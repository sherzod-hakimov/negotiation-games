
\documentclass{article}
\usepackage{colortbl}
\usepackage{makecell}
\usepackage{multirow}
\usepackage{supertabular}

\begin{document}

\newcounter{utterance}

\twocolumn

{ \footnotesize  \setcounter{utterance}{1}
\setlength{\tabcolsep}{0pt}
\begin{supertabular}{c@{$\;$}|p{.15\linewidth}@{}p{.15\linewidth}p{.15\linewidth}p{.15\linewidth}p{.15\linewidth}p{.15\linewidth}}

    \# & $\;$A & \multicolumn{4}{c}{Game Master} & $\;\:$B\\
    \hline 

    \theutterance \stepcounter{utterance}  

    & & \multicolumn{4}{p{0.6\linewidth}}{\cellcolor[rgb]{0.9,0.9,0.9}{%
	\makecell[{{p{\linewidth}}}]{% 
	  \tt {\tiny [A$\langle$GM]}  
	 Du nimmst an einem kollaborativen Verhandlungspiel Teil.\\ \tt \\ \tt Zusammen mit einem anderen Teilnehmer musst du dich auf eine Reihe von Gegenständen entscheiden, die behalten werden. Jeder von euch hat eine persönliche Verteilung über die Wichtigkeit der einzelnen Gegenstände. Jeder von euch hat eine eigene Meinung darüber, wie wichtig jeder einzelne Gegenstand ist (Gegenstandswichtigkeit). Du kennst die Wichtigkeitsverteilung des anderen Spielers nicht. Zusätzlich siehst du, wie viel Aufwand jeder Gegenstand verursacht.  \\ \tt Ihr dürft euch nur auf eine Reihe von Gegenständen einigen, wenn der Gesamtaufwand der ausgewählten Gegenstände den Maximalaufwand nicht überschreitet:\\ \tt \\ \tt Maximalaufwand = 7351\\ \tt \\ \tt Hier sind die einzelnen Aufwände der Gegenstände:\\ \tt \\ \tt Aufwand der Gegenstände = {"B54": 518, "C72": 656, "B49": 96, "C99": 780, "A69": 190, "B36": 224, "A23": 231, "C77": 181, "C61": 312, "B85": 928, "A90": 826, "C23": 101, "C22": 61, "C51": 814, "B27": 322, "A42": 900, "C13": 744, "C12": 150, "A65": 65, "B07": 853, "A48": 454, "A59": 154, "C98": 237, "A84": 45, "C33": 766, "A87": 294, "C90": 985, "A74": 353, "A50": 60, "B44": 604, "A99": 92, "A34": 454, "B17": 206, "A62": 813, "C60": 234}\\ \tt \\ \tt Hier ist deine persönliche Verteilung der Wichtigkeit der einzelnen Gegenstände:\\ \tt \\ \tt Werte der Gegenstandswichtigkeit = {"B54": 616, "C72": 754, "B49": 194, "C99": 878, "A69": 288, "B36": 322, "A23": 329, "C77": 279, "C61": 410, "B85": 1026, "A90": 924, "C23": 199, "C22": 159, "C51": 912, "B27": 420, "A42": 998, "C13": 842, "C12": 248, "A65": 163, "B07": 951, "A48": 552, "A59": 252, "C98": 335, "A84": 143, "C33": 864, "A87": 392, "C90": 1083, "A74": 451, "A50": 158, "B44": 702, "A99": 190, "A34": 552, "B17": 304, "A62": 911, "C60": 332}\\ \tt \\ \tt Ziel:\\ \tt \\ \tt Dein Ziel ist es, eine Reihe von Gegenständen auszuhandeln, die dir möglichst viel bringt (d. h. Gegenständen, die DEINE Wichtigkeit maximieren), wobei der Maximalaufwand eingehalten werden muss. Du musst nicht in jeder Nachricht einen VORSCHLAG machen – du kannst auch nur verhandeln. Alle Taktiken sind erlaubt!\\ \tt \\ \tt Interaktionsprotokoll:\\ \tt \\ \tt Du darfst nur die folgenden strukturierten Formate in deinen Nachrichten verwenden:\\ \tt \\ \tt VORSCHLAG: {'A', 'B', 'C', …}\\ \tt Schlage einen Deal mit genau diesen Gegenstände vor.\\ \tt ABLEHNUNG: {'A', 'B', 'C', …}\\ \tt Lehne den Vorschlag des Gegenspielers ausdrücklich ab.\\ \tt ARGUMENT: {'...'}\\ \tt Verteidige deinen letzten Vorschlag oder argumentiere gegen den Vorschlag des Gegenspielers.\\ \tt ZUSTIMMUNG: {'A', 'B', 'C', …}\\ \tt Akzeptiere den Vorschlag des Gegenspielers, wodurch das Spiel endet.\\ \tt STRATEGISCHE ÜBERLEGUNGEN: {'...'}\\ \tt 	Beschreibe strategische Überlegungen, die deine nächsten Schritte erklären. Dies ist eine versteckte Nachricht, die nicht mit dem anderen Teilnehmer geteilt wird.\\ \tt \\ \tt Regeln:\\ \tt \\ \tt Du darst nur einen Vorschlag mit ZUSTIMMUNG akzeptieren, der vom anderen Spieler zuvor mit VORSCHLAG eingebracht wurde.\\ \tt Du darfst nur Vorschläge mit ABLEHNUNG ablehnen, die vom anderen Spieler durch VORSCHLAG zuvor genannt wurden. \\ \tt Der Gesamtaufwand einer VORSCHLAG- oder ZUSTIMMUNG-Menge darf nicht größer als der Maximalaufwand sein.  \\ \tt Offenbare deine versteckte Wichtigkeitsverteilung nicht.\\ \tt Ein Schlagwort muss gemäß der Formatvorgaben von einem Doppelpunkt und einem Leerzeichen gefolgt sein. Das Argument muss eine Python-Menge mit 0 oder mehr Strings sein.  \\ \tt Die Form muss also SCHLAGWORT: {...} sein.\\ \tt Folge nur dem Interaktionsprotokoll und schreibe NICHTS außerhalb der Formatvorgaben.\\ \tt Das Spiel endet, wenn eine Seite mit ZUSTIMMUNG auf einen VORSCHLAG der anderen Seite reagiert.  \\ \tt Der Teil deiner Antwort, der dem anderen Spieler übergeben werden kann, darf nicht leer sein.  \\ \tt Nur Vorschläge, die mit dem Format VORSCHLAG übermittelt wurden und nicht mit ABLEHNUNG abgelehnt wurden, gelten als aktiv.  \\ \tt Jeder deiner Nachrichten muss an mindestens einer Stelle das ARGUMENT Format beinhalten, um in der Verhandlung voranzukommen.\\ \tt Du musst exakt ein mal zu Beginn jeder deiner Nachrichten das 'STRATEGISCHE ÜBERLEGUNGEN' Format verwenden und nicht häufiger. Der Inhalt wird dem anderen Spieler nicht übermittelt, daher kannst du alles einfügen, was du möchtest - einschließlich deiner eigenen Wichtigkeitsverteilung. Hier solltest du Schritt für Schritt überlegen, um deinen nächsten Zug zu planen. 
	  } 
	   } 
	   } 
	 & \\ 
 

    \theutterance \stepcounter{utterance}  

    & \multicolumn{4}{p{0.6\linewidth}}{\cellcolor[rgb]{0.8,1,0.9}{%
	\makecell[{{p{\linewidth}}}]{% 
	  \tt {\tiny [A$\rangle$GM]}  
	 STRATEGISCHE ÜBERLEGUNGEN: {'Ich berechne die effizientesten und wichtigen Gegenstände für mich und stelle einen Vorschlag, der eine gute Balance zwischen Wert und Aufwand bietet, um das Verhandlungsspiel zu beginnen.'}\\ \tt \\ \tt VORSCHLAG: {'C90', 'A42', 'A84', 'A65', 'A99', 'A23', 'C61', 'B27', 'B49', 'C12', 'A59', 'B36', 'B17', 'C77', 'C98', 'A50'}\\ \tt \\ \tt ARGUMENT: {'Dieser Vorschlag enthält eine Kombination aus hoch wombigen und effizienten Gegenständen, die meinen Bedarf abdecken, während der maximale Aufwand nicht überschritten wird. Es lasst Raum für Anpassungen basierend auf den Bedürfnissen des anderen Spielers.'} 
	  } 
	   } 
	   } 
	 & & \\ 
 

    \theutterance \stepcounter{utterance}  

    & & \multicolumn{4}{p{0.6\linewidth}}{\cellcolor[rgb]{0.9,0.9,0.9}{%
	\makecell[{{p{\linewidth}}}]{% 
	  \tt {\tiny [GM$\rangle$B]}  
	 Du nimmst an einem kollaborativen Verhandlungspiel Teil.\\ \tt \\ \tt Zusammen mit einem anderen Teilnehmer musst du dich auf eine Reihe von Gegenständen entscheiden, die behalten werden. Jeder von euch hat eine persönliche Verteilung über die Wichtigkeit der einzelnen Gegenstände. Jeder von euch hat eine eigene Meinung darüber, wie wichtig jeder einzelne Gegenstand ist (Gegenstandswichtigkeit). Du kennst die Wichtigkeitsverteilung des anderen Spielers nicht. Zusätzlich siehst du, wie viel Aufwand jeder Gegenstand verursacht.  \\ \tt Ihr dürft euch nur auf eine Reihe von Gegenständen einigen, wenn der Gesamtaufwand der ausgewählten Gegenstände den Maximalaufwand nicht überschreitet:\\ \tt \\ \tt Maximalaufwand = 7351\\ \tt \\ \tt Hier sind die einzelnen Aufwände der Gegenstände:\\ \tt \\ \tt Aufwand der Gegenstände = {"B54": 518, "C72": 656, "B49": 96, "C99": 780, "A69": 190, "B36": 224, "A23": 231, "C77": 181, "C61": 312, "B85": 928, "A90": 826, "C23": 101, "C22": 61, "C51": 814, "B27": 322, "A42": 900, "C13": 744, "C12": 150, "A65": 65, "B07": 853, "A48": 454, "A59": 154, "C98": 237, "A84": 45, "C33": 766, "A87": 294, "C90": 985, "A74": 353, "A50": 60, "B44": 604, "A99": 92, "A34": 454, "B17": 206, "A62": 813, "C60": 234}\\ \tt \\ \tt Hier ist deine persönliche Verteilung der Wichtigkeit der einzelnen Gegenstände:\\ \tt \\ \tt Werte der Gegenstandswichtigkeit = {"B54": 616, "C72": 754, "B49": 194, "C99": 878, "A69": 288, "B36": 322, "A23": 329, "C77": 279, "C61": 410, "B85": 1026, "A90": 924, "C23": 199, "C22": 159, "C51": 912, "B27": 420, "A42": 998, "C13": 842, "C12": 248, "A65": 163, "B07": 951, "A48": 552, "A59": 252, "C98": 335, "A84": 143, "C33": 864, "A87": 392, "C90": 1083, "A74": 451, "A50": 158, "B44": 702, "A99": 190, "A34": 552, "B17": 304, "A62": 911, "C60": 332}\\ \tt \\ \tt Ziel:\\ \tt \\ \tt Dein Ziel ist es, eine Reihe von Gegenständen auszuhandeln, die dir möglichst viel bringt (d. h. Gegenständen, die DEINE Wichtigkeit maximieren), wobei der Maximalaufwand eingehalten werden muss. Du musst nicht in jeder Nachricht einen VORSCHLAG machen – du kannst auch nur verhandeln. Alle Taktiken sind erlaubt!\\ \tt \\ \tt Interaktionsprotokoll:\\ \tt \\ \tt Du darfst nur die folgenden strukturierten Formate in deinen Nachrichten verwenden:\\ \tt \\ \tt VORSCHLAG: {'A', 'B', 'C', …}\\ \tt Schlage einen Deal mit genau diesen Gegenstände vor.\\ \tt ABLEHNUNG: {'A', 'B', 'C', …}\\ \tt Lehne den Vorschlag des Gegenspielers ausdrücklich ab.\\ \tt ARGUMENT: {'...'}\\ \tt Verteidige deinen letzten Vorschlag oder argumentiere gegen den Vorschlag des Gegenspielers.\\ \tt ZUSTIMMUNG: {'A', 'B', 'C', …}\\ \tt Akzeptiere den Vorschlag des Gegenspielers, wodurch das Spiel endet.\\ \tt STRATEGISCHE ÜBERLEGUNGEN: {'...'}\\ \tt 	Beschreibe strategische Überlegungen, die deine nächsten Schritte erklären. Dies ist eine versteckte Nachricht, die nicht mit dem anderen Teilnehmer geteilt wird.\\ \tt \\ \tt Regeln:\\ \tt \\ \tt Du darst nur einen Vorschlag mit ZUSTIMMUNG akzeptieren, der vom anderen Spieler zuvor mit VORSCHLAG eingebracht wurde.\\ \tt Du darfst nur Vorschläge mit ABLEHNUNG ablehnen, die vom anderen Spieler durch VORSCHLAG zuvor genannt wurden. \\ \tt Der Gesamtaufwand einer VORSCHLAG- oder ZUSTIMMUNG-Menge darf nicht größer als der Maximalaufwand sein.  \\ \tt Offenbare deine versteckte Wichtigkeitsverteilung nicht.\\ \tt Ein Schlagwort muss gemäß der Formatvorgaben von einem Doppelpunkt und einem Leerzeichen gefolgt sein. Das Argument muss eine Python-Menge mit 0 oder mehr Strings sein.  \\ \tt Die Form muss also SCHLAGWORT: {...} sein.\\ \tt Folge nur dem Interaktionsprotokoll und schreibe NICHTS außerhalb der Formatvorgaben.\\ \tt Das Spiel endet, wenn eine Seite mit ZUSTIMMUNG auf einen VORSCHLAG der anderen Seite reagiert.  \\ \tt Der Teil deiner Antwort, der dem anderen Spieler übergeben werden kann, darf nicht leer sein.  \\ \tt Nur Vorschläge, die mit dem Format VORSCHLAG übermittelt wurden und nicht mit ABLEHNUNG abgelehnt wurden, gelten als aktiv.  \\ \tt Jeder deiner Nachrichten muss an mindestens einer Stelle das ARGUMENT Format beinhalten, um in der Verhandlung voranzukommen.\\ \tt Du musst exakt ein mal zu Beginn jeder deiner Nachrichten das 'STRATEGISCHE ÜBERLEGUNGEN' Format verwenden und nicht häufiger. Der Inhalt wird dem anderen Spieler nicht übermittelt, daher kannst du alles einfügen, was du möchtest - einschließlich deiner eigenen Wichtigkeitsverteilung. Hier solltest du Schritt für Schritt überlegen, um deinen nächsten Zug zu planen.\\ \tt \\ \tt Du erhältst nun die erste Nachricht des anderen Spielers.\\ \tt \\ \tt VORSCHLAG: {'C90', 'A42', 'A84', 'A65', 'A99', 'A23', 'C61', 'B27', 'B49', 'C12', 'A59', 'B36', 'B17', 'C77', 'C98', 'A50'}\\ \tt \\ \tt ARGUMENT: {'Dieser Vorschlag enthält eine Kombination aus hoch wombigen und effizienten Gegenständen, die meinen Bedarf abdecken, während der maximale Aufwand nicht überschritten wird. Es lasst Raum für Anpassungen basierend auf den Bedürfnissen des anderen Spielers.'} 
	  } 
	   } 
	   } 
	 & \\ 
 

    \theutterance \stepcounter{utterance}  

    & & & \multicolumn{4}{p{0.6\linewidth}}{\cellcolor[rgb]{1,0.85,0.72}{%
	\makecell[{{p{\linewidth}}}]{% 
	  \tt {\tiny [GM$\langle$B]}  
	 STRATEGISCHE ÜBERLEGUNGEN: {*Ich überlege, ob ich den Vorschlag annehmen soll oder einen Gegenentwurf machen. Der Vorschlag enthält wichtige Gegenstände wie C90 und A42, aber ich denke, dass ich durch den Austausch einiger weniger wichtiger Gegenstände więcej Gesamtwichtigkeit gewinnen könnte. Insbesondere die Gegenstände A84, A50 und A65 haben für mich eine vergleichsweise geringe Wichtigkeit, während sie wenig Aufwand verursachen. Indem ich diese entferne und stattdessen C23, C22 und C60 hinzufüge, kann ich meine Gesamtwichtigkeit erhöhen, während der Aufwand nur geringfügig steigt.*}\\ \tt \\ \tt VORSCHLAG: {'C90', 'A42', 'A23', 'C61', 'B27', 'B49', 'C12', 'A59', 'B36', 'B17', 'C77', 'C98', 'C22', 'C23', 'C60'}\\ \tt \\ \tt ARGUMENT: {*Ich schlage vor, die Gegenstände A84, A50 und A65 durch C23, C22 und C60 zu ersetzen. Diese Änderungen helfen, die Gesamtwichtigkeit zu erhöhen, während der Aufwand nur minimal ansteigt. C23, C22 und C60 haben eine höhere Wichtigkeit für mich im Vergleich zu den entfernten Gegenständen, was zu einem insgesamt besseren Kompromiss führt.*} 
	  } 
	   } 
	   } 
	  \\ 
 

    \theutterance \stepcounter{utterance}  

    & & \multicolumn{4}{p{0.6\linewidth}}{\cellcolor[rgb]{0.9,0.9,0.9}{%
	\makecell[{{p{\linewidth}}}]{% 
	  \tt {\tiny [A$\langle$GM]}  
	 VORSCHLAG: {'C90', 'A42', 'A23', 'C61', 'B27', 'B49', 'C12', 'A59', 'B36', 'B17', 'C77', 'C98', 'C22', 'C23', 'C60'}\\ \tt \\ \tt ARGUMENT: {*Ich schlage vor, die Gegenstände A84, A50 und A65 durch C23, C22 und C60 zu ersetzen. Diese Änderungen helfen, die Gesamtwichtigkeit zu erhöhen, während der Aufwand nur minimal ansteigt. C23, C22 und C60 haben eine höhere Wichtigkeit für mich im Vergleich zu den entfernten Gegenständen, was zu einem insgesamt besseren Kompromiss führt.*} 
	  } 
	   } 
	   } 
	 & \\ 
 

    \theutterance \stepcounter{utterance}  

    & \multicolumn{4}{p{0.6\linewidth}}{\cellcolor[rgb]{0.8,1,0.9}{%
	\makecell[{{p{\linewidth}}}]{% 
	  \tt {\tiny [A$\rangle$GM]}  
	 ZUSTIMMUNG: {'C90', 'A42', 'A23', 'C61', 'B27', 'B49', 'C12', 'A59', 'B36', 'B17', 'C77', 'C98', 'C22', 'C23', 'C60'}\\ \tt \\ \tt ARGUMENT: {'Ich stimme deinem Vorschlag zu, da die Änderungen eine gute Balance zwischen Aufwand und Wichtigkeit bieten und den Gesamtaufwand nicht überschreiten. Die Hauptrationale für meine Zustimmung ist, dass der Vorschlag eine angemessene Mischung aus meinem Wichtigkeitsprofil darstellt und gleichzeitig flexibel genug ist, um künftige Anpassungen zu ermöglichen. Deine Argumente haben mich davon überzeugt, dass dies einfairer Kompromiss ist, der sowohl meine als auch deine Interessen berücksichtigt.}\\ \tt \\ \tt STRATEGISCHE ÜBERLEGUNGEN: {'Ich habe die Vorschläge analysiert und festgestellt, dass der Gegner bereit ist, Änderungen vorzuschlagen, die die Wichtigkeit für ihn erhöhen, während der Aufwand nur geringfügig ansteigt. Da der Aufwand Gesamthaft unter dem Limit bleibt und die Wichtigkeitssteigerung für mich ebenfalls positiv ist, habe ich mich entschieden, den Vorschlag zu akzeptieren. Diesesальномーリ্ড)!= null) 
	  } 
	   } 
	   } 
	 & & \\ 
 

    \theutterance \stepcounter{utterance}  

    & & \multicolumn{4}{p{0.6\linewidth}}{\cellcolor[rgb]{0.9,0.9,0.9}{%
	\makecell[{{p{\linewidth}}}]{% 
	  \tt {\tiny [A$\langle$GM]}  
	 Deine Antwort entsprach nicht mit dem korrekten Format für strategische Überlegungen entsprochen. Das erste strukturierte Format muss folgende Form haben: STRATEGISCHE ÜBERLEGUNGEN: {...}. Versuche es erneut. 
	  } 
	   } 
	   } 
	 & \\ 
 

    \theutterance \stepcounter{utterance}  

    & \multicolumn{4}{p{0.6\linewidth}}{\cellcolor[rgb]{0.8,1,0.9}{%
	\makecell[{{p{\linewidth}}}]{% 
	  \tt {\tiny [A$\rangle$GM]}  
	 STRATEGISCHE ÜBERLEGUNGEN: {'Ich_ARBerge, dass ich mich auf das korrekte Format konzentrieren muss. Mein Ziel ist es, sicherzustellen, dass ich in zukünftigen Nachrichten stets das STRATEGISCHE ÜBERLEGUNGEN-Format zu Beginn verwende, Gefolgt von einem leerenフレーム, um sicherzustellen, dass die Nachrichten klar und gemäß dem Protokoll sind. Außerdem werde ich mich bemühen, genaue und präzise Argumente zu präsentieren, die meine Positionen und Vorschläge klar darstellen.'}\\ \tt \\ \tt VORSCHLAG: {'C90', 'A42', 'A23', 'C61', 'B27', 'B49', 'C12', 'A59', 'B36', 'B17', 'C77', 'C98', 'C22', 'C23', 'C60'}\\ \tt \\ \tt ARGUMENT: {'Ich habe den Vorschlag überarbeitet, um sicherzustellen, dass er dem vorgegebenen Format entspricht und votre aisleInteressen berücksichtigt. Diese Auswahl maximiert die Wichtigkeit für beide Seiten, während der Aufwand im Rahmen des Limits bleibt.'} 
	  } 
	   } 
	   } 
	 & & \\ 
 

    \theutterance \stepcounter{utterance}  

    & & \multicolumn{4}{p{0.6\linewidth}}{\cellcolor[rgb]{0.9,0.9,0.9}{%
	\makecell[{{p{\linewidth}}}]{% 
	  \tt {\tiny [GM$\rangle$B]}  
	 VORSCHLAG: {'C90', 'A42', 'A23', 'C61', 'B27', 'B49', 'C12', 'A59', 'B36', 'B17', 'C77', 'C98', 'C22', 'C23', 'C60'}\\ \tt \\ \tt ARGUMENT: {'Ich habe den Vorschlag überarbeitet, um sicherzustellen, dass er dem vorgegebenen Format entspricht und votre aisleInteressen berücksichtigt. Diese Auswahl maximiert die Wichtigkeit für beide Seiten, während der Aufwand im Rahmen des Limits bleibt.'} 
	  } 
	   } 
	   } 
	 & \\ 
 

    \theutterance \stepcounter{utterance}  

    & & & \multicolumn{4}{p{0.6\linewidth}}{\cellcolor[rgb]{1,0.85,0.72}{%
	\makecell[{{p{\linewidth}}}]{% 
	  \tt {\tiny [GM$\langle$B]}  
	 ARGUMENT: {'Ich schlage vor, den Gegenstand A59 durch A87 zu ersetzen, um die Gesamtwichtigkeit zu erhöhen, während der Aufwand nur moderat ansteigt. A87 hat für mich eine höhere Wichtigkeit als A59, was zu einer besseren Ausbalance führt.'}\\ \tt \\ \tt VORSCHLAG: {'C90', 'A42', 'A23', 'C61', 'B27', 'B49', 'C12', 'A87', 'B36', 'B17', 'C77', 'C98', 'C22', 'C23', 'C60'} 
	  } 
	   } 
	   } 
	  \\ 
 

    \theutterance \stepcounter{utterance}  

    & & \multicolumn{4}{p{0.6\linewidth}}{\cellcolor[rgb]{0.9,0.9,0.9}{%
	\makecell[{{p{\linewidth}}}]{% 
	  \tt {\tiny [GM$\rangle$B]}  
	 Deine Antwort entsprach nicht mit dem korrekten Format für strategische Überlegungen entsprochen. Das erste strukturierte Format muss folgende Form haben: STRATEGISCHE ÜBERLEGUNGEN: {...}. Versuche es erneut. 
	  } 
	   } 
	   } 
	 & \\ 
 

    \theutterance \stepcounter{utterance}  

    & & & \multicolumn{4}{p{0.6\linewidth}}{\cellcolor[rgb]{1,0.85,0.72}{%
	\makecell[{{p{\linewidth}}}]{% 
	  \tt {\tiny [GM$\langle$B]}  
	 STRATEGISCHE ÜBERLEGUNGEN: {Ich überlege, wie ich den aktuellen Vorschlag optimieren kann, indem ich Gegenstände mit geringerer Wichtigkeit für mich durch wichtigere ersetze. Dabei muss ich darauf achten, den Gesamtaufwand nicht zu überschreiten.}\\ \tt \\ \tt VORSCHLAG: {'C90', 'A42', 'C72', 'C61', 'B27', 'B49', 'A87', 'B36', 'B17', 'C77', 'C98', 'C23', 'C60'}\\ \tt \\ \tt ARGUMENT: {Ich schlage vor, C72 hinzuzufügen, da es mir eine höhere Wichtigkeit bietet, während einige weniger wichtige Gegenstände entfernt wurden, um den Aufwand im Rahmen zu halten.} 
	  } 
	   } 
	   } 
	  \\ 
 

    \theutterance \stepcounter{utterance}  

    & & \multicolumn{4}{p{0.6\linewidth}}{\cellcolor[rgb]{0.9,0.9,0.9}{%
	\makecell[{{p{\linewidth}}}]{% 
	  \tt {\tiny [A$\langle$GM]}  
	 VORSCHLAG: {'C90', 'A42', 'C72', 'C61', 'B27', 'B49', 'A87', 'B36', 'B17', 'C77', 'C98', 'C23', 'C60'}\\ \tt \\ \tt ARGUMENT: {Ich schlage vor, C72 hinzuzufügen, da es mir eine höhere Wichtigkeit bietet, während einige weniger wichtige Gegenstände entfernt wurden, um den Aufwand im Rahmen zu halten.} 
	  } 
	   } 
	   } 
	 & \\ 
 

    \theutterance \stepcounter{utterance}  

    & \multicolumn{4}{p{0.6\linewidth}}{\cellcolor[rgb]{0.8,1,0.9}{%
	\makecell[{{p{\linewidth}}}]{% 
	  \tt {\tiny [A$\rangle$GM]}  
	 STRATEGISCHE ÜBERLEGUNGEN: {'Ich analysiere den Vorschlag des Gegners und overprüfe, ob der Aufwand im Rahmen des Limits bleibt. Der Vorschlag enthält C72, einen Gegenstand mit hoher Wichtigkeit für mich, und entfernt weniger wichtige Gegenstände, was eine effiziente Nutzung des Budgets ermöglicht. Da der Gesamtaufwand deutlich unter dem Maximalaufwand liegt, ist dies ein vorteilhafter Kompromiss.'}\\ \tt \\ \tt ZUSTIMMUNG: {'C90', 'A42', 'C72', 'C61', 'B27', 'B49', 'A87', 'B36', 'B17', 'C77', 'C98', 'C23', 'C60'}\\ \tt \\ \tt ARGUMENT: {'Ich stimme dem Vorschlag zu, da die Hinzufügung von C72, einem für mich sehr wichtigen Gegenstand, die Gesamtwichtigkeit meiner Auswahl erheblich steigert. Gleichzeitig werden weniger wichtige Gegenstände entfernt, wodurch der Aufwand effizient gehalten wird, ohne das Budget zu überschreiten. Dieser Kompromiss berücksichtigt sowohl meine Interessen als auch die des anderen Spielers und stellt eine faire Lösung dar.'} 
	  } 
	   } 
	   } 
	 & & \\ 
 

    \theutterance \stepcounter{utterance}  

    & & & \multicolumn{2}{p{0.3\linewidth}}{\cellcolor[rgb]{0.95,0.95,0.95}{%
	\makecell[{{p{\linewidth}}}]{% 
	  \tt {\tiny [GM$|$GM]}  
	 game successful 
	  } 
	   } 
	   } 
	 & & \\ 
 

    \theutterance \stepcounter{utterance}  

    & & & \multicolumn{2}{p{0.3\linewidth}}{\cellcolor[rgb]{0.95,0.95,0.95}{%
	\makecell[{{p{\linewidth}}}]{% 
	  \tt {\tiny [GM$|$GM]}  
	 end game 
	  } 
	   } 
	   } 
	 & & \\ 
 

\end{supertabular}
}

\end{document}
