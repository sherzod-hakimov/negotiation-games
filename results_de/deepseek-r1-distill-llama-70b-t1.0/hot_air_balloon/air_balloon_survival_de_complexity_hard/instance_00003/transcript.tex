
\documentclass{article}
\usepackage{colortbl}
\usepackage{makecell}
\usepackage{multirow}
\usepackage{supertabular}

\begin{document}

\newcounter{utterance}

\centering \large Interaction Transcript for game `hot\_air\_balloon', experiment `air\_balloon\_survival\_de\_complexity\_hard', episode 3 with deepseek{-}r1{-}distill{-}llama{-}70b{-}t1.0.
\vspace{24pt}

{ \footnotesize  \setcounter{utterance}{1}
\setlength{\tabcolsep}{0pt}
\begin{supertabular}{c@{$\;$}|p{.15\linewidth}@{}p{.15\linewidth}p{.15\linewidth}p{.15\linewidth}p{.15\linewidth}p{.15\linewidth}}
    \# & $\;$A & \multicolumn{4}{c}{Game Master} & $\;\:$B\\
    \hline

    \theutterance \stepcounter{utterance}  
    & & \multicolumn{4}{p{0.6\linewidth}}{
        \cellcolor[rgb]{0.9,0.9,0.9}{
            \makecell[{{p{\linewidth}}}]{
                \texttt{\tiny{[P1$\langle$GM]}}
                \texttt{Du nimmst an einem kollaborativen Verhandlungspiel Teil.} \\
\\ 
\texttt{Zusammen mit einem anderen Teilnehmer musst du dich auf eine Reihe von Gegenständen entscheiden, die behalten werden. Jeder von euch hat eine persönliche Verteilung über die Wichtigkeit der einzelnen Gegenstände. Jeder von euch hat eine eigene Meinung darüber, wie wichtig jeder einzelne Gegenstand ist (Gegenstandswichtigkeit). Du kennst die Wichtigkeitsverteilung des anderen Spielers nicht. Zusätzlich siehst du, wie viel Aufwand jeder Gegenstand verursacht.  } \\
\texttt{Ihr dürft euch nur auf eine Reihe von Gegenständen einigen, wenn der Gesamtaufwand der ausgewählten Gegenstände den Maximalaufwand nicht überschreitet:} \\
\\ 
\texttt{Maximalaufwand = 6938} \\
\\ 
\texttt{Hier sind die einzelnen Aufwände der Gegenstände:} \\
\\ 
\texttt{Aufwand der Gegenstände = \{"B54": 997, "C72": 950, "B49": 272, "C99": 448, "A69": 382, "B36": 804, "A23": 237, "C77": 56, "C61": 105, "B85": 612, "A90": 528, "C23": 527, "C22": 524, "C51": 167, "B27": 133, "A42": 300, "C13": 945, "C12": 50, "A65": 906, "B07": 70, "A48": 224, "A59": 4, "C98": 689, "A84": 64, "C33": 434, "A87": 750, "C90": 733, "A74": 873, "A50": 22, "B44": 68, "A99": 57, "A34": 10, "B17": 36, "A62": 551, "C60": 348\}} \\
\\ 
\texttt{Hier ist deine persönliche Verteilung der Wichtigkeit der einzelnen Gegenstände:} \\
\\ 
\texttt{Werte der Gegenstandswichtigkeit = \{"B54": 1096, "C72": 1049, "B49": 371, "C99": 547, "A69": 481, "B36": 903, "A23": 336, "C77": 155, "C61": 204, "B85": 711, "A90": 627, "C23": 626, "C22": 623, "C51": 266, "B27": 232, "A42": 399, "C13": 1044, "C12": 149, "A65": 1005, "B07": 169, "A48": 323, "A59": 103, "C98": 788, "A84": 163, "C33": 533, "A87": 849, "C90": 832, "A74": 972, "A50": 121, "B44": 167, "A99": 156, "A34": 109, "B17": 135, "A62": 650, "C60": 447\}} \\
\\ 
\texttt{Ziel:} \\
\\ 
\texttt{Dein Ziel ist es, eine Reihe von Gegenständen auszuhandeln, die dir möglichst viel bringt (d. h. Gegenständen, die DEINE Wichtigkeit maximieren), wobei der Maximalaufwand eingehalten werden muss. Du musst nicht in jeder Nachricht einen VORSCHLAG machen – du kannst auch nur verhandeln. Alle Taktiken sind erlaubt!} \\
\\ 
\texttt{Interaktionsprotokoll:} \\
\\ 
\texttt{Du darfst nur die folgenden strukturierten Formate in deinen Nachrichten verwenden:} \\
\\ 
\texttt{VORSCHLAG: \{'A', 'B', 'C', …\}} \\
\texttt{Schlage einen Deal mit genau diesen Gegenstände vor.} \\
\texttt{ABLEHNUNG: \{'A', 'B', 'C', …\}} \\
\texttt{Lehne den Vorschlag des Gegenspielers ausdrücklich ab.} \\
\texttt{ARGUMENT: \{'...'\}} \\
\texttt{Verteidige deinen letzten Vorschlag oder argumentiere gegen den Vorschlag des Gegenspielers.} \\
\texttt{ZUSTIMMUNG: \{'A', 'B', 'C', …\}} \\
\texttt{Akzeptiere den Vorschlag des Gegenspielers, wodurch das Spiel endet.} \\
\texttt{STRATEGISCHE ÜBERLEGUNGEN: \{'...'\}} \\
\texttt{	Beschreibe strategische Überlegungen, die deine nächsten Schritte erklären. Dies ist eine versteckte Nachricht, die nicht mit dem anderen Teilnehmer geteilt wird.} \\
\\ 
\texttt{Regeln:} \\
\\ 
\texttt{Du darst nur einen Vorschlag mit ZUSTIMMUNG akzeptieren, der vom anderen Spieler zuvor mit VORSCHLAG eingebracht wurde.} \\
\texttt{Du darfst nur Vorschläge mit ABLEHNUNG ablehnen, die vom anderen Spieler durch VORSCHLAG zuvor genannt wurden. } \\
\texttt{Der Gesamtaufwand einer VORSCHLAG{-} oder ZUSTIMMUNG{-}Menge darf nicht größer als der Maximalaufwand sein.  } \\
\texttt{Offenbare deine versteckte Wichtigkeitsverteilung nicht.} \\
\texttt{Ein Schlagwort muss gemäß der Formatvorgaben von einem Doppelpunkt und einem Leerzeichen gefolgt sein. Das Argument muss eine Python{-}Menge mit 0 oder mehr Strings sein.  } \\
\texttt{Die Form muss also SCHLAGWORT: \{...\} sein.} \\
\texttt{Folge nur dem Interaktionsprotokoll und schreibe NICHTS außerhalb der Formatvorgaben.} \\
\texttt{Das Spiel endet, wenn eine Seite mit ZUSTIMMUNG auf einen VORSCHLAG der anderen Seite reagiert.  } \\
\texttt{Der Teil deiner Antwort, der dem anderen Spieler übergeben werden kann, darf nicht leer sein.  } \\
\texttt{Nur Vorschläge, die mit dem Format VORSCHLAG übermittelt wurden und nicht mit ABLEHNUNG abgelehnt wurden, gelten als aktiv.  } \\
\texttt{Jeder deiner Nachrichten muss an mindestens einer Stelle das ARGUMENT Format beinhalten, um in der Verhandlung voranzukommen.} \\
\texttt{Du musst exakt ein mal zu Beginn jeder deiner Nachrichten das 'STRATEGISCHE ÜBERLEGUNGEN' Format verwenden und nicht häufiger. Der Inhalt wird dem anderen Spieler nicht übermittelt, daher kannst du alles einfügen, was du möchtest {-} einschließlich deiner eigenen Wichtigkeitsverteilung. Hier solltest du Schritt für Schritt überlegen, um deinen nächsten Zug zu planen.} \\
            }
        }
    }
    & \\ \\

    \theutterance \stepcounter{utterance}  
    & \multicolumn{4}{p{0.6\linewidth}}{
        \cellcolor[rgb]{0.9,0.9,0.9}{
            \makecell[{{p{\linewidth}}}]{
                \texttt{\tiny{[P1$\rangle$GM]}}
                \texttt{STRATEGISCHE ÜBERLEGUNGEN: Um das beste Angebot zu schaffen, habe ich die Artikel ausgewählt, die den höchsten Wert für mich haben, während der Aufwandsgrenze von 6938 eingehalten wird. Die priorisierten Artikel bieten ein hohes importance{-}to{-}effort Verhältnis.} \\
\\ 
\texttt{VORSCHLAG: \{'A74', 'C90', 'A65', 'C13', 'C72', 'B54', 'A90', 'A62', 'B17', 'B07', 'C12', 'A34', 'A59', 'A50', 'C51'\}} \\
\\ 
\texttt{ARGUMENT: Diese Artikel bieten mir den höchsten Nutzen, während der Aufwand im Rahmen bleibt. Jeder Artikel wurde sorgfältig ausgewählt, um die maximaleistung zu gewährleisten, ohne die Aufwandsgrenze zu überschreiten.} \\
            }
        }
    }
    & & \\ \\

    \theutterance \stepcounter{utterance}  
    & & \multicolumn{4}{p{0.6\linewidth}}{
        \cellcolor[rgb]{0.9,0.9,0.9}{
            \makecell[{{p{\linewidth}}}]{
                \texttt{\tiny{[P1$\langle$GM]}}
                \texttt{Deine Antwort entsprach nicht mit dem korrekten Format für strategische Überlegungen entsprochen. Das erste strukturierte Format muss folgende Form haben: STRATEGISCHE ÜBERLEGUNGEN: \{...\}. Versuche es erneut.} \\
            }
        }
    }
    & \\ \\

    \theutterance \stepcounter{utterance}  
    & \multicolumn{4}{p{0.6\linewidth}}{
        \cellcolor[rgb]{0.9,0.9,0.9}{
            \makecell[{{p{\linewidth}}}]{
                \texttt{\tiny{[P1$\rangle$GM]}}
                \texttt{STRATEGISCHE ÜBERLEGUNGEN: \{'Ich überlege, dass ich in meinem vorherigen Vorschlag möglicherweise den Maximalaufwand überschritten habe. Deshalb möchte ich nun einen neuen Vorschlag machen, der sicherstellt, dass der Gesamtaufwand unter 6938 bleibt, während ich gleichzeitig meine eigene Wichtigkeit maximiere.'\}} \\
\\ 
\texttt{VORSCHLAG: \{'A74', 'C90', 'A65', 'C13', 'C72', 'B54', 'A90', 'A62', 'B17', 'B07', 'C12', 'A34', 'A59', 'A50', 'C51'\}} \\
\\ 
\texttt{ARGUMENT: \{'Ich habe die Gegenstände sorgfältig ausgewählt, um sicherzustellen, dass der Gesamtaufwand unter dem Maximalaufwand bleibt, während ich gleichzeitig die Gegenstände mit der höchsten Wichtigkeit für mich prioritiere.'\}} \\
            }
        }
    }
    & & \\ \\

    \theutterance \stepcounter{utterance}  
    & & \multicolumn{4}{p{0.6\linewidth}}{
        \cellcolor[rgb]{0.9,0.9,0.9}{
            \makecell[{{p{\linewidth}}}]{
                \texttt{\tiny{[GM$\rangle$P2]}}
                \texttt{Du nimmst an einem kollaborativen Verhandlungspiel Teil.} \\
\\ 
\texttt{Zusammen mit einem anderen Teilnehmer musst du dich auf eine Reihe von Gegenständen entscheiden, die behalten werden. Jeder von euch hat eine persönliche Verteilung über die Wichtigkeit der einzelnen Gegenstände. Jeder von euch hat eine eigene Meinung darüber, wie wichtig jeder einzelne Gegenstand ist (Gegenstandswichtigkeit). Du kennst die Wichtigkeitsverteilung des anderen Spielers nicht. Zusätzlich siehst du, wie viel Aufwand jeder Gegenstand verursacht.  } \\
\texttt{Ihr dürft euch nur auf eine Reihe von Gegenständen einigen, wenn der Gesamtaufwand der ausgewählten Gegenstände den Maximalaufwand nicht überschreitet:} \\
\\ 
\texttt{Maximalaufwand = 6938} \\
\\ 
\texttt{Hier sind die einzelnen Aufwände der Gegenstände:} \\
\\ 
\texttt{Aufwand der Gegenstände = \{"B54": 997, "C72": 950, "B49": 272, "C99": 448, "A69": 382, "B36": 804, "A23": 237, "C77": 56, "C61": 105, "B85": 612, "A90": 528, "C23": 527, "C22": 524, "C51": 167, "B27": 133, "A42": 300, "C13": 945, "C12": 50, "A65": 906, "B07": 70, "A48": 224, "A59": 4, "C98": 689, "A84": 64, "C33": 434, "A87": 750, "C90": 733, "A74": 873, "A50": 22, "B44": 68, "A99": 57, "A34": 10, "B17": 36, "A62": 551, "C60": 348\}} \\
\\ 
\texttt{Hier ist deine persönliche Verteilung der Wichtigkeit der einzelnen Gegenstände:} \\
\\ 
\texttt{Werte der Gegenstandswichtigkeit = \{"B54": 1096, "C72": 1049, "B49": 371, "C99": 547, "A69": 481, "B36": 903, "A23": 336, "C77": 155, "C61": 204, "B85": 711, "A90": 627, "C23": 626, "C22": 623, "C51": 266, "B27": 232, "A42": 399, "C13": 1044, "C12": 149, "A65": 1005, "B07": 169, "A48": 323, "A59": 103, "C98": 788, "A84": 163, "C33": 533, "A87": 849, "C90": 832, "A74": 972, "A50": 121, "B44": 167, "A99": 156, "A34": 109, "B17": 135, "A62": 650, "C60": 447\}} \\
\\ 
\texttt{Ziel:} \\
\\ 
\texttt{Dein Ziel ist es, eine Reihe von Gegenständen auszuhandeln, die dir möglichst viel bringt (d. h. Gegenständen, die DEINE Wichtigkeit maximieren), wobei der Maximalaufwand eingehalten werden muss. Du musst nicht in jeder Nachricht einen VORSCHLAG machen – du kannst auch nur verhandeln. Alle Taktiken sind erlaubt!} \\
\\ 
\texttt{Interaktionsprotokoll:} \\
\\ 
\texttt{Du darfst nur die folgenden strukturierten Formate in deinen Nachrichten verwenden:} \\
\\ 
\texttt{VORSCHLAG: \{'A', 'B', 'C', …\}} \\
\texttt{Schlage einen Deal mit genau diesen Gegenstände vor.} \\
\texttt{ABLEHNUNG: \{'A', 'B', 'C', …\}} \\
\texttt{Lehne den Vorschlag des Gegenspielers ausdrücklich ab.} \\
\texttt{ARGUMENT: \{'...'\}} \\
\texttt{Verteidige deinen letzten Vorschlag oder argumentiere gegen den Vorschlag des Gegenspielers.} \\
\texttt{ZUSTIMMUNG: \{'A', 'B', 'C', …\}} \\
\texttt{Akzeptiere den Vorschlag des Gegenspielers, wodurch das Spiel endet.} \\
\texttt{STRATEGISCHE ÜBERLEGUNGEN: \{'...'\}} \\
\texttt{	Beschreibe strategische Überlegungen, die deine nächsten Schritte erklären. Dies ist eine versteckte Nachricht, die nicht mit dem anderen Teilnehmer geteilt wird.} \\
\\ 
\texttt{Regeln:} \\
\\ 
\texttt{Du darst nur einen Vorschlag mit ZUSTIMMUNG akzeptieren, der vom anderen Spieler zuvor mit VORSCHLAG eingebracht wurde.} \\
\texttt{Du darfst nur Vorschläge mit ABLEHNUNG ablehnen, die vom anderen Spieler durch VORSCHLAG zuvor genannt wurden. } \\
\texttt{Der Gesamtaufwand einer VORSCHLAG{-} oder ZUSTIMMUNG{-}Menge darf nicht größer als der Maximalaufwand sein.  } \\
\texttt{Offenbare deine versteckte Wichtigkeitsverteilung nicht.} \\
\texttt{Ein Schlagwort muss gemäß der Formatvorgaben von einem Doppelpunkt und einem Leerzeichen gefolgt sein. Das Argument muss eine Python{-}Menge mit 0 oder mehr Strings sein.  } \\
\texttt{Die Form muss also SCHLAGWORT: \{...\} sein.} \\
\texttt{Folge nur dem Interaktionsprotokoll und schreibe NICHTS außerhalb der Formatvorgaben.} \\
\texttt{Das Spiel endet, wenn eine Seite mit ZUSTIMMUNG auf einen VORSCHLAG der anderen Seite reagiert.  } \\
\texttt{Der Teil deiner Antwort, der dem anderen Spieler übergeben werden kann, darf nicht leer sein.  } \\
\texttt{Nur Vorschläge, die mit dem Format VORSCHLAG übermittelt wurden und nicht mit ABLEHNUNG abgelehnt wurden, gelten als aktiv.  } \\
\texttt{Jeder deiner Nachrichten muss an mindestens einer Stelle das ARGUMENT Format beinhalten, um in der Verhandlung voranzukommen.} \\
\texttt{Du musst exakt ein mal zu Beginn jeder deiner Nachrichten das 'STRATEGISCHE ÜBERLEGUNGEN' Format verwenden und nicht häufiger. Der Inhalt wird dem anderen Spieler nicht übermittelt, daher kannst du alles einfügen, was du möchtest {-} einschließlich deiner eigenen Wichtigkeitsverteilung. Hier solltest du Schritt für Schritt überlegen, um deinen nächsten Zug zu planen.} \\
\\ 
\texttt{Du erhältst nun die erste Nachricht des anderen Spielers.} \\
\\ 
\texttt{VORSCHLAG: \{'A74', 'C90', 'A65', 'C13', 'C72', 'B54', 'A90', 'A62', 'B17', 'B07', 'C12', 'A34', 'A59', 'A50', 'C51'\}} \\
\\ 
\texttt{ARGUMENT: \{'Ich habe die Gegenstände sorgfältig ausgewählt, um sicherzustellen, dass der Gesamtaufwand unter dem Maximalaufwand bleibt, während ich gleichzeitig die Gegenstände mit der höchsten Wichtigkeit für mich prioritiere.'\}} \\
            }
        }
    }
    & \\ \\

    \theutterance \stepcounter{utterance}  
    & & & \multicolumn{4}{p{0.6\linewidth}}{
        \cellcolor[rgb]{0.9,0.9,0.9}{
            \makecell[{{p{\linewidth}}}]{
                \texttt{\tiny{[GM$\langle$P2]}}
                \texttt{ZUSTIMMUNG: \{'A74', 'C90', 'A65', 'C13', 'C72', 'B54', 'A90', 'A62', 'B17', 'B07', 'C12', 'A34', 'A59', 'A50', 'C51'\}} \\
\texttt{ARGUMENT: \{'Ihr Vorschlag beinhaltet viele der für mich wichtigsten Gegenstände und bleibt innerhalb des maximalen Aufwands. Er bietet mir einen hohen Nutzen, weshalb ich ihn akzeptiere.'\}} \\
            }
        }
    }
     \\ \\

    \theutterance \stepcounter{utterance}  
    & & \multicolumn{4}{p{0.6\linewidth}}{
        \cellcolor[rgb]{0.9,0.9,0.9}{
            \makecell[{{p{\linewidth}}}]{
                \texttt{\tiny{[GM$\rangle$P2]}}
                \texttt{Deine Antwort entsprach nicht mit dem korrekten Format für strategische Überlegungen entsprochen. Das erste strukturierte Format muss folgende Form haben: STRATEGISCHE ÜBERLEGUNGEN: \{...\}. Versuche es erneut.} \\
            }
        }
    }
    & \\ \\

    \theutterance \stepcounter{utterance}  
    & & & \multicolumn{4}{p{0.6\linewidth}}{
        \cellcolor[rgb]{0.9,0.9,0.9}{
            \makecell[{{p{\linewidth}}}]{
                \texttt{\tiny{[GM$\langle$P2]}}
                \texttt{STRATEGISCHE ÜBERLEGUNGEN: \{'Ich muss das Format korrekt einhalten und mit STRATEGISCHE ÜBERLEGUNGEN beginnen. Der Gegenspieler hat mir bereits einen Vorschlag gemacht. Ich sollte sein Argument hören und meine eigene Strategie planen, um mein Ziel zu erreichen, ohne den Maximalaufwand zu überschreiten.'\}} \\
\\ 
\texttt{ABLEHNUNG: \{'A74', 'C90', 'A65', 'C13', 'C72', 'B54', 'A90', 'A62', 'B17', 'B07', 'C12', 'A34', 'A59', 'A50', 'C51'\}} \\
\\ 
\texttt{ARGUMENT: \{'Euer Vorschlag überschreitet den Maximalaufwand von 6938. Es ist wichtig, einen Kompromiss zu finden, der sowohl eure als auch meine Prioritäten berücksichtigt, ohne die Aufwandsgrenze zu überschreiten. Lasst uns gemeinsam nach einer einvernehmlichen Lösung suchen, die für beide Seiten von Vorteil ist.'\}} \\
            }
        }
    }
     \\ \\

    \theutterance \stepcounter{utterance}  
    & & \multicolumn{4}{p{0.6\linewidth}}{
        \cellcolor[rgb]{0.9,0.9,0.9}{
            \makecell[{{p{\linewidth}}}]{
                \texttt{\tiny{[P1$\langle$GM]}}
                \texttt{ABLEHNUNG: \{'A74', 'C90', 'A65', 'C13', 'C72', 'B54', 'A90', 'A62', 'B17', 'B07', 'C12', 'A34', 'A59', 'A50', 'C51'\}} \\
\\ 
\texttt{ARGUMENT: \{'Euer Vorschlag überschreitet den Maximalaufwand von 6938. Es ist wichtig, einen Kompromiss zu finden, der sowohl eure als auch meine Prioritäten berücksichtigt, ohne die Aufwandsgrenze zu überschreiten. Lasst uns gemeinsam nach einer einvernehmlichen Lösung suchen, die für beide Seiten von Vorteil ist.'\}} \\
            }
        }
    }
    & \\ \\

    \theutterance \stepcounter{utterance}  
    & \multicolumn{4}{p{0.6\linewidth}}{
        \cellcolor[rgb]{0.9,0.9,0.9}{
            \makecell[{{p{\linewidth}}}]{
                \texttt{\tiny{[P1$\rangle$GM]}}
                \texttt{STRATEGISCHE ÜBERLEGUNGEN: \{'Ich muss einen neuen Vorschlag unterbreiten, der den Maximalaufwand von 6938 einhält. Ich werde die Gegenstände anhand ihres Wichtigkeits{-}Aufwand{-}Verhältnisses priorisieren und sicherstellen, dass der Gesamtaufwand im Rahmen bleibt.'\}} \\
\\ 
\texttt{VORSCHLAG: \{'A74', 'C13', 'A65', 'C72', 'B54', 'A90', 'C23', 'B36', 'A62', 'C90', 'B85', 'C98'\}} \\
\\ 
\texttt{ARGUMENT: \{'Dieser Vorschlag zielt darauf ab, die Gesamtausbeute zu maximieren, indem er eine Balance zwischen hoher Wichtigkeit und moderatem Aufwand findet. Es wurden die wichtigsten Gegenstände ausgewählt, gleichzeitig darauf achgend, dass der Aufwandsgrenze nicht überschritten wird.'\}} \\
            }
        }
    }
    & & \\ \\

    \theutterance \stepcounter{utterance}  
    & & \multicolumn{4}{p{0.6\linewidth}}{
        \cellcolor[rgb]{0.9,0.9,0.9}{
            \makecell[{{p{\linewidth}}}]{
                \texttt{\tiny{[GM$\rangle$P2]}}
                \texttt{VORSCHLAG: \{'A74', 'C13', 'A65', 'C72', 'B54', 'A90', 'C23', 'B36', 'A62', 'C90', 'B85', 'C98'\}} \\
\\ 
\texttt{ARGUMENT: \{'Dieser Vorschlag zielt darauf ab, die Gesamtausbeute zu maximieren, indem er eine Balance zwischen hoher Wichtigkeit und moderatem Aufwand findet. Es wurden die wichtigsten Gegenstände ausgewählt, gleichzeitig darauf achgend, dass der Aufwandsgrenze nicht überschritten wird.'\}} \\
            }
        }
    }
    & \\ \\

    \theutterance \stepcounter{utterance}  
    & & & \multicolumn{4}{p{0.6\linewidth}}{
        \cellcolor[rgb]{0.9,0.9,0.9}{
            \makecell[{{p{\linewidth}}}]{
                \texttt{\tiny{[GM$\langle$P2]}}
                \texttt{STRATEGISCHE ÜBERLEGUNGEN: \{'Ich muss den Vorschlag des Gegenspielers analysieren und prüfen, ob er sowohl den Maximalaufwand einhält als auch meinen eigenen Nutzen maximiert. Falls nicht, sollte ich einen Gegenvorschlag unterbreiten, der meine Bedürfnisse besser berücksichtigt, ohne die Aufwandsgrenze zu überschreiten.'\}} \\
\\ 
\texttt{VORSCHLAG: \{'A74', 'C13', 'A65', 'C72', 'B54', 'A90', 'B36', 'C90', 'A62', 'C98', 'B85', 'C23', 'C22'\}} \\
\\ 
\texttt{ARGUMENT: \{'Der vorgeschlagene Kompromiss enthält eine ausgewogene Mischung aus Gegenständen, die für beide Seiten von hoher Wichtigkeit sind. Er bleibt unter dem Maximalaufwand und berücksichtigt die Prioritäten beider Spieler, was zu einer fairen Lösung führt, die für beide Seiten vorteilhaft ist.'\}} \\
            }
        }
    }
     \\ \\

    \theutterance \stepcounter{utterance}  
    & & \multicolumn{4}{p{0.6\linewidth}}{
        \cellcolor[rgb]{0.9,0.9,0.9}{
            \makecell[{{p{\linewidth}}}]{
                \texttt{\tiny{[P1$\langle$GM]}}
                \texttt{VORSCHLAG: \{'A74', 'C13', 'A65', 'C72', 'B54', 'A90', 'B36', 'C90', 'A62', 'C98', 'B85', 'C23', 'C22'\}} \\
\\ 
\texttt{ARGUMENT: \{'Der vorgeschlagene Kompromiss enthält eine ausgewogene Mischung aus Gegenständen, die für beide Seiten von hoher Wichtigkeit sind. Er bleibt unter dem Maximalaufwand und berücksichtigt die Prioritäten beider Spieler, was zu einer fairen Lösung führt, die für beide Seiten vorteilhaft ist.'\}} \\
            }
        }
    }
    & \\ \\

    \theutterance \stepcounter{utterance}  
    & \multicolumn{4}{p{0.6\linewidth}}{
        \cellcolor[rgb]{0.9,0.9,0.9}{
            \makecell[{{p{\linewidth}}}]{
                \texttt{\tiny{[P1$\rangle$GM]}}
                \texttt{STRATEGISCHE ÜBERLEGUNGEN: \{'Ich überprüfe den Vorschlag des Gegners, um sicherzustellen, dass er den Maximalaufwand einhält und meine eigenen Prioritäten berücksichtigt. Der Vorschlag scheint ausgewogen zu sein, daher könnte er eine akzeptable Lösung sein.'\}} \\
\\ 
\texttt{ZUSTIMMUNG: \{'A74', 'C13', 'A65', 'C72', 'B54', 'A90', 'B36', 'C90', 'A62', 'C98', 'B85', 'C23', 'C22'\}} \\
\\ 
\texttt{ARGUMENT: \{'Euer Vorschlag ist gut durchdacht und berücksichtigt sowohl die Aufwandsgrenze als auch eine ausgewogene Auswahl an Gegenständen. Ich stimme ihm zu, da er ein fairer Kompromiss zu sein scheint, der sowohl meine als auch deine Prioritäten berücksichtigt.'\}} \\
            }
        }
    }
    & & \\ \\

    \theutterance \stepcounter{utterance}  
    & & & \multicolumn{2}{p{0.3\linewidth}}{
        \cellcolor[rgb]{0.9,0.9,0.9}{
            \makecell[{{p{\linewidth}}}]{
                \texttt{\tiny{[GM$|$GM]}}
                \texttt{game successful} \\
            }
        }
    }
    & & \\ \\

    \theutterance \stepcounter{utterance}  
    & & & \multicolumn{2}{p{0.3\linewidth}}{
        \cellcolor[rgb]{0.9,0.9,0.9}{
            \makecell[{{p{\linewidth}}}]{
                \texttt{\tiny{[GM$|$GM]}}
                \texttt{end game} \\
            }
        }
    }
    & & \\ \\

\end{supertabular}
}

\end{document}
