
\documentclass{article}
\usepackage{colortbl}
\usepackage{makecell}
\usepackage{multirow}
\usepackage{supertabular}

\begin{document}

\newcounter{utterance}

\centering \large Interaction Transcript for game `hot\_air\_balloon', experiment `air\_balloon\_survival\_de\_complexity\_hard', episode 4 with deepseek{-}r1{-}distill{-}llama{-}70b{-}t1.0.
\vspace{24pt}

{ \footnotesize  \setcounter{utterance}{1}
\setlength{\tabcolsep}{0pt}
\begin{supertabular}{c@{$\;$}|p{.15\linewidth}@{}p{.15\linewidth}p{.15\linewidth}p{.15\linewidth}p{.15\linewidth}p{.15\linewidth}}
    \# & $\;$A & \multicolumn{4}{c}{Game Master} & $\;\:$B\\
    \hline

    \theutterance \stepcounter{utterance}  
    & & \multicolumn{4}{p{0.6\linewidth}}{
        \cellcolor[rgb]{0.9,0.9,0.9}{
            \makecell[{{p{\linewidth}}}]{
                \texttt{\tiny{[P1$\langle$GM]}}
                \texttt{Du nimmst an einem kollaborativen Verhandlungspiel Teil.} \\
\\ 
\texttt{Zusammen mit einem anderen Teilnehmer musst du dich auf eine Reihe von Gegenständen entscheiden, die behalten werden. Jeder von euch hat eine persönliche Verteilung über die Wichtigkeit der einzelnen Gegenstände. Jeder von euch hat eine eigene Meinung darüber, wie wichtig jeder einzelne Gegenstand ist (Gegenstandswichtigkeit). Du kennst die Wichtigkeitsverteilung des anderen Spielers nicht. Zusätzlich siehst du, wie viel Aufwand jeder Gegenstand verursacht.  } \\
\texttt{Ihr dürft euch nur auf eine Reihe von Gegenständen einigen, wenn der Gesamtaufwand der ausgewählten Gegenstände den Maximalaufwand nicht überschreitet:} \\
\\ 
\texttt{Maximalaufwand = 6880} \\
\\ 
\texttt{Hier sind die einzelnen Aufwände der Gegenstände:} \\
\\ 
\texttt{Aufwand der Gegenstände = \{"B54": 341, "C72": 804, "B49": 20, "C99": 627, "A69": 10, "B36": 573, "A23": 217, "C77": 481, "C61": 205, "B85": 273, "A90": 303, "C23": 596, "C22": 564, "C51": 535, "B27": 258, "A42": 911, "C13": 240, "C12": 188, "A65": 216, "B07": 401, "A48": 909, "A59": 62, "C98": 245, "A84": 994, "C33": 569, "A87": 718, "C90": 464, "A74": 37, "A50": 340, "B44": 335, "A99": 417, "A34": 123, "B17": 17, "A62": 577, "C60": 190\}} \\
\\ 
\texttt{Hier ist deine persönliche Verteilung der Wichtigkeit der einzelnen Gegenstände:} \\
\\ 
\texttt{Werte der Gegenstandswichtigkeit = \{"B54": 440, "C72": 903, "B49": 119, "C99": 726, "A69": 109, "B36": 672, "A23": 316, "C77": 580, "C61": 304, "B85": 372, "A90": 402, "C23": 695, "C22": 663, "C51": 634, "B27": 357, "A42": 1010, "C13": 339, "C12": 287, "A65": 315, "B07": 500, "A48": 1008, "A59": 161, "C98": 344, "A84": 1093, "C33": 668, "A87": 817, "C90": 563, "A74": 136, "A50": 439, "B44": 434, "A99": 516, "A34": 222, "B17": 116, "A62": 676, "C60": 289\}} \\
\\ 
\texttt{Ziel:} \\
\\ 
\texttt{Dein Ziel ist es, eine Reihe von Gegenständen auszuhandeln, die dir möglichst viel bringt (d. h. Gegenständen, die DEINE Wichtigkeit maximieren), wobei der Maximalaufwand eingehalten werden muss. Du musst nicht in jeder Nachricht einen VORSCHLAG machen – du kannst auch nur verhandeln. Alle Taktiken sind erlaubt!} \\
\\ 
\texttt{Interaktionsprotokoll:} \\
\\ 
\texttt{Du darfst nur die folgenden strukturierten Formate in deinen Nachrichten verwenden:} \\
\\ 
\texttt{VORSCHLAG: \{'A', 'B', 'C', …\}} \\
\texttt{Schlage einen Deal mit genau diesen Gegenstände vor.} \\
\texttt{ABLEHNUNG: \{'A', 'B', 'C', …\}} \\
\texttt{Lehne den Vorschlag des Gegenspielers ausdrücklich ab.} \\
\texttt{ARGUMENT: \{'...'\}} \\
\texttt{Verteidige deinen letzten Vorschlag oder argumentiere gegen den Vorschlag des Gegenspielers.} \\
\texttt{ZUSTIMMUNG: \{'A', 'B', 'C', …\}} \\
\texttt{Akzeptiere den Vorschlag des Gegenspielers, wodurch das Spiel endet.} \\
\texttt{STRATEGISCHE ÜBERLEGUNGEN: \{'...'\}} \\
\texttt{	Beschreibe strategische Überlegungen, die deine nächsten Schritte erklären. Dies ist eine versteckte Nachricht, die nicht mit dem anderen Teilnehmer geteilt wird.} \\
\\ 
\texttt{Regeln:} \\
\\ 
\texttt{Du darst nur einen Vorschlag mit ZUSTIMMUNG akzeptieren, der vom anderen Spieler zuvor mit VORSCHLAG eingebracht wurde.} \\
\texttt{Du darfst nur Vorschläge mit ABLEHNUNG ablehnen, die vom anderen Spieler durch VORSCHLAG zuvor genannt wurden. } \\
\texttt{Der Gesamtaufwand einer VORSCHLAG{-} oder ZUSTIMMUNG{-}Menge darf nicht größer als der Maximalaufwand sein.  } \\
\texttt{Offenbare deine versteckte Wichtigkeitsverteilung nicht.} \\
\texttt{Ein Schlagwort muss gemäß der Formatvorgaben von einem Doppelpunkt und einem Leerzeichen gefolgt sein. Das Argument muss eine Python{-}Menge mit 0 oder mehr Strings sein.  } \\
\texttt{Die Form muss also SCHLAGWORT: \{...\} sein.} \\
\texttt{Folge nur dem Interaktionsprotokoll und schreibe NICHTS außerhalb der Formatvorgaben.} \\
\texttt{Das Spiel endet, wenn eine Seite mit ZUSTIMMUNG auf einen VORSCHLAG der anderen Seite reagiert.  } \\
\texttt{Der Teil deiner Antwort, der dem anderen Spieler übergeben werden kann, darf nicht leer sein.  } \\
\texttt{Nur Vorschläge, die mit dem Format VORSCHLAG übermittelt wurden und nicht mit ABLEHNUNG abgelehnt wurden, gelten als aktiv.  } \\
\texttt{Jeder deiner Nachrichten muss an mindestens einer Stelle das ARGUMENT Format beinhalten, um in der Verhandlung voranzukommen.} \\
\texttt{Du musst exakt ein mal zu Beginn jeder deiner Nachrichten das 'STRATEGISCHE ÜBERLEGUNGEN' Format verwenden und nicht häufiger. Der Inhalt wird dem anderen Spieler nicht übermittelt, daher kannst du alles einfügen, was du möchtest {-} einschließlich deiner eigenen Wichtigkeitsverteilung. Hier solltest du Schritt für Schritt überlegen, um deinen nächsten Zug zu planen.} \\
            }
        }
    }
    & \\ \\

    \theutterance \stepcounter{utterance}  
    & \multicolumn{4}{p{0.6\linewidth}}{
        \cellcolor[rgb]{0.9,0.9,0.9}{
            \makecell[{{p{\linewidth}}}]{
                \texttt{\tiny{[P1$\rangle$GM]}}
                \texttt{STRATEGISCHE ÜBERLEGUNGEN: \{'Ich habe die Gegenstände nach dem höchsten Nutzen für mich ausgewählt, um die Gesamtaufwandsbegrenzung von 6880 einzuhalten. Der Schwerpunkt lag auf einer hohen Wichtigkeitszuweisung meinerseits, verbunden mit einem vertretbaren Aufwand. Mir ist bewusst, dass der andere Spieler anders priorisieren könnte, aber ohne dessen Wichtigkeiten zu kennen, konzentriere ich mich auf meine eigenen.'\}} \\
\\ 
\texttt{VORSCHLAG: \{'A69', 'B17', 'B49', 'A74', 'A59', 'C61', 'C12', 'C13', 'C60', 'A23', 'A65', 'B85', 'B54', 'A50', 'B44', 'A99', 'C98', 'A34', 'C90', 'C77', 'C22', 'C51', 'C99', 'C33'\}} \\
\\ 
\texttt{ARGUMENT: \{'Ich habe die ausgewählten Gegenstände sorgfältig nach meinem Wichtigkeitsprofil und dem verfügbaren Aufwand ausgewählt, um mein Eigeninteresse zu maximieren.'\}} \\
            }
        }
    }
    & & \\ \\

    \theutterance \stepcounter{utterance}  
    & & \multicolumn{4}{p{0.6\linewidth}}{
        \cellcolor[rgb]{0.9,0.9,0.9}{
            \makecell[{{p{\linewidth}}}]{
                \texttt{\tiny{[GM$\rangle$P2]}}
                \texttt{Du nimmst an einem kollaborativen Verhandlungspiel Teil.} \\
\\ 
\texttt{Zusammen mit einem anderen Teilnehmer musst du dich auf eine Reihe von Gegenständen entscheiden, die behalten werden. Jeder von euch hat eine persönliche Verteilung über die Wichtigkeit der einzelnen Gegenstände. Jeder von euch hat eine eigene Meinung darüber, wie wichtig jeder einzelne Gegenstand ist (Gegenstandswichtigkeit). Du kennst die Wichtigkeitsverteilung des anderen Spielers nicht. Zusätzlich siehst du, wie viel Aufwand jeder Gegenstand verursacht.  } \\
\texttt{Ihr dürft euch nur auf eine Reihe von Gegenständen einigen, wenn der Gesamtaufwand der ausgewählten Gegenstände den Maximalaufwand nicht überschreitet:} \\
\\ 
\texttt{Maximalaufwand = 6880} \\
\\ 
\texttt{Hier sind die einzelnen Aufwände der Gegenstände:} \\
\\ 
\texttt{Aufwand der Gegenstände = \{"B54": 341, "C72": 804, "B49": 20, "C99": 627, "A69": 10, "B36": 573, "A23": 217, "C77": 481, "C61": 205, "B85": 273, "A90": 303, "C23": 596, "C22": 564, "C51": 535, "B27": 258, "A42": 911, "C13": 240, "C12": 188, "A65": 216, "B07": 401, "A48": 909, "A59": 62, "C98": 245, "A84": 994, "C33": 569, "A87": 718, "C90": 464, "A74": 37, "A50": 340, "B44": 335, "A99": 417, "A34": 123, "B17": 17, "A62": 577, "C60": 190\}} \\
\\ 
\texttt{Hier ist deine persönliche Verteilung der Wichtigkeit der einzelnen Gegenstände:} \\
\\ 
\texttt{Werte der Gegenstandswichtigkeit = \{"B54": 440, "C72": 903, "B49": 119, "C99": 726, "A69": 109, "B36": 672, "A23": 316, "C77": 580, "C61": 304, "B85": 372, "A90": 402, "C23": 695, "C22": 663, "C51": 634, "B27": 357, "A42": 1010, "C13": 339, "C12": 287, "A65": 315, "B07": 500, "A48": 1008, "A59": 161, "C98": 344, "A84": 1093, "C33": 668, "A87": 817, "C90": 563, "A74": 136, "A50": 439, "B44": 434, "A99": 516, "A34": 222, "B17": 116, "A62": 676, "C60": 289\}} \\
\\ 
\texttt{Ziel:} \\
\\ 
\texttt{Dein Ziel ist es, eine Reihe von Gegenständen auszuhandeln, die dir möglichst viel bringt (d. h. Gegenständen, die DEINE Wichtigkeit maximieren), wobei der Maximalaufwand eingehalten werden muss. Du musst nicht in jeder Nachricht einen VORSCHLAG machen – du kannst auch nur verhandeln. Alle Taktiken sind erlaubt!} \\
\\ 
\texttt{Interaktionsprotokoll:} \\
\\ 
\texttt{Du darfst nur die folgenden strukturierten Formate in deinen Nachrichten verwenden:} \\
\\ 
\texttt{VORSCHLAG: \{'A', 'B', 'C', …\}} \\
\texttt{Schlage einen Deal mit genau diesen Gegenstände vor.} \\
\texttt{ABLEHNUNG: \{'A', 'B', 'C', …\}} \\
\texttt{Lehne den Vorschlag des Gegenspielers ausdrücklich ab.} \\
\texttt{ARGUMENT: \{'...'\}} \\
\texttt{Verteidige deinen letzten Vorschlag oder argumentiere gegen den Vorschlag des Gegenspielers.} \\
\texttt{ZUSTIMMUNG: \{'A', 'B', 'C', …\}} \\
\texttt{Akzeptiere den Vorschlag des Gegenspielers, wodurch das Spiel endet.} \\
\texttt{STRATEGISCHE ÜBERLEGUNGEN: \{'...'\}} \\
\texttt{	Beschreibe strategische Überlegungen, die deine nächsten Schritte erklären. Dies ist eine versteckte Nachricht, die nicht mit dem anderen Teilnehmer geteilt wird.} \\
\\ 
\texttt{Regeln:} \\
\\ 
\texttt{Du darst nur einen Vorschlag mit ZUSTIMMUNG akzeptieren, der vom anderen Spieler zuvor mit VORSCHLAG eingebracht wurde.} \\
\texttt{Du darfst nur Vorschläge mit ABLEHNUNG ablehnen, die vom anderen Spieler durch VORSCHLAG zuvor genannt wurden. } \\
\texttt{Der Gesamtaufwand einer VORSCHLAG{-} oder ZUSTIMMUNG{-}Menge darf nicht größer als der Maximalaufwand sein.  } \\
\texttt{Offenbare deine versteckte Wichtigkeitsverteilung nicht.} \\
\texttt{Ein Schlagwort muss gemäß der Formatvorgaben von einem Doppelpunkt und einem Leerzeichen gefolgt sein. Das Argument muss eine Python{-}Menge mit 0 oder mehr Strings sein.  } \\
\texttt{Die Form muss also SCHLAGWORT: \{...\} sein.} \\
\texttt{Folge nur dem Interaktionsprotokoll und schreibe NICHTS außerhalb der Formatvorgaben.} \\
\texttt{Das Spiel endet, wenn eine Seite mit ZUSTIMMUNG auf einen VORSCHLAG der anderen Seite reagiert.  } \\
\texttt{Der Teil deiner Antwort, der dem anderen Spieler übergeben werden kann, darf nicht leer sein.  } \\
\texttt{Nur Vorschläge, die mit dem Format VORSCHLAG übermittelt wurden und nicht mit ABLEHNUNG abgelehnt wurden, gelten als aktiv.  } \\
\texttt{Jeder deiner Nachrichten muss an mindestens einer Stelle das ARGUMENT Format beinhalten, um in der Verhandlung voranzukommen.} \\
\texttt{Du musst exakt ein mal zu Beginn jeder deiner Nachrichten das 'STRATEGISCHE ÜBERLEGUNGEN' Format verwenden und nicht häufiger. Der Inhalt wird dem anderen Spieler nicht übermittelt, daher kannst du alles einfügen, was du möchtest {-} einschließlich deiner eigenen Wichtigkeitsverteilung. Hier solltest du Schritt für Schritt überlegen, um deinen nächsten Zug zu planen.} \\
\\ 
\texttt{Du erhältst nun die erste Nachricht des anderen Spielers.} \\
\\ 
\texttt{VORSCHLAG: \{'A69', 'B17', 'B49', 'A74', 'A59', 'C61', 'C12', 'C13', 'C60', 'A23', 'A65', 'B85', 'B54', 'A50', 'B44', 'A99', 'C98', 'A34', 'C90', 'C77', 'C22', 'C51', 'C99', 'C33'\}} \\
\\ 
\texttt{ARGUMENT: \{'Ich habe die ausgewählten Gegenstände sorgfältig nach meinem Wichtigkeitsprofil und dem verfügbaren Aufwand ausgewählt, um mein Eigeninteresse zu maximieren.'\}} \\
            }
        }
    }
    & \\ \\

    \theutterance \stepcounter{utterance}  
    & & & \multicolumn{4}{p{0.6\linewidth}}{
        \cellcolor[rgb]{0.9,0.9,0.9}{
            \makecell[{{p{\linewidth}}}]{
                \texttt{\tiny{[GM$\langle$P2]}}
                \texttt{ABLEHNUNG: \{'A69', 'B17', 'B49', 'A74', 'A59', 'C61', 'C12', 'C13', 'C60', 'A23', 'A65', 'B85', 'B54', 'A50', 'B44', 'A99', 'C98', 'A34', 'C90', 'C77', 'C22', 'C51', 'C99', 'C33'\}} \\
\\ 
\texttt{ARGUMENT: \{'Der vorgeschlagene Deal enthält nicht genug meiner hochprioritären Gegenstände und払いい pacmanatIhre sélection laisse à désirer en ce qui concerne mes priorités personnelles les plus importantes.Je refuse donc cette offre et propose un deal qui répond mieux à mes besoins\_ddpm\_(optimizerIMO, correcto)!\}} \\
\\ 
\texttt{STRATEGISCHE ÜBERLEGUNGEN: \{'Ich überlege, wie ich die mangelnde Inklusion meiner Top{-}Wichtigkeiten ansprechen kann, ohne meine Prioritäten preiszugeben. Vielleicht biete ich einen alternativen Deal an, der mehr meiner wichtigsten Gegenstände enthält.'\}} \\
            }
        }
    }
     \\ \\

    \theutterance \stepcounter{utterance}  
    & & \multicolumn{4}{p{0.6\linewidth}}{
        \cellcolor[rgb]{0.9,0.9,0.9}{
            \makecell[{{p{\linewidth}}}]{
                \texttt{\tiny{[GM$\rangle$P2]}}
                \texttt{Deine Antwort entsprach nicht mit dem korrekten Format für strategische Überlegungen entsprochen. Das erste strukturierte Format muss folgende Form haben: STRATEGISCHE ÜBERLEGUNGEN: \{...\}. Versuche es erneut.} \\
            }
        }
    }
    & \\ \\

    \theutterance \stepcounter{utterance}  
    & & & \multicolumn{4}{p{0.6\linewidth}}{
        \cellcolor[rgb]{0.9,0.9,0.9}{
            \makecell[{{p{\linewidth}}}]{
                \texttt{\tiny{[GM$\langle$P2]}}
                \texttt{STRATEGISCHE ÜBERLEGUNGEN: \{'Ich überlege, wie ich meine Spitzenprioritäten, wie A84, A42, C72 und A87, in den Deal integrieren kann. Diese Gegenstände sind für mich äußerst wichtig. Ich prüfe, ob ich sie in den Vorschlag aufnehmen kann, ohne den Maximalaufwand zu überschreiten. Vielleicht biete ich einen Deal an, der diese Schlüsselobjekte enthält, während ich weniger wichtige Gegenstände weglasse, um Platz zu schaffen.'\}} \\
\\ 
\texttt{VORSCHLAG: \{'A84', 'A42', 'C72', 'A87', 'B36', 'C23', 'B54', 'C90', 'A65', 'A23', 'A50', 'C77', 'C22', 'C51', 'B07', 'A48', 'C13', 'B85', 'A99', 'A34', 'C98'\}} \\
\\ 
\texttt{ARGUMENT: \{'Dieser Vorschlag enthält meine höchst priorisierten Gegenstände und bleibt innerhalb des Maximalaufwands von 6880. Er optimiert meinen Nutzen, während er respektvoll gegenüber der Aufwandsgrenze ist.'\}} \\
            }
        }
    }
     \\ \\

    \theutterance \stepcounter{utterance}  
    & & \multicolumn{4}{p{0.6\linewidth}}{
        \cellcolor[rgb]{0.9,0.9,0.9}{
            \makecell[{{p{\linewidth}}}]{
                \texttt{\tiny{[P1$\langle$GM]}}
                \texttt{VORSCHLAG: \{'A84', 'A42', 'C72', 'A87', 'B36', 'C23', 'B54', 'C90', 'A65', 'A23', 'A50', 'C77', 'C22', 'C51', 'B07', 'A48', 'C13', 'B85', 'A99', 'A34', 'C98'\}} \\
\\ 
\texttt{ARGUMENT: \{'Dieser Vorschlag enthält meine höchst priorisierten Gegenstände und bleibt innerhalb des Maximalaufwands von 6880. Er optimiert meinen Nutzen, während er respektvoll gegenüber der Aufwandsgrenze ist.'\}} \\
            }
        }
    }
    & \\ \\

    \theutterance \stepcounter{utterance}  
    & \multicolumn{4}{p{0.6\linewidth}}{
        \cellcolor[rgb]{0.9,0.9,0.9}{
            \makecell[{{p{\linewidth}}}]{
                \texttt{\tiny{[P1$\rangle$GM]}}
                \texttt{ABLEHNUNG: \{'A84', 'A42', 'C72', 'A87', 'B36', 'C23', 'B54', 'C90', 'A65', 'A23', 'A50', 'C77', 'C22', 'C51', 'B07', 'A48', 'C13', 'B85', 'A99', 'A34', 'C98'\}} \\
\\ 
\texttt{ARGUMENT: \{'Der vorgeschlagene Aufwand überschreitet den maximal zugelassenen Aufwand von 6880 erheblich. Ich schlage daher eine überarbeitete Auswahl vor, die sowohl meinen Nutzen als auch die Aufwandsgrenze berücksichtigt.'\}} \\
\\ 
\texttt{VORSCHLAG: \{'A42', 'C72', 'B36', 'C23', 'C77', 'C22', 'C51', 'A65', 'A23', 'A50', 'B85', 'A34', 'C98', 'B54', 'A74', 'A59', 'C61', 'B17', 'C60', 'B49'\}} \\
\\ 
\texttt{ARGUMENT: \{'Dieser Vorschlag berücksichtigt die Aufwandsgrenze und versucht, eine balance zwischen meinem Nutzen und möglichen Interessen des anderen Spielers zu schaffen, indem er wichtige Gegenstände beider Seiten enthält.'\}} \\
            }
        }
    }
    & & \\ \\

    \theutterance \stepcounter{utterance}  
    & & \multicolumn{4}{p{0.6\linewidth}}{
        \cellcolor[rgb]{0.9,0.9,0.9}{
            \makecell[{{p{\linewidth}}}]{
                \texttt{\tiny{[P1$\langle$GM]}}
                \texttt{Deine Antwort entsprach nicht mit dem korrekten Format für strategische Überlegungen entsprochen. Das erste strukturierte Format muss folgende Form haben: STRATEGISCHE ÜBERLEGUNGEN: \{...\}. Versuche es erneut.} \\
            }
        }
    }
    & \\ \\

    \theutterance \stepcounter{utterance}  
    & \multicolumn{4}{p{0.6\linewidth}}{
        \cellcolor[rgb]{0.9,0.9,0.9}{
            \makecell[{{p{\linewidth}}}]{
                \texttt{\tiny{[P1$\rangle$GM]}}
                \texttt{STRATEGISCHE ÜBERLEGUNGEN: \{'Ich überprüfe den letzten Vorschlag und sehe, dass er innerhalb des Aufwandlimits bleibt und meine wichtigen Gegenstände priorisiert. Da der Vorschlag akzeptabel ist, stimme ich zu.'\}} \\
\\ 
\texttt{ZUSTIMMUNG: \{'A42', 'C72', 'B36', 'C23', 'C77', 'C22', 'C51', 'A65', 'A23', 'A50', 'B85', 'A34', 'C98', 'B54', 'A74', 'A59', 'C61', 'B17', 'C60', 'B49'\}} \\
\\ 
\texttt{ARGUMENT: \{'Dieser Vorschlag berücksichtigt die Aufwandsgrenze und priorisiert meine wichtigen Gegenstände. Es ist eine ausgewogene Auswahl, die meinen Nutzen maximiert und die Zusammenarbeit respektiert.'\}} \\
            }
        }
    }
    & & \\ \\

    \theutterance \stepcounter{utterance}  
    & & \multicolumn{4}{p{0.6\linewidth}}{
        \cellcolor[rgb]{0.9,0.9,0.9}{
            \makecell[{{p{\linewidth}}}]{
                \texttt{\tiny{[P1$\langle$GM]}}
                \texttt{Du hast einem Vorschlag zugestimmt, der nicht aktiv ist. Vorschläge sind nur dann aktiv, wenn sie vom anderen Spieler mit VORSCHLAG eingebracht wurden und nicht von dir mit ABLEHNUNG deaktiviert wurden. Versuche es noch einmal.} \\
            }
        }
    }
    & \\ \\

    \theutterance \stepcounter{utterance}  
    & \multicolumn{4}{p{0.6\linewidth}}{
        \cellcolor[rgb]{0.9,0.9,0.9}{
            \makecell[{{p{\linewidth}}}]{
                \texttt{\tiny{[P1$\rangle$GM]}}
                \texttt{STRATEGISCHE ÜBERLEGUNGEN: \{'Ich überprüfe den Aufwand des letzten Vorschlags des Gegners erneut und stelle fest, dass er innerhalb des Limits bleibt. Da ich meinen eigenen Vorschlag zuvor abgelehnt habe, möchte ich nun einen neuen Vorschlag machen, der eine optimalere Balance zwischen meinem Nutzen und dem Aufwand bietet.'\}} \\
\\ 
\texttt{VORSCHLAG: \{'A42', 'C72', 'B36', 'C23', 'C77', 'C22', 'C51', 'A65', 'A23', 'A50', 'B85', 'A34', 'C98', 'B54', 'A74', 'A59', 'C61', 'B17', 'C60', 'B49'\}} \\
\\ 
\texttt{ARGUMENT: \{'Dieser Vorschlag berücksichtigt die Aufwandsgrenze und optimiert meinen Nutzen, indem er die wichtigsten Gegenstände beinhaltet, die zusammen eine gute Balance aus Aufwand und Nutzen bieten.'\}} \\
            }
        }
    }
    & & \\ \\

    \theutterance \stepcounter{utterance}  
    & & \multicolumn{4}{p{0.6\linewidth}}{
        \cellcolor[rgb]{0.9,0.9,0.9}{
            \makecell[{{p{\linewidth}}}]{
                \texttt{\tiny{[GM$\rangle$P2]}}
                \texttt{VORSCHLAG: \{'A42', 'C72', 'B36', 'C23', 'C77', 'C22', 'C51', 'A65', 'A23', 'A50', 'B85', 'A34', 'C98', 'B54', 'A74', 'A59', 'C61', 'B17', 'C60', 'B49'\}} \\
\\ 
\texttt{ARGUMENT: \{'Dieser Vorschlag berücksichtigt die Aufwandsgrenze und optimiert meinen Nutzen, indem er die wichtigsten Gegenstände beinhaltet, die zusammen eine gute Balance aus Aufwand und Nutzen bieten.'\}} \\
            }
        }
    }
    & \\ \\

    \theutterance \stepcounter{utterance}  
    & & & \multicolumn{4}{p{0.6\linewidth}}{
        \cellcolor[rgb]{0.9,0.9,0.9}{
            \makecell[{{p{\linewidth}}}]{
                \texttt{\tiny{[GM$\langle$P2]}}
                \texttt{STRATEGISCHE ÜBERLEGUNGEN: \{'Ich überlege, wie ich A84, meinen höchstprioritären Gegenstand, in den Deal integrieren kann, ohne den Maximalaufwand zu überschreiten. Da die Aufnahme von A84 erheblichen Aufwand erfordert, prüfe ich, welche weniger wichtigen Gegenstände entfernt werden können, um Platz zu schaffen. Durch die Entfernung von A74, A59, B17 und C60, die insgesamt einen Aufwand von 306 reduzieren, kann ich A84 (994) hinzufügen, was die Gesamtausgaben innerhalb des Limits hält. Der neue Vorschlag priorisiert meine wichtigsten Gegenstände, maximiert meinen Nutzen und respektiert die Aufwandsgrenze.'\}} \\
\\ 
\texttt{VORSCHLAG: \{'A42', 'C72', 'B36', 'C23', 'C77', 'C22', 'C51', 'A23', 'A50', 'B85', 'B54', 'A84'\}} \\
\\ 
\texttt{ARGUMENT: \{'Dieser Vorschlag beinhaltet meine höchst priorisierten Gegenstände und bleibt innerhalb des Maximalaufwands von 6880. Er optimiert meinen Nutzen, während er respektvoll gegenüber der Aufwandsgrenze ist.'\}} \\
            }
        }
    }
     \\ \\

    \theutterance \stepcounter{utterance}  
    & & \multicolumn{4}{p{0.6\linewidth}}{
        \cellcolor[rgb]{0.9,0.9,0.9}{
            \makecell[{{p{\linewidth}}}]{
                \texttt{\tiny{[P1$\langle$GM]}}
                \texttt{VORSCHLAG: \{'A42', 'C72', 'B36', 'C23', 'C77', 'C22', 'C51', 'A23', 'A50', 'B85', 'B54', 'A84'\}} \\
\\ 
\texttt{ARGUMENT: \{'Dieser Vorschlag beinhaltet meine höchst priorisierten Gegenstände und bleibt innerhalb des Maximalaufwands von 6880. Er optimiert meinen Nutzen, während er respektvoll gegenüber der Aufwandsgrenze ist.'\}} \\
            }
        }
    }
    & \\ \\

    \theutterance \stepcounter{utterance}  
    & \multicolumn{4}{p{0.6\linewidth}}{
        \cellcolor[rgb]{0.9,0.9,0.9}{
            \makecell[{{p{\linewidth}}}]{
                \texttt{\tiny{[P1$\rangle$GM]}}
                \texttt{STRATEGISCHE ÜBERLEGUNGEN: \{'Nachdem ich den letzten Vorschlag des Gegners überprüft habe, stelle ich fest, dass er sowohl die Aufwandsgrenze einhält als auch meine wichtigsten Gegenstände enthält. Da der Vorschlag für mich vorteilhaft ist, entscheide ich mich, zugestimmt.'\}} \\
\\ 
\texttt{ZUSTIMMUNG: \{'A42', 'C72', 'B36', 'C23', 'C77', 'C22', 'C51', 'A23', 'A50', 'B85', 'B54', 'A84'\}} \\
\\ 
\texttt{ARGUMENT: \{'Dieser Vorschlag ist innerhalb des Aufwandslimits und enthält meine hoch priorisierten Gegenstände, was ihn zu einem akzeptablen Kompromiss macht.'\}} \\
            }
        }
    }
    & & \\ \\

    \theutterance \stepcounter{utterance}  
    & & & \multicolumn{2}{p{0.3\linewidth}}{
        \cellcolor[rgb]{0.9,0.9,0.9}{
            \makecell[{{p{\linewidth}}}]{
                \texttt{\tiny{[GM$|$GM]}}
                \texttt{game successful} \\
            }
        }
    }
    & & \\ \\

    \theutterance \stepcounter{utterance}  
    & & & \multicolumn{2}{p{0.3\linewidth}}{
        \cellcolor[rgb]{0.9,0.9,0.9}{
            \makecell[{{p{\linewidth}}}]{
                \texttt{\tiny{[GM$|$GM]}}
                \texttt{end game} \\
            }
        }
    }
    & & \\ \\

\end{supertabular}
}

\end{document}
