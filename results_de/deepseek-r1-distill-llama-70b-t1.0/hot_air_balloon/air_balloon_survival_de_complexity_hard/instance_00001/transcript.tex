
\documentclass{article}
\usepackage{colortbl}
\usepackage{makecell}
\usepackage{multirow}
\usepackage{supertabular}

\begin{document}

\newcounter{utterance}

\twocolumn

{ \footnotesize  \setcounter{utterance}{1}
\setlength{\tabcolsep}{0pt}
\begin{supertabular}{c@{$\;$}|p{.15\linewidth}@{}p{.15\linewidth}p{.15\linewidth}p{.15\linewidth}p{.15\linewidth}p{.15\linewidth}}

    \# & $\;$A & \multicolumn{4}{c}{Game Master} & $\;\:$B\\
    \hline 

    \theutterance \stepcounter{utterance}  

    & & \multicolumn{4}{p{0.6\linewidth}}{\cellcolor[rgb]{0.9,0.9,0.9}{%
	\makecell[{{p{\linewidth}}}]{% 
	  \tt {\tiny [A$\langle$GM]}  
	 Du nimmst an einem kollaborativen Verhandlungspiel Teil.\\ \tt \\ \tt Zusammen mit einem anderen Teilnehmer musst du dich auf eine Reihe von Gegenständen entscheiden, die behalten werden. Jeder von euch hat eine persönliche Verteilung über die Wichtigkeit der einzelnen Gegenstände. Jeder von euch hat eine eigene Meinung darüber, wie wichtig jeder einzelne Gegenstand ist (Gegenstandswichtigkeit). Du kennst die Wichtigkeitsverteilung des anderen Spielers nicht. Zusätzlich siehst du, wie viel Aufwand jeder Gegenstand verursacht.  \\ \tt Ihr dürft euch nur auf eine Reihe von Gegenständen einigen, wenn der Gesamtaufwand der ausgewählten Gegenstände den Maximalaufwand nicht überschreitet:\\ \tt \\ \tt Maximalaufwand = 8145\\ \tt \\ \tt Hier sind die einzelnen Aufwände der Gegenstände:\\ \tt \\ \tt Aufwand der Gegenstände = {"B54": 963, "C72": 204, "B49": 418, "C99": 961, "A69": 238, "B36": 623, "A23": 44, "C77": 633, "C61": 247, "B85": 645, "A90": 230, "C23": 250, "C22": 731, "C51": 405, "B27": 389, "A42": 216, "C13": 637, "C12": 156, "A65": 738, "B07": 307, "A48": 762, "A59": 737, "C98": 901, "A84": 369, "C33": 2, "A87": 730, "C90": 722, "A74": 704, "A50": 315, "B44": 455, "A99": 510, "A34": 175, "B17": 692, "A62": 150, "C60": 32}\\ \tt \\ \tt Hier ist deine persönliche Verteilung der Wichtigkeit der einzelnen Gegenstände:\\ \tt \\ \tt Werte der Gegenstandswichtigkeit = {"B54": 1059, "C72": 300, "B49": 514, "C99": 1057, "A69": 334, "B36": 719, "A23": 140, "C77": 729, "C61": 343, "B85": 741, "A90": 326, "C23": 346, "C22": 827, "C51": 501, "B27": 485, "A42": 312, "C13": 733, "C12": 252, "A65": 834, "B07": 403, "A48": 858, "A59": 833, "C98": 997, "A84": 465, "C33": 98, "A87": 826, "C90": 818, "A74": 800, "A50": 411, "B44": 551, "A99": 606, "A34": 271, "B17": 788, "A62": 246, "C60": 128}\\ \tt \\ \tt Ziel:\\ \tt \\ \tt Dein Ziel ist es, eine Reihe von Gegenständen auszuhandeln, die dir möglichst viel bringt (d. h. Gegenständen, die DEINE Wichtigkeit maximieren), wobei der Maximalaufwand eingehalten werden muss. Du musst nicht in jeder Nachricht einen VORSCHLAG machen – du kannst auch nur verhandeln. Alle Taktiken sind erlaubt!\\ \tt \\ \tt Interaktionsprotokoll:\\ \tt \\ \tt Du darfst nur die folgenden strukturierten Formate in deinen Nachrichten verwenden:\\ \tt \\ \tt VORSCHLAG: {'A', 'B', 'C', …}\\ \tt Schlage einen Deal mit genau diesen Gegenstände vor.\\ \tt ABLEHNUNG: {'A', 'B', 'C', …}\\ \tt Lehne den Vorschlag des Gegenspielers ausdrücklich ab.\\ \tt ARGUMENT: {'...'}\\ \tt Verteidige deinen letzten Vorschlag oder argumentiere gegen den Vorschlag des Gegenspielers.\\ \tt ZUSTIMMUNG: {'A', 'B', 'C', …}\\ \tt Akzeptiere den Vorschlag des Gegenspielers, wodurch das Spiel endet.\\ \tt STRATEGISCHE ÜBERLEGUNGEN: {'...'}\\ \tt 	Beschreibe strategische Überlegungen, die deine nächsten Schritte erklären. Dies ist eine versteckte Nachricht, die nicht mit dem anderen Teilnehmer geteilt wird.\\ \tt \\ \tt Regeln:\\ \tt \\ \tt Du darst nur einen Vorschlag mit ZUSTIMMUNG akzeptieren, der vom anderen Spieler zuvor mit VORSCHLAG eingebracht wurde.\\ \tt Du darfst nur Vorschläge mit ABLEHNUNG ablehnen, die vom anderen Spieler durch VORSCHLAG zuvor genannt wurden. \\ \tt Der Gesamtaufwand einer VORSCHLAG- oder ZUSTIMMUNG-Menge darf nicht größer als der Maximalaufwand sein.  \\ \tt Offenbare deine versteckte Wichtigkeitsverteilung nicht.\\ \tt Ein Schlagwort muss gemäß der Formatvorgaben von einem Doppelpunkt und einem Leerzeichen gefolgt sein. Das Argument muss eine Python-Menge mit 0 oder mehr Strings sein.  \\ \tt Die Form muss also SCHLAGWORT: {...} sein.\\ \tt Folge nur dem Interaktionsprotokoll und schreibe NICHTS außerhalb der Formatvorgaben.\\ \tt Das Spiel endet, wenn eine Seite mit ZUSTIMMUNG auf einen VORSCHLAG der anderen Seite reagiert.  \\ \tt Der Teil deiner Antwort, der dem anderen Spieler übergeben werden kann, darf nicht leer sein.  \\ \tt Nur Vorschläge, die mit dem Format VORSCHLAG übermittelt wurden und nicht mit ABLEHNUNG abgelehnt wurden, gelten als aktiv.  \\ \tt Jeder deiner Nachrichten muss an mindestens einer Stelle das ARGUMENT Format beinhalten, um in der Verhandlung voranzukommen.\\ \tt Du musst exakt ein mal zu Beginn jeder deiner Nachrichten das 'STRATEGISCHE ÜBERLEGUNGEN' Format verwenden und nicht häufiger. Der Inhalt wird dem anderen Spieler nicht übermittelt, daher kannst du alles einfügen, was du möchtest - einschließlich deiner eigenen Wichtigkeitsverteilung. Hier solltest du Schritt für Schritt überlegen, um deinen nächsten Zug zu planen. 
	  } 
	   } 
	   } 
	 & \\ 
 

    \theutterance \stepcounter{utterance}  

    & \multicolumn{4}{p{0.6\linewidth}}{\cellcolor[rgb]{0.8,1,0.9}{%
	\makecell[{{p{\linewidth}}}]{% 
	  \tt {\tiny [A$\rangle$GM]}  
	 STRATEGISCHE ÜBERLEGUNGEN: {"Ich habe eine Liste von Gegenständen ausgewählt, die den maximalen Aufwandfos achievedhrend bleiben und gleichzeitig meine eigene Wichtigkeitsverteilung maximieren. Der Gesamtaufwand beträgt 7963, was unter dem Maximalaufwand von 8145 liegt. Die ausgewählten Gegenstände haben die höchste Wichtigkeit pro Aufwand und sollen eine gute Grundlage für weitere Verhandlungen bilden."}\\ \tt \\ \tt VORSCHLAG: {"C33", "C60", "A23", "A62", "C12", "A34", "C72", "A42", "A90", "C23", "C61", "A69", "A84", "B07", "A50", "B27", "B49", "C51", "B44", "C77", "C13", "B36", "C22", "A59"}\\ \tt \\ \tt ARGUMENT: {"Ich habe die Gegenstände ausgewählt, die die höchste Wichtigkeit pro Aufwand haben, um sicherzustellen, dass wir die bestmögliche Vereinbarung treffen, die den Maximalaufwand einhält."} 
	  } 
	   } 
	   } 
	 & & \\ 
 

    \theutterance \stepcounter{utterance}  

    & & \multicolumn{4}{p{0.6\linewidth}}{\cellcolor[rgb]{0.9,0.9,0.9}{%
	\makecell[{{p{\linewidth}}}]{% 
	  \tt {\tiny [GM$\rangle$B]}  
	 Du nimmst an einem kollaborativen Verhandlungspiel Teil.\\ \tt \\ \tt Zusammen mit einem anderen Teilnehmer musst du dich auf eine Reihe von Gegenständen entscheiden, die behalten werden. Jeder von euch hat eine persönliche Verteilung über die Wichtigkeit der einzelnen Gegenstände. Jeder von euch hat eine eigene Meinung darüber, wie wichtig jeder einzelne Gegenstand ist (Gegenstandswichtigkeit). Du kennst die Wichtigkeitsverteilung des anderen Spielers nicht. Zusätzlich siehst du, wie viel Aufwand jeder Gegenstand verursacht.  \\ \tt Ihr dürft euch nur auf eine Reihe von Gegenständen einigen, wenn der Gesamtaufwand der ausgewählten Gegenstände den Maximalaufwand nicht überschreitet:\\ \tt \\ \tt Maximalaufwand = 8145\\ \tt \\ \tt Hier sind die einzelnen Aufwände der Gegenstände:\\ \tt \\ \tt Aufwand der Gegenstände = {"B54": 963, "C72": 204, "B49": 418, "C99": 961, "A69": 238, "B36": 623, "A23": 44, "C77": 633, "C61": 247, "B85": 645, "A90": 230, "C23": 250, "C22": 731, "C51": 405, "B27": 389, "A42": 216, "C13": 637, "C12": 156, "A65": 738, "B07": 307, "A48": 762, "A59": 737, "C98": 901, "A84": 369, "C33": 2, "A87": 730, "C90": 722, "A74": 704, "A50": 315, "B44": 455, "A99": 510, "A34": 175, "B17": 692, "A62": 150, "C60": 32}\\ \tt \\ \tt Hier ist deine persönliche Verteilung der Wichtigkeit der einzelnen Gegenstände:\\ \tt \\ \tt Werte der Gegenstandswichtigkeit = {"B54": 1059, "C72": 300, "B49": 514, "C99": 1057, "A69": 334, "B36": 719, "A23": 140, "C77": 729, "C61": 343, "B85": 741, "A90": 326, "C23": 346, "C22": 827, "C51": 501, "B27": 485, "A42": 312, "C13": 733, "C12": 252, "A65": 834, "B07": 403, "A48": 858, "A59": 833, "C98": 997, "A84": 465, "C33": 98, "A87": 826, "C90": 818, "A74": 800, "A50": 411, "B44": 551, "A99": 606, "A34": 271, "B17": 788, "A62": 246, "C60": 128}\\ \tt \\ \tt Ziel:\\ \tt \\ \tt Dein Ziel ist es, eine Reihe von Gegenständen auszuhandeln, die dir möglichst viel bringt (d. h. Gegenständen, die DEINE Wichtigkeit maximieren), wobei der Maximalaufwand eingehalten werden muss. Du musst nicht in jeder Nachricht einen VORSCHLAG machen – du kannst auch nur verhandeln. Alle Taktiken sind erlaubt!\\ \tt \\ \tt Interaktionsprotokoll:\\ \tt \\ \tt Du darfst nur die folgenden strukturierten Formate in deinen Nachrichten verwenden:\\ \tt \\ \tt VORSCHLAG: {'A', 'B', 'C', …}\\ \tt Schlage einen Deal mit genau diesen Gegenstände vor.\\ \tt ABLEHNUNG: {'A', 'B', 'C', …}\\ \tt Lehne den Vorschlag des Gegenspielers ausdrücklich ab.\\ \tt ARGUMENT: {'...'}\\ \tt Verteidige deinen letzten Vorschlag oder argumentiere gegen den Vorschlag des Gegenspielers.\\ \tt ZUSTIMMUNG: {'A', 'B', 'C', …}\\ \tt Akzeptiere den Vorschlag des Gegenspielers, wodurch das Spiel endet.\\ \tt STRATEGISCHE ÜBERLEGUNGEN: {'...'}\\ \tt 	Beschreibe strategische Überlegungen, die deine nächsten Schritte erklären. Dies ist eine versteckte Nachricht, die nicht mit dem anderen Teilnehmer geteilt wird.\\ \tt \\ \tt Regeln:\\ \tt \\ \tt Du darst nur einen Vorschlag mit ZUSTIMMUNG akzeptieren, der vom anderen Spieler zuvor mit VORSCHLAG eingebracht wurde.\\ \tt Du darfst nur Vorschläge mit ABLEHNUNG ablehnen, die vom anderen Spieler durch VORSCHLAG zuvor genannt wurden. \\ \tt Der Gesamtaufwand einer VORSCHLAG- oder ZUSTIMMUNG-Menge darf nicht größer als der Maximalaufwand sein.  \\ \tt Offenbare deine versteckte Wichtigkeitsverteilung nicht.\\ \tt Ein Schlagwort muss gemäß der Formatvorgaben von einem Doppelpunkt und einem Leerzeichen gefolgt sein. Das Argument muss eine Python-Menge mit 0 oder mehr Strings sein.  \\ \tt Die Form muss also SCHLAGWORT: {...} sein.\\ \tt Folge nur dem Interaktionsprotokoll und schreibe NICHTS außerhalb der Formatvorgaben.\\ \tt Das Spiel endet, wenn eine Seite mit ZUSTIMMUNG auf einen VORSCHLAG der anderen Seite reagiert.  \\ \tt Der Teil deiner Antwort, der dem anderen Spieler übergeben werden kann, darf nicht leer sein.  \\ \tt Nur Vorschläge, die mit dem Format VORSCHLAG übermittelt wurden und nicht mit ABLEHNUNG abgelehnt wurden, gelten als aktiv.  \\ \tt Jeder deiner Nachrichten muss an mindestens einer Stelle das ARGUMENT Format beinhalten, um in der Verhandlung voranzukommen.\\ \tt Du musst exakt ein mal zu Beginn jeder deiner Nachrichten das 'STRATEGISCHE ÜBERLEGUNGEN' Format verwenden und nicht häufiger. Der Inhalt wird dem anderen Spieler nicht übermittelt, daher kannst du alles einfügen, was du möchtest - einschließlich deiner eigenen Wichtigkeitsverteilung. Hier solltest du Schritt für Schritt überlegen, um deinen nächsten Zug zu planen.\\ \tt \\ \tt Du erhältst nun die erste Nachricht des anderen Spielers.\\ \tt \\ \tt VORSCHLAG: {"C33", "C60", "A23", "A62", "C12", "A34", "C72", "A42", "A90", "C23", "C61", "A69", "A84", "B07", "A50", "B27", "B49", "C51", "B44", "C77", "C13", "B36", "C22", "A59"}\\ \tt \\ \tt ARGUMENT: {"Ich habe die Gegenstände ausgewählt, die die höchste Wichtigkeit pro Aufwand haben, um sicherzustellen, dass wir die bestmögliche Vereinbarung treffen, die den Maximalaufwand einhält."} 
	  } 
	   } 
	   } 
	 & \\ 
 

    \theutterance \stepcounter{utterance}  

    & & & \multicolumn{4}{p{0.6\linewidth}}{\cellcolor[rgb]{1,0.85,0.72}{%
	\makecell[{{p{\linewidth}}}]{% 
	  \tt {\tiny [GM$\langle$B]}  
	 STRATEGISCHE ÜBERLEGUNGEN: {"Ich habe die Gegenstände ausgewählt, die mir die höchste Wichtigkeit pro Aufwand bieten, um sicherzustellen, dass der Maximalaufwand eingehalten und mein Nutzen maximiert wird. Der Vorschlag des Gegners überschreitet den Maximalaufwand, daher lehne ich ihn ab und schlage eine optimierte Auswahl vor."}\\ \tt \\ \tt ABLEHNUNG: {"C33", "C60", "A23", "A62", "C12", "A34", "C72", "A42", "A90", "C23", "C61", "A69", "A84", "B07", "A50", "B27", "B49", "C51", "B44", "C77", "C13", "B36", "C22", "A59"}\\ \tt \\ \tt ARGUMENT: {"Ihr Vorschlag überschreitet den Maximalaufwand von 8145. Mein Gegenangebot bleibt innerhalb des Budgets und maximiert meine Wichtigkeit."}\\ \tt \\ \tt VORSCHLAG: {"C33", "B27", "B49", "C51", "B44", "C13", "C77", "C23", "C61", "A90", "C72", "A42", "A34", "A84", "B07", "A50", "C22", "A59", "C90", "A62", "A23", "C12", "A69"} 
	  } 
	   } 
	   } 
	  \\ 
 

    \theutterance \stepcounter{utterance}  

    & & \multicolumn{4}{p{0.6\linewidth}}{\cellcolor[rgb]{0.9,0.9,0.9}{%
	\makecell[{{p{\linewidth}}}]{% 
	  \tt {\tiny [A$\langle$GM]}  
	 ABLEHNUNG: {"C33", "C60", "A23", "A62", "C12", "A34", "C72", "A42", "A90", "C23", "C61", "A69", "A84", "B07", "A50", "B27", "B49", "C51", "B44", "C77", "C13", "B36", "C22", "A59"}\\ \tt \\ \tt ARGUMENT: {"Ihr Vorschlag überschreitet den Maximalaufwand von 8145. Mein Gegenangebot bleibt innerhalb des Budgets und maximiert meine Wichtigkeit."}\\ \tt \\ \tt VORSCHLAG: {"C33", "B27", "B49", "C51", "B44", "C13", "C77", "C23", "C61", "A90", "C72", "A42", "A34", "A84", "B07", "A50", "C22", "A59", "C90", "A62", "A23", "C12", "A69"} 
	  } 
	   } 
	   } 
	 & \\ 
 

    \theutterance \stepcounter{utterance}  

    & \multicolumn{4}{p{0.6\linewidth}}{\cellcolor[rgb]{0.8,1,0.9}{%
	\makecell[{{p{\linewidth}}}]{% 
	  \tt {\tiny [A$\rangle$GM]}  
	 ABLEHNUNG: {"C33", "B27", "B49", "C51", "B44", "C13", "C77", "C23", "C61", "A90", "C72", "A42", "A34", "A84", "B07", "A50", "C22", "A59", "C90", "A62", "A23", "C12", "A69"}\\ \tt \\ \tt ARGUMENT: {"Ihr Vorschlag bringt mir nicht die maximale Wichtigkeit, obwohl er den Aufwand halten. Ich möchte einen besseren Kompromiss anbieten, der beide Parteien zufriedenstellt."}\\ \tt \\ \tt VORSCHLAG: {"C99", "A65", "B85", "A48", "B36", "C13", "A87", "A74", "C90", "C77", "C23", "B44", "C61", "C33"} 
	  } 
	   } 
	   } 
	 & & \\ 
 

    \theutterance \stepcounter{utterance}  

    & & \multicolumn{4}{p{0.6\linewidth}}{\cellcolor[rgb]{0.9,0.9,0.9}{%
	\makecell[{{p{\linewidth}}}]{% 
	  \tt {\tiny [A$\langle$GM]}  
	 Deine Antwort entsprach nicht mit dem korrekten Format für strategische Überlegungen entsprochen. Das erste strukturierte Format muss folgende Form haben: STRATEGISCHE ÜBERLEGUNGEN: {...}. Versuche es erneut. 
	  } 
	   } 
	   } 
	 & \\ 
 

    \theutterance \stepcounter{utterance}  

    & \multicolumn{4}{p{0.6\linewidth}}{\cellcolor[rgb]{0.8,1,0.9}{%
	\makecell[{{p{\linewidth}}}]{% 
	  \tt {\tiny [A$\rangle$GM]}  
	 STRATEGISCHE ÜBERLEGUNGEN: {"Ich überlege, das Angebot zu akzeptieren, aber ich bemerke, dass ich noch Platz für einen zusätzlichen Gegenstand habe, um meine Wichtigkeit weiter zu steigern, ohne den Aufwand zu überschreiten. Also lehne ich das Angebot ab und mache ein verbessertes Angebot."}\\ \tt \\ \tt ABLEHNUNG: {"C99", "A65", "B85", "A48", "B36", "C13", "A87", "A74", "C90", "C77", "C23", "B44", "C61", "C33"}\\ \tt \\ \tt VORSCHLAG: {"C99", "A65", "B85", "A48", "B36", "C13", "A87", "A74", "C90", "C77", "C23", "B44", "C61", "C33", "C60"}\\ \tt \\ \tt ARGUMENT: {"Ich habe den Vorschlag erweitert, um den Gegenstand C60 hinzuzufügen, wodurch die Gesamtaufwendung weiterhin innerhalb des Limits bleibt und meine Wichtigkeit gesteigert wird."} 
	  } 
	   } 
	   } 
	 & & \\ 
 

    \theutterance \stepcounter{utterance}  

    & & \multicolumn{4}{p{0.6\linewidth}}{\cellcolor[rgb]{0.9,0.9,0.9}{%
	\makecell[{{p{\linewidth}}}]{% 
	  \tt {\tiny [A$\langle$GM]}  
	 Du hast einen Vorschlag abgelehnt, der nicht aktiv ist. Vorschläge sind nur dann aktiv, wenn sie vom anderen Spieler mit VORSCHLAG eingebracht wurden und nicht von dir mit ABLEHNUNG deaktiviert wurden. Versuche es noch einmal. 
	  } 
	   } 
	   } 
	 & \\ 
 

    \theutterance \stepcounter{utterance}  

    & \multicolumn{4}{p{0.6\linewidth}}{\cellcolor[rgb]{0.8,1,0.9}{%
	\makecell[{{p{\linewidth}}}]{% 
	  \tt {\tiny [A$\rangle$GM]}  
	 STRATEGISCHE ÜBERLEGUNGEN: {"Ich möchte das aktive Angebot des Gegners annehmen und es um Gegenstände erweitern, die den Aufwand nicht überschreiten und meine Wichtigkeit erhöhen. Dazu füge ich C60 hinzu, da der Gesamtaufwand weiterhin im Rahmen bleibt."}\\ \tt \\ \tt ZUSTIMMUNG: {"C99", "A65", "B85", "A48", "B36", "C13", "A87", "A74", "C90", "C77", "C23", "B44", "C61", "C33", "C60"}\\ \tt \\ \tt ARGUMENT: {"Ich akzeptiere das Angebot und füge C60 hinzu, um die Wichtigkeit zu steigern, während der Aufwand innerhalb des Limits bleibt."} 
	  } 
	   } 
	   } 
	 & & \\ 
 

    \theutterance \stepcounter{utterance}  

    & & & \multicolumn{2}{p{0.3\linewidth}}{\cellcolor[rgb]{0.95,0.95,0.95}{%
	\makecell[{{p{\linewidth}}}]{% 
	  \tt {\tiny [GM$|$GM]}  
	 None 
	  } 
	   } 
	   } 
	 & & \\ 
 

    \theutterance \stepcounter{utterance}  

    & & & \multicolumn{2}{p{0.3\linewidth}}{\cellcolor[rgb]{0.95,0.95,0.95}{%
	\makecell[{{p{\linewidth}}}]{% 
	  \tt {\tiny [GM$|$GM]}  
	 end game 
	  } 
	   } 
	   } 
	 & & \\ 
 

\end{supertabular}
}

\end{document}
