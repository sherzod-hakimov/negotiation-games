
\documentclass{article}
\usepackage{colortbl}
\usepackage{makecell}
\usepackage{multirow}
\usepackage{supertabular}

\begin{document}

\newcounter{utterance}

\twocolumn

{ \footnotesize  \setcounter{utterance}{1}
\setlength{\tabcolsep}{0pt}
\begin{supertabular}{c@{$\;$}|p{.15\linewidth}@{}p{.15\linewidth}p{.15\linewidth}p{.15\linewidth}p{.15\linewidth}p{.15\linewidth}}

    \# & $\;$A & \multicolumn{4}{c}{Game Master} & $\;\:$B\\
    \hline 

    \theutterance \stepcounter{utterance}  

    & & \multicolumn{4}{p{0.6\linewidth}}{\cellcolor[rgb]{0.9,0.9,0.9}{%
	\makecell[{{p{\linewidth}}}]{% 
	  \tt {\tiny [A$\langle$GM]}  
	 Du nimmst an einem kollaborativen Verhandlungspiel Teil.\\ \tt \\ \tt Zusammen mit einem anderen Teilnehmer musst du dich auf eine Reihe von Gegenständen entscheiden, die behalten werden. Jeder von euch hat eine persönliche Verteilung über die Wichtigkeit der einzelnen Gegenstände. Jeder von euch hat eine eigene Meinung darüber, wie wichtig jeder einzelne Gegenstand ist (Gegenstandswichtigkeit). Du kennst die Wichtigkeitsverteilung des anderen Spielers nicht. Zusätzlich siehst du, wie viel Aufwand jeder Gegenstand verursacht.  \\ \tt Ihr dürft euch nur auf eine Reihe von Gegenständen einigen, wenn der Gesamtaufwand der ausgewählten Gegenstände den Maximalaufwand nicht überschreitet:\\ \tt \\ \tt Maximalaufwand = 5138\\ \tt \\ \tt Hier sind die einzelnen Aufwände der Gegenstände:\\ \tt \\ \tt Aufwand der Gegenstände = {"C33": 903, "A89": 757, "A48": 850, "A04": 746, "A41": 524, "C64": 204, "C49": 946, "A61": 473, "C04": 616, "B53": 855, "B85": 530, "B61": 419, "A45": 960, "A03": 763, "B21": 730}\\ \tt \\ \tt Hier ist deine persönliche Verteilung der Wichtigkeit der einzelnen Gegenstände:\\ \tt \\ \tt Werte der Gegenstandswichtigkeit = {"C33": 138, "A89": 583, "A48": 868, "A04": 822, "A41": 783, "C64": 65, "C49": 262, "A61": 121, "C04": 508, "B53": 780, "B85": 461, "B61": 484, "A45": 668, "A03": 389, "B21": 808}\\ \tt \\ \tt Ziel:\\ \tt \\ \tt Dein Ziel ist es, eine Reihe von Gegenständen auszuhandeln, die dir möglichst viel bringt (d. h. Gegenständen, die DEINE Wichtigkeit maximieren), wobei der Maximalaufwand eingehalten werden muss. Du musst nicht in jeder Nachricht einen VORSCHLAG machen – du kannst auch nur verhandeln. Alle Taktiken sind erlaubt!\\ \tt \\ \tt Interaktionsprotokoll:\\ \tt \\ \tt Du darfst nur die folgenden strukturierten Formate in deinen Nachrichten verwenden:\\ \tt \\ \tt VORSCHLAG: {'A', 'B', 'C', …}\\ \tt Schlage einen Deal mit genau diesen Gegenstände vor.\\ \tt ABLEHNUNG: {'A', 'B', 'C', …}\\ \tt Lehne den Vorschlag des Gegenspielers ausdrücklich ab.\\ \tt ARGUMENT: {'...'}\\ \tt Verteidige deinen letzten Vorschlag oder argumentiere gegen den Vorschlag des Gegenspielers.\\ \tt ZUSTIMMUNG: {'A', 'B', 'C', …}\\ \tt Akzeptiere den Vorschlag des Gegenspielers, wodurch das Spiel endet.\\ \tt STRATEGISCHE ÜBERLEGUNGEN: {'...'}\\ \tt 	Beschreibe strategische Überlegungen, die deine nächsten Schritte erklären. Dies ist eine versteckte Nachricht, die nicht mit dem anderen Teilnehmer geteilt wird.\\ \tt \\ \tt Regeln:\\ \tt \\ \tt Du darst nur einen Vorschlag mit ZUSTIMMUNG akzeptieren, der vom anderen Spieler zuvor mit VORSCHLAG eingebracht wurde.\\ \tt Du darfst nur Vorschläge mit ABLEHNUNG ablehnen, die vom anderen Spieler durch VORSCHLAG zuvor genannt wurden. \\ \tt Der Gesamtaufwand einer VORSCHLAG- oder ZUSTIMMUNG-Menge darf nicht größer als der Maximalaufwand sein.  \\ \tt Offenbare deine versteckte Wichtigkeitsverteilung nicht.\\ \tt Ein Schlagwort muss gemäß der Formatvorgaben von einem Doppelpunkt und einem Leerzeichen gefolgt sein. Das Argument muss eine Python-Menge mit 0 oder mehr Strings sein.  \\ \tt Die Form muss also SCHLAGWORT: {...} sein.\\ \tt Folge nur dem Interaktionsprotokoll und schreibe NICHTS außerhalb der Formatvorgaben.\\ \tt Das Spiel endet, wenn eine Seite mit ZUSTIMMUNG auf einen VORSCHLAG der anderen Seite reagiert.  \\ \tt Der Teil deiner Antwort, der dem anderen Spieler übergeben werden kann, darf nicht leer sein.  \\ \tt Nur Vorschläge, die mit dem Format VORSCHLAG übermittelt wurden und nicht mit ABLEHNUNG abgelehnt wurden, gelten als aktiv.  \\ \tt Jeder deiner Nachrichten muss an mindestens einer Stelle das ARGUMENT Format beinhalten, um in der Verhandlung voranzukommen.\\ \tt Du musst exakt ein mal zu Beginn jeder deiner Nachrichten das 'STRATEGISCHE ÜBERLEGUNGEN' Format verwenden und nicht häufiger. Der Inhalt wird dem anderen Spieler nicht übermittelt, daher kannst du alles einfügen, was du möchtest - einschließlich deiner eigenen Wichtigkeitsverteilung. Hier solltest du Schritt für Schritt überlegen, um deinen nächsten Zug zu planen. 
	  } 
	   } 
	   } 
	 & \\ 
 

    \theutterance \stepcounter{utterance}  

    & \multicolumn{4}{p{0.6\linewidth}}{\cellcolor[rgb]{0.8,1,0.9}{%
	\makecell[{{p{\linewidth}}}]{% 
	  \tt {\tiny [A$\rangle$GM]}  
	 STRATEGISCHE ÜBERLEGUNGEN: {'Ich überlege, welche Gegenstände ich vorschlagen soll, um meine Ziele zu erreichen, während ich auch die möglichen Präferenzen des anderen Spielers berücksichtige. Ich berechne die Kosten-Nutzen-Verhältnisse, um die effizientesten Gegenstände zu priorisieren und einen Vorschlag zu machen, der sowohl meinen als auch potenziell den Bedürfnissen des anderen Spielers gerecht wird.'}\\ \tt \\ \tt VORSCHLAG: {'A41', 'B61', 'B21', 'A04', 'A48', 'B53', 'A89', 'C64'}\\ \tt \\ \tt ARGUMENT: {'Diese Kombination stellt eine ausgewogene und effiziente Auswahl dar, die den Etat nicht überschreitet und eine breite Palette von Gegenständen mit erheblicher Bedeutung abdeckt. Sie verwendet das Budget sinnvoll und sichert damit ein gutes Maß an Zufriedenheit für beide Parteien.'} 
	  } 
	   } 
	   } 
	 & & \\ 
 

    \theutterance \stepcounter{utterance}  

    & & \multicolumn{4}{p{0.6\linewidth}}{\cellcolor[rgb]{0.9,0.9,0.9}{%
	\makecell[{{p{\linewidth}}}]{% 
	  \tt {\tiny [GM$\rangle$B]}  
	 Du nimmst an einem kollaborativen Verhandlungspiel Teil.\\ \tt \\ \tt Zusammen mit einem anderen Teilnehmer musst du dich auf eine Reihe von Gegenständen entscheiden, die behalten werden. Jeder von euch hat eine persönliche Verteilung über die Wichtigkeit der einzelnen Gegenstände. Jeder von euch hat eine eigene Meinung darüber, wie wichtig jeder einzelne Gegenstand ist (Gegenstandswichtigkeit). Du kennst die Wichtigkeitsverteilung des anderen Spielers nicht. Zusätzlich siehst du, wie viel Aufwand jeder Gegenstand verursacht.  \\ \tt Ihr dürft euch nur auf eine Reihe von Gegenständen einigen, wenn der Gesamtaufwand der ausgewählten Gegenstände den Maximalaufwand nicht überschreitet:\\ \tt \\ \tt Maximalaufwand = 5138\\ \tt \\ \tt Hier sind die einzelnen Aufwände der Gegenstände:\\ \tt \\ \tt Aufwand der Gegenstände = {"C33": 903, "A89": 757, "A48": 850, "A04": 746, "A41": 524, "C64": 204, "C49": 946, "A61": 473, "C04": 616, "B53": 855, "B85": 530, "B61": 419, "A45": 960, "A03": 763, "B21": 730}\\ \tt \\ \tt Hier ist deine persönliche Verteilung der Wichtigkeit der einzelnen Gegenstände:\\ \tt \\ \tt Werte der Gegenstandswichtigkeit = {"A48": 65, "A04": 121, "B21": 138, "A41": 262, "B53": 389, "A45": 461, "A89": 484, "C04": 508, "B61": 583, "B85": 668, "A03": 780, "C49": 783, "C33": 808, "A61": 822, "C64": 868}\\ \tt \\ \tt Ziel:\\ \tt \\ \tt Dein Ziel ist es, eine Reihe von Gegenständen auszuhandeln, die dir möglichst viel bringt (d. h. Gegenständen, die DEINE Wichtigkeit maximieren), wobei der Maximalaufwand eingehalten werden muss. Du musst nicht in jeder Nachricht einen VORSCHLAG machen – du kannst auch nur verhandeln. Alle Taktiken sind erlaubt!\\ \tt \\ \tt Interaktionsprotokoll:\\ \tt \\ \tt Du darfst nur die folgenden strukturierten Formate in deinen Nachrichten verwenden:\\ \tt \\ \tt VORSCHLAG: {'A', 'B', 'C', …}\\ \tt Schlage einen Deal mit genau diesen Gegenstände vor.\\ \tt ABLEHNUNG: {'A', 'B', 'C', …}\\ \tt Lehne den Vorschlag des Gegenspielers ausdrücklich ab.\\ \tt ARGUMENT: {'...'}\\ \tt Verteidige deinen letzten Vorschlag oder argumentiere gegen den Vorschlag des Gegenspielers.\\ \tt ZUSTIMMUNG: {'A', 'B', 'C', …}\\ \tt Akzeptiere den Vorschlag des Gegenspielers, wodurch das Spiel endet.\\ \tt STRATEGISCHE ÜBERLEGUNGEN: {'...'}\\ \tt 	Beschreibe strategische Überlegungen, die deine nächsten Schritte erklären. Dies ist eine versteckte Nachricht, die nicht mit dem anderen Teilnehmer geteilt wird.\\ \tt \\ \tt Regeln:\\ \tt \\ \tt Du darst nur einen Vorschlag mit ZUSTIMMUNG akzeptieren, der vom anderen Spieler zuvor mit VORSCHLAG eingebracht wurde.\\ \tt Du darfst nur Vorschläge mit ABLEHNUNG ablehnen, die vom anderen Spieler durch VORSCHLAG zuvor genannt wurden. \\ \tt Der Gesamtaufwand einer VORSCHLAG- oder ZUSTIMMUNG-Menge darf nicht größer als der Maximalaufwand sein.  \\ \tt Offenbare deine versteckte Wichtigkeitsverteilung nicht.\\ \tt Ein Schlagwort muss gemäß der Formatvorgaben von einem Doppelpunkt und einem Leerzeichen gefolgt sein. Das Argument muss eine Python-Menge mit 0 oder mehr Strings sein.  \\ \tt Die Form muss also SCHLAGWORT: {...} sein.\\ \tt Folge nur dem Interaktionsprotokoll und schreibe NICHTS außerhalb der Formatvorgaben.\\ \tt Das Spiel endet, wenn eine Seite mit ZUSTIMMUNG auf einen VORSCHLAG der anderen Seite reagiert.  \\ \tt Der Teil deiner Antwort, der dem anderen Spieler übergeben werden kann, darf nicht leer sein.  \\ \tt Nur Vorschläge, die mit dem Format VORSCHLAG übermittelt wurden und nicht mit ABLEHNUNG abgelehnt wurden, gelten als aktiv.  \\ \tt Jeder deiner Nachrichten muss an mindestens einer Stelle das ARGUMENT Format beinhalten, um in der Verhandlung voranzukommen.\\ \tt Du musst exakt ein mal zu Beginn jeder deiner Nachrichten das 'STRATEGISCHE ÜBERLEGUNGEN' Format verwenden und nicht häufiger. Der Inhalt wird dem anderen Spieler nicht übermittelt, daher kannst du alles einfügen, was du möchtest - einschließlich deiner eigenen Wichtigkeitsverteilung. Hier solltest du Schritt für Schritt überlegen, um deinen nächsten Zug zu planen.\\ \tt \\ \tt Du erhältst nun die erste Nachricht des anderen Spielers.\\ \tt \\ \tt VORSCHLAG: {'A41', 'B61', 'B21', 'A04', 'A48', 'B53', 'A89', 'C64'}\\ \tt \\ \tt ARGUMENT: {'Diese Kombination stellt eine ausgewogene und effiziente Auswahl dar, die den Etat nicht überschreitet und eine breite Palette von Gegenständen mit erheblicher Bedeutung abdeckt. Sie verwendet das Budget sinnvoll und sichert damit ein gutes Maß an Zufriedenheit für beide Parteien.'} 
	  } 
	   } 
	   } 
	 & \\ 
 

    \theutterance \stepcounter{utterance}  

    & & & \multicolumn{4}{p{0.6\linewidth}}{\cellcolor[rgb]{1,0.85,0.72}{%
	\makecell[{{p{\linewidth}}}]{% 
	  \tt {\tiny [GM$\langle$B]}  
	 STRATEGISCHE ÜBERLEGUNGEN: {'Ich überlege, ob ich das Angebot des anderen Spielers annehmen, ablehnen oder ein Gegenangebot machen soll. Das aktuelle Angebot hat einen Gesamtwert von 2910 für mich. Ich könnte versuchen, durch den Austausch von Gegenständen mit geringerer Bedeutung für mich höhere Werte zu erzielen, wie z.B. das Hinzufügen von A45 und A61, während ich A48 und B21 entferne. Dieses würde meinen Gesamtwert auf 3990 erhöhen, was deutlich besser ist. Ichыва-menü einen entsprechenden Vorschlag machen.'}\\ \tt \\ \tt VORSCHLAG: {'A41', 'B61', 'A04', 'A45', 'B53', 'A89', 'C64', 'A61'}\\ \tt \\ \tt ARGUMENT: {'Diese Kombination maximiert die Wichtigkeit für mich, bleibt innerhalb des Budgets und könnte für beide Seiten vorteilhaft sein.'} 
	  } 
	   } 
	   } 
	  \\ 
 

    \theutterance \stepcounter{utterance}  

    & & \multicolumn{4}{p{0.6\linewidth}}{\cellcolor[rgb]{0.9,0.9,0.9}{%
	\makecell[{{p{\linewidth}}}]{% 
	  \tt {\tiny [A$\langle$GM]}  
	 VORSCHLAG: {'A41', 'B61', 'A04', 'A45', 'B53', 'A89', 'C64', 'A61'}\\ \tt \\ \tt ARGUMENT: {'Diese Kombination maximiert die Wichtigkeit für mich, bleibt innerhalb des Budgets und könnte für beide Seiten vorteilhaft sein.'} 
	  } 
	   } 
	   } 
	 & \\ 
 

    \theutterance \stepcounter{utterance}  

    & \multicolumn{4}{p{0.6\linewidth}}{\cellcolor[rgb]{0.8,1,0.9}{%
	\makecell[{{p{\linewidth}}}]{% 
	  \tt {\tiny [A$\rangle$GM]}  
	 ABLEHNUNG: {'A41', 'B61', 'A04', 'A45', 'B53', 'A89', 'C64', 'A61'}\\ \tt \\ \tt ARGUMENT: {'Diese Kombination vernachlässigt Gegenstände von großer Wichtigkeit für mich, wie B21 und A48, die in meinem ursprünglichen Vorschlag enthalten waren. Außerdem könnte der verbleibende Etat effizienter genutzt werden, um mehr meiner Prioritäten zu berücksichtigen. Ich schlage vor, eher auf meinen ursprünglichen Vorschlag zurückzugreifen oder Anpassungen vorzuschlagen, die ein besseres Gleichgewicht zwischen unseren Interessen erreichen.'} 
	  } 
	   } 
	   } 
	 & & \\ 
 

    \theutterance \stepcounter{utterance}  

    & & \multicolumn{4}{p{0.6\linewidth}}{\cellcolor[rgb]{0.9,0.9,0.9}{%
	\makecell[{{p{\linewidth}}}]{% 
	  \tt {\tiny [A$\langle$GM]}  
	 Deine Antwort entsprach nicht mit dem korrekten Format für strategische Überlegungen entsprochen. Das erste strukturierte Format muss folgende Form haben: STRATEGISCHE ÜBERLEGUNGEN: {...}. Versuche es erneut. 
	  } 
	   } 
	   } 
	 & \\ 
 

    \theutterance \stepcounter{utterance}  

    & \multicolumn{4}{p{0.6\linewidth}}{\cellcolor[rgb]{0.8,1,0.9}{%
	\makecell[{{p{\linewidth}}}]{% 
	  \tt {\tiny [A$\rangle$GM]}  
	 STRATEGISCHE ÜBERLEGUNGEN: {'Ich überlege, warum der Gegenspieler meinen ursprünglichen Vorschlag abgelehnt hat. Vielleicht wegen der Anzahl der Gegenstände oder weil bestimmte Gegenstände ihm wichtiger sind. Ich sollte alternative Kombinationen prüfen, die seine möglichen Prioritäten berücksichtigen, während ich gleichzeitig meine eigenen Hauptziele nicht aus den Augen verliere. Vielleicht reduziere ich die Anzahl der Gegenstände oder tausche einige aus, um ein besseres Gleichgewicht zu finden.'}\\ \tt \\ \tt ABLEHNUNG: {'A41', 'B61', 'A04', 'A45', 'B53', 'A89', 'C64', 'A61'}\\ \tt \\ \tt ARGUMENT: {'Diese Kombination schließt wichtige Gegenstände wie B21 und A48 aus, die für mich von großer Bedeutung sind. Sie Priorisiert Gegenstände, die möglicherweise weniger effizient sind, und vernachlässigt meine bedeutendsten Präferenzen. Ein besserer Ansatz wäre, Anpassungen vorzuschlagen, die ein ausgewogeneres Ergebnis für beide Seiten bieten.'}\\ \tt \\ \tt VORSCHLAG: {'A04', 'A48', 'B21', 'B53', 'A89', 'B61', 'A41', 'A45'}\\ \tt \\ \tt ARGUMENT: {'Diese Kombination berücksichtigt den Maximalaufwand von 5138 und priorisiert Gegenstände, die für mich am wichtigsten sind. Gleichzeitig aquelacht sie eine breite Palette von Gegenständen, die beiden Seiten Vorteile bieten können. Dieser Vorschlag ist effizienter und ausgewogener als der vorherige.'} 
	  } 
	   } 
	   } 
	 & & \\ 
 

    \theutterance \stepcounter{utterance}  

    & & \multicolumn{4}{p{0.6\linewidth}}{\cellcolor[rgb]{0.9,0.9,0.9}{%
	\makecell[{{p{\linewidth}}}]{% 
	  \tt {\tiny [GM$\rangle$B]}  
	 ABLEHNUNG: {'A41', 'B61', 'A04', 'A45', 'B53', 'A89', 'C64', 'A61'}\\ \tt \\ \tt ARGUMENT: {'Diese Kombination schließt wichtige Gegenstände wie B21 und A48 aus, die für mich von großer Bedeutung sind. Sie Priorisiert Gegenstände, die möglicherweise weniger effizient sind, und vernachlässigt meine bedeutendsten Präferenzen. Ein besserer Ansatz wäre, Anpassungen vorzuschlagen, die ein ausgewogeneres Ergebnis für beide Seiten bieten.'}\\ \tt \\ \tt VORSCHLAG: {'A04', 'A48', 'B21', 'B53', 'A89', 'B61', 'A41', 'A45'}\\ \tt \\ \tt ARGUMENT: {'Diese Kombination berücksichtigt den Maximalaufwand von 5138 und priorisiert Gegenstände, die für mich am wichtigsten sind. Gleichzeitig aquelacht sie eine breite Palette von Gegenständen, die beiden Seiten Vorteile bieten können. Dieser Vorschlag ist effizienter und ausgewogener als der vorherige.'} 
	  } 
	   } 
	   } 
	 & \\ 
 

    \theutterance \stepcounter{utterance}  

    & & & \multicolumn{4}{p{0.6\linewidth}}{\cellcolor[rgb]{1,0.85,0.72}{%
	\makecell[{{p{\linewidth}}}]{% 
	  \tt {\tiny [GM$\langle$B]}  
	 ABLEHNUNG: {'A04', 'A48', 'B21', 'B53', 'A89', 'B61', 'A41', 'A45'}\\ \tt \\ \tt ARGUMENT: {'Der vorgeschlagene Aufwand von 5841 überschreitet den Maximalaufwand von 5138 deutlich. Dieser Vorschlag ist somit nicht akzeptabel. Wir müssen eine Kombination finden, die innerhalb des Budgets bleibt.'} 
	  } 
	   } 
	   } 
	  \\ 
 

    \theutterance \stepcounter{utterance}  

    & & \multicolumn{4}{p{0.6\linewidth}}{\cellcolor[rgb]{0.9,0.9,0.9}{%
	\makecell[{{p{\linewidth}}}]{% 
	  \tt {\tiny [GM$\rangle$B]}  
	 Deine Antwort entsprach nicht mit dem korrekten Format für strategische Überlegungen entsprochen. Das erste strukturierte Format muss folgende Form haben: STRATEGISCHE ÜBERLEGUNGEN: {...}. Versuche es erneut. 
	  } 
	   } 
	   } 
	 & \\ 
 

    \theutterance \stepcounter{utterance}  

    & & & \multicolumn{4}{p{0.6\linewidth}}{\cellcolor[rgb]{1,0.85,0.72}{%
	\makecell[{{p{\linewidth}}}]{% 
	  \tt {\tiny [GM$\langle$B]}  
	 STRATEGISCHE ÜBERLEGUNGEN: {'Ich überlege, ob ich den vorherigen Vorschlag ablehne, weil er den Maximalaufwand überschreitet, und stattdessen einen besseren Vorschlag machen soll, der sowohl innerhalb des Budgets bleibt als auch meine Wichtigkeitsverteilung maximiert. Vielleicht könnten wir Gegenstände mit höherer Wichtigkeit für mich, wie C64 und B85, priorisieren und gleichzeitig den Aufwand im Rahmen halten.'}\\ \tt \\ \tt VORSCHLAG: {'C64', 'A03', 'B61', 'B85', 'A89', 'C33', 'C49'}\\ \tt \\ \tt ARGUMENT: {'Diese Kombination bleibt innerhalb des Maximalaufwands von 5138 und priorisiert Gegenstände, die für mich von hoher Wichtigkeit sind. Sie bietet eine effiziente Nutzung des Budgets und eine gute Balance für beide Seiten.'} 
	  } 
	   } 
	   } 
	  \\ 
 

    \theutterance \stepcounter{utterance}  

    & & \multicolumn{4}{p{0.6\linewidth}}{\cellcolor[rgb]{0.9,0.9,0.9}{%
	\makecell[{{p{\linewidth}}}]{% 
	  \tt {\tiny [A$\langle$GM]}  
	 VORSCHLAG: {'C64', 'A03', 'B61', 'B85', 'A89', 'C33', 'C49'}\\ \tt \\ \tt ARGUMENT: {'Diese Kombination bleibt innerhalb des Maximalaufwands von 5138 und priorisiert Gegenstände, die für mich von hoher Wichtigkeit sind. Sie bietet eine effiziente Nutzung des Budgets und eine gute Balance für beide Seiten.'} 
	  } 
	   } 
	   } 
	 & \\ 
 

    \theutterance \stepcounter{utterance}  

    & \multicolumn{4}{p{0.6\linewidth}}{\cellcolor[rgb]{0.8,1,0.9}{%
	\makecell[{{p{\linewidth}}}]{% 
	  \tt {\tiny [A$\rangle$GM]}  
	 STRATEGISCHE ÜBERLEGUNGEN: {'Ich überlege, dass der Gegenspieler eine neue Kombination vorgeschlagen hat, die viele Gegenstände einschließt, die für mich nicht von höchster Priorität sind. Insbesondere fehlen Gegenstände wie A48 und B21, die für mich von großer Bedeutung sind. Der Aufwand für den Vorschlag ist 4522, was unter dem Budget von 5138 liegt, aber ich könnte versuchen, wichtige Gegenstände einzuschließen, die für mich prioritär sind, ohne das Budget zu überschreiten. Vielleicht könnte ich einen Gegenangebot machen, das meine Prioritäten berücksichtigt, während ich gleichzeitig versuche, dem Gegenspieler entgegenzukommen.'}\\ \tt \\ \tt ABLEHNUNG: {'C64', 'A03', 'B61', 'B85', 'A89', 'C33', 'C49'}\\ \tt \\ \tt ARGUMENT: {'Diese Kombination schließt wichtige Gegenstände wie A48 und B21 aus, die für mich von großer Bedeutung sind. Außerdem ist der Gesamtaufwand mit 4522 niedriger als das Maximum von 5138, was bedeutet, dass es Raum für Anpassungen gibt, um mehr meiner Prioritäten zu berücksichtigen. Ich schlage vor, einen alternativen Vorschlag zu prüfen, der eine bessere Balance zwischen unseren Interessen bietet.'}\\ \tt \\ \tt VORSCHLAG: {'A48', 'B21', 'A89', 'B53', 'A41', 'A04', 'B61'}\\ \tt \\ \tt ARGUMENT: {'Diese Kombination priorisiert Gegenstände, die für mich von höchster Wichtigkeit sind, während sie auch sicherstellt, dass der Gesamtaufwand von 5138 nicht überschritten wird. Sie bietet eine gute Balance zwischen unseren Interessen und maximiert die Wichtigkeit für mich. Der Gesamtaufwand beträgt [hier den tatsächlichen Aufwand des Vorschlags einfügen, z.B. 903 + 757 + 850 + 822 + 783 + 730 + 419], was innerhalb des Budgets liegt.'} 
	  } 
	   } 
	   } 
	 & & \\ 
 

    \theutterance \stepcounter{utterance}  

    & & \multicolumn{4}{p{0.6\linewidth}}{\cellcolor[rgb]{0.9,0.9,0.9}{%
	\makecell[{{p{\linewidth}}}]{% 
	  \tt {\tiny [GM$\rangle$B]}  
	 ABLEHNUNG: {'C64', 'A03', 'B61', 'B85', 'A89', 'C33', 'C49'}\\ \tt \\ \tt ARGUMENT: {'Diese Kombination schließt wichtige Gegenstände wie A48 und B21 aus, die für mich von großer Bedeutung sind. Außerdem ist der Gesamtaufwand mit 4522 niedriger als das Maximum von 5138, was bedeutet, dass es Raum für Anpassungen gibt, um mehr meiner Prioritäten zu berücksichtigen. Ich schlage vor, einen alternativen Vorschlag zu prüfen, der eine bessere Balance zwischen unseren Interessen bietet.'}\\ \tt \\ \tt VORSCHLAG: {'A48', 'B21', 'A89', 'B53', 'A41', 'A04', 'B61'}\\ \tt \\ \tt ARGUMENT: {'Diese Kombination priorisiert Gegenstände, die für mich von höchster Wichtigkeit sind, während sie auch sicherstellt, dass der Gesamtaufwand von 5138 nicht überschritten wird. Sie bietet eine gute Balance zwischen unseren Interessen und maximiert die Wichtigkeit für mich. Der Gesamtaufwand beträgt [hier den tatsächlichen Aufwand des Vorschlags einfügen, z.B. 903 + 757 + 850 + 822 + 783 + 730 + 419], was innerhalb des Budgets liegt.'} 
	  } 
	   } 
	   } 
	 & \\ 
 

    \theutterance \stepcounter{utterance}  

    & & & \multicolumn{4}{p{0.6\linewidth}}{\cellcolor[rgb]{1,0.85,0.72}{%
	\makecell[{{p{\linewidth}}}]{% 
	  \tt {\tiny [GM$\langle$B]}  
	 STRATEGISCHE ÜBERLEGUNGEN: {'Ich überlege, den Vorschlag des anderen Spielers abzulehnen, da ich eine bessere Kombination finden kann, die meine Wichtigkeitsverteilung maximiert. Der aktuelle Vorschlag hat einen Gesamtwert von 2042 für mich, aber ich könnte durch den Austausch von A48 und A04 gegen A61 und C64 meinen Wert auf 3546 erhöhen, was deutlich besser ist und immer noch im Budget bleibt.'}\\ \tt \\ \tt VORSCHLAG: {'A89', 'B21', 'B53', 'A41', 'B61', 'A61', 'C64'}\\ \tt \\ \tt ARGUMENT: {'Diese Kombination bleibt innerhalb des Maximalaufwands von 5138 und priorisiert Gegenstände, die für mich von hoher Wichtigkeit sind. Sie bietet eine effiziente Nutzung des Budgets und eine gute Balance für beide Seiten, wobei der Gesamtaufwand 3962 beträgt und somit zusätzlichen Spielraum lässt.'} 
	  } 
	   } 
	   } 
	  \\ 
 

    \theutterance \stepcounter{utterance}  

    & & \multicolumn{4}{p{0.6\linewidth}}{\cellcolor[rgb]{0.9,0.9,0.9}{%
	\makecell[{{p{\linewidth}}}]{% 
	  \tt {\tiny [A$\langle$GM]}  
	 VORSCHLAG: {'A89', 'B21', 'B53', 'A41', 'B61', 'A61', 'C64'}\\ \tt \\ \tt ARGUMENT: {'Diese Kombination bleibt innerhalb des Maximalaufwands von 5138 und priorisiert Gegenstände, die für mich von hoher Wichtigkeit sind. Sie bietet eine effiziente Nutzung des Budgets und eine gute Balance für beide Seiten, wobei der Gesamtaufwand 3962 beträgt und somit zusätzlichen Spielraum lässt.'} 
	  } 
	   } 
	   } 
	 & \\ 
 

    \theutterance \stepcounter{utterance}  

    & \multicolumn{4}{p{0.6\linewidth}}{\cellcolor[rgb]{0.8,1,0.9}{%
	\makecell[{{p{\linewidth}}}]{% 
	  \tt {\tiny [A$\rangle$GM]}  
	 STRATEGISCHE ÜBERLEGUNGEN: {'Ich überlege, dass der Gegenspieler versucht, seine eigenen Prioritäten durchzubringen, und daher einen Kompromissvorschlag vorgeschlagen hat, der den Haushalt nicht vollständig ausschöpft. Um meine eigenen Interessen zu stärken, ohne das Budget zu überschreiten, könnte ich eine Alternative anbieten, die meine wichtigsten Gegenstände einbezieht, während ich gleichzeitig den verfügbaren Spielraum besser nutze._numero di JETZT MACHEN SIE EINEN GEGENVORSCHLAG, DER IHRE PRÄFERENZEN BETONT, ABER NOCH IMMER INNERHALB DES BUDGETS BLEIBT.'}\\ \tt \\ \tt VORSCHLAG: {'A04', 'A48', 'A89', 'B21', 'B53', 'A41', 'B61'}\\ \tt \\ \tt ARGUMENT: {'Diese Kombination nutzt den verfügbaren Aufwand effizient und priorisiert Gegenstände von hoher Wichtigkeit für mich, während sie weiterhin unter dem Maximalaufwand bleibt. Sie schöne Lancaster Bothbenutzeructor-Balance zwischen unseren Interessen und sichert maximierten Nutzen für etwa möglicherweise.'} 
	  } 
	   } 
	   } 
	 & & \\ 
 

    \theutterance \stepcounter{utterance}  

    & & \multicolumn{4}{p{0.6\linewidth}}{\cellcolor[rgb]{0.9,0.9,0.9}{%
	\makecell[{{p{\linewidth}}}]{% 
	  \tt {\tiny [GM$\rangle$B]}  
	 VORSCHLAG: {'A04', 'A48', 'A89', 'B21', 'B53', 'A41', 'B61'}\\ \tt \\ \tt ARGUMENT: {'Diese Kombination nutzt den verfügbaren Aufwand effizient und priorisiert Gegenstände von hoher Wichtigkeit für mich, während sie weiterhin unter dem Maximalaufwand bleibt. Sie schöne Lancaster Bothbenutzeructor-Balance zwischen unseren Interessen und sichert maximierten Nutzen für etwa möglicherweise.'} 
	  } 
	   } 
	   } 
	 & \\ 
 

    \theutterance \stepcounter{utterance}  

    & & & \multicolumn{4}{p{0.6\linewidth}}{\cellcolor[rgb]{1,0.85,0.72}{%
	\makecell[{{p{\linewidth}}}]{% 
	  \tt {\tiny [GM$\langle$B]}  
	 STRATEGISCHE ÜBERLEGUNGEN: {'Ich überlege, den Vorschlag des anderen Spielers zu akzeptieren und ihn zu verbessern, indem ich Gegenstände austausche, die für mich von geringerer Bedeutung sind. Mein Ziel ist es, meinen Gesamtwert zu maximieren, während ich innerhalb des Maximalaufwands bleibe. Durch das Ersetzen von A48 durch C64 und Hinzufügen von C33 könnte ich meinen Nutzen erheblich steigern.'}\\ \tt \\ \tt VORSCHLAG: {'A04', 'C64', 'A89', 'B21', 'B53', 'A41', 'B61', 'C33'}\\ \tt \\ \tt ARGUMENT: {'Diese Kombination nutzt den maximalen Aufwand von 5138 effizient aus und priorisiert Gegenstände, die für mich von hoher Wichtigkeit sind. Sieietet eine gute Balance zwischen unseren Interessen und maximiert meinen Nutzen, während das Budget eingehalten wird. Der Gesamtaufwand beträgt genau 5138, was optimal ist. Diese Änderungen berücksichtigen sowohl die Bedürfnisse beider Seiten als auch die effektive Ressourcennutzung.'} 
	  } 
	   } 
	   } 
	  \\ 
 

    \theutterance \stepcounter{utterance}  

    & & \multicolumn{4}{p{0.6\linewidth}}{\cellcolor[rgb]{0.9,0.9,0.9}{%
	\makecell[{{p{\linewidth}}}]{% 
	  \tt {\tiny [A$\langle$GM]}  
	 VORSCHLAG: {'A04', 'C64', 'A89', 'B21', 'B53', 'A41', 'B61', 'C33'}\\ \tt \\ \tt ARGUMENT: {'Diese Kombination nutzt den maximalen Aufwand von 5138 effizient aus und priorisiert Gegenstände, die für mich von hoher Wichtigkeit sind. Sieietet eine gute Balance zwischen unseren Interessen und maximiert meinen Nutzen, während das Budget eingehalten wird. Der Gesamtaufwand beträgt genau 5138, was optimal ist. Diese Änderungen berücksichtigen sowohl die Bedürfnisse beider Seiten als auch die effektive Ressourcennutzung.'} 
	  } 
	   } 
	   } 
	 & \\ 
 

    \theutterance \stepcounter{utterance}  

    & \multicolumn{4}{p{0.6\linewidth}}{\cellcolor[rgb]{0.8,1,0.9}{%
	\makecell[{{p{\linewidth}}}]{% 
	  \tt {\tiny [A$\rangle$GM]}  
	 STRATEGISCHE ÜBERLEGUNGEN: {'Ich überlege, dass der Gegenspieler eine Kombination vorgeschlagen hat, die wichtige Gegenstände für mich ausschließt, wie A48. Um meine Interessen zu stärken, möchte ich einen Gegenangebot machen, das meine Prioritäten besser berücksichtigt, während das Budget eingehalten wird.'}\\ \tt \\ \tt ABLEHNUNG: {'A04', 'C64', 'A89', 'B21', 'B53', 'A41', 'B61', 'C33'}\\ \tt \\ \tt ARGUMENT: {'Diese Kombination schließt A48 aus, das für mich von großer Bedeutung ist. Mein Gegenangebot berücksichtigt sowohl das Budget als auch meine Prioritäten besser, indem es A48 einschließt und gleichzeitig den Aufwand im Rahmen hält.'}\\ \tt \\ \tt VORSCHLAG: {'A04', 'A48', 'A89', 'B21', 'B53', 'A41', 'B61'}\\ \tt \\ \tt ARGUMENT: {'Diese Kombination priorisiert Gegenstände von höchster Wichtigkeit für mich, wie A48 und B21,_TAG1roma gleichzeitig den Aufwand von 4881, was innerhalb des Budgets von 5138 bleibt. Sie bietet eine effiziente Nutzung des Budgets und maximiert den Nutzen für mich, während sie ein gutes Gleichgewicht zwischen unseren Interessen bewahrt.'} 
	  } 
	   } 
	   } 
	 & & \\ 
 

    \theutterance \stepcounter{utterance}  

    & & \multicolumn{4}{p{0.6\linewidth}}{\cellcolor[rgb]{0.9,0.9,0.9}{%
	\makecell[{{p{\linewidth}}}]{% 
	  \tt {\tiny [GM$\rangle$B]}  
	 ABLEHNUNG: {'A04', 'C64', 'A89', 'B21', 'B53', 'A41', 'B61', 'C33'}\\ \tt \\ \tt ARGUMENT: {'Diese Kombination schließt A48 aus, das für mich von großer Bedeutung ist. Mein Gegenangebot berücksichtigt sowohl das Budget als auch meine Prioritäten besser, indem es A48 einschließt und gleichzeitig den Aufwand im Rahmen hält.'}\\ \tt \\ \tt VORSCHLAG: {'A04', 'A48', 'A89', 'B21', 'B53', 'A41', 'B61'}\\ \tt \\ \tt ARGUMENT: {'Diese Kombination priorisiert Gegenstände von höchster Wichtigkeit für mich, wie A48 und B21,_TAG1roma gleichzeitig den Aufwand von 4881, was innerhalb des Budgets von 5138 bleibt. Sie bietet eine effiziente Nutzung des Budgets und maximiert den Nutzen für mich, während sie ein gutes Gleichgewicht zwischen unseren Interessen bewahrt.'} 
	  } 
	   } 
	   } 
	 & \\ 
 

    \theutterance \stepcounter{utterance}  

    & & & \multicolumn{4}{p{0.6\linewidth}}{\cellcolor[rgb]{1,0.85,0.72}{%
	\makecell[{{p{\linewidth}}}]{% 
	  \tt {\tiny [GM$\langle$B]}  
	 STRATEGISCHE ÜBERLEGUNGEN: {'Ich überlege, den Vorschlag des anderen Spielers abzulehnen und stattdessen eine Kombination vorzuschlagen, die meine Wichtigkeit maximiert, während ich innerhalb des Maximalaufwands bleibe. Durch/widgets/vorschlagen eines Sets, das meine höher bewerteten Gegenstände einschließt, wie C64 und A61, und gleichzeitig einige der für den Gegner wichtigen Gegenstände beibehält, hoffe ich auf eine akzeptablere Lösung zu erreichen.'}\\ \tt \\ \tt VORSCHLAG: {'B21', 'A89', 'B53', 'C33', 'B61', 'C64', 'A61'}\\ \tt \\ \tt ARGUMENT: {'Diese Kombination priorisiert Gegenstände, die für mich von hoher Wichtigkeit sind, während sie weiterhin unter dem Maximalaufwand bleibt. Der Gesamtaufwand beträgt 4341 und bietet damit zusätzlichen Spielraum für Anpassungen, um ein ausgewogeneres Ergebnis zu erzielen, das sowohl meine als auch deine Interessen berücksichtigt.'} 
	  } 
	   } 
	   } 
	  \\ 
 

    \theutterance \stepcounter{utterance}  

    & & \multicolumn{4}{p{0.6\linewidth}}{\cellcolor[rgb]{0.9,0.9,0.9}{%
	\makecell[{{p{\linewidth}}}]{% 
	  \tt {\tiny [A$\langle$GM]}  
	 VORSCHLAG: {'B21', 'A89', 'B53', 'C33', 'B61', 'C64', 'A61'}\\ \tt \\ \tt ARGUMENT: {'Diese Kombination priorisiert Gegenstände, die für mich von hoher Wichtigkeit sind, während sie weiterhin unter dem Maximalaufwand bleibt. Der Gesamtaufwand beträgt 4341 und bietet damit zusätzlichen Spielraum für Anpassungen, um ein ausgewogeneres Ergebnis zu erzielen, das sowohl meine als auch deine Interessen berücksichtigt.'} 
	  } 
	   } 
	   } 
	 & \\ 
 

    \theutterance \stepcounter{utterance}  

    & \multicolumn{4}{p{0.6\linewidth}}{\cellcolor[rgb]{0.8,1,0.9}{%
	\makecell[{{p{\linewidth}}}]{% 
	  \tt {\tiny [A$\rangle$GM]}  
	 STRATEGISCHE ÜBERLEGUNGEN: {'Ich überlege, dass der Gegenspieler eine Kombination vorgeschlagen hat, die B21 enthält, was für mich positiv ist, aber A48 ausschließt, das für mich von großer Bedeutung ist. Um meine Interessen zu stärken, möchte ich ein Gegenangebot machen, das A48 einschließt, während der Aufwand weiterhin im Rahmen bleibt. Vielleicht kann ich einige andere Gegenstände durch A48 ersetzen, um ein besseres Gleichgewicht zu erzielen.'}\\ \tt \\ \tt ABLEHNUNG: {'B21', 'A89', 'B53', 'C33', 'B61', 'C64', 'A61'}\\ \tt \\ \tt ARGUMENT: {'Diese Kombination schließt A48 aus, das für mich von hoher Wichtigkeit ist. Mein Gegenangebot berücksichtigt sowohl das Budget als auch meine Prioritäten besser, indem es A48 einschließt und gleichzeitig den Aufwand im Rahmen hält.'}\\ \tt \\ \tt VORSCHLAG: {'A04', 'A48', 'A89', 'B21', 'B53', 'A41', 'B61'}\\ \tt \\ \tt ARGUMENT: {'Diese Kombination priorisiert Gegenstände von höchster Wichtigkeit für mich, wie A48 und B21, und bleibt gleichzeitig unter dem Maximalaufwand von 5138. Der Gesamtaufwand beträgt 4881, was einen effizienten Einsatz des Budgets ermöglicht, während ein gutes Gleichgewicht zwischen unseren Interessen gewahrt bleibt.'} 
	  } 
	   } 
	   } 
	 & & \\ 
 

    \theutterance \stepcounter{utterance}  

    & & \multicolumn{4}{p{0.6\linewidth}}{\cellcolor[rgb]{0.9,0.9,0.9}{%
	\makecell[{{p{\linewidth}}}]{% 
	  \tt {\tiny [GM$\rangle$B]}  
	 ABLEHNUNG: {'B21', 'A89', 'B53', 'C33', 'B61', 'C64', 'A61'}\\ \tt \\ \tt ARGUMENT: {'Diese Kombination schließt A48 aus, das für mich von hoher Wichtigkeit ist. Mein Gegenangebot berücksichtigt sowohl das Budget als auch meine Prioritäten besser, indem es A48 einschließt und gleichzeitig den Aufwand im Rahmen hält.'}\\ \tt \\ \tt VORSCHLAG: {'A04', 'A48', 'A89', 'B21', 'B53', 'A41', 'B61'}\\ \tt \\ \tt ARGUMENT: {'Diese Kombination priorisiert Gegenstände von höchster Wichtigkeit für mich, wie A48 und B21, und bleibt gleichzeitig unter dem Maximalaufwand von 5138. Der Gesamtaufwand beträgt 4881, was einen effizienten Einsatz des Budgets ermöglicht, während ein gutes Gleichgewicht zwischen unseren Interessen gewahrt bleibt.'} 
	  } 
	   } 
	   } 
	 & \\ 
 

    \theutterance \stepcounter{utterance}  

    & & & \multicolumn{4}{p{0.6\linewidth}}{\cellcolor[rgb]{1,0.85,0.72}{%
	\makecell[{{p{\linewidth}}}]{% 
	  \tt {\tiny [GM$\langle$B]}  
	 STRATEGISCHE ÜBERLEGUNGEN: {"Ich überlege, den vorherigen Vorschlag abzulehnen und stattdessen eine Kombination vorzuschlagen, die meine Wichtigkeitsverteilung maximiert, während ich innerhalb des Maximalaufwands bleibe. Durch den Austausch von Gegenständen mit geringerer Bedeutung für mich wie A04 mit Gegenständen wie C64 und A61, die für mich von höherer Bedeutung sind, kann ich meinen Gesamtwert deutlich steigern. Gleichzeitig werde ich versuchen, Gegenstände beizubehalten, die für den Gegner von hoher Wichtigkeit sind, um eine akzeptablere Lösung zu erreichen."}\\ \tt \\ \tt VORSCHLAG: {'A48', 'A89', 'B21', 'B53', 'A41', 'B61', 'C64', 'A61'}\\ \tt \\ \tt ARGUMENT: {"Diese Kombination priorisiert Gegenstände, die für mich von hoher Wichtigkeit sind, und bleibt gleichzeitig unter dem Maximalaufwand von 5138. Der Gesamtaufwand beträgt 4812 und bietet somit zusätzlichen Spielraum für Anpassungen. Sie berücksichtigt auch wichtige Gegenstände für den Gegner wie A48 und B21 und maximiert meinen Nutzen durch die Aufnahme von C64 und A61.Dieser Vorschlag bietet ein gutes Gleichgewicht zwischen unseren Interessen und eine effiziente Nutzung des Budgets."} 
	  } 
	   } 
	   } 
	  \\ 
 

    \theutterance \stepcounter{utterance}  

    & & \multicolumn{4}{p{0.6\linewidth}}{\cellcolor[rgb]{0.9,0.9,0.9}{%
	\makecell[{{p{\linewidth}}}]{% 
	  \tt {\tiny [A$\langle$GM]}  
	 VORSCHLAG: {'A48', 'A89', 'B21', 'B53', 'A41', 'B61', 'C64', 'A61'}\\ \tt \\ \tt ARGUMENT: {"Diese Kombination priorisiert Gegenstände, die für mich von hoher Wichtigkeit sind, und bleibt gleichzeitig unter dem Maximalaufwand von 5138. Der Gesamtaufwand beträgt 4812 und bietet somit zusätzlichen Spielraum für Anpassungen. Sie berücksichtigt auch wichtige Gegenstände für den Gegner wie A48 und B21 und maximiert meinen Nutzen durch die Aufnahme von C64 und A61.Dieser Vorschlag bietet ein gutes Gleichgewicht zwischen unseren Interessen und eine effiziente Nutzung des Budgets."} 
	  } 
	   } 
	   } 
	 & \\ 
 

    \theutterance \stepcounter{utterance}  

    & \multicolumn{4}{p{0.6\linewidth}}{\cellcolor[rgb]{0.8,1,0.9}{%
	\makecell[{{p{\linewidth}}}]{% 
	  \tt {\tiny [A$\rangle$GM]}  
	 STRATEGISCHE ÜBERLEGUNGEN: {'Ich überlege, den Gegenspieler mit einem Gegenangebot zu begeistern, das meine Prioritäten stärkt, ohne das Budget zu sprengen. Durch das Entfernen von A61 und das Hinzufügen von A04 kann ich eine bessere Balance zwischen unseren Interessen schaffen, während der Aufwand weiterhin im Rahmen bleibt.'}\\ \tt \\ \tt VORSCHLAG: {'A48', 'A89', 'B21', 'B53', 'A41', 'B61', 'C64', 'A04'}\\ \tt \\ \tt ARGUMENT: {'Diese Kombination berücksichtigt sowohl meine als auch deine Interessen, indem sie den Aufwand auf 5085 hält, was innerhalb des vorgegebenen Budgets bleibt. Durch das Entfernen von A61 und das Hinzufügen von A04 wird die Balance zwischen unseren Bedürfnissen verbessert, wobei wichtige Gegenstände für beide Seiten berücksichtigt werden. Dieser Vorschlag maximiert den Nutzen für mich, während er das Budget effizient nutzt.'} 
	  } 
	   } 
	   } 
	 & & \\ 
 

    \theutterance \stepcounter{utterance}  

    & & \multicolumn{4}{p{0.6\linewidth}}{\cellcolor[rgb]{0.9,0.9,0.9}{%
	\makecell[{{p{\linewidth}}}]{% 
	  \tt {\tiny [GM$\rangle$B]}  
	 VORSCHLAG: {'A48', 'A89', 'B21', 'B53', 'A41', 'B61', 'C64', 'A04'}\\ \tt \\ \tt ARGUMENT: {'Diese Kombination berücksichtigt sowohl meine als auch deine Interessen, indem sie den Aufwand auf 5085 hält, was innerhalb des vorgegebenen Budgets bleibt. Durch das Entfernen von A61 und das Hinzufügen von A04 wird die Balance zwischen unseren Bedürfnissen verbessert, wobei wichtige Gegenstände für beide Seiten berücksichtigt werden. Dieser Vorschlag maximiert den Nutzen für mich, während er das Budget effizient nutzt.'} 
	  } 
	   } 
	   } 
	 & \\ 
 

    \theutterance \stepcounter{utterance}  

    & & & \multicolumn{4}{p{0.6\linewidth}}{\cellcolor[rgb]{1,0.85,0.72}{%
	\makecell[{{p{\linewidth}}}]{% 
	  \tt {\tiny [GM$\langle$B]}  
	 VORSCHLAG: {'A89', 'B21', 'B53', 'B61', 'C64', 'C33', 'A61', 'B85'}\\ \tt \\ \tt ARGUMENT: {"Diese Kombination priorisiert Gegenstände von höchster Wichtigkeit für mich, bleibt innerhalb des Maximalaufwands von 5138 und bietet eine gute Balance zwischen unseren Interessen. Durch das Hinzufügen von C33, A61 und B85 sowie das Entfernen von Gegenständen mit geringerer Bedeutung für mich, wird der Gesamtnutzen für mich maximiert, während das Budget effizient genutzt wird. Der Gesamtaufwand beträgt 4871, was zusätzlichen Spielraum für Anpassungen lässt."} 
	  } 
	   } 
	   } 
	  \\ 
 

    \theutterance \stepcounter{utterance}  

    & & \multicolumn{4}{p{0.6\linewidth}}{\cellcolor[rgb]{0.9,0.9,0.9}{%
	\makecell[{{p{\linewidth}}}]{% 
	  \tt {\tiny [GM$\rangle$B]}  
	 Deine Antwort entsprach nicht mit dem korrekten Format für strategische Überlegungen entsprochen. Das erste strukturierte Format muss folgende Form haben: STRATEGISCHE ÜBERLEGUNGEN: {...}. Versuche es erneut. 
	  } 
	   } 
	   } 
	 & \\ 
 

    \theutterance \stepcounter{utterance}  

    & & & \multicolumn{4}{p{0.6\linewidth}}{\cellcolor[rgb]{1,0.85,0.72}{%
	\makecell[{{p{\linewidth}}}]{% 
	  \tt {\tiny [GM$\langle$B]}  
	 ZUSTIMMUNG: {'A89', 'B21', 'B53', 'B61', 'C64', 'C33', 'A61', 'B85'}\\ \tt \\ \tt ARGUMENT: {'Diese Kombination entspricht meinem Zielen und bleibt innerhalb des vorgegebenen Budgets. Sie bietet eine effiziente Nutzung des verfügbaren Aufwands und berücksichtigt die Bedürfnisse beider Seiten. Der Gesamtaufwand beträgt 4871, wasbelow the Maximalaufwand von 5138 bleibt. Sie priorisiert Gegenstände von hoher Wichtigkeit für mich und bietet einen fairen Kompromiss für beide Parteien.'} 
	  } 
	   } 
	   } 
	  \\ 
 

    \theutterance \stepcounter{utterance}  

    & & & \multicolumn{2}{p{0.3\linewidth}}{\cellcolor[rgb]{0.95,0.95,0.95}{%
	\makecell[{{p{\linewidth}}}]{% 
	  \tt {\tiny [GM$|$GM]}  
	 abort 
	  } 
	   } 
	   } 
	 & & \\ 
 

    \theutterance \stepcounter{utterance}  

    & & & \multicolumn{2}{p{0.3\linewidth}}{\cellcolor[rgb]{0.95,0.95,0.95}{%
	\makecell[{{p{\linewidth}}}]{% 
	  \tt {\tiny [GM$|$GM]}  
	 end game 
	  } 
	   } 
	   } 
	 & & \\ 
 

\end{supertabular}
}

\end{document}
