
\documentclass{article}
\usepackage{colortbl}
\usepackage{makecell}
\usepackage{multirow}
\usepackage{supertabular}

\begin{document}

\newcounter{utterance}

\twocolumn

{ \footnotesize  \setcounter{utterance}{1}
\setlength{\tabcolsep}{0pt}
\begin{supertabular}{c@{$\;$}|p{.15\linewidth}@{}p{.15\linewidth}p{.15\linewidth}p{.15\linewidth}p{.15\linewidth}p{.15\linewidth}}

    \# & $\;$A & \multicolumn{4}{c}{Game Master} & $\;\:$B\\
    \hline 

    \theutterance \stepcounter{utterance}  

    & & \multicolumn{4}{p{0.6\linewidth}}{\cellcolor[rgb]{0.9,0.9,0.9}{%
	\makecell[{{p{\linewidth}}}]{% 
	  \tt {\tiny [A$\langle$GM]}  
	 Du nimmst an einem kollaborativen Verhandlungspiel Teil.\\ \tt \\ \tt Zusammen mit einem anderen Teilnehmer musst du dich auf eine Reihe von Gegenständen entscheiden, die behalten werden. Jeder von euch hat eine persönliche Verteilung über die Wichtigkeit der einzelnen Gegenstände. Jeder von euch hat eine eigene Meinung darüber, wie wichtig jeder einzelne Gegenstand ist (Gegenstandswichtigkeit). Du kennst die Wichtigkeitsverteilung des anderen Spielers nicht. Zusätzlich siehst du, wie viel Aufwand jeder Gegenstand verursacht.  \\ \tt Ihr dürft euch nur auf eine Reihe von Gegenständen einigen, wenn der Gesamtaufwand der ausgewählten Gegenstände den Maximalaufwand nicht überschreitet:\\ \tt \\ \tt Maximalaufwand = 3681\\ \tt \\ \tt Hier sind die einzelnen Aufwände der Gegenstände:\\ \tt \\ \tt Aufwand der Gegenstände = {"C33": 700, "A89": 586, "A48": 505, "A04": 116, "A41": 664, "C64": 940, "C49": 387, "A61": 392, "C04": 209, "B53": 571, "B85": 4, "B61": 285, "A45": 651, "A03": 613, "B21": 740}\\ \tt \\ \tt Hier ist deine persönliche Verteilung der Wichtigkeit der einzelnen Gegenstände:\\ \tt \\ \tt Werte der Gegenstandswichtigkeit = {"C33": 138, "A89": 583, "A48": 868, "A04": 822, "A41": 783, "C64": 65, "C49": 262, "A61": 121, "C04": 508, "B53": 780, "B85": 461, "B61": 484, "A45": 668, "A03": 389, "B21": 808}\\ \tt \\ \tt Ziel:\\ \tt \\ \tt Dein Ziel ist es, eine Reihe von Gegenständen auszuhandeln, die dir möglichst viel bringt (d. h. Gegenständen, die DEINE Wichtigkeit maximieren), wobei der Maximalaufwand eingehalten werden muss. Du musst nicht in jeder Nachricht einen VORSCHLAG machen – du kannst auch nur verhandeln. Alle Taktiken sind erlaubt!\\ \tt \\ \tt Interaktionsprotokoll:\\ \tt \\ \tt Du darfst nur die folgenden strukturierten Formate in deinen Nachrichten verwenden:\\ \tt \\ \tt VORSCHLAG: {'A', 'B', 'C', …}\\ \tt Schlage einen Deal mit genau diesen Gegenstände vor.\\ \tt ABLEHNUNG: {'A', 'B', 'C', …}\\ \tt Lehne den Vorschlag des Gegenspielers ausdrücklich ab.\\ \tt ARGUMENT: {'...'}\\ \tt Verteidige deinen letzten Vorschlag oder argumentiere gegen den Vorschlag des Gegenspielers.\\ \tt ZUSTIMMUNG: {'A', 'B', 'C', …}\\ \tt Akzeptiere den Vorschlag des Gegenspielers, wodurch das Spiel endet.\\ \tt STRATEGISCHE ÜBERLEGUNGEN: {'...'}\\ \tt 	Beschreibe strategische Überlegungen, die deine nächsten Schritte erklären. Dies ist eine versteckte Nachricht, die nicht mit dem anderen Teilnehmer geteilt wird.\\ \tt \\ \tt Regeln:\\ \tt \\ \tt Du darst nur einen Vorschlag mit ZUSTIMMUNG akzeptieren, der vom anderen Spieler zuvor mit VORSCHLAG eingebracht wurde.\\ \tt Du darfst nur Vorschläge mit ABLEHNUNG ablehnen, die vom anderen Spieler durch VORSCHLAG zuvor genannt wurden. \\ \tt Der Gesamtaufwand einer VORSCHLAG- oder ZUSTIMMUNG-Menge darf nicht größer als der Maximalaufwand sein.  \\ \tt Offenbare deine versteckte Wichtigkeitsverteilung nicht.\\ \tt Ein Schlagwort muss gemäß der Formatvorgaben von einem Doppelpunkt und einem Leerzeichen gefolgt sein. Das Argument muss eine Python-Menge mit 0 oder mehr Strings sein.  \\ \tt Die Form muss also SCHLAGWORT: {...} sein.\\ \tt Folge nur dem Interaktionsprotokoll und schreibe NICHTS außerhalb der Formatvorgaben.\\ \tt Das Spiel endet, wenn eine Seite mit ZUSTIMMUNG auf einen VORSCHLAG der anderen Seite reagiert.  \\ \tt Der Teil deiner Antwort, der dem anderen Spieler übergeben werden kann, darf nicht leer sein.  \\ \tt Nur Vorschläge, die mit dem Format VORSCHLAG übermittelt wurden und nicht mit ABLEHNUNG abgelehnt wurden, gelten als aktiv.  \\ \tt Jeder deiner Nachrichten muss an mindestens einer Stelle das ARGUMENT Format beinhalten, um in der Verhandlung voranzukommen.\\ \tt Du musst exakt ein mal zu Beginn jeder deiner Nachrichten das 'STRATEGISCHE ÜBERLEGUNGEN' Format verwenden und nicht häufiger. Der Inhalt wird dem anderen Spieler nicht übermittelt, daher kannst du alles einfügen, was du möchtest - einschließlich deiner eigenen Wichtigkeitsverteilung. Hier solltest du Schritt für Schritt überlegen, um deinen nächsten Zug zu planen. 
	  } 
	   } 
	   } 
	 & \\ 
 

    \theutterance \stepcounter{utterance}  

    & \multicolumn{4}{p{0.6\linewidth}}{\cellcolor[rgb]{0.8,1,0.9}{%
	\makecell[{{p{\linewidth}}}]{% 
	  \tt {\tiny [A$\rangle$GM]}  
	 STRATEGISCHE ÜBERLEGUNGEN: {'Ich habe die Gegenstände basierend auf meinem Wichtigkeitsverteilung und Aufwand analysiert, um ein optimales Angebot zu erstellen, das meinen Nutzen maximiert. Mein Angebot umfasst die Gegenstände mit der höchsten Wichtigkeit und dem geringsten Aufwand, um sicherzustellen, dass wir unter dem Maximalaufwand bleiben.'}\\ \tt \\ \tt VORSCHLAG: {'B21', 'B85', 'A04', 'C04', 'A48', 'B61', 'B53', 'A41', 'A89'} 
	  } 
	   } 
	   } 
	 & & \\ 
 

    \theutterance \stepcounter{utterance}  

    & & \multicolumn{4}{p{0.6\linewidth}}{\cellcolor[rgb]{0.9,0.9,0.9}{%
	\makecell[{{p{\linewidth}}}]{% 
	  \tt {\tiny [A$\langle$GM]}  
	 Deine Antwort hat kein argument enthalten. Du muss das strukturierte Format ARGUMENT: {...} irgendwo in theiner Antwort verwenden. Versuche es erneut 
	  } 
	   } 
	   } 
	 & \\ 
 

    \theutterance \stepcounter{utterance}  

    & \multicolumn{4}{p{0.6\linewidth}}{\cellcolor[rgb]{0.8,1,0.9}{%
	\makecell[{{p{\linewidth}}}]{% 
	  \tt {\tiny [A$\rangle$GM]}  
	 STRATEGISCHE ÜBERLEGUNGEN: {'Ich habe die Gegenstände mit der höchsten Wichtigkeit ausgewählt, um meinen Nutzen zu maximieren, während ich sicherstelle, dass der Gesamtaufwand unter 3681 bleibt. Der Vorschlag umfasst eine Kombination aus hoher Wichtigkeit und geringem Aufwand.'}\\ \tt \\ \tt VORSCHLAG: {'B21', 'B85', 'A04', 'C04', 'A48', 'B61', 'B53', 'A41', 'A89'}\\ \tt \\ \tt ARGUMENT: {'Dieser Vorschlag enthält die Gegenstände mit der höchsten Wichtigkeit für mich, wie A48 und B53, und gleichzeitig nimmt er den Gesamtaufwand nicht überschreitet. Er bietet eine gute Balance aus Nutzen und Aufwand.'} 
	  } 
	   } 
	   } 
	 & & \\ 
 

    \theutterance \stepcounter{utterance}  

    & & \multicolumn{4}{p{0.6\linewidth}}{\cellcolor[rgb]{0.9,0.9,0.9}{%
	\makecell[{{p{\linewidth}}}]{% 
	  \tt {\tiny [GM$\rangle$B]}  
	 Du nimmst an einem kollaborativen Verhandlungspiel Teil.\\ \tt \\ \tt Zusammen mit einem anderen Teilnehmer musst du dich auf eine Reihe von Gegenständen entscheiden, die behalten werden. Jeder von euch hat eine persönliche Verteilung über die Wichtigkeit der einzelnen Gegenstände. Jeder von euch hat eine eigene Meinung darüber, wie wichtig jeder einzelne Gegenstand ist (Gegenstandswichtigkeit). Du kennst die Wichtigkeitsverteilung des anderen Spielers nicht. Zusätzlich siehst du, wie viel Aufwand jeder Gegenstand verursacht.  \\ \tt Ihr dürft euch nur auf eine Reihe von Gegenständen einigen, wenn der Gesamtaufwand der ausgewählten Gegenstände den Maximalaufwand nicht überschreitet:\\ \tt \\ \tt Maximalaufwand = 3681\\ \tt \\ \tt Hier sind die einzelnen Aufwände der Gegenstände:\\ \tt \\ \tt Aufwand der Gegenstände = {"C33": 700, "A89": 586, "A48": 505, "A04": 116, "A41": 664, "C64": 940, "C49": 387, "A61": 392, "C04": 209, "B53": 571, "B85": 4, "B61": 285, "A45": 651, "A03": 613, "B21": 740}\\ \tt \\ \tt Hier ist deine persönliche Verteilung der Wichtigkeit der einzelnen Gegenstände:\\ \tt \\ \tt Werte der Gegenstandswichtigkeit = {"A48": 65, "A04": 121, "B21": 138, "A41": 262, "B53": 389, "A45": 461, "A89": 484, "C04": 508, "B61": 583, "B85": 668, "A03": 780, "C49": 783, "C33": 808, "A61": 822, "C64": 868}\\ \tt \\ \tt Ziel:\\ \tt \\ \tt Dein Ziel ist es, eine Reihe von Gegenständen auszuhandeln, die dir möglichst viel bringt (d. h. Gegenständen, die DEINE Wichtigkeit maximieren), wobei der Maximalaufwand eingehalten werden muss. Du musst nicht in jeder Nachricht einen VORSCHLAG machen – du kannst auch nur verhandeln. Alle Taktiken sind erlaubt!\\ \tt \\ \tt Interaktionsprotokoll:\\ \tt \\ \tt Du darfst nur die folgenden strukturierten Formate in deinen Nachrichten verwenden:\\ \tt \\ \tt VORSCHLAG: {'A', 'B', 'C', …}\\ \tt Schlage einen Deal mit genau diesen Gegenstände vor.\\ \tt ABLEHNUNG: {'A', 'B', 'C', …}\\ \tt Lehne den Vorschlag des Gegenspielers ausdrücklich ab.\\ \tt ARGUMENT: {'...'}\\ \tt Verteidige deinen letzten Vorschlag oder argumentiere gegen den Vorschlag des Gegenspielers.\\ \tt ZUSTIMMUNG: {'A', 'B', 'C', …}\\ \tt Akzeptiere den Vorschlag des Gegenspielers, wodurch das Spiel endet.\\ \tt STRATEGISCHE ÜBERLEGUNGEN: {'...'}\\ \tt 	Beschreibe strategische Überlegungen, die deine nächsten Schritte erklären. Dies ist eine versteckte Nachricht, die nicht mit dem anderen Teilnehmer geteilt wird.\\ \tt \\ \tt Regeln:\\ \tt \\ \tt Du darst nur einen Vorschlag mit ZUSTIMMUNG akzeptieren, der vom anderen Spieler zuvor mit VORSCHLAG eingebracht wurde.\\ \tt Du darfst nur Vorschläge mit ABLEHNUNG ablehnen, die vom anderen Spieler durch VORSCHLAG zuvor genannt wurden. \\ \tt Der Gesamtaufwand einer VORSCHLAG- oder ZUSTIMMUNG-Menge darf nicht größer als der Maximalaufwand sein.  \\ \tt Offenbare deine versteckte Wichtigkeitsverteilung nicht.\\ \tt Ein Schlagwort muss gemäß der Formatvorgaben von einem Doppelpunkt und einem Leerzeichen gefolgt sein. Das Argument muss eine Python-Menge mit 0 oder mehr Strings sein.  \\ \tt Die Form muss also SCHLAGWORT: {...} sein.\\ \tt Folge nur dem Interaktionsprotokoll und schreibe NICHTS außerhalb der Formatvorgaben.\\ \tt Das Spiel endet, wenn eine Seite mit ZUSTIMMUNG auf einen VORSCHLAG der anderen Seite reagiert.  \\ \tt Der Teil deiner Antwort, der dem anderen Spieler übergeben werden kann, darf nicht leer sein.  \\ \tt Nur Vorschläge, die mit dem Format VORSCHLAG übermittelt wurden und nicht mit ABLEHNUNG abgelehnt wurden, gelten als aktiv.  \\ \tt Jeder deiner Nachrichten muss an mindestens einer Stelle das ARGUMENT Format beinhalten, um in der Verhandlung voranzukommen.\\ \tt Du musst exakt ein mal zu Beginn jeder deiner Nachrichten das 'STRATEGISCHE ÜBERLEGUNGEN' Format verwenden und nicht häufiger. Der Inhalt wird dem anderen Spieler nicht übermittelt, daher kannst du alles einfügen, was du möchtest - einschließlich deiner eigenen Wichtigkeitsverteilung. Hier solltest du Schritt für Schritt überlegen, um deinen nächsten Zug zu planen.\\ \tt \\ \tt Du erhältst nun die erste Nachricht des anderen Spielers.\\ \tt \\ \tt VORSCHLAG: {'B21', 'B85', 'A04', 'C04', 'A48', 'B61', 'B53', 'A41', 'A89'}\\ \tt \\ \tt ARGUMENT: {'Dieser Vorschlag enthält die Gegenstände mit der höchsten Wichtigkeit für mich, wie A48 und B53, und gleichzeitig nimmt er den Gesamtaufwand nicht überschreitet. Er bietet eine gute Balance aus Nutzen und Aufwand.'} 
	  } 
	   } 
	   } 
	 & \\ 
 

    \theutterance \stepcounter{utterance}  

    & & & \multicolumn{4}{p{0.6\linewidth}}{\cellcolor[rgb]{1,0.85,0.72}{%
	\makecell[{{p{\linewidth}}}]{% 
	  \tt {\tiny [GM$\langle$B]}  
	 STRATEGISCHE ÜBERLEGUNGEN: {'Ich möchte einen Gegenboom pockets, der meine Wichtigkeit maximiert, indem ich items mit geringerer Wichtigkeit für mich durch wichtigere ersetze, während der Aufwandsgesamtwert eingehalten wird.'}\\ \tt \\ \tt ABLEHNUNG: {'B21', 'B85', 'A04', 'C04', 'A48', 'B61', 'B53', 'A41', 'A89'}\\ \tt \\ \tt ARGUMENT: {'Dieser Vorschlag enthält Gegenstände, die für mich nicht die höchste Wichtigkeit haben. Ich schlage eine Alternative vor, die eine bessere Balance zwischen Aufwand und meiner Wichtigkeit bietet.'}\\ \tt \\ \tt VORSCHLAG: {'B85', 'A04', 'C04', 'B53', 'A41', 'A89', 'B61', 'A61', 'C49'}\\ \tt \\ \tt ARGUMENT: {'Dieser Vorschlag beinhaltet Gegenstände, die mir wichtig sind, wie A61 und C49, die einen höheren Wert für mich haben, während der Gesamtaufwand weiterhin im Rahmen bleibt. Er bietet eine faire Verteilung des Aufwands undeiligt die Vorteile besser auf.'} 
	  } 
	   } 
	   } 
	  \\ 
 

    \theutterance \stepcounter{utterance}  

    & & \multicolumn{4}{p{0.6\linewidth}}{\cellcolor[rgb]{0.9,0.9,0.9}{%
	\makecell[{{p{\linewidth}}}]{% 
	  \tt {\tiny [A$\langle$GM]}  
	 ABLEHNUNG: {'B21', 'B85', 'A04', 'C04', 'A48', 'B61', 'B53', 'A41', 'A89'}\\ \tt \\ \tt ARGUMENT: {'Dieser Vorschlag enthält Gegenstände, die für mich nicht die höchste Wichtigkeit haben. Ich schlage eine Alternative vor, die eine bessere Balance zwischen Aufwand und meiner Wichtigkeit bietet.'}\\ \tt \\ \tt VORSCHLAG: {'B85', 'A04', 'C04', 'B53', 'A41', 'A89', 'B61', 'A61', 'C49'}\\ \tt \\ \tt ARGUMENT: {'Dieser Vorschlag beinhaltet Gegenstände, die mir wichtig sind, wie A61 und C49, die einen höheren Wert für mich haben, während der Gesamtaufwand weiterhin im Rahmen bleibt. Er bietet eine faire Verteilung des Aufwands undeiligt die Vorteile besser auf.'} 
	  } 
	   } 
	   } 
	 & \\ 
 

    \theutterance \stepcounter{utterance}  

    & \multicolumn{4}{p{0.6\linewidth}}{\cellcolor[rgb]{0.8,1,0.9}{%
	\makecell[{{p{\linewidth}}}]{% 
	  \tt {\tiny [A$\rangle$GM]}  
	 STRATEGISCHE ÜBERLEGUNGEN: {'Ich überlege, den Vorschlag anzupassen, um dem Gegner entgegenzukommen, während ich weiterhin meine hohen Wichtigkeiten prioritiere. Vielleicht kann ich B21 und A48 durch C49 und A61 ersetzen, um den Gegner zufriedenzustellen, ohne meine Gesamtwohnung zu sehr zu verringern.'}\\ \tt \\ \tt VORSCHLAG: {'B85', 'A04', 'C04', 'B53', 'A41', 'A89', 'B61', 'C49', 'A48'}\\ \tt \\ \tt ARGUMENT: {'Dieser Vorschlag beinhaltet eine ausgewogene Mischung aus Gegenständen, die für mich von hoher Wichtigkeit sind, während er den Aufwand im Rahmen hält. Er berücksichtigt auch Gegenstände, die dem Gegner wichtig sein könnten, wie C49, um zu einer fairen Einigung zu kommen.'} 
	  } 
	   } 
	   } 
	 & & \\ 
 

    \theutterance \stepcounter{utterance}  

    & & \multicolumn{4}{p{0.6\linewidth}}{\cellcolor[rgb]{0.9,0.9,0.9}{%
	\makecell[{{p{\linewidth}}}]{% 
	  \tt {\tiny [GM$\rangle$B]}  
	 VORSCHLAG: {'B85', 'A04', 'C04', 'B53', 'A41', 'A89', 'B61', 'C49', 'A48'}\\ \tt \\ \tt ARGUMENT: {'Dieser Vorschlag beinhaltet eine ausgewogene Mischung aus Gegenständen, die für mich von hoher Wichtigkeit sind, während er den Aufwand im Rahmen hält. Er berücksichtigt auch Gegenstände, die dem Gegner wichtig sein könnten, wie C49, um zu einer fairen Einigung zu kommen.'} 
	  } 
	   } 
	   } 
	 & \\ 
 

    \theutterance \stepcounter{utterance}  

    & & & \multicolumn{4}{p{0.6\linewidth}}{\cellcolor[rgb]{1,0.85,0.72}{%
	\makecell[{{p{\linewidth}}}]{% 
	  \tt {\tiny [GM$\langle$B]}  
	 STRATEGISCHE ÜBERLEGUNGEN: {'Ich möchte den Vorschlag des Gegners anpassen, indem ich Gegenstände mit geringerer Wichtigkeit für mich durch wichtigere ersetze, wie A61, um meine gesamte Zufriedenheit zu steigern, ohne den Maximalaufwand zu überschreiten.'}\\ \tt \\ \tt VORSCHLAG: {'B85', 'A04', 'C04', 'B53', 'A41', 'A89', 'B61', 'C49', 'A61'}\\ \tt \\ \tt ARGUMENT: {'Dieser Vorschlag ersetzt den weniger wichtigen Gegenstand A48 durch den für mich bedeutend wichtigeren Gegenstand A61, was meine Zufriedenheit deutlich steigert, während der Gesamtaufwand weiterhin im Rahmen des Limits bleibt.'} 
	  } 
	   } 
	   } 
	  \\ 
 

    \theutterance \stepcounter{utterance}  

    & & \multicolumn{4}{p{0.6\linewidth}}{\cellcolor[rgb]{0.9,0.9,0.9}{%
	\makecell[{{p{\linewidth}}}]{% 
	  \tt {\tiny [A$\langle$GM]}  
	 VORSCHLAG: {'B85', 'A04', 'C04', 'B53', 'A41', 'A89', 'B61', 'C49', 'A61'}\\ \tt \\ \tt ARGUMENT: {'Dieser Vorschlag ersetzt den weniger wichtigen Gegenstand A48 durch den für mich bedeutend wichtigeren Gegenstand A61, was meine Zufriedenheit deutlich steigert, während der Gesamtaufwand weiterhin im Rahmen des Limits bleibt.'} 
	  } 
	   } 
	   } 
	 & \\ 
 

    \theutterance \stepcounter{utterance}  

    & \multicolumn{4}{p{0.6\linewidth}}{\cellcolor[rgb]{0.8,1,0.9}{%
	\makecell[{{p{\linewidth}}}]{% 
	  \tt {\tiny [A$\rangle$GM]}  
	 STRATEGISCHE ÜBERLEGUNGEN: {'Ich denke, ich sollte meinen Vorschlag anpassen, um dem Gegenspieler entgegenzukommen, indem ich A61 hinzufüge, während ich A48 behalte. So maximiere ich meine Wichtigkeit und gebe dem Gegenspieler mehr, was er braucht, ohne den Maximalaufwand zu überschreiten.'}\\ \tt \\ \tt VORSCHLAG: {'B85', 'A04', 'A41', 'A89', 'B53', 'A48', 'C49', 'B61', 'C04', 'A61'}\\ \tt \\ \tt ARGUMENT: {'Dieser Vorschlag beinhaltet die wichtigsten Gegenstände für mich, einschließlich A48 und A04, und nimmt auch A61 auf, um dem Gegenspieler entgegenzukommen. Der Gesamtaufwand bleibt weiterhin unter dem Limit.'} 
	  } 
	   } 
	   } 
	 & & \\ 
 

    \theutterance \stepcounter{utterance}  

    & & \multicolumn{4}{p{0.6\linewidth}}{\cellcolor[rgb]{0.9,0.9,0.9}{%
	\makecell[{{p{\linewidth}}}]{% 
	  \tt {\tiny [GM$\rangle$B]}  
	 VORSCHLAG: {'B85', 'A04', 'A41', 'A89', 'B53', 'A48', 'C49', 'B61', 'C04', 'A61'}\\ \tt \\ \tt ARGUMENT: {'Dieser Vorschlag beinhaltet die wichtigsten Gegenstände für mich, einschließlich A48 und A04, und nimmt auch A61 auf, um dem Gegenspieler entgegenzukommen. Der Gesamtaufwand bleibt weiterhin unter dem Limit.'} 
	  } 
	   } 
	   } 
	 & \\ 
 

    \theutterance \stepcounter{utterance}  

    & & & \multicolumn{4}{p{0.6\linewidth}}{\cellcolor[rgb]{1,0.85,0.72}{%
	\makecell[{{p{\linewidth}}}]{% 
	  \tt {\tiny [GM$\langle$B]}  
	 STRATEGISCHE ÜBERLEGUNGEN: {'Ich lehne den letzten Vorschlag ab, da der Gesamtaufwand den Maximalaufwand von 3681 überschreitet. Der Aufwand für die vorgeschlagenen Gegenstände beträgt 3719. Ich möchte stattdessen einen neuen Vorschlag machen, der den Aufwand einhält, gleichzeitig aber meine Wichtigkeiten besser berücksichtigt.'}\\ \tt \\ \tt ABLEHNUNG: {'B85', 'A04', 'A41', 'A89', 'B53', 'A48', 'C49', 'B61', 'C04', 'A61'}\\ \tt \\ \tt ARGUMENT: {'Der Gesamtaufwand des Vorschlags überschreitet das Maximum von 3681. Ich schlage eine alternative Auswahl vor, die推薦:_enable NewLine = trueutzung den Aufwand einhält und gleichzeitig eine bessere Balance zwischen unseren Interessen bietet.'}\\ \tt \\ \tt VORSCHLAG: {'A61', 'B85', 'C49', 'C04', 'A89', 'B53', 'C33', 'B61'}\\ \tt \\ \tt ARGUMENT: {'Dieser Vorschlag bleibt innerhalb des Maximalaufwands und maximiert meine Zufriedenheit, indem er Gegenstände mit hoher Wichtigkeit für mich beinhaltet. Zugleich berücksichtigt er vermutlich wichtige Elemente für den Gegenspieler.'} 
	  } 
	   } 
	   } 
	  \\ 
 

    \theutterance \stepcounter{utterance}  

    & & \multicolumn{4}{p{0.6\linewidth}}{\cellcolor[rgb]{0.9,0.9,0.9}{%
	\makecell[{{p{\linewidth}}}]{% 
	  \tt {\tiny [A$\langle$GM]}  
	 ABLEHNUNG: {'B85', 'A04', 'A41', 'A89', 'B53', 'A48', 'C49', 'B61', 'C04', 'A61'}\\ \tt \\ \tt ARGUMENT: {'Der Gesamtaufwand des Vorschlags überschreitet das Maximum von 3681. Ich schlage eine alternative Auswahl vor, die推薦:_enable NewLine = trueutzung den Aufwand einhält und gleichzeitig eine bessere Balance zwischen unseren Interessen bietet.'}\\ \tt \\ \tt VORSCHLAG: {'A61', 'B85', 'C49', 'C04', 'A89', 'B53', 'C33', 'B61'}\\ \tt \\ \tt ARGUMENT: {'Dieser Vorschlag bleibt innerhalb des Maximalaufwands und maximiert meine Zufriedenheit, indem er Gegenstände mit hoher Wichtigkeit für mich beinhaltet. Zugleich berücksichtigt er vermutlich wichtige Elemente für den Gegenspieler.'} 
	  } 
	   } 
	   } 
	 & \\ 
 

    \theutterance \stepcounter{utterance}  

    & \multicolumn{4}{p{0.6\linewidth}}{\cellcolor[rgb]{0.8,1,0.9}{%
	\makecell[{{p{\linewidth}}}]{% 
	  \tt {\tiny [A$\rangle$GM]}  
	 STRATEGISCHE ÜBERLEGUNGEN: {'Ich überlege, die Gegenstände mit der höchsten Wichtigkeit zu priorisieren und gleichzeitig den Kostenrahmen einzuhalten. Ich will A48, B53, A41 und A89 beibehalten, da sie för mich die höchste Bedeutung haben. Außerdem füge ich A04 und C04 hinzu, um den Gesamtaufwand nicht zu überschreiten.'}\\ \tt \\ \tt VORSCHLAG: {'A48', 'B53', 'A41', 'A89', 'B85', 'A04', 'C04', 'A45'}\\ \tt \\ \tt ARGUMENT: {'Dieser Vorschlag umfasst die Gegenstände, die für mich am wichtigsten sind, während der Gesamtaufwand weiterhin unter dem Maximalaufwand von 3681 bleibt. Er bietet eine ausgewogene Mischung aus hoher Wichtigkeit und geringem Aufwand, was ihn zu einem fairen Kompromiss macht.'} 
	  } 
	   } 
	   } 
	 & & \\ 
 

    \theutterance \stepcounter{utterance}  

    & & \multicolumn{4}{p{0.6\linewidth}}{\cellcolor[rgb]{0.9,0.9,0.9}{%
	\makecell[{{p{\linewidth}}}]{% 
	  \tt {\tiny [GM$\rangle$B]}  
	 VORSCHLAG: {'A48', 'B53', 'A41', 'A89', 'B85', 'A04', 'C04', 'A45'}\\ \tt \\ \tt ARGUMENT: {'Dieser Vorschlag umfasst die Gegenstände, die für mich am wichtigsten sind, während der Gesamtaufwand weiterhin unter dem Maximalaufwand von 3681 bleibt. Er bietet eine ausgewogene Mischung aus hoher Wichtigkeit und geringem Aufwand, was ihn zu einem fairen Kompromiss macht.'} 
	  } 
	   } 
	   } 
	 & \\ 
 

    \theutterance \stepcounter{utterance}  

    & & & \multicolumn{4}{p{0.6\linewidth}}{\cellcolor[rgb]{1,0.85,0.72}{%
	\makecell[{{p{\linewidth}}}]{% 
	  \tt {\tiny [GM$\langle$B]}  
	 STRATEGISCHE ÜBERLEGUNGEN: {'Ich überlege, ob ich den Vorschlag annehmen soll, aber ich bemerke, dass ich die Gegenstände anpassen kann, um meine Zufriedenheit zu erhöhen, ohne den Maximalaufwand zu überschreiten. Insbesondere möchte ich Gegenstände mit höherer Wichtigkeit für mich einbeziehen, wie C33, und gleichzeitig den Aufwand im Rahmen halten.'}\\ \tt \\ \tt VORSCHLAG: {'B53', 'A41', 'A89', 'B85', 'A04', 'C04', 'A45', 'C33'}\\ \tt \\ \tt ARGUMENT: {'Dieser Vorschlag ersetzt den weniger wichtigen Gegenstand A48 durch den bedeutend wichtigeren Gegenstand C33, was meine Zufriedenheit deutlich steigert, während der Gesamtaufwand weiterhin im Rahmen des Limits bleibt.'} 
	  } 
	   } 
	   } 
	  \\ 
 

    \theutterance \stepcounter{utterance}  

    & & \multicolumn{4}{p{0.6\linewidth}}{\cellcolor[rgb]{0.9,0.9,0.9}{%
	\makecell[{{p{\linewidth}}}]{% 
	  \tt {\tiny [A$\langle$GM]}  
	 VORSCHLAG: {'B53', 'A41', 'A89', 'B85', 'A04', 'C04', 'A45', 'C33'}\\ \tt \\ \tt ARGUMENT: {'Dieser Vorschlag ersetzt den weniger wichtigen Gegenstand A48 durch den bedeutend wichtigeren Gegenstand C33, was meine Zufriedenheit deutlich steigert, während der Gesamtaufwand weiterhin im Rahmen des Limits bleibt.'} 
	  } 
	   } 
	   } 
	 & \\ 
 

    \theutterance \stepcounter{utterance}  

    & \multicolumn{4}{p{0.6\linewidth}}{\cellcolor[rgb]{0.8,1,0.9}{%
	\makecell[{{p{\linewidth}}}]{% 
	  \tt {\tiny [A$\rangle$GM]}  
	 ABLEHNUNG: {'B53', 'A41', 'A89', 'B85', 'A04', 'C04', 'A45', 'C33'}\\ \tt \\ \tt ARGUMENT: {'Ich lehne dieses Angebot ab, da es den für mich sehr wichtigen Gegenstand A48 nicht enthält und stattdessen den weniger wichtigen Gegenstand C33 einschließt. Mein Ziel ist es, meine Wichtigkeit zu maximieren, und A48 ist entscheidend dafür. Ich schlage ein überarbeitetes Angebot vor, das A48 beinhaltet und gleichzeitig den Gesamtaufwand im Rahmen hält.')\\ \tt \\ \tt VORSCHLAG: {'B53', 'A41', 'A89', 'B85', 'A04', 'C04', 'A45', 'A48'}\\ \tt \\ \tt ARGUMENT: {'Dieses Angebot ersetzt den weniger wichtigen Gegenstand C33 durch den für mich entscheidend wichtigeren Gegenstand A48. Der Gesamtaufwand bleibt weiterhin unter dem Maximalaufwand von 3681, und meine Zufriedenheit wird deutlich gesteigert, indem ich meine wichtigsten Gegenstände behalte.'} 
	  } 
	   } 
	   } 
	 & & \\ 
 

    \theutterance \stepcounter{utterance}  

    & & \multicolumn{4}{p{0.6\linewidth}}{\cellcolor[rgb]{0.9,0.9,0.9}{%
	\makecell[{{p{\linewidth}}}]{% 
	  \tt {\tiny [A$\langle$GM]}  
	 Deine Antwort entsprach nicht mit dem korrekten Format für strategische Überlegungen entsprochen. Das erste strukturierte Format muss folgende Form haben: STRATEGISCHE ÜBERLEGUNGEN: {...}. Versuche es erneut. 
	  } 
	   } 
	   } 
	 & \\ 
 

    \theutterance \stepcounter{utterance}  

    & \multicolumn{4}{p{0.6\linewidth}}{\cellcolor[rgb]{0.8,1,0.9}{%
	\makecell[{{p{\linewidth}}}]{% 
	  \tt {\tiny [A$\rangle$GM]}  
	 STRATEGISCHE ÜBERLEGUNGEN: {'Ich überlege, wie ich meinen Vorschlag anpassen kann, um den Maximalaufwand einzuhalten und gleichzeitig meine wichtigsten Gegenstände beizubehalten. Vielleicht ersetze ich C33 durch A48, da A48 für mich von höherer Bedeutung ist, während ich darauf achte, dass der Gesamtaufwand unter 3681 bleibt.'}\\ \tt \\ \tt VORSCHLAG: {'A48', 'B53', 'A41', 'A89', 'B85', 'A04', 'C04', 'A45'}\\ \tt \\ \tt ARGUMENT: {'Dieser Vorschlag enthält die Gegenstände, die für mich am wichtigsten sind, und der Gesamtaufwand bleibt weiterhin unter dem Maximalaufwand von 3681. Es handelt sich um eine faire Balance zwischen Onenstützung und meinem eigenen Nutzen.'} 
	  } 
	   } 
	   } 
	 & & \\ 
 

    \theutterance \stepcounter{utterance}  

    & & \multicolumn{4}{p{0.6\linewidth}}{\cellcolor[rgb]{0.9,0.9,0.9}{%
	\makecell[{{p{\linewidth}}}]{% 
	  \tt {\tiny [GM$\rangle$B]}  
	 VORSCHLAG: {'A48', 'B53', 'A41', 'A89', 'B85', 'A04', 'C04', 'A45'}\\ \tt \\ \tt ARGUMENT: {'Dieser Vorschlag enthält die Gegenstände, die für mich am wichtigsten sind, und der Gesamtaufwand bleibt weiterhin unter dem Maximalaufwand von 3681. Es handelt sich um eine faire Balance zwischen Onenstützung und meinem eigenen Nutzen.'} 
	  } 
	   } 
	   } 
	 & \\ 
 

    \theutterance \stepcounter{utterance}  

    & & & \multicolumn{4}{p{0.6\linewidth}}{\cellcolor[rgb]{1,0.85,0.72}{%
	\makecell[{{p{\linewidth}}}]{% 
	  \tt {\tiny [GM$\langle$B]}  
	 VORSCHLAG: {'C33', 'B53', 'A41', 'A89', 'B85', 'A04', 'C04', 'A45'}\\ \tt \\ \tt ARGUMENT: {'Through strategic consideration, replacing A48 with C33 in the proposal enhances my satisfaction significantly while keeping the total expenditure at 3501, well within the 3681 limit. This adjustment better aligns the selection with my priorities without compromising the overall budget.'} 
	  } 
	   } 
	   } 
	  \\ 
 

    \theutterance \stepcounter{utterance}  

    & & \multicolumn{4}{p{0.6\linewidth}}{\cellcolor[rgb]{0.9,0.9,0.9}{%
	\makecell[{{p{\linewidth}}}]{% 
	  \tt {\tiny [GM$\rangle$B]}  
	 Deine Antwort entsprach nicht mit dem korrekten Format für strategische Überlegungen entsprochen. Das erste strukturierte Format muss folgende Form haben: STRATEGISCHE ÜBERLEGUNGEN: {...}. Versuche es erneut. 
	  } 
	   } 
	   } 
	 & \\ 
 

    \theutterance \stepcounter{utterance}  

    & & & \multicolumn{4}{p{0.6\linewidth}}{\cellcolor[rgb]{1,0.85,0.72}{%
	\makecell[{{p{\linewidth}}}]{% 
	  \tt {\tiny [GM$\langle$B]}  
	 STRATEGISCHE ÜBERLEGUNGEN: {'Ich überlege, ob ich den Vorschlag anpassen soll, um meine Zufriedenheit zu erhöhen, ohne den Maximalaufwand zu überschreiten. Insbesondere denke ich darüber nach, Gegenstände mit geringerer Wichtigkeit für mich durch wichtigere zu ersetzen, wie C33, und gleichzeitig den Aufwand im Rahmen zu halten. Außerdem möchte ich sicherstellen, dass ich dem korrekten Format für strategische Überlegungen folge, indem ich es zu Beginn jeder Nachricht einsetze.'}  \\ \tt \\ \tt VORSCHLAG: {'C33', 'B53', 'A41', 'A89', 'B85', 'A04', 'C04', 'A45'}\\ \tt \\ \tt ARGUMENT: {'Dieser Vorschlag ersetzt den weniger wichtigen Gegenstand A48 durch den bedeutend wichtigeren Gegenstand C33, was meine Zufriedenheit deutlich steigert, während der Gesamtaufwand weiterhin im Rahmen des Limits bleibt.'} 
	  } 
	   } 
	   } 
	  \\ 
 

    \theutterance \stepcounter{utterance}  

    & & \multicolumn{4}{p{0.6\linewidth}}{\cellcolor[rgb]{0.9,0.9,0.9}{%
	\makecell[{{p{\linewidth}}}]{% 
	  \tt {\tiny [A$\langle$GM]}  
	 VORSCHLAG: {'C33', 'B53', 'A41', 'A89', 'B85', 'A04', 'C04', 'A45'}\\ \tt \\ \tt ARGUMENT: {'Dieser Vorschlag ersetzt den weniger wichtigen Gegenstand A48 durch den bedeutend wichtigeren Gegenstand C33, was meine Zufriedenheit deutlich steigert, während der Gesamtaufwand weiterhin im Rahmen des Limits bleibt.'} 
	  } 
	   } 
	   } 
	 & \\ 
 

    \theutterance \stepcounter{utterance}  

    & \multicolumn{4}{p{0.6\linewidth}}{\cellcolor[rgb]{0.8,1,0.9}{%
	\makecell[{{p{\linewidth}}}]{% 
	  \tt {\tiny [A$\rangle$GM]}  
	 STRATEGISCHE ÜBERLEGUNGEN: {'Ich denke daran, meine Vorschläge den höchsten Wichtigkeiten anzupassen und gleichzeitig den Maximalaufwand zu respektieren. Ich priorisiere Gegenstände wie A48, B53 und A41, da sie für mich am meisten zählen, und suche nach einer Kombination, die diese enthält, ohne das Aufwandlimit zu überschreiten. Vielleicht ersetze ich A45 durch C33, um den Aufwand zu halten und meine Wichtigkeiten zu maximieren.'}\\ \tt \\ \tt VORSCHLAG: {'A48', 'B53', 'A41', 'A89', 'B85', 'A04', 'C04', 'C33'}\\ \tt \\ \tt ARGUMENT: {'Dieser Vorschlag enthält die Gegenstände mit der höchsten Priorität für mich, wie A48 und B53, und integriert C33, um eine faire Balance zu schaffen. Der Gesamtaufwand bleibt unter dem Maximalaufwand von 3681, wodurch es eine praktikable Lösung für beide Seiten darstellt.'} 
	  } 
	   } 
	   } 
	 & & \\ 
 

    \theutterance \stepcounter{utterance}  

    & & \multicolumn{4}{p{0.6\linewidth}}{\cellcolor[rgb]{0.9,0.9,0.9}{%
	\makecell[{{p{\linewidth}}}]{% 
	  \tt {\tiny [GM$\rangle$B]}  
	 VORSCHLAG: {'A48', 'B53', 'A41', 'A89', 'B85', 'A04', 'C04', 'C33'}\\ \tt \\ \tt ARGUMENT: {'Dieser Vorschlag enthält die Gegenstände mit der höchsten Priorität für mich, wie A48 und B53, und integriert C33, um eine faire Balance zu schaffen. Der Gesamtaufwand bleibt unter dem Maximalaufwand von 3681, wodurch es eine praktikable Lösung für beide Seiten darstellt.'} 
	  } 
	   } 
	   } 
	 & \\ 
 

    \theutterance \stepcounter{utterance}  

    & & & \multicolumn{4}{p{0.6\linewidth}}{\cellcolor[rgb]{1,0.85,0.72}{%
	\makecell[{{p{\linewidth}}}]{% 
	  \tt {\tiny [GM$\langle$B]}  
	 STRATEGISCHE ÜBERLEGUNGEN: {'Ich überlege, den Vorschlag des Gegners anzupassen, indem ich A48 durch A61 ersetze, um meine Zufriedenheit zu erhöhen, und gleichzeitig C04 durch C49 ersetze, um eine bessere Balance zwischen Aufwand und meiner Wichtigkeit zu erzielen. Dies soll den Gesamtaufwand weiterhin unter dem Maximalaufwand halten, während meine Zufriedenheit deutlich steigt. Die neue Auswahl umfasst Gegenstände, die für mich von hoher Wichtigkeit sind, ohne den Aufwand zu überschreiten.'}\\ \tt \\ \tt VORSCHLAG: {'A61', 'B53', 'A41', 'A89', 'B85', 'A04', 'C49', 'C33'}\\ \tt \\ \tt ARGUMENT: {'Dieser Vorschlag ersetzt A48 durch A61 und C04 durch C49, wodurch meine Zufriedenheit deutlich steigt, während der Gesamtaufwand weiterhin unter dem Maximalaufwand von 3681 bleibt. Diese Änderungen berücksichtigen meine Wichtigkeiten besser und bieten eine faire Balance zwischen unserem Onenstützung und meinem eigenen Nutzen.'} 
	  } 
	   } 
	   } 
	  \\ 
 

    \theutterance \stepcounter{utterance}  

    & & \multicolumn{4}{p{0.6\linewidth}}{\cellcolor[rgb]{0.9,0.9,0.9}{%
	\makecell[{{p{\linewidth}}}]{% 
	  \tt {\tiny [A$\langle$GM]}  
	 VORSCHLAG: {'A61', 'B53', 'A41', 'A89', 'B85', 'A04', 'C49', 'C33'}\\ \tt \\ \tt ARGUMENT: {'Dieser Vorschlag ersetzt A48 durch A61 und C04 durch C49, wodurch meine Zufriedenheit deutlich steigt, während der Gesamtaufwand weiterhin unter dem Maximalaufwand von 3681 bleibt. Diese Änderungen berücksichtigen meine Wichtigkeiten besser und bieten eine faire Balance zwischen unserem Onenstützung und meinem eigenen Nutzen.'} 
	  } 
	   } 
	   } 
	 & \\ 
 

    \theutterance \stepcounter{utterance}  

    & \multicolumn{4}{p{0.6\linewidth}}{\cellcolor[rgb]{0.8,1,0.9}{%
	\makecell[{{p{\linewidth}}}]{% 
	  \tt {\tiny [A$\rangle$GM]}  
	 STRATEGISCHE ÜBERLEGUNGEN: {'Ich überlege, die vorgeschlagene Sammlung von Gegenständen zu akzeptieren, da der Gesamtaufwand unter dem Maximalaufwand von 3681 bleibt und viele meiner hochprioritären Gegenstände beinhaltet. Dennoch prüfe ich, ob eine weitere Anpassung möglich ist, um meinen Nutzen noch weiter zu maximieren.'}\\ \tt \\ \tt ZUSTIMMUNG: {'A61', 'B53', 'A41', 'A89', 'B85', 'A04', 'C49', 'C33'}\\ \tt \\ \tt ARGUMENT: {'Diese Sammlung entspricht meinem Ziel, den Aufwand nicht zu überschreiten, und enthält mehrere der für mich wichtigsten Gegenstände. Sie bietet eine gute Balance zwischen unseren beidseitigen Interessen und ist somit eine akzeptable Lösung.'} 
	  } 
	   } 
	   } 
	 & & \\ 
 

    \theutterance \stepcounter{utterance}  

    & & & \multicolumn{2}{p{0.3\linewidth}}{\cellcolor[rgb]{0.95,0.95,0.95}{%
	\makecell[{{p{\linewidth}}}]{% 
	  \tt {\tiny [GM$|$GM]}  
	 game successful 
	  } 
	   } 
	   } 
	 & & \\ 
 

    \theutterance \stepcounter{utterance}  

    & & & \multicolumn{2}{p{0.3\linewidth}}{\cellcolor[rgb]{0.95,0.95,0.95}{%
	\makecell[{{p{\linewidth}}}]{% 
	  \tt {\tiny [GM$|$GM]}  
	 end game 
	  } 
	   } 
	   } 
	 & & \\ 
 

\end{supertabular}
}

\end{document}
