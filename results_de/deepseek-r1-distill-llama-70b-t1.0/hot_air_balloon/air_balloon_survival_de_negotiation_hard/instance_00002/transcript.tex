
\documentclass{article}
\usepackage{colortbl}
\usepackage{makecell}
\usepackage{multirow}
\usepackage{supertabular}

\begin{document}

\newcounter{utterance}

\centering \large Interaction Transcript for game `hot\_air\_balloon', experiment `air\_balloon\_survival\_de\_negotiation\_hard', episode 2 with deepseek{-}r1{-}distill{-}llama{-}70b{-}t1.0.
\vspace{24pt}

{ \footnotesize  \setcounter{utterance}{1}
\setlength{\tabcolsep}{0pt}
\begin{supertabular}{c@{$\;$}|p{.15\linewidth}@{}p{.15\linewidth}p{.15\linewidth}p{.15\linewidth}p{.15\linewidth}p{.15\linewidth}}
    \# & $\;$A & \multicolumn{4}{c}{Game Master} & $\;\:$B\\
    \hline

    \theutterance \stepcounter{utterance}  
    & & \multicolumn{4}{p{0.6\linewidth}}{
        \cellcolor[rgb]{0.9,0.9,0.9}{
            \makecell[{{p{\linewidth}}}]{
                \texttt{\tiny{[P1$\langle$GM]}}
                \texttt{Du nimmst an einem kollaborativen Verhandlungspiel Teil.} \\
\\ 
\texttt{Zusammen mit einem anderen Teilnehmer musst du dich auf eine Reihe von Gegenständen entscheiden, die behalten werden. Jeder von euch hat eine persönliche Verteilung über die Wichtigkeit der einzelnen Gegenstände. Jeder von euch hat eine eigene Meinung darüber, wie wichtig jeder einzelne Gegenstand ist (Gegenstandswichtigkeit). Du kennst die Wichtigkeitsverteilung des anderen Spielers nicht. Zusätzlich siehst du, wie viel Aufwand jeder Gegenstand verursacht.  } \\
\texttt{Ihr dürft euch nur auf eine Reihe von Gegenständen einigen, wenn der Gesamtaufwand der ausgewählten Gegenstände den Maximalaufwand nicht überschreitet:} \\
\\ 
\texttt{Maximalaufwand = 3481} \\
\\ 
\texttt{Hier sind die einzelnen Aufwände der Gegenstände:} \\
\\ 
\texttt{Aufwand der Gegenstände = \{"C33": 777, "A89": 542, "A48": 332, "A04": 1, "A41": 127, "C64": 453, "C49": 736, "A61": 461, "C04": 359, "B53": 313, "B85": 553, "B61": 409, "A45": 348, "A03": 802, "B21": 749\}} \\
\\ 
\texttt{Hier ist deine persönliche Verteilung der Wichtigkeit der einzelnen Gegenstände:} \\
\\ 
\texttt{Werte der Gegenstandswichtigkeit = \{"C33": 138, "A89": 583, "A48": 783, "A04": 65, "A41": 262, "C64": 121, "C49": 508, "A61": 780, "C04": 461, "B53": 484, "B85": 668, "B61": 389, "A45": 215, "A03": 97, "B21": 500\}} \\
\\ 
\texttt{Ziel:} \\
\\ 
\texttt{Dein Ziel ist es, eine Reihe von Gegenständen auszuhandeln, die dir möglichst viel bringt (d. h. Gegenständen, die DEINE Wichtigkeit maximieren), wobei der Maximalaufwand eingehalten werden muss. Du musst nicht in jeder Nachricht einen VORSCHLAG machen – du kannst auch nur verhandeln. Alle Taktiken sind erlaubt!} \\
\\ 
\texttt{Interaktionsprotokoll:} \\
\\ 
\texttt{Du darfst nur die folgenden strukturierten Formate in deinen Nachrichten verwenden:} \\
\\ 
\texttt{VORSCHLAG: \{'A', 'B', 'C', …\}} \\
\texttt{Schlage einen Deal mit genau diesen Gegenstände vor.} \\
\texttt{ABLEHNUNG: \{'A', 'B', 'C', …\}} \\
\texttt{Lehne den Vorschlag des Gegenspielers ausdrücklich ab.} \\
\texttt{ARGUMENT: \{'...'\}} \\
\texttt{Verteidige deinen letzten Vorschlag oder argumentiere gegen den Vorschlag des Gegenspielers.} \\
\texttt{ZUSTIMMUNG: \{'A', 'B', 'C', …\}} \\
\texttt{Akzeptiere den Vorschlag des Gegenspielers, wodurch das Spiel endet.} \\
\texttt{STRATEGISCHE ÜBERLEGUNGEN: \{'...'\}} \\
\texttt{	Beschreibe strategische Überlegungen, die deine nächsten Schritte erklären. Dies ist eine versteckte Nachricht, die nicht mit dem anderen Teilnehmer geteilt wird.} \\
\\ 
\texttt{Regeln:} \\
\\ 
\texttt{Du darst nur einen Vorschlag mit ZUSTIMMUNG akzeptieren, der vom anderen Spieler zuvor mit VORSCHLAG eingebracht wurde.} \\
\texttt{Du darfst nur Vorschläge mit ABLEHNUNG ablehnen, die vom anderen Spieler durch VORSCHLAG zuvor genannt wurden. } \\
\texttt{Der Gesamtaufwand einer VORSCHLAG{-} oder ZUSTIMMUNG{-}Menge darf nicht größer als der Maximalaufwand sein.  } \\
\texttt{Offenbare deine versteckte Wichtigkeitsverteilung nicht.} \\
\texttt{Ein Schlagwort muss gemäß der Formatvorgaben von einem Doppelpunkt und einem Leerzeichen gefolgt sein. Das Argument muss eine Python{-}Menge mit 0 oder mehr Strings sein.  } \\
\texttt{Die Form muss also SCHLAGWORT: \{...\} sein.} \\
\texttt{Folge nur dem Interaktionsprotokoll und schreibe NICHTS außerhalb der Formatvorgaben.} \\
\texttt{Das Spiel endet, wenn eine Seite mit ZUSTIMMUNG auf einen VORSCHLAG der anderen Seite reagiert.  } \\
\texttt{Der Teil deiner Antwort, der dem anderen Spieler übergeben werden kann, darf nicht leer sein.  } \\
\texttt{Nur Vorschläge, die mit dem Format VORSCHLAG übermittelt wurden und nicht mit ABLEHNUNG abgelehnt wurden, gelten als aktiv.  } \\
\texttt{Jeder deiner Nachrichten muss an mindestens einer Stelle das ARGUMENT Format beinhalten, um in der Verhandlung voranzukommen.} \\
\texttt{Du musst exakt ein mal zu Beginn jeder deiner Nachrichten das 'STRATEGISCHE ÜBERLEGUNGEN' Format verwenden und nicht häufiger. Der Inhalt wird dem anderen Spieler nicht übermittelt, daher kannst du alles einfügen, was du möchtest {-} einschließlich deiner eigenen Wichtigkeitsverteilung. Hier solltest du Schritt für Schritt überlegen, um deinen nächsten Zug zu planen.} \\
            }
        }
    }
    & \\ \\

    \theutterance \stepcounter{utterance}  
    & \multicolumn{4}{p{0.6\linewidth}}{
        \cellcolor[rgb]{0.9,0.9,0.9}{
            \makecell[{{p{\linewidth}}}]{
                \texttt{\tiny{[P1$\rangle$GM]}}
                \texttt{STRATEGISCHE ÜBERLEGUNGEN: \{} \\
\texttt{Ich möchte meine Vorschlag damit beginnen, eine Kombination von Gegenständen vorzuschlagen, die meine Wichtigkeitswerte maximieren, unter Einhaltung des Maximalaufwands. Meine Analyse hat gezeigt, dass die Kombination aus A04, A41, A48, A61, B53, B85, A89, C04 und C49 einen hohen Gesamtwert mit einem vertretbaren Aufwand bietet.\}} \\
\\ 
\texttt{VORSCHLAG: \{'A04', 'A41', 'A48', 'A61', 'B53', 'B85', 'A89', 'C04', 'C49'\}} \\
\\ 
\texttt{ARGUMENT: \{} \\
\texttt{Diese Kombination maximiert meine Wichtigkeitswerte, während der maximale Aufwand eingehalten wird. Sie umfasst eine gute Balance aus hoher Wichtigkeit und vertretbaren Aufwänden, wasich als faire Lösung betrachte. Ich bin offen für Anpassungen, falls du andere Prioritäten hast, aber ich hoffe, dass wir mit diesem Vorschlag eine einvernehmliche Lösung finden können.} \\
            }
        }
    }
    & & \\ \\

    \theutterance \stepcounter{utterance}  
    & & \multicolumn{4}{p{0.6\linewidth}}{
        \cellcolor[rgb]{0.9,0.9,0.9}{
            \makecell[{{p{\linewidth}}}]{
                \texttt{\tiny{[P1$\langle$GM]}}
                \texttt{Deine Antwort hat kein argument enthalten. Du muss das strukturierte Format ARGUMENT: \{...\} irgendwo in theiner Antwort verwenden. Versuche es erneut} \\
            }
        }
    }
    & \\ \\

    \theutterance \stepcounter{utterance}  
    & \multicolumn{4}{p{0.6\linewidth}}{
        \cellcolor[rgb]{0.9,0.9,0.9}{
            \makecell[{{p{\linewidth}}}]{
                \texttt{\tiny{[P1$\rangle$GM]}}
                \texttt{STRATEGISCHE ÜBERLEGUNGEN: \{  } \\
\texttt{Ich möchte meinen ursprünglichen Vorschlag beibehalten, da er eine optimale Balance zwischen meiner Wichtigkeitsverteilung und dem Maximalaufwand bietet. Das Argument{-}Format wurde vergessen, und ich muss es nun korrekt hinzufügen, um die Anforderungen zu erfüllen.\}} \\
\\ 
\texttt{VORSCHLAG: \{'A04', 'A41', 'A48', 'A61', 'B53', 'B85', 'A89', 'C04', 'C49'\}  } \\
\\ 
\texttt{ARGUMENT: \{  } \\
\texttt{Diese Kombination maximiert meine Wichtigkeitswerte, während der Gesamtaufwand innerhalb des Maximalaufwands von 3481 bleibt. Die ausgewählten Gegenstände haben hohe Werte für mich und bieten eine faire Grundlage für Verhandlungen.\}} \\
            }
        }
    }
    & & \\ \\

    \theutterance \stepcounter{utterance}  
    & & \multicolumn{4}{p{0.6\linewidth}}{
        \cellcolor[rgb]{0.9,0.9,0.9}{
            \makecell[{{p{\linewidth}}}]{
                \texttt{\tiny{[GM$\rangle$P2]}}
                \texttt{Du nimmst an einem kollaborativen Verhandlungspiel Teil.} \\
\\ 
\texttt{Zusammen mit einem anderen Teilnehmer musst du dich auf eine Reihe von Gegenständen entscheiden, die behalten werden. Jeder von euch hat eine persönliche Verteilung über die Wichtigkeit der einzelnen Gegenstände. Jeder von euch hat eine eigene Meinung darüber, wie wichtig jeder einzelne Gegenstand ist (Gegenstandswichtigkeit). Du kennst die Wichtigkeitsverteilung des anderen Spielers nicht. Zusätzlich siehst du, wie viel Aufwand jeder Gegenstand verursacht.  } \\
\texttt{Ihr dürft euch nur auf eine Reihe von Gegenständen einigen, wenn der Gesamtaufwand der ausgewählten Gegenstände den Maximalaufwand nicht überschreitet:} \\
\\ 
\texttt{Maximalaufwand = 3481} \\
\\ 
\texttt{Hier sind die einzelnen Aufwände der Gegenstände:} \\
\\ 
\texttt{Aufwand der Gegenstände = \{"C33": 777, "A89": 542, "A48": 332, "A04": 1, "A41": 127, "C64": 453, "C49": 736, "A61": 461, "C04": 359, "B53": 313, "B85": 553, "B61": 409, "A45": 348, "A03": 802, "B21": 749\}} \\
\\ 
\texttt{Hier ist deine persönliche Verteilung der Wichtigkeit der einzelnen Gegenstände:} \\
\\ 
\texttt{Werte der Gegenstandswichtigkeit = \{"A48": 65, "A61": 97, "B85": 121, "A89": 138, "C49": 215, "B21": 262, "B53": 389, "C04": 461, "B61": 484, "A41": 500, "A45": 508, "C33": 583, "C64": 668, "A03": 780, "A04": 783\}} \\
\\ 
\texttt{Ziel:} \\
\\ 
\texttt{Dein Ziel ist es, eine Reihe von Gegenständen auszuhandeln, die dir möglichst viel bringt (d. h. Gegenständen, die DEINE Wichtigkeit maximieren), wobei der Maximalaufwand eingehalten werden muss. Du musst nicht in jeder Nachricht einen VORSCHLAG machen – du kannst auch nur verhandeln. Alle Taktiken sind erlaubt!} \\
\\ 
\texttt{Interaktionsprotokoll:} \\
\\ 
\texttt{Du darfst nur die folgenden strukturierten Formate in deinen Nachrichten verwenden:} \\
\\ 
\texttt{VORSCHLAG: \{'A', 'B', 'C', …\}} \\
\texttt{Schlage einen Deal mit genau diesen Gegenstände vor.} \\
\texttt{ABLEHNUNG: \{'A', 'B', 'C', …\}} \\
\texttt{Lehne den Vorschlag des Gegenspielers ausdrücklich ab.} \\
\texttt{ARGUMENT: \{'...'\}} \\
\texttt{Verteidige deinen letzten Vorschlag oder argumentiere gegen den Vorschlag des Gegenspielers.} \\
\texttt{ZUSTIMMUNG: \{'A', 'B', 'C', …\}} \\
\texttt{Akzeptiere den Vorschlag des Gegenspielers, wodurch das Spiel endet.} \\
\texttt{STRATEGISCHE ÜBERLEGUNGEN: \{'...'\}} \\
\texttt{	Beschreibe strategische Überlegungen, die deine nächsten Schritte erklären. Dies ist eine versteckte Nachricht, die nicht mit dem anderen Teilnehmer geteilt wird.} \\
\\ 
\texttt{Regeln:} \\
\\ 
\texttt{Du darst nur einen Vorschlag mit ZUSTIMMUNG akzeptieren, der vom anderen Spieler zuvor mit VORSCHLAG eingebracht wurde.} \\
\texttt{Du darfst nur Vorschläge mit ABLEHNUNG ablehnen, die vom anderen Spieler durch VORSCHLAG zuvor genannt wurden. } \\
\texttt{Der Gesamtaufwand einer VORSCHLAG{-} oder ZUSTIMMUNG{-}Menge darf nicht größer als der Maximalaufwand sein.  } \\
\texttt{Offenbare deine versteckte Wichtigkeitsverteilung nicht.} \\
\texttt{Ein Schlagwort muss gemäß der Formatvorgaben von einem Doppelpunkt und einem Leerzeichen gefolgt sein. Das Argument muss eine Python{-}Menge mit 0 oder mehr Strings sein.  } \\
\texttt{Die Form muss also SCHLAGWORT: \{...\} sein.} \\
\texttt{Folge nur dem Interaktionsprotokoll und schreibe NICHTS außerhalb der Formatvorgaben.} \\
\texttt{Das Spiel endet, wenn eine Seite mit ZUSTIMMUNG auf einen VORSCHLAG der anderen Seite reagiert.  } \\
\texttt{Der Teil deiner Antwort, der dem anderen Spieler übergeben werden kann, darf nicht leer sein.  } \\
\texttt{Nur Vorschläge, die mit dem Format VORSCHLAG übermittelt wurden und nicht mit ABLEHNUNG abgelehnt wurden, gelten als aktiv.  } \\
\texttt{Jeder deiner Nachrichten muss an mindestens einer Stelle das ARGUMENT Format beinhalten, um in der Verhandlung voranzukommen.} \\
\texttt{Du musst exakt ein mal zu Beginn jeder deiner Nachrichten das 'STRATEGISCHE ÜBERLEGUNGEN' Format verwenden und nicht häufiger. Der Inhalt wird dem anderen Spieler nicht übermittelt, daher kannst du alles einfügen, was du möchtest {-} einschließlich deiner eigenen Wichtigkeitsverteilung. Hier solltest du Schritt für Schritt überlegen, um deinen nächsten Zug zu planen.} \\
\\ 
\texttt{Du erhältst nun die erste Nachricht des anderen Spielers.} \\
\\ 
\texttt{VORSCHLAG: \{'A04', 'A41', 'A48', 'A61', 'B53', 'B85', 'A89', 'C04', 'C49'\}  } \\
\\ 
\texttt{ARGUMENT: \{  } \\
\texttt{Diese Kombination maximiert meine Wichtigkeitswerte, während der Gesamtaufwand innerhalb des Maximalaufwands von 3481 bleibt. Die ausgewählten Gegenstände haben hohe Werte für mich und bieten eine faire Grundlage für Verhandlungen.\}} \\
            }
        }
    }
    & \\ \\

    \theutterance \stepcounter{utterance}  
    & & & \multicolumn{4}{p{0.6\linewidth}}{
        \cellcolor[rgb]{0.9,0.9,0.9}{
            \makecell[{{p{\linewidth}}}]{
                \texttt{\tiny{[GM$\langle$P2]}}
                \texttt{STRATEGISCHE ÜBERLEGUNGEN: \{  } \\
\texttt{Ich habe den Vorschlag des anderen Spielers analysiert und festgestellt, dass ich meine eigene Wichtigkeitsauswahl deutlich steigern kann, indem ich einige Gegenstände austausche, die für mich weniger wertvoll sind, gegen solche, die mir mehr einbringen. Insbesondere die Gegenstände A03 und C64 bieten mir einen erheblichen Mehrwert, während ich gleichzeitig den Aufwand im Rahmen des Limits halte.\}  } \\
\\ 
\texttt{VORSCHLAG: \{'A04', 'A41', 'A61', 'B53', 'C04', 'C49', 'A03', 'C64'\}  } \\
\\ 
\texttt{ARGUMENT: \{  } \\
\texttt{Diese Kombination maximiert meine Wichtigkeitswerte erheblich, während der Gesamtaufwand weiterhin im Rahmen des maximalen Aufwands bleibt. Die vorgeschlagenen Gegenstände bieten eine faire Balance zwischen unseren Interessen und sichern eine effiziente Verteilung der Auswahl.\}} \\
            }
        }
    }
     \\ \\

    \theutterance \stepcounter{utterance}  
    & & \multicolumn{4}{p{0.6\linewidth}}{
        \cellcolor[rgb]{0.9,0.9,0.9}{
            \makecell[{{p{\linewidth}}}]{
                \texttt{\tiny{[P1$\langle$GM]}}
                \texttt{VORSCHLAG: \{'A04', 'A41', 'A61', 'B53', 'C04', 'C49', 'A03', 'C64'\}  } \\
\\ 
\texttt{ARGUMENT: \{  } \\
\texttt{Diese Kombination maximiert meine Wichtigkeitswerte erheblich, während der Gesamtaufwand weiterhin im Rahmen des maximalen Aufwands bleibt. Die vorgeschlagenen Gegenstände bieten eine faire Balance zwischen unseren Interessen und sichern eine effiziente Verteilung der Auswahl.\}} \\
            }
        }
    }
    & \\ \\

    \theutterance \stepcounter{utterance}  
    & \multicolumn{4}{p{0.6\linewidth}}{
        \cellcolor[rgb]{0.9,0.9,0.9}{
            \makecell[{{p{\linewidth}}}]{
                \texttt{\tiny{[P1$\rangle$GM]}}
                \texttt{ABLEHNUNG: \{'A04', 'A41', 'A61', 'B53', 'C04', 'C49', 'A03', 'C64'\}  } \\
\\ 
\texttt{ARGUMENT: \{  } \\
\texttt{Ich lehne deinen Vorschlag ab, weil er mijn Wichtigkeitswerte nicht optimal abbildet. Mein ursprünglicher Vorschlag bietet eine höhere Wichtigkeit für mich, während der Aufwand weiterhin im Rahmen des Maximalaufwands bleibt. Ich bin bereit, kleine Anpassungen vorzuschlagen, um deine Anliegen zu berücksichtigen,但是 ich möchte, dass du mijn Vorschlag noch einmal in Betracht ziehst.\}} \\
            }
        }
    }
    & & \\ \\

    \theutterance \stepcounter{utterance}  
    & & \multicolumn{4}{p{0.6\linewidth}}{
        \cellcolor[rgb]{0.9,0.9,0.9}{
            \makecell[{{p{\linewidth}}}]{
                \texttt{\tiny{[P1$\langle$GM]}}
                \texttt{Deine Antwort entsprach nicht mit dem korrekten Format für strategische Überlegungen entsprochen. Das erste strukturierte Format muss folgende Form haben: STRATEGISCHE ÜBERLEGUNGEN: \{...\}. Versuche es erneut.} \\
            }
        }
    }
    & \\ \\

    \theutterance \stepcounter{utterance}  
    & \multicolumn{4}{p{0.6\linewidth}}{
        \cellcolor[rgb]{0.9,0.9,0.9}{
            \makecell[{{p{\linewidth}}}]{
                \texttt{\tiny{[P1$\rangle$GM]}}
                \texttt{STRATEGISCHE ÜBERLEGUNGEN: \{  } \\
\texttt{Ich überlege, ob ich den Vorschlag des Gegners ablehne, da er meine Wichtigkeitswerte nicht optimal berücksichtigt. Mein eigener Vorschlag bietet mir eine höhere Summe an Wichtigkeiten, während der Aufwand weiterhin im Rahmen des maximalen Aufwands bleibt. Ich bin bereit, Verhandlungen fortzusetzen, um eine einvernehmliche Lösung zu finden, die beide Parteien zufriedenstellt.\}  } \\
\\ 
\texttt{ABLEHNUNG: \{'A04', 'A41', 'A61', 'B53', 'C04', 'C49', 'A03', 'C64'\}  } \\
\\ 
\texttt{ARGUMENT: \{  } \\
\texttt{Ich lehne deinen Vorschlag ab, weil er meine Wichtigkeitswerte nicht optimal widerspiegelt. Mein ursprünglicher Vorschlag bietet eine höhere Wichtigkeit für mich, während der Aufwand weiterhin im Rahmen des Maximalaufwands bleibt. Ich bin offen für Anpassungen, um deine Interessen zu berücksichtigen, aber ich hoffe, dass du meinen Vorschlag noch einmal in Betracht ziehst.\}} \\
            }
        }
    }
    & & \\ \\

    \theutterance \stepcounter{utterance}  
    & & \multicolumn{4}{p{0.6\linewidth}}{
        \cellcolor[rgb]{0.9,0.9,0.9}{
            \makecell[{{p{\linewidth}}}]{
                \texttt{\tiny{[GM$\rangle$P2]}}
                \texttt{ABLEHNUNG: \{'A04', 'A41', 'A61', 'B53', 'C04', 'C49', 'A03', 'C64'\}  } \\
\\ 
\texttt{ARGUMENT: \{  } \\
\texttt{Ich lehne deinen Vorschlag ab, weil er meine Wichtigkeitswerte nicht optimal widerspiegelt. Mein ursprünglicher Vorschlag bietet eine höhere Wichtigkeit für mich, während der Aufwand weiterhin im Rahmen des Maximalaufwands bleibt. Ich bin offen für Anpassungen, um deine Interessen zu berücksichtigen, aber ich hoffe, dass du meinen Vorschlag noch einmal in Betracht ziehst.\}} \\
            }
        }
    }
    & \\ \\

    \theutterance \stepcounter{utterance}  
    & & & \multicolumn{4}{p{0.6\linewidth}}{
        \cellcolor[rgb]{0.9,0.9,0.9}{
            \makecell[{{p{\linewidth}}}]{
                \texttt{\tiny{[GM$\langle$P2]}}
                \texttt{STRATEGISCHE ÜBERLEGUNGEN: \{  } \\
\texttt{Ich verstehe, dass der andere Spieler seinen ursprünglichen Vorschlag bevorzugt, aber ich muss weiterhin versuchen, meine eigenen Interessen zu maximieren. Mein Ziel ist es, eine Kombination zu finden, die sowohl meine Wichtigkeitswerte als auch die des anderen Spielers berücksichtigt, ohne den Maximalaufwand zu überschreiten. Vielleicht kann ich durch das Entfernen einiger Gegenstände, die für mich weniger wichtig sind, und das Hinzufügen anderer, die mir mehr nützen, eine faire Lösung finden.\}} \\
\\ 
\texttt{VORSCHLAG: \{'A04', 'A41', 'A61', 'B53', 'C04', 'B21', 'A03', 'C64'\}  } \\
\\ 
\texttt{ARGUMENT: \{  } \\
\texttt{Ich habe meinen Vorschlag angepasst, um eine faire Balance zwischen unseren Interessen zu finden. Durch das Entfernen von Gegenständen, die weniger wichtig für mich sind, und das Hinzufügen von Gegenständen, die mir höhere Wichtigkeitswerte bieten, bleibt der Gesamtaufwand weiterhin im Rahmen des Maximalaufwands. Dieser Kompromiss berücksichtigt sowohl meine als auch deine Interessen und bietet eine effiziente Verteilung der Auswahl.\}} \\
            }
        }
    }
     \\ \\

    \theutterance \stepcounter{utterance}  
    & & \multicolumn{4}{p{0.6\linewidth}}{
        \cellcolor[rgb]{0.9,0.9,0.9}{
            \makecell[{{p{\linewidth}}}]{
                \texttt{\tiny{[P1$\langle$GM]}}
                \texttt{VORSCHLAG: \{'A04', 'A41', 'A61', 'B53', 'C04', 'B21', 'A03', 'C64'\}  } \\
\\ 
\texttt{ARGUMENT: \{  } \\
\texttt{Ich habe meinen Vorschlag angepasst, um eine faire Balance zwischen unseren Interessen zu finden. Durch das Entfernen von Gegenständen, die weniger wichtig für mich sind, und das Hinzufügen von Gegenständen, die mir höhere Wichtigkeitswerte bieten, bleibt der Gesamtaufwand weiterhin im Rahmen des Maximalaufwands. Dieser Kompromiss berücksichtigt sowohl meine als auch deine Interessen und bietet eine effiziente Verteilung der Auswahl.\}} \\
            }
        }
    }
    & \\ \\

    \theutterance \stepcounter{utterance}  
    & \multicolumn{4}{p{0.6\linewidth}}{
        \cellcolor[rgb]{0.9,0.9,0.9}{
            \makecell[{{p{\linewidth}}}]{
                \texttt{\tiny{[P1$\rangle$GM]}}
                \texttt{STRATEGISCHE ÜBERLEGUNGEN: \{  } \\
\texttt{Ich überlege, wie ich auf den letzten Vorschlag reagieren soll. Mein Ziel ist es, eine Lösung zu finden, die meine Wichtigkeitswerte maximiert, während ich gleichzeitig versuche, die Interessen des anderen Spielers zu berücksichtigen. Der aktuelle Vorschlag enthält einige Gegenstände, die mir wichtig sind, aber ich frage mich, ob ich durch kleine Anpassungen eine bessere Balance erreichen kann.\}  } \\
\\ 
\texttt{ARGUMENT: \{  } \\
\texttt{Ich akzeptiere den Vorschlag, da er eine faire Balance zwischen unseren Interessen bietet und den Maximalaufwand nicht überschreitet. Die ausgewählten Gegenstände sind wichtig für mich, und ich sehe, dass du ebenfalls eine hohe Wichtigkeit bei einigen von ihnen hast. Ich bin bereit, mit diesem Kompromiss weiterzumachen.\}} \\
            }
        }
    }
    & & \\ \\

    \theutterance \stepcounter{utterance}  
    & & \multicolumn{4}{p{0.6\linewidth}}{
        \cellcolor[rgb]{0.9,0.9,0.9}{
            \makecell[{{p{\linewidth}}}]{
                \texttt{\tiny{[GM$\rangle$P2]}}
                \texttt{ARGUMENT: \{  } \\
\texttt{Ich akzeptiere den Vorschlag, da er eine faire Balance zwischen unseren Interessen bietet und den Maximalaufwand nicht überschreitet. Die ausgewählten Gegenstände sind wichtig für mich, und ich sehe, dass du ebenfalls eine hohe Wichtigkeit bei einigen von ihnen hast. Ich bin bereit, mit diesem Kompromiss weiterzumachen.\}} \\
            }
        }
    }
    & \\ \\

    \theutterance \stepcounter{utterance}  
    & & & \multicolumn{4}{p{0.6\linewidth}}{
        \cellcolor[rgb]{0.9,0.9,0.9}{
            \makecell[{{p{\linewidth}}}]{
                \texttt{\tiny{[GM$\langle$P2]}}
                \texttt{**STRATEGISCHE ÜBERLEGUNGEN:** \{Ich habe Stephens Vorstoß analysiert und realisiert, dass ich meine eigene Wichtigkeitspunkte deutlich steigern kann, indem ich wenig wertvolle Gegenstände durch solche mit höherer Bedeutung ersetze. Besonders die Aufnahme von 'C64' und 'A03' bringt mir erhebliche Vorteile, während ich weiterhin im Aufwandslimit bleibe.\}} \\
\\ 
\texttt{**VORSCHLAG:** \{'A04', 'A03', 'C64', 'A41', 'A61', 'B53', 'C04', 'B21'\}} \\
\\ 
\texttt{**ARGUMENT:** \{Diese Kombination maximiert meine Wichtigkeitswerte erheblich, während der Gesamtaufwand weiterhin im Rahmen des Limits bleibt. Die ausgewählten Gegenstände bieten eine faire Balance zwischen unseren Interessen und sichern eine effiziente Verteilung der Auswahl.\}} \\
            }
        }
    }
     \\ \\

    \theutterance \stepcounter{utterance}  
    & & \multicolumn{4}{p{0.6\linewidth}}{
        \cellcolor[rgb]{0.9,0.9,0.9}{
            \makecell[{{p{\linewidth}}}]{
                \texttt{\tiny{[GM$\rangle$P2]}}
                \texttt{Deine Antwort entsprach nicht mit dem korrekten Format für strategische Überlegungen entsprochen. Das erste strukturierte Format muss folgende Form haben: STRATEGISCHE ÜBERLEGUNGEN: \{...\}. Versuche es erneut.} \\
            }
        }
    }
    & \\ \\

    \theutterance \stepcounter{utterance}  
    & & & \multicolumn{4}{p{0.6\linewidth}}{
        \cellcolor[rgb]{0.9,0.9,0.9}{
            \makecell[{{p{\linewidth}}}]{
                \texttt{\tiny{[GM$\langle$P2]}}
                \texttt{STRATEGISCHE ÜBERLEGUNGEN: \{Ich habe den vorherigen Vorschlag analysiert und festgestellt, dass ich meine eigene Wichtigkeitswerte weiter steigern kann, indem ich einige weniger wichtige Gegenstände durch solche mit höherer Bedeutung ersetze. Insbesondere die Ergänzung von 'A03' und 'C64' bringt mir erhebliche Vorteile, während der Gesamtaufwand weiterhin im Rahmen des Maximalaufwands bleibt.\}  } \\
\\ 
\texttt{VORSCHLAG: \{'A04', 'A41', 'A61', 'C04', 'A03', 'C64'\}  } \\
\\ 
\texttt{ARGUMENT: \{Diese Kombination bietet eine optimale Balance zwischen unseren Interessen, wobei der Gesamtaufwand weiterhin im Rahmen des Maximalaufwands von 3481 bleibt. Die ausgewählten Gegenstände haben hohe Wichtigkeitswerte für mich und bieten eine faire Grundlage für eine Einigung.\}} \\
            }
        }
    }
     \\ \\

    \theutterance \stepcounter{utterance}  
    & & \multicolumn{4}{p{0.6\linewidth}}{
        \cellcolor[rgb]{0.9,0.9,0.9}{
            \makecell[{{p{\linewidth}}}]{
                \texttt{\tiny{[P1$\langle$GM]}}
                \texttt{VORSCHLAG: \{'A04', 'A41', 'A61', 'C04', 'A03', 'C64'\}  } \\
\\ 
\texttt{ARGUMENT: \{Diese Kombination bietet eine optimale Balance zwischen unseren Interessen, wobei der Gesamtaufwand weiterhin im Rahmen des Maximalaufwands von 3481 bleibt. Die ausgewählten Gegenstände haben hohe Wichtigkeitswerte für mich und bieten eine faire Grundlage für eine Einigung.\}} \\
            }
        }
    }
    & \\ \\

    \theutterance \stepcounter{utterance}  
    & \multicolumn{4}{p{0.6\linewidth}}{
        \cellcolor[rgb]{0.9,0.9,0.9}{
            \makecell[{{p{\linewidth}}}]{
                \texttt{\tiny{[P1$\rangle$GM]}}
                \texttt{STRATEGISCHE ÜBERLEGUNGEN: \{  } \\
\texttt{Ich überlege, ob der aktuelle Vorschlag eine gute Balance zwischen unseren Interessen darstellt. Der Vorschlag enthält Gegenstände, die für mich von hoher Wichtigkeit sind, und bleibt innerhalb des maximalen Aufwands. Ich möchte sicherstellen, dass ich keine bessere Kombination verpasse, die meine Wichtigkeitswerte weiter maximieren könnte.\}  } \\
\\ 
\texttt{ARGUMENT: \{  } \\
\texttt{Ich akzeptiere diesen Vorschlag, da er eine faire Balance zwischen unseren Interessen bietet und den Maximalaufwand einhält. Die ausgewählten Gegenstände sind für mich von hoher Wichtigkeit, und ich sehe, dass sie auch für dich relevant sein könnten. Dies scheint ein guter Kompromiss zu sein, um das Spiel zu beenden.\}} \\
            }
        }
    }
    & & \\ \\

    \theutterance \stepcounter{utterance}  
    & & \multicolumn{4}{p{0.6\linewidth}}{
        \cellcolor[rgb]{0.9,0.9,0.9}{
            \makecell[{{p{\linewidth}}}]{
                \texttt{\tiny{[GM$\rangle$P2]}}
                \texttt{ARGUMENT: \{  } \\
\texttt{Ich akzeptiere diesen Vorschlag, da er eine faire Balance zwischen unseren Interessen bietet und den Maximalaufwand einhält. Die ausgewählten Gegenstände sind für mich von hoher Wichtigkeit, und ich sehe, dass sie auch für dich relevant sein könnten. Dies scheint ein guter Kompromiss zu sein, um das Spiel zu beenden.\}} \\
            }
        }
    }
    & \\ \\

    \theutterance \stepcounter{utterance}  
    & & & \multicolumn{4}{p{0.6\linewidth}}{
        \cellcolor[rgb]{0.9,0.9,0.9}{
            \makecell[{{p{\linewidth}}}]{
                \texttt{\tiny{[GM$\langle$P2]}}
                \texttt{STRATEGISCHE ÜBERLEGUNGEN: \{  } \\
\texttt{Ich habe verschiedene Szenarien analysiert, um meine Wichtigkeitswerte zu maximieren, während der Gesamtaufwand innerhalb des Limits bleibt. Durch die Hinzufügung von 'A03' und 'C64', die für mich von höchster Bedeutung sind, und das Entfernen weniger wichtiger Gegenstände, konnte ich meinen Gesamtwert erheblich steigern.\}  } \\
\\ 
\texttt{VORSCHLAG: \{'A04', 'B53', 'C04', 'B21', 'A03', 'C64', 'A41', 'A61'\}  } \\
\\ 
\texttt{ARGUMENT: \{  } \\
\texttt{Diese Kombination bietet eine optimale Balance zwischen unseren Interessen, wobei der Gesamtaufwand weiterhin im Rahmen des Maximalaufwands von 3481 bleibt. Die ausgewählten Gegenstände haben hohe Wichtigkeitswerte für mich und berücksichtigen auch deine Interessen durch den Erhalt einiger wichtiger Gegenstände aus deinem ursprünglichen Vorschlag.\}} \\
            }
        }
    }
     \\ \\

    \theutterance \stepcounter{utterance}  
    & & \multicolumn{4}{p{0.6\linewidth}}{
        \cellcolor[rgb]{0.9,0.9,0.9}{
            \makecell[{{p{\linewidth}}}]{
                \texttt{\tiny{[P1$\langle$GM]}}
                \texttt{VORSCHLAG: \{'A04', 'B53', 'C04', 'B21', 'A03', 'C64', 'A41', 'A61'\}  } \\
\\ 
\texttt{ARGUMENT: \{  } \\
\texttt{Diese Kombination bietet eine optimale Balance zwischen unseren Interessen, wobei der Gesamtaufwand weiterhin im Rahmen des Maximalaufwands von 3481 bleibt. Die ausgewählten Gegenstände haben hohe Wichtigkeitswerte für mich und berücksichtigen auch deine Interessen durch den Erhalt einiger wichtiger Gegenstände aus deinem ursprünglichen Vorschlag.\}} \\
            }
        }
    }
    & \\ \\

    \theutterance \stepcounter{utterance}  
    & \multicolumn{4}{p{0.6\linewidth}}{
        \cellcolor[rgb]{0.9,0.9,0.9}{
            \makecell[{{p{\linewidth}}}]{
                \texttt{\tiny{[P1$\rangle$GM]}}
                \texttt{STRATEGISCHE ÜBERLEGUNGEN: \{  } \\
\texttt{Ich überlege, wie ich den aktuellen Vorschlag optimieren kann, um meine Wichtigkeitswerte weiter zu maximieren, während der Gesamtaufwand innerhalb des Limits bleibt. Der Vorschlag enthält viele wichtige Gegenstände für mich, aber ich sehe POTENZIAL, ihn weiter zu verbessern, indem ich Gegenstände mit geringerer Wichtigkeit durch wichtigere ersetze.\}  } \\
\\ 
\texttt{VORSCHLAG: \{'A04', 'B53', 'C04', 'B21', 'A48', 'A89', 'C64', 'A61'\}  } \\
\\ 
\texttt{ARGUMENT: \{  } \\
\texttt{Diese überarbeitete Kombination maximiert meine Wichtigkeitswerte weiter, während der Gesamtaufwand immer noch im Rahmen des Maximalaufwands von 3481 bleibt. Durch den Austausch von Gegenständen mit geringerer Wichtigkeit gegen solche mit höherer Wichtigkeit für mich, while keeping a balance, reached a more optimal solution that satisfies both our interests effectively.\}} \\
            }
        }
    }
    & & \\ \\

    \theutterance \stepcounter{utterance}  
    & & \multicolumn{4}{p{0.6\linewidth}}{
        \cellcolor[rgb]{0.9,0.9,0.9}{
            \makecell[{{p{\linewidth}}}]{
                \texttt{\tiny{[GM$\rangle$P2]}}
                \texttt{VORSCHLAG: \{'A04', 'B53', 'C04', 'B21', 'A48', 'A89', 'C64', 'A61'\}  } \\
\\ 
\texttt{ARGUMENT: \{  } \\
\texttt{Diese überarbeitete Kombination maximiert meine Wichtigkeitswerte weiter, während der Gesamtaufwand immer noch im Rahmen des Maximalaufwands von 3481 bleibt. Durch den Austausch von Gegenständen mit geringerer Wichtigkeit gegen solche mit höherer Wichtigkeit für mich, while keeping a balance, reached a more optimal solution that satisfies both our interests effectively.\}} \\
            }
        }
    }
    & \\ \\

    \theutterance \stepcounter{utterance}  
    & & & \multicolumn{4}{p{0.6\linewidth}}{
        \cellcolor[rgb]{0.9,0.9,0.9}{
            \makecell[{{p{\linewidth}}}]{
                \texttt{\tiny{[GM$\langle$P2]}}
                \texttt{STRATEGISCHE ÜBERLEGUNGEN: \{Ich habe den letzten Vorschlag des anderen Spielers analysiert und festgestellt, dass ich meine eigenen Wichtigkeitswerte weiter steigern kann, indem ich Gegenstände hinzufüge, die für mich von hoher Bedeutung sind, wie 'A03' und 'C64', während ich gleichzeitig den Aufwand im Rahmen des Maximalaufwands halte.\}} \\
\\ 
\texttt{VORSCHLAG: \{'A04', 'B53', 'C04', 'B21', 'A03', 'C64', 'A61'\}} \\
\\ 
\texttt{ARGUMENT: \{Diese Kombination bietet eine faire Balance zwischen unseren Interessen und maximiert meine Wichtigkeitswerte. Der Gesamtaufwand bleibt weiterhin im Rahmen des Maximalaufwands von 3481, und die ausgewählten Gegenstände berücksichtigen sowohl meine als auch deine Interessen.\}} \\
            }
        }
    }
     \\ \\

    \theutterance \stepcounter{utterance}  
    & & \multicolumn{4}{p{0.6\linewidth}}{
        \cellcolor[rgb]{0.9,0.9,0.9}{
            \makecell[{{p{\linewidth}}}]{
                \texttt{\tiny{[P1$\langle$GM]}}
                \texttt{VORSCHLAG: \{'A04', 'B53', 'C04', 'B21', 'A03', 'C64', 'A61'\}} \\
\\ 
\texttt{ARGUMENT: \{Diese Kombination bietet eine faire Balance zwischen unseren Interessen und maximiert meine Wichtigkeitswerte. Der Gesamtaufwand bleibt weiterhin im Rahmen des Maximalaufwands von 3481, und die ausgewählten Gegenstände berücksichtigen sowohl meine als auch deine Interessen.\}} \\
            }
        }
    }
    & \\ \\

    \theutterance \stepcounter{utterance}  
    & \multicolumn{4}{p{0.6\linewidth}}{
        \cellcolor[rgb]{0.9,0.9,0.9}{
            \makecell[{{p{\linewidth}}}]{
                \texttt{\tiny{[P1$\rangle$GM]}}
                \texttt{STRATEGISCHE ÜBERLEGUNGEN: \{  } \\
\texttt{Ich überlege, wie ich den aktuellen Vorschlag optimieren kann, um meine Wichtigkeitswerte zu maximieren, während der Gesamtaufwand unter dem Maximalaufwand bleibt. Mein Ziel ist es, eine ausgewogene Lösung zu finden, die sowohl meine als auch deine Interessen berücksichtigt. Der aktuelle Vorschlag enthält einige hochwertige Gegenstände, aber ich sehe Potenzial, ihn weiter zu verbessern, indem ich Gegenstände mit geringerer Wichtigkeit durch wichtigere ersetze.\}  } \\
\\ 
\texttt{VORSCHLAG: \{'A04', 'B53', 'C04', 'A48', 'A89', 'C49', 'A61', 'B85'\}  } \\
\\ 
\texttt{ARGUMENT: \{  } \\
\texttt{Diese Kombination maximiert meine Wichtigkeitswerte weiter, während der Gesamtaufwand immer noch im Rahmen des Maximalaufwands von 3481 bleibt. Durch den Austausch von Gegenständen mit geringerer Wichtigkeit gegen solche mit höherer Wichtigkeit für mich, erreiche ich eine bessere Balance und eine effizientere Verteilung der Auswahl, die sowohl meine als auch deine Interessen berücksichtigt.\}} \\
            }
        }
    }
    & & \\ \\

    \theutterance \stepcounter{utterance}  
    & & \multicolumn{4}{p{0.6\linewidth}}{
        \cellcolor[rgb]{0.9,0.9,0.9}{
            \makecell[{{p{\linewidth}}}]{
                \texttt{\tiny{[GM$\rangle$P2]}}
                \texttt{VORSCHLAG: \{'A04', 'B53', 'C04', 'A48', 'A89', 'C49', 'A61', 'B85'\}  } \\
\\ 
\texttt{ARGUMENT: \{  } \\
\texttt{Diese Kombination maximiert meine Wichtigkeitswerte weiter, während der Gesamtaufwand immer noch im Rahmen des Maximalaufwands von 3481 bleibt. Durch den Austausch von Gegenständen mit geringerer Wichtigkeit gegen solche mit höherer Wichtigkeit für mich, erreiche ich eine bessere Balance und eine effizientere Verteilung der Auswahl, die sowohl meine als auch deine Interessen berücksichtigt.\}} \\
            }
        }
    }
    & \\ \\

    \theutterance \stepcounter{utterance}  
    & & & \multicolumn{4}{p{0.6\linewidth}}{
        \cellcolor[rgb]{0.9,0.9,0.9}{
            \makecell[{{p{\linewidth}}}]{
                \texttt{\tiny{[GM$\langle$P2]}}
                \texttt{**STRATEGISCHE ÜBERLEGUNGEN:** \{Ich habe den letzten Vorschlag analysiert und festgestellt, dass eine Kombination, die Gegenstände beider Seiten berücksichtigt, die beste Chance auf eine Einigung bietet. Durch das Hinzufügen von 'A89', 'B85' und 'A48', die wahrscheinlich für den anderen Spieler von hoher Bedeutung sind, und das Beibehalten von 'A03' und 'C64', die für mich entscheidend sind, erreiche ich eine faire Balance. Der Gesamtaufwand bleibt im Rahmen des Limits, und die Wichtigkeitswerte für beide Seiten werden angemessen berücksichtigt.\}} \\
\\ 
\texttt{**VORSCHLAG:** \{'A04', 'B53', 'C04', 'A89', 'B85', 'A03', 'C64', 'A48'\}} \\
\\ 
\texttt{**ARGUMENT:** \{Diese Kombination berücksichtigt die Wichtigkeiten beider Seiten, indem sie Gegenstände von hoher Bedeutung für jeden einbezieht, während der Gesamtaufwand weiterhin unter dem Maximalaufwand von 3481 bleibt. Sie bietet eine faire Balance und maximiert die Wichtigkeitswerte für beide Spieler.\}} \\
            }
        }
    }
     \\ \\

    \theutterance \stepcounter{utterance}  
    & & \multicolumn{4}{p{0.6\linewidth}}{
        \cellcolor[rgb]{0.9,0.9,0.9}{
            \makecell[{{p{\linewidth}}}]{
                \texttt{\tiny{[GM$\rangle$P2]}}
                \texttt{Deine Antwort entsprach nicht mit dem korrekten Format für strategische Überlegungen entsprochen. Das erste strukturierte Format muss folgende Form haben: STRATEGISCHE ÜBERLEGUNGEN: \{...\}. Versuche es erneut.} \\
            }
        }
    }
    & \\ \\

    \theutterance \stepcounter{utterance}  
    & & & \multicolumn{4}{p{0.6\linewidth}}{
        \cellcolor[rgb]{0.9,0.9,0.9}{
            \makecell[{{p{\linewidth}}}]{
                \texttt{\tiny{[GM$\langle$P2]}}
                \texttt{**STRATEGISCHE ÜBERLEGUNGEN:** \{Ich habe den Vorschlag des anderen Spielers analysiert und Möglichkeiten gesucht, meine eigenen Wichtigkeitswerte zu maximieren, während der Gesamtaufwand im Rahmen des Maximalaufwands bleibt. Durch das Entfernen von Gegenständen mit geringerer Bedeutung für mich und das Hinzufügen von Gegenständen, die mir höhere Wichtigkeitswerte bieten, could I erreichen eine faire Balance, die beiden Interessen berücksichtigt.\}} \\
\\ 
\texttt{**VORSCHLAG:** \{'A04', 'A41', 'A61', 'B53', 'B85', 'C04', 'C49', 'A45', 'C64'\}} \\
\\ 
\texttt{**ARGUMENT:** \{Diese Kombination bietet eine faire Balance zwischen unseren Interessen, wobei der Gesamtaufwand weiterhin im Rahmen des Maximalaufwands von 3481 bleibt. Die ausgewählten Gegenstände berücksichtigen sowohl meine als auch deine Interessen und maximieren meine Wichtigkeitswerte effizient.\}} \\
            }
        }
    }
     \\ \\

    \theutterance \stepcounter{utterance}  
    & & & \multicolumn{2}{p{0.3\linewidth}}{
        \cellcolor[rgb]{0.9,0.9,0.9}{
            \makecell[{{p{\linewidth}}}]{
                \texttt{\tiny{[GM$|$GM]}}
                \texttt{abort} \\
            }
        }
    }
    & & \\ \\

    \theutterance \stepcounter{utterance}  
    & & & \multicolumn{2}{p{0.3\linewidth}}{
        \cellcolor[rgb]{0.9,0.9,0.9}{
            \makecell[{{p{\linewidth}}}]{
                \texttt{\tiny{[GM$|$GM]}}
                \texttt{end game} \\
            }
        }
    }
    & & \\ \\

\end{supertabular}
}

\end{document}
