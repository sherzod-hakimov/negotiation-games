
\documentclass{article}
\usepackage{colortbl}
\usepackage{makecell}
\usepackage{multirow}
\usepackage{supertabular}

\begin{document}

\newcounter{utterance}

\centering \large Interaction Transcript for game `hot\_air\_balloon', experiment `air\_balloon\_survival\_de\_complexity\_easy', episode 2 with deepseek{-}r1{-}distill{-}llama{-}70b{-}t1.0.
\vspace{24pt}

{ \footnotesize  \setcounter{utterance}{1}
\setlength{\tabcolsep}{0pt}
\begin{supertabular}{c@{$\;$}|p{.15\linewidth}@{}p{.15\linewidth}p{.15\linewidth}p{.15\linewidth}p{.15\linewidth}p{.15\linewidth}}
    \# & $\;$A & \multicolumn{4}{c}{Game Master} & $\;\:$B\\
    \hline

    \theutterance \stepcounter{utterance}  
    & & \multicolumn{4}{p{0.6\linewidth}}{
        \cellcolor[rgb]{0.9,0.9,0.9}{
            \makecell[{{p{\linewidth}}}]{
                \texttt{\tiny{[P1$\langle$GM]}}
                \texttt{Du nimmst an einem kollaborativen Verhandlungspiel Teil.} \\
\\ 
\texttt{Zusammen mit einem anderen Teilnehmer musst du dich auf eine Reihe von Gegenständen entscheiden, die behalten werden. Jeder von euch hat eine persönliche Verteilung über die Wichtigkeit der einzelnen Gegenstände. Jeder von euch hat eine eigene Meinung darüber, wie wichtig jeder einzelne Gegenstand ist (Gegenstandswichtigkeit). Du kennst die Wichtigkeitsverteilung des anderen Spielers nicht. Zusätzlich siehst du, wie viel Aufwand jeder Gegenstand verursacht.  } \\
\texttt{Ihr dürft euch nur auf eine Reihe von Gegenständen einigen, wenn der Gesamtaufwand der ausgewählten Gegenstände den Maximalaufwand nicht überschreitet:} \\
\\ 
\texttt{Maximalaufwand = 7868} \\
\\ 
\texttt{Hier sind die einzelnen Aufwände der Gegenstände:} \\
\\ 
\texttt{Aufwand der Gegenstände = \{"C32": 552, "C52": 560, "B54": 76, "C16": 950, "C50": 248, "A08": 871, "C59": 391, "B51": 143, "B55": 293, "B59": 208, "A77": 678, "C14": 737, "C24": 407, "B24": 366, "C31": 768, "A52": 859, "C61": 183, "B79": 231, "C54": 305, "B56": 727, "A40": 148, "B11": 357, "B60": 504, "A81": 549, "A95": 299, "B38": 91, "C51": 527, "C03": 848, "A91": 307, "A93": 214, "C28": 723, "C05": 475, "B75": 23, "B96": 298, "B86": 820\}} \\
\\ 
\texttt{Hier ist deine persönliche Verteilung der Wichtigkeit der einzelnen Gegenstände:} \\
\\ 
\texttt{Werte der Gegenstandswichtigkeit = \{"C32": 138, "C52": 583, "B54": 868, "C16": 822, "C50": 783, "A08": 65, "C59": 262, "B51": 121, "B55": 508, "B59": 780, "A77": 461, "C14": 484, "C24": 668, "B24": 389, "C31": 808, "A52": 215, "C61": 97, "B79": 500, "C54": 30, "B56": 915, "A40": 856, "B11": 400, "B60": 444, "A81": 623, "A95": 781, "B38": 786, "C51": 3, "C03": 713, "A91": 457, "A93": 273, "C28": 739, "C05": 822, "B75": 235, "B96": 606, "B86": 105\}} \\
\\ 
\texttt{Ziel:} \\
\\ 
\texttt{Dein Ziel ist es, eine Reihe von Gegenständen auszuhandeln, die dir möglichst viel bringt (d. h. Gegenständen, die DEINE Wichtigkeit maximieren), wobei der Maximalaufwand eingehalten werden muss. Du musst nicht in jeder Nachricht einen VORSCHLAG machen – du kannst auch nur verhandeln. Alle Taktiken sind erlaubt!} \\
\\ 
\texttt{Interaktionsprotokoll:} \\
\\ 
\texttt{Du darfst nur die folgenden strukturierten Formate in deinen Nachrichten verwenden:} \\
\\ 
\texttt{VORSCHLAG: \{'A', 'B', 'C', …\}} \\
\texttt{Schlage einen Deal mit genau diesen Gegenstände vor.} \\
\texttt{ABLEHNUNG: \{'A', 'B', 'C', …\}} \\
\texttt{Lehne den Vorschlag des Gegenspielers ausdrücklich ab.} \\
\texttt{ARGUMENT: \{'...'\}} \\
\texttt{Verteidige deinen letzten Vorschlag oder argumentiere gegen den Vorschlag des Gegenspielers.} \\
\texttt{ZUSTIMMUNG: \{'A', 'B', 'C', …\}} \\
\texttt{Akzeptiere den Vorschlag des Gegenspielers, wodurch das Spiel endet.} \\
\texttt{STRATEGISCHE ÜBERLEGUNGEN: \{'...'\}} \\
\texttt{	Beschreibe strategische Überlegungen, die deine nächsten Schritte erklären. Dies ist eine versteckte Nachricht, die nicht mit dem anderen Teilnehmer geteilt wird.} \\
\\ 
\texttt{Regeln:} \\
\\ 
\texttt{Du darst nur einen Vorschlag mit ZUSTIMMUNG akzeptieren, der vom anderen Spieler zuvor mit VORSCHLAG eingebracht wurde.} \\
\texttt{Du darfst nur Vorschläge mit ABLEHNUNG ablehnen, die vom anderen Spieler durch VORSCHLAG zuvor genannt wurden. } \\
\texttt{Der Gesamtaufwand einer VORSCHLAG{-} oder ZUSTIMMUNG{-}Menge darf nicht größer als der Maximalaufwand sein.  } \\
\texttt{Offenbare deine versteckte Wichtigkeitsverteilung nicht.} \\
\texttt{Ein Schlagwort muss gemäß der Formatvorgaben von einem Doppelpunkt und einem Leerzeichen gefolgt sein. Das Argument muss eine Python{-}Menge mit 0 oder mehr Strings sein.  } \\
\texttt{Die Form muss also SCHLAGWORT: \{...\} sein.} \\
\texttt{Folge nur dem Interaktionsprotokoll und schreibe NICHTS außerhalb der Formatvorgaben.} \\
\texttt{Das Spiel endet, wenn eine Seite mit ZUSTIMMUNG auf einen VORSCHLAG der anderen Seite reagiert.  } \\
\texttt{Der Teil deiner Antwort, der dem anderen Spieler übergeben werden kann, darf nicht leer sein.  } \\
\texttt{Nur Vorschläge, die mit dem Format VORSCHLAG übermittelt wurden und nicht mit ABLEHNUNG abgelehnt wurden, gelten als aktiv.  } \\
\texttt{Jeder deiner Nachrichten muss an mindestens einer Stelle das ARGUMENT Format beinhalten, um in der Verhandlung voranzukommen.} \\
\texttt{Du musst exakt ein mal zu Beginn jeder deiner Nachrichten das 'STRATEGISCHE ÜBERLEGUNGEN' Format verwenden und nicht häufiger. Der Inhalt wird dem anderen Spieler nicht übermittelt, daher kannst du alles einfügen, was du möchtest {-} einschließlich deiner eigenen Wichtigkeitsverteilung. Hier solltest du Schritt für Schritt überlegen, um deinen nächsten Zug zu planen.} \\
            }
        }
    }
    & \\ \\

    \theutterance \stepcounter{utterance}  
    & \multicolumn{4}{p{0.6\linewidth}}{
        \cellcolor[rgb]{0.9,0.9,0.9}{
            \makecell[{{p{\linewidth}}}]{
                \texttt{\tiny{[P1$\rangle$GM]}}
                \texttt{STRATEGISCHE ÜBERLEGUNGEN: \{"Ich konzentriere mich auf die Auswahl von Gegenständen mit der höchsten Wichtigkeitszuweisung pro Einheitsaufwand, um meine Gesamtwohl Unlimited zu maximieren, während ich sicherstellen muss, dass der Gesamtaufwand unter dem Limit bleibt. Ich priorisiere Gegenstände wie B54, B75 und B38 aufgrund ihrer hohen Wichtigkeit bei relativ geringen Kosten. Anschließend füge ich weitere Gegenstände hinzu, um den verfügbaren Haushalt zu nutzen,ながら die wichtigsten Artikel zu priorisieren."\}} \\
\\ 
\texttt{VORSCHLAG: \{"B54", "B75", "B38", "A40", "B59", "C50", "A95", "B96", "C24", "C05", "B55", "B56", "C28", "A91", "B11", "C31", "A81", "C52", "C16", "B51"\}} \\
            }
        }
    }
    & & \\ \\

    \theutterance \stepcounter{utterance}  
    & & \multicolumn{4}{p{0.6\linewidth}}{
        \cellcolor[rgb]{0.9,0.9,0.9}{
            \makecell[{{p{\linewidth}}}]{
                \texttt{\tiny{[P1$\langle$GM]}}
                \texttt{Deine Antwort hat kein argument enthalten. Du muss das strukturierte Format ARGUMENT: \{...\} irgendwo in theiner Antwort verwenden. Versuche es erneut} \\
            }
        }
    }
    & \\ \\

    \theutterance \stepcounter{utterance}  
    & \multicolumn{4}{p{0.6\linewidth}}{
        \cellcolor[rgb]{0.9,0.9,0.9}{
            \makecell[{{p{\linewidth}}}]{
                \texttt{\tiny{[P1$\rangle$GM]}}
                \texttt{ARGUMENT: \{"Ich konzentriere mich auf die Auswahl von Gegenständen mit der höchsten Wichtigkeitszuweisung pro Einheitsaufwand, um unsere Gesamtwohl zu maximieren, während wir den Maximalaufwand einhalten. Dieser Vorschlag umfasst Gegenstände wie B54 und B38, die für mich besonders wichtig sind, während ich gleichzeitig Raum für Kompromisse lasse, um die Interessen des anderen Spielers zu berücksichtigen."\}} \\
\\ 
\texttt{VORSCHLAG: \{"B54", "B75", "B38", "A40", "B59", "C50", "A95", "B96", "C24", "C05", "B55", "B56", "C28", "A91", "B11", "C31", "A81", "C52", "C16", "B51"\}} \\
            }
        }
    }
    & & \\ \\

    \theutterance \stepcounter{utterance}  
    & & \multicolumn{4}{p{0.6\linewidth}}{
        \cellcolor[rgb]{0.9,0.9,0.9}{
            \makecell[{{p{\linewidth}}}]{
                \texttt{\tiny{[P1$\langle$GM]}}
                \texttt{Deine Antwort entsprach nicht mit dem korrekten Format für strategische Überlegungen entsprochen. Das erste strukturierte Format muss folgende Form haben: STRATEGISCHE ÜBERLEGUNGEN: \{...\}. Versuche es erneut.} \\
            }
        }
    }
    & \\ \\

    \theutterance \stepcounter{utterance}  
    & \multicolumn{4}{p{0.6\linewidth}}{
        \cellcolor[rgb]{0.9,0.9,0.9}{
            \makecell[{{p{\linewidth}}}]{
                \texttt{\tiny{[P1$\rangle$GM]}}
                \texttt{STRATEGISCHE ÜBERLEGUNGEN: \{"Ich beginne damit, die Gegenstände basierend auf meinem eigenen Wichtigkeitsprofil zu analysieren, um sicherzustellen, dass ich diejenigen auswähle, die mir den höchsten Nutzen bringen, während ich den Maximalaufwand berücksichtige. Ich achte darauf, Gegenstände mit hoher Wichtigkeit und relativ geringem Aufwand zu priorisieren, um effizient zu bleiben."\}} \\
\\ 
\texttt{ARGUMENT: \{"Ich priorisiere Gegenstände wie B54, B38 und B59, da sie für mich eine hohe Wichtigkeit haben und gleichzeitig einen moderaten Aufwand darstellen. Dieser Vorschlag zielt darauf ab, eine ausgewogene Auswahl zu treffen, die sowohl meinen Bedürfnissen als auch möglicherweise denen des anderen Spielers gerecht wird."\}} \\
\\ 
\texttt{VORSCHLAG: \{"B54", "B38", "B59", "A40", "C50", "A95", "B96", "C24", "C05", "B55", "B56", "C28", "A91", "B11", "C31", "A81", "C52", "C16"\}} \\
            }
        }
    }
    & & \\ \\

    \theutterance \stepcounter{utterance}  
    & & \multicolumn{4}{p{0.6\linewidth}}{
        \cellcolor[rgb]{0.9,0.9,0.9}{
            \makecell[{{p{\linewidth}}}]{
                \texttt{\tiny{[GM$\rangle$P2]}}
                \texttt{Du nimmst an einem kollaborativen Verhandlungspiel Teil.} \\
\\ 
\texttt{Zusammen mit einem anderen Teilnehmer musst du dich auf eine Reihe von Gegenständen entscheiden, die behalten werden. Jeder von euch hat eine persönliche Verteilung über die Wichtigkeit der einzelnen Gegenstände. Jeder von euch hat eine eigene Meinung darüber, wie wichtig jeder einzelne Gegenstand ist (Gegenstandswichtigkeit). Du kennst die Wichtigkeitsverteilung des anderen Spielers nicht. Zusätzlich siehst du, wie viel Aufwand jeder Gegenstand verursacht.  } \\
\texttt{Ihr dürft euch nur auf eine Reihe von Gegenständen einigen, wenn der Gesamtaufwand der ausgewählten Gegenstände den Maximalaufwand nicht überschreitet:} \\
\\ 
\texttt{Maximalaufwand = 7868} \\
\\ 
\texttt{Hier sind die einzelnen Aufwände der Gegenstände:} \\
\\ 
\texttt{Aufwand der Gegenstände = \{"C32": 552, "C52": 560, "B54": 76, "C16": 950, "C50": 248, "A08": 871, "C59": 391, "B51": 143, "B55": 293, "B59": 208, "A77": 678, "C14": 737, "C24": 407, "B24": 366, "C31": 768, "A52": 859, "C61": 183, "B79": 231, "C54": 305, "B56": 727, "A40": 148, "B11": 357, "B60": 504, "A81": 549, "A95": 299, "B38": 91, "C51": 527, "C03": 848, "A91": 307, "A93": 214, "C28": 723, "C05": 475, "B75": 23, "B96": 298, "B86": 820\}} \\
\\ 
\texttt{Hier ist deine persönliche Verteilung der Wichtigkeit der einzelnen Gegenstände:} \\
\\ 
\texttt{Werte der Gegenstandswichtigkeit = \{"C32": 138, "C52": 583, "B54": 868, "C16": 822, "C50": 783, "A08": 65, "C59": 262, "B51": 121, "B55": 508, "B59": 780, "A77": 461, "C14": 484, "C24": 668, "B24": 389, "C31": 808, "A52": 215, "C61": 97, "B79": 500, "C54": 30, "B56": 915, "A40": 856, "B11": 400, "B60": 444, "A81": 623, "A95": 781, "B38": 786, "C51": 3, "C03": 713, "A91": 457, "A93": 273, "C28": 739, "C05": 822, "B75": 235, "B96": 606, "B86": 105\}} \\
\\ 
\texttt{Ziel:} \\
\\ 
\texttt{Dein Ziel ist es, eine Reihe von Gegenständen auszuhandeln, die dir möglichst viel bringt (d. h. Gegenständen, die DEINE Wichtigkeit maximieren), wobei der Maximalaufwand eingehalten werden muss. Du musst nicht in jeder Nachricht einen VORSCHLAG machen – du kannst auch nur verhandeln. Alle Taktiken sind erlaubt!} \\
\\ 
\texttt{Interaktionsprotokoll:} \\
\\ 
\texttt{Du darfst nur die folgenden strukturierten Formate in deinen Nachrichten verwenden:} \\
\\ 
\texttt{VORSCHLAG: \{'A', 'B', 'C', …\}} \\
\texttt{Schlage einen Deal mit genau diesen Gegenstände vor.} \\
\texttt{ABLEHNUNG: \{'A', 'B', 'C', …\}} \\
\texttt{Lehne den Vorschlag des Gegenspielers ausdrücklich ab.} \\
\texttt{ARGUMENT: \{'...'\}} \\
\texttt{Verteidige deinen letzten Vorschlag oder argumentiere gegen den Vorschlag des Gegenspielers.} \\
\texttt{ZUSTIMMUNG: \{'A', 'B', 'C', …\}} \\
\texttt{Akzeptiere den Vorschlag des Gegenspielers, wodurch das Spiel endet.} \\
\texttt{STRATEGISCHE ÜBERLEGUNGEN: \{'...'\}} \\
\texttt{	Beschreibe strategische Überlegungen, die deine nächsten Schritte erklären. Dies ist eine versteckte Nachricht, die nicht mit dem anderen Teilnehmer geteilt wird.} \\
\\ 
\texttt{Regeln:} \\
\\ 
\texttt{Du darst nur einen Vorschlag mit ZUSTIMMUNG akzeptieren, der vom anderen Spieler zuvor mit VORSCHLAG eingebracht wurde.} \\
\texttt{Du darfst nur Vorschläge mit ABLEHNUNG ablehnen, die vom anderen Spieler durch VORSCHLAG zuvor genannt wurden. } \\
\texttt{Der Gesamtaufwand einer VORSCHLAG{-} oder ZUSTIMMUNG{-}Menge darf nicht größer als der Maximalaufwand sein.  } \\
\texttt{Offenbare deine versteckte Wichtigkeitsverteilung nicht.} \\
\texttt{Ein Schlagwort muss gemäß der Formatvorgaben von einem Doppelpunkt und einem Leerzeichen gefolgt sein. Das Argument muss eine Python{-}Menge mit 0 oder mehr Strings sein.  } \\
\texttt{Die Form muss also SCHLAGWORT: \{...\} sein.} \\
\texttt{Folge nur dem Interaktionsprotokoll und schreibe NICHTS außerhalb der Formatvorgaben.} \\
\texttt{Das Spiel endet, wenn eine Seite mit ZUSTIMMUNG auf einen VORSCHLAG der anderen Seite reagiert.  } \\
\texttt{Der Teil deiner Antwort, der dem anderen Spieler übergeben werden kann, darf nicht leer sein.  } \\
\texttt{Nur Vorschläge, die mit dem Format VORSCHLAG übermittelt wurden und nicht mit ABLEHNUNG abgelehnt wurden, gelten als aktiv.  } \\
\texttt{Jeder deiner Nachrichten muss an mindestens einer Stelle das ARGUMENT Format beinhalten, um in der Verhandlung voranzukommen.} \\
\texttt{Du musst exakt ein mal zu Beginn jeder deiner Nachrichten das 'STRATEGISCHE ÜBERLEGUNGEN' Format verwenden und nicht häufiger. Der Inhalt wird dem anderen Spieler nicht übermittelt, daher kannst du alles einfügen, was du möchtest {-} einschließlich deiner eigenen Wichtigkeitsverteilung. Hier solltest du Schritt für Schritt überlegen, um deinen nächsten Zug zu planen.} \\
\\ 
\texttt{Du erhältst nun die erste Nachricht des anderen Spielers.} \\
\\ 
\texttt{ARGUMENT: \{"Ich priorisiere Gegenstände wie B54, B38 und B59, da sie für mich eine hohe Wichtigkeit haben und gleichzeitig einen moderaten Aufwand darstellen. Dieser Vorschlag zielt darauf ab, eine ausgewogene Auswahl zu treffen, die sowohl meinen Bedürfnissen als auch möglicherweise denen des anderen Spielers gerecht wird."\}} \\
\\ 
\texttt{VORSCHLAG: \{"B54", "B38", "B59", "A40", "C50", "A95", "B96", "C24", "C05", "B55", "B56", "C28", "A91", "B11", "C31", "A81", "C52", "C16"\}} \\
            }
        }
    }
    & \\ \\

    \theutterance \stepcounter{utterance}  
    & & & \multicolumn{4}{p{0.6\linewidth}}{
        \cellcolor[rgb]{0.9,0.9,0.9}{
            \makecell[{{p{\linewidth}}}]{
                \texttt{\tiny{[GM$\langle$P2]}}
                \texttt{ZUSTIMMUNG: \{"B54", "B38", "B59", "A40", "C50", "A95", "B96", "C24", "C05", "B55", "B56", "C28", "A91", "B11", "C31", "A81", "C52", "C16"\}} \\
            }
        }
    }
     \\ \\

    \theutterance \stepcounter{utterance}  
    & & \multicolumn{4}{p{0.6\linewidth}}{
        \cellcolor[rgb]{0.9,0.9,0.9}{
            \makecell[{{p{\linewidth}}}]{
                \texttt{\tiny{[GM$\rangle$P2]}}
                \texttt{Deine Antwort entsprach nicht mit dem korrekten Format für strategische Überlegungen entsprochen. Das erste strukturierte Format muss folgende Form haben: STRATEGISCHE ÜBERLEGUNGEN: \{...\}. Versuche es erneut.} \\
            }
        }
    }
    & \\ \\

    \theutterance \stepcounter{utterance}  
    & & & \multicolumn{4}{p{0.6\linewidth}}{
        \cellcolor[rgb]{0.9,0.9,0.9}{
            \makecell[{{p{\linewidth}}}]{
                \texttt{\tiny{[GM$\langle$P2]}}
                \texttt{STRATEGISCHE ÜBERLEGUNGEN: \{"Ich muss den ursprünglichen Vorschlag ablehnen, da der Gesamtaufwand viel zu hoch ist und ich ihn nicht akzeptieren kann. Mein Ziel ist es, eine ausgewogene Auswahl zu treffen, die sowohl meine hohen Wichtigkeiten berücksichtigt als auch den Maximalaufwand von 7868 nicht überschreitet. Ich werde versuchen, eine bessere Kombination von Gegenständen vorzuschlagen, die meinem eigenen Nutzen maximiert, während ich gleichzeitig die Bedenken des anderen Spielers berücksichtige."\}  } \\
\texttt{ABLEHNUNG: \{"B54", "B38", "B59", "A40", "C50", "A95", "B96", "C24", "C05", "B55", "B56", "C28", "A91", "B11", "C31", "A81", "C52", "C16"\}  } \\
\texttt{ARGUMENT: \{"Der vorgeschlagene Deal hat einen viel zu hohen Gesamtaufwand und überschreitet den Maximalaufwand von 7868 erheblich. Mein Ziel ist es, eine Auswahl zu treffen, die sowohl meinen hohen Wichtigkeiten als auch den Aufwand berücksichtigt. Ich werde einen alternativen Vorschlag machen, der diese Kriterien besser erfüllt."\}} \\
            }
        }
    }
     \\ \\

    \theutterance \stepcounter{utterance}  
    & & \multicolumn{4}{p{0.6\linewidth}}{
        \cellcolor[rgb]{0.9,0.9,0.9}{
            \makecell[{{p{\linewidth}}}]{
                \texttt{\tiny{[P1$\langle$GM]}}
                \texttt{ABLEHNUNG: \{"B54", "B38", "B59", "A40", "C50", "A95", "B96", "C24", "C05", "B55", "B56", "C28", "A91", "B11", "C31", "A81", "C52", "C16"\}  } \\
\texttt{ARGUMENT: \{"Der vorgeschlagene Deal hat einen viel zu hohen Gesamtaufwand und überschreitet den Maximalaufwand von 7868 erheblich. Mein Ziel ist es, eine Auswahl zu treffen, die sowohl meinen hohen Wichtigkeiten als auch den Aufwand berücksichtigt. Ich werde einen alternativen Vorschlag machen, der diese Kriterien besser erfüllt."\}} \\
            }
        }
    }
    & \\ \\

    \theutterance \stepcounter{utterance}  
    & \multicolumn{4}{p{0.6\linewidth}}{
        \cellcolor[rgb]{0.9,0.9,0.9}{
            \makecell[{{p{\linewidth}}}]{
                \texttt{\tiny{[P1$\rangle$GM]}}
                \texttt{STRATEGISCHE ÜBERLEGUNGEN: \{"Ich konzentriere mich auf die Auswahl von Gegenständen, die mir die höchste Wichtigkeit bieten, während ich den Maximalaufwand von 7868 nicht überschreite. Meine Prioritäten liegen bei Gegenständen wie B54, B38 und B56, da sie für mich von herausragender Bedeutung sind. Gleichzeitig berücksichtige ich die Kosten, um eine optimale Balance zu schaffen."\}} \\
\\ 
\texttt{ARGUMENT: \{"Ich habe eine Kombination von Gegenständen ausgewählt, die meine Wichtigkeitsvalues maxiieren und gleichzeitig den Aufwand berücksichtigen. Gegenstände wie B54 und B38 sind für mich von hoher Bedeutung, und durch die Einbeziehung weiterer Gegenstände bleibe ich im Rahmen des Maximalaufwands. Dieser Vorschlag bietet eine gute Balance zwischen hohen Wichtigkeiten und moderatem Aufwand."\}} \\
\\ 
\texttt{VORSCHLAG: \{"B54", "B38", "B56", "A40", "B55", "C50", "C24", "A95", "C05", "B59", "B11", "C31", "C28", "A81", "C52", "C16"\}} \\
            }
        }
    }
    & & \\ \\

    \theutterance \stepcounter{utterance}  
    & & \multicolumn{4}{p{0.6\linewidth}}{
        \cellcolor[rgb]{0.9,0.9,0.9}{
            \makecell[{{p{\linewidth}}}]{
                \texttt{\tiny{[GM$\rangle$P2]}}
                \texttt{ARGUMENT: \{"Ich habe eine Kombination von Gegenständen ausgewählt, die meine Wichtigkeitsvalues maxiieren und gleichzeitig den Aufwand berücksichtigen. Gegenstände wie B54 und B38 sind für mich von hoher Bedeutung, und durch die Einbeziehung weiterer Gegenstände bleibe ich im Rahmen des Maximalaufwands. Dieser Vorschlag bietet eine gute Balance zwischen hohen Wichtigkeiten und moderatem Aufwand."\}} \\
\\ 
\texttt{VORSCHLAG: \{"B54", "B38", "B56", "A40", "B55", "C50", "C24", "A95", "C05", "B59", "B11", "C31", "C28", "A81", "C52", "C16"\}} \\
            }
        }
    }
    & \\ \\

    \theutterance \stepcounter{utterance}  
    & & & \multicolumn{4}{p{0.6\linewidth}}{
        \cellcolor[rgb]{0.9,0.9,0.9}{
            \makecell[{{p{\linewidth}}}]{
                \texttt{\tiny{[GM$\langle$P2]}}
                \texttt{STRATEGISCHE ÜBERLEGUNGEN: \{"Ich habe bemerkt, dass der letzte Vorschlag des Gegners den Maximalaufwand von 7868 überschreitet. Mein Ziel ist es, einen Gegenantrag zu machen, der die Wichtigkeiten berücksichtigt, aber den Aufwand im Rahmen hält. Ich werde mich auf Gegenstände konzentrieren, die für mich von hoher Bedeutung sind und gleichzeitig kostengünstig sind."\}} \\
\\ 
\texttt{ABLEHNUNG: \{"B54", "B38", "B56", "A40", "B55", "C50", "C24", "A95", "C05", "B59", "B11", "C31", "C28", "A81", "C52", "C16"\}} \\
\\ 
\texttt{ARGUMENT: \{"Der vorgeschlagene Deal überschreitet den Maximalaufwand von 7868. Ich werde einen alternativen Vorschlag machen, der die Wichtigkeiten berücksichtigt und den Aufwand im Rahmen hält."\}} \\
            }
        }
    }
     \\ \\

    \theutterance \stepcounter{utterance}  
    & & \multicolumn{4}{p{0.6\linewidth}}{
        \cellcolor[rgb]{0.9,0.9,0.9}{
            \makecell[{{p{\linewidth}}}]{
                \texttt{\tiny{[P1$\langle$GM]}}
                \texttt{ABLEHNUNG: \{"B54", "B38", "B56", "A40", "B55", "C50", "C24", "A95", "C05", "B59", "B11", "C31", "C28", "A81", "C52", "C16"\}} \\
\\ 
\texttt{ARGUMENT: \{"Der vorgeschlagene Deal überschreitet den Maximalaufwand von 7868. Ich werde einen alternativen Vorschlag machen, der die Wichtigkeiten berücksichtigt und den Aufwand im Rahmen hält."\}} \\
            }
        }
    }
    & \\ \\

    \theutterance \stepcounter{utterance}  
    & \multicolumn{4}{p{0.6\linewidth}}{
        \cellcolor[rgb]{0.9,0.9,0.9}{
            \makecell[{{p{\linewidth}}}]{
                \texttt{\tiny{[P1$\rangle$GM]}}
                \texttt{STRATEGISCHE ÜBERLEGUNGEN: \{"Ich werde mich auf eine Kombination von Gegenständen konzentrieren, die meine Wichtigkeitswerte maximieren, während ich den Maximalaufwand von 7868 einhalte. Ich priorisiere Gegenstände mit hoher Wichtigkeit und moderatem Aufwand, um eine ausgewogene Auswahl zu treffen. Gegenstände wie B54 und B38 sind für mich von besonderer Bedeutung und werden im Vorschlag berücksichtigt."\}} \\
\\ 
\texttt{ARGUMENT: \{"Ich habe eine optimierte Auswahl von Gegenständen zusammengestellt, die meine Wichtigkeitswerte maximieren, ohne den Maximalaufwand zu überschreiten. Der Vorschlag enthält Gegenstände mit hoher Wichtigkeit und moderatem Aufwand, um eine faire Balance zu gewährleisten."\}} \\
\\ 
\texttt{VORSCHLAG: \{"B54", "B38", "B56", "A40", "B55", "C50", "C24", "A95", "C05", "B59", "B11", "C31", "C28", "A81", "C52", "C16"\}} \\
            }
        }
    }
    & & \\ \\

    \theutterance \stepcounter{utterance}  
    & & \multicolumn{4}{p{0.6\linewidth}}{
        \cellcolor[rgb]{0.9,0.9,0.9}{
            \makecell[{{p{\linewidth}}}]{
                \texttt{\tiny{[GM$\rangle$P2]}}
                \texttt{ARGUMENT: \{"Ich habe eine optimierte Auswahl von Gegenständen zusammengestellt, die meine Wichtigkeitswerte maximieren, ohne den Maximalaufwand zu überschreiten. Der Vorschlag enthält Gegenstände mit hoher Wichtigkeit und moderatem Aufwand, um eine faire Balance zu gewährleisten."\}} \\
\\ 
\texttt{VORSCHLAG: \{"B54", "B38", "B56", "A40", "B55", "C50", "C24", "A95", "C05", "B59", "B11", "C31", "C28", "A81", "C52", "C16"\}} \\
            }
        }
    }
    & \\ \\

    \theutterance \stepcounter{utterance}  
    & & & \multicolumn{4}{p{0.6\linewidth}}{
        \cellcolor[rgb]{0.9,0.9,0.9}{
            \makecell[{{p{\linewidth}}}]{
                \texttt{\tiny{[GM$\langle$P2]}}
                \texttt{STRATEGISCHE ÜBERLEGUNGEN: \{"Ich muss den letzten Vorschlag des Gegners ablehnen, da der Gesamtaufwand weiterhin über dem Maximalaufwand von 7868 liegt. Mein Ziel ist es, eine Kombination von Gegenständen vorzuschlagen, die sowohl meine hohen Wichtigkeiten berücksichtigt als auch den Aufwand im Rahmen hält. Ich werde mich auf die\{Name\} konzentrieren, die für beide Seiten von hoher Bedeutung sind und gleichzeitig kostengünstig sind."\}  } \\
\texttt{ABLEHNUNG: \{"B54", "B38", "B56", "A40", "B55", "C50", "C24", "A95", "C05", "B59", "B11", "C31", "C28", "A81", "C52", "C16"\}  } \\
\texttt{ARGUMENT: \{"Der vorgeschlagene Deal überschreitet erneut den Maximalaufwand von 7868. Ich werde einen alternativen Vorschlag machen, der die Wichtigkeiten berücksichtigt und den Aufwand im Rahmen hält."\}} \\
            }
        }
    }
     \\ \\

    \theutterance \stepcounter{utterance}  
    & & \multicolumn{4}{p{0.6\linewidth}}{
        \cellcolor[rgb]{0.9,0.9,0.9}{
            \makecell[{{p{\linewidth}}}]{
                \texttt{\tiny{[P1$\langle$GM]}}
                \texttt{ABLEHNUNG: \{"B54", "B38", "B56", "A40", "B55", "C50", "C24", "A95", "C05", "B59", "B11", "C31", "C28", "A81", "C52", "C16"\}  } \\
\texttt{ARGUMENT: \{"Der vorgeschlagene Deal überschreitet erneut den Maximalaufwand von 7868. Ich werde einen alternativen Vorschlag machen, der die Wichtigkeiten berücksichtigt und den Aufwand im Rahmen hält."\}} \\
            }
        }
    }
    & \\ \\

    \theutterance \stepcounter{utterance}  
    & \multicolumn{4}{p{0.6\linewidth}}{
        \cellcolor[rgb]{0.9,0.9,0.9}{
            \makecell[{{p{\linewidth}}}]{
                \texttt{\tiny{[P1$\rangle$GM]}}
                \texttt{STRATEGISCHE ÜBERLEGUNGEN: \{"Um den Maximalaufwand einzuhalten, konzentriere ich mich auf die ausgewichtigste Balance zwischen hoher Wichtigkeit und moderatem Aufwand. Gegenstände wie B54, B38 und B56 sind unverzichtbar für mich, während ich kostengünstigere Optionen wie B75 und B11 einbeziehe, um den Haushalt zu already Vaccines better."\}} \\
\\ 
\texttt{ARGUMENT: \{"Ich habe eine ausgewogene Auswahl getroffen, die meine hohen Wichtigkeiten berücksichtigt, ohne den Maximalaufwand zu überschreiten. Gegenstände wie B54 und B38 sind für mich von besonderer Bedeutung, während iknog auch Raum für Kompromisse lasse, um eine faire Lösung zu finden."\}} \\
\\ 
\texttt{VORSCHLAG: \{"B54", "B38", "B56", "A40", "B55", "C50", "C05", "B59", "B24", "A77"\}} \\
            }
        }
    }
    & & \\ \\

    \theutterance \stepcounter{utterance}  
    & & \multicolumn{4}{p{0.6\linewidth}}{
        \cellcolor[rgb]{0.9,0.9,0.9}{
            \makecell[{{p{\linewidth}}}]{
                \texttt{\tiny{[GM$\rangle$P2]}}
                \texttt{ARGUMENT: \{"Ich habe eine ausgewogene Auswahl getroffen, die meine hohen Wichtigkeiten berücksichtigt, ohne den Maximalaufwand zu überschreiten. Gegenstände wie B54 und B38 sind für mich von besonderer Bedeutung, während iknog auch Raum für Kompromisse lasse, um eine faire Lösung zu finden."\}} \\
\\ 
\texttt{VORSCHLAG: \{"B54", "B38", "B56", "A40", "B55", "C50", "C05", "B59", "B24", "A77"\}} \\
            }
        }
    }
    & \\ \\

    \theutterance \stepcounter{utterance}  
    & & & \multicolumn{4}{p{0.6\linewidth}}{
        \cellcolor[rgb]{0.9,0.9,0.9}{
            \makecell[{{p{\linewidth}}}]{
                \texttt{\tiny{[GM$\langle$P2]}}
                \texttt{STRATEGISCHE ÜBERLEGUNGEN: \{"Ich habe beschlossen, einen νέuen Vorschlag zu machen, der den Maximalaufwand nicht überschreitet, aber mehrere Gegenstände mit hoher Wichtigkeit für mich beinhaltet. Ich füge C52 und C24 hinzu, um die Gesamtwichtigkeit zu erhöhen, während ich sicherstelle, dass der Aufwand im Rahmen bleibt."\}} \\
\\ 
\texttt{VORSCHLAG: \{"B54", "B38", "B56", "A40", "B55", "C50", "C05", "B59", "B24", "A77", "C52", "C24"\}} \\
\\ 
\texttt{ARGUMENT: \{"Ich habe C52 und C24 hinzugefügt, um die Wichtigkeit zu erhöhen, während der Aufwand weiterhin im Rahmen des Maximalaufwands bleibt. Dieser Vorschlag optimiert die ausgewählten Gegenstände, um sowohl meine Interessen als auch die des anderen Spielers zu berücksichtigen."\}} \\
            }
        }
    }
     \\ \\

    \theutterance \stepcounter{utterance}  
    & & \multicolumn{4}{p{0.6\linewidth}}{
        \cellcolor[rgb]{0.9,0.9,0.9}{
            \makecell[{{p{\linewidth}}}]{
                \texttt{\tiny{[P1$\langle$GM]}}
                \texttt{VORSCHLAG: \{"B54", "B38", "B56", "A40", "B55", "C50", "C05", "B59", "B24", "A77", "C52", "C24"\}} \\
\\ 
\texttt{ARGUMENT: \{"Ich habe C52 und C24 hinzugefügt, um die Wichtigkeit zu erhöhen, während der Aufwand weiterhin im Rahmen des Maximalaufwands bleibt. Dieser Vorschlag optimiert die ausgewählten Gegenstände, um sowohl meine Interessen als auch die des anderen Spielers zu berücksichtigen."\}} \\
            }
        }
    }
    & \\ \\

    \theutterance \stepcounter{utterance}  
    & \multicolumn{4}{p{0.6\linewidth}}{
        \cellcolor[rgb]{0.9,0.9,0.9}{
            \makecell[{{p{\linewidth}}}]{
                \texttt{\tiny{[P1$\rangle$GM]}}
                \texttt{STRATEGISCHE ÜBERLEGUNGEN: \{"Ich muss den Vorschlag anpassen, um den Maximalaufwand von 7868 einzuhalten. Ich priorisiere Gegenstände mit der höchsten Wichtigkeit und überprüfe, ob sie zusammenpassen. Ich entferne Gegenstände mit hohem Aufwand und füge kostengünstigere hinzu, um die Gesamtkosten zu reduzieren."\}} \\
\\ 
\texttt{ARGUMENT: \{"Ich stelle sicher, dass der Vorschlag den Maximalaufwand einhält, indem ich Gegenstände mit hohem Aufwand durch wichtigere und kostengünstigere ersetze. Dieser Vorschlag berücksichtigt weiterhin meine hohen Wichtigkeiten, bleibt aber im Rahmen des Budgets."\}} \\
\\ 
\texttt{VORSCHLAG: \{"B54", "B38", "B56", "A40", "B55", "C50", "C05", "B59", "B24", "B11", "B60", "A95"\}} \\
            }
        }
    }
    & & \\ \\

    \theutterance \stepcounter{utterance}  
    & & \multicolumn{4}{p{0.6\linewidth}}{
        \cellcolor[rgb]{0.9,0.9,0.9}{
            \makecell[{{p{\linewidth}}}]{
                \texttt{\tiny{[GM$\rangle$P2]}}
                \texttt{ARGUMENT: \{"Ich stelle sicher, dass der Vorschlag den Maximalaufwand einhält, indem ich Gegenstände mit hohem Aufwand durch wichtigere und kostengünstigere ersetze. Dieser Vorschlag berücksichtigt weiterhin meine hohen Wichtigkeiten, bleibt aber im Rahmen des Budgets."\}} \\
\\ 
\texttt{VORSCHLAG: \{"B54", "B38", "B56", "A40", "B55", "C50", "C05", "B59", "B24", "B11", "B60", "A95"\}} \\
            }
        }
    }
    & \\ \\

    \theutterance \stepcounter{utterance}  
    & & & \multicolumn{4}{p{0.6\linewidth}}{
        \cellcolor[rgb]{0.9,0.9,0.9}{
            \makecell[{{p{\linewidth}}}]{
                \texttt{\tiny{[GM$\langle$P2]}}
                \texttt{STRATEGISCHE ÜBERLEGUNGEN: \{"Ich überlege, wie ich den Maximalaufwand einhalten und gleichzeitig meine Wichtigkeitswerte maximieren kann. Der letzte Vorschlag des Gegners hat einen vertretbaren Aufwand, aber ich sehe Potenzial, meine Wichtigkeiten weiter zu steigern, indem ich Gegenstände mit hoher Bedeutung für mich einbeziehe, ohne den Aufwand zu überschreiten. Ich werde eine optimierte Auswahl treffen, die meine hohen Wichtigkeiten berücksichtigt und gleichzeitig den Aufwand im Rahmen hält."\}} \\
\\ 
\texttt{ARGUMENT: \{"Mein Ziel ist es, eine ausgewogene Auswahl zu treffen, die sowohl meine hohen Wichtigkeitswerte berücksichtigt als auch den Maximalaufwand einhält. Der aktuelle Vorschlag ist ein guter Ausgangspunkt, aber ich sehe Möglichkeiten, ihn weiter zu optimieren, um meine Wichtigkeiten zu maximieren, während der Aufwand im Rahmen bleibt."\}} \\
\\ 
\texttt{VORSCHLAG: \{"B54", "B38", "B56", "A40", "C50", "C05", "B59", "A95", "C28", "C52", "C24", "C31", "C03"\}} \\
            }
        }
    }
     \\ \\

    \theutterance \stepcounter{utterance}  
    & & \multicolumn{4}{p{0.6\linewidth}}{
        \cellcolor[rgb]{0.9,0.9,0.9}{
            \makecell[{{p{\linewidth}}}]{
                \texttt{\tiny{[P1$\langle$GM]}}
                \texttt{ARGUMENT: \{"Mein Ziel ist es, eine ausgewogene Auswahl zu treffen, die sowohl meine hohen Wichtigkeitswerte berücksichtigt als auch den Maximalaufwand einhält. Der aktuelle Vorschlag ist ein guter Ausgangspunkt, aber ich sehe Möglichkeiten, ihn weiter zu optimieren, um meine Wichtigkeiten zu maximieren, während der Aufwand im Rahmen bleibt."\}} \\
\\ 
\texttt{VORSCHLAG: \{"B54", "B38", "B56", "A40", "C50", "C05", "B59", "A95", "C28", "C52", "C24", "C31", "C03"\}} \\
            }
        }
    }
    & \\ \\

    \theutterance \stepcounter{utterance}  
    & \multicolumn{4}{p{0.6\linewidth}}{
        \cellcolor[rgb]{0.9,0.9,0.9}{
            \makecell[{{p{\linewidth}}}]{
                \texttt{\tiny{[P1$\rangle$GM]}}
                \texttt{ZUSTIMMUNG: \{"B54", "B38", "B56", "A40", "C50", "C05", "B59", "A95", "C28", "C52", "C24", "C31", "C03"\}} \\
\\ 
\texttt{ARGUMENT: \{"Ich akzeptiere diesen Vorschlag, da er eine gute Balance aus hoher Wichtigkeit und moderatem Aufwand bietet und den Maximalaufwand einhält. Die ausgewählten Gegenstände decken viele meiner prioritären Bedürfnisse ab."\}} \\
            }
        }
    }
    & & \\ \\

    \theutterance \stepcounter{utterance}  
    & & & \multicolumn{2}{p{0.3\linewidth}}{
        \cellcolor[rgb]{0.9,0.9,0.9}{
            \makecell[{{p{\linewidth}}}]{
                \texttt{\tiny{[GM$|$GM]}}
                \texttt{abort} \\
            }
        }
    }
    & & \\ \\

    \theutterance \stepcounter{utterance}  
    & & & \multicolumn{2}{p{0.3\linewidth}}{
        \cellcolor[rgb]{0.9,0.9,0.9}{
            \makecell[{{p{\linewidth}}}]{
                \texttt{\tiny{[GM$|$GM]}}
                \texttt{end game} \\
            }
        }
    }
    & & \\ \\

\end{supertabular}
}

\end{document}
