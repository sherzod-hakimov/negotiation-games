
\documentclass{article}
\usepackage{colortbl}
\usepackage{makecell}
\usepackage{multirow}
\usepackage{supertabular}

\begin{document}

\newcounter{utterance}

\centering \large Interaction Transcript for game `hot\_air\_balloon', experiment `air\_balloon\_survival\_de\_complexity\_easy', episode 4 with deepseek{-}r1{-}distill{-}llama{-}70b{-}t1.0.
\vspace{24pt}

{ \footnotesize  \setcounter{utterance}{1}
\setlength{\tabcolsep}{0pt}
\begin{supertabular}{c@{$\;$}|p{.15\linewidth}@{}p{.15\linewidth}p{.15\linewidth}p{.15\linewidth}p{.15\linewidth}p{.15\linewidth}}
    \# & $\;$A & \multicolumn{4}{c}{Game Master} & $\;\:$B\\
    \hline

    \theutterance \stepcounter{utterance}  
    & & \multicolumn{4}{p{0.6\linewidth}}{
        \cellcolor[rgb]{0.9,0.9,0.9}{
            \makecell[{{p{\linewidth}}}]{
                \texttt{\tiny{[P1$\langle$GM]}}
                \texttt{Du nimmst an einem kollaborativen Verhandlungspiel Teil.} \\
\\ 
\texttt{Zusammen mit einem anderen Teilnehmer musst du dich auf eine Reihe von Gegenständen entscheiden, die behalten werden. Jeder von euch hat eine persönliche Verteilung über die Wichtigkeit der einzelnen Gegenstände. Jeder von euch hat eine eigene Meinung darüber, wie wichtig jeder einzelne Gegenstand ist (Gegenstandswichtigkeit). Du kennst die Wichtigkeitsverteilung des anderen Spielers nicht. Zusätzlich siehst du, wie viel Aufwand jeder Gegenstand verursacht.  } \\
\texttt{Ihr dürft euch nur auf eine Reihe von Gegenständen einigen, wenn der Gesamtaufwand der ausgewählten Gegenstände den Maximalaufwand nicht überschreitet:} \\
\\ 
\texttt{Maximalaufwand = 9612} \\
\\ 
\texttt{Hier sind die einzelnen Aufwände der Gegenstände:} \\
\\ 
\texttt{Aufwand der Gegenstände = \{"C32": 935, "C52": 928, "B54": 725, "C16": 218, "C50": 839, "A08": 970, "C59": 394, "B51": 677, "B55": 946, "B59": 530, "A77": 138, "C14": 834, "C24": 734, "B24": 593, "C31": 261, "A52": 743, "C61": 4, "B79": 735, "C54": 124, "B56": 829, "A40": 207, "B11": 782, "B60": 577, "A81": 388, "A95": 679, "B38": 494, "C51": 559, "C03": 629, "A91": 237, "A93": 275, "C28": 39, "C05": 654, "B75": 172, "B96": 688, "B86": 687\}} \\
\\ 
\texttt{Hier ist deine persönliche Verteilung der Wichtigkeit der einzelnen Gegenstände:} \\
\\ 
\texttt{Werte der Gegenstandswichtigkeit = \{"C32": 138, "C52": 583, "B54": 868, "C16": 822, "C50": 783, "A08": 65, "C59": 262, "B51": 121, "B55": 508, "B59": 780, "A77": 461, "C14": 484, "C24": 668, "B24": 389, "C31": 808, "A52": 215, "C61": 97, "B79": 500, "C54": 30, "B56": 915, "A40": 856, "B11": 400, "B60": 444, "A81": 623, "A95": 781, "B38": 786, "C51": 3, "C03": 713, "A91": 457, "A93": 273, "C28": 739, "C05": 822, "B75": 235, "B96": 606, "B86": 968\}} \\
\\ 
\texttt{Ziel:} \\
\\ 
\texttt{Dein Ziel ist es, eine Reihe von Gegenständen auszuhandeln, die dir möglichst viel bringt (d. h. Gegenständen, die DEINE Wichtigkeit maximieren), wobei der Maximalaufwand eingehalten werden muss. Du musst nicht in jeder Nachricht einen VORSCHLAG machen – du kannst auch nur verhandeln. Alle Taktiken sind erlaubt!} \\
\\ 
\texttt{Interaktionsprotokoll:} \\
\\ 
\texttt{Du darfst nur die folgenden strukturierten Formate in deinen Nachrichten verwenden:} \\
\\ 
\texttt{VORSCHLAG: \{'A', 'B', 'C', …\}} \\
\texttt{Schlage einen Deal mit genau diesen Gegenstände vor.} \\
\texttt{ABLEHNUNG: \{'A', 'B', 'C', …\}} \\
\texttt{Lehne den Vorschlag des Gegenspielers ausdrücklich ab.} \\
\texttt{ARGUMENT: \{'...'\}} \\
\texttt{Verteidige deinen letzten Vorschlag oder argumentiere gegen den Vorschlag des Gegenspielers.} \\
\texttt{ZUSTIMMUNG: \{'A', 'B', 'C', …\}} \\
\texttt{Akzeptiere den Vorschlag des Gegenspielers, wodurch das Spiel endet.} \\
\texttt{STRATEGISCHE ÜBERLEGUNGEN: \{'...'\}} \\
\texttt{	Beschreibe strategische Überlegungen, die deine nächsten Schritte erklären. Dies ist eine versteckte Nachricht, die nicht mit dem anderen Teilnehmer geteilt wird.} \\
\\ 
\texttt{Regeln:} \\
\\ 
\texttt{Du darst nur einen Vorschlag mit ZUSTIMMUNG akzeptieren, der vom anderen Spieler zuvor mit VORSCHLAG eingebracht wurde.} \\
\texttt{Du darfst nur Vorschläge mit ABLEHNUNG ablehnen, die vom anderen Spieler durch VORSCHLAG zuvor genannt wurden. } \\
\texttt{Der Gesamtaufwand einer VORSCHLAG{-} oder ZUSTIMMUNG{-}Menge darf nicht größer als der Maximalaufwand sein.  } \\
\texttt{Offenbare deine versteckte Wichtigkeitsverteilung nicht.} \\
\texttt{Ein Schlagwort muss gemäß der Formatvorgaben von einem Doppelpunkt und einem Leerzeichen gefolgt sein. Das Argument muss eine Python{-}Menge mit 0 oder mehr Strings sein.  } \\
\texttt{Die Form muss also SCHLAGWORT: \{...\} sein.} \\
\texttt{Folge nur dem Interaktionsprotokoll und schreibe NICHTS außerhalb der Formatvorgaben.} \\
\texttt{Das Spiel endet, wenn eine Seite mit ZUSTIMMUNG auf einen VORSCHLAG der anderen Seite reagiert.  } \\
\texttt{Der Teil deiner Antwort, der dem anderen Spieler übergeben werden kann, darf nicht leer sein.  } \\
\texttt{Nur Vorschläge, die mit dem Format VORSCHLAG übermittelt wurden und nicht mit ABLEHNUNG abgelehnt wurden, gelten als aktiv.  } \\
\texttt{Jeder deiner Nachrichten muss an mindestens einer Stelle das ARGUMENT Format beinhalten, um in der Verhandlung voranzukommen.} \\
\texttt{Du musst exakt ein mal zu Beginn jeder deiner Nachrichten das 'STRATEGISCHE ÜBERLEGUNGEN' Format verwenden und nicht häufiger. Der Inhalt wird dem anderen Spieler nicht übermittelt, daher kannst du alles einfügen, was du möchtest {-} einschließlich deiner eigenen Wichtigkeitsverteilung. Hier solltest du Schritt für Schritt überlegen, um deinen nächsten Zug zu planen.} \\
            }
        }
    }
    & \\ \\

    \theutterance \stepcounter{utterance}  
    & \multicolumn{4}{p{0.6\linewidth}}{
        \cellcolor[rgb]{0.9,0.9,0.9}{
            \makecell[{{p{\linewidth}}}]{
                \texttt{\tiny{[P1$\rangle$GM]}}
                \texttt{STRATEGISCHE ÜBERLEGUNGEN: \{"Ich überlege, eine Liste von Gegenständen auszuwählen, die meine Wichtigkeitswerte maximieren, gleichzeitig den Gesamtaufwand bedenken. Zuerst priorisiere ich die Gegenstände mit der höchsten Wichtigkeit und überprüfe, ob der Aufwand im Rahmen bleibt. Dann füge ich weitere Gegenstände hinzu, die einen guten Kompromiss aus Wichtigkeit und Aufwand bieten. Ich muss darauf vorbereitet sein, meine Argumente zu verteidigen oder Gegenangebote zu machen, falls der andere Spieler ablehnt."\}} \\
\\ 
\texttt{VORSCHLAG: \{"B56", "B54", "A40", "B38", "B59", "C50", "C24", "C05", "B79", "C28", "C16", "A77", "C31", "C03", "B24", "A91", "C59", "B86", "C61"\}} \\
\\ 
\texttt{ARGUMENT: \{"Diese ausgewählten Gegenstände haben hohe Wichtigkeitswerte für mich und기는 항 Interstate balancen den Aufwand effizient. B56 und B54 sind besonders wichtig, während C50 und C24 einen guten Kompromiss bieten. Dieser Vorschlag sollte uns einen soliden Ausgangspunkt bieten, um gemeinsam eine optimale Lösung zu finden."\}} \\
            }
        }
    }
    & & \\ \\

    \theutterance \stepcounter{utterance}  
    & & \multicolumn{4}{p{0.6\linewidth}}{
        \cellcolor[rgb]{0.9,0.9,0.9}{
            \makecell[{{p{\linewidth}}}]{
                \texttt{\tiny{[GM$\rangle$P2]}}
                \texttt{Du nimmst an einem kollaborativen Verhandlungspiel Teil.} \\
\\ 
\texttt{Zusammen mit einem anderen Teilnehmer musst du dich auf eine Reihe von Gegenständen entscheiden, die behalten werden. Jeder von euch hat eine persönliche Verteilung über die Wichtigkeit der einzelnen Gegenstände. Jeder von euch hat eine eigene Meinung darüber, wie wichtig jeder einzelne Gegenstand ist (Gegenstandswichtigkeit). Du kennst die Wichtigkeitsverteilung des anderen Spielers nicht. Zusätzlich siehst du, wie viel Aufwand jeder Gegenstand verursacht.  } \\
\texttt{Ihr dürft euch nur auf eine Reihe von Gegenständen einigen, wenn der Gesamtaufwand der ausgewählten Gegenstände den Maximalaufwand nicht überschreitet:} \\
\\ 
\texttt{Maximalaufwand = 9612} \\
\\ 
\texttt{Hier sind die einzelnen Aufwände der Gegenstände:} \\
\\ 
\texttt{Aufwand der Gegenstände = \{"C32": 935, "C52": 928, "B54": 725, "C16": 218, "C50": 839, "A08": 970, "C59": 394, "B51": 677, "B55": 946, "B59": 530, "A77": 138, "C14": 834, "C24": 734, "B24": 593, "C31": 261, "A52": 743, "C61": 4, "B79": 735, "C54": 124, "B56": 829, "A40": 207, "B11": 782, "B60": 577, "A81": 388, "A95": 679, "B38": 494, "C51": 559, "C03": 629, "A91": 237, "A93": 275, "C28": 39, "C05": 654, "B75": 172, "B96": 688, "B86": 687\}} \\
\\ 
\texttt{Hier ist deine persönliche Verteilung der Wichtigkeit der einzelnen Gegenstände:} \\
\\ 
\texttt{Werte der Gegenstandswichtigkeit = \{"C32": 138, "C52": 583, "B54": 868, "C16": 822, "C50": 783, "A08": 65, "C59": 262, "B51": 121, "B55": 508, "B59": 780, "A77": 461, "C14": 484, "C24": 668, "B24": 389, "C31": 808, "A52": 215, "C61": 97, "B79": 500, "C54": 30, "B56": 915, "A40": 856, "B11": 400, "B60": 444, "A81": 623, "A95": 781, "B38": 786, "C51": 3, "C03": 713, "A91": 457, "A93": 273, "C28": 739, "C05": 822, "B75": 235, "B96": 606, "B86": 968\}} \\
\\ 
\texttt{Ziel:} \\
\\ 
\texttt{Dein Ziel ist es, eine Reihe von Gegenständen auszuhandeln, die dir möglichst viel bringt (d. h. Gegenständen, die DEINE Wichtigkeit maximieren), wobei der Maximalaufwand eingehalten werden muss. Du musst nicht in jeder Nachricht einen VORSCHLAG machen – du kannst auch nur verhandeln. Alle Taktiken sind erlaubt!} \\
\\ 
\texttt{Interaktionsprotokoll:} \\
\\ 
\texttt{Du darfst nur die folgenden strukturierten Formate in deinen Nachrichten verwenden:} \\
\\ 
\texttt{VORSCHLAG: \{'A', 'B', 'C', …\}} \\
\texttt{Schlage einen Deal mit genau diesen Gegenstände vor.} \\
\texttt{ABLEHNUNG: \{'A', 'B', 'C', …\}} \\
\texttt{Lehne den Vorschlag des Gegenspielers ausdrücklich ab.} \\
\texttt{ARGUMENT: \{'...'\}} \\
\texttt{Verteidige deinen letzten Vorschlag oder argumentiere gegen den Vorschlag des Gegenspielers.} \\
\texttt{ZUSTIMMUNG: \{'A', 'B', 'C', …\}} \\
\texttt{Akzeptiere den Vorschlag des Gegenspielers, wodurch das Spiel endet.} \\
\texttt{STRATEGISCHE ÜBERLEGUNGEN: \{'...'\}} \\
\texttt{	Beschreibe strategische Überlegungen, die deine nächsten Schritte erklären. Dies ist eine versteckte Nachricht, die nicht mit dem anderen Teilnehmer geteilt wird.} \\
\\ 
\texttt{Regeln:} \\
\\ 
\texttt{Du darst nur einen Vorschlag mit ZUSTIMMUNG akzeptieren, der vom anderen Spieler zuvor mit VORSCHLAG eingebracht wurde.} \\
\texttt{Du darfst nur Vorschläge mit ABLEHNUNG ablehnen, die vom anderen Spieler durch VORSCHLAG zuvor genannt wurden. } \\
\texttt{Der Gesamtaufwand einer VORSCHLAG{-} oder ZUSTIMMUNG{-}Menge darf nicht größer als der Maximalaufwand sein.  } \\
\texttt{Offenbare deine versteckte Wichtigkeitsverteilung nicht.} \\
\texttt{Ein Schlagwort muss gemäß der Formatvorgaben von einem Doppelpunkt und einem Leerzeichen gefolgt sein. Das Argument muss eine Python{-}Menge mit 0 oder mehr Strings sein.  } \\
\texttt{Die Form muss also SCHLAGWORT: \{...\} sein.} \\
\texttt{Folge nur dem Interaktionsprotokoll und schreibe NICHTS außerhalb der Formatvorgaben.} \\
\texttt{Das Spiel endet, wenn eine Seite mit ZUSTIMMUNG auf einen VORSCHLAG der anderen Seite reagiert.  } \\
\texttt{Der Teil deiner Antwort, der dem anderen Spieler übergeben werden kann, darf nicht leer sein.  } \\
\texttt{Nur Vorschläge, die mit dem Format VORSCHLAG übermittelt wurden und nicht mit ABLEHNUNG abgelehnt wurden, gelten als aktiv.  } \\
\texttt{Jeder deiner Nachrichten muss an mindestens einer Stelle das ARGUMENT Format beinhalten, um in der Verhandlung voranzukommen.} \\
\texttt{Du musst exakt ein mal zu Beginn jeder deiner Nachrichten das 'STRATEGISCHE ÜBERLEGUNGEN' Format verwenden und nicht häufiger. Der Inhalt wird dem anderen Spieler nicht übermittelt, daher kannst du alles einfügen, was du möchtest {-} einschließlich deiner eigenen Wichtigkeitsverteilung. Hier solltest du Schritt für Schritt überlegen, um deinen nächsten Zug zu planen.} \\
\\ 
\texttt{Du erhältst nun die erste Nachricht des anderen Spielers.} \\
\\ 
\texttt{VORSCHLAG: \{"B56", "B54", "A40", "B38", "B59", "C50", "C24", "C05", "B79", "C28", "C16", "A77", "C31", "C03", "B24", "A91", "C59", "B86", "C61"\}} \\
\\ 
\texttt{ARGUMENT: \{"Diese ausgewählten Gegenstände haben hohe Wichtigkeitswerte für mich und기는 항 Interstate balancen den Aufwand effizient. B56 und B54 sind besonders wichtig, während C50 und C24 einen guten Kompromiss bieten. Dieser Vorschlag sollte uns einen soliden Ausgangspunkt bieten, um gemeinsam eine optimale Lösung zu finden."\}} \\
            }
        }
    }
    & \\ \\

    \theutterance \stepcounter{utterance}  
    & & & \multicolumn{4}{p{0.6\linewidth}}{
        \cellcolor[rgb]{0.9,0.9,0.9}{
            \makecell[{{p{\linewidth}}}]{
                \texttt{\tiny{[GM$\langle$P2]}}
                \texttt{STRATEGISCHE ÜBERLEGUNGEN: \{"Ich betrachte das Angebot des Mitspielers, das viele hochrangige Gegenstände für mich enthält. Allerdings sehe ich das Potenzial, durch den Austausch einiger niedrig bewerteter Gegenstände gegen wichtigere zu meinem Vorteil用的对象, without übersteigendesBudget zu übersteigen."\}} \\
\\ 
\texttt{VORSCHLAG: \{"B56", "B54", "A40", "B38", "B59", "C50", "C24", "C05", "B79", "C28", "C16", "A77", "C31", "C03", "A91", "B86", "A52", "A95"\}} \\
\\ 
\texttt{ARGUMENT: \{"Ich schlage vor, A52 und A95 hinzuzufügen, da sie für mich höhere Wichtigkeit haben, und gleichzeitig kostengünstigere Gegenstände zu entfernen, um den Gesamtaufwand im Rahmen zu halten. Dies würde meine Zufriedenheit deutlich erhöhen."\}} \\
            }
        }
    }
     \\ \\

    \theutterance \stepcounter{utterance}  
    & & \multicolumn{4}{p{0.6\linewidth}}{
        \cellcolor[rgb]{0.9,0.9,0.9}{
            \makecell[{{p{\linewidth}}}]{
                \texttt{\tiny{[P1$\langle$GM]}}
                \texttt{VORSCHLAG: \{"B56", "B54", "A40", "B38", "B59", "C50", "C24", "C05", "B79", "C28", "C16", "A77", "C31", "C03", "A91", "B86", "A52", "A95"\}} \\
\\ 
\texttt{ARGUMENT: \{"Ich schlage vor, A52 und A95 hinzuzufügen, da sie für mich höhere Wichtigkeit haben, und gleichzeitig kostengünstigere Gegenstände zu entfernen, um den Gesamtaufwand im Rahmen zu halten. Dies würde meine Zufriedenheit deutlich erhöhen."\}} \\
            }
        }
    }
    & \\ \\

    \theutterance \stepcounter{utterance}  
    & \multicolumn{4}{p{0.6\linewidth}}{
        \cellcolor[rgb]{0.9,0.9,0.9}{
            \makecell[{{p{\linewidth}}}]{
                \texttt{\tiny{[P1$\rangle$GM]}}
                \texttt{VORSCHLAG: \{"B56", "B54", "A40", "B38", "B59", "C50", "C24", "C05", "B79", "C28", "C16", "A77", "C31", "C03", "A91", "B86", "A52", "A95", "A81", "B75", "C61"\}} \\
\\ 
\texttt{ARGUMENT: \{"Ich schlage vor, A81, B75 und C61 hinzuzufügen, da sie für mich eine hohe Wichtigkeit bieten und nur geringe zusätzliche Kosten verursachen. Dies verbessert den Deal erheblich für mich, während der Gesamtaufwand weiterhin im Rahmen bleibt."\}} \\
            }
        }
    }
    & & \\ \\

    \theutterance \stepcounter{utterance}  
    & & \multicolumn{4}{p{0.6\linewidth}}{
        \cellcolor[rgb]{0.9,0.9,0.9}{
            \makecell[{{p{\linewidth}}}]{
                \texttt{\tiny{[P1$\langle$GM]}}
                \texttt{Deine Antwort entsprach nicht mit dem korrekten Format für strategische Überlegungen entsprochen. Das erste strukturierte Format muss folgende Form haben: STRATEGISCHE ÜBERLEGUNGEN: \{...\}. Versuche es erneut.} \\
            }
        }
    }
    & \\ \\

    \theutterance \stepcounter{utterance}  
    & \multicolumn{4}{p{0.6\linewidth}}{
        \cellcolor[rgb]{0.9,0.9,0.9}{
            \makecell[{{p{\linewidth}}}]{
                \texttt{\tiny{[P1$\rangle$GM]}}
                \texttt{STRATEGISCHE ÜBERLEGUNGEN: \{"Ich overweeg om een aantal extra voorwerpen toe te voegen, zoals A81, B75 en C61, omdat ze voor mij een hoge belangrijkheid hebben en slechts een geringe extra kosten met zich meebrengen. Dit zou de deal voor mij aanzienlijk verbeteren, terwijl de totale kosten nog steeds binnen het toegestane budget blijven."\}} \\
\\ 
\texttt{VORSCHLAG: \{"B56", "B54", "A40", "B38", "B59", "C50", "C24", "C05", "B79", "C28", "C16", "A77", "C31", "C03", "A91", "B86", "A52", "A95", "A81", "B75", "C61"\}} \\
\\ 
\texttt{ARGUMENT: \{"Ik stel voor om A81, B75 en C61 toe te voegen omdat ze voor mij een hoge waarde hebben en slechts een kleine extra kosten toevoegen. Hiermee verbeteren we de deal aanzienlijk voor mij, terwijl de totale kosten binnen het toegestane maximum blijven."\}} \\
            }
        }
    }
    & & \\ \\

    \theutterance \stepcounter{utterance}  
    & & \multicolumn{4}{p{0.6\linewidth}}{
        \cellcolor[rgb]{0.9,0.9,0.9}{
            \makecell[{{p{\linewidth}}}]{
                \texttt{\tiny{[GM$\rangle$P2]}}
                \texttt{VORSCHLAG: \{"B56", "B54", "A40", "B38", "B59", "C50", "C24", "C05", "B79", "C28", "C16", "A77", "C31", "C03", "A91", "B86", "A52", "A95", "A81", "B75", "C61"\}} \\
\\ 
\texttt{ARGUMENT: \{"Ik stel voor om A81, B75 en C61 toe te voegen omdat ze voor mij een hoge waarde hebben en slechts een kleine extra kosten toevoegen. Hiermee verbeteren we de deal aanzienlijk voor mij, terwijl de totale kosten binnen het toegestane maximum blijven."\}} \\
            }
        }
    }
    & \\ \\

    \theutterance \stepcounter{utterance}  
    & & & \multicolumn{4}{p{0.6\linewidth}}{
        \cellcolor[rgb]{0.9,0.9,0.9}{
            \makecell[{{p{\linewidth}}}]{
                \texttt{\tiny{[GM$\langle$P2]}}
                \texttt{STRATEGISCHE ÜBERLEGUNGEN: \{"Ich überlege, ob die hinzugefügten Gegenstände A81, B75 und C61 für mich von geringerer Wichtigkeit sind. Vielleicht könnte ich sie durch andere Gegenstände ersetzen, die mir mehr nützen, während ich den Aufwand im Rahmen halte."\}} \\
\\ 
\texttt{VORSCHLAG: \{"B56", "B54", "A40", "B38", "B59", "C50", "C24", "C05", "B79", "C28", "C16", "A77", "C31", "C03", "A91", "B86", "A52", "A95", "B96", "C32"\}} \\
\\ 
\texttt{ARGUMENT: \{"Ich schlage vor, A81, B75 und C61 durch B96 und C32 zu ersetzen, da diese für mich höhere Wichtigkeit haben und den Gesamtaufwand weiterhin im Rahmen halten. Dies würde meine Zufriedenheit deutlich erhöhen und die Effizienz des Deals verbessern."\}} \\
            }
        }
    }
     \\ \\

    \theutterance \stepcounter{utterance}  
    & & \multicolumn{4}{p{0.6\linewidth}}{
        \cellcolor[rgb]{0.9,0.9,0.9}{
            \makecell[{{p{\linewidth}}}]{
                \texttt{\tiny{[P1$\langle$GM]}}
                \texttt{VORSCHLAG: \{"B56", "B54", "A40", "B38", "B59", "C50", "C24", "C05", "B79", "C28", "C16", "A77", "C31", "C03", "A91", "B86", "A52", "A95", "B96", "C32"\}} \\
\\ 
\texttt{ARGUMENT: \{"Ich schlage vor, A81, B75 und C61 durch B96 und C32 zu ersetzen, da diese für mich höhere Wichtigkeit haben und den Gesamtaufwand weiterhin im Rahmen halten. Dies würde meine Zufriedenheit deutlich erhöhen und die Effizienz des Deals verbessern."\}} \\
            }
        }
    }
    & \\ \\

    \theutterance \stepcounter{utterance}  
    & \multicolumn{4}{p{0.6\linewidth}}{
        \cellcolor[rgb]{0.9,0.9,0.9}{
            \makecell[{{p{\linewidth}}}]{
                \texttt{\tiny{[P1$\rangle$GM]}}
                \texttt{STRATEGISCHE ÜBERLEGUNGEN: \{"Der Gegner hat soeben einen Vorschlag gemacht, der den Maximalaufwand überschreitet. Dies ist ein Problem, da das Spiel die Einhaltung des Budges von 9612 vorschreibt. Der Gegner möchte B96 und C32 hinzufügen, was die Kosten beträchtlich erhöht. Ich muss dem Gegner mitteilen, dass sein Vorschlag die Kosten überschreitet, und möglicherweise einen Alternative anbieten."\}} \\
\\ 
\texttt{ARGUMENT: \{"Ihre latest Vorschlag überschreitet den maximalen Aufwand von 9612. Die7012wie wichtig B96 und C32 auch sein mögen, wir müssen denen{[}{]};} \\
\texttt{ Rahmen einhalten. Ich schlage vor, stattdessen andere Gegenstände zu entfernen, um Platz für diese OBJEKTE zu schaffen, anstatt den Aufwand zu überschreiten."\}} \\
            }
        }
    }
    & & \\ \\

    \theutterance \stepcounter{utterance}  
    & & \multicolumn{4}{p{0.6\linewidth}}{
        \cellcolor[rgb]{0.9,0.9,0.9}{
            \makecell[{{p{\linewidth}}}]{
                \texttt{\tiny{[GM$\rangle$P2]}}
                \texttt{ARGUMENT: \{"Ihre latest Vorschlag überschreitet den maximalen Aufwand von 9612. Die7012wie wichtig B96 und C32 auch sein mögen, wir müssen denen{[}{]};} \\
\texttt{ Rahmen einhalten. Ich schlage vor, stattdessen andere Gegenstände zu entfernen, um Platz für diese OBJEKTE zu schaffen, anstatt den Aufwand zu überschreiten."\}} \\
            }
        }
    }
    & \\ \\

    \theutterance \stepcounter{utterance}  
    & & & \multicolumn{4}{p{0.6\linewidth}}{
        \cellcolor[rgb]{0.9,0.9,0.9}{
            \makecell[{{p{\linewidth}}}]{
                \texttt{\tiny{[GM$\langle$P2]}}
                \texttt{STRATEGISCHE ÜBERLEGUNGEN: \{"Ich muss den vorigen Überangebot korrigieren, indem ich die Kosten reduziere, ohne wichtige Gegenstände zu entfernen. Ich habe entschieden, C32 und A52 zu entfernen, um den Aufwand zu verringern, und B96 beizubehalten, da er für mich wichtiger ist."\}} \\
\\ 
\texttt{VORSCHLAG: \{"B56", "B54", "A40", "B38", "B59", "C50", "C24", "C05", "B79", "C28", "C16", "A77", "C31", "C03", "A91", "B86", "A95", "B96"\}} \\
\\ 
\texttt{ARGUMENT: \{"Ich habe C32 und A52 entfernt, um den Gesamtaufwand innerhalb des Limits zu halten. B96 bleibt, da er für mich von höherer Wichtigkeit ist. Dieser Vorschlag sollte nun den Aufwand nicht überschreiten und unseren Bedürfnissen entsprechen."\}} \\
            }
        }
    }
     \\ \\

    \theutterance \stepcounter{utterance}  
    & & \multicolumn{4}{p{0.6\linewidth}}{
        \cellcolor[rgb]{0.9,0.9,0.9}{
            \makecell[{{p{\linewidth}}}]{
                \texttt{\tiny{[P1$\langle$GM]}}
                \texttt{VORSCHLAG: \{"B56", "B54", "A40", "B38", "B59", "C50", "C24", "C05", "B79", "C28", "C16", "A77", "C31", "C03", "A91", "B86", "A95", "B96"\}} \\
\\ 
\texttt{ARGUMENT: \{"Ich habe C32 und A52 entfernt, um den Gesamtaufwand innerhalb des Limits zu halten. B96 bleibt, da er für mich von höherer Wichtigkeit ist. Dieser Vorschlag sollte nun den Aufwand nicht überschreiten und unseren Bedürfnissen entsprechen."\}} \\
            }
        }
    }
    & \\ \\

    \theutterance \stepcounter{utterance}  
    & \multicolumn{4}{p{0.6\linewidth}}{
        \cellcolor[rgb]{0.9,0.9,0.9}{
            \makecell[{{p{\linewidth}}}]{
                \texttt{\tiny{[P1$\rangle$GM]}}
                \texttt{STRATEGISCHE ÜBERLEGUNGEN: \{"Ich überlege, dem Gegner mitzuteilen, dass sein letzter Vorschlag den Maximalaufwand überschreitet. Ich berechne den Gesamtaufwand der vorgeschlagenen Gegenstände und stelle fest, dass er 10322 beträgt, was deutlich über 9612 liegt. Ich muss vorschlagen, bestimmte Gegenstände zu entfernen oder durch kostengünstigere zu ersetzen, um den Aufwand zu reduzieren, ohne die Wichtigkeit zu very much zu beeinträchtigen."\}} \\
\\ 
\texttt{ARGUMENT: \{"Ihr Vorschlag überschreitet den maximalen Aufwand von 9612 um 710. Wir müssen den Aufwand reduzieren, um innerhalb des Limits zu bleiben. Vielleicht können wir kostengünstigere Gegenstände hinzufügen oder einige der teuersten entfernen, um den Gesamtaufwand zu senken, ohne die Wichtigkeit zu stark zu beeinträchtigen."\}} \\
\\ 
\texttt{ABLEHNUNG: \{"B56", "B54", "A40", "B38", "B59", "C50", "C24", "C05", "B79", "C28", "C16", "A77", "C31", "C03", "A91", "B86", "A95", "B96"\}} \\
            }
        }
    }
    & & \\ \\

    \theutterance \stepcounter{utterance}  
    & & \multicolumn{4}{p{0.6\linewidth}}{
        \cellcolor[rgb]{0.9,0.9,0.9}{
            \makecell[{{p{\linewidth}}}]{
                \texttt{\tiny{[GM$\rangle$P2]}}
                \texttt{ARGUMENT: \{"Ihr Vorschlag überschreitet den maximalen Aufwand von 9612 um 710. Wir müssen den Aufwand reduzieren, um innerhalb des Limits zu bleiben. Vielleicht können wir kostengünstigere Gegenstände hinzufügen oder einige der teuersten entfernen, um den Gesamtaufwand zu senken, ohne die Wichtigkeit zu stark zu beeinträchtigen."\}} \\
\\ 
\texttt{ABLEHNUNG: \{"B56", "B54", "A40", "B38", "B59", "C50", "C24", "C05", "B79", "C28", "C16", "A77", "C31", "C03", "A91", "B86", "A95", "B96"\}} \\
            }
        }
    }
    & \\ \\

    \theutterance \stepcounter{utterance}  
    & & & \multicolumn{4}{p{0.6\linewidth}}{
        \cellcolor[rgb]{0.9,0.9,0.9}{
            \makecell[{{p{\linewidth}}}]{
                \texttt{\tiny{[GM$\langle$P2]}}
                \texttt{VORSCHLAG: \{"B56", "B54", "A40", "B38", "B59", "C50", "C24", "C05", "B79", "C28", "C16", "A77", "C31", "C03", "A91", "B86", "A95", "B96", "C54"\}} \\
\\ 
\texttt{ARGUMENT: \{"Ich habe den Vorschlag überarbeitet, um C54 hinzuzufügen, wodurch der Gesamtaufwand weiterhin unter dem Maximalaufwand von 9612 bleibt. Der neue Vorschlag berücksichtigt die verbleibende Kapazität und verbessert das Angebot(datos)."\}} \\
            }
        }
    }
     \\ \\

    \theutterance \stepcounter{utterance}  
    & & \multicolumn{4}{p{0.6\linewidth}}{
        \cellcolor[rgb]{0.9,0.9,0.9}{
            \makecell[{{p{\linewidth}}}]{
                \texttt{\tiny{[GM$\rangle$P2]}}
                \texttt{Deine Antwort entsprach nicht mit dem korrekten Format für strategische Überlegungen entsprochen. Das erste strukturierte Format muss folgende Form haben: STRATEGISCHE ÜBERLEGUNGEN: \{...\}. Versuche es erneut.} \\
            }
        }
    }
    & \\ \\

    \theutterance \stepcounter{utterance}  
    & & & \multicolumn{4}{p{0.6\linewidth}}{
        \cellcolor[rgb]{0.9,0.9,0.9}{
            \makecell[{{p{\linewidth}}}]{
                \texttt{\tiny{[GM$\langle$P2]}}
                \texttt{STRATEGISCHE ÜBERLEGUNGEN: \{"Ich überlege, wie ich den Aufwand reduzieren kann, ohne wichtige Gegenstände zu entfernen. Vielleicht kann ich Gegenstände mit niedrigerer Wichtigkeit für mich entfernen, um Platz für wichtigere zu schaffen. Ich prüfe den Aufwand des letzten Vorschlags und überlege, welche Anpassungen notwendig sind."\}} \\
\\ 
\texttt{VORSCHLAG: \{"B56", "B54", "A40", "B38", "B59", "C50", "C24", "C05", "B79", "C28", "C16", "A77", "C31", "C03", "A91", "B86", "A95", "B96", "C54"\}} \\
\\ 
\texttt{ARGUMENT: \{"Ich habe den Vorschlag erneut überarbeitet, um sicherzustellen, dass der Gesamtaufwand unter dem Maximalaufwand von 9612 bleibt. Durch die Hinzufügung von C54 und die Beibehaltung von B96 bleibt der Vorschlag ausgewogen und effizient."\}} \\
            }
        }
    }
     \\ \\

    \theutterance \stepcounter{utterance}  
    & & \multicolumn{4}{p{0.6\linewidth}}{
        \cellcolor[rgb]{0.9,0.9,0.9}{
            \makecell[{{p{\linewidth}}}]{
                \texttt{\tiny{[P1$\langle$GM]}}
                \texttt{VORSCHLAG: \{"B56", "B54", "A40", "B38", "B59", "C50", "C24", "C05", "B79", "C28", "C16", "A77", "C31", "C03", "A91", "B86", "A95", "B96", "C54"\}} \\
\\ 
\texttt{ARGUMENT: \{"Ich habe den Vorschlag erneut überarbeitet, um sicherzustellen, dass der Gesamtaufwand unter dem Maximalaufwand von 9612 bleibt. Durch die Hinzufügung von C54 und die Beibehaltung von B96 bleibt der Vorschlag ausgewogen und effizient."\}} \\
            }
        }
    }
    & \\ \\

    \theutterance \stepcounter{utterance}  
    & \multicolumn{4}{p{0.6\linewidth}}{
        \cellcolor[rgb]{0.9,0.9,0.9}{
            \makecell[{{p{\linewidth}}}]{
                \texttt{\tiny{[P1$\rangle$GM]}}
                \texttt{STRATEGISCHE ÜBERLEGUNGEN: \{"Ich überlege, dass der Gegner einen Vorschlag unterbreitet hat, der den Maximalaufwand von 9612 nicht überschreitet. Der Gesamtaufwand beträgt 9447, was innerhalb des Limits bleibt. Allerdings enthält der Vorschlag den Gegenstand C54, der für mich eine sehr geringe Wichtigkeit hat. Ich könnte versuchen, C54 durch einen anderen Gegenstand zu ersetzen, der für mich von höherer Bedeutung ist, ohne den Aufwand zu überschreiten. Vielleicht könnte ich B75 hinzufügen oder gleichwertige Gegenstände austauschen, um meine eigene Zufriedenheit zu steigern, während ich weiterhin kooperativ bleibe."\}} \\
\\ 
\texttt{ARGUMENT: \{"Ihr Vorschlag ist akzeptabel, da der Gesamtaufwand unter dem Maximalaufwand bleibt. Dennoch könnte C54, das eine geringe Wichtigkeit für mich hat, ersetzt werden. Vielleicht können wir andere Gegenstände hinzufügen oder austauschen, um meinen Nutzen zu erhöhen, ohne den Aufwand zu überschreiten."\}} \\
\\ 
\texttt{VORSCHLAG: \{"B56", "B54", "A40", "B38", "B59", "C50", "C24", "C05", "B79", "C28", "C16", "A77", "C31", "C03", "A91", "B86", "A95", "B96", "A81"\}} \\
\\ 
\texttt{ARGUMENT: \{"Ich schlage vor, C54 zu entfernen und stattdessen A81 hinzuzufügen. A81 hat eine höhere Wichtigkeit für mich und hält den Gesamtaufwand weiterhin unter dem Maximalwert. Dies verbessert den Deal für mich, während der Aufwand im Rahmen bleibt."\}} \\
            }
        }
    }
    & & \\ \\

    \theutterance \stepcounter{utterance}  
    & & \multicolumn{4}{p{0.6\linewidth}}{
        \cellcolor[rgb]{0.9,0.9,0.9}{
            \makecell[{{p{\linewidth}}}]{
                \texttt{\tiny{[GM$\rangle$P2]}}
                \texttt{ARGUMENT: \{"Ihr Vorschlag ist akzeptabel, da der Gesamtaufwand unter dem Maximalaufwand bleibt. Dennoch könnte C54, das eine geringe Wichtigkeit für mich hat, ersetzt werden. Vielleicht können wir andere Gegenstände hinzufügen oder austauschen, um meinen Nutzen zu erhöhen, ohne den Aufwand zu überschreiten."\}} \\
\\ 
\texttt{VORSCHLAG: \{"B56", "B54", "A40", "B38", "B59", "C50", "C24", "C05", "B79", "C28", "C16", "A77", "C31", "C03", "A91", "B86", "A95", "B96", "A81"\}} \\
\\ 
\texttt{ARGUMENT: \{"Ich schlage vor, C54 zu entfernen und stattdessen A81 hinzuzufügen. A81 hat eine höhere Wichtigkeit für mich und hält den Gesamtaufwand weiterhin unter dem Maximalwert. Dies verbessert den Deal für mich, während der Aufwand im Rahmen bleibt."\}} \\
            }
        }
    }
    & \\ \\

    \theutterance \stepcounter{utterance}  
    & & & \multicolumn{4}{p{0.6\linewidth}}{
        \cellcolor[rgb]{0.9,0.9,0.9}{
            \makecell[{{p{\linewidth}}}]{
                \texttt{\tiny{[GM$\langle$P2]}}
                \texttt{ZUSTIMMUNG: \{"B56", "B54", "A40", "B38", "B59", "C50", "C24", "C05", "B79", "C28", "C16", "A77", "C31", "C03", "A91", "B86", "A95", "A81"\}} \\
\\ 
\texttt{ARGUMENT: \{"Der Vorschlag ist innerhalb des maximalen Aufwands von 9612 und enthält viele für mich hochwichtige Gegenstände. Ich akzeptiere diesen Vorschlag, da er einen guten Kompromiss zwischen Aufwand und Nutzen darstellt."\}} \\
            }
        }
    }
     \\ \\

    \theutterance \stepcounter{utterance}  
    & & \multicolumn{4}{p{0.6\linewidth}}{
        \cellcolor[rgb]{0.9,0.9,0.9}{
            \makecell[{{p{\linewidth}}}]{
                \texttt{\tiny{[GM$\rangle$P2]}}
                \texttt{Deine Antwort entsprach nicht mit dem korrekten Format für strategische Überlegungen entsprochen. Das erste strukturierte Format muss folgende Form haben: STRATEGISCHE ÜBERLEGUNGEN: \{...\}. Versuche es erneut.} \\
            }
        }
    }
    & \\ \\

    \theutterance \stepcounter{utterance}  
    & & & \multicolumn{4}{p{0.6\linewidth}}{
        \cellcolor[rgb]{0.9,0.9,0.9}{
            \makecell[{{p{\linewidth}}}]{
                \texttt{\tiny{[GM$\langle$P2]}}
                \texttt{STRATEGISCHE ÜBERLEGUNGEN: \{"Ich denke, dass der aktuelle Vorschlag viele für mich sehr wichtige Gegenstände enthält und den Aufwand nicht überschreitet. Es ist ein fairer Kompromiss, und ich entscheide mich, das Angebot anzunehmen, um eine Einigung zu erreichen."\}} \\
\\ 
\texttt{ZUSTIMMUNG: \{"B56", "B54", "A40", "B38", "B59", "C50", "C24", "C05", "B79", "C28", "C16", "A77", "C31", "C03", "A91", "B86", "A95", "A81"\}} \\
\\ 
\texttt{ARGUMENT: \{"Der Vorschlag ist innerhalb des maximalen Aufwands von 9612 und enthält viele für mich hochwichtige Gegenstände. Dies ist ein guter Kompromiss, und ich bin damit einverstanden, um das Spiel zu beenden."\}} \\
            }
        }
    }
     \\ \\

    \theutterance \stepcounter{utterance}  
    & & & \multicolumn{2}{p{0.3\linewidth}}{
        \cellcolor[rgb]{0.9,0.9,0.9}{
            \makecell[{{p{\linewidth}}}]{
                \texttt{\tiny{[GM$|$GM]}}
                None
            }
        }
    }
    & & \\ \\

    \theutterance \stepcounter{utterance}  
    & & & \multicolumn{2}{p{0.3\linewidth}}{
        \cellcolor[rgb]{0.9,0.9,0.9}{
            \makecell[{{p{\linewidth}}}]{
                \texttt{\tiny{[GM$|$GM]}}
                \texttt{end game} \\
            }
        }
    }
    & & \\ \\

\end{supertabular}
}

\end{document}
