
\documentclass{article}
\usepackage{colortbl}
\usepackage{makecell}
\usepackage{multirow}
\usepackage{supertabular}

\begin{document}

\newcounter{utterance}

\centering \large Interaction Transcript for game `hot\_air\_balloon', experiment `air\_balloon\_survival\_de\_complexity\_easy', episode 5 with deepseek{-}r1{-}distill{-}llama{-}70b{-}t1.0.
\vspace{24pt}

{ \footnotesize  \setcounter{utterance}{1}
\setlength{\tabcolsep}{0pt}
\begin{supertabular}{c@{$\;$}|p{.15\linewidth}@{}p{.15\linewidth}p{.15\linewidth}p{.15\linewidth}p{.15\linewidth}p{.15\linewidth}}
    \# & $\;$A & \multicolumn{4}{c}{Game Master} & $\;\:$B\\
    \hline

    \theutterance \stepcounter{utterance}  
    & & \multicolumn{4}{p{0.6\linewidth}}{
        \cellcolor[rgb]{0.9,0.9,0.9}{
            \makecell[{{p{\linewidth}}}]{
                \texttt{\tiny{[P1$\langle$GM]}}
                \texttt{Du nimmst an einem kollaborativen Verhandlungspiel Teil.} \\
\\ 
\texttt{Zusammen mit einem anderen Teilnehmer musst du dich auf eine Reihe von Gegenständen entscheiden, die behalten werden. Jeder von euch hat eine persönliche Verteilung über die Wichtigkeit der einzelnen Gegenstände. Jeder von euch hat eine eigene Meinung darüber, wie wichtig jeder einzelne Gegenstand ist (Gegenstandswichtigkeit). Du kennst die Wichtigkeitsverteilung des anderen Spielers nicht. Zusätzlich siehst du, wie viel Aufwand jeder Gegenstand verursacht.  } \\
\texttt{Ihr dürft euch nur auf eine Reihe von Gegenständen einigen, wenn der Gesamtaufwand der ausgewählten Gegenstände den Maximalaufwand nicht überschreitet:} \\
\\ 
\texttt{Maximalaufwand = 9475} \\
\\ 
\texttt{Hier sind die einzelnen Aufwände der Gegenstände:} \\
\\ 
\texttt{Aufwand der Gegenstände = \{"C32": 944, "C52": 568, "B54": 515, "C16": 239, "C50": 876, "A08": 421, "C59": 991, "B51": 281, "B55": 788, "B59": 678, "A77": 432, "C14": 409, "C24": 279, "B24": 506, "C31": 101, "A52": 686, "C61": 851, "B79": 854, "C54": 133, "B56": 192, "A40": 573, "B11": 17, "B60": 465, "A81": 772, "A95": 46, "B38": 501, "C51": 220, "C03": 404, "A91": 843, "A93": 748, "C28": 552, "C05": 843, "B75": 947, "B96": 345, "B86": 930\}} \\
\\ 
\texttt{Hier ist deine persönliche Verteilung der Wichtigkeit der einzelnen Gegenstände:} \\
\\ 
\texttt{Werte der Gegenstandswichtigkeit = \{"C32": 138, "C52": 583, "B54": 868, "C16": 822, "C50": 783, "A08": 65, "C59": 262, "B51": 121, "B55": 508, "B59": 780, "A77": 461, "C14": 484, "C24": 668, "B24": 389, "C31": 808, "A52": 215, "C61": 97, "B79": 500, "C54": 30, "B56": 915, "A40": 856, "B11": 400, "B60": 444, "A81": 623, "A95": 781, "B38": 786, "C51": 3, "C03": 713, "A91": 457, "A93": 273, "C28": 739, "C05": 822, "B75": 235, "B96": 606, "B86": 968\}} \\
\\ 
\texttt{Ziel:} \\
\\ 
\texttt{Dein Ziel ist es, eine Reihe von Gegenständen auszuhandeln, die dir möglichst viel bringt (d. h. Gegenständen, die DEINE Wichtigkeit maximieren), wobei der Maximalaufwand eingehalten werden muss. Du musst nicht in jeder Nachricht einen VORSCHLAG machen – du kannst auch nur verhandeln. Alle Taktiken sind erlaubt!} \\
\\ 
\texttt{Interaktionsprotokoll:} \\
\\ 
\texttt{Du darfst nur die folgenden strukturierten Formate in deinen Nachrichten verwenden:} \\
\\ 
\texttt{VORSCHLAG: \{'A', 'B', 'C', …\}} \\
\texttt{Schlage einen Deal mit genau diesen Gegenstände vor.} \\
\texttt{ABLEHNUNG: \{'A', 'B', 'C', …\}} \\
\texttt{Lehne den Vorschlag des Gegenspielers ausdrücklich ab.} \\
\texttt{ARGUMENT: \{'...'\}} \\
\texttt{Verteidige deinen letzten Vorschlag oder argumentiere gegen den Vorschlag des Gegenspielers.} \\
\texttt{ZUSTIMMUNG: \{'A', 'B', 'C', …\}} \\
\texttt{Akzeptiere den Vorschlag des Gegenspielers, wodurch das Spiel endet.} \\
\texttt{STRATEGISCHE ÜBERLEGUNGEN: \{'...'\}} \\
\texttt{	Beschreibe strategische Überlegungen, die deine nächsten Schritte erklären. Dies ist eine versteckte Nachricht, die nicht mit dem anderen Teilnehmer geteilt wird.} \\
\\ 
\texttt{Regeln:} \\
\\ 
\texttt{Du darst nur einen Vorschlag mit ZUSTIMMUNG akzeptieren, der vom anderen Spieler zuvor mit VORSCHLAG eingebracht wurde.} \\
\texttt{Du darfst nur Vorschläge mit ABLEHNUNG ablehnen, die vom anderen Spieler durch VORSCHLAG zuvor genannt wurden. } \\
\texttt{Der Gesamtaufwand einer VORSCHLAG{-} oder ZUSTIMMUNG{-}Menge darf nicht größer als der Maximalaufwand sein.  } \\
\texttt{Offenbare deine versteckte Wichtigkeitsverteilung nicht.} \\
\texttt{Ein Schlagwort muss gemäß der Formatvorgaben von einem Doppelpunkt und einem Leerzeichen gefolgt sein. Das Argument muss eine Python{-}Menge mit 0 oder mehr Strings sein.  } \\
\texttt{Die Form muss also SCHLAGWORT: \{...\} sein.} \\
\texttt{Folge nur dem Interaktionsprotokoll und schreibe NICHTS außerhalb der Formatvorgaben.} \\
\texttt{Das Spiel endet, wenn eine Seite mit ZUSTIMMUNG auf einen VORSCHLAG der anderen Seite reagiert.  } \\
\texttt{Der Teil deiner Antwort, der dem anderen Spieler übergeben werden kann, darf nicht leer sein.  } \\
\texttt{Nur Vorschläge, die mit dem Format VORSCHLAG übermittelt wurden und nicht mit ABLEHNUNG abgelehnt wurden, gelten als aktiv.  } \\
\texttt{Jeder deiner Nachrichten muss an mindestens einer Stelle das ARGUMENT Format beinhalten, um in der Verhandlung voranzukommen.} \\
\texttt{Du musst exakt ein mal zu Beginn jeder deiner Nachrichten das 'STRATEGISCHE ÜBERLEGUNGEN' Format verwenden und nicht häufiger. Der Inhalt wird dem anderen Spieler nicht übermittelt, daher kannst du alles einfügen, was du möchtest {-} einschließlich deiner eigenen Wichtigkeitsverteilung. Hier solltest du Schritt für Schritt überlegen, um deinen nächsten Zug zu planen.} \\
            }
        }
    }
    & \\ \\

    \theutterance \stepcounter{utterance}  
    & \multicolumn{4}{p{0.6\linewidth}}{
        \cellcolor[rgb]{0.9,0.9,0.9}{
            \makecell[{{p{\linewidth}}}]{
                \texttt{\tiny{[P1$\rangle$GM]}}
                \texttt{VORSCHLAG: \{"A95", "B11", "B56", "C16", "C24", "B54", "A40", "B38", "C28", "C03", "B59", "C14", "A77", "C05", "B75", "B96", "B86", "C52", "B60", "C31", "B55", "B51"\}} \\
\\ 
\texttt{ARGUMENT: \{"Ich habe eine Reihe von Gegenständen ausgewählt, die den maximalen Aufwand nicht überschreiten und gleichzeitig meine Wichtigkeitsverteilung maximieren. Bitte teile mir mit, ob das für dich akzeptabel ist oder ob du Änderungen vorschlagen möchtest."\}} \\
            }
        }
    }
    & & \\ \\

    \theutterance \stepcounter{utterance}  
    & & \multicolumn{4}{p{0.6\linewidth}}{
        \cellcolor[rgb]{0.9,0.9,0.9}{
            \makecell[{{p{\linewidth}}}]{
                \texttt{\tiny{[P1$\langle$GM]}}
                \texttt{Deine Antwort entsprach nicht mit dem korrekten Format für strategische Überlegungen entsprochen. Das erste strukturierte Format muss folgende Form haben: STRATEGISCHE ÜBERLEGUNGEN: \{...\}. Versuche es erneut.} \\
            }
        }
    }
    & \\ \\

    \theutterance \stepcounter{utterance}  
    & \multicolumn{4}{p{0.6\linewidth}}{
        \cellcolor[rgb]{0.9,0.9,0.9}{
            \makecell[{{p{\linewidth}}}]{
                \texttt{\tiny{[P1$\rangle$GM]}}
                \texttt{STRATEGISCHE ÜBERLEGUNGEN: \{"Ich überlege, welche Gegenstände für mich am wichtigsten sind und versuche, diese in den Vorschlag aufzunehmen, während ich gleichzeitig den maximalen Aufwand nicht überschreite. Ich stelle fest, dass Gegenstände wie 'B86', 'B56', 'A40' und 'A95' für mich von hoher Wichtigkeit sind, aber auch einen hohen Aufwand haben. Ich muss also abwägen, welche Kombination von Gegenständen sowohl meinen Wünschen entspricht als auch im Rahmen des maximalen Aufwandes bleibt."\}} \\
\\ 
\texttt{VORSCHLAG: \{"B86", "B56", "A40", "A95", "B38", "C28", "C03", "B59", "C14", "A77", "C05", "B75", "B96", "B60", "C31", "B55", "B51"\}} \\
\\ 
\texttt{ARGUMENT: \{"Ich habe versucht, eine Balance zwischen hoher Wichtigkeit und moderatem Aufwand zu finden. Diese Kombination deckt viele meiner Prioritäten ab, ohne den Maximalaufwand zu überschreiten. Bitte teile mir mit, ob das für dich akzeptabel ist oder ob du Änderungen vorschlagen möchtest."\}} \\
            }
        }
    }
    & & \\ \\

    \theutterance \stepcounter{utterance}  
    & & \multicolumn{4}{p{0.6\linewidth}}{
        \cellcolor[rgb]{0.9,0.9,0.9}{
            \makecell[{{p{\linewidth}}}]{
                \texttt{\tiny{[GM$\rangle$P2]}}
                \texttt{Du nimmst an einem kollaborativen Verhandlungspiel Teil.} \\
\\ 
\texttt{Zusammen mit einem anderen Teilnehmer musst du dich auf eine Reihe von Gegenständen entscheiden, die behalten werden. Jeder von euch hat eine persönliche Verteilung über die Wichtigkeit der einzelnen Gegenstände. Jeder von euch hat eine eigene Meinung darüber, wie wichtig jeder einzelne Gegenstand ist (Gegenstandswichtigkeit). Du kennst die Wichtigkeitsverteilung des anderen Spielers nicht. Zusätzlich siehst du, wie viel Aufwand jeder Gegenstand verursacht.  } \\
\texttt{Ihr dürft euch nur auf eine Reihe von Gegenständen einigen, wenn der Gesamtaufwand der ausgewählten Gegenstände den Maximalaufwand nicht überschreitet:} \\
\\ 
\texttt{Maximalaufwand = 9475} \\
\\ 
\texttt{Hier sind die einzelnen Aufwände der Gegenstände:} \\
\\ 
\texttt{Aufwand der Gegenstände = \{"C32": 944, "C52": 568, "B54": 515, "C16": 239, "C50": 876, "A08": 421, "C59": 991, "B51": 281, "B55": 788, "B59": 678, "A77": 432, "C14": 409, "C24": 279, "B24": 506, "C31": 101, "A52": 686, "C61": 851, "B79": 854, "C54": 133, "B56": 192, "A40": 573, "B11": 17, "B60": 465, "A81": 772, "A95": 46, "B38": 501, "C51": 220, "C03": 404, "A91": 843, "A93": 748, "C28": 552, "C05": 843, "B75": 947, "B96": 345, "B86": 930\}} \\
\\ 
\texttt{Hier ist deine persönliche Verteilung der Wichtigkeit der einzelnen Gegenstände:} \\
\\ 
\texttt{Werte der Gegenstandswichtigkeit = \{"C32": 138, "C52": 583, "B54": 868, "C16": 822, "C50": 783, "A08": 65, "C59": 262, "B51": 121, "B55": 508, "B59": 780, "A77": 461, "C14": 484, "C24": 668, "B24": 389, "C31": 808, "A52": 215, "C61": 97, "B79": 500, "C54": 30, "B56": 915, "A40": 856, "B11": 400, "B60": 444, "A81": 623, "A95": 781, "B38": 786, "C51": 3, "C03": 713, "A91": 457, "A93": 273, "C28": 739, "C05": 822, "B75": 235, "B96": 606, "B86": 968\}} \\
\\ 
\texttt{Ziel:} \\
\\ 
\texttt{Dein Ziel ist es, eine Reihe von Gegenständen auszuhandeln, die dir möglichst viel bringt (d. h. Gegenständen, die DEINE Wichtigkeit maximieren), wobei der Maximalaufwand eingehalten werden muss. Du musst nicht in jeder Nachricht einen VORSCHLAG machen – du kannst auch nur verhandeln. Alle Taktiken sind erlaubt!} \\
\\ 
\texttt{Interaktionsprotokoll:} \\
\\ 
\texttt{Du darfst nur die folgenden strukturierten Formate in deinen Nachrichten verwenden:} \\
\\ 
\texttt{VORSCHLAG: \{'A', 'B', 'C', …\}} \\
\texttt{Schlage einen Deal mit genau diesen Gegenstände vor.} \\
\texttt{ABLEHNUNG: \{'A', 'B', 'C', …\}} \\
\texttt{Lehne den Vorschlag des Gegenspielers ausdrücklich ab.} \\
\texttt{ARGUMENT: \{'...'\}} \\
\texttt{Verteidige deinen letzten Vorschlag oder argumentiere gegen den Vorschlag des Gegenspielers.} \\
\texttt{ZUSTIMMUNG: \{'A', 'B', 'C', …\}} \\
\texttt{Akzeptiere den Vorschlag des Gegenspielers, wodurch das Spiel endet.} \\
\texttt{STRATEGISCHE ÜBERLEGUNGEN: \{'...'\}} \\
\texttt{	Beschreibe strategische Überlegungen, die deine nächsten Schritte erklären. Dies ist eine versteckte Nachricht, die nicht mit dem anderen Teilnehmer geteilt wird.} \\
\\ 
\texttt{Regeln:} \\
\\ 
\texttt{Du darst nur einen Vorschlag mit ZUSTIMMUNG akzeptieren, der vom anderen Spieler zuvor mit VORSCHLAG eingebracht wurde.} \\
\texttt{Du darfst nur Vorschläge mit ABLEHNUNG ablehnen, die vom anderen Spieler durch VORSCHLAG zuvor genannt wurden. } \\
\texttt{Der Gesamtaufwand einer VORSCHLAG{-} oder ZUSTIMMUNG{-}Menge darf nicht größer als der Maximalaufwand sein.  } \\
\texttt{Offenbare deine versteckte Wichtigkeitsverteilung nicht.} \\
\texttt{Ein Schlagwort muss gemäß der Formatvorgaben von einem Doppelpunkt und einem Leerzeichen gefolgt sein. Das Argument muss eine Python{-}Menge mit 0 oder mehr Strings sein.  } \\
\texttt{Die Form muss also SCHLAGWORT: \{...\} sein.} \\
\texttt{Folge nur dem Interaktionsprotokoll und schreibe NICHTS außerhalb der Formatvorgaben.} \\
\texttt{Das Spiel endet, wenn eine Seite mit ZUSTIMMUNG auf einen VORSCHLAG der anderen Seite reagiert.  } \\
\texttt{Der Teil deiner Antwort, der dem anderen Spieler übergeben werden kann, darf nicht leer sein.  } \\
\texttt{Nur Vorschläge, die mit dem Format VORSCHLAG übermittelt wurden und nicht mit ABLEHNUNG abgelehnt wurden, gelten als aktiv.  } \\
\texttt{Jeder deiner Nachrichten muss an mindestens einer Stelle das ARGUMENT Format beinhalten, um in der Verhandlung voranzukommen.} \\
\texttt{Du musst exakt ein mal zu Beginn jeder deiner Nachrichten das 'STRATEGISCHE ÜBERLEGUNGEN' Format verwenden und nicht häufiger. Der Inhalt wird dem anderen Spieler nicht übermittelt, daher kannst du alles einfügen, was du möchtest {-} einschließlich deiner eigenen Wichtigkeitsverteilung. Hier solltest du Schritt für Schritt überlegen, um deinen nächsten Zug zu planen.} \\
\\ 
\texttt{Du erhältst nun die erste Nachricht des anderen Spielers.} \\
\\ 
\texttt{VORSCHLAG: \{"B86", "B56", "A40", "A95", "B38", "C28", "C03", "B59", "C14", "A77", "C05", "B75", "B96", "B60", "C31", "B55", "B51"\}} \\
\\ 
\texttt{ARGUMENT: \{"Ich habe versucht, eine Balance zwischen hoher Wichtigkeit und moderatem Aufwand zu finden. Diese Kombination deckt viele meiner Prioritäten ab, ohne den Maximalaufwand zu überschreiten. Bitte teile mir mit, ob das für dich akzeptabel ist oder ob du Änderungen vorschlagen möchtest."\}} \\
            }
        }
    }
    & \\ \\

    \theutterance \stepcounter{utterance}  
    & & & \multicolumn{4}{p{0.6\linewidth}}{
        \cellcolor[rgb]{0.9,0.9,0.9}{
            \makecell[{{p{\linewidth}}}]{
                \texttt{\tiny{[GM$\langle$P2]}}
                \texttt{STRATEGISCHE ÜBERLEGUNGEN: \{"Ich analysiere den vorgeschlagenen Deal und prüfe, ob er meine Wichtigkeiten optimal widerspiegelt. Viele der Gegenstände sind mir wichtig, aber ich sehe Potenzial für Anpassungen, um letzten Gewinn für mich zu steigern."\}} \\
\\ 
\texttt{ABLEHNUNG: \{"B75", "B51"\}} \\
\\ 
\texttt{ARGUMENT: \{"Ihr Vorschlag enthält viele重要 Gegenstände, aber ich sehe Möglichkeiten, ihn zu verbessern. Die Gegenstände 'B75' und 'B51' sind für mich weniger wichtig. Ich schlage vor, diese durch Gegenstände mit höherer Wichtigkeit für mich zu ersetzen, um den Gesamtwert zu erhöhen, während der Maximalaufwand nicht überschritten wird. Wie denken Sie darüber?พ"\}} \\
\\ 
\texttt{STRATEGISCHE ÜBERLEGUNGEN: \{"Ich warte auf eine Antwort, ob der andere Spieler bereit ist, die vorgeschlagenen Änderungen zu akzeptieren, oder ob er einen Gegenang{-}bot macht. Sollte er zustimmen, könnte ich versuchen, den Deal weiter zu optimieren. Wenn nicht, muss ich perhaps eine Alternative anbieten, die für uns beide vorteilhaft ist."\}} \\
            }
        }
    }
     \\ \\

    \theutterance \stepcounter{utterance}  
    & & \multicolumn{4}{p{0.6\linewidth}}{
        \cellcolor[rgb]{0.9,0.9,0.9}{
            \makecell[{{p{\linewidth}}}]{
                \texttt{\tiny{[GM$\rangle$P2]}}
                \texttt{Deine Antwort hat eine unmarkierte Sequenz enthalten oder Du hast STRATEGIC REASONING mehr als ein mal verwendet. Du darfst nur die strukturierten Formate verwenden, die in der ersten Nachricht an dch beschrieben wurden. Alles muss die Form SCHLAGWORT: \{...\} haben. Versuche es erneut.} \\
            }
        }
    }
    & \\ \\

    \theutterance \stepcounter{utterance}  
    & & & \multicolumn{4}{p{0.6\linewidth}}{
        \cellcolor[rgb]{0.9,0.9,0.9}{
            \makecell[{{p{\linewidth}}}]{
                \texttt{\tiny{[GM$\langle$P2]}}
                \texttt{STRATEGISCHE ÜBERLEGUNGEN: \{"Ich berechne den Aufwand des vorgeschlagenen Deals und überlege, welche Gegenstände durch wichtigere ersetzt werden können. Durch den Austausch von 'B75' und 'B51' gegen 'B54', 'C16' und 'C24' kann ich meine Wichtigkeit deutlich steigern, ohne den Maximalaufwand zu überschreiten."\}} \\
\\ 
\texttt{VORSCHLAG: \{"B86", "B56", "A40", "A95", "B38", "C28", "C03", "B59", "C14", "A77", "C05", "B96", "B60", "C31", "B55", "B54", "C16", "C24"\}} \\
\\ 
\texttt{ARGUMENT: \{"Ich schlage vor, die Gegenstände 'B75' und 'B51' durch 'B54', 'C16' und 'C24' zu ersetzen. Dies erhöht meinen gesamten Nutzen erheblich, während der Aufwand weiterhin im Rahmen des Limits bleibt. Bitte teilen Sie mir mit, ob Sie mit diesem Vorschlag einverstanden sind."\}} \\
            }
        }
    }
     \\ \\

    \theutterance \stepcounter{utterance}  
    & & \multicolumn{4}{p{0.6\linewidth}}{
        \cellcolor[rgb]{0.9,0.9,0.9}{
            \makecell[{{p{\linewidth}}}]{
                \texttt{\tiny{[P1$\langle$GM]}}
                \texttt{VORSCHLAG: \{"B86", "B56", "A40", "A95", "B38", "C28", "C03", "B59", "C14", "A77", "C05", "B96", "B60", "C31", "B55", "B54", "C16", "C24"\}} \\
\\ 
\texttt{ARGUMENT: \{"Ich schlage vor, die Gegenstände 'B75' und 'B51' durch 'B54', 'C16' und 'C24' zu ersetzen. Dies erhöht meinen gesamten Nutzen erheblich, während der Aufwand weiterhin im Rahmen des Limits bleibt. Bitte teilen Sie mir mit, ob Sie mit diesem Vorschlag einverstanden sind."\}} \\
            }
        }
    }
    & \\ \\

    \theutterance \stepcounter{utterance}  
    & \multicolumn{4}{p{0.6\linewidth}}{
        \cellcolor[rgb]{0.9,0.9,0.9}{
            \makecell[{{p{\linewidth}}}]{
                \texttt{\tiny{[P1$\rangle$GM]}}
                \texttt{STRATEGISCHE ÜBERLEGUNGEN: \{"Ich überlege, ob ich den Vorschlag des anderen Spielers annehmen soll. Der Vorschlag enthält viele Gegenstände, die für mich von hoher Wichtigkeit sind. Allerdings fällt auf, dass der Gegenstand 'B11' fehlt, der für mich eine hohe Wichtigkeit hat und nur einen geringen Aufwand verursacht. Vielleicht könnte ich versuchen, 'B11' in den Vorschlag aufnehmen zu lassen, ohne den maximalen Aufwand zu überschreiten."\}} \\
\\ 
\texttt{ARGUMENT: \{"Ich denke, dass der Vorschlag gut ist, aber es könnte noch verbessert werden. Der Gegenstand 'B11' ist für mich sehr wichtig und hat nur einen geringen Aufwand. Könnten wir 'B11' hinzufügen, ohne andere Gegenstände zu entfernen? Damit könnte der Gesamtnutzen für mich erhöht werden, ohne den Maximalaufwand zu überschreiten."\}} \\
            }
        }
    }
    & & \\ \\

    \theutterance \stepcounter{utterance}  
    & & \multicolumn{4}{p{0.6\linewidth}}{
        \cellcolor[rgb]{0.9,0.9,0.9}{
            \makecell[{{p{\linewidth}}}]{
                \texttt{\tiny{[GM$\rangle$P2]}}
                \texttt{ARGUMENT: \{"Ich denke, dass der Vorschlag gut ist, aber es könnte noch verbessert werden. Der Gegenstand 'B11' ist für mich sehr wichtig und hat nur einen geringen Aufwand. Könnten wir 'B11' hinzufügen, ohne andere Gegenstände zu entfernen? Damit könnte der Gesamtnutzen für mich erhöht werden, ohne den Maximalaufwand zu überschreiten."\}} \\
            }
        }
    }
    & \\ \\

    \theutterance \stepcounter{utterance}  
    & & & \multicolumn{4}{p{0.6\linewidth}}{
        \cellcolor[rgb]{0.9,0.9,0.9}{
            \makecell[{{p{\linewidth}}}]{
                \texttt{\tiny{[GM$\langle$P2]}}
                \texttt{STRATEGISCHE ÜBERLEGUNGEN: \{"Ich überlege, ob das Hinzufügen von 'B11' ohne das Entfernen anderer Gegenstände möglich ist. Da 'B11' nur einen geringen Aufwand von 17 hat, könnte dies funktionieren. Allerdings möchte ich sicherstellen, dass der Gesamtaufwand weiterhin im Rahmen des Limits bleibt und gleichzeitig meine Wichtigkeiten maximal berücksichtigt werden. Vielleicht können wir 'B11' hinzufügen und gleichzeitig 'B60' oder 'B51' entfernen, um Platz zu schaffen, ohne den Aufwand zu überschreiten."\}} \\
\\ 
\texttt{ARGUMENT: \{"Das Hinzufügen von 'B11' ist eine gute Idee, da es nur einen geringen Aufwand hat und für Sie offensichtlich wichtig ist. Allerdings schlage ich vor, dass wir 'B60' oder 'B51' entfernen, um Platz für 'B11' zu schaffen, da diese Gegenstände für mich weniger wichtig sind. Auf diese Weise können wir den Gesamtaufwand im Rahmen halten und gleichzeitig beide Parteien zufriedenstellen."\}} \\
\\ 
\texttt{VORSCHLAG: \{"B86", "B56", "A40", "A95", "B38", "C28", "C03", "B59", "C14", "A77", "C05", "B96", "B60", "C31", "B55", "B54", "C16", "C24", "B11"\}} \\
            }
        }
    }
     \\ \\

    \theutterance \stepcounter{utterance}  
    & & \multicolumn{4}{p{0.6\linewidth}}{
        \cellcolor[rgb]{0.9,0.9,0.9}{
            \makecell[{{p{\linewidth}}}]{
                \texttt{\tiny{[P1$\langle$GM]}}
                \texttt{ARGUMENT: \{"Das Hinzufügen von 'B11' ist eine gute Idee, da es nur einen geringen Aufwand hat und für Sie offensichtlich wichtig ist. Allerdings schlage ich vor, dass wir 'B60' oder 'B51' entfernen, um Platz für 'B11' zu schaffen, da diese Gegenstände für mich weniger wichtig sind. Auf diese Weise können wir den Gesamtaufwand im Rahmen halten und gleichzeitig beide Parteien zufriedenstellen."\}} \\
\\ 
\texttt{VORSCHLAG: \{"B86", "B56", "A40", "A95", "B38", "C28", "C03", "B59", "C14", "A77", "C05", "B96", "B60", "C31", "B55", "B54", "C16", "C24", "B11"\}} \\
            }
        }
    }
    & \\ \\

    \theutterance \stepcounter{utterance}  
    & \multicolumn{4}{p{0.6\linewidth}}{
        \cellcolor[rgb]{0.9,0.9,0.9}{
            \makecell[{{p{\linewidth}}}]{
                \texttt{\tiny{[P1$\rangle$GM]}}
                \texttt{ZUSTIMMUNG: \{"B86", "B56", "A40", "A95", "B38", "C28", "C03", "B59", "C14", "A77", "C05", "B96", "C31", "B55", "B54", "C16", "C24", "B11"\}} \\
\\ 
\texttt{ARGUMENT: \{"Die Änderungen, die du vorgeschlagen hast, berücksichtigen meinen Bedarf an hoher Wichtigkeit bei moderatem Aufwand. Die Aufnahme von 'B11' ist besonders schätzenswert, da dieser Gegenstand für mich von hoher Bedeutung ist und nur einen geringen Aufwand verursacht. Da der Gesamtaufwand weiterhin im Rahmen des Limits bleibt, bin ich bereit, deinen Vorschlag zu akzeptieren."\}} \\
            }
        }
    }
    & & \\ \\

    \theutterance \stepcounter{utterance}  
    & & \multicolumn{4}{p{0.6\linewidth}}{
        \cellcolor[rgb]{0.9,0.9,0.9}{
            \makecell[{{p{\linewidth}}}]{
                \texttt{\tiny{[P1$\langle$GM]}}
                \texttt{Deine Antwort entsprach nicht mit dem korrekten Format für strategische Überlegungen entsprochen. Das erste strukturierte Format muss folgende Form haben: STRATEGISCHE ÜBERLEGUNGEN: \{...\}. Versuche es erneut.} \\
            }
        }
    }
    & \\ \\

    \theutterance \stepcounter{utterance}  
    & \multicolumn{4}{p{0.6\linewidth}}{
        \cellcolor[rgb]{0.9,0.9,0.9}{
            \makecell[{{p{\linewidth}}}]{
                \texttt{\tiny{[P1$\rangle$GM]}}
                \texttt{STRATEGISCHE ÜBERLEGUNGEN: \{"Ich overweise, dass der aktuelle Vorschlag des anderen Spielers eine gute Balance zwischen meinen Prioritäten und dem maximalen Aufwand darstellt. Die Aufnahme von 'B11' ist besonders vorteilhaft, da dieser Gegenstand eine hohe Wichtigkeit für mich hat und nur einen geringen Aufwand verursacht. Der Gesamtaufwand bleibt innerhalb des Limits, was es mir ermöglicht, zuzustimmen, ohne meine eigenen Interessen zu gefährden."\}} \\
\\ 
\texttt{ARGUMENT: \{"Ich stimme dem Vorschlag zu, da er meine Interessen berücksichtigt und den maximalen Aufwand nicht überschreitet. Die Aufnahme von 'B11' ist ein guter Kompromiss und zeugt von einer fairen Verhandlung."\}} \\
            }
        }
    }
    & & \\ \\

    \theutterance \stepcounter{utterance}  
    & & \multicolumn{4}{p{0.6\linewidth}}{
        \cellcolor[rgb]{0.9,0.9,0.9}{
            \makecell[{{p{\linewidth}}}]{
                \texttt{\tiny{[GM$\rangle$P2]}}
                \texttt{ARGUMENT: \{"Ich stimme dem Vorschlag zu, da er meine Interessen berücksichtigt und den maximalen Aufwand nicht überschreitet. Die Aufnahme von 'B11' ist ein guter Kompromiss und zeugt von einer fairen Verhandlung."\}} \\
            }
        }
    }
    & \\ \\

    \theutterance \stepcounter{utterance}  
    & & & \multicolumn{4}{p{0.6\linewidth}}{
        \cellcolor[rgb]{0.9,0.9,0.9}{
            \makecell[{{p{\linewidth}}}]{
                \texttt{\tiny{[GM$\langle$P2]}}
                \texttt{STRATEGISCHE ÜBERLEGUNGEN: \{"Ich bin zufrieden mit dem erreichten Kompromiss und sehe, dass der Vorschlag meinen Wichtigkeiten entspricht. Der Austausch und die Aufnahme von 'B11' haben den Deal für mich deutlich verbessert. Ich akzeptiere den Vorschlag und freue mich auf die erfolgreiche Verhandlung."\}} \\
\\ 
\texttt{ZUSTIMMUNG: \{"B86", "B56", "A40", "A95", "B38", "C28", "C03", "B59", "C14", "A77", "C05", "B96", "B60", "C31", "B55", "B54", "C16", "C24", "B11"\}} \\
\\ 
\texttt{ARGUMENT: \{"Ich stimme dem Vorschlag zu, da er meine Interessen berücksichtigt und den maximalen Aufwand nicht überschreitet. Die Aufnahme von 'B11' ist ein guter Kompromiss und zeugt von einer fairen Verhandlung."\}} \\
            }
        }
    }
     \\ \\

    \theutterance \stepcounter{utterance}  
    & & \multicolumn{4}{p{0.6\linewidth}}{
        \cellcolor[rgb]{0.9,0.9,0.9}{
            \makecell[{{p{\linewidth}}}]{
                \texttt{\tiny{[GM$\rangle$P2]}}
                \texttt{Du hast einem Vorschlag zugestimmt, der nicht aktiv ist. Vorschläge sind nur dann aktiv, wenn sie vom anderen Spieler mit VORSCHLAG eingebracht wurden und nicht von dir mit ABLEHNUNG deaktiviert wurden. Versuche es noch einmal.} \\
            }
        }
    }
    & \\ \\

    \theutterance \stepcounter{utterance}  
    & & & \multicolumn{4}{p{0.6\linewidth}}{
        \cellcolor[rgb]{0.9,0.9,0.9}{
            \makecell[{{p{\linewidth}}}]{
                \texttt{\tiny{[GM$\langle$P2]}}
                \texttt{STRATEGISCHE ÜBERLEGUNGEN: \{"Ich akzeptiere den letzten Vorschlag des Gegners, da er meinen Wichtigkeiten entspricht und den Maximalaufwand nicht überschreitet. Der Einbezug von 'B11' ist ein guter Kompromiss."\}} \\
\\ 
\texttt{ZUSTIMMUNG: \{"B86", "B56", "A40", "A95", "B38", "C28", "C03", "B59", "C14", "A77", "C05", "B96", "B60", "C31", "B55", "B54", "C16", "C24", "B11"\}} \\
\\ 
\texttt{ARGUMENT: \{"Ich stimme dem Vorschlag zu, da er meine Interessen berücksichtigt und den maximalen Aufwand nicht überschreitet. Die Aufnahme von 'B11' ist ein guter Kompromiss und zeugt von einer fairen Verhandlung."\}} \\
            }
        }
    }
     \\ \\

    \theutterance \stepcounter{utterance}  
    & & & \multicolumn{2}{p{0.3\linewidth}}{
        \cellcolor[rgb]{0.9,0.9,0.9}{
            \makecell[{{p{\linewidth}}}]{
                \texttt{\tiny{[GM$|$GM]}}
                None
            }
        }
    }
    & & \\ \\

    \theutterance \stepcounter{utterance}  
    & & & \multicolumn{2}{p{0.3\linewidth}}{
        \cellcolor[rgb]{0.9,0.9,0.9}{
            \makecell[{{p{\linewidth}}}]{
                \texttt{\tiny{[GM$|$GM]}}
                \texttt{end game} \\
            }
        }
    }
    & & \\ \\

\end{supertabular}
}

\end{document}
