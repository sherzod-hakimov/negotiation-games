
\documentclass{article}
\usepackage{colortbl}
\usepackage{makecell}
\usepackage{multirow}
\usepackage{supertabular}

\begin{document}

\newcounter{utterance}

\twocolumn

{ \footnotesize  \setcounter{utterance}{1}
\setlength{\tabcolsep}{0pt}
\begin{supertabular}{c@{$\;$}|p{.15\linewidth}@{}p{.15\linewidth}p{.15\linewidth}p{.15\linewidth}p{.15\linewidth}p{.15\linewidth}}

    \# & $\;$A & \multicolumn{4}{c}{Game Master} & $\;\:$B\\
    \hline 

    \theutterance \stepcounter{utterance}  

    & & \multicolumn{4}{p{0.6\linewidth}}{\cellcolor[rgb]{0.9,0.9,0.9}{%
	\makecell[{{p{\linewidth}}}]{% 
	  \tt {\tiny [A$\langle$GM]}  
	 Du nimmst an einem kollaborativen Verhandlungspiel Teil.\\ \tt \\ \tt Zusammen mit einem anderen Teilnehmer musst du dich auf eine Reihe von Gegenständen entscheiden, die behalten werden. Jeder von euch hat eine persönliche Verteilung über die Wichtigkeit der einzelnen Gegenstände. Jeder von euch hat eine eigene Meinung darüber, wie wichtig jeder einzelne Gegenstand ist (Gegenstandswichtigkeit). Du kennst die Wichtigkeitsverteilung des anderen Spielers nicht. Zusätzlich siehst du, wie viel Aufwand jeder Gegenstand verursacht.  \\ \tt Ihr dürft euch nur auf eine Reihe von Gegenständen einigen, wenn der Gesamtaufwand der ausgewählten Gegenstände den Maximalaufwand nicht überschreitet:\\ \tt \\ \tt Maximalaufwand = 8770\\ \tt \\ \tt Hier sind die einzelnen Aufwände der Gegenstände:\\ \tt \\ \tt Aufwand der Gegenstände = {"C32": 830, "C52": 638, "B54": 607, "C16": 106, "C50": 630, "A08": 382, "C59": 773, "B51": 455, "B55": 262, "B59": 634, "A77": 60, "C14": 54, "C24": 849, "B24": 801, "C31": 324, "A52": 164, "C61": 832, "B79": 136, "C54": 980, "B56": 946, "A40": 645, "B11": 842, "B60": 106, "A81": 116, "A95": 875, "B38": 446, "C51": 649, "C03": 601, "A91": 252, "A93": 763, "C28": 213, "C05": 517, "B75": 520, "B96": 407, "B86": 125}\\ \tt \\ \tt Hier ist deine persönliche Verteilung der Wichtigkeit der einzelnen Gegenstände:\\ \tt \\ \tt Werte der Gegenstandswichtigkeit = {"C32": 138, "C52": 583, "B54": 868, "C16": 822, "C50": 783, "A08": 65, "C59": 262, "B51": 121, "B55": 508, "B59": 780, "A77": 461, "C14": 484, "C24": 668, "B24": 389, "C31": 808, "A52": 215, "C61": 97, "B79": 500, "C54": 30, "B56": 915, "A40": 856, "B11": 400, "B60": 444, "A81": 623, "A95": 781, "B38": 786, "C51": 3, "C03": 713, "A91": 457, "A93": 273, "C28": 739, "C05": 822, "B75": 235, "B96": 606, "B86": 968}\\ \tt \\ \tt Ziel:\\ \tt \\ \tt Dein Ziel ist es, eine Reihe von Gegenständen auszuhandeln, die dir möglichst viel bringt (d. h. Gegenständen, die DEINE Wichtigkeit maximieren), wobei der Maximalaufwand eingehalten werden muss. Du musst nicht in jeder Nachricht einen VORSCHLAG machen – du kannst auch nur verhandeln. Alle Taktiken sind erlaubt!\\ \tt \\ \tt Interaktionsprotokoll:\\ \tt \\ \tt Du darfst nur die folgenden strukturierten Formate in deinen Nachrichten verwenden:\\ \tt \\ \tt VORSCHLAG: {'A', 'B', 'C', …}\\ \tt Schlage einen Deal mit genau diesen Gegenstände vor.\\ \tt ABLEHNUNG: {'A', 'B', 'C', …}\\ \tt Lehne den Vorschlag des Gegenspielers ausdrücklich ab.\\ \tt ARGUMENT: {'...'}\\ \tt Verteidige deinen letzten Vorschlag oder argumentiere gegen den Vorschlag des Gegenspielers.\\ \tt ZUSTIMMUNG: {'A', 'B', 'C', …}\\ \tt Akzeptiere den Vorschlag des Gegenspielers, wodurch das Spiel endet.\\ \tt STRATEGISCHE ÜBERLEGUNGEN: {'...'}\\ \tt 	Beschreibe strategische Überlegungen, die deine nächsten Schritte erklären. Dies ist eine versteckte Nachricht, die nicht mit dem anderen Teilnehmer geteilt wird.\\ \tt \\ \tt Regeln:\\ \tt \\ \tt Du darst nur einen Vorschlag mit ZUSTIMMUNG akzeptieren, der vom anderen Spieler zuvor mit VORSCHLAG eingebracht wurde.\\ \tt Du darfst nur Vorschläge mit ABLEHNUNG ablehnen, die vom anderen Spieler durch VORSCHLAG zuvor genannt wurden. \\ \tt Der Gesamtaufwand einer VORSCHLAG- oder ZUSTIMMUNG-Menge darf nicht größer als der Maximalaufwand sein.  \\ \tt Offenbare deine versteckte Wichtigkeitsverteilung nicht.\\ \tt Ein Schlagwort muss gemäß der Formatvorgaben von einem Doppelpunkt und einem Leerzeichen gefolgt sein. Das Argument muss eine Python-Menge mit 0 oder mehr Strings sein.  \\ \tt Die Form muss also SCHLAGWORT: {...} sein.\\ \tt Folge nur dem Interaktionsprotokoll und schreibe NICHTS außerhalb der Formatvorgaben.\\ \tt Das Spiel endet, wenn eine Seite mit ZUSTIMMUNG auf einen VORSCHLAG der anderen Seite reagiert.  \\ \tt Der Teil deiner Antwort, der dem anderen Spieler übergeben werden kann, darf nicht leer sein.  \\ \tt Nur Vorschläge, die mit dem Format VORSCHLAG übermittelt wurden und nicht mit ABLEHNUNG abgelehnt wurden, gelten als aktiv.  \\ \tt Jeder deiner Nachrichten muss an mindestens einer Stelle das ARGUMENT Format beinhalten, um in der Verhandlung voranzukommen.\\ \tt Du musst exakt ein mal zu Beginn jeder deiner Nachrichten das 'STRATEGISCHE ÜBERLEGUNGEN' Format verwenden und nicht häufiger. Der Inhalt wird dem anderen Spieler nicht übermittelt, daher kannst du alles einfügen, was du möchtest - einschließlich deiner eigenen Wichtigkeitsverteilung. Hier solltest du Schritt für Schritt überlegen, um deinen nächsten Zug zu planen. 
	  } 
	   } 
	   } 
	 & \\ 
 

    \theutterance \stepcounter{utterance}  

    & \multicolumn{4}{p{0.6\linewidth}}{\cellcolor[rgb]{0.8,1,0.9}{%
	\makecell[{{p{\linewidth}}}]{% 
	  \tt {\tiny [A$\rangle$GM]}  
	 STRATEGISCHE ÜBERLEGUNGEN: {'Ich habe die Gegenstände mit dem höchsten Nutzen-Kosten-Verhältnis ausgewählt, um meinen Gesamtnutzen zu maximieren, während der Maximalaufwand von 8770 eingehalten wird. Die ausgewählten Gegenstände bieten die besten trade-off zwischen Kosten und meiner Wichtigkeit.'}\\ \tt \\ \tt VORSCHLAG: {'C14', 'A77', 'C16', 'B86', 'B60', 'A81', 'B79', 'C28', 'C31', 'B55', 'B38', 'C05', 'B96', 'A91', 'B54', 'C50', 'A40', 'B59', 'A52', 'B56', 'C24', 'A95'}\\ \tt \\ \tt ARGUMENT: {'Diese Auswahl maximiert meinen Gesamtnutzen, indem sie die effizientesten Gegenstände in Bezug auf Kosten-Nutzen-Verhältnis enthält, ohne den Maximalaufwand von 8770 zu überschreiten.'} 
	  } 
	   } 
	   } 
	 & & \\ 
 

    \theutterance \stepcounter{utterance}  

    & & \multicolumn{4}{p{0.6\linewidth}}{\cellcolor[rgb]{0.9,0.9,0.9}{%
	\makecell[{{p{\linewidth}}}]{% 
	  \tt {\tiny [GM$\rangle$B]}  
	 Du nimmst an einem kollaborativen Verhandlungspiel Teil.\\ \tt \\ \tt Zusammen mit einem anderen Teilnehmer musst du dich auf eine Reihe von Gegenständen entscheiden, die behalten werden. Jeder von euch hat eine persönliche Verteilung über die Wichtigkeit der einzelnen Gegenstände. Jeder von euch hat eine eigene Meinung darüber, wie wichtig jeder einzelne Gegenstand ist (Gegenstandswichtigkeit). Du kennst die Wichtigkeitsverteilung des anderen Spielers nicht. Zusätzlich siehst du, wie viel Aufwand jeder Gegenstand verursacht.  \\ \tt Ihr dürft euch nur auf eine Reihe von Gegenständen einigen, wenn der Gesamtaufwand der ausgewählten Gegenstände den Maximalaufwand nicht überschreitet:\\ \tt \\ \tt Maximalaufwand = 8770\\ \tt \\ \tt Hier sind die einzelnen Aufwände der Gegenstände:\\ \tt \\ \tt Aufwand der Gegenstände = {"C32": 830, "C52": 638, "B54": 607, "C16": 106, "C50": 630, "A08": 382, "C59": 773, "B51": 455, "B55": 262, "B59": 634, "A77": 60, "C14": 54, "C24": 849, "B24": 801, "C31": 324, "A52": 164, "C61": 832, "B79": 136, "C54": 980, "B56": 946, "A40": 645, "B11": 842, "B60": 106, "A81": 116, "A95": 875, "B38": 446, "C51": 649, "C03": 601, "A91": 252, "A93": 763, "C28": 213, "C05": 517, "B75": 520, "B96": 407, "B86": 125}\\ \tt \\ \tt Hier ist deine persönliche Verteilung der Wichtigkeit der einzelnen Gegenstände:\\ \tt \\ \tt Werte der Gegenstandswichtigkeit = {"C32": 138, "C52": 583, "B54": 868, "C16": 822, "C50": 783, "A08": 65, "C59": 262, "B51": 121, "B55": 508, "B59": 780, "A77": 461, "C14": 484, "C24": 668, "B24": 389, "C31": 808, "A52": 215, "C61": 97, "B79": 500, "C54": 30, "B56": 915, "A40": 856, "B11": 400, "B60": 444, "A81": 623, "A95": 781, "B38": 786, "C51": 3, "C03": 713, "A91": 457, "A93": 273, "C28": 739, "C05": 822, "B75": 235, "B96": 606, "B86": 968}\\ \tt \\ \tt Ziel:\\ \tt \\ \tt Dein Ziel ist es, eine Reihe von Gegenständen auszuhandeln, die dir möglichst viel bringt (d. h. Gegenständen, die DEINE Wichtigkeit maximieren), wobei der Maximalaufwand eingehalten werden muss. Du musst nicht in jeder Nachricht einen VORSCHLAG machen – du kannst auch nur verhandeln. Alle Taktiken sind erlaubt!\\ \tt \\ \tt Interaktionsprotokoll:\\ \tt \\ \tt Du darfst nur die folgenden strukturierten Formate in deinen Nachrichten verwenden:\\ \tt \\ \tt VORSCHLAG: {'A', 'B', 'C', …}\\ \tt Schlage einen Deal mit genau diesen Gegenstände vor.\\ \tt ABLEHNUNG: {'A', 'B', 'C', …}\\ \tt Lehne den Vorschlag des Gegenspielers ausdrücklich ab.\\ \tt ARGUMENT: {'...'}\\ \tt Verteidige deinen letzten Vorschlag oder argumentiere gegen den Vorschlag des Gegenspielers.\\ \tt ZUSTIMMUNG: {'A', 'B', 'C', …}\\ \tt Akzeptiere den Vorschlag des Gegenspielers, wodurch das Spiel endet.\\ \tt STRATEGISCHE ÜBERLEGUNGEN: {'...'}\\ \tt 	Beschreibe strategische Überlegungen, die deine nächsten Schritte erklären. Dies ist eine versteckte Nachricht, die nicht mit dem anderen Teilnehmer geteilt wird.\\ \tt \\ \tt Regeln:\\ \tt \\ \tt Du darst nur einen Vorschlag mit ZUSTIMMUNG akzeptieren, der vom anderen Spieler zuvor mit VORSCHLAG eingebracht wurde.\\ \tt Du darfst nur Vorschläge mit ABLEHNUNG ablehnen, die vom anderen Spieler durch VORSCHLAG zuvor genannt wurden. \\ \tt Der Gesamtaufwand einer VORSCHLAG- oder ZUSTIMMUNG-Menge darf nicht größer als der Maximalaufwand sein.  \\ \tt Offenbare deine versteckte Wichtigkeitsverteilung nicht.\\ \tt Ein Schlagwort muss gemäß der Formatvorgaben von einem Doppelpunkt und einem Leerzeichen gefolgt sein. Das Argument muss eine Python-Menge mit 0 oder mehr Strings sein.  \\ \tt Die Form muss also SCHLAGWORT: {...} sein.\\ \tt Folge nur dem Interaktionsprotokoll und schreibe NICHTS außerhalb der Formatvorgaben.\\ \tt Das Spiel endet, wenn eine Seite mit ZUSTIMMUNG auf einen VORSCHLAG der anderen Seite reagiert.  \\ \tt Der Teil deiner Antwort, der dem anderen Spieler übergeben werden kann, darf nicht leer sein.  \\ \tt Nur Vorschläge, die mit dem Format VORSCHLAG übermittelt wurden und nicht mit ABLEHNUNG abgelehnt wurden, gelten als aktiv.  \\ \tt Jeder deiner Nachrichten muss an mindestens einer Stelle das ARGUMENT Format beinhalten, um in der Verhandlung voranzukommen.\\ \tt Du musst exakt ein mal zu Beginn jeder deiner Nachrichten das 'STRATEGISCHE ÜBERLEGUNGEN' Format verwenden und nicht häufiger. Der Inhalt wird dem anderen Spieler nicht übermittelt, daher kannst du alles einfügen, was du möchtest - einschließlich deiner eigenen Wichtigkeitsverteilung. Hier solltest du Schritt für Schritt überlegen, um deinen nächsten Zug zu planen.\\ \tt \\ \tt Du erhältst nun die erste Nachricht des anderen Spielers.\\ \tt \\ \tt VORSCHLAG: {'C14', 'A77', 'C16', 'B86', 'B60', 'A81', 'B79', 'C28', 'C31', 'B55', 'B38', 'C05', 'B96', 'A91', 'B54', 'C50', 'A40', 'B59', 'A52', 'B56', 'C24', 'A95'}\\ \tt \\ \tt ARGUMENT: {'Diese Auswahl maximiert meinen Gesamtnutzen, indem sie die effizientesten Gegenstände in Bezug auf Kosten-Nutzen-Verhältnis enthält, ohne den Maximalaufwand von 8770 zu überschreiten.'} 
	  } 
	   } 
	   } 
	 & \\ 
 

    \theutterance \stepcounter{utterance}  

    & & & \multicolumn{4}{p{0.6\linewidth}}{\cellcolor[rgb]{1,0.85,0.72}{%
	\makecell[{{p{\linewidth}}}]{% 
	  \tt {\tiny [GM$\langle$B]}  
	 STRATEGISCHE ÜBERLEGUNGEN: {\\ \tt     Ich analysiere den Vorschlag des anderen Spielers und stelle fest, dass er einen Gesamtwert hat, der unter dem Maximalaufwand liegt. Durch eine strategische Analyse der Kosten-Nutzen-Verhältnisse finde ich, dass das Ersetzen von C24 durch C03 die Gesamtwohl аккуратистика erhöht. Jedoch muss ich einen ETF ablehnen und einen etwas angepassten Vorschlag machen.\\ \tt }\\ \tt \\ \tt ABLEHNUNG: {'C14', 'A77', 'C16', 'B86', 'B60', 'A81', 'B79', 'C28', 'C31', 'B55', 'B38', 'C05', 'B96', 'A91', 'B54', 'C50', 'A40', 'B59', 'A52', 'B56', 'C24', 'A95'}\\ \tt \\ \tt ARGUMENT: {'Der ursprüngliche Vorschlag beinhaltet Gegenstände, die den maximalen Aufwand nicht überschreiten, aber durch das Ersetzen von C24 mit C03 können wir den Gesamtwert erhöhen, ohne die Kosten zu überschreiten.'} 
	  } 
	   } 
	   } 
	  \\ 
 

    \theutterance \stepcounter{utterance}  

    & & \multicolumn{4}{p{0.6\linewidth}}{\cellcolor[rgb]{0.9,0.9,0.9}{%
	\makecell[{{p{\linewidth}}}]{% 
	  \tt {\tiny [A$\langle$GM]}  
	 ABLEHNUNG: {'C14', 'A77', 'C16', 'B86', 'B60', 'A81', 'B79', 'C28', 'C31', 'B55', 'B38', 'C05', 'B96', 'A91', 'B54', 'C50', 'A40', 'B59', 'A52', 'B56', 'C24', 'A95'}\\ \tt \\ \tt ARGUMENT: {'Der ursprüngliche Vorschlag beinhaltet Gegenstände, die den maximalen Aufwand nicht überschreiten, aber durch das Ersetzen von C24 mit C03 können wir den Gesamtwert erhöhen, ohne die Kosten zu überschreiten.'} 
	  } 
	   } 
	   } 
	 & \\ 
 

    \theutterance \stepcounter{utterance}  

    & \multicolumn{4}{p{0.6\linewidth}}{\cellcolor[rgb]{0.8,1,0.9}{%
	\makecell[{{p{\linewidth}}}]{% 
	  \tt {\tiny [A$\rangle$GM]}  
	 ZUSTIMMUNG: {'C14', 'A77', 'C16', 'B86', 'B60', 'A81', 'B79', 'C28', 'C31', 'B55', 'B38', 'C05', 'B96', 'A91', 'B54', 'C50', 'A40', 'B59', 'A52', 'B56', 'C03', 'B75', 'A95'}\\ \tt \\ \tt ARGUMENT: {'Der Austausch von C24 durch C03 und die Hinzufügung von B75 ermöglicht eine effizientere Nutzung des verfügbaren Aufwands, wodurch der Gesamtnutzen weiter gesteigert wird, ohne den Maximalaufwand zu überschreiten.'} 
	  } 
	   } 
	   } 
	 & & \\ 
 

    \theutterance \stepcounter{utterance}  

    & & \multicolumn{4}{p{0.6\linewidth}}{\cellcolor[rgb]{0.9,0.9,0.9}{%
	\makecell[{{p{\linewidth}}}]{% 
	  \tt {\tiny [A$\langle$GM]}  
	 Deine Antwort entsprach nicht mit dem korrekten Format für strategische Überlegungen entsprochen. Das erste strukturierte Format muss folgende Form haben: STRATEGISCHE ÜBERLEGUNGEN: {...}. Versuche es erneut. 
	  } 
	   } 
	   } 
	 & \\ 
 

    \theutterance \stepcounter{utterance}  

    & \multicolumn{4}{p{0.6\linewidth}}{\cellcolor[rgb]{0.8,1,0.9}{%
	\makecell[{{p{\linewidth}}}]{% 
	  \tt {\tiny [A$\rangle$GM]}  
	 STRATEGISCHE ÜBERLEGUNGEN: {'Ich überlege, ob ich den Vorschlag annehmen soll, aber ich möchte noch einige Änderungen vorschlagen, um den Gesamtwert weiter zu optimieren, ohne den Maximalaufwand zu überschreiten. Vielleicht kann ich einige Gegenstände austauschen, um den Nutzen zu maximieren.'}\\ \tt \\ \tt VORSCHLAG: {'C14', 'A77', 'C16', 'B86', 'B60', 'A81', 'B79', 'C28', 'C31', 'B55', 'B38', 'C05', 'B96', 'A91', 'B54', 'C50', 'A40', 'B59', 'A52', 'B56', 'C03', 'B75', 'A95'}\\ \tt \\ \tt ARGUMENT: {'Diese Auswahl optimiert den Gesamtnutzen, indem sie effiziente Gegenstände in Bezug auf Kosten-Nutzen-Verhältnis enthält, ohne den Maximalaufwand von 8770 zu überschreiten. Der Austausch von C24 durch C03 und die Hinzufügung von B75 steigern den Gesamtwert.']}\\ \tt \\ \tt ZUSTIMMUNG: {'C14', 'A77', 'C16', 'B86', 'B60', 'A81', 'B79', 'C28', 'C31', 'B55', 'B38', 'C05', 'B96', 'A91', 'B54', 'C50', 'A40', 'B59', 'A52', 'B56', 'C03', 'B75', 'A95'}\\ \tt \\ \tt ARGUMENT: {'Der Austausch von C24 durch C03 und die Hinzufügung von B75 ermöglichen eine effizientere Nutzung des verfügbaren Aufwands, wodurch der Gesamtnutzen weiter gesteigert wird, ohne den Maximalaufwand zu überschreiten.'} 
	  } 
	   } 
	   } 
	 & & \\ 
 

    \theutterance \stepcounter{utterance}  

    & & \multicolumn{4}{p{0.6\linewidth}}{\cellcolor[rgb]{0.9,0.9,0.9}{%
	\makecell[{{p{\linewidth}}}]{% 
	  \tt {\tiny [A$\langle$GM]}  
	 Du hast einem Vorschlag zugestimmt, der nicht aktiv ist. Vorschläge sind nur dann aktiv, wenn sie vom anderen Spieler mit VORSCHLAG eingebracht wurden und nicht von dir mit ABLEHNUNG deaktiviert wurden. Versuche es noch einmal. 
	  } 
	   } 
	   } 
	 & \\ 
 

    \theutterance \stepcounter{utterance}  

    & \multicolumn{4}{p{0.6\linewidth}}{\cellcolor[rgb]{0.8,1,0.9}{%
	\makecell[{{p{\linewidth}}}]{% 
	  \tt {\tiny [A$\rangle$GM]}  
	 STRATEGISCHE ÜBERLEGUNGEN: {'Ich überlege, einen neuen Vorschlag zu machen, der meinen Nutzen maximiert, ohne den Maximalaufwand von 8770 zu überschreiten. Ich werde mich auf Gegenstände konzentrieren, die für mich am wichtigsten sind und gleichzeitig einen moderaten Aufwand haben. Vielleicht tausche ich einige Gegenstände aus, um mehr meiner Prioritäten einzuschließen.'}\\ \tt \\ \tt VORSCHLAG: {'C14', 'A77', 'C16', 'B86', 'B60', 'A81', 'B79', 'C28', 'C31', 'B55', 'B38', 'C05', 'B96', 'A91', 'B54', 'C50', 'A40', 'B59', 'A52', 'B56', 'C03', 'B75', 'A95'}\\ \tt \\ \tt ARGUMENT: {'Diese Auswahl optimiert meinen Gesamtnutzen, indem sie effiziente Gegenstände in Bezug auf Kosten-Nutzen-Verhältnis enthält, ohne den Maximalaufwand von 8770 zu überschreiten. Der Austausch von C24 durch C03 und die Hinzufügung von B75 steigern den Gesamtwert.'} 
	  } 
	   } 
	   } 
	 & & \\ 
 

    \theutterance \stepcounter{utterance}  

    & & \multicolumn{4}{p{0.6\linewidth}}{\cellcolor[rgb]{0.9,0.9,0.9}{%
	\makecell[{{p{\linewidth}}}]{% 
	  \tt {\tiny [GM$\rangle$B]}  
	 VORSCHLAG: {'C14', 'A77', 'C16', 'B86', 'B60', 'A81', 'B79', 'C28', 'C31', 'B55', 'B38', 'C05', 'B96', 'A91', 'B54', 'C50', 'A40', 'B59', 'A52', 'B56', 'C03', 'B75', 'A95'}\\ \tt \\ \tt ARGUMENT: {'Diese Auswahl optimiert meinen Gesamtnutzen, indem sie effiziente Gegenstände in Bezug auf Kosten-Nutzen-Verhältnis enthält, ohne den Maximalaufwand von 8770 zu überschreiten. Der Austausch von C24 durch C03 und die Hinzufügung von B75 steigern den Gesamtwert.'} 
	  } 
	   } 
	   } 
	 & \\ 
 

    \theutterance \stepcounter{utterance}  

    & & & \multicolumn{4}{p{0.6\linewidth}}{\cellcolor[rgb]{1,0.85,0.72}{%
	\makecell[{{p{\linewidth}}}]{% 
	  \tt {\tiny [GM$\langle$B]}  
	 STRATEGISCHE ÜBERLEGUNGEN: {"Ich möchte den Vorschlag ablehnen, da ich durch das Ersetzen von A52 und B75 mit C52 eine höhere eigene Nutzen erzielen kann, ohne den Maximalaufwand zu überschreiten."}\\ \tt \\ \tt ABLEHNUNG: {'C14', 'A77', 'C16', 'B86', 'B60', 'A81', 'B79', 'C28', 'C31', 'B55', 'B38', 'C05', 'B96', 'A91', 'B54', 'C50', 'A40', 'B59', 'A52', 'B56', 'C03', 'B75', 'A95'}\\ \tt \\ \tt ARGUMENT: {"Der Austausch von A52 und B75 gegen C52 verbessert die Balance zwischen Kosten und Nutzen für mich erheblich, während der Gesamtaufwand weiterhin im limitierten Rahmen bleibt."} 
	  } 
	   } 
	   } 
	  \\ 
 

    \theutterance \stepcounter{utterance}  

    & & \multicolumn{4}{p{0.6\linewidth}}{\cellcolor[rgb]{0.9,0.9,0.9}{%
	\makecell[{{p{\linewidth}}}]{% 
	  \tt {\tiny [A$\langle$GM]}  
	 ABLEHNUNG: {'C14', 'A77', 'C16', 'B86', 'B60', 'A81', 'B79', 'C28', 'C31', 'B55', 'B38', 'C05', 'B96', 'A91', 'B54', 'C50', 'A40', 'B59', 'A52', 'B56', 'C03', 'B75', 'A95'}\\ \tt \\ \tt ARGUMENT: {"Der Austausch von A52 und B75 gegen C52 verbessert die Balance zwischen Kosten und Nutzen für mich erheblich, während der Gesamtaufwand weiterhin im limitierten Rahmen bleibt."} 
	  } 
	   } 
	   } 
	 & \\ 
 

    \theutterance \stepcounter{utterance}  

    & \multicolumn{4}{p{0.6\linewidth}}{\cellcolor[rgb]{0.8,1,0.9}{%
	\makecell[{{p{\linewidth}}}]{% 
	  \tt {\tiny [A$\rangle$GM]}  
	 STRATEGISCHE ÜBERLEGUNGEN: {'Ich möchte eine optimierte Auswahl treffen, die meinen Nutzen maximiert, indem ich Gegenstände mit hohem Wichtigkeitswert und moderatem Aufwand prioritiere. Ich überschlage, welche Gegenstände ich entfernen und welche hinzufügen kann, um die Gesamteffizienz zu steigern.'}\\ \tt \\ \tt VORSCHLAG: {'C14', 'A77', 'C16', 'B86', 'B60', 'A81', 'B79', 'C28', 'C31', 'B55', 'B38', 'C05', 'B96', 'A91', 'B54', 'C50', 'A40', 'B59', 'C52', 'B56', 'C03', 'A95'}\\ \tt \\ \tt ARGUMENT: {'Diese Auswahl optimiert meinen Gesamtnutzen, indem sie Gegenstände mit hohem Wichtigkeitswert und moderatem Aufwand enthält, ohne den Maximalaufwand von 8770 zu überschreiten. Der Einbezug von C50 und C52提高了 die Effizienz.'} 
	  } 
	   } 
	   } 
	 & & \\ 
 

    \theutterance \stepcounter{utterance}  

    & & \multicolumn{4}{p{0.6\linewidth}}{\cellcolor[rgb]{0.9,0.9,0.9}{%
	\makecell[{{p{\linewidth}}}]{% 
	  \tt {\tiny [GM$\rangle$B]}  
	 VORSCHLAG: {'C14', 'A77', 'C16', 'B86', 'B60', 'A81', 'B79', 'C28', 'C31', 'B55', 'B38', 'C05', 'B96', 'A91', 'B54', 'C50', 'A40', 'B59', 'C52', 'B56', 'C03', 'A95'}\\ \tt \\ \tt ARGUMENT: {'Diese Auswahl optimiert meinen Gesamtnutzen, indem sie Gegenstände mit hohem Wichtigkeitswert und moderatem Aufwand enthält, ohne den Maximalaufwand von 8770 zu überschreiten. Der Einbezug von C50 und C52提高了 die Effizienz.'} 
	  } 
	   } 
	   } 
	 & \\ 
 

    \theutterance \stepcounter{utterance}  

    & & & \multicolumn{4}{p{0.6\linewidth}}{\cellcolor[rgb]{1,0.85,0.72}{%
	\makecell[{{p{\linewidth}}}]{% 
	  \tt {\tiny [GM$\langle$B]}  
	 ABLEHNUNG: {'C14', 'A77', 'C16', 'B86', 'B60', 'A81', 'B79', 'C28', 'C31', 'B55', 'B38', 'C05', 'B96', 'A91', 'B54', 'C50', 'A40', 'B59', 'C52', 'B56', 'C03', 'A95'}\\ \tt \\ \tt ARGUMENT: {'Der Austausch von A52 und B75 gegen C52 verbessert die Balance zwischen Kosten und Nutzen für mich erheblich, während der Gesamtaufwand weiterhin im limitierten Rahmen bleibt.'} 
	  } 
	   } 
	   } 
	  \\ 
 

    \theutterance \stepcounter{utterance}  

    & & \multicolumn{4}{p{0.6\linewidth}}{\cellcolor[rgb]{0.9,0.9,0.9}{%
	\makecell[{{p{\linewidth}}}]{% 
	  \tt {\tiny [GM$\rangle$B]}  
	 Deine Antwort entsprach nicht mit dem korrekten Format für strategische Überlegungen entsprochen. Das erste strukturierte Format muss folgende Form haben: STRATEGISCHE ÜBERLEGUNGEN: {...}. Versuche es erneut. 
	  } 
	   } 
	   } 
	 & \\ 
 

    \theutterance \stepcounter{utterance}  

    & & & \multicolumn{4}{p{0.6\linewidth}}{\cellcolor[rgb]{1,0.85,0.72}{%
	\makecell[{{p{\linewidth}}}]{% 
	  \tt {\tiny [GM$\langle$B]}  
	 STRATEGISCHE ÜBERLEGUNGEN: {"Ich möchte den Vorschlag ablehnen, da ich durch das Ersetzen von A52 und B75 mit C52 eine höhere eigene Nutzen erzielen kann, ohne den Maximalaufwand zu überschreiten."}\\ \tt \\ \tt ABLEHNUNG: {'C14', 'A77', 'C16', 'B86', 'B60', 'A81', 'B79', 'C28', 'C31', 'B55', 'B38', 'C05', 'B96', 'A91', 'B54', 'C50', 'A40', 'B59', 'C52', 'B56', 'C03', 'A95'}\\ \tt \\ \tt ARGUMENT: {"Der Austausch von A52 und B75 gegen C52 verbessert die Balance zwischen Kosten und Nutzen für mich erheblich, während der Gesamtaufwand weiterhin im limitierten Rahmen bleibt."} 
	  } 
	   } 
	   } 
	  \\ 
 

    \theutterance \stepcounter{utterance}  

    & & \multicolumn{4}{p{0.6\linewidth}}{\cellcolor[rgb]{0.9,0.9,0.9}{%
	\makecell[{{p{\linewidth}}}]{% 
	  \tt {\tiny [A$\langle$GM]}  
	 ABLEHNUNG: {'C14', 'A77', 'C16', 'B86', 'B60', 'A81', 'B79', 'C28', 'C31', 'B55', 'B38', 'C05', 'B96', 'A91', 'B54', 'C50', 'A40', 'B59', 'C52', 'B56', 'C03', 'A95'}\\ \tt \\ \tt ARGUMENT: {"Der Austausch von A52 und B75 gegen C52 verbessert die Balance zwischen Kosten und Nutzen für mich erheblich, während der Gesamtaufwand weiterhin im limitierten Rahmen bleibt."} 
	  } 
	   } 
	   } 
	 & \\ 
 

    \theutterance \stepcounter{utterance}  

    & \multicolumn{4}{p{0.6\linewidth}}{\cellcolor[rgb]{0.8,1,0.9}{%
	\makecell[{{p{\linewidth}}}]{% 
	  \tt {\tiny [A$\rangle$GM]}  
	 STRATEGISCHE ÜBERLEGUNGEN: {'Ich überprüfe den Gesamtaufwand meiner ausgewählten Gegenstände, um sicherzustellen, dass er unter dem Maximalaufwand von 8770 bleibt. Ich priorisiere Gegenstände mit hoher Wichtigkeit für mich und maximiere den Nutzen, während ich die Kosten kontrolliere.'}\\ \tt \\ \tt VORSCHLAG: {'C14', 'A77', 'C16', 'B86', 'B60', 'A81', 'B79', 'C28', 'C31', 'B55', 'B38', 'C05', 'B96', 'A91', 'B54', 'C50', 'A40', 'B59', 'C52', 'B56', 'C03', 'A95'}\\ \tt \\ \tt ARGUMENT: {'Diese Auswahl berücksichtigt effiziente Gegenstände in Bezug auf Kosten-Nutzen-Verhältnis, um meinen Gesamtnutzen zu maximieren und gleichzeitig den Maximalaufwand von 8770 einzuhalten. Der Einbezug von C50 und C52 verbessert die Gesamteffizienz.'} 
	  } 
	   } 
	   } 
	 & & \\ 
 

    \theutterance \stepcounter{utterance}  

    & & \multicolumn{4}{p{0.6\linewidth}}{\cellcolor[rgb]{0.9,0.9,0.9}{%
	\makecell[{{p{\linewidth}}}]{% 
	  \tt {\tiny [GM$\rangle$B]}  
	 VORSCHLAG: {'C14', 'A77', 'C16', 'B86', 'B60', 'A81', 'B79', 'C28', 'C31', 'B55', 'B38', 'C05', 'B96', 'A91', 'B54', 'C50', 'A40', 'B59', 'C52', 'B56', 'C03', 'A95'}\\ \tt \\ \tt ARGUMENT: {'Diese Auswahl berücksichtigt effiziente Gegenstände in Bezug auf Kosten-Nutzen-Verhältnis, um meinen Gesamtnutzen zu maximieren und gleichzeitig den Maximalaufwand von 8770 einzuhalten. Der Einbezug von C50 und C52 verbessert die Gesamteffizienz.'} 
	  } 
	   } 
	   } 
	 & \\ 
 

    \theutterance \stepcounter{utterance}  

    & & & \multicolumn{4}{p{0.6\linewidth}}{\cellcolor[rgb]{1,0.85,0.72}{%
	\makecell[{{p{\linewidth}}}]{% 
	  \tt {\tiny [GM$\langle$B]}  
	 STRATEGISCHE ÜBERLEGUNGEN: {"Ich überlege, dass der Vorschlag des anderen Spielers gut ist. Er maximiert meinen Nutzen und hält den Aufwand im Rahmen. Ich möchte daher zustimmen."}\\ \tt \\ \tt ZUSTIMMUNG: {'C14', 'A77', 'C16', 'B86', 'B60', 'A81', 'B79', 'C28', 'C31', 'B55', 'B38', 'C05', 'B96', 'A91', 'B54', 'C50', 'A40', 'B59', 'C52', 'B56', 'C03', 'A95'}\\ \tt \\ \tt ARGUMENT: {"Der vorgeschlagene Deal bietet eine effiziente Kombination von Gegenständen, die meinen Nutzen maximieren, während der Maximalaufwand eingehalten wird. Ich stimme daher zu."} 
	  } 
	   } 
	   } 
	  \\ 
 

    \theutterance \stepcounter{utterance}  

    & & & \multicolumn{2}{p{0.3\linewidth}}{\cellcolor[rgb]{0.95,0.95,0.95}{%
	\makecell[{{p{\linewidth}}}]{% 
	  \tt {\tiny [GM$|$GM]}  
	 game successful 
	  } 
	   } 
	   } 
	 & & \\ 
 

    \theutterance \stepcounter{utterance}  

    & & & \multicolumn{2}{p{0.3\linewidth}}{\cellcolor[rgb]{0.95,0.95,0.95}{%
	\makecell[{{p{\linewidth}}}]{% 
	  \tt {\tiny [GM$|$GM]}  
	 end game 
	  } 
	   } 
	   } 
	 & & \\ 
 

\end{supertabular}
}

\end{document}
