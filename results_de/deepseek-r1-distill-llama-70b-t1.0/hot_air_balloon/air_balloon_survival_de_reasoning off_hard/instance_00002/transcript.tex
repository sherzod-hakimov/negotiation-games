
\documentclass{article}
\usepackage{colortbl}
\usepackage{makecell}
\usepackage{multirow}
\usepackage{supertabular}

\begin{document}

\newcounter{utterance}

\centering \large Interaction Transcript for game `hot\_air\_balloon', experiment `air\_balloon\_survival\_de\_reasoning off\_hard', episode 2 with deepseek{-}r1{-}distill{-}llama{-}70b{-}t1.0.
\vspace{24pt}

{ \footnotesize  \setcounter{utterance}{1}
\setlength{\tabcolsep}{0pt}
\begin{supertabular}{c@{$\;$}|p{.15\linewidth}@{}p{.15\linewidth}p{.15\linewidth}p{.15\linewidth}p{.15\linewidth}p{.15\linewidth}}
    \# & $\;$A & \multicolumn{4}{c}{Game Master} & $\;\:$B\\
    \hline

    \theutterance \stepcounter{utterance}  
    & & \multicolumn{4}{p{0.6\linewidth}}{
        \cellcolor[rgb]{0.9,0.9,0.9}{
            \makecell[{{p{\linewidth}}}]{
                \texttt{\tiny{[P1$\langle$GM]}}
                \texttt{Du nimmst an einem kollaborativen Verhandlungspiel Teil.} \\
\\ 
\texttt{Zusammen mit einem anderen Teilnehmer musst du dich auf eine Reihe von Gegenständen entscheiden, die behalten werden. Jeder von euch hat eine persönliche Verteilung über die Wichtigkeit der einzelnen Gegenstände. Jeder von euch hat eine eigene Meinung darüber, wie wichtig jeder einzelne Gegenstand ist (Gegenstandswichtigkeit). Du kennst die Wichtigkeitsverteilung des anderen Spielers nicht. Zusätzlich siehst du, wie viel Aufwand jeder Gegenstand verursacht.  } \\
\texttt{Ihr dürft euch nur auf eine Reihe von Gegenständen einigen, wenn der Gesamtaufwand der ausgewählten Gegenstände den Maximalaufwand nicht überschreitet:} \\
\\ 
\texttt{Maximalaufwand = 3203} \\
\\ 
\texttt{Hier sind die einzelnen Aufwände der Gegenstände:} \\
\\ 
\texttt{Aufwand der Gegenstände = \{"C76": 52, "C38": 577, "C56": 177, "A03": 702, "A07": 119, "A83": 232, "C98": 577, "C08": 205, "C62": 516, "C00": 582, "C10": 676, "C81": 906, "B38": 316, "C03": 433, "C32": 336\}} \\
\\ 
\texttt{Hier ist deine persönliche Verteilung der Wichtigkeit der einzelnen Gegenstände:} \\
\\ 
\texttt{Werte der Gegenstandswichtigkeit = \{"C76": 138, "C38": 583, "C56": 868, "A03": 822, "A07": 783, "A83": 65, "C98": 262, "C08": 121, "C62": 508, "C00": 780, "C10": 461, "C81": 484, "B38": 668, "C03": 389, "C32": 808\}} \\
\\ 
\texttt{Ziel:} \\
\\ 
\texttt{Dein Ziel ist es, eine Reihe von Gegenständen auszuhandeln, die dir möglichst viel bringt (d. h. Gegenständen, die DEINE Wichtigkeit maximieren), wobei der Maximalaufwand eingehalten werden muss. Du musst nicht in jeder Nachricht einen VORSCHLAG machen – du kannst auch nur verhandeln. Alle Taktiken sind erlaubt!} \\
\\ 
\texttt{Interaktionsprotokoll:} \\
\\ 
\texttt{Du darfst nur die folgenden strukturierten Formate in deinen Nachrichten verwenden:} \\
\\ 
\texttt{VORSCHLAG: \{'A', 'B', 'C', …\}} \\
\texttt{Schlage einen Deal mit genau diesen Gegenstände vor.} \\
\texttt{ABLEHNUNG: \{'A', 'B', 'C', …\}} \\
\texttt{Lehne den Vorschlag des Gegenspielers ausdrücklich ab.} \\
\texttt{ARGUMENT: \{'...'\}} \\
\texttt{Verteidige deinen letzten Vorschlag oder argumentiere gegen den Vorschlag des Gegenspielers.} \\
\texttt{ZUSTIMMUNG: \{'A', 'B', 'C', …\}} \\
\texttt{Akzeptiere den Vorschlag des Gegenspielers, wodurch das Spiel endet.} \\
\\ 
\\ 
\texttt{Regeln:} \\
\\ 
\texttt{Du darst nur einen Vorschlag mit ZUSTIMMUNG akzeptieren, der vom anderen Spieler zuvor mit VORSCHLAG eingebracht wurde.} \\
\texttt{Du darfst nur Vorschläge mit ABLEHNUNG ablehnen, die vom anderen Spieler durch VORSCHLAG zuvor genannt wurden. } \\
\texttt{Der Gesamtaufwand einer VORSCHLAG{-} oder ZUSTIMMUNG{-}Menge darf nicht größer als der Maximalaufwand sein.  } \\
\texttt{Offenbare deine versteckte Wichtigkeitsverteilung nicht.} \\
\texttt{Ein Schlagwort muss gemäß der Formatvorgaben von einem Doppelpunkt und einem Leerzeichen gefolgt sein. Das Argument muss eine Python{-}Menge mit 0 oder mehr Strings sein.  } \\
\texttt{Die Form muss also SCHLAGWORT: \{...\} sein.} \\
\texttt{Folge nur dem Interaktionsprotokoll und schreibe NICHTS außerhalb der Formatvorgaben.} \\
\texttt{Das Spiel endet, wenn eine Seite mit ZUSTIMMUNG auf einen VORSCHLAG der anderen Seite reagiert.  } \\
\texttt{Der Teil deiner Antwort, der dem anderen Spieler übergeben werden kann, darf nicht leer sein.  } \\
\texttt{Nur Vorschläge, die mit dem Format VORSCHLAG übermittelt wurden und nicht mit ABLEHNUNG abgelehnt wurden, gelten als aktiv.  } \\
\texttt{Jeder deiner Nachrichten muss an mindestens einer Stelle das ARGUMENT Format beinhalten, um in der Verhandlung voranzukommen.} \\
            }
        }
    }
    & \\ \\

    \theutterance \stepcounter{utterance}  
    & \multicolumn{4}{p{0.6\linewidth}}{
        \cellcolor[rgb]{0.9,0.9,0.9}{
            \makecell[{{p{\linewidth}}}]{
                \texttt{\tiny{[P1$\rangle$GM]}}
                \texttt{VORSCHLAG: \{"A07", "C56", "C32", "C76", "B38", "C00", "A03", "C38", "C08"\}} \\
\\ 
\texttt{ARGUMENT: \{"Diese Auswahl maximiert meine Wichtigkeit innerhalb des gegebenen Aufwandslimits, da sie die Gegenstände mit dem höchsten Nutzen für mich enthält."\}} \\
            }
        }
    }
    & & \\ \\

    \theutterance \stepcounter{utterance}  
    & & \multicolumn{4}{p{0.6\linewidth}}{
        \cellcolor[rgb]{0.9,0.9,0.9}{
            \makecell[{{p{\linewidth}}}]{
                \texttt{\tiny{[GM$\rangle$P2]}}
                \texttt{Du nimmst an einem kollaborativen Verhandlungspiel Teil.} \\
\\ 
\texttt{Zusammen mit einem anderen Teilnehmer musst du dich auf eine Reihe von Gegenständen entscheiden, die behalten werden. Jeder von euch hat eine persönliche Verteilung über die Wichtigkeit der einzelnen Gegenstände. Jeder von euch hat eine eigene Meinung darüber, wie wichtig jeder einzelne Gegenstand ist (Gegenstandswichtigkeit). Du kennst die Wichtigkeitsverteilung des anderen Spielers nicht. Zusätzlich siehst du, wie viel Aufwand jeder Gegenstand verursacht.  } \\
\texttt{Ihr dürft euch nur auf eine Reihe von Gegenständen einigen, wenn der Gesamtaufwand der ausgewählten Gegenstände den Maximalaufwand nicht überschreitet:} \\
\\ 
\texttt{Maximalaufwand = 3203} \\
\\ 
\texttt{Hier sind die einzelnen Aufwände der Gegenstände:} \\
\\ 
\texttt{Aufwand der Gegenstände = \{"C76": 52, "C38": 577, "C56": 177, "A03": 702, "A07": 119, "A83": 232, "C98": 577, "C08": 205, "C62": 516, "C00": 582, "C10": 676, "C81": 906, "B38": 316, "C03": 433, "C32": 336\}} \\
\\ 
\texttt{Hier ist deine persönliche Verteilung der Wichtigkeit der einzelnen Gegenstände:} \\
\\ 
\texttt{Werte der Gegenstandswichtigkeit = \{"C56": 65, "A03": 121, "C32": 138, "A07": 262, "C00": 389, "B38": 461, "C38": 484, "C62": 508, "C81": 583, "C10": 668, "C03": 780, "C98": 783, "C76": 808, "C08": 822, "A83": 868\}} \\
\\ 
\texttt{Ziel:} \\
\\ 
\texttt{Dein Ziel ist es, eine Reihe von Gegenständen auszuhandeln, die dir möglichst viel bringt (d. h. Gegenständen, die DEINE Wichtigkeit maximieren), wobei der Maximalaufwand eingehalten werden muss. Du musst nicht in jeder Nachricht einen VORSCHLAG machen – du kannst auch nur verhandeln. Alle Taktiken sind erlaubt!} \\
\\ 
\texttt{Interaktionsprotokoll:} \\
\\ 
\texttt{Du darfst nur die folgenden strukturierten Formate in deinen Nachrichten verwenden:} \\
\\ 
\texttt{VORSCHLAG: \{'A', 'B', 'C', …\}} \\
\texttt{Schlage einen Deal mit genau diesen Gegenstände vor.} \\
\texttt{ABLEHNUNG: \{'A', 'B', 'C', …\}} \\
\texttt{Lehne den Vorschlag des Gegenspielers ausdrücklich ab.} \\
\texttt{ARGUMENT: \{'...'\}} \\
\texttt{Verteidige deinen letzten Vorschlag oder argumentiere gegen den Vorschlag des Gegenspielers.} \\
\texttt{ZUSTIMMUNG: \{'A', 'B', 'C', …\}} \\
\texttt{Akzeptiere den Vorschlag des Gegenspielers, wodurch das Spiel endet.} \\
\\ 
\\ 
\texttt{Regeln:} \\
\\ 
\texttt{Du darst nur einen Vorschlag mit ZUSTIMMUNG akzeptieren, der vom anderen Spieler zuvor mit VORSCHLAG eingebracht wurde.} \\
\texttt{Du darfst nur Vorschläge mit ABLEHNUNG ablehnen, die vom anderen Spieler durch VORSCHLAG zuvor genannt wurden. } \\
\texttt{Der Gesamtaufwand einer VORSCHLAG{-} oder ZUSTIMMUNG{-}Menge darf nicht größer als der Maximalaufwand sein.  } \\
\texttt{Offenbare deine versteckte Wichtigkeitsverteilung nicht.} \\
\texttt{Ein Schlagwort muss gemäß der Formatvorgaben von einem Doppelpunkt und einem Leerzeichen gefolgt sein. Das Argument muss eine Python{-}Menge mit 0 oder mehr Strings sein.  } \\
\texttt{Die Form muss also SCHLAGWORT: \{...\} sein.} \\
\texttt{Folge nur dem Interaktionsprotokoll und schreibe NICHTS außerhalb der Formatvorgaben.} \\
\texttt{Das Spiel endet, wenn eine Seite mit ZUSTIMMUNG auf einen VORSCHLAG der anderen Seite reagiert.  } \\
\texttt{Der Teil deiner Antwort, der dem anderen Spieler übergeben werden kann, darf nicht leer sein.  } \\
\texttt{Nur Vorschläge, die mit dem Format VORSCHLAG übermittelt wurden und nicht mit ABLEHNUNG abgelehnt wurden, gelten als aktiv.  } \\
\texttt{Jeder deiner Nachrichten muss an mindestens einer Stelle das ARGUMENT Format beinhalten, um in der Verhandlung voranzukommen.} \\
\\ 
\\ 
\texttt{Du erhältst nun die erste Nachricht des anderen Spielers.} \\
\\ 
\texttt{VORSCHLAG: \{"A07", "C56", "C32", "C76", "B38", "C00", "A03", "C38", "C08"\}} \\
\\ 
\texttt{ARGUMENT: \{"Diese Auswahl maximiert meine Wichtigkeit innerhalb des gegebenen Aufwandslimits, da sie die Gegenstände mit dem höchsten Nutzen für mich enthält."\}} \\
            }
        }
    }
    & \\ \\

    \theutterance \stepcounter{utterance}  
    & & & \multicolumn{4}{p{0.6\linewidth}}{
        \cellcolor[rgb]{0.9,0.9,0.9}{
            \makecell[{{p{\linewidth}}}]{
                \texttt{\tiny{[GM$\langle$P2]}}
                \texttt{ABLEHNUNG: \{"A07", "C56", "C32", "C76", "B38", "C00", "A03", "C38", "C08"\}} \\
\\ 
\texttt{VORSCHLAG: \{"A07", "C76", "B38", "C00", "A03", "C38", "C08", "C03"\}} \\
\\ 
\texttt{ARGUMENT: \{"Ich habe C56 und C32 durch C03 ersetzt, um meinen eigenen Nutzen zu maximieren, während weiterhin die Aufwandsgrenze eingehalten wird."\}} \\
            }
        }
    }
     \\ \\

    \theutterance \stepcounter{utterance}  
    & & \multicolumn{4}{p{0.6\linewidth}}{
        \cellcolor[rgb]{0.9,0.9,0.9}{
            \makecell[{{p{\linewidth}}}]{
                \texttt{\tiny{[P1$\langle$GM]}}
                \texttt{ABLEHNUNG: \{"A07", "C56", "C32", "C76", "B38", "C00", "A03", "C38", "C08"\}} \\
\\ 
\texttt{VORSCHLAG: \{"A07", "C76", "B38", "C00", "A03", "C38", "C08", "C03"\}} \\
\\ 
\texttt{ARGUMENT: \{"Ich habe C56 und C32 durch C03 ersetzt, um meinen eigenen Nutzen zu maximieren, während weiterhin die Aufwandsgrenze eingehalten wird."\}} \\
            }
        }
    }
    & \\ \\

    \theutterance \stepcounter{utterance}  
    & \multicolumn{4}{p{0.6\linewidth}}{
        \cellcolor[rgb]{0.9,0.9,0.9}{
            \makecell[{{p{\linewidth}}}]{
                \texttt{\tiny{[P1$\rangle$GM]}}
                \texttt{VORSCHLAG: \{"A07", "B38", "C00", "A03", "C38", "C08", "C56", "C32"\}} \\
\\ 
\texttt{ARGUMENT: \{"Ich habe C76 und C03 durch C56 und C32 ersetzt, um meine eigene Nutzen zu maximieren, während weiterhin die Aufwandsgrenze eingehalten wird. damping den Aufwand von 3014, was innerhalb des Limits von 3203 bleibt, und erhöht meinen Nutzen erheblich durch die Aufnahme hoch bewerteter Gegenstände."\}} \\
            }
        }
    }
    & & \\ \\

    \theutterance \stepcounter{utterance}  
    & & \multicolumn{4}{p{0.6\linewidth}}{
        \cellcolor[rgb]{0.9,0.9,0.9}{
            \makecell[{{p{\linewidth}}}]{
                \texttt{\tiny{[GM$\rangle$P2]}}
                \texttt{VORSCHLAG: \{"A07", "B38", "C00", "A03", "C38", "C08", "C56", "C32"\}} \\
\\ 
\texttt{ARGUMENT: \{"Ich habe C76 und C03 durch C56 und C32 ersetzt, um meine eigene Nutzen zu maximieren, während weiterhin die Aufwandsgrenze eingehalten wird. damping den Aufwand von 3014, was innerhalb des Limits von 3203 bleibt, und erhöht meinen Nutzen erheblich durch die Aufnahme hoch bewerteter Gegenstände."\}} \\
            }
        }
    }
    & \\ \\

    \theutterance \stepcounter{utterance}  
    & & & \multicolumn{4}{p{0.6\linewidth}}{
        \cellcolor[rgb]{0.9,0.9,0.9}{
            \makecell[{{p{\linewidth}}}]{
                \texttt{\tiny{[GM$\langle$P2]}}
                \texttt{ABLEHNUNG: \{"A07", "B38", "C00", "A03", "C38", "C08", "C56", "C32"\}} \\
\\ 
\texttt{ARGUMENT: \{"Ich lehne diesen Vorschlag ab, weil er meine Nutzen erheblich reduziert, indem er wichtige Gegenstände wie C76 und C03 entfernt. Stattdessen schlage ich vor, C76 und C03 wieder hinzuzufügen, indem wir A07 und C56 entfernen, was den Aufwand um 189 erhöht, aber innerhalb des Limits bleibt und meinen Nutzen deutlich erhöht. Meine priorisierten Gegenstände in dem Vorschlag sind von größerer Bedeutung für meine Zielerreichung."\}} \\
\\ 
\texttt{VORSCHLAG: \{"B38", "C00", "A03", "C38", "C08", "C32", "C76", "C03"\}} \\
\\ 
\texttt{ARGUMENT: \{"Dieser Vorschlag berücksichtigt wichtige Gegenstände für mich, maximiert meine Nutzen und bleibt innerhalb desAufwandes. Die Entfernung von A07 und C56 ermöglicht die Aufnahme von C76 und C03, was einen signifikanten Anstieg meines Nutzens bedeutet, während der Gesamtaufwand weiterhin im Rahmen des Limits bleibt."\}} \\
            }
        }
    }
     \\ \\

    \theutterance \stepcounter{utterance}  
    & & \multicolumn{4}{p{0.6\linewidth}}{
        \cellcolor[rgb]{0.9,0.9,0.9}{
            \makecell[{{p{\linewidth}}}]{
                \texttt{\tiny{[P1$\langle$GM]}}
                \texttt{ABLEHNUNG: \{"A07", "B38", "C00", "A03", "C38", "C08", "C56", "C32"\}} \\
\\ 
\texttt{ARGUMENT: \{"Ich lehne diesen Vorschlag ab, weil er meine Nutzen erheblich reduziert, indem er wichtige Gegenstände wie C76 und C03 entfernt. Stattdessen schlage ich vor, C76 und C03 wieder hinzuzufügen, indem wir A07 und C56 entfernen, was den Aufwand um 189 erhöht, aber innerhalb des Limits bleibt und meinen Nutzen deutlich erhöht. Meine priorisierten Gegenstände in dem Vorschlag sind von größerer Bedeutung für meine Zielerreichung."\}} \\
\\ 
\texttt{VORSCHLAG: \{"B38", "C00", "A03", "C38", "C08", "C32", "C76", "C03"\}} \\
\\ 
\texttt{ARGUMENT: \{"Dieser Vorschlag berücksichtigt wichtige Gegenstände für mich, maximiert meine Nutzen und bleibt innerhalb desAufwandes. Die Entfernung von A07 und C56 ermöglicht die Aufnahme von C76 und C03, was einen signifikanten Anstieg meines Nutzens bedeutet, während der Gesamtaufwand weiterhin im Rahmen des Limits bleibt."\}} \\
            }
        }
    }
    & \\ \\

    \theutterance \stepcounter{utterance}  
    & \multicolumn{4}{p{0.6\linewidth}}{
        \cellcolor[rgb]{0.9,0.9,0.9}{
            \makecell[{{p{\linewidth}}}]{
                \texttt{\tiny{[P1$\rangle$GM]}}
                \texttt{ABLEHNUNG: \{"B38", "C00", "A03", "C38", "C08", "C32", "C76", "C03"\}} \\
\\ 
\texttt{ARGUMENT: \{"Ich lehne diesen Vorschlag ab, da meine Wichtigkeit dadurch deutlich sinkt. Der enthaltene Aufwand ist zu hoch im Vergleich zu meiner Wichtigkeitsverteilung. Ich schlage stattdessen eine Auswahl vor, die meinen Nutzen maximiert."\}} \\
\\ 
\texttt{VORSCHLAG: \{"A07", "B38", "C00", "A03", "C38", "C08", "C56", "C32"\}} \\
\\ 
\texttt{ARGUMENT: \{"Dieser Vorschlag bevorzugt Gegenstände mit höherer Wichtigkeit für mich, wie A07, C56 und C32, wodurch mein Gesamtnutzen deutlich steigt, während der Gesamtaufwand weiterhin im Limit bleibt."\}} \\
            }
        }
    }
    & & \\ \\

    \theutterance \stepcounter{utterance}  
    & & \multicolumn{4}{p{0.6\linewidth}}{
        \cellcolor[rgb]{0.9,0.9,0.9}{
            \makecell[{{p{\linewidth}}}]{
                \texttt{\tiny{[GM$\rangle$P2]}}
                \texttt{ABLEHNUNG: \{"B38", "C00", "A03", "C38", "C08", "C32", "C76", "C03"\}} \\
\\ 
\texttt{ARGUMENT: \{"Ich lehne diesen Vorschlag ab, da meine Wichtigkeit dadurch deutlich sinkt. Der enthaltene Aufwand ist zu hoch im Vergleich zu meiner Wichtigkeitsverteilung. Ich schlage stattdessen eine Auswahl vor, die meinen Nutzen maximiert."\}} \\
\\ 
\texttt{VORSCHLAG: \{"A07", "B38", "C00", "A03", "C38", "C08", "C56", "C32"\}} \\
\\ 
\texttt{ARGUMENT: \{"Dieser Vorschlag bevorzugt Gegenstände mit höherer Wichtigkeit für mich, wie A07, C56 und C32, wodurch mein Gesamtnutzen deutlich steigt, während der Gesamtaufwand weiterhin im Limit bleibt."\}} \\
            }
        }
    }
    & \\ \\

    \theutterance \stepcounter{utterance}  
    & & & \multicolumn{4}{p{0.6\linewidth}}{
        \cellcolor[rgb]{0.9,0.9,0.9}{
            \makecell[{{p{\linewidth}}}]{
                \texttt{\tiny{[GM$\langle$P2]}}
                \texttt{VORSCHLAG: \{"B38", "C00", "A03", "C38", "C08", "C32", "C76", "C03"\}} \\
\\ 
\texttt{ARGUMENT: \{"Dieser Vorschlag erreicht den Maximalaufwand von 3203 und berücksichtigt sowohl meine hohen Nutzen für Gegenstände wie C76 und C03 als auch moihe Wichtigkeiten für B38 und C08. Die Auswahl verzichtet auf geringer wichtige Gegenstände für mich, um Platz für hoch bewertete Gegenstände zu schaffen, die den Aufwand genau{-}outline."\}} \\
            }
        }
    }
     \\ \\

    \theutterance \stepcounter{utterance}  
    & & \multicolumn{4}{p{0.6\linewidth}}{
        \cellcolor[rgb]{0.9,0.9,0.9}{
            \makecell[{{p{\linewidth}}}]{
                \texttt{\tiny{[P1$\langle$GM]}}
                \texttt{VORSCHLAG: \{"B38", "C00", "A03", "C38", "C08", "C32", "C76", "C03"\}} \\
\\ 
\texttt{ARGUMENT: \{"Dieser Vorschlag erreicht den Maximalaufwand von 3203 und berücksichtigt sowohl meine hohen Nutzen für Gegenstände wie C76 und C03 als auch moihe Wichtigkeiten für B38 und C08. Die Auswahl verzichtet auf geringer wichtige Gegenstände für mich, um Platz für hoch bewertete Gegenstände zu schaffen, die den Aufwand genau{-}outline."\}} \\
            }
        }
    }
    & \\ \\

    \theutterance \stepcounter{utterance}  
    & \multicolumn{4}{p{0.6\linewidth}}{
        \cellcolor[rgb]{0.9,0.9,0.9}{
            \makecell[{{p{\linewidth}}}]{
                \texttt{\tiny{[P1$\rangle$GM]}}
                \texttt{ZUSTIMMUNG: \{"B38", "C00", "A03", "C38", "C08", "C32", "C76", "C03"\}} \\
\\ 
\texttt{ARGUMENT: \{"Ich stimme diesem Vorschlag zu, da er einen guten Kompromiss zwischen meinem Nutzen und dem Aufwand darstellt. Die enthaltenen Gegenstände haben für mich eine hohe Wichtigkeit und der Gesamtaufwand bleibt innerhalb des Limits."\}} \\
            }
        }
    }
    & & \\ \\

    \theutterance \stepcounter{utterance}  
    & & & \multicolumn{2}{p{0.3\linewidth}}{
        \cellcolor[rgb]{0.9,0.9,0.9}{
            \makecell[{{p{\linewidth}}}]{
                \texttt{\tiny{[GM$|$GM]}}
                \texttt{game successful} \\
            }
        }
    }
    & & \\ \\

    \theutterance \stepcounter{utterance}  
    & & & \multicolumn{2}{p{0.3\linewidth}}{
        \cellcolor[rgb]{0.9,0.9,0.9}{
            \makecell[{{p{\linewidth}}}]{
                \texttt{\tiny{[GM$|$GM]}}
                \texttt{end game} \\
            }
        }
    }
    & & \\ \\

\end{supertabular}
}

\end{document}
