
\documentclass{article}
\usepackage{colortbl}
\usepackage{makecell}
\usepackage{multirow}
\usepackage{supertabular}

\begin{document}

\newcounter{utterance}

\twocolumn

{ \footnotesize  \setcounter{utterance}{1}
\setlength{\tabcolsep}{0pt}
\begin{supertabular}{c@{$\;$}|p{.15\linewidth}@{}p{.15\linewidth}p{.15\linewidth}p{.15\linewidth}p{.15\linewidth}p{.15\linewidth}}

    \# & $\;$A & \multicolumn{4}{c}{Game Master} & $\;\:$B\\
    \hline 

    \theutterance \stepcounter{utterance}  

    & & \multicolumn{4}{p{0.6\linewidth}}{\cellcolor[rgb]{0.9,0.9,0.9}{%
	\makecell[{{p{\linewidth}}}]{% 
	  \tt {\tiny [A$\langle$GM]}  
	 Du nimmst an einem kollaborativen Verhandlungspiel Teil.\\ \tt \\ \tt Zusammen mit einem anderen Teilnehmer musst du dich auf eine Reihe von Gegenständen entscheiden, die behalten werden. Jeder von euch hat eine persönliche Verteilung über die Wichtigkeit der einzelnen Gegenstände. Jeder von euch hat eine eigene Meinung darüber, wie wichtig jeder einzelne Gegenstand ist (Gegenstandswichtigkeit). Du kennst die Wichtigkeitsverteilung des anderen Spielers nicht. Zusätzlich siehst du, wie viel Aufwand jeder Gegenstand verursacht.  \\ \tt Ihr dürft euch nur auf eine Reihe von Gegenständen einigen, wenn der Gesamtaufwand der ausgewählten Gegenstände den Maximalaufwand nicht überschreitet:\\ \tt \\ \tt Maximalaufwand = 2984\\ \tt \\ \tt Hier sind die einzelnen Aufwände der Gegenstände:\\ \tt \\ \tt Aufwand der Gegenstände = {"C76": 350, "C38": 276, "C56": 616, "A03": 737, "A07": 531, "A83": 389, "C98": 24, "C08": 125, "C62": 338, "C00": 356, "C10": 143, "C81": 117, "B38": 257, "C03": 921, "C32": 789}\\ \tt \\ \tt Hier ist deine persönliche Verteilung der Wichtigkeit der einzelnen Gegenstände:\\ \tt \\ \tt Werte der Gegenstandswichtigkeit = {"C76": 138, "C38": 583, "C56": 868, "A03": 822, "A07": 783, "A83": 65, "C98": 262, "C08": 121, "C62": 508, "C00": 780, "C10": 461, "C81": 484, "B38": 668, "C03": 389, "C32": 808}\\ \tt \\ \tt Ziel:\\ \tt \\ \tt Dein Ziel ist es, eine Reihe von Gegenständen auszuhandeln, die dir möglichst viel bringt (d. h. Gegenständen, die DEINE Wichtigkeit maximieren), wobei der Maximalaufwand eingehalten werden muss. Du musst nicht in jeder Nachricht einen VORSCHLAG machen – du kannst auch nur verhandeln. Alle Taktiken sind erlaubt!\\ \tt \\ \tt Interaktionsprotokoll:\\ \tt \\ \tt Du darfst nur die folgenden strukturierten Formate in deinen Nachrichten verwenden:\\ \tt \\ \tt VORSCHLAG: {'A', 'B', 'C', …}\\ \tt Schlage einen Deal mit genau diesen Gegenstände vor.\\ \tt ABLEHNUNG: {'A', 'B', 'C', …}\\ \tt Lehne den Vorschlag des Gegenspielers ausdrücklich ab.\\ \tt ARGUMENT: {'...'}\\ \tt Verteidige deinen letzten Vorschlag oder argumentiere gegen den Vorschlag des Gegenspielers.\\ \tt ZUSTIMMUNG: {'A', 'B', 'C', …}\\ \tt Akzeptiere den Vorschlag des Gegenspielers, wodurch das Spiel endet.\\ \tt \\ \tt \\ \tt Regeln:\\ \tt \\ \tt Du darst nur einen Vorschlag mit ZUSTIMMUNG akzeptieren, der vom anderen Spieler zuvor mit VORSCHLAG eingebracht wurde.\\ \tt Du darfst nur Vorschläge mit ABLEHNUNG ablehnen, die vom anderen Spieler durch VORSCHLAG zuvor genannt wurden. \\ \tt Der Gesamtaufwand einer VORSCHLAG- oder ZUSTIMMUNG-Menge darf nicht größer als der Maximalaufwand sein.  \\ \tt Offenbare deine versteckte Wichtigkeitsverteilung nicht.\\ \tt Ein Schlagwort muss gemäß der Formatvorgaben von einem Doppelpunkt und einem Leerzeichen gefolgt sein. Das Argument muss eine Python-Menge mit 0 oder mehr Strings sein.  \\ \tt Die Form muss also SCHLAGWORT: {...} sein.\\ \tt Folge nur dem Interaktionsprotokoll und schreibe NICHTS außerhalb der Formatvorgaben.\\ \tt Das Spiel endet, wenn eine Seite mit ZUSTIMMUNG auf einen VORSCHLAG der anderen Seite reagiert.  \\ \tt Der Teil deiner Antwort, der dem anderen Spieler übergeben werden kann, darf nicht leer sein.  \\ \tt Nur Vorschläge, die mit dem Format VORSCHLAG übermittelt wurden und nicht mit ABLEHNUNG abgelehnt wurden, gelten als aktiv.  \\ \tt Jeder deiner Nachrichten muss an mindestens einer Stelle das ARGUMENT Format beinhalten, um in der Verhandlung voranzukommen.\\ \tt  
	  } 
	   } 
	   } 
	 & \\ 
 

    \theutterance \stepcounter{utterance}  

    & \multicolumn{4}{p{0.6\linewidth}}{\cellcolor[rgb]{0.8,1,0.9}{%
	\makecell[{{p{\linewidth}}}]{% 
	  \tt {\tiny [A$\rangle$GM]}  
	 VORSCHLAG: {'C98', 'C10', 'B38', 'C00', 'C38', 'A03', 'C56', 'C62', 'C81'}  \\ \tt ARGUMENT: {'Dieser Vorschlag maximiert meine Wichtigkeit, indem er die Gegenstände mit dem besten Verhältnis aus Wichtigkeit und Aufwand auswählt, wobei der Maximalaufwand eingehalten wird. Er kombiniert hochwertige Gegenstände wie C98, die einen extrem hohen Wert bei geringem Aufwand bieten, mit anderen wichtigen Items wie C10 und B38.'} 
	  } 
	   } 
	   } 
	 & & \\ 
 

    \theutterance \stepcounter{utterance}  

    & & \multicolumn{4}{p{0.6\linewidth}}{\cellcolor[rgb]{0.9,0.9,0.9}{%
	\makecell[{{p{\linewidth}}}]{% 
	  \tt {\tiny [GM$\rangle$B]}  
	 Du nimmst an einem kollaborativen Verhandlungspiel Teil.\\ \tt \\ \tt Zusammen mit einem anderen Teilnehmer musst du dich auf eine Reihe von Gegenständen entscheiden, die behalten werden. Jeder von euch hat eine persönliche Verteilung über die Wichtigkeit der einzelnen Gegenstände. Jeder von euch hat eine eigene Meinung darüber, wie wichtig jeder einzelne Gegenstand ist (Gegenstandswichtigkeit). Du kennst die Wichtigkeitsverteilung des anderen Spielers nicht. Zusätzlich siehst du, wie viel Aufwand jeder Gegenstand verursacht.  \\ \tt Ihr dürft euch nur auf eine Reihe von Gegenständen einigen, wenn der Gesamtaufwand der ausgewählten Gegenstände den Maximalaufwand nicht überschreitet:\\ \tt \\ \tt Maximalaufwand = 2984\\ \tt \\ \tt Hier sind die einzelnen Aufwände der Gegenstände:\\ \tt \\ \tt Aufwand der Gegenstände = {"C76": 350, "C38": 276, "C56": 616, "A03": 737, "A07": 531, "A83": 389, "C98": 24, "C08": 125, "C62": 338, "C00": 356, "C10": 143, "C81": 117, "B38": 257, "C03": 921, "C32": 789}\\ \tt \\ \tt Hier ist deine persönliche Verteilung der Wichtigkeit der einzelnen Gegenstände:\\ \tt \\ \tt Werte der Gegenstandswichtigkeit = {"C56": 65, "A03": 121, "C32": 138, "A07": 262, "C00": 389, "B38": 461, "C38": 484, "C62": 508, "C81": 583, "C10": 668, "C03": 780, "C98": 783, "C76": 808, "C08": 822, "A83": 868}\\ \tt \\ \tt Ziel:\\ \tt \\ \tt Dein Ziel ist es, eine Reihe von Gegenständen auszuhandeln, die dir möglichst viel bringt (d. h. Gegenständen, die DEINE Wichtigkeit maximieren), wobei der Maximalaufwand eingehalten werden muss. Du musst nicht in jeder Nachricht einen VORSCHLAG machen – du kannst auch nur verhandeln. Alle Taktiken sind erlaubt!\\ \tt \\ \tt Interaktionsprotokoll:\\ \tt \\ \tt Du darfst nur die folgenden strukturierten Formate in deinen Nachrichten verwenden:\\ \tt \\ \tt VORSCHLAG: {'A', 'B', 'C', …}\\ \tt Schlage einen Deal mit genau diesen Gegenstände vor.\\ \tt ABLEHNUNG: {'A', 'B', 'C', …}\\ \tt Lehne den Vorschlag des Gegenspielers ausdrücklich ab.\\ \tt ARGUMENT: {'...'}\\ \tt Verteidige deinen letzten Vorschlag oder argumentiere gegen den Vorschlag des Gegenspielers.\\ \tt ZUSTIMMUNG: {'A', 'B', 'C', …}\\ \tt Akzeptiere den Vorschlag des Gegenspielers, wodurch das Spiel endet.\\ \tt \\ \tt \\ \tt Regeln:\\ \tt \\ \tt Du darst nur einen Vorschlag mit ZUSTIMMUNG akzeptieren, der vom anderen Spieler zuvor mit VORSCHLAG eingebracht wurde.\\ \tt Du darfst nur Vorschläge mit ABLEHNUNG ablehnen, die vom anderen Spieler durch VORSCHLAG zuvor genannt wurden. \\ \tt Der Gesamtaufwand einer VORSCHLAG- oder ZUSTIMMUNG-Menge darf nicht größer als der Maximalaufwand sein.  \\ \tt Offenbare deine versteckte Wichtigkeitsverteilung nicht.\\ \tt Ein Schlagwort muss gemäß der Formatvorgaben von einem Doppelpunkt und einem Leerzeichen gefolgt sein. Das Argument muss eine Python-Menge mit 0 oder mehr Strings sein.  \\ \tt Die Form muss also SCHLAGWORT: {...} sein.\\ \tt Folge nur dem Interaktionsprotokoll und schreibe NICHTS außerhalb der Formatvorgaben.\\ \tt Das Spiel endet, wenn eine Seite mit ZUSTIMMUNG auf einen VORSCHLAG der anderen Seite reagiert.  \\ \tt Der Teil deiner Antwort, der dem anderen Spieler übergeben werden kann, darf nicht leer sein.  \\ \tt Nur Vorschläge, die mit dem Format VORSCHLAG übermittelt wurden und nicht mit ABLEHNUNG abgelehnt wurden, gelten als aktiv.  \\ \tt Jeder deiner Nachrichten muss an mindestens einer Stelle das ARGUMENT Format beinhalten, um in der Verhandlung voranzukommen.\\ \tt \\ \tt \\ \tt Du erhältst nun die erste Nachricht des anderen Spielers.\\ \tt \\ \tt VORSCHLAG: {'C98', 'C10', 'B38', 'C00', 'C38', 'A03', 'C56', 'C62', 'C81'}  \\ \tt ARGUMENT: {'Dieser Vorschlag maximiert meine Wichtigkeit, indem er die Gegenstände mit dem besten Verhältnis aus Wichtigkeit und Aufwand auswählt, wobei der Maximalaufwand eingehalten wird. Er kombiniert hochwertige Gegenstände wie C98, die einen extrem hohen Wert bei geringem Aufwand bieten, mit anderen wichtigen Items wie C10 und B38.'} 
	  } 
	   } 
	   } 
	 & \\ 
 

    \theutterance \stepcounter{utterance}  

    & & & \multicolumn{4}{p{0.6\linewidth}}{\cellcolor[rgb]{1,0.85,0.72}{%
	\makecell[{{p{\linewidth}}}]{% 
	  \tt {\tiny [GM$\langle$B]}  
	 ABLEHNUNG: {'C98', 'C10', 'B38', 'C00', 'C38', 'A03', 'C56', 'C62', 'C81'}  \\ \tt ARGUMENT: {'Ihr Vorschlag ist gut, aber ich kann eine bessere Kombination vorschlagen, die meinen Nutzen maximiert, ohne den Maximalaufwand zu überschreiten. Meine Auswahl enthält hochwertige Gegenstände wie A83 und C08, die für mich einen höheren Wert haben, während der Aufwand weiterhin im Rahmen bleibt.'}  \\ \tt VORSCHLAG: {'C08', 'A83', 'C76', 'C98', 'C10', 'C00', 'C81', 'C38', 'C62', 'B38', 'A07'} 
	  } 
	   } 
	   } 
	  \\ 
 

    \theutterance \stepcounter{utterance}  

    & & \multicolumn{4}{p{0.6\linewidth}}{\cellcolor[rgb]{0.9,0.9,0.9}{%
	\makecell[{{p{\linewidth}}}]{% 
	  \tt {\tiny [A$\langle$GM]}  
	 ABLEHNUNG: {'C98', 'C10', 'B38', 'C00', 'C38', 'A03', 'C56', 'C62', 'C81'}  \\ \tt ARGUMENT: {'Ihr Vorschlag ist gut, aber ich kann eine bessere Kombination vorschlagen, die meinen Nutzen maximiert, ohne den Maximalaufwand zu überschreiten. Meine Auswahl enthält hochwertige Gegenstände wie A83 und C08, die für mich einen höheren Wert haben, während der Aufwand weiterhin im Rahmen bleibt.'}  \\ \tt VORSCHLAG: {'C08', 'A83', 'C76', 'C98', 'C10', 'C00', 'C81', 'C38', 'C62', 'B38', 'A07'} 
	  } 
	   } 
	   } 
	 & \\ 
 

    \theutterance \stepcounter{utterance}  

    & \multicolumn{4}{p{0.6\linewidth}}{\cellcolor[rgb]{0.8,1,0.9}{%
	\makecell[{{p{\linewidth}}}]{% 
	  \tt {\tiny [A$\rangle$GM]}  
	 ABLEHNUNG: {'C08', 'A83', 'C76', 'C98', 'C10', 'C00', 'C81', 'C38', 'C62', 'B38', 'A07'}\\ \tt ARGUMENT: {'Ihr Vorschlag überschreitet den Maximalaufwand von 2984. Mein Gegenangebot bleibt unter diesem Limit und maximiert meinen Nutzen durch eine effiziente Auswahl von Gegenständen, die ein hohes Maß an Wichtigkeit bei angemessenem Aufwand bieten.'}\\ \tt VORSCHLAG: {'C98', 'C10', 'B38', 'C00', 'C38', 'A07', 'C56', 'C62', 'C81', 'C08'} 
	  } 
	   } 
	   } 
	 & & \\ 
 

    \theutterance \stepcounter{utterance}  

    & & \multicolumn{4}{p{0.6\linewidth}}{\cellcolor[rgb]{0.9,0.9,0.9}{%
	\makecell[{{p{\linewidth}}}]{% 
	  \tt {\tiny [GM$\rangle$B]}  
	 ABLEHNUNG: {'C08', 'A83', 'C76', 'C98', 'C10', 'C00', 'C81', 'C38', 'C62', 'B38', 'A07'}\\ \tt ARGUMENT: {'Ihr Vorschlag überschreitet den Maximalaufwand von 2984. Mein Gegenangebot bleibt unter diesem Limit und maximiert meinen Nutzen durch eine effiziente Auswahl von Gegenständen, die ein hohes Maß an Wichtigkeit bei angemessenem Aufwand bieten.'}\\ \tt VORSCHLAG: {'C98', 'C10', 'B38', 'C00', 'C38', 'A07', 'C56', 'C62', 'C81', 'C08'} 
	  } 
	   } 
	   } 
	 & \\ 
 

    \theutterance \stepcounter{utterance}  

    & & & \multicolumn{4}{p{0.6\linewidth}}{\cellcolor[rgb]{1,0.85,0.72}{%
	\makecell[{{p{\linewidth}}}]{% 
	  \tt {\tiny [GM$\langle$B]}  
	 VORSCHLAG: {'C76', 'C08', 'A83', 'C10', 'C98', 'C00', 'C38', 'B38', 'C81', 'A07'}  \\ \tt ARGUMENT: {'Dieser Vorschlag enthält eine ausgewogene Mischung aus hoch重要en und kostengünstigen Gegenständen, die minha wichtigkeit maximieren, während der Maximalaufwand eingehalten wird. Es handelt sich um eine effiziente Auswahl, die wichtige Gegenstände wie C76, C08 und A83 beinhaltet, die höchste wichtige Werte bieten, ohne die Kosten zu übertreiben. Dieses Angebot ist fair und bringt mir den größten Nutzen, während es den Spielregeln entspricht.'} 
	  } 
	   } 
	   } 
	  \\ 
 

    \theutterance \stepcounter{utterance}  

    & & \multicolumn{4}{p{0.6\linewidth}}{\cellcolor[rgb]{0.9,0.9,0.9}{%
	\makecell[{{p{\linewidth}}}]{% 
	  \tt {\tiny [A$\langle$GM]}  
	 VORSCHLAG: {'C76', 'C08', 'A83', 'C10', 'C98', 'C00', 'C38', 'B38', 'C81', 'A07'}  \\ \tt ARGUMENT: {'Dieser Vorschlag enthält eine ausgewogene Mischung aus hoch重要en und kostengünstigen Gegenständen, die minha wichtigkeit maximieren, während der Maximalaufwand eingehalten wird. Es handelt sich um eine effiziente Auswahl, die wichtige Gegenstände wie C76, C08 und A83 beinhaltet, die höchste wichtige Werte bieten, ohne die Kosten zu übertreiben. Dieses Angebot ist fair und bringt mir den größten Nutzen, während es den Spielregeln entspricht.'} 
	  } 
	   } 
	   } 
	 & \\ 
 

    \theutterance \stepcounter{utterance}  

    & \multicolumn{4}{p{0.6\linewidth}}{\cellcolor[rgb]{0.8,1,0.9}{%
	\makecell[{{p{\linewidth}}}]{% 
	  \tt {\tiny [A$\rangle$GM]}  
	 VORSCHLAG: {'C98', 'C81', 'C10', 'B38', 'C00', 'C38', 'C62', 'C56', 'A03'}  \\ \tt ARGUMENT: {'Diese Kombination maximiert meine Wichtigkeit, indem sie die effizientesten Gegenstände beinhaltet, die einen hohen Nutzen bei geringem Aufwand bieten. Der Gesamtaufwand bleibt unter dem Maximalaufwand von 2984, während der Gewinn für mich maximiert wird. Dieses Angebot ist fair und bringt mir den größten Nutzen, während es den Spielregeln entspricht.'} 
	  } 
	   } 
	   } 
	 & & \\ 
 

    \theutterance \stepcounter{utterance}  

    & & \multicolumn{4}{p{0.6\linewidth}}{\cellcolor[rgb]{0.9,0.9,0.9}{%
	\makecell[{{p{\linewidth}}}]{% 
	  \tt {\tiny [GM$\rangle$B]}  
	 VORSCHLAG: {'C98', 'C81', 'C10', 'B38', 'C00', 'C38', 'C62', 'C56', 'A03'}  \\ \tt ARGUMENT: {'Diese Kombination maximiert meine Wichtigkeit, indem sie die effizientesten Gegenstände beinhaltet, die einen hohen Nutzen bei geringem Aufwand bieten. Der Gesamtaufwand bleibt unter dem Maximalaufwand von 2984, während der Gewinn für mich maximiert wird. Dieses Angebot ist fair und bringt mir den größten Nutzen, während es den Spielregeln entspricht.'} 
	  } 
	   } 
	   } 
	 & \\ 
 

    \theutterance \stepcounter{utterance}  

    & & & \multicolumn{4}{p{0.6\linewidth}}{\cellcolor[rgb]{1,0.85,0.72}{%
	\makecell[{{p{\linewidth}}}]{% 
	  \tt {\tiny [GM$\langle$B]}  
	 ABLEHNUNG: {'C98', 'C81', 'C10', 'B38', 'C00', 'C38', 'C62', 'C56', 'A03'}  \\ \tt ARGUMENT: {'Ihr Vorschlag wird abgelehnt, da ich eine bessere Kombination vorschlagen kann, die meinen Nutzen erheblich erhöht, ohne den Maximalaufwand zu überschreiten. Mein Angebot enthält hochwertige Gegenstände, die für mich einen höheren Wert haben, während der Aufwand weiterhin im Rahmen bleibt.'}  \\ \tt VORSCHLAG: {'C76', 'C08', 'A83', 'C10', 'C98', 'C00', 'C38', 'B38', 'C81', 'C62', 'A07'}  \\ \tt ARGUMENT: {'Dieser Vorschlag maximiert meine Wichtigkeit durch die Auswahl von Gegenständen mit dem höchsten Nutzen für mich, darunter A83, C08 und C76, während der Gesamtaufwand innerhalb des Limits bleibt. Dieses Angebot ist fair und bringt mir den größten Nutzen.'} 
	  } 
	   } 
	   } 
	  \\ 
 

    \theutterance \stepcounter{utterance}  

    & & \multicolumn{4}{p{0.6\linewidth}}{\cellcolor[rgb]{0.9,0.9,0.9}{%
	\makecell[{{p{\linewidth}}}]{% 
	  \tt {\tiny [A$\langle$GM]}  
	 ABLEHNUNG: {'C98', 'C81', 'C10', 'B38', 'C00', 'C38', 'C62', 'C56', 'A03'}  \\ \tt ARGUMENT: {'Ihr Vorschlag wird abgelehnt, da ich eine bessere Kombination vorschlagen kann, die meinen Nutzen erheblich erhöht, ohne den Maximalaufwand zu überschreiten. Mein Angebot enthält hochwertige Gegenstände, die für mich einen höheren Wert haben, während der Aufwand weiterhin im Rahmen bleibt.'}  \\ \tt VORSCHLAG: {'C76', 'C08', 'A83', 'C10', 'C98', 'C00', 'C38', 'B38', 'C81', 'C62', 'A07'}  \\ \tt ARGUMENT: {'Dieser Vorschlag maximiert meine Wichtigkeit durch die Auswahl von Gegenständen mit dem höchsten Nutzen für mich, darunter A83, C08 und C76, während der Gesamtaufwand innerhalb des Limits bleibt. Dieses Angebot ist fair und bringt mir den größten Nutzen.'} 
	  } 
	   } 
	   } 
	 & \\ 
 

    \theutterance \stepcounter{utterance}  

    & \multicolumn{4}{p{0.6\linewidth}}{\cellcolor[rgb]{0.8,1,0.9}{%
	\makecell[{{p{\linewidth}}}]{% 
	  \tt {\tiny [A$\rangle$GM]}  
	 ABLEHNUNG: {'C76', 'C08', 'A83', 'C10', 'C98', 'C00', 'C38', 'B38', 'C81', 'C62', 'A07'}  \\ \tt ARGUMENT: {'Ihr Vorschlag ist smarter als ich dachte, aber mein eigener Vorschlag bringt mir persönlich mehr Nutzen, indem er eine höhere Wichtigkeit bei gleichem oder geringerem Aufwand bereitstellt. Ich schlage deshalb meine frühHERe Auswahl vor, die eine bessere Balance aus Wichtigkeit und Aufwand bietet.'}  \\ \tt VORSCHLAG: {'C98', 'C10', 'B38', 'C00', 'C38', 'A07', 'C56', 'C62', 'C81'} 
	  } 
	   } 
	   } 
	 & & \\ 
 

    \theutterance \stepcounter{utterance}  

    & & \multicolumn{4}{p{0.6\linewidth}}{\cellcolor[rgb]{0.9,0.9,0.9}{%
	\makecell[{{p{\linewidth}}}]{% 
	  \tt {\tiny [GM$\rangle$B]}  
	 ABLEHNUNG: {'C76', 'C08', 'A83', 'C10', 'C98', 'C00', 'C38', 'B38', 'C81', 'C62', 'A07'}  \\ \tt ARGUMENT: {'Ihr Vorschlag ist smarter als ich dachte, aber mein eigener Vorschlag bringt mir persönlich mehr Nutzen, indem er eine höhere Wichtigkeit bei gleichem oder geringerem Aufwand bereitstellt. Ich schlage deshalb meine frühHERe Auswahl vor, die eine bessere Balance aus Wichtigkeit und Aufwand bietet.'}  \\ \tt VORSCHLAG: {'C98', 'C10', 'B38', 'C00', 'C38', 'A07', 'C56', 'C62', 'C81'} 
	  } 
	   } 
	   } 
	 & \\ 
 

    \theutterance \stepcounter{utterance}  

    & & & \multicolumn{4}{p{0.6\linewidth}}{\cellcolor[rgb]{1,0.85,0.72}{%
	\makecell[{{p{\linewidth}}}]{% 
	  \tt {\tiny [GM$\langle$B]}  
	 ABLEHNUNG: {'C98', 'C81', 'C10', 'B38', 'C00', 'C38', 'C62', 'C56', 'A03'}  \\ \tt ARGUMENT: {'Ihr Vorschlag überschreitet den Maximalaufwand von 2984. Mein Gegenangebot bleibt unter diesem Limit und maximiert meinen Nutzen durch eine effiziente Auswahl von Gegenständen, die ein hohes Maß an Wichtigkeit bei angemessenem Aufwand bieten.'}  \\ \tt VORSCHLAG: {'C76', 'C08', 'A83', 'C10', 'C98', 'C00', 'C38', 'B38', 'C81', 'C03'}  \\ \tt ARGUMENT: {'Dieser Vorschlag maximiert meine Wichtigkeit durch die optimale Auswahl von Gegenständen mit dem höchsten Wert für mich, darunter A83, C08 und C76, während der Gesamtaufwand innerhalb des Limits bleibt. Dieses Angebot ist fair und bringt mir den größten Nutzen unter den gegebenen Bedingungen.'} 
	  } 
	   } 
	   } 
	  \\ 
 

    \theutterance \stepcounter{utterance}  

    & & \multicolumn{4}{p{0.6\linewidth}}{\cellcolor[rgb]{0.9,0.9,0.9}{%
	\makecell[{{p{\linewidth}}}]{% 
	  \tt {\tiny [GM$\rangle$B]}  
	 Du hast einen Vorschlag abgelehnt, der nicht aktiv ist. Vorschläge sind nur dann aktiv, wenn sie vom anderen Spieler mit VORSCHLAG eingebracht wurden und nicht von dir mit ABLEHNUNG deaktiviert wurden. Versuche es noch einmal. 
	  } 
	   } 
	   } 
	 & \\ 
 

    \theutterance \stepcounter{utterance}  

    & & & \multicolumn{4}{p{0.6\linewidth}}{\cellcolor[rgb]{1,0.85,0.72}{%
	\makecell[{{p{\linewidth}}}]{% 
	  \tt {\tiny [GM$\langle$B]}  
	 VORSCHLAG: {'C76', 'C08', 'A83', 'C10', 'C38', 'C00', 'B38', 'C81', 'C62', 'A07', 'C98'}  \\ \tt ARGUMENT: {'Dieser Vorschlag kombiniert Gegenstände mit hoher Wichtigkeit für mich bei gleichzeitig effizientem Aufwand. Die Auswahl umfasst wichtige Gegenstände wie C76, A83 und C08, die meinen Nutzen maximieren, während der Gesamtaufwand unter dem Maximalaufwand von 2984 bleibt. Dieses Angebot ist fair und bringt mir den größten Nutzen unter den gegebenen Bedingungen.'} 
	  } 
	   } 
	   } 
	  \\ 
 

    \theutterance \stepcounter{utterance}  

    & & \multicolumn{4}{p{0.6\linewidth}}{\cellcolor[rgb]{0.9,0.9,0.9}{%
	\makecell[{{p{\linewidth}}}]{% 
	  \tt {\tiny [A$\langle$GM]}  
	 VORSCHLAG: {'C76', 'C08', 'A83', 'C10', 'C38', 'C00', 'B38', 'C81', 'C62', 'A07', 'C98'}  \\ \tt ARGUMENT: {'Dieser Vorschlag kombiniert Gegenstände mit hoher Wichtigkeit für mich bei gleichzeitig effizientem Aufwand. Die Auswahl umfasst wichtige Gegenstände wie C76, A83 und C08, die meinen Nutzen maximieren, während der Gesamtaufwand unter dem Maximalaufwand von 2984 bleibt. Dieses Angebot ist fair und bringt mir den größten Nutzen unter den gegebenen Bedingungen.'} 
	  } 
	   } 
	   } 
	 & \\ 
 

    \theutterance \stepcounter{utterance}  

    & \multicolumn{4}{p{0.6\linewidth}}{\cellcolor[rgb]{0.8,1,0.9}{%
	\makecell[{{p{\linewidth}}}]{% 
	  \tt {\tiny [A$\rangle$GM]}  
	 ABLEHNUNG: {'C98', 'C10', 'B38', 'C00', 'C38', 'A07', 'C56', 'C62', 'C81'}  \\ \tt ARGUMENT: {'Ihr Vorschlag wurde sorgfältig analysiert und abgelehnt, da ich eine bessere Balance aus Wichtigkeit und Aufwand für meine Bedürfnisse finden kann. Mein neuer Vorschlag zielt darauf ab, eine effizientere Auswahl zu treffen, die meinen Nutzen maximiert, ohne den Maximalaufwand zu überschreiten.'}  \\ \tt VORSCHLAG: {'C76', 'C08', 'A83', 'C10', 'C38', 'C00', 'B38', 'C81', 'C62', 'C98'}  \\ \tt ARGUMENT: {'Dieser Vorschlag enthält eine optimierte Auswahl an Gegenständen, die meinen Nutzen maximieren, während der Gesamtaufwand unter dem Maximalaufwand bleibt. DieAufstellung Priorisierung hoch wichtiger Gegenstände wie C76 und A83 sorgt für eine faire und effiziente Lösung für beide Seiten.'} 
	  } 
	   } 
	   } 
	 & & \\ 
 

    \theutterance \stepcounter{utterance}  

    & & \multicolumn{4}{p{0.6\linewidth}}{\cellcolor[rgb]{0.9,0.9,0.9}{%
	\makecell[{{p{\linewidth}}}]{% 
	  \tt {\tiny [A$\langle$GM]}  
	 Du hast einen Vorschlag abgelehnt, der nicht aktiv ist. Vorschläge sind nur dann aktiv, wenn sie vom anderen Spieler mit VORSCHLAG eingebracht wurden und nicht von dir mit ABLEHNUNG deaktiviert wurden. Versuche es noch einmal. 
	  } 
	   } 
	   } 
	 & \\ 
 

    \theutterance \stepcounter{utterance}  

    & \multicolumn{4}{p{0.6\linewidth}}{\cellcolor[rgb]{0.8,1,0.9}{%
	\makecell[{{p{\linewidth}}}]{% 
	  \tt {\tiny [A$\rangle$GM]}  
	 ABLEHNUNG: {'C76', 'C08', 'A83', 'C10', 'C38', 'C00', 'B38', 'C81', 'C62', 'C98'}  \\ \tt ARGUMENT: {'Ihr Vorschlag wurde abgelehnt, da ich eine bessere Kombination vorschlagen kann, die meinen Nutzen erheblich erhöht, ohne den Maximalaufwand zu überschreiten. Mein Angebot enthält hochwertige Gegenstände, die für mich einen höheren Wert haben, während der Aufwand weiterhin im Rahmen bleibt.'}  \\ \tt VORSCHLAG: {'C98', 'C10', 'B38', 'C00', 'C38', 'A07', 'C56', 'C62', 'C81', 'C08'}  \\ \tt ARGUMENT: {'Diese Kombination maximiert meine Wichtigkeit, indem sie die effizientesten Gegenstände beinhaltet, die einen hohen Nutzen bei geringem Aufwand bieten. Der Gesamtaufwand bleibt unter dem Maximalaufwand von 2984, während der Gewinn für mich maximiert wird. Dieses Angebot ist fair und bringt mir den größten Nutzen, während es den Spielregeln entspricht.'} 
	  } 
	   } 
	   } 
	 & & \\ 
 

    \theutterance \stepcounter{utterance}  

    & & \multicolumn{4}{p{0.6\linewidth}}{\cellcolor[rgb]{0.9,0.9,0.9}{%
	\makecell[{{p{\linewidth}}}]{% 
	  \tt {\tiny [A$\langle$GM]}  
	 Du hast einen Vorschlag abgelehnt, der nicht aktiv ist. Vorschläge sind nur dann aktiv, wenn sie vom anderen Spieler mit VORSCHLAG eingebracht wurden und nicht von dir mit ABLEHNUNG deaktiviert wurden. Versuche es noch einmal. 
	  } 
	   } 
	   } 
	 & \\ 
 

    \theutterance \stepcounter{utterance}  

    & \multicolumn{4}{p{0.6\linewidth}}{\cellcolor[rgb]{0.8,1,0.9}{%
	\makecell[{{p{\linewidth}}}]{% 
	  \tt {\tiny [A$\rangle$GM]}  
	 ABLEHNUNG:  \\ \tt ARGUMENT: {'Es scheint, dass es eine Verwirrung gibt. Mein Vorschlag ist aktiv und enthält eine optimierte Auswahl an Gegenständen, die meinen Nutzen maximieren, während der Gesamtaufwand unter dem Maximalaufwand bleibt. Bitte akzeptieren Sie meinen Vorschlag, um das Spiel zu beenden.'}  \\ \tt VORSCHLAG: {'C56', 'C62', 'C10', 'C38', 'C00', 'B38', 'C81', 'C98', 'C08', 'C76'}  \\ \tt ARGUMENT: {'Diese Kombination berücksichtigt eine faire Verteilung mit einer effizienten Ausgewogenheit zwischen Aufwand und Nutzen für mich, während der Maximalaufwand eingehalten wird. Sie enthält Gegenstände, die für mich von hoher Bedeutung sind und gleichzeitig kosteneffizient sind.'} 
	  } 
	   } 
	   } 
	 & & \\ 
 

    \theutterance \stepcounter{utterance}  

    & & & \multicolumn{2}{p{0.3\linewidth}}{\cellcolor[rgb]{0.95,0.95,0.95}{%
	\makecell[{{p{\linewidth}}}]{% 
	  \tt {\tiny [GM$|$GM]}  
	 abort 
	  } 
	   } 
	   } 
	 & & \\ 
 

    \theutterance \stepcounter{utterance}  

    & & & \multicolumn{2}{p{0.3\linewidth}}{\cellcolor[rgb]{0.95,0.95,0.95}{%
	\makecell[{{p{\linewidth}}}]{% 
	  \tt {\tiny [GM$|$GM]}  
	 end game 
	  } 
	   } 
	   } 
	 & & \\ 
 

\end{supertabular}
}

\end{document}
