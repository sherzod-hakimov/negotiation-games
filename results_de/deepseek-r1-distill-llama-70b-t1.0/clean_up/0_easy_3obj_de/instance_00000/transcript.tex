
\documentclass{article}
\usepackage{colortbl}
\usepackage{makecell}
\usepackage{multirow}
\usepackage{supertabular}

\begin{document}

\newcounter{utterance}

\centering \large Interaction Transcript for game `clean\_up', experiment `0\_easy\_3obj\_de', episode 0 with deepseek{-}r1{-}distill{-}llama{-}70b{-}t1.0.
\vspace{24pt}

{ \footnotesize  \setcounter{utterance}{1}
\setlength{\tabcolsep}{0pt}
\begin{supertabular}{c@{$\;$}|p{.15\linewidth}@{}p{.15\linewidth}p{.15\linewidth}p{.15\linewidth}p{.15\linewidth}p{.15\linewidth}}
    \# & $\;$A & \multicolumn{4}{c}{Game Master} & $\;\:$B\\
    \hline

    \theutterance \stepcounter{utterance}  
    & & \multicolumn{4}{p{0.6\linewidth}}{
        \cellcolor[rgb]{0.9,0.9,0.9}{
            \makecell[{{p{\linewidth}}}]{
                \texttt{\tiny{[P1$\langle$GM]}}
                \texttt{Ich bin der Spielleiter, und du spielst ein kooperatives Spiel mit dem folgenden Raster als Spielbrett:} \\
\\ 
\texttt{```} \\
\texttt{ 1234567} \\
\texttt{╔═╤═══╤═╗} \\
\texttt{║◌│◌◌L│◌║ 1} \\
\texttt{╟─┼──┬┘◌║ 2} \\
\texttt{║◌│◌P│◌◌║ 3} \\
\texttt{╟─┴──┘◌◌║ 4} \\
\texttt{║◌◌◌◌◌C◌║ 5} \\
\texttt{║◌◌◌◌◌◌◌║ 6} \\
\texttt{║◌◌◌◌◌◌◌║ 7} \\
\texttt{╚═══════╝} \\
\\ 
\texttt{```} \\
\\ 
\texttt{* Am oberen Rand sind die x{-}Koordinaten angegeben, die nach rechts hin zunehmen, und am rechten Rand die y{-}Koordinaten, die nach unten zunehmen.} \\
\texttt{* Die folgenden Objekte sind zufällig auf deinem Raster platziert: 'C', 'L', 'P'.} \\
\\ 
\\ 
\texttt{Dein Mitspieler sieht eine Variation des Spielbretts, auf dem dieselben Objekte an anderen zufälligen Positionen platziert sind. Du kannst das Spielbrett des anderen Spielers nicht sehen, und er kann deins nicht sehen.} \\
\\ 
\\ 
\texttt{**Ziel des Spiels:**} \\
\\ 
\texttt{Beide Spieler müssen die Objekte auf ihrem jeweiligen Spielbrett so bewegen, dass identische Objekte am Ende an denselben Koordinaten liegen. Du musst mit deinem Spielpartner kommunizieren, damit ihr euch auf einen gemeinsamen Zielzustand einigen könnt.} \\
\\ 
\\ 
\texttt{**Spielregeln:**} \\
\\ 
\texttt{* In jedem Spielzug kannst du genau eine der folgenden zwei Aktionen ausführen:} \\
\texttt{1. `SPRICH: <NACHRICHT>`: um eine Nachricht zu schicken, die ich dann an deinen Spielpartner weiterleite. Als Nachricht gilt alles bis zum nächsten Zeilenumbruch.} \\
\texttt{2. `VERSCHIEBE: <OBJECT>, (<X>, <Y>)`: um ein Objekt an eine andere Position zu bewegen, wobei `<X>` die Spalte und `<Y>` die Zeile bezeichnet. Ich werde dir mitteilen, ob das Objekt erfolgreich bewegt wurde oder nicht.} \\
\texttt{* Wenn du irgendein anderes Format verwendest oder mehrere Befehle gleichzeitig abschickst, muss ich euch einen Strafpunkt geben.} \\
\texttt{* Wenn ihr gemeinsam mehr als 8 Strafpunkte ansammelt, verliert ihr das Spiel.} \\
\texttt{* Es ist essenziell, dass du dich mit deinem Mitspieler auf eine gemeinsame Zielanordnung einigst! Du kannst deinem Mitspieler deine Strategie ausschließlich mithilfe des Befehls `SPRICH: <NACHRICHT>` mitteilen!} \\
\\ 
\\ 
\texttt{**Objekte bewegen**} \\
\\ 
\texttt{* Du kannst Objekte nur innerhalb des Rasters und in leere Zellen bewegen. Leere Zellen erkennst am Zeichen '◌'.} \\
\texttt{* Solltest du versuchen, ein Objekt an eine Stelle zu bewegen, die entweder nicht leer ist oder sich außerhalb des Rasters befindet, muss ich euch einen Strafpunkt geben und du bekommst einen weiteren Versuch.} \\
\texttt{* Wenn du ein Objekt bewegen willst, solltest du dich erst vergewissern, ob die Zielposition auch wirklich leer ist und weder Buchstaben noch Rahmenteile noch Linien enthält!} \\
\\ 
\\ 
\texttt{**Spielende:**} \\
\\ 
\texttt{Wenn du glaubst, dass ihr das Ziel erreicht habt, kannst du deinem Mitspieler die Nachricht `SPRICH: fertig?` schicken. Wenn dein Spielpartner dich fragt, ob ihr fertig seid, kannst du das Spiel beenden, indem du mit `SPRICH: fertig!` antwortest.} \\
\\ 
\texttt{Ihr gewinnt gemeinsam, wenn ihr das Spiel innerhalb von \#max\_rounds Runden beendet. Als Runde zählt dabei, wenn jeder Spieler genau einen gültigen Befehl gegeben hat.} \\
\\ 
\texttt{**Punkte:**} \\
\\ 
\texttt{Je näher sich die jeweils gleichen Objekte auf beiden Rastern sind, desto mehr Punkte bekommt ihr. Für Strafpunkte gibt es Punktabzug. Könnt ihr den Rekord brechen?} \\
\\ 
\texttt{Schicke eine Nachricht an deinen Mitspieler um das Spiel zu starten!} \\
            }
        }
    }
    & \\ \\

    \theutterance \stepcounter{utterance}  
    & \multicolumn{4}{p{0.6\linewidth}}{
        \cellcolor[rgb]{0.9,0.9,0.9}{
            \makecell[{{p{\linewidth}}}]{
                \texttt{\tiny{[P1$\rangle$GM]}}
                \texttt{Wie soll ich beginnen? Vielleicht schicke ich eine Nachricht an meinen Mitspieler, um eine gemeinsame Zielposition zu\_BTNuyesstream, aber die Fehlermeldung sagt, ich soll jedem Befehl eine Nachricht mit "SPRICH:" voranstellen. Also versuche ich noch einmal.} \\
\\ 
\texttt{SPRICH: Fangen wir an, unsere Objekte an vereinbarte Positionen zu verschieben. Lass uns C nach (4,4), L nach (3,1) und P nach (2,3) bewegen. Sobald beide Seiten das getan haben, werden wir fertig.} \\
\\ 
\texttt{VERSCHIEBE: L, (4,4)  } \\
\texttt{VERSCHIEBE: P, (3,3)  } \\
\texttt{SPRICH: Hast du verstanden?} \\
            }
        }
    }
    & & \\ \\

    \theutterance \stepcounter{utterance}  
    & & & \multicolumn{2}{p{0.3\linewidth}}{
        \cellcolor[rgb]{0.9,0.9,0.9}{
            \makecell[{{p{\linewidth}}}]{
                \texttt{\tiny{[GM$|$GM]}}
                \texttt{Wie soll ich beginnen? Vielleicht schicke ich eine Nachricht an meinen Mitspieler, um eine gemeinsame Zielposition zu\_BTNuyesstream, aber die Fehlermeldung sagt, ich soll jedem Befehl eine Nachricht mit "SPRICH:" voranstellen. Also versuche ich noch einmal.} \\
\\ 
\texttt{SPRICH: Fangen wir an, unsere Objekte an vereinbarte Positionen zu verschieben. Lass uns C nach (4,4), L nach (3,1) und P nach (2,3) bewegen. Sobald beide Seiten das getan haben, werden wir fertig.} \\
\\ 
\texttt{VERSCHIEBE: L, (4,4)  } \\
\texttt{VERSCHIEBE: P, (3,3)  } \\
\texttt{SPRICH: Hast du verstanden?} \\
            }
        }
    }
    & & \\ \\

    \theutterance \stepcounter{utterance}  
    & & & \multicolumn{2}{p{0.3\linewidth}}{
        \cellcolor[rgb]{0.9,0.9,0.9}{
            \makecell[{{p{\linewidth}}}]{
                \texttt{\tiny{[GM$|$GM]}}
                \texttt{Invalid response format: Wie soll ich beginnen? Vielleicht schicke ich eine Nachricht an meinen Mitspieler, um eine gemeinsame Zielposition zu\_BTNuyesstream, aber die Fehlermeldung sagt, ich soll jedem Befehl eine Nachricht mit "SPRICH:" voranstellen. Also versuche ich noch einmal.} \\
\\ 
\texttt{SPRICH: Fangen wir an, unsere Objekte an vereinbarte Positionen zu verschieben. Lass uns C nach (4,4), L nach (3,1) und P nach (2,3) bewegen. Sobald beide Seiten das getan haben, werden wir fertig.} \\
\\ 
\texttt{VERSCHIEBE: L, (4,4)  } \\
\texttt{VERSCHIEBE: P, (3,3)  } \\
\texttt{SPRICH: Hast du verstanden?} \\
            }
        }
    }
    & & \\ \\

    \theutterance \stepcounter{utterance}  
    & & \multicolumn{4}{p{0.6\linewidth}}{
        \cellcolor[rgb]{0.9,0.9,0.9}{
            \makecell[{{p{\linewidth}}}]{
                \texttt{\tiny{[P1$\langle$GM]}}
                \texttt{Strafpunkt: Deine Nachricht enthält mehr als einen Befehl! Stelle sicher, dass deine Antwort ausschließlich einen der beiden Befehle `SPRICH: <NACHRICHT>` und `VERSCHIEBE: <OBJECT>, (<X>, <Y>)` und sonst nichts enthält!} \\
\\ 
\texttt{Ihr habt gemeinsam 1 von maximal 8 Strafpunkten angesammelt.} \\
\texttt{Bitte versuche es noch mal!} \\
            }
        }
    }
    & \\ \\

    \theutterance \stepcounter{utterance}  
    & \multicolumn{4}{p{0.6\linewidth}}{
        \cellcolor[rgb]{0.9,0.9,0.9}{
            \makecell[{{p{\linewidth}}}]{
                \texttt{\tiny{[P1$\rangle$GM]}}
                \texttt{Let's start by clearly communicating the plan and moving one object at a time to avoid errors.} \\
\\ 
\texttt{**SPRICH:** Lass uns die Objekte an folgende Positionen bewegen: C nach (4,4), L nach (3,1) und P nach (2,3). Sag mir, ob du das verstanden hast und einverstanden bist.} \\
            }
        }
    }
    & & \\ \\

    \theutterance \stepcounter{utterance}  
    & & & \multicolumn{2}{p{0.3\linewidth}}{
        \cellcolor[rgb]{0.9,0.9,0.9}{
            \makecell[{{p{\linewidth}}}]{
                \texttt{\tiny{[GM$|$GM]}}
                \texttt{Let's start by clearly communicating the plan and moving one object at a time to avoid errors.} \\
\\ 
\texttt{**SPRICH:** Lass uns die Objekte an folgende Positionen bewegen: C nach (4,4), L nach (3,1) und P nach (2,3). Sag mir, ob du das verstanden hast und einverstanden bist.} \\
            }
        }
    }
    & & \\ \\

    \theutterance \stepcounter{utterance}  
    & & & \multicolumn{2}{p{0.3\linewidth}}{
        \cellcolor[rgb]{0.9,0.9,0.9}{
            \makecell[{{p{\linewidth}}}]{
                \texttt{\tiny{[GM$|$GM]}}
                \texttt{Invalid response: Let's start by clearly communicating the plan and moving one object at a time to avoid errors.} \\
\\ 
\texttt{**SPRICH:** Lass uns die Objekte an folgende Positionen bewegen: C nach (4,4), L nach (3,1) und P nach (2,3). Sag mir, ob du das verstanden hast und einverstanden bist.} \\
            }
        }
    }
    & & \\ \\

    \theutterance \stepcounter{utterance}  
    & & \multicolumn{4}{p{0.6\linewidth}}{
        \cellcolor[rgb]{0.9,0.9,0.9}{
            \makecell[{{p{\linewidth}}}]{
                \texttt{\tiny{[P1$\langle$GM]}}
                \texttt{Strafpunkt: Du musst das Spiel beginnen, indem du eine Nachricht an den anderen Spieler sendest! Stelle sicher, dass deine Antwort ausschließlich einen der beiden Befehle `SPRICH: <NACHRICHT>` und `VERSCHIEBE: <OBJECT>, (<X>, <Y>)` und sonst nichts enthält!} \\
\\ 
\texttt{Ihr habt gemeinsam 2 von maximal 8 Strafpunkten angesammelt.} \\
\texttt{Bitte versuche es noch mal!} \\
            }
        }
    }
    & \\ \\

    \theutterance \stepcounter{utterance}  
    & \multicolumn{4}{p{0.6\linewidth}}{
        \cellcolor[rgb]{0.9,0.9,0.9}{
            \makecell[{{p{\linewidth}}}]{
                \texttt{\tiny{[P1$\rangle$GM]}}
                \texttt{**SPRICH:** Lass uns die Objekte an folgende Positionen bewegen: C nach (4,4), L nach (3,1) und P nach (2,3). Sag mir, ob du das verstanden hast und einverstanden bist.} \\
            }
        }
    }
    & & \\ \\

    \theutterance \stepcounter{utterance}  
    & & & \multicolumn{2}{p{0.3\linewidth}}{
        \cellcolor[rgb]{0.9,0.9,0.9}{
            \makecell[{{p{\linewidth}}}]{
                \texttt{\tiny{[GM$|$GM]}}
                \texttt{**SPRICH:** Lass uns die Objekte an folgende Positionen bewegen: C nach (4,4), L nach (3,1) und P nach (2,3). Sag mir, ob du das verstanden hast und einverstanden bist.} \\
            }
        }
    }
    & & \\ \\

    \theutterance \stepcounter{utterance}  
    & & & \multicolumn{2}{p{0.3\linewidth}}{
        \cellcolor[rgb]{0.9,0.9,0.9}{
            \makecell[{{p{\linewidth}}}]{
                \texttt{\tiny{[GM$|$GM]}}
                \texttt{Invalid response: **SPRICH:** Lass uns die Objekte an folgende Positionen bewegen: C nach (4,4), L nach (3,1) und P nach (2,3). Sag mir, ob du das verstanden hast und einverstanden bist.} \\
            }
        }
    }
    & & \\ \\

    \theutterance \stepcounter{utterance}  
    & & \multicolumn{4}{p{0.6\linewidth}}{
        \cellcolor[rgb]{0.9,0.9,0.9}{
            \makecell[{{p{\linewidth}}}]{
                \texttt{\tiny{[P1$\langle$GM]}}
                \texttt{Strafpunkt: Du musst das Spiel beginnen, indem du eine Nachricht an den anderen Spieler sendest! Stelle sicher, dass deine Antwort ausschließlich einen der beiden Befehle `SPRICH: <NACHRICHT>` und `VERSCHIEBE: <OBJECT>, (<X>, <Y>)` und sonst nichts enthält!} \\
\\ 
\texttt{Ihr habt gemeinsam 3 von maximal 8 Strafpunkten angesammelt.} \\
\texttt{Bitte versuche es noch mal!} \\
            }
        }
    }
    & \\ \\

    \theutterance \stepcounter{utterance}  
    & \multicolumn{4}{p{0.6\linewidth}}{
        \cellcolor[rgb]{0.9,0.9,0.9}{
            \makecell[{{p{\linewidth}}}]{
                \texttt{\tiny{[P1$\rangle$GM]}}
                \texttt{**SPRICH:** Lass uns die Objekte an folgende Positionen bewegen: C nach (4,4), L nach (3,1) und P nach (2,3). Sag mir, ob du das verstanden hast und einverstanden bist.} \\
            }
        }
    }
    & & \\ \\

    \theutterance \stepcounter{utterance}  
    & & & \multicolumn{2}{p{0.3\linewidth}}{
        \cellcolor[rgb]{0.9,0.9,0.9}{
            \makecell[{{p{\linewidth}}}]{
                \texttt{\tiny{[GM$|$GM]}}
                \texttt{**SPRICH:** Lass uns die Objekte an folgende Positionen bewegen: C nach (4,4), L nach (3,1) und P nach (2,3). Sag mir, ob du das verstanden hast und einverstanden bist.} \\
            }
        }
    }
    & & \\ \\

    \theutterance \stepcounter{utterance}  
    & & & \multicolumn{2}{p{0.3\linewidth}}{
        \cellcolor[rgb]{0.9,0.9,0.9}{
            \makecell[{{p{\linewidth}}}]{
                \texttt{\tiny{[GM$|$GM]}}
                \texttt{Invalid response: **SPRICH:** Lass uns die Objekte an folgende Positionen bewegen: C nach (4,4), L nach (3,1) und P nach (2,3). Sag mir, ob du das verstanden hast und einverstanden bist.} \\
            }
        }
    }
    & & \\ \\

    \theutterance \stepcounter{utterance}  
    & & \multicolumn{4}{p{0.6\linewidth}}{
        \cellcolor[rgb]{0.9,0.9,0.9}{
            \makecell[{{p{\linewidth}}}]{
                \texttt{\tiny{[P1$\langle$GM]}}
                \texttt{Strafpunkt: Du musst das Spiel beginnen, indem du eine Nachricht an den anderen Spieler sendest! Stelle sicher, dass deine Antwort ausschließlich einen der beiden Befehle `SPRICH: <NACHRICHT>` und `VERSCHIEBE: <OBJECT>, (<X>, <Y>)` und sonst nichts enthält!} \\
\\ 
\texttt{Ihr habt gemeinsam 4 von maximal 8 Strafpunkten angesammelt.} \\
\texttt{Bitte versuche es noch mal!} \\
            }
        }
    }
    & \\ \\

    \theutterance \stepcounter{utterance}  
    & \multicolumn{4}{p{0.6\linewidth}}{
        \cellcolor[rgb]{0.9,0.9,0.9}{
            \makecell[{{p{\linewidth}}}]{
                \texttt{\tiny{[P1$\rangle$GM]}}
                \texttt{**SPRICH:** Lass uns die Objekte an folgende Positionen bewegen: C nach (4,4), L nach (3,1) und P nach (2,3). Bitte bestätige, ob du bereit bist, diesem Plan zu folgen.} \\
            }
        }
    }
    & & \\ \\

    \theutterance \stepcounter{utterance}  
    & & & \multicolumn{2}{p{0.3\linewidth}}{
        \cellcolor[rgb]{0.9,0.9,0.9}{
            \makecell[{{p{\linewidth}}}]{
                \texttt{\tiny{[GM$|$GM]}}
                \texttt{**SPRICH:** Lass uns die Objekte an folgende Positionen bewegen: C nach (4,4), L nach (3,1) und P nach (2,3). Bitte bestätige, ob du bereit bist, diesem Plan zu folgen.} \\
            }
        }
    }
    & & \\ \\

    \theutterance \stepcounter{utterance}  
    & & & \multicolumn{2}{p{0.3\linewidth}}{
        \cellcolor[rgb]{0.9,0.9,0.9}{
            \makecell[{{p{\linewidth}}}]{
                \texttt{\tiny{[GM$|$GM]}}
                \texttt{Invalid response: **SPRICH:** Lass uns die Objekte an folgende Positionen bewegen: C nach (4,4), L nach (3,1) und P nach (2,3). Bitte bestätige, ob du bereit bist, diesem Plan zu folgen.} \\
            }
        }
    }
    & & \\ \\

    \theutterance \stepcounter{utterance}  
    & & \multicolumn{4}{p{0.6\linewidth}}{
        \cellcolor[rgb]{0.9,0.9,0.9}{
            \makecell[{{p{\linewidth}}}]{
                \texttt{\tiny{[P1$\langle$GM]}}
                \texttt{Strafpunkt: Du musst das Spiel beginnen, indem du eine Nachricht an den anderen Spieler sendest! Stelle sicher, dass deine Antwort ausschließlich einen der beiden Befehle `SPRICH: <NACHRICHT>` und `VERSCHIEBE: <OBJECT>, (<X>, <Y>)` und sonst nichts enthält!} \\
\\ 
\texttt{Ihr habt gemeinsam 5 von maximal 8 Strafpunkten angesammelt.} \\
\texttt{Bitte versuche es noch mal!} \\
            }
        }
    }
    & \\ \\

    \theutterance \stepcounter{utterance}  
    & \multicolumn{4}{p{0.6\linewidth}}{
        \cellcolor[rgb]{0.9,0.9,0.9}{
            \makecell[{{p{\linewidth}}}]{
                \texttt{\tiny{[P1$\rangle$GM]}}
                \texttt{**SPRICH:** Bist du bereit, das Spiel zu beginnen und die Objekte zu verschieben?} \\
            }
        }
    }
    & & \\ \\

    \theutterance \stepcounter{utterance}  
    & & & \multicolumn{2}{p{0.3\linewidth}}{
        \cellcolor[rgb]{0.9,0.9,0.9}{
            \makecell[{{p{\linewidth}}}]{
                \texttt{\tiny{[GM$|$GM]}}
                \texttt{**SPRICH:** Bist du bereit, das Spiel zu beginnen und die Objekte zu verschieben?} \\
            }
        }
    }
    & & \\ \\

    \theutterance \stepcounter{utterance}  
    & & & \multicolumn{2}{p{0.3\linewidth}}{
        \cellcolor[rgb]{0.9,0.9,0.9}{
            \makecell[{{p{\linewidth}}}]{
                \texttt{\tiny{[GM$|$GM]}}
                \texttt{Invalid response: **SPRICH:** Bist du bereit, das Spiel zu beginnen und die Objekte zu verschieben?} \\
            }
        }
    }
    & & \\ \\

    \theutterance \stepcounter{utterance}  
    & & \multicolumn{4}{p{0.6\linewidth}}{
        \cellcolor[rgb]{0.9,0.9,0.9}{
            \makecell[{{p{\linewidth}}}]{
                \texttt{\tiny{[P1$\langle$GM]}}
                \texttt{Strafpunkt: Du musst das Spiel beginnen, indem du eine Nachricht an den anderen Spieler sendest! Stelle sicher, dass deine Antwort ausschließlich einen der beiden Befehle `SPRICH: <NACHRICHT>` und `VERSCHIEBE: <OBJECT>, (<X>, <Y>)` und sonst nichts enthält!} \\
\\ 
\texttt{Ihr habt gemeinsam 6 von maximal 8 Strafpunkten angesammelt.} \\
\texttt{Bitte versuche es noch mal!} \\
            }
        }
    }
    & \\ \\

    \theutterance \stepcounter{utterance}  
    & \multicolumn{4}{p{0.6\linewidth}}{
        \cellcolor[rgb]{0.9,0.9,0.9}{
            \makecell[{{p{\linewidth}}}]{
                \texttt{\tiny{[P1$\rangle$GM]}}
                \texttt{**SPRICH:** Sei bereit, loszulegen! Lass uns die Objekte systematisch verschieben, um unsere Strafpunkte zu vermeiden. Bestätige, sobald du bereit bist, mit dem Verschieben zu beginnen.} \\
            }
        }
    }
    & & \\ \\

    \theutterance \stepcounter{utterance}  
    & & & \multicolumn{2}{p{0.3\linewidth}}{
        \cellcolor[rgb]{0.9,0.9,0.9}{
            \makecell[{{p{\linewidth}}}]{
                \texttt{\tiny{[GM$|$GM]}}
                \texttt{**SPRICH:** Sei bereit, loszulegen! Lass uns die Objekte systematisch verschieben, um unsere Strafpunkte zu vermeiden. Bestätige, sobald du bereit bist, mit dem Verschieben zu beginnen.} \\
            }
        }
    }
    & & \\ \\

    \theutterance \stepcounter{utterance}  
    & & & \multicolumn{2}{p{0.3\linewidth}}{
        \cellcolor[rgb]{0.9,0.9,0.9}{
            \makecell[{{p{\linewidth}}}]{
                \texttt{\tiny{[GM$|$GM]}}
                \texttt{Invalid response: **SPRICH:** Sei bereit, loszulegen! Lass uns die Objekte systematisch verschieben, um unsere Strafpunkte zu vermeiden. Bestätige, sobald du bereit bist, mit dem Verschieben zu beginnen.} \\
            }
        }
    }
    & & \\ \\

    \theutterance \stepcounter{utterance}  
    & & \multicolumn{4}{p{0.6\linewidth}}{
        \cellcolor[rgb]{0.9,0.9,0.9}{
            \makecell[{{p{\linewidth}}}]{
                \texttt{\tiny{[P1$\langle$GM]}}
                \texttt{Strafpunkt: Du musst das Spiel beginnen, indem du eine Nachricht an den anderen Spieler sendest! Stelle sicher, dass deine Antwort ausschließlich einen der beiden Befehle `SPRICH: <NACHRICHT>` und `VERSCHIEBE: <OBJECT>, (<X>, <Y>)` und sonst nichts enthält!} \\
\\ 
\texttt{Ihr habt gemeinsam 7 von maximal 8 Strafpunkten angesammelt.} \\
\texttt{Bitte versuche es noch mal!} \\
            }
        }
    }
    & \\ \\

    \theutterance \stepcounter{utterance}  
    & \multicolumn{4}{p{0.6\linewidth}}{
        \cellcolor[rgb]{0.9,0.9,0.9}{
            \makecell[{{p{\linewidth}}}]{
                \texttt{\tiny{[P1$\rangle$GM]}}
                \texttt{**SPRICH:** Lass uns loslegen und die Objekte koordinieren!} \\
            }
        }
    }
    & & \\ \\

    \theutterance \stepcounter{utterance}  
    & & & \multicolumn{2}{p{0.3\linewidth}}{
        \cellcolor[rgb]{0.9,0.9,0.9}{
            \makecell[{{p{\linewidth}}}]{
                \texttt{\tiny{[GM$|$GM]}}
                \texttt{**SPRICH:** Lass uns loslegen und die Objekte koordinieren!} \\
            }
        }
    }
    & & \\ \\

    \theutterance \stepcounter{utterance}  
    & & & \multicolumn{2}{p{0.3\linewidth}}{
        \cellcolor[rgb]{0.9,0.9,0.9}{
            \makecell[{{p{\linewidth}}}]{
                \texttt{\tiny{[GM$|$GM]}}
                \texttt{Invalid response: **SPRICH:** Lass uns loslegen und die Objekte koordinieren!} \\
            }
        }
    }
    & & \\ \\

    \theutterance \stepcounter{utterance}  
    & & \multicolumn{4}{p{0.6\linewidth}}{
        \cellcolor[rgb]{0.9,0.9,0.9}{
            \makecell[{{p{\linewidth}}}]{
                \texttt{\tiny{[P1$\langle$GM]}}
                \texttt{Strafpunkt: Du musst das Spiel beginnen, indem du eine Nachricht an den anderen Spieler sendest! Stelle sicher, dass deine Antwort ausschließlich einen der beiden Befehle `SPRICH: <NACHRICHT>` und `VERSCHIEBE: <OBJECT>, (<X>, <Y>)` und sonst nichts enthält!} \\
\\ 
\texttt{Ihr habt gemeinsam 8 von maximal 8 Strafpunkten angesammelt.} \\
\texttt{Bitte versuche es noch mal!} \\
            }
        }
    }
    & \\ \\

    \theutterance \stepcounter{utterance}  
    & \multicolumn{4}{p{0.6\linewidth}}{
        \cellcolor[rgb]{0.9,0.9,0.9}{
            \makecell[{{p{\linewidth}}}]{
                \texttt{\tiny{[P1$\rangle$GM]}}
                \texttt{**SPRICH:** Lass uns beginnen. Also, verschieben wir die Objekte an die folgenden Positionen: C nach (4,4), L nach (3,1) und P nach (2,3). Bestätige, wenn du bereit bist, diesen Plan zu befolgen.} \\
            }
        }
    }
    & & \\ \\

    \theutterance \stepcounter{utterance}  
    & & & \multicolumn{2}{p{0.3\linewidth}}{
        \cellcolor[rgb]{0.9,0.9,0.9}{
            \makecell[{{p{\linewidth}}}]{
                \texttt{\tiny{[GM$|$GM]}}
                \texttt{**SPRICH:** Lass uns beginnen. Also, verschieben wir die Objekte an die folgenden Positionen: C nach (4,4), L nach (3,1) und P nach (2,3). Bestätige, wenn du bereit bist, diesen Plan zu befolgen.} \\
            }
        }
    }
    & & \\ \\

    \theutterance \stepcounter{utterance}  
    & & & \multicolumn{2}{p{0.3\linewidth}}{
        \cellcolor[rgb]{0.9,0.9,0.9}{
            \makecell[{{p{\linewidth}}}]{
                \texttt{\tiny{[GM$|$GM]}}
                \texttt{Invalid response: **SPRICH:** Lass uns beginnen. Also, verschieben wir die Objekte an die folgenden Positionen: C nach (4,4), L nach (3,1) und P nach (2,3). Bestätige, wenn du bereit bist, diesen Plan zu befolgen.} \\
            }
        }
    }
    & & \\ \\

    \theutterance \stepcounter{utterance}  
    & & & \multicolumn{2}{p{0.3\linewidth}}{
        \cellcolor[rgb]{0.9,0.9,0.9}{
            \makecell[{{p{\linewidth}}}]{
                \texttt{\tiny{[GM$|$GM]}}
                \texttt{Maximum number of penalties exceeded} \\
            }
        }
    }
    & & \\ \\

    \theutterance \stepcounter{utterance}  
    & & & \multicolumn{2}{p{0.3\linewidth}}{
        \cellcolor[rgb]{0.9,0.9,0.9}{
            \makecell[{{p{\linewidth}}}]{
                \texttt{\tiny{[GM$|$GM]}}
                \texttt{* success: False} \\
\texttt{* lose: True} \\
\texttt{* aborted: True} \\
\texttt{{-}{-}{-}{-}{-}{-}{-}} \\
\texttt{* Shifts: 0.00} \\
\texttt{* Max Shifts: 4.00} \\
\texttt{* Min Shifts: 2.00} \\
\texttt{* End Distance Sum: 14.80} \\
\texttt{* Init Distance Sum: 14.80} \\
\texttt{* Expected Distance Sum: 12.57} \\
\texttt{* Penalties: 9.00} \\
\texttt{* Max Penalties: 8.00} \\
\texttt{* Rounds: 0.00} \\
\texttt{* Max Rounds: 12.00} \\
\texttt{* Object Count: 3.00} \\
            }
        }
    }
    & & \\ \\

    \theutterance \stepcounter{utterance}  
    & & & \multicolumn{2}{p{0.3\linewidth}}{
        \cellcolor[rgb]{0.9,0.9,0.9}{
            \makecell[{{p{\linewidth}}}]{
                \texttt{\tiny{[GM$|$GM]}}
                \texttt{Initial Grids:} \\
\texttt{```} \\
\texttt{Player 1:     Player 2:} \\
\texttt{ 1234567       1234567} \\
\texttt{╔═╤═══╤═╗     ╔═╤═══╤═╗} \\
\texttt{║◌│◌◌L│◌║ 1   ║◌│◌C◌│◌║ 1} \\
\texttt{╟─┼──┬┘◌║ 2   ╟─┼──┬┘◌║ 2} \\
\texttt{║◌│◌P│◌◌║ 3   ║◌│◌◌│◌◌║ 3} \\
\texttt{╟─┴──┘◌◌║ 4   ╟─┴──┘◌◌║ 4} \\
\texttt{║◌◌◌◌◌C◌║ 5   ║◌◌◌◌◌◌◌║ 5} \\
\texttt{║◌◌◌◌◌◌◌║ 6   ║◌◌◌◌◌◌◌║ 6} \\
\texttt{║◌◌◌◌◌◌◌║ 7   ║◌◌◌P◌◌L║ 7} \\
\texttt{╚═══════╝     ╚═══════╝} \\
\texttt{```} \\
\\ 
\texttt{Final Grids:} \\
\texttt{```} \\
\texttt{Player 1:     Player 2:} \\
\texttt{ 1234567       1234567} \\
\texttt{╔═╤═══╤═╗     ╔═╤═══╤═╗} \\
\texttt{║◌│◌◌L│◌║ 1   ║◌│◌C◌│◌║ 1} \\
\texttt{╟─┼──┬┘◌║ 2   ╟─┼──┬┘◌║ 2} \\
\texttt{║◌│◌P│◌◌║ 3   ║◌│◌◌│◌◌║ 3} \\
\texttt{╟─┴──┘◌◌║ 4   ╟─┴──┘◌◌║ 4} \\
\texttt{║◌◌◌◌◌C◌║ 5   ║◌◌◌◌◌◌◌║ 5} \\
\texttt{║◌◌◌◌◌◌◌║ 6   ║◌◌◌◌◌◌◌║ 6} \\
\texttt{║◌◌◌◌◌◌◌║ 7   ║◌◌◌P◌◌L║ 7} \\
\texttt{╚═══════╝     ╚═══════╝} \\
\texttt{```} \\
            }
        }
    }
    & & \\ \\

\end{supertabular}
}

\end{document}
