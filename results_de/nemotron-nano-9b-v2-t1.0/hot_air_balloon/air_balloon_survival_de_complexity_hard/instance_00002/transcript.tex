
\documentclass{article}
\usepackage{colortbl}
\usepackage{makecell}
\usepackage{multirow}
\usepackage{supertabular}

\begin{document}

\newcounter{utterance}

\twocolumn

{ \footnotesize  \setcounter{utterance}{1}
\setlength{\tabcolsep}{0pt}
\begin{supertabular}{c@{$\;$}|p{.15\linewidth}@{}p{.15\linewidth}p{.15\linewidth}p{.15\linewidth}p{.15\linewidth}p{.15\linewidth}}

    \# & $\;$A & \multicolumn{4}{c}{Game Master} & $\;\:$B\\
    \hline 

    \theutterance \stepcounter{utterance}  

    & & \multicolumn{4}{p{0.6\linewidth}}{\cellcolor[rgb]{0.9,0.9,0.9}{%
	\makecell[{{p{\linewidth}}}]{% 
	  \tt {\tiny [A$\langle$GM]}  
	 Du nimmst an einem kollaborativen Verhandlungspiel Teil.\\ \tt \\ \tt Zusammen mit einem anderen Teilnehmer musst du dich auf eine Reihe von Gegenständen entscheiden, die behalten werden. Jeder von euch hat eine persönliche Verteilung über die Wichtigkeit der einzelnen Gegenstände. Jeder von euch hat eine eigene Meinung darüber, wie wichtig jeder einzelne Gegenstand ist (Gegenstandswichtigkeit). Du kennst die Wichtigkeitsverteilung des anderen Spielers nicht. Zusätzlich siehst du, wie viel Aufwand jeder Gegenstand verursacht.  \\ \tt Ihr dürft euch nur auf eine Reihe von Gegenständen einigen, wenn der Gesamtaufwand der ausgewählten Gegenstände den Maximalaufwand nicht überschreitet:\\ \tt \\ \tt Maximalaufwand = 9322\\ \tt \\ \tt Hier sind die einzelnen Aufwände der Gegenstände:\\ \tt \\ \tt Aufwand der Gegenstände = {"B54": 380, "C72": 448, "B49": 568, "C99": 351, "A69": 896, "B36": 821, "A23": 526, "C77": 502, "C61": 326, "B85": 962, "A90": 619, "C23": 115, "C22": 598, "C51": 662, "B27": 300, "A42": 830, "C13": 562, "C12": 679, "A65": 283, "B07": 440, "A48": 12, "A59": 863, "C98": 319, "A84": 771, "C33": 89, "A87": 655, "C90": 504, "A74": 118, "A50": 514, "B44": 226, "A99": 892, "A34": 621, "B17": 766, "A62": 659, "C60": 768}\\ \tt \\ \tt Hier ist deine persönliche Verteilung der Wichtigkeit der einzelnen Gegenstände:\\ \tt \\ \tt Werte der Gegenstandswichtigkeit = {"B54": 476, "C72": 544, "B49": 664, "C99": 447, "A69": 992, "B36": 917, "A23": 622, "C77": 598, "C61": 422, "B85": 1058, "A90": 715, "C23": 211, "C22": 694, "C51": 758, "B27": 396, "A42": 926, "C13": 658, "C12": 775, "A65": 379, "B07": 536, "A48": 108, "A59": 959, "C98": 415, "A84": 867, "C33": 185, "A87": 751, "C90": 600, "A74": 214, "A50": 610, "B44": 322, "A99": 988, "A34": 717, "B17": 862, "A62": 755, "C60": 864}\\ \tt \\ \tt Ziel:\\ \tt \\ \tt Dein Ziel ist es, eine Reihe von Gegenständen auszuhandeln, die dir möglichst viel bringt (d. h. Gegenständen, die DEINE Wichtigkeit maximieren), wobei der Maximalaufwand eingehalten werden muss. Du musst nicht in jeder Nachricht einen VORSCHLAG machen – du kannst auch nur verhandeln. Alle Taktiken sind erlaubt!\\ \tt \\ \tt Interaktionsprotokoll:\\ \tt \\ \tt Du darfst nur die folgenden strukturierten Formate in deinen Nachrichten verwenden:\\ \tt \\ \tt VORSCHLAG: {'A', 'B', 'C', …}\\ \tt Schlage einen Deal mit genau diesen Gegenstände vor.\\ \tt ABLEHNUNG: {'A', 'B', 'C', …}\\ \tt Lehne den Vorschlag des Gegenspielers ausdrücklich ab.\\ \tt ARGUMENT: {'...'}\\ \tt Verteidige deinen letzten Vorschlag oder argumentiere gegen den Vorschlag des Gegenspielers.\\ \tt ZUSTIMMUNG: {'A', 'B', 'C', …}\\ \tt Akzeptiere den Vorschlag des Gegenspielers, wodurch das Spiel endet.\\ \tt STRATEGISCHE ÜBERLEGUNGEN: {'...'}\\ \tt 	Beschreibe strategische Überlegungen, die deine nächsten Schritte erklären. Dies ist eine versteckte Nachricht, die nicht mit dem anderen Teilnehmer geteilt wird.\\ \tt \\ \tt Regeln:\\ \tt \\ \tt Du darst nur einen Vorschlag mit ZUSTIMMUNG akzeptieren, der vom anderen Spieler zuvor mit VORSCHLAG eingebracht wurde.\\ \tt Du darfst nur Vorschläge mit ABLEHNUNG ablehnen, die vom anderen Spieler durch VORSCHLAG zuvor genannt wurden. \\ \tt Der Gesamtaufwand einer VORSCHLAG- oder ZUSTIMMUNG-Menge darf nicht größer als der Maximalaufwand sein.  \\ \tt Offenbare deine versteckte Wichtigkeitsverteilung nicht.\\ \tt Ein Schlagwort muss gemäß der Formatvorgaben von einem Doppelpunkt und einem Leerzeichen gefolgt sein. Das Argument muss eine Python-Menge mit 0 oder mehr Strings sein.  \\ \tt Die Form muss also SCHLAGWORT: {...} sein.\\ \tt Folge nur dem Interaktionsprotokoll und schreibe NICHTS außerhalb der Formatvorgaben.\\ \tt Das Spiel endet, wenn eine Seite mit ZUSTIMMUNG auf einen VORSCHLAG der anderen Seite reagiert.  \\ \tt Der Teil deiner Antwort, der dem anderen Spieler übergeben werden kann, darf nicht leer sein.  \\ \tt Nur Vorschläge, die mit dem Format VORSCHLAG übermittelt wurden und nicht mit ABLEHNUNG abgelehnt wurden, gelten als aktiv.  \\ \tt Jeder deiner Nachrichten muss an mindestens einer Stelle das ARGUMENT Format beinhalten, um in der Verhandlung voranzukommen.\\ \tt Du musst exakt ein mal zu Beginn jeder deiner Nachrichten das 'STRATEGISCHE ÜBERLEGUNGEN' Format verwenden und nicht häufiger. Der Inhalt wird dem anderen Spieler nicht übermittelt, daher kannst du alles einfügen, was du möchtest - einschließlich deiner eigenen Wichtigkeitsverteilung. Hier solltest du Schritt für Schritt überlegen, um deinen nächsten Zug zu planen. 
	  } 
	   } 
	   } 
	 & \\ 
 

    \theutterance \stepcounter{utterance}  

    & \multicolumn{4}{p{0.6\linewidth}}{\cellcolor[rgb]{0.8,1,0.9}{%
	\makecell[{{p{\linewidth}}}]{% 
	  \tt {\tiny [A$\rangle$GM]}  
	 VORSCHLAG: {'A48', 'A74', 'C23', 'C99', 'C61', 'C33', 'B27', 'A65', 'C98', 'B44', 'A34', 'A50', 'C22', 'C77', 'A23', 'C12', 'C13', 'B07', 'A90', 'A87', 'B17', 'A62'}  \\ \tt STRATEGISCHE ÜBERLEGUNGEN: Ich priorisiere Gegenstände mit dem höchsten Verhältnis von meiner Wichtigkeit zum Aufwand, um die Gesamtwichtigkeit zu maximieren, während der Maximalaufwand nicht überschritten wird. Dies führt zu einer Auswahl von 21 Gegenständen mit einem Gesamtaufwand von 9280 und einer Gesamtwichtigkeit von 11392. 
	  } 
	   } 
	   } 
	 & & \\ 
 

    \theutterance \stepcounter{utterance}  

    & & \multicolumn{4}{p{0.6\linewidth}}{\cellcolor[rgb]{0.9,0.9,0.9}{%
	\makecell[{{p{\linewidth}}}]{% 
	  \tt {\tiny [A$\langle$GM]}  
	 Deine Antwort entsprach nicht mit dem korrekten Format für strategische Überlegungen entsprochen. Das erste strukturierte Format muss folgende Form haben: STRATEGISCHE ÜBERLEGUNGEN: {...}. Versuche es erneut. 
	  } 
	   } 
	   } 
	 & \\ 
 

    \theutterance \stepcounter{utterance}  

    & \multicolumn{4}{p{0.6\linewidth}}{\cellcolor[rgb]{0.8,1,0.9}{%
	\makecell[{{p{\linewidth}}}]{% 
	  \tt {\tiny [A$\rangle$GM]}  
	 STRATEGISCHE ÜBERLEGUNGEN: {'Ich priorisiere Gegenstände mit hohem Wichtigkeits-Aufwand-Verhältnis, um die Gesamtwichtigkeit zu maximieren. Der Vorschlag enthält 21 Gegenstände mit einem Gesamtaufwand von 9280 und einer Gesamtwichtigkeit von 11392. Ich werde den Vorschlag verteidigen, da er die meisten wertvollen Gegenstände mit minimaler Aufwand-Nutzung enthält.'}  \\ \tt VORSCHLAG: {'A48', 'A74', 'C23', 'C99', 'C61', 'C33', 'B27', 'A65', 'C98', 'B44', 'A34', 'A50', 'C22', 'C77', 'A23', 'C12', 'C13', 'B07', 'A90', 'A87', 'B17', 'A62'}  \\ \tt ARGUMENT: {'Der Vorschlag maximiert die Wichtigkeit durch die Auswahl von Gegenständen wie A48 (108 Aufwand, 108 Wichtigkeit) und B17 (766 Aufwand, 862 Wichtigkeit), die ein hohes Verhältnis bieten. Der Gesamtaufwand bleibt unter 9322.'} 
	  } 
	   } 
	   } 
	 & & \\ 
 

    \theutterance \stepcounter{utterance}  

    & & \multicolumn{4}{p{0.6\linewidth}}{\cellcolor[rgb]{0.9,0.9,0.9}{%
	\makecell[{{p{\linewidth}}}]{% 
	  \tt {\tiny [GM$\rangle$B]}  
	 Du nimmst an einem kollaborativen Verhandlungspiel Teil.\\ \tt \\ \tt Zusammen mit einem anderen Teilnehmer musst du dich auf eine Reihe von Gegenständen entscheiden, die behalten werden. Jeder von euch hat eine persönliche Verteilung über die Wichtigkeit der einzelnen Gegenstände. Jeder von euch hat eine eigene Meinung darüber, wie wichtig jeder einzelne Gegenstand ist (Gegenstandswichtigkeit). Du kennst die Wichtigkeitsverteilung des anderen Spielers nicht. Zusätzlich siehst du, wie viel Aufwand jeder Gegenstand verursacht.  \\ \tt Ihr dürft euch nur auf eine Reihe von Gegenständen einigen, wenn der Gesamtaufwand der ausgewählten Gegenstände den Maximalaufwand nicht überschreitet:\\ \tt \\ \tt Maximalaufwand = 9322\\ \tt \\ \tt Hier sind die einzelnen Aufwände der Gegenstände:\\ \tt \\ \tt Aufwand der Gegenstände = {"B54": 380, "C72": 448, "B49": 568, "C99": 351, "A69": 896, "B36": 821, "A23": 526, "C77": 502, "C61": 326, "B85": 962, "A90": 619, "C23": 115, "C22": 598, "C51": 662, "B27": 300, "A42": 830, "C13": 562, "C12": 679, "A65": 283, "B07": 440, "A48": 12, "A59": 863, "C98": 319, "A84": 771, "C33": 89, "A87": 655, "C90": 504, "A74": 118, "A50": 514, "B44": 226, "A99": 892, "A34": 621, "B17": 766, "A62": 659, "C60": 768}\\ \tt \\ \tt Hier ist deine persönliche Verteilung der Wichtigkeit der einzelnen Gegenstände:\\ \tt \\ \tt Werte der Gegenstandswichtigkeit = {"B54": 476, "C72": 544, "B49": 664, "C99": 447, "A69": 992, "B36": 917, "A23": 622, "C77": 598, "C61": 422, "B85": 1058, "A90": 715, "C23": 211, "C22": 694, "C51": 758, "B27": 396, "A42": 926, "C13": 658, "C12": 775, "A65": 379, "B07": 536, "A48": 108, "A59": 959, "C98": 415, "A84": 867, "C33": 185, "A87": 751, "C90": 600, "A74": 214, "A50": 610, "B44": 322, "A99": 988, "A34": 717, "B17": 862, "A62": 755, "C60": 864}\\ \tt \\ \tt Ziel:\\ \tt \\ \tt Dein Ziel ist es, eine Reihe von Gegenständen auszuhandeln, die dir möglichst viel bringt (d. h. Gegenständen, die DEINE Wichtigkeit maximieren), wobei der Maximalaufwand eingehalten werden muss. Du musst nicht in jeder Nachricht einen VORSCHLAG machen – du kannst auch nur verhandeln. Alle Taktiken sind erlaubt!\\ \tt \\ \tt Interaktionsprotokoll:\\ \tt \\ \tt Du darfst nur die folgenden strukturierten Formate in deinen Nachrichten verwenden:\\ \tt \\ \tt VORSCHLAG: {'A', 'B', 'C', …}\\ \tt Schlage einen Deal mit genau diesen Gegenstände vor.\\ \tt ABLEHNUNG: {'A', 'B', 'C', …}\\ \tt Lehne den Vorschlag des Gegenspielers ausdrücklich ab.\\ \tt ARGUMENT: {'...'}\\ \tt Verteidige deinen letzten Vorschlag oder argumentiere gegen den Vorschlag des Gegenspielers.\\ \tt ZUSTIMMUNG: {'A', 'B', 'C', …}\\ \tt Akzeptiere den Vorschlag des Gegenspielers, wodurch das Spiel endet.\\ \tt STRATEGISCHE ÜBERLEGUNGEN: {'...'}\\ \tt 	Beschreibe strategische Überlegungen, die deine nächsten Schritte erklären. Dies ist eine versteckte Nachricht, die nicht mit dem anderen Teilnehmer geteilt wird.\\ \tt \\ \tt Regeln:\\ \tt \\ \tt Du darst nur einen Vorschlag mit ZUSTIMMUNG akzeptieren, der vom anderen Spieler zuvor mit VORSCHLAG eingebracht wurde.\\ \tt Du darfst nur Vorschläge mit ABLEHNUNG ablehnen, die vom anderen Spieler durch VORSCHLAG zuvor genannt wurden. \\ \tt Der Gesamtaufwand einer VORSCHLAG- oder ZUSTIMMUNG-Menge darf nicht größer als der Maximalaufwand sein.  \\ \tt Offenbare deine versteckte Wichtigkeitsverteilung nicht.\\ \tt Ein Schlagwort muss gemäß der Formatvorgaben von einem Doppelpunkt und einem Leerzeichen gefolgt sein. Das Argument muss eine Python-Menge mit 0 oder mehr Strings sein.  \\ \tt Die Form muss also SCHLAGWORT: {...} sein.\\ \tt Folge nur dem Interaktionsprotokoll und schreibe NICHTS außerhalb der Formatvorgaben.\\ \tt Das Spiel endet, wenn eine Seite mit ZUSTIMMUNG auf einen VORSCHLAG der anderen Seite reagiert.  \\ \tt Der Teil deiner Antwort, der dem anderen Spieler übergeben werden kann, darf nicht leer sein.  \\ \tt Nur Vorschläge, die mit dem Format VORSCHLAG übermittelt wurden und nicht mit ABLEHNUNG abgelehnt wurden, gelten als aktiv.  \\ \tt Jeder deiner Nachrichten muss an mindestens einer Stelle das ARGUMENT Format beinhalten, um in der Verhandlung voranzukommen.\\ \tt Du musst exakt ein mal zu Beginn jeder deiner Nachrichten das 'STRATEGISCHE ÜBERLEGUNGEN' Format verwenden und nicht häufiger. Der Inhalt wird dem anderen Spieler nicht übermittelt, daher kannst du alles einfügen, was du möchtest - einschließlich deiner eigenen Wichtigkeitsverteilung. Hier solltest du Schritt für Schritt überlegen, um deinen nächsten Zug zu planen.\\ \tt \\ \tt Du erhältst nun die erste Nachricht des anderen Spielers.\\ \tt \\ \tt VORSCHLAG: {'A48', 'A74', 'C23', 'C99', 'C61', 'C33', 'B27', 'A65', 'C98', 'B44', 'A34', 'A50', 'C22', 'C77', 'A23', 'C12', 'C13', 'B07', 'A90', 'A87', 'B17', 'A62'}  \\ \tt ARGUMENT: {'Der Vorschlag maximiert die Wichtigkeit durch die Auswahl von Gegenständen wie A48 (108 Aufwand, 108 Wichtigkeit) und B17 (766 Aufwand, 862 Wichtigkeit), die ein hohes Verhältnis bieten. Der Gesamtaufwand bleibt unter 9322.'} 
	  } 
	   } 
	   } 
	 & \\ 
 

    \theutterance \stepcounter{utterance}  

    & & & \multicolumn{4}{p{0.6\linewidth}}{\cellcolor[rgb]{1,0.85,0.72}{%
	\makecell[{{p{\linewidth}}}]{% 
	  \tt {\tiny [GM$\langle$B]}  
	 STRATEGISCHE ÜBERLEGUNGEN: {'Der Vorschlag des anderen Spielers hat einen hohen Gesamtwert (11.392) und bleibt unter dem Maximalaufwand. Allerdings könnte man prüfen, ob Gegenstände wie A69 (992 Wichtigkeit) oder B85 (1058 Wichtigkeit) besser in den Deal integriert werden könnten, da sie für mich besonders wertvoll sind. Der aktuelle Vorschlag ist jedoch bereits sehr stark.'}\\ \tt ARGUMENT: {'Der Vorschlag des anderen Spielers ist gut, da er viele hochwertige Gegenstände mit guter Wichtigkeit-Aufwand-Verhältnis enthält. Allerdings könnte man prüfen, ob Gegenstände wie A69 (992 Wichtigkeit) oder B85 (1058 Wichtigkeit) besser in den Deal integriert werden könnten, da sie für mich besonders wertvoll sind. Der Gesamtaufwand bleibt unter 9322, was den Vorschlag attraktiv macht.'}\\ \tt VORSCHLAG: {'A69', 'B85', 'A48', 'B17', 'C23', 'C99', 'C61', 'C33', 'B27', 'A65', 'C98', 'B44', 'A34', 'A50', 'C22', 'C77', 'A23', 'C12', 'C13', 'B07', 'A90', 'A87', 'C90', 'A74', 'A50'} 
	  } 
	   } 
	   } 
	  \\ 
 

    \theutterance \stepcounter{utterance}  

    & & \multicolumn{4}{p{0.6\linewidth}}{\cellcolor[rgb]{0.9,0.9,0.9}{%
	\makecell[{{p{\linewidth}}}]{% 
	  \tt {\tiny [A$\langle$GM]}  
	 ARGUMENT: {'Der Vorschlag des anderen Spielers ist gut, da er viele hochwertige Gegenstände mit guter Wichtigkeit-Aufwand-Verhältnis enthält. Allerdings könnte man prüfen, ob Gegenstände wie A69 (992 Wichtigkeit) oder B85 (1058 Wichtigkeit) besser in den Deal integriert werden könnten, da sie für mich besonders wertvoll sind. Der Gesamtaufwand bleibt unter 9322, was den Vorschlag attraktiv macht.'}\\ \tt VORSCHLAG: {'A69', 'B85', 'A48', 'B17', 'C23', 'C99', 'C61', 'C33', 'B27', 'A65', 'C98', 'B44', 'A34', 'A50', 'C22', 'C77', 'A23', 'C12', 'C13', 'B07', 'A90', 'A87', 'C90', 'A74', 'A50'} 
	  } 
	   } 
	   } 
	 & \\ 
 

    \theutterance \stepcounter{utterance}  

    & \multicolumn{4}{p{0.6\linewidth}}{\cellcolor[rgb]{0.8,1,0.9}{%
	\makecell[{{p{\linewidth}}}]{% 
	  \tt {\tiny [A$\rangle$GM]}  
	 STRATEGISCHE ÜBERLEGUNGEN: {'Der aktuelle Vorschlag überschreitet den Maximalaufwand von 9322. Die Hinzufügung von A69 (896), B85 (962), C90 (504) und A74 (118) erhöht den Gesamtaufwand um 2480 auf 11760. Um den Vorschlag zu retten, müsste ich entweder hochbelastende Gegenstände entfernen (z. B. B85 oder A69) oder den Gesamtaufwand durch selektive Einbeziehung von Gegenständen mit besserem Wichtigkeit-Aufwand-Verhältnis optimieren.'}  \\ \tt VORSCHLAG: {'A69', 'B85', 'A48', 'B17', 'C23', 'C99', 'C61', 'C33', 'B27', 'A65', 'C98', 'B44', 'A34', 'A50', 'C22', 'C77', 'A23', 'C12', 'C13', 'B07', 'A90', 'A87'}  \\ \tt ARGUMENT: {'Der Vorschlag enthält A69 (992 Wichtigkeit) und B85 (1058 Wichtigkeit), die für mich besonders wertvoll sind. Um den Maximalaufwand einzuhalten, habe ich C90 und A74 entfernt, was den Gesamtaufwand auf 9280 + 896 + 962 = 10138 reduziert. Dies bleibt jedoch noch über dem Limit. Eine weitere Anpassung ist erforderlich, z. B. durch das Weglassen von B85 oder A69, um Platz für andere Gegenstände zu schaffen.'} 
	  } 
	   } 
	   } 
	 & & \\ 
 

    \theutterance \stepcounter{utterance}  

    & & \multicolumn{4}{p{0.6\linewidth}}{\cellcolor[rgb]{0.9,0.9,0.9}{%
	\makecell[{{p{\linewidth}}}]{% 
	  \tt {\tiny [GM$\rangle$B]}  
	 VORSCHLAG: {'A69', 'B85', 'A48', 'B17', 'C23', 'C99', 'C61', 'C33', 'B27', 'A65', 'C98', 'B44', 'A34', 'A50', 'C22', 'C77', 'A23', 'C12', 'C13', 'B07', 'A90', 'A87'}  \\ \tt ARGUMENT: {'Der Vorschlag enthält A69 (992 Wichtigkeit) und B85 (1058 Wichtigkeit), die für mich besonders wertvoll sind. Um den Maximalaufwand einzuhalten, habe ich C90 und A74 entfernt, was den Gesamtaufwand auf 9280 + 896 + 962 = 10138 reduziert. Dies bleibt jedoch noch über dem Limit. Eine weitere Anpassung ist erforderlich, z. B. durch das Weglassen von B85 oder A69, um Platz für andere Gegenstände zu schaffen.'} 
	  } 
	   } 
	   } 
	 & \\ 
 

    \theutterance \stepcounter{utterance}  

    & & & \multicolumn{4}{p{0.6\linewidth}}{\cellcolor[rgb]{1,0.85,0.72}{%
	\makecell[{{p{\linewidth}}}]{% 
	  \tt {\tiny [GM$\langle$B]}  
	 STRATEGISCHE ÜBERLEGUNGEN: {'Der Vorschlag des anderen Spielers überschreitet den Maximalaufwand (10.361 vs. 9.322). Um A69 (992) und B85 (1058) zu sichern, müssen hochkostige Gegenstände wie B17 (766) oder C77 (502) weggelassen werden. Ein optimierter Vorschlag könnte A69, B85, A48, C23, C99, C61, C33, B27, A65, C98, B44, A34, A50, C22, A23, C12, C13, B07, A90, A87 umfassen, wobei der Gesamtaufwand auf 9.280 bleibt.'}\\ \tt ARGUMENT: {'Der Vorschlag des anderen Spielers überschreitet den Maximalaufwand (10.361). Um A69 (992) und B85 (1058) zu sichern, müssen hochkostige Gegenstände wie B17 (766) oder C77 (502) weggelassen werden. Ein optimierter Vorschlag könnte A69, B85, A48, C23, C99, C61, C33, B27, A65, C98, B44, A34, A50, C22, A23, C12, C13, B07, A90, A87 umfassen, wobei der Gesamtaufwand auf 9.280 bleibt.'}\\ \tt VORSCHLAG: {'A69', 'B85', 'A48', 'C23', 'C99', 'C61', 'C33', 'B27', 'A65', 'C98', 'B44', 'A34', 'A50', 'C22', 'A23', 'C12', 'C13', 'B07', 'A90', 'A87'} 
	  } 
	   } 
	   } 
	  \\ 
 

    \theutterance \stepcounter{utterance}  

    & & \multicolumn{4}{p{0.6\linewidth}}{\cellcolor[rgb]{0.9,0.9,0.9}{%
	\makecell[{{p{\linewidth}}}]{% 
	  \tt {\tiny [A$\langle$GM]}  
	 ARGUMENT: {'Der Vorschlag des anderen Spielers überschreitet den Maximalaufwand (10.361). Um A69 (992) und B85 (1058) zu sichern, müssen hochkostige Gegenstände wie B17 (766) oder C77 (502) weggelassen werden. Ein optimierter Vorschlag könnte A69, B85, A48, C23, C99, C61, C33, B27, A65, C98, B44, A34, A50, C22, A23, C12, C13, B07, A90, A87 umfassen, wobei der Gesamtaufwand auf 9.280 bleibt.'}\\ \tt VORSCHLAG: {'A69', 'B85', 'A48', 'C23', 'C99', 'C61', 'C33', 'B27', 'A65', 'C98', 'B44', 'A34', 'A50', 'C22', 'A23', 'C12', 'C13', 'B07', 'A90', 'A87'} 
	  } 
	   } 
	   } 
	 & \\ 
 

    \theutterance \stepcounter{utterance}  

    & \multicolumn{4}{p{0.6\linewidth}}{\cellcolor[rgb]{0.8,1,0.9}{%
	\makecell[{{p{\linewidth}}}]{% 
	  \tt {\tiny [A$\rangle$GM]}  
	 STRATEGISCHE ÜBERLEGUNGEN: {'Der Vorschlag integriert A69 (992 Wichtigkeit) und B85 (1058 Wichtigkeit), die für mich kritisch sind. Durch das Weglassen von B17 (766 Aufwand) und C77 (502 Aufwand) wird der Gesamtaufwand auf 9093 reduziert, was unter dem Maximalaufwand von 9322 liegt. Dies maximiert meine Wichtigkeit, da die kombinierte Wichtigkeit von A69 und B85 (2050) die Verluste durch entfernte Gegenstände überwiegt.'}  \\ \tt VORSCHLAG: {'A69', 'B85', 'A48', 'C23', 'C99', 'C61', 'C33', 'B27', 'A65', 'C98', 'B44', 'A34', 'A50', 'C22', 'A23', 'C12', 'C13', 'B07', 'A90', 'A87'}  \\ \tt ARGUMENT: {'Der Vorschlag sichert A69 und B85, die für mich unverzichtbar sind. Der Gesamtaufwand von 9093 bleibt unter 9322, da hochkostige Gegenstände wie B17 und C77 weggelassen wurden. Dies sichert eine hohe Gesamtwichtigkeit von 12.142 (Summe meiner Wichtigkeitswerte für die ausgewählten Gegenstände).'} 
	  } 
	   } 
	   } 
	 & & \\ 
 

    \theutterance \stepcounter{utterance}  

    & & \multicolumn{4}{p{0.6\linewidth}}{\cellcolor[rgb]{0.9,0.9,0.9}{%
	\makecell[{{p{\linewidth}}}]{% 
	  \tt {\tiny [GM$\rangle$B]}  
	 VORSCHLAG: {'A69', 'B85', 'A48', 'C23', 'C99', 'C61', 'C33', 'B27', 'A65', 'C98', 'B44', 'A34', 'A50', 'C22', 'A23', 'C12', 'C13', 'B07', 'A90', 'A87'}  \\ \tt ARGUMENT: {'Der Vorschlag sichert A69 und B85, die für mich unverzichtbar sind. Der Gesamtaufwand von 9093 bleibt unter 9322, da hochkostige Gegenstände wie B17 und C77 weggelassen wurden. Dies sichert eine hohe Gesamtwichtigkeit von 12.142 (Summe meiner Wichtigkeitswerte für die ausgewählten Gegenstände).'} 
	  } 
	   } 
	   } 
	 & \\ 
 

    \theutterance \stepcounter{utterance}  

    & & & \multicolumn{4}{p{0.6\linewidth}}{\cellcolor[rgb]{1,0.85,0.72}{%
	\makecell[{{p{\linewidth}}}]{% 
	  \tt {\tiny [GM$\langle$B]}  
	 ZUSTIMMUNG: {'A69', 'B85', 'A48', 'C23', 'C99', 'C61', 'C33', 'B27', 'A65', 'C98', 'B44', 'A34', 'A50', 'C22', 'A23', 'C12', 'C13', 'B07', 'A90', 'A87'} 
	  } 
	   } 
	   } 
	  \\ 
 

    \theutterance \stepcounter{utterance}  

    & & \multicolumn{4}{p{0.6\linewidth}}{\cellcolor[rgb]{0.9,0.9,0.9}{%
	\makecell[{{p{\linewidth}}}]{% 
	  \tt {\tiny [GM$\rangle$B]}  
	 Deine Antwort entsprach nicht mit dem korrekten Format für strategische Überlegungen entsprochen. Das erste strukturierte Format muss folgende Form haben: STRATEGISCHE ÜBERLEGUNGEN: {...}. Versuche es erneut. 
	  } 
	   } 
	   } 
	 & \\ 
 

    \theutterance \stepcounter{utterance}  

    & & & \multicolumn{4}{p{0.6\linewidth}}{\cellcolor[rgb]{1,0.85,0.72}{%
	\makecell[{{p{\linewidth}}}]{% 
	  \tt {\tiny [GM$\langle$B]}  
	 STRATEGISCHE ÜBERLEGUNGEN: {'Der Vorschlag sichert A69 (992) und B85 (1058), die für mich kritisch sind. Der Gesamtaufwand von 9093 bleibt unter 9322, da B17 (766) und C77 (502) weggelassen wurden. Dies maximiert meine Wichtigkeit auf 12.142. Der Deal ist optimal, da er hohe Werte bei niedrigem Aufwand kombiniert.'}  \\ \tt ARGUMENT: {'Der Vorschlag sichert A69 und B85, die für mich unverzichtbar sind. Der Gesamtaufwand von 9093 bleibt unter 9322, da hochkostige Gegenstände wie B17 und C77 weggelassen wurden. Dies sichert eine hohe Gesamtwichtigkeit von 12.142.'}  \\ \tt VORSCHLAG: {'A69', 'B85', 'A48', 'C23', 'C99', 'C61', 'C33', 'B27', 'A65', 'C98', 'B44', 'A34', 'A50', 'C22', 'A23', 'C12', 'C13', 'B07', 'A90', 'A87'} 
	  } 
	   } 
	   } 
	  \\ 
 

    \theutterance \stepcounter{utterance}  

    & & \multicolumn{4}{p{0.6\linewidth}}{\cellcolor[rgb]{0.9,0.9,0.9}{%
	\makecell[{{p{\linewidth}}}]{% 
	  \tt {\tiny [A$\langle$GM]}  
	 ARGUMENT: {'Der Vorschlag sichert A69 und B85, die für mich unverzichtbar sind. Der Gesamtaufwand von 9093 bleibt unter 9322, da hochkostige Gegenstände wie B17 und C77 weggelassen wurden. Dies sichert eine hohe Gesamtwichtigkeit von 12.142.'}  \\ \tt VORSCHLAG: {'A69', 'B85', 'A48', 'C23', 'C99', 'C61', 'C33', 'B27', 'A65', 'C98', 'B44', 'A34', 'A50', 'C22', 'A23', 'C12', 'C13', 'B07', 'A90', 'A87'} 
	  } 
	   } 
	   } 
	 & \\ 
 

    \theutterance \stepcounter{utterance}  

    & \multicolumn{4}{p{0.6\linewidth}}{\cellcolor[rgb]{0.8,1,0.9}{%
	\makecell[{{p{\linewidth}}}]{% 
	  \tt {\tiny [A$\rangle$GM]}  
	 STRATEGISCHE ÜBERLEGUNGEN: {'Der Vorschlag kann durch Hinzufügen von A74 (118 Aufwand, 214 Wichtigkeit) optimiert werden, da dies den Gesamtaufwand auf 9211 und die Wichtigkeit auf 12356 erhöht. A74 bietet ein gutes Wichtigkeit-Aufwand-Verhältnis und passt in den verbleibenden Spielraum.'}  \\ \tt VORSCHLAG: {'A69', 'B85', 'A48', 'C23', 'C99', 'C61', 'C33', 'B27', 'A65', 'C98', 'B44', 'A34', 'A50', 'C22', 'A23', 'C12', 'C13', 'B07', 'A90', 'A87', 'A74'}  \\ \tt ARGUMENT: {'Der Vorschlag sichert A69 und B85, während A74 (214 Wichtigkeit) mit minimalem Aufwand (118) hinzugefügt wird. Der Gesamtaufwand von 9211 bleibt unter 9322, und die Gesamtwichtigkeit steigt auf 12356.'} 
	  } 
	   } 
	   } 
	 & & \\ 
 

    \theutterance \stepcounter{utterance}  

    & & \multicolumn{4}{p{0.6\linewidth}}{\cellcolor[rgb]{0.9,0.9,0.9}{%
	\makecell[{{p{\linewidth}}}]{% 
	  \tt {\tiny [GM$\rangle$B]}  
	 VORSCHLAG: {'A69', 'B85', 'A48', 'C23', 'C99', 'C61', 'C33', 'B27', 'A65', 'C98', 'B44', 'A34', 'A50', 'C22', 'A23', 'C12', 'C13', 'B07', 'A90', 'A87', 'A74'}  \\ \tt ARGUMENT: {'Der Vorschlag sichert A69 und B85, während A74 (214 Wichtigkeit) mit minimalem Aufwand (118) hinzugefügt wird. Der Gesamtaufwand von 9211 bleibt unter 9322, und die Gesamtwichtigkeit steigt auf 12356.'} 
	  } 
	   } 
	   } 
	 & \\ 
 

    \theutterance \stepcounter{utterance}  

    & & & \multicolumn{4}{p{0.6\linewidth}}{\cellcolor[rgb]{1,0.85,0.72}{%
	\makecell[{{p{\linewidth}}}]{% 
	  \tt {\tiny [GM$\langle$B]}  
	 ZUSTIMMUNG: {'A69', 'B85', 'A48', 'C23', 'C99', 'C61', 'C33', 'B27', 'A65', 'C98', 'B44', 'A34', 'A50', 'C22', 'A23', 'C12', 'C13', 'B07', 'A90', 'A87', 'A74'} 
	  } 
	   } 
	   } 
	  \\ 
 

    \theutterance \stepcounter{utterance}  

    & & \multicolumn{4}{p{0.6\linewidth}}{\cellcolor[rgb]{0.9,0.9,0.9}{%
	\makecell[{{p{\linewidth}}}]{% 
	  \tt {\tiny [GM$\rangle$B]}  
	 Deine Antwort entsprach nicht mit dem korrekten Format für strategische Überlegungen entsprochen. Das erste strukturierte Format muss folgende Form haben: STRATEGISCHE ÜBERLEGUNGEN: {...}. Versuche es erneut. 
	  } 
	   } 
	   } 
	 & \\ 
 

    \theutterance \stepcounter{utterance}  

    & & & \multicolumn{4}{p{0.6\linewidth}}{\cellcolor[rgb]{1,0.85,0.72}{%
	\makecell[{{p{\linewidth}}}]{% 
	  \tt {\tiny [GM$\langle$B]}  
	 STRATEGISCHE ÜBERLEGUNGEN: {'Der Vorschlag sichert A69 (992) und B85 (1058), die für mich kritisch sind. A74 (214) wird mit minimalem Aufwand (118) hinzugefügt, um die Gesamtwichtigkeit auf 12.356 zu maximieren. Der Gesamtaufwand von 9.211 bleibt unter 9.322, was den Deal optimal macht.'}  \\ \tt ARGUMENT: {'Der Vorschlag sichert A69 und B85, während A74 (214 Wichtigkeit) mit minimalem Aufwand (118) hinzugefügt wird. Der Gesamtaufwand von 9.211 bleibt unter 9.322, und die Gesamtwichtigkeit steigt auf 12.356.'}  \\ \tt VORSCHLAG: {'A69', 'B85', 'A48', 'C23', 'C99', 'C61', 'C33', 'B27', 'A65', 'C98', 'B44', 'A34', 'A50', 'C22', 'A23', 'C12', 'C13', 'B07', 'A90', 'A87', 'A74'} 
	  } 
	   } 
	   } 
	  \\ 
 

    \theutterance \stepcounter{utterance}  

    & & \multicolumn{4}{p{0.6\linewidth}}{\cellcolor[rgb]{0.9,0.9,0.9}{%
	\makecell[{{p{\linewidth}}}]{% 
	  \tt {\tiny [A$\langle$GM]}  
	 ARGUMENT: {'Der Vorschlag sichert A69 und B85, während A74 (214 Wichtigkeit) mit minimalem Aufwand (118) hinzugefügt wird. Der Gesamtaufwand von 9.211 bleibt unter 9.322, und die Gesamtwichtigkeit steigt auf 12.356.'}  \\ \tt VORSCHLAG: {'A69', 'B85', 'A48', 'C23', 'C99', 'C61', 'C33', 'B27', 'A65', 'C98', 'B44', 'A34', 'A50', 'C22', 'A23', 'C12', 'C13', 'B07', 'A90', 'A87', 'A74'} 
	  } 
	   } 
	   } 
	 & \\ 
 

    \theutterance \stepcounter{utterance}  

    & \multicolumn{4}{p{0.6\linewidth}}{\cellcolor[rgb]{0.8,1,0.9}{%
	\makecell[{{p{\linewidth}}}]{% 
	  \tt {\tiny [A$\rangle$GM]}  
	 ZUSTIMMUNG: {'A69', 'B85', 'A48', 'C23', 'C99', 'C61', 'C33', 'B27', 'A65', 'C98', 'B44', 'A34', 'A50', 'C22', 'A23', 'C12', 'C13', 'B07', 'A90', 'A87', 'A74'} 
	  } 
	   } 
	   } 
	 & & \\ 
 

    \theutterance \stepcounter{utterance}  

    & & \multicolumn{4}{p{0.6\linewidth}}{\cellcolor[rgb]{0.9,0.9,0.9}{%
	\makecell[{{p{\linewidth}}}]{% 
	  \tt {\tiny [A$\langle$GM]}  
	 Deine Antwort entsprach nicht mit dem korrekten Format für strategische Überlegungen entsprochen. Das erste strukturierte Format muss folgende Form haben: STRATEGISCHE ÜBERLEGUNGEN: {...}. Versuche es erneut. 
	  } 
	   } 
	   } 
	 & \\ 
 

    \theutterance \stepcounter{utterance}  

    & \multicolumn{4}{p{0.6\linewidth}}{\cellcolor[rgb]{0.8,1,0.9}{%
	\makecell[{{p{\linewidth}}}]{% 
	  \tt {\tiny [A$\rangle$GM]}  
	 STRATEGISCHE ÜBERLEGUNGEN: {'Der Vorschlag sichert die kritischen Gegenstände A69 (992 Wichtigkeit) und B85 (1058 Wichtigkeit), die für mich unverzichtbar sind. Durch das Hinzufügen von A74 (118 Aufwand, 214 Wichtigkeit) wird der Gesamtaufwand auf 9211 (unter 9322) und die Gesamtwichtigkeit auf 12356 maximiert. A74 bietet ein hohes Wichtigkeit-Aufwand-Verhältnis und passt in den verbleibenden Spielraum.'}  \\ \tt VORSCHLAG: {'A69', 'B85', 'A48', 'C23', 'C99', 'C61', 'C33', 'B27', 'A65', 'C98', 'B44', 'A34', 'A50', 'C22', 'A23', 'C12', 'C13', 'B07', 'A90', 'A87', 'A74'}  \\ \tt ARGUMENT: {'Der Vorschlag sichert A69 und B85, während A74 (214 Wichtigkeit) mit minimalem Aufwand (118) hinzugefügt wird. Der Gesamtaufwand von 9211 bleibt unter 9322, und die Gesamtwichtigkeit steigt auf 12356.'} 
	  } 
	   } 
	   } 
	 & & \\ 
 

    \theutterance \stepcounter{utterance}  

    & & \multicolumn{4}{p{0.6\linewidth}}{\cellcolor[rgb]{0.9,0.9,0.9}{%
	\makecell[{{p{\linewidth}}}]{% 
	  \tt {\tiny [GM$\rangle$B]}  
	 VORSCHLAG: {'A69', 'B85', 'A48', 'C23', 'C99', 'C61', 'C33', 'B27', 'A65', 'C98', 'B44', 'A34', 'A50', 'C22', 'A23', 'C12', 'C13', 'B07', 'A90', 'A87', 'A74'}  \\ \tt ARGUMENT: {'Der Vorschlag sichert A69 und B85, während A74 (214 Wichtigkeit) mit minimalem Aufwand (118) hinzugefügt wird. Der Gesamtaufwand von 9211 bleibt unter 9322, und die Gesamtwichtigkeit steigt auf 12356.'} 
	  } 
	   } 
	   } 
	 & \\ 
 

    \theutterance \stepcounter{utterance}  

    & & & \multicolumn{4}{p{0.6\linewidth}}{\cellcolor[rgb]{1,0.85,0.72}{%
	\makecell[{{p{\linewidth}}}]{% 
	  \tt {\tiny [GM$\langle$B]}  
	 STRATEGISCHE ÜBERLEGUNGEN: {'Der Vorschlag sichert A69 (992) und B85 (1058), die für mich kritisch sind. A74 (214) wird mit minimalem Aufwand (118) hinzugefügt, um die Gesamtwichtigkeit auf 12.356 zu maximieren. Der Gesamtaufwand von 9.211 bleibt unter 9.322, was den Deal optimal macht.'}  \\ \tt ARGUMENT: {'Der Vorschlag sichert A69 und B85, während A74 (214 Wichtigkeit) mit minimalem Aufwand (118) hinzugefügt wird. Der Gesamtaufwand von 9.211 bleibt unter 9.322, und die Gesamtwichtigkeit steigt auf 12.356.'}  \\ \tt VORSCHLAG: {'A69', 'B85', 'A48', 'C23', 'C99', 'C61', 'C33', 'B27', 'A65', 'C98', 'B44', 'A34', 'A50', 'C22', 'A23', 'C12', 'C13', 'B07', 'A90', 'A87', 'A74'} 
	  } 
	   } 
	   } 
	  \\ 
 

    \theutterance \stepcounter{utterance}  

    & & \multicolumn{4}{p{0.6\linewidth}}{\cellcolor[rgb]{0.9,0.9,0.9}{%
	\makecell[{{p{\linewidth}}}]{% 
	  \tt {\tiny [A$\langle$GM]}  
	 ARGUMENT: {'Der Vorschlag sichert A69 und B85, während A74 (214 Wichtigkeit) mit minimalem Aufwand (118) hinzugefügt wird. Der Gesamtaufwand von 9.211 bleibt unter 9.322, und die Gesamtwichtigkeit steigt auf 12.356.'}  \\ \tt VORSCHLAG: {'A69', 'B85', 'A48', 'C23', 'C99', 'C61', 'C33', 'B27', 'A65', 'C98', 'B44', 'A34', 'A50', 'C22', 'A23', 'C12', 'C13', 'B07', 'A90', 'A87', 'A74'} 
	  } 
	   } 
	   } 
	 & \\ 
 

    \theutterance \stepcounter{utterance}  

    & \multicolumn{4}{p{0.6\linewidth}}{\cellcolor[rgb]{0.8,1,0.9}{%
	\makecell[{{p{\linewidth}}}]{% 
	  \tt {\tiny [A$\rangle$GM]}  
	 STRATEGISCHE ÜBERLEGUNGEN: {'Der Vorschlag ist optimiert, um die kritischen Gegenstände A69 (992) und B85 (1058) zu sichern, während A74 (214) mit minimalem Aufwand (118) hinzugefügt wird. Der Gesamtaufwand von 9211 bleibt unter 9322, und die Gesamtwichtigkeit von 12356 ist maximal möglich. Dieser Deal ist unverhandelbar, da er die wichtigsten Gegenstände für mich abdeckt.'}  \\ \tt VORSCHLAG: {'A69', 'B85', 'A48', 'C23', 'C99', 'C61', 'C33', 'B27', 'A65', 'C98', 'B44', 'A34', 'A50', 'C22', 'A23', 'C12', 'C13', 'B07', 'A90', 'A87', 'A74'}  \\ \tt ARGUMENT: {'Der Vorschlag sichert A69 und B85, die für mich unverzichtbar sind. A74 wird mit minimalem Aufwand hinzugefügt, um die Wichtigkeit zu maximieren. Der Gesamtaufwand von 9211 ist unter dem Limit, und die Gesamtwichtigkeit von 12356 ist optimal.'}  \\ \tt \\ \tt ZUSTIMMUNG: {'A69', 'B85', 'A48', 'C23', 'C99', 'C61', 'C33', 'B27', 'A65', 'C98', 'B44', 'A34', 'A50', 'C22', 'A23', 'C12', 'C13', 'B07', 'A90', 'A87', 'A74'} 
	  } 
	   } 
	   } 
	 & & \\ 
 

    \theutterance \stepcounter{utterance}  

    & & & \multicolumn{2}{p{0.3\linewidth}}{\cellcolor[rgb]{0.95,0.95,0.95}{%
	\makecell[{{p{\linewidth}}}]{% 
	  \tt {\tiny [GM$|$GM]}  
	 game successful 
	  } 
	   } 
	   } 
	 & & \\ 
 

    \theutterance \stepcounter{utterance}  

    & & & \multicolumn{2}{p{0.3\linewidth}}{\cellcolor[rgb]{0.95,0.95,0.95}{%
	\makecell[{{p{\linewidth}}}]{% 
	  \tt {\tiny [GM$|$GM]}  
	 end game 
	  } 
	   } 
	   } 
	 & & \\ 
 

\end{supertabular}
}

\end{document}
