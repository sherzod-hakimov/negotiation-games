
\documentclass{article}
\usepackage{colortbl}
\usepackage{makecell}
\usepackage{multirow}
\usepackage{supertabular}

\begin{document}

\newcounter{utterance}

\twocolumn

{ \footnotesize  \setcounter{utterance}{1}
\setlength{\tabcolsep}{0pt}
\begin{supertabular}{c@{$\;$}|p{.15\linewidth}@{}p{.15\linewidth}p{.15\linewidth}p{.15\linewidth}p{.15\linewidth}p{.15\linewidth}}

    \# & $\;$A & \multicolumn{4}{c}{Game Master} & $\;\:$B\\
    \hline 

    \theutterance \stepcounter{utterance}  

    & & \multicolumn{4}{p{0.6\linewidth}}{\cellcolor[rgb]{0.9,0.9,0.9}{%
	\makecell[{{p{\linewidth}}}]{% 
	  \tt {\tiny [A$\langle$GM]}  
	 Ich bin der Spielleiter, und du spielst ein kooperatives Spiel mit dem folgenden Raster als Spielbrett:\\ \tt \\ \tt ```\\ \tt  1234567\\ \tt ╔══╤════╗\\ \tt ║◌◌│◌◌IW║ 1\\ \tt ╟──┘◌◌┌─╢ 2\\ \tt ║◌◌◌◌H│◌║ 3\\ \tt ╟─┐C┌─┼─╢ 4\\ \tt ║◌├─┤◌│T║ 5\\ \tt ╟─┤◌├─┤◌║ 6\\ \tt ║◌│◌│◌│◌║ 7\\ \tt ╚═╧═╧═╧═╝\\ \tt \\ \tt ```\\ \tt \\ \tt * Am oberen Rand sind die x-Koordinaten angegeben, die nach rechts hin zunehmen, und am rechten Rand die y-Koordinaten, die nach unten zunehmen.\\ \tt * Die folgenden Objekte sind zufällig auf deinem Raster platziert: 'W', 'I', 'T', 'C', 'H'.\\ \tt \\ \tt \\ \tt Dein Mitspieler sieht eine Variation des Spielbretts, auf dem dieselben Objekte an anderen zufälligen Positionen platziert sind. Du kannst das Spielbrett des anderen Spielers nicht sehen, und er kann deins nicht sehen.\\ \tt \\ \tt \\ \tt **Ziel des Spiels:**\\ \tt \\ \tt Beide Spieler müssen die Objekte auf ihrem jeweiligen Spielbrett so bewegen, dass identische Objekte am Ende an denselben Koordinaten liegen. Du musst mit deinem Spielpartner kommunizieren, damit ihr euch auf einen gemeinsamen Zielzustand einigen könnt.\\ \tt \\ \tt \\ \tt **Spielregeln:**\\ \tt \\ \tt * In jedem Spielzug kannst du genau eine der folgenden zwei Aktionen ausführen:\\ \tt 1. `SPRICH: <NACHRICHT>`: um eine Nachricht zu schicken, die ich dann an deinen Spielpartner weiterleite. Als Nachricht gilt alles bis zum nächsten Zeilenumbruch.\\ \tt 2. `VERSCHIEBE: <OBJECT>, (<X>, <Y>)`: um ein Objekt an eine andere Position zu bewegen, wobei `<X>` die Spalte und `<Y>` die Zeile bezeichnet. Ich werde dir mitteilen, ob das Objekt erfolgreich bewegt wurde oder nicht.\\ \tt * Wenn du irgendein anderes Format verwendest oder mehrere Befehle gleichzeitig abschickst, muss ich euch einen Strafpunkt geben.\\ \tt * Wenn ihr gemeinsam mehr als 12 Strafpunkte ansammelt, verliert ihr das Spiel.\\ \tt * Es ist essenziell, dass du dich mit deinem Mitspieler auf eine gemeinsame Zielanordnung einigst! Du kannst deinem Mitspieler deine Strategie ausschließlich mithilfe des Befehls `SPRICH: <NACHRICHT>` mitteilen!\\ \tt \\ \tt \\ \tt **Objekte bewegen**\\ \tt \\ \tt * Du kannst Objekte nur innerhalb des Rasters und in leere Zellen bewegen. Leere Zellen erkennst am Zeichen '◌'.\\ \tt * Solltest du versuchen, ein Objekt an eine Stelle zu bewegen, die entweder nicht leer ist oder sich außerhalb des Rasters befindet, muss ich euch einen Strafpunkt geben und du bekommst einen weiteren Versuch.\\ \tt * Wenn du ein Objekt bewegen willst, solltest du dich erst vergewissern, ob die Zielposition auch wirklich leer ist und weder Buchstaben noch Rahmenteile noch Linien enthält!\\ \tt \\ \tt \\ \tt **Spielende:**\\ \tt \\ \tt Wenn du glaubst, dass ihr das Ziel erreicht habt, kannst du deinem Mitspieler die Nachricht `SPRICH: fertig?` schicken. Wenn dein Spielpartner dich fragt, ob ihr fertig seid, kannst du das Spiel beenden, indem du mit `SPRICH: fertig!` antwortest.\\ \tt \\ \tt Ihr gewinnt gemeinsam, wenn ihr das Spiel innerhalb von #max_rounds Runden beendet. Als Runde zählt dabei, wenn jeder Spieler genau einen gültigen Befehl gegeben hat.\\ \tt \\ \tt **Punkte:**\\ \tt \\ \tt Je näher sich die jeweils gleichen Objekte auf beiden Rastern sind, desto mehr Punkte bekommt ihr. Für Strafpunkte gibt es Punktabzug. Könnt ihr den Rekord brechen?\\ \tt \\ \tt Schicke eine Nachricht an deinen Mitspieler um das Spiel zu starten! 
	  } 
	   } 
	   } 
	 & \\ 
 

    \theutterance \stepcounter{utterance}  

    & \multicolumn{4}{p{0.6\linewidth}}{\cellcolor[rgb]{0.8,1,0.9}{%
	\makecell[{{p{\linewidth}}}]{% 
	  \tt {\tiny [A$\rangle$GM]}  
	 SPRICH: Lasst uns alle Objekte in die oberste Zeile verschieben. Ich beginne mit W an (1,1). Bitte teilt mir eure Positionen mit, damit wir uns abstimmen können. 
	  } 
	   } 
	   } 
	 & & \\ 
 

    \theutterance \stepcounter{utterance}  

    & & & \multicolumn{2}{p{0.3\linewidth}}{\cellcolor[rgb]{0.95,0.95,0.95}{%
	\makecell[{{p{\linewidth}}}]{% 
	  \tt {\tiny [GM$|$GM]}  
	 SPRICH: Lasst uns alle Objekte in die oberste Zeile verschieben. Ich beginne mit W an (1,1). Bitte teilt mir eure Positionen mit, damit wir uns abstimmen können. 
	  } 
	   } 
	   } 
	 & & \\ 
 

    \theutterance \stepcounter{utterance}  

    & & \multicolumn{4}{p{0.6\linewidth}}{\cellcolor[rgb]{0.9,0.9,0.9}{%
	\makecell[{{p{\linewidth}}}]{% 
	  \tt {\tiny [GM$\rangle$B]}  
	 Ich bin der Spielleiter, und du spielst ein kooperatives Spiel mit dem folgenden Raster als Spielbrett:\\ \tt \\ \tt ```\\ \tt  1234567\\ \tt ╔══╤════╗\\ \tt ║◌◌│◌◌◌◌║ 1\\ \tt ╟──┘◌T┌─╢ 2\\ \tt ║◌◌HW◌│◌║ 3\\ \tt ╟─┐◌┌─┼─╢ 4\\ \tt ║◌├─┤◌│◌║ 5\\ \tt ╟─┤◌├─┤◌║ 6\\ \tt ║◌│I│C│◌║ 7\\ \tt ╚═╧═╧═╧═╝\\ \tt \\ \tt ```\\ \tt \\ \tt * Am oberen Rand sind die x-Koordinaten angegeben, die nach rechts hin zunehmen, und am rechten Rand die y-Koordinaten, die nach unten zunehmen.\\ \tt * Die folgenden Objekte sind zufällig auf deinem Raster platziert: 'W', 'I', 'T', 'C', 'H'.\\ \tt \\ \tt \\ \tt Dein Mitspieler sieht eine Variation des Spielbretts, auf dem dieselben Objekte an anderen zufälligen Positionen platziert sind. Du kannst das Spielbrett des anderen Spielers nicht sehen, und er kann deins nicht sehen.\\ \tt \\ \tt \\ \tt **Ziel des Spiels:**\\ \tt \\ \tt Beide Spieler müssen die Objekte auf ihrem jeweiligen Spielbrett so bewegen, dass identische Objekte am Ende an denselben Koordinaten liegen. Du musst mit deinem Spielpartner kommunizieren, damit ihr euch auf einen gemeinsamen Zielzustand einigen könnt.\\ \tt \\ \tt \\ \tt **Spielregeln:**\\ \tt \\ \tt * In jedem Spielzug kannst du genau eine der folgenden zwei Aktionen ausführen:\\ \tt 1. `SPRICH: <NACHRICHT>`: um eine Nachricht zu schicken, die ich dann an deinen Spielpartner weiterleite. Als Nachricht gilt alles bis zum nächsten Zeilenumbruch.\\ \tt 2. `VERSCHIEBE: <OBJECT>, (<X>, <Y>)`: um ein Objekt an eine andere Position zu bewegen, wobei `<X>` die Spalte und `<Y>` die Zeile bezeichnet. Ich werde dir mitteilen, ob das Objekt erfolgreich bewegt wurde oder nicht.\\ \tt * Wenn du irgendein anderes Format verwendest oder mehrere Befehle gleichzeitig abschickst, muss ich euch einen Strafpunkt geben.\\ \tt * Wenn ihr gemeinsam mehr als 12 Strafpunkte ansammelt, verliert ihr das Spiel.\\ \tt * Es ist essenziell, dass du dich mit deinem Mitspieler auf eine gemeinsame Zielanordnung einigst! Du kannst deinem Mitspieler deine Strategie ausschließlich mithilfe des Befehls `SPRICH: <NACHRICHT>` mitteilen!\\ \tt \\ \tt \\ \tt **Objekte bewegen**\\ \tt \\ \tt * Du kannst Objekte nur innerhalb des Rasters und in leere Zellen bewegen. Leere Zellen erkennst am Zeichen '◌'.\\ \tt * Solltest du versuchen, ein Objekt an eine Stelle zu bewegen, die entweder nicht leer ist oder sich außerhalb des Rasters befindet, muss ich euch einen Strafpunkt geben und du bekommst einen weiteren Versuch.\\ \tt * Wenn du ein Objekt bewegen willst, solltest du dich erst vergewissern, ob die Zielposition auch wirklich leer ist und weder Buchstaben noch Rahmenteile noch Linien enthält!\\ \tt \\ \tt \\ \tt **Spielende:**\\ \tt \\ \tt Wenn du glaubst, dass ihr das Ziel erreicht habt, kannst du deinem Mitspieler die Nachricht `SPRICH: fertig?` schicken. Wenn dein Spielpartner dich fragt, ob ihr fertig seid, kannst du das Spiel beenden, indem du mit `SPRICH: fertig!` antwortest.\\ \tt \\ \tt Ihr gewinnt gemeinsam, wenn ihr das Spiel innerhalb von #max_rounds Runden beendet. Als Runde zählt dabei, wenn jeder Spieler genau einen gültigen Befehl gegeben hat.\\ \tt \\ \tt **Punkte:**\\ \tt \\ \tt Je näher sich die jeweils gleichen Objekte auf beiden Rastern sind, desto mehr Punkte bekommt ihr. Für Strafpunkte gibt es Punktabzug. Könnt ihr den Rekord brechen?\\ \tt \\ \tt Dein Mitspieler hat das Spiel mit dieser Nachricht begonnen:\\ \tt \\ \tt "Lasst uns alle Objekte in die oberste Zeile verschieben. Ich beginne mit W an (1,1). Bitte teilt mir eure Positionen mit, damit wir uns abstimmen können."\\ \tt \\ \tt Wie lautet dein erster Befehl? 
	  } 
	   } 
	   } 
	 & \\ 
 

    \theutterance \stepcounter{utterance}  

    & & & \multicolumn{4}{p{0.6\linewidth}}{\cellcolor[rgb]{1,0.85,0.72}{%
	\makecell[{{p{\linewidth}}}]{% 
	  \tt {\tiny [GM$\langle$B]}  
	 SPRICH: Lasst uns alle Objekte in die oberste Zeile verschieben. Ich beginne mit W an (1,1). Bitte teilt mir deine aktuellen Positionen mit, damit wir uns abstimmen können. 
	  } 
	   } 
	   } 
	  \\ 
 

    \theutterance \stepcounter{utterance}  

    & & & \multicolumn{2}{p{0.3\linewidth}}{\cellcolor[rgb]{0.95,0.95,0.95}{%
	\makecell[{{p{\linewidth}}}]{% 
	  \tt {\tiny [GM$|$GM]}  
	 SPRICH: Lasst uns alle Objekte in die oberste Zeile verschieben. Ich beginne mit W an (1,1). Bitte teilt mir deine aktuellen Positionen mit, damit wir uns abstimmen können. 
	  } 
	   } 
	   } 
	 & & \\ 
 

    \theutterance \stepcounter{utterance}  

    & & \multicolumn{4}{p{0.6\linewidth}}{\cellcolor[rgb]{0.9,0.9,0.9}{%
	\makecell[{{p{\linewidth}}}]{% 
	  \tt {\tiny [A$\langle$GM]}  
	 Deine Nachricht wurde an deinen Mitspieler übermittelt.\\ \tt \\ \tt \\ \tt Ihr seid gerade bei Runde 1 von maximal 20.\\ \tt Ihr habt gemeinsam 0 von maximal 12 Strafpunkten angesammelt.\\ \tt Dein Mitspieler hat dir diese Nachricht geschickt:\\ \tt \\ \tt \\ \tt "Lasst uns alle Objekte in die oberste Zeile verschieben. Ich beginne mit W an (1,1). Bitte teilt mir deine aktuellen Positionen mit, damit wir uns abstimmen können."\\ \tt \\ \tt \\ \tt Wie lautet dein nächster Befehl? 
	  } 
	   } 
	   } 
	 & \\ 
 

    \theutterance \stepcounter{utterance}  

    & \multicolumn{4}{p{0.6\linewidth}}{\cellcolor[rgb]{0.8,1,0.9}{%
	\makecell[{{p{\linewidth}}}]{% 
	  \tt {\tiny [A$\rangle$GM]}  
	 SPRICH: Meine Objekte sind aktuell an folgenden Positionen: W(7,2), I(6,2), H(5,3), C(2,4), T(5,5). Bitte teilt mir deine aktuellen Positionen mit, damit wir uns auf eine gemeinsame Zielanordnung einigen können. 
	  } 
	   } 
	   } 
	 & & \\ 
 

    \theutterance \stepcounter{utterance}  

    & & & \multicolumn{2}{p{0.3\linewidth}}{\cellcolor[rgb]{0.95,0.95,0.95}{%
	\makecell[{{p{\linewidth}}}]{% 
	  \tt {\tiny [GM$|$GM]}  
	 SPRICH: Meine Objekte sind aktuell an folgenden Positionen: W(7,2), I(6,2), H(5,3), C(2,4), T(5,5). Bitte teilt mir deine aktuellen Positionen mit, damit wir uns auf eine gemeinsame Zielanordnung einigen können. 
	  } 
	   } 
	   } 
	 & & \\ 
 

    \theutterance \stepcounter{utterance}  

    & & \multicolumn{4}{p{0.6\linewidth}}{\cellcolor[rgb]{0.9,0.9,0.9}{%
	\makecell[{{p{\linewidth}}}]{% 
	  \tt {\tiny [GM$\rangle$B]}  
	 Deine Nachricht wurde an deinen Mitspieler übermittelt.\\ \tt \\ \tt \\ \tt Ihr seid gerade bei Runde 2 von maximal 20.\\ \tt Ihr habt gemeinsam 0 von maximal 12 Strafpunkten angesammelt.\\ \tt Dein Mitspieler hat dir diese Nachricht geschickt:\\ \tt \\ \tt \\ \tt "Meine Objekte sind aktuell an folgenden Positionen: W(7,2), I(6,2), H(5,3), C(2,4), T(5,5). Bitte teilt mir deine aktuellen Positionen mit, damit wir uns auf eine gemeinsame Zielanordnung einigen können."\\ \tt \\ \tt \\ \tt Wie lautet dein nächster Befehl? 
	  } 
	   } 
	   } 
	 & \\ 
 

    \theutterance \stepcounter{utterance}  

    & & & \multicolumn{4}{p{0.6\linewidth}}{\cellcolor[rgb]{1,0.85,0.72}{%
	\makecell[{{p{\linewidth}}}]{% 
	  \tt {\tiny [GM$\langle$B]}  
	 SPRICH: Meine Objekte sind aktuell an W(1,1), I(2,1), H(3,1), C(4,1), T(5,1). Lasst uns alle in die oberste Zeile verschieben. Ich beginne mit W zu (1,1), dann I zu (2,1), H zu (3,1), C zu (4,1) und T zu (5,1). Teilt mir eure Bewegungen mit, damit wir uns abstimmen können. 
	  } 
	   } 
	   } 
	  \\ 
 

    \theutterance \stepcounter{utterance}  

    & & & \multicolumn{2}{p{0.3\linewidth}}{\cellcolor[rgb]{0.95,0.95,0.95}{%
	\makecell[{{p{\linewidth}}}]{% 
	  \tt {\tiny [GM$|$GM]}  
	 SPRICH: Meine Objekte sind aktuell an W(1,1), I(2,1), H(3,1), C(4,1), T(5,1). Lasst uns alle in die oberste Zeile verschieben. Ich beginne mit W zu (1,1), dann I zu (2,1), H zu (3,1), C zu (4,1) und T zu (5,1). Teilt mir eure Bewegungen mit, damit wir uns abstimmen können. 
	  } 
	   } 
	   } 
	 & & \\ 
 

    \theutterance \stepcounter{utterance}  

    & & \multicolumn{4}{p{0.6\linewidth}}{\cellcolor[rgb]{0.9,0.9,0.9}{%
	\makecell[{{p{\linewidth}}}]{% 
	  \tt {\tiny [A$\langle$GM]}  
	 Deine Nachricht wurde an deinen Mitspieler übermittelt.\\ \tt \\ \tt \\ \tt Ihr seid gerade bei Runde 2 von maximal 20.\\ \tt Ihr habt gemeinsam 0 von maximal 12 Strafpunkten angesammelt.\\ \tt Dein Mitspieler hat dir diese Nachricht geschickt:\\ \tt \\ \tt \\ \tt "Meine Objekte sind aktuell an W(1,1), I(2,1), H(3,1), C(4,1), T(5,1). Lasst uns alle in die oberste Zeile verschieben. Ich beginne mit W zu (1,1), dann I zu (2,1), H zu (3,1), C zu (4,1) und T zu (5,1). Teilt mir eure Bewegungen mit, damit wir uns abstimmen können."\\ \tt \\ \tt \\ \tt Wie lautet dein nächster Befehl? 
	  } 
	   } 
	   } 
	 & \\ 
 

    \theutterance \stepcounter{utterance}  

    & \multicolumn{4}{p{0.6\linewidth}}{\cellcolor[rgb]{0.8,1,0.9}{%
	\makecell[{{p{\linewidth}}}]{% 
	  \tt {\tiny [A$\rangle$GM]}  
	 SPRICH: Es scheint, dass die obere Zeile bereits von deinem Mitspieler belegt ist. Vielleicht sollten wir alle Objekte in die unterste Zeile (Zeile 7) verschieben? Meine Objekte sind aktuell an W(7,2), I(6,2), H(5,3), C(2,4), T(5,5). Teilt mir deine aktuellen Positionen mit, damit wir uns auf eine gemeinsame Zielanordnung einigen können. 
	  } 
	   } 
	   } 
	 & & \\ 
 

    \theutterance \stepcounter{utterance}  

    & & & \multicolumn{2}{p{0.3\linewidth}}{\cellcolor[rgb]{0.95,0.95,0.95}{%
	\makecell[{{p{\linewidth}}}]{% 
	  \tt {\tiny [GM$|$GM]}  
	 SPRICH: Es scheint, dass die obere Zeile bereits von deinem Mitspieler belegt ist. Vielleicht sollten wir alle Objekte in die unterste Zeile (Zeile 7) verschieben? Meine Objekte sind aktuell an W(7,2), I(6,2), H(5,3), C(2,4), T(5,5). Teilt mir deine aktuellen Positionen mit, damit wir uns auf eine gemeinsame Zielanordnung einigen können. 
	  } 
	   } 
	   } 
	 & & \\ 
 

    \theutterance \stepcounter{utterance}  

    & & \multicolumn{4}{p{0.6\linewidth}}{\cellcolor[rgb]{0.9,0.9,0.9}{%
	\makecell[{{p{\linewidth}}}]{% 
	  \tt {\tiny [GM$\rangle$B]}  
	 Deine Nachricht wurde an deinen Mitspieler übermittelt.\\ \tt \\ \tt \\ \tt Ihr seid gerade bei Runde 3 von maximal 20.\\ \tt Ihr habt gemeinsam 0 von maximal 12 Strafpunkten angesammelt.\\ \tt Dein Mitspieler hat dir diese Nachricht geschickt:\\ \tt \\ \tt \\ \tt "Es scheint, dass die obere Zeile bereits von deinem Mitspieler belegt ist. Vielleicht sollten wir alle Objekte in die unterste Zeile (Zeile 7) verschieben? Meine Objekte sind aktuell an W(7,2), I(6,2), H(5,3), C(2,4), T(5,5). Teilt mir deine aktuellen Positionen mit, damit wir uns auf eine gemeinsame Zielanordnung einigen können."\\ \tt \\ \tt \\ \tt Wie lautet dein nächster Befehl? 
	  } 
	   } 
	   } 
	 & \\ 
 

    \theutterance \stepcounter{utterance}  

    & & & \multicolumn{4}{p{0.6\linewidth}}{\cellcolor[rgb]{1,0.85,0.72}{%
	\makecell[{{p{\linewidth}}}]{% 
	  \tt {\tiny [GM$\langle$B]}  
	 SPRICH: Lasst uns alle Objekte in die unterste Zeile (Zeile 7) verschieben. Meine aktuellen Positionen sind W(1,1), I(2,1), H(3,1), C(4,1), T(5,1). Bitte teilt mir eure Bewegungen mit, damit wir uns abstimmen können. 
	  } 
	   } 
	   } 
	  \\ 
 

    \theutterance \stepcounter{utterance}  

    & & & \multicolumn{2}{p{0.3\linewidth}}{\cellcolor[rgb]{0.95,0.95,0.95}{%
	\makecell[{{p{\linewidth}}}]{% 
	  \tt {\tiny [GM$|$GM]}  
	 SPRICH: Lasst uns alle Objekte in die unterste Zeile (Zeile 7) verschieben. Meine aktuellen Positionen sind W(1,1), I(2,1), H(3,1), C(4,1), T(5,1). Bitte teilt mir eure Bewegungen mit, damit wir uns abstimmen können. 
	  } 
	   } 
	   } 
	 & & \\ 
 

    \theutterance \stepcounter{utterance}  

    & & \multicolumn{4}{p{0.6\linewidth}}{\cellcolor[rgb]{0.9,0.9,0.9}{%
	\makecell[{{p{\linewidth}}}]{% 
	  \tt {\tiny [A$\langle$GM]}  
	 Deine Nachricht wurde an deinen Mitspieler übermittelt.\\ \tt \\ \tt \\ \tt Ihr seid gerade bei Runde 3 von maximal 20.\\ \tt Ihr habt gemeinsam 0 von maximal 12 Strafpunkten angesammelt.\\ \tt Dein Mitspieler hat dir diese Nachricht geschickt:\\ \tt \\ \tt \\ \tt "Lasst uns alle Objekte in die unterste Zeile (Zeile 7) verschieben. Meine aktuellen Positionen sind W(1,1), I(2,1), H(3,1), C(4,1), T(5,1). Bitte teilt mir eure Bewegungen mit, damit wir uns abstimmen können."\\ \tt \\ \tt \\ \tt Wie lautet dein nächster Befehl? 
	  } 
	   } 
	   } 
	 & \\ 
 

    \theutterance \stepcounter{utterance}  

    & \multicolumn{4}{p{0.6\linewidth}}{\cellcolor[rgb]{0.8,1,0.9}{%
	\makecell[{{p{\linewidth}}}]{% 
	  \tt {\tiny [A$\rangle$GM]}  
	 VERSCHIEBE: W, (7,7) 
	  } 
	   } 
	   } 
	 & & \\ 
 

    \theutterance \stepcounter{utterance}  

    & & & \multicolumn{2}{p{0.3\linewidth}}{\cellcolor[rgb]{0.95,0.95,0.95}{%
	\makecell[{{p{\linewidth}}}]{% 
	  \tt {\tiny [GM$|$GM]}  
	 VERSCHIEBE: W, (7,7) 
	  } 
	   } 
	   } 
	 & & \\ 
 

    \theutterance \stepcounter{utterance}  

    & & & \multicolumn{2}{p{0.3\linewidth}}{\cellcolor[rgb]{0.95,0.95,0.95}{%
	\makecell[{{p{\linewidth}}}]{% 
	  \tt {\tiny [GM$|$GM]}  
	 'W' erfolgreich nach (7,7) verschoben. Dein Raster sieht jetzt so aus:\\ \tt ```\\ \tt  1234567\\ \tt ╔══╤════╗\\ \tt ║◌◌│◌◌I◌║ 1\\ \tt ╟──┘◌◌┌─╢ 2\\ \tt ║◌◌◌◌H│◌║ 3\\ \tt ╟─┐C┌─┼─╢ 4\\ \tt ║◌├─┤◌│T║ 5\\ \tt ╟─┤◌├─┤◌║ 6\\ \tt ║◌│◌│◌│W║ 7\\ \tt ╚═╧═╧═╧═╝\\ \tt ```\\ \tt  
	  } 
	   } 
	   } 
	 & & \\ 
 

    \theutterance \stepcounter{utterance}  

    & & \multicolumn{4}{p{0.6\linewidth}}{\cellcolor[rgb]{0.9,0.9,0.9}{%
	\makecell[{{p{\linewidth}}}]{% 
	  \tt {\tiny [GM$\rangle$B]}  
	 Deine Nachricht wurde an deinen Mitspieler übermittelt.\\ \tt \\ \tt \\ \tt Ihr seid gerade bei Runde 4 von maximal 20.\\ \tt Ihr habt gemeinsam 0 von maximal 12 Strafpunkten angesammelt.\\ \tt Dein Mitspieler hat ein Objekt auf seinem Raster bewegt.\\ \tt \\ \tt \\ \tt Wie lautet dein nächster Befehl? 
	  } 
	   } 
	   } 
	 & \\ 
 

    \theutterance \stepcounter{utterance}  

    & & & \multicolumn{4}{p{0.6\linewidth}}{\cellcolor[rgb]{1,0.85,0.72}{%
	\makecell[{{p{\linewidth}}}]{% 
	  \tt {\tiny [GM$\langle$B]}  
	 VERSCHIEBE: W, (1,2)  \\ \tt (Ich beginne damit, W von (1,1) nach (1,2) zu bewegen, um den Plan zu verfolgen, alle Objekte in die unterste Zeile zu bringen.) 
	  } 
	   } 
	   } 
	  \\ 
 

    \theutterance \stepcounter{utterance}  

    & & & \multicolumn{2}{p{0.3\linewidth}}{\cellcolor[rgb]{0.95,0.95,0.95}{%
	\makecell[{{p{\linewidth}}}]{% 
	  \tt {\tiny [GM$|$GM]}  
	 VERSCHIEBE: W, (1,2)  \\ \tt (Ich beginne damit, W von (1,1) nach (1,2) zu bewegen, um den Plan zu verfolgen, alle Objekte in die unterste Zeile zu bringen.) 
	  } 
	   } 
	   } 
	 & & \\ 
 

    \theutterance \stepcounter{utterance}  

    & & & \multicolumn{2}{p{0.3\linewidth}}{\cellcolor[rgb]{0.95,0.95,0.95}{%
	\makecell[{{p{\linewidth}}}]{% 
	  \tt {\tiny [GM$|$GM]}  
	 Invalid format: tail is not empty: '  \\ \tt (Ich beginne damit, W von (1,1) nach (1,2) zu bewegen, um den Plan zu verfolgen, alle Objekte in die unterste Zeile zu bringen.) 
	  } 
	   } 
	   } 
	 & & \\ 
 

    \theutterance \stepcounter{utterance}  

    & & \multicolumn{4}{p{0.6\linewidth}}{\cellcolor[rgb]{0.9,0.9,0.9}{%
	\makecell[{{p{\linewidth}}}]{% 
	  \tt {\tiny [GM$\rangle$B]}  
	 Strafpunkt: Deine Nachricht darf nichts nach dem Befehl enthalten! Stelle sicher, dass deine Antwort ausschließlich einen der beiden Befehle `SPRICH: <NACHRICHT>` und `VERSCHIEBE: <OBJECT>, (<X>, <Y>)` und sonst nichts enthält!\\ \tt \\ \tt Ihr habt gemeinsam 1 von maximal 12 Strafpunkten angesammelt.\\ \tt Bitte versuche es noch mal! 
	  } 
	   } 
	   } 
	 & \\ 
 

    \theutterance \stepcounter{utterance}  

    & & & \multicolumn{4}{p{0.6\linewidth}}{\cellcolor[rgb]{1,0.85,0.72}{%
	\makecell[{{p{\linewidth}}}]{% 
	  \tt {\tiny [GM$\langle$B]}  
	 VERSCHIEBE: W, (1,2) 
	  } 
	   } 
	   } 
	  \\ 
 

    \theutterance \stepcounter{utterance}  

    & & & \multicolumn{2}{p{0.3\linewidth}}{\cellcolor[rgb]{0.95,0.95,0.95}{%
	\makecell[{{p{\linewidth}}}]{% 
	  \tt {\tiny [GM$|$GM]}  
	 VERSCHIEBE: W, (1,2) 
	  } 
	   } 
	   } 
	 & & \\ 
 

    \theutterance \stepcounter{utterance}  

    & & & \multicolumn{2}{p{0.3\linewidth}}{\cellcolor[rgb]{0.95,0.95,0.95}{%
	\makecell[{{p{\linewidth}}}]{% 
	  \tt {\tiny [GM$|$GM]}  
	 Strafpunkt: (1,2) ist nicht leer, sondern enthält '─'.\\ \tt \\ \tt Ihr habt gemeinsam 2 von maximal 12 Strafpunkten angesammelt.\\ \tt Bitte versuche es noch mal! 
	  } 
	   } 
	   } 
	 & & \\ 
 

    \theutterance \stepcounter{utterance}  

    & & \multicolumn{4}{p{0.6\linewidth}}{\cellcolor[rgb]{0.9,0.9,0.9}{%
	\makecell[{{p{\linewidth}}}]{% 
	  \tt {\tiny [GM$\rangle$B]}  
	 Strafpunkt: (1,2) ist nicht leer, sondern enthält '─'.\\ \tt \\ \tt Ihr habt gemeinsam 2 von maximal 12 Strafpunkten angesammelt.\\ \tt Bitte versuche es noch mal! 
	  } 
	   } 
	   } 
	 & \\ 
 

    \theutterance \stepcounter{utterance}  

    & & & \multicolumn{4}{p{0.6\linewidth}}{\cellcolor[rgb]{1,0.85,0.72}{%
	\makecell[{{p{\linewidth}}}]{% 
	  \tt {\tiny [GM$\langle$B]}  
	 SPRICH: Die Position (1,2) ist blockiert. Lasst uns W stattdessen nach (1,3) oder (2,1) bewegen. Bitte teilt mir eure Anpassungen mit. 
	  } 
	   } 
	   } 
	  \\ 
 

    \theutterance \stepcounter{utterance}  

    & & & \multicolumn{2}{p{0.3\linewidth}}{\cellcolor[rgb]{0.95,0.95,0.95}{%
	\makecell[{{p{\linewidth}}}]{% 
	  \tt {\tiny [GM$|$GM]}  
	 SPRICH: Die Position (1,2) ist blockiert. Lasst uns W stattdessen nach (1,3) oder (2,1) bewegen. Bitte teilt mir eure Anpassungen mit. 
	  } 
	   } 
	   } 
	 & & \\ 
 

    \theutterance \stepcounter{utterance}  

    & & \multicolumn{4}{p{0.6\linewidth}}{\cellcolor[rgb]{0.9,0.9,0.9}{%
	\makecell[{{p{\linewidth}}}]{% 
	  \tt {\tiny [A$\langle$GM]}  
	 'W' erfolgreich nach (7,7) verschoben. Dein Raster sieht jetzt so aus:\\ \tt ```\\ \tt  1234567\\ \tt ╔══╤════╗\\ \tt ║◌◌│◌◌I◌║ 1\\ \tt ╟──┘◌◌┌─╢ 2\\ \tt ║◌◌◌◌H│◌║ 3\\ \tt ╟─┐C┌─┼─╢ 4\\ \tt ║◌├─┤◌│T║ 5\\ \tt ╟─┤◌├─┤◌║ 6\\ \tt ║◌│◌│◌│W║ 7\\ \tt ╚═╧═╧═╧═╝\\ \tt ```\\ \tt Ihr seid gerade bei Runde 4 von maximal 20.\\ \tt Ihr habt gemeinsam 2 von maximal 12 Strafpunkten angesammelt.\\ \tt Dein Mitspieler hat dir diese Nachricht geschickt:\\ \tt \\ \tt \\ \tt "Die Position (1,2) ist blockiert. Lasst uns W stattdessen nach (1,3) oder (2,1) bewegen. Bitte teilt mir eure Anpassungen mit."\\ \tt \\ \tt \\ \tt Wie lautet dein nächster Befehl? 
	  } 
	   } 
	   } 
	 & \\ 
 

    \theutterance \stepcounter{utterance}  

    & \multicolumn{4}{p{0.6\linewidth}}{\cellcolor[rgb]{0.8,1,0.9}{%
	\makecell[{{p{\linewidth}}}]{% 
	  \tt {\tiny [A$\rangle$GM]}  
	 VERSCHIEBE: W, (1,3) 
	  } 
	   } 
	   } 
	 & & \\ 
 

    \theutterance \stepcounter{utterance}  

    & & & \multicolumn{2}{p{0.3\linewidth}}{\cellcolor[rgb]{0.95,0.95,0.95}{%
	\makecell[{{p{\linewidth}}}]{% 
	  \tt {\tiny [GM$|$GM]}  
	 VERSCHIEBE: W, (1,3) 
	  } 
	   } 
	   } 
	 & & \\ 
 

    \theutterance \stepcounter{utterance}  

    & & & \multicolumn{2}{p{0.3\linewidth}}{\cellcolor[rgb]{0.95,0.95,0.95}{%
	\makecell[{{p{\linewidth}}}]{% 
	  \tt {\tiny [GM$|$GM]}  
	 'W' erfolgreich nach (1,3) verschoben. Dein Raster sieht jetzt so aus:\\ \tt ```\\ \tt  1234567\\ \tt ╔══╤════╗\\ \tt ║◌◌│◌◌I◌║ 1\\ \tt ╟──┘◌◌┌─╢ 2\\ \tt ║W◌◌◌H│◌║ 3\\ \tt ╟─┐C┌─┼─╢ 4\\ \tt ║◌├─┤◌│T║ 5\\ \tt ╟─┤◌├─┤◌║ 6\\ \tt ║◌│◌│◌│◌║ 7\\ \tt ╚═╧═╧═╧═╝\\ \tt ```\\ \tt  
	  } 
	   } 
	   } 
	 & & \\ 
 

    \theutterance \stepcounter{utterance}  

    & & \multicolumn{4}{p{0.6\linewidth}}{\cellcolor[rgb]{0.9,0.9,0.9}{%
	\makecell[{{p{\linewidth}}}]{% 
	  \tt {\tiny [GM$\rangle$B]}  
	 Deine Nachricht wurde an deinen Mitspieler übermittelt.\\ \tt \\ \tt \\ \tt Ihr seid gerade bei Runde 5 von maximal 20.\\ \tt Ihr habt gemeinsam 2 von maximal 12 Strafpunkten angesammelt.\\ \tt Dein Mitspieler hat ein Objekt auf seinem Raster bewegt.\\ \tt \\ \tt \\ \tt Wie lautet dein nächster Befehl? 
	  } 
	   } 
	   } 
	 & \\ 
 

    \theutterance \stepcounter{utterance}  

    & & & \multicolumn{4}{p{0.6\linewidth}}{\cellcolor[rgb]{1,0.85,0.72}{%
	\makecell[{{p{\linewidth}}}]{% 
	  \tt {\tiny [GM$\langle$B]}  
	 VERSCHIEBE: W, (1,3) 
	  } 
	   } 
	   } 
	  \\ 
 

    \theutterance \stepcounter{utterance}  

    & & & \multicolumn{2}{p{0.3\linewidth}}{\cellcolor[rgb]{0.95,0.95,0.95}{%
	\makecell[{{p{\linewidth}}}]{% 
	  \tt {\tiny [GM$|$GM]}  
	 VERSCHIEBE: W, (1,3) 
	  } 
	   } 
	   } 
	 & & \\ 
 

    \theutterance \stepcounter{utterance}  

    & & & \multicolumn{2}{p{0.3\linewidth}}{\cellcolor[rgb]{0.95,0.95,0.95}{%
	\makecell[{{p{\linewidth}}}]{% 
	  \tt {\tiny [GM$|$GM]}  
	 'W' erfolgreich nach (1,3) verschoben. Dein Raster sieht jetzt so aus:\\ \tt ```\\ \tt  1234567\\ \tt ╔══╤════╗\\ \tt ║◌◌│◌◌◌◌║ 1\\ \tt ╟──┘◌T┌─╢ 2\\ \tt ║W◌H◌◌│◌║ 3\\ \tt ╟─┐◌┌─┼─╢ 4\\ \tt ║◌├─┤◌│◌║ 5\\ \tt ╟─┤◌├─┤◌║ 6\\ \tt ║◌│I│C│◌║ 7\\ \tt ╚═╧═╧═╧═╝\\ \tt ```\\ \tt  
	  } 
	   } 
	   } 
	 & & \\ 
 

    \theutterance \stepcounter{utterance}  

    & & \multicolumn{4}{p{0.6\linewidth}}{\cellcolor[rgb]{0.9,0.9,0.9}{%
	\makecell[{{p{\linewidth}}}]{% 
	  \tt {\tiny [A$\langle$GM]}  
	 'W' erfolgreich nach (1,3) verschoben. Dein Raster sieht jetzt so aus:\\ \tt ```\\ \tt  1234567\\ \tt ╔══╤════╗\\ \tt ║◌◌│◌◌I◌║ 1\\ \tt ╟──┘◌◌┌─╢ 2\\ \tt ║W◌◌◌H│◌║ 3\\ \tt ╟─┐C┌─┼─╢ 4\\ \tt ║◌├─┤◌│T║ 5\\ \tt ╟─┤◌├─┤◌║ 6\\ \tt ║◌│◌│◌│◌║ 7\\ \tt ╚═╧═╧═╧═╝\\ \tt ```\\ \tt Ihr seid gerade bei Runde 5 von maximal 20.\\ \tt Ihr habt gemeinsam 2 von maximal 12 Strafpunkten angesammelt.\\ \tt Dein Mitspieler hat ein Objekt auf seinem Raster bewegt.\\ \tt \\ \tt \\ \tt Wie lautet dein nächster Befehl? 
	  } 
	   } 
	   } 
	 & \\ 
 

    \theutterance \stepcounter{utterance}  

    & \multicolumn{4}{p{0.6\linewidth}}{\cellcolor[rgb]{0.8,1,0.9}{%
	\makecell[{{p{\linewidth}}}]{% 
	  \tt {\tiny [A$\rangle$GM]}  
	 SPRICH: Meine aktuellen Positionen sind W(1,3), I(6,2), H(5,3), C(2,4), T(5,5). Dein Mitspieler hat ein Objekt bewegt – teilt mir bitte deine neuen Positionen mit, damit wir eine gemeinsame Zielanordnung planen können. Vielleicht sollten wir alle Objekte in die mittlere Zeile (Zeile 4) verschieben? 
	  } 
	   } 
	   } 
	 & & \\ 
 

    \theutterance \stepcounter{utterance}  

    & & & \multicolumn{2}{p{0.3\linewidth}}{\cellcolor[rgb]{0.95,0.95,0.95}{%
	\makecell[{{p{\linewidth}}}]{% 
	  \tt {\tiny [GM$|$GM]}  
	 SPRICH: Meine aktuellen Positionen sind W(1,3), I(6,2), H(5,3), C(2,4), T(5,5). Dein Mitspieler hat ein Objekt bewegt – teilt mir bitte deine neuen Positionen mit, damit wir eine gemeinsame Zielanordnung planen können. Vielleicht sollten wir alle Objekte in die mittlere Zeile (Zeile 4) verschieben? 
	  } 
	   } 
	   } 
	 & & \\ 
 

    \theutterance \stepcounter{utterance}  

    & & \multicolumn{4}{p{0.6\linewidth}}{\cellcolor[rgb]{0.9,0.9,0.9}{%
	\makecell[{{p{\linewidth}}}]{% 
	  \tt {\tiny [GM$\rangle$B]}  
	 'W' erfolgreich nach (1,3) verschoben. Dein Raster sieht jetzt so aus:\\ \tt ```\\ \tt  1234567\\ \tt ╔══╤════╗\\ \tt ║◌◌│◌◌◌◌║ 1\\ \tt ╟──┘◌T┌─╢ 2\\ \tt ║W◌H◌◌│◌║ 3\\ \tt ╟─┐◌┌─┼─╢ 4\\ \tt ║◌├─┤◌│◌║ 5\\ \tt ╟─┤◌├─┤◌║ 6\\ \tt ║◌│I│C│◌║ 7\\ \tt ╚═╧═╧═╧═╝\\ \tt ```\\ \tt Ihr seid gerade bei Runde 6 von maximal 20.\\ \tt Ihr habt gemeinsam 2 von maximal 12 Strafpunkten angesammelt.\\ \tt Dein Mitspieler hat dir diese Nachricht geschickt:\\ \tt \\ \tt \\ \tt "Meine aktuellen Positionen sind W(1,3), I(6,2), H(5,3), C(2,4), T(5,5). Dein Mitspieler hat ein Objekt bewegt – teilt mir bitte deine neuen Positionen mit, damit wir eine gemeinsame Zielanordnung planen können. Vielleicht sollten wir alle Objekte in die mittlere Zeile (Zeile 4) verschieben?"\\ \tt \\ \tt \\ \tt Wie lautet dein nächster Befehl? 
	  } 
	   } 
	   } 
	 & \\ 
 

    \theutterance \stepcounter{utterance}  

    & & & \multicolumn{4}{p{0.6\linewidth}}{\cellcolor[rgb]{1,0.85,0.72}{%
	\makecell[{{p{\linewidth}}}]{% 
	  \tt {\tiny [GM$\langle$B]}  
	 VERSCHIEBE: W, (1,4) 
	  } 
	   } 
	   } 
	  \\ 
 

    \theutterance \stepcounter{utterance}  

    & & & \multicolumn{2}{p{0.3\linewidth}}{\cellcolor[rgb]{0.95,0.95,0.95}{%
	\makecell[{{p{\linewidth}}}]{% 
	  \tt {\tiny [GM$|$GM]}  
	 VERSCHIEBE: W, (1,4) 
	  } 
	   } 
	   } 
	 & & \\ 
 

    \theutterance \stepcounter{utterance}  

    & & & \multicolumn{2}{p{0.3\linewidth}}{\cellcolor[rgb]{0.95,0.95,0.95}{%
	\makecell[{{p{\linewidth}}}]{% 
	  \tt {\tiny [GM$|$GM]}  
	 Strafpunkt: (1,4) ist nicht leer, sondern enthält '─'.\\ \tt \\ \tt Ihr habt gemeinsam 3 von maximal 12 Strafpunkten angesammelt.\\ \tt Bitte versuche es noch mal! 
	  } 
	   } 
	   } 
	 & & \\ 
 

    \theutterance \stepcounter{utterance}  

    & & \multicolumn{4}{p{0.6\linewidth}}{\cellcolor[rgb]{0.9,0.9,0.9}{%
	\makecell[{{p{\linewidth}}}]{% 
	  \tt {\tiny [GM$\rangle$B]}  
	 Strafpunkt: (1,4) ist nicht leer, sondern enthält '─'.\\ \tt \\ \tt Ihr habt gemeinsam 3 von maximal 12 Strafpunkten angesammelt.\\ \tt Bitte versuche es noch mal! 
	  } 
	   } 
	   } 
	 & \\ 
 

    \theutterance \stepcounter{utterance}  

    & & & \multicolumn{4}{p{0.6\linewidth}}{\cellcolor[rgb]{1,0.85,0.72}{%
	\makecell[{{p{\linewidth}}}]{% 
	  \tt {\tiny [GM$\langle$B]}  
	 SPRICH: Die mittlere Zeile (Zeile 4) hat Blockaden. Vielleicht sollten wir alle Objekte in die unterste Zeile (Zeile 7) verschieben? Meine aktuelle Position ist W(1,3). Teilt mir eure Anpassungen mit. 
	  } 
	   } 
	   } 
	  \\ 
 

    \theutterance \stepcounter{utterance}  

    & & & \multicolumn{2}{p{0.3\linewidth}}{\cellcolor[rgb]{0.95,0.95,0.95}{%
	\makecell[{{p{\linewidth}}}]{% 
	  \tt {\tiny [GM$|$GM]}  
	 SPRICH: Die mittlere Zeile (Zeile 4) hat Blockaden. Vielleicht sollten wir alle Objekte in die unterste Zeile (Zeile 7) verschieben? Meine aktuelle Position ist W(1,3). Teilt mir eure Anpassungen mit. 
	  } 
	   } 
	   } 
	 & & \\ 
 

    \theutterance \stepcounter{utterance}  

    & & \multicolumn{4}{p{0.6\linewidth}}{\cellcolor[rgb]{0.9,0.9,0.9}{%
	\makecell[{{p{\linewidth}}}]{% 
	  \tt {\tiny [A$\langle$GM]}  
	 Deine Nachricht wurde an deinen Mitspieler übermittelt.\\ \tt \\ \tt \\ \tt Ihr seid gerade bei Runde 6 von maximal 20.\\ \tt Ihr habt gemeinsam 3 von maximal 12 Strafpunkten angesammelt.\\ \tt Dein Mitspieler hat dir diese Nachricht geschickt:\\ \tt \\ \tt \\ \tt "Die mittlere Zeile (Zeile 4) hat Blockaden. Vielleicht sollten wir alle Objekte in die unterste Zeile (Zeile 7) verschieben? Meine aktuelle Position ist W(1,3). Teilt mir eure Anpassungen mit."\\ \tt \\ \tt \\ \tt Wie lautet dein nächster Befehl? 
	  } 
	   } 
	   } 
	 & \\ 
 

    \theutterance \stepcounter{utterance}  

    & \multicolumn{4}{p{0.6\linewidth}}{\cellcolor[rgb]{0.8,1,0.9}{%
	\makecell[{{p{\linewidth}}}]{% 
	  \tt {\tiny [A$\rangle$GM]}  
	 VERSCHIEBE: W, (1,7) 
	  } 
	   } 
	   } 
	 & & \\ 
 

    \theutterance \stepcounter{utterance}  

    & & & \multicolumn{2}{p{0.3\linewidth}}{\cellcolor[rgb]{0.95,0.95,0.95}{%
	\makecell[{{p{\linewidth}}}]{% 
	  \tt {\tiny [GM$|$GM]}  
	 VERSCHIEBE: W, (1,7) 
	  } 
	   } 
	   } 
	 & & \\ 
 

    \theutterance \stepcounter{utterance}  

    & & & \multicolumn{2}{p{0.3\linewidth}}{\cellcolor[rgb]{0.95,0.95,0.95}{%
	\makecell[{{p{\linewidth}}}]{% 
	  \tt {\tiny [GM$|$GM]}  
	 'W' erfolgreich nach (1,7) verschoben. Dein Raster sieht jetzt so aus:\\ \tt ```\\ \tt  1234567\\ \tt ╔══╤════╗\\ \tt ║◌◌│◌◌I◌║ 1\\ \tt ╟──┘◌◌┌─╢ 2\\ \tt ║◌◌◌◌H│◌║ 3\\ \tt ╟─┐C┌─┼─╢ 4\\ \tt ║◌├─┤◌│T║ 5\\ \tt ╟─┤◌├─┤◌║ 6\\ \tt ║W│◌│◌│◌║ 7\\ \tt ╚═╧═╧═╧═╝\\ \tt ```\\ \tt  
	  } 
	   } 
	   } 
	 & & \\ 
 

    \theutterance \stepcounter{utterance}  

    & & \multicolumn{4}{p{0.6\linewidth}}{\cellcolor[rgb]{0.9,0.9,0.9}{%
	\makecell[{{p{\linewidth}}}]{% 
	  \tt {\tiny [GM$\rangle$B]}  
	 Deine Nachricht wurde an deinen Mitspieler übermittelt.\\ \tt \\ \tt \\ \tt Ihr seid gerade bei Runde 7 von maximal 20.\\ \tt Ihr habt gemeinsam 3 von maximal 12 Strafpunkten angesammelt.\\ \tt Dein Mitspieler hat ein Objekt auf seinem Raster bewegt.\\ \tt \\ \tt \\ \tt Wie lautet dein nächster Befehl? 
	  } 
	   } 
	   } 
	 & \\ 
 

    \theutterance \stepcounter{utterance}  

    & & & \multicolumn{4}{p{0.6\linewidth}}{\cellcolor[rgb]{1,0.85,0.72}{%
	\makecell[{{p{\linewidth}}}]{% 
	  \tt {\tiny [GM$\langle$B]}  
	 VERSCHIEBE: W, (1,5) 
	  } 
	   } 
	   } 
	  \\ 
 

    \theutterance \stepcounter{utterance}  

    & & & \multicolumn{2}{p{0.3\linewidth}}{\cellcolor[rgb]{0.95,0.95,0.95}{%
	\makecell[{{p{\linewidth}}}]{% 
	  \tt {\tiny [GM$|$GM]}  
	 VERSCHIEBE: W, (1,5) 
	  } 
	   } 
	   } 
	 & & \\ 
 

    \theutterance \stepcounter{utterance}  

    & & & \multicolumn{2}{p{0.3\linewidth}}{\cellcolor[rgb]{0.95,0.95,0.95}{%
	\makecell[{{p{\linewidth}}}]{% 
	  \tt {\tiny [GM$|$GM]}  
	 'W' erfolgreich nach (1,5) verschoben. Dein Raster sieht jetzt so aus:\\ \tt ```\\ \tt  1234567\\ \tt ╔══╤════╗\\ \tt ║◌◌│◌◌◌◌║ 1\\ \tt ╟──┘◌T┌─╢ 2\\ \tt ║◌◌H◌◌│◌║ 3\\ \tt ╟─┐◌┌─┼─╢ 4\\ \tt ║W├─┤◌│◌║ 5\\ \tt ╟─┤◌├─┤◌║ 6\\ \tt ║◌│I│C│◌║ 7\\ \tt ╚═╧═╧═╧═╝\\ \tt ```\\ \tt  
	  } 
	   } 
	   } 
	 & & \\ 
 

    \theutterance \stepcounter{utterance}  

    & & \multicolumn{4}{p{0.6\linewidth}}{\cellcolor[rgb]{0.9,0.9,0.9}{%
	\makecell[{{p{\linewidth}}}]{% 
	  \tt {\tiny [A$\langle$GM]}  
	 'W' erfolgreich nach (1,7) verschoben. Dein Raster sieht jetzt so aus:\\ \tt ```\\ \tt  1234567\\ \tt ╔══╤════╗\\ \tt ║◌◌│◌◌I◌║ 1\\ \tt ╟──┘◌◌┌─╢ 2\\ \tt ║◌◌◌◌H│◌║ 3\\ \tt ╟─┐C┌─┼─╢ 4\\ \tt ║◌├─┤◌│T║ 5\\ \tt ╟─┤◌├─┤◌║ 6\\ \tt ║W│◌│◌│◌║ 7\\ \tt ╚═╧═╧═╧═╝\\ \tt ```\\ \tt Ihr seid gerade bei Runde 7 von maximal 20.\\ \tt Ihr habt gemeinsam 3 von maximal 12 Strafpunkten angesammelt.\\ \tt Dein Mitspieler hat ein Objekt auf seinem Raster bewegt.\\ \tt \\ \tt \\ \tt Wie lautet dein nächster Befehl? 
	  } 
	   } 
	   } 
	 & \\ 
 

    \theutterance \stepcounter{utterance}  

    & \multicolumn{4}{p{0.6\linewidth}}{\cellcolor[rgb]{0.8,1,0.9}{%
	\makecell[{{p{\linewidth}}}]{% 
	  \tt {\tiny [A$\rangle$GM]}  
	 VERSCHIEBE: I, (6,7) 
	  } 
	   } 
	   } 
	 & & \\ 
 

    \theutterance \stepcounter{utterance}  

    & & & \multicolumn{2}{p{0.3\linewidth}}{\cellcolor[rgb]{0.95,0.95,0.95}{%
	\makecell[{{p{\linewidth}}}]{% 
	  \tt {\tiny [GM$|$GM]}  
	 VERSCHIEBE: I, (6,7) 
	  } 
	   } 
	   } 
	 & & \\ 
 

    \theutterance \stepcounter{utterance}  

    & & & \multicolumn{2}{p{0.3\linewidth}}{\cellcolor[rgb]{0.95,0.95,0.95}{%
	\makecell[{{p{\linewidth}}}]{% 
	  \tt {\tiny [GM$|$GM]}  
	 Strafpunkt: (6,7) ist nicht leer, sondern enthält '│'.\\ \tt \\ \tt Ihr habt gemeinsam 4 von maximal 12 Strafpunkten angesammelt.\\ \tt Bitte versuche es noch mal! 
	  } 
	   } 
	   } 
	 & & \\ 
 

    \theutterance \stepcounter{utterance}  

    & & \multicolumn{4}{p{0.6\linewidth}}{\cellcolor[rgb]{0.9,0.9,0.9}{%
	\makecell[{{p{\linewidth}}}]{% 
	  \tt {\tiny [A$\langle$GM]}  
	 Strafpunkt: (6,7) ist nicht leer, sondern enthält '│'.\\ \tt \\ \tt Ihr habt gemeinsam 4 von maximal 12 Strafpunkten angesammelt.\\ \tt Bitte versuche es noch mal! 
	  } 
	   } 
	   } 
	 & \\ 
 

    \theutterance \stepcounter{utterance}  

    & \multicolumn{4}{p{0.6\linewidth}}{\cellcolor[rgb]{0.8,1,0.9}{%
	\makecell[{{p{\linewidth}}}]{% 
	  \tt {\tiny [A$\rangle$GM]}  
	 VERSCHIEBE: I, (6,3) 
	  } 
	   } 
	   } 
	 & & \\ 
 

    \theutterance \stepcounter{utterance}  

    & & & \multicolumn{2}{p{0.3\linewidth}}{\cellcolor[rgb]{0.95,0.95,0.95}{%
	\makecell[{{p{\linewidth}}}]{% 
	  \tt {\tiny [GM$|$GM]}  
	 VERSCHIEBE: I, (6,3) 
	  } 
	   } 
	   } 
	 & & \\ 
 

    \theutterance \stepcounter{utterance}  

    & & & \multicolumn{2}{p{0.3\linewidth}}{\cellcolor[rgb]{0.95,0.95,0.95}{%
	\makecell[{{p{\linewidth}}}]{% 
	  \tt {\tiny [GM$|$GM]}  
	 Strafpunkt: (6,3) ist nicht leer, sondern enthält '│'.\\ \tt \\ \tt Ihr habt gemeinsam 5 von maximal 12 Strafpunkten angesammelt.\\ \tt Bitte versuche es noch mal! 
	  } 
	   } 
	   } 
	 & & \\ 
 

    \theutterance \stepcounter{utterance}  

    & & \multicolumn{4}{p{0.6\linewidth}}{\cellcolor[rgb]{0.9,0.9,0.9}{%
	\makecell[{{p{\linewidth}}}]{% 
	  \tt {\tiny [A$\langle$GM]}  
	 Strafpunkt: (6,3) ist nicht leer, sondern enthält '│'.\\ \tt \\ \tt Ihr habt gemeinsam 5 von maximal 12 Strafpunkten angesammelt.\\ \tt Bitte versuche es noch mal! 
	  } 
	   } 
	   } 
	 & \\ 
 

    \theutterance \stepcounter{utterance}  

    & \multicolumn{4}{p{0.6\linewidth}}{\cellcolor[rgb]{0.8,1,0.9}{%
	\makecell[{{p{\linewidth}}}]{% 
	  \tt {\tiny [A$\rangle$GM]}  
	 VERSCHIEBE: I, (6,5) 
	  } 
	   } 
	   } 
	 & & \\ 
 

    \theutterance \stepcounter{utterance}  

    & & & \multicolumn{2}{p{0.3\linewidth}}{\cellcolor[rgb]{0.95,0.95,0.95}{%
	\makecell[{{p{\linewidth}}}]{% 
	  \tt {\tiny [GM$|$GM]}  
	 VERSCHIEBE: I, (6,5) 
	  } 
	   } 
	   } 
	 & & \\ 
 

    \theutterance \stepcounter{utterance}  

    & & & \multicolumn{2}{p{0.3\linewidth}}{\cellcolor[rgb]{0.95,0.95,0.95}{%
	\makecell[{{p{\linewidth}}}]{% 
	  \tt {\tiny [GM$|$GM]}  
	 Strafpunkt: (6,5) ist nicht leer, sondern enthält '│'.\\ \tt \\ \tt Ihr habt gemeinsam 6 von maximal 12 Strafpunkten angesammelt.\\ \tt Bitte versuche es noch mal! 
	  } 
	   } 
	   } 
	 & & \\ 
 

    \theutterance \stepcounter{utterance}  

    & & \multicolumn{4}{p{0.6\linewidth}}{\cellcolor[rgb]{0.9,0.9,0.9}{%
	\makecell[{{p{\linewidth}}}]{% 
	  \tt {\tiny [A$\langle$GM]}  
	 Strafpunkt: (6,5) ist nicht leer, sondern enthält '│'.\\ \tt \\ \tt Ihr habt gemeinsam 6 von maximal 12 Strafpunkten angesammelt.\\ \tt Bitte versuche es noch mal! 
	  } 
	   } 
	   } 
	 & \\ 
 

    \theutterance \stepcounter{utterance}  

    & \multicolumn{4}{p{0.6\linewidth}}{\cellcolor[rgb]{0.8,1,0.9}{%
	\makecell[{{p{\linewidth}}}]{% 
	  \tt {\tiny [A$\rangle$GM]}  
	 VERSCHIEBE: I, (5,2) 
	  } 
	   } 
	   } 
	 & & \\ 
 

    \theutterance \stepcounter{utterance}  

    & & & \multicolumn{2}{p{0.3\linewidth}}{\cellcolor[rgb]{0.95,0.95,0.95}{%
	\makecell[{{p{\linewidth}}}]{% 
	  \tt {\tiny [GM$|$GM]}  
	 VERSCHIEBE: I, (5,2) 
	  } 
	   } 
	   } 
	 & & \\ 
 

    \theutterance \stepcounter{utterance}  

    & & & \multicolumn{2}{p{0.3\linewidth}}{\cellcolor[rgb]{0.95,0.95,0.95}{%
	\makecell[{{p{\linewidth}}}]{% 
	  \tt {\tiny [GM$|$GM]}  
	 'I' erfolgreich nach (5,2) verschoben. Dein Raster sieht jetzt so aus:\\ \tt ```\\ \tt  1234567\\ \tt ╔══╤════╗\\ \tt ║◌◌│◌◌◌◌║ 1\\ \tt ╟──┘◌I┌─╢ 2\\ \tt ║◌◌◌◌H│◌║ 3\\ \tt ╟─┐C┌─┼─╢ 4\\ \tt ║◌├─┤◌│T║ 5\\ \tt ╟─┤◌├─┤◌║ 6\\ \tt ║W│◌│◌│◌║ 7\\ \tt ╚═╧═╧═╧═╝\\ \tt ```\\ \tt  
	  } 
	   } 
	   } 
	 & & \\ 
 

    \theutterance \stepcounter{utterance}  

    & & \multicolumn{4}{p{0.6\linewidth}}{\cellcolor[rgb]{0.9,0.9,0.9}{%
	\makecell[{{p{\linewidth}}}]{% 
	  \tt {\tiny [GM$\rangle$B]}  
	 'W' erfolgreich nach (1,5) verschoben. Dein Raster sieht jetzt so aus:\\ \tt ```\\ \tt  1234567\\ \tt ╔══╤════╗\\ \tt ║◌◌│◌◌◌◌║ 1\\ \tt ╟──┘◌T┌─╢ 2\\ \tt ║◌◌H◌◌│◌║ 3\\ \tt ╟─┐◌┌─┼─╢ 4\\ \tt ║W├─┤◌│◌║ 5\\ \tt ╟─┤◌├─┤◌║ 6\\ \tt ║◌│I│C│◌║ 7\\ \tt ╚═╧═╧═╧═╝\\ \tt ```\\ \tt Ihr seid gerade bei Runde 8 von maximal 20.\\ \tt Ihr habt gemeinsam 6 von maximal 12 Strafpunkten angesammelt.\\ \tt Dein Mitspieler hat ein Objekt auf seinem Raster bewegt.\\ \tt \\ \tt \\ \tt Wie lautet dein nächster Befehl? 
	  } 
	   } 
	   } 
	 & \\ 
 

    \theutterance \stepcounter{utterance}  

    & & & \multicolumn{4}{p{0.6\linewidth}}{\cellcolor[rgb]{1,0.85,0.72}{%
	\makecell[{{p{\linewidth}}}]{% 
	  \tt {\tiny [GM$\langle$B]}  
	 VERSCHIEBE: W, (1,6) 
	  } 
	   } 
	   } 
	  \\ 
 

    \theutterance \stepcounter{utterance}  

    & & & \multicolumn{2}{p{0.3\linewidth}}{\cellcolor[rgb]{0.95,0.95,0.95}{%
	\makecell[{{p{\linewidth}}}]{% 
	  \tt {\tiny [GM$|$GM]}  
	 VERSCHIEBE: W, (1,6) 
	  } 
	   } 
	   } 
	 & & \\ 
 

    \theutterance \stepcounter{utterance}  

    & & & \multicolumn{2}{p{0.3\linewidth}}{\cellcolor[rgb]{0.95,0.95,0.95}{%
	\makecell[{{p{\linewidth}}}]{% 
	  \tt {\tiny [GM$|$GM]}  
	 Strafpunkt: (1,6) ist nicht leer, sondern enthält '─'.\\ \tt \\ \tt Ihr habt gemeinsam 7 von maximal 12 Strafpunkten angesammelt.\\ \tt Bitte versuche es noch mal! 
	  } 
	   } 
	   } 
	 & & \\ 
 

    \theutterance \stepcounter{utterance}  

    & & \multicolumn{4}{p{0.6\linewidth}}{\cellcolor[rgb]{0.9,0.9,0.9}{%
	\makecell[{{p{\linewidth}}}]{% 
	  \tt {\tiny [GM$\rangle$B]}  
	 Strafpunkt: (1,6) ist nicht leer, sondern enthält '─'.\\ \tt \\ \tt Ihr habt gemeinsam 7 von maximal 12 Strafpunkten angesammelt.\\ \tt Bitte versuche es noch mal! 
	  } 
	   } 
	   } 
	 & \\ 
 

    \theutterance \stepcounter{utterance}  

    & & & \multicolumn{4}{p{0.6\linewidth}}{\cellcolor[rgb]{1,0.85,0.72}{%
	\makecell[{{p{\linewidth}}}]{% 
	  \tt {\tiny [GM$\langle$B]}  
	 VERSCHIEBE: W, (2,5) 
	  } 
	   } 
	   } 
	  \\ 
 

    \theutterance \stepcounter{utterance}  

    & & & \multicolumn{2}{p{0.3\linewidth}}{\cellcolor[rgb]{0.95,0.95,0.95}{%
	\makecell[{{p{\linewidth}}}]{% 
	  \tt {\tiny [GM$|$GM]}  
	 VERSCHIEBE: W, (2,5) 
	  } 
	   } 
	   } 
	 & & \\ 
 

    \theutterance \stepcounter{utterance}  

    & & & \multicolumn{2}{p{0.3\linewidth}}{\cellcolor[rgb]{0.95,0.95,0.95}{%
	\makecell[{{p{\linewidth}}}]{% 
	  \tt {\tiny [GM$|$GM]}  
	 Strafpunkt: (2,5) ist nicht leer, sondern enthält '├'.\\ \tt \\ \tt Ihr habt gemeinsam 8 von maximal 12 Strafpunkten angesammelt.\\ \tt Bitte versuche es noch mal! 
	  } 
	   } 
	   } 
	 & & \\ 
 

    \theutterance \stepcounter{utterance}  

    & & \multicolumn{4}{p{0.6\linewidth}}{\cellcolor[rgb]{0.9,0.9,0.9}{%
	\makecell[{{p{\linewidth}}}]{% 
	  \tt {\tiny [GM$\rangle$B]}  
	 Strafpunkt: (2,5) ist nicht leer, sondern enthält '├'.\\ \tt \\ \tt Ihr habt gemeinsam 8 von maximal 12 Strafpunkten angesammelt.\\ \tt Bitte versuche es noch mal! 
	  } 
	   } 
	   } 
	 & \\ 
 

    \theutterance \stepcounter{utterance}  

    & & & \multicolumn{4}{p{0.6\linewidth}}{\cellcolor[rgb]{1,0.85,0.72}{%
	\makecell[{{p{\linewidth}}}]{% 
	  \tt {\tiny [GM$\langle$B]}  
	 VERSCHIEBE: W, (1,7) 
	  } 
	   } 
	   } 
	  \\ 
 

    \theutterance \stepcounter{utterance}  

    & & & \multicolumn{2}{p{0.3\linewidth}}{\cellcolor[rgb]{0.95,0.95,0.95}{%
	\makecell[{{p{\linewidth}}}]{% 
	  \tt {\tiny [GM$|$GM]}  
	 VERSCHIEBE: W, (1,7) 
	  } 
	   } 
	   } 
	 & & \\ 
 

    \theutterance \stepcounter{utterance}  

    & & & \multicolumn{2}{p{0.3\linewidth}}{\cellcolor[rgb]{0.95,0.95,0.95}{%
	\makecell[{{p{\linewidth}}}]{% 
	  \tt {\tiny [GM$|$GM]}  
	 'W' erfolgreich nach (1,7) verschoben. Dein Raster sieht jetzt so aus:\\ \tt ```\\ \tt  1234567\\ \tt ╔══╤════╗\\ \tt ║◌◌│◌◌◌◌║ 1\\ \tt ╟──┘◌T┌─╢ 2\\ \tt ║◌◌H◌◌│◌║ 3\\ \tt ╟─┐◌┌─┼─╢ 4\\ \tt ║◌├─┤◌│◌║ 5\\ \tt ╟─┤◌├─┤◌║ 6\\ \tt ║W│I│C│◌║ 7\\ \tt ╚═╧═╧═╧═╝\\ \tt ```\\ \tt  
	  } 
	   } 
	   } 
	 & & \\ 
 

    \theutterance \stepcounter{utterance}  

    & & \multicolumn{4}{p{0.6\linewidth}}{\cellcolor[rgb]{0.9,0.9,0.9}{%
	\makecell[{{p{\linewidth}}}]{% 
	  \tt {\tiny [A$\langle$GM]}  
	 'I' erfolgreich nach (5,2) verschoben. Dein Raster sieht jetzt so aus:\\ \tt ```\\ \tt  1234567\\ \tt ╔══╤════╗\\ \tt ║◌◌│◌◌◌◌║ 1\\ \tt ╟──┘◌I┌─╢ 2\\ \tt ║◌◌◌◌H│◌║ 3\\ \tt ╟─┐C┌─┼─╢ 4\\ \tt ║◌├─┤◌│T║ 5\\ \tt ╟─┤◌├─┤◌║ 6\\ \tt ║W│◌│◌│◌║ 7\\ \tt ╚═╧═╧═╧═╝\\ \tt ```\\ \tt Ihr seid gerade bei Runde 8 von maximal 20.\\ \tt Ihr habt gemeinsam 8 von maximal 12 Strafpunkten angesammelt.\\ \tt Dein Mitspieler hat ein Objekt auf seinem Raster bewegt.\\ \tt \\ \tt \\ \tt Wie lautet dein nächster Befehl? 
	  } 
	   } 
	   } 
	 & \\ 
 

    \theutterance \stepcounter{utterance}  

    & \multicolumn{4}{p{0.6\linewidth}}{\cellcolor[rgb]{0.8,1,0.9}{%
	\makecell[{{p{\linewidth}}}]{% 
	  \tt {\tiny [A$\rangle$GM]}  
	 SPRICH: Lasst uns alle Objekte in die unterste Zeile (Zeile 7) verschieben. Meine aktuellen Positionen sind W(1,7), I(5,2), H(5,3), C(2,4), T(5,5). Bitte teilt mir deine aktuellen Positionen mit, damit wir uns abstimmen können. 
	  } 
	   } 
	   } 
	 & & \\ 
 

    \theutterance \stepcounter{utterance}  

    & & & \multicolumn{2}{p{0.3\linewidth}}{\cellcolor[rgb]{0.95,0.95,0.95}{%
	\makecell[{{p{\linewidth}}}]{% 
	  \tt {\tiny [GM$|$GM]}  
	 SPRICH: Lasst uns alle Objekte in die unterste Zeile (Zeile 7) verschieben. Meine aktuellen Positionen sind W(1,7), I(5,2), H(5,3), C(2,4), T(5,5). Bitte teilt mir deine aktuellen Positionen mit, damit wir uns abstimmen können. 
	  } 
	   } 
	   } 
	 & & \\ 
 

    \theutterance \stepcounter{utterance}  

    & & \multicolumn{4}{p{0.6\linewidth}}{\cellcolor[rgb]{0.9,0.9,0.9}{%
	\makecell[{{p{\linewidth}}}]{% 
	  \tt {\tiny [GM$\rangle$B]}  
	 'W' erfolgreich nach (1,7) verschoben. Dein Raster sieht jetzt so aus:\\ \tt ```\\ \tt  1234567\\ \tt ╔══╤════╗\\ \tt ║◌◌│◌◌◌◌║ 1\\ \tt ╟──┘◌T┌─╢ 2\\ \tt ║◌◌H◌◌│◌║ 3\\ \tt ╟─┐◌┌─┼─╢ 4\\ \tt ║◌├─┤◌│◌║ 5\\ \tt ╟─┤◌├─┤◌║ 6\\ \tt ║W│I│C│◌║ 7\\ \tt ╚═╧═╧═╧═╝\\ \tt ```\\ \tt Ihr seid gerade bei Runde 9 von maximal 20.\\ \tt Ihr habt gemeinsam 8 von maximal 12 Strafpunkten angesammelt.\\ \tt Dein Mitspieler hat dir diese Nachricht geschickt:\\ \tt \\ \tt \\ \tt "Lasst uns alle Objekte in die unterste Zeile (Zeile 7) verschieben. Meine aktuellen Positionen sind W(1,7), I(5,2), H(5,3), C(2,4), T(5,5). Bitte teilt mir deine aktuellen Positionen mit, damit wir uns abstimmen können."\\ \tt \\ \tt \\ \tt Wie lautet dein nächster Befehl? 
	  } 
	   } 
	   } 
	 & \\ 
 

    \theutterance \stepcounter{utterance}  

    & & & \multicolumn{4}{p{0.6\linewidth}}{\cellcolor[rgb]{1,0.85,0.72}{%
	\makecell[{{p{\linewidth}}}]{% 
	  \tt {\tiny [GM$\langle$B]}  
	 VERSCHIEBE: I, (2,7) 
	  } 
	   } 
	   } 
	  \\ 
 

    \theutterance \stepcounter{utterance}  

    & & & \multicolumn{2}{p{0.3\linewidth}}{\cellcolor[rgb]{0.95,0.95,0.95}{%
	\makecell[{{p{\linewidth}}}]{% 
	  \tt {\tiny [GM$|$GM]}  
	 VERSCHIEBE: I, (2,7) 
	  } 
	   } 
	   } 
	 & & \\ 
 

    \theutterance \stepcounter{utterance}  

    & & & \multicolumn{2}{p{0.3\linewidth}}{\cellcolor[rgb]{0.95,0.95,0.95}{%
	\makecell[{{p{\linewidth}}}]{% 
	  \tt {\tiny [GM$|$GM]}  
	 Strafpunkt: (2,7) ist nicht leer, sondern enthält '│'.\\ \tt \\ \tt Ihr habt gemeinsam 9 von maximal 12 Strafpunkten angesammelt.\\ \tt Bitte versuche es noch mal! 
	  } 
	   } 
	   } 
	 & & \\ 
 

    \theutterance \stepcounter{utterance}  

    & & \multicolumn{4}{p{0.6\linewidth}}{\cellcolor[rgb]{0.9,0.9,0.9}{%
	\makecell[{{p{\linewidth}}}]{% 
	  \tt {\tiny [GM$\rangle$B]}  
	 Strafpunkt: (2,7) ist nicht leer, sondern enthält '│'.\\ \tt \\ \tt Ihr habt gemeinsam 9 von maximal 12 Strafpunkten angesammelt.\\ \tt Bitte versuche es noch mal! 
	  } 
	   } 
	   } 
	 & \\ 
 

    \theutterance \stepcounter{utterance}  

    & & & \multicolumn{4}{p{0.6\linewidth}}{\cellcolor[rgb]{1,0.85,0.72}{%
	\makecell[{{p{\linewidth}}}]{% 
	  \tt {\tiny [GM$\langle$B]}  
	 VERSCHIEBE: I, (5,7) 
	  } 
	   } 
	   } 
	  \\ 
 

    \theutterance \stepcounter{utterance}  

    & & & \multicolumn{2}{p{0.3\linewidth}}{\cellcolor[rgb]{0.95,0.95,0.95}{%
	\makecell[{{p{\linewidth}}}]{% 
	  \tt {\tiny [GM$|$GM]}  
	 VERSCHIEBE: I, (5,7) 
	  } 
	   } 
	   } 
	 & & \\ 
 

    \theutterance \stepcounter{utterance}  

    & & & \multicolumn{2}{p{0.3\linewidth}}{\cellcolor[rgb]{0.95,0.95,0.95}{%
	\makecell[{{p{\linewidth}}}]{% 
	  \tt {\tiny [GM$|$GM]}  
	 Strafpunkt: (5,7) ist nicht leer, sondern enthält 'C'.\\ \tt \\ \tt Ihr habt gemeinsam 10 von maximal 12 Strafpunkten angesammelt.\\ \tt Bitte versuche es noch mal! 
	  } 
	   } 
	   } 
	 & & \\ 
 

    \theutterance \stepcounter{utterance}  

    & & \multicolumn{4}{p{0.6\linewidth}}{\cellcolor[rgb]{0.9,0.9,0.9}{%
	\makecell[{{p{\linewidth}}}]{% 
	  \tt {\tiny [GM$\rangle$B]}  
	 Strafpunkt: (5,7) ist nicht leer, sondern enthält 'C'.\\ \tt \\ \tt Ihr habt gemeinsam 10 von maximal 12 Strafpunkten angesammelt.\\ \tt Bitte versuche es noch mal! 
	  } 
	   } 
	   } 
	 & \\ 
 

    \theutterance \stepcounter{utterance}  

    & & & \multicolumn{4}{p{0.6\linewidth}}{\cellcolor[rgb]{1,0.85,0.72}{%
	\makecell[{{p{\linewidth}}}]{% 
	  \tt {\tiny [GM$\langle$B]}  
	 VERSCHIEBE: I, (5,6) 
	  } 
	   } 
	   } 
	  \\ 
 

    \theutterance \stepcounter{utterance}  

    & & & \multicolumn{2}{p{0.3\linewidth}}{\cellcolor[rgb]{0.95,0.95,0.95}{%
	\makecell[{{p{\linewidth}}}]{% 
	  \tt {\tiny [GM$|$GM]}  
	 VERSCHIEBE: I, (5,6) 
	  } 
	   } 
	   } 
	 & & \\ 
 

    \theutterance \stepcounter{utterance}  

    & & & \multicolumn{2}{p{0.3\linewidth}}{\cellcolor[rgb]{0.95,0.95,0.95}{%
	\makecell[{{p{\linewidth}}}]{% 
	  \tt {\tiny [GM$|$GM]}  
	 Strafpunkt: (5,6) ist nicht leer, sondern enthält '─'.\\ \tt \\ \tt Ihr habt gemeinsam 11 von maximal 12 Strafpunkten angesammelt.\\ \tt Bitte versuche es noch mal! 
	  } 
	   } 
	   } 
	 & & \\ 
 

    \theutterance \stepcounter{utterance}  

    & & \multicolumn{4}{p{0.6\linewidth}}{\cellcolor[rgb]{0.9,0.9,0.9}{%
	\makecell[{{p{\linewidth}}}]{% 
	  \tt {\tiny [GM$\rangle$B]}  
	 Strafpunkt: (5,6) ist nicht leer, sondern enthält '─'.\\ \tt \\ \tt Ihr habt gemeinsam 11 von maximal 12 Strafpunkten angesammelt.\\ \tt Bitte versuche es noch mal! 
	  } 
	   } 
	   } 
	 & \\ 
 

    \theutterance \stepcounter{utterance}  

    & & & \multicolumn{4}{p{0.6\linewidth}}{\cellcolor[rgb]{1,0.85,0.72}{%
	\makecell[{{p{\linewidth}}}]{% 
	  \tt {\tiny [GM$\langle$B]}  
	 VERSCHIEBE: I, (6,7) 
	  } 
	   } 
	   } 
	  \\ 
 

    \theutterance \stepcounter{utterance}  

    & & & \multicolumn{2}{p{0.3\linewidth}}{\cellcolor[rgb]{0.95,0.95,0.95}{%
	\makecell[{{p{\linewidth}}}]{% 
	  \tt {\tiny [GM$|$GM]}  
	 VERSCHIEBE: I, (6,7) 
	  } 
	   } 
	   } 
	 & & \\ 
 

    \theutterance \stepcounter{utterance}  

    & & & \multicolumn{2}{p{0.3\linewidth}}{\cellcolor[rgb]{0.95,0.95,0.95}{%
	\makecell[{{p{\linewidth}}}]{% 
	  \tt {\tiny [GM$|$GM]}  
	 Strafpunkt: (6,7) ist nicht leer, sondern enthält '│'.\\ \tt \\ \tt Ihr habt gemeinsam 12 von maximal 12 Strafpunkten angesammelt.\\ \tt Bitte versuche es noch mal! 
	  } 
	   } 
	   } 
	 & & \\ 
 

    \theutterance \stepcounter{utterance}  

    & & \multicolumn{4}{p{0.6\linewidth}}{\cellcolor[rgb]{0.9,0.9,0.9}{%
	\makecell[{{p{\linewidth}}}]{% 
	  \tt {\tiny [GM$\rangle$B]}  
	 Strafpunkt: (6,7) ist nicht leer, sondern enthält '│'.\\ \tt \\ \tt Ihr habt gemeinsam 12 von maximal 12 Strafpunkten angesammelt.\\ \tt Bitte versuche es noch mal! 
	  } 
	   } 
	   } 
	 & \\ 
 

    \theutterance \stepcounter{utterance}  

    & & & \multicolumn{4}{p{0.6\linewidth}}{\cellcolor[rgb]{1,0.85,0.72}{%
	\makecell[{{p{\linewidth}}}]{% 
	  \tt {\tiny [GM$\langle$B]}  
	 VERSCHIEBE: C, (3,5) 
	  } 
	   } 
	   } 
	  \\ 
 

    \theutterance \stepcounter{utterance}  

    & & & \multicolumn{2}{p{0.3\linewidth}}{\cellcolor[rgb]{0.95,0.95,0.95}{%
	\makecell[{{p{\linewidth}}}]{% 
	  \tt {\tiny [GM$|$GM]}  
	 VERSCHIEBE: C, (3,5) 
	  } 
	   } 
	   } 
	 & & \\ 
 

    \theutterance \stepcounter{utterance}  

    & & & \multicolumn{2}{p{0.3\linewidth}}{\cellcolor[rgb]{0.95,0.95,0.95}{%
	\makecell[{{p{\linewidth}}}]{% 
	  \tt {\tiny [GM$|$GM]}  
	 Strafpunkt: (3,5) ist nicht leer, sondern enthält '─'.\\ \tt \\ \tt Ihr habt gemeinsam 13 von maximal 12 Strafpunkten angesammelt.\\ \tt Bitte versuche es noch mal! 
	  } 
	   } 
	   } 
	 & & \\ 
 

    \theutterance \stepcounter{utterance}  

    & & & \multicolumn{2}{p{0.3\linewidth}}{\cellcolor[rgb]{0.95,0.95,0.95}{%
	\makecell[{{p{\linewidth}}}]{% 
	  \tt {\tiny [GM$|$GM]}  
	 Maximum number of penalties exceeded 
	  } 
	   } 
	   } 
	 & & \\ 
 

    \theutterance \stepcounter{utterance}  

    & & & \multicolumn{2}{p{0.3\linewidth}}{\cellcolor[rgb]{0.95,0.95,0.95}{%
	\makecell[{{p{\linewidth}}}]{% 
	  \tt {\tiny [GM$|$GM]}  
	 * success: False\\ \tt * lose: True\\ \tt * aborted: True\\ \tt -------\\ \tt * Shifts: 2.00\\ \tt * Max Shifts: 8.00\\ \tt * Min Shifts: 4.00\\ \tt * End Distance Sum: 14.60\\ \tt * Init Distance Sum: 19.52\\ \tt * Expected Distance Sum: 20.95\\ \tt * Penalties: 13.00\\ \tt * Max Penalties: 12.00\\ \tt * Rounds: 8.00\\ \tt * Max Rounds: 20.00\\ \tt * Object Count: 5.00\\ \tt  
	  } 
	   } 
	   } 
	 & & \\ 
 

    \theutterance \stepcounter{utterance}  

    & & & \multicolumn{2}{p{0.3\linewidth}}{\cellcolor[rgb]{0.95,0.95,0.95}{%
	\makecell[{{p{\linewidth}}}]{% 
	  \tt {\tiny [GM$|$GM]}  
	 * Main Score: 0.00\\ \tt \\ \tt -------\\ \tt * Distance Score: 0.28\\ \tt * Coverage Score: 0.17\\ \tt * Penalty Score: 0.00\\ \tt * Alternative Penalty Score: 0.00\\ \tt * Alternative Main Score: 0.00\\ \tt \\ \tt -------\\ \tt * Shifts: 2.00\\ \tt * Max Shifts: 8.00\\ \tt * Min Shifts: 4.00\\ \tt * End Distance Sum: 14.60\\ \tt * Init Distance Sum: 19.52\\ \tt * Expected Distance Sum: 20.95\\ \tt * Penalties: 13.00\\ \tt * Max Penalties: 12.00\\ \tt * Rounds: 8.00\\ \tt * Max Rounds: 20.00\\ \tt * Object Count: 5.00\\ \tt  
	  } 
	   } 
	   } 
	 & & \\ 
 

\end{supertabular}
}

\end{document}
