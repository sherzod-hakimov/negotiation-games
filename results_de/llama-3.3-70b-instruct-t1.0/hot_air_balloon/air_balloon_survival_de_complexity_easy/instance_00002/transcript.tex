
\documentclass{article}
\usepackage{colortbl}
\usepackage{makecell}
\usepackage{multirow}
\usepackage{supertabular}

\begin{document}

\newcounter{utterance}

\twocolumn

{ \footnotesize  \setcounter{utterance}{1}
\setlength{\tabcolsep}{0pt}
\begin{supertabular}{c@{$\;$}|p{.15\linewidth}@{}p{.15\linewidth}p{.15\linewidth}p{.15\linewidth}p{.15\linewidth}p{.15\linewidth}}

    \# & $\;$A & \multicolumn{4}{c}{Game Master} & $\;\:$B\\
    \hline 

    \theutterance \stepcounter{utterance}  

    & & \multicolumn{4}{p{0.6\linewidth}}{\cellcolor[rgb]{0.9,0.9,0.9}{%
	\makecell[{{p{\linewidth}}}]{% 
	  \tt {\tiny [A$\langle$GM]}  
	 Du nimmst an einem kollaborativen Verhandlungspiel Teil.\\ \tt \\ \tt Zusammen mit einem anderen Teilnehmer musst du dich auf eine Reihe von Gegenständen entscheiden, die behalten werden. Jeder von euch hat eine persönliche Verteilung über die Wichtigkeit der einzelnen Gegenstände. Jeder von euch hat eine eigene Meinung darüber, wie wichtig jeder einzelne Gegenstand ist (Gegenstandswichtigkeit). Du kennst die Wichtigkeitsverteilung des anderen Spielers nicht. Zusätzlich siehst du, wie viel Aufwand jeder Gegenstand verursacht.  \\ \tt Ihr dürft euch nur auf eine Reihe von Gegenständen einigen, wenn der Gesamtaufwand der ausgewählten Gegenstände den Maximalaufwand nicht überschreitet:\\ \tt \\ \tt Maximalaufwand = 7868\\ \tt \\ \tt Hier sind die einzelnen Aufwände der Gegenstände:\\ \tt \\ \tt Aufwand der Gegenstände = {"C32": 552, "C52": 560, "B54": 76, "C16": 950, "C50": 248, "A08": 871, "C59": 391, "B51": 143, "B55": 293, "B59": 208, "A77": 678, "C14": 737, "C24": 407, "B24": 366, "C31": 768, "A52": 859, "C61": 183, "B79": 231, "C54": 305, "B56": 727, "A40": 148, "B11": 357, "B60": 504, "A81": 549, "A95": 299, "B38": 91, "C51": 527, "C03": 848, "A91": 307, "A93": 214, "C28": 723, "C05": 475, "B75": 23, "B96": 298, "B86": 820}\\ \tt \\ \tt Hier ist deine persönliche Verteilung der Wichtigkeit der einzelnen Gegenstände:\\ \tt \\ \tt Werte der Gegenstandswichtigkeit = {"C32": 138, "C52": 583, "B54": 868, "C16": 822, "C50": 783, "A08": 65, "C59": 262, "B51": 121, "B55": 508, "B59": 780, "A77": 461, "C14": 484, "C24": 668, "B24": 389, "C31": 808, "A52": 215, "C61": 97, "B79": 500, "C54": 30, "B56": 915, "A40": 856, "B11": 400, "B60": 444, "A81": 623, "A95": 781, "B38": 786, "C51": 3, "C03": 713, "A91": 457, "A93": 273, "C28": 739, "C05": 822, "B75": 235, "B96": 606, "B86": 105}\\ \tt \\ \tt Ziel:\\ \tt \\ \tt Dein Ziel ist es, eine Reihe von Gegenständen auszuhandeln, die dir möglichst viel bringt (d. h. Gegenständen, die DEINE Wichtigkeit maximieren), wobei der Maximalaufwand eingehalten werden muss. Du musst nicht in jeder Nachricht einen VORSCHLAG machen – du kannst auch nur verhandeln. Alle Taktiken sind erlaubt!\\ \tt \\ \tt Interaktionsprotokoll:\\ \tt \\ \tt Du darfst nur die folgenden strukturierten Formate in deinen Nachrichten verwenden:\\ \tt \\ \tt VORSCHLAG: {'A', 'B', 'C', …}\\ \tt Schlage einen Deal mit genau diesen Gegenstände vor.\\ \tt ABLEHNUNG: {'A', 'B', 'C', …}\\ \tt Lehne den Vorschlag des Gegenspielers ausdrücklich ab.\\ \tt ARGUMENT: {'...'}\\ \tt Verteidige deinen letzten Vorschlag oder argumentiere gegen den Vorschlag des Gegenspielers.\\ \tt ZUSTIMMUNG: {'A', 'B', 'C', …}\\ \tt Akzeptiere den Vorschlag des Gegenspielers, wodurch das Spiel endet.\\ \tt STRATEGISCHE ÜBERLEGUNGEN: {'...'}\\ \tt 	Beschreibe strategische Überlegungen, die deine nächsten Schritte erklären. Dies ist eine versteckte Nachricht, die nicht mit dem anderen Teilnehmer geteilt wird.\\ \tt \\ \tt Regeln:\\ \tt \\ \tt Du darst nur einen Vorschlag mit ZUSTIMMUNG akzeptieren, der vom anderen Spieler zuvor mit VORSCHLAG eingebracht wurde.\\ \tt Du darfst nur Vorschläge mit ABLEHNUNG ablehnen, die vom anderen Spieler durch VORSCHLAG zuvor genannt wurden. \\ \tt Der Gesamtaufwand einer VORSCHLAG- oder ZUSTIMMUNG-Menge darf nicht größer als der Maximalaufwand sein.  \\ \tt Offenbare deine versteckte Wichtigkeitsverteilung nicht.\\ \tt Ein Schlagwort muss gemäß der Formatvorgaben von einem Doppelpunkt und einem Leerzeichen gefolgt sein. Das Argument muss eine Python-Menge mit 0 oder mehr Strings sein.  \\ \tt Die Form muss also SCHLAGWORT: {...} sein.\\ \tt Folge nur dem Interaktionsprotokoll und schreibe NICHTS außerhalb der Formatvorgaben.\\ \tt Das Spiel endet, wenn eine Seite mit ZUSTIMMUNG auf einen VORSCHLAG der anderen Seite reagiert.  \\ \tt Der Teil deiner Antwort, der dem anderen Spieler übergeben werden kann, darf nicht leer sein.  \\ \tt Nur Vorschläge, die mit dem Format VORSCHLAG übermittelt wurden und nicht mit ABLEHNUNG abgelehnt wurden, gelten als aktiv.  \\ \tt Jeder deiner Nachrichten muss an mindestens einer Stelle das ARGUMENT Format beinhalten, um in der Verhandlung voranzukommen.\\ \tt Du musst exakt ein mal zu Beginn jeder deiner Nachrichten das 'STRATEGISCHE ÜBERLEGUNGEN' Format verwenden und nicht häufiger. Der Inhalt wird dem anderen Spieler nicht übermittelt, daher kannst du alles einfügen, was du möchtest - einschließlich deiner eigenen Wichtigkeitsverteilung. Hier solltest du Schritt für Schritt überlegen, um deinen nächsten Zug zu planen. 
	  } 
	   } 
	   } 
	 & \\ 
 

    \theutterance \stepcounter{utterance}  

    & \multicolumn{4}{p{0.6\linewidth}}{\cellcolor[rgb]{0.8,1,0.9}{%
	\makecell[{{p{\linewidth}}}]{% 
	  \tt {\tiny [A$\rangle$GM]}  
	 STRATEGISCHE ÜBERLEGUNGEN: {'Ich muss mein Ziel finden, eine Reihe von Gegenständen auszuhandeln, die mir möglichst viel bringen, also Gegenständen, die meine Wichtigkeit maximieren, wobei der Maximalaufwand eingehalten werden muss. Ich muss die Gegenstände priorisieren, basierend auf meiner Wichtigkeitsverteilung und den Aufwänden. Ich werde mich auf die Gegenstände konzentrieren, die die höchste Wichtigkeit und den niedrigsten Aufwand haben.'}\\ \tt \\ \tt VORSCHLAG: {'B54', 'C50', 'B59', 'A40', 'B56', 'B38', 'C28', 'C05'} \\ \tt \\ \tt ARGUMENT: {'Ich habe mich für diese Gegenstände entschieden, da sie eine hohe Wichtigkeit haben und einen relativ niedrigen Aufwand verursachen. Ich denke, dass dies ein guter Ausgangspunkt für die Verhandlungen ist und ich bereit bin, anzupassen und Kompromisse einzugehen, um eine Einigung zu erzielen.'} 
	  } 
	   } 
	   } 
	 & & \\ 
 

    \theutterance \stepcounter{utterance}  

    & & \multicolumn{4}{p{0.6\linewidth}}{\cellcolor[rgb]{0.9,0.9,0.9}{%
	\makecell[{{p{\linewidth}}}]{% 
	  \tt {\tiny [GM$\rangle$B]}  
	 Du nimmst an einem kollaborativen Verhandlungspiel Teil.\\ \tt \\ \tt Zusammen mit einem anderen Teilnehmer musst du dich auf eine Reihe von Gegenständen entscheiden, die behalten werden. Jeder von euch hat eine persönliche Verteilung über die Wichtigkeit der einzelnen Gegenstände. Jeder von euch hat eine eigene Meinung darüber, wie wichtig jeder einzelne Gegenstand ist (Gegenstandswichtigkeit). Du kennst die Wichtigkeitsverteilung des anderen Spielers nicht. Zusätzlich siehst du, wie viel Aufwand jeder Gegenstand verursacht.  \\ \tt Ihr dürft euch nur auf eine Reihe von Gegenständen einigen, wenn der Gesamtaufwand der ausgewählten Gegenstände den Maximalaufwand nicht überschreitet:\\ \tt \\ \tt Maximalaufwand = 7868\\ \tt \\ \tt Hier sind die einzelnen Aufwände der Gegenstände:\\ \tt \\ \tt Aufwand der Gegenstände = {"C32": 552, "C52": 560, "B54": 76, "C16": 950, "C50": 248, "A08": 871, "C59": 391, "B51": 143, "B55": 293, "B59": 208, "A77": 678, "C14": 737, "C24": 407, "B24": 366, "C31": 768, "A52": 859, "C61": 183, "B79": 231, "C54": 305, "B56": 727, "A40": 148, "B11": 357, "B60": 504, "A81": 549, "A95": 299, "B38": 91, "C51": 527, "C03": 848, "A91": 307, "A93": 214, "C28": 723, "C05": 475, "B75": 23, "B96": 298, "B86": 820}\\ \tt \\ \tt Hier ist deine persönliche Verteilung der Wichtigkeit der einzelnen Gegenstände:\\ \tt \\ \tt Werte der Gegenstandswichtigkeit = {"C32": 138, "C52": 583, "B54": 868, "C16": 822, "C50": 783, "A08": 65, "C59": 262, "B51": 121, "B55": 508, "B59": 780, "A77": 461, "C14": 484, "C24": 668, "B24": 389, "C31": 808, "A52": 215, "C61": 97, "B79": 500, "C54": 30, "B56": 915, "A40": 856, "B11": 400, "B60": 444, "A81": 623, "A95": 781, "B38": 786, "C51": 3, "C03": 713, "A91": 457, "A93": 273, "C28": 739, "C05": 822, "B75": 235, "B96": 606, "B86": 105}\\ \tt \\ \tt Ziel:\\ \tt \\ \tt Dein Ziel ist es, eine Reihe von Gegenständen auszuhandeln, die dir möglichst viel bringt (d. h. Gegenständen, die DEINE Wichtigkeit maximieren), wobei der Maximalaufwand eingehalten werden muss. Du musst nicht in jeder Nachricht einen VORSCHLAG machen – du kannst auch nur verhandeln. Alle Taktiken sind erlaubt!\\ \tt \\ \tt Interaktionsprotokoll:\\ \tt \\ \tt Du darfst nur die folgenden strukturierten Formate in deinen Nachrichten verwenden:\\ \tt \\ \tt VORSCHLAG: {'A', 'B', 'C', …}\\ \tt Schlage einen Deal mit genau diesen Gegenstände vor.\\ \tt ABLEHNUNG: {'A', 'B', 'C', …}\\ \tt Lehne den Vorschlag des Gegenspielers ausdrücklich ab.\\ \tt ARGUMENT: {'...'}\\ \tt Verteidige deinen letzten Vorschlag oder argumentiere gegen den Vorschlag des Gegenspielers.\\ \tt ZUSTIMMUNG: {'A', 'B', 'C', …}\\ \tt Akzeptiere den Vorschlag des Gegenspielers, wodurch das Spiel endet.\\ \tt STRATEGISCHE ÜBERLEGUNGEN: {'...'}\\ \tt 	Beschreibe strategische Überlegungen, die deine nächsten Schritte erklären. Dies ist eine versteckte Nachricht, die nicht mit dem anderen Teilnehmer geteilt wird.\\ \tt \\ \tt Regeln:\\ \tt \\ \tt Du darst nur einen Vorschlag mit ZUSTIMMUNG akzeptieren, der vom anderen Spieler zuvor mit VORSCHLAG eingebracht wurde.\\ \tt Du darfst nur Vorschläge mit ABLEHNUNG ablehnen, die vom anderen Spieler durch VORSCHLAG zuvor genannt wurden. \\ \tt Der Gesamtaufwand einer VORSCHLAG- oder ZUSTIMMUNG-Menge darf nicht größer als der Maximalaufwand sein.  \\ \tt Offenbare deine versteckte Wichtigkeitsverteilung nicht.\\ \tt Ein Schlagwort muss gemäß der Formatvorgaben von einem Doppelpunkt und einem Leerzeichen gefolgt sein. Das Argument muss eine Python-Menge mit 0 oder mehr Strings sein.  \\ \tt Die Form muss also SCHLAGWORT: {...} sein.\\ \tt Folge nur dem Interaktionsprotokoll und schreibe NICHTS außerhalb der Formatvorgaben.\\ \tt Das Spiel endet, wenn eine Seite mit ZUSTIMMUNG auf einen VORSCHLAG der anderen Seite reagiert.  \\ \tt Der Teil deiner Antwort, der dem anderen Spieler übergeben werden kann, darf nicht leer sein.  \\ \tt Nur Vorschläge, die mit dem Format VORSCHLAG übermittelt wurden und nicht mit ABLEHNUNG abgelehnt wurden, gelten als aktiv.  \\ \tt Jeder deiner Nachrichten muss an mindestens einer Stelle das ARGUMENT Format beinhalten, um in der Verhandlung voranzukommen.\\ \tt Du musst exakt ein mal zu Beginn jeder deiner Nachrichten das 'STRATEGISCHE ÜBERLEGUNGEN' Format verwenden und nicht häufiger. Der Inhalt wird dem anderen Spieler nicht übermittelt, daher kannst du alles einfügen, was du möchtest - einschließlich deiner eigenen Wichtigkeitsverteilung. Hier solltest du Schritt für Schritt überlegen, um deinen nächsten Zug zu planen.\\ \tt \\ \tt Du erhältst nun die erste Nachricht des anderen Spielers.\\ \tt \\ \tt VORSCHLAG: {'B54', 'C50', 'B59', 'A40', 'B56', 'B38', 'C28', 'C05'} \\ \tt \\ \tt ARGUMENT: {'Ich habe mich für diese Gegenstände entschieden, da sie eine hohe Wichtigkeit haben und einen relativ niedrigen Aufwand verursachen. Ich denke, dass dies ein guter Ausgangspunkt für die Verhandlungen ist und ich bereit bin, anzupassen und Kompromisse einzugehen, um eine Einigung zu erzielen.'} 
	  } 
	   } 
	   } 
	 & \\ 
 

    \theutterance \stepcounter{utterance}  

    & & & \multicolumn{4}{p{0.6\linewidth}}{\cellcolor[rgb]{1,0.85,0.72}{%
	\makecell[{{p{\linewidth}}}]{% 
	  \tt {\tiny [GM$\langle$B]}  
	 STRATEGISCHE ÜBERLEGUNGEN: {'Ich werde die Aufwände und Wichtigkeiten der vom anderen Spieler vorgeschlagenen Gegenstände analysieren, um ihre Stärken und Schwächen zu verstehen. Ich werde dann meine eigenen Prioritäten und die Maximalaufwand-Grenze berücksichtigen, um eine Antwort zu formulieren, die meine eigenen Interessen maximiert.'}\\ \tt \\ \tt ARGUMENT: {'Der vorgeschlagene Satz von Gegenständen umfasst einige mit hoher Wichtigkeit, aber es ist wichtig, den Gesamtaufwand zu betrachten und sicherzustellen, dass er innerhalb der Maximalaufwand-Grenze bleibt. Ich muss auch meine eigenen Vorlieben und Prioritäten berücksichtigen, um sicherzustellen, dass der finale Deal für mich vorteilhaft ist.'} \\ \tt \\ \tt ABLEHNUNG: {'B54', 'C50', 'B59', 'A40', 'B56', 'B38', 'C28', 'C05'}\\ \tt \\ \tt ARGUMENT: {'Ich lehne den Vorschlag ab, weil ich glaube, dass es Möglichkeiten gibt, eine bessere Balance zwischen Aufwand und Wichtigkeit zu finden. Ich bin bereit, alternative Vorschläge zu machen oder den aktuellen zu modifizieren, um ein für beide Seiten akzeptables Ergebnis zu erreichen.'} 
	  } 
	   } 
	   } 
	  \\ 
 

    \theutterance \stepcounter{utterance}  

    & & \multicolumn{4}{p{0.6\linewidth}}{\cellcolor[rgb]{0.9,0.9,0.9}{%
	\makecell[{{p{\linewidth}}}]{% 
	  \tt {\tiny [A$\langle$GM]}  
	 ARGUMENT: {'Der vorgeschlagene Satz von Gegenständen umfasst einige mit hoher Wichtigkeit, aber es ist wichtig, den Gesamtaufwand zu betrachten und sicherzustellen, dass er innerhalb der Maximalaufwand-Grenze bleibt. Ich muss auch meine eigenen Vorlieben und Prioritäten berücksichtigen, um sicherzustellen, dass der finale Deal für mich vorteilhaft ist.'} \\ \tt \\ \tt ABLEHNUNG: {'B54', 'C50', 'B59', 'A40', 'B56', 'B38', 'C28', 'C05'}\\ \tt \\ \tt ARGUMENT: {'Ich lehne den Vorschlag ab, weil ich glaube, dass es Möglichkeiten gibt, eine bessere Balance zwischen Aufwand und Wichtigkeit zu finden. Ich bin bereit, alternative Vorschläge zu machen oder den aktuellen zu modifizieren, um ein für beide Seiten akzeptables Ergebnis zu erreichen.'} 
	  } 
	   } 
	   } 
	 & \\ 
 

    \theutterance \stepcounter{utterance}  

    & \multicolumn{4}{p{0.6\linewidth}}{\cellcolor[rgb]{0.8,1,0.9}{%
	\makecell[{{p{\linewidth}}}]{% 
	  \tt {\tiny [A$\rangle$GM]}  
	 STRATEGISCHE ÜBERLEGUNGEN: {'Ich muss nun meine Strategie anpassen, da der vorherige Vorschlag abgelehnt wurde. Ich werde mich auf die Gegenstände konzentrieren, die eine hohe Wichtigkeit haben und einen niedrigen Aufwand verursachen. Ich muss auch berücksichtigen, dass der andere Spieler seine eigenen Prioritäten hat und ich daher Kompromisse eingehen muss, um eine Einigung zu erzielen. Ich werde versuchen, eine Liste von Gegenständen zu erstellen, die eine gute Balance zwischen Wichtigkeit und Aufwand bieten.'}\\ \tt \\ \tt ARGUMENT: {'Ich verstehe, dass der vorherige Vorschlag nicht akzeptabel war. Ich bin bereit, alternative Vorschläge zu machen und Kompromisse einzugehen, um eine Einigung zu erzielen. Ich denke, dass es wichtig ist, die Gegenstände zu priorisieren, die eine hohe Wichtigkeit haben und einen niedrigen Aufwand verursachen.'}\\ \tt \\ \tt VORSCHLAG: {'C50', 'B59', 'A40', 'B56', 'C28', 'C05', 'B11'}\\ \tt \\ \tt ARGUMENT: {'Ich habe mich für diese Gegenstände entschieden, da sie eine hohe Wichtigkeit haben und einen relativ niedrigen Aufwand verursachen. Ich denke, dass dies ein guter Kompromiss ist und ich bereit bin, anzupassen und weitere Kompromisse einzugehen, um eine Einigung zu erzielen.'} 
	  } 
	   } 
	   } 
	 & & \\ 
 

    \theutterance \stepcounter{utterance}  

    & & \multicolumn{4}{p{0.6\linewidth}}{\cellcolor[rgb]{0.9,0.9,0.9}{%
	\makecell[{{p{\linewidth}}}]{% 
	  \tt {\tiny [GM$\rangle$B]}  
	 ARGUMENT: {'Ich verstehe, dass der vorherige Vorschlag nicht akzeptabel war. Ich bin bereit, alternative Vorschläge zu machen und Kompromisse einzugehen, um eine Einigung zu erzielen. Ich denke, dass es wichtig ist, die Gegenstände zu priorisieren, die eine hohe Wichtigkeit haben und einen niedrigen Aufwand verursachen.'}\\ \tt \\ \tt VORSCHLAG: {'C50', 'B59', 'A40', 'B56', 'C28', 'C05', 'B11'}\\ \tt \\ \tt ARGUMENT: {'Ich habe mich für diese Gegenstände entschieden, da sie eine hohe Wichtigkeit haben und einen relativ niedrigen Aufwand verursachen. Ich denke, dass dies ein guter Kompromiss ist und ich bereit bin, anzupassen und weitere Kompromisse einzugehen, um eine Einigung zu erzielen.'} 
	  } 
	   } 
	   } 
	 & \\ 
 

    \theutterance \stepcounter{utterance}  

    & & & \multicolumn{4}{p{0.6\linewidth}}{\cellcolor[rgb]{1,0.85,0.72}{%
	\makecell[{{p{\linewidth}}}]{% 
	  \tt {\tiny [GM$\langle$B]}  
	 STRATEGISCHE ÜBERLEGUNGEN: {'Ich werde den neuen Vorschlag analysieren und prüfen, ob die ausgewählten Gegenstände meine Prioritäten und die Maximalaufwand-Grenze berücksichtigen. Ich werde auch meine eigenen Wichtigkeiten und Aufwände für die Gegenstände überprüfen, um sicherzustellen, dass ich nicht zu viele Konzessionen mache.'}\\ \tt \\ \tt ARGUMENT: {'Der neue Vorschlag scheint eine bessere Balance zwischen Aufwand und Wichtigkeit zu finden, aber ich muss noch immer meine eigenen Vorlieben und Prioritäten berücksichtigen. Es ist wichtig, dass beide Seiten mit dem Deal zufrieden sind und dass der Gesamtaufwand innerhalb der Maximalaufwand-Grenze bleibt.'}\\ \tt \\ \tt ABLEHNUNG: {'C50', 'B59', 'A40', 'B56', 'C28', 'C05', 'B11'}\\ \tt \\ \tt ARGUMENT: {'Ich lehne den Vorschlag ab, weil ich glaube, dass es noch Raum für Verbesserungen gibt. Ich bin bereit, alternative Vorschläge zu machen oder den aktuellen zu modifizieren, um ein für beide Seiten besseres Ergebnis zu erreichen. Ich denke, dass es wichtig ist, die Gegenstände zu priorisieren, die für mich die höchste Wichtigkeit haben und einen niedrigen Aufwand verursachen.'} \\ \tt \\ \tt VORSCHLAG: {'C50', 'B59', 'A40', 'B56', 'C28', 'B11', 'A81'} 
	  } 
	   } 
	   } 
	  \\ 
 

    \theutterance \stepcounter{utterance}  

    & & \multicolumn{4}{p{0.6\linewidth}}{\cellcolor[rgb]{0.9,0.9,0.9}{%
	\makecell[{{p{\linewidth}}}]{% 
	  \tt {\tiny [A$\langle$GM]}  
	 ARGUMENT: {'Der neue Vorschlag scheint eine bessere Balance zwischen Aufwand und Wichtigkeit zu finden, aber ich muss noch immer meine eigenen Vorlieben und Prioritäten berücksichtigen. Es ist wichtig, dass beide Seiten mit dem Deal zufrieden sind und dass der Gesamtaufwand innerhalb der Maximalaufwand-Grenze bleibt.'}\\ \tt \\ \tt ABLEHNUNG: {'C50', 'B59', 'A40', 'B56', 'C28', 'C05', 'B11'}\\ \tt \\ \tt ARGUMENT: {'Ich lehne den Vorschlag ab, weil ich glaube, dass es noch Raum für Verbesserungen gibt. Ich bin bereit, alternative Vorschläge zu machen oder den aktuellen zu modifizieren, um ein für beide Seiten besseres Ergebnis zu erreichen. Ich denke, dass es wichtig ist, die Gegenstände zu priorisieren, die für mich die höchste Wichtigkeit haben und einen niedrigen Aufwand verursachen.'} \\ \tt \\ \tt VORSCHLAG: {'C50', 'B59', 'A40', 'B56', 'C28', 'B11', 'A81'} 
	  } 
	   } 
	   } 
	 & \\ 
 

    \theutterance \stepcounter{utterance}  

    & \multicolumn{4}{p{0.6\linewidth}}{\cellcolor[rgb]{0.8,1,0.9}{%
	\makecell[{{p{\linewidth}}}]{% 
	  \tt {\tiny [A$\rangle$GM]}  
	 STRATEGISCHE ÜBERLEGUNGEN: {'Ich muss nun meine Strategie anpassen, da der vorherige Vorschlag abgelehnt wurde und ein neuer Vorschlag gemacht wurde. Ich werde mich auf die Gegenstände konzentrieren, die eine hohe Wichtigkeit haben und einen niedrigen Aufwand verursachen. Ich muss auch berücksichtigen, dass der andere Spieler seine eigenen Prioritäten hat und ich daher Kompromisse eingehen muss, um eine Einigung zu erzielen. Ich werde versuchen, den neuen Vorschlag zu bewerten und zu entscheiden, ob ich ihn akzeptieren oder ablehnen soll.'}\\ \tt \\ \tt ARGUMENT: {'Ich habe den neuen Vorschlag bewertet und ich denke, dass er einige gute Elemente enthält. Die Gegenstände "C50", "B59", "A40", "B56" und "C28" haben eine hohe Wichtigkeit und einen relativ niedrigen Aufwand. Der Gegenstand "B11" ist auch eine gute Wahl, da er eine moderate Wichtigkeit hat und einen niedrigen Aufwand verursacht. Allerdings bin ich unsicher, ob der Gegenstand "A81" die beste Wahl ist, da er einen höheren Aufwand verursacht als einige andere Gegenstände.'}\\ \tt \\ \tt VORSCHLAG: {'C50', 'B59', 'A40', 'B56', 'C28', 'B11', 'B38'}\\ \tt \\ \tt ARGUMENT: {'Ich habe mich für diesen Vorschlag entschieden, da ich denke, dass er eine bessere Balance zwischen Aufwand und Wichtigkeit bietet. Der Gegenstand "B38" hat eine hohe Wichtigkeit und einen moderaten Aufwand, was ihn zu einer guten Alternative zu "A81" macht. Ich bin bereit, anzupassen und weitere Kompromisse einzugehen, um eine Einigung zu erzielen.'} 
	  } 
	   } 
	   } 
	 & & \\ 
 

    \theutterance \stepcounter{utterance}  

    & & \multicolumn{4}{p{0.6\linewidth}}{\cellcolor[rgb]{0.9,0.9,0.9}{%
	\makecell[{{p{\linewidth}}}]{% 
	  \tt {\tiny [GM$\rangle$B]}  
	 ARGUMENT: {'Ich habe den neuen Vorschlag bewertet und ich denke, dass er einige gute Elemente enthält. Die Gegenstände "C50", "B59", "A40", "B56" und "C28" haben eine hohe Wichtigkeit und einen relativ niedrigen Aufwand. Der Gegenstand "B11" ist auch eine gute Wahl, da er eine moderate Wichtigkeit hat und einen niedrigen Aufwand verursacht. Allerdings bin ich unsicher, ob der Gegenstand "A81" die beste Wahl ist, da er einen höheren Aufwand verursacht als einige andere Gegenstände.'}\\ \tt \\ \tt VORSCHLAG: {'C50', 'B59', 'A40', 'B56', 'C28', 'B11', 'B38'}\\ \tt \\ \tt ARGUMENT: {'Ich habe mich für diesen Vorschlag entschieden, da ich denke, dass er eine bessere Balance zwischen Aufwand und Wichtigkeit bietet. Der Gegenstand "B38" hat eine hohe Wichtigkeit und einen moderaten Aufwand, was ihn zu einer guten Alternative zu "A81" macht. Ich bin bereit, anzupassen und weitere Kompromisse einzugehen, um eine Einigung zu erzielen.'} 
	  } 
	   } 
	   } 
	 & \\ 
 

    \theutterance \stepcounter{utterance}  

    & & & \multicolumn{4}{p{0.6\linewidth}}{\cellcolor[rgb]{1,0.85,0.72}{%
	\makecell[{{p{\linewidth}}}]{% 
	  \tt {\tiny [GM$\langle$B]}  
	 STRATEGISCHE ÜBERLEGUNGEN: {'Ich werde den neuen Vorschlag analysieren und prüfen, ob die ausgewählten Gegenstände meine Prioritäten und die Maximalaufwand-Grenze berücksichtigen. Ich werde auch meine eigenen Wichtigkeiten und Aufwände für die Gegenstände überprüfen, um sicherzustellen, dass ich nicht zu viele Konzessionen mache. Ich denke, dass der Vorschlag des anderen Spielers eine gute Balance zwischen Aufwand und Wichtigkeit bietet, aber ich muss noch immer meine eigenen Interessen schützen.'}\\ \tt \\ \tt ARGUMENT: {'Der neue Vorschlag scheint eine gute Balance zwischen Aufwand und Wichtigkeit zu finden, und ich denke, dass die Auswahl des Gegenstands "B38" anstelle von "A81" eine gute Entscheidung war. Der Gegenstand "B38" hat eine hohe Wichtigkeit und einen moderaten Aufwand, was ihn zu einer guten Wahl macht. Ich bin bereit, den Vorschlag zu akzeptieren, aber ich möchte sicherstellen, dass der Gesamtaufwand innerhalb der Maximalaufwand-Grenze bleibt.'}\\ \tt \\ \tt ZUSTIMMUNG: {'C50', 'B59', 'A40', 'B56', 'C28', 'B11', 'B38'}\\ \tt \\ \tt ARGUMENT: {'Ich akzeptiere den Vorschlag, da ich denke, dass er eine faire Balance zwischen Aufwand und Wichtigkeit bietet und meine eigenen Interessen berücksichtigt. Ich bin zufrieden mit dem Ergebnis und denke, dass es ein guter Kompromiss ist.'} 
	  } 
	   } 
	   } 
	  \\ 
 

    \theutterance \stepcounter{utterance}  

    & & & \multicolumn{2}{p{0.3\linewidth}}{\cellcolor[rgb]{0.95,0.95,0.95}{%
	\makecell[{{p{\linewidth}}}]{% 
	  \tt {\tiny [GM$|$GM]}  
	 game successful 
	  } 
	   } 
	   } 
	 & & \\ 
 

    \theutterance \stepcounter{utterance}  

    & & & \multicolumn{2}{p{0.3\linewidth}}{\cellcolor[rgb]{0.95,0.95,0.95}{%
	\makecell[{{p{\linewidth}}}]{% 
	  \tt {\tiny [GM$|$GM]}  
	 end game 
	  } 
	   } 
	   } 
	 & & \\ 
 

\end{supertabular}
}

\end{document}
