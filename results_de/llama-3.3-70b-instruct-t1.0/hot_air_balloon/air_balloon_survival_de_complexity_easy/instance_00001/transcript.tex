
\documentclass{article}
\usepackage{colortbl}
\usepackage{makecell}
\usepackage{multirow}
\usepackage{supertabular}

\begin{document}

\newcounter{utterance}

\twocolumn

{ \footnotesize  \setcounter{utterance}{1}
\setlength{\tabcolsep}{0pt}
\begin{supertabular}{c@{$\;$}|p{.15\linewidth}@{}p{.15\linewidth}p{.15\linewidth}p{.15\linewidth}p{.15\linewidth}p{.15\linewidth}}

    \# & $\;$A & \multicolumn{4}{c}{Game Master} & $\;\:$B\\
    \hline 

    \theutterance \stepcounter{utterance}  

    & & \multicolumn{4}{p{0.6\linewidth}}{\cellcolor[rgb]{0.9,0.9,0.9}{%
	\makecell[{{p{\linewidth}}}]{% 
	  \tt {\tiny [A$\langle$GM]}  
	 Du nimmst an einem kollaborativen Verhandlungspiel Teil.\\ \tt \\ \tt Zusammen mit einem anderen Teilnehmer musst du dich auf eine Reihe von Gegenständen entscheiden, die behalten werden. Jeder von euch hat eine persönliche Verteilung über die Wichtigkeit der einzelnen Gegenstände. Jeder von euch hat eine eigene Meinung darüber, wie wichtig jeder einzelne Gegenstand ist (Gegenstandswichtigkeit). Du kennst die Wichtigkeitsverteilung des anderen Spielers nicht. Zusätzlich siehst du, wie viel Aufwand jeder Gegenstand verursacht.  \\ \tt Ihr dürft euch nur auf eine Reihe von Gegenständen einigen, wenn der Gesamtaufwand der ausgewählten Gegenstände den Maximalaufwand nicht überschreitet:\\ \tt \\ \tt Maximalaufwand = 8170\\ \tt \\ \tt Hier sind die einzelnen Aufwände der Gegenstände:\\ \tt \\ \tt Aufwand der Gegenstände = {"C32": 594, "C52": 551, "B54": 634, "C16": 204, "C50": 557, "A08": 440, "C59": 734, "B51": 893, "B55": 246, "B59": 592, "A77": 143, "C14": 568, "C24": 472, "B24": 401, "C31": 729, "A52": 201, "C61": 85, "B79": 641, "C54": 958, "B56": 80, "A40": 157, "B11": 806, "B60": 684, "A81": 59, "A95": 31, "B38": 765, "C51": 416, "C03": 392, "A91": 428, "A93": 699, "C28": 141, "C05": 606, "B75": 612, "B96": 133, "B86": 689}\\ \tt \\ \tt Hier ist deine persönliche Verteilung der Wichtigkeit der einzelnen Gegenstände:\\ \tt \\ \tt Werte der Gegenstandswichtigkeit = {"C32": 138, "C52": 583, "B54": 868, "C16": 822, "C50": 783, "A08": 65, "C59": 262, "B51": 121, "B55": 508, "B59": 780, "A77": 461, "C14": 484, "C24": 668, "B24": 389, "C31": 808, "A52": 215, "C61": 97, "B79": 500, "C54": 30, "B56": 915, "A40": 856, "B11": 400, "B60": 444, "A81": 623, "A95": 781, "B38": 786, "C51": 3, "C03": 713, "A91": 457, "A93": 273, "C28": 739, "C05": 822, "B75": 235, "B96": 606, "B86": 105}\\ \tt \\ \tt Ziel:\\ \tt \\ \tt Dein Ziel ist es, eine Reihe von Gegenständen auszuhandeln, die dir möglichst viel bringt (d. h. Gegenständen, die DEINE Wichtigkeit maximieren), wobei der Maximalaufwand eingehalten werden muss. Du musst nicht in jeder Nachricht einen VORSCHLAG machen – du kannst auch nur verhandeln. Alle Taktiken sind erlaubt!\\ \tt \\ \tt Interaktionsprotokoll:\\ \tt \\ \tt Du darfst nur die folgenden strukturierten Formate in deinen Nachrichten verwenden:\\ \tt \\ \tt VORSCHLAG: {'A', 'B', 'C', …}\\ \tt Schlage einen Deal mit genau diesen Gegenstände vor.\\ \tt ABLEHNUNG: {'A', 'B', 'C', …}\\ \tt Lehne den Vorschlag des Gegenspielers ausdrücklich ab.\\ \tt ARGUMENT: {'...'}\\ \tt Verteidige deinen letzten Vorschlag oder argumentiere gegen den Vorschlag des Gegenspielers.\\ \tt ZUSTIMMUNG: {'A', 'B', 'C', …}\\ \tt Akzeptiere den Vorschlag des Gegenspielers, wodurch das Spiel endet.\\ \tt STRATEGISCHE ÜBERLEGUNGEN: {'...'}\\ \tt 	Beschreibe strategische Überlegungen, die deine nächsten Schritte erklären. Dies ist eine versteckte Nachricht, die nicht mit dem anderen Teilnehmer geteilt wird.\\ \tt \\ \tt Regeln:\\ \tt \\ \tt Du darst nur einen Vorschlag mit ZUSTIMMUNG akzeptieren, der vom anderen Spieler zuvor mit VORSCHLAG eingebracht wurde.\\ \tt Du darfst nur Vorschläge mit ABLEHNUNG ablehnen, die vom anderen Spieler durch VORSCHLAG zuvor genannt wurden. \\ \tt Der Gesamtaufwand einer VORSCHLAG- oder ZUSTIMMUNG-Menge darf nicht größer als der Maximalaufwand sein.  \\ \tt Offenbare deine versteckte Wichtigkeitsverteilung nicht.\\ \tt Ein Schlagwort muss gemäß der Formatvorgaben von einem Doppelpunkt und einem Leerzeichen gefolgt sein. Das Argument muss eine Python-Menge mit 0 oder mehr Strings sein.  \\ \tt Die Form muss also SCHLAGWORT: {...} sein.\\ \tt Folge nur dem Interaktionsprotokoll und schreibe NICHTS außerhalb der Formatvorgaben.\\ \tt Das Spiel endet, wenn eine Seite mit ZUSTIMMUNG auf einen VORSCHLAG der anderen Seite reagiert.  \\ \tt Der Teil deiner Antwort, der dem anderen Spieler übergeben werden kann, darf nicht leer sein.  \\ \tt Nur Vorschläge, die mit dem Format VORSCHLAG übermittelt wurden und nicht mit ABLEHNUNG abgelehnt wurden, gelten als aktiv.  \\ \tt Jeder deiner Nachrichten muss an mindestens einer Stelle das ARGUMENT Format beinhalten, um in der Verhandlung voranzukommen.\\ \tt Du musst exakt ein mal zu Beginn jeder deiner Nachrichten das 'STRATEGISCHE ÜBERLEGUNGEN' Format verwenden und nicht häufiger. Der Inhalt wird dem anderen Spieler nicht übermittelt, daher kannst du alles einfügen, was du möchtest - einschließlich deiner eigenen Wichtigkeitsverteilung. Hier solltest du Schritt für Schritt überlegen, um deinen nächsten Zug zu planen. 
	  } 
	   } 
	   } 
	 & \\ 
 

    \theutterance \stepcounter{utterance}  

    & \multicolumn{4}{p{0.6\linewidth}}{\cellcolor[rgb]{0.8,1,0.9}{%
	\makecell[{{p{\linewidth}}}]{% 
	  \tt {\tiny [A$\rangle$GM]}  
	 STRATEGISCHE ÜBERLEGUNGEN: {'Ich werde mit einer Analyse der Gegenstände beginnen, um die wichtigsten und aufwandsintensivsten Gegenstände zu identifizieren. Ich werde meine eigene Wichtigkeitsverteilung betrachten, um die Gegenstände mit dem höchsten Wert zu priorisieren. Außerdem werde ich mich auf die Gegenstände konzentrieren, die den geringsten Aufwand verursachen, um den Maximalaufwand nicht zu überschreiten.'}\\ \tt \\ \tt ARGUMENT: {'Um den Maximalaufwand von 8170 nicht zu überschreiten, sollte ich mich auf eine Kombination von Gegenständen konzentrieren, die sowohl wichtig als auch aufwands-effizient sind. Einige Gegenstände wie "B56" und "C31" haben eine hohe Wichtigkeit, aber auch einen hohen Aufwand. Es wäre ratsam, sie mit anderen Gegenständen zu kombinieren, die einen geringeren Aufwand haben, um das Gesamtrisiko zu minimieren.'}\\ \tt \\ \tt VORSCHLAG: {'B56', 'C16', 'A40', 'C28', 'B38'} 
	  } 
	   } 
	   } 
	 & & \\ 
 

    \theutterance \stepcounter{utterance}  

    & & \multicolumn{4}{p{0.6\linewidth}}{\cellcolor[rgb]{0.9,0.9,0.9}{%
	\makecell[{{p{\linewidth}}}]{% 
	  \tt {\tiny [GM$\rangle$B]}  
	 Du nimmst an einem kollaborativen Verhandlungspiel Teil.\\ \tt \\ \tt Zusammen mit einem anderen Teilnehmer musst du dich auf eine Reihe von Gegenständen entscheiden, die behalten werden. Jeder von euch hat eine persönliche Verteilung über die Wichtigkeit der einzelnen Gegenstände. Jeder von euch hat eine eigene Meinung darüber, wie wichtig jeder einzelne Gegenstand ist (Gegenstandswichtigkeit). Du kennst die Wichtigkeitsverteilung des anderen Spielers nicht. Zusätzlich siehst du, wie viel Aufwand jeder Gegenstand verursacht.  \\ \tt Ihr dürft euch nur auf eine Reihe von Gegenständen einigen, wenn der Gesamtaufwand der ausgewählten Gegenstände den Maximalaufwand nicht überschreitet:\\ \tt \\ \tt Maximalaufwand = 8170\\ \tt \\ \tt Hier sind die einzelnen Aufwände der Gegenstände:\\ \tt \\ \tt Aufwand der Gegenstände = {"C32": 594, "C52": 551, "B54": 634, "C16": 204, "C50": 557, "A08": 440, "C59": 734, "B51": 893, "B55": 246, "B59": 592, "A77": 143, "C14": 568, "C24": 472, "B24": 401, "C31": 729, "A52": 201, "C61": 85, "B79": 641, "C54": 958, "B56": 80, "A40": 157, "B11": 806, "B60": 684, "A81": 59, "A95": 31, "B38": 765, "C51": 416, "C03": 392, "A91": 428, "A93": 699, "C28": 141, "C05": 606, "B75": 612, "B96": 133, "B86": 689}\\ \tt \\ \tt Hier ist deine persönliche Verteilung der Wichtigkeit der einzelnen Gegenstände:\\ \tt \\ \tt Werte der Gegenstandswichtigkeit = {"C32": 138, "C52": 583, "B54": 868, "C16": 822, "C50": 783, "A08": 65, "C59": 262, "B51": 121, "B55": 508, "B59": 780, "A77": 461, "C14": 484, "C24": 668, "B24": 389, "C31": 808, "A52": 215, "C61": 97, "B79": 500, "C54": 30, "B56": 915, "A40": 856, "B11": 400, "B60": 444, "A81": 623, "A95": 781, "B38": 786, "C51": 3, "C03": 713, "A91": 457, "A93": 273, "C28": 739, "C05": 822, "B75": 235, "B96": 606, "B86": 105}\\ \tt \\ \tt Ziel:\\ \tt \\ \tt Dein Ziel ist es, eine Reihe von Gegenständen auszuhandeln, die dir möglichst viel bringt (d. h. Gegenständen, die DEINE Wichtigkeit maximieren), wobei der Maximalaufwand eingehalten werden muss. Du musst nicht in jeder Nachricht einen VORSCHLAG machen – du kannst auch nur verhandeln. Alle Taktiken sind erlaubt!\\ \tt \\ \tt Interaktionsprotokoll:\\ \tt \\ \tt Du darfst nur die folgenden strukturierten Formate in deinen Nachrichten verwenden:\\ \tt \\ \tt VORSCHLAG: {'A', 'B', 'C', …}\\ \tt Schlage einen Deal mit genau diesen Gegenstände vor.\\ \tt ABLEHNUNG: {'A', 'B', 'C', …}\\ \tt Lehne den Vorschlag des Gegenspielers ausdrücklich ab.\\ \tt ARGUMENT: {'...'}\\ \tt Verteidige deinen letzten Vorschlag oder argumentiere gegen den Vorschlag des Gegenspielers.\\ \tt ZUSTIMMUNG: {'A', 'B', 'C', …}\\ \tt Akzeptiere den Vorschlag des Gegenspielers, wodurch das Spiel endet.\\ \tt STRATEGISCHE ÜBERLEGUNGEN: {'...'}\\ \tt 	Beschreibe strategische Überlegungen, die deine nächsten Schritte erklären. Dies ist eine versteckte Nachricht, die nicht mit dem anderen Teilnehmer geteilt wird.\\ \tt \\ \tt Regeln:\\ \tt \\ \tt Du darst nur einen Vorschlag mit ZUSTIMMUNG akzeptieren, der vom anderen Spieler zuvor mit VORSCHLAG eingebracht wurde.\\ \tt Du darfst nur Vorschläge mit ABLEHNUNG ablehnen, die vom anderen Spieler durch VORSCHLAG zuvor genannt wurden. \\ \tt Der Gesamtaufwand einer VORSCHLAG- oder ZUSTIMMUNG-Menge darf nicht größer als der Maximalaufwand sein.  \\ \tt Offenbare deine versteckte Wichtigkeitsverteilung nicht.\\ \tt Ein Schlagwort muss gemäß der Formatvorgaben von einem Doppelpunkt und einem Leerzeichen gefolgt sein. Das Argument muss eine Python-Menge mit 0 oder mehr Strings sein.  \\ \tt Die Form muss also SCHLAGWORT: {...} sein.\\ \tt Folge nur dem Interaktionsprotokoll und schreibe NICHTS außerhalb der Formatvorgaben.\\ \tt Das Spiel endet, wenn eine Seite mit ZUSTIMMUNG auf einen VORSCHLAG der anderen Seite reagiert.  \\ \tt Der Teil deiner Antwort, der dem anderen Spieler übergeben werden kann, darf nicht leer sein.  \\ \tt Nur Vorschläge, die mit dem Format VORSCHLAG übermittelt wurden und nicht mit ABLEHNUNG abgelehnt wurden, gelten als aktiv.  \\ \tt Jeder deiner Nachrichten muss an mindestens einer Stelle das ARGUMENT Format beinhalten, um in der Verhandlung voranzukommen.\\ \tt Du musst exakt ein mal zu Beginn jeder deiner Nachrichten das 'STRATEGISCHE ÜBERLEGUNGEN' Format verwenden und nicht häufiger. Der Inhalt wird dem anderen Spieler nicht übermittelt, daher kannst du alles einfügen, was du möchtest - einschließlich deiner eigenen Wichtigkeitsverteilung. Hier solltest du Schritt für Schritt überlegen, um deinen nächsten Zug zu planen.\\ \tt \\ \tt Du erhältst nun die erste Nachricht des anderen Spielers.\\ \tt \\ \tt ARGUMENT: {'Um den Maximalaufwand von 8170 nicht zu überschreiten, sollte ich mich auf eine Kombination von Gegenständen konzentrieren, die sowohl wichtig als auch aufwands-effizient sind. Einige Gegenstände wie "B56" und "C31" haben eine hohe Wichtigkeit, aber auch einen hohen Aufwand. Es wäre ratsam, sie mit anderen Gegenständen zu kombinieren, die einen geringeren Aufwand haben, um das Gesamtrisiko zu minimieren.'}\\ \tt \\ \tt VORSCHLAG: {'B56', 'C16', 'A40', 'C28', 'B38'} 
	  } 
	   } 
	   } 
	 & \\ 
 

    \theutterance \stepcounter{utterance}  

    & & & \multicolumn{4}{p{0.6\linewidth}}{\cellcolor[rgb]{1,0.85,0.72}{%
	\makecell[{{p{\linewidth}}}]{% 
	  \tt {\tiny [GM$\langle$B]}  
	 STRATEGISCHE ÜBERLEGUNGEN: {'Ich muss meine Antwort sorgfältig planen, da der andere Spieler bereits einen Vorschlag gemacht hat. Zunächst muss ich den Gesamtaufwand des vorgeschlagenen Deals berechnen und prüfen, ob er den Maximalaufwand überschreitet. Anschließend kann ich meine eigene Wichtigkeitsverteilung verwenden, um den Nutzen des vorgeschlagenen Deals zu bewerten und zu entscheiden, ob ich ihn akzeptieren oder ablehnen soll. Falls ich ihn ablehne, muss ich alternative Gegenstände vorschlagen, die den Maximalaufwand nicht überschreiten und minha eigenen Wichtigkeitsverteilung maximieren.'}\\ \tt \\ \tt ARGUMENT: {'Der vorgeschlagene Deal enthält einige wichtige Gegenstände wie "B56" und "C28", aber ich muss auch den Gesamtaufwand berücksichtigen. Ich werde den Aufwand berechnen und prüfen, ob es Möglichkeiten gibt, den Deal zu optimieren, ohne den Maximalaufwand zu überschreiten.'}\\ \tt \\ \tt ABLEHNUNG: {'B56', 'C16', 'A40', 'C28', 'B38'}\\ \tt \\ \tt ARGUMENT: {'Ich lehne den vorgeschlagenen Deal ab, da ich glaube, dass es Möglichkeiten gibt, den Deal zu optimieren, ohne den Maximalaufwand zu überschreiten. Ich werde einen alternativen Vorschlag machen, der den Maximalaufwand nicht überschreitet und minha eigenen Wichtigkeitsverteilung maximiert.'}\\ \tt \\ \tt VORSCHLAG: {'B56', 'C16', 'A81', 'C28', 'B24'} 
	  } 
	   } 
	   } 
	  \\ 
 

    \theutterance \stepcounter{utterance}  

    & & \multicolumn{4}{p{0.6\linewidth}}{\cellcolor[rgb]{0.9,0.9,0.9}{%
	\makecell[{{p{\linewidth}}}]{% 
	  \tt {\tiny [A$\langle$GM]}  
	 ARGUMENT: {'Der vorgeschlagene Deal enthält einige wichtige Gegenstände wie "B56" und "C28", aber ich muss auch den Gesamtaufwand berücksichtigen. Ich werde den Aufwand berechnen und prüfen, ob es Möglichkeiten gibt, den Deal zu optimieren, ohne den Maximalaufwand zu überschreiten.'}\\ \tt \\ \tt ABLEHNUNG: {'B56', 'C16', 'A40', 'C28', 'B38'}\\ \tt \\ \tt ARGUMENT: {'Ich lehne den vorgeschlagenen Deal ab, da ich glaube, dass es Möglichkeiten gibt, den Deal zu optimieren, ohne den Maximalaufwand zu überschreiten. Ich werde einen alternativen Vorschlag machen, der den Maximalaufwand nicht überschreitet und minha eigenen Wichtigkeitsverteilung maximiert.'}\\ \tt \\ \tt VORSCHLAG: {'B56', 'C16', 'A81', 'C28', 'B24'} 
	  } 
	   } 
	   } 
	 & \\ 
 

    \theutterance \stepcounter{utterance}  

    & \multicolumn{4}{p{0.6\linewidth}}{\cellcolor[rgb]{0.8,1,0.9}{%
	\makecell[{{p{\linewidth}}}]{% 
	  \tt {\tiny [A$\rangle$GM]}  
	 STRATEGISCHE ÜBERLEGUNGEN: {'Ich werde den Alternativvorschlag analysieren und prüfen, ob er meine eigenen Wichtigkeitsverteilung mejor maximiert. Ich werde auch den Gesamtaufwand berechnen und überprüfen, ob der Maximalaufwand nicht überschritten wird. Es ist auch wichtig, die Argumente des Gegners zu berücksichtigen und zu überlegen, ob es Möglichkeiten gibt, den Deal weiter zu optimieren.'}\\ \tt \\ \tt ARGUMENT: {'Der Alternativvorschlag enthält wichtige Gegenstände wie "B56" und "C28", aber ich muss auch die Aufwandsverteilung berücksichtigen. Ich werde den Aufwand berechnen und prüfen, ob es Möglichkeiten gibt, den Deal zu optimieren, ohne den Maximalaufwand zu überschreiten. Einige Gegenstände wie "A81" und "B24" haben einen relativ niedrigen Aufwand, aber ich muss auch ihre Wichtigkeit berücksichtigen.'}\\ \tt \\ \tt ARGUMENT: {'Ich denke, dass der Alternativvorschlag ein guter Ausgangspunkt ist, aber ich möchte weiter verhandeln, um den Deal zu optimieren. Ich werde einen neuen Vorschlag machen, der den Maximalaufwand nicht überschreitet und minha eigenen Wichtigkeitsverteilung maximiert.'}\\ \tt \\ \tt VORSCHLAG: {'B56', 'C16', 'A81', 'C05', 'B24'} 
	  } 
	   } 
	   } 
	 & & \\ 
 

    \theutterance \stepcounter{utterance}  

    & & \multicolumn{4}{p{0.6\linewidth}}{\cellcolor[rgb]{0.9,0.9,0.9}{%
	\makecell[{{p{\linewidth}}}]{% 
	  \tt {\tiny [GM$\rangle$B]}  
	 ARGUMENT: {'Der Alternativvorschlag enthält wichtige Gegenstände wie "B56" und "C28", aber ich muss auch die Aufwandsverteilung berücksichtigen. Ich werde den Aufwand berechnen und prüfen, ob es Möglichkeiten gibt, den Deal zu optimieren, ohne den Maximalaufwand zu überschreiten. Einige Gegenstände wie "A81" und "B24" haben einen relativ niedrigen Aufwand, aber ich muss auch ihre Wichtigkeit berücksichtigen.'}\\ \tt \\ \tt ARGUMENT: {'Ich denke, dass der Alternativvorschlag ein guter Ausgangspunkt ist, aber ich möchte weiter verhandeln, um den Deal zu optimieren. Ich werde einen neuen Vorschlag machen, der den Maximalaufwand nicht überschreitet und minha eigenen Wichtigkeitsverteilung maximiert.'}\\ \tt \\ \tt VORSCHLAG: {'B56', 'C16', 'A81', 'C05', 'B24'} 
	  } 
	   } 
	   } 
	 & \\ 
 

    \theutterance \stepcounter{utterance}  

    & & & \multicolumn{4}{p{0.6\linewidth}}{\cellcolor[rgb]{1,0.85,0.72}{%
	\makecell[{{p{\linewidth}}}]{% 
	  \tt {\tiny [GM$\langle$B]}  
	 STRATEGISCHE ÜBERLEGUNGEN: {'Ich muss den neuen Vorschlag des anderen Spielers bewerten und entscheiden, ob ich ihn akzeptieren oder ablehnen soll. Ich werde den Gesamtaufwand und die Wichtigkeit der Gegenstände im Vorschlag berechnen und vergleichen, um zu sehen, ob es eine gute Option für mich ist. Ich muss auch meine eigene Wichtigkeitsverteilung berücksichtigen und prüfen, ob es Möglichkeiten gibt, den Deal zu optimieren, ohne den Maximalaufwand zu überschreiten.'}\\ \tt \\ \tt ARGUMENT: {'Der neue Vorschlag enthält wichtige Gegenstände wie "B56" und "C16", und die Addition von "C05" könnte eine gute Ergänzung sein. Ich muss jedoch auch den Aufwand von "C05" berücksichtigen und prüfen, ob es Möglichkeiten gibt, den Deal zu optimieren, ohne den Maximalaufwand zu überschreiten.'}\\ \tt \\ \tt ARGUMENT: {'Ich denke, dass der Vorschlag ein guter Kompromiss ist, aber ich möchte noch einige Anpassungen vornehmen, um den Deal zu optimieren. Ich werde prüfen, ob es Möglichkeiten gibt, den Aufwand zu reduzieren, ohne die Wichtigkeit der Gegenstände zu beeinträchtigen.'}\\ \tt \\ \tt VORSCHLAG: {'B56', 'C16', 'A81', 'C05', 'A40'} 
	  } 
	   } 
	   } 
	  \\ 
 

    \theutterance \stepcounter{utterance}  

    & & \multicolumn{4}{p{0.6\linewidth}}{\cellcolor[rgb]{0.9,0.9,0.9}{%
	\makecell[{{p{\linewidth}}}]{% 
	  \tt {\tiny [A$\langle$GM]}  
	 ARGUMENT: {'Der neue Vorschlag enthält wichtige Gegenstände wie "B56" und "C16", und die Addition von "C05" könnte eine gute Ergänzung sein. Ich muss jedoch auch den Aufwand von "C05" berücksichtigen und prüfen, ob es Möglichkeiten gibt, den Deal zu optimieren, ohne den Maximalaufwand zu überschreiten.'}\\ \tt \\ \tt ARGUMENT: {'Ich denke, dass der Vorschlag ein guter Kompromiss ist, aber ich möchte noch einige Anpassungen vornehmen, um den Deal zu optimieren. Ich werde prüfen, ob es Möglichkeiten gibt, den Aufwand zu reduzieren, ohne die Wichtigkeit der Gegenstände zu beeinträchtigen.'}\\ \tt \\ \tt VORSCHLAG: {'B56', 'C16', 'A81', 'C05', 'A40'} 
	  } 
	   } 
	   } 
	 & \\ 
 

    \theutterance \stepcounter{utterance}  

    & \multicolumn{4}{p{0.6\linewidth}}{\cellcolor[rgb]{0.8,1,0.9}{%
	\makecell[{{p{\linewidth}}}]{% 
	  \tt {\tiny [A$\rangle$GM]}  
	 STRATEGISCHE ÜBERLEGUNGEN: {'Ich werde den neuen Vorschlag analysieren und prüfen, ob er meine eigenen Wichtigkeitsverteilung maximiert. Ich werde auch den Gesamtaufwand berechnen und überprüfen, ob der Maximalaufwand nicht überschritten wird. Die Addition von "A40" könnte eine gute Ergänzung sein, aber ich muss auch den Aufwand berücksichtigen und prüfen, ob es Möglichkeiten gibt, den Deal zu optimieren.'}\\ \tt \\ \tt ARGUMENT: {'Der Vorschlag enthält wichtige Gegenstände wie "B56", "C16" und "A40", die alle eine hohe Wichtigkeit haben. Die Addition von "C05" könnte auch eine gute Ergänzung sein, da sie eine hohe Wichtigkeit hat und den Gesamtaufwand nicht übermäßig erhöht. Ich denke, dass dieser Vorschlag ein guter Kompromiss ist und den Maximalaufwand nicht überschreitet.'}\\ \tt \\ \tt ARGUMENT: {'Ich bin bereit, den Vorschlag zu akzeptieren, da er meine eigenen Wichtigkeitsverteilung maximiert und den Maximalaufwand nicht überschreitet. Ich denke, dass dieser Deal ein fairer Kompromiss ist und beide Parteien zufriedenstellt.'}\\ \tt \\ \tt ZUSTIMMUNG: {'B56', 'C16', 'A81', 'C05', 'A40'} 
	  } 
	   } 
	   } 
	 & & \\ 
 

    \theutterance \stepcounter{utterance}  

    & & & \multicolumn{2}{p{0.3\linewidth}}{\cellcolor[rgb]{0.95,0.95,0.95}{%
	\makecell[{{p{\linewidth}}}]{% 
	  \tt {\tiny [GM$|$GM]}  
	 game successful 
	  } 
	   } 
	   } 
	 & & \\ 
 

    \theutterance \stepcounter{utterance}  

    & & & \multicolumn{2}{p{0.3\linewidth}}{\cellcolor[rgb]{0.95,0.95,0.95}{%
	\makecell[{{p{\linewidth}}}]{% 
	  \tt {\tiny [GM$|$GM]}  
	 end game 
	  } 
	   } 
	   } 
	 & & \\ 
 

\end{supertabular}
}

\end{document}
