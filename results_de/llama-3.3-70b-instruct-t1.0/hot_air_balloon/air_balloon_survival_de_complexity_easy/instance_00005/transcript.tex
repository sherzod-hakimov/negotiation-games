
\documentclass{article}
\usepackage{colortbl}
\usepackage{makecell}
\usepackage{multirow}
\usepackage{supertabular}

\begin{document}

\newcounter{utterance}

\centering \large Interaction Transcript for game `hot\_air\_balloon', experiment `air\_balloon\_survival\_de\_complexity\_easy', episode 5 with llama{-}3.3{-}70b{-}instruct{-}t1.0.
\vspace{24pt}

{ \footnotesize  \setcounter{utterance}{1}
\setlength{\tabcolsep}{0pt}
\begin{supertabular}{c@{$\;$}|p{.15\linewidth}@{}p{.15\linewidth}p{.15\linewidth}p{.15\linewidth}p{.15\linewidth}p{.15\linewidth}}
    \# & $\;$A & \multicolumn{4}{c}{Game Master} & $\;\:$B\\
    \hline

    \theutterance \stepcounter{utterance}  
    & & \multicolumn{4}{p{0.6\linewidth}}{
        \cellcolor[rgb]{0.9,0.9,0.9}{
            \makecell[{{p{\linewidth}}}]{
                \texttt{\tiny{[P1$\langle$GM]}}
                \texttt{Du nimmst an einem kollaborativen Verhandlungspiel Teil.} \\
\\ 
\texttt{Zusammen mit einem anderen Teilnehmer musst du dich auf eine Reihe von Gegenständen entscheiden, die behalten werden. Jeder von euch hat eine persönliche Verteilung über die Wichtigkeit der einzelnen Gegenstände. Jeder von euch hat eine eigene Meinung darüber, wie wichtig jeder einzelne Gegenstand ist (Gegenstandswichtigkeit). Du kennst die Wichtigkeitsverteilung des anderen Spielers nicht. Zusätzlich siehst du, wie viel Aufwand jeder Gegenstand verursacht.  } \\
\texttt{Ihr dürft euch nur auf eine Reihe von Gegenständen einigen, wenn der Gesamtaufwand der ausgewählten Gegenstände den Maximalaufwand nicht überschreitet:} \\
\\ 
\texttt{Maximalaufwand = 9475} \\
\\ 
\texttt{Hier sind die einzelnen Aufwände der Gegenstände:} \\
\\ 
\texttt{Aufwand der Gegenstände = \{"C32": 944, "C52": 568, "B54": 515, "C16": 239, "C50": 876, "A08": 421, "C59": 991, "B51": 281, "B55": 788, "B59": 678, "A77": 432, "C14": 409, "C24": 279, "B24": 506, "C31": 101, "A52": 686, "C61": 851, "B79": 854, "C54": 133, "B56": 192, "A40": 573, "B11": 17, "B60": 465, "A81": 772, "A95": 46, "B38": 501, "C51": 220, "C03": 404, "A91": 843, "A93": 748, "C28": 552, "C05": 843, "B75": 947, "B96": 345, "B86": 930\}} \\
\\ 
\texttt{Hier ist deine persönliche Verteilung der Wichtigkeit der einzelnen Gegenstände:} \\
\\ 
\texttt{Werte der Gegenstandswichtigkeit = \{"C32": 138, "C52": 583, "B54": 868, "C16": 822, "C50": 783, "A08": 65, "C59": 262, "B51": 121, "B55": 508, "B59": 780, "A77": 461, "C14": 484, "C24": 668, "B24": 389, "C31": 808, "A52": 215, "C61": 97, "B79": 500, "C54": 30, "B56": 915, "A40": 856, "B11": 400, "B60": 444, "A81": 623, "A95": 781, "B38": 786, "C51": 3, "C03": 713, "A91": 457, "A93": 273, "C28": 739, "C05": 822, "B75": 235, "B96": 606, "B86": 968\}} \\
\\ 
\texttt{Ziel:} \\
\\ 
\texttt{Dein Ziel ist es, eine Reihe von Gegenständen auszuhandeln, die dir möglichst viel bringt (d. h. Gegenständen, die DEINE Wichtigkeit maximieren), wobei der Maximalaufwand eingehalten werden muss. Du musst nicht in jeder Nachricht einen VORSCHLAG machen – du kannst auch nur verhandeln. Alle Taktiken sind erlaubt!} \\
\\ 
\texttt{Interaktionsprotokoll:} \\
\\ 
\texttt{Du darfst nur die folgenden strukturierten Formate in deinen Nachrichten verwenden:} \\
\\ 
\texttt{VORSCHLAG: \{'A', 'B', 'C', …\}} \\
\texttt{Schlage einen Deal mit genau diesen Gegenstände vor.} \\
\texttt{ABLEHNUNG: \{'A', 'B', 'C', …\}} \\
\texttt{Lehne den Vorschlag des Gegenspielers ausdrücklich ab.} \\
\texttt{ARGUMENT: \{'...'\}} \\
\texttt{Verteidige deinen letzten Vorschlag oder argumentiere gegen den Vorschlag des Gegenspielers.} \\
\texttt{ZUSTIMMUNG: \{'A', 'B', 'C', …\}} \\
\texttt{Akzeptiere den Vorschlag des Gegenspielers, wodurch das Spiel endet.} \\
\texttt{STRATEGISCHE ÜBERLEGUNGEN: \{'...'\}} \\
\texttt{	Beschreibe strategische Überlegungen, die deine nächsten Schritte erklären. Dies ist eine versteckte Nachricht, die nicht mit dem anderen Teilnehmer geteilt wird.} \\
\\ 
\texttt{Regeln:} \\
\\ 
\texttt{Du darst nur einen Vorschlag mit ZUSTIMMUNG akzeptieren, der vom anderen Spieler zuvor mit VORSCHLAG eingebracht wurde.} \\
\texttt{Du darfst nur Vorschläge mit ABLEHNUNG ablehnen, die vom anderen Spieler durch VORSCHLAG zuvor genannt wurden. } \\
\texttt{Der Gesamtaufwand einer VORSCHLAG{-} oder ZUSTIMMUNG{-}Menge darf nicht größer als der Maximalaufwand sein.  } \\
\texttt{Offenbare deine versteckte Wichtigkeitsverteilung nicht.} \\
\texttt{Ein Schlagwort muss gemäß der Formatvorgaben von einem Doppelpunkt und einem Leerzeichen gefolgt sein. Das Argument muss eine Python{-}Menge mit 0 oder mehr Strings sein.  } \\
\texttt{Die Form muss also SCHLAGWORT: \{...\} sein.} \\
\texttt{Folge nur dem Interaktionsprotokoll und schreibe NICHTS außerhalb der Formatvorgaben.} \\
\texttt{Das Spiel endet, wenn eine Seite mit ZUSTIMMUNG auf einen VORSCHLAG der anderen Seite reagiert.  } \\
\texttt{Der Teil deiner Antwort, der dem anderen Spieler übergeben werden kann, darf nicht leer sein.  } \\
\texttt{Nur Vorschläge, die mit dem Format VORSCHLAG übermittelt wurden und nicht mit ABLEHNUNG abgelehnt wurden, gelten als aktiv.  } \\
\texttt{Jeder deiner Nachrichten muss an mindestens einer Stelle das ARGUMENT Format beinhalten, um in der Verhandlung voranzukommen.} \\
\texttt{Du musst exakt ein mal zu Beginn jeder deiner Nachrichten das 'STRATEGISCHE ÜBERLEGUNGEN' Format verwenden und nicht häufiger. Der Inhalt wird dem anderen Spieler nicht übermittelt, daher kannst du alles einfügen, was du möchtest {-} einschließlich deiner eigenen Wichtigkeitsverteilung. Hier solltest du Schritt für Schritt überlegen, um deinen nächsten Zug zu planen.} \\
            }
        }
    }
    & \\ \\

    \theutterance \stepcounter{utterance}  
    & \multicolumn{4}{p{0.6\linewidth}}{
        \cellcolor[rgb]{0.9,0.9,0.9}{
            \makecell[{{p{\linewidth}}}]{
                \texttt{\tiny{[P1$\rangle$GM]}}
                \texttt{STRATEGISCHE ÜBERLEGUNGEN: \{"Ich werde zunächst versuchen, eine Menge von Gegenständen zu finden, die eine hohe Wichtigkeit für mich haben und gleichzeitig einen relativ kleinen Aufwand verursachen. Ich werde dann einen Vorschlag machen, der diese Gegenstände enthält, um zu sehen, wie mein Gegner reagiert. Ich werde auch darauf achten, dass der Gesamtaufwand der vorgeschlagenen Gegenstände den Maximalaufwand nicht überschreitet."\}} \\
\\ 
\texttt{ARGUMENT: \{"Ich denke, es ist sinnvoll, mit einem Vorschlag zu beginnen, der eine Mischung aus Gegenständen mit hoher Wichtigkeit und geringem Aufwand enthält. Ich werde mich auf die Gegenstände konzentrieren, die meine Wichtigkeitsverteilung widerspiegeln, aber auch darauf achten, dass der Vorschlag für meinen Gegner attraktiv ist."\}} \\
\\ 
\texttt{VORSCHLAG: \{"C16", "B54", "C31", "B56", "A40"\}} \\
            }
        }
    }
    & & \\ \\

    \theutterance \stepcounter{utterance}  
    & & \multicolumn{4}{p{0.6\linewidth}}{
        \cellcolor[rgb]{0.9,0.9,0.9}{
            \makecell[{{p{\linewidth}}}]{
                \texttt{\tiny{[GM$\rangle$P2]}}
                \texttt{Du nimmst an einem kollaborativen Verhandlungspiel Teil.} \\
\\ 
\texttt{Zusammen mit einem anderen Teilnehmer musst du dich auf eine Reihe von Gegenständen entscheiden, die behalten werden. Jeder von euch hat eine persönliche Verteilung über die Wichtigkeit der einzelnen Gegenstände. Jeder von euch hat eine eigene Meinung darüber, wie wichtig jeder einzelne Gegenstand ist (Gegenstandswichtigkeit). Du kennst die Wichtigkeitsverteilung des anderen Spielers nicht. Zusätzlich siehst du, wie viel Aufwand jeder Gegenstand verursacht.  } \\
\texttt{Ihr dürft euch nur auf eine Reihe von Gegenständen einigen, wenn der Gesamtaufwand der ausgewählten Gegenstände den Maximalaufwand nicht überschreitet:} \\
\\ 
\texttt{Maximalaufwand = 9475} \\
\\ 
\texttt{Hier sind die einzelnen Aufwände der Gegenstände:} \\
\\ 
\texttt{Aufwand der Gegenstände = \{"C32": 944, "C52": 568, "B54": 515, "C16": 239, "C50": 876, "A08": 421, "C59": 991, "B51": 281, "B55": 788, "B59": 678, "A77": 432, "C14": 409, "C24": 279, "B24": 506, "C31": 101, "A52": 686, "C61": 851, "B79": 854, "C54": 133, "B56": 192, "A40": 573, "B11": 17, "B60": 465, "A81": 772, "A95": 46, "B38": 501, "C51": 220, "C03": 404, "A91": 843, "A93": 748, "C28": 552, "C05": 843, "B75": 947, "B96": 345, "B86": 930\}} \\
\\ 
\texttt{Hier ist deine persönliche Verteilung der Wichtigkeit der einzelnen Gegenstände:} \\
\\ 
\texttt{Werte der Gegenstandswichtigkeit = \{"C32": 138, "C52": 583, "B54": 868, "C16": 822, "C50": 783, "A08": 65, "C59": 262, "B51": 121, "B55": 508, "B59": 780, "A77": 461, "C14": 484, "C24": 668, "B24": 389, "C31": 808, "A52": 215, "C61": 97, "B79": 500, "C54": 30, "B56": 915, "A40": 856, "B11": 400, "B60": 444, "A81": 623, "A95": 781, "B38": 786, "C51": 3, "C03": 713, "A91": 457, "A93": 273, "C28": 739, "C05": 822, "B75": 235, "B96": 606, "B86": 968\}} \\
\\ 
\texttt{Ziel:} \\
\\ 
\texttt{Dein Ziel ist es, eine Reihe von Gegenständen auszuhandeln, die dir möglichst viel bringt (d. h. Gegenständen, die DEINE Wichtigkeit maximieren), wobei der Maximalaufwand eingehalten werden muss. Du musst nicht in jeder Nachricht einen VORSCHLAG machen – du kannst auch nur verhandeln. Alle Taktiken sind erlaubt!} \\
\\ 
\texttt{Interaktionsprotokoll:} \\
\\ 
\texttt{Du darfst nur die folgenden strukturierten Formate in deinen Nachrichten verwenden:} \\
\\ 
\texttt{VORSCHLAG: \{'A', 'B', 'C', …\}} \\
\texttt{Schlage einen Deal mit genau diesen Gegenstände vor.} \\
\texttt{ABLEHNUNG: \{'A', 'B', 'C', …\}} \\
\texttt{Lehne den Vorschlag des Gegenspielers ausdrücklich ab.} \\
\texttt{ARGUMENT: \{'...'\}} \\
\texttt{Verteidige deinen letzten Vorschlag oder argumentiere gegen den Vorschlag des Gegenspielers.} \\
\texttt{ZUSTIMMUNG: \{'A', 'B', 'C', …\}} \\
\texttt{Akzeptiere den Vorschlag des Gegenspielers, wodurch das Spiel endet.} \\
\texttt{STRATEGISCHE ÜBERLEGUNGEN: \{'...'\}} \\
\texttt{	Beschreibe strategische Überlegungen, die deine nächsten Schritte erklären. Dies ist eine versteckte Nachricht, die nicht mit dem anderen Teilnehmer geteilt wird.} \\
\\ 
\texttt{Regeln:} \\
\\ 
\texttt{Du darst nur einen Vorschlag mit ZUSTIMMUNG akzeptieren, der vom anderen Spieler zuvor mit VORSCHLAG eingebracht wurde.} \\
\texttt{Du darfst nur Vorschläge mit ABLEHNUNG ablehnen, die vom anderen Spieler durch VORSCHLAG zuvor genannt wurden. } \\
\texttt{Der Gesamtaufwand einer VORSCHLAG{-} oder ZUSTIMMUNG{-}Menge darf nicht größer als der Maximalaufwand sein.  } \\
\texttt{Offenbare deine versteckte Wichtigkeitsverteilung nicht.} \\
\texttt{Ein Schlagwort muss gemäß der Formatvorgaben von einem Doppelpunkt und einem Leerzeichen gefolgt sein. Das Argument muss eine Python{-}Menge mit 0 oder mehr Strings sein.  } \\
\texttt{Die Form muss also SCHLAGWORT: \{...\} sein.} \\
\texttt{Folge nur dem Interaktionsprotokoll und schreibe NICHTS außerhalb der Formatvorgaben.} \\
\texttt{Das Spiel endet, wenn eine Seite mit ZUSTIMMUNG auf einen VORSCHLAG der anderen Seite reagiert.  } \\
\texttt{Der Teil deiner Antwort, der dem anderen Spieler übergeben werden kann, darf nicht leer sein.  } \\
\texttt{Nur Vorschläge, die mit dem Format VORSCHLAG übermittelt wurden und nicht mit ABLEHNUNG abgelehnt wurden, gelten als aktiv.  } \\
\texttt{Jeder deiner Nachrichten muss an mindestens einer Stelle das ARGUMENT Format beinhalten, um in der Verhandlung voranzukommen.} \\
\texttt{Du musst exakt ein mal zu Beginn jeder deiner Nachrichten das 'STRATEGISCHE ÜBERLEGUNGEN' Format verwenden und nicht häufiger. Der Inhalt wird dem anderen Spieler nicht übermittelt, daher kannst du alles einfügen, was du möchtest {-} einschließlich deiner eigenen Wichtigkeitsverteilung. Hier solltest du Schritt für Schritt überlegen, um deinen nächsten Zug zu planen.} \\
\\ 
\texttt{Du erhältst nun die erste Nachricht des anderen Spielers.} \\
\\ 
\texttt{ARGUMENT: \{"Ich denke, es ist sinnvoll, mit einem Vorschlag zu beginnen, der eine Mischung aus Gegenständen mit hoher Wichtigkeit und geringem Aufwand enthält. Ich werde mich auf die Gegenstände konzentrieren, die meine Wichtigkeitsverteilung widerspiegeln, aber auch darauf achten, dass der Vorschlag für meinen Gegner attraktiv ist."\}} \\
\\ 
\texttt{VORSCHLAG: \{"C16", "B54", "C31", "B56", "A40"\}} \\
            }
        }
    }
    & \\ \\

    \theutterance \stepcounter{utterance}  
    & & & \multicolumn{4}{p{0.6\linewidth}}{
        \cellcolor[rgb]{0.9,0.9,0.9}{
            \makecell[{{p{\linewidth}}}]{
                \texttt{\tiny{[GM$\langle$P2]}}
                \texttt{STRATEGISCHE ÜBERLEGUNGEN: \{"Ich werde zunächst die Wichtigkeit und den Aufwand der vom Gegner vorgeschlagenen Gegenstände analysieren, um zu verstehen, warum er diese Auswahl getroffen hat. Anschließend werde ich meine eigene Wichtigkeitsverteilung und den Maximalaufwand berücksichtigen, um zu entscheiden, ob ich den Vorschlag akzeptieren oder ablehnen soll. Wenn ich ablehne, muss ich einen konkurrierenden Vorschlag machen, der meine eigenen Interessen besser widerspiegelt, aber auch für den Gegner attraktiv ist."\}} \\
\\ 
\texttt{ARGUMENT: \{"Der Vorschlag enthält Gegenstände mit einer hohen Wichtigkeit für mich, wie 'B54' und 'B56', aber auch Gegenstände, die einen höheren Aufwand haben, wie 'A40'. Ich muss abwägen, ob der Gesamtaufwand innerhalb des Maximalaufwands liegt und ob der Vorschlag insgesamt für mich vorteilhaft ist. Es ist auch wichtig zu beachten, dass ich nicht zu viel über meine eigene Wichtigkeitsverteilung preisgeben möchte, um nicht den Gegner zu beeinflussen."\}} \\
\\ 
\texttt{ABLEHNUNG: \{"C16", "B54", "C31", "B56", "A40"\}} \\
\\ 
\texttt{ARGUMENT: \{"Ich lehne den Vorschlag ab, weil ich denke, dass es möglich ist, eine bessere Kombination von Gegenständen zu finden, die sowohl meine als auch die Interessen des Gegners berücksichtigt. Ich werde nun einen eigenen Vorschlag machen, der eine alternative Auswahl von Gegenständen enthält, die einen besseren Kompromiss zwischen Wichtigkeit und Aufwand bieten."\}} \\
\\ 
\texttt{VORSCHLAG: \{"B54", "B56", "C31", "C14", "B11"\}} \\
            }
        }
    }
     \\ \\

    \theutterance \stepcounter{utterance}  
    & & \multicolumn{4}{p{0.6\linewidth}}{
        \cellcolor[rgb]{0.9,0.9,0.9}{
            \makecell[{{p{\linewidth}}}]{
                \texttt{\tiny{[P1$\langle$GM]}}
                \texttt{ARGUMENT: \{"Der Vorschlag enthält Gegenstände mit einer hohen Wichtigkeit für mich, wie 'B54' und 'B56', aber auch Gegenstände, die einen höheren Aufwand haben, wie 'A40'. Ich muss abwägen, ob der Gesamtaufwand innerhalb des Maximalaufwands liegt und ob der Vorschlag insgesamt für mich vorteilhaft ist. Es ist auch wichtig zu beachten, dass ich nicht zu viel über meine eigene Wichtigkeitsverteilung preisgeben möchte, um nicht den Gegner zu beeinflussen."\}} \\
\\ 
\texttt{ABLEHNUNG: \{"C16", "B54", "C31", "B56", "A40"\}} \\
\\ 
\texttt{ARGUMENT: \{"Ich lehne den Vorschlag ab, weil ich denke, dass es möglich ist, eine bessere Kombination von Gegenständen zu finden, die sowohl meine als auch die Interessen des Gegners berücksichtigt. Ich werde nun einen eigenen Vorschlag machen, der eine alternative Auswahl von Gegenständen enthält, die einen besseren Kompromiss zwischen Wichtigkeit und Aufwand bieten."\}} \\
\\ 
\texttt{VORSCHLAG: \{"B54", "B56", "C31", "C14", "B11"\}} \\
            }
        }
    }
    & \\ \\

    \theutterance \stepcounter{utterance}  
    & \multicolumn{4}{p{0.6\linewidth}}{
        \cellcolor[rgb]{0.9,0.9,0.9}{
            \makecell[{{p{\linewidth}}}]{
                \texttt{\tiny{[P1$\rangle$GM]}}
                \texttt{STRATEGISCHE ÜBERLEGUNGEN: \{"Ich analysiere den Vorschlag meines Gegners und überlege, ob ich ihn akzeptieren oder ablehnen soll. Ich werde die Wichtigkeit und den Aufwand der einzelnen Gegenstände bewerten und prüfen, ob der Gesamtaufwand den Maximalaufwand nicht überschreitet. Ich werde auch versuchen, die Strategie meines Gegners zu verstehen und mich entsprechend anpassen. Ich denke, dass der Vorschlag einige gute Gegenstände enthält, aber ich bin nicht sicher, ob ich bereit bin, alle diese Gegenstände zu akzeptieren."\}} \\
\\ 
\texttt{ARGUMENT: \{"Der Vorschlag enthält einige Gegenstände, die für mich wichtig sind, wie 'B54' und 'B56', aber auch einige, die weniger wichtig sind, wie 'C14' und 'B11'. Ich denke, dass der Vorschlag insgesamt akzeptabel ist, aber ich bin nicht sicher, ob ich nicht noch bessere Optionen haben könnte. Ich werde den Vorschlag ablehnen und einen eigenen Vorschlag machen, der eine alternative Auswahl von Gegenständen enthält."\}} \\
\\ 
\texttt{ABLEHNUNG: \{"B54", "B56", "C31", "C14", "B11"\}} \\
\\ 
\texttt{ARGUMENT: \{"Ich lehne den Vorschlag ab, weil ich denke, dass ich eine bessere Kombination von Gegenständen finden kann, die meine Interessen besser berücksichtigt. Ich werde nun einen eigenen Vorschlag machen, der eine alternative Auswahl von Gegenständen enthält, die einen besseren Kompromiss zwischen Wichtigkeit und Aufwand bieten."\}} \\
\\ 
\texttt{VORSCHLAG: \{"C16", "B54", "C31", "B56", "C03"\}} \\
            }
        }
    }
    & & \\ \\

    \theutterance \stepcounter{utterance}  
    & & \multicolumn{4}{p{0.6\linewidth}}{
        \cellcolor[rgb]{0.9,0.9,0.9}{
            \makecell[{{p{\linewidth}}}]{
                \texttt{\tiny{[GM$\rangle$P2]}}
                \texttt{ARGUMENT: \{"Der Vorschlag enthält einige Gegenstände, die für mich wichtig sind, wie 'B54' und 'B56', aber auch einige, die weniger wichtig sind, wie 'C14' und 'B11'. Ich denke, dass der Vorschlag insgesamt akzeptabel ist, aber ich bin nicht sicher, ob ich nicht noch bessere Optionen haben könnte. Ich werde den Vorschlag ablehnen und einen eigenen Vorschlag machen, der eine alternative Auswahl von Gegenständen enthält."\}} \\
\\ 
\texttt{ABLEHNUNG: \{"B54", "B56", "C31", "C14", "B11"\}} \\
\\ 
\texttt{ARGUMENT: \{"Ich lehne den Vorschlag ab, weil ich denke, dass ich eine bessere Kombination von Gegenständen finden kann, die meine Interessen besser berücksichtigt. Ich werde nun einen eigenen Vorschlag machen, der eine alternative Auswahl von Gegenständen enthält, die einen besseren Kompromiss zwischen Wichtigkeit und Aufwand bieten."\}} \\
\\ 
\texttt{VORSCHLAG: \{"C16", "B54", "C31", "B56", "C03"\}} \\
            }
        }
    }
    & \\ \\

    \theutterance \stepcounter{utterance}  
    & & & \multicolumn{4}{p{0.6\linewidth}}{
        \cellcolor[rgb]{0.9,0.9,0.9}{
            \makecell[{{p{\linewidth}}}]{
                \texttt{\tiny{[GM$\langle$P2]}}
                \texttt{STRATEGISCHE ÜBERLEGUNGEN: \{"Ich analysiere den neuen Vorschlag und vergleiche ihn mit meinem eigenen. Ich sehe, dass der Gegner 'C03' anstelle von 'A40' vorschlägt, was den Gesamtaufwand reduziert. Ich muss prüfen, ob dieser Vorschlag meine Interessen besser berücksichtigt als mein eigener Vorschlag. Ich werde auch meine eigene Wichtigkeitsverteilung und den Maximalaufwand berücksichtigen, um zu entscheiden, ob ich den Vorschlag akzeptieren oder ablehnen soll."\}} \\
\\ 
\texttt{ARGUMENT: \{"Der Vorschlag enthält Gegenstände, die für mich wichtig sind, wie 'B54', 'B56' und 'C31'. Der Austausch von 'A40' gegen 'C03' reduziert den Gesamtaufwand und macht den Vorschlag attraktiver. Ich bin jedoch noch nicht sicher, ob ich den Vorschlag akzeptieren soll, da ich meine eigenen Interessen und den Maximalaufwand berücksichtigen muss."\}} \\
\\ 
\texttt{ARGUMENT: \{"Ich denke, dass der Vorschlag insgesamt akzeptabel ist, aber ich möchte noch einige Anpassungen vornehmen, um meine eigenen Interessen besser zu berücksichtigen. Ich werde den Vorschlag ablehnen und einen eigenen Vorschlag machen, der eine alternative Auswahl von Gegenständen enthält, die meine Interessen besser widerspiegeln."\}} \\
\\ 
\texttt{ABLEHNUNG: \{"C16", "B54", "C31", "B56", "C03"\}} \\
\\ 
\texttt{VORSCHLAG: \{"B54", "B56", "C31", "C03", "A81"\}} \\
            }
        }
    }
     \\ \\

    \theutterance \stepcounter{utterance}  
    & & \multicolumn{4}{p{0.6\linewidth}}{
        \cellcolor[rgb]{0.9,0.9,0.9}{
            \makecell[{{p{\linewidth}}}]{
                \texttt{\tiny{[P1$\langle$GM]}}
                \texttt{ARGUMENT: \{"Der Vorschlag enthält Gegenstände, die für mich wichtig sind, wie 'B54', 'B56' und 'C31'. Der Austausch von 'A40' gegen 'C03' reduziert den Gesamtaufwand und macht den Vorschlag attraktiver. Ich bin jedoch noch nicht sicher, ob ich den Vorschlag akzeptieren soll, da ich meine eigenen Interessen und den Maximalaufwand berücksichtigen muss."\}} \\
\\ 
\texttt{ARGUMENT: \{"Ich denke, dass der Vorschlag insgesamt akzeptabel ist, aber ich möchte noch einige Anpassungen vornehmen, um meine eigenen Interessen besser zu berücksichtigen. Ich werde den Vorschlag ablehnen und einen eigenen Vorschlag machen, der eine alternative Auswahl von Gegenständen enthält, die meine Interessen besser widerspiegeln."\}} \\
\\ 
\texttt{ABLEHNUNG: \{"C16", "B54", "C31", "B56", "C03"\}} \\
\\ 
\texttt{VORSCHLAG: \{"B54", "B56", "C31", "C03", "A81"\}} \\
            }
        }
    }
    & \\ \\

    \theutterance \stepcounter{utterance}  
    & \multicolumn{4}{p{0.6\linewidth}}{
        \cellcolor[rgb]{0.9,0.9,0.9}{
            \makecell[{{p{\linewidth}}}]{
                \texttt{\tiny{[P1$\rangle$GM]}}
                \texttt{STRATEGISCHE ÜBERLEGUNGEN: \{"Ich analysiere den neuen Vorschlag meines Gegners und überlege, ob ich ihn akzeptieren oder ablehnen soll. Ich werde die Wichtigkeit und den Aufwand der einzelnen Gegenstände bewerten und prüfen, ob der Gesamtaufwand den Maximalaufwand nicht überschreitet. Ich denke, dass der Vorschlag einige gute Gegenstände enthält, aber ich bin besorgt über den hohen Aufwand von 'A81'. Ich werde versuchen, eine Alternative zu finden, die den Aufwand reduziert und gleichzeitig meine Interessen berücksichtigt."\}} \\
\\ 
\texttt{ARGUMENT: \{"Der Vorschlag enthält einige Gegenstände, die für mich wichtig sind, wie 'B54', 'B56' und 'C31'. Der Einbau von 'A81' jedoch erhöht den Gesamtaufwand erheblich und macht den Vorschlag weniger attraktiv. Ich denke, dass es möglich ist, eine alternative Auswahl von Gegenständen zu finden, die den Aufwand reduziert und gleichzeitig meine Interessen berücksichtigt."\}} \\
\\ 
\texttt{ARGUMENT: \{"Ich lehne den Vorschlag ab, weil ich denke, dass der hohe Aufwand von 'A81' nicht gerechtfertigt ist. Ich werde nun einen eigenen Vorschlag machen, der eine alternative Auswahl von Gegenständen enthält, die den Aufwand reduziert und gleichzeitig meine Interessen berücksichtigt."\}} \\
\\ 
\texttt{VORSCHLAG: \{"B54", "B56", "C31", "C03", "B38"\}} \\
            }
        }
    }
    & & \\ \\

    \theutterance \stepcounter{utterance}  
    & & \multicolumn{4}{p{0.6\linewidth}}{
        \cellcolor[rgb]{0.9,0.9,0.9}{
            \makecell[{{p{\linewidth}}}]{
                \texttt{\tiny{[GM$\rangle$P2]}}
                \texttt{ARGUMENT: \{"Der Vorschlag enthält einige Gegenstände, die für mich wichtig sind, wie 'B54', 'B56' und 'C31'. Der Einbau von 'A81' jedoch erhöht den Gesamtaufwand erheblich und macht den Vorschlag weniger attraktiv. Ich denke, dass es möglich ist, eine alternative Auswahl von Gegenständen zu finden, die den Aufwand reduziert und gleichzeitig meine Interessen berücksichtigt."\}} \\
\\ 
\texttt{ARGUMENT: \{"Ich lehne den Vorschlag ab, weil ich denke, dass der hohe Aufwand von 'A81' nicht gerechtfertigt ist. Ich werde nun einen eigenen Vorschlag machen, der eine alternative Auswahl von Gegenständen enthält, die den Aufwand reduziert und gleichzeitig meine Interessen berücksichtigt."\}} \\
\\ 
\texttt{VORSCHLAG: \{"B54", "B56", "C31", "C03", "B38"\}} \\
            }
        }
    }
    & \\ \\

    \theutterance \stepcounter{utterance}  
    & & & \multicolumn{4}{p{0.6\linewidth}}{
        \cellcolor[rgb]{0.9,0.9,0.9}{
            \makecell[{{p{\linewidth}}}]{
                \texttt{\tiny{[GM$\langle$P2]}}
                \texttt{STRATEGISCHE ÜBERLEGUNGEN: \{"Ich analysiere den neuen Vorschlag und vergleiche ihn mit meinem eigenen. Ich sehe, dass der Gegner 'B38' anstelle von 'A81' vorschlägt, was den Gesamtaufwand reduziert. Ich muss prüfen, ob dieser Vorschlag meine Interessen besser berücksichtigt als mein eigener Vorschlag. Ich werde auch meine eigene Wichtigkeitsverteilung und den Maximalaufwand berücksichtigen, um zu entscheiden, ob ich den Vorschlag akzeptieren oder ablehnen soll. Ich denke, dass 'B38' eine gute Alternative zu 'A81' ist, da es einen ähnlichen Nutzen bietet, aber mit geringerem Aufwand."\}} \\
\\ 
\texttt{ARGUMENT: \{"Der Vorschlag enthält Gegenstände, die für mich wichtig sind, wie 'B54', 'B56' und 'C31'. Der Einbau von 'B38' anstelle von 'A81' reduziert den Gesamtaufwand und macht den Vorschlag attraktiver. Ich denke, dass dieser Vorschlag eine gute Balance zwischen Wichtigkeit und Aufwand bietet."\}} \\
\\ 
\texttt{ARGUMENT: \{"Ich denke, dass der Vorschlag insgesamt akzeptabel ist und meine Interessen berücksichtigt. Ich werde den Vorschlag akzeptieren, da ich denke, dass es nicht möglich ist, eine bessere Kombination von Gegenständen zu finden, die den Maximalaufwand nicht überschreitet und gleichzeitig meine Interessen berücksichtigt."\}} \\
\\ 
\texttt{ZUSTIMMUNG: \{"B54", "B56", "C31", "C03", "B38"\}} \\
            }
        }
    }
     \\ \\

    \theutterance \stepcounter{utterance}  
    & & & \multicolumn{2}{p{0.3\linewidth}}{
        \cellcolor[rgb]{0.9,0.9,0.9}{
            \makecell[{{p{\linewidth}}}]{
                \texttt{\tiny{[GM$|$GM]}}
                \texttt{game successful} \\
            }
        }
    }
    & & \\ \\

    \theutterance \stepcounter{utterance}  
    & & & \multicolumn{2}{p{0.3\linewidth}}{
        \cellcolor[rgb]{0.9,0.9,0.9}{
            \makecell[{{p{\linewidth}}}]{
                \texttt{\tiny{[GM$|$GM]}}
                \texttt{end game} \\
            }
        }
    }
    & & \\ \\

\end{supertabular}
}

\end{document}
