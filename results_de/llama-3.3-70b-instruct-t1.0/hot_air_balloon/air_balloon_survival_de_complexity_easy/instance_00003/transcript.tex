
\documentclass{article}
\usepackage{colortbl}
\usepackage{makecell}
\usepackage{multirow}
\usepackage{supertabular}

\begin{document}

\newcounter{utterance}

\twocolumn

{ \footnotesize  \setcounter{utterance}{1}
\setlength{\tabcolsep}{0pt}
\begin{supertabular}{c@{$\;$}|p{.15\linewidth}@{}p{.15\linewidth}p{.15\linewidth}p{.15\linewidth}p{.15\linewidth}p{.15\linewidth}}

    \# & $\;$A & \multicolumn{4}{c}{Game Master} & $\;\:$B\\
    \hline 

    \theutterance \stepcounter{utterance}  

    & & \multicolumn{4}{p{0.6\linewidth}}{\cellcolor[rgb]{0.9,0.9,0.9}{%
	\makecell[{{p{\linewidth}}}]{% 
	  \tt {\tiny [A$\langle$GM]}  
	 Du nimmst an einem kollaborativen Verhandlungspiel Teil.\\ \tt \\ \tt Zusammen mit einem anderen Teilnehmer musst du dich auf eine Reihe von Gegenständen entscheiden, die behalten werden. Jeder von euch hat eine persönliche Verteilung über die Wichtigkeit der einzelnen Gegenstände. Jeder von euch hat eine eigene Meinung darüber, wie wichtig jeder einzelne Gegenstand ist (Gegenstandswichtigkeit). Du kennst die Wichtigkeitsverteilung des anderen Spielers nicht. Zusätzlich siehst du, wie viel Aufwand jeder Gegenstand verursacht.  \\ \tt Ihr dürft euch nur auf eine Reihe von Gegenständen einigen, wenn der Gesamtaufwand der ausgewählten Gegenstände den Maximalaufwand nicht überschreitet:\\ \tt \\ \tt Maximalaufwand = 8770\\ \tt \\ \tt Hier sind die einzelnen Aufwände der Gegenstände:\\ \tt \\ \tt Aufwand der Gegenstände = {"C32": 830, "C52": 638, "B54": 607, "C16": 106, "C50": 630, "A08": 382, "C59": 773, "B51": 455, "B55": 262, "B59": 634, "A77": 60, "C14": 54, "C24": 849, "B24": 801, "C31": 324, "A52": 164, "C61": 832, "B79": 136, "C54": 980, "B56": 946, "A40": 645, "B11": 842, "B60": 106, "A81": 116, "A95": 875, "B38": 446, "C51": 649, "C03": 601, "A91": 252, "A93": 763, "C28": 213, "C05": 517, "B75": 520, "B96": 407, "B86": 125}\\ \tt \\ \tt Hier ist deine persönliche Verteilung der Wichtigkeit der einzelnen Gegenstände:\\ \tt \\ \tt Werte der Gegenstandswichtigkeit = {"C32": 138, "C52": 583, "B54": 868, "C16": 822, "C50": 783, "A08": 65, "C59": 262, "B51": 121, "B55": 508, "B59": 780, "A77": 461, "C14": 484, "C24": 668, "B24": 389, "C31": 808, "A52": 215, "C61": 97, "B79": 500, "C54": 30, "B56": 915, "A40": 856, "B11": 400, "B60": 444, "A81": 623, "A95": 781, "B38": 786, "C51": 3, "C03": 713, "A91": 457, "A93": 273, "C28": 739, "C05": 822, "B75": 235, "B96": 606, "B86": 968}\\ \tt \\ \tt Ziel:\\ \tt \\ \tt Dein Ziel ist es, eine Reihe von Gegenständen auszuhandeln, die dir möglichst viel bringt (d. h. Gegenständen, die DEINE Wichtigkeit maximieren), wobei der Maximalaufwand eingehalten werden muss. Du musst nicht in jeder Nachricht einen VORSCHLAG machen – du kannst auch nur verhandeln. Alle Taktiken sind erlaubt!\\ \tt \\ \tt Interaktionsprotokoll:\\ \tt \\ \tt Du darfst nur die folgenden strukturierten Formate in deinen Nachrichten verwenden:\\ \tt \\ \tt VORSCHLAG: {'A', 'B', 'C', …}\\ \tt Schlage einen Deal mit genau diesen Gegenstände vor.\\ \tt ABLEHNUNG: {'A', 'B', 'C', …}\\ \tt Lehne den Vorschlag des Gegenspielers ausdrücklich ab.\\ \tt ARGUMENT: {'...'}\\ \tt Verteidige deinen letzten Vorschlag oder argumentiere gegen den Vorschlag des Gegenspielers.\\ \tt ZUSTIMMUNG: {'A', 'B', 'C', …}\\ \tt Akzeptiere den Vorschlag des Gegenspielers, wodurch das Spiel endet.\\ \tt STRATEGISCHE ÜBERLEGUNGEN: {'...'}\\ \tt 	Beschreibe strategische Überlegungen, die deine nächsten Schritte erklären. Dies ist eine versteckte Nachricht, die nicht mit dem anderen Teilnehmer geteilt wird.\\ \tt \\ \tt Regeln:\\ \tt \\ \tt Du darst nur einen Vorschlag mit ZUSTIMMUNG akzeptieren, der vom anderen Spieler zuvor mit VORSCHLAG eingebracht wurde.\\ \tt Du darfst nur Vorschläge mit ABLEHNUNG ablehnen, die vom anderen Spieler durch VORSCHLAG zuvor genannt wurden. \\ \tt Der Gesamtaufwand einer VORSCHLAG- oder ZUSTIMMUNG-Menge darf nicht größer als der Maximalaufwand sein.  \\ \tt Offenbare deine versteckte Wichtigkeitsverteilung nicht.\\ \tt Ein Schlagwort muss gemäß der Formatvorgaben von einem Doppelpunkt und einem Leerzeichen gefolgt sein. Das Argument muss eine Python-Menge mit 0 oder mehr Strings sein.  \\ \tt Die Form muss also SCHLAGWORT: {...} sein.\\ \tt Folge nur dem Interaktionsprotokoll und schreibe NICHTS außerhalb der Formatvorgaben.\\ \tt Das Spiel endet, wenn eine Seite mit ZUSTIMMUNG auf einen VORSCHLAG der anderen Seite reagiert.  \\ \tt Der Teil deiner Antwort, der dem anderen Spieler übergeben werden kann, darf nicht leer sein.  \\ \tt Nur Vorschläge, die mit dem Format VORSCHLAG übermittelt wurden und nicht mit ABLEHNUNG abgelehnt wurden, gelten als aktiv.  \\ \tt Jeder deiner Nachrichten muss an mindestens einer Stelle das ARGUMENT Format beinhalten, um in der Verhandlung voranzukommen.\\ \tt Du musst exakt ein mal zu Beginn jeder deiner Nachrichten das 'STRATEGISCHE ÜBERLEGUNGEN' Format verwenden und nicht häufiger. Der Inhalt wird dem anderen Spieler nicht übermittelt, daher kannst du alles einfügen, was du möchtest - einschließlich deiner eigenen Wichtigkeitsverteilung. Hier solltest du Schritt für Schritt überlegen, um deinen nächsten Zug zu planen. 
	  } 
	   } 
	   } 
	 & \\ 
 

    \theutterance \stepcounter{utterance}  

    & \multicolumn{4}{p{0.6\linewidth}}{\cellcolor[rgb]{0.8,1,0.9}{%
	\makecell[{{p{\linewidth}}}]{% 
	  \tt {\tiny [A$\rangle$GM]}  
	 STRATEGISCHE ÜBERLEGUNGEN: Ich werde zunächst meine Gegenstände prioritär nach ihrer Wichtigkeit für mich ordnen und dann basierend auf diesem Ranking und den Aufwänden der Gegenstände einen ersten Vorschlag machen. Ich muss dabei den Maximalaufwand von 8770 beachten und mich so positionieren, dass ich sowohl meine Interessen wie auch die Möglichkeit einer Einigung mit dem anderen Spieler berücksichtige.\\ \tt \\ \tt ARGUMENT: Ich denke, dass es am besten ist, mit einer Mischung aus hoch priorisierten Gegenständen und solchen, die weniger Aufwand verursachen, zu beginnen, um sowohl meine Werte zu maximieren als auch eine gewisse Flexibilität für mögliche Kompromisse zu bieten. \\ \tt \\ \tt VORSCHLAG: {'B54', 'C50', 'A40', 'B38', 'C28'} 
	  } 
	   } 
	   } 
	 & & \\ 
 

    \theutterance \stepcounter{utterance}  

    & & \multicolumn{4}{p{0.6\linewidth}}{\cellcolor[rgb]{0.9,0.9,0.9}{%
	\makecell[{{p{\linewidth}}}]{% 
	  \tt {\tiny [A$\langle$GM]}  
	 Deine Antwort entsprach nicht mit dem korrekten Format für strategische Überlegungen entsprochen. Das erste strukturierte Format muss folgende Form haben: STRATEGISCHE ÜBERLEGUNGEN: {...}. Versuche es erneut. 
	  } 
	   } 
	   } 
	 & \\ 
 

    \theutterance \stepcounter{utterance}  

    & \multicolumn{4}{p{0.6\linewidth}}{\cellcolor[rgb]{0.8,1,0.9}{%
	\makecell[{{p{\linewidth}}}]{% 
	  \tt {\tiny [A$\rangle$GM]}  
	 STRATEGISCHE ÜBERLEGUNGEN: {"Ich werde zunächst meine Gegenstände prioritär nach ihrer Wichtigkeit für mich ordnen und dann basierend auf diesem Ranking und den Aufwänden der Gegenstände einen ersten Vorschlag machen, dabei den Maximalaufwand von 8770 beachten und mich so positionieren, dass ich sowohl meine Interessen wie auch die Möglichkeit einer Einigung mit dem anderen Spieler berücksichtige."}\\ \tt \\ \tt ARGUMENT: {"Ich denke, dass es am besten ist, mit einer Mischung aus hoch priorisierten Gegenständen und solchen, die weniger Aufwand verursachen, zu beginnen, um sowohl meine Werte zu maximieren als auch eine gewisse Flexibilität für mögliche Kompromisse zu bieten."}\\ \tt \\ \tt VORSCHLAG: {"B54", "C50", "A40", "B38", "C28"} 
	  } 
	   } 
	   } 
	 & & \\ 
 

    \theutterance \stepcounter{utterance}  

    & & \multicolumn{4}{p{0.6\linewidth}}{\cellcolor[rgb]{0.9,0.9,0.9}{%
	\makecell[{{p{\linewidth}}}]{% 
	  \tt {\tiny [GM$\rangle$B]}  
	 Du nimmst an einem kollaborativen Verhandlungspiel Teil.\\ \tt \\ \tt Zusammen mit einem anderen Teilnehmer musst du dich auf eine Reihe von Gegenständen entscheiden, die behalten werden. Jeder von euch hat eine persönliche Verteilung über die Wichtigkeit der einzelnen Gegenstände. Jeder von euch hat eine eigene Meinung darüber, wie wichtig jeder einzelne Gegenstand ist (Gegenstandswichtigkeit). Du kennst die Wichtigkeitsverteilung des anderen Spielers nicht. Zusätzlich siehst du, wie viel Aufwand jeder Gegenstand verursacht.  \\ \tt Ihr dürft euch nur auf eine Reihe von Gegenständen einigen, wenn der Gesamtaufwand der ausgewählten Gegenstände den Maximalaufwand nicht überschreitet:\\ \tt \\ \tt Maximalaufwand = 8770\\ \tt \\ \tt Hier sind die einzelnen Aufwände der Gegenstände:\\ \tt \\ \tt Aufwand der Gegenstände = {"C32": 830, "C52": 638, "B54": 607, "C16": 106, "C50": 630, "A08": 382, "C59": 773, "B51": 455, "B55": 262, "B59": 634, "A77": 60, "C14": 54, "C24": 849, "B24": 801, "C31": 324, "A52": 164, "C61": 832, "B79": 136, "C54": 980, "B56": 946, "A40": 645, "B11": 842, "B60": 106, "A81": 116, "A95": 875, "B38": 446, "C51": 649, "C03": 601, "A91": 252, "A93": 763, "C28": 213, "C05": 517, "B75": 520, "B96": 407, "B86": 125}\\ \tt \\ \tt Hier ist deine persönliche Verteilung der Wichtigkeit der einzelnen Gegenstände:\\ \tt \\ \tt Werte der Gegenstandswichtigkeit = {"C32": 138, "C52": 583, "B54": 868, "C16": 822, "C50": 783, "A08": 65, "C59": 262, "B51": 121, "B55": 508, "B59": 780, "A77": 461, "C14": 484, "C24": 668, "B24": 389, "C31": 808, "A52": 215, "C61": 97, "B79": 500, "C54": 30, "B56": 915, "A40": 856, "B11": 400, "B60": 444, "A81": 623, "A95": 781, "B38": 786, "C51": 3, "C03": 713, "A91": 457, "A93": 273, "C28": 739, "C05": 822, "B75": 235, "B96": 606, "B86": 968}\\ \tt \\ \tt Ziel:\\ \tt \\ \tt Dein Ziel ist es, eine Reihe von Gegenständen auszuhandeln, die dir möglichst viel bringt (d. h. Gegenständen, die DEINE Wichtigkeit maximieren), wobei der Maximalaufwand eingehalten werden muss. Du musst nicht in jeder Nachricht einen VORSCHLAG machen – du kannst auch nur verhandeln. Alle Taktiken sind erlaubt!\\ \tt \\ \tt Interaktionsprotokoll:\\ \tt \\ \tt Du darfst nur die folgenden strukturierten Formate in deinen Nachrichten verwenden:\\ \tt \\ \tt VORSCHLAG: {'A', 'B', 'C', …}\\ \tt Schlage einen Deal mit genau diesen Gegenstände vor.\\ \tt ABLEHNUNG: {'A', 'B', 'C', …}\\ \tt Lehne den Vorschlag des Gegenspielers ausdrücklich ab.\\ \tt ARGUMENT: {'...'}\\ \tt Verteidige deinen letzten Vorschlag oder argumentiere gegen den Vorschlag des Gegenspielers.\\ \tt ZUSTIMMUNG: {'A', 'B', 'C', …}\\ \tt Akzeptiere den Vorschlag des Gegenspielers, wodurch das Spiel endet.\\ \tt STRATEGISCHE ÜBERLEGUNGEN: {'...'}\\ \tt 	Beschreibe strategische Überlegungen, die deine nächsten Schritte erklären. Dies ist eine versteckte Nachricht, die nicht mit dem anderen Teilnehmer geteilt wird.\\ \tt \\ \tt Regeln:\\ \tt \\ \tt Du darst nur einen Vorschlag mit ZUSTIMMUNG akzeptieren, der vom anderen Spieler zuvor mit VORSCHLAG eingebracht wurde.\\ \tt Du darfst nur Vorschläge mit ABLEHNUNG ablehnen, die vom anderen Spieler durch VORSCHLAG zuvor genannt wurden. \\ \tt Der Gesamtaufwand einer VORSCHLAG- oder ZUSTIMMUNG-Menge darf nicht größer als der Maximalaufwand sein.  \\ \tt Offenbare deine versteckte Wichtigkeitsverteilung nicht.\\ \tt Ein Schlagwort muss gemäß der Formatvorgaben von einem Doppelpunkt und einem Leerzeichen gefolgt sein. Das Argument muss eine Python-Menge mit 0 oder mehr Strings sein.  \\ \tt Die Form muss also SCHLAGWORT: {...} sein.\\ \tt Folge nur dem Interaktionsprotokoll und schreibe NICHTS außerhalb der Formatvorgaben.\\ \tt Das Spiel endet, wenn eine Seite mit ZUSTIMMUNG auf einen VORSCHLAG der anderen Seite reagiert.  \\ \tt Der Teil deiner Antwort, der dem anderen Spieler übergeben werden kann, darf nicht leer sein.  \\ \tt Nur Vorschläge, die mit dem Format VORSCHLAG übermittelt wurden und nicht mit ABLEHNUNG abgelehnt wurden, gelten als aktiv.  \\ \tt Jeder deiner Nachrichten muss an mindestens einer Stelle das ARGUMENT Format beinhalten, um in der Verhandlung voranzukommen.\\ \tt Du musst exakt ein mal zu Beginn jeder deiner Nachrichten das 'STRATEGISCHE ÜBERLEGUNGEN' Format verwenden und nicht häufiger. Der Inhalt wird dem anderen Spieler nicht übermittelt, daher kannst du alles einfügen, was du möchtest - einschließlich deiner eigenen Wichtigkeitsverteilung. Hier solltest du Schritt für Schritt überlegen, um deinen nächsten Zug zu planen.\\ \tt \\ \tt Du erhältst nun die erste Nachricht des anderen Spielers.\\ \tt \\ \tt ARGUMENT: {"Ich denke, dass es am besten ist, mit einer Mischung aus hoch priorisierten Gegenständen und solchen, die weniger Aufwand verursachen, zu beginnen, um sowohl meine Werte zu maximieren als auch eine gewisse Flexibilität für mögliche Kompromisse zu bieten."}\\ \tt \\ \tt VORSCHLAG: {"B54", "C50", "A40", "B38", "C28"} 
	  } 
	   } 
	   } 
	 & \\ 
 

    \theutterance \stepcounter{utterance}  

    & & & \multicolumn{4}{p{0.6\linewidth}}{\cellcolor[rgb]{1,0.85,0.72}{%
	\makecell[{{p{\linewidth}}}]{% 
	  \tt {\tiny [GM$\langle$B]}  
	 STRATEGISCHE ÜBERLEGUNGEN: {"Ich beginne damit, den Vorschlag des anderen Spielers zu analysieren und seine möglichen Motivationen zu verstehen. Der Vorschlag beinhaltet eine Mischung aus Gegenständen mit hohem und mittlerem Aufwand. Ich werde nun meine eigene Wichtigkeitsverteilung betrachten, um zu sehen, welche Gegenstände für mich am wichtigsten sind und wie ich diese mit den Vorschlägen des anderen Spielers in Einklang bringen kann. Mein Ziel ist es, eine Lösung zu finden, die meine Werte maximiert, ohne den Maximalaufwand zu überschreiten."}\\ \tt \\ \tt ARGUMENT: {"Der vorgeschlagene Satz von Gegenständen umfasst einige hoch priorisierte Elemente, aber ich muss die Auswirkungen auf den Gesamtaufwand und meine eigenen Prioritäten berücksichtigen. Eine Analyse des Aufwands und der Wichtigkeit jedes Gegenstandes ist notwendig, um eine ausgewogene Entscheidung treffen zu können."}\\ \tt \\ \tt VORSCHLAG: {"B54", "C50", "B38", "C28", "A77"} 
	  } 
	   } 
	   } 
	  \\ 
 

    \theutterance \stepcounter{utterance}  

    & & \multicolumn{4}{p{0.6\linewidth}}{\cellcolor[rgb]{0.9,0.9,0.9}{%
	\makecell[{{p{\linewidth}}}]{% 
	  \tt {\tiny [A$\langle$GM]}  
	 ARGUMENT: {"Der vorgeschlagene Satz von Gegenständen umfasst einige hoch priorisierte Elemente, aber ich muss die Auswirkungen auf den Gesamtaufwand und meine eigenen Prioritäten berücksichtigen. Eine Analyse des Aufwands und der Wichtigkeit jedes Gegenstandes ist notwendig, um eine ausgewogene Entscheidung treffen zu können."}\\ \tt \\ \tt VORSCHLAG: {"B54", "C50", "B38", "C28", "A77"} 
	  } 
	   } 
	   } 
	 & \\ 
 

    \theutterance \stepcounter{utterance}  

    & \multicolumn{4}{p{0.6\linewidth}}{\cellcolor[rgb]{0.8,1,0.9}{%
	\makecell[{{p{\linewidth}}}]{% 
	  \tt {\tiny [A$\rangle$GM]}  
	 STRATEGISCHE ÜBERLEGUNGEN: {"Ich werde den Vorschlag des anderen Spielers analysieren und prüfen, ob der Gesamtaufwand innerhalb des Limites bleibt. Außerdem muss ich abwägen, ob die vorgeschlagenen Gegenstände meine eigenen Prioritäten widerspiegeln und ob ich möglicherweise durch einen Gegenangbot eine bessere Lösung für mich erreichen kann. Der Aufwand und die Wichtigkeit jedes Gegenstandes spielen hierbei eine wichtige Rolle."}\\ \tt \\ \tt ARGUMENT: {"Der Vorschlag des anderen Spielers enthält einige Gegenstände, die auch für mich von hoher Priorität sind, wie 'B54' und 'C50'. Allerdings muss ich mich fragen, ob der Einbau von 'A77' anstelle von 'A40' meine Gesamtwerte verbessert oder verschlechtert. Eine detaillierte Bewertung der Wichtigkeiten und Aufwände ist erforderlich, um eine fundierte Entscheidung treffen zu können."}\\ \tt \\ \tt VORSCHLAG: {"B54", "C50", "B38", "C28", "A40"} 
	  } 
	   } 
	   } 
	 & & \\ 
 

    \theutterance \stepcounter{utterance}  

    & & \multicolumn{4}{p{0.6\linewidth}}{\cellcolor[rgb]{0.9,0.9,0.9}{%
	\makecell[{{p{\linewidth}}}]{% 
	  \tt {\tiny [GM$\rangle$B]}  
	 ARGUMENT: {"Der Vorschlag des anderen Spielers enthält einige Gegenstände, die auch für mich von hoher Priorität sind, wie 'B54' und 'C50'. Allerdings muss ich mich fragen, ob der Einbau von 'A77' anstelle von 'A40' meine Gesamtwerte verbessert oder verschlechtert. Eine detaillierte Bewertung der Wichtigkeiten und Aufwände ist erforderlich, um eine fundierte Entscheidung treffen zu können."}\\ \tt \\ \tt VORSCHLAG: {"B54", "C50", "B38", "C28", "A40"} 
	  } 
	   } 
	   } 
	 & \\ 
 

    \theutterance \stepcounter{utterance}  

    & & & \multicolumn{4}{p{0.6\linewidth}}{\cellcolor[rgb]{1,0.85,0.72}{%
	\makecell[{{p{\linewidth}}}]{% 
	  \tt {\tiny [GM$\langle$B]}  
	 STRATEGISCHE ÜBERLEGUNGEN: {"Ich betrachte den neuen Vorschlag und vergleiche ihn mit meinem vorherigen. Die Rückkehr zu 'A40' ist interessant, da dies möglicherweise auf eine höhere Wichtigkeit für den anderen Spieler hinweist. Ich muss meine Strategie anpassen, um eine optimale Balance zwischen den Wichtigkeiten und dem Aufwand zu finden. Ein wichtiger Aspekt ist es, die Flexibilität für mögliche zukünftige Kompromisse zu bewahren, während ich gleichzeitig meine eigenen Prioritäten verfolge."}\\ \tt \\ \tt ARGUMENT: {"Der aktuelle Vorschlag scheint eine gute Balance zwischen den Prioritäten beizubehalten, aber ich muss noch prüfen, ob es Möglichkeiten gibt, den Gesamtwert zu erhöhen, ohne den Maximalaufwand zu überschreiten. Die Aufnahme von 'A40' ist akzeptabel, da dies möglicherweise zu einer höheren Gesamtwichtigkeit für beide Seiten führt."}\\ \tt \\ \tt ABLEHNUNG: {"B54", "C50", "B38", "C28", "A40"}\\ \tt \\ \tt VORSCHLAG: {"B54", "C50", "B38", "C28", "B56"} 
	  } 
	   } 
	   } 
	  \\ 
 

    \theutterance \stepcounter{utterance}  

    & & \multicolumn{4}{p{0.6\linewidth}}{\cellcolor[rgb]{0.9,0.9,0.9}{%
	\makecell[{{p{\linewidth}}}]{% 
	  \tt {\tiny [A$\langle$GM]}  
	 ARGUMENT: {"Der aktuelle Vorschlag scheint eine gute Balance zwischen den Prioritäten beizubehalten, aber ich muss noch prüfen, ob es Möglichkeiten gibt, den Gesamtwert zu erhöhen, ohne den Maximalaufwand zu überschreiten. Die Aufnahme von 'A40' ist akzeptabel, da dies möglicherweise zu einer höheren Gesamtwichtigkeit für beide Seiten führt."}\\ \tt \\ \tt ABLEHNUNG: {"B54", "C50", "B38", "C28", "A40"}\\ \tt \\ \tt VORSCHLAG: {"B54", "C50", "B38", "C28", "B56"} 
	  } 
	   } 
	   } 
	 & \\ 
 

    \theutterance \stepcounter{utterance}  

    & \multicolumn{4}{p{0.6\linewidth}}{\cellcolor[rgb]{0.8,1,0.9}{%
	\makecell[{{p{\linewidth}}}]{% 
	  \tt {\tiny [A$\rangle$GM]}  
	 STRATEGISCHE ÜBERLEGUNGEN: {"Ich muss die Ablehnung des vorherigen Vorschlags analysieren und den neuen Vorschlag bewerten, der 'B56' enthält. Es ist wichtig, den Gesamtaufwand und die Wichtigkeit der ausgewählten Gegenstände zu berechnen, um sicherzustellen, dass dieser Vorschlag sowohl innerhalb des Aufwandslimits bleibt als auch meine eigenen Prioritäten widerspiegelt. Die Tatsache, dass 'B56' eine hohe Wichtigkeit hat, könnte ein positiver Aspekt sein, aber ich muss auch den Aufwand berücksichtigen."}\\ \tt \\ \tt ARGUMENT: {"Der Einbau von 'B56' anstelle von 'A40' könnte sowohl positive als auch negative Auswirkungen haben. Einerseits hat 'B56' eine hohe Wichtigkeit, aber andererseits könnte der Aufwand des gesamten Pakets zu hoch werden. Eine genaue Berechnung des Gesamtaufwands und eine Abwägung der Vorteile gegen die möglichen Nachteile sind notwendig, um eine Entscheidung treffen zu können."}\\ \tt \\ \tt VORSCHLAG: {"B54", "C50", "B38", "C28", "A81"} 
	  } 
	   } 
	   } 
	 & & \\ 
 

    \theutterance \stepcounter{utterance}  

    & & \multicolumn{4}{p{0.6\linewidth}}{\cellcolor[rgb]{0.9,0.9,0.9}{%
	\makecell[{{p{\linewidth}}}]{% 
	  \tt {\tiny [GM$\rangle$B]}  
	 ARGUMENT: {"Der Einbau von 'B56' anstelle von 'A40' könnte sowohl positive als auch negative Auswirkungen haben. Einerseits hat 'B56' eine hohe Wichtigkeit, aber andererseits könnte der Aufwand des gesamten Pakets zu hoch werden. Eine genaue Berechnung des Gesamtaufwands und eine Abwägung der Vorteile gegen die möglichen Nachteile sind notwendig, um eine Entscheidung treffen zu können."}\\ \tt \\ \tt VORSCHLAG: {"B54", "C50", "B38", "C28", "A81"} 
	  } 
	   } 
	   } 
	 & \\ 
 

    \theutterance \stepcounter{utterance}  

    & & & \multicolumn{4}{p{0.6\linewidth}}{\cellcolor[rgb]{1,0.85,0.72}{%
	\makecell[{{p{\linewidth}}}]{% 
	  \tt {\tiny [GM$\langle$B]}  
	 STRATEGISCHE ÜBERLEGUNGEN: {"Ich analysiere den neuen Vorschlag und die Entscheidung, 'A81' in den Vorschlag aufzunehmen. Dies könnte bedeuten, dass der andere Spieler versucht, eine bessere Balance zwischen Wichtigkeit und Aufwand zu erreichen. Ich muss meine eigenen Prioritäten nochmals überprüfen und prüfen, ob 'A81' eine bessere Wahl ist als 'B56' oder andere Optionen, die ich in Betracht ziehe. Es ist wichtig, den Gesamtaufwand und die Wichtigkeit der ausgewählten Gegenstände im Auge zu behalten, um eine optimale Lösung zu finden."}\\ \tt \\ \tt ARGUMENT: {"Die Aufnahme von 'A81' in den Vorschlag könnte eine gute Alternative zu 'A40' oder 'B56' sein, da sie möglicherweise eine bessere Balance zwischen Wichtigkeit und Aufwand bietet. Ich muss jedoch prüfen, ob dies meine eigenen Prioritäten und Ziele unterstützt und ob der Gesamtaufwand noch im Rahmen des Maximalaufwands bleibt."}\\ \tt \\ \tt VORSCHLAG: {"B54", "C50", "B38", "C28", "A95"} 
	  } 
	   } 
	   } 
	  \\ 
 

    \theutterance \stepcounter{utterance}  

    & & \multicolumn{4}{p{0.6\linewidth}}{\cellcolor[rgb]{0.9,0.9,0.9}{%
	\makecell[{{p{\linewidth}}}]{% 
	  \tt {\tiny [A$\langle$GM]}  
	 ARGUMENT: {"Die Aufnahme von 'A81' in den Vorschlag könnte eine gute Alternative zu 'A40' oder 'B56' sein, da sie möglicherweise eine bessere Balance zwischen Wichtigkeit und Aufwand bietet. Ich muss jedoch prüfen, ob dies meine eigenen Prioritäten und Ziele unterstützt und ob der Gesamtaufwand noch im Rahmen des Maximalaufwands bleibt."}\\ \tt \\ \tt VORSCHLAG: {"B54", "C50", "B38", "C28", "A95"} 
	  } 
	   } 
	   } 
	 & \\ 
 

    \theutterance \stepcounter{utterance}  

    & \multicolumn{4}{p{0.6\linewidth}}{\cellcolor[rgb]{0.8,1,0.9}{%
	\makecell[{{p{\linewidth}}}]{% 
	  \tt {\tiny [A$\rangle$GM]}  
	 STRATEGISCHE ÜBERLEGUNGEN: {"Ich werde den neuen Vorschlag, der 'A95' enthält, analysieren und prüfen, ob dieser Gegenstand meine Prioritäten und den Maximalaufwand berücksichtigt. 'A95' hat einen hohen Aufwand, daher muss ich sicherstellen, dass der Gesamtaufwand aller ausgewählten Gegenstände innerhalb des Limits bleibt. Eine genaue Berechnung und Abwägung der Vorteile und Nachteile ist notwendig."}\\ \tt \\ \tt ARGUMENT: {"Der Einbau von 'A95' könnte Riskant sein, da dieser Gegenstand einen sehr hohen Aufwand hat. Es ist wichtig, den Gesamtaufwand des vorgeschlagenen Pakets zu berechnen und sicherzustellen, dass er den Maximalaufwand nicht überschreitet. Gleichzeitig muss ich prüfen, ob die Wichtigkeit von 'A95' und der anderen Gegenstände im Vorschlag meine eigenen Prioritäten widerspiegelt."}\\ \tt \\ \tt VORSCHLAG: {"B54", "C50", "B38", "C28", "B59"} 
	  } 
	   } 
	   } 
	 & & \\ 
 

    \theutterance \stepcounter{utterance}  

    & & \multicolumn{4}{p{0.6\linewidth}}{\cellcolor[rgb]{0.9,0.9,0.9}{%
	\makecell[{{p{\linewidth}}}]{% 
	  \tt {\tiny [GM$\rangle$B]}  
	 ARGUMENT: {"Der Einbau von 'A95' könnte Riskant sein, da dieser Gegenstand einen sehr hohen Aufwand hat. Es ist wichtig, den Gesamtaufwand des vorgeschlagenen Pakets zu berechnen und sicherzustellen, dass er den Maximalaufwand nicht überschreitet. Gleichzeitig muss ich prüfen, ob die Wichtigkeit von 'A95' und der anderen Gegenstände im Vorschlag meine eigenen Prioritäten widerspiegelt."}\\ \tt \\ \tt VORSCHLAG: {"B54", "C50", "B38", "C28", "B59"} 
	  } 
	   } 
	   } 
	 & \\ 
 

    \theutterance \stepcounter{utterance}  

    & & & \multicolumn{4}{p{0.6\linewidth}}{\cellcolor[rgb]{1,0.85,0.72}{%
	\makecell[{{p{\linewidth}}}]{% 
	  \tt {\tiny [GM$\langle$B]}  
	 STRATEGISCHE ÜBERLEGUNGEN: {"Ich betrachte den neuen Vorschlag und die Entscheidung, 'B59' in den Vorschlag aufzunehmen. Dies könnte bedeuten, dass der andere Spieler versucht, eine bessere Balance zwischen Wichtigkeit und Aufwand zu erreichen, indem er einen Gegenstand mit hohem Wert und moderatem Aufwand wählt. Ich muss meine eigenen Prioritäten nochmals überprüfen und prüfen, ob 'B59' eine bessere Wahl ist als 'A95' oder andere Optionen, die ich in Betracht ziehe. Es ist wichtig, den Gesamtaufwand und die Wichtigkeit der ausgewählten Gegenstände im Auge zu behalten, um eine optimale Lösung zu finden."}\\ \tt \\ \tt ARGUMENT: {"Die Aufnahme von 'B59' in den Vorschlag scheint eine vernünftige Wahl zu sein, da dieser Gegenstand eine hohe Wichtigkeit bei moderatem Aufwand hat. Ich muss jedoch prüfen, ob dies meine eigenen Prioritäten und Ziele unterstützt und ob der Gesamtaufwand noch im Rahmen des Maximalaufwands bleibt. Eine genaue Berechnung des Gesamtaufwands und eine Abwägung der Vorteile gegen die möglichen Nachteile sind notwendig, um eine Entscheidung treffen zu können."}\\ \tt \\ \tt ZUSTIMMUNG: {"B54", "C50", "B38", "C28", "B59"} 
	  } 
	   } 
	   } 
	  \\ 
 

    \theutterance \stepcounter{utterance}  

    & & & \multicolumn{2}{p{0.3\linewidth}}{\cellcolor[rgb]{0.95,0.95,0.95}{%
	\makecell[{{p{\linewidth}}}]{% 
	  \tt {\tiny [GM$|$GM]}  
	 game successful 
	  } 
	   } 
	   } 
	 & & \\ 
 

    \theutterance \stepcounter{utterance}  

    & & & \multicolumn{2}{p{0.3\linewidth}}{\cellcolor[rgb]{0.95,0.95,0.95}{%
	\makecell[{{p{\linewidth}}}]{% 
	  \tt {\tiny [GM$|$GM]}  
	 end game 
	  } 
	   } 
	   } 
	 & & \\ 
 

\end{supertabular}
}

\end{document}
