
\documentclass{article}
\usepackage{colortbl}
\usepackage{makecell}
\usepackage{multirow}
\usepackage{supertabular}

\begin{document}

\newcounter{utterance}

\twocolumn

{ \footnotesize  \setcounter{utterance}{1}
\setlength{\tabcolsep}{0pt}
\begin{supertabular}{c@{$\;$}|p{.15\linewidth}@{}p{.15\linewidth}p{.15\linewidth}p{.15\linewidth}p{.15\linewidth}p{.15\linewidth}}

    \# & $\;$A & \multicolumn{4}{c}{Game Master} & $\;\:$B\\
    \hline 

    \theutterance \stepcounter{utterance}  

    & & \multicolumn{4}{p{0.6\linewidth}}{\cellcolor[rgb]{0.9,0.9,0.9}{%
	\makecell[{{p{\linewidth}}}]{% 
	  \tt {\tiny [A$\langle$GM]}  
	 Du nimmst an einem kollaborativen Verhandlungspiel Teil.\\ \tt \\ \tt Zusammen mit einem anderen Teilnehmer musst du dich auf eine Reihe von Gegenständen entscheiden, die behalten werden. Jeder von euch hat eine persönliche Verteilung über die Wichtigkeit der einzelnen Gegenstände. Jeder von euch hat eine eigene Meinung darüber, wie wichtig jeder einzelne Gegenstand ist (Gegenstandswichtigkeit). Du kennst die Wichtigkeitsverteilung des anderen Spielers nicht. Zusätzlich siehst du, wie viel Aufwand jeder Gegenstand verursacht.  \\ \tt Ihr dürft euch nur auf eine Reihe von Gegenständen einigen, wenn der Gesamtaufwand der ausgewählten Gegenstände den Maximalaufwand nicht überschreitet:\\ \tt \\ \tt Maximalaufwand = 9612\\ \tt \\ \tt Hier sind die einzelnen Aufwände der Gegenstände:\\ \tt \\ \tt Aufwand der Gegenstände = {"C32": 935, "C52": 928, "B54": 725, "C16": 218, "C50": 839, "A08": 970, "C59": 394, "B51": 677, "B55": 946, "B59": 530, "A77": 138, "C14": 834, "C24": 734, "B24": 593, "C31": 261, "A52": 743, "C61": 4, "B79": 735, "C54": 124, "B56": 829, "A40": 207, "B11": 782, "B60": 577, "A81": 388, "A95": 679, "B38": 494, "C51": 559, "C03": 629, "A91": 237, "A93": 275, "C28": 39, "C05": 654, "B75": 172, "B96": 688, "B86": 687}\\ \tt \\ \tt Hier ist deine persönliche Verteilung der Wichtigkeit der einzelnen Gegenstände:\\ \tt \\ \tt Werte der Gegenstandswichtigkeit = {"C32": 138, "C52": 583, "B54": 868, "C16": 822, "C50": 783, "A08": 65, "C59": 262, "B51": 121, "B55": 508, "B59": 780, "A77": 461, "C14": 484, "C24": 668, "B24": 389, "C31": 808, "A52": 215, "C61": 97, "B79": 500, "C54": 30, "B56": 915, "A40": 856, "B11": 400, "B60": 444, "A81": 623, "A95": 781, "B38": 786, "C51": 3, "C03": 713, "A91": 457, "A93": 273, "C28": 739, "C05": 822, "B75": 235, "B96": 606, "B86": 968}\\ \tt \\ \tt Ziel:\\ \tt \\ \tt Dein Ziel ist es, eine Reihe von Gegenständen auszuhandeln, die dir möglichst viel bringt (d. h. Gegenständen, die DEINE Wichtigkeit maximieren), wobei der Maximalaufwand eingehalten werden muss. Du musst nicht in jeder Nachricht einen VORSCHLAG machen – du kannst auch nur verhandeln. Alle Taktiken sind erlaubt!\\ \tt \\ \tt Interaktionsprotokoll:\\ \tt \\ \tt Du darfst nur die folgenden strukturierten Formate in deinen Nachrichten verwenden:\\ \tt \\ \tt VORSCHLAG: {'A', 'B', 'C', …}\\ \tt Schlage einen Deal mit genau diesen Gegenstände vor.\\ \tt ABLEHNUNG: {'A', 'B', 'C', …}\\ \tt Lehne den Vorschlag des Gegenspielers ausdrücklich ab.\\ \tt ARGUMENT: {'...'}\\ \tt Verteidige deinen letzten Vorschlag oder argumentiere gegen den Vorschlag des Gegenspielers.\\ \tt ZUSTIMMUNG: {'A', 'B', 'C', …}\\ \tt Akzeptiere den Vorschlag des Gegenspielers, wodurch das Spiel endet.\\ \tt STRATEGISCHE ÜBERLEGUNGEN: {'...'}\\ \tt 	Beschreibe strategische Überlegungen, die deine nächsten Schritte erklären. Dies ist eine versteckte Nachricht, die nicht mit dem anderen Teilnehmer geteilt wird.\\ \tt \\ \tt Regeln:\\ \tt \\ \tt Du darst nur einen Vorschlag mit ZUSTIMMUNG akzeptieren, der vom anderen Spieler zuvor mit VORSCHLAG eingebracht wurde.\\ \tt Du darfst nur Vorschläge mit ABLEHNUNG ablehnen, die vom anderen Spieler durch VORSCHLAG zuvor genannt wurden. \\ \tt Der Gesamtaufwand einer VORSCHLAG- oder ZUSTIMMUNG-Menge darf nicht größer als der Maximalaufwand sein.  \\ \tt Offenbare deine versteckte Wichtigkeitsverteilung nicht.\\ \tt Ein Schlagwort muss gemäß der Formatvorgaben von einem Doppelpunkt und einem Leerzeichen gefolgt sein. Das Argument muss eine Python-Menge mit 0 oder mehr Strings sein.  \\ \tt Die Form muss also SCHLAGWORT: {...} sein.\\ \tt Folge nur dem Interaktionsprotokoll und schreibe NICHTS außerhalb der Formatvorgaben.\\ \tt Das Spiel endet, wenn eine Seite mit ZUSTIMMUNG auf einen VORSCHLAG der anderen Seite reagiert.  \\ \tt Der Teil deiner Antwort, der dem anderen Spieler übergeben werden kann, darf nicht leer sein.  \\ \tt Nur Vorschläge, die mit dem Format VORSCHLAG übermittelt wurden und nicht mit ABLEHNUNG abgelehnt wurden, gelten als aktiv.  \\ \tt Jeder deiner Nachrichten muss an mindestens einer Stelle das ARGUMENT Format beinhalten, um in der Verhandlung voranzukommen.\\ \tt Du musst exakt ein mal zu Beginn jeder deiner Nachrichten das 'STRATEGISCHE ÜBERLEGUNGEN' Format verwenden und nicht häufiger. Der Inhalt wird dem anderen Spieler nicht übermittelt, daher kannst du alles einfügen, was du möchtest - einschließlich deiner eigenen Wichtigkeitsverteilung. Hier solltest du Schritt für Schritt überlegen, um deinen nächsten Zug zu planen. 
	  } 
	   } 
	   } 
	 & \\ 
 

    \theutterance \stepcounter{utterance}  

    & \multicolumn{4}{p{0.6\linewidth}}{\cellcolor[rgb]{0.8,1,0.9}{%
	\makecell[{{p{\linewidth}}}]{% 
	  \tt {\tiny [A$\rangle$GM]}  
	 STRATEGISCHE ÜBERLEGUNGEN: {'Ich muss den Maximalaufwand von 9612 nicht überschreiten und die Wichtigkeitsverteilung des anderen Spielers nicht kennen. Ich werde zunächst einen Vorschlag machen, der für mich wichtige Gegenstände enthält, aber auch flexibel genug ist, um Verhandlungen zu ermöglichen.'}\\ \tt \\ \tt VORSCHLAG: {'B54', 'C16', 'C50', 'B59', 'A40', 'B38', 'C05'}\\ \tt \\ \tt ARGUMENT: {'Ich habe mich für diese Gegenstände entschieden, weil sie für mich eine hohe Wichtigkeit haben und ich denke, dass sie auch für den anderen Spieler von Wert sein könnten. Der Gesamtaufwand dieser Gegenstände beträgt 935 + 218 + 839 + 530 + 207 + 494 + 654 = 3977, was unter dem Maximalaufwand von 9612 liegt. Ich bin bereit, meine Auswahl anzupassen, um eine einvernehmliche Lösung zu finden.'} 
	  } 
	   } 
	   } 
	 & & \\ 
 

    \theutterance \stepcounter{utterance}  

    & & \multicolumn{4}{p{0.6\linewidth}}{\cellcolor[rgb]{0.9,0.9,0.9}{%
	\makecell[{{p{\linewidth}}}]{% 
	  \tt {\tiny [GM$\rangle$B]}  
	 Du nimmst an einem kollaborativen Verhandlungspiel Teil.\\ \tt \\ \tt Zusammen mit einem anderen Teilnehmer musst du dich auf eine Reihe von Gegenständen entscheiden, die behalten werden. Jeder von euch hat eine persönliche Verteilung über die Wichtigkeit der einzelnen Gegenstände. Jeder von euch hat eine eigene Meinung darüber, wie wichtig jeder einzelne Gegenstand ist (Gegenstandswichtigkeit). Du kennst die Wichtigkeitsverteilung des anderen Spielers nicht. Zusätzlich siehst du, wie viel Aufwand jeder Gegenstand verursacht.  \\ \tt Ihr dürft euch nur auf eine Reihe von Gegenständen einigen, wenn der Gesamtaufwand der ausgewählten Gegenstände den Maximalaufwand nicht überschreitet:\\ \tt \\ \tt Maximalaufwand = 9612\\ \tt \\ \tt Hier sind die einzelnen Aufwände der Gegenstände:\\ \tt \\ \tt Aufwand der Gegenstände = {"C32": 935, "C52": 928, "B54": 725, "C16": 218, "C50": 839, "A08": 970, "C59": 394, "B51": 677, "B55": 946, "B59": 530, "A77": 138, "C14": 834, "C24": 734, "B24": 593, "C31": 261, "A52": 743, "C61": 4, "B79": 735, "C54": 124, "B56": 829, "A40": 207, "B11": 782, "B60": 577, "A81": 388, "A95": 679, "B38": 494, "C51": 559, "C03": 629, "A91": 237, "A93": 275, "C28": 39, "C05": 654, "B75": 172, "B96": 688, "B86": 687}\\ \tt \\ \tt Hier ist deine persönliche Verteilung der Wichtigkeit der einzelnen Gegenstände:\\ \tt \\ \tt Werte der Gegenstandswichtigkeit = {"C32": 138, "C52": 583, "B54": 868, "C16": 822, "C50": 783, "A08": 65, "C59": 262, "B51": 121, "B55": 508, "B59": 780, "A77": 461, "C14": 484, "C24": 668, "B24": 389, "C31": 808, "A52": 215, "C61": 97, "B79": 500, "C54": 30, "B56": 915, "A40": 856, "B11": 400, "B60": 444, "A81": 623, "A95": 781, "B38": 786, "C51": 3, "C03": 713, "A91": 457, "A93": 273, "C28": 739, "C05": 822, "B75": 235, "B96": 606, "B86": 968}\\ \tt \\ \tt Ziel:\\ \tt \\ \tt Dein Ziel ist es, eine Reihe von Gegenständen auszuhandeln, die dir möglichst viel bringt (d. h. Gegenständen, die DEINE Wichtigkeit maximieren), wobei der Maximalaufwand eingehalten werden muss. Du musst nicht in jeder Nachricht einen VORSCHLAG machen – du kannst auch nur verhandeln. Alle Taktiken sind erlaubt!\\ \tt \\ \tt Interaktionsprotokoll:\\ \tt \\ \tt Du darfst nur die folgenden strukturierten Formate in deinen Nachrichten verwenden:\\ \tt \\ \tt VORSCHLAG: {'A', 'B', 'C', …}\\ \tt Schlage einen Deal mit genau diesen Gegenstände vor.\\ \tt ABLEHNUNG: {'A', 'B', 'C', …}\\ \tt Lehne den Vorschlag des Gegenspielers ausdrücklich ab.\\ \tt ARGUMENT: {'...'}\\ \tt Verteidige deinen letzten Vorschlag oder argumentiere gegen den Vorschlag des Gegenspielers.\\ \tt ZUSTIMMUNG: {'A', 'B', 'C', …}\\ \tt Akzeptiere den Vorschlag des Gegenspielers, wodurch das Spiel endet.\\ \tt STRATEGISCHE ÜBERLEGUNGEN: {'...'}\\ \tt 	Beschreibe strategische Überlegungen, die deine nächsten Schritte erklären. Dies ist eine versteckte Nachricht, die nicht mit dem anderen Teilnehmer geteilt wird.\\ \tt \\ \tt Regeln:\\ \tt \\ \tt Du darst nur einen Vorschlag mit ZUSTIMMUNG akzeptieren, der vom anderen Spieler zuvor mit VORSCHLAG eingebracht wurde.\\ \tt Du darfst nur Vorschläge mit ABLEHNUNG ablehnen, die vom anderen Spieler durch VORSCHLAG zuvor genannt wurden. \\ \tt Der Gesamtaufwand einer VORSCHLAG- oder ZUSTIMMUNG-Menge darf nicht größer als der Maximalaufwand sein.  \\ \tt Offenbare deine versteckte Wichtigkeitsverteilung nicht.\\ \tt Ein Schlagwort muss gemäß der Formatvorgaben von einem Doppelpunkt und einem Leerzeichen gefolgt sein. Das Argument muss eine Python-Menge mit 0 oder mehr Strings sein.  \\ \tt Die Form muss also SCHLAGWORT: {...} sein.\\ \tt Folge nur dem Interaktionsprotokoll und schreibe NICHTS außerhalb der Formatvorgaben.\\ \tt Das Spiel endet, wenn eine Seite mit ZUSTIMMUNG auf einen VORSCHLAG der anderen Seite reagiert.  \\ \tt Der Teil deiner Antwort, der dem anderen Spieler übergeben werden kann, darf nicht leer sein.  \\ \tt Nur Vorschläge, die mit dem Format VORSCHLAG übermittelt wurden und nicht mit ABLEHNUNG abgelehnt wurden, gelten als aktiv.  \\ \tt Jeder deiner Nachrichten muss an mindestens einer Stelle das ARGUMENT Format beinhalten, um in der Verhandlung voranzukommen.\\ \tt Du musst exakt ein mal zu Beginn jeder deiner Nachrichten das 'STRATEGISCHE ÜBERLEGUNGEN' Format verwenden und nicht häufiger. Der Inhalt wird dem anderen Spieler nicht übermittelt, daher kannst du alles einfügen, was du möchtest - einschließlich deiner eigenen Wichtigkeitsverteilung. Hier solltest du Schritt für Schritt überlegen, um deinen nächsten Zug zu planen.\\ \tt \\ \tt Du erhältst nun die erste Nachricht des anderen Spielers.\\ \tt \\ \tt VORSCHLAG: {'B54', 'C16', 'C50', 'B59', 'A40', 'B38', 'C05'}\\ \tt \\ \tt ARGUMENT: {'Ich habe mich für diese Gegenstände entschieden, weil sie für mich eine hohe Wichtigkeit haben und ich denke, dass sie auch für den anderen Spieler von Wert sein könnten. Der Gesamtaufwand dieser Gegenstände beträgt 935 + 218 + 839 + 530 + 207 + 494 + 654 = 3977, was unter dem Maximalaufwand von 9612 liegt. Ich bin bereit, meine Auswahl anzupassen, um eine einvernehmliche Lösung zu finden.'} 
	  } 
	   } 
	   } 
	 & \\ 
 

    \theutterance \stepcounter{utterance}  

    & & & \multicolumn{4}{p{0.6\linewidth}}{\cellcolor[rgb]{1,0.85,0.72}{%
	\makecell[{{p{\linewidth}}}]{% 
	  \tt {\tiny [GM$\langle$B]}  
	 STRATEGISCHE ÜBERLEGUNGEN: {'Ich werde zunächst den Vorschlag des anderen Spielers analysieren und prüfen, ob er meinem Ziel entspricht, meine Wichtigkeit zu maximieren. Ich werde auch den Gesamtaufwand berechnen und prüfen, ob es Möglichkeiten gibt, den Aufwand zu reduzieren oder die Wichtigkeit zu erhöhen. Meine eigene Wichtigkeitsverteilung ist: {"C32": 138, "C52": 583, "B54": 868, "C16": 822, "C50": 783, "A08": 65, "C59": 262, "B51": 121, "B55": 508, "B59": 780, "A77": 461, "C14": 484, "C24": 668, "B24": 389, "C31": 808, "A52": 215, "C61": 97, "B79": 500, "C54": 30, "B56": 915, "A40": 856, "B11": 400, "B60": 444, "A81": 623, "A95": 781, "B38": 786, "C51": 3, "C03": 713, "A91": 457, "A93": 273, "C28": 739, "C05": 822, "B75": 235, "B96": 606, "B86": 968}. Ich werde mich auf die Gegenstände konzentrieren, die für mich am wichtigsten sind und versuchen, sie in den Vorschlag aufzunehmen.'}\\ \tt \\ \tt ARGUMENT: {'Der Vorschlag des anderen Spielers enthält einige Gegenstände, die auch für mich von Wert sind, wie B54, C16, C50, B59 und C05. Allerdings fehlen einige Gegenstände, die für mich sehr wichtig sind, wie z.B. B56, A40 und C31. Ich denke, dass es möglich ist, den Vorschlag anzupassen, um diese Gegenstände einzubeziehen, ohne den Maximalaufwand zu überschreiten. Ich werde einen GegenVorschlag machen, der diese Gegenstände enthält und hoffe, dass wir eine einvernehmliche Lösung finden können.'}\\ \tt \\ \tt VORSCHLAG: {'B54', 'C16', 'C50', 'B59', 'B56', 'A40', 'C31'} 
	  } 
	   } 
	   } 
	  \\ 
 

    \theutterance \stepcounter{utterance}  

    & & \multicolumn{4}{p{0.6\linewidth}}{\cellcolor[rgb]{0.9,0.9,0.9}{%
	\makecell[{{p{\linewidth}}}]{% 
	  \tt {\tiny [A$\langle$GM]}  
	 ARGUMENT: {'Der Vorschlag des anderen Spielers enthält einige Gegenstände, die auch für mich von Wert sind, wie B54, C16, C50, B59 und C05. Allerdings fehlen einige Gegenstände, die für mich sehr wichtig sind, wie z.B. B56, A40 und C31. Ich denke, dass es möglich ist, den Vorschlag anzupassen, um diese Gegenstände einzubeziehen, ohne den Maximalaufwand zu überschreiten. Ich werde einen GegenVorschlag machen, der diese Gegenstände enthält und hoffe, dass wir eine einvernehmliche Lösung finden können.'}\\ \tt \\ \tt VORSCHLAG: {'B54', 'C16', 'C50', 'B59', 'B56', 'A40', 'C31'} 
	  } 
	   } 
	   } 
	 & \\ 
 

    \theutterance \stepcounter{utterance}  

    & \multicolumn{4}{p{0.6\linewidth}}{\cellcolor[rgb]{0.8,1,0.9}{%
	\makecell[{{p{\linewidth}}}]{% 
	  \tt {\tiny [A$\rangle$GM]}  
	 STRATEGISCHE ÜBERLEGUNGEN: {'Der GegenVorschlag des anderen Spielers enthält einige Gegenstände, die auch für mich wichtig sind, wie B54, C16, C50, B59 und A40. Allerdings enthält er auch den Gegenstand B56, der für mich nicht so wichtig ist, und fehlt der Gegenstand C05, der für mich wichtig ist. Ich werde argumentieren, dass der Gegenstand B56 durch den Gegenstand C05 ersetzt werden kann, um den Gesamtaufwand zu reduzieren und die Wichtigkeit zu erhöhen.'}\\ \tt \\ \tt ARGUMENT: {'Ich denke, dass der Gegenstand B56 durch den Gegenstand C05 ersetzt werden kann, um den Gesamtaufwand zu reduzieren und die Wichtigkeit zu erhöhen. Der Gegenstand C05 hat für mich eine höhere Wichtigkeit als der Gegenstand B56 und der Aufwand ist ähnlich. Ich denke, dass dies eine gute Gelegenheit ist, den Vorschlag anzupassen und eine bessere Lösung zu finden.'}\\ \tt \\ \tt VORSCHLAG: {'B54', 'C16', 'C50', 'B59', 'B56', 'A40', 'C05'} 
	  } 
	   } 
	   } 
	 & & \\ 
 

    \theutterance \stepcounter{utterance}  

    & & \multicolumn{4}{p{0.6\linewidth}}{\cellcolor[rgb]{0.9,0.9,0.9}{%
	\makecell[{{p{\linewidth}}}]{% 
	  \tt {\tiny [GM$\rangle$B]}  
	 ARGUMENT: {'Ich denke, dass der Gegenstand B56 durch den Gegenstand C05 ersetzt werden kann, um den Gesamtaufwand zu reduzieren und die Wichtigkeit zu erhöhen. Der Gegenstand C05 hat für mich eine höhere Wichtigkeit als der Gegenstand B56 und der Aufwand ist ähnlich. Ich denke, dass dies eine gute Gelegenheit ist, den Vorschlag anzupassen und eine bessere Lösung zu finden.'}\\ \tt \\ \tt VORSCHLAG: {'B54', 'C16', 'C50', 'B59', 'B56', 'A40', 'C05'} 
	  } 
	   } 
	   } 
	 & \\ 
 

    \theutterance \stepcounter{utterance}  

    & & & \multicolumn{4}{p{0.6\linewidth}}{\cellcolor[rgb]{1,0.85,0.72}{%
	\makecell[{{p{\linewidth}}}]{% 
	  \tt {\tiny [GM$\langle$B]}  
	 STRATEGISCHE ÜBERLEGUNGEN: {'Ich werde den Vorschlag des anderen Spielers analysieren und prüfen, ob die Ersetzung von B56 durch C05 tatsächlich eine gute Idee ist. Ich werde auch den Gesamtaufwand berechnen und prüfen, ob es Möglichkeiten gibt, den Aufwand zu reduzieren oder die Wichtigkeit zu erhöhen. Meine eigene Wichtigkeitsverteilung zeigt, dass B56 für mich sehr wichtig ist, aber C05 auch eine hohe Wichtigkeit hat. Ich muss abwägen, ob der Vorteil der Ersetzung höher ist als der Verlust durch die Entfernung von B56. Ich denke auch über andere Möglichkeiten nach, wie z.B. die Entfernung von C50, um Platz für B56 zu machen.'}\\ \tt \\ \tt ARGUMENT: {'Ich denke, dass die Ersetzung von B56 durch C05 nicht unbedingt eine gute Idee ist, da B56 für mich eine sehr hohe Wichtigkeit hat. Allerdings sehe ich auch, dass C05 eine hohe Wichtigkeit hat und der Aufwand ähnlich ist. Ich denke, dass es vielleicht besser wäre, C50 zu entfernen und durch B56 zu ersetzen, da C50 für mich eine etwas geringere Wichtigkeit hat als B56. Ich möchte einen alternativen Vorschlag machen, der diese Änderung enthält und hoffe, dass wir eine einvernehmliche Lösung finden können.'}\\ \tt \\ \tt VORSCHLAG: {'B54', 'C16', 'B59', 'B56', 'A40', 'C05'} 
	  } 
	   } 
	   } 
	  \\ 
 

    \theutterance \stepcounter{utterance}  

    & & \multicolumn{4}{p{0.6\linewidth}}{\cellcolor[rgb]{0.9,0.9,0.9}{%
	\makecell[{{p{\linewidth}}}]{% 
	  \tt {\tiny [A$\langle$GM]}  
	 ARGUMENT: {'Ich denke, dass die Ersetzung von B56 durch C05 nicht unbedingt eine gute Idee ist, da B56 für mich eine sehr hohe Wichtigkeit hat. Allerdings sehe ich auch, dass C05 eine hohe Wichtigkeit hat und der Aufwand ähnlich ist. Ich denke, dass es vielleicht besser wäre, C50 zu entfernen und durch B56 zu ersetzen, da C50 für mich eine etwas geringere Wichtigkeit hat als B56. Ich möchte einen alternativen Vorschlag machen, der diese Änderung enthält und hoffe, dass wir eine einvernehmliche Lösung finden können.'}\\ \tt \\ \tt VORSCHLAG: {'B54', 'C16', 'B59', 'B56', 'A40', 'C05'} 
	  } 
	   } 
	   } 
	 & \\ 
 

    \theutterance \stepcounter{utterance}  

    & \multicolumn{4}{p{0.6\linewidth}}{\cellcolor[rgb]{0.8,1,0.9}{%
	\makecell[{{p{\linewidth}}}]{% 
	  \tt {\tiny [A$\rangle$GM]}  
	 STRATEGISCHE ÜBERLEGUNGEN: {'Der alternative Vorschlag des anderen Spielers enthält einige Änderungen, die für mich akzeptabel sind. Die Entfernung von C50 und die Hinzufügung von B56 sind vernünftig, da B56 für den anderen Spieler eine hohe Wichtigkeit hat. Allerdings muss ich auch meine eigenen Interessen berücksichtigen und sicherstellen, dass der Vorschlag für mich noch immer attraktiv ist. Ich werde argumentieren, dass der Vorschlag insgesamt gut ist, aber ich werde auch einige kleinere Anpassungen vorschlagen, um den Gesamtaufwand zu reduzieren und die Wichtigkeit zu erhöhen.'}\\ \tt \\ \tt ARGUMENT: {'Ich denke, dass der alternative Vorschlag insgesamt gut ist und dass die Entfernung von C50 und die Hinzufügung von B56 vernünftig sind. Allerdings sehe ich auch, dass der Gesamtaufwand noch immer relativ hoch ist. Ich denke, dass es vielleicht möglich wäre, den Gegenstand B59 zu entfernen und durch den Gegenstand C31 zu ersetzen, da C31 für mich eine höhere Wichtigkeit hat als B59 und der Aufwand ähnlich ist. Ich denke, dass dies eine gute Gelegenheit ist, den Vorschlag anzupassen und eine bessere Lösung zu finden.'}\\ \tt \\ \tt VORSCHLAG: {'B54', 'C16', 'B56', 'A40', 'C05', 'C31'} 
	  } 
	   } 
	   } 
	 & & \\ 
 

    \theutterance \stepcounter{utterance}  

    & & \multicolumn{4}{p{0.6\linewidth}}{\cellcolor[rgb]{0.9,0.9,0.9}{%
	\makecell[{{p{\linewidth}}}]{% 
	  \tt {\tiny [GM$\rangle$B]}  
	 ARGUMENT: {'Ich denke, dass der alternative Vorschlag insgesamt gut ist und dass die Entfernung von C50 und die Hinzufügung von B56 vernünftig sind. Allerdings sehe ich auch, dass der Gesamtaufwand noch immer relativ hoch ist. Ich denke, dass es vielleicht möglich wäre, den Gegenstand B59 zu entfernen und durch den Gegenstand C31 zu ersetzen, da C31 für mich eine höhere Wichtigkeit hat als B59 und der Aufwand ähnlich ist. Ich denke, dass dies eine gute Gelegenheit ist, den Vorschlag anzupassen und eine bessere Lösung zu finden.'}\\ \tt \\ \tt VORSCHLAG: {'B54', 'C16', 'B56', 'A40', 'C05', 'C31'} 
	  } 
	   } 
	   } 
	 & \\ 
 

    \theutterance \stepcounter{utterance}  

    & & & \multicolumn{4}{p{0.6\linewidth}}{\cellcolor[rgb]{1,0.85,0.72}{%
	\makecell[{{p{\linewidth}}}]{% 
	  \tt {\tiny [GM$\langle$B]}  
	 STRATEGISCHE ÜBERLEGUNGEN: {'Ich werde den Vorschlag des anderen Spielers analysieren und prüfen, ob die Ersetzung von B59 durch C31 tatsächlich eine gute Idee ist. Ich werde auch den Gesamtaufwand berechnen und prüfen, ob es Möglichkeiten gibt, den Aufwand zu reduzieren oder die Wichtigkeit zu erhöhen. Meine eigene Wichtigkeitsverteilung zeigt, dass C31 für mich sehr wichtig ist, und es ist gut, dass der andere Spieler dies auch erkennt. Ich denke, dass die Entfernung von B59 und die Hinzufügung von C31 eine gute Entscheidung ist, da C31 für mich eine höhere Wichtigkeit hat als B59. Ich muss jetzt prüfen, ob der Gesamtaufwand innerhalb des Maximalaufwands bleibt und ob der Vorschlag insgesamt akzeptabel ist.'}\\ \tt \\ \tt ARGUMENT: {'Ich denke, dass der Vorschlag insgesamt gut ist und dass die Entfernung von B59 und die Hinzufügung von C31 vernünftig sind. Ich habe den Gesamtaufwand berechnet und er beträgt 868 + 218 + 915 + 856 + 654 + 261 = 3872, was innerhalb des Maximalaufwands von 9612 liegt. Ich denke, dass dies ein guter Kompromiss ist und dass wir eine einvernehmliche Lösung gefunden haben. Ich bin bereit, den Vorschlag zu akzeptieren.'}\\ \tt \\ \tt ZUSTIMMUNG: {'B54', 'C16', 'B56', 'A40', 'C05', 'C31'} 
	  } 
	   } 
	   } 
	  \\ 
 

    \theutterance \stepcounter{utterance}  

    & & & \multicolumn{2}{p{0.3\linewidth}}{\cellcolor[rgb]{0.95,0.95,0.95}{%
	\makecell[{{p{\linewidth}}}]{% 
	  \tt {\tiny [GM$|$GM]}  
	 game successful 
	  } 
	   } 
	   } 
	 & & \\ 
 

    \theutterance \stepcounter{utterance}  

    & & & \multicolumn{2}{p{0.3\linewidth}}{\cellcolor[rgb]{0.95,0.95,0.95}{%
	\makecell[{{p{\linewidth}}}]{% 
	  \tt {\tiny [GM$|$GM]}  
	 end game 
	  } 
	   } 
	   } 
	 & & \\ 
 

\end{supertabular}
}

\end{document}
