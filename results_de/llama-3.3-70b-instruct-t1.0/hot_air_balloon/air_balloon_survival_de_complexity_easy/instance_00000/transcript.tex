
\documentclass{article}
\usepackage{colortbl}
\usepackage{makecell}
\usepackage{multirow}
\usepackage{supertabular}

\begin{document}

\newcounter{utterance}

\centering \large Interaction Transcript for game `hot\_air\_balloon', experiment `air\_balloon\_survival\_de\_complexity\_easy', episode 0 with llama{-}3.3{-}70b{-}instruct{-}t1.0.
\vspace{24pt}

{ \footnotesize  \setcounter{utterance}{1}
\setlength{\tabcolsep}{0pt}
\begin{supertabular}{c@{$\;$}|p{.15\linewidth}@{}p{.15\linewidth}p{.15\linewidth}p{.15\linewidth}p{.15\linewidth}p{.15\linewidth}}
    \# & $\;$A & \multicolumn{4}{c}{Game Master} & $\;\:$B\\
    \hline

    \theutterance \stepcounter{utterance}  
    & & \multicolumn{4}{p{0.6\linewidth}}{
        \cellcolor[rgb]{0.9,0.9,0.9}{
            \makecell[{{p{\linewidth}}}]{
                \texttt{\tiny{[P1$\langle$GM]}}
                \texttt{Du nimmst an einem kollaborativen Verhandlungspiel Teil.} \\
\\ 
\texttt{Zusammen mit einem anderen Teilnehmer musst du dich auf eine Reihe von Gegenständen entscheiden, die behalten werden. Jeder von euch hat eine persönliche Verteilung über die Wichtigkeit der einzelnen Gegenstände. Jeder von euch hat eine eigene Meinung darüber, wie wichtig jeder einzelne Gegenstand ist (Gegenstandswichtigkeit). Du kennst die Wichtigkeitsverteilung des anderen Spielers nicht. Zusätzlich siehst du, wie viel Aufwand jeder Gegenstand verursacht.  } \\
\texttt{Ihr dürft euch nur auf eine Reihe von Gegenständen einigen, wenn der Gesamtaufwand der ausgewählten Gegenstände den Maximalaufwand nicht überschreitet:} \\
\\ 
\texttt{Maximalaufwand = 7605} \\
\\ 
\texttt{Hier sind die einzelnen Aufwände der Gegenstände:} \\
\\ 
\texttt{Aufwand der Gegenstände = \{"C32": 369, "C52": 951, "B54": 156, "C16": 266, "C50": 581, "A08": 287, "C59": 866, "B51": 180, "B55": 800, "B59": 739, "A77": 637, "C14": 86, "C24": 749, "B24": 370, "C31": 345, "A52": 951, "C61": 147, "B79": 265, "C54": 262, "B56": 259, "A40": 358, "B11": 394, "B60": 286, "A81": 580, "A95": 479, "B38": 14, "C51": 153, "C03": 977, "A91": 134, "A93": 982, "C28": 259, "C05": 232, "B75": 202, "B96": 73, "B86": 821\}} \\
\\ 
\texttt{Hier ist deine persönliche Verteilung der Wichtigkeit der einzelnen Gegenstände:} \\
\\ 
\texttt{Werte der Gegenstandswichtigkeit = \{"C32": 138, "C52": 583, "B54": 868, "C16": 822, "C50": 783, "A08": 65, "C59": 262, "B51": 121, "B55": 508, "B59": 780, "A77": 461, "C14": 484, "C24": 668, "B24": 389, "C31": 808, "A52": 215, "C61": 97, "B79": 500, "C54": 30, "B56": 915, "A40": 856, "B11": 400, "B60": 444, "A81": 623, "A95": 781, "B38": 786, "C51": 3, "C03": 713, "A91": 457, "A93": 273, "C28": 739, "C05": 822, "B75": 235, "B96": 606, "B86": 968\}} \\
\\ 
\texttt{Ziel:} \\
\\ 
\texttt{Dein Ziel ist es, eine Reihe von Gegenständen auszuhandeln, die dir möglichst viel bringt (d. h. Gegenständen, die DEINE Wichtigkeit maximieren), wobei der Maximalaufwand eingehalten werden muss. Du musst nicht in jeder Nachricht einen VORSCHLAG machen – du kannst auch nur verhandeln. Alle Taktiken sind erlaubt!} \\
\\ 
\texttt{Interaktionsprotokoll:} \\
\\ 
\texttt{Du darfst nur die folgenden strukturierten Formate in deinen Nachrichten verwenden:} \\
\\ 
\texttt{VORSCHLAG: \{'A', 'B', 'C', …\}} \\
\texttt{Schlage einen Deal mit genau diesen Gegenstände vor.} \\
\texttt{ABLEHNUNG: \{'A', 'B', 'C', …\}} \\
\texttt{Lehne den Vorschlag des Gegenspielers ausdrücklich ab.} \\
\texttt{ARGUMENT: \{'...'\}} \\
\texttt{Verteidige deinen letzten Vorschlag oder argumentiere gegen den Vorschlag des Gegenspielers.} \\
\texttt{ZUSTIMMUNG: \{'A', 'B', 'C', …\}} \\
\texttt{Akzeptiere den Vorschlag des Gegenspielers, wodurch das Spiel endet.} \\
\texttt{STRATEGISCHE ÜBERLEGUNGEN: \{'...'\}} \\
\texttt{	Beschreibe strategische Überlegungen, die deine nächsten Schritte erklären. Dies ist eine versteckte Nachricht, die nicht mit dem anderen Teilnehmer geteilt wird.} \\
\\ 
\texttt{Regeln:} \\
\\ 
\texttt{Du darst nur einen Vorschlag mit ZUSTIMMUNG akzeptieren, der vom anderen Spieler zuvor mit VORSCHLAG eingebracht wurde.} \\
\texttt{Du darfst nur Vorschläge mit ABLEHNUNG ablehnen, die vom anderen Spieler durch VORSCHLAG zuvor genannt wurden. } \\
\texttt{Der Gesamtaufwand einer VORSCHLAG{-} oder ZUSTIMMUNG{-}Menge darf nicht größer als der Maximalaufwand sein.  } \\
\texttt{Offenbare deine versteckte Wichtigkeitsverteilung nicht.} \\
\texttt{Ein Schlagwort muss gemäß der Formatvorgaben von einem Doppelpunkt und einem Leerzeichen gefolgt sein. Das Argument muss eine Python{-}Menge mit 0 oder mehr Strings sein.  } \\
\texttt{Die Form muss also SCHLAGWORT: \{...\} sein.} \\
\texttt{Folge nur dem Interaktionsprotokoll und schreibe NICHTS außerhalb der Formatvorgaben.} \\
\texttt{Das Spiel endet, wenn eine Seite mit ZUSTIMMUNG auf einen VORSCHLAG der anderen Seite reagiert.  } \\
\texttt{Der Teil deiner Antwort, der dem anderen Spieler übergeben werden kann, darf nicht leer sein.  } \\
\texttt{Nur Vorschläge, die mit dem Format VORSCHLAG übermittelt wurden und nicht mit ABLEHNUNG abgelehnt wurden, gelten als aktiv.  } \\
\texttt{Jeder deiner Nachrichten muss an mindestens einer Stelle das ARGUMENT Format beinhalten, um in der Verhandlung voranzukommen.} \\
\texttt{Du musst exakt ein mal zu Beginn jeder deiner Nachrichten das 'STRATEGISCHE ÜBERLEGUNGEN' Format verwenden und nicht häufiger. Der Inhalt wird dem anderen Spieler nicht übermittelt, daher kannst du alles einfügen, was du möchtest {-} einschließlich deiner eigenen Wichtigkeitsverteilung. Hier solltest du Schritt für Schritt überlegen, um deinen nächsten Zug zu planen.} \\
            }
        }
    }
    & \\ \\

    \theutterance \stepcounter{utterance}  
    & \multicolumn{4}{p{0.6\linewidth}}{
        \cellcolor[rgb]{0.9,0.9,0.9}{
            \makecell[{{p{\linewidth}}}]{
                \texttt{\tiny{[P1$\rangle$GM]}}
                \texttt{STRATEGISCHE ÜBERLEGUNGEN: \{"Überblick über die Situation", "Maximalaufwand: 7605", "Meine Wichtigkeitsverteilung: Werte der Gegenstandswichtigkeit = \{C32: 138, C52: 583, B54: 868, C16: 822, C50: 783, A08: 65, C59: 262, B51: 121, B55: 508, B59: 780, A77: 461, C14: 484, C24: 668, B24: 389, C31: 808, A52: 215, C61: 97, B79: 500, C54: 30, B56: 915, A40: 856, B11: 400, B60: 444, A81: 623, A95: 781, B38: 786, C51: 3, C03: 713, A91: 457, A93: 273, C28: 739, C05: 822, B75: 235, B96: 606, B86: 968\}", "Priorisierung meiner Werte", "Eine Reihe von Gegenständen auswählen, die meinen Werten entsprechen und den Maximalaufwand nicht überschreiten"\}} \\
\texttt{ARGUMENT: \{"Ich möchte eine Mischung aus Gegenständen mit hoher Wichtigkeit und geringem Aufwand", "Gegenstände wie B54, C16 und C50 scheinen eine gute Balance zwischen Wichtigkeit und Aufwand zu bieten"\}} \\
\texttt{VORSCHLAG: \{"B54", "C16", "C50", "B56", "A40"\}} \\
            }
        }
    }
    & & \\ \\

    \theutterance \stepcounter{utterance}  
    & & \multicolumn{4}{p{0.6\linewidth}}{
        \cellcolor[rgb]{0.9,0.9,0.9}{
            \makecell[{{p{\linewidth}}}]{
                \texttt{\tiny{[GM$\rangle$P2]}}
                \texttt{Du nimmst an einem kollaborativen Verhandlungspiel Teil.} \\
\\ 
\texttt{Zusammen mit einem anderen Teilnehmer musst du dich auf eine Reihe von Gegenständen entscheiden, die behalten werden. Jeder von euch hat eine persönliche Verteilung über die Wichtigkeit der einzelnen Gegenstände. Jeder von euch hat eine eigene Meinung darüber, wie wichtig jeder einzelne Gegenstand ist (Gegenstandswichtigkeit). Du kennst die Wichtigkeitsverteilung des anderen Spielers nicht. Zusätzlich siehst du, wie viel Aufwand jeder Gegenstand verursacht.  } \\
\texttt{Ihr dürft euch nur auf eine Reihe von Gegenständen einigen, wenn der Gesamtaufwand der ausgewählten Gegenstände den Maximalaufwand nicht überschreitet:} \\
\\ 
\texttt{Maximalaufwand = 7605} \\
\\ 
\texttt{Hier sind die einzelnen Aufwände der Gegenstände:} \\
\\ 
\texttt{Aufwand der Gegenstände = \{"C32": 369, "C52": 951, "B54": 156, "C16": 266, "C50": 581, "A08": 287, "C59": 866, "B51": 180, "B55": 800, "B59": 739, "A77": 637, "C14": 86, "C24": 749, "B24": 370, "C31": 345, "A52": 951, "C61": 147, "B79": 265, "C54": 262, "B56": 259, "A40": 358, "B11": 394, "B60": 286, "A81": 580, "A95": 479, "B38": 14, "C51": 153, "C03": 977, "A91": 134, "A93": 982, "C28": 259, "C05": 232, "B75": 202, "B96": 73, "B86": 821\}} \\
\\ 
\texttt{Hier ist deine persönliche Verteilung der Wichtigkeit der einzelnen Gegenstände:} \\
\\ 
\texttt{Werte der Gegenstandswichtigkeit = \{"C32": 138, "C52": 583, "B54": 868, "C16": 822, "C50": 783, "A08": 65, "C59": 262, "B51": 121, "B55": 508, "B59": 780, "A77": 461, "C14": 484, "C24": 668, "B24": 389, "C31": 808, "A52": 215, "C61": 97, "B79": 500, "C54": 30, "B56": 915, "A40": 856, "B11": 400, "B60": 444, "A81": 623, "A95": 781, "B38": 786, "C51": 3, "C03": 713, "A91": 457, "A93": 273, "C28": 739, "C05": 822, "B75": 235, "B96": 606, "B86": 968\}} \\
\\ 
\texttt{Ziel:} \\
\\ 
\texttt{Dein Ziel ist es, eine Reihe von Gegenständen auszuhandeln, die dir möglichst viel bringt (d. h. Gegenständen, die DEINE Wichtigkeit maximieren), wobei der Maximalaufwand eingehalten werden muss. Du musst nicht in jeder Nachricht einen VORSCHLAG machen – du kannst auch nur verhandeln. Alle Taktiken sind erlaubt!} \\
\\ 
\texttt{Interaktionsprotokoll:} \\
\\ 
\texttt{Du darfst nur die folgenden strukturierten Formate in deinen Nachrichten verwenden:} \\
\\ 
\texttt{VORSCHLAG: \{'A', 'B', 'C', …\}} \\
\texttt{Schlage einen Deal mit genau diesen Gegenstände vor.} \\
\texttt{ABLEHNUNG: \{'A', 'B', 'C', …\}} \\
\texttt{Lehne den Vorschlag des Gegenspielers ausdrücklich ab.} \\
\texttt{ARGUMENT: \{'...'\}} \\
\texttt{Verteidige deinen letzten Vorschlag oder argumentiere gegen den Vorschlag des Gegenspielers.} \\
\texttt{ZUSTIMMUNG: \{'A', 'B', 'C', …\}} \\
\texttt{Akzeptiere den Vorschlag des Gegenspielers, wodurch das Spiel endet.} \\
\texttt{STRATEGISCHE ÜBERLEGUNGEN: \{'...'\}} \\
\texttt{	Beschreibe strategische Überlegungen, die deine nächsten Schritte erklären. Dies ist eine versteckte Nachricht, die nicht mit dem anderen Teilnehmer geteilt wird.} \\
\\ 
\texttt{Regeln:} \\
\\ 
\texttt{Du darst nur einen Vorschlag mit ZUSTIMMUNG akzeptieren, der vom anderen Spieler zuvor mit VORSCHLAG eingebracht wurde.} \\
\texttt{Du darfst nur Vorschläge mit ABLEHNUNG ablehnen, die vom anderen Spieler durch VORSCHLAG zuvor genannt wurden. } \\
\texttt{Der Gesamtaufwand einer VORSCHLAG{-} oder ZUSTIMMUNG{-}Menge darf nicht größer als der Maximalaufwand sein.  } \\
\texttt{Offenbare deine versteckte Wichtigkeitsverteilung nicht.} \\
\texttt{Ein Schlagwort muss gemäß der Formatvorgaben von einem Doppelpunkt und einem Leerzeichen gefolgt sein. Das Argument muss eine Python{-}Menge mit 0 oder mehr Strings sein.  } \\
\texttt{Die Form muss also SCHLAGWORT: \{...\} sein.} \\
\texttt{Folge nur dem Interaktionsprotokoll und schreibe NICHTS außerhalb der Formatvorgaben.} \\
\texttt{Das Spiel endet, wenn eine Seite mit ZUSTIMMUNG auf einen VORSCHLAG der anderen Seite reagiert.  } \\
\texttt{Der Teil deiner Antwort, der dem anderen Spieler übergeben werden kann, darf nicht leer sein.  } \\
\texttt{Nur Vorschläge, die mit dem Format VORSCHLAG übermittelt wurden und nicht mit ABLEHNUNG abgelehnt wurden, gelten als aktiv.  } \\
\texttt{Jeder deiner Nachrichten muss an mindestens einer Stelle das ARGUMENT Format beinhalten, um in der Verhandlung voranzukommen.} \\
\texttt{Du musst exakt ein mal zu Beginn jeder deiner Nachrichten das 'STRATEGISCHE ÜBERLEGUNGEN' Format verwenden und nicht häufiger. Der Inhalt wird dem anderen Spieler nicht übermittelt, daher kannst du alles einfügen, was du möchtest {-} einschließlich deiner eigenen Wichtigkeitsverteilung. Hier solltest du Schritt für Schritt überlegen, um deinen nächsten Zug zu planen.} \\
\\ 
\texttt{Du erhältst nun die erste Nachricht des anderen Spielers.} \\
\\ 
\texttt{ARGUMENT: \{"Ich möchte eine Mischung aus Gegenständen mit hoher Wichtigkeit und geringem Aufwand", "Gegenstände wie B54, C16 und C50 scheinen eine gute Balance zwischen Wichtigkeit und Aufwand zu bieten"\}} \\
\texttt{VORSCHLAG: \{"B54", "C16", "C50", "B56", "A40"\}} \\
            }
        }
    }
    & \\ \\

    \theutterance \stepcounter{utterance}  
    & & & \multicolumn{4}{p{0.6\linewidth}}{
        \cellcolor[rgb]{0.9,0.9,0.9}{
            \makecell[{{p{\linewidth}}}]{
                \texttt{\tiny{[GM$\langle$P2]}}
                \texttt{STRATEGISCHE ÜBERLEGUNGEN: \{"Ich analiziere den Vorschlag des anderen Spielers und überlege, wie ich darauf reagieren kann. Ich muss meine eigene Wichtigkeitsverteilung im Hinterkopf behalten und versuchen, eine Kombination von Gegenständen zu finden, die mir maximale Wichtigkeit bringt, ohne den Maximalaufwand zu überschreiten. Ich werde den Vorschlag des anderen Spielers ablehnen und einen eigenen Vorschlag machen, der meine Interessen besser widerspiegelt."\}} \\
\\ 
\texttt{ARGUMENT: \{"Der Vorschlag des anderen Spielers enthält einige gute Gegenstände, aber ich denke, dass ich eine bessere Kombination finden kann", "Gegenstände wie B54 und C16 sind wichtig, aber ich denke, dass ich andere Gegenstände finden kann, die eine bessere Balance zwischen Wichtigkeit und Aufwand bieten"\}} \\
\\ 
\texttt{ABLEHNUNG: \{"B54", "C16", "C50", "B56", "A40"\}} \\
\\ 
\texttt{VORSCHLAG: \{"B54", "C16", "B51", "C61", "A91"\}} \\
            }
        }
    }
     \\ \\

    \theutterance \stepcounter{utterance}  
    & & \multicolumn{4}{p{0.6\linewidth}}{
        \cellcolor[rgb]{0.9,0.9,0.9}{
            \makecell[{{p{\linewidth}}}]{
                \texttt{\tiny{[P1$\langle$GM]}}
                \texttt{ARGUMENT: \{"Der Vorschlag des anderen Spielers enthält einige gute Gegenstände, aber ich denke, dass ich eine bessere Kombination finden kann", "Gegenstände wie B54 und C16 sind wichtig, aber ich denke, dass ich andere Gegenstände finden kann, die eine bessere Balance zwischen Wichtigkeit und Aufwand bieten"\}} \\
\\ 
\texttt{ABLEHNUNG: \{"B54", "C16", "C50", "B56", "A40"\}} \\
\\ 
\texttt{VORSCHLAG: \{"B54", "C16", "B51", "C61", "A91"\}} \\
            }
        }
    }
    & \\ \\

    \theutterance \stepcounter{utterance}  
    & \multicolumn{4}{p{0.6\linewidth}}{
        \cellcolor[rgb]{0.9,0.9,0.9}{
            \makecell[{{p{\linewidth}}}]{
                \texttt{\tiny{[P1$\rangle$GM]}}
                \texttt{STRATEGISCHE ÜBERLEGUNGEN: \{"Analyse des neuen Vorschlags", "B54 und C16 sind immer noch gute Auswahlmöglichkeiten", "B51 und C61 haben einen relativ geringen Aufwand, aber ihre Wichtigkeit ist nicht extrem hoch", "A91 hat eine moderate Wichtigkeit und einen geringen Aufwand", "Überprüfung der Gesamtauswahl und Vergleich mit meinen Prioritäten", "Meine nächste Strategie: Ablehnung des aktuellen Vorschlags und Gegenangebot mit einer besseren Balance zwischen Wichtigkeit und Aufwand"\}} \\
\texttt{ARGUMENT: \{"Ich denke, dass der Vorschlag noch nicht optimal ist, da einige Gegenstände wie B51 und C61 nicht meine höchsten Werte haben", "Ich möchte eine Kombination finden, die meine Werte besser widerspiegelt"\}} \\
\texttt{ABLEHNUNG: \{"B54", "C16", "B51", "C61", "A91"\}} \\
\texttt{VORSCHLAG: \{"B54", "C16", "B56", "A40", "C31"\}} \\
            }
        }
    }
    & & \\ \\

    \theutterance \stepcounter{utterance}  
    & & \multicolumn{4}{p{0.6\linewidth}}{
        \cellcolor[rgb]{0.9,0.9,0.9}{
            \makecell[{{p{\linewidth}}}]{
                \texttt{\tiny{[GM$\rangle$P2]}}
                \texttt{ARGUMENT: \{"Ich denke, dass der Vorschlag noch nicht optimal ist, da einige Gegenstände wie B51 und C61 nicht meine höchsten Werte haben", "Ich möchte eine Kombination finden, die meine Werte besser widerspiegelt"\}} \\
\texttt{ABLEHNUNG: \{"B54", "C16", "B51", "C61", "A91"\}} \\
\texttt{VORSCHLAG: \{"B54", "C16", "B56", "A40", "C31"\}} \\
            }
        }
    }
    & \\ \\

    \theutterance \stepcounter{utterance}  
    & & & \multicolumn{4}{p{0.6\linewidth}}{
        \cellcolor[rgb]{0.9,0.9,0.9}{
            \makecell[{{p{\linewidth}}}]{
                \texttt{\tiny{[GM$\langle$P2]}}
                \texttt{STRATEGISCHE ÜBERLEGUNGEN: \{"Ich muss den Vorschlag des anderen Spielers sorgfältig analysieren und überlegen, wie ich darauf reagieren kann. Ich denke, dass der Vorschlag einige gute Gegenstände enthält, aber ich muss auch sicherstellen, dass ich meine eigenen Interessen wahre. Ich werde den Vorschlag ablehnen und einen eigenen Vorschlag machen, der meine Werte besser widerspiegelt. Ich muss auch darauf achten, dass der Gesamtaufwand nicht zu hoch wird."\}} \\
\\ 
\texttt{ARGUMENT: \{"Der Vorschlag des anderen Spielers enthält einige gute Gegenstände, aber ich denke, dass ich eine bessere Kombination finden kann", "Gegenstände wie B54 und C16 sind wichtig, aber ich denke, dass ich andere Gegenstände finden kann, die eine bessere Balance zwischen Wichtigkeit und Aufwand bieten", "Ich möchte auch sicherstellen, dass ich meine eigenen Prioritäten berücksichtige"\}} \\
\\ 
\texttt{ABLEHNUNG: \{"B54", "C16", "B56", "A40", "C31"\}} \\
\\ 
\texttt{VORSCHLAG: \{"B54", "C16", "A40", "C31", "B38"\}} \\
            }
        }
    }
     \\ \\

    \theutterance \stepcounter{utterance}  
    & & \multicolumn{4}{p{0.6\linewidth}}{
        \cellcolor[rgb]{0.9,0.9,0.9}{
            \makecell[{{p{\linewidth}}}]{
                \texttt{\tiny{[P1$\langle$GM]}}
                \texttt{ARGUMENT: \{"Der Vorschlag des anderen Spielers enthält einige gute Gegenstände, aber ich denke, dass ich eine bessere Kombination finden kann", "Gegenstände wie B54 und C16 sind wichtig, aber ich denke, dass ich andere Gegenstände finden kann, die eine bessere Balance zwischen Wichtigkeit und Aufwand bieten", "Ich möchte auch sicherstellen, dass ich meine eigenen Prioritäten berücksichtige"\}} \\
\\ 
\texttt{ABLEHNUNG: \{"B54", "C16", "B56", "A40", "C31"\}} \\
\\ 
\texttt{VORSCHLAG: \{"B54", "C16", "A40", "C31", "B38"\}} \\
            }
        }
    }
    & \\ \\

    \theutterance \stepcounter{utterance}  
    & \multicolumn{4}{p{0.6\linewidth}}{
        \cellcolor[rgb]{0.9,0.9,0.9}{
            \makecell[{{p{\linewidth}}}]{
                \texttt{\tiny{[P1$\rangle$GM]}}
                \texttt{STRATEGISCHE ÜBERLEGUNGEN: \{"Überprüfung des neuen Vorschlags", "B54, C16, A40 und C31 sind alle wichtige Gegenstände mit einer guten Balance zwischen Wichtigkeit und Aufwand", "B38 ist ein zusätzlicher Gegenstand, der eine hohe Wichtigkeit hat, aber ich muss prüfen, ob der Gesamtaufwand noch im Rahmen bleibt", "Meine Prioritäten: B54, C16, A40, C31 und B38 sind alle hochwertige Gegenstände, aber ich muss den Gesamtaufwand berücksichtigen", "Gesamtaufwand des Vorschlags: 868 + 822 + 358 + 345 + 786 = 3179", "Der Gesamtaufwand ist noch im Rahmen, aber ich muss sicherstellen, dass ich keine bessere Kombination finde"\}} \\
\texttt{ARGUMENT: \{"Ich denke, dass der Vorschlag des anderen Spielers besser ist als meine vorherigen Vorschläge", "Die Kombination aus B54, C16, A40, C31 und B38 bietet eine gute Balance zwischen Wichtigkeit und Aufwand"\}} \\
\texttt{ZUSTIMMUNG: \{"B54", "C16", "A40", "C31", "B38"\}} \\
            }
        }
    }
    & & \\ \\

    \theutterance \stepcounter{utterance}  
    & & & \multicolumn{2}{p{0.3\linewidth}}{
        \cellcolor[rgb]{0.9,0.9,0.9}{
            \makecell[{{p{\linewidth}}}]{
                \texttt{\tiny{[GM$|$GM]}}
                \texttt{game successful} \\
            }
        }
    }
    & & \\ \\

    \theutterance \stepcounter{utterance}  
    & & & \multicolumn{2}{p{0.3\linewidth}}{
        \cellcolor[rgb]{0.9,0.9,0.9}{
            \makecell[{{p{\linewidth}}}]{
                \texttt{\tiny{[GM$|$GM]}}
                \texttt{end game} \\
            }
        }
    }
    & & \\ \\

\end{supertabular}
}

\end{document}
