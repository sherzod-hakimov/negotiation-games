
\documentclass{article}
\usepackage{colortbl}
\usepackage{makecell}
\usepackage{multirow}
\usepackage{supertabular}

\begin{document}

\newcounter{utterance}

\centering \large Interaction Transcript for game `hot\_air\_balloon', experiment `air\_balloon\_survival\_de\_reasoning off\_easy', episode 1 with llama{-}3.3{-}70b{-}instruct{-}t1.0.
\vspace{24pt}

{ \footnotesize  \setcounter{utterance}{1}
\setlength{\tabcolsep}{0pt}
\begin{supertabular}{c@{$\;$}|p{.15\linewidth}@{}p{.15\linewidth}p{.15\linewidth}p{.15\linewidth}p{.15\linewidth}p{.15\linewidth}}
    \# & $\;$A & \multicolumn{4}{c}{Game Master} & $\;\:$B\\
    \hline

    \theutterance \stepcounter{utterance}  
    & & \multicolumn{4}{p{0.6\linewidth}}{
        \cellcolor[rgb]{0.9,0.9,0.9}{
            \makecell[{{p{\linewidth}}}]{
                \texttt{\tiny{[P1$\langle$GM]}}
                \texttt{Du nimmst an einem kollaborativen Verhandlungspiel Teil.} \\
\\ 
\texttt{Zusammen mit einem anderen Teilnehmer musst du dich auf eine Reihe von Gegenständen entscheiden, die behalten werden. Jeder von euch hat eine persönliche Verteilung über die Wichtigkeit der einzelnen Gegenstände. Jeder von euch hat eine eigene Meinung darüber, wie wichtig jeder einzelne Gegenstand ist (Gegenstandswichtigkeit). Du kennst die Wichtigkeitsverteilung des anderen Spielers nicht. Zusätzlich siehst du, wie viel Aufwand jeder Gegenstand verursacht.  } \\
\texttt{Ihr dürft euch nur auf eine Reihe von Gegenständen einigen, wenn der Gesamtaufwand der ausgewählten Gegenstände den Maximalaufwand nicht überschreitet:} \\
\\ 
\texttt{Maximalaufwand = 4100} \\
\\ 
\texttt{Hier sind die einzelnen Aufwände der Gegenstände:} \\
\\ 
\texttt{Aufwand der Gegenstände = \{"C41": 108, "C15": 133, "C28": 669, "C48": 835, "B78": 460, "A75": 537, "A60": 573, "B05": 737, "A56": 865, "B66": 855, "C60": 596, "B49": 719, "A35": 533, "A93": 549, "B10": 31\}} \\
\\ 
\texttt{Hier ist deine persönliche Verteilung der Wichtigkeit der einzelnen Gegenstände:} \\
\\ 
\texttt{Werte der Gegenstandswichtigkeit = \{"C41": 138, "C15": 583, "C28": 822, "C48": 783, "B78": 65, "A75": 262, "A60": 121, "B05": 508, "A56": 780, "B66": 461, "C60": 484, "B49": 668, "A35": 389, "A93": 808, "B10": 215\}} \\
\\ 
\texttt{Ziel:} \\
\\ 
\texttt{Dein Ziel ist es, eine Reihe von Gegenständen auszuhandeln, die dir möglichst viel bringt (d. h. Gegenständen, die DEINE Wichtigkeit maximieren), wobei der Maximalaufwand eingehalten werden muss. Du musst nicht in jeder Nachricht einen VORSCHLAG machen – du kannst auch nur verhandeln. Alle Taktiken sind erlaubt!} \\
\\ 
\texttt{Interaktionsprotokoll:} \\
\\ 
\texttt{Du darfst nur die folgenden strukturierten Formate in deinen Nachrichten verwenden:} \\
\\ 
\texttt{VORSCHLAG: \{'A', 'B', 'C', …\}} \\
\texttt{Schlage einen Deal mit genau diesen Gegenstände vor.} \\
\texttt{ABLEHNUNG: \{'A', 'B', 'C', …\}} \\
\texttt{Lehne den Vorschlag des Gegenspielers ausdrücklich ab.} \\
\texttt{ARGUMENT: \{'...'\}} \\
\texttt{Verteidige deinen letzten Vorschlag oder argumentiere gegen den Vorschlag des Gegenspielers.} \\
\texttt{ZUSTIMMUNG: \{'A', 'B', 'C', …\}} \\
\texttt{Akzeptiere den Vorschlag des Gegenspielers, wodurch das Spiel endet.} \\
\\ 
\\ 
\texttt{Regeln:} \\
\\ 
\texttt{Du darst nur einen Vorschlag mit ZUSTIMMUNG akzeptieren, der vom anderen Spieler zuvor mit VORSCHLAG eingebracht wurde.} \\
\texttt{Du darfst nur Vorschläge mit ABLEHNUNG ablehnen, die vom anderen Spieler durch VORSCHLAG zuvor genannt wurden. } \\
\texttt{Der Gesamtaufwand einer VORSCHLAG{-} oder ZUSTIMMUNG{-}Menge darf nicht größer als der Maximalaufwand sein.  } \\
\texttt{Offenbare deine versteckte Wichtigkeitsverteilung nicht.} \\
\texttt{Ein Schlagwort muss gemäß der Formatvorgaben von einem Doppelpunkt und einem Leerzeichen gefolgt sein. Das Argument muss eine Python{-}Menge mit 0 oder mehr Strings sein.  } \\
\texttt{Die Form muss also SCHLAGWORT: \{...\} sein.} \\
\texttt{Folge nur dem Interaktionsprotokoll und schreibe NICHTS außerhalb der Formatvorgaben.} \\
\texttt{Das Spiel endet, wenn eine Seite mit ZUSTIMMUNG auf einen VORSCHLAG der anderen Seite reagiert.  } \\
\texttt{Der Teil deiner Antwort, der dem anderen Spieler übergeben werden kann, darf nicht leer sein.  } \\
\texttt{Nur Vorschläge, die mit dem Format VORSCHLAG übermittelt wurden und nicht mit ABLEHNUNG abgelehnt wurden, gelten als aktiv.  } \\
\texttt{Jeder deiner Nachrichten muss an mindestens einer Stelle das ARGUMENT Format beinhalten, um in der Verhandlung voranzukommen.} \\
            }
        }
    }
    & \\ \\

    \theutterance \stepcounter{utterance}  
    & \multicolumn{4}{p{0.6\linewidth}}{
        \cellcolor[rgb]{0.9,0.9,0.9}{
            \makecell[{{p{\linewidth}}}]{
                \texttt{\tiny{[P1$\rangle$GM]}}
                \texttt{VORSCHLAG: \{'C41', 'C15', 'C28', 'B10'\}} \\
\texttt{ARGUMENT: \{'Ich denke, diese Kombination bietet eine gute Balance zwischen Aufwand und Nutzen', 'Der Gesamtaufwand bleibt unter dem Maximum', 'Es sind eine Vielzahl von Gegenständen enthalten, die meiner Meinung nach wichtig sind'\}} \\
            }
        }
    }
    & & \\ \\

    \theutterance \stepcounter{utterance}  
    & & \multicolumn{4}{p{0.6\linewidth}}{
        \cellcolor[rgb]{0.9,0.9,0.9}{
            \makecell[{{p{\linewidth}}}]{
                \texttt{\tiny{[GM$\rangle$P2]}}
                \texttt{Du nimmst an einem kollaborativen Verhandlungspiel Teil.} \\
\\ 
\texttt{Zusammen mit einem anderen Teilnehmer musst du dich auf eine Reihe von Gegenständen entscheiden, die behalten werden. Jeder von euch hat eine persönliche Verteilung über die Wichtigkeit der einzelnen Gegenstände. Jeder von euch hat eine eigene Meinung darüber, wie wichtig jeder einzelne Gegenstand ist (Gegenstandswichtigkeit). Du kennst die Wichtigkeitsverteilung des anderen Spielers nicht. Zusätzlich siehst du, wie viel Aufwand jeder Gegenstand verursacht.  } \\
\texttt{Ihr dürft euch nur auf eine Reihe von Gegenständen einigen, wenn der Gesamtaufwand der ausgewählten Gegenstände den Maximalaufwand nicht überschreitet:} \\
\\ 
\texttt{Maximalaufwand = 4100} \\
\\ 
\texttt{Hier sind die einzelnen Aufwände der Gegenstände:} \\
\\ 
\texttt{Aufwand der Gegenstände = \{"C41": 108, "C15": 133, "C28": 669, "C48": 835, "B78": 460, "A75": 537, "A60": 573, "B05": 737, "A56": 865, "B66": 855, "C60": 596, "B49": 719, "A35": 533, "A93": 549, "B10": 31\}} \\
\\ 
\texttt{Hier ist deine persönliche Verteilung der Wichtigkeit der einzelnen Gegenstände:} \\
\\ 
\texttt{Werte der Gegenstandswichtigkeit = \{"C41": 138, "C15": 583, "C28": 822, "C48": 783, "B78": 65, "A75": 262, "A60": 121, "B05": 508, "A56": 780, "B66": 461, "C60": 484, "B49": 668, "A35": 389, "A93": 808, "B10": 215\}} \\
\\ 
\texttt{Ziel:} \\
\\ 
\texttt{Dein Ziel ist es, eine Reihe von Gegenständen auszuhandeln, die dir möglichst viel bringt (d. h. Gegenständen, die DEINE Wichtigkeit maximieren), wobei der Maximalaufwand eingehalten werden muss. Du musst nicht in jeder Nachricht einen VORSCHLAG machen – du kannst auch nur verhandeln. Alle Taktiken sind erlaubt!} \\
\\ 
\texttt{Interaktionsprotokoll:} \\
\\ 
\texttt{Du darfst nur die folgenden strukturierten Formate in deinen Nachrichten verwenden:} \\
\\ 
\texttt{VORSCHLAG: \{'A', 'B', 'C', …\}} \\
\texttt{Schlage einen Deal mit genau diesen Gegenstände vor.} \\
\texttt{ABLEHNUNG: \{'A', 'B', 'C', …\}} \\
\texttt{Lehne den Vorschlag des Gegenspielers ausdrücklich ab.} \\
\texttt{ARGUMENT: \{'...'\}} \\
\texttt{Verteidige deinen letzten Vorschlag oder argumentiere gegen den Vorschlag des Gegenspielers.} \\
\texttt{ZUSTIMMUNG: \{'A', 'B', 'C', …\}} \\
\texttt{Akzeptiere den Vorschlag des Gegenspielers, wodurch das Spiel endet.} \\
\\ 
\\ 
\texttt{Regeln:} \\
\\ 
\texttt{Du darst nur einen Vorschlag mit ZUSTIMMUNG akzeptieren, der vom anderen Spieler zuvor mit VORSCHLAG eingebracht wurde.} \\
\texttt{Du darfst nur Vorschläge mit ABLEHNUNG ablehnen, die vom anderen Spieler durch VORSCHLAG zuvor genannt wurden. } \\
\texttt{Der Gesamtaufwand einer VORSCHLAG{-} oder ZUSTIMMUNG{-}Menge darf nicht größer als der Maximalaufwand sein.  } \\
\texttt{Offenbare deine versteckte Wichtigkeitsverteilung nicht.} \\
\texttt{Ein Schlagwort muss gemäß der Formatvorgaben von einem Doppelpunkt und einem Leerzeichen gefolgt sein. Das Argument muss eine Python{-}Menge mit 0 oder mehr Strings sein.  } \\
\texttt{Die Form muss also SCHLAGWORT: \{...\} sein.} \\
\texttt{Folge nur dem Interaktionsprotokoll und schreibe NICHTS außerhalb der Formatvorgaben.} \\
\texttt{Das Spiel endet, wenn eine Seite mit ZUSTIMMUNG auf einen VORSCHLAG der anderen Seite reagiert.  } \\
\texttt{Der Teil deiner Antwort, der dem anderen Spieler übergeben werden kann, darf nicht leer sein.  } \\
\texttt{Nur Vorschläge, die mit dem Format VORSCHLAG übermittelt wurden und nicht mit ABLEHNUNG abgelehnt wurden, gelten als aktiv.  } \\
\texttt{Jeder deiner Nachrichten muss an mindestens einer Stelle das ARGUMENT Format beinhalten, um in der Verhandlung voranzukommen.} \\
\\ 
\\ 
\texttt{Du erhältst nun die erste Nachricht des anderen Spielers.} \\
\\ 
\texttt{VORSCHLAG: \{'C41', 'C15', 'C28', 'B10'\}} \\
\texttt{ARGUMENT: \{'Ich denke, diese Kombination bietet eine gute Balance zwischen Aufwand und Nutzen', 'Der Gesamtaufwand bleibt unter dem Maximum', 'Es sind eine Vielzahl von Gegenständen enthalten, die meiner Meinung nach wichtig sind'\}} \\
            }
        }
    }
    & \\ \\

    \theutterance \stepcounter{utterance}  
    & & & \multicolumn{4}{p{0.6\linewidth}}{
        \cellcolor[rgb]{0.9,0.9,0.9}{
            \makecell[{{p{\linewidth}}}]{
                \texttt{\tiny{[GM$\langle$P2]}}
                \texttt{ABLEHNUNG: \{'C41', 'C15', 'C28', 'B10'\}} \\
\texttt{ARGUMENT: \{'Ich bin nicht überzeugt, dass diese Kombination meine Bedürfnisse erfüllt', 'Einige der Gegenstände haben einen relativen hohen Aufwand, ohne dass ich ihre Wichtigkeit richtig einschätzen kann', 'Ich denke, es gibt bessere Kombinationen, die wir noch nicht betrachtet haben'\} } \\
\texttt{VORSCHLAG: \{'A75', 'A60', 'B78', 'B10'\}} \\
\texttt{ARGUMENT: \{'Diese Kombination scheint mir eine bessere Balance zwischen den Bedürfnissen zu bieten', 'Der Gesamtaufwand bleibt innerhalb des Limits', 'Ich denke, es ist eine gute Grundlage für weitere Verhandlungen'\}} \\
            }
        }
    }
     \\ \\

    \theutterance \stepcounter{utterance}  
    & & \multicolumn{4}{p{0.6\linewidth}}{
        \cellcolor[rgb]{0.9,0.9,0.9}{
            \makecell[{{p{\linewidth}}}]{
                \texttt{\tiny{[P1$\langle$GM]}}
                \texttt{ABLEHNUNG: \{'C41', 'C15', 'C28', 'B10'\}} \\
\texttt{ARGUMENT: \{'Ich bin nicht überzeugt, dass diese Kombination meine Bedürfnisse erfüllt', 'Einige der Gegenstände haben einen relativen hohen Aufwand, ohne dass ich ihre Wichtigkeit richtig einschätzen kann', 'Ich denke, es gibt bessere Kombinationen, die wir noch nicht betrachtet haben'\} } \\
\texttt{VORSCHLAG: \{'A75', 'A60', 'B78', 'B10'\}} \\
\texttt{ARGUMENT: \{'Diese Kombination scheint mir eine bessere Balance zwischen den Bedürfnissen zu bieten', 'Der Gesamtaufwand bleibt innerhalb des Limits', 'Ich denke, es ist eine gute Grundlage für weitere Verhandlungen'\}} \\
            }
        }
    }
    & \\ \\

    \theutterance \stepcounter{utterance}  
    & \multicolumn{4}{p{0.6\linewidth}}{
        \cellcolor[rgb]{0.9,0.9,0.9}{
            \makecell[{{p{\linewidth}}}]{
                \texttt{\tiny{[P1$\rangle$GM]}}
                \texttt{ABLEHNUNG: \{'A75', 'A60', 'B78', 'B10'\}} \\
\texttt{ARGUMENT: \{'Ich bin nicht vollständig überzeugt, dass diese Kombination meine Prioritäten richtig widerspiegelt', 'Einige der Gegenstände in diesem Vorschlag haben, meiner Meinung nach, eine geringere Wichtigkeit als andere, die nicht berücksichtigt wurden', 'Ich denke, wir sollten weiter nach einer Lösung suchen, die beide Seiten besser bedient'\}} \\
\texttt{VORSCHLAG: \{'C48', 'A93', 'B05', 'C41'\}} \\
\texttt{ARGUMENT: \{'Dieser Vorschlag berücksichtigt eine breitere Palette von Gegenständen, die meiner Meinung nach wichtiger sind', 'Der Gesamtaufwand bleibt innerhalb des zulässigen Rahmens', 'Ich hoffe, diese Kombination kommt den Bedürfnissen beider Seiten näher'\}} \\
            }
        }
    }
    & & \\ \\

    \theutterance \stepcounter{utterance}  
    & & \multicolumn{4}{p{0.6\linewidth}}{
        \cellcolor[rgb]{0.9,0.9,0.9}{
            \makecell[{{p{\linewidth}}}]{
                \texttt{\tiny{[GM$\rangle$P2]}}
                \texttt{ABLEHNUNG: \{'A75', 'A60', 'B78', 'B10'\}} \\
\texttt{ARGUMENT: \{'Ich bin nicht vollständig überzeugt, dass diese Kombination meine Prioritäten richtig widerspiegelt', 'Einige der Gegenstände in diesem Vorschlag haben, meiner Meinung nach, eine geringere Wichtigkeit als andere, die nicht berücksichtigt wurden', 'Ich denke, wir sollten weiter nach einer Lösung suchen, die beide Seiten besser bedient'\}} \\
\texttt{VORSCHLAG: \{'C48', 'A93', 'B05', 'C41'\}} \\
\texttt{ARGUMENT: \{'Dieser Vorschlag berücksichtigt eine breitere Palette von Gegenständen, die meiner Meinung nach wichtiger sind', 'Der Gesamtaufwand bleibt innerhalb des zulässigen Rahmens', 'Ich hoffe, diese Kombination kommt den Bedürfnissen beider Seiten näher'\}} \\
            }
        }
    }
    & \\ \\

    \theutterance \stepcounter{utterance}  
    & & & \multicolumn{4}{p{0.6\linewidth}}{
        \cellcolor[rgb]{0.9,0.9,0.9}{
            \makecell[{{p{\linewidth}}}]{
                \texttt{\tiny{[GM$\langle$P2]}}
                \texttt{ABLEHNUNG: \{'C48', 'A93', 'B05', 'C41'\}} \\
\texttt{ARGUMENT: \{'Ich teile nicht vollständig die Meinung, dass diese Gegenstände die beste Auswahl darstellen', 'Einige der Gegenstände in diesem Vorschlag haben einen erheblichen Aufwand, ohne dass ich ihre Wichtigkeit in Relation zu anderen Optionen richtig bewerten kann', 'Ich denke, es gibt noch Raum für Verhandlungen, um eine bessere Übereinstimmung zu finden'\}} \\
\texttt{VORSCHLAG: \{'A56', 'B49', 'C60', 'A35'\}} \\
\texttt{ARGUMENT: \{'Diese Kombination scheint mir eine gute Mischung aus Wichtigkeit und Aufwand zu bieten', 'Der Gesamtaufwand bleibt unter dem Maximalaufwand', 'Ich hoffe, dass dieser Vorschlag uns näher an eine Einigung bringt, die beide Seiten zufriedenstellt'\}} \\
            }
        }
    }
     \\ \\

    \theutterance \stepcounter{utterance}  
    & & \multicolumn{4}{p{0.6\linewidth}}{
        \cellcolor[rgb]{0.9,0.9,0.9}{
            \makecell[{{p{\linewidth}}}]{
                \texttt{\tiny{[P1$\langle$GM]}}
                \texttt{ABLEHNUNG: \{'C48', 'A93', 'B05', 'C41'\}} \\
\texttt{ARGUMENT: \{'Ich teile nicht vollständig die Meinung, dass diese Gegenstände die beste Auswahl darstellen', 'Einige der Gegenstände in diesem Vorschlag haben einen erheblichen Aufwand, ohne dass ich ihre Wichtigkeit in Relation zu anderen Optionen richtig bewerten kann', 'Ich denke, es gibt noch Raum für Verhandlungen, um eine bessere Übereinstimmung zu finden'\}} \\
\texttt{VORSCHLAG: \{'A56', 'B49', 'C60', 'A35'\}} \\
\texttt{ARGUMENT: \{'Diese Kombination scheint mir eine gute Mischung aus Wichtigkeit und Aufwand zu bieten', 'Der Gesamtaufwand bleibt unter dem Maximalaufwand', 'Ich hoffe, dass dieser Vorschlag uns näher an eine Einigung bringt, die beide Seiten zufriedenstellt'\}} \\
            }
        }
    }
    & \\ \\

    \theutterance \stepcounter{utterance}  
    & \multicolumn{4}{p{0.6\linewidth}}{
        \cellcolor[rgb]{0.9,0.9,0.9}{
            \makecell[{{p{\linewidth}}}]{
                \texttt{\tiny{[P1$\rangle$GM]}}
                \texttt{ABLEHNUNG: \{'A56', 'B49', 'C60', 'A35'\}} \\
\texttt{ARGUMENT: \{'Ich bin nicht vollständig überzeugt, dass diese Kombination die optimalen Gegenstände für mich enthält', 'Einige der Gegenstände in diesem Vorschlag haben, meiner Meinung nach, eine geringere Priorität als andere, die nicht berücksichtigt wurden', 'Ich denke, es gibt noch potential für eine bessere Lösung, die beide Seiten besser bedient'\}} \\
\texttt{VORSCHLAG: \{'C28', 'C48', 'A93', 'B10'\}} \\
\texttt{ARGUMENT: \{'Dieser Vorschlag berücksichtigt eine breitere Palette von Gegenständen, die meiner Meinung nach wichtiger sind', 'Der Gesamtaufwand bleibt innerhalb des zulässigen Rahmens', 'Ich hoffe, dass diese Kombination eine bessere Übereinstimmung zwischen unseren Bedürfnissen findet'\}} \\
            }
        }
    }
    & & \\ \\

    \theutterance \stepcounter{utterance}  
    & & \multicolumn{4}{p{0.6\linewidth}}{
        \cellcolor[rgb]{0.9,0.9,0.9}{
            \makecell[{{p{\linewidth}}}]{
                \texttt{\tiny{[GM$\rangle$P2]}}
                \texttt{ABLEHNUNG: \{'A56', 'B49', 'C60', 'A35'\}} \\
\texttt{ARGUMENT: \{'Ich bin nicht vollständig überzeugt, dass diese Kombination die optimalen Gegenstände für mich enthält', 'Einige der Gegenstände in diesem Vorschlag haben, meiner Meinung nach, eine geringere Priorität als andere, die nicht berücksichtigt wurden', 'Ich denke, es gibt noch potential für eine bessere Lösung, die beide Seiten besser bedient'\}} \\
\texttt{VORSCHLAG: \{'C28', 'C48', 'A93', 'B10'\}} \\
\texttt{ARGUMENT: \{'Dieser Vorschlag berücksichtigt eine breitere Palette von Gegenständen, die meiner Meinung nach wichtiger sind', 'Der Gesamtaufwand bleibt innerhalb des zulässigen Rahmens', 'Ich hoffe, dass diese Kombination eine bessere Übereinstimmung zwischen unseren Bedürfnissen findet'\}} \\
            }
        }
    }
    & \\ \\

    \theutterance \stepcounter{utterance}  
    & & & \multicolumn{4}{p{0.6\linewidth}}{
        \cellcolor[rgb]{0.9,0.9,0.9}{
            \makecell[{{p{\linewidth}}}]{
                \texttt{\tiny{[GM$\langle$P2]}}
                \texttt{ARGUMENT: \{'Ich denke, wir kommen langsam näher an eine Einigung, aber ich bin noch nicht vollständig zufrieden', 'Einige der Gegenstände in diesem Vorschlag haben einen erheblichen Aufwand, und ich möchte sicherstellen, dass sie auch entsprechend wichtig sind', 'Ich frage mich, ob es möglich ist, den Gegenstand B10 durch einen anderen zu ersetzen, der möglicherweise eine bessere Balance bietet'\}} \\
\texttt{VORSCHLAG: \{'C28', 'C48', 'A93', 'A75'\}} \\
\texttt{ARGUMENT: \{'Dieser Vorschlag ersetzt B10 durch A75, was möglicherweise eine bessere Balance zwischen Aufwand und Wichtigkeit bietet', 'Der Gesamtaufwand bleibt innerhalb des zulässigen Rahmens', 'Ich hoffe, dass diese Kombination eine gute Lösung für beide Seiten darstellt'\}} \\
            }
        }
    }
     \\ \\

    \theutterance \stepcounter{utterance}  
    & & \multicolumn{4}{p{0.6\linewidth}}{
        \cellcolor[rgb]{0.9,0.9,0.9}{
            \makecell[{{p{\linewidth}}}]{
                \texttt{\tiny{[P1$\langle$GM]}}
                \texttt{ARGUMENT: \{'Ich denke, wir kommen langsam näher an eine Einigung, aber ich bin noch nicht vollständig zufrieden', 'Einige der Gegenstände in diesem Vorschlag haben einen erheblichen Aufwand, und ich möchte sicherstellen, dass sie auch entsprechend wichtig sind', 'Ich frage mich, ob es möglich ist, den Gegenstand B10 durch einen anderen zu ersetzen, der möglicherweise eine bessere Balance bietet'\}} \\
\texttt{VORSCHLAG: \{'C28', 'C48', 'A93', 'A75'\}} \\
\texttt{ARGUMENT: \{'Dieser Vorschlag ersetzt B10 durch A75, was möglicherweise eine bessere Balance zwischen Aufwand und Wichtigkeit bietet', 'Der Gesamtaufwand bleibt innerhalb des zulässigen Rahmens', 'Ich hoffe, dass diese Kombination eine gute Lösung für beide Seiten darstellt'\}} \\
            }
        }
    }
    & \\ \\

    \theutterance \stepcounter{utterance}  
    & \multicolumn{4}{p{0.6\linewidth}}{
        \cellcolor[rgb]{0.9,0.9,0.9}{
            \makecell[{{p{\linewidth}}}]{
                \texttt{\tiny{[P1$\rangle$GM]}}
                \texttt{ARGUMENT: \{'Ich denke, der Vorschlag, B10 durch A75 zu ersetzen, ist interessant und könnte tatsächlich eine bessere Balance bieten', 'Allerdings möchte ich auch sicherstellen, dass die Gesamtwichtigkeit der ausgewählten Gegenstände maximiert wird', 'Ich frage mich, ob es sinnvoll wäre, noch einmal die Wichtigkeit und den Aufwand der einzelnen Gegenstände zu bewerten, um sicherzustellen, dass wir die beste Kombination gefunden haben'\}} \\
\texttt{VORSCHLAG: \{'C28', 'C48', 'A93', 'C15'\}} \\
\texttt{ARGUMENT: \{'Dieser Vorschlag ersetzt A75 durch C15, was möglicherweise eine bessere Kombination aus Wichtigkeit und Aufwand bietet', 'Der Gesamtaufwand bleibt innerhalb des zulässigen Rahmens', 'Ich hoffe, dass diese Kombination eine gute Lösung für beide Seiten darstellt und unsere Bedürfnisse besser erfüllt'\}} \\
            }
        }
    }
    & & \\ \\

    \theutterance \stepcounter{utterance}  
    & & \multicolumn{4}{p{0.6\linewidth}}{
        \cellcolor[rgb]{0.9,0.9,0.9}{
            \makecell[{{p{\linewidth}}}]{
                \texttt{\tiny{[GM$\rangle$P2]}}
                \texttt{ARGUMENT: \{'Ich denke, der Vorschlag, B10 durch A75 zu ersetzen, ist interessant und könnte tatsächlich eine bessere Balance bieten', 'Allerdings möchte ich auch sicherstellen, dass die Gesamtwichtigkeit der ausgewählten Gegenstände maximiert wird', 'Ich frage mich, ob es sinnvoll wäre, noch einmal die Wichtigkeit und den Aufwand der einzelnen Gegenstände zu bewerten, um sicherzustellen, dass wir die beste Kombination gefunden haben'\}} \\
\texttt{VORSCHLAG: \{'C28', 'C48', 'A93', 'C15'\}} \\
\texttt{ARGUMENT: \{'Dieser Vorschlag ersetzt A75 durch C15, was möglicherweise eine bessere Kombination aus Wichtigkeit und Aufwand bietet', 'Der Gesamtaufwand bleibt innerhalb des zulässigen Rahmens', 'Ich hoffe, dass diese Kombination eine gute Lösung für beide Seiten darstellt und unsere Bedürfnisse besser erfüllt'\}} \\
            }
        }
    }
    & \\ \\

    \theutterance \stepcounter{utterance}  
    & & & \multicolumn{4}{p{0.6\linewidth}}{
        \cellcolor[rgb]{0.9,0.9,0.9}{
            \makecell[{{p{\linewidth}}}]{
                \texttt{\tiny{[GM$\langle$P2]}}
                \texttt{ARGUMENT: \{'Ich bin beeindruckt von der Bereitschaft, alternative Kombinationen zu erkunden, und denke, dass der Vorschlag, A75 durch C15 zu ersetzen, eine interessante Option ist', 'Ich teile die Überlegung, dass der Gesamtaufwand und die Wichtigkeit der ausgewählten Gegenstände sorgfältig abgewogen werden sollten', 'Ich denke, dass der aktuelle Vorschlag eine gute Balance bietet und nah an einer optimalen Lösung zu sein scheint'\}} \\
\texttt{ZUSTIMMUNG: \{'C28', 'C48', 'A93', 'C15'\}} \\
\texttt{ARGUMENT: \{'Ich bin zu dem Schluss gekommen, dass dieser Vorschlag eine sehr gute Kombination darstellt, die sowohl den Gesamtaufwand als auch die Wichtigkeit der Gegenstände berücksichtigt', 'Ich denke, dass wir durch diese Kombination eine Lösung gefunden haben, die beide Seiten zufriedenstellt und unsere Bedürfnisse erfüllt', 'Ich stimme daher diesem Vorschlag zu und sehe ihn als eine faire und vernünftige Lösung an'\}} \\
            }
        }
    }
     \\ \\

    \theutterance \stepcounter{utterance}  
    & & & \multicolumn{2}{p{0.3\linewidth}}{
        \cellcolor[rgb]{0.9,0.9,0.9}{
            \makecell[{{p{\linewidth}}}]{
                \texttt{\tiny{[GM$|$GM]}}
                \texttt{game successful} \\
            }
        }
    }
    & & \\ \\

    \theutterance \stepcounter{utterance}  
    & & & \multicolumn{2}{p{0.3\linewidth}}{
        \cellcolor[rgb]{0.9,0.9,0.9}{
            \makecell[{{p{\linewidth}}}]{
                \texttt{\tiny{[GM$|$GM]}}
                \texttt{end game} \\
            }
        }
    }
    & & \\ \\

\end{supertabular}
}

\end{document}
