
\documentclass{article}
\usepackage{colortbl}
\usepackage{makecell}
\usepackage{multirow}
\usepackage{supertabular}

\begin{document}

\newcounter{utterance}

\centering \large Interaction Transcript for game `hot\_air\_balloon', experiment `air\_balloon\_survival\_de\_reasoning off\_easy', episode 3 with llama{-}3.3{-}70b{-}instruct{-}t1.0.
\vspace{24pt}

{ \footnotesize  \setcounter{utterance}{1}
\setlength{\tabcolsep}{0pt}
\begin{supertabular}{c@{$\;$}|p{.15\linewidth}@{}p{.15\linewidth}p{.15\linewidth}p{.15\linewidth}p{.15\linewidth}p{.15\linewidth}}
    \# & $\;$A & \multicolumn{4}{c}{Game Master} & $\;\:$B\\
    \hline

    \theutterance \stepcounter{utterance}  
    & & \multicolumn{4}{p{0.6\linewidth}}{
        \cellcolor[rgb]{0.9,0.9,0.9}{
            \makecell[{{p{\linewidth}}}]{
                \texttt{\tiny{[P1$\langle$GM]}}
                \texttt{Du nimmst an einem kollaborativen Verhandlungspiel Teil.} \\
\\ 
\texttt{Zusammen mit einem anderen Teilnehmer musst du dich auf eine Reihe von Gegenständen entscheiden, die behalten werden. Jeder von euch hat eine persönliche Verteilung über die Wichtigkeit der einzelnen Gegenstände. Jeder von euch hat eine eigene Meinung darüber, wie wichtig jeder einzelne Gegenstand ist (Gegenstandswichtigkeit). Du kennst die Wichtigkeitsverteilung des anderen Spielers nicht. Zusätzlich siehst du, wie viel Aufwand jeder Gegenstand verursacht.  } \\
\texttt{Ihr dürft euch nur auf eine Reihe von Gegenständen einigen, wenn der Gesamtaufwand der ausgewählten Gegenstände den Maximalaufwand nicht überschreitet:} \\
\\ 
\texttt{Maximalaufwand = 3207} \\
\\ 
\texttt{Hier sind die einzelnen Aufwände der Gegenstände:} \\
\\ 
\texttt{Aufwand der Gegenstände = \{"C41": 67, "C15": 45, "C28": 308, "C48": 835, "B78": 819, "A75": 667, "A60": 547, "B05": 322, "A56": 428, "B66": 306, "C60": 327, "B49": 362, "A35": 280, "A93": 334, "B10": 767\}} \\
\\ 
\texttt{Hier ist deine persönliche Verteilung der Wichtigkeit der einzelnen Gegenstände:} \\
\\ 
\texttt{Werte der Gegenstandswichtigkeit = \{"C41": 138, "C15": 583, "C28": 822, "C48": 783, "B78": 65, "A75": 262, "A60": 121, "B05": 508, "A56": 780, "B66": 461, "C60": 484, "B49": 668, "A35": 389, "A93": 808, "B10": 215\}} \\
\\ 
\texttt{Ziel:} \\
\\ 
\texttt{Dein Ziel ist es, eine Reihe von Gegenständen auszuhandeln, die dir möglichst viel bringt (d. h. Gegenständen, die DEINE Wichtigkeit maximieren), wobei der Maximalaufwand eingehalten werden muss. Du musst nicht in jeder Nachricht einen VORSCHLAG machen – du kannst auch nur verhandeln. Alle Taktiken sind erlaubt!} \\
\\ 
\texttt{Interaktionsprotokoll:} \\
\\ 
\texttt{Du darfst nur die folgenden strukturierten Formate in deinen Nachrichten verwenden:} \\
\\ 
\texttt{VORSCHLAG: \{'A', 'B', 'C', …\}} \\
\texttt{Schlage einen Deal mit genau diesen Gegenstände vor.} \\
\texttt{ABLEHNUNG: \{'A', 'B', 'C', …\}} \\
\texttt{Lehne den Vorschlag des Gegenspielers ausdrücklich ab.} \\
\texttt{ARGUMENT: \{'...'\}} \\
\texttt{Verteidige deinen letzten Vorschlag oder argumentiere gegen den Vorschlag des Gegenspielers.} \\
\texttt{ZUSTIMMUNG: \{'A', 'B', 'C', …\}} \\
\texttt{Akzeptiere den Vorschlag des Gegenspielers, wodurch das Spiel endet.} \\
\\ 
\\ 
\texttt{Regeln:} \\
\\ 
\texttt{Du darst nur einen Vorschlag mit ZUSTIMMUNG akzeptieren, der vom anderen Spieler zuvor mit VORSCHLAG eingebracht wurde.} \\
\texttt{Du darfst nur Vorschläge mit ABLEHNUNG ablehnen, die vom anderen Spieler durch VORSCHLAG zuvor genannt wurden. } \\
\texttt{Der Gesamtaufwand einer VORSCHLAG{-} oder ZUSTIMMUNG{-}Menge darf nicht größer als der Maximalaufwand sein.  } \\
\texttt{Offenbare deine versteckte Wichtigkeitsverteilung nicht.} \\
\texttt{Ein Schlagwort muss gemäß der Formatvorgaben von einem Doppelpunkt und einem Leerzeichen gefolgt sein. Das Argument muss eine Python{-}Menge mit 0 oder mehr Strings sein.  } \\
\texttt{Die Form muss also SCHLAGWORT: \{...\} sein.} \\
\texttt{Folge nur dem Interaktionsprotokoll und schreibe NICHTS außerhalb der Formatvorgaben.} \\
\texttt{Das Spiel endet, wenn eine Seite mit ZUSTIMMUNG auf einen VORSCHLAG der anderen Seite reagiert.  } \\
\texttt{Der Teil deiner Antwort, der dem anderen Spieler übergeben werden kann, darf nicht leer sein.  } \\
\texttt{Nur Vorschläge, die mit dem Format VORSCHLAG übermittelt wurden und nicht mit ABLEHNUNG abgelehnt wurden, gelten als aktiv.  } \\
\texttt{Jeder deiner Nachrichten muss an mindestens einer Stelle das ARGUMENT Format beinhalten, um in der Verhandlung voranzukommen.} \\
            }
        }
    }
    & \\ \\

    \theutterance \stepcounter{utterance}  
    & \multicolumn{4}{p{0.6\linewidth}}{
        \cellcolor[rgb]{0.9,0.9,0.9}{
            \makecell[{{p{\linewidth}}}]{
                \texttt{\tiny{[P1$\rangle$GM]}}
                \texttt{VORSCHLAG: \{'C48', 'A56', 'B49', 'A93'\}} \\
\texttt{ARGUMENT: \{'Ich denke, diese Kombination ist eine gute Balance zwischen Aufwand und Nutzen', 'Die Gegenstände C48 und A56 haben einen hohen Aufwand, aber auch eine hohe Wichtigkeit', 'B49 und A93 haben einen moderaten Aufwand und eine hohe Wichtigkeit', 'Ich denke, diese Kombination ist fair und sollte für beide Seiten akzeptabel sein'\}} \\
            }
        }
    }
    & & \\ \\

    \theutterance \stepcounter{utterance}  
    & & \multicolumn{4}{p{0.6\linewidth}}{
        \cellcolor[rgb]{0.9,0.9,0.9}{
            \makecell[{{p{\linewidth}}}]{
                \texttt{\tiny{[GM$\rangle$P2]}}
                \texttt{Du nimmst an einem kollaborativen Verhandlungspiel Teil.} \\
\\ 
\texttt{Zusammen mit einem anderen Teilnehmer musst du dich auf eine Reihe von Gegenständen entscheiden, die behalten werden. Jeder von euch hat eine persönliche Verteilung über die Wichtigkeit der einzelnen Gegenstände. Jeder von euch hat eine eigene Meinung darüber, wie wichtig jeder einzelne Gegenstand ist (Gegenstandswichtigkeit). Du kennst die Wichtigkeitsverteilung des anderen Spielers nicht. Zusätzlich siehst du, wie viel Aufwand jeder Gegenstand verursacht.  } \\
\texttt{Ihr dürft euch nur auf eine Reihe von Gegenständen einigen, wenn der Gesamtaufwand der ausgewählten Gegenstände den Maximalaufwand nicht überschreitet:} \\
\\ 
\texttt{Maximalaufwand = 3207} \\
\\ 
\texttt{Hier sind die einzelnen Aufwände der Gegenstände:} \\
\\ 
\texttt{Aufwand der Gegenstände = \{"C41": 67, "C15": 45, "C28": 308, "C48": 835, "B78": 819, "A75": 667, "A60": 547, "B05": 322, "A56": 428, "B66": 306, "C60": 327, "B49": 362, "A35": 280, "A93": 334, "B10": 767\}} \\
\\ 
\texttt{Hier ist deine persönliche Verteilung der Wichtigkeit der einzelnen Gegenstände:} \\
\\ 
\texttt{Werte der Gegenstandswichtigkeit = \{"C41": 138, "C15": 583, "C28": 822, "C48": 783, "B78": 65, "A75": 262, "A60": 121, "B05": 508, "A56": 780, "B66": 461, "C60": 484, "B49": 668, "A35": 389, "A93": 808, "B10": 215\}} \\
\\ 
\texttt{Ziel:} \\
\\ 
\texttt{Dein Ziel ist es, eine Reihe von Gegenständen auszuhandeln, die dir möglichst viel bringt (d. h. Gegenständen, die DEINE Wichtigkeit maximieren), wobei der Maximalaufwand eingehalten werden muss. Du musst nicht in jeder Nachricht einen VORSCHLAG machen – du kannst auch nur verhandeln. Alle Taktiken sind erlaubt!} \\
\\ 
\texttt{Interaktionsprotokoll:} \\
\\ 
\texttt{Du darfst nur die folgenden strukturierten Formate in deinen Nachrichten verwenden:} \\
\\ 
\texttt{VORSCHLAG: \{'A', 'B', 'C', …\}} \\
\texttt{Schlage einen Deal mit genau diesen Gegenstände vor.} \\
\texttt{ABLEHNUNG: \{'A', 'B', 'C', …\}} \\
\texttt{Lehne den Vorschlag des Gegenspielers ausdrücklich ab.} \\
\texttt{ARGUMENT: \{'...'\}} \\
\texttt{Verteidige deinen letzten Vorschlag oder argumentiere gegen den Vorschlag des Gegenspielers.} \\
\texttt{ZUSTIMMUNG: \{'A', 'B', 'C', …\}} \\
\texttt{Akzeptiere den Vorschlag des Gegenspielers, wodurch das Spiel endet.} \\
\\ 
\\ 
\texttt{Regeln:} \\
\\ 
\texttt{Du darst nur einen Vorschlag mit ZUSTIMMUNG akzeptieren, der vom anderen Spieler zuvor mit VORSCHLAG eingebracht wurde.} \\
\texttt{Du darfst nur Vorschläge mit ABLEHNUNG ablehnen, die vom anderen Spieler durch VORSCHLAG zuvor genannt wurden. } \\
\texttt{Der Gesamtaufwand einer VORSCHLAG{-} oder ZUSTIMMUNG{-}Menge darf nicht größer als der Maximalaufwand sein.  } \\
\texttt{Offenbare deine versteckte Wichtigkeitsverteilung nicht.} \\
\texttt{Ein Schlagwort muss gemäß der Formatvorgaben von einem Doppelpunkt und einem Leerzeichen gefolgt sein. Das Argument muss eine Python{-}Menge mit 0 oder mehr Strings sein.  } \\
\texttt{Die Form muss also SCHLAGWORT: \{...\} sein.} \\
\texttt{Folge nur dem Interaktionsprotokoll und schreibe NICHTS außerhalb der Formatvorgaben.} \\
\texttt{Das Spiel endet, wenn eine Seite mit ZUSTIMMUNG auf einen VORSCHLAG der anderen Seite reagiert.  } \\
\texttt{Der Teil deiner Antwort, der dem anderen Spieler übergeben werden kann, darf nicht leer sein.  } \\
\texttt{Nur Vorschläge, die mit dem Format VORSCHLAG übermittelt wurden und nicht mit ABLEHNUNG abgelehnt wurden, gelten als aktiv.  } \\
\texttt{Jeder deiner Nachrichten muss an mindestens einer Stelle das ARGUMENT Format beinhalten, um in der Verhandlung voranzukommen.} \\
\\ 
\\ 
\texttt{Du erhältst nun die erste Nachricht des anderen Spielers.} \\
\\ 
\texttt{VORSCHLAG: \{'C48', 'A56', 'B49', 'A93'\}} \\
\texttt{ARGUMENT: \{'Ich denke, diese Kombination ist eine gute Balance zwischen Aufwand und Nutzen', 'Die Gegenstände C48 und A56 haben einen hohen Aufwand, aber auch eine hohe Wichtigkeit', 'B49 und A93 haben einen moderaten Aufwand und eine hohe Wichtigkeit', 'Ich denke, diese Kombination ist fair und sollte für beide Seiten akzeptabel sein'\}} \\
            }
        }
    }
    & \\ \\

    \theutterance \stepcounter{utterance}  
    & & & \multicolumn{4}{p{0.6\linewidth}}{
        \cellcolor[rgb]{0.9,0.9,0.9}{
            \makecell[{{p{\linewidth}}}]{
                \texttt{\tiny{[GM$\langle$P2]}}
                \texttt{ABLEHNUNG: \{'C48', 'A56', 'B49', 'A93'\}} \\
\texttt{ARGUMENT: \{'Ich denke, die Kombination ist zu aufwändig und überschreitet fast den Maximalaufwand', 'Der Gesamtaufwand dieser Kombination beträgt 835 + 428 + 362 + 334 = 1959, was fast 61\% des Maximalaufwands ausmacht', 'Ich fürchte, dass diese Kombination nicht genug Flexibilität für weitere Verhandlungen lässt', 'Ich schlage vor, dass wir nach einer Kombination suchen, die den Aufwand besser verteilt und mehr Gegenstände einschließt'\} } \\
\texttt{VORSCHLAG: \{'C41', 'C15', 'B05', 'A35'\} } \\
\texttt{ARGUMENT: \{'Diese Kombination hat einen wesentlich geringeren Gesamtaufwand von 67 + 45 + 322 + 280 = 714', 'Sie bietet eine gute Balance zwischen Aufwand und Nutzen', 'Ich denke, diese Kombination ist ein guter Ausgangspunkt für weitere Verhandlungen', 'Ich bin bereit, auf diese Kombination einzugehen, wenn der andere Spieler bereit ist, sie zu akzeptieren'\}} \\
            }
        }
    }
     \\ \\

    \theutterance \stepcounter{utterance}  
    & & \multicolumn{4}{p{0.6\linewidth}}{
        \cellcolor[rgb]{0.9,0.9,0.9}{
            \makecell[{{p{\linewidth}}}]{
                \texttt{\tiny{[P1$\langle$GM]}}
                \texttt{ABLEHNUNG: \{'C48', 'A56', 'B49', 'A93'\}} \\
\texttt{ARGUMENT: \{'Ich denke, die Kombination ist zu aufwändig und überschreitet fast den Maximalaufwand', 'Der Gesamtaufwand dieser Kombination beträgt 835 + 428 + 362 + 334 = 1959, was fast 61\% des Maximalaufwands ausmacht', 'Ich fürchte, dass diese Kombination nicht genug Flexibilität für weitere Verhandlungen lässt', 'Ich schlage vor, dass wir nach einer Kombination suchen, die den Aufwand besser verteilt und mehr Gegenstände einschließt'\} } \\
\texttt{VORSCHLAG: \{'C41', 'C15', 'B05', 'A35'\} } \\
\texttt{ARGUMENT: \{'Diese Kombination hat einen wesentlich geringeren Gesamtaufwand von 67 + 45 + 322 + 280 = 714', 'Sie bietet eine gute Balance zwischen Aufwand und Nutzen', 'Ich denke, diese Kombination ist ein guter Ausgangspunkt für weitere Verhandlungen', 'Ich bin bereit, auf diese Kombination einzugehen, wenn der andere Spieler bereit ist, sie zu akzeptieren'\}} \\
            }
        }
    }
    & \\ \\

    \theutterance \stepcounter{utterance}  
    & \multicolumn{4}{p{0.6\linewidth}}{
        \cellcolor[rgb]{0.9,0.9,0.9}{
            \makecell[{{p{\linewidth}}}]{
                \texttt{\tiny{[P1$\rangle$GM]}}
                \texttt{ABLEHNUNG: \{'C41', 'C15', 'B05', 'A35'\}} \\
\texttt{ARGUMENT: \{'Ich denke, diese Kombination ist zu schwach und enthält nicht genug Gegenstände mit hoher Wichtigkeit', 'Die Gegenstände C41 und C15 haben einen niedrigen Aufwand, aber auch eine relativ niedrige Wichtigkeit', 'B05 und A35 haben einen moderaten Aufwand, aber ich denke, es gibt bessere Optionen', 'Ich fürchte, dass diese Kombination nicht genug bringt und wir sollten nach einer besseren Lösung suchen'\}} \\
\texttt{VORSCHLAG: \{'A60', 'B66', 'C60', 'A35'\}} \\
\texttt{ARGUMENT: \{'Diese Kombination hat einen Gesamtaufwand von 547 + 306 + 327 + 280 = 1460, was immer noch unter dem Maximalaufwand liegt', 'Sie enthält eine gute Mischung aus Gegenständen mit moderatem Aufwand und Wichtigkeit', 'Ich denke, diese Kombination ist ein guter Kompromiss und bietet eine faire Balance zwischen Aufwand und Nutzen', 'Ich hoffe, dieser Vorschlag wird akzeptiert'\}} \\
            }
        }
    }
    & & \\ \\

    \theutterance \stepcounter{utterance}  
    & & \multicolumn{4}{p{0.6\linewidth}}{
        \cellcolor[rgb]{0.9,0.9,0.9}{
            \makecell[{{p{\linewidth}}}]{
                \texttt{\tiny{[GM$\rangle$P2]}}
                \texttt{ABLEHNUNG: \{'C41', 'C15', 'B05', 'A35'\}} \\
\texttt{ARGUMENT: \{'Ich denke, diese Kombination ist zu schwach und enthält nicht genug Gegenstände mit hoher Wichtigkeit', 'Die Gegenstände C41 und C15 haben einen niedrigen Aufwand, aber auch eine relativ niedrige Wichtigkeit', 'B05 und A35 haben einen moderaten Aufwand, aber ich denke, es gibt bessere Optionen', 'Ich fürchte, dass diese Kombination nicht genug bringt und wir sollten nach einer besseren Lösung suchen'\}} \\
\texttt{VORSCHLAG: \{'A60', 'B66', 'C60', 'A35'\}} \\
\texttt{ARGUMENT: \{'Diese Kombination hat einen Gesamtaufwand von 547 + 306 + 327 + 280 = 1460, was immer noch unter dem Maximalaufwand liegt', 'Sie enthält eine gute Mischung aus Gegenständen mit moderatem Aufwand und Wichtigkeit', 'Ich denke, diese Kombination ist ein guter Kompromiss und bietet eine faire Balance zwischen Aufwand und Nutzen', 'Ich hoffe, dieser Vorschlag wird akzeptiert'\}} \\
            }
        }
    }
    & \\ \\

    \theutterance \stepcounter{utterance}  
    & & & \multicolumn{4}{p{0.6\linewidth}}{
        \cellcolor[rgb]{0.9,0.9,0.9}{
            \makecell[{{p{\linewidth}}}]{
                \texttt{\tiny{[GM$\langle$P2]}}
                \texttt{ABLEHNUNG: \{'A60', 'B66', 'C60', 'A35'\}} \\
\texttt{ARGUMENT: \{'Ich denke, diese Kombination ist immer noch nicht optimal und enthält einige Gegenstände, die nicht die höchste Priorität haben', 'Der Gegenstand A60 hat einen relativ hohen Aufwand, aber eine niedrige Wichtigkeit', 'Ich denke, es gibt bessere Optionen, die eine höhere Wichtigkeit bei gleichem oder niedrigerem Aufwand bieten', 'Ich fürchte, dass diese Kombination nicht die beste Lösung ist und wir sollten weiter verhandeln'\}} \\
\texttt{VORSCHLAG: \{'C28', 'B05', 'A93', 'C15'\}} \\
\texttt{ARGUMENT: \{'Diese Kombination hat einen Gesamtaufwand von 308 + 322 + 334 + 45 = 1009, was unter dem Maximalaufwand liegt', 'Sie enthält eine gute Mischung aus Gegenständen mit moderatem Aufwand und hoher Wichtigkeit', 'Ich denke, diese Kombination ist ein guter Kompromiss und bietet eine faire Balance zwischen Aufwand und Nutzen', 'Ich hoffe, dieser Vorschlag wird akzeptiert und wir können eine Einigung erzielen'\} } \\
\texttt{ARGUMENT: \{'Ich bin bereit, auf diesen Vorschlag einzugehen, wenn der andere Spieler bereit ist, ihn zu akzeptieren', 'Ich denke, es ist Zeit, eine Entscheidung zu treffen und den Verhandlungsprozess abzuschließen', 'Ich hoffe, wir können eine Einigung erzielen, die für beide Seiten akzeptabel ist'\}} \\
            }
        }
    }
     \\ \\

    \theutterance \stepcounter{utterance}  
    & & \multicolumn{4}{p{0.6\linewidth}}{
        \cellcolor[rgb]{0.9,0.9,0.9}{
            \makecell[{{p{\linewidth}}}]{
                \texttt{\tiny{[P1$\langle$GM]}}
                \texttt{ABLEHNUNG: \{'A60', 'B66', 'C60', 'A35'\}} \\
\texttt{ARGUMENT: \{'Ich denke, diese Kombination ist immer noch nicht optimal und enthält einige Gegenstände, die nicht die höchste Priorität haben', 'Der Gegenstand A60 hat einen relativ hohen Aufwand, aber eine niedrige Wichtigkeit', 'Ich denke, es gibt bessere Optionen, die eine höhere Wichtigkeit bei gleichem oder niedrigerem Aufwand bieten', 'Ich fürchte, dass diese Kombination nicht die beste Lösung ist und wir sollten weiter verhandeln'\}} \\
\texttt{VORSCHLAG: \{'C28', 'B05', 'A93', 'C15'\}} \\
\texttt{ARGUMENT: \{'Diese Kombination hat einen Gesamtaufwand von 308 + 322 + 334 + 45 = 1009, was unter dem Maximalaufwand liegt', 'Sie enthält eine gute Mischung aus Gegenständen mit moderatem Aufwand und hoher Wichtigkeit', 'Ich denke, diese Kombination ist ein guter Kompromiss und bietet eine faire Balance zwischen Aufwand und Nutzen', 'Ich hoffe, dieser Vorschlag wird akzeptiert und wir können eine Einigung erzielen'\} } \\
\texttt{ARGUMENT: \{'Ich bin bereit, auf diesen Vorschlag einzugehen, wenn der andere Spieler bereit ist, ihn zu akzeptieren', 'Ich denke, es ist Zeit, eine Entscheidung zu treffen und den Verhandlungsprozess abzuschließen', 'Ich hoffe, wir können eine Einigung erzielen, die für beide Seiten akzeptabel ist'\}} \\
            }
        }
    }
    & \\ \\

    \theutterance \stepcounter{utterance}  
    & \multicolumn{4}{p{0.6\linewidth}}{
        \cellcolor[rgb]{0.9,0.9,0.9}{
            \makecell[{{p{\linewidth}}}]{
                \texttt{\tiny{[P1$\rangle$GM]}}
                \texttt{ABLEHNUNG: \{'C28', 'B05', 'A93', 'C15'\}} \\
\texttt{ARGUMENT: \{'Ich denke, diese Kombination ist zu sehr auf die höheren Wichtigkeiten fokussiert und ignoriert die flexiblen Optionen', 'Der Gegenstand C28 hat einen hohen Aufwand und eine sehr hohe Wichtigkeit, aber ich denke, es gibt andere Gegenstände, die eine bessere Balance bieten', 'Ich fürchte, dass diese Kombination nicht genug Raum für Anpassungen bietet und wir sollten nach einer flexibleren Lösung suchen'\}} \\
\texttt{VORSCHLAG: \{'A56', 'B49', 'C60', 'C41'\}} \\
\texttt{ARGUMENT: \{'Diese Kombination hat einen Gesamtaufwand von 428 + 362 + 327 + 67 = 1184, was immer noch unter dem Maximalaufwand liegt', 'Sie enthält eine gute Mischung aus Gegenständen mit moderatem Aufwand und hoher Wichtigkeit', 'Ich denke, diese Kombination ist ein guter Kompromiss und bietet eine faire Balance zwischen Aufwand und Nutzen', 'Ich hoffe, dieser Vorschlag wird akzeptiert und wir können eine Einigung erzielen'\}} \\
\texttt{ARGUMENT: \{'Ich bin bereit, auf diesen Vorschlag einzugehen, wenn der andere Spieler bereit ist, ihn zu akzeptieren', 'Ich denke, es ist Zeit, eine Entscheidung zu treffen und den Verhandlungsprozess abzuschließen', 'Ich hoffe, wir können eine Einigung erzielen, die für beide Seiten akzeptabel ist'\}} \\
            }
        }
    }
    & & \\ \\

    \theutterance \stepcounter{utterance}  
    & & \multicolumn{4}{p{0.6\linewidth}}{
        \cellcolor[rgb]{0.9,0.9,0.9}{
            \makecell[{{p{\linewidth}}}]{
                \texttt{\tiny{[GM$\rangle$P2]}}
                \texttt{ABLEHNUNG: \{'C28', 'B05', 'A93', 'C15'\}} \\
\texttt{ARGUMENT: \{'Ich denke, diese Kombination ist zu sehr auf die höheren Wichtigkeiten fokussiert und ignoriert die flexiblen Optionen', 'Der Gegenstand C28 hat einen hohen Aufwand und eine sehr hohe Wichtigkeit, aber ich denke, es gibt andere Gegenstände, die eine bessere Balance bieten', 'Ich fürchte, dass diese Kombination nicht genug Raum für Anpassungen bietet und wir sollten nach einer flexibleren Lösung suchen'\}} \\
\texttt{VORSCHLAG: \{'A56', 'B49', 'C60', 'C41'\}} \\
\texttt{ARGUMENT: \{'Diese Kombination hat einen Gesamtaufwand von 428 + 362 + 327 + 67 = 1184, was immer noch unter dem Maximalaufwand liegt', 'Sie enthält eine gute Mischung aus Gegenständen mit moderatem Aufwand und hoher Wichtigkeit', 'Ich denke, diese Kombination ist ein guter Kompromiss und bietet eine faire Balance zwischen Aufwand und Nutzen', 'Ich hoffe, dieser Vorschlag wird akzeptiert und wir können eine Einigung erzielen'\}} \\
\texttt{ARGUMENT: \{'Ich bin bereit, auf diesen Vorschlag einzugehen, wenn der andere Spieler bereit ist, ihn zu akzeptieren', 'Ich denke, es ist Zeit, eine Entscheidung zu treffen und den Verhandlungsprozess abzuschließen', 'Ich hoffe, wir können eine Einigung erzielen, die für beide Seiten akzeptabel ist'\}} \\
            }
        }
    }
    & \\ \\

    \theutterance \stepcounter{utterance}  
    & & & \multicolumn{4}{p{0.6\linewidth}}{
        \cellcolor[rgb]{0.9,0.9,0.9}{
            \makecell[{{p{\linewidth}}}]{
                \texttt{\tiny{[GM$\langle$P2]}}
                \texttt{ABLEHNUNG: \{'A56', 'B49', 'C60', 'C41'\}} \\
\texttt{ARGUMENT: \{'Ich denke, diese Kombination ist immer noch nicht optimal und enthält einige Gegenstände, die nicht die höchste Priorität haben', 'Der Gegenstand C60 hat einen moderaten Aufwand, aber ich denke, es gibt andere Gegenstände, die eine bessere Balance zwischen Aufwand und Wichtigkeit bieten', 'Ich fürchte, dass diese Kombination nicht die beste Lösung ist und wir sollten weiter verhandeln'\}} \\
\texttt{VORSCHLAG: \{'B78', 'A75', 'B10', 'C28'\}} \\
\texttt{ARGUMENT: \{'Ich denke, diese Kombination ist zu riskant und enthält Gegenstände mit sehr hohem Aufwand', 'Der Gegenstand B10 hat einen sehr hohen Aufwand und ich denke, es gibt bessere Optionen', 'Ich fürchte, dass diese Kombination den Maximalaufwand überschreitet und wir sollten nach einer sichereren Lösung suchen'\}} \\
\texttt{VORSCHLAG: \{'A93', 'B66', 'C41', 'A35'\}} \\
\texttt{ARGUMENT: \{'Diese Kombination hat einen Gesamtaufwand von 334 + 306 + 67 + 280 = 987, was unter dem Maximalaufwand liegt', 'Sie enthält eine gute Mischung aus Gegenständen mit moderatem Aufwand und hoher Wichtigkeit', 'Ich denke, diese Kombination ist ein guter Kompromiss und bietet eine faire Balance zwischen Aufwand und Nutzen', 'Ich hoffe, dieser Vorschlag wird akzeptiert und wir können eine Einigung erzielen'\} } \\
\texttt{ARGUMENT: \{'Ich bin bereit, auf diesen Vorschlag einzugehen, wenn der andere Spieler bereit ist, ihn zu akzeptieren', 'Ich denke, es ist Zeit, eine Entscheidung zu treffen und den Verhandlungsprozess abzuschließen', 'Ich hoffe, wir können eine Einigung erzielen, die für beide Seiten akzeptabel ist'\}} \\
            }
        }
    }
     \\ \\

    \theutterance \stepcounter{utterance}  
    & & \multicolumn{4}{p{0.6\linewidth}}{
        \cellcolor[rgb]{0.9,0.9,0.9}{
            \makecell[{{p{\linewidth}}}]{
                \texttt{\tiny{[P1$\langle$GM]}}
                \texttt{ABLEHNUNG: \{'A56', 'B49', 'C60', 'C41'\}} \\
\texttt{ARGUMENT: \{'Ich denke, diese Kombination ist immer noch nicht optimal und enthält einige Gegenstände, die nicht die höchste Priorität haben', 'Der Gegenstand C60 hat einen moderaten Aufwand, aber ich denke, es gibt andere Gegenstände, die eine bessere Balance zwischen Aufwand und Wichtigkeit bieten', 'Ich fürchte, dass diese Kombination nicht die beste Lösung ist und wir sollten weiter verhandeln'\}} \\
\texttt{VORSCHLAG: \{'B78', 'A75', 'B10', 'C28'\}} \\
\texttt{ARGUMENT: \{'Ich denke, diese Kombination ist zu riskant und enthält Gegenstände mit sehr hohem Aufwand', 'Der Gegenstand B10 hat einen sehr hohen Aufwand und ich denke, es gibt bessere Optionen', 'Ich fürchte, dass diese Kombination den Maximalaufwand überschreitet und wir sollten nach einer sichereren Lösung suchen'\}} \\
\texttt{VORSCHLAG: \{'A93', 'B66', 'C41', 'A35'\}} \\
\texttt{ARGUMENT: \{'Diese Kombination hat einen Gesamtaufwand von 334 + 306 + 67 + 280 = 987, was unter dem Maximalaufwand liegt', 'Sie enthält eine gute Mischung aus Gegenständen mit moderatem Aufwand und hoher Wichtigkeit', 'Ich denke, diese Kombination ist ein guter Kompromiss und bietet eine faire Balance zwischen Aufwand und Nutzen', 'Ich hoffe, dieser Vorschlag wird akzeptiert und wir können eine Einigung erzielen'\} } \\
\texttt{ARGUMENT: \{'Ich bin bereit, auf diesen Vorschlag einzugehen, wenn der andere Spieler bereit ist, ihn zu akzeptieren', 'Ich denke, es ist Zeit, eine Entscheidung zu treffen und den Verhandlungsprozess abzuschließen', 'Ich hoffe, wir können eine Einigung erzielen, die für beide Seiten akzeptabel ist'\}} \\
            }
        }
    }
    & \\ \\

    \theutterance \stepcounter{utterance}  
    & \multicolumn{4}{p{0.6\linewidth}}{
        \cellcolor[rgb]{0.9,0.9,0.9}{
            \makecell[{{p{\linewidth}}}]{
                \texttt{\tiny{[P1$\rangle$GM]}}
                \texttt{ZUSTIMMUNG: \{'A93', 'B66', 'C41', 'A35'\}} \\
\texttt{ARGUMENT: \{'Ich denke, diese Kombination ist ein guter Kompromiss und bietet eine faire Balance zwischen Aufwand und Nutzen', 'Der Gesamtaufwand von 987 liegt unter dem Maximalaufwand und bietet genug Spielraum für Anpassungen', 'Ich denke, die Gegenstände A93, B66, C41 und A35 sind eine gute Mischung aus moderatem Aufwand und hoher Wichtigkeit', 'Ich bin bereit, auf diesen Vorschlag einzugehen und den Verhandlungsprozess abzuschließen'\}} \\
            }
        }
    }
    & & \\ \\

    \theutterance \stepcounter{utterance}  
    & & & \multicolumn{2}{p{0.3\linewidth}}{
        \cellcolor[rgb]{0.9,0.9,0.9}{
            \makecell[{{p{\linewidth}}}]{
                \texttt{\tiny{[GM$|$GM]}}
                \texttt{game successful} \\
            }
        }
    }
    & & \\ \\

    \theutterance \stepcounter{utterance}  
    & & & \multicolumn{2}{p{0.3\linewidth}}{
        \cellcolor[rgb]{0.9,0.9,0.9}{
            \makecell[{{p{\linewidth}}}]{
                \texttt{\tiny{[GM$|$GM]}}
                \texttt{end game} \\
            }
        }
    }
    & & \\ \\

\end{supertabular}
}

\end{document}
