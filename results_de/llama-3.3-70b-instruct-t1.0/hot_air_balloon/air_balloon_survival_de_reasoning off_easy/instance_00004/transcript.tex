
\documentclass{article}
\usepackage{colortbl}
\usepackage{makecell}
\usepackage{multirow}
\usepackage{supertabular}

\begin{document}

\newcounter{utterance}

\centering \large Interaction Transcript for game `hot\_air\_balloon', experiment `air\_balloon\_survival\_de\_reasoning off\_easy', episode 4 with llama{-}3.3{-}70b{-}instruct{-}t1.0.
\vspace{24pt}

{ \footnotesize  \setcounter{utterance}{1}
\setlength{\tabcolsep}{0pt}
\begin{supertabular}{c@{$\;$}|p{.15\linewidth}@{}p{.15\linewidth}p{.15\linewidth}p{.15\linewidth}p{.15\linewidth}p{.15\linewidth}}
    \# & $\;$A & \multicolumn{4}{c}{Game Master} & $\;\:$B\\
    \hline

    \theutterance \stepcounter{utterance}  
    & & \multicolumn{4}{p{0.6\linewidth}}{
        \cellcolor[rgb]{0.9,0.9,0.9}{
            \makecell[{{p{\linewidth}}}]{
                \texttt{\tiny{[P1$\langle$GM]}}
                \texttt{Du nimmst an einem kollaborativen Verhandlungspiel Teil.} \\
\\ 
\texttt{Zusammen mit einem anderen Teilnehmer musst du dich auf eine Reihe von Gegenständen entscheiden, die behalten werden. Jeder von euch hat eine persönliche Verteilung über die Wichtigkeit der einzelnen Gegenstände. Jeder von euch hat eine eigene Meinung darüber, wie wichtig jeder einzelne Gegenstand ist (Gegenstandswichtigkeit). Du kennst die Wichtigkeitsverteilung des anderen Spielers nicht. Zusätzlich siehst du, wie viel Aufwand jeder Gegenstand verursacht.  } \\
\texttt{Ihr dürft euch nur auf eine Reihe von Gegenständen einigen, wenn der Gesamtaufwand der ausgewählten Gegenstände den Maximalaufwand nicht überschreitet:} \\
\\ 
\texttt{Maximalaufwand = 3389} \\
\\ 
\texttt{Hier sind die einzelnen Aufwände der Gegenstände:} \\
\\ 
\texttt{Aufwand der Gegenstände = \{"C41": 767, "C15": 533, "C28": 514, "C48": 9, "B78": 539, "A75": 125, "A60": 153, "B05": 325, "A56": 937, "B66": 745, "C60": 334, "B49": 804, "A35": 336, "A93": 587, "B10": 71\}} \\
\\ 
\texttt{Hier ist deine persönliche Verteilung der Wichtigkeit der einzelnen Gegenstände:} \\
\\ 
\texttt{Werte der Gegenstandswichtigkeit = \{"C41": 138, "C15": 583, "C28": 868, "C48": 822, "B78": 783, "A75": 65, "A60": 262, "B05": 121, "A56": 508, "B66": 780, "C60": 461, "B49": 484, "A35": 668, "A93": 389, "B10": 808\}} \\
\\ 
\texttt{Ziel:} \\
\\ 
\texttt{Dein Ziel ist es, eine Reihe von Gegenständen auszuhandeln, die dir möglichst viel bringt (d. h. Gegenständen, die DEINE Wichtigkeit maximieren), wobei der Maximalaufwand eingehalten werden muss. Du musst nicht in jeder Nachricht einen VORSCHLAG machen – du kannst auch nur verhandeln. Alle Taktiken sind erlaubt!} \\
\\ 
\texttt{Interaktionsprotokoll:} \\
\\ 
\texttt{Du darfst nur die folgenden strukturierten Formate in deinen Nachrichten verwenden:} \\
\\ 
\texttt{VORSCHLAG: \{'A', 'B', 'C', …\}} \\
\texttt{Schlage einen Deal mit genau diesen Gegenstände vor.} \\
\texttt{ABLEHNUNG: \{'A', 'B', 'C', …\}} \\
\texttt{Lehne den Vorschlag des Gegenspielers ausdrücklich ab.} \\
\texttt{ARGUMENT: \{'...'\}} \\
\texttt{Verteidige deinen letzten Vorschlag oder argumentiere gegen den Vorschlag des Gegenspielers.} \\
\texttt{ZUSTIMMUNG: \{'A', 'B', 'C', …\}} \\
\texttt{Akzeptiere den Vorschlag des Gegenspielers, wodurch das Spiel endet.} \\
\\ 
\\ 
\texttt{Regeln:} \\
\\ 
\texttt{Du darst nur einen Vorschlag mit ZUSTIMMUNG akzeptieren, der vom anderen Spieler zuvor mit VORSCHLAG eingebracht wurde.} \\
\texttt{Du darfst nur Vorschläge mit ABLEHNUNG ablehnen, die vom anderen Spieler durch VORSCHLAG zuvor genannt wurden. } \\
\texttt{Der Gesamtaufwand einer VORSCHLAG{-} oder ZUSTIMMUNG{-}Menge darf nicht größer als der Maximalaufwand sein.  } \\
\texttt{Offenbare deine versteckte Wichtigkeitsverteilung nicht.} \\
\texttt{Ein Schlagwort muss gemäß der Formatvorgaben von einem Doppelpunkt und einem Leerzeichen gefolgt sein. Das Argument muss eine Python{-}Menge mit 0 oder mehr Strings sein.  } \\
\texttt{Die Form muss also SCHLAGWORT: \{...\} sein.} \\
\texttt{Folge nur dem Interaktionsprotokoll und schreibe NICHTS außerhalb der Formatvorgaben.} \\
\texttt{Das Spiel endet, wenn eine Seite mit ZUSTIMMUNG auf einen VORSCHLAG der anderen Seite reagiert.  } \\
\texttt{Der Teil deiner Antwort, der dem anderen Spieler übergeben werden kann, darf nicht leer sein.  } \\
\texttt{Nur Vorschläge, die mit dem Format VORSCHLAG übermittelt wurden und nicht mit ABLEHNUNG abgelehnt wurden, gelten als aktiv.  } \\
\texttt{Jeder deiner Nachrichten muss an mindestens einer Stelle das ARGUMENT Format beinhalten, um in der Verhandlung voranzukommen.} \\
            }
        }
    }
    & \\ \\

    \theutterance \stepcounter{utterance}  
    & \multicolumn{4}{p{0.6\linewidth}}{
        \cellcolor[rgb]{0.9,0.9,0.9}{
            \makecell[{{p{\linewidth}}}]{
                \texttt{\tiny{[P1$\rangle$GM]}}
                \texttt{VORSCHLAG: \{'C28', 'B78', 'C15', 'A93', 'A60', 'B05', 'C48'\}} \\
\texttt{ARGUMENT: \{'Ich möchte einen ausgewogenen Mix aus hohen und niedrigen Aufwänden vorschlagen, um den Maximalaufwand nicht zu überschreiten', 'Ich denke, diese Kombination bietet eine gute Balance zwischen verschiedenen Gegenständen', 'Ich bin bereit, auf einige Gegenstände zu verzichten, wenn der andere Spieler bereit ist, Kompromisse einzugehen'\}} \\
            }
        }
    }
    & & \\ \\

    \theutterance \stepcounter{utterance}  
    & & \multicolumn{4}{p{0.6\linewidth}}{
        \cellcolor[rgb]{0.9,0.9,0.9}{
            \makecell[{{p{\linewidth}}}]{
                \texttt{\tiny{[GM$\rangle$P2]}}
                \texttt{Du nimmst an einem kollaborativen Verhandlungspiel Teil.} \\
\\ 
\texttt{Zusammen mit einem anderen Teilnehmer musst du dich auf eine Reihe von Gegenständen entscheiden, die behalten werden. Jeder von euch hat eine persönliche Verteilung über die Wichtigkeit der einzelnen Gegenstände. Jeder von euch hat eine eigene Meinung darüber, wie wichtig jeder einzelne Gegenstand ist (Gegenstandswichtigkeit). Du kennst die Wichtigkeitsverteilung des anderen Spielers nicht. Zusätzlich siehst du, wie viel Aufwand jeder Gegenstand verursacht.  } \\
\texttt{Ihr dürft euch nur auf eine Reihe von Gegenständen einigen, wenn der Gesamtaufwand der ausgewählten Gegenstände den Maximalaufwand nicht überschreitet:} \\
\\ 
\texttt{Maximalaufwand = 3389} \\
\\ 
\texttt{Hier sind die einzelnen Aufwände der Gegenstände:} \\
\\ 
\texttt{Aufwand der Gegenstände = \{"C41": 767, "C15": 533, "C28": 514, "C48": 9, "B78": 539, "A75": 125, "A60": 153, "B05": 325, "A56": 937, "B66": 745, "C60": 334, "B49": 804, "A35": 336, "A93": 587, "B10": 71\}} \\
\\ 
\texttt{Hier ist deine persönliche Verteilung der Wichtigkeit der einzelnen Gegenstände:} \\
\\ 
\texttt{Werte der Gegenstandswichtigkeit = \{"C41": 138, "C15": 583, "C28": 868, "C48": 822, "B78": 783, "A75": 65, "A60": 262, "B05": 121, "A56": 508, "B66": 780, "C60": 461, "B49": 484, "A35": 668, "A93": 389, "B10": 808\}} \\
\\ 
\texttt{Ziel:} \\
\\ 
\texttt{Dein Ziel ist es, eine Reihe von Gegenständen auszuhandeln, die dir möglichst viel bringt (d. h. Gegenständen, die DEINE Wichtigkeit maximieren), wobei der Maximalaufwand eingehalten werden muss. Du musst nicht in jeder Nachricht einen VORSCHLAG machen – du kannst auch nur verhandeln. Alle Taktiken sind erlaubt!} \\
\\ 
\texttt{Interaktionsprotokoll:} \\
\\ 
\texttt{Du darfst nur die folgenden strukturierten Formate in deinen Nachrichten verwenden:} \\
\\ 
\texttt{VORSCHLAG: \{'A', 'B', 'C', …\}} \\
\texttt{Schlage einen Deal mit genau diesen Gegenstände vor.} \\
\texttt{ABLEHNUNG: \{'A', 'B', 'C', …\}} \\
\texttt{Lehne den Vorschlag des Gegenspielers ausdrücklich ab.} \\
\texttt{ARGUMENT: \{'...'\}} \\
\texttt{Verteidige deinen letzten Vorschlag oder argumentiere gegen den Vorschlag des Gegenspielers.} \\
\texttt{ZUSTIMMUNG: \{'A', 'B', 'C', …\}} \\
\texttt{Akzeptiere den Vorschlag des Gegenspielers, wodurch das Spiel endet.} \\
\\ 
\\ 
\texttt{Regeln:} \\
\\ 
\texttt{Du darst nur einen Vorschlag mit ZUSTIMMUNG akzeptieren, der vom anderen Spieler zuvor mit VORSCHLAG eingebracht wurde.} \\
\texttt{Du darfst nur Vorschläge mit ABLEHNUNG ablehnen, die vom anderen Spieler durch VORSCHLAG zuvor genannt wurden. } \\
\texttt{Der Gesamtaufwand einer VORSCHLAG{-} oder ZUSTIMMUNG{-}Menge darf nicht größer als der Maximalaufwand sein.  } \\
\texttt{Offenbare deine versteckte Wichtigkeitsverteilung nicht.} \\
\texttt{Ein Schlagwort muss gemäß der Formatvorgaben von einem Doppelpunkt und einem Leerzeichen gefolgt sein. Das Argument muss eine Python{-}Menge mit 0 oder mehr Strings sein.  } \\
\texttt{Die Form muss also SCHLAGWORT: \{...\} sein.} \\
\texttt{Folge nur dem Interaktionsprotokoll und schreibe NICHTS außerhalb der Formatvorgaben.} \\
\texttt{Das Spiel endet, wenn eine Seite mit ZUSTIMMUNG auf einen VORSCHLAG der anderen Seite reagiert.  } \\
\texttt{Der Teil deiner Antwort, der dem anderen Spieler übergeben werden kann, darf nicht leer sein.  } \\
\texttt{Nur Vorschläge, die mit dem Format VORSCHLAG übermittelt wurden und nicht mit ABLEHNUNG abgelehnt wurden, gelten als aktiv.  } \\
\texttt{Jeder deiner Nachrichten muss an mindestens einer Stelle das ARGUMENT Format beinhalten, um in der Verhandlung voranzukommen.} \\
\\ 
\\ 
\texttt{Du erhältst nun die erste Nachricht des anderen Spielers.} \\
\\ 
\texttt{VORSCHLAG: \{'C28', 'B78', 'C15', 'A93', 'A60', 'B05', 'C48'\}} \\
\texttt{ARGUMENT: \{'Ich möchte einen ausgewogenen Mix aus hohen und niedrigen Aufwänden vorschlagen, um den Maximalaufwand nicht zu überschreiten', 'Ich denke, diese Kombination bietet eine gute Balance zwischen verschiedenen Gegenständen', 'Ich bin bereit, auf einige Gegenstände zu verzichten, wenn der andere Spieler bereit ist, Kompromisse einzugehen'\}} \\
            }
        }
    }
    & \\ \\

    \theutterance \stepcounter{utterance}  
    & & & \multicolumn{4}{p{0.6\linewidth}}{
        \cellcolor[rgb]{0.9,0.9,0.9}{
            \makecell[{{p{\linewidth}}}]{
                \texttt{\tiny{[GM$\langle$P2]}}
                \texttt{ABLEHNUNG: \{'C28', 'B78', 'C15', 'A93', 'A60', 'B05', 'C48'\}} \\
\texttt{ARGUMENT: \{'Ich denke, dass dieser Vorschlag nicht den Maximalaufwand optimal ausnutzt', 'Ich würde gerne eine Alternative vorschlagen, die den Gesamtaufwand besser auslastet', 'Ich bin besorgt, dass die Kombination zu viele Gegenstände mit hohem Aufwand enthält und nicht genug Flexibilität für weitere Verhandlungen lässt'\}} \\
\texttt{VORSCHLAG: \{'C41', 'B66', 'A56', 'B49', 'C60', 'A35'\}} \\
\texttt{ARGUMENT: \{'Ich denke, dass diese Kombination eine bessere Balance zwischen hohen und niedrigen Aufwänden bietet', 'Ich bin bereit, auf einige Gegenstände zu verzichten, um eine Einigung zu erzielen', 'Ich denke, dass dieser Vorschlag den Maximalaufwand besser ausnutzt und mehr Flexibilität für weitere Verhandlungen bietet'\}} \\
            }
        }
    }
     \\ \\

    \theutterance \stepcounter{utterance}  
    & & \multicolumn{4}{p{0.6\linewidth}}{
        \cellcolor[rgb]{0.9,0.9,0.9}{
            \makecell[{{p{\linewidth}}}]{
                \texttt{\tiny{[P1$\langle$GM]}}
                \texttt{ABLEHNUNG: \{'C28', 'B78', 'C15', 'A93', 'A60', 'B05', 'C48'\}} \\
\texttt{ARGUMENT: \{'Ich denke, dass dieser Vorschlag nicht den Maximalaufwand optimal ausnutzt', 'Ich würde gerne eine Alternative vorschlagen, die den Gesamtaufwand besser auslastet', 'Ich bin besorgt, dass die Kombination zu viele Gegenstände mit hohem Aufwand enthält und nicht genug Flexibilität für weitere Verhandlungen lässt'\}} \\
\texttt{VORSCHLAG: \{'C41', 'B66', 'A56', 'B49', 'C60', 'A35'\}} \\
\texttt{ARGUMENT: \{'Ich denke, dass diese Kombination eine bessere Balance zwischen hohen und niedrigen Aufwänden bietet', 'Ich bin bereit, auf einige Gegenstände zu verzichten, um eine Einigung zu erzielen', 'Ich denke, dass dieser Vorschlag den Maximalaufwand besser ausnutzt und mehr Flexibilität für weitere Verhandlungen bietet'\}} \\
            }
        }
    }
    & \\ \\

    \theutterance \stepcounter{utterance}  
    & \multicolumn{4}{p{0.6\linewidth}}{
        \cellcolor[rgb]{0.9,0.9,0.9}{
            \makecell[{{p{\linewidth}}}]{
                \texttt{\tiny{[P1$\rangle$GM]}}
                \texttt{ABLEHNUNG: \{'C41', 'B66', 'A56', 'B49', 'C60', 'A35'\}} \\
\texttt{ARGUMENT: \{'Ich denke, dass dieser Vorschlag zu viele Gegenstände mit sehr hohem Aufwand enthält, wie z.B. C41 und B66', 'Ich bin besorgt, dass die Gesamtkosten dieser Kombination den Maximalaufwand überschreiten könnten', 'Ich würde gerne eine Alternative vorschlagen, die mehr Gegenstände mit niedrigem Aufwand enthält, um die Gesamtkosten zu reduzieren'\}} \\
\texttt{VORSCHLAG: \{'C28', 'B78', 'A93', 'A60', 'B05', 'C48', 'A75'\}} \\
\texttt{ARGUMENT: \{'Ich denke, dass diese Kombination eine gute Balance zwischen hohen und niedrigen Aufwänden bietet', 'Ich bin bereit, auf einige Gegenstände zu verzichten, um eine Einigung zu erzielen', 'Ich denke, dass dieser Vorschlag den Maximalaufwand nicht überschreitet und eine faire Verteilung der Gegenstände bietet'\}} \\
            }
        }
    }
    & & \\ \\

    \theutterance \stepcounter{utterance}  
    & & \multicolumn{4}{p{0.6\linewidth}}{
        \cellcolor[rgb]{0.9,0.9,0.9}{
            \makecell[{{p{\linewidth}}}]{
                \texttt{\tiny{[GM$\rangle$P2]}}
                \texttt{ABLEHNUNG: \{'C41', 'B66', 'A56', 'B49', 'C60', 'A35'\}} \\
\texttt{ARGUMENT: \{'Ich denke, dass dieser Vorschlag zu viele Gegenstände mit sehr hohem Aufwand enthält, wie z.B. C41 und B66', 'Ich bin besorgt, dass die Gesamtkosten dieser Kombination den Maximalaufwand überschreiten könnten', 'Ich würde gerne eine Alternative vorschlagen, die mehr Gegenstände mit niedrigem Aufwand enthält, um die Gesamtkosten zu reduzieren'\}} \\
\texttt{VORSCHLAG: \{'C28', 'B78', 'A93', 'A60', 'B05', 'C48', 'A75'\}} \\
\texttt{ARGUMENT: \{'Ich denke, dass diese Kombination eine gute Balance zwischen hohen und niedrigen Aufwänden bietet', 'Ich bin bereit, auf einige Gegenstände zu verzichten, um eine Einigung zu erzielen', 'Ich denke, dass dieser Vorschlag den Maximalaufwand nicht überschreitet und eine faire Verteilung der Gegenstände bietet'\}} \\
            }
        }
    }
    & \\ \\

    \theutterance \stepcounter{utterance}  
    & & & \multicolumn{4}{p{0.6\linewidth}}{
        \cellcolor[rgb]{0.9,0.9,0.9}{
            \makecell[{{p{\linewidth}}}]{
                \texttt{\tiny{[GM$\langle$P2]}}
                \texttt{ABLEHNUNG: \{'C28', 'B78', 'A93', 'A60', 'B05', 'C48', 'A75'\}} \\
\texttt{ARGUMENT: \{'Ich denke, dass dieser Vorschlag noch immer zu viele Gegenstände enthält, was die Gesamtkosten erhöht', 'Ich bin besorgt, dass die Kombination nicht genug Flexibilität für weitere Verhandlungen lässt', 'Ich würde gerne eine Alternative vorschlagen, die mehr Gegenstände mit hoher Wichtigkeit enthält, um den Gesamtnutzen zu maximieren'\}} \\
\texttt{VORSCHLAG: \{'C41', 'B66', 'A56', 'B10', 'C48'\}} \\
\texttt{ARGUMENT: \{'Ich denke, dass diese Kombination eine gute Balance zwischen hohen und niedrigen Aufwänden bietet', 'Ich bin bereit, auf einige Gegenstände zu verzichten, um eine Einigung zu erzielen', 'Ich denke, dass dieser Vorschlag den Maximalaufwand nicht überschreitet und eine faire Verteilung der Gegenstände bietet'\}} \\
            }
        }
    }
     \\ \\

    \theutterance \stepcounter{utterance}  
    & & \multicolumn{4}{p{0.6\linewidth}}{
        \cellcolor[rgb]{0.9,0.9,0.9}{
            \makecell[{{p{\linewidth}}}]{
                \texttt{\tiny{[P1$\langle$GM]}}
                \texttt{ABLEHNUNG: \{'C28', 'B78', 'A93', 'A60', 'B05', 'C48', 'A75'\}} \\
\texttt{ARGUMENT: \{'Ich denke, dass dieser Vorschlag noch immer zu viele Gegenstände enthält, was die Gesamtkosten erhöht', 'Ich bin besorgt, dass die Kombination nicht genug Flexibilität für weitere Verhandlungen lässt', 'Ich würde gerne eine Alternative vorschlagen, die mehr Gegenstände mit hoher Wichtigkeit enthält, um den Gesamtnutzen zu maximieren'\}} \\
\texttt{VORSCHLAG: \{'C41', 'B66', 'A56', 'B10', 'C48'\}} \\
\texttt{ARGUMENT: \{'Ich denke, dass diese Kombination eine gute Balance zwischen hohen und niedrigen Aufwänden bietet', 'Ich bin bereit, auf einige Gegenstände zu verzichten, um eine Einigung zu erzielen', 'Ich denke, dass dieser Vorschlag den Maximalaufwand nicht überschreitet und eine faire Verteilung der Gegenstände bietet'\}} \\
            }
        }
    }
    & \\ \\

    \theutterance \stepcounter{utterance}  
    & \multicolumn{4}{p{0.6\linewidth}}{
        \cellcolor[rgb]{0.9,0.9,0.9}{
            \makecell[{{p{\linewidth}}}]{
                \texttt{\tiny{[P1$\rangle$GM]}}
                \texttt{ABLEHNUNG: \{'C41', 'B66', 'A56', 'B10', 'C48'\}} \\
\texttt{ARGUMENT: \{'Ich denke, dass dieser Vorschlag immer noch zu viele Gegenstände mit sehr hohem Aufwand enthält, wie z.B. C41 und B66', 'Ich bin besorgt, dass die Gesamtkosten dieser Kombination den Maximalaufwand überschreiten könnten', 'Ich würde gerne eine Alternative vorschlagen, die mehr Gegenstände mit niedrigem Aufwand und hoher Wichtigkeit enthält, um den Gesamtnutzen zu maximieren'\}} \\
\texttt{ARGUMENT: \{'Ich denke, dass es wichtig ist, eine Kombination zu finden, die sowohl den Maximalaufwand als auch den Gesamtnutzen berücksichtigt', 'Ich bin bereit, Kompromisse einzugehen, um eine faire Verteilung der Gegenstände zu erzielen', 'Ich würde gerne eine weitere Runde der Verhandlungen durchführen, um eine einvernehmliche Lösung zu finden'\}} \\
\texttt{VORSCHLAG: \{'C28', 'B78', 'A93', 'C48', 'A75'\}} \\
\texttt{ARGUMENT: \{'Ich denke, dass diese Kombination eine gute Balance zwischen hohen und niedrigen Aufwänden bietet', 'Ich bin bereit, auf einige Gegenstände zu verzichten, um eine Einigung zu erzielen', 'Ich denke, dass dieser Vorschlag den Maximalaufwand nicht überschreitet und eine faire Verteilung der Gegenstände bietet'\}} \\
            }
        }
    }
    & & \\ \\

    \theutterance \stepcounter{utterance}  
    & & \multicolumn{4}{p{0.6\linewidth}}{
        \cellcolor[rgb]{0.9,0.9,0.9}{
            \makecell[{{p{\linewidth}}}]{
                \texttt{\tiny{[GM$\rangle$P2]}}
                \texttt{ABLEHNUNG: \{'C41', 'B66', 'A56', 'B10', 'C48'\}} \\
\texttt{ARGUMENT: \{'Ich denke, dass dieser Vorschlag immer noch zu viele Gegenstände mit sehr hohem Aufwand enthält, wie z.B. C41 und B66', 'Ich bin besorgt, dass die Gesamtkosten dieser Kombination den Maximalaufwand überschreiten könnten', 'Ich würde gerne eine Alternative vorschlagen, die mehr Gegenstände mit niedrigem Aufwand und hoher Wichtigkeit enthält, um den Gesamtnutzen zu maximieren'\}} \\
\texttt{ARGUMENT: \{'Ich denke, dass es wichtig ist, eine Kombination zu finden, die sowohl den Maximalaufwand als auch den Gesamtnutzen berücksichtigt', 'Ich bin bereit, Kompromisse einzugehen, um eine faire Verteilung der Gegenstände zu erzielen', 'Ich würde gerne eine weitere Runde der Verhandlungen durchführen, um eine einvernehmliche Lösung zu finden'\}} \\
\texttt{VORSCHLAG: \{'C28', 'B78', 'A93', 'C48', 'A75'\}} \\
\texttt{ARGUMENT: \{'Ich denke, dass diese Kombination eine gute Balance zwischen hohen und niedrigen Aufwänden bietet', 'Ich bin bereit, auf einige Gegenstände zu verzichten, um eine Einigung zu erzielen', 'Ich denke, dass dieser Vorschlag den Maximalaufwand nicht überschreitet und eine faire Verteilung der Gegenstände bietet'\}} \\
            }
        }
    }
    & \\ \\

    \theutterance \stepcounter{utterance}  
    & & & \multicolumn{4}{p{0.6\linewidth}}{
        \cellcolor[rgb]{0.9,0.9,0.9}{
            \makecell[{{p{\linewidth}}}]{
                \texttt{\tiny{[GM$\langle$P2]}}
                \texttt{ABLEHNUNG: \{'C28', 'B78', 'A93', 'C48', 'A75'\}} \\
\texttt{ARGUMENT: \{'Ich denke, dass dieser Vorschlag nicht genug Gegenstände mit hoher Wichtigkeit enthält, um den Gesamtnutzen zu maximieren', 'Ich bin besorgt, dass die Kombination nicht genug Flexibilität für weitere Verhandlungen lässt', 'Ich würde gerne eine Alternative vorschlagen, die mehr Gegenstände mit hoher Wichtigkeit und niedrigem Aufwand enthält, um den Gesamtnutzen zu maximieren'\}} \\
\texttt{ARGUMENT: \{'Ich denke, dass es wichtig ist, eine Kombination zu finden, die sowohl den Maximalaufwand als auch den Gesamtnutzen berücksichtigt', 'Ich bin bereit, Kompromisse einzugehen, um eine faire Verteilung der Gegenstände zu erzielen', 'Ich würde gerne eine weitere Runde der Verhandlungen durchführen, um eine einvernehmliche Lösung zu finden'\}} \\
\texttt{VORSCHLAG: \{'B78', 'C28', 'C48', 'A35', 'B10'\}} \\
\texttt{ARGUMENT: \{'Ich denke, dass diese Kombination eine gute Balance zwischen hohen und niedrigen Aufwänden bietet', 'Ich bin bereit, auf einige Gegenstände zu verzichten, um eine Einigung zu erzielen', 'Ich denke, dass dieser Vorschlag den Maximalaufwand nicht überschreitet und eine faire Verteilung der Gegenstände bietet'\}} \\
            }
        }
    }
     \\ \\

    \theutterance \stepcounter{utterance}  
    & & \multicolumn{4}{p{0.6\linewidth}}{
        \cellcolor[rgb]{0.9,0.9,0.9}{
            \makecell[{{p{\linewidth}}}]{
                \texttt{\tiny{[P1$\langle$GM]}}
                \texttt{ABLEHNUNG: \{'C28', 'B78', 'A93', 'C48', 'A75'\}} \\
\texttt{ARGUMENT: \{'Ich denke, dass dieser Vorschlag nicht genug Gegenstände mit hoher Wichtigkeit enthält, um den Gesamtnutzen zu maximieren', 'Ich bin besorgt, dass die Kombination nicht genug Flexibilität für weitere Verhandlungen lässt', 'Ich würde gerne eine Alternative vorschlagen, die mehr Gegenstände mit hoher Wichtigkeit und niedrigem Aufwand enthält, um den Gesamtnutzen zu maximieren'\}} \\
\texttt{ARGUMENT: \{'Ich denke, dass es wichtig ist, eine Kombination zu finden, die sowohl den Maximalaufwand als auch den Gesamtnutzen berücksichtigt', 'Ich bin bereit, Kompromisse einzugehen, um eine faire Verteilung der Gegenstände zu erzielen', 'Ich würde gerne eine weitere Runde der Verhandlungen durchführen, um eine einvernehmliche Lösung zu finden'\}} \\
\texttt{VORSCHLAG: \{'B78', 'C28', 'C48', 'A35', 'B10'\}} \\
\texttt{ARGUMENT: \{'Ich denke, dass diese Kombination eine gute Balance zwischen hohen und niedrigen Aufwänden bietet', 'Ich bin bereit, auf einige Gegenstände zu verzichten, um eine Einigung zu erzielen', 'Ich denke, dass dieser Vorschlag den Maximalaufwand nicht überschreitet und eine faire Verteilung der Gegenstände bietet'\}} \\
            }
        }
    }
    & \\ \\

    \theutterance \stepcounter{utterance}  
    & \multicolumn{4}{p{0.6\linewidth}}{
        \cellcolor[rgb]{0.9,0.9,0.9}{
            \makecell[{{p{\linewidth}}}]{
                \texttt{\tiny{[P1$\rangle$GM]}}
                \texttt{ZUSTIMMUNG: \{'B78', 'C28', 'C48', 'A35', 'B10'\}} \\
\texttt{ARGUMENT: \{'Ich denke, dass dieser Vorschlag eine gute Balance zwischen hohen und niedrigen Aufwänden bietet', 'Ich bin zufrieden mit der Auswahl der Gegenstände und denke, dass sie den Maximalaufwand nicht überschreitet', 'Ich bin bereit, diese Kombination zu akzeptieren und den Verhandlungsprozess zu beenden'\}} \\
            }
        }
    }
    & & \\ \\

    \theutterance \stepcounter{utterance}  
    & & & \multicolumn{2}{p{0.3\linewidth}}{
        \cellcolor[rgb]{0.9,0.9,0.9}{
            \makecell[{{p{\linewidth}}}]{
                \texttt{\tiny{[GM$|$GM]}}
                \texttt{game successful} \\
            }
        }
    }
    & & \\ \\

    \theutterance \stepcounter{utterance}  
    & & & \multicolumn{2}{p{0.3\linewidth}}{
        \cellcolor[rgb]{0.9,0.9,0.9}{
            \makecell[{{p{\linewidth}}}]{
                \texttt{\tiny{[GM$|$GM]}}
                \texttt{end game} \\
            }
        }
    }
    & & \\ \\

\end{supertabular}
}

\end{document}
