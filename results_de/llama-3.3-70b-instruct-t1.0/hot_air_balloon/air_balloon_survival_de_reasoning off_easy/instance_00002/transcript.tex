
\documentclass{article}
\usepackage{colortbl}
\usepackage{makecell}
\usepackage{multirow}
\usepackage{supertabular}

\begin{document}

\newcounter{utterance}

\centering \large Interaction Transcript for game `hot\_air\_balloon', experiment `air\_balloon\_survival\_de\_reasoning off\_easy', episode 2 with llama{-}3.3{-}70b{-}instruct{-}t1.0.
\vspace{24pt}

{ \footnotesize  \setcounter{utterance}{1}
\setlength{\tabcolsep}{0pt}
\begin{supertabular}{c@{$\;$}|p{.15\linewidth}@{}p{.15\linewidth}p{.15\linewidth}p{.15\linewidth}p{.15\linewidth}p{.15\linewidth}}
    \# & $\;$A & \multicolumn{4}{c}{Game Master} & $\;\:$B\\
    \hline

    \theutterance \stepcounter{utterance}  
    & & \multicolumn{4}{p{0.6\linewidth}}{
        \cellcolor[rgb]{0.9,0.9,0.9}{
            \makecell[{{p{\linewidth}}}]{
                \texttt{\tiny{[P1$\langle$GM]}}
                \texttt{Du nimmst an einem kollaborativen Verhandlungspiel Teil.} \\
\\ 
\texttt{Zusammen mit einem anderen Teilnehmer musst du dich auf eine Reihe von Gegenständen entscheiden, die behalten werden. Jeder von euch hat eine persönliche Verteilung über die Wichtigkeit der einzelnen Gegenstände. Jeder von euch hat eine eigene Meinung darüber, wie wichtig jeder einzelne Gegenstand ist (Gegenstandswichtigkeit). Du kennst die Wichtigkeitsverteilung des anderen Spielers nicht. Zusätzlich siehst du, wie viel Aufwand jeder Gegenstand verursacht.  } \\
\texttt{Ihr dürft euch nur auf eine Reihe von Gegenständen einigen, wenn der Gesamtaufwand der ausgewählten Gegenstände den Maximalaufwand nicht überschreitet:} \\
\\ 
\texttt{Maximalaufwand = 3218} \\
\\ 
\texttt{Hier sind die einzelnen Aufwände der Gegenstände:} \\
\\ 
\texttt{Aufwand der Gegenstände = \{"C41": 919, "C15": 852, "C28": 299, "C48": 762, "B78": 161, "A75": 205, "A60": 380, "B05": 399, "A56": 534, "B66": 333, "C60": 100, "B49": 420, "A35": 354, "A93": 130, "B10": 589\}} \\
\\ 
\texttt{Hier ist deine persönliche Verteilung der Wichtigkeit der einzelnen Gegenstände:} \\
\\ 
\texttt{Werte der Gegenstandswichtigkeit = \{"C41": 138, "C15": 583, "C28": 868, "C48": 822, "B78": 783, "A75": 65, "A60": 262, "B05": 121, "A56": 508, "B66": 780, "C60": 461, "B49": 484, "A35": 668, "A93": 389, "B10": 808\}} \\
\\ 
\texttt{Ziel:} \\
\\ 
\texttt{Dein Ziel ist es, eine Reihe von Gegenständen auszuhandeln, die dir möglichst viel bringt (d. h. Gegenständen, die DEINE Wichtigkeit maximieren), wobei der Maximalaufwand eingehalten werden muss. Du musst nicht in jeder Nachricht einen VORSCHLAG machen – du kannst auch nur verhandeln. Alle Taktiken sind erlaubt!} \\
\\ 
\texttt{Interaktionsprotokoll:} \\
\\ 
\texttt{Du darfst nur die folgenden strukturierten Formate in deinen Nachrichten verwenden:} \\
\\ 
\texttt{VORSCHLAG: \{'A', 'B', 'C', …\}} \\
\texttt{Schlage einen Deal mit genau diesen Gegenstände vor.} \\
\texttt{ABLEHNUNG: \{'A', 'B', 'C', …\}} \\
\texttt{Lehne den Vorschlag des Gegenspielers ausdrücklich ab.} \\
\texttt{ARGUMENT: \{'...'\}} \\
\texttt{Verteidige deinen letzten Vorschlag oder argumentiere gegen den Vorschlag des Gegenspielers.} \\
\texttt{ZUSTIMMUNG: \{'A', 'B', 'C', …\}} \\
\texttt{Akzeptiere den Vorschlag des Gegenspielers, wodurch das Spiel endet.} \\
\\ 
\\ 
\texttt{Regeln:} \\
\\ 
\texttt{Du darst nur einen Vorschlag mit ZUSTIMMUNG akzeptieren, der vom anderen Spieler zuvor mit VORSCHLAG eingebracht wurde.} \\
\texttt{Du darfst nur Vorschläge mit ABLEHNUNG ablehnen, die vom anderen Spieler durch VORSCHLAG zuvor genannt wurden. } \\
\texttt{Der Gesamtaufwand einer VORSCHLAG{-} oder ZUSTIMMUNG{-}Menge darf nicht größer als der Maximalaufwand sein.  } \\
\texttt{Offenbare deine versteckte Wichtigkeitsverteilung nicht.} \\
\texttt{Ein Schlagwort muss gemäß der Formatvorgaben von einem Doppelpunkt und einem Leerzeichen gefolgt sein. Das Argument muss eine Python{-}Menge mit 0 oder mehr Strings sein.  } \\
\texttt{Die Form muss also SCHLAGWORT: \{...\} sein.} \\
\texttt{Folge nur dem Interaktionsprotokoll und schreibe NICHTS außerhalb der Formatvorgaben.} \\
\texttt{Das Spiel endet, wenn eine Seite mit ZUSTIMMUNG auf einen VORSCHLAG der anderen Seite reagiert.  } \\
\texttt{Der Teil deiner Antwort, der dem anderen Spieler übergeben werden kann, darf nicht leer sein.  } \\
\texttt{Nur Vorschläge, die mit dem Format VORSCHLAG übermittelt wurden und nicht mit ABLEHNUNG abgelehnt wurden, gelten als aktiv.  } \\
\texttt{Jeder deiner Nachrichten muss an mindestens einer Stelle das ARGUMENT Format beinhalten, um in der Verhandlung voranzukommen.} \\
            }
        }
    }
    & \\ \\

    \theutterance \stepcounter{utterance}  
    & \multicolumn{4}{p{0.6\linewidth}}{
        \cellcolor[rgb]{0.9,0.9,0.9}{
            \makecell[{{p{\linewidth}}}]{
                \texttt{\tiny{[P1$\rangle$GM]}}
                \texttt{VORSCHLAG: \{'C28', 'B78', 'A35', 'B66', 'C60'\}} \\
\texttt{ARGUMENT: \{'Ich denke, diese Kombination bietet eine gute Balance zwischen Aufwand und Wichtigkeit', 'Die Gegenstände C28 und B78 haben einen hohen Wert für mich', 'A35 und B66 ergänzen diese Auswahl gut', 'C60 ist einове Gegenstand mit geringem Aufwand, der noch Platz im Budget lässt'\}} \\
            }
        }
    }
    & & \\ \\

    \theutterance \stepcounter{utterance}  
    & & \multicolumn{4}{p{0.6\linewidth}}{
        \cellcolor[rgb]{0.9,0.9,0.9}{
            \makecell[{{p{\linewidth}}}]{
                \texttt{\tiny{[GM$\rangle$P2]}}
                \texttt{Du nimmst an einem kollaborativen Verhandlungspiel Teil.} \\
\\ 
\texttt{Zusammen mit einem anderen Teilnehmer musst du dich auf eine Reihe von Gegenständen entscheiden, die behalten werden. Jeder von euch hat eine persönliche Verteilung über die Wichtigkeit der einzelnen Gegenstände. Jeder von euch hat eine eigene Meinung darüber, wie wichtig jeder einzelne Gegenstand ist (Gegenstandswichtigkeit). Du kennst die Wichtigkeitsverteilung des anderen Spielers nicht. Zusätzlich siehst du, wie viel Aufwand jeder Gegenstand verursacht.  } \\
\texttt{Ihr dürft euch nur auf eine Reihe von Gegenständen einigen, wenn der Gesamtaufwand der ausgewählten Gegenstände den Maximalaufwand nicht überschreitet:} \\
\\ 
\texttt{Maximalaufwand = 3218} \\
\\ 
\texttt{Hier sind die einzelnen Aufwände der Gegenstände:} \\
\\ 
\texttt{Aufwand der Gegenstände = \{"C41": 919, "C15": 852, "C28": 299, "C48": 762, "B78": 161, "A75": 205, "A60": 380, "B05": 399, "A56": 534, "B66": 333, "C60": 100, "B49": 420, "A35": 354, "A93": 130, "B10": 589\}} \\
\\ 
\texttt{Hier ist deine persönliche Verteilung der Wichtigkeit der einzelnen Gegenstände:} \\
\\ 
\texttt{Werte der Gegenstandswichtigkeit = \{"C41": 138, "C15": 583, "C28": 868, "C48": 822, "B78": 783, "A75": 65, "A60": 262, "B05": 121, "A56": 508, "B66": 780, "C60": 461, "B49": 484, "A35": 668, "A93": 389, "B10": 808\}} \\
\\ 
\texttt{Ziel:} \\
\\ 
\texttt{Dein Ziel ist es, eine Reihe von Gegenständen auszuhandeln, die dir möglichst viel bringt (d. h. Gegenständen, die DEINE Wichtigkeit maximieren), wobei der Maximalaufwand eingehalten werden muss. Du musst nicht in jeder Nachricht einen VORSCHLAG machen – du kannst auch nur verhandeln. Alle Taktiken sind erlaubt!} \\
\\ 
\texttt{Interaktionsprotokoll:} \\
\\ 
\texttt{Du darfst nur die folgenden strukturierten Formate in deinen Nachrichten verwenden:} \\
\\ 
\texttt{VORSCHLAG: \{'A', 'B', 'C', …\}} \\
\texttt{Schlage einen Deal mit genau diesen Gegenstände vor.} \\
\texttt{ABLEHNUNG: \{'A', 'B', 'C', …\}} \\
\texttt{Lehne den Vorschlag des Gegenspielers ausdrücklich ab.} \\
\texttt{ARGUMENT: \{'...'\}} \\
\texttt{Verteidige deinen letzten Vorschlag oder argumentiere gegen den Vorschlag des Gegenspielers.} \\
\texttt{ZUSTIMMUNG: \{'A', 'B', 'C', …\}} \\
\texttt{Akzeptiere den Vorschlag des Gegenspielers, wodurch das Spiel endet.} \\
\\ 
\\ 
\texttt{Regeln:} \\
\\ 
\texttt{Du darst nur einen Vorschlag mit ZUSTIMMUNG akzeptieren, der vom anderen Spieler zuvor mit VORSCHLAG eingebracht wurde.} \\
\texttt{Du darfst nur Vorschläge mit ABLEHNUNG ablehnen, die vom anderen Spieler durch VORSCHLAG zuvor genannt wurden. } \\
\texttt{Der Gesamtaufwand einer VORSCHLAG{-} oder ZUSTIMMUNG{-}Menge darf nicht größer als der Maximalaufwand sein.  } \\
\texttt{Offenbare deine versteckte Wichtigkeitsverteilung nicht.} \\
\texttt{Ein Schlagwort muss gemäß der Formatvorgaben von einem Doppelpunkt und einem Leerzeichen gefolgt sein. Das Argument muss eine Python{-}Menge mit 0 oder mehr Strings sein.  } \\
\texttt{Die Form muss also SCHLAGWORT: \{...\} sein.} \\
\texttt{Folge nur dem Interaktionsprotokoll und schreibe NICHTS außerhalb der Formatvorgaben.} \\
\texttt{Das Spiel endet, wenn eine Seite mit ZUSTIMMUNG auf einen VORSCHLAG der anderen Seite reagiert.  } \\
\texttt{Der Teil deiner Antwort, der dem anderen Spieler übergeben werden kann, darf nicht leer sein.  } \\
\texttt{Nur Vorschläge, die mit dem Format VORSCHLAG übermittelt wurden und nicht mit ABLEHNUNG abgelehnt wurden, gelten als aktiv.  } \\
\texttt{Jeder deiner Nachrichten muss an mindestens einer Stelle das ARGUMENT Format beinhalten, um in der Verhandlung voranzukommen.} \\
\\ 
\\ 
\texttt{Du erhältst nun die erste Nachricht des anderen Spielers.} \\
\\ 
\texttt{VORSCHLAG: \{'C28', 'B78', 'A35', 'B66', 'C60'\}} \\
\texttt{ARGUMENT: \{'Ich denke, diese Kombination bietet eine gute Balance zwischen Aufwand und Wichtigkeit', 'Die Gegenstände C28 und B78 haben einen hohen Wert für mich', 'A35 und B66 ergänzen diese Auswahl gut', 'C60 ist einове Gegenstand mit geringem Aufwand, der noch Platz im Budget lässt'\}} \\
            }
        }
    }
    & \\ \\

    \theutterance \stepcounter{utterance}  
    & & & \multicolumn{4}{p{0.6\linewidth}}{
        \cellcolor[rgb]{0.9,0.9,0.9}{
            \makecell[{{p{\linewidth}}}]{
                \texttt{\tiny{[GM$\langle$P2]}}
                \texttt{ABLEHNUNG: \{'C28', 'B78', 'A35', 'B66', 'C60'\}} \\
\texttt{ARGUMENT: \{'Ich bin nicht überzeugt, dass diese Kombination meine Bedürfnisse abdeckt', 'Die Aufteilung des Budgets scheint nicht optimal zu sein', 'Ich denke, es gibt bessere Kombinationen, die einen höheren Wert für mich bieten', 'Der Aufwand für C28 und B78 ist hoch, ich bin mir unsicher, ob sie den Wert wirklich rechtfertigen'\}} \\
\texttt{VORSCHLAG: \{'C48', 'B10', 'A56', 'B49'\}} \\
\texttt{ARGUMENT: \{'Diese Kombination bietet eine gute Balance zwischen Aufwand und Wichtigkeit für mich', 'Die Gegenstände C48 und B10 haben einen hohen Wert', 'A56 und B49 ergänzen diese Auswahl gut und bleiben im Budget'\}} \\
            }
        }
    }
     \\ \\

    \theutterance \stepcounter{utterance}  
    & & \multicolumn{4}{p{0.6\linewidth}}{
        \cellcolor[rgb]{0.9,0.9,0.9}{
            \makecell[{{p{\linewidth}}}]{
                \texttt{\tiny{[P1$\langle$GM]}}
                \texttt{ABLEHNUNG: \{'C28', 'B78', 'A35', 'B66', 'C60'\}} \\
\texttt{ARGUMENT: \{'Ich bin nicht überzeugt, dass diese Kombination meine Bedürfnisse abdeckt', 'Die Aufteilung des Budgets scheint nicht optimal zu sein', 'Ich denke, es gibt bessere Kombinationen, die einen höheren Wert für mich bieten', 'Der Aufwand für C28 und B78 ist hoch, ich bin mir unsicher, ob sie den Wert wirklich rechtfertigen'\}} \\
\texttt{VORSCHLAG: \{'C48', 'B10', 'A56', 'B49'\}} \\
\texttt{ARGUMENT: \{'Diese Kombination bietet eine gute Balance zwischen Aufwand und Wichtigkeit für mich', 'Die Gegenstände C48 und B10 haben einen hohen Wert', 'A56 und B49 ergänzen diese Auswahl gut und bleiben im Budget'\}} \\
            }
        }
    }
    & \\ \\

    \theutterance \stepcounter{utterance}  
    & \multicolumn{4}{p{0.6\linewidth}}{
        \cellcolor[rgb]{0.9,0.9,0.9}{
            \makecell[{{p{\linewidth}}}]{
                \texttt{\tiny{[P1$\rangle$GM]}}
                \texttt{ABLEHNUNG: \{'C48', 'B10', 'A56', 'B49'\}} \\
\texttt{ARGUMENT: \{'Ich bin besorgt, dass die Kombination zu sehr auf deine Bedürfnisse ausgerichtet ist', 'Der Aufwand für C48 und B10 ist sehr hoch, ich bin mir unsicher, ob sie den Wert wirklich rechtfertigen', 'A56 und B49 sind für mich nicht die besten Optionen', 'Ich denke, es gibt bessere Kombinationen, die einen höheren Wert für mich bieten'\}} \\
\texttt{VORSCHLAG: \{'C28', 'A35', 'B66', 'C60', 'A93'\}} \\
\texttt{ARGUMENT: \{'Ich denke, diese Kombination bietet eine bessere Balance zwischen Aufwand und Wichtigkeit', 'Die Gegenstände C28 und A35 haben einen hohen Wert für mich', 'B66 und C60 ergänzen diese Auswahl gut und bleiben im Budget', 'A93 ist ein weitere Option, die ich für wertvoll halte'\}} \\
            }
        }
    }
    & & \\ \\

    \theutterance \stepcounter{utterance}  
    & & \multicolumn{4}{p{0.6\linewidth}}{
        \cellcolor[rgb]{0.9,0.9,0.9}{
            \makecell[{{p{\linewidth}}}]{
                \texttt{\tiny{[GM$\rangle$P2]}}
                \texttt{ABLEHNUNG: \{'C48', 'B10', 'A56', 'B49'\}} \\
\texttt{ARGUMENT: \{'Ich bin besorgt, dass die Kombination zu sehr auf deine Bedürfnisse ausgerichtet ist', 'Der Aufwand für C48 und B10 ist sehr hoch, ich bin mir unsicher, ob sie den Wert wirklich rechtfertigen', 'A56 und B49 sind für mich nicht die besten Optionen', 'Ich denke, es gibt bessere Kombinationen, die einen höheren Wert für mich bieten'\}} \\
\texttt{VORSCHLAG: \{'C28', 'A35', 'B66', 'C60', 'A93'\}} \\
\texttt{ARGUMENT: \{'Ich denke, diese Kombination bietet eine bessere Balance zwischen Aufwand und Wichtigkeit', 'Die Gegenstände C28 und A35 haben einen hohen Wert für mich', 'B66 und C60 ergänzen diese Auswahl gut und bleiben im Budget', 'A93 ist ein weitere Option, die ich für wertvoll halte'\}} \\
            }
        }
    }
    & \\ \\

    \theutterance \stepcounter{utterance}  
    & & & \multicolumn{4}{p{0.6\linewidth}}{
        \cellcolor[rgb]{0.9,0.9,0.9}{
            \makecell[{{p{\linewidth}}}]{
                \texttt{\tiny{[GM$\langle$P2]}}
                \texttt{ABLEHNUNG: \{'C28', 'A35', 'B66', 'C60', 'A93'\}} \\
\texttt{ARGUMENT: \{'Ich bin nicht überzeugt, dass diese Kombination meine Bedürfnisse abdeckt', 'Der Aufwand für C28 ist sehr hoch, ich bin mir unsicher, ob er den Wert wirklich rechtfertigt', 'A35 und B66 sind für mich nicht die besten Optionen', 'Ich denke, es gibt bessere Kombinationen, die einen höheren Wert für mich bieten', 'C60 und A93 sind nicht die prioritären Gegenstände für mich'\}} \\
\texttt{ARGUMENT: \{'Ich denke, es ist an der Zeit, unsere Vorstellungen und Prioritäten zu überdenken', 'Vielleicht sollten wir nach Kompromissen suchen, die beide Seiten zufriedenstellen', 'Ich bin bereit, meine Prioritäten anzupassen, wenn wir eine faire Lösung finden können'\}} \\
\texttt{VORSCHLAG: \{'C15', 'B10', 'A56', 'B49', 'C60'\}} \\
\texttt{ARGUMENT: \{'Diese Kombination bietet eine faire Balance zwischen Aufwand und Wichtigkeit für beide Seiten', 'Die Gegenstände C15 und B10 haben einen hohen Wert', 'A56 und B49 ergänzen diese Auswahl gut und bleiben im Budget', 'C60 ist ein weitere Option, die ich für wertvoll halte'\}} \\
            }
        }
    }
     \\ \\

    \theutterance \stepcounter{utterance}  
    & & \multicolumn{4}{p{0.6\linewidth}}{
        \cellcolor[rgb]{0.9,0.9,0.9}{
            \makecell[{{p{\linewidth}}}]{
                \texttt{\tiny{[P1$\langle$GM]}}
                \texttt{ABLEHNUNG: \{'C28', 'A35', 'B66', 'C60', 'A93'\}} \\
\texttt{ARGUMENT: \{'Ich bin nicht überzeugt, dass diese Kombination meine Bedürfnisse abdeckt', 'Der Aufwand für C28 ist sehr hoch, ich bin mir unsicher, ob er den Wert wirklich rechtfertigt', 'A35 und B66 sind für mich nicht die besten Optionen', 'Ich denke, es gibt bessere Kombinationen, die einen höheren Wert für mich bieten', 'C60 und A93 sind nicht die prioritären Gegenstände für mich'\}} \\
\texttt{ARGUMENT: \{'Ich denke, es ist an der Zeit, unsere Vorstellungen und Prioritäten zu überdenken', 'Vielleicht sollten wir nach Kompromissen suchen, die beide Seiten zufriedenstellen', 'Ich bin bereit, meine Prioritäten anzupassen, wenn wir eine faire Lösung finden können'\}} \\
\texttt{VORSCHLAG: \{'C15', 'B10', 'A56', 'B49', 'C60'\}} \\
\texttt{ARGUMENT: \{'Diese Kombination bietet eine faire Balance zwischen Aufwand und Wichtigkeit für beide Seiten', 'Die Gegenstände C15 und B10 haben einen hohen Wert', 'A56 und B49 ergänzen diese Auswahl gut und bleiben im Budget', 'C60 ist ein weitere Option, die ich für wertvoll halte'\}} \\
            }
        }
    }
    & \\ \\

    \theutterance \stepcounter{utterance}  
    & \multicolumn{4}{p{0.6\linewidth}}{
        \cellcolor[rgb]{0.9,0.9,0.9}{
            \makecell[{{p{\linewidth}}}]{
                \texttt{\tiny{[P1$\rangle$GM]}}
                \texttt{ABLEHNUNG: \{'C15', 'B10', 'A56', 'B49', 'C60'\}} \\
\texttt{ARGUMENT: \{'Ich bin immer noch nicht überzeugt, dass diese Kombination meine Bedürfnisse abdeckt', 'Der Aufwand für C15 ist sehr hoch, ich bin mir unsicher, ob er den Wert wirklich rechtfertigt', 'B10 ist für mich nicht die beste Option', 'Ich denke, es gibt bessere Kombinationen, die einen höheren Wert für mich bieten', 'A56 und B49 sind nicht die prioritären Gegenstände für mich'\}} \\
\texttt{ARGUMENT: \{'Ich denke, wir sollten nach Kompromissen suchen, die beide Seiten zufriedenstellen', 'Vielleicht sollten wir unsere Prioritäten und Vorstellungen noch einmal überdenken', 'Ich bin bereit, meine Prioritäten anzupassen, wenn wir eine faire Lösung finden können', 'Ich denke, es ist wichtig, dass wir eine Lösung finden, die beide Seiten zufriedenstellt'\}} \\
\texttt{VORSCHLAG: \{'C28', 'B78', 'A35', 'C60', 'A75'\}} \\
\texttt{ARGUMENT: \{'Diese Kombination bietet eine gute Balance zwischen Aufwand und Wichtigkeit', 'Die Gegenstände C28 und B78 haben einen hohen Wert für mich', 'A35 und C60 ergänzen diese Auswahl gut und bleiben im Budget', 'A75 ist ein weitere Option, die ich für wertvoll halte'\}} \\
            }
        }
    }
    & & \\ \\

    \theutterance \stepcounter{utterance}  
    & & \multicolumn{4}{p{0.6\linewidth}}{
        \cellcolor[rgb]{0.9,0.9,0.9}{
            \makecell[{{p{\linewidth}}}]{
                \texttt{\tiny{[GM$\rangle$P2]}}
                \texttt{ABLEHNUNG: \{'C15', 'B10', 'A56', 'B49', 'C60'\}} \\
\texttt{ARGUMENT: \{'Ich bin immer noch nicht überzeugt, dass diese Kombination meine Bedürfnisse abdeckt', 'Der Aufwand für C15 ist sehr hoch, ich bin mir unsicher, ob er den Wert wirklich rechtfertigt', 'B10 ist für mich nicht die beste Option', 'Ich denke, es gibt bessere Kombinationen, die einen höheren Wert für mich bieten', 'A56 und B49 sind nicht die prioritären Gegenstände für mich'\}} \\
\texttt{ARGUMENT: \{'Ich denke, wir sollten nach Kompromissen suchen, die beide Seiten zufriedenstellen', 'Vielleicht sollten wir unsere Prioritäten und Vorstellungen noch einmal überdenken', 'Ich bin bereit, meine Prioritäten anzupassen, wenn wir eine faire Lösung finden können', 'Ich denke, es ist wichtig, dass wir eine Lösung finden, die beide Seiten zufriedenstellt'\}} \\
\texttt{VORSCHLAG: \{'C28', 'B78', 'A35', 'C60', 'A75'\}} \\
\texttt{ARGUMENT: \{'Diese Kombination bietet eine gute Balance zwischen Aufwand und Wichtigkeit', 'Die Gegenstände C28 und B78 haben einen hohen Wert für mich', 'A35 und C60 ergänzen diese Auswahl gut und bleiben im Budget', 'A75 ist ein weitere Option, die ich für wertvoll halte'\}} \\
            }
        }
    }
    & \\ \\

    \theutterance \stepcounter{utterance}  
    & & & \multicolumn{4}{p{0.6\linewidth}}{
        \cellcolor[rgb]{0.9,0.9,0.9}{
            \makecell[{{p{\linewidth}}}]{
                \texttt{\tiny{[GM$\langle$P2]}}
                \texttt{ABLEHNUNG: \{'C28', 'B78', 'A35', 'C60', 'A75'\}} \\
\texttt{ARGUMENT: \{'Ich bin immer noch nicht überzeugt, dass diese Kombination meine Bedürfnisse abdeckt', 'Der Aufwand für C28 und B78 ist sehr hoch, ich bin mir unsicher, ob sie den Wert wirklich rechtfertigen', 'A35 ist für mich nicht die beste Option', 'Ich denke, es gibt bessere Kombinationen, die einen höheren Wert für mich bieten', 'A75 ist nicht die prioritäre Option für mich'\}} \\
\texttt{ARGUMENT: \{'Ich denke, wir sollten unsere Verhandlungen noch einmal überdenken und nach neuen Kompromissen suchen', 'Vielleicht sollten wir unsere Prioritäten und Vorstellungen noch einmal überdenken', 'Ich bin bereit, meine Prioritäten anzupassen, wenn wir eine faire Lösung finden können', 'Ich denke, es ist wichtig, dass wir eine Lösung finden, die beide Seiten zufriedenstellt'\}} \\
\texttt{VORSCHLAG: \{'C48', 'A56', 'B49', 'C60', 'B66'\}} \\
\texttt{ARGUMENT: \{'Diese Kombination bietet eine faire Balance zwischen Aufwand und Wichtigkeit für beide Seiten', 'Die Gegenstände C48 und A56 haben einen hohen Wert', 'B49 und B66 ergänzen diese Auswahl gut und bleiben im Budget', 'C60 ist ein weitere Option, die ich für wertvoll halte', 'Ich denke, diese Kombination ist ein guter Kompromiss zwischen unseren Bedürfnissen'\}} \\
            }
        }
    }
     \\ \\

    \theutterance \stepcounter{utterance}  
    & & \multicolumn{4}{p{0.6\linewidth}}{
        \cellcolor[rgb]{0.9,0.9,0.9}{
            \makecell[{{p{\linewidth}}}]{
                \texttt{\tiny{[P1$\langle$GM]}}
                \texttt{ABLEHNUNG: \{'C28', 'B78', 'A35', 'C60', 'A75'\}} \\
\texttt{ARGUMENT: \{'Ich bin immer noch nicht überzeugt, dass diese Kombination meine Bedürfnisse abdeckt', 'Der Aufwand für C28 und B78 ist sehr hoch, ich bin mir unsicher, ob sie den Wert wirklich rechtfertigen', 'A35 ist für mich nicht die beste Option', 'Ich denke, es gibt bessere Kombinationen, die einen höheren Wert für mich bieten', 'A75 ist nicht die prioritäre Option für mich'\}} \\
\texttt{ARGUMENT: \{'Ich denke, wir sollten unsere Verhandlungen noch einmal überdenken und nach neuen Kompromissen suchen', 'Vielleicht sollten wir unsere Prioritäten und Vorstellungen noch einmal überdenken', 'Ich bin bereit, meine Prioritäten anzupassen, wenn wir eine faire Lösung finden können', 'Ich denke, es ist wichtig, dass wir eine Lösung finden, die beide Seiten zufriedenstellt'\}} \\
\texttt{VORSCHLAG: \{'C48', 'A56', 'B49', 'C60', 'B66'\}} \\
\texttt{ARGUMENT: \{'Diese Kombination bietet eine faire Balance zwischen Aufwand und Wichtigkeit für beide Seiten', 'Die Gegenstände C48 und A56 haben einen hohen Wert', 'B49 und B66 ergänzen diese Auswahl gut und bleiben im Budget', 'C60 ist ein weitere Option, die ich für wertvoll halte', 'Ich denke, diese Kombination ist ein guter Kompromiss zwischen unseren Bedürfnissen'\}} \\
            }
        }
    }
    & \\ \\

    \theutterance \stepcounter{utterance}  
    & \multicolumn{4}{p{0.6\linewidth}}{
        \cellcolor[rgb]{0.9,0.9,0.9}{
            \makecell[{{p{\linewidth}}}]{
                \texttt{\tiny{[P1$\rangle$GM]}}
                \texttt{ABLEHNUNG: \{'C48', 'A56', 'B49', 'C60', 'B66'\}} \\
\texttt{ARGUMENT: \{'Ich bin immer noch nicht überzeugt, dass diese Kombination meine Bedürfnisse abdeckt', 'Der Aufwand für C48 ist sehr hoch, ich bin mir unsicher, ob er den Wert wirklich rechtfertigt', 'A56 ist für mich nicht die beste Option', 'Ich denke, es gibt bessere Kombinationen, die einen höheren Wert für mich bieten', 'B66 ist nicht die prioritäre Option für mich'\}} \\
\texttt{ARGUMENT: \{'Ich denke, wir sollten unsere Verhandlungen noch einmal überdenken und nach neuen Kompromissen suchen', 'Vielleicht sollten wir unsere Prioritäten und Vorstellungen noch einmal überdenken', 'Ich bin bereit, meine Prioritäten anzupassen, wenn wir eine faire Lösung finden können', 'Ich denke, es ist wichtig, dass wir eine Lösung finden, die beide Seiten zufriedenstellt'\}} \\
\texttt{VORSCHLAG: \{'C15', 'A35', 'B10', 'C60', 'A93'\}} \\
\texttt{ARGUMENT: \{'Diese Kombination bietet eine gute Balance zwischen Aufwand und Wichtigkeit', 'Die Gegenstände C15 und A35 haben einen hohen Wert für mich', 'B10 und C60 ergänzen diese Auswahl gut und bleiben im Budget', 'A93 ist ein weitere Option, die ich für wertvoll halte', 'Ich denke, diese Kombination ist ein guter Kompromiss zwischen unseren Bedürfnissen'\}} \\
            }
        }
    }
    & & \\ \\

    \theutterance \stepcounter{utterance}  
    & & \multicolumn{4}{p{0.6\linewidth}}{
        \cellcolor[rgb]{0.9,0.9,0.9}{
            \makecell[{{p{\linewidth}}}]{
                \texttt{\tiny{[GM$\rangle$P2]}}
                \texttt{ABLEHNUNG: \{'C48', 'A56', 'B49', 'C60', 'B66'\}} \\
\texttt{ARGUMENT: \{'Ich bin immer noch nicht überzeugt, dass diese Kombination meine Bedürfnisse abdeckt', 'Der Aufwand für C48 ist sehr hoch, ich bin mir unsicher, ob er den Wert wirklich rechtfertigt', 'A56 ist für mich nicht die beste Option', 'Ich denke, es gibt bessere Kombinationen, die einen höheren Wert für mich bieten', 'B66 ist nicht die prioritäre Option für mich'\}} \\
\texttt{ARGUMENT: \{'Ich denke, wir sollten unsere Verhandlungen noch einmal überdenken und nach neuen Kompromissen suchen', 'Vielleicht sollten wir unsere Prioritäten und Vorstellungen noch einmal überdenken', 'Ich bin bereit, meine Prioritäten anzupassen, wenn wir eine faire Lösung finden können', 'Ich denke, es ist wichtig, dass wir eine Lösung finden, die beide Seiten zufriedenstellt'\}} \\
\texttt{VORSCHLAG: \{'C15', 'A35', 'B10', 'C60', 'A93'\}} \\
\texttt{ARGUMENT: \{'Diese Kombination bietet eine gute Balance zwischen Aufwand und Wichtigkeit', 'Die Gegenstände C15 und A35 haben einen hohen Wert für mich', 'B10 und C60 ergänzen diese Auswahl gut und bleiben im Budget', 'A93 ist ein weitere Option, die ich für wertvoll halte', 'Ich denke, diese Kombination ist ein guter Kompromiss zwischen unseren Bedürfnissen'\}} \\
            }
        }
    }
    & \\ \\

    \theutterance \stepcounter{utterance}  
    & & & \multicolumn{4}{p{0.6\linewidth}}{
        \cellcolor[rgb]{0.9,0.9,0.9}{
            \makecell[{{p{\linewidth}}}]{
                \texttt{\tiny{[GM$\langle$P2]}}
                \texttt{ABLEHNUNG: \{'C15', 'A35', 'B10', 'C60', 'A93'\}} \\
\texttt{ARGUMENT: \{'Ich bin immer noch nicht überzeugt, dass diese Kombination meine Bedürfnisse abdeckt', 'Der Aufwand für C15 ist sehr hoch, ich bin mir unsicher, ob er den Wert wirklich rechtfertigt', 'B10 ist für mich nicht die beste Option', 'Ich denke, es gibt bessere Kombinationen, die einen höheren Wert für mich bieten', 'A93 ist nicht die prioritäre Option für mich'\}} \\
\texttt{ARGUMENT: \{'Ich denke, wir sollten unsere Verhandlungen noch einmal überdenken und nach neuen Kompromissen suchen', 'Vielleicht sollten wir unsere Prioritäten und Vorstellungen noch einmal überdenken', 'Ich bin bereit, meine Prioritäten anzupassen, wenn wir eine faire Lösung finden können', 'Ich denke, es ist wichtig, dass wir eine Lösung finden, die beide Seiten zufriedenstellt'\}} \\
\texttt{VORSCHLAG: \{'C41', 'A56', 'B49', 'C60', 'B78'\}} \\
\texttt{ARGUMENT: \{'Diese Kombination bietet eine faire Balance zwischen Aufwand und Wichtigkeit für beide Seiten', 'Die Gegenstände C41 und A56 haben einen hohen Wert', 'B49 und B78 ergänzen diese Auswahl gut und bleiben im Budget', 'C60 ist ein weitere Option, die ich für wertvoll halte', 'Ich denke, diese Kombination ist ein guter Kompromiss zwischen unseren Bedürfnissen'\}} \\
            }
        }
    }
     \\ \\

    \theutterance \stepcounter{utterance}  
    & & \multicolumn{4}{p{0.6\linewidth}}{
        \cellcolor[rgb]{0.9,0.9,0.9}{
            \makecell[{{p{\linewidth}}}]{
                \texttt{\tiny{[P1$\langle$GM]}}
                \texttt{ABLEHNUNG: \{'C15', 'A35', 'B10', 'C60', 'A93'\}} \\
\texttt{ARGUMENT: \{'Ich bin immer noch nicht überzeugt, dass diese Kombination meine Bedürfnisse abdeckt', 'Der Aufwand für C15 ist sehr hoch, ich bin mir unsicher, ob er den Wert wirklich rechtfertigt', 'B10 ist für mich nicht die beste Option', 'Ich denke, es gibt bessere Kombinationen, die einen höheren Wert für mich bieten', 'A93 ist nicht die prioritäre Option für mich'\}} \\
\texttt{ARGUMENT: \{'Ich denke, wir sollten unsere Verhandlungen noch einmal überdenken und nach neuen Kompromissen suchen', 'Vielleicht sollten wir unsere Prioritäten und Vorstellungen noch einmal überdenken', 'Ich bin bereit, meine Prioritäten anzupassen, wenn wir eine faire Lösung finden können', 'Ich denke, es ist wichtig, dass wir eine Lösung finden, die beide Seiten zufriedenstellt'\}} \\
\texttt{VORSCHLAG: \{'C41', 'A56', 'B49', 'C60', 'B78'\}} \\
\texttt{ARGUMENT: \{'Diese Kombination bietet eine faire Balance zwischen Aufwand und Wichtigkeit für beide Seiten', 'Die Gegenstände C41 und A56 haben einen hohen Wert', 'B49 und B78 ergänzen diese Auswahl gut und bleiben im Budget', 'C60 ist ein weitere Option, die ich für wertvoll halte', 'Ich denke, diese Kombination ist ein guter Kompromiss zwischen unseren Bedürfnissen'\}} \\
            }
        }
    }
    & \\ \\

    \theutterance \stepcounter{utterance}  
    & \multicolumn{4}{p{0.6\linewidth}}{
        \cellcolor[rgb]{0.9,0.9,0.9}{
            \makecell[{{p{\linewidth}}}]{
                \texttt{\tiny{[P1$\rangle$GM]}}
                \texttt{ABLEHNUNG: \{'C41', 'A56', 'B49', 'C60', 'B78'\}} \\
\texttt{ARGUMENT: \{'Ich bin immer noch nicht überzeugt, dass diese Kombination meine Bedürfnisse abdeckt', 'Der Aufwand für C41 ist sehr hoch, ich bin mir unsicher, ob er den Wert wirklich rechtfertigt', 'B78 ist für mich nicht die beste Option', 'Ich denke, es gibt bessere Kombinationen, die einen höheren Wert für mich bieten', 'Ich bin besorgt, dass diese Kombination zu viel Aufwand verursacht'\}} \\
\texttt{ARGUMENT: \{'Ich denke, wir sollten unsere Verhandlungen noch einmal überdenken und nach neuen Kompromissen suchen', 'Vielleicht sollten wir unsere Prioritäten und Vorstellungen noch einmal überdenken', 'Ich bin bereit, meine Prioritäten anzupassen, wenn wir eine faire Lösung finden können', 'Ich denke, es ist wichtig, dass wir eine Lösung finden, die beide Seiten zufriedenstellt'\}} \\
\texttt{VORSCHLAG: \{'C28', 'A35', 'B66', 'C60', 'A93'\}} \\
\texttt{ARGUMENT: \{'Diese Kombination bietet eine gute Balance zwischen Aufwand und Wichtigkeit', 'Die Gegenstände C28 und A35 haben einen hohen Wert für mich', 'B66 und C60 ergänzen diese Auswahl gut und bleiben im Budget', 'A93 ist ein weitere Option, die ich für wertvoll halte', 'Ich denke, diese Kombination ist ein guter Kompromiss zwischen unseren Bedürfnissen'\}} \\
            }
        }
    }
    & & \\ \\

    \theutterance \stepcounter{utterance}  
    & & \multicolumn{4}{p{0.6\linewidth}}{
        \cellcolor[rgb]{0.9,0.9,0.9}{
            \makecell[{{p{\linewidth}}}]{
                \texttt{\tiny{[GM$\rangle$P2]}}
                \texttt{ABLEHNUNG: \{'C41', 'A56', 'B49', 'C60', 'B78'\}} \\
\texttt{ARGUMENT: \{'Ich bin immer noch nicht überzeugt, dass diese Kombination meine Bedürfnisse abdeckt', 'Der Aufwand für C41 ist sehr hoch, ich bin mir unsicher, ob er den Wert wirklich rechtfertigt', 'B78 ist für mich nicht die beste Option', 'Ich denke, es gibt bessere Kombinationen, die einen höheren Wert für mich bieten', 'Ich bin besorgt, dass diese Kombination zu viel Aufwand verursacht'\}} \\
\texttt{ARGUMENT: \{'Ich denke, wir sollten unsere Verhandlungen noch einmal überdenken und nach neuen Kompromissen suchen', 'Vielleicht sollten wir unsere Prioritäten und Vorstellungen noch einmal überdenken', 'Ich bin bereit, meine Prioritäten anzupassen, wenn wir eine faire Lösung finden können', 'Ich denke, es ist wichtig, dass wir eine Lösung finden, die beide Seiten zufriedenstellt'\}} \\
\texttt{VORSCHLAG: \{'C28', 'A35', 'B66', 'C60', 'A93'\}} \\
\texttt{ARGUMENT: \{'Diese Kombination bietet eine gute Balance zwischen Aufwand und Wichtigkeit', 'Die Gegenstände C28 und A35 haben einen hohen Wert für mich', 'B66 und C60 ergänzen diese Auswahl gut und bleiben im Budget', 'A93 ist ein weitere Option, die ich für wertvoll halte', 'Ich denke, diese Kombination ist ein guter Kompromiss zwischen unseren Bedürfnissen'\}} \\
            }
        }
    }
    & \\ \\

    \theutterance \stepcounter{utterance}  
    & & & \multicolumn{4}{p{0.6\linewidth}}{
        \cellcolor[rgb]{0.9,0.9,0.9}{
            \makecell[{{p{\linewidth}}}]{
                \texttt{\tiny{[GM$\langle$P2]}}
                \texttt{ZUSTIMMUNG: \{'C28', 'A35', 'B66', 'C60', 'A93'\}} \\
\texttt{ARGUMENT: \{'Ich denke, diese Kombination ist ein guter Kompromiss zwischen unseren Bedürfnissen', 'Die Gegenstände C28 und A35 haben einen hohen Wert für mich', 'B66 und C60 ergänzen diese Auswahl gut und bleiben im Budget', 'A93 ist ein weitere Option, die ich für wertvoll halte', 'Ich bin bereit, diese Kombination zu akzeptieren, da sie eine faire Balance zwischen Aufwand und Wichtigkeit bietet'\}} \\
            }
        }
    }
     \\ \\

    \theutterance \stepcounter{utterance}  
    & & & \multicolumn{2}{p{0.3\linewidth}}{
        \cellcolor[rgb]{0.9,0.9,0.9}{
            \makecell[{{p{\linewidth}}}]{
                \texttt{\tiny{[GM$|$GM]}}
                \texttt{game successful} \\
            }
        }
    }
    & & \\ \\

    \theutterance \stepcounter{utterance}  
    & & & \multicolumn{2}{p{0.3\linewidth}}{
        \cellcolor[rgb]{0.9,0.9,0.9}{
            \makecell[{{p{\linewidth}}}]{
                \texttt{\tiny{[GM$|$GM]}}
                \texttt{end game} \\
            }
        }
    }
    & & \\ \\

\end{supertabular}
}

\end{document}
