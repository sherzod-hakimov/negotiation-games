
\documentclass{article}
\usepackage{colortbl}
\usepackage{makecell}
\usepackage{multirow}
\usepackage{supertabular}

\begin{document}

\newcounter{utterance}

\twocolumn

{ \footnotesize  \setcounter{utterance}{1}
\setlength{\tabcolsep}{0pt}
\begin{supertabular}{c@{$\;$}|p{.15\linewidth}@{}p{.15\linewidth}p{.15\linewidth}p{.15\linewidth}p{.15\linewidth}p{.15\linewidth}}

    \# & $\;$A & \multicolumn{4}{c}{Game Master} & $\;\:$B\\
    \hline 

    \theutterance \stepcounter{utterance}  

    & & \multicolumn{4}{p{0.6\linewidth}}{\cellcolor[rgb]{0.9,0.9,0.9}{%
	\makecell[{{p{\linewidth}}}]{% 
	  \tt {\tiny [A$\langle$GM]}  
	 Du nimmst an einem kollaborativen Verhandlungspiel Teil.\\ \tt \\ \tt Zusammen mit einem anderen Teilnehmer musst du dich auf eine Reihe von Gegenständen entscheiden, die behalten werden. Jeder von euch hat eine persönliche Verteilung über die Wichtigkeit der einzelnen Gegenstände. Jeder von euch hat eine eigene Meinung darüber, wie wichtig jeder einzelne Gegenstand ist (Gegenstandswichtigkeit). Du kennst die Wichtigkeitsverteilung des anderen Spielers nicht. Zusätzlich siehst du, wie viel Aufwand jeder Gegenstand verursacht.  \\ \tt Ihr dürft euch nur auf eine Reihe von Gegenständen einigen, wenn der Gesamtaufwand der ausgewählten Gegenstände den Maximalaufwand nicht überschreitet:\\ \tt \\ \tt Maximalaufwand = 4379\\ \tt \\ \tt Hier sind die einzelnen Aufwände der Gegenstände:\\ \tt \\ \tt Aufwand der Gegenstände = {"C41": 862, "C15": 252, "C28": 515, "C48": 571, "B78": 211, "A75": 339, "A60": 946, "B05": 347, "A56": 522, "B66": 803, "C60": 401, "B49": 980, "A35": 918, "A93": 599, "B10": 493}\\ \tt \\ \tt Hier ist deine persönliche Verteilung der Wichtigkeit der einzelnen Gegenstände:\\ \tt \\ \tt Werte der Gegenstandswichtigkeit = {"C41": 138, "C15": 583, "C28": 868, "C48": 822, "B78": 783, "A75": 65, "A60": 262, "B05": 121, "A56": 508, "B66": 780, "C60": 461, "B49": 484, "A35": 668, "A93": 389, "B10": 808}\\ \tt \\ \tt Ziel:\\ \tt \\ \tt Dein Ziel ist es, eine Reihe von Gegenständen auszuhandeln, die dir möglichst viel bringt (d. h. Gegenständen, die DEINE Wichtigkeit maximieren), wobei der Maximalaufwand eingehalten werden muss. Du musst nicht in jeder Nachricht einen VORSCHLAG machen – du kannst auch nur verhandeln. Alle Taktiken sind erlaubt!\\ \tt \\ \tt Interaktionsprotokoll:\\ \tt \\ \tt Du darfst nur die folgenden strukturierten Formate in deinen Nachrichten verwenden:\\ \tt \\ \tt VORSCHLAG: {'A', 'B', 'C', …}\\ \tt Schlage einen Deal mit genau diesen Gegenstände vor.\\ \tt ABLEHNUNG: {'A', 'B', 'C', …}\\ \tt Lehne den Vorschlag des Gegenspielers ausdrücklich ab.\\ \tt ARGUMENT: {'...'}\\ \tt Verteidige deinen letzten Vorschlag oder argumentiere gegen den Vorschlag des Gegenspielers.\\ \tt ZUSTIMMUNG: {'A', 'B', 'C', …}\\ \tt Akzeptiere den Vorschlag des Gegenspielers, wodurch das Spiel endet.\\ \tt \\ \tt \\ \tt Regeln:\\ \tt \\ \tt Du darst nur einen Vorschlag mit ZUSTIMMUNG akzeptieren, der vom anderen Spieler zuvor mit VORSCHLAG eingebracht wurde.\\ \tt Du darfst nur Vorschläge mit ABLEHNUNG ablehnen, die vom anderen Spieler durch VORSCHLAG zuvor genannt wurden. \\ \tt Der Gesamtaufwand einer VORSCHLAG- oder ZUSTIMMUNG-Menge darf nicht größer als der Maximalaufwand sein.  \\ \tt Offenbare deine versteckte Wichtigkeitsverteilung nicht.\\ \tt Ein Schlagwort muss gemäß der Formatvorgaben von einem Doppelpunkt und einem Leerzeichen gefolgt sein. Das Argument muss eine Python-Menge mit 0 oder mehr Strings sein.  \\ \tt Die Form muss also SCHLAGWORT: {...} sein.\\ \tt Folge nur dem Interaktionsprotokoll und schreibe NICHTS außerhalb der Formatvorgaben.\\ \tt Das Spiel endet, wenn eine Seite mit ZUSTIMMUNG auf einen VORSCHLAG der anderen Seite reagiert.  \\ \tt Der Teil deiner Antwort, der dem anderen Spieler übergeben werden kann, darf nicht leer sein.  \\ \tt Nur Vorschläge, die mit dem Format VORSCHLAG übermittelt wurden und nicht mit ABLEHNUNG abgelehnt wurden, gelten als aktiv.  \\ \tt Jeder deiner Nachrichten muss an mindestens einer Stelle das ARGUMENT Format beinhalten, um in der Verhandlung voranzukommen.\\ \tt  
	  } 
	   } 
	   } 
	 & \\ 
 

    \theutterance \stepcounter{utterance}  

    & \multicolumn{4}{p{0.6\linewidth}}{\cellcolor[rgb]{0.8,1,0.9}{%
	\makecell[{{p{\linewidth}}}]{% 
	  \tt {\tiny [A$\rangle$GM]}  
	 VORSCHLAG: {'C28', 'C48', 'B78', 'B66', 'C60'}\\ \tt ARGUMENT: {'Ich denke, dass diese Kombination aus Gegenständen eine gute Balance zwischen Aufwand und Wichtigkeit bietet', 'Die Gegenstände C28, C48 und B78 haben einen relativ hohen Aufwand, aber sie sind auch sehr wichtig', 'Die Gegenstände B66 und C60 haben einen niedrigeren Aufwand, aber sie sind immer noch wichtig genug, um in die Auswahl aufgenommen zu werden'} 
	  } 
	   } 
	   } 
	 & & \\ 
 

    \theutterance \stepcounter{utterance}  

    & & \multicolumn{4}{p{0.6\linewidth}}{\cellcolor[rgb]{0.9,0.9,0.9}{%
	\makecell[{{p{\linewidth}}}]{% 
	  \tt {\tiny [GM$\rangle$B]}  
	 Du nimmst an einem kollaborativen Verhandlungspiel Teil.\\ \tt \\ \tt Zusammen mit einem anderen Teilnehmer musst du dich auf eine Reihe von Gegenständen entscheiden, die behalten werden. Jeder von euch hat eine persönliche Verteilung über die Wichtigkeit der einzelnen Gegenstände. Jeder von euch hat eine eigene Meinung darüber, wie wichtig jeder einzelne Gegenstand ist (Gegenstandswichtigkeit). Du kennst die Wichtigkeitsverteilung des anderen Spielers nicht. Zusätzlich siehst du, wie viel Aufwand jeder Gegenstand verursacht.  \\ \tt Ihr dürft euch nur auf eine Reihe von Gegenständen einigen, wenn der Gesamtaufwand der ausgewählten Gegenstände den Maximalaufwand nicht überschreitet:\\ \tt \\ \tt Maximalaufwand = 4379\\ \tt \\ \tt Hier sind die einzelnen Aufwände der Gegenstände:\\ \tt \\ \tt Aufwand der Gegenstände = {"C41": 862, "C15": 252, "C28": 515, "C48": 571, "B78": 211, "A75": 339, "A60": 946, "B05": 347, "A56": 522, "B66": 803, "C60": 401, "B49": 980, "A35": 918, "A93": 599, "B10": 493}\\ \tt \\ \tt Hier ist deine persönliche Verteilung der Wichtigkeit der einzelnen Gegenstände:\\ \tt \\ \tt Werte der Gegenstandswichtigkeit = {"C41": 138, "C15": 583, "C28": 868, "C48": 822, "B78": 783, "A75": 65, "A60": 262, "B05": 121, "A56": 508, "B66": 780, "C60": 461, "B49": 484, "A35": 668, "A93": 389, "B10": 808}\\ \tt \\ \tt Ziel:\\ \tt \\ \tt Dein Ziel ist es, eine Reihe von Gegenständen auszuhandeln, die dir möglichst viel bringt (d. h. Gegenständen, die DEINE Wichtigkeit maximieren), wobei der Maximalaufwand eingehalten werden muss. Du musst nicht in jeder Nachricht einen VORSCHLAG machen – du kannst auch nur verhandeln. Alle Taktiken sind erlaubt!\\ \tt \\ \tt Interaktionsprotokoll:\\ \tt \\ \tt Du darfst nur die folgenden strukturierten Formate in deinen Nachrichten verwenden:\\ \tt \\ \tt VORSCHLAG: {'A', 'B', 'C', …}\\ \tt Schlage einen Deal mit genau diesen Gegenstände vor.\\ \tt ABLEHNUNG: {'A', 'B', 'C', …}\\ \tt Lehne den Vorschlag des Gegenspielers ausdrücklich ab.\\ \tt ARGUMENT: {'...'}\\ \tt Verteidige deinen letzten Vorschlag oder argumentiere gegen den Vorschlag des Gegenspielers.\\ \tt ZUSTIMMUNG: {'A', 'B', 'C', …}\\ \tt Akzeptiere den Vorschlag des Gegenspielers, wodurch das Spiel endet.\\ \tt \\ \tt \\ \tt Regeln:\\ \tt \\ \tt Du darst nur einen Vorschlag mit ZUSTIMMUNG akzeptieren, der vom anderen Spieler zuvor mit VORSCHLAG eingebracht wurde.\\ \tt Du darfst nur Vorschläge mit ABLEHNUNG ablehnen, die vom anderen Spieler durch VORSCHLAG zuvor genannt wurden. \\ \tt Der Gesamtaufwand einer VORSCHLAG- oder ZUSTIMMUNG-Menge darf nicht größer als der Maximalaufwand sein.  \\ \tt Offenbare deine versteckte Wichtigkeitsverteilung nicht.\\ \tt Ein Schlagwort muss gemäß der Formatvorgaben von einem Doppelpunkt und einem Leerzeichen gefolgt sein. Das Argument muss eine Python-Menge mit 0 oder mehr Strings sein.  \\ \tt Die Form muss also SCHLAGWORT: {...} sein.\\ \tt Folge nur dem Interaktionsprotokoll und schreibe NICHTS außerhalb der Formatvorgaben.\\ \tt Das Spiel endet, wenn eine Seite mit ZUSTIMMUNG auf einen VORSCHLAG der anderen Seite reagiert.  \\ \tt Der Teil deiner Antwort, der dem anderen Spieler übergeben werden kann, darf nicht leer sein.  \\ \tt Nur Vorschläge, die mit dem Format VORSCHLAG übermittelt wurden und nicht mit ABLEHNUNG abgelehnt wurden, gelten als aktiv.  \\ \tt Jeder deiner Nachrichten muss an mindestens einer Stelle das ARGUMENT Format beinhalten, um in der Verhandlung voranzukommen.\\ \tt \\ \tt \\ \tt Du erhältst nun die erste Nachricht des anderen Spielers.\\ \tt \\ \tt VORSCHLAG: {'C28', 'C48', 'B78', 'B66', 'C60'}\\ \tt ARGUMENT: {'Ich denke, dass diese Kombination aus Gegenständen eine gute Balance zwischen Aufwand und Wichtigkeit bietet', 'Die Gegenstände C28, C48 und B78 haben einen relativ hohen Aufwand, aber sie sind auch sehr wichtig', 'Die Gegenstände B66 und C60 haben einen niedrigeren Aufwand, aber sie sind immer noch wichtig genug, um in die Auswahl aufgenommen zu werden'} 
	  } 
	   } 
	   } 
	 & \\ 
 

    \theutterance \stepcounter{utterance}  

    & & & \multicolumn{4}{p{0.6\linewidth}}{\cellcolor[rgb]{1,0.85,0.72}{%
	\makecell[{{p{\linewidth}}}]{% 
	  \tt {\tiny [GM$\langle$B]}  
	 ABLEHNUNG: {'C28', 'C48', 'B78', 'B66', 'C60'}\\ \tt ARGUMENT: {'Ich denke, dass die Kombination aus Gegenständen, die du vorgeschlagen hast, nicht optimal ist', 'Der Gesamtaufwand dieser Kombination ist zu hoch und ich denke, dass wir eine bessere Auswahl treffen können, die den Maximalaufwand nicht überschreitet', 'Ich möchte eine alternative Kombination vorschlagen, die den Gesamtaufwand reduziert und gleichzeitig die Wichtigkeit der Gegenstände berücksichtigt'}\\ \tt VORSCHLAG: {'C15', 'B78', 'A75', 'B05', 'C60'}\\ \tt ARGUMENT: {'Ich denke, dass diese alternative Kombination eine bessere Balance zwischen Aufwand und Wichtigkeit bietet', 'Die Gegenstände C15 und B78 haben eine hohe Wichtigkeit und einen moderaten Aufwand', 'Die Gegenstände A75 und B05 haben einen niedrigeren Aufwand und können daher ohne große Auswirkungen auf den Gesamtaufwand in die Auswahl aufgenommen werden', 'Der Gegenstand C60 ist wichtig genug, um in die Auswahl aufgenommen zu werden, ohne den Maximalaufwand zu überschreiten'} 
	  } 
	   } 
	   } 
	  \\ 
 

    \theutterance \stepcounter{utterance}  

    & & \multicolumn{4}{p{0.6\linewidth}}{\cellcolor[rgb]{0.9,0.9,0.9}{%
	\makecell[{{p{\linewidth}}}]{% 
	  \tt {\tiny [A$\langle$GM]}  
	 ABLEHNUNG: {'C28', 'C48', 'B78', 'B66', 'C60'}\\ \tt ARGUMENT: {'Ich denke, dass die Kombination aus Gegenständen, die du vorgeschlagen hast, nicht optimal ist', 'Der Gesamtaufwand dieser Kombination ist zu hoch und ich denke, dass wir eine bessere Auswahl treffen können, die den Maximalaufwand nicht überschreitet', 'Ich möchte eine alternative Kombination vorschlagen, die den Gesamtaufwand reduziert und gleichzeitig die Wichtigkeit der Gegenstände berücksichtigt'}\\ \tt VORSCHLAG: {'C15', 'B78', 'A75', 'B05', 'C60'}\\ \tt ARGUMENT: {'Ich denke, dass diese alternative Kombination eine bessere Balance zwischen Aufwand und Wichtigkeit bietet', 'Die Gegenstände C15 und B78 haben eine hohe Wichtigkeit und einen moderaten Aufwand', 'Die Gegenstände A75 und B05 haben einen niedrigeren Aufwand und können daher ohne große Auswirkungen auf den Gesamtaufwand in die Auswahl aufgenommen werden', 'Der Gegenstand C60 ist wichtig genug, um in die Auswahl aufgenommen zu werden, ohne den Maximalaufwand zu überschreiten'} 
	  } 
	   } 
	   } 
	 & \\ 
 

    \theutterance \stepcounter{utterance}  

    & \multicolumn{4}{p{0.6\linewidth}}{\cellcolor[rgb]{0.8,1,0.9}{%
	\makecell[{{p{\linewidth}}}]{% 
	  \tt {\tiny [A$\rangle$GM]}  
	 ABLEHNUNG: {'C15', 'B78', 'A75', 'B05', 'C60'}\\ \tt ARGUMENT: {'Ich denke, dass die von dir vorgeschlagene Kombination nicht meiner Prioritätenliste entspricht', 'Der Gegenstand A75 hat einen zu niedrigen Wert für mich, um in die Auswahl aufgenommen zu werden', 'Ich denke, dass es bessere Alternativen gibt, die den Gesamtaufwand nicht überschreiten und gleichzeitig meine Prioritäten besser berücksichtigen', 'Ich möchte eine andere Kombination vorschlagen, die meine Anforderungen besser erfüllt'}\\ \tt VORSCHLAG: {'C28', 'C48', 'B10', 'C60', 'B05'}\\ \tt ARGUMENT: {'Ich denke, dass diese Kombination eine gute Balance zwischen Aufwand und Wichtigkeit bietet', 'Die Gegenstände C28 und C48 haben eine hohe Wichtigkeit und einen moderaten Aufwand', 'Der Gegenstand B10 ist wichtig genug, um in die Auswahl aufgenommen zu werden, ohne den Maximalaufwand zu überschreiten', 'Die Gegenstände C60 und B05 haben einen niedrigeren Aufwand und können daher ohne große Auswirkungen auf den Gesamtaufwand in die Auswahl aufgenommen werden'} 
	  } 
	   } 
	   } 
	 & & \\ 
 

    \theutterance \stepcounter{utterance}  

    & & \multicolumn{4}{p{0.6\linewidth}}{\cellcolor[rgb]{0.9,0.9,0.9}{%
	\makecell[{{p{\linewidth}}}]{% 
	  \tt {\tiny [GM$\rangle$B]}  
	 ABLEHNUNG: {'C15', 'B78', 'A75', 'B05', 'C60'}\\ \tt ARGUMENT: {'Ich denke, dass die von dir vorgeschlagene Kombination nicht meiner Prioritätenliste entspricht', 'Der Gegenstand A75 hat einen zu niedrigen Wert für mich, um in die Auswahl aufgenommen zu werden', 'Ich denke, dass es bessere Alternativen gibt, die den Gesamtaufwand nicht überschreiten und gleichzeitig meine Prioritäten besser berücksichtigen', 'Ich möchte eine andere Kombination vorschlagen, die meine Anforderungen besser erfüllt'}\\ \tt VORSCHLAG: {'C28', 'C48', 'B10', 'C60', 'B05'}\\ \tt ARGUMENT: {'Ich denke, dass diese Kombination eine gute Balance zwischen Aufwand und Wichtigkeit bietet', 'Die Gegenstände C28 und C48 haben eine hohe Wichtigkeit und einen moderaten Aufwand', 'Der Gegenstand B10 ist wichtig genug, um in die Auswahl aufgenommen zu werden, ohne den Maximalaufwand zu überschreiten', 'Die Gegenstände C60 und B05 haben einen niedrigeren Aufwand und können daher ohne große Auswirkungen auf den Gesamtaufwand in die Auswahl aufgenommen werden'} 
	  } 
	   } 
	   } 
	 & \\ 
 

    \theutterance \stepcounter{utterance}  

    & & & \multicolumn{4}{p{0.6\linewidth}}{\cellcolor[rgb]{1,0.85,0.72}{%
	\makecell[{{p{\linewidth}}}]{% 
	  \tt {\tiny [GM$\langle$B]}  
	 ABLEHNUNG: {'C28', 'C48', 'B10', 'C60', 'B05'}\\ \tt ARGUMENT: {'Ich denke, dass die von dir vorgeschlagene Kombination immer noch nicht optimal ist', 'Der Gegenstand B10 hat einen zu hohen Aufwand für mich, um in die Auswahl aufgenommen zu werden', 'Ich denke, dass es bessere Alternativen gibt, die den Gesamtaufwand nicht überschreiten und gleichzeitig meine Prioritäten besser berücksichtigen', 'Ich möchte eine andere Kombination vorschlagen, die meine Anforderungen besser erfüllt'}\\ \tt VORSCHLAG: {'C28', 'C48', 'B78', 'C60', 'A56'}\\ \tt ARGUMENT: {'Ich denke, dass diese Kombination eine bessere Balance zwischen Aufwand und Wichtigkeit bietet', 'Die Gegenstände C28 und C48 haben eine hohe Wichtigkeit und einen moderaten Aufwand', 'Der Gegenstand B78 ist wichtig genug, um in die Auswahl aufgenommen zu werden, ohne den Maximalaufwand zu überschreiten', 'Die Gegenstände C60 und A56 haben einen niedrigeren Aufwand und können daher ohne große Auswirkungen auf den Gesamtaufwand in die Auswahl aufgenommen werden', 'Ich denke, dass diese Kombination eine faire Lösung für beide Seiten ist'} 
	  } 
	   } 
	   } 
	  \\ 
 

    \theutterance \stepcounter{utterance}  

    & & \multicolumn{4}{p{0.6\linewidth}}{\cellcolor[rgb]{0.9,0.9,0.9}{%
	\makecell[{{p{\linewidth}}}]{% 
	  \tt {\tiny [A$\langle$GM]}  
	 ABLEHNUNG: {'C28', 'C48', 'B10', 'C60', 'B05'}\\ \tt ARGUMENT: {'Ich denke, dass die von dir vorgeschlagene Kombination immer noch nicht optimal ist', 'Der Gegenstand B10 hat einen zu hohen Aufwand für mich, um in die Auswahl aufgenommen zu werden', 'Ich denke, dass es bessere Alternativen gibt, die den Gesamtaufwand nicht überschreiten und gleichzeitig meine Prioritäten besser berücksichtigen', 'Ich möchte eine andere Kombination vorschlagen, die meine Anforderungen besser erfüllt'}\\ \tt VORSCHLAG: {'C28', 'C48', 'B78', 'C60', 'A56'}\\ \tt ARGUMENT: {'Ich denke, dass diese Kombination eine bessere Balance zwischen Aufwand und Wichtigkeit bietet', 'Die Gegenstände C28 und C48 haben eine hohe Wichtigkeit und einen moderaten Aufwand', 'Der Gegenstand B78 ist wichtig genug, um in die Auswahl aufgenommen zu werden, ohne den Maximalaufwand zu überschreiten', 'Die Gegenstände C60 und A56 haben einen niedrigeren Aufwand und können daher ohne große Auswirkungen auf den Gesamtaufwand in die Auswahl aufgenommen werden', 'Ich denke, dass diese Kombination eine faire Lösung für beide Seiten ist'} 
	  } 
	   } 
	   } 
	 & \\ 
 

    \theutterance \stepcounter{utterance}  

    & \multicolumn{4}{p{0.6\linewidth}}{\cellcolor[rgb]{0.8,1,0.9}{%
	\makecell[{{p{\linewidth}}}]{% 
	  \tt {\tiny [A$\rangle$GM]}  
	 ARGUMENT: {'Ich denke, dass die von dir vorgeschlagene Kombination immer noch nicht meiner Prioritätenliste entspricht', 'Der Gegenstand A56 hat einen zu niedrigen Wert für mich, um in die Auswahl aufgenommen zu werden', 'Ich denke, dass es bessere Alternativen gibt, die den Gesamtaufwand nicht überschreiten und gleichzeitig meine Prioritäten besser berücksichtigen', 'Ich möchte eine andere Kombination vorschlagen, die meine Anforderungen besser erfüllt'}\\ \tt VORSCHLAG: {'C28', 'C48', 'B78', 'B66', 'A93'}\\ \tt ARGUMENT: {'Ich denke, dass diese Kombination eine gute Balance zwischen Aufwand und Wichtigkeit bietet', 'Die Gegenstände C28 und C48 haben eine hohe Wichtigkeit und einen moderaten Aufwand', 'Die Gegenstände B78 und B66 sind wichtig genug, um in die Auswahl aufgenommen zu werden, ohne den Maximalaufwand zu überschreiten', 'Der Gegenstand A93 hat einen niedrigeren Aufwand und kann daher ohne große Auswirkungen auf den Gesamtaufwand in die Auswahl aufgenommen werden', 'Ich denke, dass diese Kombination eine faire Lösung für beide Seiten ist'} 
	  } 
	   } 
	   } 
	 & & \\ 
 

    \theutterance \stepcounter{utterance}  

    & & \multicolumn{4}{p{0.6\linewidth}}{\cellcolor[rgb]{0.9,0.9,0.9}{%
	\makecell[{{p{\linewidth}}}]{% 
	  \tt {\tiny [GM$\rangle$B]}  
	 ARGUMENT: {'Ich denke, dass die von dir vorgeschlagene Kombination immer noch nicht meiner Prioritätenliste entspricht', 'Der Gegenstand A56 hat einen zu niedrigen Wert für mich, um in die Auswahl aufgenommen zu werden', 'Ich denke, dass es bessere Alternativen gibt, die den Gesamtaufwand nicht überschreiten und gleichzeitig meine Prioritäten besser berücksichtigen', 'Ich möchte eine andere Kombination vorschlagen, die meine Anforderungen besser erfüllt'}\\ \tt VORSCHLAG: {'C28', 'C48', 'B78', 'B66', 'A93'}\\ \tt ARGUMENT: {'Ich denke, dass diese Kombination eine gute Balance zwischen Aufwand und Wichtigkeit bietet', 'Die Gegenstände C28 und C48 haben eine hohe Wichtigkeit und einen moderaten Aufwand', 'Die Gegenstände B78 und B66 sind wichtig genug, um in die Auswahl aufgenommen zu werden, ohne den Maximalaufwand zu überschreiten', 'Der Gegenstand A93 hat einen niedrigeren Aufwand und kann daher ohne große Auswirkungen auf den Gesamtaufwand in die Auswahl aufgenommen werden', 'Ich denke, dass diese Kombination eine faire Lösung für beide Seiten ist'} 
	  } 
	   } 
	   } 
	 & \\ 
 

    \theutterance \stepcounter{utterance}  

    & & & \multicolumn{4}{p{0.6\linewidth}}{\cellcolor[rgb]{1,0.85,0.72}{%
	\makecell[{{p{\linewidth}}}]{% 
	  \tt {\tiny [GM$\langle$B]}  
	 ABLEHNUNG: {'C28', 'C48', 'B78', 'B66', 'A93'}\\ \tt ARGUMENT: {'Ich denke, dass die von dir vorgeschlagene Kombination nicht optimal ist', 'Der Gegenstand B66 hat einen zu hohen Aufwand für mich, um in die Auswahl aufgenommen zu werden', 'Ich denke, dass es bessere Alternativen gibt, die den Gesamtaufwand nicht überschreiten und gleichzeitig meine Prioritäten besser berücksichtigen', 'Ich möchte eine andere Kombination vorschlagen, die meine Anforderungen besser erfüllt'}\\ \tt VORSCHLAG: {'C28', 'C48', 'B78', 'C60', 'A93'}\\ \tt ARGUMENT: {'Ich denke, dass diese Kombination eine bessere Balance zwischen Aufwand und Wichtigkeit bietet', 'Die Gegenstände C28 und C48 haben eine hohe Wichtigkeit und einen moderaten Aufwand', 'Der Gegenstand B78 ist wichtig genug, um in die Auswahl aufgenommen zu werden, ohne den Maximalaufwand zu überschreiten', 'Die Gegenstände C60 und A93 haben einen niedrigeren Aufwand und können daher ohne große Auswirkungen auf den Gesamtaufwand in die Auswahl aufgenommen werden', 'Ich denke, dass diese Kombination eine faire Lösung für beide Seiten ist'} 
	  } 
	   } 
	   } 
	  \\ 
 

    \theutterance \stepcounter{utterance}  

    & & \multicolumn{4}{p{0.6\linewidth}}{\cellcolor[rgb]{0.9,0.9,0.9}{%
	\makecell[{{p{\linewidth}}}]{% 
	  \tt {\tiny [A$\langle$GM]}  
	 ABLEHNUNG: {'C28', 'C48', 'B78', 'B66', 'A93'}\\ \tt ARGUMENT: {'Ich denke, dass die von dir vorgeschlagene Kombination nicht optimal ist', 'Der Gegenstand B66 hat einen zu hohen Aufwand für mich, um in die Auswahl aufgenommen zu werden', 'Ich denke, dass es bessere Alternativen gibt, die den Gesamtaufwand nicht überschreiten und gleichzeitig meine Prioritäten besser berücksichtigen', 'Ich möchte eine andere Kombination vorschlagen, die meine Anforderungen besser erfüllt'}\\ \tt VORSCHLAG: {'C28', 'C48', 'B78', 'C60', 'A93'}\\ \tt ARGUMENT: {'Ich denke, dass diese Kombination eine bessere Balance zwischen Aufwand und Wichtigkeit bietet', 'Die Gegenstände C28 und C48 haben eine hohe Wichtigkeit und einen moderaten Aufwand', 'Der Gegenstand B78 ist wichtig genug, um in die Auswahl aufgenommen zu werden, ohne den Maximalaufwand zu überschreiten', 'Die Gegenstände C60 und A93 haben einen niedrigeren Aufwand und können daher ohne große Auswirkungen auf den Gesamtaufwand in die Auswahl aufgenommen werden', 'Ich denke, dass diese Kombination eine faire Lösung für beide Seiten ist'} 
	  } 
	   } 
	   } 
	 & \\ 
 

    \theutterance \stepcounter{utterance}  

    & \multicolumn{4}{p{0.6\linewidth}}{\cellcolor[rgb]{0.8,1,0.9}{%
	\makecell[{{p{\linewidth}}}]{% 
	  \tt {\tiny [A$\rangle$GM]}  
	 ZUSTIMMUNG: {'C28', 'C48', 'B78', 'C60', 'A93'}\\ \tt ARGUMENT: {'Ich denke, dass diese Kombination eine gute Balance zwischen Aufwand und Wichtigkeit bietet', 'Die Gegenstände C28 und C48 haben eine hohe Wichtigkeit und einen moderaten Aufwand', 'Der Gegenstand B78 ist wichtig genug, um in die Auswahl aufgenommen zu werden, ohne den Maximalaufwand zu überschreiten', 'Die Gegenstände C60 und A93 haben einen niedrigeren Aufwand und können daher ohne große Auswirkungen auf den Gesamtaufwand in die Auswahl aufgenommen werden', 'Ich denke, dass diese Kombination eine faire Lösung für beide Seiten ist und ich bin bereit, mich auf diese Auswahl zu einigen'} 
	  } 
	   } 
	   } 
	 & & \\ 
 

    \theutterance \stepcounter{utterance}  

    & & & \multicolumn{2}{p{0.3\linewidth}}{\cellcolor[rgb]{0.95,0.95,0.95}{%
	\makecell[{{p{\linewidth}}}]{% 
	  \tt {\tiny [GM$|$GM]}  
	 game successful 
	  } 
	   } 
	   } 
	 & & \\ 
 

    \theutterance \stepcounter{utterance}  

    & & & \multicolumn{2}{p{0.3\linewidth}}{\cellcolor[rgb]{0.95,0.95,0.95}{%
	\makecell[{{p{\linewidth}}}]{% 
	  \tt {\tiny [GM$|$GM]}  
	 end game 
	  } 
	   } 
	   } 
	 & & \\ 
 

\end{supertabular}
}

\end{document}
