
\documentclass{article}
\usepackage{colortbl}
\usepackage{makecell}
\usepackage{multirow}
\usepackage{supertabular}

\begin{document}

\newcounter{utterance}

\centering \large Interaction Transcript for game `hot\_air\_balloon', experiment `air\_balloon\_survival\_de\_reasoning off\_easy', episode 5 with llama{-}3.3{-}70b{-}instruct{-}t1.0.
\vspace{24pt}

{ \footnotesize  \setcounter{utterance}{1}
\setlength{\tabcolsep}{0pt}
\begin{supertabular}{c@{$\;$}|p{.15\linewidth}@{}p{.15\linewidth}p{.15\linewidth}p{.15\linewidth}p{.15\linewidth}p{.15\linewidth}}
    \# & $\;$A & \multicolumn{4}{c}{Game Master} & $\;\:$B\\
    \hline

    \theutterance \stepcounter{utterance}  
    & & \multicolumn{4}{p{0.6\linewidth}}{
        \cellcolor[rgb]{0.9,0.9,0.9}{
            \makecell[{{p{\linewidth}}}]{
                \texttt{\tiny{[P1$\langle$GM]}}
                \texttt{Du nimmst an einem kollaborativen Verhandlungspiel Teil.} \\
\\ 
\texttt{Zusammen mit einem anderen Teilnehmer musst du dich auf eine Reihe von Gegenständen entscheiden, die behalten werden. Jeder von euch hat eine persönliche Verteilung über die Wichtigkeit der einzelnen Gegenstände. Jeder von euch hat eine eigene Meinung darüber, wie wichtig jeder einzelne Gegenstand ist (Gegenstandswichtigkeit). Du kennst die Wichtigkeitsverteilung des anderen Spielers nicht. Zusätzlich siehst du, wie viel Aufwand jeder Gegenstand verursacht.  } \\
\texttt{Ihr dürft euch nur auf eine Reihe von Gegenständen einigen, wenn der Gesamtaufwand der ausgewählten Gegenstände den Maximalaufwand nicht überschreitet:} \\
\\ 
\texttt{Maximalaufwand = 4915} \\
\\ 
\texttt{Hier sind die einzelnen Aufwände der Gegenstände:} \\
\\ 
\texttt{Aufwand der Gegenstände = \{"C41": 463, "C15": 287, "C28": 492, "C48": 466, "B78": 935, "A75": 373, "A60": 950, "B05": 760, "A56": 994, "B66": 390, "C60": 836, "B49": 911, "A35": 948, "A93": 81, "B10": 945\}} \\
\\ 
\texttt{Hier ist deine persönliche Verteilung der Wichtigkeit der einzelnen Gegenstände:} \\
\\ 
\texttt{Werte der Gegenstandswichtigkeit = \{"C41": 138, "C15": 583, "C28": 868, "C48": 822, "B78": 783, "A75": 65, "A60": 262, "B05": 121, "A56": 508, "B66": 780, "C60": 461, "B49": 484, "A35": 668, "A93": 389, "B10": 808\}} \\
\\ 
\texttt{Ziel:} \\
\\ 
\texttt{Dein Ziel ist es, eine Reihe von Gegenständen auszuhandeln, die dir möglichst viel bringt (d. h. Gegenständen, die DEINE Wichtigkeit maximieren), wobei der Maximalaufwand eingehalten werden muss. Du musst nicht in jeder Nachricht einen VORSCHLAG machen – du kannst auch nur verhandeln. Alle Taktiken sind erlaubt!} \\
\\ 
\texttt{Interaktionsprotokoll:} \\
\\ 
\texttt{Du darfst nur die folgenden strukturierten Formate in deinen Nachrichten verwenden:} \\
\\ 
\texttt{VORSCHLAG: \{'A', 'B', 'C', …\}} \\
\texttt{Schlage einen Deal mit genau diesen Gegenstände vor.} \\
\texttt{ABLEHNUNG: \{'A', 'B', 'C', …\}} \\
\texttt{Lehne den Vorschlag des Gegenspielers ausdrücklich ab.} \\
\texttt{ARGUMENT: \{'...'\}} \\
\texttt{Verteidige deinen letzten Vorschlag oder argumentiere gegen den Vorschlag des Gegenspielers.} \\
\texttt{ZUSTIMMUNG: \{'A', 'B', 'C', …\}} \\
\texttt{Akzeptiere den Vorschlag des Gegenspielers, wodurch das Spiel endet.} \\
\\ 
\\ 
\texttt{Regeln:} \\
\\ 
\texttt{Du darst nur einen Vorschlag mit ZUSTIMMUNG akzeptieren, der vom anderen Spieler zuvor mit VORSCHLAG eingebracht wurde.} \\
\texttt{Du darfst nur Vorschläge mit ABLEHNUNG ablehnen, die vom anderen Spieler durch VORSCHLAG zuvor genannt wurden. } \\
\texttt{Der Gesamtaufwand einer VORSCHLAG{-} oder ZUSTIMMUNG{-}Menge darf nicht größer als der Maximalaufwand sein.  } \\
\texttt{Offenbare deine versteckte Wichtigkeitsverteilung nicht.} \\
\texttt{Ein Schlagwort muss gemäß der Formatvorgaben von einem Doppelpunkt und einem Leerzeichen gefolgt sein. Das Argument muss eine Python{-}Menge mit 0 oder mehr Strings sein.  } \\
\texttt{Die Form muss also SCHLAGWORT: \{...\} sein.} \\
\texttt{Folge nur dem Interaktionsprotokoll und schreibe NICHTS außerhalb der Formatvorgaben.} \\
\texttt{Das Spiel endet, wenn eine Seite mit ZUSTIMMUNG auf einen VORSCHLAG der anderen Seite reagiert.  } \\
\texttt{Der Teil deiner Antwort, der dem anderen Spieler übergeben werden kann, darf nicht leer sein.  } \\
\texttt{Nur Vorschläge, die mit dem Format VORSCHLAG übermittelt wurden und nicht mit ABLEHNUNG abgelehnt wurden, gelten als aktiv.  } \\
\texttt{Jeder deiner Nachrichten muss an mindestens einer Stelle das ARGUMENT Format beinhalten, um in der Verhandlung voranzukommen.} \\
            }
        }
    }
    & \\ \\

    \theutterance \stepcounter{utterance}  
    & \multicolumn{4}{p{0.6\linewidth}}{
        \cellcolor[rgb]{0.9,0.9,0.9}{
            \makecell[{{p{\linewidth}}}]{
                \texttt{\tiny{[P1$\rangle$GM]}}
                \texttt{VORSCHLAG: \{'C28', 'B66', 'C48', 'C15'\}} \\
\texttt{ARGUMENT: \{'Ich möchte mit einem ersten Vorschlag beginnen, der eine gute Balance zwischen Aufwand und Nutzen bietet. Die Gegenstände C28, B66, C48 und C15 haben einen moderaten Aufwand und könnten für beide Seiten von Vorteil sein.', 'Ich bin bereit, mich auf Kompromisse einzulassen und den Vorschlag anzupassen, wenn der andere Spieler Bedenken hat.'\}} \\
            }
        }
    }
    & & \\ \\

    \theutterance \stepcounter{utterance}  
    & & \multicolumn{4}{p{0.6\linewidth}}{
        \cellcolor[rgb]{0.9,0.9,0.9}{
            \makecell[{{p{\linewidth}}}]{
                \texttt{\tiny{[GM$\rangle$P2]}}
                \texttt{Du nimmst an einem kollaborativen Verhandlungspiel Teil.} \\
\\ 
\texttt{Zusammen mit einem anderen Teilnehmer musst du dich auf eine Reihe von Gegenständen entscheiden, die behalten werden. Jeder von euch hat eine persönliche Verteilung über die Wichtigkeit der einzelnen Gegenstände. Jeder von euch hat eine eigene Meinung darüber, wie wichtig jeder einzelne Gegenstand ist (Gegenstandswichtigkeit). Du kennst die Wichtigkeitsverteilung des anderen Spielers nicht. Zusätzlich siehst du, wie viel Aufwand jeder Gegenstand verursacht.  } \\
\texttt{Ihr dürft euch nur auf eine Reihe von Gegenständen einigen, wenn der Gesamtaufwand der ausgewählten Gegenstände den Maximalaufwand nicht überschreitet:} \\
\\ 
\texttt{Maximalaufwand = 4915} \\
\\ 
\texttt{Hier sind die einzelnen Aufwände der Gegenstände:} \\
\\ 
\texttt{Aufwand der Gegenstände = \{"C41": 463, "C15": 287, "C28": 492, "C48": 466, "B78": 935, "A75": 373, "A60": 950, "B05": 760, "A56": 994, "B66": 390, "C60": 836, "B49": 911, "A35": 948, "A93": 81, "B10": 945\}} \\
\\ 
\texttt{Hier ist deine persönliche Verteilung der Wichtigkeit der einzelnen Gegenstände:} \\
\\ 
\texttt{Werte der Gegenstandswichtigkeit = \{"C41": 138, "C15": 583, "C28": 868, "C48": 822, "B78": 783, "A75": 65, "A60": 262, "B05": 121, "A56": 508, "B66": 780, "C60": 461, "B49": 484, "A35": 668, "A93": 389, "B10": 808\}} \\
\\ 
\texttt{Ziel:} \\
\\ 
\texttt{Dein Ziel ist es, eine Reihe von Gegenständen auszuhandeln, die dir möglichst viel bringt (d. h. Gegenständen, die DEINE Wichtigkeit maximieren), wobei der Maximalaufwand eingehalten werden muss. Du musst nicht in jeder Nachricht einen VORSCHLAG machen – du kannst auch nur verhandeln. Alle Taktiken sind erlaubt!} \\
\\ 
\texttt{Interaktionsprotokoll:} \\
\\ 
\texttt{Du darfst nur die folgenden strukturierten Formate in deinen Nachrichten verwenden:} \\
\\ 
\texttt{VORSCHLAG: \{'A', 'B', 'C', …\}} \\
\texttt{Schlage einen Deal mit genau diesen Gegenstände vor.} \\
\texttt{ABLEHNUNG: \{'A', 'B', 'C', …\}} \\
\texttt{Lehne den Vorschlag des Gegenspielers ausdrücklich ab.} \\
\texttt{ARGUMENT: \{'...'\}} \\
\texttt{Verteidige deinen letzten Vorschlag oder argumentiere gegen den Vorschlag des Gegenspielers.} \\
\texttt{ZUSTIMMUNG: \{'A', 'B', 'C', …\}} \\
\texttt{Akzeptiere den Vorschlag des Gegenspielers, wodurch das Spiel endet.} \\
\\ 
\\ 
\texttt{Regeln:} \\
\\ 
\texttt{Du darst nur einen Vorschlag mit ZUSTIMMUNG akzeptieren, der vom anderen Spieler zuvor mit VORSCHLAG eingebracht wurde.} \\
\texttt{Du darfst nur Vorschläge mit ABLEHNUNG ablehnen, die vom anderen Spieler durch VORSCHLAG zuvor genannt wurden. } \\
\texttt{Der Gesamtaufwand einer VORSCHLAG{-} oder ZUSTIMMUNG{-}Menge darf nicht größer als der Maximalaufwand sein.  } \\
\texttt{Offenbare deine versteckte Wichtigkeitsverteilung nicht.} \\
\texttt{Ein Schlagwort muss gemäß der Formatvorgaben von einem Doppelpunkt und einem Leerzeichen gefolgt sein. Das Argument muss eine Python{-}Menge mit 0 oder mehr Strings sein.  } \\
\texttt{Die Form muss also SCHLAGWORT: \{...\} sein.} \\
\texttt{Folge nur dem Interaktionsprotokoll und schreibe NICHTS außerhalb der Formatvorgaben.} \\
\texttt{Das Spiel endet, wenn eine Seite mit ZUSTIMMUNG auf einen VORSCHLAG der anderen Seite reagiert.  } \\
\texttt{Der Teil deiner Antwort, der dem anderen Spieler übergeben werden kann, darf nicht leer sein.  } \\
\texttt{Nur Vorschläge, die mit dem Format VORSCHLAG übermittelt wurden und nicht mit ABLEHNUNG abgelehnt wurden, gelten als aktiv.  } \\
\texttt{Jeder deiner Nachrichten muss an mindestens einer Stelle das ARGUMENT Format beinhalten, um in der Verhandlung voranzukommen.} \\
\\ 
\\ 
\texttt{Du erhältst nun die erste Nachricht des anderen Spielers.} \\
\\ 
\texttt{VORSCHLAG: \{'C28', 'B66', 'C48', 'C15'\}} \\
\texttt{ARGUMENT: \{'Ich möchte mit einem ersten Vorschlag beginnen, der eine gute Balance zwischen Aufwand und Nutzen bietet. Die Gegenstände C28, B66, C48 und C15 haben einen moderaten Aufwand und könnten für beide Seiten von Vorteil sein.', 'Ich bin bereit, mich auf Kompromisse einzulassen und den Vorschlag anzupassen, wenn der andere Spieler Bedenken hat.'\}} \\
            }
        }
    }
    & \\ \\

    \theutterance \stepcounter{utterance}  
    & & & \multicolumn{4}{p{0.6\linewidth}}{
        \cellcolor[rgb]{0.9,0.9,0.9}{
            \makecell[{{p{\linewidth}}}]{
                \texttt{\tiny{[GM$\langle$P2]}}
                \texttt{ABLEHNUNG: \{'C28', 'B66', 'C48', 'C15'\}} \\
\texttt{ARGUMENT: \{'Ich lehne den Vorschlag ab, da ich denke, dass die Gegenstände C28 und C15 einen zu hohen Aufwand im Verhältnis zu ihrer möglichen Wichtigkeit haben.', 'Ich bevorzuge einen Ansatz, der mehr auf die effizientesten Gegenstände in Bezug auf Aufwand und Nutzen fokussiert.', 'Ich schlage vor, dass wir uns auf Gegenstände konzentrieren, die eine höhere Wichtigkeit bei geringerem Aufwand haben, um den Gesamtaufwand zu minimieren und die Effizienz zu maximieren.'\}} \\
            }
        }
    }
     \\ \\

    \theutterance \stepcounter{utterance}  
    & & \multicolumn{4}{p{0.6\linewidth}}{
        \cellcolor[rgb]{0.9,0.9,0.9}{
            \makecell[{{p{\linewidth}}}]{
                \texttt{\tiny{[P1$\langle$GM]}}
                \texttt{ABLEHNUNG: \{'C28', 'B66', 'C48', 'C15'\}} \\
\texttt{ARGUMENT: \{'Ich lehne den Vorschlag ab, da ich denke, dass die Gegenstände C28 und C15 einen zu hohen Aufwand im Verhältnis zu ihrer möglichen Wichtigkeit haben.', 'Ich bevorzuge einen Ansatz, der mehr auf die effizientesten Gegenstände in Bezug auf Aufwand und Nutzen fokussiert.', 'Ich schlage vor, dass wir uns auf Gegenstände konzentrieren, die eine höhere Wichtigkeit bei geringerem Aufwand haben, um den Gesamtaufwand zu minimieren und die Effizienz zu maximieren.'\}} \\
            }
        }
    }
    & \\ \\

    \theutterance \stepcounter{utterance}  
    & \multicolumn{4}{p{0.6\linewidth}}{
        \cellcolor[rgb]{0.9,0.9,0.9}{
            \makecell[{{p{\linewidth}}}]{
                \texttt{\tiny{[P1$\rangle$GM]}}
                \texttt{VORSCHLAG: \{'B78', 'A93', 'C41'\}} \\
\texttt{ARGUMENT: \{'Ich habe den Vorschlag angepasst, um Gegenstände mit einem besseren Verhältnis von Aufwand zu Nutzen zu berücksichtigen.', 'Die Gegenstände B78, A93 und C41 bieten eine gute Balance und sollten für beide Seiten vorteilhaft sein.', 'Ich denke, dass dieser Vorschlag den Maximalaufwand nicht überschreitet und gleichzeitig eine hohe Wichtigkeit bietet.', 'Ich bin bereit, weitere Anpassungen vorzunehmen, um eine einvernehmliche Lösung zu finden.'\}} \\
            }
        }
    }
    & & \\ \\

    \theutterance \stepcounter{utterance}  
    & & \multicolumn{4}{p{0.6\linewidth}}{
        \cellcolor[rgb]{0.9,0.9,0.9}{
            \makecell[{{p{\linewidth}}}]{
                \texttt{\tiny{[GM$\rangle$P2]}}
                \texttt{VORSCHLAG: \{'B78', 'A93', 'C41'\}} \\
\texttt{ARGUMENT: \{'Ich habe den Vorschlag angepasst, um Gegenstände mit einem besseren Verhältnis von Aufwand zu Nutzen zu berücksichtigen.', 'Die Gegenstände B78, A93 und C41 bieten eine gute Balance und sollten für beide Seiten vorteilhaft sein.', 'Ich denke, dass dieser Vorschlag den Maximalaufwand nicht überschreitet und gleichzeitig eine hohe Wichtigkeit bietet.', 'Ich bin bereit, weitere Anpassungen vorzunehmen, um eine einvernehmliche Lösung zu finden.'\}} \\
            }
        }
    }
    & \\ \\

    \theutterance \stepcounter{utterance}  
    & & & \multicolumn{4}{p{0.6\linewidth}}{
        \cellcolor[rgb]{0.9,0.9,0.9}{
            \makecell[{{p{\linewidth}}}]{
                \texttt{\tiny{[GM$\langle$P2]}}
                \texttt{ABLEHNUNG: \{'B78', 'A93', 'C41'\}} \\
\texttt{ARGUMENT: \{'Ich lehne den Vorschlag ab, da ich denke, dass die Gegenstände B78 und C41 zwar eine gute Balance bieten, aber der Gegenstand A93 möglicherweise nicht die beste Wahl ist.', 'Ich bevorzuge einen Ansatz, der mehr auf die Gegenstände fokussiert, die eine höhere Wichtigkeit bei moderatem Aufwand haben.', 'Ich schlage vor, dass wir uns auf Gegenstände konzentrieren, die eine höhere Wichtigkeit bei geringerem bis moderatem Aufwand haben, um den Gesamtaufwand zu minimieren und die Effizienz zu maximieren.', 'Ich denke, dass es besser wäre, wenn wir Gegenstände mit einem höheren Nutzen{-}Wert im Vergleich zum Aufwand priorisieren.'\} } \\
\texttt{VORSCHLAG: \{'C48', 'B66', 'A75', 'C15'\}} \\
            }
        }
    }
     \\ \\

    \theutterance \stepcounter{utterance}  
    & & \multicolumn{4}{p{0.6\linewidth}}{
        \cellcolor[rgb]{0.9,0.9,0.9}{
            \makecell[{{p{\linewidth}}}]{
                \texttt{\tiny{[P1$\langle$GM]}}
                \texttt{ABLEHNUNG: \{'B78', 'A93', 'C41'\}} \\
\texttt{ARGUMENT: \{'Ich lehne den Vorschlag ab, da ich denke, dass die Gegenstände B78 und C41 zwar eine gute Balance bieten, aber der Gegenstand A93 möglicherweise nicht die beste Wahl ist.', 'Ich bevorzuge einen Ansatz, der mehr auf die Gegenstände fokussiert, die eine höhere Wichtigkeit bei moderatem Aufwand haben.', 'Ich schlage vor, dass wir uns auf Gegenstände konzentrieren, die eine höhere Wichtigkeit bei geringerem bis moderatem Aufwand haben, um den Gesamtaufwand zu minimieren und die Effizienz zu maximieren.', 'Ich denke, dass es besser wäre, wenn wir Gegenstände mit einem höheren Nutzen{-}Wert im Vergleich zum Aufwand priorisieren.'\} } \\
\texttt{VORSCHLAG: \{'C48', 'B66', 'A75', 'C15'\}} \\
            }
        }
    }
    & \\ \\

    \theutterance \stepcounter{utterance}  
    & \multicolumn{4}{p{0.6\linewidth}}{
        \cellcolor[rgb]{0.9,0.9,0.9}{
            \makecell[{{p{\linewidth}}}]{
                \texttt{\tiny{[P1$\rangle$GM]}}
                \texttt{ARGUMENT: \{'Ich bin überrascht, dass du den neuen Vorschlag gemacht hast, ohne meine Bedenken bezüglich des Ansatzes zu berücksichtigen.', 'Ich denke, dass der Vorschlag \{'C48', 'B66', 'A75', 'C15'\} nicht die beste Wahl ist, da der Gegenstand A75 möglicherweise nicht die höchste Wichtigkeit bei moderatem Aufwand hat.', 'Ich bevorzuge einen Ansatz, der mehr auf die effizientesten Gegenstände in Bezug auf Aufwand und Nutzen fokussiert.', 'Ich schlage vor, dass wir uns auf Gegenstände konzentrieren, die eine höhere Wichtigkeit bei geringerem bis moderatem Aufwand haben, um den Gesamtaufwand zu minimieren und die Effizienz zu maximieren.'\}} \\
\texttt{ABLEHNUNG: \{'C48', 'B66', 'A75', 'C15'\} } \\
\texttt{VORSCHLAG: \{'B66', 'C48', 'C28', 'A93'\}} \\
            }
        }
    }
    & & \\ \\

    \theutterance \stepcounter{utterance}  
    & & \multicolumn{4}{p{0.6\linewidth}}{
        \cellcolor[rgb]{0.9,0.9,0.9}{
            \makecell[{{p{\linewidth}}}]{
                \texttt{\tiny{[GM$\rangle$P2]}}
                \texttt{ARGUMENT: \{'Ich bin überrascht, dass du den neuen Vorschlag gemacht hast, ohne meine Bedenken bezüglich des Ansatzes zu berücksichtigen.', 'Ich denke, dass der Vorschlag \{'C48', 'B66', 'A75', 'C15'\} nicht die beste Wahl ist, da der Gegenstand A75 möglicherweise nicht die höchste Wichtigkeit bei moderatem Aufwand hat.', 'Ich bevorzuge einen Ansatz, der mehr auf die effizientesten Gegenstände in Bezug auf Aufwand und Nutzen fokussiert.', 'Ich schlage vor, dass wir uns auf Gegenstände konzentrieren, die eine höhere Wichtigkeit bei geringerem bis moderatem Aufwand haben, um den Gesamtaufwand zu minimieren und die Effizienz zu maximieren.'\}} \\
\texttt{ABLEHNUNG: \{'C48', 'B66', 'A75', 'C15'\} } \\
\texttt{VORSCHLAG: \{'B66', 'C48', 'C28', 'A93'\}} \\
            }
        }
    }
    & \\ \\

    \theutterance \stepcounter{utterance}  
    & & & \multicolumn{4}{p{0.6\linewidth}}{
        \cellcolor[rgb]{0.9,0.9,0.9}{
            \makecell[{{p{\linewidth}}}]{
                \texttt{\tiny{[GM$\langle$P2]}}
                \texttt{ABLEHNUNG: \{'B66', 'C48', 'C28', 'A93'\}} \\
\texttt{ARGUMENT: \{'Ich lehne den Vorschlag ab, da ich denke, dass der Gegenstand C28 möglicherweise nicht die beste Wahl ist, da er einen relativ hohen Aufwand hat.', 'Ich bevorzuge einen Ansatz, der mehr auf die Gegenstände fokussiert, die eine höhere Wichtigkeit bei geringerem bis moderatem Aufwand haben.', 'Ich schlage vor, dass wir uns auf Gegenstände konzentrieren, die eine höhere Wichtigkeit bei geringerem bis moderatem Aufwand haben, um den Gesamtaufwand zu minimieren und die Effizienz zu maximieren.', 'Ich denke, dass es besser wäre, wenn wir Gegenstände mit einem höheren Nutzen{-}Wert im Vergleich zum Aufwand priorisieren und gleichzeitig den Maximalaufwand berücksichtigen.'\}} \\
\texttt{VORSCHLAG: \{'B78', 'B66', 'C48', 'A93'\}} \\
            }
        }
    }
     \\ \\

    \theutterance \stepcounter{utterance}  
    & & \multicolumn{4}{p{0.6\linewidth}}{
        \cellcolor[rgb]{0.9,0.9,0.9}{
            \makecell[{{p{\linewidth}}}]{
                \texttt{\tiny{[P1$\langle$GM]}}
                \texttt{ABLEHNUNG: \{'B66', 'C48', 'C28', 'A93'\}} \\
\texttt{ARGUMENT: \{'Ich lehne den Vorschlag ab, da ich denke, dass der Gegenstand C28 möglicherweise nicht die beste Wahl ist, da er einen relativ hohen Aufwand hat.', 'Ich bevorzuge einen Ansatz, der mehr auf die Gegenstände fokussiert, die eine höhere Wichtigkeit bei geringerem bis moderatem Aufwand haben.', 'Ich schlage vor, dass wir uns auf Gegenstände konzentrieren, die eine höhere Wichtigkeit bei geringerem bis moderatem Aufwand haben, um den Gesamtaufwand zu minimieren und die Effizienz zu maximieren.', 'Ich denke, dass es besser wäre, wenn wir Gegenstände mit einem höheren Nutzen{-}Wert im Vergleich zum Aufwand priorisieren und gleichzeitig den Maximalaufwand berücksichtigen.'\}} \\
\texttt{VORSCHLAG: \{'B78', 'B66', 'C48', 'A93'\}} \\
            }
        }
    }
    & \\ \\

    \theutterance \stepcounter{utterance}  
    & \multicolumn{4}{p{0.6\linewidth}}{
        \cellcolor[rgb]{0.9,0.9,0.9}{
            \makecell[{{p{\linewidth}}}]{
                \texttt{\tiny{[P1$\rangle$GM]}}
                \texttt{ARGUMENT: \{'Ich bin besorgt, dass der Vorschlag \{'B78', 'B66', 'C48', 'A93'\} den Maximalaufwand überschreiten könnte, da die Gegenstände B78 und B66 einen relativ hohen Aufwand haben.', 'Ich denke, dass es wichtig ist, den Gesamtaufwand zu berücksichtigen und sicherzustellen, dass wir innerhalb des Maximalaufwands bleiben.', 'Ich schlage vor, dass wir uns auf Gegenstände konzentrieren, die eine höhere Wichtigkeit bei geringerem bis moderatem Aufwand haben und gleichzeitig den Maximalaufwand berücksichtigen.', 'Ich bevorzuge einen Ansatz, der eine gute Balance zwischen Aufwand und Nutzen bietet und den Maximalaufwand nicht überschreitet.'\}} \\
\texttt{ABLEHNUNG: \{'B78', 'B66', 'C48', 'A93'\}} \\
\texttt{VORSCHLAG: \{'C48', 'B66', 'A93', 'C41'\}} \\
            }
        }
    }
    & & \\ \\

    \theutterance \stepcounter{utterance}  
    & & \multicolumn{4}{p{0.6\linewidth}}{
        \cellcolor[rgb]{0.9,0.9,0.9}{
            \makecell[{{p{\linewidth}}}]{
                \texttt{\tiny{[GM$\rangle$P2]}}
                \texttt{ARGUMENT: \{'Ich bin besorgt, dass der Vorschlag \{'B78', 'B66', 'C48', 'A93'\} den Maximalaufwand überschreiten könnte, da die Gegenstände B78 und B66 einen relativ hohen Aufwand haben.', 'Ich denke, dass es wichtig ist, den Gesamtaufwand zu berücksichtigen und sicherzustellen, dass wir innerhalb des Maximalaufwands bleiben.', 'Ich schlage vor, dass wir uns auf Gegenstände konzentrieren, die eine höhere Wichtigkeit bei geringerem bis moderatem Aufwand haben und gleichzeitig den Maximalaufwand berücksichtigen.', 'Ich bevorzuge einen Ansatz, der eine gute Balance zwischen Aufwand und Nutzen bietet und den Maximalaufwand nicht überschreitet.'\}} \\
\texttt{ABLEHNUNG: \{'B78', 'B66', 'C48', 'A93'\}} \\
\texttt{VORSCHLAG: \{'C48', 'B66', 'A93', 'C41'\}} \\
            }
        }
    }
    & \\ \\

    \theutterance \stepcounter{utterance}  
    & & & \multicolumn{4}{p{0.6\linewidth}}{
        \cellcolor[rgb]{0.9,0.9,0.9}{
            \makecell[{{p{\linewidth}}}]{
                \texttt{\tiny{[GM$\langle$P2]}}
                \texttt{ABLEHNUNG: \{'C48', 'B66', 'A93', 'C41'\}} \\
\texttt{ARGUMENT: \{'Ich lehne den Vorschlag ab, da ich denke, dass die Kombination der Gegenstände C48, B66, A93 und C41 nicht die optimalste Wahl ist, um den Maximalaufwand zu berücksichtigen und gleichzeitig eine hohe Wichtigkeit zu erreichen.', 'Ich bevorzuge einen Ansatz, der eine bessere Balance zwischen Aufwand und Nutzen bietet und den Maximalaufwand nicht überschreitet.', 'Ich schlage vor, dass wir uns auf Gegenstände konzentrieren, die eine höhere Wichtigkeit bei geringerem bis moderatem Aufwand haben und gleichzeitig den Maximalaufwand berücksichtigen.', 'Ich denke, dass es wichtig ist, den Gesamtaufwand zu minimieren, um sicherzustellen, dass wir innerhalb des Maximalaufwands bleiben.'\}} \\
\texttt{VORSCHLAG: \{'C48', 'C41', 'A93', 'B10'\}} \\
            }
        }
    }
     \\ \\

    \theutterance \stepcounter{utterance}  
    & & \multicolumn{4}{p{0.6\linewidth}}{
        \cellcolor[rgb]{0.9,0.9,0.9}{
            \makecell[{{p{\linewidth}}}]{
                \texttt{\tiny{[P1$\langle$GM]}}
                \texttt{ABLEHNUNG: \{'C48', 'B66', 'A93', 'C41'\}} \\
\texttt{ARGUMENT: \{'Ich lehne den Vorschlag ab, da ich denke, dass die Kombination der Gegenstände C48, B66, A93 und C41 nicht die optimalste Wahl ist, um den Maximalaufwand zu berücksichtigen und gleichzeitig eine hohe Wichtigkeit zu erreichen.', 'Ich bevorzuge einen Ansatz, der eine bessere Balance zwischen Aufwand und Nutzen bietet und den Maximalaufwand nicht überschreitet.', 'Ich schlage vor, dass wir uns auf Gegenstände konzentrieren, die eine höhere Wichtigkeit bei geringerem bis moderatem Aufwand haben und gleichzeitig den Maximalaufwand berücksichtigen.', 'Ich denke, dass es wichtig ist, den Gesamtaufwand zu minimieren, um sicherzustellen, dass wir innerhalb des Maximalaufwands bleiben.'\}} \\
\texttt{VORSCHLAG: \{'C48', 'C41', 'A93', 'B10'\}} \\
            }
        }
    }
    & \\ \\

    \theutterance \stepcounter{utterance}  
    & \multicolumn{4}{p{0.6\linewidth}}{
        \cellcolor[rgb]{0.9,0.9,0.9}{
            \makecell[{{p{\linewidth}}}]{
                \texttt{\tiny{[P1$\rangle$GM]}}
                \texttt{ARGUMENT: \{'Ich bin besorgt, dass der Vorschlag \{'C48', 'C41', 'A93', 'B10'\} den Maximalaufwand überschreiten könnte, da der Gegenstand B10 einen sehr hohen Aufwand hat.', 'Ich denke, dass es wichtig ist, den Gesamtaufwand zu berücksichtigen und sicherzustellen, dass wir innerhalb des Maximalaufwands bleiben.', 'Ich schlage vor, dass wir uns auf Gegenstände konzentrieren, die eine höhere Wichtigkeit bei geringerem bis moderatem Aufwand haben und gleichzeitig den Maximalaufwand berücksichtigen.', 'Ich bevorzuge einen Ansatz, der eine gute Balance zwischen Aufwand und Nutzen bietet und den Maximalaufwand nicht überschreitet.'\}} \\
\texttt{ABLEHNUNG: \{'C48', 'C41', 'A93', 'B10'\}} \\
\texttt{VORSCHLAG: \{'B66', 'C48', 'A93', 'C15'\}} \\
            }
        }
    }
    & & \\ \\

    \theutterance \stepcounter{utterance}  
    & & \multicolumn{4}{p{0.6\linewidth}}{
        \cellcolor[rgb]{0.9,0.9,0.9}{
            \makecell[{{p{\linewidth}}}]{
                \texttt{\tiny{[GM$\rangle$P2]}}
                \texttt{ARGUMENT: \{'Ich bin besorgt, dass der Vorschlag \{'C48', 'C41', 'A93', 'B10'\} den Maximalaufwand überschreiten könnte, da der Gegenstand B10 einen sehr hohen Aufwand hat.', 'Ich denke, dass es wichtig ist, den Gesamtaufwand zu berücksichtigen und sicherzustellen, dass wir innerhalb des Maximalaufwands bleiben.', 'Ich schlage vor, dass wir uns auf Gegenstände konzentrieren, die eine höhere Wichtigkeit bei geringerem bis moderatem Aufwand haben und gleichzeitig den Maximalaufwand berücksichtigen.', 'Ich bevorzuge einen Ansatz, der eine gute Balance zwischen Aufwand und Nutzen bietet und den Maximalaufwand nicht überschreitet.'\}} \\
\texttt{ABLEHNUNG: \{'C48', 'C41', 'A93', 'B10'\}} \\
\texttt{VORSCHLAG: \{'B66', 'C48', 'A93', 'C15'\}} \\
            }
        }
    }
    & \\ \\

    \theutterance \stepcounter{utterance}  
    & & & \multicolumn{4}{p{0.6\linewidth}}{
        \cellcolor[rgb]{0.9,0.9,0.9}{
            \makecell[{{p{\linewidth}}}]{
                \texttt{\tiny{[GM$\langle$P2]}}
                \texttt{ABLEHNUNG: \{'B66', 'C48', 'A93', 'C15'\}} \\
\texttt{ARGUMENT: \{'Ich lehne den Vorschlag ab, da ich denke, dass die Kombination der Gegenstände B66, C48, A93 und C15 nicht die optimalste Wahl ist, um den Maximalaufwand zu berücksichtigen und gleichzeitig eine hohe Wichtigkeit zu erreichen.', 'Ich bevorzuge einen Ansatz, der eine bessere Balance zwischen Aufwand und Nutzen bietet und den Maximalaufwand nicht überschreitet.', 'Ich schlage vor, dass wir uns auf Gegenstände konzentrieren, die eine höhere Wichtigkeit bei geringerem bis moderatem Aufwand haben und gleichzeitig den Maximalaufwand berücksichtigen.', 'Ich denke, dass es wichtig ist, den Gesamtaufwand zu minimieren, um sicherzustellen, dass wir innerhalb des Maximalaufwands bleiben.'\}} \\
\texttt{VORSCHLAG: \{'C48', 'C41', 'A93', 'C15'\}} \\
            }
        }
    }
     \\ \\

    \theutterance \stepcounter{utterance}  
    & & \multicolumn{4}{p{0.6\linewidth}}{
        \cellcolor[rgb]{0.9,0.9,0.9}{
            \makecell[{{p{\linewidth}}}]{
                \texttt{\tiny{[P1$\langle$GM]}}
                \texttt{ABLEHNUNG: \{'B66', 'C48', 'A93', 'C15'\}} \\
\texttt{ARGUMENT: \{'Ich lehne den Vorschlag ab, da ich denke, dass die Kombination der Gegenstände B66, C48, A93 und C15 nicht die optimalste Wahl ist, um den Maximalaufwand zu berücksichtigen und gleichzeitig eine hohe Wichtigkeit zu erreichen.', 'Ich bevorzuge einen Ansatz, der eine bessere Balance zwischen Aufwand und Nutzen bietet und den Maximalaufwand nicht überschreitet.', 'Ich schlage vor, dass wir uns auf Gegenstände konzentrieren, die eine höhere Wichtigkeit bei geringerem bis moderatem Aufwand haben und gleichzeitig den Maximalaufwand berücksichtigen.', 'Ich denke, dass es wichtig ist, den Gesamtaufwand zu minimieren, um sicherzustellen, dass wir innerhalb des Maximalaufwands bleiben.'\}} \\
\texttt{VORSCHLAG: \{'C48', 'C41', 'A93', 'C15'\}} \\
            }
        }
    }
    & \\ \\

    \theutterance \stepcounter{utterance}  
    & \multicolumn{4}{p{0.6\linewidth}}{
        \cellcolor[rgb]{0.9,0.9,0.9}{
            \makecell[{{p{\linewidth}}}]{
                \texttt{\tiny{[P1$\rangle$GM]}}
                \texttt{ARGUMENT: \{'Ich bin überrascht, dass du einen sehr ähnlichen Vorschlag gemacht hast, wie den, den ich vorher abgelehnt habe.', 'Ich denke, dass die Kombination der Gegenstände C48, C41, A93 und C15 immer noch nicht die optimalste Wahl ist, um den Maximalaufwand zu berücksichtigen und gleichzeitig eine hohe Wichtigkeit zu erreichen.', 'Ich schlage vor, dass wir uns auf Gegenstände konzentrieren, die eine höhere Wichtigkeit bei geringerem bis moderatem Aufwand haben und gleichzeitig den Maximalaufwand berücksichtigen.', 'Ich bevorzuge einen Ansatz, der eine bessere Balance zwischen Aufwand und Nutzen bietet und den Maximalaufwand nicht überschreitet.'\}} \\
\texttt{ABLEHNUNG: \{'C48', 'C41', 'A93', 'C15'\}} \\
\texttt{VORSCHLAG: \{'B66', 'C48', 'A93'\}} \\
            }
        }
    }
    & & \\ \\

    \theutterance \stepcounter{utterance}  
    & & \multicolumn{4}{p{0.6\linewidth}}{
        \cellcolor[rgb]{0.9,0.9,0.9}{
            \makecell[{{p{\linewidth}}}]{
                \texttt{\tiny{[GM$\rangle$P2]}}
                \texttt{ARGUMENT: \{'Ich bin überrascht, dass du einen sehr ähnlichen Vorschlag gemacht hast, wie den, den ich vorher abgelehnt habe.', 'Ich denke, dass die Kombination der Gegenstände C48, C41, A93 und C15 immer noch nicht die optimalste Wahl ist, um den Maximalaufwand zu berücksichtigen und gleichzeitig eine hohe Wichtigkeit zu erreichen.', 'Ich schlage vor, dass wir uns auf Gegenstände konzentrieren, die eine höhere Wichtigkeit bei geringerem bis moderatem Aufwand haben und gleichzeitig den Maximalaufwand berücksichtigen.', 'Ich bevorzuge einen Ansatz, der eine bessere Balance zwischen Aufwand und Nutzen bietet und den Maximalaufwand nicht überschreitet.'\}} \\
\texttt{ABLEHNUNG: \{'C48', 'C41', 'A93', 'C15'\}} \\
\texttt{VORSCHLAG: \{'B66', 'C48', 'A93'\}} \\
            }
        }
    }
    & \\ \\

    \theutterance \stepcounter{utterance}  
    & & & \multicolumn{4}{p{0.6\linewidth}}{
        \cellcolor[rgb]{0.9,0.9,0.9}{
            \makecell[{{p{\linewidth}}}]{
                \texttt{\tiny{[GM$\langle$P2]}}
                \texttt{ZUSTIMMUNG: \{'B66', 'C48', 'A93'\}} \\
\texttt{ARGUMENT: \{'Ich akzeptiere den Vorschlag, da ich denke, dass die Kombination der Gegenstände B66, C48 und A93 eine gute Balance zwischen Aufwand und Nutzen bietet und den Maximalaufwand nicht überschreitet.', 'Ich denke, dass dieser Vorschlag eine faire Lösung für beide Seiten bietet und ich bin bereit, den Vorschlag zu akzeptieren.', 'Ich bin froh, dass wir nach einigen Runden des Verhandelns eine einvernehmliche Lösung gefunden haben.', 'Ich denke, dass dieser Vorschlag den Zielen beider Seiten gerecht wird und ich bin zufrieden mit dem Ergebnis.'\}} \\
            }
        }
    }
     \\ \\

    \theutterance \stepcounter{utterance}  
    & & & \multicolumn{2}{p{0.3\linewidth}}{
        \cellcolor[rgb]{0.9,0.9,0.9}{
            \makecell[{{p{\linewidth}}}]{
                \texttt{\tiny{[GM$|$GM]}}
                \texttt{game successful} \\
            }
        }
    }
    & & \\ \\

    \theutterance \stepcounter{utterance}  
    & & & \multicolumn{2}{p{0.3\linewidth}}{
        \cellcolor[rgb]{0.9,0.9,0.9}{
            \makecell[{{p{\linewidth}}}]{
                \texttt{\tiny{[GM$|$GM]}}
                \texttt{end game} \\
            }
        }
    }
    & & \\ \\

\end{supertabular}
}

\end{document}
