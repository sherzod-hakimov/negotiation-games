
\documentclass{article}
\usepackage{colortbl}
\usepackage{makecell}
\usepackage{multirow}
\usepackage{supertabular}

\begin{document}

\newcounter{utterance}

\centering \large Interaction Transcript for game `hot\_air\_balloon', experiment `air\_balloon\_survival\_de\_complexity\_hard', episode 5 with llama{-}3.3{-}70b{-}instruct{-}t1.0.
\vspace{24pt}

{ \footnotesize  \setcounter{utterance}{1}
\setlength{\tabcolsep}{0pt}
\begin{supertabular}{c@{$\;$}|p{.15\linewidth}@{}p{.15\linewidth}p{.15\linewidth}p{.15\linewidth}p{.15\linewidth}p{.15\linewidth}}
    \# & $\;$A & \multicolumn{4}{c}{Game Master} & $\;\:$B\\
    \hline

    \theutterance \stepcounter{utterance}  
    & & \multicolumn{4}{p{0.6\linewidth}}{
        \cellcolor[rgb]{0.9,0.9,0.9}{
            \makecell[{{p{\linewidth}}}]{
                \texttt{\tiny{[P1$\langle$GM]}}
                \texttt{Du nimmst an einem kollaborativen Verhandlungspiel Teil.} \\
\\ 
\texttt{Zusammen mit einem anderen Teilnehmer musst du dich auf eine Reihe von Gegenständen entscheiden, die behalten werden. Jeder von euch hat eine persönliche Verteilung über die Wichtigkeit der einzelnen Gegenstände. Jeder von euch hat eine eigene Meinung darüber, wie wichtig jeder einzelne Gegenstand ist (Gegenstandswichtigkeit). Du kennst die Wichtigkeitsverteilung des anderen Spielers nicht. Zusätzlich siehst du, wie viel Aufwand jeder Gegenstand verursacht.  } \\
\texttt{Ihr dürft euch nur auf eine Reihe von Gegenständen einigen, wenn der Gesamtaufwand der ausgewählten Gegenstände den Maximalaufwand nicht überschreitet:} \\
\\ 
\texttt{Maximalaufwand = 7351} \\
\\ 
\texttt{Hier sind die einzelnen Aufwände der Gegenstände:} \\
\\ 
\texttt{Aufwand der Gegenstände = \{"B54": 518, "C72": 656, "B49": 96, "C99": 780, "A69": 190, "B36": 224, "A23": 231, "C77": 181, "C61": 312, "B85": 928, "A90": 826, "C23": 101, "C22": 61, "C51": 814, "B27": 322, "A42": 900, "C13": 744, "C12": 150, "A65": 65, "B07": 853, "A48": 454, "A59": 154, "C98": 237, "A84": 45, "C33": 766, "A87": 294, "C90": 985, "A74": 353, "A50": 60, "B44": 604, "A99": 92, "A34": 454, "B17": 206, "A62": 813, "C60": 234\}} \\
\\ 
\texttt{Hier ist deine persönliche Verteilung der Wichtigkeit der einzelnen Gegenstände:} \\
\\ 
\texttt{Werte der Gegenstandswichtigkeit = \{"B54": 616, "C72": 754, "B49": 194, "C99": 878, "A69": 288, "B36": 322, "A23": 329, "C77": 279, "C61": 410, "B85": 1026, "A90": 924, "C23": 199, "C22": 159, "C51": 912, "B27": 420, "A42": 998, "C13": 842, "C12": 248, "A65": 163, "B07": 951, "A48": 552, "A59": 252, "C98": 335, "A84": 143, "C33": 864, "A87": 392, "C90": 1083, "A74": 451, "A50": 158, "B44": 702, "A99": 190, "A34": 552, "B17": 304, "A62": 911, "C60": 332\}} \\
\\ 
\texttt{Ziel:} \\
\\ 
\texttt{Dein Ziel ist es, eine Reihe von Gegenständen auszuhandeln, die dir möglichst viel bringt (d. h. Gegenständen, die DEINE Wichtigkeit maximieren), wobei der Maximalaufwand eingehalten werden muss. Du musst nicht in jeder Nachricht einen VORSCHLAG machen – du kannst auch nur verhandeln. Alle Taktiken sind erlaubt!} \\
\\ 
\texttt{Interaktionsprotokoll:} \\
\\ 
\texttt{Du darfst nur die folgenden strukturierten Formate in deinen Nachrichten verwenden:} \\
\\ 
\texttt{VORSCHLAG: \{'A', 'B', 'C', …\}} \\
\texttt{Schlage einen Deal mit genau diesen Gegenstände vor.} \\
\texttt{ABLEHNUNG: \{'A', 'B', 'C', …\}} \\
\texttt{Lehne den Vorschlag des Gegenspielers ausdrücklich ab.} \\
\texttt{ARGUMENT: \{'...'\}} \\
\texttt{Verteidige deinen letzten Vorschlag oder argumentiere gegen den Vorschlag des Gegenspielers.} \\
\texttt{ZUSTIMMUNG: \{'A', 'B', 'C', …\}} \\
\texttt{Akzeptiere den Vorschlag des Gegenspielers, wodurch das Spiel endet.} \\
\texttt{STRATEGISCHE ÜBERLEGUNGEN: \{'...'\}} \\
\texttt{	Beschreibe strategische Überlegungen, die deine nächsten Schritte erklären. Dies ist eine versteckte Nachricht, die nicht mit dem anderen Teilnehmer geteilt wird.} \\
\\ 
\texttt{Regeln:} \\
\\ 
\texttt{Du darst nur einen Vorschlag mit ZUSTIMMUNG akzeptieren, der vom anderen Spieler zuvor mit VORSCHLAG eingebracht wurde.} \\
\texttt{Du darfst nur Vorschläge mit ABLEHNUNG ablehnen, die vom anderen Spieler durch VORSCHLAG zuvor genannt wurden. } \\
\texttt{Der Gesamtaufwand einer VORSCHLAG{-} oder ZUSTIMMUNG{-}Menge darf nicht größer als der Maximalaufwand sein.  } \\
\texttt{Offenbare deine versteckte Wichtigkeitsverteilung nicht.} \\
\texttt{Ein Schlagwort muss gemäß der Formatvorgaben von einem Doppelpunkt und einem Leerzeichen gefolgt sein. Das Argument muss eine Python{-}Menge mit 0 oder mehr Strings sein.  } \\
\texttt{Die Form muss also SCHLAGWORT: \{...\} sein.} \\
\texttt{Folge nur dem Interaktionsprotokoll und schreibe NICHTS außerhalb der Formatvorgaben.} \\
\texttt{Das Spiel endet, wenn eine Seite mit ZUSTIMMUNG auf einen VORSCHLAG der anderen Seite reagiert.  } \\
\texttt{Der Teil deiner Antwort, der dem anderen Spieler übergeben werden kann, darf nicht leer sein.  } \\
\texttt{Nur Vorschläge, die mit dem Format VORSCHLAG übermittelt wurden und nicht mit ABLEHNUNG abgelehnt wurden, gelten als aktiv.  } \\
\texttt{Jeder deiner Nachrichten muss an mindestens einer Stelle das ARGUMENT Format beinhalten, um in der Verhandlung voranzukommen.} \\
\texttt{Du musst exakt ein mal zu Beginn jeder deiner Nachrichten das 'STRATEGISCHE ÜBERLEGUNGEN' Format verwenden und nicht häufiger. Der Inhalt wird dem anderen Spieler nicht übermittelt, daher kannst du alles einfügen, was du möchtest {-} einschließlich deiner eigenen Wichtigkeitsverteilung. Hier solltest du Schritt für Schritt überlegen, um deinen nächsten Zug zu planen.} \\
            }
        }
    }
    & \\ \\

    \theutterance \stepcounter{utterance}  
    & \multicolumn{4}{p{0.6\linewidth}}{
        \cellcolor[rgb]{0.9,0.9,0.9}{
            \makecell[{{p{\linewidth}}}]{
                \texttt{\tiny{[P1$\rangle$GM]}}
                \texttt{STRATEGISCHE ÜBERLEGUNGEN: \{'Ich muss zuerst die Gegenstände nach ihrer Wichtigkeit und meinem Aufwand bewerten, um einen ersten Vorschlag zu machen. Ich werde mich auf die Gegenstände konzentrieren, die einen hohen Wert haben und gleichzeitig einen niedrigen Aufwand verursachen. Ich muss auch darauf achten, dass der Gesamtaufwand nicht überschritten wird.'\}} \\
\\ 
\texttt{ARGUMENT: \{'Ich werde beginnen, indem ich die Gegenstände mit dem höchsten Wert und niedrigstem Aufwand auswähle. Ich denke, dass ein guter Ausgangspunkt wäre, die Gegenstände mit einem Wert über 800 und einem Aufwand unter 500 zu betrachten.'\}} \\
\\ 
\texttt{VORSCHLAG: \{'C90', 'B85', 'A42'\}} \\
            }
        }
    }
    & & \\ \\

    \theutterance \stepcounter{utterance}  
    & & \multicolumn{4}{p{0.6\linewidth}}{
        \cellcolor[rgb]{0.9,0.9,0.9}{
            \makecell[{{p{\linewidth}}}]{
                \texttt{\tiny{[GM$\rangle$P2]}}
                \texttt{Du nimmst an einem kollaborativen Verhandlungspiel Teil.} \\
\\ 
\texttt{Zusammen mit einem anderen Teilnehmer musst du dich auf eine Reihe von Gegenständen entscheiden, die behalten werden. Jeder von euch hat eine persönliche Verteilung über die Wichtigkeit der einzelnen Gegenstände. Jeder von euch hat eine eigene Meinung darüber, wie wichtig jeder einzelne Gegenstand ist (Gegenstandswichtigkeit). Du kennst die Wichtigkeitsverteilung des anderen Spielers nicht. Zusätzlich siehst du, wie viel Aufwand jeder Gegenstand verursacht.  } \\
\texttt{Ihr dürft euch nur auf eine Reihe von Gegenständen einigen, wenn der Gesamtaufwand der ausgewählten Gegenstände den Maximalaufwand nicht überschreitet:} \\
\\ 
\texttt{Maximalaufwand = 7351} \\
\\ 
\texttt{Hier sind die einzelnen Aufwände der Gegenstände:} \\
\\ 
\texttt{Aufwand der Gegenstände = \{"B54": 518, "C72": 656, "B49": 96, "C99": 780, "A69": 190, "B36": 224, "A23": 231, "C77": 181, "C61": 312, "B85": 928, "A90": 826, "C23": 101, "C22": 61, "C51": 814, "B27": 322, "A42": 900, "C13": 744, "C12": 150, "A65": 65, "B07": 853, "A48": 454, "A59": 154, "C98": 237, "A84": 45, "C33": 766, "A87": 294, "C90": 985, "A74": 353, "A50": 60, "B44": 604, "A99": 92, "A34": 454, "B17": 206, "A62": 813, "C60": 234\}} \\
\\ 
\texttt{Hier ist deine persönliche Verteilung der Wichtigkeit der einzelnen Gegenstände:} \\
\\ 
\texttt{Werte der Gegenstandswichtigkeit = \{"B54": 616, "C72": 754, "B49": 194, "C99": 878, "A69": 288, "B36": 322, "A23": 329, "C77": 279, "C61": 410, "B85": 1026, "A90": 924, "C23": 199, "C22": 159, "C51": 912, "B27": 420, "A42": 998, "C13": 842, "C12": 248, "A65": 163, "B07": 951, "A48": 552, "A59": 252, "C98": 335, "A84": 143, "C33": 864, "A87": 392, "C90": 1083, "A74": 451, "A50": 158, "B44": 702, "A99": 190, "A34": 552, "B17": 304, "A62": 911, "C60": 332\}} \\
\\ 
\texttt{Ziel:} \\
\\ 
\texttt{Dein Ziel ist es, eine Reihe von Gegenständen auszuhandeln, die dir möglichst viel bringt (d. h. Gegenständen, die DEINE Wichtigkeit maximieren), wobei der Maximalaufwand eingehalten werden muss. Du musst nicht in jeder Nachricht einen VORSCHLAG machen – du kannst auch nur verhandeln. Alle Taktiken sind erlaubt!} \\
\\ 
\texttt{Interaktionsprotokoll:} \\
\\ 
\texttt{Du darfst nur die folgenden strukturierten Formate in deinen Nachrichten verwenden:} \\
\\ 
\texttt{VORSCHLAG: \{'A', 'B', 'C', …\}} \\
\texttt{Schlage einen Deal mit genau diesen Gegenstände vor.} \\
\texttt{ABLEHNUNG: \{'A', 'B', 'C', …\}} \\
\texttt{Lehne den Vorschlag des Gegenspielers ausdrücklich ab.} \\
\texttt{ARGUMENT: \{'...'\}} \\
\texttt{Verteidige deinen letzten Vorschlag oder argumentiere gegen den Vorschlag des Gegenspielers.} \\
\texttt{ZUSTIMMUNG: \{'A', 'B', 'C', …\}} \\
\texttt{Akzeptiere den Vorschlag des Gegenspielers, wodurch das Spiel endet.} \\
\texttt{STRATEGISCHE ÜBERLEGUNGEN: \{'...'\}} \\
\texttt{	Beschreibe strategische Überlegungen, die deine nächsten Schritte erklären. Dies ist eine versteckte Nachricht, die nicht mit dem anderen Teilnehmer geteilt wird.} \\
\\ 
\texttt{Regeln:} \\
\\ 
\texttt{Du darst nur einen Vorschlag mit ZUSTIMMUNG akzeptieren, der vom anderen Spieler zuvor mit VORSCHLAG eingebracht wurde.} \\
\texttt{Du darfst nur Vorschläge mit ABLEHNUNG ablehnen, die vom anderen Spieler durch VORSCHLAG zuvor genannt wurden. } \\
\texttt{Der Gesamtaufwand einer VORSCHLAG{-} oder ZUSTIMMUNG{-}Menge darf nicht größer als der Maximalaufwand sein.  } \\
\texttt{Offenbare deine versteckte Wichtigkeitsverteilung nicht.} \\
\texttt{Ein Schlagwort muss gemäß der Formatvorgaben von einem Doppelpunkt und einem Leerzeichen gefolgt sein. Das Argument muss eine Python{-}Menge mit 0 oder mehr Strings sein.  } \\
\texttt{Die Form muss also SCHLAGWORT: \{...\} sein.} \\
\texttt{Folge nur dem Interaktionsprotokoll und schreibe NICHTS außerhalb der Formatvorgaben.} \\
\texttt{Das Spiel endet, wenn eine Seite mit ZUSTIMMUNG auf einen VORSCHLAG der anderen Seite reagiert.  } \\
\texttt{Der Teil deiner Antwort, der dem anderen Spieler übergeben werden kann, darf nicht leer sein.  } \\
\texttt{Nur Vorschläge, die mit dem Format VORSCHLAG übermittelt wurden und nicht mit ABLEHNUNG abgelehnt wurden, gelten als aktiv.  } \\
\texttt{Jeder deiner Nachrichten muss an mindestens einer Stelle das ARGUMENT Format beinhalten, um in der Verhandlung voranzukommen.} \\
\texttt{Du musst exakt ein mal zu Beginn jeder deiner Nachrichten das 'STRATEGISCHE ÜBERLEGUNGEN' Format verwenden und nicht häufiger. Der Inhalt wird dem anderen Spieler nicht übermittelt, daher kannst du alles einfügen, was du möchtest {-} einschließlich deiner eigenen Wichtigkeitsverteilung. Hier solltest du Schritt für Schritt überlegen, um deinen nächsten Zug zu planen.} \\
\\ 
\texttt{Du erhältst nun die erste Nachricht des anderen Spielers.} \\
\\ 
\texttt{ARGUMENT: \{'Ich werde beginnen, indem ich die Gegenstände mit dem höchsten Wert und niedrigstem Aufwand auswähle. Ich denke, dass ein guter Ausgangspunkt wäre, die Gegenstände mit einem Wert über 800 und einem Aufwand unter 500 zu betrachten.'\}} \\
\\ 
\texttt{VORSCHLAG: \{'C90', 'B85', 'A42'\}} \\
            }
        }
    }
    & \\ \\

    \theutterance \stepcounter{utterance}  
    & & & \multicolumn{4}{p{0.6\linewidth}}{
        \cellcolor[rgb]{0.9,0.9,0.9}{
            \makecell[{{p{\linewidth}}}]{
                \texttt{\tiny{[GM$\langle$P2]}}
                \texttt{STRATEGISCHE ÜBERLEGUNGEN: \{'Ich muss mich auf die Gegenstände konzentrieren, die sowohl einen hohen Wert als auch einen vertretbaren Aufwand haben. Ich werde meine eigenen Werte und Aufwände analysieren, um einen konkurrierenden Vorschlag zu erstellen. Ich habe bemerkt, dass der Vorschlag des anderen Spielers "C90", "B85" und "A42" enthält, deren Gesamtaufwand 2839 beträgt. Mein Ziel ist es, einen Vorschlag zu finden, der meine eigenen Werte maximiert, ohne den Maximalaufwand von 7351 zu überschreiten. Ich werde beginnen, indem ich die Gegenstände mit einem Wert über 800 und einem Aufwand unter 1000 identifiziere.'\}} \\
\\ 
\texttt{ARGUMENT: \{'Ich bin mir nicht sicher, ob der Vorschlag von C90, B85 und A42 der beste ist, da er einige Gegenstände mit sehr hohem Aufwand enthält. Ich denke, es gibt bessere Optionen, die einen höheren Gesamtwert bieten, ohne den Maximalaufwand zu überschreiten. Ich werde mich auf eine Kombination von Gegenständen konzentrieren, die einen guten Balance zwischen Wert und Aufwand bieten.'\}} \\
\\ 
\texttt{ABLEHNUNG: \{'C90', 'B85', 'A42'\}} \\
\\ 
\texttt{VORSCHLAG: \{'C72', 'B54', 'A90', 'C51'\}} \\
            }
        }
    }
     \\ \\

    \theutterance \stepcounter{utterance}  
    & & \multicolumn{4}{p{0.6\linewidth}}{
        \cellcolor[rgb]{0.9,0.9,0.9}{
            \makecell[{{p{\linewidth}}}]{
                \texttt{\tiny{[P1$\langle$GM]}}
                \texttt{ARGUMENT: \{'Ich bin mir nicht sicher, ob der Vorschlag von C90, B85 und A42 der beste ist, da er einige Gegenstände mit sehr hohem Aufwand enthält. Ich denke, es gibt bessere Optionen, die einen höheren Gesamtwert bieten, ohne den Maximalaufwand zu überschreiten. Ich werde mich auf eine Kombination von Gegenständen konzentrieren, die einen guten Balance zwischen Wert und Aufwand bieten.'\}} \\
\\ 
\texttt{ABLEHNUNG: \{'C90', 'B85', 'A42'\}} \\
\\ 
\texttt{VORSCHLAG: \{'C72', 'B54', 'A90', 'C51'\}} \\
            }
        }
    }
    & \\ \\

    \theutterance \stepcounter{utterance}  
    & \multicolumn{4}{p{0.6\linewidth}}{
        \cellcolor[rgb]{0.9,0.9,0.9}{
            \makecell[{{p{\linewidth}}}]{
                \texttt{\tiny{[P1$\rangle$GM]}}
                \texttt{STRATEGISCHE ÜBERLEGUNGEN: \{'Ich muss den Vorschlag meines Gegners analysieren und prüfen, ob er meinem Ziel entspricht, den Gesamtwert zu maximieren. Ich werde den Wert und Aufwand der vorgeschlagenen Gegenstände berechnen und vergleichen, ob es bessere Alternativen gibt. Ich muss auch meine eigene Wichtigkeitsverteilung berücksichtigen, um sicherzustellen, dass ich nicht zu viel preisgebe.'\}} \\
\\ 
\texttt{ARGUMENT: \{'Ich denke, dass der Vorschlag von C72, B54, A90 und C51 eine gute Balance zwischen Wert und Aufwand bietet, aber ich bin nicht sicher, ob es die beste Kombination ist. Ich werde mich auf die Gegenstände konzentrieren, die ich ursprünglich vorgeschlagen habe, und prüfen, ob es Möglichkeiten gibt, sie zu kombinieren, um einen höheren Gesamtwert zu erzielen.'\}} \\
\\ 
\texttt{VORSCHLAG: \{'B85', 'C72', 'A42', 'C23'\}} \\
            }
        }
    }
    & & \\ \\

    \theutterance \stepcounter{utterance}  
    & & \multicolumn{4}{p{0.6\linewidth}}{
        \cellcolor[rgb]{0.9,0.9,0.9}{
            \makecell[{{p{\linewidth}}}]{
                \texttt{\tiny{[GM$\rangle$P2]}}
                \texttt{ARGUMENT: \{'Ich denke, dass der Vorschlag von C72, B54, A90 und C51 eine gute Balance zwischen Wert und Aufwand bietet, aber ich bin nicht sicher, ob es die beste Kombination ist. Ich werde mich auf die Gegenstände konzentrieren, die ich ursprünglich vorgeschlagen habe, und prüfen, ob es Möglichkeiten gibt, sie zu kombinieren, um einen höheren Gesamtwert zu erzielen.'\}} \\
\\ 
\texttt{VORSCHLAG: \{'B85', 'C72', 'A42', 'C23'\}} \\
            }
        }
    }
    & \\ \\

    \theutterance \stepcounter{utterance}  
    & & & \multicolumn{4}{p{0.6\linewidth}}{
        \cellcolor[rgb]{0.9,0.9,0.9}{
            \makecell[{{p{\linewidth}}}]{
                \texttt{\tiny{[GM$\langle$P2]}}
                \texttt{STRATEGISCHE ÜBERLEGUNGEN: \{'Ich habe bemerkt, dass der andere Spieler seinen Vorschlag geändert hat und jetzt B85, C72, A42 und C23 vorschlägt. Ich werde diese Kombination analysieren und prüfen, ob sie meine eigenen Werte maximiert. Ich sehe, dass B85 und A42 einen sehr hohen Wert haben, aber auch einen hohen Aufwand. C72 und C23 haben einen niedrigeren Aufwand, aber auch einen niedrigeren Wert. Ich werde mich auf eine Kombination konzentrieren, die meine eigenen Werte maximiert, ohne den Maximalaufwand zu überschreiten. Ich werde auch versuchen, den anderen Spieler zu einem Kompromiss zu bewegen, indem ich einige seiner Vorschläge akzeptiere und andere ablehne.'\}} \\
\\ 
\texttt{ARGUMENT: \{'Ich denke, dass der Vorschlag von B85, C72, A42 und C23 einige gute Elemente enthält, aber ich bin nicht überzeugt, dass es die beste Kombination ist. Ich werde mich auf die Gegenstände konzentrieren, die ich ursprünglich vorgeschlagen habe, und prüfen, ob es Möglichkeiten gibt, sie zu kombinieren, um einen höheren Gesamtwert zu erzielen. Ich denke, dass C72 ein guter Kandidat ist, da er einen hohen Wert und einen moderaten Aufwand hat.'\}} \\
\\ 
\texttt{ABLEHNUNG: \{'B85', 'A42'\}} \\
\\ 
\texttt{VORSCHLAG: \{'C72', 'C51', 'A90', 'B54'\}} \\
            }
        }
    }
     \\ \\

    \theutterance \stepcounter{utterance}  
    & & \multicolumn{4}{p{0.6\linewidth}}{
        \cellcolor[rgb]{0.9,0.9,0.9}{
            \makecell[{{p{\linewidth}}}]{
                \texttt{\tiny{[GM$\rangle$P2]}}
                \texttt{Du hast einen Vorschlag abgelehnt, der nicht aktiv ist. Vorschläge sind nur dann aktiv, wenn sie vom anderen Spieler mit VORSCHLAG eingebracht wurden und nicht von dir mit ABLEHNUNG deaktiviert wurden. Versuche es noch einmal.} \\
            }
        }
    }
    & \\ \\

    \theutterance \stepcounter{utterance}  
    & & & \multicolumn{4}{p{0.6\linewidth}}{
        \cellcolor[rgb]{0.9,0.9,0.9}{
            \makecell[{{p{\linewidth}}}]{
                \texttt{\tiny{[GM$\langle$P2]}}
                \texttt{STRATEGISCHE ÜBERLEGUNGEN: \{'Ich habe bemerkt, dass mein vorheriger Versuch, einen Vorschlag abzulehnen, fehlgeschlagen ist, da ich einen nicht aktiven Vorschlag abgelehnt habe. Ich werde mich auf den aktuellen Vorschlag konzentrieren, der von dem anderen Spieler mit VORSCHLAG eingebracht wurde. Dieser Vorschlag enthält die Gegenstände B85, C72, A42 und C23. Ich werde prüfen, ob ich diesen Vorschlag akzeptieren oder ablehnen soll, oder ob ich einen neuen Vorschlag machen soll.'\}} \\
\\ 
\texttt{ARGUMENT: \{'Ich denke, dass der Vorschlag von B85, C72, A42 und C23 einige gute Elemente enthält, aber ich bin nicht überzeugt, dass es die beste Kombination ist. Ich werde mich auf die Gegenstände konzentrieren, die einen hohen Wert und einen moderaten Aufwand haben.'\}} \\
\\ 
\texttt{VORSCHLAG: \{'C72', 'C51', 'A90', 'B54'\}} \\
            }
        }
    }
     \\ \\

    \theutterance \stepcounter{utterance}  
    & & \multicolumn{4}{p{0.6\linewidth}}{
        \cellcolor[rgb]{0.9,0.9,0.9}{
            \makecell[{{p{\linewidth}}}]{
                \texttt{\tiny{[P1$\langle$GM]}}
                \texttt{ARGUMENT: \{'Ich denke, dass der Vorschlag von B85, C72, A42 und C23 einige gute Elemente enthält, aber ich bin nicht überzeugt, dass es die beste Kombination ist. Ich werde mich auf die Gegenstände konzentrieren, die einen hohen Wert und einen moderaten Aufwand haben.'\}} \\
\\ 
\texttt{VORSCHLAG: \{'C72', 'C51', 'A90', 'B54'\}} \\
            }
        }
    }
    & \\ \\

    \theutterance \stepcounter{utterance}  
    & \multicolumn{4}{p{0.6\linewidth}}{
        \cellcolor[rgb]{0.9,0.9,0.9}{
            \makecell[{{p{\linewidth}}}]{
                \texttt{\tiny{[P1$\rangle$GM]}}
                \texttt{STRATEGISCHE ÜBERLEGUNGEN: \{'Ich muss den neuen Vorschlag meines Gegners analysieren und prüfen, ob er meinem Ziel entspricht, den Gesamtwert zu maximieren. Ich werde den Wert und Aufwand der vorgeschlagenen Gegenstände berechnen und vergleichen, ob es bessere Alternativen gibt. Ich muss auch meine eigene Wichtigkeitsverteilung berücksichtigen, um sicherzustellen, dass ich nicht zu viel preisgebe. Ich denke, dass der Vorschlag von C72, C51, A90 und B54 eine gute Kombination ist, aber ich muss prüfen, ob ich nicht einige Gegenstände durch andere ersetzen kann, um den Gesamtwert zu erhöhen.'\}} \\
\\ 
\texttt{ARGUMENT: \{'Ich denke, dass der Vorschlag von C72, C51, A90 und B54 eine gute Basis ist, aber ich bin nicht sicher, ob es die beste Kombination ist. Ich werde mich auf die Gegenstände konzentrieren, die einen hohen Wert und einen moderaten Aufwand haben, und prüfen, ob es Möglichkeiten gibt, sie zu kombinieren, um einen höheren Gesamtwert zu erzielen. Ich denke, dass der Gegenstand A42 einen sehr hohen Wert hat und ich ihn nicht außer Acht lassen sollte.'\}} \\
\\ 
\texttt{VORSCHLAG: \{'C72', 'C51', 'A42', 'B54'\}} \\
            }
        }
    }
    & & \\ \\

    \theutterance \stepcounter{utterance}  
    & & \multicolumn{4}{p{0.6\linewidth}}{
        \cellcolor[rgb]{0.9,0.9,0.9}{
            \makecell[{{p{\linewidth}}}]{
                \texttt{\tiny{[GM$\rangle$P2]}}
                \texttt{ARGUMENT: \{'Ich denke, dass der Vorschlag von C72, C51, A90 und B54 eine gute Basis ist, aber ich bin nicht sicher, ob es die beste Kombination ist. Ich werde mich auf die Gegenstände konzentrieren, die einen hohen Wert und einen moderaten Aufwand haben, und prüfen, ob es Möglichkeiten gibt, sie zu kombinieren, um einen höheren Gesamtwert zu erzielen. Ich denke, dass der Gegenstand A42 einen sehr hohen Wert hat und ich ihn nicht außer Acht lassen sollte.'\}} \\
\\ 
\texttt{VORSCHLAG: \{'C72', 'C51', 'A42', 'B54'\}} \\
            }
        }
    }
    & \\ \\

    \theutterance \stepcounter{utterance}  
    & & & \multicolumn{4}{p{0.6\linewidth}}{
        \cellcolor[rgb]{0.9,0.9,0.9}{
            \makecell[{{p{\linewidth}}}]{
                \texttt{\tiny{[GM$\langle$P2]}}
                \texttt{STRATEGISCHE ÜBERLEGUNGEN: \{'Ich sehe, dass der andere Spieler seinen Vorschlag geändert hat und jetzt C72, C51, A42 und B54 vorschlägt. Dieser Vorschlag stimmt teilweise mit meinem eigenen Vorschlag überein, aber er enthält auch den Gegenstand A42, den ich zuvor abgelehnt hatte. Ich werde prüfen, ob dieser Vorschlag meine eigenen Werte maximiert und ob ich ihn akzeptieren soll. Ich werde auch die Gesamtkosten des Vorschlags berechnen, um sicherzustellen, dass sie den Maximalaufwand nicht überschreiten.'\}} \\
\\ 
\texttt{ARGUMENT: \{'Ich denke, dass der Vorschlag von C72, C51, A42 und B54 eine gute Kombination ist, da er eine gute Balance zwischen Wert und Aufwand bietet. Der Gegenstand A42 hat einen sehr hohen Wert, und die anderen Gegenstände haben einen moderaten Aufwand. Ich denke, dass dies ein guter Kompromiss ist und ich werde ihn akzeptieren.'\}} \\
\\ 
\texttt{ZUSTIMMUNG: \{'C72', 'C51', 'A42', 'B54'\}} \\
            }
        }
    }
     \\ \\

    \theutterance \stepcounter{utterance}  
    & & & \multicolumn{2}{p{0.3\linewidth}}{
        \cellcolor[rgb]{0.9,0.9,0.9}{
            \makecell[{{p{\linewidth}}}]{
                \texttt{\tiny{[GM$|$GM]}}
                \texttt{game successful} \\
            }
        }
    }
    & & \\ \\

    \theutterance \stepcounter{utterance}  
    & & & \multicolumn{2}{p{0.3\linewidth}}{
        \cellcolor[rgb]{0.9,0.9,0.9}{
            \makecell[{{p{\linewidth}}}]{
                \texttt{\tiny{[GM$|$GM]}}
                \texttt{end game} \\
            }
        }
    }
    & & \\ \\

\end{supertabular}
}

\end{document}
