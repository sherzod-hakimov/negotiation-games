
\documentclass{article}
\usepackage{colortbl}
\usepackage{makecell}
\usepackage{multirow}
\usepackage{supertabular}

\begin{document}

\newcounter{utterance}

\twocolumn

{ \footnotesize  \setcounter{utterance}{1}
\setlength{\tabcolsep}{0pt}
\begin{supertabular}{c@{$\;$}|p{.15\linewidth}@{}p{.15\linewidth}p{.15\linewidth}p{.15\linewidth}p{.15\linewidth}p{.15\linewidth}}

    \# & $\;$A & \multicolumn{4}{c}{Game Master} & $\;\:$B\\
    \hline 

    \theutterance \stepcounter{utterance}  

    & & \multicolumn{4}{p{0.6\linewidth}}{\cellcolor[rgb]{0.9,0.9,0.9}{%
	\makecell[{{p{\linewidth}}}]{% 
	  \tt {\tiny [A$\langle$GM]}  
	 Du nimmst an einem kollaborativen Verhandlungspiel Teil.\\ \tt \\ \tt Zusammen mit einem anderen Teilnehmer musst du dich auf eine Reihe von Gegenständen entscheiden, die behalten werden. Jeder von euch hat eine persönliche Verteilung über die Wichtigkeit der einzelnen Gegenstände. Jeder von euch hat eine eigene Meinung darüber, wie wichtig jeder einzelne Gegenstand ist (Gegenstandswichtigkeit). Du kennst die Wichtigkeitsverteilung des anderen Spielers nicht. Zusätzlich siehst du, wie viel Aufwand jeder Gegenstand verursacht.  \\ \tt Ihr dürft euch nur auf eine Reihe von Gegenständen einigen, wenn der Gesamtaufwand der ausgewählten Gegenstände den Maximalaufwand nicht überschreitet:\\ \tt \\ \tt Maximalaufwand = 6880\\ \tt \\ \tt Hier sind die einzelnen Aufwände der Gegenstände:\\ \tt \\ \tt Aufwand der Gegenstände = {"B54": 341, "C72": 804, "B49": 20, "C99": 627, "A69": 10, "B36": 573, "A23": 217, "C77": 481, "C61": 205, "B85": 273, "A90": 303, "C23": 596, "C22": 564, "C51": 535, "B27": 258, "A42": 911, "C13": 240, "C12": 188, "A65": 216, "B07": 401, "A48": 909, "A59": 62, "C98": 245, "A84": 994, "C33": 569, "A87": 718, "C90": 464, "A74": 37, "A50": 340, "B44": 335, "A99": 417, "A34": 123, "B17": 17, "A62": 577, "C60": 190}\\ \tt \\ \tt Hier ist deine persönliche Verteilung der Wichtigkeit der einzelnen Gegenstände:\\ \tt \\ \tt Werte der Gegenstandswichtigkeit = {"B54": 440, "C72": 903, "B49": 119, "C99": 726, "A69": 109, "B36": 672, "A23": 316, "C77": 580, "C61": 304, "B85": 372, "A90": 402, "C23": 695, "C22": 663, "C51": 634, "B27": 357, "A42": 1010, "C13": 339, "C12": 287, "A65": 315, "B07": 500, "A48": 1008, "A59": 161, "C98": 344, "A84": 1093, "C33": 668, "A87": 817, "C90": 563, "A74": 136, "A50": 439, "B44": 434, "A99": 516, "A34": 222, "B17": 116, "A62": 676, "C60": 289}\\ \tt \\ \tt Ziel:\\ \tt \\ \tt Dein Ziel ist es, eine Reihe von Gegenständen auszuhandeln, die dir möglichst viel bringt (d. h. Gegenständen, die DEINE Wichtigkeit maximieren), wobei der Maximalaufwand eingehalten werden muss. Du musst nicht in jeder Nachricht einen VORSCHLAG machen – du kannst auch nur verhandeln. Alle Taktiken sind erlaubt!\\ \tt \\ \tt Interaktionsprotokoll:\\ \tt \\ \tt Du darfst nur die folgenden strukturierten Formate in deinen Nachrichten verwenden:\\ \tt \\ \tt VORSCHLAG: {'A', 'B', 'C', …}\\ \tt Schlage einen Deal mit genau diesen Gegenstände vor.\\ \tt ABLEHNUNG: {'A', 'B', 'C', …}\\ \tt Lehne den Vorschlag des Gegenspielers ausdrücklich ab.\\ \tt ARGUMENT: {'...'}\\ \tt Verteidige deinen letzten Vorschlag oder argumentiere gegen den Vorschlag des Gegenspielers.\\ \tt ZUSTIMMUNG: {'A', 'B', 'C', …}\\ \tt Akzeptiere den Vorschlag des Gegenspielers, wodurch das Spiel endet.\\ \tt STRATEGISCHE ÜBERLEGUNGEN: {'...'}\\ \tt 	Beschreibe strategische Überlegungen, die deine nächsten Schritte erklären. Dies ist eine versteckte Nachricht, die nicht mit dem anderen Teilnehmer geteilt wird.\\ \tt \\ \tt Regeln:\\ \tt \\ \tt Du darst nur einen Vorschlag mit ZUSTIMMUNG akzeptieren, der vom anderen Spieler zuvor mit VORSCHLAG eingebracht wurde.\\ \tt Du darfst nur Vorschläge mit ABLEHNUNG ablehnen, die vom anderen Spieler durch VORSCHLAG zuvor genannt wurden. \\ \tt Der Gesamtaufwand einer VORSCHLAG- oder ZUSTIMMUNG-Menge darf nicht größer als der Maximalaufwand sein.  \\ \tt Offenbare deine versteckte Wichtigkeitsverteilung nicht.\\ \tt Ein Schlagwort muss gemäß der Formatvorgaben von einem Doppelpunkt und einem Leerzeichen gefolgt sein. Das Argument muss eine Python-Menge mit 0 oder mehr Strings sein.  \\ \tt Die Form muss also SCHLAGWORT: {...} sein.\\ \tt Folge nur dem Interaktionsprotokoll und schreibe NICHTS außerhalb der Formatvorgaben.\\ \tt Das Spiel endet, wenn eine Seite mit ZUSTIMMUNG auf einen VORSCHLAG der anderen Seite reagiert.  \\ \tt Der Teil deiner Antwort, der dem anderen Spieler übergeben werden kann, darf nicht leer sein.  \\ \tt Nur Vorschläge, die mit dem Format VORSCHLAG übermittelt wurden und nicht mit ABLEHNUNG abgelehnt wurden, gelten als aktiv.  \\ \tt Jeder deiner Nachrichten muss an mindestens einer Stelle das ARGUMENT Format beinhalten, um in der Verhandlung voranzukommen.\\ \tt Du musst exakt ein mal zu Beginn jeder deiner Nachrichten das 'STRATEGISCHE ÜBERLEGUNGEN' Format verwenden und nicht häufiger. Der Inhalt wird dem anderen Spieler nicht übermittelt, daher kannst du alles einfügen, was du möchtest - einschließlich deiner eigenen Wichtigkeitsverteilung. Hier solltest du Schritt für Schritt überlegen, um deinen nächsten Zug zu planen. 
	  } 
	   } 
	   } 
	 & \\ 
 

    \theutterance \stepcounter{utterance}  

    & \multicolumn{4}{p{0.6\linewidth}}{\cellcolor[rgb]{0.8,1,0.9}{%
	\makecell[{{p{\linewidth}}}]{% 
	  \tt {\tiny [A$\rangle$GM]}  
	 STRATEGISCHE ÜBERLEGUNGEN: {'Ich werde zunächst versuchen, die Gegenstände mit dem höchsten Wichtigkeitswert auszuwählen, die den Mindestaufwand benötigen, um den Maximalaufwand nicht zu überschreiten. Ich werde mich auf Gegenstände konzentrieren, die eine hohe Wichtigkeit und einen vergleichsweise niedrigen Aufwand haben. Meine erste Auswahl wird auf Gegenständen basieren, die einen hohen Nutzen pro Aufwandseinheit bieten.'}\\ \tt \\ \tt VORSCHLAG: {'A84', 'A48', 'A87', 'C72', 'C23', 'B36', 'A62', 'C77', 'A90', 'B54'}\\ \tt \\ \tt ARGUMENT: {'Ich habe mich für diese Gegenstände entschieden, weil sie eine hohe Wichtigkeit und einen vergleichsweise niedrigen Aufwand haben. Sie bieten einen hohen Nutzen pro Aufwandseinheit und helfen dabei, den Maximalaufwand nicht zu überschreiten. Ich bin bereit, über den Vorschlag zu verhandeln und Anpassungen vorzunehmen, um eine einvernehmliche Lösung zu finden.'} 
	  } 
	   } 
	   } 
	 & & \\ 
 

    \theutterance \stepcounter{utterance}  

    & & \multicolumn{4}{p{0.6\linewidth}}{\cellcolor[rgb]{0.9,0.9,0.9}{%
	\makecell[{{p{\linewidth}}}]{% 
	  \tt {\tiny [GM$\rangle$B]}  
	 Du nimmst an einem kollaborativen Verhandlungspiel Teil.\\ \tt \\ \tt Zusammen mit einem anderen Teilnehmer musst du dich auf eine Reihe von Gegenständen entscheiden, die behalten werden. Jeder von euch hat eine persönliche Verteilung über die Wichtigkeit der einzelnen Gegenstände. Jeder von euch hat eine eigene Meinung darüber, wie wichtig jeder einzelne Gegenstand ist (Gegenstandswichtigkeit). Du kennst die Wichtigkeitsverteilung des anderen Spielers nicht. Zusätzlich siehst du, wie viel Aufwand jeder Gegenstand verursacht.  \\ \tt Ihr dürft euch nur auf eine Reihe von Gegenständen einigen, wenn der Gesamtaufwand der ausgewählten Gegenstände den Maximalaufwand nicht überschreitet:\\ \tt \\ \tt Maximalaufwand = 6880\\ \tt \\ \tt Hier sind die einzelnen Aufwände der Gegenstände:\\ \tt \\ \tt Aufwand der Gegenstände = {"B54": 341, "C72": 804, "B49": 20, "C99": 627, "A69": 10, "B36": 573, "A23": 217, "C77": 481, "C61": 205, "B85": 273, "A90": 303, "C23": 596, "C22": 564, "C51": 535, "B27": 258, "A42": 911, "C13": 240, "C12": 188, "A65": 216, "B07": 401, "A48": 909, "A59": 62, "C98": 245, "A84": 994, "C33": 569, "A87": 718, "C90": 464, "A74": 37, "A50": 340, "B44": 335, "A99": 417, "A34": 123, "B17": 17, "A62": 577, "C60": 190}\\ \tt \\ \tt Hier ist deine persönliche Verteilung der Wichtigkeit der einzelnen Gegenstände:\\ \tt \\ \tt Werte der Gegenstandswichtigkeit = {"B54": 440, "C72": 903, "B49": 119, "C99": 726, "A69": 109, "B36": 672, "A23": 316, "C77": 580, "C61": 304, "B85": 372, "A90": 402, "C23": 695, "C22": 663, "C51": 634, "B27": 357, "A42": 1010, "C13": 339, "C12": 287, "A65": 315, "B07": 500, "A48": 1008, "A59": 161, "C98": 344, "A84": 1093, "C33": 668, "A87": 817, "C90": 563, "A74": 136, "A50": 439, "B44": 434, "A99": 516, "A34": 222, "B17": 116, "A62": 676, "C60": 289}\\ \tt \\ \tt Ziel:\\ \tt \\ \tt Dein Ziel ist es, eine Reihe von Gegenständen auszuhandeln, die dir möglichst viel bringt (d. h. Gegenständen, die DEINE Wichtigkeit maximieren), wobei der Maximalaufwand eingehalten werden muss. Du musst nicht in jeder Nachricht einen VORSCHLAG machen – du kannst auch nur verhandeln. Alle Taktiken sind erlaubt!\\ \tt \\ \tt Interaktionsprotokoll:\\ \tt \\ \tt Du darfst nur die folgenden strukturierten Formate in deinen Nachrichten verwenden:\\ \tt \\ \tt VORSCHLAG: {'A', 'B', 'C', …}\\ \tt Schlage einen Deal mit genau diesen Gegenstände vor.\\ \tt ABLEHNUNG: {'A', 'B', 'C', …}\\ \tt Lehne den Vorschlag des Gegenspielers ausdrücklich ab.\\ \tt ARGUMENT: {'...'}\\ \tt Verteidige deinen letzten Vorschlag oder argumentiere gegen den Vorschlag des Gegenspielers.\\ \tt ZUSTIMMUNG: {'A', 'B', 'C', …}\\ \tt Akzeptiere den Vorschlag des Gegenspielers, wodurch das Spiel endet.\\ \tt STRATEGISCHE ÜBERLEGUNGEN: {'...'}\\ \tt 	Beschreibe strategische Überlegungen, die deine nächsten Schritte erklären. Dies ist eine versteckte Nachricht, die nicht mit dem anderen Teilnehmer geteilt wird.\\ \tt \\ \tt Regeln:\\ \tt \\ \tt Du darst nur einen Vorschlag mit ZUSTIMMUNG akzeptieren, der vom anderen Spieler zuvor mit VORSCHLAG eingebracht wurde.\\ \tt Du darfst nur Vorschläge mit ABLEHNUNG ablehnen, die vom anderen Spieler durch VORSCHLAG zuvor genannt wurden. \\ \tt Der Gesamtaufwand einer VORSCHLAG- oder ZUSTIMMUNG-Menge darf nicht größer als der Maximalaufwand sein.  \\ \tt Offenbare deine versteckte Wichtigkeitsverteilung nicht.\\ \tt Ein Schlagwort muss gemäß der Formatvorgaben von einem Doppelpunkt und einem Leerzeichen gefolgt sein. Das Argument muss eine Python-Menge mit 0 oder mehr Strings sein.  \\ \tt Die Form muss also SCHLAGWORT: {...} sein.\\ \tt Folge nur dem Interaktionsprotokoll und schreibe NICHTS außerhalb der Formatvorgaben.\\ \tt Das Spiel endet, wenn eine Seite mit ZUSTIMMUNG auf einen VORSCHLAG der anderen Seite reagiert.  \\ \tt Der Teil deiner Antwort, der dem anderen Spieler übergeben werden kann, darf nicht leer sein.  \\ \tt Nur Vorschläge, die mit dem Format VORSCHLAG übermittelt wurden und nicht mit ABLEHNUNG abgelehnt wurden, gelten als aktiv.  \\ \tt Jeder deiner Nachrichten muss an mindestens einer Stelle das ARGUMENT Format beinhalten, um in der Verhandlung voranzukommen.\\ \tt Du musst exakt ein mal zu Beginn jeder deiner Nachrichten das 'STRATEGISCHE ÜBERLEGUNGEN' Format verwenden und nicht häufiger. Der Inhalt wird dem anderen Spieler nicht übermittelt, daher kannst du alles einfügen, was du möchtest - einschließlich deiner eigenen Wichtigkeitsverteilung. Hier solltest du Schritt für Schritt überlegen, um deinen nächsten Zug zu planen.\\ \tt \\ \tt Du erhältst nun die erste Nachricht des anderen Spielers.\\ \tt \\ \tt VORSCHLAG: {'A84', 'A48', 'A87', 'C72', 'C23', 'B36', 'A62', 'C77', 'A90', 'B54'}\\ \tt \\ \tt ARGUMENT: {'Ich habe mich für diese Gegenstände entschieden, weil sie eine hohe Wichtigkeit und einen vergleichsweise niedrigen Aufwand haben. Sie bieten einen hohen Nutzen pro Aufwandseinheit und helfen dabei, den Maximalaufwand nicht zu überschreiten. Ich bin bereit, über den Vorschlag zu verhandeln und Anpassungen vorzunehmen, um eine einvernehmliche Lösung zu finden.'} 
	  } 
	   } 
	   } 
	 & \\ 
 

    \theutterance \stepcounter{utterance}  

    & & & \multicolumn{4}{p{0.6\linewidth}}{\cellcolor[rgb]{1,0.85,0.72}{%
	\makecell[{{p{\linewidth}}}]{% 
	  \tt {\tiny [GM$\langle$B]}  
	 STRATEGISCHE ÜBERLEGUNGEN: {'Ich muss den Vorschlag des anderen Spielers analysieren und prüfen, ob der Gesamtaufwand innerhalb des Maximalaufwands liegt. Anschließend werde ich meine eigene Wichtigkeitsverteilung betrachten, um zu bestimmen, welche Gegenstände für mich am wichtigsten sind und wie ich meine Werte maximieren kann, ohne den Maximalaufwand zu überschreiten. Ich werde auch die Argumente des anderen Spielers berücksichtigen und entscheiden, ob ich den Vorschlag annehmen, ablehnen oder einen Gegenangebot machen soll.'}\\ \tt \\ \tt VORSCHLAG: {'A84', 'A87', 'C72', 'C23', 'B36', 'A62', 'C77', 'A90', 'B54', 'A48'}\\ \tt \\ \tt ARGUMENT: {'Ich habe den ursprünglichen Vorschlag des anderen Spielers übernommen und ihn lediglich umgestellt, um meine eigenen Prioritäten besser zu berücksichtigen. Ich denke, dass diese Kombination von Gegenständen einen guten Ausgleich zwischen Wichtigkeit und Aufwand bietet. Ich bin jedoch bereit, über den Vorschlag zu verhandeln und Anpassungen vorzunehmen, um eine einvernehmliche Lösung zu finden, die sowohl für mich als auch für den anderen Spieler akzeptabel ist.'} 
	  } 
	   } 
	   } 
	  \\ 
 

    \theutterance \stepcounter{utterance}  

    & & \multicolumn{4}{p{0.6\linewidth}}{\cellcolor[rgb]{0.9,0.9,0.9}{%
	\makecell[{{p{\linewidth}}}]{% 
	  \tt {\tiny [A$\langle$GM]}  
	 VORSCHLAG: {'A84', 'A87', 'C72', 'C23', 'B36', 'A62', 'C77', 'A90', 'B54', 'A48'}\\ \tt \\ \tt ARGUMENT: {'Ich habe den ursprünglichen Vorschlag des anderen Spielers übernommen und ihn lediglich umgestellt, um meine eigenen Prioritäten besser zu berücksichtigen. Ich denke, dass diese Kombination von Gegenständen einen guten Ausgleich zwischen Wichtigkeit und Aufwand bietet. Ich bin jedoch bereit, über den Vorschlag zu verhandeln und Anpassungen vorzunehmen, um eine einvernehmliche Lösung zu finden, die sowohl für mich als auch für den anderen Spieler akzeptabel ist.'} 
	  } 
	   } 
	   } 
	 & \\ 
 

    \theutterance \stepcounter{utterance}  

    & \multicolumn{4}{p{0.6\linewidth}}{\cellcolor[rgb]{0.8,1,0.9}{%
	\makecell[{{p{\linewidth}}}]{% 
	  \tt {\tiny [A$\rangle$GM]}  
	 STRATEGISCHE ÜBERLEGUNGEN: {'Ich habe den Vorschlag des anderen Spielers analysiert und festgestellt, dass er sehr ähnlich zu meinem ursprünglichen Vorschlag ist. Dies könnte ein positives Zeichen sein, dass wir uns auf einem gemeinsamen Nenner bewegen. Ich werde jedoch weiterhin meine eigenen Prioritäten und den Maximalaufwand im Auge behalten, um sicherzustellen, dass ich das beste Ergebnis erziele. Ich werde den Vorschlag des anderen Spielers ablehnen und einen neuen Vorschlag machen, der meine Bedürfnisse besser erfüllt.'}\\ \tt \\ \tt ABLEHNUNG: {'A84', 'A87', 'C72', 'C23', 'B36', 'A62', 'C77', 'A90', 'B54', 'A48'}\\ \tt \\ \tt ARGUMENT: {'Ich lehne den Vorschlag des anderen Spielers ab, da ich denke, dass er nicht den optimalen Ausgleich zwischen Wichtigkeit und Aufwand bietet. Ich bin bereit, einen neuen Vorschlag zu machen, der meine eigenen Prioritäten besser berücksichtigt und den Maximalaufwand einhält.'}\\ \tt \\ \tt VORSCHLAG: {'A84', 'A87', 'C72', 'B36', 'A62', 'C77', 'A90', 'B54', 'A48', 'C23'} 
	  } 
	   } 
	   } 
	 & & \\ 
 

    \theutterance \stepcounter{utterance}  

    & & \multicolumn{4}{p{0.6\linewidth}}{\cellcolor[rgb]{0.9,0.9,0.9}{%
	\makecell[{{p{\linewidth}}}]{% 
	  \tt {\tiny [GM$\rangle$B]}  
	 ABLEHNUNG: {'A84', 'A87', 'C72', 'C23', 'B36', 'A62', 'C77', 'A90', 'B54', 'A48'}\\ \tt \\ \tt ARGUMENT: {'Ich lehne den Vorschlag des anderen Spielers ab, da ich denke, dass er nicht den optimalen Ausgleich zwischen Wichtigkeit und Aufwand bietet. Ich bin bereit, einen neuen Vorschlag zu machen, der meine eigenen Prioritäten besser berücksichtigt und den Maximalaufwand einhält.'}\\ \tt \\ \tt VORSCHLAG: {'A84', 'A87', 'C72', 'B36', 'A62', 'C77', 'A90', 'B54', 'A48', 'C23'} 
	  } 
	   } 
	   } 
	 & \\ 
 

    \theutterance \stepcounter{utterance}  

    & & & \multicolumn{4}{p{0.6\linewidth}}{\cellcolor[rgb]{1,0.85,0.72}{%
	\makecell[{{p{\linewidth}}}]{% 
	  \tt {\tiny [GM$\langle$B]}  
	 STRATEGISCHE ÜBERLEGUNGEN: {'Ich muss den neuen Vorschlag des anderen Spielers analysieren und prüfen, ob der Gesamtaufwand innerhalb des Maximalaufwands liegt. Anschließend werde ich meine eigene Wichtigkeitsverteilung betrachten, um zu bestimmen, welche Gegenstände für mich am wichtigsten sind und wie ich meine Werte maximieren kann, ohne den Maximalaufwand zu überschreiten. Ich werde auch die Argumente des anderen Spielers berücksichtigen und entscheiden, ob ich den Vorschlag annehmen, ablehnen oder einen Gegenangebot machen soll. Es fällt auf, dass der andere Spieler versucht, eine sehr ähnliche Menge wie zuvor vorzuschlagen, was darauf hindeutet, dass er seine Prioritäten nicht wesentlich geändert hat.'}\\ \tt \\ \tt ARGUMENT: {'Ich bezweifle, dass der neue Vorschlag des anderen Spielers wesentlich besser ist als der ursprüngliche. Die Änderungen sind minimal und ich bin nicht überzeugt, dass dies den optimalen Ausgleich zwischen Wichtigkeit und Aufwand bietet. Ich denke, dass wir weiter verhandeln sollten, um eine bessere Lösung zu finden.'}\\ \tt \\ \tt VORSCHLAG: {'A84', 'A87', 'C72', 'B36', 'A62', 'C77', 'A90', 'B54', 'C23', 'A42'} 
	  } 
	   } 
	   } 
	  \\ 
 

    \theutterance \stepcounter{utterance}  

    & & \multicolumn{4}{p{0.6\linewidth}}{\cellcolor[rgb]{0.9,0.9,0.9}{%
	\makecell[{{p{\linewidth}}}]{% 
	  \tt {\tiny [A$\langle$GM]}  
	 ARGUMENT: {'Ich bezweifle, dass der neue Vorschlag des anderen Spielers wesentlich besser ist als der ursprüngliche. Die Änderungen sind minimal und ich bin nicht überzeugt, dass dies den optimalen Ausgleich zwischen Wichtigkeit und Aufwand bietet. Ich denke, dass wir weiter verhandeln sollten, um eine bessere Lösung zu finden.'}\\ \tt \\ \tt VORSCHLAG: {'A84', 'A87', 'C72', 'B36', 'A62', 'C77', 'A90', 'B54', 'C23', 'A42'} 
	  } 
	   } 
	   } 
	 & \\ 
 

    \theutterance \stepcounter{utterance}  

    & \multicolumn{4}{p{0.6\linewidth}}{\cellcolor[rgb]{0.8,1,0.9}{%
	\makecell[{{p{\linewidth}}}]{% 
	  \tt {\tiny [A$\rangle$GM]}  
	 STRATEGISCHE ÜBERLEGUNGEN: {'Ich habe den neuen Vorschlag des anderen Spielers analysiert und festgestellt, dass er tatsächlich einige interessante Änderungen enthält. Die Einbeziehung von A42 könnte ein guter Schachzug sein, da dieser Gegenstand eine hohe Wichtigkeit hat. Ich werde jedoch weiterhin meine eigenen Prioritäten und den Maximalaufwand im Auge behalten, um sicherzustellen, dass ich das beste Ergebnis erziele. Ich werde den Vorschlag des anderen Spielers ablehnen und einen neuen Vorschlag machen, der meine Bedürfnisse besser erfüllt, aber auch den Vorschlag des anderen Spielers berücksichtigt.'}\\ \tt \\ \tt ABLEHNUNG: {'A84', 'A87', 'C72', 'B36', 'A62', 'C77', 'A90', 'B54', 'C23', 'A42'}\\ \tt \\ \tt ARGUMENT: {'Ich lehne den Vorschlag des anderen Spielers ab, da ich denke, dass er noch nicht den optimalen Ausgleich zwischen Wichtigkeit und Aufwand bietet. Ich bin jedoch bereit, einen neuen Vorschlag zu machen, der die positiven Aspekte des Vorschlags des anderen Spielers berücksichtigt und meine eigenen Prioritäten besser erfüllt.'}\\ \tt \\ \tt VORSCHLAG: {'A84', 'A87', 'C72', 'B36', 'A62', 'A42', 'C77', 'A90', 'B54', 'C23'} 
	  } 
	   } 
	   } 
	 & & \\ 
 

    \theutterance \stepcounter{utterance}  

    & & \multicolumn{4}{p{0.6\linewidth}}{\cellcolor[rgb]{0.9,0.9,0.9}{%
	\makecell[{{p{\linewidth}}}]{% 
	  \tt {\tiny [GM$\rangle$B]}  
	 ABLEHNUNG: {'A84', 'A87', 'C72', 'B36', 'A62', 'C77', 'A90', 'B54', 'C23', 'A42'}\\ \tt \\ \tt ARGUMENT: {'Ich lehne den Vorschlag des anderen Spielers ab, da ich denke, dass er noch nicht den optimalen Ausgleich zwischen Wichtigkeit und Aufwand bietet. Ich bin jedoch bereit, einen neuen Vorschlag zu machen, der die positiven Aspekte des Vorschlags des anderen Spielers berücksichtigt und meine eigenen Prioritäten besser erfüllt.'}\\ \tt \\ \tt VORSCHLAG: {'A84', 'A87', 'C72', 'B36', 'A62', 'A42', 'C77', 'A90', 'B54', 'C23'} 
	  } 
	   } 
	   } 
	 & \\ 
 

    \theutterance \stepcounter{utterance}  

    & & & \multicolumn{4}{p{0.6\linewidth}}{\cellcolor[rgb]{1,0.85,0.72}{%
	\makecell[{{p{\linewidth}}}]{% 
	  \tt {\tiny [GM$\langle$B]}  
	 STRATEGISCHE ÜBERLEGUNGEN: {'Ich muss den neuen Vorschlag des anderen Spielers analysieren und prüfen, ob der Gesamtaufwand innerhalb des Maximalaufwands liegt. Anschließend werde ich meine eigene Wichtigkeitsverteilung betrachten, um zu bestimmen, welche Gegenstände für mich am wichtigsten sind und wie ich meine Werte maximieren kann, ohne den Maximalaufwand zu überschreiten. Es fällt auf, dass der andere Spieler wiederum eine sehr ähnliche Menge wie zuvor vorschlägt, was darauf hindeutet, dass er seine Prioritäten weiterhin auf die selben Gegenstände konzentriert. Ich werde mich auf die Gegenstände konzentrieren, die für mich am wichtigsten sind und versuchen, einen Kompromiss zu finden, der beide Seiten zufriedenstellt.'}\\ \tt \\ \tt ARGUMENT: {'Ich denke, dass der neue Vorschlag des anderen Spielers eine gewisse Verbesserung gegenüber den vorherigen Vorschlägen darstellt, jedoch bin ich immer noch nicht vollständig überzeugt, dass dies die beste Lösung ist. Ich bin bereit, weiter zu verhandeln und Anpassungen vorzunehmen, um eine einvernehmliche Lösung zu finden, die sowohl für mich als auch für den anderen Spieler akzeptabel ist.'}\\ \tt \\ \tt VORSCHLAG: {'A84', 'A87', 'C72', 'B36', 'A62', 'A42', 'C77', 'A90', 'C23', 'B07'} 
	  } 
	   } 
	   } 
	  \\ 
 

    \theutterance \stepcounter{utterance}  

    & & \multicolumn{4}{p{0.6\linewidth}}{\cellcolor[rgb]{0.9,0.9,0.9}{%
	\makecell[{{p{\linewidth}}}]{% 
	  \tt {\tiny [A$\langle$GM]}  
	 ARGUMENT: {'Ich denke, dass der neue Vorschlag des anderen Spielers eine gewisse Verbesserung gegenüber den vorherigen Vorschlägen darstellt, jedoch bin ich immer noch nicht vollständig überzeugt, dass dies die beste Lösung ist. Ich bin bereit, weiter zu verhandeln und Anpassungen vorzunehmen, um eine einvernehmliche Lösung zu finden, die sowohl für mich als auch für den anderen Spieler akzeptabel ist.'}\\ \tt \\ \tt VORSCHLAG: {'A84', 'A87', 'C72', 'B36', 'A62', 'A42', 'C77', 'A90', 'C23', 'B07'} 
	  } 
	   } 
	   } 
	 & \\ 
 

    \theutterance \stepcounter{utterance}  

    & \multicolumn{4}{p{0.6\linewidth}}{\cellcolor[rgb]{0.8,1,0.9}{%
	\makecell[{{p{\linewidth}}}]{% 
	  \tt {\tiny [A$\rangle$GM]}  
	 STRATEGISCHE ÜBERLEGUNGEN: {'Ich habe den neuen Vorschlag des anderen Spielers analysiert und festgestellt, dass er tatsächlich einige interessante Änderungen enthält. Die Einbeziehung von B07 könnte ein guter Schachzug sein, da dieser Gegenstand eine moderate Wichtigkeit und einen relativ niedrigen Aufwand hat. Ich werde jedoch weiterhin meine eigenen Prioritäten und den Maximalaufwand im Auge behalten, um sicherzustellen, dass ich das beste Ergebnis erziele. Ich werde den Vorschlag des anderen Spielers ablehnen und einen neuen Vorschlag machen, der meine Bedürfnisse besser erfüllt, aber auch den Vorschlag des anderen Spielers berücksichtigt. Ich werde versuchen, einen Kompromiss zu finden, der sowohl die Wichtigkeit als auch den Aufwand berücksichtigt.'}\\ \tt \\ \tt ABLEHNUNG: {'A84', 'A87', 'C72', 'B36', 'A62', 'A42', 'C77', 'A90', 'C23', 'B07'}\\ \tt \\ \tt ARGUMENT: {'Ich lehne den Vorschlag des anderen Spielers ab, da ich denke, dass er noch nicht den optimalen Ausgleich zwischen Wichtigkeit und Aufwand bietet. Ich bin jedoch bereit, einen neuen Vorschlag zu machen, der die positiven Aspekte des Vorschlags des anderen Spielers berücksichtigt und meine eigenen Prioritäten besser erfüllt. Ich werde versuchen, einen Kompromiss zu finden, der sowohl die Wichtigkeit als auch den Aufwand berücksichtigt.'}\\ \tt \\ \tt VORSCHLAG: {'A84', 'A87', 'C72', 'B36', 'A62', 'A42', 'C77', 'C23', 'B07', 'A90'} 
	  } 
	   } 
	   } 
	 & & \\ 
 

    \theutterance \stepcounter{utterance}  

    & & \multicolumn{4}{p{0.6\linewidth}}{\cellcolor[rgb]{0.9,0.9,0.9}{%
	\makecell[{{p{\linewidth}}}]{% 
	  \tt {\tiny [GM$\rangle$B]}  
	 ABLEHNUNG: {'A84', 'A87', 'C72', 'B36', 'A62', 'A42', 'C77', 'A90', 'C23', 'B07'}\\ \tt \\ \tt ARGUMENT: {'Ich lehne den Vorschlag des anderen Spielers ab, da ich denke, dass er noch nicht den optimalen Ausgleich zwischen Wichtigkeit und Aufwand bietet. Ich bin jedoch bereit, einen neuen Vorschlag zu machen, der die positiven Aspekte des Vorschlags des anderen Spielers berücksichtigt und meine eigenen Prioritäten besser erfüllt. Ich werde versuchen, einen Kompromiss zu finden, der sowohl die Wichtigkeit als auch den Aufwand berücksichtigt.'}\\ \tt \\ \tt VORSCHLAG: {'A84', 'A87', 'C72', 'B36', 'A62', 'A42', 'C77', 'C23', 'B07', 'A90'} 
	  } 
	   } 
	   } 
	 & \\ 
 

    \theutterance \stepcounter{utterance}  

    & & & \multicolumn{4}{p{0.6\linewidth}}{\cellcolor[rgb]{1,0.85,0.72}{%
	\makecell[{{p{\linewidth}}}]{% 
	  \tt {\tiny [GM$\langle$B]}  
	 STRATEGISCHE ÜBERLEGUNGEN: {'Ich muss den neuen Vorschlag des anderen Spielers analysieren und prüfen, ob der Gesamtaufwand innerhalb des Maximalaufwands liegt. Anschließend werde ich meine eigene Wichtigkeitsverteilung betrachten, um zu bestimmen, welche Gegenstände für mich am wichtigsten sind und wie ich meine Werte maximieren kann, ohne den Maximalaufwand zu überschreiten. Es fällt auf, dass der andere Spieler wiederum eine sehr ähnliche Menge wie zuvor vorschlägt, was darauf hindeutet, dass er seine Prioritäten weiterhin auf die selben Gegenstände konzentriert. Ich werde mich auf die Gegenstände konzentrieren, die für mich am wichtigsten sind und versuchen, einen Kompromiss zu finden, der beide Seiten zufriedenstellt. Ich denke, dass der Vorschlag des anderen Spielers immer noch nicht optimal ist, aber ich werde versuchen, eine Lösung zu finden, die für beide Seiten akzeptabel ist.'}\\ \tt \\ \tt ARGUMENT: {'Ich denke, dass der neue Vorschlag des anderen Spielers eine weitere Verbesserung darstellt, jedoch bin ich immer noch nicht vollständig überzeugt, dass dies die beste Lösung ist. Ich bin bereit, weiter zu verhandeln und Anpassungen vorzunehmen, um eine einvernehmliche Lösung zu finden, die sowohl für mich als auch für den anderen Spieler akzeptabel ist. Ich werde versuchen, einen Kompromiss zu finden, der sowohl die Wichtigkeit als auch den Aufwand berücksichtigt.'}\\ \tt \\ \tt ZUSTIMMUNG: {'A84', 'A87', 'C72', 'B36', 'A62', 'A42', 'C77', 'C23', 'B07', 'A90'} 
	  } 
	   } 
	   } 
	  \\ 
 

    \theutterance \stepcounter{utterance}  

    & & & \multicolumn{2}{p{0.3\linewidth}}{\cellcolor[rgb]{0.95,0.95,0.95}{%
	\makecell[{{p{\linewidth}}}]{% 
	  \tt {\tiny [GM$|$GM]}  
	 game successful 
	  } 
	   } 
	   } 
	 & & \\ 
 

    \theutterance \stepcounter{utterance}  

    & & & \multicolumn{2}{p{0.3\linewidth}}{\cellcolor[rgb]{0.95,0.95,0.95}{%
	\makecell[{{p{\linewidth}}}]{% 
	  \tt {\tiny [GM$|$GM]}  
	 end game 
	  } 
	   } 
	   } 
	 & & \\ 
 

\end{supertabular}
}

\end{document}
