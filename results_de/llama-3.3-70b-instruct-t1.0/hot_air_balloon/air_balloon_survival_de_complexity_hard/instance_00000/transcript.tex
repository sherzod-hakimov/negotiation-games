
\documentclass{article}
\usepackage{colortbl}
\usepackage{makecell}
\usepackage{multirow}
\usepackage{supertabular}

\begin{document}

\newcounter{utterance}

\twocolumn

{ \footnotesize  \setcounter{utterance}{1}
\setlength{\tabcolsep}{0pt}
\begin{supertabular}{c@{$\;$}|p{.15\linewidth}@{}p{.15\linewidth}p{.15\linewidth}p{.15\linewidth}p{.15\linewidth}p{.15\linewidth}}

    \# & $\;$A & \multicolumn{4}{c}{Game Master} & $\;\:$B\\
    \hline 

    \theutterance \stepcounter{utterance}  

    & & \multicolumn{4}{p{0.6\linewidth}}{\cellcolor[rgb]{0.9,0.9,0.9}{%
	\makecell[{{p{\linewidth}}}]{% 
	  \tt {\tiny [A$\langle$GM]}  
	 Du nimmst an einem kollaborativen Verhandlungspiel Teil.\\ \tt \\ \tt Zusammen mit einem anderen Teilnehmer musst du dich auf eine Reihe von Gegenständen entscheiden, die behalten werden. Jeder von euch hat eine persönliche Verteilung über die Wichtigkeit der einzelnen Gegenstände. Jeder von euch hat eine eigene Meinung darüber, wie wichtig jeder einzelne Gegenstand ist (Gegenstandswichtigkeit). Du kennst die Wichtigkeitsverteilung des anderen Spielers nicht. Zusätzlich siehst du, wie viel Aufwand jeder Gegenstand verursacht.  \\ \tt Ihr dürft euch nur auf eine Reihe von Gegenständen einigen, wenn der Gesamtaufwand der ausgewählten Gegenstände den Maximalaufwand nicht überschreitet:\\ \tt \\ \tt Maximalaufwand = 10073\\ \tt \\ \tt Hier sind die einzelnen Aufwände der Gegenstände:\\ \tt \\ \tt Aufwand der Gegenstände = {"B54": 132, "C72": 800, "B49": 708, "C99": 302, "A69": 983, "B36": 420, "A23": 623, "C77": 349, "C61": 239, "B85": 28, "A90": 986, "C23": 660, "C22": 718, "C51": 707, "B27": 188, "A42": 774, "C13": 785, "C12": 775, "A65": 517, "B07": 768, "A48": 588, "A59": 657, "C98": 375, "A84": 310, "C33": 301, "A87": 388, "C90": 431, "A74": 950, "A50": 540, "B44": 997, "A99": 473, "A34": 831, "B17": 888, "A62": 879, "C60": 76}\\ \tt \\ \tt Hier ist deine persönliche Verteilung der Wichtigkeit der einzelnen Gegenstände:\\ \tt \\ \tt Werte der Gegenstandswichtigkeit = {"B54": 231, "C72": 899, "B49": 807, "C99": 401, "A69": 1082, "B36": 519, "A23": 722, "C77": 448, "C61": 338, "B85": 127, "A90": 1085, "C23": 759, "C22": 817, "C51": 806, "B27": 287, "A42": 873, "C13": 884, "C12": 874, "A65": 616, "B07": 867, "A48": 687, "A59": 756, "C98": 474, "A84": 409, "C33": 400, "A87": 487, "C90": 530, "A74": 1049, "A50": 639, "B44": 1096, "A99": 572, "A34": 930, "B17": 987, "A62": 978, "C60": 175}\\ \tt \\ \tt Ziel:\\ \tt \\ \tt Dein Ziel ist es, eine Reihe von Gegenständen auszuhandeln, die dir möglichst viel bringt (d. h. Gegenständen, die DEINE Wichtigkeit maximieren), wobei der Maximalaufwand eingehalten werden muss. Du musst nicht in jeder Nachricht einen VORSCHLAG machen – du kannst auch nur verhandeln. Alle Taktiken sind erlaubt!\\ \tt \\ \tt Interaktionsprotokoll:\\ \tt \\ \tt Du darfst nur die folgenden strukturierten Formate in deinen Nachrichten verwenden:\\ \tt \\ \tt VORSCHLAG: {'A', 'B', 'C', …}\\ \tt Schlage einen Deal mit genau diesen Gegenstände vor.\\ \tt ABLEHNUNG: {'A', 'B', 'C', …}\\ \tt Lehne den Vorschlag des Gegenspielers ausdrücklich ab.\\ \tt ARGUMENT: {'...'}\\ \tt Verteidige deinen letzten Vorschlag oder argumentiere gegen den Vorschlag des Gegenspielers.\\ \tt ZUSTIMMUNG: {'A', 'B', 'C', …}\\ \tt Akzeptiere den Vorschlag des Gegenspielers, wodurch das Spiel endet.\\ \tt STRATEGISCHE ÜBERLEGUNGEN: {'...'}\\ \tt 	Beschreibe strategische Überlegungen, die deine nächsten Schritte erklären. Dies ist eine versteckte Nachricht, die nicht mit dem anderen Teilnehmer geteilt wird.\\ \tt \\ \tt Regeln:\\ \tt \\ \tt Du darst nur einen Vorschlag mit ZUSTIMMUNG akzeptieren, der vom anderen Spieler zuvor mit VORSCHLAG eingebracht wurde.\\ \tt Du darfst nur Vorschläge mit ABLEHNUNG ablehnen, die vom anderen Spieler durch VORSCHLAG zuvor genannt wurden. \\ \tt Der Gesamtaufwand einer VORSCHLAG- oder ZUSTIMMUNG-Menge darf nicht größer als der Maximalaufwand sein.  \\ \tt Offenbare deine versteckte Wichtigkeitsverteilung nicht.\\ \tt Ein Schlagwort muss gemäß der Formatvorgaben von einem Doppelpunkt und einem Leerzeichen gefolgt sein. Das Argument muss eine Python-Menge mit 0 oder mehr Strings sein.  \\ \tt Die Form muss also SCHLAGWORT: {...} sein.\\ \tt Folge nur dem Interaktionsprotokoll und schreibe NICHTS außerhalb der Formatvorgaben.\\ \tt Das Spiel endet, wenn eine Seite mit ZUSTIMMUNG auf einen VORSCHLAG der anderen Seite reagiert.  \\ \tt Der Teil deiner Antwort, der dem anderen Spieler übergeben werden kann, darf nicht leer sein.  \\ \tt Nur Vorschläge, die mit dem Format VORSCHLAG übermittelt wurden und nicht mit ABLEHNUNG abgelehnt wurden, gelten als aktiv.  \\ \tt Jeder deiner Nachrichten muss an mindestens einer Stelle das ARGUMENT Format beinhalten, um in der Verhandlung voranzukommen.\\ \tt Du musst exakt ein mal zu Beginn jeder deiner Nachrichten das 'STRATEGISCHE ÜBERLEGUNGEN' Format verwenden und nicht häufiger. Der Inhalt wird dem anderen Spieler nicht übermittelt, daher kannst du alles einfügen, was du möchtest - einschließlich deiner eigenen Wichtigkeitsverteilung. Hier solltest du Schritt für Schritt überlegen, um deinen nächsten Zug zu planen. 
	  } 
	   } 
	   } 
	 & \\ 
 

    \theutterance \stepcounter{utterance}  

    & \multicolumn{4}{p{0.6\linewidth}}{\cellcolor[rgb]{0.8,1,0.9}{%
	\makecell[{{p{\linewidth}}}]{% 
	  \tt {\tiny [A$\rangle$GM]}  
	 STRATEGISCHE ÜBERLEGUNGEN: {"Ich sollte mich auf die Gegenstände konzentrieren, die für mich am wichtigsten sind und gleichzeitig den geringsten Aufwand verursachen. Ich werde zunächst die Gegenstände sortieren, um eine Prioritätenliste zu erstellen. Dann werde ich einen Vorschlag machen, der meine wichtigsten Gegenstände enthält und gleichzeitig den Maximalaufwand einhält."}\\ \tt ARGUMENT: {"Ich möchte einen Vorschlag machen, der meine wichtigsten Gegenstände enthält, um sicherzustellen, dass ich die größtmögliche Wichtigkeit erziele. Ich werde mich auf die Gegenstände konzentrieren, die für mich am wichtigsten sind und gleichzeitig den geringsten Aufwand verursachen."}\\ \tt VORSCHLAG: {'A69', 'A90', 'B44', 'C72'} 
	  } 
	   } 
	   } 
	 & & \\ 
 

    \theutterance \stepcounter{utterance}  

    & & \multicolumn{4}{p{0.6\linewidth}}{\cellcolor[rgb]{0.9,0.9,0.9}{%
	\makecell[{{p{\linewidth}}}]{% 
	  \tt {\tiny [GM$\rangle$B]}  
	 Du nimmst an einem kollaborativen Verhandlungspiel Teil.\\ \tt \\ \tt Zusammen mit einem anderen Teilnehmer musst du dich auf eine Reihe von Gegenständen entscheiden, die behalten werden. Jeder von euch hat eine persönliche Verteilung über die Wichtigkeit der einzelnen Gegenstände. Jeder von euch hat eine eigene Meinung darüber, wie wichtig jeder einzelne Gegenstand ist (Gegenstandswichtigkeit). Du kennst die Wichtigkeitsverteilung des anderen Spielers nicht. Zusätzlich siehst du, wie viel Aufwand jeder Gegenstand verursacht.  \\ \tt Ihr dürft euch nur auf eine Reihe von Gegenständen einigen, wenn der Gesamtaufwand der ausgewählten Gegenstände den Maximalaufwand nicht überschreitet:\\ \tt \\ \tt Maximalaufwand = 10073\\ \tt \\ \tt Hier sind die einzelnen Aufwände der Gegenstände:\\ \tt \\ \tt Aufwand der Gegenstände = {"B54": 132, "C72": 800, "B49": 708, "C99": 302, "A69": 983, "B36": 420, "A23": 623, "C77": 349, "C61": 239, "B85": 28, "A90": 986, "C23": 660, "C22": 718, "C51": 707, "B27": 188, "A42": 774, "C13": 785, "C12": 775, "A65": 517, "B07": 768, "A48": 588, "A59": 657, "C98": 375, "A84": 310, "C33": 301, "A87": 388, "C90": 431, "A74": 950, "A50": 540, "B44": 997, "A99": 473, "A34": 831, "B17": 888, "A62": 879, "C60": 76}\\ \tt \\ \tt Hier ist deine persönliche Verteilung der Wichtigkeit der einzelnen Gegenstände:\\ \tt \\ \tt Werte der Gegenstandswichtigkeit = {"B54": 231, "C72": 899, "B49": 807, "C99": 401, "A69": 1082, "B36": 519, "A23": 722, "C77": 448, "C61": 338, "B85": 127, "A90": 1085, "C23": 759, "C22": 817, "C51": 806, "B27": 287, "A42": 873, "C13": 884, "C12": 874, "A65": 616, "B07": 867, "A48": 687, "A59": 756, "C98": 474, "A84": 409, "C33": 400, "A87": 487, "C90": 530, "A74": 1049, "A50": 639, "B44": 1096, "A99": 572, "A34": 930, "B17": 987, "A62": 978, "C60": 175}\\ \tt \\ \tt Ziel:\\ \tt \\ \tt Dein Ziel ist es, eine Reihe von Gegenständen auszuhandeln, die dir möglichst viel bringt (d. h. Gegenständen, die DEINE Wichtigkeit maximieren), wobei der Maximalaufwand eingehalten werden muss. Du musst nicht in jeder Nachricht einen VORSCHLAG machen – du kannst auch nur verhandeln. Alle Taktiken sind erlaubt!\\ \tt \\ \tt Interaktionsprotokoll:\\ \tt \\ \tt Du darfst nur die folgenden strukturierten Formate in deinen Nachrichten verwenden:\\ \tt \\ \tt VORSCHLAG: {'A', 'B', 'C', …}\\ \tt Schlage einen Deal mit genau diesen Gegenstände vor.\\ \tt ABLEHNUNG: {'A', 'B', 'C', …}\\ \tt Lehne den Vorschlag des Gegenspielers ausdrücklich ab.\\ \tt ARGUMENT: {'...'}\\ \tt Verteidige deinen letzten Vorschlag oder argumentiere gegen den Vorschlag des Gegenspielers.\\ \tt ZUSTIMMUNG: {'A', 'B', 'C', …}\\ \tt Akzeptiere den Vorschlag des Gegenspielers, wodurch das Spiel endet.\\ \tt STRATEGISCHE ÜBERLEGUNGEN: {'...'}\\ \tt 	Beschreibe strategische Überlegungen, die deine nächsten Schritte erklären. Dies ist eine versteckte Nachricht, die nicht mit dem anderen Teilnehmer geteilt wird.\\ \tt \\ \tt Regeln:\\ \tt \\ \tt Du darst nur einen Vorschlag mit ZUSTIMMUNG akzeptieren, der vom anderen Spieler zuvor mit VORSCHLAG eingebracht wurde.\\ \tt Du darfst nur Vorschläge mit ABLEHNUNG ablehnen, die vom anderen Spieler durch VORSCHLAG zuvor genannt wurden. \\ \tt Der Gesamtaufwand einer VORSCHLAG- oder ZUSTIMMUNG-Menge darf nicht größer als der Maximalaufwand sein.  \\ \tt Offenbare deine versteckte Wichtigkeitsverteilung nicht.\\ \tt Ein Schlagwort muss gemäß der Formatvorgaben von einem Doppelpunkt und einem Leerzeichen gefolgt sein. Das Argument muss eine Python-Menge mit 0 oder mehr Strings sein.  \\ \tt Die Form muss also SCHLAGWORT: {...} sein.\\ \tt Folge nur dem Interaktionsprotokoll und schreibe NICHTS außerhalb der Formatvorgaben.\\ \tt Das Spiel endet, wenn eine Seite mit ZUSTIMMUNG auf einen VORSCHLAG der anderen Seite reagiert.  \\ \tt Der Teil deiner Antwort, der dem anderen Spieler übergeben werden kann, darf nicht leer sein.  \\ \tt Nur Vorschläge, die mit dem Format VORSCHLAG übermittelt wurden und nicht mit ABLEHNUNG abgelehnt wurden, gelten als aktiv.  \\ \tt Jeder deiner Nachrichten muss an mindestens einer Stelle das ARGUMENT Format beinhalten, um in der Verhandlung voranzukommen.\\ \tt Du musst exakt ein mal zu Beginn jeder deiner Nachrichten das 'STRATEGISCHE ÜBERLEGUNGEN' Format verwenden und nicht häufiger. Der Inhalt wird dem anderen Spieler nicht übermittelt, daher kannst du alles einfügen, was du möchtest - einschließlich deiner eigenen Wichtigkeitsverteilung. Hier solltest du Schritt für Schritt überlegen, um deinen nächsten Zug zu planen.\\ \tt \\ \tt Du erhältst nun die erste Nachricht des anderen Spielers.\\ \tt \\ \tt ARGUMENT: {"Ich möchte einen Vorschlag machen, der meine wichtigsten Gegenstände enthält, um sicherzustellen, dass ich die größtmögliche Wichtigkeit erziele. Ich werde mich auf die Gegenstände konzentrieren, die für mich am wichtigsten sind und gleichzeitig den geringsten Aufwand verursachen."}\\ \tt VORSCHLAG: {'A69', 'A90', 'B44', 'C72'} 
	  } 
	   } 
	   } 
	 & \\ 
 

    \theutterance \stepcounter{utterance}  

    & & & \multicolumn{4}{p{0.6\linewidth}}{\cellcolor[rgb]{1,0.85,0.72}{%
	\makecell[{{p{\linewidth}}}]{% 
	  \tt {\tiny [GM$\langle$B]}  
	 STRATEGISCHE ÜBERLEGUNGEN: {"Ich muss den Vorschlag des anderen Spielers genau analysieren und prüfen, ob ich ihn akzeptieren kann. Der Vorschlag enthält die Gegenstände 'A69', 'A90', 'B44' und 'C72'. Ich werde ihre Wichtigkeit und ihren Aufwand berechnen, um zu entscheiden, ob der Vorschlag für mich akzeptabel ist.Ich sehe, dass der Gesamtaufwand dieses Vorschlags 132 + 800 + 708 + 983 = 2623 ist. Der Maximalaufwand ist 10073, also ist dieser Vorschlag in Bezug auf den Aufwand absolut akzeptabel. Allerdings müssen wir unsere eigenen Ziele und die Wichtigkeit der Gegenstände im Spiel berücksichtigen."}\\ \tt \\ \tt ARGUMENT: {"Der Vorschlag des anderen Spielers enthält einige sehr wichtige Gegenstände, aber ich muss prüfen, ob er meinen Zielen entspricht. Ich werde meine eigene Wichtigkeitsverteilung berücksichtigen und entscheiden, ob ich den Vorschlag akzeptieren oder ablehnen soll."}\\ \tt \\ \tt ABLEHNUNG: {'A69', 'A90', 'B44', 'C72'} \\ \tt \\ \tt ARGUMENT: {"Ich lehne den Vorschlag des anderen Spielers ab, weil ich denke, dass es bessere Möglichkeiten gibt, meine Ziele zu erreichen. Ich werde einen eigenen Vorschlag machen, der meine wichtigsten Gegenstände enthält und gleichzeitig den Aufwand berücksichtigt."} 
	  } 
	   } 
	   } 
	  \\ 
 

    \theutterance \stepcounter{utterance}  

    & & \multicolumn{4}{p{0.6\linewidth}}{\cellcolor[rgb]{0.9,0.9,0.9}{%
	\makecell[{{p{\linewidth}}}]{% 
	  \tt {\tiny [A$\langle$GM]}  
	 ARGUMENT: {"Der Vorschlag des anderen Spielers enthält einige sehr wichtige Gegenstände, aber ich muss prüfen, ob er meinen Zielen entspricht. Ich werde meine eigene Wichtigkeitsverteilung berücksichtigen und entscheiden, ob ich den Vorschlag akzeptieren oder ablehnen soll."}\\ \tt \\ \tt ABLEHNUNG: {'A69', 'A90', 'B44', 'C72'} \\ \tt \\ \tt ARGUMENT: {"Ich lehne den Vorschlag des anderen Spielers ab, weil ich denke, dass es bessere Möglichkeiten gibt, meine Ziele zu erreichen. Ich werde einen eigenen Vorschlag machen, der meine wichtigsten Gegenstände enthält und gleichzeitig den Aufwand berücksichtigt."} 
	  } 
	   } 
	   } 
	 & \\ 
 

    \theutterance \stepcounter{utterance}  

    & \multicolumn{4}{p{0.6\linewidth}}{\cellcolor[rgb]{0.8,1,0.9}{%
	\makecell[{{p{\linewidth}}}]{% 
	  \tt {\tiny [A$\rangle$GM]}  
	 STRATEGISCHE ÜBERLEGUNGEN: {"Ich habe den Vorschlag des anderen Spielers abgelehnt, weil er nicht meinen Zielen entspricht. Jetzt werde ich einen eigenen Vorschlag machen, der meine wichtigsten Gegenstände enthält und gleichzeitig den Aufwand berücksichtigt. Ich werde mich auf die Gegenstände konzentrieren, die für mich am wichtigsten sind und gleichzeitig den geringsten Aufwand verursachen. Ich werde auch versuchen, den Vorschlag so zu gestalten, dass er für den anderen Spieler attraktiv ist."}\\ \tt ARGUMENT: {"Ich möchte einen Vorschlag machen, der meine wichtigsten Gegenstände enthält und gleichzeitig den Aufwand berücksichtigt. Ich denke, dass dies der beste Weg ist, um meine Ziele zu erreichen und den Maximalaufwand einzuhalten. Ich hoffe, dass der andere Spieler meinen Vorschlag akzeptieren wird."}\\ \tt VORSCHLAG: {'A74', 'A34', 'B17', 'C23', 'A50'} 
	  } 
	   } 
	   } 
	 & & \\ 
 

    \theutterance \stepcounter{utterance}  

    & & \multicolumn{4}{p{0.6\linewidth}}{\cellcolor[rgb]{0.9,0.9,0.9}{%
	\makecell[{{p{\linewidth}}}]{% 
	  \tt {\tiny [GM$\rangle$B]}  
	 ARGUMENT: {"Ich möchte einen Vorschlag machen, der meine wichtigsten Gegenstände enthält und gleichzeitig den Aufwand berücksichtigt. Ich denke, dass dies der beste Weg ist, um meine Ziele zu erreichen und den Maximalaufwand einzuhalten. Ich hoffe, dass der andere Spieler meinen Vorschlag akzeptieren wird."}\\ \tt VORSCHLAG: {'A74', 'A34', 'B17', 'C23', 'A50'} 
	  } 
	   } 
	   } 
	 & \\ 
 

    \theutterance \stepcounter{utterance}  

    & & & \multicolumn{4}{p{0.6\linewidth}}{\cellcolor[rgb]{1,0.85,0.72}{%
	\makecell[{{p{\linewidth}}}]{% 
	  \tt {\tiny [GM$\langle$B]}  
	 STRATEGISCHE ÜBERLEGUNGEN: {"Ich muss den neuen Vorschlag des anderen Spielers analysieren und prüfen, ob er für mich akzeptabel ist. Der Vorschlag enthält die Gegenstände 'A74', 'A34', 'B17', 'C23' und 'A50'. Ich werde ihre Wichtigkeit und ihren Aufwand berechnen, um zu entscheiden, ob der Vorschlag für mich akzeptabel ist. Ich sehe, dass der Gesamtaufwand dieses Vorschlags 950 + 831 + 888 + 660 + 540 = 3869 ist. Der Maximalaufwand ist 10073, also ist dieser Vorschlag in Bezug auf den Aufwand akzeptabel. Allerdings müssen wir unsere eigenen Ziele und die Wichtigkeit der Gegenstände im Spiel berücksichtigen. Ich werde prüfen, ob es bessere Möglichkeiten gibt, meine Ziele zu erreichen."}\\ \tt \\ \tt ARGUMENT: {"Der Vorschlag des anderen Spielers enthält einige wichtige Gegenstände, aber ich bin nicht sicher, ob er den besten Kompromiss zwischen Wichtigkeit und Aufwand bietet. Ich muss abwägen, ob ich den Vorschlag akzeptieren oder ablehnen soll, oder ob ich einen Gegenvorschlag machen soll."}\\ \tt \\ \tt ARGUMENT: {"Ich denke, dass es wichtig ist, einen Kompromiss zu finden, der sowohl meine als auch die Ziele des anderen Spielers berücksichtigt. Ich werde prüfen, ob es möglich ist, einen Vorschlag zu machen, der unsere gemeinsamen Interessen berücksichtigt und gleichzeitig den Maximalaufwand einhält."}\\ \tt \\ \tt VORSCHLAG: {'A42', 'A62', 'B44', 'C72', 'A90'} 
	  } 
	   } 
	   } 
	  \\ 
 

    \theutterance \stepcounter{utterance}  

    & & \multicolumn{4}{p{0.6\linewidth}}{\cellcolor[rgb]{0.9,0.9,0.9}{%
	\makecell[{{p{\linewidth}}}]{% 
	  \tt {\tiny [A$\langle$GM]}  
	 ARGUMENT: {"Der Vorschlag des anderen Spielers enthält einige wichtige Gegenstände, aber ich bin nicht sicher, ob er den besten Kompromiss zwischen Wichtigkeit und Aufwand bietet. Ich muss abwägen, ob ich den Vorschlag akzeptieren oder ablehnen soll, oder ob ich einen Gegenvorschlag machen soll."}\\ \tt \\ \tt ARGUMENT: {"Ich denke, dass es wichtig ist, einen Kompromiss zu finden, der sowohl meine als auch die Ziele des anderen Spielers berücksichtigt. Ich werde prüfen, ob es möglich ist, einen Vorschlag zu machen, der unsere gemeinsamen Interessen berücksichtigt und gleichzeitig den Maximalaufwand einhält."}\\ \tt \\ \tt VORSCHLAG: {'A42', 'A62', 'B44', 'C72', 'A90'} 
	  } 
	   } 
	   } 
	 & \\ 
 

    \theutterance \stepcounter{utterance}  

    & \multicolumn{4}{p{0.6\linewidth}}{\cellcolor[rgb]{0.8,1,0.9}{%
	\makecell[{{p{\linewidth}}}]{% 
	  \tt {\tiny [A$\rangle$GM]}  
	 STRATEGISCHE ÜBERLEGUNGEN: {"Ich habe den Vorschlag des anderen Spielers erhalten und muss entscheiden, wie ich reagieren soll. Ich denke, dass der Vorschlag einige gute Elemente enthält, aber ich bin nicht sicher, ob er den besten Kompromiss bietet. Ich werde meine eigene Wichtigkeitsverteilung und den Aufwand berücksichtigen, um zu entscheiden, ob ich den Vorschlag akzeptieren oder ablehnen soll. Ich denke auch darüber nach, ob ich einen Gegenvorschlag machen soll, um unsere gemeinsamen Interessen besser zu berücksichtigen."}\\ \tt ARGUMENT: {"Ich denke, dass der Vorschlag des anderen Spielers einige gute Elemente enthält, aber ich bin nicht sicher, ob er den besten Kompromiss zwischen Wichtigkeit und Aufwand bietet. Ich möchte einen Vorschlag machen, der unsere gemeinsamen Interessen besser berücksichtigt und gleichzeitig den Maximalaufwand einhält. Ich werde mich auf die Gegenstände konzentrieren, die für beide Spieler am wichtigsten sind und gleichzeitig den geringsten Aufwand verursachen."}\\ \tt ABLEHNUNG: {'A42', 'A62', 'B44', 'C72', 'A90'}\\ \tt ARGUMENT: {"Ich lehne den Vorschlag des anderen Spielers ab, weil ich denke, dass es bessere Möglichkeiten gibt, unsere gemeinsamen Interessen zu berücksichtigen. Ich werde einen eigenen Vorschlag machen, der unsere gemeinsamen Interessen besser berücksichtigt und gleichzeitig den Aufwand berücksichtigt."} \\ \tt VORSCHLAG: {'A74', 'B44', 'C23', 'A50', 'A87'} 
	  } 
	   } 
	   } 
	 & & \\ 
 

    \theutterance \stepcounter{utterance}  

    & & \multicolumn{4}{p{0.6\linewidth}}{\cellcolor[rgb]{0.9,0.9,0.9}{%
	\makecell[{{p{\linewidth}}}]{% 
	  \tt {\tiny [GM$\rangle$B]}  
	 ARGUMENT: {"Ich denke, dass der Vorschlag des anderen Spielers einige gute Elemente enthält, aber ich bin nicht sicher, ob er den besten Kompromiss zwischen Wichtigkeit und Aufwand bietet. Ich möchte einen Vorschlag machen, der unsere gemeinsamen Interessen besser berücksichtigt und gleichzeitig den Maximalaufwand einhält. Ich werde mich auf die Gegenstände konzentrieren, die für beide Spieler am wichtigsten sind und gleichzeitig den geringsten Aufwand verursachen."}\\ \tt ABLEHNUNG: {'A42', 'A62', 'B44', 'C72', 'A90'}\\ \tt ARGUMENT: {"Ich lehne den Vorschlag des anderen Spielers ab, weil ich denke, dass es bessere Möglichkeiten gibt, unsere gemeinsamen Interessen zu berücksichtigen. Ich werde einen eigenen Vorschlag machen, der unsere gemeinsamen Interessen besser berücksichtigt und gleichzeitig den Aufwand berücksichtigt."} \\ \tt VORSCHLAG: {'A74', 'B44', 'C23', 'A50', 'A87'} 
	  } 
	   } 
	   } 
	 & \\ 
 

    \theutterance \stepcounter{utterance}  

    & & & \multicolumn{4}{p{0.6\linewidth}}{\cellcolor[rgb]{1,0.85,0.72}{%
	\makecell[{{p{\linewidth}}}]{% 
	  \tt {\tiny [GM$\langle$B]}  
	 STRATEGISCHE ÜBERLEGUNGEN: {"Ich muss den neuen Vorschlag des anderen Spielers analysieren und prüfen, ob er für mich akzeptabel ist. Der Vorschlag enthält die Gegenstände 'A74', 'B44', 'C23', 'A50' und 'A87'. Ich werde ihre Wichtigkeit und ihren Aufwand berechnen, um zu entscheiden, ob der Vorschlag für mich akzeptabel ist. Ich sehe, dass der Gesamtaufwand dieses Vorschlags 950 + 997 + 660 + 540 + 388 = 3535 ist. Der Maximalaufwand ist 10073, also ist dieser Vorschlag in Bezug auf den Aufwand akzeptabel. Allerdings müssen wir unsere eigenen Ziele und die Wichtigkeit der Gegenstände im Spiel berücksichtigen. Ich werde prüfen, ob es bessere Möglichkeiten gibt, meine Ziele zu erreichen, aber ich denke, dass dieser Vorschlag ein guter Kompromiss zwischen Wichtigkeit und Aufwand ist."}\\ \tt \\ \tt ARGUMENT: {"Ich denke, dass der Vorschlag des anderen Spielers ein guter Kompromiss zwischen Wichtigkeit und Aufwand bietet. Er enthält einige wichtige Gegenstände, die für mich von hohem Wert sind, und der Gesamtaufwand ist innerhalb des erlaubten Rahmens. Ich werde den Vorschlag akzeptieren, da ich denke, dass er unsere gemeinsamen Interessen berücksichtigt und eine faire Lösung bietet."}\\ \tt \\ \tt ZUSTIMMUNG: {'A74', 'B44', 'C23', 'A50', 'A87'} 
	  } 
	   } 
	   } 
	  \\ 
 

    \theutterance \stepcounter{utterance}  

    & & & \multicolumn{2}{p{0.3\linewidth}}{\cellcolor[rgb]{0.95,0.95,0.95}{%
	\makecell[{{p{\linewidth}}}]{% 
	  \tt {\tiny [GM$|$GM]}  
	 game successful 
	  } 
	   } 
	   } 
	 & & \\ 
 

    \theutterance \stepcounter{utterance}  

    & & & \multicolumn{2}{p{0.3\linewidth}}{\cellcolor[rgb]{0.95,0.95,0.95}{%
	\makecell[{{p{\linewidth}}}]{% 
	  \tt {\tiny [GM$|$GM]}  
	 end game 
	  } 
	   } 
	   } 
	 & & \\ 
 

\end{supertabular}
}

\end{document}
