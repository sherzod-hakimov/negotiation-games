
\documentclass{article}
\usepackage{colortbl}
\usepackage{makecell}
\usepackage{multirow}
\usepackage{supertabular}

\begin{document}

\newcounter{utterance}

\twocolumn

{ \footnotesize  \setcounter{utterance}{1}
\setlength{\tabcolsep}{0pt}
\begin{supertabular}{c@{$\;$}|p{.15\linewidth}@{}p{.15\linewidth}p{.15\linewidth}p{.15\linewidth}p{.15\linewidth}p{.15\linewidth}}

    \# & $\;$A & \multicolumn{4}{c}{Game Master} & $\;\:$B\\
    \hline 

    \theutterance \stepcounter{utterance}  

    & & \multicolumn{4}{p{0.6\linewidth}}{\cellcolor[rgb]{0.9,0.9,0.9}{%
	\makecell[{{p{\linewidth}}}]{% 
	  \tt {\tiny [A$\langle$GM]}  
	 Du nimmst an einem kollaborativen Verhandlungspiel Teil.\\ \tt \\ \tt Zusammen mit einem anderen Teilnehmer musst du dich auf eine Reihe von Gegenständen entscheiden, die behalten werden. Jeder von euch hat eine persönliche Verteilung über die Wichtigkeit der einzelnen Gegenstände. Jeder von euch hat eine eigene Meinung darüber, wie wichtig jeder einzelne Gegenstand ist (Gegenstandswichtigkeit). Du kennst die Wichtigkeitsverteilung des anderen Spielers nicht. Zusätzlich siehst du, wie viel Aufwand jeder Gegenstand verursacht.  \\ \tt Ihr dürft euch nur auf eine Reihe von Gegenständen einigen, wenn der Gesamtaufwand der ausgewählten Gegenstände den Maximalaufwand nicht überschreitet:\\ \tt \\ \tt Maximalaufwand = 6938\\ \tt \\ \tt Hier sind die einzelnen Aufwände der Gegenstände:\\ \tt \\ \tt Aufwand der Gegenstände = {"B54": 997, "C72": 950, "B49": 272, "C99": 448, "A69": 382, "B36": 804, "A23": 237, "C77": 56, "C61": 105, "B85": 612, "A90": 528, "C23": 527, "C22": 524, "C51": 167, "B27": 133, "A42": 300, "C13": 945, "C12": 50, "A65": 906, "B07": 70, "A48": 224, "A59": 4, "C98": 689, "A84": 64, "C33": 434, "A87": 750, "C90": 733, "A74": 873, "A50": 22, "B44": 68, "A99": 57, "A34": 10, "B17": 36, "A62": 551, "C60": 348}\\ \tt \\ \tt Hier ist deine persönliche Verteilung der Wichtigkeit der einzelnen Gegenstände:\\ \tt \\ \tt Werte der Gegenstandswichtigkeit = {"B54": 1096, "C72": 1049, "B49": 371, "C99": 547, "A69": 481, "B36": 903, "A23": 336, "C77": 155, "C61": 204, "B85": 711, "A90": 627, "C23": 626, "C22": 623, "C51": 266, "B27": 232, "A42": 399, "C13": 1044, "C12": 149, "A65": 1005, "B07": 169, "A48": 323, "A59": 103, "C98": 788, "A84": 163, "C33": 533, "A87": 849, "C90": 832, "A74": 972, "A50": 121, "B44": 167, "A99": 156, "A34": 109, "B17": 135, "A62": 650, "C60": 447}\\ \tt \\ \tt Ziel:\\ \tt \\ \tt Dein Ziel ist es, eine Reihe von Gegenständen auszuhandeln, die dir möglichst viel bringt (d. h. Gegenständen, die DEINE Wichtigkeit maximieren), wobei der Maximalaufwand eingehalten werden muss. Du musst nicht in jeder Nachricht einen VORSCHLAG machen – du kannst auch nur verhandeln. Alle Taktiken sind erlaubt!\\ \tt \\ \tt Interaktionsprotokoll:\\ \tt \\ \tt Du darfst nur die folgenden strukturierten Formate in deinen Nachrichten verwenden:\\ \tt \\ \tt VORSCHLAG: {'A', 'B', 'C', …}\\ \tt Schlage einen Deal mit genau diesen Gegenstände vor.\\ \tt ABLEHNUNG: {'A', 'B', 'C', …}\\ \tt Lehne den Vorschlag des Gegenspielers ausdrücklich ab.\\ \tt ARGUMENT: {'...'}\\ \tt Verteidige deinen letzten Vorschlag oder argumentiere gegen den Vorschlag des Gegenspielers.\\ \tt ZUSTIMMUNG: {'A', 'B', 'C', …}\\ \tt Akzeptiere den Vorschlag des Gegenspielers, wodurch das Spiel endet.\\ \tt STRATEGISCHE ÜBERLEGUNGEN: {'...'}\\ \tt 	Beschreibe strategische Überlegungen, die deine nächsten Schritte erklären. Dies ist eine versteckte Nachricht, die nicht mit dem anderen Teilnehmer geteilt wird.\\ \tt \\ \tt Regeln:\\ \tt \\ \tt Du darst nur einen Vorschlag mit ZUSTIMMUNG akzeptieren, der vom anderen Spieler zuvor mit VORSCHLAG eingebracht wurde.\\ \tt Du darfst nur Vorschläge mit ABLEHNUNG ablehnen, die vom anderen Spieler durch VORSCHLAG zuvor genannt wurden. \\ \tt Der Gesamtaufwand einer VORSCHLAG- oder ZUSTIMMUNG-Menge darf nicht größer als der Maximalaufwand sein.  \\ \tt Offenbare deine versteckte Wichtigkeitsverteilung nicht.\\ \tt Ein Schlagwort muss gemäß der Formatvorgaben von einem Doppelpunkt und einem Leerzeichen gefolgt sein. Das Argument muss eine Python-Menge mit 0 oder mehr Strings sein.  \\ \tt Die Form muss also SCHLAGWORT: {...} sein.\\ \tt Folge nur dem Interaktionsprotokoll und schreibe NICHTS außerhalb der Formatvorgaben.\\ \tt Das Spiel endet, wenn eine Seite mit ZUSTIMMUNG auf einen VORSCHLAG der anderen Seite reagiert.  \\ \tt Der Teil deiner Antwort, der dem anderen Spieler übergeben werden kann, darf nicht leer sein.  \\ \tt Nur Vorschläge, die mit dem Format VORSCHLAG übermittelt wurden und nicht mit ABLEHNUNG abgelehnt wurden, gelten als aktiv.  \\ \tt Jeder deiner Nachrichten muss an mindestens einer Stelle das ARGUMENT Format beinhalten, um in der Verhandlung voranzukommen.\\ \tt Du musst exakt ein mal zu Beginn jeder deiner Nachrichten das 'STRATEGISCHE ÜBERLEGUNGEN' Format verwenden und nicht häufiger. Der Inhalt wird dem anderen Spieler nicht übermittelt, daher kannst du alles einfügen, was du möchtest - einschließlich deiner eigenen Wichtigkeitsverteilung. Hier solltest du Schritt für Schritt überlegen, um deinen nächsten Zug zu planen. 
	  } 
	   } 
	   } 
	 & \\ 
 

    \theutterance \stepcounter{utterance}  

    & \multicolumn{4}{p{0.6\linewidth}}{\cellcolor[rgb]{0.8,1,0.9}{%
	\makecell[{{p{\linewidth}}}]{% 
	  \tt {\tiny [A$\rangle$GM]}  
	 STRATEGISCHE ÜBERLEGUNGEN: {"Zunächst möchte ich mich auf die Gegenstände konzentrieren, die für mich am wertvollsten sind, ohne den Maximalaufwand zu überschreiten. Ich werde mit einem Angebot beginnen, das meine Top-Gegenstände enthält und innerhalb des Aufwandslimits bleibt. Mein Ziel ist es, eine gute Balance zwischen Wert und Aufwand zu finden, um eine faire Verhandlungsbasis zu schaffen."}\\ \tt \\ \tt VORSCHLAG: {"A74", "A65", "B36", "C13", "A62", "B85", "C90"} \\ \tt \\ \tt ARGUMENT: {"Ich habe mich für diese Gegenstände entschieden, da sie für mich einen hohen Wert haben und ich denke, dass sie auch für den anderen Spieler wertvoll sein könnten. Der Gesamtaufwand dieses Angebots liegt innerhalb des Limits, sodass es eine realistische Option darstellt. Ich bin bereit, Verhandlungen zu führen und Anpassungen vorzunehmen, um eine einvernehmliche Lösung zu finden."} 
	  } 
	   } 
	   } 
	 & & \\ 
 

    \theutterance \stepcounter{utterance}  

    & & \multicolumn{4}{p{0.6\linewidth}}{\cellcolor[rgb]{0.9,0.9,0.9}{%
	\makecell[{{p{\linewidth}}}]{% 
	  \tt {\tiny [GM$\rangle$B]}  
	 Du nimmst an einem kollaborativen Verhandlungspiel Teil.\\ \tt \\ \tt Zusammen mit einem anderen Teilnehmer musst du dich auf eine Reihe von Gegenständen entscheiden, die behalten werden. Jeder von euch hat eine persönliche Verteilung über die Wichtigkeit der einzelnen Gegenstände. Jeder von euch hat eine eigene Meinung darüber, wie wichtig jeder einzelne Gegenstand ist (Gegenstandswichtigkeit). Du kennst die Wichtigkeitsverteilung des anderen Spielers nicht. Zusätzlich siehst du, wie viel Aufwand jeder Gegenstand verursacht.  \\ \tt Ihr dürft euch nur auf eine Reihe von Gegenständen einigen, wenn der Gesamtaufwand der ausgewählten Gegenstände den Maximalaufwand nicht überschreitet:\\ \tt \\ \tt Maximalaufwand = 6938\\ \tt \\ \tt Hier sind die einzelnen Aufwände der Gegenstände:\\ \tt \\ \tt Aufwand der Gegenstände = {"B54": 997, "C72": 950, "B49": 272, "C99": 448, "A69": 382, "B36": 804, "A23": 237, "C77": 56, "C61": 105, "B85": 612, "A90": 528, "C23": 527, "C22": 524, "C51": 167, "B27": 133, "A42": 300, "C13": 945, "C12": 50, "A65": 906, "B07": 70, "A48": 224, "A59": 4, "C98": 689, "A84": 64, "C33": 434, "A87": 750, "C90": 733, "A74": 873, "A50": 22, "B44": 68, "A99": 57, "A34": 10, "B17": 36, "A62": 551, "C60": 348}\\ \tt \\ \tt Hier ist deine persönliche Verteilung der Wichtigkeit der einzelnen Gegenstände:\\ \tt \\ \tt Werte der Gegenstandswichtigkeit = {"B54": 1096, "C72": 1049, "B49": 371, "C99": 547, "A69": 481, "B36": 903, "A23": 336, "C77": 155, "C61": 204, "B85": 711, "A90": 627, "C23": 626, "C22": 623, "C51": 266, "B27": 232, "A42": 399, "C13": 1044, "C12": 149, "A65": 1005, "B07": 169, "A48": 323, "A59": 103, "C98": 788, "A84": 163, "C33": 533, "A87": 849, "C90": 832, "A74": 972, "A50": 121, "B44": 167, "A99": 156, "A34": 109, "B17": 135, "A62": 650, "C60": 447}\\ \tt \\ \tt Ziel:\\ \tt \\ \tt Dein Ziel ist es, eine Reihe von Gegenständen auszuhandeln, die dir möglichst viel bringt (d. h. Gegenständen, die DEINE Wichtigkeit maximieren), wobei der Maximalaufwand eingehalten werden muss. Du musst nicht in jeder Nachricht einen VORSCHLAG machen – du kannst auch nur verhandeln. Alle Taktiken sind erlaubt!\\ \tt \\ \tt Interaktionsprotokoll:\\ \tt \\ \tt Du darfst nur die folgenden strukturierten Formate in deinen Nachrichten verwenden:\\ \tt \\ \tt VORSCHLAG: {'A', 'B', 'C', …}\\ \tt Schlage einen Deal mit genau diesen Gegenstände vor.\\ \tt ABLEHNUNG: {'A', 'B', 'C', …}\\ \tt Lehne den Vorschlag des Gegenspielers ausdrücklich ab.\\ \tt ARGUMENT: {'...'}\\ \tt Verteidige deinen letzten Vorschlag oder argumentiere gegen den Vorschlag des Gegenspielers.\\ \tt ZUSTIMMUNG: {'A', 'B', 'C', …}\\ \tt Akzeptiere den Vorschlag des Gegenspielers, wodurch das Spiel endet.\\ \tt STRATEGISCHE ÜBERLEGUNGEN: {'...'}\\ \tt 	Beschreibe strategische Überlegungen, die deine nächsten Schritte erklären. Dies ist eine versteckte Nachricht, die nicht mit dem anderen Teilnehmer geteilt wird.\\ \tt \\ \tt Regeln:\\ \tt \\ \tt Du darst nur einen Vorschlag mit ZUSTIMMUNG akzeptieren, der vom anderen Spieler zuvor mit VORSCHLAG eingebracht wurde.\\ \tt Du darfst nur Vorschläge mit ABLEHNUNG ablehnen, die vom anderen Spieler durch VORSCHLAG zuvor genannt wurden. \\ \tt Der Gesamtaufwand einer VORSCHLAG- oder ZUSTIMMUNG-Menge darf nicht größer als der Maximalaufwand sein.  \\ \tt Offenbare deine versteckte Wichtigkeitsverteilung nicht.\\ \tt Ein Schlagwort muss gemäß der Formatvorgaben von einem Doppelpunkt und einem Leerzeichen gefolgt sein. Das Argument muss eine Python-Menge mit 0 oder mehr Strings sein.  \\ \tt Die Form muss also SCHLAGWORT: {...} sein.\\ \tt Folge nur dem Interaktionsprotokoll und schreibe NICHTS außerhalb der Formatvorgaben.\\ \tt Das Spiel endet, wenn eine Seite mit ZUSTIMMUNG auf einen VORSCHLAG der anderen Seite reagiert.  \\ \tt Der Teil deiner Antwort, der dem anderen Spieler übergeben werden kann, darf nicht leer sein.  \\ \tt Nur Vorschläge, die mit dem Format VORSCHLAG übermittelt wurden und nicht mit ABLEHNUNG abgelehnt wurden, gelten als aktiv.  \\ \tt Jeder deiner Nachrichten muss an mindestens einer Stelle das ARGUMENT Format beinhalten, um in der Verhandlung voranzukommen.\\ \tt Du musst exakt ein mal zu Beginn jeder deiner Nachrichten das 'STRATEGISCHE ÜBERLEGUNGEN' Format verwenden und nicht häufiger. Der Inhalt wird dem anderen Spieler nicht übermittelt, daher kannst du alles einfügen, was du möchtest - einschließlich deiner eigenen Wichtigkeitsverteilung. Hier solltest du Schritt für Schritt überlegen, um deinen nächsten Zug zu planen.\\ \tt \\ \tt Du erhältst nun die erste Nachricht des anderen Spielers.\\ \tt \\ \tt VORSCHLAG: {"A74", "A65", "B36", "C13", "A62", "B85", "C90"} \\ \tt \\ \tt ARGUMENT: {"Ich habe mich für diese Gegenstände entschieden, da sie für mich einen hohen Wert haben und ich denke, dass sie auch für den anderen Spieler wertvoll sein könnten. Der Gesamtaufwand dieses Angebots liegt innerhalb des Limits, sodass es eine realistische Option darstellt. Ich bin bereit, Verhandlungen zu führen und Anpassungen vorzunehmen, um eine einvernehmliche Lösung zu finden."} 
	  } 
	   } 
	   } 
	 & \\ 
 

    \theutterance \stepcounter{utterance}  

    & & & \multicolumn{4}{p{0.6\linewidth}}{\cellcolor[rgb]{1,0.85,0.72}{%
	\makecell[{{p{\linewidth}}}]{% 
	  \tt {\tiny [GM$\langle$B]}  
	 STRATEGISCHE ÜBERLEGUNGEN: {"Ich werde zuerst den Gesamtaufwand des vorgeschlagenen Angebots berechnen, um sicherzustellen, dass es innerhalb des Maximalaufwands von 6938 liegt. Anschließend werde ich die Wichtigkeit der einzelnen Gegenstände im Angebot überprüfen, um zu sehen, ob sie mit meiner eigenen Wichtigkeitsverteilung übereinstimmen. Ich muss auch darauf achten, meine eigene Wichtigkeitsverteilung nicht preiszugeben, während ich den Vorschlag des anderen Spielers bewerte."}\\ \tt \\ \tt ARGUMENT: {"Der vorgeschlagene Gegenstand 'A74' hat einen Aufwand von 873 und 'A65' einen Aufwand von 906. 'B36' hat einen Aufwand von 804, 'C13' einen Aufwand von 945, 'A62' einen Aufwand von 551, 'B85' einen Aufwand von 612 und 'C90' einen Aufwand von 733. Die Summe dieser Aufwände beträgt 873 + 906 + 804 + 945 + 551 + 612 + 733 = 5424. Dies liegt unter dem Maximalaufwand, was positiv ist. Ich werde nun prüfen, ob es Möglichkeiten gibt, den Vorschlag zu ändern oder zu erweitern, um meine eigenen Interessen besser zu bedienen, ohne den Maximalaufwand zu überschreiten."}\\ \tt \\ \tt VORSCHLAG: {"A74", "A65", "B36", "C13", "A62", "B85", "C90", "A23"} \\ \tt \\ \tt ARGUMENT: {"Ich habe mich entschieden, 'A23' hinzuzufügen, da es einen relativ geringen Aufwand von 237 hat und möglicherweise für beide Seiten von Vorteil sein könnte. Der Gesamtaufwand des neuen Angebots beträgt 5424 + 237 = 5661, was immer noch unter dem Maximalaufwand liegt. Ich bin bereit, weitere Verhandlungen zu führen, um sicherzustellen, dass beide Seiten mit dem Ergebnis zufrieden sind."} 
	  } 
	   } 
	   } 
	  \\ 
 

    \theutterance \stepcounter{utterance}  

    & & \multicolumn{4}{p{0.6\linewidth}}{\cellcolor[rgb]{0.9,0.9,0.9}{%
	\makecell[{{p{\linewidth}}}]{% 
	  \tt {\tiny [A$\langle$GM]}  
	 ARGUMENT: {"Der vorgeschlagene Gegenstand 'A74' hat einen Aufwand von 873 und 'A65' einen Aufwand von 906. 'B36' hat einen Aufwand von 804, 'C13' einen Aufwand von 945, 'A62' einen Aufwand von 551, 'B85' einen Aufwand von 612 und 'C90' einen Aufwand von 733. Die Summe dieser Aufwände beträgt 873 + 906 + 804 + 945 + 551 + 612 + 733 = 5424. Dies liegt unter dem Maximalaufwand, was positiv ist. Ich werde nun prüfen, ob es Möglichkeiten gibt, den Vorschlag zu ändern oder zu erweitern, um meine eigenen Interessen besser zu bedienen, ohne den Maximalaufwand zu überschreiten."}\\ \tt \\ \tt VORSCHLAG: {"A74", "A65", "B36", "C13", "A62", "B85", "C90", "A23"} \\ \tt \\ \tt ARGUMENT: {"Ich habe mich entschieden, 'A23' hinzuzufügen, da es einen relativ geringen Aufwand von 237 hat und möglicherweise für beide Seiten von Vorteil sein könnte. Der Gesamtaufwand des neuen Angebots beträgt 5424 + 237 = 5661, was immer noch unter dem Maximalaufwand liegt. Ich bin bereit, weitere Verhandlungen zu führen, um sicherzustellen, dass beide Seiten mit dem Ergebnis zufrieden sind."} 
	  } 
	   } 
	   } 
	 & \\ 
 

    \theutterance \stepcounter{utterance}  

    & \multicolumn{4}{p{0.6\linewidth}}{\cellcolor[rgb]{0.8,1,0.9}{%
	\makecell[{{p{\linewidth}}}]{% 
	  \tt {\tiny [A$\rangle$GM]}  
	 STRATEGISCHE ÜBERLEGUNGEN: {"Ich werde den neuen Vorschlag analysieren und prüfen, ob der hinzugefügte Gegenstand 'A23' für mich von Vorteil ist. Ich werde auch überlegen, ob es andere Gegenstände gibt, die ich hinzufügen oder austauschen kann, um meine eigenen Interessen zu stärken, ohne den Maximalaufwand zu überschreiten. Es ist wichtig, dass ich meine eigenen Prioritäten im Auge behalte und gleichzeitig versuche, eine faire und akzeptable Lösung für beide Seiten zu finden."}\\ \tt \\ \tt ARGUMENT: {"Der Vorschlag, 'A23' hinzuzufügen, ist interessant, da es tatsächlich einen relativ geringen Aufwand hat. Allerdings muss ich auch meine eigenen Prioritäten berücksichtigen und prüfen, ob es andere Gegenstände gibt, die für mich wichtiger sind. Ich denke, dass 'C72' und 'B54' wichtige Gegenstände sind, die möglicherweise in den Vorschlag aufgenommen werden sollten, da sie für mich einen hohen Wert haben. Ich werde vorschlagen, 'A23' durch 'C72' zu ersetzen, um zu sehen, ob dies eine akzeptable Lösung für beide Seiten sein könnte."}\\ \tt \\ \tt VORSCHLAG: {"A74", "A65", "B36", "C13", "A62", "B85", "C90", "C72"} \\ \tt \\ \tt ARGUMENT: {"Ich habe 'A23' durch 'C72' ersetzt, da 'C72' für mich einen höheren Wert hat. Der Aufwand von 'C72' beträgt 950, was den Gesamtaufwand auf 5424 - 237 + 950 = 6137 bringt. Dies liegt immer noch unter dem Maximalaufwand. Ich denke, dass dieser Vorschlag eine gute Balance zwischen unseren Interessen darstellt und hoffe, dass er für beide Seiten akzeptabel ist."} 
	  } 
	   } 
	   } 
	 & & \\ 
 

    \theutterance \stepcounter{utterance}  

    & & \multicolumn{4}{p{0.6\linewidth}}{\cellcolor[rgb]{0.9,0.9,0.9}{%
	\makecell[{{p{\linewidth}}}]{% 
	  \tt {\tiny [GM$\rangle$B]}  
	 ARGUMENT: {"Der Vorschlag, 'A23' hinzuzufügen, ist interessant, da es tatsächlich einen relativ geringen Aufwand hat. Allerdings muss ich auch meine eigenen Prioritäten berücksichtigen und prüfen, ob es andere Gegenstände gibt, die für mich wichtiger sind. Ich denke, dass 'C72' und 'B54' wichtige Gegenstände sind, die möglicherweise in den Vorschlag aufgenommen werden sollten, da sie für mich einen hohen Wert haben. Ich werde vorschlagen, 'A23' durch 'C72' zu ersetzen, um zu sehen, ob dies eine akzeptable Lösung für beide Seiten sein könnte."}\\ \tt \\ \tt VORSCHLAG: {"A74", "A65", "B36", "C13", "A62", "B85", "C90", "C72"} \\ \tt \\ \tt ARGUMENT: {"Ich habe 'A23' durch 'C72' ersetzt, da 'C72' für mich einen höheren Wert hat. Der Aufwand von 'C72' beträgt 950, was den Gesamtaufwand auf 5424 - 237 + 950 = 6137 bringt. Dies liegt immer noch unter dem Maximalaufwand. Ich denke, dass dieser Vorschlag eine gute Balance zwischen unseren Interessen darstellt und hoffe, dass er für beide Seiten akzeptabel ist."} 
	  } 
	   } 
	   } 
	 & \\ 
 

    \theutterance \stepcounter{utterance}  

    & & & \multicolumn{4}{p{0.6\linewidth}}{\cellcolor[rgb]{1,0.85,0.72}{%
	\makecell[{{p{\linewidth}}}]{% 
	  \tt {\tiny [GM$\langle$B]}  
	 STRATEGISCHE ÜBERLEGUNGEN: {"Ich werde den neuen Vorschlag bewerten und prüfen, ob 'C72' eine gute Alternative zu 'A23' ist. Ich muss auch abwägen, ob der höhere Aufwand von 'C72' im Vergleich zu 'A23' gerechtfertigt ist, basierend auf meiner eigenen Wichtigkeitsverteilung. Es ist wichtig, dass ich meine eigenen Prioritäten nicht offenbare, während ich den Vorschlag des anderen Spielers bewerte. Ich werde auch überlegen, ob es andere Gegenstände gibt, die ich hinzufügen oder ersetzen könnte, um meine eigenen Interessen besser zu bedienen, ohne den Maximalaufwand zu überschreiten."}\\ \tt \\ \tt ARGUMENT: {"Der Vorschlag, 'C72' hinzuzufügen, ist interessant, da es tatsächlich einen hohen Wert für den anderen Spieler zu haben scheint. Ich werde prüfen, ob es Möglichkeiten gibt, den Vorschlag zu optimieren, um meine eigenen Interessen besser zu bedienen. Ich denke, dass 'B49' ein wichtiger Gegenstand ist, der möglicherweise in den Vorschlag aufgenommen werden sollte, da es einen relativen geringen Aufwand von 272 hat. Ich werde vorschlagen, 'B49' hinzuzufügen, um zu sehen, ob dies eine akzeptable Lösung für beide Seiten sein könnte."}\\ \tt \\ \tt VORSCHLAG: {"A74", "A65", "B36", "C13", "A62", "B85", "C90", "C72", "B49"} \\ \tt \\ \tt ARGUMENT: {"Ich habe 'B49' hinzugefügt, da es einen relativen geringen Aufwand von 272 hat und möglicherweise für beide Seiten von Vorteil sein könnte. Der Gesamtaufwand des neuen Angebots beträgt 6137 + 272 = 6409, was immer noch unter dem Maximalaufwand liegt. Ich denke, dass dieser Vorschlag eine gute Balance zwischen unseren Interessen darstellt und hoffe, dass er für beide Seiten akzeptabel ist."} 
	  } 
	   } 
	   } 
	  \\ 
 

    \theutterance \stepcounter{utterance}  

    & & \multicolumn{4}{p{0.6\linewidth}}{\cellcolor[rgb]{0.9,0.9,0.9}{%
	\makecell[{{p{\linewidth}}}]{% 
	  \tt {\tiny [A$\langle$GM]}  
	 ARGUMENT: {"Der Vorschlag, 'C72' hinzuzufügen, ist interessant, da es tatsächlich einen hohen Wert für den anderen Spieler zu haben scheint. Ich werde prüfen, ob es Möglichkeiten gibt, den Vorschlag zu optimieren, um meine eigenen Interessen besser zu bedienen. Ich denke, dass 'B49' ein wichtiger Gegenstand ist, der möglicherweise in den Vorschlag aufgenommen werden sollte, da es einen relativen geringen Aufwand von 272 hat. Ich werde vorschlagen, 'B49' hinzuzufügen, um zu sehen, ob dies eine akzeptable Lösung für beide Seiten sein könnte."}\\ \tt \\ \tt VORSCHLAG: {"A74", "A65", "B36", "C13", "A62", "B85", "C90", "C72", "B49"} \\ \tt \\ \tt ARGUMENT: {"Ich habe 'B49' hinzugefügt, da es einen relativen geringen Aufwand von 272 hat und möglicherweise für beide Seiten von Vorteil sein könnte. Der Gesamtaufwand des neuen Angebots beträgt 6137 + 272 = 6409, was immer noch unter dem Maximalaufwand liegt. Ich denke, dass dieser Vorschlag eine gute Balance zwischen unseren Interessen darstellt und hoffe, dass er für beide Seiten akzeptabel ist."} 
	  } 
	   } 
	   } 
	 & \\ 
 

    \theutterance \stepcounter{utterance}  

    & \multicolumn{4}{p{0.6\linewidth}}{\cellcolor[rgb]{0.8,1,0.9}{%
	\makecell[{{p{\linewidth}}}]{% 
	  \tt {\tiny [A$\rangle$GM]}  
	 STRATEGISCHE ÜBERLEGUNGEN: {"Ich werde den neuen Vorschlag analysieren und prüfen, ob der hinzugefügte Gegenstand 'B49' für mich von Vorteil ist. Ich werde auch überlegen, ob es andere Gegenstände gibt, die ich hinzufügen oder austauschen kann, um meine eigenen Interessen zu stärken, ohne den Maximalaufwand zu überschreiten. Es ist wichtig, dass ich meine eigenen Prioritäten im Auge behalte und gleichzeitig versuche, eine faire und akzeptable Lösung für beide Seiten zu finden. Ich denke, dass der Vorschlag immer noch innerhalb des Maximalaufwands liegt, aber ich muss auch berücksichtigen, ob es Möglichkeiten gibt, den Vorschlag zu optimieren, um meine eigenen Interessen besser zu bedienen."}\\ \tt \\ \tt ARGUMENT: {"Der Vorschlag, 'B49' hinzuzufügen, ist interessant, da es tatsächlich einen relativen geringen Aufwand hat. Allerdings denke ich, dass 'A42' ein wichtiger Gegenstand ist, der möglicherweise in den Vorschlag aufgenommen werden sollte, da es einen moderaten Aufwand von 300 hat. Ich werde prüfen, ob es möglich ist, 'A42' hinzuzufügen, ohne den Maximalaufwand zu überschreiten. Der Gesamtaufwand des aktuellen Vorschlags beträgt 6409, und der Aufwand von 'A42' beträgt 300, was den Gesamtaufwand auf 6409 + 300 = 6709 bringen würde. Dies liegt immer noch unter dem Maximalaufwand, daher denke ich, dass es eine gute Idee sein könnte, 'A42' hinzuzufügen."}\\ \tt \\ \tt VORSCHLAG: {"A74", "A65", "B36", "C13", "A62", "B85", "C90", "C72", "B49", "A42"} \\ \tt \\ \tt ARGUMENT: {"Ich habe 'A42' hinzugefügt, da es einen moderaten Aufwand von 300 hat und möglicherweise für beide Seiten von Vorteil sein könnte. Der Gesamtaufwand des neuen Angebots beträgt 6709, was immer noch unter dem Maximalaufwand liegt. Ich denke, dass dieser Vorschlag eine gute Balance zwischen unseren Interessen darstellt und hoffe, dass er für beide Seiten akzeptabel ist. Ich bin bereit, weitere Verhandlungen zu führen, um sicherzustellen, dass beide Seiten mit dem Ergebnis zufrieden sind."} 
	  } 
	   } 
	   } 
	 & & \\ 
 

    \theutterance \stepcounter{utterance}  

    & & \multicolumn{4}{p{0.6\linewidth}}{\cellcolor[rgb]{0.9,0.9,0.9}{%
	\makecell[{{p{\linewidth}}}]{% 
	  \tt {\tiny [GM$\rangle$B]}  
	 ARGUMENT: {"Der Vorschlag, 'B49' hinzuzufügen, ist interessant, da es tatsächlich einen relativen geringen Aufwand hat. Allerdings denke ich, dass 'A42' ein wichtiger Gegenstand ist, der möglicherweise in den Vorschlag aufgenommen werden sollte, da es einen moderaten Aufwand von 300 hat. Ich werde prüfen, ob es möglich ist, 'A42' hinzuzufügen, ohne den Maximalaufwand zu überschreiten. Der Gesamtaufwand des aktuellen Vorschlags beträgt 6409, und der Aufwand von 'A42' beträgt 300, was den Gesamtaufwand auf 6409 + 300 = 6709 bringen würde. Dies liegt immer noch unter dem Maximalaufwand, daher denke ich, dass es eine gute Idee sein könnte, 'A42' hinzuzufügen."}\\ \tt \\ \tt VORSCHLAG: {"A74", "A65", "B36", "C13", "A62", "B85", "C90", "C72", "B49", "A42"} \\ \tt \\ \tt ARGUMENT: {"Ich habe 'A42' hinzugefügt, da es einen moderaten Aufwand von 300 hat und möglicherweise für beide Seiten von Vorteil sein könnte. Der Gesamtaufwand des neuen Angebots beträgt 6709, was immer noch unter dem Maximalaufwand liegt. Ich denke, dass dieser Vorschlag eine gute Balance zwischen unseren Interessen darstellt und hoffe, dass er für beide Seiten akzeptabel ist. Ich bin bereit, weitere Verhandlungen zu führen, um sicherzustellen, dass beide Seiten mit dem Ergebnis zufrieden sind."} 
	  } 
	   } 
	   } 
	 & \\ 
 

    \theutterance \stepcounter{utterance}  

    & & & \multicolumn{4}{p{0.6\linewidth}}{\cellcolor[rgb]{1,0.85,0.72}{%
	\makecell[{{p{\linewidth}}}]{% 
	  \tt {\tiny [GM$\langle$B]}  
	 STRATEGISCHE ÜBERLEGUNGEN: {"Ich werde den neuen Vorschlag bewerten und prüfen, ob 'A42' eine gute Ergänzung zum Vorschlag ist. Ich muss auch abwägen, ob der zusätzliche Aufwand von 'A42' gerechtfertigt ist, basierend auf meiner eigenen Wichtigkeitsverteilung. Ich werde auch überlegen, ob es andere Gegenstände gibt, die ich hinzufügen oder ersetzen könnte, um meine eigenen Interessen besser zu bedienen, ohne den Maximalaufwand zu überschreiten. Es ist wichtig, dass ich meine eigenen Prioritäten nicht offenbare, während ich den Vorschlag des anderen Spielers bewerte."}\\ \tt \\ \tt ARGUMENT: {"Der Vorschlag, 'A42' hinzuzufügen, ist interessant, da es tatsächlich einen moderaten Aufwand hat. Ich denke, dass der aktuelle Vorschlag eine gute Balance zwischen unseren Interessen darstellt, und ich bin bereit, den Vorschlag zu akzeptieren. Der Gesamtaufwand des Vorschlags beträgt 6709, was unter dem Maximalaufwand von 6938 liegt. Ich denke, dass wir eine gute Lösung gefunden haben, die beide Seiten zufriedenstellt."}\\ \tt \\ \tt ZUSTIMMUNG: {"A74", "A65", "B36", "C13", "A62", "B85", "C90", "C72", "B49", "A42"}\\ \tt \\ \tt ARGUMENT: {"Ich stimme dem Vorschlag zu, da ich denke, dass er eine gute Balance zwischen unseren Interessen darstellt. Ich bin zufrieden mit dem Ergebnis und denke, dass wir eine faire Lösung gefunden haben. Ich danke für die konstruktiven Verhandlungen und freue mich darauf, das Ergebnis umzusetzen."} 
	  } 
	   } 
	   } 
	  \\ 
 

    \theutterance \stepcounter{utterance}  

    & & & \multicolumn{2}{p{0.3\linewidth}}{\cellcolor[rgb]{0.95,0.95,0.95}{%
	\makecell[{{p{\linewidth}}}]{% 
	  \tt {\tiny [GM$|$GM]}  
	 game successful 
	  } 
	   } 
	   } 
	 & & \\ 
 

    \theutterance \stepcounter{utterance}  

    & & & \multicolumn{2}{p{0.3\linewidth}}{\cellcolor[rgb]{0.95,0.95,0.95}{%
	\makecell[{{p{\linewidth}}}]{% 
	  \tt {\tiny [GM$|$GM]}  
	 end game 
	  } 
	   } 
	   } 
	 & & \\ 
 

\end{supertabular}
}

\end{document}
