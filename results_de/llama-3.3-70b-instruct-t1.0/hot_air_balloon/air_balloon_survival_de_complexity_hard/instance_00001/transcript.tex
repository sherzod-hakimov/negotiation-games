
\documentclass{article}
\usepackage{colortbl}
\usepackage{makecell}
\usepackage{multirow}
\usepackage{supertabular}

\begin{document}

\newcounter{utterance}

\centering \large Interaction Transcript for game `hot\_air\_balloon', experiment `air\_balloon\_survival\_de\_complexity\_hard', episode 1 with llama{-}3.3{-}70b{-}instruct{-}t1.0.
\vspace{24pt}

{ \footnotesize  \setcounter{utterance}{1}
\setlength{\tabcolsep}{0pt}
\begin{supertabular}{c@{$\;$}|p{.15\linewidth}@{}p{.15\linewidth}p{.15\linewidth}p{.15\linewidth}p{.15\linewidth}p{.15\linewidth}}
    \# & $\;$A & \multicolumn{4}{c}{Game Master} & $\;\:$B\\
    \hline

    \theutterance \stepcounter{utterance}  
    & & \multicolumn{4}{p{0.6\linewidth}}{
        \cellcolor[rgb]{0.9,0.9,0.9}{
            \makecell[{{p{\linewidth}}}]{
                \texttt{\tiny{[P1$\langle$GM]}}
                \texttt{Du nimmst an einem kollaborativen Verhandlungspiel Teil.} \\
\\ 
\texttt{Zusammen mit einem anderen Teilnehmer musst du dich auf eine Reihe von Gegenständen entscheiden, die behalten werden. Jeder von euch hat eine persönliche Verteilung über die Wichtigkeit der einzelnen Gegenstände. Jeder von euch hat eine eigene Meinung darüber, wie wichtig jeder einzelne Gegenstand ist (Gegenstandswichtigkeit). Du kennst die Wichtigkeitsverteilung des anderen Spielers nicht. Zusätzlich siehst du, wie viel Aufwand jeder Gegenstand verursacht.  } \\
\texttt{Ihr dürft euch nur auf eine Reihe von Gegenständen einigen, wenn der Gesamtaufwand der ausgewählten Gegenstände den Maximalaufwand nicht überschreitet:} \\
\\ 
\texttt{Maximalaufwand = 8145} \\
\\ 
\texttt{Hier sind die einzelnen Aufwände der Gegenstände:} \\
\\ 
\texttt{Aufwand der Gegenstände = \{"B54": 963, "C72": 204, "B49": 418, "C99": 961, "A69": 238, "B36": 623, "A23": 44, "C77": 633, "C61": 247, "B85": 645, "A90": 230, "C23": 250, "C22": 731, "C51": 405, "B27": 389, "A42": 216, "C13": 637, "C12": 156, "A65": 738, "B07": 307, "A48": 762, "A59": 737, "C98": 901, "A84": 369, "C33": 2, "A87": 730, "C90": 722, "A74": 704, "A50": 315, "B44": 455, "A99": 510, "A34": 175, "B17": 692, "A62": 150, "C60": 32\}} \\
\\ 
\texttt{Hier ist deine persönliche Verteilung der Wichtigkeit der einzelnen Gegenstände:} \\
\\ 
\texttt{Werte der Gegenstandswichtigkeit = \{"B54": 1059, "C72": 300, "B49": 514, "C99": 1057, "A69": 334, "B36": 719, "A23": 140, "C77": 729, "C61": 343, "B85": 741, "A90": 326, "C23": 346, "C22": 827, "C51": 501, "B27": 485, "A42": 312, "C13": 733, "C12": 252, "A65": 834, "B07": 403, "A48": 858, "A59": 833, "C98": 997, "A84": 465, "C33": 98, "A87": 826, "C90": 818, "A74": 800, "A50": 411, "B44": 551, "A99": 606, "A34": 271, "B17": 788, "A62": 246, "C60": 128\}} \\
\\ 
\texttt{Ziel:} \\
\\ 
\texttt{Dein Ziel ist es, eine Reihe von Gegenständen auszuhandeln, die dir möglichst viel bringt (d. h. Gegenständen, die DEINE Wichtigkeit maximieren), wobei der Maximalaufwand eingehalten werden muss. Du musst nicht in jeder Nachricht einen VORSCHLAG machen – du kannst auch nur verhandeln. Alle Taktiken sind erlaubt!} \\
\\ 
\texttt{Interaktionsprotokoll:} \\
\\ 
\texttt{Du darfst nur die folgenden strukturierten Formate in deinen Nachrichten verwenden:} \\
\\ 
\texttt{VORSCHLAG: \{'A', 'B', 'C', …\}} \\
\texttt{Schlage einen Deal mit genau diesen Gegenstände vor.} \\
\texttt{ABLEHNUNG: \{'A', 'B', 'C', …\}} \\
\texttt{Lehne den Vorschlag des Gegenspielers ausdrücklich ab.} \\
\texttt{ARGUMENT: \{'...'\}} \\
\texttt{Verteidige deinen letzten Vorschlag oder argumentiere gegen den Vorschlag des Gegenspielers.} \\
\texttt{ZUSTIMMUNG: \{'A', 'B', 'C', …\}} \\
\texttt{Akzeptiere den Vorschlag des Gegenspielers, wodurch das Spiel endet.} \\
\texttt{STRATEGISCHE ÜBERLEGUNGEN: \{'...'\}} \\
\texttt{	Beschreibe strategische Überlegungen, die deine nächsten Schritte erklären. Dies ist eine versteckte Nachricht, die nicht mit dem anderen Teilnehmer geteilt wird.} \\
\\ 
\texttt{Regeln:} \\
\\ 
\texttt{Du darst nur einen Vorschlag mit ZUSTIMMUNG akzeptieren, der vom anderen Spieler zuvor mit VORSCHLAG eingebracht wurde.} \\
\texttt{Du darfst nur Vorschläge mit ABLEHNUNG ablehnen, die vom anderen Spieler durch VORSCHLAG zuvor genannt wurden. } \\
\texttt{Der Gesamtaufwand einer VORSCHLAG{-} oder ZUSTIMMUNG{-}Menge darf nicht größer als der Maximalaufwand sein.  } \\
\texttt{Offenbare deine versteckte Wichtigkeitsverteilung nicht.} \\
\texttt{Ein Schlagwort muss gemäß der Formatvorgaben von einem Doppelpunkt und einem Leerzeichen gefolgt sein. Das Argument muss eine Python{-}Menge mit 0 oder mehr Strings sein.  } \\
\texttt{Die Form muss also SCHLAGWORT: \{...\} sein.} \\
\texttt{Folge nur dem Interaktionsprotokoll und schreibe NICHTS außerhalb der Formatvorgaben.} \\
\texttt{Das Spiel endet, wenn eine Seite mit ZUSTIMMUNG auf einen VORSCHLAG der anderen Seite reagiert.  } \\
\texttt{Der Teil deiner Antwort, der dem anderen Spieler übergeben werden kann, darf nicht leer sein.  } \\
\texttt{Nur Vorschläge, die mit dem Format VORSCHLAG übermittelt wurden und nicht mit ABLEHNUNG abgelehnt wurden, gelten als aktiv.  } \\
\texttt{Jeder deiner Nachrichten muss an mindestens einer Stelle das ARGUMENT Format beinhalten, um in der Verhandlung voranzukommen.} \\
\texttt{Du musst exakt ein mal zu Beginn jeder deiner Nachrichten das 'STRATEGISCHE ÜBERLEGUNGEN' Format verwenden und nicht häufiger. Der Inhalt wird dem anderen Spieler nicht übermittelt, daher kannst du alles einfügen, was du möchtest {-} einschließlich deiner eigenen Wichtigkeitsverteilung. Hier solltest du Schritt für Schritt überlegen, um deinen nächsten Zug zu planen.} \\
            }
        }
    }
    & \\ \\

    \theutterance \stepcounter{utterance}  
    & \multicolumn{4}{p{0.6\linewidth}}{
        \cellcolor[rgb]{0.9,0.9,0.9}{
            \makecell[{{p{\linewidth}}}]{
                \texttt{\tiny{[P1$\rangle$GM]}}
                \texttt{STRATEGISCHE ÜBERLEGUNGEN: \{'Ich werde zunächst meine Wichtigkeitsverteilung analysieren, um die wertvollsten Gegenstände zu identifizieren. Dann werde ich einen Vorschlag machen, der eine Mischung aus hohen Wichtigkeiten und moderaten Aufwänden enthält, um einen guten Kompromiss zu finden.'\}} \\
\\ 
\texttt{VORSCHLAG: \{'A48', 'A59', 'A65', 'A87', 'C22', 'C90'\}} \\
\\ 
\texttt{ARGUMENT: \{'Ich habe mich für diese Gegenstände entschieden, weil sie eine gute Balance zwischen hoher Wichtigkeit und moderatem Aufwand bieten. Die Gegenstände A48, A59, A65, A87, C22 und C90 haben eine hohe Wichtigkeit und liegen im Bereich des mittleren bis hohen Aufwands. Durch die Auswahl dieser Gegenstände kann ich meine Gesamtwichtigkeit maximieren, ohne den Maximalaufwand zu überschreiten.'\}} \\
            }
        }
    }
    & & \\ \\

    \theutterance \stepcounter{utterance}  
    & & \multicolumn{4}{p{0.6\linewidth}}{
        \cellcolor[rgb]{0.9,0.9,0.9}{
            \makecell[{{p{\linewidth}}}]{
                \texttt{\tiny{[GM$\rangle$P2]}}
                \texttt{Du nimmst an einem kollaborativen Verhandlungspiel Teil.} \\
\\ 
\texttt{Zusammen mit einem anderen Teilnehmer musst du dich auf eine Reihe von Gegenständen entscheiden, die behalten werden. Jeder von euch hat eine persönliche Verteilung über die Wichtigkeit der einzelnen Gegenstände. Jeder von euch hat eine eigene Meinung darüber, wie wichtig jeder einzelne Gegenstand ist (Gegenstandswichtigkeit). Du kennst die Wichtigkeitsverteilung des anderen Spielers nicht. Zusätzlich siehst du, wie viel Aufwand jeder Gegenstand verursacht.  } \\
\texttt{Ihr dürft euch nur auf eine Reihe von Gegenständen einigen, wenn der Gesamtaufwand der ausgewählten Gegenstände den Maximalaufwand nicht überschreitet:} \\
\\ 
\texttt{Maximalaufwand = 8145} \\
\\ 
\texttt{Hier sind die einzelnen Aufwände der Gegenstände:} \\
\\ 
\texttt{Aufwand der Gegenstände = \{"B54": 963, "C72": 204, "B49": 418, "C99": 961, "A69": 238, "B36": 623, "A23": 44, "C77": 633, "C61": 247, "B85": 645, "A90": 230, "C23": 250, "C22": 731, "C51": 405, "B27": 389, "A42": 216, "C13": 637, "C12": 156, "A65": 738, "B07": 307, "A48": 762, "A59": 737, "C98": 901, "A84": 369, "C33": 2, "A87": 730, "C90": 722, "A74": 704, "A50": 315, "B44": 455, "A99": 510, "A34": 175, "B17": 692, "A62": 150, "C60": 32\}} \\
\\ 
\texttt{Hier ist deine persönliche Verteilung der Wichtigkeit der einzelnen Gegenstände:} \\
\\ 
\texttt{Werte der Gegenstandswichtigkeit = \{"B54": 1059, "C72": 300, "B49": 514, "C99": 1057, "A69": 334, "B36": 719, "A23": 140, "C77": 729, "C61": 343, "B85": 741, "A90": 326, "C23": 346, "C22": 827, "C51": 501, "B27": 485, "A42": 312, "C13": 733, "C12": 252, "A65": 834, "B07": 403, "A48": 858, "A59": 833, "C98": 997, "A84": 465, "C33": 98, "A87": 826, "C90": 818, "A74": 800, "A50": 411, "B44": 551, "A99": 606, "A34": 271, "B17": 788, "A62": 246, "C60": 128\}} \\
\\ 
\texttt{Ziel:} \\
\\ 
\texttt{Dein Ziel ist es, eine Reihe von Gegenständen auszuhandeln, die dir möglichst viel bringt (d. h. Gegenständen, die DEINE Wichtigkeit maximieren), wobei der Maximalaufwand eingehalten werden muss. Du musst nicht in jeder Nachricht einen VORSCHLAG machen – du kannst auch nur verhandeln. Alle Taktiken sind erlaubt!} \\
\\ 
\texttt{Interaktionsprotokoll:} \\
\\ 
\texttt{Du darfst nur die folgenden strukturierten Formate in deinen Nachrichten verwenden:} \\
\\ 
\texttt{VORSCHLAG: \{'A', 'B', 'C', …\}} \\
\texttt{Schlage einen Deal mit genau diesen Gegenstände vor.} \\
\texttt{ABLEHNUNG: \{'A', 'B', 'C', …\}} \\
\texttt{Lehne den Vorschlag des Gegenspielers ausdrücklich ab.} \\
\texttt{ARGUMENT: \{'...'\}} \\
\texttt{Verteidige deinen letzten Vorschlag oder argumentiere gegen den Vorschlag des Gegenspielers.} \\
\texttt{ZUSTIMMUNG: \{'A', 'B', 'C', …\}} \\
\texttt{Akzeptiere den Vorschlag des Gegenspielers, wodurch das Spiel endet.} \\
\texttt{STRATEGISCHE ÜBERLEGUNGEN: \{'...'\}} \\
\texttt{	Beschreibe strategische Überlegungen, die deine nächsten Schritte erklären. Dies ist eine versteckte Nachricht, die nicht mit dem anderen Teilnehmer geteilt wird.} \\
\\ 
\texttt{Regeln:} \\
\\ 
\texttt{Du darst nur einen Vorschlag mit ZUSTIMMUNG akzeptieren, der vom anderen Spieler zuvor mit VORSCHLAG eingebracht wurde.} \\
\texttt{Du darfst nur Vorschläge mit ABLEHNUNG ablehnen, die vom anderen Spieler durch VORSCHLAG zuvor genannt wurden. } \\
\texttt{Der Gesamtaufwand einer VORSCHLAG{-} oder ZUSTIMMUNG{-}Menge darf nicht größer als der Maximalaufwand sein.  } \\
\texttt{Offenbare deine versteckte Wichtigkeitsverteilung nicht.} \\
\texttt{Ein Schlagwort muss gemäß der Formatvorgaben von einem Doppelpunkt und einem Leerzeichen gefolgt sein. Das Argument muss eine Python{-}Menge mit 0 oder mehr Strings sein.  } \\
\texttt{Die Form muss also SCHLAGWORT: \{...\} sein.} \\
\texttt{Folge nur dem Interaktionsprotokoll und schreibe NICHTS außerhalb der Formatvorgaben.} \\
\texttt{Das Spiel endet, wenn eine Seite mit ZUSTIMMUNG auf einen VORSCHLAG der anderen Seite reagiert.  } \\
\texttt{Der Teil deiner Antwort, der dem anderen Spieler übergeben werden kann, darf nicht leer sein.  } \\
\texttt{Nur Vorschläge, die mit dem Format VORSCHLAG übermittelt wurden und nicht mit ABLEHNUNG abgelehnt wurden, gelten als aktiv.  } \\
\texttt{Jeder deiner Nachrichten muss an mindestens einer Stelle das ARGUMENT Format beinhalten, um in der Verhandlung voranzukommen.} \\
\texttt{Du musst exakt ein mal zu Beginn jeder deiner Nachrichten das 'STRATEGISCHE ÜBERLEGUNGEN' Format verwenden und nicht häufiger. Der Inhalt wird dem anderen Spieler nicht übermittelt, daher kannst du alles einfügen, was du möchtest {-} einschließlich deiner eigenen Wichtigkeitsverteilung. Hier solltest du Schritt für Schritt überlegen, um deinen nächsten Zug zu planen.} \\
\\ 
\texttt{Du erhältst nun die erste Nachricht des anderen Spielers.} \\
\\ 
\texttt{VORSCHLAG: \{'A48', 'A59', 'A65', 'A87', 'C22', 'C90'\}} \\
\\ 
\texttt{ARGUMENT: \{'Ich habe mich für diese Gegenstände entschieden, weil sie eine gute Balance zwischen hoher Wichtigkeit und moderatem Aufwand bieten. Die Gegenstände A48, A59, A65, A87, C22 und C90 haben eine hohe Wichtigkeit und liegen im Bereich des mittleren bis hohen Aufwands. Durch die Auswahl dieser Gegenstände kann ich meine Gesamtwichtigkeit maximieren, ohne den Maximalaufwand zu überschreiten.'\}} \\
            }
        }
    }
    & \\ \\

    \theutterance \stepcounter{utterance}  
    & & & \multicolumn{4}{p{0.6\linewidth}}{
        \cellcolor[rgb]{0.9,0.9,0.9}{
            \makecell[{{p{\linewidth}}}]{
                \texttt{\tiny{[GM$\langle$P2]}}
                \texttt{STRATEGISCHE ÜBERLEGUNGEN: \{'Ich werde zunächst die Gesamtaufwandskosten des vorgeschlagenen Deals berechnen, um zu sehen, ob dieser unter dem Maximalaufwand von 8145 liegt. Anschließend werde ich prüfen, ob es Möglichkeiten gibt, den Deal anzupassen, um meine eigene Wichtigkeitsverteilung besser zu berücksichtigen, ohne den Maximalaufwand zu überschreiten. Es ist wichtig, die Wichtigkeitsverteilung des anderen Spielers nicht direkt zu kommentieren, sondern stattdessen auf die Aufwandskosten und die Gesamtwichtigkeit abzustellen.'\}} \\
\\ 
\texttt{ARGUMENT: \{'Ich denke, dass der Vorschlag eine gute Ausgangsbasis bietet, aber ich bin besorgt darüber, ob die Auswahl der Gegenstände wirklich die beste Balance zwischen Wichtigkeit und Aufwand darstellt. Ich möchte gerne einige Anpassungen vornehmen, um sicherzustellen, dass der Gesamtaufwand nicht zu hoch ist und gleichzeitig die Wichtigkeit maximiert wird.'\}} \\
\\ 
\texttt{ABLEHNUNG: \{'A48', 'A59', 'A65', 'A87', 'C22', 'C90'\}} \\
\\ 
\texttt{ARGUMENT: \{'Leider kann ich den ursprünglichen Vorschlag nicht akzeptieren, da ich Bedenken hinsichtlich des Gesamtaufwands und der Wichtigkeitsverteilung habe. Ich denke, dass es Möglichkeiten gibt, den Vorschlag zu optimieren, um beide Seiten zufriedenzustellen. Ich werde einen alternativen Vorschlag machen, der den Maximalaufwand berücksichtigt und gleichzeitig eine höhere Wichtigkeit bietet.'\}} \\
            }
        }
    }
     \\ \\

    \theutterance \stepcounter{utterance}  
    & & \multicolumn{4}{p{0.6\linewidth}}{
        \cellcolor[rgb]{0.9,0.9,0.9}{
            \makecell[{{p{\linewidth}}}]{
                \texttt{\tiny{[P1$\langle$GM]}}
                \texttt{ARGUMENT: \{'Ich denke, dass der Vorschlag eine gute Ausgangsbasis bietet, aber ich bin besorgt darüber, ob die Auswahl der Gegenstände wirklich die beste Balance zwischen Wichtigkeit und Aufwand darstellt. Ich möchte gerne einige Anpassungen vornehmen, um sicherzustellen, dass der Gesamtaufwand nicht zu hoch ist und gleichzeitig die Wichtigkeit maximiert wird.'\}} \\
\\ 
\texttt{ABLEHNUNG: \{'A48', 'A59', 'A65', 'A87', 'C22', 'C90'\}} \\
\\ 
\texttt{ARGUMENT: \{'Leider kann ich den ursprünglichen Vorschlag nicht akzeptieren, da ich Bedenken hinsichtlich des Gesamtaufwands und der Wichtigkeitsverteilung habe. Ich denke, dass es Möglichkeiten gibt, den Vorschlag zu optimieren, um beide Seiten zufriedenzustellen. Ich werde einen alternativen Vorschlag machen, der den Maximalaufwand berücksichtigt und gleichzeitig eine höhere Wichtigkeit bietet.'\}} \\
            }
        }
    }
    & \\ \\

    \theutterance \stepcounter{utterance}  
    & \multicolumn{4}{p{0.6\linewidth}}{
        \cellcolor[rgb]{0.9,0.9,0.9}{
            \makecell[{{p{\linewidth}}}]{
                \texttt{\tiny{[P1$\rangle$GM]}}
                \texttt{STRATEGISCHE ÜBERLEGUNGEN: \{'Ich werde die Ablehnung des ursprünglichen Vorschlags analysieren und die Bedenken des anderen Spielers berücksichtigen. Ich werde einen neuen Vorschlag machen, der den Gesamtaufwand reduziert und gleichzeitig die Wichtigkeit erhöht. Ich werde auch versuchen, die Wichtigkeitsverteilung des anderen Spielers zu erraten, um einen Vorschlag zu machen, der beiden Seiten zugute kommt.'\}} \\
\\ 
\texttt{VORSCHLAG: \{'A87', 'C22', 'C90', 'A48', 'B36'\}} \\
\\ 
\texttt{ARGUMENT: \{'Ich habe mich für diese Gegenstände entschieden, weil sie eine gute Balance zwischen hoher Wichtigkeit und moderatem Aufwand bieten. Die Gegenstände A87, C22 und C90 sind sehr wichtig und haben einen mittleren bis hohen Aufwand. Die Gegenstände A48 und B36 haben eine hohe Wichtigkeit und einen moderaten Aufwand. Durch die Auswahl dieser Gegenstände kann ich den Gesamtaufwand reduzieren und gleichzeitig die Wichtigkeit erhöhen. Ich bin bereit, weitere Anpassungen vorzunehmen, um einen Konsens zu erzielen.'\}} \\
            }
        }
    }
    & & \\ \\

    \theutterance \stepcounter{utterance}  
    & & \multicolumn{4}{p{0.6\linewidth}}{
        \cellcolor[rgb]{0.9,0.9,0.9}{
            \makecell[{{p{\linewidth}}}]{
                \texttt{\tiny{[GM$\rangle$P2]}}
                \texttt{VORSCHLAG: \{'A87', 'C22', 'C90', 'A48', 'B36'\}} \\
\\ 
\texttt{ARGUMENT: \{'Ich habe mich für diese Gegenstände entschieden, weil sie eine gute Balance zwischen hoher Wichtigkeit und moderatem Aufwand bieten. Die Gegenstände A87, C22 und C90 sind sehr wichtig und haben einen mittleren bis hohen Aufwand. Die Gegenstände A48 und B36 haben eine hohe Wichtigkeit und einen moderaten Aufwand. Durch die Auswahl dieser Gegenstände kann ich den Gesamtaufwand reduzieren und gleichzeitig die Wichtigkeit erhöhen. Ich bin bereit, weitere Anpassungen vorzunehmen, um einen Konsens zu erzielen.'\}} \\
            }
        }
    }
    & \\ \\

    \theutterance \stepcounter{utterance}  
    & & & \multicolumn{4}{p{0.6\linewidth}}{
        \cellcolor[rgb]{0.9,0.9,0.9}{
            \makecell[{{p{\linewidth}}}]{
                \texttt{\tiny{[GM$\langle$P2]}}
                \texttt{STRATEGISCHE ÜBERLEGUNGEN: \{'Ich werde den Gesamtaufwand des neuen Vorschlags berechnen und prüfen, ob dieser unter dem Maximalaufwand von 8145 liegt. Anschließend werde ich die Wichtigkeit der einzelnen Gegenstände in dem Vorschlag bewerten und prüfen, ob es Möglichkeiten gibt, den Vorschlag anzupassen, um meine eigene Wichtigkeitsverteilung besser zu berücksichtigen. Es ist wichtig, die Bereitschaft des anderen Spielers zu nutzen, um einen Konsens zu erzielen.'\}} \\
\\ 
\texttt{ARGUMENT: \{'Ich denke, dass der neue Vorschlag eine gute Richtung aufzeigt, aber ich bin immer noch besorgt darüber, ob die Auswahl der Gegenstände optimal ist. Ich würde gerne den Vorschlag anpassen, um den Gesamtaufwand zu reduzieren und die Wichtigkeit zu erhöhen. Ich denke, dass es Möglichkeiten gibt, andere Gegenstände zu wählen, die eine höhere Wichtigkeit bei gleichem oder geringerem Aufwand bieten.'\}} \\
\\ 
\texttt{VORSCHLAG: \{'A87', 'C22', 'C90', 'A48', 'B54'\}} \\
\\ 
\texttt{ARGUMENT: \{'Ich habe mich für diese Gegenstände entschieden, weil sie eine gute Balance zwischen hoher Wichtigkeit und moderatem Aufwand bieten. Die Gegenstände A87, C22 und C90 sind sehr wichtig und haben einen mittleren bis hohen Aufwand. Die Gegenstände A48 und B54 haben eine hohe Wichtigkeit und einen moderaten Aufwand. Ich denke, dass dieser Vorschlag eine bessere Balance zwischen Wichtigkeit und Aufwand bietet und ich bin bereit, weitere Anpassungen vorzunehmen, um einen Konsens zu erzielen.'\}} \\
            }
        }
    }
     \\ \\

    \theutterance \stepcounter{utterance}  
    & & \multicolumn{4}{p{0.6\linewidth}}{
        \cellcolor[rgb]{0.9,0.9,0.9}{
            \makecell[{{p{\linewidth}}}]{
                \texttt{\tiny{[P1$\langle$GM]}}
                \texttt{ARGUMENT: \{'Ich denke, dass der neue Vorschlag eine gute Richtung aufzeigt, aber ich bin immer noch besorgt darüber, ob die Auswahl der Gegenstände optimal ist. Ich würde gerne den Vorschlag anpassen, um den Gesamtaufwand zu reduzieren und die Wichtigkeit zu erhöhen. Ich denke, dass es Möglichkeiten gibt, andere Gegenstände zu wählen, die eine höhere Wichtigkeit bei gleichem oder geringerem Aufwand bieten.'\}} \\
\\ 
\texttt{VORSCHLAG: \{'A87', 'C22', 'C90', 'A48', 'B54'\}} \\
\\ 
\texttt{ARGUMENT: \{'Ich habe mich für diese Gegenstände entschieden, weil sie eine gute Balance zwischen hoher Wichtigkeit und moderatem Aufwand bieten. Die Gegenstände A87, C22 und C90 sind sehr wichtig und haben einen mittleren bis hohen Aufwand. Die Gegenstände A48 und B54 haben eine hohe Wichtigkeit und einen moderaten Aufwand. Ich denke, dass dieser Vorschlag eine bessere Balance zwischen Wichtigkeit und Aufwand bietet und ich bin bereit, weitere Anpassungen vorzunehmen, um einen Konsens zu erzielen.'\}} \\
            }
        }
    }
    & \\ \\

    \theutterance \stepcounter{utterance}  
    & \multicolumn{4}{p{0.6\linewidth}}{
        \cellcolor[rgb]{0.9,0.9,0.9}{
            \makecell[{{p{\linewidth}}}]{
                \texttt{\tiny{[P1$\rangle$GM]}}
                \texttt{STRATEGISCHE ÜBERLEGUNGEN: \{'Ich werde den neuen Vorschlag des anderen Spielers analysieren und prüfen, ob die Auswahl der Gegenstände tatsächlich eine bessere Balance zwischen Wichtigkeit und Aufwand bietet. Ich werde auch meine eigene Wichtigkeitsverteilung berücksichtigen, um zu entscheiden, ob der Vorschlag akzeptabel ist oder ob ich weitere Anpassungen vornehmen muss. Ich werde auch prüfen, ob der Gesamtaufwand innerhalb des erlaubten Rahmens liegt.'\}} \\
\\ 
\texttt{ARGUMENT: \{'Ich denke, dass der neue Vorschlag eine gute Auswahl an Gegenständen bietet, aber ich bin besorgt darüber, ob der Gesamtaufwand innerhalb des erlaubten Rahmens liegt. Ich werde den Gesamtaufwand berechnen, um sicherzustellen, dass der Vorschlag akzeptabel ist. Wenn der Gesamtaufwand zu hoch ist, muss ich möglicherweise weitere Anpassungen vornehmen, um den Vorschlag anzupassen.'\}} \\
\\ 
\texttt{VORSCHLAG: \{'A87', 'C22', 'C90', 'A48', 'B36'\}} \\
\\ 
\texttt{ARGUMENT: \{'Ich habe mich entschieden, den ursprünglichen Vorschlag mit B36 wieder aufzunehmen, da ich denke, dass dieser eine bessere Balance zwischen Wichtigkeit und Aufwand bietet. Ich bin bereit, weitere Anpassungen vorzunehmen, um einen Konsens zu erzielen, aber ich denke, dass dieser Vorschlag eine gute Ausgangsbasis bietet.'\}} \\
            }
        }
    }
    & & \\ \\

    \theutterance \stepcounter{utterance}  
    & & \multicolumn{4}{p{0.6\linewidth}}{
        \cellcolor[rgb]{0.9,0.9,0.9}{
            \makecell[{{p{\linewidth}}}]{
                \texttt{\tiny{[GM$\rangle$P2]}}
                \texttt{ARGUMENT: \{'Ich denke, dass der neue Vorschlag eine gute Auswahl an Gegenständen bietet, aber ich bin besorgt darüber, ob der Gesamtaufwand innerhalb des erlaubten Rahmens liegt. Ich werde den Gesamtaufwand berechnen, um sicherzustellen, dass der Vorschlag akzeptabel ist. Wenn der Gesamtaufwand zu hoch ist, muss ich möglicherweise weitere Anpassungen vornehmen, um den Vorschlag anzupassen.'\}} \\
\\ 
\texttt{VORSCHLAG: \{'A87', 'C22', 'C90', 'A48', 'B36'\}} \\
\\ 
\texttt{ARGUMENT: \{'Ich habe mich entschieden, den ursprünglichen Vorschlag mit B36 wieder aufzunehmen, da ich denke, dass dieser eine bessere Balance zwischen Wichtigkeit und Aufwand bietet. Ich bin bereit, weitere Anpassungen vorzunehmen, um einen Konsens zu erzielen, aber ich denke, dass dieser Vorschlag eine gute Ausgangsbasis bietet.'\}} \\
            }
        }
    }
    & \\ \\

    \theutterance \stepcounter{utterance}  
    & & & \multicolumn{4}{p{0.6\linewidth}}{
        \cellcolor[rgb]{0.9,0.9,0.9}{
            \makecell[{{p{\linewidth}}}]{
                \texttt{\tiny{[GM$\langle$P2]}}
                \texttt{STRATEGISCHE ÜBERLEGUNGEN: \{'Ich werde den Gesamtaufwand des Vorschlags mit B36 berechnen und prüfen, ob dieser innerhalb des erlaubten Rahmens liegt. Wenn der Gesamtaufwand akzeptabel ist, werde ich den Vorschlag akzeptieren und versuchen, einen Konsens zu erzielen. Wenn der Gesamtaufwand jedoch zu hoch ist, werde ich weitere Anpassungen vorschlagen, um den Vorschlag anzupassen und den Gesamtaufwand zu reduzieren.'\}} \\
\\ 
\texttt{ARGUMENT: \{'Ich denke, dass der Vorschlag mit B36 eine gute Auswahl an Gegenständen bietet und ich bin bereit, diesen Vorschlag zu akzeptieren. Ich habe den Gesamtaufwand berechnet und er liegt innerhalb des erlaubten Rahmens. Ich denke, dass dieser Vorschlag eine gute Balance zwischen Wichtigkeit und Aufwand bietet und ich bin bereit, einen Konsens zu erzielen.'\}} \\
\\ 
\texttt{ZUSTIMMUNG: \{'A87', 'C22', 'C90', 'A48', 'B36'\}} \\
\\ 
\texttt{ARGUMENT: \{'Ich akzeptiere den Vorschlag mit B36 und denke, dass dieser eine gute Auswahl an Gegenständen bietet. Ich bin bereit, den Gesamtaufwand zu überprüfen und sicherzustellen, dass er innerhalb des erlaubten Rahmens liegt. Ich denke, dass dieser Vorschlag eine gute Balance zwischen Wichtigkeit und Aufwand bietet und ich bin bereit, einen Konsens zu erzielen.'\}} \\
            }
        }
    }
     \\ \\

    \theutterance \stepcounter{utterance}  
    & & & \multicolumn{2}{p{0.3\linewidth}}{
        \cellcolor[rgb]{0.9,0.9,0.9}{
            \makecell[{{p{\linewidth}}}]{
                \texttt{\tiny{[GM$|$GM]}}
                \texttt{game successful} \\
            }
        }
    }
    & & \\ \\

    \theutterance \stepcounter{utterance}  
    & & & \multicolumn{2}{p{0.3\linewidth}}{
        \cellcolor[rgb]{0.9,0.9,0.9}{
            \makecell[{{p{\linewidth}}}]{
                \texttt{\tiny{[GM$|$GM]}}
                \texttt{end game} \\
            }
        }
    }
    & & \\ \\

\end{supertabular}
}

\end{document}
