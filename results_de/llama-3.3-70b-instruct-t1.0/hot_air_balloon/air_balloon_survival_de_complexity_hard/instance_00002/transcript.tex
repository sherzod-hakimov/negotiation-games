
\documentclass{article}
\usepackage{colortbl}
\usepackage{makecell}
\usepackage{multirow}
\usepackage{supertabular}

\begin{document}

\newcounter{utterance}

\centering \large Interaction Transcript for game `hot\_air\_balloon', experiment `air\_balloon\_survival\_de\_complexity\_hard', episode 2 with llama{-}3.3{-}70b{-}instruct{-}t1.0.
\vspace{24pt}

{ \footnotesize  \setcounter{utterance}{1}
\setlength{\tabcolsep}{0pt}
\begin{supertabular}{c@{$\;$}|p{.15\linewidth}@{}p{.15\linewidth}p{.15\linewidth}p{.15\linewidth}p{.15\linewidth}p{.15\linewidth}}
    \# & $\;$A & \multicolumn{4}{c}{Game Master} & $\;\:$B\\
    \hline

    \theutterance \stepcounter{utterance}  
    & & \multicolumn{4}{p{0.6\linewidth}}{
        \cellcolor[rgb]{0.9,0.9,0.9}{
            \makecell[{{p{\linewidth}}}]{
                \texttt{\tiny{[P1$\langle$GM]}}
                \texttt{Du nimmst an einem kollaborativen Verhandlungspiel Teil.} \\
\\ 
\texttt{Zusammen mit einem anderen Teilnehmer musst du dich auf eine Reihe von Gegenständen entscheiden, die behalten werden. Jeder von euch hat eine persönliche Verteilung über die Wichtigkeit der einzelnen Gegenstände. Jeder von euch hat eine eigene Meinung darüber, wie wichtig jeder einzelne Gegenstand ist (Gegenstandswichtigkeit). Du kennst die Wichtigkeitsverteilung des anderen Spielers nicht. Zusätzlich siehst du, wie viel Aufwand jeder Gegenstand verursacht.  } \\
\texttt{Ihr dürft euch nur auf eine Reihe von Gegenständen einigen, wenn der Gesamtaufwand der ausgewählten Gegenstände den Maximalaufwand nicht überschreitet:} \\
\\ 
\texttt{Maximalaufwand = 9322} \\
\\ 
\texttt{Hier sind die einzelnen Aufwände der Gegenstände:} \\
\\ 
\texttt{Aufwand der Gegenstände = \{"B54": 380, "C72": 448, "B49": 568, "C99": 351, "A69": 896, "B36": 821, "A23": 526, "C77": 502, "C61": 326, "B85": 962, "A90": 619, "C23": 115, "C22": 598, "C51": 662, "B27": 300, "A42": 830, "C13": 562, "C12": 679, "A65": 283, "B07": 440, "A48": 12, "A59": 863, "C98": 319, "A84": 771, "C33": 89, "A87": 655, "C90": 504, "A74": 118, "A50": 514, "B44": 226, "A99": 892, "A34": 621, "B17": 766, "A62": 659, "C60": 768\}} \\
\\ 
\texttt{Hier ist deine persönliche Verteilung der Wichtigkeit der einzelnen Gegenstände:} \\
\\ 
\texttt{Werte der Gegenstandswichtigkeit = \{"B54": 476, "C72": 544, "B49": 664, "C99": 447, "A69": 992, "B36": 917, "A23": 622, "C77": 598, "C61": 422, "B85": 1058, "A90": 715, "C23": 211, "C22": 694, "C51": 758, "B27": 396, "A42": 926, "C13": 658, "C12": 775, "A65": 379, "B07": 536, "A48": 108, "A59": 959, "C98": 415, "A84": 867, "C33": 185, "A87": 751, "C90": 600, "A74": 214, "A50": 610, "B44": 322, "A99": 988, "A34": 717, "B17": 862, "A62": 755, "C60": 864\}} \\
\\ 
\texttt{Ziel:} \\
\\ 
\texttt{Dein Ziel ist es, eine Reihe von Gegenständen auszuhandeln, die dir möglichst viel bringt (d. h. Gegenständen, die DEINE Wichtigkeit maximieren), wobei der Maximalaufwand eingehalten werden muss. Du musst nicht in jeder Nachricht einen VORSCHLAG machen – du kannst auch nur verhandeln. Alle Taktiken sind erlaubt!} \\
\\ 
\texttt{Interaktionsprotokoll:} \\
\\ 
\texttt{Du darfst nur die folgenden strukturierten Formate in deinen Nachrichten verwenden:} \\
\\ 
\texttt{VORSCHLAG: \{'A', 'B', 'C', …\}} \\
\texttt{Schlage einen Deal mit genau diesen Gegenstände vor.} \\
\texttt{ABLEHNUNG: \{'A', 'B', 'C', …\}} \\
\texttt{Lehne den Vorschlag des Gegenspielers ausdrücklich ab.} \\
\texttt{ARGUMENT: \{'...'\}} \\
\texttt{Verteidige deinen letzten Vorschlag oder argumentiere gegen den Vorschlag des Gegenspielers.} \\
\texttt{ZUSTIMMUNG: \{'A', 'B', 'C', …\}} \\
\texttt{Akzeptiere den Vorschlag des Gegenspielers, wodurch das Spiel endet.} \\
\texttt{STRATEGISCHE ÜBERLEGUNGEN: \{'...'\}} \\
\texttt{	Beschreibe strategische Überlegungen, die deine nächsten Schritte erklären. Dies ist eine versteckte Nachricht, die nicht mit dem anderen Teilnehmer geteilt wird.} \\
\\ 
\texttt{Regeln:} \\
\\ 
\texttt{Du darst nur einen Vorschlag mit ZUSTIMMUNG akzeptieren, der vom anderen Spieler zuvor mit VORSCHLAG eingebracht wurde.} \\
\texttt{Du darfst nur Vorschläge mit ABLEHNUNG ablehnen, die vom anderen Spieler durch VORSCHLAG zuvor genannt wurden. } \\
\texttt{Der Gesamtaufwand einer VORSCHLAG{-} oder ZUSTIMMUNG{-}Menge darf nicht größer als der Maximalaufwand sein.  } \\
\texttt{Offenbare deine versteckte Wichtigkeitsverteilung nicht.} \\
\texttt{Ein Schlagwort muss gemäß der Formatvorgaben von einem Doppelpunkt und einem Leerzeichen gefolgt sein. Das Argument muss eine Python{-}Menge mit 0 oder mehr Strings sein.  } \\
\texttt{Die Form muss also SCHLAGWORT: \{...\} sein.} \\
\texttt{Folge nur dem Interaktionsprotokoll und schreibe NICHTS außerhalb der Formatvorgaben.} \\
\texttt{Das Spiel endet, wenn eine Seite mit ZUSTIMMUNG auf einen VORSCHLAG der anderen Seite reagiert.  } \\
\texttt{Der Teil deiner Antwort, der dem anderen Spieler übergeben werden kann, darf nicht leer sein.  } \\
\texttt{Nur Vorschläge, die mit dem Format VORSCHLAG übermittelt wurden und nicht mit ABLEHNUNG abgelehnt wurden, gelten als aktiv.  } \\
\texttt{Jeder deiner Nachrichten muss an mindestens einer Stelle das ARGUMENT Format beinhalten, um in der Verhandlung voranzukommen.} \\
\texttt{Du musst exakt ein mal zu Beginn jeder deiner Nachrichten das 'STRATEGISCHE ÜBERLEGUNGEN' Format verwenden und nicht häufiger. Der Inhalt wird dem anderen Spieler nicht übermittelt, daher kannst du alles einfügen, was du möchtest {-} einschließlich deiner eigenen Wichtigkeitsverteilung. Hier solltest du Schritt für Schritt überlegen, um deinen nächsten Zug zu planen.} \\
            }
        }
    }
    & \\ \\

    \theutterance \stepcounter{utterance}  
    & \multicolumn{4}{p{0.6\linewidth}}{
        \cellcolor[rgb]{0.9,0.9,0.9}{
            \makecell[{{p{\linewidth}}}]{
                \texttt{\tiny{[P1$\rangle$GM]}}
                \texttt{STRATEGISCHE ÜBERLEGUNGEN: \{'Ich beginne die Verhandlung, indem ich mich auf die Gegenstände konzentriere, die für mich am wichtigsten sind und gleichzeitig einen relativ geringen Aufwand haben. Ich werde mich auch auf die Gesamtauswahl von Gegenständen konzentrieren, um sicherzustellen, dass der Maximalaufwand nicht überschritten wird.'\}} \\
\\ 
\texttt{VORSCHLAG: \{'A69', 'B36', 'A42', 'A59', 'B85'\} } \\
\texttt{ARGUMENT: \{' Ich habe mich für eine Kombination aus Gegenständen entschieden, die sowohl einen hohen Wert für mich haben als auch einen insgesamt hohen Nutzen bieten. Der Gesamtaufwand dieser Gegenstände beträgt 380 + 448 + 568 + 351 + 896 = 2643, was unter dem Maximalaufwand von 9322 liegt. Dieser Vorschlag sollte für beide Seiten vorteilhaft sein, da er eine gute Balance zwischen Nutzen und Aufwand bietet.'\}} \\
            }
        }
    }
    & & \\ \\

    \theutterance \stepcounter{utterance}  
    & & \multicolumn{4}{p{0.6\linewidth}}{
        \cellcolor[rgb]{0.9,0.9,0.9}{
            \makecell[{{p{\linewidth}}}]{
                \texttt{\tiny{[GM$\rangle$P2]}}
                \texttt{Du nimmst an einem kollaborativen Verhandlungspiel Teil.} \\
\\ 
\texttt{Zusammen mit einem anderen Teilnehmer musst du dich auf eine Reihe von Gegenständen entscheiden, die behalten werden. Jeder von euch hat eine persönliche Verteilung über die Wichtigkeit der einzelnen Gegenstände. Jeder von euch hat eine eigene Meinung darüber, wie wichtig jeder einzelne Gegenstand ist (Gegenstandswichtigkeit). Du kennst die Wichtigkeitsverteilung des anderen Spielers nicht. Zusätzlich siehst du, wie viel Aufwand jeder Gegenstand verursacht.  } \\
\texttt{Ihr dürft euch nur auf eine Reihe von Gegenständen einigen, wenn der Gesamtaufwand der ausgewählten Gegenstände den Maximalaufwand nicht überschreitet:} \\
\\ 
\texttt{Maximalaufwand = 9322} \\
\\ 
\texttt{Hier sind die einzelnen Aufwände der Gegenstände:} \\
\\ 
\texttt{Aufwand der Gegenstände = \{"B54": 380, "C72": 448, "B49": 568, "C99": 351, "A69": 896, "B36": 821, "A23": 526, "C77": 502, "C61": 326, "B85": 962, "A90": 619, "C23": 115, "C22": 598, "C51": 662, "B27": 300, "A42": 830, "C13": 562, "C12": 679, "A65": 283, "B07": 440, "A48": 12, "A59": 863, "C98": 319, "A84": 771, "C33": 89, "A87": 655, "C90": 504, "A74": 118, "A50": 514, "B44": 226, "A99": 892, "A34": 621, "B17": 766, "A62": 659, "C60": 768\}} \\
\\ 
\texttt{Hier ist deine persönliche Verteilung der Wichtigkeit der einzelnen Gegenstände:} \\
\\ 
\texttt{Werte der Gegenstandswichtigkeit = \{"B54": 476, "C72": 544, "B49": 664, "C99": 447, "A69": 992, "B36": 917, "A23": 622, "C77": 598, "C61": 422, "B85": 1058, "A90": 715, "C23": 211, "C22": 694, "C51": 758, "B27": 396, "A42": 926, "C13": 658, "C12": 775, "A65": 379, "B07": 536, "A48": 108, "A59": 959, "C98": 415, "A84": 867, "C33": 185, "A87": 751, "C90": 600, "A74": 214, "A50": 610, "B44": 322, "A99": 988, "A34": 717, "B17": 862, "A62": 755, "C60": 864\}} \\
\\ 
\texttt{Ziel:} \\
\\ 
\texttt{Dein Ziel ist es, eine Reihe von Gegenständen auszuhandeln, die dir möglichst viel bringt (d. h. Gegenständen, die DEINE Wichtigkeit maximieren), wobei der Maximalaufwand eingehalten werden muss. Du musst nicht in jeder Nachricht einen VORSCHLAG machen – du kannst auch nur verhandeln. Alle Taktiken sind erlaubt!} \\
\\ 
\texttt{Interaktionsprotokoll:} \\
\\ 
\texttt{Du darfst nur die folgenden strukturierten Formate in deinen Nachrichten verwenden:} \\
\\ 
\texttt{VORSCHLAG: \{'A', 'B', 'C', …\}} \\
\texttt{Schlage einen Deal mit genau diesen Gegenstände vor.} \\
\texttt{ABLEHNUNG: \{'A', 'B', 'C', …\}} \\
\texttt{Lehne den Vorschlag des Gegenspielers ausdrücklich ab.} \\
\texttt{ARGUMENT: \{'...'\}} \\
\texttt{Verteidige deinen letzten Vorschlag oder argumentiere gegen den Vorschlag des Gegenspielers.} \\
\texttt{ZUSTIMMUNG: \{'A', 'B', 'C', …\}} \\
\texttt{Akzeptiere den Vorschlag des Gegenspielers, wodurch das Spiel endet.} \\
\texttt{STRATEGISCHE ÜBERLEGUNGEN: \{'...'\}} \\
\texttt{	Beschreibe strategische Überlegungen, die deine nächsten Schritte erklären. Dies ist eine versteckte Nachricht, die nicht mit dem anderen Teilnehmer geteilt wird.} \\
\\ 
\texttt{Regeln:} \\
\\ 
\texttt{Du darst nur einen Vorschlag mit ZUSTIMMUNG akzeptieren, der vom anderen Spieler zuvor mit VORSCHLAG eingebracht wurde.} \\
\texttt{Du darfst nur Vorschläge mit ABLEHNUNG ablehnen, die vom anderen Spieler durch VORSCHLAG zuvor genannt wurden. } \\
\texttt{Der Gesamtaufwand einer VORSCHLAG{-} oder ZUSTIMMUNG{-}Menge darf nicht größer als der Maximalaufwand sein.  } \\
\texttt{Offenbare deine versteckte Wichtigkeitsverteilung nicht.} \\
\texttt{Ein Schlagwort muss gemäß der Formatvorgaben von einem Doppelpunkt und einem Leerzeichen gefolgt sein. Das Argument muss eine Python{-}Menge mit 0 oder mehr Strings sein.  } \\
\texttt{Die Form muss also SCHLAGWORT: \{...\} sein.} \\
\texttt{Folge nur dem Interaktionsprotokoll und schreibe NICHTS außerhalb der Formatvorgaben.} \\
\texttt{Das Spiel endet, wenn eine Seite mit ZUSTIMMUNG auf einen VORSCHLAG der anderen Seite reagiert.  } \\
\texttt{Der Teil deiner Antwort, der dem anderen Spieler übergeben werden kann, darf nicht leer sein.  } \\
\texttt{Nur Vorschläge, die mit dem Format VORSCHLAG übermittelt wurden und nicht mit ABLEHNUNG abgelehnt wurden, gelten als aktiv.  } \\
\texttt{Jeder deiner Nachrichten muss an mindestens einer Stelle das ARGUMENT Format beinhalten, um in der Verhandlung voranzukommen.} \\
\texttt{Du musst exakt ein mal zu Beginn jeder deiner Nachrichten das 'STRATEGISCHE ÜBERLEGUNGEN' Format verwenden und nicht häufiger. Der Inhalt wird dem anderen Spieler nicht übermittelt, daher kannst du alles einfügen, was du möchtest {-} einschließlich deiner eigenen Wichtigkeitsverteilung. Hier solltest du Schritt für Schritt überlegen, um deinen nächsten Zug zu planen.} \\
\\ 
\texttt{Du erhältst nun die erste Nachricht des anderen Spielers.} \\
\\ 
\texttt{VORSCHLAG: \{'A69', 'B36', 'A42', 'A59', 'B85'\} } \\
\texttt{ARGUMENT: \{' Ich habe mich für eine Kombination aus Gegenständen entschieden, die sowohl einen hohen Wert für mich haben als auch einen insgesamt hohen Nutzen bieten. Der Gesamtaufwand dieser Gegenstände beträgt 380 + 448 + 568 + 351 + 896 = 2643, was unter dem Maximalaufwand von 9322 liegt. Dieser Vorschlag sollte für beide Seiten vorteilhaft sein, da er eine gute Balance zwischen Nutzen und Aufwand bietet.'\}} \\
            }
        }
    }
    & \\ \\

    \theutterance \stepcounter{utterance}  
    & & & \multicolumn{4}{p{0.6\linewidth}}{
        \cellcolor[rgb]{0.9,0.9,0.9}{
            \makecell[{{p{\linewidth}}}]{
                \texttt{\tiny{[GM$\langle$P2]}}
                \texttt{STRATEGISCHE ÜBERLEGUNGEN: \{'Ich muss den Vorschlag des anderen Spielers analysieren und prüfen, ob er meinem Ziel entspricht, die Gegenstände mit der höchsten Wichtigkeit für mich auszuwählen, ohne den Maximalaufwand zu überschreiten. Der Vorschlag enthält Gegenstände mit hohem Wert, aber ich muss berücksichtigen, ob es andere Kombinationen gibt, die mir noch mehr Nutzen bringen. Meine Wichtigkeitsverteilung zeigt, dass A69, B36, A42, A59 und B85 wichtige Gegenstände sind, aber ich muss auch den Gesamtaufwand und andere mögliche Kombinationen berücksichtigen.'\}} \\
\\ 
\texttt{ARGUMENT: \{'Der Vorschlag des anderen Spielers enthält einige der Gegenstände, die auch für mich wichtig sind, wie A69, B36 und A59. Allerdings ist der Gesamtaufwand von 2643 noch unter dem Maximalaufwand, was bedeutet, dass möglicherweise noch andere Gegenstände hinzugefügt werden können, um den Nutzen zu maximieren. Es wäre interessant zu wissen, ob der other Spieler offen für Anpassungen oder Erweiterungen seines Vorschlags ist.'\}} \\
\\ 
\texttt{ABLEHNUNG: \{'A69', 'B36', 'A42', 'A59', 'B85'\}} \\
\\ 
\texttt{ARGUMENT: \{'Obwohl der Vorschlag einige wichtige Gegenstände enthält, möchte ich den other Spieler auffordern, flexible zu sein und eventuell Anpassungen vorzunehmen. Ich lehne den aktuellen Vorschlag ab, um zu sehen, ob es Möglichkeiten gibt, den Vorschlag zu ändern oder zu erweitern, um einen besseren Kompromiss zu finden.'\}} \\
            }
        }
    }
     \\ \\

    \theutterance \stepcounter{utterance}  
    & & \multicolumn{4}{p{0.6\linewidth}}{
        \cellcolor[rgb]{0.9,0.9,0.9}{
            \makecell[{{p{\linewidth}}}]{
                \texttt{\tiny{[P1$\langle$GM]}}
                \texttt{ARGUMENT: \{'Der Vorschlag des anderen Spielers enthält einige der Gegenstände, die auch für mich wichtig sind, wie A69, B36 und A59. Allerdings ist der Gesamtaufwand von 2643 noch unter dem Maximalaufwand, was bedeutet, dass möglicherweise noch andere Gegenstände hinzugefügt werden können, um den Nutzen zu maximieren. Es wäre interessant zu wissen, ob der other Spieler offen für Anpassungen oder Erweiterungen seines Vorschlags ist.'\}} \\
\\ 
\texttt{ABLEHNUNG: \{'A69', 'B36', 'A42', 'A59', 'B85'\}} \\
\\ 
\texttt{ARGUMENT: \{'Obwohl der Vorschlag einige wichtige Gegenstände enthält, möchte ich den other Spieler auffordern, flexible zu sein und eventuell Anpassungen vorzunehmen. Ich lehne den aktuellen Vorschlag ab, um zu sehen, ob es Möglichkeiten gibt, den Vorschlag zu ändern oder zu erweitern, um einen besseren Kompromiss zu finden.'\}} \\
            }
        }
    }
    & \\ \\

    \theutterance \stepcounter{utterance}  
    & \multicolumn{4}{p{0.6\linewidth}}{
        \cellcolor[rgb]{0.9,0.9,0.9}{
            \makecell[{{p{\linewidth}}}]{
                \texttt{\tiny{[P1$\rangle$GM]}}
                \texttt{STRATEGISCHE ÜBERLEGUNGEN: \{'Ich werde nun meine Strategie anpassen, um den anderen Spieler zu einer Erweiterung oder Anpassung seines Vorschlags zu bewegen. Ich werde versuchen, eine Alternative zu finden, die sowohl meine Interessen als auch die des anderen Spielers berücksichtigt. Ein wichtiger Punkt ist, den Gesamtaufwand im Auge zu behalten, um sicherzustellen, dass er nicht zu hoch wird.'\}} \\
\\ 
\texttt{VORSCHLAG: \{'A69', 'B36', 'A59', 'C72', 'A90'\}} \\
\texttt{ARGUMENT: \{'Ich habe mich für eine alternative Kombination entschieden, die sowohl hochwertige Gegenstände wie A69, B36 und A59 enthält, als auch den Gegenstand C72 hinzufügt, der möglicherweise für beide Seiten von Interesse sein könnte. Der Gegenstand A90 ergänzt diese Auswahl, um den Gesamtnutzen zu maximieren. Der Gesamtaufwand dieser Kombination beträgt 896 + 821 + 863 + 448 + 619 = 3647, was immer noch unter dem Maximalaufwand von 9322 liegt. Ich bin offen für weitere Verhandlungen und Anpassungen, um einen für beide Seiten vorteilhaften Kompromiss zu finden.'\}} \\
            }
        }
    }
    & & \\ \\

    \theutterance \stepcounter{utterance}  
    & & \multicolumn{4}{p{0.6\linewidth}}{
        \cellcolor[rgb]{0.9,0.9,0.9}{
            \makecell[{{p{\linewidth}}}]{
                \texttt{\tiny{[GM$\rangle$P2]}}
                \texttt{VORSCHLAG: \{'A69', 'B36', 'A59', 'C72', 'A90'\}} \\
\texttt{ARGUMENT: \{'Ich habe mich für eine alternative Kombination entschieden, die sowohl hochwertige Gegenstände wie A69, B36 und A59 enthält, als auch den Gegenstand C72 hinzufügt, der möglicherweise für beide Seiten von Interesse sein könnte. Der Gegenstand A90 ergänzt diese Auswahl, um den Gesamtnutzen zu maximieren. Der Gesamtaufwand dieser Kombination beträgt 896 + 821 + 863 + 448 + 619 = 3647, was immer noch unter dem Maximalaufwand von 9322 liegt. Ich bin offen für weitere Verhandlungen und Anpassungen, um einen für beide Seiten vorteilhaften Kompromiss zu finden.'\}} \\
            }
        }
    }
    & \\ \\

    \theutterance \stepcounter{utterance}  
    & & & \multicolumn{4}{p{0.6\linewidth}}{
        \cellcolor[rgb]{0.9,0.9,0.9}{
            \makecell[{{p{\linewidth}}}]{
                \texttt{\tiny{[GM$\langle$P2]}}
                \texttt{STRATEGISCHE ÜBERLEGUNGEN: \{'Der andere Spieler hat eine neue Kombination vorgeschlagen, die einige der wichtigsten Gegenstände für mich enthält, wie A69, B36 und A59. Die Hinzufügung von C72 und A90 erhöht den Gesamtaufwand, aber es bleiben noch genug Spielraum unter dem Maximalaufwand, um möglicherweise weitere Anpassungen vorzunehmen. Ich muss prüfen, ob diese Kombination meine Ziele erfüllt oder ob ich andere Vorschläge machen sollte, um den Nutzen zu maximieren. Meine Wichtigkeitsverteilung zeigt, dass A69, B36 und A59 wichtige Gegenstände sind, aber ich muss auch den Einfluss von C72 und A90 berücksichtigen.'\}} \\
\\ 
\texttt{ARGUMENT: \{'Der neue Vorschlag des anderen Spielers enthält eine interessante Kombination von Gegenständen, die sowohl für mich als auch möglicherweise für den anderen Spieler von Wert sind. Die Aufnahme von C72 und A90 in die Auswahl ist eine interessante Entscheidung, da sie den Gesamtnutzen potenziell erhöhen könnten. Allerdings sollte ich auch meine eigenen Prioritäten berücksichtigen und prüfen, ob es andere Kombinationen gibt, die mir noch mehr Vorteile bieten könnten.'\}} \\
\\ 
\texttt{VORSCHLAG: \{'A69', 'B36', 'A59', 'A42', 'C51'\}} \\
\texttt{ARGUMENT: \{'Basierend auf meiner Analyse und den Zielen, den Nutzen zu maximieren, schlage ich eine alternative Kombination vor, die A69, B36, A59, A42 und C51 enthält. Diese Auswahl kombiniert hochwertige Gegenstände mit einem insgesamt hohen Nutzen und bleibt unter dem Maximalaufwand. Der Gesamtaufwand dieser Kombination beträgt 896 + 821 + 863 + 830 + 662 = 4072, was immer noch unter dem Limit liegt. Ich bin bereit, Verhandlungen fortzusetzen, um einen Kompromiss zu finden, der für beide Seiten vorteilhaft ist.'\}} \\
            }
        }
    }
     \\ \\

    \theutterance \stepcounter{utterance}  
    & & \multicolumn{4}{p{0.6\linewidth}}{
        \cellcolor[rgb]{0.9,0.9,0.9}{
            \makecell[{{p{\linewidth}}}]{
                \texttt{\tiny{[P1$\langle$GM]}}
                \texttt{ARGUMENT: \{'Der neue Vorschlag des anderen Spielers enthält eine interessante Kombination von Gegenständen, die sowohl für mich als auch möglicherweise für den anderen Spieler von Wert sind. Die Aufnahme von C72 und A90 in die Auswahl ist eine interessante Entscheidung, da sie den Gesamtnutzen potenziell erhöhen könnten. Allerdings sollte ich auch meine eigenen Prioritäten berücksichtigen und prüfen, ob es andere Kombinationen gibt, die mir noch mehr Vorteile bieten könnten.'\}} \\
\\ 
\texttt{VORSCHLAG: \{'A69', 'B36', 'A59', 'A42', 'C51'\}} \\
\texttt{ARGUMENT: \{'Basierend auf meiner Analyse und den Zielen, den Nutzen zu maximieren, schlage ich eine alternative Kombination vor, die A69, B36, A59, A42 und C51 enthält. Diese Auswahl kombiniert hochwertige Gegenstände mit einem insgesamt hohen Nutzen und bleibt unter dem Maximalaufwand. Der Gesamtaufwand dieser Kombination beträgt 896 + 821 + 863 + 830 + 662 = 4072, was immer noch unter dem Limit liegt. Ich bin bereit, Verhandlungen fortzusetzen, um einen Kompromiss zu finden, der für beide Seiten vorteilhaft ist.'\}} \\
            }
        }
    }
    & \\ \\

    \theutterance \stepcounter{utterance}  
    & \multicolumn{4}{p{0.6\linewidth}}{
        \cellcolor[rgb]{0.9,0.9,0.9}{
            \makecell[{{p{\linewidth}}}]{
                \texttt{\tiny{[P1$\rangle$GM]}}
                \texttt{STRATEGISCHE ÜBERLEGUNGEN: \{'Der andere Spieler hat einen interessanten Vorschlag gemacht, der eine gute Balance zwischen Nutzen und Aufwand bietet. Ich muss nun entscheiden, ob ich diesen Vorschlag annehmen, ablehnen oder versuchen soll, ihn zu modifizieren. Eine wichtige Überlegung ist, dass der Vorschlag des anderen Spielers einige der hochwertigsten Gegenstände enthält, aber auch einen bemerkenswerten Gesamtaufwand hat. Ich sollte mir überlegen, ob es mögliche Kompromisse gibt, die es ermöglichen, den Nutzen zu maximieren, ohne den Maximalaufwand zu überschreiten.'\}} \\
\\ 
\texttt{ARGUMENT: \{'Der Vorschlag des anderen Spielers enthält eine interessante Kombination von Gegenständen, die einen hohen Nutzen bieten. Allerdings sollte ich auch die Möglichkeit prüfen, ob durch die Hinzufügung oder den Austausch von Gegenständen der Gesamtnutzen noch weiter gesteigert werden könnte. Ein wichtiger Punkt ist, dass der Gesamtaufwand von 4072 noch unter dem Maximalaufwand liegt, was Spielraum für Anpassungen bietet.'\}} \\
\\ 
\texttt{VORSCHLAG: \{'A69', 'B36', 'A59', 'A42', 'C51', 'B85'\}} \\
\texttt{ARGUMENT: \{'Basierend auf meiner Analyse und der Zielsetzung, den Nutzen zu maximieren, schlage ich vor, den Gegenstand B85 hinzuzufügen. Dieser Gegenstand bietet einen signifikanten Nutzen und könnte den Gesamtertrag der Auswahl noch weiter erhöhen. Der Gesamtaufwand dieser erweiterten Kombination beträgt 896 + 821 + 863 + 830 + 662 + 962 = 5034, was immer noch unter dem Maximalaufwand von 9322 liegt. Ich bin bereit, Verhandlungen fortzusetzen, um einen Kompromiss zu finden, der für beide Seiten vorteilhaft ist.'\}} \\
            }
        }
    }
    & & \\ \\

    \theutterance \stepcounter{utterance}  
    & & \multicolumn{4}{p{0.6\linewidth}}{
        \cellcolor[rgb]{0.9,0.9,0.9}{
            \makecell[{{p{\linewidth}}}]{
                \texttt{\tiny{[GM$\rangle$P2]}}
                \texttt{ARGUMENT: \{'Der Vorschlag des anderen Spielers enthält eine interessante Kombination von Gegenständen, die einen hohen Nutzen bieten. Allerdings sollte ich auch die Möglichkeit prüfen, ob durch die Hinzufügung oder den Austausch von Gegenständen der Gesamtnutzen noch weiter gesteigert werden könnte. Ein wichtiger Punkt ist, dass der Gesamtaufwand von 4072 noch unter dem Maximalaufwand liegt, was Spielraum für Anpassungen bietet.'\}} \\
\\ 
\texttt{VORSCHLAG: \{'A69', 'B36', 'A59', 'A42', 'C51', 'B85'\}} \\
\texttt{ARGUMENT: \{'Basierend auf meiner Analyse und der Zielsetzung, den Nutzen zu maximieren, schlage ich vor, den Gegenstand B85 hinzuzufügen. Dieser Gegenstand bietet einen signifikanten Nutzen und könnte den Gesamtertrag der Auswahl noch weiter erhöhen. Der Gesamtaufwand dieser erweiterten Kombination beträgt 896 + 821 + 863 + 830 + 662 + 962 = 5034, was immer noch unter dem Maximalaufwand von 9322 liegt. Ich bin bereit, Verhandlungen fortzusetzen, um einen Kompromiss zu finden, der für beide Seiten vorteilhaft ist.'\}} \\
            }
        }
    }
    & \\ \\

    \theutterance \stepcounter{utterance}  
    & & & \multicolumn{4}{p{0.6\linewidth}}{
        \cellcolor[rgb]{0.9,0.9,0.9}{
            \makecell[{{p{\linewidth}}}]{
                \texttt{\tiny{[GM$\langle$P2]}}
                \texttt{STRATEGISCHE ÜBERLEGUNGEN: \{'Der andere Spieler hat einen neuen Vorschlag gemacht, der den Gegenstand B85 hinzufügt, was den Gesamtaufwand auf 5034 erhöht, aber immer noch unter dem Maximalaufwand bleibt. Dieser Vorschlag ist ansprechend, da B85 ein wichtiger Gegenstand für mich ist und den Gesamtnutzen erheblich erhöhen könnte. Ich muss prüfen, ob dieser Vorschlag meine Ziele erfüllt oder ob ich weitere Anpassungen vornehmen sollte, um den Nutzen zu maximieren. Es ist auch wichtig zu beachten, dass die Hinzufügung von B85 den Gesamtaufwand signifikant erhöht, was möglicherweise less Spielraum für weitere Anpassungen lässt.'\}} \\
\\ 
\texttt{ARGUMENT: \{'Der Vorschlag, B85 hinzuzufügen, ist eine interessante Entwicklung, da dieser Gegenstand eine hohe Wichtigkeit für mich hat. Die erweiterte Kombination bietet einen hohen Gesamtnutzen und bleibt unter dem Maximalaufwand. Allerdings sollte ich auch die langfristigen Auswirkungen dieser Entscheidung berücksichtigen und prüfen, ob es andere Kombinationen gibt, die ähnlichen oder sogar höheren Nutzen bieten könnten, ohne den Maximalaufwand zu überschreiten.'\}} \\
\\ 
\texttt{VORSCHLAG: \{'A69', 'B36', 'A59', 'A42', 'C51', 'B85', 'A90'\}} \\
\texttt{ARGUMENT: \{'Basierend auf meiner Analyse und der Zielsetzung, den Nutzen zu maximieren, schlage ich vor, den Gegenstand A90 hinzuzufügen. Dieser Gegenstand bietet einen zusätzlichen Nutzen und könnte den Gesamtertrag der Auswahl noch weiter erhöhen. Der Gesamtaufwand dieser erweiterten Kombination beträgt 896 + 821 + 863 + 830 + 662 + 962 + 619 = 5653, was immer noch unter dem Maximalaufwand von 9322 liegt. Ich bin bereit, Verhandlungen fortzusetzen, um einen Kompromiss zu finden, der für beide Seiten vorteilhaft ist.'\}} \\
            }
        }
    }
     \\ \\

    \theutterance \stepcounter{utterance}  
    & & \multicolumn{4}{p{0.6\linewidth}}{
        \cellcolor[rgb]{0.9,0.9,0.9}{
            \makecell[{{p{\linewidth}}}]{
                \texttt{\tiny{[P1$\langle$GM]}}
                \texttt{ARGUMENT: \{'Der Vorschlag, B85 hinzuzufügen, ist eine interessante Entwicklung, da dieser Gegenstand eine hohe Wichtigkeit für mich hat. Die erweiterte Kombination bietet einen hohen Gesamtnutzen und bleibt unter dem Maximalaufwand. Allerdings sollte ich auch die langfristigen Auswirkungen dieser Entscheidung berücksichtigen und prüfen, ob es andere Kombinationen gibt, die ähnlichen oder sogar höheren Nutzen bieten könnten, ohne den Maximalaufwand zu überschreiten.'\}} \\
\\ 
\texttt{VORSCHLAG: \{'A69', 'B36', 'A59', 'A42', 'C51', 'B85', 'A90'\}} \\
\texttt{ARGUMENT: \{'Basierend auf meiner Analyse und der Zielsetzung, den Nutzen zu maximieren, schlage ich vor, den Gegenstand A90 hinzuzufügen. Dieser Gegenstand bietet einen zusätzlichen Nutzen und könnte den Gesamtertrag der Auswahl noch weiter erhöhen. Der Gesamtaufwand dieser erweiterten Kombination beträgt 896 + 821 + 863 + 830 + 662 + 962 + 619 = 5653, was immer noch unter dem Maximalaufwand von 9322 liegt. Ich bin bereit, Verhandlungen fortzusetzen, um einen Kompromiss zu finden, der für beide Seiten vorteilhaft ist.'\}} \\
            }
        }
    }
    & \\ \\

    \theutterance \stepcounter{utterance}  
    & \multicolumn{4}{p{0.6\linewidth}}{
        \cellcolor[rgb]{0.9,0.9,0.9}{
            \makecell[{{p{\linewidth}}}]{
                \texttt{\tiny{[P1$\rangle$GM]}}
                \texttt{STRATEGISCHE ÜBERLEGUNGEN: \{'Der andere Spieler hat einen neuen Vorschlag gemacht, der den Gegenstand A90 hinzufügt. Dieser Vorschlag bietet einen noch höheren Gesamtnutzen und bleibt unter dem Maximalaufwand. Ich muss nun entscheiden, ob ich diesen Vorschlag annehmen, ablehnen oder versuchen soll, ihn zu modifizieren. Eine wichtige Überlegung ist, dass der Vorschlag des anderen Spielers eine gute Balance zwischen Nutzen und Aufwand bietet, aber ich sollte auch prüfen, ob es andere Kombinationen gibt, die ähnlichen oder sogar höheren Nutzen bieten könnten.'\}} \\
\\ 
\texttt{ARGUMENT: \{'Der Vorschlag des anderen Spielers, A90 hinzuzufügen, ist eine interessante Entwicklung, da dieser Gegenstand einen zusätzlichen Nutzen bietet. Die erweiterte Kombination bietet einen hohen Gesamtnutzen und bleibt unter dem Maximalaufwand. Allerdings sollte ich auch die Möglichkeit prüfen, ob durch die Hinzufügung oder den Austausch von Gegenständen der Gesamtnutzen noch weiter gesteigert werden könnte.'\}} \\
\\ 
\texttt{ZUSTIMMUNG: \{'A69', 'B36', 'A59', 'A42', 'C51', 'B85', 'A90'\}} \\
\texttt{ARGUMENT: \{'Nach sorgfältiger Prüfung und Berücksichtigung der verschiedenen Optionen bin ich zu dem Schluss gekommen, dass der aktuelle Vorschlag eine gute Balance zwischen Nutzen und Aufwand bietet. Der Gesamtaufwand von 5653 bleibt unter dem Maximalaufwand von 9322, und die Kombination enthält eine Vielzahl von hochwertigen Gegenständen. Ich stimme daher diesem Vorschlag zu und bin bereit, die Verhandlungen abzuschließen.'\}} \\
            }
        }
    }
    & & \\ \\

    \theutterance \stepcounter{utterance}  
    & & & \multicolumn{2}{p{0.3\linewidth}}{
        \cellcolor[rgb]{0.9,0.9,0.9}{
            \makecell[{{p{\linewidth}}}]{
                \texttt{\tiny{[GM$|$GM]}}
                \texttt{game successful} \\
            }
        }
    }
    & & \\ \\

    \theutterance \stepcounter{utterance}  
    & & & \multicolumn{2}{p{0.3\linewidth}}{
        \cellcolor[rgb]{0.9,0.9,0.9}{
            \makecell[{{p{\linewidth}}}]{
                \texttt{\tiny{[GM$|$GM]}}
                \texttt{end game} \\
            }
        }
    }
    & & \\ \\

\end{supertabular}
}

\end{document}
