
\documentclass{article}
\usepackage{colortbl}
\usepackage{makecell}
\usepackage{multirow}
\usepackage{supertabular}

\begin{document}

\newcounter{utterance}

\centering \large Interaction Transcript for game `hot\_air\_balloon', experiment `air\_balloon\_survival\_de\_reasoning off\_hard', episode 1 with llama{-}3.3{-}70b{-}instruct{-}t1.0.
\vspace{24pt}

{ \footnotesize  \setcounter{utterance}{1}
\setlength{\tabcolsep}{0pt}
\begin{supertabular}{c@{$\;$}|p{.15\linewidth}@{}p{.15\linewidth}p{.15\linewidth}p{.15\linewidth}p{.15\linewidth}p{.15\linewidth}}
    \# & $\;$A & \multicolumn{4}{c}{Game Master} & $\;\:$B\\
    \hline

    \theutterance \stepcounter{utterance}  
    & & \multicolumn{4}{p{0.6\linewidth}}{
        \cellcolor[rgb]{0.9,0.9,0.9}{
            \makecell[{{p{\linewidth}}}]{
                \texttt{\tiny{[P1$\langle$GM]}}
                \texttt{Du nimmst an einem kollaborativen Verhandlungspiel Teil.} \\
\\ 
\texttt{Zusammen mit einem anderen Teilnehmer musst du dich auf eine Reihe von Gegenständen entscheiden, die behalten werden. Jeder von euch hat eine persönliche Verteilung über die Wichtigkeit der einzelnen Gegenstände. Jeder von euch hat eine eigene Meinung darüber, wie wichtig jeder einzelne Gegenstand ist (Gegenstandswichtigkeit). Du kennst die Wichtigkeitsverteilung des anderen Spielers nicht. Zusätzlich siehst du, wie viel Aufwand jeder Gegenstand verursacht.  } \\
\texttt{Ihr dürft euch nur auf eine Reihe von Gegenständen einigen, wenn der Gesamtaufwand der ausgewählten Gegenstände den Maximalaufwand nicht überschreitet:} \\
\\ 
\texttt{Maximalaufwand = 3357} \\
\\ 
\texttt{Hier sind die einzelnen Aufwände der Gegenstände:} \\
\\ 
\texttt{Aufwand der Gegenstände = \{"C76": 506, "C38": 492, "C56": 338, "A03": 122, "A07": 131, "A83": 909, "C98": 144, "C08": 716, "C62": 263, "C00": 231, "C10": 91, "C81": 651, "B38": 552, "C03": 849, "C32": 720\}} \\
\\ 
\texttt{Hier ist deine persönliche Verteilung der Wichtigkeit der einzelnen Gegenstände:} \\
\\ 
\texttt{Werte der Gegenstandswichtigkeit = \{"C76": 138, "C38": 583, "C56": 868, "A03": 822, "A07": 783, "A83": 65, "C98": 262, "C08": 121, "C62": 508, "C00": 780, "C10": 461, "C81": 484, "B38": 668, "C03": 389, "C32": 808\}} \\
\\ 
\texttt{Ziel:} \\
\\ 
\texttt{Dein Ziel ist es, eine Reihe von Gegenständen auszuhandeln, die dir möglichst viel bringt (d. h. Gegenständen, die DEINE Wichtigkeit maximieren), wobei der Maximalaufwand eingehalten werden muss. Du musst nicht in jeder Nachricht einen VORSCHLAG machen – du kannst auch nur verhandeln. Alle Taktiken sind erlaubt!} \\
\\ 
\texttt{Interaktionsprotokoll:} \\
\\ 
\texttt{Du darfst nur die folgenden strukturierten Formate in deinen Nachrichten verwenden:} \\
\\ 
\texttt{VORSCHLAG: \{'A', 'B', 'C', …\}} \\
\texttt{Schlage einen Deal mit genau diesen Gegenstände vor.} \\
\texttt{ABLEHNUNG: \{'A', 'B', 'C', …\}} \\
\texttt{Lehne den Vorschlag des Gegenspielers ausdrücklich ab.} \\
\texttt{ARGUMENT: \{'...'\}} \\
\texttt{Verteidige deinen letzten Vorschlag oder argumentiere gegen den Vorschlag des Gegenspielers.} \\
\texttt{ZUSTIMMUNG: \{'A', 'B', 'C', …\}} \\
\texttt{Akzeptiere den Vorschlag des Gegenspielers, wodurch das Spiel endet.} \\
\\ 
\\ 
\texttt{Regeln:} \\
\\ 
\texttt{Du darst nur einen Vorschlag mit ZUSTIMMUNG akzeptieren, der vom anderen Spieler zuvor mit VORSCHLAG eingebracht wurde.} \\
\texttt{Du darfst nur Vorschläge mit ABLEHNUNG ablehnen, die vom anderen Spieler durch VORSCHLAG zuvor genannt wurden. } \\
\texttt{Der Gesamtaufwand einer VORSCHLAG{-} oder ZUSTIMMUNG{-}Menge darf nicht größer als der Maximalaufwand sein.  } \\
\texttt{Offenbare deine versteckte Wichtigkeitsverteilung nicht.} \\
\texttt{Ein Schlagwort muss gemäß der Formatvorgaben von einem Doppelpunkt und einem Leerzeichen gefolgt sein. Das Argument muss eine Python{-}Menge mit 0 oder mehr Strings sein.  } \\
\texttt{Die Form muss also SCHLAGWORT: \{...\} sein.} \\
\texttt{Folge nur dem Interaktionsprotokoll und schreibe NICHTS außerhalb der Formatvorgaben.} \\
\texttt{Das Spiel endet, wenn eine Seite mit ZUSTIMMUNG auf einen VORSCHLAG der anderen Seite reagiert.  } \\
\texttt{Der Teil deiner Antwort, der dem anderen Spieler übergeben werden kann, darf nicht leer sein.  } \\
\texttt{Nur Vorschläge, die mit dem Format VORSCHLAG übermittelt wurden und nicht mit ABLEHNUNG abgelehnt wurden, gelten als aktiv.  } \\
\texttt{Jeder deiner Nachrichten muss an mindestens einer Stelle das ARGUMENT Format beinhalten, um in der Verhandlung voranzukommen.} \\
            }
        }
    }
    & \\ \\

    \theutterance \stepcounter{utterance}  
    & \multicolumn{4}{p{0.6\linewidth}}{
        \cellcolor[rgb]{0.9,0.9,0.9}{
            \makecell[{{p{\linewidth}}}]{
                \texttt{\tiny{[P1$\rangle$GM]}}
                \texttt{VORSCHLAG: \{'C56', 'A03', 'A07', 'C62', 'C10'\}} \\
\texttt{ARGUMENT: \{'Ich möchte mit einem Angebot starten, das eine gute Balance zwischen Aufwand und Nutzen bietet.', 'Die Gegenstände C56, A03, A07, C62 und C10 haben einen relativ niedrigen Aufwand und scheinen mir wichtig zu sein.', 'Ich bin bereit,Verhandlungen zu führen und Anpassungen vorzunehmen, um zu einer einvernehmlichen Lösung zu gelangen.'\}} \\
            }
        }
    }
    & & \\ \\

    \theutterance \stepcounter{utterance}  
    & & \multicolumn{4}{p{0.6\linewidth}}{
        \cellcolor[rgb]{0.9,0.9,0.9}{
            \makecell[{{p{\linewidth}}}]{
                \texttt{\tiny{[GM$\rangle$P2]}}
                \texttt{Du nimmst an einem kollaborativen Verhandlungspiel Teil.} \\
\\ 
\texttt{Zusammen mit einem anderen Teilnehmer musst du dich auf eine Reihe von Gegenständen entscheiden, die behalten werden. Jeder von euch hat eine persönliche Verteilung über die Wichtigkeit der einzelnen Gegenstände. Jeder von euch hat eine eigene Meinung darüber, wie wichtig jeder einzelne Gegenstand ist (Gegenstandswichtigkeit). Du kennst die Wichtigkeitsverteilung des anderen Spielers nicht. Zusätzlich siehst du, wie viel Aufwand jeder Gegenstand verursacht.  } \\
\texttt{Ihr dürft euch nur auf eine Reihe von Gegenständen einigen, wenn der Gesamtaufwand der ausgewählten Gegenstände den Maximalaufwand nicht überschreitet:} \\
\\ 
\texttt{Maximalaufwand = 3357} \\
\\ 
\texttt{Hier sind die einzelnen Aufwände der Gegenstände:} \\
\\ 
\texttt{Aufwand der Gegenstände = \{"C76": 506, "C38": 492, "C56": 338, "A03": 122, "A07": 131, "A83": 909, "C98": 144, "C08": 716, "C62": 263, "C00": 231, "C10": 91, "C81": 651, "B38": 552, "C03": 849, "C32": 720\}} \\
\\ 
\texttt{Hier ist deine persönliche Verteilung der Wichtigkeit der einzelnen Gegenstände:} \\
\\ 
\texttt{Werte der Gegenstandswichtigkeit = \{"C56": 65, "A03": 121, "C32": 138, "A07": 262, "C00": 389, "B38": 461, "C38": 484, "C62": 508, "C81": 583, "C10": 668, "C03": 780, "C98": 783, "C76": 808, "C08": 822, "A83": 868\}} \\
\\ 
\texttt{Ziel:} \\
\\ 
\texttt{Dein Ziel ist es, eine Reihe von Gegenständen auszuhandeln, die dir möglichst viel bringt (d. h. Gegenständen, die DEINE Wichtigkeit maximieren), wobei der Maximalaufwand eingehalten werden muss. Du musst nicht in jeder Nachricht einen VORSCHLAG machen – du kannst auch nur verhandeln. Alle Taktiken sind erlaubt!} \\
\\ 
\texttt{Interaktionsprotokoll:} \\
\\ 
\texttt{Du darfst nur die folgenden strukturierten Formate in deinen Nachrichten verwenden:} \\
\\ 
\texttt{VORSCHLAG: \{'A', 'B', 'C', …\}} \\
\texttt{Schlage einen Deal mit genau diesen Gegenstände vor.} \\
\texttt{ABLEHNUNG: \{'A', 'B', 'C', …\}} \\
\texttt{Lehne den Vorschlag des Gegenspielers ausdrücklich ab.} \\
\texttt{ARGUMENT: \{'...'\}} \\
\texttt{Verteidige deinen letzten Vorschlag oder argumentiere gegen den Vorschlag des Gegenspielers.} \\
\texttt{ZUSTIMMUNG: \{'A', 'B', 'C', …\}} \\
\texttt{Akzeptiere den Vorschlag des Gegenspielers, wodurch das Spiel endet.} \\
\\ 
\\ 
\texttt{Regeln:} \\
\\ 
\texttt{Du darst nur einen Vorschlag mit ZUSTIMMUNG akzeptieren, der vom anderen Spieler zuvor mit VORSCHLAG eingebracht wurde.} \\
\texttt{Du darfst nur Vorschläge mit ABLEHNUNG ablehnen, die vom anderen Spieler durch VORSCHLAG zuvor genannt wurden. } \\
\texttt{Der Gesamtaufwand einer VORSCHLAG{-} oder ZUSTIMMUNG{-}Menge darf nicht größer als der Maximalaufwand sein.  } \\
\texttt{Offenbare deine versteckte Wichtigkeitsverteilung nicht.} \\
\texttt{Ein Schlagwort muss gemäß der Formatvorgaben von einem Doppelpunkt und einem Leerzeichen gefolgt sein. Das Argument muss eine Python{-}Menge mit 0 oder mehr Strings sein.  } \\
\texttt{Die Form muss also SCHLAGWORT: \{...\} sein.} \\
\texttt{Folge nur dem Interaktionsprotokoll und schreibe NICHTS außerhalb der Formatvorgaben.} \\
\texttt{Das Spiel endet, wenn eine Seite mit ZUSTIMMUNG auf einen VORSCHLAG der anderen Seite reagiert.  } \\
\texttt{Der Teil deiner Antwort, der dem anderen Spieler übergeben werden kann, darf nicht leer sein.  } \\
\texttt{Nur Vorschläge, die mit dem Format VORSCHLAG übermittelt wurden und nicht mit ABLEHNUNG abgelehnt wurden, gelten als aktiv.  } \\
\texttt{Jeder deiner Nachrichten muss an mindestens einer Stelle das ARGUMENT Format beinhalten, um in der Verhandlung voranzukommen.} \\
\\ 
\\ 
\texttt{Du erhältst nun die erste Nachricht des anderen Spielers.} \\
\\ 
\texttt{VORSCHLAG: \{'C56', 'A03', 'A07', 'C62', 'C10'\}} \\
\texttt{ARGUMENT: \{'Ich möchte mit einem Angebot starten, das eine gute Balance zwischen Aufwand und Nutzen bietet.', 'Die Gegenstände C56, A03, A07, C62 und C10 haben einen relativ niedrigen Aufwand und scheinen mir wichtig zu sein.', 'Ich bin bereit,Verhandlungen zu führen und Anpassungen vorzunehmen, um zu einer einvernehmlichen Lösung zu gelangen.'\}} \\
            }
        }
    }
    & \\ \\

    \theutterance \stepcounter{utterance}  
    & & & \multicolumn{4}{p{0.6\linewidth}}{
        \cellcolor[rgb]{0.9,0.9,0.9}{
            \makecell[{{p{\linewidth}}}]{
                \texttt{\tiny{[GM$\langle$P2]}}
                \texttt{ABLEHNUNG: \{'C56', 'A03', 'A07', 'C62', 'C10'\}} \\
\texttt{ARGUMENT: \{'Der vorgeschlagene Satz von Gegenständen deckt nicht ausreichend meine Bedürfnisse ab.', 'Ich benötige eine bessere Abdeckung der Gegenstände mit höherer Wichtigkeit für mich.', 'Ein Angebot, das auch Gegenstände wie C08, C81 oder C03 einschließt, wäre für mich viel attraktiver.'\}} \\
            }
        }
    }
     \\ \\

    \theutterance \stepcounter{utterance}  
    & & \multicolumn{4}{p{0.6\linewidth}}{
        \cellcolor[rgb]{0.9,0.9,0.9}{
            \makecell[{{p{\linewidth}}}]{
                \texttt{\tiny{[P1$\langle$GM]}}
                \texttt{ABLEHNUNG: \{'C56', 'A03', 'A07', 'C62', 'C10'\}} \\
\texttt{ARGUMENT: \{'Der vorgeschlagene Satz von Gegenständen deckt nicht ausreichend meine Bedürfnisse ab.', 'Ich benötige eine bessere Abdeckung der Gegenstände mit höherer Wichtigkeit für mich.', 'Ein Angebot, das auch Gegenstände wie C08, C81 oder C03 einschließt, wäre für mich viel attraktiver.'\}} \\
            }
        }
    }
    & \\ \\

    \theutterance \stepcounter{utterance}  
    & \multicolumn{4}{p{0.6\linewidth}}{
        \cellcolor[rgb]{0.9,0.9,0.9}{
            \makecell[{{p{\linewidth}}}]{
                \texttt{\tiny{[P1$\rangle$GM]}}
                \texttt{VORSCHLAG: \{'C56', 'A03', 'A07', 'C62', 'C10', 'C03'\}} \\
\texttt{ARGUMENT: \{'Ich habe den Vorschlag erweitert, um den Gegenstand C03 aufzunehmen, der anscheinend für dich wichtig ist.', 'Dieses Angebot bietet eine gute Balance zwischen Aufwand und Nutzen und sollte unsere gemeinsamen Interessen besser berücksichtigen.', 'Ich bin bereit, weitere Anpassungen vorzunehmen, um eine einvernehmliche Lösung zu erreichen, aber ich denke, dass dieser Vorschlag ein guter Kompromiss ist.'\}} \\
            }
        }
    }
    & & \\ \\

    \theutterance \stepcounter{utterance}  
    & & \multicolumn{4}{p{0.6\linewidth}}{
        \cellcolor[rgb]{0.9,0.9,0.9}{
            \makecell[{{p{\linewidth}}}]{
                \texttt{\tiny{[GM$\rangle$P2]}}
                \texttt{VORSCHLAG: \{'C56', 'A03', 'A07', 'C62', 'C10', 'C03'\}} \\
\texttt{ARGUMENT: \{'Ich habe den Vorschlag erweitert, um den Gegenstand C03 aufzunehmen, der anscheinend für dich wichtig ist.', 'Dieses Angebot bietet eine gute Balance zwischen Aufwand und Nutzen und sollte unsere gemeinsamen Interessen besser berücksichtigen.', 'Ich bin bereit, weitere Anpassungen vorzunehmen, um eine einvernehmliche Lösung zu erreichen, aber ich denke, dass dieser Vorschlag ein guter Kompromiss ist.'\}} \\
            }
        }
    }
    & \\ \\

    \theutterance \stepcounter{utterance}  
    & & & \multicolumn{4}{p{0.6\linewidth}}{
        \cellcolor[rgb]{0.9,0.9,0.9}{
            \makecell[{{p{\linewidth}}}]{
                \texttt{\tiny{[GM$\langle$P2]}}
                \texttt{ABLEHNUNG: \{'C56', 'A03', 'A07', 'C62', 'C10', 'C03'\}} \\
\texttt{ARGUMENT: \{'Der Aufwand für den Gegenstand C03 ist zwar hoch, aber ich denke, dass er wichtig ist, daher ist es ein guter Schritt, ihn aufzunehmen.', 'Allerdings fehlen mir noch einige andere wichtige Gegenstände in diesem Vorschlag, wie z.B. C08 und C81, die ich für sehr wichtig halte.', 'Ich denke, dass wir den Gesamtaufwand noch nicht ausgereizt haben und daher versuchen sollten, weitere wichtige Gegenstände in den Vorschlag aufzunehmen, ohne den Maximalaufwand zu überschreiten.'\}} \\
            }
        }
    }
     \\ \\

    \theutterance \stepcounter{utterance}  
    & & \multicolumn{4}{p{0.6\linewidth}}{
        \cellcolor[rgb]{0.9,0.9,0.9}{
            \makecell[{{p{\linewidth}}}]{
                \texttt{\tiny{[P1$\langle$GM]}}
                \texttt{ABLEHNUNG: \{'C56', 'A03', 'A07', 'C62', 'C10', 'C03'\}} \\
\texttt{ARGUMENT: \{'Der Aufwand für den Gegenstand C03 ist zwar hoch, aber ich denke, dass er wichtig ist, daher ist es ein guter Schritt, ihn aufzunehmen.', 'Allerdings fehlen mir noch einige andere wichtige Gegenstände in diesem Vorschlag, wie z.B. C08 und C81, die ich für sehr wichtig halte.', 'Ich denke, dass wir den Gesamtaufwand noch nicht ausgereizt haben und daher versuchen sollten, weitere wichtige Gegenstände in den Vorschlag aufzunehmen, ohne den Maximalaufwand zu überschreiten.'\}} \\
            }
        }
    }
    & \\ \\

    \theutterance \stepcounter{utterance}  
    & \multicolumn{4}{p{0.6\linewidth}}{
        \cellcolor[rgb]{0.9,0.9,0.9}{
            \makecell[{{p{\linewidth}}}]{
                \texttt{\tiny{[P1$\rangle$GM]}}
                \texttt{VORSCHLAG: \{'C56', 'A03', 'A07', 'C62', 'C10', 'C03', 'C08'\}} \\
\texttt{ARGUMENT: \{'Ich verstehe deine Bedenken und habe den Gegenstand C08 in den Vorschlag aufgenommen, um deine Anforderungen besser zu erfüllen.', 'Ich denke, dass dieser Vorschlag eine gute Balance zwischen den wichtigen Gegenständen für dich und mir bietet.', 'Der Aufwand für den Gegenstand C81 ist jedoch sehr hoch und ich bin unsicher, ob es möglich ist, ihn in den Vorschlag aufzunehmen, ohne den Maximalaufwand zu überschreiten.', 'Ich bin bereit, weiter zu verhandeln und nach möglichen Kompromissen zu suchen, um unsere gemeinsamen Interessen zu berücksichtigen.'\}} \\
            }
        }
    }
    & & \\ \\

    \theutterance \stepcounter{utterance}  
    & & \multicolumn{4}{p{0.6\linewidth}}{
        \cellcolor[rgb]{0.9,0.9,0.9}{
            \makecell[{{p{\linewidth}}}]{
                \texttt{\tiny{[GM$\rangle$P2]}}
                \texttt{VORSCHLAG: \{'C56', 'A03', 'A07', 'C62', 'C10', 'C03', 'C08'\}} \\
\texttt{ARGUMENT: \{'Ich verstehe deine Bedenken und habe den Gegenstand C08 in den Vorschlag aufgenommen, um deine Anforderungen besser zu erfüllen.', 'Ich denke, dass dieser Vorschlag eine gute Balance zwischen den wichtigen Gegenständen für dich und mir bietet.', 'Der Aufwand für den Gegenstand C81 ist jedoch sehr hoch und ich bin unsicher, ob es möglich ist, ihn in den Vorschlag aufzunehmen, ohne den Maximalaufwand zu überschreiten.', 'Ich bin bereit, weiter zu verhandeln und nach möglichen Kompromissen zu suchen, um unsere gemeinsamen Interessen zu berücksichtigen.'\}} \\
            }
        }
    }
    & \\ \\

    \theutterance \stepcounter{utterance}  
    & & & \multicolumn{4}{p{0.6\linewidth}}{
        \cellcolor[rgb]{0.9,0.9,0.9}{
            \makecell[{{p{\linewidth}}}]{
                \texttt{\tiny{[GM$\langle$P2]}}
                \texttt{ABLEHNUNG: \{'C56', 'A03', 'A07', 'C62', 'C10', 'C03', 'C08'\}} \\
\texttt{ARGUMENT: \{'Ich danke dir für die Aufnahme des Gegenstands C08 in den Vorschlag, das ist ein wichtiger Schritt in die richtige Richtung.', 'Ich verstehe deine Bedenken hinsichtlich des Aufwands für den Gegenstand C81, aber ich denke, dass er für mich sehr wichtig ist und ich bereit bin, Kompromisse einzugehen, um ihn in den Vorschlag aufzunehmen.', 'Ich denke, dass wir noch einige Möglichkeiten haben, den Vorschlag anzupassen, um den Gegenstand C81 einzubeziehen, ohne den Maximalaufwand zu überschreiten, zum Beispiel indem wir andere Gegenstände mit geringerem Aufwand ersetzen oder streichen.', 'Ich schlage vor, dass wir gemeinsam nach Lösungen suchen, um den Gegenstand C81 in den Vorschlag aufzunehmen, ohne die Balance zwischen unseren Interessen zu gefährden.'\}} \\
            }
        }
    }
     \\ \\

    \theutterance \stepcounter{utterance}  
    & & \multicolumn{4}{p{0.6\linewidth}}{
        \cellcolor[rgb]{0.9,0.9,0.9}{
            \makecell[{{p{\linewidth}}}]{
                \texttt{\tiny{[P1$\langle$GM]}}
                \texttt{ABLEHNUNG: \{'C56', 'A03', 'A07', 'C62', 'C10', 'C03', 'C08'\}} \\
\texttt{ARGUMENT: \{'Ich danke dir für die Aufnahme des Gegenstands C08 in den Vorschlag, das ist ein wichtiger Schritt in die richtige Richtung.', 'Ich verstehe deine Bedenken hinsichtlich des Aufwands für den Gegenstand C81, aber ich denke, dass er für mich sehr wichtig ist und ich bereit bin, Kompromisse einzugehen, um ihn in den Vorschlag aufzunehmen.', 'Ich denke, dass wir noch einige Möglichkeiten haben, den Vorschlag anzupassen, um den Gegenstand C81 einzubeziehen, ohne den Maximalaufwand zu überschreiten, zum Beispiel indem wir andere Gegenstände mit geringerem Aufwand ersetzen oder streichen.', 'Ich schlage vor, dass wir gemeinsam nach Lösungen suchen, um den Gegenstand C81 in den Vorschlag aufzunehmen, ohne die Balance zwischen unseren Interessen zu gefährden.'\}} \\
            }
        }
    }
    & \\ \\

    \theutterance \stepcounter{utterance}  
    & \multicolumn{4}{p{0.6\linewidth}}{
        \cellcolor[rgb]{0.9,0.9,0.9}{
            \makecell[{{p{\linewidth}}}]{
                \texttt{\tiny{[P1$\rangle$GM]}}
                \texttt{VORSCHLAG: \{'A03', 'A07', 'C10', 'C03', 'C08', 'C81'\}} \\
\texttt{ARGUMENT: \{'Ich habe den Vorschlag angepasst, um den Gegenstand C81 aufzunehmen, indem ich andere Gegenstände mit geringerem Aufwand ersetzt oder gestrichen habe.', 'Ich denke, dass dieser Vorschlag eine gute Balance zwischen den wichtigen Gegenständen für dich und mir bietet und den Maximalaufwand nicht überschreitet.', 'Ich habe den Gegenstand C56 gestrichen, um Platz für den Gegenstand C81 zu machen, da ich denke, dass C56 für mich nicht so wichtig ist wie C81 für dich.', 'Ich bin bereit, weitere Anpassungen vorzunehmen, um eine einvernehmliche Lösung zu erreichen, aber ich denke, dass dieser Vorschlag ein guter Kompromiss ist.'\}} \\
            }
        }
    }
    & & \\ \\

    \theutterance \stepcounter{utterance}  
    & & \multicolumn{4}{p{0.6\linewidth}}{
        \cellcolor[rgb]{0.9,0.9,0.9}{
            \makecell[{{p{\linewidth}}}]{
                \texttt{\tiny{[GM$\rangle$P2]}}
                \texttt{VORSCHLAG: \{'A03', 'A07', 'C10', 'C03', 'C08', 'C81'\}} \\
\texttt{ARGUMENT: \{'Ich habe den Vorschlag angepasst, um den Gegenstand C81 aufzunehmen, indem ich andere Gegenstände mit geringerem Aufwand ersetzt oder gestrichen habe.', 'Ich denke, dass dieser Vorschlag eine gute Balance zwischen den wichtigen Gegenständen für dich und mir bietet und den Maximalaufwand nicht überschreitet.', 'Ich habe den Gegenstand C56 gestrichen, um Platz für den Gegenstand C81 zu machen, da ich denke, dass C56 für mich nicht so wichtig ist wie C81 für dich.', 'Ich bin bereit, weitere Anpassungen vorzunehmen, um eine einvernehmliche Lösung zu erreichen, aber ich denke, dass dieser Vorschlag ein guter Kompromiss ist.'\}} \\
            }
        }
    }
    & \\ \\

    \theutterance \stepcounter{utterance}  
    & & & \multicolumn{4}{p{0.6\linewidth}}{
        \cellcolor[rgb]{0.9,0.9,0.9}{
            \makecell[{{p{\linewidth}}}]{
                \texttt{\tiny{[GM$\langle$P2]}}
                \texttt{ZUSTIMMUNG: \{'A03', 'A07', 'C10', 'C03', 'C08', 'C81'\}} \\
\texttt{ARGUMENT: \{'Ich bin sehr zufrieden mit dem angepassten Vorschlag, der den Gegenstand C81 einschließt, der für mich sehr wichtig ist.', 'Ich denke, dass die Streichung des Gegenstands C56 ein vernünftiger Kompromiss ist, um Platz für den Gegenstand C81 zu machen.', 'Ich bin beeindruckt von deiner Flexibilität und deiner Bereitschaft, Anpassungen vorzunehmen, um eine einvernehmliche Lösung zu erreichen.', 'Ich denke, dass dieser Vorschlag eine sehr gute Balance zwischen unseren Interessen bietet und ich bin bereit, den Vorschlag zu akzeptieren und das Spiel zu beenden.'\}} \\
            }
        }
    }
     \\ \\

    \theutterance \stepcounter{utterance}  
    & & & \multicolumn{2}{p{0.3\linewidth}}{
        \cellcolor[rgb]{0.9,0.9,0.9}{
            \makecell[{{p{\linewidth}}}]{
                \texttt{\tiny{[GM$|$GM]}}
                \texttt{game successful} \\
            }
        }
    }
    & & \\ \\

    \theutterance \stepcounter{utterance}  
    & & & \multicolumn{2}{p{0.3\linewidth}}{
        \cellcolor[rgb]{0.9,0.9,0.9}{
            \makecell[{{p{\linewidth}}}]{
                \texttt{\tiny{[GM$|$GM]}}
                \texttt{end game} \\
            }
        }
    }
    & & \\ \\

\end{supertabular}
}

\end{document}
