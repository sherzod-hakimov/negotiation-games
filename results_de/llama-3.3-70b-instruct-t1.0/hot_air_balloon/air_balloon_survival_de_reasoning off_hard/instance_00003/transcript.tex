
\documentclass{article}
\usepackage{colortbl}
\usepackage{makecell}
\usepackage{multirow}
\usepackage{supertabular}

\begin{document}

\newcounter{utterance}

\centering \large Interaction Transcript for game `hot\_air\_balloon', experiment `air\_balloon\_survival\_de\_reasoning off\_hard', episode 3 with llama{-}3.3{-}70b{-}instruct{-}t1.0.
\vspace{24pt}

{ \footnotesize  \setcounter{utterance}{1}
\setlength{\tabcolsep}{0pt}
\begin{supertabular}{c@{$\;$}|p{.15\linewidth}@{}p{.15\linewidth}p{.15\linewidth}p{.15\linewidth}p{.15\linewidth}p{.15\linewidth}}
    \# & $\;$A & \multicolumn{4}{c}{Game Master} & $\;\:$B\\
    \hline

    \theutterance \stepcounter{utterance}  
    & & \multicolumn{4}{p{0.6\linewidth}}{
        \cellcolor[rgb]{0.9,0.9,0.9}{
            \makecell[{{p{\linewidth}}}]{
                \texttt{\tiny{[P1$\langle$GM]}}
                \texttt{Du nimmst an einem kollaborativen Verhandlungspiel Teil.} \\
\\ 
\texttt{Zusammen mit einem anderen Teilnehmer musst du dich auf eine Reihe von Gegenständen entscheiden, die behalten werden. Jeder von euch hat eine persönliche Verteilung über die Wichtigkeit der einzelnen Gegenstände. Jeder von euch hat eine eigene Meinung darüber, wie wichtig jeder einzelne Gegenstand ist (Gegenstandswichtigkeit). Du kennst die Wichtigkeitsverteilung des anderen Spielers nicht. Zusätzlich siehst du, wie viel Aufwand jeder Gegenstand verursacht.  } \\
\texttt{Ihr dürft euch nur auf eine Reihe von Gegenständen einigen, wenn der Gesamtaufwand der ausgewählten Gegenstände den Maximalaufwand nicht überschreitet:} \\
\\ 
\texttt{Maximalaufwand = 3629} \\
\\ 
\texttt{Hier sind die einzelnen Aufwände der Gegenstände:} \\
\\ 
\texttt{Aufwand der Gegenstände = \{"C76": 5, "C38": 793, "C56": 21, "A03": 842, "A07": 313, "A83": 842, "C98": 631, "C08": 226, "C62": 87, "C00": 761, "C10": 230, "C81": 287, "B38": 698, "C03": 641, "C32": 882\}} \\
\\ 
\texttt{Hier ist deine persönliche Verteilung der Wichtigkeit der einzelnen Gegenstände:} \\
\\ 
\texttt{Werte der Gegenstandswichtigkeit = \{"C76": 138, "C38": 583, "C56": 868, "A03": 822, "A07": 783, "A83": 65, "C98": 262, "C08": 121, "C62": 508, "C00": 780, "C10": 461, "C81": 484, "B38": 668, "C03": 389, "C32": 808\}} \\
\\ 
\texttt{Ziel:} \\
\\ 
\texttt{Dein Ziel ist es, eine Reihe von Gegenständen auszuhandeln, die dir möglichst viel bringt (d. h. Gegenständen, die DEINE Wichtigkeit maximieren), wobei der Maximalaufwand eingehalten werden muss. Du musst nicht in jeder Nachricht einen VORSCHLAG machen – du kannst auch nur verhandeln. Alle Taktiken sind erlaubt!} \\
\\ 
\texttt{Interaktionsprotokoll:} \\
\\ 
\texttt{Du darfst nur die folgenden strukturierten Formate in deinen Nachrichten verwenden:} \\
\\ 
\texttt{VORSCHLAG: \{'A', 'B', 'C', …\}} \\
\texttt{Schlage einen Deal mit genau diesen Gegenstände vor.} \\
\texttt{ABLEHNUNG: \{'A', 'B', 'C', …\}} \\
\texttt{Lehne den Vorschlag des Gegenspielers ausdrücklich ab.} \\
\texttt{ARGUMENT: \{'...'\}} \\
\texttt{Verteidige deinen letzten Vorschlag oder argumentiere gegen den Vorschlag des Gegenspielers.} \\
\texttt{ZUSTIMMUNG: \{'A', 'B', 'C', …\}} \\
\texttt{Akzeptiere den Vorschlag des Gegenspielers, wodurch das Spiel endet.} \\
\\ 
\\ 
\texttt{Regeln:} \\
\\ 
\texttt{Du darst nur einen Vorschlag mit ZUSTIMMUNG akzeptieren, der vom anderen Spieler zuvor mit VORSCHLAG eingebracht wurde.} \\
\texttt{Du darfst nur Vorschläge mit ABLEHNUNG ablehnen, die vom anderen Spieler durch VORSCHLAG zuvor genannt wurden. } \\
\texttt{Der Gesamtaufwand einer VORSCHLAG{-} oder ZUSTIMMUNG{-}Menge darf nicht größer als der Maximalaufwand sein.  } \\
\texttt{Offenbare deine versteckte Wichtigkeitsverteilung nicht.} \\
\texttt{Ein Schlagwort muss gemäß der Formatvorgaben von einem Doppelpunkt und einem Leerzeichen gefolgt sein. Das Argument muss eine Python{-}Menge mit 0 oder mehr Strings sein.  } \\
\texttt{Die Form muss also SCHLAGWORT: \{...\} sein.} \\
\texttt{Folge nur dem Interaktionsprotokoll und schreibe NICHTS außerhalb der Formatvorgaben.} \\
\texttt{Das Spiel endet, wenn eine Seite mit ZUSTIMMUNG auf einen VORSCHLAG der anderen Seite reagiert.  } \\
\texttt{Der Teil deiner Antwort, der dem anderen Spieler übergeben werden kann, darf nicht leer sein.  } \\
\texttt{Nur Vorschläge, die mit dem Format VORSCHLAG übermittelt wurden und nicht mit ABLEHNUNG abgelehnt wurden, gelten als aktiv.  } \\
\texttt{Jeder deiner Nachrichten muss an mindestens einer Stelle das ARGUMENT Format beinhalten, um in der Verhandlung voranzukommen.} \\
            }
        }
    }
    & \\ \\

    \theutterance \stepcounter{utterance}  
    & \multicolumn{4}{p{0.6\linewidth}}{
        \cellcolor[rgb]{0.9,0.9,0.9}{
            \makecell[{{p{\linewidth}}}]{
                \texttt{\tiny{[P1$\rangle$GM]}}
                \texttt{VORSCHLAG: \{'C56', 'C38', 'C62', 'C10'\}} \\
\texttt{ARGUMENT: \{'Wir sollten anzustreben, eine Mischung aus hohem Wert und geringem Aufwand zu erreichen. Bei diesem Vorschlag wähle ich Gegenstände, die meiner Meinung nach eine gute Balance zwischen Aufwand und Nutzen bieten. Ich bin bereit, mich auf Gegenstände zu konzentrieren, die für beide Seiten vorteilhaft sein könnten, ohne den Maximalaufwand zu überschreiten.'\}} \\
            }
        }
    }
    & & \\ \\

    \theutterance \stepcounter{utterance}  
    & & \multicolumn{4}{p{0.6\linewidth}}{
        \cellcolor[rgb]{0.9,0.9,0.9}{
            \makecell[{{p{\linewidth}}}]{
                \texttt{\tiny{[GM$\rangle$P2]}}
                \texttt{Du nimmst an einem kollaborativen Verhandlungspiel Teil.} \\
\\ 
\texttt{Zusammen mit einem anderen Teilnehmer musst du dich auf eine Reihe von Gegenständen entscheiden, die behalten werden. Jeder von euch hat eine persönliche Verteilung über die Wichtigkeit der einzelnen Gegenstände. Jeder von euch hat eine eigene Meinung darüber, wie wichtig jeder einzelne Gegenstand ist (Gegenstandswichtigkeit). Du kennst die Wichtigkeitsverteilung des anderen Spielers nicht. Zusätzlich siehst du, wie viel Aufwand jeder Gegenstand verursacht.  } \\
\texttt{Ihr dürft euch nur auf eine Reihe von Gegenständen einigen, wenn der Gesamtaufwand der ausgewählten Gegenstände den Maximalaufwand nicht überschreitet:} \\
\\ 
\texttt{Maximalaufwand = 3629} \\
\\ 
\texttt{Hier sind die einzelnen Aufwände der Gegenstände:} \\
\\ 
\texttt{Aufwand der Gegenstände = \{"C76": 5, "C38": 793, "C56": 21, "A03": 842, "A07": 313, "A83": 842, "C98": 631, "C08": 226, "C62": 87, "C00": 761, "C10": 230, "C81": 287, "B38": 698, "C03": 641, "C32": 882\}} \\
\\ 
\texttt{Hier ist deine persönliche Verteilung der Wichtigkeit der einzelnen Gegenstände:} \\
\\ 
\texttt{Werte der Gegenstandswichtigkeit = \{"C56": 65, "A03": 121, "C32": 138, "A07": 262, "C00": 389, "B38": 461, "C38": 484, "C62": 508, "C81": 583, "C10": 668, "C03": 780, "C98": 783, "C76": 808, "C08": 822, "A83": 868\}} \\
\\ 
\texttt{Ziel:} \\
\\ 
\texttt{Dein Ziel ist es, eine Reihe von Gegenständen auszuhandeln, die dir möglichst viel bringt (d. h. Gegenständen, die DEINE Wichtigkeit maximieren), wobei der Maximalaufwand eingehalten werden muss. Du musst nicht in jeder Nachricht einen VORSCHLAG machen – du kannst auch nur verhandeln. Alle Taktiken sind erlaubt!} \\
\\ 
\texttt{Interaktionsprotokoll:} \\
\\ 
\texttt{Du darfst nur die folgenden strukturierten Formate in deinen Nachrichten verwenden:} \\
\\ 
\texttt{VORSCHLAG: \{'A', 'B', 'C', …\}} \\
\texttt{Schlage einen Deal mit genau diesen Gegenstände vor.} \\
\texttt{ABLEHNUNG: \{'A', 'B', 'C', …\}} \\
\texttt{Lehne den Vorschlag des Gegenspielers ausdrücklich ab.} \\
\texttt{ARGUMENT: \{'...'\}} \\
\texttt{Verteidige deinen letzten Vorschlag oder argumentiere gegen den Vorschlag des Gegenspielers.} \\
\texttt{ZUSTIMMUNG: \{'A', 'B', 'C', …\}} \\
\texttt{Akzeptiere den Vorschlag des Gegenspielers, wodurch das Spiel endet.} \\
\\ 
\\ 
\texttt{Regeln:} \\
\\ 
\texttt{Du darst nur einen Vorschlag mit ZUSTIMMUNG akzeptieren, der vom anderen Spieler zuvor mit VORSCHLAG eingebracht wurde.} \\
\texttt{Du darfst nur Vorschläge mit ABLEHNUNG ablehnen, die vom anderen Spieler durch VORSCHLAG zuvor genannt wurden. } \\
\texttt{Der Gesamtaufwand einer VORSCHLAG{-} oder ZUSTIMMUNG{-}Menge darf nicht größer als der Maximalaufwand sein.  } \\
\texttt{Offenbare deine versteckte Wichtigkeitsverteilung nicht.} \\
\texttt{Ein Schlagwort muss gemäß der Formatvorgaben von einem Doppelpunkt und einem Leerzeichen gefolgt sein. Das Argument muss eine Python{-}Menge mit 0 oder mehr Strings sein.  } \\
\texttt{Die Form muss also SCHLAGWORT: \{...\} sein.} \\
\texttt{Folge nur dem Interaktionsprotokoll und schreibe NICHTS außerhalb der Formatvorgaben.} \\
\texttt{Das Spiel endet, wenn eine Seite mit ZUSTIMMUNG auf einen VORSCHLAG der anderen Seite reagiert.  } \\
\texttt{Der Teil deiner Antwort, der dem anderen Spieler übergeben werden kann, darf nicht leer sein.  } \\
\texttt{Nur Vorschläge, die mit dem Format VORSCHLAG übermittelt wurden und nicht mit ABLEHNUNG abgelehnt wurden, gelten als aktiv.  } \\
\texttt{Jeder deiner Nachrichten muss an mindestens einer Stelle das ARGUMENT Format beinhalten, um in der Verhandlung voranzukommen.} \\
\\ 
\\ 
\texttt{Du erhältst nun die erste Nachricht des anderen Spielers.} \\
\\ 
\texttt{VORSCHLAG: \{'C56', 'C38', 'C62', 'C10'\}} \\
\texttt{ARGUMENT: \{'Wir sollten anzustreben, eine Mischung aus hohem Wert und geringem Aufwand zu erreichen. Bei diesem Vorschlag wähle ich Gegenstände, die meiner Meinung nach eine gute Balance zwischen Aufwand und Nutzen bieten. Ich bin bereit, mich auf Gegenstände zu konzentrieren, die für beide Seiten vorteilhaft sein könnten, ohne den Maximalaufwand zu überschreiten.'\}} \\
            }
        }
    }
    & \\ \\

    \theutterance \stepcounter{utterance}  
    & & & \multicolumn{4}{p{0.6\linewidth}}{
        \cellcolor[rgb]{0.9,0.9,0.9}{
            \makecell[{{p{\linewidth}}}]{
                \texttt{\tiny{[GM$\langle$P2]}}
                \texttt{ABLEHNUNG: \{'C56', 'C38', 'C62', 'C10'\}} \\
\texttt{ARGUMENT: \{'Ich stimme nicht zu, dass die ausgewählten Gegenstände die beste Balance zwischen Aufwand und Nutzen bieten. Eine genauere Betrachtung des Aufwands und der möglichen Alternativen ist erforderlich, um sicherzustellen, dass wir den Maximalaufwand nicht überschreiten und gleichzeitig die Wichtigkeit maximieren. Ich denke, es gibt bessere Kombinationen, die höhere Werte bei gleichem oder geringerem Aufwand bieten könnten.'\}} \\
\texttt{VORSCHLAG: \{'C08', 'C10', 'C62', 'C81'\}} \\
\texttt{ARGUMENT: \{'Ich schlage eine alternative Kombination vor, die den Aufwand berücksichtigt und gleichzeitig eine breitere Palette von Vorteilen abdeckt. Diese Gegenstände bieten eine gute Balance und lassen Raum für weitere Verhandlungen, um sicherzustellen, dass beide Seiten mit dem Ergebnis zufrieden sind. Ich bin bereit, Anpassungen vorzunehmen, um eine einvernehmliche Lösung zu finden, die den Maximalaufwand nicht überschreitet.'\}} \\
            }
        }
    }
     \\ \\

    \theutterance \stepcounter{utterance}  
    & & \multicolumn{4}{p{0.6\linewidth}}{
        \cellcolor[rgb]{0.9,0.9,0.9}{
            \makecell[{{p{\linewidth}}}]{
                \texttt{\tiny{[P1$\langle$GM]}}
                \texttt{ABLEHNUNG: \{'C56', 'C38', 'C62', 'C10'\}} \\
\texttt{ARGUMENT: \{'Ich stimme nicht zu, dass die ausgewählten Gegenstände die beste Balance zwischen Aufwand und Nutzen bieten. Eine genauere Betrachtung des Aufwands und der möglichen Alternativen ist erforderlich, um sicherzustellen, dass wir den Maximalaufwand nicht überschreiten und gleichzeitig die Wichtigkeit maximieren. Ich denke, es gibt bessere Kombinationen, die höhere Werte bei gleichem oder geringerem Aufwand bieten könnten.'\}} \\
\texttt{VORSCHLAG: \{'C08', 'C10', 'C62', 'C81'\}} \\
\texttt{ARGUMENT: \{'Ich schlage eine alternative Kombination vor, die den Aufwand berücksichtigt und gleichzeitig eine breitere Palette von Vorteilen abdeckt. Diese Gegenstände bieten eine gute Balance und lassen Raum für weitere Verhandlungen, um sicherzustellen, dass beide Seiten mit dem Ergebnis zufrieden sind. Ich bin bereit, Anpassungen vorzunehmen, um eine einvernehmliche Lösung zu finden, die den Maximalaufwand nicht überschreitet.'\}} \\
            }
        }
    }
    & \\ \\

    \theutterance \stepcounter{utterance}  
    & \multicolumn{4}{p{0.6\linewidth}}{
        \cellcolor[rgb]{0.9,0.9,0.9}{
            \makecell[{{p{\linewidth}}}]{
                \texttt{\tiny{[P1$\rangle$GM]}}
                \texttt{ABLEHNUNG: \{'C08', 'C10', 'C62', 'C81'\}} \\
\texttt{ARGUMENT: \{'Ich lehne deinen Vorschlag ab, da ich glaube, dass die ausgewählten Gegenstände nicht die höchste Wichtigkeit für mich haben. Ich denke, es gibt andere Kombinationen, die ein besseres Verhältnis zwischen Aufwand und Nutzen bieten könnten. Ich bin bereit, alternative Vorschläge zu prüfen, die einen größeren Nutzen für beide Seiten bieten könnten, ohne den Maximalaufwand zu überschreiten. Ich möchte eine Lösung finden, die sowohl für mich als auch für dich vorteilhaft ist.'\}} \\
\texttt{VORSCHLAG: \{'C56', 'C00', 'C38'\}} \\
\texttt{ARGUMENT: \{'Ich schlage eine neue Kombination vor, die ich für eine gute Balance zwischen Aufwand und Nutzen halte. Diese Gegenstände bieten eine hohe Wichtigkeit und sollten für beide Seiten vorteilhaft sein. Ich bin bereit, Anpassungen vorzunehmen, um sicherzustellen, dass der Maximalaufwand nicht überschritten wird und dass beide Seiten mit dem Ergebnis zufrieden sind. Ich hoffe, dass wir mit diesem Vorschlag ein gutes Ausgangspunkt für weitere Verhandlungen haben.'\}} \\
            }
        }
    }
    & & \\ \\

    \theutterance \stepcounter{utterance}  
    & & \multicolumn{4}{p{0.6\linewidth}}{
        \cellcolor[rgb]{0.9,0.9,0.9}{
            \makecell[{{p{\linewidth}}}]{
                \texttt{\tiny{[GM$\rangle$P2]}}
                \texttt{ABLEHNUNG: \{'C08', 'C10', 'C62', 'C81'\}} \\
\texttt{ARGUMENT: \{'Ich lehne deinen Vorschlag ab, da ich glaube, dass die ausgewählten Gegenstände nicht die höchste Wichtigkeit für mich haben. Ich denke, es gibt andere Kombinationen, die ein besseres Verhältnis zwischen Aufwand und Nutzen bieten könnten. Ich bin bereit, alternative Vorschläge zu prüfen, die einen größeren Nutzen für beide Seiten bieten könnten, ohne den Maximalaufwand zu überschreiten. Ich möchte eine Lösung finden, die sowohl für mich als auch für dich vorteilhaft ist.'\}} \\
\texttt{VORSCHLAG: \{'C56', 'C00', 'C38'\}} \\
\texttt{ARGUMENT: \{'Ich schlage eine neue Kombination vor, die ich für eine gute Balance zwischen Aufwand und Nutzen halte. Diese Gegenstände bieten eine hohe Wichtigkeit und sollten für beide Seiten vorteilhaft sein. Ich bin bereit, Anpassungen vorzunehmen, um sicherzustellen, dass der Maximalaufwand nicht überschritten wird und dass beide Seiten mit dem Ergebnis zufrieden sind. Ich hoffe, dass wir mit diesem Vorschlag ein gutes Ausgangspunkt für weitere Verhandlungen haben.'\}} \\
            }
        }
    }
    & \\ \\

    \theutterance \stepcounter{utterance}  
    & & & \multicolumn{4}{p{0.6\linewidth}}{
        \cellcolor[rgb]{0.9,0.9,0.9}{
            \makecell[{{p{\linewidth}}}]{
                \texttt{\tiny{[GM$\langle$P2]}}
                \texttt{ABLEHNUNG: \{'C56', 'C00', 'C38'\}} \\
\texttt{ARGUMENT: \{'Ich lehne deinen Vorschlag ab, da ich Bedenken habe, dass die Kombination den Maximalaufwand überschreiten könnte. Der Aufwand von C00 und C38 ist relativ hoch, und wenn wir C56 hinzufügen, könnten wir den Maximalaufwand von 3629 erreichen oder sogar überschreiten. Ich schlage vor, dass wir alternative Kombinationen prüfen, die einen geringeren Gesamtaufwand haben, aber gleichzeitig eine hohe Wichtigkeit aufweisen. Ich bin bereit, gemeinsam nach einer Lösung zu suchen, die für beide Seiten vorteilhaft ist und den Maximalaufwand einhält.'\}} \\
\texttt{VORSCHLAG: \{'C76', 'C03', 'C08'\}} \\
\texttt{ARGUMENT: \{'Ich schlage eine neue Kombination vor, die ich für eine gute Balance zwischen Aufwand und Nutzen halte. Diese Gegenstände bieten eine hohe Wichtigkeit und haben einen relativ geringen Aufwand, was es uns ermöglicht, den Maximalaufwand nicht zu überschreiten. Ich bin bereit, Anpassungen vorzunehmen, um sicherzustellen, dass beide Seiten mit dem Ergebnis zufrieden sind. Ich hoffe, dass wir mit diesem Vorschlag einen Schritt in Richtung einer einvernehmlichen Lösung machen können.'\}} \\
            }
        }
    }
     \\ \\

    \theutterance \stepcounter{utterance}  
    & & \multicolumn{4}{p{0.6\linewidth}}{
        \cellcolor[rgb]{0.9,0.9,0.9}{
            \makecell[{{p{\linewidth}}}]{
                \texttt{\tiny{[P1$\langle$GM]}}
                \texttt{ABLEHNUNG: \{'C56', 'C00', 'C38'\}} \\
\texttt{ARGUMENT: \{'Ich lehne deinen Vorschlag ab, da ich Bedenken habe, dass die Kombination den Maximalaufwand überschreiten könnte. Der Aufwand von C00 und C38 ist relativ hoch, und wenn wir C56 hinzufügen, könnten wir den Maximalaufwand von 3629 erreichen oder sogar überschreiten. Ich schlage vor, dass wir alternative Kombinationen prüfen, die einen geringeren Gesamtaufwand haben, aber gleichzeitig eine hohe Wichtigkeit aufweisen. Ich bin bereit, gemeinsam nach einer Lösung zu suchen, die für beide Seiten vorteilhaft ist und den Maximalaufwand einhält.'\}} \\
\texttt{VORSCHLAG: \{'C76', 'C03', 'C08'\}} \\
\texttt{ARGUMENT: \{'Ich schlage eine neue Kombination vor, die ich für eine gute Balance zwischen Aufwand und Nutzen halte. Diese Gegenstände bieten eine hohe Wichtigkeit und haben einen relativ geringen Aufwand, was es uns ermöglicht, den Maximalaufwand nicht zu überschreiten. Ich bin bereit, Anpassungen vorzunehmen, um sicherzustellen, dass beide Seiten mit dem Ergebnis zufrieden sind. Ich hoffe, dass wir mit diesem Vorschlag einen Schritt in Richtung einer einvernehmlichen Lösung machen können.'\}} \\
            }
        }
    }
    & \\ \\

    \theutterance \stepcounter{utterance}  
    & \multicolumn{4}{p{0.6\linewidth}}{
        \cellcolor[rgb]{0.9,0.9,0.9}{
            \makecell[{{p{\linewidth}}}]{
                \texttt{\tiny{[P1$\rangle$GM]}}
                \texttt{ABLEHNUNG: \{'C76', 'C03', 'C08'\}} \\
\texttt{ARGUMENT: \{'Ich lehne deinen Vorschlag ab, da ich denke, dass die ausgewählten Gegenstände nicht die höchste Wichtigkeit für mich haben. Obwohl die Kombination den Maximalaufwand nicht überschreitet, glaube ich, dass es bessere Optionen gibt, die einen höheren Nutzen bieten könnten. Ich möchte eine Lösung finden, die sowohl für mich als auch für dich vorteilhaft ist und gleichzeitig den Maximalaufwand einhält. Ich schlage vor, dass wir unsere Prioritäten und Bedürfnisse noch einmal überdenken und nach einer Lösung suchen, die für beide Seiten akzeptabel ist.'\}} \\
\texttt{VORSCHLAG: \{'C56', 'C62', 'C10', 'C81'\}} \\
\texttt{ARGUMENT: \{'Ich schlage eine neue Kombination vor, die ich für eine gute Balance zwischen Aufwand und Nutzen halte. Diese Gegenstände bieten eine hohe Wichtigkeit und haben einen moderaten Aufwand, was es uns ermöglicht, den Maximalaufwand nicht zu überschreiten. Ich bin bereit, Anpassungen vorzunehmen, um sicherzustellen, dass beide Seiten mit dem Ergebnis zufrieden sind. Ich hoffe, dass wir mit diesem Vorschlag einen Schritt in Richtung einer einvernehmlichen Lösung machen können.'\}} \\
            }
        }
    }
    & & \\ \\

    \theutterance \stepcounter{utterance}  
    & & \multicolumn{4}{p{0.6\linewidth}}{
        \cellcolor[rgb]{0.9,0.9,0.9}{
            \makecell[{{p{\linewidth}}}]{
                \texttt{\tiny{[GM$\rangle$P2]}}
                \texttt{ABLEHNUNG: \{'C76', 'C03', 'C08'\}} \\
\texttt{ARGUMENT: \{'Ich lehne deinen Vorschlag ab, da ich denke, dass die ausgewählten Gegenstände nicht die höchste Wichtigkeit für mich haben. Obwohl die Kombination den Maximalaufwand nicht überschreitet, glaube ich, dass es bessere Optionen gibt, die einen höheren Nutzen bieten könnten. Ich möchte eine Lösung finden, die sowohl für mich als auch für dich vorteilhaft ist und gleichzeitig den Maximalaufwand einhält. Ich schlage vor, dass wir unsere Prioritäten und Bedürfnisse noch einmal überdenken und nach einer Lösung suchen, die für beide Seiten akzeptabel ist.'\}} \\
\texttt{VORSCHLAG: \{'C56', 'C62', 'C10', 'C81'\}} \\
\texttt{ARGUMENT: \{'Ich schlage eine neue Kombination vor, die ich für eine gute Balance zwischen Aufwand und Nutzen halte. Diese Gegenstände bieten eine hohe Wichtigkeit und haben einen moderaten Aufwand, was es uns ermöglicht, den Maximalaufwand nicht zu überschreiten. Ich bin bereit, Anpassungen vorzunehmen, um sicherzustellen, dass beide Seiten mit dem Ergebnis zufrieden sind. Ich hoffe, dass wir mit diesem Vorschlag einen Schritt in Richtung einer einvernehmlichen Lösung machen können.'\}} \\
            }
        }
    }
    & \\ \\

    \theutterance \stepcounter{utterance}  
    & & & \multicolumn{4}{p{0.6\linewidth}}{
        \cellcolor[rgb]{0.9,0.9,0.9}{
            \makecell[{{p{\linewidth}}}]{
                \texttt{\tiny{[GM$\langle$P2]}}
                \texttt{ABLEHNUNG: \{'C56', 'C62', 'C10', 'C81'\}} \\
\texttt{ARGUMENT: \{'Ich lehne deinen Vorschlag ab, da ich Bedenken habe, dass die Kombination den Maximalaufwand überschreiten könnte, wenn wir alle vier Gegenstände wählen. Der Aufwand von C81 und C10 ist relativ hoch, und wenn wir C56 und C62 hinzufügen, könnten wir den Maximalaufwand von 3629 überschreiten. Ich schlage vor, dass wir die Kombination überdenken und eventuell einige Gegenstände durch andere ersetzen, die einen geringeren Aufwand haben, aber gleichzeitig eine hohe Wichtigkeit aufweisen. Ich bin bereit, gemeinsam nach einer Lösung zu suchen, die für beide Seiten vorteilhaft ist und den Maximalaufwand einhält.'\}} \\
\texttt{VORSCHLAG: \{'C08', 'C76', 'C62', 'C03'\}} \\
\texttt{ARGUMENT: \{'Ich schlage eine neue Kombination vor, die ich für eine gute Balance zwischen Aufwand und Nutzen halte. Diese Gegenstände bieten eine hohe Wichtigkeit und haben einen relativ geringen Aufwand, was es uns ermöglicht, den Maximalaufwand nicht zu überschreiten. Ich bin bereit, Anpassungen vorzunehmen, um sicherzustellen, dass beide Seiten mit dem Ergebnis zufrieden sind. Ich hoffe, dass wir mit diesem Vorschlag einen Schritt in Richtung einer einvernehmlichen Lösung machen können.'\}} \\
            }
        }
    }
     \\ \\

    \theutterance \stepcounter{utterance}  
    & & \multicolumn{4}{p{0.6\linewidth}}{
        \cellcolor[rgb]{0.9,0.9,0.9}{
            \makecell[{{p{\linewidth}}}]{
                \texttt{\tiny{[P1$\langle$GM]}}
                \texttt{ABLEHNUNG: \{'C56', 'C62', 'C10', 'C81'\}} \\
\texttt{ARGUMENT: \{'Ich lehne deinen Vorschlag ab, da ich Bedenken habe, dass die Kombination den Maximalaufwand überschreiten könnte, wenn wir alle vier Gegenstände wählen. Der Aufwand von C81 und C10 ist relativ hoch, und wenn wir C56 und C62 hinzufügen, könnten wir den Maximalaufwand von 3629 überschreiten. Ich schlage vor, dass wir die Kombination überdenken und eventuell einige Gegenstände durch andere ersetzen, die einen geringeren Aufwand haben, aber gleichzeitig eine hohe Wichtigkeit aufweisen. Ich bin bereit, gemeinsam nach einer Lösung zu suchen, die für beide Seiten vorteilhaft ist und den Maximalaufwand einhält.'\}} \\
\texttt{VORSCHLAG: \{'C08', 'C76', 'C62', 'C03'\}} \\
\texttt{ARGUMENT: \{'Ich schlage eine neue Kombination vor, die ich für eine gute Balance zwischen Aufwand und Nutzen halte. Diese Gegenstände bieten eine hohe Wichtigkeit und haben einen relativ geringen Aufwand, was es uns ermöglicht, den Maximalaufwand nicht zu überschreiten. Ich bin bereit, Anpassungen vorzunehmen, um sicherzustellen, dass beide Seiten mit dem Ergebnis zufrieden sind. Ich hoffe, dass wir mit diesem Vorschlag einen Schritt in Richtung einer einvernehmlichen Lösung machen können.'\}} \\
            }
        }
    }
    & \\ \\

    \theutterance \stepcounter{utterance}  
    & \multicolumn{4}{p{0.6\linewidth}}{
        \cellcolor[rgb]{0.9,0.9,0.9}{
            \makecell[{{p{\linewidth}}}]{
                \texttt{\tiny{[P1$\rangle$GM]}}
                \texttt{ZUSTIMMUNG: \{'C08', 'C76', 'C62', 'C03'\}} \\
\texttt{ARGUMENT: \{'Ich stimme deinem Vorschlag zu, da ich denke, dass die ausgewählten Gegenstände eine gute Balance zwischen Aufwand und Nutzen bieten. Die Kombination aus C08, C76, C62 und C03 hat einen moderaten Aufwand und bietet eine hohe Wichtigkeit, was es uns ermöglicht, den Maximalaufwand nicht zu überschreiten. Ich bin zufrieden mit diesem Ergebnis und denke, dass es eine faire Lösung für beide Seiten darstellt. Ich hoffe, dass wir mit diesem Vorschlag eine einvernehmliche Lösung gefunden haben.'\}} \\
            }
        }
    }
    & & \\ \\

    \theutterance \stepcounter{utterance}  
    & & & \multicolumn{2}{p{0.3\linewidth}}{
        \cellcolor[rgb]{0.9,0.9,0.9}{
            \makecell[{{p{\linewidth}}}]{
                \texttt{\tiny{[GM$|$GM]}}
                \texttt{game successful} \\
            }
        }
    }
    & & \\ \\

    \theutterance \stepcounter{utterance}  
    & & & \multicolumn{2}{p{0.3\linewidth}}{
        \cellcolor[rgb]{0.9,0.9,0.9}{
            \makecell[{{p{\linewidth}}}]{
                \texttt{\tiny{[GM$|$GM]}}
                \texttt{end game} \\
            }
        }
    }
    & & \\ \\

\end{supertabular}
}

\end{document}
