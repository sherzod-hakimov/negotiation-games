
\documentclass{article}
\usepackage{colortbl}
\usepackage{makecell}
\usepackage{multirow}
\usepackage{supertabular}

\begin{document}

\newcounter{utterance}

\centering \large Interaction Transcript for game `hot\_air\_balloon', experiment `air\_balloon\_survival\_de\_reasoning off\_hard', episode 4 with llama{-}3.3{-}70b{-}instruct{-}t1.0.
\vspace{24pt}

{ \footnotesize  \setcounter{utterance}{1}
\setlength{\tabcolsep}{0pt}
\begin{supertabular}{c@{$\;$}|p{.15\linewidth}@{}p{.15\linewidth}p{.15\linewidth}p{.15\linewidth}p{.15\linewidth}p{.15\linewidth}}
    \# & $\;$A & \multicolumn{4}{c}{Game Master} & $\;\:$B\\
    \hline

    \theutterance \stepcounter{utterance}  
    & & \multicolumn{4}{p{0.6\linewidth}}{
        \cellcolor[rgb]{0.9,0.9,0.9}{
            \makecell[{{p{\linewidth}}}]{
                \texttt{\tiny{[P1$\langle$GM]}}
                \texttt{Du nimmst an einem kollaborativen Verhandlungspiel Teil.} \\
\\ 
\texttt{Zusammen mit einem anderen Teilnehmer musst du dich auf eine Reihe von Gegenständen entscheiden, die behalten werden. Jeder von euch hat eine persönliche Verteilung über die Wichtigkeit der einzelnen Gegenstände. Jeder von euch hat eine eigene Meinung darüber, wie wichtig jeder einzelne Gegenstand ist (Gegenstandswichtigkeit). Du kennst die Wichtigkeitsverteilung des anderen Spielers nicht. Zusätzlich siehst du, wie viel Aufwand jeder Gegenstand verursacht.  } \\
\texttt{Ihr dürft euch nur auf eine Reihe von Gegenständen einigen, wenn der Gesamtaufwand der ausgewählten Gegenstände den Maximalaufwand nicht überschreitet:} \\
\\ 
\texttt{Maximalaufwand = 2984} \\
\\ 
\texttt{Hier sind die einzelnen Aufwände der Gegenstände:} \\
\\ 
\texttt{Aufwand der Gegenstände = \{"C76": 350, "C38": 276, "C56": 616, "A03": 737, "A07": 531, "A83": 389, "C98": 24, "C08": 125, "C62": 338, "C00": 356, "C10": 143, "C81": 117, "B38": 257, "C03": 921, "C32": 789\}} \\
\\ 
\texttt{Hier ist deine persönliche Verteilung der Wichtigkeit der einzelnen Gegenstände:} \\
\\ 
\texttt{Werte der Gegenstandswichtigkeit = \{"C76": 138, "C38": 583, "C56": 868, "A03": 822, "A07": 783, "A83": 65, "C98": 262, "C08": 121, "C62": 508, "C00": 780, "C10": 461, "C81": 484, "B38": 668, "C03": 389, "C32": 808\}} \\
\\ 
\texttt{Ziel:} \\
\\ 
\texttt{Dein Ziel ist es, eine Reihe von Gegenständen auszuhandeln, die dir möglichst viel bringt (d. h. Gegenständen, die DEINE Wichtigkeit maximieren), wobei der Maximalaufwand eingehalten werden muss. Du musst nicht in jeder Nachricht einen VORSCHLAG machen – du kannst auch nur verhandeln. Alle Taktiken sind erlaubt!} \\
\\ 
\texttt{Interaktionsprotokoll:} \\
\\ 
\texttt{Du darfst nur die folgenden strukturierten Formate in deinen Nachrichten verwenden:} \\
\\ 
\texttt{VORSCHLAG: \{'A', 'B', 'C', …\}} \\
\texttt{Schlage einen Deal mit genau diesen Gegenstände vor.} \\
\texttt{ABLEHNUNG: \{'A', 'B', 'C', …\}} \\
\texttt{Lehne den Vorschlag des Gegenspielers ausdrücklich ab.} \\
\texttt{ARGUMENT: \{'...'\}} \\
\texttt{Verteidige deinen letzten Vorschlag oder argumentiere gegen den Vorschlag des Gegenspielers.} \\
\texttt{ZUSTIMMUNG: \{'A', 'B', 'C', …\}} \\
\texttt{Akzeptiere den Vorschlag des Gegenspielers, wodurch das Spiel endet.} \\
\\ 
\\ 
\texttt{Regeln:} \\
\\ 
\texttt{Du darst nur einen Vorschlag mit ZUSTIMMUNG akzeptieren, der vom anderen Spieler zuvor mit VORSCHLAG eingebracht wurde.} \\
\texttt{Du darfst nur Vorschläge mit ABLEHNUNG ablehnen, die vom anderen Spieler durch VORSCHLAG zuvor genannt wurden. } \\
\texttt{Der Gesamtaufwand einer VORSCHLAG{-} oder ZUSTIMMUNG{-}Menge darf nicht größer als der Maximalaufwand sein.  } \\
\texttt{Offenbare deine versteckte Wichtigkeitsverteilung nicht.} \\
\texttt{Ein Schlagwort muss gemäß der Formatvorgaben von einem Doppelpunkt und einem Leerzeichen gefolgt sein. Das Argument muss eine Python{-}Menge mit 0 oder mehr Strings sein.  } \\
\texttt{Die Form muss also SCHLAGWORT: \{...\} sein.} \\
\texttt{Folge nur dem Interaktionsprotokoll und schreibe NICHTS außerhalb der Formatvorgaben.} \\
\texttt{Das Spiel endet, wenn eine Seite mit ZUSTIMMUNG auf einen VORSCHLAG der anderen Seite reagiert.  } \\
\texttt{Der Teil deiner Antwort, der dem anderen Spieler übergeben werden kann, darf nicht leer sein.  } \\
\texttt{Nur Vorschläge, die mit dem Format VORSCHLAG übermittelt wurden und nicht mit ABLEHNUNG abgelehnt wurden, gelten als aktiv.  } \\
\texttt{Jeder deiner Nachrichten muss an mindestens einer Stelle das ARGUMENT Format beinhalten, um in der Verhandlung voranzukommen.} \\
            }
        }
    }
    & \\ \\

    \theutterance \stepcounter{utterance}  
    & \multicolumn{4}{p{0.6\linewidth}}{
        \cellcolor[rgb]{0.9,0.9,0.9}{
            \makecell[{{p{\linewidth}}}]{
                \texttt{\tiny{[P1$\rangle$GM]}}
                \texttt{VORSCHLAG: \{'C38', 'C56', 'A03', 'A07', 'C00', 'C10', 'C81', 'B38'\}} \\
\texttt{ARGUMENT: \{'Ich denke, dass diese Kombination eine gute Balance zwischen Wichtigkeit und Aufwand bietet', 'Die Gegenstände C38, C56 und A03 haben einen hohen Wert und sollten daher bevorzugt werden', 'Die Gegenstände C00, C10 und C81 haben einen moderaten Aufwand und sollten ebenfalls berücksichtigt werden', 'Der Gegenstand B38 ist wichtig und sollteilyn aufgrund seines moderaten Aufwands ebenfalls einbezogen werden'\}} \\
            }
        }
    }
    & & \\ \\

    \theutterance \stepcounter{utterance}  
    & & \multicolumn{4}{p{0.6\linewidth}}{
        \cellcolor[rgb]{0.9,0.9,0.9}{
            \makecell[{{p{\linewidth}}}]{
                \texttt{\tiny{[GM$\rangle$P2]}}
                \texttt{Du nimmst an einem kollaborativen Verhandlungspiel Teil.} \\
\\ 
\texttt{Zusammen mit einem anderen Teilnehmer musst du dich auf eine Reihe von Gegenständen entscheiden, die behalten werden. Jeder von euch hat eine persönliche Verteilung über die Wichtigkeit der einzelnen Gegenstände. Jeder von euch hat eine eigene Meinung darüber, wie wichtig jeder einzelne Gegenstand ist (Gegenstandswichtigkeit). Du kennst die Wichtigkeitsverteilung des anderen Spielers nicht. Zusätzlich siehst du, wie viel Aufwand jeder Gegenstand verursacht.  } \\
\texttt{Ihr dürft euch nur auf eine Reihe von Gegenständen einigen, wenn der Gesamtaufwand der ausgewählten Gegenstände den Maximalaufwand nicht überschreitet:} \\
\\ 
\texttt{Maximalaufwand = 2984} \\
\\ 
\texttt{Hier sind die einzelnen Aufwände der Gegenstände:} \\
\\ 
\texttt{Aufwand der Gegenstände = \{"C76": 350, "C38": 276, "C56": 616, "A03": 737, "A07": 531, "A83": 389, "C98": 24, "C08": 125, "C62": 338, "C00": 356, "C10": 143, "C81": 117, "B38": 257, "C03": 921, "C32": 789\}} \\
\\ 
\texttt{Hier ist deine persönliche Verteilung der Wichtigkeit der einzelnen Gegenstände:} \\
\\ 
\texttt{Werte der Gegenstandswichtigkeit = \{"C56": 65, "A03": 121, "C32": 138, "A07": 262, "C00": 389, "B38": 461, "C38": 484, "C62": 508, "C81": 583, "C10": 668, "C03": 780, "C98": 783, "C76": 808, "C08": 822, "A83": 868\}} \\
\\ 
\texttt{Ziel:} \\
\\ 
\texttt{Dein Ziel ist es, eine Reihe von Gegenständen auszuhandeln, die dir möglichst viel bringt (d. h. Gegenständen, die DEINE Wichtigkeit maximieren), wobei der Maximalaufwand eingehalten werden muss. Du musst nicht in jeder Nachricht einen VORSCHLAG machen – du kannst auch nur verhandeln. Alle Taktiken sind erlaubt!} \\
\\ 
\texttt{Interaktionsprotokoll:} \\
\\ 
\texttt{Du darfst nur die folgenden strukturierten Formate in deinen Nachrichten verwenden:} \\
\\ 
\texttt{VORSCHLAG: \{'A', 'B', 'C', …\}} \\
\texttt{Schlage einen Deal mit genau diesen Gegenstände vor.} \\
\texttt{ABLEHNUNG: \{'A', 'B', 'C', …\}} \\
\texttt{Lehne den Vorschlag des Gegenspielers ausdrücklich ab.} \\
\texttt{ARGUMENT: \{'...'\}} \\
\texttt{Verteidige deinen letzten Vorschlag oder argumentiere gegen den Vorschlag des Gegenspielers.} \\
\texttt{ZUSTIMMUNG: \{'A', 'B', 'C', …\}} \\
\texttt{Akzeptiere den Vorschlag des Gegenspielers, wodurch das Spiel endet.} \\
\\ 
\\ 
\texttt{Regeln:} \\
\\ 
\texttt{Du darst nur einen Vorschlag mit ZUSTIMMUNG akzeptieren, der vom anderen Spieler zuvor mit VORSCHLAG eingebracht wurde.} \\
\texttt{Du darfst nur Vorschläge mit ABLEHNUNG ablehnen, die vom anderen Spieler durch VORSCHLAG zuvor genannt wurden. } \\
\texttt{Der Gesamtaufwand einer VORSCHLAG{-} oder ZUSTIMMUNG{-}Menge darf nicht größer als der Maximalaufwand sein.  } \\
\texttt{Offenbare deine versteckte Wichtigkeitsverteilung nicht.} \\
\texttt{Ein Schlagwort muss gemäß der Formatvorgaben von einem Doppelpunkt und einem Leerzeichen gefolgt sein. Das Argument muss eine Python{-}Menge mit 0 oder mehr Strings sein.  } \\
\texttt{Die Form muss also SCHLAGWORT: \{...\} sein.} \\
\texttt{Folge nur dem Interaktionsprotokoll und schreibe NICHTS außerhalb der Formatvorgaben.} \\
\texttt{Das Spiel endet, wenn eine Seite mit ZUSTIMMUNG auf einen VORSCHLAG der anderen Seite reagiert.  } \\
\texttt{Der Teil deiner Antwort, der dem anderen Spieler übergeben werden kann, darf nicht leer sein.  } \\
\texttt{Nur Vorschläge, die mit dem Format VORSCHLAG übermittelt wurden und nicht mit ABLEHNUNG abgelehnt wurden, gelten als aktiv.  } \\
\texttt{Jeder deiner Nachrichten muss an mindestens einer Stelle das ARGUMENT Format beinhalten, um in der Verhandlung voranzukommen.} \\
\\ 
\\ 
\texttt{Du erhältst nun die erste Nachricht des anderen Spielers.} \\
\\ 
\texttt{VORSCHLAG: \{'C38', 'C56', 'A03', 'A07', 'C00', 'C10', 'C81', 'B38'\}} \\
\texttt{ARGUMENT: \{'Ich denke, dass diese Kombination eine gute Balance zwischen Wichtigkeit und Aufwand bietet', 'Die Gegenstände C38, C56 und A03 haben einen hohen Wert und sollten daher bevorzugt werden', 'Die Gegenstände C00, C10 und C81 haben einen moderaten Aufwand und sollten ebenfalls berücksichtigt werden', 'Der Gegenstand B38 ist wichtig und sollteilyn aufgrund seines moderaten Aufwands ebenfalls einbezogen werden'\}} \\
            }
        }
    }
    & \\ \\

    \theutterance \stepcounter{utterance}  
    & & & \multicolumn{4}{p{0.6\linewidth}}{
        \cellcolor[rgb]{0.9,0.9,0.9}{
            \makecell[{{p{\linewidth}}}]{
                \texttt{\tiny{[GM$\langle$P2]}}
                \texttt{ABLEHNUNG: \{'C38', 'C56', 'A03', 'A07', 'C00', 'C10', 'C81', 'B38'\}} \\
\texttt{ARGUMENT: \{'Ich denke, dass der vorgeschlagene Satz einen zu hohen Aufwand hat', 'Der Gesamtaufwand dieser Gegenstände übersteigt möglicherweise den Maximalaufwand', 'Ich bevorzuge einen Ansatz, der den Aufwand besser kontrolliert', 'Ich schlage vor, dass wir einzelne Gegenstände priorisieren und schrittweise hinzufügen, um den optimalen Kompromiss zu finden'\}} \\
            }
        }
    }
     \\ \\

    \theutterance \stepcounter{utterance}  
    & & \multicolumn{4}{p{0.6\linewidth}}{
        \cellcolor[rgb]{0.9,0.9,0.9}{
            \makecell[{{p{\linewidth}}}]{
                \texttt{\tiny{[P1$\langle$GM]}}
                \texttt{ABLEHNUNG: \{'C38', 'C56', 'A03', 'A07', 'C00', 'C10', 'C81', 'B38'\}} \\
\texttt{ARGUMENT: \{'Ich denke, dass der vorgeschlagene Satz einen zu hohen Aufwand hat', 'Der Gesamtaufwand dieser Gegenstände übersteigt möglicherweise den Maximalaufwand', 'Ich bevorzuge einen Ansatz, der den Aufwand besser kontrolliert', 'Ich schlage vor, dass wir einzelne Gegenstände priorisieren und schrittweise hinzufügen, um den optimalen Kompromiss zu finden'\}} \\
            }
        }
    }
    & \\ \\

    \theutterance \stepcounter{utterance}  
    & \multicolumn{4}{p{0.6\linewidth}}{
        \cellcolor[rgb]{0.9,0.9,0.9}{
            \makecell[{{p{\linewidth}}}]{
                \texttt{\tiny{[P1$\rangle$GM]}}
                \texttt{VORSCHLAG: \{'C38', 'C56', 'A03', 'C00', 'C10'\}} \\
\texttt{ARGUMENT: \{'Ich verstehe die Bedenken hinsichtlich des Aufwands und habe daher einige Gegenstände entfernt, um den Gesamtaufwand zu reduzieren', 'Die Gegenstände C38, C56 und A03 sind sehr wichtig und sollten nach wie vor priorisiert werden', 'Die Gegenstände C00 und C10 haben einen moderaten Aufwand und tragen dazu bei, den Gesamtaufwand im Rahmen zu halten', 'Ich denke, dass dieser Vorschlag eine gute Balance zwischen Wichtigkeit und Aufwand bietet', 'Ich bin bereit, weitere Anpassungen vorzunehmen, um eine Einigung zu erzielen'\}} \\
            }
        }
    }
    & & \\ \\

    \theutterance \stepcounter{utterance}  
    & & \multicolumn{4}{p{0.6\linewidth}}{
        \cellcolor[rgb]{0.9,0.9,0.9}{
            \makecell[{{p{\linewidth}}}]{
                \texttt{\tiny{[GM$\rangle$P2]}}
                \texttt{VORSCHLAG: \{'C38', 'C56', 'A03', 'C00', 'C10'\}} \\
\texttt{ARGUMENT: \{'Ich verstehe die Bedenken hinsichtlich des Aufwands und habe daher einige Gegenstände entfernt, um den Gesamtaufwand zu reduzieren', 'Die Gegenstände C38, C56 und A03 sind sehr wichtig und sollten nach wie vor priorisiert werden', 'Die Gegenstände C00 und C10 haben einen moderaten Aufwand und tragen dazu bei, den Gesamtaufwand im Rahmen zu halten', 'Ich denke, dass dieser Vorschlag eine gute Balance zwischen Wichtigkeit und Aufwand bietet', 'Ich bin bereit, weitere Anpassungen vorzunehmen, um eine Einigung zu erzielen'\}} \\
            }
        }
    }
    & \\ \\

    \theutterance \stepcounter{utterance}  
    & & & \multicolumn{4}{p{0.6\linewidth}}{
        \cellcolor[rgb]{0.9,0.9,0.9}{
            \makecell[{{p{\linewidth}}}]{
                \texttt{\tiny{[GM$\langle$P2]}}
                \texttt{ARGUMENT: \{'Ich appreciate die Anstrengung, den Vorschlag anzupassen, jedoch habe ich Bedenken hinsichtlich der Priorisierung bestimmter Gegenstände', 'Ich denke, dass C56 und A03 einen sehr hohen Aufwand haben und möglicherweise durch andere Gegenstände ersetzt werden sollten, die einen besseren Nutzen{-}Aufwand{-}Kompromiss bieten', 'Ich wäre bereit, über alternative Kombinationen zu diskutieren, die den Gesamtaufwand effizienter nutzen', 'Ich schlage vor, dass wir die Gegenstände einzeln bewerten und nach ihrem Nutzen{-}Aufwand{-}Verhältnis priorisieren'\}} \\
\texttt{VORSCHLAG: \{'C38', 'C00', 'C10', 'C81', 'B38'\}} \\
            }
        }
    }
     \\ \\

    \theutterance \stepcounter{utterance}  
    & & \multicolumn{4}{p{0.6\linewidth}}{
        \cellcolor[rgb]{0.9,0.9,0.9}{
            \makecell[{{p{\linewidth}}}]{
                \texttt{\tiny{[P1$\langle$GM]}}
                \texttt{ARGUMENT: \{'Ich appreciate die Anstrengung, den Vorschlag anzupassen, jedoch habe ich Bedenken hinsichtlich der Priorisierung bestimmter Gegenstände', 'Ich denke, dass C56 und A03 einen sehr hohen Aufwand haben und möglicherweise durch andere Gegenstände ersetzt werden sollten, die einen besseren Nutzen{-}Aufwand{-}Kompromiss bieten', 'Ich wäre bereit, über alternative Kombinationen zu diskutieren, die den Gesamtaufwand effizienter nutzen', 'Ich schlage vor, dass wir die Gegenstände einzeln bewerten und nach ihrem Nutzen{-}Aufwand{-}Verhältnis priorisieren'\}} \\
\texttt{VORSCHLAG: \{'C38', 'C00', 'C10', 'C81', 'B38'\}} \\
            }
        }
    }
    & \\ \\

    \theutterance \stepcounter{utterance}  
    & \multicolumn{4}{p{0.6\linewidth}}{
        \cellcolor[rgb]{0.9,0.9,0.9}{
            \makecell[{{p{\linewidth}}}]{
                \texttt{\tiny{[P1$\rangle$GM]}}
                \texttt{ABLEHNUNG: \{'C38', 'C00', 'C10', 'C81', 'B38'\}} \\
\texttt{ARGUMENT: \{'Ich verstehe die Priorisierung nach dem Nutzen{-}Aufwand{-}Verhältnis, aber ich denke, dass dieser Vorschlag einige wichtige Gegenstände außer Acht lässt', 'Die Gegenstände C56 und A03, die in meinem vorherigen Vorschlag enthalten waren, bieten meiner Meinung nach einen hohen Wert, auch wenn sie einen höheren Aufwand haben', 'Ich bin bereit, Kompromisse einzugehen, aber ich denke, dass wir auch die langfristigen Vorteile berücksichtigen sollten, die einige der Gegenstände mit höherem Aufwand bieten könnten', 'Ich schlage vor, dass wir uns auf die Gegenstände konzentrieren, die einen hohen Wert haben und gleichzeitig einen moderaten Aufwand, um eine ausgewogene Lösung zu finden'\}} \\
            }
        }
    }
    & & \\ \\

    \theutterance \stepcounter{utterance}  
    & & \multicolumn{4}{p{0.6\linewidth}}{
        \cellcolor[rgb]{0.9,0.9,0.9}{
            \makecell[{{p{\linewidth}}}]{
                \texttt{\tiny{[GM$\rangle$P2]}}
                \texttt{ABLEHNUNG: \{'C38', 'C00', 'C10', 'C81', 'B38'\}} \\
\texttt{ARGUMENT: \{'Ich verstehe die Priorisierung nach dem Nutzen{-}Aufwand{-}Verhältnis, aber ich denke, dass dieser Vorschlag einige wichtige Gegenstände außer Acht lässt', 'Die Gegenstände C56 und A03, die in meinem vorherigen Vorschlag enthalten waren, bieten meiner Meinung nach einen hohen Wert, auch wenn sie einen höheren Aufwand haben', 'Ich bin bereit, Kompromisse einzugehen, aber ich denke, dass wir auch die langfristigen Vorteile berücksichtigen sollten, die einige der Gegenstände mit höherem Aufwand bieten könnten', 'Ich schlage vor, dass wir uns auf die Gegenstände konzentrieren, die einen hohen Wert haben und gleichzeitig einen moderaten Aufwand, um eine ausgewogene Lösung zu finden'\}} \\
            }
        }
    }
    & \\ \\

    \theutterance \stepcounter{utterance}  
    & & & \multicolumn{4}{p{0.6\linewidth}}{
        \cellcolor[rgb]{0.9,0.9,0.9}{
            \makecell[{{p{\linewidth}}}]{
                \texttt{\tiny{[GM$\langle$P2]}}
                \texttt{ARGUMENT: \{'Ich verstehe die Bedenken hinsichtlich der langfristigen Vorteile und der Wertigkeit bestimmter Gegenstände', 'Ich denke jedoch, dass wir auch den Gesamtaufwand und die effiziente Nutzung der Ressourcen berücksichtigen sollten', 'Die Gegenstände C56 und A03 haben zwar einen hohen Wert, aber auch einen sehr hohen Aufwand, der möglicherweise andere wichtige Gegenstände ausschließt', 'Ich schlage vor, dass wir eine Liste von Gegenständen erstellen, die einen hohen Wert und einen moderaten Aufwand haben, und dann gemeinsam priorisieren, um eine optimale Lösung zu finden'\}} \\
\texttt{VORSCHLAG: \{'C38', 'C10', 'C81', 'B38', 'C08'\}} \\
            }
        }
    }
     \\ \\

    \theutterance \stepcounter{utterance}  
    & & \multicolumn{4}{p{0.6\linewidth}}{
        \cellcolor[rgb]{0.9,0.9,0.9}{
            \makecell[{{p{\linewidth}}}]{
                \texttt{\tiny{[P1$\langle$GM]}}
                \texttt{ARGUMENT: \{'Ich verstehe die Bedenken hinsichtlich der langfristigen Vorteile und der Wertigkeit bestimmter Gegenstände', 'Ich denke jedoch, dass wir auch den Gesamtaufwand und die effiziente Nutzung der Ressourcen berücksichtigen sollten', 'Die Gegenstände C56 und A03 haben zwar einen hohen Wert, aber auch einen sehr hohen Aufwand, der möglicherweise andere wichtige Gegenstände ausschließt', 'Ich schlage vor, dass wir eine Liste von Gegenständen erstellen, die einen hohen Wert und einen moderaten Aufwand haben, und dann gemeinsam priorisieren, um eine optimale Lösung zu finden'\}} \\
\texttt{VORSCHLAG: \{'C38', 'C10', 'C81', 'B38', 'C08'\}} \\
            }
        }
    }
    & \\ \\

    \theutterance \stepcounter{utterance}  
    & \multicolumn{4}{p{0.6\linewidth}}{
        \cellcolor[rgb]{0.9,0.9,0.9}{
            \makecell[{{p{\linewidth}}}]{
                \texttt{\tiny{[P1$\rangle$GM]}}
                \texttt{VORSCHLAG: \{'C38', 'C10', 'C81', 'B38', 'C62'\}} \\
\texttt{ARGUMENT: \{'Ich denke, dass der vorherige Vorschlag eine gute Grundlage bietet, aber ich würde vorschlagen, den Gegenstand C08 durch C62 zu ersetzen', 'C62 hat einen höheren Wert als C08 und einen moderaten Aufwand, der den Gesamtaufwand nicht übermäßig erhöht', 'Ich denke, dass diese Kombination eine gute Balance zwischen Wert und Aufwand bietet und die Bedenken hinsichtlich der langfristigen Vorteile und der effizienten Ressourcennutzung berücksichtigt', 'Ich bin bereit, weitere Anpassungen vorzunehmen, um eine Einigung zu erzielen, aber ich denke, dass dieser Vorschlag eine gute Richtung zeigt'\}} \\
            }
        }
    }
    & & \\ \\

    \theutterance \stepcounter{utterance}  
    & & \multicolumn{4}{p{0.6\linewidth}}{
        \cellcolor[rgb]{0.9,0.9,0.9}{
            \makecell[{{p{\linewidth}}}]{
                \texttt{\tiny{[GM$\rangle$P2]}}
                \texttt{VORSCHLAG: \{'C38', 'C10', 'C81', 'B38', 'C62'\}} \\
\texttt{ARGUMENT: \{'Ich denke, dass der vorherige Vorschlag eine gute Grundlage bietet, aber ich würde vorschlagen, den Gegenstand C08 durch C62 zu ersetzen', 'C62 hat einen höheren Wert als C08 und einen moderaten Aufwand, der den Gesamtaufwand nicht übermäßig erhöht', 'Ich denke, dass diese Kombination eine gute Balance zwischen Wert und Aufwand bietet und die Bedenken hinsichtlich der langfristigen Vorteile und der effizienten Ressourcennutzung berücksichtigt', 'Ich bin bereit, weitere Anpassungen vorzunehmen, um eine Einigung zu erzielen, aber ich denke, dass dieser Vorschlag eine gute Richtung zeigt'\}} \\
            }
        }
    }
    & \\ \\

    \theutterance \stepcounter{utterance}  
    & & & \multicolumn{4}{p{0.6\linewidth}}{
        \cellcolor[rgb]{0.9,0.9,0.9}{
            \makecell[{{p{\linewidth}}}]{
                \texttt{\tiny{[GM$\langle$P2]}}
                \texttt{ARGUMENT: \{'Ich denke, dass der Vorschlag, C08 durch C62 zu ersetzen, eine interessante Idee ist', 'C62 hat tatsächlich einen höheren Wert als C08, aber ich muss auch den Gesamtaufwand und die Auswirkungen auf die anderen Gegenstände berücksichtigen', 'Ich denke, dass diese Kombination eine gute Balance zwischen Wert und Aufwand bietet, aber ich würde gerne nochmal den Gesamtaufwand berechnen, um sicherzustellen, dass wir innerhalb des Limits bleiben', 'Ich bin bereit, diesem Vorschlag zuzustimmen, wenn der Gesamtaufwand innerhalb des Limits bleibt und die Kombination eine optimale Lösung darstellt'\}} \\
\texttt{ZUSTIMMUNG: \{'C38', 'C10', 'C81', 'B38', 'C62'\}} \\
            }
        }
    }
     \\ \\

    \theutterance \stepcounter{utterance}  
    & & & \multicolumn{2}{p{0.3\linewidth}}{
        \cellcolor[rgb]{0.9,0.9,0.9}{
            \makecell[{{p{\linewidth}}}]{
                \texttt{\tiny{[GM$|$GM]}}
                \texttt{game successful} \\
            }
        }
    }
    & & \\ \\

    \theutterance \stepcounter{utterance}  
    & & & \multicolumn{2}{p{0.3\linewidth}}{
        \cellcolor[rgb]{0.9,0.9,0.9}{
            \makecell[{{p{\linewidth}}}]{
                \texttt{\tiny{[GM$|$GM]}}
                \texttt{end game} \\
            }
        }
    }
    & & \\ \\

\end{supertabular}
}

\end{document}
