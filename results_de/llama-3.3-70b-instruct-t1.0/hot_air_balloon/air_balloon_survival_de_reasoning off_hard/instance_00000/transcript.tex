
\documentclass{article}
\usepackage{colortbl}
\usepackage{makecell}
\usepackage{multirow}
\usepackage{supertabular}

\begin{document}

\newcounter{utterance}

\twocolumn

{ \footnotesize  \setcounter{utterance}{1}
\setlength{\tabcolsep}{0pt}
\begin{supertabular}{c@{$\;$}|p{.15\linewidth}@{}p{.15\linewidth}p{.15\linewidth}p{.15\linewidth}p{.15\linewidth}p{.15\linewidth}}

    \# & $\;$A & \multicolumn{4}{c}{Game Master} & $\;\:$B\\
    \hline 

    \theutterance \stepcounter{utterance}  

    & & \multicolumn{4}{p{0.6\linewidth}}{\cellcolor[rgb]{0.9,0.9,0.9}{%
	\makecell[{{p{\linewidth}}}]{% 
	  \tt {\tiny [A$\langle$GM]}  
	 Du nimmst an einem kollaborativen Verhandlungspiel Teil.\\ \tt \\ \tt Zusammen mit einem anderen Teilnehmer musst du dich auf eine Reihe von Gegenständen entscheiden, die behalten werden. Jeder von euch hat eine persönliche Verteilung über die Wichtigkeit der einzelnen Gegenstände. Jeder von euch hat eine eigene Meinung darüber, wie wichtig jeder einzelne Gegenstand ist (Gegenstandswichtigkeit). Du kennst die Wichtigkeitsverteilung des anderen Spielers nicht. Zusätzlich siehst du, wie viel Aufwand jeder Gegenstand verursacht.  \\ \tt Ihr dürft euch nur auf eine Reihe von Gegenständen einigen, wenn der Gesamtaufwand der ausgewählten Gegenstände den Maximalaufwand nicht überschreitet:\\ \tt \\ \tt Maximalaufwand = 3683\\ \tt \\ \tt Hier sind die einzelnen Aufwände der Gegenstände:\\ \tt \\ \tt Aufwand der Gegenstände = {"C76": 548, "C38": 154, "C56": 60, "A03": 517, "A07": 334, "A83": 542, "C98": 707, "C08": 139, "C62": 661, "C00": 780, "C10": 832, "C81": 913, "B38": 219, "C03": 323, "C32": 638}\\ \tt \\ \tt Hier ist deine persönliche Verteilung der Wichtigkeit der einzelnen Gegenstände:\\ \tt \\ \tt Werte der Gegenstandswichtigkeit = {"C76": 138, "C38": 583, "C56": 868, "A03": 822, "A07": 783, "A83": 65, "C98": 262, "C08": 121, "C62": 508, "C00": 780, "C10": 461, "C81": 484, "B38": 668, "C03": 389, "C32": 808}\\ \tt \\ \tt Ziel:\\ \tt \\ \tt Dein Ziel ist es, eine Reihe von Gegenständen auszuhandeln, die dir möglichst viel bringt (d. h. Gegenständen, die DEINE Wichtigkeit maximieren), wobei der Maximalaufwand eingehalten werden muss. Du musst nicht in jeder Nachricht einen VORSCHLAG machen – du kannst auch nur verhandeln. Alle Taktiken sind erlaubt!\\ \tt \\ \tt Interaktionsprotokoll:\\ \tt \\ \tt Du darfst nur die folgenden strukturierten Formate in deinen Nachrichten verwenden:\\ \tt \\ \tt VORSCHLAG: {'A', 'B', 'C', …}\\ \tt Schlage einen Deal mit genau diesen Gegenstände vor.\\ \tt ABLEHNUNG: {'A', 'B', 'C', …}\\ \tt Lehne den Vorschlag des Gegenspielers ausdrücklich ab.\\ \tt ARGUMENT: {'...'}\\ \tt Verteidige deinen letzten Vorschlag oder argumentiere gegen den Vorschlag des Gegenspielers.\\ \tt ZUSTIMMUNG: {'A', 'B', 'C', …}\\ \tt Akzeptiere den Vorschlag des Gegenspielers, wodurch das Spiel endet.\\ \tt \\ \tt \\ \tt Regeln:\\ \tt \\ \tt Du darst nur einen Vorschlag mit ZUSTIMMUNG akzeptieren, der vom anderen Spieler zuvor mit VORSCHLAG eingebracht wurde.\\ \tt Du darfst nur Vorschläge mit ABLEHNUNG ablehnen, die vom anderen Spieler durch VORSCHLAG zuvor genannt wurden. \\ \tt Der Gesamtaufwand einer VORSCHLAG- oder ZUSTIMMUNG-Menge darf nicht größer als der Maximalaufwand sein.  \\ \tt Offenbare deine versteckte Wichtigkeitsverteilung nicht.\\ \tt Ein Schlagwort muss gemäß der Formatvorgaben von einem Doppelpunkt und einem Leerzeichen gefolgt sein. Das Argument muss eine Python-Menge mit 0 oder mehr Strings sein.  \\ \tt Die Form muss also SCHLAGWORT: {...} sein.\\ \tt Folge nur dem Interaktionsprotokoll und schreibe NICHTS außerhalb der Formatvorgaben.\\ \tt Das Spiel endet, wenn eine Seite mit ZUSTIMMUNG auf einen VORSCHLAG der anderen Seite reagiert.  \\ \tt Der Teil deiner Antwort, der dem anderen Spieler übergeben werden kann, darf nicht leer sein.  \\ \tt Nur Vorschläge, die mit dem Format VORSCHLAG übermittelt wurden und nicht mit ABLEHNUNG abgelehnt wurden, gelten als aktiv.  \\ \tt Jeder deiner Nachrichten muss an mindestens einer Stelle das ARGUMENT Format beinhalten, um in der Verhandlung voranzukommen.\\ \tt  
	  } 
	   } 
	   } 
	 & \\ 
 

    \theutterance \stepcounter{utterance}  

    & \multicolumn{4}{p{0.6\linewidth}}{\cellcolor[rgb]{0.8,1,0.9}{%
	\makecell[{{p{\linewidth}}}]{% 
	  \tt {\tiny [A$\rangle$GM]}  
	 VORSCHLAG: {'C56', 'C38', 'A03', 'A07'}\\ \tt ARGUMENT: {'Ich beginne mit einem Angebot, das eine gute Mischung aus wichtigen und weniger aufwändigen Gegenständen bietet. Dieser Vorschlag sollte sowohl für mich als auch für den anderen Spieler von Vorteil sein, da die Gesamtaufwandskosten innerhalb des maximalen Limits bleiben.'} 
	  } 
	   } 
	   } 
	 & & \\ 
 

    \theutterance \stepcounter{utterance}  

    & & \multicolumn{4}{p{0.6\linewidth}}{\cellcolor[rgb]{0.9,0.9,0.9}{%
	\makecell[{{p{\linewidth}}}]{% 
	  \tt {\tiny [GM$\rangle$B]}  
	 Du nimmst an einem kollaborativen Verhandlungspiel Teil.\\ \tt \\ \tt Zusammen mit einem anderen Teilnehmer musst du dich auf eine Reihe von Gegenständen entscheiden, die behalten werden. Jeder von euch hat eine persönliche Verteilung über die Wichtigkeit der einzelnen Gegenstände. Jeder von euch hat eine eigene Meinung darüber, wie wichtig jeder einzelne Gegenstand ist (Gegenstandswichtigkeit). Du kennst die Wichtigkeitsverteilung des anderen Spielers nicht. Zusätzlich siehst du, wie viel Aufwand jeder Gegenstand verursacht.  \\ \tt Ihr dürft euch nur auf eine Reihe von Gegenständen einigen, wenn der Gesamtaufwand der ausgewählten Gegenstände den Maximalaufwand nicht überschreitet:\\ \tt \\ \tt Maximalaufwand = 3683\\ \tt \\ \tt Hier sind die einzelnen Aufwände der Gegenstände:\\ \tt \\ \tt Aufwand der Gegenstände = {"C76": 548, "C38": 154, "C56": 60, "A03": 517, "A07": 334, "A83": 542, "C98": 707, "C08": 139, "C62": 661, "C00": 780, "C10": 832, "C81": 913, "B38": 219, "C03": 323, "C32": 638}\\ \tt \\ \tt Hier ist deine persönliche Verteilung der Wichtigkeit der einzelnen Gegenstände:\\ \tt \\ \tt Werte der Gegenstandswichtigkeit = {"C56": 65, "A03": 121, "C32": 138, "A07": 262, "C00": 389, "B38": 461, "C38": 484, "C62": 508, "C81": 583, "C10": 668, "C03": 780, "C98": 783, "C76": 808, "C08": 822, "A83": 868}\\ \tt \\ \tt Ziel:\\ \tt \\ \tt Dein Ziel ist es, eine Reihe von Gegenständen auszuhandeln, die dir möglichst viel bringt (d. h. Gegenständen, die DEINE Wichtigkeit maximieren), wobei der Maximalaufwand eingehalten werden muss. Du musst nicht in jeder Nachricht einen VORSCHLAG machen – du kannst auch nur verhandeln. Alle Taktiken sind erlaubt!\\ \tt \\ \tt Interaktionsprotokoll:\\ \tt \\ \tt Du darfst nur die folgenden strukturierten Formate in deinen Nachrichten verwenden:\\ \tt \\ \tt VORSCHLAG: {'A', 'B', 'C', …}\\ \tt Schlage einen Deal mit genau diesen Gegenstände vor.\\ \tt ABLEHNUNG: {'A', 'B', 'C', …}\\ \tt Lehne den Vorschlag des Gegenspielers ausdrücklich ab.\\ \tt ARGUMENT: {'...'}\\ \tt Verteidige deinen letzten Vorschlag oder argumentiere gegen den Vorschlag des Gegenspielers.\\ \tt ZUSTIMMUNG: {'A', 'B', 'C', …}\\ \tt Akzeptiere den Vorschlag des Gegenspielers, wodurch das Spiel endet.\\ \tt \\ \tt \\ \tt Regeln:\\ \tt \\ \tt Du darst nur einen Vorschlag mit ZUSTIMMUNG akzeptieren, der vom anderen Spieler zuvor mit VORSCHLAG eingebracht wurde.\\ \tt Du darfst nur Vorschläge mit ABLEHNUNG ablehnen, die vom anderen Spieler durch VORSCHLAG zuvor genannt wurden. \\ \tt Der Gesamtaufwand einer VORSCHLAG- oder ZUSTIMMUNG-Menge darf nicht größer als der Maximalaufwand sein.  \\ \tt Offenbare deine versteckte Wichtigkeitsverteilung nicht.\\ \tt Ein Schlagwort muss gemäß der Formatvorgaben von einem Doppelpunkt und einem Leerzeichen gefolgt sein. Das Argument muss eine Python-Menge mit 0 oder mehr Strings sein.  \\ \tt Die Form muss also SCHLAGWORT: {...} sein.\\ \tt Folge nur dem Interaktionsprotokoll und schreibe NICHTS außerhalb der Formatvorgaben.\\ \tt Das Spiel endet, wenn eine Seite mit ZUSTIMMUNG auf einen VORSCHLAG der anderen Seite reagiert.  \\ \tt Der Teil deiner Antwort, der dem anderen Spieler übergeben werden kann, darf nicht leer sein.  \\ \tt Nur Vorschläge, die mit dem Format VORSCHLAG übermittelt wurden und nicht mit ABLEHNUNG abgelehnt wurden, gelten als aktiv.  \\ \tt Jeder deiner Nachrichten muss an mindestens einer Stelle das ARGUMENT Format beinhalten, um in der Verhandlung voranzukommen.\\ \tt \\ \tt \\ \tt Du erhältst nun die erste Nachricht des anderen Spielers.\\ \tt \\ \tt VORSCHLAG: {'C56', 'C38', 'A03', 'A07'}\\ \tt ARGUMENT: {'Ich beginne mit einem Angebot, das eine gute Mischung aus wichtigen und weniger aufwändigen Gegenständen bietet. Dieser Vorschlag sollte sowohl für mich als auch für den anderen Spieler von Vorteil sein, da die Gesamtaufwandskosten innerhalb des maximalen Limits bleiben.'} 
	  } 
	   } 
	   } 
	 & \\ 
 

    \theutterance \stepcounter{utterance}  

    & & & \multicolumn{4}{p{0.6\linewidth}}{\cellcolor[rgb]{1,0.85,0.72}{%
	\makecell[{{p{\linewidth}}}]{% 
	  \tt {\tiny [GM$\langle$B]}  
	 ABLEHNUNG: {'C56', 'C38', 'A03', 'A07'}\\ \tt ARGUMENT: {'Ich lehne diesen Vorschlag ab, da ich denke, dass es bessere Kombinationen gibt, die höhere Wichtigkeiten für mich abdecken, ohne den Maximalaufwand zu überschreiten. Der Vorschlag konzentriert sich auf weniger wichtige Gegenstände und lässt einige der hochwertigeren aus.'}\\ \tt VORSCHLAG: {'C08', 'C38', 'C62', 'C81'}\\ \tt ARGUMENT: {'Ich schlage eine Alternative vor, die eine bessere Balance zwischen Aufwand und Wichtigkeit bietet. Diese Kombination umfasst Gegenstände, die für mich von höherem Wert sind und gleichzeitig den Gesamtaufwand im Rahmen halten.'} 
	  } 
	   } 
	   } 
	  \\ 
 

    \theutterance \stepcounter{utterance}  

    & & \multicolumn{4}{p{0.6\linewidth}}{\cellcolor[rgb]{0.9,0.9,0.9}{%
	\makecell[{{p{\linewidth}}}]{% 
	  \tt {\tiny [A$\langle$GM]}  
	 ABLEHNUNG: {'C56', 'C38', 'A03', 'A07'}\\ \tt ARGUMENT: {'Ich lehne diesen Vorschlag ab, da ich denke, dass es bessere Kombinationen gibt, die höhere Wichtigkeiten für mich abdecken, ohne den Maximalaufwand zu überschreiten. Der Vorschlag konzentriert sich auf weniger wichtige Gegenstände und lässt einige der hochwertigeren aus.'}\\ \tt VORSCHLAG: {'C08', 'C38', 'C62', 'C81'}\\ \tt ARGUMENT: {'Ich schlage eine Alternative vor, die eine bessere Balance zwischen Aufwand und Wichtigkeit bietet. Diese Kombination umfasst Gegenstände, die für mich von höherem Wert sind und gleichzeitig den Gesamtaufwand im Rahmen halten.'} 
	  } 
	   } 
	   } 
	 & \\ 
 

    \theutterance \stepcounter{utterance}  

    & \multicolumn{4}{p{0.6\linewidth}}{\cellcolor[rgb]{0.8,1,0.9}{%
	\makecell[{{p{\linewidth}}}]{% 
	  \tt {\tiny [A$\rangle$GM]}  
	 ABLEHNUNG: {'C08', 'C38', 'C62', 'C81'}\\ \tt ARGUMENT: {'Ich lehne diesen Vorschlag ab, da ich Bedenken habe, dass die Gegenstände C62 und C81 den Aufwand deutlich erhöhen, ohne dass ich sicher bin, ob sie beide unbedingt für mich von höchster Priorität sind. Es gibt möglicherweise effizientere Kombinationen, die sowohl den Aufwand als auch die Wichtigkeit besser ausbalancieren.'}\\ \tt VORSCHLAG: {'A03', 'C38', 'C56', 'B38'}\\ \tt ARGUMENT: {'Ich schlage eine neue Kombination vor, die eine gute Balance zwischen hohen Wichtigkeiten und moderatem Aufwand bietet. Diese Auswahl deckt eine breite Palette ab und sollte für beide Seiten vorteilhaft sein, da sie den Maximalaufwand nicht überschreitet und wichtige Gegenstände einschließt.'} 
	  } 
	   } 
	   } 
	 & & \\ 
 

    \theutterance \stepcounter{utterance}  

    & & \multicolumn{4}{p{0.6\linewidth}}{\cellcolor[rgb]{0.9,0.9,0.9}{%
	\makecell[{{p{\linewidth}}}]{% 
	  \tt {\tiny [GM$\rangle$B]}  
	 ABLEHNUNG: {'C08', 'C38', 'C62', 'C81'}\\ \tt ARGUMENT: {'Ich lehne diesen Vorschlag ab, da ich Bedenken habe, dass die Gegenstände C62 und C81 den Aufwand deutlich erhöhen, ohne dass ich sicher bin, ob sie beide unbedingt für mich von höchster Priorität sind. Es gibt möglicherweise effizientere Kombinationen, die sowohl den Aufwand als auch die Wichtigkeit besser ausbalancieren.'}\\ \tt VORSCHLAG: {'A03', 'C38', 'C56', 'B38'}\\ \tt ARGUMENT: {'Ich schlage eine neue Kombination vor, die eine gute Balance zwischen hohen Wichtigkeiten und moderatem Aufwand bietet. Diese Auswahl deckt eine breite Palette ab und sollte für beide Seiten vorteilhaft sein, da sie den Maximalaufwand nicht überschreitet und wichtige Gegenstände einschließt.'} 
	  } 
	   } 
	   } 
	 & \\ 
 

    \theutterance \stepcounter{utterance}  

    & & & \multicolumn{4}{p{0.6\linewidth}}{\cellcolor[rgb]{1,0.85,0.72}{%
	\makecell[{{p{\linewidth}}}]{% 
	  \tt {\tiny [GM$\langle$B]}  
	 ABLEHNUNG: {'A03', 'C38', 'C56', 'B38'}\\ \tt ARGUMENT: {'Ich lehne diesen Vorschlag ab, da ich denke, dass er wichtige Gegenstände wie C62 und C81 auslässt, die für mich von hoher Priorität sind. Zudem ist die Auswahl zu sehr auf moderate Wichtigkeiten fokussiert und vernachlässigt die höheren Werte, die andere Gegenstände bieten könnten.'}\\ \tt ARGUMENT: {'Es ist wichtig, eine Kombination zu finden, die nicht nur den Aufwand berücksichtigt, sondern auch die Wichtigkeit der Gegenstände. Ich suche nach einer Lösung, die die höheren Werte Priorität einräumt, ohne das Aufwandlimit zu überschreiten.'}\\ \tt VORSCHLAG: {'C76', 'C38', 'C03', 'A07'}\\ \tt ARGUMENT: {'Ich schlage eine neue Kombination vor, die eine bessere Balance zwischen hohen Wichtigkeiten und moderatem Aufwand bietet. Diese Auswahl umfasst wichtige Gegenstände und sollte für beide Seiten vorteilhaft sein, da sie den Maximalaufwand nicht überschreitet und eine gute Mischung aus Wichtigkeiten bietet.'} 
	  } 
	   } 
	   } 
	  \\ 
 

    \theutterance \stepcounter{utterance}  

    & & \multicolumn{4}{p{0.6\linewidth}}{\cellcolor[rgb]{0.9,0.9,0.9}{%
	\makecell[{{p{\linewidth}}}]{% 
	  \tt {\tiny [A$\langle$GM]}  
	 ABLEHNUNG: {'A03', 'C38', 'C56', 'B38'}\\ \tt ARGUMENT: {'Ich lehne diesen Vorschlag ab, da ich denke, dass er wichtige Gegenstände wie C62 und C81 auslässt, die für mich von hoher Priorität sind. Zudem ist die Auswahl zu sehr auf moderate Wichtigkeiten fokussiert und vernachlässigt die höheren Werte, die andere Gegenstände bieten könnten.'}\\ \tt ARGUMENT: {'Es ist wichtig, eine Kombination zu finden, die nicht nur den Aufwand berücksichtigt, sondern auch die Wichtigkeit der Gegenstände. Ich suche nach einer Lösung, die die höheren Werte Priorität einräumt, ohne das Aufwandlimit zu überschreiten.'}\\ \tt VORSCHLAG: {'C76', 'C38', 'C03', 'A07'}\\ \tt ARGUMENT: {'Ich schlage eine neue Kombination vor, die eine bessere Balance zwischen hohen Wichtigkeiten und moderatem Aufwand bietet. Diese Auswahl umfasst wichtige Gegenstände und sollte für beide Seiten vorteilhaft sein, da sie den Maximalaufwand nicht überschreitet und eine gute Mischung aus Wichtigkeiten bietet.'} 
	  } 
	   } 
	   } 
	 & \\ 
 

    \theutterance \stepcounter{utterance}  

    & \multicolumn{4}{p{0.6\linewidth}}{\cellcolor[rgb]{0.8,1,0.9}{%
	\makecell[{{p{\linewidth}}}]{% 
	  \tt {\tiny [A$\rangle$GM]}  
	 ABLEHNUNG: {'C76', 'C38', 'C03', 'A07'}\\ \tt ARGUMENT: {'Ich lehne diesen Vorschlag ab, da ich denke, dass das Angebot zu sehr auf Gegenstände mit mittlerer Wichtigkeit setzt und wichtige Elemente wie A03 und C56 nicht berücksichtigt, die für mich von hoher Priorität sind. Zudem ist der Aufwand für C76 höher als bei anderen Optionen, die ähnliche Wichtigkeit bieten.'}\\ \tt ARGUMENT: {'Es ist wichtig, die Kombinationen sorgfältig zu bewerten und sicherzustellen, dass sie sowohl den Aufwand als auch die Wichtigkeit der Gegenstände berücksichtigen. Ich bin bereit, weitere Vorschläge zu diskutieren, die eine bessere Balance bieten.'}\\ \tt VORSCHLAG: {'A03', 'C56', 'C38', 'C08'}\\ \tt ARGUMENT: {'Ich schlage eine neue Kombination vor, die eine gute Balance zwischen hohen Wichtigkeiten und moderatem Aufwand bietet. Diese Auswahl umfasst wichtige Gegenstände und hält den Gesamtaufwand innerhalb des Limits, während sie eine breite Palette an Wichtigkeiten abdeckt.'} 
	  } 
	   } 
	   } 
	 & & \\ 
 

    \theutterance \stepcounter{utterance}  

    & & \multicolumn{4}{p{0.6\linewidth}}{\cellcolor[rgb]{0.9,0.9,0.9}{%
	\makecell[{{p{\linewidth}}}]{% 
	  \tt {\tiny [GM$\rangle$B]}  
	 ABLEHNUNG: {'C76', 'C38', 'C03', 'A07'}\\ \tt ARGUMENT: {'Ich lehne diesen Vorschlag ab, da ich denke, dass das Angebot zu sehr auf Gegenstände mit mittlerer Wichtigkeit setzt und wichtige Elemente wie A03 und C56 nicht berücksichtigt, die für mich von hoher Priorität sind. Zudem ist der Aufwand für C76 höher als bei anderen Optionen, die ähnliche Wichtigkeit bieten.'}\\ \tt ARGUMENT: {'Es ist wichtig, die Kombinationen sorgfältig zu bewerten und sicherzustellen, dass sie sowohl den Aufwand als auch die Wichtigkeit der Gegenstände berücksichtigen. Ich bin bereit, weitere Vorschläge zu diskutieren, die eine bessere Balance bieten.'}\\ \tt VORSCHLAG: {'A03', 'C56', 'C38', 'C08'}\\ \tt ARGUMENT: {'Ich schlage eine neue Kombination vor, die eine gute Balance zwischen hohen Wichtigkeiten und moderatem Aufwand bietet. Diese Auswahl umfasst wichtige Gegenstände und hält den Gesamtaufwand innerhalb des Limits, während sie eine breite Palette an Wichtigkeiten abdeckt.'} 
	  } 
	   } 
	   } 
	 & \\ 
 

    \theutterance \stepcounter{utterance}  

    & & & \multicolumn{4}{p{0.6\linewidth}}{\cellcolor[rgb]{1,0.85,0.72}{%
	\makecell[{{p{\linewidth}}}]{% 
	  \tt {\tiny [GM$\langle$B]}  
	 ABLEHNUNG: {'A03', 'C56', 'C38', 'C08'}\\ \tt ARGUMENT: {'Ich lehne diesen Vorschlag ab, da ich denke, dass die Auswahl zwar eine gute Mischung aus Wichtigkeiten bietet, aber wichtige Gegenstände mit höherem Wert wie C62 und C81 nicht berücksichtigt. Zudem könnte der Aufwand durch die Kombination von A03 und C56 höher sein als bei anderen möglichen Kombinationen, die ähnliche Wichtigkeiten bieten.'}\\ \tt ARGUMENT: {'Ich bin der Meinung, dass es wichtig ist, die höheren Werte Priorität einzuräumen, ohne das Aufwandlimit zu überschreiten. Eine Kombination, die dies berücksichtigt, wäre für mich akzeptabeler. Ich denke, es gibt noch Raum für eine bessere Lösung, die sowohl den Aufwand als auch die Wichtigkeit der Gegenstände optimal ausbalanciert.'}\\ \tt VORSCHLAG: {'C76', 'C38', 'C03', 'C08'}\\ \tt ARGUMENT: {'Ich schlage eine neue Kombination vor, die eine bessere Balance zwischen hohen Wichtigkeiten und moderatem Aufwand bietet. Diese Auswahl umfasst wichtige Gegenstände, hält den Gesamtaufwand innerhalb des Limits und bietet eine breite Palette an Wichtigkeiten, die sowohl für mich als auch für den anderen Spieler von Vorteil sein könnten.'} 
	  } 
	   } 
	   } 
	  \\ 
 

    \theutterance \stepcounter{utterance}  

    & & \multicolumn{4}{p{0.6\linewidth}}{\cellcolor[rgb]{0.9,0.9,0.9}{%
	\makecell[{{p{\linewidth}}}]{% 
	  \tt {\tiny [A$\langle$GM]}  
	 ABLEHNUNG: {'A03', 'C56', 'C38', 'C08'}\\ \tt ARGUMENT: {'Ich lehne diesen Vorschlag ab, da ich denke, dass die Auswahl zwar eine gute Mischung aus Wichtigkeiten bietet, aber wichtige Gegenstände mit höherem Wert wie C62 und C81 nicht berücksichtigt. Zudem könnte der Aufwand durch die Kombination von A03 und C56 höher sein als bei anderen möglichen Kombinationen, die ähnliche Wichtigkeiten bieten.'}\\ \tt ARGUMENT: {'Ich bin der Meinung, dass es wichtig ist, die höheren Werte Priorität einzuräumen, ohne das Aufwandlimit zu überschreiten. Eine Kombination, die dies berücksichtigt, wäre für mich akzeptabeler. Ich denke, es gibt noch Raum für eine bessere Lösung, die sowohl den Aufwand als auch die Wichtigkeit der Gegenstände optimal ausbalanciert.'}\\ \tt VORSCHLAG: {'C76', 'C38', 'C03', 'C08'}\\ \tt ARGUMENT: {'Ich schlage eine neue Kombination vor, die eine bessere Balance zwischen hohen Wichtigkeiten und moderatem Aufwand bietet. Diese Auswahl umfasst wichtige Gegenstände, hält den Gesamtaufwand innerhalb des Limits und bietet eine breite Palette an Wichtigkeiten, die sowohl für mich als auch für den anderen Spieler von Vorteil sein könnten.'} 
	  } 
	   } 
	   } 
	 & \\ 
 

    \theutterance \stepcounter{utterance}  

    & \multicolumn{4}{p{0.6\linewidth}}{\cellcolor[rgb]{0.8,1,0.9}{%
	\makecell[{{p{\linewidth}}}]{% 
	  \tt {\tiny [A$\rangle$GM]}  
	 ABLEHNUNG: {'C76', 'C38', 'C03', 'C08'}\\ \tt ARGUMENT: {'Ich lehne diesen Vorschlag ab, da ich denke, dass die Auswahl wiederum zu sehr auf Gegenstände mit mittlerer Wichtigkeit setzt und wichtige Elemente wie A03, C56 und B38 nicht berücksichtigt, die für mich von hoher Priorität sind. Zudem ist der Aufwand für C76 höher als bei anderen Optionen, die ähnliche Wichtigkeit bieten, und die Kombination mit C03 und C08 führt zu einer Unternutzung der verfügbaren Ressourcen.'}\\ \tt ARGUMENT: {'Es ist wichtig, eine Kombination zu finden, die die höheren Werte priorisiert, ohne das Aufwandlimit zu überschreiten. Ich bin bereit, weitere Vorschläge zu diskutieren, die eine bessere Balance zwischen Aufwand und Wichtigkeit bieten. Ich denke, dass ein Austausch von Ideen und eine gemeinsame Bewertung der Optionen zu einer fairen und optimalen Lösung führen kann.'}\\ \tt VORSCHLAG: {'A03', 'B38', 'C56', 'C08'}\\ \tt ARGUMENT: {'Ich schlage eine neue Kombination vor, die eine gute Balance zwischen hohen Wichtigkeiten und moderatem Aufwand bietet. Diese Auswahl umfasst wichtige Gegenstände, hält den Gesamtaufwand innerhalb des Limits und bietet eine breite Palette an Wichtigkeiten, die sowohl für mich als auch für den anderen Spieler von Vorteil sein könnten. Ich denke, dass diese Kombination eine faire und optimale Lösung darstellt.'} 
	  } 
	   } 
	   } 
	 & & \\ 
 

    \theutterance \stepcounter{utterance}  

    & & \multicolumn{4}{p{0.6\linewidth}}{\cellcolor[rgb]{0.9,0.9,0.9}{%
	\makecell[{{p{\linewidth}}}]{% 
	  \tt {\tiny [GM$\rangle$B]}  
	 ABLEHNUNG: {'C76', 'C38', 'C03', 'C08'}\\ \tt ARGUMENT: {'Ich lehne diesen Vorschlag ab, da ich denke, dass die Auswahl wiederum zu sehr auf Gegenstände mit mittlerer Wichtigkeit setzt und wichtige Elemente wie A03, C56 und B38 nicht berücksichtigt, die für mich von hoher Priorität sind. Zudem ist der Aufwand für C76 höher als bei anderen Optionen, die ähnliche Wichtigkeit bieten, und die Kombination mit C03 und C08 führt zu einer Unternutzung der verfügbaren Ressourcen.'}\\ \tt ARGUMENT: {'Es ist wichtig, eine Kombination zu finden, die die höheren Werte priorisiert, ohne das Aufwandlimit zu überschreiten. Ich bin bereit, weitere Vorschläge zu diskutieren, die eine bessere Balance zwischen Aufwand und Wichtigkeit bieten. Ich denke, dass ein Austausch von Ideen und eine gemeinsame Bewertung der Optionen zu einer fairen und optimalen Lösung führen kann.'}\\ \tt VORSCHLAG: {'A03', 'B38', 'C56', 'C08'}\\ \tt ARGUMENT: {'Ich schlage eine neue Kombination vor, die eine gute Balance zwischen hohen Wichtigkeiten und moderatem Aufwand bietet. Diese Auswahl umfasst wichtige Gegenstände, hält den Gesamtaufwand innerhalb des Limits und bietet eine breite Palette an Wichtigkeiten, die sowohl für mich als auch für den anderen Spieler von Vorteil sein könnten. Ich denke, dass diese Kombination eine faire und optimale Lösung darstellt.'} 
	  } 
	   } 
	   } 
	 & \\ 
 

    \theutterance \stepcounter{utterance}  

    & & & \multicolumn{4}{p{0.6\linewidth}}{\cellcolor[rgb]{1,0.85,0.72}{%
	\makecell[{{p{\linewidth}}}]{% 
	  \tt {\tiny [GM$\langle$B]}  
	 ZUSTIMMUNG: {'A03', 'B38', 'C56', 'C08'}\\ \tt ARGUMENT: {'Ich stimme diesem Vorschlag zu, da ich denke, dass die Auswahl eine gute Balance zwischen hohen Wichtigkeiten und moderatem Aufwand bietet. Die Kombination umfasst wichtige Gegenstände wie A03, B38 und C56, die für mich von hoher Priorität sind, und hält den Gesamtaufwand innerhalb des Limits. Ich denke, dass diese Kombination eine faire und optimale Lösung darstellt, die sowohl meine als auch die Bedürfnisse des anderen Spielers berücksichtigt.'} 
	  } 
	   } 
	   } 
	  \\ 
 

    \theutterance \stepcounter{utterance}  

    & & & \multicolumn{2}{p{0.3\linewidth}}{\cellcolor[rgb]{0.95,0.95,0.95}{%
	\makecell[{{p{\linewidth}}}]{% 
	  \tt {\tiny [GM$|$GM]}  
	 game successful 
	  } 
	   } 
	   } 
	 & & \\ 
 

    \theutterance \stepcounter{utterance}  

    & & & \multicolumn{2}{p{0.3\linewidth}}{\cellcolor[rgb]{0.95,0.95,0.95}{%
	\makecell[{{p{\linewidth}}}]{% 
	  \tt {\tiny [GM$|$GM]}  
	 end game 
	  } 
	   } 
	   } 
	 & & \\ 
 

\end{supertabular}
}

\end{document}
