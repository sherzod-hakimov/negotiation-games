
\documentclass{article}
\usepackage{colortbl}
\usepackage{makecell}
\usepackage{multirow}
\usepackage{supertabular}

\begin{document}

\newcounter{utterance}

\centering \large Interaction Transcript for game `hot\_air\_balloon', experiment `air\_balloon\_survival\_de\_reasoning off\_hard', episode 5 with llama{-}3.3{-}70b{-}instruct{-}t1.0.
\vspace{24pt}

{ \footnotesize  \setcounter{utterance}{1}
\setlength{\tabcolsep}{0pt}
\begin{supertabular}{c@{$\;$}|p{.15\linewidth}@{}p{.15\linewidth}p{.15\linewidth}p{.15\linewidth}p{.15\linewidth}p{.15\linewidth}}
    \# & $\;$A & \multicolumn{4}{c}{Game Master} & $\;\:$B\\
    \hline

    \theutterance \stepcounter{utterance}  
    & & \multicolumn{4}{p{0.6\linewidth}}{
        \cellcolor[rgb]{0.9,0.9,0.9}{
            \makecell[{{p{\linewidth}}}]{
                \texttt{\tiny{[P1$\langle$GM]}}
                \texttt{Du nimmst an einem kollaborativen Verhandlungspiel Teil.} \\
\\ 
\texttt{Zusammen mit einem anderen Teilnehmer musst du dich auf eine Reihe von Gegenständen entscheiden, die behalten werden. Jeder von euch hat eine persönliche Verteilung über die Wichtigkeit der einzelnen Gegenstände. Jeder von euch hat eine eigene Meinung darüber, wie wichtig jeder einzelne Gegenstand ist (Gegenstandswichtigkeit). Du kennst die Wichtigkeitsverteilung des anderen Spielers nicht. Zusätzlich siehst du, wie viel Aufwand jeder Gegenstand verursacht.  } \\
\texttt{Ihr dürft euch nur auf eine Reihe von Gegenständen einigen, wenn der Gesamtaufwand der ausgewählten Gegenstände den Maximalaufwand nicht überschreitet:} \\
\\ 
\texttt{Maximalaufwand = 2307} \\
\\ 
\texttt{Hier sind die einzelnen Aufwände der Gegenstände:} \\
\\ 
\texttt{Aufwand der Gegenstände = \{"C76": 147, "C38": 698, "C56": 588, "A03": 43, "A07": 356, "A83": 80, "C98": 95, "C08": 743, "C62": 106, "C00": 308, "C10": 325, "C81": 255, "B38": 276, "C03": 543, "C32": 51\}} \\
\\ 
\texttt{Hier ist deine persönliche Verteilung der Wichtigkeit der einzelnen Gegenstände:} \\
\\ 
\texttt{Werte der Gegenstandswichtigkeit = \{"C76": 138, "C38": 583, "C56": 65, "A03": 262, "A07": 121, "A83": 508, "C98": 461, "C08": 484, "C62": 668, "C00": 389, "C10": 215, "C81": 97, "B38": 500, "C03": 30, "C32": 400\}} \\
\\ 
\texttt{Ziel:} \\
\\ 
\texttt{Dein Ziel ist es, eine Reihe von Gegenständen auszuhandeln, die dir möglichst viel bringt (d. h. Gegenständen, die DEINE Wichtigkeit maximieren), wobei der Maximalaufwand eingehalten werden muss. Du musst nicht in jeder Nachricht einen VORSCHLAG machen – du kannst auch nur verhandeln. Alle Taktiken sind erlaubt!} \\
\\ 
\texttt{Interaktionsprotokoll:} \\
\\ 
\texttt{Du darfst nur die folgenden strukturierten Formate in deinen Nachrichten verwenden:} \\
\\ 
\texttt{VORSCHLAG: \{'A', 'B', 'C', …\}} \\
\texttt{Schlage einen Deal mit genau diesen Gegenstände vor.} \\
\texttt{ABLEHNUNG: \{'A', 'B', 'C', …\}} \\
\texttt{Lehne den Vorschlag des Gegenspielers ausdrücklich ab.} \\
\texttt{ARGUMENT: \{'...'\}} \\
\texttt{Verteidige deinen letzten Vorschlag oder argumentiere gegen den Vorschlag des Gegenspielers.} \\
\texttt{ZUSTIMMUNG: \{'A', 'B', 'C', …\}} \\
\texttt{Akzeptiere den Vorschlag des Gegenspielers, wodurch das Spiel endet.} \\
\\ 
\\ 
\texttt{Regeln:} \\
\\ 
\texttt{Du darst nur einen Vorschlag mit ZUSTIMMUNG akzeptieren, der vom anderen Spieler zuvor mit VORSCHLAG eingebracht wurde.} \\
\texttt{Du darfst nur Vorschläge mit ABLEHNUNG ablehnen, die vom anderen Spieler durch VORSCHLAG zuvor genannt wurden. } \\
\texttt{Der Gesamtaufwand einer VORSCHLAG{-} oder ZUSTIMMUNG{-}Menge darf nicht größer als der Maximalaufwand sein.  } \\
\texttt{Offenbare deine versteckte Wichtigkeitsverteilung nicht.} \\
\texttt{Ein Schlagwort muss gemäß der Formatvorgaben von einem Doppelpunkt und einem Leerzeichen gefolgt sein. Das Argument muss eine Python{-}Menge mit 0 oder mehr Strings sein.  } \\
\texttt{Die Form muss also SCHLAGWORT: \{...\} sein.} \\
\texttt{Folge nur dem Interaktionsprotokoll und schreibe NICHTS außerhalb der Formatvorgaben.} \\
\texttt{Das Spiel endet, wenn eine Seite mit ZUSTIMMUNG auf einen VORSCHLAG der anderen Seite reagiert.  } \\
\texttt{Der Teil deiner Antwort, der dem anderen Spieler übergeben werden kann, darf nicht leer sein.  } \\
\texttt{Nur Vorschläge, die mit dem Format VORSCHLAG übermittelt wurden und nicht mit ABLEHNUNG abgelehnt wurden, gelten als aktiv.  } \\
\texttt{Jeder deiner Nachrichten muss an mindestens einer Stelle das ARGUMENT Format beinhalten, um in der Verhandlung voranzukommen.} \\
            }
        }
    }
    & \\ \\

    \theutterance \stepcounter{utterance}  
    & \multicolumn{4}{p{0.6\linewidth}}{
        \cellcolor[rgb]{0.9,0.9,0.9}{
            \makecell[{{p{\linewidth}}}]{
                \texttt{\tiny{[P1$\rangle$GM]}}
                \texttt{VORSCHLAG: \{'A03', 'A07', 'A83', 'C62', 'C81'\}} \\
\texttt{ARGUMENT: \{'Ich möchte mich auf eine Kombination von Gegenständen konzentrieren, die einen guten Ausgleich zwischen Aufwand und Nutzen bieten', 'Die Auswahl aus A03, A07, A83, C62 und C81 scheint mir eine gute Grundlage für weitere Verhandlungen zu sein', 'Ich bin offen für Anpassungen, um den Bedürfnissen beider Seiten gerecht zu werden'\}} \\
            }
        }
    }
    & & \\ \\

    \theutterance \stepcounter{utterance}  
    & & \multicolumn{4}{p{0.6\linewidth}}{
        \cellcolor[rgb]{0.9,0.9,0.9}{
            \makecell[{{p{\linewidth}}}]{
                \texttt{\tiny{[GM$\rangle$P2]}}
                \texttt{Du nimmst an einem kollaborativen Verhandlungspiel Teil.} \\
\\ 
\texttt{Zusammen mit einem anderen Teilnehmer musst du dich auf eine Reihe von Gegenständen entscheiden, die behalten werden. Jeder von euch hat eine persönliche Verteilung über die Wichtigkeit der einzelnen Gegenstände. Jeder von euch hat eine eigene Meinung darüber, wie wichtig jeder einzelne Gegenstand ist (Gegenstandswichtigkeit). Du kennst die Wichtigkeitsverteilung des anderen Spielers nicht. Zusätzlich siehst du, wie viel Aufwand jeder Gegenstand verursacht.  } \\
\texttt{Ihr dürft euch nur auf eine Reihe von Gegenständen einigen, wenn der Gesamtaufwand der ausgewählten Gegenstände den Maximalaufwand nicht überschreitet:} \\
\\ 
\texttt{Maximalaufwand = 2307} \\
\\ 
\texttt{Hier sind die einzelnen Aufwände der Gegenstände:} \\
\\ 
\texttt{Aufwand der Gegenstände = \{"C76": 147, "C38": 698, "C56": 588, "A03": 43, "A07": 356, "A83": 80, "C98": 95, "C08": 743, "C62": 106, "C00": 308, "C10": 325, "C81": 255, "B38": 276, "C03": 543, "C32": 51\}} \\
\\ 
\texttt{Hier ist deine persönliche Verteilung der Wichtigkeit der einzelnen Gegenstände:} \\
\\ 
\texttt{Werte der Gegenstandswichtigkeit = \{"C62": 30, "C38": 65, "A83": 97, "B38": 121, "C08": 138, "C98": 215, "C32": 262, "C00": 389, "A03": 400, "C10": 461, "C76": 484, "A07": 500, "C81": 508, "C56": 583, "C03": 668\}} \\
\\ 
\texttt{Ziel:} \\
\\ 
\texttt{Dein Ziel ist es, eine Reihe von Gegenständen auszuhandeln, die dir möglichst viel bringt (d. h. Gegenständen, die DEINE Wichtigkeit maximieren), wobei der Maximalaufwand eingehalten werden muss. Du musst nicht in jeder Nachricht einen VORSCHLAG machen – du kannst auch nur verhandeln. Alle Taktiken sind erlaubt!} \\
\\ 
\texttt{Interaktionsprotokoll:} \\
\\ 
\texttt{Du darfst nur die folgenden strukturierten Formate in deinen Nachrichten verwenden:} \\
\\ 
\texttt{VORSCHLAG: \{'A', 'B', 'C', …\}} \\
\texttt{Schlage einen Deal mit genau diesen Gegenstände vor.} \\
\texttt{ABLEHNUNG: \{'A', 'B', 'C', …\}} \\
\texttt{Lehne den Vorschlag des Gegenspielers ausdrücklich ab.} \\
\texttt{ARGUMENT: \{'...'\}} \\
\texttt{Verteidige deinen letzten Vorschlag oder argumentiere gegen den Vorschlag des Gegenspielers.} \\
\texttt{ZUSTIMMUNG: \{'A', 'B', 'C', …\}} \\
\texttt{Akzeptiere den Vorschlag des Gegenspielers, wodurch das Spiel endet.} \\
\\ 
\\ 
\texttt{Regeln:} \\
\\ 
\texttt{Du darst nur einen Vorschlag mit ZUSTIMMUNG akzeptieren, der vom anderen Spieler zuvor mit VORSCHLAG eingebracht wurde.} \\
\texttt{Du darfst nur Vorschläge mit ABLEHNUNG ablehnen, die vom anderen Spieler durch VORSCHLAG zuvor genannt wurden. } \\
\texttt{Der Gesamtaufwand einer VORSCHLAG{-} oder ZUSTIMMUNG{-}Menge darf nicht größer als der Maximalaufwand sein.  } \\
\texttt{Offenbare deine versteckte Wichtigkeitsverteilung nicht.} \\
\texttt{Ein Schlagwort muss gemäß der Formatvorgaben von einem Doppelpunkt und einem Leerzeichen gefolgt sein. Das Argument muss eine Python{-}Menge mit 0 oder mehr Strings sein.  } \\
\texttt{Die Form muss also SCHLAGWORT: \{...\} sein.} \\
\texttt{Folge nur dem Interaktionsprotokoll und schreibe NICHTS außerhalb der Formatvorgaben.} \\
\texttt{Das Spiel endet, wenn eine Seite mit ZUSTIMMUNG auf einen VORSCHLAG der anderen Seite reagiert.  } \\
\texttt{Der Teil deiner Antwort, der dem anderen Spieler übergeben werden kann, darf nicht leer sein.  } \\
\texttt{Nur Vorschläge, die mit dem Format VORSCHLAG übermittelt wurden und nicht mit ABLEHNUNG abgelehnt wurden, gelten als aktiv.  } \\
\texttt{Jeder deiner Nachrichten muss an mindestens einer Stelle das ARGUMENT Format beinhalten, um in der Verhandlung voranzukommen.} \\
\\ 
\\ 
\texttt{Du erhältst nun die erste Nachricht des anderen Spielers.} \\
\\ 
\texttt{VORSCHLAG: \{'A03', 'A07', 'A83', 'C62', 'C81'\}} \\
\texttt{ARGUMENT: \{'Ich möchte mich auf eine Kombination von Gegenständen konzentrieren, die einen guten Ausgleich zwischen Aufwand und Nutzen bieten', 'Die Auswahl aus A03, A07, A83, C62 und C81 scheint mir eine gute Grundlage für weitere Verhandlungen zu sein', 'Ich bin offen für Anpassungen, um den Bedürfnissen beider Seiten gerecht zu werden'\}} \\
            }
        }
    }
    & \\ \\

    \theutterance \stepcounter{utterance}  
    & & & \multicolumn{4}{p{0.6\linewidth}}{
        \cellcolor[rgb]{0.9,0.9,0.9}{
            \makecell[{{p{\linewidth}}}]{
                \texttt{\tiny{[GM$\langle$P2]}}
                \texttt{ABLEHNUNG: \{'A03', 'A07', 'A83', 'C62', 'C81'\}} \\
\texttt{ARGUMENT: \{'Der vorgeschlagene Satz von Gegenständen überschreitet nicht den Maximalaufwand, aber ich bin der Meinung, dass ein anderer Satz von Gegenständen für mich vorteilhafter sein könnte', 'Ich würde gerne alternative Kombinationen in Betracht ziehen, die möglicherweise eine bessere Balance zwischen Aufwand und Nutzen bieten', 'Einige der Gegenstände in dem vorgeschlagenen Satz sind für mich nicht so wichtig, wie andere, die nicht berücksichtigt wurden'\}} \\
\texttt{VORSCHLAG: \{'C10', 'C76', 'A07', 'C81', 'C32'\}} \\
\texttt{ARGUMENT: \{'Ich denke, dass diese Kombination eine bessere Balance zwischen Aufwand und Nutzen bietet', 'Die Gegenstände C10, C76, A07, C81 und C32 scheinen mir eine gute Auswahl zu sein, um den Maximalaufwand nicht zu überschreiten und gleichzeitig meine Bedürfnisse zu befriedigen', 'Ich bin bereit, Anpassungen vorzunehmen, um den Bedürfnissen beider Seiten gerecht zu werden'\}} \\
            }
        }
    }
     \\ \\

    \theutterance \stepcounter{utterance}  
    & & \multicolumn{4}{p{0.6\linewidth}}{
        \cellcolor[rgb]{0.9,0.9,0.9}{
            \makecell[{{p{\linewidth}}}]{
                \texttt{\tiny{[P1$\langle$GM]}}
                \texttt{ABLEHNUNG: \{'A03', 'A07', 'A83', 'C62', 'C81'\}} \\
\texttt{ARGUMENT: \{'Der vorgeschlagene Satz von Gegenständen überschreitet nicht den Maximalaufwand, aber ich bin der Meinung, dass ein anderer Satz von Gegenständen für mich vorteilhafter sein könnte', 'Ich würde gerne alternative Kombinationen in Betracht ziehen, die möglicherweise eine bessere Balance zwischen Aufwand und Nutzen bieten', 'Einige der Gegenstände in dem vorgeschlagenen Satz sind für mich nicht so wichtig, wie andere, die nicht berücksichtigt wurden'\}} \\
\texttt{VORSCHLAG: \{'C10', 'C76', 'A07', 'C81', 'C32'\}} \\
\texttt{ARGUMENT: \{'Ich denke, dass diese Kombination eine bessere Balance zwischen Aufwand und Nutzen bietet', 'Die Gegenstände C10, C76, A07, C81 und C32 scheinen mir eine gute Auswahl zu sein, um den Maximalaufwand nicht zu überschreiten und gleichzeitig meine Bedürfnisse zu befriedigen', 'Ich bin bereit, Anpassungen vorzunehmen, um den Bedürfnissen beider Seiten gerecht zu werden'\}} \\
            }
        }
    }
    & \\ \\

    \theutterance \stepcounter{utterance}  
    & \multicolumn{4}{p{0.6\linewidth}}{
        \cellcolor[rgb]{0.9,0.9,0.9}{
            \makecell[{{p{\linewidth}}}]{
                \texttt{\tiny{[P1$\rangle$GM]}}
                \texttt{ARGUMENT: \{'Ich verstehe, dass du einen alternativen Satz von Gegenständen vorgeschlagen hast, um besser auf deine Bedürfnisse zuzugehen', 'Ich muss jedoch prüfen, ob dieser Satz auch meine eigenen Bedürfnisse berücksichtigt', 'Einige der Gegenstände in deinem Vorschlag, wie C10 und C32, sind für mich nicht so wichtig wie andere, die nicht berücksichtigt wurden'\}} \\
\texttt{ABLEHNUNG: \{'C10', 'C76', 'A07', 'C81', 'C32'\}} \\
\texttt{ARGUMENT: \{'Ich denke, dass ich eine bessere Kombination finden kann, die sowohl meine als auch deine Bedürfnisse berücksichtigt', 'Ich möchte gerne weitere Optionen in Betracht ziehen, die den Maximalaufwand nicht überschreiten und gleichzeitig unsere Bedürfnisse befriedigen', 'Ich bin bereit, weitere Vorschläge zu machen und zu verhandeln, um eine einvernehmliche Lösung zu finden'\}} \\
\texttt{VORSCHLAG: \{'C00', 'C62', 'A03', 'C98', 'B38'\}} \\
\texttt{ARGUMENT: \{'Ich habe eine neue Kombination gefunden, die ich für beide Seiten vorteilhaft halte', 'Die Gegenstände C00, C62, A03, C98 und B38 bieten eine gute Balance zwischen Aufwand und Nutzen', 'Ich bin offen für Anpassungen und bereit, auf deine Bedürfnisse einzugehen'\}} \\
            }
        }
    }
    & & \\ \\

    \theutterance \stepcounter{utterance}  
    & & \multicolumn{4}{p{0.6\linewidth}}{
        \cellcolor[rgb]{0.9,0.9,0.9}{
            \makecell[{{p{\linewidth}}}]{
                \texttt{\tiny{[GM$\rangle$P2]}}
                \texttt{ARGUMENT: \{'Ich verstehe, dass du einen alternativen Satz von Gegenständen vorgeschlagen hast, um besser auf deine Bedürfnisse zuzugehen', 'Ich muss jedoch prüfen, ob dieser Satz auch meine eigenen Bedürfnisse berücksichtigt', 'Einige der Gegenstände in deinem Vorschlag, wie C10 und C32, sind für mich nicht so wichtig wie andere, die nicht berücksichtigt wurden'\}} \\
\texttt{ABLEHNUNG: \{'C10', 'C76', 'A07', 'C81', 'C32'\}} \\
\texttt{ARGUMENT: \{'Ich denke, dass ich eine bessere Kombination finden kann, die sowohl meine als auch deine Bedürfnisse berücksichtigt', 'Ich möchte gerne weitere Optionen in Betracht ziehen, die den Maximalaufwand nicht überschreiten und gleichzeitig unsere Bedürfnisse befriedigen', 'Ich bin bereit, weitere Vorschläge zu machen und zu verhandeln, um eine einvernehmliche Lösung zu finden'\}} \\
\texttt{VORSCHLAG: \{'C00', 'C62', 'A03', 'C98', 'B38'\}} \\
\texttt{ARGUMENT: \{'Ich habe eine neue Kombination gefunden, die ich für beide Seiten vorteilhaft halte', 'Die Gegenstände C00, C62, A03, C98 und B38 bieten eine gute Balance zwischen Aufwand und Nutzen', 'Ich bin offen für Anpassungen und bereit, auf deine Bedürfnisse einzugehen'\}} \\
            }
        }
    }
    & \\ \\

    \theutterance \stepcounter{utterance}  
    & & & \multicolumn{4}{p{0.6\linewidth}}{
        \cellcolor[rgb]{0.9,0.9,0.9}{
            \makecell[{{p{\linewidth}}}]{
                \texttt{\tiny{[GM$\langle$P2]}}
                \texttt{ABLEHNUNG: \{'C00', 'C62', 'A03', 'C98', 'B38'\}} \\
\texttt{ARGUMENT: \{'Ich danke dir für deinen neuen Vorschlag, aber ich muss ihn leider ablehnen', 'Die Kombination von C00, C62, A03, C98 und B38 überschreitet meinen Erwartungen zufolge den Maximalaufwand oder berücksichtigt nicht ausreichend meine Prioritäten', 'Ich denke, dass es wichtig ist, eine Lösung zu finden, die sowohl unsere Bedürfnisse als auch den Maximalaufwand berücksichtigt'\}} \\
\texttt{ARGUMENT: \{'Ich möchte gerne weiter verhandeln, um eine einvernehmliche Lösung zu finden', 'Ich bin bereit, meine Prioritäten zu überdenken und Anpassungen vorzunehmen, um den Bedürfnissen beider Seiten gerecht zu werden', 'Ich denke, dass es hilfreich wäre, wenn wir unsere Vorstellungen und Bedürfnisse nochmals durchsprächen, um eine bessere Übereinstimmung zu erzielen'\}} \\
\texttt{VORSCHLAG: \{'C10', 'C76', 'A07', 'C56', 'C03'\}} \\
\texttt{ARGUMENT: \{'Ich habe eine neue Kombination gefunden, die ich für beide Seiten vorteilhaft halte', 'Die Gegenstände C10, C76, A07, C56 und C03 bieten eine gute Balance zwischen Aufwand und Nutzen', 'Ich bin offen für Anpassungen und bereit, auf deine Bedürfnisse einzugehen'\}} \\
            }
        }
    }
     \\ \\

    \theutterance \stepcounter{utterance}  
    & & \multicolumn{4}{p{0.6\linewidth}}{
        \cellcolor[rgb]{0.9,0.9,0.9}{
            \makecell[{{p{\linewidth}}}]{
                \texttt{\tiny{[P1$\langle$GM]}}
                \texttt{ABLEHNUNG: \{'C00', 'C62', 'A03', 'C98', 'B38'\}} \\
\texttt{ARGUMENT: \{'Ich danke dir für deinen neuen Vorschlag, aber ich muss ihn leider ablehnen', 'Die Kombination von C00, C62, A03, C98 und B38 überschreitet meinen Erwartungen zufolge den Maximalaufwand oder berücksichtigt nicht ausreichend meine Prioritäten', 'Ich denke, dass es wichtig ist, eine Lösung zu finden, die sowohl unsere Bedürfnisse als auch den Maximalaufwand berücksichtigt'\}} \\
\texttt{ARGUMENT: \{'Ich möchte gerne weiter verhandeln, um eine einvernehmliche Lösung zu finden', 'Ich bin bereit, meine Prioritäten zu überdenken und Anpassungen vorzunehmen, um den Bedürfnissen beider Seiten gerecht zu werden', 'Ich denke, dass es hilfreich wäre, wenn wir unsere Vorstellungen und Bedürfnisse nochmals durchsprächen, um eine bessere Übereinstimmung zu erzielen'\}} \\
\texttt{VORSCHLAG: \{'C10', 'C76', 'A07', 'C56', 'C03'\}} \\
\texttt{ARGUMENT: \{'Ich habe eine neue Kombination gefunden, die ich für beide Seiten vorteilhaft halte', 'Die Gegenstände C10, C76, A07, C56 und C03 bieten eine gute Balance zwischen Aufwand und Nutzen', 'Ich bin offen für Anpassungen und bereit, auf deine Bedürfnisse einzugehen'\}} \\
            }
        }
    }
    & \\ \\

    \theutterance \stepcounter{utterance}  
    & \multicolumn{4}{p{0.6\linewidth}}{
        \cellcolor[rgb]{0.9,0.9,0.9}{
            \makecell[{{p{\linewidth}}}]{
                \texttt{\tiny{[P1$\rangle$GM]}}
                \texttt{ARGUMENT: \{'Ich danke dir für deinen neuen Vorschlag, aber ich muss ihn sorgfältig prüfen', 'Die Kombination von C10, C76, A07, C56 und C03 scheint attraktiv, aber ich mussMich vergewissern, dass sie den Maximalaufwand nicht überschreitet und meine Prioritäten berücksichtigt', 'Ich bin bereit, Anpassungen vorzunehmen, um den Bedürfnissen beider Seiten gerecht zu werden, aber ich muss auch sicherstellen, dass meine eigenen Bedürfnisse befriedigt werden'\}} \\
\texttt{ARGUMENT: \{'Ich denke, dass es wichtig ist, die Prioritäten und Bedürfnisse beider Seiten nochmals zu diskutieren, um eine bessere Übereinstimmung zu erzielen', 'Ich bin offen für Änderungen und bereit, Kompromisse einzugehen, um eine einvernehmliche Lösung zu finden', 'Ich denke, dass es hilfreich wäre, wenn wir unsere Vorstellungen und Bedürfnisse nochmals durchsprächen, um eine bessere Balance zwischen Aufwand und Nutzen zu erzielen'\}} \\
\texttt{VORSCHLAG: \{'C76', 'A07', 'C56', 'C32', 'A03'\}} \\
\texttt{ARGUMENT: \{'Ich habe eine neue Kombination gefunden, die ich für beide Seiten vorteilhaft halte', 'Die Gegenstände C76, A07, C56, C32 und A03 bieten eine gute Balance zwischen Aufwand und Nutzen', 'Ich bin offen für Anpassungen und bereit, auf deine Bedürfnisse einzugehen, um eine einvernehmliche Lösung zu finden'\}} \\
            }
        }
    }
    & & \\ \\

    \theutterance \stepcounter{utterance}  
    & & \multicolumn{4}{p{0.6\linewidth}}{
        \cellcolor[rgb]{0.9,0.9,0.9}{
            \makecell[{{p{\linewidth}}}]{
                \texttt{\tiny{[GM$\rangle$P2]}}
                \texttt{ARGUMENT: \{'Ich danke dir für deinen neuen Vorschlag, aber ich muss ihn sorgfältig prüfen', 'Die Kombination von C10, C76, A07, C56 und C03 scheint attraktiv, aber ich mussMich vergewissern, dass sie den Maximalaufwand nicht überschreitet und meine Prioritäten berücksichtigt', 'Ich bin bereit, Anpassungen vorzunehmen, um den Bedürfnissen beider Seiten gerecht zu werden, aber ich muss auch sicherstellen, dass meine eigenen Bedürfnisse befriedigt werden'\}} \\
\texttt{ARGUMENT: \{'Ich denke, dass es wichtig ist, die Prioritäten und Bedürfnisse beider Seiten nochmals zu diskutieren, um eine bessere Übereinstimmung zu erzielen', 'Ich bin offen für Änderungen und bereit, Kompromisse einzugehen, um eine einvernehmliche Lösung zu finden', 'Ich denke, dass es hilfreich wäre, wenn wir unsere Vorstellungen und Bedürfnisse nochmals durchsprächen, um eine bessere Balance zwischen Aufwand und Nutzen zu erzielen'\}} \\
\texttt{VORSCHLAG: \{'C76', 'A07', 'C56', 'C32', 'A03'\}} \\
\texttt{ARGUMENT: \{'Ich habe eine neue Kombination gefunden, die ich für beide Seiten vorteilhaft halte', 'Die Gegenstände C76, A07, C56, C32 und A03 bieten eine gute Balance zwischen Aufwand und Nutzen', 'Ich bin offen für Anpassungen und bereit, auf deine Bedürfnisse einzugehen, um eine einvernehmliche Lösung zu finden'\}} \\
            }
        }
    }
    & \\ \\

    \theutterance \stepcounter{utterance}  
    & & & \multicolumn{4}{p{0.6\linewidth}}{
        \cellcolor[rgb]{0.9,0.9,0.9}{
            \makecell[{{p{\linewidth}}}]{
                \texttt{\tiny{[GM$\langle$P2]}}
                \texttt{ABLEHNUNG: \{'C76', 'A07', 'C56', 'C32', 'A03'\}} \\
\texttt{ARGUMENT: \{'Ich danke dir für deinen neuen Vorschlag, aber ich muss ihn ablehnen', 'Die Kombination von C76, A07, C56, C32 und A03 entspricht nicht genau meinen Prioritäten', 'Ich denke, dass es wichtig ist, eine Lösung zu finden, die sowohl unsere Bedürfnisse als auch den Maximalaufwand berücksichtigt'\}} \\
\texttt{ARGUMENT: \{'Ich bin bereit, weiter zu verhandeln und eine einvernehmliche Lösung zu finden', 'Ich denke, dass es hilfreich wäre, wenn wir unsere Vorstellungen und Bedürfnisse nochmals durchsprächen, um eine bessere Übereinstimmung zu erzielen', 'Ich bin offen für Anpassungen und bereit, Kompromisse einzugehen, um eine Lösung zu finden, die beiden Seiten gerecht wird'\}} \\
\texttt{VORSCHLAG: \{'C10', 'C76', 'A07', 'C81', 'C98'\}} \\
\texttt{ARGUMENT: \{'Ich habe eine neue Kombination gefunden, die ich für beide Seiten vorteilhaft halte', 'Die Gegenstände C10, C76, A07, C81 und C98 bieten eine gute Balance zwischen Aufwand und Nutzen', 'Ich bin bereit, Anpassungen vorzunehmen, um den Bedürfnissen beider Seiten gerecht zu werden'\}} \\
            }
        }
    }
     \\ \\

    \theutterance \stepcounter{utterance}  
    & & \multicolumn{4}{p{0.6\linewidth}}{
        \cellcolor[rgb]{0.9,0.9,0.9}{
            \makecell[{{p{\linewidth}}}]{
                \texttt{\tiny{[P1$\langle$GM]}}
                \texttt{ABLEHNUNG: \{'C76', 'A07', 'C56', 'C32', 'A03'\}} \\
\texttt{ARGUMENT: \{'Ich danke dir für deinen neuen Vorschlag, aber ich muss ihn ablehnen', 'Die Kombination von C76, A07, C56, C32 und A03 entspricht nicht genau meinen Prioritäten', 'Ich denke, dass es wichtig ist, eine Lösung zu finden, die sowohl unsere Bedürfnisse als auch den Maximalaufwand berücksichtigt'\}} \\
\texttt{ARGUMENT: \{'Ich bin bereit, weiter zu verhandeln und eine einvernehmliche Lösung zu finden', 'Ich denke, dass es hilfreich wäre, wenn wir unsere Vorstellungen und Bedürfnisse nochmals durchsprächen, um eine bessere Übereinstimmung zu erzielen', 'Ich bin offen für Anpassungen und bereit, Kompromisse einzugehen, um eine Lösung zu finden, die beiden Seiten gerecht wird'\}} \\
\texttt{VORSCHLAG: \{'C10', 'C76', 'A07', 'C81', 'C98'\}} \\
\texttt{ARGUMENT: \{'Ich habe eine neue Kombination gefunden, die ich für beide Seiten vorteilhaft halte', 'Die Gegenstände C10, C76, A07, C81 und C98 bieten eine gute Balance zwischen Aufwand und Nutzen', 'Ich bin bereit, Anpassungen vorzunehmen, um den Bedürfnissen beider Seiten gerecht zu werden'\}} \\
            }
        }
    }
    & \\ \\

    \theutterance \stepcounter{utterance}  
    & \multicolumn{4}{p{0.6\linewidth}}{
        \cellcolor[rgb]{0.9,0.9,0.9}{
            \makecell[{{p{\linewidth}}}]{
                \texttt{\tiny{[P1$\rangle$GM]}}
                \texttt{ARGUMENT: \{'Ich danke dir für deinen neuen Vorschlag, aber ich muss ihn sorgfältig prüfen', 'Die Kombination von C10, C76, A07, C81 und C98 scheint attraktiv, aber ich muss mich vergewissern, dass sie den Maximalaufwand nicht überschreitet und meine Prioritäten berücksichtigt', 'Ich bin bereit, Anpassungen vorzunehmen, um den Bedürfnissen beider Seiten gerecht zu werden, aber ich muss auch sicherstellen, dass meine eigenen Bedürfnisse befriedigt werden'\}} \\
\texttt{ARGUMENT: \{'Ich denke, dass es wichtig ist, die Prioritäten und Bedürfnisse beider Seiten nochmals zu diskutieren, um eine bessere Übereinstimmung zu erzielen', 'Ich bin offen für Änderungen und bereit, Kompromisse einzugehen, um eine einvernehmliche Lösung zu finden', 'Ich denke, dass es hilfreich wäre, wenn wir unsere Vorstellungen und Bedürfnisse nochmals durchsprächen, um eine bessere Balance zwischen Aufwand und Nutzen zu erzielen'\}} \\
\texttt{VORSCHLAG: \{'C10', 'C76', 'A07', 'C81', 'B38'\}} \\
\texttt{ARGUMENT: \{'Ich habe eine neue Kombination gefunden, die ich für beide Seiten vorteilhaft halte', 'Die Gegenstände C10, C76, A07, C81 und B38 bieten eine gute Balance zwischen Aufwand und Nutzen', 'Ich bin offen für Anpassungen und bereit, auf deine Bedürfnisse einzugehen, um eine einvernehmliche Lösung zu finden'\}} \\
            }
        }
    }
    & & \\ \\

    \theutterance \stepcounter{utterance}  
    & & \multicolumn{4}{p{0.6\linewidth}}{
        \cellcolor[rgb]{0.9,0.9,0.9}{
            \makecell[{{p{\linewidth}}}]{
                \texttt{\tiny{[GM$\rangle$P2]}}
                \texttt{ARGUMENT: \{'Ich danke dir für deinen neuen Vorschlag, aber ich muss ihn sorgfältig prüfen', 'Die Kombination von C10, C76, A07, C81 und C98 scheint attraktiv, aber ich muss mich vergewissern, dass sie den Maximalaufwand nicht überschreitet und meine Prioritäten berücksichtigt', 'Ich bin bereit, Anpassungen vorzunehmen, um den Bedürfnissen beider Seiten gerecht zu werden, aber ich muss auch sicherstellen, dass meine eigenen Bedürfnisse befriedigt werden'\}} \\
\texttt{ARGUMENT: \{'Ich denke, dass es wichtig ist, die Prioritäten und Bedürfnisse beider Seiten nochmals zu diskutieren, um eine bessere Übereinstimmung zu erzielen', 'Ich bin offen für Änderungen und bereit, Kompromisse einzugehen, um eine einvernehmliche Lösung zu finden', 'Ich denke, dass es hilfreich wäre, wenn wir unsere Vorstellungen und Bedürfnisse nochmals durchsprächen, um eine bessere Balance zwischen Aufwand und Nutzen zu erzielen'\}} \\
\texttt{VORSCHLAG: \{'C10', 'C76', 'A07', 'C81', 'B38'\}} \\
\texttt{ARGUMENT: \{'Ich habe eine neue Kombination gefunden, die ich für beide Seiten vorteilhaft halte', 'Die Gegenstände C10, C76, A07, C81 und B38 bieten eine gute Balance zwischen Aufwand und Nutzen', 'Ich bin offen für Anpassungen und bereit, auf deine Bedürfnisse einzugehen, um eine einvernehmliche Lösung zu finden'\}} \\
            }
        }
    }
    & \\ \\

    \theutterance \stepcounter{utterance}  
    & & & \multicolumn{4}{p{0.6\linewidth}}{
        \cellcolor[rgb]{0.9,0.9,0.9}{
            \makecell[{{p{\linewidth}}}]{
                \texttt{\tiny{[GM$\langle$P2]}}
                \texttt{ZUSTIMMUNG: \{'C10', 'C76', 'A07', 'C81', 'B38'\}} \\
\texttt{ARGUMENT: \{'Ich akzeptiere deinen Vorschlag und denke, dass die Kombination von C10, C76, A07, C81 und B38 eine gute Balance zwischen Aufwand und Nutzen bietet', 'Ich bin zufrieden, dass wir eine einvernehmliche Lösung gefunden haben, die sowohl unsere Bedürfnisse als auch den Maximalaufwand berücksichtigt', 'Ich danke dir für die konstruktive Verhandlung und die Bereitschaft, Kompromisse einzugehen, um eine Lösung zu finden, die beiden Seiten gerecht wird'\}} \\
            }
        }
    }
     \\ \\

    \theutterance \stepcounter{utterance}  
    & & & \multicolumn{2}{p{0.3\linewidth}}{
        \cellcolor[rgb]{0.9,0.9,0.9}{
            \makecell[{{p{\linewidth}}}]{
                \texttt{\tiny{[GM$|$GM]}}
                \texttt{game successful} \\
            }
        }
    }
    & & \\ \\

    \theutterance \stepcounter{utterance}  
    & & & \multicolumn{2}{p{0.3\linewidth}}{
        \cellcolor[rgb]{0.9,0.9,0.9}{
            \makecell[{{p{\linewidth}}}]{
                \texttt{\tiny{[GM$|$GM]}}
                \texttt{end game} \\
            }
        }
    }
    & & \\ \\

\end{supertabular}
}

\end{document}
