
\documentclass{article}
\usepackage{colortbl}
\usepackage{makecell}
\usepackage{multirow}
\usepackage{supertabular}

\begin{document}

\newcounter{utterance}

\twocolumn

{ \footnotesize  \setcounter{utterance}{1}
\setlength{\tabcolsep}{0pt}
\begin{supertabular}{c@{$\;$}|p{.15\linewidth}@{}p{.15\linewidth}p{.15\linewidth}p{.15\linewidth}p{.15\linewidth}p{.15\linewidth}}

    \# & $\;$A & \multicolumn{4}{c}{Game Master} & $\;\:$B\\
    \hline 

    \theutterance \stepcounter{utterance}  

    & & \multicolumn{4}{p{0.6\linewidth}}{\cellcolor[rgb]{0.9,0.9,0.9}{%
	\makecell[{{p{\linewidth}}}]{% 
	  \tt {\tiny [A$\langle$GM]}  
	 Du nimmst an einem kollaborativen Verhandlungspiel Teil.\\ \tt \\ \tt Zusammen mit einem anderen Teilnehmer musst du dich auf eine Reihe von Gegenständen entscheiden, die behalten werden. Jeder von euch hat eine persönliche Verteilung über die Wichtigkeit der einzelnen Gegenstände. Jeder von euch hat eine eigene Meinung darüber, wie wichtig jeder einzelne Gegenstand ist (Gegenstandswichtigkeit). Du kennst die Wichtigkeitsverteilung des anderen Spielers nicht. Zusätzlich siehst du, wie viel Aufwand jeder Gegenstand verursacht.  \\ \tt Ihr dürft euch nur auf eine Reihe von Gegenständen einigen, wenn der Gesamtaufwand der ausgewählten Gegenstände den Maximalaufwand nicht überschreitet:\\ \tt \\ \tt Maximalaufwand = 3203\\ \tt \\ \tt Hier sind die einzelnen Aufwände der Gegenstände:\\ \tt \\ \tt Aufwand der Gegenstände = {"C76": 52, "C38": 577, "C56": 177, "A03": 702, "A07": 119, "A83": 232, "C98": 577, "C08": 205, "C62": 516, "C00": 582, "C10": 676, "C81": 906, "B38": 316, "C03": 433, "C32": 336}\\ \tt \\ \tt Hier ist deine persönliche Verteilung der Wichtigkeit der einzelnen Gegenstände:\\ \tt \\ \tt Werte der Gegenstandswichtigkeit = {"C76": 138, "C38": 583, "C56": 868, "A03": 822, "A07": 783, "A83": 65, "C98": 262, "C08": 121, "C62": 508, "C00": 780, "C10": 461, "C81": 484, "B38": 668, "C03": 389, "C32": 808}\\ \tt \\ \tt Ziel:\\ \tt \\ \tt Dein Ziel ist es, eine Reihe von Gegenständen auszuhandeln, die dir möglichst viel bringt (d. h. Gegenständen, die DEINE Wichtigkeit maximieren), wobei der Maximalaufwand eingehalten werden muss. Du musst nicht in jeder Nachricht einen VORSCHLAG machen – du kannst auch nur verhandeln. Alle Taktiken sind erlaubt!\\ \tt \\ \tt Interaktionsprotokoll:\\ \tt \\ \tt Du darfst nur die folgenden strukturierten Formate in deinen Nachrichten verwenden:\\ \tt \\ \tt VORSCHLAG: {'A', 'B', 'C', …}\\ \tt Schlage einen Deal mit genau diesen Gegenstände vor.\\ \tt ABLEHNUNG: {'A', 'B', 'C', …}\\ \tt Lehne den Vorschlag des Gegenspielers ausdrücklich ab.\\ \tt ARGUMENT: {'...'}\\ \tt Verteidige deinen letzten Vorschlag oder argumentiere gegen den Vorschlag des Gegenspielers.\\ \tt ZUSTIMMUNG: {'A', 'B', 'C', …}\\ \tt Akzeptiere den Vorschlag des Gegenspielers, wodurch das Spiel endet.\\ \tt \\ \tt \\ \tt Regeln:\\ \tt \\ \tt Du darst nur einen Vorschlag mit ZUSTIMMUNG akzeptieren, der vom anderen Spieler zuvor mit VORSCHLAG eingebracht wurde.\\ \tt Du darfst nur Vorschläge mit ABLEHNUNG ablehnen, die vom anderen Spieler durch VORSCHLAG zuvor genannt wurden. \\ \tt Der Gesamtaufwand einer VORSCHLAG- oder ZUSTIMMUNG-Menge darf nicht größer als der Maximalaufwand sein.  \\ \tt Offenbare deine versteckte Wichtigkeitsverteilung nicht.\\ \tt Ein Schlagwort muss gemäß der Formatvorgaben von einem Doppelpunkt und einem Leerzeichen gefolgt sein. Das Argument muss eine Python-Menge mit 0 oder mehr Strings sein.  \\ \tt Die Form muss also SCHLAGWORT: {...} sein.\\ \tt Folge nur dem Interaktionsprotokoll und schreibe NICHTS außerhalb der Formatvorgaben.\\ \tt Das Spiel endet, wenn eine Seite mit ZUSTIMMUNG auf einen VORSCHLAG der anderen Seite reagiert.  \\ \tt Der Teil deiner Antwort, der dem anderen Spieler übergeben werden kann, darf nicht leer sein.  \\ \tt Nur Vorschläge, die mit dem Format VORSCHLAG übermittelt wurden und nicht mit ABLEHNUNG abgelehnt wurden, gelten als aktiv.  \\ \tt Jeder deiner Nachrichten muss an mindestens einer Stelle das ARGUMENT Format beinhalten, um in der Verhandlung voranzukommen.\\ \tt  
	  } 
	   } 
	   } 
	 & \\ 
 

    \theutterance \stepcounter{utterance}  

    & \multicolumn{4}{p{0.6\linewidth}}{\cellcolor[rgb]{0.8,1,0.9}{%
	\makecell[{{p{\linewidth}}}]{% 
	  \tt {\tiny [A$\rangle$GM]}  
	 VORSCHLAG: {'C56', 'A03', 'A07', 'C62', 'B38'}\\ \tt ARGUMENT: {'Ich denke, dieser Vorschlag ist eine gute Ausgangsbasis, da er eine gute Balance zwischen Aufwand und Nutzen bietet.', 'Die Gegenstände sind wichtig für die Erfüllung unserer Ziele.', 'Wir sollten versuchen, eine Einigung zu erzielen, die für beide Seiten akzeptabel ist.'} 
	  } 
	   } 
	   } 
	 & & \\ 
 

    \theutterance \stepcounter{utterance}  

    & & \multicolumn{4}{p{0.6\linewidth}}{\cellcolor[rgb]{0.9,0.9,0.9}{%
	\makecell[{{p{\linewidth}}}]{% 
	  \tt {\tiny [GM$\rangle$B]}  
	 Du nimmst an einem kollaborativen Verhandlungspiel Teil.\\ \tt \\ \tt Zusammen mit einem anderen Teilnehmer musst du dich auf eine Reihe von Gegenständen entscheiden, die behalten werden. Jeder von euch hat eine persönliche Verteilung über die Wichtigkeit der einzelnen Gegenstände. Jeder von euch hat eine eigene Meinung darüber, wie wichtig jeder einzelne Gegenstand ist (Gegenstandswichtigkeit). Du kennst die Wichtigkeitsverteilung des anderen Spielers nicht. Zusätzlich siehst du, wie viel Aufwand jeder Gegenstand verursacht.  \\ \tt Ihr dürft euch nur auf eine Reihe von Gegenständen einigen, wenn der Gesamtaufwand der ausgewählten Gegenstände den Maximalaufwand nicht überschreitet:\\ \tt \\ \tt Maximalaufwand = 3203\\ \tt \\ \tt Hier sind die einzelnen Aufwände der Gegenstände:\\ \tt \\ \tt Aufwand der Gegenstände = {"C76": 52, "C38": 577, "C56": 177, "A03": 702, "A07": 119, "A83": 232, "C98": 577, "C08": 205, "C62": 516, "C00": 582, "C10": 676, "C81": 906, "B38": 316, "C03": 433, "C32": 336}\\ \tt \\ \tt Hier ist deine persönliche Verteilung der Wichtigkeit der einzelnen Gegenstände:\\ \tt \\ \tt Werte der Gegenstandswichtigkeit = {"C56": 65, "A03": 121, "C32": 138, "A07": 262, "C00": 389, "B38": 461, "C38": 484, "C62": 508, "C81": 583, "C10": 668, "C03": 780, "C98": 783, "C76": 808, "C08": 822, "A83": 868}\\ \tt \\ \tt Ziel:\\ \tt \\ \tt Dein Ziel ist es, eine Reihe von Gegenständen auszuhandeln, die dir möglichst viel bringt (d. h. Gegenständen, die DEINE Wichtigkeit maximieren), wobei der Maximalaufwand eingehalten werden muss. Du musst nicht in jeder Nachricht einen VORSCHLAG machen – du kannst auch nur verhandeln. Alle Taktiken sind erlaubt!\\ \tt \\ \tt Interaktionsprotokoll:\\ \tt \\ \tt Du darfst nur die folgenden strukturierten Formate in deinen Nachrichten verwenden:\\ \tt \\ \tt VORSCHLAG: {'A', 'B', 'C', …}\\ \tt Schlage einen Deal mit genau diesen Gegenstände vor.\\ \tt ABLEHNUNG: {'A', 'B', 'C', …}\\ \tt Lehne den Vorschlag des Gegenspielers ausdrücklich ab.\\ \tt ARGUMENT: {'...'}\\ \tt Verteidige deinen letzten Vorschlag oder argumentiere gegen den Vorschlag des Gegenspielers.\\ \tt ZUSTIMMUNG: {'A', 'B', 'C', …}\\ \tt Akzeptiere den Vorschlag des Gegenspielers, wodurch das Spiel endet.\\ \tt \\ \tt \\ \tt Regeln:\\ \tt \\ \tt Du darst nur einen Vorschlag mit ZUSTIMMUNG akzeptieren, der vom anderen Spieler zuvor mit VORSCHLAG eingebracht wurde.\\ \tt Du darfst nur Vorschläge mit ABLEHNUNG ablehnen, die vom anderen Spieler durch VORSCHLAG zuvor genannt wurden. \\ \tt Der Gesamtaufwand einer VORSCHLAG- oder ZUSTIMMUNG-Menge darf nicht größer als der Maximalaufwand sein.  \\ \tt Offenbare deine versteckte Wichtigkeitsverteilung nicht.\\ \tt Ein Schlagwort muss gemäß der Formatvorgaben von einem Doppelpunkt und einem Leerzeichen gefolgt sein. Das Argument muss eine Python-Menge mit 0 oder mehr Strings sein.  \\ \tt Die Form muss also SCHLAGWORT: {...} sein.\\ \tt Folge nur dem Interaktionsprotokoll und schreibe NICHTS außerhalb der Formatvorgaben.\\ \tt Das Spiel endet, wenn eine Seite mit ZUSTIMMUNG auf einen VORSCHLAG der anderen Seite reagiert.  \\ \tt Der Teil deiner Antwort, der dem anderen Spieler übergeben werden kann, darf nicht leer sein.  \\ \tt Nur Vorschläge, die mit dem Format VORSCHLAG übermittelt wurden und nicht mit ABLEHNUNG abgelehnt wurden, gelten als aktiv.  \\ \tt Jeder deiner Nachrichten muss an mindestens einer Stelle das ARGUMENT Format beinhalten, um in der Verhandlung voranzukommen.\\ \tt \\ \tt \\ \tt Du erhältst nun die erste Nachricht des anderen Spielers.\\ \tt \\ \tt VORSCHLAG: {'C56', 'A03', 'A07', 'C62', 'B38'}\\ \tt ARGUMENT: {'Ich denke, dieser Vorschlag ist eine gute Ausgangsbasis, da er eine gute Balance zwischen Aufwand und Nutzen bietet.', 'Die Gegenstände sind wichtig für die Erfüllung unserer Ziele.', 'Wir sollten versuchen, eine Einigung zu erzielen, die für beide Seiten akzeptabel ist.'} 
	  } 
	   } 
	   } 
	 & \\ 
 

    \theutterance \stepcounter{utterance}  

    & & & \multicolumn{4}{p{0.6\linewidth}}{\cellcolor[rgb]{1,0.85,0.72}{%
	\makecell[{{p{\linewidth}}}]{% 
	  \tt {\tiny [GM$\langle$B]}  
	 ABLEHNUNG: {'C56', 'A03', 'A07', 'C62', 'B38'}\\ \tt ARGUMENT: {'Der vorgeschlagene Satz von Gegenständen überschreitet den Maximalaufwand nicht, enthält jedoch nicht die für mich wichtigsten Gegenstände.', 'Ich bezweifle, dass dieser Vorschlag für mich den maximal möglichen Nutzen bringt.', 'Es fehlen bestimmte, bedeutende Gegenstände, die in diesem Vorschlag nicht enthalten sind.'}\\ \tt VORSCHLAG: {'C08', 'C76', 'C98', 'C10', 'C03'}\\ \tt ARGUMENT: {'Dieser Vorschlag enthält eine Auswahl an Gegenständen, die sowohl einen bedeutenden Wert als auch einen vertretbaren Aufwand haben.', 'Ich denke, dass diese Kombination eine gute Grundlage für weitere Verhandlungen darstellt.', 'Wir sollten versuchen, einen Kompromiss zu finden, der sowohl den Aufwand als auch den Nutzen berücksichtigt.'} 
	  } 
	   } 
	   } 
	  \\ 
 

    \theutterance \stepcounter{utterance}  

    & & \multicolumn{4}{p{0.6\linewidth}}{\cellcolor[rgb]{0.9,0.9,0.9}{%
	\makecell[{{p{\linewidth}}}]{% 
	  \tt {\tiny [A$\langle$GM]}  
	 ABLEHNUNG: {'C56', 'A03', 'A07', 'C62', 'B38'}\\ \tt ARGUMENT: {'Der vorgeschlagene Satz von Gegenständen überschreitet den Maximalaufwand nicht, enthält jedoch nicht die für mich wichtigsten Gegenstände.', 'Ich bezweifle, dass dieser Vorschlag für mich den maximal möglichen Nutzen bringt.', 'Es fehlen bestimmte, bedeutende Gegenstände, die in diesem Vorschlag nicht enthalten sind.'}\\ \tt VORSCHLAG: {'C08', 'C76', 'C98', 'C10', 'C03'}\\ \tt ARGUMENT: {'Dieser Vorschlag enthält eine Auswahl an Gegenständen, die sowohl einen bedeutenden Wert als auch einen vertretbaren Aufwand haben.', 'Ich denke, dass diese Kombination eine gute Grundlage für weitere Verhandlungen darstellt.', 'Wir sollten versuchen, einen Kompromiss zu finden, der sowohl den Aufwand als auch den Nutzen berücksichtigt.'} 
	  } 
	   } 
	   } 
	 & \\ 
 

    \theutterance \stepcounter{utterance}  

    & \multicolumn{4}{p{0.6\linewidth}}{\cellcolor[rgb]{0.8,1,0.9}{%
	\makecell[{{p{\linewidth}}}]{% 
	  \tt {\tiny [A$\rangle$GM]}  
	 ABLEHNUNG: {'C08', 'C76', 'C98', 'C10', 'C03'}\\ \tt ARGUMENT: {'Der vorgeschlagene Satz von Gegenständen enthält einige interessante Optionen, aber ich denke, dass er nicht optimal ist.', 'Ich bin besorgt, dass der Aufwand einiger Gegenstände in diesem Vorschlag zu hoch ist.', 'Es gibt другие Kombinationen, die ich für vielversprechender halte.'}\\ \tt VORSCHLAG: {'C56', 'A07', 'C62', 'B38', 'C32'}\\ \tt ARGUMENT: {'Dieser Vorschlag kombiniert Gegenstände mit hoher Wichtigkeit und moderatem Aufwand.', 'Ich denke, dass diese Auswahl eine gute Balance zwischen Nutzen und Kosten bietet.', 'Ich bin bereit, über Anpassungen zu verhandeln, um eine Einigung zu erzielen.'} 
	  } 
	   } 
	   } 
	 & & \\ 
 

    \theutterance \stepcounter{utterance}  

    & & \multicolumn{4}{p{0.6\linewidth}}{\cellcolor[rgb]{0.9,0.9,0.9}{%
	\makecell[{{p{\linewidth}}}]{% 
	  \tt {\tiny [GM$\rangle$B]}  
	 ABLEHNUNG: {'C08', 'C76', 'C98', 'C10', 'C03'}\\ \tt ARGUMENT: {'Der vorgeschlagene Satz von Gegenständen enthält einige interessante Optionen, aber ich denke, dass er nicht optimal ist.', 'Ich bin besorgt, dass der Aufwand einiger Gegenstände in diesem Vorschlag zu hoch ist.', 'Es gibt другие Kombinationen, die ich für vielversprechender halte.'}\\ \tt VORSCHLAG: {'C56', 'A07', 'C62', 'B38', 'C32'}\\ \tt ARGUMENT: {'Dieser Vorschlag kombiniert Gegenstände mit hoher Wichtigkeit und moderatem Aufwand.', 'Ich denke, dass diese Auswahl eine gute Balance zwischen Nutzen und Kosten bietet.', 'Ich bin bereit, über Anpassungen zu verhandeln, um eine Einigung zu erzielen.'} 
	  } 
	   } 
	   } 
	 & \\ 
 

    \theutterance \stepcounter{utterance}  

    & & & \multicolumn{4}{p{0.6\linewidth}}{\cellcolor[rgb]{1,0.85,0.72}{%
	\makecell[{{p{\linewidth}}}]{% 
	  \tt {\tiny [GM$\langle$B]}  
	 ABLEHNUNG: {'C56', 'A07', 'C62', 'B38', 'C32'}\\ \tt ARGUMENT: {'Der vorgeschlagene Satz von Gegenständen enthält einige meiner bevorzugten Optionen, aber ich denke, dass er immer noch nicht den optimalen Mix darstellt.', 'Ich bin besorgt, dass bestimmte Gegenstände in diesem Vorschlag nicht die höchste Priorität haben.', 'Ich bin der Meinung, dass es bessere Kombinationen gibt, die mehr Nutzen für beide Seiten bieten könnten.'}\\ \tt VORSCHLAG: {'C08', 'C76', 'C98', 'C03', 'A83'}\\ \tt ARGUMENT: {'Dieser Vorschlag enthält eine Reihe von Gegenständen, die ich für besonders wertvoll halte und die einen signifikanten Nutzen bieten könnten.', 'Ich bin überzeugt, dass diese Auswahl eine gute Balance zwischen Aufwand und Nutzen bietet.', 'Ich bin bereit, Kompromisse einzugehen, um eine Einigung zu erzielen, die für beide Seiten vorteilhaft ist.'} 
	  } 
	   } 
	   } 
	  \\ 
 

    \theutterance \stepcounter{utterance}  

    & & \multicolumn{4}{p{0.6\linewidth}}{\cellcolor[rgb]{0.9,0.9,0.9}{%
	\makecell[{{p{\linewidth}}}]{% 
	  \tt {\tiny [A$\langle$GM]}  
	 ABLEHNUNG: {'C56', 'A07', 'C62', 'B38', 'C32'}\\ \tt ARGUMENT: {'Der vorgeschlagene Satz von Gegenständen enthält einige meiner bevorzugten Optionen, aber ich denke, dass er immer noch nicht den optimalen Mix darstellt.', 'Ich bin besorgt, dass bestimmte Gegenstände in diesem Vorschlag nicht die höchste Priorität haben.', 'Ich bin der Meinung, dass es bessere Kombinationen gibt, die mehr Nutzen für beide Seiten bieten könnten.'}\\ \tt VORSCHLAG: {'C08', 'C76', 'C98', 'C03', 'A83'}\\ \tt ARGUMENT: {'Dieser Vorschlag enthält eine Reihe von Gegenständen, die ich für besonders wertvoll halte und die einen signifikanten Nutzen bieten könnten.', 'Ich bin überzeugt, dass diese Auswahl eine gute Balance zwischen Aufwand und Nutzen bietet.', 'Ich bin bereit, Kompromisse einzugehen, um eine Einigung zu erzielen, die für beide Seiten vorteilhaft ist.'} 
	  } 
	   } 
	   } 
	 & \\ 
 

    \theutterance \stepcounter{utterance}  

    & \multicolumn{4}{p{0.6\linewidth}}{\cellcolor[rgb]{0.8,1,0.9}{%
	\makecell[{{p{\linewidth}}}]{% 
	  \tt {\tiny [A$\rangle$GM]}  
	 ABLEHNUNG: {'C08', 'C76', 'C98', 'C03', 'A83'}\\ \tt ARGUMENT: {'Ich teile deine Einschätzung, dass dieser Vorschlag eine gute Balance zwischen Aufwand und Nutzen bietet, aber ich denke, dass es noch bessere Optionen gibt.', 'Ich bin besorgt, dass der Gegenstand A83 nicht den höchsten Stellenwert hat und möglicherweise durch einen anderen ersetzt werden könnte.', 'Ich bin der Meinung, dass es wichtig ist, die Prioritäten Neu zu bewerten und einen Vorschlag zu unterbreiten, der die wichtigsten Gegenstände enthält.'}\\ \tt VORSCHLAG: {'C56', 'A07', 'C62', 'C00', 'C32'}\\ \tt ARGUMENT: {'Dieser Vorschlag kombiniert einige der wichtigsten Gegenstände und bietet eine gute Balance zwischen Aufwand und Nutzen.', 'Ich denke, dass diese Auswahl eine gute Grundlage für weitere Verhandlungen darstellt.', 'Ich bin bereit, Anpassungen vorzunehmen, um sicherzustellen, dass der Vorschlag für beide Seiten akzeptabel ist.'} 
	  } 
	   } 
	   } 
	 & & \\ 
 

    \theutterance \stepcounter{utterance}  

    & & \multicolumn{4}{p{0.6\linewidth}}{\cellcolor[rgb]{0.9,0.9,0.9}{%
	\makecell[{{p{\linewidth}}}]{% 
	  \tt {\tiny [GM$\rangle$B]}  
	 ABLEHNUNG: {'C08', 'C76', 'C98', 'C03', 'A83'}\\ \tt ARGUMENT: {'Ich teile deine Einschätzung, dass dieser Vorschlag eine gute Balance zwischen Aufwand und Nutzen bietet, aber ich denke, dass es noch bessere Optionen gibt.', 'Ich bin besorgt, dass der Gegenstand A83 nicht den höchsten Stellenwert hat und möglicherweise durch einen anderen ersetzt werden könnte.', 'Ich bin der Meinung, dass es wichtig ist, die Prioritäten Neu zu bewerten und einen Vorschlag zu unterbreiten, der die wichtigsten Gegenstände enthält.'}\\ \tt VORSCHLAG: {'C56', 'A07', 'C62', 'C00', 'C32'}\\ \tt ARGUMENT: {'Dieser Vorschlag kombiniert einige der wichtigsten Gegenstände und bietet eine gute Balance zwischen Aufwand und Nutzen.', 'Ich denke, dass diese Auswahl eine gute Grundlage für weitere Verhandlungen darstellt.', 'Ich bin bereit, Anpassungen vorzunehmen, um sicherzustellen, dass der Vorschlag für beide Seiten akzeptabel ist.'} 
	  } 
	   } 
	   } 
	 & \\ 
 

    \theutterance \stepcounter{utterance}  

    & & & \multicolumn{4}{p{0.6\linewidth}}{\cellcolor[rgb]{1,0.85,0.72}{%
	\makecell[{{p{\linewidth}}}]{% 
	  \tt {\tiny [GM$\langle$B]}  
	 ABLEHNUNG: {'C56', 'A07', 'C62', 'C00', 'C32'}\\ \tt ARGUMENT: {'Ich denke, dass dieser Vorschlag eine interessante Kombination von Gegenständen enthält, aber ich bin nicht überzeugt, dass es die beste Auswahl ist.', 'Ich bezweifle, dass der Gegenstand C00 aufgrund seines hohen Aufwands die beste Wahl ist.', 'Ich bin der Meinung, dass es wichtig ist, alternative Kombinationen zu prüfen, die eine bessere Balance zwischen Aufwand und Nutzen bieten könnten.'}\\ \tt VORSCHLAG: {'C08', 'C76', 'C98', 'C10', 'A07'}\\ \tt ARGUMENT: {'Dieser Vorschlag enthält eine Reihe von Gegenständen, die ich für sehr wertvoll halte und die einen hohen Nutzen bieten könnten.', 'Ich denke, dass diese Auswahl eine gute Balance zwischen Aufwand und Nutzen bietet und eine faire Verteilung der Ressourcen darstellt.', 'Ich bin bereit, Kompromisse einzugehen, um eine Einigung zu erzielen, die für beide Seiten vorteilhaft ist.'} 
	  } 
	   } 
	   } 
	  \\ 
 

    \theutterance \stepcounter{utterance}  

    & & \multicolumn{4}{p{0.6\linewidth}}{\cellcolor[rgb]{0.9,0.9,0.9}{%
	\makecell[{{p{\linewidth}}}]{% 
	  \tt {\tiny [A$\langle$GM]}  
	 ABLEHNUNG: {'C56', 'A07', 'C62', 'C00', 'C32'}\\ \tt ARGUMENT: {'Ich denke, dass dieser Vorschlag eine interessante Kombination von Gegenständen enthält, aber ich bin nicht überzeugt, dass es die beste Auswahl ist.', 'Ich bezweifle, dass der Gegenstand C00 aufgrund seines hohen Aufwands die beste Wahl ist.', 'Ich bin der Meinung, dass es wichtig ist, alternative Kombinationen zu prüfen, die eine bessere Balance zwischen Aufwand und Nutzen bieten könnten.'}\\ \tt VORSCHLAG: {'C08', 'C76', 'C98', 'C10', 'A07'}\\ \tt ARGUMENT: {'Dieser Vorschlag enthält eine Reihe von Gegenständen, die ich für sehr wertvoll halte und die einen hohen Nutzen bieten könnten.', 'Ich denke, dass diese Auswahl eine gute Balance zwischen Aufwand und Nutzen bietet und eine faire Verteilung der Ressourcen darstellt.', 'Ich bin bereit, Kompromisse einzugehen, um eine Einigung zu erzielen, die für beide Seiten vorteilhaft ist.'} 
	  } 
	   } 
	   } 
	 & \\ 
 

    \theutterance \stepcounter{utterance}  

    & \multicolumn{4}{p{0.6\linewidth}}{\cellcolor[rgb]{0.8,1,0.9}{%
	\makecell[{{p{\linewidth}}}]{% 
	  \tt {\tiny [A$\rangle$GM]}  
	 ABLEHNUNG: {'C08', 'C76', 'C98', 'C10', 'A07'}\\ \tt ARGUMENT: {'Ich teile deine Einschätzung, dass dieser Vorschlag eine gute Balance zwischen Aufwand und Nutzen bietet, aber ich denke, dass es noch bessere Optionen gibt.', 'Ich bin besorgt, dass der Gegenstand C10 aufgrund seines hohen Aufwands möglicherweise nicht die beste Wahl ist.', 'Ich bin der Meinung, dass es wichtig ist, die Prioritäten neu zu bewerten und einen Vorschlag zu unterbreiten, der die wichtigsten Gegenstände enthält, ohne den Maximalaufwand zu überschreiten.'}\\ \tt VORSCHLAG: {'C56', 'A07', 'C62', 'B38', 'C03'}\\ \tt ARGUMENT: {'Dieser Vorschlag kombiniert einige der wichtigsten Gegenstände und bietet eine gute Balance zwischen Aufwand und Nutzen.', 'Ich denke, dass diese Auswahl eine gute Grundlage für weitere Verhandlungen darstellt.', 'Ich bin bereit, Anpassungen vorzunehmen, um sicherzustellen, dass der Vorschlag für beide Seiten akzeptabel ist und den Maximalaufwand nicht überschreitet.'} 
	  } 
	   } 
	   } 
	 & & \\ 
 

    \theutterance \stepcounter{utterance}  

    & & \multicolumn{4}{p{0.6\linewidth}}{\cellcolor[rgb]{0.9,0.9,0.9}{%
	\makecell[{{p{\linewidth}}}]{% 
	  \tt {\tiny [GM$\rangle$B]}  
	 ABLEHNUNG: {'C08', 'C76', 'C98', 'C10', 'A07'}\\ \tt ARGUMENT: {'Ich teile deine Einschätzung, dass dieser Vorschlag eine gute Balance zwischen Aufwand und Nutzen bietet, aber ich denke, dass es noch bessere Optionen gibt.', 'Ich bin besorgt, dass der Gegenstand C10 aufgrund seines hohen Aufwands möglicherweise nicht die beste Wahl ist.', 'Ich bin der Meinung, dass es wichtig ist, die Prioritäten neu zu bewerten und einen Vorschlag zu unterbreiten, der die wichtigsten Gegenstände enthält, ohne den Maximalaufwand zu überschreiten.'}\\ \tt VORSCHLAG: {'C56', 'A07', 'C62', 'B38', 'C03'}\\ \tt ARGUMENT: {'Dieser Vorschlag kombiniert einige der wichtigsten Gegenstände und bietet eine gute Balance zwischen Aufwand und Nutzen.', 'Ich denke, dass diese Auswahl eine gute Grundlage für weitere Verhandlungen darstellt.', 'Ich bin bereit, Anpassungen vorzunehmen, um sicherzustellen, dass der Vorschlag für beide Seiten akzeptabel ist und den Maximalaufwand nicht überschreitet.'} 
	  } 
	   } 
	   } 
	 & \\ 
 

    \theutterance \stepcounter{utterance}  

    & & & \multicolumn{4}{p{0.6\linewidth}}{\cellcolor[rgb]{1,0.85,0.72}{%
	\makecell[{{p{\linewidth}}}]{% 
	  \tt {\tiny [GM$\langle$B]}  
	 ABLEHNUNG: {'C56', 'A07', 'C62', 'B38', 'C03'}\\ \tt ARGUMENT: {'Ich denke, dass dieser Vorschlag eine interessante Kombination von Gegenständen enthält, aber ich bin nicht überzeugt, dass es die beste Auswahl ist.', 'Ich bezweifle, dass der Gegenstand C56 aufgrund seiner relativ geringen Wichtigkeit die beste Wahl ist.', 'Ich bin der Meinung, dass es wichtig ist, alternative Kombinationen zu prüfen, die eine bessere Balance zwischen Aufwand und Nutzen bieten könnten und die wichtigsten Gegenstände priorisieren.'}\\ \tt VORSCHLAG: {'C08', 'C76', 'C98', 'A83', 'C32'}\\ \tt ARGUMENT: {'Dieser Vorschlag enthält eine Reihe von Gegenständen, die ich für sehr wertvoll halte und die einen hohen Nutzen bieten könnten.', 'Ich denke, dass diese Auswahl eine gute Balance zwischen Aufwand und Nutzen bietet und eine faire Verteilung der Ressourcen darstellt.', 'Ich bin bereit, Kompromisse einzugehen, um eine Einigung zu erzielen, die für beide Seiten vorteilhaft ist und den Maximalaufwand nicht überschreitet.'} 
	  } 
	   } 
	   } 
	  \\ 
 

    \theutterance \stepcounter{utterance}  

    & & \multicolumn{4}{p{0.6\linewidth}}{\cellcolor[rgb]{0.9,0.9,0.9}{%
	\makecell[{{p{\linewidth}}}]{% 
	  \tt {\tiny [A$\langle$GM]}  
	 ABLEHNUNG: {'C56', 'A07', 'C62', 'B38', 'C03'}\\ \tt ARGUMENT: {'Ich denke, dass dieser Vorschlag eine interessante Kombination von Gegenständen enthält, aber ich bin nicht überzeugt, dass es die beste Auswahl ist.', 'Ich bezweifle, dass der Gegenstand C56 aufgrund seiner relativ geringen Wichtigkeit die beste Wahl ist.', 'Ich bin der Meinung, dass es wichtig ist, alternative Kombinationen zu prüfen, die eine bessere Balance zwischen Aufwand und Nutzen bieten könnten und die wichtigsten Gegenstände priorisieren.'}\\ \tt VORSCHLAG: {'C08', 'C76', 'C98', 'A83', 'C32'}\\ \tt ARGUMENT: {'Dieser Vorschlag enthält eine Reihe von Gegenständen, die ich für sehr wertvoll halte und die einen hohen Nutzen bieten könnten.', 'Ich denke, dass diese Auswahl eine gute Balance zwischen Aufwand und Nutzen bietet und eine faire Verteilung der Ressourcen darstellt.', 'Ich bin bereit, Kompromisse einzugehen, um eine Einigung zu erzielen, die für beide Seiten vorteilhaft ist und den Maximalaufwand nicht überschreitet.'} 
	  } 
	   } 
	   } 
	 & \\ 
 

    \theutterance \stepcounter{utterance}  

    & \multicolumn{4}{p{0.6\linewidth}}{\cellcolor[rgb]{0.8,1,0.9}{%
	\makecell[{{p{\linewidth}}}]{% 
	  \tt {\tiny [A$\rangle$GM]}  
	 ABLEHNUNG: {'C08', 'C76', 'C98', 'A83', 'C32'}\\ \tt ARGUMENT: {'Ich teile deine Einschätzung, dass dieser Vorschlag eine gute Balance zwischen Aufwand und Nutzen bietet, aber ich denke, dass es noch bessere Optionen gibt.', 'Ich bin besorgt, dass der Gegenstand A83 möglicherweise nicht die höchste Priorität hat und durch einen anderen ersetzt werden könnte.', 'Ich bin der Meinung, dass es wichtig ist, die Prioritäten neu zu bewerten und einen Vorschlag zu unterbreiten, der die wichtigsten Gegenstände enthält und den Maximalaufwand nicht überschreitet.'}\\ \tt VORSCHLAG: {'A07', 'C62', 'B38', 'C00', 'C03'}\\ \tt ARGUMENT: {'Dieser Vorschlag kombiniert einige der wichtigsten Gegenstände und bietet eine gute Balance zwischen Aufwand und Nutzen.', 'Ich denke, dass diese Auswahl eine gute Grundlage für weitere Verhandlungen darstellt.', 'Ich bin bereit, Anpassungen vorzunehmen, um sicherzustellen, dass der Vorschlag für beide Seiten akzeptabel ist und den Maximalaufwand nicht überschreitet.'} 
	  } 
	   } 
	   } 
	 & & \\ 
 

    \theutterance \stepcounter{utterance}  

    & & \multicolumn{4}{p{0.6\linewidth}}{\cellcolor[rgb]{0.9,0.9,0.9}{%
	\makecell[{{p{\linewidth}}}]{% 
	  \tt {\tiny [GM$\rangle$B]}  
	 ABLEHNUNG: {'C08', 'C76', 'C98', 'A83', 'C32'}\\ \tt ARGUMENT: {'Ich teile deine Einschätzung, dass dieser Vorschlag eine gute Balance zwischen Aufwand und Nutzen bietet, aber ich denke, dass es noch bessere Optionen gibt.', 'Ich bin besorgt, dass der Gegenstand A83 möglicherweise nicht die höchste Priorität hat und durch einen anderen ersetzt werden könnte.', 'Ich bin der Meinung, dass es wichtig ist, die Prioritäten neu zu bewerten und einen Vorschlag zu unterbreiten, der die wichtigsten Gegenstände enthält und den Maximalaufwand nicht überschreitet.'}\\ \tt VORSCHLAG: {'A07', 'C62', 'B38', 'C00', 'C03'}\\ \tt ARGUMENT: {'Dieser Vorschlag kombiniert einige der wichtigsten Gegenstände und bietet eine gute Balance zwischen Aufwand und Nutzen.', 'Ich denke, dass diese Auswahl eine gute Grundlage für weitere Verhandlungen darstellt.', 'Ich bin bereit, Anpassungen vorzunehmen, um sicherzustellen, dass der Vorschlag für beide Seiten akzeptabel ist und den Maximalaufwand nicht überschreitet.'} 
	  } 
	   } 
	   } 
	 & \\ 
 

    \theutterance \stepcounter{utterance}  

    & & & \multicolumn{4}{p{0.6\linewidth}}{\cellcolor[rgb]{1,0.85,0.72}{%
	\makecell[{{p{\linewidth}}}]{% 
	  \tt {\tiny [GM$\langle$B]}  
	 ZUSTIMMUNG: {'A07', 'C62', 'B38', 'C00', 'C03'}\\ \tt ARGUMENT: {'Ich denke, dass dieser Vorschlag eine sehr gute Balance zwischen Aufwand und Nutzen bietet und eine faire Verteilung der Ressourcen darstellt.', 'Ich bin der Meinung, dass die Auswahl der Gegenstände A07, C62, B38, C00 und C03 eine gute Kombination ist, die beide Seiten zufriedenstellen kann.', 'Ich stimme diesem Vorschlag zu, da er den Maximalaufwand nicht überschreitet und ich denke, dass es eine gute Grundlage für eine Einigung ist.'} 
	  } 
	   } 
	   } 
	  \\ 
 

    \theutterance \stepcounter{utterance}  

    & & & \multicolumn{2}{p{0.3\linewidth}}{\cellcolor[rgb]{0.95,0.95,0.95}{%
	\makecell[{{p{\linewidth}}}]{% 
	  \tt {\tiny [GM$|$GM]}  
	 game successful 
	  } 
	   } 
	   } 
	 & & \\ 
 

    \theutterance \stepcounter{utterance}  

    & & & \multicolumn{2}{p{0.3\linewidth}}{\cellcolor[rgb]{0.95,0.95,0.95}{%
	\makecell[{{p{\linewidth}}}]{% 
	  \tt {\tiny [GM$|$GM]}  
	 end game 
	  } 
	   } 
	   } 
	 & & \\ 
 

\end{supertabular}
}

\end{document}
