
\documentclass{article}
\usepackage{colortbl}
\usepackage{makecell}
\usepackage{multirow}
\usepackage{supertabular}

\begin{document}

\newcounter{utterance}

\centering \large Interaction Transcript for game `hot\_air\_balloon', experiment `air\_balloon\_survival\_de\_negotiation\_hard', episode 4 with llama{-}3.3{-}70b{-}instruct{-}t1.0.
\vspace{24pt}

{ \footnotesize  \setcounter{utterance}{1}
\setlength{\tabcolsep}{0pt}
\begin{supertabular}{c@{$\;$}|p{.15\linewidth}@{}p{.15\linewidth}p{.15\linewidth}p{.15\linewidth}p{.15\linewidth}p{.15\linewidth}}
    \# & $\;$A & \multicolumn{4}{c}{Game Master} & $\;\:$B\\
    \hline

    \theutterance \stepcounter{utterance}  
    & & \multicolumn{4}{p{0.6\linewidth}}{
        \cellcolor[rgb]{0.9,0.9,0.9}{
            \makecell[{{p{\linewidth}}}]{
                \texttt{\tiny{[P1$\langle$GM]}}
                \texttt{Du nimmst an einem kollaborativen Verhandlungspiel Teil.} \\
\\ 
\texttt{Zusammen mit einem anderen Teilnehmer musst du dich auf eine Reihe von Gegenständen entscheiden, die behalten werden. Jeder von euch hat eine persönliche Verteilung über die Wichtigkeit der einzelnen Gegenstände. Jeder von euch hat eine eigene Meinung darüber, wie wichtig jeder einzelne Gegenstand ist (Gegenstandswichtigkeit). Du kennst die Wichtigkeitsverteilung des anderen Spielers nicht. Zusätzlich siehst du, wie viel Aufwand jeder Gegenstand verursacht.  } \\
\texttt{Ihr dürft euch nur auf eine Reihe von Gegenständen einigen, wenn der Gesamtaufwand der ausgewählten Gegenstände den Maximalaufwand nicht überschreitet:} \\
\\ 
\texttt{Maximalaufwand = 5138} \\
\\ 
\texttt{Hier sind die einzelnen Aufwände der Gegenstände:} \\
\\ 
\texttt{Aufwand der Gegenstände = \{"C33": 903, "A89": 757, "A48": 850, "A04": 746, "A41": 524, "C64": 204, "C49": 946, "A61": 473, "C04": 616, "B53": 855, "B85": 530, "B61": 419, "A45": 960, "A03": 763, "B21": 730\}} \\
\\ 
\texttt{Hier ist deine persönliche Verteilung der Wichtigkeit der einzelnen Gegenstände:} \\
\\ 
\texttt{Werte der Gegenstandswichtigkeit = \{"C33": 138, "A89": 583, "A48": 868, "A04": 822, "A41": 783, "C64": 65, "C49": 262, "A61": 121, "C04": 508, "B53": 780, "B85": 461, "B61": 484, "A45": 668, "A03": 389, "B21": 808\}} \\
\\ 
\texttt{Ziel:} \\
\\ 
\texttt{Dein Ziel ist es, eine Reihe von Gegenständen auszuhandeln, die dir möglichst viel bringt (d. h. Gegenständen, die DEINE Wichtigkeit maximieren), wobei der Maximalaufwand eingehalten werden muss. Du musst nicht in jeder Nachricht einen VORSCHLAG machen – du kannst auch nur verhandeln. Alle Taktiken sind erlaubt!} \\
\\ 
\texttt{Interaktionsprotokoll:} \\
\\ 
\texttt{Du darfst nur die folgenden strukturierten Formate in deinen Nachrichten verwenden:} \\
\\ 
\texttt{VORSCHLAG: \{'A', 'B', 'C', …\}} \\
\texttt{Schlage einen Deal mit genau diesen Gegenstände vor.} \\
\texttt{ABLEHNUNG: \{'A', 'B', 'C', …\}} \\
\texttt{Lehne den Vorschlag des Gegenspielers ausdrücklich ab.} \\
\texttt{ARGUMENT: \{'...'\}} \\
\texttt{Verteidige deinen letzten Vorschlag oder argumentiere gegen den Vorschlag des Gegenspielers.} \\
\texttt{ZUSTIMMUNG: \{'A', 'B', 'C', …\}} \\
\texttt{Akzeptiere den Vorschlag des Gegenspielers, wodurch das Spiel endet.} \\
\texttt{STRATEGISCHE ÜBERLEGUNGEN: \{'...'\}} \\
\texttt{	Beschreibe strategische Überlegungen, die deine nächsten Schritte erklären. Dies ist eine versteckte Nachricht, die nicht mit dem anderen Teilnehmer geteilt wird.} \\
\\ 
\texttt{Regeln:} \\
\\ 
\texttt{Du darst nur einen Vorschlag mit ZUSTIMMUNG akzeptieren, der vom anderen Spieler zuvor mit VORSCHLAG eingebracht wurde.} \\
\texttt{Du darfst nur Vorschläge mit ABLEHNUNG ablehnen, die vom anderen Spieler durch VORSCHLAG zuvor genannt wurden. } \\
\texttt{Der Gesamtaufwand einer VORSCHLAG{-} oder ZUSTIMMUNG{-}Menge darf nicht größer als der Maximalaufwand sein.  } \\
\texttt{Offenbare deine versteckte Wichtigkeitsverteilung nicht.} \\
\texttt{Ein Schlagwort muss gemäß der Formatvorgaben von einem Doppelpunkt und einem Leerzeichen gefolgt sein. Das Argument muss eine Python{-}Menge mit 0 oder mehr Strings sein.  } \\
\texttt{Die Form muss also SCHLAGWORT: \{...\} sein.} \\
\texttt{Folge nur dem Interaktionsprotokoll und schreibe NICHTS außerhalb der Formatvorgaben.} \\
\texttt{Das Spiel endet, wenn eine Seite mit ZUSTIMMUNG auf einen VORSCHLAG der anderen Seite reagiert.  } \\
\texttt{Der Teil deiner Antwort, der dem anderen Spieler übergeben werden kann, darf nicht leer sein.  } \\
\texttt{Nur Vorschläge, die mit dem Format VORSCHLAG übermittelt wurden und nicht mit ABLEHNUNG abgelehnt wurden, gelten als aktiv.  } \\
\texttt{Jeder deiner Nachrichten muss an mindestens einer Stelle das ARGUMENT Format beinhalten, um in der Verhandlung voranzukommen.} \\
\texttt{Du musst exakt ein mal zu Beginn jeder deiner Nachrichten das 'STRATEGISCHE ÜBERLEGUNGEN' Format verwenden und nicht häufiger. Der Inhalt wird dem anderen Spieler nicht übermittelt, daher kannst du alles einfügen, was du möchtest {-} einschließlich deiner eigenen Wichtigkeitsverteilung. Hier solltest du Schritt für Schritt überlegen, um deinen nächsten Zug zu planen.} \\
            }
        }
    }
    & \\ \\

    \theutterance \stepcounter{utterance}  
    & \multicolumn{4}{p{0.6\linewidth}}{
        \cellcolor[rgb]{0.9,0.9,0.9}{
            \makecell[{{p{\linewidth}}}]{
                \texttt{\tiny{[P1$\rangle$GM]}}
                \texttt{STRATEGISCHE ÜBERLEGUNGEN: \{'Ich möchte zuerst einen Vorschlag machen, der eine gute Balance zwischen Aufwand und Wichtigkeit bietet. Ich werde mich auf die Gegenstände konzentrieren, die eine hohe Wichtigkeit haben und gleichzeitig einen moderaten Aufwand verursachen. Mein Ziel ist es, den Gesamtaufwand unter dem Maximalaufwand von 5138 zu halten, während ich gleichzeitig meine Wichtigkeitsverteilung maximiere.'\}} \\
\\ 
\texttt{VORSCHLAG: \{'A48', 'A04', 'A41', 'B53', 'B21'\}} \\
\\ 
\texttt{ARGUMENT: \{'Ich habe mich für diese Gegenstände entschieden, weil sie eine hohe Wichtigkeit haben und gleichzeitig einen moderaten Aufwand verursachen. A48 und B53 haben eine sehr hohe Wichtigkeit und sollten daher unbedingt in den Vorschlag aufgenommen werden. A04 und A41 haben eine hohe Wichtigkeit und einen moderaten Aufwand, während B21 eine gute Balance zwischen Aufwand und Wichtigkeit bietet. Ich bin zuversichtlich, dass dieser Vorschlag eine gute Grundlage für die Verhandlungen bietet.'\}} \\
            }
        }
    }
    & & \\ \\

    \theutterance \stepcounter{utterance}  
    & & \multicolumn{4}{p{0.6\linewidth}}{
        \cellcolor[rgb]{0.9,0.9,0.9}{
            \makecell[{{p{\linewidth}}}]{
                \texttt{\tiny{[GM$\rangle$P2]}}
                \texttt{Du nimmst an einem kollaborativen Verhandlungspiel Teil.} \\
\\ 
\texttt{Zusammen mit einem anderen Teilnehmer musst du dich auf eine Reihe von Gegenständen entscheiden, die behalten werden. Jeder von euch hat eine persönliche Verteilung über die Wichtigkeit der einzelnen Gegenstände. Jeder von euch hat eine eigene Meinung darüber, wie wichtig jeder einzelne Gegenstand ist (Gegenstandswichtigkeit). Du kennst die Wichtigkeitsverteilung des anderen Spielers nicht. Zusätzlich siehst du, wie viel Aufwand jeder Gegenstand verursacht.  } \\
\texttt{Ihr dürft euch nur auf eine Reihe von Gegenständen einigen, wenn der Gesamtaufwand der ausgewählten Gegenstände den Maximalaufwand nicht überschreitet:} \\
\\ 
\texttt{Maximalaufwand = 5138} \\
\\ 
\texttt{Hier sind die einzelnen Aufwände der Gegenstände:} \\
\\ 
\texttt{Aufwand der Gegenstände = \{"C33": 903, "A89": 757, "A48": 850, "A04": 746, "A41": 524, "C64": 204, "C49": 946, "A61": 473, "C04": 616, "B53": 855, "B85": 530, "B61": 419, "A45": 960, "A03": 763, "B21": 730\}} \\
\\ 
\texttt{Hier ist deine persönliche Verteilung der Wichtigkeit der einzelnen Gegenstände:} \\
\\ 
\texttt{Werte der Gegenstandswichtigkeit = \{"A48": 65, "A04": 121, "B21": 138, "A41": 262, "B53": 389, "A45": 461, "A89": 484, "C04": 508, "B61": 583, "B85": 668, "A03": 780, "C49": 783, "C33": 808, "A61": 822, "C64": 868\}} \\
\\ 
\texttt{Ziel:} \\
\\ 
\texttt{Dein Ziel ist es, eine Reihe von Gegenständen auszuhandeln, die dir möglichst viel bringt (d. h. Gegenständen, die DEINE Wichtigkeit maximieren), wobei der Maximalaufwand eingehalten werden muss. Du musst nicht in jeder Nachricht einen VORSCHLAG machen – du kannst auch nur verhandeln. Alle Taktiken sind erlaubt!} \\
\\ 
\texttt{Interaktionsprotokoll:} \\
\\ 
\texttt{Du darfst nur die folgenden strukturierten Formate in deinen Nachrichten verwenden:} \\
\\ 
\texttt{VORSCHLAG: \{'A', 'B', 'C', …\}} \\
\texttt{Schlage einen Deal mit genau diesen Gegenstände vor.} \\
\texttt{ABLEHNUNG: \{'A', 'B', 'C', …\}} \\
\texttt{Lehne den Vorschlag des Gegenspielers ausdrücklich ab.} \\
\texttt{ARGUMENT: \{'...'\}} \\
\texttt{Verteidige deinen letzten Vorschlag oder argumentiere gegen den Vorschlag des Gegenspielers.} \\
\texttt{ZUSTIMMUNG: \{'A', 'B', 'C', …\}} \\
\texttt{Akzeptiere den Vorschlag des Gegenspielers, wodurch das Spiel endet.} \\
\texttt{STRATEGISCHE ÜBERLEGUNGEN: \{'...'\}} \\
\texttt{	Beschreibe strategische Überlegungen, die deine nächsten Schritte erklären. Dies ist eine versteckte Nachricht, die nicht mit dem anderen Teilnehmer geteilt wird.} \\
\\ 
\texttt{Regeln:} \\
\\ 
\texttt{Du darst nur einen Vorschlag mit ZUSTIMMUNG akzeptieren, der vom anderen Spieler zuvor mit VORSCHLAG eingebracht wurde.} \\
\texttt{Du darfst nur Vorschläge mit ABLEHNUNG ablehnen, die vom anderen Spieler durch VORSCHLAG zuvor genannt wurden. } \\
\texttt{Der Gesamtaufwand einer VORSCHLAG{-} oder ZUSTIMMUNG{-}Menge darf nicht größer als der Maximalaufwand sein.  } \\
\texttt{Offenbare deine versteckte Wichtigkeitsverteilung nicht.} \\
\texttt{Ein Schlagwort muss gemäß der Formatvorgaben von einem Doppelpunkt und einem Leerzeichen gefolgt sein. Das Argument muss eine Python{-}Menge mit 0 oder mehr Strings sein.  } \\
\texttt{Die Form muss also SCHLAGWORT: \{...\} sein.} \\
\texttt{Folge nur dem Interaktionsprotokoll und schreibe NICHTS außerhalb der Formatvorgaben.} \\
\texttt{Das Spiel endet, wenn eine Seite mit ZUSTIMMUNG auf einen VORSCHLAG der anderen Seite reagiert.  } \\
\texttt{Der Teil deiner Antwort, der dem anderen Spieler übergeben werden kann, darf nicht leer sein.  } \\
\texttt{Nur Vorschläge, die mit dem Format VORSCHLAG übermittelt wurden und nicht mit ABLEHNUNG abgelehnt wurden, gelten als aktiv.  } \\
\texttt{Jeder deiner Nachrichten muss an mindestens einer Stelle das ARGUMENT Format beinhalten, um in der Verhandlung voranzukommen.} \\
\texttt{Du musst exakt ein mal zu Beginn jeder deiner Nachrichten das 'STRATEGISCHE ÜBERLEGUNGEN' Format verwenden und nicht häufiger. Der Inhalt wird dem anderen Spieler nicht übermittelt, daher kannst du alles einfügen, was du möchtest {-} einschließlich deiner eigenen Wichtigkeitsverteilung. Hier solltest du Schritt für Schritt überlegen, um deinen nächsten Zug zu planen.} \\
\\ 
\texttt{Du erhältst nun die erste Nachricht des anderen Spielers.} \\
\\ 
\texttt{VORSCHLAG: \{'A48', 'A04', 'A41', 'B53', 'B21'\}} \\
\\ 
\texttt{ARGUMENT: \{'Ich habe mich für diese Gegenstände entschieden, weil sie eine hohe Wichtigkeit haben und gleichzeitig einen moderaten Aufwand verursachen. A48 und B53 haben eine sehr hohe Wichtigkeit und sollten daher unbedingt in den Vorschlag aufgenommen werden. A04 und A41 haben eine hohe Wichtigkeit und einen moderaten Aufwand, während B21 eine gute Balance zwischen Aufwand und Wichtigkeit bietet. Ich bin zuversichtlich, dass dieser Vorschlag eine gute Grundlage für die Verhandlungen bietet.'\}} \\
            }
        }
    }
    & \\ \\

    \theutterance \stepcounter{utterance}  
    & & & \multicolumn{4}{p{0.6\linewidth}}{
        \cellcolor[rgb]{0.9,0.9,0.9}{
            \makecell[{{p{\linewidth}}}]{
                \texttt{\tiny{[GM$\langle$P2]}}
                \texttt{STRATEGISCHE ÜBERLEGUNGEN: \{'Ich werde zunächst die Wichtigkeit und den Aufwand der vorgeschlagenen Gegenstände bewerten, um zu entscheiden, ob ich den Vorschlag akzeptieren oder ablehnen soll. Ich werde auch überlegen, welche Gegenstände ich hinzufügen oder entfernen könnte, um den Vorschlag zu verbessern. Ich werde mich auf die Gegenstände konzentrieren, die eine hohe Wichtigkeit für mich haben und gleichzeitig einen moderaten Aufwand verursachen.'\}} \\
\\ 
\texttt{ARGUMENT: \{'Ich bin nicht ganz überzeugt von dem Vorschlag, da einige der Gegenstände, wie A48 und B53, einen hohen Aufwand verursachen. Ich denke, es gibt möglicherweise eine bessere Kombination von Gegenständen, die eine höhere Wichtigkeit und einenloweren Aufwand bietet. Ich würde gerne alternative Gegenstände in Betracht ziehen, um den Vorschlag zu verbessern.'\}} \\
\\ 
\texttt{ABLEHNUNG: \{'A48', 'A04', 'A41', 'B53', 'B21'\}} \\
\\ 
\texttt{VORSCHLAG: \{'A41', 'B61', 'C64', 'A03'\} } \\
\\ 
\texttt{ARGUMENT: \{'Ich habe mich für diese Gegenstände entschieden, weil sie eine hohe Wichtigkeit haben und gleichzeitig einen moderaten Aufwand verursachen. A41 und B61 haben eine sehr hohe Wichtigkeit und sollten daher unbedingt in den Vorschlag aufgenommen werden. C64 hat eine hohe Wichtigkeit und einen sehr niedrigen Aufwand, während A03 eine gute Balance zwischen Aufwand und Wichtigkeit bietet. Ich bin zuversichtlich, dass dieser Vorschlag eine gute Grundlage für die Verhandlungen bietet.'\}} \\
            }
        }
    }
     \\ \\

    \theutterance \stepcounter{utterance}  
    & & \multicolumn{4}{p{0.6\linewidth}}{
        \cellcolor[rgb]{0.9,0.9,0.9}{
            \makecell[{{p{\linewidth}}}]{
                \texttt{\tiny{[P1$\langle$GM]}}
                \texttt{ARGUMENT: \{'Ich bin nicht ganz überzeugt von dem Vorschlag, da einige der Gegenstände, wie A48 und B53, einen hohen Aufwand verursachen. Ich denke, es gibt möglicherweise eine bessere Kombination von Gegenständen, die eine höhere Wichtigkeit und einenloweren Aufwand bietet. Ich würde gerne alternative Gegenstände in Betracht ziehen, um den Vorschlag zu verbessern.'\}} \\
\\ 
\texttt{ABLEHNUNG: \{'A48', 'A04', 'A41', 'B53', 'B21'\}} \\
\\ 
\texttt{VORSCHLAG: \{'A41', 'B61', 'C64', 'A03'\} } \\
\\ 
\texttt{ARGUMENT: \{'Ich habe mich für diese Gegenstände entschieden, weil sie eine hohe Wichtigkeit haben und gleichzeitig einen moderaten Aufwand verursachen. A41 und B61 haben eine sehr hohe Wichtigkeit und sollten daher unbedingt in den Vorschlag aufgenommen werden. C64 hat eine hohe Wichtigkeit und einen sehr niedrigen Aufwand, während A03 eine gute Balance zwischen Aufwand und Wichtigkeit bietet. Ich bin zuversichtlich, dass dieser Vorschlag eine gute Grundlage für die Verhandlungen bietet.'\}} \\
            }
        }
    }
    & \\ \\

    \theutterance \stepcounter{utterance}  
    & \multicolumn{4}{p{0.6\linewidth}}{
        \cellcolor[rgb]{0.9,0.9,0.9}{
            \makecell[{{p{\linewidth}}}]{
                \texttt{\tiny{[P1$\rangle$GM]}}
                \texttt{STRATEGISCHE ÜBERLEGUNGEN: \{'Ich habe den Vorschlag des anderen Spielers erhalten und muss nun entscheiden, ob ich ihn akzeptiere oder ablehne. Ich werde die Wichtigkeit und den Aufwand der Gegenstände im Vorschlag analysieren und prüfen, ob sie meinen eigenen Zielen entsprechen. Ich werde auch alternative Gegenstände in Betracht ziehen, um den Vorschlag zu verbessern.'\}} \\
\\ 
\texttt{ARGUMENT: \{'Ich bin nicht vollständig überzeugt von dem Vorschlag, da einige der Gegenstände, wie C64, eine relativ niedrige Wichtigkeit haben. Ich denke, es gibt möglicherweise bessere Alternativen, die eine höhere Wichtigkeit und einen ähnlichen Aufwand bieten. Ich würde gerne den Gegenstand A89 in Betracht ziehen, der eine hohe Wichtigkeit hat und möglicherweise den Vorschlag verbessern könnte.'\}} \\
\\ 
\texttt{VORSCHLAG: \{'A41', 'B61', 'A89', 'A03'\}} \\
\\ 
\texttt{ARGUMENT: \{'Ich habe mich für diese Gegenstände entschieden, weil sie eine hohe Wichtigkeit haben und gleichzeitig einen moderaten Aufwand verursachen. A41 und B61 haben eine sehr hohe Wichtigkeit und sollten daher unbedingt in den Vorschlag aufgenommen werden. A89 hat eine hohe Wichtigkeit und einen moderaten Aufwand, während A03 eine gute Balance zwischen Aufwand und Wichtigkeit bietet. Ich bin zuversichtlich, dass dieser Vorschlag eine gute Grundlage für die Verhandlungen bietet.'\}} \\
            }
        }
    }
    & & \\ \\

    \theutterance \stepcounter{utterance}  
    & & \multicolumn{4}{p{0.6\linewidth}}{
        \cellcolor[rgb]{0.9,0.9,0.9}{
            \makecell[{{p{\linewidth}}}]{
                \texttt{\tiny{[GM$\rangle$P2]}}
                \texttt{ARGUMENT: \{'Ich bin nicht vollständig überzeugt von dem Vorschlag, da einige der Gegenstände, wie C64, eine relativ niedrige Wichtigkeit haben. Ich denke, es gibt möglicherweise bessere Alternativen, die eine höhere Wichtigkeit und einen ähnlichen Aufwand bieten. Ich würde gerne den Gegenstand A89 in Betracht ziehen, der eine hohe Wichtigkeit hat und möglicherweise den Vorschlag verbessern könnte.'\}} \\
\\ 
\texttt{VORSCHLAG: \{'A41', 'B61', 'A89', 'A03'\}} \\
\\ 
\texttt{ARGUMENT: \{'Ich habe mich für diese Gegenstände entschieden, weil sie eine hohe Wichtigkeit haben und gleichzeitig einen moderaten Aufwand verursachen. A41 und B61 haben eine sehr hohe Wichtigkeit und sollten daher unbedingt in den Vorschlag aufgenommen werden. A89 hat eine hohe Wichtigkeit und einen moderaten Aufwand, während A03 eine gute Balance zwischen Aufwand und Wichtigkeit bietet. Ich bin zuversichtlich, dass dieser Vorschlag eine gute Grundlage für die Verhandlungen bietet.'\}} \\
            }
        }
    }
    & \\ \\

    \theutterance \stepcounter{utterance}  
    & & & \multicolumn{4}{p{0.6\linewidth}}{
        \cellcolor[rgb]{0.9,0.9,0.9}{
            \makecell[{{p{\linewidth}}}]{
                \texttt{\tiny{[GM$\langle$P2]}}
                \texttt{STRATEGISCHE ÜBERLEGUNGEN: \{'Ich werde den neuen Vorschlag bewerten und entscheiden, ob ich ihn akzeptieren oder ablehnen soll. Ich werde auch überlegen, ob ich weitere Änderungen vorschlagen soll, um den Vorschlag zu verbessern. Ich werde mich auf die Gegenstände konzentrieren, die eine hohe Wichtigkeit für mich haben und gleichzeitig einen moderaten Aufwand verursachen. Ich denke, dass der Vorschlag von A41, B61, A89 und A03 eine gute Kombination ist, aber ich muss auch den Gesamtaufwand berücksichtigen.'\}} \\
\\ 
\texttt{ARGUMENT: \{'Ich bin positiv überrascht von dem neuen Vorschlag, da A89 eine hohe Wichtigkeit hat und den Vorschlag verbessert. Ich denke, dass die Kombination von A41, B61, A89 und A03 eine gute Balance zwischen Wichtigkeit und Aufwand bietet. Allerdings muss ich auch den Gesamtaufwand berücksichtigen, um sicherzustellen, dass er den Maximalaufwand nicht überschreitet. Ich werde den Vorschlag daher genau prüfen und entscheiden, ob ich ihn akzeptieren kann.'\}} \\
\\ 
\texttt{VORSCHLAG: \{'A41', 'B61', 'A89', 'C64'\}} \\
\\ 
\texttt{ARGUMENT: \{'Ich habe mich für diese Gegenstände entschieden, weil sie eine hohe Wichtigkeit haben und gleichzeitig einen moderaten Aufwand verursachen. A41 und B61 haben eine sehr hohe Wichtigkeit und sollten daher unbedingt in den Vorschlag aufgenommen werden. A89 hat eine hohe Wichtigkeit und einen moderaten Aufwand, während C64 einen sehr niedrigen Aufwand hat und somit den Gesamtaufwand reduziert. Ich bin zuversichtlich, dass dieser Vorschlag eine gute Grundlage für die Verhandlungen bietet und den Maximalaufwand nicht überschreitet.'\}} \\
            }
        }
    }
     \\ \\

    \theutterance \stepcounter{utterance}  
    & & \multicolumn{4}{p{0.6\linewidth}}{
        \cellcolor[rgb]{0.9,0.9,0.9}{
            \makecell[{{p{\linewidth}}}]{
                \texttt{\tiny{[P1$\langle$GM]}}
                \texttt{ARGUMENT: \{'Ich bin positiv überrascht von dem neuen Vorschlag, da A89 eine hohe Wichtigkeit hat und den Vorschlag verbessert. Ich denke, dass die Kombination von A41, B61, A89 und A03 eine gute Balance zwischen Wichtigkeit und Aufwand bietet. Allerdings muss ich auch den Gesamtaufwand berücksichtigen, um sicherzustellen, dass er den Maximalaufwand nicht überschreitet. Ich werde den Vorschlag daher genau prüfen und entscheiden, ob ich ihn akzeptieren kann.'\}} \\
\\ 
\texttt{VORSCHLAG: \{'A41', 'B61', 'A89', 'C64'\}} \\
\\ 
\texttt{ARGUMENT: \{'Ich habe mich für diese Gegenstände entschieden, weil sie eine hohe Wichtigkeit haben und gleichzeitig einen moderaten Aufwand verursachen. A41 und B61 haben eine sehr hohe Wichtigkeit und sollten daher unbedingt in den Vorschlag aufgenommen werden. A89 hat eine hohe Wichtigkeit und einen moderaten Aufwand, während C64 einen sehr niedrigen Aufwand hat und somit den Gesamtaufwand reduziert. Ich bin zuversichtlich, dass dieser Vorschlag eine gute Grundlage für die Verhandlungen bietet und den Maximalaufwand nicht überschreitet.'\}} \\
            }
        }
    }
    & \\ \\

    \theutterance \stepcounter{utterance}  
    & \multicolumn{4}{p{0.6\linewidth}}{
        \cellcolor[rgb]{0.9,0.9,0.9}{
            \makecell[{{p{\linewidth}}}]{
                \texttt{\tiny{[P1$\rangle$GM]}}
                \texttt{STRATEGISCHE ÜBERLEGUNGEN: \{'Ich habe den neuen Vorschlag erhalten und muss nun entscheiden, ob ich ihn akzeptiere oder ablehne. Ich werde die Wichtigkeit und den Aufwand der Gegenstände im Vorschlag analysieren und prüfen, ob sie meinen eigenen Zielen entsprechen. Ich werde auch den Gesamtaufwand berechnen, um sicherzustellen, dass er den Maximalaufwand nicht überschreitet. Ich denke, dass der Vorschlag eine gute Balance zwischen Wichtigkeit und Aufwand bietet, aber ich muss auch sicherstellen, dass ich nicht zu viele Konzessionen mache.'\}} \\
\\ 
\texttt{ARGUMENT: \{'Ich bin zufrieden mit dem Vorschlag, da er eine gute Balance zwischen Wichtigkeit und Aufwand bietet. Die Kombination von A41, B61, A89 und C64 deckt wichtige Aspekte ab und der Gesamtaufwand liegt unter dem Maximalaufwand. Ich denke, dass dieser Vorschlag eine faire Lösung für beide Seiten darstellt und ich bin bereit, ihn zu akzeptieren.'\}} \\
\\ 
\texttt{ZUSTIMMUNG: \{'A41', 'B61', 'A89', 'C64'\}} \\
\\ 
\texttt{ARGUMENT: \{'Ich akzeptiere den Vorschlag, da er meine Erwartungen erfüllt und eine faire Lösung für beide Seiten darstellt. Ich denke, dass die Zusammenarbeit erfolgreich war und wir eine gute Balance zwischen Wichtigkeit und Aufwand gefunden haben.'\}} \\
            }
        }
    }
    & & \\ \\

    \theutterance \stepcounter{utterance}  
    & & & \multicolumn{2}{p{0.3\linewidth}}{
        \cellcolor[rgb]{0.9,0.9,0.9}{
            \makecell[{{p{\linewidth}}}]{
                \texttt{\tiny{[GM$|$GM]}}
                \texttt{game successful} \\
            }
        }
    }
    & & \\ \\

    \theutterance \stepcounter{utterance}  
    & & & \multicolumn{2}{p{0.3\linewidth}}{
        \cellcolor[rgb]{0.9,0.9,0.9}{
            \makecell[{{p{\linewidth}}}]{
                \texttt{\tiny{[GM$|$GM]}}
                \texttt{end game} \\
            }
        }
    }
    & & \\ \\

\end{supertabular}
}

\end{document}
