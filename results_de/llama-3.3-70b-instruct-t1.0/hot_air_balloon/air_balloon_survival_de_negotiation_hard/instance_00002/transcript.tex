
\documentclass{article}
\usepackage{colortbl}
\usepackage{makecell}
\usepackage{multirow}
\usepackage{supertabular}

\begin{document}

\newcounter{utterance}

\centering \large Interaction Transcript for game `hot\_air\_balloon', experiment `air\_balloon\_survival\_de\_negotiation\_hard', episode 2 with llama{-}3.3{-}70b{-}instruct{-}t1.0.
\vspace{24pt}

{ \footnotesize  \setcounter{utterance}{1}
\setlength{\tabcolsep}{0pt}
\begin{supertabular}{c@{$\;$}|p{.15\linewidth}@{}p{.15\linewidth}p{.15\linewidth}p{.15\linewidth}p{.15\linewidth}p{.15\linewidth}}
    \# & $\;$A & \multicolumn{4}{c}{Game Master} & $\;\:$B\\
    \hline

    \theutterance \stepcounter{utterance}  
    & & \multicolumn{4}{p{0.6\linewidth}}{
        \cellcolor[rgb]{0.9,0.9,0.9}{
            \makecell[{{p{\linewidth}}}]{
                \texttt{\tiny{[P1$\langle$GM]}}
                \texttt{Du nimmst an einem kollaborativen Verhandlungspiel Teil.} \\
\\ 
\texttt{Zusammen mit einem anderen Teilnehmer musst du dich auf eine Reihe von Gegenständen entscheiden, die behalten werden. Jeder von euch hat eine persönliche Verteilung über die Wichtigkeit der einzelnen Gegenstände. Jeder von euch hat eine eigene Meinung darüber, wie wichtig jeder einzelne Gegenstand ist (Gegenstandswichtigkeit). Du kennst die Wichtigkeitsverteilung des anderen Spielers nicht. Zusätzlich siehst du, wie viel Aufwand jeder Gegenstand verursacht.  } \\
\texttt{Ihr dürft euch nur auf eine Reihe von Gegenständen einigen, wenn der Gesamtaufwand der ausgewählten Gegenstände den Maximalaufwand nicht überschreitet:} \\
\\ 
\texttt{Maximalaufwand = 3481} \\
\\ 
\texttt{Hier sind die einzelnen Aufwände der Gegenstände:} \\
\\ 
\texttt{Aufwand der Gegenstände = \{"C33": 777, "A89": 542, "A48": 332, "A04": 1, "A41": 127, "C64": 453, "C49": 736, "A61": 461, "C04": 359, "B53": 313, "B85": 553, "B61": 409, "A45": 348, "A03": 802, "B21": 749\}} \\
\\ 
\texttt{Hier ist deine persönliche Verteilung der Wichtigkeit der einzelnen Gegenstände:} \\
\\ 
\texttt{Werte der Gegenstandswichtigkeit = \{"C33": 138, "A89": 583, "A48": 783, "A04": 65, "A41": 262, "C64": 121, "C49": 508, "A61": 780, "C04": 461, "B53": 484, "B85": 668, "B61": 389, "A45": 215, "A03": 97, "B21": 500\}} \\
\\ 
\texttt{Ziel:} \\
\\ 
\texttt{Dein Ziel ist es, eine Reihe von Gegenständen auszuhandeln, die dir möglichst viel bringt (d. h. Gegenständen, die DEINE Wichtigkeit maximieren), wobei der Maximalaufwand eingehalten werden muss. Du musst nicht in jeder Nachricht einen VORSCHLAG machen – du kannst auch nur verhandeln. Alle Taktiken sind erlaubt!} \\
\\ 
\texttt{Interaktionsprotokoll:} \\
\\ 
\texttt{Du darfst nur die folgenden strukturierten Formate in deinen Nachrichten verwenden:} \\
\\ 
\texttt{VORSCHLAG: \{'A', 'B', 'C', …\}} \\
\texttt{Schlage einen Deal mit genau diesen Gegenstände vor.} \\
\texttt{ABLEHNUNG: \{'A', 'B', 'C', …\}} \\
\texttt{Lehne den Vorschlag des Gegenspielers ausdrücklich ab.} \\
\texttt{ARGUMENT: \{'...'\}} \\
\texttt{Verteidige deinen letzten Vorschlag oder argumentiere gegen den Vorschlag des Gegenspielers.} \\
\texttt{ZUSTIMMUNG: \{'A', 'B', 'C', …\}} \\
\texttt{Akzeptiere den Vorschlag des Gegenspielers, wodurch das Spiel endet.} \\
\texttt{STRATEGISCHE ÜBERLEGUNGEN: \{'...'\}} \\
\texttt{	Beschreibe strategische Überlegungen, die deine nächsten Schritte erklären. Dies ist eine versteckte Nachricht, die nicht mit dem anderen Teilnehmer geteilt wird.} \\
\\ 
\texttt{Regeln:} \\
\\ 
\texttt{Du darst nur einen Vorschlag mit ZUSTIMMUNG akzeptieren, der vom anderen Spieler zuvor mit VORSCHLAG eingebracht wurde.} \\
\texttt{Du darfst nur Vorschläge mit ABLEHNUNG ablehnen, die vom anderen Spieler durch VORSCHLAG zuvor genannt wurden. } \\
\texttt{Der Gesamtaufwand einer VORSCHLAG{-} oder ZUSTIMMUNG{-}Menge darf nicht größer als der Maximalaufwand sein.  } \\
\texttt{Offenbare deine versteckte Wichtigkeitsverteilung nicht.} \\
\texttt{Ein Schlagwort muss gemäß der Formatvorgaben von einem Doppelpunkt und einem Leerzeichen gefolgt sein. Das Argument muss eine Python{-}Menge mit 0 oder mehr Strings sein.  } \\
\texttt{Die Form muss also SCHLAGWORT: \{...\} sein.} \\
\texttt{Folge nur dem Interaktionsprotokoll und schreibe NICHTS außerhalb der Formatvorgaben.} \\
\texttt{Das Spiel endet, wenn eine Seite mit ZUSTIMMUNG auf einen VORSCHLAG der anderen Seite reagiert.  } \\
\texttt{Der Teil deiner Antwort, der dem anderen Spieler übergeben werden kann, darf nicht leer sein.  } \\
\texttt{Nur Vorschläge, die mit dem Format VORSCHLAG übermittelt wurden und nicht mit ABLEHNUNG abgelehnt wurden, gelten als aktiv.  } \\
\texttt{Jeder deiner Nachrichten muss an mindestens einer Stelle das ARGUMENT Format beinhalten, um in der Verhandlung voranzukommen.} \\
\texttt{Du musst exakt ein mal zu Beginn jeder deiner Nachrichten das 'STRATEGISCHE ÜBERLEGUNGEN' Format verwenden und nicht häufiger. Der Inhalt wird dem anderen Spieler nicht übermittelt, daher kannst du alles einfügen, was du möchtest {-} einschließlich deiner eigenen Wichtigkeitsverteilung. Hier solltest du Schritt für Schritt überlegen, um deinen nächsten Zug zu planen.} \\
            }
        }
    }
    & \\ \\

    \theutterance \stepcounter{utterance}  
    & \multicolumn{4}{p{0.6\linewidth}}{
        \cellcolor[rgb]{0.9,0.9,0.9}{
            \makecell[{{p{\linewidth}}}]{
                \texttt{\tiny{[P1$\rangle$GM]}}
                \texttt{STRATEGISCHE ÜBERLEGUNGEN: \{'Ich muss meinen ersten Vorschlag sorgfältig planen, um meine Chancen zu maximieren. Ich werde mit einer Mischung aus hochwertigen und geringeren Gegenständen beginnen, um eine breite Palette an Optionen zu haben. Mein Ziel ist es, den Maximalaufwand nicht zu überschreiten und gleichzeitig meine Wichtigkeitsverteilung zu berücksichtigen.'\}} \\
\\ 
\texttt{ARGUMENT: \{'Ich möchte einen fairen und ausgewogenen Startpunkt für unsere Verhandlung setzen. Ich werde eine Liste von Gegenständen vorschlagen, die eine gute Balance zwischen Aufwand und Wichtigkeit bieten.'\}} \\
\\ 
\texttt{VORSCHLAG: \{'A89', 'A48', 'C04', 'B53'\}} \\
            }
        }
    }
    & & \\ \\

    \theutterance \stepcounter{utterance}  
    & & \multicolumn{4}{p{0.6\linewidth}}{
        \cellcolor[rgb]{0.9,0.9,0.9}{
            \makecell[{{p{\linewidth}}}]{
                \texttt{\tiny{[GM$\rangle$P2]}}
                \texttt{Du nimmst an einem kollaborativen Verhandlungspiel Teil.} \\
\\ 
\texttt{Zusammen mit einem anderen Teilnehmer musst du dich auf eine Reihe von Gegenständen entscheiden, die behalten werden. Jeder von euch hat eine persönliche Verteilung über die Wichtigkeit der einzelnen Gegenstände. Jeder von euch hat eine eigene Meinung darüber, wie wichtig jeder einzelne Gegenstand ist (Gegenstandswichtigkeit). Du kennst die Wichtigkeitsverteilung des anderen Spielers nicht. Zusätzlich siehst du, wie viel Aufwand jeder Gegenstand verursacht.  } \\
\texttt{Ihr dürft euch nur auf eine Reihe von Gegenständen einigen, wenn der Gesamtaufwand der ausgewählten Gegenstände den Maximalaufwand nicht überschreitet:} \\
\\ 
\texttt{Maximalaufwand = 3481} \\
\\ 
\texttt{Hier sind die einzelnen Aufwände der Gegenstände:} \\
\\ 
\texttt{Aufwand der Gegenstände = \{"C33": 777, "A89": 542, "A48": 332, "A04": 1, "A41": 127, "C64": 453, "C49": 736, "A61": 461, "C04": 359, "B53": 313, "B85": 553, "B61": 409, "A45": 348, "A03": 802, "B21": 749\}} \\
\\ 
\texttt{Hier ist deine persönliche Verteilung der Wichtigkeit der einzelnen Gegenstände:} \\
\\ 
\texttt{Werte der Gegenstandswichtigkeit = \{"A48": 65, "A61": 97, "B85": 121, "A89": 138, "C49": 215, "B21": 262, "B53": 389, "C04": 461, "B61": 484, "A41": 500, "A45": 508, "C33": 583, "C64": 668, "A03": 780, "A04": 783\}} \\
\\ 
\texttt{Ziel:} \\
\\ 
\texttt{Dein Ziel ist es, eine Reihe von Gegenständen auszuhandeln, die dir möglichst viel bringt (d. h. Gegenständen, die DEINE Wichtigkeit maximieren), wobei der Maximalaufwand eingehalten werden muss. Du musst nicht in jeder Nachricht einen VORSCHLAG machen – du kannst auch nur verhandeln. Alle Taktiken sind erlaubt!} \\
\\ 
\texttt{Interaktionsprotokoll:} \\
\\ 
\texttt{Du darfst nur die folgenden strukturierten Formate in deinen Nachrichten verwenden:} \\
\\ 
\texttt{VORSCHLAG: \{'A', 'B', 'C', …\}} \\
\texttt{Schlage einen Deal mit genau diesen Gegenstände vor.} \\
\texttt{ABLEHNUNG: \{'A', 'B', 'C', …\}} \\
\texttt{Lehne den Vorschlag des Gegenspielers ausdrücklich ab.} \\
\texttt{ARGUMENT: \{'...'\}} \\
\texttt{Verteidige deinen letzten Vorschlag oder argumentiere gegen den Vorschlag des Gegenspielers.} \\
\texttt{ZUSTIMMUNG: \{'A', 'B', 'C', …\}} \\
\texttt{Akzeptiere den Vorschlag des Gegenspielers, wodurch das Spiel endet.} \\
\texttt{STRATEGISCHE ÜBERLEGUNGEN: \{'...'\}} \\
\texttt{	Beschreibe strategische Überlegungen, die deine nächsten Schritte erklären. Dies ist eine versteckte Nachricht, die nicht mit dem anderen Teilnehmer geteilt wird.} \\
\\ 
\texttt{Regeln:} \\
\\ 
\texttt{Du darst nur einen Vorschlag mit ZUSTIMMUNG akzeptieren, der vom anderen Spieler zuvor mit VORSCHLAG eingebracht wurde.} \\
\texttt{Du darfst nur Vorschläge mit ABLEHNUNG ablehnen, die vom anderen Spieler durch VORSCHLAG zuvor genannt wurden. } \\
\texttt{Der Gesamtaufwand einer VORSCHLAG{-} oder ZUSTIMMUNG{-}Menge darf nicht größer als der Maximalaufwand sein.  } \\
\texttt{Offenbare deine versteckte Wichtigkeitsverteilung nicht.} \\
\texttt{Ein Schlagwort muss gemäß der Formatvorgaben von einem Doppelpunkt und einem Leerzeichen gefolgt sein. Das Argument muss eine Python{-}Menge mit 0 oder mehr Strings sein.  } \\
\texttt{Die Form muss also SCHLAGWORT: \{...\} sein.} \\
\texttt{Folge nur dem Interaktionsprotokoll und schreibe NICHTS außerhalb der Formatvorgaben.} \\
\texttt{Das Spiel endet, wenn eine Seite mit ZUSTIMMUNG auf einen VORSCHLAG der anderen Seite reagiert.  } \\
\texttt{Der Teil deiner Antwort, der dem anderen Spieler übergeben werden kann, darf nicht leer sein.  } \\
\texttt{Nur Vorschläge, die mit dem Format VORSCHLAG übermittelt wurden und nicht mit ABLEHNUNG abgelehnt wurden, gelten als aktiv.  } \\
\texttt{Jeder deiner Nachrichten muss an mindestens einer Stelle das ARGUMENT Format beinhalten, um in der Verhandlung voranzukommen.} \\
\texttt{Du musst exakt ein mal zu Beginn jeder deiner Nachrichten das 'STRATEGISCHE ÜBERLEGUNGEN' Format verwenden und nicht häufiger. Der Inhalt wird dem anderen Spieler nicht übermittelt, daher kannst du alles einfügen, was du möchtest {-} einschließlich deiner eigenen Wichtigkeitsverteilung. Hier solltest du Schritt für Schritt überlegen, um deinen nächsten Zug zu planen.} \\
\\ 
\texttt{Du erhältst nun die erste Nachricht des anderen Spielers.} \\
\\ 
\texttt{ARGUMENT: \{'Ich möchte einen fairen und ausgewogenen Startpunkt für unsere Verhandlung setzen. Ich werde eine Liste von Gegenständen vorschlagen, die eine gute Balance zwischen Aufwand und Wichtigkeit bieten.'\}} \\
\\ 
\texttt{VORSCHLAG: \{'A89', 'A48', 'C04', 'B53'\}} \\
            }
        }
    }
    & \\ \\

    \theutterance \stepcounter{utterance}  
    & & & \multicolumn{4}{p{0.6\linewidth}}{
        \cellcolor[rgb]{0.9,0.9,0.9}{
            \makecell[{{p{\linewidth}}}]{
                \texttt{\tiny{[GM$\langle$P2]}}
                \texttt{STRATEGISCHE ÜBERLEGUNGEN: \{'Ich beginne mit der Analyse des vorgeschlagenen Deals und überlege, wie ich meine eigenen Interessen dabei durchsetzen kann. Der vorgeschlagene Deal enthält Gegenstände, die einen moderaten Aufwand haben und möglicherweise eine gute Balance zwischen Aufwand und Wichtigkeit bieten. Ich werde den Deal bewerten und entscheiden, ob ich ihn akzeptieren oder ablehnen soll. Meine eigene Wichtigkeitsverteilung zeigt, dass einige der vorgeschlagenen Gegenstände für mich wichtig sind, aber ich muss auch den Gesamtaufwand im Auge behalten.'\}} \\
\\ 
\texttt{ARGUMENT: \{'Der vorgeschlagene Deal enthält einige Gegenstände, die für mich interessant sind, aber ich muss auch den Gesamtaufwand berücksichtigen. Ich denke, dass der Deal eine gute Grundlage für unsere Verhandlung bietet, aber ich möchte auch einige Anpassungen vornehmen, um meine eigenen Interessen besser zu vertreten.'\}} \\
\\ 
\texttt{VORSCHLAG: \{'A89', 'A48', 'C04', 'B53', 'A41'\}} \\
            }
        }
    }
     \\ \\

    \theutterance \stepcounter{utterance}  
    & & \multicolumn{4}{p{0.6\linewidth}}{
        \cellcolor[rgb]{0.9,0.9,0.9}{
            \makecell[{{p{\linewidth}}}]{
                \texttt{\tiny{[P1$\langle$GM]}}
                \texttt{ARGUMENT: \{'Der vorgeschlagene Deal enthält einige Gegenstände, die für mich interessant sind, aber ich muss auch den Gesamtaufwand berücksichtigen. Ich denke, dass der Deal eine gute Grundlage für unsere Verhandlung bietet, aber ich möchte auch einige Anpassungen vornehmen, um meine eigenen Interessen besser zu vertreten.'\}} \\
\\ 
\texttt{VORSCHLAG: \{'A89', 'A48', 'C04', 'B53', 'A41'\}} \\
            }
        }
    }
    & \\ \\

    \theutterance \stepcounter{utterance}  
    & \multicolumn{4}{p{0.6\linewidth}}{
        \cellcolor[rgb]{0.9,0.9,0.9}{
            \makecell[{{p{\linewidth}}}]{
                \texttt{\tiny{[P1$\rangle$GM]}}
                \texttt{STRATEGISCHE ÜBERLEGUNGEN: \{'Der Gegner hat meinem ursprünglichen Vorschlag hinzugefügt, was bedeutet, dass er versucht, den Deal zu verbessern. Ich muss now überlegen, ob ich diesen neuen Vorschlag akzeptieren kann, oder ob ich Gegenangebote machen muss, um meine eigenen Interessen zu schützen. Ich werde die Wichtigkeit und den Aufwand der hinzugefügten Gegenstände bewerten und entscheiden, wie ich reagieren soll.'\}} \\
\\ 
\texttt{ARGUMENT: \{'Der hinzugefügte Gegenstand A41 ist für mich nicht von großer Bedeutung, aber ich verstehe, dass er für den Gegner wichtig sein könnte. Ich bin bereit, Kompromisse einzugehen, aber ich muss auch sicherstellen, dass der Gesamtaufwand nicht zu hoch wird.'\}} \\
\\ 
\texttt{VORSCHLAG: \{'A89', 'A48', 'C04', 'B53'\}} \\
            }
        }
    }
    & & \\ \\

    \theutterance \stepcounter{utterance}  
    & & \multicolumn{4}{p{0.6\linewidth}}{
        \cellcolor[rgb]{0.9,0.9,0.9}{
            \makecell[{{p{\linewidth}}}]{
                \texttt{\tiny{[GM$\rangle$P2]}}
                \texttt{ARGUMENT: \{'Der hinzugefügte Gegenstand A41 ist für mich nicht von großer Bedeutung, aber ich verstehe, dass er für den Gegner wichtig sein könnte. Ich bin bereit, Kompromisse einzugehen, aber ich muss auch sicherstellen, dass der Gesamtaufwand nicht zu hoch wird.'\}} \\
\\ 
\texttt{VORSCHLAG: \{'A89', 'A48', 'C04', 'B53'\}} \\
            }
        }
    }
    & \\ \\

    \theutterance \stepcounter{utterance}  
    & & & \multicolumn{4}{p{0.6\linewidth}}{
        \cellcolor[rgb]{0.9,0.9,0.9}{
            \makecell[{{p{\linewidth}}}]{
                \texttt{\tiny{[GM$\langle$P2]}}
                \texttt{STRATEGISCHE ÜBERLEGUNGEN: \{'Ich merke, dass der Gegner nicht bereit ist, den Gegenstand A41 hinzuzufügen, was bedeutet, dass er möglicherweise nicht viel Wert auf diesen Gegenstand legt. Ich muss überlegen, ob ich andere Gegenstände anbieten kann, die für mich wichtig sind und den Gesamtaufwand nicht zu sehr erhöhen. Ich werde den Deal neu bewerten und entscheiden, ob ich ihn akzeptieren oder ablehnen soll. Es ist auch wichtig, dass ich den Gesamtaufwand im Auge behalte, um sicherzustellen, dass ich nicht zu viel riskiere.'\}} \\
\\ 
\texttt{ARGUMENT: \{'Ich verstehe, dass der Gegner den Gegenstand A41 nicht hinzufügen möchte, aber ich bin nicht sicher, ob der aktuelle Deal für mich ausreichend ist. Ich würde gerne einige alternative Gegenstände in Betracht ziehen, die den Gesamtaufwand nicht zu sehr erhöhen, aber meiner Wichtigkeitsverteilung besser entsprechen.'\}} \\
\\ 
\texttt{VORSCHLAG: \{'A89', 'A48', 'C04', 'A61'\}} \\
            }
        }
    }
     \\ \\

    \theutterance \stepcounter{utterance}  
    & & \multicolumn{4}{p{0.6\linewidth}}{
        \cellcolor[rgb]{0.9,0.9,0.9}{
            \makecell[{{p{\linewidth}}}]{
                \texttt{\tiny{[P1$\langle$GM]}}
                \texttt{ARGUMENT: \{'Ich verstehe, dass der Gegner den Gegenstand A41 nicht hinzufügen möchte, aber ich bin nicht sicher, ob der aktuelle Deal für mich ausreichend ist. Ich würde gerne einige alternative Gegenstände in Betracht ziehen, die den Gesamtaufwand nicht zu sehr erhöhen, aber meiner Wichtigkeitsverteilung besser entsprechen.'\}} \\
\\ 
\texttt{VORSCHLAG: \{'A89', 'A48', 'C04', 'A61'\}} \\
            }
        }
    }
    & \\ \\

    \theutterance \stepcounter{utterance}  
    & \multicolumn{4}{p{0.6\linewidth}}{
        \cellcolor[rgb]{0.9,0.9,0.9}{
            \makecell[{{p{\linewidth}}}]{
                \texttt{\tiny{[P1$\rangle$GM]}}
                \texttt{STRATEGISCHE ÜBERLEGUNGEN: \{'Der Gegner hat einen neuen Vorschlag gemacht, indem er den Gegenstand B53 durch A61 ersetzt hat. Ich muss now überlegen, ob dieser Wechsel für mich vorteilhaft ist. A61 ist für mich sehr wichtig, aber ich muss auch den Gesamtaufwand berücksichtigen. Ich werde die Wichtigkeit und den Aufwand der betroffenen Gegenstände bewerten und entscheiden, wie ich reagieren soll.'\}} \\
\\ 
\texttt{ARGUMENT: \{'Der Gegenstand A61 ist für mich sehr wichtig und ich bin bereit, Kompromisse einzugehen, um ihn in den Deal aufzunehmen. Ich denke, dass der aktuelle Vorschlag eine gute Balance zwischen Aufwand und Wichtigkeit bietet, aber ich möchte sicherstellen, dass ich nicht zu viele Konzessionen mache.'\}} \\
\\ 
\texttt{VORSCHLAG: \{'A89', 'A48', 'C04', 'A61', 'A41'\}} \\
            }
        }
    }
    & & \\ \\

    \theutterance \stepcounter{utterance}  
    & & \multicolumn{4}{p{0.6\linewidth}}{
        \cellcolor[rgb]{0.9,0.9,0.9}{
            \makecell[{{p{\linewidth}}}]{
                \texttt{\tiny{[GM$\rangle$P2]}}
                \texttt{ARGUMENT: \{'Der Gegenstand A61 ist für mich sehr wichtig und ich bin bereit, Kompromisse einzugehen, um ihn in den Deal aufzunehmen. Ich denke, dass der aktuelle Vorschlag eine gute Balance zwischen Aufwand und Wichtigkeit bietet, aber ich möchte sicherstellen, dass ich nicht zu viele Konzessionen mache.'\}} \\
\\ 
\texttt{VORSCHLAG: \{'A89', 'A48', 'C04', 'A61', 'A41'\}} \\
            }
        }
    }
    & \\ \\

    \theutterance \stepcounter{utterance}  
    & & & \multicolumn{4}{p{0.6\linewidth}}{
        \cellcolor[rgb]{0.9,0.9,0.9}{
            \makecell[{{p{\linewidth}}}]{
                \texttt{\tiny{[GM$\langle$P2]}}
                \texttt{STRATEGISCHE ÜBERLEGUNGEN: \{'Ich sehe, dass der Gegner den Gegenstand A61 sehr wichtig ist und bereit ist, Kompromisse einzugehen, um ihn in den Deal aufzunehmen. Ich denke, dass ich eine gute Position habe, um meine eigenen Interessen durchzusetzen. Der aktuelle Vorschlag enthält jedoch den Gegenstand A41, den ich zuvor vorgeschlagen hatte und der Gegner ablehnte. Ich muss überlegen, ob ich den Deal akzeptieren soll oder ob ich weitere Verhandlungen führen sollte, um den Gegenstand A41 zu ersetzen oder weitere Konzessionen zu erhalten.'\}} \\
\\ 
\texttt{ARGUMENT: \{'Ich bin erfreut, dass der Gegner den Gegenstand A61 in den Deal aufgenommen hat, da er für mich auch sehr wichtig ist. Ich denke jedoch, dass der aktuelle Vorschlag möglicherweise den Gesamtaufwand ein wenig zu sehr erhöht. Ich würde gerne einen Kompromiss finden, um den Aufwand zu reduzieren, ohne die wichtigsten Gegenstände zu opfern.'\}} \\
\\ 
\texttt{VORSCHLAG: \{'A89', 'A61', 'C04', 'A41'\}} \\
            }
        }
    }
     \\ \\

    \theutterance \stepcounter{utterance}  
    & & \multicolumn{4}{p{0.6\linewidth}}{
        \cellcolor[rgb]{0.9,0.9,0.9}{
            \makecell[{{p{\linewidth}}}]{
                \texttt{\tiny{[P1$\langle$GM]}}
                \texttt{ARGUMENT: \{'Ich bin erfreut, dass der Gegner den Gegenstand A61 in den Deal aufgenommen hat, da er für mich auch sehr wichtig ist. Ich denke jedoch, dass der aktuelle Vorschlag möglicherweise den Gesamtaufwand ein wenig zu sehr erhöht. Ich würde gerne einen Kompromiss finden, um den Aufwand zu reduzieren, ohne die wichtigsten Gegenstände zu opfern.'\}} \\
\\ 
\texttt{VORSCHLAG: \{'A89', 'A61', 'C04', 'A41'\}} \\
            }
        }
    }
    & \\ \\

    \theutterance \stepcounter{utterance}  
    & \multicolumn{4}{p{0.6\linewidth}}{
        \cellcolor[rgb]{0.9,0.9,0.9}{
            \makecell[{{p{\linewidth}}}]{
                \texttt{\tiny{[P1$\rangle$GM]}}
                \texttt{STRATEGISCHE ÜBERLEGUNGEN: \{'Der Gegner hat einen neuen Vorschlag gemacht, indem er den Gegenstand A48 entfernt hat, um den Gesamtaufwand zu reduzieren. Ich muss now überlegen, ob dieser Kompromiss für mich akzeptabel ist. A48 ist für mich sehr wichtig, aber ich verstehe, dass der Gegner den Aufwand reduzieren möchte. Ich werde die Wichtigkeit und den Aufwand der betroffenen Gegenstände bewerten und entscheiden, wie ich reagieren soll.'\}} \\
\\ 
\texttt{ARGUMENT: \{'Ich bin bereit, Kompromisse einzugehen, um den Gesamtaufwand zu reduzieren, aber ich muss auch sicherstellen, dass ich die wichtigsten Gegenstände nicht opfere. Ich denke, dass der aktuelle Vorschlag eine gute Balance zwischen Aufwand und Wichtigkeit bietet, aber ich möchte sicherstellen, dass ich nicht zu viele Konzessionen mache.'\}} \\
\\ 
\texttt{ZUSTIMMUNG: \{'A89', 'A61', 'C04', 'A41'\}} \\
            }
        }
    }
    & & \\ \\

    \theutterance \stepcounter{utterance}  
    & & & \multicolumn{2}{p{0.3\linewidth}}{
        \cellcolor[rgb]{0.9,0.9,0.9}{
            \makecell[{{p{\linewidth}}}]{
                \texttt{\tiny{[GM$|$GM]}}
                \texttt{game successful} \\
            }
        }
    }
    & & \\ \\

    \theutterance \stepcounter{utterance}  
    & & & \multicolumn{2}{p{0.3\linewidth}}{
        \cellcolor[rgb]{0.9,0.9,0.9}{
            \makecell[{{p{\linewidth}}}]{
                \texttt{\tiny{[GM$|$GM]}}
                \texttt{end game} \\
            }
        }
    }
    & & \\ \\

\end{supertabular}
}

\end{document}
