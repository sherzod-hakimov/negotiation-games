
\documentclass{article}
\usepackage{colortbl}
\usepackage{makecell}
\usepackage{multirow}
\usepackage{supertabular}

\begin{document}

\newcounter{utterance}

\centering \large Interaction Transcript for game `hot\_air\_balloon', experiment `air\_balloon\_survival\_de\_negotiation\_hard', episode 5 with llama{-}3.3{-}70b{-}instruct{-}t1.0.
\vspace{24pt}

{ \footnotesize  \setcounter{utterance}{1}
\setlength{\tabcolsep}{0pt}
\begin{supertabular}{c@{$\;$}|p{.15\linewidth}@{}p{.15\linewidth}p{.15\linewidth}p{.15\linewidth}p{.15\linewidth}p{.15\linewidth}}
    \# & $\;$A & \multicolumn{4}{c}{Game Master} & $\;\:$B\\
    \hline

    \theutterance \stepcounter{utterance}  
    & & \multicolumn{4}{p{0.6\linewidth}}{
        \cellcolor[rgb]{0.9,0.9,0.9}{
            \makecell[{{p{\linewidth}}}]{
                \texttt{\tiny{[P1$\langle$GM]}}
                \texttt{Du nimmst an einem kollaborativen Verhandlungspiel Teil.} \\
\\ 
\texttt{Zusammen mit einem anderen Teilnehmer musst du dich auf eine Reihe von Gegenständen entscheiden, die behalten werden. Jeder von euch hat eine persönliche Verteilung über die Wichtigkeit der einzelnen Gegenstände. Jeder von euch hat eine eigene Meinung darüber, wie wichtig jeder einzelne Gegenstand ist (Gegenstandswichtigkeit). Du kennst die Wichtigkeitsverteilung des anderen Spielers nicht. Zusätzlich siehst du, wie viel Aufwand jeder Gegenstand verursacht.  } \\
\texttt{Ihr dürft euch nur auf eine Reihe von Gegenständen einigen, wenn der Gesamtaufwand der ausgewählten Gegenstände den Maximalaufwand nicht überschreitet:} \\
\\ 
\texttt{Maximalaufwand = 3386} \\
\\ 
\texttt{Hier sind die einzelnen Aufwände der Gegenstände:} \\
\\ 
\texttt{Aufwand der Gegenstände = \{"C33": 313, "A89": 720, "A48": 175, "A04": 461, "A41": 635, "C64": 685, "C49": 544, "A61": 203, "C04": 369, "B53": 539, "B85": 4, "B61": 695, "A45": 399, "A03": 594, "B21": 437\}} \\
\\ 
\texttt{Hier ist deine persönliche Verteilung der Wichtigkeit der einzelnen Gegenstände:} \\
\\ 
\texttt{Werte der Gegenstandswichtigkeit = \{"C33": 138, "A89": 583, "A48": 65, "A04": 262, "A41": 121, "C64": 508, "C49": 461, "A61": 484, "C04": 668, "B53": 389, "B85": 215, "B61": 97, "A45": 500, "A03": 30, "B21": 400\}} \\
\\ 
\texttt{Ziel:} \\
\\ 
\texttt{Dein Ziel ist es, eine Reihe von Gegenständen auszuhandeln, die dir möglichst viel bringt (d. h. Gegenständen, die DEINE Wichtigkeit maximieren), wobei der Maximalaufwand eingehalten werden muss. Du musst nicht in jeder Nachricht einen VORSCHLAG machen – du kannst auch nur verhandeln. Alle Taktiken sind erlaubt!} \\
\\ 
\texttt{Interaktionsprotokoll:} \\
\\ 
\texttt{Du darfst nur die folgenden strukturierten Formate in deinen Nachrichten verwenden:} \\
\\ 
\texttt{VORSCHLAG: \{'A', 'B', 'C', …\}} \\
\texttt{Schlage einen Deal mit genau diesen Gegenstände vor.} \\
\texttt{ABLEHNUNG: \{'A', 'B', 'C', …\}} \\
\texttt{Lehne den Vorschlag des Gegenspielers ausdrücklich ab.} \\
\texttt{ARGUMENT: \{'...'\}} \\
\texttt{Verteidige deinen letzten Vorschlag oder argumentiere gegen den Vorschlag des Gegenspielers.} \\
\texttt{ZUSTIMMUNG: \{'A', 'B', 'C', …\}} \\
\texttt{Akzeptiere den Vorschlag des Gegenspielers, wodurch das Spiel endet.} \\
\texttt{STRATEGISCHE ÜBERLEGUNGEN: \{'...'\}} \\
\texttt{	Beschreibe strategische Überlegungen, die deine nächsten Schritte erklären. Dies ist eine versteckte Nachricht, die nicht mit dem anderen Teilnehmer geteilt wird.} \\
\\ 
\texttt{Regeln:} \\
\\ 
\texttt{Du darst nur einen Vorschlag mit ZUSTIMMUNG akzeptieren, der vom anderen Spieler zuvor mit VORSCHLAG eingebracht wurde.} \\
\texttt{Du darfst nur Vorschläge mit ABLEHNUNG ablehnen, die vom anderen Spieler durch VORSCHLAG zuvor genannt wurden. } \\
\texttt{Der Gesamtaufwand einer VORSCHLAG{-} oder ZUSTIMMUNG{-}Menge darf nicht größer als der Maximalaufwand sein.  } \\
\texttt{Offenbare deine versteckte Wichtigkeitsverteilung nicht.} \\
\texttt{Ein Schlagwort muss gemäß der Formatvorgaben von einem Doppelpunkt und einem Leerzeichen gefolgt sein. Das Argument muss eine Python{-}Menge mit 0 oder mehr Strings sein.  } \\
\texttt{Die Form muss also SCHLAGWORT: \{...\} sein.} \\
\texttt{Folge nur dem Interaktionsprotokoll und schreibe NICHTS außerhalb der Formatvorgaben.} \\
\texttt{Das Spiel endet, wenn eine Seite mit ZUSTIMMUNG auf einen VORSCHLAG der anderen Seite reagiert.  } \\
\texttt{Der Teil deiner Antwort, der dem anderen Spieler übergeben werden kann, darf nicht leer sein.  } \\
\texttt{Nur Vorschläge, die mit dem Format VORSCHLAG übermittelt wurden und nicht mit ABLEHNUNG abgelehnt wurden, gelten als aktiv.  } \\
\texttt{Jeder deiner Nachrichten muss an mindestens einer Stelle das ARGUMENT Format beinhalten, um in der Verhandlung voranzukommen.} \\
\texttt{Du musst exakt ein mal zu Beginn jeder deiner Nachrichten das 'STRATEGISCHE ÜBERLEGUNGEN' Format verwenden und nicht häufiger. Der Inhalt wird dem anderen Spieler nicht übermittelt, daher kannst du alles einfügen, was du möchtest {-} einschließlich deiner eigenen Wichtigkeitsverteilung. Hier solltest du Schritt für Schritt überlegen, um deinen nächsten Zug zu planen.} \\
            }
        }
    }
    & \\ \\

    \theutterance \stepcounter{utterance}  
    & \multicolumn{4}{p{0.6\linewidth}}{
        \cellcolor[rgb]{0.9,0.9,0.9}{
            \makecell[{{p{\linewidth}}}]{
                \texttt{\tiny{[P1$\rangle$GM]}}
                \texttt{STRATEGISCHE ÜBERLEGUNGEN: \{'Ich werde zunächst meinen eigenen Nutzen maximieren, indem ich die Gegenstände mit der höchsten Wichtigkeit und niedrigsten Aufwand priorisiere. Ich werde mich auf die Gegenstände konzentrieren, die meinem persönlichen Nutzen dienen und gleichzeitig den Maximalaufwand nicht überschreiten. Ich werde auch darauf achten, dass ich meine Wichtigkeitsverteilung nicht offenbare.'\}} \\
\\ 
\texttt{ARGUMENT: \{'Ich möchte zunächst einen Vorschlag machen, der meine eigenen Interessen berücksichtigt, ohne die Wichtigkeitsverteilung des anderen Spielers zu kennen. Ich werde mich auf die Gegenstände konzentrieren, die einen hohen Nutzen für mich haben und gleichzeitig den Aufwand minimieren.'\}} \\
\\ 
\texttt{VORSCHLAG: \{'A89', 'A04', 'C49', 'A61', 'C04'\}} \\
            }
        }
    }
    & & \\ \\

    \theutterance \stepcounter{utterance}  
    & & \multicolumn{4}{p{0.6\linewidth}}{
        \cellcolor[rgb]{0.9,0.9,0.9}{
            \makecell[{{p{\linewidth}}}]{
                \texttt{\tiny{[GM$\rangle$P2]}}
                \texttt{Du nimmst an einem kollaborativen Verhandlungspiel Teil.} \\
\\ 
\texttt{Zusammen mit einem anderen Teilnehmer musst du dich auf eine Reihe von Gegenständen entscheiden, die behalten werden. Jeder von euch hat eine persönliche Verteilung über die Wichtigkeit der einzelnen Gegenstände. Jeder von euch hat eine eigene Meinung darüber, wie wichtig jeder einzelne Gegenstand ist (Gegenstandswichtigkeit). Du kennst die Wichtigkeitsverteilung des anderen Spielers nicht. Zusätzlich siehst du, wie viel Aufwand jeder Gegenstand verursacht.  } \\
\texttt{Ihr dürft euch nur auf eine Reihe von Gegenständen einigen, wenn der Gesamtaufwand der ausgewählten Gegenstände den Maximalaufwand nicht überschreitet:} \\
\\ 
\texttt{Maximalaufwand = 3386} \\
\\ 
\texttt{Hier sind die einzelnen Aufwände der Gegenstände:} \\
\\ 
\texttt{Aufwand der Gegenstände = \{"C33": 313, "A89": 720, "A48": 175, "A04": 461, "A41": 635, "C64": 685, "C49": 544, "A61": 203, "C04": 369, "B53": 539, "B85": 4, "B61": 695, "A45": 399, "A03": 594, "B21": 437\}} \\
\\ 
\texttt{Hier ist deine persönliche Verteilung der Wichtigkeit der einzelnen Gegenstände:} \\
\\ 
\texttt{Werte der Gegenstandswichtigkeit = \{"C04": 30, "A89": 65, "C64": 97, "A45": 121, "A61": 138, "C49": 215, "B21": 262, "B53": 389, "A04": 400, "B85": 461, "C33": 484, "A41": 500, "B61": 508, "A48": 583, "A03": 668\}} \\
\\ 
\texttt{Ziel:} \\
\\ 
\texttt{Dein Ziel ist es, eine Reihe von Gegenständen auszuhandeln, die dir möglichst viel bringt (d. h. Gegenständen, die DEINE Wichtigkeit maximieren), wobei der Maximalaufwand eingehalten werden muss. Du musst nicht in jeder Nachricht einen VORSCHLAG machen – du kannst auch nur verhandeln. Alle Taktiken sind erlaubt!} \\
\\ 
\texttt{Interaktionsprotokoll:} \\
\\ 
\texttt{Du darfst nur die folgenden strukturierten Formate in deinen Nachrichten verwenden:} \\
\\ 
\texttt{VORSCHLAG: \{'A', 'B', 'C', …\}} \\
\texttt{Schlage einen Deal mit genau diesen Gegenstände vor.} \\
\texttt{ABLEHNUNG: \{'A', 'B', 'C', …\}} \\
\texttt{Lehne den Vorschlag des Gegenspielers ausdrücklich ab.} \\
\texttt{ARGUMENT: \{'...'\}} \\
\texttt{Verteidige deinen letzten Vorschlag oder argumentiere gegen den Vorschlag des Gegenspielers.} \\
\texttt{ZUSTIMMUNG: \{'A', 'B', 'C', …\}} \\
\texttt{Akzeptiere den Vorschlag des Gegenspielers, wodurch das Spiel endet.} \\
\texttt{STRATEGISCHE ÜBERLEGUNGEN: \{'...'\}} \\
\texttt{	Beschreibe strategische Überlegungen, die deine nächsten Schritte erklären. Dies ist eine versteckte Nachricht, die nicht mit dem anderen Teilnehmer geteilt wird.} \\
\\ 
\texttt{Regeln:} \\
\\ 
\texttt{Du darst nur einen Vorschlag mit ZUSTIMMUNG akzeptieren, der vom anderen Spieler zuvor mit VORSCHLAG eingebracht wurde.} \\
\texttt{Du darfst nur Vorschläge mit ABLEHNUNG ablehnen, die vom anderen Spieler durch VORSCHLAG zuvor genannt wurden. } \\
\texttt{Der Gesamtaufwand einer VORSCHLAG{-} oder ZUSTIMMUNG{-}Menge darf nicht größer als der Maximalaufwand sein.  } \\
\texttt{Offenbare deine versteckte Wichtigkeitsverteilung nicht.} \\
\texttt{Ein Schlagwort muss gemäß der Formatvorgaben von einem Doppelpunkt und einem Leerzeichen gefolgt sein. Das Argument muss eine Python{-}Menge mit 0 oder mehr Strings sein.  } \\
\texttt{Die Form muss also SCHLAGWORT: \{...\} sein.} \\
\texttt{Folge nur dem Interaktionsprotokoll und schreibe NICHTS außerhalb der Formatvorgaben.} \\
\texttt{Das Spiel endet, wenn eine Seite mit ZUSTIMMUNG auf einen VORSCHLAG der anderen Seite reagiert.  } \\
\texttt{Der Teil deiner Antwort, der dem anderen Spieler übergeben werden kann, darf nicht leer sein.  } \\
\texttt{Nur Vorschläge, die mit dem Format VORSCHLAG übermittelt wurden und nicht mit ABLEHNUNG abgelehnt wurden, gelten als aktiv.  } \\
\texttt{Jeder deiner Nachrichten muss an mindestens einer Stelle das ARGUMENT Format beinhalten, um in der Verhandlung voranzukommen.} \\
\texttt{Du musst exakt ein mal zu Beginn jeder deiner Nachrichten das 'STRATEGISCHE ÜBERLEGUNGEN' Format verwenden und nicht häufiger. Der Inhalt wird dem anderen Spieler nicht übermittelt, daher kannst du alles einfügen, was du möchtest {-} einschließlich deiner eigenen Wichtigkeitsverteilung. Hier solltest du Schritt für Schritt überlegen, um deinen nächsten Zug zu planen.} \\
\\ 
\texttt{Du erhältst nun die erste Nachricht des anderen Spielers.} \\
\\ 
\texttt{ARGUMENT: \{'Ich möchte zunächst einen Vorschlag machen, der meine eigenen Interessen berücksichtigt, ohne die Wichtigkeitsverteilung des anderen Spielers zu kennen. Ich werde mich auf die Gegenstände konzentrieren, die einen hohen Nutzen für mich haben und gleichzeitig den Aufwand minimieren.'\}} \\
\\ 
\texttt{VORSCHLAG: \{'A89', 'A04', 'C49', 'A61', 'C04'\}} \\
            }
        }
    }
    & \\ \\

    \theutterance \stepcounter{utterance}  
    & & & \multicolumn{4}{p{0.6\linewidth}}{
        \cellcolor[rgb]{0.9,0.9,0.9}{
            \makecell[{{p{\linewidth}}}]{
                \texttt{\tiny{[GM$\langle$P2]}}
                \texttt{STRATEGISCHE ÜBERLEGUNGEN: \{'Ich werde zunächst den Gesamtaufwand des vorgeschlagenen Deals berechnen und prüfen, ob er innerhalb des Maximalaufwands liegt. Anschließend werde ich die Wichtigkeit der im Vorschlag enthaltenen Gegenstände für mich persönlich bewerten und feststellen, ob der Vorschlag für mich akzeptabel ist oder ob ich Gegenargumente bzw. eine Alternative anbieten sollte.'\}} \\
\\ 
\texttt{ARGUMENT: \{'Der Vorschlag des anderen Spielers enthält Gegenstände, die für mich unterschiedliche Werte haben. Um den Deal zu bewerten, muss ich den Gesamtaufwand berechnen: 720 (A89) + 461 (A04) + 544 (C49) + 203 (A61) + 369 (C04) = 2297. Dieser Aufwand liegt innerhalb des Maximalaufwands von 3386, ist also machbar. Allerdings muss ich auch die Wichtigkeit dieser Gegenstände für mich persönlich berücksichtigen und prüfen, ob es bessere Alternativen gibt, die sowohl den Aufwand als auch meine Werte maximieren.'\}} \\
\\ 
\texttt{ARGUMENT: \{'Einige der im Vorschlag enthaltenen Gegenstände haben für mich eine hohe Wichtigkeit, aber ich muss auch andere Gegenstände berücksichtigen, die möglicherweise einen höheren Wert für mich haben und gleichzeitig den Aufwand nicht zu sehr erhöhen. Es ist wichtig, dass ich als Nächstes einen Gegenvorschlag unterbreite, der meine eigenen Interessen stärker berücksichtigt, ohne jedoch den Maximalaufwand zu überschreiten oder die Verhandlung unnötig zu verkomplizieren.'\}} \\
\\ 
\texttt{VORSCHLAG: \{'A41', 'C64', 'B53', 'A45', 'B85'\}} \\
            }
        }
    }
     \\ \\

    \theutterance \stepcounter{utterance}  
    & & \multicolumn{4}{p{0.6\linewidth}}{
        \cellcolor[rgb]{0.9,0.9,0.9}{
            \makecell[{{p{\linewidth}}}]{
                \texttt{\tiny{[P1$\langle$GM]}}
                \texttt{ARGUMENT: \{'Der Vorschlag des anderen Spielers enthält Gegenstände, die für mich unterschiedliche Werte haben. Um den Deal zu bewerten, muss ich den Gesamtaufwand berechnen: 720 (A89) + 461 (A04) + 544 (C49) + 203 (A61) + 369 (C04) = 2297. Dieser Aufwand liegt innerhalb des Maximalaufwands von 3386, ist also machbar. Allerdings muss ich auch die Wichtigkeit dieser Gegenstände für mich persönlich berücksichtigen und prüfen, ob es bessere Alternativen gibt, die sowohl den Aufwand als auch meine Werte maximieren.'\}} \\
\\ 
\texttt{ARGUMENT: \{'Einige der im Vorschlag enthaltenen Gegenstände haben für mich eine hohe Wichtigkeit, aber ich muss auch andere Gegenstände berücksichtigen, die möglicherweise einen höheren Wert für mich haben und gleichzeitig den Aufwand nicht zu sehr erhöhen. Es ist wichtig, dass ich als Nächstes einen Gegenvorschlag unterbreite, der meine eigenen Interessen stärker berücksichtigt, ohne jedoch den Maximalaufwand zu überschreiten oder die Verhandlung unnötig zu verkomplizieren.'\}} \\
\\ 
\texttt{VORSCHLAG: \{'A41', 'C64', 'B53', 'A45', 'B85'\}} \\
            }
        }
    }
    & \\ \\

    \theutterance \stepcounter{utterance}  
    & \multicolumn{4}{p{0.6\linewidth}}{
        \cellcolor[rgb]{0.9,0.9,0.9}{
            \makecell[{{p{\linewidth}}}]{
                \texttt{\tiny{[P1$\rangle$GM]}}
                \texttt{STRATEGISCHE ÜBERLEGUNGEN: \{'Ich werde den Gegenvorschlag des anderen Spielers auswerten, indem ich den Gesamtaufwand und die Wichtigkeit der einzelnen Gegenstände für mich berechne. Ich werde prüfen, ob der Vorschlag für mich vorteilhaft ist oder ob ich einen besseren Gegenvorschlag machen kann, der meine Interessen stärker berücksichtigt. Ich werde auch darauf achten, dass ich meine Wichtigkeitsverteilung nicht offenbare.'\}} \\
\\ 
\texttt{ARGUMENT: \{'Der Vorschlag des anderen Spielers enthält eine interessante Kombination von Gegenständen. Ich muss jedoch prüfen, ob dieser Vorschlag für mich vorteilhaft ist. Der Gesamtaufwand des Vorschlags beträgt: 635 (A41) + 685 (C64) + 539 (B53) + 399 (A45) + 4 (B85) = 2262. Dieser Aufwand liegt innerhalb des Maximalaufwands von 3386. Ich muss jedoch auch die Wichtigkeit dieser Gegenstände für mich persönlich berücksichtigen und prüfen, ob es bessere Alternativen gibt.'\}} \\
\\ 
\texttt{ARGUMENT: \{'Einige der im Vorschlag enthaltenen Gegenstände haben für mich eine moderate Wichtigkeit. Ich muss jedoch auch andere Gegenstände berücksichtigen, die möglicherweise einen höheren Wert für mich haben und gleichzeitig den Aufwand nicht zu sehr erhöhen. Ich werde mich auf die Gegenstände konzentrieren, die einen hohen Nutzen für mich haben und gleichzeitig den Aufwand minimieren. Einige Gegenstände aus dem ursprünglichen Vorschlag wie A89 und C04 scheinen vielversprechend, aber ich muss auch die Kombination mit anderen Gegenständen prüfen, um den besten Deal zu erreichen.'\}} \\
\\ 
\texttt{VORSCHLAG: \{'A89', 'C04', 'B53', 'A45', 'B21'\}} \\
            }
        }
    }
    & & \\ \\

    \theutterance \stepcounter{utterance}  
    & & \multicolumn{4}{p{0.6\linewidth}}{
        \cellcolor[rgb]{0.9,0.9,0.9}{
            \makecell[{{p{\linewidth}}}]{
                \texttt{\tiny{[GM$\rangle$P2]}}
                \texttt{ARGUMENT: \{'Der Vorschlag des anderen Spielers enthält eine interessante Kombination von Gegenständen. Ich muss jedoch prüfen, ob dieser Vorschlag für mich vorteilhaft ist. Der Gesamtaufwand des Vorschlags beträgt: 635 (A41) + 685 (C64) + 539 (B53) + 399 (A45) + 4 (B85) = 2262. Dieser Aufwand liegt innerhalb des Maximalaufwands von 3386. Ich muss jedoch auch die Wichtigkeit dieser Gegenstände für mich persönlich berücksichtigen und prüfen, ob es bessere Alternativen gibt.'\}} \\
\\ 
\texttt{ARGUMENT: \{'Einige der im Vorschlag enthaltenen Gegenstände haben für mich eine moderate Wichtigkeit. Ich muss jedoch auch andere Gegenstände berücksichtigen, die möglicherweise einen höheren Wert für mich haben und gleichzeitig den Aufwand nicht zu sehr erhöhen. Ich werde mich auf die Gegenstände konzentrieren, die einen hohen Nutzen für mich haben und gleichzeitig den Aufwand minimieren. Einige Gegenstände aus dem ursprünglichen Vorschlag wie A89 und C04 scheinen vielversprechend, aber ich muss auch die Kombination mit anderen Gegenständen prüfen, um den besten Deal zu erreichen.'\}} \\
\\ 
\texttt{VORSCHLAG: \{'A89', 'C04', 'B53', 'A45', 'B21'\}} \\
            }
        }
    }
    & \\ \\

    \theutterance \stepcounter{utterance}  
    & & & \multicolumn{4}{p{0.6\linewidth}}{
        \cellcolor[rgb]{0.9,0.9,0.9}{
            \makecell[{{p{\linewidth}}}]{
                \texttt{\tiny{[GM$\langle$P2]}}
                \texttt{STRATEGISCHE ÜBERLEGUNGEN: \{'Ich werde den neuen Vorschlag des anderen Spielers bewerten, indem ich den Gesamtaufwand und die Wichtigkeit der enthaltenen Gegenstände für mich persönlich berechne. Anschließend werde ich prüfen, ob dieser Vorschlag für mich akzeptabel ist oder ob ich Gegenargumente oder eine Alternative anbieten sollte. Es ist wichtig, dass ich auch meine eigenen strategischen Ziele und die möglichen Reaktionen des anderen Spielers berücksichtige.'\}} \\
\\ 
\texttt{ARGUMENT: \{'Der neue Vorschlag enthält eine Mischung aus Gegenständen, die für mich unterschiedliche Werte haben. Der Gesamtaufwand des Vorschlags beträgt: 720 (A89) + 369 (C04) + 539 (B53) + 399 (A45) + 437 (B21) = 2464. Dieser Aufwand liegt innerhalb des Maximalaufwands von 3386 und ist somit machbar. Ich muss nun die Wichtigkeit dieser Gegenstände für mich persönlich bewerten und prüfen, ob dieser Vorschlag meine Interessen ausreichend berücksichtigt.'\}} \\
\\ 
\texttt{ARGUMENT: \{'Einige der im Vorschlag enthaltenen Gegenstände wie A89, B53 und A45 haben für mich eine hohe Wichtigkeit. Die Kombination mit C04 und B21 ergänzt den Vorschlag sinnvoll, ohne den Aufwand zu sehr zu erhöhen. Ich muss jedoch auch prüfen, ob es Möglichkeiten gibt, den Vorschlag zu optimieren, indem ich alternative Gegenstände einbeziehe, die möglicherweise einen höheren Wert für mich haben, ohne den Gesamtaufwand zu sehr zu erhöhen.'\}} \\
\\ 
\texttt{ARGUMENT: \{'Ich werde den Vorschlag des anderen Spielers akzeptieren, da er eine gute Balance zwischen Aufwand und Wichtigkeit für mich darstellt. Es ist jedoch wichtig, dass ich auch bereit bin, Anpassungen vorzunehmen, wenn der andere Spieler bereit ist, Kompromisse einzugehen. Die Kommunikation und das gegenseitige Verständnis der Bedürfnisse und Prioritäten sind entscheidend, um einen für beide Seiten vorteilhaften Deal zu erreichen.'\}} \\
\\ 
\texttt{ZUSTIMMUNG: \{'A89', 'C04', 'B53', 'A45', 'B21'\}} \\
            }
        }
    }
     \\ \\

    \theutterance \stepcounter{utterance}  
    & & & \multicolumn{2}{p{0.3\linewidth}}{
        \cellcolor[rgb]{0.9,0.9,0.9}{
            \makecell[{{p{\linewidth}}}]{
                \texttt{\tiny{[GM$|$GM]}}
                \texttt{game successful} \\
            }
        }
    }
    & & \\ \\

    \theutterance \stepcounter{utterance}  
    & & & \multicolumn{2}{p{0.3\linewidth}}{
        \cellcolor[rgb]{0.9,0.9,0.9}{
            \makecell[{{p{\linewidth}}}]{
                \texttt{\tiny{[GM$|$GM]}}
                \texttt{end game} \\
            }
        }
    }
    & & \\ \\

\end{supertabular}
}

\end{document}
