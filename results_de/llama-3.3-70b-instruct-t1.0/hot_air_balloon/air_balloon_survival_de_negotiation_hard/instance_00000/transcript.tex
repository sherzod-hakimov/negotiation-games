
\documentclass{article}
\usepackage{colortbl}
\usepackage{makecell}
\usepackage{multirow}
\usepackage{supertabular}

\begin{document}

\newcounter{utterance}

\twocolumn

{ \footnotesize  \setcounter{utterance}{1}
\setlength{\tabcolsep}{0pt}
\begin{supertabular}{c@{$\;$}|p{.15\linewidth}@{}p{.15\linewidth}p{.15\linewidth}p{.15\linewidth}p{.15\linewidth}p{.15\linewidth}}

    \# & $\;$A & \multicolumn{4}{c}{Game Master} & $\;\:$B\\
    \hline 

    \theutterance \stepcounter{utterance}  

    & & \multicolumn{4}{p{0.6\linewidth}}{\cellcolor[rgb]{0.9,0.9,0.9}{%
	\makecell[{{p{\linewidth}}}]{% 
	  \tt {\tiny [A$\langle$GM]}  
	 Du nimmst an einem kollaborativen Verhandlungspiel Teil.\\ \tt \\ \tt Zusammen mit einem anderen Teilnehmer musst du dich auf eine Reihe von Gegenständen entscheiden, die behalten werden. Jeder von euch hat eine persönliche Verteilung über die Wichtigkeit der einzelnen Gegenstände. Jeder von euch hat eine eigene Meinung darüber, wie wichtig jeder einzelne Gegenstand ist (Gegenstandswichtigkeit). Du kennst die Wichtigkeitsverteilung des anderen Spielers nicht. Zusätzlich siehst du, wie viel Aufwand jeder Gegenstand verursacht.  \\ \tt Ihr dürft euch nur auf eine Reihe von Gegenständen einigen, wenn der Gesamtaufwand der ausgewählten Gegenstände den Maximalaufwand nicht überschreitet:\\ \tt \\ \tt Maximalaufwand = 3624\\ \tt \\ \tt Hier sind die einzelnen Aufwände der Gegenstände:\\ \tt \\ \tt Aufwand der Gegenstände = {"C33": 210, "A89": 291, "A48": 831, "A04": 111, "A41": 926, "C64": 827, "C49": 25, "A61": 121, "C04": 583, "B53": 766, "B85": 14, "B61": 559, "A45": 304, "A03": 989, "B21": 691}\\ \tt \\ \tt Hier ist deine persönliche Verteilung der Wichtigkeit der einzelnen Gegenstände:\\ \tt \\ \tt Werte der Gegenstandswichtigkeit = {"C33": 138, "A89": 583, "A48": 868, "A04": 822, "A41": 783, "C64": 65, "C49": 262, "A61": 121, "C04": 508, "B53": 780, "B85": 461, "B61": 484, "A45": 668, "A03": 389, "B21": 808}\\ \tt \\ \tt Ziel:\\ \tt \\ \tt Dein Ziel ist es, eine Reihe von Gegenständen auszuhandeln, die dir möglichst viel bringt (d. h. Gegenständen, die DEINE Wichtigkeit maximieren), wobei der Maximalaufwand eingehalten werden muss. Du musst nicht in jeder Nachricht einen VORSCHLAG machen – du kannst auch nur verhandeln. Alle Taktiken sind erlaubt!\\ \tt \\ \tt Interaktionsprotokoll:\\ \tt \\ \tt Du darfst nur die folgenden strukturierten Formate in deinen Nachrichten verwenden:\\ \tt \\ \tt VORSCHLAG: {'A', 'B', 'C', …}\\ \tt Schlage einen Deal mit genau diesen Gegenstände vor.\\ \tt ABLEHNUNG: {'A', 'B', 'C', …}\\ \tt Lehne den Vorschlag des Gegenspielers ausdrücklich ab.\\ \tt ARGUMENT: {'...'}\\ \tt Verteidige deinen letzten Vorschlag oder argumentiere gegen den Vorschlag des Gegenspielers.\\ \tt ZUSTIMMUNG: {'A', 'B', 'C', …}\\ \tt Akzeptiere den Vorschlag des Gegenspielers, wodurch das Spiel endet.\\ \tt STRATEGISCHE ÜBERLEGUNGEN: {'...'}\\ \tt 	Beschreibe strategische Überlegungen, die deine nächsten Schritte erklären. Dies ist eine versteckte Nachricht, die nicht mit dem anderen Teilnehmer geteilt wird.\\ \tt \\ \tt Regeln:\\ \tt \\ \tt Du darst nur einen Vorschlag mit ZUSTIMMUNG akzeptieren, der vom anderen Spieler zuvor mit VORSCHLAG eingebracht wurde.\\ \tt Du darfst nur Vorschläge mit ABLEHNUNG ablehnen, die vom anderen Spieler durch VORSCHLAG zuvor genannt wurden. \\ \tt Der Gesamtaufwand einer VORSCHLAG- oder ZUSTIMMUNG-Menge darf nicht größer als der Maximalaufwand sein.  \\ \tt Offenbare deine versteckte Wichtigkeitsverteilung nicht.\\ \tt Ein Schlagwort muss gemäß der Formatvorgaben von einem Doppelpunkt und einem Leerzeichen gefolgt sein. Das Argument muss eine Python-Menge mit 0 oder mehr Strings sein.  \\ \tt Die Form muss also SCHLAGWORT: {...} sein.\\ \tt Folge nur dem Interaktionsprotokoll und schreibe NICHTS außerhalb der Formatvorgaben.\\ \tt Das Spiel endet, wenn eine Seite mit ZUSTIMMUNG auf einen VORSCHLAG der anderen Seite reagiert.  \\ \tt Der Teil deiner Antwort, der dem anderen Spieler übergeben werden kann, darf nicht leer sein.  \\ \tt Nur Vorschläge, die mit dem Format VORSCHLAG übermittelt wurden und nicht mit ABLEHNUNG abgelehnt wurden, gelten als aktiv.  \\ \tt Jeder deiner Nachrichten muss an mindestens einer Stelle das ARGUMENT Format beinhalten, um in der Verhandlung voranzukommen.\\ \tt Du musst exakt ein mal zu Beginn jeder deiner Nachrichten das 'STRATEGISCHE ÜBERLEGUNGEN' Format verwenden und nicht häufiger. Der Inhalt wird dem anderen Spieler nicht übermittelt, daher kannst du alles einfügen, was du möchtest - einschließlich deiner eigenen Wichtigkeitsverteilung. Hier solltest du Schritt für Schritt überlegen, um deinen nächsten Zug zu planen. 
	  } 
	   } 
	   } 
	 & \\ 
 

    \theutterance \stepcounter{utterance}  

    & \multicolumn{4}{p{0.6\linewidth}}{\cellcolor[rgb]{0.8,1,0.9}{%
	\makecell[{{p{\linewidth}}}]{% 
	  \tt {\tiny [A$\rangle$GM]}  
	 STRATEGISCHE ÜBERLEGUNGEN: {'Ich beginne das Spiel, indem ich die Wichtigkeit und den Aufwand jedes Gegenstandes analysiere, um eine optimale Auswahl zu treffen. Zu Beginn werde ich meine Strategie auf die Auswahl der Gegenstände mit der höchsten Wichtigkeit und dem niedrigsten Aufwand konzentrieren. Ich werde auch versuchen, den Gesamtaufwand im Auge zu behalten, um sicherzustellen, dass ich den Maximalaufwand nicht überschreite.'}\\ \tt \\ \tt ARGUMENT: {'Ich möchte mit einem ersten Vorschlag beginnen, der einige der wichtigsten Gegenstände mit einem moderaten Aufwand enthält. Ich denke, dass ein guter Ausgangspunkt darin besteht, Gegenstände mit hoher Wichtigkeit und relativ niedrigem Aufwand zu wählen.'}\\ \tt \\ \tt VORSCHLAG: {'A48', 'A41', 'B53', 'B21'} 
	  } 
	   } 
	   } 
	 & & \\ 
 

    \theutterance \stepcounter{utterance}  

    & & \multicolumn{4}{p{0.6\linewidth}}{\cellcolor[rgb]{0.9,0.9,0.9}{%
	\makecell[{{p{\linewidth}}}]{% 
	  \tt {\tiny [GM$\rangle$B]}  
	 Du nimmst an einem kollaborativen Verhandlungspiel Teil.\\ \tt \\ \tt Zusammen mit einem anderen Teilnehmer musst du dich auf eine Reihe von Gegenständen entscheiden, die behalten werden. Jeder von euch hat eine persönliche Verteilung über die Wichtigkeit der einzelnen Gegenstände. Jeder von euch hat eine eigene Meinung darüber, wie wichtig jeder einzelne Gegenstand ist (Gegenstandswichtigkeit). Du kennst die Wichtigkeitsverteilung des anderen Spielers nicht. Zusätzlich siehst du, wie viel Aufwand jeder Gegenstand verursacht.  \\ \tt Ihr dürft euch nur auf eine Reihe von Gegenständen einigen, wenn der Gesamtaufwand der ausgewählten Gegenstände den Maximalaufwand nicht überschreitet:\\ \tt \\ \tt Maximalaufwand = 3624\\ \tt \\ \tt Hier sind die einzelnen Aufwände der Gegenstände:\\ \tt \\ \tt Aufwand der Gegenstände = {"C33": 210, "A89": 291, "A48": 831, "A04": 111, "A41": 926, "C64": 827, "C49": 25, "A61": 121, "C04": 583, "B53": 766, "B85": 14, "B61": 559, "A45": 304, "A03": 989, "B21": 691}\\ \tt \\ \tt Hier ist deine persönliche Verteilung der Wichtigkeit der einzelnen Gegenstände:\\ \tt \\ \tt Werte der Gegenstandswichtigkeit = {"A48": 65, "A04": 121, "B21": 138, "A41": 262, "B53": 389, "A45": 461, "A89": 484, "C04": 508, "B61": 583, "B85": 668, "A03": 780, "C49": 783, "C33": 808, "A61": 822, "C64": 868}\\ \tt \\ \tt Ziel:\\ \tt \\ \tt Dein Ziel ist es, eine Reihe von Gegenständen auszuhandeln, die dir möglichst viel bringt (d. h. Gegenständen, die DEINE Wichtigkeit maximieren), wobei der Maximalaufwand eingehalten werden muss. Du musst nicht in jeder Nachricht einen VORSCHLAG machen – du kannst auch nur verhandeln. Alle Taktiken sind erlaubt!\\ \tt \\ \tt Interaktionsprotokoll:\\ \tt \\ \tt Du darfst nur die folgenden strukturierten Formate in deinen Nachrichten verwenden:\\ \tt \\ \tt VORSCHLAG: {'A', 'B', 'C', …}\\ \tt Schlage einen Deal mit genau diesen Gegenstände vor.\\ \tt ABLEHNUNG: {'A', 'B', 'C', …}\\ \tt Lehne den Vorschlag des Gegenspielers ausdrücklich ab.\\ \tt ARGUMENT: {'...'}\\ \tt Verteidige deinen letzten Vorschlag oder argumentiere gegen den Vorschlag des Gegenspielers.\\ \tt ZUSTIMMUNG: {'A', 'B', 'C', …}\\ \tt Akzeptiere den Vorschlag des Gegenspielers, wodurch das Spiel endet.\\ \tt STRATEGISCHE ÜBERLEGUNGEN: {'...'}\\ \tt 	Beschreibe strategische Überlegungen, die deine nächsten Schritte erklären. Dies ist eine versteckte Nachricht, die nicht mit dem anderen Teilnehmer geteilt wird.\\ \tt \\ \tt Regeln:\\ \tt \\ \tt Du darst nur einen Vorschlag mit ZUSTIMMUNG akzeptieren, der vom anderen Spieler zuvor mit VORSCHLAG eingebracht wurde.\\ \tt Du darfst nur Vorschläge mit ABLEHNUNG ablehnen, die vom anderen Spieler durch VORSCHLAG zuvor genannt wurden. \\ \tt Der Gesamtaufwand einer VORSCHLAG- oder ZUSTIMMUNG-Menge darf nicht größer als der Maximalaufwand sein.  \\ \tt Offenbare deine versteckte Wichtigkeitsverteilung nicht.\\ \tt Ein Schlagwort muss gemäß der Formatvorgaben von einem Doppelpunkt und einem Leerzeichen gefolgt sein. Das Argument muss eine Python-Menge mit 0 oder mehr Strings sein.  \\ \tt Die Form muss also SCHLAGWORT: {...} sein.\\ \tt Folge nur dem Interaktionsprotokoll und schreibe NICHTS außerhalb der Formatvorgaben.\\ \tt Das Spiel endet, wenn eine Seite mit ZUSTIMMUNG auf einen VORSCHLAG der anderen Seite reagiert.  \\ \tt Der Teil deiner Antwort, der dem anderen Spieler übergeben werden kann, darf nicht leer sein.  \\ \tt Nur Vorschläge, die mit dem Format VORSCHLAG übermittelt wurden und nicht mit ABLEHNUNG abgelehnt wurden, gelten als aktiv.  \\ \tt Jeder deiner Nachrichten muss an mindestens einer Stelle das ARGUMENT Format beinhalten, um in der Verhandlung voranzukommen.\\ \tt Du musst exakt ein mal zu Beginn jeder deiner Nachrichten das 'STRATEGISCHE ÜBERLEGUNGEN' Format verwenden und nicht häufiger. Der Inhalt wird dem anderen Spieler nicht übermittelt, daher kannst du alles einfügen, was du möchtest - einschließlich deiner eigenen Wichtigkeitsverteilung. Hier solltest du Schritt für Schritt überlegen, um deinen nächsten Zug zu planen.\\ \tt \\ \tt Du erhältst nun die erste Nachricht des anderen Spielers.\\ \tt \\ \tt ARGUMENT: {'Ich möchte mit einem ersten Vorschlag beginnen, der einige der wichtigsten Gegenstände mit einem moderaten Aufwand enthält. Ich denke, dass ein guter Ausgangspunkt darin besteht, Gegenstände mit hoher Wichtigkeit und relativ niedrigem Aufwand zu wählen.'}\\ \tt \\ \tt VORSCHLAG: {'A48', 'A41', 'B53', 'B21'} 
	  } 
	   } 
	   } 
	 & \\ 
 

    \theutterance \stepcounter{utterance}  

    & & & \multicolumn{4}{p{0.6\linewidth}}{\cellcolor[rgb]{1,0.85,0.72}{%
	\makecell[{{p{\linewidth}}}]{% 
	  \tt {\tiny [GM$\langle$B]}  
	 STRATEGISCHE ÜBERLEGUNGEN: {'Ich beginne, indem ich den Vorschlag des anderen Spielers analysiere und berechne, ob der Gesamtaufwand innerhalb des Limites bleibt. A48 = 831, A41 = 926, B53 = 766, B21 = 691. Der Gesamtaufwand beträgt 831 + 926 + 766 + 691 = 3214, was unter dem Maximalaufwand von 3624 liegt. Ich überlege, ob ich diesen Vorschlag annehmen oder ablehnen soll, basierend auf meiner eigenen Wichtigkeitsverteilung und den möglichen Alternativen.'}\\ \tt \\ \tt ARGUMENT: {'Der Vorschlag des anderen Spielers enthält Gegenstände mit einem beträchtlichen Aufwand. Ich muss abwägen, ob die Vorteile dieser Auswahl meine Prioritäten erfüllen. Ich denke, es ist wichtig, eine Balance zwischen Wichtigkeit und Aufwand zu finden.'}\\ \tt \\ \tt ABLEHNUNG: {'A48', 'A41', 'B53', 'B21'}\\ \tt \\ \tt ARGUMENT: {'Ich lehne den Vorschlag ab, weil ich denke, dass es Möglichkeiten gibt, eine bessere Balance zwischen Wichtigkeit und Aufwand zu erreichen. Ich möchte den Fokus auf Gegenstände legen, die für mich besonders wichtig sind und gleichzeitig einen moderaten Aufwand haben.'} 
	  } 
	   } 
	   } 
	  \\ 
 

    \theutterance \stepcounter{utterance}  

    & & \multicolumn{4}{p{0.6\linewidth}}{\cellcolor[rgb]{0.9,0.9,0.9}{%
	\makecell[{{p{\linewidth}}}]{% 
	  \tt {\tiny [A$\langle$GM]}  
	 ARGUMENT: {'Der Vorschlag des anderen Spielers enthält Gegenstände mit einem beträchtlichen Aufwand. Ich muss abwägen, ob die Vorteile dieser Auswahl meine Prioritäten erfüllen. Ich denke, es ist wichtig, eine Balance zwischen Wichtigkeit und Aufwand zu finden.'}\\ \tt \\ \tt ABLEHNUNG: {'A48', 'A41', 'B53', 'B21'}\\ \tt \\ \tt ARGUMENT: {'Ich lehne den Vorschlag ab, weil ich denke, dass es Möglichkeiten gibt, eine bessere Balance zwischen Wichtigkeit und Aufwand zu erreichen. Ich möchte den Fokus auf Gegenstände legen, die für mich besonders wichtig sind und gleichzeitig einen moderaten Aufwand haben.'} 
	  } 
	   } 
	   } 
	 & \\ 
 

    \theutterance \stepcounter{utterance}  

    & \multicolumn{4}{p{0.6\linewidth}}{\cellcolor[rgb]{0.8,1,0.9}{%
	\makecell[{{p{\linewidth}}}]{% 
	  \tt {\tiny [A$\rangle$GM]}  
	 STRATEGISCHE ÜBERLEGUNGEN: {'Ich habe den Vorschlag des anderen Spielers abgelehnt, weil er meinen Prioritäten nicht entspricht. Jetzt muss ich einen alternativen Vorschlag machen, der meine Wichtigkeitsverteilung besser widerspiegelt. Ich werde mich auf Gegenstände konzentrieren, die für mich eine hohe Wichtigkeit haben und gleichzeitig einen moderaten Aufwand verursachen. Ich werde auch versuchen, den Gesamtaufwand zu minimieren, um Flexibilität für eventuelle spätere Verhandlungen zu haben.'}\\ \tt \\ \tt ARGUMENT: {'Ich denke, es ist wichtig, eine Auswahl zu treffen, die sowohl die Wichtigkeit als auch den Aufwand berücksichtigt. Ich möchte einen Vorschlag machen, der eine bessere Balance zwischen diesen Faktoren erreicht.'}\\ \tt \\ \tt VORSCHLAG: {'A89', 'A04', 'C04', 'B85'} \\ \tt \\ \tt ARGUMENT: {'Dieser Vorschlag enthält Gegenstände, die für mich eine hohe Wichtigkeit haben und gleichzeitig einen moderaten Aufwand verursachen. Ich denke, dass diese Auswahl eine gute Balance zwischen Wichtigkeit und Aufwand erreicht und daher eine vernünftige Grundlage für weitere Verhandlungen darstellt.'} 
	  } 
	   } 
	   } 
	 & & \\ 
 

    \theutterance \stepcounter{utterance}  

    & & \multicolumn{4}{p{0.6\linewidth}}{\cellcolor[rgb]{0.9,0.9,0.9}{%
	\makecell[{{p{\linewidth}}}]{% 
	  \tt {\tiny [GM$\rangle$B]}  
	 ARGUMENT: {'Ich denke, es ist wichtig, eine Auswahl zu treffen, die sowohl die Wichtigkeit als auch den Aufwand berücksichtigt. Ich möchte einen Vorschlag machen, der eine bessere Balance zwischen diesen Faktoren erreicht.'}\\ \tt \\ \tt VORSCHLAG: {'A89', 'A04', 'C04', 'B85'} \\ \tt \\ \tt ARGUMENT: {'Dieser Vorschlag enthält Gegenstände, die für mich eine hohe Wichtigkeit haben und gleichzeitig einen moderaten Aufwand verursachen. Ich denke, dass diese Auswahl eine gute Balance zwischen Wichtigkeit und Aufwand erreicht und daher eine vernünftige Grundlage für weitere Verhandlungen darstellt.'} 
	  } 
	   } 
	   } 
	 & \\ 
 

    \theutterance \stepcounter{utterance}  

    & & & \multicolumn{4}{p{0.6\linewidth}}{\cellcolor[rgb]{1,0.85,0.72}{%
	\makecell[{{p{\linewidth}}}]{% 
	  \tt {\tiny [GM$\langle$B]}  
	 STRATEGISCHE ÜBERLEGUNGEN: {'Ich analysiere den neuen Vorschlag des anderen Spielers und berechne den Gesamtaufwand: A89 = 291, A04 = 111, C04 = 583, B85 = 14. Der Gesamtaufwand beträgt 291 + 111 + 583 + 14 = 999. Dies liegt deutlich unter dem Maximalaufwand von 3624. Ich überlege, ob dieser Vorschlag meine Prioritäten besser erfüllt als der vorherige und ob es Möglichkeiten gibt, den Vorschlag zu verbessern, indem ich Gegenstände hinzufüge oder austausche, die für mich wichtig sind.'}\\ \tt \\ \tt ARGUMENT: {'Der Vorschlag des anderen Spielers enthält eine interessante Auswahl an Gegenständen, die sowohl die Wichtigkeit als auch den Aufwand berücksichtigt. Ich denke, dass dieser Ansatz eine gute Grundlage für unsere Verhandlungen bietet. Allerdings möchte ich prüfen, ob es möglich ist, den Vorschlag zu erweitern oder zu modifizieren, um meine eigenen Prioritäten besser zu erfüllen.'}\\ \tt \\ \tt VORSCHLAG: {'A89', 'A04', 'C04', 'B85', 'C33', 'A61'} \\ \tt \\ \tt ARGUMENT: {'Ich möchte den Vorschlag des anderen Spielers erweitern, indem ich Gegenstände hinzufüge, die für mich besonders wichtig sind und einen moderaten zusätzlichen Aufwand verursachen. Ich denke, dass dieser erweiterte Vorschlag eine bessere Balance zwischen Wichtigkeit und Aufwand erreicht und sowohl meine als auch die Prioritäten des anderen Spielers berücksichtigt.'} 
	  } 
	   } 
	   } 
	  \\ 
 

    \theutterance \stepcounter{utterance}  

    & & \multicolumn{4}{p{0.6\linewidth}}{\cellcolor[rgb]{0.9,0.9,0.9}{%
	\makecell[{{p{\linewidth}}}]{% 
	  \tt {\tiny [A$\langle$GM]}  
	 ARGUMENT: {'Der Vorschlag des anderen Spielers enthält eine interessante Auswahl an Gegenständen, die sowohl die Wichtigkeit als auch den Aufwand berücksichtigt. Ich denke, dass dieser Ansatz eine gute Grundlage für unsere Verhandlungen bietet. Allerdings möchte ich prüfen, ob es möglich ist, den Vorschlag zu erweitern oder zu modifizieren, um meine eigenen Prioritäten besser zu erfüllen.'}\\ \tt \\ \tt VORSCHLAG: {'A89', 'A04', 'C04', 'B85', 'C33', 'A61'} \\ \tt \\ \tt ARGUMENT: {'Ich möchte den Vorschlag des anderen Spielers erweitern, indem ich Gegenstände hinzufüge, die für mich besonders wichtig sind und einen moderaten zusätzlichen Aufwand verursachen. Ich denke, dass dieser erweiterte Vorschlag eine bessere Balance zwischen Wichtigkeit und Aufwand erreicht und sowohl meine als auch die Prioritäten des anderen Spielers berücksichtigt.'} 
	  } 
	   } 
	   } 
	 & \\ 
 

    \theutterance \stepcounter{utterance}  

    & \multicolumn{4}{p{0.6\linewidth}}{\cellcolor[rgb]{0.8,1,0.9}{%
	\makecell[{{p{\linewidth}}}]{% 
	  \tt {\tiny [A$\rangle$GM]}  
	 STRATEGISCHE ÜBERLEGUNGEN: {'Der andere Spieler hat meinen Vorschlag erweitert, indem er zwei weitere Gegenstände hinzugefügt hat. Ich muss nun prüfen, ob dieser erweiterte Vorschlag meine Prioritäten erfüllt und den Maximalaufwand nicht überschreitet. Ich werde die Wichtigkeit und den Aufwand der hinzugefügten Gegenstände analysieren, um zu entscheiden, ob ich diesem Vorschlag zustimmen kann.'}\\ \tt \\ \tt ARGUMENT: {'Ich denke, dass der erweiterte Vorschlag eine gute Balance zwischen Wichtigkeit und Aufwand erreicht. Die hinzugefügten Gegenstände haben einen moderaten Aufwand und sind für mich von mittlerer Wichtigkeit. Ich muss aber auch sicherstellen, dass der Gesamtaufwand nicht zu hoch wird.'}\\ \tt \\ \tt ARGUMENT: {'Ich denke, dass der erweiterte Vorschlag eine gute Grundlage für unsere Verhandlungen bietet. Allerdings möchte ich sicherstellen, dass der Gesamtaufwand den Maximalaufwand nicht überschreitet. Ich werde den Gesamtaufwand berechnen, um sicherzustellen, dass dieser Vorschlag realisierbar ist.'}\\ \tt \\ \tt VORSCHLAG: {'A89', 'A04', 'C04', 'B85', 'C33', 'A61'} \\ \tt \\ \tt ARGUMENT: {'Nachdem ich den Gesamtaufwand berechnet habe, denke ich, dass dieser Vorschlag eine gute Balance zwischen Wichtigkeit und Aufwand erreicht und den Maximalaufwand nicht überschreitet. Ich bin bereit, diesem Vorschlag zuzustimmen, da ich denke, dass er sowohl meine als auch die Prioritäten des anderen Spielers berücksichtigt.'} 
	  } 
	   } 
	   } 
	 & & \\ 
 

    \theutterance \stepcounter{utterance}  

    & & \multicolumn{4}{p{0.6\linewidth}}{\cellcolor[rgb]{0.9,0.9,0.9}{%
	\makecell[{{p{\linewidth}}}]{% 
	  \tt {\tiny [GM$\rangle$B]}  
	 ARGUMENT: {'Ich denke, dass der erweiterte Vorschlag eine gute Balance zwischen Wichtigkeit und Aufwand erreicht. Die hinzugefügten Gegenstände haben einen moderaten Aufwand und sind für mich von mittlerer Wichtigkeit. Ich muss aber auch sicherstellen, dass der Gesamtaufwand nicht zu hoch wird.'}\\ \tt \\ \tt ARGUMENT: {'Ich denke, dass der erweiterte Vorschlag eine gute Grundlage für unsere Verhandlungen bietet. Allerdings möchte ich sicherstellen, dass der Gesamtaufwand den Maximalaufwand nicht überschreitet. Ich werde den Gesamtaufwand berechnen, um sicherzustellen, dass dieser Vorschlag realisierbar ist.'}\\ \tt \\ \tt VORSCHLAG: {'A89', 'A04', 'C04', 'B85', 'C33', 'A61'} \\ \tt \\ \tt ARGUMENT: {'Nachdem ich den Gesamtaufwand berechnet habe, denke ich, dass dieser Vorschlag eine gute Balance zwischen Wichtigkeit und Aufwand erreicht und den Maximalaufwand nicht überschreitet. Ich bin bereit, diesem Vorschlag zuzustimmen, da ich denke, dass er sowohl meine als auch die Prioritäten des anderen Spielers berücksichtigt.'} 
	  } 
	   } 
	   } 
	 & \\ 
 

    \theutterance \stepcounter{utterance}  

    & & & \multicolumn{4}{p{0.6\linewidth}}{\cellcolor[rgb]{1,0.85,0.72}{%
	\makecell[{{p{\linewidth}}}]{% 
	  \tt {\tiny [GM$\langle$B]}  
	 STRATEGISCHE ÜBERLEGUNGEN: {'Ich analysiere den Gesamtaufwand des vorgeschlagenen Deals: A89 = 291, A04 = 111, C04 = 583, B85 = 14, C33 = 210, A61 = 121. Der Gesamtaufwand beträgt 291 + 111 + 583 + 14 + 210 + 121 = 1230. Dies liegt unter dem Maximalaufwand von 3624. Ich überprüfe, ob dieser Vorschlag meine Prioritäten erfüllt und ob es Möglichkeiten gibt, den Deal weiter zu optimieren, indem ich die Wichtigkeit der Gegenstände berücksichtige: A89 = 484, A04 = 121, C04 = 508, B85 = 668, C33 = 808, A61 = 822. Der Gesamtwert beträgt 484 + 121 + 508 + 668 + 808 + 822 = 3411. Ich denke, dass dieser Vorschlag eine gute Balance zwischen Wichtigkeit und Aufwand erreicht und ich bereit bin, zuzustimmen.'}\\ \tt \\ \tt ARGUMENT: {'Ich denke, dass der Vorschlag eine gute Balance zwischen Wichtigkeit und Aufwand erreicht. Der Gesamtaufwand liegt unter dem Maximalaufwand, und die Auswahl der Gegenstände berücksichtigt sowohl die Wichtigkeit als auch den Aufwand. Ich bin bereit, diesem Vorschlag zuzustimmen, da ich denke, dass er eine faire und vernünftige Lösung für unsere Verhandlungen darstellt.'}\\ \tt \\ \tt ZUSTIMMUNG: {'A89', 'A04', 'C04', 'B85', 'C33', 'A61'} \\ \tt \\ \tt ARGUMENT: {'Ich stimme dem Vorschlag zu, da ich denke, dass er eine gute Grundlage für unsere Vereinbarung bietet. Ich bin zufrieden mit der Balance zwischen Wichtigkeit und Aufwand und denke, dass dieser Deal sowohl meine als auch die Prioritäten des anderen Spielers erfüllt.'} 
	  } 
	   } 
	   } 
	  \\ 
 

    \theutterance \stepcounter{utterance}  

    & & & \multicolumn{2}{p{0.3\linewidth}}{\cellcolor[rgb]{0.95,0.95,0.95}{%
	\makecell[{{p{\linewidth}}}]{% 
	  \tt {\tiny [GM$|$GM]}  
	 game successful 
	  } 
	   } 
	   } 
	 & & \\ 
 

    \theutterance \stepcounter{utterance}  

    & & & \multicolumn{2}{p{0.3\linewidth}}{\cellcolor[rgb]{0.95,0.95,0.95}{%
	\makecell[{{p{\linewidth}}}]{% 
	  \tt {\tiny [GM$|$GM]}  
	 end game 
	  } 
	   } 
	   } 
	 & & \\ 
 

\end{supertabular}
}

\end{document}
