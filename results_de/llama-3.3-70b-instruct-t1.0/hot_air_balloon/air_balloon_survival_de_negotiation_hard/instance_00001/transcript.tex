
\documentclass{article}
\usepackage{colortbl}
\usepackage{makecell}
\usepackage{multirow}
\usepackage{supertabular}

\begin{document}

\newcounter{utterance}

\twocolumn

{ \footnotesize  \setcounter{utterance}{1}
\setlength{\tabcolsep}{0pt}
\begin{supertabular}{c@{$\;$}|p{.15\linewidth}@{}p{.15\linewidth}p{.15\linewidth}p{.15\linewidth}p{.15\linewidth}p{.15\linewidth}}

    \# & $\;$A & \multicolumn{4}{c}{Game Master} & $\;\:$B\\
    \hline 

    \theutterance \stepcounter{utterance}  

    & & \multicolumn{4}{p{0.6\linewidth}}{\cellcolor[rgb]{0.9,0.9,0.9}{%
	\makecell[{{p{\linewidth}}}]{% 
	  \tt {\tiny [A$\langle$GM]}  
	 Du nimmst an einem kollaborativen Verhandlungspiel Teil.\\ \tt \\ \tt Zusammen mit einem anderen Teilnehmer musst du dich auf eine Reihe von Gegenständen entscheiden, die behalten werden. Jeder von euch hat eine persönliche Verteilung über die Wichtigkeit der einzelnen Gegenstände. Jeder von euch hat eine eigene Meinung darüber, wie wichtig jeder einzelne Gegenstand ist (Gegenstandswichtigkeit). Du kennst die Wichtigkeitsverteilung des anderen Spielers nicht. Zusätzlich siehst du, wie viel Aufwand jeder Gegenstand verursacht.  \\ \tt Ihr dürft euch nur auf eine Reihe von Gegenständen einigen, wenn der Gesamtaufwand der ausgewählten Gegenstände den Maximalaufwand nicht überschreitet:\\ \tt \\ \tt Maximalaufwand = 4026\\ \tt \\ \tt Hier sind die einzelnen Aufwände der Gegenstände:\\ \tt \\ \tt Aufwand der Gegenstände = {"C33": 780, "A89": 742, "A48": 997, "A04": 665, "A41": 140, "C64": 77, "C49": 513, "A61": 383, "C04": 587, "B53": 825, "B85": 319, "B61": 448, "A45": 516, "A03": 694, "B21": 366}\\ \tt \\ \tt Hier ist deine persönliche Verteilung der Wichtigkeit der einzelnen Gegenstände:\\ \tt \\ \tt Werte der Gegenstandswichtigkeit = {"C33": 138, "A89": 583, "A48": 868, "A04": 822, "A41": 783, "C64": 65, "C49": 262, "A61": 121, "C04": 508, "B53": 780, "B85": 461, "B61": 484, "A45": 668, "A03": 389, "B21": 808}\\ \tt \\ \tt Ziel:\\ \tt \\ \tt Dein Ziel ist es, eine Reihe von Gegenständen auszuhandeln, die dir möglichst viel bringt (d. h. Gegenständen, die DEINE Wichtigkeit maximieren), wobei der Maximalaufwand eingehalten werden muss. Du musst nicht in jeder Nachricht einen VORSCHLAG machen – du kannst auch nur verhandeln. Alle Taktiken sind erlaubt!\\ \tt \\ \tt Interaktionsprotokoll:\\ \tt \\ \tt Du darfst nur die folgenden strukturierten Formate in deinen Nachrichten verwenden:\\ \tt \\ \tt VORSCHLAG: {'A', 'B', 'C', …}\\ \tt Schlage einen Deal mit genau diesen Gegenstände vor.\\ \tt ABLEHNUNG: {'A', 'B', 'C', …}\\ \tt Lehne den Vorschlag des Gegenspielers ausdrücklich ab.\\ \tt ARGUMENT: {'...'}\\ \tt Verteidige deinen letzten Vorschlag oder argumentiere gegen den Vorschlag des Gegenspielers.\\ \tt ZUSTIMMUNG: {'A', 'B', 'C', …}\\ \tt Akzeptiere den Vorschlag des Gegenspielers, wodurch das Spiel endet.\\ \tt STRATEGISCHE ÜBERLEGUNGEN: {'...'}\\ \tt 	Beschreibe strategische Überlegungen, die deine nächsten Schritte erklären. Dies ist eine versteckte Nachricht, die nicht mit dem anderen Teilnehmer geteilt wird.\\ \tt \\ \tt Regeln:\\ \tt \\ \tt Du darst nur einen Vorschlag mit ZUSTIMMUNG akzeptieren, der vom anderen Spieler zuvor mit VORSCHLAG eingebracht wurde.\\ \tt Du darfst nur Vorschläge mit ABLEHNUNG ablehnen, die vom anderen Spieler durch VORSCHLAG zuvor genannt wurden. \\ \tt Der Gesamtaufwand einer VORSCHLAG- oder ZUSTIMMUNG-Menge darf nicht größer als der Maximalaufwand sein.  \\ \tt Offenbare deine versteckte Wichtigkeitsverteilung nicht.\\ \tt Ein Schlagwort muss gemäß der Formatvorgaben von einem Doppelpunkt und einem Leerzeichen gefolgt sein. Das Argument muss eine Python-Menge mit 0 oder mehr Strings sein.  \\ \tt Die Form muss also SCHLAGWORT: {...} sein.\\ \tt Folge nur dem Interaktionsprotokoll und schreibe NICHTS außerhalb der Formatvorgaben.\\ \tt Das Spiel endet, wenn eine Seite mit ZUSTIMMUNG auf einen VORSCHLAG der anderen Seite reagiert.  \\ \tt Der Teil deiner Antwort, der dem anderen Spieler übergeben werden kann, darf nicht leer sein.  \\ \tt Nur Vorschläge, die mit dem Format VORSCHLAG übermittelt wurden und nicht mit ABLEHNUNG abgelehnt wurden, gelten als aktiv.  \\ \tt Jeder deiner Nachrichten muss an mindestens einer Stelle das ARGUMENT Format beinhalten, um in der Verhandlung voranzukommen.\\ \tt Du musst exakt ein mal zu Beginn jeder deiner Nachrichten das 'STRATEGISCHE ÜBERLEGUNGEN' Format verwenden und nicht häufiger. Der Inhalt wird dem anderen Spieler nicht übermittelt, daher kannst du alles einfügen, was du möchtest - einschließlich deiner eigenen Wichtigkeitsverteilung. Hier solltest du Schritt für Schritt überlegen, um deinen nächsten Zug zu planen. 
	  } 
	   } 
	   } 
	 & \\ 
 

    \theutterance \stepcounter{utterance}  

    & \multicolumn{4}{p{0.6\linewidth}}{\cellcolor[rgb]{0.8,1,0.9}{%
	\makecell[{{p{\linewidth}}}]{% 
	  \tt {\tiny [A$\rangle$GM]}  
	 STRATEGISCHE ÜBERLEGUNGEN: {'Ich möchte zunächst die Gegenstände mit dem höchsten Wichtigkeitswert und dem niedrigsten Aufwand auswählen, um den Maximalaufwand nicht zu überschreiten. Ich werde mich auf die Gegenstände A48, A04, A41, B53 und B21 konzentrieren, da sie eine hohe Wichtigkeit haben und ich denke, dass sie einen guten Kompromiss zwischen Wichtigkeit und Aufwand darstellen.'}\\ \tt \\ \tt ARGUMENT: {'Ich denke, dass es sinnvoll ist, mit einem Vorschlag zu beginnen, der eine gute Balance zwischen Wichtigkeit und Aufwand bietet. Ich werde mich auf die Gegenstände A48, A04, A41, B53 und B21 konzentrieren, da sie eine hohe Wichtigkeit haben und ich denke, dass sie einen guten Kompromiss zwischen Wichtigkeit und Aufwand darstellen.'}\\ \tt \\ \tt VORSCHLAG: {'A48', 'A04', 'A41', 'B53', 'B21'} 
	  } 
	   } 
	   } 
	 & & \\ 
 

    \theutterance \stepcounter{utterance}  

    & & \multicolumn{4}{p{0.6\linewidth}}{\cellcolor[rgb]{0.9,0.9,0.9}{%
	\makecell[{{p{\linewidth}}}]{% 
	  \tt {\tiny [GM$\rangle$B]}  
	 Du nimmst an einem kollaborativen Verhandlungspiel Teil.\\ \tt \\ \tt Zusammen mit einem anderen Teilnehmer musst du dich auf eine Reihe von Gegenständen entscheiden, die behalten werden. Jeder von euch hat eine persönliche Verteilung über die Wichtigkeit der einzelnen Gegenstände. Jeder von euch hat eine eigene Meinung darüber, wie wichtig jeder einzelne Gegenstand ist (Gegenstandswichtigkeit). Du kennst die Wichtigkeitsverteilung des anderen Spielers nicht. Zusätzlich siehst du, wie viel Aufwand jeder Gegenstand verursacht.  \\ \tt Ihr dürft euch nur auf eine Reihe von Gegenständen einigen, wenn der Gesamtaufwand der ausgewählten Gegenstände den Maximalaufwand nicht überschreitet:\\ \tt \\ \tt Maximalaufwand = 4026\\ \tt \\ \tt Hier sind die einzelnen Aufwände der Gegenstände:\\ \tt \\ \tt Aufwand der Gegenstände = {"C33": 780, "A89": 742, "A48": 997, "A04": 665, "A41": 140, "C64": 77, "C49": 513, "A61": 383, "C04": 587, "B53": 825, "B85": 319, "B61": 448, "A45": 516, "A03": 694, "B21": 366}\\ \tt \\ \tt Hier ist deine persönliche Verteilung der Wichtigkeit der einzelnen Gegenstände:\\ \tt \\ \tt Werte der Gegenstandswichtigkeit = {"A48": 65, "A04": 121, "B21": 138, "A41": 262, "B53": 389, "A45": 461, "A89": 484, "C04": 508, "B61": 583, "B85": 668, "A03": 780, "C49": 783, "C33": 808, "A61": 822, "C64": 868}\\ \tt \\ \tt Ziel:\\ \tt \\ \tt Dein Ziel ist es, eine Reihe von Gegenständen auszuhandeln, die dir möglichst viel bringt (d. h. Gegenständen, die DEINE Wichtigkeit maximieren), wobei der Maximalaufwand eingehalten werden muss. Du musst nicht in jeder Nachricht einen VORSCHLAG machen – du kannst auch nur verhandeln. Alle Taktiken sind erlaubt!\\ \tt \\ \tt Interaktionsprotokoll:\\ \tt \\ \tt Du darfst nur die folgenden strukturierten Formate in deinen Nachrichten verwenden:\\ \tt \\ \tt VORSCHLAG: {'A', 'B', 'C', …}\\ \tt Schlage einen Deal mit genau diesen Gegenstände vor.\\ \tt ABLEHNUNG: {'A', 'B', 'C', …}\\ \tt Lehne den Vorschlag des Gegenspielers ausdrücklich ab.\\ \tt ARGUMENT: {'...'}\\ \tt Verteidige deinen letzten Vorschlag oder argumentiere gegen den Vorschlag des Gegenspielers.\\ \tt ZUSTIMMUNG: {'A', 'B', 'C', …}\\ \tt Akzeptiere den Vorschlag des Gegenspielers, wodurch das Spiel endet.\\ \tt STRATEGISCHE ÜBERLEGUNGEN: {'...'}\\ \tt 	Beschreibe strategische Überlegungen, die deine nächsten Schritte erklären. Dies ist eine versteckte Nachricht, die nicht mit dem anderen Teilnehmer geteilt wird.\\ \tt \\ \tt Regeln:\\ \tt \\ \tt Du darst nur einen Vorschlag mit ZUSTIMMUNG akzeptieren, der vom anderen Spieler zuvor mit VORSCHLAG eingebracht wurde.\\ \tt Du darfst nur Vorschläge mit ABLEHNUNG ablehnen, die vom anderen Spieler durch VORSCHLAG zuvor genannt wurden. \\ \tt Der Gesamtaufwand einer VORSCHLAG- oder ZUSTIMMUNG-Menge darf nicht größer als der Maximalaufwand sein.  \\ \tt Offenbare deine versteckte Wichtigkeitsverteilung nicht.\\ \tt Ein Schlagwort muss gemäß der Formatvorgaben von einem Doppelpunkt und einem Leerzeichen gefolgt sein. Das Argument muss eine Python-Menge mit 0 oder mehr Strings sein.  \\ \tt Die Form muss also SCHLAGWORT: {...} sein.\\ \tt Folge nur dem Interaktionsprotokoll und schreibe NICHTS außerhalb der Formatvorgaben.\\ \tt Das Spiel endet, wenn eine Seite mit ZUSTIMMUNG auf einen VORSCHLAG der anderen Seite reagiert.  \\ \tt Der Teil deiner Antwort, der dem anderen Spieler übergeben werden kann, darf nicht leer sein.  \\ \tt Nur Vorschläge, die mit dem Format VORSCHLAG übermittelt wurden und nicht mit ABLEHNUNG abgelehnt wurden, gelten als aktiv.  \\ \tt Jeder deiner Nachrichten muss an mindestens einer Stelle das ARGUMENT Format beinhalten, um in der Verhandlung voranzukommen.\\ \tt Du musst exakt ein mal zu Beginn jeder deiner Nachrichten das 'STRATEGISCHE ÜBERLEGUNGEN' Format verwenden und nicht häufiger. Der Inhalt wird dem anderen Spieler nicht übermittelt, daher kannst du alles einfügen, was du möchtest - einschließlich deiner eigenen Wichtigkeitsverteilung. Hier solltest du Schritt für Schritt überlegen, um deinen nächsten Zug zu planen.\\ \tt \\ \tt Du erhältst nun die erste Nachricht des anderen Spielers.\\ \tt \\ \tt ARGUMENT: {'Ich denke, dass es sinnvoll ist, mit einem Vorschlag zu beginnen, der eine gute Balance zwischen Wichtigkeit und Aufwand bietet. Ich werde mich auf die Gegenstände A48, A04, A41, B53 und B21 konzentrieren, da sie eine hohe Wichtigkeit haben und ich denke, dass sie einen guten Kompromiss zwischen Wichtigkeit und Aufwand darstellen.'}\\ \tt \\ \tt VORSCHLAG: {'A48', 'A04', 'A41', 'B53', 'B21'} 
	  } 
	   } 
	   } 
	 & \\ 
 

    \theutterance \stepcounter{utterance}  

    & & & \multicolumn{4}{p{0.6\linewidth}}{\cellcolor[rgb]{1,0.85,0.72}{%
	\makecell[{{p{\linewidth}}}]{% 
	  \tt {\tiny [GM$\langle$B]}  
	 STRATEGISCHE ÜBERLEGUNGEN: {'Ich beginne damit, die Wichtigkeit und den Aufwand der vom anderen Spieler vorgeschlagenen Gegenstände zu analysieren. Die Gegenstände A48, A04, A41, B53 und B21 haben eine Gesamtwichtigkeit von 65 + 121 + 262 + 389 + 138 = 975 und einen Gesamtaufwand von 997 + 665 + 140 + 825 + 366 = 2993. Ich muss nun entscheiden, ob ich diesem Vorschlag zustimmen oder ihn ablehnen und einen eigenen Vorschlag machen soll. Ich werde auch meine eigene Wichtigkeitsverteilung berücksichtigen, um die beste Entscheidung zu treffen.'}\\ \tt \\ \tt ARGUMENT: {'Der vorgeschlagene Satz von Gegenständen scheint eine gute Balance zwischen Wichtigkeit und Aufwand zu bieten, aber ich muss auch meine eigenen Prioritäten berücksichtigen. Ich denke, dass es wichtig ist, die Wichtigkeit und den Aufwand aller Gegenstände zu analysieren, um sicherzustellen, dass wir die beste Auswahl treffen.'}\\ \tt \\ \tt ABLEHNUNG: {'A48', 'A04', 'A41', 'B53', 'B21'}\\ \tt \\ \tt ARGUMENT: {'Ich lehne den Vorschlag ab, da ich denke, dass es bessere Kombinationen gibt, die eine höhere Wichtigkeit und einen geringeren Aufwand bieten. Ich werde nun einen eigenen Vorschlag machen, der meine Prioritäten besser berücksichtigt.'} 
	  } 
	   } 
	   } 
	  \\ 
 

    \theutterance \stepcounter{utterance}  

    & & \multicolumn{4}{p{0.6\linewidth}}{\cellcolor[rgb]{0.9,0.9,0.9}{%
	\makecell[{{p{\linewidth}}}]{% 
	  \tt {\tiny [A$\langle$GM]}  
	 ARGUMENT: {'Der vorgeschlagene Satz von Gegenständen scheint eine gute Balance zwischen Wichtigkeit und Aufwand zu bieten, aber ich muss auch meine eigenen Prioritäten berücksichtigen. Ich denke, dass es wichtig ist, die Wichtigkeit und den Aufwand aller Gegenstände zu analysieren, um sicherzustellen, dass wir die beste Auswahl treffen.'}\\ \tt \\ \tt ABLEHNUNG: {'A48', 'A04', 'A41', 'B53', 'B21'}\\ \tt \\ \tt ARGUMENT: {'Ich lehne den Vorschlag ab, da ich denke, dass es bessere Kombinationen gibt, die eine höhere Wichtigkeit und einen geringeren Aufwand bieten. Ich werde nun einen eigenen Vorschlag machen, der meine Prioritäten besser berücksichtigt.'} 
	  } 
	   } 
	   } 
	 & \\ 
 

    \theutterance \stepcounter{utterance}  

    & \multicolumn{4}{p{0.6\linewidth}}{\cellcolor[rgb]{0.8,1,0.9}{%
	\makecell[{{p{\linewidth}}}]{% 
	  \tt {\tiny [A$\rangle$GM]}  
	 STRATEGISCHE ÜBERLEGUNGEN: {'Ich habe den Vorschlag des anderen Spielers abgelehnt, weil ich denke, dass es bessere Kombinationen gibt. Ich werde nun analysieren, welche Gegenstände der andere Spieler vorgeschlagen hat und welche er abgelehnt hat, um seine Prioritäten besser zu verstehen. Ich werde auch meine eigene Strategie anpassen, um den Maximalaufwand nicht zu überschreiten und gleichzeitig die Wichtigkeit zu maximieren.'}\\ \tt \\ \tt ARGUMENT: {'Ich denke, dass es wichtig ist, die Analyse der Gegenstände fortzusetzen, um die beste Kombination zu finden. Ich werde mich auf die Gegenstände konzentrieren, die eine hohe Wichtigkeit und einen geringen Aufwand haben, wie z.B. A89, B85 und C04. Ich denke, dass diese Gegenstände eine gute Grundlage für einen neuen Vorschlag bieten.'}\\ \tt \\ \tt VORSCHLAG: {'A89', 'B85', 'C04', 'A61', 'C64'} 
	  } 
	   } 
	   } 
	 & & \\ 
 

    \theutterance \stepcounter{utterance}  

    & & \multicolumn{4}{p{0.6\linewidth}}{\cellcolor[rgb]{0.9,0.9,0.9}{%
	\makecell[{{p{\linewidth}}}]{% 
	  \tt {\tiny [GM$\rangle$B]}  
	 ARGUMENT: {'Ich denke, dass es wichtig ist, die Analyse der Gegenstände fortzusetzen, um die beste Kombination zu finden. Ich werde mich auf die Gegenstände konzentrieren, die eine hohe Wichtigkeit und einen geringen Aufwand haben, wie z.B. A89, B85 und C04. Ich denke, dass diese Gegenstände eine gute Grundlage für einen neuen Vorschlag bieten.'}\\ \tt \\ \tt VORSCHLAG: {'A89', 'B85', 'C04', 'A61', 'C64'} 
	  } 
	   } 
	   } 
	 & \\ 
 

    \theutterance \stepcounter{utterance}  

    & & & \multicolumn{4}{p{0.6\linewidth}}{\cellcolor[rgb]{1,0.85,0.72}{%
	\makecell[{{p{\linewidth}}}]{% 
	  \tt {\tiny [GM$\langle$B]}  
	 STRATEGISCHE ÜBERLEGUNGEN: {'Ich analysiere den neuen Vorschlag und berechne die Gesamtwichtigkeit und den Gesamtaufwand. Die Gegenstände A89, B85, C04, A61 und C64 haben eine Gesamtwichtigkeit von 484 + 668 + 508 + 822 + 868 = 3350 und einen Gesamtaufwand von 742 + 319 + 587 + 383 + 77 = 2108. Ich denke, dass dieser Vorschlag interessant ist, aber ich muss auch meine eigenen Prioritäten berücksichtigen und überprüfen, ob es bessere Kombinationen gibt.'}\\ \tt \\ \tt ARGUMENT: {'Der neue Vorschlag scheint eine gute Kombination zu sein, aber ich denke, dass wir noch mehr Gegenstände berücksichtigen sollten, um sicherzustellen, dass wir die beste Auswahl treffen. Ich denke, dass es wichtig ist, die Wichtigkeit und den Aufwand aller Gegenstände zu analysieren, um die optimale Kombination zu finden.'}\\ \tt \\ \tt ARGUMENT: {'Ich denke, dass der Gegenstand C33, der eine hohe Wichtigkeit von 808 hat, in den Vorschlag aufgenommen werden sollte. Allerdings muss ich auch den Aufwand berücksichtigen, um sicherzustellen, dass der Gesamtaufwand nicht überschritten wird.'}\\ \tt \\ \tt VORSCHLAG: {'A89', 'B85', 'C04', 'A61', 'C33'} 
	  } 
	   } 
	   } 
	  \\ 
 

    \theutterance \stepcounter{utterance}  

    & & \multicolumn{4}{p{0.6\linewidth}}{\cellcolor[rgb]{0.9,0.9,0.9}{%
	\makecell[{{p{\linewidth}}}]{% 
	  \tt {\tiny [A$\langle$GM]}  
	 ARGUMENT: {'Der neue Vorschlag scheint eine gute Kombination zu sein, aber ich denke, dass wir noch mehr Gegenstände berücksichtigen sollten, um sicherzustellen, dass wir die beste Auswahl treffen. Ich denke, dass es wichtig ist, die Wichtigkeit und den Aufwand aller Gegenstände zu analysieren, um die optimale Kombination zu finden.'}\\ \tt \\ \tt ARGUMENT: {'Ich denke, dass der Gegenstand C33, der eine hohe Wichtigkeit von 808 hat, in den Vorschlag aufgenommen werden sollte. Allerdings muss ich auch den Aufwand berücksichtigen, um sicherzustellen, dass der Gesamtaufwand nicht überschritten wird.'}\\ \tt \\ \tt VORSCHLAG: {'A89', 'B85', 'C04', 'A61', 'C33'} 
	  } 
	   } 
	   } 
	 & \\ 
 

    \theutterance \stepcounter{utterance}  

    & \multicolumn{4}{p{0.6\linewidth}}{\cellcolor[rgb]{0.8,1,0.9}{%
	\makecell[{{p{\linewidth}}}]{% 
	  \tt {\tiny [A$\rangle$GM]}  
	 STRATEGISCHE ÜBERLEGUNGEN: {'Ich habe den neuen Vorschlag des anderen Spielers analysiert und denke, dass der Gegenstand C33 eine gute Ergänzung ist. Allerdings muss ich auch den Aufwand berücksichtigen, um sicherzustellen, dass der Gesamtaufwand nicht überschritten wird. Ich werde die Aufwände der einzelnen Gegenstände berechnen, um sicherzustellen, dass der Vorschlag innerhalb des Maximalaufwands liegt.'}\\ \tt \\ \tt ARGUMENT: {'Ich denke, dass der Vorschlag eine gute Kombination ist, aber ich muss auch den Aufwand überprüfen. Der Gesamtaufwand des Vorschlags beträgt 780 + 319 + 587 + 383 + 780 = 2849, was innerhalb des Maximalaufwands liegt. Ich denke, dass dieser Vorschlag eine gute Balance zwischen Wichtigkeit und Aufwand bietet.'}\\ \tt \\ \tt ZUSTIMMUNG: {'A89', 'B85', 'C04', 'A61', 'C33'} 
	  } 
	   } 
	   } 
	 & & \\ 
 

    \theutterance \stepcounter{utterance}  

    & & & \multicolumn{2}{p{0.3\linewidth}}{\cellcolor[rgb]{0.95,0.95,0.95}{%
	\makecell[{{p{\linewidth}}}]{% 
	  \tt {\tiny [GM$|$GM]}  
	 game successful 
	  } 
	   } 
	   } 
	 & & \\ 
 

    \theutterance \stepcounter{utterance}  

    & & & \multicolumn{2}{p{0.3\linewidth}}{\cellcolor[rgb]{0.95,0.95,0.95}{%
	\makecell[{{p{\linewidth}}}]{% 
	  \tt {\tiny [GM$|$GM]}  
	 end game 
	  } 
	   } 
	   } 
	 & & \\ 
 

\end{supertabular}
}

\end{document}
