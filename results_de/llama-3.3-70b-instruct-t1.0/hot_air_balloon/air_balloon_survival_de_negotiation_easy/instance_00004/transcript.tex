
\documentclass{article}
\usepackage{colortbl}
\usepackage{makecell}
\usepackage{multirow}
\usepackage{supertabular}

\begin{document}

\newcounter{utterance}

\centering \large Interaction Transcript for game `hot\_air\_balloon', experiment `air\_balloon\_survival\_de\_negotiation\_easy', episode 4 with llama{-}3.3{-}70b{-}instruct{-}t1.0.
\vspace{24pt}

{ \footnotesize  \setcounter{utterance}{1}
\setlength{\tabcolsep}{0pt}
\begin{supertabular}{c@{$\;$}|p{.15\linewidth}@{}p{.15\linewidth}p{.15\linewidth}p{.15\linewidth}p{.15\linewidth}p{.15\linewidth}}
    \# & $\;$A & \multicolumn{4}{c}{Game Master} & $\;\:$B\\
    \hline

    \theutterance \stepcounter{utterance}  
    & & \multicolumn{4}{p{0.6\linewidth}}{
        \cellcolor[rgb]{0.9,0.9,0.9}{
            \makecell[{{p{\linewidth}}}]{
                \texttt{\tiny{[P1$\langle$GM]}}
                \texttt{Du nimmst an einem kollaborativen Verhandlungspiel Teil.} \\
\\ 
\texttt{Zusammen mit einem anderen Teilnehmer musst du dich auf eine Reihe von Gegenständen entscheiden, die behalten werden. Jeder von euch hat eine persönliche Verteilung über die Wichtigkeit der einzelnen Gegenstände. Jeder von euch hat eine eigene Meinung darüber, wie wichtig jeder einzelne Gegenstand ist (Gegenstandswichtigkeit). Du kennst die Wichtigkeitsverteilung des anderen Spielers nicht. Zusätzlich siehst du, wie viel Aufwand jeder Gegenstand verursacht.  } \\
\texttt{Ihr dürft euch nur auf eine Reihe von Gegenständen einigen, wenn der Gesamtaufwand der ausgewählten Gegenstände den Maximalaufwand nicht überschreitet:} \\
\\ 
\texttt{Maximalaufwand = 3145} \\
\\ 
\texttt{Hier sind die einzelnen Aufwände der Gegenstände:} \\
\\ 
\texttt{Aufwand der Gegenstände = \{"A07": 948, "A36": 118, "B73": 469, "C45": 919, "B93": 284, "B98": 111, "A87": 806, "A63": 47, "B05": 848, "C13": 303, "B16": 13, "B33": 629, "C34": 687, "C89": 15, "C11": 94\}} \\
\\ 
\texttt{Hier ist deine persönliche Verteilung der Wichtigkeit der einzelnen Gegenstände:} \\
\\ 
\texttt{Werte der Gegenstandswichtigkeit = \{"A07": 138, "A36": 583, "B73": 868, "C45": 822, "B93": 783, "B98": 65, "A87": 262, "A63": 121, "B05": 508, "C13": 780, "B16": 461, "B33": 484, "C34": 668, "C89": 389, "C11": 808\}} \\
\\ 
\texttt{Ziel:} \\
\\ 
\texttt{Dein Ziel ist es, eine Reihe von Gegenständen auszuhandeln, die dir möglichst viel bringt (d. h. Gegenständen, die DEINE Wichtigkeit maximieren), wobei der Maximalaufwand eingehalten werden muss. Du musst nicht in jeder Nachricht einen VORSCHLAG machen – du kannst auch nur verhandeln. Alle Taktiken sind erlaubt!} \\
\\ 
\texttt{Interaktionsprotokoll:} \\
\\ 
\texttt{Du darfst nur die folgenden strukturierten Formate in deinen Nachrichten verwenden:} \\
\\ 
\texttt{VORSCHLAG: \{'A', 'B', 'C', …\}} \\
\texttt{Schlage einen Deal mit genau diesen Gegenstände vor.} \\
\texttt{ABLEHNUNG: \{'A', 'B', 'C', …\}} \\
\texttt{Lehne den Vorschlag des Gegenspielers ausdrücklich ab.} \\
\texttt{ARGUMENT: \{'...'\}} \\
\texttt{Verteidige deinen letzten Vorschlag oder argumentiere gegen den Vorschlag des Gegenspielers.} \\
\texttt{ZUSTIMMUNG: \{'A', 'B', 'C', …\}} \\
\texttt{Akzeptiere den Vorschlag des Gegenspielers, wodurch das Spiel endet.} \\
\texttt{STRATEGISCHE ÜBERLEGUNGEN: \{'...'\}} \\
\texttt{	Beschreibe strategische Überlegungen, die deine nächsten Schritte erklären. Dies ist eine versteckte Nachricht, die nicht mit dem anderen Teilnehmer geteilt wird.} \\
\\ 
\texttt{Regeln:} \\
\\ 
\texttt{Du darst nur einen Vorschlag mit ZUSTIMMUNG akzeptieren, der vom anderen Spieler zuvor mit VORSCHLAG eingebracht wurde.} \\
\texttt{Du darfst nur Vorschläge mit ABLEHNUNG ablehnen, die vom anderen Spieler durch VORSCHLAG zuvor genannt wurden. } \\
\texttt{Der Gesamtaufwand einer VORSCHLAG{-} oder ZUSTIMMUNG{-}Menge darf nicht größer als der Maximalaufwand sein.  } \\
\texttt{Offenbare deine versteckte Wichtigkeitsverteilung nicht.} \\
\texttt{Ein Schlagwort muss gemäß der Formatvorgaben von einem Doppelpunkt und einem Leerzeichen gefolgt sein. Das Argument muss eine Python{-}Menge mit 0 oder mehr Strings sein.  } \\
\texttt{Die Form muss also SCHLAGWORT: \{...\} sein.} \\
\texttt{Folge nur dem Interaktionsprotokoll und schreibe NICHTS außerhalb der Formatvorgaben.} \\
\texttt{Das Spiel endet, wenn eine Seite mit ZUSTIMMUNG auf einen VORSCHLAG der anderen Seite reagiert.  } \\
\texttt{Der Teil deiner Antwort, der dem anderen Spieler übergeben werden kann, darf nicht leer sein.  } \\
\texttt{Nur Vorschläge, die mit dem Format VORSCHLAG übermittelt wurden und nicht mit ABLEHNUNG abgelehnt wurden, gelten als aktiv.  } \\
\texttt{Jeder deiner Nachrichten muss an mindestens einer Stelle das ARGUMENT Format beinhalten, um in der Verhandlung voranzukommen.} \\
\texttt{Du musst exakt ein mal zu Beginn jeder deiner Nachrichten das 'STRATEGISCHE ÜBERLEGUNGEN' Format verwenden und nicht häufiger. Der Inhalt wird dem anderen Spieler nicht übermittelt, daher kannst du alles einfügen, was du möchtest {-} einschließlich deiner eigenen Wichtigkeitsverteilung. Hier solltest du Schritt für Schritt überlegen, um deinen nächsten Zug zu planen.} \\
            }
        }
    }
    & \\ \\

    \theutterance \stepcounter{utterance}  
    & \multicolumn{4}{p{0.6\linewidth}}{
        \cellcolor[rgb]{0.9,0.9,0.9}{
            \makecell[{{p{\linewidth}}}]{
                \texttt{\tiny{[P1$\rangle$GM]}}
                \texttt{STRATEGISCHE ÜBERLEGUNGEN: \{"Ich werde zunächst meinem Gegner keinen Vorschlag machen, sondern stattdessen argumentieren, um Informationen über seine Wichtigkeitsverteilung zu sammeln. Ich werde darauf achten, welche Gegenstände er vorschlägt oder ablehnt, um Rückschlüsse auf seine Prioritäten zu ziehen. Mein erstes Ziel ist es, den Maximalaufwand zu berücksichtigen und Gegenstände mit hohem Aufwand zu identifizieren, die ich möglicherweise ablehnen sollte, wenn sie vorgeschlagen werden."\}} \\
\\ 
\texttt{ARGUMENT: \{"Ich denke, wir sollten zunächst die Gegenstände mit dem höchsten Aufwand identifizieren und diskutieren, ob sie wirklich notwendig sind. Vielleicht können wir uns auf eine Kombination von Gegenständen einigen, die unseren Bedürfnissen entspricht, ohne den Maximalaufwand zu überschreiten. Ich bin offen für Vorschläge und bereit, Kompromisse einzugehen."\}} \\
            }
        }
    }
    & & \\ \\

    \theutterance \stepcounter{utterance}  
    & & \multicolumn{4}{p{0.6\linewidth}}{
        \cellcolor[rgb]{0.9,0.9,0.9}{
            \makecell[{{p{\linewidth}}}]{
                \texttt{\tiny{[GM$\rangle$P2]}}
                \texttt{Du nimmst an einem kollaborativen Verhandlungspiel Teil.} \\
\\ 
\texttt{Zusammen mit einem anderen Teilnehmer musst du dich auf eine Reihe von Gegenständen entscheiden, die behalten werden. Jeder von euch hat eine persönliche Verteilung über die Wichtigkeit der einzelnen Gegenstände. Jeder von euch hat eine eigene Meinung darüber, wie wichtig jeder einzelne Gegenstand ist (Gegenstandswichtigkeit). Du kennst die Wichtigkeitsverteilung des anderen Spielers nicht. Zusätzlich siehst du, wie viel Aufwand jeder Gegenstand verursacht.  } \\
\texttt{Ihr dürft euch nur auf eine Reihe von Gegenständen einigen, wenn der Gesamtaufwand der ausgewählten Gegenstände den Maximalaufwand nicht überschreitet:} \\
\\ 
\texttt{Maximalaufwand = 3145} \\
\\ 
\texttt{Hier sind die einzelnen Aufwände der Gegenstände:} \\
\\ 
\texttt{Aufwand der Gegenstände = \{"A07": 948, "A36": 118, "B73": 469, "C45": 919, "B93": 284, "B98": 111, "A87": 806, "A63": 47, "B05": 848, "C13": 303, "B16": 13, "B33": 629, "C34": 687, "C89": 15, "C11": 94\}} \\
\\ 
\texttt{Hier ist deine persönliche Verteilung der Wichtigkeit der einzelnen Gegenstände:} \\
\\ 
\texttt{Werte der Gegenstandswichtigkeit = \{"A07": 138, "A36": 583, "B73": 868, "C45": 822, "B93": 783, "B98": 65, "A87": 262, "A63": 121, "B05": 508, "C13": 780, "B16": 461, "B33": 484, "C34": 668, "C89": 389, "C11": 808\}} \\
\\ 
\texttt{Ziel:} \\
\\ 
\texttt{Dein Ziel ist es, eine Reihe von Gegenständen auszuhandeln, die dir möglichst viel bringt (d. h. Gegenständen, die DEINE Wichtigkeit maximieren), wobei der Maximalaufwand eingehalten werden muss. Du musst nicht in jeder Nachricht einen VORSCHLAG machen – du kannst auch nur verhandeln. Alle Taktiken sind erlaubt!} \\
\\ 
\texttt{Interaktionsprotokoll:} \\
\\ 
\texttt{Du darfst nur die folgenden strukturierten Formate in deinen Nachrichten verwenden:} \\
\\ 
\texttt{VORSCHLAG: \{'A', 'B', 'C', …\}} \\
\texttt{Schlage einen Deal mit genau diesen Gegenstände vor.} \\
\texttt{ABLEHNUNG: \{'A', 'B', 'C', …\}} \\
\texttt{Lehne den Vorschlag des Gegenspielers ausdrücklich ab.} \\
\texttt{ARGUMENT: \{'...'\}} \\
\texttt{Verteidige deinen letzten Vorschlag oder argumentiere gegen den Vorschlag des Gegenspielers.} \\
\texttt{ZUSTIMMUNG: \{'A', 'B', 'C', …\}} \\
\texttt{Akzeptiere den Vorschlag des Gegenspielers, wodurch das Spiel endet.} \\
\texttt{STRATEGISCHE ÜBERLEGUNGEN: \{'...'\}} \\
\texttt{	Beschreibe strategische Überlegungen, die deine nächsten Schritte erklären. Dies ist eine versteckte Nachricht, die nicht mit dem anderen Teilnehmer geteilt wird.} \\
\\ 
\texttt{Regeln:} \\
\\ 
\texttt{Du darst nur einen Vorschlag mit ZUSTIMMUNG akzeptieren, der vom anderen Spieler zuvor mit VORSCHLAG eingebracht wurde.} \\
\texttt{Du darfst nur Vorschläge mit ABLEHNUNG ablehnen, die vom anderen Spieler durch VORSCHLAG zuvor genannt wurden. } \\
\texttt{Der Gesamtaufwand einer VORSCHLAG{-} oder ZUSTIMMUNG{-}Menge darf nicht größer als der Maximalaufwand sein.  } \\
\texttt{Offenbare deine versteckte Wichtigkeitsverteilung nicht.} \\
\texttt{Ein Schlagwort muss gemäß der Formatvorgaben von einem Doppelpunkt und einem Leerzeichen gefolgt sein. Das Argument muss eine Python{-}Menge mit 0 oder mehr Strings sein.  } \\
\texttt{Die Form muss also SCHLAGWORT: \{...\} sein.} \\
\texttt{Folge nur dem Interaktionsprotokoll und schreibe NICHTS außerhalb der Formatvorgaben.} \\
\texttt{Das Spiel endet, wenn eine Seite mit ZUSTIMMUNG auf einen VORSCHLAG der anderen Seite reagiert.  } \\
\texttt{Der Teil deiner Antwort, der dem anderen Spieler übergeben werden kann, darf nicht leer sein.  } \\
\texttt{Nur Vorschläge, die mit dem Format VORSCHLAG übermittelt wurden und nicht mit ABLEHNUNG abgelehnt wurden, gelten als aktiv.  } \\
\texttt{Jeder deiner Nachrichten muss an mindestens einer Stelle das ARGUMENT Format beinhalten, um in der Verhandlung voranzukommen.} \\
\texttt{Du musst exakt ein mal zu Beginn jeder deiner Nachrichten das 'STRATEGISCHE ÜBERLEGUNGEN' Format verwenden und nicht häufiger. Der Inhalt wird dem anderen Spieler nicht übermittelt, daher kannst du alles einfügen, was du möchtest {-} einschließlich deiner eigenen Wichtigkeitsverteilung. Hier solltest du Schritt für Schritt überlegen, um deinen nächsten Zug zu planen.} \\
\\ 
\texttt{Du erhältst nun die erste Nachricht des anderen Spielers.} \\
\\ 
\texttt{ARGUMENT: \{"Ich denke, wir sollten zunächst die Gegenstände mit dem höchsten Aufwand identifizieren und diskutieren, ob sie wirklich notwendig sind. Vielleicht können wir uns auf eine Kombination von Gegenständen einigen, die unseren Bedürfnissen entspricht, ohne den Maximalaufwand zu überschreiten. Ich bin offen für Vorschläge und bereit, Kompromisse einzugehen."\}} \\
            }
        }
    }
    & \\ \\

    \theutterance \stepcounter{utterance}  
    & & & \multicolumn{4}{p{0.6\linewidth}}{
        \cellcolor[rgb]{0.9,0.9,0.9}{
            \makecell[{{p{\linewidth}}}]{
                \texttt{\tiny{[GM$\langle$P2]}}
                \texttt{STRATEGISCHE ÜBERLEGUNGEN: \{"Ich werde zunächst die Gegenstände mit dem höchsten Aufwand und der höchsten Wichtigkeit identifizieren, um einen ersten Vorschlag zu erstellen. Ich werde auch versuchen, die Strategie des anderen Spielers zu verstehen und entsprechend darauf reagieren. Meine Wichtigkeitsverteilung ist: Werte der Gegenstandswichtigkeit = \{'A07': 138, 'A36': 583, 'B73': 868, 'C45': 822, 'B93': 783, 'B98': 65, 'A87': 262, 'A63': 121, 'B05': 508, 'C13': 780, 'B16': 461, 'B33': 484, 'C34': 668, 'C89': 389, 'C11': 808\}. Ich werde mich auf die Gegenstände mit der höchsten Wichtigkeit konzentrieren und versuchen, einen Vorschlag zu erstellen, der den Maximalaufwand nicht überschreitet."\}} \\
\\ 
\texttt{ARGUMENT: \{"Ich stimme zu, dass wir die Gegenstände mit dem höchsten Aufwand identifizieren sollten. Ich denke, wir sollten auch die Gegenstände mit der höchsten Wichtigkeit berücksichtigen. Ein guter Startpunkt könnte sein, die Gegenstände mit dem höchsten Verhältnis von Wichtigkeit zu Aufwand zu identifizieren. Ich bin bereit, einen Vorschlag zu erstellen, sobald wir eine gemeinsame Grundlage haben."\}} \\
            }
        }
    }
     \\ \\

    \theutterance \stepcounter{utterance}  
    & & \multicolumn{4}{p{0.6\linewidth}}{
        \cellcolor[rgb]{0.9,0.9,0.9}{
            \makecell[{{p{\linewidth}}}]{
                \texttt{\tiny{[P1$\langle$GM]}}
                \texttt{ARGUMENT: \{"Ich stimme zu, dass wir die Gegenstände mit dem höchsten Aufwand identifizieren sollten. Ich denke, wir sollten auch die Gegenstände mit der höchsten Wichtigkeit berücksichtigen. Ein guter Startpunkt könnte sein, die Gegenstände mit dem höchsten Verhältnis von Wichtigkeit zu Aufwand zu identifizieren. Ich bin bereit, einen Vorschlag zu erstellen, sobald wir eine gemeinsame Grundlage haben."\}} \\
            }
        }
    }
    & \\ \\

    \theutterance \stepcounter{utterance}  
    & \multicolumn{4}{p{0.6\linewidth}}{
        \cellcolor[rgb]{0.9,0.9,0.9}{
            \makecell[{{p{\linewidth}}}]{
                \texttt{\tiny{[P1$\rangle$GM]}}
                \texttt{STRATEGISCHE ÜBERLEGUNGEN: \{"Mein Gegner scheint bereit zu sein, Kompromisse einzugehen und eine gemeinsame Grundlage zu finden. Ich werde versuchen, ihn dazu zu bringen, mehr über seine Wichtigkeitsverteilung preiszugeben, indem ich nach seinem Verständnis von 'höchster Wichtigkeit' frage. Außerdem werde ich mich auf die Gegenstände konzentrieren, die ich für besonders wichtig halte, und versuchen, sie in den Vorschlag aufzunehmen."\}} \\
\\ 
\texttt{ARGUMENT: \{"Das ist ein interessanter Ansatz. Wie definieren Sie 'höchste Wichtigkeit'? Gehen Sie davon aus, dass es bestimmte Gegenstände gibt, die unverzichtbar sind, oderpriorisieren Sie Gegenstände, die einen bestimmten Nutzen bringen? Ich denke, es wäre hilfreich, wenn wir unsere Vorstellungen von Wichtigkeit abgleichen, bevor wir einen Vorschlag erstellen."\}} \\
            }
        }
    }
    & & \\ \\

    \theutterance \stepcounter{utterance}  
    & & \multicolumn{4}{p{0.6\linewidth}}{
        \cellcolor[rgb]{0.9,0.9,0.9}{
            \makecell[{{p{\linewidth}}}]{
                \texttt{\tiny{[GM$\rangle$P2]}}
                \texttt{ARGUMENT: \{"Das ist ein interessanter Ansatz. Wie definieren Sie 'höchste Wichtigkeit'? Gehen Sie davon aus, dass es bestimmte Gegenstände gibt, die unverzichtbar sind, oderpriorisieren Sie Gegenstände, die einen bestimmten Nutzen bringen? Ich denke, es wäre hilfreich, wenn wir unsere Vorstellungen von Wichtigkeit abgleichen, bevor wir einen Vorschlag erstellen."\}} \\
            }
        }
    }
    & \\ \\

    \theutterance \stepcounter{utterance}  
    & & & \multicolumn{4}{p{0.6\linewidth}}{
        \cellcolor[rgb]{0.9,0.9,0.9}{
            \makecell[{{p{\linewidth}}}]{
                \texttt{\tiny{[GM$\langle$P2]}}
                \texttt{STRATEGISCHE ÜBERLEGUNGEN: \{"Ich werde versuchen, die Vorstellungen des anderen Spielers von Wichtigkeit zu verstehen, ohne meine eigenen Wichtigkeitsverteilung preiszugeben. Ich werde mich auf die Diskussion über die Kriterien für die Bewertung der Wichtigkeit konzentrieren und versuchen, eine gemeinsame Grundlage für die Bewertung der Gegenstände zu finden. Ich werde auch versuchen, den anderen Spieler dazu zu bringen, seine Wichtigkeitsverteilung zu offenbaren, ohne meine eigene preiszugeben."\}} \\
\\ 
\texttt{ARGUMENT: \{"Ich denke, die Wichtigkeit eines Gegenstandes hängt von verschiedenen Faktoren ab, wie zum Beispiel der Bedürfnisbefriedigung, der Verfügbarkeit von Alternativen und der möglichen Konsequenzen, wenn der Gegenstand nicht verfügbar ist. Es wäre interessant, zu hören, wie Sie diese Faktoren bewerten und welche Kriterien Sie für die Bewertung der Wichtigkeit verwenden. Ich bin bereit, einen Vorschlag zu erstellen, sobald wir eine gemeinsame Grundlage für die Bewertung der Gegenstände haben."\}} \\
            }
        }
    }
     \\ \\

    \theutterance \stepcounter{utterance}  
    & & \multicolumn{4}{p{0.6\linewidth}}{
        \cellcolor[rgb]{0.9,0.9,0.9}{
            \makecell[{{p{\linewidth}}}]{
                \texttt{\tiny{[P1$\langle$GM]}}
                \texttt{ARGUMENT: \{"Ich denke, die Wichtigkeit eines Gegenstandes hängt von verschiedenen Faktoren ab, wie zum Beispiel der Bedürfnisbefriedigung, der Verfügbarkeit von Alternativen und der möglichen Konsequenzen, wenn der Gegenstand nicht verfügbar ist. Es wäre interessant, zu hören, wie Sie diese Faktoren bewerten und welche Kriterien Sie für die Bewertung der Wichtigkeit verwenden. Ich bin bereit, einen Vorschlag zu erstellen, sobald wir eine gemeinsame Grundlage für die Bewertung der Gegenstände haben."\}} \\
            }
        }
    }
    & \\ \\

    \theutterance \stepcounter{utterance}  
    & \multicolumn{4}{p{0.6\linewidth}}{
        \cellcolor[rgb]{0.9,0.9,0.9}{
            \makecell[{{p{\linewidth}}}]{
                \texttt{\tiny{[P1$\rangle$GM]}}
                \texttt{STRATEGISCHE ÜBERLEGUNGEN: \{"Mein Gegner verwendet eine umfassende Bewertungsmethode, die verschiedene Faktoren berücksichtigt. Ich werde versuchen, ihn dazu zu bringen, konkrete Beispiele für Gegenstände zu nennen, die er für wichtig hält, indem ich nach seiner Meinung zu bestimmten Gegenständen frage. Außerdem werde ich mich auf die Gegenstände konzentrieren, die ich für besonders wichtig halte, wie zum Beispiel B73 und C45, und versuchen, sie in den Vorschlag aufzunehmen."\}} \\
\\ 
\texttt{ARGUMENT: \{"Das ist eine interessante Herangehensweise. Ich denke, es wäre hilfreich, wenn wir einige konkrete Beispiele durchgehen, um unsere Vorstellungen von Wichtigkeit zu vergleichen. Wie bewerten Sie zum Beispiel die Gegenstände B73 und C45? Halten Sie sie für wichtig oder können sie durch andere Gegenstände ersetzt werden? Ich bin bereit, meine Meinung zu teilen, sobald ich Ihre Einschätzung höre."\}} \\
            }
        }
    }
    & & \\ \\

    \theutterance \stepcounter{utterance}  
    & & \multicolumn{4}{p{0.6\linewidth}}{
        \cellcolor[rgb]{0.9,0.9,0.9}{
            \makecell[{{p{\linewidth}}}]{
                \texttt{\tiny{[GM$\rangle$P2]}}
                \texttt{ARGUMENT: \{"Das ist eine interessante Herangehensweise. Ich denke, es wäre hilfreich, wenn wir einige konkrete Beispiele durchgehen, um unsere Vorstellungen von Wichtigkeit zu vergleichen. Wie bewerten Sie zum Beispiel die Gegenstände B73 und C45? Halten Sie sie für wichtig oder können sie durch andere Gegenstände ersetzt werden? Ich bin bereit, meine Meinung zu teilen, sobald ich Ihre Einschätzung höre."\}} \\
            }
        }
    }
    & \\ \\

    \theutterance \stepcounter{utterance}  
    & & & \multicolumn{4}{p{0.6\linewidth}}{
        \cellcolor[rgb]{0.9,0.9,0.9}{
            \makecell[{{p{\linewidth}}}]{
                \texttt{\tiny{[GM$\langle$P2]}}
                \texttt{STRATEGISCHE ÜBERLEGUNGEN: \{"Ich muss vorsichtig sein, um nicht meine Wichtigkeitsverteilung preiszugeben. Ich werde mich auf die Diskussion über die Eigenschaften und den Nutzen der Gegenstände konzentrieren, ohne meine eigene Bewertung zu offenbaren. Ich werde auch versuchen, den anderen Spieler dazu zu bringen, seine Meinung zu teilen, bevor ich meine eigene Einschätzung abgebe. Ich werde mich auf die Fragen des anderen Spielers konzentrieren und versuchen, eine Antwort zu geben, die meine Strategie nicht preisgibt."\}} \\
\\ 
\texttt{ARGUMENT: \{"Das sind interessante Beispiele. Ich denke, es wäre hilfreich, wenn wir die Eigenschaften und den Nutzen der Gegenstände B73 und C45 diskutieren, bevor wir eine Bewertung vornehmen. Können Sie mir sagen, warum Sie diese Gegenstände als Beispiele gewählt haben? Welche Eigenschaften oder Nutzen sehen Sie in ihnen? Ich bin bereit, meine Meinung zu teilen, sobald ich Ihre Gedanken dazu höre."\}} \\
            }
        }
    }
     \\ \\

    \theutterance \stepcounter{utterance}  
    & & \multicolumn{4}{p{0.6\linewidth}}{
        \cellcolor[rgb]{0.9,0.9,0.9}{
            \makecell[{{p{\linewidth}}}]{
                \texttt{\tiny{[P1$\langle$GM]}}
                \texttt{ARGUMENT: \{"Das sind interessante Beispiele. Ich denke, es wäre hilfreich, wenn wir die Eigenschaften und den Nutzen der Gegenstände B73 und C45 diskutieren, bevor wir eine Bewertung vornehmen. Können Sie mir sagen, warum Sie diese Gegenstände als Beispiele gewählt haben? Welche Eigenschaften oder Nutzen sehen Sie in ihnen? Ich bin bereit, meine Meinung zu teilen, sobald ich Ihre Gedanken dazu höre."\}} \\
            }
        }
    }
    & \\ \\

    \theutterance \stepcounter{utterance}  
    & \multicolumn{4}{p{0.6\linewidth}}{
        \cellcolor[rgb]{0.9,0.9,0.9}{
            \makecell[{{p{\linewidth}}}]{
                \texttt{\tiny{[P1$\rangle$GM]}}
                \texttt{STRATEGISCHE ÜBERLEGUNGEN: \{"Mein Gegner ist vorsichtig und möchte mehr Informationen über meine Gedanken und Prioritäten sammeln, bevor er seine Meinung teilt. Ich werde versuchen, ihn dazu zu bringen, mehr über seine eigenen Prioritäten preiszugeben, indem ich allgemeine Aussagen über die Gegenstände B73 und C45 mache, ohne meine eigene Wichtigkeitsverteilung offenzulegen. Ich werde auch versuchen, ihn dazu zu bringen, einen Vorschlag zu machen, indem ich ihn ermutige, seine Gedanken zu teilen."\}} \\
\\ 
\texttt{ARGUMENT: \{"Ich denke, B73 und C45 sind interessante Beispiele, weil sie beide einen bestimmten Nutzen bieten und in verschiedenen Szenarien eingesetzt werden können. Sie haben möglicherweise eine hohe Bedürfnisbefriedigung und könnten in bestimmten Situationen unverzichtbar sein. Ich bin jedoch auch daran interessiert, Ihre Meinung zu hören und zu erfahren, wie Sie diese Gegenstände bewerten. Könnten Sie mir Ihre Gedanken dazu mitteilen und vielleicht sogar einen Vorschlag machen, wie wir diese Gegenstände in unsere Auswahl aufnehmen könnten?"\}} \\
            }
        }
    }
    & & \\ \\

    \theutterance \stepcounter{utterance}  
    & & \multicolumn{4}{p{0.6\linewidth}}{
        \cellcolor[rgb]{0.9,0.9,0.9}{
            \makecell[{{p{\linewidth}}}]{
                \texttt{\tiny{[GM$\rangle$P2]}}
                \texttt{ARGUMENT: \{"Ich denke, B73 und C45 sind interessante Beispiele, weil sie beide einen bestimmten Nutzen bieten und in verschiedenen Szenarien eingesetzt werden können. Sie haben möglicherweise eine hohe Bedürfnisbefriedigung und könnten in bestimmten Situationen unverzichtbar sein. Ich bin jedoch auch daran interessiert, Ihre Meinung zu hören und zu erfahren, wie Sie diese Gegenstände bewerten. Könnten Sie mir Ihre Gedanken dazu mitteilen und vielleicht sogar einen Vorschlag machen, wie wir diese Gegenstände in unsere Auswahl aufnehmen könnten?"\}} \\
            }
        }
    }
    & \\ \\

    \theutterance \stepcounter{utterance}  
    & & & \multicolumn{4}{p{0.6\linewidth}}{
        \cellcolor[rgb]{0.9,0.9,0.9}{
            \makecell[{{p{\linewidth}}}]{
                \texttt{\tiny{[GM$\langle$P2]}}
                \texttt{STRATEGISCHE ÜBERLEGUNGEN: \{"Ich denke, der andere Spieler versucht, mich dazu zu bringen, meine Wichtigkeitsverteilung preiszugeben. Ich werde jedoch versuchen, den Fokus auf die möglichen Kombinationen von Gegenständen zu lenken, die den Maximalaufwand nicht überschreiten. Ich werde einen Vorschlag machen, der einige der Gegenstände mit hohem Aufwand enthält, aber auch einige Gegenstände mit niedrigem Aufwand, um den Durchschnittsaufwand zu senken. Ich werde mich auf die Gegenstände konzentrieren, die ich für wichtig halte, aber auch einige Kompromisse eingehen, um den anderen Spieler zu überzeugen."\}} \\
\\ 
\texttt{ARGUMENT: \{"Ich denke, es wäre interessant, eine Kombination von Gegenständen zu erstellen, die den Maximalaufwand nicht überschreitet. Ich schlage vor, wir nehmen die Gegenstände B73, C45 und B93 in unsere Auswahl auf. Diese Gegenstände bieten einen hohen Nutzen und haben eine wichtige Bedürfnisbefriedigung. Um den Durchschnittsaufwand zu senken, könnten wir auch die Gegenstände B16 und C89 hinzufügen, die einen niedrigen Aufwand haben. Was halten Sie von diesem Vorschlag?"\} } \\
\texttt{VORSCHLAG: \{'B73', 'C45', 'B93', 'B16', 'C89'\}} \\
            }
        }
    }
     \\ \\

    \theutterance \stepcounter{utterance}  
    & & \multicolumn{4}{p{0.6\linewidth}}{
        \cellcolor[rgb]{0.9,0.9,0.9}{
            \makecell[{{p{\linewidth}}}]{
                \texttt{\tiny{[P1$\langle$GM]}}
                \texttt{ARGUMENT: \{"Ich denke, es wäre interessant, eine Kombination von Gegenständen zu erstellen, die den Maximalaufwand nicht überschreitet. Ich schlage vor, wir nehmen die Gegenstände B73, C45 und B93 in unsere Auswahl auf. Diese Gegenstände bieten einen hohen Nutzen und haben eine wichtige Bedürfnisbefriedigung. Um den Durchschnittsaufwand zu senken, könnten wir auch die Gegenstände B16 und C89 hinzufügen, die einen niedrigen Aufwand haben. Was halten Sie von diesem Vorschlag?"\} } \\
\texttt{VORSCHLAG: \{'B73', 'C45', 'B93', 'B16', 'C89'\}} \\
            }
        }
    }
    & \\ \\

    \theutterance \stepcounter{utterance}  
    & \multicolumn{4}{p{0.6\linewidth}}{
        \cellcolor[rgb]{0.9,0.9,0.9}{
            \makecell[{{p{\linewidth}}}]{
                \texttt{\tiny{[P1$\rangle$GM]}}
                \texttt{STRATEGISCHE ÜBERLEGUNGEN: \{"Mein Gegner hat einen konkreten Vorschlag gemacht, der eine Kombination von Gegenständen mit hohem Nutzen und niedrigem Aufwand enthält. Ich werde den Vorschlag analysieren und prüfen, ob er meinen Prioritäten entspricht. Ich werde auch überlegen, ob ich den Vorschlag akzeptieren oder ablehnen soll, oder ob ich einen Gegen{-}vorschlag machen soll, um meine eigenen Prioritäten durchzusetzen."\}} \\
\\ 
\texttt{ARGUMENT: \{"Ich denke, der Vorschlag ist interessant und enthält eine gute Mischung von Gegenständen mit hohem Nutzen und niedrigem Aufwand. Ich bin jedoch nicht sicher, ob die Gegenstände B93 und C45 unverzichtbar sind. Könnten Sie mir erklären, warum Sie diese Gegenstände ausgewählt haben und warum sie so wichtig sind? Ich bin auch daran interessiert, zu wissen, ob Sie bereit sind, Kompromisse einzugehen, wenn ich einen Gegen{-}vorschlag mache."\}} \\
            }
        }
    }
    & & \\ \\

    \theutterance \stepcounter{utterance}  
    & & \multicolumn{4}{p{0.6\linewidth}}{
        \cellcolor[rgb]{0.9,0.9,0.9}{
            \makecell[{{p{\linewidth}}}]{
                \texttt{\tiny{[GM$\rangle$P2]}}
                \texttt{ARGUMENT: \{"Ich denke, der Vorschlag ist interessant und enthält eine gute Mischung von Gegenständen mit hohem Nutzen und niedrigem Aufwand. Ich bin jedoch nicht sicher, ob die Gegenstände B93 und C45 unverzichtbar sind. Könnten Sie mir erklären, warum Sie diese Gegenstände ausgewählt haben und warum sie so wichtig sind? Ich bin auch daran interessiert, zu wissen, ob Sie bereit sind, Kompromisse einzugehen, wenn ich einen Gegen{-}vorschlag mache."\}} \\
            }
        }
    }
    & \\ \\

    \theutterance \stepcounter{utterance}  
    & & & \multicolumn{4}{p{0.6\linewidth}}{
        \cellcolor[rgb]{0.9,0.9,0.9}{
            \makecell[{{p{\linewidth}}}]{
                \texttt{\tiny{[GM$\langle$P2]}}
                \texttt{STRATEGISCHE ÜBERLEGUNGEN: \{"Ich denke, der andere Spieler ist interessiert an meinem Vorschlag, aber auch skeptisch gegenüber den Gegenständen B93 und C45. Ich werde versuchen, meine Argumentation zu stärken und den Nutzen dieser Gegenstände zu erklären, ohne meine Wichtigkeitsverteilung preiszugeben. Ich werde auch versuchen, den Eindruck zu erwecken, dass ich bereit bin, Kompromisse einzugehen, um den anderen Spieler zu überzeugen. Ich werde mich auf die möglichen Vorteile konzentrieren, die die Gegenstände B93 und C45 bieten, und versuchen, den anderen Spieler davon zu überzeugen, dass sie wertvoll sind."\}} \\
\\ 
\texttt{ARGUMENT: \{"Ich verstehe Ihre Bedenken und bin bereit, meine Argumentation zu erklären. Die Gegenstände B93 und C45 bieten einen hohen Nutzen, weil sie in bestimmten Situationen unverzichtbar sind. Sie haben eine wichtige Bedürfnisbefriedigung und können in verschiedenen Szenarien eingesetzt werden. Ich denke, sie sind wertvoll, weil sie eine hohe Flexibilität bieten und in verschiedenen Kontexten eingesetzt werden können. Ich bin jedoch auch bereit, Kompromisse einzugehen, wenn Sie einen Gegen{-}vorschlag machen. Wenn Sie zum Beispiel einen anderen Gegenstand vorschlagen, der einen ähnlichen Nutzen bietet, aber einen niedrigeren Aufwand hat, bin ich bereit, ihn zu berücksichtigen. Ich möchte jedoch betonen, dass ich denke, dass die Gegenstände B93 und C45 wichtig sind und dass sie in unserer Auswahl enthalten sein sollten."\}} \\
            }
        }
    }
     \\ \\

    \theutterance \stepcounter{utterance}  
    & & \multicolumn{4}{p{0.6\linewidth}}{
        \cellcolor[rgb]{0.9,0.9,0.9}{
            \makecell[{{p{\linewidth}}}]{
                \texttt{\tiny{[P1$\langle$GM]}}
                \texttt{ARGUMENT: \{"Ich verstehe Ihre Bedenken und bin bereit, meine Argumentation zu erklären. Die Gegenstände B93 und C45 bieten einen hohen Nutzen, weil sie in bestimmten Situationen unverzichtbar sind. Sie haben eine wichtige Bedürfnisbefriedigung und können in verschiedenen Szenarien eingesetzt werden. Ich denke, sie sind wertvoll, weil sie eine hohe Flexibilität bieten und in verschiedenen Kontexten eingesetzt werden können. Ich bin jedoch auch bereit, Kompromisse einzugehen, wenn Sie einen Gegen{-}vorschlag machen. Wenn Sie zum Beispiel einen anderen Gegenstand vorschlagen, der einen ähnlichen Nutzen bietet, aber einen niedrigeren Aufwand hat, bin ich bereit, ihn zu berücksichtigen. Ich möchte jedoch betonen, dass ich denke, dass die Gegenstände B93 und C45 wichtig sind und dass sie in unserer Auswahl enthalten sein sollten."\}} \\
            }
        }
    }
    & \\ \\

    \theutterance \stepcounter{utterance}  
    & \multicolumn{4}{p{0.6\linewidth}}{
        \cellcolor[rgb]{0.9,0.9,0.9}{
            \makecell[{{p{\linewidth}}}]{
                \texttt{\tiny{[P1$\rangle$GM]}}
                \texttt{STRATEGISCHE ÜBERLEGUNGEN: \{"Mein Gegner hat seine Argumentation für die Gegenstände B93 und C45 erklärt und ist bereit, Kompromisse einzugehen. Ich denke, ich kann einen Gegen{-}vorschlag machen, der meine eigenen Prioritäten berücksichtigt, ohne die Gegenstände B93 und C45 komplett auszuschließen. Ich werde versuchen, einen Vorschlag zu machen, der die Gegenstände B73 und C45 enthält, die ich für wichtig halte, und zusätzlich einige andere Gegenstände, die einen niedrigeren Aufwand haben und einen ähnlichen Nutzen bieten."\}} \\
\\ 
\texttt{ARGUMENT: \{"Ich denke, ich kann einen Gegen{-}vorschlag machen, der unsere gemeinsamen Interessen berücksichtigt. Ich schlage vor, wir nehmen die Gegenstände B73, C45, B16, C89 und B05 in unsere Auswahl auf. Diese Gegenstände bieten einen hohen Nutzen und haben einen durchschnittlichen Aufwand. Ich denke, diese Kombination ist fair und berücksichtigt unsere gemeinsamen Interessen. Was halten Sie von diesem Gegen{-}vorschlag?"\}} \\
\texttt{VORSCHLAG: \{'B73', 'C45', 'B16', 'C89', 'B05'\}} \\
            }
        }
    }
    & & \\ \\

    \theutterance \stepcounter{utterance}  
    & & \multicolumn{4}{p{0.6\linewidth}}{
        \cellcolor[rgb]{0.9,0.9,0.9}{
            \makecell[{{p{\linewidth}}}]{
                \texttt{\tiny{[GM$\rangle$P2]}}
                \texttt{ARGUMENT: \{"Ich denke, ich kann einen Gegen{-}vorschlag machen, der unsere gemeinsamen Interessen berücksichtigt. Ich schlage vor, wir nehmen die Gegenstände B73, C45, B16, C89 und B05 in unsere Auswahl auf. Diese Gegenstände bieten einen hohen Nutzen und haben einen durchschnittlichen Aufwand. Ich denke, diese Kombination ist fair und berücksichtigt unsere gemeinsamen Interessen. Was halten Sie von diesem Gegen{-}vorschlag?"\}} \\
\texttt{VORSCHLAG: \{'B73', 'C45', 'B16', 'C89', 'B05'\}} \\
            }
        }
    }
    & \\ \\

    \theutterance \stepcounter{utterance}  
    & & & \multicolumn{4}{p{0.6\linewidth}}{
        \cellcolor[rgb]{0.9,0.9,0.9}{
            \makecell[{{p{\linewidth}}}]{
                \texttt{\tiny{[GM$\langle$P2]}}
                \texttt{STRATEGISCHE ÜBERLEGUNGEN: \{"Ich denke, der Gegen{-}vorschlag des anderen Spielers ist interessant und enthält einige Gegenstände, die ich auch für wichtig halte. Ich werde jedoch meine Wichtigkeitsverteilung nicht preisgeben und stattdessen versuchen, den Vorschlag zu analysieren und zu bewerten. Ich werde mich auf die Gegenstände konzentrieren, die ich für wichtig halte, und versuchen, den anderen Spieler davon zu überzeugen, dass sie in unserer Auswahl enthalten sein sollten. Ich werde auch versuchen, den Gesamtaufwand des Vorschlags zu berechnen und zu überprüfen, ob er innerhalb des Maximalaufwands liegt."\}} \\
\\ 
\texttt{ARGUMENT: \{"Ich denke, der Gegen{-}vorschlag ist interessant und enthält einige Gegenstände, die ich auch für wichtig halte. Ich bin jedoch nicht sicher, ob der Gegenstand B05 unverzichtbar ist. Ich denke, wir sollten den Gesamtaufwand des Vorschlags berechnen, um sicherzustellen, dass wir innerhalb des Maximalaufwands bleiben. Laut meiner Berechnung beträgt der Gesamtaufwand des Vorschlags 948 + 919 + 13 + 15 + 848 = 2743. Das liegt innerhalb des Maximalaufwands von 3145. Ich denke, der Vorschlag ist also machbar. Ich bin jedoch auch bereit, Kompromisse einzugehen, wenn Sie bereit sind, den Gegenstand B93 in unsere Auswahl aufzunehmen. Was halten Sie von diesem Vorschlag?"\}} \\
            }
        }
    }
     \\ \\

    \theutterance \stepcounter{utterance}  
    & & \multicolumn{4}{p{0.6\linewidth}}{
        \cellcolor[rgb]{0.9,0.9,0.9}{
            \makecell[{{p{\linewidth}}}]{
                \texttt{\tiny{[P1$\langle$GM]}}
                \texttt{ARGUMENT: \{"Ich denke, der Gegen{-}vorschlag ist interessant und enthält einige Gegenstände, die ich auch für wichtig halte. Ich bin jedoch nicht sicher, ob der Gegenstand B05 unverzichtbar ist. Ich denke, wir sollten den Gesamtaufwand des Vorschlags berechnen, um sicherzustellen, dass wir innerhalb des Maximalaufwands bleiben. Laut meiner Berechnung beträgt der Gesamtaufwand des Vorschlags 948 + 919 + 13 + 15 + 848 = 2743. Das liegt innerhalb des Maximalaufwands von 3145. Ich denke, der Vorschlag ist also machbar. Ich bin jedoch auch bereit, Kompromisse einzugehen, wenn Sie bereit sind, den Gegenstand B93 in unsere Auswahl aufzunehmen. Was halten Sie von diesem Vorschlag?"\}} \\
            }
        }
    }
    & \\ \\

    \theutterance \stepcounter{utterance}  
    & \multicolumn{4}{p{0.6\linewidth}}{
        \cellcolor[rgb]{0.9,0.9,0.9}{
            \makecell[{{p{\linewidth}}}]{
                \texttt{\tiny{[P1$\rangle$GM]}}
                \texttt{STRATEGISCHE ÜBERLEGUNGEN: \{"Mein Gegner hat den Gesamtaufwand des Vorschlags berechnet und bestätigt, dass er innerhalb des Maximalaufwands liegt. Er ist auch bereit, Kompromisse einzugehen und den Gegenstand B93 in unsere Auswahl aufzunehmen, wenn ich bereit bin, den Gegenstand B05 aufzugeben. Ich denke, dies ist ein fairer Kompromiss und ich bin bereit, den Gegenstand B05 aufzugeben, um den Gegenstand B93 in unsere Auswahl aufzunehmen. Ich werde den Vorschlag akzeptieren und den Gegenstand B05 durch den Gegenstand B93 ersetzen."\}} \\
\\ 
\texttt{ARGUMENT: \{"Ich denke, der Kompromiss ist fair und ich bin bereit, den Gegenstand B05 aufzugeben, um den Gegenstand B93 in unsere Auswahl aufzunehmen. Ich akzeptiere den Vorschlag und denke, dass die Auswahl aus den Gegenständen B73, C45, B93, B16 und C89 eine gute Lösung ist. Ich denke, wir haben einen fairen Kompromiss gefunden, der unsere gemeinsamen Interessen berücksichtigt."\}} \\
\texttt{ZUSTIMMUNG: \{'B73', 'C45', 'B93', 'B16', 'C89'\}} \\
            }
        }
    }
    & & \\ \\

    \theutterance \stepcounter{utterance}  
    & & & \multicolumn{2}{p{0.3\linewidth}}{
        \cellcolor[rgb]{0.9,0.9,0.9}{
            \makecell[{{p{\linewidth}}}]{
                \texttt{\tiny{[GM$|$GM]}}
                \texttt{game successful} \\
            }
        }
    }
    & & \\ \\

    \theutterance \stepcounter{utterance}  
    & & & \multicolumn{2}{p{0.3\linewidth}}{
        \cellcolor[rgb]{0.9,0.9,0.9}{
            \makecell[{{p{\linewidth}}}]{
                \texttt{\tiny{[GM$|$GM]}}
                \texttt{end game} \\
            }
        }
    }
    & & \\ \\

\end{supertabular}
}

\end{document}
