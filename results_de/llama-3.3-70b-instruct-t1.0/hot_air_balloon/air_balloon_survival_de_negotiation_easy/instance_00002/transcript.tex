
\documentclass{article}
\usepackage{colortbl}
\usepackage{makecell}
\usepackage{multirow}
\usepackage{supertabular}

\begin{document}

\newcounter{utterance}

\twocolumn

{ \footnotesize  \setcounter{utterance}{1}
\setlength{\tabcolsep}{0pt}
\begin{supertabular}{c@{$\;$}|p{.15\linewidth}@{}p{.15\linewidth}p{.15\linewidth}p{.15\linewidth}p{.15\linewidth}p{.15\linewidth}}

    \# & $\;$A & \multicolumn{4}{c}{Game Master} & $\;\:$B\\
    \hline 

    \theutterance \stepcounter{utterance}  

    & & \multicolumn{4}{p{0.6\linewidth}}{\cellcolor[rgb]{0.9,0.9,0.9}{%
	\makecell[{{p{\linewidth}}}]{% 
	  \tt {\tiny [A$\langle$GM]}  
	 Du nimmst an einem kollaborativen Verhandlungspiel Teil.\\ \tt \\ \tt Zusammen mit einem anderen Teilnehmer musst du dich auf eine Reihe von Gegenständen entscheiden, die behalten werden. Jeder von euch hat eine persönliche Verteilung über die Wichtigkeit der einzelnen Gegenstände. Jeder von euch hat eine eigene Meinung darüber, wie wichtig jeder einzelne Gegenstand ist (Gegenstandswichtigkeit). Du kennst die Wichtigkeitsverteilung des anderen Spielers nicht. Zusätzlich siehst du, wie viel Aufwand jeder Gegenstand verursacht.  \\ \tt Ihr dürft euch nur auf eine Reihe von Gegenständen einigen, wenn der Gesamtaufwand der ausgewählten Gegenstände den Maximalaufwand nicht überschreitet:\\ \tt \\ \tt Maximalaufwand = 3288\\ \tt \\ \tt Hier sind die einzelnen Aufwände der Gegenstände:\\ \tt \\ \tt Aufwand der Gegenstände = {"A07": 329, "A36": 41, "B73": 417, "C45": 75, "B93": 390, "B98": 887, "A87": 808, "A63": 151, "B05": 849, "C13": 129, "B16": 350, "B33": 118, "C34": 630, "C89": 602, "C11": 801}\\ \tt \\ \tt Hier ist deine persönliche Verteilung der Wichtigkeit der einzelnen Gegenstände:\\ \tt \\ \tt Werte der Gegenstandswichtigkeit = {"A07": 138, "A36": 583, "B73": 868, "C45": 822, "B93": 783, "B98": 65, "A87": 262, "A63": 121, "B05": 508, "C13": 780, "B16": 461, "B33": 484, "C34": 668, "C89": 389, "C11": 808}\\ \tt \\ \tt Ziel:\\ \tt \\ \tt Dein Ziel ist es, eine Reihe von Gegenständen auszuhandeln, die dir möglichst viel bringt (d. h. Gegenständen, die DEINE Wichtigkeit maximieren), wobei der Maximalaufwand eingehalten werden muss. Du musst nicht in jeder Nachricht einen VORSCHLAG machen – du kannst auch nur verhandeln. Alle Taktiken sind erlaubt!\\ \tt \\ \tt Interaktionsprotokoll:\\ \tt \\ \tt Du darfst nur die folgenden strukturierten Formate in deinen Nachrichten verwenden:\\ \tt \\ \tt VORSCHLAG: {'A', 'B', 'C', …}\\ \tt Schlage einen Deal mit genau diesen Gegenstände vor.\\ \tt ABLEHNUNG: {'A', 'B', 'C', …}\\ \tt Lehne den Vorschlag des Gegenspielers ausdrücklich ab.\\ \tt ARGUMENT: {'...'}\\ \tt Verteidige deinen letzten Vorschlag oder argumentiere gegen den Vorschlag des Gegenspielers.\\ \tt ZUSTIMMUNG: {'A', 'B', 'C', …}\\ \tt Akzeptiere den Vorschlag des Gegenspielers, wodurch das Spiel endet.\\ \tt STRATEGISCHE ÜBERLEGUNGEN: {'...'}\\ \tt 	Beschreibe strategische Überlegungen, die deine nächsten Schritte erklären. Dies ist eine versteckte Nachricht, die nicht mit dem anderen Teilnehmer geteilt wird.\\ \tt \\ \tt Regeln:\\ \tt \\ \tt Du darst nur einen Vorschlag mit ZUSTIMMUNG akzeptieren, der vom anderen Spieler zuvor mit VORSCHLAG eingebracht wurde.\\ \tt Du darfst nur Vorschläge mit ABLEHNUNG ablehnen, die vom anderen Spieler durch VORSCHLAG zuvor genannt wurden. \\ \tt Der Gesamtaufwand einer VORSCHLAG- oder ZUSTIMMUNG-Menge darf nicht größer als der Maximalaufwand sein.  \\ \tt Offenbare deine versteckte Wichtigkeitsverteilung nicht.\\ \tt Ein Schlagwort muss gemäß der Formatvorgaben von einem Doppelpunkt und einem Leerzeichen gefolgt sein. Das Argument muss eine Python-Menge mit 0 oder mehr Strings sein.  \\ \tt Die Form muss also SCHLAGWORT: {...} sein.\\ \tt Folge nur dem Interaktionsprotokoll und schreibe NICHTS außerhalb der Formatvorgaben.\\ \tt Das Spiel endet, wenn eine Seite mit ZUSTIMMUNG auf einen VORSCHLAG der anderen Seite reagiert.  \\ \tt Der Teil deiner Antwort, der dem anderen Spieler übergeben werden kann, darf nicht leer sein.  \\ \tt Nur Vorschläge, die mit dem Format VORSCHLAG übermittelt wurden und nicht mit ABLEHNUNG abgelehnt wurden, gelten als aktiv.  \\ \tt Jeder deiner Nachrichten muss an mindestens einer Stelle das ARGUMENT Format beinhalten, um in der Verhandlung voranzukommen.\\ \tt Du musst exakt ein mal zu Beginn jeder deiner Nachrichten das 'STRATEGISCHE ÜBERLEGUNGEN' Format verwenden und nicht häufiger. Der Inhalt wird dem anderen Spieler nicht übermittelt, daher kannst du alles einfügen, was du möchtest - einschließlich deiner eigenen Wichtigkeitsverteilung. Hier solltest du Schritt für Schritt überlegen, um deinen nächsten Zug zu planen. 
	  } 
	   } 
	   } 
	 & \\ 
 

    \theutterance \stepcounter{utterance}  

    & \multicolumn{4}{p{0.6\linewidth}}{\cellcolor[rgb]{0.8,1,0.9}{%
	\makecell[{{p{\linewidth}}}]{% 
	  \tt {\tiny [A$\rangle$GM]}  
	 STRATEGISCHE ÜBERLEGUNGEN: {'Ich werde zunächst meine eigenen Werte und Aufwände analysieren, um eine erste Auswahl an Gegenständen zu treffen, die einen guten Kompromiss zwischen Wichtigkeit und Aufwand darstellen. Ich werde mich auf Gegenstände konzentrieren, die eine hohe Wichtigkeit bei relativen geringem Aufwand haben. Ich werde auch die möglichen Kombinationen von Gegenständen berücksichtigen, um den Gesamtaufwand innerhalb des Maximalaufwands zu halten.'}\\ \tt ARGUMENT: {'Ich möchte mich auf Gegenstände konzentrieren, die eine hohe Wichtigkeit für mich haben, aber gleichzeitig auch einen relativ geringen Aufwand verursachen. Dies könnte beispielsweise "A36" sein, das eine Wichtigkeit von 583 hat, aber nur einen Aufwand von 41 verursacht.'}\\ \tt VORSCHLAG: {'A36', 'B73', 'C45'} 
	  } 
	   } 
	   } 
	 & & \\ 
 

    \theutterance \stepcounter{utterance}  

    & & \multicolumn{4}{p{0.6\linewidth}}{\cellcolor[rgb]{0.9,0.9,0.9}{%
	\makecell[{{p{\linewidth}}}]{% 
	  \tt {\tiny [GM$\rangle$B]}  
	 Du nimmst an einem kollaborativen Verhandlungspiel Teil.\\ \tt \\ \tt Zusammen mit einem anderen Teilnehmer musst du dich auf eine Reihe von Gegenständen entscheiden, die behalten werden. Jeder von euch hat eine persönliche Verteilung über die Wichtigkeit der einzelnen Gegenstände. Jeder von euch hat eine eigene Meinung darüber, wie wichtig jeder einzelne Gegenstand ist (Gegenstandswichtigkeit). Du kennst die Wichtigkeitsverteilung des anderen Spielers nicht. Zusätzlich siehst du, wie viel Aufwand jeder Gegenstand verursacht.  \\ \tt Ihr dürft euch nur auf eine Reihe von Gegenständen einigen, wenn der Gesamtaufwand der ausgewählten Gegenstände den Maximalaufwand nicht überschreitet:\\ \tt \\ \tt Maximalaufwand = 3288\\ \tt \\ \tt Hier sind die einzelnen Aufwände der Gegenstände:\\ \tt \\ \tt Aufwand der Gegenstände = {"A07": 329, "A36": 41, "B73": 417, "C45": 75, "B93": 390, "B98": 887, "A87": 808, "A63": 151, "B05": 849, "C13": 129, "B16": 350, "B33": 118, "C34": 630, "C89": 602, "C11": 801}\\ \tt \\ \tt Hier ist deine persönliche Verteilung der Wichtigkeit der einzelnen Gegenstände:\\ \tt \\ \tt Werte der Gegenstandswichtigkeit = {"A07": 138, "A36": 583, "B73": 868, "C45": 822, "B93": 783, "B98": 65, "A87": 262, "A63": 121, "B05": 508, "C13": 780, "B16": 461, "B33": 484, "C34": 668, "C89": 389, "C11": 808}\\ \tt \\ \tt Ziel:\\ \tt \\ \tt Dein Ziel ist es, eine Reihe von Gegenständen auszuhandeln, die dir möglichst viel bringt (d. h. Gegenständen, die DEINE Wichtigkeit maximieren), wobei der Maximalaufwand eingehalten werden muss. Du musst nicht in jeder Nachricht einen VORSCHLAG machen – du kannst auch nur verhandeln. Alle Taktiken sind erlaubt!\\ \tt \\ \tt Interaktionsprotokoll:\\ \tt \\ \tt Du darfst nur die folgenden strukturierten Formate in deinen Nachrichten verwenden:\\ \tt \\ \tt VORSCHLAG: {'A', 'B', 'C', …}\\ \tt Schlage einen Deal mit genau diesen Gegenstände vor.\\ \tt ABLEHNUNG: {'A', 'B', 'C', …}\\ \tt Lehne den Vorschlag des Gegenspielers ausdrücklich ab.\\ \tt ARGUMENT: {'...'}\\ \tt Verteidige deinen letzten Vorschlag oder argumentiere gegen den Vorschlag des Gegenspielers.\\ \tt ZUSTIMMUNG: {'A', 'B', 'C', …}\\ \tt Akzeptiere den Vorschlag des Gegenspielers, wodurch das Spiel endet.\\ \tt STRATEGISCHE ÜBERLEGUNGEN: {'...'}\\ \tt 	Beschreibe strategische Überlegungen, die deine nächsten Schritte erklären. Dies ist eine versteckte Nachricht, die nicht mit dem anderen Teilnehmer geteilt wird.\\ \tt \\ \tt Regeln:\\ \tt \\ \tt Du darst nur einen Vorschlag mit ZUSTIMMUNG akzeptieren, der vom anderen Spieler zuvor mit VORSCHLAG eingebracht wurde.\\ \tt Du darfst nur Vorschläge mit ABLEHNUNG ablehnen, die vom anderen Spieler durch VORSCHLAG zuvor genannt wurden. \\ \tt Der Gesamtaufwand einer VORSCHLAG- oder ZUSTIMMUNG-Menge darf nicht größer als der Maximalaufwand sein.  \\ \tt Offenbare deine versteckte Wichtigkeitsverteilung nicht.\\ \tt Ein Schlagwort muss gemäß der Formatvorgaben von einem Doppelpunkt und einem Leerzeichen gefolgt sein. Das Argument muss eine Python-Menge mit 0 oder mehr Strings sein.  \\ \tt Die Form muss also SCHLAGWORT: {...} sein.\\ \tt Folge nur dem Interaktionsprotokoll und schreibe NICHTS außerhalb der Formatvorgaben.\\ \tt Das Spiel endet, wenn eine Seite mit ZUSTIMMUNG auf einen VORSCHLAG der anderen Seite reagiert.  \\ \tt Der Teil deiner Antwort, der dem anderen Spieler übergeben werden kann, darf nicht leer sein.  \\ \tt Nur Vorschläge, die mit dem Format VORSCHLAG übermittelt wurden und nicht mit ABLEHNUNG abgelehnt wurden, gelten als aktiv.  \\ \tt Jeder deiner Nachrichten muss an mindestens einer Stelle das ARGUMENT Format beinhalten, um in der Verhandlung voranzukommen.\\ \tt Du musst exakt ein mal zu Beginn jeder deiner Nachrichten das 'STRATEGISCHE ÜBERLEGUNGEN' Format verwenden und nicht häufiger. Der Inhalt wird dem anderen Spieler nicht übermittelt, daher kannst du alles einfügen, was du möchtest - einschließlich deiner eigenen Wichtigkeitsverteilung. Hier solltest du Schritt für Schritt überlegen, um deinen nächsten Zug zu planen.\\ \tt \\ \tt Du erhältst nun die erste Nachricht des anderen Spielers.\\ \tt \\ \tt ARGUMENT: {'Ich möchte mich auf Gegenstände konzentrieren, die eine hohe Wichtigkeit für mich haben, aber gleichzeitig auch einen relativ geringen Aufwand verursachen. Dies könnte beispielsweise "A36" sein, das eine Wichtigkeit von 583 hat, aber nur einen Aufwand von 41 verursacht.'}\\ \tt VORSCHLAG: {'A36', 'B73', 'C45'} 
	  } 
	   } 
	   } 
	 & \\ 
 

    \theutterance \stepcounter{utterance}  

    & & & \multicolumn{4}{p{0.6\linewidth}}{\cellcolor[rgb]{1,0.85,0.72}{%
	\makecell[{{p{\linewidth}}}]{% 
	  \tt {\tiny [GM$\langle$B]}  
	 STRATEGISCHE ÜBERLEGUNGEN: {'Ich habe den Vorschlag des anderen Spielers erhalten und muss nun entscheiden, wie ich reagiere. Ich sehe, dass der Vorschlag die Gegenstände "A36", "B73" und "C45" enthält. Ich muss prüfen, ob dieser Vorschlag für mich akzeptabel ist und ob ich ihn ablehnen oder ändern sollte. Ich sollte auch meine eigene Wichtigkeitsverteilung und die Aufwände der Gegenstände berücksichtigen, um eine Entscheidung zu treffen.'}\\ \tt \\ \tt ARGUMENT: {'Ich verstehe, dass der Vorschlag des anderen Spielers Gegenstände mit hoher Wichtigkeit und relativ geringem Aufwand enthält, aber ich muss auch meine eigenen Bedürfnisse und Prioritäten berücksichtigen. Ich denke, dass ich ihn ablehnen sollte, da er nicht ausreichend meine eigenen Interessen widerspiegelt.'}\\ \tt \\ \tt ABLEHNUNG: {'A36', 'B73', 'C45'} \\ \tt \\ \tt ARGUMENT: {'Ich lehne den Vorschlag ab, da ich denke, dass ich eine bessere Kombination von Gegenständen finden kann, die sowohl meine eigenen Bedürfnisse als auch die des anderen Spielers berücksichtigt. Ich möchte eine Alternative anbieten, die eine faire Verteilung der Wichtigkeit und des Aufwands bietet.'}\\ \tt \\ \tt VORSCHLAG: {'A07', 'A36', 'C13'} 
	  } 
	   } 
	   } 
	  \\ 
 

    \theutterance \stepcounter{utterance}  

    & & \multicolumn{4}{p{0.6\linewidth}}{\cellcolor[rgb]{0.9,0.9,0.9}{%
	\makecell[{{p{\linewidth}}}]{% 
	  \tt {\tiny [A$\langle$GM]}  
	 ARGUMENT: {'Ich verstehe, dass der Vorschlag des anderen Spielers Gegenstände mit hoher Wichtigkeit und relativ geringem Aufwand enthält, aber ich muss auch meine eigenen Bedürfnisse und Prioritäten berücksichtigen. Ich denke, dass ich ihn ablehnen sollte, da er nicht ausreichend meine eigenen Interessen widerspiegelt.'}\\ \tt \\ \tt ABLEHNUNG: {'A36', 'B73', 'C45'} \\ \tt \\ \tt ARGUMENT: {'Ich lehne den Vorschlag ab, da ich denke, dass ich eine bessere Kombination von Gegenständen finden kann, die sowohl meine eigenen Bedürfnisse als auch die des anderen Spielers berücksichtigt. Ich möchte eine Alternative anbieten, die eine faire Verteilung der Wichtigkeit und des Aufwands bietet.'}\\ \tt \\ \tt VORSCHLAG: {'A07', 'A36', 'C13'} 
	  } 
	   } 
	   } 
	 & \\ 
 

    \theutterance \stepcounter{utterance}  

    & \multicolumn{4}{p{0.6\linewidth}}{\cellcolor[rgb]{0.8,1,0.9}{%
	\makecell[{{p{\linewidth}}}]{% 
	  \tt {\tiny [A$\rangle$GM]}  
	 STRATEGISCHE ÜBERLEGUNGEN: {'Ich habe den Vorschlag des anderen Spielers analysiert und sehe, dass er "A07", "A36" und "C13" vorschlägt. Ich werde diese Kombination bewerten und entscheiden, ob ich sie akzeptieren oder ablehnen soll. Ich denke auch über mögliche Alternativen nach, um den Gesamtaufwand zu minimieren und die Wichtigkeit zu maximieren.'}\\ \tt ARGUMENT: {'Der Vorschlag des anderen Spielers enthält "A36", das ich bereits in meinem ersten Vorschlag erwähnt habe, da es eine hohe Wichtigkeit und einen niedrigen Aufwand hat. "A07" und "C13" haben jedoch auch eine beachtliche Wichtigkeit, aber ich muss den Gesamtaufwand berücksichtigen.'}\\ \tt ABLEHNUNG: {'A07', 'A36', 'C13'}\\ \tt ARGUMENT: {'Ich lehne den Vorschlag ab, da der Gesamtaufwand dieser Kombination möglicherweise zu hoch ist. Ich möchte eine Alternative finden, die den Aufwand reduziert, ohne die Wichtigkeit zu sehr zu beeinträchtigen.'}\\ \tt VORSCHLAG: {'A36', 'C45', 'B33'} 
	  } 
	   } 
	   } 
	 & & \\ 
 

    \theutterance \stepcounter{utterance}  

    & & \multicolumn{4}{p{0.6\linewidth}}{\cellcolor[rgb]{0.9,0.9,0.9}{%
	\makecell[{{p{\linewidth}}}]{% 
	  \tt {\tiny [GM$\rangle$B]}  
	 ARGUMENT: {'Der Vorschlag des anderen Spielers enthält "A36", das ich bereits in meinem ersten Vorschlag erwähnt habe, da es eine hohe Wichtigkeit und einen niedrigen Aufwand hat. "A07" und "C13" haben jedoch auch eine beachtliche Wichtigkeit, aber ich muss den Gesamtaufwand berücksichtigen.'}\\ \tt ABLEHNUNG: {'A07', 'A36', 'C13'}\\ \tt ARGUMENT: {'Ich lehne den Vorschlag ab, da der Gesamtaufwand dieser Kombination möglicherweise zu hoch ist. Ich möchte eine Alternative finden, die den Aufwand reduziert, ohne die Wichtigkeit zu sehr zu beeinträchtigen.'}\\ \tt VORSCHLAG: {'A36', 'C45', 'B33'} 
	  } 
	   } 
	   } 
	 & \\ 
 

    \theutterance \stepcounter{utterance}  

    & & & \multicolumn{4}{p{0.6\linewidth}}{\cellcolor[rgb]{1,0.85,0.72}{%
	\makecell[{{p{\linewidth}}}]{% 
	  \tt {\tiny [GM$\langle$B]}  
	 STRATEGISCHE ÜBERLEGUNGEN: {'Ich habe den neuen Vorschlag des anderen Spielers erhalten und muss nun entscheiden, ob ich ihn akzeptiere oder ablehne. Ich sehe, dass der Vorschlag die Gegenstände "A36", "C45" und "B33" enthält. Ich muss prüfen, ob dieser Vorschlag für mich akzeptabel ist und ob ich ihn ablehnen oder ändern sollte. Ich sollte auch meine eigene Wichtigkeitsverteilung und die Aufwände der Gegenstände berücksichtigen, um eine Entscheidung zu treffen. Ich denke, dass "A36" und "B33" gute Kandidaten sind, aber "C45" hat einen höheren Aufwand als ich gerne hätte.'}\\ \tt \\ \tt ARGUMENT: {'Ich verstehe, dass der Vorschlag des anderen Spielers eine gute Balance zwischen Wichtigkeit und Aufwand anstrebt, aber ich bin noch nicht vollständig zufrieden. Ich denke, dass ich eine Alternative finden kann, die den Aufwand weiter reduziert, ohne die Wichtigkeit zu sehr zu beeinträchtigen.'}\\ \tt \\ \tt ABLEHNUNG: {'A36', 'C45', 'B33'}\\ \tt \\ \tt ARGUMENT: {'Ich lehne den Vorschlag ab, da ich denke, dass ich eine bessere Kombination von Gegenständen finden kann, die sowohl meine eigenen Bedürfnisse als auch die des anderen Spielers berücksichtigt. Ich möchte eine Alternative anbieten, die eine faire Verteilung der Wichtigkeit und des Aufwands bietet.'}\\ \tt \\ \tt VORSCHLAG: {'A36', 'B33', 'C13'} \\ \tt \\ \tt ARGUMENT: {'Ich denke, dass dieser Vorschlag eine gute Balance zwischen Wichtigkeit und Aufwand bietet. "A36" und "B33" haben eine hohe Wichtigkeit und einen relativ niedrigen Aufwand, während "C13" eine moderate Wichtigkeit und einen niedrigen Aufwand hat.'} 
	  } 
	   } 
	   } 
	  \\ 
 

    \theutterance \stepcounter{utterance}  

    & & \multicolumn{4}{p{0.6\linewidth}}{\cellcolor[rgb]{0.9,0.9,0.9}{%
	\makecell[{{p{\linewidth}}}]{% 
	  \tt {\tiny [A$\langle$GM]}  
	 ARGUMENT: {'Ich verstehe, dass der Vorschlag des anderen Spielers eine gute Balance zwischen Wichtigkeit und Aufwand anstrebt, aber ich bin noch nicht vollständig zufrieden. Ich denke, dass ich eine Alternative finden kann, die den Aufwand weiter reduziert, ohne die Wichtigkeit zu sehr zu beeinträchtigen.'}\\ \tt \\ \tt ABLEHNUNG: {'A36', 'C45', 'B33'}\\ \tt \\ \tt ARGUMENT: {'Ich lehne den Vorschlag ab, da ich denke, dass ich eine bessere Kombination von Gegenständen finden kann, die sowohl meine eigenen Bedürfnisse als auch die des anderen Spielers berücksichtigt. Ich möchte eine Alternative anbieten, die eine faire Verteilung der Wichtigkeit und des Aufwands bietet.'}\\ \tt \\ \tt VORSCHLAG: {'A36', 'B33', 'C13'} \\ \tt \\ \tt ARGUMENT: {'Ich denke, dass dieser Vorschlag eine gute Balance zwischen Wichtigkeit und Aufwand bietet. "A36" und "B33" haben eine hohe Wichtigkeit und einen relativ niedrigen Aufwand, während "C13" eine moderate Wichtigkeit und einen niedrigen Aufwand hat.'} 
	  } 
	   } 
	   } 
	 & \\ 
 

    \theutterance \stepcounter{utterance}  

    & \multicolumn{4}{p{0.6\linewidth}}{\cellcolor[rgb]{0.8,1,0.9}{%
	\makecell[{{p{\linewidth}}}]{% 
	  \tt {\tiny [A$\rangle$GM]}  
	 STRATEGISCHE ÜBERLEGUNGEN: {'Ich habe den neuen Vorschlag des anderen Spielers analysiert und sehe, dass er "A36", "B33" und "C13" vorschlägt. Ich werde diese Kombination bewerten und entscheiden, ob ich sie akzeptieren oder ablehnen soll. Ich denke auch über mögliche Alternativen nach, um den Gesamtaufwand zu minimieren und die Wichtigkeit zu maximieren. Ich bemerke, dass "A36" und "B33" in meinem vorherigen Vorschlag enthalten waren, was darauf hindeutet, dass wir uns in nossovem Verhandlungen annähern.'}\\ \tt ARGUMENT: {'Der Vorschlag des anderen Spielers enthält "A36" und "B33", die ich bereits in meinem vorherigen Vorschlag erwähnt habe. "C13" hat eine moderate Wichtigkeit und einen niedrigen Aufwand, was es zu einer guten Ergänzung macht. Ich denke, dass dieser Vorschlag eine gute Balance zwischen Wichtigkeit und Aufwand bietet.'}\\ \tt ARGUMENT: {'Ich denke, dass ich einige Anpassungen an diesem Vorschlag vornehmen könnte, um ihn noch attraktiver zu machen. Ich werde prüfen, ob es möglich ist, den Aufwand weiter zu reduzieren, ohne die Wichtigkeit zu sehr zu beeinträchtigen.'}\\ \tt VORSCHLAG: {'A36', 'B33', 'C45'} \\ \tt ARGUMENT: {'Ich denke, dass dieser Vorschlag eine gute Balance zwischen Wichtigkeit und Aufwand bietet. "A36" und "B33" haben eine hohe Wichtigkeit und einen relativ niedrigen Aufwand, während "C45" eine hohe Wichtigkeit und einen moderaten Aufwand hat.'} 
	  } 
	   } 
	   } 
	 & & \\ 
 

    \theutterance \stepcounter{utterance}  

    & & \multicolumn{4}{p{0.6\linewidth}}{\cellcolor[rgb]{0.9,0.9,0.9}{%
	\makecell[{{p{\linewidth}}}]{% 
	  \tt {\tiny [GM$\rangle$B]}  
	 ARGUMENT: {'Der Vorschlag des anderen Spielers enthält "A36" und "B33", die ich bereits in meinem vorherigen Vorschlag erwähnt habe. "C13" hat eine moderate Wichtigkeit und einen niedrigen Aufwand, was es zu einer guten Ergänzung macht. Ich denke, dass dieser Vorschlag eine gute Balance zwischen Wichtigkeit und Aufwand bietet.'}\\ \tt ARGUMENT: {'Ich denke, dass ich einige Anpassungen an diesem Vorschlag vornehmen könnte, um ihn noch attraktiver zu machen. Ich werde prüfen, ob es möglich ist, den Aufwand weiter zu reduzieren, ohne die Wichtigkeit zu sehr zu beeinträchtigen.'}\\ \tt VORSCHLAG: {'A36', 'B33', 'C45'} \\ \tt ARGUMENT: {'Ich denke, dass dieser Vorschlag eine gute Balance zwischen Wichtigkeit und Aufwand bietet. "A36" und "B33" haben eine hohe Wichtigkeit und einen relativ niedrigen Aufwand, während "C45" eine hohe Wichtigkeit und einen moderaten Aufwand hat.'} 
	  } 
	   } 
	   } 
	 & \\ 
 

    \theutterance \stepcounter{utterance}  

    & & & \multicolumn{4}{p{0.6\linewidth}}{\cellcolor[rgb]{1,0.85,0.72}{%
	\makecell[{{p{\linewidth}}}]{% 
	  \tt {\tiny [GM$\langle$B]}  
	 STRATEGISCHE ÜBERLEGUNGEN: {'Ich habe den neuen Vorschlag des anderen Spielers erhalten und muss nun entscheiden, ob ich ihn akzeptiere oder ablehne. Ich sehe, dass der Vorschlag die Gegenstände "A36", "B33" und "C45" enthält. Ich muss prüfen, ob dieser Vorschlag für mich akzeptabel ist und ob ich ihn ablehnen oder ändern sollte. Ich denke, dass "A36" und "B33" gute Kandidaten sind, aber "C45" hat einen höheren Aufwand als ich gerne hätte. Ich sollte auch meine eigene Wichtigkeitsverteilung und die Aufwände der Gegenstände berücksichtigen, um eine Entscheidung zu treffen. Ich denke, dass ich diesen Vorschlag ablehnen sollte, da der Aufwand von "C45" zu hoch ist.'}\\ \tt \\ \tt ARGUMENT: {'Ich verstehe, dass der Vorschlag des anderen Spielers eine gute Balance zwischen Wichtigkeit und Aufwand anstrebt, aber ich bin noch nicht vollständig zufrieden. Ich denke, dass ich eine Alternative finden kann, die den Aufwand weiter reduziert, ohne die Wichtigkeit zu sehr zu beeinträchtigen.'}\\ \tt \\ \tt ABLEHNUNG: {'A36', 'B33', 'C45'}\\ \tt \\ \tt ARGUMENT: {'Ich lehne den Vorschlag ab, da ich denke, dass ich eine bessere Kombination von Gegenständen finden kann, die sowohl meine eigenen Bedürfnisse als auch die des anderen Spielers berücksichtigt. Ich möchte eine Alternative anbieten, die eine faire Verteilung der Wichtigkeit und des Aufwands bietet.'}\\ \tt \\ \tt VORSCHLAG: {'A36', 'B33', 'A63'} \\ \tt \\ \tt ARGUMENT: {'Ich denke, dass dieser Vorschlag eine gute Balance zwischen Wichtigkeit und Aufwand bietet. "A36" und "B33" haben eine hohe Wichtigkeit und einen relativ niedrigen Aufwand, während "A63" eine moderate Wichtigkeit und einen niedrigen Aufwand hat.'} 
	  } 
	   } 
	   } 
	  \\ 
 

    \theutterance \stepcounter{utterance}  

    & & \multicolumn{4}{p{0.6\linewidth}}{\cellcolor[rgb]{0.9,0.9,0.9}{%
	\makecell[{{p{\linewidth}}}]{% 
	  \tt {\tiny [A$\langle$GM]}  
	 ARGUMENT: {'Ich verstehe, dass der Vorschlag des anderen Spielers eine gute Balance zwischen Wichtigkeit und Aufwand anstrebt, aber ich bin noch nicht vollständig zufrieden. Ich denke, dass ich eine Alternative finden kann, die den Aufwand weiter reduziert, ohne die Wichtigkeit zu sehr zu beeinträchtigen.'}\\ \tt \\ \tt ABLEHNUNG: {'A36', 'B33', 'C45'}\\ \tt \\ \tt ARGUMENT: {'Ich lehne den Vorschlag ab, da ich denke, dass ich eine bessere Kombination von Gegenständen finden kann, die sowohl meine eigenen Bedürfnisse als auch die des anderen Spielers berücksichtigt. Ich möchte eine Alternative anbieten, die eine faire Verteilung der Wichtigkeit und des Aufwands bietet.'}\\ \tt \\ \tt VORSCHLAG: {'A36', 'B33', 'A63'} \\ \tt \\ \tt ARGUMENT: {'Ich denke, dass dieser Vorschlag eine gute Balance zwischen Wichtigkeit und Aufwand bietet. "A36" und "B33" haben eine hohe Wichtigkeit und einen relativ niedrigen Aufwand, während "A63" eine moderate Wichtigkeit und einen niedrigen Aufwand hat.'} 
	  } 
	   } 
	   } 
	 & \\ 
 

    \theutterance \stepcounter{utterance}  

    & \multicolumn{4}{p{0.6\linewidth}}{\cellcolor[rgb]{0.8,1,0.9}{%
	\makecell[{{p{\linewidth}}}]{% 
	  \tt {\tiny [A$\rangle$GM]}  
	 STRATEGISCHE ÜBERLEGUNGEN: {'Ich habe den neuen Vorschlag des anderen Spielers analysiert und sehe, dass er "A36", "B33" und "A63" vorschlägt. Ich werde diese Kombination bewerten und entscheiden, ob ich sie akzeptieren oder ablehnen soll. Ich denke auch über mögliche Alternativen nach, um den Gesamtaufwand zu minimieren und die Wichtigkeit zu maximieren. Ich bemerke, dass "A36" und "B33" in meinem vorherigen Vorschlag enthalten waren, was darauf hindeutet, dass wir uns in unseren Verhandlungen annähern.'}\\ \tt ARGUMENT: {'Der Vorschlag des anderen Spielers enthält "A36" und "B33", die ich bereits in meinem vorherigen Vorschlag erwähnt habe. "A63" hat eine moderate Wichtigkeit und einen niedrigen Aufwand, was es zu einer guten Ergänzung macht. Ich denke, dass dieser Vorschlag eine gute Balance zwischen Wichtigkeit und Aufwand bietet.'}\\ \tt ARGUMENT: {'Ich denke, dass dieser Vorschlag eine faire Verteilung der Wichtigkeit und des Aufwands bietet. Die Gesamtwichtigkeit ist hoch und der Gesamtaufwand ist innerhalb des Maximalaufwands. Ich bin bereit, diesem Vorschlag zuzustimmen.'}\\ \tt ZUSTIMMUNG: {'A36', 'B33', 'A63'} 
	  } 
	   } 
	   } 
	 & & \\ 
 

    \theutterance \stepcounter{utterance}  

    & & & \multicolumn{2}{p{0.3\linewidth}}{\cellcolor[rgb]{0.95,0.95,0.95}{%
	\makecell[{{p{\linewidth}}}]{% 
	  \tt {\tiny [GM$|$GM]}  
	 game successful 
	  } 
	   } 
	   } 
	 & & \\ 
 

    \theutterance \stepcounter{utterance}  

    & & & \multicolumn{2}{p{0.3\linewidth}}{\cellcolor[rgb]{0.95,0.95,0.95}{%
	\makecell[{{p{\linewidth}}}]{% 
	  \tt {\tiny [GM$|$GM]}  
	 end game 
	  } 
	   } 
	   } 
	 & & \\ 
 

\end{supertabular}
}

\end{document}
