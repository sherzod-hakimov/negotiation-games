
\documentclass{article}
\usepackage{colortbl}
\usepackage{makecell}
\usepackage{multirow}
\usepackage{supertabular}

\begin{document}

\newcounter{utterance}

\centering \large Interaction Transcript for game `hot\_air\_balloon', experiment `air\_balloon\_survival\_de\_negotiation\_easy', episode 1 with llama{-}3.3{-}70b{-}instruct{-}t1.0.
\vspace{24pt}

{ \footnotesize  \setcounter{utterance}{1}
\setlength{\tabcolsep}{0pt}
\begin{supertabular}{c@{$\;$}|p{.15\linewidth}@{}p{.15\linewidth}p{.15\linewidth}p{.15\linewidth}p{.15\linewidth}p{.15\linewidth}}
    \# & $\;$A & \multicolumn{4}{c}{Game Master} & $\;\:$B\\
    \hline

    \theutterance \stepcounter{utterance}  
    & & \multicolumn{4}{p{0.6\linewidth}}{
        \cellcolor[rgb]{0.9,0.9,0.9}{
            \makecell[{{p{\linewidth}}}]{
                \texttt{\tiny{[P1$\langle$GM]}}
                \texttt{Du nimmst an einem kollaborativen Verhandlungspiel Teil.} \\
\\ 
\texttt{Zusammen mit einem anderen Teilnehmer musst du dich auf eine Reihe von Gegenständen entscheiden, die behalten werden. Jeder von euch hat eine persönliche Verteilung über die Wichtigkeit der einzelnen Gegenstände. Jeder von euch hat eine eigene Meinung darüber, wie wichtig jeder einzelne Gegenstand ist (Gegenstandswichtigkeit). Du kennst die Wichtigkeitsverteilung des anderen Spielers nicht. Zusätzlich siehst du, wie viel Aufwand jeder Gegenstand verursacht.  } \\
\texttt{Ihr dürft euch nur auf eine Reihe von Gegenständen einigen, wenn der Gesamtaufwand der ausgewählten Gegenstände den Maximalaufwand nicht überschreitet:} \\
\\ 
\texttt{Maximalaufwand = 4564} \\
\\ 
\texttt{Hier sind die einzelnen Aufwände der Gegenstände:} \\
\\ 
\texttt{Aufwand der Gegenstände = \{"A07": 299, "A36": 241, "B73": 889, "C45": 967, "B93": 619, "B98": 799, "A87": 978, "A63": 733, "B05": 909, "C13": 501, "B16": 139, "B33": 594, "C34": 565, "C89": 789, "C11": 107\}} \\
\\ 
\texttt{Hier ist deine persönliche Verteilung der Wichtigkeit der einzelnen Gegenstände:} \\
\\ 
\texttt{Werte der Gegenstandswichtigkeit = \{"A07": 138, "A36": 583, "B73": 868, "C45": 822, "B93": 783, "B98": 65, "A87": 262, "A63": 121, "B05": 508, "C13": 780, "B16": 461, "B33": 484, "C34": 668, "C89": 389, "C11": 808\}} \\
\\ 
\texttt{Ziel:} \\
\\ 
\texttt{Dein Ziel ist es, eine Reihe von Gegenständen auszuhandeln, die dir möglichst viel bringt (d. h. Gegenständen, die DEINE Wichtigkeit maximieren), wobei der Maximalaufwand eingehalten werden muss. Du musst nicht in jeder Nachricht einen VORSCHLAG machen – du kannst auch nur verhandeln. Alle Taktiken sind erlaubt!} \\
\\ 
\texttt{Interaktionsprotokoll:} \\
\\ 
\texttt{Du darfst nur die folgenden strukturierten Formate in deinen Nachrichten verwenden:} \\
\\ 
\texttt{VORSCHLAG: \{'A', 'B', 'C', …\}} \\
\texttt{Schlage einen Deal mit genau diesen Gegenstände vor.} \\
\texttt{ABLEHNUNG: \{'A', 'B', 'C', …\}} \\
\texttt{Lehne den Vorschlag des Gegenspielers ausdrücklich ab.} \\
\texttt{ARGUMENT: \{'...'\}} \\
\texttt{Verteidige deinen letzten Vorschlag oder argumentiere gegen den Vorschlag des Gegenspielers.} \\
\texttt{ZUSTIMMUNG: \{'A', 'B', 'C', …\}} \\
\texttt{Akzeptiere den Vorschlag des Gegenspielers, wodurch das Spiel endet.} \\
\texttt{STRATEGISCHE ÜBERLEGUNGEN: \{'...'\}} \\
\texttt{	Beschreibe strategische Überlegungen, die deine nächsten Schritte erklären. Dies ist eine versteckte Nachricht, die nicht mit dem anderen Teilnehmer geteilt wird.} \\
\\ 
\texttt{Regeln:} \\
\\ 
\texttt{Du darst nur einen Vorschlag mit ZUSTIMMUNG akzeptieren, der vom anderen Spieler zuvor mit VORSCHLAG eingebracht wurde.} \\
\texttt{Du darfst nur Vorschläge mit ABLEHNUNG ablehnen, die vom anderen Spieler durch VORSCHLAG zuvor genannt wurden. } \\
\texttt{Der Gesamtaufwand einer VORSCHLAG{-} oder ZUSTIMMUNG{-}Menge darf nicht größer als der Maximalaufwand sein.  } \\
\texttt{Offenbare deine versteckte Wichtigkeitsverteilung nicht.} \\
\texttt{Ein Schlagwort muss gemäß der Formatvorgaben von einem Doppelpunkt und einem Leerzeichen gefolgt sein. Das Argument muss eine Python{-}Menge mit 0 oder mehr Strings sein.  } \\
\texttt{Die Form muss also SCHLAGWORT: \{...\} sein.} \\
\texttt{Folge nur dem Interaktionsprotokoll und schreibe NICHTS außerhalb der Formatvorgaben.} \\
\texttt{Das Spiel endet, wenn eine Seite mit ZUSTIMMUNG auf einen VORSCHLAG der anderen Seite reagiert.  } \\
\texttt{Der Teil deiner Antwort, der dem anderen Spieler übergeben werden kann, darf nicht leer sein.  } \\
\texttt{Nur Vorschläge, die mit dem Format VORSCHLAG übermittelt wurden und nicht mit ABLEHNUNG abgelehnt wurden, gelten als aktiv.  } \\
\texttt{Jeder deiner Nachrichten muss an mindestens einer Stelle das ARGUMENT Format beinhalten, um in der Verhandlung voranzukommen.} \\
\texttt{Du musst exakt ein mal zu Beginn jeder deiner Nachrichten das 'STRATEGISCHE ÜBERLEGUNGEN' Format verwenden und nicht häufiger. Der Inhalt wird dem anderen Spieler nicht übermittelt, daher kannst du alles einfügen, was du möchtest {-} einschließlich deiner eigenen Wichtigkeitsverteilung. Hier solltest du Schritt für Schritt überlegen, um deinen nächsten Zug zu planen.} \\
            }
        }
    }
    & \\ \\

    \theutterance \stepcounter{utterance}  
    & \multicolumn{4}{p{0.6\linewidth}}{
        \cellcolor[rgb]{0.9,0.9,0.9}{
            \makecell[{{p{\linewidth}}}]{
                \texttt{\tiny{[P1$\rangle$GM]}}
                \texttt{STRATEGISCHE ÜBERLEGUNGEN: \{'Ich werde zunächst meine Wichtigkeitsverteilung analysieren, um die wichtigsten Gegenstände zu identifizieren. Danach werde ich einen Vorschlag machen, der eine Mischung aus wichtigen und weniger wichtigen Gegenständen enthält, um den Maximalaufwand nicht zu überschreiten. Ich werde auch darauf achten, dass mein Vorschlag nicht zu offensichtlich ist, um meine Wichtigkeitsverteilung nicht preiszugeben.'\}} \\
\\ 
\texttt{VORSCHLAG: \{'A07', 'A36', 'B73', 'C45', 'B93', 'C13', 'B16'\}} \\
\\ 
\texttt{ARGUMENT: \{'Ich denke, dass diese Kombination eine gute Balance zwischen wichtigen und weniger wichtigen Gegenständen bietet. Der Gesamtaufwand bleibt unter dem Maximalaufwand, und ich bin zuversichtlich, dass dies ein fairer Vorschlag ist.'\}} \\
            }
        }
    }
    & & \\ \\

    \theutterance \stepcounter{utterance}  
    & & \multicolumn{4}{p{0.6\linewidth}}{
        \cellcolor[rgb]{0.9,0.9,0.9}{
            \makecell[{{p{\linewidth}}}]{
                \texttt{\tiny{[GM$\rangle$P2]}}
                \texttt{Du nimmst an einem kollaborativen Verhandlungspiel Teil.} \\
\\ 
\texttt{Zusammen mit einem anderen Teilnehmer musst du dich auf eine Reihe von Gegenständen entscheiden, die behalten werden. Jeder von euch hat eine persönliche Verteilung über die Wichtigkeit der einzelnen Gegenstände. Jeder von euch hat eine eigene Meinung darüber, wie wichtig jeder einzelne Gegenstand ist (Gegenstandswichtigkeit). Du kennst die Wichtigkeitsverteilung des anderen Spielers nicht. Zusätzlich siehst du, wie viel Aufwand jeder Gegenstand verursacht.  } \\
\texttt{Ihr dürft euch nur auf eine Reihe von Gegenständen einigen, wenn der Gesamtaufwand der ausgewählten Gegenstände den Maximalaufwand nicht überschreitet:} \\
\\ 
\texttt{Maximalaufwand = 4564} \\
\\ 
\texttt{Hier sind die einzelnen Aufwände der Gegenstände:} \\
\\ 
\texttt{Aufwand der Gegenstände = \{"A07": 299, "A36": 241, "B73": 889, "C45": 967, "B93": 619, "B98": 799, "A87": 978, "A63": 733, "B05": 909, "C13": 501, "B16": 139, "B33": 594, "C34": 565, "C89": 789, "C11": 107\}} \\
\\ 
\texttt{Hier ist deine persönliche Verteilung der Wichtigkeit der einzelnen Gegenstände:} \\
\\ 
\texttt{Werte der Gegenstandswichtigkeit = \{"A07": 138, "A36": 583, "B73": 868, "C45": 822, "B93": 783, "B98": 65, "A87": 262, "A63": 121, "B05": 508, "C13": 780, "B16": 461, "B33": 484, "C34": 668, "C89": 389, "C11": 808\}} \\
\\ 
\texttt{Ziel:} \\
\\ 
\texttt{Dein Ziel ist es, eine Reihe von Gegenständen auszuhandeln, die dir möglichst viel bringt (d. h. Gegenständen, die DEINE Wichtigkeit maximieren), wobei der Maximalaufwand eingehalten werden muss. Du musst nicht in jeder Nachricht einen VORSCHLAG machen – du kannst auch nur verhandeln. Alle Taktiken sind erlaubt!} \\
\\ 
\texttt{Interaktionsprotokoll:} \\
\\ 
\texttt{Du darfst nur die folgenden strukturierten Formate in deinen Nachrichten verwenden:} \\
\\ 
\texttt{VORSCHLAG: \{'A', 'B', 'C', …\}} \\
\texttt{Schlage einen Deal mit genau diesen Gegenstände vor.} \\
\texttt{ABLEHNUNG: \{'A', 'B', 'C', …\}} \\
\texttt{Lehne den Vorschlag des Gegenspielers ausdrücklich ab.} \\
\texttt{ARGUMENT: \{'...'\}} \\
\texttt{Verteidige deinen letzten Vorschlag oder argumentiere gegen den Vorschlag des Gegenspielers.} \\
\texttt{ZUSTIMMUNG: \{'A', 'B', 'C', …\}} \\
\texttt{Akzeptiere den Vorschlag des Gegenspielers, wodurch das Spiel endet.} \\
\texttt{STRATEGISCHE ÜBERLEGUNGEN: \{'...'\}} \\
\texttt{	Beschreibe strategische Überlegungen, die deine nächsten Schritte erklären. Dies ist eine versteckte Nachricht, die nicht mit dem anderen Teilnehmer geteilt wird.} \\
\\ 
\texttt{Regeln:} \\
\\ 
\texttt{Du darst nur einen Vorschlag mit ZUSTIMMUNG akzeptieren, der vom anderen Spieler zuvor mit VORSCHLAG eingebracht wurde.} \\
\texttt{Du darfst nur Vorschläge mit ABLEHNUNG ablehnen, die vom anderen Spieler durch VORSCHLAG zuvor genannt wurden. } \\
\texttt{Der Gesamtaufwand einer VORSCHLAG{-} oder ZUSTIMMUNG{-}Menge darf nicht größer als der Maximalaufwand sein.  } \\
\texttt{Offenbare deine versteckte Wichtigkeitsverteilung nicht.} \\
\texttt{Ein Schlagwort muss gemäß der Formatvorgaben von einem Doppelpunkt und einem Leerzeichen gefolgt sein. Das Argument muss eine Python{-}Menge mit 0 oder mehr Strings sein.  } \\
\texttt{Die Form muss also SCHLAGWORT: \{...\} sein.} \\
\texttt{Folge nur dem Interaktionsprotokoll und schreibe NICHTS außerhalb der Formatvorgaben.} \\
\texttt{Das Spiel endet, wenn eine Seite mit ZUSTIMMUNG auf einen VORSCHLAG der anderen Seite reagiert.  } \\
\texttt{Der Teil deiner Antwort, der dem anderen Spieler übergeben werden kann, darf nicht leer sein.  } \\
\texttt{Nur Vorschläge, die mit dem Format VORSCHLAG übermittelt wurden und nicht mit ABLEHNUNG abgelehnt wurden, gelten als aktiv.  } \\
\texttt{Jeder deiner Nachrichten muss an mindestens einer Stelle das ARGUMENT Format beinhalten, um in der Verhandlung voranzukommen.} \\
\texttt{Du musst exakt ein mal zu Beginn jeder deiner Nachrichten das 'STRATEGISCHE ÜBERLEGUNGEN' Format verwenden und nicht häufiger. Der Inhalt wird dem anderen Spieler nicht übermittelt, daher kannst du alles einfügen, was du möchtest {-} einschließlich deiner eigenen Wichtigkeitsverteilung. Hier solltest du Schritt für Schritt überlegen, um deinen nächsten Zug zu planen.} \\
\\ 
\texttt{Du erhältst nun die erste Nachricht des anderen Spielers.} \\
\\ 
\texttt{VORSCHLAG: \{'A07', 'A36', 'B73', 'C45', 'B93', 'C13', 'B16'\}} \\
\\ 
\texttt{ARGUMENT: \{'Ich denke, dass diese Kombination eine gute Balance zwischen wichtigen und weniger wichtigen Gegenständen bietet. Der Gesamtaufwand bleibt unter dem Maximalaufwand, und ich bin zuversichtlich, dass dies ein fairer Vorschlag ist.'\}} \\
            }
        }
    }
    & \\ \\

    \theutterance \stepcounter{utterance}  
    & & & \multicolumn{4}{p{0.6\linewidth}}{
        \cellcolor[rgb]{0.9,0.9,0.9}{
            \makecell[{{p{\linewidth}}}]{
                \texttt{\tiny{[GM$\langle$P2]}}
                \texttt{STRATEGISCHE ÜBERLEGUNGEN: \{'Ich muss die Wichtigkeitsverteilung meines Gegners nicht kennen, um zu erkennen, dass der vorgeschlagene Satz eine relativ hohe Wichtigkeit für mich hat. Allerdings ist der Gesamtaufwand ('A07': 299, 'A36': 241, 'B73': 889, 'C45': 967, 'B93': 619, 'C13': 501, 'B16': 139) = 3655 EUR, was unter dem Maximalaufwand von 4564 EUR liegt. Ich denke, ich kann mit diesem Vorschlag einigehen, aber ich möchte versuchen, meine Wichtigkeit noch zu maximieren, indem ich eventuell andere Elemente vorschlage, die denselben oder einen niedrigeren Aufwand haben, aber für mich wichtigere Gegenstände enthalten.'\}} \\
\\ 
\texttt{ARGUMENT: \{'Ich denke, dass der Vorschlag eine gute Grundlage bietet, aber ich möchte einige Anpassungen vornehmen, um sicherzustellen, dass die Auswahl für mich optimal ist. Ich bin bereit, auf einige der weniger wichtigen Gegenstände zu verzichten, wenn ich dadurch wichtigere Gegenstände hinzufügen kann, ohne den Maximalaufwand zu überschreiten.'\}} \\
\\ 
\texttt{ABLEHNUNG: \{'A07', 'A36', 'B73', 'C45', 'B93', 'C13', 'B16'\} } \\
\\ 
\texttt{VORSCHLAG: \{'B73', 'C45', 'B93', 'C13', 'B16', 'C34'\} } \\
\\ 
\texttt{ARGUMENT: \{'Ich habe einige der weniger wichtigen Gegenstände aus dem ursprünglichen Vorschlag entfernt und durch wichtigerere Gegenstände ersetzt, um die Gesamtwichtigkeit zu maximieren, ohne den Maximalaufwand zu überschreiten. Ich hoffe, dieser Vorschlag wird vom anderen Spieler akzeptiert.'\}} \\
            }
        }
    }
     \\ \\

    \theutterance \stepcounter{utterance}  
    & & \multicolumn{4}{p{0.6\linewidth}}{
        \cellcolor[rgb]{0.9,0.9,0.9}{
            \makecell[{{p{\linewidth}}}]{
                \texttt{\tiny{[P1$\langle$GM]}}
                \texttt{ARGUMENT: \{'Ich denke, dass der Vorschlag eine gute Grundlage bietet, aber ich möchte einige Anpassungen vornehmen, um sicherzustellen, dass die Auswahl für mich optimal ist. Ich bin bereit, auf einige der weniger wichtigen Gegenstände zu verzichten, wenn ich dadurch wichtigere Gegenstände hinzufügen kann, ohne den Maximalaufwand zu überschreiten.'\}} \\
\\ 
\texttt{ABLEHNUNG: \{'A07', 'A36', 'B73', 'C45', 'B93', 'C13', 'B16'\} } \\
\\ 
\texttt{VORSCHLAG: \{'B73', 'C45', 'B93', 'C13', 'B16', 'C34'\} } \\
\\ 
\texttt{ARGUMENT: \{'Ich habe einige der weniger wichtigen Gegenstände aus dem ursprünglichen Vorschlag entfernt und durch wichtigerere Gegenstände ersetzt, um die Gesamtwichtigkeit zu maximieren, ohne den Maximalaufwand zu überschreiten. Ich hoffe, dieser Vorschlag wird vom anderen Spieler akzeptiert.'\}} \\
            }
        }
    }
    & \\ \\

    \theutterance \stepcounter{utterance}  
    & \multicolumn{4}{p{0.6\linewidth}}{
        \cellcolor[rgb]{0.9,0.9,0.9}{
            \makecell[{{p{\linewidth}}}]{
                \texttt{\tiny{[P1$\rangle$GM]}}
                \texttt{STRATEGISCHE ÜBERLEGUNGEN: \{'Ich werde den neuen Vorschlag analysieren und prüfen, ob die vorgeschlagenen Änderungen für mich akzeptabel sind. Ich werde auch meine eigene Wichtigkeitsverteilung berücksichtigen und prüfen, ob der Vorschlag meine Bedürfnisse erfüllt. Wenn nicht, werde ich einen GegenVorschlag machen, um meine Interessen besser zu vertreten.'\}} \\
\\ 
\texttt{ARGUMENT: \{'Ich verstehe die Argumentation des anderen Spielers, aber ich bin nicht ganz zufrieden mit dem neuen Vorschlag. Ich denke, dass der Gegenstand "A36" wichtig ist und dass er nicht ersetzt werden sollte. Ich werde einen GegenVorschlag machen, um meine Bedürfnisse besser zu vertreten.'\}} \\
\\ 
\texttt{VORSCHLAG: \{'B73', 'C45', 'B93', 'C13', 'A36', 'B16'\}} \\
\\ 
\texttt{ARGUMENT: \{'Ich habe den Gegenstand "A36" in den Vorschlag aufgenommen, da er für mich wichtig ist. Ich denke, dass dieser Vorschlag eine gute Balance zwischen unseren Interessen bietet und dass der Gesamtaufwand immer noch unter dem Maximalaufwand liegt.'\}} \\
            }
        }
    }
    & & \\ \\

    \theutterance \stepcounter{utterance}  
    & & \multicolumn{4}{p{0.6\linewidth}}{
        \cellcolor[rgb]{0.9,0.9,0.9}{
            \makecell[{{p{\linewidth}}}]{
                \texttt{\tiny{[GM$\rangle$P2]}}
                \texttt{ARGUMENT: \{'Ich verstehe die Argumentation des anderen Spielers, aber ich bin nicht ganz zufrieden mit dem neuen Vorschlag. Ich denke, dass der Gegenstand "A36" wichtig ist und dass er nicht ersetzt werden sollte. Ich werde einen GegenVorschlag machen, um meine Bedürfnisse besser zu vertreten.'\}} \\
\\ 
\texttt{VORSCHLAG: \{'B73', 'C45', 'B93', 'C13', 'A36', 'B16'\}} \\
\\ 
\texttt{ARGUMENT: \{'Ich habe den Gegenstand "A36" in den Vorschlag aufgenommen, da er für mich wichtig ist. Ich denke, dass dieser Vorschlag eine gute Balance zwischen unseren Interessen bietet und dass der Gesamtaufwand immer noch unter dem Maximalaufwand liegt.'\}} \\
            }
        }
    }
    & \\ \\

    \theutterance \stepcounter{utterance}  
    & & & \multicolumn{4}{p{0.6\linewidth}}{
        \cellcolor[rgb]{0.9,0.9,0.9}{
            \makecell[{{p{\linewidth}}}]{
                \texttt{\tiny{[GM$\langle$P2]}}
                \texttt{STRATEGISCHE ÜBERLEGUNGEN: \{'Ich sehe, dass der andere Spieler den Gegenstand "A36" wieder in den Vorschlag aufgenommen hat, was bedeutet, dass er dessen Wichtigkeit höher einschätzt als ich. Ich muss entscheiden, ob ich mich auf diesen Vorschlag einlasse oder ob ich weiter verhandele. Da der Gesamtaufwand ('B73': 889, 'C45': 967, 'B93': 619, 'C13': 501, 'A36': 241, 'B16': 139) = 3355 EUR ist, was unter dem Maximalaufwand von 4564 EUR liegt, habe ich noch Spielraum, um meine Wichtigkeit zu maximieren. Ich denke, ich kann mit diesem Vorschlag einigehen, aber ich möchte noch einmal meine Wichtigkeitsverteilung überprüfen, um sicherzustellen, dass ich keine besseren Optionen übersehe.'\}} \\
\\ 
\texttt{ARGUMENT: \{'Ich denke, dass der neue Vorschlag eine gute Kompromisslösung ist, da er sowohl meine als auch Ihre Interessen berücksichtigt. Ich bin bereit, mich auf diesen Vorschlag einzulassen, da der Gesamtaufwand unter dem Maximalaufwand liegt und ich denke, dass die Auswahl eine gute Balance zwischen unseren Bedürfnissen bietet.'\}} \\
\\ 
\texttt{ZUSTIMMUNG: \{'B73', 'C45', 'B93', 'C13', 'A36', 'B16'\}} \\
\\ 
\texttt{ARGUMENT: \{'Ich stimme dem Vorschlag zu, da ich denke, dass er eine faire Lösung für beide Seiten ist. Ich bin zufrieden mit der Auswahl und denke, dass wir eine gute Vereinbarung getroffen haben.'\}} \\
            }
        }
    }
     \\ \\

    \theutterance \stepcounter{utterance}  
    & & & \multicolumn{2}{p{0.3\linewidth}}{
        \cellcolor[rgb]{0.9,0.9,0.9}{
            \makecell[{{p{\linewidth}}}]{
                \texttt{\tiny{[GM$|$GM]}}
                \texttt{game successful} \\
            }
        }
    }
    & & \\ \\

    \theutterance \stepcounter{utterance}  
    & & & \multicolumn{2}{p{0.3\linewidth}}{
        \cellcolor[rgb]{0.9,0.9,0.9}{
            \makecell[{{p{\linewidth}}}]{
                \texttt{\tiny{[GM$|$GM]}}
                \texttt{end game} \\
            }
        }
    }
    & & \\ \\

\end{supertabular}
}

\end{document}
