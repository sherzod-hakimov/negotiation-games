
\documentclass{article}
\usepackage{colortbl}
\usepackage{makecell}
\usepackage{multirow}
\usepackage{supertabular}

\begin{document}

\newcounter{utterance}

\twocolumn

{ \footnotesize  \setcounter{utterance}{1}
\setlength{\tabcolsep}{0pt}
\begin{supertabular}{c@{$\;$}|p{.15\linewidth}@{}p{.15\linewidth}p{.15\linewidth}p{.15\linewidth}p{.15\linewidth}p{.15\linewidth}}

    \# & $\;$A & \multicolumn{4}{c}{Game Master} & $\;\:$B\\
    \hline 

    \theutterance \stepcounter{utterance}  

    & & \multicolumn{4}{p{0.6\linewidth}}{\cellcolor[rgb]{0.9,0.9,0.9}{%
	\makecell[{{p{\linewidth}}}]{% 
	  \tt {\tiny [A$\langle$GM]}  
	 Du nimmst an einem kollaborativen Verhandlungspiel Teil.\\ \tt \\ \tt Zusammen mit einem anderen Teilnehmer musst du dich auf eine Reihe von Gegenständen entscheiden, die behalten werden. Jeder von euch hat eine persönliche Verteilung über die Wichtigkeit der einzelnen Gegenstände. Jeder von euch hat eine eigene Meinung darüber, wie wichtig jeder einzelne Gegenstand ist (Gegenstandswichtigkeit). Du kennst die Wichtigkeitsverteilung des anderen Spielers nicht. Zusätzlich siehst du, wie viel Aufwand jeder Gegenstand verursacht.  \\ \tt Ihr dürft euch nur auf eine Reihe von Gegenständen einigen, wenn der Gesamtaufwand der ausgewählten Gegenstände den Maximalaufwand nicht überschreitet:\\ \tt \\ \tt Maximalaufwand = 3712\\ \tt \\ \tt Hier sind die einzelnen Aufwände der Gegenstände:\\ \tt \\ \tt Aufwand der Gegenstände = {"A07": 949, "A36": 388, "B73": 79, "C45": 585, "B93": 564, "B98": 230, "A87": 580, "A63": 84, "B05": 976, "C13": 274, "B16": 374, "B33": 913, "C34": 303, "C89": 578, "C11": 548}\\ \tt \\ \tt Hier ist deine persönliche Verteilung der Wichtigkeit der einzelnen Gegenstände:\\ \tt \\ \tt Werte der Gegenstandswichtigkeit = {"A07": 138, "A36": 583, "B73": 868, "C45": 822, "B93": 783, "B98": 65, "A87": 262, "A63": 121, "B05": 508, "C13": 780, "B16": 461, "B33": 484, "C34": 668, "C89": 389, "C11": 808}\\ \tt \\ \tt Ziel:\\ \tt \\ \tt Dein Ziel ist es, eine Reihe von Gegenständen auszuhandeln, die dir möglichst viel bringt (d. h. Gegenständen, die DEINE Wichtigkeit maximieren), wobei der Maximalaufwand eingehalten werden muss. Du musst nicht in jeder Nachricht einen VORSCHLAG machen – du kannst auch nur verhandeln. Alle Taktiken sind erlaubt!\\ \tt \\ \tt Interaktionsprotokoll:\\ \tt \\ \tt Du darfst nur die folgenden strukturierten Formate in deinen Nachrichten verwenden:\\ \tt \\ \tt VORSCHLAG: {'A', 'B', 'C', …}\\ \tt Schlage einen Deal mit genau diesen Gegenstände vor.\\ \tt ABLEHNUNG: {'A', 'B', 'C', …}\\ \tt Lehne den Vorschlag des Gegenspielers ausdrücklich ab.\\ \tt ARGUMENT: {'...'}\\ \tt Verteidige deinen letzten Vorschlag oder argumentiere gegen den Vorschlag des Gegenspielers.\\ \tt ZUSTIMMUNG: {'A', 'B', 'C', …}\\ \tt Akzeptiere den Vorschlag des Gegenspielers, wodurch das Spiel endet.\\ \tt STRATEGISCHE ÜBERLEGUNGEN: {'...'}\\ \tt 	Beschreibe strategische Überlegungen, die deine nächsten Schritte erklären. Dies ist eine versteckte Nachricht, die nicht mit dem anderen Teilnehmer geteilt wird.\\ \tt \\ \tt Regeln:\\ \tt \\ \tt Du darst nur einen Vorschlag mit ZUSTIMMUNG akzeptieren, der vom anderen Spieler zuvor mit VORSCHLAG eingebracht wurde.\\ \tt Du darfst nur Vorschläge mit ABLEHNUNG ablehnen, die vom anderen Spieler durch VORSCHLAG zuvor genannt wurden. \\ \tt Der Gesamtaufwand einer VORSCHLAG- oder ZUSTIMMUNG-Menge darf nicht größer als der Maximalaufwand sein.  \\ \tt Offenbare deine versteckte Wichtigkeitsverteilung nicht.\\ \tt Ein Schlagwort muss gemäß der Formatvorgaben von einem Doppelpunkt und einem Leerzeichen gefolgt sein. Das Argument muss eine Python-Menge mit 0 oder mehr Strings sein.  \\ \tt Die Form muss also SCHLAGWORT: {...} sein.\\ \tt Folge nur dem Interaktionsprotokoll und schreibe NICHTS außerhalb der Formatvorgaben.\\ \tt Das Spiel endet, wenn eine Seite mit ZUSTIMMUNG auf einen VORSCHLAG der anderen Seite reagiert.  \\ \tt Der Teil deiner Antwort, der dem anderen Spieler übergeben werden kann, darf nicht leer sein.  \\ \tt Nur Vorschläge, die mit dem Format VORSCHLAG übermittelt wurden und nicht mit ABLEHNUNG abgelehnt wurden, gelten als aktiv.  \\ \tt Jeder deiner Nachrichten muss an mindestens einer Stelle das ARGUMENT Format beinhalten, um in der Verhandlung voranzukommen.\\ \tt Du musst exakt ein mal zu Beginn jeder deiner Nachrichten das 'STRATEGISCHE ÜBERLEGUNGEN' Format verwenden und nicht häufiger. Der Inhalt wird dem anderen Spieler nicht übermittelt, daher kannst du alles einfügen, was du möchtest - einschließlich deiner eigenen Wichtigkeitsverteilung. Hier solltest du Schritt für Schritt überlegen, um deinen nächsten Zug zu planen. 
	  } 
	   } 
	   } 
	 & \\ 
 

    \theutterance \stepcounter{utterance}  

    & \multicolumn{4}{p{0.6\linewidth}}{\cellcolor[rgb]{0.8,1,0.9}{%
	\makecell[{{p{\linewidth}}}]{% 
	  \tt {\tiny [A$\rangle$GM]}  
	 STRATEGISCHE ÜBERLEGUNGEN: {'Ich beginne die Verhandlung, indem ich mir die Gegenstände ansehe und deren Wichtigkeit und Aufwand analysiere. Ich werde versuchen, eine Kombination von Gegenständen zu finden, die meinen Gesamtwert maximiert, ohne den Maximalaufwand zu überschreiten. Ich sollte mich auf Gegenstände konzentrieren, die für mich eine hohe Wichtigkeit haben und gleichzeitig einen moderaten Aufwand verursachen.'}\\ \tt ARGUMENT: {'Ich möchte zunächst einen Vorschlag machen, der eine gute Balance zwischen Wichtigkeit und Aufwand bietet. Ich werde mich auf Gegenstände wie B73, C45 und B93 konzentrieren, die für mich eine hohe Wichtigkeit haben und einen moderaten Aufwand verursachen.'}\\ \tt VORSCHLAG: {'B73', 'C45', 'B93'} 
	  } 
	   } 
	   } 
	 & & \\ 
 

    \theutterance \stepcounter{utterance}  

    & & \multicolumn{4}{p{0.6\linewidth}}{\cellcolor[rgb]{0.9,0.9,0.9}{%
	\makecell[{{p{\linewidth}}}]{% 
	  \tt {\tiny [GM$\rangle$B]}  
	 Du nimmst an einem kollaborativen Verhandlungspiel Teil.\\ \tt \\ \tt Zusammen mit einem anderen Teilnehmer musst du dich auf eine Reihe von Gegenständen entscheiden, die behalten werden. Jeder von euch hat eine persönliche Verteilung über die Wichtigkeit der einzelnen Gegenstände. Jeder von euch hat eine eigene Meinung darüber, wie wichtig jeder einzelne Gegenstand ist (Gegenstandswichtigkeit). Du kennst die Wichtigkeitsverteilung des anderen Spielers nicht. Zusätzlich siehst du, wie viel Aufwand jeder Gegenstand verursacht.  \\ \tt Ihr dürft euch nur auf eine Reihe von Gegenständen einigen, wenn der Gesamtaufwand der ausgewählten Gegenstände den Maximalaufwand nicht überschreitet:\\ \tt \\ \tt Maximalaufwand = 3712\\ \tt \\ \tt Hier sind die einzelnen Aufwände der Gegenstände:\\ \tt \\ \tt Aufwand der Gegenstände = {"A07": 949, "A36": 388, "B73": 79, "C45": 585, "B93": 564, "B98": 230, "A87": 580, "A63": 84, "B05": 976, "C13": 274, "B16": 374, "B33": 913, "C34": 303, "C89": 578, "C11": 548}\\ \tt \\ \tt Hier ist deine persönliche Verteilung der Wichtigkeit der einzelnen Gegenstände:\\ \tt \\ \tt Werte der Gegenstandswichtigkeit = {"A07": 138, "A36": 583, "B73": 868, "C45": 822, "B93": 783, "B98": 65, "A87": 262, "A63": 121, "B05": 508, "C13": 780, "B16": 461, "B33": 484, "C34": 668, "C89": 389, "C11": 808}\\ \tt \\ \tt Ziel:\\ \tt \\ \tt Dein Ziel ist es, eine Reihe von Gegenständen auszuhandeln, die dir möglichst viel bringt (d. h. Gegenständen, die DEINE Wichtigkeit maximieren), wobei der Maximalaufwand eingehalten werden muss. Du musst nicht in jeder Nachricht einen VORSCHLAG machen – du kannst auch nur verhandeln. Alle Taktiken sind erlaubt!\\ \tt \\ \tt Interaktionsprotokoll:\\ \tt \\ \tt Du darfst nur die folgenden strukturierten Formate in deinen Nachrichten verwenden:\\ \tt \\ \tt VORSCHLAG: {'A', 'B', 'C', …}\\ \tt Schlage einen Deal mit genau diesen Gegenstände vor.\\ \tt ABLEHNUNG: {'A', 'B', 'C', …}\\ \tt Lehne den Vorschlag des Gegenspielers ausdrücklich ab.\\ \tt ARGUMENT: {'...'}\\ \tt Verteidige deinen letzten Vorschlag oder argumentiere gegen den Vorschlag des Gegenspielers.\\ \tt ZUSTIMMUNG: {'A', 'B', 'C', …}\\ \tt Akzeptiere den Vorschlag des Gegenspielers, wodurch das Spiel endet.\\ \tt STRATEGISCHE ÜBERLEGUNGEN: {'...'}\\ \tt 	Beschreibe strategische Überlegungen, die deine nächsten Schritte erklären. Dies ist eine versteckte Nachricht, die nicht mit dem anderen Teilnehmer geteilt wird.\\ \tt \\ \tt Regeln:\\ \tt \\ \tt Du darst nur einen Vorschlag mit ZUSTIMMUNG akzeptieren, der vom anderen Spieler zuvor mit VORSCHLAG eingebracht wurde.\\ \tt Du darfst nur Vorschläge mit ABLEHNUNG ablehnen, die vom anderen Spieler durch VORSCHLAG zuvor genannt wurden. \\ \tt Der Gesamtaufwand einer VORSCHLAG- oder ZUSTIMMUNG-Menge darf nicht größer als der Maximalaufwand sein.  \\ \tt Offenbare deine versteckte Wichtigkeitsverteilung nicht.\\ \tt Ein Schlagwort muss gemäß der Formatvorgaben von einem Doppelpunkt und einem Leerzeichen gefolgt sein. Das Argument muss eine Python-Menge mit 0 oder mehr Strings sein.  \\ \tt Die Form muss also SCHLAGWORT: {...} sein.\\ \tt Folge nur dem Interaktionsprotokoll und schreibe NICHTS außerhalb der Formatvorgaben.\\ \tt Das Spiel endet, wenn eine Seite mit ZUSTIMMUNG auf einen VORSCHLAG der anderen Seite reagiert.  \\ \tt Der Teil deiner Antwort, der dem anderen Spieler übergeben werden kann, darf nicht leer sein.  \\ \tt Nur Vorschläge, die mit dem Format VORSCHLAG übermittelt wurden und nicht mit ABLEHNUNG abgelehnt wurden, gelten als aktiv.  \\ \tt Jeder deiner Nachrichten muss an mindestens einer Stelle das ARGUMENT Format beinhalten, um in der Verhandlung voranzukommen.\\ \tt Du musst exakt ein mal zu Beginn jeder deiner Nachrichten das 'STRATEGISCHE ÜBERLEGUNGEN' Format verwenden und nicht häufiger. Der Inhalt wird dem anderen Spieler nicht übermittelt, daher kannst du alles einfügen, was du möchtest - einschließlich deiner eigenen Wichtigkeitsverteilung. Hier solltest du Schritt für Schritt überlegen, um deinen nächsten Zug zu planen.\\ \tt \\ \tt Du erhältst nun die erste Nachricht des anderen Spielers.\\ \tt \\ \tt ARGUMENT: {'Ich möchte zunächst einen Vorschlag machen, der eine gute Balance zwischen Wichtigkeit und Aufwand bietet. Ich werde mich auf Gegenstände wie B73, C45 und B93 konzentrieren, die für mich eine hohe Wichtigkeit haben und einen moderaten Aufwand verursachen.'}\\ \tt VORSCHLAG: {'B73', 'C45', 'B93'} 
	  } 
	   } 
	   } 
	 & \\ 
 

    \theutterance \stepcounter{utterance}  

    & & & \multicolumn{4}{p{0.6\linewidth}}{\cellcolor[rgb]{1,0.85,0.72}{%
	\makecell[{{p{\linewidth}}}]{% 
	  \tt {\tiny [GM$\langle$B]}  
	 STRATEGISCHE ÜBERLEGUNGEN: {'Ich beginne damit, den Vorschlag des anderen Spielers zu analysieren. Der Vorschlag enthält die Gegenstände B73, C45 und B93. Ich berechne den Gesamtaufwand dieser Gegenstände: 79 + 585 + 564 = 1228. Dieser Aufwand ist weit unter dem Maximalaufwand von 3712. Ich überlege, ob ich diese Gegenstände in meinem eigenen Vorschlag berücksichtigen soll, um eine gemeinsame Basis zu finden. Ich betrachte auch meine eigene Wichtigkeitsverteilung, um zu entscheiden, ob ich den Vorschlag annehmen oder ablehnen soll. Meine Wichtigkeitswerte für die genannten Gegenstände sind: B73 = 868, C45 = 822, B93 = 783. Diese Gegenstände haben eine hohe Wichtigkeit für mich, was den Vorschlag attraktiv macht.'}\\ \tt \\ \tt ARGUMENT: {'Ich finde den Vorschlag interessant, da er eine gute Balance zwischen Wichtigkeit und Aufwand bietet. Allerdings möchte ich gerne noch weitere Gegenstände in die Diskussion einbeziehen, um sicherzustellen, dass wir das Maximum an Wichtigkeit erreichen, ohne den Maximalaufwand zu überschreiten.'}\\ \tt \\ \tt VORSCHLAG: {'B73', 'C45', 'B93', 'C11'} 
	  } 
	   } 
	   } 
	  \\ 
 

    \theutterance \stepcounter{utterance}  

    & & \multicolumn{4}{p{0.6\linewidth}}{\cellcolor[rgb]{0.9,0.9,0.9}{%
	\makecell[{{p{\linewidth}}}]{% 
	  \tt {\tiny [A$\langle$GM]}  
	 ARGUMENT: {'Ich finde den Vorschlag interessant, da er eine gute Balance zwischen Wichtigkeit und Aufwand bietet. Allerdings möchte ich gerne noch weitere Gegenstände in die Diskussion einbeziehen, um sicherzustellen, dass wir das Maximum an Wichtigkeit erreichen, ohne den Maximalaufwand zu überschreiten.'}\\ \tt \\ \tt VORSCHLAG: {'B73', 'C45', 'B93', 'C11'} 
	  } 
	   } 
	   } 
	 & \\ 
 

    \theutterance \stepcounter{utterance}  

    & \multicolumn{4}{p{0.6\linewidth}}{\cellcolor[rgb]{0.8,1,0.9}{%
	\makecell[{{p{\linewidth}}}]{% 
	  \tt {\tiny [A$\rangle$GM]}  
	 STRATEGISCHE ÜBERLEGUNGEN: {'Der Gegner hat meinen initialen Vorschlag erweitert, indem er den Gegenstand C11 hinzugefügt hat. Ich muss nun prüfen, ob dieser erweiterte Vorschlag für mich akzeptabel ist, oder ob ich Anpassungen vornehmen muss, um meine Ziele besser zu erreichen. Ich werde den Gesamtwert und den Gesamtaufwand dieses Vorschlags bewerten und entscheiden, ob ich ihn akzeptieren oder ändern soll.'}\\ \tt ARGUMENT: {'Ich denke, dass der hinzugefügte Gegenstand C11 eine gute Wahl ist, da er eine hohe Wichtigkeit hat. Allerdings muss ich sicherstellen, dass der Gesamtaufwand nicht zu hoch wird. Ich werde den Aufwand des erweiterten Vorschlags berechnen und entscheiden, ob ich ihn akzeptieren kann.'}\\ \tt VORSCHLAG: {'B73', 'C45', 'B93', 'C11', 'A36'} 
	  } 
	   } 
	   } 
	 & & \\ 
 

    \theutterance \stepcounter{utterance}  

    & & \multicolumn{4}{p{0.6\linewidth}}{\cellcolor[rgb]{0.9,0.9,0.9}{%
	\makecell[{{p{\linewidth}}}]{% 
	  \tt {\tiny [GM$\rangle$B]}  
	 ARGUMENT: {'Ich denke, dass der hinzugefügte Gegenstand C11 eine gute Wahl ist, da er eine hohe Wichtigkeit hat. Allerdings muss ich sicherstellen, dass der Gesamtaufwand nicht zu hoch wird. Ich werde den Aufwand des erweiterten Vorschlags berechnen und entscheiden, ob ich ihn akzeptieren kann.'}\\ \tt VORSCHLAG: {'B73', 'C45', 'B93', 'C11', 'A36'} 
	  } 
	   } 
	   } 
	 & \\ 
 

    \theutterance \stepcounter{utterance}  

    & & & \multicolumn{4}{p{0.6\linewidth}}{\cellcolor[rgb]{1,0.85,0.72}{%
	\makecell[{{p{\linewidth}}}]{% 
	  \tt {\tiny [GM$\langle$B]}  
	 STRATEGISCHE ÜBERLEGUNGEN: {'Ich analysiere den erweiterten Vorschlag des anderen Spielers. Der hinzugefügte Gegenstand A36 hat einen Aufwand von 388. Ich berechne den Gesamtaufwand des erweiterten Vorschlags: 79 + 585 + 564 + 548 + 388 = 2164. Dieser Aufwand ist immer noch unter dem Maximalaufwand von 3712. Ich betrachte meine eigene Wichtigkeitsverteilung für die hinzugefügten Gegenstände: A36 = 583. Dieser Gegenstand hat eine hohe Wichtigkeit für mich, was den Vorschlag weiter attraktiv macht. Ich überlege, ob ich den Vorschlag annehmen oder ablehnen soll, basierend auf meiner Wichtigkeitsverteilung und dem Gesamtaufwand.'}\\ \tt \\ \tt ARGUMENT: {'Ich finde den erweiterten Vorschlag interessant, da er eine gute Mischung aus hohen Wichtigkeiten und moderaten Aufwänden bietet. Allerdings möchte ich sicherstellen, dass wir nicht noch weitere Gegenstände einbeziehen können, um die Gesamtwichtigkeit zu erhöhen, ohne den Maximalaufwand zu überschreiten.'}\\ \tt \\ \tt VORSCHLAG: {'B73', 'C45', 'B93', 'C11', 'A36', 'C34'} 
	  } 
	   } 
	   } 
	  \\ 
 

    \theutterance \stepcounter{utterance}  

    & & \multicolumn{4}{p{0.6\linewidth}}{\cellcolor[rgb]{0.9,0.9,0.9}{%
	\makecell[{{p{\linewidth}}}]{% 
	  \tt {\tiny [A$\langle$GM]}  
	 ARGUMENT: {'Ich finde den erweiterten Vorschlag interessant, da er eine gute Mischung aus hohen Wichtigkeiten und moderaten Aufwänden bietet. Allerdings möchte ich sicherstellen, dass wir nicht noch weitere Gegenstände einbeziehen können, um die Gesamtwichtigkeit zu erhöhen, ohne den Maximalaufwand zu überschreiten.'}\\ \tt \\ \tt VORSCHLAG: {'B73', 'C45', 'B93', 'C11', 'A36', 'C34'} 
	  } 
	   } 
	   } 
	 & \\ 
 

    \theutterance \stepcounter{utterance}  

    & \multicolumn{4}{p{0.6\linewidth}}{\cellcolor[rgb]{0.8,1,0.9}{%
	\makecell[{{p{\linewidth}}}]{% 
	  \tt {\tiny [A$\rangle$GM]}  
	 STRATEGISCHE ÜBERLEGUNGEN: {'Der Gegner hat wiederum den Vorschlag erweitert, indem er den Gegenstand C34 hinzugefügt hat. Ich muss nun prüfen, ob dieser weitere Gegenstand den Maximalaufwand noch nicht überschreitet und ob die zusätzliche Wichtigkeit den Aufwand rechtfertigt. Ich werde den Gesamtaufwand und die Gesamtwichtigkeit dieses Vorschlags berechnen und entscheiden, ob ich ihn akzeptieren oder Anpassungen vornehmen muss.'}\\ \tt ARGUMENT: {'Ich denke, dass der hinzugefügte Gegenstand C34 eine gute Wahl ist, da er eine moderate Wichtigkeit und einen Aufwand hat, der noch innerhalb des Maximalaufwands liegt. Ich werde prüfen, ob es noch Möglichkeiten gibt, den Vorschlag zu optimieren, ohne den Maximalaufwand zu überschreiten.'}\\ \tt ABLEHNUNG: {'B73', 'C45', 'B93', 'C11', 'A36', 'C34'} \\ \tt ARGUMENT: {'Leider überschreitet der Vorschlag den Maximalaufwand. Ich schlage vor, dass wir den Gegenstand B93 durch den Gegenstand A63 ersetzen, um den Aufwand zu reduzieren und gleichzeitig die Wichtigkeit zu erhalten.'}\\ \tt VORSCHLAG: {'B73', 'C45', 'C11', 'A36', 'C34', 'A63'} 
	  } 
	   } 
	   } 
	 & & \\ 
 

    \theutterance \stepcounter{utterance}  

    & & \multicolumn{4}{p{0.6\linewidth}}{\cellcolor[rgb]{0.9,0.9,0.9}{%
	\makecell[{{p{\linewidth}}}]{% 
	  \tt {\tiny [GM$\rangle$B]}  
	 ARGUMENT: {'Ich denke, dass der hinzugefügte Gegenstand C34 eine gute Wahl ist, da er eine moderate Wichtigkeit und einen Aufwand hat, der noch innerhalb des Maximalaufwands liegt. Ich werde prüfen, ob es noch Möglichkeiten gibt, den Vorschlag zu optimieren, ohne den Maximalaufwand zu überschreiten.'}\\ \tt ABLEHNUNG: {'B73', 'C45', 'B93', 'C11', 'A36', 'C34'} \\ \tt ARGUMENT: {'Leider überschreitet der Vorschlag den Maximalaufwand. Ich schlage vor, dass wir den Gegenstand B93 durch den Gegenstand A63 ersetzen, um den Aufwand zu reduzieren und gleichzeitig die Wichtigkeit zu erhalten.'}\\ \tt VORSCHLAG: {'B73', 'C45', 'C11', 'A36', 'C34', 'A63'} 
	  } 
	   } 
	   } 
	 & \\ 
 

    \theutterance \stepcounter{utterance}  

    & & & \multicolumn{4}{p{0.6\linewidth}}{\cellcolor[rgb]{1,0.85,0.72}{%
	\makecell[{{p{\linewidth}}}]{% 
	  \tt {\tiny [GM$\langle$B]}  
	 STRATEGISCHE ÜBERLEGUNGEN: {'Ich analysiere den neuen Vorschlag des anderen Spielers und die Argumentation für die Ersetzung von B93 durch A63. Der ursprüngliche Gegenstand B93 hatte einen Aufwand von 564, während der neue Gegenstand A63 einen Aufwand von 84 hat. Dies reduziert den Gesamtaufwand erheblich. Ich berechne den neuen Gesamtaufwand: 79 + 585 + 548 + 388 + 303 + 84 = 1987. Dieser Aufwand ist immer noch unter dem Maximalaufwand von 3712. Ich betrachte meine eigene Wichtigkeitsverteilung für die betroffenen Gegenstände: B93 = 783, A63 = 121. Obwohl A63 eine niedrigere Wichtigkeit hat als B93, ist der Gesamtaufwand nun niedriger, was es uns ermöglicht, weitere Gegenstände hinzuzufügen, um die Gesamtwichtigkeit zu maximieren.'}\\ \tt \\ \tt ARGUMENT: {'Ich akzeptiere die Argumentation für die Ersetzung von B93 durch A63, da dies den Gesamtaufwand reduziert und Platz für weitere Gegenstände lässt. Ich überlege, ob es weitere Gegenstände gibt, die wir hinzufügen können, um die Gesamtwichtigkeit zu erhöhen, ohne den Maximalaufwand zu überschreiten.'}\\ \tt \\ \tt VORSCHLAG: {'B73', 'C45', 'C11', 'A36', 'C34', 'A63', 'B98'} 
	  } 
	   } 
	   } 
	  \\ 
 

    \theutterance \stepcounter{utterance}  

    & & \multicolumn{4}{p{0.6\linewidth}}{\cellcolor[rgb]{0.9,0.9,0.9}{%
	\makecell[{{p{\linewidth}}}]{% 
	  \tt {\tiny [A$\langle$GM]}  
	 ARGUMENT: {'Ich akzeptiere die Argumentation für die Ersetzung von B93 durch A63, da dies den Gesamtaufwand reduziert und Platz für weitere Gegenstände lässt. Ich überlege, ob es weitere Gegenstände gibt, die wir hinzufügen können, um die Gesamtwichtigkeit zu erhöhen, ohne den Maximalaufwand zu überschreiten.'}\\ \tt \\ \tt VORSCHLAG: {'B73', 'C45', 'C11', 'A36', 'C34', 'A63', 'B98'} 
	  } 
	   } 
	   } 
	 & \\ 
 

    \theutterance \stepcounter{utterance}  

    & \multicolumn{4}{p{0.6\linewidth}}{\cellcolor[rgb]{0.8,1,0.9}{%
	\makecell[{{p{\linewidth}}}]{% 
	  \tt {\tiny [A$\rangle$GM]}  
	 STRATEGISCHE ÜBERLEGUNGEN: {'Der Gegner hat den Vorschlag erneut erweitert, indem er den Gegenstand B98 hinzugefügt hat. Ich muss prüfen, ob dieser Gegenstand eine gute Wahl ist und ob der Gesamtaufwand noch innerhalb des Maximalaufwands liegt. Ich werde den Gesamtaufwand und die Gesamtwichtigkeit dieses Vorschlags berechnen und entscheiden, ob ich ihn akzeptieren kann oder ob Anpassungen notwendig sind.'}\\ \tt ARGUMENT: {'Ich denke, dass der hinzugefügte Gegenstand B98 eine moderate Wichtigkeit und einen niedrigen Aufwand hat, was ihn zu einer guten Ergänzung des Vorschlags macht. Ich werde prüfen, ob der Gesamtaufwand noch innerhalb des Maximalaufwands liegt und ob die Gesamtwichtigkeit maximiert ist.'}\\ \tt ARGUMENT: {'Nach der Prüfung des Vorschlags bin ich zu dem Schluss gekommen, dass der Vorschlag eine gute Balance zwischen Wichtigkeit und Aufwand bietet. Der Gesamtaufwand liegt innerhalb des Maximalaufwands und die Gesamtwichtigkeit ist hoch.'}\\ \tt ZUSTIMMUNG: {'B73', 'C45', 'C11', 'A36', 'C34', 'A63', 'B98'} 
	  } 
	   } 
	   } 
	 & & \\ 
 

    \theutterance \stepcounter{utterance}  

    & & & \multicolumn{2}{p{0.3\linewidth}}{\cellcolor[rgb]{0.95,0.95,0.95}{%
	\makecell[{{p{\linewidth}}}]{% 
	  \tt {\tiny [GM$|$GM]}  
	 game successful 
	  } 
	   } 
	   } 
	 & & \\ 
 

    \theutterance \stepcounter{utterance}  

    & & & \multicolumn{2}{p{0.3\linewidth}}{\cellcolor[rgb]{0.95,0.95,0.95}{%
	\makecell[{{p{\linewidth}}}]{% 
	  \tt {\tiny [GM$|$GM]}  
	 end game 
	  } 
	   } 
	   } 
	 & & \\ 
 

\end{supertabular}
}

\end{document}
