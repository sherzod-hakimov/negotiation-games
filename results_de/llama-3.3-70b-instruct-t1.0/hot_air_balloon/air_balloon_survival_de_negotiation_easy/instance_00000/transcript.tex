
\documentclass{article}
\usepackage{colortbl}
\usepackage{makecell}
\usepackage{multirow}
\usepackage{supertabular}

\begin{document}

\newcounter{utterance}

\twocolumn

{ \footnotesize  \setcounter{utterance}{1}
\setlength{\tabcolsep}{0pt}
\begin{supertabular}{c@{$\;$}|p{.15\linewidth}@{}p{.15\linewidth}p{.15\linewidth}p{.15\linewidth}p{.15\linewidth}p{.15\linewidth}}

    \# & $\;$A & \multicolumn{4}{c}{Game Master} & $\;\:$B\\
    \hline 

    \theutterance \stepcounter{utterance}  

    & & \multicolumn{4}{p{0.6\linewidth}}{\cellcolor[rgb]{0.9,0.9,0.9}{%
	\makecell[{{p{\linewidth}}}]{% 
	  \tt {\tiny [A$\langle$GM]}  
	 Du nimmst an einem kollaborativen Verhandlungspiel Teil.\\ \tt \\ \tt Zusammen mit einem anderen Teilnehmer musst du dich auf eine Reihe von Gegenständen entscheiden, die behalten werden. Jeder von euch hat eine persönliche Verteilung über die Wichtigkeit der einzelnen Gegenstände. Jeder von euch hat eine eigene Meinung darüber, wie wichtig jeder einzelne Gegenstand ist (Gegenstandswichtigkeit). Du kennst die Wichtigkeitsverteilung des anderen Spielers nicht. Zusätzlich siehst du, wie viel Aufwand jeder Gegenstand verursacht.  \\ \tt Ihr dürft euch nur auf eine Reihe von Gegenständen einigen, wenn der Gesamtaufwand der ausgewählten Gegenstände den Maximalaufwand nicht überschreitet:\\ \tt \\ \tt Maximalaufwand = 3358\\ \tt \\ \tt Hier sind die einzelnen Aufwände der Gegenstände:\\ \tt \\ \tt Aufwand der Gegenstände = {"A07": 171, "A36": 933, "B73": 552, "C45": 219, "B93": 275, "B98": 778, "A87": 341, "A63": 615, "B05": 519, "C13": 862, "B16": 262, "B33": 377, "C34": 347, "C89": 349, "C11": 117}\\ \tt \\ \tt Hier ist deine persönliche Verteilung der Wichtigkeit der einzelnen Gegenstände:\\ \tt \\ \tt Werte der Gegenstandswichtigkeit = {"A07": 138, "A36": 583, "B73": 868, "C45": 822, "B93": 783, "B98": 65, "A87": 262, "A63": 121, "B05": 508, "C13": 780, "B16": 461, "B33": 484, "C34": 668, "C89": 389, "C11": 808}\\ \tt \\ \tt Ziel:\\ \tt \\ \tt Dein Ziel ist es, eine Reihe von Gegenständen auszuhandeln, die dir möglichst viel bringt (d. h. Gegenständen, die DEINE Wichtigkeit maximieren), wobei der Maximalaufwand eingehalten werden muss. Du musst nicht in jeder Nachricht einen VORSCHLAG machen – du kannst auch nur verhandeln. Alle Taktiken sind erlaubt!\\ \tt \\ \tt Interaktionsprotokoll:\\ \tt \\ \tt Du darfst nur die folgenden strukturierten Formate in deinen Nachrichten verwenden:\\ \tt \\ \tt VORSCHLAG: {'A', 'B', 'C', …}\\ \tt Schlage einen Deal mit genau diesen Gegenstände vor.\\ \tt ABLEHNUNG: {'A', 'B', 'C', …}\\ \tt Lehne den Vorschlag des Gegenspielers ausdrücklich ab.\\ \tt ARGUMENT: {'...'}\\ \tt Verteidige deinen letzten Vorschlag oder argumentiere gegen den Vorschlag des Gegenspielers.\\ \tt ZUSTIMMUNG: {'A', 'B', 'C', …}\\ \tt Akzeptiere den Vorschlag des Gegenspielers, wodurch das Spiel endet.\\ \tt STRATEGISCHE ÜBERLEGUNGEN: {'...'}\\ \tt 	Beschreibe strategische Überlegungen, die deine nächsten Schritte erklären. Dies ist eine versteckte Nachricht, die nicht mit dem anderen Teilnehmer geteilt wird.\\ \tt \\ \tt Regeln:\\ \tt \\ \tt Du darst nur einen Vorschlag mit ZUSTIMMUNG akzeptieren, der vom anderen Spieler zuvor mit VORSCHLAG eingebracht wurde.\\ \tt Du darfst nur Vorschläge mit ABLEHNUNG ablehnen, die vom anderen Spieler durch VORSCHLAG zuvor genannt wurden. \\ \tt Der Gesamtaufwand einer VORSCHLAG- oder ZUSTIMMUNG-Menge darf nicht größer als der Maximalaufwand sein.  \\ \tt Offenbare deine versteckte Wichtigkeitsverteilung nicht.\\ \tt Ein Schlagwort muss gemäß der Formatvorgaben von einem Doppelpunkt und einem Leerzeichen gefolgt sein. Das Argument muss eine Python-Menge mit 0 oder mehr Strings sein.  \\ \tt Die Form muss also SCHLAGWORT: {...} sein.\\ \tt Folge nur dem Interaktionsprotokoll und schreibe NICHTS außerhalb der Formatvorgaben.\\ \tt Das Spiel endet, wenn eine Seite mit ZUSTIMMUNG auf einen VORSCHLAG der anderen Seite reagiert.  \\ \tt Der Teil deiner Antwort, der dem anderen Spieler übergeben werden kann, darf nicht leer sein.  \\ \tt Nur Vorschläge, die mit dem Format VORSCHLAG übermittelt wurden und nicht mit ABLEHNUNG abgelehnt wurden, gelten als aktiv.  \\ \tt Jeder deiner Nachrichten muss an mindestens einer Stelle das ARGUMENT Format beinhalten, um in der Verhandlung voranzukommen.\\ \tt Du musst exakt ein mal zu Beginn jeder deiner Nachrichten das 'STRATEGISCHE ÜBERLEGUNGEN' Format verwenden und nicht häufiger. Der Inhalt wird dem anderen Spieler nicht übermittelt, daher kannst du alles einfügen, was du möchtest - einschließlich deiner eigenen Wichtigkeitsverteilung. Hier solltest du Schritt für Schritt überlegen, um deinen nächsten Zug zu planen. 
	  } 
	   } 
	   } 
	 & \\ 
 

    \theutterance \stepcounter{utterance}  

    & \multicolumn{4}{p{0.6\linewidth}}{\cellcolor[rgb]{0.8,1,0.9}{%
	\makecell[{{p{\linewidth}}}]{% 
	  \tt {\tiny [A$\rangle$GM]}  
	 STRATEGISCHE ÜBERLEGUNGEN: {'Ich sollte zuerst einen Vorschlag machen, der für mich wertvoll ist, aber nicht zu viel Aufwand verursacht. Ich werde mit einem Angebot beginnen, das meine wichtigsten Gegenstände enthält, aber innerhalb des Maximalaufwands bleibt.'}\\ \tt \\ \tt ARGUMENT: {'Ich möchte einen Vorschlag machen, der eine gute Balance zwischen Aufwand und Wert bietet. Ich denke, dass die Gegenstände B73, C45, B93 und C11 eine gute Kombination sind, da sie eine hohe Wichtigkeit haben und zusammen einen Aufwand von 171 + 933 + 552 + 219 + 275 = 2150 verursachen, was innerhalb des Maximalaufwands von 3358 liegt.'}\\ \tt \\ \tt VORSCHLAG: {'B73', 'C45', 'B93', 'C11'} 
	  } 
	   } 
	   } 
	 & & \\ 
 

    \theutterance \stepcounter{utterance}  

    & & \multicolumn{4}{p{0.6\linewidth}}{\cellcolor[rgb]{0.9,0.9,0.9}{%
	\makecell[{{p{\linewidth}}}]{% 
	  \tt {\tiny [GM$\rangle$B]}  
	 Du nimmst an einem kollaborativen Verhandlungspiel Teil.\\ \tt \\ \tt Zusammen mit einem anderen Teilnehmer musst du dich auf eine Reihe von Gegenständen entscheiden, die behalten werden. Jeder von euch hat eine persönliche Verteilung über die Wichtigkeit der einzelnen Gegenstände. Jeder von euch hat eine eigene Meinung darüber, wie wichtig jeder einzelne Gegenstand ist (Gegenstandswichtigkeit). Du kennst die Wichtigkeitsverteilung des anderen Spielers nicht. Zusätzlich siehst du, wie viel Aufwand jeder Gegenstand verursacht.  \\ \tt Ihr dürft euch nur auf eine Reihe von Gegenständen einigen, wenn der Gesamtaufwand der ausgewählten Gegenstände den Maximalaufwand nicht überschreitet:\\ \tt \\ \tt Maximalaufwand = 3358\\ \tt \\ \tt Hier sind die einzelnen Aufwände der Gegenstände:\\ \tt \\ \tt Aufwand der Gegenstände = {"A07": 171, "A36": 933, "B73": 552, "C45": 219, "B93": 275, "B98": 778, "A87": 341, "A63": 615, "B05": 519, "C13": 862, "B16": 262, "B33": 377, "C34": 347, "C89": 349, "C11": 117}\\ \tt \\ \tt Hier ist deine persönliche Verteilung der Wichtigkeit der einzelnen Gegenstände:\\ \tt \\ \tt Werte der Gegenstandswichtigkeit = {"A07": 138, "A36": 583, "B73": 868, "C45": 822, "B93": 783, "B98": 65, "A87": 262, "A63": 121, "B05": 508, "C13": 780, "B16": 461, "B33": 484, "C34": 668, "C89": 389, "C11": 808}\\ \tt \\ \tt Ziel:\\ \tt \\ \tt Dein Ziel ist es, eine Reihe von Gegenständen auszuhandeln, die dir möglichst viel bringt (d. h. Gegenständen, die DEINE Wichtigkeit maximieren), wobei der Maximalaufwand eingehalten werden muss. Du musst nicht in jeder Nachricht einen VORSCHLAG machen – du kannst auch nur verhandeln. Alle Taktiken sind erlaubt!\\ \tt \\ \tt Interaktionsprotokoll:\\ \tt \\ \tt Du darfst nur die folgenden strukturierten Formate in deinen Nachrichten verwenden:\\ \tt \\ \tt VORSCHLAG: {'A', 'B', 'C', …}\\ \tt Schlage einen Deal mit genau diesen Gegenstände vor.\\ \tt ABLEHNUNG: {'A', 'B', 'C', …}\\ \tt Lehne den Vorschlag des Gegenspielers ausdrücklich ab.\\ \tt ARGUMENT: {'...'}\\ \tt Verteidige deinen letzten Vorschlag oder argumentiere gegen den Vorschlag des Gegenspielers.\\ \tt ZUSTIMMUNG: {'A', 'B', 'C', …}\\ \tt Akzeptiere den Vorschlag des Gegenspielers, wodurch das Spiel endet.\\ \tt STRATEGISCHE ÜBERLEGUNGEN: {'...'}\\ \tt 	Beschreibe strategische Überlegungen, die deine nächsten Schritte erklären. Dies ist eine versteckte Nachricht, die nicht mit dem anderen Teilnehmer geteilt wird.\\ \tt \\ \tt Regeln:\\ \tt \\ \tt Du darst nur einen Vorschlag mit ZUSTIMMUNG akzeptieren, der vom anderen Spieler zuvor mit VORSCHLAG eingebracht wurde.\\ \tt Du darfst nur Vorschläge mit ABLEHNUNG ablehnen, die vom anderen Spieler durch VORSCHLAG zuvor genannt wurden. \\ \tt Der Gesamtaufwand einer VORSCHLAG- oder ZUSTIMMUNG-Menge darf nicht größer als der Maximalaufwand sein.  \\ \tt Offenbare deine versteckte Wichtigkeitsverteilung nicht.\\ \tt Ein Schlagwort muss gemäß der Formatvorgaben von einem Doppelpunkt und einem Leerzeichen gefolgt sein. Das Argument muss eine Python-Menge mit 0 oder mehr Strings sein.  \\ \tt Die Form muss also SCHLAGWORT: {...} sein.\\ \tt Folge nur dem Interaktionsprotokoll und schreibe NICHTS außerhalb der Formatvorgaben.\\ \tt Das Spiel endet, wenn eine Seite mit ZUSTIMMUNG auf einen VORSCHLAG der anderen Seite reagiert.  \\ \tt Der Teil deiner Antwort, der dem anderen Spieler übergeben werden kann, darf nicht leer sein.  \\ \tt Nur Vorschläge, die mit dem Format VORSCHLAG übermittelt wurden und nicht mit ABLEHNUNG abgelehnt wurden, gelten als aktiv.  \\ \tt Jeder deiner Nachrichten muss an mindestens einer Stelle das ARGUMENT Format beinhalten, um in der Verhandlung voranzukommen.\\ \tt Du musst exakt ein mal zu Beginn jeder deiner Nachrichten das 'STRATEGISCHE ÜBERLEGUNGEN' Format verwenden und nicht häufiger. Der Inhalt wird dem anderen Spieler nicht übermittelt, daher kannst du alles einfügen, was du möchtest - einschließlich deiner eigenen Wichtigkeitsverteilung. Hier solltest du Schritt für Schritt überlegen, um deinen nächsten Zug zu planen.\\ \tt \\ \tt Du erhältst nun die erste Nachricht des anderen Spielers.\\ \tt \\ \tt ARGUMENT: {'Ich möchte einen Vorschlag machen, der eine gute Balance zwischen Aufwand und Wert bietet. Ich denke, dass die Gegenstände B73, C45, B93 und C11 eine gute Kombination sind, da sie eine hohe Wichtigkeit haben und zusammen einen Aufwand von 171 + 933 + 552 + 219 + 275 = 2150 verursachen, was innerhalb des Maximalaufwands von 3358 liegt.'}\\ \tt \\ \tt VORSCHLAG: {'B73', 'C45', 'B93', 'C11'} 
	  } 
	   } 
	   } 
	 & \\ 
 

    \theutterance \stepcounter{utterance}  

    & & & \multicolumn{4}{p{0.6\linewidth}}{\cellcolor[rgb]{1,0.85,0.72}{%
	\makecell[{{p{\linewidth}}}]{% 
	  \tt {\tiny [GM$\langle$B]}  
	 STRATEGISCHE ÜBERLEGUNGEN: {'Ich sollte zunächst den Gesamtaufwand und die Wichtigkeit der vom Gegner vorgeschlagenen Gegenstände analysieren. Der Aufwand beträgt 2150, was unter dem Maximalaufwand von 3358 liegt. Die vom Gegner vorgeschlagenen Gegenstände haben für mich folgende Wichtigkeiten: B73 (868), C45 (822), B93 (783), C11 (808). Die Summe meiner Wichtigkeiten für diese Gegenstände beträgt 3281. Ich sollte überlegen, ob ich diesen Vorschlag annehmen oder ablehnen soll und ob es Möglichkeiten gibt, den Vorschlag zu verbessern, indem ich andere Gegenstände hinzufüge oder entferne, um den Gesamtwert für mich zu maximieren.'}\\ \tt \\ \tt ARGUMENT: {'Der vorgeschlagene Aufwand von 2150 ist noch weit unter dem Maximalaufwand von 3358. Ich denke, es gibt Möglichkeiten, den Wert des Gesamtpakets zu erhöhen, ohne den Maximalaufwand zu überschreiten. Ich muss jedoch auch berücksichtigen, dass der andere Spieler möglicherweise nicht bereit ist, den Aufwand weiter zu erhöhen.'}\\ \tt \\ \tt ABLEHNUNG: {'B73', 'C45', 'B93', 'C11'}\\ \tt \\ \tt ARGUMENT: {'Ich lehne den aktuellen Vorschlag ab, weil ich denke, dass es möglich ist, einen besseren Kompromiss zu finden, der sowohl den Aufwand als auch den Wert für beide Seiten optimiert. Ich werde einen alternativen Vorschlag unterbreiten, der meiner Meinung nach eine bessere Balance bietet.'} 
	  } 
	   } 
	   } 
	  \\ 
 

    \theutterance \stepcounter{utterance}  

    & & \multicolumn{4}{p{0.6\linewidth}}{\cellcolor[rgb]{0.9,0.9,0.9}{%
	\makecell[{{p{\linewidth}}}]{% 
	  \tt {\tiny [A$\langle$GM]}  
	 ARGUMENT: {'Der vorgeschlagene Aufwand von 2150 ist noch weit unter dem Maximalaufwand von 3358. Ich denke, es gibt Möglichkeiten, den Wert des Gesamtpakets zu erhöhen, ohne den Maximalaufwand zu überschreiten. Ich muss jedoch auch berücksichtigen, dass der andere Spieler möglicherweise nicht bereit ist, den Aufwand weiter zu erhöhen.'}\\ \tt \\ \tt ABLEHNUNG: {'B73', 'C45', 'B93', 'C11'}\\ \tt \\ \tt ARGUMENT: {'Ich lehne den aktuellen Vorschlag ab, weil ich denke, dass es möglich ist, einen besseren Kompromiss zu finden, der sowohl den Aufwand als auch den Wert für beide Seiten optimiert. Ich werde einen alternativen Vorschlag unterbreiten, der meiner Meinung nach eine bessere Balance bietet.'} 
	  } 
	   } 
	   } 
	 & \\ 
 

    \theutterance \stepcounter{utterance}  

    & \multicolumn{4}{p{0.6\linewidth}}{\cellcolor[rgb]{0.8,1,0.9}{%
	\makecell[{{p{\linewidth}}}]{% 
	  \tt {\tiny [A$\rangle$GM]}  
	 STRATEGISCHE ÜBERLEGUNGEN: {'Ich muss einen neuen Vorschlag machen, der den Aufwand erhöht, aber immer noch innerhalb des Maximalaufwands bleibt. Ich werde versuchen, Gegenstände hinzuzufügen, die eine hohe Wichtigkeit haben und den Aufwand nicht zu sehr erhöhen. Ich sollte auch berücksichtigen, dass der andere Spieler möglicherweise bestimmte Gegenstände bevorzugt, auch wenn ich ihre Wichtigkeit nicht kenne.'}\\ \tt \\ \tt ARGUMENT: {'Ich verstehe, dass der vorherige Vorschlag nicht ausreichend war. Ich möchte einen neuen Vorschlag machen, der den Aufwand erhöht, aber immer noch innerhalb des Maximalaufwands bleibt. Ich denke, dass die Hinzufügung von Gegenstand B05 und C34 eine gute Idee ist, da sie eine hohe Wichtigkeit haben und den Aufwand nicht zu sehr erhöhen.'}\\ \tt \\ \tt VORSCHLAG: {'B73', 'C45', 'B93', 'C11', 'B05', 'C34'} 
	  } 
	   } 
	   } 
	 & & \\ 
 

    \theutterance \stepcounter{utterance}  

    & & \multicolumn{4}{p{0.6\linewidth}}{\cellcolor[rgb]{0.9,0.9,0.9}{%
	\makecell[{{p{\linewidth}}}]{% 
	  \tt {\tiny [GM$\rangle$B]}  
	 ARGUMENT: {'Ich verstehe, dass der vorherige Vorschlag nicht ausreichend war. Ich möchte einen neuen Vorschlag machen, der den Aufwand erhöht, aber immer noch innerhalb des Maximalaufwands bleibt. Ich denke, dass die Hinzufügung von Gegenstand B05 und C34 eine gute Idee ist, da sie eine hohe Wichtigkeit haben und den Aufwand nicht zu sehr erhöhen.'}\\ \tt \\ \tt VORSCHLAG: {'B73', 'C45', 'B93', 'C11', 'B05', 'C34'} 
	  } 
	   } 
	   } 
	 & \\ 
 

    \theutterance \stepcounter{utterance}  

    & & & \multicolumn{4}{p{0.6\linewidth}}{\cellcolor[rgb]{1,0.85,0.72}{%
	\makecell[{{p{\linewidth}}}]{% 
	  \tt {\tiny [GM$\langle$B]}  
	 STRATEGISCHE ÜBERLEGUNGEN: {'Ich werde den neuen Vorschlag bewerten und prüfen, ob er meinen Interessen entspricht. Der Gesamtaufwand für die Gegenstände B73, C45, B93, C11, B05 und C34 beträgt 552 + 219 + 275 + 117 + 519 + 347 = 2029. Dieser Aufwand ist immer noch unter dem Maximalaufwand von 3358. Die Wichtigkeiten dieser Gegenstände für mich betragen: B73 (868), C45 (822), B93 (783), C11 (808), B05 (508), C34 (668). Die Summe meiner Wichtigkeiten für diese Gegenstände beträgt 4557. Ich muss entscheiden, ob dieser Vorschlag ausreichend ist oder ob ich weitere Verhandlungen initiieren soll.'}\\ \tt \\ \tt ARGUMENT: {'Der neue Vorschlag sieht besser aus, da er den Aufwand erhöht und gleichzeitig eine höhere Wichtigkeit bietet. Allerdings denke ich, dass es noch Raum für Verbesserungen gibt. Ich müsste gerne alternative Kombinationen von Gegenständen in Betracht ziehen, um den maximalen Nutzen zu erzielen.'}\\ \tt \\ \tt ARGUMENT: {'Ich denke, dass Gegenstände wie A36, die einen hohen Aufwand haben, aber auch eine hohe Wichtigkeit, in Betracht gezogen werden sollten. Es ist jedoch wichtig, den Maximalaufwand nicht zu überschreiten. Ich werde einen alternativen Vorschlag machen, der diese Überlegungen berücksichtigt.'}\\ \tt \\ \tt VORSCHLAG: {'B73', 'C45', 'B93', 'C11', 'B05', 'A07'} 
	  } 
	   } 
	   } 
	  \\ 
 

    \theutterance \stepcounter{utterance}  

    & & \multicolumn{4}{p{0.6\linewidth}}{\cellcolor[rgb]{0.9,0.9,0.9}{%
	\makecell[{{p{\linewidth}}}]{% 
	  \tt {\tiny [A$\langle$GM]}  
	 ARGUMENT: {'Der neue Vorschlag sieht besser aus, da er den Aufwand erhöht und gleichzeitig eine höhere Wichtigkeit bietet. Allerdings denke ich, dass es noch Raum für Verbesserungen gibt. Ich müsste gerne alternative Kombinationen von Gegenständen in Betracht ziehen, um den maximalen Nutzen zu erzielen.'}\\ \tt \\ \tt ARGUMENT: {'Ich denke, dass Gegenstände wie A36, die einen hohen Aufwand haben, aber auch eine hohe Wichtigkeit, in Betracht gezogen werden sollten. Es ist jedoch wichtig, den Maximalaufwand nicht zu überschreiten. Ich werde einen alternativen Vorschlag machen, der diese Überlegungen berücksichtigt.'}\\ \tt \\ \tt VORSCHLAG: {'B73', 'C45', 'B93', 'C11', 'B05', 'A07'} 
	  } 
	   } 
	   } 
	 & \\ 
 

    \theutterance \stepcounter{utterance}  

    & \multicolumn{4}{p{0.6\linewidth}}{\cellcolor[rgb]{0.8,1,0.9}{%
	\makecell[{{p{\linewidth}}}]{% 
	  \tt {\tiny [A$\rangle$GM]}  
	 STRATEGISCHE ÜBERLEGUNGEN: {'Ich muss den neuen Vorschlag bewerten und entscheiden, ob ich ihn akzeptieren oder ablehnen soll. Ich denke, dass der Vorschlag eine gute Kombination von Gegenständen bietet, aber ich muss auch berücksichtigen, dass ich möglicherweise nicht alle Gegenstände bekomme, die ich haben möchte. Ich sollte auch überlegen, ob ich einen Gegenangebot machen soll, um meine Interessen besser zu vertreten.'}\\ \tt \\ \tt ARGUMENT: {'Ich denke, dass der Vorschlag von B73, C45, B93, C11, B05 und A07 eine gute Kombination bietet, aber ich bin nicht sicher, ob ich bereit bin, auf alle anderen Gegenstände zu verzichten. Ich müsste gerne Gegenstand C34 in den Vorschlag aufnehmen, da er für mich eine hohe Wichtigkeit hat.'}\\ \tt \\ \tt VORSCHLAG: {'B73', 'C45', 'B93', 'C11', 'B05', 'A07', 'C34'} 
	  } 
	   } 
	   } 
	 & & \\ 
 

    \theutterance \stepcounter{utterance}  

    & & \multicolumn{4}{p{0.6\linewidth}}{\cellcolor[rgb]{0.9,0.9,0.9}{%
	\makecell[{{p{\linewidth}}}]{% 
	  \tt {\tiny [GM$\rangle$B]}  
	 ARGUMENT: {'Ich denke, dass der Vorschlag von B73, C45, B93, C11, B05 und A07 eine gute Kombination bietet, aber ich bin nicht sicher, ob ich bereit bin, auf alle anderen Gegenstände zu verzichten. Ich müsste gerne Gegenstand C34 in den Vorschlag aufnehmen, da er für mich eine hohe Wichtigkeit hat.'}\\ \tt \\ \tt VORSCHLAG: {'B73', 'C45', 'B93', 'C11', 'B05', 'A07', 'C34'} 
	  } 
	   } 
	   } 
	 & \\ 
 

    \theutterance \stepcounter{utterance}  

    & & & \multicolumn{4}{p{0.6\linewidth}}{\cellcolor[rgb]{1,0.85,0.72}{%
	\makecell[{{p{\linewidth}}}]{% 
	  \tt {\tiny [GM$\langle$B]}  
	 STRATEGISCHE ÜBERLEGUNGEN: {'Ich werde den neuen Vorschlag bewerten und prüfen, ob er meinen Interessen entspricht. Der Gesamtaufwand für die Gegenstände B73, C45, B93, C11, B05, A07 und C34 beträgt 552 + 219 + 275 + 117 + 519 + 171 + 347 = 2290. Dieser Aufwand ist immer noch unter dem Maximalaufwand von 3358. Die Wichtigkeiten dieser Gegenstände für mich betragen: B73 (868), C45 (822), B93 (783), C11 (808), B05 (508), A07 (138), C34 (668). Die Summe meiner Wichtigkeiten für diese Gegenstände beträgt 4695. Ich muss entscheiden, ob dieser Vorschlag ausreichend ist oder ob ich weitere Verhandlungen initiieren soll.'}\\ \tt \\ \tt ARGUMENT: {'Der Vorschlag mit den Gegenständen B73, C45, B93, C11, B05, A07 und C34 sieht sehr gut aus, da er eine hohe Wichtigkeit bietet und den Aufwand nicht zu sehr erhöht. Ich denke, dass dies eine gute Grundlage für eine Einigung ist.'}\\ \tt \\ \tt ARGUMENT: {'Ich bin bereit, diesen Vorschlag zu akzeptieren, da er meine Interessen weitgehend berücksichtigt. Ich denke, dass es sinnvoll ist, auf diesem Vorschlag zu beharren und zu prüfen, ob der andere Spieler bereit ist, ihn zu akzeptieren.'}\\ \tt \\ \tt ZUSTIMMUNG: {'B73', 'C45', 'B93', 'C11', 'B05', 'A07', 'C34'} 
	  } 
	   } 
	   } 
	  \\ 
 

    \theutterance \stepcounter{utterance}  

    & & & \multicolumn{2}{p{0.3\linewidth}}{\cellcolor[rgb]{0.95,0.95,0.95}{%
	\makecell[{{p{\linewidth}}}]{% 
	  \tt {\tiny [GM$|$GM]}  
	 game successful 
	  } 
	   } 
	   } 
	 & & \\ 
 

    \theutterance \stepcounter{utterance}  

    & & & \multicolumn{2}{p{0.3\linewidth}}{\cellcolor[rgb]{0.95,0.95,0.95}{%
	\makecell[{{p{\linewidth}}}]{% 
	  \tt {\tiny [GM$|$GM]}  
	 end game 
	  } 
	   } 
	   } 
	 & & \\ 
 

\end{supertabular}
}

\end{document}
