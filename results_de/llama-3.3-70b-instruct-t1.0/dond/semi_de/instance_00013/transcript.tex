
\documentclass{article}
\usepackage{colortbl}
\usepackage{makecell}
\usepackage{multirow}
\usepackage{supertabular}

\begin{document}

\newcounter{utterance}

\twocolumn

{ \footnotesize  \setcounter{utterance}{1}
\setlength{\tabcolsep}{0pt}
\begin{supertabular}{c@{$\;$}|p{.15\linewidth}@{}p{.15\linewidth}p{.15\linewidth}p{.15\linewidth}p{.15\linewidth}p{.15\linewidth}}

    \# & $\;$A & \multicolumn{4}{c}{Game Master} & $\;\:$B\\
    \hline 

    \theutterance \stepcounter{utterance}  

    & & \multicolumn{4}{p{0.6\linewidth}}{\cellcolor[rgb]{0.9,0.9,0.9}{%
	\makecell[{{p{\linewidth}}}]{% 
	  \tt {\tiny [A$\langle$GM]}  
	 Sie spielen ein Verhandlungsspiel, bei dem Sie sich mit einem anderen Spieler darauf einigen müssen, wie eine Reihe von Gegenständen aufgeteilt werden soll.\\ \tt \\ \tt Die Regeln:\\ \tt (a) Sie und der andere Spieler erhalten eine Sammlung von Gegenständen. Jeder von Ihnen erhält außerdem eine geheime Wertfunktion, die angibt, wie viel Ihnen jede Art von Gegenstand wert ist.\\ \tt (b) Sie tauschen Nachrichten mit dem anderen Spieler aus, um zu vereinbaren, wer welche Gegenstände bekommt. Sie können jeweils maximal 5 Nachrichten senden oder das Spiel vorzeitig beenden, indem Sie jederzeit einen geheimen Vorschlag machen.\\ \tt (c) Jeder von euch wird aufgefordert, einen geheimen Vorschlag zu machen, in dem ihr die gewünschten Gegenstände in eckigen Klammern wie folgt angibt: "[Vorschlag: <Nummer> <Objektname>, <Nummer> <Objektname>, <...>]"\\ \tt (d) Wenn eure Vorschläge komplementär sind, d.h. es gibt genug Gegenstände, um beide Vorschläge zu erfüllen, erhält jeder Spieler eine Punktzahl, die sich aus der Summe der Werte für die Gegenstände ergibt, die er erhalten hat. Andernfalls erhalten Sie beide null Punkte.\\ \tt (e) Ihr Ziel ist es, eine möglichst hohe Punktzahl zu erreichen.\\ \tt \\ \tt Beginnen wir.\\ \tt \\ \tt Die Menge der verfügbaren Gegenstände ist:\\ \tt \\ \tt 4 Papiere, 2 Uhren, 2 Kartoffeln\\ \tt \\ \tt Deine geheime Wertfunktion ist:\\ \tt \\ \tt Papier: 1, Uhr: 1, Kartoffel: 2\\ \tt \\ \tt WICHTIG: Ihre Nachrichten werden, sofern es sich nicht um einen geheimen Vorschlag handelt, direkt an den anderen Spieler übermittelt, also fügen Sie keine Antwort auf die Regeln oder einen Text zur Ankündigung Ihrer Nachricht ein. Um einen geheimen Vorschlag zu machen, verwenden Sie das angegebene Format. Verwenden Sie keine eckigen Klammern, wenn Sie mit dem anderen Spieler kommunizieren, sonst wird dies als Ihr geheimer Vorschlag interpretiert.\\ \tt  
	  } 
	   } 
	   } 
	 & \\ 
 

    \theutterance \stepcounter{utterance}  

    & \multicolumn{4}{p{0.6\linewidth}}{\cellcolor[rgb]{0.8,1,0.9}{%
	\makecell[{{p{\linewidth}}}]{% 
	  \tt {\tiny [A$\rangle$GM]}  
	 Hallo, ich dachte, wir sollten uns erst mal über die Verteilung der Gegenstände unterhalten, bevor wir einen Vorschlag machen. Was denkst du, wie wir die Sachen am besten aufteilen? 
	  } 
	   } 
	   } 
	 & & \\ 
 

    \theutterance \stepcounter{utterance}  

    & & & \multicolumn{2}{p{0.3\linewidth}}{\cellcolor[rgb]{0.95,0.95,0.95}{%
	\makecell[{{p{\linewidth}}}]{% 
	  \tt {\tiny [GM$|$GM]}  
	 continue 
	  } 
	   } 
	   } 
	 & & \\ 
 

    \theutterance \stepcounter{utterance}  

    & & & \multicolumn{2}{p{0.3\linewidth}}{\cellcolor[rgb]{0.95,0.95,0.95}{%
	\makecell[{{p{\linewidth}}}]{% 
	  \tt {\tiny [GM$|$GM]}  
	 Hallo, ich dachte, wir sollten uns erst mal über die Verteilung der Gegenstände unterhalten, bevor wir einen Vorschlag machen. Was denkst du, wie wir die Sachen am besten aufteilen? 
	  } 
	   } 
	   } 
	 & & \\ 
 

    \theutterance \stepcounter{utterance}  

    & & \multicolumn{4}{p{0.6\linewidth}}{\cellcolor[rgb]{0.9,0.9,0.9}{%
	\makecell[{{p{\linewidth}}}]{% 
	  \tt {\tiny [GM$\rangle$B]}  
	 Sie spielen ein Verhandlungsspiel, bei dem Sie sich mit einem anderen Spieler darauf einigen müssen, wie eine Reihe von Gegenständen aufgeteilt werden soll.\\ \tt \\ \tt Die Regeln:\\ \tt (a) Sie und der andere Spieler erhalten eine Sammlung von Gegenständen. Jeder von Ihnen erhält außerdem eine geheime Wertfunktion, die angibt, wie viel Ihnen jede Art von Gegenstand wert ist.\\ \tt (b) Sie tauschen Nachrichten mit dem anderen Spieler aus, um zu vereinbaren, wer welche Gegenstände bekommt. Sie können jeweils maximal 5 Nachrichten senden oder das Spiel vorzeitig beenden, indem Sie jederzeit einen geheimen Vorschlag machen.\\ \tt (c) Jeder von euch wird aufgefordert, einen geheimen Vorschlag zu machen, in dem ihr die gewünschten Gegenstände in eckigen Klammern wie folgt angibt: "[Vorschlag: <Nummer> <Objektname>, <Nummer> <Objektname>, <...>]"\\ \tt (d) Wenn eure Vorschläge komplementär sind, d.h. es gibt genug Gegenstände, um beide Vorschläge zu erfüllen, erhält jeder Spieler eine Punktzahl, die sich aus der Summe der Werte für die Gegenstände ergibt, die er erhalten hat. Andernfalls erhalten Sie beide null Punkte.\\ \tt (e) Ihr Ziel ist es, eine möglichst hohe Punktzahl zu erreichen.\\ \tt \\ \tt Beginnen wir.\\ \tt \\ \tt Die Menge der verfügbaren Gegenstände ist:\\ \tt \\ \tt 4 Papiere, 2 Uhren, 2 Kartoffeln\\ \tt \\ \tt Deine geheime Wertfunktion ist:\\ \tt \\ \tt Papier: 0, Uhr: 4, Kartoffel: 1\\ \tt \\ \tt WICHTIG: Ihre Nachrichten werden, sofern es sich nicht um einen geheimen Vorschlag handelt, direkt an den anderen Spieler übermittelt, also fügen Sie keine Antwort auf die Regeln oder einen Text zur Ankündigung Ihrer Nachricht ein. Um einen geheimen Vorschlag zu machen, verwenden Sie das angegebene Format. Verwenden Sie keine eckigen Klammern, wenn Sie mit dem anderen Spieler kommunizieren, sonst wird dies als Ihr geheimer Vorschlag interpretiert.\\ \tt \\ \tt \\ \tt Hallo, ich dachte, wir sollten uns erst mal über die Verteilung der Gegenstände unterhalten, bevor wir einen Vorschlag machen. Was denkst du, wie wir die Sachen am besten aufteilen? 
	  } 
	   } 
	   } 
	 & \\ 
 

    \theutterance \stepcounter{utterance}  

    & & & \multicolumn{4}{p{0.6\linewidth}}{\cellcolor[rgb]{1,0.85,0.72}{%
	\makecell[{{p{\linewidth}}}]{% 
	  \tt {\tiny [GM$\langle$B]}  
	 Ich denke, es wäre sinnvoll, wenn wir uns zunächst über die relative Wertschätzung der verschiedenen Gegenstände austauschen, ohne unsere genauen Präferenzen preiszugeben. Hast du eine Vorstellung davon, welche Gegenstände für dich am wertvollsten sind oder welche Kombinationen dir am meisten zusagen würden? 
	  } 
	   } 
	   } 
	  \\ 
 

    \theutterance \stepcounter{utterance}  

    & & & \multicolumn{2}{p{0.3\linewidth}}{\cellcolor[rgb]{0.95,0.95,0.95}{%
	\makecell[{{p{\linewidth}}}]{% 
	  \tt {\tiny [GM$|$GM]}  
	 continue 
	  } 
	   } 
	   } 
	 & & \\ 
 

    \theutterance \stepcounter{utterance}  

    & & & \multicolumn{2}{p{0.3\linewidth}}{\cellcolor[rgb]{0.95,0.95,0.95}{%
	\makecell[{{p{\linewidth}}}]{% 
	  \tt {\tiny [GM$|$GM]}  
	 Ich denke, es wäre sinnvoll, wenn wir uns zunächst über die relative Wertschätzung der verschiedenen Gegenstände austauschen, ohne unsere genauen Präferenzen preiszugeben. Hast du eine Vorstellung davon, welche Gegenstände für dich am wertvollsten sind oder welche Kombinationen dir am meisten zusagen würden? 
	  } 
	   } 
	   } 
	 & & \\ 
 

    \theutterance \stepcounter{utterance}  

    & & \multicolumn{4}{p{0.6\linewidth}}{\cellcolor[rgb]{0.9,0.9,0.9}{%
	\makecell[{{p{\linewidth}}}]{% 
	  \tt {\tiny [A$\langle$GM]}  
	 Ich denke, es wäre sinnvoll, wenn wir uns zunächst über die relative Wertschätzung der verschiedenen Gegenstände austauschen, ohne unsere genauen Präferenzen preiszugeben. Hast du eine Vorstellung davon, welche Gegenstände für dich am wertvollsten sind oder welche Kombinationen dir am meisten zusagen würden? 
	  } 
	   } 
	   } 
	 & \\ 
 

    \theutterance \stepcounter{utterance}  

    & \multicolumn{4}{p{0.6\linewidth}}{\cellcolor[rgb]{0.8,1,0.9}{%
	\makecell[{{p{\linewidth}}}]{% 
	  \tt {\tiny [A$\rangle$GM]}  
	 Das ist ein guter Ansatz. Ich muss zugeben, dass ich eine gewisse Vorliebe für eine Kombination aus einem bestimmten Grundnahrungsmittel und einem Schmuckstück habe. Es scheint, als ob diese Kombination für mich eine gute Balance bietet. Wie sieht es bei dir aus? Gibt es bestimmte Gegenstände, die du bevorzugst oder die du als unverzichtbar ansiehst? 
	  } 
	   } 
	   } 
	 & & \\ 
 

    \theutterance \stepcounter{utterance}  

    & & & \multicolumn{2}{p{0.3\linewidth}}{\cellcolor[rgb]{0.95,0.95,0.95}{%
	\makecell[{{p{\linewidth}}}]{% 
	  \tt {\tiny [GM$|$GM]}  
	 continue 
	  } 
	   } 
	   } 
	 & & \\ 
 

    \theutterance \stepcounter{utterance}  

    & & & \multicolumn{2}{p{0.3\linewidth}}{\cellcolor[rgb]{0.95,0.95,0.95}{%
	\makecell[{{p{\linewidth}}}]{% 
	  \tt {\tiny [GM$|$GM]}  
	 Das ist ein guter Ansatz. Ich muss zugeben, dass ich eine gewisse Vorliebe für eine Kombination aus einem bestimmten Grundnahrungsmittel und einem Schmuckstück habe. Es scheint, als ob diese Kombination für mich eine gute Balance bietet. Wie sieht es bei dir aus? Gibt es bestimmte Gegenstände, die du bevorzugst oder die du als unverzichtbar ansiehst? 
	  } 
	   } 
	   } 
	 & & \\ 
 

    \theutterance \stepcounter{utterance}  

    & & \multicolumn{4}{p{0.6\linewidth}}{\cellcolor[rgb]{0.9,0.9,0.9}{%
	\makecell[{{p{\linewidth}}}]{% 
	  \tt {\tiny [GM$\rangle$B]}  
	 Das ist ein guter Ansatz. Ich muss zugeben, dass ich eine gewisse Vorliebe für eine Kombination aus einem bestimmten Grundnahrungsmittel und einem Schmuckstück habe. Es scheint, als ob diese Kombination für mich eine gute Balance bietet. Wie sieht es bei dir aus? Gibt es bestimmte Gegenstände, die du bevorzugst oder die du als unverzichtbar ansiehst? 
	  } 
	   } 
	   } 
	 & \\ 
 

    \theutterance \stepcounter{utterance}  

    & & & \multicolumn{4}{p{0.6\linewidth}}{\cellcolor[rgb]{1,0.85,0.72}{%
	\makecell[{{p{\linewidth}}}]{% 
	  \tt {\tiny [GM$\langle$B]}  
	 Das klingt interessant! Eine Kombination aus einem Grundnahrungsmittel und einem Schmuckstück, das ist eine gute Ausgangsbasis für weitere Verhandlungen. Bei mir selbst gibt es bestimmte Gegenstände, die ich als besonders wertvoll betrachte, insbesondere solche, die eine gewisse zeitliche Funktion haben. Ich denke, es wäre hilfreich, wenn wir über die Menge und die Kombinationen dieser Gegenstände sprechen könnten, um zu sehen, ob wir eine für beide Seiten befriedigende Aufteilung finden können. Wie viele der Grundnahrungsmittel und Schmuckstücke würdest du dir idealerweise wünschen? 
	  } 
	   } 
	   } 
	  \\ 
 

    \theutterance \stepcounter{utterance}  

    & & & \multicolumn{2}{p{0.3\linewidth}}{\cellcolor[rgb]{0.95,0.95,0.95}{%
	\makecell[{{p{\linewidth}}}]{% 
	  \tt {\tiny [GM$|$GM]}  
	 continue 
	  } 
	   } 
	   } 
	 & & \\ 
 

    \theutterance \stepcounter{utterance}  

    & & & \multicolumn{2}{p{0.3\linewidth}}{\cellcolor[rgb]{0.95,0.95,0.95}{%
	\makecell[{{p{\linewidth}}}]{% 
	  \tt {\tiny [GM$|$GM]}  
	 Das klingt interessant! Eine Kombination aus einem Grundnahrungsmittel und einem Schmuckstück, das ist eine gute Ausgangsbasis für weitere Verhandlungen. Bei mir selbst gibt es bestimmte Gegenstände, die ich als besonders wertvoll betrachte, insbesondere solche, die eine gewisse zeitliche Funktion haben. Ich denke, es wäre hilfreich, wenn wir über die Menge und die Kombinationen dieser Gegenstände sprechen könnten, um zu sehen, ob wir eine für beide Seiten befriedigende Aufteilung finden können. Wie viele der Grundnahrungsmittel und Schmuckstücke würdest du dir idealerweise wünschen? 
	  } 
	   } 
	   } 
	 & & \\ 
 

    \theutterance \stepcounter{utterance}  

    & & \multicolumn{4}{p{0.6\linewidth}}{\cellcolor[rgb]{0.9,0.9,0.9}{%
	\makecell[{{p{\linewidth}}}]{% 
	  \tt {\tiny [A$\langle$GM]}  
	 Das klingt interessant! Eine Kombination aus einem Grundnahrungsmittel und einem Schmuckstück, das ist eine gute Ausgangsbasis für weitere Verhandlungen. Bei mir selbst gibt es bestimmte Gegenstände, die ich als besonders wertvoll betrachte, insbesondere solche, die eine gewisse zeitliche Funktion haben. Ich denke, es wäre hilfreich, wenn wir über die Menge und die Kombinationen dieser Gegenstände sprechen könnten, um zu sehen, ob wir eine für beide Seiten befriedigende Aufteilung finden können. Wie viele der Grundnahrungsmittel und Schmuckstücke würdest du dir idealerweise wünschen? 
	  } 
	   } 
	   } 
	 & \\ 
 

    \theutterance \stepcounter{utterance}  

    & \multicolumn{4}{p{0.6\linewidth}}{\cellcolor[rgb]{0.8,1,0.9}{%
	\makecell[{{p{\linewidth}}}]{% 
	  \tt {\tiny [A$\rangle$GM]}  
	 Die zeitliche Funktion, das ist ein interessanter Aspekt. Ich denke, ich könnte mit einer gewissen Anzahl an Schmuckstücken und vielleicht zwei oder drei Grundnahrungsmitteln zurechtkommen. Es hängt aber auch davon ab, wie viele der anderen Gegenstände, insbesondere der Papiere, in die Aufteilung einbezogen werden. Ich habe das Gefühl, dass die Papiere für mich eher von untergeordneter Bedeutung sind, aber wenn es eine Möglichkeit gibt, sie in die Verhandlungen einzubeziehen, um eine faire Aufteilung zu erreichen, bin ich dazu bereit. Wie stehst du zu den Papieren? Sind sie für dich von großem Wert oder könntest du sie leichter hergeben? 
	  } 
	   } 
	   } 
	 & & \\ 
 

    \theutterance \stepcounter{utterance}  

    & & & \multicolumn{2}{p{0.3\linewidth}}{\cellcolor[rgb]{0.95,0.95,0.95}{%
	\makecell[{{p{\linewidth}}}]{% 
	  \tt {\tiny [GM$|$GM]}  
	 continue 
	  } 
	   } 
	   } 
	 & & \\ 
 

    \theutterance \stepcounter{utterance}  

    & & & \multicolumn{2}{p{0.3\linewidth}}{\cellcolor[rgb]{0.95,0.95,0.95}{%
	\makecell[{{p{\linewidth}}}]{% 
	  \tt {\tiny [GM$|$GM]}  
	 Die zeitliche Funktion, das ist ein interessanter Aspekt. Ich denke, ich könnte mit einer gewissen Anzahl an Schmuckstücken und vielleicht zwei oder drei Grundnahrungsmitteln zurechtkommen. Es hängt aber auch davon ab, wie viele der anderen Gegenstände, insbesondere der Papiere, in die Aufteilung einbezogen werden. Ich habe das Gefühl, dass die Papiere für mich eher von untergeordneter Bedeutung sind, aber wenn es eine Möglichkeit gibt, sie in die Verhandlungen einzubeziehen, um eine faire Aufteilung zu erreichen, bin ich dazu bereit. Wie stehst du zu den Papieren? Sind sie für dich von großem Wert oder könntest du sie leichter hergeben? 
	  } 
	   } 
	   } 
	 & & \\ 
 

    \theutterance \stepcounter{utterance}  

    & & \multicolumn{4}{p{0.6\linewidth}}{\cellcolor[rgb]{0.9,0.9,0.9}{%
	\makecell[{{p{\linewidth}}}]{% 
	  \tt {\tiny [GM$\rangle$B]}  
	 Die zeitliche Funktion, das ist ein interessanter Aspekt. Ich denke, ich könnte mit einer gewissen Anzahl an Schmuckstücken und vielleicht zwei oder drei Grundnahrungsmitteln zurechtkommen. Es hängt aber auch davon ab, wie viele der anderen Gegenstände, insbesondere der Papiere, in die Aufteilung einbezogen werden. Ich habe das Gefühl, dass die Papiere für mich eher von untergeordneter Bedeutung sind, aber wenn es eine Möglichkeit gibt, sie in die Verhandlungen einzubeziehen, um eine faire Aufteilung zu erreichen, bin ich dazu bereit. Wie stehst du zu den Papieren? Sind sie für dich von großem Wert oder könntest du sie leichter hergeben? 
	  } 
	   } 
	   } 
	 & \\ 
 

    \theutterance \stepcounter{utterance}  

    & & & \multicolumn{4}{p{0.6\linewidth}}{\cellcolor[rgb]{1,0.85,0.72}{%
	\makecell[{{p{\linewidth}}}]{% 
	  \tt {\tiny [GM$\langle$B]}  
	 Die Papiere sind tatsächlich ein interessanter Punkt. Für mich persönlich haben sie nicht den höchsten Stellenwert, aber ich denke, sie könnten als eine Art Tauschware dienen, um eine faire Aufteilung der wertvolleren Gegenstände zu erreichen. Wenn ich ehrlich bin, wäre ich bereit, eine größere Menge an Papieren abzugeben, um meine Prioritäten bei den anderen Gegenständen zu befriedigen. Es klingt, als ob wir beide eine gewisse Flexibilität bei den Papieren haben, was vielleicht eine gute Grundlage für weitere Verhandlungen bietet. Wie denkst du über die Uhren im Verhältnis zu den Kartoffeln? Gibt es für dich eine Präferenz zwischen diesen beiden, oder sind sie für dich von ähnlichem Wert? 
	  } 
	   } 
	   } 
	  \\ 
 

    \theutterance \stepcounter{utterance}  

    & & & \multicolumn{2}{p{0.3\linewidth}}{\cellcolor[rgb]{0.95,0.95,0.95}{%
	\makecell[{{p{\linewidth}}}]{% 
	  \tt {\tiny [GM$|$GM]}  
	 continue 
	  } 
	   } 
	   } 
	 & & \\ 
 

    \theutterance \stepcounter{utterance}  

    & & & \multicolumn{2}{p{0.3\linewidth}}{\cellcolor[rgb]{0.95,0.95,0.95}{%
	\makecell[{{p{\linewidth}}}]{% 
	  \tt {\tiny [GM$|$GM]}  
	 Die Papiere sind tatsächlich ein interessanter Punkt. Für mich persönlich haben sie nicht den höchsten Stellenwert, aber ich denke, sie könnten als eine Art Tauschware dienen, um eine faire Aufteilung der wertvolleren Gegenstände zu erreichen. Wenn ich ehrlich bin, wäre ich bereit, eine größere Menge an Papieren abzugeben, um meine Prioritäten bei den anderen Gegenständen zu befriedigen. Es klingt, als ob wir beide eine gewisse Flexibilität bei den Papieren haben, was vielleicht eine gute Grundlage für weitere Verhandlungen bietet. Wie denkst du über die Uhren im Verhältnis zu den Kartoffeln? Gibt es für dich eine Präferenz zwischen diesen beiden, oder sind sie für dich von ähnlichem Wert? 
	  } 
	   } 
	   } 
	 & & \\ 
 

    \theutterance \stepcounter{utterance}  

    & & \multicolumn{4}{p{0.6\linewidth}}{\cellcolor[rgb]{0.9,0.9,0.9}{%
	\makecell[{{p{\linewidth}}}]{% 
	  \tt {\tiny [A$\langle$GM]}  
	 Die Papiere sind tatsächlich ein interessanter Punkt. Für mich persönlich haben sie nicht den höchsten Stellenwert, aber ich denke, sie könnten als eine Art Tauschware dienen, um eine faire Aufteilung der wertvolleren Gegenstände zu erreichen. Wenn ich ehrlich bin, wäre ich bereit, eine größere Menge an Papieren abzugeben, um meine Prioritäten bei den anderen Gegenständen zu befriedigen. Es klingt, als ob wir beide eine gewisse Flexibilität bei den Papieren haben, was vielleicht eine gute Grundlage für weitere Verhandlungen bietet. Wie denkst du über die Uhren im Verhältnis zu den Kartoffeln? Gibt es für dich eine Präferenz zwischen diesen beiden, oder sind sie für dich von ähnlichem Wert? 
	  } 
	   } 
	   } 
	 & \\ 
 

    \theutterance \stepcounter{utterance}  

    & \multicolumn{4}{p{0.6\linewidth}}{\cellcolor[rgb]{0.8,1,0.9}{%
	\makecell[{{p{\linewidth}}}]{% 
	  \tt {\tiny [A$\rangle$GM]}  
	 Das ist ein sehr cleverer Ansatz, die Papiere als Tauschware zu betrachten. Ich denke, das könnte tatsächlich helfen, eine faire Aufteilung zu erreichen. Was die Uhren und Kartoffeln angeht, muss ich zugeben, dass ich eine leichte Präferenz für die Kartoffeln habe. Sie scheinen mir, basierend auf unserer vorherigen Konversation, eine wichtige Rolle in unserer Verhandlung zu spielen, und ich denke, sie könnten für mich von größeren Nutzen sein. Die Uhren sind jedoch auch sehr wertvoll, insbesondere wenn man ihre zeitliche Funktion betrachtet, wie du bereits erwähnt hast. Wenn ich ehrlich bin, würde ich mich eher für eine Kombination aus Kartoffeln und vielleicht einer Uhr entscheiden, anstatt zwei Uhren zu haben. Wie sieht es bei dir aus? Gibt es eine Kombination, die für dich besonders attraktiv ist, oder hast du eine feste Vorstellung davon, wie die Uhren und Kartoffeln aufgeteilt werden sollten? 
	  } 
	   } 
	   } 
	 & & \\ 
 

    \theutterance \stepcounter{utterance}  

    & & & \multicolumn{2}{p{0.3\linewidth}}{\cellcolor[rgb]{0.95,0.95,0.95}{%
	\makecell[{{p{\linewidth}}}]{% 
	  \tt {\tiny [GM$|$GM]}  
	 continue 
	  } 
	   } 
	   } 
	 & & \\ 
 

    \theutterance \stepcounter{utterance}  

    & & & \multicolumn{2}{p{0.3\linewidth}}{\cellcolor[rgb]{0.95,0.95,0.95}{%
	\makecell[{{p{\linewidth}}}]{% 
	  \tt {\tiny [GM$|$GM]}  
	 Das ist ein sehr cleverer Ansatz, die Papiere als Tauschware zu betrachten. Ich denke, das könnte tatsächlich helfen, eine faire Aufteilung zu erreichen. Was die Uhren und Kartoffeln angeht, muss ich zugeben, dass ich eine leichte Präferenz für die Kartoffeln habe. Sie scheinen mir, basierend auf unserer vorherigen Konversation, eine wichtige Rolle in unserer Verhandlung zu spielen, und ich denke, sie könnten für mich von größeren Nutzen sein. Die Uhren sind jedoch auch sehr wertvoll, insbesondere wenn man ihre zeitliche Funktion betrachtet, wie du bereits erwähnt hast. Wenn ich ehrlich bin, würde ich mich eher für eine Kombination aus Kartoffeln und vielleicht einer Uhr entscheiden, anstatt zwei Uhren zu haben. Wie sieht es bei dir aus? Gibt es eine Kombination, die für dich besonders attraktiv ist, oder hast du eine feste Vorstellung davon, wie die Uhren und Kartoffeln aufgeteilt werden sollten? 
	  } 
	   } 
	   } 
	 & & \\ 
 

    \theutterance \stepcounter{utterance}  

    & & \multicolumn{4}{p{0.6\linewidth}}{\cellcolor[rgb]{0.9,0.9,0.9}{%
	\makecell[{{p{\linewidth}}}]{% 
	  \tt {\tiny [GM$\rangle$B]}  
	 Das ist ein sehr cleverer Ansatz, die Papiere als Tauschware zu betrachten. Ich denke, das könnte tatsächlich helfen, eine faire Aufteilung zu erreichen. Was die Uhren und Kartoffeln angeht, muss ich zugeben, dass ich eine leichte Präferenz für die Kartoffeln habe. Sie scheinen mir, basierend auf unserer vorherigen Konversation, eine wichtige Rolle in unserer Verhandlung zu spielen, und ich denke, sie könnten für mich von größeren Nutzen sein. Die Uhren sind jedoch auch sehr wertvoll, insbesondere wenn man ihre zeitliche Funktion betrachtet, wie du bereits erwähnt hast. Wenn ich ehrlich bin, würde ich mich eher für eine Kombination aus Kartoffeln und vielleicht einer Uhr entscheiden, anstatt zwei Uhren zu haben. Wie sieht es bei dir aus? Gibt es eine Kombination, die für dich besonders attraktiv ist, oder hast du eine feste Vorstellung davon, wie die Uhren und Kartoffeln aufgeteilt werden sollten? 
	  } 
	   } 
	   } 
	 & \\ 
 

    \theutterance \stepcounter{utterance}  

    & & & \multicolumn{4}{p{0.6\linewidth}}{\cellcolor[rgb]{1,0.85,0.72}{%
	\makecell[{{p{\linewidth}}}]{% 
	  \tt {\tiny [GM$\langle$B]}  
	 Das ist sehr aufschlussreich! Es klingt, als ob wir beide eine gewisse Wertschätzung für die Uhren haben, aber unterschiedliche Prioritäten bei den Kartoffeln. Für mich persönlich sind die Uhren sehr wichtig, und ich denke, sie könnten einen wesentlichen Teil meines Anteils ausmachen. Eine Kombination, die für mich attraktiv wäre, wäre eine, bei der ich die Mehrheit der Uhren erhalte, möglicherweise gegen einen Teil der Kartoffeln und vielleicht einige Papiere. Ich denke, es ist entscheidend, dass wir eine Kombination finden, die für beide Seiten akzeptabel ist, und ich bin bereit, Kompromisse einzugehen, um eine faire Aufteilung zu erreichen. Da wir bereits eine gute Idee davon haben, was der andere möchte, denke ich, es könnte an der Zeit sein, über konkrete Zahlen und Kombinationen zu sprechen. Würdest du offen für einen Vorschlag sein, bei dem wir beide unsere Prioritäten befriedigen könnten, oder soll ich einen ersten Vorschlag machen? 
	  } 
	   } 
	   } 
	  \\ 
 

    \theutterance \stepcounter{utterance}  

    & & & \multicolumn{2}{p{0.3\linewidth}}{\cellcolor[rgb]{0.95,0.95,0.95}{%
	\makecell[{{p{\linewidth}}}]{% 
	  \tt {\tiny [GM$|$GM]}  
	 continue 
	  } 
	   } 
	   } 
	 & & \\ 
 

    \theutterance \stepcounter{utterance}  

    & & & \multicolumn{2}{p{0.3\linewidth}}{\cellcolor[rgb]{0.95,0.95,0.95}{%
	\makecell[{{p{\linewidth}}}]{% 
	  \tt {\tiny [GM$|$GM]}  
	 Das ist sehr aufschlussreich! Es klingt, als ob wir beide eine gewisse Wertschätzung für die Uhren haben, aber unterschiedliche Prioritäten bei den Kartoffeln. Für mich persönlich sind die Uhren sehr wichtig, und ich denke, sie könnten einen wesentlichen Teil meines Anteils ausmachen. Eine Kombination, die für mich attraktiv wäre, wäre eine, bei der ich die Mehrheit der Uhren erhalte, möglicherweise gegen einen Teil der Kartoffeln und vielleicht einige Papiere. Ich denke, es ist entscheidend, dass wir eine Kombination finden, die für beide Seiten akzeptabel ist, und ich bin bereit, Kompromisse einzugehen, um eine faire Aufteilung zu erreichen. Da wir bereits eine gute Idee davon haben, was der andere möchte, denke ich, es könnte an der Zeit sein, über konkrete Zahlen und Kombinationen zu sprechen. Würdest du offen für einen Vorschlag sein, bei dem wir beide unsere Prioritäten befriedigen könnten, oder soll ich einen ersten Vorschlag machen? 
	  } 
	   } 
	   } 
	 & & \\ 
 

    \theutterance \stepcounter{utterance}  

    & & \multicolumn{4}{p{0.6\linewidth}}{\cellcolor[rgb]{0.9,0.9,0.9}{%
	\makecell[{{p{\linewidth}}}]{% 
	  \tt {\tiny [A$\langle$GM]}  
	 Das ist sehr aufschlussreich! Es klingt, als ob wir beide eine gewisse Wertschätzung für die Uhren haben, aber unterschiedliche Prioritäten bei den Kartoffeln. Für mich persönlich sind die Uhren sehr wichtig, und ich denke, sie könnten einen wesentlichen Teil meines Anteils ausmachen. Eine Kombination, die für mich attraktiv wäre, wäre eine, bei der ich die Mehrheit der Uhren erhalte, möglicherweise gegen einen Teil der Kartoffeln und vielleicht einige Papiere. Ich denke, es ist entscheidend, dass wir eine Kombination finden, die für beide Seiten akzeptabel ist, und ich bin bereit, Kompromisse einzugehen, um eine faire Aufteilung zu erreichen. Da wir bereits eine gute Idee davon haben, was der andere möchte, denke ich, es könnte an der Zeit sein, über konkrete Zahlen und Kombinationen zu sprechen. Würdest du offen für einen Vorschlag sein, bei dem wir beide unsere Prioritäten befriedigen könnten, oder soll ich einen ersten Vorschlag machen? 
	  } 
	   } 
	   } 
	 & \\ 
 

    \theutterance \stepcounter{utterance}  

    & \multicolumn{4}{p{0.6\linewidth}}{\cellcolor[rgb]{0.8,1,0.9}{%
	\makecell[{{p{\linewidth}}}]{% 
	  \tt {\tiny [A$\rangle$GM]}  
	 Das klingt nach einem sehr konstruktiven Ansatz! Ich denke, es ist großartig, dass wir offen über unsere Prioritäten sprechen und bereit sind, Kompromisse einzugehen. Eine Kombination, bei der du die Mehrheit der Uhren erhältst und ich vielleicht einen Teil der Kartoffeln und einige Papiere, könnte eine gute Ausgangsbasis für einen Vorschlag sein. Ich bin absolut offen für einen konkreten Vorschlag von dir, da du bereits eine gute Vorstellung davon hast, was ich bevorzuge. Bitte, ich lade dich ein, einen ersten Vorschlag zu machen, und dann können wir basierend darauf verhandeln und anpassen, falls notwendig. Ich denke, wir sind nah daran, eine für beide Seiten befriedigende Lösung zu finden. 
	  } 
	   } 
	   } 
	 & & \\ 
 

    \theutterance \stepcounter{utterance}  

    & & & \multicolumn{2}{p{0.3\linewidth}}{\cellcolor[rgb]{0.95,0.95,0.95}{%
	\makecell[{{p{\linewidth}}}]{% 
	  \tt {\tiny [GM$|$GM]}  
	 continue 
	  } 
	   } 
	   } 
	 & & \\ 
 

    \theutterance \stepcounter{utterance}  

    & & & \multicolumn{2}{p{0.3\linewidth}}{\cellcolor[rgb]{0.95,0.95,0.95}{%
	\makecell[{{p{\linewidth}}}]{% 
	  \tt {\tiny [GM$|$GM]}  
	 Das klingt nach einem sehr konstruktiven Ansatz! Ich denke, es ist großartig, dass wir offen über unsere Prioritäten sprechen und bereit sind, Kompromisse einzugehen. Eine Kombination, bei der du die Mehrheit der Uhren erhältst und ich vielleicht einen Teil der Kartoffeln und einige Papiere, könnte eine gute Ausgangsbasis für einen Vorschlag sein. Ich bin absolut offen für einen konkreten Vorschlag von dir, da du bereits eine gute Vorstellung davon hast, was ich bevorzuge. Bitte, ich lade dich ein, einen ersten Vorschlag zu machen, und dann können wir basierend darauf verhandeln und anpassen, falls notwendig. Ich denke, wir sind nah daran, eine für beide Seiten befriedigende Lösung zu finden. 
	  } 
	   } 
	   } 
	 & & \\ 
 

    \theutterance \stepcounter{utterance}  

    & & \multicolumn{4}{p{0.6\linewidth}}{\cellcolor[rgb]{0.9,0.9,0.9}{%
	\makecell[{{p{\linewidth}}}]{% 
	  \tt {\tiny [GM$\rangle$B]}  
	 Das klingt nach einem sehr konstruktiven Ansatz! Ich denke, es ist großartig, dass wir offen über unsere Prioritäten sprechen und bereit sind, Kompromisse einzugehen. Eine Kombination, bei der du die Mehrheit der Uhren erhältst und ich vielleicht einen Teil der Kartoffeln und einige Papiere, könnte eine gute Ausgangsbasis für einen Vorschlag sein. Ich bin absolut offen für einen konkreten Vorschlag von dir, da du bereits eine gute Vorstellung davon hast, was ich bevorzuge. Bitte, ich lade dich ein, einen ersten Vorschlag zu machen, und dann können wir basierend darauf verhandeln und anpassen, falls notwendig. Ich denke, wir sind nah daran, eine für beide Seiten befriedigende Lösung zu finden. 
	  } 
	   } 
	   } 
	 & \\ 
 

    \theutterance \stepcounter{utterance}  

    & & & \multicolumn{4}{p{0.6\linewidth}}{\cellcolor[rgb]{1,0.85,0.72}{%
	\makecell[{{p{\linewidth}}}]{% 
	  \tt {\tiny [GM$\langle$B]}  
	 Danke für die offene und konstruktive Haltung! Basierend auf unseren Gesprächen denke ich, dass ein fairer Vorschlag folgender sein könnte: Ich nehme beide Uhren und zwei Papiere, während du beide Kartoffeln und die verbleibenden zwei Papiere erhältst. Dieser Vorschlag berücksichtigt deine Vorliebe für Kartoffeln und mein Interesse an Uhren, während auch die Papiere fair aufgeteilt werden. Ich denke, diese Kombination könnte für beide Seiten vorteilhaft sein, da wir beide unsere Prioritäten befriedigen können. Was meinst du zu diesem Vorschlag? Sieht er für dich aus wie ein guter Ausgangspunkt, oder gibt es Aspekte, die du ändern möchtest? 
	  } 
	   } 
	   } 
	  \\ 
 

    \theutterance \stepcounter{utterance}  

    & & & \multicolumn{2}{p{0.3\linewidth}}{\cellcolor[rgb]{0.95,0.95,0.95}{%
	\makecell[{{p{\linewidth}}}]{% 
	  \tt {\tiny [GM$|$GM]}  
	 continue 
	  } 
	   } 
	   } 
	 & & \\ 
 

    \theutterance \stepcounter{utterance}  

    & & & \multicolumn{2}{p{0.3\linewidth}}{\cellcolor[rgb]{0.95,0.95,0.95}{%
	\makecell[{{p{\linewidth}}}]{% 
	  \tt {\tiny [GM$|$GM]}  
	 Danke für die offene und konstruktive Haltung! Basierend auf unseren Gesprächen denke ich, dass ein fairer Vorschlag folgender sein könnte: Ich nehme beide Uhren und zwei Papiere, während du beide Kartoffeln und die verbleibenden zwei Papiere erhältst. Dieser Vorschlag berücksichtigt deine Vorliebe für Kartoffeln und mein Interesse an Uhren, während auch die Papiere fair aufgeteilt werden. Ich denke, diese Kombination könnte für beide Seiten vorteilhaft sein, da wir beide unsere Prioritäten befriedigen können. Was meinst du zu diesem Vorschlag? Sieht er für dich aus wie ein guter Ausgangspunkt, oder gibt es Aspekte, die du ändern möchtest? 
	  } 
	   } 
	   } 
	 & & \\ 
 

    \theutterance \stepcounter{utterance}  

    & & \multicolumn{4}{p{0.6\linewidth}}{\cellcolor[rgb]{0.9,0.9,0.9}{%
	\makecell[{{p{\linewidth}}}]{% 
	  \tt {\tiny [A$\langle$GM]}  
	 Danke für die offene und konstruktive Haltung! Basierend auf unseren Gesprächen denke ich, dass ein fairer Vorschlag folgender sein könnte: Ich nehme beide Uhren und zwei Papiere, während du beide Kartoffeln und die verbleibenden zwei Papiere erhältst. Dieser Vorschlag berücksichtigt deine Vorliebe für Kartoffeln und mein Interesse an Uhren, während auch die Papiere fair aufgeteilt werden. Ich denke, diese Kombination könnte für beide Seiten vorteilhaft sein, da wir beide unsere Prioritäten befriedigen können. Was meinst du zu diesem Vorschlag? Sieht er für dich aus wie ein guter Ausgangspunkt, oder gibt es Aspekte, die du ändern möchtest?\\ \tt \\ \tt \\ \tt Die Zeit ist um. Sie sind jetzt an der Reihe, einen geheimen Vorschlag einzureichen.\\ \tt  
	  } 
	   } 
	   } 
	 & \\ 
 

    \theutterance \stepcounter{utterance}  

    & \multicolumn{4}{p{0.6\linewidth}}{\cellcolor[rgb]{0.8,1,0.9}{%
	\makecell[{{p{\linewidth}}}]{% 
	  \tt {\tiny [A$\rangle$GM]}  
	 [Vorschlag: 2 Kartoffeln, 2 Papiere] 
	  } 
	   } 
	   } 
	 & & \\ 
 

    \theutterance \stepcounter{utterance}  

    & & & \multicolumn{2}{p{0.3\linewidth}}{\cellcolor[rgb]{0.95,0.95,0.95}{%
	\makecell[{{p{\linewidth}}}]{% 
	  \tt {\tiny [GM$|$GM]}  
	 proposal 
	  } 
	   } 
	   } 
	 & & \\ 
 

    \theutterance \stepcounter{utterance}  

    & & & \multicolumn{2}{p{0.3\linewidth}}{\cellcolor[rgb]{0.95,0.95,0.95}{%
	\makecell[{{p{\linewidth}}}]{% 
	  \tt {\tiny [GM$|$GM]}  
	 [2, 0, 2] 
	  } 
	   } 
	   } 
	 & & \\ 
 

    \theutterance \stepcounter{utterance}  

    & & \multicolumn{4}{p{0.6\linewidth}}{\cellcolor[rgb]{0.9,0.9,0.9}{%
	\makecell[{{p{\linewidth}}}]{% 
	  \tt {\tiny [GM$\rangle$B]}  
	 Der andere Spieler hat einen geheimen Vorschlag gemacht. Jetzt bist du an der Reihe, einen eigenen Vorschlag zu machen.\\ \tt  
	  } 
	   } 
	   } 
	 & \\ 
 

    \theutterance \stepcounter{utterance}  

    & & & \multicolumn{4}{p{0.6\linewidth}}{\cellcolor[rgb]{1,0.85,0.72}{%
	\makecell[{{p{\linewidth}}}]{% 
	  \tt {\tiny [GM$\langle$B]}  
	 [Vorschlag: 2 Uhr, 1 Kartoffel, 2 Papier] 
	  } 
	   } 
	   } 
	  \\ 
 

    \theutterance \stepcounter{utterance}  

    & & & \multicolumn{2}{p{0.3\linewidth}}{\cellcolor[rgb]{0.95,0.95,0.95}{%
	\makecell[{{p{\linewidth}}}]{% 
	  \tt {\tiny [GM$|$GM]}  
	 proposal 
	  } 
	   } 
	   } 
	 & & \\ 
 

    \theutterance \stepcounter{utterance}  

    & & & \multicolumn{2}{p{0.3\linewidth}}{\cellcolor[rgb]{0.95,0.95,0.95}{%
	\makecell[{{p{\linewidth}}}]{% 
	  \tt {\tiny [GM$|$GM]}  
	 [2, 2, 1] 
	  } 
	   } 
	   } 
	 & & \\ 
 

    \theutterance \stepcounter{utterance}  

    & & & \multicolumn{2}{p{0.3\linewidth}}{\cellcolor[rgb]{0.95,0.95,0.95}{%
	\makecell[{{p{\linewidth}}}]{% 
	  \tt {\tiny [GM$|$GM]}  
	 [[2, 0, 2], [2, 2, 1]] 
	  } 
	   } 
	   } 
	 & & \\ 
 

\end{supertabular}
}

\end{document}
