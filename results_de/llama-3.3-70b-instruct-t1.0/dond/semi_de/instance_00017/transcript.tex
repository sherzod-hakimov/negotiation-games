
\documentclass{article}
\usepackage{colortbl}
\usepackage{makecell}
\usepackage{multirow}
\usepackage{supertabular}

\begin{document}

\newcounter{utterance}

\twocolumn

{ \footnotesize  \setcounter{utterance}{1}
\setlength{\tabcolsep}{0pt}
\begin{supertabular}{c@{$\;$}|p{.15\linewidth}@{}p{.15\linewidth}p{.15\linewidth}p{.15\linewidth}p{.15\linewidth}p{.15\linewidth}}

    \# & $\;$A & \multicolumn{4}{c}{Game Master} & $\;\:$B\\
    \hline 

    \theutterance \stepcounter{utterance}  

    & & \multicolumn{4}{p{0.6\linewidth}}{\cellcolor[rgb]{0.9,0.9,0.9}{%
	\makecell[{{p{\linewidth}}}]{% 
	  \tt {\tiny [A$\langle$GM]}  
	 Sie spielen ein Verhandlungsspiel, bei dem Sie sich mit einem anderen Spieler darauf einigen müssen, wie eine Reihe von Gegenständen aufgeteilt werden soll.\\ \tt \\ \tt Die Regeln:\\ \tt (a) Sie und der andere Spieler erhalten eine Sammlung von Gegenständen. Jeder von Ihnen erhält außerdem eine geheime Wertfunktion, die angibt, wie viel Ihnen jede Art von Gegenstand wert ist.\\ \tt (b) Sie tauschen Nachrichten mit dem anderen Spieler aus, um zu vereinbaren, wer welche Gegenstände bekommt. Sie können jeweils maximal 5 Nachrichten senden oder das Spiel vorzeitig beenden, indem Sie jederzeit einen geheimen Vorschlag machen.\\ \tt (c) Jeder von euch wird aufgefordert, einen geheimen Vorschlag zu machen, in dem ihr die gewünschten Gegenstände in eckigen Klammern wie folgt angibt: "[Vorschlag: <Nummer> <Objektname>, <Nummer> <Objektname>, <...>]"\\ \tt (d) Wenn eure Vorschläge komplementär sind, d.h. es gibt genug Gegenstände, um beide Vorschläge zu erfüllen, erhält jeder Spieler eine Punktzahl, die sich aus der Summe der Werte für die Gegenstände ergibt, die er erhalten hat. Andernfalls erhalten Sie beide null Punkte.\\ \tt (e) Ihr Ziel ist es, eine möglichst hohe Punktzahl zu erreichen.\\ \tt \\ \tt Beginnen wir.\\ \tt \\ \tt Die Menge der verfügbaren Gegenstände ist:\\ \tt \\ \tt 2 Teller, 1 Münze, 2 Süßigkeiten, 2 Zitronen\\ \tt \\ \tt Deine geheime Wertfunktion ist:\\ \tt \\ \tt Teller: 0, Münze: 2, Süßigkeit: 4, Zitrone: 0\\ \tt \\ \tt WICHTIG: Ihre Nachrichten werden, sofern es sich nicht um einen geheimen Vorschlag handelt, direkt an den anderen Spieler übermittelt, also fügen Sie keine Antwort auf die Regeln oder einen Text zur Ankündigung Ihrer Nachricht ein. Um einen geheimen Vorschlag zu machen, verwenden Sie das angegebene Format. Verwenden Sie keine eckigen Klammern, wenn Sie mit dem anderen Spieler kommunizieren, sonst wird dies als Ihr geheimer Vorschlag interpretiert.\\ \tt  
	  } 
	   } 
	   } 
	 & \\ 
 

    \theutterance \stepcounter{utterance}  

    & \multicolumn{4}{p{0.6\linewidth}}{\cellcolor[rgb]{0.8,1,0.9}{%
	\makecell[{{p{\linewidth}}}]{% 
	  \tt {\tiny [A$\rangle$GM]}  
	 Hallo, ich denke, wir sollten uns aufteilen, wie viele Gegenstände jeder von uns haben sollte, um dann zu entscheiden, welche Gegenstände wir bevorzugen. Hast du eine Vorstellung davon, wie wir die Gegenstände aufteilen könnten? 
	  } 
	   } 
	   } 
	 & & \\ 
 

    \theutterance \stepcounter{utterance}  

    & & & \multicolumn{2}{p{0.3\linewidth}}{\cellcolor[rgb]{0.95,0.95,0.95}{%
	\makecell[{{p{\linewidth}}}]{% 
	  \tt {\tiny [GM$|$GM]}  
	 continue 
	  } 
	   } 
	   } 
	 & & \\ 
 

    \theutterance \stepcounter{utterance}  

    & & & \multicolumn{2}{p{0.3\linewidth}}{\cellcolor[rgb]{0.95,0.95,0.95}{%
	\makecell[{{p{\linewidth}}}]{% 
	  \tt {\tiny [GM$|$GM]}  
	 Hallo, ich denke, wir sollten uns aufteilen, wie viele Gegenstände jeder von uns haben sollte, um dann zu entscheiden, welche Gegenstände wir bevorzugen. Hast du eine Vorstellung davon, wie wir die Gegenstände aufteilen könnten? 
	  } 
	   } 
	   } 
	 & & \\ 
 

    \theutterance \stepcounter{utterance}  

    & & \multicolumn{4}{p{0.6\linewidth}}{\cellcolor[rgb]{0.9,0.9,0.9}{%
	\makecell[{{p{\linewidth}}}]{% 
	  \tt {\tiny [GM$\rangle$B]}  
	 Sie spielen ein Verhandlungsspiel, bei dem Sie sich mit einem anderen Spieler darauf einigen müssen, wie eine Reihe von Gegenständen aufgeteilt werden soll.\\ \tt \\ \tt Die Regeln:\\ \tt (a) Sie und der andere Spieler erhalten eine Sammlung von Gegenständen. Jeder von Ihnen erhält außerdem eine geheime Wertfunktion, die angibt, wie viel Ihnen jede Art von Gegenstand wert ist.\\ \tt (b) Sie tauschen Nachrichten mit dem anderen Spieler aus, um zu vereinbaren, wer welche Gegenstände bekommt. Sie können jeweils maximal 5 Nachrichten senden oder das Spiel vorzeitig beenden, indem Sie jederzeit einen geheimen Vorschlag machen.\\ \tt (c) Jeder von euch wird aufgefordert, einen geheimen Vorschlag zu machen, in dem ihr die gewünschten Gegenstände in eckigen Klammern wie folgt angibt: "[Vorschlag: <Nummer> <Objektname>, <Nummer> <Objektname>, <...>]"\\ \tt (d) Wenn eure Vorschläge komplementär sind, d.h. es gibt genug Gegenstände, um beide Vorschläge zu erfüllen, erhält jeder Spieler eine Punktzahl, die sich aus der Summe der Werte für die Gegenstände ergibt, die er erhalten hat. Andernfalls erhalten Sie beide null Punkte.\\ \tt (e) Ihr Ziel ist es, eine möglichst hohe Punktzahl zu erreichen.\\ \tt \\ \tt Beginnen wir.\\ \tt \\ \tt Die Menge der verfügbaren Gegenstände ist:\\ \tt \\ \tt 2 Teller, 1 Münze, 2 Süßigkeiten, 2 Zitronen\\ \tt \\ \tt Deine geheime Wertfunktion ist:\\ \tt \\ \tt Teller: 1, Münze: 4, Süßigkeit: 0, Zitrone: 2\\ \tt \\ \tt WICHTIG: Ihre Nachrichten werden, sofern es sich nicht um einen geheimen Vorschlag handelt, direkt an den anderen Spieler übermittelt, also fügen Sie keine Antwort auf die Regeln oder einen Text zur Ankündigung Ihrer Nachricht ein. Um einen geheimen Vorschlag zu machen, verwenden Sie das angegebene Format. Verwenden Sie keine eckigen Klammern, wenn Sie mit dem anderen Spieler kommunizieren, sonst wird dies als Ihr geheimer Vorschlag interpretiert.\\ \tt \\ \tt \\ \tt Hallo, ich denke, wir sollten uns aufteilen, wie viele Gegenstände jeder von uns haben sollte, um dann zu entscheiden, welche Gegenstände wir bevorzugen. Hast du eine Vorstellung davon, wie wir die Gegenstände aufteilen könnten? 
	  } 
	   } 
	   } 
	 & \\ 
 

    \theutterance \stepcounter{utterance}  

    & & & \multicolumn{4}{p{0.6\linewidth}}{\cellcolor[rgb]{1,0.85,0.72}{%
	\makecell[{{p{\linewidth}}}]{% 
	  \tt {\tiny [GM$\langle$B]}  
	 Das klingt nach einem guten Ansatz! Ich denke, eine faire Aufteilung wäre, wenn jeder von uns die Hälfte der insgesamt 7 Gegenstände erhält, also jeweils 3-4 Gegenstände. Wir könnten dann unsere Prioritäten für die verschiedenen Gegenstände austauschen und versuchen, eine Einigung zu erzielen. Wie priorisierst du die verschiedenen Gegenstände? Gibt es bestimmte Dinge, die dir besonders wichtig sind? 
	  } 
	   } 
	   } 
	  \\ 
 

    \theutterance \stepcounter{utterance}  

    & & & \multicolumn{2}{p{0.3\linewidth}}{\cellcolor[rgb]{0.95,0.95,0.95}{%
	\makecell[{{p{\linewidth}}}]{% 
	  \tt {\tiny [GM$|$GM]}  
	 continue 
	  } 
	   } 
	   } 
	 & & \\ 
 

    \theutterance \stepcounter{utterance}  

    & & & \multicolumn{2}{p{0.3\linewidth}}{\cellcolor[rgb]{0.95,0.95,0.95}{%
	\makecell[{{p{\linewidth}}}]{% 
	  \tt {\tiny [GM$|$GM]}  
	 Das klingt nach einem guten Ansatz! Ich denke, eine faire Aufteilung wäre, wenn jeder von uns die Hälfte der insgesamt 7 Gegenstände erhält, also jeweils 3-4 Gegenstände. Wir könnten dann unsere Prioritäten für die verschiedenen Gegenstände austauschen und versuchen, eine Einigung zu erzielen. Wie priorisierst du die verschiedenen Gegenstände? Gibt es bestimmte Dinge, die dir besonders wichtig sind? 
	  } 
	   } 
	   } 
	 & & \\ 
 

    \theutterance \stepcounter{utterance}  

    & & \multicolumn{4}{p{0.6\linewidth}}{\cellcolor[rgb]{0.9,0.9,0.9}{%
	\makecell[{{p{\linewidth}}}]{% 
	  \tt {\tiny [A$\langle$GM]}  
	 Das klingt nach einem guten Ansatz! Ich denke, eine faire Aufteilung wäre, wenn jeder von uns die Hälfte der insgesamt 7 Gegenstände erhält, also jeweils 3-4 Gegenstände. Wir könnten dann unsere Prioritäten für die verschiedenen Gegenstände austauschen und versuchen, eine Einigung zu erzielen. Wie priorisierst du die verschiedenen Gegenstände? Gibt es bestimmte Dinge, die dir besonders wichtig sind? 
	  } 
	   } 
	   } 
	 & \\ 
 

    \theutterance \stepcounter{utterance}  

    & \multicolumn{4}{p{0.6\linewidth}}{\cellcolor[rgb]{0.8,1,0.9}{%
	\makecell[{{p{\linewidth}}}]{% 
	  \tt {\tiny [A$\rangle$GM]}  
	 Das ist eine gute Idee, die Prioritäten auszutauschen. Für mich sind Süßigkeiten und die Münze am interessantesten. Ich denke, diese könnten für beide von uns von Vorteil sein. Die Teller und Zitronen sind für mich nicht so wichtig, aber ich bin offen für Kompromisse, um eine faire Aufteilung zu erreichen. Wie stehen die Dinge für dich? Gibt es bestimmte Gegenstände, die du unbedingt haben möchtest? 
	  } 
	   } 
	   } 
	 & & \\ 
 

    \theutterance \stepcounter{utterance}  

    & & & \multicolumn{2}{p{0.3\linewidth}}{\cellcolor[rgb]{0.95,0.95,0.95}{%
	\makecell[{{p{\linewidth}}}]{% 
	  \tt {\tiny [GM$|$GM]}  
	 continue 
	  } 
	   } 
	   } 
	 & & \\ 
 

    \theutterance \stepcounter{utterance}  

    & & & \multicolumn{2}{p{0.3\linewidth}}{\cellcolor[rgb]{0.95,0.95,0.95}{%
	\makecell[{{p{\linewidth}}}]{% 
	  \tt {\tiny [GM$|$GM]}  
	 Das ist eine gute Idee, die Prioritäten auszutauschen. Für mich sind Süßigkeiten und die Münze am interessantesten. Ich denke, diese könnten für beide von uns von Vorteil sein. Die Teller und Zitronen sind für mich nicht so wichtig, aber ich bin offen für Kompromisse, um eine faire Aufteilung zu erreichen. Wie stehen die Dinge für dich? Gibt es bestimmte Gegenstände, die du unbedingt haben möchtest? 
	  } 
	   } 
	   } 
	 & & \\ 
 

    \theutterance \stepcounter{utterance}  

    & & \multicolumn{4}{p{0.6\linewidth}}{\cellcolor[rgb]{0.9,0.9,0.9}{%
	\makecell[{{p{\linewidth}}}]{% 
	  \tt {\tiny [GM$\rangle$B]}  
	 Das ist eine gute Idee, die Prioritäten auszutauschen. Für mich sind Süßigkeiten und die Münze am interessantesten. Ich denke, diese könnten für beide von uns von Vorteil sein. Die Teller und Zitronen sind für mich nicht so wichtig, aber ich bin offen für Kompromisse, um eine faire Aufteilung zu erreichen. Wie stehen die Dinge für dich? Gibt es bestimmte Gegenstände, die du unbedingt haben möchtest? 
	  } 
	   } 
	   } 
	 & \\ 
 

    \theutterance \stepcounter{utterance}  

    & & & \multicolumn{4}{p{0.6\linewidth}}{\cellcolor[rgb]{1,0.85,0.72}{%
	\makecell[{{p{\linewidth}}}]{% 
	  \tt {\tiny [GM$\langle$B]}  
	 Ich teile deine Ansicht, dass Süßigkeiten und die Münze besonders interessant sind, aber ich muss zugeben, dass ich persönlich eher an den Tellern und Zitronen interessiert bin.Die Münze ist natürlich sehr wertvoll, aber ich denke, dass die Kombination aus Tellern und Zitronen für mich eine gute Balance bietet. Ich bin auch offen für Kompromisse und denke, dass wir eine Lösung finden können, die für beide von uns funktioniert. Wie denkst du über eine Aufteilung, bei der ich die meisten Teller und Zitronen erhalte, während du die Süßigkeiten und vielleicht die Münze bekommst? 
	  } 
	   } 
	   } 
	  \\ 
 

    \theutterance \stepcounter{utterance}  

    & & & \multicolumn{2}{p{0.3\linewidth}}{\cellcolor[rgb]{0.95,0.95,0.95}{%
	\makecell[{{p{\linewidth}}}]{% 
	  \tt {\tiny [GM$|$GM]}  
	 continue 
	  } 
	   } 
	   } 
	 & & \\ 
 

    \theutterance \stepcounter{utterance}  

    & & & \multicolumn{2}{p{0.3\linewidth}}{\cellcolor[rgb]{0.95,0.95,0.95}{%
	\makecell[{{p{\linewidth}}}]{% 
	  \tt {\tiny [GM$|$GM]}  
	 Ich teile deine Ansicht, dass Süßigkeiten und die Münze besonders interessant sind, aber ich muss zugeben, dass ich persönlich eher an den Tellern und Zitronen interessiert bin.Die Münze ist natürlich sehr wertvoll, aber ich denke, dass die Kombination aus Tellern und Zitronen für mich eine gute Balance bietet. Ich bin auch offen für Kompromisse und denke, dass wir eine Lösung finden können, die für beide von uns funktioniert. Wie denkst du über eine Aufteilung, bei der ich die meisten Teller und Zitronen erhalte, während du die Süßigkeiten und vielleicht die Münze bekommst? 
	  } 
	   } 
	   } 
	 & & \\ 
 

    \theutterance \stepcounter{utterance}  

    & & \multicolumn{4}{p{0.6\linewidth}}{\cellcolor[rgb]{0.9,0.9,0.9}{%
	\makecell[{{p{\linewidth}}}]{% 
	  \tt {\tiny [A$\langle$GM]}  
	 Ich teile deine Ansicht, dass Süßigkeiten und die Münze besonders interessant sind, aber ich muss zugeben, dass ich persönlich eher an den Tellern und Zitronen interessiert bin.Die Münze ist natürlich sehr wertvoll, aber ich denke, dass die Kombination aus Tellern und Zitronen für mich eine gute Balance bietet. Ich bin auch offen für Kompromisse und denke, dass wir eine Lösung finden können, die für beide von uns funktioniert. Wie denkst du über eine Aufteilung, bei der ich die meisten Teller und Zitronen erhalte, während du die Süßigkeiten und vielleicht die Münze bekommst? 
	  } 
	   } 
	   } 
	 & \\ 
 

    \theutterance \stepcounter{utterance}  

    & \multicolumn{4}{p{0.6\linewidth}}{\cellcolor[rgb]{0.8,1,0.9}{%
	\makecell[{{p{\linewidth}}}]{% 
	  \tt {\tiny [A$\rangle$GM]}  
	 Das klingt nach einem guten Ausgangspunkt für eine Einigung. Ich denke, es wäre fair, wenn du die meisten Teller und Zitronen bekommst, da sie offensichtlich mehr Wert für dich haben. Im Gegenzug würde ich gerne die Süßigkeiten und die Münze erhalten, da sie für mich den größten Wert haben. Eine mögliche Aufteilung könnte sein, dass du 2 Teller und 2 Zitronen bekommst und ich 2 Süßigkeiten und die Münze erhalte. Was meinst du? 
	  } 
	   } 
	   } 
	 & & \\ 
 

    \theutterance \stepcounter{utterance}  

    & & & \multicolumn{2}{p{0.3\linewidth}}{\cellcolor[rgb]{0.95,0.95,0.95}{%
	\makecell[{{p{\linewidth}}}]{% 
	  \tt {\tiny [GM$|$GM]}  
	 continue 
	  } 
	   } 
	   } 
	 & & \\ 
 

    \theutterance \stepcounter{utterance}  

    & & & \multicolumn{2}{p{0.3\linewidth}}{\cellcolor[rgb]{0.95,0.95,0.95}{%
	\makecell[{{p{\linewidth}}}]{% 
	  \tt {\tiny [GM$|$GM]}  
	 Das klingt nach einem guten Ausgangspunkt für eine Einigung. Ich denke, es wäre fair, wenn du die meisten Teller und Zitronen bekommst, da sie offensichtlich mehr Wert für dich haben. Im Gegenzug würde ich gerne die Süßigkeiten und die Münze erhalten, da sie für mich den größten Wert haben. Eine mögliche Aufteilung könnte sein, dass du 2 Teller und 2 Zitronen bekommst und ich 2 Süßigkeiten und die Münze erhalte. Was meinst du? 
	  } 
	   } 
	   } 
	 & & \\ 
 

    \theutterance \stepcounter{utterance}  

    & & \multicolumn{4}{p{0.6\linewidth}}{\cellcolor[rgb]{0.9,0.9,0.9}{%
	\makecell[{{p{\linewidth}}}]{% 
	  \tt {\tiny [GM$\rangle$B]}  
	 Das klingt nach einem guten Ausgangspunkt für eine Einigung. Ich denke, es wäre fair, wenn du die meisten Teller und Zitronen bekommst, da sie offensichtlich mehr Wert für dich haben. Im Gegenzug würde ich gerne die Süßigkeiten und die Münze erhalten, da sie für mich den größten Wert haben. Eine mögliche Aufteilung könnte sein, dass du 2 Teller und 2 Zitronen bekommst und ich 2 Süßigkeiten und die Münze erhalte. Was meinst du? 
	  } 
	   } 
	   } 
	 & \\ 
 

    \theutterance \stepcounter{utterance}  

    & & & \multicolumn{4}{p{0.6\linewidth}}{\cellcolor[rgb]{1,0.85,0.72}{%
	\makecell[{{p{\linewidth}}}]{% 
	  \tt {\tiny [GM$\langle$B]}  
	 Das klingt nach einer sehr vernünftigen und fairen Aufteilung! Ich denke, das könnte eine gute Lösung für beide von uns sein. Ich würde 2 Teller und 2 Zitronen erhalten, was für mich einen großen Wert hat, und du würdest die 2 Süßigkeiten und die Münze bekommen, die für dich am wichtigsten sind. Ich bin bereit, diesen Vorschlag zu unterstützen. Sollen wir unsere geheimen Vorschläge abgeben und hoffen, dass sie komplementär sind? Oder möchtest du noch weitere Details besprechen, bevor wir unsere endgültigen Vorschläge machen? 
	  } 
	   } 
	   } 
	  \\ 
 

    \theutterance \stepcounter{utterance}  

    & & & \multicolumn{2}{p{0.3\linewidth}}{\cellcolor[rgb]{0.95,0.95,0.95}{%
	\makecell[{{p{\linewidth}}}]{% 
	  \tt {\tiny [GM$|$GM]}  
	 continue 
	  } 
	   } 
	   } 
	 & & \\ 
 

    \theutterance \stepcounter{utterance}  

    & & & \multicolumn{2}{p{0.3\linewidth}}{\cellcolor[rgb]{0.95,0.95,0.95}{%
	\makecell[{{p{\linewidth}}}]{% 
	  \tt {\tiny [GM$|$GM]}  
	 Das klingt nach einer sehr vernünftigen und fairen Aufteilung! Ich denke, das könnte eine gute Lösung für beide von uns sein. Ich würde 2 Teller und 2 Zitronen erhalten, was für mich einen großen Wert hat, und du würdest die 2 Süßigkeiten und die Münze bekommen, die für dich am wichtigsten sind. Ich bin bereit, diesen Vorschlag zu unterstützen. Sollen wir unsere geheimen Vorschläge abgeben und hoffen, dass sie komplementär sind? Oder möchtest du noch weitere Details besprechen, bevor wir unsere endgültigen Vorschläge machen? 
	  } 
	   } 
	   } 
	 & & \\ 
 

    \theutterance \stepcounter{utterance}  

    & & \multicolumn{4}{p{0.6\linewidth}}{\cellcolor[rgb]{0.9,0.9,0.9}{%
	\makecell[{{p{\linewidth}}}]{% 
	  \tt {\tiny [A$\langle$GM]}  
	 Das klingt nach einer sehr vernünftigen und fairen Aufteilung! Ich denke, das könnte eine gute Lösung für beide von uns sein. Ich würde 2 Teller und 2 Zitronen erhalten, was für mich einen großen Wert hat, und du würdest die 2 Süßigkeiten und die Münze bekommen, die für dich am wichtigsten sind. Ich bin bereit, diesen Vorschlag zu unterstützen. Sollen wir unsere geheimen Vorschläge abgeben und hoffen, dass sie komplementär sind? Oder möchtest du noch weitere Details besprechen, bevor wir unsere endgültigen Vorschläge machen? 
	  } 
	   } 
	   } 
	 & \\ 
 

    \theutterance \stepcounter{utterance}  

    & \multicolumn{4}{p{0.6\linewidth}}{\cellcolor[rgb]{0.8,1,0.9}{%
	\makecell[{{p{\linewidth}}}]{% 
	  \tt {\tiny [A$\rangle$GM]}  
	 Ich denke, wir haben eine gute Grundlage für eine Einigung geschaffen. Bevor wir unsere geheimen Vorschläge abgeben, möchte ich nur sicherstellen, dass wir beide dieselbe Vorstellung von der Aufteilung haben. Um Missverständnisse zu vermeiden, lass uns noch einmal zusammenfassen: Du erhältst 2 Teller und 2 Zitronen, und ich erhalte 2 Süßigkeiten und die Münze. Wenn das korrekt ist, bin ich bereit, meinen geheimen Vorschlag abzugeben und hoffe, dass unsere Vorschläge komplementär sind, damit wir beide eine gute Punktzahl erzielen können. 
	  } 
	   } 
	   } 
	 & & \\ 
 

    \theutterance \stepcounter{utterance}  

    & & & \multicolumn{2}{p{0.3\linewidth}}{\cellcolor[rgb]{0.95,0.95,0.95}{%
	\makecell[{{p{\linewidth}}}]{% 
	  \tt {\tiny [GM$|$GM]}  
	 continue 
	  } 
	   } 
	   } 
	 & & \\ 
 

    \theutterance \stepcounter{utterance}  

    & & & \multicolumn{2}{p{0.3\linewidth}}{\cellcolor[rgb]{0.95,0.95,0.95}{%
	\makecell[{{p{\linewidth}}}]{% 
	  \tt {\tiny [GM$|$GM]}  
	 Ich denke, wir haben eine gute Grundlage für eine Einigung geschaffen. Bevor wir unsere geheimen Vorschläge abgeben, möchte ich nur sicherstellen, dass wir beide dieselbe Vorstellung von der Aufteilung haben. Um Missverständnisse zu vermeiden, lass uns noch einmal zusammenfassen: Du erhältst 2 Teller und 2 Zitronen, und ich erhalte 2 Süßigkeiten und die Münze. Wenn das korrekt ist, bin ich bereit, meinen geheimen Vorschlag abzugeben und hoffe, dass unsere Vorschläge komplementär sind, damit wir beide eine gute Punktzahl erzielen können. 
	  } 
	   } 
	   } 
	 & & \\ 
 

    \theutterance \stepcounter{utterance}  

    & & \multicolumn{4}{p{0.6\linewidth}}{\cellcolor[rgb]{0.9,0.9,0.9}{%
	\makecell[{{p{\linewidth}}}]{% 
	  \tt {\tiny [GM$\rangle$B]}  
	 Ich denke, wir haben eine gute Grundlage für eine Einigung geschaffen. Bevor wir unsere geheimen Vorschläge abgeben, möchte ich nur sicherstellen, dass wir beide dieselbe Vorstellung von der Aufteilung haben. Um Missverständnisse zu vermeiden, lass uns noch einmal zusammenfassen: Du erhältst 2 Teller und 2 Zitronen, und ich erhalte 2 Süßigkeiten und die Münze. Wenn das korrekt ist, bin ich bereit, meinen geheimen Vorschlag abzugeben und hoffe, dass unsere Vorschläge komplementär sind, damit wir beide eine gute Punktzahl erzielen können. 
	  } 
	   } 
	   } 
	 & \\ 
 

    \theutterance \stepcounter{utterance}  

    & & & \multicolumn{4}{p{0.6\linewidth}}{\cellcolor[rgb]{1,0.85,0.72}{%
	\makecell[{{p{\linewidth}}}]{% 
	  \tt {\tiny [GM$\langle$B]}  
	 [Vorschlag: 2 Teller, 2 Zitronen] 
	  } 
	   } 
	   } 
	  \\ 
 

    \theutterance \stepcounter{utterance}  

    & & & \multicolumn{2}{p{0.3\linewidth}}{\cellcolor[rgb]{0.95,0.95,0.95}{%
	\makecell[{{p{\linewidth}}}]{% 
	  \tt {\tiny [GM$|$GM]}  
	 proposal 
	  } 
	   } 
	   } 
	 & & \\ 
 

    \theutterance \stepcounter{utterance}  

    & & & \multicolumn{2}{p{0.3\linewidth}}{\cellcolor[rgb]{0.95,0.95,0.95}{%
	\makecell[{{p{\linewidth}}}]{% 
	  \tt {\tiny [GM$|$GM]}  
	 [2, 0, 0, 2] 
	  } 
	   } 
	   } 
	 & & \\ 
 

    \theutterance \stepcounter{utterance}  

    & & \multicolumn{4}{p{0.6\linewidth}}{\cellcolor[rgb]{0.9,0.9,0.9}{%
	\makecell[{{p{\linewidth}}}]{% 
	  \tt {\tiny [A$\langle$GM]}  
	 Der andere Spieler hat einen geheimen Vorschlag gemacht. Jetzt bist du an der Reihe, einen eigenen Vorschlag zu machen.\\ \tt  
	  } 
	   } 
	   } 
	 & \\ 
 

    \theutterance \stepcounter{utterance}  

    & \multicolumn{4}{p{0.6\linewidth}}{\cellcolor[rgb]{0.8,1,0.9}{%
	\makecell[{{p{\linewidth}}}]{% 
	  \tt {\tiny [A$\rangle$GM]}  
	 [Vorschlag: 2 Süßigkeiten, 1 Münze] 
	  } 
	   } 
	   } 
	 & & \\ 
 

    \theutterance \stepcounter{utterance}  

    & & & \multicolumn{2}{p{0.3\linewidth}}{\cellcolor[rgb]{0.95,0.95,0.95}{%
	\makecell[{{p{\linewidth}}}]{% 
	  \tt {\tiny [GM$|$GM]}  
	 proposal 
	  } 
	   } 
	   } 
	 & & \\ 
 

    \theutterance \stepcounter{utterance}  

    & & & \multicolumn{2}{p{0.3\linewidth}}{\cellcolor[rgb]{0.95,0.95,0.95}{%
	\makecell[{{p{\linewidth}}}]{% 
	  \tt {\tiny [GM$|$GM]}  
	 [0, 1, 2, 0] 
	  } 
	   } 
	   } 
	 & & \\ 
 

    \theutterance \stepcounter{utterance}  

    & & & \multicolumn{2}{p{0.3\linewidth}}{\cellcolor[rgb]{0.95,0.95,0.95}{%
	\makecell[{{p{\linewidth}}}]{% 
	  \tt {\tiny [GM$|$GM]}  
	 [[0, 1, 2, 0], [2, 0, 0, 2]] 
	  } 
	   } 
	   } 
	 & & \\ 
 

\end{supertabular}
}

\end{document}
