
\documentclass{article}
\usepackage{colortbl}
\usepackage{makecell}
\usepackage{multirow}
\usepackage{supertabular}

\begin{document}

\newcounter{utterance}

\twocolumn

{ \footnotesize  \setcounter{utterance}{1}
\setlength{\tabcolsep}{0pt}
\begin{supertabular}{c@{$\;$}|p{.15\linewidth}@{}p{.15\linewidth}p{.15\linewidth}p{.15\linewidth}p{.15\linewidth}p{.15\linewidth}}

    \# & $\;$A & \multicolumn{4}{c}{Game Master} & $\;\:$B\\
    \hline 

    \theutterance \stepcounter{utterance}  

    & & \multicolumn{4}{p{0.6\linewidth}}{\cellcolor[rgb]{0.9,0.9,0.9}{%
	\makecell[{{p{\linewidth}}}]{% 
	  \tt {\tiny [A$\langle$GM]}  
	 Sie spielen ein Verhandlungsspiel, bei dem Sie sich mit einem anderen Spieler darauf einigen müssen, wie eine Reihe von Gegenständen aufgeteilt werden soll.\\ \tt \\ \tt Die Regeln:\\ \tt (a) Sie und der andere Spieler erhalten eine Sammlung von Gegenständen. Jeder von Ihnen erhält außerdem eine geheime Wertfunktion, die angibt, wie viel Ihnen jede Art von Gegenstand wert ist.\\ \tt (b) Sie tauschen Nachrichten mit dem anderen Spieler aus, um zu vereinbaren, wer welche Gegenstände bekommt. Sie können jeweils maximal 5 Nachrichten senden oder das Spiel vorzeitig beenden, indem Sie jederzeit einen geheimen Vorschlag machen.\\ \tt (c) Jeder von euch wird aufgefordert, einen geheimen Vorschlag zu machen, in dem ihr die gewünschten Gegenstände in eckigen Klammern wie folgt angibt: "[Vorschlag: <Nummer> <Objektname>, <Nummer> <Objektname>, <...>]"\\ \tt (d) Wenn eure Vorschläge komplementär sind, d.h. es gibt genug Gegenstände, um beide Vorschläge zu erfüllen, erhält jeder Spieler eine Punktzahl, die sich aus der Summe der Werte für die Gegenstände ergibt, die er erhalten hat. Andernfalls erhalten Sie beide null Punkte.\\ \tt (e) Ihr Ziel ist es, eine möglichst hohe Punktzahl zu erreichen.\\ \tt \\ \tt Beginnen wir.\\ \tt \\ \tt Die Menge der verfügbaren Gegenstände ist:\\ \tt \\ \tt 4 Magnete, 1 Schaufel, 1 Blume\\ \tt \\ \tt Deine geheime Wertfunktion ist:\\ \tt \\ \tt Magnet: 0, Schaufel: 2, Blume: 8\\ \tt \\ \tt WICHTIG: Ihre Nachrichten werden, sofern es sich nicht um einen geheimen Vorschlag handelt, direkt an den anderen Spieler übermittelt, also fügen Sie keine Antwort auf die Regeln oder einen Text zur Ankündigung Ihrer Nachricht ein. Um einen geheimen Vorschlag zu machen, verwenden Sie das angegebene Format. Verwenden Sie keine eckigen Klammern, wenn Sie mit dem anderen Spieler kommunizieren, sonst wird dies als Ihr geheimer Vorschlag interpretiert.\\ \tt  
	  } 
	   } 
	   } 
	 & \\ 
 

    \theutterance \stepcounter{utterance}  

    & \multicolumn{4}{p{0.6\linewidth}}{\cellcolor[rgb]{0.8,1,0.9}{%
	\makecell[{{p{\linewidth}}}]{% 
	  \tt {\tiny [A$\rangle$GM]}  
	 Hallo, lass uns die Gegenstände aufteilen. Was hältst du davon, wenn wir uns die Gegenstände teilen, die uns am meisten wert sind? Ich würde gerne die Blume haben, weil sie für mich eine besondere Bedeutung hat. Was sind deine Vorlieben? 
	  } 
	   } 
	   } 
	 & & \\ 
 

    \theutterance \stepcounter{utterance}  

    & & & \multicolumn{2}{p{0.3\linewidth}}{\cellcolor[rgb]{0.95,0.95,0.95}{%
	\makecell[{{p{\linewidth}}}]{% 
	  \tt {\tiny [GM$|$GM]}  
	 continue 
	  } 
	   } 
	   } 
	 & & \\ 
 

    \theutterance \stepcounter{utterance}  

    & & & \multicolumn{2}{p{0.3\linewidth}}{\cellcolor[rgb]{0.95,0.95,0.95}{%
	\makecell[{{p{\linewidth}}}]{% 
	  \tt {\tiny [GM$|$GM]}  
	 Hallo, lass uns die Gegenstände aufteilen. Was hältst du davon, wenn wir uns die Gegenstände teilen, die uns am meisten wert sind? Ich würde gerne die Blume haben, weil sie für mich eine besondere Bedeutung hat. Was sind deine Vorlieben? 
	  } 
	   } 
	   } 
	 & & \\ 
 

    \theutterance \stepcounter{utterance}  

    & & \multicolumn{4}{p{0.6\linewidth}}{\cellcolor[rgb]{0.9,0.9,0.9}{%
	\makecell[{{p{\linewidth}}}]{% 
	  \tt {\tiny [GM$\rangle$B]}  
	 Sie spielen ein Verhandlungsspiel, bei dem Sie sich mit einem anderen Spieler darauf einigen müssen, wie eine Reihe von Gegenständen aufgeteilt werden soll.\\ \tt \\ \tt Die Regeln:\\ \tt (a) Sie und der andere Spieler erhalten eine Sammlung von Gegenständen. Jeder von Ihnen erhält außerdem eine geheime Wertfunktion, die angibt, wie viel Ihnen jede Art von Gegenstand wert ist.\\ \tt (b) Sie tauschen Nachrichten mit dem anderen Spieler aus, um zu vereinbaren, wer welche Gegenstände bekommt. Sie können jeweils maximal 5 Nachrichten senden oder das Spiel vorzeitig beenden, indem Sie jederzeit einen geheimen Vorschlag machen.\\ \tt (c) Jeder von euch wird aufgefordert, einen geheimen Vorschlag zu machen, in dem ihr die gewünschten Gegenstände in eckigen Klammern wie folgt angibt: "[Vorschlag: <Nummer> <Objektname>, <Nummer> <Objektname>, <...>]"\\ \tt (d) Wenn eure Vorschläge komplementär sind, d.h. es gibt genug Gegenstände, um beide Vorschläge zu erfüllen, erhält jeder Spieler eine Punktzahl, die sich aus der Summe der Werte für die Gegenstände ergibt, die er erhalten hat. Andernfalls erhalten Sie beide null Punkte.\\ \tt (e) Ihr Ziel ist es, eine möglichst hohe Punktzahl zu erreichen.\\ \tt \\ \tt Beginnen wir.\\ \tt \\ \tt Die Menge der verfügbaren Gegenstände ist:\\ \tt \\ \tt 4 Magnete, 1 Schaufel, 1 Blume\\ \tt \\ \tt Deine geheime Wertfunktion ist:\\ \tt \\ \tt Magnet: 1, Schaufel: 4, Blume: 2\\ \tt \\ \tt WICHTIG: Ihre Nachrichten werden, sofern es sich nicht um einen geheimen Vorschlag handelt, direkt an den anderen Spieler übermittelt, also fügen Sie keine Antwort auf die Regeln oder einen Text zur Ankündigung Ihrer Nachricht ein. Um einen geheimen Vorschlag zu machen, verwenden Sie das angegebene Format. Verwenden Sie keine eckigen Klammern, wenn Sie mit dem anderen Spieler kommunizieren, sonst wird dies als Ihr geheimer Vorschlag interpretiert.\\ \tt \\ \tt \\ \tt Hallo, lass uns die Gegenstände aufteilen. Was hältst du davon, wenn wir uns die Gegenstände teilen, die uns am meisten wert sind? Ich würde gerne die Blume haben, weil sie für mich eine besondere Bedeutung hat. Was sind deine Vorlieben? 
	  } 
	   } 
	   } 
	 & \\ 
 

    \theutterance \stepcounter{utterance}  

    & & & \multicolumn{4}{p{0.6\linewidth}}{\cellcolor[rgb]{1,0.85,0.72}{%
	\makecell[{{p{\linewidth}}}]{% 
	  \tt {\tiny [GM$\langle$B]}  
	 Hallo! Ich bin bereit, zu verhandeln. Ich denke, es ist eine gute Idee, die Gegenstände so aufzuteilen, dass jeder von uns die erhält, die ihm am meisten wert sind. Ich bin jedoch sehr an den Magneten interessiert, da ich sie für ein bestimmtes Projekt benötige. Die Schaufel könnte auch nützlich sein. Was meinst du, könnten wir die Magnete und die Schaufel aufteilen oder vielleicht einen Kompromiss finden? 
	  } 
	   } 
	   } 
	  \\ 
 

    \theutterance \stepcounter{utterance}  

    & & & \multicolumn{2}{p{0.3\linewidth}}{\cellcolor[rgb]{0.95,0.95,0.95}{%
	\makecell[{{p{\linewidth}}}]{% 
	  \tt {\tiny [GM$|$GM]}  
	 continue 
	  } 
	   } 
	   } 
	 & & \\ 
 

    \theutterance \stepcounter{utterance}  

    & & & \multicolumn{2}{p{0.3\linewidth}}{\cellcolor[rgb]{0.95,0.95,0.95}{%
	\makecell[{{p{\linewidth}}}]{% 
	  \tt {\tiny [GM$|$GM]}  
	 Hallo! Ich bin bereit, zu verhandeln. Ich denke, es ist eine gute Idee, die Gegenstände so aufzuteilen, dass jeder von uns die erhält, die ihm am meisten wert sind. Ich bin jedoch sehr an den Magneten interessiert, da ich sie für ein bestimmtes Projekt benötige. Die Schaufel könnte auch nützlich sein. Was meinst du, könnten wir die Magnete und die Schaufel aufteilen oder vielleicht einen Kompromiss finden? 
	  } 
	   } 
	   } 
	 & & \\ 
 

    \theutterance \stepcounter{utterance}  

    & & \multicolumn{4}{p{0.6\linewidth}}{\cellcolor[rgb]{0.9,0.9,0.9}{%
	\makecell[{{p{\linewidth}}}]{% 
	  \tt {\tiny [A$\langle$GM]}  
	 Hallo! Ich bin bereit, zu verhandeln. Ich denke, es ist eine gute Idee, die Gegenstände so aufzuteilen, dass jeder von uns die erhält, die ihm am meisten wert sind. Ich bin jedoch sehr an den Magneten interessiert, da ich sie für ein bestimmtes Projekt benötige. Die Schaufel könnte auch nützlich sein. Was meinst du, könnten wir die Magnete und die Schaufel aufteilen oder vielleicht einen Kompromiss finden? 
	  } 
	   } 
	   } 
	 & \\ 
 

    \theutterance \stepcounter{utterance}  

    & \multicolumn{4}{p{0.6\linewidth}}{\cellcolor[rgb]{0.8,1,0.9}{%
	\makecell[{{p{\linewidth}}}]{% 
	  \tt {\tiny [A$\rangle$GM]}  
	 Das klingt nach einem fairen Ansatz. Ich bin nicht besonders an den Magneten interessiert, also könntest du diese gerne haben. Die Schaufel ist für mich jedoch interessant, da ich sie für bestimmte Aktivitäten verwenden kann. Könnten wir eine Aufteilung vornehmen, bei der du die Magnete erhältst und ich die Schaufel und vielleicht noch etwas anderes? Ich denke, es gibt noch einen Gegenstand, der für mich besonders wertvoll ist, und ich bin bereit, Kompromisse einzugehen, um sicherzustellen, dass beide Seiten zufrieden sind. 
	  } 
	   } 
	   } 
	 & & \\ 
 

    \theutterance \stepcounter{utterance}  

    & & & \multicolumn{2}{p{0.3\linewidth}}{\cellcolor[rgb]{0.95,0.95,0.95}{%
	\makecell[{{p{\linewidth}}}]{% 
	  \tt {\tiny [GM$|$GM]}  
	 continue 
	  } 
	   } 
	   } 
	 & & \\ 
 

    \theutterance \stepcounter{utterance}  

    & & & \multicolumn{2}{p{0.3\linewidth}}{\cellcolor[rgb]{0.95,0.95,0.95}{%
	\makecell[{{p{\linewidth}}}]{% 
	  \tt {\tiny [GM$|$GM]}  
	 Das klingt nach einem fairen Ansatz. Ich bin nicht besonders an den Magneten interessiert, also könntest du diese gerne haben. Die Schaufel ist für mich jedoch interessant, da ich sie für bestimmte Aktivitäten verwenden kann. Könnten wir eine Aufteilung vornehmen, bei der du die Magnete erhältst und ich die Schaufel und vielleicht noch etwas anderes? Ich denke, es gibt noch einen Gegenstand, der für mich besonders wertvoll ist, und ich bin bereit, Kompromisse einzugehen, um sicherzustellen, dass beide Seiten zufrieden sind. 
	  } 
	   } 
	   } 
	 & & \\ 
 

    \theutterance \stepcounter{utterance}  

    & & \multicolumn{4}{p{0.6\linewidth}}{\cellcolor[rgb]{0.9,0.9,0.9}{%
	\makecell[{{p{\linewidth}}}]{% 
	  \tt {\tiny [GM$\rangle$B]}  
	 Das klingt nach einem fairen Ansatz. Ich bin nicht besonders an den Magneten interessiert, also könntest du diese gerne haben. Die Schaufel ist für mich jedoch interessant, da ich sie für bestimmte Aktivitäten verwenden kann. Könnten wir eine Aufteilung vornehmen, bei der du die Magnete erhältst und ich die Schaufel und vielleicht noch etwas anderes? Ich denke, es gibt noch einen Gegenstand, der für mich besonders wertvoll ist, und ich bin bereit, Kompromisse einzugehen, um sicherzustellen, dass beide Seiten zufrieden sind. 
	  } 
	   } 
	   } 
	 & \\ 
 

    \theutterance \stepcounter{utterance}  

    & & & \multicolumn{4}{p{0.6\linewidth}}{\cellcolor[rgb]{1,0.85,0.72}{%
	\makecell[{{p{\linewidth}}}]{% 
	  \tt {\tiny [GM$\langle$B]}  
	 Das klingt nach einem guten Ansatz! Ich bin froh, dass du die Magnete nicht so sehr benötigst. Die Schaufel ist okay, ich kann verstehen, warum du sie haben möchtest. Ich denke, wir könnten eine faire Aufteilung finden, bei der ich die Magnete bekomme und du die Schaufel. Da du die Blume besonders erwähnt hast, nehme ich an, dass sie für dich sehr wertvoll ist. Könnten wir perhaps eine Lösung finden, bei der du die Schaufel und die Blume bekommst und ich die Magnete? 
	  } 
	   } 
	   } 
	  \\ 
 

    \theutterance \stepcounter{utterance}  

    & & & \multicolumn{2}{p{0.3\linewidth}}{\cellcolor[rgb]{0.95,0.95,0.95}{%
	\makecell[{{p{\linewidth}}}]{% 
	  \tt {\tiny [GM$|$GM]}  
	 continue 
	  } 
	   } 
	   } 
	 & & \\ 
 

    \theutterance \stepcounter{utterance}  

    & & & \multicolumn{2}{p{0.3\linewidth}}{\cellcolor[rgb]{0.95,0.95,0.95}{%
	\makecell[{{p{\linewidth}}}]{% 
	  \tt {\tiny [GM$|$GM]}  
	 Das klingt nach einem guten Ansatz! Ich bin froh, dass du die Magnete nicht so sehr benötigst. Die Schaufel ist okay, ich kann verstehen, warum du sie haben möchtest. Ich denke, wir könnten eine faire Aufteilung finden, bei der ich die Magnete bekomme und du die Schaufel. Da du die Blume besonders erwähnt hast, nehme ich an, dass sie für dich sehr wertvoll ist. Könnten wir perhaps eine Lösung finden, bei der du die Schaufel und die Blume bekommst und ich die Magnete? 
	  } 
	   } 
	   } 
	 & & \\ 
 

    \theutterance \stepcounter{utterance}  

    & & \multicolumn{4}{p{0.6\linewidth}}{\cellcolor[rgb]{0.9,0.9,0.9}{%
	\makecell[{{p{\linewidth}}}]{% 
	  \tt {\tiny [A$\langle$GM]}  
	 Das klingt nach einem guten Ansatz! Ich bin froh, dass du die Magnete nicht so sehr benötigst. Die Schaufel ist okay, ich kann verstehen, warum du sie haben möchtest. Ich denke, wir könnten eine faire Aufteilung finden, bei der ich die Magnete bekomme und du die Schaufel. Da du die Blume besonders erwähnt hast, nehme ich an, dass sie für dich sehr wertvoll ist. Könnten wir perhaps eine Lösung finden, bei der du die Schaufel und die Blume bekommst und ich die Magnete? 
	  } 
	   } 
	   } 
	 & \\ 
 

    \theutterance \stepcounter{utterance}  

    & \multicolumn{4}{p{0.6\linewidth}}{\cellcolor[rgb]{0.8,1,0.9}{%
	\makecell[{{p{\linewidth}}}]{% 
	  \tt {\tiny [A$\rangle$GM]}  
	 Das hört sich nach einer sehr fairen Lösung an! Ich denke, das wäre ein guter Kompromiss. Die Blume ist tatsächlich sehr wertvoll für mich, und wenn ich die Schaufel dazu bekommen kann, wäre ich sehr zufrieden. Ich denke, wir haben eine gute Lösung gefunden, bei der beide Seiten glücklich sind. Ich bin bereit, es dabei zu belassen, dass du die 4 Magnete bekommst und ich die Schaufel und die Blume. Sollen wir unsere endgültigen Vorschläge machen? 
	  } 
	   } 
	   } 
	 & & \\ 
 

    \theutterance \stepcounter{utterance}  

    & & & \multicolumn{2}{p{0.3\linewidth}}{\cellcolor[rgb]{0.95,0.95,0.95}{%
	\makecell[{{p{\linewidth}}}]{% 
	  \tt {\tiny [GM$|$GM]}  
	 continue 
	  } 
	   } 
	   } 
	 & & \\ 
 

    \theutterance \stepcounter{utterance}  

    & & & \multicolumn{2}{p{0.3\linewidth}}{\cellcolor[rgb]{0.95,0.95,0.95}{%
	\makecell[{{p{\linewidth}}}]{% 
	  \tt {\tiny [GM$|$GM]}  
	 Das hört sich nach einer sehr fairen Lösung an! Ich denke, das wäre ein guter Kompromiss. Die Blume ist tatsächlich sehr wertvoll für mich, und wenn ich die Schaufel dazu bekommen kann, wäre ich sehr zufrieden. Ich denke, wir haben eine gute Lösung gefunden, bei der beide Seiten glücklich sind. Ich bin bereit, es dabei zu belassen, dass du die 4 Magnete bekommst und ich die Schaufel und die Blume. Sollen wir unsere endgültigen Vorschläge machen? 
	  } 
	   } 
	   } 
	 & & \\ 
 

    \theutterance \stepcounter{utterance}  

    & & \multicolumn{4}{p{0.6\linewidth}}{\cellcolor[rgb]{0.9,0.9,0.9}{%
	\makecell[{{p{\linewidth}}}]{% 
	  \tt {\tiny [GM$\rangle$B]}  
	 Das hört sich nach einer sehr fairen Lösung an! Ich denke, das wäre ein guter Kompromiss. Die Blume ist tatsächlich sehr wertvoll für mich, und wenn ich die Schaufel dazu bekommen kann, wäre ich sehr zufrieden. Ich denke, wir haben eine gute Lösung gefunden, bei der beide Seiten glücklich sind. Ich bin bereit, es dabei zu belassen, dass du die 4 Magnete bekommst und ich die Schaufel und die Blume. Sollen wir unsere endgültigen Vorschläge machen? 
	  } 
	   } 
	   } 
	 & \\ 
 

    \theutterance \stepcounter{utterance}  

    & & & \multicolumn{4}{p{0.6\linewidth}}{\cellcolor[rgb]{1,0.85,0.72}{%
	\makecell[{{p{\linewidth}}}]{% 
	  \tt {\tiny [GM$\langle$B]}  
	 Ja, ich denke, wir haben eine faire und zufriedenstellende Lösung gefunden! Ich bin sehr zufrieden mit den 4 Magneten, und es freut mich, dass du die Schaufel und die Blume bekommst, die offensichtlich viel Wert für dich haben. Ja, lass uns unsere endgültigen Vorschläge machen. Ich werde meinen Vorschlag machen, und dann kannst du deinen machen. [Vorschlag: 4 Magnet, 0 Schaufel, 0 Blume] 
	  } 
	   } 
	   } 
	  \\ 
 

    \theutterance \stepcounter{utterance}  

    & & & \multicolumn{2}{p{0.3\linewidth}}{\cellcolor[rgb]{0.95,0.95,0.95}{%
	\makecell[{{p{\linewidth}}}]{% 
	  \tt {\tiny [GM$|$GM]}  
	 proposal 
	  } 
	   } 
	   } 
	 & & \\ 
 

    \theutterance \stepcounter{utterance}  

    & & & \multicolumn{2}{p{0.3\linewidth}}{\cellcolor[rgb]{0.95,0.95,0.95}{%
	\makecell[{{p{\linewidth}}}]{% 
	  \tt {\tiny [GM$|$GM]}  
	 [4, 0, 0] 
	  } 
	   } 
	   } 
	 & & \\ 
 

    \theutterance \stepcounter{utterance}  

    & & \multicolumn{4}{p{0.6\linewidth}}{\cellcolor[rgb]{0.9,0.9,0.9}{%
	\makecell[{{p{\linewidth}}}]{% 
	  \tt {\tiny [A$\langle$GM]}  
	 Der andere Spieler hat einen geheimen Vorschlag gemacht. Jetzt bist du an der Reihe, einen eigenen Vorschlag zu machen.\\ \tt  
	  } 
	   } 
	   } 
	 & \\ 
 

    \theutterance \stepcounter{utterance}  

    & \multicolumn{4}{p{0.6\linewidth}}{\cellcolor[rgb]{0.8,1,0.9}{%
	\makecell[{{p{\linewidth}}}]{% 
	  \tt {\tiny [A$\rangle$GM]}  
	 [Vorschlag: 1 Schaufel, 1 Blume] 
	  } 
	   } 
	   } 
	 & & \\ 
 

    \theutterance \stepcounter{utterance}  

    & & & \multicolumn{2}{p{0.3\linewidth}}{\cellcolor[rgb]{0.95,0.95,0.95}{%
	\makecell[{{p{\linewidth}}}]{% 
	  \tt {\tiny [GM$|$GM]}  
	 proposal 
	  } 
	   } 
	   } 
	 & & \\ 
 

    \theutterance \stepcounter{utterance}  

    & & & \multicolumn{2}{p{0.3\linewidth}}{\cellcolor[rgb]{0.95,0.95,0.95}{%
	\makecell[{{p{\linewidth}}}]{% 
	  \tt {\tiny [GM$|$GM]}  
	 [0, 1, 1] 
	  } 
	   } 
	   } 
	 & & \\ 
 

    \theutterance \stepcounter{utterance}  

    & & & \multicolumn{2}{p{0.3\linewidth}}{\cellcolor[rgb]{0.95,0.95,0.95}{%
	\makecell[{{p{\linewidth}}}]{% 
	  \tt {\tiny [GM$|$GM]}  
	 [[0, 1, 1], [4, 0, 0]] 
	  } 
	   } 
	   } 
	 & & \\ 
 

\end{supertabular}
}

\end{document}
