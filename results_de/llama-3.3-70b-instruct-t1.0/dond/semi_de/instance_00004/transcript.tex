
\documentclass{article}
\usepackage{colortbl}
\usepackage{makecell}
\usepackage{multirow}
\usepackage{supertabular}

\begin{document}

\newcounter{utterance}

\centering \large Interaction Transcript for game `dond', experiment `semi\_de', episode 4 with llama{-}3.3{-}70b{-}instruct{-}t1.0.
\vspace{24pt}

{ \footnotesize  \setcounter{utterance}{1}
\setlength{\tabcolsep}{0pt}
\begin{supertabular}{c@{$\;$}|p{.15\linewidth}@{}p{.15\linewidth}p{.15\linewidth}p{.15\linewidth}p{.15\linewidth}p{.15\linewidth}}
    \# & $\;$A & \multicolumn{4}{c}{Game Master} & $\;\:$B\\
    \hline

    \theutterance \stepcounter{utterance}  
    & & \multicolumn{4}{p{0.6\linewidth}}{
        \cellcolor[rgb]{0.9,0.9,0.9}{
            \makecell[{{p{\linewidth}}}]{
                \texttt{\tiny{[P1$\langle$GM]}}
                \texttt{Sie spielen ein Verhandlungsspiel, bei dem Sie sich mit einem anderen Spieler darauf einigen müssen, wie eine Reihe von Gegenständen aufgeteilt werden soll.} \\
\\ 
\texttt{Die Regeln:} \\
\texttt{(a) Sie und der andere Spieler erhalten eine Sammlung von Gegenständen. Jeder von Ihnen erhält außerdem eine geheime Wertfunktion, die angibt, wie viel Ihnen jede Art von Gegenstand wert ist.} \\
\texttt{(b) Sie tauschen Nachrichten mit dem anderen Spieler aus, um zu vereinbaren, wer welche Gegenstände bekommt. Sie können jeweils maximal 5 Nachrichten senden oder das Spiel vorzeitig beenden, indem Sie jederzeit einen geheimen Vorschlag machen.} \\
\texttt{(c) Jeder von euch wird aufgefordert, einen geheimen Vorschlag zu machen, in dem ihr die gewünschten Gegenstände in eckigen Klammern wie folgt angibt: "{[}Vorschlag: <Nummer> <Objektname>, <Nummer> <Objektname>, <...>{]}"} \\
\texttt{(d) Wenn eure Vorschläge komplementär sind, d.h. es gibt genug Gegenstände, um beide Vorschläge zu erfüllen, erhält jeder Spieler eine Punktzahl, die sich aus der Summe der Werte für die Gegenstände ergibt, die er erhalten hat. Andernfalls erhalten Sie beide null Punkte.} \\
\texttt{(e) Ihr Ziel ist es, eine möglichst hohe Punktzahl zu erreichen.} \\
\\ 
\texttt{Beginnen wir.} \\
\\ 
\texttt{Die Menge der verfügbaren Gegenstände ist:} \\
\\ 
\texttt{1 Orange, 1 Paprika, 3 Bilder} \\
\\ 
\texttt{Deine geheime Wertfunktion ist:} \\
\\ 
\texttt{Orange: 5, Paprika: 2, Bild: 1} \\
\\ 
\texttt{WICHTIG: Ihre Nachrichten werden, sofern es sich nicht um einen geheimen Vorschlag handelt, direkt an den anderen Spieler übermittelt, also fügen Sie keine Antwort auf die Regeln oder einen Text zur Ankündigung Ihrer Nachricht ein. Um einen geheimen Vorschlag zu machen, verwenden Sie das angegebene Format. Verwenden Sie keine eckigen Klammern, wenn Sie mit dem anderen Spieler kommunizieren, sonst wird dies als Ihr geheimer Vorschlag interpretiert.} \\
            }
        }
    }
    & \\ \\

    \theutterance \stepcounter{utterance}  
    & \multicolumn{4}{p{0.6\linewidth}}{
        \cellcolor[rgb]{0.9,0.9,0.9}{
            \makecell[{{p{\linewidth}}}]{
                \texttt{\tiny{[P1$\rangle$GM]}}
                \texttt{Hallo, ich denke, wir sollten versuchen, einen fairen Kompromiss zu finden. Wie schätzen Sie die Werte der Gegenstände ein? Haben Sie eine Vorstellung davon, wie wir die Dinge aufteilen könnten?} \\
            }
        }
    }
    & & \\ \\

    \theutterance \stepcounter{utterance}  
    & & & \multicolumn{2}{p{0.3\linewidth}}{
        \cellcolor[rgb]{0.9,0.9,0.9}{
            \makecell[{{p{\linewidth}}}]{
                \texttt{\tiny{[GM$|$GM]}}
                \texttt{continue} \\
            }
        }
    }
    & & \\ \\

    \theutterance \stepcounter{utterance}  
    & & & \multicolumn{2}{p{0.3\linewidth}}{
        \cellcolor[rgb]{0.9,0.9,0.9}{
            \makecell[{{p{\linewidth}}}]{
                \texttt{\tiny{[GM$|$GM]}}
                \texttt{Hallo, ich denke, wir sollten versuchen, einen fairen Kompromiss zu finden. Wie schätzen Sie die Werte der Gegenstände ein? Haben Sie eine Vorstellung davon, wie wir die Dinge aufteilen könnten?} \\
            }
        }
    }
    & & \\ \\

    \theutterance \stepcounter{utterance}  
    & & \multicolumn{4}{p{0.6\linewidth}}{
        \cellcolor[rgb]{0.9,0.9,0.9}{
            \makecell[{{p{\linewidth}}}]{
                \texttt{\tiny{[GM$\rangle$P2]}}
                \texttt{Sie spielen ein Verhandlungsspiel, bei dem Sie sich mit einem anderen Spieler darauf einigen müssen, wie eine Reihe von Gegenständen aufgeteilt werden soll.} \\
\\ 
\texttt{Die Regeln:} \\
\texttt{(a) Sie und der andere Spieler erhalten eine Sammlung von Gegenständen. Jeder von Ihnen erhält außerdem eine geheime Wertfunktion, die angibt, wie viel Ihnen jede Art von Gegenstand wert ist.} \\
\texttt{(b) Sie tauschen Nachrichten mit dem anderen Spieler aus, um zu vereinbaren, wer welche Gegenstände bekommt. Sie können jeweils maximal 5 Nachrichten senden oder das Spiel vorzeitig beenden, indem Sie jederzeit einen geheimen Vorschlag machen.} \\
\texttt{(c) Jeder von euch wird aufgefordert, einen geheimen Vorschlag zu machen, in dem ihr die gewünschten Gegenstände in eckigen Klammern wie folgt angibt: "{[}Vorschlag: <Nummer> <Objektname>, <Nummer> <Objektname>, <...>{]}"} \\
\texttt{(d) Wenn eure Vorschläge komplementär sind, d.h. es gibt genug Gegenstände, um beide Vorschläge zu erfüllen, erhält jeder Spieler eine Punktzahl, die sich aus der Summe der Werte für die Gegenstände ergibt, die er erhalten hat. Andernfalls erhalten Sie beide null Punkte.} \\
\texttt{(e) Ihr Ziel ist es, eine möglichst hohe Punktzahl zu erreichen.} \\
\\ 
\texttt{Beginnen wir.} \\
\\ 
\texttt{Die Menge der verfügbaren Gegenstände ist:} \\
\\ 
\texttt{1 Orange, 1 Paprika, 3 Bilder} \\
\\ 
\texttt{Deine geheime Wertfunktion ist:} \\
\\ 
\texttt{Orange: 7, Paprika: 3, Bild: 0} \\
\\ 
\texttt{WICHTIG: Ihre Nachrichten werden, sofern es sich nicht um einen geheimen Vorschlag handelt, direkt an den anderen Spieler übermittelt, also fügen Sie keine Antwort auf die Regeln oder einen Text zur Ankündigung Ihrer Nachricht ein. Um einen geheimen Vorschlag zu machen, verwenden Sie das angegebene Format. Verwenden Sie keine eckigen Klammern, wenn Sie mit dem anderen Spieler kommunizieren, sonst wird dies als Ihr geheimer Vorschlag interpretiert.} \\
\\ 
\\ 
\texttt{Hallo, ich denke, wir sollten versuchen, einen fairen Kompromiss zu finden. Wie schätzen Sie die Werte der Gegenstände ein? Haben Sie eine Vorstellung davon, wie wir die Dinge aufteilen könnten?} \\
            }
        }
    }
    & \\ \\

    \theutterance \stepcounter{utterance}  
    & & & \multicolumn{4}{p{0.6\linewidth}}{
        \cellcolor[rgb]{0.9,0.9,0.9}{
            \makecell[{{p{\linewidth}}}]{
                \texttt{\tiny{[GM$\langle$P2]}}
                \texttt{Ich denke, ein fairer Kompromiss ist ein guter Ansatz. Ich persönlich schätze den Wert der Gegenstände unterschiedlich ein. Die Orange und die Paprika scheinen mir das Wichtigste zu sein, die Bilder weniger. Wie wäre es, wenn wir uns auf eine Aufteilung konzentrieren, die es uns ermöglicht, unsere wichtigsten Bedürfnisse zu befriedigen? Ich bin noch nicht sicher, wie genau wir vorgehen sollen, aber ich denke, es wäre hilfreich, wenn wir unsere Vorlieben und Bedürfnisse austauschen könnten, ohne direkt zu sagen, was wir wollen. Was halten Sie davon, wenn wir versuchen, unsere Interessen zu beschreiben, ohne direkt zu sagen, was wir wollen?} \\
            }
        }
    }
     \\ \\

    \theutterance \stepcounter{utterance}  
    & & & \multicolumn{2}{p{0.3\linewidth}}{
        \cellcolor[rgb]{0.9,0.9,0.9}{
            \makecell[{{p{\linewidth}}}]{
                \texttt{\tiny{[GM$|$GM]}}
                \texttt{continue} \\
            }
        }
    }
    & & \\ \\

    \theutterance \stepcounter{utterance}  
    & & & \multicolumn{2}{p{0.3\linewidth}}{
        \cellcolor[rgb]{0.9,0.9,0.9}{
            \makecell[{{p{\linewidth}}}]{
                \texttt{\tiny{[GM$|$GM]}}
                \texttt{Ich denke, ein fairer Kompromiss ist ein guter Ansatz. Ich persönlich schätze den Wert der Gegenstände unterschiedlich ein. Die Orange und die Paprika scheinen mir das Wichtigste zu sein, die Bilder weniger. Wie wäre es, wenn wir uns auf eine Aufteilung konzentrieren, die es uns ermöglicht, unsere wichtigsten Bedürfnisse zu befriedigen? Ich bin noch nicht sicher, wie genau wir vorgehen sollen, aber ich denke, es wäre hilfreich, wenn wir unsere Vorlieben und Bedürfnisse austauschen könnten, ohne direkt zu sagen, was wir wollen. Was halten Sie davon, wenn wir versuchen, unsere Interessen zu beschreiben, ohne direkt zu sagen, was wir wollen?} \\
            }
        }
    }
    & & \\ \\

    \theutterance \stepcounter{utterance}  
    & & \multicolumn{4}{p{0.6\linewidth}}{
        \cellcolor[rgb]{0.9,0.9,0.9}{
            \makecell[{{p{\linewidth}}}]{
                \texttt{\tiny{[P1$\langle$GM]}}
                \texttt{Ich denke, ein fairer Kompromiss ist ein guter Ansatz. Ich persönlich schätze den Wert der Gegenstände unterschiedlich ein. Die Orange und die Paprika scheinen mir das Wichtigste zu sein, die Bilder weniger. Wie wäre es, wenn wir uns auf eine Aufteilung konzentrieren, die es uns ermöglicht, unsere wichtigsten Bedürfnisse zu befriedigen? Ich bin noch nicht sicher, wie genau wir vorgehen sollen, aber ich denke, es wäre hilfreich, wenn wir unsere Vorlieben und Bedürfnisse austauschen könnten, ohne direkt zu sagen, was wir wollen. Was halten Sie davon, wenn wir versuchen, unsere Interessen zu beschreiben, ohne direkt zu sagen, was wir wollen?} \\
            }
        }
    }
    & \\ \\

    \theutterance \stepcounter{utterance}  
    & \multicolumn{4}{p{0.6\linewidth}}{
        \cellcolor[rgb]{0.9,0.9,0.9}{
            \makecell[{{p{\linewidth}}}]{
                \texttt{\tiny{[P1$\rangle$GM]}}
                \texttt{Das klingt nach einem guten Ansatz. Ich denke, es wäre hilfreich, wenn wir unsere allgemeinen Vorlieben und Prioritäten besprechen könnten, ohne zu spezifisch zu werden. Ich persönlich tendiere dazu, Gegenstände zu bevorzugen, die vielleicht nicht so leicht wiederzubekommen sind oder einzigartig sind. Gegenstände, die in größerer Menge vorhanden sind, scheinen mir weniger prioritätswert zu sein. Wie sehen Sie das? Gibt es bestimmte Aspekte oder Eigenschaften der Gegenstände, die für Sie besonders wichtig sind?} \\
            }
        }
    }
    & & \\ \\

    \theutterance \stepcounter{utterance}  
    & & & \multicolumn{2}{p{0.3\linewidth}}{
        \cellcolor[rgb]{0.9,0.9,0.9}{
            \makecell[{{p{\linewidth}}}]{
                \texttt{\tiny{[GM$|$GM]}}
                \texttt{continue} \\
            }
        }
    }
    & & \\ \\

    \theutterance \stepcounter{utterance}  
    & & & \multicolumn{2}{p{0.3\linewidth}}{
        \cellcolor[rgb]{0.9,0.9,0.9}{
            \makecell[{{p{\linewidth}}}]{
                \texttt{\tiny{[GM$|$GM]}}
                \texttt{Das klingt nach einem guten Ansatz. Ich denke, es wäre hilfreich, wenn wir unsere allgemeinen Vorlieben und Prioritäten besprechen könnten, ohne zu spezifisch zu werden. Ich persönlich tendiere dazu, Gegenstände zu bevorzugen, die vielleicht nicht so leicht wiederzubekommen sind oder einzigartig sind. Gegenstände, die in größerer Menge vorhanden sind, scheinen mir weniger prioritätswert zu sein. Wie sehen Sie das? Gibt es bestimmte Aspekte oder Eigenschaften der Gegenstände, die für Sie besonders wichtig sind?} \\
            }
        }
    }
    & & \\ \\

    \theutterance \stepcounter{utterance}  
    & & \multicolumn{4}{p{0.6\linewidth}}{
        \cellcolor[rgb]{0.9,0.9,0.9}{
            \makecell[{{p{\linewidth}}}]{
                \texttt{\tiny{[GM$\rangle$P2]}}
                \texttt{Das klingt nach einem guten Ansatz. Ich denke, es wäre hilfreich, wenn wir unsere allgemeinen Vorlieben und Prioritäten besprechen könnten, ohne zu spezifisch zu werden. Ich persönlich tendiere dazu, Gegenstände zu bevorzugen, die vielleicht nicht so leicht wiederzubekommen sind oder einzigartig sind. Gegenstände, die in größerer Menge vorhanden sind, scheinen mir weniger prioritätswert zu sein. Wie sehen Sie das? Gibt es bestimmte Aspekte oder Eigenschaften der Gegenstände, die für Sie besonders wichtig sind?} \\
            }
        }
    }
    & \\ \\

    \theutterance \stepcounter{utterance}  
    & & & \multicolumn{4}{p{0.6\linewidth}}{
        \cellcolor[rgb]{0.9,0.9,0.9}{
            \makecell[{{p{\linewidth}}}]{
                \texttt{\tiny{[GM$\langle$P2]}}
                \texttt{Das ist ein interessanter Punkt, den Sie ansprechen. Ich denke, Einzigartigkeit und Seltenheit können definitiv einen Wertaspekt darstellen. Für mich persönlich sind es nicht nur die Menge oder die Seltenheit, sondern auch die potenzielle Nutzung oder der Nutzen, den ich aus einem Gegenstand ziehen kann. Ich neige dazu, Gegenstände zu bevorzugen, die eine bestimmte Funktion erfüllen oder einen praktischen Nutzen haben. Gegenstände, die mehr dekorativ sind, scheinen mir weniger relevant zu sein, es sei denn, sie haben einen besonderen emotionalen Wert. Wie sehen Sie den Aspekt des Nutzens oder der Funktionalität? Spielt das für Sie eine Rolle bei der Bewertung der Gegenstände?} \\
            }
        }
    }
     \\ \\

    \theutterance \stepcounter{utterance}  
    & & & \multicolumn{2}{p{0.3\linewidth}}{
        \cellcolor[rgb]{0.9,0.9,0.9}{
            \makecell[{{p{\linewidth}}}]{
                \texttt{\tiny{[GM$|$GM]}}
                \texttt{continue} \\
            }
        }
    }
    & & \\ \\

    \theutterance \stepcounter{utterance}  
    & & & \multicolumn{2}{p{0.3\linewidth}}{
        \cellcolor[rgb]{0.9,0.9,0.9}{
            \makecell[{{p{\linewidth}}}]{
                \texttt{\tiny{[GM$|$GM]}}
                \texttt{Das ist ein interessanter Punkt, den Sie ansprechen. Ich denke, Einzigartigkeit und Seltenheit können definitiv einen Wertaspekt darstellen. Für mich persönlich sind es nicht nur die Menge oder die Seltenheit, sondern auch die potenzielle Nutzung oder der Nutzen, den ich aus einem Gegenstand ziehen kann. Ich neige dazu, Gegenstände zu bevorzugen, die eine bestimmte Funktion erfüllen oder einen praktischen Nutzen haben. Gegenstände, die mehr dekorativ sind, scheinen mir weniger relevant zu sein, es sei denn, sie haben einen besonderen emotionalen Wert. Wie sehen Sie den Aspekt des Nutzens oder der Funktionalität? Spielt das für Sie eine Rolle bei der Bewertung der Gegenstände?} \\
            }
        }
    }
    & & \\ \\

    \theutterance \stepcounter{utterance}  
    & & \multicolumn{4}{p{0.6\linewidth}}{
        \cellcolor[rgb]{0.9,0.9,0.9}{
            \makecell[{{p{\linewidth}}}]{
                \texttt{\tiny{[P1$\langle$GM]}}
                \texttt{Das ist ein interessanter Punkt, den Sie ansprechen. Ich denke, Einzigartigkeit und Seltenheit können definitiv einen Wertaspekt darstellen. Für mich persönlich sind es nicht nur die Menge oder die Seltenheit, sondern auch die potenzielle Nutzung oder der Nutzen, den ich aus einem Gegenstand ziehen kann. Ich neige dazu, Gegenstände zu bevorzugen, die eine bestimmte Funktion erfüllen oder einen praktischen Nutzen haben. Gegenstände, die mehr dekorativ sind, scheinen mir weniger relevant zu sein, es sei denn, sie haben einen besonderen emotionalen Wert. Wie sehen Sie den Aspekt des Nutzens oder der Funktionalität? Spielt das für Sie eine Rolle bei der Bewertung der Gegenstände?} \\
            }
        }
    }
    & \\ \\

    \theutterance \stepcounter{utterance}  
    & \multicolumn{4}{p{0.6\linewidth}}{
        \cellcolor[rgb]{0.9,0.9,0.9}{
            \makecell[{{p{\linewidth}}}]{
                \texttt{\tiny{[P1$\rangle$GM]}}
                \texttt{Das ist ein wichtiger Gesichtspunkt. Für mich spielt der Nutzen oder die Funktionalität einerseits eine Rolle, aber auch der emotionale Wert oder der persönliche Reiz, den ein Gegenstand auslöst. Ich denke, es ist eine gute Balance zwischen praktischem Nutzen und persönlicher Vorliebe zu finden. Gegenstände, die sowohl nützlich als auch angenehm zu besitzen sind, scheinen mir besonders wertvoll. Es ist auch interessant, dass Sie den emotionalen Wert erwähnen. Ich glaube, dass dies ein Aspekt ist, der oft übersehen wird, aber letztendlich unsere Entscheidungen stark beeinflusst. Wie denken Sie, dass wir diese unterschiedlichen Aspekte bei unserer Entscheidung berücksichtigen können, um eine faire und für beide Seiten zufriedenstellende Lösung zu finden?} \\
            }
        }
    }
    & & \\ \\

    \theutterance \stepcounter{utterance}  
    & & & \multicolumn{2}{p{0.3\linewidth}}{
        \cellcolor[rgb]{0.9,0.9,0.9}{
            \makecell[{{p{\linewidth}}}]{
                \texttt{\tiny{[GM$|$GM]}}
                \texttt{continue} \\
            }
        }
    }
    & & \\ \\

    \theutterance \stepcounter{utterance}  
    & & & \multicolumn{2}{p{0.3\linewidth}}{
        \cellcolor[rgb]{0.9,0.9,0.9}{
            \makecell[{{p{\linewidth}}}]{
                \texttt{\tiny{[GM$|$GM]}}
                \texttt{Das ist ein wichtiger Gesichtspunkt. Für mich spielt der Nutzen oder die Funktionalität einerseits eine Rolle, aber auch der emotionale Wert oder der persönliche Reiz, den ein Gegenstand auslöst. Ich denke, es ist eine gute Balance zwischen praktischem Nutzen und persönlicher Vorliebe zu finden. Gegenstände, die sowohl nützlich als auch angenehm zu besitzen sind, scheinen mir besonders wertvoll. Es ist auch interessant, dass Sie den emotionalen Wert erwähnen. Ich glaube, dass dies ein Aspekt ist, der oft übersehen wird, aber letztendlich unsere Entscheidungen stark beeinflusst. Wie denken Sie, dass wir diese unterschiedlichen Aspekte bei unserer Entscheidung berücksichtigen können, um eine faire und für beide Seiten zufriedenstellende Lösung zu finden?} \\
            }
        }
    }
    & & \\ \\

    \theutterance \stepcounter{utterance}  
    & & \multicolumn{4}{p{0.6\linewidth}}{
        \cellcolor[rgb]{0.9,0.9,0.9}{
            \makecell[{{p{\linewidth}}}]{
                \texttt{\tiny{[GM$\rangle$P2]}}
                \texttt{Das ist ein wichtiger Gesichtspunkt. Für mich spielt der Nutzen oder die Funktionalität einerseits eine Rolle, aber auch der emotionale Wert oder der persönliche Reiz, den ein Gegenstand auslöst. Ich denke, es ist eine gute Balance zwischen praktischem Nutzen und persönlicher Vorliebe zu finden. Gegenstände, die sowohl nützlich als auch angenehm zu besitzen sind, scheinen mir besonders wertvoll. Es ist auch interessant, dass Sie den emotionalen Wert erwähnen. Ich glaube, dass dies ein Aspekt ist, der oft übersehen wird, aber letztendlich unsere Entscheidungen stark beeinflusst. Wie denken Sie, dass wir diese unterschiedlichen Aspekte bei unserer Entscheidung berücksichtigen können, um eine faire und für beide Seiten zufriedenstellende Lösung zu finden?} \\
            }
        }
    }
    & \\ \\

    \theutterance \stepcounter{utterance}  
    & & & \multicolumn{4}{p{0.6\linewidth}}{
        \cellcolor[rgb]{0.9,0.9,0.9}{
            \makecell[{{p{\linewidth}}}]{
                \texttt{\tiny{[GM$\langle$P2]}}
                \texttt{Ich denke, das ist ein sehr gültiger Punkt. Die Balance zwischen praktischem Nutzen und persönlicher Vorliebe ist entscheidend. Um eine faire Lösung zu finden, könnten wir perhaps versuchen, die Gegenstände in Kategorien einzuteilen, basierend auf ihren Eigenschaften und unserem wahrgenommenen Wert. Das könnte uns helfen, besser zu verstehen, welche Gegenstände für jeden von uns am wertvollsten sind und wie wir diese am besten aufteilen können. Es könnte auch hilfreich sein, wenn wir offen über unsere Prioritäten sprechen, ohne unsere genauen Vorlieben preiszugeben, um eine Art von Verständnis oder gegenseitigem Respekt für unsere Bedürfnisse zu entwickeln. Wie wäre es, wenn wir eine Art von Listen oder Kategorien erstellen, um unsere Überlegungen zu strukturieren, ohne direkt unsere Endziele zu enthüllen?} \\
            }
        }
    }
     \\ \\

    \theutterance \stepcounter{utterance}  
    & & & \multicolumn{2}{p{0.3\linewidth}}{
        \cellcolor[rgb]{0.9,0.9,0.9}{
            \makecell[{{p{\linewidth}}}]{
                \texttt{\tiny{[GM$|$GM]}}
                \texttt{continue} \\
            }
        }
    }
    & & \\ \\

    \theutterance \stepcounter{utterance}  
    & & & \multicolumn{2}{p{0.3\linewidth}}{
        \cellcolor[rgb]{0.9,0.9,0.9}{
            \makecell[{{p{\linewidth}}}]{
                \texttt{\tiny{[GM$|$GM]}}
                \texttt{Ich denke, das ist ein sehr gültiger Punkt. Die Balance zwischen praktischem Nutzen und persönlicher Vorliebe ist entscheidend. Um eine faire Lösung zu finden, könnten wir perhaps versuchen, die Gegenstände in Kategorien einzuteilen, basierend auf ihren Eigenschaften und unserem wahrgenommenen Wert. Das könnte uns helfen, besser zu verstehen, welche Gegenstände für jeden von uns am wertvollsten sind und wie wir diese am besten aufteilen können. Es könnte auch hilfreich sein, wenn wir offen über unsere Prioritäten sprechen, ohne unsere genauen Vorlieben preiszugeben, um eine Art von Verständnis oder gegenseitigem Respekt für unsere Bedürfnisse zu entwickeln. Wie wäre es, wenn wir eine Art von Listen oder Kategorien erstellen, um unsere Überlegungen zu strukturieren, ohne direkt unsere Endziele zu enthüllen?} \\
            }
        }
    }
    & & \\ \\

    \theutterance \stepcounter{utterance}  
    & & \multicolumn{4}{p{0.6\linewidth}}{
        \cellcolor[rgb]{0.9,0.9,0.9}{
            \makecell[{{p{\linewidth}}}]{
                \texttt{\tiny{[P1$\langle$GM]}}
                \texttt{Ich denke, das ist ein sehr gültiger Punkt. Die Balance zwischen praktischem Nutzen und persönlicher Vorliebe ist entscheidend. Um eine faire Lösung zu finden, könnten wir perhaps versuchen, die Gegenstände in Kategorien einzuteilen, basierend auf ihren Eigenschaften und unserem wahrgenommenen Wert. Das könnte uns helfen, besser zu verstehen, welche Gegenstände für jeden von uns am wertvollsten sind und wie wir diese am besten aufteilen können. Es könnte auch hilfreich sein, wenn wir offen über unsere Prioritäten sprechen, ohne unsere genauen Vorlieben preiszugeben, um eine Art von Verständnis oder gegenseitigem Respekt für unsere Bedürfnisse zu entwickeln. Wie wäre es, wenn wir eine Art von Listen oder Kategorien erstellen, um unsere Überlegungen zu strukturieren, ohne direkt unsere Endziele zu enthüllen?} \\
            }
        }
    }
    & \\ \\

    \theutterance \stepcounter{utterance}  
    & \multicolumn{4}{p{0.6\linewidth}}{
        \cellcolor[rgb]{0.9,0.9,0.9}{
            \makecell[{{p{\linewidth}}}]{
                \texttt{\tiny{[P1$\rangle$GM]}}
                \texttt{Das klingt nach einem sehr vernünftigen Ansatz. Ich denke, wenn wir die Gegenstände in Kategorien einordnen, wie Sie vorgeschlagen haben, könnten wir vielleicht Kategorien wie "Essen", "Dekoration" oder "Sonstiges" erstellen. Dies könnte uns helfen, die Gegenstände nach ihrem Zweck oder ihrer Funktion zu gliedern und so unsere Gespräche zu strukturieren. Es ist auch eine gute Idee, offen über unsere Prioritäten zu sprechen, ohne zu viel über unsere spezifischen Vorlieben preiszugeben. Indem wir unsere Bedürfnisse und Interessen auf eine Weise diskutieren, die respektvoll und nicht zu direkt ist, können wir vielleicht eine Lösung finden, die für beide Seiten akzeptabel ist. Wie wäre es, wenn wir als Nächstes versuchen, die Kategorien zu definieren und dann zu sehen, wie wir innerhalb dieser Kategorien unsere Interessen ausdrücken können, ohne unsere genauen Wünsche zu enthüllen? } \\
\\ 
\texttt{{[}Vorschlag: 1 Orange, 2 Bilder{]}} \\
            }
        }
    }
    & & \\ \\

    \theutterance \stepcounter{utterance}  
    & & & \multicolumn{2}{p{0.3\linewidth}}{
        \cellcolor[rgb]{0.9,0.9,0.9}{
            \makecell[{{p{\linewidth}}}]{
                \texttt{\tiny{[GM$|$GM]}}
                \texttt{proposal} \\
            }
        }
    }
    & & \\ \\

    \theutterance \stepcounter{utterance}  
    & & & \multicolumn{2}{p{0.3\linewidth}}{
        \cellcolor[rgb]{0.9,0.9,0.9}{
            \makecell[{{p{\linewidth}}}]{
                \texttt{\tiny{[GM$|$GM]}}
                [1, 0, 2]
            }
        }
    }
    & & \\ \\

    \theutterance \stepcounter{utterance}  
    & & \multicolumn{4}{p{0.6\linewidth}}{
        \cellcolor[rgb]{0.9,0.9,0.9}{
            \makecell[{{p{\linewidth}}}]{
                \texttt{\tiny{[GM$\rangle$P2]}}
                \texttt{Der andere Spieler hat einen geheimen Vorschlag gemacht. Jetzt bist du an der Reihe, einen eigenen Vorschlag zu machen.} \\
            }
        }
    }
    & \\ \\

    \theutterance \stepcounter{utterance}  
    & & & \multicolumn{4}{p{0.6\linewidth}}{
        \cellcolor[rgb]{0.9,0.9,0.9}{
            \makecell[{{p{\linewidth}}}]{
                \texttt{\tiny{[GM$\langle$P2]}}
                \texttt{{[}Vorschlag: 1 Orange, 1 Paprika, 0 Bilder{]}} \\
            }
        }
    }
     \\ \\

    \theutterance \stepcounter{utterance}  
    & & & \multicolumn{2}{p{0.3\linewidth}}{
        \cellcolor[rgb]{0.9,0.9,0.9}{
            \makecell[{{p{\linewidth}}}]{
                \texttt{\tiny{[GM$|$GM]}}
                \texttt{proposal} \\
            }
        }
    }
    & & \\ \\

    \theutterance \stepcounter{utterance}  
    & & & \multicolumn{2}{p{0.3\linewidth}}{
        \cellcolor[rgb]{0.9,0.9,0.9}{
            \makecell[{{p{\linewidth}}}]{
                \texttt{\tiny{[GM$|$GM]}}
                [1, 1, 0]
            }
        }
    }
    & & \\ \\

    \theutterance \stepcounter{utterance}  
    & & & \multicolumn{2}{p{0.3\linewidth}}{
        \cellcolor[rgb]{0.9,0.9,0.9}{
            \makecell[{{p{\linewidth}}}]{
                \texttt{\tiny{[GM$|$GM]}}
                [[1, 0, 2], [1, 1, 0]]
            }
        }
    }
    & & \\ \\

\end{supertabular}
}

\end{document}
