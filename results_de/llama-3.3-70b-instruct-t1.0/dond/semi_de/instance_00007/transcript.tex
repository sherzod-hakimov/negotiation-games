
\documentclass{article}
\usepackage{colortbl}
\usepackage{makecell}
\usepackage{multirow}
\usepackage{supertabular}

\begin{document}

\newcounter{utterance}

\twocolumn

{ \footnotesize  \setcounter{utterance}{1}
\setlength{\tabcolsep}{0pt}
\begin{supertabular}{c@{$\;$}|p{.15\linewidth}@{}p{.15\linewidth}p{.15\linewidth}p{.15\linewidth}p{.15\linewidth}p{.15\linewidth}}

    \# & $\;$A & \multicolumn{4}{c}{Game Master} & $\;\:$B\\
    \hline 

    \theutterance \stepcounter{utterance}  

    & & \multicolumn{4}{p{0.6\linewidth}}{\cellcolor[rgb]{0.9,0.9,0.9}{%
	\makecell[{{p{\linewidth}}}]{% 
	  \tt {\tiny [A$\langle$GM]}  
	 Sie spielen ein Verhandlungsspiel, bei dem Sie sich mit einem anderen Spieler darauf einigen müssen, wie eine Reihe von Gegenständen aufgeteilt werden soll.\\ \tt \\ \tt Die Regeln:\\ \tt (a) Sie und der andere Spieler erhalten eine Sammlung von Gegenständen. Jeder von Ihnen erhält außerdem eine geheime Wertfunktion, die angibt, wie viel Ihnen jede Art von Gegenstand wert ist.\\ \tt (b) Sie tauschen Nachrichten mit dem anderen Spieler aus, um zu vereinbaren, wer welche Gegenstände bekommt. Sie können jeweils maximal 5 Nachrichten senden oder das Spiel vorzeitig beenden, indem Sie jederzeit einen geheimen Vorschlag machen.\\ \tt (c) Jeder von euch wird aufgefordert, einen geheimen Vorschlag zu machen, in dem ihr die gewünschten Gegenstände in eckigen Klammern wie folgt angibt: "[Vorschlag: <Nummer> <Objektname>, <Nummer> <Objektname>, <...>]"\\ \tt (d) Wenn eure Vorschläge komplementär sind, d.h. es gibt genug Gegenstände, um beide Vorschläge zu erfüllen, erhält jeder Spieler eine Punktzahl, die sich aus der Summe der Werte für die Gegenstände ergibt, die er erhalten hat. Andernfalls erhalten Sie beide null Punkte.\\ \tt (e) Ihr Ziel ist es, eine möglichst hohe Punktzahl zu erreichen.\\ \tt \\ \tt Beginnen wir.\\ \tt \\ \tt Die Menge der verfügbaren Gegenstände ist:\\ \tt \\ \tt 2 Drähte, 2 Gläser, 2 Nadeln, 2 Tassen\\ \tt \\ \tt Deine geheime Wertfunktion ist:\\ \tt \\ \tt Draht: 1, Glas: 0, Nadel: 0, Tasse: 4\\ \tt \\ \tt WICHTIG: Ihre Nachrichten werden, sofern es sich nicht um einen geheimen Vorschlag handelt, direkt an den anderen Spieler übermittelt, also fügen Sie keine Antwort auf die Regeln oder einen Text zur Ankündigung Ihrer Nachricht ein. Um einen geheimen Vorschlag zu machen, verwenden Sie das angegebene Format. Verwenden Sie keine eckigen Klammern, wenn Sie mit dem anderen Spieler kommunizieren, sonst wird dies als Ihr geheimer Vorschlag interpretiert.\\ \tt  
	  } 
	   } 
	   } 
	 & \\ 
 

    \theutterance \stepcounter{utterance}  

    & \multicolumn{4}{p{0.6\linewidth}}{\cellcolor[rgb]{0.8,1,0.9}{%
	\makecell[{{p{\linewidth}}}]{% 
	  \tt {\tiny [A$\rangle$GM]}  
	 Hallo, ich würde gerne wissen, welche Art von Gegenständen du priorisierst. Hast du eine Präferenz für Nadeln, Drähte, Gläser oder Tassen? 
	  } 
	   } 
	   } 
	 & & \\ 
 

    \theutterance \stepcounter{utterance}  

    & & & \multicolumn{2}{p{0.3\linewidth}}{\cellcolor[rgb]{0.95,0.95,0.95}{%
	\makecell[{{p{\linewidth}}}]{% 
	  \tt {\tiny [GM$|$GM]}  
	 continue 
	  } 
	   } 
	   } 
	 & & \\ 
 

    \theutterance \stepcounter{utterance}  

    & & & \multicolumn{2}{p{0.3\linewidth}}{\cellcolor[rgb]{0.95,0.95,0.95}{%
	\makecell[{{p{\linewidth}}}]{% 
	  \tt {\tiny [GM$|$GM]}  
	 Hallo, ich würde gerne wissen, welche Art von Gegenständen du priorisierst. Hast du eine Präferenz für Nadeln, Drähte, Gläser oder Tassen? 
	  } 
	   } 
	   } 
	 & & \\ 
 

    \theutterance \stepcounter{utterance}  

    & & \multicolumn{4}{p{0.6\linewidth}}{\cellcolor[rgb]{0.9,0.9,0.9}{%
	\makecell[{{p{\linewidth}}}]{% 
	  \tt {\tiny [GM$\rangle$B]}  
	 Sie spielen ein Verhandlungsspiel, bei dem Sie sich mit einem anderen Spieler darauf einigen müssen, wie eine Reihe von Gegenständen aufgeteilt werden soll.\\ \tt \\ \tt Die Regeln:\\ \tt (a) Sie und der andere Spieler erhalten eine Sammlung von Gegenständen. Jeder von Ihnen erhält außerdem eine geheime Wertfunktion, die angibt, wie viel Ihnen jede Art von Gegenstand wert ist.\\ \tt (b) Sie tauschen Nachrichten mit dem anderen Spieler aus, um zu vereinbaren, wer welche Gegenstände bekommt. Sie können jeweils maximal 5 Nachrichten senden oder das Spiel vorzeitig beenden, indem Sie jederzeit einen geheimen Vorschlag machen.\\ \tt (c) Jeder von euch wird aufgefordert, einen geheimen Vorschlag zu machen, in dem ihr die gewünschten Gegenstände in eckigen Klammern wie folgt angibt: "[Vorschlag: <Nummer> <Objektname>, <Nummer> <Objektname>, <...>]"\\ \tt (d) Wenn eure Vorschläge komplementär sind, d.h. es gibt genug Gegenstände, um beide Vorschläge zu erfüllen, erhält jeder Spieler eine Punktzahl, die sich aus der Summe der Werte für die Gegenstände ergibt, die er erhalten hat. Andernfalls erhalten Sie beide null Punkte.\\ \tt (e) Ihr Ziel ist es, eine möglichst hohe Punktzahl zu erreichen.\\ \tt \\ \tt Beginnen wir.\\ \tt \\ \tt Die Menge der verfügbaren Gegenstände ist:\\ \tt \\ \tt 2 Drähte, 2 Gläser, 2 Nadeln, 2 Tassen\\ \tt \\ \tt Deine geheime Wertfunktion ist:\\ \tt \\ \tt Draht: 1, Glas: 2, Nadel: 2, Tasse: 0\\ \tt \\ \tt WICHTIG: Ihre Nachrichten werden, sofern es sich nicht um einen geheimen Vorschlag handelt, direkt an den anderen Spieler übermittelt, also fügen Sie keine Antwort auf die Regeln oder einen Text zur Ankündigung Ihrer Nachricht ein. Um einen geheimen Vorschlag zu machen, verwenden Sie das angegebene Format. Verwenden Sie keine eckigen Klammern, wenn Sie mit dem anderen Spieler kommunizieren, sonst wird dies als Ihr geheimer Vorschlag interpretiert.\\ \tt \\ \tt \\ \tt Hallo, ich würde gerne wissen, welche Art von Gegenständen du priorisierst. Hast du eine Präferenz für Nadeln, Drähte, Gläser oder Tassen? 
	  } 
	   } 
	   } 
	 & \\ 
 

    \theutterance \stepcounter{utterance}  

    & & & \multicolumn{4}{p{0.6\linewidth}}{\cellcolor[rgb]{1,0.85,0.72}{%
	\makecell[{{p{\linewidth}}}]{% 
	  \tt {\tiny [GM$\langle$B]}  
	 Hallo! Ich denke, ich priorisiere Gläser und Nadeln, ich finde sie both ziemlich nützlich. Wie sieht es bei dir aus, hast du eine Vorliebe für eine bestimmte Art von Gegenständen? 
	  } 
	   } 
	   } 
	  \\ 
 

    \theutterance \stepcounter{utterance}  

    & & & \multicolumn{2}{p{0.3\linewidth}}{\cellcolor[rgb]{0.95,0.95,0.95}{%
	\makecell[{{p{\linewidth}}}]{% 
	  \tt {\tiny [GM$|$GM]}  
	 continue 
	  } 
	   } 
	   } 
	 & & \\ 
 

    \theutterance \stepcounter{utterance}  

    & & & \multicolumn{2}{p{0.3\linewidth}}{\cellcolor[rgb]{0.95,0.95,0.95}{%
	\makecell[{{p{\linewidth}}}]{% 
	  \tt {\tiny [GM$|$GM]}  
	 Hallo! Ich denke, ich priorisiere Gläser und Nadeln, ich finde sie both ziemlich nützlich. Wie sieht es bei dir aus, hast du eine Vorliebe für eine bestimmte Art von Gegenständen? 
	  } 
	   } 
	   } 
	 & & \\ 
 

    \theutterance \stepcounter{utterance}  

    & & \multicolumn{4}{p{0.6\linewidth}}{\cellcolor[rgb]{0.9,0.9,0.9}{%
	\makecell[{{p{\linewidth}}}]{% 
	  \tt {\tiny [A$\langle$GM]}  
	 Hallo! Ich denke, ich priorisiere Gläser und Nadeln, ich finde sie both ziemlich nützlich. Wie sieht es bei dir aus, hast du eine Vorliebe für eine bestimmte Art von Gegenständen? 
	  } 
	   } 
	   } 
	 & \\ 
 

    \theutterance \stepcounter{utterance}  

    & \multicolumn{4}{p{0.6\linewidth}}{\cellcolor[rgb]{0.8,1,0.9}{%
	\makecell[{{p{\linewidth}}}]{% 
	  \tt {\tiny [A$\rangle$GM]}  
	 Das klingt interessant! Ich persönlich halte Drähte für ziemlich praktisch, aber ich bin auch an Tassen interessiert. Ich denke, es wäre hilfreich, wenn wir unsere Interessen austauschen, um zu sehen, ob wir eine faire Aufteilung finden können. Hast du ein Interesse an Drähten oder Tassen, oder bist du eher auf Gläser und Nadeln fokussiert? 
	  } 
	   } 
	   } 
	 & & \\ 
 

    \theutterance \stepcounter{utterance}  

    & & & \multicolumn{2}{p{0.3\linewidth}}{\cellcolor[rgb]{0.95,0.95,0.95}{%
	\makecell[{{p{\linewidth}}}]{% 
	  \tt {\tiny [GM$|$GM]}  
	 continue 
	  } 
	   } 
	   } 
	 & & \\ 
 

    \theutterance \stepcounter{utterance}  

    & & & \multicolumn{2}{p{0.3\linewidth}}{\cellcolor[rgb]{0.95,0.95,0.95}{%
	\makecell[{{p{\linewidth}}}]{% 
	  \tt {\tiny [GM$|$GM]}  
	 Das klingt interessant! Ich persönlich halte Drähte für ziemlich praktisch, aber ich bin auch an Tassen interessiert. Ich denke, es wäre hilfreich, wenn wir unsere Interessen austauschen, um zu sehen, ob wir eine faire Aufteilung finden können. Hast du ein Interesse an Drähten oder Tassen, oder bist du eher auf Gläser und Nadeln fokussiert? 
	  } 
	   } 
	   } 
	 & & \\ 
 

    \theutterance \stepcounter{utterance}  

    & & \multicolumn{4}{p{0.6\linewidth}}{\cellcolor[rgb]{0.9,0.9,0.9}{%
	\makecell[{{p{\linewidth}}}]{% 
	  \tt {\tiny [GM$\rangle$B]}  
	 Das klingt interessant! Ich persönlich halte Drähte für ziemlich praktisch, aber ich bin auch an Tassen interessiert. Ich denke, es wäre hilfreich, wenn wir unsere Interessen austauschen, um zu sehen, ob wir eine faire Aufteilung finden können. Hast du ein Interesse an Drähten oder Tassen, oder bist du eher auf Gläser und Nadeln fokussiert? 
	  } 
	   } 
	   } 
	 & \\ 
 

    \theutterance \stepcounter{utterance}  

    & & & \multicolumn{4}{p{0.6\linewidth}}{\cellcolor[rgb]{1,0.85,0.72}{%
	\makecell[{{p{\linewidth}}}]{% 
	  \tt {\tiny [GM$\langle$B]}  
	 Das ist eine gute Idee, unsere Interessen auszutauschen. Ich muss zugeben, dass ich nicht so sehr an Drähten interessiert bin, aber ich denke, Tassen könnten nützlich sein. Mein Hauptfokus liegt jedoch auf Gläsern und Nadeln, wie ich bereits erwähnt habe. Wenn du Drähte und Tassen haben möchtest, könnten wir vielleicht eine Aufteilung finden, bei der du diese Gegenstände bekommst und ich die Gläser und Nadeln. Was meinst du? 
	  } 
	   } 
	   } 
	  \\ 
 

    \theutterance \stepcounter{utterance}  

    & & & \multicolumn{2}{p{0.3\linewidth}}{\cellcolor[rgb]{0.95,0.95,0.95}{%
	\makecell[{{p{\linewidth}}}]{% 
	  \tt {\tiny [GM$|$GM]}  
	 continue 
	  } 
	   } 
	   } 
	 & & \\ 
 

    \theutterance \stepcounter{utterance}  

    & & & \multicolumn{2}{p{0.3\linewidth}}{\cellcolor[rgb]{0.95,0.95,0.95}{%
	\makecell[{{p{\linewidth}}}]{% 
	  \tt {\tiny [GM$|$GM]}  
	 Das ist eine gute Idee, unsere Interessen auszutauschen. Ich muss zugeben, dass ich nicht so sehr an Drähten interessiert bin, aber ich denke, Tassen könnten nützlich sein. Mein Hauptfokus liegt jedoch auf Gläsern und Nadeln, wie ich bereits erwähnt habe. Wenn du Drähte und Tassen haben möchtest, könnten wir vielleicht eine Aufteilung finden, bei der du diese Gegenstände bekommst und ich die Gläser und Nadeln. Was meinst du? 
	  } 
	   } 
	   } 
	 & & \\ 
 

    \theutterance \stepcounter{utterance}  

    & & \multicolumn{4}{p{0.6\linewidth}}{\cellcolor[rgb]{0.9,0.9,0.9}{%
	\makecell[{{p{\linewidth}}}]{% 
	  \tt {\tiny [A$\langle$GM]}  
	 Das ist eine gute Idee, unsere Interessen auszutauschen. Ich muss zugeben, dass ich nicht so sehr an Drähten interessiert bin, aber ich denke, Tassen könnten nützlich sein. Mein Hauptfokus liegt jedoch auf Gläsern und Nadeln, wie ich bereits erwähnt habe. Wenn du Drähte und Tassen haben möchtest, könnten wir vielleicht eine Aufteilung finden, bei der du diese Gegenstände bekommst und ich die Gläser und Nadeln. Was meinst du? 
	  } 
	   } 
	   } 
	 & \\ 
 

    \theutterance \stepcounter{utterance}  

    & \multicolumn{4}{p{0.6\linewidth}}{\cellcolor[rgb]{0.8,1,0.9}{%
	\makecell[{{p{\linewidth}}}]{% 
	  \tt {\tiny [A$\rangle$GM]}  
	 Das klingt nach einem fairen Vorschlag! Ich denke, es wäre großartig, wenn wir eine Aufteilung finden könnten, bei der ich Drähte und Tassen bekomme und du Gläser und Nadeln. Ich bin bereit, mich auf diesen Vorschlag einzulassen, aber ich möchte nur noch sicherstellen, dass wir beide zufrieden sind. Wie viele Tassen und Gläser würdest du gerne haben? Sollten wir sie gleichmäßig aufteilen oder hast du eine andere Vorstellung? 
	  } 
	   } 
	   } 
	 & & \\ 
 

    \theutterance \stepcounter{utterance}  

    & & & \multicolumn{2}{p{0.3\linewidth}}{\cellcolor[rgb]{0.95,0.95,0.95}{%
	\makecell[{{p{\linewidth}}}]{% 
	  \tt {\tiny [GM$|$GM]}  
	 continue 
	  } 
	   } 
	   } 
	 & & \\ 
 

    \theutterance \stepcounter{utterance}  

    & & & \multicolumn{2}{p{0.3\linewidth}}{\cellcolor[rgb]{0.95,0.95,0.95}{%
	\makecell[{{p{\linewidth}}}]{% 
	  \tt {\tiny [GM$|$GM]}  
	 Das klingt nach einem fairen Vorschlag! Ich denke, es wäre großartig, wenn wir eine Aufteilung finden könnten, bei der ich Drähte und Tassen bekomme und du Gläser und Nadeln. Ich bin bereit, mich auf diesen Vorschlag einzulassen, aber ich möchte nur noch sicherstellen, dass wir beide zufrieden sind. Wie viele Tassen und Gläser würdest du gerne haben? Sollten wir sie gleichmäßig aufteilen oder hast du eine andere Vorstellung? 
	  } 
	   } 
	   } 
	 & & \\ 
 

    \theutterance \stepcounter{utterance}  

    & & \multicolumn{4}{p{0.6\linewidth}}{\cellcolor[rgb]{0.9,0.9,0.9}{%
	\makecell[{{p{\linewidth}}}]{% 
	  \tt {\tiny [GM$\rangle$B]}  
	 Das klingt nach einem fairen Vorschlag! Ich denke, es wäre großartig, wenn wir eine Aufteilung finden könnten, bei der ich Drähte und Tassen bekomme und du Gläser und Nadeln. Ich bin bereit, mich auf diesen Vorschlag einzulassen, aber ich möchte nur noch sicherstellen, dass wir beide zufrieden sind. Wie viele Tassen und Gläser würdest du gerne haben? Sollten wir sie gleichmäßig aufteilen oder hast du eine andere Vorstellung? 
	  } 
	   } 
	   } 
	 & \\ 
 

    \theutterance \stepcounter{utterance}  

    & & & \multicolumn{4}{p{0.6\linewidth}}{\cellcolor[rgb]{1,0.85,0.72}{%
	\makecell[{{p{\linewidth}}}]{% 
	  \tt {\tiny [GM$\langle$B]}  
	 Ich denke, eine gleichmäßige Aufteilung der Gläser und Nadeln wäre fair, da es ja nur 2 von jedem gibt. Ich würde also gerne 1 Glas und 1 Nadel haben, und du könntest die andere Glas und Nadel bekommen, zusammen mit den Drähten und Tassen. Das würde bedeuten, dass du 2 Drähte, 2 Tassen, 1 Glas und 1 Nadel bekommst, und ich 1 Glas, 1 Nadel, 0 Drähte und 0 Tassen. Das klingt für mich nach einer fairen und angemessenen Aufteilung. Was meinst du? 
	  } 
	   } 
	   } 
	  \\ 
 

    \theutterance \stepcounter{utterance}  

    & & & \multicolumn{2}{p{0.3\linewidth}}{\cellcolor[rgb]{0.95,0.95,0.95}{%
	\makecell[{{p{\linewidth}}}]{% 
	  \tt {\tiny [GM$|$GM]}  
	 continue 
	  } 
	   } 
	   } 
	 & & \\ 
 

    \theutterance \stepcounter{utterance}  

    & & & \multicolumn{2}{p{0.3\linewidth}}{\cellcolor[rgb]{0.95,0.95,0.95}{%
	\makecell[{{p{\linewidth}}}]{% 
	  \tt {\tiny [GM$|$GM]}  
	 Ich denke, eine gleichmäßige Aufteilung der Gläser und Nadeln wäre fair, da es ja nur 2 von jedem gibt. Ich würde also gerne 1 Glas und 1 Nadel haben, und du könntest die andere Glas und Nadel bekommen, zusammen mit den Drähten und Tassen. Das würde bedeuten, dass du 2 Drähte, 2 Tassen, 1 Glas und 1 Nadel bekommst, und ich 1 Glas, 1 Nadel, 0 Drähte und 0 Tassen. Das klingt für mich nach einer fairen und angemessenen Aufteilung. Was meinst du? 
	  } 
	   } 
	   } 
	 & & \\ 
 

    \theutterance \stepcounter{utterance}  

    & & \multicolumn{4}{p{0.6\linewidth}}{\cellcolor[rgb]{0.9,0.9,0.9}{%
	\makecell[{{p{\linewidth}}}]{% 
	  \tt {\tiny [A$\langle$GM]}  
	 Ich denke, eine gleichmäßige Aufteilung der Gläser und Nadeln wäre fair, da es ja nur 2 von jedem gibt. Ich würde also gerne 1 Glas und 1 Nadel haben, und du könntest die andere Glas und Nadel bekommen, zusammen mit den Drähten und Tassen. Das würde bedeuten, dass du 2 Drähte, 2 Tassen, 1 Glas und 1 Nadel bekommst, und ich 1 Glas, 1 Nadel, 0 Drähte und 0 Tassen. Das klingt für mich nach einer fairen und angemessenen Aufteilung. Was meinst du? 
	  } 
	   } 
	   } 
	 & \\ 
 

    \theutterance \stepcounter{utterance}  

    & \multicolumn{4}{p{0.6\linewidth}}{\cellcolor[rgb]{0.8,1,0.9}{%
	\makecell[{{p{\linewidth}}}]{% 
	  \tt {\tiny [A$\rangle$GM]}  
	 Ich denke, du hast eine prima Idee, aber ich muss zugeben, dass ich mich ein bisschen unwohl fühle, wenn ich 2 Tassen und 2 Drähte bekomme, während du keine bekommst. Ich denke, es wäre besser, wenn wir die Tassen und Drähte auch gleichmäßig aufteilen, sodass wir beide 1 Tasse und 1 Draht bekommen. Das würde bedeuten, dass ich 1 Draht, 1 Tasse, 1 Glas und 1 Nadel bekomme, und du 1 Glas, 1 Nadel, 1 Tasse und 1 Draht. Wie gefällt dir das? 
	  } 
	   } 
	   } 
	 & & \\ 
 

    \theutterance \stepcounter{utterance}  

    & & & \multicolumn{2}{p{0.3\linewidth}}{\cellcolor[rgb]{0.95,0.95,0.95}{%
	\makecell[{{p{\linewidth}}}]{% 
	  \tt {\tiny [GM$|$GM]}  
	 continue 
	  } 
	   } 
	   } 
	 & & \\ 
 

    \theutterance \stepcounter{utterance}  

    & & & \multicolumn{2}{p{0.3\linewidth}}{\cellcolor[rgb]{0.95,0.95,0.95}{%
	\makecell[{{p{\linewidth}}}]{% 
	  \tt {\tiny [GM$|$GM]}  
	 Ich denke, du hast eine prima Idee, aber ich muss zugeben, dass ich mich ein bisschen unwohl fühle, wenn ich 2 Tassen und 2 Drähte bekomme, während du keine bekommst. Ich denke, es wäre besser, wenn wir die Tassen und Drähte auch gleichmäßig aufteilen, sodass wir beide 1 Tasse und 1 Draht bekommen. Das würde bedeuten, dass ich 1 Draht, 1 Tasse, 1 Glas und 1 Nadel bekomme, und du 1 Glas, 1 Nadel, 1 Tasse und 1 Draht. Wie gefällt dir das? 
	  } 
	   } 
	   } 
	 & & \\ 
 

    \theutterance \stepcounter{utterance}  

    & & \multicolumn{4}{p{0.6\linewidth}}{\cellcolor[rgb]{0.9,0.9,0.9}{%
	\makecell[{{p{\linewidth}}}]{% 
	  \tt {\tiny [GM$\rangle$B]}  
	 Ich denke, du hast eine prima Idee, aber ich muss zugeben, dass ich mich ein bisschen unwohl fühle, wenn ich 2 Tassen und 2 Drähte bekomme, während du keine bekommst. Ich denke, es wäre besser, wenn wir die Tassen und Drähte auch gleichmäßig aufteilen, sodass wir beide 1 Tasse und 1 Draht bekommen. Das würde bedeuten, dass ich 1 Draht, 1 Tasse, 1 Glas und 1 Nadel bekomme, und du 1 Glas, 1 Nadel, 1 Tasse und 1 Draht. Wie gefällt dir das? 
	  } 
	   } 
	   } 
	 & \\ 
 

    \theutterance \stepcounter{utterance}  

    & & & \multicolumn{4}{p{0.6\linewidth}}{\cellcolor[rgb]{1,0.85,0.72}{%
	\makecell[{{p{\linewidth}}}]{% 
	  \tt {\tiny [GM$\langle$B]}  
	 Das ist ein sehr großzügiger Vorschlag! Ich denke, das ist eine sehr faire und ausgewogene Aufteilung. Ich bin einverstanden, dass wir die Tassen und Drähte gleichmäßig aufteilen, so dass wir beide 1 Tasse und 1 Draht bekommen. Ich denke, das ist ein guter Kompromiss und wir beide bekommen eine angemessene Menge an Gegenständen. Ich bin bereit, diese Aufteilung zu akzeptieren. Sollen wir unsere endgültigen Vorschläge machen? 
	  } 
	   } 
	   } 
	  \\ 
 

    \theutterance \stepcounter{utterance}  

    & & & \multicolumn{2}{p{0.3\linewidth}}{\cellcolor[rgb]{0.95,0.95,0.95}{%
	\makecell[{{p{\linewidth}}}]{% 
	  \tt {\tiny [GM$|$GM]}  
	 continue 
	  } 
	   } 
	   } 
	 & & \\ 
 

    \theutterance \stepcounter{utterance}  

    & & & \multicolumn{2}{p{0.3\linewidth}}{\cellcolor[rgb]{0.95,0.95,0.95}{%
	\makecell[{{p{\linewidth}}}]{% 
	  \tt {\tiny [GM$|$GM]}  
	 Das ist ein sehr großzügiger Vorschlag! Ich denke, das ist eine sehr faire und ausgewogene Aufteilung. Ich bin einverstanden, dass wir die Tassen und Drähte gleichmäßig aufteilen, so dass wir beide 1 Tasse und 1 Draht bekommen. Ich denke, das ist ein guter Kompromiss und wir beide bekommen eine angemessene Menge an Gegenständen. Ich bin bereit, diese Aufteilung zu akzeptieren. Sollen wir unsere endgültigen Vorschläge machen? 
	  } 
	   } 
	   } 
	 & & \\ 
 

    \theutterance \stepcounter{utterance}  

    & & \multicolumn{4}{p{0.6\linewidth}}{\cellcolor[rgb]{0.9,0.9,0.9}{%
	\makecell[{{p{\linewidth}}}]{% 
	  \tt {\tiny [A$\langle$GM]}  
	 Das ist ein sehr großzügiger Vorschlag! Ich denke, das ist eine sehr faire und ausgewogene Aufteilung. Ich bin einverstanden, dass wir die Tassen und Drähte gleichmäßig aufteilen, so dass wir beide 1 Tasse und 1 Draht bekommen. Ich denke, das ist ein guter Kompromiss und wir beide bekommen eine angemessene Menge an Gegenständen. Ich bin bereit, diese Aufteilung zu akzeptieren. Sollen wir unsere endgültigen Vorschläge machen? 
	  } 
	   } 
	   } 
	 & \\ 
 

    \theutterance \stepcounter{utterance}  

    & \multicolumn{4}{p{0.6\linewidth}}{\cellcolor[rgb]{0.8,1,0.9}{%
	\makecell[{{p{\linewidth}}}]{% 
	  \tt {\tiny [A$\rangle$GM]}  
	 Ja, ich denke, wir haben eine gute Vereinbarung erzielt! Ich bin auch bereit, unsere endgültigen Vorschläge zu machen. Ich werde meinen Vorschlag wie folgt machen: [Vorschlag: 1 Draht, 1 Tasse, 1 Glas, 1 Nadel]. Ich hoffe, du machst deinen Vorschlag entsprechend, damit wir beide zufrieden sind. Lass uns hoffen, dass unsere Vorschläge komplementär sind und wir beide eine gute Punktzahl erhalten! 
	  } 
	   } 
	   } 
	 & & \\ 
 

    \theutterance \stepcounter{utterance}  

    & & & \multicolumn{2}{p{0.3\linewidth}}{\cellcolor[rgb]{0.95,0.95,0.95}{%
	\makecell[{{p{\linewidth}}}]{% 
	  \tt {\tiny [GM$|$GM]}  
	 proposal 
	  } 
	   } 
	   } 
	 & & \\ 
 

    \theutterance \stepcounter{utterance}  

    & & & \multicolumn{2}{p{0.3\linewidth}}{\cellcolor[rgb]{0.95,0.95,0.95}{%
	\makecell[{{p{\linewidth}}}]{% 
	  \tt {\tiny [GM$|$GM]}  
	 [1, 1, 1, 1] 
	  } 
	   } 
	   } 
	 & & \\ 
 

    \theutterance \stepcounter{utterance}  

    & & \multicolumn{4}{p{0.6\linewidth}}{\cellcolor[rgb]{0.9,0.9,0.9}{%
	\makecell[{{p{\linewidth}}}]{% 
	  \tt {\tiny [GM$\rangle$B]}  
	 Der andere Spieler hat einen geheimen Vorschlag gemacht. Jetzt bist du an der Reihe, einen eigenen Vorschlag zu machen.\\ \tt  
	  } 
	   } 
	   } 
	 & \\ 
 

    \theutterance \stepcounter{utterance}  

    & & & \multicolumn{4}{p{0.6\linewidth}}{\cellcolor[rgb]{1,0.85,0.72}{%
	\makecell[{{p{\linewidth}}}]{% 
	  \tt {\tiny [GM$\langle$B]}  
	 [Vorschlag: 1 Draht, 1 Tasse, 1 Glas, 1 Nadel] 
	  } 
	   } 
	   } 
	  \\ 
 

    \theutterance \stepcounter{utterance}  

    & & & \multicolumn{2}{p{0.3\linewidth}}{\cellcolor[rgb]{0.95,0.95,0.95}{%
	\makecell[{{p{\linewidth}}}]{% 
	  \tt {\tiny [GM$|$GM]}  
	 proposal 
	  } 
	   } 
	   } 
	 & & \\ 
 

    \theutterance \stepcounter{utterance}  

    & & & \multicolumn{2}{p{0.3\linewidth}}{\cellcolor[rgb]{0.95,0.95,0.95}{%
	\makecell[{{p{\linewidth}}}]{% 
	  \tt {\tiny [GM$|$GM]}  
	 [1, 1, 1, 1] 
	  } 
	   } 
	   } 
	 & & \\ 
 

    \theutterance \stepcounter{utterance}  

    & & & \multicolumn{2}{p{0.3\linewidth}}{\cellcolor[rgb]{0.95,0.95,0.95}{%
	\makecell[{{p{\linewidth}}}]{% 
	  \tt {\tiny [GM$|$GM]}  
	 [[1, 1, 1, 1], [1, 1, 1, 1]] 
	  } 
	   } 
	   } 
	 & & \\ 
 

\end{supertabular}
}

\end{document}
