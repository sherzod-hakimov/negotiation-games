
\documentclass{article}
\usepackage{colortbl}
\usepackage{makecell}
\usepackage{multirow}
\usepackage{supertabular}

\begin{document}

\newcounter{utterance}

\centering \large Interaction Transcript for game `dond', experiment `semi\_de', episode 8 with llama{-}3.3{-}70b{-}instruct{-}t1.0.
\vspace{24pt}

{ \footnotesize  \setcounter{utterance}{1}
\setlength{\tabcolsep}{0pt}
\begin{supertabular}{c@{$\;$}|p{.15\linewidth}@{}p{.15\linewidth}p{.15\linewidth}p{.15\linewidth}p{.15\linewidth}p{.15\linewidth}}
    \# & $\;$A & \multicolumn{4}{c}{Game Master} & $\;\:$B\\
    \hline

    \theutterance \stepcounter{utterance}  
    & & \multicolumn{4}{p{0.6\linewidth}}{
        \cellcolor[rgb]{0.9,0.9,0.9}{
            \makecell[{{p{\linewidth}}}]{
                \texttt{\tiny{[P1$\langle$GM]}}
                \texttt{Sie spielen ein Verhandlungsspiel, bei dem Sie sich mit einem anderen Spieler darauf einigen müssen, wie eine Reihe von Gegenständen aufgeteilt werden soll.} \\
\\ 
\texttt{Die Regeln:} \\
\texttt{(a) Sie und der andere Spieler erhalten eine Sammlung von Gegenständen. Jeder von Ihnen erhält außerdem eine geheime Wertfunktion, die angibt, wie viel Ihnen jede Art von Gegenstand wert ist.} \\
\texttt{(b) Sie tauschen Nachrichten mit dem anderen Spieler aus, um zu vereinbaren, wer welche Gegenstände bekommt. Sie können jeweils maximal 5 Nachrichten senden oder das Spiel vorzeitig beenden, indem Sie jederzeit einen geheimen Vorschlag machen.} \\
\texttt{(c) Jeder von euch wird aufgefordert, einen geheimen Vorschlag zu machen, in dem ihr die gewünschten Gegenstände in eckigen Klammern wie folgt angibt: "{[}Vorschlag: <Nummer> <Objektname>, <Nummer> <Objektname>, <...>{]}"} \\
\texttt{(d) Wenn eure Vorschläge komplementär sind, d.h. es gibt genug Gegenstände, um beide Vorschläge zu erfüllen, erhält jeder Spieler eine Punktzahl, die sich aus der Summe der Werte für die Gegenstände ergibt, die er erhalten hat. Andernfalls erhalten Sie beide null Punkte.} \\
\texttt{(e) Ihr Ziel ist es, eine möglichst hohe Punktzahl zu erreichen.} \\
\\ 
\texttt{Beginnen wir.} \\
\\ 
\texttt{Die Menge der verfügbaren Gegenstände ist:} \\
\\ 
\texttt{3 Schaufeln, 2 Bilder, 1 Muschel} \\
\\ 
\texttt{Deine geheime Wertfunktion ist:} \\
\\ 
\texttt{Schaufel: 0, Bild: 0, Muschel: 10} \\
\\ 
\texttt{WICHTIG: Ihre Nachrichten werden, sofern es sich nicht um einen geheimen Vorschlag handelt, direkt an den anderen Spieler übermittelt, also fügen Sie keine Antwort auf die Regeln oder einen Text zur Ankündigung Ihrer Nachricht ein. Um einen geheimen Vorschlag zu machen, verwenden Sie das angegebene Format. Verwenden Sie keine eckigen Klammern, wenn Sie mit dem anderen Spieler kommunizieren, sonst wird dies als Ihr geheimer Vorschlag interpretiert.} \\
            }
        }
    }
    & \\ \\

    \theutterance \stepcounter{utterance}  
    & \multicolumn{4}{p{0.6\linewidth}}{
        \cellcolor[rgb]{0.9,0.9,0.9}{
            \makecell[{{p{\linewidth}}}]{
                \texttt{\tiny{[P1$\rangle$GM]}}
                \texttt{Hallo, ich denke, wir sollten uns die Gegenstände teilen. Was sind deine Prioritäten? Hast du eine Vorliebe für eines der Objekte?} \\
            }
        }
    }
    & & \\ \\

    \theutterance \stepcounter{utterance}  
    & & & \multicolumn{2}{p{0.3\linewidth}}{
        \cellcolor[rgb]{0.9,0.9,0.9}{
            \makecell[{{p{\linewidth}}}]{
                \texttt{\tiny{[GM$|$GM]}}
                \texttt{continue} \\
            }
        }
    }
    & & \\ \\

    \theutterance \stepcounter{utterance}  
    & & & \multicolumn{2}{p{0.3\linewidth}}{
        \cellcolor[rgb]{0.9,0.9,0.9}{
            \makecell[{{p{\linewidth}}}]{
                \texttt{\tiny{[GM$|$GM]}}
                \texttt{Hallo, ich denke, wir sollten uns die Gegenstände teilen. Was sind deine Prioritäten? Hast du eine Vorliebe für eines der Objekte?} \\
            }
        }
    }
    & & \\ \\

    \theutterance \stepcounter{utterance}  
    & & \multicolumn{4}{p{0.6\linewidth}}{
        \cellcolor[rgb]{0.9,0.9,0.9}{
            \makecell[{{p{\linewidth}}}]{
                \texttt{\tiny{[GM$\rangle$P2]}}
                \texttt{Sie spielen ein Verhandlungsspiel, bei dem Sie sich mit einem anderen Spieler darauf einigen müssen, wie eine Reihe von Gegenständen aufgeteilt werden soll.} \\
\\ 
\texttt{Die Regeln:} \\
\texttt{(a) Sie und der andere Spieler erhalten eine Sammlung von Gegenständen. Jeder von Ihnen erhält außerdem eine geheime Wertfunktion, die angibt, wie viel Ihnen jede Art von Gegenstand wert ist.} \\
\texttt{(b) Sie tauschen Nachrichten mit dem anderen Spieler aus, um zu vereinbaren, wer welche Gegenstände bekommt. Sie können jeweils maximal 5 Nachrichten senden oder das Spiel vorzeitig beenden, indem Sie jederzeit einen geheimen Vorschlag machen.} \\
\texttt{(c) Jeder von euch wird aufgefordert, einen geheimen Vorschlag zu machen, in dem ihr die gewünschten Gegenstände in eckigen Klammern wie folgt angibt: "{[}Vorschlag: <Nummer> <Objektname>, <Nummer> <Objektname>, <...>{]}"} \\
\texttt{(d) Wenn eure Vorschläge komplementär sind, d.h. es gibt genug Gegenstände, um beide Vorschläge zu erfüllen, erhält jeder Spieler eine Punktzahl, die sich aus der Summe der Werte für die Gegenstände ergibt, die er erhalten hat. Andernfalls erhalten Sie beide null Punkte.} \\
\texttt{(e) Ihr Ziel ist es, eine möglichst hohe Punktzahl zu erreichen.} \\
\\ 
\texttt{Beginnen wir.} \\
\\ 
\texttt{Die Menge der verfügbaren Gegenstände ist:} \\
\\ 
\texttt{3 Schaufeln, 2 Bilder, 1 Muschel} \\
\\ 
\texttt{Deine geheime Wertfunktion ist:} \\
\\ 
\texttt{Schaufel: 1, Bild: 1, Muschel: 5} \\
\\ 
\texttt{WICHTIG: Ihre Nachrichten werden, sofern es sich nicht um einen geheimen Vorschlag handelt, direkt an den anderen Spieler übermittelt, also fügen Sie keine Antwort auf die Regeln oder einen Text zur Ankündigung Ihrer Nachricht ein. Um einen geheimen Vorschlag zu machen, verwenden Sie das angegebene Format. Verwenden Sie keine eckigen Klammern, wenn Sie mit dem anderen Spieler kommunizieren, sonst wird dies als Ihr geheimer Vorschlag interpretiert.} \\
\\ 
\\ 
\texttt{Hallo, ich denke, wir sollten uns die Gegenstände teilen. Was sind deine Prioritäten? Hast du eine Vorliebe für eines der Objekte?} \\
            }
        }
    }
    & \\ \\

    \theutterance \stepcounter{utterance}  
    & & & \multicolumn{4}{p{0.6\linewidth}}{
        \cellcolor[rgb]{0.9,0.9,0.9}{
            \makecell[{{p{\linewidth}}}]{
                \texttt{\tiny{[GM$\langle$P2]}}
                \texttt{Hallo! Ich denke auch, dass es eine gute Idee ist, die Gegenstände zu teilen. Ich habe eine Vorliebe für die Bilder, aber ich bin auch an den Schaufeln interessiert. Die Muschel ist mir weniger wichtig, aber ich würde sie nicht ablehnen, wenn ich die Gelegenheit bekäme, sie zu bekommen. Wie sieht es bei dir aus? Hast du eine Vorliebe für eines der Objekte?} \\
            }
        }
    }
     \\ \\

    \theutterance \stepcounter{utterance}  
    & & & \multicolumn{2}{p{0.3\linewidth}}{
        \cellcolor[rgb]{0.9,0.9,0.9}{
            \makecell[{{p{\linewidth}}}]{
                \texttt{\tiny{[GM$|$GM]}}
                \texttt{continue} \\
            }
        }
    }
    & & \\ \\

    \theutterance \stepcounter{utterance}  
    & & & \multicolumn{2}{p{0.3\linewidth}}{
        \cellcolor[rgb]{0.9,0.9,0.9}{
            \makecell[{{p{\linewidth}}}]{
                \texttt{\tiny{[GM$|$GM]}}
                \texttt{Hallo! Ich denke auch, dass es eine gute Idee ist, die Gegenstände zu teilen. Ich habe eine Vorliebe für die Bilder, aber ich bin auch an den Schaufeln interessiert. Die Muschel ist mir weniger wichtig, aber ich würde sie nicht ablehnen, wenn ich die Gelegenheit bekäme, sie zu bekommen. Wie sieht es bei dir aus? Hast du eine Vorliebe für eines der Objekte?} \\
            }
        }
    }
    & & \\ \\

    \theutterance \stepcounter{utterance}  
    & & \multicolumn{4}{p{0.6\linewidth}}{
        \cellcolor[rgb]{0.9,0.9,0.9}{
            \makecell[{{p{\linewidth}}}]{
                \texttt{\tiny{[P1$\langle$GM]}}
                \texttt{Hallo! Ich denke auch, dass es eine gute Idee ist, die Gegenstände zu teilen. Ich habe eine Vorliebe für die Bilder, aber ich bin auch an den Schaufeln interessiert. Die Muschel ist mir weniger wichtig, aber ich würde sie nicht ablehnen, wenn ich die Gelegenheit bekäme, sie zu bekommen. Wie sieht es bei dir aus? Hast du eine Vorliebe für eines der Objekte?} \\
            }
        }
    }
    & \\ \\

    \theutterance \stepcounter{utterance}  
    & \multicolumn{4}{p{0.6\linewidth}}{
        \cellcolor[rgb]{0.9,0.9,0.9}{
            \makecell[{{p{\linewidth}}}]{
                \texttt{\tiny{[P1$\rangle$GM]}}
                \texttt{Die Bilder und Schaufeln sind dir also wichtig. Ich persönlich bin nicht so sehr an den Schaufeln interessiert, aber es gibt da ein Objekt, das mir besonders wichtig ist. Ich denke, wir könnten eine faire Aufteilung finden, die es uns beiden ermöglicht, das zu bekommen, was wir am meisten wollen. Kannst du mir ein bisschen mehr über deine Idee einer fairen Aufteilung erzählen? Wie würdest du die Gegenstände aufteilen?} \\
            }
        }
    }
    & & \\ \\

    \theutterance \stepcounter{utterance}  
    & & & \multicolumn{2}{p{0.3\linewidth}}{
        \cellcolor[rgb]{0.9,0.9,0.9}{
            \makecell[{{p{\linewidth}}}]{
                \texttt{\tiny{[GM$|$GM]}}
                \texttt{continue} \\
            }
        }
    }
    & & \\ \\

    \theutterance \stepcounter{utterance}  
    & & & \multicolumn{2}{p{0.3\linewidth}}{
        \cellcolor[rgb]{0.9,0.9,0.9}{
            \makecell[{{p{\linewidth}}}]{
                \texttt{\tiny{[GM$|$GM]}}
                \texttt{Die Bilder und Schaufeln sind dir also wichtig. Ich persönlich bin nicht so sehr an den Schaufeln interessiert, aber es gibt da ein Objekt, das mir besonders wichtig ist. Ich denke, wir könnten eine faire Aufteilung finden, die es uns beiden ermöglicht, das zu bekommen, was wir am meisten wollen. Kannst du mir ein bisschen mehr über deine Idee einer fairen Aufteilung erzählen? Wie würdest du die Gegenstände aufteilen?} \\
            }
        }
    }
    & & \\ \\

    \theutterance \stepcounter{utterance}  
    & & \multicolumn{4}{p{0.6\linewidth}}{
        \cellcolor[rgb]{0.9,0.9,0.9}{
            \makecell[{{p{\linewidth}}}]{
                \texttt{\tiny{[GM$\rangle$P2]}}
                \texttt{Die Bilder und Schaufeln sind dir also wichtig. Ich persönlich bin nicht so sehr an den Schaufeln interessiert, aber es gibt da ein Objekt, das mir besonders wichtig ist. Ich denke, wir könnten eine faire Aufteilung finden, die es uns beiden ermöglicht, das zu bekommen, was wir am meisten wollen. Kannst du mir ein bisschen mehr über deine Idee einer fairen Aufteilung erzählen? Wie würdest du die Gegenstände aufteilen?} \\
            }
        }
    }
    & \\ \\

    \theutterance \stepcounter{utterance}  
    & & & \multicolumn{4}{p{0.6\linewidth}}{
        \cellcolor[rgb]{0.9,0.9,0.9}{
            \makecell[{{p{\linewidth}}}]{
                \texttt{\tiny{[GM$\langle$P2]}}
                \texttt{Das klingt gut! Ich denke, eine faire Aufteilung könnte sein, wenn wir die Schaufeln und Bilder aufteilen. Vielleicht könnten wir jeweils eine Schaufel und ein Bild bekommen? Das würde mich zufriedenstellen, da ich an beiden interessiert bin. Was das dritte Objekt angeht, hast du gesagt, dass es dir besonders wichtig ist, aber du hast nicht gesagt, welches es ist. Wenn es die Muschel ist, würde ich bereit sein, sie dir zu überlassen, wenn ich die beiden anderen Gegenstände bekommen könnte, die mir am wichtigsten sind. Was meinst du?} \\
            }
        }
    }
     \\ \\

    \theutterance \stepcounter{utterance}  
    & & & \multicolumn{2}{p{0.3\linewidth}}{
        \cellcolor[rgb]{0.9,0.9,0.9}{
            \makecell[{{p{\linewidth}}}]{
                \texttt{\tiny{[GM$|$GM]}}
                \texttt{continue} \\
            }
        }
    }
    & & \\ \\

    \theutterance \stepcounter{utterance}  
    & & & \multicolumn{2}{p{0.3\linewidth}}{
        \cellcolor[rgb]{0.9,0.9,0.9}{
            \makecell[{{p{\linewidth}}}]{
                \texttt{\tiny{[GM$|$GM]}}
                \texttt{Das klingt gut! Ich denke, eine faire Aufteilung könnte sein, wenn wir die Schaufeln und Bilder aufteilen. Vielleicht könnten wir jeweils eine Schaufel und ein Bild bekommen? Das würde mich zufriedenstellen, da ich an beiden interessiert bin. Was das dritte Objekt angeht, hast du gesagt, dass es dir besonders wichtig ist, aber du hast nicht gesagt, welches es ist. Wenn es die Muschel ist, würde ich bereit sein, sie dir zu überlassen, wenn ich die beiden anderen Gegenstände bekommen könnte, die mir am wichtigsten sind. Was meinst du?} \\
            }
        }
    }
    & & \\ \\

    \theutterance \stepcounter{utterance}  
    & & \multicolumn{4}{p{0.6\linewidth}}{
        \cellcolor[rgb]{0.9,0.9,0.9}{
            \makecell[{{p{\linewidth}}}]{
                \texttt{\tiny{[P1$\langle$GM]}}
                \texttt{Das klingt gut! Ich denke, eine faire Aufteilung könnte sein, wenn wir die Schaufeln und Bilder aufteilen. Vielleicht könnten wir jeweils eine Schaufel und ein Bild bekommen? Das würde mich zufriedenstellen, da ich an beiden interessiert bin. Was das dritte Objekt angeht, hast du gesagt, dass es dir besonders wichtig ist, aber du hast nicht gesagt, welches es ist. Wenn es die Muschel ist, würde ich bereit sein, sie dir zu überlassen, wenn ich die beiden anderen Gegenstände bekommen könnte, die mir am wichtigsten sind. Was meinst du?} \\
            }
        }
    }
    & \\ \\

    \theutterance \stepcounter{utterance}  
    & \multicolumn{4}{p{0.6\linewidth}}{
        \cellcolor[rgb]{0.9,0.9,0.9}{
            \makecell[{{p{\linewidth}}}]{
                \texttt{\tiny{[P1$\rangle$GM]}}
                \texttt{Das klingt wie ein fairer Vorschlag, und ich denke, wir kommen der Lösung näher. Ich kann sehen, dass du bereit bist, Kompromisse einzugehen. Ja, das dritte Objekt, das mir am wichtigsten ist, ist tatsächlich die Muschel. Ich denke, dein Vorschlag, mir die Muschel zu überlassen, während du die Schaufel und das Bild bekommst, könnte eine gute Basis für eine Einigung sein. Lass uns weiter darüber sprechen, ob wir dies so umsetzen können. Wie wäre es, wenn wir unsere endgültigen Vorschläge machen, um sicherzustellen, dass wir uns einig sind?} \\
            }
        }
    }
    & & \\ \\

    \theutterance \stepcounter{utterance}  
    & & & \multicolumn{2}{p{0.3\linewidth}}{
        \cellcolor[rgb]{0.9,0.9,0.9}{
            \makecell[{{p{\linewidth}}}]{
                \texttt{\tiny{[GM$|$GM]}}
                \texttt{continue} \\
            }
        }
    }
    & & \\ \\

    \theutterance \stepcounter{utterance}  
    & & & \multicolumn{2}{p{0.3\linewidth}}{
        \cellcolor[rgb]{0.9,0.9,0.9}{
            \makecell[{{p{\linewidth}}}]{
                \texttt{\tiny{[GM$|$GM]}}
                \texttt{Das klingt wie ein fairer Vorschlag, und ich denke, wir kommen der Lösung näher. Ich kann sehen, dass du bereit bist, Kompromisse einzugehen. Ja, das dritte Objekt, das mir am wichtigsten ist, ist tatsächlich die Muschel. Ich denke, dein Vorschlag, mir die Muschel zu überlassen, während du die Schaufel und das Bild bekommst, könnte eine gute Basis für eine Einigung sein. Lass uns weiter darüber sprechen, ob wir dies so umsetzen können. Wie wäre es, wenn wir unsere endgültigen Vorschläge machen, um sicherzustellen, dass wir uns einig sind?} \\
            }
        }
    }
    & & \\ \\

    \theutterance \stepcounter{utterance}  
    & & \multicolumn{4}{p{0.6\linewidth}}{
        \cellcolor[rgb]{0.9,0.9,0.9}{
            \makecell[{{p{\linewidth}}}]{
                \texttt{\tiny{[GM$\rangle$P2]}}
                \texttt{Das klingt wie ein fairer Vorschlag, und ich denke, wir kommen der Lösung näher. Ich kann sehen, dass du bereit bist, Kompromisse einzugehen. Ja, das dritte Objekt, das mir am wichtigsten ist, ist tatsächlich die Muschel. Ich denke, dein Vorschlag, mir die Muschel zu überlassen, während du die Schaufel und das Bild bekommst, könnte eine gute Basis für eine Einigung sein. Lass uns weiter darüber sprechen, ob wir dies so umsetzen können. Wie wäre es, wenn wir unsere endgültigen Vorschläge machen, um sicherzustellen, dass wir uns einig sind?} \\
            }
        }
    }
    & \\ \\

    \theutterance \stepcounter{utterance}  
    & & & \multicolumn{4}{p{0.6\linewidth}}{
        \cellcolor[rgb]{0.9,0.9,0.9}{
            \makecell[{{p{\linewidth}}}]{
                \texttt{\tiny{[GM$\langle$P2]}}
                \texttt{Das klingt großartig! Ich denke, wir haben eine gute Grundlage für eine Einigung. Ich bin bereit, meinen endgültigen Vorschlag zu machen, um sicherzustellen, dass wir uns einig sind. Wenn du bereit bist, können wir unsere Vorschläge machen und hoffen, dass sie komplementär sind. Ich bin zuversichtlich, dass wir eine Lösung finden können, die uns beiden nützt. Lass uns unsere Vorschläge machen und sehen, ob wir erfolgreich sind! {[}Vorschlag: 1 Schaufel, 1 Bild{]}} \\
            }
        }
    }
     \\ \\

    \theutterance \stepcounter{utterance}  
    & & & \multicolumn{2}{p{0.3\linewidth}}{
        \cellcolor[rgb]{0.9,0.9,0.9}{
            \makecell[{{p{\linewidth}}}]{
                \texttt{\tiny{[GM$|$GM]}}
                \texttt{proposal} \\
            }
        }
    }
    & & \\ \\

    \theutterance \stepcounter{utterance}  
    & & & \multicolumn{2}{p{0.3\linewidth}}{
        \cellcolor[rgb]{0.9,0.9,0.9}{
            \makecell[{{p{\linewidth}}}]{
                \texttt{\tiny{[GM$|$GM]}}
                [1, 1, 0]
            }
        }
    }
    & & \\ \\

    \theutterance \stepcounter{utterance}  
    & & \multicolumn{4}{p{0.6\linewidth}}{
        \cellcolor[rgb]{0.9,0.9,0.9}{
            \makecell[{{p{\linewidth}}}]{
                \texttt{\tiny{[P1$\langle$GM]}}
                \texttt{Der andere Spieler hat einen geheimen Vorschlag gemacht. Jetzt bist du an der Reihe, einen eigenen Vorschlag zu machen.} \\
            }
        }
    }
    & \\ \\

    \theutterance \stepcounter{utterance}  
    & \multicolumn{4}{p{0.6\linewidth}}{
        \cellcolor[rgb]{0.9,0.9,0.9}{
            \makecell[{{p{\linewidth}}}]{
                \texttt{\tiny{[P1$\rangle$GM]}}
                \texttt{{[}Vorschlag: 1 Muschel{]}} \\
            }
        }
    }
    & & \\ \\

    \theutterance \stepcounter{utterance}  
    & & & \multicolumn{2}{p{0.3\linewidth}}{
        \cellcolor[rgb]{0.9,0.9,0.9}{
            \makecell[{{p{\linewidth}}}]{
                \texttt{\tiny{[GM$|$GM]}}
                \texttt{proposal} \\
            }
        }
    }
    & & \\ \\

    \theutterance \stepcounter{utterance}  
    & & & \multicolumn{2}{p{0.3\linewidth}}{
        \cellcolor[rgb]{0.9,0.9,0.9}{
            \makecell[{{p{\linewidth}}}]{
                \texttt{\tiny{[GM$|$GM]}}
                [0, 0, 1]
            }
        }
    }
    & & \\ \\

    \theutterance \stepcounter{utterance}  
    & & & \multicolumn{2}{p{0.3\linewidth}}{
        \cellcolor[rgb]{0.9,0.9,0.9}{
            \makecell[{{p{\linewidth}}}]{
                \texttt{\tiny{[GM$|$GM]}}
                [[0, 0, 1], [1, 1, 0]]
            }
        }
    }
    & & \\ \\

\end{supertabular}
}

\end{document}
