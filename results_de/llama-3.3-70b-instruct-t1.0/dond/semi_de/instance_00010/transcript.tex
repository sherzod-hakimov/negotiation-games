
\documentclass{article}
\usepackage{colortbl}
\usepackage{makecell}
\usepackage{multirow}
\usepackage{supertabular}

\begin{document}

\newcounter{utterance}

\twocolumn

{ \footnotesize  \setcounter{utterance}{1}
\setlength{\tabcolsep}{0pt}
\begin{supertabular}{c@{$\;$}|p{.15\linewidth}@{}p{.15\linewidth}p{.15\linewidth}p{.15\linewidth}p{.15\linewidth}p{.15\linewidth}}

    \# & $\;$A & \multicolumn{4}{c}{Game Master} & $\;\:$B\\
    \hline 

    \theutterance \stepcounter{utterance}  

    & & \multicolumn{4}{p{0.6\linewidth}}{\cellcolor[rgb]{0.9,0.9,0.9}{%
	\makecell[{{p{\linewidth}}}]{% 
	  \tt {\tiny [A$\langle$GM]}  
	 Sie spielen ein Verhandlungsspiel, bei dem Sie sich mit einem anderen Spieler darauf einigen müssen, wie eine Reihe von Gegenständen aufgeteilt werden soll.\\ \tt \\ \tt Die Regeln:\\ \tt (a) Sie und der andere Spieler erhalten eine Sammlung von Gegenständen. Jeder von Ihnen erhält außerdem eine geheime Wertfunktion, die angibt, wie viel Ihnen jede Art von Gegenstand wert ist.\\ \tt (b) Sie tauschen Nachrichten mit dem anderen Spieler aus, um zu vereinbaren, wer welche Gegenstände bekommt. Sie können jeweils maximal 5 Nachrichten senden oder das Spiel vorzeitig beenden, indem Sie jederzeit einen geheimen Vorschlag machen.\\ \tt (c) Jeder von euch wird aufgefordert, einen geheimen Vorschlag zu machen, in dem ihr die gewünschten Gegenstände in eckigen Klammern wie folgt angibt: "[Vorschlag: <Nummer> <Objektname>, <Nummer> <Objektname>, <...>]"\\ \tt (d) Wenn eure Vorschläge komplementär sind, d.h. es gibt genug Gegenstände, um beide Vorschläge zu erfüllen, erhält jeder Spieler eine Punktzahl, die sich aus der Summe der Werte für die Gegenstände ergibt, die er erhalten hat. Andernfalls erhalten Sie beide null Punkte.\\ \tt (e) Ihr Ziel ist es, eine möglichst hohe Punktzahl zu erreichen.\\ \tt \\ \tt Beginnen wir.\\ \tt \\ \tt Die Menge der verfügbaren Gegenstände ist:\\ \tt \\ \tt 1 Eimer, 3 Messer, 1 Batterie\\ \tt \\ \tt Deine geheime Wertfunktion ist:\\ \tt \\ \tt Eimer: 2, Messer: 2, Batterie: 2\\ \tt \\ \tt WICHTIG: Ihre Nachrichten werden, sofern es sich nicht um einen geheimen Vorschlag handelt, direkt an den anderen Spieler übermittelt, also fügen Sie keine Antwort auf die Regeln oder einen Text zur Ankündigung Ihrer Nachricht ein. Um einen geheimen Vorschlag zu machen, verwenden Sie das angegebene Format. Verwenden Sie keine eckigen Klammern, wenn Sie mit dem anderen Spieler kommunizieren, sonst wird dies als Ihr geheimer Vorschlag interpretiert.\\ \tt  
	  } 
	   } 
	   } 
	 & \\ 
 

    \theutterance \stepcounter{utterance}  

    & \multicolumn{4}{p{0.6\linewidth}}{\cellcolor[rgb]{0.8,1,0.9}{%
	\makecell[{{p{\linewidth}}}]{% 
	  \tt {\tiny [A$\rangle$GM]}  
	 Ich frage mich, was du über die Verteilung der Gegenstände denkst. Hast du eine Vorstellung davon, wie wir sie aufteilen sollten? 
	  } 
	   } 
	   } 
	 & & \\ 
 

    \theutterance \stepcounter{utterance}  

    & & & \multicolumn{2}{p{0.3\linewidth}}{\cellcolor[rgb]{0.95,0.95,0.95}{%
	\makecell[{{p{\linewidth}}}]{% 
	  \tt {\tiny [GM$|$GM]}  
	 continue 
	  } 
	   } 
	   } 
	 & & \\ 
 

    \theutterance \stepcounter{utterance}  

    & & & \multicolumn{2}{p{0.3\linewidth}}{\cellcolor[rgb]{0.95,0.95,0.95}{%
	\makecell[{{p{\linewidth}}}]{% 
	  \tt {\tiny [GM$|$GM]}  
	 Ich frage mich, was du über die Verteilung der Gegenstände denkst. Hast du eine Vorstellung davon, wie wir sie aufteilen sollten? 
	  } 
	   } 
	   } 
	 & & \\ 
 

    \theutterance \stepcounter{utterance}  

    & & \multicolumn{4}{p{0.6\linewidth}}{\cellcolor[rgb]{0.9,0.9,0.9}{%
	\makecell[{{p{\linewidth}}}]{% 
	  \tt {\tiny [GM$\rangle$B]}  
	 Sie spielen ein Verhandlungsspiel, bei dem Sie sich mit einem anderen Spieler darauf einigen müssen, wie eine Reihe von Gegenständen aufgeteilt werden soll.\\ \tt \\ \tt Die Regeln:\\ \tt (a) Sie und der andere Spieler erhalten eine Sammlung von Gegenständen. Jeder von Ihnen erhält außerdem eine geheime Wertfunktion, die angibt, wie viel Ihnen jede Art von Gegenstand wert ist.\\ \tt (b) Sie tauschen Nachrichten mit dem anderen Spieler aus, um zu vereinbaren, wer welche Gegenstände bekommt. Sie können jeweils maximal 5 Nachrichten senden oder das Spiel vorzeitig beenden, indem Sie jederzeit einen geheimen Vorschlag machen.\\ \tt (c) Jeder von euch wird aufgefordert, einen geheimen Vorschlag zu machen, in dem ihr die gewünschten Gegenstände in eckigen Klammern wie folgt angibt: "[Vorschlag: <Nummer> <Objektname>, <Nummer> <Objektname>, <...>]"\\ \tt (d) Wenn eure Vorschläge komplementär sind, d.h. es gibt genug Gegenstände, um beide Vorschläge zu erfüllen, erhält jeder Spieler eine Punktzahl, die sich aus der Summe der Werte für die Gegenstände ergibt, die er erhalten hat. Andernfalls erhalten Sie beide null Punkte.\\ \tt (e) Ihr Ziel ist es, eine möglichst hohe Punktzahl zu erreichen.\\ \tt \\ \tt Beginnen wir.\\ \tt \\ \tt Die Menge der verfügbaren Gegenstände ist:\\ \tt \\ \tt 1 Eimer, 3 Messer, 1 Batterie\\ \tt \\ \tt Deine geheime Wertfunktion ist:\\ \tt \\ \tt Eimer: 3, Messer: 2, Batterie: 1\\ \tt \\ \tt WICHTIG: Ihre Nachrichten werden, sofern es sich nicht um einen geheimen Vorschlag handelt, direkt an den anderen Spieler übermittelt, also fügen Sie keine Antwort auf die Regeln oder einen Text zur Ankündigung Ihrer Nachricht ein. Um einen geheimen Vorschlag zu machen, verwenden Sie das angegebene Format. Verwenden Sie keine eckigen Klammern, wenn Sie mit dem anderen Spieler kommunizieren, sonst wird dies als Ihr geheimer Vorschlag interpretiert.\\ \tt \\ \tt \\ \tt Ich frage mich, was du über die Verteilung der Gegenstände denkst. Hast du eine Vorstellung davon, wie wir sie aufteilen sollten? 
	  } 
	   } 
	   } 
	 & \\ 
 

    \theutterance \stepcounter{utterance}  

    & & & \multicolumn{4}{p{0.6\linewidth}}{\cellcolor[rgb]{1,0.85,0.72}{%
	\makecell[{{p{\linewidth}}}]{% 
	  \tt {\tiny [GM$\langle$B]}  
	 Ich denke, es wäre am besten, wenn wir die Gegenstände so aufteilen, dass jeder von uns die Gegenstände bekommt, die für ihn am wertvollsten sind. Ich bin jedoch nicht sicher, was du für die verschiedenen Gegenstände hältst. Hast du eine Vorliebe für bestimmte Gegenstände? 
	  } 
	   } 
	   } 
	  \\ 
 

    \theutterance \stepcounter{utterance}  

    & & & \multicolumn{2}{p{0.3\linewidth}}{\cellcolor[rgb]{0.95,0.95,0.95}{%
	\makecell[{{p{\linewidth}}}]{% 
	  \tt {\tiny [GM$|$GM]}  
	 continue 
	  } 
	   } 
	   } 
	 & & \\ 
 

    \theutterance \stepcounter{utterance}  

    & & & \multicolumn{2}{p{0.3\linewidth}}{\cellcolor[rgb]{0.95,0.95,0.95}{%
	\makecell[{{p{\linewidth}}}]{% 
	  \tt {\tiny [GM$|$GM]}  
	 Ich denke, es wäre am besten, wenn wir die Gegenstände so aufteilen, dass jeder von uns die Gegenstände bekommt, die für ihn am wertvollsten sind. Ich bin jedoch nicht sicher, was du für die verschiedenen Gegenstände hältst. Hast du eine Vorliebe für bestimmte Gegenstände? 
	  } 
	   } 
	   } 
	 & & \\ 
 

    \theutterance \stepcounter{utterance}  

    & & \multicolumn{4}{p{0.6\linewidth}}{\cellcolor[rgb]{0.9,0.9,0.9}{%
	\makecell[{{p{\linewidth}}}]{% 
	  \tt {\tiny [A$\langle$GM]}  
	 Ich denke, es wäre am besten, wenn wir die Gegenstände so aufteilen, dass jeder von uns die Gegenstände bekommt, die für ihn am wertvollsten sind. Ich bin jedoch nicht sicher, was du für die verschiedenen Gegenstände hältst. Hast du eine Vorliebe für bestimmte Gegenstände? 
	  } 
	   } 
	   } 
	 & \\ 
 

    \theutterance \stepcounter{utterance}  

    & \multicolumn{4}{p{0.6\linewidth}}{\cellcolor[rgb]{0.8,1,0.9}{%
	\makecell[{{p{\linewidth}}}]{% 
	  \tt {\tiny [A$\rangle$GM]}  
	 Das klingt nach einem fairen Ansatz. Ich denke, es wäre hilfreich, wenn wir uns über unsere Präferenzen unterhalten, ohne jedoch unsere genauen Werte preiszugeben. Kannst du mir sagen, ob es einen Gegenstand gibt, der für dich besonders wertvoll ist, oder ob du eher an einer ausgewogenen Verteilung der verschiedenen Arten von Gegenständen interessiert bist? 
	  } 
	   } 
	   } 
	 & & \\ 
 

    \theutterance \stepcounter{utterance}  

    & & & \multicolumn{2}{p{0.3\linewidth}}{\cellcolor[rgb]{0.95,0.95,0.95}{%
	\makecell[{{p{\linewidth}}}]{% 
	  \tt {\tiny [GM$|$GM]}  
	 continue 
	  } 
	   } 
	   } 
	 & & \\ 
 

    \theutterance \stepcounter{utterance}  

    & & & \multicolumn{2}{p{0.3\linewidth}}{\cellcolor[rgb]{0.95,0.95,0.95}{%
	\makecell[{{p{\linewidth}}}]{% 
	  \tt {\tiny [GM$|$GM]}  
	 Das klingt nach einem fairen Ansatz. Ich denke, es wäre hilfreich, wenn wir uns über unsere Präferenzen unterhalten, ohne jedoch unsere genauen Werte preiszugeben. Kannst du mir sagen, ob es einen Gegenstand gibt, der für dich besonders wertvoll ist, oder ob du eher an einer ausgewogenen Verteilung der verschiedenen Arten von Gegenständen interessiert bist? 
	  } 
	   } 
	   } 
	 & & \\ 
 

    \theutterance \stepcounter{utterance}  

    & & \multicolumn{4}{p{0.6\linewidth}}{\cellcolor[rgb]{0.9,0.9,0.9}{%
	\makecell[{{p{\linewidth}}}]{% 
	  \tt {\tiny [GM$\rangle$B]}  
	 Das klingt nach einem fairen Ansatz. Ich denke, es wäre hilfreich, wenn wir uns über unsere Präferenzen unterhalten, ohne jedoch unsere genauen Werte preiszugeben. Kannst du mir sagen, ob es einen Gegenstand gibt, der für dich besonders wertvoll ist, oder ob du eher an einer ausgewogenen Verteilung der verschiedenen Arten von Gegenständen interessiert bist? 
	  } 
	   } 
	   } 
	 & \\ 
 

    \theutterance \stepcounter{utterance}  

    & & & \multicolumn{4}{p{0.6\linewidth}}{\cellcolor[rgb]{1,0.85,0.72}{%
	\makecell[{{p{\linewidth}}}]{% 
	  \tt {\tiny [GM$\langle$B]}  
	 Ich denke, ich könnte mit einem ausgewogenen Ansatz leben, aber es gibt einen Gegenstand, der für mich ein bisschen mehr Wert hat als die anderen. Ich bin vor allem an der Vielfalt der Gegenstände interessiert, aber wenn ich ehrlich bin, gibt es einen, der mir ein bisschen mehr imponiert. Wie steht es mit dir? Hast du auch einen Favoriten oder geht es dir mehr um die Menge? 
	  } 
	   } 
	   } 
	  \\ 
 

    \theutterance \stepcounter{utterance}  

    & & & \multicolumn{2}{p{0.3\linewidth}}{\cellcolor[rgb]{0.95,0.95,0.95}{%
	\makecell[{{p{\linewidth}}}]{% 
	  \tt {\tiny [GM$|$GM]}  
	 continue 
	  } 
	   } 
	   } 
	 & & \\ 
 

    \theutterance \stepcounter{utterance}  

    & & & \multicolumn{2}{p{0.3\linewidth}}{\cellcolor[rgb]{0.95,0.95,0.95}{%
	\makecell[{{p{\linewidth}}}]{% 
	  \tt {\tiny [GM$|$GM]}  
	 Ich denke, ich könnte mit einem ausgewogenen Ansatz leben, aber es gibt einen Gegenstand, der für mich ein bisschen mehr Wert hat als die anderen. Ich bin vor allem an der Vielfalt der Gegenstände interessiert, aber wenn ich ehrlich bin, gibt es einen, der mir ein bisschen mehr imponiert. Wie steht es mit dir? Hast du auch einen Favoriten oder geht es dir mehr um die Menge? 
	  } 
	   } 
	   } 
	 & & \\ 
 

    \theutterance \stepcounter{utterance}  

    & & \multicolumn{4}{p{0.6\linewidth}}{\cellcolor[rgb]{0.9,0.9,0.9}{%
	\makecell[{{p{\linewidth}}}]{% 
	  \tt {\tiny [A$\langle$GM]}  
	 Ich denke, ich könnte mit einem ausgewogenen Ansatz leben, aber es gibt einen Gegenstand, der für mich ein bisschen mehr Wert hat als die anderen. Ich bin vor allem an der Vielfalt der Gegenstände interessiert, aber wenn ich ehrlich bin, gibt es einen, der mir ein bisschen mehr imponiert. Wie steht es mit dir? Hast du auch einen Favoriten oder geht es dir mehr um die Menge? 
	  } 
	   } 
	   } 
	 & \\ 
 

    \theutterance \stepcounter{utterance}  

    & \multicolumn{4}{p{0.6\linewidth}}{\cellcolor[rgb]{0.8,1,0.9}{%
	\makecell[{{p{\linewidth}}}]{% 
	  \tt {\tiny [A$\rangle$GM]}  
	 Ich denke auch, dass Vielfalt wichtig ist, aber ich muss zugeben, dass ich ein bestimmtes Interesse an einem der Gegenstände habe, ohne dass ich sagen kann, ob es der gleiche ist, der dir am meisten imponiert. Es ist interessant, dass du erwähnt hast, dass du einen Favoriten hast, aber auch Wert auf Vielfalt legst. Glaubst du, dass es eine Möglichkeit gibt, uns beiden eine ausgewogene Auswahl zu geben, ohne dass einer von uns zu sehr verliert? 
	  } 
	   } 
	   } 
	 & & \\ 
 

    \theutterance \stepcounter{utterance}  

    & & & \multicolumn{2}{p{0.3\linewidth}}{\cellcolor[rgb]{0.95,0.95,0.95}{%
	\makecell[{{p{\linewidth}}}]{% 
	  \tt {\tiny [GM$|$GM]}  
	 continue 
	  } 
	   } 
	   } 
	 & & \\ 
 

    \theutterance \stepcounter{utterance}  

    & & & \multicolumn{2}{p{0.3\linewidth}}{\cellcolor[rgb]{0.95,0.95,0.95}{%
	\makecell[{{p{\linewidth}}}]{% 
	  \tt {\tiny [GM$|$GM]}  
	 Ich denke auch, dass Vielfalt wichtig ist, aber ich muss zugeben, dass ich ein bestimmtes Interesse an einem der Gegenstände habe, ohne dass ich sagen kann, ob es der gleiche ist, der dir am meisten imponiert. Es ist interessant, dass du erwähnt hast, dass du einen Favoriten hast, aber auch Wert auf Vielfalt legst. Glaubst du, dass es eine Möglichkeit gibt, uns beiden eine ausgewogene Auswahl zu geben, ohne dass einer von uns zu sehr verliert? 
	  } 
	   } 
	   } 
	 & & \\ 
 

    \theutterance \stepcounter{utterance}  

    & & \multicolumn{4}{p{0.6\linewidth}}{\cellcolor[rgb]{0.9,0.9,0.9}{%
	\makecell[{{p{\linewidth}}}]{% 
	  \tt {\tiny [GM$\rangle$B]}  
	 Ich denke auch, dass Vielfalt wichtig ist, aber ich muss zugeben, dass ich ein bestimmtes Interesse an einem der Gegenstände habe, ohne dass ich sagen kann, ob es der gleiche ist, der dir am meisten imponiert. Es ist interessant, dass du erwähnt hast, dass du einen Favoriten hast, aber auch Wert auf Vielfalt legst. Glaubst du, dass es eine Möglichkeit gibt, uns beiden eine ausgewogene Auswahl zu geben, ohne dass einer von uns zu sehr verliert? 
	  } 
	   } 
	   } 
	 & \\ 
 

    \theutterance \stepcounter{utterance}  

    & & & \multicolumn{4}{p{0.6\linewidth}}{\cellcolor[rgb]{1,0.85,0.72}{%
	\makecell[{{p{\linewidth}}}]{% 
	  \tt {\tiny [GM$\langle$B]}  
	 Das ist eine gute Frage! Ich denke, es gibt definitiv Möglichkeiten, eine ausgewogene Verteilung zu finden, die beiden Seiten gerecht wird. Eine Idee könnte sein, dass wir uns einen der Gegenstände teilen, der für beide von uns wichtig ist, und dann die anderen Gegenstände so aufteilen, dass jeder von uns eine gute Balance erhält. Oder vielleicht könnten wir auch einen Gegenstand finden, der für einen von uns weniger wichtig ist, und den anderen Gegenständen mehr Gewicht geben. Ich bin offen für Vorschläge, wie wir das erreichen könnten. Hast du maybe eine Vorstellung davon, wie wir mit den Messern umgehen sollten, da es davon ja mehrere gibt? 
	  } 
	   } 
	   } 
	  \\ 
 

    \theutterance \stepcounter{utterance}  

    & & & \multicolumn{2}{p{0.3\linewidth}}{\cellcolor[rgb]{0.95,0.95,0.95}{%
	\makecell[{{p{\linewidth}}}]{% 
	  \tt {\tiny [GM$|$GM]}  
	 continue 
	  } 
	   } 
	   } 
	 & & \\ 
 

    \theutterance \stepcounter{utterance}  

    & & & \multicolumn{2}{p{0.3\linewidth}}{\cellcolor[rgb]{0.95,0.95,0.95}{%
	\makecell[{{p{\linewidth}}}]{% 
	  \tt {\tiny [GM$|$GM]}  
	 Das ist eine gute Frage! Ich denke, es gibt definitiv Möglichkeiten, eine ausgewogene Verteilung zu finden, die beiden Seiten gerecht wird. Eine Idee könnte sein, dass wir uns einen der Gegenstände teilen, der für beide von uns wichtig ist, und dann die anderen Gegenstände so aufteilen, dass jeder von uns eine gute Balance erhält. Oder vielleicht könnten wir auch einen Gegenstand finden, der für einen von uns weniger wichtig ist, und den anderen Gegenständen mehr Gewicht geben. Ich bin offen für Vorschläge, wie wir das erreichen könnten. Hast du maybe eine Vorstellung davon, wie wir mit den Messern umgehen sollten, da es davon ja mehrere gibt? 
	  } 
	   } 
	   } 
	 & & \\ 
 

    \theutterance \stepcounter{utterance}  

    & & \multicolumn{4}{p{0.6\linewidth}}{\cellcolor[rgb]{0.9,0.9,0.9}{%
	\makecell[{{p{\linewidth}}}]{% 
	  \tt {\tiny [A$\langle$GM]}  
	 Das ist eine gute Frage! Ich denke, es gibt definitiv Möglichkeiten, eine ausgewogene Verteilung zu finden, die beiden Seiten gerecht wird. Eine Idee könnte sein, dass wir uns einen der Gegenstände teilen, der für beide von uns wichtig ist, und dann die anderen Gegenstände so aufteilen, dass jeder von uns eine gute Balance erhält. Oder vielleicht könnten wir auch einen Gegenstand finden, der für einen von uns weniger wichtig ist, und den anderen Gegenständen mehr Gewicht geben. Ich bin offen für Vorschläge, wie wir das erreichen könnten. Hast du maybe eine Vorstellung davon, wie wir mit den Messern umgehen sollten, da es davon ja mehrere gibt? 
	  } 
	   } 
	   } 
	 & \\ 
 

    \theutterance \stepcounter{utterance}  

    & \multicolumn{4}{p{0.6\linewidth}}{\cellcolor[rgb]{0.8,1,0.9}{%
	\makecell[{{p{\linewidth}}}]{% 
	  \tt {\tiny [A$\rangle$GM]}  
	 Die Messer sind tatsächlich ein interessanter Punkt, da es davon mehrere gibt. Eine Möglichkeit wäre, dass einer von uns die Mehrheit der Messer bekommt, während der andere vielleicht den Eimer oder die Batterie erhält. Oder wir könnten die Messer auch gleichmäßig aufteilen, je nachdem, wie wichtig sie für jeden von uns sind. Ich denke, es wäre hilfreich, wenn wir uns über die Batterie und den Eimer unterhalten, da es davon nur jeweils einen gibt. Hast du eine starke Vorliebe für einen dieser beiden Gegenstände oder könntest du perhaps mit einem Tauschgeschäft leben, bei dem du einen dieser Gegenstände erhältst, wenn ich die Mehrheit der Messer bekomme? 
	  } 
	   } 
	   } 
	 & & \\ 
 

    \theutterance \stepcounter{utterance}  

    & & & \multicolumn{2}{p{0.3\linewidth}}{\cellcolor[rgb]{0.95,0.95,0.95}{%
	\makecell[{{p{\linewidth}}}]{% 
	  \tt {\tiny [GM$|$GM]}  
	 continue 
	  } 
	   } 
	   } 
	 & & \\ 
 

    \theutterance \stepcounter{utterance}  

    & & & \multicolumn{2}{p{0.3\linewidth}}{\cellcolor[rgb]{0.95,0.95,0.95}{%
	\makecell[{{p{\linewidth}}}]{% 
	  \tt {\tiny [GM$|$GM]}  
	 Die Messer sind tatsächlich ein interessanter Punkt, da es davon mehrere gibt. Eine Möglichkeit wäre, dass einer von uns die Mehrheit der Messer bekommt, während der andere vielleicht den Eimer oder die Batterie erhält. Oder wir könnten die Messer auch gleichmäßig aufteilen, je nachdem, wie wichtig sie für jeden von uns sind. Ich denke, es wäre hilfreich, wenn wir uns über die Batterie und den Eimer unterhalten, da es davon nur jeweils einen gibt. Hast du eine starke Vorliebe für einen dieser beiden Gegenstände oder könntest du perhaps mit einem Tauschgeschäft leben, bei dem du einen dieser Gegenstände erhältst, wenn ich die Mehrheit der Messer bekomme? 
	  } 
	   } 
	   } 
	 & & \\ 
 

    \theutterance \stepcounter{utterance}  

    & & \multicolumn{4}{p{0.6\linewidth}}{\cellcolor[rgb]{0.9,0.9,0.9}{%
	\makecell[{{p{\linewidth}}}]{% 
	  \tt {\tiny [GM$\rangle$B]}  
	 Die Messer sind tatsächlich ein interessanter Punkt, da es davon mehrere gibt. Eine Möglichkeit wäre, dass einer von uns die Mehrheit der Messer bekommt, während der andere vielleicht den Eimer oder die Batterie erhält. Oder wir könnten die Messer auch gleichmäßig aufteilen, je nachdem, wie wichtig sie für jeden von uns sind. Ich denke, es wäre hilfreich, wenn wir uns über die Batterie und den Eimer unterhalten, da es davon nur jeweils einen gibt. Hast du eine starke Vorliebe für einen dieser beiden Gegenstände oder könntest du perhaps mit einem Tauschgeschäft leben, bei dem du einen dieser Gegenstände erhältst, wenn ich die Mehrheit der Messer bekomme? 
	  } 
	   } 
	   } 
	 & \\ 
 

    \theutterance \stepcounter{utterance}  

    & & & \multicolumn{4}{p{0.6\linewidth}}{\cellcolor[rgb]{1,0.85,0.72}{%
	\makecell[{{p{\linewidth}}}]{% 
	  \tt {\tiny [GM$\langle$B]}  
	 Die Batterie und der Eimer sind beide interessante Punkte. Ich denke, ich könnte mit einem Tauschgeschäft leben, bei dem ich einen dieser Gegenstände erhalte, wenn du die Mehrheit der Messer bekommst. Ich bin jedoch auch offen für andere Vorschläge. Die Batterie ist für mich nicht so wichtig wie der Eimer, aber ich denke, es hängt auch davon ab, wie viele Messer du bekommst und wie viele ich bekomme. Wenn du die Mehrheit der Messer bekommst, würde ich mich perhaps mehr für den Eimer interessieren. Aber ich bin auch bereit, Kompromisse zu finden. Wie wichtig ist dir die Batterie im Vergleich zu den Messern? 
	  } 
	   } 
	   } 
	  \\ 
 

    \theutterance \stepcounter{utterance}  

    & & & \multicolumn{2}{p{0.3\linewidth}}{\cellcolor[rgb]{0.95,0.95,0.95}{%
	\makecell[{{p{\linewidth}}}]{% 
	  \tt {\tiny [GM$|$GM]}  
	 continue 
	  } 
	   } 
	   } 
	 & & \\ 
 

    \theutterance \stepcounter{utterance}  

    & & & \multicolumn{2}{p{0.3\linewidth}}{\cellcolor[rgb]{0.95,0.95,0.95}{%
	\makecell[{{p{\linewidth}}}]{% 
	  \tt {\tiny [GM$|$GM]}  
	 Die Batterie und der Eimer sind beide interessante Punkte. Ich denke, ich könnte mit einem Tauschgeschäft leben, bei dem ich einen dieser Gegenstände erhalte, wenn du die Mehrheit der Messer bekommst. Ich bin jedoch auch offen für andere Vorschläge. Die Batterie ist für mich nicht so wichtig wie der Eimer, aber ich denke, es hängt auch davon ab, wie viele Messer du bekommst und wie viele ich bekomme. Wenn du die Mehrheit der Messer bekommst, würde ich mich perhaps mehr für den Eimer interessieren. Aber ich bin auch bereit, Kompromisse zu finden. Wie wichtig ist dir die Batterie im Vergleich zu den Messern? 
	  } 
	   } 
	   } 
	 & & \\ 
 

    \theutterance \stepcounter{utterance}  

    & & \multicolumn{4}{p{0.6\linewidth}}{\cellcolor[rgb]{0.9,0.9,0.9}{%
	\makecell[{{p{\linewidth}}}]{% 
	  \tt {\tiny [A$\langle$GM]}  
	 Die Batterie und der Eimer sind beide interessante Punkte. Ich denke, ich könnte mit einem Tauschgeschäft leben, bei dem ich einen dieser Gegenstände erhalte, wenn du die Mehrheit der Messer bekommst. Ich bin jedoch auch offen für andere Vorschläge. Die Batterie ist für mich nicht so wichtig wie der Eimer, aber ich denke, es hängt auch davon ab, wie viele Messer du bekommst und wie viele ich bekomme. Wenn du die Mehrheit der Messer bekommst, würde ich mich perhaps mehr für den Eimer interessieren. Aber ich bin auch bereit, Kompromisse zu finden. Wie wichtig ist dir die Batterie im Vergleich zu den Messern? 
	  } 
	   } 
	   } 
	 & \\ 
 

    \theutterance \stepcounter{utterance}  

    & \multicolumn{4}{p{0.6\linewidth}}{\cellcolor[rgb]{0.8,1,0.9}{%
	\makecell[{{p{\linewidth}}}]{% 
	  \tt {\tiny [A$\rangle$GM]}  
	 Das ist ein guter Punkt. Ich denke, ich könnte mit einem Tauschgeschäft leben, bei dem ich die Mehrheit der Messer bekomme und du den Eimer erhältst. Die Batterie ist für mich nicht so wichtig wie die Messer, aber ich denke, sie könnte ein guter Ausgleich für den Tausch sein. Wenn ich die Mehrheit der Messer bekomme, wäre ich perhaps bereit, dir den Eimer und die Batterie zu überlassen, oder wir könnten auch einen anderen Kompromiss finden. Ich denke, es ist wichtig, dass wir beide eine faire Chance haben, unsere Ziele zu erreichen. Was meinst du, wenn wir einen Vorschlag machen, der darauf basiert, dass ich zwei Messer bekomme und du den Eimer und die Batterie erhältst? Oder sollten wir Perhaps einen anderen Ansatz wählen? [Vorschlag: 2 Messer, 0 Eimer, 0 Batterie] 
	  } 
	   } 
	   } 
	 & & \\ 
 

    \theutterance \stepcounter{utterance}  

    & & & \multicolumn{2}{p{0.3\linewidth}}{\cellcolor[rgb]{0.95,0.95,0.95}{%
	\makecell[{{p{\linewidth}}}]{% 
	  \tt {\tiny [GM$|$GM]}  
	 proposal 
	  } 
	   } 
	   } 
	 & & \\ 
 

    \theutterance \stepcounter{utterance}  

    & & & \multicolumn{2}{p{0.3\linewidth}}{\cellcolor[rgb]{0.95,0.95,0.95}{%
	\makecell[{{p{\linewidth}}}]{% 
	  \tt {\tiny [GM$|$GM]}  
	 [0, 2, 0] 
	  } 
	   } 
	   } 
	 & & \\ 
 

    \theutterance \stepcounter{utterance}  

    & & \multicolumn{4}{p{0.6\linewidth}}{\cellcolor[rgb]{0.9,0.9,0.9}{%
	\makecell[{{p{\linewidth}}}]{% 
	  \tt {\tiny [GM$\rangle$B]}  
	 Der andere Spieler hat einen geheimen Vorschlag gemacht. Jetzt bist du an der Reihe, einen eigenen Vorschlag zu machen.\\ \tt  
	  } 
	   } 
	   } 
	 & \\ 
 

    \theutterance \stepcounter{utterance}  

    & & & \multicolumn{4}{p{0.6\linewidth}}{\cellcolor[rgb]{1,0.85,0.72}{%
	\makecell[{{p{\linewidth}}}]{% 
	  \tt {\tiny [GM$\langle$B]}  
	 [Vorschlag: 1 Eimer, 1 Messer] 
	  } 
	   } 
	   } 
	  \\ 
 

    \theutterance \stepcounter{utterance}  

    & & & \multicolumn{2}{p{0.3\linewidth}}{\cellcolor[rgb]{0.95,0.95,0.95}{%
	\makecell[{{p{\linewidth}}}]{% 
	  \tt {\tiny [GM$|$GM]}  
	 proposal 
	  } 
	   } 
	   } 
	 & & \\ 
 

    \theutterance \stepcounter{utterance}  

    & & & \multicolumn{2}{p{0.3\linewidth}}{\cellcolor[rgb]{0.95,0.95,0.95}{%
	\makecell[{{p{\linewidth}}}]{% 
	  \tt {\tiny [GM$|$GM]}  
	 [1, 1, 0] 
	  } 
	   } 
	   } 
	 & & \\ 
 

    \theutterance \stepcounter{utterance}  

    & & & \multicolumn{2}{p{0.3\linewidth}}{\cellcolor[rgb]{0.95,0.95,0.95}{%
	\makecell[{{p{\linewidth}}}]{% 
	  \tt {\tiny [GM$|$GM]}  
	 [[0, 2, 0], [1, 1, 0]] 
	  } 
	   } 
	   } 
	 & & \\ 
 

\end{supertabular}
}

\end{document}
