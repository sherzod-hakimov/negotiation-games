
\documentclass{article}
\usepackage{colortbl}
\usepackage{makecell}
\usepackage{multirow}
\usepackage{supertabular}

\begin{document}

\newcounter{utterance}

\twocolumn

{ \footnotesize  \setcounter{utterance}{1}
\setlength{\tabcolsep}{0pt}
\begin{supertabular}{c@{$\;$}|p{.15\linewidth}@{}p{.15\linewidth}p{.15\linewidth}p{.15\linewidth}p{.15\linewidth}p{.15\linewidth}}

    \# & $\;$A & \multicolumn{4}{c}{Game Master} & $\;\:$B\\
    \hline 

    \theutterance \stepcounter{utterance}  

    & & \multicolumn{4}{p{0.6\linewidth}}{\cellcolor[rgb]{0.9,0.9,0.9}{%
	\makecell[{{p{\linewidth}}}]{% 
	  \tt {\tiny [A$\langle$GM]}  
	 Sie spielen ein Verhandlungsspiel, bei dem Sie sich mit einem anderen Spieler darauf einigen müssen, wie eine Reihe von Gegenständen aufgeteilt werden soll.\\ \tt \\ \tt Die Regeln:\\ \tt (a) Sie und der andere Spieler erhalten eine Sammlung von Gegenständen. Jeder von Ihnen erhält außerdem eine geheime Wertfunktion, die angibt, wie viel Ihnen jede Art von Gegenstand wert ist.\\ \tt (b) Sie tauschen Nachrichten mit dem anderen Spieler aus, um zu vereinbaren, wer welche Gegenstände bekommt. Sie können jeweils maximal 5 Nachrichten senden oder das Spiel vorzeitig beenden, indem Sie jederzeit einen geheimen Vorschlag machen.\\ \tt (c) Jeder von euch wird aufgefordert, einen geheimen Vorschlag zu machen, in dem ihr die gewünschten Gegenstände in eckigen Klammern wie folgt angibt: "[Vorschlag: <Nummer> <Objektname>, <Nummer> <Objektname>, <...>]"\\ \tt (d) Wenn eure Vorschläge komplementär sind, d.h. es gibt genug Gegenstände, um beide Vorschläge zu erfüllen, erhält jeder Spieler eine Punktzahl, die sich aus der Summe der Werte für die Gegenstände ergibt, die er erhalten hat. Andernfalls erhalten Sie beide null Punkte.\\ \tt (e) Ihr Ziel ist es, eine möglichst hohe Punktzahl zu erreichen.\\ \tt \\ \tt Beginnen wir.\\ \tt \\ \tt Die Menge der verfügbaren Gegenstände ist:\\ \tt \\ \tt 2 Fäden, 3 Bolzen, 1 Bild\\ \tt \\ \tt Deine geheime Wertfunktion ist:\\ \tt \\ \tt Faden: 0, Bolzen: 1, Bild: 7\\ \tt \\ \tt WICHTIG: Ihre Nachrichten werden, sofern es sich nicht um einen geheimen Vorschlag handelt, direkt an den anderen Spieler übermittelt, also fügen Sie keine Antwort auf die Regeln oder einen Text zur Ankündigung Ihrer Nachricht ein. Um einen geheimen Vorschlag zu machen, verwenden Sie das angegebene Format. Verwenden Sie keine eckigen Klammern, wenn Sie mit dem anderen Spieler kommunizieren, sonst wird dies als Ihr geheimer Vorschlag interpretiert.\\ \tt  
	  } 
	   } 
	   } 
	 & \\ 
 

    \theutterance \stepcounter{utterance}  

    & \multicolumn{4}{p{0.6\linewidth}}{\cellcolor[rgb]{0.8,1,0.9}{%
	\makecell[{{p{\linewidth}}}]{% 
	  \tt {\tiny [A$\rangle$GM]}  
	 Hallo, ich denke, wir sollten versuchen, die Gegenstände so aufzuteilen, dass beide Seiten zufrieden sind. Was halten Sie von den Fäden? Sind Sie daran interessiert oder kann ich die beiden Fäden haben? 
	  } 
	   } 
	   } 
	 & & \\ 
 

    \theutterance \stepcounter{utterance}  

    & & & \multicolumn{2}{p{0.3\linewidth}}{\cellcolor[rgb]{0.95,0.95,0.95}{%
	\makecell[{{p{\linewidth}}}]{% 
	  \tt {\tiny [GM$|$GM]}  
	 continue 
	  } 
	   } 
	   } 
	 & & \\ 
 

    \theutterance \stepcounter{utterance}  

    & & & \multicolumn{2}{p{0.3\linewidth}}{\cellcolor[rgb]{0.95,0.95,0.95}{%
	\makecell[{{p{\linewidth}}}]{% 
	  \tt {\tiny [GM$|$GM]}  
	 Hallo, ich denke, wir sollten versuchen, die Gegenstände so aufzuteilen, dass beide Seiten zufrieden sind. Was halten Sie von den Fäden? Sind Sie daran interessiert oder kann ich die beiden Fäden haben? 
	  } 
	   } 
	   } 
	 & & \\ 
 

    \theutterance \stepcounter{utterance}  

    & & \multicolumn{4}{p{0.6\linewidth}}{\cellcolor[rgb]{0.9,0.9,0.9}{%
	\makecell[{{p{\linewidth}}}]{% 
	  \tt {\tiny [GM$\rangle$B]}  
	 Sie spielen ein Verhandlungsspiel, bei dem Sie sich mit einem anderen Spieler darauf einigen müssen, wie eine Reihe von Gegenständen aufgeteilt werden soll.\\ \tt \\ \tt Die Regeln:\\ \tt (a) Sie und der andere Spieler erhalten eine Sammlung von Gegenständen. Jeder von Ihnen erhält außerdem eine geheime Wertfunktion, die angibt, wie viel Ihnen jede Art von Gegenstand wert ist.\\ \tt (b) Sie tauschen Nachrichten mit dem anderen Spieler aus, um zu vereinbaren, wer welche Gegenstände bekommt. Sie können jeweils maximal 5 Nachrichten senden oder das Spiel vorzeitig beenden, indem Sie jederzeit einen geheimen Vorschlag machen.\\ \tt (c) Jeder von euch wird aufgefordert, einen geheimen Vorschlag zu machen, in dem ihr die gewünschten Gegenstände in eckigen Klammern wie folgt angibt: "[Vorschlag: <Nummer> <Objektname>, <Nummer> <Objektname>, <...>]"\\ \tt (d) Wenn eure Vorschläge komplementär sind, d.h. es gibt genug Gegenstände, um beide Vorschläge zu erfüllen, erhält jeder Spieler eine Punktzahl, die sich aus der Summe der Werte für die Gegenstände ergibt, die er erhalten hat. Andernfalls erhalten Sie beide null Punkte.\\ \tt (e) Ihr Ziel ist es, eine möglichst hohe Punktzahl zu erreichen.\\ \tt \\ \tt Beginnen wir.\\ \tt \\ \tt Die Menge der verfügbaren Gegenstände ist:\\ \tt \\ \tt 2 Fäden, 3 Bolzen, 1 Bild\\ \tt \\ \tt Deine geheime Wertfunktion ist:\\ \tt \\ \tt Faden: 3, Bolzen: 0, Bild: 4\\ \tt \\ \tt WICHTIG: Ihre Nachrichten werden, sofern es sich nicht um einen geheimen Vorschlag handelt, direkt an den anderen Spieler übermittelt, also fügen Sie keine Antwort auf die Regeln oder einen Text zur Ankündigung Ihrer Nachricht ein. Um einen geheimen Vorschlag zu machen, verwenden Sie das angegebene Format. Verwenden Sie keine eckigen Klammern, wenn Sie mit dem anderen Spieler kommunizieren, sonst wird dies als Ihr geheimer Vorschlag interpretiert.\\ \tt \\ \tt \\ \tt Hallo, ich denke, wir sollten versuchen, die Gegenstände so aufzuteilen, dass beide Seiten zufrieden sind. Was halten Sie von den Fäden? Sind Sie daran interessiert oder kann ich die beiden Fäden haben? 
	  } 
	   } 
	   } 
	 & \\ 
 

    \theutterance \stepcounter{utterance}  

    & & & \multicolumn{4}{p{0.6\linewidth}}{\cellcolor[rgb]{1,0.85,0.72}{%
	\makecell[{{p{\linewidth}}}]{% 
	  \tt {\tiny [GM$\langle$B]}  
	 Ich bin auch daran interessiert, eine faire Aufteilung zu finden. Ich bin nicht besonders an den Fäden interessiert, aber ich würde gerne wissen, was Sie mir im Gegenzug anbieten würden, wenn ich Ihnen die Fäden gebe. Haben Sie Interesse an den Bolzen oder dem Bild? 
	  } 
	   } 
	   } 
	  \\ 
 

    \theutterance \stepcounter{utterance}  

    & & & \multicolumn{2}{p{0.3\linewidth}}{\cellcolor[rgb]{0.95,0.95,0.95}{%
	\makecell[{{p{\linewidth}}}]{% 
	  \tt {\tiny [GM$|$GM]}  
	 continue 
	  } 
	   } 
	   } 
	 & & \\ 
 

    \theutterance \stepcounter{utterance}  

    & & & \multicolumn{2}{p{0.3\linewidth}}{\cellcolor[rgb]{0.95,0.95,0.95}{%
	\makecell[{{p{\linewidth}}}]{% 
	  \tt {\tiny [GM$|$GM]}  
	 Ich bin auch daran interessiert, eine faire Aufteilung zu finden. Ich bin nicht besonders an den Fäden interessiert, aber ich würde gerne wissen, was Sie mir im Gegenzug anbieten würden, wenn ich Ihnen die Fäden gebe. Haben Sie Interesse an den Bolzen oder dem Bild? 
	  } 
	   } 
	   } 
	 & & \\ 
 

    \theutterance \stepcounter{utterance}  

    & & \multicolumn{4}{p{0.6\linewidth}}{\cellcolor[rgb]{0.9,0.9,0.9}{%
	\makecell[{{p{\linewidth}}}]{% 
	  \tt {\tiny [A$\langle$GM]}  
	 Ich bin auch daran interessiert, eine faire Aufteilung zu finden. Ich bin nicht besonders an den Fäden interessiert, aber ich würde gerne wissen, was Sie mir im Gegenzug anbieten würden, wenn ich Ihnen die Fäden gebe. Haben Sie Interesse an den Bolzen oder dem Bild? 
	  } 
	   } 
	   } 
	 & \\ 
 

    \theutterance \stepcounter{utterance}  

    & \multicolumn{4}{p{0.6\linewidth}}{\cellcolor[rgb]{0.8,1,0.9}{%
	\makecell[{{p{\linewidth}}}]{% 
	  \tt {\tiny [A$\rangle$GM]}  
	 Das ist großartig, dass wir auf dem gleichen Weg sind! Ich bin tatsächlich sehr an dem Bild interessiert, es scheint mir, als ob es für beide Seiten am wertvollsten sein könnte. Wenn Sie mir die Fäden geben, könnte ich Ihnen im Gegenzug ein oder zwei der Bolzen anbieten. Was halten Sie davon? 
	  } 
	   } 
	   } 
	 & & \\ 
 

    \theutterance \stepcounter{utterance}  

    & & & \multicolumn{2}{p{0.3\linewidth}}{\cellcolor[rgb]{0.95,0.95,0.95}{%
	\makecell[{{p{\linewidth}}}]{% 
	  \tt {\tiny [GM$|$GM]}  
	 continue 
	  } 
	   } 
	   } 
	 & & \\ 
 

    \theutterance \stepcounter{utterance}  

    & & & \multicolumn{2}{p{0.3\linewidth}}{\cellcolor[rgb]{0.95,0.95,0.95}{%
	\makecell[{{p{\linewidth}}}]{% 
	  \tt {\tiny [GM$|$GM]}  
	 Das ist großartig, dass wir auf dem gleichen Weg sind! Ich bin tatsächlich sehr an dem Bild interessiert, es scheint mir, als ob es für beide Seiten am wertvollsten sein könnte. Wenn Sie mir die Fäden geben, könnte ich Ihnen im Gegenzug ein oder zwei der Bolzen anbieten. Was halten Sie davon? 
	  } 
	   } 
	   } 
	 & & \\ 
 

    \theutterance \stepcounter{utterance}  

    & & \multicolumn{4}{p{0.6\linewidth}}{\cellcolor[rgb]{0.9,0.9,0.9}{%
	\makecell[{{p{\linewidth}}}]{% 
	  \tt {\tiny [GM$\rangle$B]}  
	 Das ist großartig, dass wir auf dem gleichen Weg sind! Ich bin tatsächlich sehr an dem Bild interessiert, es scheint mir, als ob es für beide Seiten am wertvollsten sein könnte. Wenn Sie mir die Fäden geben, könnte ich Ihnen im Gegenzug ein oder zwei der Bolzen anbieten. Was halten Sie davon? 
	  } 
	   } 
	   } 
	 & \\ 
 

    \theutterance \stepcounter{utterance}  

    & & & \multicolumn{4}{p{0.6\linewidth}}{\cellcolor[rgb]{1,0.85,0.72}{%
	\makecell[{{p{\linewidth}}}]{% 
	  \tt {\tiny [GM$\langle$B]}  
	 Das Bild ist tatsächlich einzigartig und ich bin auch daran interessiert. Ein Angebot von einem oder zwei Bolzen für die Fäden klingt fair, aber ich muss zugeben, dass ich nicht besonders an den Bolzen interessiert bin. Wenn ich Ihnen die Fäden gebe, würde ich mir eher das Bild wünschen. Könnten wir ein Angebot austauschen, bei dem ich Ihnen die Fäden gebe und Sie mir im Gegenzug das Bild geben? 
	  } 
	   } 
	   } 
	  \\ 
 

    \theutterance \stepcounter{utterance}  

    & & & \multicolumn{2}{p{0.3\linewidth}}{\cellcolor[rgb]{0.95,0.95,0.95}{%
	\makecell[{{p{\linewidth}}}]{% 
	  \tt {\tiny [GM$|$GM]}  
	 continue 
	  } 
	   } 
	   } 
	 & & \\ 
 

    \theutterance \stepcounter{utterance}  

    & & & \multicolumn{2}{p{0.3\linewidth}}{\cellcolor[rgb]{0.95,0.95,0.95}{%
	\makecell[{{p{\linewidth}}}]{% 
	  \tt {\tiny [GM$|$GM]}  
	 Das Bild ist tatsächlich einzigartig und ich bin auch daran interessiert. Ein Angebot von einem oder zwei Bolzen für die Fäden klingt fair, aber ich muss zugeben, dass ich nicht besonders an den Bolzen interessiert bin. Wenn ich Ihnen die Fäden gebe, würde ich mir eher das Bild wünschen. Könnten wir ein Angebot austauschen, bei dem ich Ihnen die Fäden gebe und Sie mir im Gegenzug das Bild geben? 
	  } 
	   } 
	   } 
	 & & \\ 
 

    \theutterance \stepcounter{utterance}  

    & & \multicolumn{4}{p{0.6\linewidth}}{\cellcolor[rgb]{0.9,0.9,0.9}{%
	\makecell[{{p{\linewidth}}}]{% 
	  \tt {\tiny [A$\langle$GM]}  
	 Das Bild ist tatsächlich einzigartig und ich bin auch daran interessiert. Ein Angebot von einem oder zwei Bolzen für die Fäden klingt fair, aber ich muss zugeben, dass ich nicht besonders an den Bolzen interessiert bin. Wenn ich Ihnen die Fäden gebe, würde ich mir eher das Bild wünschen. Könnten wir ein Angebot austauschen, bei dem ich Ihnen die Fäden gebe und Sie mir im Gegenzug das Bild geben? 
	  } 
	   } 
	   } 
	 & \\ 
 

    \theutterance \stepcounter{utterance}  

    & \multicolumn{4}{p{0.6\linewidth}}{\cellcolor[rgb]{0.8,1,0.9}{%
	\makecell[{{p{\linewidth}}}]{% 
	  \tt {\tiny [A$\rangle$GM]}  
	 Das ist ein interessantes Angebot! Ich verstehe, dass Sie das Bild haben möchten, und ich bin bereit, Kompromisse einzugehen. Allerdings wäre es für mich schwierig, Ihnen das Bild zu geben, ohne dass ich im Gegenzug die Bolzen bekomme. Wie wäre es, wenn wir einen Mittelweg finden? Ich könnte Ihnen das Bild geben, wenn Sie mir die Fäden und einen der Bolzen geben. Das würde uns beiden etwas geben, was wir wirklich wollen. Was halten Sie davon? 
	  } 
	   } 
	   } 
	 & & \\ 
 

    \theutterance \stepcounter{utterance}  

    & & & \multicolumn{2}{p{0.3\linewidth}}{\cellcolor[rgb]{0.95,0.95,0.95}{%
	\makecell[{{p{\linewidth}}}]{% 
	  \tt {\tiny [GM$|$GM]}  
	 continue 
	  } 
	   } 
	   } 
	 & & \\ 
 

    \theutterance \stepcounter{utterance}  

    & & & \multicolumn{2}{p{0.3\linewidth}}{\cellcolor[rgb]{0.95,0.95,0.95}{%
	\makecell[{{p{\linewidth}}}]{% 
	  \tt {\tiny [GM$|$GM]}  
	 Das ist ein interessantes Angebot! Ich verstehe, dass Sie das Bild haben möchten, und ich bin bereit, Kompromisse einzugehen. Allerdings wäre es für mich schwierig, Ihnen das Bild zu geben, ohne dass ich im Gegenzug die Bolzen bekomme. Wie wäre es, wenn wir einen Mittelweg finden? Ich könnte Ihnen das Bild geben, wenn Sie mir die Fäden und einen der Bolzen geben. Das würde uns beiden etwas geben, was wir wirklich wollen. Was halten Sie davon? 
	  } 
	   } 
	   } 
	 & & \\ 
 

    \theutterance \stepcounter{utterance}  

    & & \multicolumn{4}{p{0.6\linewidth}}{\cellcolor[rgb]{0.9,0.9,0.9}{%
	\makecell[{{p{\linewidth}}}]{% 
	  \tt {\tiny [GM$\rangle$B]}  
	 Das ist ein interessantes Angebot! Ich verstehe, dass Sie das Bild haben möchten, und ich bin bereit, Kompromisse einzugehen. Allerdings wäre es für mich schwierig, Ihnen das Bild zu geben, ohne dass ich im Gegenzug die Bolzen bekomme. Wie wäre es, wenn wir einen Mittelweg finden? Ich könnte Ihnen das Bild geben, wenn Sie mir die Fäden und einen der Bolzen geben. Das würde uns beiden etwas geben, was wir wirklich wollen. Was halten Sie davon? 
	  } 
	   } 
	   } 
	 & \\ 
 

    \theutterance \stepcounter{utterance}  

    & & & \multicolumn{4}{p{0.6\linewidth}}{\cellcolor[rgb]{1,0.85,0.72}{%
	\makecell[{{p{\linewidth}}}]{% 
	  \tt {\tiny [GM$\langle$B]}  
	 Das ist ein interessanter Vorschlag! Ich denke, wir kommen der Lösung näher. Ich bin bereit, einen Bolzen abzugeben, aber ich muss zugeben, dass ich nicht sicher bin, ob ich bereit bin, alle Fäden abzugeben. Könnten wir einen anderen Mittelweg finden? Wie wäre es, wenn ich Ihnen das Bild gebe und Sie mir im Gegenzug einen der Fäden geben? Das wäre für mich ein guter Kompromiss. 
	  } 
	   } 
	   } 
	  \\ 
 

    \theutterance \stepcounter{utterance}  

    & & & \multicolumn{2}{p{0.3\linewidth}}{\cellcolor[rgb]{0.95,0.95,0.95}{%
	\makecell[{{p{\linewidth}}}]{% 
	  \tt {\tiny [GM$|$GM]}  
	 continue 
	  } 
	   } 
	   } 
	 & & \\ 
 

    \theutterance \stepcounter{utterance}  

    & & & \multicolumn{2}{p{0.3\linewidth}}{\cellcolor[rgb]{0.95,0.95,0.95}{%
	\makecell[{{p{\linewidth}}}]{% 
	  \tt {\tiny [GM$|$GM]}  
	 Das ist ein interessanter Vorschlag! Ich denke, wir kommen der Lösung näher. Ich bin bereit, einen Bolzen abzugeben, aber ich muss zugeben, dass ich nicht sicher bin, ob ich bereit bin, alle Fäden abzugeben. Könnten wir einen anderen Mittelweg finden? Wie wäre es, wenn ich Ihnen das Bild gebe und Sie mir im Gegenzug einen der Fäden geben? Das wäre für mich ein guter Kompromiss. 
	  } 
	   } 
	   } 
	 & & \\ 
 

    \theutterance \stepcounter{utterance}  

    & & \multicolumn{4}{p{0.6\linewidth}}{\cellcolor[rgb]{0.9,0.9,0.9}{%
	\makecell[{{p{\linewidth}}}]{% 
	  \tt {\tiny [A$\langle$GM]}  
	 Das ist ein interessanter Vorschlag! Ich denke, wir kommen der Lösung näher. Ich bin bereit, einen Bolzen abzugeben, aber ich muss zugeben, dass ich nicht sicher bin, ob ich bereit bin, alle Fäden abzugeben. Könnten wir einen anderen Mittelweg finden? Wie wäre es, wenn ich Ihnen das Bild gebe und Sie mir im Gegenzug einen der Fäden geben? Das wäre für mich ein guter Kompromiss. 
	  } 
	   } 
	   } 
	 & \\ 
 

    \theutterance \stepcounter{utterance}  

    & \multicolumn{4}{p{0.6\linewidth}}{\cellcolor[rgb]{0.8,1,0.9}{%
	\makecell[{{p{\linewidth}}}]{% 
	  \tt {\tiny [A$\rangle$GM]}  
	 Das ist ein cleverer Gegenentwurf! Ich muss zugeben, dass ich das Bild sehr gerne haben würde, aber ich bin auch daran interessiert, die Fäden zu behalten. Ein Tausch, bei dem ich Ihnen das Bild gebe und Sie mir im Gegenzug einen der Fäden geben, klingt für mich nicht ganz fairen. Ich denke, ich kann ein besseres Angebot machen. Wie wäre es, wenn ich Ihnen das Bild gebe und Sie mir im Gegenzug alle Bolzen geben? Das wäre für mich ein guter Kompromiss, da ich das Bild sehr schätze, aber ich auch die Bolzen benötige. 
	  } 
	   } 
	   } 
	 & & \\ 
 

    \theutterance \stepcounter{utterance}  

    & & & \multicolumn{2}{p{0.3\linewidth}}{\cellcolor[rgb]{0.95,0.95,0.95}{%
	\makecell[{{p{\linewidth}}}]{% 
	  \tt {\tiny [GM$|$GM]}  
	 continue 
	  } 
	   } 
	   } 
	 & & \\ 
 

    \theutterance \stepcounter{utterance}  

    & & & \multicolumn{2}{p{0.3\linewidth}}{\cellcolor[rgb]{0.95,0.95,0.95}{%
	\makecell[{{p{\linewidth}}}]{% 
	  \tt {\tiny [GM$|$GM]}  
	 Das ist ein cleverer Gegenentwurf! Ich muss zugeben, dass ich das Bild sehr gerne haben würde, aber ich bin auch daran interessiert, die Fäden zu behalten. Ein Tausch, bei dem ich Ihnen das Bild gebe und Sie mir im Gegenzug einen der Fäden geben, klingt für mich nicht ganz fairen. Ich denke, ich kann ein besseres Angebot machen. Wie wäre es, wenn ich Ihnen das Bild gebe und Sie mir im Gegenzug alle Bolzen geben? Das wäre für mich ein guter Kompromiss, da ich das Bild sehr schätze, aber ich auch die Bolzen benötige. 
	  } 
	   } 
	   } 
	 & & \\ 
 

    \theutterance \stepcounter{utterance}  

    & & \multicolumn{4}{p{0.6\linewidth}}{\cellcolor[rgb]{0.9,0.9,0.9}{%
	\makecell[{{p{\linewidth}}}]{% 
	  \tt {\tiny [GM$\rangle$B]}  
	 Das ist ein cleverer Gegenentwurf! Ich muss zugeben, dass ich das Bild sehr gerne haben würde, aber ich bin auch daran interessiert, die Fäden zu behalten. Ein Tausch, bei dem ich Ihnen das Bild gebe und Sie mir im Gegenzug einen der Fäden geben, klingt für mich nicht ganz fairen. Ich denke, ich kann ein besseres Angebot machen. Wie wäre es, wenn ich Ihnen das Bild gebe und Sie mir im Gegenzug alle Bolzen geben? Das wäre für mich ein guter Kompromiss, da ich das Bild sehr schätze, aber ich auch die Bolzen benötige. 
	  } 
	   } 
	   } 
	 & \\ 
 

    \theutterance \stepcounter{utterance}  

    & & & \multicolumn{4}{p{0.6\linewidth}}{\cellcolor[rgb]{1,0.85,0.72}{%
	\makecell[{{p{\linewidth}}}]{% 
	  \tt {\tiny [GM$\langle$B]}  
	 Das ist ein interessantes Angebot! Ich denke, ich könnte damit leben, alle Bolzen abzugeben, wenn ich das Bild bekomme. Die Bolzen sind für mich nicht besonders wertvoll, und das Bild ist einzigartig. Ich denke, das ist ein fairer Tausch. Ich bin einverstanden, Ihnen alle Bolzen zu geben, wenn Sie mir das Bild geben. Wir sollten unsere endgültigen Vorschläge machen, um sicherzustellen, dass wir die beste Lösung finden. Ich denke, wir haben eine gute Grundlage für eine Einigung geschaffen. Sollen wir unsere Vorschläge jetzt machen? 
	  } 
	   } 
	   } 
	  \\ 
 

    \theutterance \stepcounter{utterance}  

    & & & \multicolumn{2}{p{0.3\linewidth}}{\cellcolor[rgb]{0.95,0.95,0.95}{%
	\makecell[{{p{\linewidth}}}]{% 
	  \tt {\tiny [GM$|$GM]}  
	 continue 
	  } 
	   } 
	   } 
	 & & \\ 
 

    \theutterance \stepcounter{utterance}  

    & & & \multicolumn{2}{p{0.3\linewidth}}{\cellcolor[rgb]{0.95,0.95,0.95}{%
	\makecell[{{p{\linewidth}}}]{% 
	  \tt {\tiny [GM$|$GM]}  
	 Das ist ein interessantes Angebot! Ich denke, ich könnte damit leben, alle Bolzen abzugeben, wenn ich das Bild bekomme. Die Bolzen sind für mich nicht besonders wertvoll, und das Bild ist einzigartig. Ich denke, das ist ein fairer Tausch. Ich bin einverstanden, Ihnen alle Bolzen zu geben, wenn Sie mir das Bild geben. Wir sollten unsere endgültigen Vorschläge machen, um sicherzustellen, dass wir die beste Lösung finden. Ich denke, wir haben eine gute Grundlage für eine Einigung geschaffen. Sollen wir unsere Vorschläge jetzt machen? 
	  } 
	   } 
	   } 
	 & & \\ 
 

    \theutterance \stepcounter{utterance}  

    & & \multicolumn{4}{p{0.6\linewidth}}{\cellcolor[rgb]{0.9,0.9,0.9}{%
	\makecell[{{p{\linewidth}}}]{% 
	  \tt {\tiny [A$\langle$GM]}  
	 Das ist ein interessantes Angebot! Ich denke, ich könnte damit leben, alle Bolzen abzugeben, wenn ich das Bild bekomme. Die Bolzen sind für mich nicht besonders wertvoll, und das Bild ist einzigartig. Ich denke, das ist ein fairer Tausch. Ich bin einverstanden, Ihnen alle Bolzen zu geben, wenn Sie mir das Bild geben. Wir sollten unsere endgültigen Vorschläge machen, um sicherzustellen, dass wir die beste Lösung finden. Ich denke, wir haben eine gute Grundlage für eine Einigung geschaffen. Sollen wir unsere Vorschläge jetzt machen? 
	  } 
	   } 
	   } 
	 & \\ 
 

    \theutterance \stepcounter{utterance}  

    & \multicolumn{4}{p{0.6\linewidth}}{\cellcolor[rgb]{0.8,1,0.9}{%
	\makecell[{{p{\linewidth}}}]{% 
	  \tt {\tiny [A$\rangle$GM]}  
	 Ich denke, wir haben eine gute Vereinbarung erzielt! Ich bin zufrieden mit dem Tausch, bei dem ich Ihnen das Bild gebe und Sie mir alle Bolzen geben. Da wir uns auf eine Lösung geeinigt haben, denke ich, es ist an der Zeit, unsere endgültigen Vorschläge zu machen. Ich werde meinen Vorschlag jetzt machen und hoffe, dass er mit Ihrem übereinstimmt.\\ \tt \\ \tt [Vorschlag: 3 Bolzen, 2 Fäden] 
	  } 
	   } 
	   } 
	 & & \\ 
 

    \theutterance \stepcounter{utterance}  

    & & & \multicolumn{2}{p{0.3\linewidth}}{\cellcolor[rgb]{0.95,0.95,0.95}{%
	\makecell[{{p{\linewidth}}}]{% 
	  \tt {\tiny [GM$|$GM]}  
	 proposal 
	  } 
	   } 
	   } 
	 & & \\ 
 

    \theutterance \stepcounter{utterance}  

    & & & \multicolumn{2}{p{0.3\linewidth}}{\cellcolor[rgb]{0.95,0.95,0.95}{%
	\makecell[{{p{\linewidth}}}]{% 
	  \tt {\tiny [GM$|$GM]}  
	 [2, 3, 0] 
	  } 
	   } 
	   } 
	 & & \\ 
 

    \theutterance \stepcounter{utterance}  

    & & \multicolumn{4}{p{0.6\linewidth}}{\cellcolor[rgb]{0.9,0.9,0.9}{%
	\makecell[{{p{\linewidth}}}]{% 
	  \tt {\tiny [GM$\rangle$B]}  
	 Der andere Spieler hat einen geheimen Vorschlag gemacht. Jetzt bist du an der Reihe, einen eigenen Vorschlag zu machen.\\ \tt  
	  } 
	   } 
	   } 
	 & \\ 
 

    \theutterance \stepcounter{utterance}  

    & & & \multicolumn{4}{p{0.6\linewidth}}{\cellcolor[rgb]{1,0.85,0.72}{%
	\makecell[{{p{\linewidth}}}]{% 
	  \tt {\tiny [GM$\langle$B]}  
	 [Vorschlag: 1 Bild, 2 Fäden] 
	  } 
	   } 
	   } 
	  \\ 
 

    \theutterance \stepcounter{utterance}  

    & & & \multicolumn{2}{p{0.3\linewidth}}{\cellcolor[rgb]{0.95,0.95,0.95}{%
	\makecell[{{p{\linewidth}}}]{% 
	  \tt {\tiny [GM$|$GM]}  
	 proposal 
	  } 
	   } 
	   } 
	 & & \\ 
 

    \theutterance \stepcounter{utterance}  

    & & & \multicolumn{2}{p{0.3\linewidth}}{\cellcolor[rgb]{0.95,0.95,0.95}{%
	\makecell[{{p{\linewidth}}}]{% 
	  \tt {\tiny [GM$|$GM]}  
	 [2, 0, 1] 
	  } 
	   } 
	   } 
	 & & \\ 
 

    \theutterance \stepcounter{utterance}  

    & & & \multicolumn{2}{p{0.3\linewidth}}{\cellcolor[rgb]{0.95,0.95,0.95}{%
	\makecell[{{p{\linewidth}}}]{% 
	  \tt {\tiny [GM$|$GM]}  
	 [[2, 3, 0], [2, 0, 1]] 
	  } 
	   } 
	   } 
	 & & \\ 
 

\end{supertabular}
}

\end{document}
