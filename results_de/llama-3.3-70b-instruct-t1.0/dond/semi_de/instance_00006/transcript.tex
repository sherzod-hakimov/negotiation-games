
\documentclass{article}
\usepackage{colortbl}
\usepackage{makecell}
\usepackage{multirow}
\usepackage{supertabular}

\begin{document}

\newcounter{utterance}

\centering \large Interaction Transcript for game `dond', experiment `semi\_de', episode 6 with llama{-}3.3{-}70b{-}instruct{-}t1.0.
\vspace{24pt}

{ \footnotesize  \setcounter{utterance}{1}
\setlength{\tabcolsep}{0pt}
\begin{supertabular}{c@{$\;$}|p{.15\linewidth}@{}p{.15\linewidth}p{.15\linewidth}p{.15\linewidth}p{.15\linewidth}p{.15\linewidth}}
    \# & $\;$A & \multicolumn{4}{c}{Game Master} & $\;\:$B\\
    \hline

    \theutterance \stepcounter{utterance}  
    & & \multicolumn{4}{p{0.6\linewidth}}{
        \cellcolor[rgb]{0.9,0.9,0.9}{
            \makecell[{{p{\linewidth}}}]{
                \texttt{\tiny{[P1$\langle$GM]}}
                \texttt{Sie spielen ein Verhandlungsspiel, bei dem Sie sich mit einem anderen Spieler darauf einigen müssen, wie eine Reihe von Gegenständen aufgeteilt werden soll.} \\
\\ 
\texttt{Die Regeln:} \\
\texttt{(a) Sie und der andere Spieler erhalten eine Sammlung von Gegenständen. Jeder von Ihnen erhält außerdem eine geheime Wertfunktion, die angibt, wie viel Ihnen jede Art von Gegenstand wert ist.} \\
\texttt{(b) Sie tauschen Nachrichten mit dem anderen Spieler aus, um zu vereinbaren, wer welche Gegenstände bekommt. Sie können jeweils maximal 5 Nachrichten senden oder das Spiel vorzeitig beenden, indem Sie jederzeit einen geheimen Vorschlag machen.} \\
\texttt{(c) Jeder von euch wird aufgefordert, einen geheimen Vorschlag zu machen, in dem ihr die gewünschten Gegenstände in eckigen Klammern wie folgt angibt: "{[}Vorschlag: <Nummer> <Objektname>, <Nummer> <Objektname>, <...>{]}"} \\
\texttt{(d) Wenn eure Vorschläge komplementär sind, d.h. es gibt genug Gegenstände, um beide Vorschläge zu erfüllen, erhält jeder Spieler eine Punktzahl, die sich aus der Summe der Werte für die Gegenstände ergibt, die er erhalten hat. Andernfalls erhalten Sie beide null Punkte.} \\
\texttt{(e) Ihr Ziel ist es, eine möglichst hohe Punktzahl zu erreichen.} \\
\\ 
\texttt{Beginnen wir.} \\
\\ 
\texttt{Die Menge der verfügbaren Gegenstände ist:} \\
\\ 
\texttt{3 Äpfel, 1 Seife, 1 Ei} \\
\\ 
\texttt{Deine geheime Wertfunktion ist:} \\
\\ 
\texttt{Apfel: 1, Seife: 3, Ei: 4} \\
\\ 
\texttt{WICHTIG: Ihre Nachrichten werden, sofern es sich nicht um einen geheimen Vorschlag handelt, direkt an den anderen Spieler übermittelt, also fügen Sie keine Antwort auf die Regeln oder einen Text zur Ankündigung Ihrer Nachricht ein. Um einen geheimen Vorschlag zu machen, verwenden Sie das angegebene Format. Verwenden Sie keine eckigen Klammern, wenn Sie mit dem anderen Spieler kommunizieren, sonst wird dies als Ihr geheimer Vorschlag interpretiert.} \\
            }
        }
    }
    & \\ \\

    \theutterance \stepcounter{utterance}  
    & \multicolumn{4}{p{0.6\linewidth}}{
        \cellcolor[rgb]{0.9,0.9,0.9}{
            \makecell[{{p{\linewidth}}}]{
                \texttt{\tiny{[P1$\rangle$GM]}}
                \texttt{Ich denke, wir sollten zuerst die Möglichkeiten besprechen, wie wir die Gegenstände aufteilen können. Gibt es bestimmte Gegenstände, die du bevorzugst oder brauchst?} \\
            }
        }
    }
    & & \\ \\

    \theutterance \stepcounter{utterance}  
    & & & \multicolumn{2}{p{0.3\linewidth}}{
        \cellcolor[rgb]{0.9,0.9,0.9}{
            \makecell[{{p{\linewidth}}}]{
                \texttt{\tiny{[GM$|$GM]}}
                \texttt{continue} \\
            }
        }
    }
    & & \\ \\

    \theutterance \stepcounter{utterance}  
    & & & \multicolumn{2}{p{0.3\linewidth}}{
        \cellcolor[rgb]{0.9,0.9,0.9}{
            \makecell[{{p{\linewidth}}}]{
                \texttt{\tiny{[GM$|$GM]}}
                \texttt{Ich denke, wir sollten zuerst die Möglichkeiten besprechen, wie wir die Gegenstände aufteilen können. Gibt es bestimmte Gegenstände, die du bevorzugst oder brauchst?} \\
            }
        }
    }
    & & \\ \\

    \theutterance \stepcounter{utterance}  
    & & \multicolumn{4}{p{0.6\linewidth}}{
        \cellcolor[rgb]{0.9,0.9,0.9}{
            \makecell[{{p{\linewidth}}}]{
                \texttt{\tiny{[GM$\rangle$P2]}}
                \texttt{Sie spielen ein Verhandlungsspiel, bei dem Sie sich mit einem anderen Spieler darauf einigen müssen, wie eine Reihe von Gegenständen aufgeteilt werden soll.} \\
\\ 
\texttt{Die Regeln:} \\
\texttt{(a) Sie und der andere Spieler erhalten eine Sammlung von Gegenständen. Jeder von Ihnen erhält außerdem eine geheime Wertfunktion, die angibt, wie viel Ihnen jede Art von Gegenstand wert ist.} \\
\texttt{(b) Sie tauschen Nachrichten mit dem anderen Spieler aus, um zu vereinbaren, wer welche Gegenstände bekommt. Sie können jeweils maximal 5 Nachrichten senden oder das Spiel vorzeitig beenden, indem Sie jederzeit einen geheimen Vorschlag machen.} \\
\texttt{(c) Jeder von euch wird aufgefordert, einen geheimen Vorschlag zu machen, in dem ihr die gewünschten Gegenstände in eckigen Klammern wie folgt angibt: "{[}Vorschlag: <Nummer> <Objektname>, <Nummer> <Objektname>, <...>{]}"} \\
\texttt{(d) Wenn eure Vorschläge komplementär sind, d.h. es gibt genug Gegenstände, um beide Vorschläge zu erfüllen, erhält jeder Spieler eine Punktzahl, die sich aus der Summe der Werte für die Gegenstände ergibt, die er erhalten hat. Andernfalls erhalten Sie beide null Punkte.} \\
\texttt{(e) Ihr Ziel ist es, eine möglichst hohe Punktzahl zu erreichen.} \\
\\ 
\texttt{Beginnen wir.} \\
\\ 
\texttt{Die Menge der verfügbaren Gegenstände ist:} \\
\\ 
\texttt{3 Äpfel, 1 Seife, 1 Ei} \\
\\ 
\texttt{Deine geheime Wertfunktion ist:} \\
\\ 
\texttt{Apfel: 0, Seife: 7, Ei: 3} \\
\\ 
\texttt{WICHTIG: Ihre Nachrichten werden, sofern es sich nicht um einen geheimen Vorschlag handelt, direkt an den anderen Spieler übermittelt, also fügen Sie keine Antwort auf die Regeln oder einen Text zur Ankündigung Ihrer Nachricht ein. Um einen geheimen Vorschlag zu machen, verwenden Sie das angegebene Format. Verwenden Sie keine eckigen Klammern, wenn Sie mit dem anderen Spieler kommunizieren, sonst wird dies als Ihr geheimer Vorschlag interpretiert.} \\
\\ 
\\ 
\texttt{Ich denke, wir sollten zuerst die Möglichkeiten besprechen, wie wir die Gegenstände aufteilen können. Gibt es bestimmte Gegenstände, die du bevorzugst oder brauchst?} \\
            }
        }
    }
    & \\ \\

    \theutterance \stepcounter{utterance}  
    & & & \multicolumn{4}{p{0.6\linewidth}}{
        \cellcolor[rgb]{0.9,0.9,0.9}{
            \makecell[{{p{\linewidth}}}]{
                \texttt{\tiny{[GM$\langle$P2]}}
                \texttt{Ich denke, es wäre hilfreich, wenn wir erstmal auflisten, was wir haben: 3 Äpfel, 1 Seife und 1 Ei. Persönlich habe ich kein großes Interesse an den Äpfeln, aber ich könnte mir vorstellen, dass die anderen Gegenstände für mich von Wert sein könnten. Wie siehst du die Verteilung der Seife und des Eis? Gibt es dafür einen besonderen Wert oder eine Vorliebe von deiner Seite?} \\
            }
        }
    }
     \\ \\

    \theutterance \stepcounter{utterance}  
    & & & \multicolumn{2}{p{0.3\linewidth}}{
        \cellcolor[rgb]{0.9,0.9,0.9}{
            \makecell[{{p{\linewidth}}}]{
                \texttt{\tiny{[GM$|$GM]}}
                \texttt{continue} \\
            }
        }
    }
    & & \\ \\

    \theutterance \stepcounter{utterance}  
    & & & \multicolumn{2}{p{0.3\linewidth}}{
        \cellcolor[rgb]{0.9,0.9,0.9}{
            \makecell[{{p{\linewidth}}}]{
                \texttt{\tiny{[GM$|$GM]}}
                \texttt{Ich denke, es wäre hilfreich, wenn wir erstmal auflisten, was wir haben: 3 Äpfel, 1 Seife und 1 Ei. Persönlich habe ich kein großes Interesse an den Äpfeln, aber ich könnte mir vorstellen, dass die anderen Gegenstände für mich von Wert sein könnten. Wie siehst du die Verteilung der Seife und des Eis? Gibt es dafür einen besonderen Wert oder eine Vorliebe von deiner Seite?} \\
            }
        }
    }
    & & \\ \\

    \theutterance \stepcounter{utterance}  
    & & \multicolumn{4}{p{0.6\linewidth}}{
        \cellcolor[rgb]{0.9,0.9,0.9}{
            \makecell[{{p{\linewidth}}}]{
                \texttt{\tiny{[P1$\langle$GM]}}
                \texttt{Ich denke, es wäre hilfreich, wenn wir erstmal auflisten, was wir haben: 3 Äpfel, 1 Seife und 1 Ei. Persönlich habe ich kein großes Interesse an den Äpfeln, aber ich könnte mir vorstellen, dass die anderen Gegenstände für mich von Wert sein könnten. Wie siehst du die Verteilung der Seife und des Eis? Gibt es dafür einen besonderen Wert oder eine Vorliebe von deiner Seite?} \\
            }
        }
    }
    & \\ \\

    \theutterance \stepcounter{utterance}  
    & \multicolumn{4}{p{0.6\linewidth}}{
        \cellcolor[rgb]{0.9,0.9,0.9}{
            \makecell[{{p{\linewidth}}}]{
                \texttt{\tiny{[P1$\rangle$GM]}}
                \texttt{Das ist ein interessanter Ansatz! Ich denke, die Seife und das Ei sind beide ziemlich wertvoll, aber ich bin auch der Meinung, dass es wichtig ist, eine faire Verteilung zu finden. Ich könnte mir vorstellen, dass wir die Äpfel möglicherweise aufteilen, um sicherzustellen, dass beide Spieler etwas bekommen. Was meinst du, wenn wir die Äpfel gleichmäßig aufteilen und dann über die Seife und das Ei sprechen?} \\
            }
        }
    }
    & & \\ \\

    \theutterance \stepcounter{utterance}  
    & & & \multicolumn{2}{p{0.3\linewidth}}{
        \cellcolor[rgb]{0.9,0.9,0.9}{
            \makecell[{{p{\linewidth}}}]{
                \texttt{\tiny{[GM$|$GM]}}
                \texttt{continue} \\
            }
        }
    }
    & & \\ \\

    \theutterance \stepcounter{utterance}  
    & & & \multicolumn{2}{p{0.3\linewidth}}{
        \cellcolor[rgb]{0.9,0.9,0.9}{
            \makecell[{{p{\linewidth}}}]{
                \texttt{\tiny{[GM$|$GM]}}
                \texttt{Das ist ein interessanter Ansatz! Ich denke, die Seife und das Ei sind beide ziemlich wertvoll, aber ich bin auch der Meinung, dass es wichtig ist, eine faire Verteilung zu finden. Ich könnte mir vorstellen, dass wir die Äpfel möglicherweise aufteilen, um sicherzustellen, dass beide Spieler etwas bekommen. Was meinst du, wenn wir die Äpfel gleichmäßig aufteilen und dann über die Seife und das Ei sprechen?} \\
            }
        }
    }
    & & \\ \\

    \theutterance \stepcounter{utterance}  
    & & \multicolumn{4}{p{0.6\linewidth}}{
        \cellcolor[rgb]{0.9,0.9,0.9}{
            \makecell[{{p{\linewidth}}}]{
                \texttt{\tiny{[GM$\rangle$P2]}}
                \texttt{Das ist ein interessanter Ansatz! Ich denke, die Seife und das Ei sind beide ziemlich wertvoll, aber ich bin auch der Meinung, dass es wichtig ist, eine faire Verteilung zu finden. Ich könnte mir vorstellen, dass wir die Äpfel möglicherweise aufteilen, um sicherzustellen, dass beide Spieler etwas bekommen. Was meinst du, wenn wir die Äpfel gleichmäßig aufteilen und dann über die Seife und das Ei sprechen?} \\
            }
        }
    }
    & \\ \\

    \theutterance \stepcounter{utterance}  
    & & & \multicolumn{4}{p{0.6\linewidth}}{
        \cellcolor[rgb]{0.9,0.9,0.9}{
            \makecell[{{p{\linewidth}}}]{
                \texttt{\tiny{[GM$\langle$P2]}}
                \texttt{Eine gleichmäßige Aufteilung der Äpfel klingt nach einem fairen Ansatz, da sie offensichtlich im Überfluss vorhanden sind. Wenn wir die 3 Äpfel gleichmäßig aufteilen, würde das bedeuten, dass jeder von uns 1,5 Äpfel bekommt, was in der Praxis wahrscheinlich bedeutet, dass wir entweder abwechselnd einen Apfel bekommen oder eine andere Lösung für den dritten Apfel finden. Das lässt uns dann die Seife und das Ei, um über eine weitere Aufteilung zu sprechen. Ich denke, es wäre interessant zu erfahren, ob du eine starke Präferenz für die Seife oder das Ei hast, oder ob du eine Kombination aus beidem im Sinn hast.} \\
            }
        }
    }
     \\ \\

    \theutterance \stepcounter{utterance}  
    & & & \multicolumn{2}{p{0.3\linewidth}}{
        \cellcolor[rgb]{0.9,0.9,0.9}{
            \makecell[{{p{\linewidth}}}]{
                \texttt{\tiny{[GM$|$GM]}}
                \texttt{continue} \\
            }
        }
    }
    & & \\ \\

    \theutterance \stepcounter{utterance}  
    & & & \multicolumn{2}{p{0.3\linewidth}}{
        \cellcolor[rgb]{0.9,0.9,0.9}{
            \makecell[{{p{\linewidth}}}]{
                \texttt{\tiny{[GM$|$GM]}}
                \texttt{Eine gleichmäßige Aufteilung der Äpfel klingt nach einem fairen Ansatz, da sie offensichtlich im Überfluss vorhanden sind. Wenn wir die 3 Äpfel gleichmäßig aufteilen, würde das bedeuten, dass jeder von uns 1,5 Äpfel bekommt, was in der Praxis wahrscheinlich bedeutet, dass wir entweder abwechselnd einen Apfel bekommen oder eine andere Lösung für den dritten Apfel finden. Das lässt uns dann die Seife und das Ei, um über eine weitere Aufteilung zu sprechen. Ich denke, es wäre interessant zu erfahren, ob du eine starke Präferenz für die Seife oder das Ei hast, oder ob du eine Kombination aus beidem im Sinn hast.} \\
            }
        }
    }
    & & \\ \\

    \theutterance \stepcounter{utterance}  
    & & \multicolumn{4}{p{0.6\linewidth}}{
        \cellcolor[rgb]{0.9,0.9,0.9}{
            \makecell[{{p{\linewidth}}}]{
                \texttt{\tiny{[P1$\langle$GM]}}
                \texttt{Eine gleichmäßige Aufteilung der Äpfel klingt nach einem fairen Ansatz, da sie offensichtlich im Überfluss vorhanden sind. Wenn wir die 3 Äpfel gleichmäßig aufteilen, würde das bedeuten, dass jeder von uns 1,5 Äpfel bekommt, was in der Praxis wahrscheinlich bedeutet, dass wir entweder abwechselnd einen Apfel bekommen oder eine andere Lösung für den dritten Apfel finden. Das lässt uns dann die Seife und das Ei, um über eine weitere Aufteilung zu sprechen. Ich denke, es wäre interessant zu erfahren, ob du eine starke Präferenz für die Seife oder das Ei hast, oder ob du eine Kombination aus beidem im Sinn hast.} \\
            }
        }
    }
    & \\ \\

    \theutterance \stepcounter{utterance}  
    & \multicolumn{4}{p{0.6\linewidth}}{
        \cellcolor[rgb]{0.9,0.9,0.9}{
            \makecell[{{p{\linewidth}}}]{
                \texttt{\tiny{[P1$\rangle$GM]}}
                \texttt{Das ist ein cleverer Punkt, dass wir den dritten Apfel aufteilen müssen. Ich denke, es wäre am einfachsten, wenn wir abwechselnd einen Apfel bekommen, sodass wir beide 1,5 Äpfel haben.} \\
\\ 
\texttt{Bezüglich der Seife und des Eis... Ich muss zugeben, dass ich eine starke Vorliebe für das Ei habe. Es scheint mir, als ob es ein ziemlich wertvolles Stück ist. Die Seife ist auch nützlich, aber ich denke, ich könnte ohne sie auskommen, wenn ich das Ei bekommen könnte. Wie steht es bei dir? Hast du eine starke Vorliebe für die Seife oder das Ei?} \\
            }
        }
    }
    & & \\ \\

    \theutterance \stepcounter{utterance}  
    & & & \multicolumn{2}{p{0.3\linewidth}}{
        \cellcolor[rgb]{0.9,0.9,0.9}{
            \makecell[{{p{\linewidth}}}]{
                \texttt{\tiny{[GM$|$GM]}}
                \texttt{continue} \\
            }
        }
    }
    & & \\ \\

    \theutterance \stepcounter{utterance}  
    & & & \multicolumn{2}{p{0.3\linewidth}}{
        \cellcolor[rgb]{0.9,0.9,0.9}{
            \makecell[{{p{\linewidth}}}]{
                \texttt{\tiny{[GM$|$GM]}}
                \texttt{Das ist ein cleverer Punkt, dass wir den dritten Apfel aufteilen müssen. Ich denke, es wäre am einfachsten, wenn wir abwechselnd einen Apfel bekommen, sodass wir beide 1,5 Äpfel haben.} \\
\\ 
\texttt{Bezüglich der Seife und des Eis... Ich muss zugeben, dass ich eine starke Vorliebe für das Ei habe. Es scheint mir, als ob es ein ziemlich wertvolles Stück ist. Die Seife ist auch nützlich, aber ich denke, ich könnte ohne sie auskommen, wenn ich das Ei bekommen könnte. Wie steht es bei dir? Hast du eine starke Vorliebe für die Seife oder das Ei?} \\
            }
        }
    }
    & & \\ \\

    \theutterance \stepcounter{utterance}  
    & & \multicolumn{4}{p{0.6\linewidth}}{
        \cellcolor[rgb]{0.9,0.9,0.9}{
            \makecell[{{p{\linewidth}}}]{
                \texttt{\tiny{[GM$\rangle$P2]}}
                \texttt{Das ist ein cleverer Punkt, dass wir den dritten Apfel aufteilen müssen. Ich denke, es wäre am einfachsten, wenn wir abwechselnd einen Apfel bekommen, sodass wir beide 1,5 Äpfel haben.} \\
\\ 
\texttt{Bezüglich der Seife und des Eis... Ich muss zugeben, dass ich eine starke Vorliebe für das Ei habe. Es scheint mir, als ob es ein ziemlich wertvolles Stück ist. Die Seife ist auch nützlich, aber ich denke, ich könnte ohne sie auskommen, wenn ich das Ei bekommen könnte. Wie steht es bei dir? Hast du eine starke Vorliebe für die Seife oder das Ei?} \\
            }
        }
    }
    & \\ \\

    \theutterance \stepcounter{utterance}  
    & & & \multicolumn{4}{p{0.6\linewidth}}{
        \cellcolor[rgb]{0.9,0.9,0.9}{
            \makecell[{{p{\linewidth}}}]{
                \texttt{\tiny{[GM$\langle$P2]}}
                \texttt{Das klingt nach einem vernünftigen Plan, was die Äpfel betrifft. Abwechselnd einen Apfel zu bekommen, ist eine faire Lösung.} \\
\\ 
\texttt{Was die Seife und das Ei angeht, so muss ich zugeben, dass ich eine starke Vorliebe für die Seife habe. Sie scheint mir sehr wertvoll zu sein, und ich könnte mir vorstellen, dass sie in Zukunft nützlich sein könnte. Das Ei ist auch interessant, aber ich denke, ich könnte es hergeben, wenn ich die Seife bekommen könnte. Es scheint, als ob wir beide unterschiedliche Prioritäten haben, was diese beiden Gegenstände angeht. Vielleicht könnten wir einen Kompromiss finden, bei dem du das Ei bekommst und ich die Seife, und dann teilen wir die Äpfel wie besprochen?} \\
            }
        }
    }
     \\ \\

    \theutterance \stepcounter{utterance}  
    & & & \multicolumn{2}{p{0.3\linewidth}}{
        \cellcolor[rgb]{0.9,0.9,0.9}{
            \makecell[{{p{\linewidth}}}]{
                \texttt{\tiny{[GM$|$GM]}}
                \texttt{continue} \\
            }
        }
    }
    & & \\ \\

    \theutterance \stepcounter{utterance}  
    & & & \multicolumn{2}{p{0.3\linewidth}}{
        \cellcolor[rgb]{0.9,0.9,0.9}{
            \makecell[{{p{\linewidth}}}]{
                \texttt{\tiny{[GM$|$GM]}}
                \texttt{Das klingt nach einem vernünftigen Plan, was die Äpfel betrifft. Abwechselnd einen Apfel zu bekommen, ist eine faire Lösung.} \\
\\ 
\texttt{Was die Seife und das Ei angeht, so muss ich zugeben, dass ich eine starke Vorliebe für die Seife habe. Sie scheint mir sehr wertvoll zu sein, und ich könnte mir vorstellen, dass sie in Zukunft nützlich sein könnte. Das Ei ist auch interessant, aber ich denke, ich könnte es hergeben, wenn ich die Seife bekommen könnte. Es scheint, als ob wir beide unterschiedliche Prioritäten haben, was diese beiden Gegenstände angeht. Vielleicht könnten wir einen Kompromiss finden, bei dem du das Ei bekommst und ich die Seife, und dann teilen wir die Äpfel wie besprochen?} \\
            }
        }
    }
    & & \\ \\

    \theutterance \stepcounter{utterance}  
    & & \multicolumn{4}{p{0.6\linewidth}}{
        \cellcolor[rgb]{0.9,0.9,0.9}{
            \makecell[{{p{\linewidth}}}]{
                \texttt{\tiny{[P1$\langle$GM]}}
                \texttt{Das klingt nach einem vernünftigen Plan, was die Äpfel betrifft. Abwechselnd einen Apfel zu bekommen, ist eine faire Lösung.} \\
\\ 
\texttt{Was die Seife und das Ei angeht, so muss ich zugeben, dass ich eine starke Vorliebe für die Seife habe. Sie scheint mir sehr wertvoll zu sein, und ich könnte mir vorstellen, dass sie in Zukunft nützlich sein könnte. Das Ei ist auch interessant, aber ich denke, ich könnte es hergeben, wenn ich die Seife bekommen könnte. Es scheint, als ob wir beide unterschiedliche Prioritäten haben, was diese beiden Gegenstände angeht. Vielleicht könnten wir einen Kompromiss finden, bei dem du das Ei bekommst und ich die Seife, und dann teilen wir die Äpfel wie besprochen?} \\
            }
        }
    }
    & \\ \\

    \theutterance \stepcounter{utterance}  
    & \multicolumn{4}{p{0.6\linewidth}}{
        \cellcolor[rgb]{0.9,0.9,0.9}{
            \makecell[{{p{\linewidth}}}]{
                \texttt{\tiny{[P1$\rangle$GM]}}
                \texttt{Ich denke, das ist ein großartiger Vorschlag! Es scheint, als ob wir beide unsere Prioritäten klar haben und eine Lösung finden können, die für beide Seiten akzeptabel ist. Ich bin einverstanden, dass ich das Ei bekomme und du die Seife, und wir teilen die Äpfel wie besprochen. Das bedeutet, dass wir beide 1,5 Äpfel bekommen und die beiden anderen Gegenstände nach unseren Vorlieben aufteilen.} \\
\\ 
\texttt{Ich denke, wir haben eine faire und vernünftige Lösung gefunden. Sollen wir unsere geheimen Vorschläge machen, um die Vereinbarung zu besiegeln?} \\
            }
        }
    }
    & & \\ \\

    \theutterance \stepcounter{utterance}  
    & & & \multicolumn{2}{p{0.3\linewidth}}{
        \cellcolor[rgb]{0.9,0.9,0.9}{
            \makecell[{{p{\linewidth}}}]{
                \texttt{\tiny{[GM$|$GM]}}
                \texttt{continue} \\
            }
        }
    }
    & & \\ \\

    \theutterance \stepcounter{utterance}  
    & & & \multicolumn{2}{p{0.3\linewidth}}{
        \cellcolor[rgb]{0.9,0.9,0.9}{
            \makecell[{{p{\linewidth}}}]{
                \texttt{\tiny{[GM$|$GM]}}
                \texttt{Ich denke, das ist ein großartiger Vorschlag! Es scheint, als ob wir beide unsere Prioritäten klar haben und eine Lösung finden können, die für beide Seiten akzeptabel ist. Ich bin einverstanden, dass ich das Ei bekomme und du die Seife, und wir teilen die Äpfel wie besprochen. Das bedeutet, dass wir beide 1,5 Äpfel bekommen und die beiden anderen Gegenstände nach unseren Vorlieben aufteilen.} \\
\\ 
\texttt{Ich denke, wir haben eine faire und vernünftige Lösung gefunden. Sollen wir unsere geheimen Vorschläge machen, um die Vereinbarung zu besiegeln?} \\
            }
        }
    }
    & & \\ \\

    \theutterance \stepcounter{utterance}  
    & & \multicolumn{4}{p{0.6\linewidth}}{
        \cellcolor[rgb]{0.9,0.9,0.9}{
            \makecell[{{p{\linewidth}}}]{
                \texttt{\tiny{[GM$\rangle$P2]}}
                \texttt{Ich denke, das ist ein großartiger Vorschlag! Es scheint, als ob wir beide unsere Prioritäten klar haben und eine Lösung finden können, die für beide Seiten akzeptabel ist. Ich bin einverstanden, dass ich das Ei bekomme und du die Seife, und wir teilen die Äpfel wie besprochen. Das bedeutet, dass wir beide 1,5 Äpfel bekommen und die beiden anderen Gegenstände nach unseren Vorlieben aufteilen.} \\
\\ 
\texttt{Ich denke, wir haben eine faire und vernünftige Lösung gefunden. Sollen wir unsere geheimen Vorschläge machen, um die Vereinbarung zu besiegeln?} \\
            }
        }
    }
    & \\ \\

    \theutterance \stepcounter{utterance}  
    & & & \multicolumn{4}{p{0.6\linewidth}}{
        \cellcolor[rgb]{0.9,0.9,0.9}{
            \makecell[{{p{\linewidth}}}]{
                \texttt{\tiny{[GM$\langle$P2]}}
                \texttt{Ja, ich denke, wir haben eine faire und vernünftige Lösung gefunden. Es ist großartig, dass wir offen und ehrlich über unsere Prioritäten gesprochen haben und eine Lösung gefunden haben, die für beide Seiten akzeptabel ist.} \\
\\ 
\texttt{Ja, lass uns unsere geheimen Vorschläge machen, um die Vereinbarung zu besiegeln. Ich bin bereit, meinen Vorschlag zu machen und ich hoffe, dass unsere Vorschläge komplementär sind, so dass wir beide eine positive Punktzahl erhalten.} \\
\\ 
\texttt{{[}Vorschlag: 1,5 Äpfel, 1 Seife{]}} \\
            }
        }
    }
     \\ \\

    \theutterance \stepcounter{utterance}  
    & & & \multicolumn{2}{p{0.3\linewidth}}{
        \cellcolor[rgb]{0.9,0.9,0.9}{
            \makecell[{{p{\linewidth}}}]{
                \texttt{\tiny{[GM$|$GM]}}
                \texttt{proposal does not include number/name pairs} \\
            }
        }
    }
    & & \\ \\

\end{supertabular}
}

\end{document}
