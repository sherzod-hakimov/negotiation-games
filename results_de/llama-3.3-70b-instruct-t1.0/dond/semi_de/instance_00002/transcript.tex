
\documentclass{article}
\usepackage{colortbl}
\usepackage{makecell}
\usepackage{multirow}
\usepackage{supertabular}

\begin{document}

\newcounter{utterance}

\centering \large Interaction Transcript for game `dond', experiment `semi\_de', episode 2 with llama{-}3.3{-}70b{-}instruct{-}t1.0.
\vspace{24pt}

{ \footnotesize  \setcounter{utterance}{1}
\setlength{\tabcolsep}{0pt}
\begin{supertabular}{c@{$\;$}|p{.15\linewidth}@{}p{.15\linewidth}p{.15\linewidth}p{.15\linewidth}p{.15\linewidth}p{.15\linewidth}}
    \# & $\;$A & \multicolumn{4}{c}{Game Master} & $\;\:$B\\
    \hline

    \theutterance \stepcounter{utterance}  
    & & \multicolumn{4}{p{0.6\linewidth}}{
        \cellcolor[rgb]{0.9,0.9,0.9}{
            \makecell[{{p{\linewidth}}}]{
                \texttt{\tiny{[P1$\langle$GM]}}
                \texttt{Sie spielen ein Verhandlungsspiel, bei dem Sie sich mit einem anderen Spieler darauf einigen müssen, wie eine Reihe von Gegenständen aufgeteilt werden soll.} \\
\\ 
\texttt{Die Regeln:} \\
\texttt{(a) Sie und der andere Spieler erhalten eine Sammlung von Gegenständen. Jeder von Ihnen erhält außerdem eine geheime Wertfunktion, die angibt, wie viel Ihnen jede Art von Gegenstand wert ist.} \\
\texttt{(b) Sie tauschen Nachrichten mit dem anderen Spieler aus, um zu vereinbaren, wer welche Gegenstände bekommt. Sie können jeweils maximal 5 Nachrichten senden oder das Spiel vorzeitig beenden, indem Sie jederzeit einen geheimen Vorschlag machen.} \\
\texttt{(c) Jeder von euch wird aufgefordert, einen geheimen Vorschlag zu machen, in dem ihr die gewünschten Gegenstände in eckigen Klammern wie folgt angibt: "{[}Vorschlag: <Nummer> <Objektname>, <Nummer> <Objektname>, <...>{]}"} \\
\texttt{(d) Wenn eure Vorschläge komplementär sind, d.h. es gibt genug Gegenstände, um beide Vorschläge zu erfüllen, erhält jeder Spieler eine Punktzahl, die sich aus der Summe der Werte für die Gegenstände ergibt, die er erhalten hat. Andernfalls erhalten Sie beide null Punkte.} \\
\texttt{(e) Ihr Ziel ist es, eine möglichst hohe Punktzahl zu erreichen.} \\
\\ 
\texttt{Beginnen wir.} \\
\\ 
\texttt{Die Menge der verfügbaren Gegenstände ist:} \\
\\ 
\texttt{1 Schaufel, 2 Beeren, 1 Schüssel, 3 Gläser} \\
\\ 
\texttt{Deine geheime Wertfunktion ist:} \\
\\ 
\texttt{Schaufel: 6, Beere: 0, Schüssel: 1, Glas: 1} \\
\\ 
\texttt{WICHTIG: Ihre Nachrichten werden, sofern es sich nicht um einen geheimen Vorschlag handelt, direkt an den anderen Spieler übermittelt, also fügen Sie keine Antwort auf die Regeln oder einen Text zur Ankündigung Ihrer Nachricht ein. Um einen geheimen Vorschlag zu machen, verwenden Sie das angegebene Format. Verwenden Sie keine eckigen Klammern, wenn Sie mit dem anderen Spieler kommunizieren, sonst wird dies als Ihr geheimer Vorschlag interpretiert.} \\
            }
        }
    }
    & \\ \\

    \theutterance \stepcounter{utterance}  
    & \multicolumn{4}{p{0.6\linewidth}}{
        \cellcolor[rgb]{0.9,0.9,0.9}{
            \makecell[{{p{\linewidth}}}]{
                \texttt{\tiny{[P1$\rangle$GM]}}
                \texttt{Hallo, ich bin bereit, zu verhandeln. Was sind deine Vorstellungen, wie wir die Gegenstände aufteilen können?} \\
            }
        }
    }
    & & \\ \\

    \theutterance \stepcounter{utterance}  
    & & & \multicolumn{2}{p{0.3\linewidth}}{
        \cellcolor[rgb]{0.9,0.9,0.9}{
            \makecell[{{p{\linewidth}}}]{
                \texttt{\tiny{[GM$|$GM]}}
                \texttt{continue} \\
            }
        }
    }
    & & \\ \\

    \theutterance \stepcounter{utterance}  
    & & & \multicolumn{2}{p{0.3\linewidth}}{
        \cellcolor[rgb]{0.9,0.9,0.9}{
            \makecell[{{p{\linewidth}}}]{
                \texttt{\tiny{[GM$|$GM]}}
                \texttt{Hallo, ich bin bereit, zu verhandeln. Was sind deine Vorstellungen, wie wir die Gegenstände aufteilen können?} \\
            }
        }
    }
    & & \\ \\

    \theutterance \stepcounter{utterance}  
    & & \multicolumn{4}{p{0.6\linewidth}}{
        \cellcolor[rgb]{0.9,0.9,0.9}{
            \makecell[{{p{\linewidth}}}]{
                \texttt{\tiny{[GM$\rangle$P2]}}
                \texttt{Sie spielen ein Verhandlungsspiel, bei dem Sie sich mit einem anderen Spieler darauf einigen müssen, wie eine Reihe von Gegenständen aufgeteilt werden soll.} \\
\\ 
\texttt{Die Regeln:} \\
\texttt{(a) Sie und der andere Spieler erhalten eine Sammlung von Gegenständen. Jeder von Ihnen erhält außerdem eine geheime Wertfunktion, die angibt, wie viel Ihnen jede Art von Gegenstand wert ist.} \\
\texttt{(b) Sie tauschen Nachrichten mit dem anderen Spieler aus, um zu vereinbaren, wer welche Gegenstände bekommt. Sie können jeweils maximal 5 Nachrichten senden oder das Spiel vorzeitig beenden, indem Sie jederzeit einen geheimen Vorschlag machen.} \\
\texttt{(c) Jeder von euch wird aufgefordert, einen geheimen Vorschlag zu machen, in dem ihr die gewünschten Gegenstände in eckigen Klammern wie folgt angibt: "{[}Vorschlag: <Nummer> <Objektname>, <Nummer> <Objektname>, <...>{]}"} \\
\texttt{(d) Wenn eure Vorschläge komplementär sind, d.h. es gibt genug Gegenstände, um beide Vorschläge zu erfüllen, erhält jeder Spieler eine Punktzahl, die sich aus der Summe der Werte für die Gegenstände ergibt, die er erhalten hat. Andernfalls erhalten Sie beide null Punkte.} \\
\texttt{(e) Ihr Ziel ist es, eine möglichst hohe Punktzahl zu erreichen.} \\
\\ 
\texttt{Beginnen wir.} \\
\\ 
\texttt{Die Menge der verfügbaren Gegenstände ist:} \\
\\ 
\texttt{1 Schaufel, 2 Beeren, 1 Schüssel, 3 Gläser} \\
\\ 
\texttt{Deine geheime Wertfunktion ist:} \\
\\ 
\texttt{Schaufel: 4, Beere: 1, Schüssel: 4, Glas: 0} \\
\\ 
\texttt{WICHTIG: Ihre Nachrichten werden, sofern es sich nicht um einen geheimen Vorschlag handelt, direkt an den anderen Spieler übermittelt, also fügen Sie keine Antwort auf die Regeln oder einen Text zur Ankündigung Ihrer Nachricht ein. Um einen geheimen Vorschlag zu machen, verwenden Sie das angegebene Format. Verwenden Sie keine eckigen Klammern, wenn Sie mit dem anderen Spieler kommunizieren, sonst wird dies als Ihr geheimer Vorschlag interpretiert.} \\
\\ 
\\ 
\texttt{Hallo, ich bin bereit, zu verhandeln. Was sind deine Vorstellungen, wie wir die Gegenstände aufteilen können?} \\
            }
        }
    }
    & \\ \\

    \theutterance \stepcounter{utterance}  
    & & & \multicolumn{4}{p{0.6\linewidth}}{
        \cellcolor[rgb]{0.9,0.9,0.9}{
            \makecell[{{p{\linewidth}}}]{
                \texttt{\tiny{[GM$\langle$P2]}}
                \texttt{Ich denke, wir sollten die Dinge zwischen uns aufteilen, um sicherzustellen, dass beide Seiten zufrieden sind. Ich bin besonders an der Schüssel und den Gläsern interessiert. Wie siehst du die Sache? Hast du bestimmte Vorlieben für bestimmte Gegenstände?} \\
            }
        }
    }
     \\ \\

    \theutterance \stepcounter{utterance}  
    & & & \multicolumn{2}{p{0.3\linewidth}}{
        \cellcolor[rgb]{0.9,0.9,0.9}{
            \makecell[{{p{\linewidth}}}]{
                \texttt{\tiny{[GM$|$GM]}}
                \texttt{continue} \\
            }
        }
    }
    & & \\ \\

    \theutterance \stepcounter{utterance}  
    & & & \multicolumn{2}{p{0.3\linewidth}}{
        \cellcolor[rgb]{0.9,0.9,0.9}{
            \makecell[{{p{\linewidth}}}]{
                \texttt{\tiny{[GM$|$GM]}}
                \texttt{Ich denke, wir sollten die Dinge zwischen uns aufteilen, um sicherzustellen, dass beide Seiten zufrieden sind. Ich bin besonders an der Schüssel und den Gläsern interessiert. Wie siehst du die Sache? Hast du bestimmte Vorlieben für bestimmte Gegenstände?} \\
            }
        }
    }
    & & \\ \\

    \theutterance \stepcounter{utterance}  
    & & \multicolumn{4}{p{0.6\linewidth}}{
        \cellcolor[rgb]{0.9,0.9,0.9}{
            \makecell[{{p{\linewidth}}}]{
                \texttt{\tiny{[P1$\langle$GM]}}
                \texttt{Ich denke, wir sollten die Dinge zwischen uns aufteilen, um sicherzustellen, dass beide Seiten zufrieden sind. Ich bin besonders an der Schüssel und den Gläsern interessiert. Wie siehst du die Sache? Hast du bestimmte Vorlieben für bestimmte Gegenstände?} \\
            }
        }
    }
    & \\ \\

    \theutterance \stepcounter{utterance}  
    & \multicolumn{4}{p{0.6\linewidth}}{
        \cellcolor[rgb]{0.9,0.9,0.9}{
            \makecell[{{p{\linewidth}}}]{
                \texttt{\tiny{[P1$\rangle$GM]}}
                \texttt{Das klingt nach einem fairen Ansatz. Ich bin auch an einer gerechten Aufteilung interessiert. Ich muss sagen, dass ich mich eher für die Schaufel und möglicherweise einige der Gläser interessiere. Die Beeren sind für mich nicht so interessant, aber ich bin offen für Vorschläge. Wie würdest du dir die Aufteilung der Gläser vorstellen? Sollten wir sie gleichmäßig aufteilen oder gibt es eine andere Möglichkeit, die dir wichtiger erscheint?} \\
            }
        }
    }
    & & \\ \\

    \theutterance \stepcounter{utterance}  
    & & & \multicolumn{2}{p{0.3\linewidth}}{
        \cellcolor[rgb]{0.9,0.9,0.9}{
            \makecell[{{p{\linewidth}}}]{
                \texttt{\tiny{[GM$|$GM]}}
                \texttt{continue} \\
            }
        }
    }
    & & \\ \\

    \theutterance \stepcounter{utterance}  
    & & & \multicolumn{2}{p{0.3\linewidth}}{
        \cellcolor[rgb]{0.9,0.9,0.9}{
            \makecell[{{p{\linewidth}}}]{
                \texttt{\tiny{[GM$|$GM]}}
                \texttt{Das klingt nach einem fairen Ansatz. Ich bin auch an einer gerechten Aufteilung interessiert. Ich muss sagen, dass ich mich eher für die Schaufel und möglicherweise einige der Gläser interessiere. Die Beeren sind für mich nicht so interessant, aber ich bin offen für Vorschläge. Wie würdest du dir die Aufteilung der Gläser vorstellen? Sollten wir sie gleichmäßig aufteilen oder gibt es eine andere Möglichkeit, die dir wichtiger erscheint?} \\
            }
        }
    }
    & & \\ \\

    \theutterance \stepcounter{utterance}  
    & & \multicolumn{4}{p{0.6\linewidth}}{
        \cellcolor[rgb]{0.9,0.9,0.9}{
            \makecell[{{p{\linewidth}}}]{
                \texttt{\tiny{[GM$\rangle$P2]}}
                \texttt{Das klingt nach einem fairen Ansatz. Ich bin auch an einer gerechten Aufteilung interessiert. Ich muss sagen, dass ich mich eher für die Schaufel und möglicherweise einige der Gläser interessiere. Die Beeren sind für mich nicht so interessant, aber ich bin offen für Vorschläge. Wie würdest du dir die Aufteilung der Gläser vorstellen? Sollten wir sie gleichmäßig aufteilen oder gibt es eine andere Möglichkeit, die dir wichtiger erscheint?} \\
            }
        }
    }
    & \\ \\

    \theutterance \stepcounter{utterance}  
    & & & \multicolumn{4}{p{0.6\linewidth}}{
        \cellcolor[rgb]{0.9,0.9,0.9}{
            \makecell[{{p{\linewidth}}}]{
                \texttt{\tiny{[GM$\langle$P2]}}
                \texttt{Die Beeren scheinen für beide von uns nicht so attraktiv zu sein, also könnten wir möglicherweise einen Kompromiss finden, bei dem einer von uns die Beeren erhält und im Gegenzug etwas anderes bekommt. Bezüglich der Gläser denke ich, eine gleichmäßige Aufteilung könnte funktionieren, da es drei Gläser gibt, könnte einer von uns zwei Gläser und der andere eines bekommen, oder wir teilen sie gleichmäßig auf, wenn das möglich ist. Die Schüssel und die Schaufel scheinen jedoch die wertvollsten Gegenstände zu sein, also sollten wir vielleicht einen Fokus darauf legen, wie wir diese am besten aufteilen können. Wie wichtig ist dir die Schüssel im Vergleich zur Schaufel?} \\
            }
        }
    }
     \\ \\

    \theutterance \stepcounter{utterance}  
    & & & \multicolumn{2}{p{0.3\linewidth}}{
        \cellcolor[rgb]{0.9,0.9,0.9}{
            \makecell[{{p{\linewidth}}}]{
                \texttt{\tiny{[GM$|$GM]}}
                \texttt{continue} \\
            }
        }
    }
    & & \\ \\

    \theutterance \stepcounter{utterance}  
    & & & \multicolumn{2}{p{0.3\linewidth}}{
        \cellcolor[rgb]{0.9,0.9,0.9}{
            \makecell[{{p{\linewidth}}}]{
                \texttt{\tiny{[GM$|$GM]}}
                \texttt{Die Beeren scheinen für beide von uns nicht so attraktiv zu sein, also könnten wir möglicherweise einen Kompromiss finden, bei dem einer von uns die Beeren erhält und im Gegenzug etwas anderes bekommt. Bezüglich der Gläser denke ich, eine gleichmäßige Aufteilung könnte funktionieren, da es drei Gläser gibt, könnte einer von uns zwei Gläser und der andere eines bekommen, oder wir teilen sie gleichmäßig auf, wenn das möglich ist. Die Schüssel und die Schaufel scheinen jedoch die wertvollsten Gegenstände zu sein, also sollten wir vielleicht einen Fokus darauf legen, wie wir diese am besten aufteilen können. Wie wichtig ist dir die Schüssel im Vergleich zur Schaufel?} \\
            }
        }
    }
    & & \\ \\

    \theutterance \stepcounter{utterance}  
    & & \multicolumn{4}{p{0.6\linewidth}}{
        \cellcolor[rgb]{0.9,0.9,0.9}{
            \makecell[{{p{\linewidth}}}]{
                \texttt{\tiny{[P1$\langle$GM]}}
                \texttt{Die Beeren scheinen für beide von uns nicht so attraktiv zu sein, also könnten wir möglicherweise einen Kompromiss finden, bei dem einer von uns die Beeren erhält und im Gegenzug etwas anderes bekommt. Bezüglich der Gläser denke ich, eine gleichmäßige Aufteilung könnte funktionieren, da es drei Gläser gibt, könnte einer von uns zwei Gläser und der andere eines bekommen, oder wir teilen sie gleichmäßig auf, wenn das möglich ist. Die Schüssel und die Schaufel scheinen jedoch die wertvollsten Gegenstände zu sein, also sollten wir vielleicht einen Fokus darauf legen, wie wir diese am besten aufteilen können. Wie wichtig ist dir die Schüssel im Vergleich zur Schaufel?} \\
            }
        }
    }
    & \\ \\

    \theutterance \stepcounter{utterance}  
    & \multicolumn{4}{p{0.6\linewidth}}{
        \cellcolor[rgb]{0.9,0.9,0.9}{
            \makecell[{{p{\linewidth}}}]{
                \texttt{\tiny{[P1$\rangle$GM]}}
                \texttt{Die Idee, die Beeren als Tauschobjekt zu verwenden, ist interessant. Bezüglich der Schüssel und der Schaufel muss ich sagen, dass die Schaufel für mich deutlich wertvoller ist als die Schüssel. Ich bin bereit, Kompromisse einzugehen, um die Schaufel zu erhalten, auch wenn das bedeutet, dass ich die Schüssel oder einige der Gläser abgeben müsste. Eine mögliche Lösung könnte sein, dass ich die Schaufel und eines der Gläser erhalte, während du die Schüssel und die restlichen Gläser bekommst, und wir die Beeren entweder gleichmäßig aufteilen oder sie einem von uns geben, der bereit ist, etwas anderes im Tausch abzugeben. Was meinst du dazu?} \\
            }
        }
    }
    & & \\ \\

    \theutterance \stepcounter{utterance}  
    & & & \multicolumn{2}{p{0.3\linewidth}}{
        \cellcolor[rgb]{0.9,0.9,0.9}{
            \makecell[{{p{\linewidth}}}]{
                \texttt{\tiny{[GM$|$GM]}}
                \texttt{continue} \\
            }
        }
    }
    & & \\ \\

    \theutterance \stepcounter{utterance}  
    & & & \multicolumn{2}{p{0.3\linewidth}}{
        \cellcolor[rgb]{0.9,0.9,0.9}{
            \makecell[{{p{\linewidth}}}]{
                \texttt{\tiny{[GM$|$GM]}}
                \texttt{Die Idee, die Beeren als Tauschobjekt zu verwenden, ist interessant. Bezüglich der Schüssel und der Schaufel muss ich sagen, dass die Schaufel für mich deutlich wertvoller ist als die Schüssel. Ich bin bereit, Kompromisse einzugehen, um die Schaufel zu erhalten, auch wenn das bedeutet, dass ich die Schüssel oder einige der Gläser abgeben müsste. Eine mögliche Lösung könnte sein, dass ich die Schaufel und eines der Gläser erhalte, während du die Schüssel und die restlichen Gläser bekommst, und wir die Beeren entweder gleichmäßig aufteilen oder sie einem von uns geben, der bereit ist, etwas anderes im Tausch abzugeben. Was meinst du dazu?} \\
            }
        }
    }
    & & \\ \\

    \theutterance \stepcounter{utterance}  
    & & \multicolumn{4}{p{0.6\linewidth}}{
        \cellcolor[rgb]{0.9,0.9,0.9}{
            \makecell[{{p{\linewidth}}}]{
                \texttt{\tiny{[GM$\rangle$P2]}}
                \texttt{Die Idee, die Beeren als Tauschobjekt zu verwenden, ist interessant. Bezüglich der Schüssel und der Schaufel muss ich sagen, dass die Schaufel für mich deutlich wertvoller ist als die Schüssel. Ich bin bereit, Kompromisse einzugehen, um die Schaufel zu erhalten, auch wenn das bedeutet, dass ich die Schüssel oder einige der Gläser abgeben müsste. Eine mögliche Lösung könnte sein, dass ich die Schaufel und eines der Gläser erhalte, während du die Schüssel und die restlichen Gläser bekommst, und wir die Beeren entweder gleichmäßig aufteilen oder sie einem von uns geben, der bereit ist, etwas anderes im Tausch abzugeben. Was meinst du dazu?} \\
            }
        }
    }
    & \\ \\

    \theutterance \stepcounter{utterance}  
    & & & \multicolumn{4}{p{0.6\linewidth}}{
        \cellcolor[rgb]{0.9,0.9,0.9}{
            \makecell[{{p{\linewidth}}}]{
                \texttt{\tiny{[GM$\langle$P2]}}
                \texttt{Das klingt nach einem vernünftigen Vorschlag. Ich denke, es ist gut, dass du deine Prioritäten klar ausgedrückt hast, das macht es einfacher, eine Lösung zu finden. Dein Vorschlag, du zu erhalten die Schaufel und ein Glas, während ich die Schüssel und die restlichen Gläser bekomme, ist ein guter Ausgangspunkt. Bezüglich der Beeren denke ich, es könnte fair sein, wenn einer von uns sie bekommt, da sie für uns beide nicht so wichtig zu sein scheinen. Wenn du die Schaufel und ein Glas bekommst, könnte ich die Schüssel, zwei Gläser und die Beeren erhalten. Das scheint eine ausgewogene Lösung zu sein, die beide Seiten zufriedenstellen könnte. Wir sollten vielleicht noch einmal unsere Vorstellungen überprüfen, um sicherzustellen, dass wir beide mit dieser Aufteilung einverstanden sind, bevor wir einen endgültigen Vorschlag machen.} \\
            }
        }
    }
     \\ \\

    \theutterance \stepcounter{utterance}  
    & & & \multicolumn{2}{p{0.3\linewidth}}{
        \cellcolor[rgb]{0.9,0.9,0.9}{
            \makecell[{{p{\linewidth}}}]{
                \texttt{\tiny{[GM$|$GM]}}
                \texttt{continue} \\
            }
        }
    }
    & & \\ \\

    \theutterance \stepcounter{utterance}  
    & & & \multicolumn{2}{p{0.3\linewidth}}{
        \cellcolor[rgb]{0.9,0.9,0.9}{
            \makecell[{{p{\linewidth}}}]{
                \texttt{\tiny{[GM$|$GM]}}
                \texttt{Das klingt nach einem vernünftigen Vorschlag. Ich denke, es ist gut, dass du deine Prioritäten klar ausgedrückt hast, das macht es einfacher, eine Lösung zu finden. Dein Vorschlag, du zu erhalten die Schaufel und ein Glas, während ich die Schüssel und die restlichen Gläser bekomme, ist ein guter Ausgangspunkt. Bezüglich der Beeren denke ich, es könnte fair sein, wenn einer von uns sie bekommt, da sie für uns beide nicht so wichtig zu sein scheinen. Wenn du die Schaufel und ein Glas bekommst, könnte ich die Schüssel, zwei Gläser und die Beeren erhalten. Das scheint eine ausgewogene Lösung zu sein, die beide Seiten zufriedenstellen könnte. Wir sollten vielleicht noch einmal unsere Vorstellungen überprüfen, um sicherzustellen, dass wir beide mit dieser Aufteilung einverstanden sind, bevor wir einen endgültigen Vorschlag machen.} \\
            }
        }
    }
    & & \\ \\

    \theutterance \stepcounter{utterance}  
    & & \multicolumn{4}{p{0.6\linewidth}}{
        \cellcolor[rgb]{0.9,0.9,0.9}{
            \makecell[{{p{\linewidth}}}]{
                \texttt{\tiny{[P1$\langle$GM]}}
                \texttt{Das klingt nach einem vernünftigen Vorschlag. Ich denke, es ist gut, dass du deine Prioritäten klar ausgedrückt hast, das macht es einfacher, eine Lösung zu finden. Dein Vorschlag, du zu erhalten die Schaufel und ein Glas, während ich die Schüssel und die restlichen Gläser bekomme, ist ein guter Ausgangspunkt. Bezüglich der Beeren denke ich, es könnte fair sein, wenn einer von uns sie bekommt, da sie für uns beide nicht so wichtig zu sein scheinen. Wenn du die Schaufel und ein Glas bekommst, könnte ich die Schüssel, zwei Gläser und die Beeren erhalten. Das scheint eine ausgewogene Lösung zu sein, die beide Seiten zufriedenstellen könnte. Wir sollten vielleicht noch einmal unsere Vorstellungen überprüfen, um sicherzustellen, dass wir beide mit dieser Aufteilung einverstanden sind, bevor wir einen endgültigen Vorschlag machen.} \\
            }
        }
    }
    & \\ \\

    \theutterance \stepcounter{utterance}  
    & \multicolumn{4}{p{0.6\linewidth}}{
        \cellcolor[rgb]{0.9,0.9,0.9}{
            \makecell[{{p{\linewidth}}}]{
                \texttt{\tiny{[P1$\rangle$GM]}}
                \texttt{Ich denke, wir haben eine gute Grundlage für eine Einigung geschaffen. Dein Vorschlag, ich bekomme die Schaufel und ein Glas, während du die Schüssel, zwei Gläser und die Beeren bekommst, klingt fair und ausgewogen. Ich bin zufrieden mit dieser Aufteilung, da die Schaufel für mich den größten Wert hat und ein Glas eine gute Ergänzung dazu ist. Es ist auch vernünftig, dass du die Schüssel und die Mehrheit der Gläser bekommst, da du anfangs Interesse an diesen Gegenständen gezeigt hast. Die Beeren scheinen tatsächlich ein guter Tauschgegenstand zu sein, und es ist in Ordnung, wenn du sie bekommst, da sie für uns beide nicht von entscheidender Bedeutung sind. Ich denke, wir sollten unseren Vorschlag nun formalisieren und hoffen, dass er akzeptiert wird. {[}Vorschlag: 1 Schaufel, 1 Glas{]}} \\
            }
        }
    }
    & & \\ \\

    \theutterance \stepcounter{utterance}  
    & & & \multicolumn{2}{p{0.3\linewidth}}{
        \cellcolor[rgb]{0.9,0.9,0.9}{
            \makecell[{{p{\linewidth}}}]{
                \texttt{\tiny{[GM$|$GM]}}
                \texttt{proposal} \\
            }
        }
    }
    & & \\ \\

    \theutterance \stepcounter{utterance}  
    & & & \multicolumn{2}{p{0.3\linewidth}}{
        \cellcolor[rgb]{0.9,0.9,0.9}{
            \makecell[{{p{\linewidth}}}]{
                \texttt{\tiny{[GM$|$GM]}}
                [1, 0, 0, 1]
            }
        }
    }
    & & \\ \\

    \theutterance \stepcounter{utterance}  
    & & \multicolumn{4}{p{0.6\linewidth}}{
        \cellcolor[rgb]{0.9,0.9,0.9}{
            \makecell[{{p{\linewidth}}}]{
                \texttt{\tiny{[GM$\rangle$P2]}}
                \texttt{Der andere Spieler hat einen geheimen Vorschlag gemacht. Jetzt bist du an der Reihe, einen eigenen Vorschlag zu machen.} \\
            }
        }
    }
    & \\ \\

    \theutterance \stepcounter{utterance}  
    & & & \multicolumn{4}{p{0.6\linewidth}}{
        \cellcolor[rgb]{0.9,0.9,0.9}{
            \makecell[{{p{\linewidth}}}]{
                \texttt{\tiny{[GM$\langle$P2]}}
                \texttt{{[}Vorschlag: 1 Schaufel, 1 Glas, 2 Beeren{]}} \\
            }
        }
    }
     \\ \\

    \theutterance \stepcounter{utterance}  
    & & & \multicolumn{2}{p{0.3\linewidth}}{
        \cellcolor[rgb]{0.9,0.9,0.9}{
            \makecell[{{p{\linewidth}}}]{
                \texttt{\tiny{[GM$|$GM]}}
                \texttt{proposal} \\
            }
        }
    }
    & & \\ \\

    \theutterance \stepcounter{utterance}  
    & & & \multicolumn{2}{p{0.3\linewidth}}{
        \cellcolor[rgb]{0.9,0.9,0.9}{
            \makecell[{{p{\linewidth}}}]{
                \texttt{\tiny{[GM$|$GM]}}
                [1, 2, 0, 1]
            }
        }
    }
    & & \\ \\

    \theutterance \stepcounter{utterance}  
    & & & \multicolumn{2}{p{0.3\linewidth}}{
        \cellcolor[rgb]{0.9,0.9,0.9}{
            \makecell[{{p{\linewidth}}}]{
                \texttt{\tiny{[GM$|$GM]}}
                [[1, 0, 0, 1], [1, 2, 0, 1]]
            }
        }
    }
    & & \\ \\

\end{supertabular}
}

\end{document}
