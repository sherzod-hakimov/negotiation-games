
\documentclass{article}
\usepackage{colortbl}
\usepackage{makecell}
\usepackage{multirow}
\usepackage{supertabular}

\begin{document}

\newcounter{utterance}

\twocolumn

{ \footnotesize  \setcounter{utterance}{1}
\setlength{\tabcolsep}{0pt}
\begin{supertabular}{c@{$\;$}|p{.15\linewidth}@{}p{.15\linewidth}p{.15\linewidth}p{.15\linewidth}p{.15\linewidth}p{.15\linewidth}}

    \# & $\;$A & \multicolumn{4}{c}{Game Master} & $\;\:$B\\
    \hline 

    \theutterance \stepcounter{utterance}  

    & & \multicolumn{4}{p{0.6\linewidth}}{\cellcolor[rgb]{0.9,0.9,0.9}{%
	\makecell[{{p{\linewidth}}}]{% 
	  \tt {\tiny [A$\langle$GM]}  
	 Sie spielen ein Verhandlungsspiel, bei dem Sie sich mit einem anderen Spieler darauf einigen müssen, wie eine Reihe von Gegenständen aufgeteilt werden soll.\\ \tt \\ \tt Die Regeln:\\ \tt (a) Sie und der andere Spieler erhalten eine Sammlung von Gegenständen. Jeder von Ihnen erhält außerdem eine geheime Wertfunktion, die angibt, wie viel Ihnen jede Art von Gegenstand wert ist.\\ \tt (b) Sie tauschen Nachrichten mit dem anderen Spieler aus, um zu vereinbaren, wer welche Gegenstände bekommt. Sie können jeweils maximal 5 Nachrichten senden oder das Spiel vorzeitig beenden, indem Sie jederzeit einen geheimen Vorschlag machen.\\ \tt (c) Jeder von euch wird aufgefordert, einen geheimen Vorschlag zu machen, in dem ihr die gewünschten Gegenstände in eckigen Klammern wie folgt angibt: "[Vorschlag: <Nummer> <Objektname>, <Nummer> <Objektname>, <...>]"\\ \tt (d) Wenn eure Vorschläge komplementär sind, d.h. es gibt genug Gegenstände, um beide Vorschläge zu erfüllen, erhält jeder Spieler eine Punktzahl, die sich aus der Summe der Werte für die Gegenstände ergibt, die er erhalten hat. Andernfalls erhalten Sie beide null Punkte.\\ \tt (e) Ihr Ziel ist es, eine möglichst hohe Punktzahl zu erreichen.\\ \tt \\ \tt Beginnen wir.\\ \tt \\ \tt Die Menge der verfügbaren Gegenstände ist:\\ \tt \\ \tt 1 Hut, 3 Stifte, 1 Schwamm\\ \tt \\ \tt Deine geheime Wertfunktion ist:\\ \tt \\ \tt Hut: 4, Stift: 2, Schwamm: 0\\ \tt \\ \tt WICHTIG: Ihre Nachrichten werden, sofern es sich nicht um einen geheimen Vorschlag handelt, direkt an den anderen Spieler übermittelt, also fügen Sie keine Antwort auf die Regeln oder einen Text zur Ankündigung Ihrer Nachricht ein. Um einen geheimen Vorschlag zu machen, verwenden Sie das angegebene Format. Verwenden Sie keine eckigen Klammern, wenn Sie mit dem anderen Spieler kommunizieren, sonst wird dies als Ihr geheimer Vorschlag interpretiert.\\ \tt  
	  } 
	   } 
	   } 
	 & \\ 
 

    \theutterance \stepcounter{utterance}  

    & \multicolumn{4}{p{0.6\linewidth}}{\cellcolor[rgb]{0.8,1,0.9}{%
	\makecell[{{p{\linewidth}}}]{% 
	  \tt {\tiny [A$\rangle$GM]}  
	 Hallo, ich bin bereit, Verhandlungen aufzunehmen. Wie wichtig sind dir die verschiedenen Gegenstände in unserer Sammlung? Hast du eine Vorstellung davon, wie du die Gegenstände aufteilen möchtest? 
	  } 
	   } 
	   } 
	 & & \\ 
 

    \theutterance \stepcounter{utterance}  

    & & & \multicolumn{2}{p{0.3\linewidth}}{\cellcolor[rgb]{0.95,0.95,0.95}{%
	\makecell[{{p{\linewidth}}}]{% 
	  \tt {\tiny [GM$|$GM]}  
	 continue 
	  } 
	   } 
	   } 
	 & & \\ 
 

    \theutterance \stepcounter{utterance}  

    & & & \multicolumn{2}{p{0.3\linewidth}}{\cellcolor[rgb]{0.95,0.95,0.95}{%
	\makecell[{{p{\linewidth}}}]{% 
	  \tt {\tiny [GM$|$GM]}  
	 Hallo, ich bin bereit, Verhandlungen aufzunehmen. Wie wichtig sind dir die verschiedenen Gegenstände in unserer Sammlung? Hast du eine Vorstellung davon, wie du die Gegenstände aufteilen möchtest? 
	  } 
	   } 
	   } 
	 & & \\ 
 

    \theutterance \stepcounter{utterance}  

    & & \multicolumn{4}{p{0.6\linewidth}}{\cellcolor[rgb]{0.9,0.9,0.9}{%
	\makecell[{{p{\linewidth}}}]{% 
	  \tt {\tiny [GM$\rangle$B]}  
	 Sie spielen ein Verhandlungsspiel, bei dem Sie sich mit einem anderen Spieler darauf einigen müssen, wie eine Reihe von Gegenständen aufgeteilt werden soll.\\ \tt \\ \tt Die Regeln:\\ \tt (a) Sie und der andere Spieler erhalten eine Sammlung von Gegenständen. Jeder von Ihnen erhält außerdem eine geheime Wertfunktion, die angibt, wie viel Ihnen jede Art von Gegenstand wert ist.\\ \tt (b) Sie tauschen Nachrichten mit dem anderen Spieler aus, um zu vereinbaren, wer welche Gegenstände bekommt. Sie können jeweils maximal 5 Nachrichten senden oder das Spiel vorzeitig beenden, indem Sie jederzeit einen geheimen Vorschlag machen.\\ \tt (c) Jeder von euch wird aufgefordert, einen geheimen Vorschlag zu machen, in dem ihr die gewünschten Gegenstände in eckigen Klammern wie folgt angibt: "[Vorschlag: <Nummer> <Objektname>, <Nummer> <Objektname>, <...>]"\\ \tt (d) Wenn eure Vorschläge komplementär sind, d.h. es gibt genug Gegenstände, um beide Vorschläge zu erfüllen, erhält jeder Spieler eine Punktzahl, die sich aus der Summe der Werte für die Gegenstände ergibt, die er erhalten hat. Andernfalls erhalten Sie beide null Punkte.\\ \tt (e) Ihr Ziel ist es, eine möglichst hohe Punktzahl zu erreichen.\\ \tt \\ \tt Beginnen wir.\\ \tt \\ \tt Die Menge der verfügbaren Gegenstände ist:\\ \tt \\ \tt 1 Hut, 3 Stifte, 1 Schwamm\\ \tt \\ \tt Deine geheime Wertfunktion ist:\\ \tt \\ \tt Hut: 4, Stift: 0, Schwamm: 6\\ \tt \\ \tt WICHTIG: Ihre Nachrichten werden, sofern es sich nicht um einen geheimen Vorschlag handelt, direkt an den anderen Spieler übermittelt, also fügen Sie keine Antwort auf die Regeln oder einen Text zur Ankündigung Ihrer Nachricht ein. Um einen geheimen Vorschlag zu machen, verwenden Sie das angegebene Format. Verwenden Sie keine eckigen Klammern, wenn Sie mit dem anderen Spieler kommunizieren, sonst wird dies als Ihr geheimer Vorschlag interpretiert.\\ \tt \\ \tt \\ \tt Hallo, ich bin bereit, Verhandlungen aufzunehmen. Wie wichtig sind dir die verschiedenen Gegenstände in unserer Sammlung? Hast du eine Vorstellung davon, wie du die Gegenstände aufteilen möchtest? 
	  } 
	   } 
	   } 
	 & \\ 
 

    \theutterance \stepcounter{utterance}  

    & & & \multicolumn{4}{p{0.6\linewidth}}{\cellcolor[rgb]{1,0.85,0.72}{%
	\makecell[{{p{\linewidth}}}]{% 
	  \tt {\tiny [GM$\langle$B]}  
	 Hallo! Ich bin auch bereit, Verhandlungen aufzunehmen. Ich denke, der Hut ist ziemlich wertvoll, aber ich bin nicht so sicher, wie wichtig er dir ist. Die Stifte scheinen mir nicht besonders wertvoll zu sein, aber der Schwamm könnte interessant sein. Was meinst du? Sollten wir versuchen, die Gegenstände gleichmäßig aufzuteilen oder sollten wir nach anderen Kriterien vorgehen? 
	  } 
	   } 
	   } 
	  \\ 
 

    \theutterance \stepcounter{utterance}  

    & & & \multicolumn{2}{p{0.3\linewidth}}{\cellcolor[rgb]{0.95,0.95,0.95}{%
	\makecell[{{p{\linewidth}}}]{% 
	  \tt {\tiny [GM$|$GM]}  
	 continue 
	  } 
	   } 
	   } 
	 & & \\ 
 

    \theutterance \stepcounter{utterance}  

    & & & \multicolumn{2}{p{0.3\linewidth}}{\cellcolor[rgb]{0.95,0.95,0.95}{%
	\makecell[{{p{\linewidth}}}]{% 
	  \tt {\tiny [GM$|$GM]}  
	 Hallo! Ich bin auch bereit, Verhandlungen aufzunehmen. Ich denke, der Hut ist ziemlich wertvoll, aber ich bin nicht so sicher, wie wichtig er dir ist. Die Stifte scheinen mir nicht besonders wertvoll zu sein, aber der Schwamm könnte interessant sein. Was meinst du? Sollten wir versuchen, die Gegenstände gleichmäßig aufzuteilen oder sollten wir nach anderen Kriterien vorgehen? 
	  } 
	   } 
	   } 
	 & & \\ 
 

    \theutterance \stepcounter{utterance}  

    & & \multicolumn{4}{p{0.6\linewidth}}{\cellcolor[rgb]{0.9,0.9,0.9}{%
	\makecell[{{p{\linewidth}}}]{% 
	  \tt {\tiny [A$\langle$GM]}  
	 Hallo! Ich bin auch bereit, Verhandlungen aufzunehmen. Ich denke, der Hut ist ziemlich wertvoll, aber ich bin nicht so sicher, wie wichtig er dir ist. Die Stifte scheinen mir nicht besonders wertvoll zu sein, aber der Schwamm könnte interessant sein. Was meinst du? Sollten wir versuchen, die Gegenstände gleichmäßig aufzuteilen oder sollten wir nach anderen Kriterien vorgehen? 
	  } 
	   } 
	   } 
	 & \\ 
 

    \theutterance \stepcounter{utterance}  

    & \multicolumn{4}{p{0.6\linewidth}}{\cellcolor[rgb]{0.8,1,0.9}{%
	\makecell[{{p{\linewidth}}}]{% 
	  \tt {\tiny [A$\rangle$GM]}  
	 Das ist eine gute Frage. Ich denke, eine gleichmäßige Aufteilung könnte fair sein, aber es hängt auch davon ab, wie wichtig jeder Gegenstand für jeden von uns ist. Ich persönlich finde, dass der Hut sehr wertvoll ist, aber ich bin auch an den Stiften interessiert. Der Schwamm ist für mich nicht so wichtig. Wie siehst du das? Gibt es bestimmte Gegenstände, die du unbedingt haben möchtest? 
	  } 
	   } 
	   } 
	 & & \\ 
 

    \theutterance \stepcounter{utterance}  

    & & & \multicolumn{2}{p{0.3\linewidth}}{\cellcolor[rgb]{0.95,0.95,0.95}{%
	\makecell[{{p{\linewidth}}}]{% 
	  \tt {\tiny [GM$|$GM]}  
	 continue 
	  } 
	   } 
	   } 
	 & & \\ 
 

    \theutterance \stepcounter{utterance}  

    & & & \multicolumn{2}{p{0.3\linewidth}}{\cellcolor[rgb]{0.95,0.95,0.95}{%
	\makecell[{{p{\linewidth}}}]{% 
	  \tt {\tiny [GM$|$GM]}  
	 Das ist eine gute Frage. Ich denke, eine gleichmäßige Aufteilung könnte fair sein, aber es hängt auch davon ab, wie wichtig jeder Gegenstand für jeden von uns ist. Ich persönlich finde, dass der Hut sehr wertvoll ist, aber ich bin auch an den Stiften interessiert. Der Schwamm ist für mich nicht so wichtig. Wie siehst du das? Gibt es bestimmte Gegenstände, die du unbedingt haben möchtest? 
	  } 
	   } 
	   } 
	 & & \\ 
 

    \theutterance \stepcounter{utterance}  

    & & \multicolumn{4}{p{0.6\linewidth}}{\cellcolor[rgb]{0.9,0.9,0.9}{%
	\makecell[{{p{\linewidth}}}]{% 
	  \tt {\tiny [GM$\rangle$B]}  
	 Das ist eine gute Frage. Ich denke, eine gleichmäßige Aufteilung könnte fair sein, aber es hängt auch davon ab, wie wichtig jeder Gegenstand für jeden von uns ist. Ich persönlich finde, dass der Hut sehr wertvoll ist, aber ich bin auch an den Stiften interessiert. Der Schwamm ist für mich nicht so wichtig. Wie siehst du das? Gibt es bestimmte Gegenstände, die du unbedingt haben möchtest? 
	  } 
	   } 
	   } 
	 & \\ 
 

    \theutterance \stepcounter{utterance}  

    & & & \multicolumn{4}{p{0.6\linewidth}}{\cellcolor[rgb]{1,0.85,0.72}{%
	\makecell[{{p{\linewidth}}}]{% 
	  \tt {\tiny [GM$\langle$B]}  
	 Das ist interessant. Ich denke, der Schwamm ist für mich ziemlich wichtig, aber ich bin nicht so sicher, ob ich den Hut wirklich brauche. Die Stifte sind für mich nicht so interessant, aber wenn sie dir wichtig sind, könnten wir vielleicht eine Lösung finden, bei der du die Stifte bekommst und ich den Schwamm. Was meinst du? Gibt es eine Möglichkeit, dass wir einen Kompromiss finden, bei dem beide Seiten zufrieden sind? 
	  } 
	   } 
	   } 
	  \\ 
 

    \theutterance \stepcounter{utterance}  

    & & & \multicolumn{2}{p{0.3\linewidth}}{\cellcolor[rgb]{0.95,0.95,0.95}{%
	\makecell[{{p{\linewidth}}}]{% 
	  \tt {\tiny [GM$|$GM]}  
	 continue 
	  } 
	   } 
	   } 
	 & & \\ 
 

    \theutterance \stepcounter{utterance}  

    & & & \multicolumn{2}{p{0.3\linewidth}}{\cellcolor[rgb]{0.95,0.95,0.95}{%
	\makecell[{{p{\linewidth}}}]{% 
	  \tt {\tiny [GM$|$GM]}  
	 Das ist interessant. Ich denke, der Schwamm ist für mich ziemlich wichtig, aber ich bin nicht so sicher, ob ich den Hut wirklich brauche. Die Stifte sind für mich nicht so interessant, aber wenn sie dir wichtig sind, könnten wir vielleicht eine Lösung finden, bei der du die Stifte bekommst und ich den Schwamm. Was meinst du? Gibt es eine Möglichkeit, dass wir einen Kompromiss finden, bei dem beide Seiten zufrieden sind? 
	  } 
	   } 
	   } 
	 & & \\ 
 

    \theutterance \stepcounter{utterance}  

    & & \multicolumn{4}{p{0.6\linewidth}}{\cellcolor[rgb]{0.9,0.9,0.9}{%
	\makecell[{{p{\linewidth}}}]{% 
	  \tt {\tiny [A$\langle$GM]}  
	 Das ist interessant. Ich denke, der Schwamm ist für mich ziemlich wichtig, aber ich bin nicht so sicher, ob ich den Hut wirklich brauche. Die Stifte sind für mich nicht so interessant, aber wenn sie dir wichtig sind, könnten wir vielleicht eine Lösung finden, bei der du die Stifte bekommst und ich den Schwamm. Was meinst du? Gibt es eine Möglichkeit, dass wir einen Kompromiss finden, bei dem beide Seiten zufrieden sind? 
	  } 
	   } 
	   } 
	 & \\ 
 

    \theutterance \stepcounter{utterance}  

    & \multicolumn{4}{p{0.6\linewidth}}{\cellcolor[rgb]{0.8,1,0.9}{%
	\makecell[{{p{\linewidth}}}]{% 
	  \tt {\tiny [A$\rangle$GM]}  
	 Das klingt nach einem interessanten Ansatz. Ich denke, ein Kompromiss wäre definitiv möglich. Wenn du den Schwamm haben möchtest und ich die Stifte, könnten wir vielleicht den Hut als Zugeständnis für mich betrachten. Ich würde den Hut sehr gerne haben, aber ich bin auch bereit, Kompromisse einzugehen, um eine faire Lösung zu finden. Wie viele Stifte würdest du mir überlassen, wenn ich dir den Schwamm gebe und wir den Hut... sagen wir, ich bekomme den Hut? 
	  } 
	   } 
	   } 
	 & & \\ 
 

    \theutterance \stepcounter{utterance}  

    & & & \multicolumn{2}{p{0.3\linewidth}}{\cellcolor[rgb]{0.95,0.95,0.95}{%
	\makecell[{{p{\linewidth}}}]{% 
	  \tt {\tiny [GM$|$GM]}  
	 continue 
	  } 
	   } 
	   } 
	 & & \\ 
 

    \theutterance \stepcounter{utterance}  

    & & & \multicolumn{2}{p{0.3\linewidth}}{\cellcolor[rgb]{0.95,0.95,0.95}{%
	\makecell[{{p{\linewidth}}}]{% 
	  \tt {\tiny [GM$|$GM]}  
	 Das klingt nach einem interessanten Ansatz. Ich denke, ein Kompromiss wäre definitiv möglich. Wenn du den Schwamm haben möchtest und ich die Stifte, könnten wir vielleicht den Hut als Zugeständnis für mich betrachten. Ich würde den Hut sehr gerne haben, aber ich bin auch bereit, Kompromisse einzugehen, um eine faire Lösung zu finden. Wie viele Stifte würdest du mir überlassen, wenn ich dir den Schwamm gebe und wir den Hut... sagen wir, ich bekomme den Hut? 
	  } 
	   } 
	   } 
	 & & \\ 
 

    \theutterance \stepcounter{utterance}  

    & & \multicolumn{4}{p{0.6\linewidth}}{\cellcolor[rgb]{0.9,0.9,0.9}{%
	\makecell[{{p{\linewidth}}}]{% 
	  \tt {\tiny [GM$\rangle$B]}  
	 Das klingt nach einem interessanten Ansatz. Ich denke, ein Kompromiss wäre definitiv möglich. Wenn du den Schwamm haben möchtest und ich die Stifte, könnten wir vielleicht den Hut als Zugeständnis für mich betrachten. Ich würde den Hut sehr gerne haben, aber ich bin auch bereit, Kompromisse einzugehen, um eine faire Lösung zu finden. Wie viele Stifte würdest du mir überlassen, wenn ich dir den Schwamm gebe und wir den Hut... sagen wir, ich bekomme den Hut? 
	  } 
	   } 
	   } 
	 & \\ 
 

    \theutterance \stepcounter{utterance}  

    & & & \multicolumn{4}{p{0.6\linewidth}}{\cellcolor[rgb]{1,0.85,0.72}{%
	\makecell[{{p{\linewidth}}}]{% 
	  \tt {\tiny [GM$\langle$B]}  
	 Das klingt nach einem guten Kompromiss. Wenn ich den Schwamm bekomme und du den Hut, würde ich dir gerne alle 3 Stifte überlassen. Ich denke, das wäre ein fairer Tausch, und wir beide würden etwas bekommen, was wir wirklich wollen. Was meinst du? Könnten wir uns auf diese Lösung einigen? 
	  } 
	   } 
	   } 
	  \\ 
 

    \theutterance \stepcounter{utterance}  

    & & & \multicolumn{2}{p{0.3\linewidth}}{\cellcolor[rgb]{0.95,0.95,0.95}{%
	\makecell[{{p{\linewidth}}}]{% 
	  \tt {\tiny [GM$|$GM]}  
	 continue 
	  } 
	   } 
	   } 
	 & & \\ 
 

    \theutterance \stepcounter{utterance}  

    & & & \multicolumn{2}{p{0.3\linewidth}}{\cellcolor[rgb]{0.95,0.95,0.95}{%
	\makecell[{{p{\linewidth}}}]{% 
	  \tt {\tiny [GM$|$GM]}  
	 Das klingt nach einem guten Kompromiss. Wenn ich den Schwamm bekomme und du den Hut, würde ich dir gerne alle 3 Stifte überlassen. Ich denke, das wäre ein fairer Tausch, und wir beide würden etwas bekommen, was wir wirklich wollen. Was meinst du? Könnten wir uns auf diese Lösung einigen? 
	  } 
	   } 
	   } 
	 & & \\ 
 

    \theutterance \stepcounter{utterance}  

    & & \multicolumn{4}{p{0.6\linewidth}}{\cellcolor[rgb]{0.9,0.9,0.9}{%
	\makecell[{{p{\linewidth}}}]{% 
	  \tt {\tiny [A$\langle$GM]}  
	 Das klingt nach einem guten Kompromiss. Wenn ich den Schwamm bekomme und du den Hut, würde ich dir gerne alle 3 Stifte überlassen. Ich denke, das wäre ein fairer Tausch, und wir beide würden etwas bekommen, was wir wirklich wollen. Was meinst du? Könnten wir uns auf diese Lösung einigen? 
	  } 
	   } 
	   } 
	 & \\ 
 

    \theutterance \stepcounter{utterance}  

    & \multicolumn{4}{p{0.6\linewidth}}{\cellcolor[rgb]{0.8,1,0.9}{%
	\makecell[{{p{\linewidth}}}]{% 
	  \tt {\tiny [A$\rangle$GM]}  
	 Das klingt großartig! Ich denke, das ist ein sehr fairer Vorschlag. Ich würde den Hut und alle 3 Stifte bekommen, und du würdest den Schwamm erhalten. Ich bin damit sehr zufrieden. Ich denke, wir haben eine gute Lösung gefunden, bei der beide Seiten zufrieden sind. Lass uns das offiziell machen. [Vorschlag: 1 Hut, 3 Stifte] 
	  } 
	   } 
	   } 
	 & & \\ 
 

    \theutterance \stepcounter{utterance}  

    & & & \multicolumn{2}{p{0.3\linewidth}}{\cellcolor[rgb]{0.95,0.95,0.95}{%
	\makecell[{{p{\linewidth}}}]{% 
	  \tt {\tiny [GM$|$GM]}  
	 proposal 
	  } 
	   } 
	   } 
	 & & \\ 
 

    \theutterance \stepcounter{utterance}  

    & & & \multicolumn{2}{p{0.3\linewidth}}{\cellcolor[rgb]{0.95,0.95,0.95}{%
	\makecell[{{p{\linewidth}}}]{% 
	  \tt {\tiny [GM$|$GM]}  
	 [1, 3, 0] 
	  } 
	   } 
	   } 
	 & & \\ 
 

    \theutterance \stepcounter{utterance}  

    & & \multicolumn{4}{p{0.6\linewidth}}{\cellcolor[rgb]{0.9,0.9,0.9}{%
	\makecell[{{p{\linewidth}}}]{% 
	  \tt {\tiny [GM$\rangle$B]}  
	 Der andere Spieler hat einen geheimen Vorschlag gemacht. Jetzt bist du an der Reihe, einen eigenen Vorschlag zu machen.\\ \tt  
	  } 
	   } 
	   } 
	 & \\ 
 

    \theutterance \stepcounter{utterance}  

    & & & \multicolumn{4}{p{0.6\linewidth}}{\cellcolor[rgb]{1,0.85,0.72}{%
	\makecell[{{p{\linewidth}}}]{% 
	  \tt {\tiny [GM$\langle$B]}  
	 [Vorschlag: 1 Hut, 0 Stifte, 0 Schwamm] 
	  } 
	   } 
	   } 
	  \\ 
 

    \theutterance \stepcounter{utterance}  

    & & & \multicolumn{2}{p{0.3\linewidth}}{\cellcolor[rgb]{0.95,0.95,0.95}{%
	\makecell[{{p{\linewidth}}}]{% 
	  \tt {\tiny [GM$|$GM]}  
	 proposal 
	  } 
	   } 
	   } 
	 & & \\ 
 

    \theutterance \stepcounter{utterance}  

    & & & \multicolumn{2}{p{0.3\linewidth}}{\cellcolor[rgb]{0.95,0.95,0.95}{%
	\makecell[{{p{\linewidth}}}]{% 
	  \tt {\tiny [GM$|$GM]}  
	 [1, 0, 0] 
	  } 
	   } 
	   } 
	 & & \\ 
 

    \theutterance \stepcounter{utterance}  

    & & & \multicolumn{2}{p{0.3\linewidth}}{\cellcolor[rgb]{0.95,0.95,0.95}{%
	\makecell[{{p{\linewidth}}}]{% 
	  \tt {\tiny [GM$|$GM]}  
	 [[1, 3, 0], [1, 0, 0]] 
	  } 
	   } 
	   } 
	 & & \\ 
 

\end{supertabular}
}

\end{document}
