
\documentclass{article}
\usepackage{colortbl}
\usepackage{makecell}
\usepackage{multirow}
\usepackage{supertabular}

\begin{document}

\newcounter{utterance}

\twocolumn

{ \footnotesize  \setcounter{utterance}{1}
\setlength{\tabcolsep}{0pt}
\begin{supertabular}{c@{$\;$}|p{.15\linewidth}@{}p{.15\linewidth}p{.15\linewidth}p{.15\linewidth}p{.15\linewidth}p{.15\linewidth}}

    \# & $\;$A & \multicolumn{4}{c}{Game Master} & $\;\:$B\\
    \hline 

    \theutterance \stepcounter{utterance}  

    & & \multicolumn{4}{p{0.6\linewidth}}{\cellcolor[rgb]{0.9,0.9,0.9}{%
	\makecell[{{p{\linewidth}}}]{% 
	  \tt {\tiny [A$\langle$GM]}  
	 Sie spielen ein Verhandlungsspiel, bei dem Sie sich mit einem anderen Spieler darauf einigen müssen, wie eine Reihe von Gegenständen aufgeteilt werden soll.\\ \tt \\ \tt Die Regeln:\\ \tt (a) Sie und der andere Spieler erhalten eine Sammlung von Gegenständen. Jeder von Ihnen erhält außerdem eine geheime Wertfunktion, die angibt, wie viel Ihnen jede Art von Gegenstand wert ist.\\ \tt (b) Sie tauschen Nachrichten mit dem anderen Spieler aus, um zu vereinbaren, wer welche Gegenstände bekommt. Sie können jeweils maximal 5 Nachrichten senden oder das Spiel vorzeitig beenden, indem Sie jederzeit einen geheimen Vorschlag machen.\\ \tt (c) Jeder von euch wird aufgefordert, einen geheimen Vorschlag zu machen, in dem ihr die gewünschten Gegenstände in eckigen Klammern wie folgt angibt: "[Vorschlag: <Nummer> <Objektname>, <Nummer> <Objektname>, <...>]"\\ \tt (d) Wenn eure Vorschläge komplementär sind, d.h. es gibt genug Gegenstände, um beide Vorschläge zu erfüllen, erhält jeder Spieler eine Punktzahl, die sich aus der Summe der Werte für die Gegenstände ergibt, die er erhalten hat. Andernfalls erhalten Sie beide null Punkte.\\ \tt (e) Ihr Ziel ist es, eine möglichst hohe Punktzahl zu erreichen.\\ \tt \\ \tt Beginnen wir.\\ \tt \\ \tt Die Menge der verfügbaren Gegenstände ist:\\ \tt \\ \tt 2 Würfel, 1 Ball, 1 Klebstoff, 1 Batterie\\ \tt \\ \tt Deine geheime Wertfunktion ist:\\ \tt \\ \tt Würfel: 1, Ball: 7, Klebstoff: 0, Batterie: 1\\ \tt \\ \tt WICHTIG: Ihre Nachrichten werden, sofern es sich nicht um einen geheimen Vorschlag handelt, direkt an den anderen Spieler übermittelt, also fügen Sie keine Antwort auf die Regeln oder einen Text zur Ankündigung Ihrer Nachricht ein. Um einen geheimen Vorschlag zu machen, verwenden Sie das angegebene Format. Verwenden Sie keine eckigen Klammern, wenn Sie mit dem anderen Spieler kommunizieren, sonst wird dies als Ihr geheimer Vorschlag interpretiert.\\ \tt  
	  } 
	   } 
	   } 
	 & \\ 
 

    \theutterance \stepcounter{utterance}  

    & \multicolumn{4}{p{0.6\linewidth}}{\cellcolor[rgb]{0.8,1,0.9}{%
	\makecell[{{p{\linewidth}}}]{% 
	  \tt {\tiny [A$\rangle$GM]}  
	 Hallo, ich bin bereit, zu beginnen. Was sind deine Prioritäten bei den Gegenständen? 
	  } 
	   } 
	   } 
	 & & \\ 
 

    \theutterance \stepcounter{utterance}  

    & & & \multicolumn{2}{p{0.3\linewidth}}{\cellcolor[rgb]{0.95,0.95,0.95}{%
	\makecell[{{p{\linewidth}}}]{% 
	  \tt {\tiny [GM$|$GM]}  
	 continue 
	  } 
	   } 
	   } 
	 & & \\ 
 

    \theutterance \stepcounter{utterance}  

    & & & \multicolumn{2}{p{0.3\linewidth}}{\cellcolor[rgb]{0.95,0.95,0.95}{%
	\makecell[{{p{\linewidth}}}]{% 
	  \tt {\tiny [GM$|$GM]}  
	 Hallo, ich bin bereit, zu beginnen. Was sind deine Prioritäten bei den Gegenständen? 
	  } 
	   } 
	   } 
	 & & \\ 
 

    \theutterance \stepcounter{utterance}  

    & & \multicolumn{4}{p{0.6\linewidth}}{\cellcolor[rgb]{0.9,0.9,0.9}{%
	\makecell[{{p{\linewidth}}}]{% 
	  \tt {\tiny [GM$\rangle$B]}  
	 Sie spielen ein Verhandlungsspiel, bei dem Sie sich mit einem anderen Spieler darauf einigen müssen, wie eine Reihe von Gegenständen aufgeteilt werden soll.\\ \tt \\ \tt Die Regeln:\\ \tt (a) Sie und der andere Spieler erhalten eine Sammlung von Gegenständen. Jeder von Ihnen erhält außerdem eine geheime Wertfunktion, die angibt, wie viel Ihnen jede Art von Gegenstand wert ist.\\ \tt (b) Sie tauschen Nachrichten mit dem anderen Spieler aus, um zu vereinbaren, wer welche Gegenstände bekommt. Sie können jeweils maximal 5 Nachrichten senden oder das Spiel vorzeitig beenden, indem Sie jederzeit einen geheimen Vorschlag machen.\\ \tt (c) Jeder von euch wird aufgefordert, einen geheimen Vorschlag zu machen, in dem ihr die gewünschten Gegenstände in eckigen Klammern wie folgt angibt: "[Vorschlag: <Nummer> <Objektname>, <Nummer> <Objektname>, <...>]"\\ \tt (d) Wenn eure Vorschläge komplementär sind, d.h. es gibt genug Gegenstände, um beide Vorschläge zu erfüllen, erhält jeder Spieler eine Punktzahl, die sich aus der Summe der Werte für die Gegenstände ergibt, die er erhalten hat. Andernfalls erhalten Sie beide null Punkte.\\ \tt (e) Ihr Ziel ist es, eine möglichst hohe Punktzahl zu erreichen.\\ \tt \\ \tt Beginnen wir.\\ \tt \\ \tt Die Menge der verfügbaren Gegenstände ist:\\ \tt \\ \tt 2 Würfel, 1 Ball, 1 Klebstoff, 1 Batterie\\ \tt \\ \tt Deine geheime Wertfunktion ist:\\ \tt \\ \tt Würfel: 0, Ball: 2, Klebstoff: 8, Batterie: 0\\ \tt \\ \tt WICHTIG: Ihre Nachrichten werden, sofern es sich nicht um einen geheimen Vorschlag handelt, direkt an den anderen Spieler übermittelt, also fügen Sie keine Antwort auf die Regeln oder einen Text zur Ankündigung Ihrer Nachricht ein. Um einen geheimen Vorschlag zu machen, verwenden Sie das angegebene Format. Verwenden Sie keine eckigen Klammern, wenn Sie mit dem anderen Spieler kommunizieren, sonst wird dies als Ihr geheimer Vorschlag interpretiert.\\ \tt \\ \tt \\ \tt Hallo, ich bin bereit, zu beginnen. Was sind deine Prioritäten bei den Gegenständen? 
	  } 
	   } 
	   } 
	 & \\ 
 

    \theutterance \stepcounter{utterance}  

    & & & \multicolumn{4}{p{0.6\linewidth}}{\cellcolor[rgb]{1,0.85,0.72}{%
	\makecell[{{p{\linewidth}}}]{% 
	  \tt {\tiny [GM$\langle$B]}  
	 Hallo! Ich bin auch bereit, loszulegen. Ich denke, wir sollten uns zunächst auf die Gegenstände konzentrieren, die für uns beide am wertvollsten sind. Ich bin an dem Klebstoff sehr interessiert, weil ich denke, dass er sehr nützlich sein könnte. Wie sieht es bei dir aus? Hast du bestimmte Prioritäten oder bevorzugst du bestimmte Gegenstände? 
	  } 
	   } 
	   } 
	  \\ 
 

    \theutterance \stepcounter{utterance}  

    & & & \multicolumn{2}{p{0.3\linewidth}}{\cellcolor[rgb]{0.95,0.95,0.95}{%
	\makecell[{{p{\linewidth}}}]{% 
	  \tt {\tiny [GM$|$GM]}  
	 continue 
	  } 
	   } 
	   } 
	 & & \\ 
 

    \theutterance \stepcounter{utterance}  

    & & & \multicolumn{2}{p{0.3\linewidth}}{\cellcolor[rgb]{0.95,0.95,0.95}{%
	\makecell[{{p{\linewidth}}}]{% 
	  \tt {\tiny [GM$|$GM]}  
	 Hallo! Ich bin auch bereit, loszulegen. Ich denke, wir sollten uns zunächst auf die Gegenstände konzentrieren, die für uns beide am wertvollsten sind. Ich bin an dem Klebstoff sehr interessiert, weil ich denke, dass er sehr nützlich sein könnte. Wie sieht es bei dir aus? Hast du bestimmte Prioritäten oder bevorzugst du bestimmte Gegenstände? 
	  } 
	   } 
	   } 
	 & & \\ 
 

    \theutterance \stepcounter{utterance}  

    & & \multicolumn{4}{p{0.6\linewidth}}{\cellcolor[rgb]{0.9,0.9,0.9}{%
	\makecell[{{p{\linewidth}}}]{% 
	  \tt {\tiny [A$\langle$GM]}  
	 Hallo! Ich bin auch bereit, loszulegen. Ich denke, wir sollten uns zunächst auf die Gegenstände konzentrieren, die für uns beide am wertvollsten sind. Ich bin an dem Klebstoff sehr interessiert, weil ich denke, dass er sehr nützlich sein könnte. Wie sieht es bei dir aus? Hast du bestimmte Prioritäten oder bevorzugst du bestimmte Gegenstände? 
	  } 
	   } 
	   } 
	 & \\ 
 

    \theutterance \stepcounter{utterance}  

    & \multicolumn{4}{p{0.6\linewidth}}{\cellcolor[rgb]{0.8,1,0.9}{%
	\makecell[{{p{\linewidth}}}]{% 
	  \tt {\tiny [A$\rangle$GM]}  
	 Das ist interessant, dass du den Klebstoff so schätzt! Bei mir liegt der Fokus eher auf dem Ball. Ich denke, er könnte sehr wertvoll sein, und ich bin bereit, einen fairen Deal zu machen, um ihn zu bekommen. Die Würfel und die Batterie sind für mich auch interessant, aber der Ball hat für mich Priorität. Was denkst du, könnten wir einen Kompromiss finden, der für beide Seiten vorteilhaft ist? 
	  } 
	   } 
	   } 
	 & & \\ 
 

    \theutterance \stepcounter{utterance}  

    & & & \multicolumn{2}{p{0.3\linewidth}}{\cellcolor[rgb]{0.95,0.95,0.95}{%
	\makecell[{{p{\linewidth}}}]{% 
	  \tt {\tiny [GM$|$GM]}  
	 continue 
	  } 
	   } 
	   } 
	 & & \\ 
 

    \theutterance \stepcounter{utterance}  

    & & & \multicolumn{2}{p{0.3\linewidth}}{\cellcolor[rgb]{0.95,0.95,0.95}{%
	\makecell[{{p{\linewidth}}}]{% 
	  \tt {\tiny [GM$|$GM]}  
	 Das ist interessant, dass du den Klebstoff so schätzt! Bei mir liegt der Fokus eher auf dem Ball. Ich denke, er könnte sehr wertvoll sein, und ich bin bereit, einen fairen Deal zu machen, um ihn zu bekommen. Die Würfel und die Batterie sind für mich auch interessant, aber der Ball hat für mich Priorität. Was denkst du, könnten wir einen Kompromiss finden, der für beide Seiten vorteilhaft ist? 
	  } 
	   } 
	   } 
	 & & \\ 
 

    \theutterance \stepcounter{utterance}  

    & & \multicolumn{4}{p{0.6\linewidth}}{\cellcolor[rgb]{0.9,0.9,0.9}{%
	\makecell[{{p{\linewidth}}}]{% 
	  \tt {\tiny [GM$\rangle$B]}  
	 Das ist interessant, dass du den Klebstoff so schätzt! Bei mir liegt der Fokus eher auf dem Ball. Ich denke, er könnte sehr wertvoll sein, und ich bin bereit, einen fairen Deal zu machen, um ihn zu bekommen. Die Würfel und die Batterie sind für mich auch interessant, aber der Ball hat für mich Priorität. Was denkst du, könnten wir einen Kompromiss finden, der für beide Seiten vorteilhaft ist? 
	  } 
	   } 
	   } 
	 & \\ 
 

    \theutterance \stepcounter{utterance}  

    & & & \multicolumn{4}{p{0.6\linewidth}}{\cellcolor[rgb]{1,0.85,0.72}{%
	\makecell[{{p{\linewidth}}}]{% 
	  \tt {\tiny [GM$\langle$B]}  
	 Ein Kompromiss klingt nach einer guten Idee! Ich bin auch bereit, über einen fairen Deal nachzudenken. Ich denke, es wäre großartig, wenn wir eine Lösung finden könnten, bei der wir beide zufrieden sind. Da du den Ball so sehr schätzt, könnte ich dir eventuell den Ball anbieten, wenn ich im Gegenzug den Klebstoff bekomme. Was meinst du? Würdest du auch bereit sein, einen der Würfel abzugeben, oder hast du andere Vorstellungen, wie wir den Deal gestalten könnten? 
	  } 
	   } 
	   } 
	  \\ 
 

    \theutterance \stepcounter{utterance}  

    & & & \multicolumn{2}{p{0.3\linewidth}}{\cellcolor[rgb]{0.95,0.95,0.95}{%
	\makecell[{{p{\linewidth}}}]{% 
	  \tt {\tiny [GM$|$GM]}  
	 continue 
	  } 
	   } 
	   } 
	 & & \\ 
 

    \theutterance \stepcounter{utterance}  

    & & & \multicolumn{2}{p{0.3\linewidth}}{\cellcolor[rgb]{0.95,0.95,0.95}{%
	\makecell[{{p{\linewidth}}}]{% 
	  \tt {\tiny [GM$|$GM]}  
	 Ein Kompromiss klingt nach einer guten Idee! Ich bin auch bereit, über einen fairen Deal nachzudenken. Ich denke, es wäre großartig, wenn wir eine Lösung finden könnten, bei der wir beide zufrieden sind. Da du den Ball so sehr schätzt, könnte ich dir eventuell den Ball anbieten, wenn ich im Gegenzug den Klebstoff bekomme. Was meinst du? Würdest du auch bereit sein, einen der Würfel abzugeben, oder hast du andere Vorstellungen, wie wir den Deal gestalten könnten? 
	  } 
	   } 
	   } 
	 & & \\ 
 

    \theutterance \stepcounter{utterance}  

    & & \multicolumn{4}{p{0.6\linewidth}}{\cellcolor[rgb]{0.9,0.9,0.9}{%
	\makecell[{{p{\linewidth}}}]{% 
	  \tt {\tiny [A$\langle$GM]}  
	 Ein Kompromiss klingt nach einer guten Idee! Ich bin auch bereit, über einen fairen Deal nachzudenken. Ich denke, es wäre großartig, wenn wir eine Lösung finden könnten, bei der wir beide zufrieden sind. Da du den Ball so sehr schätzt, könnte ich dir eventuell den Ball anbieten, wenn ich im Gegenzug den Klebstoff bekomme. Was meinst du? Würdest du auch bereit sein, einen der Würfel abzugeben, oder hast du andere Vorstellungen, wie wir den Deal gestalten könnten? 
	  } 
	   } 
	   } 
	 & \\ 
 

    \theutterance \stepcounter{utterance}  

    & \multicolumn{4}{p{0.6\linewidth}}{\cellcolor[rgb]{0.8,1,0.9}{%
	\makecell[{{p{\linewidth}}}]{% 
	  \tt {\tiny [A$\rangle$GM]}  
	 Das ist ein interessantes Angebot! Der Ball ist für mich sehr wichtig, und ich denke, es wäre großartig, wenn ich ihn bekommen könnte. Ich bin auch bereit, einen Würfel abzugeben, um den Deal zu besiegeln. Ich denke, ein fairer Tausch wäre, wenn ich den Ball und einen Würfel bekomme und du den Klebstoff und den anderen Würfel bekommst. Die Batterie könnten wir dann noch separat besprechen. Was meinst du? Würde dir dieser Vorschlag zusagen? 
	  } 
	   } 
	   } 
	 & & \\ 
 

    \theutterance \stepcounter{utterance}  

    & & & \multicolumn{2}{p{0.3\linewidth}}{\cellcolor[rgb]{0.95,0.95,0.95}{%
	\makecell[{{p{\linewidth}}}]{% 
	  \tt {\tiny [GM$|$GM]}  
	 continue 
	  } 
	   } 
	   } 
	 & & \\ 
 

    \theutterance \stepcounter{utterance}  

    & & & \multicolumn{2}{p{0.3\linewidth}}{\cellcolor[rgb]{0.95,0.95,0.95}{%
	\makecell[{{p{\linewidth}}}]{% 
	  \tt {\tiny [GM$|$GM]}  
	 Das ist ein interessantes Angebot! Der Ball ist für mich sehr wichtig, und ich denke, es wäre großartig, wenn ich ihn bekommen könnte. Ich bin auch bereit, einen Würfel abzugeben, um den Deal zu besiegeln. Ich denke, ein fairer Tausch wäre, wenn ich den Ball und einen Würfel bekomme und du den Klebstoff und den anderen Würfel bekommst. Die Batterie könnten wir dann noch separat besprechen. Was meinst du? Würde dir dieser Vorschlag zusagen? 
	  } 
	   } 
	   } 
	 & & \\ 
 

    \theutterance \stepcounter{utterance}  

    & & \multicolumn{4}{p{0.6\linewidth}}{\cellcolor[rgb]{0.9,0.9,0.9}{%
	\makecell[{{p{\linewidth}}}]{% 
	  \tt {\tiny [GM$\rangle$B]}  
	 Das ist ein interessantes Angebot! Der Ball ist für mich sehr wichtig, und ich denke, es wäre großartig, wenn ich ihn bekommen könnte. Ich bin auch bereit, einen Würfel abzugeben, um den Deal zu besiegeln. Ich denke, ein fairer Tausch wäre, wenn ich den Ball und einen Würfel bekomme und du den Klebstoff und den anderen Würfel bekommst. Die Batterie könnten wir dann noch separat besprechen. Was meinst du? Würde dir dieser Vorschlag zusagen? 
	  } 
	   } 
	   } 
	 & \\ 
 

    \theutterance \stepcounter{utterance}  

    & & & \multicolumn{4}{p{0.6\linewidth}}{\cellcolor[rgb]{1,0.85,0.72}{%
	\makecell[{{p{\linewidth}}}]{% 
	  \tt {\tiny [GM$\langle$B]}  
	 Das klingt nach einem sehr vernünftigen Vorschlag! Ich denke, es ist fair, dass wir den Ball und den Klebstoff wie besprochen tauschen. Die Aufteilung der Würfel ist auch okay für mich. Die Batterie ist für mich nicht so wichtig, also könnten wir sie eventuell aus dem Deal herauslassen oder sie dir anbieten, wenn du bereit bist, einen kleinen Zusatz zu bieten. Aber ich denke, wir sind auf dem richtigen Weg! Lass uns nochmal alles durchgehen, um sicherzustellen, dass wir beide zufrieden sind, und dann können wir unsere endgültigen Vorschläge machen. Wie sieht es aus, bist du bereit, deinen endgültigen Vorschlag zu machen? 
	  } 
	   } 
	   } 
	  \\ 
 

    \theutterance \stepcounter{utterance}  

    & & & \multicolumn{2}{p{0.3\linewidth}}{\cellcolor[rgb]{0.95,0.95,0.95}{%
	\makecell[{{p{\linewidth}}}]{% 
	  \tt {\tiny [GM$|$GM]}  
	 continue 
	  } 
	   } 
	   } 
	 & & \\ 
 

    \theutterance \stepcounter{utterance}  

    & & & \multicolumn{2}{p{0.3\linewidth}}{\cellcolor[rgb]{0.95,0.95,0.95}{%
	\makecell[{{p{\linewidth}}}]{% 
	  \tt {\tiny [GM$|$GM]}  
	 Das klingt nach einem sehr vernünftigen Vorschlag! Ich denke, es ist fair, dass wir den Ball und den Klebstoff wie besprochen tauschen. Die Aufteilung der Würfel ist auch okay für mich. Die Batterie ist für mich nicht so wichtig, also könnten wir sie eventuell aus dem Deal herauslassen oder sie dir anbieten, wenn du bereit bist, einen kleinen Zusatz zu bieten. Aber ich denke, wir sind auf dem richtigen Weg! Lass uns nochmal alles durchgehen, um sicherzustellen, dass wir beide zufrieden sind, und dann können wir unsere endgültigen Vorschläge machen. Wie sieht es aus, bist du bereit, deinen endgültigen Vorschlag zu machen? 
	  } 
	   } 
	   } 
	 & & \\ 
 

    \theutterance \stepcounter{utterance}  

    & & \multicolumn{4}{p{0.6\linewidth}}{\cellcolor[rgb]{0.9,0.9,0.9}{%
	\makecell[{{p{\linewidth}}}]{% 
	  \tt {\tiny [A$\langle$GM]}  
	 Das klingt nach einem sehr vernünftigen Vorschlag! Ich denke, es ist fair, dass wir den Ball und den Klebstoff wie besprochen tauschen. Die Aufteilung der Würfel ist auch okay für mich. Die Batterie ist für mich nicht so wichtig, also könnten wir sie eventuell aus dem Deal herauslassen oder sie dir anbieten, wenn du bereit bist, einen kleinen Zusatz zu bieten. Aber ich denke, wir sind auf dem richtigen Weg! Lass uns nochmal alles durchgehen, um sicherzustellen, dass wir beide zufrieden sind, und dann können wir unsere endgültigen Vorschläge machen. Wie sieht es aus, bist du bereit, deinen endgültigen Vorschlag zu machen? 
	  } 
	   } 
	   } 
	 & \\ 
 

    \theutterance \stepcounter{utterance}  

    & \multicolumn{4}{p{0.6\linewidth}}{\cellcolor[rgb]{0.8,1,0.9}{%
	\makecell[{{p{\linewidth}}}]{% 
	  \tt {\tiny [A$\rangle$GM]}  
	 Ja, ich denke, wir haben eine gute Grundlage für einen fairen Deal. Ich bin bereit, meinen endgültigen Vorschlag zu machen. Ich denke, wir sollten unsere Vorschläge jetzt machen, um sicherzustellen, dass wir beide zufrieden sind und dass der Deal für beide Seiten fair ist. Ich werde jetzt meinen Vorschlag machen, und ich hoffe, dass er deinen Erwartungen entspricht. [Vorschlag: 1 Ball, 1 Würfel, 1 Batterie] 
	  } 
	   } 
	   } 
	 & & \\ 
 

    \theutterance \stepcounter{utterance}  

    & & & \multicolumn{2}{p{0.3\linewidth}}{\cellcolor[rgb]{0.95,0.95,0.95}{%
	\makecell[{{p{\linewidth}}}]{% 
	  \tt {\tiny [GM$|$GM]}  
	 proposal 
	  } 
	   } 
	   } 
	 & & \\ 
 

    \theutterance \stepcounter{utterance}  

    & & & \multicolumn{2}{p{0.3\linewidth}}{\cellcolor[rgb]{0.95,0.95,0.95}{%
	\makecell[{{p{\linewidth}}}]{% 
	  \tt {\tiny [GM$|$GM]}  
	 [1, 1, 0, 1] 
	  } 
	   } 
	   } 
	 & & \\ 
 

    \theutterance \stepcounter{utterance}  

    & & \multicolumn{4}{p{0.6\linewidth}}{\cellcolor[rgb]{0.9,0.9,0.9}{%
	\makecell[{{p{\linewidth}}}]{% 
	  \tt {\tiny [GM$\rangle$B]}  
	 Der andere Spieler hat einen geheimen Vorschlag gemacht. Jetzt bist du an der Reihe, einen eigenen Vorschlag zu machen.\\ \tt  
	  } 
	   } 
	   } 
	 & \\ 
 

    \theutterance \stepcounter{utterance}  

    & & & \multicolumn{4}{p{0.6\linewidth}}{\cellcolor[rgb]{1,0.85,0.72}{%
	\makecell[{{p{\linewidth}}}]{% 
	  \tt {\tiny [GM$\langle$B]}  
	 [Vorschlag: 1 Ball, 1 Würfel] 
	  } 
	   } 
	   } 
	  \\ 
 

    \theutterance \stepcounter{utterance}  

    & & & \multicolumn{2}{p{0.3\linewidth}}{\cellcolor[rgb]{0.95,0.95,0.95}{%
	\makecell[{{p{\linewidth}}}]{% 
	  \tt {\tiny [GM$|$GM]}  
	 proposal 
	  } 
	   } 
	   } 
	 & & \\ 
 

    \theutterance \stepcounter{utterance}  

    & & & \multicolumn{2}{p{0.3\linewidth}}{\cellcolor[rgb]{0.95,0.95,0.95}{%
	\makecell[{{p{\linewidth}}}]{% 
	  \tt {\tiny [GM$|$GM]}  
	 [1, 1, 0, 0] 
	  } 
	   } 
	   } 
	 & & \\ 
 

    \theutterance \stepcounter{utterance}  

    & & & \multicolumn{2}{p{0.3\linewidth}}{\cellcolor[rgb]{0.95,0.95,0.95}{%
	\makecell[{{p{\linewidth}}}]{% 
	  \tt {\tiny [GM$|$GM]}  
	 [[1, 1, 0, 1], [1, 1, 0, 0]] 
	  } 
	   } 
	   } 
	 & & \\ 
 

\end{supertabular}
}

\end{document}
