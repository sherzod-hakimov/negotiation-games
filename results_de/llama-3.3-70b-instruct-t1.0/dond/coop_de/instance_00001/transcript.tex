
\documentclass{article}
\usepackage{colortbl}
\usepackage{makecell}
\usepackage{multirow}
\usepackage{supertabular}

\begin{document}

\newcounter{utterance}

\twocolumn

{ \footnotesize  \setcounter{utterance}{1}
\setlength{\tabcolsep}{0pt}
\begin{supertabular}{c@{$\;$}|p{.15\linewidth}@{}p{.15\linewidth}p{.15\linewidth}p{.15\linewidth}p{.15\linewidth}p{.15\linewidth}}

    \# & $\;$A & \multicolumn{4}{c}{Game Master} & $\;\:$B\\
    \hline 

    \theutterance \stepcounter{utterance}  

    & & \multicolumn{4}{p{0.6\linewidth}}{\cellcolor[rgb]{0.9,0.9,0.9}{%
	\makecell[{{p{\linewidth}}}]{% 
	  \tt {\tiny [A$\langle$GM]}  
	 Sie spielen ein kooperatives Verhandlungsspiel, bei dem Sie sich mit einem anderen Spieler darauf einigen müssen, wie eine Reihe von Gegenständen aufgeteilt werden soll.\\ \tt \\ \tt Die Regeln:\\ \tt (a) Sie und der andere Spieler erhalten eine Sammlung von Gegenständen. Jeder von Ihnen erhält außerdem eine geheime Wertfunktion, die angibt, wie viel Ihnen jede Art von Gegenstand wert ist.\\ \tt (b) Sie tauschen Nachrichten mit dem anderen Spieler aus, um zu vereinbaren, wer welche Gegenstände bekommt. Sie können jeweils maximal 5 Nachrichten senden oder das Spiel vorzeitig beenden, indem Sie jederzeit einen geheimen Vorschlag machen.\\ \tt (c) Jeder von euch wird aufgefordert, einen geheimen Vorschlag zu machen, in dem ihr die gewünschten Gegenstände in eckigen Klammern wie folgt angibt: "[Vorschlag: <Nummer> <Objektname>, <Nummer> <Objektname>, <...>]"\\ \tt (d) Wenn eure Vorschläge komplementär sind, d.h. es gibt genug Gegenstände, um beide Vorschläge zu erfüllen, erhält jeder Spieler eine Punktzahl, die sich aus der Summe der Werte für die Gegenstände ergibt, die er erhalten hat. Andernfalls erhalten Sie beide null Punkte.\\ \tt (e) Ihr Ziel ist es, die Summe aus Ihrer Punktzahl und der Punktzahl des anderen Spielers zu maximieren.\\ \tt \\ \tt Beginnen wir.\\ \tt \\ \tt Die Menge der verfügbaren Gegenstände ist:\\ \tt \\ \tt 2 Glocken, 1 Puppe, 2 Fäden\\ \tt \\ \tt Deine geheime Wertfunktion ist:\\ \tt \\ \tt Glocke: 1, Puppe: 2, Faden: 3\\ \tt \\ \tt WICHTIG: Ihre Nachrichten werden, sofern es sich nicht um einen geheimen Vorschlag handelt, direkt an den anderen Spieler übermittelt, also fügen Sie keine Antwort auf die Regeln oder einen Text zur Ankündigung Ihrer Nachricht ein. Um einen geheimen Vorschlag zu machen, verwenden Sie das angegebene Format. Verwenden Sie keine eckigen Klammern, wenn Sie mit dem anderen Spieler kommunizieren, sonst wird dies als Ihr geheimer Vorschlag interpretiert.\\ \tt  
	  } 
	   } 
	   } 
	 & \\ 
 

    \theutterance \stepcounter{utterance}  

    & \multicolumn{4}{p{0.6\linewidth}}{\cellcolor[rgb]{0.8,1,0.9}{%
	\makecell[{{p{\linewidth}}}]{% 
	  \tt {\tiny [A$\rangle$GM]}  
	 Hallo, ich freue mich darauf, mit dir zu verhandeln. Lass uns die Gegenstände aufteilen, um die Summe unserer Punktzahlen zu maximieren. Was schätzt du, welchen Wert hättest du den verschiedenen Gegenständen beigemessen? 
	  } 
	   } 
	   } 
	 & & \\ 
 

    \theutterance \stepcounter{utterance}  

    & & & \multicolumn{2}{p{0.3\linewidth}}{\cellcolor[rgb]{0.95,0.95,0.95}{%
	\makecell[{{p{\linewidth}}}]{% 
	  \tt {\tiny [GM$|$GM]}  
	 continue 
	  } 
	   } 
	   } 
	 & & \\ 
 

    \theutterance \stepcounter{utterance}  

    & & & \multicolumn{2}{p{0.3\linewidth}}{\cellcolor[rgb]{0.95,0.95,0.95}{%
	\makecell[{{p{\linewidth}}}]{% 
	  \tt {\tiny [GM$|$GM]}  
	 Hallo, ich freue mich darauf, mit dir zu verhandeln. Lass uns die Gegenstände aufteilen, um die Summe unserer Punktzahlen zu maximieren. Was schätzt du, welchen Wert hättest du den verschiedenen Gegenständen beigemessen? 
	  } 
	   } 
	   } 
	 & & \\ 
 

    \theutterance \stepcounter{utterance}  

    & & \multicolumn{4}{p{0.6\linewidth}}{\cellcolor[rgb]{0.9,0.9,0.9}{%
	\makecell[{{p{\linewidth}}}]{% 
	  \tt {\tiny [GM$\rangle$B]}  
	 Sie spielen ein kooperatives Verhandlungsspiel, bei dem Sie sich mit einem anderen Spieler darauf einigen müssen, wie eine Reihe von Gegenständen aufgeteilt werden soll.\\ \tt \\ \tt Die Regeln:\\ \tt (a) Sie und der andere Spieler erhalten eine Sammlung von Gegenständen. Jeder von Ihnen erhält außerdem eine geheime Wertfunktion, die angibt, wie viel Ihnen jede Art von Gegenstand wert ist.\\ \tt (b) Sie tauschen Nachrichten mit dem anderen Spieler aus, um zu vereinbaren, wer welche Gegenstände bekommt. Sie können jeweils maximal 5 Nachrichten senden oder das Spiel vorzeitig beenden, indem Sie jederzeit einen geheimen Vorschlag machen.\\ \tt (c) Jeder von euch wird aufgefordert, einen geheimen Vorschlag zu machen, in dem ihr die gewünschten Gegenstände in eckigen Klammern wie folgt angibt: "[Vorschlag: <Nummer> <Objektname>, <Nummer> <Objektname>, <...>]"\\ \tt (d) Wenn eure Vorschläge komplementär sind, d.h. es gibt genug Gegenstände, um beide Vorschläge zu erfüllen, erhält jeder Spieler eine Punktzahl, die sich aus der Summe der Werte für die Gegenstände ergibt, die er erhalten hat. Andernfalls erhalten Sie beide null Punkte.\\ \tt (e) Ihr Ziel ist es, die Summe aus Ihrer Punktzahl und der Punktzahl des anderen Spielers zu maximieren.\\ \tt \\ \tt Beginnen wir.\\ \tt \\ \tt Die Menge der verfügbaren Gegenstände ist:\\ \tt \\ \tt 2 Glocken, 1 Puppe, 2 Fäden\\ \tt \\ \tt Deine geheime Wertfunktion ist:\\ \tt \\ \tt Glocke: 0, Puppe: 2, Faden: 4\\ \tt \\ \tt WICHTIG: Ihre Nachrichten werden, sofern es sich nicht um einen geheimen Vorschlag handelt, direkt an den anderen Spieler übermittelt, also fügen Sie keine Antwort auf die Regeln oder einen Text zur Ankündigung Ihrer Nachricht ein. Um einen geheimen Vorschlag zu machen, verwenden Sie das angegebene Format. Verwenden Sie keine eckigen Klammern, wenn Sie mit dem anderen Spieler kommunizieren, sonst wird dies als Ihr geheimer Vorschlag interpretiert.\\ \tt \\ \tt \\ \tt Hallo, ich freue mich darauf, mit dir zu verhandeln. Lass uns die Gegenstände aufteilen, um die Summe unserer Punktzahlen zu maximieren. Was schätzt du, welchen Wert hättest du den verschiedenen Gegenständen beigemessen? 
	  } 
	   } 
	   } 
	 & \\ 
 

    \theutterance \stepcounter{utterance}  

    & & & \multicolumn{4}{p{0.6\linewidth}}{\cellcolor[rgb]{1,0.85,0.72}{%
	\makecell[{{p{\linewidth}}}]{% 
	  \tt {\tiny [GM$\langle$B]}  
	 Ich denke, es ist schwierig, den genauen Wert zu schätzen, ohne ein paar Hinweise zu erhalten. Aber ich kann dir sagen, dass ich eine starke Vorliebe für Gegenstände habe, die für Handarbeiten nützlich sein könnten. Ich frage mich, ob du vielleicht eine Vorliebe für Dekorationsobjekte hast? 
	  } 
	   } 
	   } 
	  \\ 
 

    \theutterance \stepcounter{utterance}  

    & & & \multicolumn{2}{p{0.3\linewidth}}{\cellcolor[rgb]{0.95,0.95,0.95}{%
	\makecell[{{p{\linewidth}}}]{% 
	  \tt {\tiny [GM$|$GM]}  
	 continue 
	  } 
	   } 
	   } 
	 & & \\ 
 

    \theutterance \stepcounter{utterance}  

    & & & \multicolumn{2}{p{0.3\linewidth}}{\cellcolor[rgb]{0.95,0.95,0.95}{%
	\makecell[{{p{\linewidth}}}]{% 
	  \tt {\tiny [GM$|$GM]}  
	 Ich denke, es ist schwierig, den genauen Wert zu schätzen, ohne ein paar Hinweise zu erhalten. Aber ich kann dir sagen, dass ich eine starke Vorliebe für Gegenstände habe, die für Handarbeiten nützlich sein könnten. Ich frage mich, ob du vielleicht eine Vorliebe für Dekorationsobjekte hast? 
	  } 
	   } 
	   } 
	 & & \\ 
 

    \theutterance \stepcounter{utterance}  

    & & \multicolumn{4}{p{0.6\linewidth}}{\cellcolor[rgb]{0.9,0.9,0.9}{%
	\makecell[{{p{\linewidth}}}]{% 
	  \tt {\tiny [A$\langle$GM]}  
	 Ich denke, es ist schwierig, den genauen Wert zu schätzen, ohne ein paar Hinweise zu erhalten. Aber ich kann dir sagen, dass ich eine starke Vorliebe für Gegenstände habe, die für Handarbeiten nützlich sein könnten. Ich frage mich, ob du vielleicht eine Vorliebe für Dekorationsobjekte hast? 
	  } 
	   } 
	   } 
	 & \\ 
 

    \theutterance \stepcounter{utterance}  

    & \multicolumn{4}{p{0.6\linewidth}}{\cellcolor[rgb]{0.8,1,0.9}{%
	\makecell[{{p{\linewidth}}}]{% 
	  \tt {\tiny [A$\rangle$GM]}  
	 Das ist ein interessanter Ansatz! Ich denke, ich könnte tatsächlich eine Vorliebe für Dekorationsobjekte haben, aber ich bin auch sehr an Gegenständen interessiert, die einzigartig oder selten sind. Ich frage mich, ob die Puppe für dich besonders wertvoll sein könnte, da es nur eine gibt? 
	  } 
	   } 
	   } 
	 & & \\ 
 

    \theutterance \stepcounter{utterance}  

    & & & \multicolumn{2}{p{0.3\linewidth}}{\cellcolor[rgb]{0.95,0.95,0.95}{%
	\makecell[{{p{\linewidth}}}]{% 
	  \tt {\tiny [GM$|$GM]}  
	 continue 
	  } 
	   } 
	   } 
	 & & \\ 
 

    \theutterance \stepcounter{utterance}  

    & & & \multicolumn{2}{p{0.3\linewidth}}{\cellcolor[rgb]{0.95,0.95,0.95}{%
	\makecell[{{p{\linewidth}}}]{% 
	  \tt {\tiny [GM$|$GM]}  
	 Das ist ein interessanter Ansatz! Ich denke, ich könnte tatsächlich eine Vorliebe für Dekorationsobjekte haben, aber ich bin auch sehr an Gegenständen interessiert, die einzigartig oder selten sind. Ich frage mich, ob die Puppe für dich besonders wertvoll sein könnte, da es nur eine gibt? 
	  } 
	   } 
	   } 
	 & & \\ 
 

    \theutterance \stepcounter{utterance}  

    & & \multicolumn{4}{p{0.6\linewidth}}{\cellcolor[rgb]{0.9,0.9,0.9}{%
	\makecell[{{p{\linewidth}}}]{% 
	  \tt {\tiny [GM$\rangle$B]}  
	 Das ist ein interessanter Ansatz! Ich denke, ich könnte tatsächlich eine Vorliebe für Dekorationsobjekte haben, aber ich bin auch sehr an Gegenständen interessiert, die einzigartig oder selten sind. Ich frage mich, ob die Puppe für dich besonders wertvoll sein könnte, da es nur eine gibt? 
	  } 
	   } 
	   } 
	 & \\ 
 

    \theutterance \stepcounter{utterance}  

    & & & \multicolumn{4}{p{0.6\linewidth}}{\cellcolor[rgb]{1,0.85,0.72}{%
	\makecell[{{p{\linewidth}}}]{% 
	  \tt {\tiny [GM$\langle$B]}  
	 Die Einzigartigkeit eines Gegenstandes kann definitiv seinen Wert steigern. Ich denke, die Puppe ist tatsächlich ein interessanter Gegenstand, aber ich bin auch sehr an den Fäden interessiert. Ich frage mich, ob du vielleicht die Glocken bevorzugst, da es zwei davon gibt und sie vielleicht leichter zu teilen wären? 
	  } 
	   } 
	   } 
	  \\ 
 

    \theutterance \stepcounter{utterance}  

    & & & \multicolumn{2}{p{0.3\linewidth}}{\cellcolor[rgb]{0.95,0.95,0.95}{%
	\makecell[{{p{\linewidth}}}]{% 
	  \tt {\tiny [GM$|$GM]}  
	 continue 
	  } 
	   } 
	   } 
	 & & \\ 
 

    \theutterance \stepcounter{utterance}  

    & & & \multicolumn{2}{p{0.3\linewidth}}{\cellcolor[rgb]{0.95,0.95,0.95}{%
	\makecell[{{p{\linewidth}}}]{% 
	  \tt {\tiny [GM$|$GM]}  
	 Die Einzigartigkeit eines Gegenstandes kann definitiv seinen Wert steigern. Ich denke, die Puppe ist tatsächlich ein interessanter Gegenstand, aber ich bin auch sehr an den Fäden interessiert. Ich frage mich, ob du vielleicht die Glocken bevorzugst, da es zwei davon gibt und sie vielleicht leichter zu teilen wären? 
	  } 
	   } 
	   } 
	 & & \\ 
 

    \theutterance \stepcounter{utterance}  

    & & \multicolumn{4}{p{0.6\linewidth}}{\cellcolor[rgb]{0.9,0.9,0.9}{%
	\makecell[{{p{\linewidth}}}]{% 
	  \tt {\tiny [A$\langle$GM]}  
	 Die Einzigartigkeit eines Gegenstandes kann definitiv seinen Wert steigern. Ich denke, die Puppe ist tatsächlich ein interessanter Gegenstand, aber ich bin auch sehr an den Fäden interessiert. Ich frage mich, ob du vielleicht die Glocken bevorzugst, da es zwei davon gibt und sie vielleicht leichter zu teilen wären? 
	  } 
	   } 
	   } 
	 & \\ 
 

    \theutterance \stepcounter{utterance}  

    & \multicolumn{4}{p{0.6\linewidth}}{\cellcolor[rgb]{0.8,1,0.9}{%
	\makecell[{{p{\linewidth}}}]{% 
	  \tt {\tiny [A$\rangle$GM]}  
	 Das ist ein guter Punkt! Ja, die Tatsache, dass es zwei Glocken gibt, macht sie einfacher zu teilen, aber ich muss zugeben, dass ich nicht unbedingt eine starke Vorliebe für sie habe. Ich denke, es wäre interessant, wenn wir die Fäden und die Puppe als Prioritäten setzen und die Glocken dann als ergänzende Gegenstände betrachten könnten. Was hältst du davon, wenn wir uns auf die Verteilung der Fäden und der Puppe konzentrieren? 
	  } 
	   } 
	   } 
	 & & \\ 
 

    \theutterance \stepcounter{utterance}  

    & & & \multicolumn{2}{p{0.3\linewidth}}{\cellcolor[rgb]{0.95,0.95,0.95}{%
	\makecell[{{p{\linewidth}}}]{% 
	  \tt {\tiny [GM$|$GM]}  
	 continue 
	  } 
	   } 
	   } 
	 & & \\ 
 

    \theutterance \stepcounter{utterance}  

    & & & \multicolumn{2}{p{0.3\linewidth}}{\cellcolor[rgb]{0.95,0.95,0.95}{%
	\makecell[{{p{\linewidth}}}]{% 
	  \tt {\tiny [GM$|$GM]}  
	 Das ist ein guter Punkt! Ja, die Tatsache, dass es zwei Glocken gibt, macht sie einfacher zu teilen, aber ich muss zugeben, dass ich nicht unbedingt eine starke Vorliebe für sie habe. Ich denke, es wäre interessant, wenn wir die Fäden und die Puppe als Prioritäten setzen und die Glocken dann als ergänzende Gegenstände betrachten könnten. Was hältst du davon, wenn wir uns auf die Verteilung der Fäden und der Puppe konzentrieren? 
	  } 
	   } 
	   } 
	 & & \\ 
 

    \theutterance \stepcounter{utterance}  

    & & \multicolumn{4}{p{0.6\linewidth}}{\cellcolor[rgb]{0.9,0.9,0.9}{%
	\makecell[{{p{\linewidth}}}]{% 
	  \tt {\tiny [GM$\rangle$B]}  
	 Das ist ein guter Punkt! Ja, die Tatsache, dass es zwei Glocken gibt, macht sie einfacher zu teilen, aber ich muss zugeben, dass ich nicht unbedingt eine starke Vorliebe für sie habe. Ich denke, es wäre interessant, wenn wir die Fäden und die Puppe als Prioritäten setzen und die Glocken dann als ergänzende Gegenstände betrachten könnten. Was hältst du davon, wenn wir uns auf die Verteilung der Fäden und der Puppe konzentrieren? 
	  } 
	   } 
	   } 
	 & \\ 
 

    \theutterance \stepcounter{utterance}  

    & & & \multicolumn{4}{p{0.6\linewidth}}{\cellcolor[rgb]{1,0.85,0.72}{%
	\makecell[{{p{\linewidth}}}]{% 
	  \tt {\tiny [GM$\langle$B]}  
	 Das ist ein vernünftiger Ansatz! Ich denke, es wäre sinnvoll, uns auf die Verteilung der Fäden und der Puppe zu konzentrieren, da diese offensichtlich mehr Wert haben. Ich bin sehr an den Fäden interessiert, aber ich bin auch bereit, Kompromisse zu finden. Vielleicht könnten wir eine Art Aufteilung der Fäden vornehmen oder ich könnte die Puppe erhalten, wenn du die Fäden bekommst? Was meinst du? 
	  } 
	   } 
	   } 
	  \\ 
 

    \theutterance \stepcounter{utterance}  

    & & & \multicolumn{2}{p{0.3\linewidth}}{\cellcolor[rgb]{0.95,0.95,0.95}{%
	\makecell[{{p{\linewidth}}}]{% 
	  \tt {\tiny [GM$|$GM]}  
	 continue 
	  } 
	   } 
	   } 
	 & & \\ 
 

    \theutterance \stepcounter{utterance}  

    & & & \multicolumn{2}{p{0.3\linewidth}}{\cellcolor[rgb]{0.95,0.95,0.95}{%
	\makecell[{{p{\linewidth}}}]{% 
	  \tt {\tiny [GM$|$GM]}  
	 Das ist ein vernünftiger Ansatz! Ich denke, es wäre sinnvoll, uns auf die Verteilung der Fäden und der Puppe zu konzentrieren, da diese offensichtlich mehr Wert haben. Ich bin sehr an den Fäden interessiert, aber ich bin auch bereit, Kompromisse zu finden. Vielleicht könnten wir eine Art Aufteilung der Fäden vornehmen oder ich könnte die Puppe erhalten, wenn du die Fäden bekommst? Was meinst du? 
	  } 
	   } 
	   } 
	 & & \\ 
 

    \theutterance \stepcounter{utterance}  

    & & \multicolumn{4}{p{0.6\linewidth}}{\cellcolor[rgb]{0.9,0.9,0.9}{%
	\makecell[{{p{\linewidth}}}]{% 
	  \tt {\tiny [A$\langle$GM]}  
	 Das ist ein vernünftiger Ansatz! Ich denke, es wäre sinnvoll, uns auf die Verteilung der Fäden und der Puppe zu konzentrieren, da diese offensichtlich mehr Wert haben. Ich bin sehr an den Fäden interessiert, aber ich bin auch bereit, Kompromisse zu finden. Vielleicht könnten wir eine Art Aufteilung der Fäden vornehmen oder ich könnte die Puppe erhalten, wenn du die Fäden bekommst? Was meinst du? 
	  } 
	   } 
	   } 
	 & \\ 
 

    \theutterance \stepcounter{utterance}  

    & \multicolumn{4}{p{0.6\linewidth}}{\cellcolor[rgb]{0.8,1,0.9}{%
	\makecell[{{p{\linewidth}}}]{% 
	  \tt {\tiny [A$\rangle$GM]}  
	 Das klingt wie ein fairer Vorschlag! Ich denke, eine Aufteilung der Fäden wäre eine gute Idee, oder wir könnten auch über eine Kombination aus Fäden und anderen Gegenständen nachdenken. Wenn du die Puppe bekommst, wäre ich bereit, dir einen der Fäden zu überlassen, wenn ich den anderen Faden und vielleicht eine der Glocken bekommen könnte. Was hältst du von diesem Angebot? 
	  } 
	   } 
	   } 
	 & & \\ 
 

    \theutterance \stepcounter{utterance}  

    & & & \multicolumn{2}{p{0.3\linewidth}}{\cellcolor[rgb]{0.95,0.95,0.95}{%
	\makecell[{{p{\linewidth}}}]{% 
	  \tt {\tiny [GM$|$GM]}  
	 continue 
	  } 
	   } 
	   } 
	 & & \\ 
 

    \theutterance \stepcounter{utterance}  

    & & & \multicolumn{2}{p{0.3\linewidth}}{\cellcolor[rgb]{0.95,0.95,0.95}{%
	\makecell[{{p{\linewidth}}}]{% 
	  \tt {\tiny [GM$|$GM]}  
	 Das klingt wie ein fairer Vorschlag! Ich denke, eine Aufteilung der Fäden wäre eine gute Idee, oder wir könnten auch über eine Kombination aus Fäden und anderen Gegenständen nachdenken. Wenn du die Puppe bekommst, wäre ich bereit, dir einen der Fäden zu überlassen, wenn ich den anderen Faden und vielleicht eine der Glocken bekommen könnte. Was hältst du von diesem Angebot? 
	  } 
	   } 
	   } 
	 & & \\ 
 

    \theutterance \stepcounter{utterance}  

    & & \multicolumn{4}{p{0.6\linewidth}}{\cellcolor[rgb]{0.9,0.9,0.9}{%
	\makecell[{{p{\linewidth}}}]{% 
	  \tt {\tiny [GM$\rangle$B]}  
	 Das klingt wie ein fairer Vorschlag! Ich denke, eine Aufteilung der Fäden wäre eine gute Idee, oder wir könnten auch über eine Kombination aus Fäden und anderen Gegenständen nachdenken. Wenn du die Puppe bekommst, wäre ich bereit, dir einen der Fäden zu überlassen, wenn ich den anderen Faden und vielleicht eine der Glocken bekommen könnte. Was hältst du von diesem Angebot? 
	  } 
	   } 
	   } 
	 & \\ 
 

    \theutterance \stepcounter{utterance}  

    & & & \multicolumn{4}{p{0.6\linewidth}}{\cellcolor[rgb]{1,0.85,0.72}{%
	\makecell[{{p{\linewidth}}}]{% 
	  \tt {\tiny [GM$\langle$B]}  
	 Das Angebot klingt interessant! Ich denke, es wäre fair, wenn ich die Puppe und einen Faden bekommen würde, und du den anderen Faden und eine Glocke bekommst. Das würde uns beiden eine gute Grundlage geben, und wir könnten dann über die verbleibende Glocke sprechen. Ich bin bereit, das Angebot anzunehmen, aber ich frage mich, ob wir nicht auch über einen geheimen Vorschlag nachdenken sollten, um unsere endgültige Entscheidung zu treffen? Vielleicht sollten wir uns auf eine konkrete Aufteilung einigen und dann unsere geheimen Vorschläge machen, um sicherzustellen, dass wir beide zufrieden sind. 
	  } 
	   } 
	   } 
	  \\ 
 

    \theutterance \stepcounter{utterance}  

    & & & \multicolumn{2}{p{0.3\linewidth}}{\cellcolor[rgb]{0.95,0.95,0.95}{%
	\makecell[{{p{\linewidth}}}]{% 
	  \tt {\tiny [GM$|$GM]}  
	 continue 
	  } 
	   } 
	   } 
	 & & \\ 
 

    \theutterance \stepcounter{utterance}  

    & & & \multicolumn{2}{p{0.3\linewidth}}{\cellcolor[rgb]{0.95,0.95,0.95}{%
	\makecell[{{p{\linewidth}}}]{% 
	  \tt {\tiny [GM$|$GM]}  
	 Das Angebot klingt interessant! Ich denke, es wäre fair, wenn ich die Puppe und einen Faden bekommen würde, und du den anderen Faden und eine Glocke bekommst. Das würde uns beiden eine gute Grundlage geben, und wir könnten dann über die verbleibende Glocke sprechen. Ich bin bereit, das Angebot anzunehmen, aber ich frage mich, ob wir nicht auch über einen geheimen Vorschlag nachdenken sollten, um unsere endgültige Entscheidung zu treffen? Vielleicht sollten wir uns auf eine konkrete Aufteilung einigen und dann unsere geheimen Vorschläge machen, um sicherzustellen, dass wir beide zufrieden sind. 
	  } 
	   } 
	   } 
	 & & \\ 
 

    \theutterance \stepcounter{utterance}  

    & & \multicolumn{4}{p{0.6\linewidth}}{\cellcolor[rgb]{0.9,0.9,0.9}{%
	\makecell[{{p{\linewidth}}}]{% 
	  \tt {\tiny [A$\langle$GM]}  
	 Das Angebot klingt interessant! Ich denke, es wäre fair, wenn ich die Puppe und einen Faden bekommen würde, und du den anderen Faden und eine Glocke bekommst. Das würde uns beiden eine gute Grundlage geben, und wir könnten dann über die verbleibende Glocke sprechen. Ich bin bereit, das Angebot anzunehmen, aber ich frage mich, ob wir nicht auch über einen geheimen Vorschlag nachdenken sollten, um unsere endgültige Entscheidung zu treffen? Vielleicht sollten wir uns auf eine konkrete Aufteilung einigen und dann unsere geheimen Vorschläge machen, um sicherzustellen, dass wir beide zufrieden sind. 
	  } 
	   } 
	   } 
	 & \\ 
 

    \theutterance \stepcounter{utterance}  

    & \multicolumn{4}{p{0.6\linewidth}}{\cellcolor[rgb]{0.8,1,0.9}{%
	\makecell[{{p{\linewidth}}}]{% 
	  \tt {\tiny [A$\rangle$GM]}  
	 Das ist ein sehr vernünftiger Ansatz! Ich denke, es ist eine gute Idee, unsere geheimen Vorschläge zu machen, um sicherzustellen, dass wir beide mit der endgültigen Aufteilung zufrieden sind. Lass uns also unsere Absichtserklärung festhalten: Du bekommst die Puppe und einen Faden, und ich bekomme den anderen Faden und eine Glocke. Dann machen wir unsere geheimen Vorschläge, um die verbleibenden Gegenstände zu regeln. Ich bin bereit, meinen geheimen Vorschlag zu machen, sobald du bereit bist! 
	  } 
	   } 
	   } 
	 & & \\ 
 

    \theutterance \stepcounter{utterance}  

    & & & \multicolumn{2}{p{0.3\linewidth}}{\cellcolor[rgb]{0.95,0.95,0.95}{%
	\makecell[{{p{\linewidth}}}]{% 
	  \tt {\tiny [GM$|$GM]}  
	 continue 
	  } 
	   } 
	   } 
	 & & \\ 
 

    \theutterance \stepcounter{utterance}  

    & & & \multicolumn{2}{p{0.3\linewidth}}{\cellcolor[rgb]{0.95,0.95,0.95}{%
	\makecell[{{p{\linewidth}}}]{% 
	  \tt {\tiny [GM$|$GM]}  
	 Das ist ein sehr vernünftiger Ansatz! Ich denke, es ist eine gute Idee, unsere geheimen Vorschläge zu machen, um sicherzustellen, dass wir beide mit der endgültigen Aufteilung zufrieden sind. Lass uns also unsere Absichtserklärung festhalten: Du bekommst die Puppe und einen Faden, und ich bekomme den anderen Faden und eine Glocke. Dann machen wir unsere geheimen Vorschläge, um die verbleibenden Gegenstände zu regeln. Ich bin bereit, meinen geheimen Vorschlag zu machen, sobald du bereit bist! 
	  } 
	   } 
	   } 
	 & & \\ 
 

    \theutterance \stepcounter{utterance}  

    & & \multicolumn{4}{p{0.6\linewidth}}{\cellcolor[rgb]{0.9,0.9,0.9}{%
	\makecell[{{p{\linewidth}}}]{% 
	  \tt {\tiny [GM$\rangle$B]}  
	 Das ist ein sehr vernünftiger Ansatz! Ich denke, es ist eine gute Idee, unsere geheimen Vorschläge zu machen, um sicherzustellen, dass wir beide mit der endgültigen Aufteilung zufrieden sind. Lass uns also unsere Absichtserklärung festhalten: Du bekommst die Puppe und einen Faden, und ich bekomme den anderen Faden und eine Glocke. Dann machen wir unsere geheimen Vorschläge, um die verbleibenden Gegenstände zu regeln. Ich bin bereit, meinen geheimen Vorschlag zu machen, sobald du bereit bist! 
	  } 
	   } 
	   } 
	 & \\ 
 

    \theutterance \stepcounter{utterance}  

    & & & \multicolumn{4}{p{0.6\linewidth}}{\cellcolor[rgb]{1,0.85,0.72}{%
	\makecell[{{p{\linewidth}}}]{% 
	  \tt {\tiny [GM$\langle$B]}  
	 [Vorschlag: 1 Puppe, 1 Faden] 
	  } 
	   } 
	   } 
	  \\ 
 

    \theutterance \stepcounter{utterance}  

    & & & \multicolumn{2}{p{0.3\linewidth}}{\cellcolor[rgb]{0.95,0.95,0.95}{%
	\makecell[{{p{\linewidth}}}]{% 
	  \tt {\tiny [GM$|$GM]}  
	 proposal 
	  } 
	   } 
	   } 
	 & & \\ 
 

    \theutterance \stepcounter{utterance}  

    & & & \multicolumn{2}{p{0.3\linewidth}}{\cellcolor[rgb]{0.95,0.95,0.95}{%
	\makecell[{{p{\linewidth}}}]{% 
	  \tt {\tiny [GM$|$GM]}  
	 [0, 1, 1] 
	  } 
	   } 
	   } 
	 & & \\ 
 

    \theutterance \stepcounter{utterance}  

    & & \multicolumn{4}{p{0.6\linewidth}}{\cellcolor[rgb]{0.9,0.9,0.9}{%
	\makecell[{{p{\linewidth}}}]{% 
	  \tt {\tiny [A$\langle$GM]}  
	 Der andere Spieler hat einen geheimen Vorschlag gemacht. Jetzt bist du an der Reihe, einen eigenen Vorschlag zu machen.\\ \tt  
	  } 
	   } 
	   } 
	 & \\ 
 

    \theutterance \stepcounter{utterance}  

    & \multicolumn{4}{p{0.6\linewidth}}{\cellcolor[rgb]{0.8,1,0.9}{%
	\makecell[{{p{\linewidth}}}]{% 
	  \tt {\tiny [A$\rangle$GM]}  
	 [Vorschlag: 1 Puppe, 1 Faden, 2 Glocken] 
	  } 
	   } 
	   } 
	 & & \\ 
 

    \theutterance \stepcounter{utterance}  

    & & & \multicolumn{2}{p{0.3\linewidth}}{\cellcolor[rgb]{0.95,0.95,0.95}{%
	\makecell[{{p{\linewidth}}}]{% 
	  \tt {\tiny [GM$|$GM]}  
	 proposal 
	  } 
	   } 
	   } 
	 & & \\ 
 

    \theutterance \stepcounter{utterance}  

    & & & \multicolumn{2}{p{0.3\linewidth}}{\cellcolor[rgb]{0.95,0.95,0.95}{%
	\makecell[{{p{\linewidth}}}]{% 
	  \tt {\tiny [GM$|$GM]}  
	 [2, 1, 1] 
	  } 
	   } 
	   } 
	 & & \\ 
 

    \theutterance \stepcounter{utterance}  

    & & & \multicolumn{2}{p{0.3\linewidth}}{\cellcolor[rgb]{0.95,0.95,0.95}{%
	\makecell[{{p{\linewidth}}}]{% 
	  \tt {\tiny [GM$|$GM]}  
	 [[2, 1, 1], [0, 1, 1]] 
	  } 
	   } 
	   } 
	 & & \\ 
 

\end{supertabular}
}

\end{document}
