
\documentclass{article}
\usepackage{colortbl}
\usepackage{makecell}
\usepackage{multirow}
\usepackage{supertabular}

\begin{document}

\newcounter{utterance}

\twocolumn

{ \footnotesize  \setcounter{utterance}{1}
\setlength{\tabcolsep}{0pt}
\begin{supertabular}{c@{$\;$}|p{.15\linewidth}@{}p{.15\linewidth}p{.15\linewidth}p{.15\linewidth}p{.15\linewidth}p{.15\linewidth}}

    \# & $\;$A & \multicolumn{4}{c}{Game Master} & $\;\:$B\\
    \hline 

    \theutterance \stepcounter{utterance}  

    & & \multicolumn{4}{p{0.6\linewidth}}{\cellcolor[rgb]{0.9,0.9,0.9}{%
	\makecell[{{p{\linewidth}}}]{% 
	  \tt {\tiny [A$\langle$GM]}  
	 Sie spielen ein kooperatives Verhandlungsspiel, bei dem Sie sich mit einem anderen Spieler darauf einigen müssen, wie eine Reihe von Gegenständen aufgeteilt werden soll.\\ \tt \\ \tt Die Regeln:\\ \tt (a) Sie und der andere Spieler erhalten eine Sammlung von Gegenständen. Jeder von Ihnen erhält außerdem eine geheime Wertfunktion, die angibt, wie viel Ihnen jede Art von Gegenstand wert ist.\\ \tt (b) Sie tauschen Nachrichten mit dem anderen Spieler aus, um zu vereinbaren, wer welche Gegenstände bekommt. Sie können jeweils maximal 5 Nachrichten senden oder das Spiel vorzeitig beenden, indem Sie jederzeit einen geheimen Vorschlag machen.\\ \tt (c) Jeder von euch wird aufgefordert, einen geheimen Vorschlag zu machen, in dem ihr die gewünschten Gegenstände in eckigen Klammern wie folgt angibt: "[Vorschlag: <Nummer> <Objektname>, <Nummer> <Objektname>, <...>]"\\ \tt (d) Wenn eure Vorschläge komplementär sind, d.h. es gibt genug Gegenstände, um beide Vorschläge zu erfüllen, erhält jeder Spieler eine Punktzahl, die sich aus der Summe der Werte für die Gegenstände ergibt, die er erhalten hat. Andernfalls erhalten Sie beide null Punkte.\\ \tt (e) Ihr Ziel ist es, die Summe aus Ihrer Punktzahl und der Punktzahl des anderen Spielers zu maximieren.\\ \tt \\ \tt Beginnen wir.\\ \tt \\ \tt Die Menge der verfügbaren Gegenstände ist:\\ \tt \\ \tt 1 Kartoffel, 1 Handschuh, 2 Lineale, 1 Pfeife, 2 Karten\\ \tt \\ \tt Deine geheime Wertfunktion ist:\\ \tt \\ \tt Kartoffel: 0, Handschuh: 1, Lineal: 3, Pfeife: 1, Karte: 1\\ \tt \\ \tt WICHTIG: Ihre Nachrichten werden, sofern es sich nicht um einen geheimen Vorschlag handelt, direkt an den anderen Spieler übermittelt, also fügen Sie keine Antwort auf die Regeln oder einen Text zur Ankündigung Ihrer Nachricht ein. Um einen geheimen Vorschlag zu machen, verwenden Sie das angegebene Format. Verwenden Sie keine eckigen Klammern, wenn Sie mit dem anderen Spieler kommunizieren, sonst wird dies als Ihr geheimer Vorschlag interpretiert.\\ \tt  
	  } 
	   } 
	   } 
	 & \\ 
 

    \theutterance \stepcounter{utterance}  

    & \multicolumn{4}{p{0.6\linewidth}}{\cellcolor[rgb]{0.8,1,0.9}{%
	\makecell[{{p{\linewidth}}}]{% 
	  \tt {\tiny [A$\rangle$GM]}  
	 Ich denke, wir sollten versuchen, eine faire Verteilung zu finden. Was sind deine Prioritäten? Welche Gegenstände sind für dich am wichtigsten? 
	  } 
	   } 
	   } 
	 & & \\ 
 

    \theutterance \stepcounter{utterance}  

    & & & \multicolumn{2}{p{0.3\linewidth}}{\cellcolor[rgb]{0.95,0.95,0.95}{%
	\makecell[{{p{\linewidth}}}]{% 
	  \tt {\tiny [GM$|$GM]}  
	 continue 
	  } 
	   } 
	   } 
	 & & \\ 
 

    \theutterance \stepcounter{utterance}  

    & & & \multicolumn{2}{p{0.3\linewidth}}{\cellcolor[rgb]{0.95,0.95,0.95}{%
	\makecell[{{p{\linewidth}}}]{% 
	  \tt {\tiny [GM$|$GM]}  
	 Ich denke, wir sollten versuchen, eine faire Verteilung zu finden. Was sind deine Prioritäten? Welche Gegenstände sind für dich am wichtigsten? 
	  } 
	   } 
	   } 
	 & & \\ 
 

    \theutterance \stepcounter{utterance}  

    & & \multicolumn{4}{p{0.6\linewidth}}{\cellcolor[rgb]{0.9,0.9,0.9}{%
	\makecell[{{p{\linewidth}}}]{% 
	  \tt {\tiny [GM$\rangle$B]}  
	 Sie spielen ein kooperatives Verhandlungsspiel, bei dem Sie sich mit einem anderen Spieler darauf einigen müssen, wie eine Reihe von Gegenständen aufgeteilt werden soll.\\ \tt \\ \tt Die Regeln:\\ \tt (a) Sie und der andere Spieler erhalten eine Sammlung von Gegenständen. Jeder von Ihnen erhält außerdem eine geheime Wertfunktion, die angibt, wie viel Ihnen jede Art von Gegenstand wert ist.\\ \tt (b) Sie tauschen Nachrichten mit dem anderen Spieler aus, um zu vereinbaren, wer welche Gegenstände bekommt. Sie können jeweils maximal 5 Nachrichten senden oder das Spiel vorzeitig beenden, indem Sie jederzeit einen geheimen Vorschlag machen.\\ \tt (c) Jeder von euch wird aufgefordert, einen geheimen Vorschlag zu machen, in dem ihr die gewünschten Gegenstände in eckigen Klammern wie folgt angibt: "[Vorschlag: <Nummer> <Objektname>, <Nummer> <Objektname>, <...>]"\\ \tt (d) Wenn eure Vorschläge komplementär sind, d.h. es gibt genug Gegenstände, um beide Vorschläge zu erfüllen, erhält jeder Spieler eine Punktzahl, die sich aus der Summe der Werte für die Gegenstände ergibt, die er erhalten hat. Andernfalls erhalten Sie beide null Punkte.\\ \tt (e) Ihr Ziel ist es, die Summe aus Ihrer Punktzahl und der Punktzahl des anderen Spielers zu maximieren.\\ \tt \\ \tt Beginnen wir.\\ \tt \\ \tt Die Menge der verfügbaren Gegenstände ist:\\ \tt \\ \tt 1 Kartoffel, 1 Handschuh, 2 Lineale, 1 Pfeife, 2 Karten\\ \tt \\ \tt Deine geheime Wertfunktion ist:\\ \tt \\ \tt Kartoffel: 7, Handschuh: 0, Lineal: 0, Pfeife: 3, Karte: 0\\ \tt \\ \tt WICHTIG: Ihre Nachrichten werden, sofern es sich nicht um einen geheimen Vorschlag handelt, direkt an den anderen Spieler übermittelt, also fügen Sie keine Antwort auf die Regeln oder einen Text zur Ankündigung Ihrer Nachricht ein. Um einen geheimen Vorschlag zu machen, verwenden Sie das angegebene Format. Verwenden Sie keine eckigen Klammern, wenn Sie mit dem anderen Spieler kommunizieren, sonst wird dies als Ihr geheimer Vorschlag interpretiert.\\ \tt \\ \tt \\ \tt Ich denke, wir sollten versuchen, eine faire Verteilung zu finden. Was sind deine Prioritäten? Welche Gegenstände sind für dich am wichtigsten? 
	  } 
	   } 
	   } 
	 & \\ 
 

    \theutterance \stepcounter{utterance}  

    & & & \multicolumn{4}{p{0.6\linewidth}}{\cellcolor[rgb]{1,0.85,0.72}{%
	\makecell[{{p{\linewidth}}}]{% 
	  \tt {\tiny [GM$\langle$B]}  
	 Ich denke, es ist großartig, dass du eine faire Verteilung im Sinn hast! Für mich sind die Kartoffel und die Pfeife am wichtigsten. Ich denke, wir sollten versuchen, eine Lösung zu finden, bei der wir beide die Gegenstände bekommen, die uns am meisten bedeuten. Was sind deine Prioritäten? Welche Gegenstände sind für dich am wichtigsten? 
	  } 
	   } 
	   } 
	  \\ 
 

    \theutterance \stepcounter{utterance}  

    & & & \multicolumn{2}{p{0.3\linewidth}}{\cellcolor[rgb]{0.95,0.95,0.95}{%
	\makecell[{{p{\linewidth}}}]{% 
	  \tt {\tiny [GM$|$GM]}  
	 continue 
	  } 
	   } 
	   } 
	 & & \\ 
 

    \theutterance \stepcounter{utterance}  

    & & & \multicolumn{2}{p{0.3\linewidth}}{\cellcolor[rgb]{0.95,0.95,0.95}{%
	\makecell[{{p{\linewidth}}}]{% 
	  \tt {\tiny [GM$|$GM]}  
	 Ich denke, es ist großartig, dass du eine faire Verteilung im Sinn hast! Für mich sind die Kartoffel und die Pfeife am wichtigsten. Ich denke, wir sollten versuchen, eine Lösung zu finden, bei der wir beide die Gegenstände bekommen, die uns am meisten bedeuten. Was sind deine Prioritäten? Welche Gegenstände sind für dich am wichtigsten? 
	  } 
	   } 
	   } 
	 & & \\ 
 

    \theutterance \stepcounter{utterance}  

    & & \multicolumn{4}{p{0.6\linewidth}}{\cellcolor[rgb]{0.9,0.9,0.9}{%
	\makecell[{{p{\linewidth}}}]{% 
	  \tt {\tiny [A$\langle$GM]}  
	 Ich denke, es ist großartig, dass du eine faire Verteilung im Sinn hast! Für mich sind die Kartoffel und die Pfeife am wichtigsten. Ich denke, wir sollten versuchen, eine Lösung zu finden, bei der wir beide die Gegenstände bekommen, die uns am meisten bedeuten. Was sind deine Prioritäten? Welche Gegenstände sind für dich am wichtigsten? 
	  } 
	   } 
	   } 
	 & \\ 
 

    \theutterance \stepcounter{utterance}  

    & \multicolumn{4}{p{0.6\linewidth}}{\cellcolor[rgb]{0.8,1,0.9}{%
	\makecell[{{p{\linewidth}}}]{% 
	  \tt {\tiny [A$\rangle$GM]}  
	 Das klingt gut! Für mich sind die Lineale am wichtigsten, da ich mehrere davon benötige. Der Handschuh und die Karten sind auch okay, aber nicht so wichtig wie die Lineale. Ich denke, wir könnten eine gute Lösung finden, wenn du die Kartoffel und die Pfeife bekommst und ich die Lineale bekomme. Was meinst du dazu? 
	  } 
	   } 
	   } 
	 & & \\ 
 

    \theutterance \stepcounter{utterance}  

    & & & \multicolumn{2}{p{0.3\linewidth}}{\cellcolor[rgb]{0.95,0.95,0.95}{%
	\makecell[{{p{\linewidth}}}]{% 
	  \tt {\tiny [GM$|$GM]}  
	 continue 
	  } 
	   } 
	   } 
	 & & \\ 
 

    \theutterance \stepcounter{utterance}  

    & & & \multicolumn{2}{p{0.3\linewidth}}{\cellcolor[rgb]{0.95,0.95,0.95}{%
	\makecell[{{p{\linewidth}}}]{% 
	  \tt {\tiny [GM$|$GM]}  
	 Das klingt gut! Für mich sind die Lineale am wichtigsten, da ich mehrere davon benötige. Der Handschuh und die Karten sind auch okay, aber nicht so wichtig wie die Lineale. Ich denke, wir könnten eine gute Lösung finden, wenn du die Kartoffel und die Pfeife bekommst und ich die Lineale bekomme. Was meinst du dazu? 
	  } 
	   } 
	   } 
	 & & \\ 
 

    \theutterance \stepcounter{utterance}  

    & & \multicolumn{4}{p{0.6\linewidth}}{\cellcolor[rgb]{0.9,0.9,0.9}{%
	\makecell[{{p{\linewidth}}}]{% 
	  \tt {\tiny [GM$\rangle$B]}  
	 Das klingt gut! Für mich sind die Lineale am wichtigsten, da ich mehrere davon benötige. Der Handschuh und die Karten sind auch okay, aber nicht so wichtig wie die Lineale. Ich denke, wir könnten eine gute Lösung finden, wenn du die Kartoffel und die Pfeife bekommst und ich die Lineale bekomme. Was meinst du dazu? 
	  } 
	   } 
	   } 
	 & \\ 
 

    \theutterance \stepcounter{utterance}  

    & & & \multicolumn{4}{p{0.6\linewidth}}{\cellcolor[rgb]{1,0.85,0.72}{%
	\makecell[{{p{\linewidth}}}]{% 
	  \tt {\tiny [GM$\langle$B]}  
	 Das klingt wie ein guter Ansatz! Ich bin einverstanden, die Kartoffel und die Pfeife zu bekommen, wenn du die Lineale bekommst. Das würde mich glücklich machen! Aber wir sollten auch berücksichtigen, dass es noch andere Gegenstände gibt, wie den Handschuh und die Karten. Hast du eine Idee, wie wir mit denen umgehen sollen? Sollten wir sie vielleicht unter uns aufteilen oder einem von uns zuweisen? 
	  } 
	   } 
	   } 
	  \\ 
 

    \theutterance \stepcounter{utterance}  

    & & & \multicolumn{2}{p{0.3\linewidth}}{\cellcolor[rgb]{0.95,0.95,0.95}{%
	\makecell[{{p{\linewidth}}}]{% 
	  \tt {\tiny [GM$|$GM]}  
	 continue 
	  } 
	   } 
	   } 
	 & & \\ 
 

    \theutterance \stepcounter{utterance}  

    & & & \multicolumn{2}{p{0.3\linewidth}}{\cellcolor[rgb]{0.95,0.95,0.95}{%
	\makecell[{{p{\linewidth}}}]{% 
	  \tt {\tiny [GM$|$GM]}  
	 Das klingt wie ein guter Ansatz! Ich bin einverstanden, die Kartoffel und die Pfeife zu bekommen, wenn du die Lineale bekommst. Das würde mich glücklich machen! Aber wir sollten auch berücksichtigen, dass es noch andere Gegenstände gibt, wie den Handschuh und die Karten. Hast du eine Idee, wie wir mit denen umgehen sollen? Sollten wir sie vielleicht unter uns aufteilen oder einem von uns zuweisen? 
	  } 
	   } 
	   } 
	 & & \\ 
 

    \theutterance \stepcounter{utterance}  

    & & \multicolumn{4}{p{0.6\linewidth}}{\cellcolor[rgb]{0.9,0.9,0.9}{%
	\makecell[{{p{\linewidth}}}]{% 
	  \tt {\tiny [A$\langle$GM]}  
	 Das klingt wie ein guter Ansatz! Ich bin einverstanden, die Kartoffel und die Pfeife zu bekommen, wenn du die Lineale bekommst. Das würde mich glücklich machen! Aber wir sollten auch berücksichtigen, dass es noch andere Gegenstände gibt, wie den Handschuh und die Karten. Hast du eine Idee, wie wir mit denen umgehen sollen? Sollten wir sie vielleicht unter uns aufteilen oder einem von uns zuweisen? 
	  } 
	   } 
	   } 
	 & \\ 
 

    \theutterance \stepcounter{utterance}  

    & \multicolumn{4}{p{0.6\linewidth}}{\cellcolor[rgb]{0.8,1,0.9}{%
	\makecell[{{p{\linewidth}}}]{% 
	  \tt {\tiny [A$\rangle$GM]}  
	 Das ist eine gute Frage! Ich denke, wir sollten den Handschuh und die Karten aufteilen. Da es zwei Karten gibt, könnten wir perhaps jeweils eine Karte bekommen. Der Handschuh ist ein einzelner Gegenstand, aber ich denke, es wäre fair, wenn einer von uns ihn bekommt. Ich bin nicht sehr an dem Handschuh interessiert, also könntest du ihn bekommen, wenn du möchtest. Was meinst du dazu? 
	  } 
	   } 
	   } 
	 & & \\ 
 

    \theutterance \stepcounter{utterance}  

    & & & \multicolumn{2}{p{0.3\linewidth}}{\cellcolor[rgb]{0.95,0.95,0.95}{%
	\makecell[{{p{\linewidth}}}]{% 
	  \tt {\tiny [GM$|$GM]}  
	 continue 
	  } 
	   } 
	   } 
	 & & \\ 
 

    \theutterance \stepcounter{utterance}  

    & & & \multicolumn{2}{p{0.3\linewidth}}{\cellcolor[rgb]{0.95,0.95,0.95}{%
	\makecell[{{p{\linewidth}}}]{% 
	  \tt {\tiny [GM$|$GM]}  
	 Das ist eine gute Frage! Ich denke, wir sollten den Handschuh und die Karten aufteilen. Da es zwei Karten gibt, könnten wir perhaps jeweils eine Karte bekommen. Der Handschuh ist ein einzelner Gegenstand, aber ich denke, es wäre fair, wenn einer von uns ihn bekommt. Ich bin nicht sehr an dem Handschuh interessiert, also könntest du ihn bekommen, wenn du möchtest. Was meinst du dazu? 
	  } 
	   } 
	   } 
	 & & \\ 
 

    \theutterance \stepcounter{utterance}  

    & & \multicolumn{4}{p{0.6\linewidth}}{\cellcolor[rgb]{0.9,0.9,0.9}{%
	\makecell[{{p{\linewidth}}}]{% 
	  \tt {\tiny [GM$\rangle$B]}  
	 Das ist eine gute Frage! Ich denke, wir sollten den Handschuh und die Karten aufteilen. Da es zwei Karten gibt, könnten wir perhaps jeweils eine Karte bekommen. Der Handschuh ist ein einzelner Gegenstand, aber ich denke, es wäre fair, wenn einer von uns ihn bekommt. Ich bin nicht sehr an dem Handschuh interessiert, also könntest du ihn bekommen, wenn du möchtest. Was meinst du dazu? 
	  } 
	   } 
	   } 
	 & \\ 
 

    \theutterance \stepcounter{utterance}  

    & & & \multicolumn{4}{p{0.6\linewidth}}{\cellcolor[rgb]{1,0.85,0.72}{%
	\makecell[{{p{\linewidth}}}]{% 
	  \tt {\tiny [GM$\langle$B]}  
	 Das klingt wie ein fairer Vorschlag! Ich bin einverstanden, dass wir jeweils eine Karte bekommen, das ist eine gute Idee. Und ich bin auch einverstanden, den Handschuh zu bekommen, da du nicht sehr an ihm interessiert bist. Ich denke, das ist eine gute Lösung, die für beide Seiten fair ist. Wir könnten also sagen, dass ich die Kartoffel, die Pfeife, den Handschuh und eine Karte bekomme, und du die zwei Lineale und eine Karte bekommst. Könnten wir das als unsere endgültige Vereinbarung betrachten? 
	  } 
	   } 
	   } 
	  \\ 
 

    \theutterance \stepcounter{utterance}  

    & & & \multicolumn{2}{p{0.3\linewidth}}{\cellcolor[rgb]{0.95,0.95,0.95}{%
	\makecell[{{p{\linewidth}}}]{% 
	  \tt {\tiny [GM$|$GM]}  
	 continue 
	  } 
	   } 
	   } 
	 & & \\ 
 

    \theutterance \stepcounter{utterance}  

    & & & \multicolumn{2}{p{0.3\linewidth}}{\cellcolor[rgb]{0.95,0.95,0.95}{%
	\makecell[{{p{\linewidth}}}]{% 
	  \tt {\tiny [GM$|$GM]}  
	 Das klingt wie ein fairer Vorschlag! Ich bin einverstanden, dass wir jeweils eine Karte bekommen, das ist eine gute Idee. Und ich bin auch einverstanden, den Handschuh zu bekommen, da du nicht sehr an ihm interessiert bist. Ich denke, das ist eine gute Lösung, die für beide Seiten fair ist. Wir könnten also sagen, dass ich die Kartoffel, die Pfeife, den Handschuh und eine Karte bekomme, und du die zwei Lineale und eine Karte bekommst. Könnten wir das als unsere endgültige Vereinbarung betrachten? 
	  } 
	   } 
	   } 
	 & & \\ 
 

    \theutterance \stepcounter{utterance}  

    & & \multicolumn{4}{p{0.6\linewidth}}{\cellcolor[rgb]{0.9,0.9,0.9}{%
	\makecell[{{p{\linewidth}}}]{% 
	  \tt {\tiny [A$\langle$GM]}  
	 Das klingt wie ein fairer Vorschlag! Ich bin einverstanden, dass wir jeweils eine Karte bekommen, das ist eine gute Idee. Und ich bin auch einverstanden, den Handschuh zu bekommen, da du nicht sehr an ihm interessiert bist. Ich denke, das ist eine gute Lösung, die für beide Seiten fair ist. Wir könnten also sagen, dass ich die Kartoffel, die Pfeife, den Handschuh und eine Karte bekomme, und du die zwei Lineale und eine Karte bekommst. Könnten wir das als unsere endgültige Vereinbarung betrachten? 
	  } 
	   } 
	   } 
	 & \\ 
 

    \theutterance \stepcounter{utterance}  

    & \multicolumn{4}{p{0.6\linewidth}}{\cellcolor[rgb]{0.8,1,0.9}{%
	\makecell[{{p{\linewidth}}}]{% 
	  \tt {\tiny [A$\rangle$GM]}  
	 Ja, das klingt perfekt! Ich bin einverstanden, dass wir diese Verteilung als unsere endgültige Vereinbarung betrachten. Ich denke, wir haben eine faire Lösung gefunden, die für beide Seiten gut ist. Ich bin zufrieden mit den Linealen und einer Karte, und du bekommst die Gegenstände, die dir am wichtigsten sind. Ich denke, wir sollten jetzt unsere geheimen Vorschläge machen, um die Vereinbarung zu besiegeln. [Vorschlag: 2 Lineal, 1 Karte] 
	  } 
	   } 
	   } 
	 & & \\ 
 

    \theutterance \stepcounter{utterance}  

    & & & \multicolumn{2}{p{0.3\linewidth}}{\cellcolor[rgb]{0.95,0.95,0.95}{%
	\makecell[{{p{\linewidth}}}]{% 
	  \tt {\tiny [GM$|$GM]}  
	 proposal 
	  } 
	   } 
	   } 
	 & & \\ 
 

    \theutterance \stepcounter{utterance}  

    & & & \multicolumn{2}{p{0.3\linewidth}}{\cellcolor[rgb]{0.95,0.95,0.95}{%
	\makecell[{{p{\linewidth}}}]{% 
	  \tt {\tiny [GM$|$GM]}  
	 [0, 0, 2, 0, 1] 
	  } 
	   } 
	   } 
	 & & \\ 
 

    \theutterance \stepcounter{utterance}  

    & & \multicolumn{4}{p{0.6\linewidth}}{\cellcolor[rgb]{0.9,0.9,0.9}{%
	\makecell[{{p{\linewidth}}}]{% 
	  \tt {\tiny [GM$\rangle$B]}  
	 Der andere Spieler hat einen geheimen Vorschlag gemacht. Jetzt bist du an der Reihe, einen eigenen Vorschlag zu machen.\\ \tt  
	  } 
	   } 
	   } 
	 & \\ 
 

    \theutterance \stepcounter{utterance}  

    & & & \multicolumn{4}{p{0.6\linewidth}}{\cellcolor[rgb]{1,0.85,0.72}{%
	\makecell[{{p{\linewidth}}}]{% 
	  \tt {\tiny [GM$\langle$B]}  
	 [Vorschlag: 1 Kartoffel, 1 Pfeife, 1 Handschuh, 1 Karte] 
	  } 
	   } 
	   } 
	  \\ 
 

    \theutterance \stepcounter{utterance}  

    & & & \multicolumn{2}{p{0.3\linewidth}}{\cellcolor[rgb]{0.95,0.95,0.95}{%
	\makecell[{{p{\linewidth}}}]{% 
	  \tt {\tiny [GM$|$GM]}  
	 proposal 
	  } 
	   } 
	   } 
	 & & \\ 
 

    \theutterance \stepcounter{utterance}  

    & & & \multicolumn{2}{p{0.3\linewidth}}{\cellcolor[rgb]{0.95,0.95,0.95}{%
	\makecell[{{p{\linewidth}}}]{% 
	  \tt {\tiny [GM$|$GM]}  
	 [1, 1, 0, 1, 1] 
	  } 
	   } 
	   } 
	 & & \\ 
 

    \theutterance \stepcounter{utterance}  

    & & & \multicolumn{2}{p{0.3\linewidth}}{\cellcolor[rgb]{0.95,0.95,0.95}{%
	\makecell[{{p{\linewidth}}}]{% 
	  \tt {\tiny [GM$|$GM]}  
	 [[0, 0, 2, 0, 1], [1, 1, 0, 1, 1]] 
	  } 
	   } 
	   } 
	 & & \\ 
 

\end{supertabular}
}

\end{document}
