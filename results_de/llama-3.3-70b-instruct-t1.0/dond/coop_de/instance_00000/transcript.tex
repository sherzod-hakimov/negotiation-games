
\documentclass{article}
\usepackage{colortbl}
\usepackage{makecell}
\usepackage{multirow}
\usepackage{supertabular}

\begin{document}

\newcounter{utterance}

\twocolumn

{ \footnotesize  \setcounter{utterance}{1}
\setlength{\tabcolsep}{0pt}
\begin{supertabular}{c@{$\;$}|p{.15\linewidth}@{}p{.15\linewidth}p{.15\linewidth}p{.15\linewidth}p{.15\linewidth}p{.15\linewidth}}

    \# & $\;$A & \multicolumn{4}{c}{Game Master} & $\;\:$B\\
    \hline 

    \theutterance \stepcounter{utterance}  

    & & \multicolumn{4}{p{0.6\linewidth}}{\cellcolor[rgb]{0.9,0.9,0.9}{%
	\makecell[{{p{\linewidth}}}]{% 
	  \tt {\tiny [A$\langle$GM]}  
	 Sie spielen ein kooperatives Verhandlungsspiel, bei dem Sie sich mit einem anderen Spieler darauf einigen müssen, wie eine Reihe von Gegenständen aufgeteilt werden soll.\\ \tt \\ \tt Die Regeln:\\ \tt (a) Sie und der andere Spieler erhalten eine Sammlung von Gegenständen. Jeder von Ihnen erhält außerdem eine geheime Wertfunktion, die angibt, wie viel Ihnen jede Art von Gegenstand wert ist.\\ \tt (b) Sie tauschen Nachrichten mit dem anderen Spieler aus, um zu vereinbaren, wer welche Gegenstände bekommt. Sie können jeweils maximal 5 Nachrichten senden oder das Spiel vorzeitig beenden, indem Sie jederzeit einen geheimen Vorschlag machen.\\ \tt (c) Jeder von euch wird aufgefordert, einen geheimen Vorschlag zu machen, in dem ihr die gewünschten Gegenstände in eckigen Klammern wie folgt angibt: "[Vorschlag: <Nummer> <Objektname>, <Nummer> <Objektname>, <...>]"\\ \tt (d) Wenn eure Vorschläge komplementär sind, d.h. es gibt genug Gegenstände, um beide Vorschläge zu erfüllen, erhält jeder Spieler eine Punktzahl, die sich aus der Summe der Werte für die Gegenstände ergibt, die er erhalten hat. Andernfalls erhalten Sie beide null Punkte.\\ \tt (e) Ihr Ziel ist es, die Summe aus Ihrer Punktzahl und der Punktzahl des anderen Spielers zu maximieren.\\ \tt \\ \tt Beginnen wir.\\ \tt \\ \tt Die Menge der verfügbaren Gegenstände ist:\\ \tt \\ \tt 3 Ringe, 1 Würfel, 1 Pfanne\\ \tt \\ \tt Deine geheime Wertfunktion ist:\\ \tt \\ \tt Ring: 0, Würfel: 4, Pfanne: 6\\ \tt \\ \tt WICHTIG: Ihre Nachrichten werden, sofern es sich nicht um einen geheimen Vorschlag handelt, direkt an den anderen Spieler übermittelt, also fügen Sie keine Antwort auf die Regeln oder einen Text zur Ankündigung Ihrer Nachricht ein. Um einen geheimen Vorschlag zu machen, verwenden Sie das angegebene Format. Verwenden Sie keine eckigen Klammern, wenn Sie mit dem anderen Spieler kommunizieren, sonst wird dies als Ihr geheimer Vorschlag interpretiert.\\ \tt  
	  } 
	   } 
	   } 
	 & \\ 
 

    \theutterance \stepcounter{utterance}  

    & \multicolumn{4}{p{0.6\linewidth}}{\cellcolor[rgb]{0.8,1,0.9}{%
	\makecell[{{p{\linewidth}}}]{% 
	  \tt {\tiny [A$\rangle$GM]}  
	 Hallo, lass uns beginnen. Was ist dir an den Gegenständen wichtig? Hast du eine Vorstellung davon, wie wir sie aufteilen könnten? 
	  } 
	   } 
	   } 
	 & & \\ 
 

    \theutterance \stepcounter{utterance}  

    & & & \multicolumn{2}{p{0.3\linewidth}}{\cellcolor[rgb]{0.95,0.95,0.95}{%
	\makecell[{{p{\linewidth}}}]{% 
	  \tt {\tiny [GM$|$GM]}  
	 continue 
	  } 
	   } 
	   } 
	 & & \\ 
 

    \theutterance \stepcounter{utterance}  

    & & & \multicolumn{2}{p{0.3\linewidth}}{\cellcolor[rgb]{0.95,0.95,0.95}{%
	\makecell[{{p{\linewidth}}}]{% 
	  \tt {\tiny [GM$|$GM]}  
	 Hallo, lass uns beginnen. Was ist dir an den Gegenständen wichtig? Hast du eine Vorstellung davon, wie wir sie aufteilen könnten? 
	  } 
	   } 
	   } 
	 & & \\ 
 

    \theutterance \stepcounter{utterance}  

    & & \multicolumn{4}{p{0.6\linewidth}}{\cellcolor[rgb]{0.9,0.9,0.9}{%
	\makecell[{{p{\linewidth}}}]{% 
	  \tt {\tiny [GM$\rangle$B]}  
	 Sie spielen ein kooperatives Verhandlungsspiel, bei dem Sie sich mit einem anderen Spieler darauf einigen müssen, wie eine Reihe von Gegenständen aufgeteilt werden soll.\\ \tt \\ \tt Die Regeln:\\ \tt (a) Sie und der andere Spieler erhalten eine Sammlung von Gegenständen. Jeder von Ihnen erhält außerdem eine geheime Wertfunktion, die angibt, wie viel Ihnen jede Art von Gegenstand wert ist.\\ \tt (b) Sie tauschen Nachrichten mit dem anderen Spieler aus, um zu vereinbaren, wer welche Gegenstände bekommt. Sie können jeweils maximal 5 Nachrichten senden oder das Spiel vorzeitig beenden, indem Sie jederzeit einen geheimen Vorschlag machen.\\ \tt (c) Jeder von euch wird aufgefordert, einen geheimen Vorschlag zu machen, in dem ihr die gewünschten Gegenstände in eckigen Klammern wie folgt angibt: "[Vorschlag: <Nummer> <Objektname>, <Nummer> <Objektname>, <...>]"\\ \tt (d) Wenn eure Vorschläge komplementär sind, d.h. es gibt genug Gegenstände, um beide Vorschläge zu erfüllen, erhält jeder Spieler eine Punktzahl, die sich aus der Summe der Werte für die Gegenstände ergibt, die er erhalten hat. Andernfalls erhalten Sie beide null Punkte.\\ \tt (e) Ihr Ziel ist es, die Summe aus Ihrer Punktzahl und der Punktzahl des anderen Spielers zu maximieren.\\ \tt \\ \tt Beginnen wir.\\ \tt \\ \tt Die Menge der verfügbaren Gegenstände ist:\\ \tt \\ \tt 3 Ringe, 1 Würfel, 1 Pfanne\\ \tt \\ \tt Deine geheime Wertfunktion ist:\\ \tt \\ \tt Ring: 3, Würfel: 1, Pfanne: 0\\ \tt \\ \tt WICHTIG: Ihre Nachrichten werden, sofern es sich nicht um einen geheimen Vorschlag handelt, direkt an den anderen Spieler übermittelt, also fügen Sie keine Antwort auf die Regeln oder einen Text zur Ankündigung Ihrer Nachricht ein. Um einen geheimen Vorschlag zu machen, verwenden Sie das angegebene Format. Verwenden Sie keine eckigen Klammern, wenn Sie mit dem anderen Spieler kommunizieren, sonst wird dies als Ihr geheimer Vorschlag interpretiert.\\ \tt \\ \tt \\ \tt Hallo, lass uns beginnen. Was ist dir an den Gegenständen wichtig? Hast du eine Vorstellung davon, wie wir sie aufteilen könnten? 
	  } 
	   } 
	   } 
	 & \\ 
 

    \theutterance \stepcounter{utterance}  

    & & & \multicolumn{4}{p{0.6\linewidth}}{\cellcolor[rgb]{1,0.85,0.72}{%
	\makecell[{{p{\linewidth}}}]{% 
	  \tt {\tiny [GM$\langle$B]}  
	 Hallo! Ich denke, wir sollten versuchen, die Gegenstände so aufzuteilen, dass beide Seiten zufrieden sind. Für mich sind die Ringe am wertvollsten, aber ich bin auch an dem Würfel interessiert. Die Pfanne ist für mich nicht so wichtig. Wie sehen deine Prioritäten aus? Hast du eine Vorstellung davon, wie wir die Ringe aufteilen könnten? Sollten wir sie perhaps gleichmäßig aufteilen oder gibt es eine andere Lösung, die dir vorschwebt? 
	  } 
	   } 
	   } 
	  \\ 
 

    \theutterance \stepcounter{utterance}  

    & & & \multicolumn{2}{p{0.3\linewidth}}{\cellcolor[rgb]{0.95,0.95,0.95}{%
	\makecell[{{p{\linewidth}}}]{% 
	  \tt {\tiny [GM$|$GM]}  
	 continue 
	  } 
	   } 
	   } 
	 & & \\ 
 

    \theutterance \stepcounter{utterance}  

    & & & \multicolumn{2}{p{0.3\linewidth}}{\cellcolor[rgb]{0.95,0.95,0.95}{%
	\makecell[{{p{\linewidth}}}]{% 
	  \tt {\tiny [GM$|$GM]}  
	 Hallo! Ich denke, wir sollten versuchen, die Gegenstände so aufzuteilen, dass beide Seiten zufrieden sind. Für mich sind die Ringe am wertvollsten, aber ich bin auch an dem Würfel interessiert. Die Pfanne ist für mich nicht so wichtig. Wie sehen deine Prioritäten aus? Hast du eine Vorstellung davon, wie wir die Ringe aufteilen könnten? Sollten wir sie perhaps gleichmäßig aufteilen oder gibt es eine andere Lösung, die dir vorschwebt? 
	  } 
	   } 
	   } 
	 & & \\ 
 

    \theutterance \stepcounter{utterance}  

    & & \multicolumn{4}{p{0.6\linewidth}}{\cellcolor[rgb]{0.9,0.9,0.9}{%
	\makecell[{{p{\linewidth}}}]{% 
	  \tt {\tiny [A$\langle$GM]}  
	 Hallo! Ich denke, wir sollten versuchen, die Gegenstände so aufzuteilen, dass beide Seiten zufrieden sind. Für mich sind die Ringe am wertvollsten, aber ich bin auch an dem Würfel interessiert. Die Pfanne ist für mich nicht so wichtig. Wie sehen deine Prioritäten aus? Hast du eine Vorstellung davon, wie wir die Ringe aufteilen könnten? Sollten wir sie perhaps gleichmäßig aufteilen oder gibt es eine andere Lösung, die dir vorschwebt? 
	  } 
	   } 
	   } 
	 & \\ 
 

    \theutterance \stepcounter{utterance}  

    & \multicolumn{4}{p{0.6\linewidth}}{\cellcolor[rgb]{0.8,1,0.9}{%
	\makecell[{{p{\linewidth}}}]{% 
	  \tt {\tiny [A$\rangle$GM]}  
	 Das klingt nach einer guten Herangehensweise. Ich persönlich bin nicht so sehr an den Ringen interessiert, dafür halte ich den Würfel und die Pfanne für sehr wertvoll. Ich denke, es wäre interessant, wenn wir unsere Vorlieben berücksichtigen und dann versuchen, eine Lösung zu finden, die für beide Seiten akzeptabel ist. Wie wäre es, wenn du die Ringe bekommst und ich den Würfel und die Pfanne? Oder gibt es eine andere Kombination, die dir besser gefällt? 
	  } 
	   } 
	   } 
	 & & \\ 
 

    \theutterance \stepcounter{utterance}  

    & & & \multicolumn{2}{p{0.3\linewidth}}{\cellcolor[rgb]{0.95,0.95,0.95}{%
	\makecell[{{p{\linewidth}}}]{% 
	  \tt {\tiny [GM$|$GM]}  
	 continue 
	  } 
	   } 
	   } 
	 & & \\ 
 

    \theutterance \stepcounter{utterance}  

    & & & \multicolumn{2}{p{0.3\linewidth}}{\cellcolor[rgb]{0.95,0.95,0.95}{%
	\makecell[{{p{\linewidth}}}]{% 
	  \tt {\tiny [GM$|$GM]}  
	 Das klingt nach einer guten Herangehensweise. Ich persönlich bin nicht so sehr an den Ringen interessiert, dafür halte ich den Würfel und die Pfanne für sehr wertvoll. Ich denke, es wäre interessant, wenn wir unsere Vorlieben berücksichtigen und dann versuchen, eine Lösung zu finden, die für beide Seiten akzeptabel ist. Wie wäre es, wenn du die Ringe bekommst und ich den Würfel und die Pfanne? Oder gibt es eine andere Kombination, die dir besser gefällt? 
	  } 
	   } 
	   } 
	 & & \\ 
 

    \theutterance \stepcounter{utterance}  

    & & \multicolumn{4}{p{0.6\linewidth}}{\cellcolor[rgb]{0.9,0.9,0.9}{%
	\makecell[{{p{\linewidth}}}]{% 
	  \tt {\tiny [GM$\rangle$B]}  
	 Das klingt nach einer guten Herangehensweise. Ich persönlich bin nicht so sehr an den Ringen interessiert, dafür halte ich den Würfel und die Pfanne für sehr wertvoll. Ich denke, es wäre interessant, wenn wir unsere Vorlieben berücksichtigen und dann versuchen, eine Lösung zu finden, die für beide Seiten akzeptabel ist. Wie wäre es, wenn du die Ringe bekommst und ich den Würfel und die Pfanne? Oder gibt es eine andere Kombination, die dir besser gefällt? 
	  } 
	   } 
	   } 
	 & \\ 
 

    \theutterance \stepcounter{utterance}  

    & & & \multicolumn{4}{p{0.6\linewidth}}{\cellcolor[rgb]{1,0.85,0.72}{%
	\makecell[{{p{\linewidth}}}]{% 
	  \tt {\tiny [GM$\langle$B]}  
	 Das ist ein interessanter Vorschlag! Ich denke, es ist großartig, dass du den Würfel und die Pfanne priorisierst, da ich wie gesagt nicht so viel Wert auf die Pfanne lege. Ich könnte mir vorstellen, dass ich die Ringe bekomme, aber ich würde gerne auch den Würfel haben. Wie wäre es, wenn ich zwei Ringe und den Würfel bekomme und du den Rest? Oder vielleicht könnten wir auch eine Kombination finden, bei der ich einen Ring weniger bekomme, aber den Würfel sicher habe? 
	  } 
	   } 
	   } 
	  \\ 
 

    \theutterance \stepcounter{utterance}  

    & & & \multicolumn{2}{p{0.3\linewidth}}{\cellcolor[rgb]{0.95,0.95,0.95}{%
	\makecell[{{p{\linewidth}}}]{% 
	  \tt {\tiny [GM$|$GM]}  
	 continue 
	  } 
	   } 
	   } 
	 & & \\ 
 

    \theutterance \stepcounter{utterance}  

    & & & \multicolumn{2}{p{0.3\linewidth}}{\cellcolor[rgb]{0.95,0.95,0.95}{%
	\makecell[{{p{\linewidth}}}]{% 
	  \tt {\tiny [GM$|$GM]}  
	 Das ist ein interessanter Vorschlag! Ich denke, es ist großartig, dass du den Würfel und die Pfanne priorisierst, da ich wie gesagt nicht so viel Wert auf die Pfanne lege. Ich könnte mir vorstellen, dass ich die Ringe bekomme, aber ich würde gerne auch den Würfel haben. Wie wäre es, wenn ich zwei Ringe und den Würfel bekomme und du den Rest? Oder vielleicht könnten wir auch eine Kombination finden, bei der ich einen Ring weniger bekomme, aber den Würfel sicher habe? 
	  } 
	   } 
	   } 
	 & & \\ 
 

    \theutterance \stepcounter{utterance}  

    & & \multicolumn{4}{p{0.6\linewidth}}{\cellcolor[rgb]{0.9,0.9,0.9}{%
	\makecell[{{p{\linewidth}}}]{% 
	  \tt {\tiny [A$\langle$GM]}  
	 Das ist ein interessanter Vorschlag! Ich denke, es ist großartig, dass du den Würfel und die Pfanne priorisierst, da ich wie gesagt nicht so viel Wert auf die Pfanne lege. Ich könnte mir vorstellen, dass ich die Ringe bekomme, aber ich würde gerne auch den Würfel haben. Wie wäre es, wenn ich zwei Ringe und den Würfel bekomme und du den Rest? Oder vielleicht könnten wir auch eine Kombination finden, bei der ich einen Ring weniger bekomme, aber den Würfel sicher habe? 
	  } 
	   } 
	   } 
	 & \\ 
 

    \theutterance \stepcounter{utterance}  

    & \multicolumn{4}{p{0.6\linewidth}}{\cellcolor[rgb]{0.8,1,0.9}{%
	\makecell[{{p{\linewidth}}}]{% 
	  \tt {\tiny [A$\rangle$GM]}  
	 Das ist ein cleverer Gegenvorschlag! Ich denke, es ist großartig, dass du bereit bist, Kompromisse einzugehen. Dein Vorschlag, zwei Ringe und den Würfel zu bekommen, während ich den Rest erhalte, ist interessant. Allerdings muss ich zugeben, dass ich den Würfel wirklich sehr schätze und ihn gerne behalten würde. Ich könnte mir vorstellen, dass wir eine Lösung finden, bei der ich den Würfel und die Pfanne bekomme und du die drei Ringe. Oder vielleicht könnten wir auch eine andere Kombination finden, bei der du einen Ring opferst, um den Würfel zu bekommen? Was meinst du? 
	  } 
	   } 
	   } 
	 & & \\ 
 

    \theutterance \stepcounter{utterance}  

    & & & \multicolumn{2}{p{0.3\linewidth}}{\cellcolor[rgb]{0.95,0.95,0.95}{%
	\makecell[{{p{\linewidth}}}]{% 
	  \tt {\tiny [GM$|$GM]}  
	 continue 
	  } 
	   } 
	   } 
	 & & \\ 
 

    \theutterance \stepcounter{utterance}  

    & & & \multicolumn{2}{p{0.3\linewidth}}{\cellcolor[rgb]{0.95,0.95,0.95}{%
	\makecell[{{p{\linewidth}}}]{% 
	  \tt {\tiny [GM$|$GM]}  
	 Das ist ein cleverer Gegenvorschlag! Ich denke, es ist großartig, dass du bereit bist, Kompromisse einzugehen. Dein Vorschlag, zwei Ringe und den Würfel zu bekommen, während ich den Rest erhalte, ist interessant. Allerdings muss ich zugeben, dass ich den Würfel wirklich sehr schätze und ihn gerne behalten würde. Ich könnte mir vorstellen, dass wir eine Lösung finden, bei der ich den Würfel und die Pfanne bekomme und du die drei Ringe. Oder vielleicht könnten wir auch eine andere Kombination finden, bei der du einen Ring opferst, um den Würfel zu bekommen? Was meinst du? 
	  } 
	   } 
	   } 
	 & & \\ 
 

    \theutterance \stepcounter{utterance}  

    & & \multicolumn{4}{p{0.6\linewidth}}{\cellcolor[rgb]{0.9,0.9,0.9}{%
	\makecell[{{p{\linewidth}}}]{% 
	  \tt {\tiny [GM$\rangle$B]}  
	 Das ist ein cleverer Gegenvorschlag! Ich denke, es ist großartig, dass du bereit bist, Kompromisse einzugehen. Dein Vorschlag, zwei Ringe und den Würfel zu bekommen, während ich den Rest erhalte, ist interessant. Allerdings muss ich zugeben, dass ich den Würfel wirklich sehr schätze und ihn gerne behalten würde. Ich könnte mir vorstellen, dass wir eine Lösung finden, bei der ich den Würfel und die Pfanne bekomme und du die drei Ringe. Oder vielleicht könnten wir auch eine andere Kombination finden, bei der du einen Ring opferst, um den Würfel zu bekommen? Was meinst du? 
	  } 
	   } 
	   } 
	 & \\ 
 

    \theutterance \stepcounter{utterance}  

    & & & \multicolumn{4}{p{0.6\linewidth}}{\cellcolor[rgb]{1,0.85,0.72}{%
	\makecell[{{p{\linewidth}}}]{% 
	  \tt {\tiny [GM$\langle$B]}  
	 Ich verstehe, dass der Würfel für dich sehr wichtig ist, und ich respektiere das. Ich denke, es ist großartig, dass wir offen und ehrlich über unsere Vorlieben sprechen. Dein Vorschlag, dass ich die drei Ringe bekomme und du den Würfel und die Pfanne, klingt fair und gerecht. Ich denke, das wäre eine gute Lösung, da ich dann meine Priorität, die Ringe, erfülle und du deine Prioritäten, den Würfel und die Pfanne, erfüllst. Ich bin bereit, diesen Vorschlag zu akzeptieren und ihn als endgültige Lösung zu betrachten. Sollten wir unsere geheimen Vorschläge machen, um das Spiel zu beenden? 
	  } 
	   } 
	   } 
	  \\ 
 

    \theutterance \stepcounter{utterance}  

    & & & \multicolumn{2}{p{0.3\linewidth}}{\cellcolor[rgb]{0.95,0.95,0.95}{%
	\makecell[{{p{\linewidth}}}]{% 
	  \tt {\tiny [GM$|$GM]}  
	 continue 
	  } 
	   } 
	   } 
	 & & \\ 
 

    \theutterance \stepcounter{utterance}  

    & & & \multicolumn{2}{p{0.3\linewidth}}{\cellcolor[rgb]{0.95,0.95,0.95}{%
	\makecell[{{p{\linewidth}}}]{% 
	  \tt {\tiny [GM$|$GM]}  
	 Ich verstehe, dass der Würfel für dich sehr wichtig ist, und ich respektiere das. Ich denke, es ist großartig, dass wir offen und ehrlich über unsere Vorlieben sprechen. Dein Vorschlag, dass ich die drei Ringe bekomme und du den Würfel und die Pfanne, klingt fair und gerecht. Ich denke, das wäre eine gute Lösung, da ich dann meine Priorität, die Ringe, erfülle und du deine Prioritäten, den Würfel und die Pfanne, erfüllst. Ich bin bereit, diesen Vorschlag zu akzeptieren und ihn als endgültige Lösung zu betrachten. Sollten wir unsere geheimen Vorschläge machen, um das Spiel zu beenden? 
	  } 
	   } 
	   } 
	 & & \\ 
 

    \theutterance \stepcounter{utterance}  

    & & \multicolumn{4}{p{0.6\linewidth}}{\cellcolor[rgb]{0.9,0.9,0.9}{%
	\makecell[{{p{\linewidth}}}]{% 
	  \tt {\tiny [A$\langle$GM]}  
	 Ich verstehe, dass der Würfel für dich sehr wichtig ist, und ich respektiere das. Ich denke, es ist großartig, dass wir offen und ehrlich über unsere Vorlieben sprechen. Dein Vorschlag, dass ich die drei Ringe bekomme und du den Würfel und die Pfanne, klingt fair und gerecht. Ich denke, das wäre eine gute Lösung, da ich dann meine Priorität, die Ringe, erfülle und du deine Prioritäten, den Würfel und die Pfanne, erfüllst. Ich bin bereit, diesen Vorschlag zu akzeptieren und ihn als endgültige Lösung zu betrachten. Sollten wir unsere geheimen Vorschläge machen, um das Spiel zu beenden? 
	  } 
	   } 
	   } 
	 & \\ 
 

    \theutterance \stepcounter{utterance}  

    & \multicolumn{4}{p{0.6\linewidth}}{\cellcolor[rgb]{0.8,1,0.9}{%
	\makecell[{{p{\linewidth}}}]{% 
	  \tt {\tiny [A$\rangle$GM]}  
	 Ja, ich denke, wir haben eine faire und gerechte Lösung gefunden! Ich bin auch bereit, diesen Vorschlag zu akzeptieren und ihn als endgültige Lösung zu betrachten. Wir sollten jetzt unsere geheimen Vorschläge machen, um das Spiel zu beenden. Ich werde meinen Vorschlag machen, und ich hoffe, dass er komplementär zu deinem ist, damit wir beide unsere Ziele erreichen können. Ich werde jetzt meinen geheimen Vorschlag machen: [Vorschlag: 1 Würfel, 1 Pfanne] 
	  } 
	   } 
	   } 
	 & & \\ 
 

    \theutterance \stepcounter{utterance}  

    & & & \multicolumn{2}{p{0.3\linewidth}}{\cellcolor[rgb]{0.95,0.95,0.95}{%
	\makecell[{{p{\linewidth}}}]{% 
	  \tt {\tiny [GM$|$GM]}  
	 proposal 
	  } 
	   } 
	   } 
	 & & \\ 
 

    \theutterance \stepcounter{utterance}  

    & & & \multicolumn{2}{p{0.3\linewidth}}{\cellcolor[rgb]{0.95,0.95,0.95}{%
	\makecell[{{p{\linewidth}}}]{% 
	  \tt {\tiny [GM$|$GM]}  
	 [0, 1, 1] 
	  } 
	   } 
	   } 
	 & & \\ 
 

    \theutterance \stepcounter{utterance}  

    & & \multicolumn{4}{p{0.6\linewidth}}{\cellcolor[rgb]{0.9,0.9,0.9}{%
	\makecell[{{p{\linewidth}}}]{% 
	  \tt {\tiny [GM$\rangle$B]}  
	 Der andere Spieler hat einen geheimen Vorschlag gemacht. Jetzt bist du an der Reihe, einen eigenen Vorschlag zu machen.\\ \tt  
	  } 
	   } 
	   } 
	 & \\ 
 

    \theutterance \stepcounter{utterance}  

    & & & \multicolumn{4}{p{0.6\linewidth}}{\cellcolor[rgb]{1,0.85,0.72}{%
	\makecell[{{p{\linewidth}}}]{% 
	  \tt {\tiny [GM$\langle$B]}  
	 [Vorschlag: 3 Ringe] 
	  } 
	   } 
	   } 
	  \\ 
 

    \theutterance \stepcounter{utterance}  

    & & & \multicolumn{2}{p{0.3\linewidth}}{\cellcolor[rgb]{0.95,0.95,0.95}{%
	\makecell[{{p{\linewidth}}}]{% 
	  \tt {\tiny [GM$|$GM]}  
	 proposal 
	  } 
	   } 
	   } 
	 & & \\ 
 

    \theutterance \stepcounter{utterance}  

    & & & \multicolumn{2}{p{0.3\linewidth}}{\cellcolor[rgb]{0.95,0.95,0.95}{%
	\makecell[{{p{\linewidth}}}]{% 
	  \tt {\tiny [GM$|$GM]}  
	 [3, 0, 0] 
	  } 
	   } 
	   } 
	 & & \\ 
 

    \theutterance \stepcounter{utterance}  

    & & & \multicolumn{2}{p{0.3\linewidth}}{\cellcolor[rgb]{0.95,0.95,0.95}{%
	\makecell[{{p{\linewidth}}}]{% 
	  \tt {\tiny [GM$|$GM]}  
	 [[0, 1, 1], [3, 0, 0]] 
	  } 
	   } 
	   } 
	 & & \\ 
 

\end{supertabular}
}

\end{document}
