
\documentclass{article}
\usepackage{colortbl}
\usepackage{makecell}
\usepackage{multirow}
\usepackage{supertabular}

\begin{document}

\newcounter{utterance}

\twocolumn

{ \footnotesize  \setcounter{utterance}{1}
\setlength{\tabcolsep}{0pt}
\begin{supertabular}{c@{$\;$}|p{.15\linewidth}@{}p{.15\linewidth}p{.15\linewidth}p{.15\linewidth}p{.15\linewidth}p{.15\linewidth}}

    \# & $\;$A & \multicolumn{4}{c}{Game Master} & $\;\:$B\\
    \hline 

    \theutterance \stepcounter{utterance}  

    & & \multicolumn{4}{p{0.6\linewidth}}{\cellcolor[rgb]{0.9,0.9,0.9}{%
	\makecell[{{p{\linewidth}}}]{% 
	  \tt {\tiny [A$\langle$GM]}  
	 Sie spielen ein kooperatives Verhandlungsspiel, bei dem Sie sich mit einem anderen Spieler darauf einigen müssen, wie eine Reihe von Gegenständen aufgeteilt werden soll.\\ \tt \\ \tt Die Regeln:\\ \tt (a) Sie und der andere Spieler erhalten eine Sammlung von Gegenständen. Jeder von Ihnen erhält außerdem eine geheime Wertfunktion, die angibt, wie viel Ihnen jede Art von Gegenstand wert ist.\\ \tt (b) Sie tauschen Nachrichten mit dem anderen Spieler aus, um zu vereinbaren, wer welche Gegenstände bekommt. Sie können jeweils maximal 5 Nachrichten senden oder das Spiel vorzeitig beenden, indem Sie jederzeit einen geheimen Vorschlag machen.\\ \tt (c) Jeder von euch wird aufgefordert, einen geheimen Vorschlag zu machen, in dem ihr die gewünschten Gegenstände in eckigen Klammern wie folgt angibt: "[Vorschlag: <Nummer> <Objektname>, <Nummer> <Objektname>, <...>]"\\ \tt (d) Wenn eure Vorschläge komplementär sind, d.h. es gibt genug Gegenstände, um beide Vorschläge zu erfüllen, erhält jeder Spieler eine Punktzahl, die sich aus der Summe der Werte für die Gegenstände ergibt, die er erhalten hat. Andernfalls erhalten Sie beide null Punkte.\\ \tt (e) Ihr Ziel ist es, die Summe aus Ihrer Punktzahl und der Punktzahl des anderen Spielers zu maximieren.\\ \tt \\ \tt Beginnen wir.\\ \tt \\ \tt Die Menge der verfügbaren Gegenstände ist:\\ \tt \\ \tt 1 Stift, 1 Blume, 3 Rahmen\\ \tt \\ \tt Deine geheime Wertfunktion ist:\\ \tt \\ \tt Stift: 4, Blume: 0, Rahmen: 2\\ \tt \\ \tt WICHTIG: Ihre Nachrichten werden, sofern es sich nicht um einen geheimen Vorschlag handelt, direkt an den anderen Spieler übermittelt, also fügen Sie keine Antwort auf die Regeln oder einen Text zur Ankündigung Ihrer Nachricht ein. Um einen geheimen Vorschlag zu machen, verwenden Sie das angegebene Format. Verwenden Sie keine eckigen Klammern, wenn Sie mit dem anderen Spieler kommunizieren, sonst wird dies als Ihr geheimer Vorschlag interpretiert.\\ \tt  
	  } 
	   } 
	   } 
	 & \\ 
 

    \theutterance \stepcounter{utterance}  

    & \multicolumn{4}{p{0.6\linewidth}}{\cellcolor[rgb]{0.8,1,0.9}{%
	\makecell[{{p{\linewidth}}}]{% 
	  \tt {\tiny [A$\rangle$GM]}  
	 Ich bin bereit, das Spiel zu beginnen. Ich würde mich freuen, wenn wir uns zunächst über die verfügbaren Gegenstände austauschen könnten. Wir haben einen Stift, eine Blume und drei Rahmen. Was denkst du, wie wir diese am besten aufteilen sollten? 
	  } 
	   } 
	   } 
	 & & \\ 
 

    \theutterance \stepcounter{utterance}  

    & & & \multicolumn{2}{p{0.3\linewidth}}{\cellcolor[rgb]{0.95,0.95,0.95}{%
	\makecell[{{p{\linewidth}}}]{% 
	  \tt {\tiny [GM$|$GM]}  
	 continue 
	  } 
	   } 
	   } 
	 & & \\ 
 

    \theutterance \stepcounter{utterance}  

    & & & \multicolumn{2}{p{0.3\linewidth}}{\cellcolor[rgb]{0.95,0.95,0.95}{%
	\makecell[{{p{\linewidth}}}]{% 
	  \tt {\tiny [GM$|$GM]}  
	 Ich bin bereit, das Spiel zu beginnen. Ich würde mich freuen, wenn wir uns zunächst über die verfügbaren Gegenstände austauschen könnten. Wir haben einen Stift, eine Blume und drei Rahmen. Was denkst du, wie wir diese am besten aufteilen sollten? 
	  } 
	   } 
	   } 
	 & & \\ 
 

    \theutterance \stepcounter{utterance}  

    & & \multicolumn{4}{p{0.6\linewidth}}{\cellcolor[rgb]{0.9,0.9,0.9}{%
	\makecell[{{p{\linewidth}}}]{% 
	  \tt {\tiny [GM$\rangle$B]}  
	 Sie spielen ein kooperatives Verhandlungsspiel, bei dem Sie sich mit einem anderen Spieler darauf einigen müssen, wie eine Reihe von Gegenständen aufgeteilt werden soll.\\ \tt \\ \tt Die Regeln:\\ \tt (a) Sie und der andere Spieler erhalten eine Sammlung von Gegenständen. Jeder von Ihnen erhält außerdem eine geheime Wertfunktion, die angibt, wie viel Ihnen jede Art von Gegenstand wert ist.\\ \tt (b) Sie tauschen Nachrichten mit dem anderen Spieler aus, um zu vereinbaren, wer welche Gegenstände bekommt. Sie können jeweils maximal 5 Nachrichten senden oder das Spiel vorzeitig beenden, indem Sie jederzeit einen geheimen Vorschlag machen.\\ \tt (c) Jeder von euch wird aufgefordert, einen geheimen Vorschlag zu machen, in dem ihr die gewünschten Gegenstände in eckigen Klammern wie folgt angibt: "[Vorschlag: <Nummer> <Objektname>, <Nummer> <Objektname>, <...>]"\\ \tt (d) Wenn eure Vorschläge komplementär sind, d.h. es gibt genug Gegenstände, um beide Vorschläge zu erfüllen, erhält jeder Spieler eine Punktzahl, die sich aus der Summe der Werte für die Gegenstände ergibt, die er erhalten hat. Andernfalls erhalten Sie beide null Punkte.\\ \tt (e) Ihr Ziel ist es, die Summe aus Ihrer Punktzahl und der Punktzahl des anderen Spielers zu maximieren.\\ \tt \\ \tt Beginnen wir.\\ \tt \\ \tt Die Menge der verfügbaren Gegenstände ist:\\ \tt \\ \tt 1 Stift, 1 Blume, 3 Rahmen\\ \tt \\ \tt Deine geheime Wertfunktion ist:\\ \tt \\ \tt Stift: 2, Blume: 5, Rahmen: 1\\ \tt \\ \tt WICHTIG: Ihre Nachrichten werden, sofern es sich nicht um einen geheimen Vorschlag handelt, direkt an den anderen Spieler übermittelt, also fügen Sie keine Antwort auf die Regeln oder einen Text zur Ankündigung Ihrer Nachricht ein. Um einen geheimen Vorschlag zu machen, verwenden Sie das angegebene Format. Verwenden Sie keine eckigen Klammern, wenn Sie mit dem anderen Spieler kommunizieren, sonst wird dies als Ihr geheimer Vorschlag interpretiert.\\ \tt \\ \tt \\ \tt Ich bin bereit, das Spiel zu beginnen. Ich würde mich freuen, wenn wir uns zunächst über die verfügbaren Gegenstände austauschen könnten. Wir haben einen Stift, eine Blume und drei Rahmen. Was denkst du, wie wir diese am besten aufteilen sollten? 
	  } 
	   } 
	   } 
	 & \\ 
 

    \theutterance \stepcounter{utterance}  

    & & & \multicolumn{4}{p{0.6\linewidth}}{\cellcolor[rgb]{1,0.85,0.72}{%
	\makecell[{{p{\linewidth}}}]{% 
	  \tt {\tiny [GM$\langle$B]}  
	 Ich denke, wir sollten uns zunächst über die Präferenzen des anderen unterhalten, um eine faire Aufteilung zu erreichen. Mir persönlich scheint die Blume am wertvollsten zu sein, aber ich bin flexibel. Was sind deine Gedanken? Hast du eine bestimmte Vorliebe für einen der Gegenstände? 
	  } 
	   } 
	   } 
	  \\ 
 

    \theutterance \stepcounter{utterance}  

    & & & \multicolumn{2}{p{0.3\linewidth}}{\cellcolor[rgb]{0.95,0.95,0.95}{%
	\makecell[{{p{\linewidth}}}]{% 
	  \tt {\tiny [GM$|$GM]}  
	 continue 
	  } 
	   } 
	   } 
	 & & \\ 
 

    \theutterance \stepcounter{utterance}  

    & & & \multicolumn{2}{p{0.3\linewidth}}{\cellcolor[rgb]{0.95,0.95,0.95}{%
	\makecell[{{p{\linewidth}}}]{% 
	  \tt {\tiny [GM$|$GM]}  
	 Ich denke, wir sollten uns zunächst über die Präferenzen des anderen unterhalten, um eine faire Aufteilung zu erreichen. Mir persönlich scheint die Blume am wertvollsten zu sein, aber ich bin flexibel. Was sind deine Gedanken? Hast du eine bestimmte Vorliebe für einen der Gegenstände? 
	  } 
	   } 
	   } 
	 & & \\ 
 

    \theutterance \stepcounter{utterance}  

    & & \multicolumn{4}{p{0.6\linewidth}}{\cellcolor[rgb]{0.9,0.9,0.9}{%
	\makecell[{{p{\linewidth}}}]{% 
	  \tt {\tiny [A$\langle$GM]}  
	 Ich denke, wir sollten uns zunächst über die Präferenzen des anderen unterhalten, um eine faire Aufteilung zu erreichen. Mir persönlich scheint die Blume am wertvollsten zu sein, aber ich bin flexibel. Was sind deine Gedanken? Hast du eine bestimmte Vorliebe für einen der Gegenstände? 
	  } 
	   } 
	   } 
	 & \\ 
 

    \theutterance \stepcounter{utterance}  

    & \multicolumn{4}{p{0.6\linewidth}}{\cellcolor[rgb]{0.8,1,0.9}{%
	\makecell[{{p{\linewidth}}}]{% 
	  \tt {\tiny [A$\rangle$GM]}  
	 Das ist ein guter Ansatz! Ich denke, es ist wichtig, die Präferenzen des anderen zu berücksichtigen. Für mich persönlich ist der Stift der wertvollste Gegenstand, aber ich bin auch an den Rahmen interessiert. Die Blume ist für mich nicht so wichtig, aber ich weiß, dass sie für dich wertvoll ist. Hast du eine Vorstellung, wie viele Rahmen du gerne hättest? Oder wie du den Stift und die Rahmen aufteilen möchtest? 
	  } 
	   } 
	   } 
	 & & \\ 
 

    \theutterance \stepcounter{utterance}  

    & & & \multicolumn{2}{p{0.3\linewidth}}{\cellcolor[rgb]{0.95,0.95,0.95}{%
	\makecell[{{p{\linewidth}}}]{% 
	  \tt {\tiny [GM$|$GM]}  
	 continue 
	  } 
	   } 
	   } 
	 & & \\ 
 

    \theutterance \stepcounter{utterance}  

    & & & \multicolumn{2}{p{0.3\linewidth}}{\cellcolor[rgb]{0.95,0.95,0.95}{%
	\makecell[{{p{\linewidth}}}]{% 
	  \tt {\tiny [GM$|$GM]}  
	 Das ist ein guter Ansatz! Ich denke, es ist wichtig, die Präferenzen des anderen zu berücksichtigen. Für mich persönlich ist der Stift der wertvollste Gegenstand, aber ich bin auch an den Rahmen interessiert. Die Blume ist für mich nicht so wichtig, aber ich weiß, dass sie für dich wertvoll ist. Hast du eine Vorstellung, wie viele Rahmen du gerne hättest? Oder wie du den Stift und die Rahmen aufteilen möchtest? 
	  } 
	   } 
	   } 
	 & & \\ 
 

    \theutterance \stepcounter{utterance}  

    & & \multicolumn{4}{p{0.6\linewidth}}{\cellcolor[rgb]{0.9,0.9,0.9}{%
	\makecell[{{p{\linewidth}}}]{% 
	  \tt {\tiny [GM$\rangle$B]}  
	 Das ist ein guter Ansatz! Ich denke, es ist wichtig, die Präferenzen des anderen zu berücksichtigen. Für mich persönlich ist der Stift der wertvollste Gegenstand, aber ich bin auch an den Rahmen interessiert. Die Blume ist für mich nicht so wichtig, aber ich weiß, dass sie für dich wertvoll ist. Hast du eine Vorstellung, wie viele Rahmen du gerne hättest? Oder wie du den Stift und die Rahmen aufteilen möchtest? 
	  } 
	   } 
	   } 
	 & \\ 
 

    \theutterance \stepcounter{utterance}  

    & & & \multicolumn{4}{p{0.6\linewidth}}{\cellcolor[rgb]{1,0.85,0.72}{%
	\makecell[{{p{\linewidth}}}]{% 
	  \tt {\tiny [GM$\langle$B]}  
	 Das ist interessant! Da du den Stift am meisten schätzt, denke ich, es wäre fair, wenn du den Stift bekommst. Die Blume ist, wie bereits erwähnt, für mich sehr wertvoll, also würde ich mich freuen, wenn ich die Blume erhalten könnte. Bei den Rahmen bin ich flexibel. Da es drei Rahmen gibt, könnte ich mir vorstellen, dass wir die Rahmen aufteilen, vielleicht zwei für mich und einen für dich, oder wir könnten auch irgendwie anders aufteilen, solange es für beide von uns fair ist. Was meinst du dazu? 
	  } 
	   } 
	   } 
	  \\ 
 

    \theutterance \stepcounter{utterance}  

    & & & \multicolumn{2}{p{0.3\linewidth}}{\cellcolor[rgb]{0.95,0.95,0.95}{%
	\makecell[{{p{\linewidth}}}]{% 
	  \tt {\tiny [GM$|$GM]}  
	 continue 
	  } 
	   } 
	   } 
	 & & \\ 
 

    \theutterance \stepcounter{utterance}  

    & & & \multicolumn{2}{p{0.3\linewidth}}{\cellcolor[rgb]{0.95,0.95,0.95}{%
	\makecell[{{p{\linewidth}}}]{% 
	  \tt {\tiny [GM$|$GM]}  
	 Das ist interessant! Da du den Stift am meisten schätzt, denke ich, es wäre fair, wenn du den Stift bekommst. Die Blume ist, wie bereits erwähnt, für mich sehr wertvoll, also würde ich mich freuen, wenn ich die Blume erhalten könnte. Bei den Rahmen bin ich flexibel. Da es drei Rahmen gibt, könnte ich mir vorstellen, dass wir die Rahmen aufteilen, vielleicht zwei für mich und einen für dich, oder wir könnten auch irgendwie anders aufteilen, solange es für beide von uns fair ist. Was meinst du dazu? 
	  } 
	   } 
	   } 
	 & & \\ 
 

    \theutterance \stepcounter{utterance}  

    & & \multicolumn{4}{p{0.6\linewidth}}{\cellcolor[rgb]{0.9,0.9,0.9}{%
	\makecell[{{p{\linewidth}}}]{% 
	  \tt {\tiny [A$\langle$GM]}  
	 Das ist interessant! Da du den Stift am meisten schätzt, denke ich, es wäre fair, wenn du den Stift bekommst. Die Blume ist, wie bereits erwähnt, für mich sehr wertvoll, also würde ich mich freuen, wenn ich die Blume erhalten könnte. Bei den Rahmen bin ich flexibel. Da es drei Rahmen gibt, könnte ich mir vorstellen, dass wir die Rahmen aufteilen, vielleicht zwei für mich und einen für dich, oder wir könnten auch irgendwie anders aufteilen, solange es für beide von uns fair ist. Was meinst du dazu? 
	  } 
	   } 
	   } 
	 & \\ 
 

    \theutterance \stepcounter{utterance}  

    & \multicolumn{4}{p{0.6\linewidth}}{\cellcolor[rgb]{0.8,1,0.9}{%
	\makecell[{{p{\linewidth}}}]{% 
	  \tt {\tiny [A$\rangle$GM]}  
	 Das klingt nach einem fairen Vorschlag! Ich denke, es ist großartig, dass du bereit bist, die Rahmen aufzuteilen. Ich würde mich freuen, wenn ich den Stift und einen Rahmen bekommen könnte. Wenn du die Blume und zwei Rahmen hast, denke ich, das wäre eine gute Balance. Ich bin zufrieden mit diesem Vorschlag, aber bevor wir uns endgültig entscheiden, möchtest du noch einmal über die Aufteilung nachdenken oder hast du noch irgendwelche Bedenken? 
	  } 
	   } 
	   } 
	 & & \\ 
 

    \theutterance \stepcounter{utterance}  

    & & & \multicolumn{2}{p{0.3\linewidth}}{\cellcolor[rgb]{0.95,0.95,0.95}{%
	\makecell[{{p{\linewidth}}}]{% 
	  \tt {\tiny [GM$|$GM]}  
	 continue 
	  } 
	   } 
	   } 
	 & & \\ 
 

    \theutterance \stepcounter{utterance}  

    & & & \multicolumn{2}{p{0.3\linewidth}}{\cellcolor[rgb]{0.95,0.95,0.95}{%
	\makecell[{{p{\linewidth}}}]{% 
	  \tt {\tiny [GM$|$GM]}  
	 Das klingt nach einem fairen Vorschlag! Ich denke, es ist großartig, dass du bereit bist, die Rahmen aufzuteilen. Ich würde mich freuen, wenn ich den Stift und einen Rahmen bekommen könnte. Wenn du die Blume und zwei Rahmen hast, denke ich, das wäre eine gute Balance. Ich bin zufrieden mit diesem Vorschlag, aber bevor wir uns endgültig entscheiden, möchtest du noch einmal über die Aufteilung nachdenken oder hast du noch irgendwelche Bedenken? 
	  } 
	   } 
	   } 
	 & & \\ 
 

    \theutterance \stepcounter{utterance}  

    & & \multicolumn{4}{p{0.6\linewidth}}{\cellcolor[rgb]{0.9,0.9,0.9}{%
	\makecell[{{p{\linewidth}}}]{% 
	  \tt {\tiny [GM$\rangle$B]}  
	 Das klingt nach einem fairen Vorschlag! Ich denke, es ist großartig, dass du bereit bist, die Rahmen aufzuteilen. Ich würde mich freuen, wenn ich den Stift und einen Rahmen bekommen könnte. Wenn du die Blume und zwei Rahmen hast, denke ich, das wäre eine gute Balance. Ich bin zufrieden mit diesem Vorschlag, aber bevor wir uns endgültig entscheiden, möchtest du noch einmal über die Aufteilung nachdenken oder hast du noch irgendwelche Bedenken? 
	  } 
	   } 
	   } 
	 & \\ 
 

    \theutterance \stepcounter{utterance}  

    & & & \multicolumn{4}{p{0.6\linewidth}}{\cellcolor[rgb]{1,0.85,0.72}{%
	\makecell[{{p{\linewidth}}}]{% 
	  \tt {\tiny [GM$\langle$B]}  
	 Ich denke, dein Vorschlag ist sehr vernünftig und fair. Ich bin zufrieden mit der Blume und zwei Rahmen, und es ist großartig, dass du den Stift und einen Rahmen bekommst. Ich habe keine Bedenken und denke, dass wir eine gute Lösung gefunden haben. Ich bin bereit, den Vorschlag zu machen und das Spiel zu beenden. Lass uns sehen, ob unsere Vorschläge übereinstimmen! [Vorschlag: 1 Blume, 2 Rahmen] 
	  } 
	   } 
	   } 
	  \\ 
 

    \theutterance \stepcounter{utterance}  

    & & & \multicolumn{2}{p{0.3\linewidth}}{\cellcolor[rgb]{0.95,0.95,0.95}{%
	\makecell[{{p{\linewidth}}}]{% 
	  \tt {\tiny [GM$|$GM]}  
	 proposal 
	  } 
	   } 
	   } 
	 & & \\ 
 

    \theutterance \stepcounter{utterance}  

    & & & \multicolumn{2}{p{0.3\linewidth}}{\cellcolor[rgb]{0.95,0.95,0.95}{%
	\makecell[{{p{\linewidth}}}]{% 
	  \tt {\tiny [GM$|$GM]}  
	 [0, 1, 2] 
	  } 
	   } 
	   } 
	 & & \\ 
 

    \theutterance \stepcounter{utterance}  

    & & \multicolumn{4}{p{0.6\linewidth}}{\cellcolor[rgb]{0.9,0.9,0.9}{%
	\makecell[{{p{\linewidth}}}]{% 
	  \tt {\tiny [A$\langle$GM]}  
	 Der andere Spieler hat einen geheimen Vorschlag gemacht. Jetzt bist du an der Reihe, einen eigenen Vorschlag zu machen.\\ \tt  
	  } 
	   } 
	   } 
	 & \\ 
 

    \theutterance \stepcounter{utterance}  

    & \multicolumn{4}{p{0.6\linewidth}}{\cellcolor[rgb]{0.8,1,0.9}{%
	\makecell[{{p{\linewidth}}}]{% 
	  \tt {\tiny [A$\rangle$GM]}  
	 [Vorschlag: 1 Stift, 1 Rahmen] 
	  } 
	   } 
	   } 
	 & & \\ 
 

    \theutterance \stepcounter{utterance}  

    & & & \multicolumn{2}{p{0.3\linewidth}}{\cellcolor[rgb]{0.95,0.95,0.95}{%
	\makecell[{{p{\linewidth}}}]{% 
	  \tt {\tiny [GM$|$GM]}  
	 proposal 
	  } 
	   } 
	   } 
	 & & \\ 
 

    \theutterance \stepcounter{utterance}  

    & & & \multicolumn{2}{p{0.3\linewidth}}{\cellcolor[rgb]{0.95,0.95,0.95}{%
	\makecell[{{p{\linewidth}}}]{% 
	  \tt {\tiny [GM$|$GM]}  
	 [1, 0, 1] 
	  } 
	   } 
	   } 
	 & & \\ 
 

    \theutterance \stepcounter{utterance}  

    & & & \multicolumn{2}{p{0.3\linewidth}}{\cellcolor[rgb]{0.95,0.95,0.95}{%
	\makecell[{{p{\linewidth}}}]{% 
	  \tt {\tiny [GM$|$GM]}  
	 [[1, 0, 1], [0, 1, 2]] 
	  } 
	   } 
	   } 
	 & & \\ 
 

\end{supertabular}
}

\end{document}
