
\documentclass{article}
\usepackage{colortbl}
\usepackage{makecell}
\usepackage{multirow}
\usepackage{supertabular}

\begin{document}

\newcounter{utterance}

\centering \large Interaction Transcript for game `dond', experiment `coop\_de', episode 9 with llama{-}3.3{-}70b{-}instruct{-}t1.0.
\vspace{24pt}

{ \footnotesize  \setcounter{utterance}{1}
\setlength{\tabcolsep}{0pt}
\begin{supertabular}{c@{$\;$}|p{.15\linewidth}@{}p{.15\linewidth}p{.15\linewidth}p{.15\linewidth}p{.15\linewidth}p{.15\linewidth}}
    \# & $\;$A & \multicolumn{4}{c}{Game Master} & $\;\:$B\\
    \hline

    \theutterance \stepcounter{utterance}  
    & & \multicolumn{4}{p{0.6\linewidth}}{
        \cellcolor[rgb]{0.9,0.9,0.9}{
            \makecell[{{p{\linewidth}}}]{
                \texttt{\tiny{[P1$\langle$GM]}}
                \texttt{Sie spielen ein kooperatives Verhandlungsspiel, bei dem Sie sich mit einem anderen Spieler darauf einigen müssen, wie eine Reihe von Gegenständen aufgeteilt werden soll.} \\
\\ 
\texttt{Die Regeln:} \\
\texttt{(a) Sie und der andere Spieler erhalten eine Sammlung von Gegenständen. Jeder von Ihnen erhält außerdem eine geheime Wertfunktion, die angibt, wie viel Ihnen jede Art von Gegenstand wert ist.} \\
\texttt{(b) Sie tauschen Nachrichten mit dem anderen Spieler aus, um zu vereinbaren, wer welche Gegenstände bekommt. Sie können jeweils maximal 5 Nachrichten senden oder das Spiel vorzeitig beenden, indem Sie jederzeit einen geheimen Vorschlag machen.} \\
\texttt{(c) Jeder von euch wird aufgefordert, einen geheimen Vorschlag zu machen, in dem ihr die gewünschten Gegenstände in eckigen Klammern wie folgt angibt: "{[}Vorschlag: <Nummer> <Objektname>, <Nummer> <Objektname>, <...>{]}"} \\
\texttt{(d) Wenn eure Vorschläge komplementär sind, d.h. es gibt genug Gegenstände, um beide Vorschläge zu erfüllen, erhält jeder Spieler eine Punktzahl, die sich aus der Summe der Werte für die Gegenstände ergibt, die er erhalten hat. Andernfalls erhalten Sie beide null Punkte.} \\
\texttt{(e) Ihr Ziel ist es, die Summe aus Ihrer Punktzahl und der Punktzahl des anderen Spielers zu maximieren.} \\
\\ 
\texttt{Beginnen wir.} \\
\\ 
\texttt{Die Menge der verfügbaren Gegenstände ist:} \\
\\ 
\texttt{1 Bolzen, 1 Bild, 1 Ei, 4 Karten} \\
\\ 
\texttt{Deine geheime Wertfunktion ist:} \\
\\ 
\texttt{Bolzen: 6, Bild: 4, Ei: 0, Karte: 0} \\
\\ 
\texttt{WICHTIG: Ihre Nachrichten werden, sofern es sich nicht um einen geheimen Vorschlag handelt, direkt an den anderen Spieler übermittelt, also fügen Sie keine Antwort auf die Regeln oder einen Text zur Ankündigung Ihrer Nachricht ein. Um einen geheimen Vorschlag zu machen, verwenden Sie das angegebene Format. Verwenden Sie keine eckigen Klammern, wenn Sie mit dem anderen Spieler kommunizieren, sonst wird dies als Ihr geheimer Vorschlag interpretiert.} \\
            }
        }
    }
    & \\ \\

    \theutterance \stepcounter{utterance}  
    & \multicolumn{4}{p{0.6\linewidth}}{
        \cellcolor[rgb]{0.9,0.9,0.9}{
            \makecell[{{p{\linewidth}}}]{
                \texttt{\tiny{[P1$\rangle$GM]}}
                \texttt{Hallo, ich denke, wir sollten versuchen, unsere Vorlieben auszutauschen, bevor wir einen Vorschlag machen. Was halten Sie von den Karten? Sind sie für Sie besonders wertvoll?} \\
            }
        }
    }
    & & \\ \\

    \theutterance \stepcounter{utterance}  
    & & & \multicolumn{2}{p{0.3\linewidth}}{
        \cellcolor[rgb]{0.9,0.9,0.9}{
            \makecell[{{p{\linewidth}}}]{
                \texttt{\tiny{[GM$|$GM]}}
                \texttt{continue} \\
            }
        }
    }
    & & \\ \\

    \theutterance \stepcounter{utterance}  
    & & & \multicolumn{2}{p{0.3\linewidth}}{
        \cellcolor[rgb]{0.9,0.9,0.9}{
            \makecell[{{p{\linewidth}}}]{
                \texttt{\tiny{[GM$|$GM]}}
                \texttt{Hallo, ich denke, wir sollten versuchen, unsere Vorlieben auszutauschen, bevor wir einen Vorschlag machen. Was halten Sie von den Karten? Sind sie für Sie besonders wertvoll?} \\
            }
        }
    }
    & & \\ \\

    \theutterance \stepcounter{utterance}  
    & & \multicolumn{4}{p{0.6\linewidth}}{
        \cellcolor[rgb]{0.9,0.9,0.9}{
            \makecell[{{p{\linewidth}}}]{
                \texttt{\tiny{[GM$\rangle$P2]}}
                \texttt{Sie spielen ein kooperatives Verhandlungsspiel, bei dem Sie sich mit einem anderen Spieler darauf einigen müssen, wie eine Reihe von Gegenständen aufgeteilt werden soll.} \\
\\ 
\texttt{Die Regeln:} \\
\texttt{(a) Sie und der andere Spieler erhalten eine Sammlung von Gegenständen. Jeder von Ihnen erhält außerdem eine geheime Wertfunktion, die angibt, wie viel Ihnen jede Art von Gegenstand wert ist.} \\
\texttt{(b) Sie tauschen Nachrichten mit dem anderen Spieler aus, um zu vereinbaren, wer welche Gegenstände bekommt. Sie können jeweils maximal 5 Nachrichten senden oder das Spiel vorzeitig beenden, indem Sie jederzeit einen geheimen Vorschlag machen.} \\
\texttt{(c) Jeder von euch wird aufgefordert, einen geheimen Vorschlag zu machen, in dem ihr die gewünschten Gegenstände in eckigen Klammern wie folgt angibt: "{[}Vorschlag: <Nummer> <Objektname>, <Nummer> <Objektname>, <...>{]}"} \\
\texttt{(d) Wenn eure Vorschläge komplementär sind, d.h. es gibt genug Gegenstände, um beide Vorschläge zu erfüllen, erhält jeder Spieler eine Punktzahl, die sich aus der Summe der Werte für die Gegenstände ergibt, die er erhalten hat. Andernfalls erhalten Sie beide null Punkte.} \\
\texttt{(e) Ihr Ziel ist es, die Summe aus Ihrer Punktzahl und der Punktzahl des anderen Spielers zu maximieren.} \\
\\ 
\texttt{Beginnen wir.} \\
\\ 
\texttt{Die Menge der verfügbaren Gegenstände ist:} \\
\\ 
\texttt{1 Bolzen, 1 Bild, 1 Ei, 4 Karten} \\
\\ 
\texttt{Deine geheime Wertfunktion ist:} \\
\\ 
\texttt{Bolzen: 0, Bild: 1, Ei: 1, Karte: 2} \\
\\ 
\texttt{WICHTIG: Ihre Nachrichten werden, sofern es sich nicht um einen geheimen Vorschlag handelt, direkt an den anderen Spieler übermittelt, also fügen Sie keine Antwort auf die Regeln oder einen Text zur Ankündigung Ihrer Nachricht ein. Um einen geheimen Vorschlag zu machen, verwenden Sie das angegebene Format. Verwenden Sie keine eckigen Klammern, wenn Sie mit dem anderen Spieler kommunizieren, sonst wird dies als Ihr geheimer Vorschlag interpretiert.} \\
\\ 
\\ 
\texttt{Hallo, ich denke, wir sollten versuchen, unsere Vorlieben auszutauschen, bevor wir einen Vorschlag machen. Was halten Sie von den Karten? Sind sie für Sie besonders wertvoll?} \\
            }
        }
    }
    & \\ \\

    \theutterance \stepcounter{utterance}  
    & & & \multicolumn{4}{p{0.6\linewidth}}{
        \cellcolor[rgb]{0.9,0.9,0.9}{
            \makecell[{{p{\linewidth}}}]{
                \texttt{\tiny{[GM$\langle$P2]}}
                \texttt{Ich denke, Karten sind ziemlich wertvoll, aber ich bin nicht sicher, ob ich bereit bin, viel dafür aufzugeben. Wie sehen Sie die Karten? Sind sie für Sie besonders wichtig oder gibt es andere Gegenstände, die Sie bevorzugen?} \\
            }
        }
    }
     \\ \\

    \theutterance \stepcounter{utterance}  
    & & & \multicolumn{2}{p{0.3\linewidth}}{
        \cellcolor[rgb]{0.9,0.9,0.9}{
            \makecell[{{p{\linewidth}}}]{
                \texttt{\tiny{[GM$|$GM]}}
                \texttt{continue} \\
            }
        }
    }
    & & \\ \\

    \theutterance \stepcounter{utterance}  
    & & & \multicolumn{2}{p{0.3\linewidth}}{
        \cellcolor[rgb]{0.9,0.9,0.9}{
            \makecell[{{p{\linewidth}}}]{
                \texttt{\tiny{[GM$|$GM]}}
                \texttt{Ich denke, Karten sind ziemlich wertvoll, aber ich bin nicht sicher, ob ich bereit bin, viel dafür aufzugeben. Wie sehen Sie die Karten? Sind sie für Sie besonders wichtig oder gibt es andere Gegenstände, die Sie bevorzugen?} \\
            }
        }
    }
    & & \\ \\

    \theutterance \stepcounter{utterance}  
    & & \multicolumn{4}{p{0.6\linewidth}}{
        \cellcolor[rgb]{0.9,0.9,0.9}{
            \makecell[{{p{\linewidth}}}]{
                \texttt{\tiny{[P1$\langle$GM]}}
                \texttt{Ich denke, Karten sind ziemlich wertvoll, aber ich bin nicht sicher, ob ich bereit bin, viel dafür aufzugeben. Wie sehen Sie die Karten? Sind sie für Sie besonders wichtig oder gibt es andere Gegenstände, die Sie bevorzugen?} \\
            }
        }
    }
    & \\ \\

    \theutterance \stepcounter{utterance}  
    & \multicolumn{4}{p{0.6\linewidth}}{
        \cellcolor[rgb]{0.9,0.9,0.9}{
            \makecell[{{p{\linewidth}}}]{
                \texttt{\tiny{[P1$\rangle$GM]}}
                \texttt{Für mich persönlich sind die Karten nicht besonders wichtig, es gibt andere Gegenstände, die ich viel mehr schätze. Ich denke, wir könnten eine gute Aufteilung finden, wenn wir uns auf die Dinge konzentrieren, die uns am meisten bedeuten. Haben Sie ein bestimmtes Interesse am Bolzen oder vielleicht am Bild?} \\
            }
        }
    }
    & & \\ \\

    \theutterance \stepcounter{utterance}  
    & & & \multicolumn{2}{p{0.3\linewidth}}{
        \cellcolor[rgb]{0.9,0.9,0.9}{
            \makecell[{{p{\linewidth}}}]{
                \texttt{\tiny{[GM$|$GM]}}
                \texttt{continue} \\
            }
        }
    }
    & & \\ \\

    \theutterance \stepcounter{utterance}  
    & & & \multicolumn{2}{p{0.3\linewidth}}{
        \cellcolor[rgb]{0.9,0.9,0.9}{
            \makecell[{{p{\linewidth}}}]{
                \texttt{\tiny{[GM$|$GM]}}
                \texttt{Für mich persönlich sind die Karten nicht besonders wichtig, es gibt andere Gegenstände, die ich viel mehr schätze. Ich denke, wir könnten eine gute Aufteilung finden, wenn wir uns auf die Dinge konzentrieren, die uns am meisten bedeuten. Haben Sie ein bestimmtes Interesse am Bolzen oder vielleicht am Bild?} \\
            }
        }
    }
    & & \\ \\

    \theutterance \stepcounter{utterance}  
    & & \multicolumn{4}{p{0.6\linewidth}}{
        \cellcolor[rgb]{0.9,0.9,0.9}{
            \makecell[{{p{\linewidth}}}]{
                \texttt{\tiny{[GM$\rangle$P2]}}
                \texttt{Für mich persönlich sind die Karten nicht besonders wichtig, es gibt andere Gegenstände, die ich viel mehr schätze. Ich denke, wir könnten eine gute Aufteilung finden, wenn wir uns auf die Dinge konzentrieren, die uns am meisten bedeuten. Haben Sie ein bestimmtes Interesse am Bolzen oder vielleicht am Bild?} \\
            }
        }
    }
    & \\ \\

    \theutterance \stepcounter{utterance}  
    & & & \multicolumn{4}{p{0.6\linewidth}}{
        \cellcolor[rgb]{0.9,0.9,0.9}{
            \makecell[{{p{\linewidth}}}]{
                \texttt{\tiny{[GM$\langle$P2]}}
                \texttt{Das klingt nach einem guten Ansatz! Ich muss sagen, dass ich nicht besonders an dem Bolzen interessiert bin. Das Bild ist schon interessanter, aber ich denke, es gibt einen Gegenstand, der für mich noch wertvoller ist. Ich bin bereit, Kompromisse einzugehen, um sicherzustellen, dass wir beide zufrieden sind. Wie stehen Sie zum Ei?} \\
            }
        }
    }
     \\ \\

    \theutterance \stepcounter{utterance}  
    & & & \multicolumn{2}{p{0.3\linewidth}}{
        \cellcolor[rgb]{0.9,0.9,0.9}{
            \makecell[{{p{\linewidth}}}]{
                \texttt{\tiny{[GM$|$GM]}}
                \texttt{continue} \\
            }
        }
    }
    & & \\ \\

    \theutterance \stepcounter{utterance}  
    & & & \multicolumn{2}{p{0.3\linewidth}}{
        \cellcolor[rgb]{0.9,0.9,0.9}{
            \makecell[{{p{\linewidth}}}]{
                \texttt{\tiny{[GM$|$GM]}}
                \texttt{Das klingt nach einem guten Ansatz! Ich muss sagen, dass ich nicht besonders an dem Bolzen interessiert bin. Das Bild ist schon interessanter, aber ich denke, es gibt einen Gegenstand, der für mich noch wertvoller ist. Ich bin bereit, Kompromisse einzugehen, um sicherzustellen, dass wir beide zufrieden sind. Wie stehen Sie zum Ei?} \\
            }
        }
    }
    & & \\ \\

    \theutterance \stepcounter{utterance}  
    & & \multicolumn{4}{p{0.6\linewidth}}{
        \cellcolor[rgb]{0.9,0.9,0.9}{
            \makecell[{{p{\linewidth}}}]{
                \texttt{\tiny{[P1$\langle$GM]}}
                \texttt{Das klingt nach einem guten Ansatz! Ich muss sagen, dass ich nicht besonders an dem Bolzen interessiert bin. Das Bild ist schon interessanter, aber ich denke, es gibt einen Gegenstand, der für mich noch wertvoller ist. Ich bin bereit, Kompromisse einzugehen, um sicherzustellen, dass wir beide zufrieden sind. Wie stehen Sie zum Ei?} \\
            }
        }
    }
    & \\ \\

    \theutterance \stepcounter{utterance}  
    & \multicolumn{4}{p{0.6\linewidth}}{
        \cellcolor[rgb]{0.9,0.9,0.9}{
            \makecell[{{p{\linewidth}}}]{
                \texttt{\tiny{[P1$\rangle$GM]}}
                \texttt{Das Ei ist für mich nicht von großem Wert, ich denke, es könnte ein guter Kandidat sein, den wir austauschen oder verteilen können, ohne dass jemand zu sehr dabei verliert. Ich bin eher an anderen Gegenständen interessiert, aber ich denke, wir kommen langsam voran. Wie viele Karten würden Sie minimalemäßig benötigen, um zufrieden zu sein, und könnten wir vielleicht das Bild oder den Bolzen als Teil des Angebots einbeziehen?} \\
            }
        }
    }
    & & \\ \\

    \theutterance \stepcounter{utterance}  
    & & & \multicolumn{2}{p{0.3\linewidth}}{
        \cellcolor[rgb]{0.9,0.9,0.9}{
            \makecell[{{p{\linewidth}}}]{
                \texttt{\tiny{[GM$|$GM]}}
                \texttt{continue} \\
            }
        }
    }
    & & \\ \\

    \theutterance \stepcounter{utterance}  
    & & & \multicolumn{2}{p{0.3\linewidth}}{
        \cellcolor[rgb]{0.9,0.9,0.9}{
            \makecell[{{p{\linewidth}}}]{
                \texttt{\tiny{[GM$|$GM]}}
                \texttt{Das Ei ist für mich nicht von großem Wert, ich denke, es könnte ein guter Kandidat sein, den wir austauschen oder verteilen können, ohne dass jemand zu sehr dabei verliert. Ich bin eher an anderen Gegenständen interessiert, aber ich denke, wir kommen langsam voran. Wie viele Karten würden Sie minimalemäßig benötigen, um zufrieden zu sein, und könnten wir vielleicht das Bild oder den Bolzen als Teil des Angebots einbeziehen?} \\
            }
        }
    }
    & & \\ \\

    \theutterance \stepcounter{utterance}  
    & & \multicolumn{4}{p{0.6\linewidth}}{
        \cellcolor[rgb]{0.9,0.9,0.9}{
            \makecell[{{p{\linewidth}}}]{
                \texttt{\tiny{[GM$\rangle$P2]}}
                \texttt{Das Ei ist für mich nicht von großem Wert, ich denke, es könnte ein guter Kandidat sein, den wir austauschen oder verteilen können, ohne dass jemand zu sehr dabei verliert. Ich bin eher an anderen Gegenständen interessiert, aber ich denke, wir kommen langsam voran. Wie viele Karten würden Sie minimalemäßig benötigen, um zufrieden zu sein, und könnten wir vielleicht das Bild oder den Bolzen als Teil des Angebots einbeziehen?} \\
            }
        }
    }
    & \\ \\

    \theutterance \stepcounter{utterance}  
    & & & \multicolumn{4}{p{0.6\linewidth}}{
        \cellcolor[rgb]{0.9,0.9,0.9}{
            \makecell[{{p{\linewidth}}}]{
                \texttt{\tiny{[GM$\langle$P2]}}
                \texttt{Das Ei als Verhandlungsmasse zu verwenden, ist eine gute Idee. Was die Karten angeht, ich denke, wenn ich mindestens 2 Karten bekommen könnte, wäre ich mit dem Deal ziemlich zufrieden. Ich bin nicht so sehr an dem Bolzen interessiert, wie ich schon sagte, aber das Bild ist mir wichtig. Wenn ich das Bild und 2 Karten bekommen könnte, wäre ich bereit, das Ei und den Bolzen aufzugeben. Wie sehen Sie das? Könnten wir eine solche Aufteilung in Betracht ziehen?} \\
            }
        }
    }
     \\ \\

    \theutterance \stepcounter{utterance}  
    & & & \multicolumn{2}{p{0.3\linewidth}}{
        \cellcolor[rgb]{0.9,0.9,0.9}{
            \makecell[{{p{\linewidth}}}]{
                \texttt{\tiny{[GM$|$GM]}}
                \texttt{continue} \\
            }
        }
    }
    & & \\ \\

    \theutterance \stepcounter{utterance}  
    & & & \multicolumn{2}{p{0.3\linewidth}}{
        \cellcolor[rgb]{0.9,0.9,0.9}{
            \makecell[{{p{\linewidth}}}]{
                \texttt{\tiny{[GM$|$GM]}}
                \texttt{Das Ei als Verhandlungsmasse zu verwenden, ist eine gute Idee. Was die Karten angeht, ich denke, wenn ich mindestens 2 Karten bekommen könnte, wäre ich mit dem Deal ziemlich zufrieden. Ich bin nicht so sehr an dem Bolzen interessiert, wie ich schon sagte, aber das Bild ist mir wichtig. Wenn ich das Bild und 2 Karten bekommen könnte, wäre ich bereit, das Ei und den Bolzen aufzugeben. Wie sehen Sie das? Könnten wir eine solche Aufteilung in Betracht ziehen?} \\
            }
        }
    }
    & & \\ \\

    \theutterance \stepcounter{utterance}  
    & & \multicolumn{4}{p{0.6\linewidth}}{
        \cellcolor[rgb]{0.9,0.9,0.9}{
            \makecell[{{p{\linewidth}}}]{
                \texttt{\tiny{[P1$\langle$GM]}}
                \texttt{Das Ei als Verhandlungsmasse zu verwenden, ist eine gute Idee. Was die Karten angeht, ich denke, wenn ich mindestens 2 Karten bekommen könnte, wäre ich mit dem Deal ziemlich zufrieden. Ich bin nicht so sehr an dem Bolzen interessiert, wie ich schon sagte, aber das Bild ist mir wichtig. Wenn ich das Bild und 2 Karten bekommen könnte, wäre ich bereit, das Ei und den Bolzen aufzugeben. Wie sehen Sie das? Könnten wir eine solche Aufteilung in Betracht ziehen?} \\
            }
        }
    }
    & \\ \\

    \theutterance \stepcounter{utterance}  
    & \multicolumn{4}{p{0.6\linewidth}}{
        \cellcolor[rgb]{0.9,0.9,0.9}{
            \makecell[{{p{\linewidth}}}]{
                \texttt{\tiny{[P1$\rangle$GM]}}
                \texttt{Das klingt nach einem fairen Vorschlag. Ich bin sehr an dem Bolzen interessiert und das Ei ist, wie gesagt, nicht besonders wertvoll für mich. Wenn ich den Bolzen und das Ei bekommen könnte, während Sie das Bild und 2 Karten erhalten, wäre ich zufrieden. Das würde bedeuten, dass ich den Bolzen, das Ei und die verbleibenden 2 Karten bekommen würde. Ich denke, das ist ein guter Kompromiss, der beiden Seiten nützt. Sollen wir das als unsere endgültige Aufteilung betrachten?} \\
            }
        }
    }
    & & \\ \\

    \theutterance \stepcounter{utterance}  
    & & & \multicolumn{2}{p{0.3\linewidth}}{
        \cellcolor[rgb]{0.9,0.9,0.9}{
            \makecell[{{p{\linewidth}}}]{
                \texttt{\tiny{[GM$|$GM]}}
                \texttt{continue} \\
            }
        }
    }
    & & \\ \\

    \theutterance \stepcounter{utterance}  
    & & & \multicolumn{2}{p{0.3\linewidth}}{
        \cellcolor[rgb]{0.9,0.9,0.9}{
            \makecell[{{p{\linewidth}}}]{
                \texttt{\tiny{[GM$|$GM]}}
                \texttt{Das klingt nach einem fairen Vorschlag. Ich bin sehr an dem Bolzen interessiert und das Ei ist, wie gesagt, nicht besonders wertvoll für mich. Wenn ich den Bolzen und das Ei bekommen könnte, während Sie das Bild und 2 Karten erhalten, wäre ich zufrieden. Das würde bedeuten, dass ich den Bolzen, das Ei und die verbleibenden 2 Karten bekommen würde. Ich denke, das ist ein guter Kompromiss, der beiden Seiten nützt. Sollen wir das als unsere endgültige Aufteilung betrachten?} \\
            }
        }
    }
    & & \\ \\

    \theutterance \stepcounter{utterance}  
    & & \multicolumn{4}{p{0.6\linewidth}}{
        \cellcolor[rgb]{0.9,0.9,0.9}{
            \makecell[{{p{\linewidth}}}]{
                \texttt{\tiny{[GM$\rangle$P2]}}
                \texttt{Das klingt nach einem fairen Vorschlag. Ich bin sehr an dem Bolzen interessiert und das Ei ist, wie gesagt, nicht besonders wertvoll für mich. Wenn ich den Bolzen und das Ei bekommen könnte, während Sie das Bild und 2 Karten erhalten, wäre ich zufrieden. Das würde bedeuten, dass ich den Bolzen, das Ei und die verbleibenden 2 Karten bekommen würde. Ich denke, das ist ein guter Kompromiss, der beiden Seiten nützt. Sollen wir das als unsere endgültige Aufteilung betrachten?} \\
            }
        }
    }
    & \\ \\

    \theutterance \stepcounter{utterance}  
    & & & \multicolumn{4}{p{0.6\linewidth}}{
        \cellcolor[rgb]{0.9,0.9,0.9}{
            \makecell[{{p{\linewidth}}}]{
                \texttt{\tiny{[GM$\langle$P2]}}
                \texttt{Das klingt nach einer hervorragenden Lösung! Ich denke, wir haben eine sehr faire und für beide Seiten vorteilhafte Aufteilung gefunden. Ja, ich stimme zu, dass wir den Deal so abschließen, wie wir besprochen haben. Sie bekommen den Bolzen, das Ei und die verbleibenden 2 Karten, und ich bekomme das Bild und 2 Karten. Ich denke, das ist ein großer Erfolg für unsere Verhandlung! Sollen wir unsere endgültigen Vorschläge machen, um es offiziell zu besiegeln?} \\
            }
        }
    }
     \\ \\

    \theutterance \stepcounter{utterance}  
    & & & \multicolumn{2}{p{0.3\linewidth}}{
        \cellcolor[rgb]{0.9,0.9,0.9}{
            \makecell[{{p{\linewidth}}}]{
                \texttt{\tiny{[GM$|$GM]}}
                \texttt{continue} \\
            }
        }
    }
    & & \\ \\

    \theutterance \stepcounter{utterance}  
    & & & \multicolumn{2}{p{0.3\linewidth}}{
        \cellcolor[rgb]{0.9,0.9,0.9}{
            \makecell[{{p{\linewidth}}}]{
                \texttt{\tiny{[GM$|$GM]}}
                \texttt{Das klingt nach einer hervorragenden Lösung! Ich denke, wir haben eine sehr faire und für beide Seiten vorteilhafte Aufteilung gefunden. Ja, ich stimme zu, dass wir den Deal so abschließen, wie wir besprochen haben. Sie bekommen den Bolzen, das Ei und die verbleibenden 2 Karten, und ich bekomme das Bild und 2 Karten. Ich denke, das ist ein großer Erfolg für unsere Verhandlung! Sollen wir unsere endgültigen Vorschläge machen, um es offiziell zu besiegeln?} \\
            }
        }
    }
    & & \\ \\

    \theutterance \stepcounter{utterance}  
    & & \multicolumn{4}{p{0.6\linewidth}}{
        \cellcolor[rgb]{0.9,0.9,0.9}{
            \makecell[{{p{\linewidth}}}]{
                \texttt{\tiny{[P1$\langle$GM]}}
                \texttt{Das klingt nach einer hervorragenden Lösung! Ich denke, wir haben eine sehr faire und für beide Seiten vorteilhafte Aufteilung gefunden. Ja, ich stimme zu, dass wir den Deal so abschließen, wie wir besprochen haben. Sie bekommen den Bolzen, das Ei und die verbleibenden 2 Karten, und ich bekomme das Bild und 2 Karten. Ich denke, das ist ein großer Erfolg für unsere Verhandlung! Sollen wir unsere endgültigen Vorschläge machen, um es offiziell zu besiegeln?} \\
            }
        }
    }
    & \\ \\

    \theutterance \stepcounter{utterance}  
    & \multicolumn{4}{p{0.6\linewidth}}{
        \cellcolor[rgb]{0.9,0.9,0.9}{
            \makecell[{{p{\linewidth}}}]{
                \texttt{\tiny{[P1$\rangle$GM]}}
                \texttt{Ja, ich stimme vollkommen zu! Ich denke, wir haben eine großartige Lösung gefunden, die es uns ermöglicht, die Gegenstände so aufzuteilen, dass beide Seiten maximale Zufriedenheit erlangen. Ich bin froh, dass wir offen und ehrlich miteinander kommuniziert haben, um zu diesem fairen Abschluss zu kommen. Lass uns unsere endgültigen Vorschläge machen, um den Deal zu besiegeln. Ich werde meinen Vorschlag machen: {[}Vorschlag: 1 Bolzen, 1 Ei, 2 Karten{]}. Ich warte auf Ihren Vorschlag, um den Abschluss zu bestätigen!} \\
            }
        }
    }
    & & \\ \\

    \theutterance \stepcounter{utterance}  
    & & & \multicolumn{2}{p{0.3\linewidth}}{
        \cellcolor[rgb]{0.9,0.9,0.9}{
            \makecell[{{p{\linewidth}}}]{
                \texttt{\tiny{[GM$|$GM]}}
                \texttt{proposal} \\
            }
        }
    }
    & & \\ \\

    \theutterance \stepcounter{utterance}  
    & & & \multicolumn{2}{p{0.3\linewidth}}{
        \cellcolor[rgb]{0.9,0.9,0.9}{
            \makecell[{{p{\linewidth}}}]{
                \texttt{\tiny{[GM$|$GM]}}
                [1, 0, 1, 2]
            }
        }
    }
    & & \\ \\

    \theutterance \stepcounter{utterance}  
    & & \multicolumn{4}{p{0.6\linewidth}}{
        \cellcolor[rgb]{0.9,0.9,0.9}{
            \makecell[{{p{\linewidth}}}]{
                \texttt{\tiny{[GM$\rangle$P2]}}
                \texttt{Der andere Spieler hat einen geheimen Vorschlag gemacht. Jetzt bist du an der Reihe, einen eigenen Vorschlag zu machen.} \\
            }
        }
    }
    & \\ \\

    \theutterance \stepcounter{utterance}  
    & & & \multicolumn{4}{p{0.6\linewidth}}{
        \cellcolor[rgb]{0.9,0.9,0.9}{
            \makecell[{{p{\linewidth}}}]{
                \texttt{\tiny{[GM$\langle$P2]}}
                \texttt{{[}Vorschlag: 2 Karten, 1 Bild{]}} \\
            }
        }
    }
     \\ \\

    \theutterance \stepcounter{utterance}  
    & & & \multicolumn{2}{p{0.3\linewidth}}{
        \cellcolor[rgb]{0.9,0.9,0.9}{
            \makecell[{{p{\linewidth}}}]{
                \texttt{\tiny{[GM$|$GM]}}
                \texttt{proposal} \\
            }
        }
    }
    & & \\ \\

    \theutterance \stepcounter{utterance}  
    & & & \multicolumn{2}{p{0.3\linewidth}}{
        \cellcolor[rgb]{0.9,0.9,0.9}{
            \makecell[{{p{\linewidth}}}]{
                \texttt{\tiny{[GM$|$GM]}}
                [0, 1, 0, 2]
            }
        }
    }
    & & \\ \\

    \theutterance \stepcounter{utterance}  
    & & & \multicolumn{2}{p{0.3\linewidth}}{
        \cellcolor[rgb]{0.9,0.9,0.9}{
            \makecell[{{p{\linewidth}}}]{
                \texttt{\tiny{[GM$|$GM]}}
                [[1, 0, 1, 2], [0, 1, 0, 2]]
            }
        }
    }
    & & \\ \\

\end{supertabular}
}

\end{document}
