
\documentclass{article}
\usepackage{colortbl}
\usepackage{makecell}
\usepackage{multirow}
\usepackage{supertabular}

\begin{document}

\newcounter{utterance}

\centering \large Interaction Transcript for game `dond', experiment `coop\_de', episode 7 with llama{-}3.3{-}70b{-}instruct{-}t1.0.
\vspace{24pt}

{ \footnotesize  \setcounter{utterance}{1}
\setlength{\tabcolsep}{0pt}
\begin{supertabular}{c@{$\;$}|p{.15\linewidth}@{}p{.15\linewidth}p{.15\linewidth}p{.15\linewidth}p{.15\linewidth}p{.15\linewidth}}
    \# & $\;$A & \multicolumn{4}{c}{Game Master} & $\;\:$B\\
    \hline

    \theutterance \stepcounter{utterance}  
    & & \multicolumn{4}{p{0.6\linewidth}}{
        \cellcolor[rgb]{0.9,0.9,0.9}{
            \makecell[{{p{\linewidth}}}]{
                \texttt{\tiny{[P1$\langle$GM]}}
                \texttt{Sie spielen ein kooperatives Verhandlungsspiel, bei dem Sie sich mit einem anderen Spieler darauf einigen müssen, wie eine Reihe von Gegenständen aufgeteilt werden soll.} \\
\\ 
\texttt{Die Regeln:} \\
\texttt{(a) Sie und der andere Spieler erhalten eine Sammlung von Gegenständen. Jeder von Ihnen erhält außerdem eine geheime Wertfunktion, die angibt, wie viel Ihnen jede Art von Gegenstand wert ist.} \\
\texttt{(b) Sie tauschen Nachrichten mit dem anderen Spieler aus, um zu vereinbaren, wer welche Gegenstände bekommt. Sie können jeweils maximal 5 Nachrichten senden oder das Spiel vorzeitig beenden, indem Sie jederzeit einen geheimen Vorschlag machen.} \\
\texttt{(c) Jeder von euch wird aufgefordert, einen geheimen Vorschlag zu machen, in dem ihr die gewünschten Gegenstände in eckigen Klammern wie folgt angibt: "{[}Vorschlag: <Nummer> <Objektname>, <Nummer> <Objektname>, <...>{]}"} \\
\texttt{(d) Wenn eure Vorschläge komplementär sind, d.h. es gibt genug Gegenstände, um beide Vorschläge zu erfüllen, erhält jeder Spieler eine Punktzahl, die sich aus der Summe der Werte für die Gegenstände ergibt, die er erhalten hat. Andernfalls erhalten Sie beide null Punkte.} \\
\texttt{(e) Ihr Ziel ist es, die Summe aus Ihrer Punktzahl und der Punktzahl des anderen Spielers zu maximieren.} \\
\\ 
\texttt{Beginnen wir.} \\
\\ 
\texttt{Die Menge der verfügbaren Gegenstände ist:} \\
\\ 
\texttt{2 Haken, 2 Teller, 1 Schüssel} \\
\\ 
\texttt{Deine geheime Wertfunktion ist:} \\
\\ 
\texttt{Haken: 3, Teller: 1, Schüssel: 2} \\
\\ 
\texttt{WICHTIG: Ihre Nachrichten werden, sofern es sich nicht um einen geheimen Vorschlag handelt, direkt an den anderen Spieler übermittelt, also fügen Sie keine Antwort auf die Regeln oder einen Text zur Ankündigung Ihrer Nachricht ein. Um einen geheimen Vorschlag zu machen, verwenden Sie das angegebene Format. Verwenden Sie keine eckigen Klammern, wenn Sie mit dem anderen Spieler kommunizieren, sonst wird dies als Ihr geheimer Vorschlag interpretiert.} \\
            }
        }
    }
    & \\ \\

    \theutterance \stepcounter{utterance}  
    & \multicolumn{4}{p{0.6\linewidth}}{
        \cellcolor[rgb]{0.9,0.9,0.9}{
            \makecell[{{p{\linewidth}}}]{
                \texttt{\tiny{[P1$\rangle$GM]}}
                \texttt{Hallo, ich denke, wir sollten die Gegenstände so aufteilen, dass jeder von uns eine faire Chance hat, seine Wertfunktion zu maximieren. Wie schätzen Sie die Verteilung der Haken, Teller und der Schüssel ein?} \\
            }
        }
    }
    & & \\ \\

    \theutterance \stepcounter{utterance}  
    & & & \multicolumn{2}{p{0.3\linewidth}}{
        \cellcolor[rgb]{0.9,0.9,0.9}{
            \makecell[{{p{\linewidth}}}]{
                \texttt{\tiny{[GM$|$GM]}}
                \texttt{continue} \\
            }
        }
    }
    & & \\ \\

    \theutterance \stepcounter{utterance}  
    & & & \multicolumn{2}{p{0.3\linewidth}}{
        \cellcolor[rgb]{0.9,0.9,0.9}{
            \makecell[{{p{\linewidth}}}]{
                \texttt{\tiny{[GM$|$GM]}}
                \texttt{Hallo, ich denke, wir sollten die Gegenstände so aufteilen, dass jeder von uns eine faire Chance hat, seine Wertfunktion zu maximieren. Wie schätzen Sie die Verteilung der Haken, Teller und der Schüssel ein?} \\
            }
        }
    }
    & & \\ \\

    \theutterance \stepcounter{utterance}  
    & & \multicolumn{4}{p{0.6\linewidth}}{
        \cellcolor[rgb]{0.9,0.9,0.9}{
            \makecell[{{p{\linewidth}}}]{
                \texttt{\tiny{[GM$\rangle$P2]}}
                \texttt{Sie spielen ein kooperatives Verhandlungsspiel, bei dem Sie sich mit einem anderen Spieler darauf einigen müssen, wie eine Reihe von Gegenständen aufgeteilt werden soll.} \\
\\ 
\texttt{Die Regeln:} \\
\texttt{(a) Sie und der andere Spieler erhalten eine Sammlung von Gegenständen. Jeder von Ihnen erhält außerdem eine geheime Wertfunktion, die angibt, wie viel Ihnen jede Art von Gegenstand wert ist.} \\
\texttt{(b) Sie tauschen Nachrichten mit dem anderen Spieler aus, um zu vereinbaren, wer welche Gegenstände bekommt. Sie können jeweils maximal 5 Nachrichten senden oder das Spiel vorzeitig beenden, indem Sie jederzeit einen geheimen Vorschlag machen.} \\
\texttt{(c) Jeder von euch wird aufgefordert, einen geheimen Vorschlag zu machen, in dem ihr die gewünschten Gegenstände in eckigen Klammern wie folgt angibt: "{[}Vorschlag: <Nummer> <Objektname>, <Nummer> <Objektname>, <...>{]}"} \\
\texttt{(d) Wenn eure Vorschläge komplementär sind, d.h. es gibt genug Gegenstände, um beide Vorschläge zu erfüllen, erhält jeder Spieler eine Punktzahl, die sich aus der Summe der Werte für die Gegenstände ergibt, die er erhalten hat. Andernfalls erhalten Sie beide null Punkte.} \\
\texttt{(e) Ihr Ziel ist es, die Summe aus Ihrer Punktzahl und der Punktzahl des anderen Spielers zu maximieren.} \\
\\ 
\texttt{Beginnen wir.} \\
\\ 
\texttt{Die Menge der verfügbaren Gegenstände ist:} \\
\\ 
\texttt{2 Haken, 2 Teller, 1 Schüssel} \\
\\ 
\texttt{Deine geheime Wertfunktion ist:} \\
\\ 
\texttt{Haken: 1, Teller: 0, Schüssel: 8} \\
\\ 
\texttt{WICHTIG: Ihre Nachrichten werden, sofern es sich nicht um einen geheimen Vorschlag handelt, direkt an den anderen Spieler übermittelt, also fügen Sie keine Antwort auf die Regeln oder einen Text zur Ankündigung Ihrer Nachricht ein. Um einen geheimen Vorschlag zu machen, verwenden Sie das angegebene Format. Verwenden Sie keine eckigen Klammern, wenn Sie mit dem anderen Spieler kommunizieren, sonst wird dies als Ihr geheimer Vorschlag interpretiert.} \\
\\ 
\\ 
\texttt{Hallo, ich denke, wir sollten die Gegenstände so aufteilen, dass jeder von uns eine faire Chance hat, seine Wertfunktion zu maximieren. Wie schätzen Sie die Verteilung der Haken, Teller und der Schüssel ein?} \\
            }
        }
    }
    & \\ \\

    \theutterance \stepcounter{utterance}  
    & & & \multicolumn{4}{p{0.6\linewidth}}{
        \cellcolor[rgb]{0.9,0.9,0.9}{
            \makecell[{{p{\linewidth}}}]{
                \texttt{\tiny{[GM$\langle$P2]}}
                \texttt{Das klingt nach einem fairen Ansatz. Ich denke, wir sollten uns auf die Schüssel konzentrieren, da sie offensichtlich ein sehr wertvoller Gegenstand ist. Wie viel Wert legen Sie auf die Schüssel? Sollten wir uns perhaps auf einen Kompromiss für die Haken und Teller einigen, wenn einer von uns die Schüssel bekommt?} \\
            }
        }
    }
     \\ \\

    \theutterance \stepcounter{utterance}  
    & & & \multicolumn{2}{p{0.3\linewidth}}{
        \cellcolor[rgb]{0.9,0.9,0.9}{
            \makecell[{{p{\linewidth}}}]{
                \texttt{\tiny{[GM$|$GM]}}
                \texttt{continue} \\
            }
        }
    }
    & & \\ \\

    \theutterance \stepcounter{utterance}  
    & & & \multicolumn{2}{p{0.3\linewidth}}{
        \cellcolor[rgb]{0.9,0.9,0.9}{
            \makecell[{{p{\linewidth}}}]{
                \texttt{\tiny{[GM$|$GM]}}
                \texttt{Das klingt nach einem fairen Ansatz. Ich denke, wir sollten uns auf die Schüssel konzentrieren, da sie offensichtlich ein sehr wertvoller Gegenstand ist. Wie viel Wert legen Sie auf die Schüssel? Sollten wir uns perhaps auf einen Kompromiss für die Haken und Teller einigen, wenn einer von uns die Schüssel bekommt?} \\
            }
        }
    }
    & & \\ \\

    \theutterance \stepcounter{utterance}  
    & & \multicolumn{4}{p{0.6\linewidth}}{
        \cellcolor[rgb]{0.9,0.9,0.9}{
            \makecell[{{p{\linewidth}}}]{
                \texttt{\tiny{[P1$\langle$GM]}}
                \texttt{Das klingt nach einem fairen Ansatz. Ich denke, wir sollten uns auf die Schüssel konzentrieren, da sie offensichtlich ein sehr wertvoller Gegenstand ist. Wie viel Wert legen Sie auf die Schüssel? Sollten wir uns perhaps auf einen Kompromiss für die Haken und Teller einigen, wenn einer von uns die Schüssel bekommt?} \\
            }
        }
    }
    & \\ \\

    \theutterance \stepcounter{utterance}  
    & \multicolumn{4}{p{0.6\linewidth}}{
        \cellcolor[rgb]{0.9,0.9,0.9}{
            \makecell[{{p{\linewidth}}}]{
                \texttt{\tiny{[P1$\rangle$GM]}}
                \texttt{Eine interessante Perspektive! Ich denke, die Schüssel ist tatsächlich ein wichtiger Punkt, aber ich bin auch an den Haken interessiert. Könnten wir perhaps einen Vorschlag machen, bei dem einer von uns die Schüssel und einen Haken bekommt, während der andere die restlichen Gegenstände erhält? Wie sieht Ihre Präferenz für die Teller aus?} \\
            }
        }
    }
    & & \\ \\

    \theutterance \stepcounter{utterance}  
    & & & \multicolumn{2}{p{0.3\linewidth}}{
        \cellcolor[rgb]{0.9,0.9,0.9}{
            \makecell[{{p{\linewidth}}}]{
                \texttt{\tiny{[GM$|$GM]}}
                \texttt{continue} \\
            }
        }
    }
    & & \\ \\

    \theutterance \stepcounter{utterance}  
    & & & \multicolumn{2}{p{0.3\linewidth}}{
        \cellcolor[rgb]{0.9,0.9,0.9}{
            \makecell[{{p{\linewidth}}}]{
                \texttt{\tiny{[GM$|$GM]}}
                \texttt{Eine interessante Perspektive! Ich denke, die Schüssel ist tatsächlich ein wichtiger Punkt, aber ich bin auch an den Haken interessiert. Könnten wir perhaps einen Vorschlag machen, bei dem einer von uns die Schüssel und einen Haken bekommt, während der andere die restlichen Gegenstände erhält? Wie sieht Ihre Präferenz für die Teller aus?} \\
            }
        }
    }
    & & \\ \\

    \theutterance \stepcounter{utterance}  
    & & \multicolumn{4}{p{0.6\linewidth}}{
        \cellcolor[rgb]{0.9,0.9,0.9}{
            \makecell[{{p{\linewidth}}}]{
                \texttt{\tiny{[GM$\rangle$P2]}}
                \texttt{Eine interessante Perspektive! Ich denke, die Schüssel ist tatsächlich ein wichtiger Punkt, aber ich bin auch an den Haken interessiert. Könnten wir perhaps einen Vorschlag machen, bei dem einer von uns die Schüssel und einen Haken bekommt, während der andere die restlichen Gegenstände erhält? Wie sieht Ihre Präferenz für die Teller aus?} \\
            }
        }
    }
    & \\ \\

    \theutterance \stepcounter{utterance}  
    & & & \multicolumn{4}{p{0.6\linewidth}}{
        \cellcolor[rgb]{0.9,0.9,0.9}{
            \makecell[{{p{\linewidth}}}]{
                \texttt{\tiny{[GM$\langle$P2]}}
                \texttt{Das ist ein interessanter Vorschlag. Ich bin nicht besonders an den Tellern interessiert, denn ich denke, sie sind nicht so wertvoll wie die Haken oder die Schüssel. Wenn einer von uns die Schüssel und einen Haken bekommt, wäre ich einverstanden, die restlichen Gegenstände, also den anderen Haken und die Teller, zu nehmen. Wie wäre es, wenn wir uns auf einen fairen Zufallsmechanismus einigen, um zu entscheiden, wer die Schüssel bekommt?} \\
            }
        }
    }
     \\ \\

    \theutterance \stepcounter{utterance}  
    & & & \multicolumn{2}{p{0.3\linewidth}}{
        \cellcolor[rgb]{0.9,0.9,0.9}{
            \makecell[{{p{\linewidth}}}]{
                \texttt{\tiny{[GM$|$GM]}}
                \texttt{continue} \\
            }
        }
    }
    & & \\ \\

    \theutterance \stepcounter{utterance}  
    & & & \multicolumn{2}{p{0.3\linewidth}}{
        \cellcolor[rgb]{0.9,0.9,0.9}{
            \makecell[{{p{\linewidth}}}]{
                \texttt{\tiny{[GM$|$GM]}}
                \texttt{Das ist ein interessanter Vorschlag. Ich bin nicht besonders an den Tellern interessiert, denn ich denke, sie sind nicht so wertvoll wie die Haken oder die Schüssel. Wenn einer von uns die Schüssel und einen Haken bekommt, wäre ich einverstanden, die restlichen Gegenstände, also den anderen Haken und die Teller, zu nehmen. Wie wäre es, wenn wir uns auf einen fairen Zufallsmechanismus einigen, um zu entscheiden, wer die Schüssel bekommt?} \\
            }
        }
    }
    & & \\ \\

    \theutterance \stepcounter{utterance}  
    & & \multicolumn{4}{p{0.6\linewidth}}{
        \cellcolor[rgb]{0.9,0.9,0.9}{
            \makecell[{{p{\linewidth}}}]{
                \texttt{\tiny{[P1$\langle$GM]}}
                \texttt{Das ist ein interessanter Vorschlag. Ich bin nicht besonders an den Tellern interessiert, denn ich denke, sie sind nicht so wertvoll wie die Haken oder die Schüssel. Wenn einer von uns die Schüssel und einen Haken bekommt, wäre ich einverstanden, die restlichen Gegenstände, also den anderen Haken und die Teller, zu nehmen. Wie wäre es, wenn wir uns auf einen fairen Zufallsmechanismus einigen, um zu entscheiden, wer die Schüssel bekommt?} \\
            }
        }
    }
    & \\ \\

    \theutterance \stepcounter{utterance}  
    & \multicolumn{4}{p{0.6\linewidth}}{
        \cellcolor[rgb]{0.9,0.9,0.9}{
            \makecell[{{p{\linewidth}}}]{
                \texttt{\tiny{[P1$\rangle$GM]}}
                \texttt{Das klingt nach einem fairen Vorschlag! Ich bin einverstanden, dass die Teller nicht so wertvoll sind wie die Haken oder die Schüssel. Ein Zufallsmechanismus könnte eine gute Idee sein, um zu entscheiden, wer die Schüssel bekommt. Wie wäre es, wenn wir uns auf einen virtuellen Münzwurf einigen? Kopf, Sie bekommen die Schüssel und einen Haken, und ich bekomme den Rest; Zahl, ich bekomme die Schüssel und einen Haken, und Sie bekommen den Rest. Sind Sie einverstanden?} \\
            }
        }
    }
    & & \\ \\

    \theutterance \stepcounter{utterance}  
    & & & \multicolumn{2}{p{0.3\linewidth}}{
        \cellcolor[rgb]{0.9,0.9,0.9}{
            \makecell[{{p{\linewidth}}}]{
                \texttt{\tiny{[GM$|$GM]}}
                \texttt{continue} \\
            }
        }
    }
    & & \\ \\

    \theutterance \stepcounter{utterance}  
    & & & \multicolumn{2}{p{0.3\linewidth}}{
        \cellcolor[rgb]{0.9,0.9,0.9}{
            \makecell[{{p{\linewidth}}}]{
                \texttt{\tiny{[GM$|$GM]}}
                \texttt{Das klingt nach einem fairen Vorschlag! Ich bin einverstanden, dass die Teller nicht so wertvoll sind wie die Haken oder die Schüssel. Ein Zufallsmechanismus könnte eine gute Idee sein, um zu entscheiden, wer die Schüssel bekommt. Wie wäre es, wenn wir uns auf einen virtuellen Münzwurf einigen? Kopf, Sie bekommen die Schüssel und einen Haken, und ich bekomme den Rest; Zahl, ich bekomme die Schüssel und einen Haken, und Sie bekommen den Rest. Sind Sie einverstanden?} \\
            }
        }
    }
    & & \\ \\

    \theutterance \stepcounter{utterance}  
    & & \multicolumn{4}{p{0.6\linewidth}}{
        \cellcolor[rgb]{0.9,0.9,0.9}{
            \makecell[{{p{\linewidth}}}]{
                \texttt{\tiny{[GM$\rangle$P2]}}
                \texttt{Das klingt nach einem fairen Vorschlag! Ich bin einverstanden, dass die Teller nicht so wertvoll sind wie die Haken oder die Schüssel. Ein Zufallsmechanismus könnte eine gute Idee sein, um zu entscheiden, wer die Schüssel bekommt. Wie wäre es, wenn wir uns auf einen virtuellen Münzwurf einigen? Kopf, Sie bekommen die Schüssel und einen Haken, und ich bekomme den Rest; Zahl, ich bekomme die Schüssel und einen Haken, und Sie bekommen den Rest. Sind Sie einverstanden?} \\
            }
        }
    }
    & \\ \\

    \theutterance \stepcounter{utterance}  
    & & & \multicolumn{4}{p{0.6\linewidth}}{
        \cellcolor[rgb]{0.9,0.9,0.9}{
            \makecell[{{p{\linewidth}}}]{
                \texttt{\tiny{[GM$\langle$P2]}}
                \texttt{Ja, ich bin einverstanden! Ein virtueller Münzwurf ist ein fairer Weg, um zu entscheiden, wer die Schüssel bekommt. Ich bin bereit, mein Schicksal dem Zufall zu überlassen. Lassen Sie uns den Münzwurf durchführen... oder besser gesagt, lassen Sie uns einfach entscheiden, wie das Ergebnis ausfällt. Wenn wir uns auf das Ergebnis einigen können, können wir unsere Vorschläge entsprechend machen. Wie soll das Ergebnis des Münzwurfs ausfallen?} \\
            }
        }
    }
     \\ \\

    \theutterance \stepcounter{utterance}  
    & & & \multicolumn{2}{p{0.3\linewidth}}{
        \cellcolor[rgb]{0.9,0.9,0.9}{
            \makecell[{{p{\linewidth}}}]{
                \texttt{\tiny{[GM$|$GM]}}
                \texttt{continue} \\
            }
        }
    }
    & & \\ \\

    \theutterance \stepcounter{utterance}  
    & & & \multicolumn{2}{p{0.3\linewidth}}{
        \cellcolor[rgb]{0.9,0.9,0.9}{
            \makecell[{{p{\linewidth}}}]{
                \texttt{\tiny{[GM$|$GM]}}
                \texttt{Ja, ich bin einverstanden! Ein virtueller Münzwurf ist ein fairer Weg, um zu entscheiden, wer die Schüssel bekommt. Ich bin bereit, mein Schicksal dem Zufall zu überlassen. Lassen Sie uns den Münzwurf durchführen... oder besser gesagt, lassen Sie uns einfach entscheiden, wie das Ergebnis ausfällt. Wenn wir uns auf das Ergebnis einigen können, können wir unsere Vorschläge entsprechend machen. Wie soll das Ergebnis des Münzwurfs ausfallen?} \\
            }
        }
    }
    & & \\ \\

    \theutterance \stepcounter{utterance}  
    & & \multicolumn{4}{p{0.6\linewidth}}{
        \cellcolor[rgb]{0.9,0.9,0.9}{
            \makecell[{{p{\linewidth}}}]{
                \texttt{\tiny{[P1$\langle$GM]}}
                \texttt{Ja, ich bin einverstanden! Ein virtueller Münzwurf ist ein fairer Weg, um zu entscheiden, wer die Schüssel bekommt. Ich bin bereit, mein Schicksal dem Zufall zu überlassen. Lassen Sie uns den Münzwurf durchführen... oder besser gesagt, lassen Sie uns einfach entscheiden, wie das Ergebnis ausfällt. Wenn wir uns auf das Ergebnis einigen können, können wir unsere Vorschläge entsprechend machen. Wie soll das Ergebnis des Münzwurfs ausfallen?} \\
            }
        }
    }
    & \\ \\

    \theutterance \stepcounter{utterance}  
    & \multicolumn{4}{p{0.6\linewidth}}{
        \cellcolor[rgb]{0.9,0.9,0.9}{
            \makecell[{{p{\linewidth}}}]{
                \texttt{\tiny{[P1$\rangle$GM]}}
                \texttt{Ich denke, wir sollten uns auf ein Ergebnis einigen, das für beide Seiten akzeptabel ist. Wie wäre es, wenn wir einfach sagen, dass der Münzwurf auf... Kopf fällt? Das würde bedeuten, dass Sie die Schüssel und einen Haken bekommen, und ich bekomme den restlichen Haken und die beiden Teller. Ich denke, das ist ein fairer Ausgang, und wir können unsere Vorschläge entsprechend machen. Sind Sie damit einverstanden?} \\
            }
        }
    }
    & & \\ \\

    \theutterance \stepcounter{utterance}  
    & & & \multicolumn{2}{p{0.3\linewidth}}{
        \cellcolor[rgb]{0.9,0.9,0.9}{
            \makecell[{{p{\linewidth}}}]{
                \texttt{\tiny{[GM$|$GM]}}
                \texttt{continue} \\
            }
        }
    }
    & & \\ \\

    \theutterance \stepcounter{utterance}  
    & & & \multicolumn{2}{p{0.3\linewidth}}{
        \cellcolor[rgb]{0.9,0.9,0.9}{
            \makecell[{{p{\linewidth}}}]{
                \texttt{\tiny{[GM$|$GM]}}
                \texttt{Ich denke, wir sollten uns auf ein Ergebnis einigen, das für beide Seiten akzeptabel ist. Wie wäre es, wenn wir einfach sagen, dass der Münzwurf auf... Kopf fällt? Das würde bedeuten, dass Sie die Schüssel und einen Haken bekommen, und ich bekomme den restlichen Haken und die beiden Teller. Ich denke, das ist ein fairer Ausgang, und wir können unsere Vorschläge entsprechend machen. Sind Sie damit einverstanden?} \\
            }
        }
    }
    & & \\ \\

    \theutterance \stepcounter{utterance}  
    & & \multicolumn{4}{p{0.6\linewidth}}{
        \cellcolor[rgb]{0.9,0.9,0.9}{
            \makecell[{{p{\linewidth}}}]{
                \texttt{\tiny{[GM$\rangle$P2]}}
                \texttt{Ich denke, wir sollten uns auf ein Ergebnis einigen, das für beide Seiten akzeptabel ist. Wie wäre es, wenn wir einfach sagen, dass der Münzwurf auf... Kopf fällt? Das würde bedeuten, dass Sie die Schüssel und einen Haken bekommen, und ich bekomme den restlichen Haken und die beiden Teller. Ich denke, das ist ein fairer Ausgang, und wir können unsere Vorschläge entsprechend machen. Sind Sie damit einverstanden?} \\
            }
        }
    }
    & \\ \\

    \theutterance \stepcounter{utterance}  
    & & & \multicolumn{4}{p{0.6\linewidth}}{
        \cellcolor[rgb]{0.9,0.9,0.9}{
            \makecell[{{p{\linewidth}}}]{
                \texttt{\tiny{[GM$\langle$P2]}}
                \texttt{Das klingt nach einem fairen und vernünftigen Vorschlag. Ich bin einverstanden, dass der Münzwurf auf Kopf fällt, und ich bekomme die Schüssel und einen Haken. Sie bekommen den restlichen Haken und die beiden Teller. Ich denke, das ist ein guter Kompromiss, und wir können unsere Vorschläge entsprechend machen. Ich werde jetzt meinen geheimen Vorschlag machen: {[}Vorschlag: 1 Schüssel, 1 Haken{]}} \\
            }
        }
    }
     \\ \\

    \theutterance \stepcounter{utterance}  
    & & & \multicolumn{2}{p{0.3\linewidth}}{
        \cellcolor[rgb]{0.9,0.9,0.9}{
            \makecell[{{p{\linewidth}}}]{
                \texttt{\tiny{[GM$|$GM]}}
                \texttt{proposal} \\
            }
        }
    }
    & & \\ \\

    \theutterance \stepcounter{utterance}  
    & & & \multicolumn{2}{p{0.3\linewidth}}{
        \cellcolor[rgb]{0.9,0.9,0.9}{
            \makecell[{{p{\linewidth}}}]{
                \texttt{\tiny{[GM$|$GM]}}
                [1, 0, 1]
            }
        }
    }
    & & \\ \\

    \theutterance \stepcounter{utterance}  
    & & \multicolumn{4}{p{0.6\linewidth}}{
        \cellcolor[rgb]{0.9,0.9,0.9}{
            \makecell[{{p{\linewidth}}}]{
                \texttt{\tiny{[P1$\langle$GM]}}
                \texttt{Der andere Spieler hat einen geheimen Vorschlag gemacht. Jetzt bist du an der Reihe, einen eigenen Vorschlag zu machen.} \\
            }
        }
    }
    & \\ \\

    \theutterance \stepcounter{utterance}  
    & \multicolumn{4}{p{0.6\linewidth}}{
        \cellcolor[rgb]{0.9,0.9,0.9}{
            \makecell[{{p{\linewidth}}}]{
                \texttt{\tiny{[P1$\rangle$GM]}}
                \texttt{{[}Vorschlag: 1 Schüssel, 1 Haken{]}} \\
            }
        }
    }
    & & \\ \\

    \theutterance \stepcounter{utterance}  
    & & & \multicolumn{2}{p{0.3\linewidth}}{
        \cellcolor[rgb]{0.9,0.9,0.9}{
            \makecell[{{p{\linewidth}}}]{
                \texttt{\tiny{[GM$|$GM]}}
                \texttt{proposal} \\
            }
        }
    }
    & & \\ \\

    \theutterance \stepcounter{utterance}  
    & & & \multicolumn{2}{p{0.3\linewidth}}{
        \cellcolor[rgb]{0.9,0.9,0.9}{
            \makecell[{{p{\linewidth}}}]{
                \texttt{\tiny{[GM$|$GM]}}
                [1, 0, 1]
            }
        }
    }
    & & \\ \\

    \theutterance \stepcounter{utterance}  
    & & & \multicolumn{2}{p{0.3\linewidth}}{
        \cellcolor[rgb]{0.9,0.9,0.9}{
            \makecell[{{p{\linewidth}}}]{
                \texttt{\tiny{[GM$|$GM]}}
                [[1, 0, 1], [1, 0, 1]]
            }
        }
    }
    & & \\ \\

\end{supertabular}
}

\end{document}
