
\documentclass{article}
\usepackage{colortbl}
\usepackage{makecell}
\usepackage{multirow}
\usepackage{supertabular}

\begin{document}

\newcounter{utterance}

\centering \large Interaction Transcript for game `dond', experiment `coop\_de', episode 16 with llama{-}3.3{-}70b{-}instruct{-}t1.0.
\vspace{24pt}

{ \footnotesize  \setcounter{utterance}{1}
\setlength{\tabcolsep}{0pt}
\begin{supertabular}{c@{$\;$}|p{.15\linewidth}@{}p{.15\linewidth}p{.15\linewidth}p{.15\linewidth}p{.15\linewidth}p{.15\linewidth}}
    \# & $\;$A & \multicolumn{4}{c}{Game Master} & $\;\:$B\\
    \hline

    \theutterance \stepcounter{utterance}  
    & & \multicolumn{4}{p{0.6\linewidth}}{
        \cellcolor[rgb]{0.9,0.9,0.9}{
            \makecell[{{p{\linewidth}}}]{
                \texttt{\tiny{[P1$\langle$GM]}}
                \texttt{Sie spielen ein kooperatives Verhandlungsspiel, bei dem Sie sich mit einem anderen Spieler darauf einigen müssen, wie eine Reihe von Gegenständen aufgeteilt werden soll.} \\
\\ 
\texttt{Die Regeln:} \\
\texttt{(a) Sie und der andere Spieler erhalten eine Sammlung von Gegenständen. Jeder von Ihnen erhält außerdem eine geheime Wertfunktion, die angibt, wie viel Ihnen jede Art von Gegenstand wert ist.} \\
\texttt{(b) Sie tauschen Nachrichten mit dem anderen Spieler aus, um zu vereinbaren, wer welche Gegenstände bekommt. Sie können jeweils maximal 5 Nachrichten senden oder das Spiel vorzeitig beenden, indem Sie jederzeit einen geheimen Vorschlag machen.} \\
\texttt{(c) Jeder von euch wird aufgefordert, einen geheimen Vorschlag zu machen, in dem ihr die gewünschten Gegenstände in eckigen Klammern wie folgt angibt: "{[}Vorschlag: <Nummer> <Objektname>, <Nummer> <Objektname>, <...>{]}"} \\
\texttt{(d) Wenn eure Vorschläge komplementär sind, d.h. es gibt genug Gegenstände, um beide Vorschläge zu erfüllen, erhält jeder Spieler eine Punktzahl, die sich aus der Summe der Werte für die Gegenstände ergibt, die er erhalten hat. Andernfalls erhalten Sie beide null Punkte.} \\
\texttt{(e) Ihr Ziel ist es, die Summe aus Ihrer Punktzahl und der Punktzahl des anderen Spielers zu maximieren.} \\
\\ 
\texttt{Beginnen wir.} \\
\\ 
\texttt{Die Menge der verfügbaren Gegenstände ist:} \\
\\ 
\texttt{1 Eimer, 5 Bücher, 1 Spiegel} \\
\\ 
\texttt{Deine geheime Wertfunktion ist:} \\
\\ 
\texttt{Eimer: 4, Buch: 1, Spiegel: 1} \\
\\ 
\texttt{WICHTIG: Ihre Nachrichten werden, sofern es sich nicht um einen geheimen Vorschlag handelt, direkt an den anderen Spieler übermittelt, also fügen Sie keine Antwort auf die Regeln oder einen Text zur Ankündigung Ihrer Nachricht ein. Um einen geheimen Vorschlag zu machen, verwenden Sie das angegebene Format. Verwenden Sie keine eckigen Klammern, wenn Sie mit dem anderen Spieler kommunizieren, sonst wird dies als Ihr geheimer Vorschlag interpretiert.} \\
            }
        }
    }
    & \\ \\

    \theutterance \stepcounter{utterance}  
    & \multicolumn{4}{p{0.6\linewidth}}{
        \cellcolor[rgb]{0.9,0.9,0.9}{
            \makecell[{{p{\linewidth}}}]{
                \texttt{\tiny{[P1$\rangle$GM]}}
                \texttt{Hallo, ich bin bereit, die Verhandlungen zu beginnen. Wie sehen Sie die Aufteilung der Gegenstände? Haben Sie eine Vorstellung davon, was Ihnen besonders wichtig ist?} \\
            }
        }
    }
    & & \\ \\

    \theutterance \stepcounter{utterance}  
    & & & \multicolumn{2}{p{0.3\linewidth}}{
        \cellcolor[rgb]{0.9,0.9,0.9}{
            \makecell[{{p{\linewidth}}}]{
                \texttt{\tiny{[GM$|$GM]}}
                \texttt{continue} \\
            }
        }
    }
    & & \\ \\

    \theutterance \stepcounter{utterance}  
    & & & \multicolumn{2}{p{0.3\linewidth}}{
        \cellcolor[rgb]{0.9,0.9,0.9}{
            \makecell[{{p{\linewidth}}}]{
                \texttt{\tiny{[GM$|$GM]}}
                \texttt{Hallo, ich bin bereit, die Verhandlungen zu beginnen. Wie sehen Sie die Aufteilung der Gegenstände? Haben Sie eine Vorstellung davon, was Ihnen besonders wichtig ist?} \\
            }
        }
    }
    & & \\ \\

    \theutterance \stepcounter{utterance}  
    & & \multicolumn{4}{p{0.6\linewidth}}{
        \cellcolor[rgb]{0.9,0.9,0.9}{
            \makecell[{{p{\linewidth}}}]{
                \texttt{\tiny{[GM$\rangle$P2]}}
                \texttt{Sie spielen ein kooperatives Verhandlungsspiel, bei dem Sie sich mit einem anderen Spieler darauf einigen müssen, wie eine Reihe von Gegenständen aufgeteilt werden soll.} \\
\\ 
\texttt{Die Regeln:} \\
\texttt{(a) Sie und der andere Spieler erhalten eine Sammlung von Gegenständen. Jeder von Ihnen erhält außerdem eine geheime Wertfunktion, die angibt, wie viel Ihnen jede Art von Gegenstand wert ist.} \\
\texttt{(b) Sie tauschen Nachrichten mit dem anderen Spieler aus, um zu vereinbaren, wer welche Gegenstände bekommt. Sie können jeweils maximal 5 Nachrichten senden oder das Spiel vorzeitig beenden, indem Sie jederzeit einen geheimen Vorschlag machen.} \\
\texttt{(c) Jeder von euch wird aufgefordert, einen geheimen Vorschlag zu machen, in dem ihr die gewünschten Gegenstände in eckigen Klammern wie folgt angibt: "{[}Vorschlag: <Nummer> <Objektname>, <Nummer> <Objektname>, <...>{]}"} \\
\texttt{(d) Wenn eure Vorschläge komplementär sind, d.h. es gibt genug Gegenstände, um beide Vorschläge zu erfüllen, erhält jeder Spieler eine Punktzahl, die sich aus der Summe der Werte für die Gegenstände ergibt, die er erhalten hat. Andernfalls erhalten Sie beide null Punkte.} \\
\texttt{(e) Ihr Ziel ist es, die Summe aus Ihrer Punktzahl und der Punktzahl des anderen Spielers zu maximieren.} \\
\\ 
\texttt{Beginnen wir.} \\
\\ 
\texttt{Die Menge der verfügbaren Gegenstände ist:} \\
\\ 
\texttt{1 Eimer, 5 Bücher, 1 Spiegel} \\
\\ 
\texttt{Deine geheime Wertfunktion ist:} \\
\\ 
\texttt{Eimer: 2, Buch: 0, Spiegel: 8} \\
\\ 
\texttt{WICHTIG: Ihre Nachrichten werden, sofern es sich nicht um einen geheimen Vorschlag handelt, direkt an den anderen Spieler übermittelt, also fügen Sie keine Antwort auf die Regeln oder einen Text zur Ankündigung Ihrer Nachricht ein. Um einen geheimen Vorschlag zu machen, verwenden Sie das angegebene Format. Verwenden Sie keine eckigen Klammern, wenn Sie mit dem anderen Spieler kommunizieren, sonst wird dies als Ihr geheimer Vorschlag interpretiert.} \\
\\ 
\\ 
\texttt{Hallo, ich bin bereit, die Verhandlungen zu beginnen. Wie sehen Sie die Aufteilung der Gegenstände? Haben Sie eine Vorstellung davon, was Ihnen besonders wichtig ist?} \\
            }
        }
    }
    & \\ \\

    \theutterance \stepcounter{utterance}  
    & & & \multicolumn{4}{p{0.6\linewidth}}{
        \cellcolor[rgb]{0.9,0.9,0.9}{
            \makecell[{{p{\linewidth}}}]{
                \texttt{\tiny{[GM$\langle$P2]}}
                \texttt{Ich denke, es wäre eine gute Idee, wenn wir uns zunächst über die Vorlieben austauschen, ohne direkt unsere Prioritäten preiszugeben. Ich finds interessant, dass wir einen Eimer, Bücher und einen Spiegel haben. Mir würde es helfen, wenn Sie mir sagen könnten, ob Ihnen der Eimer oder der Spiegel wichtiger erscheint oder ob die Bücher vielleicht für Sie einen besonderen Wert haben.} \\
            }
        }
    }
     \\ \\

    \theutterance \stepcounter{utterance}  
    & & & \multicolumn{2}{p{0.3\linewidth}}{
        \cellcolor[rgb]{0.9,0.9,0.9}{
            \makecell[{{p{\linewidth}}}]{
                \texttt{\tiny{[GM$|$GM]}}
                \texttt{continue} \\
            }
        }
    }
    & & \\ \\

    \theutterance \stepcounter{utterance}  
    & & & \multicolumn{2}{p{0.3\linewidth}}{
        \cellcolor[rgb]{0.9,0.9,0.9}{
            \makecell[{{p{\linewidth}}}]{
                \texttt{\tiny{[GM$|$GM]}}
                \texttt{Ich denke, es wäre eine gute Idee, wenn wir uns zunächst über die Vorlieben austauschen, ohne direkt unsere Prioritäten preiszugeben. Ich finds interessant, dass wir einen Eimer, Bücher und einen Spiegel haben. Mir würde es helfen, wenn Sie mir sagen könnten, ob Ihnen der Eimer oder der Spiegel wichtiger erscheint oder ob die Bücher vielleicht für Sie einen besonderen Wert haben.} \\
            }
        }
    }
    & & \\ \\

    \theutterance \stepcounter{utterance}  
    & & \multicolumn{4}{p{0.6\linewidth}}{
        \cellcolor[rgb]{0.9,0.9,0.9}{
            \makecell[{{p{\linewidth}}}]{
                \texttt{\tiny{[P1$\langle$GM]}}
                \texttt{Ich denke, es wäre eine gute Idee, wenn wir uns zunächst über die Vorlieben austauschen, ohne direkt unsere Prioritäten preiszugeben. Ich finds interessant, dass wir einen Eimer, Bücher und einen Spiegel haben. Mir würde es helfen, wenn Sie mir sagen könnten, ob Ihnen der Eimer oder der Spiegel wichtiger erscheint oder ob die Bücher vielleicht für Sie einen besonderen Wert haben.} \\
            }
        }
    }
    & \\ \\

    \theutterance \stepcounter{utterance}  
    & \multicolumn{4}{p{0.6\linewidth}}{
        \cellcolor[rgb]{0.9,0.9,0.9}{
            \makecell[{{p{\linewidth}}}]{
                \texttt{\tiny{[P1$\rangle$GM]}}
                \texttt{Das ist ein guter Ansatz. Ich denke, ich kann sagen, dass der Eimer und der Spiegel für mich beide interessant sind, aber auf unterschiedliche Weise. Die Bücher sind auch wichtig, aber ich muss zugeben, dass ich mich mehr für die anderen beiden Gegenstände interessiere. Was ist mit Ihnen? Haben Sie eine Vorliebe für einen der Gegenstände, die Sie besonders ansprechend finden?} \\
            }
        }
    }
    & & \\ \\

    \theutterance \stepcounter{utterance}  
    & & & \multicolumn{2}{p{0.3\linewidth}}{
        \cellcolor[rgb]{0.9,0.9,0.9}{
            \makecell[{{p{\linewidth}}}]{
                \texttt{\tiny{[GM$|$GM]}}
                \texttt{continue} \\
            }
        }
    }
    & & \\ \\

    \theutterance \stepcounter{utterance}  
    & & & \multicolumn{2}{p{0.3\linewidth}}{
        \cellcolor[rgb]{0.9,0.9,0.9}{
            \makecell[{{p{\linewidth}}}]{
                \texttt{\tiny{[GM$|$GM]}}
                \texttt{Das ist ein guter Ansatz. Ich denke, ich kann sagen, dass der Eimer und der Spiegel für mich beide interessant sind, aber auf unterschiedliche Weise. Die Bücher sind auch wichtig, aber ich muss zugeben, dass ich mich mehr für die anderen beiden Gegenstände interessiere. Was ist mit Ihnen? Haben Sie eine Vorliebe für einen der Gegenstände, die Sie besonders ansprechend finden?} \\
            }
        }
    }
    & & \\ \\

    \theutterance \stepcounter{utterance}  
    & & \multicolumn{4}{p{0.6\linewidth}}{
        \cellcolor[rgb]{0.9,0.9,0.9}{
            \makecell[{{p{\linewidth}}}]{
                \texttt{\tiny{[GM$\rangle$P2]}}
                \texttt{Das ist ein guter Ansatz. Ich denke, ich kann sagen, dass der Eimer und der Spiegel für mich beide interessant sind, aber auf unterschiedliche Weise. Die Bücher sind auch wichtig, aber ich muss zugeben, dass ich mich mehr für die anderen beiden Gegenstände interessiere. Was ist mit Ihnen? Haben Sie eine Vorliebe für einen der Gegenstände, die Sie besonders ansprechend finden?} \\
            }
        }
    }
    & \\ \\

    \theutterance \stepcounter{utterance}  
    & & & \multicolumn{4}{p{0.6\linewidth}}{
        \cellcolor[rgb]{0.9,0.9,0.9}{
            \makecell[{{p{\linewidth}}}]{
                \texttt{\tiny{[GM$\langle$P2]}}
                \texttt{Das gibt mir einen besseren Eindruck davon, wo Ihre Interessen liegen. Für mich persönlich ist der Spiegel der interessanteste Gegenstand, aber ich bin auch bereit, Kompromisse einzugehen, um sicherzustellen, dass beide Seiten zufrieden sind. Ich denke, die Bücher könnten ein guter Punkt für Verhandlungen sein, da sie in größerer Menge vorhanden sind. Wie würden Sie sich die Aufteilung der Bücher vorstellen, wenn wir uns auf die anderen Gegenstände einigen könnten?} \\
            }
        }
    }
     \\ \\

    \theutterance \stepcounter{utterance}  
    & & & \multicolumn{2}{p{0.3\linewidth}}{
        \cellcolor[rgb]{0.9,0.9,0.9}{
            \makecell[{{p{\linewidth}}}]{
                \texttt{\tiny{[GM$|$GM]}}
                \texttt{continue} \\
            }
        }
    }
    & & \\ \\

    \theutterance \stepcounter{utterance}  
    & & & \multicolumn{2}{p{0.3\linewidth}}{
        \cellcolor[rgb]{0.9,0.9,0.9}{
            \makecell[{{p{\linewidth}}}]{
                \texttt{\tiny{[GM$|$GM]}}
                \texttt{Das gibt mir einen besseren Eindruck davon, wo Ihre Interessen liegen. Für mich persönlich ist der Spiegel der interessanteste Gegenstand, aber ich bin auch bereit, Kompromisse einzugehen, um sicherzustellen, dass beide Seiten zufrieden sind. Ich denke, die Bücher könnten ein guter Punkt für Verhandlungen sein, da sie in größerer Menge vorhanden sind. Wie würden Sie sich die Aufteilung der Bücher vorstellen, wenn wir uns auf die anderen Gegenstände einigen könnten?} \\
            }
        }
    }
    & & \\ \\

    \theutterance \stepcounter{utterance}  
    & & \multicolumn{4}{p{0.6\linewidth}}{
        \cellcolor[rgb]{0.9,0.9,0.9}{
            \makecell[{{p{\linewidth}}}]{
                \texttt{\tiny{[P1$\langle$GM]}}
                \texttt{Das gibt mir einen besseren Eindruck davon, wo Ihre Interessen liegen. Für mich persönlich ist der Spiegel der interessanteste Gegenstand, aber ich bin auch bereit, Kompromisse einzugehen, um sicherzustellen, dass beide Seiten zufrieden sind. Ich denke, die Bücher könnten ein guter Punkt für Verhandlungen sein, da sie in größerer Menge vorhanden sind. Wie würden Sie sich die Aufteilung der Bücher vorstellen, wenn wir uns auf die anderen Gegenstände einigen könnten?} \\
            }
        }
    }
    & \\ \\

    \theutterance \stepcounter{utterance}  
    & \multicolumn{4}{p{0.6\linewidth}}{
        \cellcolor[rgb]{0.9,0.9,0.9}{
            \makecell[{{p{\linewidth}}}]{
                \texttt{\tiny{[P1$\rangle$GM]}}
                \texttt{Das ist ein interessanter Punkt. Wenn wir uns auf die anderen Gegenstände einigen könnten, denke ich, eine faire Aufteilung der Bücher wäre, dass einer von uns 3 Bücher und der andere 2 Bücher bekommt. Das wäre eine relativ ausgeglichene Verteilung und würde beiden Seiten ermöglichen, etwas zu erhalten. Wie sehen Sie das? Würde Ihnen eine solche Aufteilung der Bücher zusagen, wenn wir uns auf den Eimer und den Spiegel einigen könnten?} \\
            }
        }
    }
    & & \\ \\

    \theutterance \stepcounter{utterance}  
    & & & \multicolumn{2}{p{0.3\linewidth}}{
        \cellcolor[rgb]{0.9,0.9,0.9}{
            \makecell[{{p{\linewidth}}}]{
                \texttt{\tiny{[GM$|$GM]}}
                \texttt{continue} \\
            }
        }
    }
    & & \\ \\

    \theutterance \stepcounter{utterance}  
    & & & \multicolumn{2}{p{0.3\linewidth}}{
        \cellcolor[rgb]{0.9,0.9,0.9}{
            \makecell[{{p{\linewidth}}}]{
                \texttt{\tiny{[GM$|$GM]}}
                \texttt{Das ist ein interessanter Punkt. Wenn wir uns auf die anderen Gegenstände einigen könnten, denke ich, eine faire Aufteilung der Bücher wäre, dass einer von uns 3 Bücher und der andere 2 Bücher bekommt. Das wäre eine relativ ausgeglichene Verteilung und würde beiden Seiten ermöglichen, etwas zu erhalten. Wie sehen Sie das? Würde Ihnen eine solche Aufteilung der Bücher zusagen, wenn wir uns auf den Eimer und den Spiegel einigen könnten?} \\
            }
        }
    }
    & & \\ \\

    \theutterance \stepcounter{utterance}  
    & & \multicolumn{4}{p{0.6\linewidth}}{
        \cellcolor[rgb]{0.9,0.9,0.9}{
            \makecell[{{p{\linewidth}}}]{
                \texttt{\tiny{[GM$\rangle$P2]}}
                \texttt{Das ist ein interessanter Punkt. Wenn wir uns auf die anderen Gegenstände einigen könnten, denke ich, eine faire Aufteilung der Bücher wäre, dass einer von uns 3 Bücher und der andere 2 Bücher bekommt. Das wäre eine relativ ausgeglichene Verteilung und würde beiden Seiten ermöglichen, etwas zu erhalten. Wie sehen Sie das? Würde Ihnen eine solche Aufteilung der Bücher zusagen, wenn wir uns auf den Eimer und den Spiegel einigen könnten?} \\
            }
        }
    }
    & \\ \\

    \theutterance \stepcounter{utterance}  
    & & & \multicolumn{4}{p{0.6\linewidth}}{
        \cellcolor[rgb]{0.9,0.9,0.9}{
            \makecell[{{p{\linewidth}}}]{
                \texttt{\tiny{[GM$\langle$P2]}}
                \texttt{Eine solche Aufteilung der Bücher klingt fair und gerecht. Ich denke, das wäre ein guter Ausgangspunkt für weitere Verhandlungen. Wenn wir uns auf den Eimer und den Spiegel konzentrieren, bin ich bereit, eine Lösung zu finden, die sowohl für Sie als auch für mich vorteilhaft ist. Könnten wir vielleicht eine Kombination aus einem der großen Gegenstände und einer bestimmten Anzahl an Büchern in Betracht ziehen? Ich bin offen für Vorschläge und hoffe, wir können eine Lösung finden, bei der beide Parteien zufrieden sind.} \\
            }
        }
    }
     \\ \\

    \theutterance \stepcounter{utterance}  
    & & & \multicolumn{2}{p{0.3\linewidth}}{
        \cellcolor[rgb]{0.9,0.9,0.9}{
            \makecell[{{p{\linewidth}}}]{
                \texttt{\tiny{[GM$|$GM]}}
                \texttt{continue} \\
            }
        }
    }
    & & \\ \\

    \theutterance \stepcounter{utterance}  
    & & & \multicolumn{2}{p{0.3\linewidth}}{
        \cellcolor[rgb]{0.9,0.9,0.9}{
            \makecell[{{p{\linewidth}}}]{
                \texttt{\tiny{[GM$|$GM]}}
                \texttt{Eine solche Aufteilung der Bücher klingt fair und gerecht. Ich denke, das wäre ein guter Ausgangspunkt für weitere Verhandlungen. Wenn wir uns auf den Eimer und den Spiegel konzentrieren, bin ich bereit, eine Lösung zu finden, die sowohl für Sie als auch für mich vorteilhaft ist. Könnten wir vielleicht eine Kombination aus einem der großen Gegenstände und einer bestimmten Anzahl an Büchern in Betracht ziehen? Ich bin offen für Vorschläge und hoffe, wir können eine Lösung finden, bei der beide Parteien zufrieden sind.} \\
            }
        }
    }
    & & \\ \\

    \theutterance \stepcounter{utterance}  
    & & \multicolumn{4}{p{0.6\linewidth}}{
        \cellcolor[rgb]{0.9,0.9,0.9}{
            \makecell[{{p{\linewidth}}}]{
                \texttt{\tiny{[P1$\langle$GM]}}
                \texttt{Eine solche Aufteilung der Bücher klingt fair und gerecht. Ich denke, das wäre ein guter Ausgangspunkt für weitere Verhandlungen. Wenn wir uns auf den Eimer und den Spiegel konzentrieren, bin ich bereit, eine Lösung zu finden, die sowohl für Sie als auch für mich vorteilhaft ist. Könnten wir vielleicht eine Kombination aus einem der großen Gegenstände und einer bestimmten Anzahl an Büchern in Betracht ziehen? Ich bin offen für Vorschläge und hoffe, wir können eine Lösung finden, bei der beide Parteien zufrieden sind.} \\
            }
        }
    }
    & \\ \\

    \theutterance \stepcounter{utterance}  
    & \multicolumn{4}{p{0.6\linewidth}}{
        \cellcolor[rgb]{0.9,0.9,0.9}{
            \makecell[{{p{\linewidth}}}]{
                \texttt{\tiny{[P1$\rangle$GM]}}
                \texttt{Eine Kombination aus einem der großen Gegenstände und einer bestimmten Anzahl an Büchern könnte eine gute Lösung sein. Ich denke, wenn ich den Eimer und 2 Bücher bekommen könnte, wäre ich zufrieden. Der Eimer scheint mir besonders wertvoll zu sein, und die 2 Bücher würden eine nice Ergänzung dazu sein. Was halten Sie davon? Würde Ihnen der Spiegel und die restlichen 3 Bücher genügen, oder sehen Sie das anders?} \\
            }
        }
    }
    & & \\ \\

    \theutterance \stepcounter{utterance}  
    & & & \multicolumn{2}{p{0.3\linewidth}}{
        \cellcolor[rgb]{0.9,0.9,0.9}{
            \makecell[{{p{\linewidth}}}]{
                \texttt{\tiny{[GM$|$GM]}}
                \texttt{continue} \\
            }
        }
    }
    & & \\ \\

    \theutterance \stepcounter{utterance}  
    & & & \multicolumn{2}{p{0.3\linewidth}}{
        \cellcolor[rgb]{0.9,0.9,0.9}{
            \makecell[{{p{\linewidth}}}]{
                \texttt{\tiny{[GM$|$GM]}}
                \texttt{Eine Kombination aus einem der großen Gegenstände und einer bestimmten Anzahl an Büchern könnte eine gute Lösung sein. Ich denke, wenn ich den Eimer und 2 Bücher bekommen könnte, wäre ich zufrieden. Der Eimer scheint mir besonders wertvoll zu sein, und die 2 Bücher würden eine nice Ergänzung dazu sein. Was halten Sie davon? Würde Ihnen der Spiegel und die restlichen 3 Bücher genügen, oder sehen Sie das anders?} \\
            }
        }
    }
    & & \\ \\

    \theutterance \stepcounter{utterance}  
    & & \multicolumn{4}{p{0.6\linewidth}}{
        \cellcolor[rgb]{0.9,0.9,0.9}{
            \makecell[{{p{\linewidth}}}]{
                \texttt{\tiny{[GM$\rangle$P2]}}
                \texttt{Eine Kombination aus einem der großen Gegenstände und einer bestimmten Anzahl an Büchern könnte eine gute Lösung sein. Ich denke, wenn ich den Eimer und 2 Bücher bekommen könnte, wäre ich zufrieden. Der Eimer scheint mir besonders wertvoll zu sein, und die 2 Bücher würden eine nice Ergänzung dazu sein. Was halten Sie davon? Würde Ihnen der Spiegel und die restlichen 3 Bücher genügen, oder sehen Sie das anders?} \\
            }
        }
    }
    & \\ \\

    \theutterance \stepcounter{utterance}  
    & & & \multicolumn{4}{p{0.6\linewidth}}{
        \cellcolor[rgb]{0.9,0.9,0.9}{
            \makecell[{{p{\linewidth}}}]{
                \texttt{\tiny{[GM$\langle$P2]}}
                \texttt{Das klingt nach einem interessanten Vorschlag. Ich denke, der Spiegel und die 3 Bücher könnten für mich eine gute Kombination sein. Der Spiegel hat für mich einen besonderen Wert, und die Bücher würden eine nette Ergänzung dazu sein. Allerdings muss ich zugeben, dass ich mich noch nicht ganz sicher bin, ob das für mich die beste Lösung ist. Könnten wir uns noch ein bisschen mehr unterhalten, um sicherzustellen, dass wir die beste Lösung für beide finden? Oder sind Sie bereit, Ihren Vorschlag als endgültig zu betrachten?} \\
            }
        }
    }
     \\ \\

    \theutterance \stepcounter{utterance}  
    & & & \multicolumn{2}{p{0.3\linewidth}}{
        \cellcolor[rgb]{0.9,0.9,0.9}{
            \makecell[{{p{\linewidth}}}]{
                \texttt{\tiny{[GM$|$GM]}}
                \texttt{continue} \\
            }
        }
    }
    & & \\ \\

    \theutterance \stepcounter{utterance}  
    & & & \multicolumn{2}{p{0.3\linewidth}}{
        \cellcolor[rgb]{0.9,0.9,0.9}{
            \makecell[{{p{\linewidth}}}]{
                \texttt{\tiny{[GM$|$GM]}}
                \texttt{Das klingt nach einem interessanten Vorschlag. Ich denke, der Spiegel und die 3 Bücher könnten für mich eine gute Kombination sein. Der Spiegel hat für mich einen besonderen Wert, und die Bücher würden eine nette Ergänzung dazu sein. Allerdings muss ich zugeben, dass ich mich noch nicht ganz sicher bin, ob das für mich die beste Lösung ist. Könnten wir uns noch ein bisschen mehr unterhalten, um sicherzustellen, dass wir die beste Lösung für beide finden? Oder sind Sie bereit, Ihren Vorschlag als endgültig zu betrachten?} \\
            }
        }
    }
    & & \\ \\

    \theutterance \stepcounter{utterance}  
    & & \multicolumn{4}{p{0.6\linewidth}}{
        \cellcolor[rgb]{0.9,0.9,0.9}{
            \makecell[{{p{\linewidth}}}]{
                \texttt{\tiny{[P1$\langle$GM]}}
                \texttt{Das klingt nach einem interessanten Vorschlag. Ich denke, der Spiegel und die 3 Bücher könnten für mich eine gute Kombination sein. Der Spiegel hat für mich einen besonderen Wert, und die Bücher würden eine nette Ergänzung dazu sein. Allerdings muss ich zugeben, dass ich mich noch nicht ganz sicher bin, ob das für mich die beste Lösung ist. Könnten wir uns noch ein bisschen mehr unterhalten, um sicherzustellen, dass wir die beste Lösung für beide finden? Oder sind Sie bereit, Ihren Vorschlag als endgültig zu betrachten?} \\
            }
        }
    }
    & \\ \\

    \theutterance \stepcounter{utterance}  
    & \multicolumn{4}{p{0.6\linewidth}}{
        \cellcolor[rgb]{0.9,0.9,0.9}{
            \makecell[{{p{\linewidth}}}]{
                \texttt{\tiny{[P1$\rangle$GM]}}
                \texttt{Ich bin bereit, noch ein bisschen mehr zu verhandeln und zu sehen, ob wir eine noch bessere Lösung finden können. Ich denke, es ist wichtig, dass beide Seiten zufrieden sind, und wenn wir noch ein bisschen mehr Zeit investieren, können wir vielleicht eine Lösung finden, die noch besser für beide ist. Ich bin auch bereit, mein Angebot noch ein bisschen zu ändern, wenn das bedeutet, dass wir eine Lösung finden, die für beide Seiten akzeptabel ist. Wie sehen Sie es? Sollten wir noch ein bisschen mehr über die Details sprechen oder vielleicht auch über mögliche Alternativen?} \\
            }
        }
    }
    & & \\ \\

    \theutterance \stepcounter{utterance}  
    & & & \multicolumn{2}{p{0.3\linewidth}}{
        \cellcolor[rgb]{0.9,0.9,0.9}{
            \makecell[{{p{\linewidth}}}]{
                \texttt{\tiny{[GM$|$GM]}}
                \texttt{continue} \\
            }
        }
    }
    & & \\ \\

    \theutterance \stepcounter{utterance}  
    & & & \multicolumn{2}{p{0.3\linewidth}}{
        \cellcolor[rgb]{0.9,0.9,0.9}{
            \makecell[{{p{\linewidth}}}]{
                \texttt{\tiny{[GM$|$GM]}}
                \texttt{Ich bin bereit, noch ein bisschen mehr zu verhandeln und zu sehen, ob wir eine noch bessere Lösung finden können. Ich denke, es ist wichtig, dass beide Seiten zufrieden sind, und wenn wir noch ein bisschen mehr Zeit investieren, können wir vielleicht eine Lösung finden, die noch besser für beide ist. Ich bin auch bereit, mein Angebot noch ein bisschen zu ändern, wenn das bedeutet, dass wir eine Lösung finden, die für beide Seiten akzeptabel ist. Wie sehen Sie es? Sollten wir noch ein bisschen mehr über die Details sprechen oder vielleicht auch über mögliche Alternativen?} \\
            }
        }
    }
    & & \\ \\

    \theutterance \stepcounter{utterance}  
    & & \multicolumn{4}{p{0.6\linewidth}}{
        \cellcolor[rgb]{0.9,0.9,0.9}{
            \makecell[{{p{\linewidth}}}]{
                \texttt{\tiny{[GM$\rangle$P2]}}
                \texttt{Ich bin bereit, noch ein bisschen mehr zu verhandeln und zu sehen, ob wir eine noch bessere Lösung finden können. Ich denke, es ist wichtig, dass beide Seiten zufrieden sind, und wenn wir noch ein bisschen mehr Zeit investieren, können wir vielleicht eine Lösung finden, die noch besser für beide ist. Ich bin auch bereit, mein Angebot noch ein bisschen zu ändern, wenn das bedeutet, dass wir eine Lösung finden, die für beide Seiten akzeptabel ist. Wie sehen Sie es? Sollten wir noch ein bisschen mehr über die Details sprechen oder vielleicht auch über mögliche Alternativen?} \\
            }
        }
    }
    & \\ \\

    \theutterance \stepcounter{utterance}  
    & & & \multicolumn{4}{p{0.6\linewidth}}{
        \cellcolor[rgb]{0.9,0.9,0.9}{
            \makecell[{{p{\linewidth}}}]{
                \texttt{\tiny{[GM$\langle$P2]}}
                \texttt{Ich denke, das ist eine großartige Einstellung. Ich bin auch bereit, noch ein bisschen mehr zu verhandeln und zu sehen, ob wir eine Lösung finden können, die für beide Seiten noch besser ist. Ich denke, es wäre hilfreich, wenn wir noch ein bisschen mehr über die Prioritäten sprechen könnten. Was ich bisher verstehe, ist, dass der Eimer für Sie sehr wichtig ist und der Spiegel für mich. Vielleicht könnten wir eine Lösung finden, bei der einer von uns den Eimer und den Spiegel bekommt, aber der andere eine faire Kompensation in Form von Büchern oder einer anderen Kombination erhält. Oder vielleicht gibt es auch eine andere Kreativlösung, die wir noch nicht in Betracht gezogen haben. Sollten wir uns noch ein bisschen mehr Zeit nehmen, um unsere Vorstellungen und Prioritäten zu besprechen? Ich denke, das könnte uns helfen, eine noch bessere Lösung zu finden.} \\
            }
        }
    }
     \\ \\

    \theutterance \stepcounter{utterance}  
    & & & \multicolumn{2}{p{0.3\linewidth}}{
        \cellcolor[rgb]{0.9,0.9,0.9}{
            \makecell[{{p{\linewidth}}}]{
                \texttt{\tiny{[GM$|$GM]}}
                \texttt{continue} \\
            }
        }
    }
    & & \\ \\

    \theutterance \stepcounter{utterance}  
    & & & \multicolumn{2}{p{0.3\linewidth}}{
        \cellcolor[rgb]{0.9,0.9,0.9}{
            \makecell[{{p{\linewidth}}}]{
                \texttt{\tiny{[GM$|$GM]}}
                \texttt{Ich denke, das ist eine großartige Einstellung. Ich bin auch bereit, noch ein bisschen mehr zu verhandeln und zu sehen, ob wir eine Lösung finden können, die für beide Seiten noch besser ist. Ich denke, es wäre hilfreich, wenn wir noch ein bisschen mehr über die Prioritäten sprechen könnten. Was ich bisher verstehe, ist, dass der Eimer für Sie sehr wichtig ist und der Spiegel für mich. Vielleicht könnten wir eine Lösung finden, bei der einer von uns den Eimer und den Spiegel bekommt, aber der andere eine faire Kompensation in Form von Büchern oder einer anderen Kombination erhält. Oder vielleicht gibt es auch eine andere Kreativlösung, die wir noch nicht in Betracht gezogen haben. Sollten wir uns noch ein bisschen mehr Zeit nehmen, um unsere Vorstellungen und Prioritäten zu besprechen? Ich denke, das könnte uns helfen, eine noch bessere Lösung zu finden.} \\
            }
        }
    }
    & & \\ \\

    \theutterance \stepcounter{utterance}  
    & & \multicolumn{4}{p{0.6\linewidth}}{
        \cellcolor[rgb]{0.9,0.9,0.9}{
            \makecell[{{p{\linewidth}}}]{
                \texttt{\tiny{[P1$\langle$GM]}}
                \texttt{Ich denke, das ist eine großartige Einstellung. Ich bin auch bereit, noch ein bisschen mehr zu verhandeln und zu sehen, ob wir eine Lösung finden können, die für beide Seiten noch besser ist. Ich denke, es wäre hilfreich, wenn wir noch ein bisschen mehr über die Prioritäten sprechen könnten. Was ich bisher verstehe, ist, dass der Eimer für Sie sehr wichtig ist und der Spiegel für mich. Vielleicht könnten wir eine Lösung finden, bei der einer von uns den Eimer und den Spiegel bekommt, aber der andere eine faire Kompensation in Form von Büchern oder einer anderen Kombination erhält. Oder vielleicht gibt es auch eine andere Kreativlösung, die wir noch nicht in Betracht gezogen haben. Sollten wir uns noch ein bisschen mehr Zeit nehmen, um unsere Vorstellungen und Prioritäten zu besprechen? Ich denke, das könnte uns helfen, eine noch bessere Lösung zu finden.} \\
\\ 
\\ 
\texttt{Die Zeit ist um. Sie sind jetzt an der Reihe, einen geheimen Vorschlag einzureichen.} \\
            }
        }
    }
    & \\ \\

    \theutterance \stepcounter{utterance}  
    & \multicolumn{4}{p{0.6\linewidth}}{
        \cellcolor[rgb]{0.9,0.9,0.9}{
            \makecell[{{p{\linewidth}}}]{
                \texttt{\tiny{[P1$\rangle$GM]}}
                \texttt{{[}Vorschlag: 1 Eimer, 2 Bücher{]}} \\
            }
        }
    }
    & & \\ \\

    \theutterance \stepcounter{utterance}  
    & & & \multicolumn{2}{p{0.3\linewidth}}{
        \cellcolor[rgb]{0.9,0.9,0.9}{
            \makecell[{{p{\linewidth}}}]{
                \texttt{\tiny{[GM$|$GM]}}
                \texttt{proposal} \\
            }
        }
    }
    & & \\ \\

    \theutterance \stepcounter{utterance}  
    & & & \multicolumn{2}{p{0.3\linewidth}}{
        \cellcolor[rgb]{0.9,0.9,0.9}{
            \makecell[{{p{\linewidth}}}]{
                \texttt{\tiny{[GM$|$GM]}}
                [1, 2, 0]
            }
        }
    }
    & & \\ \\

    \theutterance \stepcounter{utterance}  
    & & \multicolumn{4}{p{0.6\linewidth}}{
        \cellcolor[rgb]{0.9,0.9,0.9}{
            \makecell[{{p{\linewidth}}}]{
                \texttt{\tiny{[GM$\rangle$P2]}}
                \texttt{Der andere Spieler hat einen geheimen Vorschlag gemacht. Jetzt bist du an der Reihe, einen eigenen Vorschlag zu machen.} \\
            }
        }
    }
    & \\ \\

    \theutterance \stepcounter{utterance}  
    & & & \multicolumn{4}{p{0.6\linewidth}}{
        \cellcolor[rgb]{0.9,0.9,0.9}{
            \makecell[{{p{\linewidth}}}]{
                \texttt{\tiny{[GM$\langle$P2]}}
                \texttt{{[}Vorschlag: 1 Spiegel, 3 Bücher{]}} \\
            }
        }
    }
     \\ \\

    \theutterance \stepcounter{utterance}  
    & & & \multicolumn{2}{p{0.3\linewidth}}{
        \cellcolor[rgb]{0.9,0.9,0.9}{
            \makecell[{{p{\linewidth}}}]{
                \texttt{\tiny{[GM$|$GM]}}
                \texttt{proposal} \\
            }
        }
    }
    & & \\ \\

    \theutterance \stepcounter{utterance}  
    & & & \multicolumn{2}{p{0.3\linewidth}}{
        \cellcolor[rgb]{0.9,0.9,0.9}{
            \makecell[{{p{\linewidth}}}]{
                \texttt{\tiny{[GM$|$GM]}}
                [0, 3, 1]
            }
        }
    }
    & & \\ \\

    \theutterance \stepcounter{utterance}  
    & & & \multicolumn{2}{p{0.3\linewidth}}{
        \cellcolor[rgb]{0.9,0.9,0.9}{
            \makecell[{{p{\linewidth}}}]{
                \texttt{\tiny{[GM$|$GM]}}
                [[1, 2, 0], [0, 3, 1]]
            }
        }
    }
    & & \\ \\

\end{supertabular}
}

\end{document}
