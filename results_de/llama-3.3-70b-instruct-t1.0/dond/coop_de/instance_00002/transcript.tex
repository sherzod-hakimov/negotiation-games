
\documentclass{article}
\usepackage{colortbl}
\usepackage{makecell}
\usepackage{multirow}
\usepackage{supertabular}

\begin{document}

\newcounter{utterance}

\centering \large Interaction Transcript for game `dond', experiment `coop\_de', episode 2 with llama{-}3.3{-}70b{-}instruct{-}t1.0.
\vspace{24pt}

{ \footnotesize  \setcounter{utterance}{1}
\setlength{\tabcolsep}{0pt}
\begin{supertabular}{c@{$\;$}|p{.15\linewidth}@{}p{.15\linewidth}p{.15\linewidth}p{.15\linewidth}p{.15\linewidth}p{.15\linewidth}}
    \# & $\;$A & \multicolumn{4}{c}{Game Master} & $\;\:$B\\
    \hline

    \theutterance \stepcounter{utterance}  
    & & \multicolumn{4}{p{0.6\linewidth}}{
        \cellcolor[rgb]{0.9,0.9,0.9}{
            \makecell[{{p{\linewidth}}}]{
                \texttt{\tiny{[P1$\langle$GM]}}
                \texttt{Sie spielen ein kooperatives Verhandlungsspiel, bei dem Sie sich mit einem anderen Spieler darauf einigen müssen, wie eine Reihe von Gegenständen aufgeteilt werden soll.} \\
\\ 
\texttt{Die Regeln:} \\
\texttt{(a) Sie und der andere Spieler erhalten eine Sammlung von Gegenständen. Jeder von Ihnen erhält außerdem eine geheime Wertfunktion, die angibt, wie viel Ihnen jede Art von Gegenstand wert ist.} \\
\texttt{(b) Sie tauschen Nachrichten mit dem anderen Spieler aus, um zu vereinbaren, wer welche Gegenstände bekommt. Sie können jeweils maximal 5 Nachrichten senden oder das Spiel vorzeitig beenden, indem Sie jederzeit einen geheimen Vorschlag machen.} \\
\texttt{(c) Jeder von euch wird aufgefordert, einen geheimen Vorschlag zu machen, in dem ihr die gewünschten Gegenstände in eckigen Klammern wie folgt angibt: "{[}Vorschlag: <Nummer> <Objektname>, <Nummer> <Objektname>, <...>{]}"} \\
\texttt{(d) Wenn eure Vorschläge komplementär sind, d.h. es gibt genug Gegenstände, um beide Vorschläge zu erfüllen, erhält jeder Spieler eine Punktzahl, die sich aus der Summe der Werte für die Gegenstände ergibt, die er erhalten hat. Andernfalls erhalten Sie beide null Punkte.} \\
\texttt{(e) Ihr Ziel ist es, die Summe aus Ihrer Punktzahl und der Punktzahl des anderen Spielers zu maximieren.} \\
\\ 
\texttt{Beginnen wir.} \\
\\ 
\texttt{Die Menge der verfügbaren Gegenstände ist:} \\
\\ 
\texttt{1 Telefon, 2 Spiegel, 2 Früchte} \\
\\ 
\texttt{Deine geheime Wertfunktion ist:} \\
\\ 
\texttt{Telefon: 4, Spiegel: 2, Frucht: 1} \\
\\ 
\texttt{WICHTIG: Ihre Nachrichten werden, sofern es sich nicht um einen geheimen Vorschlag handelt, direkt an den anderen Spieler übermittelt, also fügen Sie keine Antwort auf die Regeln oder einen Text zur Ankündigung Ihrer Nachricht ein. Um einen geheimen Vorschlag zu machen, verwenden Sie das angegebene Format. Verwenden Sie keine eckigen Klammern, wenn Sie mit dem anderen Spieler kommunizieren, sonst wird dies als Ihr geheimer Vorschlag interpretiert.} \\
            }
        }
    }
    & \\ \\

    \theutterance \stepcounter{utterance}  
    & \multicolumn{4}{p{0.6\linewidth}}{
        \cellcolor[rgb]{0.9,0.9,0.9}{
            \makecell[{{p{\linewidth}}}]{
                \texttt{\tiny{[P1$\rangle$GM]}}
                \texttt{Hallo, ich denke, wir sollten zunächst versuchen, unsere Vorlieben für die Gegenstände auszutauschen, um eine faire Aufteilung zu erreichen. Was hältst du vom Telefon?} \\
            }
        }
    }
    & & \\ \\

    \theutterance \stepcounter{utterance}  
    & & & \multicolumn{2}{p{0.3\linewidth}}{
        \cellcolor[rgb]{0.9,0.9,0.9}{
            \makecell[{{p{\linewidth}}}]{
                \texttt{\tiny{[GM$|$GM]}}
                \texttt{continue} \\
            }
        }
    }
    & & \\ \\

    \theutterance \stepcounter{utterance}  
    & & & \multicolumn{2}{p{0.3\linewidth}}{
        \cellcolor[rgb]{0.9,0.9,0.9}{
            \makecell[{{p{\linewidth}}}]{
                \texttt{\tiny{[GM$|$GM]}}
                \texttt{Hallo, ich denke, wir sollten zunächst versuchen, unsere Vorlieben für die Gegenstände auszutauschen, um eine faire Aufteilung zu erreichen. Was hältst du vom Telefon?} \\
            }
        }
    }
    & & \\ \\

    \theutterance \stepcounter{utterance}  
    & & \multicolumn{4}{p{0.6\linewidth}}{
        \cellcolor[rgb]{0.9,0.9,0.9}{
            \makecell[{{p{\linewidth}}}]{
                \texttt{\tiny{[GM$\rangle$P2]}}
                \texttt{Sie spielen ein kooperatives Verhandlungsspiel, bei dem Sie sich mit einem anderen Spieler darauf einigen müssen, wie eine Reihe von Gegenständen aufgeteilt werden soll.} \\
\\ 
\texttt{Die Regeln:} \\
\texttt{(a) Sie und der andere Spieler erhalten eine Sammlung von Gegenständen. Jeder von Ihnen erhält außerdem eine geheime Wertfunktion, die angibt, wie viel Ihnen jede Art von Gegenstand wert ist.} \\
\texttt{(b) Sie tauschen Nachrichten mit dem anderen Spieler aus, um zu vereinbaren, wer welche Gegenstände bekommt. Sie können jeweils maximal 5 Nachrichten senden oder das Spiel vorzeitig beenden, indem Sie jederzeit einen geheimen Vorschlag machen.} \\
\texttt{(c) Jeder von euch wird aufgefordert, einen geheimen Vorschlag zu machen, in dem ihr die gewünschten Gegenstände in eckigen Klammern wie folgt angibt: "{[}Vorschlag: <Nummer> <Objektname>, <Nummer> <Objektname>, <...>{]}"} \\
\texttt{(d) Wenn eure Vorschläge komplementär sind, d.h. es gibt genug Gegenstände, um beide Vorschläge zu erfüllen, erhält jeder Spieler eine Punktzahl, die sich aus der Summe der Werte für die Gegenstände ergibt, die er erhalten hat. Andernfalls erhalten Sie beide null Punkte.} \\
\texttt{(e) Ihr Ziel ist es, die Summe aus Ihrer Punktzahl und der Punktzahl des anderen Spielers zu maximieren.} \\
\\ 
\texttt{Beginnen wir.} \\
\\ 
\texttt{Die Menge der verfügbaren Gegenstände ist:} \\
\\ 
\texttt{1 Telefon, 2 Spiegel, 2 Früchte} \\
\\ 
\texttt{Deine geheime Wertfunktion ist:} \\
\\ 
\texttt{Telefon: 6, Spiegel: 0, Frucht: 2} \\
\\ 
\texttt{WICHTIG: Ihre Nachrichten werden, sofern es sich nicht um einen geheimen Vorschlag handelt, direkt an den anderen Spieler übermittelt, also fügen Sie keine Antwort auf die Regeln oder einen Text zur Ankündigung Ihrer Nachricht ein. Um einen geheimen Vorschlag zu machen, verwenden Sie das angegebene Format. Verwenden Sie keine eckigen Klammern, wenn Sie mit dem anderen Spieler kommunizieren, sonst wird dies als Ihr geheimer Vorschlag interpretiert.} \\
\\ 
\\ 
\texttt{Hallo, ich denke, wir sollten zunächst versuchen, unsere Vorlieben für die Gegenstände auszutauschen, um eine faire Aufteilung zu erreichen. Was hältst du vom Telefon?} \\
            }
        }
    }
    & \\ \\

    \theutterance \stepcounter{utterance}  
    & & & \multicolumn{4}{p{0.6\linewidth}}{
        \cellcolor[rgb]{0.9,0.9,0.9}{
            \makecell[{{p{\linewidth}}}]{
                \texttt{\tiny{[GM$\langle$P2]}}
                \texttt{Ich denke, das Telefon ist ein sehr wertvolles Objekt, aber ich bin nicht sicher, ob es mir mehr wert ist als den anderen Gegenständen. Ich bin eher an den Früchten interessiert, wie steht es mit dir?} \\
            }
        }
    }
     \\ \\

    \theutterance \stepcounter{utterance}  
    & & & \multicolumn{2}{p{0.3\linewidth}}{
        \cellcolor[rgb]{0.9,0.9,0.9}{
            \makecell[{{p{\linewidth}}}]{
                \texttt{\tiny{[GM$|$GM]}}
                \texttt{continue} \\
            }
        }
    }
    & & \\ \\

    \theutterance \stepcounter{utterance}  
    & & & \multicolumn{2}{p{0.3\linewidth}}{
        \cellcolor[rgb]{0.9,0.9,0.9}{
            \makecell[{{p{\linewidth}}}]{
                \texttt{\tiny{[GM$|$GM]}}
                \texttt{Ich denke, das Telefon ist ein sehr wertvolles Objekt, aber ich bin nicht sicher, ob es mir mehr wert ist als den anderen Gegenständen. Ich bin eher an den Früchten interessiert, wie steht es mit dir?} \\
            }
        }
    }
    & & \\ \\

    \theutterance \stepcounter{utterance}  
    & & \multicolumn{4}{p{0.6\linewidth}}{
        \cellcolor[rgb]{0.9,0.9,0.9}{
            \makecell[{{p{\linewidth}}}]{
                \texttt{\tiny{[P1$\langle$GM]}}
                \texttt{Ich denke, das Telefon ist ein sehr wertvolles Objekt, aber ich bin nicht sicher, ob es mir mehr wert ist als den anderen Gegenständen. Ich bin eher an den Früchten interessiert, wie steht es mit dir?} \\
            }
        }
    }
    & \\ \\

    \theutterance \stepcounter{utterance}  
    & \multicolumn{4}{p{0.6\linewidth}}{
        \cellcolor[rgb]{0.9,0.9,0.9}{
            \makecell[{{p{\linewidth}}}]{
                \texttt{\tiny{[P1$\rangle$GM]}}
                \texttt{Früchte sind interessant, aber für mich persönlich sind sie nicht so wertvoll wie das Telefon. Ich könnte mir vorstellen, dass wir das Telefon und einen Teil der anderen Gegenstände aufteilen, um sicherzustellen, dass beide Spieler zufrieden sind. Was meinst du zu den Spiegeln? Sind sie für dich besonders wertvoll?} \\
            }
        }
    }
    & & \\ \\

    \theutterance \stepcounter{utterance}  
    & & & \multicolumn{2}{p{0.3\linewidth}}{
        \cellcolor[rgb]{0.9,0.9,0.9}{
            \makecell[{{p{\linewidth}}}]{
                \texttt{\tiny{[GM$|$GM]}}
                \texttt{continue} \\
            }
        }
    }
    & & \\ \\

    \theutterance \stepcounter{utterance}  
    & & & \multicolumn{2}{p{0.3\linewidth}}{
        \cellcolor[rgb]{0.9,0.9,0.9}{
            \makecell[{{p{\linewidth}}}]{
                \texttt{\tiny{[GM$|$GM]}}
                \texttt{Früchte sind interessant, aber für mich persönlich sind sie nicht so wertvoll wie das Telefon. Ich könnte mir vorstellen, dass wir das Telefon und einen Teil der anderen Gegenstände aufteilen, um sicherzustellen, dass beide Spieler zufrieden sind. Was meinst du zu den Spiegeln? Sind sie für dich besonders wertvoll?} \\
            }
        }
    }
    & & \\ \\

    \theutterance \stepcounter{utterance}  
    & & \multicolumn{4}{p{0.6\linewidth}}{
        \cellcolor[rgb]{0.9,0.9,0.9}{
            \makecell[{{p{\linewidth}}}]{
                \texttt{\tiny{[GM$\rangle$P2]}}
                \texttt{Früchte sind interessant, aber für mich persönlich sind sie nicht so wertvoll wie das Telefon. Ich könnte mir vorstellen, dass wir das Telefon und einen Teil der anderen Gegenstände aufteilen, um sicherzustellen, dass beide Spieler zufrieden sind. Was meinst du zu den Spiegeln? Sind sie für dich besonders wertvoll?} \\
            }
        }
    }
    & \\ \\

    \theutterance \stepcounter{utterance}  
    & & & \multicolumn{4}{p{0.6\linewidth}}{
        \cellcolor[rgb]{0.9,0.9,0.9}{
            \makecell[{{p{\linewidth}}}]{
                \texttt{\tiny{[GM$\langle$P2]}}
                \texttt{Die Spiegel sind für mich nicht besonders wertvoll, ich denke, sie könnten leicht geteilt werden, ohne dass jemand großen Verlust erleidet. Es gibt ja auch zwei Stück, also könnten wir jeweils einen nehmen, wenn wir das Telefon und die Früchte anders aufteilen. Ich frage mich, wie du dir die Aufteilung der Früchte vorstellst, da es ja nur zwei gibt.} \\
            }
        }
    }
     \\ \\

    \theutterance \stepcounter{utterance}  
    & & & \multicolumn{2}{p{0.3\linewidth}}{
        \cellcolor[rgb]{0.9,0.9,0.9}{
            \makecell[{{p{\linewidth}}}]{
                \texttt{\tiny{[GM$|$GM]}}
                \texttt{continue} \\
            }
        }
    }
    & & \\ \\

    \theutterance \stepcounter{utterance}  
    & & & \multicolumn{2}{p{0.3\linewidth}}{
        \cellcolor[rgb]{0.9,0.9,0.9}{
            \makecell[{{p{\linewidth}}}]{
                \texttt{\tiny{[GM$|$GM]}}
                \texttt{Die Spiegel sind für mich nicht besonders wertvoll, ich denke, sie könnten leicht geteilt werden, ohne dass jemand großen Verlust erleidet. Es gibt ja auch zwei Stück, also könnten wir jeweils einen nehmen, wenn wir das Telefon und die Früchte anders aufteilen. Ich frage mich, wie du dir die Aufteilung der Früchte vorstellst, da es ja nur zwei gibt.} \\
            }
        }
    }
    & & \\ \\

    \theutterance \stepcounter{utterance}  
    & & \multicolumn{4}{p{0.6\linewidth}}{
        \cellcolor[rgb]{0.9,0.9,0.9}{
            \makecell[{{p{\linewidth}}}]{
                \texttt{\tiny{[P1$\langle$GM]}}
                \texttt{Die Spiegel sind für mich nicht besonders wertvoll, ich denke, sie könnten leicht geteilt werden, ohne dass jemand großen Verlust erleidet. Es gibt ja auch zwei Stück, also könnten wir jeweils einen nehmen, wenn wir das Telefon und die Früchte anders aufteilen. Ich frage mich, wie du dir die Aufteilung der Früchte vorstellst, da es ja nur zwei gibt.} \\
            }
        }
    }
    & \\ \\

    \theutterance \stepcounter{utterance}  
    & \multicolumn{4}{p{0.6\linewidth}}{
        \cellcolor[rgb]{0.9,0.9,0.9}{
            \makecell[{{p{\linewidth}}}]{
                \texttt{\tiny{[P1$\rangle$GM]}}
                \texttt{Das ist ein guter Punkt mit den Spiegeln, das wäre eine faire Lösung. Bei den Früchten denke ich, dass wir sie vielleicht aufteilen könnten, oder einer von uns nimmt beide, je nachdem, wie wichtig sie dem anderen sind. Ich bin bereit, Kompromisse einzugehen, um eine faire Lösung zu finden. Wie wäre es, wenn einer von uns das Telefon bekommt und der andere beide Früchte? Oder vielleicht können wir das Telefon und eine Frucht tauschen gegen etwas anderes?} \\
            }
        }
    }
    & & \\ \\

    \theutterance \stepcounter{utterance}  
    & & & \multicolumn{2}{p{0.3\linewidth}}{
        \cellcolor[rgb]{0.9,0.9,0.9}{
            \makecell[{{p{\linewidth}}}]{
                \texttt{\tiny{[GM$|$GM]}}
                \texttt{continue} \\
            }
        }
    }
    & & \\ \\

    \theutterance \stepcounter{utterance}  
    & & & \multicolumn{2}{p{0.3\linewidth}}{
        \cellcolor[rgb]{0.9,0.9,0.9}{
            \makecell[{{p{\linewidth}}}]{
                \texttt{\tiny{[GM$|$GM]}}
                \texttt{Das ist ein guter Punkt mit den Spiegeln, das wäre eine faire Lösung. Bei den Früchten denke ich, dass wir sie vielleicht aufteilen könnten, oder einer von uns nimmt beide, je nachdem, wie wichtig sie dem anderen sind. Ich bin bereit, Kompromisse einzugehen, um eine faire Lösung zu finden. Wie wäre es, wenn einer von uns das Telefon bekommt und der andere beide Früchte? Oder vielleicht können wir das Telefon und eine Frucht tauschen gegen etwas anderes?} \\
            }
        }
    }
    & & \\ \\

    \theutterance \stepcounter{utterance}  
    & & \multicolumn{4}{p{0.6\linewidth}}{
        \cellcolor[rgb]{0.9,0.9,0.9}{
            \makecell[{{p{\linewidth}}}]{
                \texttt{\tiny{[GM$\rangle$P2]}}
                \texttt{Das ist ein guter Punkt mit den Spiegeln, das wäre eine faire Lösung. Bei den Früchten denke ich, dass wir sie vielleicht aufteilen könnten, oder einer von uns nimmt beide, je nachdem, wie wichtig sie dem anderen sind. Ich bin bereit, Kompromisse einzugehen, um eine faire Lösung zu finden. Wie wäre es, wenn einer von uns das Telefon bekommt und der andere beide Früchte? Oder vielleicht können wir das Telefon und eine Frucht tauschen gegen etwas anderes?} \\
            }
        }
    }
    & \\ \\

    \theutterance \stepcounter{utterance}  
    & & & \multicolumn{4}{p{0.6\linewidth}}{
        \cellcolor[rgb]{0.9,0.9,0.9}{
            \makecell[{{p{\linewidth}}}]{
                \texttt{\tiny{[GM$\langle$P2]}}
                \texttt{Das klingt nach einem guten Ansatz. Ich denke, es wäre fair, wenn einer von uns das Telefon und der andere beide Früchte bekommt. Das würde bedeuten, dass wir beide ziemlich zufrieden wären, da das Telefon für dich wichtig ist und die Früchte für mich interessant sind. Ich bin bereit, beide Früchte zu nehmen, wenn du das Telefon bekommst. Was meinst du, sollten wir auch die Spiegel aufteilen, wie wir besprochen haben, oder gibt es einen anderen Vorschlag, den du hast?} \\
            }
        }
    }
     \\ \\

    \theutterance \stepcounter{utterance}  
    & & & \multicolumn{2}{p{0.3\linewidth}}{
        \cellcolor[rgb]{0.9,0.9,0.9}{
            \makecell[{{p{\linewidth}}}]{
                \texttt{\tiny{[GM$|$GM]}}
                \texttt{continue} \\
            }
        }
    }
    & & \\ \\

    \theutterance \stepcounter{utterance}  
    & & & \multicolumn{2}{p{0.3\linewidth}}{
        \cellcolor[rgb]{0.9,0.9,0.9}{
            \makecell[{{p{\linewidth}}}]{
                \texttt{\tiny{[GM$|$GM]}}
                \texttt{Das klingt nach einem guten Ansatz. Ich denke, es wäre fair, wenn einer von uns das Telefon und der andere beide Früchte bekommt. Das würde bedeuten, dass wir beide ziemlich zufrieden wären, da das Telefon für dich wichtig ist und die Früchte für mich interessant sind. Ich bin bereit, beide Früchte zu nehmen, wenn du das Telefon bekommst. Was meinst du, sollten wir auch die Spiegel aufteilen, wie wir besprochen haben, oder gibt es einen anderen Vorschlag, den du hast?} \\
            }
        }
    }
    & & \\ \\

    \theutterance \stepcounter{utterance}  
    & & \multicolumn{4}{p{0.6\linewidth}}{
        \cellcolor[rgb]{0.9,0.9,0.9}{
            \makecell[{{p{\linewidth}}}]{
                \texttt{\tiny{[P1$\langle$GM]}}
                \texttt{Das klingt nach einem guten Ansatz. Ich denke, es wäre fair, wenn einer von uns das Telefon und der andere beide Früchte bekommt. Das würde bedeuten, dass wir beide ziemlich zufrieden wären, da das Telefon für dich wichtig ist und die Früchte für mich interessant sind. Ich bin bereit, beide Früchte zu nehmen, wenn du das Telefon bekommst. Was meinst du, sollten wir auch die Spiegel aufteilen, wie wir besprochen haben, oder gibt es einen anderen Vorschlag, den du hast?} \\
            }
        }
    }
    & \\ \\

    \theutterance \stepcounter{utterance}  
    & \multicolumn{4}{p{0.6\linewidth}}{
        \cellcolor[rgb]{0.9,0.9,0.9}{
            \makecell[{{p{\linewidth}}}]{
                \texttt{\tiny{[P1$\rangle$GM]}}
                \texttt{Ich denke, das ist ein sehr fairer Vorschlag. Wenn du beide Früchte nimmst und ich das Telefon bekomme, dann sollten wir die Spiegel tatsächlich aufteilen, wie wir besprochen haben. Jeder von uns könnte einen Spiegel bekommen, das wäre eine gute Lösung. Ich denke, wir haben eine gute Grundlage für einen Vorschlag. Sollen wir jetzt unsere Vorschläge machen, um sicherzustellen, dass wir uns einig sind?} \\
            }
        }
    }
    & & \\ \\

    \theutterance \stepcounter{utterance}  
    & & & \multicolumn{2}{p{0.3\linewidth}}{
        \cellcolor[rgb]{0.9,0.9,0.9}{
            \makecell[{{p{\linewidth}}}]{
                \texttt{\tiny{[GM$|$GM]}}
                \texttt{continue} \\
            }
        }
    }
    & & \\ \\

    \theutterance \stepcounter{utterance}  
    & & & \multicolumn{2}{p{0.3\linewidth}}{
        \cellcolor[rgb]{0.9,0.9,0.9}{
            \makecell[{{p{\linewidth}}}]{
                \texttt{\tiny{[GM$|$GM]}}
                \texttt{Ich denke, das ist ein sehr fairer Vorschlag. Wenn du beide Früchte nimmst und ich das Telefon bekomme, dann sollten wir die Spiegel tatsächlich aufteilen, wie wir besprochen haben. Jeder von uns könnte einen Spiegel bekommen, das wäre eine gute Lösung. Ich denke, wir haben eine gute Grundlage für einen Vorschlag. Sollen wir jetzt unsere Vorschläge machen, um sicherzustellen, dass wir uns einig sind?} \\
            }
        }
    }
    & & \\ \\

    \theutterance \stepcounter{utterance}  
    & & \multicolumn{4}{p{0.6\linewidth}}{
        \cellcolor[rgb]{0.9,0.9,0.9}{
            \makecell[{{p{\linewidth}}}]{
                \texttt{\tiny{[GM$\rangle$P2]}}
                \texttt{Ich denke, das ist ein sehr fairer Vorschlag. Wenn du beide Früchte nimmst und ich das Telefon bekomme, dann sollten wir die Spiegel tatsächlich aufteilen, wie wir besprochen haben. Jeder von uns könnte einen Spiegel bekommen, das wäre eine gute Lösung. Ich denke, wir haben eine gute Grundlage für einen Vorschlag. Sollen wir jetzt unsere Vorschläge machen, um sicherzustellen, dass wir uns einig sind?} \\
            }
        }
    }
    & \\ \\

    \theutterance \stepcounter{utterance}  
    & & & \multicolumn{4}{p{0.6\linewidth}}{
        \cellcolor[rgb]{0.9,0.9,0.9}{
            \makecell[{{p{\linewidth}}}]{
                \texttt{\tiny{[GM$\langle$P2]}}
                \texttt{Ja, das klingt gut. Ich denke, wir haben eine faire Lösung gefunden. Ich bin bereit, meinen Vorschlag zu machen. Ich werde {[}Vorschlag: 2 Frucht, 1 Spiegel{]} schreiben, um sicherzustellen, dass ich beide Früchte und einen Spiegel bekomme. Ich hoffe, du machst einen entsprechenden Vorschlag, um das Telefon und den anderen Spiegel zu bekommen. Lass uns sehen, ob unsere Vorschläge übereinstimmen!} \\
            }
        }
    }
     \\ \\

    \theutterance \stepcounter{utterance}  
    & & & \multicolumn{2}{p{0.3\linewidth}}{
        \cellcolor[rgb]{0.9,0.9,0.9}{
            \makecell[{{p{\linewidth}}}]{
                \texttt{\tiny{[GM$|$GM]}}
                \texttt{proposal} \\
            }
        }
    }
    & & \\ \\

    \theutterance \stepcounter{utterance}  
    & & & \multicolumn{2}{p{0.3\linewidth}}{
        \cellcolor[rgb]{0.9,0.9,0.9}{
            \makecell[{{p{\linewidth}}}]{
                \texttt{\tiny{[GM$|$GM]}}
                [0, 1, 2]
            }
        }
    }
    & & \\ \\

    \theutterance \stepcounter{utterance}  
    & & \multicolumn{4}{p{0.6\linewidth}}{
        \cellcolor[rgb]{0.9,0.9,0.9}{
            \makecell[{{p{\linewidth}}}]{
                \texttt{\tiny{[P1$\langle$GM]}}
                \texttt{Der andere Spieler hat einen geheimen Vorschlag gemacht. Jetzt bist du an der Reihe, einen eigenen Vorschlag zu machen.} \\
            }
        }
    }
    & \\ \\

    \theutterance \stepcounter{utterance}  
    & \multicolumn{4}{p{0.6\linewidth}}{
        \cellcolor[rgb]{0.9,0.9,0.9}{
            \makecell[{{p{\linewidth}}}]{
                \texttt{\tiny{[P1$\rangle$GM]}}
                \texttt{{[}Vorschlag: 1 Telefon, 1 Spiegel{]}} \\
            }
        }
    }
    & & \\ \\

    \theutterance \stepcounter{utterance}  
    & & & \multicolumn{2}{p{0.3\linewidth}}{
        \cellcolor[rgb]{0.9,0.9,0.9}{
            \makecell[{{p{\linewidth}}}]{
                \texttt{\tiny{[GM$|$GM]}}
                \texttt{proposal} \\
            }
        }
    }
    & & \\ \\

    \theutterance \stepcounter{utterance}  
    & & & \multicolumn{2}{p{0.3\linewidth}}{
        \cellcolor[rgb]{0.9,0.9,0.9}{
            \makecell[{{p{\linewidth}}}]{
                \texttt{\tiny{[GM$|$GM]}}
                [1, 1, 0]
            }
        }
    }
    & & \\ \\

    \theutterance \stepcounter{utterance}  
    & & & \multicolumn{2}{p{0.3\linewidth}}{
        \cellcolor[rgb]{0.9,0.9,0.9}{
            \makecell[{{p{\linewidth}}}]{
                \texttt{\tiny{[GM$|$GM]}}
                [[1, 1, 0], [0, 1, 2]]
            }
        }
    }
    & & \\ \\

\end{supertabular}
}

\end{document}
