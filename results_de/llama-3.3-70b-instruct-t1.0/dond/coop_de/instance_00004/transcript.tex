
\documentclass{article}
\usepackage{colortbl}
\usepackage{makecell}
\usepackage{multirow}
\usepackage{supertabular}

\begin{document}

\newcounter{utterance}

\twocolumn

{ \footnotesize  \setcounter{utterance}{1}
\setlength{\tabcolsep}{0pt}
\begin{supertabular}{c@{$\;$}|p{.15\linewidth}@{}p{.15\linewidth}p{.15\linewidth}p{.15\linewidth}p{.15\linewidth}p{.15\linewidth}}

    \# & $\;$A & \multicolumn{4}{c}{Game Master} & $\;\:$B\\
    \hline 

    \theutterance \stepcounter{utterance}  

    & & \multicolumn{4}{p{0.6\linewidth}}{\cellcolor[rgb]{0.9,0.9,0.9}{%
	\makecell[{{p{\linewidth}}}]{% 
	  \tt {\tiny [A$\langle$GM]}  
	 Sie spielen ein kooperatives Verhandlungsspiel, bei dem Sie sich mit einem anderen Spieler darauf einigen müssen, wie eine Reihe von Gegenständen aufgeteilt werden soll.\\ \tt \\ \tt Die Regeln:\\ \tt (a) Sie und der andere Spieler erhalten eine Sammlung von Gegenständen. Jeder von Ihnen erhält außerdem eine geheime Wertfunktion, die angibt, wie viel Ihnen jede Art von Gegenstand wert ist.\\ \tt (b) Sie tauschen Nachrichten mit dem anderen Spieler aus, um zu vereinbaren, wer welche Gegenstände bekommt. Sie können jeweils maximal 5 Nachrichten senden oder das Spiel vorzeitig beenden, indem Sie jederzeit einen geheimen Vorschlag machen.\\ \tt (c) Jeder von euch wird aufgefordert, einen geheimen Vorschlag zu machen, in dem ihr die gewünschten Gegenstände in eckigen Klammern wie folgt angibt: "[Vorschlag: <Nummer> <Objektname>, <Nummer> <Objektname>, <...>]"\\ \tt (d) Wenn eure Vorschläge komplementär sind, d.h. es gibt genug Gegenstände, um beide Vorschläge zu erfüllen, erhält jeder Spieler eine Punktzahl, die sich aus der Summe der Werte für die Gegenstände ergibt, die er erhalten hat. Andernfalls erhalten Sie beide null Punkte.\\ \tt (e) Ihr Ziel ist es, die Summe aus Ihrer Punktzahl und der Punktzahl des anderen Spielers zu maximieren.\\ \tt \\ \tt Beginnen wir.\\ \tt \\ \tt Die Menge der verfügbaren Gegenstände ist:\\ \tt \\ \tt 1 Fingerhut, 1 Seife, 4 Briefmarken\\ \tt \\ \tt Deine geheime Wertfunktion ist:\\ \tt \\ \tt Fingerhut: 3, Seife: 7, Briefmarke: 0\\ \tt \\ \tt WICHTIG: Ihre Nachrichten werden, sofern es sich nicht um einen geheimen Vorschlag handelt, direkt an den anderen Spieler übermittelt, also fügen Sie keine Antwort auf die Regeln oder einen Text zur Ankündigung Ihrer Nachricht ein. Um einen geheimen Vorschlag zu machen, verwenden Sie das angegebene Format. Verwenden Sie keine eckigen Klammern, wenn Sie mit dem anderen Spieler kommunizieren, sonst wird dies als Ihr geheimer Vorschlag interpretiert.\\ \tt  
	  } 
	   } 
	   } 
	 & \\ 
 

    \theutterance \stepcounter{utterance}  

    & \multicolumn{4}{p{0.6\linewidth}}{\cellcolor[rgb]{0.8,1,0.9}{%
	\makecell[{{p{\linewidth}}}]{% 
	  \tt {\tiny [A$\rangle$GM]}  
	 Hallo, ich denke, wir sollten uns um die Verteilung der Gegenstände kümmern. Was denkst du über den Wert der einzelnen Gegenstände? Sollten wir versuchen, sie fair aufzuteilen oder nach einem bestimmten Kriterium zu verteilen? Lass uns diskutieren! 
	  } 
	   } 
	   } 
	 & & \\ 
 

    \theutterance \stepcounter{utterance}  

    & & & \multicolumn{2}{p{0.3\linewidth}}{\cellcolor[rgb]{0.95,0.95,0.95}{%
	\makecell[{{p{\linewidth}}}]{% 
	  \tt {\tiny [GM$|$GM]}  
	 continue 
	  } 
	   } 
	   } 
	 & & \\ 
 

    \theutterance \stepcounter{utterance}  

    & & & \multicolumn{2}{p{0.3\linewidth}}{\cellcolor[rgb]{0.95,0.95,0.95}{%
	\makecell[{{p{\linewidth}}}]{% 
	  \tt {\tiny [GM$|$GM]}  
	 Hallo, ich denke, wir sollten uns um die Verteilung der Gegenstände kümmern. Was denkst du über den Wert der einzelnen Gegenstände? Sollten wir versuchen, sie fair aufzuteilen oder nach einem bestimmten Kriterium zu verteilen? Lass uns diskutieren! 
	  } 
	   } 
	   } 
	 & & \\ 
 

    \theutterance \stepcounter{utterance}  

    & & \multicolumn{4}{p{0.6\linewidth}}{\cellcolor[rgb]{0.9,0.9,0.9}{%
	\makecell[{{p{\linewidth}}}]{% 
	  \tt {\tiny [GM$\rangle$B]}  
	 Sie spielen ein kooperatives Verhandlungsspiel, bei dem Sie sich mit einem anderen Spieler darauf einigen müssen, wie eine Reihe von Gegenständen aufgeteilt werden soll.\\ \tt \\ \tt Die Regeln:\\ \tt (a) Sie und der andere Spieler erhalten eine Sammlung von Gegenständen. Jeder von Ihnen erhält außerdem eine geheime Wertfunktion, die angibt, wie viel Ihnen jede Art von Gegenstand wert ist.\\ \tt (b) Sie tauschen Nachrichten mit dem anderen Spieler aus, um zu vereinbaren, wer welche Gegenstände bekommt. Sie können jeweils maximal 5 Nachrichten senden oder das Spiel vorzeitig beenden, indem Sie jederzeit einen geheimen Vorschlag machen.\\ \tt (c) Jeder von euch wird aufgefordert, einen geheimen Vorschlag zu machen, in dem ihr die gewünschten Gegenstände in eckigen Klammern wie folgt angibt: "[Vorschlag: <Nummer> <Objektname>, <Nummer> <Objektname>, <...>]"\\ \tt (d) Wenn eure Vorschläge komplementär sind, d.h. es gibt genug Gegenstände, um beide Vorschläge zu erfüllen, erhält jeder Spieler eine Punktzahl, die sich aus der Summe der Werte für die Gegenstände ergibt, die er erhalten hat. Andernfalls erhalten Sie beide null Punkte.\\ \tt (e) Ihr Ziel ist es, die Summe aus Ihrer Punktzahl und der Punktzahl des anderen Spielers zu maximieren.\\ \tt \\ \tt Beginnen wir.\\ \tt \\ \tt Die Menge der verfügbaren Gegenstände ist:\\ \tt \\ \tt 1 Fingerhut, 1 Seife, 4 Briefmarken\\ \tt \\ \tt Deine geheime Wertfunktion ist:\\ \tt \\ \tt Fingerhut: 2, Seife: 0, Briefmarke: 2\\ \tt \\ \tt WICHTIG: Ihre Nachrichten werden, sofern es sich nicht um einen geheimen Vorschlag handelt, direkt an den anderen Spieler übermittelt, also fügen Sie keine Antwort auf die Regeln oder einen Text zur Ankündigung Ihrer Nachricht ein. Um einen geheimen Vorschlag zu machen, verwenden Sie das angegebene Format. Verwenden Sie keine eckigen Klammern, wenn Sie mit dem anderen Spieler kommunizieren, sonst wird dies als Ihr geheimer Vorschlag interpretiert.\\ \tt \\ \tt \\ \tt Hallo, ich denke, wir sollten uns um die Verteilung der Gegenstände kümmern. Was denkst du über den Wert der einzelnen Gegenstände? Sollten wir versuchen, sie fair aufzuteilen oder nach einem bestimmten Kriterium zu verteilen? Lass uns diskutieren! 
	  } 
	   } 
	   } 
	 & \\ 
 

    \theutterance \stepcounter{utterance}  

    & & & \multicolumn{4}{p{0.6\linewidth}}{\cellcolor[rgb]{1,0.85,0.72}{%
	\makecell[{{p{\linewidth}}}]{% 
	  \tt {\tiny [GM$\langle$B]}  
	 Ich bin auch der Meinung, dass es sinnvoll ist, die Verteilung zu besprechen. Ich denke, dass ein faires Verhältnis und die persönlichen Vorlieben beide eine Rolle spielen sollten. Ich persönlich würde mich gerne auf die Briefmarken konzentrieren, da ich denke, dass sie einen gewissen Wert haben könnten. Wie wichtig sind dir die einzelnen Gegenstände? Gibt es einen, an dem du besonders interessiert bist? 
	  } 
	   } 
	   } 
	  \\ 
 

    \theutterance \stepcounter{utterance}  

    & & & \multicolumn{2}{p{0.3\linewidth}}{\cellcolor[rgb]{0.95,0.95,0.95}{%
	\makecell[{{p{\linewidth}}}]{% 
	  \tt {\tiny [GM$|$GM]}  
	 continue 
	  } 
	   } 
	   } 
	 & & \\ 
 

    \theutterance \stepcounter{utterance}  

    & & & \multicolumn{2}{p{0.3\linewidth}}{\cellcolor[rgb]{0.95,0.95,0.95}{%
	\makecell[{{p{\linewidth}}}]{% 
	  \tt {\tiny [GM$|$GM]}  
	 Ich bin auch der Meinung, dass es sinnvoll ist, die Verteilung zu besprechen. Ich denke, dass ein faires Verhältnis und die persönlichen Vorlieben beide eine Rolle spielen sollten. Ich persönlich würde mich gerne auf die Briefmarken konzentrieren, da ich denke, dass sie einen gewissen Wert haben könnten. Wie wichtig sind dir die einzelnen Gegenstände? Gibt es einen, an dem du besonders interessiert bist? 
	  } 
	   } 
	   } 
	 & & \\ 
 

    \theutterance \stepcounter{utterance}  

    & & \multicolumn{4}{p{0.6\linewidth}}{\cellcolor[rgb]{0.9,0.9,0.9}{%
	\makecell[{{p{\linewidth}}}]{% 
	  \tt {\tiny [A$\langle$GM]}  
	 Ich bin auch der Meinung, dass es sinnvoll ist, die Verteilung zu besprechen. Ich denke, dass ein faires Verhältnis und die persönlichen Vorlieben beide eine Rolle spielen sollten. Ich persönlich würde mich gerne auf die Briefmarken konzentrieren, da ich denke, dass sie einen gewissen Wert haben könnten. Wie wichtig sind dir die einzelnen Gegenstände? Gibt es einen, an dem du besonders interessiert bist? 
	  } 
	   } 
	   } 
	 & \\ 
 

    \theutterance \stepcounter{utterance}  

    & \multicolumn{4}{p{0.6\linewidth}}{\cellcolor[rgb]{0.8,1,0.9}{%
	\makecell[{{p{\linewidth}}}]{% 
	  \tt {\tiny [A$\rangle$GM]}  
	 Die Briefmarken sind interessant, aber ich muss zugeben, dass ich nicht viel von ihnen halte. Ich denke, dass der Fingerhut und die Seife möglicherweise mehr Wert haben könnten, aber ich bin nicht sicher, wie du darüber denkst. Hast du eine Vorstellung davon, wie wir die Seife und den Fingerhut aufteilen könnten? Sollten wir perhaps einen von ihnen aufteilen oder versuchen, sie als Ganzes zu erhalten? 
	  } 
	   } 
	   } 
	 & & \\ 
 

    \theutterance \stepcounter{utterance}  

    & & & \multicolumn{2}{p{0.3\linewidth}}{\cellcolor[rgb]{0.95,0.95,0.95}{%
	\makecell[{{p{\linewidth}}}]{% 
	  \tt {\tiny [GM$|$GM]}  
	 continue 
	  } 
	   } 
	   } 
	 & & \\ 
 

    \theutterance \stepcounter{utterance}  

    & & & \multicolumn{2}{p{0.3\linewidth}}{\cellcolor[rgb]{0.95,0.95,0.95}{%
	\makecell[{{p{\linewidth}}}]{% 
	  \tt {\tiny [GM$|$GM]}  
	 Die Briefmarken sind interessant, aber ich muss zugeben, dass ich nicht viel von ihnen halte. Ich denke, dass der Fingerhut und die Seife möglicherweise mehr Wert haben könnten, aber ich bin nicht sicher, wie du darüber denkst. Hast du eine Vorstellung davon, wie wir die Seife und den Fingerhut aufteilen könnten? Sollten wir perhaps einen von ihnen aufteilen oder versuchen, sie als Ganzes zu erhalten? 
	  } 
	   } 
	   } 
	 & & \\ 
 

    \theutterance \stepcounter{utterance}  

    & & \multicolumn{4}{p{0.6\linewidth}}{\cellcolor[rgb]{0.9,0.9,0.9}{%
	\makecell[{{p{\linewidth}}}]{% 
	  \tt {\tiny [GM$\rangle$B]}  
	 Die Briefmarken sind interessant, aber ich muss zugeben, dass ich nicht viel von ihnen halte. Ich denke, dass der Fingerhut und die Seife möglicherweise mehr Wert haben könnten, aber ich bin nicht sicher, wie du darüber denkst. Hast du eine Vorstellung davon, wie wir die Seife und den Fingerhut aufteilen könnten? Sollten wir perhaps einen von ihnen aufteilen oder versuchen, sie als Ganzes zu erhalten? 
	  } 
	   } 
	   } 
	 & \\ 
 

    \theutterance \stepcounter{utterance}  

    & & & \multicolumn{4}{p{0.6\linewidth}}{\cellcolor[rgb]{1,0.85,0.72}{%
	\makecell[{{p{\linewidth}}}]{% 
	  \tt {\tiny [GM$\langle$B]}  
	 Ich denke, dass ein Aufteilen von Gegenständen wie dem Fingerhut und der Seife nicht praktikabel wäre, da sie einzigartige Gegenstände sind. Es wäre vielleicht besser, wenn einer von uns den Fingerhut und der andere die Seife bekommt. Ich selbst habe eher Interesse am Fingerhut, aber ich bin auch bereit, Kompromisse einzugehen, um eine faire Verteilung zu erreichen. Wie stehst du zu den Briefmarken? Könntest du dir vorstellen, sie alle zu bekommen, oder würdest du sie gerne aufteilen? 
	  } 
	   } 
	   } 
	  \\ 
 

    \theutterance \stepcounter{utterance}  

    & & & \multicolumn{2}{p{0.3\linewidth}}{\cellcolor[rgb]{0.95,0.95,0.95}{%
	\makecell[{{p{\linewidth}}}]{% 
	  \tt {\tiny [GM$|$GM]}  
	 continue 
	  } 
	   } 
	   } 
	 & & \\ 
 

    \theutterance \stepcounter{utterance}  

    & & & \multicolumn{2}{p{0.3\linewidth}}{\cellcolor[rgb]{0.95,0.95,0.95}{%
	\makecell[{{p{\linewidth}}}]{% 
	  \tt {\tiny [GM$|$GM]}  
	 Ich denke, dass ein Aufteilen von Gegenständen wie dem Fingerhut und der Seife nicht praktikabel wäre, da sie einzigartige Gegenstände sind. Es wäre vielleicht besser, wenn einer von uns den Fingerhut und der andere die Seife bekommt. Ich selbst habe eher Interesse am Fingerhut, aber ich bin auch bereit, Kompromisse einzugehen, um eine faire Verteilung zu erreichen. Wie stehst du zu den Briefmarken? Könntest du dir vorstellen, sie alle zu bekommen, oder würdest du sie gerne aufteilen? 
	  } 
	   } 
	   } 
	 & & \\ 
 

    \theutterance \stepcounter{utterance}  

    & & \multicolumn{4}{p{0.6\linewidth}}{\cellcolor[rgb]{0.9,0.9,0.9}{%
	\makecell[{{p{\linewidth}}}]{% 
	  \tt {\tiny [A$\langle$GM]}  
	 Ich denke, dass ein Aufteilen von Gegenständen wie dem Fingerhut und der Seife nicht praktikabel wäre, da sie einzigartige Gegenstände sind. Es wäre vielleicht besser, wenn einer von uns den Fingerhut und der andere die Seife bekommt. Ich selbst habe eher Interesse am Fingerhut, aber ich bin auch bereit, Kompromisse einzugehen, um eine faire Verteilung zu erreichen. Wie stehst du zu den Briefmarken? Könntest du dir vorstellen, sie alle zu bekommen, oder würdest du sie gerne aufteilen? 
	  } 
	   } 
	   } 
	 & \\ 
 

    \theutterance \stepcounter{utterance}  

    & \multicolumn{4}{p{0.6\linewidth}}{\cellcolor[rgb]{0.8,1,0.9}{%
	\makecell[{{p{\linewidth}}}]{% 
	  \tt {\tiny [A$\rangle$GM]}  
	 Das ist ein guter Punkt, dass der Fingerhut und die Seife einzigartige Gegenstände sind. Ich denke, dass es fair wäre, wenn einer von uns den Fingerhut und der andere die Seife bekommt. Was die Briefmarken betrifft, so bin ich nicht besonders an ihnen interessiert. Ich denke, dass es okay wäre, wenn du alle Briefmarken bekämst, da du dich bereits auf sie konzentriert hast. Ich selbst würde mich eher auf den Fingerhut oder die Seife konzentrieren. Wie wäre es, wenn wir einen Vorschlag machen, um die Dinge zu klären? Ich denke, dass ich den Fingerhut und du die Seife und die Briefmarken bekommst. Wie gefällt dir das? 
	  } 
	   } 
	   } 
	 & & \\ 
 

    \theutterance \stepcounter{utterance}  

    & & & \multicolumn{2}{p{0.3\linewidth}}{\cellcolor[rgb]{0.95,0.95,0.95}{%
	\makecell[{{p{\linewidth}}}]{% 
	  \tt {\tiny [GM$|$GM]}  
	 continue 
	  } 
	   } 
	   } 
	 & & \\ 
 

    \theutterance \stepcounter{utterance}  

    & & & \multicolumn{2}{p{0.3\linewidth}}{\cellcolor[rgb]{0.95,0.95,0.95}{%
	\makecell[{{p{\linewidth}}}]{% 
	  \tt {\tiny [GM$|$GM]}  
	 Das ist ein guter Punkt, dass der Fingerhut und die Seife einzigartige Gegenstände sind. Ich denke, dass es fair wäre, wenn einer von uns den Fingerhut und der andere die Seife bekommt. Was die Briefmarken betrifft, so bin ich nicht besonders an ihnen interessiert. Ich denke, dass es okay wäre, wenn du alle Briefmarken bekämst, da du dich bereits auf sie konzentriert hast. Ich selbst würde mich eher auf den Fingerhut oder die Seife konzentrieren. Wie wäre es, wenn wir einen Vorschlag machen, um die Dinge zu klären? Ich denke, dass ich den Fingerhut und du die Seife und die Briefmarken bekommst. Wie gefällt dir das? 
	  } 
	   } 
	   } 
	 & & \\ 
 

    \theutterance \stepcounter{utterance}  

    & & \multicolumn{4}{p{0.6\linewidth}}{\cellcolor[rgb]{0.9,0.9,0.9}{%
	\makecell[{{p{\linewidth}}}]{% 
	  \tt {\tiny [GM$\rangle$B]}  
	 Das ist ein guter Punkt, dass der Fingerhut und die Seife einzigartige Gegenstände sind. Ich denke, dass es fair wäre, wenn einer von uns den Fingerhut und der andere die Seife bekommt. Was die Briefmarken betrifft, so bin ich nicht besonders an ihnen interessiert. Ich denke, dass es okay wäre, wenn du alle Briefmarken bekämst, da du dich bereits auf sie konzentriert hast. Ich selbst würde mich eher auf den Fingerhut oder die Seife konzentrieren. Wie wäre es, wenn wir einen Vorschlag machen, um die Dinge zu klären? Ich denke, dass ich den Fingerhut und du die Seife und die Briefmarken bekommst. Wie gefällt dir das? 
	  } 
	   } 
	   } 
	 & \\ 
 

    \theutterance \stepcounter{utterance}  

    & & & \multicolumn{4}{p{0.6\linewidth}}{\cellcolor[rgb]{1,0.85,0.72}{%
	\makecell[{{p{\linewidth}}}]{% 
	  \tt {\tiny [GM$\langle$B]}  
	 Das klingt nach einem vernünftigen Vorschlag, aber ich denke, dass ich vielleicht eine bessere Idee habe. Da ich mich auf den Fingerhut und die Briefmarken konzentriere, wie wäre es, wenn ich den Fingerhut und die Briefmarken bekomme und du die Seife? Auf diese Weise bekommen wir beide die Dinge, die uns am meisten interessieren. Ich denke, dass das eine faire und ausgewogene Verteilung wäre. Was meinst du dazu? 
	  } 
	   } 
	   } 
	  \\ 
 

    \theutterance \stepcounter{utterance}  

    & & & \multicolumn{2}{p{0.3\linewidth}}{\cellcolor[rgb]{0.95,0.95,0.95}{%
	\makecell[{{p{\linewidth}}}]{% 
	  \tt {\tiny [GM$|$GM]}  
	 continue 
	  } 
	   } 
	   } 
	 & & \\ 
 

    \theutterance \stepcounter{utterance}  

    & & & \multicolumn{2}{p{0.3\linewidth}}{\cellcolor[rgb]{0.95,0.95,0.95}{%
	\makecell[{{p{\linewidth}}}]{% 
	  \tt {\tiny [GM$|$GM]}  
	 Das klingt nach einem vernünftigen Vorschlag, aber ich denke, dass ich vielleicht eine bessere Idee habe. Da ich mich auf den Fingerhut und die Briefmarken konzentriere, wie wäre es, wenn ich den Fingerhut und die Briefmarken bekomme und du die Seife? Auf diese Weise bekommen wir beide die Dinge, die uns am meisten interessieren. Ich denke, dass das eine faire und ausgewogene Verteilung wäre. Was meinst du dazu? 
	  } 
	   } 
	   } 
	 & & \\ 
 

    \theutterance \stepcounter{utterance}  

    & & \multicolumn{4}{p{0.6\linewidth}}{\cellcolor[rgb]{0.9,0.9,0.9}{%
	\makecell[{{p{\linewidth}}}]{% 
	  \tt {\tiny [A$\langle$GM]}  
	 Das klingt nach einem vernünftigen Vorschlag, aber ich denke, dass ich vielleicht eine bessere Idee habe. Da ich mich auf den Fingerhut und die Briefmarken konzentriere, wie wäre es, wenn ich den Fingerhut und die Briefmarken bekomme und du die Seife? Auf diese Weise bekommen wir beide die Dinge, die uns am meisten interessieren. Ich denke, dass das eine faire und ausgewogene Verteilung wäre. Was meinst du dazu? 
	  } 
	   } 
	   } 
	 & \\ 
 

    \theutterance \stepcounter{utterance}  

    & \multicolumn{4}{p{0.6\linewidth}}{\cellcolor[rgb]{0.8,1,0.9}{%
	\makecell[{{p{\linewidth}}}]{% 
	  \tt {\tiny [A$\rangle$GM]}  
	 Das ist ein interessanter Gegenvorschlag! Ich denke, dass das eine gute Lösung wäre, da wir beide die Dinge bekommen, die uns am meisten interessieren. Ich bin einverstanden, dass du den Fingerhut und die Briefmarken bekommst und ich die Seife. Das sieht nach einer fairen und ausgewogenen Verteilung aus. Ich denke, wir haben eine gute Lösung gefunden! Sollen wir jetzt unseren geheimen Vorschlag machen, um die Sache offiziell zu machen? 
	  } 
	   } 
	   } 
	 & & \\ 
 

    \theutterance \stepcounter{utterance}  

    & & & \multicolumn{2}{p{0.3\linewidth}}{\cellcolor[rgb]{0.95,0.95,0.95}{%
	\makecell[{{p{\linewidth}}}]{% 
	  \tt {\tiny [GM$|$GM]}  
	 continue 
	  } 
	   } 
	   } 
	 & & \\ 
 

    \theutterance \stepcounter{utterance}  

    & & & \multicolumn{2}{p{0.3\linewidth}}{\cellcolor[rgb]{0.95,0.95,0.95}{%
	\makecell[{{p{\linewidth}}}]{% 
	  \tt {\tiny [GM$|$GM]}  
	 Das ist ein interessanter Gegenvorschlag! Ich denke, dass das eine gute Lösung wäre, da wir beide die Dinge bekommen, die uns am meisten interessieren. Ich bin einverstanden, dass du den Fingerhut und die Briefmarken bekommst und ich die Seife. Das sieht nach einer fairen und ausgewogenen Verteilung aus. Ich denke, wir haben eine gute Lösung gefunden! Sollen wir jetzt unseren geheimen Vorschlag machen, um die Sache offiziell zu machen? 
	  } 
	   } 
	   } 
	 & & \\ 
 

    \theutterance \stepcounter{utterance}  

    & & \multicolumn{4}{p{0.6\linewidth}}{\cellcolor[rgb]{0.9,0.9,0.9}{%
	\makecell[{{p{\linewidth}}}]{% 
	  \tt {\tiny [GM$\rangle$B]}  
	 Das ist ein interessanter Gegenvorschlag! Ich denke, dass das eine gute Lösung wäre, da wir beide die Dinge bekommen, die uns am meisten interessieren. Ich bin einverstanden, dass du den Fingerhut und die Briefmarken bekommst und ich die Seife. Das sieht nach einer fairen und ausgewogenen Verteilung aus. Ich denke, wir haben eine gute Lösung gefunden! Sollen wir jetzt unseren geheimen Vorschlag machen, um die Sache offiziell zu machen? 
	  } 
	   } 
	   } 
	 & \\ 
 

    \theutterance \stepcounter{utterance}  

    & & & \multicolumn{4}{p{0.6\linewidth}}{\cellcolor[rgb]{1,0.85,0.72}{%
	\makecell[{{p{\linewidth}}}]{% 
	  \tt {\tiny [GM$\langle$B]}  
	 Ja, ich denke, dass wir eine gute Lösung gefunden haben! Ich bin zufrieden mit dem Ergebnis und ich denke, dass es fair für beide Seiten ist. Lass uns unseren geheimen Vorschlag machen, um die Sache offiziell zu machen. Ich werde meinen Vorschlag machen und du machst deinen, und dann können wir sehen, ob sie komplementär sind. Ich bin gespannt, ob wir erfolgreich sind! [Vorschlag: 1 Fingerhut, 4 Briefmarken] 
	  } 
	   } 
	   } 
	  \\ 
 

    \theutterance \stepcounter{utterance}  

    & & & \multicolumn{2}{p{0.3\linewidth}}{\cellcolor[rgb]{0.95,0.95,0.95}{%
	\makecell[{{p{\linewidth}}}]{% 
	  \tt {\tiny [GM$|$GM]}  
	 proposal 
	  } 
	   } 
	   } 
	 & & \\ 
 

    \theutterance \stepcounter{utterance}  

    & & & \multicolumn{2}{p{0.3\linewidth}}{\cellcolor[rgb]{0.95,0.95,0.95}{%
	\makecell[{{p{\linewidth}}}]{% 
	  \tt {\tiny [GM$|$GM]}  
	 [1, 0, 4] 
	  } 
	   } 
	   } 
	 & & \\ 
 

    \theutterance \stepcounter{utterance}  

    & & \multicolumn{4}{p{0.6\linewidth}}{\cellcolor[rgb]{0.9,0.9,0.9}{%
	\makecell[{{p{\linewidth}}}]{% 
	  \tt {\tiny [A$\langle$GM]}  
	 Der andere Spieler hat einen geheimen Vorschlag gemacht. Jetzt bist du an der Reihe, einen eigenen Vorschlag zu machen.\\ \tt  
	  } 
	   } 
	   } 
	 & \\ 
 

    \theutterance \stepcounter{utterance}  

    & \multicolumn{4}{p{0.6\linewidth}}{\cellcolor[rgb]{0.8,1,0.9}{%
	\makecell[{{p{\linewidth}}}]{% 
	  \tt {\tiny [A$\rangle$GM]}  
	 [Vorschlag: 0 Fingerhut, 1 Seife, 0 Briefmarke] 
	  } 
	   } 
	   } 
	 & & \\ 
 

    \theutterance \stepcounter{utterance}  

    & & & \multicolumn{2}{p{0.3\linewidth}}{\cellcolor[rgb]{0.95,0.95,0.95}{%
	\makecell[{{p{\linewidth}}}]{% 
	  \tt {\tiny [GM$|$GM]}  
	 proposal 
	  } 
	   } 
	   } 
	 & & \\ 
 

    \theutterance \stepcounter{utterance}  

    & & & \multicolumn{2}{p{0.3\linewidth}}{\cellcolor[rgb]{0.95,0.95,0.95}{%
	\makecell[{{p{\linewidth}}}]{% 
	  \tt {\tiny [GM$|$GM]}  
	 [0, 1, 0] 
	  } 
	   } 
	   } 
	 & & \\ 
 

    \theutterance \stepcounter{utterance}  

    & & & \multicolumn{2}{p{0.3\linewidth}}{\cellcolor[rgb]{0.95,0.95,0.95}{%
	\makecell[{{p{\linewidth}}}]{% 
	  \tt {\tiny [GM$|$GM]}  
	 [[0, 1, 0], [1, 0, 4]] 
	  } 
	   } 
	   } 
	 & & \\ 
 

\end{supertabular}
}

\end{document}
