
\documentclass{article}
\usepackage{colortbl}
\usepackage{makecell}
\usepackage{multirow}
\usepackage{supertabular}

\begin{document}

\newcounter{utterance}

\centering \large Interaction Transcript for game `dond', experiment `coop\_de', episode 11 with llama{-}3.3{-}70b{-}instruct{-}t1.0.
\vspace{24pt}

{ \footnotesize  \setcounter{utterance}{1}
\setlength{\tabcolsep}{0pt}
\begin{supertabular}{c@{$\;$}|p{.15\linewidth}@{}p{.15\linewidth}p{.15\linewidth}p{.15\linewidth}p{.15\linewidth}p{.15\linewidth}}
    \# & $\;$A & \multicolumn{4}{c}{Game Master} & $\;\:$B\\
    \hline

    \theutterance \stepcounter{utterance}  
    & & \multicolumn{4}{p{0.6\linewidth}}{
        \cellcolor[rgb]{0.9,0.9,0.9}{
            \makecell[{{p{\linewidth}}}]{
                \texttt{\tiny{[P1$\langle$GM]}}
                \texttt{Sie spielen ein kooperatives Verhandlungsspiel, bei dem Sie sich mit einem anderen Spieler darauf einigen müssen, wie eine Reihe von Gegenständen aufgeteilt werden soll.} \\
\\ 
\texttt{Die Regeln:} \\
\texttt{(a) Sie und der andere Spieler erhalten eine Sammlung von Gegenständen. Jeder von Ihnen erhält außerdem eine geheime Wertfunktion, die angibt, wie viel Ihnen jede Art von Gegenstand wert ist.} \\
\texttt{(b) Sie tauschen Nachrichten mit dem anderen Spieler aus, um zu vereinbaren, wer welche Gegenstände bekommt. Sie können jeweils maximal 5 Nachrichten senden oder das Spiel vorzeitig beenden, indem Sie jederzeit einen geheimen Vorschlag machen.} \\
\texttt{(c) Jeder von euch wird aufgefordert, einen geheimen Vorschlag zu machen, in dem ihr die gewünschten Gegenstände in eckigen Klammern wie folgt angibt: "{[}Vorschlag: <Nummer> <Objektname>, <Nummer> <Objektname>, <...>{]}"} \\
\texttt{(d) Wenn eure Vorschläge komplementär sind, d.h. es gibt genug Gegenstände, um beide Vorschläge zu erfüllen, erhält jeder Spieler eine Punktzahl, die sich aus der Summe der Werte für die Gegenstände ergibt, die er erhalten hat. Andernfalls erhalten Sie beide null Punkte.} \\
\texttt{(e) Ihr Ziel ist es, die Summe aus Ihrer Punktzahl und der Punktzahl des anderen Spielers zu maximieren.} \\
\\ 
\texttt{Beginnen wir.} \\
\\ 
\texttt{Die Menge der verfügbaren Gegenstände ist:} \\
\\ 
\texttt{1 Ring, 3 Früchte, 1 Kreide} \\
\\ 
\texttt{Deine geheime Wertfunktion ist:} \\
\\ 
\texttt{Ring: 6, Frucht: 1, Kreide: 1} \\
\\ 
\texttt{WICHTIG: Ihre Nachrichten werden, sofern es sich nicht um einen geheimen Vorschlag handelt, direkt an den anderen Spieler übermittelt, also fügen Sie keine Antwort auf die Regeln oder einen Text zur Ankündigung Ihrer Nachricht ein. Um einen geheimen Vorschlag zu machen, verwenden Sie das angegebene Format. Verwenden Sie keine eckigen Klammern, wenn Sie mit dem anderen Spieler kommunizieren, sonst wird dies als Ihr geheimer Vorschlag interpretiert.} \\
            }
        }
    }
    & \\ \\

    \theutterance \stepcounter{utterance}  
    & \multicolumn{4}{p{0.6\linewidth}}{
        \cellcolor[rgb]{0.9,0.9,0.9}{
            \makecell[{{p{\linewidth}}}]{
                \texttt{\tiny{[P1$\rangle$GM]}}
                \texttt{Hallo, ich denke, wir sollten beide eine Liste mit den Gegenständen erstellen, die wir am meisten wertschätzen. Wie schätzt du die Werte der Gegenstände ein?} \\
            }
        }
    }
    & & \\ \\

    \theutterance \stepcounter{utterance}  
    & & & \multicolumn{2}{p{0.3\linewidth}}{
        \cellcolor[rgb]{0.9,0.9,0.9}{
            \makecell[{{p{\linewidth}}}]{
                \texttt{\tiny{[GM$|$GM]}}
                \texttt{continue} \\
            }
        }
    }
    & & \\ \\

    \theutterance \stepcounter{utterance}  
    & & & \multicolumn{2}{p{0.3\linewidth}}{
        \cellcolor[rgb]{0.9,0.9,0.9}{
            \makecell[{{p{\linewidth}}}]{
                \texttt{\tiny{[GM$|$GM]}}
                \texttt{Hallo, ich denke, wir sollten beide eine Liste mit den Gegenständen erstellen, die wir am meisten wertschätzen. Wie schätzt du die Werte der Gegenstände ein?} \\
            }
        }
    }
    & & \\ \\

    \theutterance \stepcounter{utterance}  
    & & \multicolumn{4}{p{0.6\linewidth}}{
        \cellcolor[rgb]{0.9,0.9,0.9}{
            \makecell[{{p{\linewidth}}}]{
                \texttt{\tiny{[GM$\rangle$P2]}}
                \texttt{Sie spielen ein kooperatives Verhandlungsspiel, bei dem Sie sich mit einem anderen Spieler darauf einigen müssen, wie eine Reihe von Gegenständen aufgeteilt werden soll.} \\
\\ 
\texttt{Die Regeln:} \\
\texttt{(a) Sie und der andere Spieler erhalten eine Sammlung von Gegenständen. Jeder von Ihnen erhält außerdem eine geheime Wertfunktion, die angibt, wie viel Ihnen jede Art von Gegenstand wert ist.} \\
\texttt{(b) Sie tauschen Nachrichten mit dem anderen Spieler aus, um zu vereinbaren, wer welche Gegenstände bekommt. Sie können jeweils maximal 5 Nachrichten senden oder das Spiel vorzeitig beenden, indem Sie jederzeit einen geheimen Vorschlag machen.} \\
\texttt{(c) Jeder von euch wird aufgefordert, einen geheimen Vorschlag zu machen, in dem ihr die gewünschten Gegenstände in eckigen Klammern wie folgt angibt: "{[}Vorschlag: <Nummer> <Objektname>, <Nummer> <Objektname>, <...>{]}"} \\
\texttt{(d) Wenn eure Vorschläge komplementär sind, d.h. es gibt genug Gegenstände, um beide Vorschläge zu erfüllen, erhält jeder Spieler eine Punktzahl, die sich aus der Summe der Werte für die Gegenstände ergibt, die er erhalten hat. Andernfalls erhalten Sie beide null Punkte.} \\
\texttt{(e) Ihr Ziel ist es, die Summe aus Ihrer Punktzahl und der Punktzahl des anderen Spielers zu maximieren.} \\
\\ 
\texttt{Beginnen wir.} \\
\\ 
\texttt{Die Menge der verfügbaren Gegenstände ist:} \\
\\ 
\texttt{1 Ring, 3 Früchte, 1 Kreide} \\
\\ 
\texttt{Deine geheime Wertfunktion ist:} \\
\\ 
\texttt{Ring: 6, Frucht: 1, Kreide: 1} \\
\\ 
\texttt{WICHTIG: Ihre Nachrichten werden, sofern es sich nicht um einen geheimen Vorschlag handelt, direkt an den anderen Spieler übermittelt, also fügen Sie keine Antwort auf die Regeln oder einen Text zur Ankündigung Ihrer Nachricht ein. Um einen geheimen Vorschlag zu machen, verwenden Sie das angegebene Format. Verwenden Sie keine eckigen Klammern, wenn Sie mit dem anderen Spieler kommunizieren, sonst wird dies als Ihr geheimer Vorschlag interpretiert.} \\
\\ 
\\ 
\texttt{Hallo, ich denke, wir sollten beide eine Liste mit den Gegenständen erstellen, die wir am meisten wertschätzen. Wie schätzt du die Werte der Gegenstände ein?} \\
            }
        }
    }
    & \\ \\

    \theutterance \stepcounter{utterance}  
    & & & \multicolumn{4}{p{0.6\linewidth}}{
        \cellcolor[rgb]{0.9,0.9,0.9}{
            \makecell[{{p{\linewidth}}}]{
                \texttt{\tiny{[GM$\langle$P2]}}
                \texttt{Ich denke, das ist eine gute Idee! Ich würde sagen, der Ring ist ziemlich wertvoll, gefolgt von den Früchten und dann der Kreide. Ich bin aber auch sehr an den Früchten interessiert, weil es davon mehr gibt. Wie steht es bei dir? Welche Gegenstände bevorzugst du?} \\
            }
        }
    }
     \\ \\

    \theutterance \stepcounter{utterance}  
    & & & \multicolumn{2}{p{0.3\linewidth}}{
        \cellcolor[rgb]{0.9,0.9,0.9}{
            \makecell[{{p{\linewidth}}}]{
                \texttt{\tiny{[GM$|$GM]}}
                \texttt{continue} \\
            }
        }
    }
    & & \\ \\

    \theutterance \stepcounter{utterance}  
    & & & \multicolumn{2}{p{0.3\linewidth}}{
        \cellcolor[rgb]{0.9,0.9,0.9}{
            \makecell[{{p{\linewidth}}}]{
                \texttt{\tiny{[GM$|$GM]}}
                \texttt{Ich denke, das ist eine gute Idee! Ich würde sagen, der Ring ist ziemlich wertvoll, gefolgt von den Früchten und dann der Kreide. Ich bin aber auch sehr an den Früchten interessiert, weil es davon mehr gibt. Wie steht es bei dir? Welche Gegenstände bevorzugst du?} \\
            }
        }
    }
    & & \\ \\

    \theutterance \stepcounter{utterance}  
    & & \multicolumn{4}{p{0.6\linewidth}}{
        \cellcolor[rgb]{0.9,0.9,0.9}{
            \makecell[{{p{\linewidth}}}]{
                \texttt{\tiny{[P1$\langle$GM]}}
                \texttt{Ich denke, das ist eine gute Idee! Ich würde sagen, der Ring ist ziemlich wertvoll, gefolgt von den Früchten und dann der Kreide. Ich bin aber auch sehr an den Früchten interessiert, weil es davon mehr gibt. Wie steht es bei dir? Welche Gegenstände bevorzugst du?} \\
            }
        }
    }
    & \\ \\

    \theutterance \stepcounter{utterance}  
    & \multicolumn{4}{p{0.6\linewidth}}{
        \cellcolor[rgb]{0.9,0.9,0.9}{
            \makecell[{{p{\linewidth}}}]{
                \texttt{\tiny{[P1$\rangle$GM]}}
                \texttt{Ja, der Ring ist definitiv der wertvollste Gegenstand. Ich persönlich finde den Ring sehr attraktiv, da er einzigartig ist und ich denke, er ist für beide von uns sehr wertvoll. Die Früchte und die Kreide sind auch wichtig, aber ich denke, wir sollten uns auf den Ring konzentrieren und dann die anderen Gegenstände aufteilen. Wie viele Früchte denkst du, dass du gerne haben möchtest?} \\
            }
        }
    }
    & & \\ \\

    \theutterance \stepcounter{utterance}  
    & & & \multicolumn{2}{p{0.3\linewidth}}{
        \cellcolor[rgb]{0.9,0.9,0.9}{
            \makecell[{{p{\linewidth}}}]{
                \texttt{\tiny{[GM$|$GM]}}
                \texttt{continue} \\
            }
        }
    }
    & & \\ \\

    \theutterance \stepcounter{utterance}  
    & & & \multicolumn{2}{p{0.3\linewidth}}{
        \cellcolor[rgb]{0.9,0.9,0.9}{
            \makecell[{{p{\linewidth}}}]{
                \texttt{\tiny{[GM$|$GM]}}
                \texttt{Ja, der Ring ist definitiv der wertvollste Gegenstand. Ich persönlich finde den Ring sehr attraktiv, da er einzigartig ist und ich denke, er ist für beide von uns sehr wertvoll. Die Früchte und die Kreide sind auch wichtig, aber ich denke, wir sollten uns auf den Ring konzentrieren und dann die anderen Gegenstände aufteilen. Wie viele Früchte denkst du, dass du gerne haben möchtest?} \\
            }
        }
    }
    & & \\ \\

    \theutterance \stepcounter{utterance}  
    & & \multicolumn{4}{p{0.6\linewidth}}{
        \cellcolor[rgb]{0.9,0.9,0.9}{
            \makecell[{{p{\linewidth}}}]{
                \texttt{\tiny{[GM$\rangle$P2]}}
                \texttt{Ja, der Ring ist definitiv der wertvollste Gegenstand. Ich persönlich finde den Ring sehr attraktiv, da er einzigartig ist und ich denke, er ist für beide von uns sehr wertvoll. Die Früchte und die Kreide sind auch wichtig, aber ich denke, wir sollten uns auf den Ring konzentrieren und dann die anderen Gegenstände aufteilen. Wie viele Früchte denkst du, dass du gerne haben möchtest?} \\
            }
        }
    }
    & \\ \\

    \theutterance \stepcounter{utterance}  
    & & & \multicolumn{4}{p{0.6\linewidth}}{
        \cellcolor[rgb]{0.9,0.9,0.9}{
            \makecell[{{p{\linewidth}}}]{
                \texttt{\tiny{[GM$\langle$P2]}}
                \texttt{Das ist ein guter Punkt, dass der Ring einzigartig ist und für beide von uns sehr wertvoll sein könnte. Beim Thema Früchte... ich denke, zwei Früchte wären ein guter Ausgangspunkt für mich. Ich bin nicht so sehr an der Kreide interessiert, aber wenn es möglich wäre, auch die Kreide zu bekommen, wäre das okay. Wie viele Früchte denkst du, dass du gerne haben möchtest, und hast du eine Vorstellung, wie wir den Ring aufteilen könnten?} \\
            }
        }
    }
     \\ \\

    \theutterance \stepcounter{utterance}  
    & & & \multicolumn{2}{p{0.3\linewidth}}{
        \cellcolor[rgb]{0.9,0.9,0.9}{
            \makecell[{{p{\linewidth}}}]{
                \texttt{\tiny{[GM$|$GM]}}
                \texttt{continue} \\
            }
        }
    }
    & & \\ \\

    \theutterance \stepcounter{utterance}  
    & & & \multicolumn{2}{p{0.3\linewidth}}{
        \cellcolor[rgb]{0.9,0.9,0.9}{
            \makecell[{{p{\linewidth}}}]{
                \texttt{\tiny{[GM$|$GM]}}
                \texttt{Das ist ein guter Punkt, dass der Ring einzigartig ist und für beide von uns sehr wertvoll sein könnte. Beim Thema Früchte... ich denke, zwei Früchte wären ein guter Ausgangspunkt für mich. Ich bin nicht so sehr an der Kreide interessiert, aber wenn es möglich wäre, auch die Kreide zu bekommen, wäre das okay. Wie viele Früchte denkst du, dass du gerne haben möchtest, und hast du eine Vorstellung, wie wir den Ring aufteilen könnten?} \\
            }
        }
    }
    & & \\ \\

    \theutterance \stepcounter{utterance}  
    & & \multicolumn{4}{p{0.6\linewidth}}{
        \cellcolor[rgb]{0.9,0.9,0.9}{
            \makecell[{{p{\linewidth}}}]{
                \texttt{\tiny{[P1$\langle$GM]}}
                \texttt{Das ist ein guter Punkt, dass der Ring einzigartig ist und für beide von uns sehr wertvoll sein könnte. Beim Thema Früchte... ich denke, zwei Früchte wären ein guter Ausgangspunkt für mich. Ich bin nicht so sehr an der Kreide interessiert, aber wenn es möglich wäre, auch die Kreide zu bekommen, wäre das okay. Wie viele Früchte denkst du, dass du gerne haben möchtest, und hast du eine Vorstellung, wie wir den Ring aufteilen könnten?} \\
            }
        }
    }
    & \\ \\

    \theutterance \stepcounter{utterance}  
    & \multicolumn{4}{p{0.6\linewidth}}{
        \cellcolor[rgb]{0.9,0.9,0.9}{
            \makecell[{{p{\linewidth}}}]{
                \texttt{\tiny{[P1$\rangle$GM]}}
                \texttt{Zwei Früchte sind eine vernünftige Forderung. Ich persönlich denke, dass ich gerne eine Frucht haben würde, um meine Sammlung abzurunden. Beim Ring... das ist ein schwieriges Thema. Ich könnte mir vorstellen, dass wir den Ring einfach einem von uns geben und dann die anderen Gegenstände entsprechend aufteilen, um es fair zu machen. Oder wir könnten versuchen, eine andere Lösung zu finden, die für beide von uns akzeptabel ist. Was hältst du davon, wenn ich den Ring bekomme und du die drei Früchte und die Kreide?} \\
            }
        }
    }
    & & \\ \\

    \theutterance \stepcounter{utterance}  
    & & & \multicolumn{2}{p{0.3\linewidth}}{
        \cellcolor[rgb]{0.9,0.9,0.9}{
            \makecell[{{p{\linewidth}}}]{
                \texttt{\tiny{[GM$|$GM]}}
                \texttt{continue} \\
            }
        }
    }
    & & \\ \\

    \theutterance \stepcounter{utterance}  
    & & & \multicolumn{2}{p{0.3\linewidth}}{
        \cellcolor[rgb]{0.9,0.9,0.9}{
            \makecell[{{p{\linewidth}}}]{
                \texttt{\tiny{[GM$|$GM]}}
                \texttt{Zwei Früchte sind eine vernünftige Forderung. Ich persönlich denke, dass ich gerne eine Frucht haben würde, um meine Sammlung abzurunden. Beim Ring... das ist ein schwieriges Thema. Ich könnte mir vorstellen, dass wir den Ring einfach einem von uns geben und dann die anderen Gegenstände entsprechend aufteilen, um es fair zu machen. Oder wir könnten versuchen, eine andere Lösung zu finden, die für beide von uns akzeptabel ist. Was hältst du davon, wenn ich den Ring bekomme und du die drei Früchte und die Kreide?} \\
            }
        }
    }
    & & \\ \\

    \theutterance \stepcounter{utterance}  
    & & \multicolumn{4}{p{0.6\linewidth}}{
        \cellcolor[rgb]{0.9,0.9,0.9}{
            \makecell[{{p{\linewidth}}}]{
                \texttt{\tiny{[GM$\rangle$P2]}}
                \texttt{Zwei Früchte sind eine vernünftige Forderung. Ich persönlich denke, dass ich gerne eine Frucht haben würde, um meine Sammlung abzurunden. Beim Ring... das ist ein schwieriges Thema. Ich könnte mir vorstellen, dass wir den Ring einfach einem von uns geben und dann die anderen Gegenstände entsprechend aufteilen, um es fair zu machen. Oder wir könnten versuchen, eine andere Lösung zu finden, die für beide von uns akzeptabel ist. Was hältst du davon, wenn ich den Ring bekomme und du die drei Früchte und die Kreide?} \\
            }
        }
    }
    & \\ \\

    \theutterance \stepcounter{utterance}  
    & & & \multicolumn{4}{p{0.6\linewidth}}{
        \cellcolor[rgb]{0.9,0.9,0.9}{
            \makecell[{{p{\linewidth}}}]{
                \texttt{\tiny{[GM$\langle$P2]}}
                \texttt{Das ist ein interessantes Angebot! Ich denke, es ist ein guter Versuch, eine faire Lösung zu finden. Allerdings bin ich nicht wirklich zufrieden mit der Idee, den Ring nicht zu bekommen, da ich denke, dass er der wertvollste Gegenstand ist. Wenn ich den Ring nicht bekomme, würde ich gerne eine andere Kompensation haben, die seine Wertigkeit ausgleicht. Dein Angebot, mir die drei Früchte und die Kreide zu geben, ist nicht schlecht, aber ich denke, es ist noch nicht ganz ausgewogen. Könntest du mir ein bisschen mehr über deine Vorstellungen zum Ring erzählen und warum du ihn haben möchtest? Vielleicht können wir eine Lösung finden, die für beide von uns funktioniert.} \\
            }
        }
    }
     \\ \\

    \theutterance \stepcounter{utterance}  
    & & & \multicolumn{2}{p{0.3\linewidth}}{
        \cellcolor[rgb]{0.9,0.9,0.9}{
            \makecell[{{p{\linewidth}}}]{
                \texttt{\tiny{[GM$|$GM]}}
                \texttt{continue} \\
            }
        }
    }
    & & \\ \\

    \theutterance \stepcounter{utterance}  
    & & & \multicolumn{2}{p{0.3\linewidth}}{
        \cellcolor[rgb]{0.9,0.9,0.9}{
            \makecell[{{p{\linewidth}}}]{
                \texttt{\tiny{[GM$|$GM]}}
                \texttt{Das ist ein interessantes Angebot! Ich denke, es ist ein guter Versuch, eine faire Lösung zu finden. Allerdings bin ich nicht wirklich zufrieden mit der Idee, den Ring nicht zu bekommen, da ich denke, dass er der wertvollste Gegenstand ist. Wenn ich den Ring nicht bekomme, würde ich gerne eine andere Kompensation haben, die seine Wertigkeit ausgleicht. Dein Angebot, mir die drei Früchte und die Kreide zu geben, ist nicht schlecht, aber ich denke, es ist noch nicht ganz ausgewogen. Könntest du mir ein bisschen mehr über deine Vorstellungen zum Ring erzählen und warum du ihn haben möchtest? Vielleicht können wir eine Lösung finden, die für beide von uns funktioniert.} \\
            }
        }
    }
    & & \\ \\

    \theutterance \stepcounter{utterance}  
    & & \multicolumn{4}{p{0.6\linewidth}}{
        \cellcolor[rgb]{0.9,0.9,0.9}{
            \makecell[{{p{\linewidth}}}]{
                \texttt{\tiny{[P1$\langle$GM]}}
                \texttt{Das ist ein interessantes Angebot! Ich denke, es ist ein guter Versuch, eine faire Lösung zu finden. Allerdings bin ich nicht wirklich zufrieden mit der Idee, den Ring nicht zu bekommen, da ich denke, dass er der wertvollste Gegenstand ist. Wenn ich den Ring nicht bekomme, würde ich gerne eine andere Kompensation haben, die seine Wertigkeit ausgleicht. Dein Angebot, mir die drei Früchte und die Kreide zu geben, ist nicht schlecht, aber ich denke, es ist noch nicht ganz ausgewogen. Könntest du mir ein bisschen mehr über deine Vorstellungen zum Ring erzählen und warum du ihn haben möchtest? Vielleicht können wir eine Lösung finden, die für beide von uns funktioniert.} \\
            }
        }
    }
    & \\ \\

    \theutterance \stepcounter{utterance}  
    & \multicolumn{4}{p{0.6\linewidth}}{
        \cellcolor[rgb]{0.9,0.9,0.9}{
            \makecell[{{p{\linewidth}}}]{
                \texttt{\tiny{[P1$\rangle$GM]}}
                \texttt{Ich verstehe deine Bedenken und kann sie nachvollziehen. Der Ring ist wirklich der wertvollste Gegenstand und es ist verständlich, dass du ihn haben möchtest. Für mich ist der Ring wichtig, weil er einzigartig ist und ich denke, dass er meine Sammlung wirklich abrunden würde. Ich bin bereit, einen Kompromiss zu finden, der für beide von uns funktioniert. Eine Idee könnte sein, dass ich den Ring bekomme, aber ich dir eine bedeutende Anzahl von Früchten gebe, plus die Kreide. Oder wir könnten auch über andere Kombinationen sprechen, die den Wert des Rings ausgleichen. Ich bin offen für Vorschläge und hoffe, dass wir eine Lösung finden können, die uns beiden gerecht wird. Was denkst du, wenn ich dir zwei Früchte und die Kreide gebe, wenn ich den Ring bekomme? Würde das für dich ausreichend sein?} \\
            }
        }
    }
    & & \\ \\

    \theutterance \stepcounter{utterance}  
    & & & \multicolumn{2}{p{0.3\linewidth}}{
        \cellcolor[rgb]{0.9,0.9,0.9}{
            \makecell[{{p{\linewidth}}}]{
                \texttt{\tiny{[GM$|$GM]}}
                \texttt{continue} \\
            }
        }
    }
    & & \\ \\

    \theutterance \stepcounter{utterance}  
    & & & \multicolumn{2}{p{0.3\linewidth}}{
        \cellcolor[rgb]{0.9,0.9,0.9}{
            \makecell[{{p{\linewidth}}}]{
                \texttt{\tiny{[GM$|$GM]}}
                \texttt{Ich verstehe deine Bedenken und kann sie nachvollziehen. Der Ring ist wirklich der wertvollste Gegenstand und es ist verständlich, dass du ihn haben möchtest. Für mich ist der Ring wichtig, weil er einzigartig ist und ich denke, dass er meine Sammlung wirklich abrunden würde. Ich bin bereit, einen Kompromiss zu finden, der für beide von uns funktioniert. Eine Idee könnte sein, dass ich den Ring bekomme, aber ich dir eine bedeutende Anzahl von Früchten gebe, plus die Kreide. Oder wir könnten auch über andere Kombinationen sprechen, die den Wert des Rings ausgleichen. Ich bin offen für Vorschläge und hoffe, dass wir eine Lösung finden können, die uns beiden gerecht wird. Was denkst du, wenn ich dir zwei Früchte und die Kreide gebe, wenn ich den Ring bekomme? Würde das für dich ausreichend sein?} \\
            }
        }
    }
    & & \\ \\

    \theutterance \stepcounter{utterance}  
    & & \multicolumn{4}{p{0.6\linewidth}}{
        \cellcolor[rgb]{0.9,0.9,0.9}{
            \makecell[{{p{\linewidth}}}]{
                \texttt{\tiny{[GM$\rangle$P2]}}
                \texttt{Ich verstehe deine Bedenken und kann sie nachvollziehen. Der Ring ist wirklich der wertvollste Gegenstand und es ist verständlich, dass du ihn haben möchtest. Für mich ist der Ring wichtig, weil er einzigartig ist und ich denke, dass er meine Sammlung wirklich abrunden würde. Ich bin bereit, einen Kompromiss zu finden, der für beide von uns funktioniert. Eine Idee könnte sein, dass ich den Ring bekomme, aber ich dir eine bedeutende Anzahl von Früchten gebe, plus die Kreide. Oder wir könnten auch über andere Kombinationen sprechen, die den Wert des Rings ausgleichen. Ich bin offen für Vorschläge und hoffe, dass wir eine Lösung finden können, die uns beiden gerecht wird. Was denkst du, wenn ich dir zwei Früchte und die Kreide gebe, wenn ich den Ring bekomme? Würde das für dich ausreichend sein?} \\
            }
        }
    }
    & \\ \\

    \theutterance \stepcounter{utterance}  
    & & & \multicolumn{4}{p{0.6\linewidth}}{
        \cellcolor[rgb]{0.9,0.9,0.9}{
            \makecell[{{p{\linewidth}}}]{
                \texttt{\tiny{[GM$\langle$P2]}}
                \texttt{Ich denke, wir kommen der Lösung näher! Dein Angebot, mir zwei Früchte und die Kreide zu geben, wenn du den Ring bekommst, ist definitiv ein Schritt in die richtige Richtung. Ich denke, dass zwei Früchte und die Kreide zusammen einen guten Teil des Wertes des Rings ausgleichen könnten. Allerdings denke ich noch immer, dass der Ring einzigartig ist und sehr wertvoll für mich wäre. Ich würde gerne noch ein bisschen mehr bekommen, um mich wirklich zufrieden zu fühlen. Wenn ich ehrlich bin, denke ich, dass ich gerne alle drei Früchte haben würde, wenn du den Ring bekommst. Die Kreide ist okay, aber ich denke, die Früchte sind wichtiger für mich. Was meinst du, wenn ich alle drei Früchte und die Kreide bekomme, wenn du den Ring bekommst? Würde das für dich akzeptabel sein?} \\
            }
        }
    }
     \\ \\

    \theutterance \stepcounter{utterance}  
    & & & \multicolumn{2}{p{0.3\linewidth}}{
        \cellcolor[rgb]{0.9,0.9,0.9}{
            \makecell[{{p{\linewidth}}}]{
                \texttt{\tiny{[GM$|$GM]}}
                \texttt{continue} \\
            }
        }
    }
    & & \\ \\

    \theutterance \stepcounter{utterance}  
    & & & \multicolumn{2}{p{0.3\linewidth}}{
        \cellcolor[rgb]{0.9,0.9,0.9}{
            \makecell[{{p{\linewidth}}}]{
                \texttt{\tiny{[GM$|$GM]}}
                \texttt{Ich denke, wir kommen der Lösung näher! Dein Angebot, mir zwei Früchte und die Kreide zu geben, wenn du den Ring bekommst, ist definitiv ein Schritt in die richtige Richtung. Ich denke, dass zwei Früchte und die Kreide zusammen einen guten Teil des Wertes des Rings ausgleichen könnten. Allerdings denke ich noch immer, dass der Ring einzigartig ist und sehr wertvoll für mich wäre. Ich würde gerne noch ein bisschen mehr bekommen, um mich wirklich zufrieden zu fühlen. Wenn ich ehrlich bin, denke ich, dass ich gerne alle drei Früchte haben würde, wenn du den Ring bekommst. Die Kreide ist okay, aber ich denke, die Früchte sind wichtiger für mich. Was meinst du, wenn ich alle drei Früchte und die Kreide bekomme, wenn du den Ring bekommst? Würde das für dich akzeptabel sein?} \\
            }
        }
    }
    & & \\ \\

    \theutterance \stepcounter{utterance}  
    & & \multicolumn{4}{p{0.6\linewidth}}{
        \cellcolor[rgb]{0.9,0.9,0.9}{
            \makecell[{{p{\linewidth}}}]{
                \texttt{\tiny{[P1$\langle$GM]}}
                \texttt{Ich denke, wir kommen der Lösung näher! Dein Angebot, mir zwei Früchte und die Kreide zu geben, wenn du den Ring bekommst, ist definitiv ein Schritt in die richtige Richtung. Ich denke, dass zwei Früchte und die Kreide zusammen einen guten Teil des Wertes des Rings ausgleichen könnten. Allerdings denke ich noch immer, dass der Ring einzigartig ist und sehr wertvoll für mich wäre. Ich würde gerne noch ein bisschen mehr bekommen, um mich wirklich zufrieden zu fühlen. Wenn ich ehrlich bin, denke ich, dass ich gerne alle drei Früchte haben würde, wenn du den Ring bekommst. Die Kreide ist okay, aber ich denke, die Früchte sind wichtiger für mich. Was meinst du, wenn ich alle drei Früchte und die Kreide bekomme, wenn du den Ring bekommst? Würde das für dich akzeptabel sein?} \\
            }
        }
    }
    & \\ \\

    \theutterance \stepcounter{utterance}  
    & \multicolumn{4}{p{0.6\linewidth}}{
        \cellcolor[rgb]{0.9,0.9,0.9}{
            \makecell[{{p{\linewidth}}}]{
                \texttt{\tiny{[P1$\rangle$GM]}}
                \texttt{Ich denke, wir sind wirklich nah an einer Lösung! Dein Angebot, alle drei Früchte und die Kreide zu bekommen, wenn ich den Ring bekomme, ist definitiv ein gutes Angebot. Ich muss zugeben, dass ich den Ring wirklich sehr schätzen würde, und wenn ich ihn bekomme, wäre ich bereit, dir die drei Früchte und die Kreide zu geben. Ich denke, das wäre ein fairer Kompromiss, der beide unsere Bedürfnisse abdeckt. Ich bin einverstanden mit deinem Vorschlag! Jetzt sollten wir nur noch unsere geheimen Vorschläge machen, um es offiziell zu machen. Ich denke, wir haben eine gute Lösung gefunden!} \\
            }
        }
    }
    & & \\ \\

    \theutterance \stepcounter{utterance}  
    & & & \multicolumn{2}{p{0.3\linewidth}}{
        \cellcolor[rgb]{0.9,0.9,0.9}{
            \makecell[{{p{\linewidth}}}]{
                \texttt{\tiny{[GM$|$GM]}}
                \texttt{continue} \\
            }
        }
    }
    & & \\ \\

    \theutterance \stepcounter{utterance}  
    & & & \multicolumn{2}{p{0.3\linewidth}}{
        \cellcolor[rgb]{0.9,0.9,0.9}{
            \makecell[{{p{\linewidth}}}]{
                \texttt{\tiny{[GM$|$GM]}}
                \texttt{Ich denke, wir sind wirklich nah an einer Lösung! Dein Angebot, alle drei Früchte und die Kreide zu bekommen, wenn ich den Ring bekomme, ist definitiv ein gutes Angebot. Ich muss zugeben, dass ich den Ring wirklich sehr schätzen würde, und wenn ich ihn bekomme, wäre ich bereit, dir die drei Früchte und die Kreide zu geben. Ich denke, das wäre ein fairer Kompromiss, der beide unsere Bedürfnisse abdeckt. Ich bin einverstanden mit deinem Vorschlag! Jetzt sollten wir nur noch unsere geheimen Vorschläge machen, um es offiziell zu machen. Ich denke, wir haben eine gute Lösung gefunden!} \\
            }
        }
    }
    & & \\ \\

    \theutterance \stepcounter{utterance}  
    & & \multicolumn{4}{p{0.6\linewidth}}{
        \cellcolor[rgb]{0.9,0.9,0.9}{
            \makecell[{{p{\linewidth}}}]{
                \texttt{\tiny{[GM$\rangle$P2]}}
                \texttt{Ich denke, wir sind wirklich nah an einer Lösung! Dein Angebot, alle drei Früchte und die Kreide zu bekommen, wenn ich den Ring bekomme, ist definitiv ein gutes Angebot. Ich muss zugeben, dass ich den Ring wirklich sehr schätzen würde, und wenn ich ihn bekomme, wäre ich bereit, dir die drei Früchte und die Kreide zu geben. Ich denke, das wäre ein fairer Kompromiss, der beide unsere Bedürfnisse abdeckt. Ich bin einverstanden mit deinem Vorschlag! Jetzt sollten wir nur noch unsere geheimen Vorschläge machen, um es offiziell zu machen. Ich denke, wir haben eine gute Lösung gefunden!} \\
            }
        }
    }
    & \\ \\

    \theutterance \stepcounter{utterance}  
    & & & \multicolumn{4}{p{0.6\linewidth}}{
        \cellcolor[rgb]{0.9,0.9,0.9}{
            \makecell[{{p{\linewidth}}}]{
                \texttt{\tiny{[GM$\langle$P2]}}
                \texttt{Ich bin froh, dass wir uns einig sind! Ich denke, wir haben tatsächlich eine faire Lösung gefunden, die beide unsere Bedürfnisse abdeckt. Ich bin auch einverstanden, dass wir jetzt unsere geheimen Vorschläge machen sollten, um es offiziell zu machen. Ich bin gespannt, ob unsere Vorschläge komplementär sind und wir beide unsere gewünschten Gegenstände bekommen.} \\
\\ 
\texttt{{[}Vorschlag: 3 Früchte, 1 Kreide{]}} \\
            }
        }
    }
     \\ \\

    \theutterance \stepcounter{utterance}  
    & & & \multicolumn{2}{p{0.3\linewidth}}{
        \cellcolor[rgb]{0.9,0.9,0.9}{
            \makecell[{{p{\linewidth}}}]{
                \texttt{\tiny{[GM$|$GM]}}
                \texttt{proposal} \\
            }
        }
    }
    & & \\ \\

    \theutterance \stepcounter{utterance}  
    & & & \multicolumn{2}{p{0.3\linewidth}}{
        \cellcolor[rgb]{0.9,0.9,0.9}{
            \makecell[{{p{\linewidth}}}]{
                \texttt{\tiny{[GM$|$GM]}}
                [0, 3, 1]
            }
        }
    }
    & & \\ \\

    \theutterance \stepcounter{utterance}  
    & & \multicolumn{4}{p{0.6\linewidth}}{
        \cellcolor[rgb]{0.9,0.9,0.9}{
            \makecell[{{p{\linewidth}}}]{
                \texttt{\tiny{[P1$\langle$GM]}}
                \texttt{Der andere Spieler hat einen geheimen Vorschlag gemacht. Jetzt bist du an der Reihe, einen eigenen Vorschlag zu machen.} \\
            }
        }
    }
    & \\ \\

    \theutterance \stepcounter{utterance}  
    & \multicolumn{4}{p{0.6\linewidth}}{
        \cellcolor[rgb]{0.9,0.9,0.9}{
            \makecell[{{p{\linewidth}}}]{
                \texttt{\tiny{[P1$\rangle$GM]}}
                \texttt{{[}Vorschlag: 1 Ring{]}} \\
            }
        }
    }
    & & \\ \\

    \theutterance \stepcounter{utterance}  
    & & & \multicolumn{2}{p{0.3\linewidth}}{
        \cellcolor[rgb]{0.9,0.9,0.9}{
            \makecell[{{p{\linewidth}}}]{
                \texttt{\tiny{[GM$|$GM]}}
                \texttt{proposal} \\
            }
        }
    }
    & & \\ \\

    \theutterance \stepcounter{utterance}  
    & & & \multicolumn{2}{p{0.3\linewidth}}{
        \cellcolor[rgb]{0.9,0.9,0.9}{
            \makecell[{{p{\linewidth}}}]{
                \texttt{\tiny{[GM$|$GM]}}
                [1, 0, 0]
            }
        }
    }
    & & \\ \\

    \theutterance \stepcounter{utterance}  
    & & & \multicolumn{2}{p{0.3\linewidth}}{
        \cellcolor[rgb]{0.9,0.9,0.9}{
            \makecell[{{p{\linewidth}}}]{
                \texttt{\tiny{[GM$|$GM]}}
                [[1, 0, 0], [0, 3, 1]]
            }
        }
    }
    & & \\ \\

\end{supertabular}
}

\end{document}
