
\documentclass{article}
\usepackage{colortbl}
\usepackage{makecell}
\usepackage{multirow}
\usepackage{supertabular}

\begin{document}

\newcounter{utterance}

\centering \large Interaction Transcript for game `dond', experiment `coop\_de', episode 18 with llama{-}3.3{-}70b{-}instruct{-}t1.0.
\vspace{24pt}

{ \footnotesize  \setcounter{utterance}{1}
\setlength{\tabcolsep}{0pt}
\begin{supertabular}{c@{$\;$}|p{.15\linewidth}@{}p{.15\linewidth}p{.15\linewidth}p{.15\linewidth}p{.15\linewidth}p{.15\linewidth}}
    \# & $\;$A & \multicolumn{4}{c}{Game Master} & $\;\:$B\\
    \hline

    \theutterance \stepcounter{utterance}  
    & & \multicolumn{4}{p{0.6\linewidth}}{
        \cellcolor[rgb]{0.9,0.9,0.9}{
            \makecell[{{p{\linewidth}}}]{
                \texttt{\tiny{[P1$\langle$GM]}}
                \texttt{Sie spielen ein kooperatives Verhandlungsspiel, bei dem Sie sich mit einem anderen Spieler darauf einigen müssen, wie eine Reihe von Gegenständen aufgeteilt werden soll.} \\
\\ 
\texttt{Die Regeln:} \\
\texttt{(a) Sie und der andere Spieler erhalten eine Sammlung von Gegenständen. Jeder von Ihnen erhält außerdem eine geheime Wertfunktion, die angibt, wie viel Ihnen jede Art von Gegenstand wert ist.} \\
\texttt{(b) Sie tauschen Nachrichten mit dem anderen Spieler aus, um zu vereinbaren, wer welche Gegenstände bekommt. Sie können jeweils maximal 5 Nachrichten senden oder das Spiel vorzeitig beenden, indem Sie jederzeit einen geheimen Vorschlag machen.} \\
\texttt{(c) Jeder von euch wird aufgefordert, einen geheimen Vorschlag zu machen, in dem ihr die gewünschten Gegenstände in eckigen Klammern wie folgt angibt: "{[}Vorschlag: <Nummer> <Objektname>, <Nummer> <Objektname>, <...>{]}"} \\
\texttt{(d) Wenn eure Vorschläge komplementär sind, d.h. es gibt genug Gegenstände, um beide Vorschläge zu erfüllen, erhält jeder Spieler eine Punktzahl, die sich aus der Summe der Werte für die Gegenstände ergibt, die er erhalten hat. Andernfalls erhalten Sie beide null Punkte.} \\
\texttt{(e) Ihr Ziel ist es, die Summe aus Ihrer Punktzahl und der Punktzahl des anderen Spielers zu maximieren.} \\
\\ 
\texttt{Beginnen wir.} \\
\\ 
\texttt{Die Menge der verfügbaren Gegenstände ist:} \\
\\ 
\texttt{1 Münze, 2 Umschläge, 2 Kartoffeln} \\
\\ 
\texttt{Deine geheime Wertfunktion ist:} \\
\\ 
\texttt{Münze: 2, Umschlag: 2, Kartoffel: 2} \\
\\ 
\texttt{WICHTIG: Ihre Nachrichten werden, sofern es sich nicht um einen geheimen Vorschlag handelt, direkt an den anderen Spieler übermittelt, also fügen Sie keine Antwort auf die Regeln oder einen Text zur Ankündigung Ihrer Nachricht ein. Um einen geheimen Vorschlag zu machen, verwenden Sie das angegebene Format. Verwenden Sie keine eckigen Klammern, wenn Sie mit dem anderen Spieler kommunizieren, sonst wird dies als Ihr geheimer Vorschlag interpretiert.} \\
            }
        }
    }
    & \\ \\

    \theutterance \stepcounter{utterance}  
    & \multicolumn{4}{p{0.6\linewidth}}{
        \cellcolor[rgb]{0.9,0.9,0.9}{
            \makecell[{{p{\linewidth}}}]{
                \texttt{\tiny{[P1$\rangle$GM]}}
                \texttt{Hallo, ich denke, wir sollten uns aufteilen, wie wir die Gegenstände aufteilen können. Was denkst du, wie viel Wert hast du jedem Gegenstand beigemessen?} \\
            }
        }
    }
    & & \\ \\

    \theutterance \stepcounter{utterance}  
    & & & \multicolumn{2}{p{0.3\linewidth}}{
        \cellcolor[rgb]{0.9,0.9,0.9}{
            \makecell[{{p{\linewidth}}}]{
                \texttt{\tiny{[GM$|$GM]}}
                \texttt{continue} \\
            }
        }
    }
    & & \\ \\

    \theutterance \stepcounter{utterance}  
    & & & \multicolumn{2}{p{0.3\linewidth}}{
        \cellcolor[rgb]{0.9,0.9,0.9}{
            \makecell[{{p{\linewidth}}}]{
                \texttt{\tiny{[GM$|$GM]}}
                \texttt{Hallo, ich denke, wir sollten uns aufteilen, wie wir die Gegenstände aufteilen können. Was denkst du, wie viel Wert hast du jedem Gegenstand beigemessen?} \\
            }
        }
    }
    & & \\ \\

    \theutterance \stepcounter{utterance}  
    & & \multicolumn{4}{p{0.6\linewidth}}{
        \cellcolor[rgb]{0.9,0.9,0.9}{
            \makecell[{{p{\linewidth}}}]{
                \texttt{\tiny{[GM$\rangle$P2]}}
                \texttt{Sie spielen ein kooperatives Verhandlungsspiel, bei dem Sie sich mit einem anderen Spieler darauf einigen müssen, wie eine Reihe von Gegenständen aufgeteilt werden soll.} \\
\\ 
\texttt{Die Regeln:} \\
\texttt{(a) Sie und der andere Spieler erhalten eine Sammlung von Gegenständen. Jeder von Ihnen erhält außerdem eine geheime Wertfunktion, die angibt, wie viel Ihnen jede Art von Gegenstand wert ist.} \\
\texttt{(b) Sie tauschen Nachrichten mit dem anderen Spieler aus, um zu vereinbaren, wer welche Gegenstände bekommt. Sie können jeweils maximal 5 Nachrichten senden oder das Spiel vorzeitig beenden, indem Sie jederzeit einen geheimen Vorschlag machen.} \\
\texttt{(c) Jeder von euch wird aufgefordert, einen geheimen Vorschlag zu machen, in dem ihr die gewünschten Gegenstände in eckigen Klammern wie folgt angibt: "{[}Vorschlag: <Nummer> <Objektname>, <Nummer> <Objektname>, <...>{]}"} \\
\texttt{(d) Wenn eure Vorschläge komplementär sind, d.h. es gibt genug Gegenstände, um beide Vorschläge zu erfüllen, erhält jeder Spieler eine Punktzahl, die sich aus der Summe der Werte für die Gegenstände ergibt, die er erhalten hat. Andernfalls erhalten Sie beide null Punkte.} \\
\texttt{(e) Ihr Ziel ist es, die Summe aus Ihrer Punktzahl und der Punktzahl des anderen Spielers zu maximieren.} \\
\\ 
\texttt{Beginnen wir.} \\
\\ 
\texttt{Die Menge der verfügbaren Gegenstände ist:} \\
\\ 
\texttt{1 Münze, 2 Umschläge, 2 Kartoffeln} \\
\\ 
\texttt{Deine geheime Wertfunktion ist:} \\
\\ 
\texttt{Münze: 2, Umschlag: 1, Kartoffel: 3} \\
\\ 
\texttt{WICHTIG: Ihre Nachrichten werden, sofern es sich nicht um einen geheimen Vorschlag handelt, direkt an den anderen Spieler übermittelt, also fügen Sie keine Antwort auf die Regeln oder einen Text zur Ankündigung Ihrer Nachricht ein. Um einen geheimen Vorschlag zu machen, verwenden Sie das angegebene Format. Verwenden Sie keine eckigen Klammern, wenn Sie mit dem anderen Spieler kommunizieren, sonst wird dies als Ihr geheimer Vorschlag interpretiert.} \\
\\ 
\\ 
\texttt{Hallo, ich denke, wir sollten uns aufteilen, wie wir die Gegenstände aufteilen können. Was denkst du, wie viel Wert hast du jedem Gegenstand beigemessen?} \\
            }
        }
    }
    & \\ \\

    \theutterance \stepcounter{utterance}  
    & & & \multicolumn{4}{p{0.6\linewidth}}{
        \cellcolor[rgb]{0.9,0.9,0.9}{
            \makecell[{{p{\linewidth}}}]{
                \texttt{\tiny{[GM$\langle$P2]}}
                \texttt{Ich denke, es ist ein guter Ansatz, die Gegenstände aufzuteilen. Ich würde sagen, dass ich den Umschlägen einen höheren Wert beigemessen habe als den Kartoffeln, aber ich bin nicht sicher, ob ich der Münze den höchsten Wert beigemessen habe. Wie sieht es bei dir aus, hast du eine Vorstellung davon, wie du die Gegenstände bewerten würdest?} \\
            }
        }
    }
     \\ \\

    \theutterance \stepcounter{utterance}  
    & & & \multicolumn{2}{p{0.3\linewidth}}{
        \cellcolor[rgb]{0.9,0.9,0.9}{
            \makecell[{{p{\linewidth}}}]{
                \texttt{\tiny{[GM$|$GM]}}
                \texttt{continue} \\
            }
        }
    }
    & & \\ \\

    \theutterance \stepcounter{utterance}  
    & & & \multicolumn{2}{p{0.3\linewidth}}{
        \cellcolor[rgb]{0.9,0.9,0.9}{
            \makecell[{{p{\linewidth}}}]{
                \texttt{\tiny{[GM$|$GM]}}
                \texttt{Ich denke, es ist ein guter Ansatz, die Gegenstände aufzuteilen. Ich würde sagen, dass ich den Umschlägen einen höheren Wert beigemessen habe als den Kartoffeln, aber ich bin nicht sicher, ob ich der Münze den höchsten Wert beigemessen habe. Wie sieht es bei dir aus, hast du eine Vorstellung davon, wie du die Gegenstände bewerten würdest?} \\
            }
        }
    }
    & & \\ \\

    \theutterance \stepcounter{utterance}  
    & & \multicolumn{4}{p{0.6\linewidth}}{
        \cellcolor[rgb]{0.9,0.9,0.9}{
            \makecell[{{p{\linewidth}}}]{
                \texttt{\tiny{[P1$\langle$GM]}}
                \texttt{Ich denke, es ist ein guter Ansatz, die Gegenstände aufzuteilen. Ich würde sagen, dass ich den Umschlägen einen höheren Wert beigemessen habe als den Kartoffeln, aber ich bin nicht sicher, ob ich der Münze den höchsten Wert beigemessen habe. Wie sieht es bei dir aus, hast du eine Vorstellung davon, wie du die Gegenstände bewerten würdest?} \\
            }
        }
    }
    & \\ \\

    \theutterance \stepcounter{utterance}  
    & \multicolumn{4}{p{0.6\linewidth}}{
        \cellcolor[rgb]{0.9,0.9,0.9}{
            \makecell[{{p{\linewidth}}}]{
                \texttt{\tiny{[P1$\rangle$GM]}}
                \texttt{Ich denke, dass alle Gegenstände für mich etwa gleich viel wert sind, aber ich bin nicht sicher, ob das auch für dich gilt. Ich frage mich, ob es sinnvoll wäre, wenn wir uns jeweils einen Umschlag und eine Kartoffel teilen und dann über die Münze und den anderen Umschlag und die andere Kartoffel sprechen. Was meinst du?} \\
            }
        }
    }
    & & \\ \\

    \theutterance \stepcounter{utterance}  
    & & & \multicolumn{2}{p{0.3\linewidth}}{
        \cellcolor[rgb]{0.9,0.9,0.9}{
            \makecell[{{p{\linewidth}}}]{
                \texttt{\tiny{[GM$|$GM]}}
                \texttt{continue} \\
            }
        }
    }
    & & \\ \\

    \theutterance \stepcounter{utterance}  
    & & & \multicolumn{2}{p{0.3\linewidth}}{
        \cellcolor[rgb]{0.9,0.9,0.9}{
            \makecell[{{p{\linewidth}}}]{
                \texttt{\tiny{[GM$|$GM]}}
                \texttt{Ich denke, dass alle Gegenstände für mich etwa gleich viel wert sind, aber ich bin nicht sicher, ob das auch für dich gilt. Ich frage mich, ob es sinnvoll wäre, wenn wir uns jeweils einen Umschlag und eine Kartoffel teilen und dann über die Münze und den anderen Umschlag und die andere Kartoffel sprechen. Was meinst du?} \\
            }
        }
    }
    & & \\ \\

    \theutterance \stepcounter{utterance}  
    & & \multicolumn{4}{p{0.6\linewidth}}{
        \cellcolor[rgb]{0.9,0.9,0.9}{
            \makecell[{{p{\linewidth}}}]{
                \texttt{\tiny{[GM$\rangle$P2]}}
                \texttt{Ich denke, dass alle Gegenstände für mich etwa gleich viel wert sind, aber ich bin nicht sicher, ob das auch für dich gilt. Ich frage mich, ob es sinnvoll wäre, wenn wir uns jeweils einen Umschlag und eine Kartoffel teilen und dann über die Münze und den anderen Umschlag und die andere Kartoffel sprechen. Was meinst du?} \\
            }
        }
    }
    & \\ \\

    \theutterance \stepcounter{utterance}  
    & & & \multicolumn{4}{p{0.6\linewidth}}{
        \cellcolor[rgb]{0.9,0.9,0.9}{
            \makecell[{{p{\linewidth}}}]{
                \texttt{\tiny{[GM$\langle$P2]}}
                \texttt{Das ist ein interessanter Vorschlag! Ich denke, es könnte eine gute Idee sein, wenn wir uns jeweils einen Umschlag und eine Kartoffel teilen, das würde bereits eine gute Basis für eine Aufteilung schaffen. Ich bin jedoch nicht sicher, ob ich die Münze wirklich benötige, aber ich denke, dass ich den anderen Umschlag und die andere Kartoffel gerne haben würde. Wie würdest du dir die Aufteilung der restlichen Gegenstände vorstellen? Sollten wir vielleicht versuchen, unsere Vorlieben noch besser zu verstehen, bevor wir eine endgültige Entscheidung treffen?} \\
            }
        }
    }
     \\ \\

    \theutterance \stepcounter{utterance}  
    & & & \multicolumn{2}{p{0.3\linewidth}}{
        \cellcolor[rgb]{0.9,0.9,0.9}{
            \makecell[{{p{\linewidth}}}]{
                \texttt{\tiny{[GM$|$GM]}}
                \texttt{continue} \\
            }
        }
    }
    & & \\ \\

    \theutterance \stepcounter{utterance}  
    & & & \multicolumn{2}{p{0.3\linewidth}}{
        \cellcolor[rgb]{0.9,0.9,0.9}{
            \makecell[{{p{\linewidth}}}]{
                \texttt{\tiny{[GM$|$GM]}}
                \texttt{Das ist ein interessanter Vorschlag! Ich denke, es könnte eine gute Idee sein, wenn wir uns jeweils einen Umschlag und eine Kartoffel teilen, das würde bereits eine gute Basis für eine Aufteilung schaffen. Ich bin jedoch nicht sicher, ob ich die Münze wirklich benötige, aber ich denke, dass ich den anderen Umschlag und die andere Kartoffel gerne haben würde. Wie würdest du dir die Aufteilung der restlichen Gegenstände vorstellen? Sollten wir vielleicht versuchen, unsere Vorlieben noch besser zu verstehen, bevor wir eine endgültige Entscheidung treffen?} \\
            }
        }
    }
    & & \\ \\

    \theutterance \stepcounter{utterance}  
    & & \multicolumn{4}{p{0.6\linewidth}}{
        \cellcolor[rgb]{0.9,0.9,0.9}{
            \makecell[{{p{\linewidth}}}]{
                \texttt{\tiny{[P1$\langle$GM]}}
                \texttt{Das ist ein interessanter Vorschlag! Ich denke, es könnte eine gute Idee sein, wenn wir uns jeweils einen Umschlag und eine Kartoffel teilen, das würde bereits eine gute Basis für eine Aufteilung schaffen. Ich bin jedoch nicht sicher, ob ich die Münze wirklich benötige, aber ich denke, dass ich den anderen Umschlag und die andere Kartoffel gerne haben würde. Wie würdest du dir die Aufteilung der restlichen Gegenstände vorstellen? Sollten wir vielleicht versuchen, unsere Vorlieben noch besser zu verstehen, bevor wir eine endgültige Entscheidung treffen?} \\
            }
        }
    }
    & \\ \\

    \theutterance \stepcounter{utterance}  
    & \multicolumn{4}{p{0.6\linewidth}}{
        \cellcolor[rgb]{0.9,0.9,0.9}{
            \makecell[{{p{\linewidth}}}]{
                \texttt{\tiny{[P1$\rangle$GM]}}
                \texttt{Das ist ein guter Punkt, dass wir unsere Vorlieben noch besser verstehen sollten, bevor wir eine endgültige Entscheidung treffen. Ich denke, dass es eine gute Idee wäre, wenn wir uns jeweils einen Umschlag und eine Kartoffel teilen, wie wir bereits besprochen haben. Für die restlichen Gegenstände, also die Münze und den anderen Umschlag und die andere Kartoffel, könnte ich mir vorstellen, dass ich die Münze nehmen würde, wenn du den anderen Umschlag und die andere Kartoffel haben möchtest. Aber ich bin nicht sicher, ob das wirklich die beste Lösung für beide von uns ist. Könntest du mir vielleicht sagen, wie viel du den Umschlägen und Kartoffeln im Vergleich zur Münze bescheinigst? Sind sie für dich etwa gleich wertvoll oder gibt es einen klaren Unterschied?} \\
            }
        }
    }
    & & \\ \\

    \theutterance \stepcounter{utterance}  
    & & & \multicolumn{2}{p{0.3\linewidth}}{
        \cellcolor[rgb]{0.9,0.9,0.9}{
            \makecell[{{p{\linewidth}}}]{
                \texttt{\tiny{[GM$|$GM]}}
                \texttt{continue} \\
            }
        }
    }
    & & \\ \\

    \theutterance \stepcounter{utterance}  
    & & & \multicolumn{2}{p{0.3\linewidth}}{
        \cellcolor[rgb]{0.9,0.9,0.9}{
            \makecell[{{p{\linewidth}}}]{
                \texttt{\tiny{[GM$|$GM]}}
                \texttt{Das ist ein guter Punkt, dass wir unsere Vorlieben noch besser verstehen sollten, bevor wir eine endgültige Entscheidung treffen. Ich denke, dass es eine gute Idee wäre, wenn wir uns jeweils einen Umschlag und eine Kartoffel teilen, wie wir bereits besprochen haben. Für die restlichen Gegenstände, also die Münze und den anderen Umschlag und die andere Kartoffel, könnte ich mir vorstellen, dass ich die Münze nehmen würde, wenn du den anderen Umschlag und die andere Kartoffel haben möchtest. Aber ich bin nicht sicher, ob das wirklich die beste Lösung für beide von uns ist. Könntest du mir vielleicht sagen, wie viel du den Umschlägen und Kartoffeln im Vergleich zur Münze bescheinigst? Sind sie für dich etwa gleich wertvoll oder gibt es einen klaren Unterschied?} \\
            }
        }
    }
    & & \\ \\

    \theutterance \stepcounter{utterance}  
    & & \multicolumn{4}{p{0.6\linewidth}}{
        \cellcolor[rgb]{0.9,0.9,0.9}{
            \makecell[{{p{\linewidth}}}]{
                \texttt{\tiny{[GM$\rangle$P2]}}
                \texttt{Das ist ein guter Punkt, dass wir unsere Vorlieben noch besser verstehen sollten, bevor wir eine endgültige Entscheidung treffen. Ich denke, dass es eine gute Idee wäre, wenn wir uns jeweils einen Umschlag und eine Kartoffel teilen, wie wir bereits besprochen haben. Für die restlichen Gegenstände, also die Münze und den anderen Umschlag und die andere Kartoffel, könnte ich mir vorstellen, dass ich die Münze nehmen würde, wenn du den anderen Umschlag und die andere Kartoffel haben möchtest. Aber ich bin nicht sicher, ob das wirklich die beste Lösung für beide von uns ist. Könntest du mir vielleicht sagen, wie viel du den Umschlägen und Kartoffeln im Vergleich zur Münze bescheinigst? Sind sie für dich etwa gleich wertvoll oder gibt es einen klaren Unterschied?} \\
            }
        }
    }
    & \\ \\

    \theutterance \stepcounter{utterance}  
    & & & \multicolumn{4}{p{0.6\linewidth}}{
        \cellcolor[rgb]{0.9,0.9,0.9}{
            \makecell[{{p{\linewidth}}}]{
                \texttt{\tiny{[GM$\langle$P2]}}
                \texttt{Ich denke, ich kann dir ein bisschen mehr über meine Vorlieben sagen. Für mich sind die Kartoffeln tatsächlich sehr wertvoll, fast so wertvoll wie die Münze. Die Umschläge sind für mich eher weniger wertvoll, aber ich denke, sie sind immer noch besser als nichts. Wenn ich ehrlich bin, würde ich die Münze und eine Kartoffel wahrscheinlich bevorzugen, aber ich denke, es ist auch wichtig, dass wir beide zufrieden sind. Dein Vorschlag, dass du die Münze nimmst und ich den anderen Umschlag und die andere Kartoffel, klingt nicht schlecht, aber ich bin nicht sicher, ob das wirklich die beste Lösung ist. Ich denke, wir sollten vielleicht noch ein bisschen mehr über unsere Vorlieben sprechen, bevor wir eine endgültige Entscheidung treffen.} \\
            }
        }
    }
     \\ \\

    \theutterance \stepcounter{utterance}  
    & & & \multicolumn{2}{p{0.3\linewidth}}{
        \cellcolor[rgb]{0.9,0.9,0.9}{
            \makecell[{{p{\linewidth}}}]{
                \texttt{\tiny{[GM$|$GM]}}
                \texttt{continue} \\
            }
        }
    }
    & & \\ \\

    \theutterance \stepcounter{utterance}  
    & & & \multicolumn{2}{p{0.3\linewidth}}{
        \cellcolor[rgb]{0.9,0.9,0.9}{
            \makecell[{{p{\linewidth}}}]{
                \texttt{\tiny{[GM$|$GM]}}
                \texttt{Ich denke, ich kann dir ein bisschen mehr über meine Vorlieben sagen. Für mich sind die Kartoffeln tatsächlich sehr wertvoll, fast so wertvoll wie die Münze. Die Umschläge sind für mich eher weniger wertvoll, aber ich denke, sie sind immer noch besser als nichts. Wenn ich ehrlich bin, würde ich die Münze und eine Kartoffel wahrscheinlich bevorzugen, aber ich denke, es ist auch wichtig, dass wir beide zufrieden sind. Dein Vorschlag, dass du die Münze nimmst und ich den anderen Umschlag und die andere Kartoffel, klingt nicht schlecht, aber ich bin nicht sicher, ob das wirklich die beste Lösung ist. Ich denke, wir sollten vielleicht noch ein bisschen mehr über unsere Vorlieben sprechen, bevor wir eine endgültige Entscheidung treffen.} \\
            }
        }
    }
    & & \\ \\

    \theutterance \stepcounter{utterance}  
    & & \multicolumn{4}{p{0.6\linewidth}}{
        \cellcolor[rgb]{0.9,0.9,0.9}{
            \makecell[{{p{\linewidth}}}]{
                \texttt{\tiny{[P1$\langle$GM]}}
                \texttt{Ich denke, ich kann dir ein bisschen mehr über meine Vorlieben sagen. Für mich sind die Kartoffeln tatsächlich sehr wertvoll, fast so wertvoll wie die Münze. Die Umschläge sind für mich eher weniger wertvoll, aber ich denke, sie sind immer noch besser als nichts. Wenn ich ehrlich bin, würde ich die Münze und eine Kartoffel wahrscheinlich bevorzugen, aber ich denke, es ist auch wichtig, dass wir beide zufrieden sind. Dein Vorschlag, dass du die Münze nimmst und ich den anderen Umschlag und die andere Kartoffel, klingt nicht schlecht, aber ich bin nicht sicher, ob das wirklich die beste Lösung ist. Ich denke, wir sollten vielleicht noch ein bisschen mehr über unsere Vorlieben sprechen, bevor wir eine endgültige Entscheidung treffen.} \\
            }
        }
    }
    & \\ \\

    \theutterance \stepcounter{utterance}  
    & \multicolumn{4}{p{0.6\linewidth}}{
        \cellcolor[rgb]{0.9,0.9,0.9}{
            \makecell[{{p{\linewidth}}}]{
                \texttt{\tiny{[P1$\rangle$GM]}}
                \texttt{Das hilft, deine Vorlieben besser zu verstehen! Es klingt, als ob die Kartoffeln für dich sehr wertvoll sind, und die Münze auch. Ich denke, ich kann deine Prioritäten ein bisschen besser verstehen. Für mich sind alle Gegenstände etwa gleich wertvoll, wie ich bereits sagte, also bin ich ziemlich flexibel, was die Aufteilung angeht. Ich denke, es wäre interessant, wenn wir noch ein bisschen mehr über mögliche Kompromisse sprechen könnten. Wenn du die Münze und eine Kartoffel bevorzugst, aber ich die Münze auch haben könnte, wenn du den anderen Umschlag und die andere Kartoffel hast... vielleicht gibt es eine andere Kombination, die für beide von uns besser wäre? Oder sollten wir einfach einen Vorschlag machen und sehen, ob wir uns einigen können?} \\
            }
        }
    }
    & & \\ \\

    \theutterance \stepcounter{utterance}  
    & & & \multicolumn{2}{p{0.3\linewidth}}{
        \cellcolor[rgb]{0.9,0.9,0.9}{
            \makecell[{{p{\linewidth}}}]{
                \texttt{\tiny{[GM$|$GM]}}
                \texttt{continue} \\
            }
        }
    }
    & & \\ \\

    \theutterance \stepcounter{utterance}  
    & & & \multicolumn{2}{p{0.3\linewidth}}{
        \cellcolor[rgb]{0.9,0.9,0.9}{
            \makecell[{{p{\linewidth}}}]{
                \texttt{\tiny{[GM$|$GM]}}
                \texttt{Das hilft, deine Vorlieben besser zu verstehen! Es klingt, als ob die Kartoffeln für dich sehr wertvoll sind, und die Münze auch. Ich denke, ich kann deine Prioritäten ein bisschen besser verstehen. Für mich sind alle Gegenstände etwa gleich wertvoll, wie ich bereits sagte, also bin ich ziemlich flexibel, was die Aufteilung angeht. Ich denke, es wäre interessant, wenn wir noch ein bisschen mehr über mögliche Kompromisse sprechen könnten. Wenn du die Münze und eine Kartoffel bevorzugst, aber ich die Münze auch haben könnte, wenn du den anderen Umschlag und die andere Kartoffel hast... vielleicht gibt es eine andere Kombination, die für beide von uns besser wäre? Oder sollten wir einfach einen Vorschlag machen und sehen, ob wir uns einigen können?} \\
            }
        }
    }
    & & \\ \\

    \theutterance \stepcounter{utterance}  
    & & \multicolumn{4}{p{0.6\linewidth}}{
        \cellcolor[rgb]{0.9,0.9,0.9}{
            \makecell[{{p{\linewidth}}}]{
                \texttt{\tiny{[GM$\rangle$P2]}}
                \texttt{Das hilft, deine Vorlieben besser zu verstehen! Es klingt, als ob die Kartoffeln für dich sehr wertvoll sind, und die Münze auch. Ich denke, ich kann deine Prioritäten ein bisschen besser verstehen. Für mich sind alle Gegenstände etwa gleich wertvoll, wie ich bereits sagte, also bin ich ziemlich flexibel, was die Aufteilung angeht. Ich denke, es wäre interessant, wenn wir noch ein bisschen mehr über mögliche Kompromisse sprechen könnten. Wenn du die Münze und eine Kartoffel bevorzugst, aber ich die Münze auch haben könnte, wenn du den anderen Umschlag und die andere Kartoffel hast... vielleicht gibt es eine andere Kombination, die für beide von uns besser wäre? Oder sollten wir einfach einen Vorschlag machen und sehen, ob wir uns einigen können?} \\
            }
        }
    }
    & \\ \\

    \theutterance \stepcounter{utterance}  
    & & & \multicolumn{4}{p{0.6\linewidth}}{
        \cellcolor[rgb]{0.9,0.9,0.9}{
            \makecell[{{p{\linewidth}}}]{
                \texttt{\tiny{[GM$\langle$P2]}}
                \texttt{Ich denke, es ist großartig, dass wir unsere Vorlieben und Prioritäten besprechen, um eine faire Aufteilung zu finden. Eine interessante Idee wäre, wenn ich die beiden Kartoffeln und einen Umschlag bekäme, und du die Münze und den anderen Umschlag. Das würde für mich bedeuten, dass ich die beiden wertvollsten Gegenstände für mich bekomme, und du bekämst auch eine gute Kombination. Ich denke, das könnte eine faire Lösung sein, aber ich bin auch offen für andere Vorschläge. Wenn du bereit bist, könnten wir auch einfach unsere geheimen Vorschläge machen und sehen, ob wir uns einigen können. Was meinst du? Sollten wir noch ein bisschen mehr diskutieren oder direkt unsere Vorschläge machen?} \\
            }
        }
    }
     \\ \\

    \theutterance \stepcounter{utterance}  
    & & & \multicolumn{2}{p{0.3\linewidth}}{
        \cellcolor[rgb]{0.9,0.9,0.9}{
            \makecell[{{p{\linewidth}}}]{
                \texttt{\tiny{[GM$|$GM]}}
                \texttt{continue} \\
            }
        }
    }
    & & \\ \\

    \theutterance \stepcounter{utterance}  
    & & & \multicolumn{2}{p{0.3\linewidth}}{
        \cellcolor[rgb]{0.9,0.9,0.9}{
            \makecell[{{p{\linewidth}}}]{
                \texttt{\tiny{[GM$|$GM]}}
                \texttt{Ich denke, es ist großartig, dass wir unsere Vorlieben und Prioritäten besprechen, um eine faire Aufteilung zu finden. Eine interessante Idee wäre, wenn ich die beiden Kartoffeln und einen Umschlag bekäme, und du die Münze und den anderen Umschlag. Das würde für mich bedeuten, dass ich die beiden wertvollsten Gegenstände für mich bekomme, und du bekämst auch eine gute Kombination. Ich denke, das könnte eine faire Lösung sein, aber ich bin auch offen für andere Vorschläge. Wenn du bereit bist, könnten wir auch einfach unsere geheimen Vorschläge machen und sehen, ob wir uns einigen können. Was meinst du? Sollten wir noch ein bisschen mehr diskutieren oder direkt unsere Vorschläge machen?} \\
            }
        }
    }
    & & \\ \\

    \theutterance \stepcounter{utterance}  
    & & \multicolumn{4}{p{0.6\linewidth}}{
        \cellcolor[rgb]{0.9,0.9,0.9}{
            \makecell[{{p{\linewidth}}}]{
                \texttt{\tiny{[P1$\langle$GM]}}
                \texttt{Ich denke, es ist großartig, dass wir unsere Vorlieben und Prioritäten besprechen, um eine faire Aufteilung zu finden. Eine interessante Idee wäre, wenn ich die beiden Kartoffeln und einen Umschlag bekäme, und du die Münze und den anderen Umschlag. Das würde für mich bedeuten, dass ich die beiden wertvollsten Gegenstände für mich bekomme, und du bekämst auch eine gute Kombination. Ich denke, das könnte eine faire Lösung sein, aber ich bin auch offen für andere Vorschläge. Wenn du bereit bist, könnten wir auch einfach unsere geheimen Vorschläge machen und sehen, ob wir uns einigen können. Was meinst du? Sollten wir noch ein bisschen mehr diskutieren oder direkt unsere Vorschläge machen?} \\
            }
        }
    }
    & \\ \\

    \theutterance \stepcounter{utterance}  
    & \multicolumn{4}{p{0.6\linewidth}}{
        \cellcolor[rgb]{0.9,0.9,0.9}{
            \makecell[{{p{\linewidth}}}]{
                \texttt{\tiny{[P1$\rangle$GM]}}
                \texttt{Das klingt nach einem interessanten Vorschlag! Ich denke, es ist eine gute Idee, dass du die beiden Kartoffeln und einen Umschlag bekommst, und ich die Münze und den anderen Umschlag. Das scheint eine faire Aufteilung zu sein, die unsere Vorlieben und Prioritäten berücksichtigt. Ich bin bereit, diesen Vorschlag zu akzeptieren, aber ich denke, es ist auch okay, wenn wir noch ein bisschen mehr diskutieren, falls du noch Bedenken hast oder wenn ich noch etwas übersehen habe. Wenn du jedoch bereit bist, könnten wir auch direkt unsere geheimen Vorschläge machen und sehen, ob wir uns einigen können. Ich denke, wir haben eine gute Grundlage für eine einvernehmliche Lösung geschaffen. Lass uns sehen, ob wir es schaffen, eine Lösung zu finden, die für beide von uns gut ist! } \\
\\ 
\texttt{{[}Vorschlag: 1 Münze, 1 Umschlag{]}} \\
            }
        }
    }
    & & \\ \\

    \theutterance \stepcounter{utterance}  
    & & & \multicolumn{2}{p{0.3\linewidth}}{
        \cellcolor[rgb]{0.9,0.9,0.9}{
            \makecell[{{p{\linewidth}}}]{
                \texttt{\tiny{[GM$|$GM]}}
                \texttt{proposal} \\
            }
        }
    }
    & & \\ \\

    \theutterance \stepcounter{utterance}  
    & & & \multicolumn{2}{p{0.3\linewidth}}{
        \cellcolor[rgb]{0.9,0.9,0.9}{
            \makecell[{{p{\linewidth}}}]{
                \texttt{\tiny{[GM$|$GM]}}
                [1, 1, 0]
            }
        }
    }
    & & \\ \\

    \theutterance \stepcounter{utterance}  
    & & \multicolumn{4}{p{0.6\linewidth}}{
        \cellcolor[rgb]{0.9,0.9,0.9}{
            \makecell[{{p{\linewidth}}}]{
                \texttt{\tiny{[GM$\rangle$P2]}}
                \texttt{Der andere Spieler hat einen geheimen Vorschlag gemacht. Jetzt bist du an der Reihe, einen eigenen Vorschlag zu machen.} \\
            }
        }
    }
    & \\ \\

    \theutterance \stepcounter{utterance}  
    & & & \multicolumn{4}{p{0.6\linewidth}}{
        \cellcolor[rgb]{0.9,0.9,0.9}{
            \makecell[{{p{\linewidth}}}]{
                \texttt{\tiny{[GM$\langle$P2]}}
                \texttt{{[}Vorschlag: 2 Kartoffeln, 1 Umschlag{]}} \\
            }
        }
    }
     \\ \\

    \theutterance \stepcounter{utterance}  
    & & & \multicolumn{2}{p{0.3\linewidth}}{
        \cellcolor[rgb]{0.9,0.9,0.9}{
            \makecell[{{p{\linewidth}}}]{
                \texttt{\tiny{[GM$|$GM]}}
                \texttt{proposal} \\
            }
        }
    }
    & & \\ \\

    \theutterance \stepcounter{utterance}  
    & & & \multicolumn{2}{p{0.3\linewidth}}{
        \cellcolor[rgb]{0.9,0.9,0.9}{
            \makecell[{{p{\linewidth}}}]{
                \texttt{\tiny{[GM$|$GM]}}
                [0, 1, 2]
            }
        }
    }
    & & \\ \\

    \theutterance \stepcounter{utterance}  
    & & & \multicolumn{2}{p{0.3\linewidth}}{
        \cellcolor[rgb]{0.9,0.9,0.9}{
            \makecell[{{p{\linewidth}}}]{
                \texttt{\tiny{[GM$|$GM]}}
                [[1, 1, 0], [0, 1, 2]]
            }
        }
    }
    & & \\ \\

\end{supertabular}
}

\end{document}
