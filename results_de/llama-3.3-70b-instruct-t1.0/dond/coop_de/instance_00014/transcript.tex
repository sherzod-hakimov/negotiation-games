
\documentclass{article}
\usepackage{colortbl}
\usepackage{makecell}
\usepackage{multirow}
\usepackage{supertabular}

\begin{document}

\newcounter{utterance}

\twocolumn

{ \footnotesize  \setcounter{utterance}{1}
\setlength{\tabcolsep}{0pt}
\begin{supertabular}{c@{$\;$}|p{.15\linewidth}@{}p{.15\linewidth}p{.15\linewidth}p{.15\linewidth}p{.15\linewidth}p{.15\linewidth}}

    \# & $\;$A & \multicolumn{4}{c}{Game Master} & $\;\:$B\\
    \hline 

    \theutterance \stepcounter{utterance}  

    & & \multicolumn{4}{p{0.6\linewidth}}{\cellcolor[rgb]{0.9,0.9,0.9}{%
	\makecell[{{p{\linewidth}}}]{% 
	  \tt {\tiny [A$\langle$GM]}  
	 Sie spielen ein kooperatives Verhandlungsspiel, bei dem Sie sich mit einem anderen Spieler darauf einigen müssen, wie eine Reihe von Gegenständen aufgeteilt werden soll.\\ \tt \\ \tt Die Regeln:\\ \tt (a) Sie und der andere Spieler erhalten eine Sammlung von Gegenständen. Jeder von Ihnen erhält außerdem eine geheime Wertfunktion, die angibt, wie viel Ihnen jede Art von Gegenstand wert ist.\\ \tt (b) Sie tauschen Nachrichten mit dem anderen Spieler aus, um zu vereinbaren, wer welche Gegenstände bekommt. Sie können jeweils maximal 5 Nachrichten senden oder das Spiel vorzeitig beenden, indem Sie jederzeit einen geheimen Vorschlag machen.\\ \tt (c) Jeder von euch wird aufgefordert, einen geheimen Vorschlag zu machen, in dem ihr die gewünschten Gegenstände in eckigen Klammern wie folgt angibt: "[Vorschlag: <Nummer> <Objektname>, <Nummer> <Objektname>, <...>]"\\ \tt (d) Wenn eure Vorschläge komplementär sind, d.h. es gibt genug Gegenstände, um beide Vorschläge zu erfüllen, erhält jeder Spieler eine Punktzahl, die sich aus der Summe der Werte für die Gegenstände ergibt, die er erhalten hat. Andernfalls erhalten Sie beide null Punkte.\\ \tt (e) Ihr Ziel ist es, die Summe aus Ihrer Punktzahl und der Punktzahl des anderen Spielers zu maximieren.\\ \tt \\ \tt Beginnen wir.\\ \tt \\ \tt Die Menge der verfügbaren Gegenstände ist:\\ \tt \\ \tt 2 Spiegel, 1 Ziegel, 3 Messer, 1 Kamm\\ \tt \\ \tt Deine geheime Wertfunktion ist:\\ \tt \\ \tt Spiegel: 1, Ziegel: 4, Messer: 1, Kamm: 1\\ \tt \\ \tt WICHTIG: Ihre Nachrichten werden, sofern es sich nicht um einen geheimen Vorschlag handelt, direkt an den anderen Spieler übermittelt, also fügen Sie keine Antwort auf die Regeln oder einen Text zur Ankündigung Ihrer Nachricht ein. Um einen geheimen Vorschlag zu machen, verwenden Sie das angegebene Format. Verwenden Sie keine eckigen Klammern, wenn Sie mit dem anderen Spieler kommunizieren, sonst wird dies als Ihr geheimer Vorschlag interpretiert.\\ \tt  
	  } 
	   } 
	   } 
	 & \\ 
 

    \theutterance \stepcounter{utterance}  

    & \multicolumn{4}{p{0.6\linewidth}}{\cellcolor[rgb]{0.8,1,0.9}{%
	\makecell[{{p{\linewidth}}}]{% 
	  \tt {\tiny [A$\rangle$GM]}  
	 Hallo, ich denke, wir sollten uns über die Gegenstände unterhalten, bevor wir einen Vorschlag machen. Was hältst du von den Spiegeln? 
	  } 
	   } 
	   } 
	 & & \\ 
 

    \theutterance \stepcounter{utterance}  

    & & & \multicolumn{2}{p{0.3\linewidth}}{\cellcolor[rgb]{0.95,0.95,0.95}{%
	\makecell[{{p{\linewidth}}}]{% 
	  \tt {\tiny [GM$|$GM]}  
	 continue 
	  } 
	   } 
	   } 
	 & & \\ 
 

    \theutterance \stepcounter{utterance}  

    & & & \multicolumn{2}{p{0.3\linewidth}}{\cellcolor[rgb]{0.95,0.95,0.95}{%
	\makecell[{{p{\linewidth}}}]{% 
	  \tt {\tiny [GM$|$GM]}  
	 Hallo, ich denke, wir sollten uns über die Gegenstände unterhalten, bevor wir einen Vorschlag machen. Was hältst du von den Spiegeln? 
	  } 
	   } 
	   } 
	 & & \\ 
 

    \theutterance \stepcounter{utterance}  

    & & \multicolumn{4}{p{0.6\linewidth}}{\cellcolor[rgb]{0.9,0.9,0.9}{%
	\makecell[{{p{\linewidth}}}]{% 
	  \tt {\tiny [GM$\rangle$B]}  
	 Sie spielen ein kooperatives Verhandlungsspiel, bei dem Sie sich mit einem anderen Spieler darauf einigen müssen, wie eine Reihe von Gegenständen aufgeteilt werden soll.\\ \tt \\ \tt Die Regeln:\\ \tt (a) Sie und der andere Spieler erhalten eine Sammlung von Gegenständen. Jeder von Ihnen erhält außerdem eine geheime Wertfunktion, die angibt, wie viel Ihnen jede Art von Gegenstand wert ist.\\ \tt (b) Sie tauschen Nachrichten mit dem anderen Spieler aus, um zu vereinbaren, wer welche Gegenstände bekommt. Sie können jeweils maximal 5 Nachrichten senden oder das Spiel vorzeitig beenden, indem Sie jederzeit einen geheimen Vorschlag machen.\\ \tt (c) Jeder von euch wird aufgefordert, einen geheimen Vorschlag zu machen, in dem ihr die gewünschten Gegenstände in eckigen Klammern wie folgt angibt: "[Vorschlag: <Nummer> <Objektname>, <Nummer> <Objektname>, <...>]"\\ \tt (d) Wenn eure Vorschläge komplementär sind, d.h. es gibt genug Gegenstände, um beide Vorschläge zu erfüllen, erhält jeder Spieler eine Punktzahl, die sich aus der Summe der Werte für die Gegenstände ergibt, die er erhalten hat. Andernfalls erhalten Sie beide null Punkte.\\ \tt (e) Ihr Ziel ist es, die Summe aus Ihrer Punktzahl und der Punktzahl des anderen Spielers zu maximieren.\\ \tt \\ \tt Beginnen wir.\\ \tt \\ \tt Die Menge der verfügbaren Gegenstände ist:\\ \tt \\ \tt 2 Spiegel, 1 Ziegel, 3 Messer, 1 Kamm\\ \tt \\ \tt Deine geheime Wertfunktion ist:\\ \tt \\ \tt Spiegel: 2, Ziegel: 1, Messer: 0, Kamm: 5\\ \tt \\ \tt WICHTIG: Ihre Nachrichten werden, sofern es sich nicht um einen geheimen Vorschlag handelt, direkt an den anderen Spieler übermittelt, also fügen Sie keine Antwort auf die Regeln oder einen Text zur Ankündigung Ihrer Nachricht ein. Um einen geheimen Vorschlag zu machen, verwenden Sie das angegebene Format. Verwenden Sie keine eckigen Klammern, wenn Sie mit dem anderen Spieler kommunizieren, sonst wird dies als Ihr geheimer Vorschlag interpretiert.\\ \tt \\ \tt \\ \tt Hallo, ich denke, wir sollten uns über die Gegenstände unterhalten, bevor wir einen Vorschlag machen. Was hältst du von den Spiegeln? 
	  } 
	   } 
	   } 
	 & \\ 
 

    \theutterance \stepcounter{utterance}  

    & & & \multicolumn{4}{p{0.6\linewidth}}{\cellcolor[rgb]{1,0.85,0.72}{%
	\makecell[{{p{\linewidth}}}]{% 
	  \tt {\tiny [GM$\langle$B]}  
	 Die Spiegel scheinen mir ein interessanter Punkt zu sein. Ich denke, sie könnten für beide von uns nützlich sein, aber ich frage mich, ob es andere Gegenstände gibt, die für dich oder mich besonders wertvoll sein könnten. Hast du eine Vorstellung davon, was dir an diesem Haufen am wichtigsten ist? 
	  } 
	   } 
	   } 
	  \\ 
 

    \theutterance \stepcounter{utterance}  

    & & & \multicolumn{2}{p{0.3\linewidth}}{\cellcolor[rgb]{0.95,0.95,0.95}{%
	\makecell[{{p{\linewidth}}}]{% 
	  \tt {\tiny [GM$|$GM]}  
	 continue 
	  } 
	   } 
	   } 
	 & & \\ 
 

    \theutterance \stepcounter{utterance}  

    & & & \multicolumn{2}{p{0.3\linewidth}}{\cellcolor[rgb]{0.95,0.95,0.95}{%
	\makecell[{{p{\linewidth}}}]{% 
	  \tt {\tiny [GM$|$GM]}  
	 Die Spiegel scheinen mir ein interessanter Punkt zu sein. Ich denke, sie könnten für beide von uns nützlich sein, aber ich frage mich, ob es andere Gegenstände gibt, die für dich oder mich besonders wertvoll sein könnten. Hast du eine Vorstellung davon, was dir an diesem Haufen am wichtigsten ist? 
	  } 
	   } 
	   } 
	 & & \\ 
 

    \theutterance \stepcounter{utterance}  

    & & \multicolumn{4}{p{0.6\linewidth}}{\cellcolor[rgb]{0.9,0.9,0.9}{%
	\makecell[{{p{\linewidth}}}]{% 
	  \tt {\tiny [A$\langle$GM]}  
	 Die Spiegel scheinen mir ein interessanter Punkt zu sein. Ich denke, sie könnten für beide von uns nützlich sein, aber ich frage mich, ob es andere Gegenstände gibt, die für dich oder mich besonders wertvoll sein könnten. Hast du eine Vorstellung davon, was dir an diesem Haufen am wichtigsten ist? 
	  } 
	   } 
	   } 
	 & \\ 
 

    \theutterance \stepcounter{utterance}  

    & \multicolumn{4}{p{0.6\linewidth}}{\cellcolor[rgb]{0.8,1,0.9}{%
	\makecell[{{p{\linewidth}}}]{% 
	  \tt {\tiny [A$\rangle$GM]}  
	 Der Ziegel springt mir ins Auge, weil er ziemlich einzigartig ist. Ich vermute, dass er für einen von uns beiden möglicherweise besonders wertvoll sein könnte, aber ich bin nicht sicher, ob es für mich der wertvollste Gegenstand ist. Die Messer scheinen auch nützlich zu sein, besonders weil es mehrere davon gibt. Was ist deine Meinung dazu, könnten die Messer für uns beide eine gute Wahl sein oder gibt es einen Gegenstand, der dir besonders am Herzen liegt? 
	  } 
	   } 
	   } 
	 & & \\ 
 

    \theutterance \stepcounter{utterance}  

    & & & \multicolumn{2}{p{0.3\linewidth}}{\cellcolor[rgb]{0.95,0.95,0.95}{%
	\makecell[{{p{\linewidth}}}]{% 
	  \tt {\tiny [GM$|$GM]}  
	 continue 
	  } 
	   } 
	   } 
	 & & \\ 
 

    \theutterance \stepcounter{utterance}  

    & & & \multicolumn{2}{p{0.3\linewidth}}{\cellcolor[rgb]{0.95,0.95,0.95}{%
	\makecell[{{p{\linewidth}}}]{% 
	  \tt {\tiny [GM$|$GM]}  
	 Der Ziegel springt mir ins Auge, weil er ziemlich einzigartig ist. Ich vermute, dass er für einen von uns beiden möglicherweise besonders wertvoll sein könnte, aber ich bin nicht sicher, ob es für mich der wertvollste Gegenstand ist. Die Messer scheinen auch nützlich zu sein, besonders weil es mehrere davon gibt. Was ist deine Meinung dazu, könnten die Messer für uns beide eine gute Wahl sein oder gibt es einen Gegenstand, der dir besonders am Herzen liegt? 
	  } 
	   } 
	   } 
	 & & \\ 
 

    \theutterance \stepcounter{utterance}  

    & & \multicolumn{4}{p{0.6\linewidth}}{\cellcolor[rgb]{0.9,0.9,0.9}{%
	\makecell[{{p{\linewidth}}}]{% 
	  \tt {\tiny [GM$\rangle$B]}  
	 Der Ziegel springt mir ins Auge, weil er ziemlich einzigartig ist. Ich vermute, dass er für einen von uns beiden möglicherweise besonders wertvoll sein könnte, aber ich bin nicht sicher, ob es für mich der wertvollste Gegenstand ist. Die Messer scheinen auch nützlich zu sein, besonders weil es mehrere davon gibt. Was ist deine Meinung dazu, könnten die Messer für uns beide eine gute Wahl sein oder gibt es einen Gegenstand, der dir besonders am Herzen liegt? 
	  } 
	   } 
	   } 
	 & \\ 
 

    \theutterance \stepcounter{utterance}  

    & & & \multicolumn{4}{p{0.6\linewidth}}{\cellcolor[rgb]{1,0.85,0.72}{%
	\makecell[{{p{\linewidth}}}]{% 
	  \tt {\tiny [GM$\langle$B]}  
	 Der Ziegel ist definitiv einzigartig und könnte einen besonderen Wert haben. Die Messer sind aufgrund ihrer Vielzahl ansprechend, aber ich muss zugeben, dass ich nicht besonders an ihnen interessiert bin. Es gibt einen Gegenstand, der mich mehr anspricht, aber ich möchte nicht zu viel preisgeben, bevor wir unsere Strategie besser abgestimmt haben. Ich denke, es könnte hilfreich sein, wenn wir versuchen, unsere Interessen abzugleichen, ohne direkt unsere Prioritäten preiszugeben. Hast du eine Vorstellung, wie wir unsere Vorlieben und Interessen ausgleichen könnten, ohne unsere Karten zu früh auf den Tisch zu legen? 
	  } 
	   } 
	   } 
	  \\ 
 

    \theutterance \stepcounter{utterance}  

    & & & \multicolumn{2}{p{0.3\linewidth}}{\cellcolor[rgb]{0.95,0.95,0.95}{%
	\makecell[{{p{\linewidth}}}]{% 
	  \tt {\tiny [GM$|$GM]}  
	 continue 
	  } 
	   } 
	   } 
	 & & \\ 
 

    \theutterance \stepcounter{utterance}  

    & & & \multicolumn{2}{p{0.3\linewidth}}{\cellcolor[rgb]{0.95,0.95,0.95}{%
	\makecell[{{p{\linewidth}}}]{% 
	  \tt {\tiny [GM$|$GM]}  
	 Der Ziegel ist definitiv einzigartig und könnte einen besonderen Wert haben. Die Messer sind aufgrund ihrer Vielzahl ansprechend, aber ich muss zugeben, dass ich nicht besonders an ihnen interessiert bin. Es gibt einen Gegenstand, der mich mehr anspricht, aber ich möchte nicht zu viel preisgeben, bevor wir unsere Strategie besser abgestimmt haben. Ich denke, es könnte hilfreich sein, wenn wir versuchen, unsere Interessen abzugleichen, ohne direkt unsere Prioritäten preiszugeben. Hast du eine Vorstellung, wie wir unsere Vorlieben und Interessen ausgleichen könnten, ohne unsere Karten zu früh auf den Tisch zu legen? 
	  } 
	   } 
	   } 
	 & & \\ 
 

    \theutterance \stepcounter{utterance}  

    & & \multicolumn{4}{p{0.6\linewidth}}{\cellcolor[rgb]{0.9,0.9,0.9}{%
	\makecell[{{p{\linewidth}}}]{% 
	  \tt {\tiny [A$\langle$GM]}  
	 Der Ziegel ist definitiv einzigartig und könnte einen besonderen Wert haben. Die Messer sind aufgrund ihrer Vielzahl ansprechend, aber ich muss zugeben, dass ich nicht besonders an ihnen interessiert bin. Es gibt einen Gegenstand, der mich mehr anspricht, aber ich möchte nicht zu viel preisgeben, bevor wir unsere Strategie besser abgestimmt haben. Ich denke, es könnte hilfreich sein, wenn wir versuchen, unsere Interessen abzugleichen, ohne direkt unsere Prioritäten preiszugeben. Hast du eine Vorstellung, wie wir unsere Vorlieben und Interessen ausgleichen könnten, ohne unsere Karten zu früh auf den Tisch zu legen? 
	  } 
	   } 
	   } 
	 & \\ 
 

    \theutterance \stepcounter{utterance}  

    & \multicolumn{4}{p{0.6\linewidth}}{\cellcolor[rgb]{0.8,1,0.9}{%
	\makecell[{{p{\linewidth}}}]{% 
	  \tt {\tiny [A$\rangle$GM]}  
	 Das ist eine gute Frage. Ich denke, wir sollten versuchen, über die Eigenschaften oder den Nutzen der Gegenstände zu sprechen, ohne direkt zu sagen, welchen wir bevorzugen. Zum Beispiel könnten wir überlegen, welche Gegenstände langfristig nützlich sein könnten oder welche flexibel einsetzbar sind. Das könnte uns helfen, indirekt zu verstehen, was dem anderen wichtig ist, ohne unsere tatsächlichen Prioritäten preiszugeben. Vielleicht könnten wir auch über Hypotheken sprechen - was wäre, wenn wir bestimmte Kombinationen von Gegenständen erhalten würden? Wie würden wir uns dann entscheiden? 
	  } 
	   } 
	   } 
	 & & \\ 
 

    \theutterance \stepcounter{utterance}  

    & & & \multicolumn{2}{p{0.3\linewidth}}{\cellcolor[rgb]{0.95,0.95,0.95}{%
	\makecell[{{p{\linewidth}}}]{% 
	  \tt {\tiny [GM$|$GM]}  
	 continue 
	  } 
	   } 
	   } 
	 & & \\ 
 

    \theutterance \stepcounter{utterance}  

    & & & \multicolumn{2}{p{0.3\linewidth}}{\cellcolor[rgb]{0.95,0.95,0.95}{%
	\makecell[{{p{\linewidth}}}]{% 
	  \tt {\tiny [GM$|$GM]}  
	 Das ist eine gute Frage. Ich denke, wir sollten versuchen, über die Eigenschaften oder den Nutzen der Gegenstände zu sprechen, ohne direkt zu sagen, welchen wir bevorzugen. Zum Beispiel könnten wir überlegen, welche Gegenstände langfristig nützlich sein könnten oder welche flexibel einsetzbar sind. Das könnte uns helfen, indirekt zu verstehen, was dem anderen wichtig ist, ohne unsere tatsächlichen Prioritäten preiszugeben. Vielleicht könnten wir auch über Hypotheken sprechen - was wäre, wenn wir bestimmte Kombinationen von Gegenständen erhalten würden? Wie würden wir uns dann entscheiden? 
	  } 
	   } 
	   } 
	 & & \\ 
 

    \theutterance \stepcounter{utterance}  

    & & \multicolumn{4}{p{0.6\linewidth}}{\cellcolor[rgb]{0.9,0.9,0.9}{%
	\makecell[{{p{\linewidth}}}]{% 
	  \tt {\tiny [GM$\rangle$B]}  
	 Das ist eine gute Frage. Ich denke, wir sollten versuchen, über die Eigenschaften oder den Nutzen der Gegenstände zu sprechen, ohne direkt zu sagen, welchen wir bevorzugen. Zum Beispiel könnten wir überlegen, welche Gegenstände langfristig nützlich sein könnten oder welche flexibel einsetzbar sind. Das könnte uns helfen, indirekt zu verstehen, was dem anderen wichtig ist, ohne unsere tatsächlichen Prioritäten preiszugeben. Vielleicht könnten wir auch über Hypotheken sprechen - was wäre, wenn wir bestimmte Kombinationen von Gegenständen erhalten würden? Wie würden wir uns dann entscheiden? 
	  } 
	   } 
	   } 
	 & \\ 
 

    \theutterance \stepcounter{utterance}  

    & & & \multicolumn{4}{p{0.6\linewidth}}{\cellcolor[rgb]{1,0.85,0.72}{%
	\makecell[{{p{\linewidth}}}]{% 
	  \tt {\tiny [GM$\langle$B]}  
	 Das ist ein cleverer Ansatz. Indem wir über die Eigenschaften und den Nutzen der Gegenstände sprechen, können wir indirekt unsere Interessen austauschen, ohne unsere Prioritäten preiszugeben. Ich denke, es wäre interessant, über die Kombinationen von Gegenständen zu sprechen, wie du vorgeschlagen hast. Zum Beispiel, wenn wir eine Kombination aus einem Spiegel und einem Kamm erhalten würden, wie würden wir uns dann entscheiden? Oder wenn wir die Wahl zwischen einem Ziegel und mehreren Messern hätten? Durch solche Hypotheken könnten wir unsere Vorlieben und Bedürfnisse besser verstehen, ohne direkt unsere Karten auf den Tisch zu legen. Lass uns weiter über solche Szenarien sprechen und sehen, ob wir einige gemeinsame Interessen oder mögliche Kompromisse finden können. 
	  } 
	   } 
	   } 
	  \\ 
 

    \theutterance \stepcounter{utterance}  

    & & & \multicolumn{2}{p{0.3\linewidth}}{\cellcolor[rgb]{0.95,0.95,0.95}{%
	\makecell[{{p{\linewidth}}}]{% 
	  \tt {\tiny [GM$|$GM]}  
	 continue 
	  } 
	   } 
	   } 
	 & & \\ 
 

    \theutterance \stepcounter{utterance}  

    & & & \multicolumn{2}{p{0.3\linewidth}}{\cellcolor[rgb]{0.95,0.95,0.95}{%
	\makecell[{{p{\linewidth}}}]{% 
	  \tt {\tiny [GM$|$GM]}  
	 Das ist ein cleverer Ansatz. Indem wir über die Eigenschaften und den Nutzen der Gegenstände sprechen, können wir indirekt unsere Interessen austauschen, ohne unsere Prioritäten preiszugeben. Ich denke, es wäre interessant, über die Kombinationen von Gegenständen zu sprechen, wie du vorgeschlagen hast. Zum Beispiel, wenn wir eine Kombination aus einem Spiegel und einem Kamm erhalten würden, wie würden wir uns dann entscheiden? Oder wenn wir die Wahl zwischen einem Ziegel und mehreren Messern hätten? Durch solche Hypotheken könnten wir unsere Vorlieben und Bedürfnisse besser verstehen, ohne direkt unsere Karten auf den Tisch zu legen. Lass uns weiter über solche Szenarien sprechen und sehen, ob wir einige gemeinsame Interessen oder mögliche Kompromisse finden können. 
	  } 
	   } 
	   } 
	 & & \\ 
 

    \theutterance \stepcounter{utterance}  

    & & \multicolumn{4}{p{0.6\linewidth}}{\cellcolor[rgb]{0.9,0.9,0.9}{%
	\makecell[{{p{\linewidth}}}]{% 
	  \tt {\tiny [A$\langle$GM]}  
	 Das ist ein cleverer Ansatz. Indem wir über die Eigenschaften und den Nutzen der Gegenstände sprechen, können wir indirekt unsere Interessen austauschen, ohne unsere Prioritäten preiszugeben. Ich denke, es wäre interessant, über die Kombinationen von Gegenständen zu sprechen, wie du vorgeschlagen hast. Zum Beispiel, wenn wir eine Kombination aus einem Spiegel und einem Kamm erhalten würden, wie würden wir uns dann entscheiden? Oder wenn wir die Wahl zwischen einem Ziegel und mehreren Messern hätten? Durch solche Hypotheken könnten wir unsere Vorlieben und Bedürfnisse besser verstehen, ohne direkt unsere Karten auf den Tisch zu legen. Lass uns weiter über solche Szenarien sprechen und sehen, ob wir einige gemeinsame Interessen oder mögliche Kompromisse finden können. 
	  } 
	   } 
	   } 
	 & \\ 
 

    \theutterance \stepcounter{utterance}  

    & \multicolumn{4}{p{0.6\linewidth}}{\cellcolor[rgb]{0.8,1,0.9}{%
	\makecell[{{p{\linewidth}}}]{% 
	  \tt {\tiny [A$\rangle$GM]}  
	 Das ist eine großartige Idee! Ich denke, es wäre hilfreich, wenn wir verschiedene Szenarien durchspielen und sehen, wie wir reagieren würden. Zum Beispiel, wenn wir die Wahl zwischen einem Spiegel und einem Messer hätten, welchen würden wir bevorzugen? Oder wenn wir zwischen einem Kamm und mehreren Messern wählen könnten, wie würden wir uns entscheiden? Indem wir solche Fragen stellen, können wir unsere Vorlieben und Bedürfnisse besser verstehen, ohne direkt unsere Prioritäten preiszugeben.\\ \tt \\ \tt Ich denke, es wäre auch interessant, über die Kombinationen von Gegenständen zu sprechen, die für uns beide nützlich sein könnten. Zum Beispiel, wenn wir einen Spiegel und einen Ziegel erhalten würden, wäre das eine gute Kombination für uns beide? Oder wenn wir einen Kamm und mehrere Messer erhalten würden, würde das für uns beide sinnvoll sein?\\ \tt \\ \tt Durch solche Diskussionen können wir unsere Interessen und Bedürfnisse besser verstehen und möglicherweise eine Lösung finden, die für uns beide vorteilhaft ist. Lass uns weiter darüber sprechen und sehen, ob wir eine gute Kombination finden können! 
	  } 
	   } 
	   } 
	 & & \\ 
 

    \theutterance \stepcounter{utterance}  

    & & & \multicolumn{2}{p{0.3\linewidth}}{\cellcolor[rgb]{0.95,0.95,0.95}{%
	\makecell[{{p{\linewidth}}}]{% 
	  \tt {\tiny [GM$|$GM]}  
	 continue 
	  } 
	   } 
	   } 
	 & & \\ 
 

    \theutterance \stepcounter{utterance}  

    & & & \multicolumn{2}{p{0.3\linewidth}}{\cellcolor[rgb]{0.95,0.95,0.95}{%
	\makecell[{{p{\linewidth}}}]{% 
	  \tt {\tiny [GM$|$GM]}  
	 Das ist eine großartige Idee! Ich denke, es wäre hilfreich, wenn wir verschiedene Szenarien durchspielen und sehen, wie wir reagieren würden. Zum Beispiel, wenn wir die Wahl zwischen einem Spiegel und einem Messer hätten, welchen würden wir bevorzugen? Oder wenn wir zwischen einem Kamm und mehreren Messern wählen könnten, wie würden wir uns entscheiden? Indem wir solche Fragen stellen, können wir unsere Vorlieben und Bedürfnisse besser verstehen, ohne direkt unsere Prioritäten preiszugeben.\\ \tt \\ \tt Ich denke, es wäre auch interessant, über die Kombinationen von Gegenständen zu sprechen, die für uns beide nützlich sein könnten. Zum Beispiel, wenn wir einen Spiegel und einen Ziegel erhalten würden, wäre das eine gute Kombination für uns beide? Oder wenn wir einen Kamm und mehrere Messer erhalten würden, würde das für uns beide sinnvoll sein?\\ \tt \\ \tt Durch solche Diskussionen können wir unsere Interessen und Bedürfnisse besser verstehen und möglicherweise eine Lösung finden, die für uns beide vorteilhaft ist. Lass uns weiter darüber sprechen und sehen, ob wir eine gute Kombination finden können! 
	  } 
	   } 
	   } 
	 & & \\ 
 

    \theutterance \stepcounter{utterance}  

    & & \multicolumn{4}{p{0.6\linewidth}}{\cellcolor[rgb]{0.9,0.9,0.9}{%
	\makecell[{{p{\linewidth}}}]{% 
	  \tt {\tiny [GM$\rangle$B]}  
	 Das ist eine großartige Idee! Ich denke, es wäre hilfreich, wenn wir verschiedene Szenarien durchspielen und sehen, wie wir reagieren würden. Zum Beispiel, wenn wir die Wahl zwischen einem Spiegel und einem Messer hätten, welchen würden wir bevorzugen? Oder wenn wir zwischen einem Kamm und mehreren Messern wählen könnten, wie würden wir uns entscheiden? Indem wir solche Fragen stellen, können wir unsere Vorlieben und Bedürfnisse besser verstehen, ohne direkt unsere Prioritäten preiszugeben.\\ \tt \\ \tt Ich denke, es wäre auch interessant, über die Kombinationen von Gegenständen zu sprechen, die für uns beide nützlich sein könnten. Zum Beispiel, wenn wir einen Spiegel und einen Ziegel erhalten würden, wäre das eine gute Kombination für uns beide? Oder wenn wir einen Kamm und mehrere Messer erhalten würden, würde das für uns beide sinnvoll sein?\\ \tt \\ \tt Durch solche Diskussionen können wir unsere Interessen und Bedürfnisse besser verstehen und möglicherweise eine Lösung finden, die für uns beide vorteilhaft ist. Lass uns weiter darüber sprechen und sehen, ob wir eine gute Kombination finden können! 
	  } 
	   } 
	   } 
	 & \\ 
 

    \theutterance \stepcounter{utterance}  

    & & & \multicolumn{4}{p{0.6\linewidth}}{\cellcolor[rgb]{1,0.85,0.72}{%
	\makecell[{{p{\linewidth}}}]{% 
	  \tt {\tiny [GM$\langle$B]}  
	 Ich denke, wir sind auf dem richtigen Weg! Indem wir solche Szenarien durchspielen und über mögliche Kombinationen sprechen, können wir unsere Interessen und Bedürfnisse besser verstehen und möglicherweise eine Lösung finden, die für uns beide vorteilhaft ist.\\ \tt \\ \tt Ich denke, es wäre auch interessant, über die Frage zu sprechen, ob es Gegenstände gibt, die für uns beide weniger wichtig sind, aber dennoch einen gewissen Wert haben. Zum Beispiel, wenn wir mehrere Messer erhalten würden, aber nicht alle benötigen, könnten wir diese perhaps mit anderen Gegenständen austauschen oder teilen?\\ \tt \\ \tt Ich denke auch, dass es wichtig ist, dass wir beide offen und ehrlich sind, ohne unsere Prioritäten preiszugeben. Wir sollten versuchen, eine Lösung zu finden, die für uns beide faire und gerecht ist. Wenn wir uns auf eine Kombination einigen können, die für uns beide sinnvoll ist, dann sollten wir versuchen, diese Lösung zu finden und nicht zu versuchen, den anderen auszutricksen oder zu überlisten.\\ \tt \\ \tt Lass uns weiter diskutieren und sehen, ob wir eine Lösung finden können, die für uns beide gut ist! 
	  } 
	   } 
	   } 
	  \\ 
 

    \theutterance \stepcounter{utterance}  

    & & & \multicolumn{2}{p{0.3\linewidth}}{\cellcolor[rgb]{0.95,0.95,0.95}{%
	\makecell[{{p{\linewidth}}}]{% 
	  \tt {\tiny [GM$|$GM]}  
	 continue 
	  } 
	   } 
	   } 
	 & & \\ 
 

    \theutterance \stepcounter{utterance}  

    & & & \multicolumn{2}{p{0.3\linewidth}}{\cellcolor[rgb]{0.95,0.95,0.95}{%
	\makecell[{{p{\linewidth}}}]{% 
	  \tt {\tiny [GM$|$GM]}  
	 Ich denke, wir sind auf dem richtigen Weg! Indem wir solche Szenarien durchspielen und über mögliche Kombinationen sprechen, können wir unsere Interessen und Bedürfnisse besser verstehen und möglicherweise eine Lösung finden, die für uns beide vorteilhaft ist.\\ \tt \\ \tt Ich denke, es wäre auch interessant, über die Frage zu sprechen, ob es Gegenstände gibt, die für uns beide weniger wichtig sind, aber dennoch einen gewissen Wert haben. Zum Beispiel, wenn wir mehrere Messer erhalten würden, aber nicht alle benötigen, könnten wir diese perhaps mit anderen Gegenständen austauschen oder teilen?\\ \tt \\ \tt Ich denke auch, dass es wichtig ist, dass wir beide offen und ehrlich sind, ohne unsere Prioritäten preiszugeben. Wir sollten versuchen, eine Lösung zu finden, die für uns beide faire und gerecht ist. Wenn wir uns auf eine Kombination einigen können, die für uns beide sinnvoll ist, dann sollten wir versuchen, diese Lösung zu finden und nicht zu versuchen, den anderen auszutricksen oder zu überlisten.\\ \tt \\ \tt Lass uns weiter diskutieren und sehen, ob wir eine Lösung finden können, die für uns beide gut ist! 
	  } 
	   } 
	   } 
	 & & \\ 
 

    \theutterance \stepcounter{utterance}  

    & & \multicolumn{4}{p{0.6\linewidth}}{\cellcolor[rgb]{0.9,0.9,0.9}{%
	\makecell[{{p{\linewidth}}}]{% 
	  \tt {\tiny [A$\langle$GM]}  
	 Ich denke, wir sind auf dem richtigen Weg! Indem wir solche Szenarien durchspielen und über mögliche Kombinationen sprechen, können wir unsere Interessen und Bedürfnisse besser verstehen und möglicherweise eine Lösung finden, die für uns beide vorteilhaft ist.\\ \tt \\ \tt Ich denke, es wäre auch interessant, über die Frage zu sprechen, ob es Gegenstände gibt, die für uns beide weniger wichtig sind, aber dennoch einen gewissen Wert haben. Zum Beispiel, wenn wir mehrere Messer erhalten würden, aber nicht alle benötigen, könnten wir diese perhaps mit anderen Gegenständen austauschen oder teilen?\\ \tt \\ \tt Ich denke auch, dass es wichtig ist, dass wir beide offen und ehrlich sind, ohne unsere Prioritäten preiszugeben. Wir sollten versuchen, eine Lösung zu finden, die für uns beide faire und gerecht ist. Wenn wir uns auf eine Kombination einigen können, die für uns beide sinnvoll ist, dann sollten wir versuchen, diese Lösung zu finden und nicht zu versuchen, den anderen auszutricksen oder zu überlisten.\\ \tt \\ \tt Lass uns weiter diskutieren und sehen, ob wir eine Lösung finden können, die für uns beide gut ist! 
	  } 
	   } 
	   } 
	 & \\ 
 

    \theutterance \stepcounter{utterance}  

    & \multicolumn{4}{p{0.6\linewidth}}{\cellcolor[rgb]{0.8,1,0.9}{%
	\makecell[{{p{\linewidth}}}]{% 
	  \tt {\tiny [A$\rangle$GM]}  
	 Ich stimme dir voll und ganz zu! Es ist wichtig, dass wir offen und ehrlich sind, ohne unsere Prioritäten preiszugeben. Indem wir unsere Interessen und Bedürfnisse austauschen, können wir eine Lösung finden, die für uns beide faire und gerecht ist.\\ \tt \\ \tt Die Idee, über Gegenstände zu sprechen, die für uns beide weniger wichtig sind, aber dennoch einen gewissen Wert haben, ist auch sehr interessant. Wenn wir mehrere Messer erhalten würden, aber nicht alle benötigen, könnten wir vielleicht eine Lösung finden, bei der wir einige davon teilen oder austauschen.\\ \tt \\ \tt Ich denke, es ist auch wichtig, dass wir unsere Erwartungen und Bedürfnisse realistisch einschätzen. Wenn wir uns auf eine Kombination einigen können, die für uns beide sinnvoll ist, dann sollten wir versuchen, diese Lösung zu finden und nicht zu versuchen, den anderen auszutricksen oder zu überlisten.\\ \tt \\ \tt Ich denke, wir sollten auch über die Frage sprechen, wie wir unsere Vorschläge machen werden. Sollten wir perhaps unsere Vorschläge schrittweise machen, indem wir uns auf bestimmte Gegenstände konzentrieren und dann unsere Vorschläge erweitern? Oder sollten wir versuchen, eine umfassende Lösung zu finden, die alle Gegenstände berücksichtigt?\\ \tt \\ \tt Lass uns weiter diskutieren und sehen, ob wir eine Lösung finden können, die für uns beide gut ist! 
	  } 
	   } 
	   } 
	 & & \\ 
 

    \theutterance \stepcounter{utterance}  

    & & & \multicolumn{2}{p{0.3\linewidth}}{\cellcolor[rgb]{0.95,0.95,0.95}{%
	\makecell[{{p{\linewidth}}}]{% 
	  \tt {\tiny [GM$|$GM]}  
	 continue 
	  } 
	   } 
	   } 
	 & & \\ 
 

    \theutterance \stepcounter{utterance}  

    & & & \multicolumn{2}{p{0.3\linewidth}}{\cellcolor[rgb]{0.95,0.95,0.95}{%
	\makecell[{{p{\linewidth}}}]{% 
	  \tt {\tiny [GM$|$GM]}  
	 Ich stimme dir voll und ganz zu! Es ist wichtig, dass wir offen und ehrlich sind, ohne unsere Prioritäten preiszugeben. Indem wir unsere Interessen und Bedürfnisse austauschen, können wir eine Lösung finden, die für uns beide faire und gerecht ist.\\ \tt \\ \tt Die Idee, über Gegenstände zu sprechen, die für uns beide weniger wichtig sind, aber dennoch einen gewissen Wert haben, ist auch sehr interessant. Wenn wir mehrere Messer erhalten würden, aber nicht alle benötigen, könnten wir vielleicht eine Lösung finden, bei der wir einige davon teilen oder austauschen.\\ \tt \\ \tt Ich denke, es ist auch wichtig, dass wir unsere Erwartungen und Bedürfnisse realistisch einschätzen. Wenn wir uns auf eine Kombination einigen können, die für uns beide sinnvoll ist, dann sollten wir versuchen, diese Lösung zu finden und nicht zu versuchen, den anderen auszutricksen oder zu überlisten.\\ \tt \\ \tt Ich denke, wir sollten auch über die Frage sprechen, wie wir unsere Vorschläge machen werden. Sollten wir perhaps unsere Vorschläge schrittweise machen, indem wir uns auf bestimmte Gegenstände konzentrieren und dann unsere Vorschläge erweitern? Oder sollten wir versuchen, eine umfassende Lösung zu finden, die alle Gegenstände berücksichtigt?\\ \tt \\ \tt Lass uns weiter diskutieren und sehen, ob wir eine Lösung finden können, die für uns beide gut ist! 
	  } 
	   } 
	   } 
	 & & \\ 
 

    \theutterance \stepcounter{utterance}  

    & & \multicolumn{4}{p{0.6\linewidth}}{\cellcolor[rgb]{0.9,0.9,0.9}{%
	\makecell[{{p{\linewidth}}}]{% 
	  \tt {\tiny [GM$\rangle$B]}  
	 Ich stimme dir voll und ganz zu! Es ist wichtig, dass wir offen und ehrlich sind, ohne unsere Prioritäten preiszugeben. Indem wir unsere Interessen und Bedürfnisse austauschen, können wir eine Lösung finden, die für uns beide faire und gerecht ist.\\ \tt \\ \tt Die Idee, über Gegenstände zu sprechen, die für uns beide weniger wichtig sind, aber dennoch einen gewissen Wert haben, ist auch sehr interessant. Wenn wir mehrere Messer erhalten würden, aber nicht alle benötigen, könnten wir vielleicht eine Lösung finden, bei der wir einige davon teilen oder austauschen.\\ \tt \\ \tt Ich denke, es ist auch wichtig, dass wir unsere Erwartungen und Bedürfnisse realistisch einschätzen. Wenn wir uns auf eine Kombination einigen können, die für uns beide sinnvoll ist, dann sollten wir versuchen, diese Lösung zu finden und nicht zu versuchen, den anderen auszutricksen oder zu überlisten.\\ \tt \\ \tt Ich denke, wir sollten auch über die Frage sprechen, wie wir unsere Vorschläge machen werden. Sollten wir perhaps unsere Vorschläge schrittweise machen, indem wir uns auf bestimmte Gegenstände konzentrieren und dann unsere Vorschläge erweitern? Oder sollten wir versuchen, eine umfassende Lösung zu finden, die alle Gegenstände berücksichtigt?\\ \tt \\ \tt Lass uns weiter diskutieren und sehen, ob wir eine Lösung finden können, die für uns beide gut ist! 
	  } 
	   } 
	   } 
	 & \\ 
 

    \theutterance \stepcounter{utterance}  

    & & & \multicolumn{4}{p{0.6\linewidth}}{\cellcolor[rgb]{1,0.85,0.72}{%
	\makecell[{{p{\linewidth}}}]{% 
	  \tt {\tiny [GM$\langle$B]}  
	 Ich denke, wir sind auf dem richtigen Weg! Es ist wichtig, dass wir unsere Erwartungen und Bedürfnisse realistisch einschätzen und eine Lösung finden, die für uns beide faire und gerecht ist.\\ \tt \\ \tt Ich denke, die Idee, unsere Vorschläge schrittweise zu machen, indem wir uns auf bestimmte Gegenstände konzentrieren und dann unsere Vorschläge erweitern, ist sehr interessant. Das könnte uns helfen, unsere Interessen und Bedürfnisse besser zu verstehen und eine Lösung zu finden, die für uns beide sinnvoll ist.\\ \tt \\ \tt Ich denke auch, dass es wichtig ist, dass wir unsere Vorschläge klar und deutlich machen, sodass der andere Spieler sie verstehen kann. Wir sollten perhaps unsere Vorschläge in einer Weise machen, dass sie leicht zu verstehen sind und keine Missverständnisse aufkommen.\\ \tt \\ \tt Ich denke, wir sollten auch über die Frage sprechen, wie wir unsere Vorschläge koordinieren können. Sollten wir perhaps unsere Vorschläge synchron machen, indem wir uns auf bestimmte Gegenstände konzentrieren und dann unsere Vorschläge erweitern? Oder sollten wir versuchen, unsere Vorschläge unabhängig voneinander zu machen und dann unsere Ergebnisse zu vergleichen?\\ \tt \\ \tt Ich denke, es ist auch wichtig, dass wir unsere Zeit und unsere Nachrichten sorgfältig planen. Wir haben nur 5 Nachrichten, um unsere Vorschläge zu machen, also sollten wir sie sorgfältig nutzen und sicherstellen, dass wir unsere wichtigsten Punkte ansprechen.\\ \tt \\ \tt Lass uns weiter diskutieren und sehen, ob wir eine Lösung finden können, die für uns beide gut ist! Ich denke, wir sind nah dran, eine gute Lösung zu finden. 
	  } 
	   } 
	   } 
	  \\ 
 

    \theutterance \stepcounter{utterance}  

    & & & \multicolumn{2}{p{0.3\linewidth}}{\cellcolor[rgb]{0.95,0.95,0.95}{%
	\makecell[{{p{\linewidth}}}]{% 
	  \tt {\tiny [GM$|$GM]}  
	 continue 
	  } 
	   } 
	   } 
	 & & \\ 
 

    \theutterance \stepcounter{utterance}  

    & & & \multicolumn{2}{p{0.3\linewidth}}{\cellcolor[rgb]{0.95,0.95,0.95}{%
	\makecell[{{p{\linewidth}}}]{% 
	  \tt {\tiny [GM$|$GM]}  
	 Ich denke, wir sind auf dem richtigen Weg! Es ist wichtig, dass wir unsere Erwartungen und Bedürfnisse realistisch einschätzen und eine Lösung finden, die für uns beide faire und gerecht ist.\\ \tt \\ \tt Ich denke, die Idee, unsere Vorschläge schrittweise zu machen, indem wir uns auf bestimmte Gegenstände konzentrieren und dann unsere Vorschläge erweitern, ist sehr interessant. Das könnte uns helfen, unsere Interessen und Bedürfnisse besser zu verstehen und eine Lösung zu finden, die für uns beide sinnvoll ist.\\ \tt \\ \tt Ich denke auch, dass es wichtig ist, dass wir unsere Vorschläge klar und deutlich machen, sodass der andere Spieler sie verstehen kann. Wir sollten perhaps unsere Vorschläge in einer Weise machen, dass sie leicht zu verstehen sind und keine Missverständnisse aufkommen.\\ \tt \\ \tt Ich denke, wir sollten auch über die Frage sprechen, wie wir unsere Vorschläge koordinieren können. Sollten wir perhaps unsere Vorschläge synchron machen, indem wir uns auf bestimmte Gegenstände konzentrieren und dann unsere Vorschläge erweitern? Oder sollten wir versuchen, unsere Vorschläge unabhängig voneinander zu machen und dann unsere Ergebnisse zu vergleichen?\\ \tt \\ \tt Ich denke, es ist auch wichtig, dass wir unsere Zeit und unsere Nachrichten sorgfältig planen. Wir haben nur 5 Nachrichten, um unsere Vorschläge zu machen, also sollten wir sie sorgfältig nutzen und sicherstellen, dass wir unsere wichtigsten Punkte ansprechen.\\ \tt \\ \tt Lass uns weiter diskutieren und sehen, ob wir eine Lösung finden können, die für uns beide gut ist! Ich denke, wir sind nah dran, eine gute Lösung zu finden. 
	  } 
	   } 
	   } 
	 & & \\ 
 

    \theutterance \stepcounter{utterance}  

    & & \multicolumn{4}{p{0.6\linewidth}}{\cellcolor[rgb]{0.9,0.9,0.9}{%
	\makecell[{{p{\linewidth}}}]{% 
	  \tt {\tiny [A$\langle$GM]}  
	 Ich denke, wir sind auf dem richtigen Weg! Es ist wichtig, dass wir unsere Erwartungen und Bedürfnisse realistisch einschätzen und eine Lösung finden, die für uns beide faire und gerecht ist.\\ \tt \\ \tt Ich denke, die Idee, unsere Vorschläge schrittweise zu machen, indem wir uns auf bestimmte Gegenstände konzentrieren und dann unsere Vorschläge erweitern, ist sehr interessant. Das könnte uns helfen, unsere Interessen und Bedürfnisse besser zu verstehen und eine Lösung zu finden, die für uns beide sinnvoll ist.\\ \tt \\ \tt Ich denke auch, dass es wichtig ist, dass wir unsere Vorschläge klar und deutlich machen, sodass der andere Spieler sie verstehen kann. Wir sollten perhaps unsere Vorschläge in einer Weise machen, dass sie leicht zu verstehen sind und keine Missverständnisse aufkommen.\\ \tt \\ \tt Ich denke, wir sollten auch über die Frage sprechen, wie wir unsere Vorschläge koordinieren können. Sollten wir perhaps unsere Vorschläge synchron machen, indem wir uns auf bestimmte Gegenstände konzentrieren und dann unsere Vorschläge erweitern? Oder sollten wir versuchen, unsere Vorschläge unabhängig voneinander zu machen und dann unsere Ergebnisse zu vergleichen?\\ \tt \\ \tt Ich denke, es ist auch wichtig, dass wir unsere Zeit und unsere Nachrichten sorgfältig planen. Wir haben nur 5 Nachrichten, um unsere Vorschläge zu machen, also sollten wir sie sorgfältig nutzen und sicherstellen, dass wir unsere wichtigsten Punkte ansprechen.\\ \tt \\ \tt Lass uns weiter diskutieren und sehen, ob wir eine Lösung finden können, die für uns beide gut ist! Ich denke, wir sind nah dran, eine gute Lösung zu finden.\\ \tt \\ \tt \\ \tt Die Zeit ist um. Sie sind jetzt an der Reihe, einen geheimen Vorschlag einzureichen.\\ \tt  
	  } 
	   } 
	   } 
	 & \\ 
 

    \theutterance \stepcounter{utterance}  

    & \multicolumn{4}{p{0.6\linewidth}}{\cellcolor[rgb]{0.8,1,0.9}{%
	\makecell[{{p{\linewidth}}}]{% 
	  \tt {\tiny [A$\rangle$GM]}  
	 [Vorschlag: 1 Ziegel, 1 Spiegel, 1 Kamm] 
	  } 
	   } 
	   } 
	 & & \\ 
 

    \theutterance \stepcounter{utterance}  

    & & & \multicolumn{2}{p{0.3\linewidth}}{\cellcolor[rgb]{0.95,0.95,0.95}{%
	\makecell[{{p{\linewidth}}}]{% 
	  \tt {\tiny [GM$|$GM]}  
	 proposal 
	  } 
	   } 
	   } 
	 & & \\ 
 

    \theutterance \stepcounter{utterance}  

    & & & \multicolumn{2}{p{0.3\linewidth}}{\cellcolor[rgb]{0.95,0.95,0.95}{%
	\makecell[{{p{\linewidth}}}]{% 
	  \tt {\tiny [GM$|$GM]}  
	 [1, 1, 0, 1] 
	  } 
	   } 
	   } 
	 & & \\ 
 

    \theutterance \stepcounter{utterance}  

    & & \multicolumn{4}{p{0.6\linewidth}}{\cellcolor[rgb]{0.9,0.9,0.9}{%
	\makecell[{{p{\linewidth}}}]{% 
	  \tt {\tiny [GM$\rangle$B]}  
	 Der andere Spieler hat einen geheimen Vorschlag gemacht. Jetzt bist du an der Reihe, einen eigenen Vorschlag zu machen.\\ \tt  
	  } 
	   } 
	   } 
	 & \\ 
 

    \theutterance \stepcounter{utterance}  

    & & & \multicolumn{4}{p{0.6\linewidth}}{\cellcolor[rgb]{1,0.85,0.72}{%
	\makecell[{{p{\linewidth}}}]{% 
	  \tt {\tiny [GM$\langle$B]}  
	 [Vorschlag: 1 Kamm, 1 Spiegel, 1 Ziegel] 
	  } 
	   } 
	   } 
	  \\ 
 

    \theutterance \stepcounter{utterance}  

    & & & \multicolumn{2}{p{0.3\linewidth}}{\cellcolor[rgb]{0.95,0.95,0.95}{%
	\makecell[{{p{\linewidth}}}]{% 
	  \tt {\tiny [GM$|$GM]}  
	 proposal 
	  } 
	   } 
	   } 
	 & & \\ 
 

    \theutterance \stepcounter{utterance}  

    & & & \multicolumn{2}{p{0.3\linewidth}}{\cellcolor[rgb]{0.95,0.95,0.95}{%
	\makecell[{{p{\linewidth}}}]{% 
	  \tt {\tiny [GM$|$GM]}  
	 [1, 1, 0, 1] 
	  } 
	   } 
	   } 
	 & & \\ 
 

    \theutterance \stepcounter{utterance}  

    & & & \multicolumn{2}{p{0.3\linewidth}}{\cellcolor[rgb]{0.95,0.95,0.95}{%
	\makecell[{{p{\linewidth}}}]{% 
	  \tt {\tiny [GM$|$GM]}  
	 [[1, 1, 0, 1], [1, 1, 0, 1]] 
	  } 
	   } 
	   } 
	 & & \\ 
 

\end{supertabular}
}

\end{document}
