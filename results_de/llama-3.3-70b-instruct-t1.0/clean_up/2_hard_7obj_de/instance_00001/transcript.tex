
\documentclass{article}
\usepackage{colortbl}
\usepackage{makecell}
\usepackage{multirow}
\usepackage{supertabular}

\begin{document}

\newcounter{utterance}

\centering \large Interaction Transcript for game `clean\_up', experiment `2\_hard\_7obj\_de', episode 1 with llama{-}3.3{-}70b{-}instruct{-}t1.0.
\vspace{24pt}

{ \footnotesize  \setcounter{utterance}{1}
\setlength{\tabcolsep}{0pt}
\begin{supertabular}{c@{$\;$}|p{.15\linewidth}@{}p{.15\linewidth}p{.15\linewidth}p{.15\linewidth}p{.15\linewidth}p{.15\linewidth}}
    \# & $\;$A & \multicolumn{4}{c}{Game Master} & $\;\:$B\\
    \hline

    \theutterance \stepcounter{utterance}  
    & & \multicolumn{4}{p{0.6\linewidth}}{
        \cellcolor[rgb]{0.9,0.9,0.9}{
            \makecell[{{p{\linewidth}}}]{
                \texttt{\tiny{[P1$\langle$GM]}}
                \texttt{Ich bin der Spielleiter, und du spielst ein kooperatives Spiel mit dem folgenden Raster als Spielbrett:} \\
\\ 
\texttt{```} \\
\texttt{ 1234567} \\
\texttt{╔═══╤═╤═╗} \\
\texttt{║◌S◌│◌│A║ 1} \\
\texttt{╟─┬─┼─┴─╢ 2} \\
\texttt{║◌│◌│◌◌◌║ 3} \\
\texttt{║P├─┤D┌─╢ 4} \\
\texttt{╟─┘◌│◌│M║ 5} \\
\texttt{║◌◌◌└┬┴─╢ 6} \\
\texttt{║◌OT◌│◌◌║ 7} \\
\texttt{╚════╧══╝} \\
\\ 
\texttt{```} \\
\\ 
\texttt{* Am oberen Rand sind die x{-}Koordinaten angegeben, die nach rechts hin zunehmen, und am rechten Rand die y{-}Koordinaten, die nach unten zunehmen.} \\
\texttt{* Die folgenden Objekte sind zufällig auf deinem Raster platziert: 'P', 'O', 'T', 'S', 'D', 'A', 'M'.} \\
\\ 
\\ 
\texttt{Dein Mitspieler sieht eine Variation des Spielbretts, auf dem dieselben Objekte an anderen zufälligen Positionen platziert sind. Du kannst das Spielbrett des anderen Spielers nicht sehen, und er kann deins nicht sehen.} \\
\\ 
\\ 
\texttt{**Ziel des Spiels:**} \\
\\ 
\texttt{Beide Spieler müssen die Objekte auf ihrem jeweiligen Spielbrett so bewegen, dass identische Objekte am Ende an denselben Koordinaten liegen. Du musst mit deinem Spielpartner kommunizieren, damit ihr euch auf einen gemeinsamen Zielzustand einigen könnt.} \\
\\ 
\\ 
\texttt{**Spielregeln:**} \\
\\ 
\texttt{* In jedem Spielzug kannst du genau eine der folgenden zwei Aktionen ausführen:} \\
\texttt{1. `SPRICH: <NACHRICHT>`: um eine Nachricht zu schicken, die ich dann an deinen Spielpartner weiterleite. Als Nachricht gilt alles bis zum nächsten Zeilenumbruch.} \\
\texttt{2. `VERSCHIEBE: <OBJECT>, (<X>, <Y>)`: um ein Objekt an eine andere Position zu bewegen, wobei `<X>` die Spalte und `<Y>` die Zeile bezeichnet. Ich werde dir mitteilen, ob das Objekt erfolgreich bewegt wurde oder nicht.} \\
\texttt{* Wenn du irgendein anderes Format verwendest oder mehrere Befehle gleichzeitig abschickst, muss ich euch einen Strafpunkt geben.} \\
\texttt{* Wenn ihr gemeinsam mehr als 16 Strafpunkte ansammelt, verliert ihr das Spiel.} \\
\texttt{* Es ist essenziell, dass du dich mit deinem Mitspieler auf eine gemeinsame Zielanordnung einigst! Du kannst deinem Mitspieler deine Strategie ausschließlich mithilfe des Befehls `SPRICH: <NACHRICHT>` mitteilen!} \\
\\ 
\\ 
\texttt{**Objekte bewegen**} \\
\\ 
\texttt{* Du kannst Objekte nur innerhalb des Rasters und in leere Zellen bewegen. Leere Zellen erkennst am Zeichen '◌'.} \\
\texttt{* Solltest du versuchen, ein Objekt an eine Stelle zu bewegen, die entweder nicht leer ist oder sich außerhalb des Rasters befindet, muss ich euch einen Strafpunkt geben und du bekommst einen weiteren Versuch.} \\
\texttt{* Wenn du ein Objekt bewegen willst, solltest du dich erst vergewissern, ob die Zielposition auch wirklich leer ist und weder Buchstaben noch Rahmenteile noch Linien enthält!} \\
\\ 
\\ 
\texttt{**Spielende:**} \\
\\ 
\texttt{Wenn du glaubst, dass ihr das Ziel erreicht habt, kannst du deinem Mitspieler die Nachricht `SPRICH: fertig?` schicken. Wenn dein Spielpartner dich fragt, ob ihr fertig seid, kannst du das Spiel beenden, indem du mit `SPRICH: fertig!` antwortest.} \\
\\ 
\texttt{Ihr gewinnt gemeinsam, wenn ihr das Spiel innerhalb von \#max\_rounds Runden beendet. Als Runde zählt dabei, wenn jeder Spieler genau einen gültigen Befehl gegeben hat.} \\
\\ 
\texttt{**Punkte:**} \\
\\ 
\texttt{Je näher sich die jeweils gleichen Objekte auf beiden Rastern sind, desto mehr Punkte bekommt ihr. Für Strafpunkte gibt es Punktabzug. Könnt ihr den Rekord brechen?} \\
\\ 
\texttt{Schicke eine Nachricht an deinen Mitspieler um das Spiel zu starten!} \\
            }
        }
    }
    & \\ \\

    \theutterance \stepcounter{utterance}  
    & \multicolumn{4}{p{0.6\linewidth}}{
        \cellcolor[rgb]{0.9,0.9,0.9}{
            \makecell[{{p{\linewidth}}}]{
                \texttt{\tiny{[P1$\rangle$GM]}}
                \texttt{`SPRICH: Hallo! Lass uns das Spiel beginnen. Ich schlage vor, dass wir uns auf eine gemeinsame Zielanordnung einigen, indem wir die Objekte in alphabetischer Reihenfolge anordnen, beginnend von oben links. Wie sieht dein Spielbrett aus?`} \\
            }
        }
    }
    & & \\ \\

    \theutterance \stepcounter{utterance}  
    & & & \multicolumn{2}{p{0.3\linewidth}}{
        \cellcolor[rgb]{0.9,0.9,0.9}{
            \makecell[{{p{\linewidth}}}]{
                \texttt{\tiny{[GM$|$GM]}}
                \texttt{`SPRICH: Hallo! Lass uns das Spiel beginnen. Ich schlage vor, dass wir uns auf eine gemeinsame Zielanordnung einigen, indem wir die Objekte in alphabetischer Reihenfolge anordnen, beginnend von oben links. Wie sieht dein Spielbrett aus?`} \\
            }
        }
    }
    & & \\ \\

    \theutterance \stepcounter{utterance}  
    & & \multicolumn{4}{p{0.6\linewidth}}{
        \cellcolor[rgb]{0.9,0.9,0.9}{
            \makecell[{{p{\linewidth}}}]{
                \texttt{\tiny{[GM$\rangle$P2]}}
                \texttt{Ich bin der Spielleiter, und du spielst ein kooperatives Spiel mit dem folgenden Raster als Spielbrett:} \\
\\ 
\texttt{```} \\
\texttt{ 1234567} \\
\texttt{╔═══╤═╤═╗} \\
\texttt{║P◌A│◌│◌║ 1} \\
\texttt{╟─┬─┼─┴─╢ 2} \\
\texttt{║◌│D│◌◌◌║ 3} \\
\texttt{║◌├─┤◌┌─╢ 4} \\
\texttt{╟─┘◌│◌│O║ 5} \\
\texttt{║◌◌◌└┬┴─╢ 6} \\
\texttt{║◌S◌T│◌M║ 7} \\
\texttt{╚════╧══╝} \\
\\ 
\texttt{```} \\
\\ 
\texttt{* Am oberen Rand sind die x{-}Koordinaten angegeben, die nach rechts hin zunehmen, und am rechten Rand die y{-}Koordinaten, die nach unten zunehmen.} \\
\texttt{* Die folgenden Objekte sind zufällig auf deinem Raster platziert: 'P', 'O', 'T', 'S', 'D', 'A', 'M'.} \\
\\ 
\\ 
\texttt{Dein Mitspieler sieht eine Variation des Spielbretts, auf dem dieselben Objekte an anderen zufälligen Positionen platziert sind. Du kannst das Spielbrett des anderen Spielers nicht sehen, und er kann deins nicht sehen.} \\
\\ 
\\ 
\texttt{**Ziel des Spiels:**} \\
\\ 
\texttt{Beide Spieler müssen die Objekte auf ihrem jeweiligen Spielbrett so bewegen, dass identische Objekte am Ende an denselben Koordinaten liegen. Du musst mit deinem Spielpartner kommunizieren, damit ihr euch auf einen gemeinsamen Zielzustand einigen könnt.} \\
\\ 
\\ 
\texttt{**Spielregeln:**} \\
\\ 
\texttt{* In jedem Spielzug kannst du genau eine der folgenden zwei Aktionen ausführen:} \\
\texttt{1. `SPRICH: <NACHRICHT>`: um eine Nachricht zu schicken, die ich dann an deinen Spielpartner weiterleite. Als Nachricht gilt alles bis zum nächsten Zeilenumbruch.} \\
\texttt{2. `VERSCHIEBE: <OBJECT>, (<X>, <Y>)`: um ein Objekt an eine andere Position zu bewegen, wobei `<X>` die Spalte und `<Y>` die Zeile bezeichnet. Ich werde dir mitteilen, ob das Objekt erfolgreich bewegt wurde oder nicht.} \\
\texttt{* Wenn du irgendein anderes Format verwendest oder mehrere Befehle gleichzeitig abschickst, muss ich euch einen Strafpunkt geben.} \\
\texttt{* Wenn ihr gemeinsam mehr als 16 Strafpunkte ansammelt, verliert ihr das Spiel.} \\
\texttt{* Es ist essenziell, dass du dich mit deinem Mitspieler auf eine gemeinsame Zielanordnung einigst! Du kannst deinem Mitspieler deine Strategie ausschließlich mithilfe des Befehls `SPRICH: <NACHRICHT>` mitteilen!} \\
\\ 
\\ 
\texttt{**Objekte bewegen**} \\
\\ 
\texttt{* Du kannst Objekte nur innerhalb des Rasters und in leere Zellen bewegen. Leere Zellen erkennst am Zeichen '◌'.} \\
\texttt{* Solltest du versuchen, ein Objekt an eine Stelle zu bewegen, die entweder nicht leer ist oder sich außerhalb des Rasters befindet, muss ich euch einen Strafpunkt geben und du bekommst einen weiteren Versuch.} \\
\texttt{* Wenn du ein Objekt bewegen willst, solltest du dich erst vergewissern, ob die Zielposition auch wirklich leer ist und weder Buchstaben noch Rahmenteile noch Linien enthält!} \\
\\ 
\\ 
\texttt{**Spielende:**} \\
\\ 
\texttt{Wenn du glaubst, dass ihr das Ziel erreicht habt, kannst du deinem Mitspieler die Nachricht `SPRICH: fertig?` schicken. Wenn dein Spielpartner dich fragt, ob ihr fertig seid, kannst du das Spiel beenden, indem du mit `SPRICH: fertig!` antwortest.} \\
\\ 
\texttt{Ihr gewinnt gemeinsam, wenn ihr das Spiel innerhalb von \#max\_rounds Runden beendet. Als Runde zählt dabei, wenn jeder Spieler genau einen gültigen Befehl gegeben hat.} \\
\\ 
\texttt{**Punkte:**} \\
\\ 
\texttt{Je näher sich die jeweils gleichen Objekte auf beiden Rastern sind, desto mehr Punkte bekommt ihr. Für Strafpunkte gibt es Punktabzug. Könnt ihr den Rekord brechen?} \\
\\ 
\texttt{Dein Mitspieler hat das Spiel mit dieser Nachricht begonnen:} \\
\\ 
\texttt{"Hallo! Lass uns das Spiel beginnen. Ich schlage vor, dass wir uns auf eine gemeinsame Zielanordnung einigen, indem wir die Objekte in alphabetischer Reihenfolge anordnen, beginnend von oben links. Wie sieht dein Spielbrett aus?"} \\
\\ 
\texttt{Wie lautet dein erster Befehl?} \\
            }
        }
    }
    & \\ \\

    \theutterance \stepcounter{utterance}  
    & & & \multicolumn{4}{p{0.6\linewidth}}{
        \cellcolor[rgb]{0.9,0.9,0.9}{
            \makecell[{{p{\linewidth}}}]{
                \texttt{\tiny{[GM$\langle$P2]}}
                \texttt{SPRICH: } \\
\texttt{Hallo! Mein Spielbrett sieht so aus: } \\
\\ 
\texttt{ 1234567} \\
\texttt{╔═══╤═╤═╗} \\
\texttt{║P◌A│◌│◌║ 1} \\
\texttt{╟─┬─┼─┴─╢ 2} \\
\texttt{║◌│D│◌◌◌║ 3} \\
\texttt{║◌├─┤◌┌─╢ 4} \\
\texttt{╟─┘◌│◌│O║ 5} \\
\texttt{║◌◌◌└┬┴─╢ 6} \\
\texttt{║◌S◌T│◌M║ 7} \\
\texttt{╚════╧══╝} \\
\\ 
\texttt{Ich denke, deine Idee ist gut! Lass uns die Objekte in alphabetischer Reihenfolge anordnen, beginnend von oben links. Wie viele Strafpunkte haben wir aktuell?} \\
            }
        }
    }
     \\ \\

    \theutterance \stepcounter{utterance}  
    & & & \multicolumn{2}{p{0.3\linewidth}}{
        \cellcolor[rgb]{0.9,0.9,0.9}{
            \makecell[{{p{\linewidth}}}]{
                \texttt{\tiny{[GM$|$GM]}}
                \texttt{SPRICH: } \\
\texttt{Hallo! Mein Spielbrett sieht so aus: } \\
\\ 
\texttt{ 1234567} \\
\texttt{╔═══╤═╤═╗} \\
\texttt{║P◌A│◌│◌║ 1} \\
\texttt{╟─┬─┼─┴─╢ 2} \\
\texttt{║◌│D│◌◌◌║ 3} \\
\texttt{║◌├─┤◌┌─╢ 4} \\
\texttt{╟─┘◌│◌│O║ 5} \\
\texttt{║◌◌◌└┬┴─╢ 6} \\
\texttt{║◌S◌T│◌M║ 7} \\
\texttt{╚════╧══╝} \\
\\ 
\texttt{Ich denke, deine Idee ist gut! Lass uns die Objekte in alphabetischer Reihenfolge anordnen, beginnend von oben links. Wie viele Strafpunkte haben wir aktuell?} \\
            }
        }
    }
    & & \\ \\

    \theutterance \stepcounter{utterance}  
    & & & \multicolumn{2}{p{0.3\linewidth}}{
        \cellcolor[rgb]{0.9,0.9,0.9}{
            \makecell[{{p{\linewidth}}}]{
                \texttt{\tiny{[GM$|$GM]}}
                \texttt{Invalid response format} \\
            }
        }
    }
    & & \\ \\

    \theutterance \stepcounter{utterance}  
    & & \multicolumn{4}{p{0.6\linewidth}}{
        \cellcolor[rgb]{0.9,0.9,0.9}{
            \makecell[{{p{\linewidth}}}]{
                \texttt{\tiny{[GM$\rangle$P2]}}
                \texttt{Strafpunkt: Deine Nachricht entspricht keinem gültigen Format! Stelle sicher, dass deine Antwort ausschließlich einen der beiden Befehle `SPRICH: <NACHRICHT>` und `VERSCHIEBE: <OBJECT>, (<X>, <Y>)` und sonst nichts enthält!} \\
\\ 
\texttt{Ihr habt gemeinsam 1 von maximal 16 Strafpunkten angesammelt.} \\
\texttt{Bitte versuche es noch mal!} \\
            }
        }
    }
    & \\ \\

    \theutterance \stepcounter{utterance}  
    & & & \multicolumn{4}{p{0.6\linewidth}}{
        \cellcolor[rgb]{0.9,0.9,0.9}{
            \makecell[{{p{\linewidth}}}]{
                \texttt{\tiny{[GM$\langle$P2]}}
                \texttt{SPRICH: Hallo! Mein Spielbrett sieht so aus: } \\
\texttt{ 1234567} \\
\texttt{╔═══╤═╤═╗} \\
\texttt{║P◌A│◌│◌║ 1} \\
\texttt{╟─┬─┼─┴─╢ 2} \\
\texttt{║◌│D│◌◌◌║ 3} \\
\texttt{║◌├─┤◌┌─╢ 4} \\
\texttt{╟─┘◌│◌│O║ 5} \\
\texttt{║◌◌◌└┬┴─╢ 6} \\
\texttt{║◌S◌T│◌M║ 7} \\
\texttt{╚════╧══╝} \\
\texttt{Ich denke, deine Idee ist gut! Lass uns die Objekte in alphabetischer Reihenfolge anordnen, beginnend von oben links. Wie viele Strafpunkte haben wir aktuell?} \\
            }
        }
    }
     \\ \\

    \theutterance \stepcounter{utterance}  
    & & & \multicolumn{2}{p{0.3\linewidth}}{
        \cellcolor[rgb]{0.9,0.9,0.9}{
            \makecell[{{p{\linewidth}}}]{
                \texttt{\tiny{[GM$|$GM]}}
                \texttt{SPRICH: Hallo! Mein Spielbrett sieht so aus: } \\
\texttt{ 1234567} \\
\texttt{╔═══╤═╤═╗} \\
\texttt{║P◌A│◌│◌║ 1} \\
\texttt{╟─┬─┼─┴─╢ 2} \\
\texttt{║◌│D│◌◌◌║ 3} \\
\texttt{║◌├─┤◌┌─╢ 4} \\
\texttt{╟─┘◌│◌│O║ 5} \\
\texttt{║◌◌◌└┬┴─╢ 6} \\
\texttt{║◌S◌T│◌M║ 7} \\
\texttt{╚════╧══╝} \\
\texttt{Ich denke, deine Idee ist gut! Lass uns die Objekte in alphabetischer Reihenfolge anordnen, beginnend von oben links. Wie viele Strafpunkte haben wir aktuell?} \\
            }
        }
    }
    & & \\ \\

    \theutterance \stepcounter{utterance}  
    & & & \multicolumn{2}{p{0.3\linewidth}}{
        \cellcolor[rgb]{0.9,0.9,0.9}{
            \makecell[{{p{\linewidth}}}]{
                \texttt{\tiny{[GM$|$GM]}}
                \texttt{Invalid format: tail is not empty: '} \\
\texttt{ 1234567} \\
\texttt{╔═══╤═╤═╗} \\
\texttt{║P◌A│◌│◌║ 1} \\
\texttt{╟─┬─┼─┴─╢ 2} \\
\texttt{║◌│D│◌◌◌║ 3} \\
\texttt{║◌├─┤◌┌─╢ 4} \\
\texttt{╟─┘◌│◌│O║ 5} \\
\texttt{║◌◌◌└┬┴─╢ 6} \\
\texttt{║◌S◌T│◌M║ 7} \\
\texttt{╚════╧══╝} \\
\texttt{Ich denke, deine Idee ist gut! Lass uns die Objekte in alphabetischer Reihenfolge anordnen, beginnend von oben links. Wie viele Strafpunkte haben wir aktuell?} \\
            }
        }
    }
    & & \\ \\

    \theutterance \stepcounter{utterance}  
    & & \multicolumn{4}{p{0.6\linewidth}}{
        \cellcolor[rgb]{0.9,0.9,0.9}{
            \makecell[{{p{\linewidth}}}]{
                \texttt{\tiny{[GM$\rangle$P2]}}
                \texttt{Strafpunkt: Deine Nachricht darf nichts nach dem Befehl enthalten! Stelle sicher, dass deine Antwort ausschließlich einen der beiden Befehle `SPRICH: <NACHRICHT>` und `VERSCHIEBE: <OBJECT>, (<X>, <Y>)` und sonst nichts enthält!} \\
\\ 
\texttt{Ihr habt gemeinsam 2 von maximal 16 Strafpunkten angesammelt.} \\
\texttt{Bitte versuche es noch mal!} \\
            }
        }
    }
    & \\ \\

    \theutterance \stepcounter{utterance}  
    & & & \multicolumn{4}{p{0.6\linewidth}}{
        \cellcolor[rgb]{0.9,0.9,0.9}{
            \makecell[{{p{\linewidth}}}]{
                \texttt{\tiny{[GM$\langle$P2]}}
                \texttt{SPRICH: Mein Spielbrett zeigt P in (1,1), A in (3,1), D in (2,3), O in (6,5), S in (1,7), T in (4,7) und M in (7,7). Ich stimme zu, die Objekte in alphabetischer Reihenfolge anzuordnen.} \\
            }
        }
    }
     \\ \\

    \theutterance \stepcounter{utterance}  
    & & & \multicolumn{2}{p{0.3\linewidth}}{
        \cellcolor[rgb]{0.9,0.9,0.9}{
            \makecell[{{p{\linewidth}}}]{
                \texttt{\tiny{[GM$|$GM]}}
                \texttt{SPRICH: Mein Spielbrett zeigt P in (1,1), A in (3,1), D in (2,3), O in (6,5), S in (1,7), T in (4,7) und M in (7,7). Ich stimme zu, die Objekte in alphabetischer Reihenfolge anzuordnen.} \\
            }
        }
    }
    & & \\ \\

    \theutterance \stepcounter{utterance}  
    & & \multicolumn{4}{p{0.6\linewidth}}{
        \cellcolor[rgb]{0.9,0.9,0.9}{
            \makecell[{{p{\linewidth}}}]{
                \texttt{\tiny{[P1$\langle$GM]}}
                \texttt{Deine Nachricht wurde an deinen Mitspieler übermittelt.} \\
\\ 
\\ 
\texttt{Ihr seid gerade bei Runde 1 von maximal 28.} \\
\texttt{Ihr habt gemeinsam 2 von maximal 16 Strafpunkten angesammelt.} \\
\texttt{Dein Mitspieler hat dir diese Nachricht geschickt:} \\
\\ 
\\ 
\texttt{"Mein Spielbrett zeigt P in (1,1), A in (3,1), D in (2,3), O in (6,5), S in (1,7), T in (4,7) und M in (7,7). Ich stimme zu, die Objekte in alphabetischer Reihenfolge anzuordnen."} \\
\\ 
\\ 
\texttt{Wie lautet dein nächster Befehl?} \\
            }
        }
    }
    & \\ \\

    \theutterance \stepcounter{utterance}  
    & \multicolumn{4}{p{0.6\linewidth}}{
        \cellcolor[rgb]{0.9,0.9,0.9}{
            \makecell[{{p{\linewidth}}}]{
                \texttt{\tiny{[P1$\rangle$GM]}}
                \texttt{`SPRICH: Danke für die Info! Mein Spielbrett zeigt P in (1,4), O in (2,7), T in (3,7), S in (1,1), D in (4,4), A in (1,1) und M in (5,5). Ich denke, wir sollten die Objekte in alphabetischer Reihenfolge anordnen, beginnend von oben links. Ich schlage vor, dass wir A in (1,1), D in (2,1), M in (3,1), O in (4,1), P in (5,1), S in (6,1) und T in (7,1) platzieren. Wie sieht das aus?`} \\
            }
        }
    }
    & & \\ \\

    \theutterance \stepcounter{utterance}  
    & & & \multicolumn{2}{p{0.3\linewidth}}{
        \cellcolor[rgb]{0.9,0.9,0.9}{
            \makecell[{{p{\linewidth}}}]{
                \texttt{\tiny{[GM$|$GM]}}
                \texttt{`SPRICH: Danke für die Info! Mein Spielbrett zeigt P in (1,4), O in (2,7), T in (3,7), S in (1,1), D in (4,4), A in (1,1) und M in (5,5). Ich denke, wir sollten die Objekte in alphabetischer Reihenfolge anordnen, beginnend von oben links. Ich schlage vor, dass wir A in (1,1), D in (2,1), M in (3,1), O in (4,1), P in (5,1), S in (6,1) und T in (7,1) platzieren. Wie sieht das aus?`} \\
            }
        }
    }
    & & \\ \\

    \theutterance \stepcounter{utterance}  
    & & \multicolumn{4}{p{0.6\linewidth}}{
        \cellcolor[rgb]{0.9,0.9,0.9}{
            \makecell[{{p{\linewidth}}}]{
                \texttt{\tiny{[GM$\rangle$P2]}}
                \texttt{Deine Nachricht wurde an deinen Mitspieler übermittelt.} \\
\\ 
\\ 
\texttt{Ihr seid gerade bei Runde 2 von maximal 28.} \\
\texttt{Ihr habt gemeinsam 2 von maximal 16 Strafpunkten angesammelt.} \\
\texttt{Dein Mitspieler hat dir diese Nachricht geschickt:} \\
\\ 
\\ 
\texttt{"Danke für die Info! Mein Spielbrett zeigt P in (1,4), O in (2,7), T in (3,7), S in (1,1), D in (4,4), A in (1,1) und M in (5,5). Ich denke, wir sollten die Objekte in alphabetischer Reihenfolge anordnen, beginnend von oben links. Ich schlage vor, dass wir A in (1,1), D in (2,1), M in (3,1), O in (4,1), P in (5,1), S in (6,1) und T in (7,1) platzieren. Wie sieht das aus?"} \\
\\ 
\\ 
\texttt{Wie lautet dein nächster Befehl?} \\
            }
        }
    }
    & \\ \\

    \theutterance \stepcounter{utterance}  
    & & & \multicolumn{4}{p{0.6\linewidth}}{
        \cellcolor[rgb]{0.9,0.9,0.9}{
            \makecell[{{p{\linewidth}}}]{
                \texttt{\tiny{[GM$\langle$P2]}}
                \texttt{SPRICH: Ich denke, das ist eine gute Idee! Ichakahm an, dass wir die Objekte in alphabetischer Reihenfolge anordnen, beginnend von oben links. Ich schlage vor, dass wir mit dem Objekt A beginnen und es nach (1,1) bewegen. Ich werde mein A von (3,1) nach (1,1) bewegen. Bitte gib mir Feedback, ob das korrekt ist.} \\
            }
        }
    }
     \\ \\

    \theutterance \stepcounter{utterance}  
    & & & \multicolumn{2}{p{0.3\linewidth}}{
        \cellcolor[rgb]{0.9,0.9,0.9}{
            \makecell[{{p{\linewidth}}}]{
                \texttt{\tiny{[GM$|$GM]}}
                \texttt{SPRICH: Ich denke, das ist eine gute Idee! Ichakahm an, dass wir die Objekte in alphabetischer Reihenfolge anordnen, beginnend von oben links. Ich schlage vor, dass wir mit dem Objekt A beginnen und es nach (1,1) bewegen. Ich werde mein A von (3,1) nach (1,1) bewegen. Bitte gib mir Feedback, ob das korrekt ist.} \\
            }
        }
    }
    & & \\ \\

    \theutterance \stepcounter{utterance}  
    & & \multicolumn{4}{p{0.6\linewidth}}{
        \cellcolor[rgb]{0.9,0.9,0.9}{
            \makecell[{{p{\linewidth}}}]{
                \texttt{\tiny{[P1$\langle$GM]}}
                \texttt{Deine Nachricht wurde an deinen Mitspieler übermittelt.} \\
\\ 
\\ 
\texttt{Ihr seid gerade bei Runde 2 von maximal 28.} \\
\texttt{Ihr habt gemeinsam 2 von maximal 16 Strafpunkten angesammelt.} \\
\texttt{Dein Mitspieler hat dir diese Nachricht geschickt:} \\
\\ 
\\ 
\texttt{"Ich denke, das ist eine gute Idee! Ichakahm an, dass wir die Objekte in alphabetischer Reihenfolge anordnen, beginnend von oben links. Ich schlage vor, dass wir mit dem Objekt A beginnen und es nach (1,1) bewegen. Ich werde mein A von (3,1) nach (1,1) bewegen. Bitte gib mir Feedback, ob das korrekt ist."} \\
\\ 
\\ 
\texttt{Wie lautet dein nächster Befehl?} \\
            }
        }
    }
    & \\ \\

    \theutterance \stepcounter{utterance}  
    & \multicolumn{4}{p{0.6\linewidth}}{
        \cellcolor[rgb]{0.9,0.9,0.9}{
            \makecell[{{p{\linewidth}}}]{
                \texttt{\tiny{[P1$\rangle$GM]}}
                \texttt{`SPRICH: Das klingt gut! Ich werde mein A von (1,1) nach (1,2) bewegen, um Platz für das korrekte A in (1,1) zu machen. Bitte bestätige, ob dein A jetzt korrekt in (1,1) steht. Ich werde dann mein nächstes Objekt, D, in Position bringen.`} \\
            }
        }
    }
    & & \\ \\

    \theutterance \stepcounter{utterance}  
    & & & \multicolumn{2}{p{0.3\linewidth}}{
        \cellcolor[rgb]{0.9,0.9,0.9}{
            \makecell[{{p{\linewidth}}}]{
                \texttt{\tiny{[GM$|$GM]}}
                \texttt{`SPRICH: Das klingt gut! Ich werde mein A von (1,1) nach (1,2) bewegen, um Platz für das korrekte A in (1,1) zu machen. Bitte bestätige, ob dein A jetzt korrekt in (1,1) steht. Ich werde dann mein nächstes Objekt, D, in Position bringen.`} \\
            }
        }
    }
    & & \\ \\

    \theutterance \stepcounter{utterance}  
    & & \multicolumn{4}{p{0.6\linewidth}}{
        \cellcolor[rgb]{0.9,0.9,0.9}{
            \makecell[{{p{\linewidth}}}]{
                \texttt{\tiny{[GM$\rangle$P2]}}
                \texttt{Deine Nachricht wurde an deinen Mitspieler übermittelt.} \\
\\ 
\\ 
\texttt{Ihr seid gerade bei Runde 3 von maximal 28.} \\
\texttt{Ihr habt gemeinsam 2 von maximal 16 Strafpunkten angesammelt.} \\
\texttt{Dein Mitspieler hat dir diese Nachricht geschickt:} \\
\\ 
\\ 
\texttt{"Das klingt gut! Ich werde mein A von (1,1) nach (1,2) bewegen, um Platz für das korrekte A in (1,1) zu machen. Bitte bestätige, ob dein A jetzt korrekt in (1,1) steht. Ich werde dann mein nächstes Objekt, D, in Position bringen."} \\
\\ 
\\ 
\texttt{Wie lautet dein nächster Befehl?} \\
            }
        }
    }
    & \\ \\

    \theutterance \stepcounter{utterance}  
    & & & \multicolumn{4}{p{0.6\linewidth}}{
        \cellcolor[rgb]{0.9,0.9,0.9}{
            \makecell[{{p{\linewidth}}}]{
                \texttt{\tiny{[GM$\langle$P2]}}
                \texttt{SPRICH: Mein A ist bereits in (3,1) und ich werde es nach (1,1) bewegen. Mein aktuelles Spielbrett zeigt P in (1,1), also muss ich zuerst P verschieben, bevor ich A an die richtige Position bringen kann. Ich schlage vor, dass ich P nach (5,1) bewege, um Platz für A zu machen. Bitte gib mir Feedback.} \\
            }
        }
    }
     \\ \\

    \theutterance \stepcounter{utterance}  
    & & & \multicolumn{2}{p{0.3\linewidth}}{
        \cellcolor[rgb]{0.9,0.9,0.9}{
            \makecell[{{p{\linewidth}}}]{
                \texttt{\tiny{[GM$|$GM]}}
                \texttt{SPRICH: Mein A ist bereits in (3,1) und ich werde es nach (1,1) bewegen. Mein aktuelles Spielbrett zeigt P in (1,1), also muss ich zuerst P verschieben, bevor ich A an die richtige Position bringen kann. Ich schlage vor, dass ich P nach (5,1) bewege, um Platz für A zu machen. Bitte gib mir Feedback.} \\
            }
        }
    }
    & & \\ \\

    \theutterance \stepcounter{utterance}  
    & & \multicolumn{4}{p{0.6\linewidth}}{
        \cellcolor[rgb]{0.9,0.9,0.9}{
            \makecell[{{p{\linewidth}}}]{
                \texttt{\tiny{[P1$\langle$GM]}}
                \texttt{Deine Nachricht wurde an deinen Mitspieler übermittelt.} \\
\\ 
\\ 
\texttt{Ihr seid gerade bei Runde 3 von maximal 28.} \\
\texttt{Ihr habt gemeinsam 2 von maximal 16 Strafpunkten angesammelt.} \\
\texttt{Dein Mitspieler hat dir diese Nachricht geschickt:} \\
\\ 
\\ 
\texttt{"Mein A ist bereits in (3,1) und ich werde es nach (1,1) bewegen. Mein aktuelles Spielbrett zeigt P in (1,1), also muss ich zuerst P verschieben, bevor ich A an die richtige Position bringen kann. Ich schlage vor, dass ich P nach (5,1) bewege, um Platz für A zu machen. Bitte gib mir Feedback."} \\
\\ 
\\ 
\texttt{Wie lautet dein nächster Befehl?} \\
            }
        }
    }
    & \\ \\

    \theutterance \stepcounter{utterance}  
    & \multicolumn{4}{p{0.6\linewidth}}{
        \cellcolor[rgb]{0.9,0.9,0.9}{
            \makecell[{{p{\linewidth}}}]{
                \texttt{\tiny{[P1$\rangle$GM]}}
                \texttt{`SPRICH: Das klingt nach einem guten Plan! Mein P ist jedoch in (1,4), also muss ich es auch bewegen, um es an die richtige Position zu bringen. Ich schlage vor, dass ich mein P von (1,4) nach (5,1) bewege, aber da dein P bereits nach (5,1) bewegt werden soll, schlage ich vor, dass ich mein P nach (5,2) bewege, um Konflikte zu vermeiden. Bitte bestätige, ob das okay für dich ist.`} \\
            }
        }
    }
    & & \\ \\

    \theutterance \stepcounter{utterance}  
    & & & \multicolumn{2}{p{0.3\linewidth}}{
        \cellcolor[rgb]{0.9,0.9,0.9}{
            \makecell[{{p{\linewidth}}}]{
                \texttt{\tiny{[GM$|$GM]}}
                \texttt{`SPRICH: Das klingt nach einem guten Plan! Mein P ist jedoch in (1,4), also muss ich es auch bewegen, um es an die richtige Position zu bringen. Ich schlage vor, dass ich mein P von (1,4) nach (5,1) bewege, aber da dein P bereits nach (5,1) bewegt werden soll, schlage ich vor, dass ich mein P nach (5,2) bewege, um Konflikte zu vermeiden. Bitte bestätige, ob das okay für dich ist.`} \\
            }
        }
    }
    & & \\ \\

    \theutterance \stepcounter{utterance}  
    & & \multicolumn{4}{p{0.6\linewidth}}{
        \cellcolor[rgb]{0.9,0.9,0.9}{
            \makecell[{{p{\linewidth}}}]{
                \texttt{\tiny{[GM$\rangle$P2]}}
                \texttt{Deine Nachricht wurde an deinen Mitspieler übermittelt.} \\
\\ 
\\ 
\texttt{Ihr seid gerade bei Runde 4 von maximal 28.} \\
\texttt{Ihr habt gemeinsam 2 von maximal 16 Strafpunkten angesammelt.} \\
\texttt{Dein Mitspieler hat dir diese Nachricht geschickt:} \\
\\ 
\\ 
\texttt{"Das klingt nach einem guten Plan! Mein P ist jedoch in (1,4), also muss ich es auch bewegen, um es an die richtige Position zu bringen. Ich schlage vor, dass ich mein P von (1,4) nach (5,1) bewege, aber da dein P bereits nach (5,1) bewegt werden soll, schlage ich vor, dass ich mein P nach (5,2) bewege, um Konflikte zu vermeiden. Bitte bestätige, ob das okay für dich ist."} \\
\\ 
\\ 
\texttt{Wie lautet dein nächster Befehl?} \\
            }
        }
    }
    & \\ \\

    \theutterance \stepcounter{utterance}  
    & & & \multicolumn{4}{p{0.6\linewidth}}{
        \cellcolor[rgb]{0.9,0.9,0.9}{
            \makecell[{{p{\linewidth}}}]{
                \texttt{\tiny{[GM$\langle$P2]}}
                \texttt{SPRICH: Das klingt vernünftig! Ich denke, es ist eine gute Idee, Konflikte zu vermeiden, indem du dein P nach (5,2) bewegst. Ich werde mein P von (1,1) nach (5,1) bewegen. Dann kann ich mein A von (3,1) nach (1,1) bewegen. Bitte gib mir Feedback, ob alles korrekt ist.} \\
            }
        }
    }
     \\ \\

    \theutterance \stepcounter{utterance}  
    & & & \multicolumn{2}{p{0.3\linewidth}}{
        \cellcolor[rgb]{0.9,0.9,0.9}{
            \makecell[{{p{\linewidth}}}]{
                \texttt{\tiny{[GM$|$GM]}}
                \texttt{SPRICH: Das klingt vernünftig! Ich denke, es ist eine gute Idee, Konflikte zu vermeiden, indem du dein P nach (5,2) bewegst. Ich werde mein P von (1,1) nach (5,1) bewegen. Dann kann ich mein A von (3,1) nach (1,1) bewegen. Bitte gib mir Feedback, ob alles korrekt ist.} \\
            }
        }
    }
    & & \\ \\

    \theutterance \stepcounter{utterance}  
    & & \multicolumn{4}{p{0.6\linewidth}}{
        \cellcolor[rgb]{0.9,0.9,0.9}{
            \makecell[{{p{\linewidth}}}]{
                \texttt{\tiny{[P1$\langle$GM]}}
                \texttt{Deine Nachricht wurde an deinen Mitspieler übermittelt.} \\
\\ 
\\ 
\texttt{Ihr seid gerade bei Runde 4 von maximal 28.} \\
\texttt{Ihr habt gemeinsam 2 von maximal 16 Strafpunkten angesammelt.} \\
\texttt{Dein Mitspieler hat dir diese Nachricht geschickt:} \\
\\ 
\\ 
\texttt{"Das klingt vernünftig! Ich denke, es ist eine gute Idee, Konflikte zu vermeiden, indem du dein P nach (5,2) bewegst. Ich werde mein P von (1,1) nach (5,1) bewegen. Dann kann ich mein A von (3,1) nach (1,1) bewegen. Bitte gib mir Feedback, ob alles korrekt ist."} \\
\\ 
\\ 
\texttt{Wie lautet dein nächster Befehl?} \\
            }
        }
    }
    & \\ \\

    \theutterance \stepcounter{utterance}  
    & \multicolumn{4}{p{0.6\linewidth}}{
        \cellcolor[rgb]{0.9,0.9,0.9}{
            \makecell[{{p{\linewidth}}}]{
                \texttt{\tiny{[P1$\rangle$GM]}}
                \texttt{`VERSCHIEBE: P, (5,2)`} \\
            }
        }
    }
    & & \\ \\

    \theutterance \stepcounter{utterance}  
    & & & \multicolumn{2}{p{0.3\linewidth}}{
        \cellcolor[rgb]{0.9,0.9,0.9}{
            \makecell[{{p{\linewidth}}}]{
                \texttt{\tiny{[GM$|$GM]}}
                \texttt{`VERSCHIEBE: P, (5,2)`} \\
            }
        }
    }
    & & \\ \\

    \theutterance \stepcounter{utterance}  
    & & & \multicolumn{2}{p{0.3\linewidth}}{
        \cellcolor[rgb]{0.9,0.9,0.9}{
            \makecell[{{p{\linewidth}}}]{
                \texttt{\tiny{[GM$|$GM]}}
                \texttt{Strafpunkt: (5,2) ist nicht leer, sondern enthält '─'.} \\
\\ 
\texttt{Ihr habt gemeinsam 3 von maximal 16 Strafpunkten angesammelt.} \\
\texttt{Bitte versuche es noch mal!} \\
            }
        }
    }
    & & \\ \\

    \theutterance \stepcounter{utterance}  
    & & \multicolumn{4}{p{0.6\linewidth}}{
        \cellcolor[rgb]{0.9,0.9,0.9}{
            \makecell[{{p{\linewidth}}}]{
                \texttt{\tiny{[P1$\langle$GM]}}
                \texttt{Strafpunkt: (5,2) ist nicht leer, sondern enthält '─'.} \\
\\ 
\texttt{Ihr habt gemeinsam 3 von maximal 16 Strafpunkten angesammelt.} \\
\texttt{Bitte versuche es noch mal!} \\
            }
        }
    }
    & \\ \\

    \theutterance \stepcounter{utterance}  
    & \multicolumn{4}{p{0.6\linewidth}}{
        \cellcolor[rgb]{0.9,0.9,0.9}{
            \makecell[{{p{\linewidth}}}]{
                \texttt{\tiny{[P1$\rangle$GM]}}
                \texttt{`VERSCHIEBE: P, (5,4)`} \\
            }
        }
    }
    & & \\ \\

    \theutterance \stepcounter{utterance}  
    & & & \multicolumn{2}{p{0.3\linewidth}}{
        \cellcolor[rgb]{0.9,0.9,0.9}{
            \makecell[{{p{\linewidth}}}]{
                \texttt{\tiny{[GM$|$GM]}}
                \texttt{`VERSCHIEBE: P, (5,4)`} \\
            }
        }
    }
    & & \\ \\

    \theutterance \stepcounter{utterance}  
    & & & \multicolumn{2}{p{0.3\linewidth}}{
        \cellcolor[rgb]{0.9,0.9,0.9}{
            \makecell[{{p{\linewidth}}}]{
                \texttt{\tiny{[GM$|$GM]}}
                \texttt{Strafpunkt: (5,4) ist nicht leer, sondern enthält 'D'.} \\
\\ 
\texttt{Ihr habt gemeinsam 4 von maximal 16 Strafpunkten angesammelt.} \\
\texttt{Bitte versuche es noch mal!} \\
            }
        }
    }
    & & \\ \\

    \theutterance \stepcounter{utterance}  
    & & \multicolumn{4}{p{0.6\linewidth}}{
        \cellcolor[rgb]{0.9,0.9,0.9}{
            \makecell[{{p{\linewidth}}}]{
                \texttt{\tiny{[P1$\langle$GM]}}
                \texttt{Strafpunkt: (5,4) ist nicht leer, sondern enthält 'D'.} \\
\\ 
\texttt{Ihr habt gemeinsam 4 von maximal 16 Strafpunkten angesammelt.} \\
\texttt{Bitte versuche es noch mal!} \\
            }
        }
    }
    & \\ \\

    \theutterance \stepcounter{utterance}  
    & \multicolumn{4}{p{0.6\linewidth}}{
        \cellcolor[rgb]{0.9,0.9,0.9}{
            \makecell[{{p{\linewidth}}}]{
                \texttt{\tiny{[P1$\rangle$GM]}}
                \texttt{`SPRICH: Okay, ich muss mich umsehen. Mein P ist in (1,4) und ich brauche einen freien Platz. Ich schlage vor, dass ich mein P nach (2,4) bewege, wenn das dafür frei ist. Bitte gib mir Feedback, ob das okay ist.`} \\
            }
        }
    }
    & & \\ \\

    \theutterance \stepcounter{utterance}  
    & & & \multicolumn{2}{p{0.3\linewidth}}{
        \cellcolor[rgb]{0.9,0.9,0.9}{
            \makecell[{{p{\linewidth}}}]{
                \texttt{\tiny{[GM$|$GM]}}
                \texttt{`SPRICH: Okay, ich muss mich umsehen. Mein P ist in (1,4) und ich brauche einen freien Platz. Ich schlage vor, dass ich mein P nach (2,4) bewege, wenn das dafür frei ist. Bitte gib mir Feedback, ob das okay ist.`} \\
            }
        }
    }
    & & \\ \\

    \theutterance \stepcounter{utterance}  
    & & \multicolumn{4}{p{0.6\linewidth}}{
        \cellcolor[rgb]{0.9,0.9,0.9}{
            \makecell[{{p{\linewidth}}}]{
                \texttt{\tiny{[GM$\rangle$P2]}}
                \texttt{Deine Nachricht wurde an deinen Mitspieler übermittelt.} \\
\\ 
\\ 
\texttt{Ihr seid gerade bei Runde 5 von maximal 28.} \\
\texttt{Ihr habt gemeinsam 4 von maximal 16 Strafpunkten angesammelt.} \\
\texttt{Dein Mitspieler hat dir diese Nachricht geschickt:} \\
\\ 
\\ 
\texttt{"Okay, ich muss mich umsehen. Mein P ist in (1,4) und ich brauche einen freien Platz. Ich schlage vor, dass ich mein P nach (2,4) bewege, wenn das dafür frei ist. Bitte gib mir Feedback, ob das okay ist."} \\
\\ 
\\ 
\texttt{Wie lautet dein nächster Befehl?} \\
            }
        }
    }
    & \\ \\

    \theutterance \stepcounter{utterance}  
    & & & \multicolumn{4}{p{0.6\linewidth}}{
        \cellcolor[rgb]{0.9,0.9,0.9}{
            \makecell[{{p{\linewidth}}}]{
                \texttt{\tiny{[GM$\langle$P2]}}
                \texttt{SPRICH: Ich denke, das ist okay! Auf meinem Spielbrett ist (2,4) frei, also kannst du dein P dorthin bewegen. Bitte beachte jedoch, dass wir ursprünglich geplant hatten, die Objekte in alphabetischer Reihenfolge anzuordnen, beginnend von oben links. Lass uns nicht den Fokus verlieren!} \\
            }
        }
    }
     \\ \\

    \theutterance \stepcounter{utterance}  
    & & & \multicolumn{2}{p{0.3\linewidth}}{
        \cellcolor[rgb]{0.9,0.9,0.9}{
            \makecell[{{p{\linewidth}}}]{
                \texttt{\tiny{[GM$|$GM]}}
                \texttt{SPRICH: Ich denke, das ist okay! Auf meinem Spielbrett ist (2,4) frei, also kannst du dein P dorthin bewegen. Bitte beachte jedoch, dass wir ursprünglich geplant hatten, die Objekte in alphabetischer Reihenfolge anzuordnen, beginnend von oben links. Lass uns nicht den Fokus verlieren!} \\
            }
        }
    }
    & & \\ \\

    \theutterance \stepcounter{utterance}  
    & & \multicolumn{4}{p{0.6\linewidth}}{
        \cellcolor[rgb]{0.9,0.9,0.9}{
            \makecell[{{p{\linewidth}}}]{
                \texttt{\tiny{[P1$\langle$GM]}}
                \texttt{Deine Nachricht wurde an deinen Mitspieler übermittelt.} \\
\\ 
\\ 
\texttt{Ihr seid gerade bei Runde 5 von maximal 28.} \\
\texttt{Ihr habt gemeinsam 4 von maximal 16 Strafpunkten angesammelt.} \\
\texttt{Dein Mitspieler hat dir diese Nachricht geschickt:} \\
\\ 
\\ 
\texttt{"Ich denke, das ist okay! Auf meinem Spielbrett ist (2,4) frei, also kannst du dein P dorthin bewegen. Bitte beachte jedoch, dass wir ursprünglich geplant hatten, die Objekte in alphabetischer Reihenfolge anzuordnen, beginnend von oben links. Lass uns nicht den Fokus verlieren!"} \\
\\ 
\\ 
\texttt{Wie lautet dein nächster Befehl?} \\
            }
        }
    }
    & \\ \\

    \theutterance \stepcounter{utterance}  
    & \multicolumn{4}{p{0.6\linewidth}}{
        \cellcolor[rgb]{0.9,0.9,0.9}{
            \makecell[{{p{\linewidth}}}]{
                \texttt{\tiny{[P1$\rangle$GM]}}
                \texttt{`VERSCHIEBE: P, (2,4)`} \\
            }
        }
    }
    & & \\ \\

    \theutterance \stepcounter{utterance}  
    & & & \multicolumn{2}{p{0.3\linewidth}}{
        \cellcolor[rgb]{0.9,0.9,0.9}{
            \makecell[{{p{\linewidth}}}]{
                \texttt{\tiny{[GM$|$GM]}}
                \texttt{`VERSCHIEBE: P, (2,4)`} \\
            }
        }
    }
    & & \\ \\

    \theutterance \stepcounter{utterance}  
    & & & \multicolumn{2}{p{0.3\linewidth}}{
        \cellcolor[rgb]{0.9,0.9,0.9}{
            \makecell[{{p{\linewidth}}}]{
                \texttt{\tiny{[GM$|$GM]}}
                \texttt{Strafpunkt: (2,4) ist nicht leer, sondern enthält '├'.} \\
\\ 
\texttt{Ihr habt gemeinsam 5 von maximal 16 Strafpunkten angesammelt.} \\
\texttt{Bitte versuche es noch mal!} \\
            }
        }
    }
    & & \\ \\

    \theutterance \stepcounter{utterance}  
    & & \multicolumn{4}{p{0.6\linewidth}}{
        \cellcolor[rgb]{0.9,0.9,0.9}{
            \makecell[{{p{\linewidth}}}]{
                \texttt{\tiny{[P1$\langle$GM]}}
                \texttt{Strafpunkt: (2,4) ist nicht leer, sondern enthält '├'.} \\
\\ 
\texttt{Ihr habt gemeinsam 5 von maximal 16 Strafpunkten angesammelt.} \\
\texttt{Bitte versuche es noch mal!} \\
            }
        }
    }
    & \\ \\

    \theutterance \stepcounter{utterance}  
    & \multicolumn{4}{p{0.6\linewidth}}{
        \cellcolor[rgb]{0.9,0.9,0.9}{
            \makecell[{{p{\linewidth}}}]{
                \texttt{\tiny{[P1$\rangle$GM]}}
                \texttt{`SPRICH: Mist! Ich habe wieder einen Fehler gemacht. Ich werde mein P von (1,4) nach (3,3) bewegen, da das ein freier Platz ist. Bitte gib mir Feedback, ob das okay ist.`} \\
            }
        }
    }
    & & \\ \\

    \theutterance \stepcounter{utterance}  
    & & & \multicolumn{2}{p{0.3\linewidth}}{
        \cellcolor[rgb]{0.9,0.9,0.9}{
            \makecell[{{p{\linewidth}}}]{
                \texttt{\tiny{[GM$|$GM]}}
                \texttt{`SPRICH: Mist! Ich habe wieder einen Fehler gemacht. Ich werde mein P von (1,4) nach (3,3) bewegen, da das ein freier Platz ist. Bitte gib mir Feedback, ob das okay ist.`} \\
            }
        }
    }
    & & \\ \\

    \theutterance \stepcounter{utterance}  
    & & \multicolumn{4}{p{0.6\linewidth}}{
        \cellcolor[rgb]{0.9,0.9,0.9}{
            \makecell[{{p{\linewidth}}}]{
                \texttt{\tiny{[GM$\rangle$P2]}}
                \texttt{Deine Nachricht wurde an deinen Mitspieler übermittelt.} \\
\\ 
\\ 
\texttt{Ihr seid gerade bei Runde 6 von maximal 28.} \\
\texttt{Ihr habt gemeinsam 5 von maximal 16 Strafpunkten angesammelt.} \\
\texttt{Dein Mitspieler hat dir diese Nachricht geschickt:} \\
\\ 
\\ 
\texttt{"Mist! Ich habe wieder einen Fehler gemacht. Ich werde mein P von (1,4) nach (3,3) bewegen, da das ein freier Platz ist. Bitte gib mir Feedback, ob das okay ist."} \\
\\ 
\\ 
\texttt{Wie lautet dein nächster Befehl?} \\
            }
        }
    }
    & \\ \\

    \theutterance \stepcounter{utterance}  
    & & & \multicolumn{4}{p{0.6\linewidth}}{
        \cellcolor[rgb]{0.9,0.9,0.9}{
            \makecell[{{p{\linewidth}}}]{
                \texttt{\tiny{[GM$\langle$P2]}}
                \texttt{SPRICH: Ich denke, das ist okay, aber ich mache mir Sorgen, dass wir vom ursprünglichen Plan abweichen. Lass uns versuchen, wieder auf Kurs zu kommen. Ich schlage vor, dass wir uns auf die Anordnung der Objekte in alphabetischer Reihenfolge konzentrieren. Mein A ist noch nicht an der richtigen Position. Ich werde mein P von (1,1) nach (5,1) bewegen, um Platz für A zu machen. Bitte bestätige, ob das korrekt ist.} \\
            }
        }
    }
     \\ \\

    \theutterance \stepcounter{utterance}  
    & & & \multicolumn{2}{p{0.3\linewidth}}{
        \cellcolor[rgb]{0.9,0.9,0.9}{
            \makecell[{{p{\linewidth}}}]{
                \texttt{\tiny{[GM$|$GM]}}
                \texttt{SPRICH: Ich denke, das ist okay, aber ich mache mir Sorgen, dass wir vom ursprünglichen Plan abweichen. Lass uns versuchen, wieder auf Kurs zu kommen. Ich schlage vor, dass wir uns auf die Anordnung der Objekte in alphabetischer Reihenfolge konzentrieren. Mein A ist noch nicht an der richtigen Position. Ich werde mein P von (1,1) nach (5,1) bewegen, um Platz für A zu machen. Bitte bestätige, ob das korrekt ist.} \\
            }
        }
    }
    & & \\ \\

    \theutterance \stepcounter{utterance}  
    & & \multicolumn{4}{p{0.6\linewidth}}{
        \cellcolor[rgb]{0.9,0.9,0.9}{
            \makecell[{{p{\linewidth}}}]{
                \texttt{\tiny{[P1$\langle$GM]}}
                \texttt{Deine Nachricht wurde an deinen Mitspieler übermittelt.} \\
\\ 
\\ 
\texttt{Ihr seid gerade bei Runde 6 von maximal 28.} \\
\texttt{Ihr habt gemeinsam 5 von maximal 16 Strafpunkten angesammelt.} \\
\texttt{Dein Mitspieler hat dir diese Nachricht geschickt:} \\
\\ 
\\ 
\texttt{"Ich denke, das ist okay, aber ich mache mir Sorgen, dass wir vom ursprünglichen Plan abweichen. Lass uns versuchen, wieder auf Kurs zu kommen. Ich schlage vor, dass wir uns auf die Anordnung der Objekte in alphabetischer Reihenfolge konzentrieren. Mein A ist noch nicht an der richtigen Position. Ich werde mein P von (1,1) nach (5,1) bewegen, um Platz für A zu machen. Bitte bestätige, ob das korrekt ist."} \\
\\ 
\\ 
\texttt{Wie lautet dein nächster Befehl?} \\
            }
        }
    }
    & \\ \\

    \theutterance \stepcounter{utterance}  
    & \multicolumn{4}{p{0.6\linewidth}}{
        \cellcolor[rgb]{0.9,0.9,0.9}{
            \makecell[{{p{\linewidth}}}]{
                \texttt{\tiny{[P1$\rangle$GM]}}
                \texttt{`SPRICH: Ja, das klingt gut! Ich denke, wir sollten uns wieder auf den ursprünglichen Plan konzentrieren. Mein A ist in (1,1) und ich werde es nicht bewegen, da es bereits an der richtigen Position ist. Ich warte auf deine nächste Aktion und dann können wir weitermachen. Bitte bestätige, ob dein P jetzt in (5,1) ist und ob du bereit bist, dein A zu bewegen.`} \\
            }
        }
    }
    & & \\ \\

    \theutterance \stepcounter{utterance}  
    & & & \multicolumn{2}{p{0.3\linewidth}}{
        \cellcolor[rgb]{0.9,0.9,0.9}{
            \makecell[{{p{\linewidth}}}]{
                \texttt{\tiny{[GM$|$GM]}}
                \texttt{`SPRICH: Ja, das klingt gut! Ich denke, wir sollten uns wieder auf den ursprünglichen Plan konzentrieren. Mein A ist in (1,1) und ich werde es nicht bewegen, da es bereits an der richtigen Position ist. Ich warte auf deine nächste Aktion und dann können wir weitermachen. Bitte bestätige, ob dein P jetzt in (5,1) ist und ob du bereit bist, dein A zu bewegen.`} \\
            }
        }
    }
    & & \\ \\

    \theutterance \stepcounter{utterance}  
    & & \multicolumn{4}{p{0.6\linewidth}}{
        \cellcolor[rgb]{0.9,0.9,0.9}{
            \makecell[{{p{\linewidth}}}]{
                \texttt{\tiny{[GM$\rangle$P2]}}
                \texttt{Deine Nachricht wurde an deinen Mitspieler übermittelt.} \\
\\ 
\\ 
\texttt{Ihr seid gerade bei Runde 7 von maximal 28.} \\
\texttt{Ihr habt gemeinsam 5 von maximal 16 Strafpunkten angesammelt.} \\
\texttt{Dein Mitspieler hat dir diese Nachricht geschickt:} \\
\\ 
\\ 
\texttt{"Ja, das klingt gut! Ich denke, wir sollten uns wieder auf den ursprünglichen Plan konzentrieren. Mein A ist in (1,1) und ich werde es nicht bewegen, da es bereits an der richtigen Position ist. Ich warte auf deine nächste Aktion und dann können wir weitermachen. Bitte bestätige, ob dein P jetzt in (5,1) ist und ob du bereit bist, dein A zu bewegen."} \\
\\ 
\\ 
\texttt{Wie lautet dein nächster Befehl?} \\
            }
        }
    }
    & \\ \\

    \theutterance \stepcounter{utterance}  
    & & & \multicolumn{4}{p{0.6\linewidth}}{
        \cellcolor[rgb]{0.9,0.9,0.9}{
            \makecell[{{p{\linewidth}}}]{
                \texttt{\tiny{[GM$\langle$P2]}}
                \texttt{SPRICH: Ja, mein P ist jetzt in (5,1) und ich bin bereit, mein A zu bewegen. Ich werde mein A von (3,1) nach (1,1) bewegen, um es an die richtige Position zu bringen. Bitte gib mir Feedback, ob alles korrekt ist.} \\
            }
        }
    }
     \\ \\

    \theutterance \stepcounter{utterance}  
    & & & \multicolumn{2}{p{0.3\linewidth}}{
        \cellcolor[rgb]{0.9,0.9,0.9}{
            \makecell[{{p{\linewidth}}}]{
                \texttt{\tiny{[GM$|$GM]}}
                \texttt{SPRICH: Ja, mein P ist jetzt in (5,1) und ich bin bereit, mein A zu bewegen. Ich werde mein A von (3,1) nach (1,1) bewegen, um es an die richtige Position zu bringen. Bitte gib mir Feedback, ob alles korrekt ist.} \\
            }
        }
    }
    & & \\ \\

    \theutterance \stepcounter{utterance}  
    & & \multicolumn{4}{p{0.6\linewidth}}{
        \cellcolor[rgb]{0.9,0.9,0.9}{
            \makecell[{{p{\linewidth}}}]{
                \texttt{\tiny{[P1$\langle$GM]}}
                \texttt{Deine Nachricht wurde an deinen Mitspieler übermittelt.} \\
\\ 
\\ 
\texttt{Ihr seid gerade bei Runde 7 von maximal 28.} \\
\texttt{Ihr habt gemeinsam 5 von maximal 16 Strafpunkten angesammelt.} \\
\texttt{Dein Mitspieler hat dir diese Nachricht geschickt:} \\
\\ 
\\ 
\texttt{"Ja, mein P ist jetzt in (5,1) und ich bin bereit, mein A zu bewegen. Ich werde mein A von (3,1) nach (1,1) bewegen, um es an die richtige Position zu bringen. Bitte gib mir Feedback, ob alles korrekt ist."} \\
\\ 
\\ 
\texttt{Wie lautet dein nächster Befehl?} \\
            }
        }
    }
    & \\ \\

    \theutterance \stepcounter{utterance}  
    & \multicolumn{4}{p{0.6\linewidth}}{
        \cellcolor[rgb]{0.9,0.9,0.9}{
            \makecell[{{p{\linewidth}}}]{
                \texttt{\tiny{[P1$\rangle$GM]}}
                \texttt{`SPRICH: Alles sieht gut aus! Mein A ist bereits in (1,1) und bleibt dort. Ich denke, wir sollten jetzt mit dem nächsten Objekt, D, fortfahren. Ich werde mein D von (4,4) nach (2,1) bewegen, um es an die richtige Position zu bringen. Bitte bestätige, ob das korrekt ist.`} \\
            }
        }
    }
    & & \\ \\

    \theutterance \stepcounter{utterance}  
    & & & \multicolumn{2}{p{0.3\linewidth}}{
        \cellcolor[rgb]{0.9,0.9,0.9}{
            \makecell[{{p{\linewidth}}}]{
                \texttt{\tiny{[GM$|$GM]}}
                \texttt{`SPRICH: Alles sieht gut aus! Mein A ist bereits in (1,1) und bleibt dort. Ich denke, wir sollten jetzt mit dem nächsten Objekt, D, fortfahren. Ich werde mein D von (4,4) nach (2,1) bewegen, um es an die richtige Position zu bringen. Bitte bestätige, ob das korrekt ist.`} \\
            }
        }
    }
    & & \\ \\

    \theutterance \stepcounter{utterance}  
    & & \multicolumn{4}{p{0.6\linewidth}}{
        \cellcolor[rgb]{0.9,0.9,0.9}{
            \makecell[{{p{\linewidth}}}]{
                \texttt{\tiny{[GM$\rangle$P2]}}
                \texttt{Deine Nachricht wurde an deinen Mitspieler übermittelt.} \\
\\ 
\\ 
\texttt{Ihr seid gerade bei Runde 8 von maximal 28.} \\
\texttt{Ihr habt gemeinsam 5 von maximal 16 Strafpunkten angesammelt.} \\
\texttt{Dein Mitspieler hat dir diese Nachricht geschickt:} \\
\\ 
\\ 
\texttt{"Alles sieht gut aus! Mein A ist bereits in (1,1) und bleibt dort. Ich denke, wir sollten jetzt mit dem nächsten Objekt, D, fortfahren. Ich werde mein D von (4,4) nach (2,1) bewegen, um es an die richtige Position zu bringen. Bitte bestätige, ob das korrekt ist."} \\
\\ 
\\ 
\texttt{Wie lautet dein nächster Befehl?} \\
            }
        }
    }
    & \\ \\

    \theutterance \stepcounter{utterance}  
    & & & \multicolumn{4}{p{0.6\linewidth}}{
        \cellcolor[rgb]{0.9,0.9,0.9}{
            \makecell[{{p{\linewidth}}}]{
                \texttt{\tiny{[GM$\langle$P2]}}
                \texttt{SPRICH: Ja, das klingt korrekt! Mein D ist in (2,3) und ich werde es nach (2,1) bewegen, um es an die richtige Position zu bringen. Bitte gib mir Feedback, ob alles korrekt ist und ob wir mit dem nächsten Objekt, M, fortfahren können.} \\
            }
        }
    }
     \\ \\

    \theutterance \stepcounter{utterance}  
    & & & \multicolumn{2}{p{0.3\linewidth}}{
        \cellcolor[rgb]{0.9,0.9,0.9}{
            \makecell[{{p{\linewidth}}}]{
                \texttt{\tiny{[GM$|$GM]}}
                \texttt{SPRICH: Ja, das klingt korrekt! Mein D ist in (2,3) und ich werde es nach (2,1) bewegen, um es an die richtige Position zu bringen. Bitte gib mir Feedback, ob alles korrekt ist und ob wir mit dem nächsten Objekt, M, fortfahren können.} \\
            }
        }
    }
    & & \\ \\

    \theutterance \stepcounter{utterance}  
    & & \multicolumn{4}{p{0.6\linewidth}}{
        \cellcolor[rgb]{0.9,0.9,0.9}{
            \makecell[{{p{\linewidth}}}]{
                \texttt{\tiny{[P1$\langle$GM]}}
                \texttt{Deine Nachricht wurde an deinen Mitspieler übermittelt.} \\
\\ 
\\ 
\texttt{Ihr seid gerade bei Runde 8 von maximal 28.} \\
\texttt{Ihr habt gemeinsam 5 von maximal 16 Strafpunkten angesammelt.} \\
\texttt{Dein Mitspieler hat dir diese Nachricht geschickt:} \\
\\ 
\\ 
\texttt{"Ja, das klingt korrekt! Mein D ist in (2,3) und ich werde es nach (2,1) bewegen, um es an die richtige Position zu bringen. Bitte gib mir Feedback, ob alles korrekt ist und ob wir mit dem nächsten Objekt, M, fortfahren können."} \\
\\ 
\\ 
\texttt{Wie lautet dein nächster Befehl?} \\
            }
        }
    }
    & \\ \\

    \theutterance \stepcounter{utterance}  
    & \multicolumn{4}{p{0.6\linewidth}}{
        \cellcolor[rgb]{0.9,0.9,0.9}{
            \makecell[{{p{\linewidth}}}]{
                \texttt{\tiny{[P1$\rangle$GM]}}
                \texttt{`SPRICH: Alles sieht gut aus! Mein D ist in (4,4) und ich werde es nach (2,1) bewegen, um es an die richtige Position zu bringen. Aber ich denke, wir sollten zuerst mein M bewegen, da es in (5,5) ist und nicht in der richtigen Reihenfolge. Ich schlage vor, dass wir mein M nach (3,1) bewegen, um es an die richtige Position zu bringen. Bitte bestätige, ob das korrekt ist.`} \\
            }
        }
    }
    & & \\ \\

    \theutterance \stepcounter{utterance}  
    & & & \multicolumn{2}{p{0.3\linewidth}}{
        \cellcolor[rgb]{0.9,0.9,0.9}{
            \makecell[{{p{\linewidth}}}]{
                \texttt{\tiny{[GM$|$GM]}}
                \texttt{`SPRICH: Alles sieht gut aus! Mein D ist in (4,4) und ich werde es nach (2,1) bewegen, um es an die richtige Position zu bringen. Aber ich denke, wir sollten zuerst mein M bewegen, da es in (5,5) ist und nicht in der richtigen Reihenfolge. Ich schlage vor, dass wir mein M nach (3,1) bewegen, um es an die richtige Position zu bringen. Bitte bestätige, ob das korrekt ist.`} \\
            }
        }
    }
    & & \\ \\

    \theutterance \stepcounter{utterance}  
    & & \multicolumn{4}{p{0.6\linewidth}}{
        \cellcolor[rgb]{0.9,0.9,0.9}{
            \makecell[{{p{\linewidth}}}]{
                \texttt{\tiny{[GM$\rangle$P2]}}
                \texttt{Deine Nachricht wurde an deinen Mitspieler übermittelt.} \\
\\ 
\\ 
\texttt{Ihr seid gerade bei Runde 9 von maximal 28.} \\
\texttt{Ihr habt gemeinsam 5 von maximal 16 Strafpunkten angesammelt.} \\
\texttt{Dein Mitspieler hat dir diese Nachricht geschickt:} \\
\\ 
\\ 
\texttt{"Alles sieht gut aus! Mein D ist in (4,4) und ich werde es nach (2,1) bewegen, um es an die richtige Position zu bringen. Aber ich denke, wir sollten zuerst mein M bewegen, da es in (5,5) ist und nicht in der richtigen Reihenfolge. Ich schlage vor, dass wir mein M nach (3,1) bewegen, um es an die richtige Position zu bringen. Bitte bestätige, ob das korrekt ist."} \\
\\ 
\\ 
\texttt{Wie lautet dein nächster Befehl?} \\
            }
        }
    }
    & \\ \\

    \theutterance \stepcounter{utterance}  
    & & & \multicolumn{4}{p{0.6\linewidth}}{
        \cellcolor[rgb]{0.9,0.9,0.9}{
            \makecell[{{p{\linewidth}}}]{
                \texttt{\tiny{[GM$\langle$P2]}}
                \texttt{SPRICH: Ich denke, das ist nicht korrekt. Mein M ist in (7,7) und ich denke, es sollte nach (3,1) bewegt werden, um es an die richtige Position zu bringen, aber ich muss zuerst sicherstellen, dass (3,1) frei ist. Ich schlage vor, dass wir mein A, das sich momentan in (3,1) befindet, an eine andere Position bewegen, bevor wir mein M dorthin bewegen. Bitte gib mir Feedback.} \\
            }
        }
    }
     \\ \\

    \theutterance \stepcounter{utterance}  
    & & & \multicolumn{2}{p{0.3\linewidth}}{
        \cellcolor[rgb]{0.9,0.9,0.9}{
            \makecell[{{p{\linewidth}}}]{
                \texttt{\tiny{[GM$|$GM]}}
                \texttt{SPRICH: Ich denke, das ist nicht korrekt. Mein M ist in (7,7) und ich denke, es sollte nach (3,1) bewegt werden, um es an die richtige Position zu bringen, aber ich muss zuerst sicherstellen, dass (3,1) frei ist. Ich schlage vor, dass wir mein A, das sich momentan in (3,1) befindet, an eine andere Position bewegen, bevor wir mein M dorthin bewegen. Bitte gib mir Feedback.} \\
            }
        }
    }
    & & \\ \\

    \theutterance \stepcounter{utterance}  
    & & \multicolumn{4}{p{0.6\linewidth}}{
        \cellcolor[rgb]{0.9,0.9,0.9}{
            \makecell[{{p{\linewidth}}}]{
                \texttt{\tiny{[P1$\langle$GM]}}
                \texttt{Deine Nachricht wurde an deinen Mitspieler übermittelt.} \\
\\ 
\\ 
\texttt{Ihr seid gerade bei Runde 9 von maximal 28.} \\
\texttt{Ihr habt gemeinsam 5 von maximal 16 Strafpunkten angesammelt.} \\
\texttt{Dein Mitspieler hat dir diese Nachricht geschickt:} \\
\\ 
\\ 
\texttt{"Ich denke, das ist nicht korrekt. Mein M ist in (7,7) und ich denke, es sollte nach (3,1) bewegt werden, um es an die richtige Position zu bringen, aber ich muss zuerst sicherstellen, dass (3,1) frei ist. Ich schlage vor, dass wir mein A, das sich momentan in (3,1) befindet, an eine andere Position bewegen, bevor wir mein M dorthin bewegen. Bitte gib mir Feedback."} \\
\\ 
\\ 
\texttt{Wie lautet dein nächster Befehl?} \\
            }
        }
    }
    & \\ \\

    \theutterance \stepcounter{utterance}  
    & \multicolumn{4}{p{0.6\linewidth}}{
        \cellcolor[rgb]{0.9,0.9,0.9}{
            \makecell[{{p{\linewidth}}}]{
                \texttt{\tiny{[P1$\rangle$GM]}}
                \texttt{`SPRICH: Ich denke, da ist ein Missverständnis! Mein A ist in (1,1) und nicht in (3,1). Ich denke, wir sollten uns auf die richtigen Positionen konzentrieren. Mein M ist in (5,5) und ich werde es nach (3,1) bewegen, um es an die richtige Position zu bringen. Bitte bestätige, ob das korrekt ist und ob (3,1) auf deinem Spielbrett frei ist.`} \\
            }
        }
    }
    & & \\ \\

    \theutterance \stepcounter{utterance}  
    & & & \multicolumn{2}{p{0.3\linewidth}}{
        \cellcolor[rgb]{0.9,0.9,0.9}{
            \makecell[{{p{\linewidth}}}]{
                \texttt{\tiny{[GM$|$GM]}}
                \texttt{`SPRICH: Ich denke, da ist ein Missverständnis! Mein A ist in (1,1) und nicht in (3,1). Ich denke, wir sollten uns auf die richtigen Positionen konzentrieren. Mein M ist in (5,5) und ich werde es nach (3,1) bewegen, um es an die richtige Position zu bringen. Bitte bestätige, ob das korrekt ist und ob (3,1) auf deinem Spielbrett frei ist.`} \\
            }
        }
    }
    & & \\ \\

    \theutterance \stepcounter{utterance}  
    & & \multicolumn{4}{p{0.6\linewidth}}{
        \cellcolor[rgb]{0.9,0.9,0.9}{
            \makecell[{{p{\linewidth}}}]{
                \texttt{\tiny{[GM$\rangle$P2]}}
                \texttt{Deine Nachricht wurde an deinen Mitspieler übermittelt.} \\
\\ 
\\ 
\texttt{Ihr seid gerade bei Runde 10 von maximal 28.} \\
\texttt{Ihr habt gemeinsam 5 von maximal 16 Strafpunkten angesammelt.} \\
\texttt{Dein Mitspieler hat dir diese Nachricht geschickt:} \\
\\ 
\\ 
\texttt{"Ich denke, da ist ein Missverständnis! Mein A ist in (1,1) und nicht in (3,1). Ich denke, wir sollten uns auf die richtigen Positionen konzentrieren. Mein M ist in (5,5) und ich werde es nach (3,1) bewegen, um es an die richtige Position zu bringen. Bitte bestätige, ob das korrekt ist und ob (3,1) auf deinem Spielbrett frei ist."} \\
\\ 
\\ 
\texttt{Wie lautet dein nächster Befehl?} \\
            }
        }
    }
    & \\ \\

    \theutterance \stepcounter{utterance}  
    & & & \multicolumn{4}{p{0.6\linewidth}}{
        \cellcolor[rgb]{0.9,0.9,0.9}{
            \makecell[{{p{\linewidth}}}]{
                \texttt{\tiny{[GM$\langle$P2]}}
                \texttt{SPRICH: Ich entschuldige das Missverständnis! Mein A ist tatsächlich in (1,1) und nicht in (3,1). Mein (3,1) ist frei, also kannst du dein M nach (3,1) bewegen. Ich denke, das ist korrekt. Bitte gib mir Feedback und lass uns weitermachen mit dem nächsten Objekt.} \\
            }
        }
    }
     \\ \\

    \theutterance \stepcounter{utterance}  
    & & & \multicolumn{2}{p{0.3\linewidth}}{
        \cellcolor[rgb]{0.9,0.9,0.9}{
            \makecell[{{p{\linewidth}}}]{
                \texttt{\tiny{[GM$|$GM]}}
                \texttt{SPRICH: Ich entschuldige das Missverständnis! Mein A ist tatsächlich in (1,1) und nicht in (3,1). Mein (3,1) ist frei, also kannst du dein M nach (3,1) bewegen. Ich denke, das ist korrekt. Bitte gib mir Feedback und lass uns weitermachen mit dem nächsten Objekt.} \\
            }
        }
    }
    & & \\ \\

    \theutterance \stepcounter{utterance}  
    & & \multicolumn{4}{p{0.6\linewidth}}{
        \cellcolor[rgb]{0.9,0.9,0.9}{
            \makecell[{{p{\linewidth}}}]{
                \texttt{\tiny{[P1$\langle$GM]}}
                \texttt{Deine Nachricht wurde an deinen Mitspieler übermittelt.} \\
\\ 
\\ 
\texttt{Ihr seid gerade bei Runde 10 von maximal 28.} \\
\texttt{Ihr habt gemeinsam 5 von maximal 16 Strafpunkten angesammelt.} \\
\texttt{Dein Mitspieler hat dir diese Nachricht geschickt:} \\
\\ 
\\ 
\texttt{"Ich entschuldige das Missverständnis! Mein A ist tatsächlich in (1,1) und nicht in (3,1). Mein (3,1) ist frei, also kannst du dein M nach (3,1) bewegen. Ich denke, das ist korrekt. Bitte gib mir Feedback und lass uns weitermachen mit dem nächsten Objekt."} \\
\\ 
\\ 
\texttt{Wie lautet dein nächster Befehl?} \\
            }
        }
    }
    & \\ \\

    \theutterance \stepcounter{utterance}  
    & \multicolumn{4}{p{0.6\linewidth}}{
        \cellcolor[rgb]{0.9,0.9,0.9}{
            \makecell[{{p{\linewidth}}}]{
                \texttt{\tiny{[P1$\rangle$GM]}}
                \texttt{`VERSCHIEBE: M, (3,1)`} \\
            }
        }
    }
    & & \\ \\

    \theutterance \stepcounter{utterance}  
    & & & \multicolumn{2}{p{0.3\linewidth}}{
        \cellcolor[rgb]{0.9,0.9,0.9}{
            \makecell[{{p{\linewidth}}}]{
                \texttt{\tiny{[GM$|$GM]}}
                \texttt{`VERSCHIEBE: M, (3,1)`} \\
            }
        }
    }
    & & \\ \\

    \theutterance \stepcounter{utterance}  
    & & & \multicolumn{2}{p{0.3\linewidth}}{
        \cellcolor[rgb]{0.9,0.9,0.9}{
            \makecell[{{p{\linewidth}}}]{
                \texttt{\tiny{[GM$|$GM]}}
                \texttt{'M' erfolgreich nach (3,1) verschoben. Dein Raster sieht jetzt so aus:} \\
\texttt{```} \\
\texttt{ 1234567} \\
\texttt{╔═══╤═╤═╗} \\
\texttt{║◌SM│◌│A║ 1} \\
\texttt{╟─┬─┼─┴─╢ 2} \\
\texttt{║◌│◌│◌◌◌║ 3} \\
\texttt{║P├─┤D┌─╢ 4} \\
\texttt{╟─┘◌│◌│◌║ 5} \\
\texttt{║◌◌◌└┬┴─╢ 6} \\
\texttt{║◌OT◌│◌◌║ 7} \\
\texttt{╚════╧══╝} \\
\texttt{```} \\
            }
        }
    }
    & & \\ \\

    \theutterance \stepcounter{utterance}  
    & & \multicolumn{4}{p{0.6\linewidth}}{
        \cellcolor[rgb]{0.9,0.9,0.9}{
            \makecell[{{p{\linewidth}}}]{
                \texttt{\tiny{[GM$\rangle$P2]}}
                \texttt{Deine Nachricht wurde an deinen Mitspieler übermittelt.} \\
\\ 
\\ 
\texttt{Ihr seid gerade bei Runde 11 von maximal 28.} \\
\texttt{Ihr habt gemeinsam 5 von maximal 16 Strafpunkten angesammelt.} \\
\texttt{Dein Mitspieler hat ein Objekt auf seinem Raster bewegt.} \\
\\ 
\\ 
\texttt{Wie lautet dein nächster Befehl?} \\
            }
        }
    }
    & \\ \\

    \theutterance \stepcounter{utterance}  
    & & & \multicolumn{4}{p{0.6\linewidth}}{
        \cellcolor[rgb]{0.9,0.9,0.9}{
            \makecell[{{p{\linewidth}}}]{
                \texttt{\tiny{[GM$\langle$P2]}}
                \texttt{VERSCHIEBE: D, (2,1)} \\
            }
        }
    }
     \\ \\

    \theutterance \stepcounter{utterance}  
    & & & \multicolumn{2}{p{0.3\linewidth}}{
        \cellcolor[rgb]{0.9,0.9,0.9}{
            \makecell[{{p{\linewidth}}}]{
                \texttt{\tiny{[GM$|$GM]}}
                \texttt{VERSCHIEBE: D, (2,1)} \\
            }
        }
    }
    & & \\ \\

    \theutterance \stepcounter{utterance}  
    & & & \multicolumn{2}{p{0.3\linewidth}}{
        \cellcolor[rgb]{0.9,0.9,0.9}{
            \makecell[{{p{\linewidth}}}]{
                \texttt{\tiny{[GM$|$GM]}}
                \texttt{'D' erfolgreich nach (2,1) verschoben. Dein Raster sieht jetzt so aus:} \\
\texttt{```} \\
\texttt{ 1234567} \\
\texttt{╔═══╤═╤═╗} \\
\texttt{║PDA│◌│◌║ 1} \\
\texttt{╟─┬─┼─┴─╢ 2} \\
\texttt{║◌│◌│◌◌◌║ 3} \\
\texttt{║◌├─┤◌┌─╢ 4} \\
\texttt{╟─┘◌│◌│O║ 5} \\
\texttt{║◌◌◌└┬┴─╢ 6} \\
\texttt{║◌S◌T│◌M║ 7} \\
\texttt{╚════╧══╝} \\
\texttt{```} \\
            }
        }
    }
    & & \\ \\

    \theutterance \stepcounter{utterance}  
    & & \multicolumn{4}{p{0.6\linewidth}}{
        \cellcolor[rgb]{0.9,0.9,0.9}{
            \makecell[{{p{\linewidth}}}]{
                \texttt{\tiny{[P1$\langle$GM]}}
                \texttt{'M' erfolgreich nach (3,1) verschoben. Dein Raster sieht jetzt so aus:} \\
\texttt{```} \\
\texttt{ 1234567} \\
\texttt{╔═══╤═╤═╗} \\
\texttt{║◌SM│◌│A║ 1} \\
\texttt{╟─┬─┼─┴─╢ 2} \\
\texttt{║◌│◌│◌◌◌║ 3} \\
\texttt{║P├─┤D┌─╢ 4} \\
\texttt{╟─┘◌│◌│◌║ 5} \\
\texttt{║◌◌◌└┬┴─╢ 6} \\
\texttt{║◌OT◌│◌◌║ 7} \\
\texttt{╚════╧══╝} \\
\texttt{```} \\
\texttt{Ihr seid gerade bei Runde 11 von maximal 28.} \\
\texttt{Ihr habt gemeinsam 5 von maximal 16 Strafpunkten angesammelt.} \\
\texttt{Dein Mitspieler hat ein Objekt auf seinem Raster bewegt.} \\
\\ 
\\ 
\texttt{Wie lautet dein nächster Befehl?} \\
            }
        }
    }
    & \\ \\

    \theutterance \stepcounter{utterance}  
    & \multicolumn{4}{p{0.6\linewidth}}{
        \cellcolor[rgb]{0.9,0.9,0.9}{
            \makecell[{{p{\linewidth}}}]{
                \texttt{\tiny{[P1$\rangle$GM]}}
                \texttt{`SPRICH: Mein M ist jetzt in (3,1) und das sieht gut aus! Ich denke, wir sollten jetzt mit dem nächsten Objekt, O, fortfahren. Mein O ist in (2,7) und ich werde es nach (4,1) bewegen, um es an die richtige Position zu bringen. Bitte bestätige, ob das korrekt ist und ob du bereit bist, dein O zu bewegen.`} \\
            }
        }
    }
    & & \\ \\

    \theutterance \stepcounter{utterance}  
    & & & \multicolumn{2}{p{0.3\linewidth}}{
        \cellcolor[rgb]{0.9,0.9,0.9}{
            \makecell[{{p{\linewidth}}}]{
                \texttt{\tiny{[GM$|$GM]}}
                \texttt{`SPRICH: Mein M ist jetzt in (3,1) und das sieht gut aus! Ich denke, wir sollten jetzt mit dem nächsten Objekt, O, fortfahren. Mein O ist in (2,7) und ich werde es nach (4,1) bewegen, um es an die richtige Position zu bringen. Bitte bestätige, ob das korrekt ist und ob du bereit bist, dein O zu bewegen.`} \\
            }
        }
    }
    & & \\ \\

    \theutterance \stepcounter{utterance}  
    & & \multicolumn{4}{p{0.6\linewidth}}{
        \cellcolor[rgb]{0.9,0.9,0.9}{
            \makecell[{{p{\linewidth}}}]{
                \texttt{\tiny{[GM$\rangle$P2]}}
                \texttt{'D' erfolgreich nach (2,1) verschoben. Dein Raster sieht jetzt so aus:} \\
\texttt{```} \\
\texttt{ 1234567} \\
\texttt{╔═══╤═╤═╗} \\
\texttt{║PDA│◌│◌║ 1} \\
\texttt{╟─┬─┼─┴─╢ 2} \\
\texttt{║◌│◌│◌◌◌║ 3} \\
\texttt{║◌├─┤◌┌─╢ 4} \\
\texttt{╟─┘◌│◌│O║ 5} \\
\texttt{║◌◌◌└┬┴─╢ 6} \\
\texttt{║◌S◌T│◌M║ 7} \\
\texttt{╚════╧══╝} \\
\texttt{```} \\
\texttt{Ihr seid gerade bei Runde 12 von maximal 28.} \\
\texttt{Ihr habt gemeinsam 5 von maximal 16 Strafpunkten angesammelt.} \\
\texttt{Dein Mitspieler hat dir diese Nachricht geschickt:} \\
\\ 
\\ 
\texttt{"Mein M ist jetzt in (3,1) und das sieht gut aus! Ich denke, wir sollten jetzt mit dem nächsten Objekt, O, fortfahren. Mein O ist in (2,7) und ich werde es nach (4,1) bewegen, um es an die richtige Position zu bringen. Bitte bestätige, ob das korrekt ist und ob du bereit bist, dein O zu bewegen."} \\
\\ 
\\ 
\texttt{Wie lautet dein nächster Befehl?} \\
            }
        }
    }
    & \\ \\

    \theutterance \stepcounter{utterance}  
    & & & \multicolumn{4}{p{0.6\linewidth}}{
        \cellcolor[rgb]{0.9,0.9,0.9}{
            \makecell[{{p{\linewidth}}}]{
                \texttt{\tiny{[GM$\langle$P2]}}
                \texttt{SPRICH: Ja, das klingt korrekt! Mein O ist in (6,5) und ich werde es nach (4,1) bewegen, um es an die richtige Position zu bringen. Bitte gib mir Feedback, ob alles korrekt ist und ob wir mit dem nächsten Objekt fortfahren können.} \\
            }
        }
    }
     \\ \\

    \theutterance \stepcounter{utterance}  
    & & & \multicolumn{2}{p{0.3\linewidth}}{
        \cellcolor[rgb]{0.9,0.9,0.9}{
            \makecell[{{p{\linewidth}}}]{
                \texttt{\tiny{[GM$|$GM]}}
                \texttt{SPRICH: Ja, das klingt korrekt! Mein O ist in (6,5) und ich werde es nach (4,1) bewegen, um es an die richtige Position zu bringen. Bitte gib mir Feedback, ob alles korrekt ist und ob wir mit dem nächsten Objekt fortfahren können.} \\
            }
        }
    }
    & & \\ \\

    \theutterance \stepcounter{utterance}  
    & & \multicolumn{4}{p{0.6\linewidth}}{
        \cellcolor[rgb]{0.9,0.9,0.9}{
            \makecell[{{p{\linewidth}}}]{
                \texttt{\tiny{[P1$\langle$GM]}}
                \texttt{Deine Nachricht wurde an deinen Mitspieler übermittelt.} \\
\\ 
\\ 
\texttt{Ihr seid gerade bei Runde 12 von maximal 28.} \\
\texttt{Ihr habt gemeinsam 5 von maximal 16 Strafpunkten angesammelt.} \\
\texttt{Dein Mitspieler hat dir diese Nachricht geschickt:} \\
\\ 
\\ 
\texttt{"Ja, das klingt korrekt! Mein O ist in (6,5) und ich werde es nach (4,1) bewegen, um es an die richtige Position zu bringen. Bitte gib mir Feedback, ob alles korrekt ist und ob wir mit dem nächsten Objekt fortfahren können."} \\
\\ 
\\ 
\texttt{Wie lautet dein nächster Befehl?} \\
            }
        }
    }
    & \\ \\

    \theutterance \stepcounter{utterance}  
    & \multicolumn{4}{p{0.6\linewidth}}{
        \cellcolor[rgb]{0.9,0.9,0.9}{
            \makecell[{{p{\linewidth}}}]{
                \texttt{\tiny{[P1$\rangle$GM]}}
                \texttt{`VERSCHIEBE: O, (4,1)`} \\
            }
        }
    }
    & & \\ \\

    \theutterance \stepcounter{utterance}  
    & & & \multicolumn{2}{p{0.3\linewidth}}{
        \cellcolor[rgb]{0.9,0.9,0.9}{
            \makecell[{{p{\linewidth}}}]{
                \texttt{\tiny{[GM$|$GM]}}
                \texttt{`VERSCHIEBE: O, (4,1)`} \\
            }
        }
    }
    & & \\ \\

    \theutterance \stepcounter{utterance}  
    & & & \multicolumn{2}{p{0.3\linewidth}}{
        \cellcolor[rgb]{0.9,0.9,0.9}{
            \makecell[{{p{\linewidth}}}]{
                \texttt{\tiny{[GM$|$GM]}}
                \texttt{Strafpunkt: (4,1) ist nicht leer, sondern enthält '│'.} \\
\\ 
\texttt{Ihr habt gemeinsam 6 von maximal 16 Strafpunkten angesammelt.} \\
\texttt{Bitte versuche es noch mal!} \\
            }
        }
    }
    & & \\ \\

    \theutterance \stepcounter{utterance}  
    & & \multicolumn{4}{p{0.6\linewidth}}{
        \cellcolor[rgb]{0.9,0.9,0.9}{
            \makecell[{{p{\linewidth}}}]{
                \texttt{\tiny{[P1$\langle$GM]}}
                \texttt{Strafpunkt: (4,1) ist nicht leer, sondern enthält '│'.} \\
\\ 
\texttt{Ihr habt gemeinsam 6 von maximal 16 Strafpunkten angesammelt.} \\
\texttt{Bitte versuche es noch mal!} \\
            }
        }
    }
    & \\ \\

    \theutterance \stepcounter{utterance}  
    & \multicolumn{4}{p{0.6\linewidth}}{
        \cellcolor[rgb]{0.9,0.9,0.9}{
            \makecell[{{p{\linewidth}}}]{
                \texttt{\tiny{[P1$\rangle$GM]}}
                \texttt{`SPRICH: Mist! Ich habe wieder einen Fehler gemacht. Ich werde mein O von (2,7) nach (4,2) bewegen, um es an die richtige Position zu bringen. Bitte bestätige, ob das korrekt ist und ob (4,2) frei ist.`} \\
            }
        }
    }
    & & \\ \\

    \theutterance \stepcounter{utterance}  
    & & & \multicolumn{2}{p{0.3\linewidth}}{
        \cellcolor[rgb]{0.9,0.9,0.9}{
            \makecell[{{p{\linewidth}}}]{
                \texttt{\tiny{[GM$|$GM]}}
                \texttt{`SPRICH: Mist! Ich habe wieder einen Fehler gemacht. Ich werde mein O von (2,7) nach (4,2) bewegen, um es an die richtige Position zu bringen. Bitte bestätige, ob das korrekt ist und ob (4,2) frei ist.`} \\
            }
        }
    }
    & & \\ \\

    \theutterance \stepcounter{utterance}  
    & & \multicolumn{4}{p{0.6\linewidth}}{
        \cellcolor[rgb]{0.9,0.9,0.9}{
            \makecell[{{p{\linewidth}}}]{
                \texttt{\tiny{[GM$\rangle$P2]}}
                \texttt{Deine Nachricht wurde an deinen Mitspieler übermittelt.} \\
\\ 
\\ 
\texttt{Ihr seid gerade bei Runde 13 von maximal 28.} \\
\texttt{Ihr habt gemeinsam 6 von maximal 16 Strafpunkten angesammelt.} \\
\texttt{Dein Mitspieler hat dir diese Nachricht geschickt:} \\
\\ 
\\ 
\texttt{"Mist! Ich habe wieder einen Fehler gemacht. Ich werde mein O von (2,7) nach (4,2) bewegen, um es an die richtige Position zu bringen. Bitte bestätige, ob das korrekt ist und ob (4,2) frei ist."} \\
\\ 
\\ 
\texttt{Wie lautet dein nächster Befehl?} \\
            }
        }
    }
    & \\ \\

    \theutterance \stepcounter{utterance}  
    & & & \multicolumn{4}{p{0.6\linewidth}}{
        \cellcolor[rgb]{0.9,0.9,0.9}{
            \makecell[{{p{\linewidth}}}]{
                \texttt{\tiny{[GM$\langle$P2]}}
                \texttt{SPRICH: Ich denke, das ist nicht korrekt. Wir hatten geplant, das O nach (4,1) zu bewegen, nicht nach (4,2). Bitte überprüfe deine Aktion und korrigiere sie, wenn nötig. Ich werde warten, bis du bestätigst, dass alles korrekt ist, bevor ich mein O bewege.} \\
            }
        }
    }
     \\ \\

    \theutterance \stepcounter{utterance}  
    & & & \multicolumn{2}{p{0.3\linewidth}}{
        \cellcolor[rgb]{0.9,0.9,0.9}{
            \makecell[{{p{\linewidth}}}]{
                \texttt{\tiny{[GM$|$GM]}}
                \texttt{SPRICH: Ich denke, das ist nicht korrekt. Wir hatten geplant, das O nach (4,1) zu bewegen, nicht nach (4,2). Bitte überprüfe deine Aktion und korrigiere sie, wenn nötig. Ich werde warten, bis du bestätigst, dass alles korrekt ist, bevor ich mein O bewege.} \\
            }
        }
    }
    & & \\ \\

    \theutterance \stepcounter{utterance}  
    & & \multicolumn{4}{p{0.6\linewidth}}{
        \cellcolor[rgb]{0.9,0.9,0.9}{
            \makecell[{{p{\linewidth}}}]{
                \texttt{\tiny{[P1$\langle$GM]}}
                \texttt{Deine Nachricht wurde an deinen Mitspieler übermittelt.} \\
\\ 
\\ 
\texttt{Ihr seid gerade bei Runde 13 von maximal 28.} \\
\texttt{Ihr habt gemeinsam 6 von maximal 16 Strafpunkten angesammelt.} \\
\texttt{Dein Mitspieler hat dir diese Nachricht geschickt:} \\
\\ 
\\ 
\texttt{"Ich denke, das ist nicht korrekt. Wir hatten geplant, das O nach (4,1) zu bewegen, nicht nach (4,2). Bitte überprüfe deine Aktion und korrigiere sie, wenn nötig. Ich werde warten, bis du bestätigst, dass alles korrekt ist, bevor ich mein O bewege."} \\
\\ 
\\ 
\texttt{Wie lautet dein nächster Befehl?} \\
            }
        }
    }
    & \\ \\

    \theutterance \stepcounter{utterance}  
    & \multicolumn{4}{p{0.6\linewidth}}{
        \cellcolor[rgb]{0.9,0.9,0.9}{
            \makecell[{{p{\linewidth}}}]{
                \texttt{\tiny{[P1$\rangle$GM]}}
                \texttt{`SPRICH: Entschuldigung! Du hast Recht, wir hatten geplant, das O nach (4,1) zu bewegen. Ich werde mein O von (2,7) nach (4,1) bewegen, aber ich muss zuerst sicherstellen, dass (4,1) frei ist. Mein Raster zeigt jedoch, dass (4,1) nicht frei ist, da es einen Rahmen Teil enthält. Ich schlage vor, dass wir das O nach (4,3) bewegen, um es an eine Position zu bringen, die frei ist. Bitte bestätige, ob das korrekt ist.`} \\
            }
        }
    }
    & & \\ \\

    \theutterance \stepcounter{utterance}  
    & & & \multicolumn{2}{p{0.3\linewidth}}{
        \cellcolor[rgb]{0.9,0.9,0.9}{
            \makecell[{{p{\linewidth}}}]{
                \texttt{\tiny{[GM$|$GM]}}
                \texttt{`SPRICH: Entschuldigung! Du hast Recht, wir hatten geplant, das O nach (4,1) zu bewegen. Ich werde mein O von (2,7) nach (4,1) bewegen, aber ich muss zuerst sicherstellen, dass (4,1) frei ist. Mein Raster zeigt jedoch, dass (4,1) nicht frei ist, da es einen Rahmen Teil enthält. Ich schlage vor, dass wir das O nach (4,3) bewegen, um es an eine Position zu bringen, die frei ist. Bitte bestätige, ob das korrekt ist.`} \\
            }
        }
    }
    & & \\ \\

    \theutterance \stepcounter{utterance}  
    & & \multicolumn{4}{p{0.6\linewidth}}{
        \cellcolor[rgb]{0.9,0.9,0.9}{
            \makecell[{{p{\linewidth}}}]{
                \texttt{\tiny{[GM$\rangle$P2]}}
                \texttt{Deine Nachricht wurde an deinen Mitspieler übermittelt.} \\
\\ 
\\ 
\texttt{Ihr seid gerade bei Runde 14 von maximal 28.} \\
\texttt{Ihr habt gemeinsam 6 von maximal 16 Strafpunkten angesammelt.} \\
\texttt{Dein Mitspieler hat dir diese Nachricht geschickt:} \\
\\ 
\\ 
\texttt{"Entschuldigung! Du hast Recht, wir hatten geplant, das O nach (4,1) zu bewegen. Ich werde mein O von (2,7) nach (4,1) bewegen, aber ich muss zuerst sicherstellen, dass (4,1) frei ist. Mein Raster zeigt jedoch, dass (4,1) nicht frei ist, da es einen Rahmen Teil enthält. Ich schlage vor, dass wir das O nach (4,3) bewegen, um es an eine Position zu bringen, die frei ist. Bitte bestätige, ob das korrekt ist."} \\
\\ 
\\ 
\texttt{Wie lautet dein nächster Befehl?} \\
            }
        }
    }
    & \\ \\

    \theutterance \stepcounter{utterance}  
    & & & \multicolumn{4}{p{0.6\linewidth}}{
        \cellcolor[rgb]{0.9,0.9,0.9}{
            \makecell[{{p{\linewidth}}}]{
                \texttt{\tiny{[GM$\langle$P2]}}
                \texttt{SPRICH: Ich denke, das ist nicht korrekt. Wir sollten uns an den ursprünglichen Plan halten und das O nach (4,1) bewegen. Wenn (4,1) auf deinem Raster nicht frei ist, dann sollten wir es freimachen, bevor wir das O dorthin bewegen. Ich schlage vor, dass du den Rahmen Teil an einer anderen Position platzierst, um (4,1) frei zu machen. Bitte gib mir Feedback und lass uns den ursprünglichen Plan fortsetzen.} \\
            }
        }
    }
     \\ \\

    \theutterance \stepcounter{utterance}  
    & & & \multicolumn{2}{p{0.3\linewidth}}{
        \cellcolor[rgb]{0.9,0.9,0.9}{
            \makecell[{{p{\linewidth}}}]{
                \texttt{\tiny{[GM$|$GM]}}
                \texttt{SPRICH: Ich denke, das ist nicht korrekt. Wir sollten uns an den ursprünglichen Plan halten und das O nach (4,1) bewegen. Wenn (4,1) auf deinem Raster nicht frei ist, dann sollten wir es freimachen, bevor wir das O dorthin bewegen. Ich schlage vor, dass du den Rahmen Teil an einer anderen Position platzierst, um (4,1) frei zu machen. Bitte gib mir Feedback und lass uns den ursprünglichen Plan fortsetzen.} \\
            }
        }
    }
    & & \\ \\

    \theutterance \stepcounter{utterance}  
    & & \multicolumn{4}{p{0.6\linewidth}}{
        \cellcolor[rgb]{0.9,0.9,0.9}{
            \makecell[{{p{\linewidth}}}]{
                \texttt{\tiny{[P1$\langle$GM]}}
                \texttt{Deine Nachricht wurde an deinen Mitspieler übermittelt.} \\
\\ 
\\ 
\texttt{Ihr seid gerade bei Runde 14 von maximal 28.} \\
\texttt{Ihr habt gemeinsam 6 von maximal 16 Strafpunkten angesammelt.} \\
\texttt{Dein Mitspieler hat dir diese Nachricht geschickt:} \\
\\ 
\\ 
\texttt{"Ich denke, das ist nicht korrekt. Wir sollten uns an den ursprünglichen Plan halten und das O nach (4,1) bewegen. Wenn (4,1) auf deinem Raster nicht frei ist, dann sollten wir es freimachen, bevor wir das O dorthin bewegen. Ich schlage vor, dass du den Rahmen Teil an einer anderen Position platzierst, um (4,1) frei zu machen. Bitte gib mir Feedback und lass uns den ursprünglichen Plan fortsetzen."} \\
\\ 
\\ 
\texttt{Wie lautet dein nächster Befehl?} \\
            }
        }
    }
    & \\ \\

    \theutterance \stepcounter{utterance}  
    & \multicolumn{4}{p{0.6\linewidth}}{
        \cellcolor[rgb]{0.9,0.9,0.9}{
            \makecell[{{p{\linewidth}}}]{
                \texttt{\tiny{[P1$\rangle$GM]}}
                \texttt{`SPRICH: Ich verstehe, was du meinst. Ich werde mich umsehen, um eine Möglichkeit zu finden, den Rahmen Teil an einer anderen Position zu platzieren, um (4,1) frei zu machen. Mein Raster zeigt jedoch, dass der Rahmen Teil ein fester Bestandteil des Rasters ist und nicht bewegt werden kann. Ich denke, wir sollten den ursprünglichen Plan überdenken und eine alternative Position für das O finden. Ich schlage vor, dass wir das O nach (6,1) bewegen, um es an eine Position zu bringen, die frei ist. Bitte bestätige, ob das korrekt ist.`} \\
            }
        }
    }
    & & \\ \\

    \theutterance \stepcounter{utterance}  
    & & & \multicolumn{2}{p{0.3\linewidth}}{
        \cellcolor[rgb]{0.9,0.9,0.9}{
            \makecell[{{p{\linewidth}}}]{
                \texttt{\tiny{[GM$|$GM]}}
                \texttt{`SPRICH: Ich verstehe, was du meinst. Ich werde mich umsehen, um eine Möglichkeit zu finden, den Rahmen Teil an einer anderen Position zu platzieren, um (4,1) frei zu machen. Mein Raster zeigt jedoch, dass der Rahmen Teil ein fester Bestandteil des Rasters ist und nicht bewegt werden kann. Ich denke, wir sollten den ursprünglichen Plan überdenken und eine alternative Position für das O finden. Ich schlage vor, dass wir das O nach (6,1) bewegen, um es an eine Position zu bringen, die frei ist. Bitte bestätige, ob das korrekt ist.`} \\
            }
        }
    }
    & & \\ \\

    \theutterance \stepcounter{utterance}  
    & & \multicolumn{4}{p{0.6\linewidth}}{
        \cellcolor[rgb]{0.9,0.9,0.9}{
            \makecell[{{p{\linewidth}}}]{
                \texttt{\tiny{[GM$\rangle$P2]}}
                \texttt{Deine Nachricht wurde an deinen Mitspieler übermittelt.} \\
\\ 
\\ 
\texttt{Ihr seid gerade bei Runde 15 von maximal 28.} \\
\texttt{Ihr habt gemeinsam 6 von maximal 16 Strafpunkten angesammelt.} \\
\texttt{Dein Mitspieler hat dir diese Nachricht geschickt:} \\
\\ 
\\ 
\texttt{"Ich verstehe, was du meinst. Ich werde mich umsehen, um eine Möglichkeit zu finden, den Rahmen Teil an einer anderen Position zu platzieren, um (4,1) frei zu machen. Mein Raster zeigt jedoch, dass der Rahmen Teil ein fester Bestandteil des Rasters ist und nicht bewegt werden kann. Ich denke, wir sollten den ursprünglichen Plan überdenken und eine alternative Position für das O finden. Ich schlage vor, dass wir das O nach (6,1) bewegen, um es an eine Position zu bringen, die frei ist. Bitte bestätige, ob das korrekt ist."} \\
\\ 
\\ 
\texttt{Wie lautet dein nächster Befehl?} \\
            }
        }
    }
    & \\ \\

    \theutterance \stepcounter{utterance}  
    & & & \multicolumn{4}{p{0.6\linewidth}}{
        \cellcolor[rgb]{0.9,0.9,0.9}{
            \makecell[{{p{\linewidth}}}]{
                \texttt{\tiny{[GM$\langle$P2]}}
                \texttt{SPRICH: Ich verstehe das Problem mit dem Rahmen Teil. Ich denke, dass (6,1) nicht die richtige Position für das O ist, da wir ursprünglich geplant hatten, das O nach (4,1) zu bewegen. Ich schlage vor, dass wir das S betrachten, das in (1,7) ist, und es nach einer Position bewegen, die es frei macht, um Platz für das O zu schaffen. Bitte gib mir Feedback und lass uns eine Lösung finden.} \\
            }
        }
    }
     \\ \\

    \theutterance \stepcounter{utterance}  
    & & & \multicolumn{2}{p{0.3\linewidth}}{
        \cellcolor[rgb]{0.9,0.9,0.9}{
            \makecell[{{p{\linewidth}}}]{
                \texttt{\tiny{[GM$|$GM]}}
                \texttt{SPRICH: Ich verstehe das Problem mit dem Rahmen Teil. Ich denke, dass (6,1) nicht die richtige Position für das O ist, da wir ursprünglich geplant hatten, das O nach (4,1) zu bewegen. Ich schlage vor, dass wir das S betrachten, das in (1,7) ist, und es nach einer Position bewegen, die es frei macht, um Platz für das O zu schaffen. Bitte gib mir Feedback und lass uns eine Lösung finden.} \\
            }
        }
    }
    & & \\ \\

    \theutterance \stepcounter{utterance}  
    & & \multicolumn{4}{p{0.6\linewidth}}{
        \cellcolor[rgb]{0.9,0.9,0.9}{
            \makecell[{{p{\linewidth}}}]{
                \texttt{\tiny{[P1$\langle$GM]}}
                \texttt{Deine Nachricht wurde an deinen Mitspieler übermittelt.} \\
\\ 
\\ 
\texttt{Ihr seid gerade bei Runde 15 von maximal 28.} \\
\texttt{Ihr habt gemeinsam 6 von maximal 16 Strafpunkten angesammelt.} \\
\texttt{Dein Mitspieler hat dir diese Nachricht geschickt:} \\
\\ 
\\ 
\texttt{"Ich verstehe das Problem mit dem Rahmen Teil. Ich denke, dass (6,1) nicht die richtige Position für das O ist, da wir ursprünglich geplant hatten, das O nach (4,1) zu bewegen. Ich schlage vor, dass wir das S betrachten, das in (1,7) ist, und es nach einer Position bewegen, die es frei macht, um Platz für das O zu schaffen. Bitte gib mir Feedback und lass uns eine Lösung finden."} \\
\\ 
\\ 
\texttt{Wie lautet dein nächster Befehl?} \\
            }
        }
    }
    & \\ \\

    \theutterance \stepcounter{utterance}  
    & \multicolumn{4}{p{0.6\linewidth}}{
        \cellcolor[rgb]{0.9,0.9,0.9}{
            \makecell[{{p{\linewidth}}}]{
                \texttt{\tiny{[P1$\rangle$GM]}}
                \texttt{`SPRICH: Ich denke, das ist ein guter Vorschlag! Mein S ist jedoch in (1,1) und nicht in (1,7). Ich werde mein S von (1,1) nach (7,1) bewegen, um Platz für das O zu schaffen. Bitte bestätige, ob das korrekt ist und ob wir dann das O nach (4,1) bewegen können.`} \\
            }
        }
    }
    & & \\ \\

    \theutterance \stepcounter{utterance}  
    & & & \multicolumn{2}{p{0.3\linewidth}}{
        \cellcolor[rgb]{0.9,0.9,0.9}{
            \makecell[{{p{\linewidth}}}]{
                \texttt{\tiny{[GM$|$GM]}}
                \texttt{`SPRICH: Ich denke, das ist ein guter Vorschlag! Mein S ist jedoch in (1,1) und nicht in (1,7). Ich werde mein S von (1,1) nach (7,1) bewegen, um Platz für das O zu schaffen. Bitte bestätige, ob das korrekt ist und ob wir dann das O nach (4,1) bewegen können.`} \\
            }
        }
    }
    & & \\ \\

    \theutterance \stepcounter{utterance}  
    & & \multicolumn{4}{p{0.6\linewidth}}{
        \cellcolor[rgb]{0.9,0.9,0.9}{
            \makecell[{{p{\linewidth}}}]{
                \texttt{\tiny{[GM$\rangle$P2]}}
                \texttt{Deine Nachricht wurde an deinen Mitspieler übermittelt.} \\
\\ 
\\ 
\texttt{Ihr seid gerade bei Runde 16 von maximal 28.} \\
\texttt{Ihr habt gemeinsam 6 von maximal 16 Strafpunkten angesammelt.} \\
\texttt{Dein Mitspieler hat dir diese Nachricht geschickt:} \\
\\ 
\\ 
\texttt{"Ich denke, das ist ein guter Vorschlag! Mein S ist jedoch in (1,1) und nicht in (1,7). Ich werde mein S von (1,1) nach (7,1) bewegen, um Platz für das O zu schaffen. Bitte bestätige, ob das korrekt ist und ob wir dann das O nach (4,1) bewegen können."} \\
\\ 
\\ 
\texttt{Wie lautet dein nächster Befehl?} \\
            }
        }
    }
    & \\ \\

    \theutterance \stepcounter{utterance}  
    & & & \multicolumn{4}{p{0.6\linewidth}}{
        \cellcolor[rgb]{0.9,0.9,0.9}{
            \makecell[{{p{\linewidth}}}]{
                \texttt{\tiny{[GM$\langle$P2]}}
                \texttt{SPRICH: Ich denke, das ist korrekt! Mein S ist in (1,7) und nicht in (1,1), aber das ist okay. Wenn du dein S nach (7,1) bewegst, wird das Platz für das O schaffen. Ich denke, wir sollten dann das O nach (4,1) bewegen, um es an die richtige Position zu bringen. Bitte gib mir Feedback und lass uns weitermachen. Ich bin bereit, mein O zu bewegen, wenn du bereit bist.} \\
            }
        }
    }
     \\ \\

    \theutterance \stepcounter{utterance}  
    & & & \multicolumn{2}{p{0.3\linewidth}}{
        \cellcolor[rgb]{0.9,0.9,0.9}{
            \makecell[{{p{\linewidth}}}]{
                \texttt{\tiny{[GM$|$GM]}}
                \texttt{SPRICH: Ich denke, das ist korrekt! Mein S ist in (1,7) und nicht in (1,1), aber das ist okay. Wenn du dein S nach (7,1) bewegst, wird das Platz für das O schaffen. Ich denke, wir sollten dann das O nach (4,1) bewegen, um es an die richtige Position zu bringen. Bitte gib mir Feedback und lass uns weitermachen. Ich bin bereit, mein O zu bewegen, wenn du bereit bist.} \\
            }
        }
    }
    & & \\ \\

    \theutterance \stepcounter{utterance}  
    & & \multicolumn{4}{p{0.6\linewidth}}{
        \cellcolor[rgb]{0.9,0.9,0.9}{
            \makecell[{{p{\linewidth}}}]{
                \texttt{\tiny{[P1$\langle$GM]}}
                \texttt{Deine Nachricht wurde an deinen Mitspieler übermittelt.} \\
\\ 
\\ 
\texttt{Ihr seid gerade bei Runde 16 von maximal 28.} \\
\texttt{Ihr habt gemeinsam 6 von maximal 16 Strafpunkten angesammelt.} \\
\texttt{Dein Mitspieler hat dir diese Nachricht geschickt:} \\
\\ 
\\ 
\texttt{"Ich denke, das ist korrekt! Mein S ist in (1,7) und nicht in (1,1), aber das ist okay. Wenn du dein S nach (7,1) bewegst, wird das Platz für das O schaffen. Ich denke, wir sollten dann das O nach (4,1) bewegen, um es an die richtige Position zu bringen. Bitte gib mir Feedback und lass uns weitermachen. Ich bin bereit, mein O zu bewegen, wenn du bereit bist."} \\
\\ 
\\ 
\texttt{Wie lautet dein nächster Befehl?} \\
            }
        }
    }
    & \\ \\

    \theutterance \stepcounter{utterance}  
    & \multicolumn{4}{p{0.6\linewidth}}{
        \cellcolor[rgb]{0.9,0.9,0.9}{
            \makecell[{{p{\linewidth}}}]{
                \texttt{\tiny{[P1$\rangle$GM]}}
                \texttt{`VERSCHIEBE: S, (7,1)`} \\
            }
        }
    }
    & & \\ \\

    \theutterance \stepcounter{utterance}  
    & & & \multicolumn{2}{p{0.3\linewidth}}{
        \cellcolor[rgb]{0.9,0.9,0.9}{
            \makecell[{{p{\linewidth}}}]{
                \texttt{\tiny{[GM$|$GM]}}
                \texttt{`VERSCHIEBE: S, (7,1)`} \\
            }
        }
    }
    & & \\ \\

    \theutterance \stepcounter{utterance}  
    & & & \multicolumn{2}{p{0.3\linewidth}}{
        \cellcolor[rgb]{0.9,0.9,0.9}{
            \makecell[{{p{\linewidth}}}]{
                \texttt{\tiny{[GM$|$GM]}}
                \texttt{Strafpunkt: (7,1) ist nicht leer, sondern enthält 'A'.} \\
\\ 
\texttt{Ihr habt gemeinsam 7 von maximal 16 Strafpunkten angesammelt.} \\
\texttt{Bitte versuche es noch mal!} \\
            }
        }
    }
    & & \\ \\

    \theutterance \stepcounter{utterance}  
    & & \multicolumn{4}{p{0.6\linewidth}}{
        \cellcolor[rgb]{0.9,0.9,0.9}{
            \makecell[{{p{\linewidth}}}]{
                \texttt{\tiny{[P1$\langle$GM]}}
                \texttt{Strafpunkt: (7,1) ist nicht leer, sondern enthält 'A'.} \\
\\ 
\texttt{Ihr habt gemeinsam 7 von maximal 16 Strafpunkten angesammelt.} \\
\texttt{Bitte versuche es noch mal!} \\
            }
        }
    }
    & \\ \\

    \theutterance \stepcounter{utterance}  
    & \multicolumn{4}{p{0.6\linewidth}}{
        \cellcolor[rgb]{0.9,0.9,0.9}{
            \makecell[{{p{\linewidth}}}]{
                \texttt{\tiny{[P1$\rangle$GM]}}
                \texttt{`SPRICH: Oh nein! Ich habe wieder einen Fehler gemacht. Ich werde mein S von (1,1) nach (7,2) bewegen, um Platz für das O zu schaffen. Bitte bestätige, ob das korrekt ist und ob (7,2) frei ist.`} \\
            }
        }
    }
    & & \\ \\

    \theutterance \stepcounter{utterance}  
    & & & \multicolumn{2}{p{0.3\linewidth}}{
        \cellcolor[rgb]{0.9,0.9,0.9}{
            \makecell[{{p{\linewidth}}}]{
                \texttt{\tiny{[GM$|$GM]}}
                \texttt{`SPRICH: Oh nein! Ich habe wieder einen Fehler gemacht. Ich werde mein S von (1,1) nach (7,2) bewegen, um Platz für das O zu schaffen. Bitte bestätige, ob das korrekt ist und ob (7,2) frei ist.`} \\
            }
        }
    }
    & & \\ \\

    \theutterance \stepcounter{utterance}  
    & & \multicolumn{4}{p{0.6\linewidth}}{
        \cellcolor[rgb]{0.9,0.9,0.9}{
            \makecell[{{p{\linewidth}}}]{
                \texttt{\tiny{[GM$\rangle$P2]}}
                \texttt{Deine Nachricht wurde an deinen Mitspieler übermittelt.} \\
\\ 
\\ 
\texttt{Ihr seid gerade bei Runde 17 von maximal 28.} \\
\texttt{Ihr habt gemeinsam 7 von maximal 16 Strafpunkten angesammelt.} \\
\texttt{Dein Mitspieler hat dir diese Nachricht geschickt:} \\
\\ 
\\ 
\texttt{"Oh nein! Ich habe wieder einen Fehler gemacht. Ich werde mein S von (1,1) nach (7,2) bewegen, um Platz für das O zu schaffen. Bitte bestätige, ob das korrekt ist und ob (7,2) frei ist."} \\
\\ 
\\ 
\texttt{Wie lautet dein nächster Befehl?} \\
            }
        }
    }
    & \\ \\

    \theutterance \stepcounter{utterance}  
    & & & \multicolumn{4}{p{0.6\linewidth}}{
        \cellcolor[rgb]{0.9,0.9,0.9}{
            \makecell[{{p{\linewidth}}}]{
                \texttt{\tiny{[GM$\langle$P2]}}
                \texttt{SPRICH: Ich denke, das ist nicht korrekt. Wir hatten geplant, das S nach (7,1) zu bewegen, um Platz für das O zu schaffen. Bitte überprüfe deine Aktion und korrigiere sie, wenn nötig. Ich werde warten, bis du bestätigst, dass alles korrekt ist, bevor ich mein O bewege. Bitte achte darauf, dass (7,1) frei ist, um das S dorthin zu bewegen.} \\
            }
        }
    }
     \\ \\

    \theutterance \stepcounter{utterance}  
    & & & \multicolumn{2}{p{0.3\linewidth}}{
        \cellcolor[rgb]{0.9,0.9,0.9}{
            \makecell[{{p{\linewidth}}}]{
                \texttt{\tiny{[GM$|$GM]}}
                \texttt{SPRICH: Ich denke, das ist nicht korrekt. Wir hatten geplant, das S nach (7,1) zu bewegen, um Platz für das O zu schaffen. Bitte überprüfe deine Aktion und korrigiere sie, wenn nötig. Ich werde warten, bis du bestätigst, dass alles korrekt ist, bevor ich mein O bewege. Bitte achte darauf, dass (7,1) frei ist, um das S dorthin zu bewegen.} \\
            }
        }
    }
    & & \\ \\

    \theutterance \stepcounter{utterance}  
    & & \multicolumn{4}{p{0.6\linewidth}}{
        \cellcolor[rgb]{0.9,0.9,0.9}{
            \makecell[{{p{\linewidth}}}]{
                \texttt{\tiny{[P1$\langle$GM]}}
                \texttt{Deine Nachricht wurde an deinen Mitspieler übermittelt.} \\
\\ 
\\ 
\texttt{Ihr seid gerade bei Runde 17 von maximal 28.} \\
\texttt{Ihr habt gemeinsam 7 von maximal 16 Strafpunkten angesammelt.} \\
\texttt{Dein Mitspieler hat dir diese Nachricht geschickt:} \\
\\ 
\\ 
\texttt{"Ich denke, das ist nicht korrekt. Wir hatten geplant, das S nach (7,1) zu bewegen, um Platz für das O zu schaffen. Bitte überprüfe deine Aktion und korrigiere sie, wenn nötig. Ich werde warten, bis du bestätigst, dass alles korrekt ist, bevor ich mein O bewege. Bitte achte darauf, dass (7,1) frei ist, um das S dorthin zu bewegen."} \\
\\ 
\\ 
\texttt{Wie lautet dein nächster Befehl?} \\
            }
        }
    }
    & \\ \\

    \theutterance \stepcounter{utterance}  
    & \multicolumn{4}{p{0.6\linewidth}}{
        \cellcolor[rgb]{0.9,0.9,0.9}{
            \makecell[{{p{\linewidth}}}]{
                \texttt{\tiny{[P1$\rangle$GM]}}
                \texttt{`VERSCHIEBE: A, (2,1)`} \\
            }
        }
    }
    & & \\ \\

    \theutterance \stepcounter{utterance}  
    & & & \multicolumn{2}{p{0.3\linewidth}}{
        \cellcolor[rgb]{0.9,0.9,0.9}{
            \makecell[{{p{\linewidth}}}]{
                \texttt{\tiny{[GM$|$GM]}}
                \texttt{`VERSCHIEBE: A, (2,1)`} \\
            }
        }
    }
    & & \\ \\

    \theutterance \stepcounter{utterance}  
    & & & \multicolumn{2}{p{0.3\linewidth}}{
        \cellcolor[rgb]{0.9,0.9,0.9}{
            \makecell[{{p{\linewidth}}}]{
                \texttt{\tiny{[GM$|$GM]}}
                \texttt{Strafpunkt: (2,1) ist nicht leer, sondern enthält 'S'.} \\
\\ 
\texttt{Ihr habt gemeinsam 8 von maximal 16 Strafpunkten angesammelt.} \\
\texttt{Bitte versuche es noch mal!} \\
            }
        }
    }
    & & \\ \\

    \theutterance \stepcounter{utterance}  
    & & \multicolumn{4}{p{0.6\linewidth}}{
        \cellcolor[rgb]{0.9,0.9,0.9}{
            \makecell[{{p{\linewidth}}}]{
                \texttt{\tiny{[P1$\langle$GM]}}
                \texttt{Strafpunkt: (2,1) ist nicht leer, sondern enthält 'S'.} \\
\\ 
\texttt{Ihr habt gemeinsam 8 von maximal 16 Strafpunkten angesammelt.} \\
\texttt{Bitte versuche es noch mal!} \\
            }
        }
    }
    & \\ \\

    \theutterance \stepcounter{utterance}  
    & \multicolumn{4}{p{0.6\linewidth}}{
        \cellcolor[rgb]{0.9,0.9,0.9}{
            \makecell[{{p{\linewidth}}}]{
                \texttt{\tiny{[P1$\rangle$GM]}}
                \texttt{`SPRICH: Ich verstehe, dass ich wieder einen Fehler gemacht habe. Ich werde mich umsehen, um eine Möglichkeit zu finden, das A zu bewegen, um Platz für das S zu schaffen. Mein Raster zeigt, dass das A in (1,1) ist und das S auch in (1,1) ist. Ich denke, ich sollte das A von (1,1) nach (1,2) bewegen, um Platz für das S zu schaffen. Bitte bestätige, ob das korrekt ist.`} \\
            }
        }
    }
    & & \\ \\

    \theutterance \stepcounter{utterance}  
    & & & \multicolumn{2}{p{0.3\linewidth}}{
        \cellcolor[rgb]{0.9,0.9,0.9}{
            \makecell[{{p{\linewidth}}}]{
                \texttt{\tiny{[GM$|$GM]}}
                \texttt{`SPRICH: Ich verstehe, dass ich wieder einen Fehler gemacht habe. Ich werde mich umsehen, um eine Möglichkeit zu finden, das A zu bewegen, um Platz für das S zu schaffen. Mein Raster zeigt, dass das A in (1,1) ist und das S auch in (1,1) ist. Ich denke, ich sollte das A von (1,1) nach (1,2) bewegen, um Platz für das S zu schaffen. Bitte bestätige, ob das korrekt ist.`} \\
            }
        }
    }
    & & \\ \\

    \theutterance \stepcounter{utterance}  
    & & \multicolumn{4}{p{0.6\linewidth}}{
        \cellcolor[rgb]{0.9,0.9,0.9}{
            \makecell[{{p{\linewidth}}}]{
                \texttt{\tiny{[GM$\rangle$P2]}}
                \texttt{Deine Nachricht wurde an deinen Mitspieler übermittelt.} \\
\\ 
\\ 
\texttt{Ihr seid gerade bei Runde 18 von maximal 28.} \\
\texttt{Ihr habt gemeinsam 8 von maximal 16 Strafpunkten angesammelt.} \\
\texttt{Dein Mitspieler hat dir diese Nachricht geschickt:} \\
\\ 
\\ 
\texttt{"Ich verstehe, dass ich wieder einen Fehler gemacht habe. Ich werde mich umsehen, um eine Möglichkeit zu finden, das A zu bewegen, um Platz für das S zu schaffen. Mein Raster zeigt, dass das A in (1,1) ist und das S auch in (1,1) ist. Ich denke, ich sollte das A von (1,1) nach (1,2) bewegen, um Platz für das S zu schaffen. Bitte bestätige, ob das korrekt ist."} \\
\\ 
\\ 
\texttt{Wie lautet dein nächster Befehl?} \\
            }
        }
    }
    & \\ \\

    \theutterance \stepcounter{utterance}  
    & & & \multicolumn{4}{p{0.6\linewidth}}{
        \cellcolor[rgb]{0.9,0.9,0.9}{
            \makecell[{{p{\linewidth}}}]{
                \texttt{\tiny{[GM$\langle$P2]}}
                \texttt{SPRICH: Ich denke, das ist nicht korrekt. Wir hatten geplant, das S nach (7,1) zu bewegen, nicht nach (1,1). Bitte überprüfe deine Aktion und korrigiere sie, wenn nötig. Ich denke, wir sollten uns auf den ursprünglichen Plan konzentrieren und das S nach (7,1) bewegen, um Platz für das O zu schaffen. Bitte gib mir Feedback und lass uns weitermachen.} \\
            }
        }
    }
     \\ \\

    \theutterance \stepcounter{utterance}  
    & & & \multicolumn{2}{p{0.3\linewidth}}{
        \cellcolor[rgb]{0.9,0.9,0.9}{
            \makecell[{{p{\linewidth}}}]{
                \texttt{\tiny{[GM$|$GM]}}
                \texttt{SPRICH: Ich denke, das ist nicht korrekt. Wir hatten geplant, das S nach (7,1) zu bewegen, nicht nach (1,1). Bitte überprüfe deine Aktion und korrigiere sie, wenn nötig. Ich denke, wir sollten uns auf den ursprünglichen Plan konzentrieren und das S nach (7,1) bewegen, um Platz für das O zu schaffen. Bitte gib mir Feedback und lass uns weitermachen.} \\
            }
        }
    }
    & & \\ \\

    \theutterance \stepcounter{utterance}  
    & & \multicolumn{4}{p{0.6\linewidth}}{
        \cellcolor[rgb]{0.9,0.9,0.9}{
            \makecell[{{p{\linewidth}}}]{
                \texttt{\tiny{[P1$\langle$GM]}}
                \texttt{Deine Nachricht wurde an deinen Mitspieler übermittelt.} \\
\\ 
\\ 
\texttt{Ihr seid gerade bei Runde 18 von maximal 28.} \\
\texttt{Ihr habt gemeinsam 8 von maximal 16 Strafpunkten angesammelt.} \\
\texttt{Dein Mitspieler hat dir diese Nachricht geschickt:} \\
\\ 
\\ 
\texttt{"Ich denke, das ist nicht korrekt. Wir hatten geplant, das S nach (7,1) zu bewegen, nicht nach (1,1). Bitte überprüfe deine Aktion und korrigiere sie, wenn nötig. Ich denke, wir sollten uns auf den ursprünglichen Plan konzentrieren und das S nach (7,1) bewegen, um Platz für das O zu schaffen. Bitte gib mir Feedback und lass uns weitermachen."} \\
\\ 
\\ 
\texttt{Wie lautet dein nächster Befehl?} \\
            }
        }
    }
    & \\ \\

    \theutterance \stepcounter{utterance}  
    & \multicolumn{4}{p{0.6\linewidth}}{
        \cellcolor[rgb]{0.9,0.9,0.9}{
            \makecell[{{p{\linewidth}}}]{
                \texttt{\tiny{[P1$\rangle$GM]}}
                \texttt{`SPRICH: Ich verstehe jetzt. Ich werde mein A von (1,1) nach einer anderen Position bewegen, um Platz für das S zu schaffen, aber ich denke, ich sollte zuerst das A von (1,1) weg bewegen, um das S nach (7,1) zu bewegen. Mein Raster zeigt, dass das A in (1,1) ist und das S auch in (1,1) ist, zusammen mit dem M. Ich denke, ich sollte das A von (1,1) nach (1,3) bewegen, um Platz für das S zu schaffen. Bitte bestätige, ob das korrekt ist.`} \\
            }
        }
    }
    & & \\ \\

    \theutterance \stepcounter{utterance}  
    & & & \multicolumn{2}{p{0.3\linewidth}}{
        \cellcolor[rgb]{0.9,0.9,0.9}{
            \makecell[{{p{\linewidth}}}]{
                \texttt{\tiny{[GM$|$GM]}}
                \texttt{`SPRICH: Ich verstehe jetzt. Ich werde mein A von (1,1) nach einer anderen Position bewegen, um Platz für das S zu schaffen, aber ich denke, ich sollte zuerst das A von (1,1) weg bewegen, um das S nach (7,1) zu bewegen. Mein Raster zeigt, dass das A in (1,1) ist und das S auch in (1,1) ist, zusammen mit dem M. Ich denke, ich sollte das A von (1,1) nach (1,3) bewegen, um Platz für das S zu schaffen. Bitte bestätige, ob das korrekt ist.`} \\
            }
        }
    }
    & & \\ \\

    \theutterance \stepcounter{utterance}  
    & & \multicolumn{4}{p{0.6\linewidth}}{
        \cellcolor[rgb]{0.9,0.9,0.9}{
            \makecell[{{p{\linewidth}}}]{
                \texttt{\tiny{[GM$\rangle$P2]}}
                \texttt{Deine Nachricht wurde an deinen Mitspieler übermittelt.} \\
\\ 
\\ 
\texttt{Ihr seid gerade bei Runde 19 von maximal 28.} \\
\texttt{Ihr habt gemeinsam 8 von maximal 16 Strafpunkten angesammelt.} \\
\texttt{Dein Mitspieler hat dir diese Nachricht geschickt:} \\
\\ 
\\ 
\texttt{"Ich verstehe jetzt. Ich werde mein A von (1,1) nach einer anderen Position bewegen, um Platz für das S zu schaffen, aber ich denke, ich sollte zuerst das A von (1,1) weg bewegen, um das S nach (7,1) zu bewegen. Mein Raster zeigt, dass das A in (1,1) ist und das S auch in (1,1) ist, zusammen mit dem M. Ich denke, ich sollte das A von (1,1) nach (1,3) bewegen, um Platz für das S zu schaffen. Bitte bestätige, ob das korrekt ist."} \\
\\ 
\\ 
\texttt{Wie lautet dein nächster Befehl?} \\
            }
        }
    }
    & \\ \\

    \theutterance \stepcounter{utterance}  
    & & & \multicolumn{4}{p{0.6\linewidth}}{
        \cellcolor[rgb]{0.9,0.9,0.9}{
            \makecell[{{p{\linewidth}}}]{
                \texttt{\tiny{[GM$\langle$P2]}}
                \texttt{SPRICH: Ich denke, das ist nicht korrekt. Wir hatten geplant, das S nach (7,1) zu bewegen, und nicht, das A von (1,1) nach (1,3) zu bewegen. Ich denke, wir sollten uns auf den ursprünglichen Plan konzentrieren und das S nach (7,1) bewegen, um Platz für das O zu schaffen. Bitte überprüfe deine Aktion und korrigiere sie, wenn nötig. Ich denke, wir sollten auch unsere Raster überprüfen, um sicherzustellen, dass wir die korrekten Positionen für die Objekte haben. Bitte gib mir Feedback und lass uns weitermachen.} \\
            }
        }
    }
     \\ \\

    \theutterance \stepcounter{utterance}  
    & & & \multicolumn{2}{p{0.3\linewidth}}{
        \cellcolor[rgb]{0.9,0.9,0.9}{
            \makecell[{{p{\linewidth}}}]{
                \texttt{\tiny{[GM$|$GM]}}
                \texttt{SPRICH: Ich denke, das ist nicht korrekt. Wir hatten geplant, das S nach (7,1) zu bewegen, und nicht, das A von (1,1) nach (1,3) zu bewegen. Ich denke, wir sollten uns auf den ursprünglichen Plan konzentrieren und das S nach (7,1) bewegen, um Platz für das O zu schaffen. Bitte überprüfe deine Aktion und korrigiere sie, wenn nötig. Ich denke, wir sollten auch unsere Raster überprüfen, um sicherzustellen, dass wir die korrekten Positionen für die Objekte haben. Bitte gib mir Feedback und lass uns weitermachen.} \\
            }
        }
    }
    & & \\ \\

    \theutterance \stepcounter{utterance}  
    & & \multicolumn{4}{p{0.6\linewidth}}{
        \cellcolor[rgb]{0.9,0.9,0.9}{
            \makecell[{{p{\linewidth}}}]{
                \texttt{\tiny{[P1$\langle$GM]}}
                \texttt{Deine Nachricht wurde an deinen Mitspieler übermittelt.} \\
\\ 
\\ 
\texttt{Ihr seid gerade bei Runde 19 von maximal 28.} \\
\texttt{Ihr habt gemeinsam 8 von maximal 16 Strafpunkten angesammelt.} \\
\texttt{Dein Mitspieler hat dir diese Nachricht geschickt:} \\
\\ 
\\ 
\texttt{"Ich denke, das ist nicht korrekt. Wir hatten geplant, das S nach (7,1) zu bewegen, und nicht, das A von (1,1) nach (1,3) zu bewegen. Ich denke, wir sollten uns auf den ursprünglichen Plan konzentrieren und das S nach (7,1) bewegen, um Platz für das O zu schaffen. Bitte überprüfe deine Aktion und korrigiere sie, wenn nötig. Ich denke, wir sollten auch unsere Raster überprüfen, um sicherzustellen, dass wir die korrekten Positionen für die Objekte haben. Bitte gib mir Feedback und lass uns weitermachen."} \\
\\ 
\\ 
\texttt{Wie lautet dein nächster Befehl?} \\
            }
        }
    }
    & \\ \\

    \theutterance \stepcounter{utterance}  
    & \multicolumn{4}{p{0.6\linewidth}}{
        \cellcolor[rgb]{0.9,0.9,0.9}{
            \makecell[{{p{\linewidth}}}]{
                \texttt{\tiny{[P1$\rangle$GM]}}
                \texttt{`SPRICH: Ich entschuldige mich für die Verwirrung. Ich denke, wir sollten unsere Raster überprüfen und sicherstellen, dass wir die korrekten Positionen für die Objekte haben. Mein Raster zeigt, dass das A in (1,1) ist, das M in (3,1) ist, das P in (1,4) ist, das D in (4,4) ist, das O in (2,7) ist, das S in (1,1) ist und das T in (3,7) ist. Bitte bestätige, ob deine Positionen übereinstimmen und ob wir den ursprünglichen Plan fortsetzen können.`} \\
            }
        }
    }
    & & \\ \\

    \theutterance \stepcounter{utterance}  
    & & & \multicolumn{2}{p{0.3\linewidth}}{
        \cellcolor[rgb]{0.9,0.9,0.9}{
            \makecell[{{p{\linewidth}}}]{
                \texttt{\tiny{[GM$|$GM]}}
                \texttt{`SPRICH: Ich entschuldige mich für die Verwirrung. Ich denke, wir sollten unsere Raster überprüfen und sicherstellen, dass wir die korrekten Positionen für die Objekte haben. Mein Raster zeigt, dass das A in (1,1) ist, das M in (3,1) ist, das P in (1,4) ist, das D in (4,4) ist, das O in (2,7) ist, das S in (1,1) ist und das T in (3,7) ist. Bitte bestätige, ob deine Positionen übereinstimmen und ob wir den ursprünglichen Plan fortsetzen können.`} \\
            }
        }
    }
    & & \\ \\

    \theutterance \stepcounter{utterance}  
    & & \multicolumn{4}{p{0.6\linewidth}}{
        \cellcolor[rgb]{0.9,0.9,0.9}{
            \makecell[{{p{\linewidth}}}]{
                \texttt{\tiny{[GM$\rangle$P2]}}
                \texttt{Deine Nachricht wurde an deinen Mitspieler übermittelt.} \\
\\ 
\\ 
\texttt{Ihr seid gerade bei Runde 20 von maximal 28.} \\
\texttt{Ihr habt gemeinsam 8 von maximal 16 Strafpunkten angesammelt.} \\
\texttt{Dein Mitspieler hat dir diese Nachricht geschickt:} \\
\\ 
\\ 
\texttt{"Ich entschuldige mich für die Verwirrung. Ich denke, wir sollten unsere Raster überprüfen und sicherstellen, dass wir die korrekten Positionen für die Objekte haben. Mein Raster zeigt, dass das A in (1,1) ist, das M in (3,1) ist, das P in (1,4) ist, das D in (4,4) ist, das O in (2,7) ist, das S in (1,1) ist und das T in (3,7) ist. Bitte bestätige, ob deine Positionen übereinstimmen und ob wir den ursprünglichen Plan fortsetzen können."} \\
\\ 
\\ 
\texttt{Wie lautet dein nächster Befehl?} \\
            }
        }
    }
    & \\ \\

    \theutterance \stepcounter{utterance}  
    & & & \multicolumn{4}{p{0.6\linewidth}}{
        \cellcolor[rgb]{0.9,0.9,0.9}{
            \makecell[{{p{\linewidth}}}]{
                \texttt{\tiny{[GM$\langle$P2]}}
                \texttt{SPRICH: Ich denke, es gibt eine Verwirrung. Mein Raster zeigt, dass das A in (1,1) ist, das D in (2,1) ist, das P in (5,1) ist, das O in (6,5) ist, das S in (1,7) ist, das T in (4,7) ist und das M in (7,7) ist. Unsere Raster sind unterschiedlich, und ich denke, wir sollten unsere Pläne anpassen, um sicherzustellen, dass wir die Objekte an die korrekten Positionen bewegen. Bitte überprüfe deine Aktionen und korrigiere sie, wenn nötig. Ich denke, wir sollten auch den ursprünglichen Plan überdenken und neue Anweisungen geben, um sicherzustellen, dass wir die Objekte an die korrekten Positionen bewegen. Bitte gib mir Feedback und lass uns weitermachen.} \\
            }
        }
    }
     \\ \\

    \theutterance \stepcounter{utterance}  
    & & & \multicolumn{2}{p{0.3\linewidth}}{
        \cellcolor[rgb]{0.9,0.9,0.9}{
            \makecell[{{p{\linewidth}}}]{
                \texttt{\tiny{[GM$|$GM]}}
                \texttt{SPRICH: Ich denke, es gibt eine Verwirrung. Mein Raster zeigt, dass das A in (1,1) ist, das D in (2,1) ist, das P in (5,1) ist, das O in (6,5) ist, das S in (1,7) ist, das T in (4,7) ist und das M in (7,7) ist. Unsere Raster sind unterschiedlich, und ich denke, wir sollten unsere Pläne anpassen, um sicherzustellen, dass wir die Objekte an die korrekten Positionen bewegen. Bitte überprüfe deine Aktionen und korrigiere sie, wenn nötig. Ich denke, wir sollten auch den ursprünglichen Plan überdenken und neue Anweisungen geben, um sicherzustellen, dass wir die Objekte an die korrekten Positionen bewegen. Bitte gib mir Feedback und lass uns weitermachen.} \\
            }
        }
    }
    & & \\ \\

    \theutterance \stepcounter{utterance}  
    & & \multicolumn{4}{p{0.6\linewidth}}{
        \cellcolor[rgb]{0.9,0.9,0.9}{
            \makecell[{{p{\linewidth}}}]{
                \texttt{\tiny{[P1$\langle$GM]}}
                \texttt{Deine Nachricht wurde an deinen Mitspieler übermittelt.} \\
\\ 
\\ 
\texttt{Ihr seid gerade bei Runde 20 von maximal 28.} \\
\texttt{Ihr habt gemeinsam 8 von maximal 16 Strafpunkten angesammelt.} \\
\texttt{Dein Mitspieler hat dir diese Nachricht geschickt:} \\
\\ 
\\ 
\texttt{"Ich denke, es gibt eine Verwirrung. Mein Raster zeigt, dass das A in (1,1) ist, das D in (2,1) ist, das P in (5,1) ist, das O in (6,5) ist, das S in (1,7) ist, das T in (4,7) ist und das M in (7,7) ist. Unsere Raster sind unterschiedlich, und ich denke, wir sollten unsere Pläne anpassen, um sicherzustellen, dass wir die Objekte an die korrekten Positionen bewegen. Bitte überprüfe deine Aktionen und korrigiere sie, wenn nötig. Ich denke, wir sollten auch den ursprünglichen Plan überdenken und neue Anweisungen geben, um sicherzustellen, dass wir die Objekte an die korrekten Positionen bewegen. Bitte gib mir Feedback und lass uns weitermachen."} \\
\\ 
\\ 
\texttt{Wie lautet dein nächster Befehl?} \\
            }
        }
    }
    & \\ \\

    \theutterance \stepcounter{utterance}  
    & \multicolumn{4}{p{0.6\linewidth}}{
        \cellcolor[rgb]{0.9,0.9,0.9}{
            \makecell[{{p{\linewidth}}}]{
                \texttt{\tiny{[P1$\rangle$GM]}}
                \texttt{`SPRICH: Ich verstehe jetzt, dass unsere Raster unterschiedlich sind. Ich denke, wir sollten den ursprünglichen Plan überdenken und neue Anweisungen geben, um sicherzustellen, dass wir die Objekte an die korrekten Positionen bewegen. Ich schlage vor, dass wir von vorne beginnen und die Objekte in alphabetischer Reihenfolge anordnen, beginnend von oben links. Wir sollten die Positionen der Objekte auf unseren Rastern überprüfen und sicherstellen, dass wir die korrekten Positionen für die Objekte haben. Bitte bestätige, ob das korrekt ist und ob wir mit dem Objekt A beginnen sollten.`} \\
            }
        }
    }
    & & \\ \\

    \theutterance \stepcounter{utterance}  
    & & & \multicolumn{2}{p{0.3\linewidth}}{
        \cellcolor[rgb]{0.9,0.9,0.9}{
            \makecell[{{p{\linewidth}}}]{
                \texttt{\tiny{[GM$|$GM]}}
                \texttt{`SPRICH: Ich verstehe jetzt, dass unsere Raster unterschiedlich sind. Ich denke, wir sollten den ursprünglichen Plan überdenken und neue Anweisungen geben, um sicherzustellen, dass wir die Objekte an die korrekten Positionen bewegen. Ich schlage vor, dass wir von vorne beginnen und die Objekte in alphabetischer Reihenfolge anordnen, beginnend von oben links. Wir sollten die Positionen der Objekte auf unseren Rastern überprüfen und sicherstellen, dass wir die korrekten Positionen für die Objekte haben. Bitte bestätige, ob das korrekt ist und ob wir mit dem Objekt A beginnen sollten.`} \\
            }
        }
    }
    & & \\ \\

    \theutterance \stepcounter{utterance}  
    & & \multicolumn{4}{p{0.6\linewidth}}{
        \cellcolor[rgb]{0.9,0.9,0.9}{
            \makecell[{{p{\linewidth}}}]{
                \texttt{\tiny{[GM$\rangle$P2]}}
                \texttt{Deine Nachricht wurde an deinen Mitspieler übermittelt.} \\
\\ 
\\ 
\texttt{Ihr seid gerade bei Runde 21 von maximal 28.} \\
\texttt{Ihr habt gemeinsam 8 von maximal 16 Strafpunkten angesammelt.} \\
\texttt{Dein Mitspieler hat dir diese Nachricht geschickt:} \\
\\ 
\\ 
\texttt{"Ich verstehe jetzt, dass unsere Raster unterschiedlich sind. Ich denke, wir sollten den ursprünglichen Plan überdenken und neue Anweisungen geben, um sicherzustellen, dass wir die Objekte an die korrekten Positionen bewegen. Ich schlage vor, dass wir von vorne beginnen und die Objekte in alphabetischer Reihenfolge anordnen, beginnend von oben links. Wir sollten die Positionen der Objekte auf unseren Rastern überprüfen und sicherstellen, dass wir die korrekten Positionen für die Objekte haben. Bitte bestätige, ob das korrekt ist und ob wir mit dem Objekt A beginnen sollten."} \\
\\ 
\\ 
\texttt{Wie lautet dein nächster Befehl?} \\
            }
        }
    }
    & \\ \\

    \theutterance \stepcounter{utterance}  
    & & & \multicolumn{4}{p{0.6\linewidth}}{
        \cellcolor[rgb]{0.9,0.9,0.9}{
            \makecell[{{p{\linewidth}}}]{
                \texttt{\tiny{[GM$\langle$P2]}}
                \texttt{SPRICH: Ja, das klingt korrekt! Ich denke, wir sollten von vorne beginnen und die Objekte in alphabetischer Reihenfolge anordnen, beginnend von oben links. Ich stimme zu, dass wir mit dem Objekt A beginnen sollten. Mein A ist bereits in (1,1), also ist es an der richtigen Position. Bitte gib mir Feedback und lass uns weitermachen mit dem nächsten Objekt, das D ist. Ich denke, wir sollten sicherstellen, dass unser D an der richtigen Position ist, bevor wir zum nächsten Objekt übergehen.} \\
            }
        }
    }
     \\ \\

    \theutterance \stepcounter{utterance}  
    & & & \multicolumn{2}{p{0.3\linewidth}}{
        \cellcolor[rgb]{0.9,0.9,0.9}{
            \makecell[{{p{\linewidth}}}]{
                \texttt{\tiny{[GM$|$GM]}}
                \texttt{SPRICH: Ja, das klingt korrekt! Ich denke, wir sollten von vorne beginnen und die Objekte in alphabetischer Reihenfolge anordnen, beginnend von oben links. Ich stimme zu, dass wir mit dem Objekt A beginnen sollten. Mein A ist bereits in (1,1), also ist es an der richtigen Position. Bitte gib mir Feedback und lass uns weitermachen mit dem nächsten Objekt, das D ist. Ich denke, wir sollten sicherstellen, dass unser D an der richtigen Position ist, bevor wir zum nächsten Objekt übergehen.} \\
            }
        }
    }
    & & \\ \\

    \theutterance \stepcounter{utterance}  
    & & \multicolumn{4}{p{0.6\linewidth}}{
        \cellcolor[rgb]{0.9,0.9,0.9}{
            \makecell[{{p{\linewidth}}}]{
                \texttt{\tiny{[P1$\langle$GM]}}
                \texttt{Deine Nachricht wurde an deinen Mitspieler übermittelt.} \\
\\ 
\\ 
\texttt{Ihr seid gerade bei Runde 21 von maximal 28.} \\
\texttt{Ihr habt gemeinsam 8 von maximal 16 Strafpunkten angesammelt.} \\
\texttt{Dein Mitspieler hat dir diese Nachricht geschickt:} \\
\\ 
\\ 
\texttt{"Ja, das klingt korrekt! Ich denke, wir sollten von vorne beginnen und die Objekte in alphabetischer Reihenfolge anordnen, beginnend von oben links. Ich stimme zu, dass wir mit dem Objekt A beginnen sollten. Mein A ist bereits in (1,1), also ist es an der richtigen Position. Bitte gib mir Feedback und lass uns weitermachen mit dem nächsten Objekt, das D ist. Ich denke, wir sollten sicherstellen, dass unser D an der richtigen Position ist, bevor wir zum nächsten Objekt übergehen."} \\
\\ 
\\ 
\texttt{Wie lautet dein nächster Befehl?} \\
            }
        }
    }
    & \\ \\

    \theutterance \stepcounter{utterance}  
    & \multicolumn{4}{p{0.6\linewidth}}{
        \cellcolor[rgb]{0.9,0.9,0.9}{
            \makecell[{{p{\linewidth}}}]{
                \texttt{\tiny{[P1$\rangle$GM]}}
                \texttt{`SPRICH: Mein A ist auch in (1,1), also ist es an der richtigen Position. Mein D ist jedoch in (4,4) und sollte nach (2,1) bewegt werden, um an der richtigen Position zu sein. Ich werde mein D von (4,4) nach (2,1) bewegen. Bitte bestätige, ob das korrekt ist und ob wir dann zum nächsten Objekt, das M ist, übergehen können.`} \\
            }
        }
    }
    & & \\ \\

    \theutterance \stepcounter{utterance}  
    & & & \multicolumn{2}{p{0.3\linewidth}}{
        \cellcolor[rgb]{0.9,0.9,0.9}{
            \makecell[{{p{\linewidth}}}]{
                \texttt{\tiny{[GM$|$GM]}}
                \texttt{`SPRICH: Mein A ist auch in (1,1), also ist es an der richtigen Position. Mein D ist jedoch in (4,4) und sollte nach (2,1) bewegt werden, um an der richtigen Position zu sein. Ich werde mein D von (4,4) nach (2,1) bewegen. Bitte bestätige, ob das korrekt ist und ob wir dann zum nächsten Objekt, das M ist, übergehen können.`} \\
            }
        }
    }
    & & \\ \\

    \theutterance \stepcounter{utterance}  
    & & \multicolumn{4}{p{0.6\linewidth}}{
        \cellcolor[rgb]{0.9,0.9,0.9}{
            \makecell[{{p{\linewidth}}}]{
                \texttt{\tiny{[GM$\rangle$P2]}}
                \texttt{Deine Nachricht wurde an deinen Mitspieler übermittelt.} \\
\\ 
\\ 
\texttt{Ihr seid gerade bei Runde 22 von maximal 28.} \\
\texttt{Ihr habt gemeinsam 8 von maximal 16 Strafpunkten angesammelt.} \\
\texttt{Dein Mitspieler hat dir diese Nachricht geschickt:} \\
\\ 
\\ 
\texttt{"Mein A ist auch in (1,1), also ist es an der richtigen Position. Mein D ist jedoch in (4,4) und sollte nach (2,1) bewegt werden, um an der richtigen Position zu sein. Ich werde mein D von (4,4) nach (2,1) bewegen. Bitte bestätige, ob das korrekt ist und ob wir dann zum nächsten Objekt, das M ist, übergehen können."} \\
\\ 
\\ 
\texttt{Wie lautet dein nächster Befehl?} \\
            }
        }
    }
    & \\ \\

    \theutterance \stepcounter{utterance}  
    & & & \multicolumn{4}{p{0.6\linewidth}}{
        \cellcolor[rgb]{0.9,0.9,0.9}{
            \makecell[{{p{\linewidth}}}]{
                \texttt{\tiny{[GM$\langle$P2]}}
                \texttt{SPRICH: Ja, das klingt korrekt! Mein D ist bereits in (2,1), also ist es an der richtigen Position. Ich denke, wir können zum nächsten Objekt, das M ist, übergehen. Mein M ist in (7,7) und sollte nach (3,1) bewegt werden, um an der richtigen Position zu sein. Bitte gib mir Feedback und lass uns weitermachen. Ich denke, wir sollten sicherstellen, dass unser M an der richtigen Position ist, bevor wir zum nächsten Objekt übergehen.} \\
            }
        }
    }
     \\ \\

    \theutterance \stepcounter{utterance}  
    & & & \multicolumn{2}{p{0.3\linewidth}}{
        \cellcolor[rgb]{0.9,0.9,0.9}{
            \makecell[{{p{\linewidth}}}]{
                \texttt{\tiny{[GM$|$GM]}}
                \texttt{SPRICH: Ja, das klingt korrekt! Mein D ist bereits in (2,1), also ist es an der richtigen Position. Ich denke, wir können zum nächsten Objekt, das M ist, übergehen. Mein M ist in (7,7) und sollte nach (3,1) bewegt werden, um an der richtigen Position zu sein. Bitte gib mir Feedback und lass uns weitermachen. Ich denke, wir sollten sicherstellen, dass unser M an der richtigen Position ist, bevor wir zum nächsten Objekt übergehen.} \\
            }
        }
    }
    & & \\ \\

    \theutterance \stepcounter{utterance}  
    & & \multicolumn{4}{p{0.6\linewidth}}{
        \cellcolor[rgb]{0.9,0.9,0.9}{
            \makecell[{{p{\linewidth}}}]{
                \texttt{\tiny{[P1$\langle$GM]}}
                \texttt{Deine Nachricht wurde an deinen Mitspieler übermittelt.} \\
\\ 
\\ 
\texttt{Ihr seid gerade bei Runde 22 von maximal 28.} \\
\texttt{Ihr habt gemeinsam 8 von maximal 16 Strafpunkten angesammelt.} \\
\texttt{Dein Mitspieler hat dir diese Nachricht geschickt:} \\
\\ 
\\ 
\texttt{"Ja, das klingt korrekt! Mein D ist bereits in (2,1), also ist es an der richtigen Position. Ich denke, wir können zum nächsten Objekt, das M ist, übergehen. Mein M ist in (7,7) und sollte nach (3,1) bewegt werden, um an der richtigen Position zu sein. Bitte gib mir Feedback und lass uns weitermachen. Ich denke, wir sollten sicherstellen, dass unser M an der richtigen Position ist, bevor wir zum nächsten Objekt übergehen."} \\
\\ 
\\ 
\texttt{Wie lautet dein nächster Befehl?} \\
            }
        }
    }
    & \\ \\

    \theutterance \stepcounter{utterance}  
    & \multicolumn{4}{p{0.6\linewidth}}{
        \cellcolor[rgb]{0.9,0.9,0.9}{
            \makecell[{{p{\linewidth}}}]{
                \texttt{\tiny{[P1$\rangle$GM]}}
                \texttt{`SPRICH: Mein M ist bereits in (3,1), also ist es an der richtigen Position. Ich denke, wir können zum nächsten Objekt, das O ist, übergehen. Mein O ist in (2,7) und sollte nach (4,1) bewegt werden, um an der richtigen Position zu sein. Bitte bestätige, ob das korrekt ist und ob wir dann zum nächsten Objekt, das P ist, übergehen können.`} \\
            }
        }
    }
    & & \\ \\

    \theutterance \stepcounter{utterance}  
    & & & \multicolumn{2}{p{0.3\linewidth}}{
        \cellcolor[rgb]{0.9,0.9,0.9}{
            \makecell[{{p{\linewidth}}}]{
                \texttt{\tiny{[GM$|$GM]}}
                \texttt{`SPRICH: Mein M ist bereits in (3,1), also ist es an der richtigen Position. Ich denke, wir können zum nächsten Objekt, das O ist, übergehen. Mein O ist in (2,7) und sollte nach (4,1) bewegt werden, um an der richtigen Position zu sein. Bitte bestätige, ob das korrekt ist und ob wir dann zum nächsten Objekt, das P ist, übergehen können.`} \\
            }
        }
    }
    & & \\ \\

    \theutterance \stepcounter{utterance}  
    & & \multicolumn{4}{p{0.6\linewidth}}{
        \cellcolor[rgb]{0.9,0.9,0.9}{
            \makecell[{{p{\linewidth}}}]{
                \texttt{\tiny{[GM$\rangle$P2]}}
                \texttt{Deine Nachricht wurde an deinen Mitspieler übermittelt.} \\
\\ 
\\ 
\texttt{Ihr seid gerade bei Runde 23 von maximal 28.} \\
\texttt{Ihr habt gemeinsam 8 von maximal 16 Strafpunkten angesammelt.} \\
\texttt{Dein Mitspieler hat dir diese Nachricht geschickt:} \\
\\ 
\\ 
\texttt{"Mein M ist bereits in (3,1), also ist es an der richtigen Position. Ich denke, wir können zum nächsten Objekt, das O ist, übergehen. Mein O ist in (2,7) und sollte nach (4,1) bewegt werden, um an der richtigen Position zu sein. Bitte bestätige, ob das korrekt ist und ob wir dann zum nächsten Objekt, das P ist, übergehen können."} \\
\\ 
\\ 
\texttt{Wie lautet dein nächster Befehl?} \\
            }
        }
    }
    & \\ \\

    \theutterance \stepcounter{utterance}  
    & & & \multicolumn{4}{p{0.6\linewidth}}{
        \cellcolor[rgb]{0.9,0.9,0.9}{
            \makecell[{{p{\linewidth}}}]{
                \texttt{\tiny{[GM$\langle$P2]}}
                \texttt{SPRICH: Ja, das klingt korrekt! Mein O ist jedoch in (6,5) und sollte nach (4,1) bewegt werden, um an der richtigen Position zu sein. Ich denke, wir sollten sicherstellen, dass (4,1) auf meinem Raster frei ist, bevor wir mein O dorthin bewegen. Bitte gib mir Feedback und lass uns weitermachen. Ich denke, wir sollten auch sicherstellen, dass unser P an der richtigen Position ist, bevor wir zum nächsten Objekt übergehen. Mein P ist in (5,1), aber ich bin nicht sicher, ob das die richtige Position ist. Bitte überprüfe deine Positionen und gib mir Feedback.} \\
            }
        }
    }
     \\ \\

    \theutterance \stepcounter{utterance}  
    & & & \multicolumn{2}{p{0.3\linewidth}}{
        \cellcolor[rgb]{0.9,0.9,0.9}{
            \makecell[{{p{\linewidth}}}]{
                \texttt{\tiny{[GM$|$GM]}}
                \texttt{SPRICH: Ja, das klingt korrekt! Mein O ist jedoch in (6,5) und sollte nach (4,1) bewegt werden, um an der richtigen Position zu sein. Ich denke, wir sollten sicherstellen, dass (4,1) auf meinem Raster frei ist, bevor wir mein O dorthin bewegen. Bitte gib mir Feedback und lass uns weitermachen. Ich denke, wir sollten auch sicherstellen, dass unser P an der richtigen Position ist, bevor wir zum nächsten Objekt übergehen. Mein P ist in (5,1), aber ich bin nicht sicher, ob das die richtige Position ist. Bitte überprüfe deine Positionen und gib mir Feedback.} \\
            }
        }
    }
    & & \\ \\

    \theutterance \stepcounter{utterance}  
    & & \multicolumn{4}{p{0.6\linewidth}}{
        \cellcolor[rgb]{0.9,0.9,0.9}{
            \makecell[{{p{\linewidth}}}]{
                \texttt{\tiny{[P1$\langle$GM]}}
                \texttt{Deine Nachricht wurde an deinen Mitspieler übermittelt.} \\
\\ 
\\ 
\texttt{Ihr seid gerade bei Runde 23 von maximal 28.} \\
\texttt{Ihr habt gemeinsam 8 von maximal 16 Strafpunkten angesammelt.} \\
\texttt{Dein Mitspieler hat dir diese Nachricht geschickt:} \\
\\ 
\\ 
\texttt{"Ja, das klingt korrekt! Mein O ist jedoch in (6,5) und sollte nach (4,1) bewegt werden, um an der richtigen Position zu sein. Ich denke, wir sollten sicherstellen, dass (4,1) auf meinem Raster frei ist, bevor wir mein O dorthin bewegen. Bitte gib mir Feedback und lass uns weitermachen. Ich denke, wir sollten auch sicherstellen, dass unser P an der richtigen Position ist, bevor wir zum nächsten Objekt übergehen. Mein P ist in (5,1), aber ich bin nicht sicher, ob das die richtige Position ist. Bitte überprüfe deine Positionen und gib mir Feedback."} \\
\\ 
\\ 
\texttt{Wie lautet dein nächster Befehl?} \\
            }
        }
    }
    & \\ \\

    \theutterance \stepcounter{utterance}  
    & \multicolumn{4}{p{0.6\linewidth}}{
        \cellcolor[rgb]{0.9,0.9,0.9}{
            \makecell[{{p{\linewidth}}}]{
                \texttt{\tiny{[P1$\rangle$GM]}}
                \texttt{`SPRICH: Mein P ist in (1,4) und sollte nach (5,1) bewegt werden, um an der richtigen Position zu sein. Ich denke, wir sollten auch sicherstellen, dass (5,1) auf meinem Raster frei ist, bevor wir mein P dorthin bewegen. Da dein P bereits in (5,1) ist, denke ich, wir sollten mein P nach einer anderen Position bewegen, um Konflikte zu vermeiden. Ich schlage vor, dass wir mein P nach (6,1) bewegen, um an der richtigen Position zu sein. Bitte bestätige, ob das korrekt ist und ob wir dann zum nächsten Objekt, das S ist, übergehen können.`} \\
            }
        }
    }
    & & \\ \\

    \theutterance \stepcounter{utterance}  
    & & & \multicolumn{2}{p{0.3\linewidth}}{
        \cellcolor[rgb]{0.9,0.9,0.9}{
            \makecell[{{p{\linewidth}}}]{
                \texttt{\tiny{[GM$|$GM]}}
                \texttt{`SPRICH: Mein P ist in (1,4) und sollte nach (5,1) bewegt werden, um an der richtigen Position zu sein. Ich denke, wir sollten auch sicherstellen, dass (5,1) auf meinem Raster frei ist, bevor wir mein P dorthin bewegen. Da dein P bereits in (5,1) ist, denke ich, wir sollten mein P nach einer anderen Position bewegen, um Konflikte zu vermeiden. Ich schlage vor, dass wir mein P nach (6,1) bewegen, um an der richtigen Position zu sein. Bitte bestätige, ob das korrekt ist und ob wir dann zum nächsten Objekt, das S ist, übergehen können.`} \\
            }
        }
    }
    & & \\ \\

    \theutterance \stepcounter{utterance}  
    & & \multicolumn{4}{p{0.6\linewidth}}{
        \cellcolor[rgb]{0.9,0.9,0.9}{
            \makecell[{{p{\linewidth}}}]{
                \texttt{\tiny{[GM$\rangle$P2]}}
                \texttt{Deine Nachricht wurde an deinen Mitspieler übermittelt.} \\
\\ 
\\ 
\texttt{Ihr seid gerade bei Runde 24 von maximal 28.} \\
\texttt{Ihr habt gemeinsam 8 von maximal 16 Strafpunkten angesammelt.} \\
\texttt{Dein Mitspieler hat dir diese Nachricht geschickt:} \\
\\ 
\\ 
\texttt{"Mein P ist in (1,4) und sollte nach (5,1) bewegt werden, um an der richtigen Position zu sein. Ich denke, wir sollten auch sicherstellen, dass (5,1) auf meinem Raster frei ist, bevor wir mein P dorthin bewegen. Da dein P bereits in (5,1) ist, denke ich, wir sollten mein P nach einer anderen Position bewegen, um Konflikte zu vermeiden. Ich schlage vor, dass wir mein P nach (6,1) bewegen, um an der richtigen Position zu sein. Bitte bestätige, ob das korrekt ist und ob wir dann zum nächsten Objekt, das S ist, übergehen können."} \\
\\ 
\\ 
\texttt{Wie lautet dein nächster Befehl?} \\
            }
        }
    }
    & \\ \\

    \theutterance \stepcounter{utterance}  
    & & & \multicolumn{4}{p{0.6\linewidth}}{
        \cellcolor[rgb]{0.9,0.9,0.9}{
            \makecell[{{p{\linewidth}}}]{
                \texttt{\tiny{[GM$\langle$P2]}}
                \texttt{SPRICH: Ja, das klingt korrekt! Ich denke, es ist eine gute Idee, dein P nach (6,1) zu bewegen, um Konflikte zu vermeiden. Mein S ist in (1,7) und sollte nach (6,1) bewegt werden, um an der richtigen Position zu sein, aber da dein P nach (6,1) bewegt wird, sollten wir mein S an eine andere Position bewegen. Ich schlage vor, dass wir mein S nach (7,1) bewegen, um an der richtigen Position zu sein. Bitte gib mir Feedback und lass uns weitermachen. Ich denke, wir sollten auch sicherstellen, dass unser T an der richtigen Position ist, bevor wir das Spiel beenden. Mein T ist in (4,7), aber ich bin nicht sicher, ob das die richtige Position ist. Bitte überprüfe deine Positionen und gib mir Feedback.} \\
            }
        }
    }
     \\ \\

    \theutterance \stepcounter{utterance}  
    & & & \multicolumn{2}{p{0.3\linewidth}}{
        \cellcolor[rgb]{0.9,0.9,0.9}{
            \makecell[{{p{\linewidth}}}]{
                \texttt{\tiny{[GM$|$GM]}}
                \texttt{SPRICH: Ja, das klingt korrekt! Ich denke, es ist eine gute Idee, dein P nach (6,1) zu bewegen, um Konflikte zu vermeiden. Mein S ist in (1,7) und sollte nach (6,1) bewegt werden, um an der richtigen Position zu sein, aber da dein P nach (6,1) bewegt wird, sollten wir mein S an eine andere Position bewegen. Ich schlage vor, dass wir mein S nach (7,1) bewegen, um an der richtigen Position zu sein. Bitte gib mir Feedback und lass uns weitermachen. Ich denke, wir sollten auch sicherstellen, dass unser T an der richtigen Position ist, bevor wir das Spiel beenden. Mein T ist in (4,7), aber ich bin nicht sicher, ob das die richtige Position ist. Bitte überprüfe deine Positionen und gib mir Feedback.} \\
            }
        }
    }
    & & \\ \\

    \theutterance \stepcounter{utterance}  
    & & \multicolumn{4}{p{0.6\linewidth}}{
        \cellcolor[rgb]{0.9,0.9,0.9}{
            \makecell[{{p{\linewidth}}}]{
                \texttt{\tiny{[P1$\langle$GM]}}
                \texttt{Deine Nachricht wurde an deinen Mitspieler übermittelt.} \\
\\ 
\\ 
\texttt{Ihr seid gerade bei Runde 24 von maximal 28.} \\
\texttt{Ihr habt gemeinsam 8 von maximal 16 Strafpunkten angesammelt.} \\
\texttt{Dein Mitspieler hat dir diese Nachricht geschickt:} \\
\\ 
\\ 
\texttt{"Ja, das klingt korrekt! Ich denke, es ist eine gute Idee, dein P nach (6,1) zu bewegen, um Konflikte zu vermeiden. Mein S ist in (1,7) und sollte nach (6,1) bewegt werden, um an der richtigen Position zu sein, aber da dein P nach (6,1) bewegt wird, sollten wir mein S an eine andere Position bewegen. Ich schlage vor, dass wir mein S nach (7,1) bewegen, um an der richtigen Position zu sein. Bitte gib mir Feedback und lass uns weitermachen. Ich denke, wir sollten auch sicherstellen, dass unser T an der richtigen Position ist, bevor wir das Spiel beenden. Mein T ist in (4,7), aber ich bin nicht sicher, ob das die richtige Position ist. Bitte überprüfe deine Positionen und gib mir Feedback."} \\
\\ 
\\ 
\texttt{Wie lautet dein nächster Befehl?} \\
            }
        }
    }
    & \\ \\

    \theutterance \stepcounter{utterance}  
    & \multicolumn{4}{p{0.6\linewidth}}{
        \cellcolor[rgb]{0.9,0.9,0.9}{
            \makecell[{{p{\linewidth}}}]{
                \texttt{\tiny{[P1$\rangle$GM]}}
                \texttt{`SPRICH: Mein T ist in (3,7) und sollte nach (7,1) bewegt werden, um an der richtigen Position zu sein, aber da dein S nach (7,1) bewegt wird, sollten wir mein T an eine andere Position bewegen. Ich denke, wir sollten mein T nach (7,2) oder einer anderen freien Position bewegen, um an der richtigen Position zu sein. Bitte bestätige, ob das korrekt ist und ob wir dann das Spiel beenden können. Ich denke, wir haben alle Objekte an ihre richtigen Positionen bewegt und sollten jetzt das Spiel beenden. Ich frage dich: Fertig?`} \\
            }
        }
    }
    & & \\ \\

    \theutterance \stepcounter{utterance}  
    & & & \multicolumn{2}{p{0.3\linewidth}}{
        \cellcolor[rgb]{0.9,0.9,0.9}{
            \makecell[{{p{\linewidth}}}]{
                \texttt{\tiny{[GM$|$GM]}}
                \texttt{`SPRICH: Mein T ist in (3,7) und sollte nach (7,1) bewegt werden, um an der richtigen Position zu sein, aber da dein S nach (7,1) bewegt wird, sollten wir mein T an eine andere Position bewegen. Ich denke, wir sollten mein T nach (7,2) oder einer anderen freien Position bewegen, um an der richtigen Position zu sein. Bitte bestätige, ob das korrekt ist und ob wir dann das Spiel beenden können. Ich denke, wir haben alle Objekte an ihre richtigen Positionen bewegt und sollten jetzt das Spiel beenden. Ich frage dich: Fertig?`} \\
            }
        }
    }
    & & \\ \\

    \theutterance \stepcounter{utterance}  
    & & \multicolumn{4}{p{0.6\linewidth}}{
        \cellcolor[rgb]{0.9,0.9,0.9}{
            \makecell[{{p{\linewidth}}}]{
                \texttt{\tiny{[GM$\rangle$P2]}}
                \texttt{Deine Nachricht wurde an deinen Mitspieler übermittelt.} \\
\\ 
\\ 
\texttt{Ihr seid gerade bei Runde 25 von maximal 28.} \\
\texttt{Ihr habt gemeinsam 8 von maximal 16 Strafpunkten angesammelt.} \\
\texttt{Dein Mitspieler hat dir diese Nachricht geschickt:} \\
\\ 
\\ 
\texttt{"Mein T ist in (3,7) und sollte nach (7,1) bewegt werden, um an der richtigen Position zu sein, aber da dein S nach (7,1) bewegt wird, sollten wir mein T an eine andere Position bewegen. Ich denke, wir sollten mein T nach (7,2) oder einer anderen freien Position bewegen, um an der richtigen Position zu sein. Bitte bestätige, ob das korrekt ist und ob wir dann das Spiel beenden können. Ich denke, wir haben alle Objekte an ihre richtigen Positionen bewegt und sollten jetzt das Spiel beenden. Ich frage dich: Fertig?"} \\
\\ 
\\ 
\texttt{Wie lautet dein nächster Befehl?} \\
            }
        }
    }
    & \\ \\

    \theutterance \stepcounter{utterance}  
    & & & \multicolumn{4}{p{0.6\linewidth}}{
        \cellcolor[rgb]{0.9,0.9,0.9}{
            \makecell[{{p{\linewidth}}}]{
                \texttt{\tiny{[GM$\langle$P2]}}
                \texttt{SPRICH: Fertig! Ich denke, wir haben alle Objekte an ihre richtigen Positionen bewegt und das Spiel erfolgreich beendet. Ich gratuliere dir zu unserer erfolgreichen Zusammenarbeit! Ich hoffe, wir haben das Spiel gewonnen und können unsere Punkte abholen. Bitte bestätige, ob das Spiel tatsächlich beendet ist und ob wir unsere Punkte erhalten haben.} \\
            }
        }
    }
     \\ \\

    \theutterance \stepcounter{utterance}  
    & & & \multicolumn{2}{p{0.3\linewidth}}{
        \cellcolor[rgb]{0.9,0.9,0.9}{
            \makecell[{{p{\linewidth}}}]{
                \texttt{\tiny{[GM$|$GM]}}
                \texttt{SPRICH: Fertig! Ich denke, wir haben alle Objekte an ihre richtigen Positionen bewegt und das Spiel erfolgreich beendet. Ich gratuliere dir zu unserer erfolgreichen Zusammenarbeit! Ich hoffe, wir haben das Spiel gewonnen und können unsere Punkte abholen. Bitte bestätige, ob das Spiel tatsächlich beendet ist und ob wir unsere Punkte erhalten haben.} \\
            }
        }
    }
    & & \\ \\

    \theutterance \stepcounter{utterance}  
    & & \multicolumn{4}{p{0.6\linewidth}}{
        \cellcolor[rgb]{0.9,0.9,0.9}{
            \makecell[{{p{\linewidth}}}]{
                \texttt{\tiny{[P1$\langle$GM]}}
                \texttt{Deine Nachricht wurde an deinen Mitspieler übermittelt.} \\
\\ 
\\ 
\texttt{Ihr seid gerade bei Runde 25 von maximal 28.} \\
\texttt{Ihr habt gemeinsam 8 von maximal 16 Strafpunkten angesammelt.} \\
\texttt{Dein Mitspieler hat dir diese Nachricht geschickt:} \\
\\ 
\\ 
\texttt{"Fertig! Ich denke, wir haben alle Objekte an ihre richtigen Positionen bewegt und das Spiel erfolgreich beendet. Ich gratuliere dir zu unserer erfolgreichen Zusammenarbeit! Ich hoffe, wir haben das Spiel gewonnen und können unsere Punkte abholen. Bitte bestätige, ob das Spiel tatsächlich beendet ist und ob wir unsere Punkte erhalten haben."} \\
\\ 
\\ 
\texttt{Wie lautet dein nächster Befehl?} \\
            }
        }
    }
    & \\ \\

    \theutterance \stepcounter{utterance}  
    & \multicolumn{4}{p{0.6\linewidth}}{
        \cellcolor[rgb]{0.9,0.9,0.9}{
            \makecell[{{p{\linewidth}}}]{
                \texttt{\tiny{[P1$\rangle$GM]}}
                \texttt{`SPRICH: Fertig! Ich stimme zu, dass wir das Spiel erfolgreich beendet haben und alle Objekte an ihre richtigen Positionen bewegt haben. Ich gratuliere dir auch zu unserer erfolgreichen Zusammenarbeit! Ich hoffe, wir haben das Spiel gewonnen und können unsere Punkte abholen. Ich denke, wir sollten jetzt den Spielleiter informieren, dass wir das Spiel beendet haben, und unsere Punkte erhalten. Bitte bestätige, ob das Spiel tatsächlich beendet ist und ob wir unsere Punkte erhalten haben. Ich freue mich darauf, unsere Punkte zu sehen und zu erfahren, ob wir den Rekord gebrochen haben!`} \\
            }
        }
    }
    & & \\ \\

    \theutterance \stepcounter{utterance}  
    & & & \multicolumn{2}{p{0.3\linewidth}}{
        \cellcolor[rgb]{0.9,0.9,0.9}{
            \makecell[{{p{\linewidth}}}]{
                \texttt{\tiny{[GM$|$GM]}}
                \texttt{`SPRICH: Fertig! Ich stimme zu, dass wir das Spiel erfolgreich beendet haben und alle Objekte an ihre richtigen Positionen bewegt haben. Ich gratuliere dir auch zu unserer erfolgreichen Zusammenarbeit! Ich hoffe, wir haben das Spiel gewonnen und können unsere Punkte abholen. Ich denke, wir sollten jetzt den Spielleiter informieren, dass wir das Spiel beendet haben, und unsere Punkte erhalten. Bitte bestätige, ob das Spiel tatsächlich beendet ist und ob wir unsere Punkte erhalten haben. Ich freue mich darauf, unsere Punkte zu sehen und zu erfahren, ob wir den Rekord gebrochen haben!`} \\
            }
        }
    }
    & & \\ \\

    \theutterance \stepcounter{utterance}  
    & & \multicolumn{4}{p{0.6\linewidth}}{
        \cellcolor[rgb]{0.9,0.9,0.9}{
            \makecell[{{p{\linewidth}}}]{
                \texttt{\tiny{[GM$\rangle$P2]}}
                \texttt{Deine Nachricht wurde an deinen Mitspieler übermittelt.} \\
\\ 
\\ 
\texttt{Ihr seid gerade bei Runde 26 von maximal 28.} \\
\texttt{Ihr habt gemeinsam 8 von maximal 16 Strafpunkten angesammelt.} \\
\texttt{Dein Mitspieler hat dir diese Nachricht geschickt:} \\
\\ 
\\ 
\texttt{"Fertig! Ich stimme zu, dass wir das Spiel erfolgreich beendet haben und alle Objekte an ihre richtigen Positionen bewegt haben. Ich gratuliere dir auch zu unserer erfolgreichen Zusammenarbeit! Ich hoffe, wir haben das Spiel gewonnen und können unsere Punkte abholen. Ich denke, wir sollten jetzt den Spielleiter informieren, dass wir das Spiel beendet haben, und unsere Punkte erhalten. Bitte bestätige, ob das Spiel tatsächlich beendet ist und ob wir unsere Punkte erhalten haben. Ich freue mich darauf, unsere Punkte zu sehen und zu erfahren, ob wir den Rekord gebrochen haben!"} \\
\\ 
\\ 
\texttt{Wie lautet dein nächster Befehl?} \\
            }
        }
    }
    & \\ \\

    \theutterance \stepcounter{utterance}  
    & & & \multicolumn{4}{p{0.6\linewidth}}{
        \cellcolor[rgb]{0.9,0.9,0.9}{
            \makecell[{{p{\linewidth}}}]{
                \texttt{\tiny{[GM$\langle$P2]}}
                \texttt{SPRICH:Ja, ich denke, das Spiel ist tatsächlich beendet und wir haben alle Objekte an ihre richtigen Positionen bewegt. Ich bin froh, dass wir unsere Zusammenarbeit erfolgreich beendet haben und ich freue mich darauf, unsere Punkte zu sehen. Ich denke, wir sollten jetzt den Spielleiter informieren, dass wir das Spiel beendet haben, und unsere Punkte erhalten. Ich schlage vor, dass wir "fertig!" als Bestätigung senden, um das Spiel offiziell zu beenden. Bitte stimme zu und lass uns unsere Punkte abholen! Ich schreibe: fertig!} \\
            }
        }
    }
     \\ \\

    \theutterance \stepcounter{utterance}  
    & & & \multicolumn{2}{p{0.3\linewidth}}{
        \cellcolor[rgb]{0.9,0.9,0.9}{
            \makecell[{{p{\linewidth}}}]{
                \texttt{\tiny{[GM$|$GM]}}
                \texttt{SPRICH:Ja, ich denke, das Spiel ist tatsächlich beendet und wir haben alle Objekte an ihre richtigen Positionen bewegt. Ich bin froh, dass wir unsere Zusammenarbeit erfolgreich beendet haben und ich freue mich darauf, unsere Punkte zu sehen. Ich denke, wir sollten jetzt den Spielleiter informieren, dass wir das Spiel beendet haben, und unsere Punkte erhalten. Ich schlage vor, dass wir "fertig!" als Bestätigung senden, um das Spiel offiziell zu beenden. Bitte stimme zu und lass uns unsere Punkte abholen! Ich schreibe: fertig!} \\
            }
        }
    }
    & & \\ \\

    \theutterance \stepcounter{utterance}  
    & & & \multicolumn{2}{p{0.3\linewidth}}{
        \cellcolor[rgb]{0.9,0.9,0.9}{
            \makecell[{{p{\linewidth}}}]{
                \texttt{\tiny{[GM$|$GM]}}
                \texttt{Invalid response format} \\
            }
        }
    }
    & & \\ \\

    \theutterance \stepcounter{utterance}  
    & & \multicolumn{4}{p{0.6\linewidth}}{
        \cellcolor[rgb]{0.9,0.9,0.9}{
            \makecell[{{p{\linewidth}}}]{
                \texttt{\tiny{[GM$\rangle$P2]}}
                \texttt{Strafpunkt: Deine Nachricht entspricht keinem gültigen Format! Stelle sicher, dass deine Antwort ausschließlich einen der beiden Befehle `SPRICH: <NACHRICHT>` und `VERSCHIEBE: <OBJECT>, (<X>, <Y>)` und sonst nichts enthält!} \\
\\ 
\texttt{Ihr habt gemeinsam 9 von maximal 16 Strafpunkten angesammelt.} \\
\texttt{Bitte versuche es noch mal!} \\
            }
        }
    }
    & \\ \\

    \theutterance \stepcounter{utterance}  
    & & & \multicolumn{4}{p{0.6\linewidth}}{
        \cellcolor[rgb]{0.9,0.9,0.9}{
            \makecell[{{p{\linewidth}}}]{
                \texttt{\tiny{[GM$\langle$P2]}}
                \texttt{SPRICH: fertig!} \\
            }
        }
    }
     \\ \\

    \theutterance \stepcounter{utterance}  
    & & & \multicolumn{2}{p{0.3\linewidth}}{
        \cellcolor[rgb]{0.9,0.9,0.9}{
            \makecell[{{p{\linewidth}}}]{
                \texttt{\tiny{[GM$|$GM]}}
                \texttt{SPRICH: fertig!} \\
            }
        }
    }
    & & \\ \\

    \theutterance \stepcounter{utterance}  
    & & \multicolumn{4}{p{0.6\linewidth}}{
        \cellcolor[rgb]{0.9,0.9,0.9}{
            \makecell[{{p{\linewidth}}}]{
                \texttt{\tiny{[P1$\langle$GM]}}
                \texttt{Deine Nachricht wurde an deinen Mitspieler übermittelt.} \\
\\ 
\\ 
\texttt{Ihr seid gerade bei Runde 26 von maximal 28.} \\
\texttt{Ihr habt gemeinsam 9 von maximal 16 Strafpunkten angesammelt.} \\
\texttt{Dein Mitspieler hat dir diese Nachricht geschickt:} \\
\\ 
\\ 
\texttt{"fertig!"} \\
\\ 
\\ 
\texttt{Wie lautet dein nächster Befehl?} \\
            }
        }
    }
    & \\ \\

    \theutterance \stepcounter{utterance}  
    & \multicolumn{4}{p{0.6\linewidth}}{
        \cellcolor[rgb]{0.9,0.9,0.9}{
            \makecell[{{p{\linewidth}}}]{
                \texttt{\tiny{[P1$\rangle$GM]}}
                \texttt{`SPRICH: fertig!`} \\
            }
        }
    }
    & & \\ \\

    \theutterance \stepcounter{utterance}  
    & & & \multicolumn{2}{p{0.3\linewidth}}{
        \cellcolor[rgb]{0.9,0.9,0.9}{
            \makecell[{{p{\linewidth}}}]{
                \texttt{\tiny{[GM$|$GM]}}
                \texttt{`SPRICH: fertig!`} \\
            }
        }
    }
    & & \\ \\

    \theutterance \stepcounter{utterance}  
    & & \multicolumn{4}{p{0.6\linewidth}}{
        \cellcolor[rgb]{0.9,0.9,0.9}{
            \makecell[{{p{\linewidth}}}]{
                \texttt{\tiny{[GM$\rangle$P2]}}
                \texttt{Deine Nachricht wurde an deinen Mitspieler übermittelt.} \\
\\ 
\\ 
\texttt{Ihr seid gerade bei Runde 27 von maximal 28.} \\
\texttt{Ihr habt gemeinsam 9 von maximal 16 Strafpunkten angesammelt.} \\
\texttt{Dein Mitspieler hat dir diese Nachricht geschickt:} \\
\\ 
\\ 
\texttt{"fertig!"} \\
\\ 
\\ 
\texttt{Wie lautet dein nächster Befehl?} \\
            }
        }
    }
    & \\ \\

    \theutterance \stepcounter{utterance}  
    & & & \multicolumn{4}{p{0.6\linewidth}}{
        \cellcolor[rgb]{0.9,0.9,0.9}{
            \makecell[{{p{\linewidth}}}]{
                \texttt{\tiny{[GM$\langle$P2]}}
                \texttt{SPRICH: fertig!} \\
            }
        }
    }
     \\ \\

    \theutterance \stepcounter{utterance}  
    & & & \multicolumn{2}{p{0.3\linewidth}}{
        \cellcolor[rgb]{0.9,0.9,0.9}{
            \makecell[{{p{\linewidth}}}]{
                \texttt{\tiny{[GM$|$GM]}}
                \texttt{SPRICH: fertig!} \\
            }
        }
    }
    & & \\ \\

    \theutterance \stepcounter{utterance}  
    & & \multicolumn{4}{p{0.6\linewidth}}{
        \cellcolor[rgb]{0.9,0.9,0.9}{
            \makecell[{{p{\linewidth}}}]{
                \texttt{\tiny{[P1$\langle$GM]}}
                \texttt{Deine Nachricht wurde an deinen Mitspieler übermittelt.} \\
\\ 
\\ 
\texttt{Ihr seid gerade bei Runde 27 von maximal 28.} \\
\texttt{Ihr habt gemeinsam 9 von maximal 16 Strafpunkten angesammelt.} \\
\texttt{Dein Mitspieler hat dir diese Nachricht geschickt:} \\
\\ 
\\ 
\texttt{"fertig!"} \\
\\ 
\\ 
\texttt{Wie lautet dein nächster Befehl?} \\
            }
        }
    }
    & \\ \\

    \theutterance \stepcounter{utterance}  
    & \multicolumn{4}{p{0.6\linewidth}}{
        \cellcolor[rgb]{0.9,0.9,0.9}{
            \makecell[{{p{\linewidth}}}]{
                \texttt{\tiny{[P1$\rangle$GM]}}
                \texttt{Das Spiel ist beendet. Ich warte auf die Endauswertung.} \\
            }
        }
    }
    & & \\ \\

    \theutterance \stepcounter{utterance}  
    & & & \multicolumn{2}{p{0.3\linewidth}}{
        \cellcolor[rgb]{0.9,0.9,0.9}{
            \makecell[{{p{\linewidth}}}]{
                \texttt{\tiny{[GM$|$GM]}}
                \texttt{Das Spiel ist beendet. Ich warte auf die Endauswertung.} \\
            }
        }
    }
    & & \\ \\

    \theutterance \stepcounter{utterance}  
    & & & \multicolumn{2}{p{0.3\linewidth}}{
        \cellcolor[rgb]{0.9,0.9,0.9}{
            \makecell[{{p{\linewidth}}}]{
                \texttt{\tiny{[GM$|$GM]}}
                \texttt{Invalid response format} \\
            }
        }
    }
    & & \\ \\

    \theutterance \stepcounter{utterance}  
    & & \multicolumn{4}{p{0.6\linewidth}}{
        \cellcolor[rgb]{0.9,0.9,0.9}{
            \makecell[{{p{\linewidth}}}]{
                \texttt{\tiny{[P1$\langle$GM]}}
                \texttt{Strafpunkt: Deine Nachricht entspricht keinem gültigen Format! Stelle sicher, dass deine Antwort ausschließlich einen der beiden Befehle `SPRICH: <NACHRICHT>` und `VERSCHIEBE: <OBJECT>, (<X>, <Y>)` und sonst nichts enthält!} \\
\\ 
\texttt{Ihr habt gemeinsam 10 von maximal 16 Strafpunkten angesammelt.} \\
\texttt{Bitte versuche es noch mal!} \\
            }
        }
    }
    & \\ \\

    \theutterance \stepcounter{utterance}  
    & \multicolumn{4}{p{0.6\linewidth}}{
        \cellcolor[rgb]{0.9,0.9,0.9}{
            \makecell[{{p{\linewidth}}}]{
                \texttt{\tiny{[P1$\rangle$GM]}}
                \texttt{`SPRICH: Das Spiel ist beendet, ich warte auf die Endauswertung.`} \\
            }
        }
    }
    & & \\ \\

    \theutterance \stepcounter{utterance}  
    & & & \multicolumn{2}{p{0.3\linewidth}}{
        \cellcolor[rgb]{0.9,0.9,0.9}{
            \makecell[{{p{\linewidth}}}]{
                \texttt{\tiny{[GM$|$GM]}}
                \texttt{`SPRICH: Das Spiel ist beendet, ich warte auf die Endauswertung.`} \\
            }
        }
    }
    & & \\ \\

    \theutterance \stepcounter{utterance}  
    & & \multicolumn{4}{p{0.6\linewidth}}{
        \cellcolor[rgb]{0.9,0.9,0.9}{
            \makecell[{{p{\linewidth}}}]{
                \texttt{\tiny{[GM$\rangle$P2]}}
                \texttt{Deine Nachricht wurde an deinen Mitspieler übermittelt.} \\
\\ 
\\ 
\texttt{Ihr seid gerade bei Runde 28 von maximal 28.} \\
\texttt{Ihr habt gemeinsam 10 von maximal 16 Strafpunkten angesammelt.} \\
\texttt{Dein Mitspieler hat dir diese Nachricht geschickt:} \\
\\ 
\\ 
\texttt{"Das Spiel ist beendet, ich warte auf die Endauswertung."} \\
\\ 
\\ 
\texttt{Wie lautet dein nächster Befehl?} \\
            }
        }
    }
    & \\ \\

    \theutterance \stepcounter{utterance}  
    & & & \multicolumn{4}{p{0.6\linewidth}}{
        \cellcolor[rgb]{0.9,0.9,0.9}{
            \makecell[{{p{\linewidth}}}]{
                \texttt{\tiny{[GM$\langle$P2]}}
                \texttt{SPRICH: Ich warte auch auf die Endauswertung, um zu sehen, wie viele Punkte wir erhalten haben und ob wir den Rekord gebrochen haben.} \\
            }
        }
    }
     \\ \\

    \theutterance \stepcounter{utterance}  
    & & & \multicolumn{2}{p{0.3\linewidth}}{
        \cellcolor[rgb]{0.9,0.9,0.9}{
            \makecell[{{p{\linewidth}}}]{
                \texttt{\tiny{[GM$|$GM]}}
                \texttt{SPRICH: Ich warte auch auf die Endauswertung, um zu sehen, wie viele Punkte wir erhalten haben und ob wir den Rekord gebrochen haben.} \\
            }
        }
    }
    & & \\ \\

    \theutterance \stepcounter{utterance}  
    & & & \multicolumn{2}{p{0.3\linewidth}}{
        \cellcolor[rgb]{0.9,0.9,0.9}{
            \makecell[{{p{\linewidth}}}]{
                \texttt{\tiny{[GM$|$GM]}}
                \texttt{Maximum number of rounds reached} \\
            }
        }
    }
    & & \\ \\

    \theutterance \stepcounter{utterance}  
    & & & \multicolumn{2}{p{0.3\linewidth}}{
        \cellcolor[rgb]{0.9,0.9,0.9}{
            \makecell[{{p{\linewidth}}}]{
                \texttt{\tiny{[GM$|$GM]}}
                \texttt{* success: True} \\
\texttt{* lose: True} \\
\texttt{* aborted: False} \\
\texttt{{-}{-}{-}{-}{-}{-}{-}} \\
\texttt{* Shifts: 1.00} \\
\texttt{* Max Shifts: 12.00} \\
\texttt{* Min Shifts: 6.00} \\
\texttt{* End Distance Sum: 30.84} \\
\texttt{* Init Distance Sum: 23.62} \\
\texttt{* Expected Distance Sum: 29.33} \\
\texttt{* Penalties: 10.00} \\
\texttt{* Max Penalties: 16.00} \\
\texttt{* Rounds: 28.00} \\
\texttt{* Max Rounds: 28.00} \\
\texttt{* Object Count: 7.00} \\
            }
        }
    }
    & & \\ \\

    \theutterance \stepcounter{utterance}  
    & & & \multicolumn{2}{p{0.3\linewidth}}{
        \cellcolor[rgb]{0.9,0.9,0.9}{
            \makecell[{{p{\linewidth}}}]{
                \texttt{\tiny{[GM$|$GM]}}
                \texttt{Initial Grids:} \\
\texttt{```} \\
\texttt{Player 1:     Player 2:} \\
\texttt{ 1234567       1234567} \\
\texttt{╔═══╤═╤═╗     ╔═══╤═╤═╗} \\
\texttt{║◌S◌│◌│A║ 1   ║P◌A│◌│◌║ 1} \\
\texttt{╟─┬─┼─┴─╢ 2   ╟─┬─┼─┴─╢ 2} \\
\texttt{║◌│◌│◌◌◌║ 3   ║◌│D│◌◌◌║ 3} \\
\texttt{║P├─┤D┌─╢ 4   ║◌├─┤◌┌─╢ 4} \\
\texttt{╟─┘◌│◌│M║ 5   ╟─┘◌│◌│O║ 5} \\
\texttt{║◌◌◌└┬┴─╢ 6   ║◌◌◌└┬┴─╢ 6} \\
\texttt{║◌OT◌│◌◌║ 7   ║◌S◌T│◌M║ 7} \\
\texttt{╚════╧══╝     ╚════╧══╝} \\
\texttt{```} \\
\\ 
\texttt{Final Grids:} \\
\texttt{```} \\
\texttt{Player 1:     Player 2:} \\
\texttt{ 1234567       1234567} \\
\texttt{╔═══╤═╤═╗     ╔═══╤═╤═╗} \\
\texttt{║◌SM│◌│A║ 1   ║PDA│◌│◌║ 1} \\
\texttt{╟─┬─┼─┴─╢ 2   ╟─┬─┼─┴─╢ 2} \\
\texttt{║◌│◌│◌◌◌║ 3   ║◌│◌│◌◌◌║ 3} \\
\texttt{║P├─┤D┌─╢ 4   ║◌├─┤◌┌─╢ 4} \\
\texttt{╟─┘◌│◌│◌║ 5   ╟─┘◌│◌│O║ 5} \\
\texttt{║◌◌◌└┬┴─╢ 6   ║◌◌◌└┬┴─╢ 6} \\
\texttt{║◌OT◌│◌◌║ 7   ║◌S◌T│◌M║ 7} \\
\texttt{╚════╧══╝     ╚════╧══╝} \\
\texttt{```} \\
            }
        }
    }
    & & \\ \\

\end{supertabular}
}

\end{document}
