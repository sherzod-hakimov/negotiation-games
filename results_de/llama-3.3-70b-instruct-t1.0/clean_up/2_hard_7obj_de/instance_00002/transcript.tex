
\documentclass{article}
\usepackage{colortbl}
\usepackage{makecell}
\usepackage{multirow}
\usepackage{supertabular}

\begin{document}

\newcounter{utterance}

\twocolumn

{ \footnotesize  \setcounter{utterance}{1}
\setlength{\tabcolsep}{0pt}
\begin{supertabular}{c@{$\;$}|p{.15\linewidth}@{}p{.15\linewidth}p{.15\linewidth}p{.15\linewidth}p{.15\linewidth}p{.15\linewidth}}

    \# & $\;$A & \multicolumn{4}{c}{Game Master} & $\;\:$B\\
    \hline 

    \theutterance \stepcounter{utterance}  

    & & \multicolumn{4}{p{0.6\linewidth}}{\cellcolor[rgb]{0.9,0.9,0.9}{%
	\makecell[{{p{\linewidth}}}]{% 
	  \tt {\tiny [A$\langle$GM]}  
	 Ich bin der Spielleiter, und du spielst ein kooperatives Spiel mit dem folgenden Raster als Spielbrett:\\ \tt \\ \tt ```\\ \tt  1234567\\ \tt ╔═╤═╤═╤═╗\\ \tt ║A│◌│O│◌║ 1\\ \tt ╟─┼─┤◌│◌║ 2\\ \tt ║◌│T│◌│P║ 3\\ \tt ║◌│◌├─┴─╢ 4\\ \tt ╟─┴─┤◌◌◌║ 5\\ \tt ║D◌◌└─┬─╢ 6\\ \tt ║◌◌◌S◌│M║ 7\\ \tt ╚═════╧═╝\\ \tt \\ \tt ```\\ \tt \\ \tt * Am oberen Rand sind die x-Koordinaten angegeben, die nach rechts hin zunehmen, und am rechten Rand die y-Koordinaten, die nach unten zunehmen.\\ \tt * Die folgenden Objekte sind zufällig auf deinem Raster platziert: 'P', 'O', 'T', 'S', 'D', 'A', 'M'.\\ \tt \\ \tt \\ \tt Dein Mitspieler sieht eine Variation des Spielbretts, auf dem dieselben Objekte an anderen zufälligen Positionen platziert sind. Du kannst das Spielbrett des anderen Spielers nicht sehen, und er kann deins nicht sehen.\\ \tt \\ \tt \\ \tt **Ziel des Spiels:**\\ \tt \\ \tt Beide Spieler müssen die Objekte auf ihrem jeweiligen Spielbrett so bewegen, dass identische Objekte am Ende an denselben Koordinaten liegen. Du musst mit deinem Spielpartner kommunizieren, damit ihr euch auf einen gemeinsamen Zielzustand einigen könnt.\\ \tt \\ \tt \\ \tt **Spielregeln:**\\ \tt \\ \tt * In jedem Spielzug kannst du genau eine der folgenden zwei Aktionen ausführen:\\ \tt 1. `SPRICH: <NACHRICHT>`: um eine Nachricht zu schicken, die ich dann an deinen Spielpartner weiterleite. Als Nachricht gilt alles bis zum nächsten Zeilenumbruch.\\ \tt 2. `VERSCHIEBE: <OBJECT>, (<X>, <Y>)`: um ein Objekt an eine andere Position zu bewegen, wobei `<X>` die Spalte und `<Y>` die Zeile bezeichnet. Ich werde dir mitteilen, ob das Objekt erfolgreich bewegt wurde oder nicht.\\ \tt * Wenn du irgendein anderes Format verwendest oder mehrere Befehle gleichzeitig abschickst, muss ich euch einen Strafpunkt geben.\\ \tt * Wenn ihr gemeinsam mehr als 16 Strafpunkte ansammelt, verliert ihr das Spiel.\\ \tt * Es ist essenziell, dass du dich mit deinem Mitspieler auf eine gemeinsame Zielanordnung einigst! Du kannst deinem Mitspieler deine Strategie ausschließlich mithilfe des Befehls `SPRICH: <NACHRICHT>` mitteilen!\\ \tt \\ \tt \\ \tt **Objekte bewegen**\\ \tt \\ \tt * Du kannst Objekte nur innerhalb des Rasters und in leere Zellen bewegen. Leere Zellen erkennst am Zeichen '◌'.\\ \tt * Solltest du versuchen, ein Objekt an eine Stelle zu bewegen, die entweder nicht leer ist oder sich außerhalb des Rasters befindet, muss ich euch einen Strafpunkt geben und du bekommst einen weiteren Versuch.\\ \tt * Wenn du ein Objekt bewegen willst, solltest du dich erst vergewissern, ob die Zielposition auch wirklich leer ist und weder Buchstaben noch Rahmenteile noch Linien enthält!\\ \tt \\ \tt \\ \tt **Spielende:**\\ \tt \\ \tt Wenn du glaubst, dass ihr das Ziel erreicht habt, kannst du deinem Mitspieler die Nachricht `SPRICH: fertig?` schicken. Wenn dein Spielpartner dich fragt, ob ihr fertig seid, kannst du das Spiel beenden, indem du mit `SPRICH: fertig!` antwortest.\\ \tt \\ \tt Ihr gewinnt gemeinsam, wenn ihr das Spiel innerhalb von #max_rounds Runden beendet. Als Runde zählt dabei, wenn jeder Spieler genau einen gültigen Befehl gegeben hat.\\ \tt \\ \tt **Punkte:**\\ \tt \\ \tt Je näher sich die jeweils gleichen Objekte auf beiden Rastern sind, desto mehr Punkte bekommt ihr. Für Strafpunkte gibt es Punktabzug. Könnt ihr den Rekord brechen?\\ \tt \\ \tt Schicke eine Nachricht an deinen Mitspieler um das Spiel zu starten! 
	  } 
	   } 
	   } 
	 & \\ 
 

    \theutterance \stepcounter{utterance}  

    & \multicolumn{4}{p{0.6\linewidth}}{\cellcolor[rgb]{0.8,1,0.9}{%
	\makecell[{{p{\linewidth}}}]{% 
	  \tt {\tiny [A$\rangle$GM]}  
	 Ich möchte das Spiel beginnen, indem ich eine Nachricht an meinen Mitspieler sende. Hier ist mein erster Befehl:\\ \tt \\ \tt `SPRICH: Hallo, ich bin bereit, loszulegen. Lass uns zunächst unsere Strategie besprechen. Ich schlage vor, dass wir uns auf eine feste Position für jedes Objekt einigen und dann unsere Züge darauf abstimmen. Wie sieht es bei dir aus? Hast du bereits eine Idee, wie wir vorgehen können?` 
	  } 
	   } 
	   } 
	 & & \\ 
 

    \theutterance \stepcounter{utterance}  

    & & & \multicolumn{2}{p{0.3\linewidth}}{\cellcolor[rgb]{0.95,0.95,0.95}{%
	\makecell[{{p{\linewidth}}}]{% 
	  \tt {\tiny [GM$|$GM]}  
	 Ich möchte das Spiel beginnen, indem ich eine Nachricht an meinen Mitspieler sende. Hier ist mein erster Befehl:\\ \tt \\ \tt `SPRICH: Hallo, ich bin bereit, loszulegen. Lass uns zunächst unsere Strategie besprechen. Ich schlage vor, dass wir uns auf eine feste Position für jedes Objekt einigen und dann unsere Züge darauf abstimmen. Wie sieht es bei dir aus? Hast du bereits eine Idee, wie wir vorgehen können?` 
	  } 
	   } 
	   } 
	 & & \\ 
 

    \theutterance \stepcounter{utterance}  

    & & & \multicolumn{2}{p{0.3\linewidth}}{\cellcolor[rgb]{0.95,0.95,0.95}{%
	\makecell[{{p{\linewidth}}}]{% 
	  \tt {\tiny [GM$|$GM]}  
	 Invalid format: head is not empty: 'Ich möchte das Spiel beginnen, indem ich eine Nachricht an meinen Mitspieler sende. Hier ist mein erster Befehl:\\ \tt \\ \tt ' 
	  } 
	   } 
	   } 
	 & & \\ 
 

    \theutterance \stepcounter{utterance}  

    & & \multicolumn{4}{p{0.6\linewidth}}{\cellcolor[rgb]{0.9,0.9,0.9}{%
	\makecell[{{p{\linewidth}}}]{% 
	  \tt {\tiny [A$\langle$GM]}  
	 Strafpunkt: Deine Nachricht darf nichts vor dem Befehl enthalten! Stelle sicher, dass deine Antwort ausschließlich einen der beiden Befehle `SPRICH: <NACHRICHT>` und `VERSCHIEBE: <OBJECT>, (<X>, <Y>)` und sonst nichts enthält!\\ \tt \\ \tt Ihr habt gemeinsam 1 von maximal 16 Strafpunkten angesammelt.\\ \tt Bitte versuche es noch mal! 
	  } 
	   } 
	   } 
	 & \\ 
 

    \theutterance \stepcounter{utterance}  

    & \multicolumn{4}{p{0.6\linewidth}}{\cellcolor[rgb]{0.8,1,0.9}{%
	\makecell[{{p{\linewidth}}}]{% 
	  \tt {\tiny [A$\rangle$GM]}  
	 SPRICH: Hallo, ich bin bereit, loszulegen. Lass uns zunächst unsere Strategie besprechen. Ich schlage vor, dass wir uns auf eine feste Position für jedes Objekt einigen und dann unsere Züge darauf abstimmen. Wie sieht es bei dir aus? Hast du bereits eine Idee, wie wir vorgehen können? 
	  } 
	   } 
	   } 
	 & & \\ 
 

    \theutterance \stepcounter{utterance}  

    & & & \multicolumn{2}{p{0.3\linewidth}}{\cellcolor[rgb]{0.95,0.95,0.95}{%
	\makecell[{{p{\linewidth}}}]{% 
	  \tt {\tiny [GM$|$GM]}  
	 SPRICH: Hallo, ich bin bereit, loszulegen. Lass uns zunächst unsere Strategie besprechen. Ich schlage vor, dass wir uns auf eine feste Position für jedes Objekt einigen und dann unsere Züge darauf abstimmen. Wie sieht es bei dir aus? Hast du bereits eine Idee, wie wir vorgehen können? 
	  } 
	   } 
	   } 
	 & & \\ 
 

    \theutterance \stepcounter{utterance}  

    & & \multicolumn{4}{p{0.6\linewidth}}{\cellcolor[rgb]{0.9,0.9,0.9}{%
	\makecell[{{p{\linewidth}}}]{% 
	  \tt {\tiny [GM$\rangle$B]}  
	 Ich bin der Spielleiter, und du spielst ein kooperatives Spiel mit dem folgenden Raster als Spielbrett:\\ \tt \\ \tt ```\\ \tt  1234567\\ \tt ╔═╤═╤═╤═╗\\ \tt ║A│◌│◌│M║ 1\\ \tt ╟─┼─┤O│T║ 2\\ \tt ║◌│D│◌│◌║ 3\\ \tt ║◌│◌├─┴─╢ 4\\ \tt ╟─┴─┤◌◌◌║ 5\\ \tt ║PS◌└─┬─╢ 6\\ \tt ║◌◌◌◌◌│◌║ 7\\ \tt ╚═════╧═╝\\ \tt \\ \tt ```\\ \tt \\ \tt * Am oberen Rand sind die x-Koordinaten angegeben, die nach rechts hin zunehmen, und am rechten Rand die y-Koordinaten, die nach unten zunehmen.\\ \tt * Die folgenden Objekte sind zufällig auf deinem Raster platziert: 'P', 'O', 'T', 'S', 'D', 'A', 'M'.\\ \tt \\ \tt \\ \tt Dein Mitspieler sieht eine Variation des Spielbretts, auf dem dieselben Objekte an anderen zufälligen Positionen platziert sind. Du kannst das Spielbrett des anderen Spielers nicht sehen, und er kann deins nicht sehen.\\ \tt \\ \tt \\ \tt **Ziel des Spiels:**\\ \tt \\ \tt Beide Spieler müssen die Objekte auf ihrem jeweiligen Spielbrett so bewegen, dass identische Objekte am Ende an denselben Koordinaten liegen. Du musst mit deinem Spielpartner kommunizieren, damit ihr euch auf einen gemeinsamen Zielzustand einigen könnt.\\ \tt \\ \tt \\ \tt **Spielregeln:**\\ \tt \\ \tt * In jedem Spielzug kannst du genau eine der folgenden zwei Aktionen ausführen:\\ \tt 1. `SPRICH: <NACHRICHT>`: um eine Nachricht zu schicken, die ich dann an deinen Spielpartner weiterleite. Als Nachricht gilt alles bis zum nächsten Zeilenumbruch.\\ \tt 2. `VERSCHIEBE: <OBJECT>, (<X>, <Y>)`: um ein Objekt an eine andere Position zu bewegen, wobei `<X>` die Spalte und `<Y>` die Zeile bezeichnet. Ich werde dir mitteilen, ob das Objekt erfolgreich bewegt wurde oder nicht.\\ \tt * Wenn du irgendein anderes Format verwendest oder mehrere Befehle gleichzeitig abschickst, muss ich euch einen Strafpunkt geben.\\ \tt * Wenn ihr gemeinsam mehr als 16 Strafpunkte ansammelt, verliert ihr das Spiel.\\ \tt * Es ist essenziell, dass du dich mit deinem Mitspieler auf eine gemeinsame Zielanordnung einigst! Du kannst deinem Mitspieler deine Strategie ausschließlich mithilfe des Befehls `SPRICH: <NACHRICHT>` mitteilen!\\ \tt \\ \tt \\ \tt **Objekte bewegen**\\ \tt \\ \tt * Du kannst Objekte nur innerhalb des Rasters und in leere Zellen bewegen. Leere Zellen erkennst am Zeichen '◌'.\\ \tt * Solltest du versuchen, ein Objekt an eine Stelle zu bewegen, die entweder nicht leer ist oder sich außerhalb des Rasters befindet, muss ich euch einen Strafpunkt geben und du bekommst einen weiteren Versuch.\\ \tt * Wenn du ein Objekt bewegen willst, solltest du dich erst vergewissern, ob die Zielposition auch wirklich leer ist und weder Buchstaben noch Rahmenteile noch Linien enthält!\\ \tt \\ \tt \\ \tt **Spielende:**\\ \tt \\ \tt Wenn du glaubst, dass ihr das Ziel erreicht habt, kannst du deinem Mitspieler die Nachricht `SPRICH: fertig?` schicken. Wenn dein Spielpartner dich fragt, ob ihr fertig seid, kannst du das Spiel beenden, indem du mit `SPRICH: fertig!` antwortest.\\ \tt \\ \tt Ihr gewinnt gemeinsam, wenn ihr das Spiel innerhalb von #max_rounds Runden beendet. Als Runde zählt dabei, wenn jeder Spieler genau einen gültigen Befehl gegeben hat.\\ \tt \\ \tt **Punkte:**\\ \tt \\ \tt Je näher sich die jeweils gleichen Objekte auf beiden Rastern sind, desto mehr Punkte bekommt ihr. Für Strafpunkte gibt es Punktabzug. Könnt ihr den Rekord brechen?\\ \tt \\ \tt Dein Mitspieler hat das Spiel mit dieser Nachricht begonnen:\\ \tt \\ \tt "Hallo, ich bin bereit, loszulegen. Lass uns zunächst unsere Strategie besprechen. Ich schlage vor, dass wir uns auf eine feste Position für jedes Objekt einigen und dann unsere Züge darauf abstimmen. Wie sieht es bei dir aus? Hast du bereits eine Idee, wie wir vorgehen können?"\\ \tt \\ \tt Wie lautet dein erster Befehl? 
	  } 
	   } 
	   } 
	 & \\ 
 

    \theutterance \stepcounter{utterance}  

    & & & \multicolumn{4}{p{0.6\linewidth}}{\cellcolor[rgb]{1,0.85,0.72}{%
	\makecell[{{p{\linewidth}}}]{% 
	  \tt {\tiny [GM$\langle$B]}  
	 SPRICH: Hallo! Ich bin auch bereit, loszulegen. Dein Vorschlag, eine feste Position für jedes Objekt zu bestimmen, klingt sinnvoll. Lass uns zunächst die Objekte auflisten und ihre derzeitigen Positionen auf unserem jeweiligen Spielbrett notieren. Auf meinem Spielbrett sehe ich folgende Objekte an den entsprechenden Positionen: A(1,1), O(2,2), T(3,2), D(2,3), P(1,6), S(2,6), M(4,1). Ich schlage vor, dass wir eine Prioritätenliste erstellen, um die Objekte in einer bestimmten Reihenfolge zu platzieren. Wie sieht es bei dir aus? Hast du die Positionen deiner Objekte notiert? 
	  } 
	   } 
	   } 
	  \\ 
 

    \theutterance \stepcounter{utterance}  

    & & & \multicolumn{2}{p{0.3\linewidth}}{\cellcolor[rgb]{0.95,0.95,0.95}{%
	\makecell[{{p{\linewidth}}}]{% 
	  \tt {\tiny [GM$|$GM]}  
	 SPRICH: Hallo! Ich bin auch bereit, loszulegen. Dein Vorschlag, eine feste Position für jedes Objekt zu bestimmen, klingt sinnvoll. Lass uns zunächst die Objekte auflisten und ihre derzeitigen Positionen auf unserem jeweiligen Spielbrett notieren. Auf meinem Spielbrett sehe ich folgende Objekte an den entsprechenden Positionen: A(1,1), O(2,2), T(3,2), D(2,3), P(1,6), S(2,6), M(4,1). Ich schlage vor, dass wir eine Prioritätenliste erstellen, um die Objekte in einer bestimmten Reihenfolge zu platzieren. Wie sieht es bei dir aus? Hast du die Positionen deiner Objekte notiert? 
	  } 
	   } 
	   } 
	 & & \\ 
 

    \theutterance \stepcounter{utterance}  

    & & \multicolumn{4}{p{0.6\linewidth}}{\cellcolor[rgb]{0.9,0.9,0.9}{%
	\makecell[{{p{\linewidth}}}]{% 
	  \tt {\tiny [A$\langle$GM]}  
	 Deine Nachricht wurde an deinen Mitspieler übermittelt.\\ \tt \\ \tt \\ \tt Ihr seid gerade bei Runde 1 von maximal 28.\\ \tt Ihr habt gemeinsam 1 von maximal 16 Strafpunkten angesammelt.\\ \tt Dein Mitspieler hat dir diese Nachricht geschickt:\\ \tt \\ \tt \\ \tt "Hallo! Ich bin auch bereit, loszulegen. Dein Vorschlag, eine feste Position für jedes Objekt zu bestimmen, klingt sinnvoll. Lass uns zunächst die Objekte auflisten und ihre derzeitigen Positionen auf unserem jeweiligen Spielbrett notieren. Auf meinem Spielbrett sehe ich folgende Objekte an den entsprechenden Positionen: A(1,1), O(2,2), T(3,2), D(2,3), P(1,6), S(2,6), M(4,1). Ich schlage vor, dass wir eine Prioritätenliste erstellen, um die Objekte in einer bestimmten Reihenfolge zu platzieren. Wie sieht es bei dir aus? Hast du die Positionen deiner Objekte notiert?"\\ \tt \\ \tt \\ \tt Wie lautet dein nächster Befehl? 
	  } 
	   } 
	   } 
	 & \\ 
 

    \theutterance \stepcounter{utterance}  

    & \multicolumn{4}{p{0.6\linewidth}}{\cellcolor[rgb]{0.8,1,0.9}{%
	\makecell[{{p{\linewidth}}}]{% 
	  \tt {\tiny [A$\rangle$GM]}  
	 SPRICH: Ich habe die Positionen meiner Objekte notiert. Auf meinem Spielbrett sehe ich folgende Objekte an den entsprechenden Positionen: A(1,1), O(1,3), T(2,3), P(3,3), D(1,6), S(7,6), M(7,2). Ich denke, wir sollten die Objekte in alphabetischer Reihenfolge platzieren, um eine einheitliche Strategie zu haben. Was meinst du dazu? Sollen wir mit der Platzierung beginnen? 
	  } 
	   } 
	   } 
	 & & \\ 
 

    \theutterance \stepcounter{utterance}  

    & & & \multicolumn{2}{p{0.3\linewidth}}{\cellcolor[rgb]{0.95,0.95,0.95}{%
	\makecell[{{p{\linewidth}}}]{% 
	  \tt {\tiny [GM$|$GM]}  
	 SPRICH: Ich habe die Positionen meiner Objekte notiert. Auf meinem Spielbrett sehe ich folgende Objekte an den entsprechenden Positionen: A(1,1), O(1,3), T(2,3), P(3,3), D(1,6), S(7,6), M(7,2). Ich denke, wir sollten die Objekte in alphabetischer Reihenfolge platzieren, um eine einheitliche Strategie zu haben. Was meinst du dazu? Sollen wir mit der Platzierung beginnen? 
	  } 
	   } 
	   } 
	 & & \\ 
 

    \theutterance \stepcounter{utterance}  

    & & \multicolumn{4}{p{0.6\linewidth}}{\cellcolor[rgb]{0.9,0.9,0.9}{%
	\makecell[{{p{\linewidth}}}]{% 
	  \tt {\tiny [GM$\rangle$B]}  
	 Deine Nachricht wurde an deinen Mitspieler übermittelt.\\ \tt \\ \tt \\ \tt Ihr seid gerade bei Runde 2 von maximal 28.\\ \tt Ihr habt gemeinsam 1 von maximal 16 Strafpunkten angesammelt.\\ \tt Dein Mitspieler hat dir diese Nachricht geschickt:\\ \tt \\ \tt \\ \tt "Ich habe die Positionen meiner Objekte notiert. Auf meinem Spielbrett sehe ich folgende Objekte an den entsprechenden Positionen: A(1,1), O(1,3), T(2,3), P(3,3), D(1,6), S(7,6), M(7,2). Ich denke, wir sollten die Objekte in alphabetischer Reihenfolge platzieren, um eine einheitliche Strategie zu haben. Was meinst du dazu? Sollen wir mit der Platzierung beginnen?"\\ \tt \\ \tt \\ \tt Wie lautet dein nächster Befehl? 
	  } 
	   } 
	   } 
	 & \\ 
 

    \theutterance \stepcounter{utterance}  

    & & & \multicolumn{4}{p{0.6\linewidth}}{\cellcolor[rgb]{1,0.85,0.72}{%
	\makecell[{{p{\linewidth}}}]{% 
	  \tt {\tiny [GM$\langle$B]}  
	 SPRICH: Ich verstehe deine Strategie und denke, dass die alphabetische Reihenfolge eine gute Idee ist. Allerdings sollten wir uns zunächst auf eine feste Position für jedes Objekt einigen, bevor wir mit der Platzierung beginnen. Wie wäre es, wenn wir die Objekte in alphabetischer Reihenfolge von links nach rechts und von oben nach unten platzieren? Zum Beispiel könnte A in der ersten Spalte, O in der zweiten Spalte, P in der dritten Spalte, S in der vierten Spalte, T in der fünften Spalte, D in der sechsten Spalte und M in der siebten Spalte platziert werden. Allerdings müssen wir noch die y-Koordinaten bestimmen. Was meinst du dazu? Sollten wir die y-Koordinaten auch in einer bestimmten Reihenfolge platzieren? 
	  } 
	   } 
	   } 
	  \\ 
 

    \theutterance \stepcounter{utterance}  

    & & & \multicolumn{2}{p{0.3\linewidth}}{\cellcolor[rgb]{0.95,0.95,0.95}{%
	\makecell[{{p{\linewidth}}}]{% 
	  \tt {\tiny [GM$|$GM]}  
	 SPRICH: Ich verstehe deine Strategie und denke, dass die alphabetische Reihenfolge eine gute Idee ist. Allerdings sollten wir uns zunächst auf eine feste Position für jedes Objekt einigen, bevor wir mit der Platzierung beginnen. Wie wäre es, wenn wir die Objekte in alphabetischer Reihenfolge von links nach rechts und von oben nach unten platzieren? Zum Beispiel könnte A in der ersten Spalte, O in der zweiten Spalte, P in der dritten Spalte, S in der vierten Spalte, T in der fünften Spalte, D in der sechsten Spalte und M in der siebten Spalte platziert werden. Allerdings müssen wir noch die y-Koordinaten bestimmen. Was meinst du dazu? Sollten wir die y-Koordinaten auch in einer bestimmten Reihenfolge platzieren? 
	  } 
	   } 
	   } 
	 & & \\ 
 

    \theutterance \stepcounter{utterance}  

    & & \multicolumn{4}{p{0.6\linewidth}}{\cellcolor[rgb]{0.9,0.9,0.9}{%
	\makecell[{{p{\linewidth}}}]{% 
	  \tt {\tiny [A$\langle$GM]}  
	 Deine Nachricht wurde an deinen Mitspieler übermittelt.\\ \tt \\ \tt \\ \tt Ihr seid gerade bei Runde 2 von maximal 28.\\ \tt Ihr habt gemeinsam 1 von maximal 16 Strafpunkten angesammelt.\\ \tt Dein Mitspieler hat dir diese Nachricht geschickt:\\ \tt \\ \tt \\ \tt "Ich verstehe deine Strategie und denke, dass die alphabetische Reihenfolge eine gute Idee ist. Allerdings sollten wir uns zunächst auf eine feste Position für jedes Objekt einigen, bevor wir mit der Platzierung beginnen. Wie wäre es, wenn wir die Objekte in alphabetischer Reihenfolge von links nach rechts und von oben nach unten platzieren? Zum Beispiel könnte A in der ersten Spalte, O in der zweiten Spalte, P in der dritten Spalte, S in der vierten Spalte, T in der fünften Spalte, D in der sechsten Spalte und M in der siebten Spalte platziert werden. Allerdings müssen wir noch die y-Koordinaten bestimmen. Was meinst du dazu? Sollten wir die y-Koordinaten auch in einer bestimmten Reihenfolge platzieren?"\\ \tt \\ \tt \\ \tt Wie lautet dein nächster Befehl? 
	  } 
	   } 
	   } 
	 & \\ 
 

    \theutterance \stepcounter{utterance}  

    & \multicolumn{4}{p{0.6\linewidth}}{\cellcolor[rgb]{0.8,1,0.9}{%
	\makecell[{{p{\linewidth}}}]{% 
	  \tt {\tiny [A$\rangle$GM]}  
	 SPRICH: Ich denke, deine Idee, die Objekte in alphabetischer Reihenfolge von links nach rechts zu platzieren, ist eine gute Grundlage. Für die y-Koordinaten schlage ich vor, dass wir die Objekte von oben nach unten in einer festen Reihenfolge platzieren, zum Beispiel in der Reihenfolge, in der sie im Alphabet vorkommen. Wenn wir also A, O, P, S, T, D, M in dieser Reihenfolge platzieren, könnten wir sie in den ersten, zweiten, dritten usw. Zeilen platzieren. Was meinst du dazu? Sollten wir uns auf eine feste Anzahl von Zeilen einigen, in denen wir die Objekte platzieren? 
	  } 
	   } 
	   } 
	 & & \\ 
 

    \theutterance \stepcounter{utterance}  

    & & & \multicolumn{2}{p{0.3\linewidth}}{\cellcolor[rgb]{0.95,0.95,0.95}{%
	\makecell[{{p{\linewidth}}}]{% 
	  \tt {\tiny [GM$|$GM]}  
	 SPRICH: Ich denke, deine Idee, die Objekte in alphabetischer Reihenfolge von links nach rechts zu platzieren, ist eine gute Grundlage. Für die y-Koordinaten schlage ich vor, dass wir die Objekte von oben nach unten in einer festen Reihenfolge platzieren, zum Beispiel in der Reihenfolge, in der sie im Alphabet vorkommen. Wenn wir also A, O, P, S, T, D, M in dieser Reihenfolge platzieren, könnten wir sie in den ersten, zweiten, dritten usw. Zeilen platzieren. Was meinst du dazu? Sollten wir uns auf eine feste Anzahl von Zeilen einigen, in denen wir die Objekte platzieren? 
	  } 
	   } 
	   } 
	 & & \\ 
 

    \theutterance \stepcounter{utterance}  

    & & \multicolumn{4}{p{0.6\linewidth}}{\cellcolor[rgb]{0.9,0.9,0.9}{%
	\makecell[{{p{\linewidth}}}]{% 
	  \tt {\tiny [GM$\rangle$B]}  
	 Deine Nachricht wurde an deinen Mitspieler übermittelt.\\ \tt \\ \tt \\ \tt Ihr seid gerade bei Runde 3 von maximal 28.\\ \tt Ihr habt gemeinsam 1 von maximal 16 Strafpunkten angesammelt.\\ \tt Dein Mitspieler hat dir diese Nachricht geschickt:\\ \tt \\ \tt \\ \tt "Ich denke, deine Idee, die Objekte in alphabetischer Reihenfolge von links nach rechts zu platzieren, ist eine gute Grundlage. Für die y-Koordinaten schlage ich vor, dass wir die Objekte von oben nach unten in einer festen Reihenfolge platzieren, zum Beispiel in der Reihenfolge, in der sie im Alphabet vorkommen. Wenn wir also A, O, P, S, T, D, M in dieser Reihenfolge platzieren, könnten wir sie in den ersten, zweiten, dritten usw. Zeilen platzieren. Was meinst du dazu? Sollten wir uns auf eine feste Anzahl von Zeilen einigen, in denen wir die Objekte platzieren?"\\ \tt \\ \tt \\ \tt Wie lautet dein nächster Befehl? 
	  } 
	   } 
	   } 
	 & \\ 
 

    \theutterance \stepcounter{utterance}  

    & & & \multicolumn{4}{p{0.6\linewidth}}{\cellcolor[rgb]{1,0.85,0.72}{%
	\makecell[{{p{\linewidth}}}]{% 
	  \tt {\tiny [GM$\langle$B]}  
	 SPRICH: Ich denke, deine Idee, die Objekte in der y-Richtung in der Reihenfolge ihres Alphabets zu platzieren, ist eine gute Ergänzung zu unserer Strategie. Eine feste Anzahl von Zeilen wäre auch sinnvoll, um unsere Platzierung zu beschränken und zu vereinfachen. Wie wäre es, wenn wir uns auf 3 Zeilen einigen? Wir könnten die Objekte in drei Zeilen platzieren, wobei jeder Buchstabe in einer bestimmten Spalte und Zeile landet. Zum Beispiel könnte die erste Zeile A, D, M enthalten, die zweite Zeile O, P, S und die dritte Zeile T. Allerdings müssen wir noch die genauen Positionen für jeden Buchstaben bestimmen. Sollten wir uns auf eine festgelegte Spalten-Zuweisung für jedes Objekt einigen, wie zum Beispiel A=1, D=2, M=3 usw.? 
	  } 
	   } 
	   } 
	  \\ 
 

    \theutterance \stepcounter{utterance}  

    & & & \multicolumn{2}{p{0.3\linewidth}}{\cellcolor[rgb]{0.95,0.95,0.95}{%
	\makecell[{{p{\linewidth}}}]{% 
	  \tt {\tiny [GM$|$GM]}  
	 SPRICH: Ich denke, deine Idee, die Objekte in der y-Richtung in der Reihenfolge ihres Alphabets zu platzieren, ist eine gute Ergänzung zu unserer Strategie. Eine feste Anzahl von Zeilen wäre auch sinnvoll, um unsere Platzierung zu beschränken und zu vereinfachen. Wie wäre es, wenn wir uns auf 3 Zeilen einigen? Wir könnten die Objekte in drei Zeilen platzieren, wobei jeder Buchstabe in einer bestimmten Spalte und Zeile landet. Zum Beispiel könnte die erste Zeile A, D, M enthalten, die zweite Zeile O, P, S und die dritte Zeile T. Allerdings müssen wir noch die genauen Positionen für jeden Buchstaben bestimmen. Sollten wir uns auf eine festgelegte Spalten-Zuweisung für jedes Objekt einigen, wie zum Beispiel A=1, D=2, M=3 usw.? 
	  } 
	   } 
	   } 
	 & & \\ 
 

    \theutterance \stepcounter{utterance}  

    & & \multicolumn{4}{p{0.6\linewidth}}{\cellcolor[rgb]{0.9,0.9,0.9}{%
	\makecell[{{p{\linewidth}}}]{% 
	  \tt {\tiny [A$\langle$GM]}  
	 Deine Nachricht wurde an deinen Mitspieler übermittelt.\\ \tt \\ \tt \\ \tt Ihr seid gerade bei Runde 3 von maximal 28.\\ \tt Ihr habt gemeinsam 1 von maximal 16 Strafpunkten angesammelt.\\ \tt Dein Mitspieler hat dir diese Nachricht geschickt:\\ \tt \\ \tt \\ \tt "Ich denke, deine Idee, die Objekte in der y-Richtung in der Reihenfolge ihres Alphabets zu platzieren, ist eine gute Ergänzung zu unserer Strategie. Eine feste Anzahl von Zeilen wäre auch sinnvoll, um unsere Platzierung zu beschränken und zu vereinfachen. Wie wäre es, wenn wir uns auf 3 Zeilen einigen? Wir könnten die Objekte in drei Zeilen platzieren, wobei jeder Buchstabe in einer bestimmten Spalte und Zeile landet. Zum Beispiel könnte die erste Zeile A, D, M enthalten, die zweite Zeile O, P, S und die dritte Zeile T. Allerdings müssen wir noch die genauen Positionen für jeden Buchstaben bestimmen. Sollten wir uns auf eine festgelegte Spalten-Zuweisung für jedes Objekt einigen, wie zum Beispiel A=1, D=2, M=3 usw.?"\\ \tt \\ \tt \\ \tt Wie lautet dein nächster Befehl? 
	  } 
	   } 
	   } 
	 & \\ 
 

    \theutterance \stepcounter{utterance}  

    & \multicolumn{4}{p{0.6\linewidth}}{\cellcolor[rgb]{0.8,1,0.9}{%
	\makecell[{{p{\linewidth}}}]{% 
	  \tt {\tiny [A$\rangle$GM]}  
	 SPRICH: Ich denke, deine Idee, die Objekte in 3 Zeilen zu platzieren, ist eine gute Vereinfachung. Die Verteilung der Objekte auf die Zeilen, wie du vorgeschlagen hast, sieht auch gut aus. Für die Spalten-Zuweisung schlage ich vor, dass wir eine einfache alphabetische Zuordnung verwenden, wie du vorgeschlagen hast: A=1, B=2, usw. Allerdings müssen wir beachten, dass wir nur 7 Objekte haben, also könnten wir die Zuordnung wie folgt erstellen: A=1, D=2, M=3, O=4, P=5, S=6, T=7. Wie sieht es bei dir aus? Sollten wir jetzt mit der Platzierung der Objekte beginnen? 
	  } 
	   } 
	   } 
	 & & \\ 
 

    \theutterance \stepcounter{utterance}  

    & & & \multicolumn{2}{p{0.3\linewidth}}{\cellcolor[rgb]{0.95,0.95,0.95}{%
	\makecell[{{p{\linewidth}}}]{% 
	  \tt {\tiny [GM$|$GM]}  
	 SPRICH: Ich denke, deine Idee, die Objekte in 3 Zeilen zu platzieren, ist eine gute Vereinfachung. Die Verteilung der Objekte auf die Zeilen, wie du vorgeschlagen hast, sieht auch gut aus. Für die Spalten-Zuweisung schlage ich vor, dass wir eine einfache alphabetische Zuordnung verwenden, wie du vorgeschlagen hast: A=1, B=2, usw. Allerdings müssen wir beachten, dass wir nur 7 Objekte haben, also könnten wir die Zuordnung wie folgt erstellen: A=1, D=2, M=3, O=4, P=5, S=6, T=7. Wie sieht es bei dir aus? Sollten wir jetzt mit der Platzierung der Objekte beginnen? 
	  } 
	   } 
	   } 
	 & & \\ 
 

    \theutterance \stepcounter{utterance}  

    & & \multicolumn{4}{p{0.6\linewidth}}{\cellcolor[rgb]{0.9,0.9,0.9}{%
	\makecell[{{p{\linewidth}}}]{% 
	  \tt {\tiny [GM$\rangle$B]}  
	 Deine Nachricht wurde an deinen Mitspieler übermittelt.\\ \tt \\ \tt \\ \tt Ihr seid gerade bei Runde 4 von maximal 28.\\ \tt Ihr habt gemeinsam 1 von maximal 16 Strafpunkten angesammelt.\\ \tt Dein Mitspieler hat dir diese Nachricht geschickt:\\ \tt \\ \tt \\ \tt "Ich denke, deine Idee, die Objekte in 3 Zeilen zu platzieren, ist eine gute Vereinfachung. Die Verteilung der Objekte auf die Zeilen, wie du vorgeschlagen hast, sieht auch gut aus. Für die Spalten-Zuweisung schlage ich vor, dass wir eine einfache alphabetische Zuordnung verwenden, wie du vorgeschlagen hast: A=1, B=2, usw. Allerdings müssen wir beachten, dass wir nur 7 Objekte haben, also könnten wir die Zuordnung wie folgt erstellen: A=1, D=2, M=3, O=4, P=5, S=6, T=7. Wie sieht es bei dir aus? Sollten wir jetzt mit der Platzierung der Objekte beginnen?"\\ \tt \\ \tt \\ \tt Wie lautet dein nächster Befehl? 
	  } 
	   } 
	   } 
	 & \\ 
 

    \theutterance \stepcounter{utterance}  

    & & & \multicolumn{4}{p{0.6\linewidth}}{\cellcolor[rgb]{1,0.85,0.72}{%
	\makecell[{{p{\linewidth}}}]{% 
	  \tt {\tiny [GM$\langle$B]}  
	 SPRICH: Ich denke, unsere Strategie ist jetzt gut ausgearbeitet. Die Spalten-Zuweisung, die du vorgeschlagen hast, sieht logisch aus und passt zu unserer alphabetischen Reihenfolge. Ich stimme zu, dass wir mit der Platzierung der Objekte beginnen sollten. Bevor wir jedoch beginnen, möchten wir uns vergewissern, dass wir beide die gleiche Vorstellung davon haben, wie die endgültige Anordnung aussehen sollte. Könntest du bitte die endgültige Position für jedes Objekt auflisten, basierend auf unserer Strategie? Zum Beispiel, wo würde A, D, M, O, P, S und T platziert werden, unter Berücksichtigung der Zeilen und Spalten, die wir besprochen haben? Damit können wir sicherstellen, dass wir auf dem gleichen Stand sind, bevor wir mit den Bewegungen beginnen. 
	  } 
	   } 
	   } 
	  \\ 
 

    \theutterance \stepcounter{utterance}  

    & & & \multicolumn{2}{p{0.3\linewidth}}{\cellcolor[rgb]{0.95,0.95,0.95}{%
	\makecell[{{p{\linewidth}}}]{% 
	  \tt {\tiny [GM$|$GM]}  
	 SPRICH: Ich denke, unsere Strategie ist jetzt gut ausgearbeitet. Die Spalten-Zuweisung, die du vorgeschlagen hast, sieht logisch aus und passt zu unserer alphabetischen Reihenfolge. Ich stimme zu, dass wir mit der Platzierung der Objekte beginnen sollten. Bevor wir jedoch beginnen, möchten wir uns vergewissern, dass wir beide die gleiche Vorstellung davon haben, wie die endgültige Anordnung aussehen sollte. Könntest du bitte die endgültige Position für jedes Objekt auflisten, basierend auf unserer Strategie? Zum Beispiel, wo würde A, D, M, O, P, S und T platziert werden, unter Berücksichtigung der Zeilen und Spalten, die wir besprochen haben? Damit können wir sicherstellen, dass wir auf dem gleichen Stand sind, bevor wir mit den Bewegungen beginnen. 
	  } 
	   } 
	   } 
	 & & \\ 
 

    \theutterance \stepcounter{utterance}  

    & & \multicolumn{4}{p{0.6\linewidth}}{\cellcolor[rgb]{0.9,0.9,0.9}{%
	\makecell[{{p{\linewidth}}}]{% 
	  \tt {\tiny [A$\langle$GM]}  
	 Deine Nachricht wurde an deinen Mitspieler übermittelt.\\ \tt \\ \tt \\ \tt Ihr seid gerade bei Runde 4 von maximal 28.\\ \tt Ihr habt gemeinsam 1 von maximal 16 Strafpunkten angesammelt.\\ \tt Dein Mitspieler hat dir diese Nachricht geschickt:\\ \tt \\ \tt \\ \tt "Ich denke, unsere Strategie ist jetzt gut ausgearbeitet. Die Spalten-Zuweisung, die du vorgeschlagen hast, sieht logisch aus und passt zu unserer alphabetischen Reihenfolge. Ich stimme zu, dass wir mit der Platzierung der Objekte beginnen sollten. Bevor wir jedoch beginnen, möchten wir uns vergewissern, dass wir beide die gleiche Vorstellung davon haben, wie die endgültige Anordnung aussehen sollte. Könntest du bitte die endgültige Position für jedes Objekt auflisten, basierend auf unserer Strategie? Zum Beispiel, wo würde A, D, M, O, P, S und T platziert werden, unter Berücksichtigung der Zeilen und Spalten, die wir besprochen haben? Damit können wir sicherstellen, dass wir auf dem gleichen Stand sind, bevor wir mit den Bewegungen beginnen."\\ \tt \\ \tt \\ \tt Wie lautet dein nächster Befehl? 
	  } 
	   } 
	   } 
	 & \\ 
 

    \theutterance \stepcounter{utterance}  

    & \multicolumn{4}{p{0.6\linewidth}}{\cellcolor[rgb]{0.8,1,0.9}{%
	\makecell[{{p{\linewidth}}}]{% 
	  \tt {\tiny [A$\rangle$GM]}  
	 SPRICH: Basierend auf unserer Strategie, würde ich die endgültige Anordnung wie folgt vorstellen: \\ \tt A (1,1), D (2,1), M (3,1) in der ersten Zeile, \\ \tt O (4,2), P (5,2), S (6,2) in der zweiten Zeile, \\ \tt T (7,3) in der dritten Zeile. \\ \tt Das würde bedeuten, dass jedes Objekt in seiner entsprechenden Spalte und Zeile platziert wird, wie wir besprochen haben. Bitte überprüfe, ob das mit deiner Vorstellung übereinstimmt, bevor wir mit den Bewegungen beginnen. 
	  } 
	   } 
	   } 
	 & & \\ 
 

    \theutterance \stepcounter{utterance}  

    & & & \multicolumn{2}{p{0.3\linewidth}}{\cellcolor[rgb]{0.95,0.95,0.95}{%
	\makecell[{{p{\linewidth}}}]{% 
	  \tt {\tiny [GM$|$GM]}  
	 SPRICH: Basierend auf unserer Strategie, würde ich die endgültige Anordnung wie folgt vorstellen: \\ \tt A (1,1), D (2,1), M (3,1) in der ersten Zeile, \\ \tt O (4,2), P (5,2), S (6,2) in der zweiten Zeile, \\ \tt T (7,3) in der dritten Zeile. \\ \tt Das würde bedeuten, dass jedes Objekt in seiner entsprechenden Spalte und Zeile platziert wird, wie wir besprochen haben. Bitte überprüfe, ob das mit deiner Vorstellung übereinstimmt, bevor wir mit den Bewegungen beginnen. 
	  } 
	   } 
	   } 
	 & & \\ 
 

    \theutterance \stepcounter{utterance}  

    & & & \multicolumn{2}{p{0.3\linewidth}}{\cellcolor[rgb]{0.95,0.95,0.95}{%
	\makecell[{{p{\linewidth}}}]{% 
	  \tt {\tiny [GM$|$GM]}  
	 Invalid format: tail is not empty: '\\ \tt A (1,1), D (2,1), M (3,1) in der ersten Zeile, \\ \tt O (4,2), P (5,2), S (6,2) in der zweiten Zeile, \\ \tt T (7,3) in der dritten Zeile. \\ \tt Das würde bedeuten, dass jedes Objekt in seiner entsprechenden Spalte und Zeile platziert wird, wie wir besprochen haben. Bitte überprüfe, ob das mit deiner Vorstellung übereinstimmt, bevor wir mit den Bewegungen beginnen. 
	  } 
	   } 
	   } 
	 & & \\ 
 

    \theutterance \stepcounter{utterance}  

    & & \multicolumn{4}{p{0.6\linewidth}}{\cellcolor[rgb]{0.9,0.9,0.9}{%
	\makecell[{{p{\linewidth}}}]{% 
	  \tt {\tiny [A$\langle$GM]}  
	 Strafpunkt: Deine Nachricht darf nichts nach dem Befehl enthalten! Stelle sicher, dass deine Antwort ausschließlich einen der beiden Befehle `SPRICH: <NACHRICHT>` und `VERSCHIEBE: <OBJECT>, (<X>, <Y>)` und sonst nichts enthält!\\ \tt \\ \tt Ihr habt gemeinsam 2 von maximal 16 Strafpunkten angesammelt.\\ \tt Bitte versuche es noch mal! 
	  } 
	   } 
	   } 
	 & \\ 
 

    \theutterance \stepcounter{utterance}  

    & \multicolumn{4}{p{0.6\linewidth}}{\cellcolor[rgb]{0.8,1,0.9}{%
	\makecell[{{p{\linewidth}}}]{% 
	  \tt {\tiny [A$\rangle$GM]}  
	 SPRICH: Basierend auf unserer Strategie, würde ich die endgültige Anordnung wie folgt vorstellen: A (1,1), D (2,1), M (3,1) in der ersten Zeile, O (4,2), P (5,2), S (6,2) in der zweiten Zeile, T (7,3) in der dritten Zeile. 
	  } 
	   } 
	   } 
	 & & \\ 
 

    \theutterance \stepcounter{utterance}  

    & & & \multicolumn{2}{p{0.3\linewidth}}{\cellcolor[rgb]{0.95,0.95,0.95}{%
	\makecell[{{p{\linewidth}}}]{% 
	  \tt {\tiny [GM$|$GM]}  
	 SPRICH: Basierend auf unserer Strategie, würde ich die endgültige Anordnung wie folgt vorstellen: A (1,1), D (2,1), M (3,1) in der ersten Zeile, O (4,2), P (5,2), S (6,2) in der zweiten Zeile, T (7,3) in der dritten Zeile. 
	  } 
	   } 
	   } 
	 & & \\ 
 

    \theutterance \stepcounter{utterance}  

    & & \multicolumn{4}{p{0.6\linewidth}}{\cellcolor[rgb]{0.9,0.9,0.9}{%
	\makecell[{{p{\linewidth}}}]{% 
	  \tt {\tiny [GM$\rangle$B]}  
	 Deine Nachricht wurde an deinen Mitspieler übermittelt.\\ \tt \\ \tt \\ \tt Ihr seid gerade bei Runde 5 von maximal 28.\\ \tt Ihr habt gemeinsam 2 von maximal 16 Strafpunkten angesammelt.\\ \tt Dein Mitspieler hat dir diese Nachricht geschickt:\\ \tt \\ \tt \\ \tt "Basierend auf unserer Strategie, würde ich die endgültige Anordnung wie folgt vorstellen: A (1,1), D (2,1), M (3,1) in der ersten Zeile, O (4,2), P (5,2), S (6,2) in der zweiten Zeile, T (7,3) in der dritten Zeile."\\ \tt \\ \tt \\ \tt Wie lautet dein nächster Befehl? 
	  } 
	   } 
	   } 
	 & \\ 
 

    \theutterance \stepcounter{utterance}  

    & & & \multicolumn{4}{p{0.6\linewidth}}{\cellcolor[rgb]{1,0.85,0.72}{%
	\makecell[{{p{\linewidth}}}]{% 
	  \tt {\tiny [GM$\langle$B]}  
	 SPRICH: Ich denke, es gibt einen kleinen Fehler in deiner Anordnung. Basierend auf unserer alphabetischen Reihenfolge und Spalten-Zuweisung (A=1, D=2, M=3, O=4, P=5, S=6, T=7) und unter Berücksichtigung der Zeilen, die wir besprochen haben, sollte die Anordnung wie folgt aussehen: A (1,1), D (2,1), M (3,1) in der ersten Zeile, O (4,2), P (5,2), S (6,2) in der zweiten Zeile, und T müsste dann in der dritten Zeile bei der 7. Spalte platziert werden, also T (7,3). Allerdings fehlt uns noch die Position für die restlichen Objekte, da wir nur 7 Objekte haben und 3 Zeilen mit jeweils 3 oder 4 Positionen besetzen. Ich denke, wir sollten uns auf eine Zeilenanordnung einigen, die alle Objekte berücksichtigt und sicherstellt, dass wir keine Positionen doppelt belegen. Lass uns die Anordnung noch einmal überprüfen und sicherstellen, dass wir alle Objekte in unserer Strategie berücksichtigt haben. Wie sieht es bei dir aus? Sollten wir unsere Strategie anpassen, um alle Objekte in den 3 Zeilen unterzubringen? 
	  } 
	   } 
	   } 
	  \\ 
 

    \theutterance \stepcounter{utterance}  

    & & & \multicolumn{2}{p{0.3\linewidth}}{\cellcolor[rgb]{0.95,0.95,0.95}{%
	\makecell[{{p{\linewidth}}}]{% 
	  \tt {\tiny [GM$|$GM]}  
	 SPRICH: Ich denke, es gibt einen kleinen Fehler in deiner Anordnung. Basierend auf unserer alphabetischen Reihenfolge und Spalten-Zuweisung (A=1, D=2, M=3, O=4, P=5, S=6, T=7) und unter Berücksichtigung der Zeilen, die wir besprochen haben, sollte die Anordnung wie folgt aussehen: A (1,1), D (2,1), M (3,1) in der ersten Zeile, O (4,2), P (5,2), S (6,2) in der zweiten Zeile, und T müsste dann in der dritten Zeile bei der 7. Spalte platziert werden, also T (7,3). Allerdings fehlt uns noch die Position für die restlichen Objekte, da wir nur 7 Objekte haben und 3 Zeilen mit jeweils 3 oder 4 Positionen besetzen. Ich denke, wir sollten uns auf eine Zeilenanordnung einigen, die alle Objekte berücksichtigt und sicherstellt, dass wir keine Positionen doppelt belegen. Lass uns die Anordnung noch einmal überprüfen und sicherstellen, dass wir alle Objekte in unserer Strategie berücksichtigt haben. Wie sieht es bei dir aus? Sollten wir unsere Strategie anpassen, um alle Objekte in den 3 Zeilen unterzubringen? 
	  } 
	   } 
	   } 
	 & & \\ 
 

    \theutterance \stepcounter{utterance}  

    & & \multicolumn{4}{p{0.6\linewidth}}{\cellcolor[rgb]{0.9,0.9,0.9}{%
	\makecell[{{p{\linewidth}}}]{% 
	  \tt {\tiny [A$\langle$GM]}  
	 Deine Nachricht wurde an deinen Mitspieler übermittelt.\\ \tt \\ \tt \\ \tt Ihr seid gerade bei Runde 5 von maximal 28.\\ \tt Ihr habt gemeinsam 2 von maximal 16 Strafpunkten angesammelt.\\ \tt Dein Mitspieler hat dir diese Nachricht geschickt:\\ \tt \\ \tt \\ \tt "Ich denke, es gibt einen kleinen Fehler in deiner Anordnung. Basierend auf unserer alphabetischen Reihenfolge und Spalten-Zuweisung (A=1, D=2, M=3, O=4, P=5, S=6, T=7) und unter Berücksichtigung der Zeilen, die wir besprochen haben, sollte die Anordnung wie folgt aussehen: A (1,1), D (2,1), M (3,1) in der ersten Zeile, O (4,2), P (5,2), S (6,2) in der zweiten Zeile, und T müsste dann in der dritten Zeile bei der 7. Spalte platziert werden, also T (7,3). Allerdings fehlt uns noch die Position für die restlichen Objekte, da wir nur 7 Objekte haben und 3 Zeilen mit jeweils 3 oder 4 Positionen besetzen. Ich denke, wir sollten uns auf eine Zeilenanordnung einigen, die alle Objekte berücksichtigt und sicherstellt, dass wir keine Positionen doppelt belegen. Lass uns die Anordnung noch einmal überprüfen und sicherstellen, dass wir alle Objekte in unserer Strategie berücksichtigt haben. Wie sieht es bei dir aus? Sollten wir unsere Strategie anpassen, um alle Objekte in den 3 Zeilen unterzubringen?"\\ \tt \\ \tt \\ \tt Wie lautet dein nächster Befehl? 
	  } 
	   } 
	   } 
	 & \\ 
 

    \theutterance \stepcounter{utterance}  

    & \multicolumn{4}{p{0.6\linewidth}}{\cellcolor[rgb]{0.8,1,0.9}{%
	\makecell[{{p{\linewidth}}}]{% 
	  \tt {\tiny [A$\rangle$GM]}  
	 SPRICH: Ich denke, wir sollten unsere Strategie anpassen, um alle Objekte in den 3 Zeilen unterzubringen. Wir könnten die erste Zeile mit A, D, M füllen, die zweite Zeile mit O, P, S und die dritte Zeile mit T und den beiden fehlenden Positionen, die wir bisher nicht berücksichtigt haben, aber diese fehlenden Positionen gibt es ja nicht, da wir nur 7 Objekte haben. Ich denke, ich habe mich geirrt, als ich dachte, wir hätten noch Platz für weitere Objekte. Lass uns einfach mit der Anordnung fortfahren, die wir bisher besprochen haben: A (1,1), D (2,1), M (3,1) in der ersten Zeile, O (4,2), P (5,2), S (6,2) in der zweiten Zeile, und T (7,3) in der dritten Zeile. Ich denke, das sollte funktionieren. 
	  } 
	   } 
	   } 
	 & & \\ 
 

    \theutterance \stepcounter{utterance}  

    & & & \multicolumn{2}{p{0.3\linewidth}}{\cellcolor[rgb]{0.95,0.95,0.95}{%
	\makecell[{{p{\linewidth}}}]{% 
	  \tt {\tiny [GM$|$GM]}  
	 SPRICH: Ich denke, wir sollten unsere Strategie anpassen, um alle Objekte in den 3 Zeilen unterzubringen. Wir könnten die erste Zeile mit A, D, M füllen, die zweite Zeile mit O, P, S und die dritte Zeile mit T und den beiden fehlenden Positionen, die wir bisher nicht berücksichtigt haben, aber diese fehlenden Positionen gibt es ja nicht, da wir nur 7 Objekte haben. Ich denke, ich habe mich geirrt, als ich dachte, wir hätten noch Platz für weitere Objekte. Lass uns einfach mit der Anordnung fortfahren, die wir bisher besprochen haben: A (1,1), D (2,1), M (3,1) in der ersten Zeile, O (4,2), P (5,2), S (6,2) in der zweiten Zeile, und T (7,3) in der dritten Zeile. Ich denke, das sollte funktionieren. 
	  } 
	   } 
	   } 
	 & & \\ 
 

    \theutterance \stepcounter{utterance}  

    & & \multicolumn{4}{p{0.6\linewidth}}{\cellcolor[rgb]{0.9,0.9,0.9}{%
	\makecell[{{p{\linewidth}}}]{% 
	  \tt {\tiny [GM$\rangle$B]}  
	 Deine Nachricht wurde an deinen Mitspieler übermittelt.\\ \tt \\ \tt \\ \tt Ihr seid gerade bei Runde 6 von maximal 28.\\ \tt Ihr habt gemeinsam 2 von maximal 16 Strafpunkten angesammelt.\\ \tt Dein Mitspieler hat dir diese Nachricht geschickt:\\ \tt \\ \tt \\ \tt "Ich denke, wir sollten unsere Strategie anpassen, um alle Objekte in den 3 Zeilen unterzubringen. Wir könnten die erste Zeile mit A, D, M füllen, die zweite Zeile mit O, P, S und die dritte Zeile mit T und den beiden fehlenden Positionen, die wir bisher nicht berücksichtigt haben, aber diese fehlenden Positionen gibt es ja nicht, da wir nur 7 Objekte haben. Ich denke, ich habe mich geirrt, als ich dachte, wir hätten noch Platz für weitere Objekte. Lass uns einfach mit der Anordnung fortfahren, die wir bisher besprochen haben: A (1,1), D (2,1), M (3,1) in der ersten Zeile, O (4,2), P (5,2), S (6,2) in der zweiten Zeile, und T (7,3) in der dritten Zeile. Ich denke, das sollte funktionieren."\\ \tt \\ \tt \\ \tt Wie lautet dein nächster Befehl? 
	  } 
	   } 
	   } 
	 & \\ 
 

    \theutterance \stepcounter{utterance}  

    & & & \multicolumn{4}{p{0.6\linewidth}}{\cellcolor[rgb]{1,0.85,0.72}{%
	\makecell[{{p{\linewidth}}}]{% 
	  \tt {\tiny [GM$\langle$B]}  
	 SPRICH: Ich denke, wir haben jetzt eine klare Vorstellung davon, wie die endgültige Anordnung aussehen sollte. Ich stimme zu, dass wir mit der Anordnung fortfahren sollten, die wir bisher besprochen haben: A (1,1), D (2,1), M (3,1) in der ersten Zeile, O (4,2), P (5,2), S (6,2) in der zweiten Zeile, und T (7,3) in der dritten Zeile. Bevor wir beginnen, die Objekte zu bewegen, lass uns noch einmal unsere aktuellen Spielbretter vergleichen, um sicherzustellen, dass wir alle Objekte berücksichtigt haben. Auf meinem Spielbrett sehe ich, dass ich einige Objekte bewegen muss, um sie in die richtigen Positionen zu bringen. Ich werde mich auf die Bewegungen vorbereiten. Wie sieht es bei dir aus? Bist du bereit, die Objekte auf deinem Spielbrett zu bewegen, um die gewünschte Anordnung zu erreichen? 
	  } 
	   } 
	   } 
	  \\ 
 

    \theutterance \stepcounter{utterance}  

    & & & \multicolumn{2}{p{0.3\linewidth}}{\cellcolor[rgb]{0.95,0.95,0.95}{%
	\makecell[{{p{\linewidth}}}]{% 
	  \tt {\tiny [GM$|$GM]}  
	 SPRICH: Ich denke, wir haben jetzt eine klare Vorstellung davon, wie die endgültige Anordnung aussehen sollte. Ich stimme zu, dass wir mit der Anordnung fortfahren sollten, die wir bisher besprochen haben: A (1,1), D (2,1), M (3,1) in der ersten Zeile, O (4,2), P (5,2), S (6,2) in der zweiten Zeile, und T (7,3) in der dritten Zeile. Bevor wir beginnen, die Objekte zu bewegen, lass uns noch einmal unsere aktuellen Spielbretter vergleichen, um sicherzustellen, dass wir alle Objekte berücksichtigt haben. Auf meinem Spielbrett sehe ich, dass ich einige Objekte bewegen muss, um sie in die richtigen Positionen zu bringen. Ich werde mich auf die Bewegungen vorbereiten. Wie sieht es bei dir aus? Bist du bereit, die Objekte auf deinem Spielbrett zu bewegen, um die gewünschte Anordnung zu erreichen? 
	  } 
	   } 
	   } 
	 & & \\ 
 

    \theutterance \stepcounter{utterance}  

    & & \multicolumn{4}{p{0.6\linewidth}}{\cellcolor[rgb]{0.9,0.9,0.9}{%
	\makecell[{{p{\linewidth}}}]{% 
	  \tt {\tiny [A$\langle$GM]}  
	 Deine Nachricht wurde an deinen Mitspieler übermittelt.\\ \tt \\ \tt \\ \tt Ihr seid gerade bei Runde 6 von maximal 28.\\ \tt Ihr habt gemeinsam 2 von maximal 16 Strafpunkten angesammelt.\\ \tt Dein Mitspieler hat dir diese Nachricht geschickt:\\ \tt \\ \tt \\ \tt "Ich denke, wir haben jetzt eine klare Vorstellung davon, wie die endgültige Anordnung aussehen sollte. Ich stimme zu, dass wir mit der Anordnung fortfahren sollten, die wir bisher besprochen haben: A (1,1), D (2,1), M (3,1) in der ersten Zeile, O (4,2), P (5,2), S (6,2) in der zweiten Zeile, und T (7,3) in der dritten Zeile. Bevor wir beginnen, die Objekte zu bewegen, lass uns noch einmal unsere aktuellen Spielbretter vergleichen, um sicherzustellen, dass wir alle Objekte berücksichtigt haben. Auf meinem Spielbrett sehe ich, dass ich einige Objekte bewegen muss, um sie in die richtigen Positionen zu bringen. Ich werde mich auf die Bewegungen vorbereiten. Wie sieht es bei dir aus? Bist du bereit, die Objekte auf deinem Spielbrett zu bewegen, um die gewünschte Anordnung zu erreichen?"\\ \tt \\ \tt \\ \tt Wie lautet dein nächster Befehl? 
	  } 
	   } 
	   } 
	 & \\ 
 

    \theutterance \stepcounter{utterance}  

    & \multicolumn{4}{p{0.6\linewidth}}{\cellcolor[rgb]{0.8,1,0.9}{%
	\makecell[{{p{\linewidth}}}]{% 
	  \tt {\tiny [A$\rangle$GM]}  
	 SPRICH: Ja, ich bin bereit, die Objekte auf meinem Spielbrett zu bewegen, um die gewünschte Anordnung zu erreichen. Ich werde mich auf die Bewegungen vorbereiten und darauf achten, dass ich die Objekte in die richtigen Positionen bringe. Lass uns beginnen! 
	  } 
	   } 
	   } 
	 & & \\ 
 

    \theutterance \stepcounter{utterance}  

    & & & \multicolumn{2}{p{0.3\linewidth}}{\cellcolor[rgb]{0.95,0.95,0.95}{%
	\makecell[{{p{\linewidth}}}]{% 
	  \tt {\tiny [GM$|$GM]}  
	 SPRICH: Ja, ich bin bereit, die Objekte auf meinem Spielbrett zu bewegen, um die gewünschte Anordnung zu erreichen. Ich werde mich auf die Bewegungen vorbereiten und darauf achten, dass ich die Objekte in die richtigen Positionen bringe. Lass uns beginnen! 
	  } 
	   } 
	   } 
	 & & \\ 
 

    \theutterance \stepcounter{utterance}  

    & & \multicolumn{4}{p{0.6\linewidth}}{\cellcolor[rgb]{0.9,0.9,0.9}{%
	\makecell[{{p{\linewidth}}}]{% 
	  \tt {\tiny [GM$\rangle$B]}  
	 Deine Nachricht wurde an deinen Mitspieler übermittelt.\\ \tt \\ \tt \\ \tt Ihr seid gerade bei Runde 7 von maximal 28.\\ \tt Ihr habt gemeinsam 2 von maximal 16 Strafpunkten angesammelt.\\ \tt Dein Mitspieler hat dir diese Nachricht geschickt:\\ \tt \\ \tt \\ \tt "Ja, ich bin bereit, die Objekte auf meinem Spielbrett zu bewegen, um die gewünschte Anordnung zu erreichen. Ich werde mich auf die Bewegungen vorbereiten und darauf achten, dass ich die Objekte in die richtigen Positionen bringe. Lass uns beginnen!"\\ \tt \\ \tt \\ \tt Wie lautet dein nächster Befehl? 
	  } 
	   } 
	   } 
	 & \\ 
 

    \theutterance \stepcounter{utterance}  

    & & & \multicolumn{4}{p{0.6\linewidth}}{\cellcolor[rgb]{1,0.85,0.72}{%
	\makecell[{{p{\linewidth}}}]{% 
	  \tt {\tiny [GM$\langle$B]}  
	 VERSCHIEBE: A, (1,1) \\ \tt \\ \tt Ich möchte das Objekt A an die Position (1,1) bewegen, um die Anordnung zu beginnen. 
	  } 
	   } 
	   } 
	  \\ 
 

    \theutterance \stepcounter{utterance}  

    & & & \multicolumn{2}{p{0.3\linewidth}}{\cellcolor[rgb]{0.95,0.95,0.95}{%
	\makecell[{{p{\linewidth}}}]{% 
	  \tt {\tiny [GM$|$GM]}  
	 VERSCHIEBE: A, (1,1) \\ \tt \\ \tt Ich möchte das Objekt A an die Position (1,1) bewegen, um die Anordnung zu beginnen. 
	  } 
	   } 
	   } 
	 & & \\ 
 

    \theutterance \stepcounter{utterance}  

    & & & \multicolumn{2}{p{0.3\linewidth}}{\cellcolor[rgb]{0.95,0.95,0.95}{%
	\makecell[{{p{\linewidth}}}]{% 
	  \tt {\tiny [GM$|$GM]}  
	 Invalid format: tail is not empty: ' \\ \tt \\ \tt Ich möchte das Objekt A an die Position (1,1) bewegen, um die Anordnung zu beginnen. 
	  } 
	   } 
	   } 
	 & & \\ 
 

    \theutterance \stepcounter{utterance}  

    & & \multicolumn{4}{p{0.6\linewidth}}{\cellcolor[rgb]{0.9,0.9,0.9}{%
	\makecell[{{p{\linewidth}}}]{% 
	  \tt {\tiny [GM$\rangle$B]}  
	 Strafpunkt: Deine Nachricht darf nichts nach dem Befehl enthalten! Stelle sicher, dass deine Antwort ausschließlich einen der beiden Befehle `SPRICH: <NACHRICHT>` und `VERSCHIEBE: <OBJECT>, (<X>, <Y>)` und sonst nichts enthält!\\ \tt \\ \tt Ihr habt gemeinsam 3 von maximal 16 Strafpunkten angesammelt.\\ \tt Bitte versuche es noch mal! 
	  } 
	   } 
	   } 
	 & \\ 
 

    \theutterance \stepcounter{utterance}  

    & & & \multicolumn{4}{p{0.6\linewidth}}{\cellcolor[rgb]{1,0.85,0.72}{%
	\makecell[{{p{\linewidth}}}]{% 
	  \tt {\tiny [GM$\langle$B]}  
	 VERSCHIEBE: A, (1,1) 
	  } 
	   } 
	   } 
	  \\ 
 

    \theutterance \stepcounter{utterance}  

    & & & \multicolumn{2}{p{0.3\linewidth}}{\cellcolor[rgb]{0.95,0.95,0.95}{%
	\makecell[{{p{\linewidth}}}]{% 
	  \tt {\tiny [GM$|$GM]}  
	 VERSCHIEBE: A, (1,1) 
	  } 
	   } 
	   } 
	 & & \\ 
 

    \theutterance \stepcounter{utterance}  

    & & & \multicolumn{2}{p{0.3\linewidth}}{\cellcolor[rgb]{0.95,0.95,0.95}{%
	\makecell[{{p{\linewidth}}}]{% 
	  \tt {\tiny [GM$|$GM]}  
	 Strafpunkt: (1,1) ist nicht leer, sondern enthält 'A'.\\ \tt \\ \tt Ihr habt gemeinsam 4 von maximal 16 Strafpunkten angesammelt.\\ \tt Bitte versuche es noch mal! 
	  } 
	   } 
	   } 
	 & & \\ 
 

    \theutterance \stepcounter{utterance}  

    & & \multicolumn{4}{p{0.6\linewidth}}{\cellcolor[rgb]{0.9,0.9,0.9}{%
	\makecell[{{p{\linewidth}}}]{% 
	  \tt {\tiny [GM$\rangle$B]}  
	 Strafpunkt: (1,1) ist nicht leer, sondern enthält 'A'.\\ \tt \\ \tt Ihr habt gemeinsam 4 von maximal 16 Strafpunkten angesammelt.\\ \tt Bitte versuche es noch mal! 
	  } 
	   } 
	   } 
	 & \\ 
 

    \theutterance \stepcounter{utterance}  

    & & & \multicolumn{4}{p{0.6\linewidth}}{\cellcolor[rgb]{1,0.85,0.72}{%
	\makecell[{{p{\linewidth}}}]{% 
	  \tt {\tiny [GM$\langle$B]}  
	 VERSCHIEBE: O, (4,2) 
	  } 
	   } 
	   } 
	  \\ 
 

    \theutterance \stepcounter{utterance}  

    & & & \multicolumn{2}{p{0.3\linewidth}}{\cellcolor[rgb]{0.95,0.95,0.95}{%
	\makecell[{{p{\linewidth}}}]{% 
	  \tt {\tiny [GM$|$GM]}  
	 VERSCHIEBE: O, (4,2) 
	  } 
	   } 
	   } 
	 & & \\ 
 

    \theutterance \stepcounter{utterance}  

    & & & \multicolumn{2}{p{0.3\linewidth}}{\cellcolor[rgb]{0.95,0.95,0.95}{%
	\makecell[{{p{\linewidth}}}]{% 
	  \tt {\tiny [GM$|$GM]}  
	 Strafpunkt: (4,2) ist nicht leer, sondern enthält '┤'.\\ \tt \\ \tt Ihr habt gemeinsam 5 von maximal 16 Strafpunkten angesammelt.\\ \tt Bitte versuche es noch mal! 
	  } 
	   } 
	   } 
	 & & \\ 
 

    \theutterance \stepcounter{utterance}  

    & & \multicolumn{4}{p{0.6\linewidth}}{\cellcolor[rgb]{0.9,0.9,0.9}{%
	\makecell[{{p{\linewidth}}}]{% 
	  \tt {\tiny [GM$\rangle$B]}  
	 Strafpunkt: (4,2) ist nicht leer, sondern enthält '┤'.\\ \tt \\ \tt Ihr habt gemeinsam 5 von maximal 16 Strafpunkten angesammelt.\\ \tt Bitte versuche es noch mal! 
	  } 
	   } 
	   } 
	 & \\ 
 

    \theutterance \stepcounter{utterance}  

    & & & \multicolumn{4}{p{0.6\linewidth}}{\cellcolor[rgb]{1,0.85,0.72}{%
	\makecell[{{p{\linewidth}}}]{% 
	  \tt {\tiny [GM$\langle$B]}  
	 VERSCHIEBE: D, (2,1) 
	  } 
	   } 
	   } 
	  \\ 
 

    \theutterance \stepcounter{utterance}  

    & & & \multicolumn{2}{p{0.3\linewidth}}{\cellcolor[rgb]{0.95,0.95,0.95}{%
	\makecell[{{p{\linewidth}}}]{% 
	  \tt {\tiny [GM$|$GM]}  
	 VERSCHIEBE: D, (2,1) 
	  } 
	   } 
	   } 
	 & & \\ 
 

    \theutterance \stepcounter{utterance}  

    & & & \multicolumn{2}{p{0.3\linewidth}}{\cellcolor[rgb]{0.95,0.95,0.95}{%
	\makecell[{{p{\linewidth}}}]{% 
	  \tt {\tiny [GM$|$GM]}  
	 Strafpunkt: (2,1) ist nicht leer, sondern enthält '│'.\\ \tt \\ \tt Ihr habt gemeinsam 6 von maximal 16 Strafpunkten angesammelt.\\ \tt Bitte versuche es noch mal! 
	  } 
	   } 
	   } 
	 & & \\ 
 

    \theutterance \stepcounter{utterance}  

    & & \multicolumn{4}{p{0.6\linewidth}}{\cellcolor[rgb]{0.9,0.9,0.9}{%
	\makecell[{{p{\linewidth}}}]{% 
	  \tt {\tiny [GM$\rangle$B]}  
	 Strafpunkt: (2,1) ist nicht leer, sondern enthält '│'.\\ \tt \\ \tt Ihr habt gemeinsam 6 von maximal 16 Strafpunkten angesammelt.\\ \tt Bitte versuche es noch mal! 
	  } 
	   } 
	   } 
	 & \\ 
 

    \theutterance \stepcounter{utterance}  

    & & & \multicolumn{4}{p{0.6\linewidth}}{\cellcolor[rgb]{1,0.85,0.72}{%
	\makecell[{{p{\linewidth}}}]{% 
	  \tt {\tiny [GM$\langle$B]}  
	 VERSCHIEBE: P, (5,2) 
	  } 
	   } 
	   } 
	  \\ 
 

    \theutterance \stepcounter{utterance}  

    & & & \multicolumn{2}{p{0.3\linewidth}}{\cellcolor[rgb]{0.95,0.95,0.95}{%
	\makecell[{{p{\linewidth}}}]{% 
	  \tt {\tiny [GM$|$GM]}  
	 VERSCHIEBE: P, (5,2) 
	  } 
	   } 
	   } 
	 & & \\ 
 

    \theutterance \stepcounter{utterance}  

    & & & \multicolumn{2}{p{0.3\linewidth}}{\cellcolor[rgb]{0.95,0.95,0.95}{%
	\makecell[{{p{\linewidth}}}]{% 
	  \tt {\tiny [GM$|$GM]}  
	 Strafpunkt: (5,2) ist nicht leer, sondern enthält 'O'.\\ \tt \\ \tt Ihr habt gemeinsam 7 von maximal 16 Strafpunkten angesammelt.\\ \tt Bitte versuche es noch mal! 
	  } 
	   } 
	   } 
	 & & \\ 
 

    \theutterance \stepcounter{utterance}  

    & & \multicolumn{4}{p{0.6\linewidth}}{\cellcolor[rgb]{0.9,0.9,0.9}{%
	\makecell[{{p{\linewidth}}}]{% 
	  \tt {\tiny [GM$\rangle$B]}  
	 Strafpunkt: (5,2) ist nicht leer, sondern enthält 'O'.\\ \tt \\ \tt Ihr habt gemeinsam 7 von maximal 16 Strafpunkten angesammelt.\\ \tt Bitte versuche es noch mal! 
	  } 
	   } 
	   } 
	 & \\ 
 

    \theutterance \stepcounter{utterance}  

    & & & \multicolumn{4}{p{0.6\linewidth}}{\cellcolor[rgb]{1,0.85,0.72}{%
	\makecell[{{p{\linewidth}}}]{% 
	  \tt {\tiny [GM$\langle$B]}  
	 VERSCHIEBE: M, (3,1) 
	  } 
	   } 
	   } 
	  \\ 
 

    \theutterance \stepcounter{utterance}  

    & & & \multicolumn{2}{p{0.3\linewidth}}{\cellcolor[rgb]{0.95,0.95,0.95}{%
	\makecell[{{p{\linewidth}}}]{% 
	  \tt {\tiny [GM$|$GM]}  
	 VERSCHIEBE: M, (3,1) 
	  } 
	   } 
	   } 
	 & & \\ 
 

    \theutterance \stepcounter{utterance}  

    & & & \multicolumn{2}{p{0.3\linewidth}}{\cellcolor[rgb]{0.95,0.95,0.95}{%
	\makecell[{{p{\linewidth}}}]{% 
	  \tt {\tiny [GM$|$GM]}  
	 'M' erfolgreich nach (3,1) verschoben. Dein Raster sieht jetzt so aus:\\ \tt ```\\ \tt  1234567\\ \tt ╔═╤═╤═╤═╗\\ \tt ║A│M│◌│◌║ 1\\ \tt ╟─┼─┤O│T║ 2\\ \tt ║◌│D│◌│◌║ 3\\ \tt ║◌│◌├─┴─╢ 4\\ \tt ╟─┴─┤◌◌◌║ 5\\ \tt ║PS◌└─┬─╢ 6\\ \tt ║◌◌◌◌◌│◌║ 7\\ \tt ╚═════╧═╝\\ \tt ```\\ \tt  
	  } 
	   } 
	   } 
	 & & \\ 
 

    \theutterance \stepcounter{utterance}  

    & & \multicolumn{4}{p{0.6\linewidth}}{\cellcolor[rgb]{0.9,0.9,0.9}{%
	\makecell[{{p{\linewidth}}}]{% 
	  \tt {\tiny [A$\langle$GM]}  
	 Deine Nachricht wurde an deinen Mitspieler übermittelt.\\ \tt \\ \tt \\ \tt Ihr seid gerade bei Runde 7 von maximal 28.\\ \tt Ihr habt gemeinsam 7 von maximal 16 Strafpunkten angesammelt.\\ \tt Dein Mitspieler hat ein Objekt auf seinem Raster bewegt.\\ \tt \\ \tt \\ \tt Wie lautet dein nächster Befehl? 
	  } 
	   } 
	   } 
	 & \\ 
 

    \theutterance \stepcounter{utterance}  

    & \multicolumn{4}{p{0.6\linewidth}}{\cellcolor[rgb]{0.8,1,0.9}{%
	\makecell[{{p{\linewidth}}}]{% 
	  \tt {\tiny [A$\rangle$GM]}  
	 SPRICH: Ich werde jetzt mein erstes Objekt bewegen, um es in die richtige Position zu bringen. Ich werde das Objekt A, das sich derzeit bei (1,1) befindet, nicht bewegen, da es bereits an der richtigen Position ist. Als nächstes werde ich mir das Objekt D ansehen, das sich derzeit bei (1,6) befindet und es zu seiner Zielposition (2,1) bewegen. 
	  } 
	   } 
	   } 
	 & & \\ 
 

    \theutterance \stepcounter{utterance}  

    & & & \multicolumn{2}{p{0.3\linewidth}}{\cellcolor[rgb]{0.95,0.95,0.95}{%
	\makecell[{{p{\linewidth}}}]{% 
	  \tt {\tiny [GM$|$GM]}  
	 SPRICH: Ich werde jetzt mein erstes Objekt bewegen, um es in die richtige Position zu bringen. Ich werde das Objekt A, das sich derzeit bei (1,1) befindet, nicht bewegen, da es bereits an der richtigen Position ist. Als nächstes werde ich mir das Objekt D ansehen, das sich derzeit bei (1,6) befindet und es zu seiner Zielposition (2,1) bewegen. 
	  } 
	   } 
	   } 
	 & & \\ 
 

    \theutterance \stepcounter{utterance}  

    & & \multicolumn{4}{p{0.6\linewidth}}{\cellcolor[rgb]{0.9,0.9,0.9}{%
	\makecell[{{p{\linewidth}}}]{% 
	  \tt {\tiny [GM$\rangle$B]}  
	 'M' erfolgreich nach (3,1) verschoben. Dein Raster sieht jetzt so aus:\\ \tt ```\\ \tt  1234567\\ \tt ╔═╤═╤═╤═╗\\ \tt ║A│M│◌│◌║ 1\\ \tt ╟─┼─┤O│T║ 2\\ \tt ║◌│D│◌│◌║ 3\\ \tt ║◌│◌├─┴─╢ 4\\ \tt ╟─┴─┤◌◌◌║ 5\\ \tt ║PS◌└─┬─╢ 6\\ \tt ║◌◌◌◌◌│◌║ 7\\ \tt ╚═════╧═╝\\ \tt ```\\ \tt Ihr seid gerade bei Runde 8 von maximal 28.\\ \tt Ihr habt gemeinsam 7 von maximal 16 Strafpunkten angesammelt.\\ \tt Dein Mitspieler hat dir diese Nachricht geschickt:\\ \tt \\ \tt \\ \tt "Ich werde jetzt mein erstes Objekt bewegen, um es in die richtige Position zu bringen. Ich werde das Objekt A, das sich derzeit bei (1,1) befindet, nicht bewegen, da es bereits an der richtigen Position ist. Als nächstes werde ich mir das Objekt D ansehen, das sich derzeit bei (1,6) befindet und es zu seiner Zielposition (2,1) bewegen."\\ \tt \\ \tt \\ \tt Wie lautet dein nächster Befehl? 
	  } 
	   } 
	   } 
	 & \\ 
 

    \theutterance \stepcounter{utterance}  

    & & & \multicolumn{4}{p{0.6\linewidth}}{\cellcolor[rgb]{1,0.85,0.72}{%
	\makecell[{{p{\linewidth}}}]{% 
	  \tt {\tiny [GM$\langle$B]}  
	 VERSCHIEBE: D, (2,1) 
	  } 
	   } 
	   } 
	  \\ 
 

    \theutterance \stepcounter{utterance}  

    & & & \multicolumn{2}{p{0.3\linewidth}}{\cellcolor[rgb]{0.95,0.95,0.95}{%
	\makecell[{{p{\linewidth}}}]{% 
	  \tt {\tiny [GM$|$GM]}  
	 VERSCHIEBE: D, (2,1) 
	  } 
	   } 
	   } 
	 & & \\ 
 

    \theutterance \stepcounter{utterance}  

    & & & \multicolumn{2}{p{0.3\linewidth}}{\cellcolor[rgb]{0.95,0.95,0.95}{%
	\makecell[{{p{\linewidth}}}]{% 
	  \tt {\tiny [GM$|$GM]}  
	 Strafpunkt: (2,1) ist nicht leer, sondern enthält '│'.\\ \tt \\ \tt Ihr habt gemeinsam 8 von maximal 16 Strafpunkten angesammelt.\\ \tt Bitte versuche es noch mal! 
	  } 
	   } 
	   } 
	 & & \\ 
 

    \theutterance \stepcounter{utterance}  

    & & \multicolumn{4}{p{0.6\linewidth}}{\cellcolor[rgb]{0.9,0.9,0.9}{%
	\makecell[{{p{\linewidth}}}]{% 
	  \tt {\tiny [GM$\rangle$B]}  
	 Strafpunkt: (2,1) ist nicht leer, sondern enthält '│'.\\ \tt \\ \tt Ihr habt gemeinsam 8 von maximal 16 Strafpunkten angesammelt.\\ \tt Bitte versuche es noch mal! 
	  } 
	   } 
	   } 
	 & \\ 
 

    \theutterance \stepcounter{utterance}  

    & & & \multicolumn{4}{p{0.6\linewidth}}{\cellcolor[rgb]{1,0.85,0.72}{%
	\makecell[{{p{\linewidth}}}]{% 
	  \tt {\tiny [GM$\langle$B]}  
	 VERSCHIEBE: T, (7,3) 
	  } 
	   } 
	   } 
	  \\ 
 

    \theutterance \stepcounter{utterance}  

    & & & \multicolumn{2}{p{0.3\linewidth}}{\cellcolor[rgb]{0.95,0.95,0.95}{%
	\makecell[{{p{\linewidth}}}]{% 
	  \tt {\tiny [GM$|$GM]}  
	 VERSCHIEBE: T, (7,3) 
	  } 
	   } 
	   } 
	 & & \\ 
 

    \theutterance \stepcounter{utterance}  

    & & & \multicolumn{2}{p{0.3\linewidth}}{\cellcolor[rgb]{0.95,0.95,0.95}{%
	\makecell[{{p{\linewidth}}}]{% 
	  \tt {\tiny [GM$|$GM]}  
	 'T' erfolgreich nach (7,3) verschoben. Dein Raster sieht jetzt so aus:\\ \tt ```\\ \tt  1234567\\ \tt ╔═╤═╤═╤═╗\\ \tt ║A│M│◌│◌║ 1\\ \tt ╟─┼─┤O│◌║ 2\\ \tt ║◌│D│◌│T║ 3\\ \tt ║◌│◌├─┴─╢ 4\\ \tt ╟─┴─┤◌◌◌║ 5\\ \tt ║PS◌└─┬─╢ 6\\ \tt ║◌◌◌◌◌│◌║ 7\\ \tt ╚═════╧═╝\\ \tt ```\\ \tt  
	  } 
	   } 
	   } 
	 & & \\ 
 

    \theutterance \stepcounter{utterance}  

    & & \multicolumn{4}{p{0.6\linewidth}}{\cellcolor[rgb]{0.9,0.9,0.9}{%
	\makecell[{{p{\linewidth}}}]{% 
	  \tt {\tiny [A$\langle$GM]}  
	 Deine Nachricht wurde an deinen Mitspieler übermittelt.\\ \tt \\ \tt \\ \tt Ihr seid gerade bei Runde 8 von maximal 28.\\ \tt Ihr habt gemeinsam 8 von maximal 16 Strafpunkten angesammelt.\\ \tt Dein Mitspieler hat ein Objekt auf seinem Raster bewegt.\\ \tt \\ \tt \\ \tt Wie lautet dein nächster Befehl? 
	  } 
	   } 
	   } 
	 & \\ 
 

    \theutterance \stepcounter{utterance}  

    & \multicolumn{4}{p{0.6\linewidth}}{\cellcolor[rgb]{0.8,1,0.9}{%
	\makecell[{{p{\linewidth}}}]{% 
	  \tt {\tiny [A$\rangle$GM]}  
	 VERSCHIEBE: D, (2,1) 
	  } 
	   } 
	   } 
	 & & \\ 
 

    \theutterance \stepcounter{utterance}  

    & & & \multicolumn{2}{p{0.3\linewidth}}{\cellcolor[rgb]{0.95,0.95,0.95}{%
	\makecell[{{p{\linewidth}}}]{% 
	  \tt {\tiny [GM$|$GM]}  
	 VERSCHIEBE: D, (2,1) 
	  } 
	   } 
	   } 
	 & & \\ 
 

    \theutterance \stepcounter{utterance}  

    & & & \multicolumn{2}{p{0.3\linewidth}}{\cellcolor[rgb]{0.95,0.95,0.95}{%
	\makecell[{{p{\linewidth}}}]{% 
	  \tt {\tiny [GM$|$GM]}  
	 Strafpunkt: (2,1) ist nicht leer, sondern enthält '│'.\\ \tt \\ \tt Ihr habt gemeinsam 9 von maximal 16 Strafpunkten angesammelt.\\ \tt Bitte versuche es noch mal! 
	  } 
	   } 
	   } 
	 & & \\ 
 

    \theutterance \stepcounter{utterance}  

    & & \multicolumn{4}{p{0.6\linewidth}}{\cellcolor[rgb]{0.9,0.9,0.9}{%
	\makecell[{{p{\linewidth}}}]{% 
	  \tt {\tiny [A$\langle$GM]}  
	 Strafpunkt: (2,1) ist nicht leer, sondern enthält '│'.\\ \tt \\ \tt Ihr habt gemeinsam 9 von maximal 16 Strafpunkten angesammelt.\\ \tt Bitte versuche es noch mal! 
	  } 
	   } 
	   } 
	 & \\ 
 

    \theutterance \stepcounter{utterance}  

    & \multicolumn{4}{p{0.6\linewidth}}{\cellcolor[rgb]{0.8,1,0.9}{%
	\makecell[{{p{\linewidth}}}]{% 
	  \tt {\tiny [A$\rangle$GM]}  
	 VERSCHIEBE: D, (2,3) 
	  } 
	   } 
	   } 
	 & & \\ 
 

    \theutterance \stepcounter{utterance}  

    & & & \multicolumn{2}{p{0.3\linewidth}}{\cellcolor[rgb]{0.95,0.95,0.95}{%
	\makecell[{{p{\linewidth}}}]{% 
	  \tt {\tiny [GM$|$GM]}  
	 VERSCHIEBE: D, (2,3) 
	  } 
	   } 
	   } 
	 & & \\ 
 

    \theutterance \stepcounter{utterance}  

    & & & \multicolumn{2}{p{0.3\linewidth}}{\cellcolor[rgb]{0.95,0.95,0.95}{%
	\makecell[{{p{\linewidth}}}]{% 
	  \tt {\tiny [GM$|$GM]}  
	 Strafpunkt: (2,3) ist nicht leer, sondern enthält '│'.\\ \tt \\ \tt Ihr habt gemeinsam 10 von maximal 16 Strafpunkten angesammelt.\\ \tt Bitte versuche es noch mal! 
	  } 
	   } 
	   } 
	 & & \\ 
 

    \theutterance \stepcounter{utterance}  

    & & \multicolumn{4}{p{0.6\linewidth}}{\cellcolor[rgb]{0.9,0.9,0.9}{%
	\makecell[{{p{\linewidth}}}]{% 
	  \tt {\tiny [A$\langle$GM]}  
	 Strafpunkt: (2,3) ist nicht leer, sondern enthält '│'.\\ \tt \\ \tt Ihr habt gemeinsam 10 von maximal 16 Strafpunkten angesammelt.\\ \tt Bitte versuche es noch mal! 
	  } 
	   } 
	   } 
	 & \\ 
 

    \theutterance \stepcounter{utterance}  

    & \multicolumn{4}{p{0.6\linewidth}}{\cellcolor[rgb]{0.8,1,0.9}{%
	\makecell[{{p{\linewidth}}}]{% 
	  \tt {\tiny [A$\rangle$GM]}  
	 VERSCHIEBE: D, (2,2) 
	  } 
	   } 
	   } 
	 & & \\ 
 

    \theutterance \stepcounter{utterance}  

    & & & \multicolumn{2}{p{0.3\linewidth}}{\cellcolor[rgb]{0.95,0.95,0.95}{%
	\makecell[{{p{\linewidth}}}]{% 
	  \tt {\tiny [GM$|$GM]}  
	 VERSCHIEBE: D, (2,2) 
	  } 
	   } 
	   } 
	 & & \\ 
 

    \theutterance \stepcounter{utterance}  

    & & & \multicolumn{2}{p{0.3\linewidth}}{\cellcolor[rgb]{0.95,0.95,0.95}{%
	\makecell[{{p{\linewidth}}}]{% 
	  \tt {\tiny [GM$|$GM]}  
	 Strafpunkt: (2,2) ist nicht leer, sondern enthält '┼'.\\ \tt \\ \tt Ihr habt gemeinsam 11 von maximal 16 Strafpunkten angesammelt.\\ \tt Bitte versuche es noch mal! 
	  } 
	   } 
	   } 
	 & & \\ 
 

    \theutterance \stepcounter{utterance}  

    & & \multicolumn{4}{p{0.6\linewidth}}{\cellcolor[rgb]{0.9,0.9,0.9}{%
	\makecell[{{p{\linewidth}}}]{% 
	  \tt {\tiny [A$\langle$GM]}  
	 Strafpunkt: (2,2) ist nicht leer, sondern enthält '┼'.\\ \tt \\ \tt Ihr habt gemeinsam 11 von maximal 16 Strafpunkten angesammelt.\\ \tt Bitte versuche es noch mal! 
	  } 
	   } 
	   } 
	 & \\ 
 

    \theutterance \stepcounter{utterance}  

    & \multicolumn{4}{p{0.6\linewidth}}{\cellcolor[rgb]{0.8,1,0.9}{%
	\makecell[{{p{\linewidth}}}]{% 
	  \tt {\tiny [A$\rangle$GM]}  
	 VERSCHIEBE: D, (1,2) 
	  } 
	   } 
	   } 
	 & & \\ 
 

    \theutterance \stepcounter{utterance}  

    & & & \multicolumn{2}{p{0.3\linewidth}}{\cellcolor[rgb]{0.95,0.95,0.95}{%
	\makecell[{{p{\linewidth}}}]{% 
	  \tt {\tiny [GM$|$GM]}  
	 VERSCHIEBE: D, (1,2) 
	  } 
	   } 
	   } 
	 & & \\ 
 

    \theutterance \stepcounter{utterance}  

    & & & \multicolumn{2}{p{0.3\linewidth}}{\cellcolor[rgb]{0.95,0.95,0.95}{%
	\makecell[{{p{\linewidth}}}]{% 
	  \tt {\tiny [GM$|$GM]}  
	 Strafpunkt: (1,2) ist nicht leer, sondern enthält '─'.\\ \tt \\ \tt Ihr habt gemeinsam 12 von maximal 16 Strafpunkten angesammelt.\\ \tt Bitte versuche es noch mal! 
	  } 
	   } 
	   } 
	 & & \\ 
 

    \theutterance \stepcounter{utterance}  

    & & \multicolumn{4}{p{0.6\linewidth}}{\cellcolor[rgb]{0.9,0.9,0.9}{%
	\makecell[{{p{\linewidth}}}]{% 
	  \tt {\tiny [A$\langle$GM]}  
	 Strafpunkt: (1,2) ist nicht leer, sondern enthält '─'.\\ \tt \\ \tt Ihr habt gemeinsam 12 von maximal 16 Strafpunkten angesammelt.\\ \tt Bitte versuche es noch mal! 
	  } 
	   } 
	   } 
	 & \\ 
 

    \theutterance \stepcounter{utterance}  

    & \multicolumn{4}{p{0.6\linewidth}}{\cellcolor[rgb]{0.8,1,0.9}{%
	\makecell[{{p{\linewidth}}}]{% 
	  \tt {\tiny [A$\rangle$GM]}  
	 VERSCHIEBE: D, (3,1) 
	  } 
	   } 
	   } 
	 & & \\ 
 

    \theutterance \stepcounter{utterance}  

    & & & \multicolumn{2}{p{0.3\linewidth}}{\cellcolor[rgb]{0.95,0.95,0.95}{%
	\makecell[{{p{\linewidth}}}]{% 
	  \tt {\tiny [GM$|$GM]}  
	 VERSCHIEBE: D, (3,1) 
	  } 
	   } 
	   } 
	 & & \\ 
 

    \theutterance \stepcounter{utterance}  

    & & & \multicolumn{2}{p{0.3\linewidth}}{\cellcolor[rgb]{0.95,0.95,0.95}{%
	\makecell[{{p{\linewidth}}}]{% 
	  \tt {\tiny [GM$|$GM]}  
	 'D' erfolgreich nach (3,1) verschoben. Dein Raster sieht jetzt so aus:\\ \tt ```\\ \tt  1234567\\ \tt ╔═╤═╤═╤═╗\\ \tt ║A│D│O│◌║ 1\\ \tt ╟─┼─┤◌│◌║ 2\\ \tt ║◌│T│◌│P║ 3\\ \tt ║◌│◌├─┴─╢ 4\\ \tt ╟─┴─┤◌◌◌║ 5\\ \tt ║◌◌◌└─┬─╢ 6\\ \tt ║◌◌◌S◌│M║ 7\\ \tt ╚═════╧═╝\\ \tt ```\\ \tt  
	  } 
	   } 
	   } 
	 & & \\ 
 

    \theutterance \stepcounter{utterance}  

    & & \multicolumn{4}{p{0.6\linewidth}}{\cellcolor[rgb]{0.9,0.9,0.9}{%
	\makecell[{{p{\linewidth}}}]{% 
	  \tt {\tiny [GM$\rangle$B]}  
	 'T' erfolgreich nach (7,3) verschoben. Dein Raster sieht jetzt so aus:\\ \tt ```\\ \tt  1234567\\ \tt ╔═╤═╤═╤═╗\\ \tt ║A│M│◌│◌║ 1\\ \tt ╟─┼─┤O│◌║ 2\\ \tt ║◌│D│◌│T║ 3\\ \tt ║◌│◌├─┴─╢ 4\\ \tt ╟─┴─┤◌◌◌║ 5\\ \tt ║PS◌└─┬─╢ 6\\ \tt ║◌◌◌◌◌│◌║ 7\\ \tt ╚═════╧═╝\\ \tt ```\\ \tt Ihr seid gerade bei Runde 9 von maximal 28.\\ \tt Ihr habt gemeinsam 12 von maximal 16 Strafpunkten angesammelt.\\ \tt Dein Mitspieler hat ein Objekt auf seinem Raster bewegt.\\ \tt \\ \tt \\ \tt Wie lautet dein nächster Befehl? 
	  } 
	   } 
	   } 
	 & \\ 
 

    \theutterance \stepcounter{utterance}  

    & & & \multicolumn{4}{p{0.6\linewidth}}{\cellcolor[rgb]{1,0.85,0.72}{%
	\makecell[{{p{\linewidth}}}]{% 
	  \tt {\tiny [GM$\langle$B]}  
	 VERSCHIEBE: O, (4,2) 
	  } 
	   } 
	   } 
	  \\ 
 

    \theutterance \stepcounter{utterance}  

    & & & \multicolumn{2}{p{0.3\linewidth}}{\cellcolor[rgb]{0.95,0.95,0.95}{%
	\makecell[{{p{\linewidth}}}]{% 
	  \tt {\tiny [GM$|$GM]}  
	 VERSCHIEBE: O, (4,2) 
	  } 
	   } 
	   } 
	 & & \\ 
 

    \theutterance \stepcounter{utterance}  

    & & & \multicolumn{2}{p{0.3\linewidth}}{\cellcolor[rgb]{0.95,0.95,0.95}{%
	\makecell[{{p{\linewidth}}}]{% 
	  \tt {\tiny [GM$|$GM]}  
	 Strafpunkt: (4,2) ist nicht leer, sondern enthält '┤'.\\ \tt \\ \tt Ihr habt gemeinsam 13 von maximal 16 Strafpunkten angesammelt.\\ \tt Bitte versuche es noch mal! 
	  } 
	   } 
	   } 
	 & & \\ 
 

    \theutterance \stepcounter{utterance}  

    & & \multicolumn{4}{p{0.6\linewidth}}{\cellcolor[rgb]{0.9,0.9,0.9}{%
	\makecell[{{p{\linewidth}}}]{% 
	  \tt {\tiny [GM$\rangle$B]}  
	 Strafpunkt: (4,2) ist nicht leer, sondern enthält '┤'.\\ \tt \\ \tt Ihr habt gemeinsam 13 von maximal 16 Strafpunkten angesammelt.\\ \tt Bitte versuche es noch mal! 
	  } 
	   } 
	   } 
	 & \\ 
 

    \theutterance \stepcounter{utterance}  

    & & & \multicolumn{4}{p{0.6\linewidth}}{\cellcolor[rgb]{1,0.85,0.72}{%
	\makecell[{{p{\linewidth}}}]{% 
	  \tt {\tiny [GM$\langle$B]}  
	 VERSCHIEBE: P, (5,2) 
	  } 
	   } 
	   } 
	  \\ 
 

    \theutterance \stepcounter{utterance}  

    & & & \multicolumn{2}{p{0.3\linewidth}}{\cellcolor[rgb]{0.95,0.95,0.95}{%
	\makecell[{{p{\linewidth}}}]{% 
	  \tt {\tiny [GM$|$GM]}  
	 VERSCHIEBE: P, (5,2) 
	  } 
	   } 
	   } 
	 & & \\ 
 

    \theutterance \stepcounter{utterance}  

    & & & \multicolumn{2}{p{0.3\linewidth}}{\cellcolor[rgb]{0.95,0.95,0.95}{%
	\makecell[{{p{\linewidth}}}]{% 
	  \tt {\tiny [GM$|$GM]}  
	 Strafpunkt: (5,2) ist nicht leer, sondern enthält 'O'.\\ \tt \\ \tt Ihr habt gemeinsam 14 von maximal 16 Strafpunkten angesammelt.\\ \tt Bitte versuche es noch mal! 
	  } 
	   } 
	   } 
	 & & \\ 
 

    \theutterance \stepcounter{utterance}  

    & & \multicolumn{4}{p{0.6\linewidth}}{\cellcolor[rgb]{0.9,0.9,0.9}{%
	\makecell[{{p{\linewidth}}}]{% 
	  \tt {\tiny [GM$\rangle$B]}  
	 Strafpunkt: (5,2) ist nicht leer, sondern enthält 'O'.\\ \tt \\ \tt Ihr habt gemeinsam 14 von maximal 16 Strafpunkten angesammelt.\\ \tt Bitte versuche es noch mal! 
	  } 
	   } 
	   } 
	 & \\ 
 

    \theutterance \stepcounter{utterance}  

    & & & \multicolumn{4}{p{0.6\linewidth}}{\cellcolor[rgb]{1,0.85,0.72}{%
	\makecell[{{p{\linewidth}}}]{% 
	  \tt {\tiny [GM$\langle$B]}  
	 VERSCHIEBE: S, (6,2) 
	  } 
	   } 
	   } 
	  \\ 
 

    \theutterance \stepcounter{utterance}  

    & & & \multicolumn{2}{p{0.3\linewidth}}{\cellcolor[rgb]{0.95,0.95,0.95}{%
	\makecell[{{p{\linewidth}}}]{% 
	  \tt {\tiny [GM$|$GM]}  
	 VERSCHIEBE: S, (6,2) 
	  } 
	   } 
	   } 
	 & & \\ 
 

    \theutterance \stepcounter{utterance}  

    & & & \multicolumn{2}{p{0.3\linewidth}}{\cellcolor[rgb]{0.95,0.95,0.95}{%
	\makecell[{{p{\linewidth}}}]{% 
	  \tt {\tiny [GM$|$GM]}  
	 Strafpunkt: (6,2) ist nicht leer, sondern enthält '│'.\\ \tt \\ \tt Ihr habt gemeinsam 15 von maximal 16 Strafpunkten angesammelt.\\ \tt Bitte versuche es noch mal! 
	  } 
	   } 
	   } 
	 & & \\ 
 

    \theutterance \stepcounter{utterance}  

    & & \multicolumn{4}{p{0.6\linewidth}}{\cellcolor[rgb]{0.9,0.9,0.9}{%
	\makecell[{{p{\linewidth}}}]{% 
	  \tt {\tiny [GM$\rangle$B]}  
	 Strafpunkt: (6,2) ist nicht leer, sondern enthält '│'.\\ \tt \\ \tt Ihr habt gemeinsam 15 von maximal 16 Strafpunkten angesammelt.\\ \tt Bitte versuche es noch mal! 
	  } 
	   } 
	   } 
	 & \\ 
 

    \theutterance \stepcounter{utterance}  

    & & & \multicolumn{4}{p{0.6\linewidth}}{\cellcolor[rgb]{1,0.85,0.72}{%
	\makecell[{{p{\linewidth}}}]{% 
	  \tt {\tiny [GM$\langle$B]}  
	 VERSCHIEBE: D, (3,3) 
	  } 
	   } 
	   } 
	  \\ 
 

    \theutterance \stepcounter{utterance}  

    & & & \multicolumn{2}{p{0.3\linewidth}}{\cellcolor[rgb]{0.95,0.95,0.95}{%
	\makecell[{{p{\linewidth}}}]{% 
	  \tt {\tiny [GM$|$GM]}  
	 VERSCHIEBE: D, (3,3) 
	  } 
	   } 
	   } 
	 & & \\ 
 

    \theutterance \stepcounter{utterance}  

    & & & \multicolumn{2}{p{0.3\linewidth}}{\cellcolor[rgb]{0.95,0.95,0.95}{%
	\makecell[{{p{\linewidth}}}]{% 
	  \tt {\tiny [GM$|$GM]}  
	 Strafpunkt: (3,3) ist nicht leer, sondern enthält 'D'.\\ \tt \\ \tt Ihr habt gemeinsam 16 von maximal 16 Strafpunkten angesammelt.\\ \tt Bitte versuche es noch mal! 
	  } 
	   } 
	   } 
	 & & \\ 
 

    \theutterance \stepcounter{utterance}  

    & & \multicolumn{4}{p{0.6\linewidth}}{\cellcolor[rgb]{0.9,0.9,0.9}{%
	\makecell[{{p{\linewidth}}}]{% 
	  \tt {\tiny [GM$\rangle$B]}  
	 Strafpunkt: (3,3) ist nicht leer, sondern enthält 'D'.\\ \tt \\ \tt Ihr habt gemeinsam 16 von maximal 16 Strafpunkten angesammelt.\\ \tt Bitte versuche es noch mal! 
	  } 
	   } 
	   } 
	 & \\ 
 

    \theutterance \stepcounter{utterance}  

    & & & \multicolumn{4}{p{0.6\linewidth}}{\cellcolor[rgb]{1,0.85,0.72}{%
	\makecell[{{p{\linewidth}}}]{% 
	  \tt {\tiny [GM$\langle$B]}  
	 SPRICH: Ich denke, wir sollten unsere Strategie überdenken und versuchen, die Objekte auf eine andere Weise zu bewegen. Wir haben zu viele Strafpunkte gesammelt und sollten uns auf eine neue Vorgehensweise konzentrieren. Wie sehen Sie unsere Chancen, das Spiel noch zu gewinnen? Sollten wir weitermachen oder einen anderen Ansatz wählen? 
	  } 
	   } 
	   } 
	  \\ 
 

    \theutterance \stepcounter{utterance}  

    & & & \multicolumn{2}{p{0.3\linewidth}}{\cellcolor[rgb]{0.95,0.95,0.95}{%
	\makecell[{{p{\linewidth}}}]{% 
	  \tt {\tiny [GM$|$GM]}  
	 SPRICH: Ich denke, wir sollten unsere Strategie überdenken und versuchen, die Objekte auf eine andere Weise zu bewegen. Wir haben zu viele Strafpunkte gesammelt und sollten uns auf eine neue Vorgehensweise konzentrieren. Wie sehen Sie unsere Chancen, das Spiel noch zu gewinnen? Sollten wir weitermachen oder einen anderen Ansatz wählen? 
	  } 
	   } 
	   } 
	 & & \\ 
 

    \theutterance \stepcounter{utterance}  

    & & \multicolumn{4}{p{0.6\linewidth}}{\cellcolor[rgb]{0.9,0.9,0.9}{%
	\makecell[{{p{\linewidth}}}]{% 
	  \tt {\tiny [A$\langle$GM]}  
	 'D' erfolgreich nach (3,1) verschoben. Dein Raster sieht jetzt so aus:\\ \tt ```\\ \tt  1234567\\ \tt ╔═╤═╤═╤═╗\\ \tt ║A│D│O│◌║ 1\\ \tt ╟─┼─┤◌│◌║ 2\\ \tt ║◌│T│◌│P║ 3\\ \tt ║◌│◌├─┴─╢ 4\\ \tt ╟─┴─┤◌◌◌║ 5\\ \tt ║◌◌◌└─┬─╢ 6\\ \tt ║◌◌◌S◌│M║ 7\\ \tt ╚═════╧═╝\\ \tt ```\\ \tt Ihr seid gerade bei Runde 9 von maximal 28.\\ \tt Ihr habt gemeinsam 16 von maximal 16 Strafpunkten angesammelt.\\ \tt Dein Mitspieler hat dir diese Nachricht geschickt:\\ \tt \\ \tt \\ \tt "Ich denke, wir sollten unsere Strategie überdenken und versuchen, die Objekte auf eine andere Weise zu bewegen. Wir haben zu viele Strafpunkte gesammelt und sollten uns auf eine neue Vorgehensweise konzentrieren. Wie sehen Sie unsere Chancen, das Spiel noch zu gewinnen? Sollten wir weitermachen oder einen anderen Ansatz wählen?"\\ \tt \\ \tt \\ \tt Wie lautet dein nächster Befehl? 
	  } 
	   } 
	   } 
	 & \\ 
 

    \theutterance \stepcounter{utterance}  

    & \multicolumn{4}{p{0.6\linewidth}}{\cellcolor[rgb]{0.8,1,0.9}{%
	\makecell[{{p{\linewidth}}}]{% 
	  \tt {\tiny [A$\rangle$GM]}  
	 SPRICH: Ich denke, wir sollten aufhören, da wir die maximale Anzahl an Strafpunkten erreicht haben und es unwahrscheinlich ist, dass wir das Spiel noch gewinnen können. Es wäre besser, das Spiel zu beenden und aus unseren Fehlern zu lernen, anstatt weiter zu versuchen und möglicherweise noch mehr Strafpunkte zu sammeln. 
	  } 
	   } 
	   } 
	 & & \\ 
 

    \theutterance \stepcounter{utterance}  

    & & & \multicolumn{2}{p{0.3\linewidth}}{\cellcolor[rgb]{0.95,0.95,0.95}{%
	\makecell[{{p{\linewidth}}}]{% 
	  \tt {\tiny [GM$|$GM]}  
	 SPRICH: Ich denke, wir sollten aufhören, da wir die maximale Anzahl an Strafpunkten erreicht haben und es unwahrscheinlich ist, dass wir das Spiel noch gewinnen können. Es wäre besser, das Spiel zu beenden und aus unseren Fehlern zu lernen, anstatt weiter zu versuchen und möglicherweise noch mehr Strafpunkte zu sammeln. 
	  } 
	   } 
	   } 
	 & & \\ 
 

    \theutterance \stepcounter{utterance}  

    & & \multicolumn{4}{p{0.6\linewidth}}{\cellcolor[rgb]{0.9,0.9,0.9}{%
	\makecell[{{p{\linewidth}}}]{% 
	  \tt {\tiny [GM$\rangle$B]}  
	 Deine Nachricht wurde an deinen Mitspieler übermittelt.\\ \tt \\ \tt \\ \tt Ihr seid gerade bei Runde 10 von maximal 28.\\ \tt Ihr habt gemeinsam 16 von maximal 16 Strafpunkten angesammelt.\\ \tt Dein Mitspieler hat dir diese Nachricht geschickt:\\ \tt \\ \tt \\ \tt "Ich denke, wir sollten aufhören, da wir die maximale Anzahl an Strafpunkten erreicht haben und es unwahrscheinlich ist, dass wir das Spiel noch gewinnen können. Es wäre besser, das Spiel zu beenden und aus unseren Fehlern zu lernen, anstatt weiter zu versuchen und möglicherweise noch mehr Strafpunkte zu sammeln."\\ \tt \\ \tt \\ \tt Wie lautet dein nächster Befehl? 
	  } 
	   } 
	   } 
	 & \\ 
 

    \theutterance \stepcounter{utterance}  

    & & & \multicolumn{4}{p{0.6\linewidth}}{\cellcolor[rgb]{1,0.85,0.72}{%
	\makecell[{{p{\linewidth}}}]{% 
	  \tt {\tiny [GM$\langle$B]}  
	 SPRICH: fertig! 
	  } 
	   } 
	   } 
	  \\ 
 

    \theutterance \stepcounter{utterance}  

    & & & \multicolumn{2}{p{0.3\linewidth}}{\cellcolor[rgb]{0.95,0.95,0.95}{%
	\makecell[{{p{\linewidth}}}]{% 
	  \tt {\tiny [GM$|$GM]}  
	 SPRICH: fertig! 
	  } 
	   } 
	   } 
	 & & \\ 
 

    \theutterance \stepcounter{utterance}  

    & & \multicolumn{4}{p{0.6\linewidth}}{\cellcolor[rgb]{0.9,0.9,0.9}{%
	\makecell[{{p{\linewidth}}}]{% 
	  \tt {\tiny [A$\langle$GM]}  
	 Deine Nachricht wurde an deinen Mitspieler übermittelt.\\ \tt \\ \tt \\ \tt Ihr seid gerade bei Runde 10 von maximal 28.\\ \tt Ihr habt gemeinsam 16 von maximal 16 Strafpunkten angesammelt.\\ \tt Dein Mitspieler hat dir diese Nachricht geschickt:\\ \tt \\ \tt \\ \tt "fertig!"\\ \tt \\ \tt \\ \tt Wie lautet dein nächster Befehl? 
	  } 
	   } 
	   } 
	 & \\ 
 

    \theutterance \stepcounter{utterance}  

    & \multicolumn{4}{p{0.6\linewidth}}{\cellcolor[rgb]{0.8,1,0.9}{%
	\makecell[{{p{\linewidth}}}]{% 
	  \tt {\tiny [A$\rangle$GM]}  
	 SPRICH: fertig! 
	  } 
	   } 
	   } 
	 & & \\ 
 

    \theutterance \stepcounter{utterance}  

    & & & \multicolumn{2}{p{0.3\linewidth}}{\cellcolor[rgb]{0.95,0.95,0.95}{%
	\makecell[{{p{\linewidth}}}]{% 
	  \tt {\tiny [GM$|$GM]}  
	 SPRICH: fertig! 
	  } 
	   } 
	   } 
	 & & \\ 
 

    \theutterance \stepcounter{utterance}  

    & & \multicolumn{4}{p{0.6\linewidth}}{\cellcolor[rgb]{0.9,0.9,0.9}{%
	\makecell[{{p{\linewidth}}}]{% 
	  \tt {\tiny [GM$\rangle$B]}  
	 Deine Nachricht wurde an deinen Mitspieler übermittelt.\\ \tt \\ \tt \\ \tt Ihr seid gerade bei Runde 11 von maximal 28.\\ \tt Ihr habt gemeinsam 16 von maximal 16 Strafpunkten angesammelt.\\ \tt Dein Mitspieler hat dir diese Nachricht geschickt:\\ \tt \\ \tt \\ \tt "fertig!"\\ \tt \\ \tt \\ \tt Wie lautet dein nächster Befehl? 
	  } 
	   } 
	   } 
	 & \\ 
 

    \theutterance \stepcounter{utterance}  

    & & & \multicolumn{4}{p{0.6\linewidth}}{\cellcolor[rgb]{1,0.85,0.72}{%
	\makecell[{{p{\linewidth}}}]{% 
	  \tt {\tiny [GM$\langle$B]}  
	 Das Spiel ist zu Ende. Es gibt keine weiteren Befehle. 
	  } 
	   } 
	   } 
	  \\ 
 

    \theutterance \stepcounter{utterance}  

    & & & \multicolumn{2}{p{0.3\linewidth}}{\cellcolor[rgb]{0.95,0.95,0.95}{%
	\makecell[{{p{\linewidth}}}]{% 
	  \tt {\tiny [GM$|$GM]}  
	 Das Spiel ist zu Ende. Es gibt keine weiteren Befehle. 
	  } 
	   } 
	   } 
	 & & \\ 
 

    \theutterance \stepcounter{utterance}  

    & & & \multicolumn{2}{p{0.3\linewidth}}{\cellcolor[rgb]{0.95,0.95,0.95}{%
	\makecell[{{p{\linewidth}}}]{% 
	  \tt {\tiny [GM$|$GM]}  
	 Invalid response format 
	  } 
	   } 
	   } 
	 & & \\ 
 

    \theutterance \stepcounter{utterance}  

    & & & \multicolumn{2}{p{0.3\linewidth}}{\cellcolor[rgb]{0.95,0.95,0.95}{%
	\makecell[{{p{\linewidth}}}]{% 
	  \tt {\tiny [GM$|$GM]}  
	 Maximum number of penalties exceeded 
	  } 
	   } 
	   } 
	 & & \\ 
 

    \theutterance \stepcounter{utterance}  

    & & & \multicolumn{2}{p{0.3\linewidth}}{\cellcolor[rgb]{0.95,0.95,0.95}{%
	\makecell[{{p{\linewidth}}}]{% 
	  \tt {\tiny [GM$|$GM]}  
	 * success: False\\ \tt * lose: True\\ \tt * aborted: True\\ \tt -------\\ \tt * Shifts: 2.00\\ \tt * Max Shifts: 12.00\\ \tt * Min Shifts: 6.00\\ \tt * End Distance Sum: 23.16\\ \tt * Init Distance Sum: 23.67\\ \tt * Expected Distance Sum: 29.33\\ \tt * Penalties: 17.00\\ \tt * Max Penalties: 16.00\\ \tt * Rounds: 10.00\\ \tt * Max Rounds: 28.00\\ \tt * Object Count: 7.00\\ \tt  
	  } 
	   } 
	   } 
	 & & \\ 
 

    \theutterance \stepcounter{utterance}  

    & & & \multicolumn{2}{p{0.3\linewidth}}{\cellcolor[rgb]{0.95,0.95,0.95}{%
	\makecell[{{p{\linewidth}}}]{% 
	  \tt {\tiny [GM$|$GM]}  
	 * Main Score: 0.00\\ \tt \\ \tt -------\\ \tt * Distance Score: 0.12\\ \tt * Coverage Score: 0.09\\ \tt * Penalty Score: 0.00\\ \tt * Alternative Penalty Score: 0.00\\ \tt * Alternative Main Score: 0.00\\ \tt \\ \tt -------\\ \tt * Shifts: 2.00\\ \tt * Max Shifts: 12.00\\ \tt * Min Shifts: 6.00\\ \tt * End Distance Sum: 23.16\\ \tt * Init Distance Sum: 23.67\\ \tt * Expected Distance Sum: 29.33\\ \tt * Penalties: 17.00\\ \tt * Max Penalties: 16.00\\ \tt * Rounds: 10.00\\ \tt * Max Rounds: 28.00\\ \tt * Object Count: 7.00\\ \tt  
	  } 
	   } 
	   } 
	 & & \\ 
 

\end{supertabular}
}

\end{document}
