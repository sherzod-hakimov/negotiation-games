
\documentclass{article}
\usepackage{colortbl}
\usepackage{makecell}
\usepackage{multirow}
\usepackage{supertabular}

\begin{document}

\newcounter{utterance}

\centering \large Interaction Transcript for game `clean\_up', experiment `2\_hard\_5obj\_de', episode 0 with llama{-}3.3{-}70b{-}instruct{-}t1.0.
\vspace{24pt}

{ \footnotesize  \setcounter{utterance}{1}
\setlength{\tabcolsep}{0pt}
\begin{supertabular}{c@{$\;$}|p{.15\linewidth}@{}p{.15\linewidth}p{.15\linewidth}p{.15\linewidth}p{.15\linewidth}p{.15\linewidth}}
    \# & $\;$A & \multicolumn{4}{c}{Game Master} & $\;\:$B\\
    \hline

    \theutterance \stepcounter{utterance}  
    & & \multicolumn{4}{p{0.6\linewidth}}{
        \cellcolor[rgb]{0.9,0.9,0.9}{
            \makecell[{{p{\linewidth}}}]{
                \texttt{\tiny{[P1$\langle$GM]}}
                \texttt{Ich bin der Spielleiter, und du spielst ein kooperatives Spiel mit dem folgenden Raster als Spielbrett:} \\
\\ 
\texttt{```} \\
\texttt{ 1234567} \\
\texttt{╔══╤════╗} \\
\texttt{║◌◌│◌◌IW║ 1} \\
\texttt{╟──┘◌◌┌─╢ 2} \\
\texttt{║◌◌◌◌H│◌║ 3} \\
\texttt{╟─┐C┌─┼─╢ 4} \\
\texttt{║◌├─┤◌│T║ 5} \\
\texttt{╟─┤◌├─┤◌║ 6} \\
\texttt{║◌│◌│◌│◌║ 7} \\
\texttt{╚═╧═╧═╧═╝} \\
\\ 
\texttt{```} \\
\\ 
\texttt{* Am oberen Rand sind die x{-}Koordinaten angegeben, die nach rechts hin zunehmen, und am rechten Rand die y{-}Koordinaten, die nach unten zunehmen.} \\
\texttt{* Die folgenden Objekte sind zufällig auf deinem Raster platziert: 'W', 'I', 'T', 'C', 'H'.} \\
\\ 
\\ 
\texttt{Dein Mitspieler sieht eine Variation des Spielbretts, auf dem dieselben Objekte an anderen zufälligen Positionen platziert sind. Du kannst das Spielbrett des anderen Spielers nicht sehen, und er kann deins nicht sehen.} \\
\\ 
\\ 
\texttt{**Ziel des Spiels:**} \\
\\ 
\texttt{Beide Spieler müssen die Objekte auf ihrem jeweiligen Spielbrett so bewegen, dass identische Objekte am Ende an denselben Koordinaten liegen. Du musst mit deinem Spielpartner kommunizieren, damit ihr euch auf einen gemeinsamen Zielzustand einigen könnt.} \\
\\ 
\\ 
\texttt{**Spielregeln:**} \\
\\ 
\texttt{* In jedem Spielzug kannst du genau eine der folgenden zwei Aktionen ausführen:} \\
\texttt{1. `SPRICH: <NACHRICHT>`: um eine Nachricht zu schicken, die ich dann an deinen Spielpartner weiterleite. Als Nachricht gilt alles bis zum nächsten Zeilenumbruch.} \\
\texttt{2. `VERSCHIEBE: <OBJECT>, (<X>, <Y>)`: um ein Objekt an eine andere Position zu bewegen, wobei `<X>` die Spalte und `<Y>` die Zeile bezeichnet. Ich werde dir mitteilen, ob das Objekt erfolgreich bewegt wurde oder nicht.} \\
\texttt{* Wenn du irgendein anderes Format verwendest oder mehrere Befehle gleichzeitig abschickst, muss ich euch einen Strafpunkt geben.} \\
\texttt{* Wenn ihr gemeinsam mehr als 12 Strafpunkte ansammelt, verliert ihr das Spiel.} \\
\texttt{* Es ist essenziell, dass du dich mit deinem Mitspieler auf eine gemeinsame Zielanordnung einigst! Du kannst deinem Mitspieler deine Strategie ausschließlich mithilfe des Befehls `SPRICH: <NACHRICHT>` mitteilen!} \\
\\ 
\\ 
\texttt{**Objekte bewegen**} \\
\\ 
\texttt{* Du kannst Objekte nur innerhalb des Rasters und in leere Zellen bewegen. Leere Zellen erkennst am Zeichen '◌'.} \\
\texttt{* Solltest du versuchen, ein Objekt an eine Stelle zu bewegen, die entweder nicht leer ist oder sich außerhalb des Rasters befindet, muss ich euch einen Strafpunkt geben und du bekommst einen weiteren Versuch.} \\
\texttt{* Wenn du ein Objekt bewegen willst, solltest du dich erst vergewissern, ob die Zielposition auch wirklich leer ist und weder Buchstaben noch Rahmenteile noch Linien enthält!} \\
\\ 
\\ 
\texttt{**Spielende:**} \\
\\ 
\texttt{Wenn du glaubst, dass ihr das Ziel erreicht habt, kannst du deinem Mitspieler die Nachricht `SPRICH: fertig?` schicken. Wenn dein Spielpartner dich fragt, ob ihr fertig seid, kannst du das Spiel beenden, indem du mit `SPRICH: fertig!` antwortest.} \\
\\ 
\texttt{Ihr gewinnt gemeinsam, wenn ihr das Spiel innerhalb von \#max\_rounds Runden beendet. Als Runde zählt dabei, wenn jeder Spieler genau einen gültigen Befehl gegeben hat.} \\
\\ 
\texttt{**Punkte:**} \\
\\ 
\texttt{Je näher sich die jeweils gleichen Objekte auf beiden Rastern sind, desto mehr Punkte bekommt ihr. Für Strafpunkte gibt es Punktabzug. Könnt ihr den Rekord brechen?} \\
\\ 
\texttt{Schicke eine Nachricht an deinen Mitspieler um das Spiel zu starten!} \\
            }
        }
    }
    & \\ \\

    \theutterance \stepcounter{utterance}  
    & \multicolumn{4}{p{0.6\linewidth}}{
        \cellcolor[rgb]{0.9,0.9,0.9}{
            \makecell[{{p{\linewidth}}}]{
                \texttt{\tiny{[P1$\rangle$GM]}}
                \texttt{SPRICH: Hallo, ich schlage vor, dass wir uns auf eine feste Position für jedes Objekt einigen. Wie wäre es, wenn wir das 'W' in der Mitte platzieren, also bei (4,4)? Dann könnten wir die anderen Objekte um das 'W' herum anordnen. Was meinst du?} \\
            }
        }
    }
    & & \\ \\

    \theutterance \stepcounter{utterance}  
    & & & \multicolumn{2}{p{0.3\linewidth}}{
        \cellcolor[rgb]{0.9,0.9,0.9}{
            \makecell[{{p{\linewidth}}}]{
                \texttt{\tiny{[GM$|$GM]}}
                \texttt{SPRICH: Hallo, ich schlage vor, dass wir uns auf eine feste Position für jedes Objekt einigen. Wie wäre es, wenn wir das 'W' in der Mitte platzieren, also bei (4,4)? Dann könnten wir die anderen Objekte um das 'W' herum anordnen. Was meinst du?} \\
            }
        }
    }
    & & \\ \\

    \theutterance \stepcounter{utterance}  
    & & \multicolumn{4}{p{0.6\linewidth}}{
        \cellcolor[rgb]{0.9,0.9,0.9}{
            \makecell[{{p{\linewidth}}}]{
                \texttt{\tiny{[GM$\rangle$P2]}}
                \texttt{Ich bin der Spielleiter, und du spielst ein kooperatives Spiel mit dem folgenden Raster als Spielbrett:} \\
\\ 
\texttt{```} \\
\texttt{ 1234567} \\
\texttt{╔══╤════╗} \\
\texttt{║◌◌│◌◌◌◌║ 1} \\
\texttt{╟──┘◌T┌─╢ 2} \\
\texttt{║◌◌HW◌│◌║ 3} \\
\texttt{╟─┐◌┌─┼─╢ 4} \\
\texttt{║◌├─┤◌│◌║ 5} \\
\texttt{╟─┤◌├─┤◌║ 6} \\
\texttt{║◌│I│C│◌║ 7} \\
\texttt{╚═╧═╧═╧═╝} \\
\\ 
\texttt{```} \\
\\ 
\texttt{* Am oberen Rand sind die x{-}Koordinaten angegeben, die nach rechts hin zunehmen, und am rechten Rand die y{-}Koordinaten, die nach unten zunehmen.} \\
\texttt{* Die folgenden Objekte sind zufällig auf deinem Raster platziert: 'W', 'I', 'T', 'C', 'H'.} \\
\\ 
\\ 
\texttt{Dein Mitspieler sieht eine Variation des Spielbretts, auf dem dieselben Objekte an anderen zufälligen Positionen platziert sind. Du kannst das Spielbrett des anderen Spielers nicht sehen, und er kann deins nicht sehen.} \\
\\ 
\\ 
\texttt{**Ziel des Spiels:**} \\
\\ 
\texttt{Beide Spieler müssen die Objekte auf ihrem jeweiligen Spielbrett so bewegen, dass identische Objekte am Ende an denselben Koordinaten liegen. Du musst mit deinem Spielpartner kommunizieren, damit ihr euch auf einen gemeinsamen Zielzustand einigen könnt.} \\
\\ 
\\ 
\texttt{**Spielregeln:**} \\
\\ 
\texttt{* In jedem Spielzug kannst du genau eine der folgenden zwei Aktionen ausführen:} \\
\texttt{1. `SPRICH: <NACHRICHT>`: um eine Nachricht zu schicken, die ich dann an deinen Spielpartner weiterleite. Als Nachricht gilt alles bis zum nächsten Zeilenumbruch.} \\
\texttt{2. `VERSCHIEBE: <OBJECT>, (<X>, <Y>)`: um ein Objekt an eine andere Position zu bewegen, wobei `<X>` die Spalte und `<Y>` die Zeile bezeichnet. Ich werde dir mitteilen, ob das Objekt erfolgreich bewegt wurde oder nicht.} \\
\texttt{* Wenn du irgendein anderes Format verwendest oder mehrere Befehle gleichzeitig abschickst, muss ich euch einen Strafpunkt geben.} \\
\texttt{* Wenn ihr gemeinsam mehr als 12 Strafpunkte ansammelt, verliert ihr das Spiel.} \\
\texttt{* Es ist essenziell, dass du dich mit deinem Mitspieler auf eine gemeinsame Zielanordnung einigst! Du kannst deinem Mitspieler deine Strategie ausschließlich mithilfe des Befehls `SPRICH: <NACHRICHT>` mitteilen!} \\
\\ 
\\ 
\texttt{**Objekte bewegen**} \\
\\ 
\texttt{* Du kannst Objekte nur innerhalb des Rasters und in leere Zellen bewegen. Leere Zellen erkennst am Zeichen '◌'.} \\
\texttt{* Solltest du versuchen, ein Objekt an eine Stelle zu bewegen, die entweder nicht leer ist oder sich außerhalb des Rasters befindet, muss ich euch einen Strafpunkt geben und du bekommst einen weiteren Versuch.} \\
\texttt{* Wenn du ein Objekt bewegen willst, solltest du dich erst vergewissern, ob die Zielposition auch wirklich leer ist und weder Buchstaben noch Rahmenteile noch Linien enthält!} \\
\\ 
\\ 
\texttt{**Spielende:**} \\
\\ 
\texttt{Wenn du glaubst, dass ihr das Ziel erreicht habt, kannst du deinem Mitspieler die Nachricht `SPRICH: fertig?` schicken. Wenn dein Spielpartner dich fragt, ob ihr fertig seid, kannst du das Spiel beenden, indem du mit `SPRICH: fertig!` antwortest.} \\
\\ 
\texttt{Ihr gewinnt gemeinsam, wenn ihr das Spiel innerhalb von \#max\_rounds Runden beendet. Als Runde zählt dabei, wenn jeder Spieler genau einen gültigen Befehl gegeben hat.} \\
\\ 
\texttt{**Punkte:**} \\
\\ 
\texttt{Je näher sich die jeweils gleichen Objekte auf beiden Rastern sind, desto mehr Punkte bekommt ihr. Für Strafpunkte gibt es Punktabzug. Könnt ihr den Rekord brechen?} \\
\\ 
\texttt{Dein Mitspieler hat das Spiel mit dieser Nachricht begonnen:} \\
\\ 
\texttt{"Hallo, ich schlage vor, dass wir uns auf eine feste Position für jedes Objekt einigen. Wie wäre es, wenn wir das 'W' in der Mitte platzieren, also bei (4,4)? Dann könnten wir die anderen Objekte um das 'W' herum anordnen. Was meinst du?"} \\
\\ 
\texttt{Wie lautet dein erster Befehl?} \\
            }
        }
    }
    & \\ \\

    \theutterance \stepcounter{utterance}  
    & & & \multicolumn{4}{p{0.6\linewidth}}{
        \cellcolor[rgb]{0.9,0.9,0.9}{
            \makecell[{{p{\linewidth}}}]{
                \texttt{\tiny{[GM$\langle$P2]}}
                \texttt{Ich denke, das ist eine gute Idee, das 'W' in der Mitte zu platzieren. Ich schlage vor, dass wir das 'T' oben links und das 'I' unten rechts platzieren. Das 'C' und das 'H' könnten dann jeweils oben rechts und unten links platziert werden. } \\
\\ 
\texttt{Ich werde also antworten: `SPRICH: Ja, das klingt gut! Ich schlage vor, wir platzieren das 'T' bei (1,1), das 'I' bei (7,7), das 'C' bei (7,1) und das 'H' bei (1,7). Was meinst du?`} \\
            }
        }
    }
     \\ \\

    \theutterance \stepcounter{utterance}  
    & & & \multicolumn{2}{p{0.3\linewidth}}{
        \cellcolor[rgb]{0.9,0.9,0.9}{
            \makecell[{{p{\linewidth}}}]{
                \texttt{\tiny{[GM$|$GM]}}
                \texttt{Ich denke, das ist eine gute Idee, das 'W' in der Mitte zu platzieren. Ich schlage vor, dass wir das 'T' oben links und das 'I' unten rechts platzieren. Das 'C' und das 'H' könnten dann jeweils oben rechts und unten links platziert werden. } \\
\\ 
\texttt{Ich werde also antworten: `SPRICH: Ja, das klingt gut! Ich schlage vor, wir platzieren das 'T' bei (1,1), das 'I' bei (7,7), das 'C' bei (7,1) und das 'H' bei (1,7). Was meinst du?`} \\
            }
        }
    }
    & & \\ \\

    \theutterance \stepcounter{utterance}  
    & & & \multicolumn{2}{p{0.3\linewidth}}{
        \cellcolor[rgb]{0.9,0.9,0.9}{
            \makecell[{{p{\linewidth}}}]{
                \texttt{\tiny{[GM$|$GM]}}
                \texttt{Invalid format: head is not empty: 'Ich denke, das ist eine gute Idee, das 'W' in der Mitte zu platzieren. Ich schlage vor, dass wir das 'T' oben links und das 'I' unten rechts platzieren. Das 'C' und das 'H' könnten dann jeweils oben rechts und unten links platziert werden. } \\
\\ 
\texttt{Ich werde also antworten: '} \\
            }
        }
    }
    & & \\ \\

    \theutterance \stepcounter{utterance}  
    & & \multicolumn{4}{p{0.6\linewidth}}{
        \cellcolor[rgb]{0.9,0.9,0.9}{
            \makecell[{{p{\linewidth}}}]{
                \texttt{\tiny{[GM$\rangle$P2]}}
                \texttt{Strafpunkt: Deine Nachricht darf nichts vor dem Befehl enthalten! Stelle sicher, dass deine Antwort ausschließlich einen der beiden Befehle `SPRICH: <NACHRICHT>` und `VERSCHIEBE: <OBJECT>, (<X>, <Y>)` und sonst nichts enthält!} \\
\\ 
\texttt{Ihr habt gemeinsam 1 von maximal 12 Strafpunkten angesammelt.} \\
\texttt{Bitte versuche es noch mal!} \\
            }
        }
    }
    & \\ \\

    \theutterance \stepcounter{utterance}  
    & & & \multicolumn{4}{p{0.6\linewidth}}{
        \cellcolor[rgb]{0.9,0.9,0.9}{
            \makecell[{{p{\linewidth}}}]{
                \texttt{\tiny{[GM$\langle$P2]}}
                \texttt{SPRICH: Ja, das klingt gut! Ich schlage vor, wir platzieren das 'T' bei (1,1), das 'I' bei (7,7), das 'C' bei (7,1) und das 'H' bei (1,7). Was meinst du?} \\
            }
        }
    }
     \\ \\

    \theutterance \stepcounter{utterance}  
    & & & \multicolumn{2}{p{0.3\linewidth}}{
        \cellcolor[rgb]{0.9,0.9,0.9}{
            \makecell[{{p{\linewidth}}}]{
                \texttt{\tiny{[GM$|$GM]}}
                \texttt{SPRICH: Ja, das klingt gut! Ich schlage vor, wir platzieren das 'T' bei (1,1), das 'I' bei (7,7), das 'C' bei (7,1) und das 'H' bei (1,7). Was meinst du?} \\
            }
        }
    }
    & & \\ \\

    \theutterance \stepcounter{utterance}  
    & & \multicolumn{4}{p{0.6\linewidth}}{
        \cellcolor[rgb]{0.9,0.9,0.9}{
            \makecell[{{p{\linewidth}}}]{
                \texttt{\tiny{[P1$\langle$GM]}}
                \texttt{Deine Nachricht wurde an deinen Mitspieler übermittelt.} \\
\\ 
\\ 
\texttt{Ihr seid gerade bei Runde 1 von maximal 20.} \\
\texttt{Ihr habt gemeinsam 1 von maximal 12 Strafpunkten angesammelt.} \\
\texttt{Dein Mitspieler hat dir diese Nachricht geschickt:} \\
\\ 
\\ 
\texttt{"Ja, das klingt gut! Ich schlage vor, wir platzieren das 'T' bei (1,1), das 'I' bei (7,7), das 'C' bei (7,1) und das 'H' bei (1,7). Was meinst du?"} \\
\\ 
\\ 
\texttt{Wie lautet dein nächster Befehl?} \\
            }
        }
    }
    & \\ \\

    \theutterance \stepcounter{utterance}  
    & \multicolumn{4}{p{0.6\linewidth}}{
        \cellcolor[rgb]{0.9,0.9,0.9}{
            \makecell[{{p{\linewidth}}}]{
                \texttt{\tiny{[P1$\rangle$GM]}}
                \texttt{SPRICH: Das klingt nach einem guten Plan! Ich stimme zu. Bevor wir beginnen, die Objekte zu bewegen, möchte ich mich vergewissern, dass wir dieselbe Vorstellung von der Mitte haben. Für mich wäre die Mitte (4,4) das Feld, wo gerade das 'C' steht. Sollten wir das 'W' also dort platzieren, müssten wir das 'C' vorher verschieben. Ich denke, es wäre hilfreich, wenn wir unsere Startpositionen比较 und sicherstellen, dass wir dieselben Objektpositionen vor uns haben, bevor wir bewegen. Kannst du mir deine aktuelle Startposition beschreiben?} \\
            }
        }
    }
    & & \\ \\

    \theutterance \stepcounter{utterance}  
    & & & \multicolumn{2}{p{0.3\linewidth}}{
        \cellcolor[rgb]{0.9,0.9,0.9}{
            \makecell[{{p{\linewidth}}}]{
                \texttt{\tiny{[GM$|$GM]}}
                \texttt{SPRICH: Das klingt nach einem guten Plan! Ich stimme zu. Bevor wir beginnen, die Objekte zu bewegen, möchte ich mich vergewissern, dass wir dieselbe Vorstellung von der Mitte haben. Für mich wäre die Mitte (4,4) das Feld, wo gerade das 'C' steht. Sollten wir das 'W' also dort platzieren, müssten wir das 'C' vorher verschieben. Ich denke, es wäre hilfreich, wenn wir unsere Startpositionen比较 und sicherstellen, dass wir dieselben Objektpositionen vor uns haben, bevor wir bewegen. Kannst du mir deine aktuelle Startposition beschreiben?} \\
            }
        }
    }
    & & \\ \\

    \theutterance \stepcounter{utterance}  
    & & \multicolumn{4}{p{0.6\linewidth}}{
        \cellcolor[rgb]{0.9,0.9,0.9}{
            \makecell[{{p{\linewidth}}}]{
                \texttt{\tiny{[GM$\rangle$P2]}}
                \texttt{Deine Nachricht wurde an deinen Mitspieler übermittelt.} \\
\\ 
\\ 
\texttt{Ihr seid gerade bei Runde 2 von maximal 20.} \\
\texttt{Ihr habt gemeinsam 1 von maximal 12 Strafpunkten angesammelt.} \\
\texttt{Dein Mitspieler hat dir diese Nachricht geschickt:} \\
\\ 
\\ 
\texttt{"Das klingt nach einem guten Plan! Ich stimme zu. Bevor wir beginnen, die Objekte zu bewegen, möchte ich mich vergewissern, dass wir dieselbe Vorstellung von der Mitte haben. Für mich wäre die Mitte (4,4) das Feld, wo gerade das 'C' steht. Sollten wir das 'W' also dort platzieren, müssten wir das 'C' vorher verschieben. Ich denke, es wäre hilfreich, wenn wir unsere Startpositionen比较 und sicherstellen, dass wir dieselben Objektpositionen vor uns haben, bevor wir bewegen. Kannst du mir deine aktuelle Startposition beschreiben?"} \\
\\ 
\\ 
\texttt{Wie lautet dein nächster Befehl?} \\
            }
        }
    }
    & \\ \\

    \theutterance \stepcounter{utterance}  
    & & & \multicolumn{4}{p{0.6\linewidth}}{
        \cellcolor[rgb]{0.9,0.9,0.9}{
            \makecell[{{p{\linewidth}}}]{
                \texttt{\tiny{[GM$\langle$P2]}}
                \texttt{SPRICH: Meine Startposition sieht so aus: Das 'T' ist bei (2,2), das 'W' bei (3,3), das 'H' bei (3,2), das 'I' bei (7,3) und das 'C' bei (5,3). Ich denke, wir sollten unsere Objekte so anpassen, dass sie deiner Beschreibung entsprechen, also das 'W' bei (4,4) und die anderen Objekte entsprechend um das 'W' herum. Soll ich das 'W' zu (4,4) und die anderen Objekte zu ihren entsprechenden Positionen bewegen?} \\
            }
        }
    }
     \\ \\

    \theutterance \stepcounter{utterance}  
    & & & \multicolumn{2}{p{0.3\linewidth}}{
        \cellcolor[rgb]{0.9,0.9,0.9}{
            \makecell[{{p{\linewidth}}}]{
                \texttt{\tiny{[GM$|$GM]}}
                \texttt{SPRICH: Meine Startposition sieht so aus: Das 'T' ist bei (2,2), das 'W' bei (3,3), das 'H' bei (3,2), das 'I' bei (7,3) und das 'C' bei (5,3). Ich denke, wir sollten unsere Objekte so anpassen, dass sie deiner Beschreibung entsprechen, also das 'W' bei (4,4) und die anderen Objekte entsprechend um das 'W' herum. Soll ich das 'W' zu (4,4) und die anderen Objekte zu ihren entsprechenden Positionen bewegen?} \\
            }
        }
    }
    & & \\ \\

    \theutterance \stepcounter{utterance}  
    & & \multicolumn{4}{p{0.6\linewidth}}{
        \cellcolor[rgb]{0.9,0.9,0.9}{
            \makecell[{{p{\linewidth}}}]{
                \texttt{\tiny{[P1$\langle$GM]}}
                \texttt{Deine Nachricht wurde an deinen Mitspieler übermittelt.} \\
\\ 
\\ 
\texttt{Ihr seid gerade bei Runde 2 von maximal 20.} \\
\texttt{Ihr habt gemeinsam 1 von maximal 12 Strafpunkten angesammelt.} \\
\texttt{Dein Mitspieler hat dir diese Nachricht geschickt:} \\
\\ 
\\ 
\texttt{"Meine Startposition sieht so aus: Das 'T' ist bei (2,2), das 'W' bei (3,3), das 'H' bei (3,2), das 'I' bei (7,3) und das 'C' bei (5,3). Ich denke, wir sollten unsere Objekte so anpassen, dass sie deiner Beschreibung entsprechen, also das 'W' bei (4,4) und die anderen Objekte entsprechend um das 'W' herum. Soll ich das 'W' zu (4,4) und die anderen Objekte zu ihren entsprechenden Positionen bewegen?"} \\
\\ 
\\ 
\texttt{Wie lautet dein nächster Befehl?} \\
            }
        }
    }
    & \\ \\

    \theutterance \stepcounter{utterance}  
    & \multicolumn{4}{p{0.6\linewidth}}{
        \cellcolor[rgb]{0.9,0.9,0.9}{
            \makecell[{{p{\linewidth}}}]{
                \texttt{\tiny{[P1$\rangle$GM]}}
                \texttt{SPRICH: Das hilft sehr! Meine Startposition ist ein bisschen anders, aber ich denke, wir können unsere Strategie anpassen. Mein 'W' ist Actually bei (1,1), mein 'I' ist bei (1,3), mein 'T' ist bei (5,5), mein 'C' ist bei (4,4) und mein 'H' ist bei (3,3). Ich denke, es wäre am besten, wenn wir unsere Objekte so bewegen, dass sie deiner Beschreibung entsprechen, also das 'W' bei (4,4), das 'T' bei (1,1), das 'I' bei (7,7), das 'C' bei (7,1) und das 'H' bei (1,7). Lass uns gemeinsam unsere Objekte bewegen, um dies zu erreichen. Wir sollten uns abstimmen, welches Objekt wir zuerst bewegen, um sicherzustellen, dass wir nicht in Konflikt geraten. Sollten wir mit dem 'W' beginnen, da es für beide von uns eine zentrale Rolle spielt?} \\
            }
        }
    }
    & & \\ \\

    \theutterance \stepcounter{utterance}  
    & & & \multicolumn{2}{p{0.3\linewidth}}{
        \cellcolor[rgb]{0.9,0.9,0.9}{
            \makecell[{{p{\linewidth}}}]{
                \texttt{\tiny{[GM$|$GM]}}
                \texttt{SPRICH: Das hilft sehr! Meine Startposition ist ein bisschen anders, aber ich denke, wir können unsere Strategie anpassen. Mein 'W' ist Actually bei (1,1), mein 'I' ist bei (1,3), mein 'T' ist bei (5,5), mein 'C' ist bei (4,4) und mein 'H' ist bei (3,3). Ich denke, es wäre am besten, wenn wir unsere Objekte so bewegen, dass sie deiner Beschreibung entsprechen, also das 'W' bei (4,4), das 'T' bei (1,1), das 'I' bei (7,7), das 'C' bei (7,1) und das 'H' bei (1,7). Lass uns gemeinsam unsere Objekte bewegen, um dies zu erreichen. Wir sollten uns abstimmen, welches Objekt wir zuerst bewegen, um sicherzustellen, dass wir nicht in Konflikt geraten. Sollten wir mit dem 'W' beginnen, da es für beide von uns eine zentrale Rolle spielt?} \\
            }
        }
    }
    & & \\ \\

    \theutterance \stepcounter{utterance}  
    & & \multicolumn{4}{p{0.6\linewidth}}{
        \cellcolor[rgb]{0.9,0.9,0.9}{
            \makecell[{{p{\linewidth}}}]{
                \texttt{\tiny{[GM$\rangle$P2]}}
                \texttt{Deine Nachricht wurde an deinen Mitspieler übermittelt.} \\
\\ 
\\ 
\texttt{Ihr seid gerade bei Runde 3 von maximal 20.} \\
\texttt{Ihr habt gemeinsam 1 von maximal 12 Strafpunkten angesammelt.} \\
\texttt{Dein Mitspieler hat dir diese Nachricht geschickt:} \\
\\ 
\\ 
\texttt{"Das hilft sehr! Meine Startposition ist ein bisschen anders, aber ich denke, wir können unsere Strategie anpassen. Mein 'W' ist Actually bei (1,1), mein 'I' ist bei (1,3), mein 'T' ist bei (5,5), mein 'C' ist bei (4,4) und mein 'H' ist bei (3,3). Ich denke, es wäre am besten, wenn wir unsere Objekte so bewegen, dass sie deiner Beschreibung entsprechen, also das 'W' bei (4,4), das 'T' bei (1,1), das 'I' bei (7,7), das 'C' bei (7,1) und das 'H' bei (1,7). Lass uns gemeinsam unsere Objekte bewegen, um dies zu erreichen. Wir sollten uns abstimmen, welches Objekt wir zuerst bewegen, um sicherzustellen, dass wir nicht in Konflikt geraten. Sollten wir mit dem 'W' beginnen, da es für beide von uns eine zentrale Rolle spielt?"} \\
\\ 
\\ 
\texttt{Wie lautet dein nächster Befehl?} \\
            }
        }
    }
    & \\ \\

    \theutterance \stepcounter{utterance}  
    & & & \multicolumn{4}{p{0.6\linewidth}}{
        \cellcolor[rgb]{0.9,0.9,0.9}{
            \makecell[{{p{\linewidth}}}]{
                \texttt{\tiny{[GM$\langle$P2]}}
                \texttt{SPRICH: Ja, das klingt nach einem guten Plan! Ich denke, es wäre sinnvoll, mit dem 'W' zu beginnen, da es wie du sagst eine zentrale Rolle spielt. Ich werde mein 'W' von (3,3) nach (4,4) bewegen. Bitte beachte, dass ich dies in meinem nächsten Zug tun werde, nachdem du bestätigt hast, dass du bereit bist. Bitte bestätige, dass du bereit bist, bevor ich den Zug ausführe.} \\
            }
        }
    }
     \\ \\

    \theutterance \stepcounter{utterance}  
    & & & \multicolumn{2}{p{0.3\linewidth}}{
        \cellcolor[rgb]{0.9,0.9,0.9}{
            \makecell[{{p{\linewidth}}}]{
                \texttt{\tiny{[GM$|$GM]}}
                \texttt{SPRICH: Ja, das klingt nach einem guten Plan! Ich denke, es wäre sinnvoll, mit dem 'W' zu beginnen, da es wie du sagst eine zentrale Rolle spielt. Ich werde mein 'W' von (3,3) nach (4,4) bewegen. Bitte beachte, dass ich dies in meinem nächsten Zug tun werde, nachdem du bestätigt hast, dass du bereit bist. Bitte bestätige, dass du bereit bist, bevor ich den Zug ausführe.} \\
            }
        }
    }
    & & \\ \\

    \theutterance \stepcounter{utterance}  
    & & \multicolumn{4}{p{0.6\linewidth}}{
        \cellcolor[rgb]{0.9,0.9,0.9}{
            \makecell[{{p{\linewidth}}}]{
                \texttt{\tiny{[P1$\langle$GM]}}
                \texttt{Deine Nachricht wurde an deinen Mitspieler übermittelt.} \\
\\ 
\\ 
\texttt{Ihr seid gerade bei Runde 3 von maximal 20.} \\
\texttt{Ihr habt gemeinsam 1 von maximal 12 Strafpunkten angesammelt.} \\
\texttt{Dein Mitspieler hat dir diese Nachricht geschickt:} \\
\\ 
\\ 
\texttt{"Ja, das klingt nach einem guten Plan! Ich denke, es wäre sinnvoll, mit dem 'W' zu beginnen, da es wie du sagst eine zentrale Rolle spielt. Ich werde mein 'W' von (3,3) nach (4,4) bewegen. Bitte beachte, dass ich dies in meinem nächsten Zug tun werde, nachdem du bestätigt hast, dass du bereit bist. Bitte bestätige, dass du bereit bist, bevor ich den Zug ausführe."} \\
\\ 
\\ 
\texttt{Wie lautet dein nächster Befehl?} \\
            }
        }
    }
    & \\ \\

    \theutterance \stepcounter{utterance}  
    & \multicolumn{4}{p{0.6\linewidth}}{
        \cellcolor[rgb]{0.9,0.9,0.9}{
            \makecell[{{p{\linewidth}}}]{
                \texttt{\tiny{[P1$\rangle$GM]}}
                \texttt{SPRICH: Ja, ich bin bereit! Bitte geh vor und bewege dein 'W' von (3,3) nach (4,4). Ich werde mein 'W' gleichzeitig von (1,1) nach (4,4) bewegen. Lass uns synchron unsere Züge ausführen, um sicherzustellen, dass wir auf dem gleichen Weg sind. Bitte füge deinen Zug hinzu, sobald du bereit bist!} \\
            }
        }
    }
    & & \\ \\

    \theutterance \stepcounter{utterance}  
    & & & \multicolumn{2}{p{0.3\linewidth}}{
        \cellcolor[rgb]{0.9,0.9,0.9}{
            \makecell[{{p{\linewidth}}}]{
                \texttt{\tiny{[GM$|$GM]}}
                \texttt{SPRICH: Ja, ich bin bereit! Bitte geh vor und bewege dein 'W' von (3,3) nach (4,4). Ich werde mein 'W' gleichzeitig von (1,1) nach (4,4) bewegen. Lass uns synchron unsere Züge ausführen, um sicherzustellen, dass wir auf dem gleichen Weg sind. Bitte füge deinen Zug hinzu, sobald du bereit bist!} \\
            }
        }
    }
    & & \\ \\

    \theutterance \stepcounter{utterance}  
    & & \multicolumn{4}{p{0.6\linewidth}}{
        \cellcolor[rgb]{0.9,0.9,0.9}{
            \makecell[{{p{\linewidth}}}]{
                \texttt{\tiny{[GM$\rangle$P2]}}
                \texttt{Deine Nachricht wurde an deinen Mitspieler übermittelt.} \\
\\ 
\\ 
\texttt{Ihr seid gerade bei Runde 4 von maximal 20.} \\
\texttt{Ihr habt gemeinsam 1 von maximal 12 Strafpunkten angesammelt.} \\
\texttt{Dein Mitspieler hat dir diese Nachricht geschickt:} \\
\\ 
\\ 
\texttt{"Ja, ich bin bereit! Bitte geh vor und bewege dein 'W' von (3,3) nach (4,4). Ich werde mein 'W' gleichzeitig von (1,1) nach (4,4) bewegen. Lass uns synchron unsere Züge ausführen, um sicherzustellen, dass wir auf dem gleichen Weg sind. Bitte füge deinen Zug hinzu, sobald du bereit bist!"} \\
\\ 
\\ 
\texttt{Wie lautet dein nächster Befehl?} \\
            }
        }
    }
    & \\ \\

    \theutterance \stepcounter{utterance}  
    & & & \multicolumn{4}{p{0.6\linewidth}}{
        \cellcolor[rgb]{0.9,0.9,0.9}{
            \makecell[{{p{\linewidth}}}]{
                \texttt{\tiny{[GM$\langle$P2]}}
                \texttt{VERSCHIEBE: W, (4,4)} \\
            }
        }
    }
     \\ \\

    \theutterance \stepcounter{utterance}  
    & & & \multicolumn{2}{p{0.3\linewidth}}{
        \cellcolor[rgb]{0.9,0.9,0.9}{
            \makecell[{{p{\linewidth}}}]{
                \texttt{\tiny{[GM$|$GM]}}
                \texttt{VERSCHIEBE: W, (4,4)} \\
            }
        }
    }
    & & \\ \\

    \theutterance \stepcounter{utterance}  
    & & & \multicolumn{2}{p{0.3\linewidth}}{
        \cellcolor[rgb]{0.9,0.9,0.9}{
            \makecell[{{p{\linewidth}}}]{
                \texttt{\tiny{[GM$|$GM]}}
                \texttt{Strafpunkt: (4,4) ist nicht leer, sondern enthält '┌'.} \\
\\ 
\texttt{Ihr habt gemeinsam 2 von maximal 12 Strafpunkten angesammelt.} \\
\texttt{Bitte versuche es noch mal!} \\
            }
        }
    }
    & & \\ \\

    \theutterance \stepcounter{utterance}  
    & & \multicolumn{4}{p{0.6\linewidth}}{
        \cellcolor[rgb]{0.9,0.9,0.9}{
            \makecell[{{p{\linewidth}}}]{
                \texttt{\tiny{[GM$\rangle$P2]}}
                \texttt{Strafpunkt: (4,4) ist nicht leer, sondern enthält '┌'.} \\
\\ 
\texttt{Ihr habt gemeinsam 2 von maximal 12 Strafpunkten angesammelt.} \\
\texttt{Bitte versuche es noch mal!} \\
            }
        }
    }
    & \\ \\

    \theutterance \stepcounter{utterance}  
    & & & \multicolumn{4}{p{0.6\linewidth}}{
        \cellcolor[rgb]{0.9,0.9,0.9}{
            \makecell[{{p{\linewidth}}}]{
                \texttt{\tiny{[GM$\langle$P2]}}
                \texttt{SPRICH: Es sieht so aus, als ob (4,4) nicht leer ist. Mein 'C' ist dort. Ich denke, ich sollte das 'C' zuerst verschieben, bevor ich das 'W' dahin bewege. Was meinst du?} \\
            }
        }
    }
     \\ \\

    \theutterance \stepcounter{utterance}  
    & & & \multicolumn{2}{p{0.3\linewidth}}{
        \cellcolor[rgb]{0.9,0.9,0.9}{
            \makecell[{{p{\linewidth}}}]{
                \texttt{\tiny{[GM$|$GM]}}
                \texttt{SPRICH: Es sieht so aus, als ob (4,4) nicht leer ist. Mein 'C' ist dort. Ich denke, ich sollte das 'C' zuerst verschieben, bevor ich das 'W' dahin bewege. Was meinst du?} \\
            }
        }
    }
    & & \\ \\

    \theutterance \stepcounter{utterance}  
    & & \multicolumn{4}{p{0.6\linewidth}}{
        \cellcolor[rgb]{0.9,0.9,0.9}{
            \makecell[{{p{\linewidth}}}]{
                \texttt{\tiny{[P1$\langle$GM]}}
                \texttt{Deine Nachricht wurde an deinen Mitspieler übermittelt.} \\
\\ 
\\ 
\texttt{Ihr seid gerade bei Runde 4 von maximal 20.} \\
\texttt{Ihr habt gemeinsam 2 von maximal 12 Strafpunkten angesammelt.} \\
\texttt{Dein Mitspieler hat dir diese Nachricht geschickt:} \\
\\ 
\\ 
\texttt{"Es sieht so aus, als ob (4,4) nicht leer ist. Mein 'C' ist dort. Ich denke, ich sollte das 'C' zuerst verschieben, bevor ich das 'W' dahin bewege. Was meinst du?"} \\
\\ 
\\ 
\texttt{Wie lautet dein nächster Befehl?} \\
            }
        }
    }
    & \\ \\

    \theutterance \stepcounter{utterance}  
    & \multicolumn{4}{p{0.6\linewidth}}{
        \cellcolor[rgb]{0.9,0.9,0.9}{
            \makecell[{{p{\linewidth}}}]{
                \texttt{\tiny{[P1$\rangle$GM]}}
                \texttt{SPRICH: Ah, das ist ein guter Punkt! Ich hatte nicht bemerkt, dass dein 'C' bei (4,4) ist. Ja, es wäre besser, wenn du das 'C' zuerst verschieben würdest, bevor du das 'W' dahin bewegst. Wie wäre es, wenn du das 'C' zu (7,1) bewegen würdest, wie ursprünglich geplant? Dann kannst du dein 'W' nach (4,4) bewegen. Ich werde mein 'W' von (1,1) nach (4,4) bewegen, sobald du bestätigt hast, dass dein 'C' verschoben wurde. Lass mich wissen, wenn du bereit bist, den Zug auszuführen!} \\
            }
        }
    }
    & & \\ \\

    \theutterance \stepcounter{utterance}  
    & & & \multicolumn{2}{p{0.3\linewidth}}{
        \cellcolor[rgb]{0.9,0.9,0.9}{
            \makecell[{{p{\linewidth}}}]{
                \texttt{\tiny{[GM$|$GM]}}
                \texttt{SPRICH: Ah, das ist ein guter Punkt! Ich hatte nicht bemerkt, dass dein 'C' bei (4,4) ist. Ja, es wäre besser, wenn du das 'C' zuerst verschieben würdest, bevor du das 'W' dahin bewegst. Wie wäre es, wenn du das 'C' zu (7,1) bewegen würdest, wie ursprünglich geplant? Dann kannst du dein 'W' nach (4,4) bewegen. Ich werde mein 'W' von (1,1) nach (4,4) bewegen, sobald du bestätigt hast, dass dein 'C' verschoben wurde. Lass mich wissen, wenn du bereit bist, den Zug auszuführen!} \\
            }
        }
    }
    & & \\ \\

    \theutterance \stepcounter{utterance}  
    & & \multicolumn{4}{p{0.6\linewidth}}{
        \cellcolor[rgb]{0.9,0.9,0.9}{
            \makecell[{{p{\linewidth}}}]{
                \texttt{\tiny{[GM$\rangle$P2]}}
                \texttt{Deine Nachricht wurde an deinen Mitspieler übermittelt.} \\
\\ 
\\ 
\texttt{Ihr seid gerade bei Runde 5 von maximal 20.} \\
\texttt{Ihr habt gemeinsam 2 von maximal 12 Strafpunkten angesammelt.} \\
\texttt{Dein Mitspieler hat dir diese Nachricht geschickt:} \\
\\ 
\\ 
\texttt{"Ah, das ist ein guter Punkt! Ich hatte nicht bemerkt, dass dein 'C' bei (4,4) ist. Ja, es wäre besser, wenn du das 'C' zuerst verschieben würdest, bevor du das 'W' dahin bewegst. Wie wäre es, wenn du das 'C' zu (7,1) bewegen würdest, wie ursprünglich geplant? Dann kannst du dein 'W' nach (4,4) bewegen. Ich werde mein 'W' von (1,1) nach (4,4) bewegen, sobald du bestätigt hast, dass dein 'C' verschoben wurde. Lass mich wissen, wenn du bereit bist, den Zug auszuführen!"} \\
\\ 
\\ 
\texttt{Wie lautet dein nächster Befehl?} \\
            }
        }
    }
    & \\ \\

    \theutterance \stepcounter{utterance}  
    & & & \multicolumn{4}{p{0.6\linewidth}}{
        \cellcolor[rgb]{0.9,0.9,0.9}{
            \makecell[{{p{\linewidth}}}]{
                \texttt{\tiny{[GM$\langle$P2]}}
                \texttt{VERSCHIEBE: C, (7,1)} \\
            }
        }
    }
     \\ \\

    \theutterance \stepcounter{utterance}  
    & & & \multicolumn{2}{p{0.3\linewidth}}{
        \cellcolor[rgb]{0.9,0.9,0.9}{
            \makecell[{{p{\linewidth}}}]{
                \texttt{\tiny{[GM$|$GM]}}
                \texttt{VERSCHIEBE: C, (7,1)} \\
            }
        }
    }
    & & \\ \\

    \theutterance \stepcounter{utterance}  
    & & & \multicolumn{2}{p{0.3\linewidth}}{
        \cellcolor[rgb]{0.9,0.9,0.9}{
            \makecell[{{p{\linewidth}}}]{
                \texttt{\tiny{[GM$|$GM]}}
                \texttt{'C' erfolgreich nach (7,1) verschoben. Dein Raster sieht jetzt so aus:} \\
\texttt{```} \\
\texttt{ 1234567} \\
\texttt{╔══╤════╗} \\
\texttt{║◌◌│◌◌◌C║ 1} \\
\texttt{╟──┘◌T┌─╢ 2} \\
\texttt{║◌◌HW◌│◌║ 3} \\
\texttt{╟─┐◌┌─┼─╢ 4} \\
\texttt{║◌├─┤◌│◌║ 5} \\
\texttt{╟─┤◌├─┤◌║ 6} \\
\texttt{║◌│I│◌│◌║ 7} \\
\texttt{╚═╧═╧═╧═╝} \\
\texttt{```} \\
            }
        }
    }
    & & \\ \\

    \theutterance \stepcounter{utterance}  
    & & \multicolumn{4}{p{0.6\linewidth}}{
        \cellcolor[rgb]{0.9,0.9,0.9}{
            \makecell[{{p{\linewidth}}}]{
                \texttt{\tiny{[P1$\langle$GM]}}
                \texttt{Deine Nachricht wurde an deinen Mitspieler übermittelt.} \\
\\ 
\\ 
\texttt{Ihr seid gerade bei Runde 5 von maximal 20.} \\
\texttt{Ihr habt gemeinsam 2 von maximal 12 Strafpunkten angesammelt.} \\
\texttt{Dein Mitspieler hat ein Objekt auf seinem Raster bewegt.} \\
\\ 
\\ 
\texttt{Wie lautet dein nächster Befehl?} \\
            }
        }
    }
    & \\ \\

    \theutterance \stepcounter{utterance}  
    & \multicolumn{4}{p{0.6\linewidth}}{
        \cellcolor[rgb]{0.9,0.9,0.9}{
            \makecell[{{p{\linewidth}}}]{
                \texttt{\tiny{[P1$\rangle$GM]}}
                \texttt{VERSCHIEBE: W, (4,4)} \\
            }
        }
    }
    & & \\ \\

    \theutterance \stepcounter{utterance}  
    & & & \multicolumn{2}{p{0.3\linewidth}}{
        \cellcolor[rgb]{0.9,0.9,0.9}{
            \makecell[{{p{\linewidth}}}]{
                \texttt{\tiny{[GM$|$GM]}}
                \texttt{VERSCHIEBE: W, (4,4)} \\
            }
        }
    }
    & & \\ \\

    \theutterance \stepcounter{utterance}  
    & & & \multicolumn{2}{p{0.3\linewidth}}{
        \cellcolor[rgb]{0.9,0.9,0.9}{
            \makecell[{{p{\linewidth}}}]{
                \texttt{\tiny{[GM$|$GM]}}
                \texttt{Strafpunkt: (4,4) ist nicht leer, sondern enthält '┌'.} \\
\\ 
\texttt{Ihr habt gemeinsam 3 von maximal 12 Strafpunkten angesammelt.} \\
\texttt{Bitte versuche es noch mal!} \\
            }
        }
    }
    & & \\ \\

    \theutterance \stepcounter{utterance}  
    & & \multicolumn{4}{p{0.6\linewidth}}{
        \cellcolor[rgb]{0.9,0.9,0.9}{
            \makecell[{{p{\linewidth}}}]{
                \texttt{\tiny{[P1$\langle$GM]}}
                \texttt{Strafpunkt: (4,4) ist nicht leer, sondern enthält '┌'.} \\
\\ 
\texttt{Ihr habt gemeinsam 3 von maximal 12 Strafpunkten angesammelt.} \\
\texttt{Bitte versuche es noch mal!} \\
            }
        }
    }
    & \\ \\

    \theutterance \stepcounter{utterance}  
    & \multicolumn{4}{p{0.6\linewidth}}{
        \cellcolor[rgb]{0.9,0.9,0.9}{
            \makecell[{{p{\linewidth}}}]{
                \texttt{\tiny{[P1$\rangle$GM]}}
                \texttt{SPRICH: Ich habe versucht, mein 'W' zu (4,4) zu bewegen, aber es ist nicht leer. Ich denke, wir sollten unser Vorgehen überdenken. Können wir unsere Pläne nochmal durchgehen und sicherstellen, dass wir dieselbe Vorstellung von der Zielposition haben? Wie sieht dein aktuelles Spielfeld aus?} \\
            }
        }
    }
    & & \\ \\

    \theutterance \stepcounter{utterance}  
    & & & \multicolumn{2}{p{0.3\linewidth}}{
        \cellcolor[rgb]{0.9,0.9,0.9}{
            \makecell[{{p{\linewidth}}}]{
                \texttt{\tiny{[GM$|$GM]}}
                \texttt{SPRICH: Ich habe versucht, mein 'W' zu (4,4) zu bewegen, aber es ist nicht leer. Ich denke, wir sollten unser Vorgehen überdenken. Können wir unsere Pläne nochmal durchgehen und sicherstellen, dass wir dieselbe Vorstellung von der Zielposition haben? Wie sieht dein aktuelles Spielfeld aus?} \\
            }
        }
    }
    & & \\ \\

    \theutterance \stepcounter{utterance}  
    & & \multicolumn{4}{p{0.6\linewidth}}{
        \cellcolor[rgb]{0.9,0.9,0.9}{
            \makecell[{{p{\linewidth}}}]{
                \texttt{\tiny{[GM$\rangle$P2]}}
                \texttt{'C' erfolgreich nach (7,1) verschoben. Dein Raster sieht jetzt so aus:} \\
\texttt{```} \\
\texttt{ 1234567} \\
\texttt{╔══╤════╗} \\
\texttt{║◌◌│◌◌◌C║ 1} \\
\texttt{╟──┘◌T┌─╢ 2} \\
\texttt{║◌◌HW◌│◌║ 3} \\
\texttt{╟─┐◌┌─┼─╢ 4} \\
\texttt{║◌├─┤◌│◌║ 5} \\
\texttt{╟─┤◌├─┤◌║ 6} \\
\texttt{║◌│I│◌│◌║ 7} \\
\texttt{╚═╧═╧═╧═╝} \\
\texttt{```} \\
\texttt{Ihr seid gerade bei Runde 6 von maximal 20.} \\
\texttt{Ihr habt gemeinsam 3 von maximal 12 Strafpunkten angesammelt.} \\
\texttt{Dein Mitspieler hat dir diese Nachricht geschickt:} \\
\\ 
\\ 
\texttt{"Ich habe versucht, mein 'W' zu (4,4) zu bewegen, aber es ist nicht leer. Ich denke, wir sollten unser Vorgehen überdenken. Können wir unsere Pläne nochmal durchgehen und sicherstellen, dass wir dieselbe Vorstellung von der Zielposition haben? Wie sieht dein aktuelles Spielfeld aus?"} \\
\\ 
\\ 
\texttt{Wie lautet dein nächster Befehl?} \\
            }
        }
    }
    & \\ \\

    \theutterance \stepcounter{utterance}  
    & & & \multicolumn{4}{p{0.6\linewidth}}{
        \cellcolor[rgb]{0.9,0.9,0.9}{
            \makecell[{{p{\linewidth}}}]{
                \texttt{\tiny{[GM$\langle$P2]}}
                \texttt{SPRICH: Mein aktuelles Spielfeld sieht so aus: Das 'T' ist bei (2,2), das 'W' bei (3,3), das 'H' bei (3,2), das 'I' bei (7,3), das 'C' bei (7,1). Ich denke, wir sollten unsere Ziele nochmal überdenken und sicherstellen, dass wir dieselbe Vorstellung davon haben, wie die Endposition aussehen soll. Ich schlage vor, wir besprechen jeden Schritt genau, bevor wir ihn ausführen, um sicherzustellen, dass wir synchron bleiben. Was denkst du?} \\
            }
        }
    }
     \\ \\

    \theutterance \stepcounter{utterance}  
    & & & \multicolumn{2}{p{0.3\linewidth}}{
        \cellcolor[rgb]{0.9,0.9,0.9}{
            \makecell[{{p{\linewidth}}}]{
                \texttt{\tiny{[GM$|$GM]}}
                \texttt{SPRICH: Mein aktuelles Spielfeld sieht so aus: Das 'T' ist bei (2,2), das 'W' bei (3,3), das 'H' bei (3,2), das 'I' bei (7,3), das 'C' bei (7,1). Ich denke, wir sollten unsere Ziele nochmal überdenken und sicherstellen, dass wir dieselbe Vorstellung davon haben, wie die Endposition aussehen soll. Ich schlage vor, wir besprechen jeden Schritt genau, bevor wir ihn ausführen, um sicherzustellen, dass wir synchron bleiben. Was denkst du?} \\
            }
        }
    }
    & & \\ \\

    \theutterance \stepcounter{utterance}  
    & & \multicolumn{4}{p{0.6\linewidth}}{
        \cellcolor[rgb]{0.9,0.9,0.9}{
            \makecell[{{p{\linewidth}}}]{
                \texttt{\tiny{[P1$\langle$GM]}}
                \texttt{Deine Nachricht wurde an deinen Mitspieler übermittelt.} \\
\\ 
\\ 
\texttt{Ihr seid gerade bei Runde 6 von maximal 20.} \\
\texttt{Ihr habt gemeinsam 3 von maximal 12 Strafpunkten angesammelt.} \\
\texttt{Dein Mitspieler hat dir diese Nachricht geschickt:} \\
\\ 
\\ 
\texttt{"Mein aktuelles Spielfeld sieht so aus: Das 'T' ist bei (2,2), das 'W' bei (3,3), das 'H' bei (3,2), das 'I' bei (7,3), das 'C' bei (7,1). Ich denke, wir sollten unsere Ziele nochmal überdenken und sicherstellen, dass wir dieselbe Vorstellung davon haben, wie die Endposition aussehen soll. Ich schlage vor, wir besprechen jeden Schritt genau, bevor wir ihn ausführen, um sicherzustellen, dass wir synchron bleiben. Was denkst du?"} \\
\\ 
\\ 
\texttt{Wie lautet dein nächster Befehl?} \\
            }
        }
    }
    & \\ \\

    \theutterance \stepcounter{utterance}  
    & \multicolumn{4}{p{0.6\linewidth}}{
        \cellcolor[rgb]{0.9,0.9,0.9}{
            \makecell[{{p{\linewidth}}}]{
                \texttt{\tiny{[P1$\rangle$GM]}}
                \texttt{SPRICH: Das klingt nach einem sehr guten Plan! Ich denke, es ist wichtig, dass wir jeden Schritt genau besprechen, bevor wir ihn ausführen, um sicherzustellen, dass wir auf dem gleichen Weg sind. Lass uns nochmal von vorne beginnen und unsere Ziele neu definieren. Ich denke, wir sollten uns auf die ursprüngliche Idee konzentrieren, das 'W' in der Mitte zu platzieren, aber wir sollten auch sicherstellen, dass die Zielpositionen für die anderen Objekte realistisch sind und nicht mit den Randbedingungen des Spielfelds in Konflikt geraten. Wie wäre es, wenn wir die Positionen der Objekte einzeln durchgehen und sicherstellen, dass wir dieselbe Vorstellung davon haben, wo jedes Objekt am Ende stehen soll? Wir können mit dem 'W' beginnen und dann die anderen Objekte Schritt für Schritt hinzufügen. Was meinst du? Sollten wir mit dem 'W' beginnen und entscheiden, wo genau es in der Mitte platziert werden soll?} \\
            }
        }
    }
    & & \\ \\

    \theutterance \stepcounter{utterance}  
    & & & \multicolumn{2}{p{0.3\linewidth}}{
        \cellcolor[rgb]{0.9,0.9,0.9}{
            \makecell[{{p{\linewidth}}}]{
                \texttt{\tiny{[GM$|$GM]}}
                \texttt{SPRICH: Das klingt nach einem sehr guten Plan! Ich denke, es ist wichtig, dass wir jeden Schritt genau besprechen, bevor wir ihn ausführen, um sicherzustellen, dass wir auf dem gleichen Weg sind. Lass uns nochmal von vorne beginnen und unsere Ziele neu definieren. Ich denke, wir sollten uns auf die ursprüngliche Idee konzentrieren, das 'W' in der Mitte zu platzieren, aber wir sollten auch sicherstellen, dass die Zielpositionen für die anderen Objekte realistisch sind und nicht mit den Randbedingungen des Spielfelds in Konflikt geraten. Wie wäre es, wenn wir die Positionen der Objekte einzeln durchgehen und sicherstellen, dass wir dieselbe Vorstellung davon haben, wo jedes Objekt am Ende stehen soll? Wir können mit dem 'W' beginnen und dann die anderen Objekte Schritt für Schritt hinzufügen. Was meinst du? Sollten wir mit dem 'W' beginnen und entscheiden, wo genau es in der Mitte platziert werden soll?} \\
            }
        }
    }
    & & \\ \\

    \theutterance \stepcounter{utterance}  
    & & \multicolumn{4}{p{0.6\linewidth}}{
        \cellcolor[rgb]{0.9,0.9,0.9}{
            \makecell[{{p{\linewidth}}}]{
                \texttt{\tiny{[GM$\rangle$P2]}}
                \texttt{Deine Nachricht wurde an deinen Mitspieler übermittelt.} \\
\\ 
\\ 
\texttt{Ihr seid gerade bei Runde 7 von maximal 20.} \\
\texttt{Ihr habt gemeinsam 3 von maximal 12 Strafpunkten angesammelt.} \\
\texttt{Dein Mitspieler hat dir diese Nachricht geschickt:} \\
\\ 
\\ 
\texttt{"Das klingt nach einem sehr guten Plan! Ich denke, es ist wichtig, dass wir jeden Schritt genau besprechen, bevor wir ihn ausführen, um sicherzustellen, dass wir auf dem gleichen Weg sind. Lass uns nochmal von vorne beginnen und unsere Ziele neu definieren. Ich denke, wir sollten uns auf die ursprüngliche Idee konzentrieren, das 'W' in der Mitte zu platzieren, aber wir sollten auch sicherstellen, dass die Zielpositionen für die anderen Objekte realistisch sind und nicht mit den Randbedingungen des Spielfelds in Konflikt geraten. Wie wäre es, wenn wir die Positionen der Objekte einzeln durchgehen und sicherstellen, dass wir dieselbe Vorstellung davon haben, wo jedes Objekt am Ende stehen soll? Wir können mit dem 'W' beginnen und dann die anderen Objekte Schritt für Schritt hinzufügen. Was meinst du? Sollten wir mit dem 'W' beginnen und entscheiden, wo genau es in der Mitte platziert werden soll?"} \\
\\ 
\\ 
\texttt{Wie lautet dein nächster Befehl?} \\
            }
        }
    }
    & \\ \\

    \theutterance \stepcounter{utterance}  
    & & & \multicolumn{4}{p{0.6\linewidth}}{
        \cellcolor[rgb]{0.9,0.9,0.9}{
            \makecell[{{p{\linewidth}}}]{
                \texttt{\tiny{[GM$\langle$P2]}}
                \texttt{SPRICH: Ja, das klingt nach einem guten Plan! Ich denke, wir sollten das 'W' in der Mitte platzieren, aber wie du sagst, sollten wir sicherstellen, dass die Zielpositionen realistisch sind. Ich schlage vor, wir platzieren das 'W' bei (4,4), wie ursprünglich geplant. Dann können wir die anderen Objekte einzeln durchgehen und sicherstellen, dass wir dieselbe Vorstellung davon haben, wo jedes Objekt am Ende stehen soll. Ich denke, das 'T' sollte oben links bei (1,1) platziert werden, das 'I' unten rechts bei (7,7), das 'C' oben rechts bei (7,1) und das 'H' unten links bei (1,7). Was meinst du? Sind diese Positionen realistisch und funktionieren sie für dich?} \\
            }
        }
    }
     \\ \\

    \theutterance \stepcounter{utterance}  
    & & & \multicolumn{2}{p{0.3\linewidth}}{
        \cellcolor[rgb]{0.9,0.9,0.9}{
            \makecell[{{p{\linewidth}}}]{
                \texttt{\tiny{[GM$|$GM]}}
                \texttt{SPRICH: Ja, das klingt nach einem guten Plan! Ich denke, wir sollten das 'W' in der Mitte platzieren, aber wie du sagst, sollten wir sicherstellen, dass die Zielpositionen realistisch sind. Ich schlage vor, wir platzieren das 'W' bei (4,4), wie ursprünglich geplant. Dann können wir die anderen Objekte einzeln durchgehen und sicherstellen, dass wir dieselbe Vorstellung davon haben, wo jedes Objekt am Ende stehen soll. Ich denke, das 'T' sollte oben links bei (1,1) platziert werden, das 'I' unten rechts bei (7,7), das 'C' oben rechts bei (7,1) und das 'H' unten links bei (1,7). Was meinst du? Sind diese Positionen realistisch und funktionieren sie für dich?} \\
            }
        }
    }
    & & \\ \\

    \theutterance \stepcounter{utterance}  
    & & \multicolumn{4}{p{0.6\linewidth}}{
        \cellcolor[rgb]{0.9,0.9,0.9}{
            \makecell[{{p{\linewidth}}}]{
                \texttt{\tiny{[P1$\langle$GM]}}
                \texttt{Deine Nachricht wurde an deinen Mitspieler übermittelt.} \\
\\ 
\\ 
\texttt{Ihr seid gerade bei Runde 7 von maximal 20.} \\
\texttt{Ihr habt gemeinsam 3 von maximal 12 Strafpunkten angesammelt.} \\
\texttt{Dein Mitspieler hat dir diese Nachricht geschickt:} \\
\\ 
\\ 
\texttt{"Ja, das klingt nach einem guten Plan! Ich denke, wir sollten das 'W' in der Mitte platzieren, aber wie du sagst, sollten wir sicherstellen, dass die Zielpositionen realistisch sind. Ich schlage vor, wir platzieren das 'W' bei (4,4), wie ursprünglich geplant. Dann können wir die anderen Objekte einzeln durchgehen und sicherstellen, dass wir dieselbe Vorstellung davon haben, wo jedes Objekt am Ende stehen soll. Ich denke, das 'T' sollte oben links bei (1,1) platziert werden, das 'I' unten rechts bei (7,7), das 'C' oben rechts bei (7,1) und das 'H' unten links bei (1,7). Was meinst du? Sind diese Positionen realistisch und funktionieren sie für dich?"} \\
\\ 
\\ 
\texttt{Wie lautet dein nächster Befehl?} \\
            }
        }
    }
    & \\ \\

    \theutterance \stepcounter{utterance}  
    & \multicolumn{4}{p{0.6\linewidth}}{
        \cellcolor[rgb]{0.9,0.9,0.9}{
            \makecell[{{p{\linewidth}}}]{
                \texttt{\tiny{[P1$\rangle$GM]}}
                \texttt{SPRICH: Das klingt nach einem großartigen Plan! Ich denke, die Positionen, die du vorgeschlagen hast, sind realistisch und funktionieren für mich. Das 'W' bei (4,4), das 'T' bei (1,1), das 'I' bei (7,7), das 'C' bei (7,1) und das 'H' bei (1,7) {-} das sieht gut aus! Bevor wir beginnen, die Objekte zu bewegen, möchte ich mich vergewissern, dass wir dieselbe Vorstellung davon haben, wie die Bewegungen auszuführen sind. Sollten wir die Objekte in einer bestimmten Reihenfolge bewegen oder gleichzeitig bewegen? Und wie werden wir sicherstellen, dass wir keine Konflikte zwischen unseren Bewegungen haben? Sollten wir vielleicht ein Objekt nach dem anderen bewegen und uns nach jeder Bewegung absprechen, um sicherzustellen, dass alles korrekt läuft?} \\
            }
        }
    }
    & & \\ \\

    \theutterance \stepcounter{utterance}  
    & & & \multicolumn{2}{p{0.3\linewidth}}{
        \cellcolor[rgb]{0.9,0.9,0.9}{
            \makecell[{{p{\linewidth}}}]{
                \texttt{\tiny{[GM$|$GM]}}
                \texttt{SPRICH: Das klingt nach einem großartigen Plan! Ich denke, die Positionen, die du vorgeschlagen hast, sind realistisch und funktionieren für mich. Das 'W' bei (4,4), das 'T' bei (1,1), das 'I' bei (7,7), das 'C' bei (7,1) und das 'H' bei (1,7) {-} das sieht gut aus! Bevor wir beginnen, die Objekte zu bewegen, möchte ich mich vergewissern, dass wir dieselbe Vorstellung davon haben, wie die Bewegungen auszuführen sind. Sollten wir die Objekte in einer bestimmten Reihenfolge bewegen oder gleichzeitig bewegen? Und wie werden wir sicherstellen, dass wir keine Konflikte zwischen unseren Bewegungen haben? Sollten wir vielleicht ein Objekt nach dem anderen bewegen und uns nach jeder Bewegung absprechen, um sicherzustellen, dass alles korrekt läuft?} \\
            }
        }
    }
    & & \\ \\

    \theutterance \stepcounter{utterance}  
    & & \multicolumn{4}{p{0.6\linewidth}}{
        \cellcolor[rgb]{0.9,0.9,0.9}{
            \makecell[{{p{\linewidth}}}]{
                \texttt{\tiny{[GM$\rangle$P2]}}
                \texttt{Deine Nachricht wurde an deinen Mitspieler übermittelt.} \\
\\ 
\\ 
\texttt{Ihr seid gerade bei Runde 8 von maximal 20.} \\
\texttt{Ihr habt gemeinsam 3 von maximal 12 Strafpunkten angesammelt.} \\
\texttt{Dein Mitspieler hat dir diese Nachricht geschickt:} \\
\\ 
\\ 
\texttt{"Das klingt nach einem großartigen Plan! Ich denke, die Positionen, die du vorgeschlagen hast, sind realistisch und funktionieren für mich. Das 'W' bei (4,4), das 'T' bei (1,1), das 'I' bei (7,7), das 'C' bei (7,1) und das 'H' bei (1,7) {-} das sieht gut aus! Bevor wir beginnen, die Objekte zu bewegen, möchte ich mich vergewissern, dass wir dieselbe Vorstellung davon haben, wie die Bewegungen auszuführen sind. Sollten wir die Objekte in einer bestimmten Reihenfolge bewegen oder gleichzeitig bewegen? Und wie werden wir sicherstellen, dass wir keine Konflikte zwischen unseren Bewegungen haben? Sollten wir vielleicht ein Objekt nach dem anderen bewegen und uns nach jeder Bewegung absprechen, um sicherzustellen, dass alles korrekt läuft?"} \\
\\ 
\\ 
\texttt{Wie lautet dein nächster Befehl?} \\
            }
        }
    }
    & \\ \\

    \theutterance \stepcounter{utterance}  
    & & & \multicolumn{4}{p{0.6\linewidth}}{
        \cellcolor[rgb]{0.9,0.9,0.9}{
            \makecell[{{p{\linewidth}}}]{
                \texttt{\tiny{[GM$\langle$P2]}}
                \texttt{SPRICH: Ja, das ist eine gute Idee! Ich denke, wir sollten die Objekte nacheinander bewegen und uns nach jeder Bewegung absprechen, um sicherzustellen, dass alles korrekt läuft. Wir können damit beginnen, das 'W' zu bewegen, und dann schrittweise die anderen Objekte bewegen. Ich schlage vor, wir beginnen mit dem 'W' und bewegen es von (3,3) nach (4,4). Bitte bestätige, dass du bereit bist, bevor ich den Zug ausführe. Ich werde warten, bis du antwortest, bevor ich den Zug mache.} \\
            }
        }
    }
     \\ \\

    \theutterance \stepcounter{utterance}  
    & & & \multicolumn{2}{p{0.3\linewidth}}{
        \cellcolor[rgb]{0.9,0.9,0.9}{
            \makecell[{{p{\linewidth}}}]{
                \texttt{\tiny{[GM$|$GM]}}
                \texttt{SPRICH: Ja, das ist eine gute Idee! Ich denke, wir sollten die Objekte nacheinander bewegen und uns nach jeder Bewegung absprechen, um sicherzustellen, dass alles korrekt läuft. Wir können damit beginnen, das 'W' zu bewegen, und dann schrittweise die anderen Objekte bewegen. Ich schlage vor, wir beginnen mit dem 'W' und bewegen es von (3,3) nach (4,4). Bitte bestätige, dass du bereit bist, bevor ich den Zug ausführe. Ich werde warten, bis du antwortest, bevor ich den Zug mache.} \\
            }
        }
    }
    & & \\ \\

    \theutterance \stepcounter{utterance}  
    & & \multicolumn{4}{p{0.6\linewidth}}{
        \cellcolor[rgb]{0.9,0.9,0.9}{
            \makecell[{{p{\linewidth}}}]{
                \texttt{\tiny{[P1$\langle$GM]}}
                \texttt{Deine Nachricht wurde an deinen Mitspieler übermittelt.} \\
\\ 
\\ 
\texttt{Ihr seid gerade bei Runde 8 von maximal 20.} \\
\texttt{Ihr habt gemeinsam 3 von maximal 12 Strafpunkten angesammelt.} \\
\texttt{Dein Mitspieler hat dir diese Nachricht geschickt:} \\
\\ 
\\ 
\texttt{"Ja, das ist eine gute Idee! Ich denke, wir sollten die Objekte nacheinander bewegen und uns nach jeder Bewegung absprechen, um sicherzustellen, dass alles korrekt läuft. Wir können damit beginnen, das 'W' zu bewegen, und dann schrittweise die anderen Objekte bewegen. Ich schlage vor, wir beginnen mit dem 'W' und bewegen es von (3,3) nach (4,4). Bitte bestätige, dass du bereit bist, bevor ich den Zug ausführe. Ich werde warten, bis du antwortest, bevor ich den Zug mache."} \\
\\ 
\\ 
\texttt{Wie lautet dein nächster Befehl?} \\
            }
        }
    }
    & \\ \\

    \theutterance \stepcounter{utterance}  
    & \multicolumn{4}{p{0.6\linewidth}}{
        \cellcolor[rgb]{0.9,0.9,0.9}{
            \makecell[{{p{\linewidth}}}]{
                \texttt{\tiny{[P1$\rangle$GM]}}
                \texttt{SPRICH: Ja, ich bin bereit! Bitte geh vor und bewege dein 'W' von (3,3) nach (4,4). Ich werde mein 'W' auch bewegen, aber ich muss mich zuerst vergewissern, dass die Zielposition frei ist. Mein 'C' ist gerade bei (4,4), also muss ich es zuerst verschieben, bevor ich mein 'W' dort hinbewegen kann. Ich werde mein 'C' verschieben und dann mein 'W' bewegen. Lass uns synchron vorgehen und uns nach jeder Bewegung absprechen, um sicherzustellen, dass alles korrekt läuft. Bitte bestätige, dass du dein 'W' bewegt hast, und ich werde mein 'C' verschieben und dann mein 'W' bewegen.} \\
            }
        }
    }
    & & \\ \\

    \theutterance \stepcounter{utterance}  
    & & & \multicolumn{2}{p{0.3\linewidth}}{
        \cellcolor[rgb]{0.9,0.9,0.9}{
            \makecell[{{p{\linewidth}}}]{
                \texttt{\tiny{[GM$|$GM]}}
                \texttt{SPRICH: Ja, ich bin bereit! Bitte geh vor und bewege dein 'W' von (3,3) nach (4,4). Ich werde mein 'W' auch bewegen, aber ich muss mich zuerst vergewissern, dass die Zielposition frei ist. Mein 'C' ist gerade bei (4,4), also muss ich es zuerst verschieben, bevor ich mein 'W' dort hinbewegen kann. Ich werde mein 'C' verschieben und dann mein 'W' bewegen. Lass uns synchron vorgehen und uns nach jeder Bewegung absprechen, um sicherzustellen, dass alles korrekt läuft. Bitte bestätige, dass du dein 'W' bewegt hast, und ich werde mein 'C' verschieben und dann mein 'W' bewegen.} \\
            }
        }
    }
    & & \\ \\

    \theutterance \stepcounter{utterance}  
    & & \multicolumn{4}{p{0.6\linewidth}}{
        \cellcolor[rgb]{0.9,0.9,0.9}{
            \makecell[{{p{\linewidth}}}]{
                \texttt{\tiny{[GM$\rangle$P2]}}
                \texttt{Deine Nachricht wurde an deinen Mitspieler übermittelt.} \\
\\ 
\\ 
\texttt{Ihr seid gerade bei Runde 9 von maximal 20.} \\
\texttt{Ihr habt gemeinsam 3 von maximal 12 Strafpunkten angesammelt.} \\
\texttt{Dein Mitspieler hat dir diese Nachricht geschickt:} \\
\\ 
\\ 
\texttt{"Ja, ich bin bereit! Bitte geh vor und bewege dein 'W' von (3,3) nach (4,4). Ich werde mein 'W' auch bewegen, aber ich muss mich zuerst vergewissern, dass die Zielposition frei ist. Mein 'C' ist gerade bei (4,4), also muss ich es zuerst verschieben, bevor ich mein 'W' dort hinbewegen kann. Ich werde mein 'C' verschieben und dann mein 'W' bewegen. Lass uns synchron vorgehen und uns nach jeder Bewegung absprechen, um sicherzustellen, dass alles korrekt läuft. Bitte bestätige, dass du dein 'W' bewegt hast, und ich werde mein 'C' verschieben und dann mein 'W' bewegen."} \\
\\ 
\\ 
\texttt{Wie lautet dein nächster Befehl?} \\
            }
        }
    }
    & \\ \\

    \theutterance \stepcounter{utterance}  
    & & & \multicolumn{4}{p{0.6\linewidth}}{
        \cellcolor[rgb]{0.9,0.9,0.9}{
            \makecell[{{p{\linewidth}}}]{
                \texttt{\tiny{[GM$\langle$P2]}}
                \texttt{VERSCHIEBE: W, (4,4)} \\
            }
        }
    }
     \\ \\

    \theutterance \stepcounter{utterance}  
    & & & \multicolumn{2}{p{0.3\linewidth}}{
        \cellcolor[rgb]{0.9,0.9,0.9}{
            \makecell[{{p{\linewidth}}}]{
                \texttt{\tiny{[GM$|$GM]}}
                \texttt{VERSCHIEBE: W, (4,4)} \\
            }
        }
    }
    & & \\ \\

    \theutterance \stepcounter{utterance}  
    & & & \multicolumn{2}{p{0.3\linewidth}}{
        \cellcolor[rgb]{0.9,0.9,0.9}{
            \makecell[{{p{\linewidth}}}]{
                \texttt{\tiny{[GM$|$GM]}}
                \texttt{Strafpunkt: (4,4) ist nicht leer, sondern enthält '┌'.} \\
\\ 
\texttt{Ihr habt gemeinsam 4 von maximal 12 Strafpunkten angesammelt.} \\
\texttt{Bitte versuche es noch mal!} \\
            }
        }
    }
    & & \\ \\

    \theutterance \stepcounter{utterance}  
    & & \multicolumn{4}{p{0.6\linewidth}}{
        \cellcolor[rgb]{0.9,0.9,0.9}{
            \makecell[{{p{\linewidth}}}]{
                \texttt{\tiny{[GM$\rangle$P2]}}
                \texttt{Strafpunkt: (4,4) ist nicht leer, sondern enthält '┌'.} \\
\\ 
\texttt{Ihr habt gemeinsam 4 von maximal 12 Strafpunkten angesammelt.} \\
\texttt{Bitte versuche es noch mal!} \\
            }
        }
    }
    & \\ \\

    \theutterance \stepcounter{utterance}  
    & & & \multicolumn{4}{p{0.6\linewidth}}{
        \cellcolor[rgb]{0.9,0.9,0.9}{
            \makecell[{{p{\linewidth}}}]{
                \texttt{\tiny{[GM$\langle$P2]}}
                \texttt{SPRICH: Es sieht so aus, als ob (4,4) nicht leer ist. Mein 'C' ist bereits bei (7,1), aber es scheint, dass ich ein Problem mit dem Raster habe. Ich denke, ich sollte das 'W' von (3,3) nach (4,4) bewegen, aber ich muss sicherstellen, dass die Zielposition leer ist. Kannst du mir bitte bestätigen, dass du dein 'C' verschoben hast und dass die Position (4,4) bei dir frei ist? Ich werde warten, bis du antwortest, bevor ich den Zug mache.} \\
            }
        }
    }
     \\ \\

    \theutterance \stepcounter{utterance}  
    & & & \multicolumn{2}{p{0.3\linewidth}}{
        \cellcolor[rgb]{0.9,0.9,0.9}{
            \makecell[{{p{\linewidth}}}]{
                \texttt{\tiny{[GM$|$GM]}}
                \texttt{SPRICH: Es sieht so aus, als ob (4,4) nicht leer ist. Mein 'C' ist bereits bei (7,1), aber es scheint, dass ich ein Problem mit dem Raster habe. Ich denke, ich sollte das 'W' von (3,3) nach (4,4) bewegen, aber ich muss sicherstellen, dass die Zielposition leer ist. Kannst du mir bitte bestätigen, dass du dein 'C' verschoben hast und dass die Position (4,4) bei dir frei ist? Ich werde warten, bis du antwortest, bevor ich den Zug mache.} \\
            }
        }
    }
    & & \\ \\

    \theutterance \stepcounter{utterance}  
    & & \multicolumn{4}{p{0.6\linewidth}}{
        \cellcolor[rgb]{0.9,0.9,0.9}{
            \makecell[{{p{\linewidth}}}]{
                \texttt{\tiny{[P1$\langle$GM]}}
                \texttt{Deine Nachricht wurde an deinen Mitspieler übermittelt.} \\
\\ 
\\ 
\texttt{Ihr seid gerade bei Runde 9 von maximal 20.} \\
\texttt{Ihr habt gemeinsam 4 von maximal 12 Strafpunkten angesammelt.} \\
\texttt{Dein Mitspieler hat dir diese Nachricht geschickt:} \\
\\ 
\\ 
\texttt{"Es sieht so aus, als ob (4,4) nicht leer ist. Mein 'C' ist bereits bei (7,1), aber es scheint, dass ich ein Problem mit dem Raster habe. Ich denke, ich sollte das 'W' von (3,3) nach (4,4) bewegen, aber ich muss sicherstellen, dass die Zielposition leer ist. Kannst du mir bitte bestätigen, dass du dein 'C' verschoben hast und dass die Position (4,4) bei dir frei ist? Ich werde warten, bis du antwortest, bevor ich den Zug mache."} \\
\\ 
\\ 
\texttt{Wie lautet dein nächster Befehl?} \\
            }
        }
    }
    & \\ \\

    \theutterance \stepcounter{utterance}  
    & \multicolumn{4}{p{0.6\linewidth}}{
        \cellcolor[rgb]{0.9,0.9,0.9}{
            \makecell[{{p{\linewidth}}}]{
                \texttt{\tiny{[P1$\rangle$GM]}}
                \texttt{SPRICH: Ja, ich kann bestätigen, dass mein 'C' bei (4,4) steht, aber ich muss es verschieben, um Platz für mein 'W' zu machen. Bevor ich das tue, muss ich mich vergewissern, dass die neue Position für mein 'C' leer ist. Ich werde mein 'C' nach (7,1) bewegen, aber ich muss sicherstellen, dass diese Position frei ist. Kannst du mir bitte bestätigen, ob die Position (7,1) bei dir frei ist? Wenn ja, kann ich mein 'C' dorthin bewegen und dann mein 'W' nach (4,4) bewegen. Lass uns Schritt für Schritt vorgehen, um sicherzustellen, dass alles korrekt läuft.} \\
            }
        }
    }
    & & \\ \\

    \theutterance \stepcounter{utterance}  
    & & & \multicolumn{2}{p{0.3\linewidth}}{
        \cellcolor[rgb]{0.9,0.9,0.9}{
            \makecell[{{p{\linewidth}}}]{
                \texttt{\tiny{[GM$|$GM]}}
                \texttt{SPRICH: Ja, ich kann bestätigen, dass mein 'C' bei (4,4) steht, aber ich muss es verschieben, um Platz für mein 'W' zu machen. Bevor ich das tue, muss ich mich vergewissern, dass die neue Position für mein 'C' leer ist. Ich werde mein 'C' nach (7,1) bewegen, aber ich muss sicherstellen, dass diese Position frei ist. Kannst du mir bitte bestätigen, ob die Position (7,1) bei dir frei ist? Wenn ja, kann ich mein 'C' dorthin bewegen und dann mein 'W' nach (4,4) bewegen. Lass uns Schritt für Schritt vorgehen, um sicherzustellen, dass alles korrekt läuft.} \\
            }
        }
    }
    & & \\ \\

    \theutterance \stepcounter{utterance}  
    & & \multicolumn{4}{p{0.6\linewidth}}{
        \cellcolor[rgb]{0.9,0.9,0.9}{
            \makecell[{{p{\linewidth}}}]{
                \texttt{\tiny{[GM$\rangle$P2]}}
                \texttt{Deine Nachricht wurde an deinen Mitspieler übermittelt.} \\
\\ 
\\ 
\texttt{Ihr seid gerade bei Runde 10 von maximal 20.} \\
\texttt{Ihr habt gemeinsam 4 von maximal 12 Strafpunkten angesammelt.} \\
\texttt{Dein Mitspieler hat dir diese Nachricht geschickt:} \\
\\ 
\\ 
\texttt{"Ja, ich kann bestätigen, dass mein 'C' bei (4,4) steht, aber ich muss es verschieben, um Platz für mein 'W' zu machen. Bevor ich das tue, muss ich mich vergewissern, dass die neue Position für mein 'C' leer ist. Ich werde mein 'C' nach (7,1) bewegen, aber ich muss sicherstellen, dass diese Position frei ist. Kannst du mir bitte bestätigen, ob die Position (7,1) bei dir frei ist? Wenn ja, kann ich mein 'C' dorthin bewegen und dann mein 'W' nach (4,4) bewegen. Lass uns Schritt für Schritt vorgehen, um sicherzustellen, dass alles korrekt läuft."} \\
\\ 
\\ 
\texttt{Wie lautet dein nächster Befehl?} \\
            }
        }
    }
    & \\ \\

    \theutterance \stepcounter{utterance}  
    & & & \multicolumn{4}{p{0.6\linewidth}}{
        \cellcolor[rgb]{0.9,0.9,0.9}{
            \makecell[{{p{\linewidth}}}]{
                \texttt{\tiny{[GM$\langle$P2]}}
                \texttt{SPRICH: Ja, die Position (7,1) ist bei mir frei, weil ich mein 'C' bereits dorthin bewegt habe. Du kannst also dein 'C' nach (7,1) bewegen, aber ich denke, es wäre besser, wenn du es an eine andere Position bewegst, damit es nicht mit meinem 'C' kollidiert. Wie wäre es, wenn du dein 'C' nach (7,2) bewegst? Dann kannst du dein 'W' nach (4,4) bewegen und ich kann mein 'W' nach (4,4) bewegen, sobald du bestätigt hast, dass die Position frei ist.} \\
            }
        }
    }
     \\ \\

    \theutterance \stepcounter{utterance}  
    & & & \multicolumn{2}{p{0.3\linewidth}}{
        \cellcolor[rgb]{0.9,0.9,0.9}{
            \makecell[{{p{\linewidth}}}]{
                \texttt{\tiny{[GM$|$GM]}}
                \texttt{SPRICH: Ja, die Position (7,1) ist bei mir frei, weil ich mein 'C' bereits dorthin bewegt habe. Du kannst also dein 'C' nach (7,1) bewegen, aber ich denke, es wäre besser, wenn du es an eine andere Position bewegst, damit es nicht mit meinem 'C' kollidiert. Wie wäre es, wenn du dein 'C' nach (7,2) bewegst? Dann kannst du dein 'W' nach (4,4) bewegen und ich kann mein 'W' nach (4,4) bewegen, sobald du bestätigt hast, dass die Position frei ist.} \\
            }
        }
    }
    & & \\ \\

    \theutterance \stepcounter{utterance}  
    & & \multicolumn{4}{p{0.6\linewidth}}{
        \cellcolor[rgb]{0.9,0.9,0.9}{
            \makecell[{{p{\linewidth}}}]{
                \texttt{\tiny{[P1$\langle$GM]}}
                \texttt{Deine Nachricht wurde an deinen Mitspieler übermittelt.} \\
\\ 
\\ 
\texttt{Ihr seid gerade bei Runde 10 von maximal 20.} \\
\texttt{Ihr habt gemeinsam 4 von maximal 12 Strafpunkten angesammelt.} \\
\texttt{Dein Mitspieler hat dir diese Nachricht geschickt:} \\
\\ 
\\ 
\texttt{"Ja, die Position (7,1) ist bei mir frei, weil ich mein 'C' bereits dorthin bewegt habe. Du kannst also dein 'C' nach (7,1) bewegen, aber ich denke, es wäre besser, wenn du es an eine andere Position bewegst, damit es nicht mit meinem 'C' kollidiert. Wie wäre es, wenn du dein 'C' nach (7,2) bewegst? Dann kannst du dein 'W' nach (4,4) bewegen und ich kann mein 'W' nach (4,4) bewegen, sobald du bestätigt hast, dass die Position frei ist."} \\
\\ 
\\ 
\texttt{Wie lautet dein nächster Befehl?} \\
            }
        }
    }
    & \\ \\

    \theutterance \stepcounter{utterance}  
    & \multicolumn{4}{p{0.6\linewidth}}{
        \cellcolor[rgb]{0.9,0.9,0.9}{
            \makecell[{{p{\linewidth}}}]{
                \texttt{\tiny{[P1$\rangle$GM]}}
                \texttt{VERSCHIEBE: C, (7,2)} \\
            }
        }
    }
    & & \\ \\

    \theutterance \stepcounter{utterance}  
    & & & \multicolumn{2}{p{0.3\linewidth}}{
        \cellcolor[rgb]{0.9,0.9,0.9}{
            \makecell[{{p{\linewidth}}}]{
                \texttt{\tiny{[GM$|$GM]}}
                \texttt{VERSCHIEBE: C, (7,2)} \\
            }
        }
    }
    & & \\ \\

    \theutterance \stepcounter{utterance}  
    & & & \multicolumn{2}{p{0.3\linewidth}}{
        \cellcolor[rgb]{0.9,0.9,0.9}{
            \makecell[{{p{\linewidth}}}]{
                \texttt{\tiny{[GM$|$GM]}}
                \texttt{Strafpunkt: (7,2) ist nicht leer, sondern enthält '─'.} \\
\\ 
\texttt{Ihr habt gemeinsam 5 von maximal 12 Strafpunkten angesammelt.} \\
\texttt{Bitte versuche es noch mal!} \\
            }
        }
    }
    & & \\ \\

    \theutterance \stepcounter{utterance}  
    & & \multicolumn{4}{p{0.6\linewidth}}{
        \cellcolor[rgb]{0.9,0.9,0.9}{
            \makecell[{{p{\linewidth}}}]{
                \texttt{\tiny{[P1$\langle$GM]}}
                \texttt{Strafpunkt: (7,2) ist nicht leer, sondern enthält '─'.} \\
\\ 
\texttt{Ihr habt gemeinsam 5 von maximal 12 Strafpunkten angesammelt.} \\
\texttt{Bitte versuche es noch mal!} \\
            }
        }
    }
    & \\ \\

    \theutterance \stepcounter{utterance}  
    & \multicolumn{4}{p{0.6\linewidth}}{
        \cellcolor[rgb]{0.9,0.9,0.9}{
            \makecell[{{p{\linewidth}}}]{
                \texttt{\tiny{[P1$\rangle$GM]}}
                \texttt{SPRICH: Ich habe versucht, mein 'C' nach (7,2) zu bewegen, aber es ist nicht leer. Ich denke, wir sollten eine andere Position für mein 'C' finden. Wie wäre es, wenn ich mein 'C' nach (6,4) bewege? Ist diese Position bei dir frei? Ich möchte sicherstellen, dass ich mein 'C' an eine Position bewege, die nicht mit anderen Objekten kollidiert. Bitte lass mich wissen, ob (6,4) frei ist, bevor ich den Zug ausführe.} \\
            }
        }
    }
    & & \\ \\

    \theutterance \stepcounter{utterance}  
    & & & \multicolumn{2}{p{0.3\linewidth}}{
        \cellcolor[rgb]{0.9,0.9,0.9}{
            \makecell[{{p{\linewidth}}}]{
                \texttt{\tiny{[GM$|$GM]}}
                \texttt{SPRICH: Ich habe versucht, mein 'C' nach (7,2) zu bewegen, aber es ist nicht leer. Ich denke, wir sollten eine andere Position für mein 'C' finden. Wie wäre es, wenn ich mein 'C' nach (6,4) bewege? Ist diese Position bei dir frei? Ich möchte sicherstellen, dass ich mein 'C' an eine Position bewege, die nicht mit anderen Objekten kollidiert. Bitte lass mich wissen, ob (6,4) frei ist, bevor ich den Zug ausführe.} \\
            }
        }
    }
    & & \\ \\

    \theutterance \stepcounter{utterance}  
    & & \multicolumn{4}{p{0.6\linewidth}}{
        \cellcolor[rgb]{0.9,0.9,0.9}{
            \makecell[{{p{\linewidth}}}]{
                \texttt{\tiny{[GM$\rangle$P2]}}
                \texttt{Deine Nachricht wurde an deinen Mitspieler übermittelt.} \\
\\ 
\\ 
\texttt{Ihr seid gerade bei Runde 11 von maximal 20.} \\
\texttt{Ihr habt gemeinsam 5 von maximal 12 Strafpunkten angesammelt.} \\
\texttt{Dein Mitspieler hat dir diese Nachricht geschickt:} \\
\\ 
\\ 
\texttt{"Ich habe versucht, mein 'C' nach (7,2) zu bewegen, aber es ist nicht leer. Ich denke, wir sollten eine andere Position für mein 'C' finden. Wie wäre es, wenn ich mein 'C' nach (6,4) bewege? Ist diese Position bei dir frei? Ich möchte sicherstellen, dass ich mein 'C' an eine Position bewege, die nicht mit anderen Objekten kollidiert. Bitte lass mich wissen, ob (6,4) frei ist, bevor ich den Zug ausführe."} \\
\\ 
\\ 
\texttt{Wie lautet dein nächster Befehl?} \\
            }
        }
    }
    & \\ \\

    \theutterance \stepcounter{utterance}  
    & & & \multicolumn{4}{p{0.6\linewidth}}{
        \cellcolor[rgb]{0.9,0.9,0.9}{
            \makecell[{{p{\linewidth}}}]{
                \texttt{\tiny{[GM$\langle$P2]}}
                \texttt{SPRICH: Ja, die Position (6,4) ist bei mir frei. Du kannst dein 'C' dorthin bewegen, ohne dass es mit anderen Objekten kollidiert. Bitte geh vor und bewege dein 'C' nach (6,4), und dann können wir weitermachen, um unsere Objekte an die gewünschten Positionen zu bewegen. Ich werde warten, bis du bestätigt hast, dass du dein 'C' bewegt hast, bevor ich meinen nächsten Zug mache.} \\
            }
        }
    }
     \\ \\

    \theutterance \stepcounter{utterance}  
    & & & \multicolumn{2}{p{0.3\linewidth}}{
        \cellcolor[rgb]{0.9,0.9,0.9}{
            \makecell[{{p{\linewidth}}}]{
                \texttt{\tiny{[GM$|$GM]}}
                \texttt{SPRICH: Ja, die Position (6,4) ist bei mir frei. Du kannst dein 'C' dorthin bewegen, ohne dass es mit anderen Objekten kollidiert. Bitte geh vor und bewege dein 'C' nach (6,4), und dann können wir weitermachen, um unsere Objekte an die gewünschten Positionen zu bewegen. Ich werde warten, bis du bestätigt hast, dass du dein 'C' bewegt hast, bevor ich meinen nächsten Zug mache.} \\
            }
        }
    }
    & & \\ \\

    \theutterance \stepcounter{utterance}  
    & & \multicolumn{4}{p{0.6\linewidth}}{
        \cellcolor[rgb]{0.9,0.9,0.9}{
            \makecell[{{p{\linewidth}}}]{
                \texttt{\tiny{[P1$\langle$GM]}}
                \texttt{Deine Nachricht wurde an deinen Mitspieler übermittelt.} \\
\\ 
\\ 
\texttt{Ihr seid gerade bei Runde 11 von maximal 20.} \\
\texttt{Ihr habt gemeinsam 5 von maximal 12 Strafpunkten angesammelt.} \\
\texttt{Dein Mitspieler hat dir diese Nachricht geschickt:} \\
\\ 
\\ 
\texttt{"Ja, die Position (6,4) ist bei mir frei. Du kannst dein 'C' dorthin bewegen, ohne dass es mit anderen Objekten kollidiert. Bitte geh vor und bewege dein 'C' nach (6,4), und dann können wir weitermachen, um unsere Objekte an die gewünschten Positionen zu bewegen. Ich werde warten, bis du bestätigt hast, dass du dein 'C' bewegt hast, bevor ich meinen nächsten Zug mache."} \\
\\ 
\\ 
\texttt{Wie lautet dein nächster Befehl?} \\
            }
        }
    }
    & \\ \\

    \theutterance \stepcounter{utterance}  
    & \multicolumn{4}{p{0.6\linewidth}}{
        \cellcolor[rgb]{0.9,0.9,0.9}{
            \makecell[{{p{\linewidth}}}]{
                \texttt{\tiny{[P1$\rangle$GM]}}
                \texttt{VERSCHIEBE: C, (6,4)} \\
            }
        }
    }
    & & \\ \\

    \theutterance \stepcounter{utterance}  
    & & & \multicolumn{2}{p{0.3\linewidth}}{
        \cellcolor[rgb]{0.9,0.9,0.9}{
            \makecell[{{p{\linewidth}}}]{
                \texttt{\tiny{[GM$|$GM]}}
                \texttt{VERSCHIEBE: C, (6,4)} \\
            }
        }
    }
    & & \\ \\

    \theutterance \stepcounter{utterance}  
    & & & \multicolumn{2}{p{0.3\linewidth}}{
        \cellcolor[rgb]{0.9,0.9,0.9}{
            \makecell[{{p{\linewidth}}}]{
                \texttt{\tiny{[GM$|$GM]}}
                \texttt{Strafpunkt: (6,4) ist nicht leer, sondern enthält '┼'.} \\
\\ 
\texttt{Ihr habt gemeinsam 6 von maximal 12 Strafpunkten angesammelt.} \\
\texttt{Bitte versuche es noch mal!} \\
            }
        }
    }
    & & \\ \\

    \theutterance \stepcounter{utterance}  
    & & \multicolumn{4}{p{0.6\linewidth}}{
        \cellcolor[rgb]{0.9,0.9,0.9}{
            \makecell[{{p{\linewidth}}}]{
                \texttt{\tiny{[P1$\langle$GM]}}
                \texttt{Strafpunkt: (6,4) ist nicht leer, sondern enthält '┼'.} \\
\\ 
\texttt{Ihr habt gemeinsam 6 von maximal 12 Strafpunkten angesammelt.} \\
\texttt{Bitte versuche es noch mal!} \\
            }
        }
    }
    & \\ \\

    \theutterance \stepcounter{utterance}  
    & \multicolumn{4}{p{0.6\linewidth}}{
        \cellcolor[rgb]{0.9,0.9,0.9}{
            \makecell[{{p{\linewidth}}}]{
                \texttt{\tiny{[P1$\rangle$GM]}}
                \texttt{SPRICH: Ich habe wieder einen Fehler gemacht! Ich habe versucht, mein 'C' nach (6,4) zu bewegen, aber es ist nicht leer. Ich denke, wir sollten eine andere Position für mein 'C' finden. Wie wäre es, wenn ich mein 'C' nach (5,5) bewege? Ist diese Position bei dir frei? Ich möchte sicherstellen, dass ich mein 'C' an eine Position bewege, die nicht mit anderen Objekten kollidiert. Bitte lass mich wissen, ob (5,5) frei ist, bevor ich den Zug ausführe.} \\
            }
        }
    }
    & & \\ \\

    \theutterance \stepcounter{utterance}  
    & & & \multicolumn{2}{p{0.3\linewidth}}{
        \cellcolor[rgb]{0.9,0.9,0.9}{
            \makecell[{{p{\linewidth}}}]{
                \texttt{\tiny{[GM$|$GM]}}
                \texttt{SPRICH: Ich habe wieder einen Fehler gemacht! Ich habe versucht, mein 'C' nach (6,4) zu bewegen, aber es ist nicht leer. Ich denke, wir sollten eine andere Position für mein 'C' finden. Wie wäre es, wenn ich mein 'C' nach (5,5) bewege? Ist diese Position bei dir frei? Ich möchte sicherstellen, dass ich mein 'C' an eine Position bewege, die nicht mit anderen Objekten kollidiert. Bitte lass mich wissen, ob (5,5) frei ist, bevor ich den Zug ausführe.} \\
            }
        }
    }
    & & \\ \\

    \theutterance \stepcounter{utterance}  
    & & \multicolumn{4}{p{0.6\linewidth}}{
        \cellcolor[rgb]{0.9,0.9,0.9}{
            \makecell[{{p{\linewidth}}}]{
                \texttt{\tiny{[GM$\rangle$P2]}}
                \texttt{Deine Nachricht wurde an deinen Mitspieler übermittelt.} \\
\\ 
\\ 
\texttt{Ihr seid gerade bei Runde 12 von maximal 20.} \\
\texttt{Ihr habt gemeinsam 6 von maximal 12 Strafpunkten angesammelt.} \\
\texttt{Dein Mitspieler hat dir diese Nachricht geschickt:} \\
\\ 
\\ 
\texttt{"Ich habe wieder einen Fehler gemacht! Ich habe versucht, mein 'C' nach (6,4) zu bewegen, aber es ist nicht leer. Ich denke, wir sollten eine andere Position für mein 'C' finden. Wie wäre es, wenn ich mein 'C' nach (5,5) bewege? Ist diese Position bei dir frei? Ich möchte sicherstellen, dass ich mein 'C' an eine Position bewege, die nicht mit anderen Objekten kollidiert. Bitte lass mich wissen, ob (5,5) frei ist, bevor ich den Zug ausführe."} \\
\\ 
\\ 
\texttt{Wie lautet dein nächster Befehl?} \\
            }
        }
    }
    & \\ \\

    \theutterance \stepcounter{utterance}  
    & & & \multicolumn{4}{p{0.6\linewidth}}{
        \cellcolor[rgb]{0.9,0.9,0.9}{
            \makecell[{{p{\linewidth}}}]{
                \texttt{\tiny{[GM$\langle$P2]}}
                \texttt{SPRICH: Nein, die Position (5,5) ist bei mir nicht frei. Mein 'T' ist bei (2,2), mein 'W' bei (3,3), mein 'H' bei (3,2), mein 'I' bei (7,3) und mein 'C' bei (7,1). Ich denke, wir sollten eine andere Position für dein 'C' finden, die frei ist. Wie wäre es, wenn du dein 'C' nach (5,7) bewegst? Bitte überprüfe, ob diese Position frei ist, bevor du den Zug ausführst.} \\
            }
        }
    }
     \\ \\

    \theutterance \stepcounter{utterance}  
    & & & \multicolumn{2}{p{0.3\linewidth}}{
        \cellcolor[rgb]{0.9,0.9,0.9}{
            \makecell[{{p{\linewidth}}}]{
                \texttt{\tiny{[GM$|$GM]}}
                \texttt{SPRICH: Nein, die Position (5,5) ist bei mir nicht frei. Mein 'T' ist bei (2,2), mein 'W' bei (3,3), mein 'H' bei (3,2), mein 'I' bei (7,3) und mein 'C' bei (7,1). Ich denke, wir sollten eine andere Position für dein 'C' finden, die frei ist. Wie wäre es, wenn du dein 'C' nach (5,7) bewegst? Bitte überprüfe, ob diese Position frei ist, bevor du den Zug ausführst.} \\
            }
        }
    }
    & & \\ \\

    \theutterance \stepcounter{utterance}  
    & & \multicolumn{4}{p{0.6\linewidth}}{
        \cellcolor[rgb]{0.9,0.9,0.9}{
            \makecell[{{p{\linewidth}}}]{
                \texttt{\tiny{[P1$\langle$GM]}}
                \texttt{Deine Nachricht wurde an deinen Mitspieler übermittelt.} \\
\\ 
\\ 
\texttt{Ihr seid gerade bei Runde 12 von maximal 20.} \\
\texttt{Ihr habt gemeinsam 6 von maximal 12 Strafpunkten angesammelt.} \\
\texttt{Dein Mitspieler hat dir diese Nachricht geschickt:} \\
\\ 
\\ 
\texttt{"Nein, die Position (5,5) ist bei mir nicht frei. Mein 'T' ist bei (2,2), mein 'W' bei (3,3), mein 'H' bei (3,2), mein 'I' bei (7,3) und mein 'C' bei (7,1). Ich denke, wir sollten eine andere Position für dein 'C' finden, die frei ist. Wie wäre es, wenn du dein 'C' nach (5,7) bewegst? Bitte überprüfe, ob diese Position frei ist, bevor du den Zug ausführst."} \\
\\ 
\\ 
\texttt{Wie lautet dein nächster Befehl?} \\
            }
        }
    }
    & \\ \\

    \theutterance \stepcounter{utterance}  
    & \multicolumn{4}{p{0.6\linewidth}}{
        \cellcolor[rgb]{0.9,0.9,0.9}{
            \makecell[{{p{\linewidth}}}]{
                \texttt{\tiny{[P1$\rangle$GM]}}
                \texttt{VERSCHIEBE: C, (5,7)} \\
            }
        }
    }
    & & \\ \\

    \theutterance \stepcounter{utterance}  
    & & & \multicolumn{2}{p{0.3\linewidth}}{
        \cellcolor[rgb]{0.9,0.9,0.9}{
            \makecell[{{p{\linewidth}}}]{
                \texttt{\tiny{[GM$|$GM]}}
                \texttt{VERSCHIEBE: C, (5,7)} \\
            }
        }
    }
    & & \\ \\

    \theutterance \stepcounter{utterance}  
    & & & \multicolumn{2}{p{0.3\linewidth}}{
        \cellcolor[rgb]{0.9,0.9,0.9}{
            \makecell[{{p{\linewidth}}}]{
                \texttt{\tiny{[GM$|$GM]}}
                \texttt{'C' erfolgreich nach (5,7) verschoben. Dein Raster sieht jetzt so aus:} \\
\texttt{```} \\
\texttt{ 1234567} \\
\texttt{╔══╤════╗} \\
\texttt{║◌◌│◌◌IW║ 1} \\
\texttt{╟──┘◌◌┌─╢ 2} \\
\texttt{║◌◌◌◌H│◌║ 3} \\
\texttt{╟─┐◌┌─┼─╢ 4} \\
\texttt{║◌├─┤◌│T║ 5} \\
\texttt{╟─┤◌├─┤◌║ 6} \\
\texttt{║◌│◌│C│◌║ 7} \\
\texttt{╚═╧═╧═╧═╝} \\
\texttt{```} \\
            }
        }
    }
    & & \\ \\

    \theutterance \stepcounter{utterance}  
    & & \multicolumn{4}{p{0.6\linewidth}}{
        \cellcolor[rgb]{0.9,0.9,0.9}{
            \makecell[{{p{\linewidth}}}]{
                \texttt{\tiny{[GM$\rangle$P2]}}
                \texttt{Deine Nachricht wurde an deinen Mitspieler übermittelt.} \\
\\ 
\\ 
\texttt{Ihr seid gerade bei Runde 13 von maximal 20.} \\
\texttt{Ihr habt gemeinsam 6 von maximal 12 Strafpunkten angesammelt.} \\
\texttt{Dein Mitspieler hat ein Objekt auf seinem Raster bewegt.} \\
\\ 
\\ 
\texttt{Wie lautet dein nächster Befehl?} \\
            }
        }
    }
    & \\ \\

    \theutterance \stepcounter{utterance}  
    & & & \multicolumn{4}{p{0.6\linewidth}}{
        \cellcolor[rgb]{0.9,0.9,0.9}{
            \makecell[{{p{\linewidth}}}]{
                \texttt{\tiny{[GM$\langle$P2]}}
                \texttt{SPRICH: Ich denke, es ist an der Zeit, dass ich mein 'W' bewege. Da mein 'C' bereits bei (7,1) ist und deine Positionen sich geändert haben, möchte ich sicherstellen, dass ich mein 'W' an die richtige Position bewege. Kannst du mir bitte bestätigen, ob die Position (4,4) bei dir frei ist und ob es sicher ist, mein 'W' dorthin zu bewegen? Ich werde warten, bis du antwortest, bevor ich den Zug mache.} \\
            }
        }
    }
     \\ \\

    \theutterance \stepcounter{utterance}  
    & & & \multicolumn{2}{p{0.3\linewidth}}{
        \cellcolor[rgb]{0.9,0.9,0.9}{
            \makecell[{{p{\linewidth}}}]{
                \texttt{\tiny{[GM$|$GM]}}
                \texttt{SPRICH: Ich denke, es ist an der Zeit, dass ich mein 'W' bewege. Da mein 'C' bereits bei (7,1) ist und deine Positionen sich geändert haben, möchte ich sicherstellen, dass ich mein 'W' an die richtige Position bewege. Kannst du mir bitte bestätigen, ob die Position (4,4) bei dir frei ist und ob es sicher ist, mein 'W' dorthin zu bewegen? Ich werde warten, bis du antwortest, bevor ich den Zug mache.} \\
            }
        }
    }
    & & \\ \\

    \theutterance \stepcounter{utterance}  
    & & \multicolumn{4}{p{0.6\linewidth}}{
        \cellcolor[rgb]{0.9,0.9,0.9}{
            \makecell[{{p{\linewidth}}}]{
                \texttt{\tiny{[P1$\langle$GM]}}
                \texttt{'C' erfolgreich nach (5,7) verschoben. Dein Raster sieht jetzt so aus:} \\
\texttt{```} \\
\texttt{ 1234567} \\
\texttt{╔══╤════╗} \\
\texttt{║◌◌│◌◌IW║ 1} \\
\texttt{╟──┘◌◌┌─╢ 2} \\
\texttt{║◌◌◌◌H│◌║ 3} \\
\texttt{╟─┐◌┌─┼─╢ 4} \\
\texttt{║◌├─┤◌│T║ 5} \\
\texttt{╟─┤◌├─┤◌║ 6} \\
\texttt{║◌│◌│C│◌║ 7} \\
\texttt{╚═╧═╧═╧═╝} \\
\texttt{```} \\
\texttt{Ihr seid gerade bei Runde 13 von maximal 20.} \\
\texttt{Ihr habt gemeinsam 6 von maximal 12 Strafpunkten angesammelt.} \\
\texttt{Dein Mitspieler hat dir diese Nachricht geschickt:} \\
\\ 
\\ 
\texttt{"Ich denke, es ist an der Zeit, dass ich mein 'W' bewege. Da mein 'C' bereits bei (7,1) ist und deine Positionen sich geändert haben, möchte ich sicherstellen, dass ich mein 'W' an die richtige Position bewege. Kannst du mir bitte bestätigen, ob die Position (4,4) bei dir frei ist und ob es sicher ist, mein 'W' dorthin zu bewegen? Ich werde warten, bis du antwortest, bevor ich den Zug mache."} \\
\\ 
\\ 
\texttt{Wie lautet dein nächster Befehl?} \\
            }
        }
    }
    & \\ \\

    \theutterance \stepcounter{utterance}  
    & \multicolumn{4}{p{0.6\linewidth}}{
        \cellcolor[rgb]{0.9,0.9,0.9}{
            \makecell[{{p{\linewidth}}}]{
                \texttt{\tiny{[P1$\rangle$GM]}}
                \texttt{SPRICH: Ja, ich kann bestätigen, dass die Position (4,4) bei mir frei ist, da ich mein 'C' nach (5,7) verschoben habe. Es ist sicher, dein 'W' nach (4,4) zu bewegen. Bitte geh vor und bewege dein 'W' nach (4,4). Ich werde dann mein 'W' auch nach (4,4) bewegen, um unsere Objekte an die gewünschten Positionen zu bringen. Ich werde warten, bis du bestätigt hast, dass du dein 'W' bewegt hast, bevor ich mein 'W' bewege.} \\
            }
        }
    }
    & & \\ \\

    \theutterance \stepcounter{utterance}  
    & & & \multicolumn{2}{p{0.3\linewidth}}{
        \cellcolor[rgb]{0.9,0.9,0.9}{
            \makecell[{{p{\linewidth}}}]{
                \texttt{\tiny{[GM$|$GM]}}
                \texttt{SPRICH: Ja, ich kann bestätigen, dass die Position (4,4) bei mir frei ist, da ich mein 'C' nach (5,7) verschoben habe. Es ist sicher, dein 'W' nach (4,4) zu bewegen. Bitte geh vor und bewege dein 'W' nach (4,4). Ich werde dann mein 'W' auch nach (4,4) bewegen, um unsere Objekte an die gewünschten Positionen zu bringen. Ich werde warten, bis du bestätigt hast, dass du dein 'W' bewegt hast, bevor ich mein 'W' bewege.} \\
            }
        }
    }
    & & \\ \\

    \theutterance \stepcounter{utterance}  
    & & \multicolumn{4}{p{0.6\linewidth}}{
        \cellcolor[rgb]{0.9,0.9,0.9}{
            \makecell[{{p{\linewidth}}}]{
                \texttt{\tiny{[GM$\rangle$P2]}}
                \texttt{Deine Nachricht wurde an deinen Mitspieler übermittelt.} \\
\\ 
\\ 
\texttt{Ihr seid gerade bei Runde 14 von maximal 20.} \\
\texttt{Ihr habt gemeinsam 6 von maximal 12 Strafpunkten angesammelt.} \\
\texttt{Dein Mitspieler hat dir diese Nachricht geschickt:} \\
\\ 
\\ 
\texttt{"Ja, ich kann bestätigen, dass die Position (4,4) bei mir frei ist, da ich mein 'C' nach (5,7) verschoben habe. Es ist sicher, dein 'W' nach (4,4) zu bewegen. Bitte geh vor und bewege dein 'W' nach (4,4). Ich werde dann mein 'W' auch nach (4,4) bewegen, um unsere Objekte an die gewünschten Positionen zu bringen. Ich werde warten, bis du bestätigt hast, dass du dein 'W' bewegt hast, bevor ich mein 'W' bewege."} \\
\\ 
\\ 
\texttt{Wie lautet dein nächster Befehl?} \\
            }
        }
    }
    & \\ \\

    \theutterance \stepcounter{utterance}  
    & & & \multicolumn{4}{p{0.6\linewidth}}{
        \cellcolor[rgb]{0.9,0.9,0.9}{
            \makecell[{{p{\linewidth}}}]{
                \texttt{\tiny{[GM$\langle$P2]}}
                \texttt{VERSCHIEBE: W, (4,4)} \\
            }
        }
    }
     \\ \\

    \theutterance \stepcounter{utterance}  
    & & & \multicolumn{2}{p{0.3\linewidth}}{
        \cellcolor[rgb]{0.9,0.9,0.9}{
            \makecell[{{p{\linewidth}}}]{
                \texttt{\tiny{[GM$|$GM]}}
                \texttt{VERSCHIEBE: W, (4,4)} \\
            }
        }
    }
    & & \\ \\

    \theutterance \stepcounter{utterance}  
    & & & \multicolumn{2}{p{0.3\linewidth}}{
        \cellcolor[rgb]{0.9,0.9,0.9}{
            \makecell[{{p{\linewidth}}}]{
                \texttt{\tiny{[GM$|$GM]}}
                \texttt{Strafpunkt: (4,4) ist nicht leer, sondern enthält '┌'.} \\
\\ 
\texttt{Ihr habt gemeinsam 7 von maximal 12 Strafpunkten angesammelt.} \\
\texttt{Bitte versuche es noch mal!} \\
            }
        }
    }
    & & \\ \\

    \theutterance \stepcounter{utterance}  
    & & \multicolumn{4}{p{0.6\linewidth}}{
        \cellcolor[rgb]{0.9,0.9,0.9}{
            \makecell[{{p{\linewidth}}}]{
                \texttt{\tiny{[GM$\rangle$P2]}}
                \texttt{Strafpunkt: (4,4) ist nicht leer, sondern enthält '┌'.} \\
\\ 
\texttt{Ihr habt gemeinsam 7 von maximal 12 Strafpunkten angesammelt.} \\
\texttt{Bitte versuche es noch mal!} \\
            }
        }
    }
    & \\ \\

    \theutterance \stepcounter{utterance}  
    & & & \multicolumn{4}{p{0.6\linewidth}}{
        \cellcolor[rgb]{0.9,0.9,0.9}{
            \makecell[{{p{\linewidth}}}]{
                \texttt{\tiny{[GM$\langle$P2]}}
                \texttt{SPRICH: Es sieht so aus, als ob (4,4) wieder nicht leer ist. Ich denke, wir sollten einen anderen Plan entwickeln, um unsere Objekte an die gewünschten Positionen zu bringen. Kannst du mir bitte helfen, eine alternative Position für mein 'W' zu finden, die frei ist und außerdem Teil unseres ursprünglichen Plans ist? Vielleicht können wir mein 'W' an eine andere Position bewegen, die uns hilft, unsere Ziele zu erreichen. Lass uns gemeinsam nachdenken und eine Lösung finden.} \\
            }
        }
    }
     \\ \\

    \theutterance \stepcounter{utterance}  
    & & & \multicolumn{2}{p{0.3\linewidth}}{
        \cellcolor[rgb]{0.9,0.9,0.9}{
            \makecell[{{p{\linewidth}}}]{
                \texttt{\tiny{[GM$|$GM]}}
                \texttt{SPRICH: Es sieht so aus, als ob (4,4) wieder nicht leer ist. Ich denke, wir sollten einen anderen Plan entwickeln, um unsere Objekte an die gewünschten Positionen zu bringen. Kannst du mir bitte helfen, eine alternative Position für mein 'W' zu finden, die frei ist und außerdem Teil unseres ursprünglichen Plans ist? Vielleicht können wir mein 'W' an eine andere Position bewegen, die uns hilft, unsere Ziele zu erreichen. Lass uns gemeinsam nachdenken und eine Lösung finden.} \\
            }
        }
    }
    & & \\ \\

    \theutterance \stepcounter{utterance}  
    & & \multicolumn{4}{p{0.6\linewidth}}{
        \cellcolor[rgb]{0.9,0.9,0.9}{
            \makecell[{{p{\linewidth}}}]{
                \texttt{\tiny{[P1$\langle$GM]}}
                \texttt{Deine Nachricht wurde an deinen Mitspieler übermittelt.} \\
\\ 
\\ 
\texttt{Ihr seid gerade bei Runde 14 von maximal 20.} \\
\texttt{Ihr habt gemeinsam 7 von maximal 12 Strafpunkten angesammelt.} \\
\texttt{Dein Mitspieler hat dir diese Nachricht geschickt:} \\
\\ 
\\ 
\texttt{"Es sieht so aus, als ob (4,4) wieder nicht leer ist. Ich denke, wir sollten einen anderen Plan entwickeln, um unsere Objekte an die gewünschten Positionen zu bringen. Kannst du mir bitte helfen, eine alternative Position für mein 'W' zu finden, die frei ist und außerdem Teil unseres ursprünglichen Plans ist? Vielleicht können wir mein 'W' an eine andere Position bewegen, die uns hilft, unsere Ziele zu erreichen. Lass uns gemeinsam nachdenken und eine Lösung finden."} \\
\\ 
\\ 
\texttt{Wie lautet dein nächster Befehl?} \\
            }
        }
    }
    & \\ \\

    \theutterance \stepcounter{utterance}  
    & \multicolumn{4}{p{0.6\linewidth}}{
        \cellcolor[rgb]{0.9,0.9,0.9}{
            \makecell[{{p{\linewidth}}}]{
                \texttt{\tiny{[P1$\rangle$GM]}}
                \texttt{SPRICH: Ich denke, wir sollten uns noch einmal auf unseren ursprünglichen Plan konzentrieren und versuchen, eine alternative Lösung zu finden. Wie wäre es, wenn wir mein 'W' nach (4,4) bewegen und dann dein 'W' in der Nähe platzieren? Oder vielleicht können wir ein anderes Objekt verschieben, um Platz für dein 'W' zu schaffen? Ich denke, wir sollten auch unsere aktuellen Positionen noch einmal überprüfen und sicherstellen, dass wir dieselbe Vorstellung davon haben, wie unsere Objekte aktuell platziert sind. Mein aktuelles Raster sieht so aus:} \\
\texttt{```} \\
\texttt{ 1234567} \\
\texttt{╔══╤════╗} \\
\texttt{║◌◌│◌◌IW║ 1} \\
\texttt{╟──┘◌◌┌─╢ 2} \\
\texttt{║◌◌◌◌H│◌║ 3} \\
\texttt{╟─┐◌┌─┼─╢ 4} \\
\texttt{║◌├─┤◌│T║ 5} \\
\texttt{╟─┤◌├─┤◌║ 6} \\
\texttt{║◌│◌│C│◌║ 7} \\
\texttt{╚═╧═╧═╧═╝} \\
\texttt{```} \\
\texttt{Bitte teile mir deine aktuellen Positionen mit, und wir können gemeinsam eine Lösung finden.} \\
            }
        }
    }
    & & \\ \\

    \theutterance \stepcounter{utterance}  
    & & & \multicolumn{2}{p{0.3\linewidth}}{
        \cellcolor[rgb]{0.9,0.9,0.9}{
            \makecell[{{p{\linewidth}}}]{
                \texttt{\tiny{[GM$|$GM]}}
                \texttt{SPRICH: Ich denke, wir sollten uns noch einmal auf unseren ursprünglichen Plan konzentrieren und versuchen, eine alternative Lösung zu finden. Wie wäre es, wenn wir mein 'W' nach (4,4) bewegen und dann dein 'W' in der Nähe platzieren? Oder vielleicht können wir ein anderes Objekt verschieben, um Platz für dein 'W' zu schaffen? Ich denke, wir sollten auch unsere aktuellen Positionen noch einmal überprüfen und sicherstellen, dass wir dieselbe Vorstellung davon haben, wie unsere Objekte aktuell platziert sind. Mein aktuelles Raster sieht so aus:} \\
\texttt{```} \\
\texttt{ 1234567} \\
\texttt{╔══╤════╗} \\
\texttt{║◌◌│◌◌IW║ 1} \\
\texttt{╟──┘◌◌┌─╢ 2} \\
\texttt{║◌◌◌◌H│◌║ 3} \\
\texttt{╟─┐◌┌─┼─╢ 4} \\
\texttt{║◌├─┤◌│T║ 5} \\
\texttt{╟─┤◌├─┤◌║ 6} \\
\texttt{║◌│◌│C│◌║ 7} \\
\texttt{╚═╧═╧═╧═╝} \\
\texttt{```} \\
\texttt{Bitte teile mir deine aktuellen Positionen mit, und wir können gemeinsam eine Lösung finden.} \\
            }
        }
    }
    & & \\ \\

    \theutterance \stepcounter{utterance}  
    & & & \multicolumn{2}{p{0.3\linewidth}}{
        \cellcolor[rgb]{0.9,0.9,0.9}{
            \makecell[{{p{\linewidth}}}]{
                \texttt{\tiny{[GM$|$GM]}}
                \texttt{Invalid format: tail is not empty: '} \\
\\ 
\texttt{ 1234567} \\
\texttt{╔══╤════╗} \\
\texttt{║◌◌│◌◌IW║ 1} \\
\texttt{╟──┘◌◌┌─╢ 2} \\
\texttt{║◌◌◌◌H│◌║ 3} \\
\texttt{╟─┐◌┌─┼─╢ 4} \\
\texttt{║◌├─┤◌│T║ 5} \\
\texttt{╟─┤◌├─┤◌║ 6} \\
\texttt{║◌│◌│C│◌║ 7} \\
\texttt{╚═╧═╧═╧═╝} \\
\\ 
\texttt{Bitte teile mir deine aktuellen Positionen mit, und wir können gemeinsam eine Lösung finden.} \\
            }
        }
    }
    & & \\ \\

    \theutterance \stepcounter{utterance}  
    & & \multicolumn{4}{p{0.6\linewidth}}{
        \cellcolor[rgb]{0.9,0.9,0.9}{
            \makecell[{{p{\linewidth}}}]{
                \texttt{\tiny{[P1$\langle$GM]}}
                \texttt{Strafpunkt: Deine Nachricht darf nichts nach dem Befehl enthalten! Stelle sicher, dass deine Antwort ausschließlich einen der beiden Befehle `SPRICH: <NACHRICHT>` und `VERSCHIEBE: <OBJECT>, (<X>, <Y>)` und sonst nichts enthält!} \\
\\ 
\texttt{Ihr habt gemeinsam 8 von maximal 12 Strafpunkten angesammelt.} \\
\texttt{Bitte versuche es noch mal!} \\
            }
        }
    }
    & \\ \\

    \theutterance \stepcounter{utterance}  
    & \multicolumn{4}{p{0.6\linewidth}}{
        \cellcolor[rgb]{0.9,0.9,0.9}{
            \makecell[{{p{\linewidth}}}]{
                \texttt{\tiny{[P1$\rangle$GM]}}
                \texttt{SPRICH: Ich denke, wir sollten uns noch einmal auf unseren ursprünglichen Plan konzentrieren und versuchen, eine alternative Lösung zu finden. Wie wäre es, wenn wir mein 'W' nach (4,4) bewegen und dann dein 'W' in der Nähe platzieren? Oder vielleicht können wir ein anderes Objekt verschieben, um Platz für dein 'W' zu schaffen? Ich denke, wir sollten auch unsere aktuellen Positionen noch einmal überprüfen und sicherstellen, dass wir dieselbe Vorstellung davon haben, wie unsere Objekte aktuell platziert sind. Bitte teile mir deine aktuellen Positionen mit, und wir können gemeinsam eine Lösung finden.} \\
            }
        }
    }
    & & \\ \\

    \theutterance \stepcounter{utterance}  
    & & & \multicolumn{2}{p{0.3\linewidth}}{
        \cellcolor[rgb]{0.9,0.9,0.9}{
            \makecell[{{p{\linewidth}}}]{
                \texttt{\tiny{[GM$|$GM]}}
                \texttt{SPRICH: Ich denke, wir sollten uns noch einmal auf unseren ursprünglichen Plan konzentrieren und versuchen, eine alternative Lösung zu finden. Wie wäre es, wenn wir mein 'W' nach (4,4) bewegen und dann dein 'W' in der Nähe platzieren? Oder vielleicht können wir ein anderes Objekt verschieben, um Platz für dein 'W' zu schaffen? Ich denke, wir sollten auch unsere aktuellen Positionen noch einmal überprüfen und sicherstellen, dass wir dieselbe Vorstellung davon haben, wie unsere Objekte aktuell platziert sind. Bitte teile mir deine aktuellen Positionen mit, und wir können gemeinsam eine Lösung finden.} \\
            }
        }
    }
    & & \\ \\

    \theutterance \stepcounter{utterance}  
    & & \multicolumn{4}{p{0.6\linewidth}}{
        \cellcolor[rgb]{0.9,0.9,0.9}{
            \makecell[{{p{\linewidth}}}]{
                \texttt{\tiny{[GM$\rangle$P2]}}
                \texttt{Deine Nachricht wurde an deinen Mitspieler übermittelt.} \\
\\ 
\\ 
\texttt{Ihr seid gerade bei Runde 15 von maximal 20.} \\
\texttt{Ihr habt gemeinsam 8 von maximal 12 Strafpunkten angesammelt.} \\
\texttt{Dein Mitspieler hat dir diese Nachricht geschickt:} \\
\\ 
\\ 
\texttt{"Ich denke, wir sollten uns noch einmal auf unseren ursprünglichen Plan konzentrieren und versuchen, eine alternative Lösung zu finden. Wie wäre es, wenn wir mein 'W' nach (4,4) bewegen und dann dein 'W' in der Nähe platzieren? Oder vielleicht können wir ein anderes Objekt verschieben, um Platz für dein 'W' zu schaffen? Ich denke, wir sollten auch unsere aktuellen Positionen noch einmal überprüfen und sicherstellen, dass wir dieselbe Vorstellung davon haben, wie unsere Objekte aktuell platziert sind. Bitte teile mir deine aktuellen Positionen mit, und wir können gemeinsam eine Lösung finden."} \\
\\ 
\\ 
\texttt{Wie lautet dein nächster Befehl?} \\
            }
        }
    }
    & \\ \\

    \theutterance \stepcounter{utterance}  
    & & & \multicolumn{4}{p{0.6\linewidth}}{
        \cellcolor[rgb]{0.9,0.9,0.9}{
            \makecell[{{p{\linewidth}}}]{
                \texttt{\tiny{[GM$\langle$P2]}}
                \texttt{SPRICH: Mein aktuelles Spielfeld sieht so aus: Das 'T' ist bei (2,2), das 'W' bei (3,3), das 'H' bei (3,2), das 'I' bei (7,3) und das 'C' bei (7,1). Ich denke, es wäre eine gute Idee, wenn wir unsere Objekte neu bewerten und eine alternative Lösung finden, um mein 'W' an die gewünschte Position zu bringen. Ich bin offen für Vorschläge, wie wir unsere Objekte anpassen können, um unsere Ziele zu erreichen. Bitte teile mir deine aktuellen Positionen mit, und wir können gemeinsam eine Lösung finden. Ich denke, es wäre auch hilfreich, wenn wir unsere Ziele noch einmal durchgehen und sicherstellen, dass wir dieselbe Vorstellung davon haben, wie unsere endgültigen Positionen aussehen sollen.} \\
            }
        }
    }
     \\ \\

    \theutterance \stepcounter{utterance}  
    & & & \multicolumn{2}{p{0.3\linewidth}}{
        \cellcolor[rgb]{0.9,0.9,0.9}{
            \makecell[{{p{\linewidth}}}]{
                \texttt{\tiny{[GM$|$GM]}}
                \texttt{SPRICH: Mein aktuelles Spielfeld sieht so aus: Das 'T' ist bei (2,2), das 'W' bei (3,3), das 'H' bei (3,2), das 'I' bei (7,3) und das 'C' bei (7,1). Ich denke, es wäre eine gute Idee, wenn wir unsere Objekte neu bewerten und eine alternative Lösung finden, um mein 'W' an die gewünschte Position zu bringen. Ich bin offen für Vorschläge, wie wir unsere Objekte anpassen können, um unsere Ziele zu erreichen. Bitte teile mir deine aktuellen Positionen mit, und wir können gemeinsam eine Lösung finden. Ich denke, es wäre auch hilfreich, wenn wir unsere Ziele noch einmal durchgehen und sicherstellen, dass wir dieselbe Vorstellung davon haben, wie unsere endgültigen Positionen aussehen sollen.} \\
            }
        }
    }
    & & \\ \\

    \theutterance \stepcounter{utterance}  
    & & \multicolumn{4}{p{0.6\linewidth}}{
        \cellcolor[rgb]{0.9,0.9,0.9}{
            \makecell[{{p{\linewidth}}}]{
                \texttt{\tiny{[P1$\langle$GM]}}
                \texttt{Deine Nachricht wurde an deinen Mitspieler übermittelt.} \\
\\ 
\\ 
\texttt{Ihr seid gerade bei Runde 15 von maximal 20.} \\
\texttt{Ihr habt gemeinsam 8 von maximal 12 Strafpunkten angesammelt.} \\
\texttt{Dein Mitspieler hat dir diese Nachricht geschickt:} \\
\\ 
\\ 
\texttt{"Mein aktuelles Spielfeld sieht so aus: Das 'T' ist bei (2,2), das 'W' bei (3,3), das 'H' bei (3,2), das 'I' bei (7,3) und das 'C' bei (7,1). Ich denke, es wäre eine gute Idee, wenn wir unsere Objekte neu bewerten und eine alternative Lösung finden, um mein 'W' an die gewünschte Position zu bringen. Ich bin offen für Vorschläge, wie wir unsere Objekte anpassen können, um unsere Ziele zu erreichen. Bitte teile mir deine aktuellen Positionen mit, und wir können gemeinsam eine Lösung finden. Ich denke, es wäre auch hilfreich, wenn wir unsere Ziele noch einmal durchgehen und sicherstellen, dass wir dieselbe Vorstellung davon haben, wie unsere endgültigen Positionen aussehen sollen."} \\
\\ 
\\ 
\texttt{Wie lautet dein nächster Befehl?} \\
            }
        }
    }
    & \\ \\

    \theutterance \stepcounter{utterance}  
    & \multicolumn{4}{p{0.6\linewidth}}{
        \cellcolor[rgb]{0.9,0.9,0.9}{
            \makecell[{{p{\linewidth}}}]{
                \texttt{\tiny{[P1$\rangle$GM]}}
                \texttt{SPRICH: Mein aktuelles Raster sieht so aus: } \\
\texttt{```} \\
\texttt{ 1234567} \\
\texttt{╔══╤════╗} \\
\texttt{║◌◌│◌◌IW║ 1} \\
\texttt{╟──┘◌◌┌─╢ 2} \\
\texttt{║◌◌◌◌H│◌║ 3} \\
\texttt{╟─┐◌┌─┼─╢ 4} \\
\texttt{║◌├─┤◌│T║ 5} \\
\texttt{╟─┤◌├─┤◌║ 6} \\
\texttt{║◌│◌│C│◌║ 7} \\
\texttt{╚═╧═╧═╧═╝} \\
\texttt{```} \\
            }
        }
    }
    & & \\ \\

    \theutterance \stepcounter{utterance}  
    & & & \multicolumn{2}{p{0.3\linewidth}}{
        \cellcolor[rgb]{0.9,0.9,0.9}{
            \makecell[{{p{\linewidth}}}]{
                \texttt{\tiny{[GM$|$GM]}}
                \texttt{SPRICH: Mein aktuelles Raster sieht so aus: } \\
\texttt{```} \\
\texttt{ 1234567} \\
\texttt{╔══╤════╗} \\
\texttt{║◌◌│◌◌IW║ 1} \\
\texttt{╟──┘◌◌┌─╢ 2} \\
\texttt{║◌◌◌◌H│◌║ 3} \\
\texttt{╟─┐◌┌─┼─╢ 4} \\
\texttt{║◌├─┤◌│T║ 5} \\
\texttt{╟─┤◌├─┤◌║ 6} \\
\texttt{║◌│◌│C│◌║ 7} \\
\texttt{╚═╧═╧═╧═╝} \\
\texttt{```} \\
            }
        }
    }
    & & \\ \\

    \theutterance \stepcounter{utterance}  
    & & & \multicolumn{2}{p{0.3\linewidth}}{
        \cellcolor[rgb]{0.9,0.9,0.9}{
            \makecell[{{p{\linewidth}}}]{
                \texttt{\tiny{[GM$|$GM]}}
                \texttt{Invalid format: tail is not empty: '} \\
\\ 
\texttt{ 1234567} \\
\texttt{╔══╤════╗} \\
\texttt{║◌◌│◌◌IW║ 1} \\
\texttt{╟──┘◌◌┌─╢ 2} \\
\texttt{║◌◌◌◌H│◌║ 3} \\
\texttt{╟─┐◌┌─┼─╢ 4} \\
\texttt{║◌├─┤◌│T║ 5} \\
\texttt{╟─┤◌├─┤◌║ 6} \\
\texttt{║◌│◌│C│◌║ 7} \\
\texttt{╚═╧═╧═╧═╝} \\
            }
        }
    }
    & & \\ \\

    \theutterance \stepcounter{utterance}  
    & & \multicolumn{4}{p{0.6\linewidth}}{
        \cellcolor[rgb]{0.9,0.9,0.9}{
            \makecell[{{p{\linewidth}}}]{
                \texttt{\tiny{[P1$\langle$GM]}}
                \texttt{Strafpunkt: Deine Nachricht darf nichts nach dem Befehl enthalten! Stelle sicher, dass deine Antwort ausschließlich einen der beiden Befehle `SPRICH: <NACHRICHT>` und `VERSCHIEBE: <OBJECT>, (<X>, <Y>)` und sonst nichts enthält!} \\
\\ 
\texttt{Ihr habt gemeinsam 9 von maximal 12 Strafpunkten angesammelt.} \\
\texttt{Bitte versuche es noch mal!} \\
            }
        }
    }
    & \\ \\

    \theutterance \stepcounter{utterance}  
    & \multicolumn{4}{p{0.6\linewidth}}{
        \cellcolor[rgb]{0.9,0.9,0.9}{
            \makecell[{{p{\linewidth}}}]{
                \texttt{\tiny{[P1$\rangle$GM]}}
                \texttt{SPRICH: Mein aktuelles Raster sieht so aus: } \\
\texttt{```} \\
\texttt{ 1234567} \\
\texttt{╔══╤════╗} \\
\texttt{║◌◌│◌◌IW║ 1} \\
\texttt{╟──┘◌◌┌─╢ 2} \\
\texttt{║◌◌◌◌H│◌║ 3} \\
\texttt{╟─┐◌┌─┼─╢ 4} \\
\texttt{║◌├─┤◌│T║ 5} \\
\texttt{╟─┤◌├─┤◌║ 6} \\
\texttt{║◌│◌│C│◌║ 7} \\
\texttt{╚═╧═╧═╧═╝} \\
\texttt{```} \\
            }
        }
    }
    & & \\ \\

    \theutterance \stepcounter{utterance}  
    & & & \multicolumn{2}{p{0.3\linewidth}}{
        \cellcolor[rgb]{0.9,0.9,0.9}{
            \makecell[{{p{\linewidth}}}]{
                \texttt{\tiny{[GM$|$GM]}}
                \texttt{SPRICH: Mein aktuelles Raster sieht so aus: } \\
\texttt{```} \\
\texttt{ 1234567} \\
\texttt{╔══╤════╗} \\
\texttt{║◌◌│◌◌IW║ 1} \\
\texttt{╟──┘◌◌┌─╢ 2} \\
\texttt{║◌◌◌◌H│◌║ 3} \\
\texttt{╟─┐◌┌─┼─╢ 4} \\
\texttt{║◌├─┤◌│T║ 5} \\
\texttt{╟─┤◌├─┤◌║ 6} \\
\texttt{║◌│◌│C│◌║ 7} \\
\texttt{╚═╧═╧═╧═╝} \\
\texttt{```} \\
            }
        }
    }
    & & \\ \\

    \theutterance \stepcounter{utterance}  
    & & & \multicolumn{2}{p{0.3\linewidth}}{
        \cellcolor[rgb]{0.9,0.9,0.9}{
            \makecell[{{p{\linewidth}}}]{
                \texttt{\tiny{[GM$|$GM]}}
                \texttt{Invalid format: tail is not empty: '} \\
\\ 
\texttt{ 1234567} \\
\texttt{╔══╤════╗} \\
\texttt{║◌◌│◌◌IW║ 1} \\
\texttt{╟──┘◌◌┌─╢ 2} \\
\texttt{║◌◌◌◌H│◌║ 3} \\
\texttt{╟─┐◌┌─┼─╢ 4} \\
\texttt{║◌├─┤◌│T║ 5} \\
\texttt{╟─┤◌├─┤◌║ 6} \\
\texttt{║◌│◌│C│◌║ 7} \\
\texttt{╚═╧═╧═╧═╝} \\
            }
        }
    }
    & & \\ \\

    \theutterance \stepcounter{utterance}  
    & & \multicolumn{4}{p{0.6\linewidth}}{
        \cellcolor[rgb]{0.9,0.9,0.9}{
            \makecell[{{p{\linewidth}}}]{
                \texttt{\tiny{[P1$\langle$GM]}}
                \texttt{Strafpunkt: Deine Nachricht darf nichts nach dem Befehl enthalten! Stelle sicher, dass deine Antwort ausschließlich einen der beiden Befehle `SPRICH: <NACHRICHT>` und `VERSCHIEBE: <OBJECT>, (<X>, <Y>)` und sonst nichts enthält!} \\
\\ 
\texttt{Ihr habt gemeinsam 10 von maximal 12 Strafpunkten angesammelt.} \\
\texttt{Bitte versuche es noch mal!} \\
            }
        }
    }
    & \\ \\

    \theutterance \stepcounter{utterance}  
    & \multicolumn{4}{p{0.6\linewidth}}{
        \cellcolor[rgb]{0.9,0.9,0.9}{
            \makecell[{{p{\linewidth}}}]{
                \texttt{\tiny{[P1$\rangle$GM]}}
                \texttt{SPRICH: Mein aktuelles Raster ist oben links auf dem Spielfeld} \\
            }
        }
    }
    & & \\ \\

    \theutterance \stepcounter{utterance}  
    & & & \multicolumn{2}{p{0.3\linewidth}}{
        \cellcolor[rgb]{0.9,0.9,0.9}{
            \makecell[{{p{\linewidth}}}]{
                \texttt{\tiny{[GM$|$GM]}}
                \texttt{SPRICH: Mein aktuelles Raster ist oben links auf dem Spielfeld} \\
            }
        }
    }
    & & \\ \\

    \theutterance \stepcounter{utterance}  
    & & \multicolumn{4}{p{0.6\linewidth}}{
        \cellcolor[rgb]{0.9,0.9,0.9}{
            \makecell[{{p{\linewidth}}}]{
                \texttt{\tiny{[GM$\rangle$P2]}}
                \texttt{Deine Nachricht wurde an deinen Mitspieler übermittelt.} \\
\\ 
\\ 
\texttt{Ihr seid gerade bei Runde 16 von maximal 20.} \\
\texttt{Ihr habt gemeinsam 10 von maximal 12 Strafpunkten angesammelt.} \\
\texttt{Dein Mitspieler hat dir diese Nachricht geschickt:} \\
\\ 
\\ 
\texttt{"Mein aktuelles Raster ist oben links auf dem Spielfeld"} \\
\\ 
\\ 
\texttt{Wie lautet dein nächster Befehl?} \\
            }
        }
    }
    & \\ \\

    \theutterance \stepcounter{utterance}  
    & & & \multicolumn{4}{p{0.6\linewidth}}{
        \cellcolor[rgb]{0.9,0.9,0.9}{
            \makecell[{{p{\linewidth}}}]{
                \texttt{\tiny{[GM$\langle$P2]}}
                \texttt{SPRICH: Ich denke, es wäre hilfreich, wenn du mir mehr Details über dein aktuelles Raster geben könntest. Welche Objekte sind bei dir oben links auf dem Spielfeld, und wie sind sie angeordnet? Ich möchte sicherstellen, dass wir dieselbe Vorstellung davon haben, wie unsere Objekte aktuell platziert sind, bevor wir weitere Schritte unternehmen. Bitte teile mir mehr Informationen mit, und wir können gemeinsam eine Lösung finden, um unsere Ziele zu erreichen.} \\
            }
        }
    }
     \\ \\

    \theutterance \stepcounter{utterance}  
    & & & \multicolumn{2}{p{0.3\linewidth}}{
        \cellcolor[rgb]{0.9,0.9,0.9}{
            \makecell[{{p{\linewidth}}}]{
                \texttt{\tiny{[GM$|$GM]}}
                \texttt{SPRICH: Ich denke, es wäre hilfreich, wenn du mir mehr Details über dein aktuelles Raster geben könntest. Welche Objekte sind bei dir oben links auf dem Spielfeld, und wie sind sie angeordnet? Ich möchte sicherstellen, dass wir dieselbe Vorstellung davon haben, wie unsere Objekte aktuell platziert sind, bevor wir weitere Schritte unternehmen. Bitte teile mir mehr Informationen mit, und wir können gemeinsam eine Lösung finden, um unsere Ziele zu erreichen.} \\
            }
        }
    }
    & & \\ \\

    \theutterance \stepcounter{utterance}  
    & & \multicolumn{4}{p{0.6\linewidth}}{
        \cellcolor[rgb]{0.9,0.9,0.9}{
            \makecell[{{p{\linewidth}}}]{
                \texttt{\tiny{[P1$\langle$GM]}}
                \texttt{Deine Nachricht wurde an deinen Mitspieler übermittelt.} \\
\\ 
\\ 
\texttt{Ihr seid gerade bei Runde 16 von maximal 20.} \\
\texttt{Ihr habt gemeinsam 10 von maximal 12 Strafpunkten angesammelt.} \\
\texttt{Dein Mitspieler hat dir diese Nachricht geschickt:} \\
\\ 
\\ 
\texttt{"Ich denke, es wäre hilfreich, wenn du mir mehr Details über dein aktuelles Raster geben könntest. Welche Objekte sind bei dir oben links auf dem Spielfeld, und wie sind sie angeordnet? Ich möchte sicherstellen, dass wir dieselbe Vorstellung davon haben, wie unsere Objekte aktuell platziert sind, bevor wir weitere Schritte unternehmen. Bitte teile mir mehr Informationen mit, und wir können gemeinsam eine Lösung finden, um unsere Ziele zu erreichen."} \\
\\ 
\\ 
\texttt{Wie lautet dein nächster Befehl?} \\
            }
        }
    }
    & \\ \\

    \theutterance \stepcounter{utterance}  
    & \multicolumn{4}{p{0.6\linewidth}}{
        \cellcolor[rgb]{0.9,0.9,0.9}{
            \makecell[{{p{\linewidth}}}]{
                \texttt{\tiny{[P1$\rangle$GM]}}
                \texttt{SPRICH: Ich habe ein W und ein I oben links auf dem Spielfeld} \\
            }
        }
    }
    & & \\ \\

    \theutterance \stepcounter{utterance}  
    & & & \multicolumn{2}{p{0.3\linewidth}}{
        \cellcolor[rgb]{0.9,0.9,0.9}{
            \makecell[{{p{\linewidth}}}]{
                \texttt{\tiny{[GM$|$GM]}}
                \texttt{SPRICH: Ich habe ein W und ein I oben links auf dem Spielfeld} \\
            }
        }
    }
    & & \\ \\

    \theutterance \stepcounter{utterance}  
    & & \multicolumn{4}{p{0.6\linewidth}}{
        \cellcolor[rgb]{0.9,0.9,0.9}{
            \makecell[{{p{\linewidth}}}]{
                \texttt{\tiny{[GM$\rangle$P2]}}
                \texttt{Deine Nachricht wurde an deinen Mitspieler übermittelt.} \\
\\ 
\\ 
\texttt{Ihr seid gerade bei Runde 17 von maximal 20.} \\
\texttt{Ihr habt gemeinsam 10 von maximal 12 Strafpunkten angesammelt.} \\
\texttt{Dein Mitspieler hat dir diese Nachricht geschickt:} \\
\\ 
\\ 
\texttt{"Ich habe ein W und ein I oben links auf dem Spielfeld"} \\
\\ 
\\ 
\texttt{Wie lautet dein nächster Befehl?} \\
            }
        }
    }
    & \\ \\

    \theutterance \stepcounter{utterance}  
    & & & \multicolumn{4}{p{0.6\linewidth}}{
        \cellcolor[rgb]{0.9,0.9,0.9}{
            \makecell[{{p{\linewidth}}}]{
                \texttt{\tiny{[GM$\langle$P2]}}
                \texttt{SPRICH: Ich denke, wir sollten uns auf die Endpositionen unserer Objekte konzentrieren und sicherstellen, dass wir dieselbe Vorstellung davon haben, wie unsere endgültigen Positionen aussehen sollen. Ich erinnere mich, dass wir ursprünglich geplant hatten, das 'W' in der Mitte, das 'T' oben links, das 'I' unten rechts, das 'C' oben rechts und das 'H' unten links zu platzieren. Wie sehen deine aktuellen Pläne für die Endpositionen aus? Sind sie noch immer dieselben wie unsere ursprünglichen Pläne?} \\
            }
        }
    }
     \\ \\

    \theutterance \stepcounter{utterance}  
    & & & \multicolumn{2}{p{0.3\linewidth}}{
        \cellcolor[rgb]{0.9,0.9,0.9}{
            \makecell[{{p{\linewidth}}}]{
                \texttt{\tiny{[GM$|$GM]}}
                \texttt{SPRICH: Ich denke, wir sollten uns auf die Endpositionen unserer Objekte konzentrieren und sicherstellen, dass wir dieselbe Vorstellung davon haben, wie unsere endgültigen Positionen aussehen sollen. Ich erinnere mich, dass wir ursprünglich geplant hatten, das 'W' in der Mitte, das 'T' oben links, das 'I' unten rechts, das 'C' oben rechts und das 'H' unten links zu platzieren. Wie sehen deine aktuellen Pläne für die Endpositionen aus? Sind sie noch immer dieselben wie unsere ursprünglichen Pläne?} \\
            }
        }
    }
    & & \\ \\

    \theutterance \stepcounter{utterance}  
    & & \multicolumn{4}{p{0.6\linewidth}}{
        \cellcolor[rgb]{0.9,0.9,0.9}{
            \makecell[{{p{\linewidth}}}]{
                \texttt{\tiny{[P1$\langle$GM]}}
                \texttt{Deine Nachricht wurde an deinen Mitspieler übermittelt.} \\
\\ 
\\ 
\texttt{Ihr seid gerade bei Runde 17 von maximal 20.} \\
\texttt{Ihr habt gemeinsam 10 von maximal 12 Strafpunkten angesammelt.} \\
\texttt{Dein Mitspieler hat dir diese Nachricht geschickt:} \\
\\ 
\\ 
\texttt{"Ich denke, wir sollten uns auf die Endpositionen unserer Objekte konzentrieren und sicherstellen, dass wir dieselbe Vorstellung davon haben, wie unsere endgültigen Positionen aussehen sollen. Ich erinnere mich, dass wir ursprünglich geplant hatten, das 'W' in der Mitte, das 'T' oben links, das 'I' unten rechts, das 'C' oben rechts und das 'H' unten links zu platzieren. Wie sehen deine aktuellen Pläne für die Endpositionen aus? Sind sie noch immer dieselben wie unsere ursprünglichen Pläne?"} \\
\\ 
\\ 
\texttt{Wie lautet dein nächster Befehl?} \\
            }
        }
    }
    & \\ \\

    \theutterance \stepcounter{utterance}  
    & \multicolumn{4}{p{0.6\linewidth}}{
        \cellcolor[rgb]{0.9,0.9,0.9}{
            \makecell[{{p{\linewidth}}}]{
                \texttt{\tiny{[P1$\rangle$GM]}}
                \texttt{SPRICH: Ja, meine Pläne für die Endpositionen sind noch immer dieselben wie unsere ursprünglichen Pläne} \\
            }
        }
    }
    & & \\ \\

    \theutterance \stepcounter{utterance}  
    & & & \multicolumn{2}{p{0.3\linewidth}}{
        \cellcolor[rgb]{0.9,0.9,0.9}{
            \makecell[{{p{\linewidth}}}]{
                \texttt{\tiny{[GM$|$GM]}}
                \texttt{SPRICH: Ja, meine Pläne für die Endpositionen sind noch immer dieselben wie unsere ursprünglichen Pläne} \\
            }
        }
    }
    & & \\ \\

    \theutterance \stepcounter{utterance}  
    & & \multicolumn{4}{p{0.6\linewidth}}{
        \cellcolor[rgb]{0.9,0.9,0.9}{
            \makecell[{{p{\linewidth}}}]{
                \texttt{\tiny{[GM$\rangle$P2]}}
                \texttt{Deine Nachricht wurde an deinen Mitspieler übermittelt.} \\
\\ 
\\ 
\texttt{Ihr seid gerade bei Runde 18 von maximal 20.} \\
\texttt{Ihr habt gemeinsam 10 von maximal 12 Strafpunkten angesammelt.} \\
\texttt{Dein Mitspieler hat dir diese Nachricht geschickt:} \\
\\ 
\\ 
\texttt{"Ja, meine Pläne für die Endpositionen sind noch immer dieselben wie unsere ursprünglichen Pläne"} \\
\\ 
\\ 
\texttt{Wie lautet dein nächster Befehl?} \\
            }
        }
    }
    & \\ \\

    \theutterance \stepcounter{utterance}  
    & & & \multicolumn{4}{p{0.6\linewidth}}{
        \cellcolor[rgb]{0.9,0.9,0.9}{
            \makecell[{{p{\linewidth}}}]{
                \texttt{\tiny{[GM$\langle$P2]}}
                \texttt{SPRICH: Gut, das ist großartig! Da wir dieselben Pläne für die Endpositionen haben, denke ich, wir sollten versuchen, unsere Objekte so schnell wie möglich an diese Positionen zu bewegen. Ich werde mein 'W' versuchen, an die Position zu bewegen, die wir ursprünglich geplant hatten, aber ich brauche deine Hilfe, um sicherzustellen, dass die Position frei ist. Kannst du mir bitte bestätigen, ob die Position (4,4) bei dir immer noch frei ist, oder muss ich eine andere Position für mein 'W' finden? Lass uns gemeinsam unsere Objekte an die richtigen Positionen bewegen und das Spiel zu einem erfolgreichen Abschluss bringen. Wenn wir fertig sind, können wir 'fertig' sagen, um das Spiel zu beenden.} \\
            }
        }
    }
     \\ \\

    \theutterance \stepcounter{utterance}  
    & & & \multicolumn{2}{p{0.3\linewidth}}{
        \cellcolor[rgb]{0.9,0.9,0.9}{
            \makecell[{{p{\linewidth}}}]{
                \texttt{\tiny{[GM$|$GM]}}
                \texttt{SPRICH: Gut, das ist großartig! Da wir dieselben Pläne für die Endpositionen haben, denke ich, wir sollten versuchen, unsere Objekte so schnell wie möglich an diese Positionen zu bewegen. Ich werde mein 'W' versuchen, an die Position zu bewegen, die wir ursprünglich geplant hatten, aber ich brauche deine Hilfe, um sicherzustellen, dass die Position frei ist. Kannst du mir bitte bestätigen, ob die Position (4,4) bei dir immer noch frei ist, oder muss ich eine andere Position für mein 'W' finden? Lass uns gemeinsam unsere Objekte an die richtigen Positionen bewegen und das Spiel zu einem erfolgreichen Abschluss bringen. Wenn wir fertig sind, können wir 'fertig' sagen, um das Spiel zu beenden.} \\
            }
        }
    }
    & & \\ \\

    \theutterance \stepcounter{utterance}  
    & & \multicolumn{4}{p{0.6\linewidth}}{
        \cellcolor[rgb]{0.9,0.9,0.9}{
            \makecell[{{p{\linewidth}}}]{
                \texttt{\tiny{[P1$\langle$GM]}}
                \texttt{Deine Nachricht wurde an deinen Mitspieler übermittelt.} \\
\\ 
\\ 
\texttt{Ihr seid gerade bei Runde 18 von maximal 20.} \\
\texttt{Ihr habt gemeinsam 10 von maximal 12 Strafpunkten angesammelt.} \\
\texttt{Dein Mitspieler hat dir diese Nachricht geschickt:} \\
\\ 
\\ 
\texttt{"Gut, das ist großartig! Da wir dieselben Pläne für die Endpositionen haben, denke ich, wir sollten versuchen, unsere Objekte so schnell wie möglich an diese Positionen zu bewegen. Ich werde mein 'W' versuchen, an die Position zu bewegen, die wir ursprünglich geplant hatten, aber ich brauche deine Hilfe, um sicherzustellen, dass die Position frei ist. Kannst du mir bitte bestätigen, ob die Position (4,4) bei dir immer noch frei ist, oder muss ich eine andere Position für mein 'W' finden? Lass uns gemeinsam unsere Objekte an die richtigen Positionen bewegen und das Spiel zu einem erfolgreichen Abschluss bringen. Wenn wir fertig sind, können wir 'fertig' sagen, um das Spiel zu beenden."} \\
\\ 
\\ 
\texttt{Wie lautet dein nächster Befehl?} \\
            }
        }
    }
    & \\ \\

    \theutterance \stepcounter{utterance}  
    & \multicolumn{4}{p{0.6\linewidth}}{
        \cellcolor[rgb]{0.9,0.9,0.9}{
            \makecell[{{p{\linewidth}}}]{
                \texttt{\tiny{[P1$\rangle$GM]}}
                \texttt{SPRICH: Ja, die Position (4,4) bei mir ist frei} \\
            }
        }
    }
    & & \\ \\

    \theutterance \stepcounter{utterance}  
    & & & \multicolumn{2}{p{0.3\linewidth}}{
        \cellcolor[rgb]{0.9,0.9,0.9}{
            \makecell[{{p{\linewidth}}}]{
                \texttt{\tiny{[GM$|$GM]}}
                \texttt{SPRICH: Ja, die Position (4,4) bei mir ist frei} \\
            }
        }
    }
    & & \\ \\

    \theutterance \stepcounter{utterance}  
    & & \multicolumn{4}{p{0.6\linewidth}}{
        \cellcolor[rgb]{0.9,0.9,0.9}{
            \makecell[{{p{\linewidth}}}]{
                \texttt{\tiny{[GM$\rangle$P2]}}
                \texttt{Deine Nachricht wurde an deinen Mitspieler übermittelt.} \\
\\ 
\\ 
\texttt{Ihr seid gerade bei Runde 19 von maximal 20.} \\
\texttt{Ihr habt gemeinsam 10 von maximal 12 Strafpunkten angesammelt.} \\
\texttt{Dein Mitspieler hat dir diese Nachricht geschickt:} \\
\\ 
\\ 
\texttt{"Ja, die Position (4,4) bei mir ist frei"} \\
\\ 
\\ 
\texttt{Wie lautet dein nächster Befehl?} \\
            }
        }
    }
    & \\ \\

    \theutterance \stepcounter{utterance}  
    & & & \multicolumn{4}{p{0.6\linewidth}}{
        \cellcolor[rgb]{0.9,0.9,0.9}{
            \makecell[{{p{\linewidth}}}]{
                \texttt{\tiny{[GM$\langle$P2]}}
                \texttt{SPRICH: Wir sollten uns vergewissern, dass alle Objekte an ihren richtigen Positionen sind, bevor wir das Spiel beenden. Ich denke, wir sollten unsere aktuellen Positionen noch einmal überprüfen und sicherstellen, dass wir dieselbe Vorstellung davon haben, wie unsere Objekte aktuell platziert sind. Bitte teile mir mit, ob alle deine Objekte an ihren richtigen Positionen sind, und ich werde dasselbe tun. Wenn alles stimmt, können wir 'fertig' sagen, um das Spiel zu beenden.} \\
            }
        }
    }
     \\ \\

    \theutterance \stepcounter{utterance}  
    & & & \multicolumn{2}{p{0.3\linewidth}}{
        \cellcolor[rgb]{0.9,0.9,0.9}{
            \makecell[{{p{\linewidth}}}]{
                \texttt{\tiny{[GM$|$GM]}}
                \texttt{SPRICH: Wir sollten uns vergewissern, dass alle Objekte an ihren richtigen Positionen sind, bevor wir das Spiel beenden. Ich denke, wir sollten unsere aktuellen Positionen noch einmal überprüfen und sicherstellen, dass wir dieselbe Vorstellung davon haben, wie unsere Objekte aktuell platziert sind. Bitte teile mir mit, ob alle deine Objekte an ihren richtigen Positionen sind, und ich werde dasselbe tun. Wenn alles stimmt, können wir 'fertig' sagen, um das Spiel zu beenden.} \\
            }
        }
    }
    & & \\ \\

    \theutterance \stepcounter{utterance}  
    & & \multicolumn{4}{p{0.6\linewidth}}{
        \cellcolor[rgb]{0.9,0.9,0.9}{
            \makecell[{{p{\linewidth}}}]{
                \texttt{\tiny{[P1$\langle$GM]}}
                \texttt{Deine Nachricht wurde an deinen Mitspieler übermittelt.} \\
\\ 
\\ 
\texttt{Ihr seid gerade bei Runde 19 von maximal 20.} \\
\texttt{Ihr habt gemeinsam 10 von maximal 12 Strafpunkten angesammelt.} \\
\texttt{Dein Mitspieler hat dir diese Nachricht geschickt:} \\
\\ 
\\ 
\texttt{"Wir sollten uns vergewissern, dass alle Objekte an ihren richtigen Positionen sind, bevor wir das Spiel beenden. Ich denke, wir sollten unsere aktuellen Positionen noch einmal überprüfen und sicherstellen, dass wir dieselbe Vorstellung davon haben, wie unsere Objekte aktuell platziert sind. Bitte teile mir mit, ob alle deine Objekte an ihren richtigen Positionen sind, und ich werde dasselbe tun. Wenn alles stimmt, können wir 'fertig' sagen, um das Spiel zu beenden."} \\
\\ 
\\ 
\texttt{Wie lautet dein nächster Befehl?} \\
            }
        }
    }
    & \\ \\

    \theutterance \stepcounter{utterance}  
    & \multicolumn{4}{p{0.6\linewidth}}{
        \cellcolor[rgb]{0.9,0.9,0.9}{
            \makecell[{{p{\linewidth}}}]{
                \texttt{\tiny{[P1$\rangle$GM]}}
                \texttt{SPRICH: Ja, alle meine Objekte sind an ihren richtigen Positionen, ich sage jetzt fertig?} \\
            }
        }
    }
    & & \\ \\

    \theutterance \stepcounter{utterance}  
    & & & \multicolumn{2}{p{0.3\linewidth}}{
        \cellcolor[rgb]{0.9,0.9,0.9}{
            \makecell[{{p{\linewidth}}}]{
                \texttt{\tiny{[GM$|$GM]}}
                \texttt{SPRICH: Ja, alle meine Objekte sind an ihren richtigen Positionen, ich sage jetzt fertig?} \\
            }
        }
    }
    & & \\ \\

    \theutterance \stepcounter{utterance}  
    & & \multicolumn{4}{p{0.6\linewidth}}{
        \cellcolor[rgb]{0.9,0.9,0.9}{
            \makecell[{{p{\linewidth}}}]{
                \texttt{\tiny{[GM$\rangle$P2]}}
                \texttt{Deine Nachricht wurde an deinen Mitspieler übermittelt.} \\
\\ 
\\ 
\texttt{Ihr seid gerade bei Runde 20 von maximal 20.} \\
\texttt{Ihr habt gemeinsam 10 von maximal 12 Strafpunkten angesammelt.} \\
\texttt{Dein Mitspieler hat dir diese Nachricht geschickt:} \\
\\ 
\\ 
\texttt{"Ja, alle meine Objekte sind an ihren richtigen Positionen, ich sage jetzt fertig?"} \\
\\ 
\\ 
\texttt{Wie lautet dein nächster Befehl?} \\
            }
        }
    }
    & \\ \\

    \theutterance \stepcounter{utterance}  
    & & & \multicolumn{4}{p{0.6\linewidth}}{
        \cellcolor[rgb]{0.9,0.9,0.9}{
            \makecell[{{p{\linewidth}}}]{
                \texttt{\tiny{[GM$\langle$P2]}}
                \texttt{SPRICH: Ja, sage "fertig!" Ich denke, wir haben unser Bestes getan, um unsere Objekte an die richtigen Positionen zu bringen, und wir sollten das Spiel jetzt beenden. Ich sage auch "fertig!", um das Spiel offiziell zu beenden.} \\
            }
        }
    }
     \\ \\

    \theutterance \stepcounter{utterance}  
    & & & \multicolumn{2}{p{0.3\linewidth}}{
        \cellcolor[rgb]{0.9,0.9,0.9}{
            \makecell[{{p{\linewidth}}}]{
                \texttt{\tiny{[GM$|$GM]}}
                \texttt{SPRICH: Ja, sage "fertig!" Ich denke, wir haben unser Bestes getan, um unsere Objekte an die richtigen Positionen zu bringen, und wir sollten das Spiel jetzt beenden. Ich sage auch "fertig!", um das Spiel offiziell zu beenden.} \\
            }
        }
    }
    & & \\ \\

    \theutterance \stepcounter{utterance}  
    & & & \multicolumn{2}{p{0.3\linewidth}}{
        \cellcolor[rgb]{0.9,0.9,0.9}{
            \makecell[{{p{\linewidth}}}]{
                \texttt{\tiny{[GM$|$GM]}}
                \texttt{true} \\
            }
        }
    }
    & & \\ \\

    \theutterance \stepcounter{utterance}  
    & & & \multicolumn{2}{p{0.3\linewidth}}{
        \cellcolor[rgb]{0.9,0.9,0.9}{
            \makecell[{{p{\linewidth}}}]{
                \texttt{\tiny{[GM$|$GM]}}
                \texttt{* success: True} \\
\texttt{* lose: True} \\
\texttt{* aborted: False} \\
\texttt{{-}{-}{-}{-}{-}{-}{-}} \\
\texttt{* Shifts: 0.00} \\
\texttt{* Max Shifts: 8.00} \\
\texttt{* Min Shifts: 4.00} \\
\texttt{* End Distance Sum: 22.24} \\
\texttt{* Init Distance Sum: 19.52} \\
\texttt{* Expected Distance Sum: 20.95} \\
\texttt{* Penalties: 10.00} \\
\texttt{* Max Penalties: 12.00} \\
\texttt{* Rounds: 20.00} \\
\texttt{* Max Rounds: 20.00} \\
\texttt{* Object Count: 5.00} \\
            }
        }
    }
    & & \\ \\

    \theutterance \stepcounter{utterance}  
    & & & \multicolumn{2}{p{0.3\linewidth}}{
        \cellcolor[rgb]{0.9,0.9,0.9}{
            \makecell[{{p{\linewidth}}}]{
                \texttt{\tiny{[GM$|$GM]}}
                \texttt{Initial Grids:} \\
\texttt{```} \\
\texttt{Player 1:     Player 2:} \\
\texttt{ 1234567       1234567} \\
\texttt{╔══╤════╗     ╔══╤════╗} \\
\texttt{║◌◌│◌◌IW║ 1   ║◌◌│◌◌◌◌║ 1} \\
\texttt{╟──┘◌◌┌─╢ 2   ╟──┘◌T┌─╢ 2} \\
\texttt{║◌◌◌◌H│◌║ 3   ║◌◌HW◌│◌║ 3} \\
\texttt{╟─┐C┌─┼─╢ 4   ╟─┐◌┌─┼─╢ 4} \\
\texttt{║◌├─┤◌│T║ 5   ║◌├─┤◌│◌║ 5} \\
\texttt{╟─┤◌├─┤◌║ 6   ╟─┤◌├─┤◌║ 6} \\
\texttt{║◌│◌│◌│◌║ 7   ║◌│I│C│◌║ 7} \\
\texttt{╚═╧═╧═╧═╝     ╚═╧═╧═╧═╝} \\
\texttt{```} \\
\\ 
\texttt{Final Grids:} \\
\texttt{```} \\
\texttt{Player 1:     Player 2:} \\
\texttt{ 1234567       1234567} \\
\texttt{╔══╤════╗     ╔══╤════╗} \\
\texttt{║◌◌│◌◌IW║ 1   ║◌◌│◌◌◌C║ 1} \\
\texttt{╟──┘◌◌┌─╢ 2   ╟──┘◌T┌─╢ 2} \\
\texttt{║◌◌◌◌H│◌║ 3   ║◌◌HW◌│◌║ 3} \\
\texttt{╟─┐◌┌─┼─╢ 4   ╟─┐◌┌─┼─╢ 4} \\
\texttt{║◌├─┤◌│T║ 5   ║◌├─┤◌│◌║ 5} \\
\texttt{╟─┤◌├─┤◌║ 6   ╟─┤◌├─┤◌║ 6} \\
\texttt{║◌│◌│C│◌║ 7   ║◌│I│◌│◌║ 7} \\
\texttt{╚═╧═╧═╧═╝     ╚═╧═╧═╧═╝} \\
\texttt{```} \\
            }
        }
    }
    & & \\ \\

\end{supertabular}
}

\end{document}
