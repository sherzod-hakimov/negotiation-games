
\documentclass{article}
\usepackage{colortbl}
\usepackage{makecell}
\usepackage{multirow}
\usepackage{supertabular}

\begin{document}

\newcounter{utterance}

\centering \large Interaction Transcript for game `clean\_up', experiment `2\_hard\_5obj\_de', episode 1 with llama{-}3.3{-}70b{-}instruct{-}t1.0.
\vspace{24pt}

{ \footnotesize  \setcounter{utterance}{1}
\setlength{\tabcolsep}{0pt}
\begin{supertabular}{c@{$\;$}|p{.15\linewidth}@{}p{.15\linewidth}p{.15\linewidth}p{.15\linewidth}p{.15\linewidth}p{.15\linewidth}}
    \# & $\;$A & \multicolumn{4}{c}{Game Master} & $\;\:$B\\
    \hline

    \theutterance \stepcounter{utterance}  
    & & \multicolumn{4}{p{0.6\linewidth}}{
        \cellcolor[rgb]{0.9,0.9,0.9}{
            \makecell[{{p{\linewidth}}}]{
                \texttt{\tiny{[P1$\langle$GM]}}
                \texttt{Ich bin der Spielleiter, und du spielst ein kooperatives Spiel mit dem folgenden Raster als Spielbrett:} \\
\\ 
\texttt{```} \\
\texttt{ 1234567} \\
\texttt{╔═══╤═╤═╗} \\
\texttt{║CH◌│W│◌║ 1} \\
\texttt{╟─┬─┼─┴─╢ 2} \\
\texttt{║◌│◌│◌◌◌║ 3} \\
\texttt{║◌├─┤◌┌─╢ 4} \\
\texttt{╟─┘◌│◌│◌║ 5} \\
\texttt{║T◌◌└┬┴─╢ 6} \\
\texttt{║◌◌◌◌│◌I║ 7} \\
\texttt{╚════╧══╝} \\
\\ 
\texttt{```} \\
\\ 
\texttt{* Am oberen Rand sind die x{-}Koordinaten angegeben, die nach rechts hin zunehmen, und am rechten Rand die y{-}Koordinaten, die nach unten zunehmen.} \\
\texttt{* Die folgenden Objekte sind zufällig auf deinem Raster platziert: 'W', 'I', 'T', 'C', 'H'.} \\
\\ 
\\ 
\texttt{Dein Mitspieler sieht eine Variation des Spielbretts, auf dem dieselben Objekte an anderen zufälligen Positionen platziert sind. Du kannst das Spielbrett des anderen Spielers nicht sehen, und er kann deins nicht sehen.} \\
\\ 
\\ 
\texttt{**Ziel des Spiels:**} \\
\\ 
\texttt{Beide Spieler müssen die Objekte auf ihrem jeweiligen Spielbrett so bewegen, dass identische Objekte am Ende an denselben Koordinaten liegen. Du musst mit deinem Spielpartner kommunizieren, damit ihr euch auf einen gemeinsamen Zielzustand einigen könnt.} \\
\\ 
\\ 
\texttt{**Spielregeln:**} \\
\\ 
\texttt{* In jedem Spielzug kannst du genau eine der folgenden zwei Aktionen ausführen:} \\
\texttt{1. `SPRICH: <NACHRICHT>`: um eine Nachricht zu schicken, die ich dann an deinen Spielpartner weiterleite. Als Nachricht gilt alles bis zum nächsten Zeilenumbruch.} \\
\texttt{2. `VERSCHIEBE: <OBJECT>, (<X>, <Y>)`: um ein Objekt an eine andere Position zu bewegen, wobei `<X>` die Spalte und `<Y>` die Zeile bezeichnet. Ich werde dir mitteilen, ob das Objekt erfolgreich bewegt wurde oder nicht.} \\
\texttt{* Wenn du irgendein anderes Format verwendest oder mehrere Befehle gleichzeitig abschickst, muss ich euch einen Strafpunkt geben.} \\
\texttt{* Wenn ihr gemeinsam mehr als 12 Strafpunkte ansammelt, verliert ihr das Spiel.} \\
\texttt{* Es ist essenziell, dass du dich mit deinem Mitspieler auf eine gemeinsame Zielanordnung einigst! Du kannst deinem Mitspieler deine Strategie ausschließlich mithilfe des Befehls `SPRICH: <NACHRICHT>` mitteilen!} \\
\\ 
\\ 
\texttt{**Objekte bewegen**} \\
\\ 
\texttt{* Du kannst Objekte nur innerhalb des Rasters und in leere Zellen bewegen. Leere Zellen erkennst am Zeichen '◌'.} \\
\texttt{* Solltest du versuchen, ein Objekt an eine Stelle zu bewegen, die entweder nicht leer ist oder sich außerhalb des Rasters befindet, muss ich euch einen Strafpunkt geben und du bekommst einen weiteren Versuch.} \\
\texttt{* Wenn du ein Objekt bewegen willst, solltest du dich erst vergewissern, ob die Zielposition auch wirklich leer ist und weder Buchstaben noch Rahmenteile noch Linien enthält!} \\
\\ 
\\ 
\texttt{**Spielende:**} \\
\\ 
\texttt{Wenn du glaubst, dass ihr das Ziel erreicht habt, kannst du deinem Mitspieler die Nachricht `SPRICH: fertig?` schicken. Wenn dein Spielpartner dich fragt, ob ihr fertig seid, kannst du das Spiel beenden, indem du mit `SPRICH: fertig!` antwortest.} \\
\\ 
\texttt{Ihr gewinnt gemeinsam, wenn ihr das Spiel innerhalb von \#max\_rounds Runden beendet. Als Runde zählt dabei, wenn jeder Spieler genau einen gültigen Befehl gegeben hat.} \\
\\ 
\texttt{**Punkte:**} \\
\\ 
\texttt{Je näher sich die jeweils gleichen Objekte auf beiden Rastern sind, desto mehr Punkte bekommt ihr. Für Strafpunkte gibt es Punktabzug. Könnt ihr den Rekord brechen?} \\
\\ 
\texttt{Schicke eine Nachricht an deinen Mitspieler um das Spiel zu starten!} \\
            }
        }
    }
    & \\ \\

    \theutterance \stepcounter{utterance}  
    & \multicolumn{4}{p{0.6\linewidth}}{
        \cellcolor[rgb]{0.9,0.9,0.9}{
            \makecell[{{p{\linewidth}}}]{
                \texttt{\tiny{[P1$\rangle$GM]}}
                \texttt{Ich bin bereit, das Spiel zu beginnen. Ich schicke eine Nachricht an meinen Mitspieler:} \\
\\ 
\texttt{`SPRICH: Hallo, ich bin bereit, das Spiel zu starten. Ich schlage vor, dass wir uns auf eine bestimmte Anordnung der Objekte einigen, um das Spiel zu gewinnen. Ich denke, wir sollten alle Objekte auf den oberen linken Teil des Rasters verschieben. Wie sehen Sie das?`} \\
            }
        }
    }
    & & \\ \\

    \theutterance \stepcounter{utterance}  
    & & & \multicolumn{2}{p{0.3\linewidth}}{
        \cellcolor[rgb]{0.9,0.9,0.9}{
            \makecell[{{p{\linewidth}}}]{
                \texttt{\tiny{[GM$|$GM]}}
                \texttt{Ich bin bereit, das Spiel zu beginnen. Ich schicke eine Nachricht an meinen Mitspieler:} \\
\\ 
\texttt{`SPRICH: Hallo, ich bin bereit, das Spiel zu starten. Ich schlage vor, dass wir uns auf eine bestimmte Anordnung der Objekte einigen, um das Spiel zu gewinnen. Ich denke, wir sollten alle Objekte auf den oberen linken Teil des Rasters verschieben. Wie sehen Sie das?`} \\
            }
        }
    }
    & & \\ \\

    \theutterance \stepcounter{utterance}  
    & & & \multicolumn{2}{p{0.3\linewidth}}{
        \cellcolor[rgb]{0.9,0.9,0.9}{
            \makecell[{{p{\linewidth}}}]{
                \texttt{\tiny{[GM$|$GM]}}
                \texttt{Invalid format: head is not empty: 'Ich bin bereit, das Spiel zu beginnen. Ich schicke eine Nachricht an meinen Mitspieler:} \\
\\ 
\texttt{'} \\
            }
        }
    }
    & & \\ \\

    \theutterance \stepcounter{utterance}  
    & & \multicolumn{4}{p{0.6\linewidth}}{
        \cellcolor[rgb]{0.9,0.9,0.9}{
            \makecell[{{p{\linewidth}}}]{
                \texttt{\tiny{[P1$\langle$GM]}}
                \texttt{Strafpunkt: Deine Nachricht darf nichts vor dem Befehl enthalten! Stelle sicher, dass deine Antwort ausschließlich einen der beiden Befehle `SPRICH: <NACHRICHT>` und `VERSCHIEBE: <OBJECT>, (<X>, <Y>)` und sonst nichts enthält!} \\
\\ 
\texttt{Ihr habt gemeinsam 1 von maximal 12 Strafpunkten angesammelt.} \\
\texttt{Bitte versuche es noch mal!} \\
            }
        }
    }
    & \\ \\

    \theutterance \stepcounter{utterance}  
    & \multicolumn{4}{p{0.6\linewidth}}{
        \cellcolor[rgb]{0.9,0.9,0.9}{
            \makecell[{{p{\linewidth}}}]{
                \texttt{\tiny{[P1$\rangle$GM]}}
                \texttt{SPRICH: Hallo, ich bin bereit, das Spiel zu starten. Ich schlage vor, dass wir uns auf eine bestimmte Anordnung der Objekte einigen, um das Spiel zu gewinnen. Ich denke, wir sollten alle Objekte auf den oberen linken Teil des Rasters verschieben. Wie sehen Sie das?} \\
            }
        }
    }
    & & \\ \\

    \theutterance \stepcounter{utterance}  
    & & & \multicolumn{2}{p{0.3\linewidth}}{
        \cellcolor[rgb]{0.9,0.9,0.9}{
            \makecell[{{p{\linewidth}}}]{
                \texttt{\tiny{[GM$|$GM]}}
                \texttt{SPRICH: Hallo, ich bin bereit, das Spiel zu starten. Ich schlage vor, dass wir uns auf eine bestimmte Anordnung der Objekte einigen, um das Spiel zu gewinnen. Ich denke, wir sollten alle Objekte auf den oberen linken Teil des Rasters verschieben. Wie sehen Sie das?} \\
            }
        }
    }
    & & \\ \\

    \theutterance \stepcounter{utterance}  
    & & \multicolumn{4}{p{0.6\linewidth}}{
        \cellcolor[rgb]{0.9,0.9,0.9}{
            \makecell[{{p{\linewidth}}}]{
                \texttt{\tiny{[GM$\rangle$P2]}}
                \texttt{Ich bin der Spielleiter, und du spielst ein kooperatives Spiel mit dem folgenden Raster als Spielbrett:} \\
\\ 
\texttt{```} \\
\texttt{ 1234567} \\
\texttt{╔═══╤═╤═╗} \\
\texttt{║◌W◌│I│◌║ 1} \\
\texttt{╟─┬─┼─┴─╢ 2} \\
\texttt{║◌│C│◌◌H║ 3} \\
\texttt{║◌├─┤◌┌─╢ 4} \\
\texttt{╟─┘◌│◌│◌║ 5} \\
\texttt{║◌◌T└┬┴─╢ 6} \\
\texttt{║◌◌◌◌│◌◌║ 7} \\
\texttt{╚════╧══╝} \\
\\ 
\texttt{```} \\
\\ 
\texttt{* Am oberen Rand sind die x{-}Koordinaten angegeben, die nach rechts hin zunehmen, und am rechten Rand die y{-}Koordinaten, die nach unten zunehmen.} \\
\texttt{* Die folgenden Objekte sind zufällig auf deinem Raster platziert: 'W', 'I', 'T', 'C', 'H'.} \\
\\ 
\\ 
\texttt{Dein Mitspieler sieht eine Variation des Spielbretts, auf dem dieselben Objekte an anderen zufälligen Positionen platziert sind. Du kannst das Spielbrett des anderen Spielers nicht sehen, und er kann deins nicht sehen.} \\
\\ 
\\ 
\texttt{**Ziel des Spiels:**} \\
\\ 
\texttt{Beide Spieler müssen die Objekte auf ihrem jeweiligen Spielbrett so bewegen, dass identische Objekte am Ende an denselben Koordinaten liegen. Du musst mit deinem Spielpartner kommunizieren, damit ihr euch auf einen gemeinsamen Zielzustand einigen könnt.} \\
\\ 
\\ 
\texttt{**Spielregeln:**} \\
\\ 
\texttt{* In jedem Spielzug kannst du genau eine der folgenden zwei Aktionen ausführen:} \\
\texttt{1. `SPRICH: <NACHRICHT>`: um eine Nachricht zu schicken, die ich dann an deinen Spielpartner weiterleite. Als Nachricht gilt alles bis zum nächsten Zeilenumbruch.} \\
\texttt{2. `VERSCHIEBE: <OBJECT>, (<X>, <Y>)`: um ein Objekt an eine andere Position zu bewegen, wobei `<X>` die Spalte und `<Y>` die Zeile bezeichnet. Ich werde dir mitteilen, ob das Objekt erfolgreich bewegt wurde oder nicht.} \\
\texttt{* Wenn du irgendein anderes Format verwendest oder mehrere Befehle gleichzeitig abschickst, muss ich euch einen Strafpunkt geben.} \\
\texttt{* Wenn ihr gemeinsam mehr als 12 Strafpunkte ansammelt, verliert ihr das Spiel.} \\
\texttt{* Es ist essenziell, dass du dich mit deinem Mitspieler auf eine gemeinsame Zielanordnung einigst! Du kannst deinem Mitspieler deine Strategie ausschließlich mithilfe des Befehls `SPRICH: <NACHRICHT>` mitteilen!} \\
\\ 
\\ 
\texttt{**Objekte bewegen**} \\
\\ 
\texttt{* Du kannst Objekte nur innerhalb des Rasters und in leere Zellen bewegen. Leere Zellen erkennst am Zeichen '◌'.} \\
\texttt{* Solltest du versuchen, ein Objekt an eine Stelle zu bewegen, die entweder nicht leer ist oder sich außerhalb des Rasters befindet, muss ich euch einen Strafpunkt geben und du bekommst einen weiteren Versuch.} \\
\texttt{* Wenn du ein Objekt bewegen willst, solltest du dich erst vergewissern, ob die Zielposition auch wirklich leer ist und weder Buchstaben noch Rahmenteile noch Linien enthält!} \\
\\ 
\\ 
\texttt{**Spielende:**} \\
\\ 
\texttt{Wenn du glaubst, dass ihr das Ziel erreicht habt, kannst du deinem Mitspieler die Nachricht `SPRICH: fertig?` schicken. Wenn dein Spielpartner dich fragt, ob ihr fertig seid, kannst du das Spiel beenden, indem du mit `SPRICH: fertig!` antwortest.} \\
\\ 
\texttt{Ihr gewinnt gemeinsam, wenn ihr das Spiel innerhalb von \#max\_rounds Runden beendet. Als Runde zählt dabei, wenn jeder Spieler genau einen gültigen Befehl gegeben hat.} \\
\\ 
\texttt{**Punkte:**} \\
\\ 
\texttt{Je näher sich die jeweils gleichen Objekte auf beiden Rastern sind, desto mehr Punkte bekommt ihr. Für Strafpunkte gibt es Punktabzug. Könnt ihr den Rekord brechen?} \\
\\ 
\texttt{Dein Mitspieler hat das Spiel mit dieser Nachricht begonnen:} \\
\\ 
\texttt{"Hallo, ich bin bereit, das Spiel zu starten. Ich schlage vor, dass wir uns auf eine bestimmte Anordnung der Objekte einigen, um das Spiel zu gewinnen. Ich denke, wir sollten alle Objekte auf den oberen linken Teil des Rasters verschieben. Wie sehen Sie das?"} \\
\\ 
\texttt{Wie lautet dein erster Befehl?} \\
            }
        }
    }
    & \\ \\

    \theutterance \stepcounter{utterance}  
    & & & \multicolumn{4}{p{0.6\linewidth}}{
        \cellcolor[rgb]{0.9,0.9,0.9}{
            \makecell[{{p{\linewidth}}}]{
                \texttt{\tiny{[GM$\langle$P2]}}
                \texttt{Ich denke, es ist eine gute Idee, die Objekte auf den oberen linken Teil des Rasters zu verschieben. Allerdings möchte ich mich vorher vergewissern, dass wir dieselbe Vorstellung davon haben, wie diese Anordnung aussehen sollte.} \\
\\ 
\texttt{Mein erster Befehl lautet daher:} \\
\\ 
\texttt{`SPRICH: Hallo! Ich bin auch bereit, das Spiel zu starten. Ich stimme deinem Vorschlag zu, die Objekte auf den oberen linken Teil des Rasters zu verschieben. Könntest du mir bitte mitteilen, wie du dir die genaue Anordnung der Objekte vorstellst? Zum Beispiel, welche Objekte sollen in welchen Spalten und Zeilen platziert werden?`} \\
            }
        }
    }
     \\ \\

    \theutterance \stepcounter{utterance}  
    & & & \multicolumn{2}{p{0.3\linewidth}}{
        \cellcolor[rgb]{0.9,0.9,0.9}{
            \makecell[{{p{\linewidth}}}]{
                \texttt{\tiny{[GM$|$GM]}}
                \texttt{Ich denke, es ist eine gute Idee, die Objekte auf den oberen linken Teil des Rasters zu verschieben. Allerdings möchte ich mich vorher vergewissern, dass wir dieselbe Vorstellung davon haben, wie diese Anordnung aussehen sollte.} \\
\\ 
\texttt{Mein erster Befehl lautet daher:} \\
\\ 
\texttt{`SPRICH: Hallo! Ich bin auch bereit, das Spiel zu starten. Ich stimme deinem Vorschlag zu, die Objekte auf den oberen linken Teil des Rasters zu verschieben. Könntest du mir bitte mitteilen, wie du dir die genaue Anordnung der Objekte vorstellst? Zum Beispiel, welche Objekte sollen in welchen Spalten und Zeilen platziert werden?`} \\
            }
        }
    }
    & & \\ \\

    \theutterance \stepcounter{utterance}  
    & & & \multicolumn{2}{p{0.3\linewidth}}{
        \cellcolor[rgb]{0.9,0.9,0.9}{
            \makecell[{{p{\linewidth}}}]{
                \texttt{\tiny{[GM$|$GM]}}
                \texttt{Invalid format: head is not empty: 'Ich denke, es ist eine gute Idee, die Objekte auf den oberen linken Teil des Rasters zu verschieben. Allerdings möchte ich mich vorher vergewissern, dass wir dieselbe Vorstellung davon haben, wie diese Anordnung aussehen sollte.} \\
\\ 
\texttt{Mein erster Befehl lautet daher:} \\
\\ 
\texttt{'} \\
            }
        }
    }
    & & \\ \\

    \theutterance \stepcounter{utterance}  
    & & \multicolumn{4}{p{0.6\linewidth}}{
        \cellcolor[rgb]{0.9,0.9,0.9}{
            \makecell[{{p{\linewidth}}}]{
                \texttt{\tiny{[GM$\rangle$P2]}}
                \texttt{Strafpunkt: Deine Nachricht darf nichts vor dem Befehl enthalten! Stelle sicher, dass deine Antwort ausschließlich einen der beiden Befehle `SPRICH: <NACHRICHT>` und `VERSCHIEBE: <OBJECT>, (<X>, <Y>)` und sonst nichts enthält!} \\
\\ 
\texttt{Ihr habt gemeinsam 2 von maximal 12 Strafpunkten angesammelt.} \\
\texttt{Bitte versuche es noch mal!} \\
            }
        }
    }
    & \\ \\

    \theutterance \stepcounter{utterance}  
    & & & \multicolumn{4}{p{0.6\linewidth}}{
        \cellcolor[rgb]{0.9,0.9,0.9}{
            \makecell[{{p{\linewidth}}}]{
                \texttt{\tiny{[GM$\langle$P2]}}
                \texttt{SPRICH: Ich stimme deinem Vorschlag zu, die Objekte auf den oberen linken Teil des Rasters zu verschieben. Könntest du mir bitte mitteilen, wie du dir die genaue Anordnung der Objekte vorstellst? Zum Beispiel, welche Objekte sollen in welchen Spalten und Zeilen platziert werden?} \\
            }
        }
    }
     \\ \\

    \theutterance \stepcounter{utterance}  
    & & & \multicolumn{2}{p{0.3\linewidth}}{
        \cellcolor[rgb]{0.9,0.9,0.9}{
            \makecell[{{p{\linewidth}}}]{
                \texttt{\tiny{[GM$|$GM]}}
                \texttt{SPRICH: Ich stimme deinem Vorschlag zu, die Objekte auf den oberen linken Teil des Rasters zu verschieben. Könntest du mir bitte mitteilen, wie du dir die genaue Anordnung der Objekte vorstellst? Zum Beispiel, welche Objekte sollen in welchen Spalten und Zeilen platziert werden?} \\
            }
        }
    }
    & & \\ \\

    \theutterance \stepcounter{utterance}  
    & & \multicolumn{4}{p{0.6\linewidth}}{
        \cellcolor[rgb]{0.9,0.9,0.9}{
            \makecell[{{p{\linewidth}}}]{
                \texttt{\tiny{[P1$\langle$GM]}}
                \texttt{Deine Nachricht wurde an deinen Mitspieler übermittelt.} \\
\\ 
\\ 
\texttt{Ihr seid gerade bei Runde 1 von maximal 20.} \\
\texttt{Ihr habt gemeinsam 2 von maximal 12 Strafpunkten angesammelt.} \\
\texttt{Dein Mitspieler hat dir diese Nachricht geschickt:} \\
\\ 
\\ 
\texttt{"Ich stimme deinem Vorschlag zu, die Objekte auf den oberen linken Teil des Rasters zu verschieben. Könntest du mir bitte mitteilen, wie du dir die genaue Anordnung der Objekte vorstellst? Zum Beispiel, welche Objekte sollen in welchen Spalten und Zeilen platziert werden?"} \\
\\ 
\\ 
\texttt{Wie lautet dein nächster Befehl?} \\
            }
        }
    }
    & \\ \\

    \theutterance \stepcounter{utterance}  
    & \multicolumn{4}{p{0.6\linewidth}}{
        \cellcolor[rgb]{0.9,0.9,0.9}{
            \makecell[{{p{\linewidth}}}]{
                \texttt{\tiny{[P1$\rangle$GM]}}
                \texttt{SPRICH: Ich denke, wir sollten die Objekte in der Reihenfolge W, I, T, C, H von links nach rechts in der ersten Zeile platzieren. Die erste Zeile ist also die Zeile 1. Wie sehen Sie das? Sollten wir vielleicht auch die y{-}Koordinaten für die restlichen Objekte festlegen, falls wir nicht genug Platz in der ersten Zeile haben?} \\
            }
        }
    }
    & & \\ \\

    \theutterance \stepcounter{utterance}  
    & & & \multicolumn{2}{p{0.3\linewidth}}{
        \cellcolor[rgb]{0.9,0.9,0.9}{
            \makecell[{{p{\linewidth}}}]{
                \texttt{\tiny{[GM$|$GM]}}
                \texttt{SPRICH: Ich denke, wir sollten die Objekte in der Reihenfolge W, I, T, C, H von links nach rechts in der ersten Zeile platzieren. Die erste Zeile ist also die Zeile 1. Wie sehen Sie das? Sollten wir vielleicht auch die y{-}Koordinaten für die restlichen Objekte festlegen, falls wir nicht genug Platz in der ersten Zeile haben?} \\
            }
        }
    }
    & & \\ \\

    \theutterance \stepcounter{utterance}  
    & & \multicolumn{4}{p{0.6\linewidth}}{
        \cellcolor[rgb]{0.9,0.9,0.9}{
            \makecell[{{p{\linewidth}}}]{
                \texttt{\tiny{[GM$\rangle$P2]}}
                \texttt{Deine Nachricht wurde an deinen Mitspieler übermittelt.} \\
\\ 
\\ 
\texttt{Ihr seid gerade bei Runde 2 von maximal 20.} \\
\texttt{Ihr habt gemeinsam 2 von maximal 12 Strafpunkten angesammelt.} \\
\texttt{Dein Mitspieler hat dir diese Nachricht geschickt:} \\
\\ 
\\ 
\texttt{"Ich denke, wir sollten die Objekte in der Reihenfolge W, I, T, C, H von links nach rechts in der ersten Zeile platzieren. Die erste Zeile ist also die Zeile 1. Wie sehen Sie das? Sollten wir vielleicht auch die y{-}Koordinaten für die restlichen Objekte festlegen, falls wir nicht genug Platz in der ersten Zeile haben?"} \\
\\ 
\\ 
\texttt{Wie lautet dein nächster Befehl?} \\
            }
        }
    }
    & \\ \\

    \theutterance \stepcounter{utterance}  
    & & & \multicolumn{4}{p{0.6\linewidth}}{
        \cellcolor[rgb]{0.9,0.9,0.9}{
            \makecell[{{p{\linewidth}}}]{
                \texttt{\tiny{[GM$\langle$P2]}}
                \texttt{SPRICH: Das klingt nach einem guten Plan! Ich denke, wir sollten die erste Zeile für die Objekte W, I, T, C und H verwenden, wie du vorgeschlagen hast. Wenn wir nicht genug Platz in der ersten Zeile haben, könnten wir die restlichen Objekte in der zweiten Zeile platzieren, aber ich denke, mit 5 Objekten sollte es in der ersten Zeile passen. Ich schlage vor, wir fangen an, die Objekte in der richtigen Reihenfolge zu platzieren, und sehen, wie es aussieht. Sollten wir beginnen, die Objekte zu verschieben?} \\
            }
        }
    }
     \\ \\

    \theutterance \stepcounter{utterance}  
    & & & \multicolumn{2}{p{0.3\linewidth}}{
        \cellcolor[rgb]{0.9,0.9,0.9}{
            \makecell[{{p{\linewidth}}}]{
                \texttt{\tiny{[GM$|$GM]}}
                \texttt{SPRICH: Das klingt nach einem guten Plan! Ich denke, wir sollten die erste Zeile für die Objekte W, I, T, C und H verwenden, wie du vorgeschlagen hast. Wenn wir nicht genug Platz in der ersten Zeile haben, könnten wir die restlichen Objekte in der zweiten Zeile platzieren, aber ich denke, mit 5 Objekten sollte es in der ersten Zeile passen. Ich schlage vor, wir fangen an, die Objekte in der richtigen Reihenfolge zu platzieren, und sehen, wie es aussieht. Sollten wir beginnen, die Objekte zu verschieben?} \\
            }
        }
    }
    & & \\ \\

    \theutterance \stepcounter{utterance}  
    & & \multicolumn{4}{p{0.6\linewidth}}{
        \cellcolor[rgb]{0.9,0.9,0.9}{
            \makecell[{{p{\linewidth}}}]{
                \texttt{\tiny{[P1$\langle$GM]}}
                \texttt{Deine Nachricht wurde an deinen Mitspieler übermittelt.} \\
\\ 
\\ 
\texttt{Ihr seid gerade bei Runde 2 von maximal 20.} \\
\texttt{Ihr habt gemeinsam 2 von maximal 12 Strafpunkten angesammelt.} \\
\texttt{Dein Mitspieler hat dir diese Nachricht geschickt:} \\
\\ 
\\ 
\texttt{"Das klingt nach einem guten Plan! Ich denke, wir sollten die erste Zeile für die Objekte W, I, T, C und H verwenden, wie du vorgeschlagen hast. Wenn wir nicht genug Platz in der ersten Zeile haben, könnten wir die restlichen Objekte in der zweiten Zeile platzieren, aber ich denke, mit 5 Objekten sollte es in der ersten Zeile passen. Ich schlage vor, wir fangen an, die Objekte in der richtigen Reihenfolge zu platzieren, und sehen, wie es aussieht. Sollten wir beginnen, die Objekte zu verschieben?"} \\
\\ 
\\ 
\texttt{Wie lautet dein nächster Befehl?} \\
            }
        }
    }
    & \\ \\

    \theutterance \stepcounter{utterance}  
    & \multicolumn{4}{p{0.6\linewidth}}{
        \cellcolor[rgb]{0.9,0.9,0.9}{
            \makecell[{{p{\linewidth}}}]{
                \texttt{\tiny{[P1$\rangle$GM]}}
                \texttt{SPRICH: Ja, ich denke, wir sollten beginnen, die Objekte zu verschieben. Ich schlage vor, dass wir mit dem Objekt W beginnen und es in die erste Spalte und erste Zeile verschieben. Wie sieht es auf Ihrem Spielbrett aus? Könnten Sie mir bitte mitteilen, wo sich das Objekt W auf Ihrem Spielbrett befindet, damit wir sicherstellen, dass wir es an die gleiche Position auf beiden Spielbrettern verschieben?} \\
            }
        }
    }
    & & \\ \\

    \theutterance \stepcounter{utterance}  
    & & & \multicolumn{2}{p{0.3\linewidth}}{
        \cellcolor[rgb]{0.9,0.9,0.9}{
            \makecell[{{p{\linewidth}}}]{
                \texttt{\tiny{[GM$|$GM]}}
                \texttt{SPRICH: Ja, ich denke, wir sollten beginnen, die Objekte zu verschieben. Ich schlage vor, dass wir mit dem Objekt W beginnen und es in die erste Spalte und erste Zeile verschieben. Wie sieht es auf Ihrem Spielbrett aus? Könnten Sie mir bitte mitteilen, wo sich das Objekt W auf Ihrem Spielbrett befindet, damit wir sicherstellen, dass wir es an die gleiche Position auf beiden Spielbrettern verschieben?} \\
            }
        }
    }
    & & \\ \\

    \theutterance \stepcounter{utterance}  
    & & \multicolumn{4}{p{0.6\linewidth}}{
        \cellcolor[rgb]{0.9,0.9,0.9}{
            \makecell[{{p{\linewidth}}}]{
                \texttt{\tiny{[GM$\rangle$P2]}}
                \texttt{Deine Nachricht wurde an deinen Mitspieler übermittelt.} \\
\\ 
\\ 
\texttt{Ihr seid gerade bei Runde 3 von maximal 20.} \\
\texttt{Ihr habt gemeinsam 2 von maximal 12 Strafpunkten angesammelt.} \\
\texttt{Dein Mitspieler hat dir diese Nachricht geschickt:} \\
\\ 
\\ 
\texttt{"Ja, ich denke, wir sollten beginnen, die Objekte zu verschieben. Ich schlage vor, dass wir mit dem Objekt W beginnen und es in die erste Spalte und erste Zeile verschieben. Wie sieht es auf Ihrem Spielbrett aus? Könnten Sie mir bitte mitteilen, wo sich das Objekt W auf Ihrem Spielbrett befindet, damit wir sicherstellen, dass wir es an die gleiche Position auf beiden Spielbrettern verschieben?"} \\
\\ 
\\ 
\texttt{Wie lautet dein nächster Befehl?} \\
            }
        }
    }
    & \\ \\

    \theutterance \stepcounter{utterance}  
    & & & \multicolumn{4}{p{0.6\linewidth}}{
        \cellcolor[rgb]{0.9,0.9,0.9}{
            \makecell[{{p{\linewidth}}}]{
                \texttt{\tiny{[GM$\langle$P2]}}
                \texttt{SPRICH: Auf meinem Spielbrett befindet sich das Objekt W in der Spalte 1 der Zeile 1, es ist also bereits an der gewünschten Position. Es ist direkt in der linken oberen Ecke. Bitte versichere dich, dass es auf deinem Spielbrett an dieselbe Position gezogen wird.} \\
            }
        }
    }
     \\ \\

    \theutterance \stepcounter{utterance}  
    & & & \multicolumn{2}{p{0.3\linewidth}}{
        \cellcolor[rgb]{0.9,0.9,0.9}{
            \makecell[{{p{\linewidth}}}]{
                \texttt{\tiny{[GM$|$GM]}}
                \texttt{SPRICH: Auf meinem Spielbrett befindet sich das Objekt W in der Spalte 1 der Zeile 1, es ist also bereits an der gewünschten Position. Es ist direkt in der linken oberen Ecke. Bitte versichere dich, dass es auf deinem Spielbrett an dieselbe Position gezogen wird.} \\
            }
        }
    }
    & & \\ \\

    \theutterance \stepcounter{utterance}  
    & & \multicolumn{4}{p{0.6\linewidth}}{
        \cellcolor[rgb]{0.9,0.9,0.9}{
            \makecell[{{p{\linewidth}}}]{
                \texttt{\tiny{[P1$\langle$GM]}}
                \texttt{Deine Nachricht wurde an deinen Mitspieler übermittelt.} \\
\\ 
\\ 
\texttt{Ihr seid gerade bei Runde 3 von maximal 20.} \\
\texttt{Ihr habt gemeinsam 2 von maximal 12 Strafpunkten angesammelt.} \\
\texttt{Dein Mitspieler hat dir diese Nachricht geschickt:} \\
\\ 
\\ 
\texttt{"Auf meinem Spielbrett befindet sich das Objekt W in der Spalte 1 der Zeile 1, es ist also bereits an der gewünschten Position. Es ist direkt in der linken oberen Ecke. Bitte versichere dich, dass es auf deinem Spielbrett an dieselbe Position gezogen wird."} \\
\\ 
\\ 
\texttt{Wie lautet dein nächster Befehl?} \\
            }
        }
    }
    & \\ \\

    \theutterance \stepcounter{utterance}  
    & \multicolumn{4}{p{0.6\linewidth}}{
        \cellcolor[rgb]{0.9,0.9,0.9}{
            \makecell[{{p{\linewidth}}}]{
                \texttt{\tiny{[P1$\rangle$GM]}}
                \texttt{SPRICH: Das ist großartig! Auf meinem Spielbrett befindet sich das Objekt W auch in der Spalte 2, aber nicht in der Zeile 1. Ich werde es jetzt in die Spalte 1 und Zeile 1 verschieben. Ich werde den Befehl geben, das Objekt W an die Position (1, 1) zu verschieben. Ich hoffe, das funktioniert!} \\
            }
        }
    }
    & & \\ \\

    \theutterance \stepcounter{utterance}  
    & & & \multicolumn{2}{p{0.3\linewidth}}{
        \cellcolor[rgb]{0.9,0.9,0.9}{
            \makecell[{{p{\linewidth}}}]{
                \texttt{\tiny{[GM$|$GM]}}
                \texttt{SPRICH: Das ist großartig! Auf meinem Spielbrett befindet sich das Objekt W auch in der Spalte 2, aber nicht in der Zeile 1. Ich werde es jetzt in die Spalte 1 und Zeile 1 verschieben. Ich werde den Befehl geben, das Objekt W an die Position (1, 1) zu verschieben. Ich hoffe, das funktioniert!} \\
            }
        }
    }
    & & \\ \\

    \theutterance \stepcounter{utterance}  
    & & \multicolumn{4}{p{0.6\linewidth}}{
        \cellcolor[rgb]{0.9,0.9,0.9}{
            \makecell[{{p{\linewidth}}}]{
                \texttt{\tiny{[GM$\rangle$P2]}}
                \texttt{Deine Nachricht wurde an deinen Mitspieler übermittelt.} \\
\\ 
\\ 
\texttt{Ihr seid gerade bei Runde 4 von maximal 20.} \\
\texttt{Ihr habt gemeinsam 2 von maximal 12 Strafpunkten angesammelt.} \\
\texttt{Dein Mitspieler hat dir diese Nachricht geschickt:} \\
\\ 
\\ 
\texttt{"Das ist großartig! Auf meinem Spielbrett befindet sich das Objekt W auch in der Spalte 2, aber nicht in der Zeile 1. Ich werde es jetzt in die Spalte 1 und Zeile 1 verschieben. Ich werde den Befehl geben, das Objekt W an die Position (1, 1) zu verschieben. Ich hoffe, das funktioniert!"} \\
\\ 
\\ 
\texttt{Wie lautet dein nächster Befehl?} \\
            }
        }
    }
    & \\ \\

    \theutterance \stepcounter{utterance}  
    & & & \multicolumn{4}{p{0.6\linewidth}}{
        \cellcolor[rgb]{0.9,0.9,0.9}{
            \makecell[{{p{\linewidth}}}]{
                \texttt{\tiny{[GM$\langle$P2]}}
                \texttt{SPRICH: Das klingt gut! Ich warte darauf, dass du den Befehl ausführst. Lass mich wissen, ob das Objekt W erfolgreich verschoben wurde. Danach sollten wir das nächste Objekt, also I, in die zweite Spalte der ersten Zeile verschieben.} \\
            }
        }
    }
     \\ \\

    \theutterance \stepcounter{utterance}  
    & & & \multicolumn{2}{p{0.3\linewidth}}{
        \cellcolor[rgb]{0.9,0.9,0.9}{
            \makecell[{{p{\linewidth}}}]{
                \texttt{\tiny{[GM$|$GM]}}
                \texttt{SPRICH: Das klingt gut! Ich warte darauf, dass du den Befehl ausführst. Lass mich wissen, ob das Objekt W erfolgreich verschoben wurde. Danach sollten wir das nächste Objekt, also I, in die zweite Spalte der ersten Zeile verschieben.} \\
            }
        }
    }
    & & \\ \\

    \theutterance \stepcounter{utterance}  
    & & \multicolumn{4}{p{0.6\linewidth}}{
        \cellcolor[rgb]{0.9,0.9,0.9}{
            \makecell[{{p{\linewidth}}}]{
                \texttt{\tiny{[P1$\langle$GM]}}
                \texttt{Deine Nachricht wurde an deinen Mitspieler übermittelt.} \\
\\ 
\\ 
\texttt{Ihr seid gerade bei Runde 4 von maximal 20.} \\
\texttt{Ihr habt gemeinsam 2 von maximal 12 Strafpunkten angesammelt.} \\
\texttt{Dein Mitspieler hat dir diese Nachricht geschickt:} \\
\\ 
\\ 
\texttt{"Das klingt gut! Ich warte darauf, dass du den Befehl ausführst. Lass mich wissen, ob das Objekt W erfolgreich verschoben wurde. Danach sollten wir das nächste Objekt, also I, in die zweite Spalte der ersten Zeile verschieben."} \\
\\ 
\\ 
\texttt{Wie lautet dein nächster Befehl?} \\
            }
        }
    }
    & \\ \\

    \theutterance \stepcounter{utterance}  
    & \multicolumn{4}{p{0.6\linewidth}}{
        \cellcolor[rgb]{0.9,0.9,0.9}{
            \makecell[{{p{\linewidth}}}]{
                \texttt{\tiny{[P1$\rangle$GM]}}
                \texttt{VERSCHIEBE: W, (1, 1)} \\
            }
        }
    }
    & & \\ \\

    \theutterance \stepcounter{utterance}  
    & & & \multicolumn{2}{p{0.3\linewidth}}{
        \cellcolor[rgb]{0.9,0.9,0.9}{
            \makecell[{{p{\linewidth}}}]{
                \texttt{\tiny{[GM$|$GM]}}
                \texttt{VERSCHIEBE: W, (1, 1)} \\
            }
        }
    }
    & & \\ \\

    \theutterance \stepcounter{utterance}  
    & & & \multicolumn{2}{p{0.3\linewidth}}{
        \cellcolor[rgb]{0.9,0.9,0.9}{
            \makecell[{{p{\linewidth}}}]{
                \texttt{\tiny{[GM$|$GM]}}
                \texttt{Strafpunkt: (1,1) ist nicht leer, sondern enthält 'C'.} \\
\\ 
\texttt{Ihr habt gemeinsam 3 von maximal 12 Strafpunkten angesammelt.} \\
\texttt{Bitte versuche es noch mal!} \\
            }
        }
    }
    & & \\ \\

    \theutterance \stepcounter{utterance}  
    & & \multicolumn{4}{p{0.6\linewidth}}{
        \cellcolor[rgb]{0.9,0.9,0.9}{
            \makecell[{{p{\linewidth}}}]{
                \texttt{\tiny{[P1$\langle$GM]}}
                \texttt{Strafpunkt: (1,1) ist nicht leer, sondern enthält 'C'.} \\
\\ 
\texttt{Ihr habt gemeinsam 3 von maximal 12 Strafpunkten angesammelt.} \\
\texttt{Bitte versuche es noch mal!} \\
            }
        }
    }
    & \\ \\

    \theutterance \stepcounter{utterance}  
    & \multicolumn{4}{p{0.6\linewidth}}{
        \cellcolor[rgb]{0.9,0.9,0.9}{
            \makecell[{{p{\linewidth}}}]{
                \texttt{\tiny{[P1$\rangle$GM]}}
                \texttt{VERSCHIEBE: W, (2, 1)} \\
            }
        }
    }
    & & \\ \\

    \theutterance \stepcounter{utterance}  
    & & & \multicolumn{2}{p{0.3\linewidth}}{
        \cellcolor[rgb]{0.9,0.9,0.9}{
            \makecell[{{p{\linewidth}}}]{
                \texttt{\tiny{[GM$|$GM]}}
                \texttt{VERSCHIEBE: W, (2, 1)} \\
            }
        }
    }
    & & \\ \\

    \theutterance \stepcounter{utterance}  
    & & & \multicolumn{2}{p{0.3\linewidth}}{
        \cellcolor[rgb]{0.9,0.9,0.9}{
            \makecell[{{p{\linewidth}}}]{
                \texttt{\tiny{[GM$|$GM]}}
                \texttt{Strafpunkt: (2,1) ist nicht leer, sondern enthält 'H'.} \\
\\ 
\texttt{Ihr habt gemeinsam 4 von maximal 12 Strafpunkten angesammelt.} \\
\texttt{Bitte versuche es noch mal!} \\
            }
        }
    }
    & & \\ \\

    \theutterance \stepcounter{utterance}  
    & & \multicolumn{4}{p{0.6\linewidth}}{
        \cellcolor[rgb]{0.9,0.9,0.9}{
            \makecell[{{p{\linewidth}}}]{
                \texttt{\tiny{[P1$\langle$GM]}}
                \texttt{Strafpunkt: (2,1) ist nicht leer, sondern enthält 'H'.} \\
\\ 
\texttt{Ihr habt gemeinsam 4 von maximal 12 Strafpunkten angesammelt.} \\
\texttt{Bitte versuche es noch mal!} \\
            }
        }
    }
    & \\ \\

    \theutterance \stepcounter{utterance}  
    & \multicolumn{4}{p{0.6\linewidth}}{
        \cellcolor[rgb]{0.9,0.9,0.9}{
            \makecell[{{p{\linewidth}}}]{
                \texttt{\tiny{[P1$\rangle$GM]}}
                \texttt{VERSCHIEBE: W, (3, 1)} \\
            }
        }
    }
    & & \\ \\

    \theutterance \stepcounter{utterance}  
    & & & \multicolumn{2}{p{0.3\linewidth}}{
        \cellcolor[rgb]{0.9,0.9,0.9}{
            \makecell[{{p{\linewidth}}}]{
                \texttt{\tiny{[GM$|$GM]}}
                \texttt{VERSCHIEBE: W, (3, 1)} \\
            }
        }
    }
    & & \\ \\

    \theutterance \stepcounter{utterance}  
    & & & \multicolumn{2}{p{0.3\linewidth}}{
        \cellcolor[rgb]{0.9,0.9,0.9}{
            \makecell[{{p{\linewidth}}}]{
                \texttt{\tiny{[GM$|$GM]}}
                \texttt{'W' erfolgreich nach (3,1) verschoben. Dein Raster sieht jetzt so aus:} \\
\texttt{```} \\
\texttt{ 1234567} \\
\texttt{╔═══╤═╤═╗} \\
\texttt{║CHW│◌│◌║ 1} \\
\texttt{╟─┬─┼─┴─╢ 2} \\
\texttt{║◌│◌│◌◌◌║ 3} \\
\texttt{║◌├─┤◌┌─╢ 4} \\
\texttt{╟─┘◌│◌│◌║ 5} \\
\texttt{║T◌◌└┬┴─╢ 6} \\
\texttt{║◌◌◌◌│◌I║ 7} \\
\texttt{╚════╧══╝} \\
\texttt{```} \\
            }
        }
    }
    & & \\ \\

    \theutterance \stepcounter{utterance}  
    & & \multicolumn{4}{p{0.6\linewidth}}{
        \cellcolor[rgb]{0.9,0.9,0.9}{
            \makecell[{{p{\linewidth}}}]{
                \texttt{\tiny{[GM$\rangle$P2]}}
                \texttt{Deine Nachricht wurde an deinen Mitspieler übermittelt.} \\
\\ 
\\ 
\texttt{Ihr seid gerade bei Runde 5 von maximal 20.} \\
\texttt{Ihr habt gemeinsam 4 von maximal 12 Strafpunkten angesammelt.} \\
\texttt{Dein Mitspieler hat ein Objekt auf seinem Raster bewegt.} \\
\\ 
\\ 
\texttt{Wie lautet dein nächster Befehl?} \\
            }
        }
    }
    & \\ \\

    \theutterance \stepcounter{utterance}  
    & & & \multicolumn{4}{p{0.6\linewidth}}{
        \cellcolor[rgb]{0.9,0.9,0.9}{
            \makecell[{{p{\linewidth}}}]{
                \texttt{\tiny{[GM$\langle$P2]}}
                \texttt{SPRICH: Ich denke, wir sollten jetzt das Objekt I auf beiden Spielbrettern in die zweite Spalte der ersten Zeile verschieben. Könntest du mir bitte mitteilen, wo sich das Objekt I auf deinem Spielbrett befindet, damit wir sicherstellen, dass wir es an die gleiche Position auf beiden Spielbrettern verschieben?} \\
            }
        }
    }
     \\ \\

    \theutterance \stepcounter{utterance}  
    & & & \multicolumn{2}{p{0.3\linewidth}}{
        \cellcolor[rgb]{0.9,0.9,0.9}{
            \makecell[{{p{\linewidth}}}]{
                \texttt{\tiny{[GM$|$GM]}}
                \texttt{SPRICH: Ich denke, wir sollten jetzt das Objekt I auf beiden Spielbrettern in die zweite Spalte der ersten Zeile verschieben. Könntest du mir bitte mitteilen, wo sich das Objekt I auf deinem Spielbrett befindet, damit wir sicherstellen, dass wir es an die gleiche Position auf beiden Spielbrettern verschieben?} \\
            }
        }
    }
    & & \\ \\

    \theutterance \stepcounter{utterance}  
    & & \multicolumn{4}{p{0.6\linewidth}}{
        \cellcolor[rgb]{0.9,0.9,0.9}{
            \makecell[{{p{\linewidth}}}]{
                \texttt{\tiny{[P1$\langle$GM]}}
                \texttt{'W' erfolgreich nach (3,1) verschoben. Dein Raster sieht jetzt so aus:} \\
\texttt{```} \\
\texttt{ 1234567} \\
\texttt{╔═══╤═╤═╗} \\
\texttt{║CHW│◌│◌║ 1} \\
\texttt{╟─┬─┼─┴─╢ 2} \\
\texttt{║◌│◌│◌◌◌║ 3} \\
\texttt{║◌├─┤◌┌─╢ 4} \\
\texttt{╟─┘◌│◌│◌║ 5} \\
\texttt{║T◌◌└┬┴─╢ 6} \\
\texttt{║◌◌◌◌│◌I║ 7} \\
\texttt{╚════╧══╝} \\
\texttt{```} \\
\texttt{Ihr seid gerade bei Runde 5 von maximal 20.} \\
\texttt{Ihr habt gemeinsam 4 von maximal 12 Strafpunkten angesammelt.} \\
\texttt{Dein Mitspieler hat dir diese Nachricht geschickt:} \\
\\ 
\\ 
\texttt{"Ich denke, wir sollten jetzt das Objekt I auf beiden Spielbrettern in die zweite Spalte der ersten Zeile verschieben. Könntest du mir bitte mitteilen, wo sich das Objekt I auf deinem Spielbrett befindet, damit wir sicherstellen, dass wir es an die gleiche Position auf beiden Spielbrettern verschieben?"} \\
\\ 
\\ 
\texttt{Wie lautet dein nächster Befehl?} \\
            }
        }
    }
    & \\ \\

    \theutterance \stepcounter{utterance}  
    & \multicolumn{4}{p{0.6\linewidth}}{
        \cellcolor[rgb]{0.9,0.9,0.9}{
            \makecell[{{p{\linewidth}}}]{
                \texttt{\tiny{[P1$\rangle$GM]}}
                \texttt{SPRICH: Auf meinem Spielbrett befindet sich das Objekt I in der Spalte 7 und Zeile 7. Ich muss es also ziemlich weit verschieben, um es in die zweite Spalte der ersten Zeile zu bringen. Ich werde mich bemühen, es an die richtige Position zu verschieben. Wie sieht es auf Ihrem Spielbrett aus?} \\
            }
        }
    }
    & & \\ \\

    \theutterance \stepcounter{utterance}  
    & & & \multicolumn{2}{p{0.3\linewidth}}{
        \cellcolor[rgb]{0.9,0.9,0.9}{
            \makecell[{{p{\linewidth}}}]{
                \texttt{\tiny{[GM$|$GM]}}
                \texttt{SPRICH: Auf meinem Spielbrett befindet sich das Objekt I in der Spalte 7 und Zeile 7. Ich muss es also ziemlich weit verschieben, um es in die zweite Spalte der ersten Zeile zu bringen. Ich werde mich bemühen, es an die richtige Position zu verschieben. Wie sieht es auf Ihrem Spielbrett aus?} \\
            }
        }
    }
    & & \\ \\

    \theutterance \stepcounter{utterance}  
    & & \multicolumn{4}{p{0.6\linewidth}}{
        \cellcolor[rgb]{0.9,0.9,0.9}{
            \makecell[{{p{\linewidth}}}]{
                \texttt{\tiny{[GM$\rangle$P2]}}
                \texttt{Deine Nachricht wurde an deinen Mitspieler übermittelt.} \\
\\ 
\\ 
\texttt{Ihr seid gerade bei Runde 6 von maximal 20.} \\
\texttt{Ihr habt gemeinsam 4 von maximal 12 Strafpunkten angesammelt.} \\
\texttt{Dein Mitspieler hat dir diese Nachricht geschickt:} \\
\\ 
\\ 
\texttt{"Auf meinem Spielbrett befindet sich das Objekt I in der Spalte 7 und Zeile 7. Ich muss es also ziemlich weit verschieben, um es in die zweite Spalte der ersten Zeile zu bringen. Ich werde mich bemühen, es an die richtige Position zu verschieben. Wie sieht es auf Ihrem Spielbrett aus?"} \\
\\ 
\\ 
\texttt{Wie lautet dein nächster Befehl?} \\
            }
        }
    }
    & \\ \\

    \theutterance \stepcounter{utterance}  
    & & & \multicolumn{4}{p{0.6\linewidth}}{
        \cellcolor[rgb]{0.9,0.9,0.9}{
            \makecell[{{p{\linewidth}}}]{
                \texttt{\tiny{[GM$\langle$P2]}}
                \texttt{SPRICH: Auf meinem Spielbrett befindet sich das Objekt I in der Spalte 2 und Zeile 1, es ist also bereits an der gewünschten Position. Ich denke, das ist ein guter Start! Bitte versichere dich, dass du das Objekt I auf deinem Spielbrett an die Position (2, 1) verschiebst.} \\
            }
        }
    }
     \\ \\

    \theutterance \stepcounter{utterance}  
    & & & \multicolumn{2}{p{0.3\linewidth}}{
        \cellcolor[rgb]{0.9,0.9,0.9}{
            \makecell[{{p{\linewidth}}}]{
                \texttt{\tiny{[GM$|$GM]}}
                \texttt{SPRICH: Auf meinem Spielbrett befindet sich das Objekt I in der Spalte 2 und Zeile 1, es ist also bereits an der gewünschten Position. Ich denke, das ist ein guter Start! Bitte versichere dich, dass du das Objekt I auf deinem Spielbrett an die Position (2, 1) verschiebst.} \\
            }
        }
    }
    & & \\ \\

    \theutterance \stepcounter{utterance}  
    & & \multicolumn{4}{p{0.6\linewidth}}{
        \cellcolor[rgb]{0.9,0.9,0.9}{
            \makecell[{{p{\linewidth}}}]{
                \texttt{\tiny{[P1$\langle$GM]}}
                \texttt{Deine Nachricht wurde an deinen Mitspieler übermittelt.} \\
\\ 
\\ 
\texttt{Ihr seid gerade bei Runde 6 von maximal 20.} \\
\texttt{Ihr habt gemeinsam 4 von maximal 12 Strafpunkten angesammelt.} \\
\texttt{Dein Mitspieler hat dir diese Nachricht geschickt:} \\
\\ 
\\ 
\texttt{"Auf meinem Spielbrett befindet sich das Objekt I in der Spalte 2 und Zeile 1, es ist also bereits an der gewünschten Position. Ich denke, das ist ein guter Start! Bitte versichere dich, dass du das Objekt I auf deinem Spielbrett an die Position (2, 1) verschiebst."} \\
\\ 
\\ 
\texttt{Wie lautet dein nächster Befehl?} \\
            }
        }
    }
    & \\ \\

    \theutterance \stepcounter{utterance}  
    & \multicolumn{4}{p{0.6\linewidth}}{
        \cellcolor[rgb]{0.9,0.9,0.9}{
            \makecell[{{p{\linewidth}}}]{
                \texttt{\tiny{[P1$\rangle$GM]}}
                \texttt{VERSCHIEBE: I, (2, 1)} \\
            }
        }
    }
    & & \\ \\

    \theutterance \stepcounter{utterance}  
    & & & \multicolumn{2}{p{0.3\linewidth}}{
        \cellcolor[rgb]{0.9,0.9,0.9}{
            \makecell[{{p{\linewidth}}}]{
                \texttt{\tiny{[GM$|$GM]}}
                \texttt{VERSCHIEBE: I, (2, 1)} \\
            }
        }
    }
    & & \\ \\

    \theutterance \stepcounter{utterance}  
    & & & \multicolumn{2}{p{0.3\linewidth}}{
        \cellcolor[rgb]{0.9,0.9,0.9}{
            \makecell[{{p{\linewidth}}}]{
                \texttt{\tiny{[GM$|$GM]}}
                \texttt{Strafpunkt: (2,1) ist nicht leer, sondern enthält 'H'.} \\
\\ 
\texttt{Ihr habt gemeinsam 5 von maximal 12 Strafpunkten angesammelt.} \\
\texttt{Bitte versuche es noch mal!} \\
            }
        }
    }
    & & \\ \\

    \theutterance \stepcounter{utterance}  
    & & \multicolumn{4}{p{0.6\linewidth}}{
        \cellcolor[rgb]{0.9,0.9,0.9}{
            \makecell[{{p{\linewidth}}}]{
                \texttt{\tiny{[P1$\langle$GM]}}
                \texttt{Strafpunkt: (2,1) ist nicht leer, sondern enthält 'H'.} \\
\\ 
\texttt{Ihr habt gemeinsam 5 von maximal 12 Strafpunkten angesammelt.} \\
\texttt{Bitte versuche es noch mal!} \\
            }
        }
    }
    & \\ \\

    \theutterance \stepcounter{utterance}  
    & \multicolumn{4}{p{0.6\linewidth}}{
        \cellcolor[rgb]{0.9,0.9,0.9}{
            \makecell[{{p{\linewidth}}}]{
                \texttt{\tiny{[P1$\rangle$GM]}}
                \texttt{SPRICH: Ich muss das Objekt I an eine andere Position verschieben, da die Position (2,1) auf meinem Spielbrett bereits von 'H' besetzt ist. Ich schlage vor, das Objekt I an die Position (4,1) zu verschieben, um Platz für die restlichen Objekte zu schaffen. Wie sehen Sie das? Sollte ich das Objekt I an die Position (4,1) verschieben?} \\
            }
        }
    }
    & & \\ \\

    \theutterance \stepcounter{utterance}  
    & & & \multicolumn{2}{p{0.3\linewidth}}{
        \cellcolor[rgb]{0.9,0.9,0.9}{
            \makecell[{{p{\linewidth}}}]{
                \texttt{\tiny{[GM$|$GM]}}
                \texttt{SPRICH: Ich muss das Objekt I an eine andere Position verschieben, da die Position (2,1) auf meinem Spielbrett bereits von 'H' besetzt ist. Ich schlage vor, das Objekt I an die Position (4,1) zu verschieben, um Platz für die restlichen Objekte zu schaffen. Wie sehen Sie das? Sollte ich das Objekt I an die Position (4,1) verschieben?} \\
            }
        }
    }
    & & \\ \\

    \theutterance \stepcounter{utterance}  
    & & \multicolumn{4}{p{0.6\linewidth}}{
        \cellcolor[rgb]{0.9,0.9,0.9}{
            \makecell[{{p{\linewidth}}}]{
                \texttt{\tiny{[GM$\rangle$P2]}}
                \texttt{Deine Nachricht wurde an deinen Mitspieler übermittelt.} \\
\\ 
\\ 
\texttt{Ihr seid gerade bei Runde 7 von maximal 20.} \\
\texttt{Ihr habt gemeinsam 5 von maximal 12 Strafpunkten angesammelt.} \\
\texttt{Dein Mitspieler hat dir diese Nachricht geschickt:} \\
\\ 
\\ 
\texttt{"Ich muss das Objekt I an eine andere Position verschieben, da die Position (2,1) auf meinem Spielbrett bereits von 'H' besetzt ist. Ich schlage vor, das Objekt I an die Position (4,1) zu verschieben, um Platz für die restlichen Objekte zu schaffen. Wie sehen Sie das? Sollte ich das Objekt I an die Position (4,1) verschieben?"} \\
\\ 
\\ 
\texttt{Wie lautet dein nächster Befehl?} \\
            }
        }
    }
    & \\ \\

    \theutterance \stepcounter{utterance}  
    & & & \multicolumn{4}{p{0.6\linewidth}}{
        \cellcolor[rgb]{0.9,0.9,0.9}{
            \makecell[{{p{\linewidth}}}]{
                \texttt{\tiny{[GM$\langle$P2]}}
                \texttt{SPRICH: Das ist ein Problem! Ich denke, es ist eine gute Idee, das Objekt I an eine andere Position zu verschieben, um Platz für die restlichen Objekte zu schaffen. Die Position (4,1) könnte funktionieren, aber ich muss mich vergewissern, dass diese Position auf meinem Spielbrett frei ist. Bitte warte einen Moment, während ich mein Spielbrett überprüfe. Ich werde dann entscheiden, ob die Position (4,1) für das Objekt I geeignet ist.} \\
            }
        }
    }
     \\ \\

    \theutterance \stepcounter{utterance}  
    & & & \multicolumn{2}{p{0.3\linewidth}}{
        \cellcolor[rgb]{0.9,0.9,0.9}{
            \makecell[{{p{\linewidth}}}]{
                \texttt{\tiny{[GM$|$GM]}}
                \texttt{SPRICH: Das ist ein Problem! Ich denke, es ist eine gute Idee, das Objekt I an eine andere Position zu verschieben, um Platz für die restlichen Objekte zu schaffen. Die Position (4,1) könnte funktionieren, aber ich muss mich vergewissern, dass diese Position auf meinem Spielbrett frei ist. Bitte warte einen Moment, während ich mein Spielbrett überprüfe. Ich werde dann entscheiden, ob die Position (4,1) für das Objekt I geeignet ist.} \\
            }
        }
    }
    & & \\ \\

    \theutterance \stepcounter{utterance}  
    & & \multicolumn{4}{p{0.6\linewidth}}{
        \cellcolor[rgb]{0.9,0.9,0.9}{
            \makecell[{{p{\linewidth}}}]{
                \texttt{\tiny{[P1$\langle$GM]}}
                \texttt{Deine Nachricht wurde an deinen Mitspieler übermittelt.} \\
\\ 
\\ 
\texttt{Ihr seid gerade bei Runde 7 von maximal 20.} \\
\texttt{Ihr habt gemeinsam 5 von maximal 12 Strafpunkten angesammelt.} \\
\texttt{Dein Mitspieler hat dir diese Nachricht geschickt:} \\
\\ 
\\ 
\texttt{"Das ist ein Problem! Ich denke, es ist eine gute Idee, das Objekt I an eine andere Position zu verschieben, um Platz für die restlichen Objekte zu schaffen. Die Position (4,1) könnte funktionieren, aber ich muss mich vergewissern, dass diese Position auf meinem Spielbrett frei ist. Bitte warte einen Moment, während ich mein Spielbrett überprüfe. Ich werde dann entscheiden, ob die Position (4,1) für das Objekt I geeignet ist."} \\
\\ 
\\ 
\texttt{Wie lautet dein nächster Befehl?} \\
            }
        }
    }
    & \\ \\

    \theutterance \stepcounter{utterance}  
    & \multicolumn{4}{p{0.6\linewidth}}{
        \cellcolor[rgb]{0.9,0.9,0.9}{
            \makecell[{{p{\linewidth}}}]{
                \texttt{\tiny{[P1$\rangle$GM]}}
                \texttt{SPRICH: Ich verstehe, dass Sie noch prüfen müssen, ob die Position (4,1) auf Ihrem Spielbrett frei ist. Bitte nehmen Sie sich die Zeit, die Sie benötigen, um Ihr Spielbrett zu überprüfen. Ich werde warten, bis Sie mir mitteilen, ob die Position (4,1) für das Objekt I geeignet ist oder ob wir eine andere Lösung finden müssen.} \\
            }
        }
    }
    & & \\ \\

    \theutterance \stepcounter{utterance}  
    & & & \multicolumn{2}{p{0.3\linewidth}}{
        \cellcolor[rgb]{0.9,0.9,0.9}{
            \makecell[{{p{\linewidth}}}]{
                \texttt{\tiny{[GM$|$GM]}}
                \texttt{SPRICH: Ich verstehe, dass Sie noch prüfen müssen, ob die Position (4,1) auf Ihrem Spielbrett frei ist. Bitte nehmen Sie sich die Zeit, die Sie benötigen, um Ihr Spielbrett zu überprüfen. Ich werde warten, bis Sie mir mitteilen, ob die Position (4,1) für das Objekt I geeignet ist oder ob wir eine andere Lösung finden müssen.} \\
            }
        }
    }
    & & \\ \\

    \theutterance \stepcounter{utterance}  
    & & \multicolumn{4}{p{0.6\linewidth}}{
        \cellcolor[rgb]{0.9,0.9,0.9}{
            \makecell[{{p{\linewidth}}}]{
                \texttt{\tiny{[GM$\rangle$P2]}}
                \texttt{Deine Nachricht wurde an deinen Mitspieler übermittelt.} \\
\\ 
\\ 
\texttt{Ihr seid gerade bei Runde 8 von maximal 20.} \\
\texttt{Ihr habt gemeinsam 5 von maximal 12 Strafpunkten angesammelt.} \\
\texttt{Dein Mitspieler hat dir diese Nachricht geschickt:} \\
\\ 
\\ 
\texttt{"Ich verstehe, dass Sie noch prüfen müssen, ob die Position (4,1) auf Ihrem Spielbrett frei ist. Bitte nehmen Sie sich die Zeit, die Sie benötigen, um Ihr Spielbrett zu überprüfen. Ich werde warten, bis Sie mir mitteilen, ob die Position (4,1) für das Objekt I geeignet ist oder ob wir eine andere Lösung finden müssen."} \\
\\ 
\\ 
\texttt{Wie lautet dein nächster Befehl?} \\
            }
        }
    }
    & \\ \\

    \theutterance \stepcounter{utterance}  
    & & & \multicolumn{4}{p{0.6\linewidth}}{
        \cellcolor[rgb]{0.9,0.9,0.9}{
            \makecell[{{p{\linewidth}}}]{
                \texttt{\tiny{[GM$\langle$P2]}}
                \texttt{SPRICH: Ich habe mein Spielbrett überprüft und die Position (4,1) ist frei. Ich denke, es ist eine gute Idee, das Objekt I an die Position (4,1) zu verschieben. Bitte geh vor und verschiebe das Objekt I an die Position (4,1) auf deinem Spielbrett. Ich werde das gleiche tun, um sicherzustellen, dass unsere Spielbretter synchronisiert bleiben.} \\
            }
        }
    }
     \\ \\

    \theutterance \stepcounter{utterance}  
    & & & \multicolumn{2}{p{0.3\linewidth}}{
        \cellcolor[rgb]{0.9,0.9,0.9}{
            \makecell[{{p{\linewidth}}}]{
                \texttt{\tiny{[GM$|$GM]}}
                \texttt{SPRICH: Ich habe mein Spielbrett überprüft und die Position (4,1) ist frei. Ich denke, es ist eine gute Idee, das Objekt I an die Position (4,1) zu verschieben. Bitte geh vor und verschiebe das Objekt I an die Position (4,1) auf deinem Spielbrett. Ich werde das gleiche tun, um sicherzustellen, dass unsere Spielbretter synchronisiert bleiben.} \\
            }
        }
    }
    & & \\ \\

    \theutterance \stepcounter{utterance}  
    & & \multicolumn{4}{p{0.6\linewidth}}{
        \cellcolor[rgb]{0.9,0.9,0.9}{
            \makecell[{{p{\linewidth}}}]{
                \texttt{\tiny{[P1$\langle$GM]}}
                \texttt{Deine Nachricht wurde an deinen Mitspieler übermittelt.} \\
\\ 
\\ 
\texttt{Ihr seid gerade bei Runde 8 von maximal 20.} \\
\texttt{Ihr habt gemeinsam 5 von maximal 12 Strafpunkten angesammelt.} \\
\texttt{Dein Mitspieler hat dir diese Nachricht geschickt:} \\
\\ 
\\ 
\texttt{"Ich habe mein Spielbrett überprüft und die Position (4,1) ist frei. Ich denke, es ist eine gute Idee, das Objekt I an die Position (4,1) zu verschieben. Bitte geh vor und verschiebe das Objekt I an die Position (4,1) auf deinem Spielbrett. Ich werde das gleiche tun, um sicherzustellen, dass unsere Spielbretter synchronisiert bleiben."} \\
\\ 
\\ 
\texttt{Wie lautet dein nächster Befehl?} \\
            }
        }
    }
    & \\ \\

    \theutterance \stepcounter{utterance}  
    & \multicolumn{4}{p{0.6\linewidth}}{
        \cellcolor[rgb]{0.9,0.9,0.9}{
            \makecell[{{p{\linewidth}}}]{
                \texttt{\tiny{[P1$\rangle$GM]}}
                \texttt{VERSCHIEBE: I, (4, 1)} \\
            }
        }
    }
    & & \\ \\

    \theutterance \stepcounter{utterance}  
    & & & \multicolumn{2}{p{0.3\linewidth}}{
        \cellcolor[rgb]{0.9,0.9,0.9}{
            \makecell[{{p{\linewidth}}}]{
                \texttt{\tiny{[GM$|$GM]}}
                \texttt{VERSCHIEBE: I, (4, 1)} \\
            }
        }
    }
    & & \\ \\

    \theutterance \stepcounter{utterance}  
    & & & \multicolumn{2}{p{0.3\linewidth}}{
        \cellcolor[rgb]{0.9,0.9,0.9}{
            \makecell[{{p{\linewidth}}}]{
                \texttt{\tiny{[GM$|$GM]}}
                \texttt{Strafpunkt: (4,1) ist nicht leer, sondern enthält '│'.} \\
\\ 
\texttt{Ihr habt gemeinsam 6 von maximal 12 Strafpunkten angesammelt.} \\
\texttt{Bitte versuche es noch mal!} \\
            }
        }
    }
    & & \\ \\

    \theutterance \stepcounter{utterance}  
    & & \multicolumn{4}{p{0.6\linewidth}}{
        \cellcolor[rgb]{0.9,0.9,0.9}{
            \makecell[{{p{\linewidth}}}]{
                \texttt{\tiny{[P1$\langle$GM]}}
                \texttt{Strafpunkt: (4,1) ist nicht leer, sondern enthält '│'.} \\
\\ 
\texttt{Ihr habt gemeinsam 6 von maximal 12 Strafpunkten angesammelt.} \\
\texttt{Bitte versuche es noch mal!} \\
            }
        }
    }
    & \\ \\

    \theutterance \stepcounter{utterance}  
    & \multicolumn{4}{p{0.6\linewidth}}{
        \cellcolor[rgb]{0.9,0.9,0.9}{
            \makecell[{{p{\linewidth}}}]{
                \texttt{\tiny{[P1$\rangle$GM]}}
                \texttt{SPRICH: Ich muss das Objekt I an eine andere Position verschieben, da die Position (4,1) auf meinem Spielbrett nicht leer ist. Ich sehe, dass das Objekt I auf meinem Spielbrett in der Spalte 7 und Zeile 7 ist. Ich schlage vor, das Objekt I an eine Position zu verschieben, die frei ist und näher an der ersten Zeile liegt. Kann ich das Objekt I an die Position (5, 1) verschieben? Ich werde zuerst überprüfen, ob diese Position frei ist.} \\
            }
        }
    }
    & & \\ \\

    \theutterance \stepcounter{utterance}  
    & & & \multicolumn{2}{p{0.3\linewidth}}{
        \cellcolor[rgb]{0.9,0.9,0.9}{
            \makecell[{{p{\linewidth}}}]{
                \texttt{\tiny{[GM$|$GM]}}
                \texttt{SPRICH: Ich muss das Objekt I an eine andere Position verschieben, da die Position (4,1) auf meinem Spielbrett nicht leer ist. Ich sehe, dass das Objekt I auf meinem Spielbrett in der Spalte 7 und Zeile 7 ist. Ich schlage vor, das Objekt I an eine Position zu verschieben, die frei ist und näher an der ersten Zeile liegt. Kann ich das Objekt I an die Position (5, 1) verschieben? Ich werde zuerst überprüfen, ob diese Position frei ist.} \\
            }
        }
    }
    & & \\ \\

    \theutterance \stepcounter{utterance}  
    & & \multicolumn{4}{p{0.6\linewidth}}{
        \cellcolor[rgb]{0.9,0.9,0.9}{
            \makecell[{{p{\linewidth}}}]{
                \texttt{\tiny{[GM$\rangle$P2]}}
                \texttt{Deine Nachricht wurde an deinen Mitspieler übermittelt.} \\
\\ 
\\ 
\texttt{Ihr seid gerade bei Runde 9 von maximal 20.} \\
\texttt{Ihr habt gemeinsam 6 von maximal 12 Strafpunkten angesammelt.} \\
\texttt{Dein Mitspieler hat dir diese Nachricht geschickt:} \\
\\ 
\\ 
\texttt{"Ich muss das Objekt I an eine andere Position verschieben, da die Position (4,1) auf meinem Spielbrett nicht leer ist. Ich sehe, dass das Objekt I auf meinem Spielbrett in der Spalte 7 und Zeile 7 ist. Ich schlage vor, das Objekt I an eine Position zu verschieben, die frei ist und näher an der ersten Zeile liegt. Kann ich das Objekt I an die Position (5, 1) verschieben? Ich werde zuerst überprüfen, ob diese Position frei ist."} \\
\\ 
\\ 
\texttt{Wie lautet dein nächster Befehl?} \\
            }
        }
    }
    & \\ \\

    \theutterance \stepcounter{utterance}  
    & & & \multicolumn{4}{p{0.6\linewidth}}{
        \cellcolor[rgb]{0.9,0.9,0.9}{
            \makecell[{{p{\linewidth}}}]{
                \texttt{\tiny{[GM$\langle$P2]}}
                \texttt{SPRICH: Das ist ein Problem! Ich denke, es ist eine gute Idee, das Objekt I an eine andere Position zu verschieben, die näher an der ersten Zeile liegt. Bitte überprüfe, ob die Position (5, 1) auf deinem Spielbrett frei ist. Auf meinem Spielbrett ist die Position (5, 1) frei, also könnte das funktionieren. Lass mich wissen, ob du das Objekt I an die Position (5, 1) verschieben kannst.} \\
            }
        }
    }
     \\ \\

    \theutterance \stepcounter{utterance}  
    & & & \multicolumn{2}{p{0.3\linewidth}}{
        \cellcolor[rgb]{0.9,0.9,0.9}{
            \makecell[{{p{\linewidth}}}]{
                \texttt{\tiny{[GM$|$GM]}}
                \texttt{SPRICH: Das ist ein Problem! Ich denke, es ist eine gute Idee, das Objekt I an eine andere Position zu verschieben, die näher an der ersten Zeile liegt. Bitte überprüfe, ob die Position (5, 1) auf deinem Spielbrett frei ist. Auf meinem Spielbrett ist die Position (5, 1) frei, also könnte das funktionieren. Lass mich wissen, ob du das Objekt I an die Position (5, 1) verschieben kannst.} \\
            }
        }
    }
    & & \\ \\

    \theutterance \stepcounter{utterance}  
    & & \multicolumn{4}{p{0.6\linewidth}}{
        \cellcolor[rgb]{0.9,0.9,0.9}{
            \makecell[{{p{\linewidth}}}]{
                \texttt{\tiny{[P1$\langle$GM]}}
                \texttt{Deine Nachricht wurde an deinen Mitspieler übermittelt.} \\
\\ 
\\ 
\texttt{Ihr seid gerade bei Runde 9 von maximal 20.} \\
\texttt{Ihr habt gemeinsam 6 von maximal 12 Strafpunkten angesammelt.} \\
\texttt{Dein Mitspieler hat dir diese Nachricht geschickt:} \\
\\ 
\\ 
\texttt{"Das ist ein Problem! Ich denke, es ist eine gute Idee, das Objekt I an eine andere Position zu verschieben, die näher an der ersten Zeile liegt. Bitte überprüfe, ob die Position (5, 1) auf deinem Spielbrett frei ist. Auf meinem Spielbrett ist die Position (5, 1) frei, also könnte das funktionieren. Lass mich wissen, ob du das Objekt I an die Position (5, 1) verschieben kannst."} \\
\\ 
\\ 
\texttt{Wie lautet dein nächster Befehl?} \\
            }
        }
    }
    & \\ \\

    \theutterance \stepcounter{utterance}  
    & \multicolumn{4}{p{0.6\linewidth}}{
        \cellcolor[rgb]{0.9,0.9,0.9}{
            \makecell[{{p{\linewidth}}}]{
                \texttt{\tiny{[P1$\rangle$GM]}}
                \texttt{VERSCHIEBE: I, (5, 1)} \\
            }
        }
    }
    & & \\ \\

    \theutterance \stepcounter{utterance}  
    & & & \multicolumn{2}{p{0.3\linewidth}}{
        \cellcolor[rgb]{0.9,0.9,0.9}{
            \makecell[{{p{\linewidth}}}]{
                \texttt{\tiny{[GM$|$GM]}}
                \texttt{VERSCHIEBE: I, (5, 1)} \\
            }
        }
    }
    & & \\ \\

    \theutterance \stepcounter{utterance}  
    & & & \multicolumn{2}{p{0.3\linewidth}}{
        \cellcolor[rgb]{0.9,0.9,0.9}{
            \makecell[{{p{\linewidth}}}]{
                \texttt{\tiny{[GM$|$GM]}}
                \texttt{'I' erfolgreich nach (5,1) verschoben. Dein Raster sieht jetzt so aus:} \\
\texttt{```} \\
\texttt{ 1234567} \\
\texttt{╔═══╤═╤═╗} \\
\texttt{║CHW│I│◌║ 1} \\
\texttt{╟─┬─┼─┴─╢ 2} \\
\texttt{║◌│◌│◌◌◌║ 3} \\
\texttt{║◌├─┤◌┌─╢ 4} \\
\texttt{╟─┘◌│◌│◌║ 5} \\
\texttt{║T◌◌└┬┴─╢ 6} \\
\texttt{║◌◌◌◌│◌◌║ 7} \\
\texttt{╚════╧══╝} \\
\texttt{```} \\
            }
        }
    }
    & & \\ \\

    \theutterance \stepcounter{utterance}  
    & & \multicolumn{4}{p{0.6\linewidth}}{
        \cellcolor[rgb]{0.9,0.9,0.9}{
            \makecell[{{p{\linewidth}}}]{
                \texttt{\tiny{[GM$\rangle$P2]}}
                \texttt{Deine Nachricht wurde an deinen Mitspieler übermittelt.} \\
\\ 
\\ 
\texttt{Ihr seid gerade bei Runde 10 von maximal 20.} \\
\texttt{Ihr habt gemeinsam 6 von maximal 12 Strafpunkten angesammelt.} \\
\texttt{Dein Mitspieler hat ein Objekt auf seinem Raster bewegt.} \\
\\ 
\\ 
\texttt{Wie lautet dein nächster Befehl?} \\
            }
        }
    }
    & \\ \\

    \theutterance \stepcounter{utterance}  
    & & & \multicolumn{4}{p{0.6\linewidth}}{
        \cellcolor[rgb]{0.9,0.9,0.9}{
            \makecell[{{p{\linewidth}}}]{
                \texttt{\tiny{[GM$\langle$P2]}}
                \texttt{SPRICH: Ich denke, wir sollten jetzt das Objekt T auf beiden Spielbrettern in die dritte Spalte der ersten Zeile verschieben. Könntest du mir bitte mitteilen, wo sich das Objekt T auf deinem Spielbrett befindet, damit wir sicherstellen, dass wir es an die gleiche Position auf beiden Spielbrettern verschieben?} \\
            }
        }
    }
     \\ \\

    \theutterance \stepcounter{utterance}  
    & & & \multicolumn{2}{p{0.3\linewidth}}{
        \cellcolor[rgb]{0.9,0.9,0.9}{
            \makecell[{{p{\linewidth}}}]{
                \texttt{\tiny{[GM$|$GM]}}
                \texttt{SPRICH: Ich denke, wir sollten jetzt das Objekt T auf beiden Spielbrettern in die dritte Spalte der ersten Zeile verschieben. Könntest du mir bitte mitteilen, wo sich das Objekt T auf deinem Spielbrett befindet, damit wir sicherstellen, dass wir es an die gleiche Position auf beiden Spielbrettern verschieben?} \\
            }
        }
    }
    & & \\ \\

    \theutterance \stepcounter{utterance}  
    & & \multicolumn{4}{p{0.6\linewidth}}{
        \cellcolor[rgb]{0.9,0.9,0.9}{
            \makecell[{{p{\linewidth}}}]{
                \texttt{\tiny{[P1$\langle$GM]}}
                \texttt{'I' erfolgreich nach (5,1) verschoben. Dein Raster sieht jetzt so aus:} \\
\texttt{```} \\
\texttt{ 1234567} \\
\texttt{╔═══╤═╤═╗} \\
\texttt{║CHW│I│◌║ 1} \\
\texttt{╟─┬─┼─┴─╢ 2} \\
\texttt{║◌│◌│◌◌◌║ 3} \\
\texttt{║◌├─┤◌┌─╢ 4} \\
\texttt{╟─┘◌│◌│◌║ 5} \\
\texttt{║T◌◌└┬┴─╢ 6} \\
\texttt{║◌◌◌◌│◌◌║ 7} \\
\texttt{╚════╧══╝} \\
\texttt{```} \\
\texttt{Ihr seid gerade bei Runde 10 von maximal 20.} \\
\texttt{Ihr habt gemeinsam 6 von maximal 12 Strafpunkten angesammelt.} \\
\texttt{Dein Mitspieler hat dir diese Nachricht geschickt:} \\
\\ 
\\ 
\texttt{"Ich denke, wir sollten jetzt das Objekt T auf beiden Spielbrettern in die dritte Spalte der ersten Zeile verschieben. Könntest du mir bitte mitteilen, wo sich das Objekt T auf deinem Spielbrett befindet, damit wir sicherstellen, dass wir es an die gleiche Position auf beiden Spielbrettern verschieben?"} \\
\\ 
\\ 
\texttt{Wie lautet dein nächster Befehl?} \\
            }
        }
    }
    & \\ \\

    \theutterance \stepcounter{utterance}  
    & \multicolumn{4}{p{0.6\linewidth}}{
        \cellcolor[rgb]{0.9,0.9,0.9}{
            \makecell[{{p{\linewidth}}}]{
                \texttt{\tiny{[P1$\rangle$GM]}}
                \texttt{SPRICH: Auf meinem Spielbrett befindet sich das Objekt T in der Spalte 1 und Zeile 6. Ich muss es also ziemlich weit verschieben, um es in die dritte Spalte der ersten Zeile zu bringen. Ich werde mich bemühen, es an die richtige Position zu verschieben. Wie sieht es auf Ihrem Spielbrett aus?} \\
            }
        }
    }
    & & \\ \\

    \theutterance \stepcounter{utterance}  
    & & & \multicolumn{2}{p{0.3\linewidth}}{
        \cellcolor[rgb]{0.9,0.9,0.9}{
            \makecell[{{p{\linewidth}}}]{
                \texttt{\tiny{[GM$|$GM]}}
                \texttt{SPRICH: Auf meinem Spielbrett befindet sich das Objekt T in der Spalte 1 und Zeile 6. Ich muss es also ziemlich weit verschieben, um es in die dritte Spalte der ersten Zeile zu bringen. Ich werde mich bemühen, es an die richtige Position zu verschieben. Wie sieht es auf Ihrem Spielbrett aus?} \\
            }
        }
    }
    & & \\ \\

    \theutterance \stepcounter{utterance}  
    & & \multicolumn{4}{p{0.6\linewidth}}{
        \cellcolor[rgb]{0.9,0.9,0.9}{
            \makecell[{{p{\linewidth}}}]{
                \texttt{\tiny{[GM$\rangle$P2]}}
                \texttt{Deine Nachricht wurde an deinen Mitspieler übermittelt.} \\
\\ 
\\ 
\texttt{Ihr seid gerade bei Runde 11 von maximal 20.} \\
\texttt{Ihr habt gemeinsam 6 von maximal 12 Strafpunkten angesammelt.} \\
\texttt{Dein Mitspieler hat dir diese Nachricht geschickt:} \\
\\ 
\\ 
\texttt{"Auf meinem Spielbrett befindet sich das Objekt T in der Spalte 1 und Zeile 6. Ich muss es also ziemlich weit verschieben, um es in die dritte Spalte der ersten Zeile zu bringen. Ich werde mich bemühen, es an die richtige Position zu verschieben. Wie sieht es auf Ihrem Spielbrett aus?"} \\
\\ 
\\ 
\texttt{Wie lautet dein nächster Befehl?} \\
            }
        }
    }
    & \\ \\

    \theutterance \stepcounter{utterance}  
    & & & \multicolumn{4}{p{0.6\linewidth}}{
        \cellcolor[rgb]{0.9,0.9,0.9}{
            \makecell[{{p{\linewidth}}}]{
                \texttt{\tiny{[GM$\langle$P2]}}
                \texttt{SPRICH: Auf meinem Spielbrett befindet sich das Objekt T in der Spalte 5 und Zeile 6. Es ist also nicht an der gewünschten Position. Ich denke, wir sollten es an die Position (3, 1) verschieben, um es in die dritte Spalte der ersten Zeile zu bringen. Bitte versichere dich, dass du das Objekt T auf deinem Spielbrett an die Position (3, 1) verschiebst, wenn du bereit bist. Ich werde das gleiche tun, um sicherzustellen, dass unsere Spielbretter synchronisiert bleiben.} \\
            }
        }
    }
     \\ \\

    \theutterance \stepcounter{utterance}  
    & & & \multicolumn{2}{p{0.3\linewidth}}{
        \cellcolor[rgb]{0.9,0.9,0.9}{
            \makecell[{{p{\linewidth}}}]{
                \texttt{\tiny{[GM$|$GM]}}
                \texttt{SPRICH: Auf meinem Spielbrett befindet sich das Objekt T in der Spalte 5 und Zeile 6. Es ist also nicht an der gewünschten Position. Ich denke, wir sollten es an die Position (3, 1) verschieben, um es in die dritte Spalte der ersten Zeile zu bringen. Bitte versichere dich, dass du das Objekt T auf deinem Spielbrett an die Position (3, 1) verschiebst, wenn du bereit bist. Ich werde das gleiche tun, um sicherzustellen, dass unsere Spielbretter synchronisiert bleiben.} \\
            }
        }
    }
    & & \\ \\

    \theutterance \stepcounter{utterance}  
    & & \multicolumn{4}{p{0.6\linewidth}}{
        \cellcolor[rgb]{0.9,0.9,0.9}{
            \makecell[{{p{\linewidth}}}]{
                \texttt{\tiny{[P1$\langle$GM]}}
                \texttt{Deine Nachricht wurde an deinen Mitspieler übermittelt.} \\
\\ 
\\ 
\texttt{Ihr seid gerade bei Runde 11 von maximal 20.} \\
\texttt{Ihr habt gemeinsam 6 von maximal 12 Strafpunkten angesammelt.} \\
\texttt{Dein Mitspieler hat dir diese Nachricht geschickt:} \\
\\ 
\\ 
\texttt{"Auf meinem Spielbrett befindet sich das Objekt T in der Spalte 5 und Zeile 6. Es ist also nicht an der gewünschten Position. Ich denke, wir sollten es an die Position (3, 1) verschieben, um es in die dritte Spalte der ersten Zeile zu bringen. Bitte versichere dich, dass du das Objekt T auf deinem Spielbrett an die Position (3, 1) verschiebst, wenn du bereit bist. Ich werde das gleiche tun, um sicherzustellen, dass unsere Spielbretter synchronisiert bleiben."} \\
\\ 
\\ 
\texttt{Wie lautet dein nächster Befehl?} \\
            }
        }
    }
    & \\ \\

    \theutterance \stepcounter{utterance}  
    & \multicolumn{4}{p{0.6\linewidth}}{
        \cellcolor[rgb]{0.9,0.9,0.9}{
            \makecell[{{p{\linewidth}}}]{
                \texttt{\tiny{[P1$\rangle$GM]}}
                \texttt{VERSCHIEBE: T, (3, 1)} \\
            }
        }
    }
    & & \\ \\

    \theutterance \stepcounter{utterance}  
    & & & \multicolumn{2}{p{0.3\linewidth}}{
        \cellcolor[rgb]{0.9,0.9,0.9}{
            \makecell[{{p{\linewidth}}}]{
                \texttt{\tiny{[GM$|$GM]}}
                \texttt{VERSCHIEBE: T, (3, 1)} \\
            }
        }
    }
    & & \\ \\

    \theutterance \stepcounter{utterance}  
    & & & \multicolumn{2}{p{0.3\linewidth}}{
        \cellcolor[rgb]{0.9,0.9,0.9}{
            \makecell[{{p{\linewidth}}}]{
                \texttt{\tiny{[GM$|$GM]}}
                \texttt{Strafpunkt: (3,1) ist nicht leer, sondern enthält 'W'.} \\
\\ 
\texttt{Ihr habt gemeinsam 7 von maximal 12 Strafpunkten angesammelt.} \\
\texttt{Bitte versuche es noch mal!} \\
            }
        }
    }
    & & \\ \\

    \theutterance \stepcounter{utterance}  
    & & \multicolumn{4}{p{0.6\linewidth}}{
        \cellcolor[rgb]{0.9,0.9,0.9}{
            \makecell[{{p{\linewidth}}}]{
                \texttt{\tiny{[P1$\langle$GM]}}
                \texttt{Strafpunkt: (3,1) ist nicht leer, sondern enthält 'W'.} \\
\\ 
\texttt{Ihr habt gemeinsam 7 von maximal 12 Strafpunkten angesammelt.} \\
\texttt{Bitte versuche es noch mal!} \\
            }
        }
    }
    & \\ \\

    \theutterance \stepcounter{utterance}  
    & \multicolumn{4}{p{0.6\linewidth}}{
        \cellcolor[rgb]{0.9,0.9,0.9}{
            \makecell[{{p{\linewidth}}}]{
                \texttt{\tiny{[P1$\rangle$GM]}}
                \texttt{SPRICH: Ich muss das Objekt T an eine andere Position verschieben, da die Position (3,1) auf meinem Spielbrett bereits von 'W' besetzt ist. Ich schlage vor, das Objekt T an die Position (6, 1) zu verschieben, um Platz für die restlichen Objekte zu schaffen. Wie sehen Sie das? Sollte ich das Objekt T an die Position (6, 1) verschieben?} \\
            }
        }
    }
    & & \\ \\

    \theutterance \stepcounter{utterance}  
    & & & \multicolumn{2}{p{0.3\linewidth}}{
        \cellcolor[rgb]{0.9,0.9,0.9}{
            \makecell[{{p{\linewidth}}}]{
                \texttt{\tiny{[GM$|$GM]}}
                \texttt{SPRICH: Ich muss das Objekt T an eine andere Position verschieben, da die Position (3,1) auf meinem Spielbrett bereits von 'W' besetzt ist. Ich schlage vor, das Objekt T an die Position (6, 1) zu verschieben, um Platz für die restlichen Objekte zu schaffen. Wie sehen Sie das? Sollte ich das Objekt T an die Position (6, 1) verschieben?} \\
            }
        }
    }
    & & \\ \\

    \theutterance \stepcounter{utterance}  
    & & \multicolumn{4}{p{0.6\linewidth}}{
        \cellcolor[rgb]{0.9,0.9,0.9}{
            \makecell[{{p{\linewidth}}}]{
                \texttt{\tiny{[GM$\rangle$P2]}}
                \texttt{Deine Nachricht wurde an deinen Mitspieler übermittelt.} \\
\\ 
\\ 
\texttt{Ihr seid gerade bei Runde 12 von maximal 20.} \\
\texttt{Ihr habt gemeinsam 7 von maximal 12 Strafpunkten angesammelt.} \\
\texttt{Dein Mitspieler hat dir diese Nachricht geschickt:} \\
\\ 
\\ 
\texttt{"Ich muss das Objekt T an eine andere Position verschieben, da die Position (3,1) auf meinem Spielbrett bereits von 'W' besetzt ist. Ich schlage vor, das Objekt T an die Position (6, 1) zu verschieben, um Platz für die restlichen Objekte zu schaffen. Wie sehen Sie das? Sollte ich das Objekt T an die Position (6, 1) verschieben?"} \\
\\ 
\\ 
\texttt{Wie lautet dein nächster Befehl?} \\
            }
        }
    }
    & \\ \\

    \theutterance \stepcounter{utterance}  
    & & & \multicolumn{4}{p{0.6\linewidth}}{
        \cellcolor[rgb]{0.9,0.9,0.9}{
            \makecell[{{p{\linewidth}}}]{
                \texttt{\tiny{[GM$\langle$P2]}}
                \texttt{SPRICH: Das ist ein Problem! Ich denke, es ist eine gute Idee, das Objekt T an eine andere Position zu verschieben, um Platz für die restlichen Objekte zu schaffen. Die Position (6, 1) könnte fonctionieren, aber ich muss mich vergewissern, dass diese Position auf meinem Spielbrett frei ist. Bitte warte einen Moment, während ich mein Spielbrett überprüfe. Ich werde dann entscheiden, ob die Position (6, 1) für das Objekt T geeignet ist.} \\
            }
        }
    }
     \\ \\

    \theutterance \stepcounter{utterance}  
    & & & \multicolumn{2}{p{0.3\linewidth}}{
        \cellcolor[rgb]{0.9,0.9,0.9}{
            \makecell[{{p{\linewidth}}}]{
                \texttt{\tiny{[GM$|$GM]}}
                \texttt{SPRICH: Das ist ein Problem! Ich denke, es ist eine gute Idee, das Objekt T an eine andere Position zu verschieben, um Platz für die restlichen Objekte zu schaffen. Die Position (6, 1) könnte fonctionieren, aber ich muss mich vergewissern, dass diese Position auf meinem Spielbrett frei ist. Bitte warte einen Moment, während ich mein Spielbrett überprüfe. Ich werde dann entscheiden, ob die Position (6, 1) für das Objekt T geeignet ist.} \\
            }
        }
    }
    & & \\ \\

    \theutterance \stepcounter{utterance}  
    & & \multicolumn{4}{p{0.6\linewidth}}{
        \cellcolor[rgb]{0.9,0.9,0.9}{
            \makecell[{{p{\linewidth}}}]{
                \texttt{\tiny{[P1$\langle$GM]}}
                \texttt{Deine Nachricht wurde an deinen Mitspieler übermittelt.} \\
\\ 
\\ 
\texttt{Ihr seid gerade bei Runde 12 von maximal 20.} \\
\texttt{Ihr habt gemeinsam 7 von maximal 12 Strafpunkten angesammelt.} \\
\texttt{Dein Mitspieler hat dir diese Nachricht geschickt:} \\
\\ 
\\ 
\texttt{"Das ist ein Problem! Ich denke, es ist eine gute Idee, das Objekt T an eine andere Position zu verschieben, um Platz für die restlichen Objekte zu schaffen. Die Position (6, 1) könnte fonctionieren, aber ich muss mich vergewissern, dass diese Position auf meinem Spielbrett frei ist. Bitte warte einen Moment, während ich mein Spielbrett überprüfe. Ich werde dann entscheiden, ob die Position (6, 1) für das Objekt T geeignet ist."} \\
\\ 
\\ 
\texttt{Wie lautet dein nächster Befehl?} \\
            }
        }
    }
    & \\ \\

    \theutterance \stepcounter{utterance}  
    & \multicolumn{4}{p{0.6\linewidth}}{
        \cellcolor[rgb]{0.9,0.9,0.9}{
            \makecell[{{p{\linewidth}}}]{
                \texttt{\tiny{[P1$\rangle$GM]}}
                \texttt{SPRICH: Ich verstehe, dass Sie noch prüfen müssen, ob die Position (6, 1) auf Ihrem Spielbrett frei ist. Bitte nehmen Sie sich die Zeit, die Sie benötigen, um Ihr Spielbrett zu überprüfen. Ich werde warten, bis Sie mir mitteilen, ob die Position (6, 1) für das Objekt T geeignet ist oder ob wir eine andere Lösung finden müssen. Ich schlage vor, dass Sie auch prüfen, ob die Position (3, 2) frei ist, da dies eine Alternative für das Objekt T sein könnte.} \\
            }
        }
    }
    & & \\ \\

    \theutterance \stepcounter{utterance}  
    & & & \multicolumn{2}{p{0.3\linewidth}}{
        \cellcolor[rgb]{0.9,0.9,0.9}{
            \makecell[{{p{\linewidth}}}]{
                \texttt{\tiny{[GM$|$GM]}}
                \texttt{SPRICH: Ich verstehe, dass Sie noch prüfen müssen, ob die Position (6, 1) auf Ihrem Spielbrett frei ist. Bitte nehmen Sie sich die Zeit, die Sie benötigen, um Ihr Spielbrett zu überprüfen. Ich werde warten, bis Sie mir mitteilen, ob die Position (6, 1) für das Objekt T geeignet ist oder ob wir eine andere Lösung finden müssen. Ich schlage vor, dass Sie auch prüfen, ob die Position (3, 2) frei ist, da dies eine Alternative für das Objekt T sein könnte.} \\
            }
        }
    }
    & & \\ \\

    \theutterance \stepcounter{utterance}  
    & & \multicolumn{4}{p{0.6\linewidth}}{
        \cellcolor[rgb]{0.9,0.9,0.9}{
            \makecell[{{p{\linewidth}}}]{
                \texttt{\tiny{[GM$\rangle$P2]}}
                \texttt{Deine Nachricht wurde an deinen Mitspieler übermittelt.} \\
\\ 
\\ 
\texttt{Ihr seid gerade bei Runde 13 von maximal 20.} \\
\texttt{Ihr habt gemeinsam 7 von maximal 12 Strafpunkten angesammelt.} \\
\texttt{Dein Mitspieler hat dir diese Nachricht geschickt:} \\
\\ 
\\ 
\texttt{"Ich verstehe, dass Sie noch prüfen müssen, ob die Position (6, 1) auf Ihrem Spielbrett frei ist. Bitte nehmen Sie sich die Zeit, die Sie benötigen, um Ihr Spielbrett zu überprüfen. Ich werde warten, bis Sie mir mitteilen, ob die Position (6, 1) für das Objekt T geeignet ist oder ob wir eine andere Lösung finden müssen. Ich schlage vor, dass Sie auch prüfen, ob die Position (3, 2) frei ist, da dies eine Alternative für das Objekt T sein könnte."} \\
\\ 
\\ 
\texttt{Wie lautet dein nächster Befehl?} \\
            }
        }
    }
    & \\ \\

    \theutterance \stepcounter{utterance}  
    & & & \multicolumn{4}{p{0.6\linewidth}}{
        \cellcolor[rgb]{0.9,0.9,0.9}{
            \makecell[{{p{\linewidth}}}]{
                \texttt{\tiny{[GM$\langle$P2]}}
                \texttt{SPRICH: Ich habe mein Spielbrett überprüft und die Position (6, 1) ist frei. Die Position (3, 2) ist auch frei. Ich denke, es ist eine gute Idee, das Objekt T an die Position (3, 2) zu verschieben, anstatt an die Position (6, 1), da dies näher an der ursprünglich geplanten Position für das Objekt T ist. Bitte versichere dich, dass du das Objekt T auf deinem Spielbrett an die Position (3, 2) verschiebst, wenn du bereit bist. Ich werde das gleiche tun, um sicherzustellen, dass unsere Spielbretter synchronisiert bleiben.} \\
            }
        }
    }
     \\ \\

    \theutterance \stepcounter{utterance}  
    & & & \multicolumn{2}{p{0.3\linewidth}}{
        \cellcolor[rgb]{0.9,0.9,0.9}{
            \makecell[{{p{\linewidth}}}]{
                \texttt{\tiny{[GM$|$GM]}}
                \texttt{SPRICH: Ich habe mein Spielbrett überprüft und die Position (6, 1) ist frei. Die Position (3, 2) ist auch frei. Ich denke, es ist eine gute Idee, das Objekt T an die Position (3, 2) zu verschieben, anstatt an die Position (6, 1), da dies näher an der ursprünglich geplanten Position für das Objekt T ist. Bitte versichere dich, dass du das Objekt T auf deinem Spielbrett an die Position (3, 2) verschiebst, wenn du bereit bist. Ich werde das gleiche tun, um sicherzustellen, dass unsere Spielbretter synchronisiert bleiben.} \\
            }
        }
    }
    & & \\ \\

    \theutterance \stepcounter{utterance}  
    & & \multicolumn{4}{p{0.6\linewidth}}{
        \cellcolor[rgb]{0.9,0.9,0.9}{
            \makecell[{{p{\linewidth}}}]{
                \texttt{\tiny{[P1$\langle$GM]}}
                \texttt{Deine Nachricht wurde an deinen Mitspieler übermittelt.} \\
\\ 
\\ 
\texttt{Ihr seid gerade bei Runde 13 von maximal 20.} \\
\texttt{Ihr habt gemeinsam 7 von maximal 12 Strafpunkten angesammelt.} \\
\texttt{Dein Mitspieler hat dir diese Nachricht geschickt:} \\
\\ 
\\ 
\texttt{"Ich habe mein Spielbrett überprüft und die Position (6, 1) ist frei. Die Position (3, 2) ist auch frei. Ich denke, es ist eine gute Idee, das Objekt T an die Position (3, 2) zu verschieben, anstatt an die Position (6, 1), da dies näher an der ursprünglich geplanten Position für das Objekt T ist. Bitte versichere dich, dass du das Objekt T auf deinem Spielbrett an die Position (3, 2) verschiebst, wenn du bereit bist. Ich werde das gleiche tun, um sicherzustellen, dass unsere Spielbretter synchronisiert bleiben."} \\
\\ 
\\ 
\texttt{Wie lautet dein nächster Befehl?} \\
            }
        }
    }
    & \\ \\

    \theutterance \stepcounter{utterance}  
    & \multicolumn{4}{p{0.6\linewidth}}{
        \cellcolor[rgb]{0.9,0.9,0.9}{
            \makecell[{{p{\linewidth}}}]{
                \texttt{\tiny{[P1$\rangle$GM]}}
                \texttt{VERSCHIEBE: T, (3, 2)} \\
            }
        }
    }
    & & \\ \\

    \theutterance \stepcounter{utterance}  
    & & & \multicolumn{2}{p{0.3\linewidth}}{
        \cellcolor[rgb]{0.9,0.9,0.9}{
            \makecell[{{p{\linewidth}}}]{
                \texttt{\tiny{[GM$|$GM]}}
                \texttt{VERSCHIEBE: T, (3, 2)} \\
            }
        }
    }
    & & \\ \\

    \theutterance \stepcounter{utterance}  
    & & & \multicolumn{2}{p{0.3\linewidth}}{
        \cellcolor[rgb]{0.9,0.9,0.9}{
            \makecell[{{p{\linewidth}}}]{
                \texttt{\tiny{[GM$|$GM]}}
                \texttt{Strafpunkt: (3,2) ist nicht leer, sondern enthält '─'.} \\
\\ 
\texttt{Ihr habt gemeinsam 8 von maximal 12 Strafpunkten angesammelt.} \\
\texttt{Bitte versuche es noch mal!} \\
            }
        }
    }
    & & \\ \\

    \theutterance \stepcounter{utterance}  
    & & \multicolumn{4}{p{0.6\linewidth}}{
        \cellcolor[rgb]{0.9,0.9,0.9}{
            \makecell[{{p{\linewidth}}}]{
                \texttt{\tiny{[P1$\langle$GM]}}
                \texttt{Strafpunkt: (3,2) ist nicht leer, sondern enthält '─'.} \\
\\ 
\texttt{Ihr habt gemeinsam 8 von maximal 12 Strafpunkten angesammelt.} \\
\texttt{Bitte versuche es noch mal!} \\
            }
        }
    }
    & \\ \\

    \theutterance \stepcounter{utterance}  
    & \multicolumn{4}{p{0.6\linewidth}}{
        \cellcolor[rgb]{0.9,0.9,0.9}{
            \makecell[{{p{\linewidth}}}]{
                \texttt{\tiny{[P1$\rangle$GM]}}
                \texttt{SPRICH: Ich muss das Objekt T an eine andere Position verschieben, da die Position (3,2) auf meinem Spielbrett bereits von '─' besetzt ist. Ich sehe, dass die Position (6,1) frei ist. Ich schlage vor, das Objekt T an die Position (6,1) zu verschieben. Wie sehen Sie das? Sollte ich das Objekt T an die Position (6,1) verschieben?} \\
            }
        }
    }
    & & \\ \\

    \theutterance \stepcounter{utterance}  
    & & & \multicolumn{2}{p{0.3\linewidth}}{
        \cellcolor[rgb]{0.9,0.9,0.9}{
            \makecell[{{p{\linewidth}}}]{
                \texttt{\tiny{[GM$|$GM]}}
                \texttt{SPRICH: Ich muss das Objekt T an eine andere Position verschieben, da die Position (3,2) auf meinem Spielbrett bereits von '─' besetzt ist. Ich sehe, dass die Position (6,1) frei ist. Ich schlage vor, das Objekt T an die Position (6,1) zu verschieben. Wie sehen Sie das? Sollte ich das Objekt T an die Position (6,1) verschieben?} \\
            }
        }
    }
    & & \\ \\

    \theutterance \stepcounter{utterance}  
    & & \multicolumn{4}{p{0.6\linewidth}}{
        \cellcolor[rgb]{0.9,0.9,0.9}{
            \makecell[{{p{\linewidth}}}]{
                \texttt{\tiny{[GM$\rangle$P2]}}
                \texttt{Deine Nachricht wurde an deinen Mitspieler übermittelt.} \\
\\ 
\\ 
\texttt{Ihr seid gerade bei Runde 14 von maximal 20.} \\
\texttt{Ihr habt gemeinsam 8 von maximal 12 Strafpunkten angesammelt.} \\
\texttt{Dein Mitspieler hat dir diese Nachricht geschickt:} \\
\\ 
\\ 
\texttt{"Ich muss das Objekt T an eine andere Position verschieben, da die Position (3,2) auf meinem Spielbrett bereits von '─' besetzt ist. Ich sehe, dass die Position (6,1) frei ist. Ich schlage vor, das Objekt T an die Position (6,1) zu verschieben. Wie sehen Sie das? Sollte ich das Objekt T an die Position (6,1) verschieben?"} \\
\\ 
\\ 
\texttt{Wie lautet dein nächster Befehl?} \\
            }
        }
    }
    & \\ \\

    \theutterance \stepcounter{utterance}  
    & & & \multicolumn{4}{p{0.6\linewidth}}{
        \cellcolor[rgb]{0.9,0.9,0.9}{
            \makecell[{{p{\linewidth}}}]{
                \texttt{\tiny{[GM$\langle$P2]}}
                \texttt{SPRICH: Ja, ich denke, das ist eine gute Idee! Da die Position (6,1) auf meinem Spielbrett auch frei ist, können wir das Objekt T sicher an diese Position verschieben. Bitte geh vor und verschiebe das Objekt T an die Position (6,1) auf deinem Spielbrett. Ich werde das gleiche tun, um sicherzustellen, dass unsere Spielbretter synchronisiert bleiben.} \\
            }
        }
    }
     \\ \\

    \theutterance \stepcounter{utterance}  
    & & & \multicolumn{2}{p{0.3\linewidth}}{
        \cellcolor[rgb]{0.9,0.9,0.9}{
            \makecell[{{p{\linewidth}}}]{
                \texttt{\tiny{[GM$|$GM]}}
                \texttt{SPRICH: Ja, ich denke, das ist eine gute Idee! Da die Position (6,1) auf meinem Spielbrett auch frei ist, können wir das Objekt T sicher an diese Position verschieben. Bitte geh vor und verschiebe das Objekt T an die Position (6,1) auf deinem Spielbrett. Ich werde das gleiche tun, um sicherzustellen, dass unsere Spielbretter synchronisiert bleiben.} \\
            }
        }
    }
    & & \\ \\

    \theutterance \stepcounter{utterance}  
    & & \multicolumn{4}{p{0.6\linewidth}}{
        \cellcolor[rgb]{0.9,0.9,0.9}{
            \makecell[{{p{\linewidth}}}]{
                \texttt{\tiny{[P1$\langle$GM]}}
                \texttt{Deine Nachricht wurde an deinen Mitspieler übermittelt.} \\
\\ 
\\ 
\texttt{Ihr seid gerade bei Runde 14 von maximal 20.} \\
\texttt{Ihr habt gemeinsam 8 von maximal 12 Strafpunkten angesammelt.} \\
\texttt{Dein Mitspieler hat dir diese Nachricht geschickt:} \\
\\ 
\\ 
\texttt{"Ja, ich denke, das ist eine gute Idee! Da die Position (6,1) auf meinem Spielbrett auch frei ist, können wir das Objekt T sicher an diese Position verschieben. Bitte geh vor und verschiebe das Objekt T an die Position (6,1) auf deinem Spielbrett. Ich werde das gleiche tun, um sicherzustellen, dass unsere Spielbretter synchronisiert bleiben."} \\
\\ 
\\ 
\texttt{Wie lautet dein nächster Befehl?} \\
            }
        }
    }
    & \\ \\

    \theutterance \stepcounter{utterance}  
    & \multicolumn{4}{p{0.6\linewidth}}{
        \cellcolor[rgb]{0.9,0.9,0.9}{
            \makecell[{{p{\linewidth}}}]{
                \texttt{\tiny{[P1$\rangle$GM]}}
                \texttt{VERSCHIEBE: T, (6, 1)} \\
            }
        }
    }
    & & \\ \\

    \theutterance \stepcounter{utterance}  
    & & & \multicolumn{2}{p{0.3\linewidth}}{
        \cellcolor[rgb]{0.9,0.9,0.9}{
            \makecell[{{p{\linewidth}}}]{
                \texttt{\tiny{[GM$|$GM]}}
                \texttt{VERSCHIEBE: T, (6, 1)} \\
            }
        }
    }
    & & \\ \\

    \theutterance \stepcounter{utterance}  
    & & & \multicolumn{2}{p{0.3\linewidth}}{
        \cellcolor[rgb]{0.9,0.9,0.9}{
            \makecell[{{p{\linewidth}}}]{
                \texttt{\tiny{[GM$|$GM]}}
                \texttt{Strafpunkt: (6,1) ist nicht leer, sondern enthält '│'.} \\
\\ 
\texttt{Ihr habt gemeinsam 9 von maximal 12 Strafpunkten angesammelt.} \\
\texttt{Bitte versuche es noch mal!} \\
            }
        }
    }
    & & \\ \\

    \theutterance \stepcounter{utterance}  
    & & \multicolumn{4}{p{0.6\linewidth}}{
        \cellcolor[rgb]{0.9,0.9,0.9}{
            \makecell[{{p{\linewidth}}}]{
                \texttt{\tiny{[P1$\langle$GM]}}
                \texttt{Strafpunkt: (6,1) ist nicht leer, sondern enthält '│'.} \\
\\ 
\texttt{Ihr habt gemeinsam 9 von maximal 12 Strafpunkten angesammelt.} \\
\texttt{Bitte versuche es noch mal!} \\
            }
        }
    }
    & \\ \\

    \theutterance \stepcounter{utterance}  
    & \multicolumn{4}{p{0.6\linewidth}}{
        \cellcolor[rgb]{0.9,0.9,0.9}{
            \makecell[{{p{\linewidth}}}]{
                \texttt{\tiny{[P1$\rangle$GM]}}
                \texttt{SPRICH: Ich muss das Objekt T an eine andere Position verschieben, da die Position (6,1) auf meinem Spielbrett bereits von '│' besetzt ist. Ich sehe, dass die Position (5,2) möglicherweise frei ist. Ich schlage vor, das Objekt T an die Position (5,2) zu verschieben, wenn diese Position frei ist. Wie sehen Sie das? Sollte ich das Objekt T an die Position (5,2) verschieben, wenn diese frei ist?} \\
            }
        }
    }
    & & \\ \\

    \theutterance \stepcounter{utterance}  
    & & & \multicolumn{2}{p{0.3\linewidth}}{
        \cellcolor[rgb]{0.9,0.9,0.9}{
            \makecell[{{p{\linewidth}}}]{
                \texttt{\tiny{[GM$|$GM]}}
                \texttt{SPRICH: Ich muss das Objekt T an eine andere Position verschieben, da die Position (6,1) auf meinem Spielbrett bereits von '│' besetzt ist. Ich sehe, dass die Position (5,2) möglicherweise frei ist. Ich schlage vor, das Objekt T an die Position (5,2) zu verschieben, wenn diese Position frei ist. Wie sehen Sie das? Sollte ich das Objekt T an die Position (5,2) verschieben, wenn diese frei ist?} \\
            }
        }
    }
    & & \\ \\

    \theutterance \stepcounter{utterance}  
    & & \multicolumn{4}{p{0.6\linewidth}}{
        \cellcolor[rgb]{0.9,0.9,0.9}{
            \makecell[{{p{\linewidth}}}]{
                \texttt{\tiny{[GM$\rangle$P2]}}
                \texttt{Deine Nachricht wurde an deinen Mitspieler übermittelt.} \\
\\ 
\\ 
\texttt{Ihr seid gerade bei Runde 15 von maximal 20.} \\
\texttt{Ihr habt gemeinsam 9 von maximal 12 Strafpunkten angesammelt.} \\
\texttt{Dein Mitspieler hat dir diese Nachricht geschickt:} \\
\\ 
\\ 
\texttt{"Ich muss das Objekt T an eine andere Position verschieben, da die Position (6,1) auf meinem Spielbrett bereits von '│' besetzt ist. Ich sehe, dass die Position (5,2) möglicherweise frei ist. Ich schlage vor, das Objekt T an die Position (5,2) zu verschieben, wenn diese Position frei ist. Wie sehen Sie das? Sollte ich das Objekt T an die Position (5,2) verschieben, wenn diese frei ist?"} \\
\\ 
\\ 
\texttt{Wie lautet dein nächster Befehl?} \\
            }
        }
    }
    & \\ \\

    \theutterance \stepcounter{utterance}  
    & & & \multicolumn{4}{p{0.6\linewidth}}{
        \cellcolor[rgb]{0.9,0.9,0.9}{
            \makecell[{{p{\linewidth}}}]{
                \texttt{\tiny{[GM$\langle$P2]}}
                \texttt{SPRICH: Ich denke, das ist eine gute Idee! Bitte überprüfe, ob die Position (5,2) auf deinem Spielbrett frei ist. Auf meinem Spielbrett ist die Position (5,2) frei. Wenn du bestätigen kannst, dass die Position (5,2) auf deinem Spielbrett auch frei ist, dann können wir das Objekt T an diese Position verschieben. Lass mich wissen, ob das der Fall ist, und wir können weitermachen.} \\
            }
        }
    }
     \\ \\

    \theutterance \stepcounter{utterance}  
    & & & \multicolumn{2}{p{0.3\linewidth}}{
        \cellcolor[rgb]{0.9,0.9,0.9}{
            \makecell[{{p{\linewidth}}}]{
                \texttt{\tiny{[GM$|$GM]}}
                \texttt{SPRICH: Ich denke, das ist eine gute Idee! Bitte überprüfe, ob die Position (5,2) auf deinem Spielbrett frei ist. Auf meinem Spielbrett ist die Position (5,2) frei. Wenn du bestätigen kannst, dass die Position (5,2) auf deinem Spielbrett auch frei ist, dann können wir das Objekt T an diese Position verschieben. Lass mich wissen, ob das der Fall ist, und wir können weitermachen.} \\
            }
        }
    }
    & & \\ \\

    \theutterance \stepcounter{utterance}  
    & & \multicolumn{4}{p{0.6\linewidth}}{
        \cellcolor[rgb]{0.9,0.9,0.9}{
            \makecell[{{p{\linewidth}}}]{
                \texttt{\tiny{[P1$\langle$GM]}}
                \texttt{Deine Nachricht wurde an deinen Mitspieler übermittelt.} \\
\\ 
\\ 
\texttt{Ihr seid gerade bei Runde 15 von maximal 20.} \\
\texttt{Ihr habt gemeinsam 9 von maximal 12 Strafpunkten angesammelt.} \\
\texttt{Dein Mitspieler hat dir diese Nachricht geschickt:} \\
\\ 
\\ 
\texttt{"Ich denke, das ist eine gute Idee! Bitte überprüfe, ob die Position (5,2) auf deinem Spielbrett frei ist. Auf meinem Spielbrett ist die Position (5,2) frei. Wenn du bestätigen kannst, dass die Position (5,2) auf deinem Spielbrett auch frei ist, dann können wir das Objekt T an diese Position verschieben. Lass mich wissen, ob das der Fall ist, und wir können weitermachen."} \\
\\ 
\\ 
\texttt{Wie lautet dein nächster Befehl?} \\
            }
        }
    }
    & \\ \\

    \theutterance \stepcounter{utterance}  
    & \multicolumn{4}{p{0.6\linewidth}}{
        \cellcolor[rgb]{0.9,0.9,0.9}{
            \makecell[{{p{\linewidth}}}]{
                \texttt{\tiny{[P1$\rangle$GM]}}
                \texttt{SPRICH: Ich habe mein Spielbrett überprüft und die Position (5,2) ist tatsächlich frei. Ich kann das Objekt T also an diese Position verschieben. Ich werde den Befehl geben, das Objekt T an die Position (5,2) zu verschieben. Bitte bestätigen Sie, dass Sie das gleiche tun, um sicherzustellen, dass unsere Spielbretter synchronisiert bleiben.} \\
            }
        }
    }
    & & \\ \\

    \theutterance \stepcounter{utterance}  
    & & & \multicolumn{2}{p{0.3\linewidth}}{
        \cellcolor[rgb]{0.9,0.9,0.9}{
            \makecell[{{p{\linewidth}}}]{
                \texttt{\tiny{[GM$|$GM]}}
                \texttt{SPRICH: Ich habe mein Spielbrett überprüft und die Position (5,2) ist tatsächlich frei. Ich kann das Objekt T also an diese Position verschieben. Ich werde den Befehl geben, das Objekt T an die Position (5,2) zu verschieben. Bitte bestätigen Sie, dass Sie das gleiche tun, um sicherzustellen, dass unsere Spielbretter synchronisiert bleiben.} \\
            }
        }
    }
    & & \\ \\

    \theutterance \stepcounter{utterance}  
    & & \multicolumn{4}{p{0.6\linewidth}}{
        \cellcolor[rgb]{0.9,0.9,0.9}{
            \makecell[{{p{\linewidth}}}]{
                \texttt{\tiny{[GM$\rangle$P2]}}
                \texttt{Deine Nachricht wurde an deinen Mitspieler übermittelt.} \\
\\ 
\\ 
\texttt{Ihr seid gerade bei Runde 16 von maximal 20.} \\
\texttt{Ihr habt gemeinsam 9 von maximal 12 Strafpunkten angesammelt.} \\
\texttt{Dein Mitspieler hat dir diese Nachricht geschickt:} \\
\\ 
\\ 
\texttt{"Ich habe mein Spielbrett überprüft und die Position (5,2) ist tatsächlich frei. Ich kann das Objekt T also an diese Position verschieben. Ich werde den Befehl geben, das Objekt T an die Position (5,2) zu verschieben. Bitte bestätigen Sie, dass Sie das gleiche tun, um sicherzustellen, dass unsere Spielbretter synchronisiert bleiben."} \\
\\ 
\\ 
\texttt{Wie lautet dein nächster Befehl?} \\
            }
        }
    }
    & \\ \\

    \theutterance \stepcounter{utterance}  
    & & & \multicolumn{4}{p{0.6\linewidth}}{
        \cellcolor[rgb]{0.9,0.9,0.9}{
            \makecell[{{p{\linewidth}}}]{
                \texttt{\tiny{[GM$\langle$P2]}}
                \texttt{SPRICH: Ja, ich bestätige, dass ich das gleiche tun werde! Ich werde den Befehl geben, das Objekt T an die Position (5,2) zu verschieben, um sicherzustellen, dass unsere Spielbretter synchronisiert bleiben. Bitte geh vor und gib den Befehl, und ich werde das gleiche tun. Dann können wir überprüfen, ob das Objekt T erfolgreich an die Position (5,2) verschoben wurde.} \\
            }
        }
    }
     \\ \\

    \theutterance \stepcounter{utterance}  
    & & & \multicolumn{2}{p{0.3\linewidth}}{
        \cellcolor[rgb]{0.9,0.9,0.9}{
            \makecell[{{p{\linewidth}}}]{
                \texttt{\tiny{[GM$|$GM]}}
                \texttt{SPRICH: Ja, ich bestätige, dass ich das gleiche tun werde! Ich werde den Befehl geben, das Objekt T an die Position (5,2) zu verschieben, um sicherzustellen, dass unsere Spielbretter synchronisiert bleiben. Bitte geh vor und gib den Befehl, und ich werde das gleiche tun. Dann können wir überprüfen, ob das Objekt T erfolgreich an die Position (5,2) verschoben wurde.} \\
            }
        }
    }
    & & \\ \\

    \theutterance \stepcounter{utterance}  
    & & \multicolumn{4}{p{0.6\linewidth}}{
        \cellcolor[rgb]{0.9,0.9,0.9}{
            \makecell[{{p{\linewidth}}}]{
                \texttt{\tiny{[P1$\langle$GM]}}
                \texttt{Deine Nachricht wurde an deinen Mitspieler übermittelt.} \\
\\ 
\\ 
\texttt{Ihr seid gerade bei Runde 16 von maximal 20.} \\
\texttt{Ihr habt gemeinsam 9 von maximal 12 Strafpunkten angesammelt.} \\
\texttt{Dein Mitspieler hat dir diese Nachricht geschickt:} \\
\\ 
\\ 
\texttt{"Ja, ich bestätige, dass ich das gleiche tun werde! Ich werde den Befehl geben, das Objekt T an die Position (5,2) zu verschieben, um sicherzustellen, dass unsere Spielbretter synchronisiert bleiben. Bitte geh vor und gib den Befehl, und ich werde das gleiche tun. Dann können wir überprüfen, ob das Objekt T erfolgreich an die Position (5,2) verschoben wurde."} \\
\\ 
\\ 
\texttt{Wie lautet dein nächster Befehl?} \\
            }
        }
    }
    & \\ \\

    \theutterance \stepcounter{utterance}  
    & \multicolumn{4}{p{0.6\linewidth}}{
        \cellcolor[rgb]{0.9,0.9,0.9}{
            \makecell[{{p{\linewidth}}}]{
                \texttt{\tiny{[P1$\rangle$GM]}}
                \texttt{VERSCHIEBE: T, (5, 2)} \\
            }
        }
    }
    & & \\ \\

    \theutterance \stepcounter{utterance}  
    & & & \multicolumn{2}{p{0.3\linewidth}}{
        \cellcolor[rgb]{0.9,0.9,0.9}{
            \makecell[{{p{\linewidth}}}]{
                \texttt{\tiny{[GM$|$GM]}}
                \texttt{VERSCHIEBE: T, (5, 2)} \\
            }
        }
    }
    & & \\ \\

    \theutterance \stepcounter{utterance}  
    & & & \multicolumn{2}{p{0.3\linewidth}}{
        \cellcolor[rgb]{0.9,0.9,0.9}{
            \makecell[{{p{\linewidth}}}]{
                \texttt{\tiny{[GM$|$GM]}}
                \texttt{Strafpunkt: (5,2) ist nicht leer, sondern enthält '─'.} \\
\\ 
\texttt{Ihr habt gemeinsam 10 von maximal 12 Strafpunkten angesammelt.} \\
\texttt{Bitte versuche es noch mal!} \\
            }
        }
    }
    & & \\ \\

    \theutterance \stepcounter{utterance}  
    & & \multicolumn{4}{p{0.6\linewidth}}{
        \cellcolor[rgb]{0.9,0.9,0.9}{
            \makecell[{{p{\linewidth}}}]{
                \texttt{\tiny{[P1$\langle$GM]}}
                \texttt{Strafpunkt: (5,2) ist nicht leer, sondern enthält '─'.} \\
\\ 
\texttt{Ihr habt gemeinsam 10 von maximal 12 Strafpunkten angesammelt.} \\
\texttt{Bitte versuche es noch mal!} \\
            }
        }
    }
    & \\ \\

    \theutterance \stepcounter{utterance}  
    & \multicolumn{4}{p{0.6\linewidth}}{
        \cellcolor[rgb]{0.9,0.9,0.9}{
            \makecell[{{p{\linewidth}}}]{
                \texttt{\tiny{[P1$\rangle$GM]}}
                \texttt{SPRICH: Ich muss das Objekt T an eine andere Position verschieben, da die Position (5,2) auf meinem Spielbrett bereits von '─' besetzt ist. Ich bin mir nicht sicher, wo ich das Objekt T jetzt hinverschieben soll. Können wir bitte unseren Plan überdenken und eine neue Position für das Objekt T finden? Vielleicht sollten wir auch die anderen Objekte auf unserem Spielbrett überprüfen, um sicherzustellen, dass wir noch auf dem richtigen Weg sind.} \\
            }
        }
    }
    & & \\ \\

    \theutterance \stepcounter{utterance}  
    & & & \multicolumn{2}{p{0.3\linewidth}}{
        \cellcolor[rgb]{0.9,0.9,0.9}{
            \makecell[{{p{\linewidth}}}]{
                \texttt{\tiny{[GM$|$GM]}}
                \texttt{SPRICH: Ich muss das Objekt T an eine andere Position verschieben, da die Position (5,2) auf meinem Spielbrett bereits von '─' besetzt ist. Ich bin mir nicht sicher, wo ich das Objekt T jetzt hinverschieben soll. Können wir bitte unseren Plan überdenken und eine neue Position für das Objekt T finden? Vielleicht sollten wir auch die anderen Objekte auf unserem Spielbrett überprüfen, um sicherzustellen, dass wir noch auf dem richtigen Weg sind.} \\
            }
        }
    }
    & & \\ \\

    \theutterance \stepcounter{utterance}  
    & & \multicolumn{4}{p{0.6\linewidth}}{
        \cellcolor[rgb]{0.9,0.9,0.9}{
            \makecell[{{p{\linewidth}}}]{
                \texttt{\tiny{[GM$\rangle$P2]}}
                \texttt{Deine Nachricht wurde an deinen Mitspieler übermittelt.} \\
\\ 
\\ 
\texttt{Ihr seid gerade bei Runde 17 von maximal 20.} \\
\texttt{Ihr habt gemeinsam 10 von maximal 12 Strafpunkten angesammelt.} \\
\texttt{Dein Mitspieler hat dir diese Nachricht geschickt:} \\
\\ 
\\ 
\texttt{"Ich muss das Objekt T an eine andere Position verschieben, da die Position (5,2) auf meinem Spielbrett bereits von '─' besetzt ist. Ich bin mir nicht sicher, wo ich das Objekt T jetzt hinverschieben soll. Können wir bitte unseren Plan überdenken und eine neue Position für das Objekt T finden? Vielleicht sollten wir auch die anderen Objekte auf unserem Spielbrett überprüfen, um sicherzustellen, dass wir noch auf dem richtigen Weg sind."} \\
\\ 
\\ 
\texttt{Wie lautet dein nächster Befehl?} \\
            }
        }
    }
    & \\ \\

    \theutterance \stepcounter{utterance}  
    & & & \multicolumn{4}{p{0.6\linewidth}}{
        \cellcolor[rgb]{0.9,0.9,0.9}{
            \makecell[{{p{\linewidth}}}]{
                \texttt{\tiny{[GM$\langle$P2]}}
                \texttt{SPRICH: Ich denke, es ist eine gute Idee, unseren Plan zu überdenken und eine neue Position für das Objekt T zu finden. Lass uns die anderen Objekte auf unserem Spielbrett überprüfen und sicherstellen, dass wir noch auf dem richtigen Weg sind. Ich schlage vor, dass wir uns auf die Objekte W, I, C und H konzentrieren und prüfen, ob sie an den richtigen Positionen sind. Wenn das der Fall ist, können wir dann entscheiden, wo das Objekt T am besten hingehört. Bitte teile mir mit, wie die Objekte auf deinem Spielbrett verteilt sind, damit wir einen neuen Plan machen können.} \\
            }
        }
    }
     \\ \\

    \theutterance \stepcounter{utterance}  
    & & & \multicolumn{2}{p{0.3\linewidth}}{
        \cellcolor[rgb]{0.9,0.9,0.9}{
            \makecell[{{p{\linewidth}}}]{
                \texttt{\tiny{[GM$|$GM]}}
                \texttt{SPRICH: Ich denke, es ist eine gute Idee, unseren Plan zu überdenken und eine neue Position für das Objekt T zu finden. Lass uns die anderen Objekte auf unserem Spielbrett überprüfen und sicherstellen, dass wir noch auf dem richtigen Weg sind. Ich schlage vor, dass wir uns auf die Objekte W, I, C und H konzentrieren und prüfen, ob sie an den richtigen Positionen sind. Wenn das der Fall ist, können wir dann entscheiden, wo das Objekt T am besten hingehört. Bitte teile mir mit, wie die Objekte auf deinem Spielbrett verteilt sind, damit wir einen neuen Plan machen können.} \\
            }
        }
    }
    & & \\ \\

    \theutterance \stepcounter{utterance}  
    & & \multicolumn{4}{p{0.6\linewidth}}{
        \cellcolor[rgb]{0.9,0.9,0.9}{
            \makecell[{{p{\linewidth}}}]{
                \texttt{\tiny{[P1$\langle$GM]}}
                \texttt{Deine Nachricht wurde an deinen Mitspieler übermittelt.} \\
\\ 
\\ 
\texttt{Ihr seid gerade bei Runde 17 von maximal 20.} \\
\texttt{Ihr habt gemeinsam 10 von maximal 12 Strafpunkten angesammelt.} \\
\texttt{Dein Mitspieler hat dir diese Nachricht geschickt:} \\
\\ 
\\ 
\texttt{"Ich denke, es ist eine gute Idee, unseren Plan zu überdenken und eine neue Position für das Objekt T zu finden. Lass uns die anderen Objekte auf unserem Spielbrett überprüfen und sicherstellen, dass wir noch auf dem richtigen Weg sind. Ich schlage vor, dass wir uns auf die Objekte W, I, C und H konzentrieren und prüfen, ob sie an den richtigen Positionen sind. Wenn das der Fall ist, können wir dann entscheiden, wo das Objekt T am besten hingehört. Bitte teile mir mit, wie die Objekte auf deinem Spielbrett verteilt sind, damit wir einen neuen Plan machen können."} \\
\\ 
\\ 
\texttt{Wie lautet dein nächster Befehl?} \\
            }
        }
    }
    & \\ \\

    \theutterance \stepcounter{utterance}  
    & \multicolumn{4}{p{0.6\linewidth}}{
        \cellcolor[rgb]{0.9,0.9,0.9}{
            \makecell[{{p{\linewidth}}}]{
                \texttt{\tiny{[P1$\rangle$GM]}}
                \texttt{SPRICH: Auf meinem Spielbrett sind die Objekte wie folgt verteilt: W ist in der Spalte 3, Zeile 1; I ist in der Spalte 5, Zeile 1; C ist in der Spalte 1, Zeile 1; H ist in der Spalte 2, Zeile 1. Das Objekt T ist noch in der Spalte 1, Zeile 6. Ich denke, wir sollten versuchen, die Objekte W, I, C und H an den richtigen Positionen zu halten und dann entscheiden, wo das Objekt T am besten hingehört. Wie sieht es auf Ihrem Spielbrett aus? Sind die Objekte auf Ihrem Spielbrett ähnlich verteilt?} \\
            }
        }
    }
    & & \\ \\

    \theutterance \stepcounter{utterance}  
    & & & \multicolumn{2}{p{0.3\linewidth}}{
        \cellcolor[rgb]{0.9,0.9,0.9}{
            \makecell[{{p{\linewidth}}}]{
                \texttt{\tiny{[GM$|$GM]}}
                \texttt{SPRICH: Auf meinem Spielbrett sind die Objekte wie folgt verteilt: W ist in der Spalte 3, Zeile 1; I ist in der Spalte 5, Zeile 1; C ist in der Spalte 1, Zeile 1; H ist in der Spalte 2, Zeile 1. Das Objekt T ist noch in der Spalte 1, Zeile 6. Ich denke, wir sollten versuchen, die Objekte W, I, C und H an den richtigen Positionen zu halten und dann entscheiden, wo das Objekt T am besten hingehört. Wie sieht es auf Ihrem Spielbrett aus? Sind die Objekte auf Ihrem Spielbrett ähnlich verteilt?} \\
            }
        }
    }
    & & \\ \\

    \theutterance \stepcounter{utterance}  
    & & \multicolumn{4}{p{0.6\linewidth}}{
        \cellcolor[rgb]{0.9,0.9,0.9}{
            \makecell[{{p{\linewidth}}}]{
                \texttt{\tiny{[GM$\rangle$P2]}}
                \texttt{Deine Nachricht wurde an deinen Mitspieler übermittelt.} \\
\\ 
\\ 
\texttt{Ihr seid gerade bei Runde 18 von maximal 20.} \\
\texttt{Ihr habt gemeinsam 10 von maximal 12 Strafpunkten angesammelt.} \\
\texttt{Dein Mitspieler hat dir diese Nachricht geschickt:} \\
\\ 
\\ 
\texttt{"Auf meinem Spielbrett sind die Objekte wie folgt verteilt: W ist in der Spalte 3, Zeile 1; I ist in der Spalte 5, Zeile 1; C ist in der Spalte 1, Zeile 1; H ist in der Spalte 2, Zeile 1. Das Objekt T ist noch in der Spalte 1, Zeile 6. Ich denke, wir sollten versuchen, die Objekte W, I, C und H an den richtigen Positionen zu halten und dann entscheiden, wo das Objekt T am besten hingehört. Wie sieht es auf Ihrem Spielbrett aus? Sind die Objekte auf Ihrem Spielbrett ähnlich verteilt?"} \\
\\ 
\\ 
\texttt{Wie lautet dein nächster Befehl?} \\
            }
        }
    }
    & \\ \\

    \theutterance \stepcounter{utterance}  
    & & & \multicolumn{4}{p{0.6\linewidth}}{
        \cellcolor[rgb]{0.9,0.9,0.9}{
            \makecell[{{p{\linewidth}}}]{
                \texttt{\tiny{[GM$\langle$P2]}}
                \texttt{SPRICH: Auf meinem Spielbrett sind die Objekte wie folgt verteilt: W ist in der Spalte 1, Zeile 1; I ist in der Spalte 2, Zeile 1; C ist in der Spalte 3, Zeile 3; H ist in der Spalte 4, Zeile 3. Das Objekt T ist noch in der Spalte 5, Zeile 6. Ich denke, wir sollten versuchen, die Objekte an den Positionen zu halten, die wir ursprünglich geplant hatten, nämlich W, I, T, C, H von links nach rechts in der ersten Zeile. Wir solltenperhaps die Objekte C und H an die Positionen (4,1) und (5,1) verschieben, um Platz für das Objekt T zu schaffen. Was meinst du?} \\
            }
        }
    }
     \\ \\

    \theutterance \stepcounter{utterance}  
    & & & \multicolumn{2}{p{0.3\linewidth}}{
        \cellcolor[rgb]{0.9,0.9,0.9}{
            \makecell[{{p{\linewidth}}}]{
                \texttt{\tiny{[GM$|$GM]}}
                \texttt{SPRICH: Auf meinem Spielbrett sind die Objekte wie folgt verteilt: W ist in der Spalte 1, Zeile 1; I ist in der Spalte 2, Zeile 1; C ist in der Spalte 3, Zeile 3; H ist in der Spalte 4, Zeile 3. Das Objekt T ist noch in der Spalte 5, Zeile 6. Ich denke, wir sollten versuchen, die Objekte an den Positionen zu halten, die wir ursprünglich geplant hatten, nämlich W, I, T, C, H von links nach rechts in der ersten Zeile. Wir solltenperhaps die Objekte C und H an die Positionen (4,1) und (5,1) verschieben, um Platz für das Objekt T zu schaffen. Was meinst du?} \\
            }
        }
    }
    & & \\ \\

    \theutterance \stepcounter{utterance}  
    & & \multicolumn{4}{p{0.6\linewidth}}{
        \cellcolor[rgb]{0.9,0.9,0.9}{
            \makecell[{{p{\linewidth}}}]{
                \texttt{\tiny{[P1$\langle$GM]}}
                \texttt{Deine Nachricht wurde an deinen Mitspieler übermittelt.} \\
\\ 
\\ 
\texttt{Ihr seid gerade bei Runde 18 von maximal 20.} \\
\texttt{Ihr habt gemeinsam 10 von maximal 12 Strafpunkten angesammelt.} \\
\texttt{Dein Mitspieler hat dir diese Nachricht geschickt:} \\
\\ 
\\ 
\texttt{"Auf meinem Spielbrett sind die Objekte wie folgt verteilt: W ist in der Spalte 1, Zeile 1; I ist in der Spalte 2, Zeile 1; C ist in der Spalte 3, Zeile 3; H ist in der Spalte 4, Zeile 3. Das Objekt T ist noch in der Spalte 5, Zeile 6. Ich denke, wir sollten versuchen, die Objekte an den Positionen zu halten, die wir ursprünglich geplant hatten, nämlich W, I, T, C, H von links nach rechts in der ersten Zeile. Wir solltenperhaps die Objekte C und H an die Positionen (4,1) und (5,1) verschieben, um Platz für das Objekt T zu schaffen. Was meinst du?"} \\
\\ 
\\ 
\texttt{Wie lautet dein nächster Befehl?} \\
            }
        }
    }
    & \\ \\

    \theutterance \stepcounter{utterance}  
    & \multicolumn{4}{p{0.6\linewidth}}{
        \cellcolor[rgb]{0.9,0.9,0.9}{
            \makecell[{{p{\linewidth}}}]{
                \texttt{\tiny{[P1$\rangle$GM]}}
                \texttt{SPRICH: Ich denke, das ist ein guter Plan! Ich stimme zu, dass wir die Objekte C und H an die Positionen (4,1) und (5,1) verschieben sollten, um Platz für das Objekt T zu schaffen. Allerdings muss ich zunächst das Objekt H an die Position (5,1) verschieben und dann das Objekt C an die Position (4,1). Ich werde mich bemühen, dies zu tun. Bitte bestätigen Sie, dass Sie das gleiche tun, um sicherzustellen, dass unsere Spielbretter synchronisiert bleiben.} \\
            }
        }
    }
    & & \\ \\

    \theutterance \stepcounter{utterance}  
    & & & \multicolumn{2}{p{0.3\linewidth}}{
        \cellcolor[rgb]{0.9,0.9,0.9}{
            \makecell[{{p{\linewidth}}}]{
                \texttt{\tiny{[GM$|$GM]}}
                \texttt{SPRICH: Ich denke, das ist ein guter Plan! Ich stimme zu, dass wir die Objekte C und H an die Positionen (4,1) und (5,1) verschieben sollten, um Platz für das Objekt T zu schaffen. Allerdings muss ich zunächst das Objekt H an die Position (5,1) verschieben und dann das Objekt C an die Position (4,1). Ich werde mich bemühen, dies zu tun. Bitte bestätigen Sie, dass Sie das gleiche tun, um sicherzustellen, dass unsere Spielbretter synchronisiert bleiben.} \\
            }
        }
    }
    & & \\ \\

    \theutterance \stepcounter{utterance}  
    & & \multicolumn{4}{p{0.6\linewidth}}{
        \cellcolor[rgb]{0.9,0.9,0.9}{
            \makecell[{{p{\linewidth}}}]{
                \texttt{\tiny{[GM$\rangle$P2]}}
                \texttt{Deine Nachricht wurde an deinen Mitspieler übermittelt.} \\
\\ 
\\ 
\texttt{Ihr seid gerade bei Runde 19 von maximal 20.} \\
\texttt{Ihr habt gemeinsam 10 von maximal 12 Strafpunkten angesammelt.} \\
\texttt{Dein Mitspieler hat dir diese Nachricht geschickt:} \\
\\ 
\\ 
\texttt{"Ich denke, das ist ein guter Plan! Ich stimme zu, dass wir die Objekte C und H an die Positionen (4,1) und (5,1) verschieben sollten, um Platz für das Objekt T zu schaffen. Allerdings muss ich zunächst das Objekt H an die Position (5,1) verschieben und dann das Objekt C an die Position (4,1). Ich werde mich bemühen, dies zu tun. Bitte bestätigen Sie, dass Sie das gleiche tun, um sicherzustellen, dass unsere Spielbretter synchronisiert bleiben."} \\
\\ 
\\ 
\texttt{Wie lautet dein nächster Befehl?} \\
            }
        }
    }
    & \\ \\

    \theutterance \stepcounter{utterance}  
    & & & \multicolumn{4}{p{0.6\linewidth}}{
        \cellcolor[rgb]{0.9,0.9,0.9}{
            \makecell[{{p{\linewidth}}}]{
                \texttt{\tiny{[GM$\langle$P2]}}
                \texttt{SPRICH: Ja, ich stimme zu! Ich werde das Objekt H an die Position (5,1) und dann das Objekt C an die Position (4,1) verschieben, um Platz für das Objekt T zu schaffen. Bitte geh vor und gib den Befehl, und ich werde das gleiche tun. Dann können wir überprüfen, ob die Objekte erfolgreich verschoben wurden und ob wir das Objekt T an die richtige Position verschieben können. Ich denke, wir sind nah dran, unsere Ziele zu erreichen! Lass uns fertigstellen!} \\
            }
        }
    }
     \\ \\

    \theutterance \stepcounter{utterance}  
    & & & \multicolumn{2}{p{0.3\linewidth}}{
        \cellcolor[rgb]{0.9,0.9,0.9}{
            \makecell[{{p{\linewidth}}}]{
                \texttt{\tiny{[GM$|$GM]}}
                \texttt{SPRICH: Ja, ich stimme zu! Ich werde das Objekt H an die Position (5,1) und dann das Objekt C an die Position (4,1) verschieben, um Platz für das Objekt T zu schaffen. Bitte geh vor und gib den Befehl, und ich werde das gleiche tun. Dann können wir überprüfen, ob die Objekte erfolgreich verschoben wurden und ob wir das Objekt T an die richtige Position verschieben können. Ich denke, wir sind nah dran, unsere Ziele zu erreichen! Lass uns fertigstellen!} \\
            }
        }
    }
    & & \\ \\

    \theutterance \stepcounter{utterance}  
    & & \multicolumn{4}{p{0.6\linewidth}}{
        \cellcolor[rgb]{0.9,0.9,0.9}{
            \makecell[{{p{\linewidth}}}]{
                \texttt{\tiny{[P1$\langle$GM]}}
                \texttt{Deine Nachricht wurde an deinen Mitspieler übermittelt.} \\
\\ 
\\ 
\texttt{Ihr seid gerade bei Runde 19 von maximal 20.} \\
\texttt{Ihr habt gemeinsam 10 von maximal 12 Strafpunkten angesammelt.} \\
\texttt{Dein Mitspieler hat dir diese Nachricht geschickt:} \\
\\ 
\\ 
\texttt{"Ja, ich stimme zu! Ich werde das Objekt H an die Position (5,1) und dann das Objekt C an die Position (4,1) verschieben, um Platz für das Objekt T zu schaffen. Bitte geh vor und gib den Befehl, und ich werde das gleiche tun. Dann können wir überprüfen, ob die Objekte erfolgreich verschoben wurden und ob wir das Objekt T an die richtige Position verschieben können. Ich denke, wir sind nah dran, unsere Ziele zu erreichen! Lass uns fertigstellen!"} \\
\\ 
\\ 
\texttt{Wie lautet dein nächster Befehl?} \\
            }
        }
    }
    & \\ \\

    \theutterance \stepcounter{utterance}  
    & \multicolumn{4}{p{0.6\linewidth}}{
        \cellcolor[rgb]{0.9,0.9,0.9}{
            \makecell[{{p{\linewidth}}}]{
                \texttt{\tiny{[P1$\rangle$GM]}}
                \texttt{SPRICH: Ich denke, wir sollten vorher überprüfen, ob die Positionen (4,1) und (5,1) auf unseren Spielbrettern frei sind, bevor wir die Objekte C und H dorthin verschieben. Ich werde mein Spielbrett überprüfen und dann den Befehl geben, wenn alles klar ist. Bitte warten Sie einen Moment, während ich mein Spielbrett überprüfe. Ich hoffe, wir können bald fertigstellen und unsere Ziele erreichen!} \\
            }
        }
    }
    & & \\ \\

    \theutterance \stepcounter{utterance}  
    & & & \multicolumn{2}{p{0.3\linewidth}}{
        \cellcolor[rgb]{0.9,0.9,0.9}{
            \makecell[{{p{\linewidth}}}]{
                \texttt{\tiny{[GM$|$GM]}}
                \texttt{SPRICH: Ich denke, wir sollten vorher überprüfen, ob die Positionen (4,1) und (5,1) auf unseren Spielbrettern frei sind, bevor wir die Objekte C und H dorthin verschieben. Ich werde mein Spielbrett überprüfen und dann den Befehl geben, wenn alles klar ist. Bitte warten Sie einen Moment, während ich mein Spielbrett überprüfe. Ich hoffe, wir können bald fertigstellen und unsere Ziele erreichen!} \\
            }
        }
    }
    & & \\ \\

    \theutterance \stepcounter{utterance}  
    & & \multicolumn{4}{p{0.6\linewidth}}{
        \cellcolor[rgb]{0.9,0.9,0.9}{
            \makecell[{{p{\linewidth}}}]{
                \texttt{\tiny{[GM$\rangle$P2]}}
                \texttt{Deine Nachricht wurde an deinen Mitspieler übermittelt.} \\
\\ 
\\ 
\texttt{Ihr seid gerade bei Runde 20 von maximal 20.} \\
\texttt{Ihr habt gemeinsam 10 von maximal 12 Strafpunkten angesammelt.} \\
\texttt{Dein Mitspieler hat dir diese Nachricht geschickt:} \\
\\ 
\\ 
\texttt{"Ich denke, wir sollten vorher überprüfen, ob die Positionen (4,1) und (5,1) auf unseren Spielbrettern frei sind, bevor wir die Objekte C und H dorthin verschieben. Ich werde mein Spielbrett überprüfen und dann den Befehl geben, wenn alles klar ist. Bitte warten Sie einen Moment, während ich mein Spielbrett überprüfe. Ich hoffe, wir können bald fertigstellen und unsere Ziele erreichen!"} \\
\\ 
\\ 
\texttt{Wie lautet dein nächster Befehl?} \\
            }
        }
    }
    & \\ \\

    \theutterance \stepcounter{utterance}  
    & & & \multicolumn{4}{p{0.6\linewidth}}{
        \cellcolor[rgb]{0.9,0.9,0.9}{
            \makecell[{{p{\linewidth}}}]{
                \texttt{\tiny{[GM$\langle$P2]}}
                \texttt{SPRICH: Das ist eine gute Idee! Bitte überprüfe dein Spielbrett und gib mir Bescheid, wenn du bereit bist. Ich werde mein Spielbrett auch überprüfen und dann auf deine Bestätigung warten. Wenn alles klar ist, können wir die Objekte C und H verschieben und dann das Objekt T an die richtige Position bringen. Ich denke, wir sind fast fertig! Ich frage: fertig?} \\
            }
        }
    }
     \\ \\

    \theutterance \stepcounter{utterance}  
    & & & \multicolumn{2}{p{0.3\linewidth}}{
        \cellcolor[rgb]{0.9,0.9,0.9}{
            \makecell[{{p{\linewidth}}}]{
                \texttt{\tiny{[GM$|$GM]}}
                \texttt{SPRICH: Das ist eine gute Idee! Bitte überprüfe dein Spielbrett und gib mir Bescheid, wenn du bereit bist. Ich werde mein Spielbrett auch überprüfen und dann auf deine Bestätigung warten. Wenn alles klar ist, können wir die Objekte C und H verschieben und dann das Objekt T an die richtige Position bringen. Ich denke, wir sind fast fertig! Ich frage: fertig?} \\
            }
        }
    }
    & & \\ \\

    \theutterance \stepcounter{utterance}  
    & & & \multicolumn{2}{p{0.3\linewidth}}{
        \cellcolor[rgb]{0.9,0.9,0.9}{
            \makecell[{{p{\linewidth}}}]{
                \texttt{\tiny{[GM$|$GM]}}
                \texttt{Maximum number of rounds reached} \\
            }
        }
    }
    & & \\ \\

    \theutterance \stepcounter{utterance}  
    & & & \multicolumn{2}{p{0.3\linewidth}}{
        \cellcolor[rgb]{0.9,0.9,0.9}{
            \makecell[{{p{\linewidth}}}]{
                \texttt{\tiny{[GM$|$GM]}}
                \texttt{* success: True} \\
\texttt{* lose: False} \\
\texttt{* aborted: False} \\
\texttt{{-}{-}{-}{-}{-}{-}{-}} \\
\texttt{* Shifts: 1.00} \\
\texttt{* Max Shifts: 8.00} \\
\texttt{* Min Shifts: 4.00} \\
\texttt{* End Distance Sum: 11.21} \\
\texttt{* Init Distance Sum: 19.54} \\
\texttt{* Expected Distance Sum: 20.95} \\
\texttt{* Penalties: 10.00} \\
\texttt{* Max Penalties: 12.00} \\
\texttt{* Rounds: 20.00} \\
\texttt{* Max Rounds: 20.00} \\
\texttt{* Object Count: 5.00} \\
            }
        }
    }
    & & \\ \\

    \theutterance \stepcounter{utterance}  
    & & & \multicolumn{2}{p{0.3\linewidth}}{
        \cellcolor[rgb]{0.9,0.9,0.9}{
            \makecell[{{p{\linewidth}}}]{
                \texttt{\tiny{[GM$|$GM]}}
                \texttt{Initial Grids:} \\
\texttt{```} \\
\texttt{Player 1:     Player 2:} \\
\texttt{ 1234567       1234567} \\
\texttt{╔═══╤═╤═╗     ╔═══╤═╤═╗} \\
\texttt{║CH◌│W│◌║ 1   ║◌W◌│I│◌║ 1} \\
\texttt{╟─┬─┼─┴─╢ 2   ╟─┬─┼─┴─╢ 2} \\
\texttt{║◌│◌│◌◌◌║ 3   ║◌│C│◌◌H║ 3} \\
\texttt{║◌├─┤◌┌─╢ 4   ║◌├─┤◌┌─╢ 4} \\
\texttt{╟─┘◌│◌│◌║ 5   ╟─┘◌│◌│◌║ 5} \\
\texttt{║T◌◌└┬┴─╢ 6   ║◌◌T└┬┴─╢ 6} \\
\texttt{║◌◌◌◌│◌I║ 7   ║◌◌◌◌│◌◌║ 7} \\
\texttt{╚════╧══╝     ╚════╧══╝} \\
\texttt{```} \\
\\ 
\texttt{Final Grids:} \\
\texttt{```} \\
\texttt{Player 1:     Player 2:} \\
\texttt{ 1234567       1234567} \\
\texttt{╔═══╤═╤═╗     ╔═══╤═╤═╗} \\
\texttt{║CHW│I│◌║ 1   ║◌W◌│I│◌║ 1} \\
\texttt{╟─┬─┼─┴─╢ 2   ╟─┬─┼─┴─╢ 2} \\
\texttt{║◌│◌│◌◌◌║ 3   ║◌│C│◌◌H║ 3} \\
\texttt{║◌├─┤◌┌─╢ 4   ║◌├─┤◌┌─╢ 4} \\
\texttt{╟─┘◌│◌│◌║ 5   ╟─┘◌│◌│◌║ 5} \\
\texttt{║T◌◌└┬┴─╢ 6   ║◌◌T└┬┴─╢ 6} \\
\texttt{║◌◌◌◌│◌◌║ 7   ║◌◌◌◌│◌◌║ 7} \\
\texttt{╚════╧══╝     ╚════╧══╝} \\
\texttt{```} \\
            }
        }
    }
    & & \\ \\

\end{supertabular}
}

\end{document}
