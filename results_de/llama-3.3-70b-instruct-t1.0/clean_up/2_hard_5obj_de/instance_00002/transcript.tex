
\documentclass{article}
\usepackage{colortbl}
\usepackage{makecell}
\usepackage{multirow}
\usepackage{supertabular}

\begin{document}

\newcounter{utterance}

\twocolumn

{ \footnotesize  \setcounter{utterance}{1}
\setlength{\tabcolsep}{0pt}
\begin{supertabular}{c@{$\;$}|p{.15\linewidth}@{}p{.15\linewidth}p{.15\linewidth}p{.15\linewidth}p{.15\linewidth}p{.15\linewidth}}

    \# & $\;$A & \multicolumn{4}{c}{Game Master} & $\;\:$B\\
    \hline 

    \theutterance \stepcounter{utterance}  

    & & \multicolumn{4}{p{0.6\linewidth}}{\cellcolor[rgb]{0.9,0.9,0.9}{%
	\makecell[{{p{\linewidth}}}]{% 
	  \tt {\tiny [A$\langle$GM]}  
	 Ich bin der Spielleiter, und du spielst ein kooperatives Spiel mit dem folgenden Raster als Spielbrett:\\ \tt \\ \tt ```\\ \tt  1234567\\ \tt ╔═╤═╤═╤═╗\\ \tt ║T│◌│◌│◌║ 1\\ \tt ╟─┼─┤I│◌║ 2\\ \tt ║W│◌│◌│H║ 3\\ \tt ║◌│◌├─┴─╢ 4\\ \tt ╟─┴─┤◌◌◌║ 5\\ \tt ║◌C◌└─┬─╢ 6\\ \tt ║◌◌◌◌◌│◌║ 7\\ \tt ╚═════╧═╝\\ \tt \\ \tt ```\\ \tt \\ \tt * Am oberen Rand sind die x-Koordinaten angegeben, die nach rechts hin zunehmen, und am rechten Rand die y-Koordinaten, die nach unten zunehmen.\\ \tt * Die folgenden Objekte sind zufällig auf deinem Raster platziert: 'W', 'I', 'T', 'C', 'H'.\\ \tt \\ \tt \\ \tt Dein Mitspieler sieht eine Variation des Spielbretts, auf dem dieselben Objekte an anderen zufälligen Positionen platziert sind. Du kannst das Spielbrett des anderen Spielers nicht sehen, und er kann deins nicht sehen.\\ \tt \\ \tt \\ \tt **Ziel des Spiels:**\\ \tt \\ \tt Beide Spieler müssen die Objekte auf ihrem jeweiligen Spielbrett so bewegen, dass identische Objekte am Ende an denselben Koordinaten liegen. Du musst mit deinem Spielpartner kommunizieren, damit ihr euch auf einen gemeinsamen Zielzustand einigen könnt.\\ \tt \\ \tt \\ \tt **Spielregeln:**\\ \tt \\ \tt * In jedem Spielzug kannst du genau eine der folgenden zwei Aktionen ausführen:\\ \tt 1. `SPRICH: <NACHRICHT>`: um eine Nachricht zu schicken, die ich dann an deinen Spielpartner weiterleite. Als Nachricht gilt alles bis zum nächsten Zeilenumbruch.\\ \tt 2. `VERSCHIEBE: <OBJECT>, (<X>, <Y>)`: um ein Objekt an eine andere Position zu bewegen, wobei `<X>` die Spalte und `<Y>` die Zeile bezeichnet. Ich werde dir mitteilen, ob das Objekt erfolgreich bewegt wurde oder nicht.\\ \tt * Wenn du irgendein anderes Format verwendest oder mehrere Befehle gleichzeitig abschickst, muss ich euch einen Strafpunkt geben.\\ \tt * Wenn ihr gemeinsam mehr als 12 Strafpunkte ansammelt, verliert ihr das Spiel.\\ \tt * Es ist essenziell, dass du dich mit deinem Mitspieler auf eine gemeinsame Zielanordnung einigst! Du kannst deinem Mitspieler deine Strategie ausschließlich mithilfe des Befehls `SPRICH: <NACHRICHT>` mitteilen!\\ \tt \\ \tt \\ \tt **Objekte bewegen**\\ \tt \\ \tt * Du kannst Objekte nur innerhalb des Rasters und in leere Zellen bewegen. Leere Zellen erkennst am Zeichen '◌'.\\ \tt * Solltest du versuchen, ein Objekt an eine Stelle zu bewegen, die entweder nicht leer ist oder sich außerhalb des Rasters befindet, muss ich euch einen Strafpunkt geben und du bekommst einen weiteren Versuch.\\ \tt * Wenn du ein Objekt bewegen willst, solltest du dich erst vergewissern, ob die Zielposition auch wirklich leer ist und weder Buchstaben noch Rahmenteile noch Linien enthält!\\ \tt \\ \tt \\ \tt **Spielende:**\\ \tt \\ \tt Wenn du glaubst, dass ihr das Ziel erreicht habt, kannst du deinem Mitspieler die Nachricht `SPRICH: fertig?` schicken. Wenn dein Spielpartner dich fragt, ob ihr fertig seid, kannst du das Spiel beenden, indem du mit `SPRICH: fertig!` antwortest.\\ \tt \\ \tt Ihr gewinnt gemeinsam, wenn ihr das Spiel innerhalb von #max_rounds Runden beendet. Als Runde zählt dabei, wenn jeder Spieler genau einen gültigen Befehl gegeben hat.\\ \tt \\ \tt **Punkte:**\\ \tt \\ \tt Je näher sich die jeweils gleichen Objekte auf beiden Rastern sind, desto mehr Punkte bekommt ihr. Für Strafpunkte gibt es Punktabzug. Könnt ihr den Rekord brechen?\\ \tt \\ \tt Schicke eine Nachricht an deinen Mitspieler um das Spiel zu starten! 
	  } 
	   } 
	   } 
	 & \\ 
 

    \theutterance \stepcounter{utterance}  

    & \multicolumn{4}{p{0.6\linewidth}}{\cellcolor[rgb]{0.8,1,0.9}{%
	\makecell[{{p{\linewidth}}}]{% 
	  \tt {\tiny [A$\rangle$GM]}  
	 SPRICH: Hallo, ich schlage vor, wir beschreiben unsere Startpositionen, um eine gemeinsame Strategie zu entwickeln. Ich beginne: Ich habe ein 'T' bei (1,1), ein 'I' bei (4,2), ein 'W' bei (1,3) und ein 'H' bei (4,3) sowie ein 'C' bei (2,6). Wie sieht deine Startposition aus? 
	  } 
	   } 
	   } 
	 & & \\ 
 

    \theutterance \stepcounter{utterance}  

    & & & \multicolumn{2}{p{0.3\linewidth}}{\cellcolor[rgb]{0.95,0.95,0.95}{%
	\makecell[{{p{\linewidth}}}]{% 
	  \tt {\tiny [GM$|$GM]}  
	 SPRICH: Hallo, ich schlage vor, wir beschreiben unsere Startpositionen, um eine gemeinsame Strategie zu entwickeln. Ich beginne: Ich habe ein 'T' bei (1,1), ein 'I' bei (4,2), ein 'W' bei (1,3) und ein 'H' bei (4,3) sowie ein 'C' bei (2,6). Wie sieht deine Startposition aus? 
	  } 
	   } 
	   } 
	 & & \\ 
 

    \theutterance \stepcounter{utterance}  

    & & \multicolumn{4}{p{0.6\linewidth}}{\cellcolor[rgb]{0.9,0.9,0.9}{%
	\makecell[{{p{\linewidth}}}]{% 
	  \tt {\tiny [GM$\rangle$B]}  
	 Ich bin der Spielleiter, und du spielst ein kooperatives Spiel mit dem folgenden Raster als Spielbrett:\\ \tt \\ \tt ```\\ \tt  1234567\\ \tt ╔═╤═╤═╤═╗\\ \tt ║◌│◌│◌│◌║ 1\\ \tt ╟─┼─┤◌│◌║ 2\\ \tt ║◌│W│◌│◌║ 3\\ \tt ║◌│C├─┴─╢ 4\\ \tt ╟─┴─┤I◌◌║ 5\\ \tt ║◌H◌└─┬─╢ 6\\ \tt ║◌◌T◌◌│◌║ 7\\ \tt ╚═════╧═╝\\ \tt \\ \tt ```\\ \tt \\ \tt * Am oberen Rand sind die x-Koordinaten angegeben, die nach rechts hin zunehmen, und am rechten Rand die y-Koordinaten, die nach unten zunehmen.\\ \tt * Die folgenden Objekte sind zufällig auf deinem Raster platziert: 'W', 'I', 'T', 'C', 'H'.\\ \tt \\ \tt \\ \tt Dein Mitspieler sieht eine Variation des Spielbretts, auf dem dieselben Objekte an anderen zufälligen Positionen platziert sind. Du kannst das Spielbrett des anderen Spielers nicht sehen, und er kann deins nicht sehen.\\ \tt \\ \tt \\ \tt **Ziel des Spiels:**\\ \tt \\ \tt Beide Spieler müssen die Objekte auf ihrem jeweiligen Spielbrett so bewegen, dass identische Objekte am Ende an denselben Koordinaten liegen. Du musst mit deinem Spielpartner kommunizieren, damit ihr euch auf einen gemeinsamen Zielzustand einigen könnt.\\ \tt \\ \tt \\ \tt **Spielregeln:**\\ \tt \\ \tt * In jedem Spielzug kannst du genau eine der folgenden zwei Aktionen ausführen:\\ \tt 1. `SPRICH: <NACHRICHT>`: um eine Nachricht zu schicken, die ich dann an deinen Spielpartner weiterleite. Als Nachricht gilt alles bis zum nächsten Zeilenumbruch.\\ \tt 2. `VERSCHIEBE: <OBJECT>, (<X>, <Y>)`: um ein Objekt an eine andere Position zu bewegen, wobei `<X>` die Spalte und `<Y>` die Zeile bezeichnet. Ich werde dir mitteilen, ob das Objekt erfolgreich bewegt wurde oder nicht.\\ \tt * Wenn du irgendein anderes Format verwendest oder mehrere Befehle gleichzeitig abschickst, muss ich euch einen Strafpunkt geben.\\ \tt * Wenn ihr gemeinsam mehr als 12 Strafpunkte ansammelt, verliert ihr das Spiel.\\ \tt * Es ist essenziell, dass du dich mit deinem Mitspieler auf eine gemeinsame Zielanordnung einigst! Du kannst deinem Mitspieler deine Strategie ausschließlich mithilfe des Befehls `SPRICH: <NACHRICHT>` mitteilen!\\ \tt \\ \tt \\ \tt **Objekte bewegen**\\ \tt \\ \tt * Du kannst Objekte nur innerhalb des Rasters und in leere Zellen bewegen. Leere Zellen erkennst am Zeichen '◌'.\\ \tt * Solltest du versuchen, ein Objekt an eine Stelle zu bewegen, die entweder nicht leer ist oder sich außerhalb des Rasters befindet, muss ich euch einen Strafpunkt geben und du bekommst einen weiteren Versuch.\\ \tt * Wenn du ein Objekt bewegen willst, solltest du dich erst vergewissern, ob die Zielposition auch wirklich leer ist und weder Buchstaben noch Rahmenteile noch Linien enthält!\\ \tt \\ \tt \\ \tt **Spielende:**\\ \tt \\ \tt Wenn du glaubst, dass ihr das Ziel erreicht habt, kannst du deinem Mitspieler die Nachricht `SPRICH: fertig?` schicken. Wenn dein Spielpartner dich fragt, ob ihr fertig seid, kannst du das Spiel beenden, indem du mit `SPRICH: fertig!` antwortest.\\ \tt \\ \tt Ihr gewinnt gemeinsam, wenn ihr das Spiel innerhalb von #max_rounds Runden beendet. Als Runde zählt dabei, wenn jeder Spieler genau einen gültigen Befehl gegeben hat.\\ \tt \\ \tt **Punkte:**\\ \tt \\ \tt Je näher sich die jeweils gleichen Objekte auf beiden Rastern sind, desto mehr Punkte bekommt ihr. Für Strafpunkte gibt es Punktabzug. Könnt ihr den Rekord brechen?\\ \tt \\ \tt Dein Mitspieler hat das Spiel mit dieser Nachricht begonnen:\\ \tt \\ \tt "Hallo, ich schlage vor, wir beschreiben unsere Startpositionen, um eine gemeinsame Strategie zu entwickeln. Ich beginne: Ich habe ein 'T' bei (1,1), ein 'I' bei (4,2), ein 'W' bei (1,3) und ein 'H' bei (4,3) sowie ein 'C' bei (2,6). Wie sieht deine Startposition aus?"\\ \tt \\ \tt Wie lautet dein erster Befehl? 
	  } 
	   } 
	   } 
	 & \\ 
 

    \theutterance \stepcounter{utterance}  

    & & & \multicolumn{4}{p{0.6\linewidth}}{\cellcolor[rgb]{1,0.85,0.72}{%
	\makecell[{{p{\linewidth}}}]{% 
	  \tt {\tiny [GM$\langle$B]}  
	 Mein erster Befehl ist: `SPRICH: Hallo, ich habe ein 'W' bei (2,3), ein 'I' bei (5,4), ein 'T' bei (3,7), ein 'H' bei (1,6) und ein 'C' bei (2,4). Ich stimme deiner Strategie zu, unsere Startpositionen auszutauschen, um gemeinsam eine optimale Anordnung zu finden. Als nächstes könnten wir überlegen, wo wir die Objekte platzieren wollen, um sie auf beiden Spielbrettern an den gleichen Koordinaten zu haben. Was schlägst du als Zielpositionen für die Objekte vor?` 
	  } 
	   } 
	   } 
	  \\ 
 

    \theutterance \stepcounter{utterance}  

    & & & \multicolumn{2}{p{0.3\linewidth}}{\cellcolor[rgb]{0.95,0.95,0.95}{%
	\makecell[{{p{\linewidth}}}]{% 
	  \tt {\tiny [GM$|$GM]}  
	 Mein erster Befehl ist: `SPRICH: Hallo, ich habe ein 'W' bei (2,3), ein 'I' bei (5,4), ein 'T' bei (3,7), ein 'H' bei (1,6) und ein 'C' bei (2,4). Ich stimme deiner Strategie zu, unsere Startpositionen auszutauschen, um gemeinsam eine optimale Anordnung zu finden. Als nächstes könnten wir überlegen, wo wir die Objekte platzieren wollen, um sie auf beiden Spielbrettern an den gleichen Koordinaten zu haben. Was schlägst du als Zielpositionen für die Objekte vor?` 
	  } 
	   } 
	   } 
	 & & \\ 
 

    \theutterance \stepcounter{utterance}  

    & & & \multicolumn{2}{p{0.3\linewidth}}{\cellcolor[rgb]{0.95,0.95,0.95}{%
	\makecell[{{p{\linewidth}}}]{% 
	  \tt {\tiny [GM$|$GM]}  
	 Invalid format: head is not empty: 'Mein erster Befehl ist: ' 
	  } 
	   } 
	   } 
	 & & \\ 
 

    \theutterance \stepcounter{utterance}  

    & & \multicolumn{4}{p{0.6\linewidth}}{\cellcolor[rgb]{0.9,0.9,0.9}{%
	\makecell[{{p{\linewidth}}}]{% 
	  \tt {\tiny [GM$\rangle$B]}  
	 Strafpunkt: Deine Nachricht darf nichts vor dem Befehl enthalten! Stelle sicher, dass deine Antwort ausschließlich einen der beiden Befehle `SPRICH: <NACHRICHT>` und `VERSCHIEBE: <OBJECT>, (<X>, <Y>)` und sonst nichts enthält!\\ \tt \\ \tt Ihr habt gemeinsam 1 von maximal 12 Strafpunkten angesammelt.\\ \tt Bitte versuche es noch mal! 
	  } 
	   } 
	   } 
	 & \\ 
 

    \theutterance \stepcounter{utterance}  

    & & & \multicolumn{4}{p{0.6\linewidth}}{\cellcolor[rgb]{1,0.85,0.72}{%
	\makecell[{{p{\linewidth}}}]{% 
	  \tt {\tiny [GM$\langle$B]}  
	 SPRICH: Ich habe ein 'W' bei (2,3), ein 'I' bei (5,4), ein 'T' bei (3,7), ein 'H' bei (1,6) und ein 'C' bei (2,4). Ich stimme deiner Strategie zu, unsere Startpositionen auszutauschen, um gemeinsam eine optimale Anordnung zu finden. Als nächstes könnten wir überlegen, wo wir die Objekte platzieren wollen, um sie auf beiden Spielbrettern an den gleichen Koordinaten zu haben. Was schlägst du als Zielpositionen für die Objekte vor? 
	  } 
	   } 
	   } 
	  \\ 
 

    \theutterance \stepcounter{utterance}  

    & & & \multicolumn{2}{p{0.3\linewidth}}{\cellcolor[rgb]{0.95,0.95,0.95}{%
	\makecell[{{p{\linewidth}}}]{% 
	  \tt {\tiny [GM$|$GM]}  
	 SPRICH: Ich habe ein 'W' bei (2,3), ein 'I' bei (5,4), ein 'T' bei (3,7), ein 'H' bei (1,6) und ein 'C' bei (2,4). Ich stimme deiner Strategie zu, unsere Startpositionen auszutauschen, um gemeinsam eine optimale Anordnung zu finden. Als nächstes könnten wir überlegen, wo wir die Objekte platzieren wollen, um sie auf beiden Spielbrettern an den gleichen Koordinaten zu haben. Was schlägst du als Zielpositionen für die Objekte vor? 
	  } 
	   } 
	   } 
	 & & \\ 
 

    \theutterance \stepcounter{utterance}  

    & & \multicolumn{4}{p{0.6\linewidth}}{\cellcolor[rgb]{0.9,0.9,0.9}{%
	\makecell[{{p{\linewidth}}}]{% 
	  \tt {\tiny [A$\langle$GM]}  
	 Deine Nachricht wurde an deinen Mitspieler übermittelt.\\ \tt \\ \tt \\ \tt Ihr seid gerade bei Runde 1 von maximal 20.\\ \tt Ihr habt gemeinsam 1 von maximal 12 Strafpunkten angesammelt.\\ \tt Dein Mitspieler hat dir diese Nachricht geschickt:\\ \tt \\ \tt \\ \tt "Ich habe ein 'W' bei (2,3), ein 'I' bei (5,4), ein 'T' bei (3,7), ein 'H' bei (1,6) und ein 'C' bei (2,4). Ich stimme deiner Strategie zu, unsere Startpositionen auszutauschen, um gemeinsam eine optimale Anordnung zu finden. Als nächstes könnten wir überlegen, wo wir die Objekte platzieren wollen, um sie auf beiden Spielbrettern an den gleichen Koordinaten zu haben. Was schlägst du als Zielpositionen für die Objekte vor?"\\ \tt \\ \tt \\ \tt Wie lautet dein nächster Befehl? 
	  } 
	   } 
	   } 
	 & \\ 
 

    \theutterance \stepcounter{utterance}  

    & \multicolumn{4}{p{0.6\linewidth}}{\cellcolor[rgb]{0.8,1,0.9}{%
	\makecell[{{p{\linewidth}}}]{% 
	  \tt {\tiny [A$\rangle$GM]}  
	 SPRICH: Basierend auf unseren Startpositionen schlage ich vor, die Objekte an den folgenden Koordinaten zu platzieren: 'T' bei (1,1), 'I' bei (2,2), 'W' bei (3,3), 'H' bei (4,4) und 'C' bei (5,5). Dies würde eine symmetrische Anordnung auf beiden Spielbrettern erlauben und die Bewegungen minimieren. Was meinst du dazu? Sollten wir diese Zielpositionen anstreben? 
	  } 
	   } 
	   } 
	 & & \\ 
 

    \theutterance \stepcounter{utterance}  

    & & & \multicolumn{2}{p{0.3\linewidth}}{\cellcolor[rgb]{0.95,0.95,0.95}{%
	\makecell[{{p{\linewidth}}}]{% 
	  \tt {\tiny [GM$|$GM]}  
	 SPRICH: Basierend auf unseren Startpositionen schlage ich vor, die Objekte an den folgenden Koordinaten zu platzieren: 'T' bei (1,1), 'I' bei (2,2), 'W' bei (3,3), 'H' bei (4,4) und 'C' bei (5,5). Dies würde eine symmetrische Anordnung auf beiden Spielbrettern erlauben und die Bewegungen minimieren. Was meinst du dazu? Sollten wir diese Zielpositionen anstreben? 
	  } 
	   } 
	   } 
	 & & \\ 
 

    \theutterance \stepcounter{utterance}  

    & & \multicolumn{4}{p{0.6\linewidth}}{\cellcolor[rgb]{0.9,0.9,0.9}{%
	\makecell[{{p{\linewidth}}}]{% 
	  \tt {\tiny [GM$\rangle$B]}  
	 Deine Nachricht wurde an deinen Mitspieler übermittelt.\\ \tt \\ \tt \\ \tt Ihr seid gerade bei Runde 2 von maximal 20.\\ \tt Ihr habt gemeinsam 1 von maximal 12 Strafpunkten angesammelt.\\ \tt Dein Mitspieler hat dir diese Nachricht geschickt:\\ \tt \\ \tt \\ \tt "Basierend auf unseren Startpositionen schlage ich vor, die Objekte an den folgenden Koordinaten zu platzieren: 'T' bei (1,1), 'I' bei (2,2), 'W' bei (3,3), 'H' bei (4,4) und 'C' bei (5,5). Dies würde eine symmetrische Anordnung auf beiden Spielbrettern erlauben und die Bewegungen minimieren. Was meinst du dazu? Sollten wir diese Zielpositionen anstreben?"\\ \tt \\ \tt \\ \tt Wie lautet dein nächster Befehl? 
	  } 
	   } 
	   } 
	 & \\ 
 

    \theutterance \stepcounter{utterance}  

    & & & \multicolumn{4}{p{0.6\linewidth}}{\cellcolor[rgb]{1,0.85,0.72}{%
	\makecell[{{p{\linewidth}}}]{% 
	  \tt {\tiny [GM$\langle$B]}  
	 SPRICH: Ich denke, deine vorgeschlagene Anordnung ist eine gute Idee, da sie symmetrisch ist und die Anzahl der notwendigen Bewegungen minimiert. Ich stimme zu, diese Zielpositionen anzustreben. Bevor wir beginnen, die Objekte zu bewegen, sollten wir sicherstellen, dass wir beide die gleiche Strategie verfolgen, um Konflikte zu vermeiden. Wir könnten abwechselnd die Objekte bewegen, um unsere Fortschritte zu koordinieren. Wie sollen wir vorgehen, um unsere Schritte zu synchronisieren? 
	  } 
	   } 
	   } 
	  \\ 
 

    \theutterance \stepcounter{utterance}  

    & & & \multicolumn{2}{p{0.3\linewidth}}{\cellcolor[rgb]{0.95,0.95,0.95}{%
	\makecell[{{p{\linewidth}}}]{% 
	  \tt {\tiny [GM$|$GM]}  
	 SPRICH: Ich denke, deine vorgeschlagene Anordnung ist eine gute Idee, da sie symmetrisch ist und die Anzahl der notwendigen Bewegungen minimiert. Ich stimme zu, diese Zielpositionen anzustreben. Bevor wir beginnen, die Objekte zu bewegen, sollten wir sicherstellen, dass wir beide die gleiche Strategie verfolgen, um Konflikte zu vermeiden. Wir könnten abwechselnd die Objekte bewegen, um unsere Fortschritte zu koordinieren. Wie sollen wir vorgehen, um unsere Schritte zu synchronisieren? 
	  } 
	   } 
	   } 
	 & & \\ 
 

    \theutterance \stepcounter{utterance}  

    & & \multicolumn{4}{p{0.6\linewidth}}{\cellcolor[rgb]{0.9,0.9,0.9}{%
	\makecell[{{p{\linewidth}}}]{% 
	  \tt {\tiny [A$\langle$GM]}  
	 Deine Nachricht wurde an deinen Mitspieler übermittelt.\\ \tt \\ \tt \\ \tt Ihr seid gerade bei Runde 2 von maximal 20.\\ \tt Ihr habt gemeinsam 1 von maximal 12 Strafpunkten angesammelt.\\ \tt Dein Mitspieler hat dir diese Nachricht geschickt:\\ \tt \\ \tt \\ \tt "Ich denke, deine vorgeschlagene Anordnung ist eine gute Idee, da sie symmetrisch ist und die Anzahl der notwendigen Bewegungen minimiert. Ich stimme zu, diese Zielpositionen anzustreben. Bevor wir beginnen, die Objekte zu bewegen, sollten wir sicherstellen, dass wir beide die gleiche Strategie verfolgen, um Konflikte zu vermeiden. Wir könnten abwechselnd die Objekte bewegen, um unsere Fortschritte zu koordinieren. Wie sollen wir vorgehen, um unsere Schritte zu synchronisieren?"\\ \tt \\ \tt \\ \tt Wie lautet dein nächster Befehl? 
	  } 
	   } 
	   } 
	 & \\ 
 

    \theutterance \stepcounter{utterance}  

    & \multicolumn{4}{p{0.6\linewidth}}{\cellcolor[rgb]{0.8,1,0.9}{%
	\makecell[{{p{\linewidth}}}]{% 
	  \tt {\tiny [A$\rangle$GM]}  
	 SPRICH: Ich denke, es ist eine gute Idee, abwechselnd die Objekte zu bewegen, um unsere Fortschritte zu koordinieren. Ich schlage vor, dass wir beginnen, indem wir das Objekt bewegen, das am weitesten von seiner Zielposition entfernt ist. Auf meinem Spielbrett ist das 'C', das von (2,6) zu (5,5) bewegt werden muss. Ich werde beginnen, indem ich das 'C' um eine Position nach rechts und oben bewege. Ich werde dich über meine Fortschritte informieren, und dann kannst du deine nächste Bewegung planen. Lass uns loslegen! 
	  } 
	   } 
	   } 
	 & & \\ 
 

    \theutterance \stepcounter{utterance}  

    & & & \multicolumn{2}{p{0.3\linewidth}}{\cellcolor[rgb]{0.95,0.95,0.95}{%
	\makecell[{{p{\linewidth}}}]{% 
	  \tt {\tiny [GM$|$GM]}  
	 SPRICH: Ich denke, es ist eine gute Idee, abwechselnd die Objekte zu bewegen, um unsere Fortschritte zu koordinieren. Ich schlage vor, dass wir beginnen, indem wir das Objekt bewegen, das am weitesten von seiner Zielposition entfernt ist. Auf meinem Spielbrett ist das 'C', das von (2,6) zu (5,5) bewegt werden muss. Ich werde beginnen, indem ich das 'C' um eine Position nach rechts und oben bewege. Ich werde dich über meine Fortschritte informieren, und dann kannst du deine nächste Bewegung planen. Lass uns loslegen! 
	  } 
	   } 
	   } 
	 & & \\ 
 

    \theutterance \stepcounter{utterance}  

    & & \multicolumn{4}{p{0.6\linewidth}}{\cellcolor[rgb]{0.9,0.9,0.9}{%
	\makecell[{{p{\linewidth}}}]{% 
	  \tt {\tiny [GM$\rangle$B]}  
	 Deine Nachricht wurde an deinen Mitspieler übermittelt.\\ \tt \\ \tt \\ \tt Ihr seid gerade bei Runde 3 von maximal 20.\\ \tt Ihr habt gemeinsam 1 von maximal 12 Strafpunkten angesammelt.\\ \tt Dein Mitspieler hat dir diese Nachricht geschickt:\\ \tt \\ \tt \\ \tt "Ich denke, es ist eine gute Idee, abwechselnd die Objekte zu bewegen, um unsere Fortschritte zu koordinieren. Ich schlage vor, dass wir beginnen, indem wir das Objekt bewegen, das am weitesten von seiner Zielposition entfernt ist. Auf meinem Spielbrett ist das 'C', das von (2,6) zu (5,5) bewegt werden muss. Ich werde beginnen, indem ich das 'C' um eine Position nach rechts und oben bewege. Ich werde dich über meine Fortschritte informieren, und dann kannst du deine nächste Bewegung planen. Lass uns loslegen!"\\ \tt \\ \tt \\ \tt Wie lautet dein nächster Befehl? 
	  } 
	   } 
	   } 
	 & \\ 
 

    \theutterance \stepcounter{utterance}  

    & & & \multicolumn{4}{p{0.6\linewidth}}{\cellcolor[rgb]{1,0.85,0.72}{%
	\makecell[{{p{\linewidth}}}]{% 
	  \tt {\tiny [GM$\langle$B]}  
	 SPRICH: Ich verstehe deinen Plan und denke, es ist eine gute Strategie, mit dem Objekt zu beginnen, das am weitesten von seiner Zielposition entfernt ist. Da du bereits angefangen hast, das 'C' auf deinem Spielbrett zu bewegen, werde ich mein 'C' von (2,4) zu (5,5) bewegen, indem ich es Schritt für Schritt in die richtige Position bringe. Ich begins mit dem Versuch, es nach rechts zu bewegen. Bitte halte mich über deine Fortschritte auf dem Laufenden, damit ich meine nächsten Schritte entsprechend planen kann. 
	  } 
	   } 
	   } 
	  \\ 
 

    \theutterance \stepcounter{utterance}  

    & & & \multicolumn{2}{p{0.3\linewidth}}{\cellcolor[rgb]{0.95,0.95,0.95}{%
	\makecell[{{p{\linewidth}}}]{% 
	  \tt {\tiny [GM$|$GM]}  
	 SPRICH: Ich verstehe deinen Plan und denke, es ist eine gute Strategie, mit dem Objekt zu beginnen, das am weitesten von seiner Zielposition entfernt ist. Da du bereits angefangen hast, das 'C' auf deinem Spielbrett zu bewegen, werde ich mein 'C' von (2,4) zu (5,5) bewegen, indem ich es Schritt für Schritt in die richtige Position bringe. Ich begins mit dem Versuch, es nach rechts zu bewegen. Bitte halte mich über deine Fortschritte auf dem Laufenden, damit ich meine nächsten Schritte entsprechend planen kann. 
	  } 
	   } 
	   } 
	 & & \\ 
 

    \theutterance \stepcounter{utterance}  

    & & \multicolumn{4}{p{0.6\linewidth}}{\cellcolor[rgb]{0.9,0.9,0.9}{%
	\makecell[{{p{\linewidth}}}]{% 
	  \tt {\tiny [A$\langle$GM]}  
	 Deine Nachricht wurde an deinen Mitspieler übermittelt.\\ \tt \\ \tt \\ \tt Ihr seid gerade bei Runde 3 von maximal 20.\\ \tt Ihr habt gemeinsam 1 von maximal 12 Strafpunkten angesammelt.\\ \tt Dein Mitspieler hat dir diese Nachricht geschickt:\\ \tt \\ \tt \\ \tt "Ich verstehe deinen Plan und denke, es ist eine gute Strategie, mit dem Objekt zu beginnen, das am weitesten von seiner Zielposition entfernt ist. Da du bereits angefangen hast, das 'C' auf deinem Spielbrett zu bewegen, werde ich mein 'C' von (2,4) zu (5,5) bewegen, indem ich es Schritt für Schritt in die richtige Position bringe. Ich begins mit dem Versuch, es nach rechts zu bewegen. Bitte halte mich über deine Fortschritte auf dem Laufenden, damit ich meine nächsten Schritte entsprechend planen kann."\\ \tt \\ \tt \\ \tt Wie lautet dein nächster Befehl? 
	  } 
	   } 
	   } 
	 & \\ 
 

    \theutterance \stepcounter{utterance}  

    & \multicolumn{4}{p{0.6\linewidth}}{\cellcolor[rgb]{0.8,1,0.9}{%
	\makecell[{{p{\linewidth}}}]{% 
	  \tt {\tiny [A$\rangle$GM]}  
	 VERSCHIEBE: C, (3,6) 
	  } 
	   } 
	   } 
	 & & \\ 
 

    \theutterance \stepcounter{utterance}  

    & & & \multicolumn{2}{p{0.3\linewidth}}{\cellcolor[rgb]{0.95,0.95,0.95}{%
	\makecell[{{p{\linewidth}}}]{% 
	  \tt {\tiny [GM$|$GM]}  
	 VERSCHIEBE: C, (3,6) 
	  } 
	   } 
	   } 
	 & & \\ 
 

    \theutterance \stepcounter{utterance}  

    & & & \multicolumn{2}{p{0.3\linewidth}}{\cellcolor[rgb]{0.95,0.95,0.95}{%
	\makecell[{{p{\linewidth}}}]{% 
	  \tt {\tiny [GM$|$GM]}  
	 'C' erfolgreich nach (3,6) verschoben. Dein Raster sieht jetzt so aus:\\ \tt ```\\ \tt  1234567\\ \tt ╔═╤═╤═╤═╗\\ \tt ║T│◌│◌│◌║ 1\\ \tt ╟─┼─┤I│◌║ 2\\ \tt ║W│◌│◌│H║ 3\\ \tt ║◌│◌├─┴─╢ 4\\ \tt ╟─┴─┤◌◌◌║ 5\\ \tt ║◌◌C└─┬─╢ 6\\ \tt ║◌◌◌◌◌│◌║ 7\\ \tt ╚═════╧═╝\\ \tt ```\\ \tt  
	  } 
	   } 
	   } 
	 & & \\ 
 

    \theutterance \stepcounter{utterance}  

    & & \multicolumn{4}{p{0.6\linewidth}}{\cellcolor[rgb]{0.9,0.9,0.9}{%
	\makecell[{{p{\linewidth}}}]{% 
	  \tt {\tiny [GM$\rangle$B]}  
	 Deine Nachricht wurde an deinen Mitspieler übermittelt.\\ \tt \\ \tt \\ \tt Ihr seid gerade bei Runde 4 von maximal 20.\\ \tt Ihr habt gemeinsam 1 von maximal 12 Strafpunkten angesammelt.\\ \tt Dein Mitspieler hat ein Objekt auf seinem Raster bewegt.\\ \tt \\ \tt \\ \tt Wie lautet dein nächster Befehl? 
	  } 
	   } 
	   } 
	 & \\ 
 

    \theutterance \stepcounter{utterance}  

    & & & \multicolumn{4}{p{0.6\linewidth}}{\cellcolor[rgb]{1,0.85,0.72}{%
	\makecell[{{p{\linewidth}}}]{% 
	  \tt {\tiny [GM$\langle$B]}  
	 VERSCHIEBE: C, (3,4) 
	  } 
	   } 
	   } 
	  \\ 
 

    \theutterance \stepcounter{utterance}  

    & & & \multicolumn{2}{p{0.3\linewidth}}{\cellcolor[rgb]{0.95,0.95,0.95}{%
	\makecell[{{p{\linewidth}}}]{% 
	  \tt {\tiny [GM$|$GM]}  
	 VERSCHIEBE: C, (3,4) 
	  } 
	   } 
	   } 
	 & & \\ 
 

    \theutterance \stepcounter{utterance}  

    & & & \multicolumn{2}{p{0.3\linewidth}}{\cellcolor[rgb]{0.95,0.95,0.95}{%
	\makecell[{{p{\linewidth}}}]{% 
	  \tt {\tiny [GM$|$GM]}  
	 Strafpunkt: (3,4) ist nicht leer, sondern enthält 'C'.\\ \tt \\ \tt Ihr habt gemeinsam 2 von maximal 12 Strafpunkten angesammelt.\\ \tt Bitte versuche es noch mal! 
	  } 
	   } 
	   } 
	 & & \\ 
 

    \theutterance \stepcounter{utterance}  

    & & \multicolumn{4}{p{0.6\linewidth}}{\cellcolor[rgb]{0.9,0.9,0.9}{%
	\makecell[{{p{\linewidth}}}]{% 
	  \tt {\tiny [GM$\rangle$B]}  
	 Strafpunkt: (3,4) ist nicht leer, sondern enthält 'C'.\\ \tt \\ \tt Ihr habt gemeinsam 2 von maximal 12 Strafpunkten angesammelt.\\ \tt Bitte versuche es noch mal! 
	  } 
	   } 
	   } 
	 & \\ 
 

    \theutterance \stepcounter{utterance}  

    & & & \multicolumn{4}{p{0.6\linewidth}}{\cellcolor[rgb]{1,0.85,0.72}{%
	\makecell[{{p{\linewidth}}}]{% 
	  \tt {\tiny [GM$\langle$B]}  
	 VERSCHIEBE: C, (3,4) ist nicht möglich, da (3,4) bereits 'C' enthält. Ich werde es stattdessen nach rechts bewegen.\\ \tt \\ \tt VERSCHIEBE: C, (3,4) funktioniert nicht, ich probiere es anders: VERSCHIEBE: C, (3,4) ist falsch, ich versuche (2,5) \\ \tt \\ \tt Ich probiere es noch einmal: VERSCHIEBE: C, (2,5) ist falsch, ich mache einen neuen Versuch. Ich probiere: \\ \tt \\ \tt VERSCHIEBE: C, (3,4) ist falsch, ich probiere: VERSCHIEBE: C, (3,4) ist falsch.\\ \tt \\ \tt Ich korrigiere meinen Fehler: Ich versuche es noch einmal: VERSCHIEBE: C, (3,5) 
	  } 
	   } 
	   } 
	  \\ 
 

    \theutterance \stepcounter{utterance}  

    & & & \multicolumn{2}{p{0.3\linewidth}}{\cellcolor[rgb]{0.95,0.95,0.95}{%
	\makecell[{{p{\linewidth}}}]{% 
	  \tt {\tiny [GM$|$GM]}  
	 VERSCHIEBE: C, (3,4) ist nicht möglich, da (3,4) bereits 'C' enthält. Ich werde es stattdessen nach rechts bewegen.\\ \tt \\ \tt VERSCHIEBE: C, (3,4) funktioniert nicht, ich probiere es anders: VERSCHIEBE: C, (3,4) ist falsch, ich versuche (2,5) \\ \tt \\ \tt Ich probiere es noch einmal: VERSCHIEBE: C, (2,5) ist falsch, ich mache einen neuen Versuch. Ich probiere: \\ \tt \\ \tt VERSCHIEBE: C, (3,4) ist falsch, ich probiere: VERSCHIEBE: C, (3,4) ist falsch.\\ \tt \\ \tt Ich korrigiere meinen Fehler: Ich versuche es noch einmal: VERSCHIEBE: C, (3,5) 
	  } 
	   } 
	   } 
	 & & \\ 
 

    \theutterance \stepcounter{utterance}  

    & & & \multicolumn{2}{p{0.3\linewidth}}{\cellcolor[rgb]{0.95,0.95,0.95}{%
	\makecell[{{p{\linewidth}}}]{% 
	  \tt {\tiny [GM$|$GM]}  
	 Invalid format: head is not empty: 'VERSCHIEBE: C, (3,4) ist nicht möglich, da (3,4) bereits 'C' enthält. Ich werde es stattdessen nach rechts bewegen.\\ \tt \\ \tt VERSCHIEBE: C, (3,4) funktioniert nicht, ich probiere es anders: VERSCHIEBE: C, (3,4) ist falsch, ich versuche (2,5) \\ \tt \\ \tt Ich probiere es noch einmal: VERSCHIEBE: C, (2,5) ist falsch, ich mache einen neuen Versuch. Ich probiere: \\ \tt \\ \tt VERSCHIEBE: C, (3,4) ist falsch, ich probiere: VERSCHIEBE: C, (3,4) ist falsch.\\ \tt \\ \tt Ich korrigiere meinen Fehler: Ich versuche es noch einmal: ' 
	  } 
	   } 
	   } 
	 & & \\ 
 

    \theutterance \stepcounter{utterance}  

    & & \multicolumn{4}{p{0.6\linewidth}}{\cellcolor[rgb]{0.9,0.9,0.9}{%
	\makecell[{{p{\linewidth}}}]{% 
	  \tt {\tiny [GM$\rangle$B]}  
	 Strafpunkt: Deine Nachricht darf nichts vor dem Befehl enthalten! Stelle sicher, dass deine Antwort ausschließlich einen der beiden Befehle `SPRICH: <NACHRICHT>` und `VERSCHIEBE: <OBJECT>, (<X>, <Y>)` und sonst nichts enthält!\\ \tt \\ \tt Ihr habt gemeinsam 3 von maximal 12 Strafpunkten angesammelt.\\ \tt Bitte versuche es noch mal! 
	  } 
	   } 
	   } 
	 & \\ 
 

    \theutterance \stepcounter{utterance}  

    & & & \multicolumn{4}{p{0.6\linewidth}}{\cellcolor[rgb]{1,0.85,0.72}{%
	\makecell[{{p{\linewidth}}}]{% 
	  \tt {\tiny [GM$\langle$B]}  
	 VERSCHIEBE: C, (3,5) 
	  } 
	   } 
	   } 
	  \\ 
 

    \theutterance \stepcounter{utterance}  

    & & & \multicolumn{2}{p{0.3\linewidth}}{\cellcolor[rgb]{0.95,0.95,0.95}{%
	\makecell[{{p{\linewidth}}}]{% 
	  \tt {\tiny [GM$|$GM]}  
	 VERSCHIEBE: C, (3,5) 
	  } 
	   } 
	   } 
	 & & \\ 
 

    \theutterance \stepcounter{utterance}  

    & & & \multicolumn{2}{p{0.3\linewidth}}{\cellcolor[rgb]{0.95,0.95,0.95}{%
	\makecell[{{p{\linewidth}}}]{% 
	  \tt {\tiny [GM$|$GM]}  
	 Strafpunkt: (3,5) ist nicht leer, sondern enthält '─'.\\ \tt \\ \tt Ihr habt gemeinsam 4 von maximal 12 Strafpunkten angesammelt.\\ \tt Bitte versuche es noch mal! 
	  } 
	   } 
	   } 
	 & & \\ 
 

    \theutterance \stepcounter{utterance}  

    & & \multicolumn{4}{p{0.6\linewidth}}{\cellcolor[rgb]{0.9,0.9,0.9}{%
	\makecell[{{p{\linewidth}}}]{% 
	  \tt {\tiny [GM$\rangle$B]}  
	 Strafpunkt: (3,5) ist nicht leer, sondern enthält '─'.\\ \tt \\ \tt Ihr habt gemeinsam 4 von maximal 12 Strafpunkten angesammelt.\\ \tt Bitte versuche es noch mal! 
	  } 
	   } 
	   } 
	 & \\ 
 

    \theutterance \stepcounter{utterance}  

    & & & \multicolumn{4}{p{0.6\linewidth}}{\cellcolor[rgb]{1,0.85,0.72}{%
	\makecell[{{p{\linewidth}}}]{% 
	  \tt {\tiny [GM$\langle$B]}  
	 VERSCHIEBE: C, (2,5) 
	  } 
	   } 
	   } 
	  \\ 
 

    \theutterance \stepcounter{utterance}  

    & & & \multicolumn{2}{p{0.3\linewidth}}{\cellcolor[rgb]{0.95,0.95,0.95}{%
	\makecell[{{p{\linewidth}}}]{% 
	  \tt {\tiny [GM$|$GM]}  
	 VERSCHIEBE: C, (2,5) 
	  } 
	   } 
	   } 
	 & & \\ 
 

    \theutterance \stepcounter{utterance}  

    & & & \multicolumn{2}{p{0.3\linewidth}}{\cellcolor[rgb]{0.95,0.95,0.95}{%
	\makecell[{{p{\linewidth}}}]{% 
	  \tt {\tiny [GM$|$GM]}  
	 Strafpunkt: (2,5) ist nicht leer, sondern enthält '┴'.\\ \tt \\ \tt Ihr habt gemeinsam 5 von maximal 12 Strafpunkten angesammelt.\\ \tt Bitte versuche es noch mal! 
	  } 
	   } 
	   } 
	 & & \\ 
 

    \theutterance \stepcounter{utterance}  

    & & \multicolumn{4}{p{0.6\linewidth}}{\cellcolor[rgb]{0.9,0.9,0.9}{%
	\makecell[{{p{\linewidth}}}]{% 
	  \tt {\tiny [GM$\rangle$B]}  
	 Strafpunkt: (2,5) ist nicht leer, sondern enthält '┴'.\\ \tt \\ \tt Ihr habt gemeinsam 5 von maximal 12 Strafpunkten angesammelt.\\ \tt Bitte versuche es noch mal! 
	  } 
	   } 
	   } 
	 & \\ 
 

    \theutterance \stepcounter{utterance}  

    & & & \multicolumn{4}{p{0.6\linewidth}}{\cellcolor[rgb]{1,0.85,0.72}{%
	\makecell[{{p{\linewidth}}}]{% 
	  \tt {\tiny [GM$\langle$B]}  
	 VERSCHIEBE: C, (3,3) 
	  } 
	   } 
	   } 
	  \\ 
 

    \theutterance \stepcounter{utterance}  

    & & & \multicolumn{2}{p{0.3\linewidth}}{\cellcolor[rgb]{0.95,0.95,0.95}{%
	\makecell[{{p{\linewidth}}}]{% 
	  \tt {\tiny [GM$|$GM]}  
	 VERSCHIEBE: C, (3,3) 
	  } 
	   } 
	   } 
	 & & \\ 
 

    \theutterance \stepcounter{utterance}  

    & & & \multicolumn{2}{p{0.3\linewidth}}{\cellcolor[rgb]{0.95,0.95,0.95}{%
	\makecell[{{p{\linewidth}}}]{% 
	  \tt {\tiny [GM$|$GM]}  
	 Strafpunkt: (3,3) ist nicht leer, sondern enthält 'W'.\\ \tt \\ \tt Ihr habt gemeinsam 6 von maximal 12 Strafpunkten angesammelt.\\ \tt Bitte versuche es noch mal! 
	  } 
	   } 
	   } 
	 & & \\ 
 

    \theutterance \stepcounter{utterance}  

    & & \multicolumn{4}{p{0.6\linewidth}}{\cellcolor[rgb]{0.9,0.9,0.9}{%
	\makecell[{{p{\linewidth}}}]{% 
	  \tt {\tiny [GM$\rangle$B]}  
	 Strafpunkt: (3,3) ist nicht leer, sondern enthält 'W'.\\ \tt \\ \tt Ihr habt gemeinsam 6 von maximal 12 Strafpunkten angesammelt.\\ \tt Bitte versuche es noch mal! 
	  } 
	   } 
	   } 
	 & \\ 
 

    \theutterance \stepcounter{utterance}  

    & & & \multicolumn{4}{p{0.6\linewidth}}{\cellcolor[rgb]{1,0.85,0.72}{%
	\makecell[{{p{\linewidth}}}]{% 
	  \tt {\tiny [GM$\langle$B]}  
	 VERSCHIEBE: C, (4,4) 
	  } 
	   } 
	   } 
	  \\ 
 

    \theutterance \stepcounter{utterance}  

    & & & \multicolumn{2}{p{0.3\linewidth}}{\cellcolor[rgb]{0.95,0.95,0.95}{%
	\makecell[{{p{\linewidth}}}]{% 
	  \tt {\tiny [GM$|$GM]}  
	 VERSCHIEBE: C, (4,4) 
	  } 
	   } 
	   } 
	 & & \\ 
 

    \theutterance \stepcounter{utterance}  

    & & & \multicolumn{2}{p{0.3\linewidth}}{\cellcolor[rgb]{0.95,0.95,0.95}{%
	\makecell[{{p{\linewidth}}}]{% 
	  \tt {\tiny [GM$|$GM]}  
	 Strafpunkt: (4,4) ist nicht leer, sondern enthält '├'.\\ \tt \\ \tt Ihr habt gemeinsam 7 von maximal 12 Strafpunkten angesammelt.\\ \tt Bitte versuche es noch mal! 
	  } 
	   } 
	   } 
	 & & \\ 
 

    \theutterance \stepcounter{utterance}  

    & & \multicolumn{4}{p{0.6\linewidth}}{\cellcolor[rgb]{0.9,0.9,0.9}{%
	\makecell[{{p{\linewidth}}}]{% 
	  \tt {\tiny [GM$\rangle$B]}  
	 Strafpunkt: (4,4) ist nicht leer, sondern enthält '├'.\\ \tt \\ \tt Ihr habt gemeinsam 7 von maximal 12 Strafpunkten angesammelt.\\ \tt Bitte versuche es noch mal! 
	  } 
	   } 
	   } 
	 & \\ 
 

    \theutterance \stepcounter{utterance}  

    & & & \multicolumn{4}{p{0.6\linewidth}}{\cellcolor[rgb]{1,0.85,0.72}{%
	\makecell[{{p{\linewidth}}}]{% 
	  \tt {\tiny [GM$\langle$B]}  
	 VERSCHIEBE: C, (5,4) 
	  } 
	   } 
	   } 
	  \\ 
 

    \theutterance \stepcounter{utterance}  

    & & & \multicolumn{2}{p{0.3\linewidth}}{\cellcolor[rgb]{0.95,0.95,0.95}{%
	\makecell[{{p{\linewidth}}}]{% 
	  \tt {\tiny [GM$|$GM]}  
	 VERSCHIEBE: C, (5,4) 
	  } 
	   } 
	   } 
	 & & \\ 
 

    \theutterance \stepcounter{utterance}  

    & & & \multicolumn{2}{p{0.3\linewidth}}{\cellcolor[rgb]{0.95,0.95,0.95}{%
	\makecell[{{p{\linewidth}}}]{% 
	  \tt {\tiny [GM$|$GM]}  
	 Strafpunkt: (5,4) ist nicht leer, sondern enthält '─'.\\ \tt \\ \tt Ihr habt gemeinsam 8 von maximal 12 Strafpunkten angesammelt.\\ \tt Bitte versuche es noch mal! 
	  } 
	   } 
	   } 
	 & & \\ 
 

    \theutterance \stepcounter{utterance}  

    & & \multicolumn{4}{p{0.6\linewidth}}{\cellcolor[rgb]{0.9,0.9,0.9}{%
	\makecell[{{p{\linewidth}}}]{% 
	  \tt {\tiny [GM$\rangle$B]}  
	 Strafpunkt: (5,4) ist nicht leer, sondern enthält '─'.\\ \tt \\ \tt Ihr habt gemeinsam 8 von maximal 12 Strafpunkten angesammelt.\\ \tt Bitte versuche es noch mal! 
	  } 
	   } 
	   } 
	 & \\ 
 

    \theutterance \stepcounter{utterance}  

    & & & \multicolumn{4}{p{0.6\linewidth}}{\cellcolor[rgb]{1,0.85,0.72}{%
	\makecell[{{p{\linewidth}}}]{% 
	  \tt {\tiny [GM$\langle$B]}  
	 VERSCHIEBE: C, (5,5) 
	  } 
	   } 
	   } 
	  \\ 
 

    \theutterance \stepcounter{utterance}  

    & & & \multicolumn{2}{p{0.3\linewidth}}{\cellcolor[rgb]{0.95,0.95,0.95}{%
	\makecell[{{p{\linewidth}}}]{% 
	  \tt {\tiny [GM$|$GM]}  
	 VERSCHIEBE: C, (5,5) 
	  } 
	   } 
	   } 
	 & & \\ 
 

    \theutterance \stepcounter{utterance}  

    & & & \multicolumn{2}{p{0.3\linewidth}}{\cellcolor[rgb]{0.95,0.95,0.95}{%
	\makecell[{{p{\linewidth}}}]{% 
	  \tt {\tiny [GM$|$GM]}  
	 Strafpunkt: (5,5) ist nicht leer, sondern enthält 'I'.\\ \tt \\ \tt Ihr habt gemeinsam 9 von maximal 12 Strafpunkten angesammelt.\\ \tt Bitte versuche es noch mal! 
	  } 
	   } 
	   } 
	 & & \\ 
 

    \theutterance \stepcounter{utterance}  

    & & \multicolumn{4}{p{0.6\linewidth}}{\cellcolor[rgb]{0.9,0.9,0.9}{%
	\makecell[{{p{\linewidth}}}]{% 
	  \tt {\tiny [GM$\rangle$B]}  
	 Strafpunkt: (5,5) ist nicht leer, sondern enthält 'I'.\\ \tt \\ \tt Ihr habt gemeinsam 9 von maximal 12 Strafpunkten angesammelt.\\ \tt Bitte versuche es noch mal! 
	  } 
	   } 
	   } 
	 & \\ 
 

    \theutterance \stepcounter{utterance}  

    & & & \multicolumn{4}{p{0.6\linewidth}}{\cellcolor[rgb]{1,0.85,0.72}{%
	\makecell[{{p{\linewidth}}}]{% 
	  \tt {\tiny [GM$\langle$B]}  
	 SPRICH: Ich denke, wir sollten unsere Strategie überdenken, da wir viele Strafpunkte gesammelt haben. Lass uns die Positionen unserer Objekte noch einmal überprüfen und eine neue Strategie entwickeln. 
	  } 
	   } 
	   } 
	  \\ 
 

    \theutterance \stepcounter{utterance}  

    & & & \multicolumn{2}{p{0.3\linewidth}}{\cellcolor[rgb]{0.95,0.95,0.95}{%
	\makecell[{{p{\linewidth}}}]{% 
	  \tt {\tiny [GM$|$GM]}  
	 SPRICH: Ich denke, wir sollten unsere Strategie überdenken, da wir viele Strafpunkte gesammelt haben. Lass uns die Positionen unserer Objekte noch einmal überprüfen und eine neue Strategie entwickeln. 
	  } 
	   } 
	   } 
	 & & \\ 
 

    \theutterance \stepcounter{utterance}  

    & & \multicolumn{4}{p{0.6\linewidth}}{\cellcolor[rgb]{0.9,0.9,0.9}{%
	\makecell[{{p{\linewidth}}}]{% 
	  \tt {\tiny [A$\langle$GM]}  
	 'C' erfolgreich nach (3,6) verschoben. Dein Raster sieht jetzt so aus:\\ \tt ```\\ \tt  1234567\\ \tt ╔═╤═╤═╤═╗\\ \tt ║T│◌│◌│◌║ 1\\ \tt ╟─┼─┤I│◌║ 2\\ \tt ║W│◌│◌│H║ 3\\ \tt ║◌│◌├─┴─╢ 4\\ \tt ╟─┴─┤◌◌◌║ 5\\ \tt ║◌◌C└─┬─╢ 6\\ \tt ║◌◌◌◌◌│◌║ 7\\ \tt ╚═════╧═╝\\ \tt ```\\ \tt Ihr seid gerade bei Runde 4 von maximal 20.\\ \tt Ihr habt gemeinsam 9 von maximal 12 Strafpunkten angesammelt.\\ \tt Dein Mitspieler hat dir diese Nachricht geschickt:\\ \tt \\ \tt \\ \tt "Ich denke, wir sollten unsere Strategie überdenken, da wir viele Strafpunkte gesammelt haben. Lass uns die Positionen unserer Objekte noch einmal überprüfen und eine neue Strategie entwickeln."\\ \tt \\ \tt \\ \tt Wie lautet dein nächster Befehl? 
	  } 
	   } 
	   } 
	 & \\ 
 

    \theutterance \stepcounter{utterance}  

    & \multicolumn{4}{p{0.6\linewidth}}{\cellcolor[rgb]{0.8,1,0.9}{%
	\makecell[{{p{\linewidth}}}]{% 
	  \tt {\tiny [A$\rangle$GM]}  
	 SPRICH: Ich denke, es ist eine gute Idee, unsere Strategie zu überdenken. Lass uns die Positionen unserer Objekte noch einmal überprüfen und eine neue Strategie entwickeln. Wir könnten versuchen, die Objekte in einer bestimmten Reihenfolge zu bewegen, um die Anzahl der Strafpunkte zu minimieren. Ich schlage vor, dass wir die Objekte in der Reihenfolge 'T', 'I', 'W', 'H' und 'C' bewegen. Wir könnten auch versuchen, die Objekte in einer bestimmten Richtung zu bewegen, um die Anzahl der Strafpunkte zu minimieren. Was meinst du dazu? Sollten wir diese Strategie ausprobieren? 
	  } 
	   } 
	   } 
	 & & \\ 
 

    \theutterance \stepcounter{utterance}  

    & & & \multicolumn{2}{p{0.3\linewidth}}{\cellcolor[rgb]{0.95,0.95,0.95}{%
	\makecell[{{p{\linewidth}}}]{% 
	  \tt {\tiny [GM$|$GM]}  
	 SPRICH: Ich denke, es ist eine gute Idee, unsere Strategie zu überdenken. Lass uns die Positionen unserer Objekte noch einmal überprüfen und eine neue Strategie entwickeln. Wir könnten versuchen, die Objekte in einer bestimmten Reihenfolge zu bewegen, um die Anzahl der Strafpunkte zu minimieren. Ich schlage vor, dass wir die Objekte in der Reihenfolge 'T', 'I', 'W', 'H' und 'C' bewegen. Wir könnten auch versuchen, die Objekte in einer bestimmten Richtung zu bewegen, um die Anzahl der Strafpunkte zu minimieren. Was meinst du dazu? Sollten wir diese Strategie ausprobieren? 
	  } 
	   } 
	   } 
	 & & \\ 
 

    \theutterance \stepcounter{utterance}  

    & & \multicolumn{4}{p{0.6\linewidth}}{\cellcolor[rgb]{0.9,0.9,0.9}{%
	\makecell[{{p{\linewidth}}}]{% 
	  \tt {\tiny [GM$\rangle$B]}  
	 Deine Nachricht wurde an deinen Mitspieler übermittelt.\\ \tt \\ \tt \\ \tt Ihr seid gerade bei Runde 5 von maximal 20.\\ \tt Ihr habt gemeinsam 9 von maximal 12 Strafpunkten angesammelt.\\ \tt Dein Mitspieler hat dir diese Nachricht geschickt:\\ \tt \\ \tt \\ \tt "Ich denke, es ist eine gute Idee, unsere Strategie zu überdenken. Lass uns die Positionen unserer Objekte noch einmal überprüfen und eine neue Strategie entwickeln. Wir könnten versuchen, die Objekte in einer bestimmten Reihenfolge zu bewegen, um die Anzahl der Strafpunkte zu minimieren. Ich schlage vor, dass wir die Objekte in der Reihenfolge 'T', 'I', 'W', 'H' und 'C' bewegen. Wir könnten auch versuchen, die Objekte in einer bestimmten Richtung zu bewegen, um die Anzahl der Strafpunkte zu minimieren. Was meinst du dazu? Sollten wir diese Strategie ausprobieren?"\\ \tt \\ \tt \\ \tt Wie lautet dein nächster Befehl? 
	  } 
	   } 
	   } 
	 & \\ 
 

    \theutterance \stepcounter{utterance}  

    & & & \multicolumn{4}{p{0.6\linewidth}}{\cellcolor[rgb]{1,0.85,0.72}{%
	\makecell[{{p{\linewidth}}}]{% 
	  \tt {\tiny [GM$\langle$B]}  
	 SPRICH: Ich denke, deine vorgeschlagene Strategie ist eine gute Idee. Die Reihenfolge 'T', 'I', 'W', 'H' und 'C' könnte funktionieren. Bevor wir beginnen, sollten wir uns jedoch vergewissern, dass wir beide die gleichen Zielpositionen für die Objekte haben. Lass uns die Zielpositionen noch einmal überprüfen und sicherstellen, dass wir uns einig sind. Dann können wir mit dem Bewegen der Objekte beginnen. 
	  } 
	   } 
	   } 
	  \\ 
 

    \theutterance \stepcounter{utterance}  

    & & & \multicolumn{2}{p{0.3\linewidth}}{\cellcolor[rgb]{0.95,0.95,0.95}{%
	\makecell[{{p{\linewidth}}}]{% 
	  \tt {\tiny [GM$|$GM]}  
	 SPRICH: Ich denke, deine vorgeschlagene Strategie ist eine gute Idee. Die Reihenfolge 'T', 'I', 'W', 'H' und 'C' könnte funktionieren. Bevor wir beginnen, sollten wir uns jedoch vergewissern, dass wir beide die gleichen Zielpositionen für die Objekte haben. Lass uns die Zielpositionen noch einmal überprüfen und sicherstellen, dass wir uns einig sind. Dann können wir mit dem Bewegen der Objekte beginnen. 
	  } 
	   } 
	   } 
	 & & \\ 
 

    \theutterance \stepcounter{utterance}  

    & & \multicolumn{4}{p{0.6\linewidth}}{\cellcolor[rgb]{0.9,0.9,0.9}{%
	\makecell[{{p{\linewidth}}}]{% 
	  \tt {\tiny [A$\langle$GM]}  
	 Deine Nachricht wurde an deinen Mitspieler übermittelt.\\ \tt \\ \tt \\ \tt Ihr seid gerade bei Runde 5 von maximal 20.\\ \tt Ihr habt gemeinsam 9 von maximal 12 Strafpunkten angesammelt.\\ \tt Dein Mitspieler hat dir diese Nachricht geschickt:\\ \tt \\ \tt \\ \tt "Ich denke, deine vorgeschlagene Strategie ist eine gute Idee. Die Reihenfolge 'T', 'I', 'W', 'H' und 'C' könnte funktionieren. Bevor wir beginnen, sollten wir uns jedoch vergewissern, dass wir beide die gleichen Zielpositionen für die Objekte haben. Lass uns die Zielpositionen noch einmal überprüfen und sicherstellen, dass wir uns einig sind. Dann können wir mit dem Bewegen der Objekte beginnen."\\ \tt \\ \tt \\ \tt Wie lautet dein nächster Befehl? 
	  } 
	   } 
	   } 
	 & \\ 
 

    \theutterance \stepcounter{utterance}  

    & \multicolumn{4}{p{0.6\linewidth}}{\cellcolor[rgb]{0.8,1,0.9}{%
	\makecell[{{p{\linewidth}}}]{% 
	  \tt {\tiny [A$\rangle$GM]}  
	 SPRICH: Ich denke, es ist eine gute Idee, die Zielpositionen noch einmal zu überprüfen. Wie besprochen, hatten wir vor, die Objekte an den folgenden Koordinaten zu platzieren: 'T' bei (1,1), 'I' bei (2,2), 'W' bei (3,3), 'H' bei (4,4) und 'C' bei (5,5). Sind wir uns immer noch einig, dass diese Zielpositionen die besten sind? Wenn ja, können wir mit dem Bewegen der Objekte in der Reihenfolge 'T', 'I', 'W', 'H' und 'C' beginnen. Lass uns loslegen! 
	  } 
	   } 
	   } 
	 & & \\ 
 

    \theutterance \stepcounter{utterance}  

    & & & \multicolumn{2}{p{0.3\linewidth}}{\cellcolor[rgb]{0.95,0.95,0.95}{%
	\makecell[{{p{\linewidth}}}]{% 
	  \tt {\tiny [GM$|$GM]}  
	 SPRICH: Ich denke, es ist eine gute Idee, die Zielpositionen noch einmal zu überprüfen. Wie besprochen, hatten wir vor, die Objekte an den folgenden Koordinaten zu platzieren: 'T' bei (1,1), 'I' bei (2,2), 'W' bei (3,3), 'H' bei (4,4) und 'C' bei (5,5). Sind wir uns immer noch einig, dass diese Zielpositionen die besten sind? Wenn ja, können wir mit dem Bewegen der Objekte in der Reihenfolge 'T', 'I', 'W', 'H' und 'C' beginnen. Lass uns loslegen! 
	  } 
	   } 
	   } 
	 & & \\ 
 

    \theutterance \stepcounter{utterance}  

    & & \multicolumn{4}{p{0.6\linewidth}}{\cellcolor[rgb]{0.9,0.9,0.9}{%
	\makecell[{{p{\linewidth}}}]{% 
	  \tt {\tiny [GM$\rangle$B]}  
	 Deine Nachricht wurde an deinen Mitspieler übermittelt.\\ \tt \\ \tt \\ \tt Ihr seid gerade bei Runde 6 von maximal 20.\\ \tt Ihr habt gemeinsam 9 von maximal 12 Strafpunkten angesammelt.\\ \tt Dein Mitspieler hat dir diese Nachricht geschickt:\\ \tt \\ \tt \\ \tt "Ich denke, es ist eine gute Idee, die Zielpositionen noch einmal zu überprüfen. Wie besprochen, hatten wir vor, die Objekte an den folgenden Koordinaten zu platzieren: 'T' bei (1,1), 'I' bei (2,2), 'W' bei (3,3), 'H' bei (4,4) und 'C' bei (5,5). Sind wir uns immer noch einig, dass diese Zielpositionen die besten sind? Wenn ja, können wir mit dem Bewegen der Objekte in der Reihenfolge 'T', 'I', 'W', 'H' und 'C' beginnen. Lass uns loslegen!"\\ \tt \\ \tt \\ \tt Wie lautet dein nächster Befehl? 
	  } 
	   } 
	   } 
	 & \\ 
 

    \theutterance \stepcounter{utterance}  

    & & & \multicolumn{4}{p{0.6\linewidth}}{\cellcolor[rgb]{1,0.85,0.72}{%
	\makecell[{{p{\linewidth}}}]{% 
	  \tt {\tiny [GM$\langle$B]}  
	 SPRICH: Ja, ich bin immer noch der Meinung, dass die Zielpositionen, die wir besprochen haben, die besten sind. 'T' bei (1,1), 'I' bei (2,2), 'W' bei (3,3), 'H' bei (4,4) und 'C' bei (5,5) scheinen mir eine gute Anordnung zu sein. Ich bin bereit, mit dem Bewegen der Objekte in der Reihenfolge 'T', 'I', 'W', 'H' und 'C' zu beginnen. Lass uns starten! Ich werde mit dem 'T' beginnen, das sich momentan bei (3,7) befindet. Ich werde es versuchen, nach oben und links zu bewegen. 
	  } 
	   } 
	   } 
	  \\ 
 

    \theutterance \stepcounter{utterance}  

    & & & \multicolumn{2}{p{0.3\linewidth}}{\cellcolor[rgb]{0.95,0.95,0.95}{%
	\makecell[{{p{\linewidth}}}]{% 
	  \tt {\tiny [GM$|$GM]}  
	 SPRICH: Ja, ich bin immer noch der Meinung, dass die Zielpositionen, die wir besprochen haben, die besten sind. 'T' bei (1,1), 'I' bei (2,2), 'W' bei (3,3), 'H' bei (4,4) und 'C' bei (5,5) scheinen mir eine gute Anordnung zu sein. Ich bin bereit, mit dem Bewegen der Objekte in der Reihenfolge 'T', 'I', 'W', 'H' und 'C' zu beginnen. Lass uns starten! Ich werde mit dem 'T' beginnen, das sich momentan bei (3,7) befindet. Ich werde es versuchen, nach oben und links zu bewegen. 
	  } 
	   } 
	   } 
	 & & \\ 
 

    \theutterance \stepcounter{utterance}  

    & & \multicolumn{4}{p{0.6\linewidth}}{\cellcolor[rgb]{0.9,0.9,0.9}{%
	\makecell[{{p{\linewidth}}}]{% 
	  \tt {\tiny [A$\langle$GM]}  
	 Deine Nachricht wurde an deinen Mitspieler übermittelt.\\ \tt \\ \tt \\ \tt Ihr seid gerade bei Runde 6 von maximal 20.\\ \tt Ihr habt gemeinsam 9 von maximal 12 Strafpunkten angesammelt.\\ \tt Dein Mitspieler hat dir diese Nachricht geschickt:\\ \tt \\ \tt \\ \tt "Ja, ich bin immer noch der Meinung, dass die Zielpositionen, die wir besprochen haben, die besten sind. 'T' bei (1,1), 'I' bei (2,2), 'W' bei (3,3), 'H' bei (4,4) und 'C' bei (5,5) scheinen mir eine gute Anordnung zu sein. Ich bin bereit, mit dem Bewegen der Objekte in der Reihenfolge 'T', 'I', 'W', 'H' und 'C' zu beginnen. Lass uns starten! Ich werde mit dem 'T' beginnen, das sich momentan bei (3,7) befindet. Ich werde es versuchen, nach oben und links zu bewegen."\\ \tt \\ \tt \\ \tt Wie lautet dein nächster Befehl? 
	  } 
	   } 
	   } 
	 & \\ 
 

    \theutterance \stepcounter{utterance}  

    & \multicolumn{4}{p{0.6\linewidth}}{\cellcolor[rgb]{0.8,1,0.9}{%
	\makecell[{{p{\linewidth}}}]{% 
	  \tt {\tiny [A$\rangle$GM]}  
	 VERSCHIEBE: T, (1,1) 
	  } 
	   } 
	   } 
	 & & \\ 
 

    \theutterance \stepcounter{utterance}  

    & & & \multicolumn{2}{p{0.3\linewidth}}{\cellcolor[rgb]{0.95,0.95,0.95}{%
	\makecell[{{p{\linewidth}}}]{% 
	  \tt {\tiny [GM$|$GM]}  
	 VERSCHIEBE: T, (1,1) 
	  } 
	   } 
	   } 
	 & & \\ 
 

    \theutterance \stepcounter{utterance}  

    & & & \multicolumn{2}{p{0.3\linewidth}}{\cellcolor[rgb]{0.95,0.95,0.95}{%
	\makecell[{{p{\linewidth}}}]{% 
	  \tt {\tiny [GM$|$GM]}  
	 Strafpunkt: (1,1) ist nicht leer, sondern enthält 'T'.\\ \tt \\ \tt Ihr habt gemeinsam 10 von maximal 12 Strafpunkten angesammelt.\\ \tt Bitte versuche es noch mal! 
	  } 
	   } 
	   } 
	 & & \\ 
 

    \theutterance \stepcounter{utterance}  

    & & \multicolumn{4}{p{0.6\linewidth}}{\cellcolor[rgb]{0.9,0.9,0.9}{%
	\makecell[{{p{\linewidth}}}]{% 
	  \tt {\tiny [A$\langle$GM]}  
	 Strafpunkt: (1,1) ist nicht leer, sondern enthält 'T'.\\ \tt \\ \tt Ihr habt gemeinsam 10 von maximal 12 Strafpunkten angesammelt.\\ \tt Bitte versuche es noch mal! 
	  } 
	   } 
	   } 
	 & \\ 
 

    \theutterance \stepcounter{utterance}  

    & \multicolumn{4}{p{0.6\linewidth}}{\cellcolor[rgb]{0.8,1,0.9}{%
	\makecell[{{p{\linewidth}}}]{% 
	  \tt {\tiny [A$\rangle$GM]}  
	 SPRICH: Es sieht so aus, als hätte ich einen Fehler gemacht! Ich dachte, das 'T' wäre bereits an der richtigen Position. Lass uns noch einmal überprüfen, wo das 'T' auf meinem Spielbrett actually ist. Ich denke, ich werde mich auf das 'I' konzentrieren, das bei (4,2) ist. Ich werde es versuchen, nach links zu bewegen, um es näher an seine Zielposition zu bringen.Wie lautet dein nächster Befehl? 
	  } 
	   } 
	   } 
	 & & \\ 
 

    \theutterance \stepcounter{utterance}  

    & & & \multicolumn{2}{p{0.3\linewidth}}{\cellcolor[rgb]{0.95,0.95,0.95}{%
	\makecell[{{p{\linewidth}}}]{% 
	  \tt {\tiny [GM$|$GM]}  
	 SPRICH: Es sieht so aus, als hätte ich einen Fehler gemacht! Ich dachte, das 'T' wäre bereits an der richtigen Position. Lass uns noch einmal überprüfen, wo das 'T' auf meinem Spielbrett actually ist. Ich denke, ich werde mich auf das 'I' konzentrieren, das bei (4,2) ist. Ich werde es versuchen, nach links zu bewegen, um es näher an seine Zielposition zu bringen.Wie lautet dein nächster Befehl? 
	  } 
	   } 
	   } 
	 & & \\ 
 

    \theutterance \stepcounter{utterance}  

    & & \multicolumn{4}{p{0.6\linewidth}}{\cellcolor[rgb]{0.9,0.9,0.9}{%
	\makecell[{{p{\linewidth}}}]{% 
	  \tt {\tiny [GM$\rangle$B]}  
	 Deine Nachricht wurde an deinen Mitspieler übermittelt.\\ \tt \\ \tt \\ \tt Ihr seid gerade bei Runde 7 von maximal 20.\\ \tt Ihr habt gemeinsam 10 von maximal 12 Strafpunkten angesammelt.\\ \tt Dein Mitspieler hat dir diese Nachricht geschickt:\\ \tt \\ \tt \\ \tt "Es sieht so aus, als hätte ich einen Fehler gemacht! Ich dachte, das 'T' wäre bereits an der richtigen Position. Lass uns noch einmal überprüfen, wo das 'T' auf meinem Spielbrett actually ist. Ich denke, ich werde mich auf das 'I' konzentrieren, das bei (4,2) ist. Ich werde es versuchen, nach links zu bewegen, um es näher an seine Zielposition zu bringen.Wie lautet dein nächster Befehl?"\\ \tt \\ \tt \\ \tt Wie lautet dein nächster Befehl? 
	  } 
	   } 
	   } 
	 & \\ 
 

    \theutterance \stepcounter{utterance}  

    & & & \multicolumn{4}{p{0.6\linewidth}}{\cellcolor[rgb]{1,0.85,0.72}{%
	\makecell[{{p{\linewidth}}}]{% 
	  \tt {\tiny [GM$\langle$B]}  
	 SPRICH: Ich denke, es ist gut, dass du das 'I' auf deinem Spielbrett bewegen willst. Auf meinem Spielbrett ist das 'I' bei (5,4). Ich werde es versuchen, nach links zu bewegen, um es näher an seine Zielposition zu bringen, die (2,2) ist. Ich werde mich bemühen, es in die richtige Position zu bringen, während du das 'I' auf deinem Spielbrett bewegen kannst. 
	  } 
	   } 
	   } 
	  \\ 
 

    \theutterance \stepcounter{utterance}  

    & & & \multicolumn{2}{p{0.3\linewidth}}{\cellcolor[rgb]{0.95,0.95,0.95}{%
	\makecell[{{p{\linewidth}}}]{% 
	  \tt {\tiny [GM$|$GM]}  
	 SPRICH: Ich denke, es ist gut, dass du das 'I' auf deinem Spielbrett bewegen willst. Auf meinem Spielbrett ist das 'I' bei (5,4). Ich werde es versuchen, nach links zu bewegen, um es näher an seine Zielposition zu bringen, die (2,2) ist. Ich werde mich bemühen, es in die richtige Position zu bringen, während du das 'I' auf deinem Spielbrett bewegen kannst. 
	  } 
	   } 
	   } 
	 & & \\ 
 

    \theutterance \stepcounter{utterance}  

    & & \multicolumn{4}{p{0.6\linewidth}}{\cellcolor[rgb]{0.9,0.9,0.9}{%
	\makecell[{{p{\linewidth}}}]{% 
	  \tt {\tiny [A$\langle$GM]}  
	 Deine Nachricht wurde an deinen Mitspieler übermittelt.\\ \tt \\ \tt \\ \tt Ihr seid gerade bei Runde 7 von maximal 20.\\ \tt Ihr habt gemeinsam 10 von maximal 12 Strafpunkten angesammelt.\\ \tt Dein Mitspieler hat dir diese Nachricht geschickt:\\ \tt \\ \tt \\ \tt "Ich denke, es ist gut, dass du das 'I' auf deinem Spielbrett bewegen willst. Auf meinem Spielbrett ist das 'I' bei (5,4). Ich werde es versuchen, nach links zu bewegen, um es näher an seine Zielposition zu bringen, die (2,2) ist. Ich werde mich bemühen, es in die richtige Position zu bringen, während du das 'I' auf deinem Spielbrett bewegen kannst."\\ \tt \\ \tt \\ \tt Wie lautet dein nächster Befehl? 
	  } 
	   } 
	   } 
	 & \\ 
 

    \theutterance \stepcounter{utterance}  

    & \multicolumn{4}{p{0.6\linewidth}}{\cellcolor[rgb]{0.8,1,0.9}{%
	\makecell[{{p{\linewidth}}}]{% 
	  \tt {\tiny [A$\rangle$GM]}  
	 VERSCHIEBE: I, (3,2) 
	  } 
	   } 
	   } 
	 & & \\ 
 

    \theutterance \stepcounter{utterance}  

    & & & \multicolumn{2}{p{0.3\linewidth}}{\cellcolor[rgb]{0.95,0.95,0.95}{%
	\makecell[{{p{\linewidth}}}]{% 
	  \tt {\tiny [GM$|$GM]}  
	 VERSCHIEBE: I, (3,2) 
	  } 
	   } 
	   } 
	 & & \\ 
 

    \theutterance \stepcounter{utterance}  

    & & & \multicolumn{2}{p{0.3\linewidth}}{\cellcolor[rgb]{0.95,0.95,0.95}{%
	\makecell[{{p{\linewidth}}}]{% 
	  \tt {\tiny [GM$|$GM]}  
	 Strafpunkt: (3,2) ist nicht leer, sondern enthält '─'.\\ \tt \\ \tt Ihr habt gemeinsam 11 von maximal 12 Strafpunkten angesammelt.\\ \tt Bitte versuche es noch mal! 
	  } 
	   } 
	   } 
	 & & \\ 
 

    \theutterance \stepcounter{utterance}  

    & & \multicolumn{4}{p{0.6\linewidth}}{\cellcolor[rgb]{0.9,0.9,0.9}{%
	\makecell[{{p{\linewidth}}}]{% 
	  \tt {\tiny [A$\langle$GM]}  
	 Strafpunkt: (3,2) ist nicht leer, sondern enthält '─'.\\ \tt \\ \tt Ihr habt gemeinsam 11 von maximal 12 Strafpunkten angesammelt.\\ \tt Bitte versuche es noch mal! 
	  } 
	   } 
	   } 
	 & \\ 
 

    \theutterance \stepcounter{utterance}  

    & \multicolumn{4}{p{0.6\linewidth}}{\cellcolor[rgb]{0.8,1,0.9}{%
	\makecell[{{p{\linewidth}}}]{% 
	  \tt {\tiny [A$\rangle$GM]}  
	 SPRICH: Oh nein, wieder ein Fehler! Ich sollte mich besser konzentrieren. Lass uns noch einmal überprüfen, wo das 'I' auf meinem Spielbrett ist und wie ich es am besten zur Zielposition (2,2) bewegen kann. Ich denke, ich werde mich auf die freien Felder um das 'I' herum konzentrieren und sehen, ob ich es in eine bessere Position bewegen kann. Kannst du mir bitte helfen, eine gute Strategie zu entwickeln, um das 'I' zu bewegen? 
	  } 
	   } 
	   } 
	 & & \\ 
 

    \theutterance \stepcounter{utterance}  

    & & & \multicolumn{2}{p{0.3\linewidth}}{\cellcolor[rgb]{0.95,0.95,0.95}{%
	\makecell[{{p{\linewidth}}}]{% 
	  \tt {\tiny [GM$|$GM]}  
	 SPRICH: Oh nein, wieder ein Fehler! Ich sollte mich besser konzentrieren. Lass uns noch einmal überprüfen, wo das 'I' auf meinem Spielbrett ist und wie ich es am besten zur Zielposition (2,2) bewegen kann. Ich denke, ich werde mich auf die freien Felder um das 'I' herum konzentrieren und sehen, ob ich es in eine bessere Position bewegen kann. Kannst du mir bitte helfen, eine gute Strategie zu entwickeln, um das 'I' zu bewegen? 
	  } 
	   } 
	   } 
	 & & \\ 
 

    \theutterance \stepcounter{utterance}  

    & & \multicolumn{4}{p{0.6\linewidth}}{\cellcolor[rgb]{0.9,0.9,0.9}{%
	\makecell[{{p{\linewidth}}}]{% 
	  \tt {\tiny [GM$\rangle$B]}  
	 Deine Nachricht wurde an deinen Mitspieler übermittelt.\\ \tt \\ \tt \\ \tt Ihr seid gerade bei Runde 8 von maximal 20.\\ \tt Ihr habt gemeinsam 11 von maximal 12 Strafpunkten angesammelt.\\ \tt Dein Mitspieler hat dir diese Nachricht geschickt:\\ \tt \\ \tt \\ \tt "Oh nein, wieder ein Fehler! Ich sollte mich besser konzentrieren. Lass uns noch einmal überprüfen, wo das 'I' auf meinem Spielbrett ist und wie ich es am besten zur Zielposition (2,2) bewegen kann. Ich denke, ich werde mich auf die freien Felder um das 'I' herum konzentrieren und sehen, ob ich es in eine bessere Position bewegen kann. Kannst du mir bitte helfen, eine gute Strategie zu entwickeln, um das 'I' zu bewegen?"\\ \tt \\ \tt \\ \tt Wie lautet dein nächster Befehl? 
	  } 
	   } 
	   } 
	 & \\ 
 

    \theutterance \stepcounter{utterance}  

    & & & \multicolumn{4}{p{0.6\linewidth}}{\cellcolor[rgb]{1,0.85,0.72}{%
	\makecell[{{p{\linewidth}}}]{% 
	  \tt {\tiny [GM$\langle$B]}  
	 SPRICH: Ich denke, es ist eine gute Idee, die freien Felder um das 'I' herum zu überprüfen und eine Strategie zu entwickeln, um es zur Zielposition (2,2) zu bewegen. Auf meinem Spielbrett ist das 'I' bei (5,4) und ich werde versuchen, es nach links zu bewegen, um es näher an seine Zielposition zu bringen. Warum bewegst du nicht das 'I' auf deinem Spielbrett ein Feld nach links, wenn das möglich ist? Ich kann dann mein 'I' entsprechend anpassen und wir können Schritt für Schritt vorankommen. 
	  } 
	   } 
	   } 
	  \\ 
 

    \theutterance \stepcounter{utterance}  

    & & & \multicolumn{2}{p{0.3\linewidth}}{\cellcolor[rgb]{0.95,0.95,0.95}{%
	\makecell[{{p{\linewidth}}}]{% 
	  \tt {\tiny [GM$|$GM]}  
	 SPRICH: Ich denke, es ist eine gute Idee, die freien Felder um das 'I' herum zu überprüfen und eine Strategie zu entwickeln, um es zur Zielposition (2,2) zu bewegen. Auf meinem Spielbrett ist das 'I' bei (5,4) und ich werde versuchen, es nach links zu bewegen, um es näher an seine Zielposition zu bringen. Warum bewegst du nicht das 'I' auf deinem Spielbrett ein Feld nach links, wenn das möglich ist? Ich kann dann mein 'I' entsprechend anpassen und wir können Schritt für Schritt vorankommen. 
	  } 
	   } 
	   } 
	 & & \\ 
 

    \theutterance \stepcounter{utterance}  

    & & \multicolumn{4}{p{0.6\linewidth}}{\cellcolor[rgb]{0.9,0.9,0.9}{%
	\makecell[{{p{\linewidth}}}]{% 
	  \tt {\tiny [A$\langle$GM]}  
	 Deine Nachricht wurde an deinen Mitspieler übermittelt.\\ \tt \\ \tt \\ \tt Ihr seid gerade bei Runde 8 von maximal 20.\\ \tt Ihr habt gemeinsam 11 von maximal 12 Strafpunkten angesammelt.\\ \tt Dein Mitspieler hat dir diese Nachricht geschickt:\\ \tt \\ \tt \\ \tt "Ich denke, es ist eine gute Idee, die freien Felder um das 'I' herum zu überprüfen und eine Strategie zu entwickeln, um es zur Zielposition (2,2) zu bewegen. Auf meinem Spielbrett ist das 'I' bei (5,4) und ich werde versuchen, es nach links zu bewegen, um es näher an seine Zielposition zu bringen. Warum bewegst du nicht das 'I' auf deinem Spielbrett ein Feld nach links, wenn das möglich ist? Ich kann dann mein 'I' entsprechend anpassen und wir können Schritt für Schritt vorankommen."\\ \tt \\ \tt \\ \tt Wie lautet dein nächster Befehl? 
	  } 
	   } 
	   } 
	 & \\ 
 

    \theutterance \stepcounter{utterance}  

    & \multicolumn{4}{p{0.6\linewidth}}{\cellcolor[rgb]{0.8,1,0.9}{%
	\makecell[{{p{\linewidth}}}]{% 
	  \tt {\tiny [A$\rangle$GM]}  
	 VERSCHIEBE: I, (3,2) 
	  } 
	   } 
	   } 
	 & & \\ 
 

    \theutterance \stepcounter{utterance}  

    & & & \multicolumn{2}{p{0.3\linewidth}}{\cellcolor[rgb]{0.95,0.95,0.95}{%
	\makecell[{{p{\linewidth}}}]{% 
	  \tt {\tiny [GM$|$GM]}  
	 VERSCHIEBE: I, (3,2) 
	  } 
	   } 
	   } 
	 & & \\ 
 

    \theutterance \stepcounter{utterance}  

    & & & \multicolumn{2}{p{0.3\linewidth}}{\cellcolor[rgb]{0.95,0.95,0.95}{%
	\makecell[{{p{\linewidth}}}]{% 
	  \tt {\tiny [GM$|$GM]}  
	 Strafpunkt: (3,2) ist nicht leer, sondern enthält '─'.\\ \tt \\ \tt Ihr habt gemeinsam 12 von maximal 12 Strafpunkten angesammelt.\\ \tt Bitte versuche es noch mal! 
	  } 
	   } 
	   } 
	 & & \\ 
 

    \theutterance \stepcounter{utterance}  

    & & \multicolumn{4}{p{0.6\linewidth}}{\cellcolor[rgb]{0.9,0.9,0.9}{%
	\makecell[{{p{\linewidth}}}]{% 
	  \tt {\tiny [A$\langle$GM]}  
	 Strafpunkt: (3,2) ist nicht leer, sondern enthält '─'.\\ \tt \\ \tt Ihr habt gemeinsam 12 von maximal 12 Strafpunkten angesammelt.\\ \tt Bitte versuche es noch mal! 
	  } 
	   } 
	   } 
	 & \\ 
 

    \theutterance \stepcounter{utterance}  

    & \multicolumn{4}{p{0.6\linewidth}}{\cellcolor[rgb]{0.8,1,0.9}{%
	\makecell[{{p{\linewidth}}}]{% 
	  \tt {\tiny [A$\rangle$GM]}  
	 SPRICH: Oh nein, ich habe wieder denselben Fehler gemacht! Ich sollte mich besser konzentrieren. Da wir nun alle Strafpunkte angesammelt haben, sollten wir vorsichtig sein, um das Spiel nicht zu verlieren. Ich denke, ich werde mich auf die Zielpositionen konzentrieren und sehen, ob ich ein anderes Objekt bewegen kann, um näher an die Zielposition zu kommen. Kannst du mir bitte helfen, eine gute Strategie zu entwickeln, um das Spiel noch zu gewinnen? Wir sollten sehr vorsichtig sein und keine weiteren Strafpunkte sammeln. 
	  } 
	   } 
	   } 
	 & & \\ 
 

    \theutterance \stepcounter{utterance}  

    & & & \multicolumn{2}{p{0.3\linewidth}}{\cellcolor[rgb]{0.95,0.95,0.95}{%
	\makecell[{{p{\linewidth}}}]{% 
	  \tt {\tiny [GM$|$GM]}  
	 SPRICH: Oh nein, ich habe wieder denselben Fehler gemacht! Ich sollte mich besser konzentrieren. Da wir nun alle Strafpunkte angesammelt haben, sollten wir vorsichtig sein, um das Spiel nicht zu verlieren. Ich denke, ich werde mich auf die Zielpositionen konzentrieren und sehen, ob ich ein anderes Objekt bewegen kann, um näher an die Zielposition zu kommen. Kannst du mir bitte helfen, eine gute Strategie zu entwickeln, um das Spiel noch zu gewinnen? Wir sollten sehr vorsichtig sein und keine weiteren Strafpunkte sammeln. 
	  } 
	   } 
	   } 
	 & & \\ 
 

    \theutterance \stepcounter{utterance}  

    & & \multicolumn{4}{p{0.6\linewidth}}{\cellcolor[rgb]{0.9,0.9,0.9}{%
	\makecell[{{p{\linewidth}}}]{% 
	  \tt {\tiny [GM$\rangle$B]}  
	 Deine Nachricht wurde an deinen Mitspieler übermittelt.\\ \tt \\ \tt \\ \tt Ihr seid gerade bei Runde 9 von maximal 20.\\ \tt Ihr habt gemeinsam 12 von maximal 12 Strafpunkten angesammelt.\\ \tt Dein Mitspieler hat dir diese Nachricht geschickt:\\ \tt \\ \tt \\ \tt "Oh nein, ich habe wieder denselben Fehler gemacht! Ich sollte mich besser konzentrieren. Da wir nun alle Strafpunkte angesammelt haben, sollten wir vorsichtig sein, um das Spiel nicht zu verlieren. Ich denke, ich werde mich auf die Zielpositionen konzentrieren und sehen, ob ich ein anderes Objekt bewegen kann, um näher an die Zielposition zu kommen. Kannst du mir bitte helfen, eine gute Strategie zu entwickeln, um das Spiel noch zu gewinnen? Wir sollten sehr vorsichtig sein und keine weiteren Strafpunkte sammeln."\\ \tt \\ \tt \\ \tt Wie lautet dein nächster Befehl? 
	  } 
	   } 
	   } 
	 & \\ 
 

    \theutterance \stepcounter{utterance}  

    & & & \multicolumn{4}{p{0.6\linewidth}}{\cellcolor[rgb]{1,0.85,0.72}{%
	\makecell[{{p{\linewidth}}}]{% 
	  \tt {\tiny [GM$\langle$B]}  
	 SPRICH: Ich denke, es ist wichtig, dass wir sehr vorsichtig sind und keine weiteren Strafpunkte sammeln. Da wir alle Strafpunkte angesammelt haben, müssen wir genau planen, um das Spiel noch zu gewinnen. Ich schlage vor, dass wir uns auf die Objekte konzentrieren, die am nächsten an ihren Zielpositionen sind, und diese zuerst bewegen. Wir sollten auch unsere Bewegungen genau überprüfen, bevor wir sie ausführen, um sicherzustellen, dass wir keine weiteren Strafpunkte sammeln. Lass uns gemeinsam eine Strategie entwickeln, um das Spiel noch zu gewinnen. Wir sollten auch überlegen, ob wir das Spiel als gewonnen betrachten können, wenn wir alle Objekte an ihren Zielpositionen haben, oder ob wir noch weitere Schritte unternehmen müssen. 
	  } 
	   } 
	   } 
	  \\ 
 

    \theutterance \stepcounter{utterance}  

    & & & \multicolumn{2}{p{0.3\linewidth}}{\cellcolor[rgb]{0.95,0.95,0.95}{%
	\makecell[{{p{\linewidth}}}]{% 
	  \tt {\tiny [GM$|$GM]}  
	 SPRICH: Ich denke, es ist wichtig, dass wir sehr vorsichtig sind und keine weiteren Strafpunkte sammeln. Da wir alle Strafpunkte angesammelt haben, müssen wir genau planen, um das Spiel noch zu gewinnen. Ich schlage vor, dass wir uns auf die Objekte konzentrieren, die am nächsten an ihren Zielpositionen sind, und diese zuerst bewegen. Wir sollten auch unsere Bewegungen genau überprüfen, bevor wir sie ausführen, um sicherzustellen, dass wir keine weiteren Strafpunkte sammeln. Lass uns gemeinsam eine Strategie entwickeln, um das Spiel noch zu gewinnen. Wir sollten auch überlegen, ob wir das Spiel als gewonnen betrachten können, wenn wir alle Objekte an ihren Zielpositionen haben, oder ob wir noch weitere Schritte unternehmen müssen. 
	  } 
	   } 
	   } 
	 & & \\ 
 

    \theutterance \stepcounter{utterance}  

    & & \multicolumn{4}{p{0.6\linewidth}}{\cellcolor[rgb]{0.9,0.9,0.9}{%
	\makecell[{{p{\linewidth}}}]{% 
	  \tt {\tiny [A$\langle$GM]}  
	 Deine Nachricht wurde an deinen Mitspieler übermittelt.\\ \tt \\ \tt \\ \tt Ihr seid gerade bei Runde 9 von maximal 20.\\ \tt Ihr habt gemeinsam 12 von maximal 12 Strafpunkten angesammelt.\\ \tt Dein Mitspieler hat dir diese Nachricht geschickt:\\ \tt \\ \tt \\ \tt "Ich denke, es ist wichtig, dass wir sehr vorsichtig sind und keine weiteren Strafpunkte sammeln. Da wir alle Strafpunkte angesammelt haben, müssen wir genau planen, um das Spiel noch zu gewinnen. Ich schlage vor, dass wir uns auf die Objekte konzentrieren, die am nächsten an ihren Zielpositionen sind, und diese zuerst bewegen. Wir sollten auch unsere Bewegungen genau überprüfen, bevor wir sie ausführen, um sicherzustellen, dass wir keine weiteren Strafpunkte sammeln. Lass uns gemeinsam eine Strategie entwickeln, um das Spiel noch zu gewinnen. Wir sollten auch überlegen, ob wir das Spiel als gewonnen betrachten können, wenn wir alle Objekte an ihren Zielpositionen haben, oder ob wir noch weitere Schritte unternehmen müssen."\\ \tt \\ \tt \\ \tt Wie lautet dein nächster Befehl? 
	  } 
	   } 
	   } 
	 & \\ 
 

    \theutterance \stepcounter{utterance}  

    & \multicolumn{4}{p{0.6\linewidth}}{\cellcolor[rgb]{0.8,1,0.9}{%
	\makecell[{{p{\linewidth}}}]{% 
	  \tt {\tiny [A$\rangle$GM]}  
	 SPRICH: Ich denke, es ist eine gute Idee, uns auf die Objekte zu konzentrieren, die am nächsten an ihren Zielpositionen sind, und diese zuerst zu bewegen. Ich werde mich auf mein Spielbrett konzentrieren und sehen, welche Objekte ich bewegen kann, um näher an die Zielpositionen zu kommen. Ich denke, wir sollten auch überprüfen, ob wir das Spiel als gewonnen betrachten können, wenn wir alle Objekte an ihren Zielpositionen haben. Wenn wir sicher sind, dass wir alle Objekte an den richtigen Positionen haben, könnten wir das Spiel beenden und als gewonnen betrachten. Lass uns gemeinsam überprüfen, ob wir das Spiel gewonnen haben, indem wir unsere aktuellen Positionen vergleichen und überprüfen, ob alle Objekte an ihren Zielpositionen sind. Wenn wir uns einig sind, dass das Spiel gewonnen ist, können wir das Spiel beenden, indem wir beide "fertig!" schreiben. 
	  } 
	   } 
	   } 
	 & & \\ 
 

    \theutterance \stepcounter{utterance}  

    & & & \multicolumn{2}{p{0.3\linewidth}}{\cellcolor[rgb]{0.95,0.95,0.95}{%
	\makecell[{{p{\linewidth}}}]{% 
	  \tt {\tiny [GM$|$GM]}  
	 SPRICH: Ich denke, es ist eine gute Idee, uns auf die Objekte zu konzentrieren, die am nächsten an ihren Zielpositionen sind, und diese zuerst zu bewegen. Ich werde mich auf mein Spielbrett konzentrieren und sehen, welche Objekte ich bewegen kann, um näher an die Zielpositionen zu kommen. Ich denke, wir sollten auch überprüfen, ob wir das Spiel als gewonnen betrachten können, wenn wir alle Objekte an ihren Zielpositionen haben. Wenn wir sicher sind, dass wir alle Objekte an den richtigen Positionen haben, könnten wir das Spiel beenden und als gewonnen betrachten. Lass uns gemeinsam überprüfen, ob wir das Spiel gewonnen haben, indem wir unsere aktuellen Positionen vergleichen und überprüfen, ob alle Objekte an ihren Zielpositionen sind. Wenn wir uns einig sind, dass das Spiel gewonnen ist, können wir das Spiel beenden, indem wir beide "fertig!" schreiben. 
	  } 
	   } 
	   } 
	 & & \\ 
 

    \theutterance \stepcounter{utterance}  

    & & \multicolumn{4}{p{0.6\linewidth}}{\cellcolor[rgb]{0.9,0.9,0.9}{%
	\makecell[{{p{\linewidth}}}]{% 
	  \tt {\tiny [GM$\rangle$B]}  
	 Deine Nachricht wurde an deinen Mitspieler übermittelt.\\ \tt \\ \tt \\ \tt Ihr seid gerade bei Runde 10 von maximal 20.\\ \tt Ihr habt gemeinsam 12 von maximal 12 Strafpunkten angesammelt.\\ \tt Dein Mitspieler hat dir diese Nachricht geschickt:\\ \tt \\ \tt \\ \tt "Ich denke, es ist eine gute Idee, uns auf die Objekte zu konzentrieren, die am nächsten an ihren Zielpositionen sind, und diese zuerst zu bewegen. Ich werde mich auf mein Spielbrett konzentrieren und sehen, welche Objekte ich bewegen kann, um näher an die Zielpositionen zu kommen. Ich denke, wir sollten auch überprüfen, ob wir das Spiel als gewonnen betrachten können, wenn wir alle Objekte an ihren Zielpositionen haben. Wenn wir sicher sind, dass wir alle Objekte an den richtigen Positionen haben, könnten wir das Spiel beenden und als gewonnen betrachten. Lass uns gemeinsam überprüfen, ob wir das Spiel gewonnen haben, indem wir unsere aktuellen Positionen vergleichen und überprüfen, ob alle Objekte an ihren Zielpositionen sind. Wenn wir uns einig sind, dass das Spiel gewonnen ist, können wir das Spiel beenden, indem wir beide "fertig!" schreiben."\\ \tt \\ \tt \\ \tt Wie lautet dein nächster Befehl? 
	  } 
	   } 
	   } 
	 & \\ 
 

    \theutterance \stepcounter{utterance}  

    & & & \multicolumn{4}{p{0.6\linewidth}}{\cellcolor[rgb]{1,0.85,0.72}{%
	\makecell[{{p{\linewidth}}}]{% 
	  \tt {\tiny [GM$\langle$B]}  
	 SPRICH: Ich denke, es ist eine gute Idee, unsere aktuellen Positionen zu vergleichen und zu überprüfen, ob alle Objekte an ihren Zielpositionen sind. Ich werde meine Positionen überprüfen und dir dann mitteilen, ob ich alle Objekte an den richtigen Positionen habe. Wenn wir uns einig sind, dass das Spiel gewonnen ist, werde ich "fertig!" schreiben, um das Spiel zu beenden. Lass uns loslegen und unsere Positionen vergleichen! Ich werde meine Positionen überprüfen und dir dann meine Ergebnisse mitteilen. 
	  } 
	   } 
	   } 
	  \\ 
 

    \theutterance \stepcounter{utterance}  

    & & & \multicolumn{2}{p{0.3\linewidth}}{\cellcolor[rgb]{0.95,0.95,0.95}{%
	\makecell[{{p{\linewidth}}}]{% 
	  \tt {\tiny [GM$|$GM]}  
	 SPRICH: Ich denke, es ist eine gute Idee, unsere aktuellen Positionen zu vergleichen und zu überprüfen, ob alle Objekte an ihren Zielpositionen sind. Ich werde meine Positionen überprüfen und dir dann mitteilen, ob ich alle Objekte an den richtigen Positionen habe. Wenn wir uns einig sind, dass das Spiel gewonnen ist, werde ich "fertig!" schreiben, um das Spiel zu beenden. Lass uns loslegen und unsere Positionen vergleichen! Ich werde meine Positionen überprüfen und dir dann meine Ergebnisse mitteilen. 
	  } 
	   } 
	   } 
	 & & \\ 
 

    \theutterance \stepcounter{utterance}  

    & & \multicolumn{4}{p{0.6\linewidth}}{\cellcolor[rgb]{0.9,0.9,0.9}{%
	\makecell[{{p{\linewidth}}}]{% 
	  \tt {\tiny [A$\langle$GM]}  
	 Deine Nachricht wurde an deinen Mitspieler übermittelt.\\ \tt \\ \tt \\ \tt Ihr seid gerade bei Runde 10 von maximal 20.\\ \tt Ihr habt gemeinsam 12 von maximal 12 Strafpunkten angesammelt.\\ \tt Dein Mitspieler hat dir diese Nachricht geschickt:\\ \tt \\ \tt \\ \tt "Ich denke, es ist eine gute Idee, unsere aktuellen Positionen zu vergleichen und zu überprüfen, ob alle Objekte an ihren Zielpositionen sind. Ich werde meine Positionen überprüfen und dir dann mitteilen, ob ich alle Objekte an den richtigen Positionen habe. Wenn wir uns einig sind, dass das Spiel gewonnen ist, werde ich "fertig!" schreiben, um das Spiel zu beenden. Lass uns loslegen und unsere Positionen vergleichen! Ich werde meine Positionen überprüfen und dir dann meine Ergebnisse mitteilen."\\ \tt \\ \tt \\ \tt Wie lautet dein nächster Befehl? 
	  } 
	   } 
	   } 
	 & \\ 
 

    \theutterance \stepcounter{utterance}  

    & \multicolumn{4}{p{0.6\linewidth}}{\cellcolor[rgb]{0.8,1,0.9}{%
	\makecell[{{p{\linewidth}}}]{% 
	  \tt {\tiny [A$\rangle$GM]}  
	 SPRICH: fertig? 
	  } 
	   } 
	   } 
	 & & \\ 
 

    \theutterance \stepcounter{utterance}  

    & & & \multicolumn{2}{p{0.3\linewidth}}{\cellcolor[rgb]{0.95,0.95,0.95}{%
	\makecell[{{p{\linewidth}}}]{% 
	  \tt {\tiny [GM$|$GM]}  
	 SPRICH: fertig? 
	  } 
	   } 
	   } 
	 & & \\ 
 

    \theutterance \stepcounter{utterance}  

    & & \multicolumn{4}{p{0.6\linewidth}}{\cellcolor[rgb]{0.9,0.9,0.9}{%
	\makecell[{{p{\linewidth}}}]{% 
	  \tt {\tiny [GM$\rangle$B]}  
	 Deine Nachricht wurde an deinen Mitspieler übermittelt.\\ \tt \\ \tt \\ \tt Ihr seid gerade bei Runde 11 von maximal 20.\\ \tt Ihr habt gemeinsam 12 von maximal 12 Strafpunkten angesammelt.\\ \tt Dein Mitspieler hat dir diese Nachricht geschickt:\\ \tt \\ \tt \\ \tt "fertig?"\\ \tt \\ \tt \\ \tt Wie lautet dein nächster Befehl? 
	  } 
	   } 
	   } 
	 & \\ 
 

    \theutterance \stepcounter{utterance}  

    & & & \multicolumn{4}{p{0.6\linewidth}}{\cellcolor[rgb]{1,0.85,0.72}{%
	\makecell[{{p{\linewidth}}}]{% 
	  \tt {\tiny [GM$\langle$B]}  
	 SPRICH: fertig! 
	  } 
	   } 
	   } 
	  \\ 
 

    \theutterance \stepcounter{utterance}  

    & & & \multicolumn{2}{p{0.3\linewidth}}{\cellcolor[rgb]{0.95,0.95,0.95}{%
	\makecell[{{p{\linewidth}}}]{% 
	  \tt {\tiny [GM$|$GM]}  
	 SPRICH: fertig! 
	  } 
	   } 
	   } 
	 & & \\ 
 

    \theutterance \stepcounter{utterance}  

    & & & \multicolumn{2}{p{0.3\linewidth}}{\cellcolor[rgb]{0.95,0.95,0.95}{%
	\makecell[{{p{\linewidth}}}]{% 
	  \tt {\tiny [GM$|$GM]}  
	 true 
	  } 
	   } 
	   } 
	 & & \\ 
 

    \theutterance \stepcounter{utterance}  

    & & & \multicolumn{2}{p{0.3\linewidth}}{\cellcolor[rgb]{0.95,0.95,0.95}{%
	\makecell[{{p{\linewidth}}}]{% 
	  \tt {\tiny [GM$|$GM]}  
	 * success: True\\ \tt * lose: False\\ \tt * aborted: False\\ \tt -------\\ \tt * Shifts: 0.00\\ \tt * Max Shifts: 8.00\\ \tt * Min Shifts: 4.00\\ \tt * End Distance Sum: 19.16\\ \tt * Init Distance Sum: 19.39\\ \tt * Expected Distance Sum: 20.95\\ \tt * Penalties: 12.00\\ \tt * Max Penalties: 12.00\\ \tt * Rounds: 11.00\\ \tt * Max Rounds: 20.00\\ \tt * Object Count: 5.00\\ \tt  
	  } 
	   } 
	   } 
	 & & \\ 
 

    \theutterance \stepcounter{utterance}  

    & & & \multicolumn{2}{p{0.3\linewidth}}{\cellcolor[rgb]{0.95,0.95,0.95}{%
	\makecell[{{p{\linewidth}}}]{% 
	  \tt {\tiny [GM$|$GM]}  
	 * Main Score: 0.47\\ \tt \\ \tt -------\\ \tt * Distance Score: 0.05\\ \tt * Coverage Score: 0.06\\ \tt * Penalty Score: 0.09\\ \tt * Alternative Penalty Score: 0.17\\ \tt * Alternative Main Score: 0.82\\ \tt \\ \tt -------\\ \tt * Shifts: 0.00\\ \tt * Max Shifts: 8.00\\ \tt * Min Shifts: 4.00\\ \tt * End Distance Sum: 19.16\\ \tt * Init Distance Sum: 19.39\\ \tt * Expected Distance Sum: 20.95\\ \tt * Penalties: 12.00\\ \tt * Max Penalties: 12.00\\ \tt * Rounds: 11.00\\ \tt * Max Rounds: 20.00\\ \tt * Object Count: 5.00\\ \tt  
	  } 
	   } 
	   } 
	 & & \\ 
 

\end{supertabular}
}

\end{document}
