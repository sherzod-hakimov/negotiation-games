
\documentclass{article}
\usepackage{colortbl}
\usepackage{makecell}
\usepackage{multirow}
\usepackage{supertabular}

\begin{document}

\newcounter{utterance}

\twocolumn

{ \footnotesize  \setcounter{utterance}{1}
\setlength{\tabcolsep}{0pt}
\begin{supertabular}{c@{$\;$}|p{.15\linewidth}@{}p{.15\linewidth}p{.15\linewidth}p{.15\linewidth}p{.15\linewidth}p{.15\linewidth}}

    \# & $\;$A & \multicolumn{4}{c}{Game Master} & $\;\:$B\\
    \hline 

    \theutterance \stepcounter{utterance}  

    & & \multicolumn{4}{p{0.6\linewidth}}{\cellcolor[rgb]{0.9,0.9,0.9}{%
	\makecell[{{p{\linewidth}}}]{% 
	  \tt {\tiny [A$\langle$GM]}  
	 Ich bin der Spielleiter, und du spielst ein kooperatives Spiel mit dem folgenden Raster als Spielbrett:\\ \tt \\ \tt ```\\ \tt  1234567\\ \tt ╔═══╤═╤═╗\\ \tt ║C◌◌│◌│◌║ 1\\ \tt ║◌I◌│◌├─╢ 2\\ \tt ║◌◌◌├─┘◌║ 3\\ \tt ║◌◌◌│◌◌◌║ 4\\ \tt ╟─┬─┼───╢ 5\\ \tt ║◌│◌│◌WT║ 6\\ \tt ║◌│◌│◌◌H║ 7\\ \tt ╚═╧═╧═══╝\\ \tt \\ \tt ```\\ \tt \\ \tt * Am oberen Rand sind die x-Koordinaten angegeben, die nach rechts hin zunehmen, und am rechten Rand die y-Koordinaten, die nach unten zunehmen.\\ \tt * Die folgenden Objekte sind zufällig auf deinem Raster platziert: 'W', 'I', 'T', 'C', 'H'.\\ \tt \\ \tt \\ \tt Dein Mitspieler sieht eine Variation des Spielbretts, auf dem dieselben Objekte an anderen zufälligen Positionen platziert sind. Du kannst das Spielbrett des anderen Spielers nicht sehen, und er kann deins nicht sehen.\\ \tt \\ \tt \\ \tt **Ziel des Spiels:**\\ \tt \\ \tt Beide Spieler müssen die Objekte auf ihrem jeweiligen Spielbrett so bewegen, dass identische Objekte am Ende an denselben Koordinaten liegen. Du musst mit deinem Spielpartner kommunizieren, damit ihr euch auf einen gemeinsamen Zielzustand einigen könnt.\\ \tt \\ \tt \\ \tt **Spielregeln:**\\ \tt \\ \tt * In jedem Spielzug kannst du genau eine der folgenden zwei Aktionen ausführen:\\ \tt 1. `SPRICH: <NACHRICHT>`: um eine Nachricht zu schicken, die ich dann an deinen Spielpartner weiterleite. Als Nachricht gilt alles bis zum nächsten Zeilenumbruch.\\ \tt 2. `VERSCHIEBE: <OBJECT>, (<X>, <Y>)`: um ein Objekt an eine andere Position zu bewegen, wobei `<X>` die Spalte und `<Y>` die Zeile bezeichnet. Ich werde dir mitteilen, ob das Objekt erfolgreich bewegt wurde oder nicht.\\ \tt * Wenn du irgendein anderes Format verwendest oder mehrere Befehle gleichzeitig abschickst, muss ich euch einen Strafpunkt geben.\\ \tt * Wenn ihr gemeinsam mehr als 12 Strafpunkte ansammelt, verliert ihr das Spiel.\\ \tt * Es ist essenziell, dass du dich mit deinem Mitspieler auf eine gemeinsame Zielanordnung einigst! Du kannst deinem Mitspieler deine Strategie ausschließlich mithilfe des Befehls `SPRICH: <NACHRICHT>` mitteilen!\\ \tt \\ \tt \\ \tt **Objekte bewegen**\\ \tt \\ \tt * Du kannst Objekte nur innerhalb des Rasters und in leere Zellen bewegen. Leere Zellen erkennst am Zeichen '◌'.\\ \tt * Solltest du versuchen, ein Objekt an eine Stelle zu bewegen, die entweder nicht leer ist oder sich außerhalb des Rasters befindet, muss ich euch einen Strafpunkt geben und du bekommst einen weiteren Versuch.\\ \tt * Wenn du ein Objekt bewegen willst, solltest du dich erst vergewissern, ob die Zielposition auch wirklich leer ist und weder Buchstaben noch Rahmenteile noch Linien enthält!\\ \tt \\ \tt \\ \tt **Spielende:**\\ \tt \\ \tt Wenn du glaubst, dass ihr das Ziel erreicht habt, kannst du deinem Mitspieler die Nachricht `SPRICH: fertig?` schicken. Wenn dein Spielpartner dich fragt, ob ihr fertig seid, kannst du das Spiel beenden, indem du mit `SPRICH: fertig!` antwortest.\\ \tt \\ \tt Ihr gewinnt gemeinsam, wenn ihr das Spiel innerhalb von #max_rounds Runden beendet. Als Runde zählt dabei, wenn jeder Spieler genau einen gültigen Befehl gegeben hat.\\ \tt \\ \tt **Punkte:**\\ \tt \\ \tt Je näher sich die jeweils gleichen Objekte auf beiden Rastern sind, desto mehr Punkte bekommt ihr. Für Strafpunkte gibt es Punktabzug. Könnt ihr den Rekord brechen?\\ \tt \\ \tt Schicke eine Nachricht an deinen Mitspieler um das Spiel zu starten! 
	  } 
	   } 
	   } 
	 & \\ 
 

    \theutterance \stepcounter{utterance}  

    & \multicolumn{4}{p{0.6\linewidth}}{\cellcolor[rgb]{0.8,1,0.9}{%
	\makecell[{{p{\linewidth}}}]{% 
	  \tt {\tiny [A$\rangle$GM]}  
	 SPRICH: Hallo, ich schlage vor, dass wir uns auf eine feste Position für jedes Objekt einigen, um das Spiel zu vereinfachen. Wie wäre es, wenn wir die Objekte in der Mitte des Rasters platzieren? 'W' bei (4,4), 'I' bei (3,4), 'T' bei (5,4), 'C' bei (4,3) und 'H' bei (4,5)? 
	  } 
	   } 
	   } 
	 & & \\ 
 

    \theutterance \stepcounter{utterance}  

    & & & \multicolumn{2}{p{0.3\linewidth}}{\cellcolor[rgb]{0.95,0.95,0.95}{%
	\makecell[{{p{\linewidth}}}]{% 
	  \tt {\tiny [GM$|$GM]}  
	 SPRICH: Hallo, ich schlage vor, dass wir uns auf eine feste Position für jedes Objekt einigen, um das Spiel zu vereinfachen. Wie wäre es, wenn wir die Objekte in der Mitte des Rasters platzieren? 'W' bei (4,4), 'I' bei (3,4), 'T' bei (5,4), 'C' bei (4,3) und 'H' bei (4,5)? 
	  } 
	   } 
	   } 
	 & & \\ 
 

    \theutterance \stepcounter{utterance}  

    & & \multicolumn{4}{p{0.6\linewidth}}{\cellcolor[rgb]{0.9,0.9,0.9}{%
	\makecell[{{p{\linewidth}}}]{% 
	  \tt {\tiny [GM$\rangle$B]}  
	 Ich bin der Spielleiter, und du spielst ein kooperatives Spiel mit dem folgenden Raster als Spielbrett:\\ \tt \\ \tt ```\\ \tt  1234567\\ \tt ╔═══╤═╤═╗\\ \tt ║◌◌◌│C│◌║ 1\\ \tt ║◌W◌│◌├─╢ 2\\ \tt ║◌◌◌├─┘◌║ 3\\ \tt ║◌◌◌│◌◌H║ 4\\ \tt ╟─┬─┼───╢ 5\\ \tt ║T│◌│I◌◌║ 6\\ \tt ║◌│◌│◌◌◌║ 7\\ \tt ╚═╧═╧═══╝\\ \tt \\ \tt ```\\ \tt \\ \tt * Am oberen Rand sind die x-Koordinaten angegeben, die nach rechts hin zunehmen, und am rechten Rand die y-Koordinaten, die nach unten zunehmen.\\ \tt * Die folgenden Objekte sind zufällig auf deinem Raster platziert: 'W', 'I', 'T', 'C', 'H'.\\ \tt \\ \tt \\ \tt Dein Mitspieler sieht eine Variation des Spielbretts, auf dem dieselben Objekte an anderen zufälligen Positionen platziert sind. Du kannst das Spielbrett des anderen Spielers nicht sehen, und er kann deins nicht sehen.\\ \tt \\ \tt \\ \tt **Ziel des Spiels:**\\ \tt \\ \tt Beide Spieler müssen die Objekte auf ihrem jeweiligen Spielbrett so bewegen, dass identische Objekte am Ende an denselben Koordinaten liegen. Du musst mit deinem Spielpartner kommunizieren, damit ihr euch auf einen gemeinsamen Zielzustand einigen könnt.\\ \tt \\ \tt \\ \tt **Spielregeln:**\\ \tt \\ \tt * In jedem Spielzug kannst du genau eine der folgenden zwei Aktionen ausführen:\\ \tt 1. `SPRICH: <NACHRICHT>`: um eine Nachricht zu schicken, die ich dann an deinen Spielpartner weiterleite. Als Nachricht gilt alles bis zum nächsten Zeilenumbruch.\\ \tt 2. `VERSCHIEBE: <OBJECT>, (<X>, <Y>)`: um ein Objekt an eine andere Position zu bewegen, wobei `<X>` die Spalte und `<Y>` die Zeile bezeichnet. Ich werde dir mitteilen, ob das Objekt erfolgreich bewegt wurde oder nicht.\\ \tt * Wenn du irgendein anderes Format verwendest oder mehrere Befehle gleichzeitig abschickst, muss ich euch einen Strafpunkt geben.\\ \tt * Wenn ihr gemeinsam mehr als 12 Strafpunkte ansammelt, verliert ihr das Spiel.\\ \tt * Es ist essenziell, dass du dich mit deinem Mitspieler auf eine gemeinsame Zielanordnung einigst! Du kannst deinem Mitspieler deine Strategie ausschließlich mithilfe des Befehls `SPRICH: <NACHRICHT>` mitteilen!\\ \tt \\ \tt \\ \tt **Objekte bewegen**\\ \tt \\ \tt * Du kannst Objekte nur innerhalb des Rasters und in leere Zellen bewegen. Leere Zellen erkennst am Zeichen '◌'.\\ \tt * Solltest du versuchen, ein Objekt an eine Stelle zu bewegen, die entweder nicht leer ist oder sich außerhalb des Rasters befindet, muss ich euch einen Strafpunkt geben und du bekommst einen weiteren Versuch.\\ \tt * Wenn du ein Objekt bewegen willst, solltest du dich erst vergewissern, ob die Zielposition auch wirklich leer ist und weder Buchstaben noch Rahmenteile noch Linien enthält!\\ \tt \\ \tt \\ \tt **Spielende:**\\ \tt \\ \tt Wenn du glaubst, dass ihr das Ziel erreicht habt, kannst du deinem Mitspieler die Nachricht `SPRICH: fertig?` schicken. Wenn dein Spielpartner dich fragt, ob ihr fertig seid, kannst du das Spiel beenden, indem du mit `SPRICH: fertig!` antwortest.\\ \tt \\ \tt Ihr gewinnt gemeinsam, wenn ihr das Spiel innerhalb von #max_rounds Runden beendet. Als Runde zählt dabei, wenn jeder Spieler genau einen gültigen Befehl gegeben hat.\\ \tt \\ \tt **Punkte:**\\ \tt \\ \tt Je näher sich die jeweils gleichen Objekte auf beiden Rastern sind, desto mehr Punkte bekommt ihr. Für Strafpunkte gibt es Punktabzug. Könnt ihr den Rekord brechen?\\ \tt \\ \tt Dein Mitspieler hat das Spiel mit dieser Nachricht begonnen:\\ \tt \\ \tt "Hallo, ich schlage vor, dass wir uns auf eine feste Position für jedes Objekt einigen, um das Spiel zu vereinfachen. Wie wäre es, wenn wir die Objekte in der Mitte des Rasters platzieren? 'W' bei (4,4), 'I' bei (3,4), 'T' bei (5,4), 'C' bei (4,3) und 'H' bei (4,5)?"\\ \tt \\ \tt Wie lautet dein erster Befehl? 
	  } 
	   } 
	   } 
	 & \\ 
 

    \theutterance \stepcounter{utterance}  

    & & & \multicolumn{4}{p{0.6\linewidth}}{\cellcolor[rgb]{1,0.85,0.72}{%
	\makecell[{{p{\linewidth}}}]{% 
	  \tt {\tiny [GM$\langle$B]}  
	 `SPRICH: Hallo! Dein Vorschlag klingt gut, aber ich denke, wir sollten uns auf eine andere Anordnung einigen, um sicherzustellen, dass wir beide das gleiche Ziel verfolgen. Wie wäre es, wenn wir die Objekte in der oberen Hälfte des Rasters platzieren? 'W' bei (3,2), 'I' bei (5,2), 'T' bei (4,1), 'C' bei (2,3) und 'H' bei (6,3)? Das würde uns mehr Flexibilität geben, falls wir während des Spiels Anpassungen vornehmen müssen.` 
	  } 
	   } 
	   } 
	  \\ 
 

    \theutterance \stepcounter{utterance}  

    & & & \multicolumn{2}{p{0.3\linewidth}}{\cellcolor[rgb]{0.95,0.95,0.95}{%
	\makecell[{{p{\linewidth}}}]{% 
	  \tt {\tiny [GM$|$GM]}  
	 `SPRICH: Hallo! Dein Vorschlag klingt gut, aber ich denke, wir sollten uns auf eine andere Anordnung einigen, um sicherzustellen, dass wir beide das gleiche Ziel verfolgen. Wie wäre es, wenn wir die Objekte in der oberen Hälfte des Rasters platzieren? 'W' bei (3,2), 'I' bei (5,2), 'T' bei (4,1), 'C' bei (2,3) und 'H' bei (6,3)? Das würde uns mehr Flexibilität geben, falls wir während des Spiels Anpassungen vornehmen müssen.` 
	  } 
	   } 
	   } 
	 & & \\ 
 

    \theutterance \stepcounter{utterance}  

    & & \multicolumn{4}{p{0.6\linewidth}}{\cellcolor[rgb]{0.9,0.9,0.9}{%
	\makecell[{{p{\linewidth}}}]{% 
	  \tt {\tiny [A$\langle$GM]}  
	 Deine Nachricht wurde an deinen Mitspieler übermittelt.\\ \tt \\ \tt \\ \tt Ihr seid gerade bei Runde 1 von maximal 20.\\ \tt Ihr habt gemeinsam 0 von maximal 12 Strafpunkten angesammelt.\\ \tt Dein Mitspieler hat dir diese Nachricht geschickt:\\ \tt \\ \tt \\ \tt "Hallo! Dein Vorschlag klingt gut, aber ich denke, wir sollten uns auf eine andere Anordnung einigen, um sicherzustellen, dass wir beide das gleiche Ziel verfolgen. Wie wäre es, wenn wir die Objekte in der oberen Hälfte des Rasters platzieren? 'W' bei (3,2), 'I' bei (5,2), 'T' bei (4,1), 'C' bei (2,3) und 'H' bei (6,3)? Das würde uns mehr Flexibilität geben, falls wir während des Spiels Anpassungen vornehmen müssen."\\ \tt \\ \tt \\ \tt Wie lautet dein nächster Befehl? 
	  } 
	   } 
	   } 
	 & \\ 
 

    \theutterance \stepcounter{utterance}  

    & \multicolumn{4}{p{0.6\linewidth}}{\cellcolor[rgb]{0.8,1,0.9}{%
	\makecell[{{p{\linewidth}}}]{% 
	  \tt {\tiny [A$\rangle$GM]}  
	 SPRICH: Ich verstehe deinen Vorschlag und denke, dass er auch gut ist. Allerdings sollte ich dir mitteilen, wie meine aktuelle Spielfeldaussage aussieht, bevor wir eine endgültige Entscheidung treffen. Auf meinem Brett ist 'C' bei (1,1), 'I' bei (2,2), 'T' bei (7,6), 'W' bei (7,6) und 'H' bei (7,7). Kannst du mir bitte mitteilen, wie deine Anordnung aussieht, damit wir ein gemeinsames Ziel definieren können? 
	  } 
	   } 
	   } 
	 & & \\ 
 

    \theutterance \stepcounter{utterance}  

    & & & \multicolumn{2}{p{0.3\linewidth}}{\cellcolor[rgb]{0.95,0.95,0.95}{%
	\makecell[{{p{\linewidth}}}]{% 
	  \tt {\tiny [GM$|$GM]}  
	 SPRICH: Ich verstehe deinen Vorschlag und denke, dass er auch gut ist. Allerdings sollte ich dir mitteilen, wie meine aktuelle Spielfeldaussage aussieht, bevor wir eine endgültige Entscheidung treffen. Auf meinem Brett ist 'C' bei (1,1), 'I' bei (2,2), 'T' bei (7,6), 'W' bei (7,6) und 'H' bei (7,7). Kannst du mir bitte mitteilen, wie deine Anordnung aussieht, damit wir ein gemeinsames Ziel definieren können? 
	  } 
	   } 
	   } 
	 & & \\ 
 

    \theutterance \stepcounter{utterance}  

    & & \multicolumn{4}{p{0.6\linewidth}}{\cellcolor[rgb]{0.9,0.9,0.9}{%
	\makecell[{{p{\linewidth}}}]{% 
	  \tt {\tiny [GM$\rangle$B]}  
	 Deine Nachricht wurde an deinen Mitspieler übermittelt.\\ \tt \\ \tt \\ \tt Ihr seid gerade bei Runde 2 von maximal 20.\\ \tt Ihr habt gemeinsam 0 von maximal 12 Strafpunkten angesammelt.\\ \tt Dein Mitspieler hat dir diese Nachricht geschickt:\\ \tt \\ \tt \\ \tt "Ich verstehe deinen Vorschlag und denke, dass er auch gut ist. Allerdings sollte ich dir mitteilen, wie meine aktuelle Spielfeldaussage aussieht, bevor wir eine endgültige Entscheidung treffen. Auf meinem Brett ist 'C' bei (1,1), 'I' bei (2,2), 'T' bei (7,6), 'W' bei (7,6) und 'H' bei (7,7). Kannst du mir bitte mitteilen, wie deine Anordnung aussieht, damit wir ein gemeinsames Ziel definieren können?"\\ \tt \\ \tt \\ \tt Wie lautet dein nächster Befehl? 
	  } 
	   } 
	   } 
	 & \\ 
 

    \theutterance \stepcounter{utterance}  

    & & & \multicolumn{4}{p{0.6\linewidth}}{\cellcolor[rgb]{1,0.85,0.72}{%
	\makecell[{{p{\linewidth}}}]{% 
	  \tt {\tiny [GM$\langle$B]}  
	 `SPRICH: Danke für die Informationen! Auf meinem Brett ist 'C' bei (2,1), 'I' bei (6,6), 'T' bei (1,6), 'W' bei (2,2) und 'H' bei (4,4). Ich denke, wir sollten uns auf eine Anordnung konzentrieren, die für beide Bretter machbar ist. Basierend auf unseren aktuellen Anordnungen, könnte ich mir vorstellen, 'T' bei (4,1) zu platzieren, 'W' bei (3,2), 'I' bei (5,2), 'C' bei (2,3) und 'H' bei (6,3). Was meinst du? Sollten wir es versuchen?` 
	  } 
	   } 
	   } 
	  \\ 
 

    \theutterance \stepcounter{utterance}  

    & & & \multicolumn{2}{p{0.3\linewidth}}{\cellcolor[rgb]{0.95,0.95,0.95}{%
	\makecell[{{p{\linewidth}}}]{% 
	  \tt {\tiny [GM$|$GM]}  
	 `SPRICH: Danke für die Informationen! Auf meinem Brett ist 'C' bei (2,1), 'I' bei (6,6), 'T' bei (1,6), 'W' bei (2,2) und 'H' bei (4,4). Ich denke, wir sollten uns auf eine Anordnung konzentrieren, die für beide Bretter machbar ist. Basierend auf unseren aktuellen Anordnungen, könnte ich mir vorstellen, 'T' bei (4,1) zu platzieren, 'W' bei (3,2), 'I' bei (5,2), 'C' bei (2,3) und 'H' bei (6,3). Was meinst du? Sollten wir es versuchen?` 
	  } 
	   } 
	   } 
	 & & \\ 
 

    \theutterance \stepcounter{utterance}  

    & & \multicolumn{4}{p{0.6\linewidth}}{\cellcolor[rgb]{0.9,0.9,0.9}{%
	\makecell[{{p{\linewidth}}}]{% 
	  \tt {\tiny [A$\langle$GM]}  
	 Deine Nachricht wurde an deinen Mitspieler übermittelt.\\ \tt \\ \tt \\ \tt Ihr seid gerade bei Runde 2 von maximal 20.\\ \tt Ihr habt gemeinsam 0 von maximal 12 Strafpunkten angesammelt.\\ \tt Dein Mitspieler hat dir diese Nachricht geschickt:\\ \tt \\ \tt \\ \tt "Danke für die Informationen! Auf meinem Brett ist 'C' bei (2,1), 'I' bei (6,6), 'T' bei (1,6), 'W' bei (2,2) und 'H' bei (4,4). Ich denke, wir sollten uns auf eine Anordnung konzentrieren, die für beide Bretter machbar ist. Basierend auf unseren aktuellen Anordnungen, könnte ich mir vorstellen, 'T' bei (4,1) zu platzieren, 'W' bei (3,2), 'I' bei (5,2), 'C' bei (2,3) und 'H' bei (6,3). Was meinst du? Sollten wir es versuchen?"\\ \tt \\ \tt \\ \tt Wie lautet dein nächster Befehl? 
	  } 
	   } 
	   } 
	 & \\ 
 

    \theutterance \stepcounter{utterance}  

    & \multicolumn{4}{p{0.6\linewidth}}{\cellcolor[rgb]{0.8,1,0.9}{%
	\makecell[{{p{\linewidth}}}]{% 
	  \tt {\tiny [A$\rangle$GM]}  
	 SPRICH: Dein Vorschlag klingt nach einer guten Kompromisslösung, die für beide unserer Bretter machbar sein könnte. Ich denke, wir sollten es versuchen. Bevor wir beginnen, die Objekte zu bewegen, sollten wir noch einmal die Zielpositionen festhalten: 'T' bei (4,1), 'W' bei (3,2), 'I' bei (5,2), 'C' bei (2,3) und 'H' bei (6,3). Ich werde mich bemühen, meine Objekte in diese Positionen zu bewegen. Bitte bestätige, dass du bereit bist, das gleiche zu tun, und wir können loslegen. 
	  } 
	   } 
	   } 
	 & & \\ 
 

    \theutterance \stepcounter{utterance}  

    & & & \multicolumn{2}{p{0.3\linewidth}}{\cellcolor[rgb]{0.95,0.95,0.95}{%
	\makecell[{{p{\linewidth}}}]{% 
	  \tt {\tiny [GM$|$GM]}  
	 SPRICH: Dein Vorschlag klingt nach einer guten Kompromisslösung, die für beide unserer Bretter machbar sein könnte. Ich denke, wir sollten es versuchen. Bevor wir beginnen, die Objekte zu bewegen, sollten wir noch einmal die Zielpositionen festhalten: 'T' bei (4,1), 'W' bei (3,2), 'I' bei (5,2), 'C' bei (2,3) und 'H' bei (6,3). Ich werde mich bemühen, meine Objekte in diese Positionen zu bewegen. Bitte bestätige, dass du bereit bist, das gleiche zu tun, und wir können loslegen. 
	  } 
	   } 
	   } 
	 & & \\ 
 

    \theutterance \stepcounter{utterance}  

    & & \multicolumn{4}{p{0.6\linewidth}}{\cellcolor[rgb]{0.9,0.9,0.9}{%
	\makecell[{{p{\linewidth}}}]{% 
	  \tt {\tiny [GM$\rangle$B]}  
	 Deine Nachricht wurde an deinen Mitspieler übermittelt.\\ \tt \\ \tt \\ \tt Ihr seid gerade bei Runde 3 von maximal 20.\\ \tt Ihr habt gemeinsam 0 von maximal 12 Strafpunkten angesammelt.\\ \tt Dein Mitspieler hat dir diese Nachricht geschickt:\\ \tt \\ \tt \\ \tt "Dein Vorschlag klingt nach einer guten Kompromisslösung, die für beide unserer Bretter machbar sein könnte. Ich denke, wir sollten es versuchen. Bevor wir beginnen, die Objekte zu bewegen, sollten wir noch einmal die Zielpositionen festhalten: 'T' bei (4,1), 'W' bei (3,2), 'I' bei (5,2), 'C' bei (2,3) und 'H' bei (6,3). Ich werde mich bemühen, meine Objekte in diese Positionen zu bewegen. Bitte bestätige, dass du bereit bist, das gleiche zu tun, und wir können loslegen."\\ \tt \\ \tt \\ \tt Wie lautet dein nächster Befehl? 
	  } 
	   } 
	   } 
	 & \\ 
 

    \theutterance \stepcounter{utterance}  

    & & & \multicolumn{4}{p{0.6\linewidth}}{\cellcolor[rgb]{1,0.85,0.72}{%
	\makecell[{{p{\linewidth}}}]{% 
	  \tt {\tiny [GM$\langle$B]}  
	 `SPRICH: Ja, ich bin bereit! Ich bestätige, dass unsere Zielpositionen 'T' bei (4,1), 'W' bei (3,2), 'I' bei (5,2), 'C' bei (2,3) und 'H' bei (6,3) sind. Ich werde mich bemühen, meine Objekte in diese Positionen zu bewegen. Lass uns loslegen! Ich werde zuerst 'T' verschieben, um Platz für die anderen Objekte zu schaffen. Bitte gib mir ein Signal, wenn du bereit bist, deine Objekte zu bewegen, damit wir sicherstellen, dass wir nicht in die Quere kommen.` 
	  } 
	   } 
	   } 
	  \\ 
 

    \theutterance \stepcounter{utterance}  

    & & & \multicolumn{2}{p{0.3\linewidth}}{\cellcolor[rgb]{0.95,0.95,0.95}{%
	\makecell[{{p{\linewidth}}}]{% 
	  \tt {\tiny [GM$|$GM]}  
	 `SPRICH: Ja, ich bin bereit! Ich bestätige, dass unsere Zielpositionen 'T' bei (4,1), 'W' bei (3,2), 'I' bei (5,2), 'C' bei (2,3) und 'H' bei (6,3) sind. Ich werde mich bemühen, meine Objekte in diese Positionen zu bewegen. Lass uns loslegen! Ich werde zuerst 'T' verschieben, um Platz für die anderen Objekte zu schaffen. Bitte gib mir ein Signal, wenn du bereit bist, deine Objekte zu bewegen, damit wir sicherstellen, dass wir nicht in die Quere kommen.` 
	  } 
	   } 
	   } 
	 & & \\ 
 

    \theutterance \stepcounter{utterance}  

    & & \multicolumn{4}{p{0.6\linewidth}}{\cellcolor[rgb]{0.9,0.9,0.9}{%
	\makecell[{{p{\linewidth}}}]{% 
	  \tt {\tiny [A$\langle$GM]}  
	 Deine Nachricht wurde an deinen Mitspieler übermittelt.\\ \tt \\ \tt \\ \tt Ihr seid gerade bei Runde 3 von maximal 20.\\ \tt Ihr habt gemeinsam 0 von maximal 12 Strafpunkten angesammelt.\\ \tt Dein Mitspieler hat dir diese Nachricht geschickt:\\ \tt \\ \tt \\ \tt "Ja, ich bin bereit! Ich bestätige, dass unsere Zielpositionen 'T' bei (4,1), 'W' bei (3,2), 'I' bei (5,2), 'C' bei (2,3) und 'H' bei (6,3) sind. Ich werde mich bemühen, meine Objekte in diese Positionen zu bewegen. Lass uns loslegen! Ich werde zuerst 'T' verschieben, um Platz für die anderen Objekte zu schaffen. Bitte gib mir ein Signal, wenn du bereit bist, deine Objekte zu bewegen, damit wir sicherstellen, dass wir nicht in die Quere kommen."\\ \tt \\ \tt \\ \tt Wie lautet dein nächster Befehl? 
	  } 
	   } 
	   } 
	 & \\ 
 

    \theutterance \stepcounter{utterance}  

    & \multicolumn{4}{p{0.6\linewidth}}{\cellcolor[rgb]{0.8,1,0.9}{%
	\makecell[{{p{\linewidth}}}]{% 
	  \tt {\tiny [A$\rangle$GM]}  
	 SPRICH: Ich bin bereit, meine Objekte zu bewegen. Ich werde mit 'C' beginnen, das sich derzeit bei (1,1) befindet, und es zur Zielposition (2,3) bewegen. Bitte gib mir ein Update, wie deine Bewegung von 'T' verläuft, bevor ich meine Bewegung ausführe. Ich werde dann den Befehl `VERSCHIEBE: C, (2,3)` geben, um meine Bewegung zu starten. 
	  } 
	   } 
	   } 
	 & & \\ 
 

    \theutterance \stepcounter{utterance}  

    & & & \multicolumn{2}{p{0.3\linewidth}}{\cellcolor[rgb]{0.95,0.95,0.95}{%
	\makecell[{{p{\linewidth}}}]{% 
	  \tt {\tiny [GM$|$GM]}  
	 SPRICH: Ich bin bereit, meine Objekte zu bewegen. Ich werde mit 'C' beginnen, das sich derzeit bei (1,1) befindet, und es zur Zielposition (2,3) bewegen. Bitte gib mir ein Update, wie deine Bewegung von 'T' verläuft, bevor ich meine Bewegung ausführe. Ich werde dann den Befehl `VERSCHIEBE: C, (2,3)` geben, um meine Bewegung zu starten. 
	  } 
	   } 
	   } 
	 & & \\ 
 

    \theutterance \stepcounter{utterance}  

    & & & \multicolumn{2}{p{0.3\linewidth}}{\cellcolor[rgb]{0.95,0.95,0.95}{%
	\makecell[{{p{\linewidth}}}]{% 
	  \tt {\tiny [GM$|$GM]}  
	 Invalid response format: SPRICH: Ich bin bereit, meine Objekte zu bewegen. Ich werde mit 'C' beginnen, das sich derzeit bei (1,1) befindet, und es zur Zielposition (2,3) bewegen. Bitte gib mir ein Update, wie deine Bewegung von 'T' verläuft, bevor ich meine Bewegung ausführe. Ich werde dann den Befehl VERSCHIEBE: C, (2,3) geben, um meine Bewegung zu starten. 
	  } 
	   } 
	   } 
	 & & \\ 
 

    \theutterance \stepcounter{utterance}  

    & & \multicolumn{4}{p{0.6\linewidth}}{\cellcolor[rgb]{0.9,0.9,0.9}{%
	\makecell[{{p{\linewidth}}}]{% 
	  \tt {\tiny [A$\langle$GM]}  
	 Strafpunkt: Deine Nachricht enthält mehr als einen Befehl! Stelle sicher, dass deine Antwort ausschließlich einen der beiden Befehle `SPRICH: <NACHRICHT>` und `VERSCHIEBE: <OBJECT>, (<X>, <Y>)` und sonst nichts enthält!\\ \tt \\ \tt Ihr habt gemeinsam 1 von maximal 12 Strafpunkten angesammelt.\\ \tt Bitte versuche es noch mal! 
	  } 
	   } 
	   } 
	 & \\ 
 

    \theutterance \stepcounter{utterance}  

    & \multicolumn{4}{p{0.6\linewidth}}{\cellcolor[rgb]{0.8,1,0.9}{%
	\makecell[{{p{\linewidth}}}]{% 
	  \tt {\tiny [A$\rangle$GM]}  
	 SPRICH: Ich bin bereit, meine Objekte zu bewegen. Ich werde mit 'C' beginnen, das sich derzeit bei (1,1) befindet, und es zur Zielposition (2,3) bewegen. Bitte gib mir ein Update, wie deine Bewegung von 'T' verläuft, bevor ich meine Bewegung ausführe. 
	  } 
	   } 
	   } 
	 & & \\ 
 

    \theutterance \stepcounter{utterance}  

    & & & \multicolumn{2}{p{0.3\linewidth}}{\cellcolor[rgb]{0.95,0.95,0.95}{%
	\makecell[{{p{\linewidth}}}]{% 
	  \tt {\tiny [GM$|$GM]}  
	 SPRICH: Ich bin bereit, meine Objekte zu bewegen. Ich werde mit 'C' beginnen, das sich derzeit bei (1,1) befindet, und es zur Zielposition (2,3) bewegen. Bitte gib mir ein Update, wie deine Bewegung von 'T' verläuft, bevor ich meine Bewegung ausführe. 
	  } 
	   } 
	   } 
	 & & \\ 
 

    \theutterance \stepcounter{utterance}  

    & & \multicolumn{4}{p{0.6\linewidth}}{\cellcolor[rgb]{0.9,0.9,0.9}{%
	\makecell[{{p{\linewidth}}}]{% 
	  \tt {\tiny [GM$\rangle$B]}  
	 Deine Nachricht wurde an deinen Mitspieler übermittelt.\\ \tt \\ \tt \\ \tt Ihr seid gerade bei Runde 4 von maximal 20.\\ \tt Ihr habt gemeinsam 1 von maximal 12 Strafpunkten angesammelt.\\ \tt Dein Mitspieler hat dir diese Nachricht geschickt:\\ \tt \\ \tt \\ \tt "Ich bin bereit, meine Objekte zu bewegen. Ich werde mit 'C' beginnen, das sich derzeit bei (1,1) befindet, und es zur Zielposition (2,3) bewegen. Bitte gib mir ein Update, wie deine Bewegung von 'T' verläuft, bevor ich meine Bewegung ausführe."\\ \tt \\ \tt \\ \tt Wie lautet dein nächster Befehl? 
	  } 
	   } 
	   } 
	 & \\ 
 

    \theutterance \stepcounter{utterance}  

    & & & \multicolumn{4}{p{0.6\linewidth}}{\cellcolor[rgb]{1,0.85,0.72}{%
	\makecell[{{p{\linewidth}}}]{% 
	  \tt {\tiny [GM$\langle$B]}  
	 `VERSCHIEBE: T, (4,1)` 
	  } 
	   } 
	   } 
	  \\ 
 

    \theutterance \stepcounter{utterance}  

    & & & \multicolumn{2}{p{0.3\linewidth}}{\cellcolor[rgb]{0.95,0.95,0.95}{%
	\makecell[{{p{\linewidth}}}]{% 
	  \tt {\tiny [GM$|$GM]}  
	 `VERSCHIEBE: T, (4,1)` 
	  } 
	   } 
	   } 
	 & & \\ 
 

    \theutterance \stepcounter{utterance}  

    & & & \multicolumn{2}{p{0.3\linewidth}}{\cellcolor[rgb]{0.95,0.95,0.95}{%
	\makecell[{{p{\linewidth}}}]{% 
	  \tt {\tiny [GM$|$GM]}  
	 Strafpunkt: (4,1) ist nicht leer, sondern enthält '│'.\\ \tt \\ \tt Ihr habt gemeinsam 2 von maximal 12 Strafpunkten angesammelt.\\ \tt Bitte versuche es noch mal! 
	  } 
	   } 
	   } 
	 & & \\ 
 

    \theutterance \stepcounter{utterance}  

    & & \multicolumn{4}{p{0.6\linewidth}}{\cellcolor[rgb]{0.9,0.9,0.9}{%
	\makecell[{{p{\linewidth}}}]{% 
	  \tt {\tiny [GM$\rangle$B]}  
	 Strafpunkt: (4,1) ist nicht leer, sondern enthält '│'.\\ \tt \\ \tt Ihr habt gemeinsam 2 von maximal 12 Strafpunkten angesammelt.\\ \tt Bitte versuche es noch mal! 
	  } 
	   } 
	   } 
	 & \\ 
 

    \theutterance \stepcounter{utterance}  

    & & & \multicolumn{4}{p{0.6\linewidth}}{\cellcolor[rgb]{1,0.85,0.72}{%
	\makecell[{{p{\linewidth}}}]{% 
	  \tt {\tiny [GM$\langle$B]}  
	 `VERSCHIEBE: T, (1,1)` 
	  } 
	   } 
	   } 
	  \\ 
 

    \theutterance \stepcounter{utterance}  

    & & & \multicolumn{2}{p{0.3\linewidth}}{\cellcolor[rgb]{0.95,0.95,0.95}{%
	\makecell[{{p{\linewidth}}}]{% 
	  \tt {\tiny [GM$|$GM]}  
	 `VERSCHIEBE: T, (1,1)` 
	  } 
	   } 
	   } 
	 & & \\ 
 

    \theutterance \stepcounter{utterance}  

    & & & \multicolumn{2}{p{0.3\linewidth}}{\cellcolor[rgb]{0.95,0.95,0.95}{%
	\makecell[{{p{\linewidth}}}]{% 
	  \tt {\tiny [GM$|$GM]}  
	 'T' erfolgreich nach (1,1) verschoben. Dein Raster sieht jetzt so aus:\\ \tt ```\\ \tt  1234567\\ \tt ╔═══╤═╤═╗\\ \tt ║T◌◌│C│◌║ 1\\ \tt ║◌W◌│◌├─╢ 2\\ \tt ║◌◌◌├─┘◌║ 3\\ \tt ║◌◌◌│◌◌H║ 4\\ \tt ╟─┬─┼───╢ 5\\ \tt ║◌│◌│I◌◌║ 6\\ \tt ║◌│◌│◌◌◌║ 7\\ \tt ╚═╧═╧═══╝\\ \tt ```\\ \tt  
	  } 
	   } 
	   } 
	 & & \\ 
 

    \theutterance \stepcounter{utterance}  

    & & \multicolumn{4}{p{0.6\linewidth}}{\cellcolor[rgb]{0.9,0.9,0.9}{%
	\makecell[{{p{\linewidth}}}]{% 
	  \tt {\tiny [A$\langle$GM]}  
	 Deine Nachricht wurde an deinen Mitspieler übermittelt.\\ \tt \\ \tt \\ \tt Ihr seid gerade bei Runde 4 von maximal 20.\\ \tt Ihr habt gemeinsam 2 von maximal 12 Strafpunkten angesammelt.\\ \tt Dein Mitspieler hat ein Objekt auf seinem Raster bewegt.\\ \tt \\ \tt \\ \tt Wie lautet dein nächster Befehl? 
	  } 
	   } 
	   } 
	 & \\ 
 

    \theutterance \stepcounter{utterance}  

    & \multicolumn{4}{p{0.6\linewidth}}{\cellcolor[rgb]{0.8,1,0.9}{%
	\makecell[{{p{\linewidth}}}]{% 
	  \tt {\tiny [A$\rangle$GM]}  
	 VERSCHIEBE: C, (2,3) 
	  } 
	   } 
	   } 
	 & & \\ 
 

    \theutterance \stepcounter{utterance}  

    & & & \multicolumn{2}{p{0.3\linewidth}}{\cellcolor[rgb]{0.95,0.95,0.95}{%
	\makecell[{{p{\linewidth}}}]{% 
	  \tt {\tiny [GM$|$GM]}  
	 VERSCHIEBE: C, (2,3) 
	  } 
	   } 
	   } 
	 & & \\ 
 

    \theutterance \stepcounter{utterance}  

    & & & \multicolumn{2}{p{0.3\linewidth}}{\cellcolor[rgb]{0.95,0.95,0.95}{%
	\makecell[{{p{\linewidth}}}]{% 
	  \tt {\tiny [GM$|$GM]}  
	 'C' erfolgreich nach (2,3) verschoben. Dein Raster sieht jetzt so aus:\\ \tt ```\\ \tt  1234567\\ \tt ╔═══╤═╤═╗\\ \tt ║◌◌◌│◌│◌║ 1\\ \tt ║◌I◌│◌├─╢ 2\\ \tt ║◌C◌├─┘◌║ 3\\ \tt ║◌◌◌│◌◌◌║ 4\\ \tt ╟─┬─┼───╢ 5\\ \tt ║◌│◌│◌WT║ 6\\ \tt ║◌│◌│◌◌H║ 7\\ \tt ╚═╧═╧═══╝\\ \tt ```\\ \tt  
	  } 
	   } 
	   } 
	 & & \\ 
 

    \theutterance \stepcounter{utterance}  

    & & \multicolumn{4}{p{0.6\linewidth}}{\cellcolor[rgb]{0.9,0.9,0.9}{%
	\makecell[{{p{\linewidth}}}]{% 
	  \tt {\tiny [GM$\rangle$B]}  
	 'T' erfolgreich nach (1,1) verschoben. Dein Raster sieht jetzt so aus:\\ \tt ```\\ \tt  1234567\\ \tt ╔═══╤═╤═╗\\ \tt ║T◌◌│C│◌║ 1\\ \tt ║◌W◌│◌├─╢ 2\\ \tt ║◌◌◌├─┘◌║ 3\\ \tt ║◌◌◌│◌◌H║ 4\\ \tt ╟─┬─┼───╢ 5\\ \tt ║◌│◌│I◌◌║ 6\\ \tt ║◌│◌│◌◌◌║ 7\\ \tt ╚═╧═╧═══╝\\ \tt ```\\ \tt Ihr seid gerade bei Runde 5 von maximal 20.\\ \tt Ihr habt gemeinsam 2 von maximal 12 Strafpunkten angesammelt.\\ \tt Dein Mitspieler hat ein Objekt auf seinem Raster bewegt.\\ \tt \\ \tt \\ \tt Wie lautet dein nächster Befehl? 
	  } 
	   } 
	   } 
	 & \\ 
 

    \theutterance \stepcounter{utterance}  

    & & & \multicolumn{4}{p{0.6\linewidth}}{\cellcolor[rgb]{1,0.85,0.72}{%
	\makecell[{{p{\linewidth}}}]{% 
	  \tt {\tiny [GM$\langle$B]}  
	 `SPRICH: Ich habe 'T' erfolgreich nach (1,1) verschoben. Wie sieht dein aktuelles Raster aus? Hast du 'C' erfolgreich verschoben? Bitte gib mir ein Update, damit wir unsere nächsten Schritte koordinieren können.` 
	  } 
	   } 
	   } 
	  \\ 
 

    \theutterance \stepcounter{utterance}  

    & & & \multicolumn{2}{p{0.3\linewidth}}{\cellcolor[rgb]{0.95,0.95,0.95}{%
	\makecell[{{p{\linewidth}}}]{% 
	  \tt {\tiny [GM$|$GM]}  
	 `SPRICH: Ich habe 'T' erfolgreich nach (1,1) verschoben. Wie sieht dein aktuelles Raster aus? Hast du 'C' erfolgreich verschoben? Bitte gib mir ein Update, damit wir unsere nächsten Schritte koordinieren können.` 
	  } 
	   } 
	   } 
	 & & \\ 
 

    \theutterance \stepcounter{utterance}  

    & & \multicolumn{4}{p{0.6\linewidth}}{\cellcolor[rgb]{0.9,0.9,0.9}{%
	\makecell[{{p{\linewidth}}}]{% 
	  \tt {\tiny [A$\langle$GM]}  
	 'C' erfolgreich nach (2,3) verschoben. Dein Raster sieht jetzt so aus:\\ \tt ```\\ \tt  1234567\\ \tt ╔═══╤═╤═╗\\ \tt ║◌◌◌│◌│◌║ 1\\ \tt ║◌I◌│◌├─╢ 2\\ \tt ║◌C◌├─┘◌║ 3\\ \tt ║◌◌◌│◌◌◌║ 4\\ \tt ╟─┬─┼───╢ 5\\ \tt ║◌│◌│◌WT║ 6\\ \tt ║◌│◌│◌◌H║ 7\\ \tt ╚═╧═╧═══╝\\ \tt ```\\ \tt Ihr seid gerade bei Runde 5 von maximal 20.\\ \tt Ihr habt gemeinsam 2 von maximal 12 Strafpunkten angesammelt.\\ \tt Dein Mitspieler hat dir diese Nachricht geschickt:\\ \tt \\ \tt \\ \tt "Ich habe 'T' erfolgreich nach (1,1) verschoben. Wie sieht dein aktuelles Raster aus? Hast du 'C' erfolgreich verschoben? Bitte gib mir ein Update, damit wir unsere nächsten Schritte koordinieren können."\\ \tt \\ \tt \\ \tt Wie lautet dein nächster Befehl? 
	  } 
	   } 
	   } 
	 & \\ 
 

    \theutterance \stepcounter{utterance}  

    & \multicolumn{4}{p{0.6\linewidth}}{\cellcolor[rgb]{0.8,1,0.9}{%
	\makecell[{{p{\linewidth}}}]{% 
	  \tt {\tiny [A$\rangle$GM]}  
	 SPRICH: Ja, ich habe 'C' erfolgreich nach (2,3) verschoben. Mein aktuelles Raster sieht so aus: \\ \tt ```\\ \tt  1234567\\ \tt ╔═══╤═╤═╗\\ \tt ║◌◌◌│◌│◌║ 1\\ \tt ║◌I◌│◌├─╢ 2\\ \tt ║◌C◌├─┘◌║ 3\\ \tt ║◌◌◌│◌◌◌║ 4\\ \tt ╟─┬─┼───╢ 5\\ \tt ║◌│◌│◌WT║ 6\\ \tt ║◌│◌│◌◌H║ 7\\ \tt ╚═╧═╧═══╝\\ \tt ```\\ \tt Als nächstes werde ich 'I' zur Zielposition (5,2) bewegen. Bitte gib mir ein Update, wie deine nächsten Schritte aussehen. 
	  } 
	   } 
	   } 
	 & & \\ 
 

    \theutterance \stepcounter{utterance}  

    & & & \multicolumn{2}{p{0.3\linewidth}}{\cellcolor[rgb]{0.95,0.95,0.95}{%
	\makecell[{{p{\linewidth}}}]{% 
	  \tt {\tiny [GM$|$GM]}  
	 SPRICH: Ja, ich habe 'C' erfolgreich nach (2,3) verschoben. Mein aktuelles Raster sieht so aus: \\ \tt ```\\ \tt  1234567\\ \tt ╔═══╤═╤═╗\\ \tt ║◌◌◌│◌│◌║ 1\\ \tt ║◌I◌│◌├─╢ 2\\ \tt ║◌C◌├─┘◌║ 3\\ \tt ║◌◌◌│◌◌◌║ 4\\ \tt ╟─┬─┼───╢ 5\\ \tt ║◌│◌│◌WT║ 6\\ \tt ║◌│◌│◌◌H║ 7\\ \tt ╚═╧═╧═══╝\\ \tt ```\\ \tt Als nächstes werde ich 'I' zur Zielposition (5,2) bewegen. Bitte gib mir ein Update, wie deine nächsten Schritte aussehen. 
	  } 
	   } 
	   } 
	 & & \\ 
 

    \theutterance \stepcounter{utterance}  

    & & & \multicolumn{2}{p{0.3\linewidth}}{\cellcolor[rgb]{0.95,0.95,0.95}{%
	\makecell[{{p{\linewidth}}}]{% 
	  \tt {\tiny [GM$|$GM]}  
	 Invalid format: tail is not empty: '\\ \tt \\ \tt  1234567\\ \tt ╔═══╤═╤═╗\\ \tt ║◌◌◌│◌│◌║ 1\\ \tt ║◌I◌│◌├─╢ 2\\ \tt ║◌C◌├─┘◌║ 3\\ \tt ║◌◌◌│◌◌◌║ 4\\ \tt ╟─┬─┼───╢ 5\\ \tt ║◌│◌│◌WT║ 6\\ \tt ║◌│◌│◌◌H║ 7\\ \tt ╚═╧═╧═══╝\\ \tt \\ \tt Als nächstes werde ich 'I' zur Zielposition (5,2) bewegen. Bitte gib mir ein Update, wie deine nächsten Schritte aussehen. 
	  } 
	   } 
	   } 
	 & & \\ 
 

    \theutterance \stepcounter{utterance}  

    & & \multicolumn{4}{p{0.6\linewidth}}{\cellcolor[rgb]{0.9,0.9,0.9}{%
	\makecell[{{p{\linewidth}}}]{% 
	  \tt {\tiny [A$\langle$GM]}  
	 Strafpunkt: Deine Nachricht darf nichts nach dem Befehl enthalten! Stelle sicher, dass deine Antwort ausschließlich einen der beiden Befehle `SPRICH: <NACHRICHT>` und `VERSCHIEBE: <OBJECT>, (<X>, <Y>)` und sonst nichts enthält!\\ \tt \\ \tt Ihr habt gemeinsam 3 von maximal 12 Strafpunkten angesammelt.\\ \tt Bitte versuche es noch mal! 
	  } 
	   } 
	   } 
	 & \\ 
 

    \theutterance \stepcounter{utterance}  

    & \multicolumn{4}{p{0.6\linewidth}}{\cellcolor[rgb]{0.8,1,0.9}{%
	\makecell[{{p{\linewidth}}}]{% 
	  \tt {\tiny [A$\rangle$GM]}  
	 SPRICH: Ja, ich habe 'C' erfolgreich nach (2,3) verschoben. Mein aktuelles Raster sieht so aus: \\ \tt ```\\ \tt  1234567\\ \tt ╔═══╤═╤═╗\\ \tt ║◌◌◌│◌│◌║ 1\\ \tt ║◌I◌│◌├─╢ 2\\ \tt ║◌C◌├─┘◌║ 3\\ \tt ║◌◌◌│◌◌◌║ 4\\ \tt ╟─┬─┼───╢ 5\\ \tt ║◌│◌│◌WT║ 6\\ \tt ║◌│◌│◌◌H║ 7\\ \tt ╚═╧═╧═══╝\\ \tt ``` 
	  } 
	   } 
	   } 
	 & & \\ 
 

    \theutterance \stepcounter{utterance}  

    & & & \multicolumn{2}{p{0.3\linewidth}}{\cellcolor[rgb]{0.95,0.95,0.95}{%
	\makecell[{{p{\linewidth}}}]{% 
	  \tt {\tiny [GM$|$GM]}  
	 SPRICH: Ja, ich habe 'C' erfolgreich nach (2,3) verschoben. Mein aktuelles Raster sieht so aus: \\ \tt ```\\ \tt  1234567\\ \tt ╔═══╤═╤═╗\\ \tt ║◌◌◌│◌│◌║ 1\\ \tt ║◌I◌│◌├─╢ 2\\ \tt ║◌C◌├─┘◌║ 3\\ \tt ║◌◌◌│◌◌◌║ 4\\ \tt ╟─┬─┼───╢ 5\\ \tt ║◌│◌│◌WT║ 6\\ \tt ║◌│◌│◌◌H║ 7\\ \tt ╚═╧═╧═══╝\\ \tt ``` 
	  } 
	   } 
	   } 
	 & & \\ 
 

    \theutterance \stepcounter{utterance}  

    & & & \multicolumn{2}{p{0.3\linewidth}}{\cellcolor[rgb]{0.95,0.95,0.95}{%
	\makecell[{{p{\linewidth}}}]{% 
	  \tt {\tiny [GM$|$GM]}  
	 Invalid format: tail is not empty: '\\ \tt \\ \tt  1234567\\ \tt ╔═══╤═╤═╗\\ \tt ║◌◌◌│◌│◌║ 1\\ \tt ║◌I◌│◌├─╢ 2\\ \tt ║◌C◌├─┘◌║ 3\\ \tt ║◌◌◌│◌◌◌║ 4\\ \tt ╟─┬─┼───╢ 5\\ \tt ║◌│◌│◌WT║ 6\\ \tt ║◌│◌│◌◌H║ 7\\ \tt ╚═╧═╧═══╝ 
	  } 
	   } 
	   } 
	 & & \\ 
 

    \theutterance \stepcounter{utterance}  

    & & \multicolumn{4}{p{0.6\linewidth}}{\cellcolor[rgb]{0.9,0.9,0.9}{%
	\makecell[{{p{\linewidth}}}]{% 
	  \tt {\tiny [A$\langle$GM]}  
	 Strafpunkt: Deine Nachricht darf nichts nach dem Befehl enthalten! Stelle sicher, dass deine Antwort ausschließlich einen der beiden Befehle `SPRICH: <NACHRICHT>` und `VERSCHIEBE: <OBJECT>, (<X>, <Y>)` und sonst nichts enthält!\\ \tt \\ \tt Ihr habt gemeinsam 4 von maximal 12 Strafpunkten angesammelt.\\ \tt Bitte versuche es noch mal! 
	  } 
	   } 
	   } 
	 & \\ 
 

    \theutterance \stepcounter{utterance}  

    & \multicolumn{4}{p{0.6\linewidth}}{\cellcolor[rgb]{0.8,1,0.9}{%
	\makecell[{{p{\linewidth}}}]{% 
	  \tt {\tiny [A$\rangle$GM]}  
	 SPRICH: Ich habe 'C' erfolgreich nach (2,3) verschoben. Mein aktuelles Raster ist oben. 
	  } 
	   } 
	   } 
	 & & \\ 
 

    \theutterance \stepcounter{utterance}  

    & & & \multicolumn{2}{p{0.3\linewidth}}{\cellcolor[rgb]{0.95,0.95,0.95}{%
	\makecell[{{p{\linewidth}}}]{% 
	  \tt {\tiny [GM$|$GM]}  
	 SPRICH: Ich habe 'C' erfolgreich nach (2,3) verschoben. Mein aktuelles Raster ist oben. 
	  } 
	   } 
	   } 
	 & & \\ 
 

    \theutterance \stepcounter{utterance}  

    & & \multicolumn{4}{p{0.6\linewidth}}{\cellcolor[rgb]{0.9,0.9,0.9}{%
	\makecell[{{p{\linewidth}}}]{% 
	  \tt {\tiny [GM$\rangle$B]}  
	 Deine Nachricht wurde an deinen Mitspieler übermittelt.\\ \tt \\ \tt \\ \tt Ihr seid gerade bei Runde 6 von maximal 20.\\ \tt Ihr habt gemeinsam 4 von maximal 12 Strafpunkten angesammelt.\\ \tt Dein Mitspieler hat dir diese Nachricht geschickt:\\ \tt \\ \tt \\ \tt "Ich habe 'C' erfolgreich nach (2,3) verschoben. Mein aktuelles Raster ist oben."\\ \tt \\ \tt \\ \tt Wie lautet dein nächster Befehl? 
	  } 
	   } 
	   } 
	 & \\ 
 

    \theutterance \stepcounter{utterance}  

    & & & \multicolumn{4}{p{0.6\linewidth}}{\cellcolor[rgb]{1,0.85,0.72}{%
	\makecell[{{p{\linewidth}}}]{% 
	  \tt {\tiny [GM$\langle$B]}  
	 `SPRICH: Ich bin froh, dass du 'C' erfolgreich verschoben hast! Ich werde jetzt 'W' von (2,2) nach (3,2) versuchen zu bewegen, um unsere Zielpositionen zu erreichen. Bitte gib mir ein Update, wenn du bereit bist, dein nächstes Objekt zu bewegen.` 
	  } 
	   } 
	   } 
	  \\ 
 

    \theutterance \stepcounter{utterance}  

    & & & \multicolumn{2}{p{0.3\linewidth}}{\cellcolor[rgb]{0.95,0.95,0.95}{%
	\makecell[{{p{\linewidth}}}]{% 
	  \tt {\tiny [GM$|$GM]}  
	 `SPRICH: Ich bin froh, dass du 'C' erfolgreich verschoben hast! Ich werde jetzt 'W' von (2,2) nach (3,2) versuchen zu bewegen, um unsere Zielpositionen zu erreichen. Bitte gib mir ein Update, wenn du bereit bist, dein nächstes Objekt zu bewegen.` 
	  } 
	   } 
	   } 
	 & & \\ 
 

    \theutterance \stepcounter{utterance}  

    & & \multicolumn{4}{p{0.6\linewidth}}{\cellcolor[rgb]{0.9,0.9,0.9}{%
	\makecell[{{p{\linewidth}}}]{% 
	  \tt {\tiny [A$\langle$GM]}  
	 Deine Nachricht wurde an deinen Mitspieler übermittelt.\\ \tt \\ \tt \\ \tt Ihr seid gerade bei Runde 6 von maximal 20.\\ \tt Ihr habt gemeinsam 4 von maximal 12 Strafpunkten angesammelt.\\ \tt Dein Mitspieler hat dir diese Nachricht geschickt:\\ \tt \\ \tt \\ \tt "Ich bin froh, dass du 'C' erfolgreich verschoben hast! Ich werde jetzt 'W' von (2,2) nach (3,2) versuchen zu bewegen, um unsere Zielpositionen zu erreichen. Bitte gib mir ein Update, wenn du bereit bist, dein nächstes Objekt zu bewegen."\\ \tt \\ \tt \\ \tt Wie lautet dein nächster Befehl? 
	  } 
	   } 
	   } 
	 & \\ 
 

    \theutterance \stepcounter{utterance}  

    & \multicolumn{4}{p{0.6\linewidth}}{\cellcolor[rgb]{0.8,1,0.9}{%
	\makecell[{{p{\linewidth}}}]{% 
	  \tt {\tiny [A$\rangle$GM]}  
	 SPRICH: Ich bin bereit, mein nächstes Objekt zu bewegen. Als nächstes werde ich 'I' zur Zielposition (5,2) bewegen. 
	  } 
	   } 
	   } 
	 & & \\ 
 

    \theutterance \stepcounter{utterance}  

    & & & \multicolumn{2}{p{0.3\linewidth}}{\cellcolor[rgb]{0.95,0.95,0.95}{%
	\makecell[{{p{\linewidth}}}]{% 
	  \tt {\tiny [GM$|$GM]}  
	 SPRICH: Ich bin bereit, mein nächstes Objekt zu bewegen. Als nächstes werde ich 'I' zur Zielposition (5,2) bewegen. 
	  } 
	   } 
	   } 
	 & & \\ 
 

    \theutterance \stepcounter{utterance}  

    & & \multicolumn{4}{p{0.6\linewidth}}{\cellcolor[rgb]{0.9,0.9,0.9}{%
	\makecell[{{p{\linewidth}}}]{% 
	  \tt {\tiny [GM$\rangle$B]}  
	 Deine Nachricht wurde an deinen Mitspieler übermittelt.\\ \tt \\ \tt \\ \tt Ihr seid gerade bei Runde 7 von maximal 20.\\ \tt Ihr habt gemeinsam 4 von maximal 12 Strafpunkten angesammelt.\\ \tt Dein Mitspieler hat dir diese Nachricht geschickt:\\ \tt \\ \tt \\ \tt "Ich bin bereit, mein nächstes Objekt zu bewegen. Als nächstes werde ich 'I' zur Zielposition (5,2) bewegen."\\ \tt \\ \tt \\ \tt Wie lautet dein nächster Befehl? 
	  } 
	   } 
	   } 
	 & \\ 
 

    \theutterance \stepcounter{utterance}  

    & & & \multicolumn{4}{p{0.6\linewidth}}{\cellcolor[rgb]{1,0.85,0.72}{%
	\makecell[{{p{\linewidth}}}]{% 
	  \tt {\tiny [GM$\langle$B]}  
	 `VERSCHIEBE: W, (3,2)` 
	  } 
	   } 
	   } 
	  \\ 
 

    \theutterance \stepcounter{utterance}  

    & & & \multicolumn{2}{p{0.3\linewidth}}{\cellcolor[rgb]{0.95,0.95,0.95}{%
	\makecell[{{p{\linewidth}}}]{% 
	  \tt {\tiny [GM$|$GM]}  
	 `VERSCHIEBE: W, (3,2)` 
	  } 
	   } 
	   } 
	 & & \\ 
 

    \theutterance \stepcounter{utterance}  

    & & & \multicolumn{2}{p{0.3\linewidth}}{\cellcolor[rgb]{0.95,0.95,0.95}{%
	\makecell[{{p{\linewidth}}}]{% 
	  \tt {\tiny [GM$|$GM]}  
	 'W' erfolgreich nach (3,2) verschoben. Dein Raster sieht jetzt so aus:\\ \tt ```\\ \tt  1234567\\ \tt ╔═══╤═╤═╗\\ \tt ║T◌◌│C│◌║ 1\\ \tt ║◌◌W│◌├─╢ 2\\ \tt ║◌◌◌├─┘◌║ 3\\ \tt ║◌◌◌│◌◌H║ 4\\ \tt ╟─┬─┼───╢ 5\\ \tt ║◌│◌│I◌◌║ 6\\ \tt ║◌│◌│◌◌◌║ 7\\ \tt ╚═╧═╧═══╝\\ \tt ```\\ \tt  
	  } 
	   } 
	   } 
	 & & \\ 
 

    \theutterance \stepcounter{utterance}  

    & & \multicolumn{4}{p{0.6\linewidth}}{\cellcolor[rgb]{0.9,0.9,0.9}{%
	\makecell[{{p{\linewidth}}}]{% 
	  \tt {\tiny [A$\langle$GM]}  
	 Deine Nachricht wurde an deinen Mitspieler übermittelt.\\ \tt \\ \tt \\ \tt Ihr seid gerade bei Runde 7 von maximal 20.\\ \tt Ihr habt gemeinsam 4 von maximal 12 Strafpunkten angesammelt.\\ \tt Dein Mitspieler hat ein Objekt auf seinem Raster bewegt.\\ \tt \\ \tt \\ \tt Wie lautet dein nächster Befehl? 
	  } 
	   } 
	   } 
	 & \\ 
 

    \theutterance \stepcounter{utterance}  

    & \multicolumn{4}{p{0.6\linewidth}}{\cellcolor[rgb]{0.8,1,0.9}{%
	\makecell[{{p{\linewidth}}}]{% 
	  \tt {\tiny [A$\rangle$GM]}  
	 VERSCHIEBE: I, (5,2) 
	  } 
	   } 
	   } 
	 & & \\ 
 

    \theutterance \stepcounter{utterance}  

    & & & \multicolumn{2}{p{0.3\linewidth}}{\cellcolor[rgb]{0.95,0.95,0.95}{%
	\makecell[{{p{\linewidth}}}]{% 
	  \tt {\tiny [GM$|$GM]}  
	 VERSCHIEBE: I, (5,2) 
	  } 
	   } 
	   } 
	 & & \\ 
 

    \theutterance \stepcounter{utterance}  

    & & & \multicolumn{2}{p{0.3\linewidth}}{\cellcolor[rgb]{0.95,0.95,0.95}{%
	\makecell[{{p{\linewidth}}}]{% 
	  \tt {\tiny [GM$|$GM]}  
	 'I' erfolgreich nach (5,2) verschoben. Dein Raster sieht jetzt so aus:\\ \tt ```\\ \tt  1234567\\ \tt ╔═══╤═╤═╗\\ \tt ║◌◌◌│◌│◌║ 1\\ \tt ║◌◌◌│I├─╢ 2\\ \tt ║◌C◌├─┘◌║ 3\\ \tt ║◌◌◌│◌◌◌║ 4\\ \tt ╟─┬─┼───╢ 5\\ \tt ║◌│◌│◌WT║ 6\\ \tt ║◌│◌│◌◌H║ 7\\ \tt ╚═╧═╧═══╝\\ \tt ```\\ \tt  
	  } 
	   } 
	   } 
	 & & \\ 
 

    \theutterance \stepcounter{utterance}  

    & & \multicolumn{4}{p{0.6\linewidth}}{\cellcolor[rgb]{0.9,0.9,0.9}{%
	\makecell[{{p{\linewidth}}}]{% 
	  \tt {\tiny [GM$\rangle$B]}  
	 'W' erfolgreich nach (3,2) verschoben. Dein Raster sieht jetzt so aus:\\ \tt ```\\ \tt  1234567\\ \tt ╔═══╤═╤═╗\\ \tt ║T◌◌│C│◌║ 1\\ \tt ║◌◌W│◌├─╢ 2\\ \tt ║◌◌◌├─┘◌║ 3\\ \tt ║◌◌◌│◌◌H║ 4\\ \tt ╟─┬─┼───╢ 5\\ \tt ║◌│◌│I◌◌║ 6\\ \tt ║◌│◌│◌◌◌║ 7\\ \tt ╚═╧═╧═══╝\\ \tt ```\\ \tt Ihr seid gerade bei Runde 8 von maximal 20.\\ \tt Ihr habt gemeinsam 4 von maximal 12 Strafpunkten angesammelt.\\ \tt Dein Mitspieler hat ein Objekt auf seinem Raster bewegt.\\ \tt \\ \tt \\ \tt Wie lautet dein nächster Befehl? 
	  } 
	   } 
	   } 
	 & \\ 
 

    \theutterance \stepcounter{utterance}  

    & & & \multicolumn{4}{p{0.6\linewidth}}{\cellcolor[rgb]{1,0.85,0.72}{%
	\makecell[{{p{\linewidth}}}]{% 
	  \tt {\tiny [GM$\langle$B]}  
	 `SPRICH: Ich habe 'W' erfolgreich nach (3,2) verschoben. Wie sieht dein aktuelles Raster aus? Hast du 'I' erfolgreich nach (5,2) verschoben? Bitte gib mir ein Update, damit wir unsere nächsten Schritte koordinieren können.` 
	  } 
	   } 
	   } 
	  \\ 
 

    \theutterance \stepcounter{utterance}  

    & & & \multicolumn{2}{p{0.3\linewidth}}{\cellcolor[rgb]{0.95,0.95,0.95}{%
	\makecell[{{p{\linewidth}}}]{% 
	  \tt {\tiny [GM$|$GM]}  
	 `SPRICH: Ich habe 'W' erfolgreich nach (3,2) verschoben. Wie sieht dein aktuelles Raster aus? Hast du 'I' erfolgreich nach (5,2) verschoben? Bitte gib mir ein Update, damit wir unsere nächsten Schritte koordinieren können.` 
	  } 
	   } 
	   } 
	 & & \\ 
 

    \theutterance \stepcounter{utterance}  

    & & \multicolumn{4}{p{0.6\linewidth}}{\cellcolor[rgb]{0.9,0.9,0.9}{%
	\makecell[{{p{\linewidth}}}]{% 
	  \tt {\tiny [A$\langle$GM]}  
	 'I' erfolgreich nach (5,2) verschoben. Dein Raster sieht jetzt so aus:\\ \tt ```\\ \tt  1234567\\ \tt ╔═══╤═╤═╗\\ \tt ║◌◌◌│◌│◌║ 1\\ \tt ║◌◌◌│I├─╢ 2\\ \tt ║◌C◌├─┘◌║ 3\\ \tt ║◌◌◌│◌◌◌║ 4\\ \tt ╟─┬─┼───╢ 5\\ \tt ║◌│◌│◌WT║ 6\\ \tt ║◌│◌│◌◌H║ 7\\ \tt ╚═╧═╧═══╝\\ \tt ```\\ \tt Ihr seid gerade bei Runde 8 von maximal 20.\\ \tt Ihr habt gemeinsam 4 von maximal 12 Strafpunkten angesammelt.\\ \tt Dein Mitspieler hat dir diese Nachricht geschickt:\\ \tt \\ \tt \\ \tt "Ich habe 'W' erfolgreich nach (3,2) verschoben. Wie sieht dein aktuelles Raster aus? Hast du 'I' erfolgreich nach (5,2) verschoben? Bitte gib mir ein Update, damit wir unsere nächsten Schritte koordinieren können."\\ \tt \\ \tt \\ \tt Wie lautet dein nächster Befehl? 
	  } 
	   } 
	   } 
	 & \\ 
 

    \theutterance \stepcounter{utterance}  

    & \multicolumn{4}{p{0.6\linewidth}}{\cellcolor[rgb]{0.8,1,0.9}{%
	\makecell[{{p{\linewidth}}}]{% 
	  \tt {\tiny [A$\rangle$GM]}  
	 SPRICH: Ja, ich habe 'I' erfolgreich nach (5,2) verschoben. Mein aktuelles Raster sieht so aus. 
	  } 
	   } 
	   } 
	 & & \\ 
 

    \theutterance \stepcounter{utterance}  

    & & & \multicolumn{2}{p{0.3\linewidth}}{\cellcolor[rgb]{0.95,0.95,0.95}{%
	\makecell[{{p{\linewidth}}}]{% 
	  \tt {\tiny [GM$|$GM]}  
	 SPRICH: Ja, ich habe 'I' erfolgreich nach (5,2) verschoben. Mein aktuelles Raster sieht so aus. 
	  } 
	   } 
	   } 
	 & & \\ 
 

    \theutterance \stepcounter{utterance}  

    & & \multicolumn{4}{p{0.6\linewidth}}{\cellcolor[rgb]{0.9,0.9,0.9}{%
	\makecell[{{p{\linewidth}}}]{% 
	  \tt {\tiny [GM$\rangle$B]}  
	 Deine Nachricht wurde an deinen Mitspieler übermittelt.\\ \tt \\ \tt \\ \tt Ihr seid gerade bei Runde 9 von maximal 20.\\ \tt Ihr habt gemeinsam 4 von maximal 12 Strafpunkten angesammelt.\\ \tt Dein Mitspieler hat dir diese Nachricht geschickt:\\ \tt \\ \tt \\ \tt "Ja, ich habe 'I' erfolgreich nach (5,2) verschoben. Mein aktuelles Raster sieht so aus."\\ \tt \\ \tt \\ \tt Wie lautet dein nächster Befehl? 
	  } 
	   } 
	   } 
	 & \\ 
 

    \theutterance \stepcounter{utterance}  

    & & & \multicolumn{4}{p{0.6\linewidth}}{\cellcolor[rgb]{1,0.85,0.72}{%
	\makecell[{{p{\linewidth}}}]{% 
	  \tt {\tiny [GM$\langle$B]}  
	 `VERSCHIEBE: I, (5,2)` 
	  } 
	   } 
	   } 
	  \\ 
 

    \theutterance \stepcounter{utterance}  

    & & & \multicolumn{2}{p{0.3\linewidth}}{\cellcolor[rgb]{0.95,0.95,0.95}{%
	\makecell[{{p{\linewidth}}}]{% 
	  \tt {\tiny [GM$|$GM]}  
	 `VERSCHIEBE: I, (5,2)` 
	  } 
	   } 
	   } 
	 & & \\ 
 

    \theutterance \stepcounter{utterance}  

    & & & \multicolumn{2}{p{0.3\linewidth}}{\cellcolor[rgb]{0.95,0.95,0.95}{%
	\makecell[{{p{\linewidth}}}]{% 
	  \tt {\tiny [GM$|$GM]}  
	 'I' erfolgreich nach (5,2) verschoben. Dein Raster sieht jetzt so aus:\\ \tt ```\\ \tt  1234567\\ \tt ╔═══╤═╤═╗\\ \tt ║T◌◌│C│◌║ 1\\ \tt ║◌◌W│I├─╢ 2\\ \tt ║◌◌◌├─┘◌║ 3\\ \tt ║◌◌◌│◌◌H║ 4\\ \tt ╟─┬─┼───╢ 5\\ \tt ║◌│◌│◌◌◌║ 6\\ \tt ║◌│◌│◌◌◌║ 7\\ \tt ╚═╧═╧═══╝\\ \tt ```\\ \tt  
	  } 
	   } 
	   } 
	 & & \\ 
 

    \theutterance \stepcounter{utterance}  

    & & \multicolumn{4}{p{0.6\linewidth}}{\cellcolor[rgb]{0.9,0.9,0.9}{%
	\makecell[{{p{\linewidth}}}]{% 
	  \tt {\tiny [A$\langle$GM]}  
	 Deine Nachricht wurde an deinen Mitspieler übermittelt.\\ \tt \\ \tt \\ \tt Ihr seid gerade bei Runde 9 von maximal 20.\\ \tt Ihr habt gemeinsam 4 von maximal 12 Strafpunkten angesammelt.\\ \tt Dein Mitspieler hat ein Objekt auf seinem Raster bewegt.\\ \tt \\ \tt \\ \tt Wie lautet dein nächster Befehl? 
	  } 
	   } 
	   } 
	 & \\ 
 

    \theutterance \stepcounter{utterance}  

    & \multicolumn{4}{p{0.6\linewidth}}{\cellcolor[rgb]{0.8,1,0.9}{%
	\makecell[{{p{\linewidth}}}]{% 
	  \tt {\tiny [A$\rangle$GM]}  
	 VERSCHIEBE: T, (4,1) 
	  } 
	   } 
	   } 
	 & & \\ 
 

    \theutterance \stepcounter{utterance}  

    & & & \multicolumn{2}{p{0.3\linewidth}}{\cellcolor[rgb]{0.95,0.95,0.95}{%
	\makecell[{{p{\linewidth}}}]{% 
	  \tt {\tiny [GM$|$GM]}  
	 VERSCHIEBE: T, (4,1) 
	  } 
	   } 
	   } 
	 & & \\ 
 

    \theutterance \stepcounter{utterance}  

    & & & \multicolumn{2}{p{0.3\linewidth}}{\cellcolor[rgb]{0.95,0.95,0.95}{%
	\makecell[{{p{\linewidth}}}]{% 
	  \tt {\tiny [GM$|$GM]}  
	 Strafpunkt: (4,1) ist nicht leer, sondern enthält '│'.\\ \tt \\ \tt Ihr habt gemeinsam 5 von maximal 12 Strafpunkten angesammelt.\\ \tt Bitte versuche es noch mal! 
	  } 
	   } 
	   } 
	 & & \\ 
 

    \theutterance \stepcounter{utterance}  

    & & \multicolumn{4}{p{0.6\linewidth}}{\cellcolor[rgb]{0.9,0.9,0.9}{%
	\makecell[{{p{\linewidth}}}]{% 
	  \tt {\tiny [A$\langle$GM]}  
	 Strafpunkt: (4,1) ist nicht leer, sondern enthält '│'.\\ \tt \\ \tt Ihr habt gemeinsam 5 von maximal 12 Strafpunkten angesammelt.\\ \tt Bitte versuche es noch mal! 
	  } 
	   } 
	   } 
	 & \\ 
 

    \theutterance \stepcounter{utterance}  

    & \multicolumn{4}{p{0.6\linewidth}}{\cellcolor[rgb]{0.8,1,0.9}{%
	\makecell[{{p{\linewidth}}}]{% 
	  \tt {\tiny [A$\rangle$GM]}  
	 VERSCHIEBE: H, (6,3) 
	  } 
	   } 
	   } 
	 & & \\ 
 

    \theutterance \stepcounter{utterance}  

    & & & \multicolumn{2}{p{0.3\linewidth}}{\cellcolor[rgb]{0.95,0.95,0.95}{%
	\makecell[{{p{\linewidth}}}]{% 
	  \tt {\tiny [GM$|$GM]}  
	 VERSCHIEBE: H, (6,3) 
	  } 
	   } 
	   } 
	 & & \\ 
 

    \theutterance \stepcounter{utterance}  

    & & & \multicolumn{2}{p{0.3\linewidth}}{\cellcolor[rgb]{0.95,0.95,0.95}{%
	\makecell[{{p{\linewidth}}}]{% 
	  \tt {\tiny [GM$|$GM]}  
	 Strafpunkt: (6,3) ist nicht leer, sondern enthält '┘'.\\ \tt \\ \tt Ihr habt gemeinsam 6 von maximal 12 Strafpunkten angesammelt.\\ \tt Bitte versuche es noch mal! 
	  } 
	   } 
	   } 
	 & & \\ 
 

    \theutterance \stepcounter{utterance}  

    & & \multicolumn{4}{p{0.6\linewidth}}{\cellcolor[rgb]{0.9,0.9,0.9}{%
	\makecell[{{p{\linewidth}}}]{% 
	  \tt {\tiny [A$\langle$GM]}  
	 Strafpunkt: (6,3) ist nicht leer, sondern enthält '┘'.\\ \tt \\ \tt Ihr habt gemeinsam 6 von maximal 12 Strafpunkten angesammelt.\\ \tt Bitte versuche es noch mal! 
	  } 
	   } 
	   } 
	 & \\ 
 

    \theutterance \stepcounter{utterance}  

    & \multicolumn{4}{p{0.6\linewidth}}{\cellcolor[rgb]{0.8,1,0.9}{%
	\makecell[{{p{\linewidth}}}]{% 
	  \tt {\tiny [A$\rangle$GM]}  
	 VERSCHIEBE: W, (3,2) 
	  } 
	   } 
	   } 
	 & & \\ 
 

    \theutterance \stepcounter{utterance}  

    & & & \multicolumn{2}{p{0.3\linewidth}}{\cellcolor[rgb]{0.95,0.95,0.95}{%
	\makecell[{{p{\linewidth}}}]{% 
	  \tt {\tiny [GM$|$GM]}  
	 VERSCHIEBE: W, (3,2) 
	  } 
	   } 
	   } 
	 & & \\ 
 

    \theutterance \stepcounter{utterance}  

    & & & \multicolumn{2}{p{0.3\linewidth}}{\cellcolor[rgb]{0.95,0.95,0.95}{%
	\makecell[{{p{\linewidth}}}]{% 
	  \tt {\tiny [GM$|$GM]}  
	 'W' erfolgreich nach (3,2) verschoben. Dein Raster sieht jetzt so aus:\\ \tt ```\\ \tt  1234567\\ \tt ╔═══╤═╤═╗\\ \tt ║◌◌◌│◌│◌║ 1\\ \tt ║◌◌W│I├─╢ 2\\ \tt ║◌C◌├─┘◌║ 3\\ \tt ║◌◌◌│◌◌◌║ 4\\ \tt ╟─┬─┼───╢ 5\\ \tt ║◌│◌│◌◌T║ 6\\ \tt ║◌│◌│◌◌H║ 7\\ \tt ╚═╧═╧═══╝\\ \tt ```\\ \tt  
	  } 
	   } 
	   } 
	 & & \\ 
 

    \theutterance \stepcounter{utterance}  

    & & \multicolumn{4}{p{0.6\linewidth}}{\cellcolor[rgb]{0.9,0.9,0.9}{%
	\makecell[{{p{\linewidth}}}]{% 
	  \tt {\tiny [GM$\rangle$B]}  
	 'I' erfolgreich nach (5,2) verschoben. Dein Raster sieht jetzt so aus:\\ \tt ```\\ \tt  1234567\\ \tt ╔═══╤═╤═╗\\ \tt ║T◌◌│C│◌║ 1\\ \tt ║◌◌W│I├─╢ 2\\ \tt ║◌◌◌├─┘◌║ 3\\ \tt ║◌◌◌│◌◌H║ 4\\ \tt ╟─┬─┼───╢ 5\\ \tt ║◌│◌│◌◌◌║ 6\\ \tt ║◌│◌│◌◌◌║ 7\\ \tt ╚═╧═╧═══╝\\ \tt ```\\ \tt Ihr seid gerade bei Runde 10 von maximal 20.\\ \tt Ihr habt gemeinsam 6 von maximal 12 Strafpunkten angesammelt.\\ \tt Dein Mitspieler hat ein Objekt auf seinem Raster bewegt.\\ \tt \\ \tt \\ \tt Wie lautet dein nächster Befehl? 
	  } 
	   } 
	   } 
	 & \\ 
 

    \theutterance \stepcounter{utterance}  

    & & & \multicolumn{4}{p{0.6\linewidth}}{\cellcolor[rgb]{1,0.85,0.72}{%
	\makecell[{{p{\linewidth}}}]{% 
	  \tt {\tiny [GM$\langle$B]}  
	 `SPRICH: Ich habe 'I' erfolgreich nach (5,2) verschoben. Wie sieht dein aktuelles Raster aus? Hast du ein weiteres Objekt bewegt? Bitte gib mir ein Update, damit wir unsere nächsten Schritte koordinieren können.` 
	  } 
	   } 
	   } 
	  \\ 
 

    \theutterance \stepcounter{utterance}  

    & & & \multicolumn{2}{p{0.3\linewidth}}{\cellcolor[rgb]{0.95,0.95,0.95}{%
	\makecell[{{p{\linewidth}}}]{% 
	  \tt {\tiny [GM$|$GM]}  
	 `SPRICH: Ich habe 'I' erfolgreich nach (5,2) verschoben. Wie sieht dein aktuelles Raster aus? Hast du ein weiteres Objekt bewegt? Bitte gib mir ein Update, damit wir unsere nächsten Schritte koordinieren können.` 
	  } 
	   } 
	   } 
	 & & \\ 
 

    \theutterance \stepcounter{utterance}  

    & & \multicolumn{4}{p{0.6\linewidth}}{\cellcolor[rgb]{0.9,0.9,0.9}{%
	\makecell[{{p{\linewidth}}}]{% 
	  \tt {\tiny [A$\langle$GM]}  
	 'W' erfolgreich nach (3,2) verschoben. Dein Raster sieht jetzt so aus:\\ \tt ```\\ \tt  1234567\\ \tt ╔═══╤═╤═╗\\ \tt ║◌◌◌│◌│◌║ 1\\ \tt ║◌◌W│I├─╢ 2\\ \tt ║◌C◌├─┘◌║ 3\\ \tt ║◌◌◌│◌◌◌║ 4\\ \tt ╟─┬─┼───╢ 5\\ \tt ║◌│◌│◌◌T║ 6\\ \tt ║◌│◌│◌◌H║ 7\\ \tt ╚═╧═╧═══╝\\ \tt ```\\ \tt Ihr seid gerade bei Runde 10 von maximal 20.\\ \tt Ihr habt gemeinsam 6 von maximal 12 Strafpunkten angesammelt.\\ \tt Dein Mitspieler hat dir diese Nachricht geschickt:\\ \tt \\ \tt \\ \tt "Ich habe 'I' erfolgreich nach (5,2) verschoben. Wie sieht dein aktuelles Raster aus? Hast du ein weiteres Objekt bewegt? Bitte gib mir ein Update, damit wir unsere nächsten Schritte koordinieren können."\\ \tt \\ \tt \\ \tt Wie lautet dein nächster Befehl? 
	  } 
	   } 
	   } 
	 & \\ 
 

    \theutterance \stepcounter{utterance}  

    & \multicolumn{4}{p{0.6\linewidth}}{\cellcolor[rgb]{0.8,1,0.9}{%
	\makecell[{{p{\linewidth}}}]{% 
	  \tt {\tiny [A$\rangle$GM]}  
	 SPRICH: Ja, ich habe 'W' erfolgreich nach (3,2) verschoben. Mein aktuelles Raster sieht so aus. 
	  } 
	   } 
	   } 
	 & & \\ 
 

    \theutterance \stepcounter{utterance}  

    & & & \multicolumn{2}{p{0.3\linewidth}}{\cellcolor[rgb]{0.95,0.95,0.95}{%
	\makecell[{{p{\linewidth}}}]{% 
	  \tt {\tiny [GM$|$GM]}  
	 SPRICH: Ja, ich habe 'W' erfolgreich nach (3,2) verschoben. Mein aktuelles Raster sieht so aus. 
	  } 
	   } 
	   } 
	 & & \\ 
 

    \theutterance \stepcounter{utterance}  

    & & \multicolumn{4}{p{0.6\linewidth}}{\cellcolor[rgb]{0.9,0.9,0.9}{%
	\makecell[{{p{\linewidth}}}]{% 
	  \tt {\tiny [GM$\rangle$B]}  
	 Deine Nachricht wurde an deinen Mitspieler übermittelt.\\ \tt \\ \tt \\ \tt Ihr seid gerade bei Runde 11 von maximal 20.\\ \tt Ihr habt gemeinsam 6 von maximal 12 Strafpunkten angesammelt.\\ \tt Dein Mitspieler hat dir diese Nachricht geschickt:\\ \tt \\ \tt \\ \tt "Ja, ich habe 'W' erfolgreich nach (3,2) verschoben. Mein aktuelles Raster sieht so aus."\\ \tt \\ \tt \\ \tt Wie lautet dein nächster Befehl? 
	  } 
	   } 
	   } 
	 & \\ 
 

    \theutterance \stepcounter{utterance}  

    & & & \multicolumn{4}{p{0.6\linewidth}}{\cellcolor[rgb]{1,0.85,0.72}{%
	\makecell[{{p{\linewidth}}}]{% 
	  \tt {\tiny [GM$\langle$B]}  
	 `SPRICH: Ich denke, wir sind auf dem richtigen Weg! Als nächstes möchte ich 'H' nach (6,3) bewegen, um unserer Zielposition näher zu kommen. Bitte gib mir ein Update, wenn du bereit bist, dein nächstes Objekt zu bewegen.` 
	  } 
	   } 
	   } 
	  \\ 
 

    \theutterance \stepcounter{utterance}  

    & & & \multicolumn{2}{p{0.3\linewidth}}{\cellcolor[rgb]{0.95,0.95,0.95}{%
	\makecell[{{p{\linewidth}}}]{% 
	  \tt {\tiny [GM$|$GM]}  
	 `SPRICH: Ich denke, wir sind auf dem richtigen Weg! Als nächstes möchte ich 'H' nach (6,3) bewegen, um unserer Zielposition näher zu kommen. Bitte gib mir ein Update, wenn du bereit bist, dein nächstes Objekt zu bewegen.` 
	  } 
	   } 
	   } 
	 & & \\ 
 

    \theutterance \stepcounter{utterance}  

    & & \multicolumn{4}{p{0.6\linewidth}}{\cellcolor[rgb]{0.9,0.9,0.9}{%
	\makecell[{{p{\linewidth}}}]{% 
	  \tt {\tiny [A$\langle$GM]}  
	 Deine Nachricht wurde an deinen Mitspieler übermittelt.\\ \tt \\ \tt \\ \tt Ihr seid gerade bei Runde 11 von maximal 20.\\ \tt Ihr habt gemeinsam 6 von maximal 12 Strafpunkten angesammelt.\\ \tt Dein Mitspieler hat dir diese Nachricht geschickt:\\ \tt \\ \tt \\ \tt "Ich denke, wir sind auf dem richtigen Weg! Als nächstes möchte ich 'H' nach (6,3) bewegen, um unserer Zielposition näher zu kommen. Bitte gib mir ein Update, wenn du bereit bist, dein nächstes Objekt zu bewegen."\\ \tt \\ \tt \\ \tt Wie lautet dein nächster Befehl? 
	  } 
	   } 
	   } 
	 & \\ 
 

    \theutterance \stepcounter{utterance}  

    & \multicolumn{4}{p{0.6\linewidth}}{\cellcolor[rgb]{0.8,1,0.9}{%
	\makecell[{{p{\linewidth}}}]{% 
	  \tt {\tiny [A$\rangle$GM]}  
	 VERSCHIEBE: H, (4,5) 
	  } 
	   } 
	   } 
	 & & \\ 
 

    \theutterance \stepcounter{utterance}  

    & & & \multicolumn{2}{p{0.3\linewidth}}{\cellcolor[rgb]{0.95,0.95,0.95}{%
	\makecell[{{p{\linewidth}}}]{% 
	  \tt {\tiny [GM$|$GM]}  
	 VERSCHIEBE: H, (4,5) 
	  } 
	   } 
	   } 
	 & & \\ 
 

    \theutterance \stepcounter{utterance}  

    & & & \multicolumn{2}{p{0.3\linewidth}}{\cellcolor[rgb]{0.95,0.95,0.95}{%
	\makecell[{{p{\linewidth}}}]{% 
	  \tt {\tiny [GM$|$GM]}  
	 Strafpunkt: (4,5) ist nicht leer, sondern enthält '┼'.\\ \tt \\ \tt Ihr habt gemeinsam 7 von maximal 12 Strafpunkten angesammelt.\\ \tt Bitte versuche es noch mal! 
	  } 
	   } 
	   } 
	 & & \\ 
 

    \theutterance \stepcounter{utterance}  

    & & \multicolumn{4}{p{0.6\linewidth}}{\cellcolor[rgb]{0.9,0.9,0.9}{%
	\makecell[{{p{\linewidth}}}]{% 
	  \tt {\tiny [A$\langle$GM]}  
	 Strafpunkt: (4,5) ist nicht leer, sondern enthält '┼'.\\ \tt \\ \tt Ihr habt gemeinsam 7 von maximal 12 Strafpunkten angesammelt.\\ \tt Bitte versuche es noch mal! 
	  } 
	   } 
	   } 
	 & \\ 
 

    \theutterance \stepcounter{utterance}  

    & \multicolumn{4}{p{0.6\linewidth}}{\cellcolor[rgb]{0.8,1,0.9}{%
	\makecell[{{p{\linewidth}}}]{% 
	  \tt {\tiny [A$\rangle$GM]}  
	 VERSCHIEBE: T, (4,6) 
	  } 
	   } 
	   } 
	 & & \\ 
 

    \theutterance \stepcounter{utterance}  

    & & & \multicolumn{2}{p{0.3\linewidth}}{\cellcolor[rgb]{0.95,0.95,0.95}{%
	\makecell[{{p{\linewidth}}}]{% 
	  \tt {\tiny [GM$|$GM]}  
	 VERSCHIEBE: T, (4,6) 
	  } 
	   } 
	   } 
	 & & \\ 
 

    \theutterance \stepcounter{utterance}  

    & & & \multicolumn{2}{p{0.3\linewidth}}{\cellcolor[rgb]{0.95,0.95,0.95}{%
	\makecell[{{p{\linewidth}}}]{% 
	  \tt {\tiny [GM$|$GM]}  
	 Strafpunkt: (4,6) ist nicht leer, sondern enthält '│'.\\ \tt \\ \tt Ihr habt gemeinsam 8 von maximal 12 Strafpunkten angesammelt.\\ \tt Bitte versuche es noch mal! 
	  } 
	   } 
	   } 
	 & & \\ 
 

    \theutterance \stepcounter{utterance}  

    & & \multicolumn{4}{p{0.6\linewidth}}{\cellcolor[rgb]{0.9,0.9,0.9}{%
	\makecell[{{p{\linewidth}}}]{% 
	  \tt {\tiny [A$\langle$GM]}  
	 Strafpunkt: (4,6) ist nicht leer, sondern enthält '│'.\\ \tt \\ \tt Ihr habt gemeinsam 8 von maximal 12 Strafpunkten angesammelt.\\ \tt Bitte versuche es noch mal! 
	  } 
	   } 
	   } 
	 & \\ 
 

    \theutterance \stepcounter{utterance}  

    & \multicolumn{4}{p{0.6\linewidth}}{\cellcolor[rgb]{0.8,1,0.9}{%
	\makecell[{{p{\linewidth}}}]{% 
	  \tt {\tiny [A$\rangle$GM]}  
	 VERSCHIEBE: H, (6,7) 
	  } 
	   } 
	   } 
	 & & \\ 
 

    \theutterance \stepcounter{utterance}  

    & & & \multicolumn{2}{p{0.3\linewidth}}{\cellcolor[rgb]{0.95,0.95,0.95}{%
	\makecell[{{p{\linewidth}}}]{% 
	  \tt {\tiny [GM$|$GM]}  
	 VERSCHIEBE: H, (6,7) 
	  } 
	   } 
	   } 
	 & & \\ 
 

    \theutterance \stepcounter{utterance}  

    & & & \multicolumn{2}{p{0.3\linewidth}}{\cellcolor[rgb]{0.95,0.95,0.95}{%
	\makecell[{{p{\linewidth}}}]{% 
	  \tt {\tiny [GM$|$GM]}  
	 'H' erfolgreich nach (6,7) verschoben. Dein Raster sieht jetzt so aus:\\ \tt ```\\ \tt  1234567\\ \tt ╔═══╤═╤═╗\\ \tt ║◌◌◌│◌│◌║ 1\\ \tt ║◌◌W│I├─╢ 2\\ \tt ║◌C◌├─┘◌║ 3\\ \tt ║◌◌◌│◌◌◌║ 4\\ \tt ╟─┬─┼───╢ 5\\ \tt ║◌│◌│◌◌T║ 6\\ \tt ║◌│◌│◌H◌║ 7\\ \tt ╚═╧═╧═══╝\\ \tt ```\\ \tt  
	  } 
	   } 
	   } 
	 & & \\ 
 

    \theutterance \stepcounter{utterance}  

    & & \multicolumn{4}{p{0.6\linewidth}}{\cellcolor[rgb]{0.9,0.9,0.9}{%
	\makecell[{{p{\linewidth}}}]{% 
	  \tt {\tiny [GM$\rangle$B]}  
	 Deine Nachricht wurde an deinen Mitspieler übermittelt.\\ \tt \\ \tt \\ \tt Ihr seid gerade bei Runde 12 von maximal 20.\\ \tt Ihr habt gemeinsam 8 von maximal 12 Strafpunkten angesammelt.\\ \tt Dein Mitspieler hat ein Objekt auf seinem Raster bewegt.\\ \tt \\ \tt \\ \tt Wie lautet dein nächster Befehl? 
	  } 
	   } 
	   } 
	 & \\ 
 

    \theutterance \stepcounter{utterance}  

    & & & \multicolumn{4}{p{0.6\linewidth}}{\cellcolor[rgb]{1,0.85,0.72}{%
	\makecell[{{p{\linewidth}}}]{% 
	  \tt {\tiny [GM$\langle$B]}  
	 `VERSCHIEBE: H, (6,3)` 
	  } 
	   } 
	   } 
	  \\ 
 

    \theutterance \stepcounter{utterance}  

    & & & \multicolumn{2}{p{0.3\linewidth}}{\cellcolor[rgb]{0.95,0.95,0.95}{%
	\makecell[{{p{\linewidth}}}]{% 
	  \tt {\tiny [GM$|$GM]}  
	 `VERSCHIEBE: H, (6,3)` 
	  } 
	   } 
	   } 
	 & & \\ 
 

    \theutterance \stepcounter{utterance}  

    & & & \multicolumn{2}{p{0.3\linewidth}}{\cellcolor[rgb]{0.95,0.95,0.95}{%
	\makecell[{{p{\linewidth}}}]{% 
	  \tt {\tiny [GM$|$GM]}  
	 Strafpunkt: (6,3) ist nicht leer, sondern enthält '┘'.\\ \tt \\ \tt Ihr habt gemeinsam 9 von maximal 12 Strafpunkten angesammelt.\\ \tt Bitte versuche es noch mal! 
	  } 
	   } 
	   } 
	 & & \\ 
 

    \theutterance \stepcounter{utterance}  

    & & \multicolumn{4}{p{0.6\linewidth}}{\cellcolor[rgb]{0.9,0.9,0.9}{%
	\makecell[{{p{\linewidth}}}]{% 
	  \tt {\tiny [GM$\rangle$B]}  
	 Strafpunkt: (6,3) ist nicht leer, sondern enthält '┘'.\\ \tt \\ \tt Ihr habt gemeinsam 9 von maximal 12 Strafpunkten angesammelt.\\ \tt Bitte versuche es noch mal! 
	  } 
	   } 
	   } 
	 & \\ 
 

    \theutterance \stepcounter{utterance}  

    & & & \multicolumn{4}{p{0.6\linewidth}}{\cellcolor[rgb]{1,0.85,0.72}{%
	\makecell[{{p{\linewidth}}}]{% 
	  \tt {\tiny [GM$\langle$B]}  
	 `VERSCHIEBE: H, (6,4)` 
	  } 
	   } 
	   } 
	  \\ 
 

    \theutterance \stepcounter{utterance}  

    & & & \multicolumn{2}{p{0.3\linewidth}}{\cellcolor[rgb]{0.95,0.95,0.95}{%
	\makecell[{{p{\linewidth}}}]{% 
	  \tt {\tiny [GM$|$GM]}  
	 `VERSCHIEBE: H, (6,4)` 
	  } 
	   } 
	   } 
	 & & \\ 
 

    \theutterance \stepcounter{utterance}  

    & & & \multicolumn{2}{p{0.3\linewidth}}{\cellcolor[rgb]{0.95,0.95,0.95}{%
	\makecell[{{p{\linewidth}}}]{% 
	  \tt {\tiny [GM$|$GM]}  
	 'H' erfolgreich nach (6,4) verschoben. Dein Raster sieht jetzt so aus:\\ \tt ```\\ \tt  1234567\\ \tt ╔═══╤═╤═╗\\ \tt ║T◌◌│C│◌║ 1\\ \tt ║◌◌W│I├─╢ 2\\ \tt ║◌◌◌├─┘◌║ 3\\ \tt ║◌◌◌│◌H◌║ 4\\ \tt ╟─┬─┼───╢ 5\\ \tt ║◌│◌│◌◌◌║ 6\\ \tt ║◌│◌│◌◌◌║ 7\\ \tt ╚═╧═╧═══╝\\ \tt ```\\ \tt  
	  } 
	   } 
	   } 
	 & & \\ 
 

    \theutterance \stepcounter{utterance}  

    & & \multicolumn{4}{p{0.6\linewidth}}{\cellcolor[rgb]{0.9,0.9,0.9}{%
	\makecell[{{p{\linewidth}}}]{% 
	  \tt {\tiny [A$\langle$GM]}  
	 'H' erfolgreich nach (6,7) verschoben. Dein Raster sieht jetzt so aus:\\ \tt ```\\ \tt  1234567\\ \tt ╔═══╤═╤═╗\\ \tt ║◌◌◌│◌│◌║ 1\\ \tt ║◌◌W│I├─╢ 2\\ \tt ║◌C◌├─┘◌║ 3\\ \tt ║◌◌◌│◌◌◌║ 4\\ \tt ╟─┬─┼───╢ 5\\ \tt ║◌│◌│◌◌T║ 6\\ \tt ║◌│◌│◌H◌║ 7\\ \tt ╚═╧═╧═══╝\\ \tt ```\\ \tt Ihr seid gerade bei Runde 12 von maximal 20.\\ \tt Ihr habt gemeinsam 9 von maximal 12 Strafpunkten angesammelt.\\ \tt Dein Mitspieler hat ein Objekt auf seinem Raster bewegt.\\ \tt \\ \tt \\ \tt Wie lautet dein nächster Befehl? 
	  } 
	   } 
	   } 
	 & \\ 
 

    \theutterance \stepcounter{utterance}  

    & \multicolumn{4}{p{0.6\linewidth}}{\cellcolor[rgb]{0.8,1,0.9}{%
	\makecell[{{p{\linewidth}}}]{% 
	  \tt {\tiny [A$\rangle$GM]}  
	 SPRICH: Ich denke, wir sollten unsere Fortschritte überprüfen und entscheiden, ob wir fertig sind oder weitere Anpassungen vornehmen müssen. Ich denke, wir sind nah dran, aber ich möchte sicherstellen, dass wir unsere Zielpositionen erreicht haben. Kannst du mir bitte ein Update geben, wie dein Raster aussieht und ob du glaubst, dass wir fertig sind? 
	  } 
	   } 
	   } 
	 & & \\ 
 

    \theutterance \stepcounter{utterance}  

    & & & \multicolumn{2}{p{0.3\linewidth}}{\cellcolor[rgb]{0.95,0.95,0.95}{%
	\makecell[{{p{\linewidth}}}]{% 
	  \tt {\tiny [GM$|$GM]}  
	 SPRICH: Ich denke, wir sollten unsere Fortschritte überprüfen und entscheiden, ob wir fertig sind oder weitere Anpassungen vornehmen müssen. Ich denke, wir sind nah dran, aber ich möchte sicherstellen, dass wir unsere Zielpositionen erreicht haben. Kannst du mir bitte ein Update geben, wie dein Raster aussieht und ob du glaubst, dass wir fertig sind? 
	  } 
	   } 
	   } 
	 & & \\ 
 

    \theutterance \stepcounter{utterance}  

    & & \multicolumn{4}{p{0.6\linewidth}}{\cellcolor[rgb]{0.9,0.9,0.9}{%
	\makecell[{{p{\linewidth}}}]{% 
	  \tt {\tiny [GM$\rangle$B]}  
	 'H' erfolgreich nach (6,4) verschoben. Dein Raster sieht jetzt so aus:\\ \tt ```\\ \tt  1234567\\ \tt ╔═══╤═╤═╗\\ \tt ║T◌◌│C│◌║ 1\\ \tt ║◌◌W│I├─╢ 2\\ \tt ║◌◌◌├─┘◌║ 3\\ \tt ║◌◌◌│◌H◌║ 4\\ \tt ╟─┬─┼───╢ 5\\ \tt ║◌│◌│◌◌◌║ 6\\ \tt ║◌│◌│◌◌◌║ 7\\ \tt ╚═╧═╧═══╝\\ \tt ```\\ \tt Ihr seid gerade bei Runde 13 von maximal 20.\\ \tt Ihr habt gemeinsam 9 von maximal 12 Strafpunkten angesammelt.\\ \tt Dein Mitspieler hat dir diese Nachricht geschickt:\\ \tt \\ \tt \\ \tt "Ich denke, wir sollten unsere Fortschritte überprüfen und entscheiden, ob wir fertig sind oder weitere Anpassungen vornehmen müssen. Ich denke, wir sind nah dran, aber ich möchte sicherstellen, dass wir unsere Zielpositionen erreicht haben. Kannst du mir bitte ein Update geben, wie dein Raster aussieht und ob du glaubst, dass wir fertig sind?"\\ \tt \\ \tt \\ \tt Wie lautet dein nächster Befehl? 
	  } 
	   } 
	   } 
	 & \\ 
 

    \theutterance \stepcounter{utterance}  

    & & & \multicolumn{4}{p{0.6\linewidth}}{\cellcolor[rgb]{1,0.85,0.72}{%
	\makecell[{{p{\linewidth}}}]{% 
	  \tt {\tiny [GM$\langle$B]}  
	 `SPRICH: Mein aktuelles Raster sieht so aus. Ich denke, wir sind nah dran, aber ich bin nicht sicher, ob wir alle Zielpositionen erreicht haben. Ich denke, wir sollten noch einmal unsere Zielpositionen überprüfen und entscheiden, ob wir weitere Anpassungen vornehmen müssen. Ich schlage vor, dass wir noch eine Runde spielen und dann entscheiden, ob wir fertig sind. Wenn du damit einverstanden bist, kann ich 'C' nach (2,3) bewegen, um unsere Zielposition zu erreichen. Was meinst du?` 
	  } 
	   } 
	   } 
	  \\ 
 

    \theutterance \stepcounter{utterance}  

    & & & \multicolumn{2}{p{0.3\linewidth}}{\cellcolor[rgb]{0.95,0.95,0.95}{%
	\makecell[{{p{\linewidth}}}]{% 
	  \tt {\tiny [GM$|$GM]}  
	 `SPRICH: Mein aktuelles Raster sieht so aus. Ich denke, wir sind nah dran, aber ich bin nicht sicher, ob wir alle Zielpositionen erreicht haben. Ich denke, wir sollten noch einmal unsere Zielpositionen überprüfen und entscheiden, ob wir weitere Anpassungen vornehmen müssen. Ich schlage vor, dass wir noch eine Runde spielen und dann entscheiden, ob wir fertig sind. Wenn du damit einverstanden bist, kann ich 'C' nach (2,3) bewegen, um unsere Zielposition zu erreichen. Was meinst du?` 
	  } 
	   } 
	   } 
	 & & \\ 
 

    \theutterance \stepcounter{utterance}  

    & & \multicolumn{4}{p{0.6\linewidth}}{\cellcolor[rgb]{0.9,0.9,0.9}{%
	\makecell[{{p{\linewidth}}}]{% 
	  \tt {\tiny [A$\langle$GM]}  
	 Deine Nachricht wurde an deinen Mitspieler übermittelt.\\ \tt \\ \tt \\ \tt Ihr seid gerade bei Runde 13 von maximal 20.\\ \tt Ihr habt gemeinsam 9 von maximal 12 Strafpunkten angesammelt.\\ \tt Dein Mitspieler hat dir diese Nachricht geschickt:\\ \tt \\ \tt \\ \tt "Mein aktuelles Raster sieht so aus. Ich denke, wir sind nah dran, aber ich bin nicht sicher, ob wir alle Zielpositionen erreicht haben. Ich denke, wir sollten noch einmal unsere Zielpositionen überprüfen und entscheiden, ob wir weitere Anpassungen vornehmen müssen. Ich schlage vor, dass wir noch eine Runde spielen und dann entscheiden, ob wir fertig sind. Wenn du damit einverstanden bist, kann ich 'C' nach (2,3) bewegen, um unsere Zielposition zu erreichen. Was meinst du?"\\ \tt \\ \tt \\ \tt Wie lautet dein nächster Befehl? 
	  } 
	   } 
	   } 
	 & \\ 
 

    \theutterance \stepcounter{utterance}  

    & \multicolumn{4}{p{0.6\linewidth}}{\cellcolor[rgb]{0.8,1,0.9}{%
	\makecell[{{p{\linewidth}}}]{% 
	  \tt {\tiny [A$\rangle$GM]}  
	 SPRICH: Ich denke, das ist eine gute Idee. Lass uns noch eine Runde spielen und unsere Zielpositionen überprüfen. Ich bin einverstanden, dass du 'C' nach (2,3) bewegst. Ich werde mich auf mein nächstes Objekt konzentrieren. Bitte gib mir ein Update, wenn du bereit bist. 
	  } 
	   } 
	   } 
	 & & \\ 
 

    \theutterance \stepcounter{utterance}  

    & & & \multicolumn{2}{p{0.3\linewidth}}{\cellcolor[rgb]{0.95,0.95,0.95}{%
	\makecell[{{p{\linewidth}}}]{% 
	  \tt {\tiny [GM$|$GM]}  
	 SPRICH: Ich denke, das ist eine gute Idee. Lass uns noch eine Runde spielen und unsere Zielpositionen überprüfen. Ich bin einverstanden, dass du 'C' nach (2,3) bewegst. Ich werde mich auf mein nächstes Objekt konzentrieren. Bitte gib mir ein Update, wenn du bereit bist. 
	  } 
	   } 
	   } 
	 & & \\ 
 

    \theutterance \stepcounter{utterance}  

    & & \multicolumn{4}{p{0.6\linewidth}}{\cellcolor[rgb]{0.9,0.9,0.9}{%
	\makecell[{{p{\linewidth}}}]{% 
	  \tt {\tiny [GM$\rangle$B]}  
	 Deine Nachricht wurde an deinen Mitspieler übermittelt.\\ \tt \\ \tt \\ \tt Ihr seid gerade bei Runde 14 von maximal 20.\\ \tt Ihr habt gemeinsam 9 von maximal 12 Strafpunkten angesammelt.\\ \tt Dein Mitspieler hat dir diese Nachricht geschickt:\\ \tt \\ \tt \\ \tt "Ich denke, das ist eine gute Idee. Lass uns noch eine Runde spielen und unsere Zielpositionen überprüfen. Ich bin einverstanden, dass du 'C' nach (2,3) bewegst. Ich werde mich auf mein nächstes Objekt konzentrieren. Bitte gib mir ein Update, wenn du bereit bist."\\ \tt \\ \tt \\ \tt Wie lautet dein nächster Befehl? 
	  } 
	   } 
	   } 
	 & \\ 
 

    \theutterance \stepcounter{utterance}  

    & & & \multicolumn{4}{p{0.6\linewidth}}{\cellcolor[rgb]{1,0.85,0.72}{%
	\makecell[{{p{\linewidth}}}]{% 
	  \tt {\tiny [GM$\langle$B]}  
	 `VERSCHIEBE: C, (2,3)` 
	  } 
	   } 
	   } 
	  \\ 
 

    \theutterance \stepcounter{utterance}  

    & & & \multicolumn{2}{p{0.3\linewidth}}{\cellcolor[rgb]{0.95,0.95,0.95}{%
	\makecell[{{p{\linewidth}}}]{% 
	  \tt {\tiny [GM$|$GM]}  
	 `VERSCHIEBE: C, (2,3)` 
	  } 
	   } 
	   } 
	 & & \\ 
 

    \theutterance \stepcounter{utterance}  

    & & & \multicolumn{2}{p{0.3\linewidth}}{\cellcolor[rgb]{0.95,0.95,0.95}{%
	\makecell[{{p{\linewidth}}}]{% 
	  \tt {\tiny [GM$|$GM]}  
	 'C' erfolgreich nach (2,3) verschoben. Dein Raster sieht jetzt so aus:\\ \tt ```\\ \tt  1234567\\ \tt ╔═══╤═╤═╗\\ \tt ║T◌◌│◌│◌║ 1\\ \tt ║◌◌W│I├─╢ 2\\ \tt ║◌C◌├─┘◌║ 3\\ \tt ║◌◌◌│◌H◌║ 4\\ \tt ╟─┬─┼───╢ 5\\ \tt ║◌│◌│◌◌◌║ 6\\ \tt ║◌│◌│◌◌◌║ 7\\ \tt ╚═╧═╧═══╝\\ \tt ```\\ \tt  
	  } 
	   } 
	   } 
	 & & \\ 
 

    \theutterance \stepcounter{utterance}  

    & & \multicolumn{4}{p{0.6\linewidth}}{\cellcolor[rgb]{0.9,0.9,0.9}{%
	\makecell[{{p{\linewidth}}}]{% 
	  \tt {\tiny [A$\langle$GM]}  
	 Deine Nachricht wurde an deinen Mitspieler übermittelt.\\ \tt \\ \tt \\ \tt Ihr seid gerade bei Runde 14 von maximal 20.\\ \tt Ihr habt gemeinsam 9 von maximal 12 Strafpunkten angesammelt.\\ \tt Dein Mitspieler hat ein Objekt auf seinem Raster bewegt.\\ \tt \\ \tt \\ \tt Wie lautet dein nächster Befehl? 
	  } 
	   } 
	   } 
	 & \\ 
 

    \theutterance \stepcounter{utterance}  

    & \multicolumn{4}{p{0.6\linewidth}}{\cellcolor[rgb]{0.8,1,0.9}{%
	\makecell[{{p{\linewidth}}}]{% 
	  \tt {\tiny [A$\rangle$GM]}  
	 SPRICH: Ich denke, wir sollten jetzt überprüfen, ob wir unsere Zielpositionen erreicht haben. Ich schlage vor, dass wir unsere Raster vergleichen und entscheiden, ob wir fertig sind. Wenn du bereit bist, kann ich dir meine endgültige Rasterkonfiguration mitteilen. 
	  } 
	   } 
	   } 
	 & & \\ 
 

    \theutterance \stepcounter{utterance}  

    & & & \multicolumn{2}{p{0.3\linewidth}}{\cellcolor[rgb]{0.95,0.95,0.95}{%
	\makecell[{{p{\linewidth}}}]{% 
	  \tt {\tiny [GM$|$GM]}  
	 SPRICH: Ich denke, wir sollten jetzt überprüfen, ob wir unsere Zielpositionen erreicht haben. Ich schlage vor, dass wir unsere Raster vergleichen und entscheiden, ob wir fertig sind. Wenn du bereit bist, kann ich dir meine endgültige Rasterkonfiguration mitteilen. 
	  } 
	   } 
	   } 
	 & & \\ 
 

    \theutterance \stepcounter{utterance}  

    & & \multicolumn{4}{p{0.6\linewidth}}{\cellcolor[rgb]{0.9,0.9,0.9}{%
	\makecell[{{p{\linewidth}}}]{% 
	  \tt {\tiny [GM$\rangle$B]}  
	 'C' erfolgreich nach (2,3) verschoben. Dein Raster sieht jetzt so aus:\\ \tt ```\\ \tt  1234567\\ \tt ╔═══╤═╤═╗\\ \tt ║T◌◌│◌│◌║ 1\\ \tt ║◌◌W│I├─╢ 2\\ \tt ║◌C◌├─┘◌║ 3\\ \tt ║◌◌◌│◌H◌║ 4\\ \tt ╟─┬─┼───╢ 5\\ \tt ║◌│◌│◌◌◌║ 6\\ \tt ║◌│◌│◌◌◌║ 7\\ \tt ╚═╧═╧═══╝\\ \tt ```\\ \tt Ihr seid gerade bei Runde 15 von maximal 20.\\ \tt Ihr habt gemeinsam 9 von maximal 12 Strafpunkten angesammelt.\\ \tt Dein Mitspieler hat dir diese Nachricht geschickt:\\ \tt \\ \tt \\ \tt "Ich denke, wir sollten jetzt überprüfen, ob wir unsere Zielpositionen erreicht haben. Ich schlage vor, dass wir unsere Raster vergleichen und entscheiden, ob wir fertig sind. Wenn du bereit bist, kann ich dir meine endgültige Rasterkonfiguration mitteilen."\\ \tt \\ \tt \\ \tt Wie lautet dein nächster Befehl? 
	  } 
	   } 
	   } 
	 & \\ 
 

    \theutterance \stepcounter{utterance}  

    & & & \multicolumn{4}{p{0.6\linewidth}}{\cellcolor[rgb]{1,0.85,0.72}{%
	\makecell[{{p{\linewidth}}}]{% 
	  \tt {\tiny [GM$\langle$B]}  
	 `SPRICH: Ja, ich bin bereit! Ich denke, wir sollten unsere Raster vergleichen und entscheiden, ob wir fertig sind. Bitte teile mir deine endgültige Rasterkonfiguration mit, und ich werde meine mitteilen. Dann können wir entscheiden, ob wir fertig sind oder weitere Anpassungen vornehmen müssen. Ich denke, wir sind nah dran, und ich bin optimistisch, dass wir es geschafft haben!` 
	  } 
	   } 
	   } 
	  \\ 
 

    \theutterance \stepcounter{utterance}  

    & & & \multicolumn{2}{p{0.3\linewidth}}{\cellcolor[rgb]{0.95,0.95,0.95}{%
	\makecell[{{p{\linewidth}}}]{% 
	  \tt {\tiny [GM$|$GM]}  
	 `SPRICH: Ja, ich bin bereit! Ich denke, wir sollten unsere Raster vergleichen und entscheiden, ob wir fertig sind. Bitte teile mir deine endgültige Rasterkonfiguration mit, und ich werde meine mitteilen. Dann können wir entscheiden, ob wir fertig sind oder weitere Anpassungen vornehmen müssen. Ich denke, wir sind nah dran, und ich bin optimistisch, dass wir es geschafft haben!` 
	  } 
	   } 
	   } 
	 & & \\ 
 

    \theutterance \stepcounter{utterance}  

    & & \multicolumn{4}{p{0.6\linewidth}}{\cellcolor[rgb]{0.9,0.9,0.9}{%
	\makecell[{{p{\linewidth}}}]{% 
	  \tt {\tiny [A$\langle$GM]}  
	 Deine Nachricht wurde an deinen Mitspieler übermittelt.\\ \tt \\ \tt \\ \tt Ihr seid gerade bei Runde 15 von maximal 20.\\ \tt Ihr habt gemeinsam 9 von maximal 12 Strafpunkten angesammelt.\\ \tt Dein Mitspieler hat dir diese Nachricht geschickt:\\ \tt \\ \tt \\ \tt "Ja, ich bin bereit! Ich denke, wir sollten unsere Raster vergleichen und entscheiden, ob wir fertig sind. Bitte teile mir deine endgültige Rasterkonfiguration mit, und ich werde meine mitteilen. Dann können wir entscheiden, ob wir fertig sind oder weitere Anpassungen vornehmen müssen. Ich denke, wir sind nah dran, und ich bin optimistisch, dass wir es geschafft haben!"\\ \tt \\ \tt \\ \tt Wie lautet dein nächster Befehl? 
	  } 
	   } 
	   } 
	 & \\ 
 

    \theutterance \stepcounter{utterance}  

    & \multicolumn{4}{p{0.6\linewidth}}{\cellcolor[rgb]{0.8,1,0.9}{%
	\makecell[{{p{\linewidth}}}]{% 
	  \tt {\tiny [A$\rangle$GM]}  
	 SPRICH: Ich denke, wir sind fertig! Mein endgültiges Raster sieht so aus: \\ \tt ```\\ \tt  1234567\\ \tt ╔═══╤═╤═╗\\ \tt ║◌◌◌│◌│◌║ 1\\ \tt ║◌◌W│I├─╢ 2\\ \tt ║◌C◌├─┘◌║ 3\\ \tt ║◌◌◌│◌◌◌║ 4\\ \tt ╟─┬─┼───╢ 5\\ \tt ║◌│◌│◌◌T║ 6\\ \tt ║◌│◌│◌H◌║ 7\\ \tt ╚═╧═╧═══╝\\ \tt ```\\ \tt Ich denke, wir haben unsere Zielpositionen erreicht. Bitte überprüfe dein Raster und teile mir mit, ob du auch fertig bist. Wenn ja, können wir das Spiel beenden. 
	  } 
	   } 
	   } 
	 & & \\ 
 

    \theutterance \stepcounter{utterance}  

    & & & \multicolumn{2}{p{0.3\linewidth}}{\cellcolor[rgb]{0.95,0.95,0.95}{%
	\makecell[{{p{\linewidth}}}]{% 
	  \tt {\tiny [GM$|$GM]}  
	 SPRICH: Ich denke, wir sind fertig! Mein endgültiges Raster sieht so aus: \\ \tt ```\\ \tt  1234567\\ \tt ╔═══╤═╤═╗\\ \tt ║◌◌◌│◌│◌║ 1\\ \tt ║◌◌W│I├─╢ 2\\ \tt ║◌C◌├─┘◌║ 3\\ \tt ║◌◌◌│◌◌◌║ 4\\ \tt ╟─┬─┼───╢ 5\\ \tt ║◌│◌│◌◌T║ 6\\ \tt ║◌│◌│◌H◌║ 7\\ \tt ╚═╧═╧═══╝\\ \tt ```\\ \tt Ich denke, wir haben unsere Zielpositionen erreicht. Bitte überprüfe dein Raster und teile mir mit, ob du auch fertig bist. Wenn ja, können wir das Spiel beenden. 
	  } 
	   } 
	   } 
	 & & \\ 
 

    \theutterance \stepcounter{utterance}  

    & & & \multicolumn{2}{p{0.3\linewidth}}{\cellcolor[rgb]{0.95,0.95,0.95}{%
	\makecell[{{p{\linewidth}}}]{% 
	  \tt {\tiny [GM$|$GM]}  
	 Invalid format: tail is not empty: '\\ \tt \\ \tt  1234567\\ \tt ╔═══╤═╤═╗\\ \tt ║◌◌◌│◌│◌║ 1\\ \tt ║◌◌W│I├─╢ 2\\ \tt ║◌C◌├─┘◌║ 3\\ \tt ║◌◌◌│◌◌◌║ 4\\ \tt ╟─┬─┼───╢ 5\\ \tt ║◌│◌│◌◌T║ 6\\ \tt ║◌│◌│◌H◌║ 7\\ \tt ╚═╧═╧═══╝\\ \tt \\ \tt Ich denke, wir haben unsere Zielpositionen erreicht. Bitte überprüfe dein Raster und teile mir mit, ob du auch fertig bist. Wenn ja, können wir das Spiel beenden. 
	  } 
	   } 
	   } 
	 & & \\ 
 

    \theutterance \stepcounter{utterance}  

    & & \multicolumn{4}{p{0.6\linewidth}}{\cellcolor[rgb]{0.9,0.9,0.9}{%
	\makecell[{{p{\linewidth}}}]{% 
	  \tt {\tiny [A$\langle$GM]}  
	 Strafpunkt: Deine Nachricht darf nichts nach dem Befehl enthalten! Stelle sicher, dass deine Antwort ausschließlich einen der beiden Befehle `SPRICH: <NACHRICHT>` und `VERSCHIEBE: <OBJECT>, (<X>, <Y>)` und sonst nichts enthält!\\ \tt \\ \tt Ihr habt gemeinsam 10 von maximal 12 Strafpunkten angesammelt.\\ \tt Bitte versuche es noch mal! 
	  } 
	   } 
	   } 
	 & \\ 
 

    \theutterance \stepcounter{utterance}  

    & \multicolumn{4}{p{0.6\linewidth}}{\cellcolor[rgb]{0.8,1,0.9}{%
	\makecell[{{p{\linewidth}}}]{% 
	  \tt {\tiny [A$\rangle$GM]}  
	 SPRICH: fertig? 
	  } 
	   } 
	   } 
	 & & \\ 
 

    \theutterance \stepcounter{utterance}  

    & & & \multicolumn{2}{p{0.3\linewidth}}{\cellcolor[rgb]{0.95,0.95,0.95}{%
	\makecell[{{p{\linewidth}}}]{% 
	  \tt {\tiny [GM$|$GM]}  
	 SPRICH: fertig? 
	  } 
	   } 
	   } 
	 & & \\ 
 

    \theutterance \stepcounter{utterance}  

    & & \multicolumn{4}{p{0.6\linewidth}}{\cellcolor[rgb]{0.9,0.9,0.9}{%
	\makecell[{{p{\linewidth}}}]{% 
	  \tt {\tiny [GM$\rangle$B]}  
	 Deine Nachricht wurde an deinen Mitspieler übermittelt.\\ \tt \\ \tt \\ \tt Ihr seid gerade bei Runde 16 von maximal 20.\\ \tt Ihr habt gemeinsam 10 von maximal 12 Strafpunkten angesammelt.\\ \tt Dein Mitspieler hat dir diese Nachricht geschickt:\\ \tt \\ \tt \\ \tt "fertig?"\\ \tt \\ \tt \\ \tt Wie lautet dein nächster Befehl? 
	  } 
	   } 
	   } 
	 & \\ 
 

    \theutterance \stepcounter{utterance}  

    & & & \multicolumn{4}{p{0.6\linewidth}}{\cellcolor[rgb]{1,0.85,0.72}{%
	\makecell[{{p{\linewidth}}}]{% 
	  \tt {\tiny [GM$\langle$B]}  
	 `SPRICH: fertig!` 
	  } 
	   } 
	   } 
	  \\ 
 

    \theutterance \stepcounter{utterance}  

    & & & \multicolumn{2}{p{0.3\linewidth}}{\cellcolor[rgb]{0.95,0.95,0.95}{%
	\makecell[{{p{\linewidth}}}]{% 
	  \tt {\tiny [GM$|$GM]}  
	 `SPRICH: fertig!` 
	  } 
	   } 
	   } 
	 & & \\ 
 

    \theutterance \stepcounter{utterance}  

    & & & \multicolumn{2}{p{0.3\linewidth}}{\cellcolor[rgb]{0.95,0.95,0.95}{%
	\makecell[{{p{\linewidth}}}]{% 
	  \tt {\tiny [GM$|$GM]}  
	 true 
	  } 
	   } 
	   } 
	 & & \\ 
 

    \theutterance \stepcounter{utterance}  

    & & & \multicolumn{2}{p{0.3\linewidth}}{\cellcolor[rgb]{0.95,0.95,0.95}{%
	\makecell[{{p{\linewidth}}}]{% 
	  \tt {\tiny [GM$|$GM]}  
	 * success: True\\ \tt * lose: False\\ \tt * aborted: False\\ \tt -------\\ \tt * Shifts: 6.00\\ \tt * Max Shifts: 8.00\\ \tt * Min Shifts: 4.00\\ \tt * End Distance Sum: 10.81\\ \tt * Init Distance Sum: 23.66\\ \tt * Expected Distance Sum: 20.95\\ \tt * Penalties: 10.00\\ \tt * Max Penalties: 12.00\\ \tt * Rounds: 16.00\\ \tt * Max Rounds: 20.00\\ \tt * Object Count: 5.00\\ \tt  
	  } 
	   } 
	   } 
	 & & \\ 
 

    \theutterance \stepcounter{utterance}  

    & & & \multicolumn{2}{p{0.3\linewidth}}{\cellcolor[rgb]{0.95,0.95,0.95}{%
	\makecell[{{p{\linewidth}}}]{% 
	  \tt {\tiny [GM$|$GM]}  
	 * Main Score: 17.00\\ \tt \\ \tt -------\\ \tt * Distance Score: 0.51\\ \tt * Consistency Score: 0.60\\ \tt * Coverage Score: 0.83\\ \tt * Penalty Score: 0.27\\ \tt * Alternative Penalty Score: 0.43\\ \tt * Alternative Main Score: 22.01\\ \tt \\ \tt -------\\ \tt * Shifts: 6.00\\ \tt * Max Shifts: 8.00\\ \tt * Min Shifts: 4.00\\ \tt * End Distance Sum: 10.81\\ \tt * Init Distance Sum: 23.66\\ \tt * Expected Distance Sum: 20.95\\ \tt * Penalties: 10.00\\ \tt * Max Penalties: 12.00\\ \tt * Rounds: 16.00\\ \tt * Max Rounds: 20.00\\ \tt * Object Count: 5.00\\ \tt  
	  } 
	   } 
	   } 
	 & & \\ 
 

\end{supertabular}
}

\end{document}
