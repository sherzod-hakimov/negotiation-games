
\documentclass{article}
\usepackage{colortbl}
\usepackage{makecell}
\usepackage{multirow}
\usepackage{supertabular}

\begin{document}

\newcounter{utterance}

\centering \large Interaction Transcript for game `clean\_up', experiment `1\_medium\_5obj\_de', episode 2 with llama{-}3.3{-}70b{-}instruct{-}t1.0.
\vspace{24pt}

{ \footnotesize  \setcounter{utterance}{1}
\setlength{\tabcolsep}{0pt}
\begin{supertabular}{c@{$\;$}|p{.15\linewidth}@{}p{.15\linewidth}p{.15\linewidth}p{.15\linewidth}p{.15\linewidth}p{.15\linewidth}}
    \# & $\;$A & \multicolumn{4}{c}{Game Master} & $\;\:$B\\
    \hline

    \theutterance \stepcounter{utterance}  
    & & \multicolumn{4}{p{0.6\linewidth}}{
        \cellcolor[rgb]{0.9,0.9,0.9}{
            \makecell[{{p{\linewidth}}}]{
                \texttt{\tiny{[P1$\langle$GM]}}
                \texttt{Ich bin der Spielleiter, und du spielst ein kooperatives Spiel mit dem folgenden Raster als Spielbrett:} \\
\\ 
\texttt{```} \\
\texttt{ 1234567} \\
\texttt{╔═╤═╤═╤═╗} \\
\texttt{║◌│◌│H│◌║ 1} \\
\texttt{╟─┴─┼─┤◌║ 2} \\
\texttt{║◌T◌│◌└─╢ 3} \\
\texttt{╟───┘◌◌◌║ 4} \\
\texttt{║◌◌◌I◌┌─╢ 5} \\
\texttt{║◌◌◌C◌│◌║ 6} \\
\texttt{║◌◌◌W◌│◌║ 7} \\
\texttt{╚═════╧═╝} \\
\\ 
\texttt{```} \\
\\ 
\texttt{* Am oberen Rand sind die x{-}Koordinaten angegeben, die nach rechts hin zunehmen, und am rechten Rand die y{-}Koordinaten, die nach unten zunehmen.} \\
\texttt{* Die folgenden Objekte sind zufällig auf deinem Raster platziert: 'W', 'I', 'T', 'C', 'H'.} \\
\\ 
\\ 
\texttt{Dein Mitspieler sieht eine Variation des Spielbretts, auf dem dieselben Objekte an anderen zufälligen Positionen platziert sind. Du kannst das Spielbrett des anderen Spielers nicht sehen, und er kann deins nicht sehen.} \\
\\ 
\\ 
\texttt{**Ziel des Spiels:**} \\
\\ 
\texttt{Beide Spieler müssen die Objekte auf ihrem jeweiligen Spielbrett so bewegen, dass identische Objekte am Ende an denselben Koordinaten liegen. Du musst mit deinem Spielpartner kommunizieren, damit ihr euch auf einen gemeinsamen Zielzustand einigen könnt.} \\
\\ 
\\ 
\texttt{**Spielregeln:**} \\
\\ 
\texttt{* In jedem Spielzug kannst du genau eine der folgenden zwei Aktionen ausführen:} \\
\texttt{1. `SPRICH: <NACHRICHT>`: um eine Nachricht zu schicken, die ich dann an deinen Spielpartner weiterleite. Als Nachricht gilt alles bis zum nächsten Zeilenumbruch.} \\
\texttt{2. `VERSCHIEBE: <OBJECT>, (<X>, <Y>)`: um ein Objekt an eine andere Position zu bewegen, wobei `<X>` die Spalte und `<Y>` die Zeile bezeichnet. Ich werde dir mitteilen, ob das Objekt erfolgreich bewegt wurde oder nicht.} \\
\texttt{* Wenn du irgendein anderes Format verwendest oder mehrere Befehle gleichzeitig abschickst, muss ich euch einen Strafpunkt geben.} \\
\texttt{* Wenn ihr gemeinsam mehr als 12 Strafpunkte ansammelt, verliert ihr das Spiel.} \\
\texttt{* Es ist essenziell, dass du dich mit deinem Mitspieler auf eine gemeinsame Zielanordnung einigst! Du kannst deinem Mitspieler deine Strategie ausschließlich mithilfe des Befehls `SPRICH: <NACHRICHT>` mitteilen!} \\
\\ 
\\ 
\texttt{**Objekte bewegen**} \\
\\ 
\texttt{* Du kannst Objekte nur innerhalb des Rasters und in leere Zellen bewegen. Leere Zellen erkennst am Zeichen '◌'.} \\
\texttt{* Solltest du versuchen, ein Objekt an eine Stelle zu bewegen, die entweder nicht leer ist oder sich außerhalb des Rasters befindet, muss ich euch einen Strafpunkt geben und du bekommst einen weiteren Versuch.} \\
\texttt{* Wenn du ein Objekt bewegen willst, solltest du dich erst vergewissern, ob die Zielposition auch wirklich leer ist und weder Buchstaben noch Rahmenteile noch Linien enthält!} \\
\\ 
\\ 
\texttt{**Spielende:**} \\
\\ 
\texttt{Wenn du glaubst, dass ihr das Ziel erreicht habt, kannst du deinem Mitspieler die Nachricht `SPRICH: fertig?` schicken. Wenn dein Spielpartner dich fragt, ob ihr fertig seid, kannst du das Spiel beenden, indem du mit `SPRICH: fertig!` antwortest.} \\
\\ 
\texttt{Ihr gewinnt gemeinsam, wenn ihr das Spiel innerhalb von \#max\_rounds Runden beendet. Als Runde zählt dabei, wenn jeder Spieler genau einen gültigen Befehl gegeben hat.} \\
\\ 
\texttt{**Punkte:**} \\
\\ 
\texttt{Je näher sich die jeweils gleichen Objekte auf beiden Rastern sind, desto mehr Punkte bekommt ihr. Für Strafpunkte gibt es Punktabzug. Könnt ihr den Rekord brechen?} \\
\\ 
\texttt{Schicke eine Nachricht an deinen Mitspieler um das Spiel zu starten!} \\
            }
        }
    }
    & \\ \\

    \theutterance \stepcounter{utterance}  
    & \multicolumn{4}{p{0.6\linewidth}}{
        \cellcolor[rgb]{0.9,0.9,0.9}{
            \makecell[{{p{\linewidth}}}]{
                \texttt{\tiny{[P1$\rangle$GM]}}
                \texttt{SPRICH: Hallo, lass uns eine Strategie entwickeln, um die Objekte auf unseren Spielbrettern anzuordnen. Ich schlage vor, dass wir alle Objekte in der ersten Zeile platzieren, sortiert nach ihrem Alphabet. Bitte bestätige, ob das für dich auch machbar ist.} \\
            }
        }
    }
    & & \\ \\

    \theutterance \stepcounter{utterance}  
    & & & \multicolumn{2}{p{0.3\linewidth}}{
        \cellcolor[rgb]{0.9,0.9,0.9}{
            \makecell[{{p{\linewidth}}}]{
                \texttt{\tiny{[GM$|$GM]}}
                \texttt{SPRICH: Hallo, lass uns eine Strategie entwickeln, um die Objekte auf unseren Spielbrettern anzuordnen. Ich schlage vor, dass wir alle Objekte in der ersten Zeile platzieren, sortiert nach ihrem Alphabet. Bitte bestätige, ob das für dich auch machbar ist.} \\
            }
        }
    }
    & & \\ \\

    \theutterance \stepcounter{utterance}  
    & & \multicolumn{4}{p{0.6\linewidth}}{
        \cellcolor[rgb]{0.9,0.9,0.9}{
            \makecell[{{p{\linewidth}}}]{
                \texttt{\tiny{[GM$\rangle$P2]}}
                \texttt{Ich bin der Spielleiter, und du spielst ein kooperatives Spiel mit dem folgenden Raster als Spielbrett:} \\
\\ 
\texttt{```} \\
\texttt{ 1234567} \\
\texttt{╔═╤═╤═╤═╗} \\
\texttt{║W│◌│T│◌║ 1} \\
\texttt{╟─┴─┼─┤◌║ 2} \\
\texttt{║◌◌◌│◌└─╢ 3} \\
\texttt{╟───┘IC◌║ 4} \\
\texttt{║H◌◌◌◌┌─╢ 5} \\
\texttt{║◌◌◌◌◌│◌║ 6} \\
\texttt{║◌◌◌◌◌│◌║ 7} \\
\texttt{╚═════╧═╝} \\
\\ 
\texttt{```} \\
\\ 
\texttt{* Am oberen Rand sind die x{-}Koordinaten angegeben, die nach rechts hin zunehmen, und am rechten Rand die y{-}Koordinaten, die nach unten zunehmen.} \\
\texttt{* Die folgenden Objekte sind zufällig auf deinem Raster platziert: 'W', 'I', 'T', 'C', 'H'.} \\
\\ 
\\ 
\texttt{Dein Mitspieler sieht eine Variation des Spielbretts, auf dem dieselben Objekte an anderen zufälligen Positionen platziert sind. Du kannst das Spielbrett des anderen Spielers nicht sehen, und er kann deins nicht sehen.} \\
\\ 
\\ 
\texttt{**Ziel des Spiels:**} \\
\\ 
\texttt{Beide Spieler müssen die Objekte auf ihrem jeweiligen Spielbrett so bewegen, dass identische Objekte am Ende an denselben Koordinaten liegen. Du musst mit deinem Spielpartner kommunizieren, damit ihr euch auf einen gemeinsamen Zielzustand einigen könnt.} \\
\\ 
\\ 
\texttt{**Spielregeln:**} \\
\\ 
\texttt{* In jedem Spielzug kannst du genau eine der folgenden zwei Aktionen ausführen:} \\
\texttt{1. `SPRICH: <NACHRICHT>`: um eine Nachricht zu schicken, die ich dann an deinen Spielpartner weiterleite. Als Nachricht gilt alles bis zum nächsten Zeilenumbruch.} \\
\texttt{2. `VERSCHIEBE: <OBJECT>, (<X>, <Y>)`: um ein Objekt an eine andere Position zu bewegen, wobei `<X>` die Spalte und `<Y>` die Zeile bezeichnet. Ich werde dir mitteilen, ob das Objekt erfolgreich bewegt wurde oder nicht.} \\
\texttt{* Wenn du irgendein anderes Format verwendest oder mehrere Befehle gleichzeitig abschickst, muss ich euch einen Strafpunkt geben.} \\
\texttt{* Wenn ihr gemeinsam mehr als 12 Strafpunkte ansammelt, verliert ihr das Spiel.} \\
\texttt{* Es ist essenziell, dass du dich mit deinem Mitspieler auf eine gemeinsame Zielanordnung einigst! Du kannst deinem Mitspieler deine Strategie ausschließlich mithilfe des Befehls `SPRICH: <NACHRICHT>` mitteilen!} \\
\\ 
\\ 
\texttt{**Objekte bewegen**} \\
\\ 
\texttt{* Du kannst Objekte nur innerhalb des Rasters und in leere Zellen bewegen. Leere Zellen erkennst am Zeichen '◌'.} \\
\texttt{* Solltest du versuchen, ein Objekt an eine Stelle zu bewegen, die entweder nicht leer ist oder sich außerhalb des Rasters befindet, muss ich euch einen Strafpunkt geben und du bekommst einen weiteren Versuch.} \\
\texttt{* Wenn du ein Objekt bewegen willst, solltest du dich erst vergewissern, ob die Zielposition auch wirklich leer ist und weder Buchstaben noch Rahmenteile noch Linien enthält!} \\
\\ 
\\ 
\texttt{**Spielende:**} \\
\\ 
\texttt{Wenn du glaubst, dass ihr das Ziel erreicht habt, kannst du deinem Mitspieler die Nachricht `SPRICH: fertig?` schicken. Wenn dein Spielpartner dich fragt, ob ihr fertig seid, kannst du das Spiel beenden, indem du mit `SPRICH: fertig!` antwortest.} \\
\\ 
\texttt{Ihr gewinnt gemeinsam, wenn ihr das Spiel innerhalb von \#max\_rounds Runden beendet. Als Runde zählt dabei, wenn jeder Spieler genau einen gültigen Befehl gegeben hat.} \\
\\ 
\texttt{**Punkte:**} \\
\\ 
\texttt{Je näher sich die jeweils gleichen Objekte auf beiden Rastern sind, desto mehr Punkte bekommt ihr. Für Strafpunkte gibt es Punktabzug. Könnt ihr den Rekord brechen?} \\
\\ 
\texttt{Dein Mitspieler hat das Spiel mit dieser Nachricht begonnen:} \\
\\ 
\texttt{"Hallo, lass uns eine Strategie entwickeln, um die Objekte auf unseren Spielbrettern anzuordnen. Ich schlage vor, dass wir alle Objekte in der ersten Zeile platzieren, sortiert nach ihrem Alphabet. Bitte bestätige, ob das für dich auch machbar ist."} \\
\\ 
\texttt{Wie lautet dein erster Befehl?} \\
            }
        }
    }
    & \\ \\

    \theutterance \stepcounter{utterance}  
    & & & \multicolumn{4}{p{0.6\linewidth}}{
        \cellcolor[rgb]{0.9,0.9,0.9}{
            \makecell[{{p{\linewidth}}}]{
                \texttt{\tiny{[GM$\langle$P2]}}
                \texttt{Das klingt nach einem guten Ansatz! Ich denke, es ist eine gute Idee, die Objekte in der ersten Zeile zu platzieren, sortiert nach ihrem Alphabet. Ich möchte jedoch auch sicherstellen, dass wir uns über die genaue Reihenfolge einig sind.} \\
\\ 
\texttt{Mein erster Befehl lautet also:} \\
\\ 
\texttt{`SPRICH: Hallo! Ich stimme deiner Strategie zu, die Objekte in der ersten Zeile zu platzieren, sortiert nach ihrem Alphabet. Um sicherzustellen, dass wir uns über die Reihenfolge einig sind, schlage ich vor, dass wir die Objekte in der Reihenfolge 'C', 'H', 'I', 'T', 'W' platzieren. Bitte bestätige, ob das für dich auch okay ist.`} \\
            }
        }
    }
     \\ \\

    \theutterance \stepcounter{utterance}  
    & & & \multicolumn{2}{p{0.3\linewidth}}{
        \cellcolor[rgb]{0.9,0.9,0.9}{
            \makecell[{{p{\linewidth}}}]{
                \texttt{\tiny{[GM$|$GM]}}
                \texttt{Das klingt nach einem guten Ansatz! Ich denke, es ist eine gute Idee, die Objekte in der ersten Zeile zu platzieren, sortiert nach ihrem Alphabet. Ich möchte jedoch auch sicherstellen, dass wir uns über die genaue Reihenfolge einig sind.} \\
\\ 
\texttt{Mein erster Befehl lautet also:} \\
\\ 
\texttt{`SPRICH: Hallo! Ich stimme deiner Strategie zu, die Objekte in der ersten Zeile zu platzieren, sortiert nach ihrem Alphabet. Um sicherzustellen, dass wir uns über die Reihenfolge einig sind, schlage ich vor, dass wir die Objekte in der Reihenfolge 'C', 'H', 'I', 'T', 'W' platzieren. Bitte bestätige, ob das für dich auch okay ist.`} \\
            }
        }
    }
    & & \\ \\

    \theutterance \stepcounter{utterance}  
    & & & \multicolumn{2}{p{0.3\linewidth}}{
        \cellcolor[rgb]{0.9,0.9,0.9}{
            \makecell[{{p{\linewidth}}}]{
                \texttt{\tiny{[GM$|$GM]}}
                \texttt{Invalid format: head is not empty: 'Das klingt nach einem guten Ansatz! Ich denke, es ist eine gute Idee, die Objekte in der ersten Zeile zu platzieren, sortiert nach ihrem Alphabet. Ich möchte jedoch auch sicherstellen, dass wir uns über die genaue Reihenfolge einig sind.} \\
\\ 
\texttt{Mein erster Befehl lautet also:} \\
\\ 
\texttt{'} \\
            }
        }
    }
    & & \\ \\

    \theutterance \stepcounter{utterance}  
    & & \multicolumn{4}{p{0.6\linewidth}}{
        \cellcolor[rgb]{0.9,0.9,0.9}{
            \makecell[{{p{\linewidth}}}]{
                \texttt{\tiny{[GM$\rangle$P2]}}
                \texttt{Strafpunkt: Deine Nachricht darf nichts vor dem Befehl enthalten! Stelle sicher, dass deine Antwort ausschließlich einen der beiden Befehle `SPRICH: <NACHRICHT>` und `VERSCHIEBE: <OBJECT>, (<X>, <Y>)` und sonst nichts enthält!} \\
\\ 
\texttt{Ihr habt gemeinsam 1 von maximal 12 Strafpunkten angesammelt.} \\
\texttt{Bitte versuche es noch mal!} \\
            }
        }
    }
    & \\ \\

    \theutterance \stepcounter{utterance}  
    & & & \multicolumn{4}{p{0.6\linewidth}}{
        \cellcolor[rgb]{0.9,0.9,0.9}{
            \makecell[{{p{\linewidth}}}]{
                \texttt{\tiny{[GM$\langle$P2]}}
                \texttt{SPRICH: Ich stimme deiner Strategie zu, die Objekte in der ersten Zeile zu platzieren, sortiert nach ihrem Alphabet. Um sicherzustellen, dass wir uns über die Reihenfolge einig sind, schlage ich vor, dass wir die Objekte in der Reihenfolge 'C', 'H', 'I', 'T', 'W' platzieren. Bitte bestätige, ob das für dich auch okay ist.} \\
            }
        }
    }
     \\ \\

    \theutterance \stepcounter{utterance}  
    & & & \multicolumn{2}{p{0.3\linewidth}}{
        \cellcolor[rgb]{0.9,0.9,0.9}{
            \makecell[{{p{\linewidth}}}]{
                \texttt{\tiny{[GM$|$GM]}}
                \texttt{SPRICH: Ich stimme deiner Strategie zu, die Objekte in der ersten Zeile zu platzieren, sortiert nach ihrem Alphabet. Um sicherzustellen, dass wir uns über die Reihenfolge einig sind, schlage ich vor, dass wir die Objekte in der Reihenfolge 'C', 'H', 'I', 'T', 'W' platzieren. Bitte bestätige, ob das für dich auch okay ist.} \\
            }
        }
    }
    & & \\ \\

    \theutterance \stepcounter{utterance}  
    & & \multicolumn{4}{p{0.6\linewidth}}{
        \cellcolor[rgb]{0.9,0.9,0.9}{
            \makecell[{{p{\linewidth}}}]{
                \texttt{\tiny{[P1$\langle$GM]}}
                \texttt{Deine Nachricht wurde an deinen Mitspieler übermittelt.} \\
\\ 
\\ 
\texttt{Ihr seid gerade bei Runde 1 von maximal 20.} \\
\texttt{Ihr habt gemeinsam 1 von maximal 12 Strafpunkten angesammelt.} \\
\texttt{Dein Mitspieler hat dir diese Nachricht geschickt:} \\
\\ 
\\ 
\texttt{"Ich stimme deiner Strategie zu, die Objekte in der ersten Zeile zu platzieren, sortiert nach ihrem Alphabet. Um sicherzustellen, dass wir uns über die Reihenfolge einig sind, schlage ich vor, dass wir die Objekte in der Reihenfolge 'C', 'H', 'I', 'T', 'W' platzieren. Bitte bestätige, ob das für dich auch okay ist."} \\
\\ 
\\ 
\texttt{Wie lautet dein nächster Befehl?} \\
            }
        }
    }
    & \\ \\

    \theutterance \stepcounter{utterance}  
    & \multicolumn{4}{p{0.6\linewidth}}{
        \cellcolor[rgb]{0.9,0.9,0.9}{
            \makecell[{{p{\linewidth}}}]{
                \texttt{\tiny{[P1$\rangle$GM]}}
                \texttt{SPRICH: Ja, das ist absolut okay für mich! Die Reihenfolge 'C', 'H', 'I', 'T', 'W' ist logisch und einfach nachzuvollziehen. Lass uns nun unsere Objekte entsprechend anordnen. Ich werde mich um das bewegen meiner Objekte kümmern und hoffe, dass wir ein gemeinsames Ziel erreichen können. Bitte bestätige, wenn du bereit bist, Änderungen auf deinem Brett vorzunehmen.} \\
            }
        }
    }
    & & \\ \\

    \theutterance \stepcounter{utterance}  
    & & & \multicolumn{2}{p{0.3\linewidth}}{
        \cellcolor[rgb]{0.9,0.9,0.9}{
            \makecell[{{p{\linewidth}}}]{
                \texttt{\tiny{[GM$|$GM]}}
                \texttt{SPRICH: Ja, das ist absolut okay für mich! Die Reihenfolge 'C', 'H', 'I', 'T', 'W' ist logisch und einfach nachzuvollziehen. Lass uns nun unsere Objekte entsprechend anordnen. Ich werde mich um das bewegen meiner Objekte kümmern und hoffe, dass wir ein gemeinsames Ziel erreichen können. Bitte bestätige, wenn du bereit bist, Änderungen auf deinem Brett vorzunehmen.} \\
            }
        }
    }
    & & \\ \\

    \theutterance \stepcounter{utterance}  
    & & \multicolumn{4}{p{0.6\linewidth}}{
        \cellcolor[rgb]{0.9,0.9,0.9}{
            \makecell[{{p{\linewidth}}}]{
                \texttt{\tiny{[GM$\rangle$P2]}}
                \texttt{Deine Nachricht wurde an deinen Mitspieler übermittelt.} \\
\\ 
\\ 
\texttt{Ihr seid gerade bei Runde 2 von maximal 20.} \\
\texttt{Ihr habt gemeinsam 1 von maximal 12 Strafpunkten angesammelt.} \\
\texttt{Dein Mitspieler hat dir diese Nachricht geschickt:} \\
\\ 
\\ 
\texttt{"Ja, das ist absolut okay für mich! Die Reihenfolge 'C', 'H', 'I', 'T', 'W' ist logisch und einfach nachzuvollziehen. Lass uns nun unsere Objekte entsprechend anordnen. Ich werde mich um das bewegen meiner Objekte kümmern und hoffe, dass wir ein gemeinsames Ziel erreichen können. Bitte bestätige, wenn du bereit bist, Änderungen auf deinem Brett vorzunehmen."} \\
\\ 
\\ 
\texttt{Wie lautet dein nächster Befehl?} \\
            }
        }
    }
    & \\ \\

    \theutterance \stepcounter{utterance}  
    & & & \multicolumn{4}{p{0.6\linewidth}}{
        \cellcolor[rgb]{0.9,0.9,0.9}{
            \makecell[{{p{\linewidth}}}]{
                \texttt{\tiny{[GM$\langle$P2]}}
                \texttt{SPRICH: Ja, ich bin bereit, Änderungen auf meinem Brett vorzunehmen. Ich werde mich um das Bewegen meiner Objekte kümmern und hoffe, dass wir ein gemeinsames Ziel erreichen können. Lass uns loslegen!} \\
            }
        }
    }
     \\ \\

    \theutterance \stepcounter{utterance}  
    & & & \multicolumn{2}{p{0.3\linewidth}}{
        \cellcolor[rgb]{0.9,0.9,0.9}{
            \makecell[{{p{\linewidth}}}]{
                \texttt{\tiny{[GM$|$GM]}}
                \texttt{SPRICH: Ja, ich bin bereit, Änderungen auf meinem Brett vorzunehmen. Ich werde mich um das Bewegen meiner Objekte kümmern und hoffe, dass wir ein gemeinsames Ziel erreichen können. Lass uns loslegen!} \\
            }
        }
    }
    & & \\ \\

    \theutterance \stepcounter{utterance}  
    & & \multicolumn{4}{p{0.6\linewidth}}{
        \cellcolor[rgb]{0.9,0.9,0.9}{
            \makecell[{{p{\linewidth}}}]{
                \texttt{\tiny{[P1$\langle$GM]}}
                \texttt{Deine Nachricht wurde an deinen Mitspieler übermittelt.} \\
\\ 
\\ 
\texttt{Ihr seid gerade bei Runde 2 von maximal 20.} \\
\texttt{Ihr habt gemeinsam 1 von maximal 12 Strafpunkten angesammelt.} \\
\texttt{Dein Mitspieler hat dir diese Nachricht geschickt:} \\
\\ 
\\ 
\texttt{"Ja, ich bin bereit, Änderungen auf meinem Brett vorzunehmen. Ich werde mich um das Bewegen meiner Objekte kümmern und hoffe, dass wir ein gemeinsames Ziel erreichen können. Lass uns loslegen!"} \\
\\ 
\\ 
\texttt{Wie lautet dein nächster Befehl?} \\
            }
        }
    }
    & \\ \\

    \theutterance \stepcounter{utterance}  
    & \multicolumn{4}{p{0.6\linewidth}}{
        \cellcolor[rgb]{0.9,0.9,0.9}{
            \makecell[{{p{\linewidth}}}]{
                \texttt{\tiny{[P1$\rangle$GM]}}
                \texttt{VERSCHIEBE: H, (3, 1)} \\
            }
        }
    }
    & & \\ \\

    \theutterance \stepcounter{utterance}  
    & & & \multicolumn{2}{p{0.3\linewidth}}{
        \cellcolor[rgb]{0.9,0.9,0.9}{
            \makecell[{{p{\linewidth}}}]{
                \texttt{\tiny{[GM$|$GM]}}
                \texttt{VERSCHIEBE: H, (3, 1)} \\
            }
        }
    }
    & & \\ \\

    \theutterance \stepcounter{utterance}  
    & & & \multicolumn{2}{p{0.3\linewidth}}{
        \cellcolor[rgb]{0.9,0.9,0.9}{
            \makecell[{{p{\linewidth}}}]{
                \texttt{\tiny{[GM$|$GM]}}
                \texttt{'H' erfolgreich nach (3,1) verschoben. Dein Raster sieht jetzt so aus:} \\
\texttt{```} \\
\texttt{ 1234567} \\
\texttt{╔═╤═╤═╤═╗} \\
\texttt{║◌│H│◌│◌║ 1} \\
\texttt{╟─┴─┼─┤◌║ 2} \\
\texttt{║◌T◌│◌└─╢ 3} \\
\texttt{╟───┘◌◌◌║ 4} \\
\texttt{║◌◌◌I◌┌─╢ 5} \\
\texttt{║◌◌◌C◌│◌║ 6} \\
\texttt{║◌◌◌W◌│◌║ 7} \\
\texttt{╚═════╧═╝} \\
\texttt{```} \\
            }
        }
    }
    & & \\ \\

    \theutterance \stepcounter{utterance}  
    & & \multicolumn{4}{p{0.6\linewidth}}{
        \cellcolor[rgb]{0.9,0.9,0.9}{
            \makecell[{{p{\linewidth}}}]{
                \texttt{\tiny{[GM$\rangle$P2]}}
                \texttt{Deine Nachricht wurde an deinen Mitspieler übermittelt.} \\
\\ 
\\ 
\texttt{Ihr seid gerade bei Runde 3 von maximal 20.} \\
\texttt{Ihr habt gemeinsam 1 von maximal 12 Strafpunkten angesammelt.} \\
\texttt{Dein Mitspieler hat ein Objekt auf seinem Raster bewegt.} \\
\\ 
\\ 
\texttt{Wie lautet dein nächster Befehl?} \\
            }
        }
    }
    & \\ \\

    \theutterance \stepcounter{utterance}  
    & & & \multicolumn{4}{p{0.6\linewidth}}{
        \cellcolor[rgb]{0.9,0.9,0.9}{
            \makecell[{{p{\linewidth}}}]{
                \texttt{\tiny{[GM$\langle$P2]}}
                \texttt{SPRICH: Ich werde nun auch mit dem Bewegen meiner Objekte beginnen. Ich werde das Objekt 'W' als erstes bewegen, da es bereits in der ersten Zeile ist. Ich werde es an die Position (1, 1) bewegen, um sicherzustellen, dass es an der richtigen Stelle ist. Bitte beachte, dass ich möglicherweise weitere Anpassungen vornehmen werde, um die Objekte in der richtigen Reihenfolge zu platzieren.} \\
            }
        }
    }
     \\ \\

    \theutterance \stepcounter{utterance}  
    & & & \multicolumn{2}{p{0.3\linewidth}}{
        \cellcolor[rgb]{0.9,0.9,0.9}{
            \makecell[{{p{\linewidth}}}]{
                \texttt{\tiny{[GM$|$GM]}}
                \texttt{SPRICH: Ich werde nun auch mit dem Bewegen meiner Objekte beginnen. Ich werde das Objekt 'W' als erstes bewegen, da es bereits in der ersten Zeile ist. Ich werde es an die Position (1, 1) bewegen, um sicherzustellen, dass es an der richtigen Stelle ist. Bitte beachte, dass ich möglicherweise weitere Anpassungen vornehmen werde, um die Objekte in der richtigen Reihenfolge zu platzieren.} \\
            }
        }
    }
    & & \\ \\

    \theutterance \stepcounter{utterance}  
    & & \multicolumn{4}{p{0.6\linewidth}}{
        \cellcolor[rgb]{0.9,0.9,0.9}{
            \makecell[{{p{\linewidth}}}]{
                \texttt{\tiny{[P1$\langle$GM]}}
                \texttt{'H' erfolgreich nach (3,1) verschoben. Dein Raster sieht jetzt so aus:} \\
\texttt{```} \\
\texttt{ 1234567} \\
\texttt{╔═╤═╤═╤═╗} \\
\texttt{║◌│H│◌│◌║ 1} \\
\texttt{╟─┴─┼─┤◌║ 2} \\
\texttt{║◌T◌│◌└─╢ 3} \\
\texttt{╟───┘◌◌◌║ 4} \\
\texttt{║◌◌◌I◌┌─╢ 5} \\
\texttt{║◌◌◌C◌│◌║ 6} \\
\texttt{║◌◌◌W◌│◌║ 7} \\
\texttt{╚═════╧═╝} \\
\texttt{```} \\
\texttt{Ihr seid gerade bei Runde 3 von maximal 20.} \\
\texttt{Ihr habt gemeinsam 1 von maximal 12 Strafpunkten angesammelt.} \\
\texttt{Dein Mitspieler hat dir diese Nachricht geschickt:} \\
\\ 
\\ 
\texttt{"Ich werde nun auch mit dem Bewegen meiner Objekte beginnen. Ich werde das Objekt 'W' als erstes bewegen, da es bereits in der ersten Zeile ist. Ich werde es an die Position (1, 1) bewegen, um sicherzustellen, dass es an der richtigen Stelle ist. Bitte beachte, dass ich möglicherweise weitere Anpassungen vornehmen werde, um die Objekte in der richtigen Reihenfolge zu platzieren."} \\
\\ 
\\ 
\texttt{Wie lautet dein nächster Befehl?} \\
            }
        }
    }
    & \\ \\

    \theutterance \stepcounter{utterance}  
    & \multicolumn{4}{p{0.6\linewidth}}{
        \cellcolor[rgb]{0.9,0.9,0.9}{
            \makecell[{{p{\linewidth}}}]{
                \texttt{\tiny{[P1$\rangle$GM]}}
                \texttt{SPRICH: Ich verstehe, dass du das Objekt 'W' an die Position (1, 1) bewegen möchtest. Das klingt nach einem guten Plan. Ich werde jetzt das Objekt 'C' bewegen, um es in der richtigen Reihenfolge zu platzieren. Ich werde es an die Position (5, 1) bewegen. Bitte beachte, dass ich möglicherweise weitere Anpassungen vornehmen werde, um die Objekte in der richtigen Reihenfolge zu platzieren. Lass uns weitermachen und unsere Objekte anordnen!} \\
            }
        }
    }
    & & \\ \\

    \theutterance \stepcounter{utterance}  
    & & & \multicolumn{2}{p{0.3\linewidth}}{
        \cellcolor[rgb]{0.9,0.9,0.9}{
            \makecell[{{p{\linewidth}}}]{
                \texttt{\tiny{[GM$|$GM]}}
                \texttt{SPRICH: Ich verstehe, dass du das Objekt 'W' an die Position (1, 1) bewegen möchtest. Das klingt nach einem guten Plan. Ich werde jetzt das Objekt 'C' bewegen, um es in der richtigen Reihenfolge zu platzieren. Ich werde es an die Position (5, 1) bewegen. Bitte beachte, dass ich möglicherweise weitere Anpassungen vornehmen werde, um die Objekte in der richtigen Reihenfolge zu platzieren. Lass uns weitermachen und unsere Objekte anordnen!} \\
            }
        }
    }
    & & \\ \\

    \theutterance \stepcounter{utterance}  
    & & \multicolumn{4}{p{0.6\linewidth}}{
        \cellcolor[rgb]{0.9,0.9,0.9}{
            \makecell[{{p{\linewidth}}}]{
                \texttt{\tiny{[GM$\rangle$P2]}}
                \texttt{Deine Nachricht wurde an deinen Mitspieler übermittelt.} \\
\\ 
\\ 
\texttt{Ihr seid gerade bei Runde 4 von maximal 20.} \\
\texttt{Ihr habt gemeinsam 1 von maximal 12 Strafpunkten angesammelt.} \\
\texttt{Dein Mitspieler hat dir diese Nachricht geschickt:} \\
\\ 
\\ 
\texttt{"Ich verstehe, dass du das Objekt 'W' an die Position (1, 1) bewegen möchtest. Das klingt nach einem guten Plan. Ich werde jetzt das Objekt 'C' bewegen, um es in der richtigen Reihenfolge zu platzieren. Ich werde es an die Position (5, 1) bewegen. Bitte beachte, dass ich möglicherweise weitere Anpassungen vornehmen werde, um die Objekte in der richtigen Reihenfolge zu platzieren. Lass uns weitermachen und unsere Objekte anordnen!"} \\
\\ 
\\ 
\texttt{Wie lautet dein nächster Befehl?} \\
            }
        }
    }
    & \\ \\

    \theutterance \stepcounter{utterance}  
    & & & \multicolumn{4}{p{0.6\linewidth}}{
        \cellcolor[rgb]{0.9,0.9,0.9}{
            \makecell[{{p{\linewidth}}}]{
                \texttt{\tiny{[GM$\langle$P2]}}
                \texttt{VERSCHIEBE: W, (1, 1)} \\
            }
        }
    }
     \\ \\

    \theutterance \stepcounter{utterance}  
    & & & \multicolumn{2}{p{0.3\linewidth}}{
        \cellcolor[rgb]{0.9,0.9,0.9}{
            \makecell[{{p{\linewidth}}}]{
                \texttt{\tiny{[GM$|$GM]}}
                \texttt{VERSCHIEBE: W, (1, 1)} \\
            }
        }
    }
    & & \\ \\

    \theutterance \stepcounter{utterance}  
    & & & \multicolumn{2}{p{0.3\linewidth}}{
        \cellcolor[rgb]{0.9,0.9,0.9}{
            \makecell[{{p{\linewidth}}}]{
                \texttt{\tiny{[GM$|$GM]}}
                \texttt{Strafpunkt: (1,1) ist nicht leer, sondern enthält 'W'.} \\
\\ 
\texttt{Ihr habt gemeinsam 2 von maximal 12 Strafpunkten angesammelt.} \\
\texttt{Bitte versuche es noch mal!} \\
            }
        }
    }
    & & \\ \\

    \theutterance \stepcounter{utterance}  
    & & \multicolumn{4}{p{0.6\linewidth}}{
        \cellcolor[rgb]{0.9,0.9,0.9}{
            \makecell[{{p{\linewidth}}}]{
                \texttt{\tiny{[GM$\rangle$P2]}}
                \texttt{Strafpunkt: (1,1) ist nicht leer, sondern enthält 'W'.} \\
\\ 
\texttt{Ihr habt gemeinsam 2 von maximal 12 Strafpunkten angesammelt.} \\
\texttt{Bitte versuche es noch mal!} \\
            }
        }
    }
    & \\ \\

    \theutterance \stepcounter{utterance}  
    & & & \multicolumn{4}{p{0.6\linewidth}}{
        \cellcolor[rgb]{0.9,0.9,0.9}{
            \makecell[{{p{\linewidth}}}]{
                \texttt{\tiny{[GM$\langle$P2]}}
                \texttt{VERSCHIEBE: T, (3, 1)} \\
            }
        }
    }
     \\ \\

    \theutterance \stepcounter{utterance}  
    & & & \multicolumn{2}{p{0.3\linewidth}}{
        \cellcolor[rgb]{0.9,0.9,0.9}{
            \makecell[{{p{\linewidth}}}]{
                \texttt{\tiny{[GM$|$GM]}}
                \texttt{VERSCHIEBE: T, (3, 1)} \\
            }
        }
    }
    & & \\ \\

    \theutterance \stepcounter{utterance}  
    & & & \multicolumn{2}{p{0.3\linewidth}}{
        \cellcolor[rgb]{0.9,0.9,0.9}{
            \makecell[{{p{\linewidth}}}]{
                \texttt{\tiny{[GM$|$GM]}}
                \texttt{'T' erfolgreich nach (3,1) verschoben. Dein Raster sieht jetzt so aus:} \\
\texttt{```} \\
\texttt{ 1234567} \\
\texttt{╔═╤═╤═╤═╗} \\
\texttt{║W│T│◌│◌║ 1} \\
\texttt{╟─┴─┼─┤◌║ 2} \\
\texttt{║◌◌◌│◌└─╢ 3} \\
\texttt{╟───┘IC◌║ 4} \\
\texttt{║H◌◌◌◌┌─╢ 5} \\
\texttt{║◌◌◌◌◌│◌║ 6} \\
\texttt{║◌◌◌◌◌│◌║ 7} \\
\texttt{╚═════╧═╝} \\
\texttt{```} \\
            }
        }
    }
    & & \\ \\

    \theutterance \stepcounter{utterance}  
    & & \multicolumn{4}{p{0.6\linewidth}}{
        \cellcolor[rgb]{0.9,0.9,0.9}{
            \makecell[{{p{\linewidth}}}]{
                \texttt{\tiny{[P1$\langle$GM]}}
                \texttt{Deine Nachricht wurde an deinen Mitspieler übermittelt.} \\
\\ 
\\ 
\texttt{Ihr seid gerade bei Runde 4 von maximal 20.} \\
\texttt{Ihr habt gemeinsam 2 von maximal 12 Strafpunkten angesammelt.} \\
\texttt{Dein Mitspieler hat ein Objekt auf seinem Raster bewegt.} \\
\\ 
\\ 
\texttt{Wie lautet dein nächster Befehl?} \\
            }
        }
    }
    & \\ \\

    \theutterance \stepcounter{utterance}  
    & \multicolumn{4}{p{0.6\linewidth}}{
        \cellcolor[rgb]{0.9,0.9,0.9}{
            \makecell[{{p{\linewidth}}}]{
                \texttt{\tiny{[P1$\rangle$GM]}}
                \texttt{VERSCHIEBE: C, (1, 1)} \\
            }
        }
    }
    & & \\ \\

    \theutterance \stepcounter{utterance}  
    & & & \multicolumn{2}{p{0.3\linewidth}}{
        \cellcolor[rgb]{0.9,0.9,0.9}{
            \makecell[{{p{\linewidth}}}]{
                \texttt{\tiny{[GM$|$GM]}}
                \texttt{VERSCHIEBE: C, (1, 1)} \\
            }
        }
    }
    & & \\ \\

    \theutterance \stepcounter{utterance}  
    & & & \multicolumn{2}{p{0.3\linewidth}}{
        \cellcolor[rgb]{0.9,0.9,0.9}{
            \makecell[{{p{\linewidth}}}]{
                \texttt{\tiny{[GM$|$GM]}}
                \texttt{'C' erfolgreich nach (1,1) verschoben. Dein Raster sieht jetzt so aus:} \\
\texttt{```} \\
\texttt{ 1234567} \\
\texttt{╔═╤═╤═╤═╗} \\
\texttt{║C│H│◌│◌║ 1} \\
\texttt{╟─┴─┼─┤◌║ 2} \\
\texttt{║◌T◌│◌└─╢ 3} \\
\texttt{╟───┘◌◌◌║ 4} \\
\texttt{║◌◌◌I◌┌─╢ 5} \\
\texttt{║◌◌◌◌◌│◌║ 6} \\
\texttt{║◌◌◌W◌│◌║ 7} \\
\texttt{╚═════╧═╝} \\
\texttt{```} \\
            }
        }
    }
    & & \\ \\

    \theutterance \stepcounter{utterance}  
    & & \multicolumn{4}{p{0.6\linewidth}}{
        \cellcolor[rgb]{0.9,0.9,0.9}{
            \makecell[{{p{\linewidth}}}]{
                \texttt{\tiny{[GM$\rangle$P2]}}
                \texttt{'T' erfolgreich nach (3,1) verschoben. Dein Raster sieht jetzt so aus:} \\
\texttt{```} \\
\texttt{ 1234567} \\
\texttt{╔═╤═╤═╤═╗} \\
\texttt{║W│T│◌│◌║ 1} \\
\texttt{╟─┴─┼─┤◌║ 2} \\
\texttt{║◌◌◌│◌└─╢ 3} \\
\texttt{╟───┘IC◌║ 4} \\
\texttt{║H◌◌◌◌┌─╢ 5} \\
\texttt{║◌◌◌◌◌│◌║ 6} \\
\texttt{║◌◌◌◌◌│◌║ 7} \\
\texttt{╚═════╧═╝} \\
\texttt{```} \\
\texttt{Ihr seid gerade bei Runde 5 von maximal 20.} \\
\texttt{Ihr habt gemeinsam 2 von maximal 12 Strafpunkten angesammelt.} \\
\texttt{Dein Mitspieler hat ein Objekt auf seinem Raster bewegt.} \\
\\ 
\\ 
\texttt{Wie lautet dein nächster Befehl?} \\
            }
        }
    }
    & \\ \\

    \theutterance \stepcounter{utterance}  
    & & & \multicolumn{4}{p{0.6\linewidth}}{
        \cellcolor[rgb]{0.9,0.9,0.9}{
            \makecell[{{p{\linewidth}}}]{
                \texttt{\tiny{[GM$\langle$P2]}}
                \texttt{VERSCHIEBE: I, (4, 1)} \\
            }
        }
    }
     \\ \\

    \theutterance \stepcounter{utterance}  
    & & & \multicolumn{2}{p{0.3\linewidth}}{
        \cellcolor[rgb]{0.9,0.9,0.9}{
            \makecell[{{p{\linewidth}}}]{
                \texttt{\tiny{[GM$|$GM]}}
                \texttt{VERSCHIEBE: I, (4, 1)} \\
            }
        }
    }
    & & \\ \\

    \theutterance \stepcounter{utterance}  
    & & & \multicolumn{2}{p{0.3\linewidth}}{
        \cellcolor[rgb]{0.9,0.9,0.9}{
            \makecell[{{p{\linewidth}}}]{
                \texttt{\tiny{[GM$|$GM]}}
                \texttt{Strafpunkt: (4,1) ist nicht leer, sondern enthält '│'.} \\
\\ 
\texttt{Ihr habt gemeinsam 3 von maximal 12 Strafpunkten angesammelt.} \\
\texttt{Bitte versuche es noch mal!} \\
            }
        }
    }
    & & \\ \\

    \theutterance \stepcounter{utterance}  
    & & \multicolumn{4}{p{0.6\linewidth}}{
        \cellcolor[rgb]{0.9,0.9,0.9}{
            \makecell[{{p{\linewidth}}}]{
                \texttt{\tiny{[GM$\rangle$P2]}}
                \texttt{Strafpunkt: (4,1) ist nicht leer, sondern enthält '│'.} \\
\\ 
\texttt{Ihr habt gemeinsam 3 von maximal 12 Strafpunkten angesammelt.} \\
\texttt{Bitte versuche es noch mal!} \\
            }
        }
    }
    & \\ \\

    \theutterance \stepcounter{utterance}  
    & & & \multicolumn{4}{p{0.6\linewidth}}{
        \cellcolor[rgb]{0.9,0.9,0.9}{
            \makecell[{{p{\linewidth}}}]{
                \texttt{\tiny{[GM$\langle$P2]}}
                \texttt{VERSCHIEBE: I, (2, 1)} \\
            }
        }
    }
     \\ \\

    \theutterance \stepcounter{utterance}  
    & & & \multicolumn{2}{p{0.3\linewidth}}{
        \cellcolor[rgb]{0.9,0.9,0.9}{
            \makecell[{{p{\linewidth}}}]{
                \texttt{\tiny{[GM$|$GM]}}
                \texttt{VERSCHIEBE: I, (2, 1)} \\
            }
        }
    }
    & & \\ \\

    \theutterance \stepcounter{utterance}  
    & & & \multicolumn{2}{p{0.3\linewidth}}{
        \cellcolor[rgb]{0.9,0.9,0.9}{
            \makecell[{{p{\linewidth}}}]{
                \texttt{\tiny{[GM$|$GM]}}
                \texttt{Strafpunkt: (2,1) ist nicht leer, sondern enthält '│'.} \\
\\ 
\texttt{Ihr habt gemeinsam 4 von maximal 12 Strafpunkten angesammelt.} \\
\texttt{Bitte versuche es noch mal!} \\
            }
        }
    }
    & & \\ \\

    \theutterance \stepcounter{utterance}  
    & & \multicolumn{4}{p{0.6\linewidth}}{
        \cellcolor[rgb]{0.9,0.9,0.9}{
            \makecell[{{p{\linewidth}}}]{
                \texttt{\tiny{[GM$\rangle$P2]}}
                \texttt{Strafpunkt: (2,1) ist nicht leer, sondern enthält '│'.} \\
\\ 
\texttt{Ihr habt gemeinsam 4 von maximal 12 Strafpunkten angesammelt.} \\
\texttt{Bitte versuche es noch mal!} \\
            }
        }
    }
    & \\ \\

    \theutterance \stepcounter{utterance}  
    & & & \multicolumn{4}{p{0.6\linewidth}}{
        \cellcolor[rgb]{0.9,0.9,0.9}{
            \makecell[{{p{\linewidth}}}]{
                \texttt{\tiny{[GM$\langle$P2]}}
                \texttt{VERSCHIEBE: C, (5, 1)} \\
            }
        }
    }
     \\ \\

    \theutterance \stepcounter{utterance}  
    & & & \multicolumn{2}{p{0.3\linewidth}}{
        \cellcolor[rgb]{0.9,0.9,0.9}{
            \makecell[{{p{\linewidth}}}]{
                \texttt{\tiny{[GM$|$GM]}}
                \texttt{VERSCHIEBE: C, (5, 1)} \\
            }
        }
    }
    & & \\ \\

    \theutterance \stepcounter{utterance}  
    & & & \multicolumn{2}{p{0.3\linewidth}}{
        \cellcolor[rgb]{0.9,0.9,0.9}{
            \makecell[{{p{\linewidth}}}]{
                \texttt{\tiny{[GM$|$GM]}}
                \texttt{'C' erfolgreich nach (5,1) verschoben. Dein Raster sieht jetzt so aus:} \\
\texttt{```} \\
\texttt{ 1234567} \\
\texttt{╔═╤═╤═╤═╗} \\
\texttt{║W│T│C│◌║ 1} \\
\texttt{╟─┴─┼─┤◌║ 2} \\
\texttt{║◌◌◌│◌└─╢ 3} \\
\texttt{╟───┘I◌◌║ 4} \\
\texttt{║H◌◌◌◌┌─╢ 5} \\
\texttt{║◌◌◌◌◌│◌║ 6} \\
\texttt{║◌◌◌◌◌│◌║ 7} \\
\texttt{╚═════╧═╝} \\
\texttt{```} \\
            }
        }
    }
    & & \\ \\

    \theutterance \stepcounter{utterance}  
    & & \multicolumn{4}{p{0.6\linewidth}}{
        \cellcolor[rgb]{0.9,0.9,0.9}{
            \makecell[{{p{\linewidth}}}]{
                \texttt{\tiny{[P1$\langle$GM]}}
                \texttt{'C' erfolgreich nach (1,1) verschoben. Dein Raster sieht jetzt so aus:} \\
\texttt{```} \\
\texttt{ 1234567} \\
\texttt{╔═╤═╤═╤═╗} \\
\texttt{║C│H│◌│◌║ 1} \\
\texttt{╟─┴─┼─┤◌║ 2} \\
\texttt{║◌T◌│◌└─╢ 3} \\
\texttt{╟───┘◌◌◌║ 4} \\
\texttt{║◌◌◌I◌┌─╢ 5} \\
\texttt{║◌◌◌◌◌│◌║ 6} \\
\texttt{║◌◌◌W◌│◌║ 7} \\
\texttt{╚═════╧═╝} \\
\texttt{```} \\
\texttt{Ihr seid gerade bei Runde 5 von maximal 20.} \\
\texttt{Ihr habt gemeinsam 4 von maximal 12 Strafpunkten angesammelt.} \\
\texttt{Dein Mitspieler hat ein Objekt auf seinem Raster bewegt.} \\
\\ 
\\ 
\texttt{Wie lautet dein nächster Befehl?} \\
            }
        }
    }
    & \\ \\

    \theutterance \stepcounter{utterance}  
    & \multicolumn{4}{p{0.6\linewidth}}{
        \cellcolor[rgb]{0.9,0.9,0.9}{
            \makecell[{{p{\linewidth}}}]{
                \texttt{\tiny{[P1$\rangle$GM]}}
                \texttt{SPRICH: Ich sehe, dass wir beide Objekte bewegt haben. Ich denke, es ist an der Zeit, dass wir unsere Fortschritte überprüfen und sicherstellen, dass wir auf dem richtigen Weg sind. Kannst du mir bitte mitteilen, wie dein Raster aktuell aussieht, oder wie du dir das Endresultat vorstellst?} \\
            }
        }
    }
    & & \\ \\

    \theutterance \stepcounter{utterance}  
    & & & \multicolumn{2}{p{0.3\linewidth}}{
        \cellcolor[rgb]{0.9,0.9,0.9}{
            \makecell[{{p{\linewidth}}}]{
                \texttt{\tiny{[GM$|$GM]}}
                \texttt{SPRICH: Ich sehe, dass wir beide Objekte bewegt haben. Ich denke, es ist an der Zeit, dass wir unsere Fortschritte überprüfen und sicherstellen, dass wir auf dem richtigen Weg sind. Kannst du mir bitte mitteilen, wie dein Raster aktuell aussieht, oder wie du dir das Endresultat vorstellst?} \\
            }
        }
    }
    & & \\ \\

    \theutterance \stepcounter{utterance}  
    & & \multicolumn{4}{p{0.6\linewidth}}{
        \cellcolor[rgb]{0.9,0.9,0.9}{
            \makecell[{{p{\linewidth}}}]{
                \texttt{\tiny{[GM$\rangle$P2]}}
                \texttt{'C' erfolgreich nach (5,1) verschoben. Dein Raster sieht jetzt so aus:} \\
\texttt{```} \\
\texttt{ 1234567} \\
\texttt{╔═╤═╤═╤═╗} \\
\texttt{║W│T│C│◌║ 1} \\
\texttt{╟─┴─┼─┤◌║ 2} \\
\texttt{║◌◌◌│◌└─╢ 3} \\
\texttt{╟───┘I◌◌║ 4} \\
\texttt{║H◌◌◌◌┌─╢ 5} \\
\texttt{║◌◌◌◌◌│◌║ 6} \\
\texttt{║◌◌◌◌◌│◌║ 7} \\
\texttt{╚═════╧═╝} \\
\texttt{```} \\
\texttt{Ihr seid gerade bei Runde 6 von maximal 20.} \\
\texttt{Ihr habt gemeinsam 4 von maximal 12 Strafpunkten angesammelt.} \\
\texttt{Dein Mitspieler hat dir diese Nachricht geschickt:} \\
\\ 
\\ 
\texttt{"Ich sehe, dass wir beide Objekte bewegt haben. Ich denke, es ist an der Zeit, dass wir unsere Fortschritte überprüfen und sicherstellen, dass wir auf dem richtigen Weg sind. Kannst du mir bitte mitteilen, wie dein Raster aktuell aussieht, oder wie du dir das Endresultat vorstellst?"} \\
\\ 
\\ 
\texttt{Wie lautet dein nächster Befehl?} \\
            }
        }
    }
    & \\ \\

    \theutterance \stepcounter{utterance}  
    & & & \multicolumn{4}{p{0.6\linewidth}}{
        \cellcolor[rgb]{0.9,0.9,0.9}{
            \makecell[{{p{\linewidth}}}]{
                \texttt{\tiny{[GM$\langle$P2]}}
                \texttt{SPRICH: Mein Raster sieht aktuell so aus: W, T, C sind in der ersten Zeile platziert, 'I' ist bei (4,4) und 'H' ist bei (1,5). Ich denke, wir sollten versuchen, 'H' und 'I' an die richtigen Positionen in der ersten Zeile zu bewegen, um die korrekte Reihenfolge 'C', 'H', 'I', 'T', 'W' zu erreichen. Wie sieht dein Raster aus und wie siehst du unsere nächsten Schritte?} \\
            }
        }
    }
     \\ \\

    \theutterance \stepcounter{utterance}  
    & & & \multicolumn{2}{p{0.3\linewidth}}{
        \cellcolor[rgb]{0.9,0.9,0.9}{
            \makecell[{{p{\linewidth}}}]{
                \texttt{\tiny{[GM$|$GM]}}
                \texttt{SPRICH: Mein Raster sieht aktuell so aus: W, T, C sind in der ersten Zeile platziert, 'I' ist bei (4,4) und 'H' ist bei (1,5). Ich denke, wir sollten versuchen, 'H' und 'I' an die richtigen Positionen in der ersten Zeile zu bewegen, um die korrekte Reihenfolge 'C', 'H', 'I', 'T', 'W' zu erreichen. Wie sieht dein Raster aus und wie siehst du unsere nächsten Schritte?} \\
            }
        }
    }
    & & \\ \\

    \theutterance \stepcounter{utterance}  
    & & \multicolumn{4}{p{0.6\linewidth}}{
        \cellcolor[rgb]{0.9,0.9,0.9}{
            \makecell[{{p{\linewidth}}}]{
                \texttt{\tiny{[P1$\langle$GM]}}
                \texttt{Deine Nachricht wurde an deinen Mitspieler übermittelt.} \\
\\ 
\\ 
\texttt{Ihr seid gerade bei Runde 6 von maximal 20.} \\
\texttt{Ihr habt gemeinsam 4 von maximal 12 Strafpunkten angesammelt.} \\
\texttt{Dein Mitspieler hat dir diese Nachricht geschickt:} \\
\\ 
\\ 
\texttt{"Mein Raster sieht aktuell so aus: W, T, C sind in der ersten Zeile platziert, 'I' ist bei (4,4) und 'H' ist bei (1,5). Ich denke, wir sollten versuchen, 'H' und 'I' an die richtigen Positionen in der ersten Zeile zu bewegen, um die korrekte Reihenfolge 'C', 'H', 'I', 'T', 'W' zu erreichen. Wie sieht dein Raster aus und wie siehst du unsere nächsten Schritte?"} \\
\\ 
\\ 
\texttt{Wie lautet dein nächster Befehl?} \\
            }
        }
    }
    & \\ \\

    \theutterance \stepcounter{utterance}  
    & \multicolumn{4}{p{0.6\linewidth}}{
        \cellcolor[rgb]{0.9,0.9,0.9}{
            \makecell[{{p{\linewidth}}}]{
                \texttt{\tiny{[P1$\rangle$GM]}}
                \texttt{SPRICH: Mein Raster sieht aktuell so aus: 'C' und 'H' sind in der ersten Zeile platziert, 'T' ist bei (3,3), 'I' ist bei (4,5) und 'W' ist bei (4,7). Ich denke, wir sollten uns auf die Bewegung von 'I' konzentrieren, um es an die richtige Position in der ersten Zeile zu bringen. Ich schlage vor, dass ich 'I' nach (3,1) bewege, um Platz für 'H' zu schaffen. Wie sieht das für dich aus? Sollten wir auch 'T' und 'W' anpassen, um die korrekte Reihenfolge zu erreichen?} \\
            }
        }
    }
    & & \\ \\

    \theutterance \stepcounter{utterance}  
    & & & \multicolumn{2}{p{0.3\linewidth}}{
        \cellcolor[rgb]{0.9,0.9,0.9}{
            \makecell[{{p{\linewidth}}}]{
                \texttt{\tiny{[GM$|$GM]}}
                \texttt{SPRICH: Mein Raster sieht aktuell so aus: 'C' und 'H' sind in der ersten Zeile platziert, 'T' ist bei (3,3), 'I' ist bei (4,5) und 'W' ist bei (4,7). Ich denke, wir sollten uns auf die Bewegung von 'I' konzentrieren, um es an die richtige Position in der ersten Zeile zu bringen. Ich schlage vor, dass ich 'I' nach (3,1) bewege, um Platz für 'H' zu schaffen. Wie sieht das für dich aus? Sollten wir auch 'T' und 'W' anpassen, um die korrekte Reihenfolge zu erreichen?} \\
            }
        }
    }
    & & \\ \\

    \theutterance \stepcounter{utterance}  
    & & \multicolumn{4}{p{0.6\linewidth}}{
        \cellcolor[rgb]{0.9,0.9,0.9}{
            \makecell[{{p{\linewidth}}}]{
                \texttt{\tiny{[GM$\rangle$P2]}}
                \texttt{Deine Nachricht wurde an deinen Mitspieler übermittelt.} \\
\\ 
\\ 
\texttt{Ihr seid gerade bei Runde 7 von maximal 20.} \\
\texttt{Ihr habt gemeinsam 4 von maximal 12 Strafpunkten angesammelt.} \\
\texttt{Dein Mitspieler hat dir diese Nachricht geschickt:} \\
\\ 
\\ 
\texttt{"Mein Raster sieht aktuell so aus: 'C' und 'H' sind in der ersten Zeile platziert, 'T' ist bei (3,3), 'I' ist bei (4,5) und 'W' ist bei (4,7). Ich denke, wir sollten uns auf die Bewegung von 'I' konzentrieren, um es an die richtige Position in der ersten Zeile zu bringen. Ich schlage vor, dass ich 'I' nach (3,1) bewege, um Platz für 'H' zu schaffen. Wie sieht das für dich aus? Sollten wir auch 'T' und 'W' anpassen, um die korrekte Reihenfolge zu erreichen?"} \\
\\ 
\\ 
\texttt{Wie lautet dein nächster Befehl?} \\
            }
        }
    }
    & \\ \\

    \theutterance \stepcounter{utterance}  
    & & & \multicolumn{4}{p{0.6\linewidth}}{
        \cellcolor[rgb]{0.9,0.9,0.9}{
            \makecell[{{p{\linewidth}}}]{
                \texttt{\tiny{[GM$\langle$P2]}}
                \texttt{SPRICH: Ich denke, es ist ein guter Ansatz, 'I' nach (3,1) zu bewegen, aber ich bin mir nicht sicher, ob das auf meinem Raster möglich ist. Ich denke, wir sollten auch 'T' und 'W' anpassen, um die korrekte Reihenfolge zu erreichen. Ich schlage vor, dass wir 'H' nach (2,1) bewegen, um Platz für 'I' zu schaffen. Dann könnten wir 'I' nach (3,1) bewegen und 'T' und 'W' entsprechend anpassen. Wie sieht das für dich aus? Sollten wir auch unsere Strategie anpassen, um die Objekte in der korrekten Reihenfolge zu platzieren?} \\
            }
        }
    }
     \\ \\

    \theutterance \stepcounter{utterance}  
    & & & \multicolumn{2}{p{0.3\linewidth}}{
        \cellcolor[rgb]{0.9,0.9,0.9}{
            \makecell[{{p{\linewidth}}}]{
                \texttt{\tiny{[GM$|$GM]}}
                \texttt{SPRICH: Ich denke, es ist ein guter Ansatz, 'I' nach (3,1) zu bewegen, aber ich bin mir nicht sicher, ob das auf meinem Raster möglich ist. Ich denke, wir sollten auch 'T' und 'W' anpassen, um die korrekte Reihenfolge zu erreichen. Ich schlage vor, dass wir 'H' nach (2,1) bewegen, um Platz für 'I' zu schaffen. Dann könnten wir 'I' nach (3,1) bewegen und 'T' und 'W' entsprechend anpassen. Wie sieht das für dich aus? Sollten wir auch unsere Strategie anpassen, um die Objekte in der korrekten Reihenfolge zu platzieren?} \\
            }
        }
    }
    & & \\ \\

    \theutterance \stepcounter{utterance}  
    & & \multicolumn{4}{p{0.6\linewidth}}{
        \cellcolor[rgb]{0.9,0.9,0.9}{
            \makecell[{{p{\linewidth}}}]{
                \texttt{\tiny{[P1$\langle$GM]}}
                \texttt{Deine Nachricht wurde an deinen Mitspieler übermittelt.} \\
\\ 
\\ 
\texttt{Ihr seid gerade bei Runde 7 von maximal 20.} \\
\texttt{Ihr habt gemeinsam 4 von maximal 12 Strafpunkten angesammelt.} \\
\texttt{Dein Mitspieler hat dir diese Nachricht geschickt:} \\
\\ 
\\ 
\texttt{"Ich denke, es ist ein guter Ansatz, 'I' nach (3,1) zu bewegen, aber ich bin mir nicht sicher, ob das auf meinem Raster möglich ist. Ich denke, wir sollten auch 'T' und 'W' anpassen, um die korrekte Reihenfolge zu erreichen. Ich schlage vor, dass wir 'H' nach (2,1) bewegen, um Platz für 'I' zu schaffen. Dann könnten wir 'I' nach (3,1) bewegen und 'T' und 'W' entsprechend anpassen. Wie sieht das für dich aus? Sollten wir auch unsere Strategie anpassen, um die Objekte in der korrekten Reihenfolge zu platzieren?"} \\
\\ 
\\ 
\texttt{Wie lautet dein nächster Befehl?} \\
            }
        }
    }
    & \\ \\

    \theutterance \stepcounter{utterance}  
    & \multicolumn{4}{p{0.6\linewidth}}{
        \cellcolor[rgb]{0.9,0.9,0.9}{
            \makecell[{{p{\linewidth}}}]{
                \texttt{\tiny{[P1$\rangle$GM]}}
                \texttt{SPRICH: Ich denke, das ist ein guter Vorschlag! Ich stimme zu, dass wir 'H' nach (2,1) bewegen sollten, um Platz für 'I' zu schaffen. Dann können wir 'I' nach (3,1) bewegen und 'T' und 'W' entsprechend anpassen. Ich denke, wir sollten auch unsere Strategie anpassen, um die Objekte in der korrekten Reihenfolge zu platzieren. Ich schlage vor, dass wir 'T' nach (4,1) und 'W' nach (5,1) bewegen, um die korrekte Reihenfolge 'C', 'H', 'I', 'T', 'W' zu erreichen. Wie sieht das für dich aus? Sollen wir loslegen und unsere Objekte anpassen?} \\
            }
        }
    }
    & & \\ \\

    \theutterance \stepcounter{utterance}  
    & & & \multicolumn{2}{p{0.3\linewidth}}{
        \cellcolor[rgb]{0.9,0.9,0.9}{
            \makecell[{{p{\linewidth}}}]{
                \texttt{\tiny{[GM$|$GM]}}
                \texttt{SPRICH: Ich denke, das ist ein guter Vorschlag! Ich stimme zu, dass wir 'H' nach (2,1) bewegen sollten, um Platz für 'I' zu schaffen. Dann können wir 'I' nach (3,1) bewegen und 'T' und 'W' entsprechend anpassen. Ich denke, wir sollten auch unsere Strategie anpassen, um die Objekte in der korrekten Reihenfolge zu platzieren. Ich schlage vor, dass wir 'T' nach (4,1) und 'W' nach (5,1) bewegen, um die korrekte Reihenfolge 'C', 'H', 'I', 'T', 'W' zu erreichen. Wie sieht das für dich aus? Sollen wir loslegen und unsere Objekte anpassen?} \\
            }
        }
    }
    & & \\ \\

    \theutterance \stepcounter{utterance}  
    & & \multicolumn{4}{p{0.6\linewidth}}{
        \cellcolor[rgb]{0.9,0.9,0.9}{
            \makecell[{{p{\linewidth}}}]{
                \texttt{\tiny{[GM$\rangle$P2]}}
                \texttt{Deine Nachricht wurde an deinen Mitspieler übermittelt.} \\
\\ 
\\ 
\texttt{Ihr seid gerade bei Runde 8 von maximal 20.} \\
\texttt{Ihr habt gemeinsam 4 von maximal 12 Strafpunkten angesammelt.} \\
\texttt{Dein Mitspieler hat dir diese Nachricht geschickt:} \\
\\ 
\\ 
\texttt{"Ich denke, das ist ein guter Vorschlag! Ich stimme zu, dass wir 'H' nach (2,1) bewegen sollten, um Platz für 'I' zu schaffen. Dann können wir 'I' nach (3,1) bewegen und 'T' und 'W' entsprechend anpassen. Ich denke, wir sollten auch unsere Strategie anpassen, um die Objekte in der korrekten Reihenfolge zu platzieren. Ich schlage vor, dass wir 'T' nach (4,1) und 'W' nach (5,1) bewegen, um die korrekte Reihenfolge 'C', 'H', 'I', 'T', 'W' zu erreichen. Wie sieht das für dich aus? Sollen wir loslegen und unsere Objekte anpassen?"} \\
\\ 
\\ 
\texttt{Wie lautet dein nächster Befehl?} \\
            }
        }
    }
    & \\ \\

    \theutterance \stepcounter{utterance}  
    & & & \multicolumn{4}{p{0.6\linewidth}}{
        \cellcolor[rgb]{0.9,0.9,0.9}{
            \makecell[{{p{\linewidth}}}]{
                \texttt{\tiny{[GM$\langle$P2]}}
                \texttt{SPRICH: Ich denke, das ist ein großartiger Plan! Ich stimme zu, dass wir 'H' nach (2,1) bewegen sollten, um Platz für 'I' zu schaffen. Dann können wir 'I' nach (4,1) bewegen, da 'T' bereits bei (3,1) ist. Wir sollten 'H' nach (2,1) bewegen, um Platz für 'I' zu schaffen. Ich werde 'H' nach (2,1) bewegen, um den Prozess zu starten.} \\
            }
        }
    }
     \\ \\

    \theutterance \stepcounter{utterance}  
    & & & \multicolumn{2}{p{0.3\linewidth}}{
        \cellcolor[rgb]{0.9,0.9,0.9}{
            \makecell[{{p{\linewidth}}}]{
                \texttt{\tiny{[GM$|$GM]}}
                \texttt{SPRICH: Ich denke, das ist ein großartiger Plan! Ich stimme zu, dass wir 'H' nach (2,1) bewegen sollten, um Platz für 'I' zu schaffen. Dann können wir 'I' nach (4,1) bewegen, da 'T' bereits bei (3,1) ist. Wir sollten 'H' nach (2,1) bewegen, um Platz für 'I' zu schaffen. Ich werde 'H' nach (2,1) bewegen, um den Prozess zu starten.} \\
            }
        }
    }
    & & \\ \\

    \theutterance \stepcounter{utterance}  
    & & \multicolumn{4}{p{0.6\linewidth}}{
        \cellcolor[rgb]{0.9,0.9,0.9}{
            \makecell[{{p{\linewidth}}}]{
                \texttt{\tiny{[P1$\langle$GM]}}
                \texttt{Deine Nachricht wurde an deinen Mitspieler übermittelt.} \\
\\ 
\\ 
\texttt{Ihr seid gerade bei Runde 8 von maximal 20.} \\
\texttt{Ihr habt gemeinsam 4 von maximal 12 Strafpunkten angesammelt.} \\
\texttt{Dein Mitspieler hat dir diese Nachricht geschickt:} \\
\\ 
\\ 
\texttt{"Ich denke, das ist ein großartiger Plan! Ich stimme zu, dass wir 'H' nach (2,1) bewegen sollten, um Platz für 'I' zu schaffen. Dann können wir 'I' nach (4,1) bewegen, da 'T' bereits bei (3,1) ist. Wir sollten 'H' nach (2,1) bewegen, um Platz für 'I' zu schaffen. Ich werde 'H' nach (2,1) bewegen, um den Prozess zu starten."} \\
\\ 
\\ 
\texttt{Wie lautet dein nächster Befehl?} \\
            }
        }
    }
    & \\ \\

    \theutterance \stepcounter{utterance}  
    & \multicolumn{4}{p{0.6\linewidth}}{
        \cellcolor[rgb]{0.9,0.9,0.9}{
            \makecell[{{p{\linewidth}}}]{
                \texttt{\tiny{[P1$\rangle$GM]}}
                \texttt{VERSCHIEBE: I, (3, 1)} \\
            }
        }
    }
    & & \\ \\

    \theutterance \stepcounter{utterance}  
    & & & \multicolumn{2}{p{0.3\linewidth}}{
        \cellcolor[rgb]{0.9,0.9,0.9}{
            \makecell[{{p{\linewidth}}}]{
                \texttt{\tiny{[GM$|$GM]}}
                \texttt{VERSCHIEBE: I, (3, 1)} \\
            }
        }
    }
    & & \\ \\

    \theutterance \stepcounter{utterance}  
    & & & \multicolumn{2}{p{0.3\linewidth}}{
        \cellcolor[rgb]{0.9,0.9,0.9}{
            \makecell[{{p{\linewidth}}}]{
                \texttt{\tiny{[GM$|$GM]}}
                \texttt{Strafpunkt: (3,1) ist nicht leer, sondern enthält 'H'.} \\
\\ 
\texttt{Ihr habt gemeinsam 5 von maximal 12 Strafpunkten angesammelt.} \\
\texttt{Bitte versuche es noch mal!} \\
            }
        }
    }
    & & \\ \\

    \theutterance \stepcounter{utterance}  
    & & \multicolumn{4}{p{0.6\linewidth}}{
        \cellcolor[rgb]{0.9,0.9,0.9}{
            \makecell[{{p{\linewidth}}}]{
                \texttt{\tiny{[P1$\langle$GM]}}
                \texttt{Strafpunkt: (3,1) ist nicht leer, sondern enthält 'H'.} \\
\\ 
\texttt{Ihr habt gemeinsam 5 von maximal 12 Strafpunkten angesammelt.} \\
\texttt{Bitte versuche es noch mal!} \\
            }
        }
    }
    & \\ \\

    \theutterance \stepcounter{utterance}  
    & \multicolumn{4}{p{0.6\linewidth}}{
        \cellcolor[rgb]{0.9,0.9,0.9}{
            \makecell[{{p{\linewidth}}}]{
                \texttt{\tiny{[P1$\rangle$GM]}}
                \texttt{VERSCHIEBE: I, (4, 1)} \\
            }
        }
    }
    & & \\ \\

    \theutterance \stepcounter{utterance}  
    & & & \multicolumn{2}{p{0.3\linewidth}}{
        \cellcolor[rgb]{0.9,0.9,0.9}{
            \makecell[{{p{\linewidth}}}]{
                \texttt{\tiny{[GM$|$GM]}}
                \texttt{VERSCHIEBE: I, (4, 1)} \\
            }
        }
    }
    & & \\ \\

    \theutterance \stepcounter{utterance}  
    & & & \multicolumn{2}{p{0.3\linewidth}}{
        \cellcolor[rgb]{0.9,0.9,0.9}{
            \makecell[{{p{\linewidth}}}]{
                \texttt{\tiny{[GM$|$GM]}}
                \texttt{Strafpunkt: (4,1) ist nicht leer, sondern enthält '│'.} \\
\\ 
\texttt{Ihr habt gemeinsam 6 von maximal 12 Strafpunkten angesammelt.} \\
\texttt{Bitte versuche es noch mal!} \\
            }
        }
    }
    & & \\ \\

    \theutterance \stepcounter{utterance}  
    & & \multicolumn{4}{p{0.6\linewidth}}{
        \cellcolor[rgb]{0.9,0.9,0.9}{
            \makecell[{{p{\linewidth}}}]{
                \texttt{\tiny{[P1$\langle$GM]}}
                \texttt{Strafpunkt: (4,1) ist nicht leer, sondern enthält '│'.} \\
\\ 
\texttt{Ihr habt gemeinsam 6 von maximal 12 Strafpunkten angesammelt.} \\
\texttt{Bitte versuche es noch mal!} \\
            }
        }
    }
    & \\ \\

    \theutterance \stepcounter{utterance}  
    & \multicolumn{4}{p{0.6\linewidth}}{
        \cellcolor[rgb]{0.9,0.9,0.9}{
            \makecell[{{p{\linewidth}}}]{
                \texttt{\tiny{[P1$\rangle$GM]}}
                \texttt{SPRICH: Ich denke, ich habe einen Fehler gemacht. Ich sollte mich erst vergewissern, ob die Zielposition leer ist. Kannst du mir bitte mitteilen, welche Positionen auf deinem Raster noch leer sind? Ich möchte mich vergewissern, dass ich die richtige Position für 'I' wähle.} \\
            }
        }
    }
    & & \\ \\

    \theutterance \stepcounter{utterance}  
    & & & \multicolumn{2}{p{0.3\linewidth}}{
        \cellcolor[rgb]{0.9,0.9,0.9}{
            \makecell[{{p{\linewidth}}}]{
                \texttt{\tiny{[GM$|$GM]}}
                \texttt{SPRICH: Ich denke, ich habe einen Fehler gemacht. Ich sollte mich erst vergewissern, ob die Zielposition leer ist. Kannst du mir bitte mitteilen, welche Positionen auf deinem Raster noch leer sind? Ich möchte mich vergewissern, dass ich die richtige Position für 'I' wähle.} \\
            }
        }
    }
    & & \\ \\

    \theutterance \stepcounter{utterance}  
    & & \multicolumn{4}{p{0.6\linewidth}}{
        \cellcolor[rgb]{0.9,0.9,0.9}{
            \makecell[{{p{\linewidth}}}]{
                \texttt{\tiny{[GM$\rangle$P2]}}
                \texttt{Deine Nachricht wurde an deinen Mitspieler übermittelt.} \\
\\ 
\\ 
\texttt{Ihr seid gerade bei Runde 9 von maximal 20.} \\
\texttt{Ihr habt gemeinsam 6 von maximal 12 Strafpunkten angesammelt.} \\
\texttt{Dein Mitspieler hat dir diese Nachricht geschickt:} \\
\\ 
\\ 
\texttt{"Ich denke, ich habe einen Fehler gemacht. Ich sollte mich erst vergewissern, ob die Zielposition leer ist. Kannst du mir bitte mitteilen, welche Positionen auf deinem Raster noch leer sind? Ich möchte mich vergewissern, dass ich die richtige Position für 'I' wähle."} \\
\\ 
\\ 
\texttt{Wie lautet dein nächster Befehl?} \\
            }
        }
    }
    & \\ \\

    \theutterance \stepcounter{utterance}  
    & & & \multicolumn{4}{p{0.6\linewidth}}{
        \cellcolor[rgb]{0.9,0.9,0.9}{
            \makecell[{{p{\linewidth}}}]{
                \texttt{\tiny{[GM$\langle$P2]}}
                \texttt{SPRICH: Auf meinem Raster sind noch die Positionen (4,1), (5,2), (5,3), (5,4), (5,5), (5,6) und (5,7) leer. Ich denke, wir sollten 'H' nach (2,1) bewegen, um Platz für 'I' zu schaffen. Dann können wir 'I' nach (4,1) bewegen. Bitte beachte, dass ich möglicherweise weitere Anpassungen vornehmen werde, um die Objekte in der richtigen Reihenfolge zu platzieren.} \\
            }
        }
    }
     \\ \\

    \theutterance \stepcounter{utterance}  
    & & & \multicolumn{2}{p{0.3\linewidth}}{
        \cellcolor[rgb]{0.9,0.9,0.9}{
            \makecell[{{p{\linewidth}}}]{
                \texttt{\tiny{[GM$|$GM]}}
                \texttt{SPRICH: Auf meinem Raster sind noch die Positionen (4,1), (5,2), (5,3), (5,4), (5,5), (5,6) und (5,7) leer. Ich denke, wir sollten 'H' nach (2,1) bewegen, um Platz für 'I' zu schaffen. Dann können wir 'I' nach (4,1) bewegen. Bitte beachte, dass ich möglicherweise weitere Anpassungen vornehmen werde, um die Objekte in der richtigen Reihenfolge zu platzieren.} \\
            }
        }
    }
    & & \\ \\

    \theutterance \stepcounter{utterance}  
    & & \multicolumn{4}{p{0.6\linewidth}}{
        \cellcolor[rgb]{0.9,0.9,0.9}{
            \makecell[{{p{\linewidth}}}]{
                \texttt{\tiny{[P1$\langle$GM]}}
                \texttt{Deine Nachricht wurde an deinen Mitspieler übermittelt.} \\
\\ 
\\ 
\texttt{Ihr seid gerade bei Runde 9 von maximal 20.} \\
\texttt{Ihr habt gemeinsam 6 von maximal 12 Strafpunkten angesammelt.} \\
\texttt{Dein Mitspieler hat dir diese Nachricht geschickt:} \\
\\ 
\\ 
\texttt{"Auf meinem Raster sind noch die Positionen (4,1), (5,2), (5,3), (5,4), (5,5), (5,6) und (5,7) leer. Ich denke, wir sollten 'H' nach (2,1) bewegen, um Platz für 'I' zu schaffen. Dann können wir 'I' nach (4,1) bewegen. Bitte beachte, dass ich möglicherweise weitere Anpassungen vornehmen werde, um die Objekte in der richtigen Reihenfolge zu platzieren."} \\
\\ 
\\ 
\texttt{Wie lautet dein nächster Befehl?} \\
            }
        }
    }
    & \\ \\

    \theutterance \stepcounter{utterance}  
    & \multicolumn{4}{p{0.6\linewidth}}{
        \cellcolor[rgb]{0.9,0.9,0.9}{
            \makecell[{{p{\linewidth}}}]{
                \texttt{\tiny{[P1$\rangle$GM]}}
                \texttt{VERSCHIEBE: H, (2, 1)} \\
            }
        }
    }
    & & \\ \\

    \theutterance \stepcounter{utterance}  
    & & & \multicolumn{2}{p{0.3\linewidth}}{
        \cellcolor[rgb]{0.9,0.9,0.9}{
            \makecell[{{p{\linewidth}}}]{
                \texttt{\tiny{[GM$|$GM]}}
                \texttt{VERSCHIEBE: H, (2, 1)} \\
            }
        }
    }
    & & \\ \\

    \theutterance \stepcounter{utterance}  
    & & & \multicolumn{2}{p{0.3\linewidth}}{
        \cellcolor[rgb]{0.9,0.9,0.9}{
            \makecell[{{p{\linewidth}}}]{
                \texttt{\tiny{[GM$|$GM]}}
                \texttt{Strafpunkt: (2,1) ist nicht leer, sondern enthält '│'.} \\
\\ 
\texttt{Ihr habt gemeinsam 7 von maximal 12 Strafpunkten angesammelt.} \\
\texttt{Bitte versuche es noch mal!} \\
            }
        }
    }
    & & \\ \\

    \theutterance \stepcounter{utterance}  
    & & \multicolumn{4}{p{0.6\linewidth}}{
        \cellcolor[rgb]{0.9,0.9,0.9}{
            \makecell[{{p{\linewidth}}}]{
                \texttt{\tiny{[P1$\langle$GM]}}
                \texttt{Strafpunkt: (2,1) ist nicht leer, sondern enthält '│'.} \\
\\ 
\texttt{Ihr habt gemeinsam 7 von maximal 12 Strafpunkten angesammelt.} \\
\texttt{Bitte versuche es noch mal!} \\
            }
        }
    }
    & \\ \\

    \theutterance \stepcounter{utterance}  
    & \multicolumn{4}{p{0.6\linewidth}}{
        \cellcolor[rgb]{0.9,0.9,0.9}{
            \makecell[{{p{\linewidth}}}]{
                \texttt{\tiny{[P1$\rangle$GM]}}
                \texttt{VERSCHIEBE: I, (5, 1)} \\
            }
        }
    }
    & & \\ \\

    \theutterance \stepcounter{utterance}  
    & & & \multicolumn{2}{p{0.3\linewidth}}{
        \cellcolor[rgb]{0.9,0.9,0.9}{
            \makecell[{{p{\linewidth}}}]{
                \texttt{\tiny{[GM$|$GM]}}
                \texttt{VERSCHIEBE: I, (5, 1)} \\
            }
        }
    }
    & & \\ \\

    \theutterance \stepcounter{utterance}  
    & & & \multicolumn{2}{p{0.3\linewidth}}{
        \cellcolor[rgb]{0.9,0.9,0.9}{
            \makecell[{{p{\linewidth}}}]{
                \texttt{\tiny{[GM$|$GM]}}
                \texttt{'I' erfolgreich nach (5,1) verschoben. Dein Raster sieht jetzt so aus:} \\
\texttt{```} \\
\texttt{ 1234567} \\
\texttt{╔═╤═╤═╤═╗} \\
\texttt{║C│H│I│◌║ 1} \\
\texttt{╟─┴─┼─┤◌║ 2} \\
\texttt{║◌T◌│◌└─╢ 3} \\
\texttt{╟───┘◌◌◌║ 4} \\
\texttt{║◌◌◌◌◌┌─╢ 5} \\
\texttt{║◌◌◌◌◌│◌║ 6} \\
\texttt{║◌◌◌W◌│◌║ 7} \\
\texttt{╚═════╧═╝} \\
\texttt{```} \\
            }
        }
    }
    & & \\ \\

    \theutterance \stepcounter{utterance}  
    & & \multicolumn{4}{p{0.6\linewidth}}{
        \cellcolor[rgb]{0.9,0.9,0.9}{
            \makecell[{{p{\linewidth}}}]{
                \texttt{\tiny{[GM$\rangle$P2]}}
                \texttt{Deine Nachricht wurde an deinen Mitspieler übermittelt.} \\
\\ 
\\ 
\texttt{Ihr seid gerade bei Runde 10 von maximal 20.} \\
\texttt{Ihr habt gemeinsam 7 von maximal 12 Strafpunkten angesammelt.} \\
\texttt{Dein Mitspieler hat ein Objekt auf seinem Raster bewegt.} \\
\\ 
\\ 
\texttt{Wie lautet dein nächster Befehl?} \\
            }
        }
    }
    & \\ \\

    \theutterance \stepcounter{utterance}  
    & & & \multicolumn{4}{p{0.6\linewidth}}{
        \cellcolor[rgb]{0.9,0.9,0.9}{
            \makecell[{{p{\linewidth}}}]{
                \texttt{\tiny{[GM$\langle$P2]}}
                \texttt{VERSCHIEBE: H, (2, 1)} \\
            }
        }
    }
     \\ \\

    \theutterance \stepcounter{utterance}  
    & & & \multicolumn{2}{p{0.3\linewidth}}{
        \cellcolor[rgb]{0.9,0.9,0.9}{
            \makecell[{{p{\linewidth}}}]{
                \texttt{\tiny{[GM$|$GM]}}
                \texttt{VERSCHIEBE: H, (2, 1)} \\
            }
        }
    }
    & & \\ \\

    \theutterance \stepcounter{utterance}  
    & & & \multicolumn{2}{p{0.3\linewidth}}{
        \cellcolor[rgb]{0.9,0.9,0.9}{
            \makecell[{{p{\linewidth}}}]{
                \texttt{\tiny{[GM$|$GM]}}
                \texttt{Strafpunkt: (2,1) ist nicht leer, sondern enthält '│'.} \\
\\ 
\texttt{Ihr habt gemeinsam 8 von maximal 12 Strafpunkten angesammelt.} \\
\texttt{Bitte versuche es noch mal!} \\
            }
        }
    }
    & & \\ \\

    \theutterance \stepcounter{utterance}  
    & & \multicolumn{4}{p{0.6\linewidth}}{
        \cellcolor[rgb]{0.9,0.9,0.9}{
            \makecell[{{p{\linewidth}}}]{
                \texttt{\tiny{[GM$\rangle$P2]}}
                \texttt{Strafpunkt: (2,1) ist nicht leer, sondern enthält '│'.} \\
\\ 
\texttt{Ihr habt gemeinsam 8 von maximal 12 Strafpunkten angesammelt.} \\
\texttt{Bitte versuche es noch mal!} \\
            }
        }
    }
    & \\ \\

    \theutterance \stepcounter{utterance}  
    & & & \multicolumn{4}{p{0.6\linewidth}}{
        \cellcolor[rgb]{0.9,0.9,0.9}{
            \makecell[{{p{\linewidth}}}]{
                \texttt{\tiny{[GM$\langle$P2]}}
                \texttt{VERSCHIEBE: H, (2, 5)} \\
            }
        }
    }
     \\ \\

    \theutterance \stepcounter{utterance}  
    & & & \multicolumn{2}{p{0.3\linewidth}}{
        \cellcolor[rgb]{0.9,0.9,0.9}{
            \makecell[{{p{\linewidth}}}]{
                \texttt{\tiny{[GM$|$GM]}}
                \texttt{VERSCHIEBE: H, (2, 5)} \\
            }
        }
    }
    & & \\ \\

    \theutterance \stepcounter{utterance}  
    & & & \multicolumn{2}{p{0.3\linewidth}}{
        \cellcolor[rgb]{0.9,0.9,0.9}{
            \makecell[{{p{\linewidth}}}]{
                \texttt{\tiny{[GM$|$GM]}}
                \texttt{'H' erfolgreich nach (2,5) verschoben. Dein Raster sieht jetzt so aus:} \\
\texttt{```} \\
\texttt{ 1234567} \\
\texttt{╔═╤═╤═╤═╗} \\
\texttt{║W│T│C│◌║ 1} \\
\texttt{╟─┴─┼─┤◌║ 2} \\
\texttt{║◌◌◌│◌└─╢ 3} \\
\texttt{╟───┘I◌◌║ 4} \\
\texttt{║◌H◌◌◌┌─╢ 5} \\
\texttt{║◌◌◌◌◌│◌║ 6} \\
\texttt{║◌◌◌◌◌│◌║ 7} \\
\texttt{╚═════╧═╝} \\
\texttt{```} \\
            }
        }
    }
    & & \\ \\

    \theutterance \stepcounter{utterance}  
    & & \multicolumn{4}{p{0.6\linewidth}}{
        \cellcolor[rgb]{0.9,0.9,0.9}{
            \makecell[{{p{\linewidth}}}]{
                \texttt{\tiny{[P1$\langle$GM]}}
                \texttt{'I' erfolgreich nach (5,1) verschoben. Dein Raster sieht jetzt so aus:} \\
\texttt{```} \\
\texttt{ 1234567} \\
\texttt{╔═╤═╤═╤═╗} \\
\texttt{║C│H│I│◌║ 1} \\
\texttt{╟─┴─┼─┤◌║ 2} \\
\texttt{║◌T◌│◌└─╢ 3} \\
\texttt{╟───┘◌◌◌║ 4} \\
\texttt{║◌◌◌◌◌┌─╢ 5} \\
\texttt{║◌◌◌◌◌│◌║ 6} \\
\texttt{║◌◌◌W◌│◌║ 7} \\
\texttt{╚═════╧═╝} \\
\texttt{```} \\
\texttt{Ihr seid gerade bei Runde 10 von maximal 20.} \\
\texttt{Ihr habt gemeinsam 8 von maximal 12 Strafpunkten angesammelt.} \\
\texttt{Dein Mitspieler hat ein Objekt auf seinem Raster bewegt.} \\
\\ 
\\ 
\texttt{Wie lautet dein nächster Befehl?} \\
            }
        }
    }
    & \\ \\

    \theutterance \stepcounter{utterance}  
    & \multicolumn{4}{p{0.6\linewidth}}{
        \cellcolor[rgb]{0.9,0.9,0.9}{
            \makecell[{{p{\linewidth}}}]{
                \texttt{\tiny{[P1$\rangle$GM]}}
                \texttt{SPRICH: Ich denke, wir sind auf dem richtigen Weg! Die Objekte 'C', 'H' und 'I' sind jetzt in der ersten Zeile. Ich schlage vor, dass wir 'T' und 'W' auch in die erste Zeile bewegen, um die korrekte Reihenfolge 'C', 'H', 'I', 'T', 'W' zu erreichen. Wie sieht das für dich aus? Sollten wir 'T' nach (4,1) und 'W' nach (5,1) bewegen, oder hast du eine andere Idee?} \\
            }
        }
    }
    & & \\ \\

    \theutterance \stepcounter{utterance}  
    & & & \multicolumn{2}{p{0.3\linewidth}}{
        \cellcolor[rgb]{0.9,0.9,0.9}{
            \makecell[{{p{\linewidth}}}]{
                \texttt{\tiny{[GM$|$GM]}}
                \texttt{SPRICH: Ich denke, wir sind auf dem richtigen Weg! Die Objekte 'C', 'H' und 'I' sind jetzt in der ersten Zeile. Ich schlage vor, dass wir 'T' und 'W' auch in die erste Zeile bewegen, um die korrekte Reihenfolge 'C', 'H', 'I', 'T', 'W' zu erreichen. Wie sieht das für dich aus? Sollten wir 'T' nach (4,1) und 'W' nach (5,1) bewegen, oder hast du eine andere Idee?} \\
            }
        }
    }
    & & \\ \\

    \theutterance \stepcounter{utterance}  
    & & \multicolumn{4}{p{0.6\linewidth}}{
        \cellcolor[rgb]{0.9,0.9,0.9}{
            \makecell[{{p{\linewidth}}}]{
                \texttt{\tiny{[GM$\rangle$P2]}}
                \texttt{'H' erfolgreich nach (2,5) verschoben. Dein Raster sieht jetzt so aus:} \\
\texttt{```} \\
\texttt{ 1234567} \\
\texttt{╔═╤═╤═╤═╗} \\
\texttt{║W│T│C│◌║ 1} \\
\texttt{╟─┴─┼─┤◌║ 2} \\
\texttt{║◌◌◌│◌└─╢ 3} \\
\texttt{╟───┘I◌◌║ 4} \\
\texttt{║◌H◌◌◌┌─╢ 5} \\
\texttt{║◌◌◌◌◌│◌║ 6} \\
\texttt{║◌◌◌◌◌│◌║ 7} \\
\texttt{╚═════╧═╝} \\
\texttt{```} \\
\texttt{Ihr seid gerade bei Runde 11 von maximal 20.} \\
\texttt{Ihr habt gemeinsam 8 von maximal 12 Strafpunkten angesammelt.} \\
\texttt{Dein Mitspieler hat dir diese Nachricht geschickt:} \\
\\ 
\\ 
\texttt{"Ich denke, wir sind auf dem richtigen Weg! Die Objekte 'C', 'H' und 'I' sind jetzt in der ersten Zeile. Ich schlage vor, dass wir 'T' und 'W' auch in die erste Zeile bewegen, um die korrekte Reihenfolge 'C', 'H', 'I', 'T', 'W' zu erreichen. Wie sieht das für dich aus? Sollten wir 'T' nach (4,1) und 'W' nach (5,1) bewegen, oder hast du eine andere Idee?"} \\
\\ 
\\ 
\texttt{Wie lautet dein nächster Befehl?} \\
            }
        }
    }
    & \\ \\

    \theutterance \stepcounter{utterance}  
    & & & \multicolumn{4}{p{0.6\linewidth}}{
        \cellcolor[rgb]{0.9,0.9,0.9}{
            \makecell[{{p{\linewidth}}}]{
                \texttt{\tiny{[GM$\langle$P2]}}
                \texttt{SPRICH: Ich denke, das ist ein guter Plan! Ich stimme zu, dass wir 'T' und 'W' in die erste Zeile bewegen sollten, um die korrekte Reihenfolge 'C', 'H', 'I', 'T', 'W' zu erreichen. Allerdings ist 'T' bereits bei (3,1) und 'W' bei (1,1). Wir sollten 'I' nach (3,1) bewegen und 'T' nach (4,1) und dann 'W' bleibt an der aktuellen Position oder wird ggf. nach (5,1) verschoben. Ich denke, wir sollten 'I' nach (3,1) bewegen, um den Prozess zu starten. Ich werde 'I' nach (3,1) bewegen.} \\
            }
        }
    }
     \\ \\

    \theutterance \stepcounter{utterance}  
    & & & \multicolumn{2}{p{0.3\linewidth}}{
        \cellcolor[rgb]{0.9,0.9,0.9}{
            \makecell[{{p{\linewidth}}}]{
                \texttt{\tiny{[GM$|$GM]}}
                \texttt{SPRICH: Ich denke, das ist ein guter Plan! Ich stimme zu, dass wir 'T' und 'W' in die erste Zeile bewegen sollten, um die korrekte Reihenfolge 'C', 'H', 'I', 'T', 'W' zu erreichen. Allerdings ist 'T' bereits bei (3,1) und 'W' bei (1,1). Wir sollten 'I' nach (3,1) bewegen und 'T' nach (4,1) und dann 'W' bleibt an der aktuellen Position oder wird ggf. nach (5,1) verschoben. Ich denke, wir sollten 'I' nach (3,1) bewegen, um den Prozess zu starten. Ich werde 'I' nach (3,1) bewegen.} \\
            }
        }
    }
    & & \\ \\

    \theutterance \stepcounter{utterance}  
    & & \multicolumn{4}{p{0.6\linewidth}}{
        \cellcolor[rgb]{0.9,0.9,0.9}{
            \makecell[{{p{\linewidth}}}]{
                \texttt{\tiny{[P1$\langle$GM]}}
                \texttt{Deine Nachricht wurde an deinen Mitspieler übermittelt.} \\
\\ 
\\ 
\texttt{Ihr seid gerade bei Runde 11 von maximal 20.} \\
\texttt{Ihr habt gemeinsam 8 von maximal 12 Strafpunkten angesammelt.} \\
\texttt{Dein Mitspieler hat dir diese Nachricht geschickt:} \\
\\ 
\\ 
\texttt{"Ich denke, das ist ein guter Plan! Ich stimme zu, dass wir 'T' und 'W' in die erste Zeile bewegen sollten, um die korrekte Reihenfolge 'C', 'H', 'I', 'T', 'W' zu erreichen. Allerdings ist 'T' bereits bei (3,1) und 'W' bei (1,1). Wir sollten 'I' nach (3,1) bewegen und 'T' nach (4,1) und dann 'W' bleibt an der aktuellen Position oder wird ggf. nach (5,1) verschoben. Ich denke, wir sollten 'I' nach (3,1) bewegen, um den Prozess zu starten. Ich werde 'I' nach (3,1) bewegen."} \\
\\ 
\\ 
\texttt{Wie lautet dein nächster Befehl?} \\
            }
        }
    }
    & \\ \\

    \theutterance \stepcounter{utterance}  
    & \multicolumn{4}{p{0.6\linewidth}}{
        \cellcolor[rgb]{0.9,0.9,0.9}{
            \makecell[{{p{\linewidth}}}]{
                \texttt{\tiny{[P1$\rangle$GM]}}
                \texttt{VERSCHIEBE: T, (4, 1)} \\
            }
        }
    }
    & & \\ \\

    \theutterance \stepcounter{utterance}  
    & & & \multicolumn{2}{p{0.3\linewidth}}{
        \cellcolor[rgb]{0.9,0.9,0.9}{
            \makecell[{{p{\linewidth}}}]{
                \texttt{\tiny{[GM$|$GM]}}
                \texttt{VERSCHIEBE: T, (4, 1)} \\
            }
        }
    }
    & & \\ \\

    \theutterance \stepcounter{utterance}  
    & & & \multicolumn{2}{p{0.3\linewidth}}{
        \cellcolor[rgb]{0.9,0.9,0.9}{
            \makecell[{{p{\linewidth}}}]{
                \texttt{\tiny{[GM$|$GM]}}
                \texttt{Strafpunkt: (4,1) ist nicht leer, sondern enthält '│'.} \\
\\ 
\texttt{Ihr habt gemeinsam 9 von maximal 12 Strafpunkten angesammelt.} \\
\texttt{Bitte versuche es noch mal!} \\
            }
        }
    }
    & & \\ \\

    \theutterance \stepcounter{utterance}  
    & & \multicolumn{4}{p{0.6\linewidth}}{
        \cellcolor[rgb]{0.9,0.9,0.9}{
            \makecell[{{p{\linewidth}}}]{
                \texttt{\tiny{[P1$\langle$GM]}}
                \texttt{Strafpunkt: (4,1) ist nicht leer, sondern enthält '│'.} \\
\\ 
\texttt{Ihr habt gemeinsam 9 von maximal 12 Strafpunkten angesammelt.} \\
\texttt{Bitte versuche es noch mal!} \\
            }
        }
    }
    & \\ \\

    \theutterance \stepcounter{utterance}  
    & \multicolumn{4}{p{0.6\linewidth}}{
        \cellcolor[rgb]{0.9,0.9,0.9}{
            \makecell[{{p{\linewidth}}}]{
                \texttt{\tiny{[P1$\rangle$GM]}}
                \texttt{SPRICH: Ich denke, ich habe wieder einen Fehler gemacht. Ich sollte mich erst vergewissern, ob die Zielposition leer ist. Ich denke, wir sollten 'T' nach (5,1) bewegen, aber ich sehe, dass 'I' bereits dort ist. Ich bin verwirrt. Kannst du mir bitte mitteilen, wie dein Raster aktuell aussieht und welche Schritte wir als Nächstes unternehmen sollten?} \\
            }
        }
    }
    & & \\ \\

    \theutterance \stepcounter{utterance}  
    & & & \multicolumn{2}{p{0.3\linewidth}}{
        \cellcolor[rgb]{0.9,0.9,0.9}{
            \makecell[{{p{\linewidth}}}]{
                \texttt{\tiny{[GM$|$GM]}}
                \texttt{SPRICH: Ich denke, ich habe wieder einen Fehler gemacht. Ich sollte mich erst vergewissern, ob die Zielposition leer ist. Ich denke, wir sollten 'T' nach (5,1) bewegen, aber ich sehe, dass 'I' bereits dort ist. Ich bin verwirrt. Kannst du mir bitte mitteilen, wie dein Raster aktuell aussieht und welche Schritte wir als Nächstes unternehmen sollten?} \\
            }
        }
    }
    & & \\ \\

    \theutterance \stepcounter{utterance}  
    & & \multicolumn{4}{p{0.6\linewidth}}{
        \cellcolor[rgb]{0.9,0.9,0.9}{
            \makecell[{{p{\linewidth}}}]{
                \texttt{\tiny{[GM$\rangle$P2]}}
                \texttt{Deine Nachricht wurde an deinen Mitspieler übermittelt.} \\
\\ 
\\ 
\texttt{Ihr seid gerade bei Runde 12 von maximal 20.} \\
\texttt{Ihr habt gemeinsam 9 von maximal 12 Strafpunkten angesammelt.} \\
\texttt{Dein Mitspieler hat dir diese Nachricht geschickt:} \\
\\ 
\\ 
\texttt{"Ich denke, ich habe wieder einen Fehler gemacht. Ich sollte mich erst vergewissern, ob die Zielposition leer ist. Ich denke, wir sollten 'T' nach (5,1) bewegen, aber ich sehe, dass 'I' bereits dort ist. Ich bin verwirrt. Kannst du mir bitte mitteilen, wie dein Raster aktuell aussieht und welche Schritte wir als Nächstes unternehmen sollten?"} \\
\\ 
\\ 
\texttt{Wie lautet dein nächster Befehl?} \\
            }
        }
    }
    & \\ \\

    \theutterance \stepcounter{utterance}  
    & & & \multicolumn{4}{p{0.6\linewidth}}{
        \cellcolor[rgb]{0.9,0.9,0.9}{
            \makecell[{{p{\linewidth}}}]{
                \texttt{\tiny{[GM$\langle$P2]}}
                \texttt{SPRICH: Mein Raster sieht aktuell so aus: } \\
\texttt{```} \\
\texttt{ 1234567} \\
\texttt{╔═╤═╤═╤═╗} \\
\texttt{║W│T│C│◌║ 1} \\
\texttt{╟─┴─┼─┤◌║ 2} \\
\texttt{║◌◌◌│◌└─╢ 3} \\
\texttt{╟───┘I◌◌║ 4} \\
\texttt{║◌H◌◌◌┌─╢ 5} \\
\texttt{║◌◌◌◌◌│◌║ 6} \\
\texttt{║◌◌◌◌◌│◌║ 7} \\
\texttt{╚═════╧═╝} \\
\texttt{```} \\
\texttt{Ich denke, wir sollten 'I' nach (4,1) bewegen, um Platz für 'T' zu schaffen. Dann können wir 'T' nach (4,1) bewegen und 'I' nach (5,1) bewegen, um die korrekte Reihenfolge 'C', 'H', 'I', 'T', 'W' zu erreichen. Bitte beachte, dass ich möglicherweise weitere Anpassungen vornehmen werde, um die Objekte in der richtigen Reihenfolge zu platzieren. Ich werde 'I' nach (4,1) bewegen, um den Prozess zu starten.} \\
            }
        }
    }
     \\ \\

    \theutterance \stepcounter{utterance}  
    & & & \multicolumn{2}{p{0.3\linewidth}}{
        \cellcolor[rgb]{0.9,0.9,0.9}{
            \makecell[{{p{\linewidth}}}]{
                \texttt{\tiny{[GM$|$GM]}}
                \texttt{SPRICH: Mein Raster sieht aktuell so aus: } \\
\texttt{```} \\
\texttt{ 1234567} \\
\texttt{╔═╤═╤═╤═╗} \\
\texttt{║W│T│C│◌║ 1} \\
\texttt{╟─┴─┼─┤◌║ 2} \\
\texttt{║◌◌◌│◌└─╢ 3} \\
\texttt{╟───┘I◌◌║ 4} \\
\texttt{║◌H◌◌◌┌─╢ 5} \\
\texttt{║◌◌◌◌◌│◌║ 6} \\
\texttt{║◌◌◌◌◌│◌║ 7} \\
\texttt{╚═════╧═╝} \\
\texttt{```} \\
\texttt{Ich denke, wir sollten 'I' nach (4,1) bewegen, um Platz für 'T' zu schaffen. Dann können wir 'T' nach (4,1) bewegen und 'I' nach (5,1) bewegen, um die korrekte Reihenfolge 'C', 'H', 'I', 'T', 'W' zu erreichen. Bitte beachte, dass ich möglicherweise weitere Anpassungen vornehmen werde, um die Objekte in der richtigen Reihenfolge zu platzieren. Ich werde 'I' nach (4,1) bewegen, um den Prozess zu starten.} \\
            }
        }
    }
    & & \\ \\

    \theutterance \stepcounter{utterance}  
    & & & \multicolumn{2}{p{0.3\linewidth}}{
        \cellcolor[rgb]{0.9,0.9,0.9}{
            \makecell[{{p{\linewidth}}}]{
                \texttt{\tiny{[GM$|$GM]}}
                \texttt{Invalid format: tail is not empty: '} \\
\\ 
\texttt{ 1234567} \\
\texttt{╔═╤═╤═╤═╗} \\
\texttt{║W│T│C│◌║ 1} \\
\texttt{╟─┴─┼─┤◌║ 2} \\
\texttt{║◌◌◌│◌└─╢ 3} \\
\texttt{╟───┘I◌◌║ 4} \\
\texttt{║◌H◌◌◌┌─╢ 5} \\
\texttt{║◌◌◌◌◌│◌║ 6} \\
\texttt{║◌◌◌◌◌│◌║ 7} \\
\texttt{╚═════╧═╝} \\
\\ 
\texttt{Ich denke, wir sollten 'I' nach (4,1) bewegen, um Platz für 'T' zu schaffen. Dann können wir 'T' nach (4,1) bewegen und 'I' nach (5,1) bewegen, um die korrekte Reihenfolge 'C', 'H', 'I', 'T', 'W' zu erreichen. Bitte beachte, dass ich möglicherweise weitere Anpassungen vornehmen werde, um die Objekte in der richtigen Reihenfolge zu platzieren. Ich werde 'I' nach (4,1) bewegen, um den Prozess zu starten.} \\
            }
        }
    }
    & & \\ \\

    \theutterance \stepcounter{utterance}  
    & & \multicolumn{4}{p{0.6\linewidth}}{
        \cellcolor[rgb]{0.9,0.9,0.9}{
            \makecell[{{p{\linewidth}}}]{
                \texttt{\tiny{[GM$\rangle$P2]}}
                \texttt{Strafpunkt: Deine Nachricht darf nichts nach dem Befehl enthalten! Stelle sicher, dass deine Antwort ausschließlich einen der beiden Befehle `SPRICH: <NACHRICHT>` und `VERSCHIEBE: <OBJECT>, (<X>, <Y>)` und sonst nichts enthält!} \\
\\ 
\texttt{Ihr habt gemeinsam 10 von maximal 12 Strafpunkten angesammelt.} \\
\texttt{Bitte versuche es noch mal!} \\
            }
        }
    }
    & \\ \\

    \theutterance \stepcounter{utterance}  
    & & & \multicolumn{4}{p{0.6\linewidth}}{
        \cellcolor[rgb]{0.9,0.9,0.9}{
            \makecell[{{p{\linewidth}}}]{
                \texttt{\tiny{[GM$\langle$P2]}}
                \texttt{SPRICH: Mein Raster sieht aktuell so aus: } \\
\texttt{```} \\
\texttt{ 1234567} \\
\texttt{╔═╤═╤═╤═╗} \\
\texttt{║W│T│C│◌║ 1} \\
\texttt{╟─┴─┼─┤◌║ 2} \\
\texttt{║◌◌◌│◌└─╢ 3} \\
\texttt{╟───┘I◌◌║ 4} \\
\texttt{║◌H◌◌◌┌─╢ 5} \\
\texttt{║◌◌◌◌◌│◌║ 6} \\
\texttt{║◌◌◌◌◌│◌║ 7} \\
\texttt{╚═════╧═╝} \\
\texttt{```} \\
\texttt{Ich denke, wir sollten 'I' nach (4,1) bewegen, um Platz für 'T' zu schaffen. Dann können wir 'T' nach (4,1) bewegen und 'I' nach (5,1) bewegen, um die korrekte Reihenfolge 'C', 'H', 'I', 'T', 'W' zu erreichen. Bitte beachte, dass ich möglicherweise weitere Anpassungen vornehmen werde, um die Objekte in der richtigen Reihenfolge zu platzieren. Ich werde 'I' nach (4,1) bewegen, um den Prozess zu starten.} \\
            }
        }
    }
     \\ \\

    \theutterance \stepcounter{utterance}  
    & & & \multicolumn{2}{p{0.3\linewidth}}{
        \cellcolor[rgb]{0.9,0.9,0.9}{
            \makecell[{{p{\linewidth}}}]{
                \texttt{\tiny{[GM$|$GM]}}
                \texttt{SPRICH: Mein Raster sieht aktuell so aus: } \\
\texttt{```} \\
\texttt{ 1234567} \\
\texttt{╔═╤═╤═╤═╗} \\
\texttt{║W│T│C│◌║ 1} \\
\texttt{╟─┴─┼─┤◌║ 2} \\
\texttt{║◌◌◌│◌└─╢ 3} \\
\texttt{╟───┘I◌◌║ 4} \\
\texttt{║◌H◌◌◌┌─╢ 5} \\
\texttt{║◌◌◌◌◌│◌║ 6} \\
\texttt{║◌◌◌◌◌│◌║ 7} \\
\texttt{╚═════╧═╝} \\
\texttt{```} \\
\texttt{Ich denke, wir sollten 'I' nach (4,1) bewegen, um Platz für 'T' zu schaffen. Dann können wir 'T' nach (4,1) bewegen und 'I' nach (5,1) bewegen, um die korrekte Reihenfolge 'C', 'H', 'I', 'T', 'W' zu erreichen. Bitte beachte, dass ich möglicherweise weitere Anpassungen vornehmen werde, um die Objekte in der richtigen Reihenfolge zu platzieren. Ich werde 'I' nach (4,1) bewegen, um den Prozess zu starten.} \\
            }
        }
    }
    & & \\ \\

    \theutterance \stepcounter{utterance}  
    & & & \multicolumn{2}{p{0.3\linewidth}}{
        \cellcolor[rgb]{0.9,0.9,0.9}{
            \makecell[{{p{\linewidth}}}]{
                \texttt{\tiny{[GM$|$GM]}}
                \texttt{Invalid format: tail is not empty: '} \\
\\ 
\texttt{ 1234567} \\
\texttt{╔═╤═╤═╤═╗} \\
\texttt{║W│T│C│◌║ 1} \\
\texttt{╟─┴─┼─┤◌║ 2} \\
\texttt{║◌◌◌│◌└─╢ 3} \\
\texttt{╟───┘I◌◌║ 4} \\
\texttt{║◌H◌◌◌┌─╢ 5} \\
\texttt{║◌◌◌◌◌│◌║ 6} \\
\texttt{║◌◌◌◌◌│◌║ 7} \\
\texttt{╚═════╧═╝} \\
\\ 
\texttt{Ich denke, wir sollten 'I' nach (4,1) bewegen, um Platz für 'T' zu schaffen. Dann können wir 'T' nach (4,1) bewegen und 'I' nach (5,1) bewegen, um die korrekte Reihenfolge 'C', 'H', 'I', 'T', 'W' zu erreichen. Bitte beachte, dass ich möglicherweise weitere Anpassungen vornehmen werde, um die Objekte in der richtigen Reihenfolge zu platzieren. Ich werde 'I' nach (4,1) bewegen, um den Prozess zu starten.} \\
            }
        }
    }
    & & \\ \\

    \theutterance \stepcounter{utterance}  
    & & \multicolumn{4}{p{0.6\linewidth}}{
        \cellcolor[rgb]{0.9,0.9,0.9}{
            \makecell[{{p{\linewidth}}}]{
                \texttt{\tiny{[GM$\rangle$P2]}}
                \texttt{Strafpunkt: Deine Nachricht darf nichts nach dem Befehl enthalten! Stelle sicher, dass deine Antwort ausschließlich einen der beiden Befehle `SPRICH: <NACHRICHT>` und `VERSCHIEBE: <OBJECT>, (<X>, <Y>)` und sonst nichts enthält!} \\
\\ 
\texttt{Ihr habt gemeinsam 11 von maximal 12 Strafpunkten angesammelt.} \\
\texttt{Bitte versuche es noch mal!} \\
            }
        }
    }
    & \\ \\

    \theutterance \stepcounter{utterance}  
    & & & \multicolumn{4}{p{0.6\linewidth}}{
        \cellcolor[rgb]{0.9,0.9,0.9}{
            \makecell[{{p{\linewidth}}}]{
                \texttt{\tiny{[GM$\langle$P2]}}
                \texttt{SPRICH: Ich denke, wir sollten 'I' nach (4,1) bewegen, um Platz für 'T' zu schaffen. Dann können wir 'T' nach (4,1) bewegen und 'I' nach (5,1) bewegen, um die korrekte Reihenfolge 'C', 'H', 'I', 'T', 'W' zu erreichen.} \\
            }
        }
    }
     \\ \\

    \theutterance \stepcounter{utterance}  
    & & & \multicolumn{2}{p{0.3\linewidth}}{
        \cellcolor[rgb]{0.9,0.9,0.9}{
            \makecell[{{p{\linewidth}}}]{
                \texttt{\tiny{[GM$|$GM]}}
                \texttt{SPRICH: Ich denke, wir sollten 'I' nach (4,1) bewegen, um Platz für 'T' zu schaffen. Dann können wir 'T' nach (4,1) bewegen und 'I' nach (5,1) bewegen, um die korrekte Reihenfolge 'C', 'H', 'I', 'T', 'W' zu erreichen.} \\
            }
        }
    }
    & & \\ \\

    \theutterance \stepcounter{utterance}  
    & & \multicolumn{4}{p{0.6\linewidth}}{
        \cellcolor[rgb]{0.9,0.9,0.9}{
            \makecell[{{p{\linewidth}}}]{
                \texttt{\tiny{[P1$\langle$GM]}}
                \texttt{Deine Nachricht wurde an deinen Mitspieler übermittelt.} \\
\\ 
\\ 
\texttt{Ihr seid gerade bei Runde 12 von maximal 20.} \\
\texttt{Ihr habt gemeinsam 11 von maximal 12 Strafpunkten angesammelt.} \\
\texttt{Dein Mitspieler hat dir diese Nachricht geschickt:} \\
\\ 
\\ 
\texttt{"Ich denke, wir sollten 'I' nach (4,1) bewegen, um Platz für 'T' zu schaffen. Dann können wir 'T' nach (4,1) bewegen und 'I' nach (5,1) bewegen, um die korrekte Reihenfolge 'C', 'H', 'I', 'T', 'W' zu erreichen."} \\
\\ 
\\ 
\texttt{Wie lautet dein nächster Befehl?} \\
            }
        }
    }
    & \\ \\

    \theutterance \stepcounter{utterance}  
    & \multicolumn{4}{p{0.6\linewidth}}{
        \cellcolor[rgb]{0.9,0.9,0.9}{
            \makecell[{{p{\linewidth}}}]{
                \texttt{\tiny{[P1$\rangle$GM]}}
                \texttt{VERSCHIEBE: W, (5, 1)} \\
            }
        }
    }
    & & \\ \\

    \theutterance \stepcounter{utterance}  
    & & & \multicolumn{2}{p{0.3\linewidth}}{
        \cellcolor[rgb]{0.9,0.9,0.9}{
            \makecell[{{p{\linewidth}}}]{
                \texttt{\tiny{[GM$|$GM]}}
                \texttt{VERSCHIEBE: W, (5, 1)} \\
            }
        }
    }
    & & \\ \\

    \theutterance \stepcounter{utterance}  
    & & & \multicolumn{2}{p{0.3\linewidth}}{
        \cellcolor[rgb]{0.9,0.9,0.9}{
            \makecell[{{p{\linewidth}}}]{
                \texttt{\tiny{[GM$|$GM]}}
                \texttt{Strafpunkt: (5,1) ist nicht leer, sondern enthält 'I'.} \\
\\ 
\texttt{Ihr habt gemeinsam 12 von maximal 12 Strafpunkten angesammelt.} \\
\texttt{Bitte versuche es noch mal!} \\
            }
        }
    }
    & & \\ \\

    \theutterance \stepcounter{utterance}  
    & & \multicolumn{4}{p{0.6\linewidth}}{
        \cellcolor[rgb]{0.9,0.9,0.9}{
            \makecell[{{p{\linewidth}}}]{
                \texttt{\tiny{[P1$\langle$GM]}}
                \texttt{Strafpunkt: (5,1) ist nicht leer, sondern enthält 'I'.} \\
\\ 
\texttt{Ihr habt gemeinsam 12 von maximal 12 Strafpunkten angesammelt.} \\
\texttt{Bitte versuche es noch mal!} \\
            }
        }
    }
    & \\ \\

    \theutterance \stepcounter{utterance}  
    & \multicolumn{4}{p{0.6\linewidth}}{
        \cellcolor[rgb]{0.9,0.9,0.9}{
            \makecell[{{p{\linewidth}}}]{
                \texttt{\tiny{[P1$\rangle$GM]}}
                \texttt{SPRICH: Ich denke, wir haben zu viele Strafpunkte gesammelt. Ich schlage vor, dass wir das Spiel beenden und die Ergebnisse überprüfen. Wir können nicht mehr gewinnen, da wir die maximale Anzahl an Strafpunkten erreicht haben. Wenn du einverstanden bist, können wir das Spiel beenden und die Ergebnisse diskutieren.Bitte antworte mit "fertig!" wenn du einverstanden bist.} \\
            }
        }
    }
    & & \\ \\

    \theutterance \stepcounter{utterance}  
    & & & \multicolumn{2}{p{0.3\linewidth}}{
        \cellcolor[rgb]{0.9,0.9,0.9}{
            \makecell[{{p{\linewidth}}}]{
                \texttt{\tiny{[GM$|$GM]}}
                \texttt{SPRICH: Ich denke, wir haben zu viele Strafpunkte gesammelt. Ich schlage vor, dass wir das Spiel beenden und die Ergebnisse überprüfen. Wir können nicht mehr gewinnen, da wir die maximale Anzahl an Strafpunkten erreicht haben. Wenn du einverstanden bist, können wir das Spiel beenden und die Ergebnisse diskutieren.Bitte antworte mit "fertig!" wenn du einverstanden bist.} \\
            }
        }
    }
    & & \\ \\

    \theutterance \stepcounter{utterance}  
    & & \multicolumn{4}{p{0.6\linewidth}}{
        \cellcolor[rgb]{0.9,0.9,0.9}{
            \makecell[{{p{\linewidth}}}]{
                \texttt{\tiny{[GM$\rangle$P2]}}
                \texttt{Deine Nachricht wurde an deinen Mitspieler übermittelt.} \\
\\ 
\\ 
\texttt{Ihr seid gerade bei Runde 13 von maximal 20.} \\
\texttt{Ihr habt gemeinsam 12 von maximal 12 Strafpunkten angesammelt.} \\
\texttt{Dein Mitspieler hat dir diese Nachricht geschickt:} \\
\\ 
\\ 
\texttt{"Ich denke, wir haben zu viele Strafpunkte gesammelt. Ich schlage vor, dass wir das Spiel beenden und die Ergebnisse überprüfen. Wir können nicht mehr gewinnen, da wir die maximale Anzahl an Strafpunkten erreicht haben. Wenn du einverstanden bist, können wir das Spiel beenden und die Ergebnisse diskutieren.Bitte antworte mit "fertig!" wenn du einverstanden bist."} \\
\\ 
\\ 
\texttt{Wie lautet dein nächster Befehl?} \\
            }
        }
    }
    & \\ \\

    \theutterance \stepcounter{utterance}  
    & & & \multicolumn{4}{p{0.6\linewidth}}{
        \cellcolor[rgb]{0.9,0.9,0.9}{
            \makecell[{{p{\linewidth}}}]{
                \texttt{\tiny{[GM$\langle$P2]}}
                \texttt{SPRICH: fertig!} \\
            }
        }
    }
     \\ \\

    \theutterance \stepcounter{utterance}  
    & & & \multicolumn{2}{p{0.3\linewidth}}{
        \cellcolor[rgb]{0.9,0.9,0.9}{
            \makecell[{{p{\linewidth}}}]{
                \texttt{\tiny{[GM$|$GM]}}
                \texttt{SPRICH: fertig!} \\
            }
        }
    }
    & & \\ \\

    \theutterance \stepcounter{utterance}  
    & & \multicolumn{4}{p{0.6\linewidth}}{
        \cellcolor[rgb]{0.9,0.9,0.9}{
            \makecell[{{p{\linewidth}}}]{
                \texttt{\tiny{[P1$\langle$GM]}}
                \texttt{Deine Nachricht wurde an deinen Mitspieler übermittelt.} \\
\\ 
\\ 
\texttt{Ihr seid gerade bei Runde 13 von maximal 20.} \\
\texttt{Ihr habt gemeinsam 12 von maximal 12 Strafpunkten angesammelt.} \\
\texttt{Dein Mitspieler hat dir diese Nachricht geschickt:} \\
\\ 
\\ 
\texttt{"fertig!"} \\
\\ 
\\ 
\texttt{Wie lautet dein nächster Befehl?} \\
            }
        }
    }
    & \\ \\

    \theutterance \stepcounter{utterance}  
    & \multicolumn{4}{p{0.6\linewidth}}{
        \cellcolor[rgb]{0.9,0.9,0.9}{
            \makecell[{{p{\linewidth}}}]{
                \texttt{\tiny{[P1$\rangle$GM]}}
                \texttt{SPRICH: fertig!} \\
            }
        }
    }
    & & \\ \\

    \theutterance \stepcounter{utterance}  
    & & & \multicolumn{2}{p{0.3\linewidth}}{
        \cellcolor[rgb]{0.9,0.9,0.9}{
            \makecell[{{p{\linewidth}}}]{
                \texttt{\tiny{[GM$|$GM]}}
                \texttt{SPRICH: fertig!} \\
            }
        }
    }
    & & \\ \\

    \theutterance \stepcounter{utterance}  
    & & \multicolumn{4}{p{0.6\linewidth}}{
        \cellcolor[rgb]{0.9,0.9,0.9}{
            \makecell[{{p{\linewidth}}}]{
                \texttt{\tiny{[GM$\rangle$P2]}}
                \texttt{Deine Nachricht wurde an deinen Mitspieler übermittelt.} \\
\\ 
\\ 
\texttt{Ihr seid gerade bei Runde 14 von maximal 20.} \\
\texttt{Ihr habt gemeinsam 12 von maximal 12 Strafpunkten angesammelt.} \\
\texttt{Dein Mitspieler hat dir diese Nachricht geschickt:} \\
\\ 
\\ 
\texttt{"fertig!"} \\
\\ 
\\ 
\texttt{Wie lautet dein nächster Befehl?} \\
            }
        }
    }
    & \\ \\

    \theutterance \stepcounter{utterance}  
    & & & \multicolumn{4}{p{0.6\linewidth}}{
        \cellcolor[rgb]{0.9,0.9,0.9}{
            \makecell[{{p{\linewidth}}}]{
                \texttt{\tiny{[GM$\langle$P2]}}
                \texttt{SPRICH: fertig!} \\
            }
        }
    }
     \\ \\

    \theutterance \stepcounter{utterance}  
    & & & \multicolumn{2}{p{0.3\linewidth}}{
        \cellcolor[rgb]{0.9,0.9,0.9}{
            \makecell[{{p{\linewidth}}}]{
                \texttt{\tiny{[GM$|$GM]}}
                \texttt{SPRICH: fertig!} \\
            }
        }
    }
    & & \\ \\

    \theutterance \stepcounter{utterance}  
    & & \multicolumn{4}{p{0.6\linewidth}}{
        \cellcolor[rgb]{0.9,0.9,0.9}{
            \makecell[{{p{\linewidth}}}]{
                \texttt{\tiny{[P1$\langle$GM]}}
                \texttt{Deine Nachricht wurde an deinen Mitspieler übermittelt.} \\
\\ 
\\ 
\texttt{Ihr seid gerade bei Runde 14 von maximal 20.} \\
\texttt{Ihr habt gemeinsam 12 von maximal 12 Strafpunkten angesammelt.} \\
\texttt{Dein Mitspieler hat dir diese Nachricht geschickt:} \\
\\ 
\\ 
\texttt{"fertig!"} \\
\\ 
\\ 
\texttt{Wie lautet dein nächster Befehl?} \\
            }
        }
    }
    & \\ \\

    \theutterance \stepcounter{utterance}  
    & \multicolumn{4}{p{0.6\linewidth}}{
        \cellcolor[rgb]{0.9,0.9,0.9}{
            \makecell[{{p{\linewidth}}}]{
                \texttt{\tiny{[P1$\rangle$GM]}}
                \texttt{Das Spiel ist beendet. Ich denke, wir haben gemeinsam entschieden, das Spiel zu beenden, da wir die maximale Anzahl an Strafpunkten erreicht haben. Ich warte auf die Ergebnisse und die Bewertung unseres Spiels.} \\
            }
        }
    }
    & & \\ \\

    \theutterance \stepcounter{utterance}  
    & & & \multicolumn{2}{p{0.3\linewidth}}{
        \cellcolor[rgb]{0.9,0.9,0.9}{
            \makecell[{{p{\linewidth}}}]{
                \texttt{\tiny{[GM$|$GM]}}
                \texttt{Das Spiel ist beendet. Ich denke, wir haben gemeinsam entschieden, das Spiel zu beenden, da wir die maximale Anzahl an Strafpunkten erreicht haben. Ich warte auf die Ergebnisse und die Bewertung unseres Spiels.} \\
            }
        }
    }
    & & \\ \\

    \theutterance \stepcounter{utterance}  
    & & & \multicolumn{2}{p{0.3\linewidth}}{
        \cellcolor[rgb]{0.9,0.9,0.9}{
            \makecell[{{p{\linewidth}}}]{
                \texttt{\tiny{[GM$|$GM]}}
                \texttt{Invalid response format} \\
            }
        }
    }
    & & \\ \\

    \theutterance \stepcounter{utterance}  
    & & & \multicolumn{2}{p{0.3\linewidth}}{
        \cellcolor[rgb]{0.9,0.9,0.9}{
            \makecell[{{p{\linewidth}}}]{
                \texttt{\tiny{[GM$|$GM]}}
                \texttt{Maximum number of penalties exceeded} \\
            }
        }
    }
    & & \\ \\

    \theutterance \stepcounter{utterance}  
    & & & \multicolumn{2}{p{0.3\linewidth}}{
        \cellcolor[rgb]{0.9,0.9,0.9}{
            \makecell[{{p{\linewidth}}}]{
                \texttt{\tiny{[GM$|$GM]}}
                \texttt{* success: False} \\
\texttt{* lose: True} \\
\texttt{* aborted: True} \\
\texttt{{-}{-}{-}{-}{-}{-}{-}} \\
\texttt{* Shifts: 4.00} \\
\texttt{* Max Shifts: 8.00} \\
\texttt{* Min Shifts: 4.00} \\
\texttt{* End Distance Sum: 20.07} \\
\texttt{* Init Distance Sum: 20.21} \\
\texttt{* Expected Distance Sum: 20.95} \\
\texttt{* Penalties: 13.00} \\
\texttt{* Max Penalties: 12.00} \\
\texttt{* Rounds: 14.00} \\
\texttt{* Max Rounds: 20.00} \\
\texttt{* Object Count: 5.00} \\
            }
        }
    }
    & & \\ \\

    \theutterance \stepcounter{utterance}  
    & & & \multicolumn{2}{p{0.3\linewidth}}{
        \cellcolor[rgb]{0.9,0.9,0.9}{
            \makecell[{{p{\linewidth}}}]{
                \texttt{\tiny{[GM$|$GM]}}
                \texttt{Initial Grids:} \\
\texttt{```} \\
\texttt{Player 1:     Player 2:} \\
\texttt{ 1234567       1234567} \\
\texttt{╔═╤═╤═╤═╗     ╔═╤═╤═╤═╗} \\
\texttt{║◌│◌│H│◌║ 1   ║W│◌│T│◌║ 1} \\
\texttt{╟─┴─┼─┤◌║ 2   ╟─┴─┼─┤◌║ 2} \\
\texttt{║◌T◌│◌└─╢ 3   ║◌◌◌│◌└─╢ 3} \\
\texttt{╟───┘◌◌◌║ 4   ╟───┘IC◌║ 4} \\
\texttt{║◌◌◌I◌┌─╢ 5   ║H◌◌◌◌┌─╢ 5} \\
\texttt{║◌◌◌C◌│◌║ 6   ║◌◌◌◌◌│◌║ 6} \\
\texttt{║◌◌◌W◌│◌║ 7   ║◌◌◌◌◌│◌║ 7} \\
\texttt{╚═════╧═╝     ╚═════╧═╝} \\
\texttt{```} \\
\\ 
\texttt{Final Grids:} \\
\texttt{```} \\
\texttt{Player 1:     Player 2:} \\
\texttt{ 1234567       1234567} \\
\texttt{╔═╤═╤═╤═╗     ╔═╤═╤═╤═╗} \\
\texttt{║C│H│I│◌║ 1   ║W│T│C│◌║ 1} \\
\texttt{╟─┴─┼─┤◌║ 2   ╟─┴─┼─┤◌║ 2} \\
\texttt{║◌T◌│◌└─╢ 3   ║◌◌◌│◌└─╢ 3} \\
\texttt{╟───┘◌◌◌║ 4   ╟───┘I◌◌║ 4} \\
\texttt{║◌◌◌◌◌┌─╢ 5   ║◌H◌◌◌┌─╢ 5} \\
\texttt{║◌◌◌◌◌│◌║ 6   ║◌◌◌◌◌│◌║ 6} \\
\texttt{║◌◌◌W◌│◌║ 7   ║◌◌◌◌◌│◌║ 7} \\
\texttt{╚═════╧═╝     ╚═════╧═╝} \\
\texttt{```} \\
            }
        }
    }
    & & \\ \\

\end{supertabular}
}

\end{document}
