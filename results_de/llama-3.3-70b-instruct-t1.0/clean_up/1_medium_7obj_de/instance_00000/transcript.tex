
\documentclass{article}
\usepackage{colortbl}
\usepackage{makecell}
\usepackage{multirow}
\usepackage{supertabular}

\begin{document}

\newcounter{utterance}

\twocolumn

{ \footnotesize  \setcounter{utterance}{1}
\setlength{\tabcolsep}{0pt}
\begin{supertabular}{c@{$\;$}|p{.15\linewidth}@{}p{.15\linewidth}p{.15\linewidth}p{.15\linewidth}p{.15\linewidth}p{.15\linewidth}}

    \# & $\;$A & \multicolumn{4}{c}{Game Master} & $\;\:$B\\
    \hline 

    \theutterance \stepcounter{utterance}  

    & & \multicolumn{4}{p{0.6\linewidth}}{\cellcolor[rgb]{0.9,0.9,0.9}{%
	\makecell[{{p{\linewidth}}}]{% 
	  \tt {\tiny [A$\langle$GM]}  
	 Ich bin der Spielleiter, und du spielst ein kooperatives Spiel mit dem folgenden Raster als Spielbrett:\\ \tt \\ \tt ```\\ \tt  1234567\\ \tt ╔═╤═╤═╤═╗\\ \tt ║◌│◌│◌│◌║ 1\\ \tt ║P└─┼─┴─╢ 2\\ \tt ║◌SA│◌◌◌║ 3\\ \tt ╟──┬┴───╢ 4\\ \tt ║◌◌│◌◌O◌║ 5\\ \tt ║◌◌│◌◌MD║ 6\\ \tt ║◌◌│◌◌◌T║ 7\\ \tt ╚══╧════╝\\ \tt \\ \tt ```\\ \tt \\ \tt * Am oberen Rand sind die x-Koordinaten angegeben, die nach rechts hin zunehmen, und am rechten Rand die y-Koordinaten, die nach unten zunehmen.\\ \tt * Die folgenden Objekte sind zufällig auf deinem Raster platziert: 'P', 'O', 'T', 'S', 'D', 'A', 'M'.\\ \tt \\ \tt \\ \tt Dein Mitspieler sieht eine Variation des Spielbretts, auf dem dieselben Objekte an anderen zufälligen Positionen platziert sind. Du kannst das Spielbrett des anderen Spielers nicht sehen, und er kann deins nicht sehen.\\ \tt \\ \tt \\ \tt **Ziel des Spiels:**\\ \tt \\ \tt Beide Spieler müssen die Objekte auf ihrem jeweiligen Spielbrett so bewegen, dass identische Objekte am Ende an denselben Koordinaten liegen. Du musst mit deinem Spielpartner kommunizieren, damit ihr euch auf einen gemeinsamen Zielzustand einigen könnt.\\ \tt \\ \tt \\ \tt **Spielregeln:**\\ \tt \\ \tt * In jedem Spielzug kannst du genau eine der folgenden zwei Aktionen ausführen:\\ \tt 1. `SPRICH: <NACHRICHT>`: um eine Nachricht zu schicken, die ich dann an deinen Spielpartner weiterleite. Als Nachricht gilt alles bis zum nächsten Zeilenumbruch.\\ \tt 2. `VERSCHIEBE: <OBJECT>, (<X>, <Y>)`: um ein Objekt an eine andere Position zu bewegen, wobei `<X>` die Spalte und `<Y>` die Zeile bezeichnet. Ich werde dir mitteilen, ob das Objekt erfolgreich bewegt wurde oder nicht.\\ \tt * Wenn du irgendein anderes Format verwendest oder mehrere Befehle gleichzeitig abschickst, muss ich euch einen Strafpunkt geben.\\ \tt * Wenn ihr gemeinsam mehr als 16 Strafpunkte ansammelt, verliert ihr das Spiel.\\ \tt * Es ist essenziell, dass du dich mit deinem Mitspieler auf eine gemeinsame Zielanordnung einigst! Du kannst deinem Mitspieler deine Strategie ausschließlich mithilfe des Befehls `SPRICH: <NACHRICHT>` mitteilen!\\ \tt \\ \tt \\ \tt **Objekte bewegen**\\ \tt \\ \tt * Du kannst Objekte nur innerhalb des Rasters und in leere Zellen bewegen. Leere Zellen erkennst am Zeichen '◌'.\\ \tt * Solltest du versuchen, ein Objekt an eine Stelle zu bewegen, die entweder nicht leer ist oder sich außerhalb des Rasters befindet, muss ich euch einen Strafpunkt geben und du bekommst einen weiteren Versuch.\\ \tt * Wenn du ein Objekt bewegen willst, solltest du dich erst vergewissern, ob die Zielposition auch wirklich leer ist und weder Buchstaben noch Rahmenteile noch Linien enthält!\\ \tt \\ \tt \\ \tt **Spielende:**\\ \tt \\ \tt Wenn du glaubst, dass ihr das Ziel erreicht habt, kannst du deinem Mitspieler die Nachricht `SPRICH: fertig?` schicken. Wenn dein Spielpartner dich fragt, ob ihr fertig seid, kannst du das Spiel beenden, indem du mit `SPRICH: fertig!` antwortest.\\ \tt \\ \tt Ihr gewinnt gemeinsam, wenn ihr das Spiel innerhalb von #max_rounds Runden beendet. Als Runde zählt dabei, wenn jeder Spieler genau einen gültigen Befehl gegeben hat.\\ \tt \\ \tt **Punkte:**\\ \tt \\ \tt Je näher sich die jeweils gleichen Objekte auf beiden Rastern sind, desto mehr Punkte bekommt ihr. Für Strafpunkte gibt es Punktabzug. Könnt ihr den Rekord brechen?\\ \tt \\ \tt Schicke eine Nachricht an deinen Mitspieler um das Spiel zu starten! 
	  } 
	   } 
	   } 
	 & \\ 
 

    \theutterance \stepcounter{utterance}  

    & \multicolumn{4}{p{0.6\linewidth}}{\cellcolor[rgb]{0.8,1,0.9}{%
	\makecell[{{p{\linewidth}}}]{% 
	  \tt {\tiny [A$\rangle$GM]}  
	 SPRICH: Hallo, lass uns das Spiel beginnen! Mein Vorschlag ist, dass wir uns auf eine feste Position für jedes Objekt einigen, bevor wir anfangen, sie zu bewegen. Wir könnten die Positionen basierend auf den Buchstaben bestimmen, z.B. 'P' in der oberen linken Ecke, 'O' in der Mitte, 'T' in der unteren rechten Ecke usw. Was meinst du? 
	  } 
	   } 
	   } 
	 & & \\ 
 

    \theutterance \stepcounter{utterance}  

    & & & \multicolumn{2}{p{0.3\linewidth}}{\cellcolor[rgb]{0.95,0.95,0.95}{%
	\makecell[{{p{\linewidth}}}]{% 
	  \tt {\tiny [GM$|$GM]}  
	 SPRICH: Hallo, lass uns das Spiel beginnen! Mein Vorschlag ist, dass wir uns auf eine feste Position für jedes Objekt einigen, bevor wir anfangen, sie zu bewegen. Wir könnten die Positionen basierend auf den Buchstaben bestimmen, z.B. 'P' in der oberen linken Ecke, 'O' in der Mitte, 'T' in der unteren rechten Ecke usw. Was meinst du? 
	  } 
	   } 
	   } 
	 & & \\ 
 

    \theutterance \stepcounter{utterance}  

    & & \multicolumn{4}{p{0.6\linewidth}}{\cellcolor[rgb]{0.9,0.9,0.9}{%
	\makecell[{{p{\linewidth}}}]{% 
	  \tt {\tiny [GM$\rangle$B]}  
	 Ich bin der Spielleiter, und du spielst ein kooperatives Spiel mit dem folgenden Raster als Spielbrett:\\ \tt \\ \tt ```\\ \tt  1234567\\ \tt ╔═╤═╤═╤═╗\\ \tt ║◌│◌│◌│P║ 1\\ \tt ║T└─┼─┴─╢ 2\\ \tt ║◌◌◌│M◌◌║ 3\\ \tt ╟──┬┴───╢ 4\\ \tt ║◌D│◌◌O◌║ 5\\ \tt ║◌◌│◌S◌◌║ 6\\ \tt ║A◌│◌◌◌◌║ 7\\ \tt ╚══╧════╝\\ \tt \\ \tt ```\\ \tt \\ \tt * Am oberen Rand sind die x-Koordinaten angegeben, die nach rechts hin zunehmen, und am rechten Rand die y-Koordinaten, die nach unten zunehmen.\\ \tt * Die folgenden Objekte sind zufällig auf deinem Raster platziert: 'P', 'O', 'T', 'S', 'D', 'A', 'M'.\\ \tt \\ \tt \\ \tt Dein Mitspieler sieht eine Variation des Spielbretts, auf dem dieselben Objekte an anderen zufälligen Positionen platziert sind. Du kannst das Spielbrett des anderen Spielers nicht sehen, und er kann deins nicht sehen.\\ \tt \\ \tt \\ \tt **Ziel des Spiels:**\\ \tt \\ \tt Beide Spieler müssen die Objekte auf ihrem jeweiligen Spielbrett so bewegen, dass identische Objekte am Ende an denselben Koordinaten liegen. Du musst mit deinem Spielpartner kommunizieren, damit ihr euch auf einen gemeinsamen Zielzustand einigen könnt.\\ \tt \\ \tt \\ \tt **Spielregeln:**\\ \tt \\ \tt * In jedem Spielzug kannst du genau eine der folgenden zwei Aktionen ausführen:\\ \tt 1. `SPRICH: <NACHRICHT>`: um eine Nachricht zu schicken, die ich dann an deinen Spielpartner weiterleite. Als Nachricht gilt alles bis zum nächsten Zeilenumbruch.\\ \tt 2. `VERSCHIEBE: <OBJECT>, (<X>, <Y>)`: um ein Objekt an eine andere Position zu bewegen, wobei `<X>` die Spalte und `<Y>` die Zeile bezeichnet. Ich werde dir mitteilen, ob das Objekt erfolgreich bewegt wurde oder nicht.\\ \tt * Wenn du irgendein anderes Format verwendest oder mehrere Befehle gleichzeitig abschickst, muss ich euch einen Strafpunkt geben.\\ \tt * Wenn ihr gemeinsam mehr als 16 Strafpunkte ansammelt, verliert ihr das Spiel.\\ \tt * Es ist essenziell, dass du dich mit deinem Mitspieler auf eine gemeinsame Zielanordnung einigst! Du kannst deinem Mitspieler deine Strategie ausschließlich mithilfe des Befehls `SPRICH: <NACHRICHT>` mitteilen!\\ \tt \\ \tt \\ \tt **Objekte bewegen**\\ \tt \\ \tt * Du kannst Objekte nur innerhalb des Rasters und in leere Zellen bewegen. Leere Zellen erkennst am Zeichen '◌'.\\ \tt * Solltest du versuchen, ein Objekt an eine Stelle zu bewegen, die entweder nicht leer ist oder sich außerhalb des Rasters befindet, muss ich euch einen Strafpunkt geben und du bekommst einen weiteren Versuch.\\ \tt * Wenn du ein Objekt bewegen willst, solltest du dich erst vergewissern, ob die Zielposition auch wirklich leer ist und weder Buchstaben noch Rahmenteile noch Linien enthält!\\ \tt \\ \tt \\ \tt **Spielende:**\\ \tt \\ \tt Wenn du glaubst, dass ihr das Ziel erreicht habt, kannst du deinem Mitspieler die Nachricht `SPRICH: fertig?` schicken. Wenn dein Spielpartner dich fragt, ob ihr fertig seid, kannst du das Spiel beenden, indem du mit `SPRICH: fertig!` antwortest.\\ \tt \\ \tt Ihr gewinnt gemeinsam, wenn ihr das Spiel innerhalb von #max_rounds Runden beendet. Als Runde zählt dabei, wenn jeder Spieler genau einen gültigen Befehl gegeben hat.\\ \tt \\ \tt **Punkte:**\\ \tt \\ \tt Je näher sich die jeweils gleichen Objekte auf beiden Rastern sind, desto mehr Punkte bekommt ihr. Für Strafpunkte gibt es Punktabzug. Könnt ihr den Rekord brechen?\\ \tt \\ \tt Dein Mitspieler hat das Spiel mit dieser Nachricht begonnen:\\ \tt \\ \tt "Hallo, lass uns das Spiel beginnen! Mein Vorschlag ist, dass wir uns auf eine feste Position für jedes Objekt einigen, bevor wir anfangen, sie zu bewegen. Wir könnten die Positionen basierend auf den Buchstaben bestimmen, z.B. 'P' in der oberen linken Ecke, 'O' in der Mitte, 'T' in der unteren rechten Ecke usw. Was meinst du?"\\ \tt \\ \tt Wie lautet dein erster Befehl? 
	  } 
	   } 
	   } 
	 & \\ 
 

    \theutterance \stepcounter{utterance}  

    & & & \multicolumn{4}{p{0.6\linewidth}}{\cellcolor[rgb]{1,0.85,0.72}{%
	\makecell[{{p{\linewidth}}}]{% 
	  \tt {\tiny [GM$\langle$B]}  
	 Ich denke, das ist ein guter Ansatz. Ich schlage vor, dass wir die Positionen wie folgt zuweisen:\\ \tt \\ \tt * 'P' in der oberen linken Ecke (1,1)\\ \tt * 'O' in der Mitte (4,4)\\ \tt * 'T' in der unteren rechten Ecke (7,7)\\ \tt * 'S' in der unteren linken Ecke (1,7)\\ \tt * 'D' in der oberen rechten Ecke (7,1)\\ \tt * 'A' in der mittleren linken Ecke (1,4)\\ \tt * 'M' in der mittleren rechten Ecke (7,4)\\ \tt \\ \tt Ich denke, das ist eine gute Ausgangsbasis. Ich werde jetzt eine Nachricht an meinen Mitspieler senden, um ihm meine Vorschläge mitzuteilen.\\ \tt \\ \tt `SPRICH: Hallo, ich denke, dein Vorschlag ist gut. Ich schlage vor, dass wir die Positionen wie folgt zuweisen: 'P' in (1,1), 'O' in (4,4), 'T' in (7,7), 'S' in (1,7), 'D' in (7,1), 'A' in (1,4) und 'M' in (7,4). Was meinst du?` 
	  } 
	   } 
	   } 
	  \\ 
 

    \theutterance \stepcounter{utterance}  

    & & & \multicolumn{2}{p{0.3\linewidth}}{\cellcolor[rgb]{0.95,0.95,0.95}{%
	\makecell[{{p{\linewidth}}}]{% 
	  \tt {\tiny [GM$|$GM]}  
	 Ich denke, das ist ein guter Ansatz. Ich schlage vor, dass wir die Positionen wie folgt zuweisen:\\ \tt \\ \tt * 'P' in der oberen linken Ecke (1,1)\\ \tt * 'O' in der Mitte (4,4)\\ \tt * 'T' in der unteren rechten Ecke (7,7)\\ \tt * 'S' in der unteren linken Ecke (1,7)\\ \tt * 'D' in der oberen rechten Ecke (7,1)\\ \tt * 'A' in der mittleren linken Ecke (1,4)\\ \tt * 'M' in der mittleren rechten Ecke (7,4)\\ \tt \\ \tt Ich denke, das ist eine gute Ausgangsbasis. Ich werde jetzt eine Nachricht an meinen Mitspieler senden, um ihm meine Vorschläge mitzuteilen.\\ \tt \\ \tt `SPRICH: Hallo, ich denke, dein Vorschlag ist gut. Ich schlage vor, dass wir die Positionen wie folgt zuweisen: 'P' in (1,1), 'O' in (4,4), 'T' in (7,7), 'S' in (1,7), 'D' in (7,1), 'A' in (1,4) und 'M' in (7,4). Was meinst du?` 
	  } 
	   } 
	   } 
	 & & \\ 
 

    \theutterance \stepcounter{utterance}  

    & & & \multicolumn{2}{p{0.3\linewidth}}{\cellcolor[rgb]{0.95,0.95,0.95}{%
	\makecell[{{p{\linewidth}}}]{% 
	  \tt {\tiny [GM$|$GM]}  
	 Invalid format: head is not empty: 'Ich denke, das ist ein guter Ansatz. Ich schlage vor, dass wir die Positionen wie folgt zuweisen:\\ \tt \\ \tt * 'P' in der oberen linken Ecke (1,1)\\ \tt * 'O' in der Mitte (4,4)\\ \tt * 'T' in der unteren rechten Ecke (7,7)\\ \tt * 'S' in der unteren linken Ecke (1,7)\\ \tt * 'D' in der oberen rechten Ecke (7,1)\\ \tt * 'A' in der mittleren linken Ecke (1,4)\\ \tt * 'M' in der mittleren rechten Ecke (7,4)\\ \tt \\ \tt Ich denke, das ist eine gute Ausgangsbasis. Ich werde jetzt eine Nachricht an meinen Mitspieler senden, um ihm meine Vorschläge mitzuteilen.\\ \tt \\ \tt ' 
	  } 
	   } 
	   } 
	 & & \\ 
 

    \theutterance \stepcounter{utterance}  

    & & \multicolumn{4}{p{0.6\linewidth}}{\cellcolor[rgb]{0.9,0.9,0.9}{%
	\makecell[{{p{\linewidth}}}]{% 
	  \tt {\tiny [GM$\rangle$B]}  
	 Strafpunkt: Deine Nachricht darf nichts vor dem Befehl enthalten! Stelle sicher, dass deine Antwort ausschließlich einen der beiden Befehle `SPRICH: <NACHRICHT>` und `VERSCHIEBE: <OBJECT>, (<X>, <Y>)` und sonst nichts enthält!\\ \tt \\ \tt Ihr habt gemeinsam 1 von maximal 16 Strafpunkten angesammelt.\\ \tt Bitte versuche es noch mal! 
	  } 
	   } 
	   } 
	 & \\ 
 

    \theutterance \stepcounter{utterance}  

    & & & \multicolumn{4}{p{0.6\linewidth}}{\cellcolor[rgb]{1,0.85,0.72}{%
	\makecell[{{p{\linewidth}}}]{% 
	  \tt {\tiny [GM$\langle$B]}  
	 SPRICH: Ich denke, dein Vorschlag ist gut. Ich schlage vor, dass wir die Positionen wie folgt zuweisen: 'P' in (1,1), 'O' in (4,4), 'T' in (7,7), 'S' in (1,7), 'D' in (7,1), 'A' in (1,4) und 'M' in (7,4). Was meinst du? 
	  } 
	   } 
	   } 
	  \\ 
 

    \theutterance \stepcounter{utterance}  

    & & & \multicolumn{2}{p{0.3\linewidth}}{\cellcolor[rgb]{0.95,0.95,0.95}{%
	\makecell[{{p{\linewidth}}}]{% 
	  \tt {\tiny [GM$|$GM]}  
	 SPRICH: Ich denke, dein Vorschlag ist gut. Ich schlage vor, dass wir die Positionen wie folgt zuweisen: 'P' in (1,1), 'O' in (4,4), 'T' in (7,7), 'S' in (1,7), 'D' in (7,1), 'A' in (1,4) und 'M' in (7,4). Was meinst du? 
	  } 
	   } 
	   } 
	 & & \\ 
 

    \theutterance \stepcounter{utterance}  

    & & \multicolumn{4}{p{0.6\linewidth}}{\cellcolor[rgb]{0.9,0.9,0.9}{%
	\makecell[{{p{\linewidth}}}]{% 
	  \tt {\tiny [A$\langle$GM]}  
	 Deine Nachricht wurde an deinen Mitspieler übermittelt.\\ \tt \\ \tt \\ \tt Ihr seid gerade bei Runde 1 von maximal 28.\\ \tt Ihr habt gemeinsam 1 von maximal 16 Strafpunkten angesammelt.\\ \tt Dein Mitspieler hat dir diese Nachricht geschickt:\\ \tt \\ \tt \\ \tt "Ich denke, dein Vorschlag ist gut. Ich schlage vor, dass wir die Positionen wie folgt zuweisen: 'P' in (1,1), 'O' in (4,4), 'T' in (7,7), 'S' in (1,7), 'D' in (7,1), 'A' in (1,4) und 'M' in (7,4). Was meinst du?"\\ \tt \\ \tt \\ \tt Wie lautet dein nächster Befehl? 
	  } 
	   } 
	   } 
	 & \\ 
 

    \theutterance \stepcounter{utterance}  

    & \multicolumn{4}{p{0.6\linewidth}}{\cellcolor[rgb]{0.8,1,0.9}{%
	\makecell[{{p{\linewidth}}}]{% 
	  \tt {\tiny [A$\rangle$GM]}  
	 SPRICH: Das klingt nach einem guten Plan! Ich stimme deinem Vorschlag zu. Bevor wir beginnen, die Objekte zu bewegen, möchte ich mich vergewissern, dass wir beide das gleiche Verständnis von den Koordinaten haben. Die Koordinaten (1,1) bedeuten also die obere linke Ecke und (7,7) die untere rechte Ecke, richtig? Lass uns sicherstellen, dass wir dieselbe Strategie verfolgen, bevor wir mit dem Bewegen beginnen. 
	  } 
	   } 
	   } 
	 & & \\ 
 

    \theutterance \stepcounter{utterance}  

    & & & \multicolumn{2}{p{0.3\linewidth}}{\cellcolor[rgb]{0.95,0.95,0.95}{%
	\makecell[{{p{\linewidth}}}]{% 
	  \tt {\tiny [GM$|$GM]}  
	 SPRICH: Das klingt nach einem guten Plan! Ich stimme deinem Vorschlag zu. Bevor wir beginnen, die Objekte zu bewegen, möchte ich mich vergewissern, dass wir beide das gleiche Verständnis von den Koordinaten haben. Die Koordinaten (1,1) bedeuten also die obere linke Ecke und (7,7) die untere rechte Ecke, richtig? Lass uns sicherstellen, dass wir dieselbe Strategie verfolgen, bevor wir mit dem Bewegen beginnen. 
	  } 
	   } 
	   } 
	 & & \\ 
 

    \theutterance \stepcounter{utterance}  

    & & \multicolumn{4}{p{0.6\linewidth}}{\cellcolor[rgb]{0.9,0.9,0.9}{%
	\makecell[{{p{\linewidth}}}]{% 
	  \tt {\tiny [GM$\rangle$B]}  
	 Deine Nachricht wurde an deinen Mitspieler übermittelt.\\ \tt \\ \tt \\ \tt Ihr seid gerade bei Runde 2 von maximal 28.\\ \tt Ihr habt gemeinsam 1 von maximal 16 Strafpunkten angesammelt.\\ \tt Dein Mitspieler hat dir diese Nachricht geschickt:\\ \tt \\ \tt \\ \tt "Das klingt nach einem guten Plan! Ich stimme deinem Vorschlag zu. Bevor wir beginnen, die Objekte zu bewegen, möchte ich mich vergewissern, dass wir beide das gleiche Verständnis von den Koordinaten haben. Die Koordinaten (1,1) bedeuten also die obere linke Ecke und (7,7) die untere rechte Ecke, richtig? Lass uns sicherstellen, dass wir dieselbe Strategie verfolgen, bevor wir mit dem Bewegen beginnen."\\ \tt \\ \tt \\ \tt Wie lautet dein nächster Befehl? 
	  } 
	   } 
	   } 
	 & \\ 
 

    \theutterance \stepcounter{utterance}  

    & & & \multicolumn{4}{p{0.6\linewidth}}{\cellcolor[rgb]{1,0.85,0.72}{%
	\makecell[{{p{\linewidth}}}]{% 
	  \tt {\tiny [GM$\langle$B]}  
	 SPRICH: Ja, das ist richtig! Die Koordinaten (1,1) bedeuten die obere linke Ecke und (7,7) die untere rechte Ecke. Ich bin bereit, mit dem Bewegen der Objekte zu beginnen, sobald wir uns einig sind. Lass uns überprüfen, ob unsere Ausgangspositionen identisch sind. Kannst du mir deine aktuelle Ausgangsposition mitteilen? 
	  } 
	   } 
	   } 
	  \\ 
 

    \theutterance \stepcounter{utterance}  

    & & & \multicolumn{2}{p{0.3\linewidth}}{\cellcolor[rgb]{0.95,0.95,0.95}{%
	\makecell[{{p{\linewidth}}}]{% 
	  \tt {\tiny [GM$|$GM]}  
	 SPRICH: Ja, das ist richtig! Die Koordinaten (1,1) bedeuten die obere linke Ecke und (7,7) die untere rechte Ecke. Ich bin bereit, mit dem Bewegen der Objekte zu beginnen, sobald wir uns einig sind. Lass uns überprüfen, ob unsere Ausgangspositionen identisch sind. Kannst du mir deine aktuelle Ausgangsposition mitteilen? 
	  } 
	   } 
	   } 
	 & & \\ 
 

    \theutterance \stepcounter{utterance}  

    & & \multicolumn{4}{p{0.6\linewidth}}{\cellcolor[rgb]{0.9,0.9,0.9}{%
	\makecell[{{p{\linewidth}}}]{% 
	  \tt {\tiny [A$\langle$GM]}  
	 Deine Nachricht wurde an deinen Mitspieler übermittelt.\\ \tt \\ \tt \\ \tt Ihr seid gerade bei Runde 2 von maximal 28.\\ \tt Ihr habt gemeinsam 1 von maximal 16 Strafpunkten angesammelt.\\ \tt Dein Mitspieler hat dir diese Nachricht geschickt:\\ \tt \\ \tt \\ \tt "Ja, das ist richtig! Die Koordinaten (1,1) bedeuten die obere linke Ecke und (7,7) die untere rechte Ecke. Ich bin bereit, mit dem Bewegen der Objekte zu beginnen, sobald wir uns einig sind. Lass uns überprüfen, ob unsere Ausgangspositionen identisch sind. Kannst du mir deine aktuelle Ausgangsposition mitteilen?"\\ \tt \\ \tt \\ \tt Wie lautet dein nächster Befehl? 
	  } 
	   } 
	   } 
	 & \\ 
 

    \theutterance \stepcounter{utterance}  

    & \multicolumn{4}{p{0.6\linewidth}}{\cellcolor[rgb]{0.8,1,0.9}{%
	\makecell[{{p{\linewidth}}}]{% 
	  \tt {\tiny [A$\rangle$GM]}  
	 SPRICH: Okay, ich verstehe. Meine aktuelle Ausgangsposition sieht wie folgt aus:\\ \tt \\ \tt \\ \tt  1234567\\ \tt ╔═╤═╤═╤═╗\\ \tt ║◌│◌│◌│◌║ 1\\ \tt ║P└─┼─┴─╢ 2\\ \tt ║◌SA│◌◌◌║ 3\\ \tt ╟──┬┴───╢ 4\\ \tt ║◌◌│◌◌O◌║ 5\\ \tt ║◌◌│◌◌MD║ 6\\ \tt ║◌◌│◌◌◌T║ 7\\ \tt ╚══╧════╝\\ \tt \\ \tt \\ \tt Ich habe 'P' in (1,2), 'O' in (5,5), 'T' in (7,7), 'S' in (3,3), 'D' in (6,6), 'A' in (3,3) und 'M' in (6,6). Es sieht so aus, als ob unsere Ausgangspositionen nicht identisch sind. Wir sollten uns auf die Zielpositionen konzentrieren und die Objekte entsprechend bewegen. Sollen wir ein Objekt nach dem anderen bewegen, um sicherzustellen, dass wir auf dem gleichen Weg sind? 
	  } 
	   } 
	   } 
	 & & \\ 
 

    \theutterance \stepcounter{utterance}  

    & & & \multicolumn{2}{p{0.3\linewidth}}{\cellcolor[rgb]{0.95,0.95,0.95}{%
	\makecell[{{p{\linewidth}}}]{% 
	  \tt {\tiny [GM$|$GM]}  
	 SPRICH: Okay, ich verstehe. Meine aktuelle Ausgangsposition sieht wie folgt aus:\\ \tt \\ \tt \\ \tt  1234567\\ \tt ╔═╤═╤═╤═╗\\ \tt ║◌│◌│◌│◌║ 1\\ \tt ║P└─┼─┴─╢ 2\\ \tt ║◌SA│◌◌◌║ 3\\ \tt ╟──┬┴───╢ 4\\ \tt ║◌◌│◌◌O◌║ 5\\ \tt ║◌◌│◌◌MD║ 6\\ \tt ║◌◌│◌◌◌T║ 7\\ \tt ╚══╧════╝\\ \tt \\ \tt \\ \tt Ich habe 'P' in (1,2), 'O' in (5,5), 'T' in (7,7), 'S' in (3,3), 'D' in (6,6), 'A' in (3,3) und 'M' in (6,6). Es sieht so aus, als ob unsere Ausgangspositionen nicht identisch sind. Wir sollten uns auf die Zielpositionen konzentrieren und die Objekte entsprechend bewegen. Sollen wir ein Objekt nach dem anderen bewegen, um sicherzustellen, dass wir auf dem gleichen Weg sind? 
	  } 
	   } 
	   } 
	 & & \\ 
 

    \theutterance \stepcounter{utterance}  

    & & & \multicolumn{2}{p{0.3\linewidth}}{\cellcolor[rgb]{0.95,0.95,0.95}{%
	\makecell[{{p{\linewidth}}}]{% 
	  \tt {\tiny [GM$|$GM]}  
	 Invalid format: tail is not empty: '\\ \tt \\ \tt \\ \tt  1234567\\ \tt ╔═╤═╤═╤═╗\\ \tt ║◌│◌│◌│◌║ 1\\ \tt ║P└─┼─┴─╢ 2\\ \tt ║◌SA│◌◌◌║ 3\\ \tt ╟──┬┴───╢ 4\\ \tt ║◌◌│◌◌O◌║ 5\\ \tt ║◌◌│◌◌MD║ 6\\ \tt ║◌◌│◌◌◌T║ 7\\ \tt ╚══╧════╝\\ \tt \\ \tt \\ \tt Ich habe 'P' in (1,2), 'O' in (5,5), 'T' in (7,7), 'S' in (3,3), 'D' in (6,6), 'A' in (3,3) und 'M' in (6,6). Es sieht so aus, als ob unsere Ausgangspositionen nicht identisch sind. Wir sollten uns auf die Zielpositionen konzentrieren und die Objekte entsprechend bewegen. Sollen wir ein Objekt nach dem anderen bewegen, um sicherzustellen, dass wir auf dem gleichen Weg sind? 
	  } 
	   } 
	   } 
	 & & \\ 
 

    \theutterance \stepcounter{utterance}  

    & & \multicolumn{4}{p{0.6\linewidth}}{\cellcolor[rgb]{0.9,0.9,0.9}{%
	\makecell[{{p{\linewidth}}}]{% 
	  \tt {\tiny [A$\langle$GM]}  
	 Strafpunkt: Deine Nachricht darf nichts nach dem Befehl enthalten! Stelle sicher, dass deine Antwort ausschließlich einen der beiden Befehle `SPRICH: <NACHRICHT>` und `VERSCHIEBE: <OBJECT>, (<X>, <Y>)` und sonst nichts enthält!\\ \tt \\ \tt Ihr habt gemeinsam 2 von maximal 16 Strafpunkten angesammelt.\\ \tt Bitte versuche es noch mal! 
	  } 
	   } 
	   } 
	 & \\ 
 

    \theutterance \stepcounter{utterance}  

    & \multicolumn{4}{p{0.6\linewidth}}{\cellcolor[rgb]{0.8,1,0.9}{%
	\makecell[{{p{\linewidth}}}]{% 
	  \tt {\tiny [A$\rangle$GM]}  
	 SPRICH: Meine aktuelle Ausgangsposition sieht wie folgt aus: \\ \tt ╔═╤═╤═╤═╗\\ \tt ║◌│◌│◌│◌║ 1\\ \tt ║P└─┼─┴─╢ 2\\ \tt ║◌SA│◌◌◌║ 3\\ \tt ╟──┬┴───╢ 4\\ \tt ║◌◌│◌◌O◌║ 5\\ \tt ║◌◌│◌◌MD║ 6\\ \tt ║◌◌│◌◌◌T║ 7\\ \tt ╚══╧════╝ 
	  } 
	   } 
	   } 
	 & & \\ 
 

    \theutterance \stepcounter{utterance}  

    & & & \multicolumn{2}{p{0.3\linewidth}}{\cellcolor[rgb]{0.95,0.95,0.95}{%
	\makecell[{{p{\linewidth}}}]{% 
	  \tt {\tiny [GM$|$GM]}  
	 SPRICH: Meine aktuelle Ausgangsposition sieht wie folgt aus: \\ \tt ╔═╤═╤═╤═╗\\ \tt ║◌│◌│◌│◌║ 1\\ \tt ║P└─┼─┴─╢ 2\\ \tt ║◌SA│◌◌◌║ 3\\ \tt ╟──┬┴───╢ 4\\ \tt ║◌◌│◌◌O◌║ 5\\ \tt ║◌◌│◌◌MD║ 6\\ \tt ║◌◌│◌◌◌T║ 7\\ \tt ╚══╧════╝ 
	  } 
	   } 
	   } 
	 & & \\ 
 

    \theutterance \stepcounter{utterance}  

    & & & \multicolumn{2}{p{0.3\linewidth}}{\cellcolor[rgb]{0.95,0.95,0.95}{%
	\makecell[{{p{\linewidth}}}]{% 
	  \tt {\tiny [GM$|$GM]}  
	 Invalid format: tail is not empty: '\\ \tt ╔═╤═╤═╤═╗\\ \tt ║◌│◌│◌│◌║ 1\\ \tt ║P└─┼─┴─╢ 2\\ \tt ║◌SA│◌◌◌║ 3\\ \tt ╟──┬┴───╢ 4\\ \tt ║◌◌│◌◌O◌║ 5\\ \tt ║◌◌│◌◌MD║ 6\\ \tt ║◌◌│◌◌◌T║ 7\\ \tt ╚══╧════╝ 
	  } 
	   } 
	   } 
	 & & \\ 
 

    \theutterance \stepcounter{utterance}  

    & & \multicolumn{4}{p{0.6\linewidth}}{\cellcolor[rgb]{0.9,0.9,0.9}{%
	\makecell[{{p{\linewidth}}}]{% 
	  \tt {\tiny [A$\langle$GM]}  
	 Strafpunkt: Deine Nachricht darf nichts nach dem Befehl enthalten! Stelle sicher, dass deine Antwort ausschließlich einen der beiden Befehle `SPRICH: <NACHRICHT>` und `VERSCHIEBE: <OBJECT>, (<X>, <Y>)` und sonst nichts enthält!\\ \tt \\ \tt Ihr habt gemeinsam 3 von maximal 16 Strafpunkten angesammelt.\\ \tt Bitte versuche es noch mal! 
	  } 
	   } 
	   } 
	 & \\ 
 

    \theutterance \stepcounter{utterance}  

    & \multicolumn{4}{p{0.6\linewidth}}{\cellcolor[rgb]{0.8,1,0.9}{%
	\makecell[{{p{\linewidth}}}]{% 
	  \tt {\tiny [A$\rangle$GM]}  
	 SPRICH: Meine Ausgangsposition ist ◌◌◌◌, P, ◌SA, ◌◌◌O◌, ◌◌◌MD, ◌◌◌◌T 
	  } 
	   } 
	   } 
	 & & \\ 
 

    \theutterance \stepcounter{utterance}  

    & & & \multicolumn{2}{p{0.3\linewidth}}{\cellcolor[rgb]{0.95,0.95,0.95}{%
	\makecell[{{p{\linewidth}}}]{% 
	  \tt {\tiny [GM$|$GM]}  
	 SPRICH: Meine Ausgangsposition ist ◌◌◌◌, P, ◌SA, ◌◌◌O◌, ◌◌◌MD, ◌◌◌◌T 
	  } 
	   } 
	   } 
	 & & \\ 
 

    \theutterance \stepcounter{utterance}  

    & & \multicolumn{4}{p{0.6\linewidth}}{\cellcolor[rgb]{0.9,0.9,0.9}{%
	\makecell[{{p{\linewidth}}}]{% 
	  \tt {\tiny [GM$\rangle$B]}  
	 Deine Nachricht wurde an deinen Mitspieler übermittelt.\\ \tt \\ \tt \\ \tt Ihr seid gerade bei Runde 3 von maximal 28.\\ \tt Ihr habt gemeinsam 3 von maximal 16 Strafpunkten angesammelt.\\ \tt Dein Mitspieler hat dir diese Nachricht geschickt:\\ \tt \\ \tt \\ \tt "Meine Ausgangsposition ist ◌◌◌◌, P, ◌SA, ◌◌◌O◌, ◌◌◌MD, ◌◌◌◌T"\\ \tt \\ \tt \\ \tt Wie lautet dein nächster Befehl? 
	  } 
	   } 
	   } 
	 & \\ 
 

    \theutterance \stepcounter{utterance}  

    & & & \multicolumn{4}{p{0.6\linewidth}}{\cellcolor[rgb]{1,0.85,0.72}{%
	\makecell[{{p{\linewidth}}}]{% 
	  \tt {\tiny [GM$\langle$B]}  
	 SPRICH: Meine Ausgangsposition ist P, ◌, ◌, ◌, T, ◌, ◌, ◌, M, ◌, ◌, ◌, ◌, ◌, D, ◌, ◌, O, ◌, ◌, S, A, ◌, ◌, ◌, ◌, ◌. Ich sehe, dass unsere Ausgangspositionen nicht identisch sind. Lass uns die Positionen der Objekte vergleichen und einen Plan erstellen, um sie an die gewünschten Positionen zu bewegen. Wo soll ich anfangen? 
	  } 
	   } 
	   } 
	  \\ 
 

    \theutterance \stepcounter{utterance}  

    & & & \multicolumn{2}{p{0.3\linewidth}}{\cellcolor[rgb]{0.95,0.95,0.95}{%
	\makecell[{{p{\linewidth}}}]{% 
	  \tt {\tiny [GM$|$GM]}  
	 SPRICH: Meine Ausgangsposition ist P, ◌, ◌, ◌, T, ◌, ◌, ◌, M, ◌, ◌, ◌, ◌, ◌, D, ◌, ◌, O, ◌, ◌, S, A, ◌, ◌, ◌, ◌, ◌. Ich sehe, dass unsere Ausgangspositionen nicht identisch sind. Lass uns die Positionen der Objekte vergleichen und einen Plan erstellen, um sie an die gewünschten Positionen zu bewegen. Wo soll ich anfangen? 
	  } 
	   } 
	   } 
	 & & \\ 
 

    \theutterance \stepcounter{utterance}  

    & & \multicolumn{4}{p{0.6\linewidth}}{\cellcolor[rgb]{0.9,0.9,0.9}{%
	\makecell[{{p{\linewidth}}}]{% 
	  \tt {\tiny [A$\langle$GM]}  
	 Deine Nachricht wurde an deinen Mitspieler übermittelt.\\ \tt \\ \tt \\ \tt Ihr seid gerade bei Runde 3 von maximal 28.\\ \tt Ihr habt gemeinsam 3 von maximal 16 Strafpunkten angesammelt.\\ \tt Dein Mitspieler hat dir diese Nachricht geschickt:\\ \tt \\ \tt \\ \tt "Meine Ausgangsposition ist P, ◌, ◌, ◌, T, ◌, ◌, ◌, M, ◌, ◌, ◌, ◌, ◌, D, ◌, ◌, O, ◌, ◌, S, A, ◌, ◌, ◌, ◌, ◌. Ich sehe, dass unsere Ausgangspositionen nicht identisch sind. Lass uns die Positionen der Objekte vergleichen und einen Plan erstellen, um sie an die gewünschten Positionen zu bewegen. Wo soll ich anfangen?"\\ \tt \\ \tt \\ \tt Wie lautet dein nächster Befehl? 
	  } 
	   } 
	   } 
	 & \\ 
 

    \theutterance \stepcounter{utterance}  

    & \multicolumn{4}{p{0.6\linewidth}}{\cellcolor[rgb]{0.8,1,0.9}{%
	\makecell[{{p{\linewidth}}}]{% 
	  \tt {\tiny [A$\rangle$GM]}  
	 SPRICH: Fangen wir mit dem Objekt P an, das in unserer Zielposition in (1,1) sein soll. 
	  } 
	   } 
	   } 
	 & & \\ 
 

    \theutterance \stepcounter{utterance}  

    & & & \multicolumn{2}{p{0.3\linewidth}}{\cellcolor[rgb]{0.95,0.95,0.95}{%
	\makecell[{{p{\linewidth}}}]{% 
	  \tt {\tiny [GM$|$GM]}  
	 SPRICH: Fangen wir mit dem Objekt P an, das in unserer Zielposition in (1,1) sein soll. 
	  } 
	   } 
	   } 
	 & & \\ 
 

    \theutterance \stepcounter{utterance}  

    & & \multicolumn{4}{p{0.6\linewidth}}{\cellcolor[rgb]{0.9,0.9,0.9}{%
	\makecell[{{p{\linewidth}}}]{% 
	  \tt {\tiny [GM$\rangle$B]}  
	 Deine Nachricht wurde an deinen Mitspieler übermittelt.\\ \tt \\ \tt \\ \tt Ihr seid gerade bei Runde 4 von maximal 28.\\ \tt Ihr habt gemeinsam 3 von maximal 16 Strafpunkten angesammelt.\\ \tt Dein Mitspieler hat dir diese Nachricht geschickt:\\ \tt \\ \tt \\ \tt "Fangen wir mit dem Objekt P an, das in unserer Zielposition in (1,1) sein soll."\\ \tt \\ \tt \\ \tt Wie lautet dein nächster Befehl? 
	  } 
	   } 
	   } 
	 & \\ 
 

    \theutterance \stepcounter{utterance}  

    & & & \multicolumn{4}{p{0.6\linewidth}}{\cellcolor[rgb]{1,0.85,0.72}{%
	\makecell[{{p{\linewidth}}}]{% 
	  \tt {\tiny [GM$\langle$B]}  
	 SPRICH: Mein Objekt P ist bereits in der Position (1,1), also muss ich es nicht bewegen. Wie sieht es bei dir aus? Muss dein Objekt P bewegt werden? 
	  } 
	   } 
	   } 
	  \\ 
 

    \theutterance \stepcounter{utterance}  

    & & & \multicolumn{2}{p{0.3\linewidth}}{\cellcolor[rgb]{0.95,0.95,0.95}{%
	\makecell[{{p{\linewidth}}}]{% 
	  \tt {\tiny [GM$|$GM]}  
	 SPRICH: Mein Objekt P ist bereits in der Position (1,1), also muss ich es nicht bewegen. Wie sieht es bei dir aus? Muss dein Objekt P bewegt werden? 
	  } 
	   } 
	   } 
	 & & \\ 
 

    \theutterance \stepcounter{utterance}  

    & & \multicolumn{4}{p{0.6\linewidth}}{\cellcolor[rgb]{0.9,0.9,0.9}{%
	\makecell[{{p{\linewidth}}}]{% 
	  \tt {\tiny [A$\langle$GM]}  
	 Deine Nachricht wurde an deinen Mitspieler übermittelt.\\ \tt \\ \tt \\ \tt Ihr seid gerade bei Runde 4 von maximal 28.\\ \tt Ihr habt gemeinsam 3 von maximal 16 Strafpunkten angesammelt.\\ \tt Dein Mitspieler hat dir diese Nachricht geschickt:\\ \tt \\ \tt \\ \tt "Mein Objekt P ist bereits in der Position (1,1), also muss ich es nicht bewegen. Wie sieht es bei dir aus? Muss dein Objekt P bewegt werden?"\\ \tt \\ \tt \\ \tt Wie lautet dein nächster Befehl? 
	  } 
	   } 
	   } 
	 & \\ 
 

    \theutterance \stepcounter{utterance}  

    & \multicolumn{4}{p{0.6\linewidth}}{\cellcolor[rgb]{0.8,1,0.9}{%
	\makecell[{{p{\linewidth}}}]{% 
	  \tt {\tiny [A$\rangle$GM]}  
	 SPRICH: Nein, mein Objekt P ist in (1,2), also muss es nach (1,1) bewegt werden. Ich werde es als nächstes bewegen. 
	  } 
	   } 
	   } 
	 & & \\ 
 

    \theutterance \stepcounter{utterance}  

    & & & \multicolumn{2}{p{0.3\linewidth}}{\cellcolor[rgb]{0.95,0.95,0.95}{%
	\makecell[{{p{\linewidth}}}]{% 
	  \tt {\tiny [GM$|$GM]}  
	 SPRICH: Nein, mein Objekt P ist in (1,2), also muss es nach (1,1) bewegt werden. Ich werde es als nächstes bewegen. 
	  } 
	   } 
	   } 
	 & & \\ 
 

    \theutterance \stepcounter{utterance}  

    & & \multicolumn{4}{p{0.6\linewidth}}{\cellcolor[rgb]{0.9,0.9,0.9}{%
	\makecell[{{p{\linewidth}}}]{% 
	  \tt {\tiny [GM$\rangle$B]}  
	 Deine Nachricht wurde an deinen Mitspieler übermittelt.\\ \tt \\ \tt \\ \tt Ihr seid gerade bei Runde 5 von maximal 28.\\ \tt Ihr habt gemeinsam 3 von maximal 16 Strafpunkten angesammelt.\\ \tt Dein Mitspieler hat dir diese Nachricht geschickt:\\ \tt \\ \tt \\ \tt "Nein, mein Objekt P ist in (1,2), also muss es nach (1,1) bewegt werden. Ich werde es als nächstes bewegen."\\ \tt \\ \tt \\ \tt Wie lautet dein nächster Befehl? 
	  } 
	   } 
	   } 
	 & \\ 
 

    \theutterance \stepcounter{utterance}  

    & & & \multicolumn{4}{p{0.6\linewidth}}{\cellcolor[rgb]{1,0.85,0.72}{%
	\makecell[{{p{\linewidth}}}]{% 
	  \tt {\tiny [GM$\langle$B]}  
	 SPRICH: Okay, ich warte ab, bis du dein Objekt P bewegt hast. Als nächstes sollten wir uns um das Objekt O kümmern, das in unserer Zielposition in (4,4) sein soll. Wie ist die aktuelle Position deines Objekts O? 
	  } 
	   } 
	   } 
	  \\ 
 

    \theutterance \stepcounter{utterance}  

    & & & \multicolumn{2}{p{0.3\linewidth}}{\cellcolor[rgb]{0.95,0.95,0.95}{%
	\makecell[{{p{\linewidth}}}]{% 
	  \tt {\tiny [GM$|$GM]}  
	 SPRICH: Okay, ich warte ab, bis du dein Objekt P bewegt hast. Als nächstes sollten wir uns um das Objekt O kümmern, das in unserer Zielposition in (4,4) sein soll. Wie ist die aktuelle Position deines Objekts O? 
	  } 
	   } 
	   } 
	 & & \\ 
 

    \theutterance \stepcounter{utterance}  

    & & \multicolumn{4}{p{0.6\linewidth}}{\cellcolor[rgb]{0.9,0.9,0.9}{%
	\makecell[{{p{\linewidth}}}]{% 
	  \tt {\tiny [A$\langle$GM]}  
	 Deine Nachricht wurde an deinen Mitspieler übermittelt.\\ \tt \\ \tt \\ \tt Ihr seid gerade bei Runde 5 von maximal 28.\\ \tt Ihr habt gemeinsam 3 von maximal 16 Strafpunkten angesammelt.\\ \tt Dein Mitspieler hat dir diese Nachricht geschickt:\\ \tt \\ \tt \\ \tt "Okay, ich warte ab, bis du dein Objekt P bewegt hast. Als nächstes sollten wir uns um das Objekt O kümmern, das in unserer Zielposition in (4,4) sein soll. Wie ist die aktuelle Position deines Objekts O?"\\ \tt \\ \tt \\ \tt Wie lautet dein nächster Befehl? 
	  } 
	   } 
	   } 
	 & \\ 
 

    \theutterance \stepcounter{utterance}  

    & \multicolumn{4}{p{0.6\linewidth}}{\cellcolor[rgb]{0.8,1,0.9}{%
	\makecell[{{p{\linewidth}}}]{% 
	  \tt {\tiny [A$\rangle$GM]}  
	 SPRICH: Mein Objekt O ist in (5,5), also muss es nach (4,4) bewegt werden, aber erst muss ich mein Objekt P bewegen. 
	  } 
	   } 
	   } 
	 & & \\ 
 

    \theutterance \stepcounter{utterance}  

    & & & \multicolumn{2}{p{0.3\linewidth}}{\cellcolor[rgb]{0.95,0.95,0.95}{%
	\makecell[{{p{\linewidth}}}]{% 
	  \tt {\tiny [GM$|$GM]}  
	 SPRICH: Mein Objekt O ist in (5,5), also muss es nach (4,4) bewegt werden, aber erst muss ich mein Objekt P bewegen. 
	  } 
	   } 
	   } 
	 & & \\ 
 

    \theutterance \stepcounter{utterance}  

    & & \multicolumn{4}{p{0.6\linewidth}}{\cellcolor[rgb]{0.9,0.9,0.9}{%
	\makecell[{{p{\linewidth}}}]{% 
	  \tt {\tiny [GM$\rangle$B]}  
	 Deine Nachricht wurde an deinen Mitspieler übermittelt.\\ \tt \\ \tt \\ \tt Ihr seid gerade bei Runde 6 von maximal 28.\\ \tt Ihr habt gemeinsam 3 von maximal 16 Strafpunkten angesammelt.\\ \tt Dein Mitspieler hat dir diese Nachricht geschickt:\\ \tt \\ \tt \\ \tt "Mein Objekt O ist in (5,5), also muss es nach (4,4) bewegt werden, aber erst muss ich mein Objekt P bewegen."\\ \tt \\ \tt \\ \tt Wie lautet dein nächster Befehl? 
	  } 
	   } 
	   } 
	 & \\ 
 

    \theutterance \stepcounter{utterance}  

    & & & \multicolumn{4}{p{0.6\linewidth}}{\cellcolor[rgb]{1,0.85,0.72}{%
	\makecell[{{p{\linewidth}}}]{% 
	  \tt {\tiny [GM$\langle$B]}  
	 SPRICH: Verstehe, du musst erst dein Objekt P bewegen, bevor du dein Objekt O bewegen kannst. Ich werde abwarten, bis du dein Objekt P bewegt hast. Mein Objekt O ist übrigens in (4,5), also muss ich es auch noch bewegen, um es an die Zielposition (4,4) zu bringen. Ich warte auf deine nächste Aktion. 
	  } 
	   } 
	   } 
	  \\ 
 

    \theutterance \stepcounter{utterance}  

    & & & \multicolumn{2}{p{0.3\linewidth}}{\cellcolor[rgb]{0.95,0.95,0.95}{%
	\makecell[{{p{\linewidth}}}]{% 
	  \tt {\tiny [GM$|$GM]}  
	 SPRICH: Verstehe, du musst erst dein Objekt P bewegen, bevor du dein Objekt O bewegen kannst. Ich werde abwarten, bis du dein Objekt P bewegt hast. Mein Objekt O ist übrigens in (4,5), also muss ich es auch noch bewegen, um es an die Zielposition (4,4) zu bringen. Ich warte auf deine nächste Aktion. 
	  } 
	   } 
	   } 
	 & & \\ 
 

    \theutterance \stepcounter{utterance}  

    & & \multicolumn{4}{p{0.6\linewidth}}{\cellcolor[rgb]{0.9,0.9,0.9}{%
	\makecell[{{p{\linewidth}}}]{% 
	  \tt {\tiny [A$\langle$GM]}  
	 Deine Nachricht wurde an deinen Mitspieler übermittelt.\\ \tt \\ \tt \\ \tt Ihr seid gerade bei Runde 6 von maximal 28.\\ \tt Ihr habt gemeinsam 3 von maximal 16 Strafpunkten angesammelt.\\ \tt Dein Mitspieler hat dir diese Nachricht geschickt:\\ \tt \\ \tt \\ \tt "Verstehe, du musst erst dein Objekt P bewegen, bevor du dein Objekt O bewegen kannst. Ich werde abwarten, bis du dein Objekt P bewegt hast. Mein Objekt O ist übrigens in (4,5), also muss ich es auch noch bewegen, um es an die Zielposition (4,4) zu bringen. Ich warte auf deine nächste Aktion."\\ \tt \\ \tt \\ \tt Wie lautet dein nächster Befehl? 
	  } 
	   } 
	   } 
	 & \\ 
 

    \theutterance \stepcounter{utterance}  

    & \multicolumn{4}{p{0.6\linewidth}}{\cellcolor[rgb]{0.8,1,0.9}{%
	\makecell[{{p{\linewidth}}}]{% 
	  \tt {\tiny [A$\rangle$GM]}  
	 VERSCHIEBE: P, (1,1) 
	  } 
	   } 
	   } 
	 & & \\ 
 

    \theutterance \stepcounter{utterance}  

    & & & \multicolumn{2}{p{0.3\linewidth}}{\cellcolor[rgb]{0.95,0.95,0.95}{%
	\makecell[{{p{\linewidth}}}]{% 
	  \tt {\tiny [GM$|$GM]}  
	 VERSCHIEBE: P, (1,1) 
	  } 
	   } 
	   } 
	 & & \\ 
 

    \theutterance \stepcounter{utterance}  

    & & & \multicolumn{2}{p{0.3\linewidth}}{\cellcolor[rgb]{0.95,0.95,0.95}{%
	\makecell[{{p{\linewidth}}}]{% 
	  \tt {\tiny [GM$|$GM]}  
	 'P' erfolgreich nach (1,1) verschoben. Dein Raster sieht jetzt so aus:\\ \tt ```\\ \tt  1234567\\ \tt ╔═╤═╤═╤═╗\\ \tt ║P│◌│◌│◌║ 1\\ \tt ║◌└─┼─┴─╢ 2\\ \tt ║◌SA│◌◌◌║ 3\\ \tt ╟──┬┴───╢ 4\\ \tt ║◌◌│◌◌O◌║ 5\\ \tt ║◌◌│◌◌MD║ 6\\ \tt ║◌◌│◌◌◌T║ 7\\ \tt ╚══╧════╝\\ \tt ```\\ \tt  
	  } 
	   } 
	   } 
	 & & \\ 
 

    \theutterance \stepcounter{utterance}  

    & & \multicolumn{4}{p{0.6\linewidth}}{\cellcolor[rgb]{0.9,0.9,0.9}{%
	\makecell[{{p{\linewidth}}}]{% 
	  \tt {\tiny [GM$\rangle$B]}  
	 Deine Nachricht wurde an deinen Mitspieler übermittelt.\\ \tt \\ \tt \\ \tt Ihr seid gerade bei Runde 7 von maximal 28.\\ \tt Ihr habt gemeinsam 3 von maximal 16 Strafpunkten angesammelt.\\ \tt Dein Mitspieler hat ein Objekt auf seinem Raster bewegt.\\ \tt \\ \tt \\ \tt Wie lautet dein nächster Befehl? 
	  } 
	   } 
	   } 
	 & \\ 
 

    \theutterance \stepcounter{utterance}  

    & & & \multicolumn{4}{p{0.6\linewidth}}{\cellcolor[rgb]{1,0.85,0.72}{%
	\makecell[{{p{\linewidth}}}]{% 
	  \tt {\tiny [GM$\langle$B]}  
	 SPRICH: Ich gehe davon aus, dass du dein Objekt P bewegt hast. Kannst du mir bitte bestätigen, ob das korrekt ist und wie die neue Position deines Objekts P ist? 
	  } 
	   } 
	   } 
	  \\ 
 

    \theutterance \stepcounter{utterance}  

    & & & \multicolumn{2}{p{0.3\linewidth}}{\cellcolor[rgb]{0.95,0.95,0.95}{%
	\makecell[{{p{\linewidth}}}]{% 
	  \tt {\tiny [GM$|$GM]}  
	 SPRICH: Ich gehe davon aus, dass du dein Objekt P bewegt hast. Kannst du mir bitte bestätigen, ob das korrekt ist und wie die neue Position deines Objekts P ist? 
	  } 
	   } 
	   } 
	 & & \\ 
 

    \theutterance \stepcounter{utterance}  

    & & \multicolumn{4}{p{0.6\linewidth}}{\cellcolor[rgb]{0.9,0.9,0.9}{%
	\makecell[{{p{\linewidth}}}]{% 
	  \tt {\tiny [A$\langle$GM]}  
	 'P' erfolgreich nach (1,1) verschoben. Dein Raster sieht jetzt so aus:\\ \tt ```\\ \tt  1234567\\ \tt ╔═╤═╤═╤═╗\\ \tt ║P│◌│◌│◌║ 1\\ \tt ║◌└─┼─┴─╢ 2\\ \tt ║◌SA│◌◌◌║ 3\\ \tt ╟──┬┴───╢ 4\\ \tt ║◌◌│◌◌O◌║ 5\\ \tt ║◌◌│◌◌MD║ 6\\ \tt ║◌◌│◌◌◌T║ 7\\ \tt ╚══╧════╝\\ \tt ```\\ \tt Ihr seid gerade bei Runde 7 von maximal 28.\\ \tt Ihr habt gemeinsam 3 von maximal 16 Strafpunkten angesammelt.\\ \tt Dein Mitspieler hat dir diese Nachricht geschickt:\\ \tt \\ \tt \\ \tt "Ich gehe davon aus, dass du dein Objekt P bewegt hast. Kannst du mir bitte bestätigen, ob das korrekt ist und wie die neue Position deines Objekts P ist?"\\ \tt \\ \tt \\ \tt Wie lautet dein nächster Befehl? 
	  } 
	   } 
	   } 
	 & \\ 
 

    \theutterance \stepcounter{utterance}  

    & \multicolumn{4}{p{0.6\linewidth}}{\cellcolor[rgb]{0.8,1,0.9}{%
	\makecell[{{p{\linewidth}}}]{% 
	  \tt {\tiny [A$\rangle$GM]}  
	 SPRICH: Ja, das ist korrekt! Mein Objekt P ist jetzt in (1,1). 
	  } 
	   } 
	   } 
	 & & \\ 
 

    \theutterance \stepcounter{utterance}  

    & & & \multicolumn{2}{p{0.3\linewidth}}{\cellcolor[rgb]{0.95,0.95,0.95}{%
	\makecell[{{p{\linewidth}}}]{% 
	  \tt {\tiny [GM$|$GM]}  
	 SPRICH: Ja, das ist korrekt! Mein Objekt P ist jetzt in (1,1). 
	  } 
	   } 
	   } 
	 & & \\ 
 

    \theutterance \stepcounter{utterance}  

    & & \multicolumn{4}{p{0.6\linewidth}}{\cellcolor[rgb]{0.9,0.9,0.9}{%
	\makecell[{{p{\linewidth}}}]{% 
	  \tt {\tiny [GM$\rangle$B]}  
	 Deine Nachricht wurde an deinen Mitspieler übermittelt.\\ \tt \\ \tt \\ \tt Ihr seid gerade bei Runde 8 von maximal 28.\\ \tt Ihr habt gemeinsam 3 von maximal 16 Strafpunkten angesammelt.\\ \tt Dein Mitspieler hat dir diese Nachricht geschickt:\\ \tt \\ \tt \\ \tt "Ja, das ist korrekt! Mein Objekt P ist jetzt in (1,1)."\\ \tt \\ \tt \\ \tt Wie lautet dein nächster Befehl? 
	  } 
	   } 
	   } 
	 & \\ 
 

    \theutterance \stepcounter{utterance}  

    & & & \multicolumn{4}{p{0.6\linewidth}}{\cellcolor[rgb]{1,0.85,0.72}{%
	\makecell[{{p{\linewidth}}}]{% 
	  \tt {\tiny [GM$\langle$B]}  
	 SPRICH: Gut, jetzt dass mein Objekt P und dein Objekt P an der gleichen Position sind, können wir uns auf das nächste Objekt konzentrieren. Mein Objekt O ist in (4,5) und muss nach (4,4) bewegt werden. Ich denke, ich werde es als nächstes bewegen. Wie ist die Situation bei dir mit dem Objekt O? 
	  } 
	   } 
	   } 
	  \\ 
 

    \theutterance \stepcounter{utterance}  

    & & & \multicolumn{2}{p{0.3\linewidth}}{\cellcolor[rgb]{0.95,0.95,0.95}{%
	\makecell[{{p{\linewidth}}}]{% 
	  \tt {\tiny [GM$|$GM]}  
	 SPRICH: Gut, jetzt dass mein Objekt P und dein Objekt P an der gleichen Position sind, können wir uns auf das nächste Objekt konzentrieren. Mein Objekt O ist in (4,5) und muss nach (4,4) bewegt werden. Ich denke, ich werde es als nächstes bewegen. Wie ist die Situation bei dir mit dem Objekt O? 
	  } 
	   } 
	   } 
	 & & \\ 
 

    \theutterance \stepcounter{utterance}  

    & & \multicolumn{4}{p{0.6\linewidth}}{\cellcolor[rgb]{0.9,0.9,0.9}{%
	\makecell[{{p{\linewidth}}}]{% 
	  \tt {\tiny [A$\langle$GM]}  
	 Deine Nachricht wurde an deinen Mitspieler übermittelt.\\ \tt \\ \tt \\ \tt Ihr seid gerade bei Runde 8 von maximal 28.\\ \tt Ihr habt gemeinsam 3 von maximal 16 Strafpunkten angesammelt.\\ \tt Dein Mitspieler hat dir diese Nachricht geschickt:\\ \tt \\ \tt \\ \tt "Gut, jetzt dass mein Objekt P und dein Objekt P an der gleichen Position sind, können wir uns auf das nächste Objekt konzentrieren. Mein Objekt O ist in (4,5) und muss nach (4,4) bewegt werden. Ich denke, ich werde es als nächstes bewegen. Wie ist die Situation bei dir mit dem Objekt O?"\\ \tt \\ \tt \\ \tt Wie lautet dein nächster Befehl? 
	  } 
	   } 
	   } 
	 & \\ 
 

    \theutterance \stepcounter{utterance}  

    & \multicolumn{4}{p{0.6\linewidth}}{\cellcolor[rgb]{0.8,1,0.9}{%
	\makecell[{{p{\linewidth}}}]{% 
	  \tt {\tiny [A$\rangle$GM]}  
	 SPRICH: Mein Objekt O ist in (5,5) und muss nach (4,4) bewegt werden, ich werde es als nächstes bewegen. 
	  } 
	   } 
	   } 
	 & & \\ 
 

    \theutterance \stepcounter{utterance}  

    & & & \multicolumn{2}{p{0.3\linewidth}}{\cellcolor[rgb]{0.95,0.95,0.95}{%
	\makecell[{{p{\linewidth}}}]{% 
	  \tt {\tiny [GM$|$GM]}  
	 SPRICH: Mein Objekt O ist in (5,5) und muss nach (4,4) bewegt werden, ich werde es als nächstes bewegen. 
	  } 
	   } 
	   } 
	 & & \\ 
 

    \theutterance \stepcounter{utterance}  

    & & \multicolumn{4}{p{0.6\linewidth}}{\cellcolor[rgb]{0.9,0.9,0.9}{%
	\makecell[{{p{\linewidth}}}]{% 
	  \tt {\tiny [GM$\rangle$B]}  
	 Deine Nachricht wurde an deinen Mitspieler übermittelt.\\ \tt \\ \tt \\ \tt Ihr seid gerade bei Runde 9 von maximal 28.\\ \tt Ihr habt gemeinsam 3 von maximal 16 Strafpunkten angesammelt.\\ \tt Dein Mitspieler hat dir diese Nachricht geschickt:\\ \tt \\ \tt \\ \tt "Mein Objekt O ist in (5,5) und muss nach (4,4) bewegt werden, ich werde es als nächstes bewegen."\\ \tt \\ \tt \\ \tt Wie lautet dein nächster Befehl? 
	  } 
	   } 
	   } 
	 & \\ 
 

    \theutterance \stepcounter{utterance}  

    & & & \multicolumn{4}{p{0.6\linewidth}}{\cellcolor[rgb]{1,0.85,0.72}{%
	\makecell[{{p{\linewidth}}}]{% 
	  \tt {\tiny [GM$\langle$B]}  
	 VERSCHIEBE: O, (4,4) 
	  } 
	   } 
	   } 
	  \\ 
 

    \theutterance \stepcounter{utterance}  

    & & & \multicolumn{2}{p{0.3\linewidth}}{\cellcolor[rgb]{0.95,0.95,0.95}{%
	\makecell[{{p{\linewidth}}}]{% 
	  \tt {\tiny [GM$|$GM]}  
	 VERSCHIEBE: O, (4,4) 
	  } 
	   } 
	   } 
	 & & \\ 
 

    \theutterance \stepcounter{utterance}  

    & & & \multicolumn{2}{p{0.3\linewidth}}{\cellcolor[rgb]{0.95,0.95,0.95}{%
	\makecell[{{p{\linewidth}}}]{% 
	  \tt {\tiny [GM$|$GM]}  
	 Strafpunkt: (4,4) ist nicht leer, sondern enthält '┴'.\\ \tt \\ \tt Ihr habt gemeinsam 4 von maximal 16 Strafpunkten angesammelt.\\ \tt Bitte versuche es noch mal! 
	  } 
	   } 
	   } 
	 & & \\ 
 

    \theutterance \stepcounter{utterance}  

    & & \multicolumn{4}{p{0.6\linewidth}}{\cellcolor[rgb]{0.9,0.9,0.9}{%
	\makecell[{{p{\linewidth}}}]{% 
	  \tt {\tiny [GM$\rangle$B]}  
	 Strafpunkt: (4,4) ist nicht leer, sondern enthält '┴'.\\ \tt \\ \tt Ihr habt gemeinsam 4 von maximal 16 Strafpunkten angesammelt.\\ \tt Bitte versuche es noch mal! 
	  } 
	   } 
	   } 
	 & \\ 
 

    \theutterance \stepcounter{utterance}  

    & & & \multicolumn{4}{p{0.6\linewidth}}{\cellcolor[rgb]{1,0.85,0.72}{%
	\makecell[{{p{\linewidth}}}]{% 
	  \tt {\tiny [GM$\langle$B]}  
	 VERSCHIEBE: O, (4,5) 
	  } 
	   } 
	   } 
	  \\ 
 

    \theutterance \stepcounter{utterance}  

    & & & \multicolumn{2}{p{0.3\linewidth}}{\cellcolor[rgb]{0.95,0.95,0.95}{%
	\makecell[{{p{\linewidth}}}]{% 
	  \tt {\tiny [GM$|$GM]}  
	 VERSCHIEBE: O, (4,5) 
	  } 
	   } 
	   } 
	 & & \\ 
 

    \theutterance \stepcounter{utterance}  

    & & & \multicolumn{2}{p{0.3\linewidth}}{\cellcolor[rgb]{0.95,0.95,0.95}{%
	\makecell[{{p{\linewidth}}}]{% 
	  \tt {\tiny [GM$|$GM]}  
	 'O' erfolgreich nach (4,5) verschoben. Dein Raster sieht jetzt so aus:\\ \tt ```\\ \tt  1234567\\ \tt ╔═╤═╤═╤═╗\\ \tt ║◌│◌│◌│P║ 1\\ \tt ║T└─┼─┴─╢ 2\\ \tt ║◌◌◌│M◌◌║ 3\\ \tt ╟──┬┴───╢ 4\\ \tt ║◌D│O◌◌◌║ 5\\ \tt ║◌◌│◌S◌◌║ 6\\ \tt ║A◌│◌◌◌◌║ 7\\ \tt ╚══╧════╝\\ \tt ```\\ \tt  
	  } 
	   } 
	   } 
	 & & \\ 
 

    \theutterance \stepcounter{utterance}  

    & & \multicolumn{4}{p{0.6\linewidth}}{\cellcolor[rgb]{0.9,0.9,0.9}{%
	\makecell[{{p{\linewidth}}}]{% 
	  \tt {\tiny [A$\langle$GM]}  
	 Deine Nachricht wurde an deinen Mitspieler übermittelt.\\ \tt \\ \tt \\ \tt Ihr seid gerade bei Runde 9 von maximal 28.\\ \tt Ihr habt gemeinsam 4 von maximal 16 Strafpunkten angesammelt.\\ \tt Dein Mitspieler hat ein Objekt auf seinem Raster bewegt.\\ \tt \\ \tt \\ \tt Wie lautet dein nächster Befehl? 
	  } 
	   } 
	   } 
	 & \\ 
 

    \theutterance \stepcounter{utterance}  

    & \multicolumn{4}{p{0.6\linewidth}}{\cellcolor[rgb]{0.8,1,0.9}{%
	\makecell[{{p{\linewidth}}}]{% 
	  \tt {\tiny [A$\rangle$GM]}  
	 VERSCHIEBE: O, (4,4) 
	  } 
	   } 
	   } 
	 & & \\ 
 

    \theutterance \stepcounter{utterance}  

    & & & \multicolumn{2}{p{0.3\linewidth}}{\cellcolor[rgb]{0.95,0.95,0.95}{%
	\makecell[{{p{\linewidth}}}]{% 
	  \tt {\tiny [GM$|$GM]}  
	 VERSCHIEBE: O, (4,4) 
	  } 
	   } 
	   } 
	 & & \\ 
 

    \theutterance \stepcounter{utterance}  

    & & & \multicolumn{2}{p{0.3\linewidth}}{\cellcolor[rgb]{0.95,0.95,0.95}{%
	\makecell[{{p{\linewidth}}}]{% 
	  \tt {\tiny [GM$|$GM]}  
	 Strafpunkt: (4,4) ist nicht leer, sondern enthält '┴'.\\ \tt \\ \tt Ihr habt gemeinsam 5 von maximal 16 Strafpunkten angesammelt.\\ \tt Bitte versuche es noch mal! 
	  } 
	   } 
	   } 
	 & & \\ 
 

    \theutterance \stepcounter{utterance}  

    & & \multicolumn{4}{p{0.6\linewidth}}{\cellcolor[rgb]{0.9,0.9,0.9}{%
	\makecell[{{p{\linewidth}}}]{% 
	  \tt {\tiny [A$\langle$GM]}  
	 Strafpunkt: (4,4) ist nicht leer, sondern enthält '┴'.\\ \tt \\ \tt Ihr habt gemeinsam 5 von maximal 16 Strafpunkten angesammelt.\\ \tt Bitte versuche es noch mal! 
	  } 
	   } 
	   } 
	 & \\ 
 

    \theutterance \stepcounter{utterance}  

    & \multicolumn{4}{p{0.6\linewidth}}{\cellcolor[rgb]{0.8,1,0.9}{%
	\makecell[{{p{\linewidth}}}]{% 
	  \tt {\tiny [A$\rangle$GM]}  
	 VERSCHIEBE: O, (4,5) 
	  } 
	   } 
	   } 
	 & & \\ 
 

    \theutterance \stepcounter{utterance}  

    & & & \multicolumn{2}{p{0.3\linewidth}}{\cellcolor[rgb]{0.95,0.95,0.95}{%
	\makecell[{{p{\linewidth}}}]{% 
	  \tt {\tiny [GM$|$GM]}  
	 VERSCHIEBE: O, (4,5) 
	  } 
	   } 
	   } 
	 & & \\ 
 

    \theutterance \stepcounter{utterance}  

    & & & \multicolumn{2}{p{0.3\linewidth}}{\cellcolor[rgb]{0.95,0.95,0.95}{%
	\makecell[{{p{\linewidth}}}]{% 
	  \tt {\tiny [GM$|$GM]}  
	 'O' erfolgreich nach (4,5) verschoben. Dein Raster sieht jetzt so aus:\\ \tt ```\\ \tt  1234567\\ \tt ╔═╤═╤═╤═╗\\ \tt ║P│◌│◌│◌║ 1\\ \tt ║◌└─┼─┴─╢ 2\\ \tt ║◌SA│◌◌◌║ 3\\ \tt ╟──┬┴───╢ 4\\ \tt ║◌◌│O◌◌◌║ 5\\ \tt ║◌◌│◌◌MD║ 6\\ \tt ║◌◌│◌◌◌T║ 7\\ \tt ╚══╧════╝\\ \tt ```\\ \tt  
	  } 
	   } 
	   } 
	 & & \\ 
 

    \theutterance \stepcounter{utterance}  

    & & \multicolumn{4}{p{0.6\linewidth}}{\cellcolor[rgb]{0.9,0.9,0.9}{%
	\makecell[{{p{\linewidth}}}]{% 
	  \tt {\tiny [GM$\rangle$B]}  
	 'O' erfolgreich nach (4,5) verschoben. Dein Raster sieht jetzt so aus:\\ \tt ```\\ \tt  1234567\\ \tt ╔═╤═╤═╤═╗\\ \tt ║◌│◌│◌│P║ 1\\ \tt ║T└─┼─┴─╢ 2\\ \tt ║◌◌◌│M◌◌║ 3\\ \tt ╟──┬┴───╢ 4\\ \tt ║◌D│O◌◌◌║ 5\\ \tt ║◌◌│◌S◌◌║ 6\\ \tt ║A◌│◌◌◌◌║ 7\\ \tt ╚══╧════╝\\ \tt ```\\ \tt Ihr seid gerade bei Runde 10 von maximal 28.\\ \tt Ihr habt gemeinsam 5 von maximal 16 Strafpunkten angesammelt.\\ \tt Dein Mitspieler hat ein Objekt auf seinem Raster bewegt.\\ \tt \\ \tt \\ \tt Wie lautet dein nächster Befehl? 
	  } 
	   } 
	   } 
	 & \\ 
 

    \theutterance \stepcounter{utterance}  

    & & & \multicolumn{4}{p{0.6\linewidth}}{\cellcolor[rgb]{1,0.85,0.72}{%
	\makecell[{{p{\linewidth}}}]{% 
	  \tt {\tiny [GM$\langle$B]}  
	 SPRICH: Mein Objekt O ist jetzt in (4,5), aber es sollte eigentlich in (4,4) sein. Da (4,4) jedoch durch '┴' blockiert ist, muss ich einen anderen Weg finden. Wie sieht es bei dir aus? Hast du dein Objekt O erfolgreich bewegt? 
	  } 
	   } 
	   } 
	  \\ 
 

    \theutterance \stepcounter{utterance}  

    & & & \multicolumn{2}{p{0.3\linewidth}}{\cellcolor[rgb]{0.95,0.95,0.95}{%
	\makecell[{{p{\linewidth}}}]{% 
	  \tt {\tiny [GM$|$GM]}  
	 SPRICH: Mein Objekt O ist jetzt in (4,5), aber es sollte eigentlich in (4,4) sein. Da (4,4) jedoch durch '┴' blockiert ist, muss ich einen anderen Weg finden. Wie sieht es bei dir aus? Hast du dein Objekt O erfolgreich bewegt? 
	  } 
	   } 
	   } 
	 & & \\ 
 

    \theutterance \stepcounter{utterance}  

    & & \multicolumn{4}{p{0.6\linewidth}}{\cellcolor[rgb]{0.9,0.9,0.9}{%
	\makecell[{{p{\linewidth}}}]{% 
	  \tt {\tiny [A$\langle$GM]}  
	 'O' erfolgreich nach (4,5) verschoben. Dein Raster sieht jetzt so aus:\\ \tt ```\\ \tt  1234567\\ \tt ╔═╤═╤═╤═╗\\ \tt ║P│◌│◌│◌║ 1\\ \tt ║◌└─┼─┴─╢ 2\\ \tt ║◌SA│◌◌◌║ 3\\ \tt ╟──┬┴───╢ 4\\ \tt ║◌◌│O◌◌◌║ 5\\ \tt ║◌◌│◌◌MD║ 6\\ \tt ║◌◌│◌◌◌T║ 7\\ \tt ╚══╧════╝\\ \tt ```\\ \tt Ihr seid gerade bei Runde 10 von maximal 28.\\ \tt Ihr habt gemeinsam 5 von maximal 16 Strafpunkten angesammelt.\\ \tt Dein Mitspieler hat dir diese Nachricht geschickt:\\ \tt \\ \tt \\ \tt "Mein Objekt O ist jetzt in (4,5), aber es sollte eigentlich in (4,4) sein. Da (4,4) jedoch durch '┴' blockiert ist, muss ich einen anderen Weg finden. Wie sieht es bei dir aus? Hast du dein Objekt O erfolgreich bewegt?"\\ \tt \\ \tt \\ \tt Wie lautet dein nächster Befehl? 
	  } 
	   } 
	   } 
	 & \\ 
 

    \theutterance \stepcounter{utterance}  

    & \multicolumn{4}{p{0.6\linewidth}}{\cellcolor[rgb]{0.8,1,0.9}{%
	\makecell[{{p{\linewidth}}}]{% 
	  \tt {\tiny [A$\rangle$GM]}  
	 SPRICH: Ja, mein Objekt O ist jetzt in (4,5), aber es sollte auch in (4,4) sein. Ich verstehe das Problem mit dem '┴'. Ich denke, wir sollten einen anderen Weg finden, um die Objekte an ihre Zielpositionen zu bringen. Vielleicht können wir einige Objekte vorübergehend an andere Positionen bewegen, um Platz zu schaffen? 
	  } 
	   } 
	   } 
	 & & \\ 
 

    \theutterance \stepcounter{utterance}  

    & & & \multicolumn{2}{p{0.3\linewidth}}{\cellcolor[rgb]{0.95,0.95,0.95}{%
	\makecell[{{p{\linewidth}}}]{% 
	  \tt {\tiny [GM$|$GM]}  
	 SPRICH: Ja, mein Objekt O ist jetzt in (4,5), aber es sollte auch in (4,4) sein. Ich verstehe das Problem mit dem '┴'. Ich denke, wir sollten einen anderen Weg finden, um die Objekte an ihre Zielpositionen zu bringen. Vielleicht können wir einige Objekte vorübergehend an andere Positionen bewegen, um Platz zu schaffen? 
	  } 
	   } 
	   } 
	 & & \\ 
 

    \theutterance \stepcounter{utterance}  

    & & \multicolumn{4}{p{0.6\linewidth}}{\cellcolor[rgb]{0.9,0.9,0.9}{%
	\makecell[{{p{\linewidth}}}]{% 
	  \tt {\tiny [GM$\rangle$B]}  
	 Deine Nachricht wurde an deinen Mitspieler übermittelt.\\ \tt \\ \tt \\ \tt Ihr seid gerade bei Runde 11 von maximal 28.\\ \tt Ihr habt gemeinsam 5 von maximal 16 Strafpunkten angesammelt.\\ \tt Dein Mitspieler hat dir diese Nachricht geschickt:\\ \tt \\ \tt \\ \tt "Ja, mein Objekt O ist jetzt in (4,5), aber es sollte auch in (4,4) sein. Ich verstehe das Problem mit dem '┴'. Ich denke, wir sollten einen anderen Weg finden, um die Objekte an ihre Zielpositionen zu bringen. Vielleicht können wir einige Objekte vorübergehend an andere Positionen bewegen, um Platz zu schaffen?"\\ \tt \\ \tt \\ \tt Wie lautet dein nächster Befehl? 
	  } 
	   } 
	   } 
	 & \\ 
 

    \theutterance \stepcounter{utterance}  

    & & & \multicolumn{4}{p{0.6\linewidth}}{\cellcolor[rgb]{1,0.85,0.72}{%
	\makecell[{{p{\linewidth}}}]{% 
	  \tt {\tiny [GM$\langle$B]}  
	 SPRICH: Das ist ein guter Vorschlag! Ich denke, wir sollten versuchen, das Objekt M zu bewegen, um Platz für das Objekt O zu schaffen. Mein Objekt M ist in (4,3), vielleicht kann ich es nach (5,3) bewegen, um Platz für das Objekt O in (4,4) zu schaffen. Was meinst du? Soll ich das Objekt M bewegen? 
	  } 
	   } 
	   } 
	  \\ 
 

    \theutterance \stepcounter{utterance}  

    & & & \multicolumn{2}{p{0.3\linewidth}}{\cellcolor[rgb]{0.95,0.95,0.95}{%
	\makecell[{{p{\linewidth}}}]{% 
	  \tt {\tiny [GM$|$GM]}  
	 SPRICH: Das ist ein guter Vorschlag! Ich denke, wir sollten versuchen, das Objekt M zu bewegen, um Platz für das Objekt O zu schaffen. Mein Objekt M ist in (4,3), vielleicht kann ich es nach (5,3) bewegen, um Platz für das Objekt O in (4,4) zu schaffen. Was meinst du? Soll ich das Objekt M bewegen? 
	  } 
	   } 
	   } 
	 & & \\ 
 

    \theutterance \stepcounter{utterance}  

    & & \multicolumn{4}{p{0.6\linewidth}}{\cellcolor[rgb]{0.9,0.9,0.9}{%
	\makecell[{{p{\linewidth}}}]{% 
	  \tt {\tiny [A$\langle$GM]}  
	 Deine Nachricht wurde an deinen Mitspieler übermittelt.\\ \tt \\ \tt \\ \tt Ihr seid gerade bei Runde 11 von maximal 28.\\ \tt Ihr habt gemeinsam 5 von maximal 16 Strafpunkten angesammelt.\\ \tt Dein Mitspieler hat dir diese Nachricht geschickt:\\ \tt \\ \tt \\ \tt "Das ist ein guter Vorschlag! Ich denke, wir sollten versuchen, das Objekt M zu bewegen, um Platz für das Objekt O zu schaffen. Mein Objekt M ist in (4,3), vielleicht kann ich es nach (5,3) bewegen, um Platz für das Objekt O in (4,4) zu schaffen. Was meinst du? Soll ich das Objekt M bewegen?"\\ \tt \\ \tt \\ \tt Wie lautet dein nächster Befehl? 
	  } 
	   } 
	   } 
	 & \\ 
 

    \theutterance \stepcounter{utterance}  

    & \multicolumn{4}{p{0.6\linewidth}}{\cellcolor[rgb]{0.8,1,0.9}{%
	\makecell[{{p{\linewidth}}}]{% 
	  \tt {\tiny [A$\rangle$GM]}  
	 SPRICH: Ja, das ist eine gute Idee! Mein Objekt M ist in (6,6), aber ich denke, wir sollten versuchen, es so zu bewegen, dass es nicht im Weg ist. Wenn du dein Objekt M nach (5,3) bewegen kannst, könnte ich vielleicht mein Objekt M nach (7,6) oder einer anderen freien Position bewegen, um Platz zu schaffen. Lass uns abwarten und sehen, wie es klappt. 
	  } 
	   } 
	   } 
	 & & \\ 
 

    \theutterance \stepcounter{utterance}  

    & & & \multicolumn{2}{p{0.3\linewidth}}{\cellcolor[rgb]{0.95,0.95,0.95}{%
	\makecell[{{p{\linewidth}}}]{% 
	  \tt {\tiny [GM$|$GM]}  
	 SPRICH: Ja, das ist eine gute Idee! Mein Objekt M ist in (6,6), aber ich denke, wir sollten versuchen, es so zu bewegen, dass es nicht im Weg ist. Wenn du dein Objekt M nach (5,3) bewegen kannst, könnte ich vielleicht mein Objekt M nach (7,6) oder einer anderen freien Position bewegen, um Platz zu schaffen. Lass uns abwarten und sehen, wie es klappt. 
	  } 
	   } 
	   } 
	 & & \\ 
 

    \theutterance \stepcounter{utterance}  

    & & \multicolumn{4}{p{0.6\linewidth}}{\cellcolor[rgb]{0.9,0.9,0.9}{%
	\makecell[{{p{\linewidth}}}]{% 
	  \tt {\tiny [GM$\rangle$B]}  
	 Deine Nachricht wurde an deinen Mitspieler übermittelt.\\ \tt \\ \tt \\ \tt Ihr seid gerade bei Runde 12 von maximal 28.\\ \tt Ihr habt gemeinsam 5 von maximal 16 Strafpunkten angesammelt.\\ \tt Dein Mitspieler hat dir diese Nachricht geschickt:\\ \tt \\ \tt \\ \tt "Ja, das ist eine gute Idee! Mein Objekt M ist in (6,6), aber ich denke, wir sollten versuchen, es so zu bewegen, dass es nicht im Weg ist. Wenn du dein Objekt M nach (5,3) bewegen kannst, könnte ich vielleicht mein Objekt M nach (7,6) oder einer anderen freien Position bewegen, um Platz zu schaffen. Lass uns abwarten und sehen, wie es klappt."\\ \tt \\ \tt \\ \tt Wie lautet dein nächster Befehl? 
	  } 
	   } 
	   } 
	 & \\ 
 

    \theutterance \stepcounter{utterance}  

    & & & \multicolumn{4}{p{0.6\linewidth}}{\cellcolor[rgb]{1,0.85,0.72}{%
	\makecell[{{p{\linewidth}}}]{% 
	  \tt {\tiny [GM$\langle$B]}  
	 VERSCHIEBE: M, (5,3) 
	  } 
	   } 
	   } 
	  \\ 
 

    \theutterance \stepcounter{utterance}  

    & & & \multicolumn{2}{p{0.3\linewidth}}{\cellcolor[rgb]{0.95,0.95,0.95}{%
	\makecell[{{p{\linewidth}}}]{% 
	  \tt {\tiny [GM$|$GM]}  
	 VERSCHIEBE: M, (5,3) 
	  } 
	   } 
	   } 
	 & & \\ 
 

    \theutterance \stepcounter{utterance}  

    & & & \multicolumn{2}{p{0.3\linewidth}}{\cellcolor[rgb]{0.95,0.95,0.95}{%
	\makecell[{{p{\linewidth}}}]{% 
	  \tt {\tiny [GM$|$GM]}  
	 Strafpunkt: (5,3) ist nicht leer, sondern enthält 'M'.\\ \tt \\ \tt Ihr habt gemeinsam 6 von maximal 16 Strafpunkten angesammelt.\\ \tt Bitte versuche es noch mal! 
	  } 
	   } 
	   } 
	 & & \\ 
 

    \theutterance \stepcounter{utterance}  

    & & \multicolumn{4}{p{0.6\linewidth}}{\cellcolor[rgb]{0.9,0.9,0.9}{%
	\makecell[{{p{\linewidth}}}]{% 
	  \tt {\tiny [GM$\rangle$B]}  
	 Strafpunkt: (5,3) ist nicht leer, sondern enthält 'M'.\\ \tt \\ \tt Ihr habt gemeinsam 6 von maximal 16 Strafpunkten angesammelt.\\ \tt Bitte versuche es noch mal! 
	  } 
	   } 
	   } 
	 & \\ 
 

    \theutterance \stepcounter{utterance}  

    & & & \multicolumn{4}{p{0.6\linewidth}}{\cellcolor[rgb]{1,0.85,0.72}{%
	\makecell[{{p{\linewidth}}}]{% 
	  \tt {\tiny [GM$\langle$B]}  
	 VERSCHIEBE: M, (3,3) 
	  } 
	   } 
	   } 
	  \\ 
 

    \theutterance \stepcounter{utterance}  

    & & & \multicolumn{2}{p{0.3\linewidth}}{\cellcolor[rgb]{0.95,0.95,0.95}{%
	\makecell[{{p{\linewidth}}}]{% 
	  \tt {\tiny [GM$|$GM]}  
	 VERSCHIEBE: M, (3,3) 
	  } 
	   } 
	   } 
	 & & \\ 
 

    \theutterance \stepcounter{utterance}  

    & & & \multicolumn{2}{p{0.3\linewidth}}{\cellcolor[rgb]{0.95,0.95,0.95}{%
	\makecell[{{p{\linewidth}}}]{% 
	  \tt {\tiny [GM$|$GM]}  
	 'M' erfolgreich nach (3,3) verschoben. Dein Raster sieht jetzt so aus:\\ \tt ```\\ \tt  1234567\\ \tt ╔═╤═╤═╤═╗\\ \tt ║◌│◌│◌│P║ 1\\ \tt ║T└─┼─┴─╢ 2\\ \tt ║◌◌M│◌◌◌║ 3\\ \tt ╟──┬┴───╢ 4\\ \tt ║◌D│O◌◌◌║ 5\\ \tt ║◌◌│◌S◌◌║ 6\\ \tt ║A◌│◌◌◌◌║ 7\\ \tt ╚══╧════╝\\ \tt ```\\ \tt  
	  } 
	   } 
	   } 
	 & & \\ 
 

    \theutterance \stepcounter{utterance}  

    & & \multicolumn{4}{p{0.6\linewidth}}{\cellcolor[rgb]{0.9,0.9,0.9}{%
	\makecell[{{p{\linewidth}}}]{% 
	  \tt {\tiny [A$\langle$GM]}  
	 Deine Nachricht wurde an deinen Mitspieler übermittelt.\\ \tt \\ \tt \\ \tt Ihr seid gerade bei Runde 12 von maximal 28.\\ \tt Ihr habt gemeinsam 6 von maximal 16 Strafpunkten angesammelt.\\ \tt Dein Mitspieler hat ein Objekt auf seinem Raster bewegt.\\ \tt \\ \tt \\ \tt Wie lautet dein nächster Befehl? 
	  } 
	   } 
	   } 
	 & \\ 
 

    \theutterance \stepcounter{utterance}  

    & \multicolumn{4}{p{0.6\linewidth}}{\cellcolor[rgb]{0.8,1,0.9}{%
	\makecell[{{p{\linewidth}}}]{% 
	  \tt {\tiny [A$\rangle$GM]}  
	 SPRICH: Wie sieht es jetzt bei dir aus? Hast du dein Objekt M erfolgreich bewegt? 
	  } 
	   } 
	   } 
	 & & \\ 
 

    \theutterance \stepcounter{utterance}  

    & & & \multicolumn{2}{p{0.3\linewidth}}{\cellcolor[rgb]{0.95,0.95,0.95}{%
	\makecell[{{p{\linewidth}}}]{% 
	  \tt {\tiny [GM$|$GM]}  
	 SPRICH: Wie sieht es jetzt bei dir aus? Hast du dein Objekt M erfolgreich bewegt? 
	  } 
	   } 
	   } 
	 & & \\ 
 

    \theutterance \stepcounter{utterance}  

    & & \multicolumn{4}{p{0.6\linewidth}}{\cellcolor[rgb]{0.9,0.9,0.9}{%
	\makecell[{{p{\linewidth}}}]{% 
	  \tt {\tiny [GM$\rangle$B]}  
	 'M' erfolgreich nach (3,3) verschoben. Dein Raster sieht jetzt so aus:\\ \tt ```\\ \tt  1234567\\ \tt ╔═╤═╤═╤═╗\\ \tt ║◌│◌│◌│P║ 1\\ \tt ║T└─┼─┴─╢ 2\\ \tt ║◌◌M│◌◌◌║ 3\\ \tt ╟──┬┴───╢ 4\\ \tt ║◌D│O◌◌◌║ 5\\ \tt ║◌◌│◌S◌◌║ 6\\ \tt ║A◌│◌◌◌◌║ 7\\ \tt ╚══╧════╝\\ \tt ```\\ \tt Ihr seid gerade bei Runde 13 von maximal 28.\\ \tt Ihr habt gemeinsam 6 von maximal 16 Strafpunkten angesammelt.\\ \tt Dein Mitspieler hat dir diese Nachricht geschickt:\\ \tt \\ \tt \\ \tt "Wie sieht es jetzt bei dir aus? Hast du dein Objekt M erfolgreich bewegt?"\\ \tt \\ \tt \\ \tt Wie lautet dein nächster Befehl? 
	  } 
	   } 
	   } 
	 & \\ 
 

    \theutterance \stepcounter{utterance}  

    & & & \multicolumn{4}{p{0.6\linewidth}}{\cellcolor[rgb]{1,0.85,0.72}{%
	\makecell[{{p{\linewidth}}}]{% 
	  \tt {\tiny [GM$\langle$B]}  
	 SPRICH: Ja, ich habe mein Objekt M erfolgreich nach (3,3) bewegt. Mein Raster sieht jetzt so aus, dass das Objekt O in (4,5) ist und das Objekt M in (3,3) ist. Ich denke, wir sollten jetzt versuchen, das Objekt O nach (4,4) zu bewegen, aber ich weiß nicht, ob das möglich ist, da das '┴' noch immer im Weg ist. Was meinst du? Soll ich versuchen, das Objekt O zu bewegen? 
	  } 
	   } 
	   } 
	  \\ 
 

    \theutterance \stepcounter{utterance}  

    & & & \multicolumn{2}{p{0.3\linewidth}}{\cellcolor[rgb]{0.95,0.95,0.95}{%
	\makecell[{{p{\linewidth}}}]{% 
	  \tt {\tiny [GM$|$GM]}  
	 SPRICH: Ja, ich habe mein Objekt M erfolgreich nach (3,3) bewegt. Mein Raster sieht jetzt so aus, dass das Objekt O in (4,5) ist und das Objekt M in (3,3) ist. Ich denke, wir sollten jetzt versuchen, das Objekt O nach (4,4) zu bewegen, aber ich weiß nicht, ob das möglich ist, da das '┴' noch immer im Weg ist. Was meinst du? Soll ich versuchen, das Objekt O zu bewegen? 
	  } 
	   } 
	   } 
	 & & \\ 
 

    \theutterance \stepcounter{utterance}  

    & & \multicolumn{4}{p{0.6\linewidth}}{\cellcolor[rgb]{0.9,0.9,0.9}{%
	\makecell[{{p{\linewidth}}}]{% 
	  \tt {\tiny [A$\langle$GM]}  
	 Deine Nachricht wurde an deinen Mitspieler übermittelt.\\ \tt \\ \tt \\ \tt Ihr seid gerade bei Runde 13 von maximal 28.\\ \tt Ihr habt gemeinsam 6 von maximal 16 Strafpunkten angesammelt.\\ \tt Dein Mitspieler hat dir diese Nachricht geschickt:\\ \tt \\ \tt \\ \tt "Ja, ich habe mein Objekt M erfolgreich nach (3,3) bewegt. Mein Raster sieht jetzt so aus, dass das Objekt O in (4,5) ist und das Objekt M in (3,3) ist. Ich denke, wir sollten jetzt versuchen, das Objekt O nach (4,4) zu bewegen, aber ich weiß nicht, ob das möglich ist, da das '┴' noch immer im Weg ist. Was meinst du? Soll ich versuchen, das Objekt O zu bewegen?"\\ \tt \\ \tt \\ \tt Wie lautet dein nächster Befehl? 
	  } 
	   } 
	   } 
	 & \\ 
 

    \theutterance \stepcounter{utterance}  

    & \multicolumn{4}{p{0.6\linewidth}}{\cellcolor[rgb]{0.8,1,0.9}{%
	\makecell[{{p{\linewidth}}}]{% 
	  \tt {\tiny [A$\rangle$GM]}  
	 SPRICH: Ich denke, wir sollten versuchen, das '┴' aus dem Weg zu räumen, bevor wir das Objekt O bewegen. Kannst du vielleicht ein anderes Objekt bewegen, um Platz zu schaffen? Oder sollten wir einen anderen Plan erstellen? 
	  } 
	   } 
	   } 
	 & & \\ 
 

    \theutterance \stepcounter{utterance}  

    & & & \multicolumn{2}{p{0.3\linewidth}}{\cellcolor[rgb]{0.95,0.95,0.95}{%
	\makecell[{{p{\linewidth}}}]{% 
	  \tt {\tiny [GM$|$GM]}  
	 SPRICH: Ich denke, wir sollten versuchen, das '┴' aus dem Weg zu räumen, bevor wir das Objekt O bewegen. Kannst du vielleicht ein anderes Objekt bewegen, um Platz zu schaffen? Oder sollten wir einen anderen Plan erstellen? 
	  } 
	   } 
	   } 
	 & & \\ 
 

    \theutterance \stepcounter{utterance}  

    & & \multicolumn{4}{p{0.6\linewidth}}{\cellcolor[rgb]{0.9,0.9,0.9}{%
	\makecell[{{p{\linewidth}}}]{% 
	  \tt {\tiny [GM$\rangle$B]}  
	 Deine Nachricht wurde an deinen Mitspieler übermittelt.\\ \tt \\ \tt \\ \tt Ihr seid gerade bei Runde 14 von maximal 28.\\ \tt Ihr habt gemeinsam 6 von maximal 16 Strafpunkten angesammelt.\\ \tt Dein Mitspieler hat dir diese Nachricht geschickt:\\ \tt \\ \tt \\ \tt "Ich denke, wir sollten versuchen, das '┴' aus dem Weg zu räumen, bevor wir das Objekt O bewegen. Kannst du vielleicht ein anderes Objekt bewegen, um Platz zu schaffen? Oder sollten wir einen anderen Plan erstellen?"\\ \tt \\ \tt \\ \tt Wie lautet dein nächster Befehl? 
	  } 
	   } 
	   } 
	 & \\ 
 

    \theutterance \stepcounter{utterance}  

    & & & \multicolumn{4}{p{0.6\linewidth}}{\cellcolor[rgb]{1,0.85,0.72}{%
	\makecell[{{p{\linewidth}}}]{% 
	  \tt {\tiny [GM$\langle$B]}  
	 SPRICH: Ich denke, wir sollten versuchen, das Objekt D zu bewegen, um Platz zu schaffen. Mein Objekt D ist in (2,5), vielleicht kann ich es nach (2,6) bewegen, um Platz für die anderen Objekte zu schaffen. Was meinst du? Soll ich das Objekt D bewegen? 
	  } 
	   } 
	   } 
	  \\ 
 

    \theutterance \stepcounter{utterance}  

    & & & \multicolumn{2}{p{0.3\linewidth}}{\cellcolor[rgb]{0.95,0.95,0.95}{%
	\makecell[{{p{\linewidth}}}]{% 
	  \tt {\tiny [GM$|$GM]}  
	 SPRICH: Ich denke, wir sollten versuchen, das Objekt D zu bewegen, um Platz zu schaffen. Mein Objekt D ist in (2,5), vielleicht kann ich es nach (2,6) bewegen, um Platz für die anderen Objekte zu schaffen. Was meinst du? Soll ich das Objekt D bewegen? 
	  } 
	   } 
	   } 
	 & & \\ 
 

    \theutterance \stepcounter{utterance}  

    & & \multicolumn{4}{p{0.6\linewidth}}{\cellcolor[rgb]{0.9,0.9,0.9}{%
	\makecell[{{p{\linewidth}}}]{% 
	  \tt {\tiny [A$\langle$GM]}  
	 Deine Nachricht wurde an deinen Mitspieler übermittelt.\\ \tt \\ \tt \\ \tt Ihr seid gerade bei Runde 14 von maximal 28.\\ \tt Ihr habt gemeinsam 6 von maximal 16 Strafpunkten angesammelt.\\ \tt Dein Mitspieler hat dir diese Nachricht geschickt:\\ \tt \\ \tt \\ \tt "Ich denke, wir sollten versuchen, das Objekt D zu bewegen, um Platz zu schaffen. Mein Objekt D ist in (2,5), vielleicht kann ich es nach (2,6) bewegen, um Platz für die anderen Objekte zu schaffen. Was meinst du? Soll ich das Objekt D bewegen?"\\ \tt \\ \tt \\ \tt Wie lautet dein nächster Befehl? 
	  } 
	   } 
	   } 
	 & \\ 
 

    \theutterance \stepcounter{utterance}  

    & \multicolumn{4}{p{0.6\linewidth}}{\cellcolor[rgb]{0.8,1,0.9}{%
	\makecell[{{p{\linewidth}}}]{% 
	  \tt {\tiny [A$\rangle$GM]}  
	 SPRICH: Ja, das ist eine gute Idee! Mein Objekt D ist in (7,1), aber ich denke, es wäre hilfreich, wenn du dein Objekt D nach (2,6) bewegen kannst, um Platz zu schaffen. Ich frage mich, ob ich vielleicht mein Objekt D nach (7,2) oder einer anderen freien Position bewegen kann, um auch Platz zu schaffen. Lass uns abwarten und sehen, wie es klappt. 
	  } 
	   } 
	   } 
	 & & \\ 
 

    \theutterance \stepcounter{utterance}  

    & & & \multicolumn{2}{p{0.3\linewidth}}{\cellcolor[rgb]{0.95,0.95,0.95}{%
	\makecell[{{p{\linewidth}}}]{% 
	  \tt {\tiny [GM$|$GM]}  
	 SPRICH: Ja, das ist eine gute Idee! Mein Objekt D ist in (7,1), aber ich denke, es wäre hilfreich, wenn du dein Objekt D nach (2,6) bewegen kannst, um Platz zu schaffen. Ich frage mich, ob ich vielleicht mein Objekt D nach (7,2) oder einer anderen freien Position bewegen kann, um auch Platz zu schaffen. Lass uns abwarten und sehen, wie es klappt. 
	  } 
	   } 
	   } 
	 & & \\ 
 

    \theutterance \stepcounter{utterance}  

    & & \multicolumn{4}{p{0.6\linewidth}}{\cellcolor[rgb]{0.9,0.9,0.9}{%
	\makecell[{{p{\linewidth}}}]{% 
	  \tt {\tiny [GM$\rangle$B]}  
	 Deine Nachricht wurde an deinen Mitspieler übermittelt.\\ \tt \\ \tt \\ \tt Ihr seid gerade bei Runde 15 von maximal 28.\\ \tt Ihr habt gemeinsam 6 von maximal 16 Strafpunkten angesammelt.\\ \tt Dein Mitspieler hat dir diese Nachricht geschickt:\\ \tt \\ \tt \\ \tt "Ja, das ist eine gute Idee! Mein Objekt D ist in (7,1), aber ich denke, es wäre hilfreich, wenn du dein Objekt D nach (2,6) bewegen kannst, um Platz zu schaffen. Ich frage mich, ob ich vielleicht mein Objekt D nach (7,2) oder einer anderen freien Position bewegen kann, um auch Platz zu schaffen. Lass uns abwarten und sehen, wie es klappt."\\ \tt \\ \tt \\ \tt Wie lautet dein nächster Befehl? 
	  } 
	   } 
	   } 
	 & \\ 
 

    \theutterance \stepcounter{utterance}  

    & & & \multicolumn{4}{p{0.6\linewidth}}{\cellcolor[rgb]{1,0.85,0.72}{%
	\makecell[{{p{\linewidth}}}]{% 
	  \tt {\tiny [GM$\langle$B]}  
	 VERSCHIEBE: D, (2,6) 
	  } 
	   } 
	   } 
	  \\ 
 

    \theutterance \stepcounter{utterance}  

    & & & \multicolumn{2}{p{0.3\linewidth}}{\cellcolor[rgb]{0.95,0.95,0.95}{%
	\makecell[{{p{\linewidth}}}]{% 
	  \tt {\tiny [GM$|$GM]}  
	 VERSCHIEBE: D, (2,6) 
	  } 
	   } 
	   } 
	 & & \\ 
 

    \theutterance \stepcounter{utterance}  

    & & & \multicolumn{2}{p{0.3\linewidth}}{\cellcolor[rgb]{0.95,0.95,0.95}{%
	\makecell[{{p{\linewidth}}}]{% 
	  \tt {\tiny [GM$|$GM]}  
	 'D' erfolgreich nach (2,6) verschoben. Dein Raster sieht jetzt so aus:\\ \tt ```\\ \tt  1234567\\ \tt ╔═╤═╤═╤═╗\\ \tt ║◌│◌│◌│P║ 1\\ \tt ║T└─┼─┴─╢ 2\\ \tt ║◌◌M│◌◌◌║ 3\\ \tt ╟──┬┴───╢ 4\\ \tt ║◌◌│O◌◌◌║ 5\\ \tt ║◌D│◌S◌◌║ 6\\ \tt ║A◌│◌◌◌◌║ 7\\ \tt ╚══╧════╝\\ \tt ```\\ \tt  
	  } 
	   } 
	   } 
	 & & \\ 
 

    \theutterance \stepcounter{utterance}  

    & & \multicolumn{4}{p{0.6\linewidth}}{\cellcolor[rgb]{0.9,0.9,0.9}{%
	\makecell[{{p{\linewidth}}}]{% 
	  \tt {\tiny [A$\langle$GM]}  
	 Deine Nachricht wurde an deinen Mitspieler übermittelt.\\ \tt \\ \tt \\ \tt Ihr seid gerade bei Runde 15 von maximal 28.\\ \tt Ihr habt gemeinsam 6 von maximal 16 Strafpunkten angesammelt.\\ \tt Dein Mitspieler hat ein Objekt auf seinem Raster bewegt.\\ \tt \\ \tt \\ \tt Wie lautet dein nächster Befehl? 
	  } 
	   } 
	   } 
	 & \\ 
 

    \theutterance \stepcounter{utterance}  

    & \multicolumn{4}{p{0.6\linewidth}}{\cellcolor[rgb]{0.8,1,0.9}{%
	\makecell[{{p{\linewidth}}}]{% 
	  \tt {\tiny [A$\rangle$GM]}  
	 SPRICH: Wie sieht es jetzt bei dir aus? Hast du dein Objekt D erfolgreich bewegt? 
	  } 
	   } 
	   } 
	 & & \\ 
 

    \theutterance \stepcounter{utterance}  

    & & & \multicolumn{2}{p{0.3\linewidth}}{\cellcolor[rgb]{0.95,0.95,0.95}{%
	\makecell[{{p{\linewidth}}}]{% 
	  \tt {\tiny [GM$|$GM]}  
	 SPRICH: Wie sieht es jetzt bei dir aus? Hast du dein Objekt D erfolgreich bewegt? 
	  } 
	   } 
	   } 
	 & & \\ 
 

    \theutterance \stepcounter{utterance}  

    & & \multicolumn{4}{p{0.6\linewidth}}{\cellcolor[rgb]{0.9,0.9,0.9}{%
	\makecell[{{p{\linewidth}}}]{% 
	  \tt {\tiny [GM$\rangle$B]}  
	 'D' erfolgreich nach (2,6) verschoben. Dein Raster sieht jetzt so aus:\\ \tt ```\\ \tt  1234567\\ \tt ╔═╤═╤═╤═╗\\ \tt ║◌│◌│◌│P║ 1\\ \tt ║T└─┼─┴─╢ 2\\ \tt ║◌◌M│◌◌◌║ 3\\ \tt ╟──┬┴───╢ 4\\ \tt ║◌◌│O◌◌◌║ 5\\ \tt ║◌D│◌S◌◌║ 6\\ \tt ║A◌│◌◌◌◌║ 7\\ \tt ╚══╧════╝\\ \tt ```\\ \tt Ihr seid gerade bei Runde 16 von maximal 28.\\ \tt Ihr habt gemeinsam 6 von maximal 16 Strafpunkten angesammelt.\\ \tt Dein Mitspieler hat dir diese Nachricht geschickt:\\ \tt \\ \tt \\ \tt "Wie sieht es jetzt bei dir aus? Hast du dein Objekt D erfolgreich bewegt?"\\ \tt \\ \tt \\ \tt Wie lautet dein nächster Befehl? 
	  } 
	   } 
	   } 
	 & \\ 
 

    \theutterance \stepcounter{utterance}  

    & & & \multicolumn{4}{p{0.6\linewidth}}{\cellcolor[rgb]{1,0.85,0.72}{%
	\makecell[{{p{\linewidth}}}]{% 
	  \tt {\tiny [GM$\langle$B]}  
	 SPRICH: Ja, ich habe mein Objekt D erfolgreich nach (2,6) bewegt. Mein Raster sieht jetzt so aus, dass das Objekt D in (2,6) ist und das Objekt O in (4,5) ist. Ich denke, wir sollten jetzt versuchen, das Objekt O nach (4,4) zu bewegen. Da das '┴' immer noch im Weg ist, frage ich mich, ob wir es perhaps entfernen oder umgehen können. Was meinst du? Soll ich versuchen, das Objekt O zu bewegen oder einen anderen Plan erstellen? 
	  } 
	   } 
	   } 
	  \\ 
 

    \theutterance \stepcounter{utterance}  

    & & & \multicolumn{2}{p{0.3\linewidth}}{\cellcolor[rgb]{0.95,0.95,0.95}{%
	\makecell[{{p{\linewidth}}}]{% 
	  \tt {\tiny [GM$|$GM]}  
	 SPRICH: Ja, ich habe mein Objekt D erfolgreich nach (2,6) bewegt. Mein Raster sieht jetzt so aus, dass das Objekt D in (2,6) ist und das Objekt O in (4,5) ist. Ich denke, wir sollten jetzt versuchen, das Objekt O nach (4,4) zu bewegen. Da das '┴' immer noch im Weg ist, frage ich mich, ob wir es perhaps entfernen oder umgehen können. Was meinst du? Soll ich versuchen, das Objekt O zu bewegen oder einen anderen Plan erstellen? 
	  } 
	   } 
	   } 
	 & & \\ 
 

    \theutterance \stepcounter{utterance}  

    & & \multicolumn{4}{p{0.6\linewidth}}{\cellcolor[rgb]{0.9,0.9,0.9}{%
	\makecell[{{p{\linewidth}}}]{% 
	  \tt {\tiny [A$\langle$GM]}  
	 Deine Nachricht wurde an deinen Mitspieler übermittelt.\\ \tt \\ \tt \\ \tt Ihr seid gerade bei Runde 16 von maximal 28.\\ \tt Ihr habt gemeinsam 6 von maximal 16 Strafpunkten angesammelt.\\ \tt Dein Mitspieler hat dir diese Nachricht geschickt:\\ \tt \\ \tt \\ \tt "Ja, ich habe mein Objekt D erfolgreich nach (2,6) bewegt. Mein Raster sieht jetzt so aus, dass das Objekt D in (2,6) ist und das Objekt O in (4,5) ist. Ich denke, wir sollten jetzt versuchen, das Objekt O nach (4,4) zu bewegen. Da das '┴' immer noch im Weg ist, frage ich mich, ob wir es perhaps entfernen oder umgehen können. Was meinst du? Soll ich versuchen, das Objekt O zu bewegen oder einen anderen Plan erstellen?"\\ \tt \\ \tt \\ \tt Wie lautet dein nächster Befehl? 
	  } 
	   } 
	   } 
	 & \\ 
 

    \theutterance \stepcounter{utterance}  

    & \multicolumn{4}{p{0.6\linewidth}}{\cellcolor[rgb]{0.8,1,0.9}{%
	\makecell[{{p{\linewidth}}}]{% 
	  \tt {\tiny [A$\rangle$GM]}  
	 SPRICH: Ich denke, wir sollten nochmal unser Raster und die Objekte überprüfen, um einen besseren Plan zu erstellen. Vielleicht gibt es eine Möglichkeit, das '┴' zu umgehen oder andere Objekte zu bewegen, um Platz zu schaffen. Lass uns nochmal unsere Objekte und ihre Positionen überprüfen, bevor wir weitere Schritte unternehmen. 
	  } 
	   } 
	   } 
	 & & \\ 
 

    \theutterance \stepcounter{utterance}  

    & & & \multicolumn{2}{p{0.3\linewidth}}{\cellcolor[rgb]{0.95,0.95,0.95}{%
	\makecell[{{p{\linewidth}}}]{% 
	  \tt {\tiny [GM$|$GM]}  
	 SPRICH: Ich denke, wir sollten nochmal unser Raster und die Objekte überprüfen, um einen besseren Plan zu erstellen. Vielleicht gibt es eine Möglichkeit, das '┴' zu umgehen oder andere Objekte zu bewegen, um Platz zu schaffen. Lass uns nochmal unsere Objekte und ihre Positionen überprüfen, bevor wir weitere Schritte unternehmen. 
	  } 
	   } 
	   } 
	 & & \\ 
 

    \theutterance \stepcounter{utterance}  

    & & \multicolumn{4}{p{0.6\linewidth}}{\cellcolor[rgb]{0.9,0.9,0.9}{%
	\makecell[{{p{\linewidth}}}]{% 
	  \tt {\tiny [GM$\rangle$B]}  
	 Deine Nachricht wurde an deinen Mitspieler übermittelt.\\ \tt \\ \tt \\ \tt Ihr seid gerade bei Runde 17 von maximal 28.\\ \tt Ihr habt gemeinsam 6 von maximal 16 Strafpunkten angesammelt.\\ \tt Dein Mitspieler hat dir diese Nachricht geschickt:\\ \tt \\ \tt \\ \tt "Ich denke, wir sollten nochmal unser Raster und die Objekte überprüfen, um einen besseren Plan zu erstellen. Vielleicht gibt es eine Möglichkeit, das '┴' zu umgehen oder andere Objekte zu bewegen, um Platz zu schaffen. Lass uns nochmal unsere Objekte und ihre Positionen überprüfen, bevor wir weitere Schritte unternehmen."\\ \tt \\ \tt \\ \tt Wie lautet dein nächster Befehl? 
	  } 
	   } 
	   } 
	 & \\ 
 

    \theutterance \stepcounter{utterance}  

    & & & \multicolumn{4}{p{0.6\linewidth}}{\cellcolor[rgb]{1,0.85,0.72}{%
	\makecell[{{p{\linewidth}}}]{% 
	  \tt {\tiny [GM$\langle$B]}  
	 SPRICH: Okay, ich überprüfe mein Raster nochmal. Mein Raster sieht so aus: P in (1,1), T in (2,2), M in (3,3), O in (4,5), D in (2,6), S in (5,6), A in (1,7). Ich denke, wir sollten nochmal unsere Zielpositionen überprüfen und einen Plan erstellen, um die Objekte dorthin zu bewegen. Wie sehen deine Objekte und ihre Positionen aus? Können wir unsere Zielpositionen nochmal durchgehen und einen neuen Plan erstellen? 
	  } 
	   } 
	   } 
	  \\ 
 

    \theutterance \stepcounter{utterance}  

    & & & \multicolumn{2}{p{0.3\linewidth}}{\cellcolor[rgb]{0.95,0.95,0.95}{%
	\makecell[{{p{\linewidth}}}]{% 
	  \tt {\tiny [GM$|$GM]}  
	 SPRICH: Okay, ich überprüfe mein Raster nochmal. Mein Raster sieht so aus: P in (1,1), T in (2,2), M in (3,3), O in (4,5), D in (2,6), S in (5,6), A in (1,7). Ich denke, wir sollten nochmal unsere Zielpositionen überprüfen und einen Plan erstellen, um die Objekte dorthin zu bewegen. Wie sehen deine Objekte und ihre Positionen aus? Können wir unsere Zielpositionen nochmal durchgehen und einen neuen Plan erstellen? 
	  } 
	   } 
	   } 
	 & & \\ 
 

    \theutterance \stepcounter{utterance}  

    & & \multicolumn{4}{p{0.6\linewidth}}{\cellcolor[rgb]{0.9,0.9,0.9}{%
	\makecell[{{p{\linewidth}}}]{% 
	  \tt {\tiny [A$\langle$GM]}  
	 Deine Nachricht wurde an deinen Mitspieler übermittelt.\\ \tt \\ \tt \\ \tt Ihr seid gerade bei Runde 17 von maximal 28.\\ \tt Ihr habt gemeinsam 6 von maximal 16 Strafpunkten angesammelt.\\ \tt Dein Mitspieler hat dir diese Nachricht geschickt:\\ \tt \\ \tt \\ \tt "Okay, ich überprüfe mein Raster nochmal. Mein Raster sieht so aus: P in (1,1), T in (2,2), M in (3,3), O in (4,5), D in (2,6), S in (5,6), A in (1,7). Ich denke, wir sollten nochmal unsere Zielpositionen überprüfen und einen Plan erstellen, um die Objekte dorthin zu bewegen. Wie sehen deine Objekte und ihre Positionen aus? Können wir unsere Zielpositionen nochmal durchgehen und einen neuen Plan erstellen?"\\ \tt \\ \tt \\ \tt Wie lautet dein nächster Befehl? 
	  } 
	   } 
	   } 
	 & \\ 
 

    \theutterance \stepcounter{utterance}  

    & \multicolumn{4}{p{0.6\linewidth}}{\cellcolor[rgb]{0.8,1,0.9}{%
	\makecell[{{p{\linewidth}}}]{% 
	  \tt {\tiny [A$\rangle$GM]}  
	 SPRICH: Mein Raster sieht so aus: P in (1,1), O in (4,5), T in (7,7), S in (3,3), A in (3,3), M in (6,6), D in (7,1). Unsere Zielpositionen sind: P in (1,1), O in (4,4), T in (7,7), S in (1,7), D in (7,1), A in (1,4), M in (7,4). Ich denke, wir sollten unsere Objekte schrittweise an diese Zielpositionen bewegen, beginnend mit den Objekten, die am wenigsten bewegt werden müssen. Lass uns einen Schritt-für-Schritt-Plan erstellen. 
	  } 
	   } 
	   } 
	 & & \\ 
 

    \theutterance \stepcounter{utterance}  

    & & & \multicolumn{2}{p{0.3\linewidth}}{\cellcolor[rgb]{0.95,0.95,0.95}{%
	\makecell[{{p{\linewidth}}}]{% 
	  \tt {\tiny [GM$|$GM]}  
	 SPRICH: Mein Raster sieht so aus: P in (1,1), O in (4,5), T in (7,7), S in (3,3), A in (3,3), M in (6,6), D in (7,1). Unsere Zielpositionen sind: P in (1,1), O in (4,4), T in (7,7), S in (1,7), D in (7,1), A in (1,4), M in (7,4). Ich denke, wir sollten unsere Objekte schrittweise an diese Zielpositionen bewegen, beginnend mit den Objekten, die am wenigsten bewegt werden müssen. Lass uns einen Schritt-für-Schritt-Plan erstellen. 
	  } 
	   } 
	   } 
	 & & \\ 
 

    \theutterance \stepcounter{utterance}  

    & & \multicolumn{4}{p{0.6\linewidth}}{\cellcolor[rgb]{0.9,0.9,0.9}{%
	\makecell[{{p{\linewidth}}}]{% 
	  \tt {\tiny [GM$\rangle$B]}  
	 Deine Nachricht wurde an deinen Mitspieler übermittelt.\\ \tt \\ \tt \\ \tt Ihr seid gerade bei Runde 18 von maximal 28.\\ \tt Ihr habt gemeinsam 6 von maximal 16 Strafpunkten angesammelt.\\ \tt Dein Mitspieler hat dir diese Nachricht geschickt:\\ \tt \\ \tt \\ \tt "Mein Raster sieht so aus: P in (1,1), O in (4,5), T in (7,7), S in (3,3), A in (3,3), M in (6,6), D in (7,1). Unsere Zielpositionen sind: P in (1,1), O in (4,4), T in (7,7), S in (1,7), D in (7,1), A in (1,4), M in (7,4). Ich denke, wir sollten unsere Objekte schrittweise an diese Zielpositionen bewegen, beginnend mit den Objekten, die am wenigsten bewegt werden müssen. Lass uns einen Schritt-für-Schritt-Plan erstellen."\\ \tt \\ \tt \\ \tt Wie lautet dein nächster Befehl? 
	  } 
	   } 
	   } 
	 & \\ 
 

    \theutterance \stepcounter{utterance}  

    & & & \multicolumn{4}{p{0.6\linewidth}}{\cellcolor[rgb]{1,0.85,0.72}{%
	\makecell[{{p{\linewidth}}}]{% 
	  \tt {\tiny [GM$\langle$B]}  
	 SPRICH: Ich denke, das ist ein guter Ansatz. Lass uns die Objekte überprüfen, die am wenigsten bewegt werden müssen. Mein Objekt P ist bereits an der Zielposition (1,1). Mein Objekt T ist nicht an der Zielposition, aber es ist auch nicht nötig, es zu bewegen, da es bei dir bereits an der Zielposition (7,7) ist. Mein Objekt O muss nach (4,4) bewegt werden, aber wie ich bereits erwähnt habe, ist das '┴' im Weg. Wir sollten perhaps einen Plan erstellen, um das '┴' zu umgehen oder zu entfernen. Was denkst du? Sollten wir versuchen, mein Objekt O nach (4,4) zu bewegen oder ein anderes Objekt bewegen, um Platz zu schaffen? 
	  } 
	   } 
	   } 
	  \\ 
 

    \theutterance \stepcounter{utterance}  

    & & & \multicolumn{2}{p{0.3\linewidth}}{\cellcolor[rgb]{0.95,0.95,0.95}{%
	\makecell[{{p{\linewidth}}}]{% 
	  \tt {\tiny [GM$|$GM]}  
	 SPRICH: Ich denke, das ist ein guter Ansatz. Lass uns die Objekte überprüfen, die am wenigsten bewegt werden müssen. Mein Objekt P ist bereits an der Zielposition (1,1). Mein Objekt T ist nicht an der Zielposition, aber es ist auch nicht nötig, es zu bewegen, da es bei dir bereits an der Zielposition (7,7) ist. Mein Objekt O muss nach (4,4) bewegt werden, aber wie ich bereits erwähnt habe, ist das '┴' im Weg. Wir sollten perhaps einen Plan erstellen, um das '┴' zu umgehen oder zu entfernen. Was denkst du? Sollten wir versuchen, mein Objekt O nach (4,4) zu bewegen oder ein anderes Objekt bewegen, um Platz zu schaffen? 
	  } 
	   } 
	   } 
	 & & \\ 
 

    \theutterance \stepcounter{utterance}  

    & & \multicolumn{4}{p{0.6\linewidth}}{\cellcolor[rgb]{0.9,0.9,0.9}{%
	\makecell[{{p{\linewidth}}}]{% 
	  \tt {\tiny [A$\langle$GM]}  
	 Deine Nachricht wurde an deinen Mitspieler übermittelt.\\ \tt \\ \tt \\ \tt Ihr seid gerade bei Runde 18 von maximal 28.\\ \tt Ihr habt gemeinsam 6 von maximal 16 Strafpunkten angesammelt.\\ \tt Dein Mitspieler hat dir diese Nachricht geschickt:\\ \tt \\ \tt \\ \tt "Ich denke, das ist ein guter Ansatz. Lass uns die Objekte überprüfen, die am wenigsten bewegt werden müssen. Mein Objekt P ist bereits an der Zielposition (1,1). Mein Objekt T ist nicht an der Zielposition, aber es ist auch nicht nötig, es zu bewegen, da es bei dir bereits an der Zielposition (7,7) ist. Mein Objekt O muss nach (4,4) bewegt werden, aber wie ich bereits erwähnt habe, ist das '┴' im Weg. Wir sollten perhaps einen Plan erstellen, um das '┴' zu umgehen oder zu entfernen. Was denkst du? Sollten wir versuchen, mein Objekt O nach (4,4) zu bewegen oder ein anderes Objekt bewegen, um Platz zu schaffen?"\\ \tt \\ \tt \\ \tt Wie lautet dein nächster Befehl? 
	  } 
	   } 
	   } 
	 & \\ 
 

    \theutterance \stepcounter{utterance}  

    & \multicolumn{4}{p{0.6\linewidth}}{\cellcolor[rgb]{0.8,1,0.9}{%
	\makecell[{{p{\linewidth}}}]{% 
	  \tt {\tiny [A$\rangle$GM]}  
	 SPRICH: Ich denke, wir sollten versuchen, dein Objekt O nach (4,4) zu bewegen, indem wir das '┴' umgehen. Da mein Objekt O bereits in (4,5) ist, denke ich, dass ich mein Objekt O nicht bewegen muss, um Platz zu schaffen. Ich frage mich, ob du dein Objekt O nach (4,4) bewegen kannst, wenn du dein Objekt S bewegst. Kannst du perhaps dein Objekt S bewegen, um Platz für dein Objekt O zu schaffen? 
	  } 
	   } 
	   } 
	 & & \\ 
 

    \theutterance \stepcounter{utterance}  

    & & & \multicolumn{2}{p{0.3\linewidth}}{\cellcolor[rgb]{0.95,0.95,0.95}{%
	\makecell[{{p{\linewidth}}}]{% 
	  \tt {\tiny [GM$|$GM]}  
	 SPRICH: Ich denke, wir sollten versuchen, dein Objekt O nach (4,4) zu bewegen, indem wir das '┴' umgehen. Da mein Objekt O bereits in (4,5) ist, denke ich, dass ich mein Objekt O nicht bewegen muss, um Platz zu schaffen. Ich frage mich, ob du dein Objekt O nach (4,4) bewegen kannst, wenn du dein Objekt S bewegst. Kannst du perhaps dein Objekt S bewegen, um Platz für dein Objekt O zu schaffen? 
	  } 
	   } 
	   } 
	 & & \\ 
 

    \theutterance \stepcounter{utterance}  

    & & \multicolumn{4}{p{0.6\linewidth}}{\cellcolor[rgb]{0.9,0.9,0.9}{%
	\makecell[{{p{\linewidth}}}]{% 
	  \tt {\tiny [GM$\rangle$B]}  
	 Deine Nachricht wurde an deinen Mitspieler übermittelt.\\ \tt \\ \tt \\ \tt Ihr seid gerade bei Runde 19 von maximal 28.\\ \tt Ihr habt gemeinsam 6 von maximal 16 Strafpunkten angesammelt.\\ \tt Dein Mitspieler hat dir diese Nachricht geschickt:\\ \tt \\ \tt \\ \tt "Ich denke, wir sollten versuchen, dein Objekt O nach (4,4) zu bewegen, indem wir das '┴' umgehen. Da mein Objekt O bereits in (4,5) ist, denke ich, dass ich mein Objekt O nicht bewegen muss, um Platz zu schaffen. Ich frage mich, ob du dein Objekt O nach (4,4) bewegen kannst, wenn du dein Objekt S bewegst. Kannst du perhaps dein Objekt S bewegen, um Platz für dein Objekt O zu schaffen?"\\ \tt \\ \tt \\ \tt Wie lautet dein nächster Befehl? 
	  } 
	   } 
	   } 
	 & \\ 
 

    \theutterance \stepcounter{utterance}  

    & & & \multicolumn{4}{p{0.6\linewidth}}{\cellcolor[rgb]{1,0.85,0.72}{%
	\makecell[{{p{\linewidth}}}]{% 
	  \tt {\tiny [GM$\langle$B]}  
	 SPRICH: Mein Objekt S ist in (5,6), und ich denke, ich könnte es perhaps nach (1,7) bewegen, um Platz für mein Objekt O zu schaffen. Aber bevor ich das tue, möchte ich sicherstellen, dass das mein Objekt O tatsächlich frei bewegen kann. Ich werde versuchen, mein Objekt S nach (1,7) zu bewegen und dann mein Objekt O nach (4,4) zu bewegen. Was meinst du? Soll ich mein Objekt S bewegen und dann mein Objekt O versuchen zu bewegen? 
	  } 
	   } 
	   } 
	  \\ 
 

    \theutterance \stepcounter{utterance}  

    & & & \multicolumn{2}{p{0.3\linewidth}}{\cellcolor[rgb]{0.95,0.95,0.95}{%
	\makecell[{{p{\linewidth}}}]{% 
	  \tt {\tiny [GM$|$GM]}  
	 SPRICH: Mein Objekt S ist in (5,6), und ich denke, ich könnte es perhaps nach (1,7) bewegen, um Platz für mein Objekt O zu schaffen. Aber bevor ich das tue, möchte ich sicherstellen, dass das mein Objekt O tatsächlich frei bewegen kann. Ich werde versuchen, mein Objekt S nach (1,7) zu bewegen und dann mein Objekt O nach (4,4) zu bewegen. Was meinst du? Soll ich mein Objekt S bewegen und dann mein Objekt O versuchen zu bewegen? 
	  } 
	   } 
	   } 
	 & & \\ 
 

    \theutterance \stepcounter{utterance}  

    & & \multicolumn{4}{p{0.6\linewidth}}{\cellcolor[rgb]{0.9,0.9,0.9}{%
	\makecell[{{p{\linewidth}}}]{% 
	  \tt {\tiny [A$\langle$GM]}  
	 Deine Nachricht wurde an deinen Mitspieler übermittelt.\\ \tt \\ \tt \\ \tt Ihr seid gerade bei Runde 19 von maximal 28.\\ \tt Ihr habt gemeinsam 6 von maximal 16 Strafpunkten angesammelt.\\ \tt Dein Mitspieler hat dir diese Nachricht geschickt:\\ \tt \\ \tt \\ \tt "Mein Objekt S ist in (5,6), und ich denke, ich könnte es perhaps nach (1,7) bewegen, um Platz für mein Objekt O zu schaffen. Aber bevor ich das tue, möchte ich sicherstellen, dass das mein Objekt O tatsächlich frei bewegen kann. Ich werde versuchen, mein Objekt S nach (1,7) zu bewegen und dann mein Objekt O nach (4,4) zu bewegen. Was meinst du? Soll ich mein Objekt S bewegen und dann mein Objekt O versuchen zu bewegen?"\\ \tt \\ \tt \\ \tt Wie lautet dein nächster Befehl? 
	  } 
	   } 
	   } 
	 & \\ 
 

    \theutterance \stepcounter{utterance}  

    & \multicolumn{4}{p{0.6\linewidth}}{\cellcolor[rgb]{0.8,1,0.9}{%
	\makecell[{{p{\linewidth}}}]{% 
	  \tt {\tiny [A$\rangle$GM]}  
	 SPRICH: Ja, das ist ein guter Plan! Ich denke, es ist eine gute Idee, dein Objekt S nach (1,7) zu bewegen, um Platz für dein Objekt O zu schaffen. Ich bin gespannt, ob dein Objekt O dann nach (4,4) bewegt werden kann. Bitte versuche, dein Objekt S nach (1,7) zu bewegen und dann dein Objekt O nach (4,4) zu bewegen. Ich werde abwarten und sehen, wie es klappt. 
	  } 
	   } 
	   } 
	 & & \\ 
 

    \theutterance \stepcounter{utterance}  

    & & & \multicolumn{2}{p{0.3\linewidth}}{\cellcolor[rgb]{0.95,0.95,0.95}{%
	\makecell[{{p{\linewidth}}}]{% 
	  \tt {\tiny [GM$|$GM]}  
	 SPRICH: Ja, das ist ein guter Plan! Ich denke, es ist eine gute Idee, dein Objekt S nach (1,7) zu bewegen, um Platz für dein Objekt O zu schaffen. Ich bin gespannt, ob dein Objekt O dann nach (4,4) bewegt werden kann. Bitte versuche, dein Objekt S nach (1,7) zu bewegen und dann dein Objekt O nach (4,4) zu bewegen. Ich werde abwarten und sehen, wie es klappt. 
	  } 
	   } 
	   } 
	 & & \\ 
 

    \theutterance \stepcounter{utterance}  

    & & \multicolumn{4}{p{0.6\linewidth}}{\cellcolor[rgb]{0.9,0.9,0.9}{%
	\makecell[{{p{\linewidth}}}]{% 
	  \tt {\tiny [GM$\rangle$B]}  
	 Deine Nachricht wurde an deinen Mitspieler übermittelt.\\ \tt \\ \tt \\ \tt Ihr seid gerade bei Runde 20 von maximal 28.\\ \tt Ihr habt gemeinsam 6 von maximal 16 Strafpunkten angesammelt.\\ \tt Dein Mitspieler hat dir diese Nachricht geschickt:\\ \tt \\ \tt \\ \tt "Ja, das ist ein guter Plan! Ich denke, es ist eine gute Idee, dein Objekt S nach (1,7) zu bewegen, um Platz für dein Objekt O zu schaffen. Ich bin gespannt, ob dein Objekt O dann nach (4,4) bewegt werden kann. Bitte versuche, dein Objekt S nach (1,7) zu bewegen und dann dein Objekt O nach (4,4) zu bewegen. Ich werde abwarten und sehen, wie es klappt."\\ \tt \\ \tt \\ \tt Wie lautet dein nächster Befehl? 
	  } 
	   } 
	   } 
	 & \\ 
 

    \theutterance \stepcounter{utterance}  

    & & & \multicolumn{4}{p{0.6\linewidth}}{\cellcolor[rgb]{1,0.85,0.72}{%
	\makecell[{{p{\linewidth}}}]{% 
	  \tt {\tiny [GM$\langle$B]}  
	 VERSCHIEBE: S, (1,7) 
	  } 
	   } 
	   } 
	  \\ 
 

    \theutterance \stepcounter{utterance}  

    & & & \multicolumn{2}{p{0.3\linewidth}}{\cellcolor[rgb]{0.95,0.95,0.95}{%
	\makecell[{{p{\linewidth}}}]{% 
	  \tt {\tiny [GM$|$GM]}  
	 VERSCHIEBE: S, (1,7) 
	  } 
	   } 
	   } 
	 & & \\ 
 

    \theutterance \stepcounter{utterance}  

    & & & \multicolumn{2}{p{0.3\linewidth}}{\cellcolor[rgb]{0.95,0.95,0.95}{%
	\makecell[{{p{\linewidth}}}]{% 
	  \tt {\tiny [GM$|$GM]}  
	 Strafpunkt: (1,7) ist nicht leer, sondern enthält 'A'.\\ \tt \\ \tt Ihr habt gemeinsam 7 von maximal 16 Strafpunkten angesammelt.\\ \tt Bitte versuche es noch mal! 
	  } 
	   } 
	   } 
	 & & \\ 
 

    \theutterance \stepcounter{utterance}  

    & & \multicolumn{4}{p{0.6\linewidth}}{\cellcolor[rgb]{0.9,0.9,0.9}{%
	\makecell[{{p{\linewidth}}}]{% 
	  \tt {\tiny [GM$\rangle$B]}  
	 Strafpunkt: (1,7) ist nicht leer, sondern enthält 'A'.\\ \tt \\ \tt Ihr habt gemeinsam 7 von maximal 16 Strafpunkten angesammelt.\\ \tt Bitte versuche es noch mal! 
	  } 
	   } 
	   } 
	 & \\ 
 

    \theutterance \stepcounter{utterance}  

    & & & \multicolumn{4}{p{0.6\linewidth}}{\cellcolor[rgb]{1,0.85,0.72}{%
	\makecell[{{p{\linewidth}}}]{% 
	  \tt {\tiny [GM$\langle$B]}  
	 VERSCHIEBE: S, (6,7) 
	  } 
	   } 
	   } 
	  \\ 
 

    \theutterance \stepcounter{utterance}  

    & & & \multicolumn{2}{p{0.3\linewidth}}{\cellcolor[rgb]{0.95,0.95,0.95}{%
	\makecell[{{p{\linewidth}}}]{% 
	  \tt {\tiny [GM$|$GM]}  
	 VERSCHIEBE: S, (6,7) 
	  } 
	   } 
	   } 
	 & & \\ 
 

    \theutterance \stepcounter{utterance}  

    & & & \multicolumn{2}{p{0.3\linewidth}}{\cellcolor[rgb]{0.95,0.95,0.95}{%
	\makecell[{{p{\linewidth}}}]{% 
	  \tt {\tiny [GM$|$GM]}  
	 'S' erfolgreich nach (6,7) verschoben. Dein Raster sieht jetzt so aus:\\ \tt ```\\ \tt  1234567\\ \tt ╔═╤═╤═╤═╗\\ \tt ║◌│◌│◌│P║ 1\\ \tt ║T└─┼─┴─╢ 2\\ \tt ║◌◌M│◌◌◌║ 3\\ \tt ╟──┬┴───╢ 4\\ \tt ║◌◌│O◌◌◌║ 5\\ \tt ║◌D│◌◌◌◌║ 6\\ \tt ║A◌│◌◌S◌║ 7\\ \tt ╚══╧════╝\\ \tt ```\\ \tt  
	  } 
	   } 
	   } 
	 & & \\ 
 

    \theutterance \stepcounter{utterance}  

    & & \multicolumn{4}{p{0.6\linewidth}}{\cellcolor[rgb]{0.9,0.9,0.9}{%
	\makecell[{{p{\linewidth}}}]{% 
	  \tt {\tiny [A$\langle$GM]}  
	 Deine Nachricht wurde an deinen Mitspieler übermittelt.\\ \tt \\ \tt \\ \tt Ihr seid gerade bei Runde 20 von maximal 28.\\ \tt Ihr habt gemeinsam 7 von maximal 16 Strafpunkten angesammelt.\\ \tt Dein Mitspieler hat ein Objekt auf seinem Raster bewegt.\\ \tt \\ \tt \\ \tt Wie lautet dein nächster Befehl? 
	  } 
	   } 
	   } 
	 & \\ 
 

    \theutterance \stepcounter{utterance}  

    & \multicolumn{4}{p{0.6\linewidth}}{\cellcolor[rgb]{0.8,1,0.9}{%
	\makecell[{{p{\linewidth}}}]{% 
	  \tt {\tiny [A$\rangle$GM]}  
	 SPRICH: Wie sieht es jetzt bei dir aus? Hast du dein Objekt S erfolgreich nach (1,7) bewegt und kannst du jetzt dein Objekt O nach (4,4) bewegen? 
	  } 
	   } 
	   } 
	 & & \\ 
 

    \theutterance \stepcounter{utterance}  

    & & & \multicolumn{2}{p{0.3\linewidth}}{\cellcolor[rgb]{0.95,0.95,0.95}{%
	\makecell[{{p{\linewidth}}}]{% 
	  \tt {\tiny [GM$|$GM]}  
	 SPRICH: Wie sieht es jetzt bei dir aus? Hast du dein Objekt S erfolgreich nach (1,7) bewegt und kannst du jetzt dein Objekt O nach (4,4) bewegen? 
	  } 
	   } 
	   } 
	 & & \\ 
 

    \theutterance \stepcounter{utterance}  

    & & \multicolumn{4}{p{0.6\linewidth}}{\cellcolor[rgb]{0.9,0.9,0.9}{%
	\makecell[{{p{\linewidth}}}]{% 
	  \tt {\tiny [GM$\rangle$B]}  
	 'S' erfolgreich nach (6,7) verschoben. Dein Raster sieht jetzt so aus:\\ \tt ```\\ \tt  1234567\\ \tt ╔═╤═╤═╤═╗\\ \tt ║◌│◌│◌│P║ 1\\ \tt ║T└─┼─┴─╢ 2\\ \tt ║◌◌M│◌◌◌║ 3\\ \tt ╟──┬┴───╢ 4\\ \tt ║◌◌│O◌◌◌║ 5\\ \tt ║◌D│◌◌◌◌║ 6\\ \tt ║A◌│◌◌S◌║ 7\\ \tt ╚══╧════╝\\ \tt ```\\ \tt Ihr seid gerade bei Runde 21 von maximal 28.\\ \tt Ihr habt gemeinsam 7 von maximal 16 Strafpunkten angesammelt.\\ \tt Dein Mitspieler hat dir diese Nachricht geschickt:\\ \tt \\ \tt \\ \tt "Wie sieht es jetzt bei dir aus? Hast du dein Objekt S erfolgreich nach (1,7) bewegt und kannst du jetzt dein Objekt O nach (4,4) bewegen?"\\ \tt \\ \tt \\ \tt Wie lautet dein nächster Befehl? 
	  } 
	   } 
	   } 
	 & \\ 
 

    \theutterance \stepcounter{utterance}  

    & & & \multicolumn{4}{p{0.6\linewidth}}{\cellcolor[rgb]{1,0.85,0.72}{%
	\makecell[{{p{\linewidth}}}]{% 
	  \tt {\tiny [GM$\langle$B]}  
	 SPRICH: Nein, ich habe mein Objekt S leider nicht nach (1,7) bewegt, weil (1,7) bereits von 'A' besetzt war. Stattdessen habe ich es nach (6,7) bewegt. Mein Objekt O ist immer noch in (4,5) und kann nicht nach (4,4) bewegt werden, weil das '┴' im Weg ist. Ich denke, wir sollten einen neuen Plan entwickeln, um mein Objekt O nach (4,4) zu bewegen. Was meinst du? Sollten wir perhaps ein anderes Objekt bewegen, um Platz zu schaffen? 
	  } 
	   } 
	   } 
	  \\ 
 

    \theutterance \stepcounter{utterance}  

    & & & \multicolumn{2}{p{0.3\linewidth}}{\cellcolor[rgb]{0.95,0.95,0.95}{%
	\makecell[{{p{\linewidth}}}]{% 
	  \tt {\tiny [GM$|$GM]}  
	 SPRICH: Nein, ich habe mein Objekt S leider nicht nach (1,7) bewegt, weil (1,7) bereits von 'A' besetzt war. Stattdessen habe ich es nach (6,7) bewegt. Mein Objekt O ist immer noch in (4,5) und kann nicht nach (4,4) bewegt werden, weil das '┴' im Weg ist. Ich denke, wir sollten einen neuen Plan entwickeln, um mein Objekt O nach (4,4) zu bewegen. Was meinst du? Sollten wir perhaps ein anderes Objekt bewegen, um Platz zu schaffen? 
	  } 
	   } 
	   } 
	 & & \\ 
 

    \theutterance \stepcounter{utterance}  

    & & \multicolumn{4}{p{0.6\linewidth}}{\cellcolor[rgb]{0.9,0.9,0.9}{%
	\makecell[{{p{\linewidth}}}]{% 
	  \tt {\tiny [A$\langle$GM]}  
	 Deine Nachricht wurde an deinen Mitspieler übermittelt.\\ \tt \\ \tt \\ \tt Ihr seid gerade bei Runde 21 von maximal 28.\\ \tt Ihr habt gemeinsam 7 von maximal 16 Strafpunkten angesammelt.\\ \tt Dein Mitspieler hat dir diese Nachricht geschickt:\\ \tt \\ \tt \\ \tt "Nein, ich habe mein Objekt S leider nicht nach (1,7) bewegt, weil (1,7) bereits von 'A' besetzt war. Stattdessen habe ich es nach (6,7) bewegt. Mein Objekt O ist immer noch in (4,5) und kann nicht nach (4,4) bewegt werden, weil das '┴' im Weg ist. Ich denke, wir sollten einen neuen Plan entwickeln, um mein Objekt O nach (4,4) zu bewegen. Was meinst du? Sollten wir perhaps ein anderes Objekt bewegen, um Platz zu schaffen?"\\ \tt \\ \tt \\ \tt Wie lautet dein nächster Befehl? 
	  } 
	   } 
	   } 
	 & \\ 
 

    \theutterance \stepcounter{utterance}  

    & \multicolumn{4}{p{0.6\linewidth}}{\cellcolor[rgb]{0.8,1,0.9}{%
	\makecell[{{p{\linewidth}}}]{% 
	  \tt {\tiny [A$\rangle$GM]}  
	 SPRICH: Ich denke, wir sollten nochmal unser Raster und die Objekte überprüfen, um einen neuen Plan zu erstellen. Vielleicht gibt es eine Möglichkeit, ein anderes Objekt zu bewegen, um Platz für dein Objekt O zu schaffen. Ich frage mich, ob du perhaps dein Objekt A bewegen kannst, um Platz zu schaffen. Oder vielleicht gibt es eine andere Möglichkeit, das '┴' zu umgehen. Lass uns nochmal unsere Optionen überprüfen. 
	  } 
	   } 
	   } 
	 & & \\ 
 

    \theutterance \stepcounter{utterance}  

    & & & \multicolumn{2}{p{0.3\linewidth}}{\cellcolor[rgb]{0.95,0.95,0.95}{%
	\makecell[{{p{\linewidth}}}]{% 
	  \tt {\tiny [GM$|$GM]}  
	 SPRICH: Ich denke, wir sollten nochmal unser Raster und die Objekte überprüfen, um einen neuen Plan zu erstellen. Vielleicht gibt es eine Möglichkeit, ein anderes Objekt zu bewegen, um Platz für dein Objekt O zu schaffen. Ich frage mich, ob du perhaps dein Objekt A bewegen kannst, um Platz zu schaffen. Oder vielleicht gibt es eine andere Möglichkeit, das '┴' zu umgehen. Lass uns nochmal unsere Optionen überprüfen. 
	  } 
	   } 
	   } 
	 & & \\ 
 

    \theutterance \stepcounter{utterance}  

    & & \multicolumn{4}{p{0.6\linewidth}}{\cellcolor[rgb]{0.9,0.9,0.9}{%
	\makecell[{{p{\linewidth}}}]{% 
	  \tt {\tiny [GM$\rangle$B]}  
	 Deine Nachricht wurde an deinen Mitspieler übermittelt.\\ \tt \\ \tt \\ \tt Ihr seid gerade bei Runde 22 von maximal 28.\\ \tt Ihr habt gemeinsam 7 von maximal 16 Strafpunkten angesammelt.\\ \tt Dein Mitspieler hat dir diese Nachricht geschickt:\\ \tt \\ \tt \\ \tt "Ich denke, wir sollten nochmal unser Raster und die Objekte überprüfen, um einen neuen Plan zu erstellen. Vielleicht gibt es eine Möglichkeit, ein anderes Objekt zu bewegen, um Platz für dein Objekt O zu schaffen. Ich frage mich, ob du perhaps dein Objekt A bewegen kannst, um Platz zu schaffen. Oder vielleicht gibt es eine andere Möglichkeit, das '┴' zu umgehen. Lass uns nochmal unsere Optionen überprüfen."\\ \tt \\ \tt \\ \tt Wie lautet dein nächster Befehl? 
	  } 
	   } 
	   } 
	 & \\ 
 

    \theutterance \stepcounter{utterance}  

    & & & \multicolumn{4}{p{0.6\linewidth}}{\cellcolor[rgb]{1,0.85,0.72}{%
	\makecell[{{p{\linewidth}}}]{% 
	  \tt {\tiny [GM$\langle$B]}  
	 SPRICH: Ich denke, das ist eine gute Idee. Ich werde mein Raster nochmal überprüfen. Mein Objekt A ist in (1,7) und mein Objekt O ist in (4,5). Ich frage mich, ob ich mein Objekt A bewegen kann, um Platz zu schaffen. Wenn ich mein Objekt A nach (1,6) oder einer anderen freien Position bewegen könnte, könnte ich vielleicht mein Objekt O nach (4,4) bewegen. Was meinst du? Sollte ich versuchen, mein Objekt A zu bewegen, um Platz zu schaffen? 
	  } 
	   } 
	   } 
	  \\ 
 

    \theutterance \stepcounter{utterance}  

    & & & \multicolumn{2}{p{0.3\linewidth}}{\cellcolor[rgb]{0.95,0.95,0.95}{%
	\makecell[{{p{\linewidth}}}]{% 
	  \tt {\tiny [GM$|$GM]}  
	 SPRICH: Ich denke, das ist eine gute Idee. Ich werde mein Raster nochmal überprüfen. Mein Objekt A ist in (1,7) und mein Objekt O ist in (4,5). Ich frage mich, ob ich mein Objekt A bewegen kann, um Platz zu schaffen. Wenn ich mein Objekt A nach (1,6) oder einer anderen freien Position bewegen könnte, könnte ich vielleicht mein Objekt O nach (4,4) bewegen. Was meinst du? Sollte ich versuchen, mein Objekt A zu bewegen, um Platz zu schaffen? 
	  } 
	   } 
	   } 
	 & & \\ 
 

    \theutterance \stepcounter{utterance}  

    & & \multicolumn{4}{p{0.6\linewidth}}{\cellcolor[rgb]{0.9,0.9,0.9}{%
	\makecell[{{p{\linewidth}}}]{% 
	  \tt {\tiny [A$\langle$GM]}  
	 Deine Nachricht wurde an deinen Mitspieler übermittelt.\\ \tt \\ \tt \\ \tt Ihr seid gerade bei Runde 22 von maximal 28.\\ \tt Ihr habt gemeinsam 7 von maximal 16 Strafpunkten angesammelt.\\ \tt Dein Mitspieler hat dir diese Nachricht geschickt:\\ \tt \\ \tt \\ \tt "Ich denke, das ist eine gute Idee. Ich werde mein Raster nochmal überprüfen. Mein Objekt A ist in (1,7) und mein Objekt O ist in (4,5). Ich frage mich, ob ich mein Objekt A bewegen kann, um Platz zu schaffen. Wenn ich mein Objekt A nach (1,6) oder einer anderen freien Position bewegen könnte, könnte ich vielleicht mein Objekt O nach (4,4) bewegen. Was meinst du? Sollte ich versuchen, mein Objekt A zu bewegen, um Platz zu schaffen?"\\ \tt \\ \tt \\ \tt Wie lautet dein nächster Befehl? 
	  } 
	   } 
	   } 
	 & \\ 
 

    \theutterance \stepcounter{utterance}  

    & \multicolumn{4}{p{0.6\linewidth}}{\cellcolor[rgb]{0.8,1,0.9}{%
	\makecell[{{p{\linewidth}}}]{% 
	  \tt {\tiny [A$\rangle$GM]}  
	 SPRICH: Ja, das ist eine gute Idee! Ich denke, es wäre eine gute Option, wenn du dein Objekt A bewegen könntest, um Platz zu schaffen. Wenn du dein Objekt A nach (1,6) oder einer anderen freien Position bewegen kannst, könnte das perhaps die Möglichkeit eröffnen, dein Objekt O nach (4,4) zu bewegen. Bitte versuche, dein Objekt A zu bewegen und sieh, ob das funktioniert. 
	  } 
	   } 
	   } 
	 & & \\ 
 

    \theutterance \stepcounter{utterance}  

    & & & \multicolumn{2}{p{0.3\linewidth}}{\cellcolor[rgb]{0.95,0.95,0.95}{%
	\makecell[{{p{\linewidth}}}]{% 
	  \tt {\tiny [GM$|$GM]}  
	 SPRICH: Ja, das ist eine gute Idee! Ich denke, es wäre eine gute Option, wenn du dein Objekt A bewegen könntest, um Platz zu schaffen. Wenn du dein Objekt A nach (1,6) oder einer anderen freien Position bewegen kannst, könnte das perhaps die Möglichkeit eröffnen, dein Objekt O nach (4,4) zu bewegen. Bitte versuche, dein Objekt A zu bewegen und sieh, ob das funktioniert. 
	  } 
	   } 
	   } 
	 & & \\ 
 

    \theutterance \stepcounter{utterance}  

    & & \multicolumn{4}{p{0.6\linewidth}}{\cellcolor[rgb]{0.9,0.9,0.9}{%
	\makecell[{{p{\linewidth}}}]{% 
	  \tt {\tiny [GM$\rangle$B]}  
	 Deine Nachricht wurde an deinen Mitspieler übermittelt.\\ \tt \\ \tt \\ \tt Ihr seid gerade bei Runde 23 von maximal 28.\\ \tt Ihr habt gemeinsam 7 von maximal 16 Strafpunkten angesammelt.\\ \tt Dein Mitspieler hat dir diese Nachricht geschickt:\\ \tt \\ \tt \\ \tt "Ja, das ist eine gute Idee! Ich denke, es wäre eine gute Option, wenn du dein Objekt A bewegen könntest, um Platz zu schaffen. Wenn du dein Objekt A nach (1,6) oder einer anderen freien Position bewegen kannst, könnte das perhaps die Möglichkeit eröffnen, dein Objekt O nach (4,4) zu bewegen. Bitte versuche, dein Objekt A zu bewegen und sieh, ob das funktioniert."\\ \tt \\ \tt \\ \tt Wie lautet dein nächster Befehl? 
	  } 
	   } 
	   } 
	 & \\ 
 

    \theutterance \stepcounter{utterance}  

    & & & \multicolumn{4}{p{0.6\linewidth}}{\cellcolor[rgb]{1,0.85,0.72}{%
	\makecell[{{p{\linewidth}}}]{% 
	  \tt {\tiny [GM$\langle$B]}  
	 VERSCHIEBE: A, (1,6) 
	  } 
	   } 
	   } 
	  \\ 
 

    \theutterance \stepcounter{utterance}  

    & & & \multicolumn{2}{p{0.3\linewidth}}{\cellcolor[rgb]{0.95,0.95,0.95}{%
	\makecell[{{p{\linewidth}}}]{% 
	  \tt {\tiny [GM$|$GM]}  
	 VERSCHIEBE: A, (1,6) 
	  } 
	   } 
	   } 
	 & & \\ 
 

    \theutterance \stepcounter{utterance}  

    & & & \multicolumn{2}{p{0.3\linewidth}}{\cellcolor[rgb]{0.95,0.95,0.95}{%
	\makecell[{{p{\linewidth}}}]{% 
	  \tt {\tiny [GM$|$GM]}  
	 'A' erfolgreich nach (1,6) verschoben. Dein Raster sieht jetzt so aus:\\ \tt ```\\ \tt  1234567\\ \tt ╔═╤═╤═╤═╗\\ \tt ║◌│◌│◌│P║ 1\\ \tt ║T└─┼─┴─╢ 2\\ \tt ║◌◌M│◌◌◌║ 3\\ \tt ╟──┬┴───╢ 4\\ \tt ║◌◌│O◌◌◌║ 5\\ \tt ║AD│◌◌◌◌║ 6\\ \tt ║◌◌│◌◌S◌║ 7\\ \tt ╚══╧════╝\\ \tt ```\\ \tt  
	  } 
	   } 
	   } 
	 & & \\ 
 

    \theutterance \stepcounter{utterance}  

    & & \multicolumn{4}{p{0.6\linewidth}}{\cellcolor[rgb]{0.9,0.9,0.9}{%
	\makecell[{{p{\linewidth}}}]{% 
	  \tt {\tiny [A$\langle$GM]}  
	 Deine Nachricht wurde an deinen Mitspieler übermittelt.\\ \tt \\ \tt \\ \tt Ihr seid gerade bei Runde 23 von maximal 28.\\ \tt Ihr habt gemeinsam 7 von maximal 16 Strafpunkten angesammelt.\\ \tt Dein Mitspieler hat ein Objekt auf seinem Raster bewegt.\\ \tt \\ \tt \\ \tt Wie lautet dein nächster Befehl? 
	  } 
	   } 
	   } 
	 & \\ 
 

    \theutterance \stepcounter{utterance}  

    & \multicolumn{4}{p{0.6\linewidth}}{\cellcolor[rgb]{0.8,1,0.9}{%
	\makecell[{{p{\linewidth}}}]{% 
	  \tt {\tiny [A$\rangle$GM]}  
	 SPRICH: Wie sieht es jetzt bei dir aus? Hast du dein Objekt A erfolgreich bewegt und kannst du jetzt dein Objekt O nach (4,4) bewegen? 
	  } 
	   } 
	   } 
	 & & \\ 
 

    \theutterance \stepcounter{utterance}  

    & & & \multicolumn{2}{p{0.3\linewidth}}{\cellcolor[rgb]{0.95,0.95,0.95}{%
	\makecell[{{p{\linewidth}}}]{% 
	  \tt {\tiny [GM$|$GM]}  
	 SPRICH: Wie sieht es jetzt bei dir aus? Hast du dein Objekt A erfolgreich bewegt und kannst du jetzt dein Objekt O nach (4,4) bewegen? 
	  } 
	   } 
	   } 
	 & & \\ 
 

    \theutterance \stepcounter{utterance}  

    & & \multicolumn{4}{p{0.6\linewidth}}{\cellcolor[rgb]{0.9,0.9,0.9}{%
	\makecell[{{p{\linewidth}}}]{% 
	  \tt {\tiny [GM$\rangle$B]}  
	 'A' erfolgreich nach (1,6) verschoben. Dein Raster sieht jetzt so aus:\\ \tt ```\\ \tt  1234567\\ \tt ╔═╤═╤═╤═╗\\ \tt ║◌│◌│◌│P║ 1\\ \tt ║T└─┼─┴─╢ 2\\ \tt ║◌◌M│◌◌◌║ 3\\ \tt ╟──┬┴───╢ 4\\ \tt ║◌◌│O◌◌◌║ 5\\ \tt ║AD│◌◌◌◌║ 6\\ \tt ║◌◌│◌◌S◌║ 7\\ \tt ╚══╧════╝\\ \tt ```\\ \tt Ihr seid gerade bei Runde 24 von maximal 28.\\ \tt Ihr habt gemeinsam 7 von maximal 16 Strafpunkten angesammelt.\\ \tt Dein Mitspieler hat dir diese Nachricht geschickt:\\ \tt \\ \tt \\ \tt "Wie sieht es jetzt bei dir aus? Hast du dein Objekt A erfolgreich bewegt und kannst du jetzt dein Objekt O nach (4,4) bewegen?"\\ \tt \\ \tt \\ \tt Wie lautet dein nächster Befehl? 
	  } 
	   } 
	   } 
	 & \\ 
 

    \theutterance \stepcounter{utterance}  

    & & & \multicolumn{4}{p{0.6\linewidth}}{\cellcolor[rgb]{1,0.85,0.72}{%
	\makecell[{{p{\linewidth}}}]{% 
	  \tt {\tiny [GM$\langle$B]}  
	 SPRICH: Ja, ich habe mein Objekt A erfolgreich nach (1,6) bewegt. Leider hat sich dadurch nichts geändert, was das '┴' in (4,4) angeht. Mein Objekt O ist immer noch in (4,5) und kann nicht nach (4,4) bewegt werden, weil das '┴' im Weg ist. Ich denke, wir sollten nochmal überlegen, wie wir das '┴' umgehen oder entfernen können, um mein Objekt O nach (4,4) zu bewegen. Was meinst du? Sollten wir perhaps ein anderes Objekt bewegen oder einen anderen Plan entwickeln? 
	  } 
	   } 
	   } 
	  \\ 
 

    \theutterance \stepcounter{utterance}  

    & & & \multicolumn{2}{p{0.3\linewidth}}{\cellcolor[rgb]{0.95,0.95,0.95}{%
	\makecell[{{p{\linewidth}}}]{% 
	  \tt {\tiny [GM$|$GM]}  
	 SPRICH: Ja, ich habe mein Objekt A erfolgreich nach (1,6) bewegt. Leider hat sich dadurch nichts geändert, was das '┴' in (4,4) angeht. Mein Objekt O ist immer noch in (4,5) und kann nicht nach (4,4) bewegt werden, weil das '┴' im Weg ist. Ich denke, wir sollten nochmal überlegen, wie wir das '┴' umgehen oder entfernen können, um mein Objekt O nach (4,4) zu bewegen. Was meinst du? Sollten wir perhaps ein anderes Objekt bewegen oder einen anderen Plan entwickeln? 
	  } 
	   } 
	   } 
	 & & \\ 
 

    \theutterance \stepcounter{utterance}  

    & & \multicolumn{4}{p{0.6\linewidth}}{\cellcolor[rgb]{0.9,0.9,0.9}{%
	\makecell[{{p{\linewidth}}}]{% 
	  \tt {\tiny [A$\langle$GM]}  
	 Deine Nachricht wurde an deinen Mitspieler übermittelt.\\ \tt \\ \tt \\ \tt Ihr seid gerade bei Runde 24 von maximal 28.\\ \tt Ihr habt gemeinsam 7 von maximal 16 Strafpunkten angesammelt.\\ \tt Dein Mitspieler hat dir diese Nachricht geschickt:\\ \tt \\ \tt \\ \tt "Ja, ich habe mein Objekt A erfolgreich nach (1,6) bewegt. Leider hat sich dadurch nichts geändert, was das '┴' in (4,4) angeht. Mein Objekt O ist immer noch in (4,5) und kann nicht nach (4,4) bewegt werden, weil das '┴' im Weg ist. Ich denke, wir sollten nochmal überlegen, wie wir das '┴' umgehen oder entfernen können, um mein Objekt O nach (4,4) zu bewegen. Was meinst du? Sollten wir perhaps ein anderes Objekt bewegen oder einen anderen Plan entwickeln?"\\ \tt \\ \tt \\ \tt Wie lautet dein nächster Befehl? 
	  } 
	   } 
	   } 
	 & \\ 
 

    \theutterance \stepcounter{utterance}  

    & \multicolumn{4}{p{0.6\linewidth}}{\cellcolor[rgb]{0.8,1,0.9}{%
	\makecell[{{p{\linewidth}}}]{% 
	  \tt {\tiny [A$\rangle$GM]}  
	 SPRICH: Ich denke, wir sollten nochmal unser Raster und die Objekte überprüfen, um einen neuen Plan zu erstellen. Vielleicht gibt es eine Möglichkeit, ein anderes Objekt zu bewegen, um das '┴' zu umgehen oder zu entfernen. Ich frage mich, ob es eine Möglichkeit gibt, das '┴' selbst zu bewegen oder zu entfernen. Oder vielleicht gibt es eine andere Möglichkeit, mein Objekt O nach (4,4) zu bewegen, indem ich andere Objekte auf meinem Raster bewege. Lass uns nochmal unsere Optionen überprüfen. 
	  } 
	   } 
	   } 
	 & & \\ 
 

    \theutterance \stepcounter{utterance}  

    & & & \multicolumn{2}{p{0.3\linewidth}}{\cellcolor[rgb]{0.95,0.95,0.95}{%
	\makecell[{{p{\linewidth}}}]{% 
	  \tt {\tiny [GM$|$GM]}  
	 SPRICH: Ich denke, wir sollten nochmal unser Raster und die Objekte überprüfen, um einen neuen Plan zu erstellen. Vielleicht gibt es eine Möglichkeit, ein anderes Objekt zu bewegen, um das '┴' zu umgehen oder zu entfernen. Ich frage mich, ob es eine Möglichkeit gibt, das '┴' selbst zu bewegen oder zu entfernen. Oder vielleicht gibt es eine andere Möglichkeit, mein Objekt O nach (4,4) zu bewegen, indem ich andere Objekte auf meinem Raster bewege. Lass uns nochmal unsere Optionen überprüfen. 
	  } 
	   } 
	   } 
	 & & \\ 
 

    \theutterance \stepcounter{utterance}  

    & & \multicolumn{4}{p{0.6\linewidth}}{\cellcolor[rgb]{0.9,0.9,0.9}{%
	\makecell[{{p{\linewidth}}}]{% 
	  \tt {\tiny [GM$\rangle$B]}  
	 Deine Nachricht wurde an deinen Mitspieler übermittelt.\\ \tt \\ \tt \\ \tt Ihr seid gerade bei Runde 25 von maximal 28.\\ \tt Ihr habt gemeinsam 7 von maximal 16 Strafpunkten angesammelt.\\ \tt Dein Mitspieler hat dir diese Nachricht geschickt:\\ \tt \\ \tt \\ \tt "Ich denke, wir sollten nochmal unser Raster und die Objekte überprüfen, um einen neuen Plan zu erstellen. Vielleicht gibt es eine Möglichkeit, ein anderes Objekt zu bewegen, um das '┴' zu umgehen oder zu entfernen. Ich frage mich, ob es eine Möglichkeit gibt, das '┴' selbst zu bewegen oder zu entfernen. Oder vielleicht gibt es eine andere Möglichkeit, mein Objekt O nach (4,4) zu bewegen, indem ich andere Objekte auf meinem Raster bewege. Lass uns nochmal unsere Optionen überprüfen."\\ \tt \\ \tt \\ \tt Wie lautet dein nächster Befehl? 
	  } 
	   } 
	   } 
	 & \\ 
 

    \theutterance \stepcounter{utterance}  

    & & & \multicolumn{4}{p{0.6\linewidth}}{\cellcolor[rgb]{1,0.85,0.72}{%
	\makecell[{{p{\linewidth}}}]{% 
	  \tt {\tiny [GM$\langle$B]}  
	 SPRICH: Ich denke, das ist eine gute Idee. Ich werde mein Raster nochmal überprüfen. Ich frage mich, ob es eine Möglichkeit gibt, das '┴' selbst zu bewegen oder zu entfernen. Leider denke ich, dass das '┴' ein fester Bestandteil des Rasters ist und nicht bewegt oder entfernt werden kann. Ich denke, wir sollten uns auf die Bewegung der Objekte konzentrieren, um mein Objekt O nach (4,4) zu bewegen. Ich werde nochmal meine Objekte überprüfen und sehen, ob ich perhaps ein anderes Objekt bewegen kann, um Platz zu schaffen. Was meinst du? Sollten wir perhaps mein Objekt T oder mein Objekt M bewegen, um Platz zu schaffen? 
	  } 
	   } 
	   } 
	  \\ 
 

    \theutterance \stepcounter{utterance}  

    & & & \multicolumn{2}{p{0.3\linewidth}}{\cellcolor[rgb]{0.95,0.95,0.95}{%
	\makecell[{{p{\linewidth}}}]{% 
	  \tt {\tiny [GM$|$GM]}  
	 SPRICH: Ich denke, das ist eine gute Idee. Ich werde mein Raster nochmal überprüfen. Ich frage mich, ob es eine Möglichkeit gibt, das '┴' selbst zu bewegen oder zu entfernen. Leider denke ich, dass das '┴' ein fester Bestandteil des Rasters ist und nicht bewegt oder entfernt werden kann. Ich denke, wir sollten uns auf die Bewegung der Objekte konzentrieren, um mein Objekt O nach (4,4) zu bewegen. Ich werde nochmal meine Objekte überprüfen und sehen, ob ich perhaps ein anderes Objekt bewegen kann, um Platz zu schaffen. Was meinst du? Sollten wir perhaps mein Objekt T oder mein Objekt M bewegen, um Platz zu schaffen? 
	  } 
	   } 
	   } 
	 & & \\ 
 

    \theutterance \stepcounter{utterance}  

    & & \multicolumn{4}{p{0.6\linewidth}}{\cellcolor[rgb]{0.9,0.9,0.9}{%
	\makecell[{{p{\linewidth}}}]{% 
	  \tt {\tiny [A$\langle$GM]}  
	 Deine Nachricht wurde an deinen Mitspieler übermittelt.\\ \tt \\ \tt \\ \tt Ihr seid gerade bei Runde 25 von maximal 28.\\ \tt Ihr habt gemeinsam 7 von maximal 16 Strafpunkten angesammelt.\\ \tt Dein Mitspieler hat dir diese Nachricht geschickt:\\ \tt \\ \tt \\ \tt "Ich denke, das ist eine gute Idee. Ich werde mein Raster nochmal überprüfen. Ich frage mich, ob es eine Möglichkeit gibt, das '┴' selbst zu bewegen oder zu entfernen. Leider denke ich, dass das '┴' ein fester Bestandteil des Rasters ist und nicht bewegt oder entfernt werden kann. Ich denke, wir sollten uns auf die Bewegung der Objekte konzentrieren, um mein Objekt O nach (4,4) zu bewegen. Ich werde nochmal meine Objekte überprüfen und sehen, ob ich perhaps ein anderes Objekt bewegen kann, um Platz zu schaffen. Was meinst du? Sollten wir perhaps mein Objekt T oder mein Objekt M bewegen, um Platz zu schaffen?"\\ \tt \\ \tt \\ \tt Wie lautet dein nächster Befehl? 
	  } 
	   } 
	   } 
	 & \\ 
 

    \theutterance \stepcounter{utterance}  

    & \multicolumn{4}{p{0.6\linewidth}}{\cellcolor[rgb]{0.8,1,0.9}{%
	\makecell[{{p{\linewidth}}}]{% 
	  \tt {\tiny [A$\rangle$GM]}  
	 SPRICH: Ich denke, das ist eine gute Idee! Wenn das '┴' nicht bewegt oder entfernt werden kann, sollten wir uns auf die Bewegung der Objekte konzentrieren. Ich denke, es wäre eine gute Option, wenn du mein Objekt T oder mein Objekt M bewegen könntest, um Platz zu schaffen. Aber ich denke, wir sollten auch meine Objekte überprüfen, um zu sehen, ob ich perhaps ein Objekt bewegen kann, um Platz zu schaffen. Lass uns nochmal unsere Objekte und ihre Positionen überprüfen, um einen neuen Plan zu erstellen. 
	  } 
	   } 
	   } 
	 & & \\ 
 

    \theutterance \stepcounter{utterance}  

    & & & \multicolumn{2}{p{0.3\linewidth}}{\cellcolor[rgb]{0.95,0.95,0.95}{%
	\makecell[{{p{\linewidth}}}]{% 
	  \tt {\tiny [GM$|$GM]}  
	 SPRICH: Ich denke, das ist eine gute Idee! Wenn das '┴' nicht bewegt oder entfernt werden kann, sollten wir uns auf die Bewegung der Objekte konzentrieren. Ich denke, es wäre eine gute Option, wenn du mein Objekt T oder mein Objekt M bewegen könntest, um Platz zu schaffen. Aber ich denke, wir sollten auch meine Objekte überprüfen, um zu sehen, ob ich perhaps ein Objekt bewegen kann, um Platz zu schaffen. Lass uns nochmal unsere Objekte und ihre Positionen überprüfen, um einen neuen Plan zu erstellen. 
	  } 
	   } 
	   } 
	 & & \\ 
 

    \theutterance \stepcounter{utterance}  

    & & \multicolumn{4}{p{0.6\linewidth}}{\cellcolor[rgb]{0.9,0.9,0.9}{%
	\makecell[{{p{\linewidth}}}]{% 
	  \tt {\tiny [GM$\rangle$B]}  
	 Deine Nachricht wurde an deinen Mitspieler übermittelt.\\ \tt \\ \tt \\ \tt Ihr seid gerade bei Runde 26 von maximal 28.\\ \tt Ihr habt gemeinsam 7 von maximal 16 Strafpunkten angesammelt.\\ \tt Dein Mitspieler hat dir diese Nachricht geschickt:\\ \tt \\ \tt \\ \tt "Ich denke, das ist eine gute Idee! Wenn das '┴' nicht bewegt oder entfernt werden kann, sollten wir uns auf die Bewegung der Objekte konzentrieren. Ich denke, es wäre eine gute Option, wenn du mein Objekt T oder mein Objekt M bewegen könntest, um Platz zu schaffen. Aber ich denke, wir sollten auch meine Objekte überprüfen, um zu sehen, ob ich perhaps ein Objekt bewegen kann, um Platz zu schaffen. Lass uns nochmal unsere Objekte und ihre Positionen überprüfen, um einen neuen Plan zu erstellen."\\ \tt \\ \tt \\ \tt Wie lautet dein nächster Befehl? 
	  } 
	   } 
	   } 
	 & \\ 
 

    \theutterance \stepcounter{utterance}  

    & & & \multicolumn{4}{p{0.6\linewidth}}{\cellcolor[rgb]{1,0.85,0.72}{%
	\makecell[{{p{\linewidth}}}]{% 
	  \tt {\tiny [GM$\langle$B]}  
	 SPRICH: Ich denke, das ist eine gute Idee. Ich werde mein Raster nochmal überprüfen. Mein Objekt T ist in (2,2) und mein Objekt M ist in (3,3). Ich denke, es wäre schwierig, mein Objekt T zu bewegen, da es in einer Ecke des Rasters ist. Mein Objekt M könnte jedoch perhaps nach (2,3) oder einer anderen freien Position bewegt werden, um Platz zu schaffen. Ich frage mich, ob das helfen würde, mein Objekt O nach (4,4) zu bewegen. Was meinst du? Sollte ich versuchen, mein Objekt M zu bewegen, um Platz zu schaffen? 
	  } 
	   } 
	   } 
	  \\ 
 

    \theutterance \stepcounter{utterance}  

    & & & \multicolumn{2}{p{0.3\linewidth}}{\cellcolor[rgb]{0.95,0.95,0.95}{%
	\makecell[{{p{\linewidth}}}]{% 
	  \tt {\tiny [GM$|$GM]}  
	 SPRICH: Ich denke, das ist eine gute Idee. Ich werde mein Raster nochmal überprüfen. Mein Objekt T ist in (2,2) und mein Objekt M ist in (3,3). Ich denke, es wäre schwierig, mein Objekt T zu bewegen, da es in einer Ecke des Rasters ist. Mein Objekt M könnte jedoch perhaps nach (2,3) oder einer anderen freien Position bewegt werden, um Platz zu schaffen. Ich frage mich, ob das helfen würde, mein Objekt O nach (4,4) zu bewegen. Was meinst du? Sollte ich versuchen, mein Objekt M zu bewegen, um Platz zu schaffen? 
	  } 
	   } 
	   } 
	 & & \\ 
 

    \theutterance \stepcounter{utterance}  

    & & \multicolumn{4}{p{0.6\linewidth}}{\cellcolor[rgb]{0.9,0.9,0.9}{%
	\makecell[{{p{\linewidth}}}]{% 
	  \tt {\tiny [A$\langle$GM]}  
	 Deine Nachricht wurde an deinen Mitspieler übermittelt.\\ \tt \\ \tt \\ \tt Ihr seid gerade bei Runde 26 von maximal 28.\\ \tt Ihr habt gemeinsam 7 von maximal 16 Strafpunkten angesammelt.\\ \tt Dein Mitspieler hat dir diese Nachricht geschickt:\\ \tt \\ \tt \\ \tt "Ich denke, das ist eine gute Idee. Ich werde mein Raster nochmal überprüfen. Mein Objekt T ist in (2,2) und mein Objekt M ist in (3,3). Ich denke, es wäre schwierig, mein Objekt T zu bewegen, da es in einer Ecke des Rasters ist. Mein Objekt M könnte jedoch perhaps nach (2,3) oder einer anderen freien Position bewegt werden, um Platz zu schaffen. Ich frage mich, ob das helfen würde, mein Objekt O nach (4,4) zu bewegen. Was meinst du? Sollte ich versuchen, mein Objekt M zu bewegen, um Platz zu schaffen?"\\ \tt \\ \tt \\ \tt Wie lautet dein nächster Befehl? 
	  } 
	   } 
	   } 
	 & \\ 
 

    \theutterance \stepcounter{utterance}  

    & \multicolumn{4}{p{0.6\linewidth}}{\cellcolor[rgb]{0.8,1,0.9}{%
	\makecell[{{p{\linewidth}}}]{% 
	  \tt {\tiny [A$\rangle$GM]}  
	 SPRICH: Ja, das ist eine gute Idee! Ich denke, es wäre eine gute Option, wenn du dein Objekt M nach (2,3) oder einer anderen freien Position bewegen könntest, um Platz zu schaffen. Ich denke, das könnte perhaps helfen, dein Objekt O nach (4,4) zu bewegen. Bitte versuche, dein Objekt M zu bewegen und sieh, ob das funktioniert. Ich bin gespannt, ob das den gewünschten Effekt hat. 
	  } 
	   } 
	   } 
	 & & \\ 
 

    \theutterance \stepcounter{utterance}  

    & & & \multicolumn{2}{p{0.3\linewidth}}{\cellcolor[rgb]{0.95,0.95,0.95}{%
	\makecell[{{p{\linewidth}}}]{% 
	  \tt {\tiny [GM$|$GM]}  
	 SPRICH: Ja, das ist eine gute Idee! Ich denke, es wäre eine gute Option, wenn du dein Objekt M nach (2,3) oder einer anderen freien Position bewegen könntest, um Platz zu schaffen. Ich denke, das könnte perhaps helfen, dein Objekt O nach (4,4) zu bewegen. Bitte versuche, dein Objekt M zu bewegen und sieh, ob das funktioniert. Ich bin gespannt, ob das den gewünschten Effekt hat. 
	  } 
	   } 
	   } 
	 & & \\ 
 

    \theutterance \stepcounter{utterance}  

    & & \multicolumn{4}{p{0.6\linewidth}}{\cellcolor[rgb]{0.9,0.9,0.9}{%
	\makecell[{{p{\linewidth}}}]{% 
	  \tt {\tiny [GM$\rangle$B]}  
	 Deine Nachricht wurde an deinen Mitspieler übermittelt.\\ \tt \\ \tt \\ \tt Ihr seid gerade bei Runde 27 von maximal 28.\\ \tt Ihr habt gemeinsam 7 von maximal 16 Strafpunkten angesammelt.\\ \tt Dein Mitspieler hat dir diese Nachricht geschickt:\\ \tt \\ \tt \\ \tt "Ja, das ist eine gute Idee! Ich denke, es wäre eine gute Option, wenn du dein Objekt M nach (2,3) oder einer anderen freien Position bewegen könntest, um Platz zu schaffen. Ich denke, das könnte perhaps helfen, dein Objekt O nach (4,4) zu bewegen. Bitte versuche, dein Objekt M zu bewegen und sieh, ob das funktioniert. Ich bin gespannt, ob das den gewünschten Effekt hat."\\ \tt \\ \tt \\ \tt Wie lautet dein nächster Befehl? 
	  } 
	   } 
	   } 
	 & \\ 
 

    \theutterance \stepcounter{utterance}  

    & & & \multicolumn{4}{p{0.6\linewidth}}{\cellcolor[rgb]{1,0.85,0.72}{%
	\makecell[{{p{\linewidth}}}]{% 
	  \tt {\tiny [GM$\langle$B]}  
	 VERSCHIEBE: M, (2,3) 
	  } 
	   } 
	   } 
	  \\ 
 

    \theutterance \stepcounter{utterance}  

    & & & \multicolumn{2}{p{0.3\linewidth}}{\cellcolor[rgb]{0.95,0.95,0.95}{%
	\makecell[{{p{\linewidth}}}]{% 
	  \tt {\tiny [GM$|$GM]}  
	 VERSCHIEBE: M, (2,3) 
	  } 
	   } 
	   } 
	 & & \\ 
 

    \theutterance \stepcounter{utterance}  

    & & & \multicolumn{2}{p{0.3\linewidth}}{\cellcolor[rgb]{0.95,0.95,0.95}{%
	\makecell[{{p{\linewidth}}}]{% 
	  \tt {\tiny [GM$|$GM]}  
	 'M' erfolgreich nach (2,3) verschoben. Dein Raster sieht jetzt so aus:\\ \tt ```\\ \tt  1234567\\ \tt ╔═╤═╤═╤═╗\\ \tt ║◌│◌│◌│P║ 1\\ \tt ║T└─┼─┴─╢ 2\\ \tt ║◌M◌│◌◌◌║ 3\\ \tt ╟──┬┴───╢ 4\\ \tt ║◌◌│O◌◌◌║ 5\\ \tt ║AD│◌◌◌◌║ 6\\ \tt ║◌◌│◌◌S◌║ 7\\ \tt ╚══╧════╝\\ \tt ```\\ \tt  
	  } 
	   } 
	   } 
	 & & \\ 
 

    \theutterance \stepcounter{utterance}  

    & & \multicolumn{4}{p{0.6\linewidth}}{\cellcolor[rgb]{0.9,0.9,0.9}{%
	\makecell[{{p{\linewidth}}}]{% 
	  \tt {\tiny [A$\langle$GM]}  
	 Deine Nachricht wurde an deinen Mitspieler übermittelt.\\ \tt \\ \tt \\ \tt Ihr seid gerade bei Runde 27 von maximal 28.\\ \tt Ihr habt gemeinsam 7 von maximal 16 Strafpunkten angesammelt.\\ \tt Dein Mitspieler hat ein Objekt auf seinem Raster bewegt.\\ \tt \\ \tt \\ \tt Wie lautet dein nächster Befehl? 
	  } 
	   } 
	   } 
	 & \\ 
 

    \theutterance \stepcounter{utterance}  

    & \multicolumn{4}{p{0.6\linewidth}}{\cellcolor[rgb]{0.8,1,0.9}{%
	\makecell[{{p{\linewidth}}}]{% 
	  \tt {\tiny [A$\rangle$GM]}  
	 SPRICH: Wie sieht es jetzt bei dir aus? Hast du dein Objekt M erfolgreich bewegt und kannst du jetzt dein Objekt O nach (4,4) bewegen? 
	  } 
	   } 
	   } 
	 & & \\ 
 

    \theutterance \stepcounter{utterance}  

    & & & \multicolumn{2}{p{0.3\linewidth}}{\cellcolor[rgb]{0.95,0.95,0.95}{%
	\makecell[{{p{\linewidth}}}]{% 
	  \tt {\tiny [GM$|$GM]}  
	 SPRICH: Wie sieht es jetzt bei dir aus? Hast du dein Objekt M erfolgreich bewegt und kannst du jetzt dein Objekt O nach (4,4) bewegen? 
	  } 
	   } 
	   } 
	 & & \\ 
 

    \theutterance \stepcounter{utterance}  

    & & \multicolumn{4}{p{0.6\linewidth}}{\cellcolor[rgb]{0.9,0.9,0.9}{%
	\makecell[{{p{\linewidth}}}]{% 
	  \tt {\tiny [GM$\rangle$B]}  
	 'M' erfolgreich nach (2,3) verschoben. Dein Raster sieht jetzt so aus:\\ \tt ```\\ \tt  1234567\\ \tt ╔═╤═╤═╤═╗\\ \tt ║◌│◌│◌│P║ 1\\ \tt ║T└─┼─┴─╢ 2\\ \tt ║◌M◌│◌◌◌║ 3\\ \tt ╟──┬┴───╢ 4\\ \tt ║◌◌│O◌◌◌║ 5\\ \tt ║AD│◌◌◌◌║ 6\\ \tt ║◌◌│◌◌S◌║ 7\\ \tt ╚══╧════╝\\ \tt ```\\ \tt Ihr seid gerade bei Runde 28 von maximal 28.\\ \tt Ihr habt gemeinsam 7 von maximal 16 Strafpunkten angesammelt.\\ \tt Dein Mitspieler hat dir diese Nachricht geschickt:\\ \tt \\ \tt \\ \tt "Wie sieht es jetzt bei dir aus? Hast du dein Objekt M erfolgreich bewegt und kannst du jetzt dein Objekt O nach (4,4) bewegen?"\\ \tt \\ \tt \\ \tt Wie lautet dein nächster Befehl? 
	  } 
	   } 
	   } 
	 & \\ 
 

    \theutterance \stepcounter{utterance}  

    & & & \multicolumn{4}{p{0.6\linewidth}}{\cellcolor[rgb]{1,0.85,0.72}{%
	\makecell[{{p{\linewidth}}}]{% 
	  \tt {\tiny [GM$\langle$B]}  
	 SPRICH: Ja, ich habe mein Objekt M erfolgreich nach (2,3) bewegt. Leider hat sich dadurch nichts geändert, was das '┴' in (4,4) angeht. Mein Objekt O ist immer noch in (4,5) und kann nicht nach (4,4) bewegt werden, weil das '┴' im Weg ist. Ich denke, wir haben alle möglichen Schritte unternommen, um die Objekte an ihre Zielpositionen zu bewegen. Wir sollten jetzt überprüfen, ob wir das Spiel gewonnen haben oder nicht. Ich denke, wir sollten "fertig" sagen und sehen, ob wir das Spiel gewonnen haben. Was meinst du? Sollten wir "fertig" sagen? 
	  } 
	   } 
	   } 
	  \\ 
 

    \theutterance \stepcounter{utterance}  

    & & & \multicolumn{2}{p{0.3\linewidth}}{\cellcolor[rgb]{0.95,0.95,0.95}{%
	\makecell[{{p{\linewidth}}}]{% 
	  \tt {\tiny [GM$|$GM]}  
	 SPRICH: Ja, ich habe mein Objekt M erfolgreich nach (2,3) bewegt. Leider hat sich dadurch nichts geändert, was das '┴' in (4,4) angeht. Mein Objekt O ist immer noch in (4,5) und kann nicht nach (4,4) bewegt werden, weil das '┴' im Weg ist. Ich denke, wir haben alle möglichen Schritte unternommen, um die Objekte an ihre Zielpositionen zu bewegen. Wir sollten jetzt überprüfen, ob wir das Spiel gewonnen haben oder nicht. Ich denke, wir sollten "fertig" sagen und sehen, ob wir das Spiel gewonnen haben. Was meinst du? Sollten wir "fertig" sagen? 
	  } 
	   } 
	   } 
	 & & \\ 
 

    \theutterance \stepcounter{utterance}  

    & & & \multicolumn{2}{p{0.3\linewidth}}{\cellcolor[rgb]{0.95,0.95,0.95}{%
	\makecell[{{p{\linewidth}}}]{% 
	  \tt {\tiny [GM$|$GM]}  
	 Maximum number of rounds reached 
	  } 
	   } 
	   } 
	 & & \\ 
 

    \theutterance \stepcounter{utterance}  

    & & & \multicolumn{2}{p{0.3\linewidth}}{\cellcolor[rgb]{0.95,0.95,0.95}{%
	\makecell[{{p{\linewidth}}}]{% 
	  \tt {\tiny [GM$|$GM]}  
	 * success: True\\ \tt * lose: True\\ \tt * aborted: False\\ \tt -------\\ \tt * Shifts: 6.00\\ \tt * Max Shifts: 12.00\\ \tt * Min Shifts: 6.00\\ \tt * End Distance Sum: 33.07\\ \tt * Init Distance Sum: 30.87\\ \tt * Expected Distance Sum: 29.33\\ \tt * Penalties: 7.00\\ \tt * Max Penalties: 16.00\\ \tt * Rounds: 28.00\\ \tt * Max Rounds: 28.00\\ \tt * Object Count: 7.00\\ \tt  
	  } 
	   } 
	   } 
	 & & \\ 
 

    \theutterance \stepcounter{utterance}  

    & & & \multicolumn{2}{p{0.3\linewidth}}{\cellcolor[rgb]{0.95,0.95,0.95}{%
	\makecell[{{p{\linewidth}}}]{% 
	  \tt {\tiny [GM$|$GM]}  
	 * Main Score: 0.00\\ \tt \\ \tt -------\\ \tt * Distance Score: 0.00\\ \tt * Consistency Score: 1.00\\ \tt * Coverage Score: 0.28\\ \tt * Penalty Score: 0.67\\ \tt * Alternative Penalty Score: 0.80\\ \tt * Alternative Main Score: 0.00\\ \tt \\ \tt -------\\ \tt * Shifts: 6.00\\ \tt * Max Shifts: 12.00\\ \tt * Min Shifts: 6.00\\ \tt * End Distance Sum: 33.07\\ \tt * Init Distance Sum: 30.87\\ \tt * Expected Distance Sum: 29.33\\ \tt * Penalties: 7.00\\ \tt * Max Penalties: 16.00\\ \tt * Rounds: 28.00\\ \tt * Max Rounds: 28.00\\ \tt * Object Count: 7.00\\ \tt  
	  } 
	   } 
	   } 
	 & & \\ 
 

\end{supertabular}
}

\end{document}
