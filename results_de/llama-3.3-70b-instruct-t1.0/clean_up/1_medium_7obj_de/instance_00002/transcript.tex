
\documentclass{article}
\usepackage{colortbl}
\usepackage{makecell}
\usepackage{multirow}
\usepackage{supertabular}

\begin{document}

\newcounter{utterance}

\centering \large Interaction Transcript for game `clean\_up', experiment `1\_medium\_7obj\_de', episode 2 with llama{-}3.3{-}70b{-}instruct{-}t1.0.
\vspace{24pt}

{ \footnotesize  \setcounter{utterance}{1}
\setlength{\tabcolsep}{0pt}
\begin{supertabular}{c@{$\;$}|p{.15\linewidth}@{}p{.15\linewidth}p{.15\linewidth}p{.15\linewidth}p{.15\linewidth}p{.15\linewidth}}
    \# & $\;$A & \multicolumn{4}{c}{Game Master} & $\;\:$B\\
    \hline

    \theutterance \stepcounter{utterance}  
    & & \multicolumn{4}{p{0.6\linewidth}}{
        \cellcolor[rgb]{0.9,0.9,0.9}{
            \makecell[{{p{\linewidth}}}]{
                \texttt{\tiny{[P1$\langle$GM]}}
                \texttt{Ich bin der Spielleiter, und du spielst ein kooperatives Spiel mit dem folgenden Raster als Spielbrett:} \\
\\ 
\texttt{```} \\
\texttt{ 1234567} \\
\texttt{╔═╤═╤═╤═╗} \\
\texttt{║◌│◌│◌│A║ 1} \\
\texttt{╟─┴─┼─┤◌║ 2} \\
\texttt{║◌◌M│◌└─╢ 3} \\
\texttt{╟───┘◌S◌║ 4} \\
\texttt{║◌◌◌◌◌┌─╢ 5} \\
\texttt{║◌◌◌PT│O║ 6} \\
\texttt{║◌D◌◌◌│◌║ 7} \\
\texttt{╚═════╧═╝} \\
\\ 
\texttt{```} \\
\\ 
\texttt{* Am oberen Rand sind die x{-}Koordinaten angegeben, die nach rechts hin zunehmen, und am rechten Rand die y{-}Koordinaten, die nach unten zunehmen.} \\
\texttt{* Die folgenden Objekte sind zufällig auf deinem Raster platziert: 'P', 'O', 'T', 'S', 'D', 'A', 'M'.} \\
\\ 
\\ 
\texttt{Dein Mitspieler sieht eine Variation des Spielbretts, auf dem dieselben Objekte an anderen zufälligen Positionen platziert sind. Du kannst das Spielbrett des anderen Spielers nicht sehen, und er kann deins nicht sehen.} \\
\\ 
\\ 
\texttt{**Ziel des Spiels:**} \\
\\ 
\texttt{Beide Spieler müssen die Objekte auf ihrem jeweiligen Spielbrett so bewegen, dass identische Objekte am Ende an denselben Koordinaten liegen. Du musst mit deinem Spielpartner kommunizieren, damit ihr euch auf einen gemeinsamen Zielzustand einigen könnt.} \\
\\ 
\\ 
\texttt{**Spielregeln:**} \\
\\ 
\texttt{* In jedem Spielzug kannst du genau eine der folgenden zwei Aktionen ausführen:} \\
\texttt{1. `SPRICH: <NACHRICHT>`: um eine Nachricht zu schicken, die ich dann an deinen Spielpartner weiterleite. Als Nachricht gilt alles bis zum nächsten Zeilenumbruch.} \\
\texttt{2. `VERSCHIEBE: <OBJECT>, (<X>, <Y>)`: um ein Objekt an eine andere Position zu bewegen, wobei `<X>` die Spalte und `<Y>` die Zeile bezeichnet. Ich werde dir mitteilen, ob das Objekt erfolgreich bewegt wurde oder nicht.} \\
\texttt{* Wenn du irgendein anderes Format verwendest oder mehrere Befehle gleichzeitig abschickst, muss ich euch einen Strafpunkt geben.} \\
\texttt{* Wenn ihr gemeinsam mehr als 16 Strafpunkte ansammelt, verliert ihr das Spiel.} \\
\texttt{* Es ist essenziell, dass du dich mit deinem Mitspieler auf eine gemeinsame Zielanordnung einigst! Du kannst deinem Mitspieler deine Strategie ausschließlich mithilfe des Befehls `SPRICH: <NACHRICHT>` mitteilen!} \\
\\ 
\\ 
\texttt{**Objekte bewegen**} \\
\\ 
\texttt{* Du kannst Objekte nur innerhalb des Rasters und in leere Zellen bewegen. Leere Zellen erkennst am Zeichen '◌'.} \\
\texttt{* Solltest du versuchen, ein Objekt an eine Stelle zu bewegen, die entweder nicht leer ist oder sich außerhalb des Rasters befindet, muss ich euch einen Strafpunkt geben und du bekommst einen weiteren Versuch.} \\
\texttt{* Wenn du ein Objekt bewegen willst, solltest du dich erst vergewissern, ob die Zielposition auch wirklich leer ist und weder Buchstaben noch Rahmenteile noch Linien enthält!} \\
\\ 
\\ 
\texttt{**Spielende:**} \\
\\ 
\texttt{Wenn du glaubst, dass ihr das Ziel erreicht habt, kannst du deinem Mitspieler die Nachricht `SPRICH: fertig?` schicken. Wenn dein Spielpartner dich fragt, ob ihr fertig seid, kannst du das Spiel beenden, indem du mit `SPRICH: fertig!` antwortest.} \\
\\ 
\texttt{Ihr gewinnt gemeinsam, wenn ihr das Spiel innerhalb von \#max\_rounds Runden beendet. Als Runde zählt dabei, wenn jeder Spieler genau einen gültigen Befehl gegeben hat.} \\
\\ 
\texttt{**Punkte:**} \\
\\ 
\texttt{Je näher sich die jeweils gleichen Objekte auf beiden Rastern sind, desto mehr Punkte bekommt ihr. Für Strafpunkte gibt es Punktabzug. Könnt ihr den Rekord brechen?} \\
\\ 
\texttt{Schicke eine Nachricht an deinen Mitspieler um das Spiel zu starten!} \\
            }
        }
    }
    & \\ \\

    \theutterance \stepcounter{utterance}  
    & \multicolumn{4}{p{0.6\linewidth}}{
        \cellcolor[rgb]{0.9,0.9,0.9}{
            \makecell[{{p{\linewidth}}}]{
                \texttt{\tiny{[P1$\rangle$GM]}}
                \texttt{SPRICH: Hallo, ich schlage vor, dass wir zunächst alle Objekte in der Mitte des Rasters platzieren, um eine gemeinsame Ausgangsposition zu haben. Dann können wir versuchen, die Objekte Schritt für Schritt an die gleichen Positionen zu bewegen. Wie sieht es bei dir aus?} \\
            }
        }
    }
    & & \\ \\

    \theutterance \stepcounter{utterance}  
    & & & \multicolumn{2}{p{0.3\linewidth}}{
        \cellcolor[rgb]{0.9,0.9,0.9}{
            \makecell[{{p{\linewidth}}}]{
                \texttt{\tiny{[GM$|$GM]}}
                \texttt{SPRICH: Hallo, ich schlage vor, dass wir zunächst alle Objekte in der Mitte des Rasters platzieren, um eine gemeinsame Ausgangsposition zu haben. Dann können wir versuchen, die Objekte Schritt für Schritt an die gleichen Positionen zu bewegen. Wie sieht es bei dir aus?} \\
            }
        }
    }
    & & \\ \\

    \theutterance \stepcounter{utterance}  
    & & \multicolumn{4}{p{0.6\linewidth}}{
        \cellcolor[rgb]{0.9,0.9,0.9}{
            \makecell[{{p{\linewidth}}}]{
                \texttt{\tiny{[GM$\rangle$P2]}}
                \texttt{Ich bin der Spielleiter, und du spielst ein kooperatives Spiel mit dem folgenden Raster als Spielbrett:} \\
\\ 
\texttt{```} \\
\texttt{ 1234567} \\
\texttt{╔═╤═╤═╤═╗} \\
\texttt{║◌│◌│◌│◌║ 1} \\
\texttt{╟─┴─┼─┤◌║ 2} \\
\texttt{║◌M◌│◌└─╢ 3} \\
\texttt{╟───┘P◌◌║ 4} \\
\texttt{║O◌◌◌◌┌─╢ 5} \\
\texttt{║◌D◌◌A│S║ 6} \\
\texttt{║◌◌◌◌T│◌║ 7} \\
\texttt{╚═════╧═╝} \\
\\ 
\texttt{```} \\
\\ 
\texttt{* Am oberen Rand sind die x{-}Koordinaten angegeben, die nach rechts hin zunehmen, und am rechten Rand die y{-}Koordinaten, die nach unten zunehmen.} \\
\texttt{* Die folgenden Objekte sind zufällig auf deinem Raster platziert: 'P', 'O', 'T', 'S', 'D', 'A', 'M'.} \\
\\ 
\\ 
\texttt{Dein Mitspieler sieht eine Variation des Spielbretts, auf dem dieselben Objekte an anderen zufälligen Positionen platziert sind. Du kannst das Spielbrett des anderen Spielers nicht sehen, und er kann deins nicht sehen.} \\
\\ 
\\ 
\texttt{**Ziel des Spiels:**} \\
\\ 
\texttt{Beide Spieler müssen die Objekte auf ihrem jeweiligen Spielbrett so bewegen, dass identische Objekte am Ende an denselben Koordinaten liegen. Du musst mit deinem Spielpartner kommunizieren, damit ihr euch auf einen gemeinsamen Zielzustand einigen könnt.} \\
\\ 
\\ 
\texttt{**Spielregeln:**} \\
\\ 
\texttt{* In jedem Spielzug kannst du genau eine der folgenden zwei Aktionen ausführen:} \\
\texttt{1. `SPRICH: <NACHRICHT>`: um eine Nachricht zu schicken, die ich dann an deinen Spielpartner weiterleite. Als Nachricht gilt alles bis zum nächsten Zeilenumbruch.} \\
\texttt{2. `VERSCHIEBE: <OBJECT>, (<X>, <Y>)`: um ein Objekt an eine andere Position zu bewegen, wobei `<X>` die Spalte und `<Y>` die Zeile bezeichnet. Ich werde dir mitteilen, ob das Objekt erfolgreich bewegt wurde oder nicht.} \\
\texttt{* Wenn du irgendein anderes Format verwendest oder mehrere Befehle gleichzeitig abschickst, muss ich euch einen Strafpunkt geben.} \\
\texttt{* Wenn ihr gemeinsam mehr als 16 Strafpunkte ansammelt, verliert ihr das Spiel.} \\
\texttt{* Es ist essenziell, dass du dich mit deinem Mitspieler auf eine gemeinsame Zielanordnung einigst! Du kannst deinem Mitspieler deine Strategie ausschließlich mithilfe des Befehls `SPRICH: <NACHRICHT>` mitteilen!} \\
\\ 
\\ 
\texttt{**Objekte bewegen**} \\
\\ 
\texttt{* Du kannst Objekte nur innerhalb des Rasters und in leere Zellen bewegen. Leere Zellen erkennst am Zeichen '◌'.} \\
\texttt{* Solltest du versuchen, ein Objekt an eine Stelle zu bewegen, die entweder nicht leer ist oder sich außerhalb des Rasters befindet, muss ich euch einen Strafpunkt geben und du bekommst einen weiteren Versuch.} \\
\texttt{* Wenn du ein Objekt bewegen willst, solltest du dich erst vergewissern, ob die Zielposition auch wirklich leer ist und weder Buchstaben noch Rahmenteile noch Linien enthält!} \\
\\ 
\\ 
\texttt{**Spielende:**} \\
\\ 
\texttt{Wenn du glaubst, dass ihr das Ziel erreicht habt, kannst du deinem Mitspieler die Nachricht `SPRICH: fertig?` schicken. Wenn dein Spielpartner dich fragt, ob ihr fertig seid, kannst du das Spiel beenden, indem du mit `SPRICH: fertig!` antwortest.} \\
\\ 
\texttt{Ihr gewinnt gemeinsam, wenn ihr das Spiel innerhalb von \#max\_rounds Runden beendet. Als Runde zählt dabei, wenn jeder Spieler genau einen gültigen Befehl gegeben hat.} \\
\\ 
\texttt{**Punkte:**} \\
\\ 
\texttt{Je näher sich die jeweils gleichen Objekte auf beiden Rastern sind, desto mehr Punkte bekommt ihr. Für Strafpunkte gibt es Punktabzug. Könnt ihr den Rekord brechen?} \\
\\ 
\texttt{Dein Mitspieler hat das Spiel mit dieser Nachricht begonnen:} \\
\\ 
\texttt{"Hallo, ich schlage vor, dass wir zunächst alle Objekte in der Mitte des Rasters platzieren, um eine gemeinsame Ausgangsposition zu haben. Dann können wir versuchen, die Objekte Schritt für Schritt an die gleichen Positionen zu bewegen. Wie sieht es bei dir aus?"} \\
\\ 
\texttt{Wie lautet dein erster Befehl?} \\
            }
        }
    }
    & \\ \\

    \theutterance \stepcounter{utterance}  
    & & & \multicolumn{4}{p{0.6\linewidth}}{
        \cellcolor[rgb]{0.9,0.9,0.9}{
            \makecell[{{p{\linewidth}}}]{
                \texttt{\tiny{[GM$\langle$P2]}}
                \texttt{Das klingt nach einem guten Ansatz! Ich denke, es ist eine gute Idee, eine gemeinsame Ausgangsposition zu haben, um die gleichen Objekte an die gleichen Positionen zu bewegen.} \\
\\ 
\texttt{Mein erster Befehl ist: `SPRICH: Hallo, das klingt gut! Ich stimme deinem Vorschlag zu. Ich werde meine Objekte in der Mitte des Rasters platzieren. Wie viele Objekte hast du insgesamt auf deinem Raster? Ich habe 7 Objekte: P, O, T, S, D, A und M.`} \\
\\ 
\texttt{Ich möchte wissen, ob mein Mitspieler die gleichen Objekte hat und wie viele Objekte insgesamt auf seinem Raster sind, um eine gemeinsame Strategie zu entwickeln.} \\
            }
        }
    }
     \\ \\

    \theutterance \stepcounter{utterance}  
    & & & \multicolumn{2}{p{0.3\linewidth}}{
        \cellcolor[rgb]{0.9,0.9,0.9}{
            \makecell[{{p{\linewidth}}}]{
                \texttt{\tiny{[GM$|$GM]}}
                \texttt{Das klingt nach einem guten Ansatz! Ich denke, es ist eine gute Idee, eine gemeinsame Ausgangsposition zu haben, um die gleichen Objekte an die gleichen Positionen zu bewegen.} \\
\\ 
\texttt{Mein erster Befehl ist: `SPRICH: Hallo, das klingt gut! Ich stimme deinem Vorschlag zu. Ich werde meine Objekte in der Mitte des Rasters platzieren. Wie viele Objekte hast du insgesamt auf deinem Raster? Ich habe 7 Objekte: P, O, T, S, D, A und M.`} \\
\\ 
\texttt{Ich möchte wissen, ob mein Mitspieler die gleichen Objekte hat und wie viele Objekte insgesamt auf seinem Raster sind, um eine gemeinsame Strategie zu entwickeln.} \\
            }
        }
    }
    & & \\ \\

    \theutterance \stepcounter{utterance}  
    & & & \multicolumn{2}{p{0.3\linewidth}}{
        \cellcolor[rgb]{0.9,0.9,0.9}{
            \makecell[{{p{\linewidth}}}]{
                \texttt{\tiny{[GM$|$GM]}}
                \texttt{Invalid format: head and tail are not empty} \\
\texttt{head: 'Das klingt nach einem guten Ansatz! Ich denke, es ist eine gute Idee, eine gemeinsame Ausgangsposition zu haben, um die gleichen Objekte an die gleichen Positionen zu bewegen.} \\
\\ 
\texttt{Mein erster Befehl ist: '} \\
\texttt{tail: '} \\
\\ 
\texttt{Ich möchte wissen, ob mein Mitspieler die gleichen Objekte hat und wie viele Objekte insgesamt auf seinem Raster sind, um eine gemeinsame Strategie zu entwickeln.'} \\
            }
        }
    }
    & & \\ \\

    \theutterance \stepcounter{utterance}  
    & & \multicolumn{4}{p{0.6\linewidth}}{
        \cellcolor[rgb]{0.9,0.9,0.9}{
            \makecell[{{p{\linewidth}}}]{
                \texttt{\tiny{[GM$\rangle$P2]}}
                \texttt{Strafpunkt: Deine Nachricht darf nichts vor oder nach dem Befehl enthalten! Stelle sicher, dass deine Antwort ausschließlich einen der beiden Befehle `SPRICH: <NACHRICHT>` und `VERSCHIEBE: <OBJECT>, (<X>, <Y>)` und sonst nichts enthält!} \\
\\ 
\texttt{Ihr habt gemeinsam 1 von maximal 16 Strafpunkten angesammelt.} \\
\texttt{Bitte versuche es noch mal!} \\
            }
        }
    }
    & \\ \\

    \theutterance \stepcounter{utterance}  
    & & & \multicolumn{4}{p{0.6\linewidth}}{
        \cellcolor[rgb]{0.9,0.9,0.9}{
            \makecell[{{p{\linewidth}}}]{
                \texttt{\tiny{[GM$\langle$P2]}}
                \texttt{SPRICH: Hallo, das klingt gut! Ich stimme deinem Vorschlag zu. Ich werde meine Objekte in der Mitte des Rasters platzieren. Wie viele Objekte hast du insgesamt auf deinem Raster? Ich habe 7 Objekte: P, O, T, S, D, A und M.} \\
            }
        }
    }
     \\ \\

    \theutterance \stepcounter{utterance}  
    & & & \multicolumn{2}{p{0.3\linewidth}}{
        \cellcolor[rgb]{0.9,0.9,0.9}{
            \makecell[{{p{\linewidth}}}]{
                \texttt{\tiny{[GM$|$GM]}}
                \texttt{SPRICH: Hallo, das klingt gut! Ich stimme deinem Vorschlag zu. Ich werde meine Objekte in der Mitte des Rasters platzieren. Wie viele Objekte hast du insgesamt auf deinem Raster? Ich habe 7 Objekte: P, O, T, S, D, A und M.} \\
            }
        }
    }
    & & \\ \\

    \theutterance \stepcounter{utterance}  
    & & \multicolumn{4}{p{0.6\linewidth}}{
        \cellcolor[rgb]{0.9,0.9,0.9}{
            \makecell[{{p{\linewidth}}}]{
                \texttt{\tiny{[P1$\langle$GM]}}
                \texttt{Deine Nachricht wurde an deinen Mitspieler übermittelt.} \\
\\ 
\\ 
\texttt{Ihr seid gerade bei Runde 1 von maximal 28.} \\
\texttt{Ihr habt gemeinsam 1 von maximal 16 Strafpunkten angesammelt.} \\
\texttt{Dein Mitspieler hat dir diese Nachricht geschickt:} \\
\\ 
\\ 
\texttt{"Hallo, das klingt gut! Ich stimme deinem Vorschlag zu. Ich werde meine Objekte in der Mitte des Rasters platzieren. Wie viele Objekte hast du insgesamt auf deinem Raster? Ich habe 7 Objekte: P, O, T, S, D, A und M."} \\
\\ 
\\ 
\texttt{Wie lautet dein nächster Befehl?} \\
            }
        }
    }
    & \\ \\

    \theutterance \stepcounter{utterance}  
    & \multicolumn{4}{p{0.6\linewidth}}{
        \cellcolor[rgb]{0.9,0.9,0.9}{
            \makecell[{{p{\linewidth}}}]{
                \texttt{\tiny{[P1$\rangle$GM]}}
                \texttt{SPRICH: Ich habe auch 7 Objekte: P, O, T, S, D, A und M. Ich denke, wir sollten uns auf eine feste Reihenfolge für die Objekte einigen, um sie in der Mitte des Rasters zu platzieren. Wie wäre es, wenn wir sie alphabetisch anordnen? Das würde bedeuten: A, D, M, O, P, S, T. Wenn du einverstanden bist, können wir начnen, unsere Objekte entsprechend zu platzieren.} \\
            }
        }
    }
    & & \\ \\

    \theutterance \stepcounter{utterance}  
    & & & \multicolumn{2}{p{0.3\linewidth}}{
        \cellcolor[rgb]{0.9,0.9,0.9}{
            \makecell[{{p{\linewidth}}}]{
                \texttt{\tiny{[GM$|$GM]}}
                \texttt{SPRICH: Ich habe auch 7 Objekte: P, O, T, S, D, A und M. Ich denke, wir sollten uns auf eine feste Reihenfolge für die Objekte einigen, um sie in der Mitte des Rasters zu platzieren. Wie wäre es, wenn wir sie alphabetisch anordnen? Das würde bedeuten: A, D, M, O, P, S, T. Wenn du einverstanden bist, können wir начnen, unsere Objekte entsprechend zu platzieren.} \\
            }
        }
    }
    & & \\ \\

    \theutterance \stepcounter{utterance}  
    & & \multicolumn{4}{p{0.6\linewidth}}{
        \cellcolor[rgb]{0.9,0.9,0.9}{
            \makecell[{{p{\linewidth}}}]{
                \texttt{\tiny{[GM$\rangle$P2]}}
                \texttt{Deine Nachricht wurde an deinen Mitspieler übermittelt.} \\
\\ 
\\ 
\texttt{Ihr seid gerade bei Runde 2 von maximal 28.} \\
\texttt{Ihr habt gemeinsam 1 von maximal 16 Strafpunkten angesammelt.} \\
\texttt{Dein Mitspieler hat dir diese Nachricht geschickt:} \\
\\ 
\\ 
\texttt{"Ich habe auch 7 Objekte: P, O, T, S, D, A und M. Ich denke, wir sollten uns auf eine feste Reihenfolge für die Objekte einigen, um sie in der Mitte des Rasters zu platzieren. Wie wäre es, wenn wir sie alphabetisch anordnen? Das würde bedeuten: A, D, M, O, P, S, T. Wenn du einverstanden bist, können wir начnen, unsere Objekte entsprechend zu platzieren."} \\
\\ 
\\ 
\texttt{Wie lautet dein nächster Befehl?} \\
            }
        }
    }
    & \\ \\

    \theutterance \stepcounter{utterance}  
    & & & \multicolumn{4}{p{0.6\linewidth}}{
        \cellcolor[rgb]{0.9,0.9,0.9}{
            \makecell[{{p{\linewidth}}}]{
                \texttt{\tiny{[GM$\langle$P2]}}
                \texttt{SPRICH: Ja, das klingt gut! Ich stimme der alphabetischen Reihenfolge zu: A, D, M, O, P, S, T. Lass uns beginnen, unsere Objekte entsprechend zu platzieren. Ich werde meine Objekte in der Mitte des Rasters anordnen und dann weiter 暫usavanaughnodiscard anzeigen, wie es bei dir aussieht.} \\
            }
        }
    }
     \\ \\

    \theutterance \stepcounter{utterance}  
    & & & \multicolumn{2}{p{0.3\linewidth}}{
        \cellcolor[rgb]{0.9,0.9,0.9}{
            \makecell[{{p{\linewidth}}}]{
                \texttt{\tiny{[GM$|$GM]}}
                \texttt{SPRICH: Ja, das klingt gut! Ich stimme der alphabetischen Reihenfolge zu: A, D, M, O, P, S, T. Lass uns beginnen, unsere Objekte entsprechend zu platzieren. Ich werde meine Objekte in der Mitte des Rasters anordnen und dann weiter 暫usavanaughnodiscard anzeigen, wie es bei dir aussieht.} \\
            }
        }
    }
    & & \\ \\

    \theutterance \stepcounter{utterance}  
    & & \multicolumn{4}{p{0.6\linewidth}}{
        \cellcolor[rgb]{0.9,0.9,0.9}{
            \makecell[{{p{\linewidth}}}]{
                \texttt{\tiny{[P1$\langle$GM]}}
                \texttt{Deine Nachricht wurde an deinen Mitspieler übermittelt.} \\
\\ 
\\ 
\texttt{Ihr seid gerade bei Runde 2 von maximal 28.} \\
\texttt{Ihr habt gemeinsam 1 von maximal 16 Strafpunkten angesammelt.} \\
\texttt{Dein Mitspieler hat dir diese Nachricht geschickt:} \\
\\ 
\\ 
\texttt{"Ja, das klingt gut! Ich stimme der alphabetischen Reihenfolge zu: A, D, M, O, P, S, T. Lass uns beginnen, unsere Objekte entsprechend zu platzieren. Ich werde meine Objekte in der Mitte des Rasters anordnen und dann weiter 暫usavanaughnodiscard anzeigen, wie es bei dir aussieht."} \\
\\ 
\\ 
\texttt{Wie lautet dein nächster Befehl?} \\
            }
        }
    }
    & \\ \\

    \theutterance \stepcounter{utterance}  
    & \multicolumn{4}{p{0.6\linewidth}}{
        \cellcolor[rgb]{0.9,0.9,0.9}{
            \makecell[{{p{\linewidth}}}]{
                \texttt{\tiny{[P1$\rangle$GM]}}
                \texttt{SPRICH: Ich verstehe, dass du deine Objekte in der Mitte des Rasters anordnen wirst. Ich werde dasselbe tun. Da unser Raster 7 Spalten und 7 Zeilen hat, würde ich vorschlagen, die Objekte in der 4. Zeile und den Spalten 2{-}8 zu platzieren, um sie möglichst zentral zu halten. Ich werde beginnen, meine Objekte entsprechend zu platzieren. Bitte teile mir mit, wenn duReady bist, damit wir synchron weitermachen können.} \\
            }
        }
    }
    & & \\ \\

    \theutterance \stepcounter{utterance}  
    & & & \multicolumn{2}{p{0.3\linewidth}}{
        \cellcolor[rgb]{0.9,0.9,0.9}{
            \makecell[{{p{\linewidth}}}]{
                \texttt{\tiny{[GM$|$GM]}}
                \texttt{SPRICH: Ich verstehe, dass du deine Objekte in der Mitte des Rasters anordnen wirst. Ich werde dasselbe tun. Da unser Raster 7 Spalten und 7 Zeilen hat, würde ich vorschlagen, die Objekte in der 4. Zeile und den Spalten 2{-}8 zu platzieren, um sie möglichst zentral zu halten. Ich werde beginnen, meine Objekte entsprechend zu platzieren. Bitte teile mir mit, wenn duReady bist, damit wir synchron weitermachen können.} \\
            }
        }
    }
    & & \\ \\

    \theutterance \stepcounter{utterance}  
    & & \multicolumn{4}{p{0.6\linewidth}}{
        \cellcolor[rgb]{0.9,0.9,0.9}{
            \makecell[{{p{\linewidth}}}]{
                \texttt{\tiny{[GM$\rangle$P2]}}
                \texttt{Deine Nachricht wurde an deinen Mitspieler übermittelt.} \\
\\ 
\\ 
\texttt{Ihr seid gerade bei Runde 3 von maximal 28.} \\
\texttt{Ihr habt gemeinsam 1 von maximal 16 Strafpunkten angesammelt.} \\
\texttt{Dein Mitspieler hat dir diese Nachricht geschickt:} \\
\\ 
\\ 
\texttt{"Ich verstehe, dass du deine Objekte in der Mitte des Rasters anordnen wirst. Ich werde dasselbe tun. Da unser Raster 7 Spalten und 7 Zeilen hat, würde ich vorschlagen, die Objekte in der 4. Zeile und den Spalten 2{-}8 zu platzieren, um sie möglichst zentral zu halten. Ich werde beginnen, meine Objekte entsprechend zu platzieren. Bitte teile mir mit, wenn duReady bist, damit wir synchron weitermachen können."} \\
\\ 
\\ 
\texttt{Wie lautet dein nächster Befehl?} \\
            }
        }
    }
    & \\ \\

    \theutterance \stepcounter{utterance}  
    & & & \multicolumn{4}{p{0.6\linewidth}}{
        \cellcolor[rgb]{0.9,0.9,0.9}{
            \makecell[{{p{\linewidth}}}]{
                \texttt{\tiny{[GM$\langle$P2]}}
                \texttt{SPRICH: Okay, ich verstehe. Ich werde meine Objekte in der 4. Zeile und den Spalten 2{-}8 platzieren, alphabetisch geordnet: A, D, M, O, P, S, T. Ich bin bereit, loszulegen. Bitte gib mir ein Zeichen, wenn du bereit bist, damit wir synchron weitermachen können.} \\
            }
        }
    }
     \\ \\

    \theutterance \stepcounter{utterance}  
    & & & \multicolumn{2}{p{0.3\linewidth}}{
        \cellcolor[rgb]{0.9,0.9,0.9}{
            \makecell[{{p{\linewidth}}}]{
                \texttt{\tiny{[GM$|$GM]}}
                \texttt{SPRICH: Okay, ich verstehe. Ich werde meine Objekte in der 4. Zeile und den Spalten 2{-}8 platzieren, alphabetisch geordnet: A, D, M, O, P, S, T. Ich bin bereit, loszulegen. Bitte gib mir ein Zeichen, wenn du bereit bist, damit wir synchron weitermachen können.} \\
            }
        }
    }
    & & \\ \\

    \theutterance \stepcounter{utterance}  
    & & \multicolumn{4}{p{0.6\linewidth}}{
        \cellcolor[rgb]{0.9,0.9,0.9}{
            \makecell[{{p{\linewidth}}}]{
                \texttt{\tiny{[P1$\langle$GM]}}
                \texttt{Deine Nachricht wurde an deinen Mitspieler übermittelt.} \\
\\ 
\\ 
\texttt{Ihr seid gerade bei Runde 3 von maximal 28.} \\
\texttt{Ihr habt gemeinsam 1 von maximal 16 Strafpunkten angesammelt.} \\
\texttt{Dein Mitspieler hat dir diese Nachricht geschickt:} \\
\\ 
\\ 
\texttt{"Okay, ich verstehe. Ich werde meine Objekte in der 4. Zeile und den Spalten 2{-}8 platzieren, alphabetisch geordnet: A, D, M, O, P, S, T. Ich bin bereit, loszulegen. Bitte gib mir ein Zeichen, wenn du bereit bist, damit wir synchron weitermachen können."} \\
\\ 
\\ 
\texttt{Wie lautet dein nächster Befehl?} \\
            }
        }
    }
    & \\ \\

    \theutterance \stepcounter{utterance}  
    & \multicolumn{4}{p{0.6\linewidth}}{
        \cellcolor[rgb]{0.9,0.9,0.9}{
            \makecell[{{p{\linewidth}}}]{
                \texttt{\tiny{[P1$\rangle$GM]}}
                \texttt{SPRICH: Ich bin bereit! Lass uns loslegen. Ich werde jetzt mein Objekt 'A' an die Position (2, 4) bewegen. Bitte mach das Gleiche auf deinem Raster. Ich werde dann warten, bis du bestätigst, dass du deine Änderung vorgenommen hast, bevor ich zum nächsten Objekt übergehe.} \\
            }
        }
    }
    & & \\ \\

    \theutterance \stepcounter{utterance}  
    & & & \multicolumn{2}{p{0.3\linewidth}}{
        \cellcolor[rgb]{0.9,0.9,0.9}{
            \makecell[{{p{\linewidth}}}]{
                \texttt{\tiny{[GM$|$GM]}}
                \texttt{SPRICH: Ich bin bereit! Lass uns loslegen. Ich werde jetzt mein Objekt 'A' an die Position (2, 4) bewegen. Bitte mach das Gleiche auf deinem Raster. Ich werde dann warten, bis du bestätigst, dass du deine Änderung vorgenommen hast, bevor ich zum nächsten Objekt übergehe.} \\
            }
        }
    }
    & & \\ \\

    \theutterance \stepcounter{utterance}  
    & & \multicolumn{4}{p{0.6\linewidth}}{
        \cellcolor[rgb]{0.9,0.9,0.9}{
            \makecell[{{p{\linewidth}}}]{
                \texttt{\tiny{[GM$\rangle$P2]}}
                \texttt{Deine Nachricht wurde an deinen Mitspieler übermittelt.} \\
\\ 
\\ 
\texttt{Ihr seid gerade bei Runde 4 von maximal 28.} \\
\texttt{Ihr habt gemeinsam 1 von maximal 16 Strafpunkten angesammelt.} \\
\texttt{Dein Mitspieler hat dir diese Nachricht geschickt:} \\
\\ 
\\ 
\texttt{"Ich bin bereit! Lass uns loslegen. Ich werde jetzt mein Objekt 'A' an die Position (2, 4) bewegen. Bitte mach das Gleiche auf deinem Raster. Ich werde dann warten, bis du bestätigst, dass du deine Änderung vorgenommen hast, bevor ich zum nächsten Objekt übergehe."} \\
\\ 
\\ 
\texttt{Wie lautet dein nächster Befehl?} \\
            }
        }
    }
    & \\ \\

    \theutterance \stepcounter{utterance}  
    & & & \multicolumn{4}{p{0.6\linewidth}}{
        \cellcolor[rgb]{0.9,0.9,0.9}{
            \makecell[{{p{\linewidth}}}]{
                \texttt{\tiny{[GM$\langle$P2]}}
                \texttt{VERSCHIEBE: A, (2, 4)} \\
            }
        }
    }
     \\ \\

    \theutterance \stepcounter{utterance}  
    & & & \multicolumn{2}{p{0.3\linewidth}}{
        \cellcolor[rgb]{0.9,0.9,0.9}{
            \makecell[{{p{\linewidth}}}]{
                \texttt{\tiny{[GM$|$GM]}}
                \texttt{VERSCHIEBE: A, (2, 4)} \\
            }
        }
    }
    & & \\ \\

    \theutterance \stepcounter{utterance}  
    & & & \multicolumn{2}{p{0.3\linewidth}}{
        \cellcolor[rgb]{0.9,0.9,0.9}{
            \makecell[{{p{\linewidth}}}]{
                \texttt{\tiny{[GM$|$GM]}}
                \texttt{Strafpunkt: (2,4) ist nicht leer, sondern enthält '─'.} \\
\\ 
\texttt{Ihr habt gemeinsam 2 von maximal 16 Strafpunkten angesammelt.} \\
\texttt{Bitte versuche es noch mal!} \\
            }
        }
    }
    & & \\ \\

    \theutterance \stepcounter{utterance}  
    & & \multicolumn{4}{p{0.6\linewidth}}{
        \cellcolor[rgb]{0.9,0.9,0.9}{
            \makecell[{{p{\linewidth}}}]{
                \texttt{\tiny{[GM$\rangle$P2]}}
                \texttt{Strafpunkt: (2,4) ist nicht leer, sondern enthält '─'.} \\
\\ 
\texttt{Ihr habt gemeinsam 2 von maximal 16 Strafpunkten angesammelt.} \\
\texttt{Bitte versuche es noch mal!} \\
            }
        }
    }
    & \\ \\

    \theutterance \stepcounter{utterance}  
    & & & \multicolumn{4}{p{0.6\linewidth}}{
        \cellcolor[rgb]{0.9,0.9,0.9}{
            \makecell[{{p{\linewidth}}}]{
                \texttt{\tiny{[GM$\langle$P2]}}
                \texttt{VERSCHIEBE: A, (3, 4)} \\
            }
        }
    }
     \\ \\

    \theutterance \stepcounter{utterance}  
    & & & \multicolumn{2}{p{0.3\linewidth}}{
        \cellcolor[rgb]{0.9,0.9,0.9}{
            \makecell[{{p{\linewidth}}}]{
                \texttt{\tiny{[GM$|$GM]}}
                \texttt{VERSCHIEBE: A, (3, 4)} \\
            }
        }
    }
    & & \\ \\

    \theutterance \stepcounter{utterance}  
    & & & \multicolumn{2}{p{0.3\linewidth}}{
        \cellcolor[rgb]{0.9,0.9,0.9}{
            \makecell[{{p{\linewidth}}}]{
                \texttt{\tiny{[GM$|$GM]}}
                \texttt{Strafpunkt: (3,4) ist nicht leer, sondern enthält '─'.} \\
\\ 
\texttt{Ihr habt gemeinsam 3 von maximal 16 Strafpunkten angesammelt.} \\
\texttt{Bitte versuche es noch mal!} \\
            }
        }
    }
    & & \\ \\

    \theutterance \stepcounter{utterance}  
    & & \multicolumn{4}{p{0.6\linewidth}}{
        \cellcolor[rgb]{0.9,0.9,0.9}{
            \makecell[{{p{\linewidth}}}]{
                \texttt{\tiny{[GM$\rangle$P2]}}
                \texttt{Strafpunkt: (3,4) ist nicht leer, sondern enthält '─'.} \\
\\ 
\texttt{Ihr habt gemeinsam 3 von maximal 16 Strafpunkten angesammelt.} \\
\texttt{Bitte versuche es noch mal!} \\
            }
        }
    }
    & \\ \\

    \theutterance \stepcounter{utterance}  
    & & & \multicolumn{4}{p{0.6\linewidth}}{
        \cellcolor[rgb]{0.9,0.9,0.9}{
            \makecell[{{p{\linewidth}}}]{
                \texttt{\tiny{[GM$\langle$P2]}}
                \texttt{VERSCHIEBE: A, (2, 5)} \\
            }
        }
    }
     \\ \\

    \theutterance \stepcounter{utterance}  
    & & & \multicolumn{2}{p{0.3\linewidth}}{
        \cellcolor[rgb]{0.9,0.9,0.9}{
            \makecell[{{p{\linewidth}}}]{
                \texttt{\tiny{[GM$|$GM]}}
                \texttt{VERSCHIEBE: A, (2, 5)} \\
            }
        }
    }
    & & \\ \\

    \theutterance \stepcounter{utterance}  
    & & & \multicolumn{2}{p{0.3\linewidth}}{
        \cellcolor[rgb]{0.9,0.9,0.9}{
            \makecell[{{p{\linewidth}}}]{
                \texttt{\tiny{[GM$|$GM]}}
                \texttt{'A' erfolgreich nach (2,5) verschoben. Dein Raster sieht jetzt so aus:} \\
\texttt{```} \\
\texttt{ 1234567} \\
\texttt{╔═╤═╤═╤═╗} \\
\texttt{║◌│◌│◌│◌║ 1} \\
\texttt{╟─┴─┼─┤◌║ 2} \\
\texttt{║◌M◌│◌└─╢ 3} \\
\texttt{╟───┘P◌◌║ 4} \\
\texttt{║OA◌◌◌┌─╢ 5} \\
\texttt{║◌D◌◌◌│S║ 6} \\
\texttt{║◌◌◌◌T│◌║ 7} \\
\texttt{╚═════╧═╝} \\
\texttt{```} \\
            }
        }
    }
    & & \\ \\

    \theutterance \stepcounter{utterance}  
    & & \multicolumn{4}{p{0.6\linewidth}}{
        \cellcolor[rgb]{0.9,0.9,0.9}{
            \makecell[{{p{\linewidth}}}]{
                \texttt{\tiny{[P1$\langle$GM]}}
                \texttt{Deine Nachricht wurde an deinen Mitspieler übermittelt.} \\
\\ 
\\ 
\texttt{Ihr seid gerade bei Runde 4 von maximal 28.} \\
\texttt{Ihr habt gemeinsam 3 von maximal 16 Strafpunkten angesammelt.} \\
\texttt{Dein Mitspieler hat ein Objekt auf seinem Raster bewegt.} \\
\\ 
\\ 
\texttt{Wie lautet dein nächster Befehl?} \\
            }
        }
    }
    & \\ \\

    \theutterance \stepcounter{utterance}  
    & \multicolumn{4}{p{0.6\linewidth}}{
        \cellcolor[rgb]{0.9,0.9,0.9}{
            \makecell[{{p{\linewidth}}}]{
                \texttt{\tiny{[P1$\rangle$GM]}}
                \texttt{VERSCHIEBE: A, (2, 4)} \\
            }
        }
    }
    & & \\ \\

    \theutterance \stepcounter{utterance}  
    & & & \multicolumn{2}{p{0.3\linewidth}}{
        \cellcolor[rgb]{0.9,0.9,0.9}{
            \makecell[{{p{\linewidth}}}]{
                \texttt{\tiny{[GM$|$GM]}}
                \texttt{VERSCHIEBE: A, (2, 4)} \\
            }
        }
    }
    & & \\ \\

    \theutterance \stepcounter{utterance}  
    & & & \multicolumn{2}{p{0.3\linewidth}}{
        \cellcolor[rgb]{0.9,0.9,0.9}{
            \makecell[{{p{\linewidth}}}]{
                \texttt{\tiny{[GM$|$GM]}}
                \texttt{Strafpunkt: (2,4) ist nicht leer, sondern enthält '─'.} \\
\\ 
\texttt{Ihr habt gemeinsam 4 von maximal 16 Strafpunkten angesammelt.} \\
\texttt{Bitte versuche es noch mal!} \\
            }
        }
    }
    & & \\ \\

    \theutterance \stepcounter{utterance}  
    & & \multicolumn{4}{p{0.6\linewidth}}{
        \cellcolor[rgb]{0.9,0.9,0.9}{
            \makecell[{{p{\linewidth}}}]{
                \texttt{\tiny{[P1$\langle$GM]}}
                \texttt{Strafpunkt: (2,4) ist nicht leer, sondern enthält '─'.} \\
\\ 
\texttt{Ihr habt gemeinsam 4 von maximal 16 Strafpunkten angesammelt.} \\
\texttt{Bitte versuche es noch mal!} \\
            }
        }
    }
    & \\ \\

    \theutterance \stepcounter{utterance}  
    & \multicolumn{4}{p{0.6\linewidth}}{
        \cellcolor[rgb]{0.9,0.9,0.9}{
            \makecell[{{p{\linewidth}}}]{
                \texttt{\tiny{[P1$\rangle$GM]}}
                \texttt{VERSCHIEBE: A, (3, 4)} \\
            }
        }
    }
    & & \\ \\

    \theutterance \stepcounter{utterance}  
    & & & \multicolumn{2}{p{0.3\linewidth}}{
        \cellcolor[rgb]{0.9,0.9,0.9}{
            \makecell[{{p{\linewidth}}}]{
                \texttt{\tiny{[GM$|$GM]}}
                \texttt{VERSCHIEBE: A, (3, 4)} \\
            }
        }
    }
    & & \\ \\

    \theutterance \stepcounter{utterance}  
    & & & \multicolumn{2}{p{0.3\linewidth}}{
        \cellcolor[rgb]{0.9,0.9,0.9}{
            \makecell[{{p{\linewidth}}}]{
                \texttt{\tiny{[GM$|$GM]}}
                \texttt{Strafpunkt: (3,4) ist nicht leer, sondern enthält '─'.} \\
\\ 
\texttt{Ihr habt gemeinsam 5 von maximal 16 Strafpunkten angesammelt.} \\
\texttt{Bitte versuche es noch mal!} \\
            }
        }
    }
    & & \\ \\

    \theutterance \stepcounter{utterance}  
    & & \multicolumn{4}{p{0.6\linewidth}}{
        \cellcolor[rgb]{0.9,0.9,0.9}{
            \makecell[{{p{\linewidth}}}]{
                \texttt{\tiny{[P1$\langle$GM]}}
                \texttt{Strafpunkt: (3,4) ist nicht leer, sondern enthält '─'.} \\
\\ 
\texttt{Ihr habt gemeinsam 5 von maximal 16 Strafpunkten angesammelt.} \\
\texttt{Bitte versuche es noch mal!} \\
            }
        }
    }
    & \\ \\

    \theutterance \stepcounter{utterance}  
    & \multicolumn{4}{p{0.6\linewidth}}{
        \cellcolor[rgb]{0.9,0.9,0.9}{
            \makecell[{{p{\linewidth}}}]{
                \texttt{\tiny{[P1$\rangle$GM]}}
                \texttt{SPRICH: Es sieht so aus, als ob unsere Raster sich unterscheiden. Ich denke, wir sollten unsere Strategie anpassen. Ich schlage vor, dass wir unsere Objekte anhand ihrer aktuellen Positionen auf dem Raster ausrichten, anstatt eine feste Position zu wählen. Kannst du mir bitte mitteilen, wo sich jedes deiner Objekte gerade befindet? Ich werde dasselbe tun, und wir können dann unsere Strategie neu planen.} \\
            }
        }
    }
    & & \\ \\

    \theutterance \stepcounter{utterance}  
    & & & \multicolumn{2}{p{0.3\linewidth}}{
        \cellcolor[rgb]{0.9,0.9,0.9}{
            \makecell[{{p{\linewidth}}}]{
                \texttt{\tiny{[GM$|$GM]}}
                \texttt{SPRICH: Es sieht so aus, als ob unsere Raster sich unterscheiden. Ich denke, wir sollten unsere Strategie anpassen. Ich schlage vor, dass wir unsere Objekte anhand ihrer aktuellen Positionen auf dem Raster ausrichten, anstatt eine feste Position zu wählen. Kannst du mir bitte mitteilen, wo sich jedes deiner Objekte gerade befindet? Ich werde dasselbe tun, und wir können dann unsere Strategie neu planen.} \\
            }
        }
    }
    & & \\ \\

    \theutterance \stepcounter{utterance}  
    & & \multicolumn{4}{p{0.6\linewidth}}{
        \cellcolor[rgb]{0.9,0.9,0.9}{
            \makecell[{{p{\linewidth}}}]{
                \texttt{\tiny{[GM$\rangle$P2]}}
                \texttt{'A' erfolgreich nach (2,5) verschoben. Dein Raster sieht jetzt so aus:} \\
\texttt{```} \\
\texttt{ 1234567} \\
\texttt{╔═╤═╤═╤═╗} \\
\texttt{║◌│◌│◌│◌║ 1} \\
\texttt{╟─┴─┼─┤◌║ 2} \\
\texttt{║◌M◌│◌└─╢ 3} \\
\texttt{╟───┘P◌◌║ 4} \\
\texttt{║OA◌◌◌┌─╢ 5} \\
\texttt{║◌D◌◌◌│S║ 6} \\
\texttt{║◌◌◌◌T│◌║ 7} \\
\texttt{╚═════╧═╝} \\
\texttt{```} \\
\texttt{Ihr seid gerade bei Runde 5 von maximal 28.} \\
\texttt{Ihr habt gemeinsam 5 von maximal 16 Strafpunkten angesammelt.} \\
\texttt{Dein Mitspieler hat dir diese Nachricht geschickt:} \\
\\ 
\\ 
\texttt{"Es sieht so aus, als ob unsere Raster sich unterscheiden. Ich denke, wir sollten unsere Strategie anpassen. Ich schlage vor, dass wir unsere Objekte anhand ihrer aktuellen Positionen auf dem Raster ausrichten, anstatt eine feste Position zu wählen. Kannst du mir bitte mitteilen, wo sich jedes deiner Objekte gerade befindet? Ich werde dasselbe tun, und wir können dann unsere Strategie neu planen."} \\
\\ 
\\ 
\texttt{Wie lautet dein nächster Befehl?} \\
            }
        }
    }
    & \\ \\

    \theutterance \stepcounter{utterance}  
    & & & \multicolumn{4}{p{0.6\linewidth}}{
        \cellcolor[rgb]{0.9,0.9,0.9}{
            \makecell[{{p{\linewidth}}}]{
                \texttt{\tiny{[GM$\langle$P2]}}
                \texttt{SPRICH: Mein Objekt 'A' ist jetzt an Position (2,5), 'D' ist an (6,2), 'M' ist an (3,3), 'O' ist an (1,5), 'P' ist an (4,4), 'S' ist an (7,6) und 'T' ist an (7,7). Bitte teile mir die Positionen deiner Objekte mit.} \\
            }
        }
    }
     \\ \\

    \theutterance \stepcounter{utterance}  
    & & & \multicolumn{2}{p{0.3\linewidth}}{
        \cellcolor[rgb]{0.9,0.9,0.9}{
            \makecell[{{p{\linewidth}}}]{
                \texttt{\tiny{[GM$|$GM]}}
                \texttt{SPRICH: Mein Objekt 'A' ist jetzt an Position (2,5), 'D' ist an (6,2), 'M' ist an (3,3), 'O' ist an (1,5), 'P' ist an (4,4), 'S' ist an (7,6) und 'T' ist an (7,7). Bitte teile mir die Positionen deiner Objekte mit.} \\
            }
        }
    }
    & & \\ \\

    \theutterance \stepcounter{utterance}  
    & & \multicolumn{4}{p{0.6\linewidth}}{
        \cellcolor[rgb]{0.9,0.9,0.9}{
            \makecell[{{p{\linewidth}}}]{
                \texttt{\tiny{[P1$\langle$GM]}}
                \texttt{Deine Nachricht wurde an deinen Mitspieler übermittelt.} \\
\\ 
\\ 
\texttt{Ihr seid gerade bei Runde 5 von maximal 28.} \\
\texttt{Ihr habt gemeinsam 5 von maximal 16 Strafpunkten angesammelt.} \\
\texttt{Dein Mitspieler hat dir diese Nachricht geschickt:} \\
\\ 
\\ 
\texttt{"Mein Objekt 'A' ist jetzt an Position (2,5), 'D' ist an (6,2), 'M' ist an (3,3), 'O' ist an (1,5), 'P' ist an (4,4), 'S' ist an (7,6) und 'T' ist an (7,7). Bitte teile mir die Positionen deiner Objekte mit."} \\
\\ 
\\ 
\texttt{Wie lautet dein nächster Befehl?} \\
            }
        }
    }
    & \\ \\

    \theutterance \stepcounter{utterance}  
    & \multicolumn{4}{p{0.6\linewidth}}{
        \cellcolor[rgb]{0.9,0.9,0.9}{
            \makecell[{{p{\linewidth}}}]{
                \texttt{\tiny{[P1$\rangle$GM]}}
                \texttt{SPRICH: Danke für die Informationen! Mein Objekt 'A' ist an Position (1,1), 'D' ist an (7,7), 'M' ist an (3,3), 'O' ist an (6,6), 'P' ist an (6,6) neben 'O', 'S' ist an (4,4) und 'T' ist an (6,6) neben 'O' und 'P'. Jetzt, da wir die Positionen unserer Objekte kennen, können wir beginnen, sie zu manipulieren, um unsere Ziele zu erreichen. Ich denke, wir sollten ein Objekt nach dem anderen bewegen, um unsere Anordnungen zu synchronisieren. Lass uns mit 'A' beginnen. Ich schlage vor, dass wir 'A' an die Position (2,5) bewegen, wie bei dir. Kannst du bitte bestätigen, ob das okay für dich ist?} \\
            }
        }
    }
    & & \\ \\

    \theutterance \stepcounter{utterance}  
    & & & \multicolumn{2}{p{0.3\linewidth}}{
        \cellcolor[rgb]{0.9,0.9,0.9}{
            \makecell[{{p{\linewidth}}}]{
                \texttt{\tiny{[GM$|$GM]}}
                \texttt{SPRICH: Danke für die Informationen! Mein Objekt 'A' ist an Position (1,1), 'D' ist an (7,7), 'M' ist an (3,3), 'O' ist an (6,6), 'P' ist an (6,6) neben 'O', 'S' ist an (4,4) und 'T' ist an (6,6) neben 'O' und 'P'. Jetzt, da wir die Positionen unserer Objekte kennen, können wir beginnen, sie zu manipulieren, um unsere Ziele zu erreichen. Ich denke, wir sollten ein Objekt nach dem anderen bewegen, um unsere Anordnungen zu synchronisieren. Lass uns mit 'A' beginnen. Ich schlage vor, dass wir 'A' an die Position (2,5) bewegen, wie bei dir. Kannst du bitte bestätigen, ob das okay für dich ist?} \\
            }
        }
    }
    & & \\ \\

    \theutterance \stepcounter{utterance}  
    & & \multicolumn{4}{p{0.6\linewidth}}{
        \cellcolor[rgb]{0.9,0.9,0.9}{
            \makecell[{{p{\linewidth}}}]{
                \texttt{\tiny{[GM$\rangle$P2]}}
                \texttt{Deine Nachricht wurde an deinen Mitspieler übermittelt.} \\
\\ 
\\ 
\texttt{Ihr seid gerade bei Runde 6 von maximal 28.} \\
\texttt{Ihr habt gemeinsam 5 von maximal 16 Strafpunkten angesammelt.} \\
\texttt{Dein Mitspieler hat dir diese Nachricht geschickt:} \\
\\ 
\\ 
\texttt{"Danke für die Informationen! Mein Objekt 'A' ist an Position (1,1), 'D' ist an (7,7), 'M' ist an (3,3), 'O' ist an (6,6), 'P' ist an (6,6) neben 'O', 'S' ist an (4,4) und 'T' ist an (6,6) neben 'O' und 'P'. Jetzt, da wir die Positionen unserer Objekte kennen, können wir beginnen, sie zu manipulieren, um unsere Ziele zu erreichen. Ich denke, wir sollten ein Objekt nach dem anderen bewegen, um unsere Anordnungen zu synchronisieren. Lass uns mit 'A' beginnen. Ich schlage vor, dass wir 'A' an die Position (2,5) bewegen, wie bei dir. Kannst du bitte bestätigen, ob das okay für dich ist?"} \\
\\ 
\\ 
\texttt{Wie lautet dein nächster Befehl?} \\
            }
        }
    }
    & \\ \\

    \theutterance \stepcounter{utterance}  
    & & & \multicolumn{4}{p{0.6\linewidth}}{
        \cellcolor[rgb]{0.9,0.9,0.9}{
            \makecell[{{p{\linewidth}}}]{
                \texttt{\tiny{[GM$\langle$P2]}}
                \texttt{SPRICH: Ja, das ist okay für mich. Bitte bewege dein 'A' an die Position (2,5), wie bei mir. Ich werde warten, bis du bestätigst, dass du die Änderung vorgenommen hast, bevor ich zum nächsten Schritt übergehe.} \\
            }
        }
    }
     \\ \\

    \theutterance \stepcounter{utterance}  
    & & & \multicolumn{2}{p{0.3\linewidth}}{
        \cellcolor[rgb]{0.9,0.9,0.9}{
            \makecell[{{p{\linewidth}}}]{
                \texttt{\tiny{[GM$|$GM]}}
                \texttt{SPRICH: Ja, das ist okay für mich. Bitte bewege dein 'A' an die Position (2,5), wie bei mir. Ich werde warten, bis du bestätigst, dass du die Änderung vorgenommen hast, bevor ich zum nächsten Schritt übergehe.} \\
            }
        }
    }
    & & \\ \\

    \theutterance \stepcounter{utterance}  
    & & \multicolumn{4}{p{0.6\linewidth}}{
        \cellcolor[rgb]{0.9,0.9,0.9}{
            \makecell[{{p{\linewidth}}}]{
                \texttt{\tiny{[P1$\langle$GM]}}
                \texttt{Deine Nachricht wurde an deinen Mitspieler übermittelt.} \\
\\ 
\\ 
\texttt{Ihr seid gerade bei Runde 6 von maximal 28.} \\
\texttt{Ihr habt gemeinsam 5 von maximal 16 Strafpunkten angesammelt.} \\
\texttt{Dein Mitspieler hat dir diese Nachricht geschickt:} \\
\\ 
\\ 
\texttt{"Ja, das ist okay für mich. Bitte bewege dein 'A' an die Position (2,5), wie bei mir. Ich werde warten, bis du bestätigst, dass du die Änderung vorgenommen hast, bevor ich zum nächsten Schritt übergehe."} \\
\\ 
\\ 
\texttt{Wie lautet dein nächster Befehl?} \\
            }
        }
    }
    & \\ \\

    \theutterance \stepcounter{utterance}  
    & \multicolumn{4}{p{0.6\linewidth}}{
        \cellcolor[rgb]{0.9,0.9,0.9}{
            \makecell[{{p{\linewidth}}}]{
                \texttt{\tiny{[P1$\rangle$GM]}}
                \texttt{VERSCHIEBE: A, (2, 5)} \\
            }
        }
    }
    & & \\ \\

    \theutterance \stepcounter{utterance}  
    & & & \multicolumn{2}{p{0.3\linewidth}}{
        \cellcolor[rgb]{0.9,0.9,0.9}{
            \makecell[{{p{\linewidth}}}]{
                \texttt{\tiny{[GM$|$GM]}}
                \texttt{VERSCHIEBE: A, (2, 5)} \\
            }
        }
    }
    & & \\ \\

    \theutterance \stepcounter{utterance}  
    & & & \multicolumn{2}{p{0.3\linewidth}}{
        \cellcolor[rgb]{0.9,0.9,0.9}{
            \makecell[{{p{\linewidth}}}]{
                \texttt{\tiny{[GM$|$GM]}}
                \texttt{'A' erfolgreich nach (2,5) verschoben. Dein Raster sieht jetzt so aus:} \\
\texttt{```} \\
\texttt{ 1234567} \\
\texttt{╔═╤═╤═╤═╗} \\
\texttt{║◌│◌│◌│◌║ 1} \\
\texttt{╟─┴─┼─┤◌║ 2} \\
\texttt{║◌◌M│◌└─╢ 3} \\
\texttt{╟───┘◌S◌║ 4} \\
\texttt{║◌A◌◌◌┌─╢ 5} \\
\texttt{║◌◌◌PT│O║ 6} \\
\texttt{║◌D◌◌◌│◌║ 7} \\
\texttt{╚═════╧═╝} \\
\texttt{```} \\
            }
        }
    }
    & & \\ \\

    \theutterance \stepcounter{utterance}  
    & & \multicolumn{4}{p{0.6\linewidth}}{
        \cellcolor[rgb]{0.9,0.9,0.9}{
            \makecell[{{p{\linewidth}}}]{
                \texttt{\tiny{[GM$\rangle$P2]}}
                \texttt{Deine Nachricht wurde an deinen Mitspieler übermittelt.} \\
\\ 
\\ 
\texttt{Ihr seid gerade bei Runde 7 von maximal 28.} \\
\texttt{Ihr habt gemeinsam 5 von maximal 16 Strafpunkten angesammelt.} \\
\texttt{Dein Mitspieler hat ein Objekt auf seinem Raster bewegt.} \\
\\ 
\\ 
\texttt{Wie lautet dein nächster Befehl?} \\
            }
        }
    }
    & \\ \\

    \theutterance \stepcounter{utterance}  
    & & & \multicolumn{4}{p{0.6\linewidth}}{
        \cellcolor[rgb]{0.9,0.9,0.9}{
            \makecell[{{p{\linewidth}}}]{
                \texttt{\tiny{[GM$\langle$P2]}}
                \texttt{SPRICH: Hast du dein 'A' erfolgreich an die Position (2,5) bewegt? Bitte bestätige, damit wir zum nächsten Schritt übergehen können.} \\
            }
        }
    }
     \\ \\

    \theutterance \stepcounter{utterance}  
    & & & \multicolumn{2}{p{0.3\linewidth}}{
        \cellcolor[rgb]{0.9,0.9,0.9}{
            \makecell[{{p{\linewidth}}}]{
                \texttt{\tiny{[GM$|$GM]}}
                \texttt{SPRICH: Hast du dein 'A' erfolgreich an die Position (2,5) bewegt? Bitte bestätige, damit wir zum nächsten Schritt übergehen können.} \\
            }
        }
    }
    & & \\ \\

    \theutterance \stepcounter{utterance}  
    & & \multicolumn{4}{p{0.6\linewidth}}{
        \cellcolor[rgb]{0.9,0.9,0.9}{
            \makecell[{{p{\linewidth}}}]{
                \texttt{\tiny{[P1$\langle$GM]}}
                \texttt{'A' erfolgreich nach (2,5) verschoben. Dein Raster sieht jetzt so aus:} \\
\texttt{```} \\
\texttt{ 1234567} \\
\texttt{╔═╤═╤═╤═╗} \\
\texttt{║◌│◌│◌│◌║ 1} \\
\texttt{╟─┴─┼─┤◌║ 2} \\
\texttt{║◌◌M│◌└─╢ 3} \\
\texttt{╟───┘◌S◌║ 4} \\
\texttt{║◌A◌◌◌┌─╢ 5} \\
\texttt{║◌◌◌PT│O║ 6} \\
\texttt{║◌D◌◌◌│◌║ 7} \\
\texttt{╚═════╧═╝} \\
\texttt{```} \\
\texttt{Ihr seid gerade bei Runde 7 von maximal 28.} \\
\texttt{Ihr habt gemeinsam 5 von maximal 16 Strafpunkten angesammelt.} \\
\texttt{Dein Mitspieler hat dir diese Nachricht geschickt:} \\
\\ 
\\ 
\texttt{"Hast du dein 'A' erfolgreich an die Position (2,5) bewegt? Bitte bestätige, damit wir zum nächsten Schritt übergehen können."} \\
\\ 
\\ 
\texttt{Wie lautet dein nächster Befehl?} \\
            }
        }
    }
    & \\ \\

    \theutterance \stepcounter{utterance}  
    & \multicolumn{4}{p{0.6\linewidth}}{
        \cellcolor[rgb]{0.9,0.9,0.9}{
            \makecell[{{p{\linewidth}}}]{
                \texttt{\tiny{[P1$\rangle$GM]}}
                \texttt{SPRICH: Ja, ich habe mein 'A' erfolgreich an die Position (2,5) bewegt. Mein Raster sieht jetzt so aus, dass 'A' an der richtigen Stelle ist. Bitte geh zum nächsten Schritt über. Welches Objekt sollen wir als Nächstes bewegen?} \\
            }
        }
    }
    & & \\ \\

    \theutterance \stepcounter{utterance}  
    & & & \multicolumn{2}{p{0.3\linewidth}}{
        \cellcolor[rgb]{0.9,0.9,0.9}{
            \makecell[{{p{\linewidth}}}]{
                \texttt{\tiny{[GM$|$GM]}}
                \texttt{SPRICH: Ja, ich habe mein 'A' erfolgreich an die Position (2,5) bewegt. Mein Raster sieht jetzt so aus, dass 'A' an der richtigen Stelle ist. Bitte geh zum nächsten Schritt über. Welches Objekt sollen wir als Nächstes bewegen?} \\
            }
        }
    }
    & & \\ \\

    \theutterance \stepcounter{utterance}  
    & & \multicolumn{4}{p{0.6\linewidth}}{
        \cellcolor[rgb]{0.9,0.9,0.9}{
            \makecell[{{p{\linewidth}}}]{
                \texttt{\tiny{[GM$\rangle$P2]}}
                \texttt{Deine Nachricht wurde an deinen Mitspieler übermittelt.} \\
\\ 
\\ 
\texttt{Ihr seid gerade bei Runde 8 von maximal 28.} \\
\texttt{Ihr habt gemeinsam 5 von maximal 16 Strafpunkten angesammelt.} \\
\texttt{Dein Mitspieler hat dir diese Nachricht geschickt:} \\
\\ 
\\ 
\texttt{"Ja, ich habe mein 'A' erfolgreich an die Position (2,5) bewegt. Mein Raster sieht jetzt so aus, dass 'A' an der richtigen Stelle ist. Bitte geh zum nächsten Schritt über. Welches Objekt sollen wir als Nächstes bewegen?"} \\
\\ 
\\ 
\texttt{Wie lautet dein nächster Befehl?} \\
            }
        }
    }
    & \\ \\

    \theutterance \stepcounter{utterance}  
    & & & \multicolumn{4}{p{0.6\linewidth}}{
        \cellcolor[rgb]{0.9,0.9,0.9}{
            \makecell[{{p{\linewidth}}}]{
                \texttt{\tiny{[GM$\langle$P2]}}
                \texttt{SPRICH: Lass uns 'D' als Nächstes bewegen. Wir sollten es an eine Position bringen, die es uns ermöglicht, unsere Anordnungen zu synchronisieren. Ich schlage vor, dass wir 'D' an die Position (6,2) bewegen, wie bei mir. Was meinst du?} \\
            }
        }
    }
     \\ \\

    \theutterance \stepcounter{utterance}  
    & & & \multicolumn{2}{p{0.3\linewidth}}{
        \cellcolor[rgb]{0.9,0.9,0.9}{
            \makecell[{{p{\linewidth}}}]{
                \texttt{\tiny{[GM$|$GM]}}
                \texttt{SPRICH: Lass uns 'D' als Nächstes bewegen. Wir sollten es an eine Position bringen, die es uns ermöglicht, unsere Anordnungen zu synchronisieren. Ich schlage vor, dass wir 'D' an die Position (6,2) bewegen, wie bei mir. Was meinst du?} \\
            }
        }
    }
    & & \\ \\

    \theutterance \stepcounter{utterance}  
    & & \multicolumn{4}{p{0.6\linewidth}}{
        \cellcolor[rgb]{0.9,0.9,0.9}{
            \makecell[{{p{\linewidth}}}]{
                \texttt{\tiny{[P1$\langle$GM]}}
                \texttt{Deine Nachricht wurde an deinen Mitspieler übermittelt.} \\
\\ 
\\ 
\texttt{Ihr seid gerade bei Runde 8 von maximal 28.} \\
\texttt{Ihr habt gemeinsam 5 von maximal 16 Strafpunkten angesammelt.} \\
\texttt{Dein Mitspieler hat dir diese Nachricht geschickt:} \\
\\ 
\\ 
\texttt{"Lass uns 'D' als Nächstes bewegen. Wir sollten es an eine Position bringen, die es uns ermöglicht, unsere Anordnungen zu synchronisieren. Ich schlage vor, dass wir 'D' an die Position (6,2) bewegen, wie bei mir. Was meinst du?"} \\
\\ 
\\ 
\texttt{Wie lautet dein nächster Befehl?} \\
            }
        }
    }
    & \\ \\

    \theutterance \stepcounter{utterance}  
    & \multicolumn{4}{p{0.6\linewidth}}{
        \cellcolor[rgb]{0.9,0.9,0.9}{
            \makecell[{{p{\linewidth}}}]{
                \texttt{\tiny{[P1$\rangle$GM]}}
                \texttt{SPRICH: Das klingt gut! Ich stimme zu, 'D' an die Position (6,2) zu bewegen, wie bei dir. Bitte warte, bis ich bestätige, dass ich die Änderung vorgenommen habe, bevor du zum nächsten Schritt übergehst. Ich werde jetzt 'D' bewegen.} \\
            }
        }
    }
    & & \\ \\

    \theutterance \stepcounter{utterance}  
    & & & \multicolumn{2}{p{0.3\linewidth}}{
        \cellcolor[rgb]{0.9,0.9,0.9}{
            \makecell[{{p{\linewidth}}}]{
                \texttt{\tiny{[GM$|$GM]}}
                \texttt{SPRICH: Das klingt gut! Ich stimme zu, 'D' an die Position (6,2) zu bewegen, wie bei dir. Bitte warte, bis ich bestätige, dass ich die Änderung vorgenommen habe, bevor du zum nächsten Schritt übergehst. Ich werde jetzt 'D' bewegen.} \\
            }
        }
    }
    & & \\ \\

    \theutterance \stepcounter{utterance}  
    & & \multicolumn{4}{p{0.6\linewidth}}{
        \cellcolor[rgb]{0.9,0.9,0.9}{
            \makecell[{{p{\linewidth}}}]{
                \texttt{\tiny{[GM$\rangle$P2]}}
                \texttt{Deine Nachricht wurde an deinen Mitspieler übermittelt.} \\
\\ 
\\ 
\texttt{Ihr seid gerade bei Runde 9 von maximal 28.} \\
\texttt{Ihr habt gemeinsam 5 von maximal 16 Strafpunkten angesammelt.} \\
\texttt{Dein Mitspieler hat dir diese Nachricht geschickt:} \\
\\ 
\\ 
\texttt{"Das klingt gut! Ich stimme zu, 'D' an die Position (6,2) zu bewegen, wie bei dir. Bitte warte, bis ich bestätige, dass ich die Änderung vorgenommen habe, bevor du zum nächsten Schritt übergehst. Ich werde jetzt 'D' bewegen."} \\
\\ 
\\ 
\texttt{Wie lautet dein nächster Befehl?} \\
            }
        }
    }
    & \\ \\

    \theutterance \stepcounter{utterance}  
    & & & \multicolumn{4}{p{0.6\linewidth}}{
        \cellcolor[rgb]{0.9,0.9,0.9}{
            \makecell[{{p{\linewidth}}}]{
                \texttt{\tiny{[GM$\langle$P2]}}
                \texttt{SPRICH: Okay, ich warte. Bitte teile mir mit, wenn du 'D' erfolgreich an die Position (6,2) bewegt hast.} \\
            }
        }
    }
     \\ \\

    \theutterance \stepcounter{utterance}  
    & & & \multicolumn{2}{p{0.3\linewidth}}{
        \cellcolor[rgb]{0.9,0.9,0.9}{
            \makecell[{{p{\linewidth}}}]{
                \texttt{\tiny{[GM$|$GM]}}
                \texttt{SPRICH: Okay, ich warte. Bitte teile mir mit, wenn du 'D' erfolgreich an die Position (6,2) bewegt hast.} \\
            }
        }
    }
    & & \\ \\

    \theutterance \stepcounter{utterance}  
    & & \multicolumn{4}{p{0.6\linewidth}}{
        \cellcolor[rgb]{0.9,0.9,0.9}{
            \makecell[{{p{\linewidth}}}]{
                \texttt{\tiny{[P1$\langle$GM]}}
                \texttt{Deine Nachricht wurde an deinen Mitspieler übermittelt.} \\
\\ 
\\ 
\texttt{Ihr seid gerade bei Runde 9 von maximal 28.} \\
\texttt{Ihr habt gemeinsam 5 von maximal 16 Strafpunkten angesammelt.} \\
\texttt{Dein Mitspieler hat dir diese Nachricht geschickt:} \\
\\ 
\\ 
\texttt{"Okay, ich warte. Bitte teile mir mit, wenn du 'D' erfolgreich an die Position (6,2) bewegt hast."} \\
\\ 
\\ 
\texttt{Wie lautet dein nächster Befehl?} \\
            }
        }
    }
    & \\ \\

    \theutterance \stepcounter{utterance}  
    & \multicolumn{4}{p{0.6\linewidth}}{
        \cellcolor[rgb]{0.9,0.9,0.9}{
            \makecell[{{p{\linewidth}}}]{
                \texttt{\tiny{[P1$\rangle$GM]}}
                \texttt{VERSCHIEBE: D, (6, 2)} \\
            }
        }
    }
    & & \\ \\

    \theutterance \stepcounter{utterance}  
    & & & \multicolumn{2}{p{0.3\linewidth}}{
        \cellcolor[rgb]{0.9,0.9,0.9}{
            \makecell[{{p{\linewidth}}}]{
                \texttt{\tiny{[GM$|$GM]}}
                \texttt{VERSCHIEBE: D, (6, 2)} \\
            }
        }
    }
    & & \\ \\

    \theutterance \stepcounter{utterance}  
    & & & \multicolumn{2}{p{0.3\linewidth}}{
        \cellcolor[rgb]{0.9,0.9,0.9}{
            \makecell[{{p{\linewidth}}}]{
                \texttt{\tiny{[GM$|$GM]}}
                \texttt{Strafpunkt: (6,2) ist nicht leer, sondern enthält '┤'.} \\
\\ 
\texttt{Ihr habt gemeinsam 6 von maximal 16 Strafpunkten angesammelt.} \\
\texttt{Bitte versuche es noch mal!} \\
            }
        }
    }
    & & \\ \\

    \theutterance \stepcounter{utterance}  
    & & \multicolumn{4}{p{0.6\linewidth}}{
        \cellcolor[rgb]{0.9,0.9,0.9}{
            \makecell[{{p{\linewidth}}}]{
                \texttt{\tiny{[P1$\langle$GM]}}
                \texttt{Strafpunkt: (6,2) ist nicht leer, sondern enthält '┤'.} \\
\\ 
\texttt{Ihr habt gemeinsam 6 von maximal 16 Strafpunkten angesammelt.} \\
\texttt{Bitte versuche es noch mal!} \\
            }
        }
    }
    & \\ \\

    \theutterance \stepcounter{utterance}  
    & \multicolumn{4}{p{0.6\linewidth}}{
        \cellcolor[rgb]{0.9,0.9,0.9}{
            \makecell[{{p{\linewidth}}}]{
                \texttt{\tiny{[P1$\rangle$GM]}}
                \texttt{SPRICH: Es sieht so aus, als ob die Position (6,2) nicht leer ist. Ich denke, wir sollten eine andere Position für 'D' finden. Kannst du mir bitte vorschlagen, wo ich 'D' hinbewegen soll? Ich werde mich bemühen, es an eine passende Stelle zu bewegen.} \\
            }
        }
    }
    & & \\ \\

    \theutterance \stepcounter{utterance}  
    & & & \multicolumn{2}{p{0.3\linewidth}}{
        \cellcolor[rgb]{0.9,0.9,0.9}{
            \makecell[{{p{\linewidth}}}]{
                \texttt{\tiny{[GM$|$GM]}}
                \texttt{SPRICH: Es sieht so aus, als ob die Position (6,2) nicht leer ist. Ich denke, wir sollten eine andere Position für 'D' finden. Kannst du mir bitte vorschlagen, wo ich 'D' hinbewegen soll? Ich werde mich bemühen, es an eine passende Stelle zu bewegen.} \\
            }
        }
    }
    & & \\ \\

    \theutterance \stepcounter{utterance}  
    & & \multicolumn{4}{p{0.6\linewidth}}{
        \cellcolor[rgb]{0.9,0.9,0.9}{
            \makecell[{{p{\linewidth}}}]{
                \texttt{\tiny{[GM$\rangle$P2]}}
                \texttt{Deine Nachricht wurde an deinen Mitspieler übermittelt.} \\
\\ 
\\ 
\texttt{Ihr seid gerade bei Runde 10 von maximal 28.} \\
\texttt{Ihr habt gemeinsam 6 von maximal 16 Strafpunkten angesammelt.} \\
\texttt{Dein Mitspieler hat dir diese Nachricht geschickt:} \\
\\ 
\\ 
\texttt{"Es sieht so aus, als ob die Position (6,2) nicht leer ist. Ich denke, wir sollten eine andere Position für 'D' finden. Kannst du mir bitte vorschlagen, wo ich 'D' hinbewegen soll? Ich werde mich bemühen, es an eine passende Stelle zu bewegen."} \\
\\ 
\\ 
\texttt{Wie lautet dein nächster Befehl?} \\
            }
        }
    }
    & \\ \\

    \theutterance \stepcounter{utterance}  
    & & & \multicolumn{4}{p{0.6\linewidth}}{
        \cellcolor[rgb]{0.9,0.9,0.9}{
            \makecell[{{p{\linewidth}}}]{
                \texttt{\tiny{[GM$\langle$P2]}}
                \texttt{SPRICH: Ich denke, wir sollten 'D' an eine Position bewegen, die nah an der Mitte des Rasters ist. Wie wäre es, wenn wir 'D' an die Position (3,4) bewegen? Bitte teile mir mit, ob diese Position bei dir frei ist.} \\
            }
        }
    }
     \\ \\

    \theutterance \stepcounter{utterance}  
    & & & \multicolumn{2}{p{0.3\linewidth}}{
        \cellcolor[rgb]{0.9,0.9,0.9}{
            \makecell[{{p{\linewidth}}}]{
                \texttt{\tiny{[GM$|$GM]}}
                \texttt{SPRICH: Ich denke, wir sollten 'D' an eine Position bewegen, die nah an der Mitte des Rasters ist. Wie wäre es, wenn wir 'D' an die Position (3,4) bewegen? Bitte teile mir mit, ob diese Position bei dir frei ist.} \\
            }
        }
    }
    & & \\ \\

    \theutterance \stepcounter{utterance}  
    & & \multicolumn{4}{p{0.6\linewidth}}{
        \cellcolor[rgb]{0.9,0.9,0.9}{
            \makecell[{{p{\linewidth}}}]{
                \texttt{\tiny{[P1$\langle$GM]}}
                \texttt{Deine Nachricht wurde an deinen Mitspieler übermittelt.} \\
\\ 
\\ 
\texttt{Ihr seid gerade bei Runde 10 von maximal 28.} \\
\texttt{Ihr habt gemeinsam 6 von maximal 16 Strafpunkten angesammelt.} \\
\texttt{Dein Mitspieler hat dir diese Nachricht geschickt:} \\
\\ 
\\ 
\texttt{"Ich denke, wir sollten 'D' an eine Position bewegen, die nah an der Mitte des Rasters ist. Wie wäre es, wenn wir 'D' an die Position (3,4) bewegen? Bitte teile mir mit, ob diese Position bei dir frei ist."} \\
\\ 
\\ 
\texttt{Wie lautet dein nächster Befehl?} \\
            }
        }
    }
    & \\ \\

    \theutterance \stepcounter{utterance}  
    & \multicolumn{4}{p{0.6\linewidth}}{
        \cellcolor[rgb]{0.9,0.9,0.9}{
            \makecell[{{p{\linewidth}}}]{
                \texttt{\tiny{[P1$\rangle$GM]}}
                \texttt{SPRICH: Die Position (3,4) ist bei mir nicht frei, weil 'M' bereits an Position (3,3) ist und die Position (3,4) durch die Trennlinie '┤' blockiert ist. Ich denke, wir sollten eine andere Position für 'D' finden. Wie wäre es, wenn wir 'D' an die Position (5,5) bewegen? Bitte teile mir mit, ob diese Position bei dir frei ist.} \\
            }
        }
    }
    & & \\ \\

    \theutterance \stepcounter{utterance}  
    & & & \multicolumn{2}{p{0.3\linewidth}}{
        \cellcolor[rgb]{0.9,0.9,0.9}{
            \makecell[{{p{\linewidth}}}]{
                \texttt{\tiny{[GM$|$GM]}}
                \texttt{SPRICH: Die Position (3,4) ist bei mir nicht frei, weil 'M' bereits an Position (3,3) ist und die Position (3,4) durch die Trennlinie '┤' blockiert ist. Ich denke, wir sollten eine andere Position für 'D' finden. Wie wäre es, wenn wir 'D' an die Position (5,5) bewegen? Bitte teile mir mit, ob diese Position bei dir frei ist.} \\
            }
        }
    }
    & & \\ \\

    \theutterance \stepcounter{utterance}  
    & & \multicolumn{4}{p{0.6\linewidth}}{
        \cellcolor[rgb]{0.9,0.9,0.9}{
            \makecell[{{p{\linewidth}}}]{
                \texttt{\tiny{[GM$\rangle$P2]}}
                \texttt{Deine Nachricht wurde an deinen Mitspieler übermittelt.} \\
\\ 
\\ 
\texttt{Ihr seid gerade bei Runde 11 von maximal 28.} \\
\texttt{Ihr habt gemeinsam 6 von maximal 16 Strafpunkten angesammelt.} \\
\texttt{Dein Mitspieler hat dir diese Nachricht geschickt:} \\
\\ 
\\ 
\texttt{"Die Position (3,4) ist bei mir nicht frei, weil 'M' bereits an Position (3,3) ist und die Position (3,4) durch die Trennlinie '┤' blockiert ist. Ich denke, wir sollten eine andere Position für 'D' finden. Wie wäre es, wenn wir 'D' an die Position (5,5) bewegen? Bitte teile mir mit, ob diese Position bei dir frei ist."} \\
\\ 
\\ 
\texttt{Wie lautet dein nächster Befehl?} \\
            }
        }
    }
    & \\ \\

    \theutterance \stepcounter{utterance}  
    & & & \multicolumn{4}{p{0.6\linewidth}}{
        \cellcolor[rgb]{0.9,0.9,0.9}{
            \makecell[{{p{\linewidth}}}]{
                \texttt{\tiny{[GM$\langle$P2]}}
                \texttt{SPRICH: Die Position (5,5) ist bei mir frei. Ich denke, das ist eine gute Wahl. Bitte bewege 'D' an die Position (5,5). Ich werde warten, bis du bestätigst, dass du die Änderung vorgenommen hast, bevor wir zum nächsten Schritt übergehen.} \\
            }
        }
    }
     \\ \\

    \theutterance \stepcounter{utterance}  
    & & & \multicolumn{2}{p{0.3\linewidth}}{
        \cellcolor[rgb]{0.9,0.9,0.9}{
            \makecell[{{p{\linewidth}}}]{
                \texttt{\tiny{[GM$|$GM]}}
                \texttt{SPRICH: Die Position (5,5) ist bei mir frei. Ich denke, das ist eine gute Wahl. Bitte bewege 'D' an die Position (5,5). Ich werde warten, bis du bestätigst, dass du die Änderung vorgenommen hast, bevor wir zum nächsten Schritt übergehen.} \\
            }
        }
    }
    & & \\ \\

    \theutterance \stepcounter{utterance}  
    & & \multicolumn{4}{p{0.6\linewidth}}{
        \cellcolor[rgb]{0.9,0.9,0.9}{
            \makecell[{{p{\linewidth}}}]{
                \texttt{\tiny{[P1$\langle$GM]}}
                \texttt{Deine Nachricht wurde an deinen Mitspieler übermittelt.} \\
\\ 
\\ 
\texttt{Ihr seid gerade bei Runde 11 von maximal 28.} \\
\texttt{Ihr habt gemeinsam 6 von maximal 16 Strafpunkten angesammelt.} \\
\texttt{Dein Mitspieler hat dir diese Nachricht geschickt:} \\
\\ 
\\ 
\texttt{"Die Position (5,5) ist bei mir frei. Ich denke, das ist eine gute Wahl. Bitte bewege 'D' an die Position (5,5). Ich werde warten, bis du bestätigst, dass du die Änderung vorgenommen hast, bevor wir zum nächsten Schritt übergehen."} \\
\\ 
\\ 
\texttt{Wie lautet dein nächster Befehl?} \\
            }
        }
    }
    & \\ \\

    \theutterance \stepcounter{utterance}  
    & \multicolumn{4}{p{0.6\linewidth}}{
        \cellcolor[rgb]{0.9,0.9,0.9}{
            \makecell[{{p{\linewidth}}}]{
                \texttt{\tiny{[P1$\rangle$GM]}}
                \texttt{VERSCHIEBE: D, (5, 5)} \\
            }
        }
    }
    & & \\ \\

    \theutterance \stepcounter{utterance}  
    & & & \multicolumn{2}{p{0.3\linewidth}}{
        \cellcolor[rgb]{0.9,0.9,0.9}{
            \makecell[{{p{\linewidth}}}]{
                \texttt{\tiny{[GM$|$GM]}}
                \texttt{VERSCHIEBE: D, (5, 5)} \\
            }
        }
    }
    & & \\ \\

    \theutterance \stepcounter{utterance}  
    & & & \multicolumn{2}{p{0.3\linewidth}}{
        \cellcolor[rgb]{0.9,0.9,0.9}{
            \makecell[{{p{\linewidth}}}]{
                \texttt{\tiny{[GM$|$GM]}}
                \texttt{'D' erfolgreich nach (5,5) verschoben. Dein Raster sieht jetzt so aus:} \\
\texttt{```} \\
\texttt{ 1234567} \\
\texttt{╔═╤═╤═╤═╗} \\
\texttt{║◌│◌│◌│◌║ 1} \\
\texttt{╟─┴─┼─┤◌║ 2} \\
\texttt{║◌◌M│◌└─╢ 3} \\
\texttt{╟───┘◌S◌║ 4} \\
\texttt{║◌A◌◌D┌─╢ 5} \\
\texttt{║◌◌◌PT│O║ 6} \\
\texttt{║◌◌◌◌◌│◌║ 7} \\
\texttt{╚═════╧═╝} \\
\texttt{```} \\
            }
        }
    }
    & & \\ \\

    \theutterance \stepcounter{utterance}  
    & & \multicolumn{4}{p{0.6\linewidth}}{
        \cellcolor[rgb]{0.9,0.9,0.9}{
            \makecell[{{p{\linewidth}}}]{
                \texttt{\tiny{[GM$\rangle$P2]}}
                \texttt{Deine Nachricht wurde an deinen Mitspieler übermittelt.} \\
\\ 
\\ 
\texttt{Ihr seid gerade bei Runde 12 von maximal 28.} \\
\texttt{Ihr habt gemeinsam 6 von maximal 16 Strafpunkten angesammelt.} \\
\texttt{Dein Mitspieler hat ein Objekt auf seinem Raster bewegt.} \\
\\ 
\\ 
\texttt{Wie lautet dein nächster Befehl?} \\
            }
        }
    }
    & \\ \\

    \theutterance \stepcounter{utterance}  
    & & & \multicolumn{4}{p{0.6\linewidth}}{
        \cellcolor[rgb]{0.9,0.9,0.9}{
            \makecell[{{p{\linewidth}}}]{
                \texttt{\tiny{[GM$\langle$P2]}}
                \texttt{SPRICH: Hast du 'D' erfolgreich an die Position (5,5) bewegt? Bitte bestätige, damit wir zum nächsten Schritt übergehen können.} \\
            }
        }
    }
     \\ \\

    \theutterance \stepcounter{utterance}  
    & & & \multicolumn{2}{p{0.3\linewidth}}{
        \cellcolor[rgb]{0.9,0.9,0.9}{
            \makecell[{{p{\linewidth}}}]{
                \texttt{\tiny{[GM$|$GM]}}
                \texttt{SPRICH: Hast du 'D' erfolgreich an die Position (5,5) bewegt? Bitte bestätige, damit wir zum nächsten Schritt übergehen können.} \\
            }
        }
    }
    & & \\ \\

    \theutterance \stepcounter{utterance}  
    & & \multicolumn{4}{p{0.6\linewidth}}{
        \cellcolor[rgb]{0.9,0.9,0.9}{
            \makecell[{{p{\linewidth}}}]{
                \texttt{\tiny{[P1$\langle$GM]}}
                \texttt{'D' erfolgreich nach (5,5) verschoben. Dein Raster sieht jetzt so aus:} \\
\texttt{```} \\
\texttt{ 1234567} \\
\texttt{╔═╤═╤═╤═╗} \\
\texttt{║◌│◌│◌│◌║ 1} \\
\texttt{╟─┴─┼─┤◌║ 2} \\
\texttt{║◌◌M│◌└─╢ 3} \\
\texttt{╟───┘◌S◌║ 4} \\
\texttt{║◌A◌◌D┌─╢ 5} \\
\texttt{║◌◌◌PT│O║ 6} \\
\texttt{║◌◌◌◌◌│◌║ 7} \\
\texttt{╚═════╧═╝} \\
\texttt{```} \\
\texttt{Ihr seid gerade bei Runde 12 von maximal 28.} \\
\texttt{Ihr habt gemeinsam 6 von maximal 16 Strafpunkten angesammelt.} \\
\texttt{Dein Mitspieler hat dir diese Nachricht geschickt:} \\
\\ 
\\ 
\texttt{"Hast du 'D' erfolgreich an die Position (5,5) bewegt? Bitte bestätige, damit wir zum nächsten Schritt übergehen können."} \\
\\ 
\\ 
\texttt{Wie lautet dein nächster Befehl?} \\
            }
        }
    }
    & \\ \\

    \theutterance \stepcounter{utterance}  
    & \multicolumn{4}{p{0.6\linewidth}}{
        \cellcolor[rgb]{0.9,0.9,0.9}{
            \makecell[{{p{\linewidth}}}]{
                \texttt{\tiny{[P1$\rangle$GM]}}
                \texttt{SPRICH: Ja, ich habe 'D' erfolgreich an die Position (5,5) bewegt. Mein Raster sieht jetzt so aus, dass 'D' an der richtigen Stelle ist. Bitte geh zum nächsten Schritt über. Welches Objekt sollen wir als Nächstes bewegen?} \\
            }
        }
    }
    & & \\ \\

    \theutterance \stepcounter{utterance}  
    & & & \multicolumn{2}{p{0.3\linewidth}}{
        \cellcolor[rgb]{0.9,0.9,0.9}{
            \makecell[{{p{\linewidth}}}]{
                \texttt{\tiny{[GM$|$GM]}}
                \texttt{SPRICH: Ja, ich habe 'D' erfolgreich an die Position (5,5) bewegt. Mein Raster sieht jetzt so aus, dass 'D' an der richtigen Stelle ist. Bitte geh zum nächsten Schritt über. Welches Objekt sollen wir als Nächstes bewegen?} \\
            }
        }
    }
    & & \\ \\

    \theutterance \stepcounter{utterance}  
    & & \multicolumn{4}{p{0.6\linewidth}}{
        \cellcolor[rgb]{0.9,0.9,0.9}{
            \makecell[{{p{\linewidth}}}]{
                \texttt{\tiny{[GM$\rangle$P2]}}
                \texttt{Deine Nachricht wurde an deinen Mitspieler übermittelt.} \\
\\ 
\\ 
\texttt{Ihr seid gerade bei Runde 13 von maximal 28.} \\
\texttt{Ihr habt gemeinsam 6 von maximal 16 Strafpunkten angesammelt.} \\
\texttt{Dein Mitspieler hat dir diese Nachricht geschickt:} \\
\\ 
\\ 
\texttt{"Ja, ich habe 'D' erfolgreich an die Position (5,5) bewegt. Mein Raster sieht jetzt so aus, dass 'D' an der richtigen Stelle ist. Bitte geh zum nächsten Schritt über. Welches Objekt sollen wir als Nächstes bewegen?"} \\
\\ 
\\ 
\texttt{Wie lautet dein nächster Befehl?} \\
            }
        }
    }
    & \\ \\

    \theutterance \stepcounter{utterance}  
    & & & \multicolumn{4}{p{0.6\linewidth}}{
        \cellcolor[rgb]{0.9,0.9,0.9}{
            \makecell[{{p{\linewidth}}}]{
                \texttt{\tiny{[GM$\langle$P2]}}
                \texttt{SPRICH: Lass uns 'M' als Nächstes bewegen. Ich denke, wir sollten 'M' an eine Position bringen, die es uns ermöglicht, unsere Anordnungen zu synchronisieren. Ich schlage vor, dass wir 'M' an die Position (3,3) bewegen, wie bei mir. Was meinst du?} \\
            }
        }
    }
     \\ \\

    \theutterance \stepcounter{utterance}  
    & & & \multicolumn{2}{p{0.3\linewidth}}{
        \cellcolor[rgb]{0.9,0.9,0.9}{
            \makecell[{{p{\linewidth}}}]{
                \texttt{\tiny{[GM$|$GM]}}
                \texttt{SPRICH: Lass uns 'M' als Nächstes bewegen. Ich denke, wir sollten 'M' an eine Position bringen, die es uns ermöglicht, unsere Anordnungen zu synchronisieren. Ich schlage vor, dass wir 'M' an die Position (3,3) bewegen, wie bei mir. Was meinst du?} \\
            }
        }
    }
    & & \\ \\

    \theutterance \stepcounter{utterance}  
    & & \multicolumn{4}{p{0.6\linewidth}}{
        \cellcolor[rgb]{0.9,0.9,0.9}{
            \makecell[{{p{\linewidth}}}]{
                \texttt{\tiny{[P1$\langle$GM]}}
                \texttt{Deine Nachricht wurde an deinen Mitspieler übermittelt.} \\
\\ 
\\ 
\texttt{Ihr seid gerade bei Runde 13 von maximal 28.} \\
\texttt{Ihr habt gemeinsam 6 von maximal 16 Strafpunkten angesammelt.} \\
\texttt{Dein Mitspieler hat dir diese Nachricht geschickt:} \\
\\ 
\\ 
\texttt{"Lass uns 'M' als Nächstes bewegen. Ich denke, wir sollten 'M' an eine Position bringen, die es uns ermöglicht, unsere Anordnungen zu synchronisieren. Ich schlage vor, dass wir 'M' an die Position (3,3) bewegen, wie bei mir. Was meinst du?"} \\
\\ 
\\ 
\texttt{Wie lautet dein nächster Befehl?} \\
            }
        }
    }
    & \\ \\

    \theutterance \stepcounter{utterance}  
    & \multicolumn{4}{p{0.6\linewidth}}{
        \cellcolor[rgb]{0.9,0.9,0.9}{
            \makecell[{{p{\linewidth}}}]{
                \texttt{\tiny{[P1$\rangle$GM]}}
                \texttt{SPRICH: Das ist interessant! Bei mir ist 'M' bereits an der Position (3,3). Es sieht so aus, als ob unsere Raster in diesem Punkt bereits synchronisiert sind. Wir können also zum nächsten Schritt übergehen. Wie wäre es, wenn wir 'S' als Nächstes bewegen? Bitte teile mir mit, wo 'S' bei dir ist und wo wir es hinbewegen sollten.} \\
            }
        }
    }
    & & \\ \\

    \theutterance \stepcounter{utterance}  
    & & & \multicolumn{2}{p{0.3\linewidth}}{
        \cellcolor[rgb]{0.9,0.9,0.9}{
            \makecell[{{p{\linewidth}}}]{
                \texttt{\tiny{[GM$|$GM]}}
                \texttt{SPRICH: Das ist interessant! Bei mir ist 'M' bereits an der Position (3,3). Es sieht so aus, als ob unsere Raster in diesem Punkt bereits synchronisiert sind. Wir können also zum nächsten Schritt übergehen. Wie wäre es, wenn wir 'S' als Nächstes bewegen? Bitte teile mir mit, wo 'S' bei dir ist und wo wir es hinbewegen sollten.} \\
            }
        }
    }
    & & \\ \\

    \theutterance \stepcounter{utterance}  
    & & \multicolumn{4}{p{0.6\linewidth}}{
        \cellcolor[rgb]{0.9,0.9,0.9}{
            \makecell[{{p{\linewidth}}}]{
                \texttt{\tiny{[GM$\rangle$P2]}}
                \texttt{Deine Nachricht wurde an deinen Mitspieler übermittelt.} \\
\\ 
\\ 
\texttt{Ihr seid gerade bei Runde 14 von maximal 28.} \\
\texttt{Ihr habt gemeinsam 6 von maximal 16 Strafpunkten angesammelt.} \\
\texttt{Dein Mitspieler hat dir diese Nachricht geschickt:} \\
\\ 
\\ 
\texttt{"Das ist interessant! Bei mir ist 'M' bereits an der Position (3,3). Es sieht so aus, als ob unsere Raster in diesem Punkt bereits synchronisiert sind. Wir können also zum nächsten Schritt übergehen. Wie wäre es, wenn wir 'S' als Nächstes bewegen? Bitte teile mir mit, wo 'S' bei dir ist und wo wir es hinbewegen sollten."} \\
\\ 
\\ 
\texttt{Wie lautet dein nächster Befehl?} \\
            }
        }
    }
    & \\ \\

    \theutterance \stepcounter{utterance}  
    & & & \multicolumn{4}{p{0.6\linewidth}}{
        \cellcolor[rgb]{0.9,0.9,0.9}{
            \makecell[{{p{\linewidth}}}]{
                \texttt{\tiny{[GM$\langle$P2]}}
                \texttt{SPRICH: Bei mir ist 'S' an der Position (7,6). Ich denke, wir sollten 'S' an eine Position bringen, die es uns ermöglicht, unsere Anordnungen zu synchronisieren. Ich schlage vor, dass wir 'S' an die Position (7,6) lassen, da es bei mir bereits an dieser Position ist. Wenn du dein 'S' an eine andere Position bewegen möchtest, bitte teile mir mit, wo es bei dir ist und wo wir es hinbewegen sollten.} \\
            }
        }
    }
     \\ \\

    \theutterance \stepcounter{utterance}  
    & & & \multicolumn{2}{p{0.3\linewidth}}{
        \cellcolor[rgb]{0.9,0.9,0.9}{
            \makecell[{{p{\linewidth}}}]{
                \texttt{\tiny{[GM$|$GM]}}
                \texttt{SPRICH: Bei mir ist 'S' an der Position (7,6). Ich denke, wir sollten 'S' an eine Position bringen, die es uns ermöglicht, unsere Anordnungen zu synchronisieren. Ich schlage vor, dass wir 'S' an die Position (7,6) lassen, da es bei mir bereits an dieser Position ist. Wenn du dein 'S' an eine andere Position bewegen möchtest, bitte teile mir mit, wo es bei dir ist und wo wir es hinbewegen sollten.} \\
            }
        }
    }
    & & \\ \\

    \theutterance \stepcounter{utterance}  
    & & \multicolumn{4}{p{0.6\linewidth}}{
        \cellcolor[rgb]{0.9,0.9,0.9}{
            \makecell[{{p{\linewidth}}}]{
                \texttt{\tiny{[P1$\langle$GM]}}
                \texttt{Deine Nachricht wurde an deinen Mitspieler übermittelt.} \\
\\ 
\\ 
\texttt{Ihr seid gerade bei Runde 14 von maximal 28.} \\
\texttt{Ihr habt gemeinsam 6 von maximal 16 Strafpunkten angesammelt.} \\
\texttt{Dein Mitspieler hat dir diese Nachricht geschickt:} \\
\\ 
\\ 
\texttt{"Bei mir ist 'S' an der Position (7,6). Ich denke, wir sollten 'S' an eine Position bringen, die es uns ermöglicht, unsere Anordnungen zu synchronisieren. Ich schlage vor, dass wir 'S' an die Position (7,6) lassen, da es bei mir bereits an dieser Position ist. Wenn du dein 'S' an eine andere Position bewegen möchtest, bitte teile mir mit, wo es bei dir ist und wo wir es hinbewegen sollten."} \\
\\ 
\\ 
\texttt{Wie lautet dein nächster Befehl?} \\
            }
        }
    }
    & \\ \\

    \theutterance \stepcounter{utterance}  
    & \multicolumn{4}{p{0.6\linewidth}}{
        \cellcolor[rgb]{0.9,0.9,0.9}{
            \makecell[{{p{\linewidth}}}]{
                \texttt{\tiny{[P1$\rangle$GM]}}
                \texttt{SPRICH: Bei mir ist 'S' an der Position (4,4). Ich denke, wir sollten 'S' an eine Position bringen, die es uns ermöglicht, unsere Anordnungen zu synchronisieren. Da 'S' bei dir bereits an der Position (7,6) ist, schlage ich vor, dass wir mein 'S' an die Position (7,6) bewegen. Bitte warte, bis ich bestätige, dass ich die Änderung vorgenommen habe, bevor wir zum nächsten Schritt übergehen. Ich werde jetzt 'S' bewegen.} \\
            }
        }
    }
    & & \\ \\

    \theutterance \stepcounter{utterance}  
    & & & \multicolumn{2}{p{0.3\linewidth}}{
        \cellcolor[rgb]{0.9,0.9,0.9}{
            \makecell[{{p{\linewidth}}}]{
                \texttt{\tiny{[GM$|$GM]}}
                \texttt{SPRICH: Bei mir ist 'S' an der Position (4,4). Ich denke, wir sollten 'S' an eine Position bringen, die es uns ermöglicht, unsere Anordnungen zu synchronisieren. Da 'S' bei dir bereits an der Position (7,6) ist, schlage ich vor, dass wir mein 'S' an die Position (7,6) bewegen. Bitte warte, bis ich bestätige, dass ich die Änderung vorgenommen habe, bevor wir zum nächsten Schritt übergehen. Ich werde jetzt 'S' bewegen.} \\
            }
        }
    }
    & & \\ \\

    \theutterance \stepcounter{utterance}  
    & & \multicolumn{4}{p{0.6\linewidth}}{
        \cellcolor[rgb]{0.9,0.9,0.9}{
            \makecell[{{p{\linewidth}}}]{
                \texttt{\tiny{[GM$\rangle$P2]}}
                \texttt{Deine Nachricht wurde an deinen Mitspieler übermittelt.} \\
\\ 
\\ 
\texttt{Ihr seid gerade bei Runde 15 von maximal 28.} \\
\texttt{Ihr habt gemeinsam 6 von maximal 16 Strafpunkten angesammelt.} \\
\texttt{Dein Mitspieler hat dir diese Nachricht geschickt:} \\
\\ 
\\ 
\texttt{"Bei mir ist 'S' an der Position (4,4). Ich denke, wir sollten 'S' an eine Position bringen, die es uns ermöglicht, unsere Anordnungen zu synchronisieren. Da 'S' bei dir bereits an der Position (7,6) ist, schlage ich vor, dass wir mein 'S' an die Position (7,6) bewegen. Bitte warte, bis ich bestätige, dass ich die Änderung vorgenommen habe, bevor wir zum nächsten Schritt übergehen. Ich werde jetzt 'S' bewegen."} \\
\\ 
\\ 
\texttt{Wie lautet dein nächster Befehl?} \\
            }
        }
    }
    & \\ \\

    \theutterance \stepcounter{utterance}  
    & & & \multicolumn{4}{p{0.6\linewidth}}{
        \cellcolor[rgb]{0.9,0.9,0.9}{
            \makecell[{{p{\linewidth}}}]{
                \texttt{\tiny{[GM$\langle$P2]}}
                \texttt{SPRICH: Okay, ich warte. Bitte teile mir mit, wenn du 'S' erfolgreich an die Position (7,6) bewegt hast.} \\
            }
        }
    }
     \\ \\

    \theutterance \stepcounter{utterance}  
    & & & \multicolumn{2}{p{0.3\linewidth}}{
        \cellcolor[rgb]{0.9,0.9,0.9}{
            \makecell[{{p{\linewidth}}}]{
                \texttt{\tiny{[GM$|$GM]}}
                \texttt{SPRICH: Okay, ich warte. Bitte teile mir mit, wenn du 'S' erfolgreich an die Position (7,6) bewegt hast.} \\
            }
        }
    }
    & & \\ \\

    \theutterance \stepcounter{utterance}  
    & & \multicolumn{4}{p{0.6\linewidth}}{
        \cellcolor[rgb]{0.9,0.9,0.9}{
            \makecell[{{p{\linewidth}}}]{
                \texttt{\tiny{[P1$\langle$GM]}}
                \texttt{Deine Nachricht wurde an deinen Mitspieler übermittelt.} \\
\\ 
\\ 
\texttt{Ihr seid gerade bei Runde 15 von maximal 28.} \\
\texttt{Ihr habt gemeinsam 6 von maximal 16 Strafpunkten angesammelt.} \\
\texttt{Dein Mitspieler hat dir diese Nachricht geschickt:} \\
\\ 
\\ 
\texttt{"Okay, ich warte. Bitte teile mir mit, wenn du 'S' erfolgreich an die Position (7,6) bewegt hast."} \\
\\ 
\\ 
\texttt{Wie lautet dein nächster Befehl?} \\
            }
        }
    }
    & \\ \\

    \theutterance \stepcounter{utterance}  
    & \multicolumn{4}{p{0.6\linewidth}}{
        \cellcolor[rgb]{0.9,0.9,0.9}{
            \makecell[{{p{\linewidth}}}]{
                \texttt{\tiny{[P1$\rangle$GM]}}
                \texttt{VERSCHIEBE: S, (7, 6)} \\
            }
        }
    }
    & & \\ \\

    \theutterance \stepcounter{utterance}  
    & & & \multicolumn{2}{p{0.3\linewidth}}{
        \cellcolor[rgb]{0.9,0.9,0.9}{
            \makecell[{{p{\linewidth}}}]{
                \texttt{\tiny{[GM$|$GM]}}
                \texttt{VERSCHIEBE: S, (7, 6)} \\
            }
        }
    }
    & & \\ \\

    \theutterance \stepcounter{utterance}  
    & & & \multicolumn{2}{p{0.3\linewidth}}{
        \cellcolor[rgb]{0.9,0.9,0.9}{
            \makecell[{{p{\linewidth}}}]{
                \texttt{\tiny{[GM$|$GM]}}
                \texttt{Strafpunkt: (7,6) ist nicht leer, sondern enthält 'O'.} \\
\\ 
\texttt{Ihr habt gemeinsam 7 von maximal 16 Strafpunkten angesammelt.} \\
\texttt{Bitte versuche es noch mal!} \\
            }
        }
    }
    & & \\ \\

    \theutterance \stepcounter{utterance}  
    & & \multicolumn{4}{p{0.6\linewidth}}{
        \cellcolor[rgb]{0.9,0.9,0.9}{
            \makecell[{{p{\linewidth}}}]{
                \texttt{\tiny{[P1$\langle$GM]}}
                \texttt{Strafpunkt: (7,6) ist nicht leer, sondern enthält 'O'.} \\
\\ 
\texttt{Ihr habt gemeinsam 7 von maximal 16 Strafpunkten angesammelt.} \\
\texttt{Bitte versuche es noch mal!} \\
            }
        }
    }
    & \\ \\

    \theutterance \stepcounter{utterance}  
    & \multicolumn{4}{p{0.6\linewidth}}{
        \cellcolor[rgb]{0.9,0.9,0.9}{
            \makecell[{{p{\linewidth}}}]{
                \texttt{\tiny{[P1$\rangle$GM]}}
                \texttt{SPRICH: Es sieht so aus, als ob die Position (7,6) nicht frei ist, weil 'O' bereits dort ist. Ich denke, wir sollten eine andere Position für 'S' finden. Da 'S' bei dir bereits an der Position (7,6) ist und wir das nicht bei mir wiederholen können, schlage ich vor, dass wir mein 'S' an eine andere Position bewegen, die frei ist. Wie wäre es, wenn wir mein 'S' an die Position (7,7) bewegen? Bitte teile mir mit, ob diese Position bei dir frei ist.} \\
            }
        }
    }
    & & \\ \\

    \theutterance \stepcounter{utterance}  
    & & & \multicolumn{2}{p{0.3\linewidth}}{
        \cellcolor[rgb]{0.9,0.9,0.9}{
            \makecell[{{p{\linewidth}}}]{
                \texttt{\tiny{[GM$|$GM]}}
                \texttt{SPRICH: Es sieht so aus, als ob die Position (7,6) nicht frei ist, weil 'O' bereits dort ist. Ich denke, wir sollten eine andere Position für 'S' finden. Da 'S' bei dir bereits an der Position (7,6) ist und wir das nicht bei mir wiederholen können, schlage ich vor, dass wir mein 'S' an eine andere Position bewegen, die frei ist. Wie wäre es, wenn wir mein 'S' an die Position (7,7) bewegen? Bitte teile mir mit, ob diese Position bei dir frei ist.} \\
            }
        }
    }
    & & \\ \\

    \theutterance \stepcounter{utterance}  
    & & \multicolumn{4}{p{0.6\linewidth}}{
        \cellcolor[rgb]{0.9,0.9,0.9}{
            \makecell[{{p{\linewidth}}}]{
                \texttt{\tiny{[GM$\rangle$P2]}}
                \texttt{Deine Nachricht wurde an deinen Mitspieler übermittelt.} \\
\\ 
\\ 
\texttt{Ihr seid gerade bei Runde 16 von maximal 28.} \\
\texttt{Ihr habt gemeinsam 7 von maximal 16 Strafpunkten angesammelt.} \\
\texttt{Dein Mitspieler hat dir diese Nachricht geschickt:} \\
\\ 
\\ 
\texttt{"Es sieht so aus, als ob die Position (7,6) nicht frei ist, weil 'O' bereits dort ist. Ich denke, wir sollten eine andere Position für 'S' finden. Da 'S' bei dir bereits an der Position (7,6) ist und wir das nicht bei mir wiederholen können, schlage ich vor, dass wir mein 'S' an eine andere Position bewegen, die frei ist. Wie wäre es, wenn wir mein 'S' an die Position (7,7) bewegen? Bitte teile mir mit, ob diese Position bei dir frei ist."} \\
\\ 
\\ 
\texttt{Wie lautet dein nächster Befehl?} \\
            }
        }
    }
    & \\ \\

    \theutterance \stepcounter{utterance}  
    & & & \multicolumn{4}{p{0.6\linewidth}}{
        \cellcolor[rgb]{0.9,0.9,0.9}{
            \makecell[{{p{\linewidth}}}]{
                \texttt{\tiny{[GM$\langle$P2]}}
                \texttt{SPRICH: Die Position (7,7) ist bei mir nicht frei, weil 'T' bereits dort ist. Ich denke, wir sollten eine andere Position für 'S' finden. Wie wäre es, wenn wir deine 'O' an eine andere Position bewegen, um Platz für 'S' an der Position (7,6) zu schaffen? Bitte teile mir mit, ob das möglich ist.} \\
            }
        }
    }
     \\ \\

    \theutterance \stepcounter{utterance}  
    & & & \multicolumn{2}{p{0.3\linewidth}}{
        \cellcolor[rgb]{0.9,0.9,0.9}{
            \makecell[{{p{\linewidth}}}]{
                \texttt{\tiny{[GM$|$GM]}}
                \texttt{SPRICH: Die Position (7,7) ist bei mir nicht frei, weil 'T' bereits dort ist. Ich denke, wir sollten eine andere Position für 'S' finden. Wie wäre es, wenn wir deine 'O' an eine andere Position bewegen, um Platz für 'S' an der Position (7,6) zu schaffen? Bitte teile mir mit, ob das möglich ist.} \\
            }
        }
    }
    & & \\ \\

    \theutterance \stepcounter{utterance}  
    & & \multicolumn{4}{p{0.6\linewidth}}{
        \cellcolor[rgb]{0.9,0.9,0.9}{
            \makecell[{{p{\linewidth}}}]{
                \texttt{\tiny{[P1$\langle$GM]}}
                \texttt{Deine Nachricht wurde an deinen Mitspieler übermittelt.} \\
\\ 
\\ 
\texttt{Ihr seid gerade bei Runde 16 von maximal 28.} \\
\texttt{Ihr habt gemeinsam 7 von maximal 16 Strafpunkten angesammelt.} \\
\texttt{Dein Mitspieler hat dir diese Nachricht geschickt:} \\
\\ 
\\ 
\texttt{"Die Position (7,7) ist bei mir nicht frei, weil 'T' bereits dort ist. Ich denke, wir sollten eine andere Position für 'S' finden. Wie wäre es, wenn wir deine 'O' an eine andere Position bewegen, um Platz für 'S' an der Position (7,6) zu schaffen? Bitte teile mir mit, ob das möglich ist."} \\
\\ 
\\ 
\texttt{Wie lautet dein nächster Befehl?} \\
            }
        }
    }
    & \\ \\

    \theutterance \stepcounter{utterance}  
    & \multicolumn{4}{p{0.6\linewidth}}{
        \cellcolor[rgb]{0.9,0.9,0.9}{
            \makecell[{{p{\linewidth}}}]{
                \texttt{\tiny{[P1$\rangle$GM]}}
                \texttt{SPRICH: Das ist eine gute Idee! Bei mir ist 'O' an der Position (6,6). Ich denke, wir könnten 'O' an eine andere Position bewegen, um Platz für 'S' an der Position (7,6) zu schaffen. Wie wäre es, wenn wir 'O' an die Position (6,7) bewegen? Bitte warte, bis ich bestätige, dass ich die Änderung vorgenommen habe, bevor wir zum nächsten Schritt übergehen. Ich werde jetzt 'O' bewegen.} \\
            }
        }
    }
    & & \\ \\

    \theutterance \stepcounter{utterance}  
    & & & \multicolumn{2}{p{0.3\linewidth}}{
        \cellcolor[rgb]{0.9,0.9,0.9}{
            \makecell[{{p{\linewidth}}}]{
                \texttt{\tiny{[GM$|$GM]}}
                \texttt{SPRICH: Das ist eine gute Idee! Bei mir ist 'O' an der Position (6,6). Ich denke, wir könnten 'O' an eine andere Position bewegen, um Platz für 'S' an der Position (7,6) zu schaffen. Wie wäre es, wenn wir 'O' an die Position (6,7) bewegen? Bitte warte, bis ich bestätige, dass ich die Änderung vorgenommen habe, bevor wir zum nächsten Schritt übergehen. Ich werde jetzt 'O' bewegen.} \\
            }
        }
    }
    & & \\ \\

    \theutterance \stepcounter{utterance}  
    & & \multicolumn{4}{p{0.6\linewidth}}{
        \cellcolor[rgb]{0.9,0.9,0.9}{
            \makecell[{{p{\linewidth}}}]{
                \texttt{\tiny{[GM$\rangle$P2]}}
                \texttt{Deine Nachricht wurde an deinen Mitspieler übermittelt.} \\
\\ 
\\ 
\texttt{Ihr seid gerade bei Runde 17 von maximal 28.} \\
\texttt{Ihr habt gemeinsam 7 von maximal 16 Strafpunkten angesammelt.} \\
\texttt{Dein Mitspieler hat dir diese Nachricht geschickt:} \\
\\ 
\\ 
\texttt{"Das ist eine gute Idee! Bei mir ist 'O' an der Position (6,6). Ich denke, wir könnten 'O' an eine andere Position bewegen, um Platz für 'S' an der Position (7,6) zu schaffen. Wie wäre es, wenn wir 'O' an die Position (6,7) bewegen? Bitte warte, bis ich bestätige, dass ich die Änderung vorgenommen habe, bevor wir zum nächsten Schritt übergehen. Ich werde jetzt 'O' bewegen."} \\
\\ 
\\ 
\texttt{Wie lautet dein nächster Befehl?} \\
            }
        }
    }
    & \\ \\

    \theutterance \stepcounter{utterance}  
    & & & \multicolumn{4}{p{0.6\linewidth}}{
        \cellcolor[rgb]{0.9,0.9,0.9}{
            \makecell[{{p{\linewidth}}}]{
                \texttt{\tiny{[GM$\langle$P2]}}
                \texttt{SPRICH: Okay, ich warte. Bitte teile mir mit, wenn du 'O' erfolgreich an die Position (6,7) bewegt hast, damit wir 'S' an die Position (7,6) bewegen können.} \\
            }
        }
    }
     \\ \\

    \theutterance \stepcounter{utterance}  
    & & & \multicolumn{2}{p{0.3\linewidth}}{
        \cellcolor[rgb]{0.9,0.9,0.9}{
            \makecell[{{p{\linewidth}}}]{
                \texttt{\tiny{[GM$|$GM]}}
                \texttt{SPRICH: Okay, ich warte. Bitte teile mir mit, wenn du 'O' erfolgreich an die Position (6,7) bewegt hast, damit wir 'S' an die Position (7,6) bewegen können.} \\
            }
        }
    }
    & & \\ \\

    \theutterance \stepcounter{utterance}  
    & & \multicolumn{4}{p{0.6\linewidth}}{
        \cellcolor[rgb]{0.9,0.9,0.9}{
            \makecell[{{p{\linewidth}}}]{
                \texttt{\tiny{[P1$\langle$GM]}}
                \texttt{Deine Nachricht wurde an deinen Mitspieler übermittelt.} \\
\\ 
\\ 
\texttt{Ihr seid gerade bei Runde 17 von maximal 28.} \\
\texttt{Ihr habt gemeinsam 7 von maximal 16 Strafpunkten angesammelt.} \\
\texttt{Dein Mitspieler hat dir diese Nachricht geschickt:} \\
\\ 
\\ 
\texttt{"Okay, ich warte. Bitte teile mir mit, wenn du 'O' erfolgreich an die Position (6,7) bewegt hast, damit wir 'S' an die Position (7,6) bewegen können."} \\
\\ 
\\ 
\texttt{Wie lautet dein nächster Befehl?} \\
            }
        }
    }
    & \\ \\

    \theutterance \stepcounter{utterance}  
    & \multicolumn{4}{p{0.6\linewidth}}{
        \cellcolor[rgb]{0.9,0.9,0.9}{
            \makecell[{{p{\linewidth}}}]{
                \texttt{\tiny{[P1$\rangle$GM]}}
                \texttt{VERSCHIEBE: O, (6, 7)} \\
            }
        }
    }
    & & \\ \\

    \theutterance \stepcounter{utterance}  
    & & & \multicolumn{2}{p{0.3\linewidth}}{
        \cellcolor[rgb]{0.9,0.9,0.9}{
            \makecell[{{p{\linewidth}}}]{
                \texttt{\tiny{[GM$|$GM]}}
                \texttt{VERSCHIEBE: O, (6, 7)} \\
            }
        }
    }
    & & \\ \\

    \theutterance \stepcounter{utterance}  
    & & & \multicolumn{2}{p{0.3\linewidth}}{
        \cellcolor[rgb]{0.9,0.9,0.9}{
            \makecell[{{p{\linewidth}}}]{
                \texttt{\tiny{[GM$|$GM]}}
                \texttt{Strafpunkt: (6,7) ist nicht leer, sondern enthält '│'.} \\
\\ 
\texttt{Ihr habt gemeinsam 8 von maximal 16 Strafpunkten angesammelt.} \\
\texttt{Bitte versuche es noch mal!} \\
            }
        }
    }
    & & \\ \\

    \theutterance \stepcounter{utterance}  
    & & \multicolumn{4}{p{0.6\linewidth}}{
        \cellcolor[rgb]{0.9,0.9,0.9}{
            \makecell[{{p{\linewidth}}}]{
                \texttt{\tiny{[P1$\langle$GM]}}
                \texttt{Strafpunkt: (6,7) ist nicht leer, sondern enthält '│'.} \\
\\ 
\texttt{Ihr habt gemeinsam 8 von maximal 16 Strafpunkten angesammelt.} \\
\texttt{Bitte versuche es noch mal!} \\
            }
        }
    }
    & \\ \\

    \theutterance \stepcounter{utterance}  
    & \multicolumn{4}{p{0.6\linewidth}}{
        \cellcolor[rgb]{0.9,0.9,0.9}{
            \makecell[{{p{\linewidth}}}]{
                \texttt{\tiny{[P1$\rangle$GM]}}
                \texttt{SPRICH: Es sieht so aus, als ob die Position (6,7) nicht frei ist, weil '│' bereits dort ist. Ich denke, wir sollten eine andere Position für 'O' finden. Wie wäre es, wenn wir 'O' an die Position (5,7) bewegen? Bitte teile mir mit, ob diese Position bei dir frei ist. Ich werde dann 'O' dorthin bewegen und wir können weitermachen.} \\
            }
        }
    }
    & & \\ \\

    \theutterance \stepcounter{utterance}  
    & & & \multicolumn{2}{p{0.3\linewidth}}{
        \cellcolor[rgb]{0.9,0.9,0.9}{
            \makecell[{{p{\linewidth}}}]{
                \texttt{\tiny{[GM$|$GM]}}
                \texttt{SPRICH: Es sieht so aus, als ob die Position (6,7) nicht frei ist, weil '│' bereits dort ist. Ich denke, wir sollten eine andere Position für 'O' finden. Wie wäre es, wenn wir 'O' an die Position (5,7) bewegen? Bitte teile mir mit, ob diese Position bei dir frei ist. Ich werde dann 'O' dorthin bewegen und wir können weitermachen.} \\
            }
        }
    }
    & & \\ \\

    \theutterance \stepcounter{utterance}  
    & & \multicolumn{4}{p{0.6\linewidth}}{
        \cellcolor[rgb]{0.9,0.9,0.9}{
            \makecell[{{p{\linewidth}}}]{
                \texttt{\tiny{[GM$\rangle$P2]}}
                \texttt{Deine Nachricht wurde an deinen Mitspieler übermittelt.} \\
\\ 
\\ 
\texttt{Ihr seid gerade bei Runde 18 von maximal 28.} \\
\texttt{Ihr habt gemeinsam 8 von maximal 16 Strafpunkten angesammelt.} \\
\texttt{Dein Mitspieler hat dir diese Nachricht geschickt:} \\
\\ 
\\ 
\texttt{"Es sieht so aus, als ob die Position (6,7) nicht frei ist, weil '│' bereits dort ist. Ich denke, wir sollten eine andere Position für 'O' finden. Wie wäre es, wenn wir 'O' an die Position (5,7) bewegen? Bitte teile mir mit, ob diese Position bei dir frei ist. Ich werde dann 'O' dorthin bewegen und wir können weitermachen."} \\
\\ 
\\ 
\texttt{Wie lautet dein nächster Befehl?} \\
            }
        }
    }
    & \\ \\

    \theutterance \stepcounter{utterance}  
    & & & \multicolumn{4}{p{0.6\linewidth}}{
        \cellcolor[rgb]{0.9,0.9,0.9}{
            \makecell[{{p{\linewidth}}}]{
                \texttt{\tiny{[GM$\langle$P2]}}
                \texttt{SPRICH: Die Position (5,7) ist bei mir nicht frei, weil sie außerhalb des Rasters liegt, da meine Rasterzeilen nur bis 7 reichen und die Spaltenzeichen nicht mit den Objekten interagieren. Ich denke, wir sollten eine andere Position für 'O' finden, die innerhalb des Rasters liegt. Wie wäre es, wenn wir deine 'O' an die Position (5,4) bewegen? Bitte teile mir mit, ob das möglich ist.} \\
            }
        }
    }
     \\ \\

    \theutterance \stepcounter{utterance}  
    & & & \multicolumn{2}{p{0.3\linewidth}}{
        \cellcolor[rgb]{0.9,0.9,0.9}{
            \makecell[{{p{\linewidth}}}]{
                \texttt{\tiny{[GM$|$GM]}}
                \texttt{SPRICH: Die Position (5,7) ist bei mir nicht frei, weil sie außerhalb des Rasters liegt, da meine Rasterzeilen nur bis 7 reichen und die Spaltenzeichen nicht mit den Objekten interagieren. Ich denke, wir sollten eine andere Position für 'O' finden, die innerhalb des Rasters liegt. Wie wäre es, wenn wir deine 'O' an die Position (5,4) bewegen? Bitte teile mir mit, ob das möglich ist.} \\
            }
        }
    }
    & & \\ \\

    \theutterance \stepcounter{utterance}  
    & & \multicolumn{4}{p{0.6\linewidth}}{
        \cellcolor[rgb]{0.9,0.9,0.9}{
            \makecell[{{p{\linewidth}}}]{
                \texttt{\tiny{[P1$\langle$GM]}}
                \texttt{Deine Nachricht wurde an deinen Mitspieler übermittelt.} \\
\\ 
\\ 
\texttt{Ihr seid gerade bei Runde 18 von maximal 28.} \\
\texttt{Ihr habt gemeinsam 8 von maximal 16 Strafpunkten angesammelt.} \\
\texttt{Dein Mitspieler hat dir diese Nachricht geschickt:} \\
\\ 
\\ 
\texttt{"Die Position (5,7) ist bei mir nicht frei, weil sie außerhalb des Rasters liegt, da meine Rasterzeilen nur bis 7 reichen und die Spaltenzeichen nicht mit den Objekten interagieren. Ich denke, wir sollten eine andere Position für 'O' finden, die innerhalb des Rasters liegt. Wie wäre es, wenn wir deine 'O' an die Position (5,4) bewegen? Bitte teile mir mit, ob das möglich ist."} \\
\\ 
\\ 
\texttt{Wie lautet dein nächster Befehl?} \\
            }
        }
    }
    & \\ \\

    \theutterance \stepcounter{utterance}  
    & \multicolumn{4}{p{0.6\linewidth}}{
        \cellcolor[rgb]{0.9,0.9,0.9}{
            \makecell[{{p{\linewidth}}}]{
                \texttt{\tiny{[P1$\rangle$GM]}}
                \texttt{VERSCHIEBE: O, (5, 4)} \\
            }
        }
    }
    & & \\ \\

    \theutterance \stepcounter{utterance}  
    & & & \multicolumn{2}{p{0.3\linewidth}}{
        \cellcolor[rgb]{0.9,0.9,0.9}{
            \makecell[{{p{\linewidth}}}]{
                \texttt{\tiny{[GM$|$GM]}}
                \texttt{VERSCHIEBE: O, (5, 4)} \\
            }
        }
    }
    & & \\ \\

    \theutterance \stepcounter{utterance}  
    & & & \multicolumn{2}{p{0.3\linewidth}}{
        \cellcolor[rgb]{0.9,0.9,0.9}{
            \makecell[{{p{\linewidth}}}]{
                \texttt{\tiny{[GM$|$GM]}}
                \texttt{'O' erfolgreich nach (5,4) verschoben. Dein Raster sieht jetzt so aus:} \\
\texttt{```} \\
\texttt{ 1234567} \\
\texttt{╔═╤═╤═╤═╗} \\
\texttt{║◌│◌│◌│◌║ 1} \\
\texttt{╟─┴─┼─┤◌║ 2} \\
\texttt{║◌◌M│◌└─╢ 3} \\
\texttt{╟───┘OS◌║ 4} \\
\texttt{║◌A◌◌D┌─╢ 5} \\
\texttt{║◌◌◌PT│◌║ 6} \\
\texttt{║◌◌◌◌◌│◌║ 7} \\
\texttt{╚═════╧═╝} \\
\texttt{```} \\
            }
        }
    }
    & & \\ \\

    \theutterance \stepcounter{utterance}  
    & & \multicolumn{4}{p{0.6\linewidth}}{
        \cellcolor[rgb]{0.9,0.9,0.9}{
            \makecell[{{p{\linewidth}}}]{
                \texttt{\tiny{[GM$\rangle$P2]}}
                \texttt{Deine Nachricht wurde an deinen Mitspieler übermittelt.} \\
\\ 
\\ 
\texttt{Ihr seid gerade bei Runde 19 von maximal 28.} \\
\texttt{Ihr habt gemeinsam 8 von maximal 16 Strafpunkten angesammelt.} \\
\texttt{Dein Mitspieler hat ein Objekt auf seinem Raster bewegt.} \\
\\ 
\\ 
\texttt{Wie lautet dein nächster Befehl?} \\
            }
        }
    }
    & \\ \\

    \theutterance \stepcounter{utterance}  
    & & & \multicolumn{4}{p{0.6\linewidth}}{
        \cellcolor[rgb]{0.9,0.9,0.9}{
            \makecell[{{p{\linewidth}}}]{
                \texttt{\tiny{[GM$\langle$P2]}}
                \texttt{SPRICH: Hast du 'O' erfolgreich an die Position (5,4) bewegt? Bitte bestätige, damit wir zum nächsten Schritt übergehen können und 'S' an die Position (7,6) bewegen können.} \\
            }
        }
    }
     \\ \\

    \theutterance \stepcounter{utterance}  
    & & & \multicolumn{2}{p{0.3\linewidth}}{
        \cellcolor[rgb]{0.9,0.9,0.9}{
            \makecell[{{p{\linewidth}}}]{
                \texttt{\tiny{[GM$|$GM]}}
                \texttt{SPRICH: Hast du 'O' erfolgreich an die Position (5,4) bewegt? Bitte bestätige, damit wir zum nächsten Schritt übergehen können und 'S' an die Position (7,6) bewegen können.} \\
            }
        }
    }
    & & \\ \\

    \theutterance \stepcounter{utterance}  
    & & \multicolumn{4}{p{0.6\linewidth}}{
        \cellcolor[rgb]{0.9,0.9,0.9}{
            \makecell[{{p{\linewidth}}}]{
                \texttt{\tiny{[P1$\langle$GM]}}
                \texttt{'O' erfolgreich nach (5,4) verschoben. Dein Raster sieht jetzt so aus:} \\
\texttt{```} \\
\texttt{ 1234567} \\
\texttt{╔═╤═╤═╤═╗} \\
\texttt{║◌│◌│◌│◌║ 1} \\
\texttt{╟─┴─┼─┤◌║ 2} \\
\texttt{║◌◌M│◌└─╢ 3} \\
\texttt{╟───┘OS◌║ 4} \\
\texttt{║◌A◌◌D┌─╢ 5} \\
\texttt{║◌◌◌PT│◌║ 6} \\
\texttt{║◌◌◌◌◌│◌║ 7} \\
\texttt{╚═════╧═╝} \\
\texttt{```} \\
\texttt{Ihr seid gerade bei Runde 19 von maximal 28.} \\
\texttt{Ihr habt gemeinsam 8 von maximal 16 Strafpunkten angesammelt.} \\
\texttt{Dein Mitspieler hat dir diese Nachricht geschickt:} \\
\\ 
\\ 
\texttt{"Hast du 'O' erfolgreich an die Position (5,4) bewegt? Bitte bestätige, damit wir zum nächsten Schritt übergehen können und 'S' an die Position (7,6) bewegen können."} \\
\\ 
\\ 
\texttt{Wie lautet dein nächster Befehl?} \\
            }
        }
    }
    & \\ \\

    \theutterance \stepcounter{utterance}  
    & \multicolumn{4}{p{0.6\linewidth}}{
        \cellcolor[rgb]{0.9,0.9,0.9}{
            \makecell[{{p{\linewidth}}}]{
                \texttt{\tiny{[P1$\rangle$GM]}}
                \texttt{SPRICH: Ja, ich habe 'O' erfolgreich an die Position (5,4) bewegt. Mein Raster sieht jetzt so aus, dass 'O' an der richtigen Stelle ist. Bitte geh zum nächsten Schritt über. Ich werde jetzt 'S' an die Position (7,6) bewegen.} \\
            }
        }
    }
    & & \\ \\

    \theutterance \stepcounter{utterance}  
    & & & \multicolumn{2}{p{0.3\linewidth}}{
        \cellcolor[rgb]{0.9,0.9,0.9}{
            \makecell[{{p{\linewidth}}}]{
                \texttt{\tiny{[GM$|$GM]}}
                \texttt{SPRICH: Ja, ich habe 'O' erfolgreich an die Position (5,4) bewegt. Mein Raster sieht jetzt so aus, dass 'O' an der richtigen Stelle ist. Bitte geh zum nächsten Schritt über. Ich werde jetzt 'S' an die Position (7,6) bewegen.} \\
            }
        }
    }
    & & \\ \\

    \theutterance \stepcounter{utterance}  
    & & \multicolumn{4}{p{0.6\linewidth}}{
        \cellcolor[rgb]{0.9,0.9,0.9}{
            \makecell[{{p{\linewidth}}}]{
                \texttt{\tiny{[GM$\rangle$P2]}}
                \texttt{Deine Nachricht wurde an deinen Mitspieler übermittelt.} \\
\\ 
\\ 
\texttt{Ihr seid gerade bei Runde 20 von maximal 28.} \\
\texttt{Ihr habt gemeinsam 8 von maximal 16 Strafpunkten angesammelt.} \\
\texttt{Dein Mitspieler hat dir diese Nachricht geschickt:} \\
\\ 
\\ 
\texttt{"Ja, ich habe 'O' erfolgreich an die Position (5,4) bewegt. Mein Raster sieht jetzt so aus, dass 'O' an der richtigen Stelle ist. Bitte geh zum nächsten Schritt über. Ich werde jetzt 'S' an die Position (7,6) bewegen."} \\
\\ 
\\ 
\texttt{Wie lautet dein nächster Befehl?} \\
            }
        }
    }
    & \\ \\

    \theutterance \stepcounter{utterance}  
    & & & \multicolumn{4}{p{0.6\linewidth}}{
        \cellcolor[rgb]{0.9,0.9,0.9}{
            \makecell[{{p{\linewidth}}}]{
                \texttt{\tiny{[GM$\langle$P2]}}
                \texttt{SPRICH: Okay, ich warte. Bitte teile mir mit, wenn du 'S' erfolgreich an die Position (7,6) bewegt hast. Dann können wir überprüfen, ob unsere Raster synchronisiert sind und zum nächsten Schritt übergehen.} \\
            }
        }
    }
     \\ \\

    \theutterance \stepcounter{utterance}  
    & & & \multicolumn{2}{p{0.3\linewidth}}{
        \cellcolor[rgb]{0.9,0.9,0.9}{
            \makecell[{{p{\linewidth}}}]{
                \texttt{\tiny{[GM$|$GM]}}
                \texttt{SPRICH: Okay, ich warte. Bitte teile mir mit, wenn du 'S' erfolgreich an die Position (7,6) bewegt hast. Dann können wir überprüfen, ob unsere Raster synchronisiert sind und zum nächsten Schritt übergehen.} \\
            }
        }
    }
    & & \\ \\

    \theutterance \stepcounter{utterance}  
    & & \multicolumn{4}{p{0.6\linewidth}}{
        \cellcolor[rgb]{0.9,0.9,0.9}{
            \makecell[{{p{\linewidth}}}]{
                \texttt{\tiny{[P1$\langle$GM]}}
                \texttt{Deine Nachricht wurde an deinen Mitspieler übermittelt.} \\
\\ 
\\ 
\texttt{Ihr seid gerade bei Runde 20 von maximal 28.} \\
\texttt{Ihr habt gemeinsam 8 von maximal 16 Strafpunkten angesammelt.} \\
\texttt{Dein Mitspieler hat dir diese Nachricht geschickt:} \\
\\ 
\\ 
\texttt{"Okay, ich warte. Bitte teile mir mit, wenn du 'S' erfolgreich an die Position (7,6) bewegt hast. Dann können wir überprüfen, ob unsere Raster synchronisiert sind und zum nächsten Schritt übergehen."} \\
\\ 
\\ 
\texttt{Wie lautet dein nächster Befehl?} \\
            }
        }
    }
    & \\ \\

    \theutterance \stepcounter{utterance}  
    & \multicolumn{4}{p{0.6\linewidth}}{
        \cellcolor[rgb]{0.9,0.9,0.9}{
            \makecell[{{p{\linewidth}}}]{
                \texttt{\tiny{[P1$\rangle$GM]}}
                \texttt{VERSCHIEBE: S, (7, 6)} \\
            }
        }
    }
    & & \\ \\

    \theutterance \stepcounter{utterance}  
    & & & \multicolumn{2}{p{0.3\linewidth}}{
        \cellcolor[rgb]{0.9,0.9,0.9}{
            \makecell[{{p{\linewidth}}}]{
                \texttt{\tiny{[GM$|$GM]}}
                \texttt{VERSCHIEBE: S, (7, 6)} \\
            }
        }
    }
    & & \\ \\

    \theutterance \stepcounter{utterance}  
    & & & \multicolumn{2}{p{0.3\linewidth}}{
        \cellcolor[rgb]{0.9,0.9,0.9}{
            \makecell[{{p{\linewidth}}}]{
                \texttt{\tiny{[GM$|$GM]}}
                \texttt{'S' erfolgreich nach (7,6) verschoben. Dein Raster sieht jetzt so aus:} \\
\texttt{```} \\
\texttt{ 1234567} \\
\texttt{╔═╤═╤═╤═╗} \\
\texttt{║◌│◌│◌│◌║ 1} \\
\texttt{╟─┴─┼─┤◌║ 2} \\
\texttt{║◌◌M│◌└─╢ 3} \\
\texttt{╟───┘O◌◌║ 4} \\
\texttt{║◌A◌◌D┌─╢ 5} \\
\texttt{║◌◌◌PT│S║ 6} \\
\texttt{║◌◌◌◌◌│◌║ 7} \\
\texttt{╚═════╧═╝} \\
\texttt{```} \\
            }
        }
    }
    & & \\ \\

    \theutterance \stepcounter{utterance}  
    & & \multicolumn{4}{p{0.6\linewidth}}{
        \cellcolor[rgb]{0.9,0.9,0.9}{
            \makecell[{{p{\linewidth}}}]{
                \texttt{\tiny{[GM$\rangle$P2]}}
                \texttt{Deine Nachricht wurde an deinen Mitspieler übermittelt.} \\
\\ 
\\ 
\texttt{Ihr seid gerade bei Runde 21 von maximal 28.} \\
\texttt{Ihr habt gemeinsam 8 von maximal 16 Strafpunkten angesammelt.} \\
\texttt{Dein Mitspieler hat ein Objekt auf seinem Raster bewegt.} \\
\\ 
\\ 
\texttt{Wie lautet dein nächster Befehl?} \\
            }
        }
    }
    & \\ \\

    \theutterance \stepcounter{utterance}  
    & & & \multicolumn{4}{p{0.6\linewidth}}{
        \cellcolor[rgb]{0.9,0.9,0.9}{
            \makecell[{{p{\linewidth}}}]{
                \texttt{\tiny{[GM$\langle$P2]}}
                \texttt{SPRICH: Hast du 'S' erfolgreich an die Position (7,6) bewegt? Bitte bestätige, damit wir überprüfen können, ob unsere Raster synchronisiert sind und zum nächsten Schritt übergehen können. Wir sollten überprüfen, ob alle Objekte an den gleichen Positionen sind.} \\
            }
        }
    }
     \\ \\

    \theutterance \stepcounter{utterance}  
    & & & \multicolumn{2}{p{0.3\linewidth}}{
        \cellcolor[rgb]{0.9,0.9,0.9}{
            \makecell[{{p{\linewidth}}}]{
                \texttt{\tiny{[GM$|$GM]}}
                \texttt{SPRICH: Hast du 'S' erfolgreich an die Position (7,6) bewegt? Bitte bestätige, damit wir überprüfen können, ob unsere Raster synchronisiert sind und zum nächsten Schritt übergehen können. Wir sollten überprüfen, ob alle Objekte an den gleichen Positionen sind.} \\
            }
        }
    }
    & & \\ \\

    \theutterance \stepcounter{utterance}  
    & & \multicolumn{4}{p{0.6\linewidth}}{
        \cellcolor[rgb]{0.9,0.9,0.9}{
            \makecell[{{p{\linewidth}}}]{
                \texttt{\tiny{[P1$\langle$GM]}}
                \texttt{'S' erfolgreich nach (7,6) verschoben. Dein Raster sieht jetzt so aus:} \\
\texttt{```} \\
\texttt{ 1234567} \\
\texttt{╔═╤═╤═╤═╗} \\
\texttt{║◌│◌│◌│◌║ 1} \\
\texttt{╟─┴─┼─┤◌║ 2} \\
\texttt{║◌◌M│◌└─╢ 3} \\
\texttt{╟───┘O◌◌║ 4} \\
\texttt{║◌A◌◌D┌─╢ 5} \\
\texttt{║◌◌◌PT│S║ 6} \\
\texttt{║◌◌◌◌◌│◌║ 7} \\
\texttt{╚═════╧═╝} \\
\texttt{```} \\
\texttt{Ihr seid gerade bei Runde 21 von maximal 28.} \\
\texttt{Ihr habt gemeinsam 8 von maximal 16 Strafpunkten angesammelt.} \\
\texttt{Dein Mitspieler hat dir diese Nachricht geschickt:} \\
\\ 
\\ 
\texttt{"Hast du 'S' erfolgreich an die Position (7,6) bewegt? Bitte bestätige, damit wir überprüfen können, ob unsere Raster synchronisiert sind und zum nächsten Schritt übergehen können. Wir sollten überprüfen, ob alle Objekte an den gleichen Positionen sind."} \\
\\ 
\\ 
\texttt{Wie lautet dein nächster Befehl?} \\
            }
        }
    }
    & \\ \\

    \theutterance \stepcounter{utterance}  
    & \multicolumn{4}{p{0.6\linewidth}}{
        \cellcolor[rgb]{0.9,0.9,0.9}{
            \makecell[{{p{\linewidth}}}]{
                \texttt{\tiny{[P1$\rangle$GM]}}
                \texttt{SPRICH: Ja, ich habe 'S' erfolgreich an die Position (7,6) bewegt. Mein Raster sieht jetzt so aus, dass 'S' an der richtigen Stelle ist. Lass uns unsere Raster überprüfen, um zu sehen, ob sie synchronisiert sind. Wir sollten alle Objekte vergleichen und überprüfen, ob sie an den gleichen Positionen sind. Ich denke, wir sind nah dran, unser Ziel zu erreichen. Bitte teile mir mit, wie unsere Raster aussehen und ob wir fertig sind.} \\
            }
        }
    }
    & & \\ \\

    \theutterance \stepcounter{utterance}  
    & & & \multicolumn{2}{p{0.3\linewidth}}{
        \cellcolor[rgb]{0.9,0.9,0.9}{
            \makecell[{{p{\linewidth}}}]{
                \texttt{\tiny{[GM$|$GM]}}
                \texttt{SPRICH: Ja, ich habe 'S' erfolgreich an die Position (7,6) bewegt. Mein Raster sieht jetzt so aus, dass 'S' an der richtigen Stelle ist. Lass uns unsere Raster überprüfen, um zu sehen, ob sie synchronisiert sind. Wir sollten alle Objekte vergleichen und überprüfen, ob sie an den gleichen Positionen sind. Ich denke, wir sind nah dran, unser Ziel zu erreichen. Bitte teile mir mit, wie unsere Raster aussehen und ob wir fertig sind.} \\
            }
        }
    }
    & & \\ \\

    \theutterance \stepcounter{utterance}  
    & & \multicolumn{4}{p{0.6\linewidth}}{
        \cellcolor[rgb]{0.9,0.9,0.9}{
            \makecell[{{p{\linewidth}}}]{
                \texttt{\tiny{[GM$\rangle$P2]}}
                \texttt{Deine Nachricht wurde an deinen Mitspieler übermittelt.} \\
\\ 
\\ 
\texttt{Ihr seid gerade bei Runde 22 von maximal 28.} \\
\texttt{Ihr habt gemeinsam 8 von maximal 16 Strafpunkten angesammelt.} \\
\texttt{Dein Mitspieler hat dir diese Nachricht geschickt:} \\
\\ 
\\ 
\texttt{"Ja, ich habe 'S' erfolgreich an die Position (7,6) bewegt. Mein Raster sieht jetzt so aus, dass 'S' an der richtigen Stelle ist. Lass uns unsere Raster überprüfen, um zu sehen, ob sie synchronisiert sind. Wir sollten alle Objekte vergleichen und überprüfen, ob sie an den gleichen Positionen sind. Ich denke, wir sind nah dran, unser Ziel zu erreichen. Bitte teile mir mit, wie unsere Raster aussehen und ob wir fertig sind."} \\
\\ 
\\ 
\texttt{Wie lautet dein nächster Befehl?} \\
            }
        }
    }
    & \\ \\

    \theutterance \stepcounter{utterance}  
    & & & \multicolumn{4}{p{0.6\linewidth}}{
        \cellcolor[rgb]{0.9,0.9,0.9}{
            \makecell[{{p{\linewidth}}}]{
                \texttt{\tiny{[GM$\langle$P2]}}
                \texttt{SPRICH: Mein Raster sieht so aus, dass 'A' an (2,5), 'D' an (6,2), 'M' an (3,3), 'O' an (1,5), 'P' an (4,4), 'S' an (7,6) und 'T' an (7,7) ist. Bitte teile mir mit, wie dein Raster aussieht und ob wir unsere Ziele erreicht haben. Wenn ja, können wir das Spiel beenden und "fertig" sagen.} \\
            }
        }
    }
     \\ \\

    \theutterance \stepcounter{utterance}  
    & & & \multicolumn{2}{p{0.3\linewidth}}{
        \cellcolor[rgb]{0.9,0.9,0.9}{
            \makecell[{{p{\linewidth}}}]{
                \texttt{\tiny{[GM$|$GM]}}
                \texttt{SPRICH: Mein Raster sieht so aus, dass 'A' an (2,5), 'D' an (6,2), 'M' an (3,3), 'O' an (1,5), 'P' an (4,4), 'S' an (7,6) und 'T' an (7,7) ist. Bitte teile mir mit, wie dein Raster aussieht und ob wir unsere Ziele erreicht haben. Wenn ja, können wir das Spiel beenden und "fertig" sagen.} \\
            }
        }
    }
    & & \\ \\

    \theutterance \stepcounter{utterance}  
    & & \multicolumn{4}{p{0.6\linewidth}}{
        \cellcolor[rgb]{0.9,0.9,0.9}{
            \makecell[{{p{\linewidth}}}]{
                \texttt{\tiny{[P1$\langle$GM]}}
                \texttt{Deine Nachricht wurde an deinen Mitspieler übermittelt.} \\
\\ 
\\ 
\texttt{Ihr seid gerade bei Runde 22 von maximal 28.} \\
\texttt{Ihr habt gemeinsam 8 von maximal 16 Strafpunkten angesammelt.} \\
\texttt{Dein Mitspieler hat dir diese Nachricht geschickt:} \\
\\ 
\\ 
\texttt{"Mein Raster sieht so aus, dass 'A' an (2,5), 'D' an (6,2), 'M' an (3,3), 'O' an (1,5), 'P' an (4,4), 'S' an (7,6) und 'T' an (7,7) ist. Bitte teile mir mit, wie dein Raster aussieht und ob wir unsere Ziele erreicht haben. Wenn ja, können wir das Spiel beenden und "fertig" sagen."} \\
\\ 
\\ 
\texttt{Wie lautet dein nächster Befehl?} \\
            }
        }
    }
    & \\ \\

    \theutterance \stepcounter{utterance}  
    & \multicolumn{4}{p{0.6\linewidth}}{
        \cellcolor[rgb]{0.9,0.9,0.9}{
            \makecell[{{p{\linewidth}}}]{
                \texttt{\tiny{[P1$\rangle$GM]}}
                \texttt{SPRICH: Mein Raster sieht so aus, dass 'A' an (2,5), 'D' an (5,5), 'M' an (3,3), 'O' an (5,4), 'P' an (6,6), 'S' an (7,6) und 'T' an (6,6) ist. Es sieht so aus, als ob unsere Raster nicht vollständig synchronisiert sind. Wir sollten noch einige Anpassungen vornehmen, um unsere Ziele zu erreichen. Ich denke, wir sollten noch einige Runden spielen, um unsere Raster zu synchronisieren. Wie wäre es, wenn wir 'P' und 'T' anpassen, um sie an die gleichen Positionen wie bei dir zu bringen? Bitte teile mir mit, wie wir vorgehen sollen.} \\
            }
        }
    }
    & & \\ \\

    \theutterance \stepcounter{utterance}  
    & & & \multicolumn{2}{p{0.3\linewidth}}{
        \cellcolor[rgb]{0.9,0.9,0.9}{
            \makecell[{{p{\linewidth}}}]{
                \texttt{\tiny{[GM$|$GM]}}
                \texttt{SPRICH: Mein Raster sieht so aus, dass 'A' an (2,5), 'D' an (5,5), 'M' an (3,3), 'O' an (5,4), 'P' an (6,6), 'S' an (7,6) und 'T' an (6,6) ist. Es sieht so aus, als ob unsere Raster nicht vollständig synchronisiert sind. Wir sollten noch einige Anpassungen vornehmen, um unsere Ziele zu erreichen. Ich denke, wir sollten noch einige Runden spielen, um unsere Raster zu synchronisieren. Wie wäre es, wenn wir 'P' und 'T' anpassen, um sie an die gleichen Positionen wie bei dir zu bringen? Bitte teile mir mit, wie wir vorgehen sollen.} \\
            }
        }
    }
    & & \\ \\

    \theutterance \stepcounter{utterance}  
    & & \multicolumn{4}{p{0.6\linewidth}}{
        \cellcolor[rgb]{0.9,0.9,0.9}{
            \makecell[{{p{\linewidth}}}]{
                \texttt{\tiny{[GM$\rangle$P2]}}
                \texttt{Deine Nachricht wurde an deinen Mitspieler übermittelt.} \\
\\ 
\\ 
\texttt{Ihr seid gerade bei Runde 23 von maximal 28.} \\
\texttt{Ihr habt gemeinsam 8 von maximal 16 Strafpunkten angesammelt.} \\
\texttt{Dein Mitspieler hat dir diese Nachricht geschickt:} \\
\\ 
\\ 
\texttt{"Mein Raster sieht so aus, dass 'A' an (2,5), 'D' an (5,5), 'M' an (3,3), 'O' an (5,4), 'P' an (6,6), 'S' an (7,6) und 'T' an (6,6) ist. Es sieht so aus, als ob unsere Raster nicht vollständig synchronisiert sind. Wir sollten noch einige Anpassungen vornehmen, um unsere Ziele zu erreichen. Ich denke, wir sollten noch einige Runden spielen, um unsere Raster zu synchronisieren. Wie wäre es, wenn wir 'P' und 'T' anpassen, um sie an die gleichen Positionen wie bei dir zu bringen? Bitte teile mir mit, wie wir vorgehen sollen."} \\
\\ 
\\ 
\texttt{Wie lautet dein nächster Befehl?} \\
            }
        }
    }
    & \\ \\

    \theutterance \stepcounter{utterance}  
    & & & \multicolumn{4}{p{0.6\linewidth}}{
        \cellcolor[rgb]{0.9,0.9,0.9}{
            \makecell[{{p{\linewidth}}}]{
                \texttt{\tiny{[GM$\langle$P2]}}
                \texttt{SPRICH: Ich denke, wir sollten 'P' an die Position (4,4) und 'T' an die Position (7,7) bewegen, um sie an die gleichen Positionen wie bei mir zu bringen. Bitte teile mir mit, ob das möglich ist und wie wir vorgehen sollen, um unsere Raster zu synchronisieren. Wir sollten auch überprüfen, ob 'D' an der Position (5,5) bleiben kann oder ob es an die Position (6,2) bewegt werden sollte, um es an die gleiche Position wie bei mir zu bringen.} \\
            }
        }
    }
     \\ \\

    \theutterance \stepcounter{utterance}  
    & & & \multicolumn{2}{p{0.3\linewidth}}{
        \cellcolor[rgb]{0.9,0.9,0.9}{
            \makecell[{{p{\linewidth}}}]{
                \texttt{\tiny{[GM$|$GM]}}
                \texttt{SPRICH: Ich denke, wir sollten 'P' an die Position (4,4) und 'T' an die Position (7,7) bewegen, um sie an die gleichen Positionen wie bei mir zu bringen. Bitte teile mir mit, ob das möglich ist und wie wir vorgehen sollen, um unsere Raster zu synchronisieren. Wir sollten auch überprüfen, ob 'D' an der Position (5,5) bleiben kann oder ob es an die Position (6,2) bewegt werden sollte, um es an die gleiche Position wie bei mir zu bringen.} \\
            }
        }
    }
    & & \\ \\

    \theutterance \stepcounter{utterance}  
    & & \multicolumn{4}{p{0.6\linewidth}}{
        \cellcolor[rgb]{0.9,0.9,0.9}{
            \makecell[{{p{\linewidth}}}]{
                \texttt{\tiny{[P1$\langle$GM]}}
                \texttt{Deine Nachricht wurde an deinen Mitspieler übermittelt.} \\
\\ 
\\ 
\texttt{Ihr seid gerade bei Runde 23 von maximal 28.} \\
\texttt{Ihr habt gemeinsam 8 von maximal 16 Strafpunkten angesammelt.} \\
\texttt{Dein Mitspieler hat dir diese Nachricht geschickt:} \\
\\ 
\\ 
\texttt{"Ich denke, wir sollten 'P' an die Position (4,4) und 'T' an die Position (7,7) bewegen, um sie an die gleichen Positionen wie bei mir zu bringen. Bitte teile mir mit, ob das möglich ist und wie wir vorgehen sollen, um unsere Raster zu synchronisieren. Wir sollten auch überprüfen, ob 'D' an der Position (5,5) bleiben kann oder ob es an die Position (6,2) bewegt werden sollte, um es an die gleiche Position wie bei mir zu bringen."} \\
\\ 
\\ 
\texttt{Wie lautet dein nächster Befehl?} \\
            }
        }
    }
    & \\ \\

    \theutterance \stepcounter{utterance}  
    & \multicolumn{4}{p{0.6\linewidth}}{
        \cellcolor[rgb]{0.9,0.9,0.9}{
            \makecell[{{p{\linewidth}}}]{
                \texttt{\tiny{[P1$\rangle$GM]}}
                \texttt{SPRICH: Das klingt wie ein guter Plan! Ich denke, wir sollten 'P' an die Position (4,4) und 'T' an die Position (7,7) bewegen, um sie an die gleichen Positionen wie bei dir zu bringen. Ich werde 'P' von (6,6) an die Position (4,4) bewegen und 'T' von (6,6) an die Position (7,7) bewegen. Bezüglich 'D' denke ich, es sollte an die Position (6,2) bewegt werden, um es an die gleiche Position wie bei dir zu bringen. Ich werde 'D' von (5,5) an die Position (6,2) bewegen. Bitte warte, bis ich bestätige, dass ich die Änderungen vorgenommen habe, bevor wir zum nächsten Schritt übergehen. Ich werde jetzt die Änderungen vornehmen.} \\
            }
        }
    }
    & & \\ \\

    \theutterance \stepcounter{utterance}  
    & & & \multicolumn{2}{p{0.3\linewidth}}{
        \cellcolor[rgb]{0.9,0.9,0.9}{
            \makecell[{{p{\linewidth}}}]{
                \texttt{\tiny{[GM$|$GM]}}
                \texttt{SPRICH: Das klingt wie ein guter Plan! Ich denke, wir sollten 'P' an die Position (4,4) und 'T' an die Position (7,7) bewegen, um sie an die gleichen Positionen wie bei dir zu bringen. Ich werde 'P' von (6,6) an die Position (4,4) bewegen und 'T' von (6,6) an die Position (7,7) bewegen. Bezüglich 'D' denke ich, es sollte an die Position (6,2) bewegt werden, um es an die gleiche Position wie bei dir zu bringen. Ich werde 'D' von (5,5) an die Position (6,2) bewegen. Bitte warte, bis ich bestätige, dass ich die Änderungen vorgenommen habe, bevor wir zum nächsten Schritt übergehen. Ich werde jetzt die Änderungen vornehmen.} \\
            }
        }
    }
    & & \\ \\

    \theutterance \stepcounter{utterance}  
    & & \multicolumn{4}{p{0.6\linewidth}}{
        \cellcolor[rgb]{0.9,0.9,0.9}{
            \makecell[{{p{\linewidth}}}]{
                \texttt{\tiny{[GM$\rangle$P2]}}
                \texttt{Deine Nachricht wurde an deinen Mitspieler übermittelt.} \\
\\ 
\\ 
\texttt{Ihr seid gerade bei Runde 24 von maximal 28.} \\
\texttt{Ihr habt gemeinsam 8 von maximal 16 Strafpunkten angesammelt.} \\
\texttt{Dein Mitspieler hat dir diese Nachricht geschickt:} \\
\\ 
\\ 
\texttt{"Das klingt wie ein guter Plan! Ich denke, wir sollten 'P' an die Position (4,4) und 'T' an die Position (7,7) bewegen, um sie an die gleichen Positionen wie bei dir zu bringen. Ich werde 'P' von (6,6) an die Position (4,4) bewegen und 'T' von (6,6) an die Position (7,7) bewegen. Bezüglich 'D' denke ich, es sollte an die Position (6,2) bewegt werden, um es an die gleiche Position wie bei dir zu bringen. Ich werde 'D' von (5,5) an die Position (6,2) bewegen. Bitte warte, bis ich bestätige, dass ich die Änderungen vorgenommen habe, bevor wir zum nächsten Schritt übergehen. Ich werde jetzt die Änderungen vornehmen."} \\
\\ 
\\ 
\texttt{Wie lautet dein nächster Befehl?} \\
            }
        }
    }
    & \\ \\

    \theutterance \stepcounter{utterance}  
    & & & \multicolumn{4}{p{0.6\linewidth}}{
        \cellcolor[rgb]{0.9,0.9,0.9}{
            \makecell[{{p{\linewidth}}}]{
                \texttt{\tiny{[GM$\langle$P2]}}
                \texttt{SPRICH: Okay, ich warte. Bitte teile mir mit, wenn du 'P', 'T' und 'D' erfolgreich an die Positionen (4,4), (7,7) und (6,2) bewegt hast. Dann können wir unsere Raster noch einmal überprüfen, um zu sehen, ob sie jetzt vollständig synchronisiert sind. Wenn ja, können wir das Spiel beenden und "fertig" sagen.} \\
            }
        }
    }
     \\ \\

    \theutterance \stepcounter{utterance}  
    & & & \multicolumn{2}{p{0.3\linewidth}}{
        \cellcolor[rgb]{0.9,0.9,0.9}{
            \makecell[{{p{\linewidth}}}]{
                \texttt{\tiny{[GM$|$GM]}}
                \texttt{SPRICH: Okay, ich warte. Bitte teile mir mit, wenn du 'P', 'T' und 'D' erfolgreich an die Positionen (4,4), (7,7) und (6,2) bewegt hast. Dann können wir unsere Raster noch einmal überprüfen, um zu sehen, ob sie jetzt vollständig synchronisiert sind. Wenn ja, können wir das Spiel beenden und "fertig" sagen.} \\
            }
        }
    }
    & & \\ \\

    \theutterance \stepcounter{utterance}  
    & & \multicolumn{4}{p{0.6\linewidth}}{
        \cellcolor[rgb]{0.9,0.9,0.9}{
            \makecell[{{p{\linewidth}}}]{
                \texttt{\tiny{[P1$\langle$GM]}}
                \texttt{Deine Nachricht wurde an deinen Mitspieler übermittelt.} \\
\\ 
\\ 
\texttt{Ihr seid gerade bei Runde 24 von maximal 28.} \\
\texttt{Ihr habt gemeinsam 8 von maximal 16 Strafpunkten angesammelt.} \\
\texttt{Dein Mitspieler hat dir diese Nachricht geschickt:} \\
\\ 
\\ 
\texttt{"Okay, ich warte. Bitte teile mir mit, wenn du 'P', 'T' und 'D' erfolgreich an die Positionen (4,4), (7,7) und (6,2) bewegt hast. Dann können wir unsere Raster noch einmal überprüfen, um zu sehen, ob sie jetzt vollständig synchronisiert sind. Wenn ja, können wir das Spiel beenden und "fertig" sagen."} \\
\\ 
\\ 
\texttt{Wie lautet dein nächster Befehl?} \\
            }
        }
    }
    & \\ \\

    \theutterance \stepcounter{utterance}  
    & \multicolumn{4}{p{0.6\linewidth}}{
        \cellcolor[rgb]{0.9,0.9,0.9}{
            \makecell[{{p{\linewidth}}}]{
                \texttt{\tiny{[P1$\rangle$GM]}}
                \texttt{VERSCHIEBE: P, (4, 4)} \\
\\ 
\texttt{VERSCHIEBE: T, (7, 7)} \\
\\ 
\texttt{VERSCHIEBE: D, (6, 2)} \\
            }
        }
    }
    & & \\ \\

    \theutterance \stepcounter{utterance}  
    & & & \multicolumn{2}{p{0.3\linewidth}}{
        \cellcolor[rgb]{0.9,0.9,0.9}{
            \makecell[{{p{\linewidth}}}]{
                \texttt{\tiny{[GM$|$GM]}}
                \texttt{VERSCHIEBE: P, (4, 4)} \\
\\ 
\texttt{VERSCHIEBE: T, (7, 7)} \\
\\ 
\texttt{VERSCHIEBE: D, (6, 2)} \\
            }
        }
    }
    & & \\ \\

    \theutterance \stepcounter{utterance}  
    & & & \multicolumn{2}{p{0.3\linewidth}}{
        \cellcolor[rgb]{0.9,0.9,0.9}{
            \makecell[{{p{\linewidth}}}]{
                \texttt{\tiny{[GM$|$GM]}}
                \texttt{Invalid format: head is not empty: 'VERSCHIEBE: P, (4, 4)} \\
\\ 
\texttt{VERSCHIEBE: T, (7, 7)} \\
\\ 
\texttt{'} \\
            }
        }
    }
    & & \\ \\

    \theutterance \stepcounter{utterance}  
    & & \multicolumn{4}{p{0.6\linewidth}}{
        \cellcolor[rgb]{0.9,0.9,0.9}{
            \makecell[{{p{\linewidth}}}]{
                \texttt{\tiny{[P1$\langle$GM]}}
                \texttt{Strafpunkt: Deine Nachricht darf nichts vor dem Befehl enthalten! Stelle sicher, dass deine Antwort ausschließlich einen der beiden Befehle `SPRICH: <NACHRICHT>` und `VERSCHIEBE: <OBJECT>, (<X>, <Y>)` und sonst nichts enthält!} \\
\\ 
\texttt{Ihr habt gemeinsam 9 von maximal 16 Strafpunkten angesammelt.} \\
\texttt{Bitte versuche es noch mal!} \\
            }
        }
    }
    & \\ \\

    \theutterance \stepcounter{utterance}  
    & \multicolumn{4}{p{0.6\linewidth}}{
        \cellcolor[rgb]{0.9,0.9,0.9}{
            \makecell[{{p{\linewidth}}}]{
                \texttt{\tiny{[P1$\rangle$GM]}}
                \texttt{VERSCHIEBE: P, (4, 4)} \\
            }
        }
    }
    & & \\ \\

    \theutterance \stepcounter{utterance}  
    & & & \multicolumn{2}{p{0.3\linewidth}}{
        \cellcolor[rgb]{0.9,0.9,0.9}{
            \makecell[{{p{\linewidth}}}]{
                \texttt{\tiny{[GM$|$GM]}}
                \texttt{VERSCHIEBE: P, (4, 4)} \\
            }
        }
    }
    & & \\ \\

    \theutterance \stepcounter{utterance}  
    & & & \multicolumn{2}{p{0.3\linewidth}}{
        \cellcolor[rgb]{0.9,0.9,0.9}{
            \makecell[{{p{\linewidth}}}]{
                \texttt{\tiny{[GM$|$GM]}}
                \texttt{Strafpunkt: (4,4) ist nicht leer, sondern enthält '┘'.} \\
\\ 
\texttt{Ihr habt gemeinsam 10 von maximal 16 Strafpunkten angesammelt.} \\
\texttt{Bitte versuche es noch mal!} \\
            }
        }
    }
    & & \\ \\

    \theutterance \stepcounter{utterance}  
    & & \multicolumn{4}{p{0.6\linewidth}}{
        \cellcolor[rgb]{0.9,0.9,0.9}{
            \makecell[{{p{\linewidth}}}]{
                \texttt{\tiny{[P1$\langle$GM]}}
                \texttt{Strafpunkt: (4,4) ist nicht leer, sondern enthält '┘'.} \\
\\ 
\texttt{Ihr habt gemeinsam 10 von maximal 16 Strafpunkten angesammelt.} \\
\texttt{Bitte versuche es noch mal!} \\
            }
        }
    }
    & \\ \\

    \theutterance \stepcounter{utterance}  
    & \multicolumn{4}{p{0.6\linewidth}}{
        \cellcolor[rgb]{0.9,0.9,0.9}{
            \makecell[{{p{\linewidth}}}]{
                \texttt{\tiny{[P1$\rangle$GM]}}
                \texttt{VERSCHIEBE: D, (6, 2)} \\
            }
        }
    }
    & & \\ \\

    \theutterance \stepcounter{utterance}  
    & & & \multicolumn{2}{p{0.3\linewidth}}{
        \cellcolor[rgb]{0.9,0.9,0.9}{
            \makecell[{{p{\linewidth}}}]{
                \texttt{\tiny{[GM$|$GM]}}
                \texttt{VERSCHIEBE: D, (6, 2)} \\
            }
        }
    }
    & & \\ \\

    \theutterance \stepcounter{utterance}  
    & & & \multicolumn{2}{p{0.3\linewidth}}{
        \cellcolor[rgb]{0.9,0.9,0.9}{
            \makecell[{{p{\linewidth}}}]{
                \texttt{\tiny{[GM$|$GM]}}
                \texttt{Strafpunkt: (6,2) ist nicht leer, sondern enthält '┤'.} \\
\\ 
\texttt{Ihr habt gemeinsam 11 von maximal 16 Strafpunkten angesammelt.} \\
\texttt{Bitte versuche es noch mal!} \\
            }
        }
    }
    & & \\ \\

    \theutterance \stepcounter{utterance}  
    & & \multicolumn{4}{p{0.6\linewidth}}{
        \cellcolor[rgb]{0.9,0.9,0.9}{
            \makecell[{{p{\linewidth}}}]{
                \texttt{\tiny{[P1$\langle$GM]}}
                \texttt{Strafpunkt: (6,2) ist nicht leer, sondern enthält '┤'.} \\
\\ 
\texttt{Ihr habt gemeinsam 11 von maximal 16 Strafpunkten angesammelt.} \\
\texttt{Bitte versuche es noch mal!} \\
            }
        }
    }
    & \\ \\

    \theutterance \stepcounter{utterance}  
    & \multicolumn{4}{p{0.6\linewidth}}{
        \cellcolor[rgb]{0.9,0.9,0.9}{
            \makecell[{{p{\linewidth}}}]{
                \texttt{\tiny{[P1$\rangle$GM]}}
                \texttt{VERSCHIEBE: T, (7, 7)} \\
            }
        }
    }
    & & \\ \\

    \theutterance \stepcounter{utterance}  
    & & & \multicolumn{2}{p{0.3\linewidth}}{
        \cellcolor[rgb]{0.9,0.9,0.9}{
            \makecell[{{p{\linewidth}}}]{
                \texttt{\tiny{[GM$|$GM]}}
                \texttt{VERSCHIEBE: T, (7, 7)} \\
            }
        }
    }
    & & \\ \\

    \theutterance \stepcounter{utterance}  
    & & & \multicolumn{2}{p{0.3\linewidth}}{
        \cellcolor[rgb]{0.9,0.9,0.9}{
            \makecell[{{p{\linewidth}}}]{
                \texttt{\tiny{[GM$|$GM]}}
                \texttt{'T' erfolgreich nach (7,7) verschoben. Dein Raster sieht jetzt so aus:} \\
\texttt{```} \\
\texttt{ 1234567} \\
\texttt{╔═╤═╤═╤═╗} \\
\texttt{║◌│◌│◌│◌║ 1} \\
\texttt{╟─┴─┼─┤◌║ 2} \\
\texttt{║◌◌M│◌└─╢ 3} \\
\texttt{╟───┘O◌◌║ 4} \\
\texttt{║◌A◌◌D┌─╢ 5} \\
\texttt{║◌◌◌P◌│S║ 6} \\
\texttt{║◌◌◌◌◌│T║ 7} \\
\texttt{╚═════╧═╝} \\
\texttt{```} \\
            }
        }
    }
    & & \\ \\

    \theutterance \stepcounter{utterance}  
    & & \multicolumn{4}{p{0.6\linewidth}}{
        \cellcolor[rgb]{0.9,0.9,0.9}{
            \makecell[{{p{\linewidth}}}]{
                \texttt{\tiny{[GM$\rangle$P2]}}
                \texttt{Deine Nachricht wurde an deinen Mitspieler übermittelt.} \\
\\ 
\\ 
\texttt{Ihr seid gerade bei Runde 25 von maximal 28.} \\
\texttt{Ihr habt gemeinsam 11 von maximal 16 Strafpunkten angesammelt.} \\
\texttt{Dein Mitspieler hat ein Objekt auf seinem Raster bewegt.} \\
\\ 
\\ 
\texttt{Wie lautet dein nächster Befehl?} \\
            }
        }
    }
    & \\ \\

    \theutterance \stepcounter{utterance}  
    & & & \multicolumn{4}{p{0.6\linewidth}}{
        \cellcolor[rgb]{0.9,0.9,0.9}{
            \makecell[{{p{\linewidth}}}]{
                \texttt{\tiny{[GM$\langle$P2]}}
                \texttt{SPRICH: Hast du 'P', 'T' und 'D' erfolgreich an die Positionen (4,4), (7,7) und (6,2) bewegt? Bitte bestätige, damit wir unsere Raster überprüfen können, um zu sehen, ob sie jetzt vollständig synchronisiert sind. Wenn ja, können wir das Spiel beenden und "fertig" sagen.} \\
            }
        }
    }
     \\ \\

    \theutterance \stepcounter{utterance}  
    & & & \multicolumn{2}{p{0.3\linewidth}}{
        \cellcolor[rgb]{0.9,0.9,0.9}{
            \makecell[{{p{\linewidth}}}]{
                \texttt{\tiny{[GM$|$GM]}}
                \texttt{SPRICH: Hast du 'P', 'T' und 'D' erfolgreich an die Positionen (4,4), (7,7) und (6,2) bewegt? Bitte bestätige, damit wir unsere Raster überprüfen können, um zu sehen, ob sie jetzt vollständig synchronisiert sind. Wenn ja, können wir das Spiel beenden und "fertig" sagen.} \\
            }
        }
    }
    & & \\ \\

    \theutterance \stepcounter{utterance}  
    & & \multicolumn{4}{p{0.6\linewidth}}{
        \cellcolor[rgb]{0.9,0.9,0.9}{
            \makecell[{{p{\linewidth}}}]{
                \texttt{\tiny{[P1$\langle$GM]}}
                \texttt{'T' erfolgreich nach (7,7) verschoben. Dein Raster sieht jetzt so aus:} \\
\texttt{```} \\
\texttt{ 1234567} \\
\texttt{╔═╤═╤═╤═╗} \\
\texttt{║◌│◌│◌│◌║ 1} \\
\texttt{╟─┴─┼─┤◌║ 2} \\
\texttt{║◌◌M│◌└─╢ 3} \\
\texttt{╟───┘O◌◌║ 4} \\
\texttt{║◌A◌◌D┌─╢ 5} \\
\texttt{║◌◌◌P◌│S║ 6} \\
\texttt{║◌◌◌◌◌│T║ 7} \\
\texttt{╚═════╧═╝} \\
\texttt{```} \\
\texttt{Ihr seid gerade bei Runde 25 von maximal 28.} \\
\texttt{Ihr habt gemeinsam 11 von maximal 16 Strafpunkten angesammelt.} \\
\texttt{Dein Mitspieler hat dir diese Nachricht geschickt:} \\
\\ 
\\ 
\texttt{"Hast du 'P', 'T' und 'D' erfolgreich an die Positionen (4,4), (7,7) und (6,2) bewegt? Bitte bestätige, damit wir unsere Raster überprüfen können, um zu sehen, ob sie jetzt vollständig synchronisiert sind. Wenn ja, können wir das Spiel beenden und "fertig" sagen."} \\
\\ 
\\ 
\texttt{Wie lautet dein nächster Befehl?} \\
            }
        }
    }
    & \\ \\

    \theutterance \stepcounter{utterance}  
    & \multicolumn{4}{p{0.6\linewidth}}{
        \cellcolor[rgb]{0.9,0.9,0.9}{
            \makecell[{{p{\linewidth}}}]{
                \texttt{\tiny{[P1$\rangle$GM]}}
                \texttt{SPRICH: Nein, ich habe noch nicht alle Änderungen vorgenommen. Ich habe nur 'T' an die Position (7,7) bewegt. Ich werde jetzt 'P' und 'D' an die richtigen Positionen bewegen. Bitte warte, bis ich bestätige, dass ich die Änderungen vorgenommen habe, bevor wir zum nächsten Schritt übergehen.} \\
            }
        }
    }
    & & \\ \\

    \theutterance \stepcounter{utterance}  
    & & & \multicolumn{2}{p{0.3\linewidth}}{
        \cellcolor[rgb]{0.9,0.9,0.9}{
            \makecell[{{p{\linewidth}}}]{
                \texttt{\tiny{[GM$|$GM]}}
                \texttt{SPRICH: Nein, ich habe noch nicht alle Änderungen vorgenommen. Ich habe nur 'T' an die Position (7,7) bewegt. Ich werde jetzt 'P' und 'D' an die richtigen Positionen bewegen. Bitte warte, bis ich bestätige, dass ich die Änderungen vorgenommen habe, bevor wir zum nächsten Schritt übergehen.} \\
            }
        }
    }
    & & \\ \\

    \theutterance \stepcounter{utterance}  
    & & \multicolumn{4}{p{0.6\linewidth}}{
        \cellcolor[rgb]{0.9,0.9,0.9}{
            \makecell[{{p{\linewidth}}}]{
                \texttt{\tiny{[GM$\rangle$P2]}}
                \texttt{Deine Nachricht wurde an deinen Mitspieler übermittelt.} \\
\\ 
\\ 
\texttt{Ihr seid gerade bei Runde 26 von maximal 28.} \\
\texttt{Ihr habt gemeinsam 11 von maximal 16 Strafpunkten angesammelt.} \\
\texttt{Dein Mitspieler hat dir diese Nachricht geschickt:} \\
\\ 
\\ 
\texttt{"Nein, ich habe noch nicht alle Änderungen vorgenommen. Ich habe nur 'T' an die Position (7,7) bewegt. Ich werde jetzt 'P' und 'D' an die richtigen Positionen bewegen. Bitte warte, bis ich bestätige, dass ich die Änderungen vorgenommen habe, bevor wir zum nächsten Schritt übergehen."} \\
\\ 
\\ 
\texttt{Wie lautet dein nächster Befehl?} \\
            }
        }
    }
    & \\ \\

    \theutterance \stepcounter{utterance}  
    & & & \multicolumn{4}{p{0.6\linewidth}}{
        \cellcolor[rgb]{0.9,0.9,0.9}{
            \makecell[{{p{\linewidth}}}]{
                \texttt{\tiny{[GM$\langle$P2]}}
                \texttt{SPRICH: Okay, ich warte. Bitte teile mir mit, wenn du 'P' und 'D' erfolgreich an die Positionen (4,4) und (6,2) bewegt hast. Dann können wir unsere Raster überprüfen, um zu sehen, ob sie jetzt vollständig synchronisiert sind. Wenn ja, können wir das Spiel beenden und "fertig" sagen.} \\
            }
        }
    }
     \\ \\

    \theutterance \stepcounter{utterance}  
    & & & \multicolumn{2}{p{0.3\linewidth}}{
        \cellcolor[rgb]{0.9,0.9,0.9}{
            \makecell[{{p{\linewidth}}}]{
                \texttt{\tiny{[GM$|$GM]}}
                \texttt{SPRICH: Okay, ich warte. Bitte teile mir mit, wenn du 'P' und 'D' erfolgreich an die Positionen (4,4) und (6,2) bewegt hast. Dann können wir unsere Raster überprüfen, um zu sehen, ob sie jetzt vollständig synchronisiert sind. Wenn ja, können wir das Spiel beenden und "fertig" sagen.} \\
            }
        }
    }
    & & \\ \\

    \theutterance \stepcounter{utterance}  
    & & \multicolumn{4}{p{0.6\linewidth}}{
        \cellcolor[rgb]{0.9,0.9,0.9}{
            \makecell[{{p{\linewidth}}}]{
                \texttt{\tiny{[P1$\langle$GM]}}
                \texttt{Deine Nachricht wurde an deinen Mitspieler übermittelt.} \\
\\ 
\\ 
\texttt{Ihr seid gerade bei Runde 26 von maximal 28.} \\
\texttt{Ihr habt gemeinsam 11 von maximal 16 Strafpunkten angesammelt.} \\
\texttt{Dein Mitspieler hat dir diese Nachricht geschickt:} \\
\\ 
\\ 
\texttt{"Okay, ich warte. Bitte teile mir mit, wenn du 'P' und 'D' erfolgreich an die Positionen (4,4) und (6,2) bewegt hast. Dann können wir unsere Raster überprüfen, um zu sehen, ob sie jetzt vollständig synchronisiert sind. Wenn ja, können wir das Spiel beenden und "fertig" sagen."} \\
\\ 
\\ 
\texttt{Wie lautet dein nächster Befehl?} \\
            }
        }
    }
    & \\ \\

    \theutterance \stepcounter{utterance}  
    & \multicolumn{4}{p{0.6\linewidth}}{
        \cellcolor[rgb]{0.9,0.9,0.9}{
            \makecell[{{p{\linewidth}}}]{
                \texttt{\tiny{[P1$\rangle$GM]}}
                \texttt{VERSCHIEBE: D, (6, 2)} \\
            }
        }
    }
    & & \\ \\

    \theutterance \stepcounter{utterance}  
    & & & \multicolumn{2}{p{0.3\linewidth}}{
        \cellcolor[rgb]{0.9,0.9,0.9}{
            \makecell[{{p{\linewidth}}}]{
                \texttt{\tiny{[GM$|$GM]}}
                \texttt{VERSCHIEBE: D, (6, 2)} \\
            }
        }
    }
    & & \\ \\

    \theutterance \stepcounter{utterance}  
    & & & \multicolumn{2}{p{0.3\linewidth}}{
        \cellcolor[rgb]{0.9,0.9,0.9}{
            \makecell[{{p{\linewidth}}}]{
                \texttt{\tiny{[GM$|$GM]}}
                \texttt{Strafpunkt: (6,2) ist nicht leer, sondern enthält '┤'.} \\
\\ 
\texttt{Ihr habt gemeinsam 12 von maximal 16 Strafpunkten angesammelt.} \\
\texttt{Bitte versuche es noch mal!} \\
            }
        }
    }
    & & \\ \\

    \theutterance \stepcounter{utterance}  
    & & \multicolumn{4}{p{0.6\linewidth}}{
        \cellcolor[rgb]{0.9,0.9,0.9}{
            \makecell[{{p{\linewidth}}}]{
                \texttt{\tiny{[P1$\langle$GM]}}
                \texttt{Strafpunkt: (6,2) ist nicht leer, sondern enthält '┤'.} \\
\\ 
\texttt{Ihr habt gemeinsam 12 von maximal 16 Strafpunkten angesammelt.} \\
\texttt{Bitte versuche es noch mal!} \\
            }
        }
    }
    & \\ \\

    \theutterance \stepcounter{utterance}  
    & \multicolumn{4}{p{0.6\linewidth}}{
        \cellcolor[rgb]{0.9,0.9,0.9}{
            \makecell[{{p{\linewidth}}}]{
                \texttt{\tiny{[P1$\rangle$GM]}}
                \texttt{VERSCHIEBE: P, (4, 5)} \\
            }
        }
    }
    & & \\ \\

    \theutterance \stepcounter{utterance}  
    & & & \multicolumn{2}{p{0.3\linewidth}}{
        \cellcolor[rgb]{0.9,0.9,0.9}{
            \makecell[{{p{\linewidth}}}]{
                \texttt{\tiny{[GM$|$GM]}}
                \texttt{VERSCHIEBE: P, (4, 5)} \\
            }
        }
    }
    & & \\ \\

    \theutterance \stepcounter{utterance}  
    & & & \multicolumn{2}{p{0.3\linewidth}}{
        \cellcolor[rgb]{0.9,0.9,0.9}{
            \makecell[{{p{\linewidth}}}]{
                \texttt{\tiny{[GM$|$GM]}}
                \texttt{'P' erfolgreich nach (4,5) verschoben. Dein Raster sieht jetzt so aus:} \\
\texttt{```} \\
\texttt{ 1234567} \\
\texttt{╔═╤═╤═╤═╗} \\
\texttt{║◌│◌│◌│◌║ 1} \\
\texttt{╟─┴─┼─┤◌║ 2} \\
\texttt{║◌◌M│◌└─╢ 3} \\
\texttt{╟───┘O◌◌║ 4} \\
\texttt{║◌A◌PD┌─╢ 5} \\
\texttt{║◌◌◌◌◌│S║ 6} \\
\texttt{║◌◌◌◌◌│T║ 7} \\
\texttt{╚═════╧═╝} \\
\texttt{```} \\
            }
        }
    }
    & & \\ \\

    \theutterance \stepcounter{utterance}  
    & & \multicolumn{4}{p{0.6\linewidth}}{
        \cellcolor[rgb]{0.9,0.9,0.9}{
            \makecell[{{p{\linewidth}}}]{
                \texttt{\tiny{[GM$\rangle$P2]}}
                \texttt{Deine Nachricht wurde an deinen Mitspieler übermittelt.} \\
\\ 
\\ 
\texttt{Ihr seid gerade bei Runde 27 von maximal 28.} \\
\texttt{Ihr habt gemeinsam 12 von maximal 16 Strafpunkten angesammelt.} \\
\texttt{Dein Mitspieler hat ein Objekt auf seinem Raster bewegt.} \\
\\ 
\\ 
\texttt{Wie lautet dein nächster Befehl?} \\
            }
        }
    }
    & \\ \\

    \theutterance \stepcounter{utterance}  
    & & & \multicolumn{4}{p{0.6\linewidth}}{
        \cellcolor[rgb]{0.9,0.9,0.9}{
            \makecell[{{p{\linewidth}}}]{
                \texttt{\tiny{[GM$\langle$P2]}}
                \texttt{SPRICH: Hast du 'P' und 'D' erfolgreich an die Positionen (4,4) und (6,2) bewegt? Bitte bestätige, damit wir unsere Raster überprüfen können, um zu sehen, ob sie jetzt vollständig synchronisiert sind. Wenn ja, kann ich "fertig?" sagen und wir können das Spiel beenden, wenn du mit "fertig!" antwortest.} \\
            }
        }
    }
     \\ \\

    \theutterance \stepcounter{utterance}  
    & & & \multicolumn{2}{p{0.3\linewidth}}{
        \cellcolor[rgb]{0.9,0.9,0.9}{
            \makecell[{{p{\linewidth}}}]{
                \texttt{\tiny{[GM$|$GM]}}
                \texttt{SPRICH: Hast du 'P' und 'D' erfolgreich an die Positionen (4,4) und (6,2) bewegt? Bitte bestätige, damit wir unsere Raster überprüfen können, um zu sehen, ob sie jetzt vollständig synchronisiert sind. Wenn ja, kann ich "fertig?" sagen und wir können das Spiel beenden, wenn du mit "fertig!" antwortest.} \\
            }
        }
    }
    & & \\ \\

    \theutterance \stepcounter{utterance}  
    & & \multicolumn{4}{p{0.6\linewidth}}{
        \cellcolor[rgb]{0.9,0.9,0.9}{
            \makecell[{{p{\linewidth}}}]{
                \texttt{\tiny{[P1$\langle$GM]}}
                \texttt{'P' erfolgreich nach (4,5) verschoben. Dein Raster sieht jetzt so aus:} \\
\texttt{```} \\
\texttt{ 1234567} \\
\texttt{╔═╤═╤═╤═╗} \\
\texttt{║◌│◌│◌│◌║ 1} \\
\texttt{╟─┴─┼─┤◌║ 2} \\
\texttt{║◌◌M│◌└─╢ 3} \\
\texttt{╟───┘O◌◌║ 4} \\
\texttt{║◌A◌PD┌─╢ 5} \\
\texttt{║◌◌◌◌◌│S║ 6} \\
\texttt{║◌◌◌◌◌│T║ 7} \\
\texttt{╚═════╧═╝} \\
\texttt{```} \\
\texttt{Ihr seid gerade bei Runde 27 von maximal 28.} \\
\texttt{Ihr habt gemeinsam 12 von maximal 16 Strafpunkten angesammelt.} \\
\texttt{Dein Mitspieler hat dir diese Nachricht geschickt:} \\
\\ 
\\ 
\texttt{"Hast du 'P' und 'D' erfolgreich an die Positionen (4,4) und (6,2) bewegt? Bitte bestätige, damit wir unsere Raster überprüfen können, um zu sehen, ob sie jetzt vollständig synchronisiert sind. Wenn ja, kann ich "fertig?" sagen und wir können das Spiel beenden, wenn du mit "fertig!" antwortest."} \\
\\ 
\\ 
\texttt{Wie lautet dein nächster Befehl?} \\
            }
        }
    }
    & \\ \\

    \theutterance \stepcounter{utterance}  
    & \multicolumn{4}{p{0.6\linewidth}}{
        \cellcolor[rgb]{0.9,0.9,0.9}{
            \makecell[{{p{\linewidth}}}]{
                \texttt{\tiny{[P1$\rangle$GM]}}
                \texttt{SPRICH: Nein, ich habe 'P' an die Position (4,5) bewegt, aber nicht an die Position (4,4), und 'D' ist noch nicht bewegt worden. Ich denke, wir sollten noch eine Runde spielen, um unsere Raster zu synchronisieren. Bitte teile mir mit, wie wir vorgehen sollen, um unsere Ziele zu erreichen.} \\
            }
        }
    }
    & & \\ \\

    \theutterance \stepcounter{utterance}  
    & & & \multicolumn{2}{p{0.3\linewidth}}{
        \cellcolor[rgb]{0.9,0.9,0.9}{
            \makecell[{{p{\linewidth}}}]{
                \texttt{\tiny{[GM$|$GM]}}
                \texttt{SPRICH: Nein, ich habe 'P' an die Position (4,5) bewegt, aber nicht an die Position (4,4), und 'D' ist noch nicht bewegt worden. Ich denke, wir sollten noch eine Runde spielen, um unsere Raster zu synchronisieren. Bitte teile mir mit, wie wir vorgehen sollen, um unsere Ziele zu erreichen.} \\
            }
        }
    }
    & & \\ \\

    \theutterance \stepcounter{utterance}  
    & & \multicolumn{4}{p{0.6\linewidth}}{
        \cellcolor[rgb]{0.9,0.9,0.9}{
            \makecell[{{p{\linewidth}}}]{
                \texttt{\tiny{[GM$\rangle$P2]}}
                \texttt{Deine Nachricht wurde an deinen Mitspieler übermittelt.} \\
\\ 
\\ 
\texttt{Ihr seid gerade bei Runde 28 von maximal 28.} \\
\texttt{Ihr habt gemeinsam 12 von maximal 16 Strafpunkten angesammelt.} \\
\texttt{Dein Mitspieler hat dir diese Nachricht geschickt:} \\
\\ 
\\ 
\texttt{"Nein, ich habe 'P' an die Position (4,5) bewegt, aber nicht an die Position (4,4), und 'D' ist noch nicht bewegt worden. Ich denke, wir sollten noch eine Runde spielen, um unsere Raster zu synchronisieren. Bitte teile mir mit, wie wir vorgehen sollen, um unsere Ziele zu erreichen."} \\
\\ 
\\ 
\texttt{Wie lautet dein nächster Befehl?} \\
            }
        }
    }
    & \\ \\

    \theutterance \stepcounter{utterance}  
    & & & \multicolumn{4}{p{0.6\linewidth}}{
        \cellcolor[rgb]{0.9,0.9,0.9}{
            \makecell[{{p{\linewidth}}}]{
                \texttt{\tiny{[GM$\langle$P2]}}
                \texttt{SPRICH: Es tut mir leid, aber wir haben keine Runden mehr übrig. Ich denke, wir sollten das Spiel beenden und die Ergebnisse überprüfen. Ich werde "fertig?" sagen, um das Spiel zu beenden. Wir können unsere Ergebnisse überprüfen und sehen, wie nah wir unsere Ziele erreicht haben. SPRICH: fertig?} \\
            }
        }
    }
     \\ \\

    \theutterance \stepcounter{utterance}  
    & & & \multicolumn{2}{p{0.3\linewidth}}{
        \cellcolor[rgb]{0.9,0.9,0.9}{
            \makecell[{{p{\linewidth}}}]{
                \texttt{\tiny{[GM$|$GM]}}
                \texttt{SPRICH: Es tut mir leid, aber wir haben keine Runden mehr übrig. Ich denke, wir sollten das Spiel beenden und die Ergebnisse überprüfen. Ich werde "fertig?" sagen, um das Spiel zu beenden. Wir können unsere Ergebnisse überprüfen und sehen, wie nah wir unsere Ziele erreicht haben. SPRICH: fertig?} \\
            }
        }
    }
    & & \\ \\

    \theutterance \stepcounter{utterance}  
    & & & \multicolumn{2}{p{0.3\linewidth}}{
        \cellcolor[rgb]{0.9,0.9,0.9}{
            \makecell[{{p{\linewidth}}}]{
                \texttt{\tiny{[GM$|$GM]}}
                \texttt{Invalid format: head is not empty: 'SPRICH: Es tut mir leid, aber wir haben keine Runden mehr übrig. Ich denke, wir sollten das Spiel beenden und die Ergebnisse überprüfen. Ich werde "fertig?" sagen, um das Spiel zu beenden. Wir können unsere Ergebnisse überprüfen und sehen, wie nah wir unsere Ziele erreicht haben. '} \\
            }
        }
    }
    & & \\ \\

    \theutterance \stepcounter{utterance}  
    & & \multicolumn{4}{p{0.6\linewidth}}{
        \cellcolor[rgb]{0.9,0.9,0.9}{
            \makecell[{{p{\linewidth}}}]{
                \texttt{\tiny{[GM$\rangle$P2]}}
                \texttt{Strafpunkt: Deine Nachricht darf nichts vor dem Befehl enthalten! Stelle sicher, dass deine Antwort ausschließlich einen der beiden Befehle `SPRICH: <NACHRICHT>` und `VERSCHIEBE: <OBJECT>, (<X>, <Y>)` und sonst nichts enthält!} \\
\\ 
\texttt{Ihr habt gemeinsam 13 von maximal 16 Strafpunkten angesammelt.} \\
\texttt{Bitte versuche es noch mal!} \\
            }
        }
    }
    & \\ \\

    \theutterance \stepcounter{utterance}  
    & & & \multicolumn{4}{p{0.6\linewidth}}{
        \cellcolor[rgb]{0.9,0.9,0.9}{
            \makecell[{{p{\linewidth}}}]{
                \texttt{\tiny{[GM$\langle$P2]}}
                \texttt{SPRICH: fertig?} \\
            }
        }
    }
     \\ \\

    \theutterance \stepcounter{utterance}  
    & & & \multicolumn{2}{p{0.3\linewidth}}{
        \cellcolor[rgb]{0.9,0.9,0.9}{
            \makecell[{{p{\linewidth}}}]{
                \texttt{\tiny{[GM$|$GM]}}
                \texttt{SPRICH: fertig?} \\
            }
        }
    }
    & & \\ \\

    \theutterance \stepcounter{utterance}  
    & & & \multicolumn{2}{p{0.3\linewidth}}{
        \cellcolor[rgb]{0.9,0.9,0.9}{
            \makecell[{{p{\linewidth}}}]{
                \texttt{\tiny{[GM$|$GM]}}
                \texttt{Maximum number of rounds reached} \\
            }
        }
    }
    & & \\ \\

    \theutterance \stepcounter{utterance}  
    & & & \multicolumn{2}{p{0.3\linewidth}}{
        \cellcolor[rgb]{0.9,0.9,0.9}{
            \makecell[{{p{\linewidth}}}]{
                \texttt{\tiny{[GM$|$GM]}}
                \texttt{* success: True} \\
\texttt{* lose: False} \\
\texttt{* aborted: False} \\
\texttt{{-}{-}{-}{-}{-}{-}{-}} \\
\texttt{* Shifts: 5.00} \\
\texttt{* Max Shifts: 12.00} \\
\texttt{* Min Shifts: 6.00} \\
\texttt{* End Distance Sum: 11.70} \\
\texttt{* Init Distance Sum: 18.94} \\
\texttt{* Expected Distance Sum: 29.33} \\
\texttt{* Penalties: 13.00} \\
\texttt{* Max Penalties: 16.00} \\
\texttt{* Rounds: 28.00} \\
\texttt{* Max Rounds: 28.00} \\
\texttt{* Object Count: 7.00} \\
            }
        }
    }
    & & \\ \\

    \theutterance \stepcounter{utterance}  
    & & & \multicolumn{2}{p{0.3\linewidth}}{
        \cellcolor[rgb]{0.9,0.9,0.9}{
            \makecell[{{p{\linewidth}}}]{
                \texttt{\tiny{[GM$|$GM]}}
                \texttt{Initial Grids:} \\
\texttt{```} \\
\texttt{Player 1:     Player 2:} \\
\texttt{ 1234567       1234567} \\
\texttt{╔═╤═╤═╤═╗     ╔═╤═╤═╤═╗} \\
\texttt{║◌│◌│◌│A║ 1   ║◌│◌│◌│◌║ 1} \\
\texttt{╟─┴─┼─┤◌║ 2   ╟─┴─┼─┤◌║ 2} \\
\texttt{║◌◌M│◌└─╢ 3   ║◌M◌│◌└─╢ 3} \\
\texttt{╟───┘◌S◌║ 4   ╟───┘P◌◌║ 4} \\
\texttt{║◌◌◌◌◌┌─╢ 5   ║O◌◌◌◌┌─╢ 5} \\
\texttt{║◌◌◌PT│O║ 6   ║◌D◌◌A│S║ 6} \\
\texttt{║◌D◌◌◌│◌║ 7   ║◌◌◌◌T│◌║ 7} \\
\texttt{╚═════╧═╝     ╚═════╧═╝} \\
\texttt{```} \\
\\ 
\texttt{Final Grids:} \\
\texttt{```} \\
\texttt{Player 1:     Player 2:} \\
\texttt{ 1234567       1234567} \\
\texttt{╔═╤═╤═╤═╗     ╔═╤═╤═╤═╗} \\
\texttt{║◌│◌│◌│◌║ 1   ║◌│◌│◌│◌║ 1} \\
\texttt{╟─┴─┼─┤◌║ 2   ╟─┴─┼─┤◌║ 2} \\
\texttt{║◌◌M│◌└─╢ 3   ║◌M◌│◌└─╢ 3} \\
\texttt{╟───┘O◌◌║ 4   ╟───┘P◌◌║ 4} \\
\texttt{║◌A◌PD┌─╢ 5   ║OA◌◌◌┌─╢ 5} \\
\texttt{║◌◌◌◌◌│S║ 6   ║◌D◌◌◌│S║ 6} \\
\texttt{║◌◌◌◌◌│T║ 7   ║◌◌◌◌T│◌║ 7} \\
\texttt{╚═════╧═╝     ╚═════╧═╝} \\
\texttt{```} \\
            }
        }
    }
    & & \\ \\

\end{supertabular}
}

\end{document}
