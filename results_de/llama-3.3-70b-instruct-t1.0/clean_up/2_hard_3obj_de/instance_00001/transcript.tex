
\documentclass{article}
\usepackage{colortbl}
\usepackage{makecell}
\usepackage{multirow}
\usepackage{supertabular}

\begin{document}

\newcounter{utterance}

\centering \large Interaction Transcript for game `clean\_up', experiment `2\_hard\_3obj\_de', episode 1 with llama{-}3.3{-}70b{-}instruct{-}t1.0.
\vspace{24pt}

{ \footnotesize  \setcounter{utterance}{1}
\setlength{\tabcolsep}{0pt}
\begin{supertabular}{c@{$\;$}|p{.15\linewidth}@{}p{.15\linewidth}p{.15\linewidth}p{.15\linewidth}p{.15\linewidth}p{.15\linewidth}}
    \# & $\;$A & \multicolumn{4}{c}{Game Master} & $\;\:$B\\
    \hline

    \theutterance \stepcounter{utterance}  
    & & \multicolumn{4}{p{0.6\linewidth}}{
        \cellcolor[rgb]{0.9,0.9,0.9}{
            \makecell[{{p{\linewidth}}}]{
                \texttt{\tiny{[P1$\langle$GM]}}
                \texttt{Ich bin der Spielleiter, und du spielst ein kooperatives Spiel mit dem folgenden Raster als Spielbrett:} \\
\\ 
\texttt{```} \\
\texttt{ 1234567} \\
\texttt{╔═══╤═╤═╗} \\
\texttt{║◌◌◌│◌│◌║ 1} \\
\texttt{╟─┬─┼─┴─╢ 2} \\
\texttt{║◌│◌│C◌◌║ 3} \\
\texttt{║◌├─┤◌┌─╢ 4} \\
\texttt{╟─┘◌│◌│L║ 5} \\
\texttt{║◌◌◌└┬┴─╢ 6} \\
\texttt{║P◌◌◌│◌◌║ 7} \\
\texttt{╚════╧══╝} \\
\\ 
\texttt{```} \\
\\ 
\texttt{* Am oberen Rand sind die x{-}Koordinaten angegeben, die nach rechts hin zunehmen, und am rechten Rand die y{-}Koordinaten, die nach unten zunehmen.} \\
\texttt{* Die folgenden Objekte sind zufällig auf deinem Raster platziert: 'C', 'L', 'P'.} \\
\\ 
\\ 
\texttt{Dein Mitspieler sieht eine Variation des Spielbretts, auf dem dieselben Objekte an anderen zufälligen Positionen platziert sind. Du kannst das Spielbrett des anderen Spielers nicht sehen, und er kann deins nicht sehen.} \\
\\ 
\\ 
\texttt{**Ziel des Spiels:**} \\
\\ 
\texttt{Beide Spieler müssen die Objekte auf ihrem jeweiligen Spielbrett so bewegen, dass identische Objekte am Ende an denselben Koordinaten liegen. Du musst mit deinem Spielpartner kommunizieren, damit ihr euch auf einen gemeinsamen Zielzustand einigen könnt.} \\
\\ 
\\ 
\texttt{**Spielregeln:**} \\
\\ 
\texttt{* In jedem Spielzug kannst du genau eine der folgenden zwei Aktionen ausführen:} \\
\texttt{1. `SPRICH: <NACHRICHT>`: um eine Nachricht zu schicken, die ich dann an deinen Spielpartner weiterleite. Als Nachricht gilt alles bis zum nächsten Zeilenumbruch.} \\
\texttt{2. `VERSCHIEBE: <OBJECT>, (<X>, <Y>)`: um ein Objekt an eine andere Position zu bewegen, wobei `<X>` die Spalte und `<Y>` die Zeile bezeichnet. Ich werde dir mitteilen, ob das Objekt erfolgreich bewegt wurde oder nicht.} \\
\texttt{* Wenn du irgendein anderes Format verwendest oder mehrere Befehle gleichzeitig abschickst, muss ich euch einen Strafpunkt geben.} \\
\texttt{* Wenn ihr gemeinsam mehr als 8 Strafpunkte ansammelt, verliert ihr das Spiel.} \\
\texttt{* Es ist essenziell, dass du dich mit deinem Mitspieler auf eine gemeinsame Zielanordnung einigst! Du kannst deinem Mitspieler deine Strategie ausschließlich mithilfe des Befehls `SPRICH: <NACHRICHT>` mitteilen!} \\
\\ 
\\ 
\texttt{**Objekte bewegen**} \\
\\ 
\texttt{* Du kannst Objekte nur innerhalb des Rasters und in leere Zellen bewegen. Leere Zellen erkennst am Zeichen '◌'.} \\
\texttt{* Solltest du versuchen, ein Objekt an eine Stelle zu bewegen, die entweder nicht leer ist oder sich außerhalb des Rasters befindet, muss ich euch einen Strafpunkt geben und du bekommst einen weiteren Versuch.} \\
\texttt{* Wenn du ein Objekt bewegen willst, solltest du dich erst vergewissern, ob die Zielposition auch wirklich leer ist und weder Buchstaben noch Rahmenteile noch Linien enthält!} \\
\\ 
\\ 
\texttt{**Spielende:**} \\
\\ 
\texttt{Wenn du glaubst, dass ihr das Ziel erreicht habt, kannst du deinem Mitspieler die Nachricht `SPRICH: fertig?` schicken. Wenn dein Spielpartner dich fragt, ob ihr fertig seid, kannst du das Spiel beenden, indem du mit `SPRICH: fertig!` antwortest.} \\
\\ 
\texttt{Ihr gewinnt gemeinsam, wenn ihr das Spiel innerhalb von \#max\_rounds Runden beendet. Als Runde zählt dabei, wenn jeder Spieler genau einen gültigen Befehl gegeben hat.} \\
\\ 
\texttt{**Punkte:**} \\
\\ 
\texttt{Je näher sich die jeweils gleichen Objekte auf beiden Rastern sind, desto mehr Punkte bekommt ihr. Für Strafpunkte gibt es Punktabzug. Könnt ihr den Rekord brechen?} \\
\\ 
\texttt{Schicke eine Nachricht an deinen Mitspieler um das Spiel zu starten!} \\
            }
        }
    }
    & \\ \\

    \theutterance \stepcounter{utterance}  
    & \multicolumn{4}{p{0.6\linewidth}}{
        \cellcolor[rgb]{0.9,0.9,0.9}{
            \makecell[{{p{\linewidth}}}]{
                \texttt{\tiny{[P1$\rangle$GM]}}
                \texttt{SPRICH: Hallo, lass uns das Spiel beginnen! Wir sollten uns auf eine Strategie einigen, um die Objekte 'C', 'L' und 'P' auf beiden Rastern an denselben Koordinaten zu platzieren. Ich schlage vor, dass wir uns auf eine feste Position für jedes Objekt einigen, um die Kommunikation zu erleichtern. Wie wäre es, wenn wir 'C' auf (4,4), 'L' auf (2,2) und 'P' auf (6,6) platzieren? Bitte antworte mit deiner Meinung oder einem alternativen Vorschlag.} \\
            }
        }
    }
    & & \\ \\

    \theutterance \stepcounter{utterance}  
    & & & \multicolumn{2}{p{0.3\linewidth}}{
        \cellcolor[rgb]{0.9,0.9,0.9}{
            \makecell[{{p{\linewidth}}}]{
                \texttt{\tiny{[GM$|$GM]}}
                \texttt{SPRICH: Hallo, lass uns das Spiel beginnen! Wir sollten uns auf eine Strategie einigen, um die Objekte 'C', 'L' und 'P' auf beiden Rastern an denselben Koordinaten zu platzieren. Ich schlage vor, dass wir uns auf eine feste Position für jedes Objekt einigen, um die Kommunikation zu erleichtern. Wie wäre es, wenn wir 'C' auf (4,4), 'L' auf (2,2) und 'P' auf (6,6) platzieren? Bitte antworte mit deiner Meinung oder einem alternativen Vorschlag.} \\
            }
        }
    }
    & & \\ \\

    \theutterance \stepcounter{utterance}  
    & & \multicolumn{4}{p{0.6\linewidth}}{
        \cellcolor[rgb]{0.9,0.9,0.9}{
            \makecell[{{p{\linewidth}}}]{
                \texttt{\tiny{[GM$\rangle$P2]}}
                \texttt{Ich bin der Spielleiter, und du spielst ein kooperatives Spiel mit dem folgenden Raster als Spielbrett:} \\
\\ 
\texttt{```} \\
\texttt{ 1234567} \\
\texttt{╔═══╤═╤═╗} \\
\texttt{║◌◌◌│◌│◌║ 1} \\
\texttt{╟─┬─┼─┴─╢ 2} \\
\texttt{║◌│L│◌P◌║ 3} \\
\texttt{║◌├─┤◌┌─╢ 4} \\
\texttt{╟─┘◌│◌│◌║ 5} \\
\texttt{║◌◌◌└┬┴─╢ 6} \\
\texttt{║◌C◌◌│◌◌║ 7} \\
\texttt{╚════╧══╝} \\
\\ 
\texttt{```} \\
\\ 
\texttt{* Am oberen Rand sind die x{-}Koordinaten angegeben, die nach rechts hin zunehmen, und am rechten Rand die y{-}Koordinaten, die nach unten zunehmen.} \\
\texttt{* Die folgenden Objekte sind zufällig auf deinem Raster platziert: 'C', 'L', 'P'.} \\
\\ 
\\ 
\texttt{Dein Mitspieler sieht eine Variation des Spielbretts, auf dem dieselben Objekte an anderen zufälligen Positionen platziert sind. Du kannst das Spielbrett des anderen Spielers nicht sehen, und er kann deins nicht sehen.} \\
\\ 
\\ 
\texttt{**Ziel des Spiels:**} \\
\\ 
\texttt{Beide Spieler müssen die Objekte auf ihrem jeweiligen Spielbrett so bewegen, dass identische Objekte am Ende an denselben Koordinaten liegen. Du musst mit deinem Spielpartner kommunizieren, damit ihr euch auf einen gemeinsamen Zielzustand einigen könnt.} \\
\\ 
\\ 
\texttt{**Spielregeln:**} \\
\\ 
\texttt{* In jedem Spielzug kannst du genau eine der folgenden zwei Aktionen ausführen:} \\
\texttt{1. `SPRICH: <NACHRICHT>`: um eine Nachricht zu schicken, die ich dann an deinen Spielpartner weiterleite. Als Nachricht gilt alles bis zum nächsten Zeilenumbruch.} \\
\texttt{2. `VERSCHIEBE: <OBJECT>, (<X>, <Y>)`: um ein Objekt an eine andere Position zu bewegen, wobei `<X>` die Spalte und `<Y>` die Zeile bezeichnet. Ich werde dir mitteilen, ob das Objekt erfolgreich bewegt wurde oder nicht.} \\
\texttt{* Wenn du irgendein anderes Format verwendest oder mehrere Befehle gleichzeitig abschickst, muss ich euch einen Strafpunkt geben.} \\
\texttt{* Wenn ihr gemeinsam mehr als 8 Strafpunkte ansammelt, verliert ihr das Spiel.} \\
\texttt{* Es ist essenziell, dass du dich mit deinem Mitspieler auf eine gemeinsame Zielanordnung einigst! Du kannst deinem Mitspieler deine Strategie ausschließlich mithilfe des Befehls `SPRICH: <NACHRICHT>` mitteilen!} \\
\\ 
\\ 
\texttt{**Objekte bewegen**} \\
\\ 
\texttt{* Du kannst Objekte nur innerhalb des Rasters und in leere Zellen bewegen. Leere Zellen erkennst am Zeichen '◌'.} \\
\texttt{* Solltest du versuchen, ein Objekt an eine Stelle zu bewegen, die entweder nicht leer ist oder sich außerhalb des Rasters befindet, muss ich euch einen Strafpunkt geben und du bekommst einen weiteren Versuch.} \\
\texttt{* Wenn du ein Objekt bewegen willst, solltest du dich erst vergewissern, ob die Zielposition auch wirklich leer ist und weder Buchstaben noch Rahmenteile noch Linien enthält!} \\
\\ 
\\ 
\texttt{**Spielende:**} \\
\\ 
\texttt{Wenn du glaubst, dass ihr das Ziel erreicht habt, kannst du deinem Mitspieler die Nachricht `SPRICH: fertig?` schicken. Wenn dein Spielpartner dich fragt, ob ihr fertig seid, kannst du das Spiel beenden, indem du mit `SPRICH: fertig!` antwortest.} \\
\\ 
\texttt{Ihr gewinnt gemeinsam, wenn ihr das Spiel innerhalb von \#max\_rounds Runden beendet. Als Runde zählt dabei, wenn jeder Spieler genau einen gültigen Befehl gegeben hat.} \\
\\ 
\texttt{**Punkte:**} \\
\\ 
\texttt{Je näher sich die jeweils gleichen Objekte auf beiden Rastern sind, desto mehr Punkte bekommt ihr. Für Strafpunkte gibt es Punktabzug. Könnt ihr den Rekord brechen?} \\
\\ 
\texttt{Dein Mitspieler hat das Spiel mit dieser Nachricht begonnen:} \\
\\ 
\texttt{"Hallo, lass uns das Spiel beginnen! Wir sollten uns auf eine Strategie einigen, um die Objekte 'C', 'L' und 'P' auf beiden Rastern an denselben Koordinaten zu platzieren. Ich schlage vor, dass wir uns auf eine feste Position für jedes Objekt einigen, um die Kommunikation zu erleichtern. Wie wäre es, wenn wir 'C' auf (4,4), 'L' auf (2,2) und 'P' auf (6,6) platzieren? Bitte antworte mit deiner Meinung oder einem alternativen Vorschlag."} \\
\\ 
\texttt{Wie lautet dein erster Befehl?} \\
            }
        }
    }
    & \\ \\

    \theutterance \stepcounter{utterance}  
    & & & \multicolumn{4}{p{0.6\linewidth}}{
        \cellcolor[rgb]{0.9,0.9,0.9}{
            \makecell[{{p{\linewidth}}}]{
                \texttt{\tiny{[GM$\langle$P2]}}
                \texttt{SPRICH: Hallo! Dein Vorschlag klingt gut, aber ich denke, wir sollten die Positionen basierend auf unseren aktuellen Spielbrettern auswählen. Ich könnte dir meine aktuelle Anordnung beschreiben, und dann könnten wir gemeinsam entscheiden, wo wir die Objekte platzieren wollen. Mein 'C' ist ziemlich weit unten, 'L' ist irgendwo in der Mitte links, und 'P' ist oben rechts. Was meinst du? Soll ich meine genauen Koordinaten teilen?} \\
            }
        }
    }
     \\ \\

    \theutterance \stepcounter{utterance}  
    & & & \multicolumn{2}{p{0.3\linewidth}}{
        \cellcolor[rgb]{0.9,0.9,0.9}{
            \makecell[{{p{\linewidth}}}]{
                \texttt{\tiny{[GM$|$GM]}}
                \texttt{SPRICH: Hallo! Dein Vorschlag klingt gut, aber ich denke, wir sollten die Positionen basierend auf unseren aktuellen Spielbrettern auswählen. Ich könnte dir meine aktuelle Anordnung beschreiben, und dann könnten wir gemeinsam entscheiden, wo wir die Objekte platzieren wollen. Mein 'C' ist ziemlich weit unten, 'L' ist irgendwo in der Mitte links, und 'P' ist oben rechts. Was meinst du? Soll ich meine genauen Koordinaten teilen?} \\
            }
        }
    }
    & & \\ \\

    \theutterance \stepcounter{utterance}  
    & & \multicolumn{4}{p{0.6\linewidth}}{
        \cellcolor[rgb]{0.9,0.9,0.9}{
            \makecell[{{p{\linewidth}}}]{
                \texttt{\tiny{[P1$\langle$GM]}}
                \texttt{Deine Nachricht wurde an deinen Mitspieler übermittelt.} \\
\\ 
\\ 
\texttt{Ihr seid gerade bei Runde 1 von maximal 12.} \\
\texttt{Ihr habt gemeinsam 0 von maximal 8 Strafpunkten angesammelt.} \\
\texttt{Dein Mitspieler hat dir diese Nachricht geschickt:} \\
\\ 
\\ 
\texttt{"Hallo! Dein Vorschlag klingt gut, aber ich denke, wir sollten die Positionen basierend auf unseren aktuellen Spielbrettern auswählen. Ich könnte dir meine aktuelle Anordnung beschreiben, und dann könnten wir gemeinsam entscheiden, wo wir die Objekte platzieren wollen. Mein 'C' ist ziemlich weit unten, 'L' ist irgendwo in der Mitte links, und 'P' ist oben rechts. Was meinst du? Soll ich meine genauen Koordinaten teilen?"} \\
\\ 
\\ 
\texttt{Wie lautet dein nächster Befehl?} \\
            }
        }
    }
    & \\ \\

    \theutterance \stepcounter{utterance}  
    & \multicolumn{4}{p{0.6\linewidth}}{
        \cellcolor[rgb]{0.9,0.9,0.9}{
            \makecell[{{p{\linewidth}}}]{
                \texttt{\tiny{[P1$\rangle$GM]}}
                \texttt{SPRICH: Hallo zurück! Ich denke, es ist eine gute Idee, unsere aktuellen Anordnungen zu berücksichtigen. Bitte teile mir deine genauen Koordinaten mit, und ich werde meine ebenfalls mitteilen. Dann können wir gemeinsam entscheiden, wo wir die Objekte platzieren wollen. Ich bin bereit, meine Strategie anzupassen, um sicherzustellen, dass wir ein gemeinsames Ziel erreichen. Bitte teile mir deine Koordinaten mit, und wir können weitermachen!} \\
            }
        }
    }
    & & \\ \\

    \theutterance \stepcounter{utterance}  
    & & & \multicolumn{2}{p{0.3\linewidth}}{
        \cellcolor[rgb]{0.9,0.9,0.9}{
            \makecell[{{p{\linewidth}}}]{
                \texttt{\tiny{[GM$|$GM]}}
                \texttt{SPRICH: Hallo zurück! Ich denke, es ist eine gute Idee, unsere aktuellen Anordnungen zu berücksichtigen. Bitte teile mir deine genauen Koordinaten mit, und ich werde meine ebenfalls mitteilen. Dann können wir gemeinsam entscheiden, wo wir die Objekte platzieren wollen. Ich bin bereit, meine Strategie anzupassen, um sicherzustellen, dass wir ein gemeinsames Ziel erreichen. Bitte teile mir deine Koordinaten mit, und wir können weitermachen!} \\
            }
        }
    }
    & & \\ \\

    \theutterance \stepcounter{utterance}  
    & & \multicolumn{4}{p{0.6\linewidth}}{
        \cellcolor[rgb]{0.9,0.9,0.9}{
            \makecell[{{p{\linewidth}}}]{
                \texttt{\tiny{[GM$\rangle$P2]}}
                \texttt{Deine Nachricht wurde an deinen Mitspieler übermittelt.} \\
\\ 
\\ 
\texttt{Ihr seid gerade bei Runde 2 von maximal 12.} \\
\texttt{Ihr habt gemeinsam 0 von maximal 8 Strafpunkten angesammelt.} \\
\texttt{Dein Mitspieler hat dir diese Nachricht geschickt:} \\
\\ 
\\ 
\texttt{"Hallo zurück! Ich denke, es ist eine gute Idee, unsere aktuellen Anordnungen zu berücksichtigen. Bitte teile mir deine genauen Koordinaten mit, und ich werde meine ebenfalls mitteilen. Dann können wir gemeinsam entscheiden, wo wir die Objekte platzieren wollen. Ich bin bereit, meine Strategie anzupassen, um sicherzustellen, dass wir ein gemeinsames Ziel erreichen. Bitte teile mir deine Koordinaten mit, und wir können weitermachen!"} \\
\\ 
\\ 
\texttt{Wie lautet dein nächster Befehl?} \\
            }
        }
    }
    & \\ \\

    \theutterance \stepcounter{utterance}  
    & & & \multicolumn{4}{p{0.6\linewidth}}{
        \cellcolor[rgb]{0.9,0.9,0.9}{
            \makecell[{{p{\linewidth}}}]{
                \texttt{\tiny{[GM$\langle$P2]}}
                \texttt{SPRICH: Mein 'C' ist bei (4,7), 'L' bei (2,3) und 'P' bei (5,3). Ich denke, wir sollten versuchen, sie an Positionen zu platzieren, die für beide von uns leicht erreichbar sind. Wie sind deine Koordinaten? Gibt es eine Position, die du bevorzugst oder die du als besonders einfach empfindest, Objekte dorthin zu bewegen?} \\
            }
        }
    }
     \\ \\

    \theutterance \stepcounter{utterance}  
    & & & \multicolumn{2}{p{0.3\linewidth}}{
        \cellcolor[rgb]{0.9,0.9,0.9}{
            \makecell[{{p{\linewidth}}}]{
                \texttt{\tiny{[GM$|$GM]}}
                \texttt{SPRICH: Mein 'C' ist bei (4,7), 'L' bei (2,3) und 'P' bei (5,3). Ich denke, wir sollten versuchen, sie an Positionen zu platzieren, die für beide von uns leicht erreichbar sind. Wie sind deine Koordinaten? Gibt es eine Position, die du bevorzugst oder die du als besonders einfach empfindest, Objekte dorthin zu bewegen?} \\
            }
        }
    }
    & & \\ \\

    \theutterance \stepcounter{utterance}  
    & & \multicolumn{4}{p{0.6\linewidth}}{
        \cellcolor[rgb]{0.9,0.9,0.9}{
            \makecell[{{p{\linewidth}}}]{
                \texttt{\tiny{[P1$\langle$GM]}}
                \texttt{Deine Nachricht wurde an deinen Mitspieler übermittelt.} \\
\\ 
\\ 
\texttt{Ihr seid gerade bei Runde 2 von maximal 12.} \\
\texttt{Ihr habt gemeinsam 0 von maximal 8 Strafpunkten angesammelt.} \\
\texttt{Dein Mitspieler hat dir diese Nachricht geschickt:} \\
\\ 
\\ 
\texttt{"Mein 'C' ist bei (4,7), 'L' bei (2,3) und 'P' bei (5,3). Ich denke, wir sollten versuchen, sie an Positionen zu platzieren, die für beide von uns leicht erreichbar sind. Wie sind deine Koordinaten? Gibt es eine Position, die du bevorzugst oder die du als besonders einfach empfindest, Objekte dorthin zu bewegen?"} \\
\\ 
\\ 
\texttt{Wie lautet dein nächster Befehl?} \\
            }
        }
    }
    & \\ \\

    \theutterance \stepcounter{utterance}  
    & \multicolumn{4}{p{0.6\linewidth}}{
        \cellcolor[rgb]{0.9,0.9,0.9}{
            \makecell[{{p{\linewidth}}}]{
                \texttt{\tiny{[P1$\rangle$GM]}}
                \texttt{SPRICH: Danke für die Koordinaten! Mein 'C' ist bei (3,3), 'L' bei (5,5) und 'P' bei (1,7). Basierend auf unseren Anordnungen denke ich, dass die Position (3,5) für 'C', (5,3) für 'L' und (1,3) für 'P' gut geeignet wären. Diese Positionen scheinen für beide von uns relativ leicht erreichbar zu sein. Was meinst du? Sollten wir uns auf diese Zielpositionen einigen und versuchen, unsere Objekte dorthin zu bewegen?} \\
            }
        }
    }
    & & \\ \\

    \theutterance \stepcounter{utterance}  
    & & & \multicolumn{2}{p{0.3\linewidth}}{
        \cellcolor[rgb]{0.9,0.9,0.9}{
            \makecell[{{p{\linewidth}}}]{
                \texttt{\tiny{[GM$|$GM]}}
                \texttt{SPRICH: Danke für die Koordinaten! Mein 'C' ist bei (3,3), 'L' bei (5,5) und 'P' bei (1,7). Basierend auf unseren Anordnungen denke ich, dass die Position (3,5) für 'C', (5,3) für 'L' und (1,3) für 'P' gut geeignet wären. Diese Positionen scheinen für beide von uns relativ leicht erreichbar zu sein. Was meinst du? Sollten wir uns auf diese Zielpositionen einigen und versuchen, unsere Objekte dorthin zu bewegen?} \\
            }
        }
    }
    & & \\ \\

    \theutterance \stepcounter{utterance}  
    & & \multicolumn{4}{p{0.6\linewidth}}{
        \cellcolor[rgb]{0.9,0.9,0.9}{
            \makecell[{{p{\linewidth}}}]{
                \texttt{\tiny{[GM$\rangle$P2]}}
                \texttt{Deine Nachricht wurde an deinen Mitspieler übermittelt.} \\
\\ 
\\ 
\texttt{Ihr seid gerade bei Runde 3 von maximal 12.} \\
\texttt{Ihr habt gemeinsam 0 von maximal 8 Strafpunkten angesammelt.} \\
\texttt{Dein Mitspieler hat dir diese Nachricht geschickt:} \\
\\ 
\\ 
\texttt{"Danke für die Koordinaten! Mein 'C' ist bei (3,3), 'L' bei (5,5) und 'P' bei (1,7). Basierend auf unseren Anordnungen denke ich, dass die Position (3,5) für 'C', (5,3) für 'L' und (1,3) für 'P' gut geeignet wären. Diese Positionen scheinen für beide von uns relativ leicht erreichbar zu sein. Was meinst du? Sollten wir uns auf diese Zielpositionen einigen und versuchen, unsere Objekte dorthin zu bewegen?"} \\
\\ 
\\ 
\texttt{Wie lautet dein nächster Befehl?} \\
            }
        }
    }
    & \\ \\

    \theutterance \stepcounter{utterance}  
    & & & \multicolumn{4}{p{0.6\linewidth}}{
        \cellcolor[rgb]{0.9,0.9,0.9}{
            \makecell[{{p{\linewidth}}}]{
                \texttt{\tiny{[GM$\langle$P2]}}
                \texttt{SPRICH: Dein Vorschlag klingt vernünftig, aber ich denke, wir sollten unsere Zielpositionen basierend auf den spezifischen Herausforderungen unseres Spielbretts auswählen. Meines 'C' ist bei (4,7), also ziemlich weit unten, und 'L' ist bei (2,3), was ziemlich gut zu deinem 'L' bei (5,5) passt, wenn wir es nach (5,3) bewegen. ‘P’ bei (5,3) bei mir und (1,7) bei dir lässt darauf schließen, dass wir versuchen sollten, es in eine Position zu bringen, die für beide von uns zugänglich ist. Was halten wir von (4,5) für 'C', (5,3) für 'L' und (3,5) für 'P'? Wir sollten auch überlegen, wie wir die Objekte am effizientesten dorthin bewegen können, da wir nur wenige Runden haben. Sollen wir unsere Bewegungen koordinieren, um sicherzustellen, dass wir unsere Ziele erreichen, ohne uns in die Quere zu kommen?} \\
            }
        }
    }
     \\ \\

    \theutterance \stepcounter{utterance}  
    & & & \multicolumn{2}{p{0.3\linewidth}}{
        \cellcolor[rgb]{0.9,0.9,0.9}{
            \makecell[{{p{\linewidth}}}]{
                \texttt{\tiny{[GM$|$GM]}}
                \texttt{SPRICH: Dein Vorschlag klingt vernünftig, aber ich denke, wir sollten unsere Zielpositionen basierend auf den spezifischen Herausforderungen unseres Spielbretts auswählen. Meines 'C' ist bei (4,7), also ziemlich weit unten, und 'L' ist bei (2,3), was ziemlich gut zu deinem 'L' bei (5,5) passt, wenn wir es nach (5,3) bewegen. ‘P’ bei (5,3) bei mir und (1,7) bei dir lässt darauf schließen, dass wir versuchen sollten, es in eine Position zu bringen, die für beide von uns zugänglich ist. Was halten wir von (4,5) für 'C', (5,3) für 'L' und (3,5) für 'P'? Wir sollten auch überlegen, wie wir die Objekte am effizientesten dorthin bewegen können, da wir nur wenige Runden haben. Sollen wir unsere Bewegungen koordinieren, um sicherzustellen, dass wir unsere Ziele erreichen, ohne uns in die Quere zu kommen?} \\
            }
        }
    }
    & & \\ \\

    \theutterance \stepcounter{utterance}  
    & & \multicolumn{4}{p{0.6\linewidth}}{
        \cellcolor[rgb]{0.9,0.9,0.9}{
            \makecell[{{p{\linewidth}}}]{
                \texttt{\tiny{[P1$\langle$GM]}}
                \texttt{Deine Nachricht wurde an deinen Mitspieler übermittelt.} \\
\\ 
\\ 
\texttt{Ihr seid gerade bei Runde 3 von maximal 12.} \\
\texttt{Ihr habt gemeinsam 0 von maximal 8 Strafpunkten angesammelt.} \\
\texttt{Dein Mitspieler hat dir diese Nachricht geschickt:} \\
\\ 
\\ 
\texttt{"Dein Vorschlag klingt vernünftig, aber ich denke, wir sollten unsere Zielpositionen basierend auf den spezifischen Herausforderungen unseres Spielbretts auswählen. Meines 'C' ist bei (4,7), also ziemlich weit unten, und 'L' ist bei (2,3), was ziemlich gut zu deinem 'L' bei (5,5) passt, wenn wir es nach (5,3) bewegen. ‘P’ bei (5,3) bei mir und (1,7) bei dir lässt darauf schließen, dass wir versuchen sollten, es in eine Position zu bringen, die für beide von uns zugänglich ist. Was halten wir von (4,5) für 'C', (5,3) für 'L' und (3,5) für 'P'? Wir sollten auch überlegen, wie wir die Objekte am effizientesten dorthin bewegen können, da wir nur wenige Runden haben. Sollen wir unsere Bewegungen koordinieren, um sicherzustellen, dass wir unsere Ziele erreichen, ohne uns in die Quere zu kommen?"} \\
\\ 
\\ 
\texttt{Wie lautet dein nächster Befehl?} \\
            }
        }
    }
    & \\ \\

    \theutterance \stepcounter{utterance}  
    & \multicolumn{4}{p{0.6\linewidth}}{
        \cellcolor[rgb]{0.9,0.9,0.9}{
            \makecell[{{p{\linewidth}}}]{
                \texttt{\tiny{[P1$\rangle$GM]}}
                \texttt{SPRICH: Ich denke, deine Vorschläge für die Zielpositionen sind sehr vernünftig. (4,5) für 'C', (5,3) für 'L' und (3,5) für 'P' scheinen gute Kompromisse zu sein, die unsere aktuellen Anordnungen berücksichtigen. Um unsere Bewegungen zu koordinieren, schlage ich vor, dass wir uns auf eine Reihenfolge einigen, in der wir die Objekte bewegen. Wie wäre es, wenn wir zuerst 'L' bewegen, dann 'P' und schließlich 'C'? Auf diese Weise können wir sicherstellen, dass wir nicht versehentlich ein Objekt blockieren, das der andere Spieler bewegen muss. Außerdem könnten wir uns abstimmen, wie viele Schritte wir in jeder Runde unternehmen, um unsere Fortschritte zu überwachen. Was meinst du? Sollten wir mit 'L' beginnen und dann Schritt für Schritt vorgehen?} \\
            }
        }
    }
    & & \\ \\

    \theutterance \stepcounter{utterance}  
    & & & \multicolumn{2}{p{0.3\linewidth}}{
        \cellcolor[rgb]{0.9,0.9,0.9}{
            \makecell[{{p{\linewidth}}}]{
                \texttt{\tiny{[GM$|$GM]}}
                \texttt{SPRICH: Ich denke, deine Vorschläge für die Zielpositionen sind sehr vernünftig. (4,5) für 'C', (5,3) für 'L' und (3,5) für 'P' scheinen gute Kompromisse zu sein, die unsere aktuellen Anordnungen berücksichtigen. Um unsere Bewegungen zu koordinieren, schlage ich vor, dass wir uns auf eine Reihenfolge einigen, in der wir die Objekte bewegen. Wie wäre es, wenn wir zuerst 'L' bewegen, dann 'P' und schließlich 'C'? Auf diese Weise können wir sicherstellen, dass wir nicht versehentlich ein Objekt blockieren, das der andere Spieler bewegen muss. Außerdem könnten wir uns abstimmen, wie viele Schritte wir in jeder Runde unternehmen, um unsere Fortschritte zu überwachen. Was meinst du? Sollten wir mit 'L' beginnen und dann Schritt für Schritt vorgehen?} \\
            }
        }
    }
    & & \\ \\

    \theutterance \stepcounter{utterance}  
    & & \multicolumn{4}{p{0.6\linewidth}}{
        \cellcolor[rgb]{0.9,0.9,0.9}{
            \makecell[{{p{\linewidth}}}]{
                \texttt{\tiny{[GM$\rangle$P2]}}
                \texttt{Deine Nachricht wurde an deinen Mitspieler übermittelt.} \\
\\ 
\\ 
\texttt{Ihr seid gerade bei Runde 4 von maximal 12.} \\
\texttt{Ihr habt gemeinsam 0 von maximal 8 Strafpunkten angesammelt.} \\
\texttt{Dein Mitspieler hat dir diese Nachricht geschickt:} \\
\\ 
\\ 
\texttt{"Ich denke, deine Vorschläge für die Zielpositionen sind sehr vernünftig. (4,5) für 'C', (5,3) für 'L' und (3,5) für 'P' scheinen gute Kompromisse zu sein, die unsere aktuellen Anordnungen berücksichtigen. Um unsere Bewegungen zu koordinieren, schlage ich vor, dass wir uns auf eine Reihenfolge einigen, in der wir die Objekte bewegen. Wie wäre es, wenn wir zuerst 'L' bewegen, dann 'P' und schließlich 'C'? Auf diese Weise können wir sicherstellen, dass wir nicht versehentlich ein Objekt blockieren, das der andere Spieler bewegen muss. Außerdem könnten wir uns abstimmen, wie viele Schritte wir in jeder Runde unternehmen, um unsere Fortschritte zu überwachen. Was meinst du? Sollten wir mit 'L' beginnen und dann Schritt für Schritt vorgehen?"} \\
\\ 
\\ 
\texttt{Wie lautet dein nächster Befehl?} \\
            }
        }
    }
    & \\ \\

    \theutterance \stepcounter{utterance}  
    & & & \multicolumn{4}{p{0.6\linewidth}}{
        \cellcolor[rgb]{0.9,0.9,0.9}{
            \makecell[{{p{\linewidth}}}]{
                \texttt{\tiny{[GM$\langle$P2]}}
                \texttt{SPRICH: Ich stimme deinem Vorschlag zu, unsere Bewegungen in der Reihenfolge 'L', 'P' und dann 'C' durchzuführen. Das sieht nach einem klaren und pragmatischen Ansatz aus, um Konflikte zu vermeiden. Bevor wir beginnen, denke ich, es wäre hilfreich, wenn wir beide unsere aktuellen Bretter überprüfen, um sicherzustellen, dass der Weg für die erste Bewegung von 'L' frei ist. Ich werde 'L' von (2,3) nach (5,3) bewegen. Bitte überprüfe, ob dies auf deinem Brett problemlos möglich ist und gib mir ein Update, bevor wir fortfahren. Außerdem sollten wir uns darauf einigen, in jeder Runde idealerweise nur ein Objekt zu bewegen, um die Koordination zu erleichtern und Fehler zu minimieren. Wie sieht es bei dir aus? Bist du bereit, mit 'L' zu beginnen?} \\
            }
        }
    }
     \\ \\

    \theutterance \stepcounter{utterance}  
    & & & \multicolumn{2}{p{0.3\linewidth}}{
        \cellcolor[rgb]{0.9,0.9,0.9}{
            \makecell[{{p{\linewidth}}}]{
                \texttt{\tiny{[GM$|$GM]}}
                \texttt{SPRICH: Ich stimme deinem Vorschlag zu, unsere Bewegungen in der Reihenfolge 'L', 'P' und dann 'C' durchzuführen. Das sieht nach einem klaren und pragmatischen Ansatz aus, um Konflikte zu vermeiden. Bevor wir beginnen, denke ich, es wäre hilfreich, wenn wir beide unsere aktuellen Bretter überprüfen, um sicherzustellen, dass der Weg für die erste Bewegung von 'L' frei ist. Ich werde 'L' von (2,3) nach (5,3) bewegen. Bitte überprüfe, ob dies auf deinem Brett problemlos möglich ist und gib mir ein Update, bevor wir fortfahren. Außerdem sollten wir uns darauf einigen, in jeder Runde idealerweise nur ein Objekt zu bewegen, um die Koordination zu erleichtern und Fehler zu minimieren. Wie sieht es bei dir aus? Bist du bereit, mit 'L' zu beginnen?} \\
            }
        }
    }
    & & \\ \\

    \theutterance \stepcounter{utterance}  
    & & \multicolumn{4}{p{0.6\linewidth}}{
        \cellcolor[rgb]{0.9,0.9,0.9}{
            \makecell[{{p{\linewidth}}}]{
                \texttt{\tiny{[P1$\langle$GM]}}
                \texttt{Deine Nachricht wurde an deinen Mitspieler übermittelt.} \\
\\ 
\\ 
\texttt{Ihr seid gerade bei Runde 4 von maximal 12.} \\
\texttt{Ihr habt gemeinsam 0 von maximal 8 Strafpunkten angesammelt.} \\
\texttt{Dein Mitspieler hat dir diese Nachricht geschickt:} \\
\\ 
\\ 
\texttt{"Ich stimme deinem Vorschlag zu, unsere Bewegungen in der Reihenfolge 'L', 'P' und dann 'C' durchzuführen. Das sieht nach einem klaren und pragmatischen Ansatz aus, um Konflikte zu vermeiden. Bevor wir beginnen, denke ich, es wäre hilfreich, wenn wir beide unsere aktuellen Bretter überprüfen, um sicherzustellen, dass der Weg für die erste Bewegung von 'L' frei ist. Ich werde 'L' von (2,3) nach (5,3) bewegen. Bitte überprüfe, ob dies auf deinem Brett problemlos möglich ist und gib mir ein Update, bevor wir fortfahren. Außerdem sollten wir uns darauf einigen, in jeder Runde idealerweise nur ein Objekt zu bewegen, um die Koordination zu erleichtern und Fehler zu minimieren. Wie sieht es bei dir aus? Bist du bereit, mit 'L' zu beginnen?"} \\
\\ 
\\ 
\texttt{Wie lautet dein nächster Befehl?} \\
            }
        }
    }
    & \\ \\

    \theutterance \stepcounter{utterance}  
    & \multicolumn{4}{p{0.6\linewidth}}{
        \cellcolor[rgb]{0.9,0.9,0.9}{
            \makecell[{{p{\linewidth}}}]{
                \texttt{\tiny{[P1$\rangle$GM]}}
                \texttt{SPRICH: Ich habe mein Brett überprüft und ich kann 'L' von (5,5) nach (5,3) bewegen, ohne dass Hindernisse im Weg sind. Dein Plan, 'L' von (2,3) nach (5,3) zu bewegen, scheint also auf beiden Brettern machbar zu sein. Ich stimme dir zu, dass wir in jeder Runde nur ein Objekt bewegen sollten, um die Koordination zu erleichtern und Fehler zu minimieren. Da wir uns auf die Reihenfolge 'L', 'P' und dann 'C' geeinigt haben, können wir fortfahren. Ich werde 'L' von (5,5) nach (5,3) bewegen. Bitte bestätige, dass du 'L' erfolgreich bewegt hast, und wir können zum nächsten Schritt übergehen, nämlich 'P' zu bewegen. VERSCHIEBE: L, (5,3)} \\
            }
        }
    }
    & & \\ \\

    \theutterance \stepcounter{utterance}  
    & & & \multicolumn{2}{p{0.3\linewidth}}{
        \cellcolor[rgb]{0.9,0.9,0.9}{
            \makecell[{{p{\linewidth}}}]{
                \texttt{\tiny{[GM$|$GM]}}
                \texttt{SPRICH: Ich habe mein Brett überprüft und ich kann 'L' von (5,5) nach (5,3) bewegen, ohne dass Hindernisse im Weg sind. Dein Plan, 'L' von (2,3) nach (5,3) zu bewegen, scheint also auf beiden Brettern machbar zu sein. Ich stimme dir zu, dass wir in jeder Runde nur ein Objekt bewegen sollten, um die Koordination zu erleichtern und Fehler zu minimieren. Da wir uns auf die Reihenfolge 'L', 'P' und dann 'C' geeinigt haben, können wir fortfahren. Ich werde 'L' von (5,5) nach (5,3) bewegen. Bitte bestätige, dass du 'L' erfolgreich bewegt hast, und wir können zum nächsten Schritt übergehen, nämlich 'P' zu bewegen. VERSCHIEBE: L, (5,3)} \\
            }
        }
    }
    & & \\ \\

    \theutterance \stepcounter{utterance}  
    & & & \multicolumn{2}{p{0.3\linewidth}}{
        \cellcolor[rgb]{0.9,0.9,0.9}{
            \makecell[{{p{\linewidth}}}]{
                \texttt{\tiny{[GM$|$GM]}}
                \texttt{Invalid response format: SPRICH: Ich habe mein Brett überprüft und ich kann 'L' von (5,5) nach (5,3) bewegen, ohne dass Hindernisse im Weg sind. Dein Plan, 'L' von (2,3) nach (5,3) zu bewegen, scheint also auf beiden Brettern machbar zu sein. Ich stimme dir zu, dass wir in jeder Runde nur ein Objekt bewegen sollten, um die Koordination zu erleichtern und Fehler zu minimieren. Da wir uns auf die Reihenfolge 'L', 'P' und dann 'C' geeinigt haben, können wir fortfahren. Ich werde 'L' von (5,5) nach (5,3) bewegen. Bitte bestätige, dass du 'L' erfolgreich bewegt hast, und wir können zum nächsten Schritt übergehen, nämlich 'P' zu bewegen. VERSCHIEBE: L, (5,3)} \\
            }
        }
    }
    & & \\ \\

    \theutterance \stepcounter{utterance}  
    & & \multicolumn{4}{p{0.6\linewidth}}{
        \cellcolor[rgb]{0.9,0.9,0.9}{
            \makecell[{{p{\linewidth}}}]{
                \texttt{\tiny{[P1$\langle$GM]}}
                \texttt{Strafpunkt: Deine Nachricht enthält mehr als einen Befehl! Stelle sicher, dass deine Antwort ausschließlich einen der beiden Befehle `SPRICH: <NACHRICHT>` und `VERSCHIEBE: <OBJECT>, (<X>, <Y>)` und sonst nichts enthält!} \\
\\ 
\texttt{Ihr habt gemeinsam 1 von maximal 8 Strafpunkten angesammelt.} \\
\texttt{Bitte versuche es noch mal!} \\
            }
        }
    }
    & \\ \\

    \theutterance \stepcounter{utterance}  
    & \multicolumn{4}{p{0.6\linewidth}}{
        \cellcolor[rgb]{0.9,0.9,0.9}{
            \makecell[{{p{\linewidth}}}]{
                \texttt{\tiny{[P1$\rangle$GM]}}
                \texttt{SPRICH: Ich habe mein Brett überprüft und ich kann 'L' von (5,5) nach (5,3) bewegen, ohne dass Hindernisse im Weg sind. Dein Plan, 'L' von (2,3) nach (5,3) zu bewegen, scheint also auf beiden Brettern machbar zu sein. Ich stimme dir zu, dass wir in jeder Runde nur ein Objekt bewegen sollten, um die Koordination zu erleichtern und Fehler zu minimieren. Da wir uns auf die Reihenfolge 'L', 'P' und dann 'C' geeinigt haben, können wir fortfahren. Bitte bestätige, dass du 'L' erfolgreich bewegt hast, und wir können zum nächsten Schritt übergehen, nämlich 'P' zu bewegen. Ich werde dann mein 'L' von (5,5) nach (5,3) bewegen.} \\
            }
        }
    }
    & & \\ \\

    \theutterance \stepcounter{utterance}  
    & & & \multicolumn{2}{p{0.3\linewidth}}{
        \cellcolor[rgb]{0.9,0.9,0.9}{
            \makecell[{{p{\linewidth}}}]{
                \texttt{\tiny{[GM$|$GM]}}
                \texttt{SPRICH: Ich habe mein Brett überprüft und ich kann 'L' von (5,5) nach (5,3) bewegen, ohne dass Hindernisse im Weg sind. Dein Plan, 'L' von (2,3) nach (5,3) zu bewegen, scheint also auf beiden Brettern machbar zu sein. Ich stimme dir zu, dass wir in jeder Runde nur ein Objekt bewegen sollten, um die Koordination zu erleichtern und Fehler zu minimieren. Da wir uns auf die Reihenfolge 'L', 'P' und dann 'C' geeinigt haben, können wir fortfahren. Bitte bestätige, dass du 'L' erfolgreich bewegt hast, und wir können zum nächsten Schritt übergehen, nämlich 'P' zu bewegen. Ich werde dann mein 'L' von (5,5) nach (5,3) bewegen.} \\
            }
        }
    }
    & & \\ \\

    \theutterance \stepcounter{utterance}  
    & & \multicolumn{4}{p{0.6\linewidth}}{
        \cellcolor[rgb]{0.9,0.9,0.9}{
            \makecell[{{p{\linewidth}}}]{
                \texttt{\tiny{[GM$\rangle$P2]}}
                \texttt{Deine Nachricht wurde an deinen Mitspieler übermittelt.} \\
\\ 
\\ 
\texttt{Ihr seid gerade bei Runde 5 von maximal 12.} \\
\texttt{Ihr habt gemeinsam 1 von maximal 8 Strafpunkten angesammelt.} \\
\texttt{Dein Mitspieler hat dir diese Nachricht geschickt:} \\
\\ 
\\ 
\texttt{"Ich habe mein Brett überprüft und ich kann 'L' von (5,5) nach (5,3) bewegen, ohne dass Hindernisse im Weg sind. Dein Plan, 'L' von (2,3) nach (5,3) zu bewegen, scheint also auf beiden Brettern machbar zu sein. Ich stimme dir zu, dass wir in jeder Runde nur ein Objekt bewegen sollten, um die Koordination zu erleichtern und Fehler zu minimieren. Da wir uns auf die Reihenfolge 'L', 'P' und dann 'C' geeinigt haben, können wir fortfahren. Bitte bestätige, dass du 'L' erfolgreich bewegt hast, und wir können zum nächsten Schritt übergehen, nämlich 'P' zu bewegen. Ich werde dann mein 'L' von (5,5) nach (5,3) bewegen."} \\
\\ 
\\ 
\texttt{Wie lautet dein nächster Befehl?} \\
            }
        }
    }
    & \\ \\

    \theutterance \stepcounter{utterance}  
    & & & \multicolumn{4}{p{0.6\linewidth}}{
        \cellcolor[rgb]{0.9,0.9,0.9}{
            \makecell[{{p{\linewidth}}}]{
                \texttt{\tiny{[GM$\langle$P2]}}
                \texttt{VERSCHIEBE: L, (5,3)} \\
            }
        }
    }
     \\ \\

    \theutterance \stepcounter{utterance}  
    & & & \multicolumn{2}{p{0.3\linewidth}}{
        \cellcolor[rgb]{0.9,0.9,0.9}{
            \makecell[{{p{\linewidth}}}]{
                \texttt{\tiny{[GM$|$GM]}}
                \texttt{VERSCHIEBE: L, (5,3)} \\
            }
        }
    }
    & & \\ \\

    \theutterance \stepcounter{utterance}  
    & & & \multicolumn{2}{p{0.3\linewidth}}{
        \cellcolor[rgb]{0.9,0.9,0.9}{
            \makecell[{{p{\linewidth}}}]{
                \texttt{\tiny{[GM$|$GM]}}
                \texttt{'L' erfolgreich nach (5,3) verschoben. Dein Raster sieht jetzt so aus:} \\
\texttt{```} \\
\texttt{ 1234567} \\
\texttt{╔═══╤═╤═╗} \\
\texttt{║◌◌◌│◌│◌║ 1} \\
\texttt{╟─┬─┼─┴─╢ 2} \\
\texttt{║◌│◌│LP◌║ 3} \\
\texttt{║◌├─┤◌┌─╢ 4} \\
\texttt{╟─┘◌│◌│◌║ 5} \\
\texttt{║◌◌◌└┬┴─╢ 6} \\
\texttt{║◌C◌◌│◌◌║ 7} \\
\texttt{╚════╧══╝} \\
\texttt{```} \\
            }
        }
    }
    & & \\ \\

    \theutterance \stepcounter{utterance}  
    & & \multicolumn{4}{p{0.6\linewidth}}{
        \cellcolor[rgb]{0.9,0.9,0.9}{
            \makecell[{{p{\linewidth}}}]{
                \texttt{\tiny{[P1$\langle$GM]}}
                \texttt{Deine Nachricht wurde an deinen Mitspieler übermittelt.} \\
\\ 
\\ 
\texttt{Ihr seid gerade bei Runde 5 von maximal 12.} \\
\texttt{Ihr habt gemeinsam 1 von maximal 8 Strafpunkten angesammelt.} \\
\texttt{Dein Mitspieler hat ein Objekt auf seinem Raster bewegt.} \\
\\ 
\\ 
\texttt{Wie lautet dein nächster Befehl?} \\
            }
        }
    }
    & \\ \\

    \theutterance \stepcounter{utterance}  
    & \multicolumn{4}{p{0.6\linewidth}}{
        \cellcolor[rgb]{0.9,0.9,0.9}{
            \makecell[{{p{\linewidth}}}]{
                \texttt{\tiny{[P1$\rangle$GM]}}
                \texttt{VERSCHIEBE: L, (5,3)} \\
            }
        }
    }
    & & \\ \\

    \theutterance \stepcounter{utterance}  
    & & & \multicolumn{2}{p{0.3\linewidth}}{
        \cellcolor[rgb]{0.9,0.9,0.9}{
            \makecell[{{p{\linewidth}}}]{
                \texttt{\tiny{[GM$|$GM]}}
                \texttt{VERSCHIEBE: L, (5,3)} \\
            }
        }
    }
    & & \\ \\

    \theutterance \stepcounter{utterance}  
    & & & \multicolumn{2}{p{0.3\linewidth}}{
        \cellcolor[rgb]{0.9,0.9,0.9}{
            \makecell[{{p{\linewidth}}}]{
                \texttt{\tiny{[GM$|$GM]}}
                \texttt{Strafpunkt: (5,3) ist nicht leer, sondern enthält 'C'.} \\
\\ 
\texttt{Ihr habt gemeinsam 2 von maximal 8 Strafpunkten angesammelt.} \\
\texttt{Bitte versuche es noch mal!} \\
            }
        }
    }
    & & \\ \\

    \theutterance \stepcounter{utterance}  
    & & \multicolumn{4}{p{0.6\linewidth}}{
        \cellcolor[rgb]{0.9,0.9,0.9}{
            \makecell[{{p{\linewidth}}}]{
                \texttt{\tiny{[P1$\langle$GM]}}
                \texttt{Strafpunkt: (5,3) ist nicht leer, sondern enthält 'C'.} \\
\\ 
\texttt{Ihr habt gemeinsam 2 von maximal 8 Strafpunkten angesammelt.} \\
\texttt{Bitte versuche es noch mal!} \\
            }
        }
    }
    & \\ \\

    \theutterance \stepcounter{utterance}  
    & \multicolumn{4}{p{0.6\linewidth}}{
        \cellcolor[rgb]{0.9,0.9,0.9}{
            \makecell[{{p{\linewidth}}}]{
                \texttt{\tiny{[P1$\rangle$GM]}}
                \texttt{VERSCHIEBE: L, (4,5)} \\
            }
        }
    }
    & & \\ \\

    \theutterance \stepcounter{utterance}  
    & & & \multicolumn{2}{p{0.3\linewidth}}{
        \cellcolor[rgb]{0.9,0.9,0.9}{
            \makecell[{{p{\linewidth}}}]{
                \texttt{\tiny{[GM$|$GM]}}
                \texttt{VERSCHIEBE: L, (4,5)} \\
            }
        }
    }
    & & \\ \\

    \theutterance \stepcounter{utterance}  
    & & & \multicolumn{2}{p{0.3\linewidth}}{
        \cellcolor[rgb]{0.9,0.9,0.9}{
            \makecell[{{p{\linewidth}}}]{
                \texttt{\tiny{[GM$|$GM]}}
                \texttt{Strafpunkt: (4,5) ist nicht leer, sondern enthält '│'.} \\
\\ 
\texttt{Ihr habt gemeinsam 3 von maximal 8 Strafpunkten angesammelt.} \\
\texttt{Bitte versuche es noch mal!} \\
            }
        }
    }
    & & \\ \\

    \theutterance \stepcounter{utterance}  
    & & \multicolumn{4}{p{0.6\linewidth}}{
        \cellcolor[rgb]{0.9,0.9,0.9}{
            \makecell[{{p{\linewidth}}}]{
                \texttt{\tiny{[P1$\langle$GM]}}
                \texttt{Strafpunkt: (4,5) ist nicht leer, sondern enthält '│'.} \\
\\ 
\texttt{Ihr habt gemeinsam 3 von maximal 8 Strafpunkten angesammelt.} \\
\texttt{Bitte versuche es noch mal!} \\
            }
        }
    }
    & \\ \\

    \theutterance \stepcounter{utterance}  
    & \multicolumn{4}{p{0.6\linewidth}}{
        \cellcolor[rgb]{0.9,0.9,0.9}{
            \makecell[{{p{\linewidth}}}]{
                \texttt{\tiny{[P1$\rangle$GM]}}
                \texttt{VERSCHIEBE: L, (3,5)} \\
            }
        }
    }
    & & \\ \\

    \theutterance \stepcounter{utterance}  
    & & & \multicolumn{2}{p{0.3\linewidth}}{
        \cellcolor[rgb]{0.9,0.9,0.9}{
            \makecell[{{p{\linewidth}}}]{
                \texttt{\tiny{[GM$|$GM]}}
                \texttt{VERSCHIEBE: L, (3,5)} \\
            }
        }
    }
    & & \\ \\

    \theutterance \stepcounter{utterance}  
    & & & \multicolumn{2}{p{0.3\linewidth}}{
        \cellcolor[rgb]{0.9,0.9,0.9}{
            \makecell[{{p{\linewidth}}}]{
                \texttt{\tiny{[GM$|$GM]}}
                \texttt{'L' erfolgreich nach (3,5) verschoben. Dein Raster sieht jetzt so aus:} \\
\texttt{```} \\
\texttt{ 1234567} \\
\texttt{╔═══╤═╤═╗} \\
\texttt{║◌◌◌│◌│◌║ 1} \\
\texttt{╟─┬─┼─┴─╢ 2} \\
\texttt{║◌│◌│C◌◌║ 3} \\
\texttt{║◌├─┤◌┌─╢ 4} \\
\texttt{╟─┘L│◌│◌║ 5} \\
\texttt{║◌◌◌└┬┴─╢ 6} \\
\texttt{║P◌◌◌│◌◌║ 7} \\
\texttt{╚════╧══╝} \\
\texttt{```} \\
            }
        }
    }
    & & \\ \\

    \theutterance \stepcounter{utterance}  
    & & \multicolumn{4}{p{0.6\linewidth}}{
        \cellcolor[rgb]{0.9,0.9,0.9}{
            \makecell[{{p{\linewidth}}}]{
                \texttt{\tiny{[GM$\rangle$P2]}}
                \texttt{'L' erfolgreich nach (5,3) verschoben. Dein Raster sieht jetzt so aus:} \\
\texttt{```} \\
\texttt{ 1234567} \\
\texttt{╔═══╤═╤═╗} \\
\texttt{║◌◌◌│◌│◌║ 1} \\
\texttt{╟─┬─┼─┴─╢ 2} \\
\texttt{║◌│◌│LP◌║ 3} \\
\texttt{║◌├─┤◌┌─╢ 4} \\
\texttt{╟─┘◌│◌│◌║ 5} \\
\texttt{║◌◌◌└┬┴─╢ 6} \\
\texttt{║◌C◌◌│◌◌║ 7} \\
\texttt{╚════╧══╝} \\
\texttt{```} \\
\texttt{Ihr seid gerade bei Runde 6 von maximal 12.} \\
\texttt{Ihr habt gemeinsam 3 von maximal 8 Strafpunkten angesammelt.} \\
\texttt{Dein Mitspieler hat ein Objekt auf seinem Raster bewegt.} \\
\\ 
\\ 
\texttt{Wie lautet dein nächster Befehl?} \\
            }
        }
    }
    & \\ \\

    \theutterance \stepcounter{utterance}  
    & & & \multicolumn{4}{p{0.6\linewidth}}{
        \cellcolor[rgb]{0.9,0.9,0.9}{
            \makecell[{{p{\linewidth}}}]{
                \texttt{\tiny{[GM$\langle$P2]}}
                \texttt{SPRICH: Ich habe 'L' erfolgreich zu (5,3) bewegt. Da wir uns auf die Reihenfolge 'L', 'P' und dann 'C' geeinigt haben, ist 'P' als nächstes dran. Mein 'P' ist derzeit bei (5,3), aber ich denke, wir sollten es zu (3,5) bewegen, wie wir besprochen haben. Bevor ich jedoch eine Bewegung durchführe, möchte ich sicherstellen, dass der Weg auf deinem Brett frei ist. Hast du dein 'P' von (1,7) in die richtige Richtung bewegt? Bitte teile mir den aktuellen Stand deines 'P' mit, damit wir unsere nächsten Schritte koordinieren können.} \\
            }
        }
    }
     \\ \\

    \theutterance \stepcounter{utterance}  
    & & & \multicolumn{2}{p{0.3\linewidth}}{
        \cellcolor[rgb]{0.9,0.9,0.9}{
            \makecell[{{p{\linewidth}}}]{
                \texttt{\tiny{[GM$|$GM]}}
                \texttt{SPRICH: Ich habe 'L' erfolgreich zu (5,3) bewegt. Da wir uns auf die Reihenfolge 'L', 'P' und dann 'C' geeinigt haben, ist 'P' als nächstes dran. Mein 'P' ist derzeit bei (5,3), aber ich denke, wir sollten es zu (3,5) bewegen, wie wir besprochen haben. Bevor ich jedoch eine Bewegung durchführe, möchte ich sicherstellen, dass der Weg auf deinem Brett frei ist. Hast du dein 'P' von (1,7) in die richtige Richtung bewegt? Bitte teile mir den aktuellen Stand deines 'P' mit, damit wir unsere nächsten Schritte koordinieren können.} \\
            }
        }
    }
    & & \\ \\

    \theutterance \stepcounter{utterance}  
    & & \multicolumn{4}{p{0.6\linewidth}}{
        \cellcolor[rgb]{0.9,0.9,0.9}{
            \makecell[{{p{\linewidth}}}]{
                \texttt{\tiny{[P1$\langle$GM]}}
                \texttt{'L' erfolgreich nach (3,5) verschoben. Dein Raster sieht jetzt so aus:} \\
\texttt{```} \\
\texttt{ 1234567} \\
\texttt{╔═══╤═╤═╗} \\
\texttt{║◌◌◌│◌│◌║ 1} \\
\texttt{╟─┬─┼─┴─╢ 2} \\
\texttt{║◌│◌│C◌◌║ 3} \\
\texttt{║◌├─┤◌┌─╢ 4} \\
\texttt{╟─┘L│◌│◌║ 5} \\
\texttt{║◌◌◌└┬┴─╢ 6} \\
\texttt{║P◌◌◌│◌◌║ 7} \\
\texttt{╚════╧══╝} \\
\texttt{```} \\
\texttt{Ihr seid gerade bei Runde 6 von maximal 12.} \\
\texttt{Ihr habt gemeinsam 3 von maximal 8 Strafpunkten angesammelt.} \\
\texttt{Dein Mitspieler hat dir diese Nachricht geschickt:} \\
\\ 
\\ 
\texttt{"Ich habe 'L' erfolgreich zu (5,3) bewegt. Da wir uns auf die Reihenfolge 'L', 'P' und dann 'C' geeinigt haben, ist 'P' als nächstes dran. Mein 'P' ist derzeit bei (5,3), aber ich denke, wir sollten es zu (3,5) bewegen, wie wir besprochen haben. Bevor ich jedoch eine Bewegung durchführe, möchte ich sicherstellen, dass der Weg auf deinem Brett frei ist. Hast du dein 'P' von (1,7) in die richtige Richtung bewegt? Bitte teile mir den aktuellen Stand deines 'P' mit, damit wir unsere nächsten Schritte koordinieren können."} \\
\\ 
\\ 
\texttt{Wie lautet dein nächster Befehl?} \\
            }
        }
    }
    & \\ \\

    \theutterance \stepcounter{utterance}  
    & \multicolumn{4}{p{0.6\linewidth}}{
        \cellcolor[rgb]{0.9,0.9,0.9}{
            \makecell[{{p{\linewidth}}}]{
                \texttt{\tiny{[P1$\rangle$GM]}}
                \texttt{SPRICH: Mein 'P' ist noch bei (1,7) und ich habe es noch nicht bewegt. Ich denke, es wäre sinnvoll, mein 'P' nach (3,5) zu bewegen, um es in die richtige Position zu bringen. Allerdings ist (3,5) auf meinem Brett bereits von 'L' besetzt. Ich denke, wir sollten unsere Strategie anpassen und ein alternatives Ziel für 'P' finden. Wie wäre es, wenn wir 'P' zu (3,4) bewegen? Bitte teile mir deine Gedanken mit, und wir können unsere nächsten Schritte koordinieren.} \\
            }
        }
    }
    & & \\ \\

    \theutterance \stepcounter{utterance}  
    & & & \multicolumn{2}{p{0.3\linewidth}}{
        \cellcolor[rgb]{0.9,0.9,0.9}{
            \makecell[{{p{\linewidth}}}]{
                \texttt{\tiny{[GM$|$GM]}}
                \texttt{SPRICH: Mein 'P' ist noch bei (1,7) und ich habe es noch nicht bewegt. Ich denke, es wäre sinnvoll, mein 'P' nach (3,5) zu bewegen, um es in die richtige Position zu bringen. Allerdings ist (3,5) auf meinem Brett bereits von 'L' besetzt. Ich denke, wir sollten unsere Strategie anpassen und ein alternatives Ziel für 'P' finden. Wie wäre es, wenn wir 'P' zu (3,4) bewegen? Bitte teile mir deine Gedanken mit, und wir können unsere nächsten Schritte koordinieren.} \\
            }
        }
    }
    & & \\ \\

    \theutterance \stepcounter{utterance}  
    & & \multicolumn{4}{p{0.6\linewidth}}{
        \cellcolor[rgb]{0.9,0.9,0.9}{
            \makecell[{{p{\linewidth}}}]{
                \texttt{\tiny{[GM$\rangle$P2]}}
                \texttt{Deine Nachricht wurde an deinen Mitspieler übermittelt.} \\
\\ 
\\ 
\texttt{Ihr seid gerade bei Runde 7 von maximal 12.} \\
\texttt{Ihr habt gemeinsam 3 von maximal 8 Strafpunkten angesammelt.} \\
\texttt{Dein Mitspieler hat dir diese Nachricht geschickt:} \\
\\ 
\\ 
\texttt{"Mein 'P' ist noch bei (1,7) und ich habe es noch nicht bewegt. Ich denke, es wäre sinnvoll, mein 'P' nach (3,5) zu bewegen, um es in die richtige Position zu bringen. Allerdings ist (3,5) auf meinem Brett bereits von 'L' besetzt. Ich denke, wir sollten unsere Strategie anpassen und ein alternatives Ziel für 'P' finden. Wie wäre es, wenn wir 'P' zu (3,4) bewegen? Bitte teile mir deine Gedanken mit, und wir können unsere nächsten Schritte koordinieren."} \\
\\ 
\\ 
\texttt{Wie lautet dein nächster Befehl?} \\
            }
        }
    }
    & \\ \\

    \theutterance \stepcounter{utterance}  
    & & & \multicolumn{4}{p{0.6\linewidth}}{
        \cellcolor[rgb]{0.9,0.9,0.9}{
            \makecell[{{p{\linewidth}}}]{
                \texttt{\tiny{[GM$\langle$P2]}}
                \texttt{SPRICH: Das ist eine gute Beobachtung, dass (3,5) auf deinem Brett von 'L' besetzt ist. Ich denke, (3,4) als alternatives Ziel für 'P' ist eine vernünftige Wahl. Auf meinem Brett ist (3,4) derzeit frei, also könnte ich mein 'P' theoretisch dorthin bewegen. Bevor wir fortfahren, sollten wir jedoch sicherstellen, dass wir beide dieselbe Vorstellung davon haben, wie unsere Bretter aussehen und wie die Objekte bewegt werden sollen. Da wir bereits 'L' bewegt haben, sollten wir uns nun auf 'P' konzentrieren. Ich werde mein 'P' von (5,3) nach (3,4) bewegen. Bitte überprüfe, ob auf deinem Brett der Weg für 'P' von (1,7) nach (3,4) frei ist und gib mir ein Update, bevor du deine Bewegung durchführst.} \\
            }
        }
    }
     \\ \\

    \theutterance \stepcounter{utterance}  
    & & & \multicolumn{2}{p{0.3\linewidth}}{
        \cellcolor[rgb]{0.9,0.9,0.9}{
            \makecell[{{p{\linewidth}}}]{
                \texttt{\tiny{[GM$|$GM]}}
                \texttt{SPRICH: Das ist eine gute Beobachtung, dass (3,5) auf deinem Brett von 'L' besetzt ist. Ich denke, (3,4) als alternatives Ziel für 'P' ist eine vernünftige Wahl. Auf meinem Brett ist (3,4) derzeit frei, also könnte ich mein 'P' theoretisch dorthin bewegen. Bevor wir fortfahren, sollten wir jedoch sicherstellen, dass wir beide dieselbe Vorstellung davon haben, wie unsere Bretter aussehen und wie die Objekte bewegt werden sollen. Da wir bereits 'L' bewegt haben, sollten wir uns nun auf 'P' konzentrieren. Ich werde mein 'P' von (5,3) nach (3,4) bewegen. Bitte überprüfe, ob auf deinem Brett der Weg für 'P' von (1,7) nach (3,4) frei ist und gib mir ein Update, bevor du deine Bewegung durchführst.} \\
            }
        }
    }
    & & \\ \\

    \theutterance \stepcounter{utterance}  
    & & \multicolumn{4}{p{0.6\linewidth}}{
        \cellcolor[rgb]{0.9,0.9,0.9}{
            \makecell[{{p{\linewidth}}}]{
                \texttt{\tiny{[P1$\langle$GM]}}
                \texttt{Deine Nachricht wurde an deinen Mitspieler übermittelt.} \\
\\ 
\\ 
\texttt{Ihr seid gerade bei Runde 7 von maximal 12.} \\
\texttt{Ihr habt gemeinsam 3 von maximal 8 Strafpunkten angesammelt.} \\
\texttt{Dein Mitspieler hat dir diese Nachricht geschickt:} \\
\\ 
\\ 
\texttt{"Das ist eine gute Beobachtung, dass (3,5) auf deinem Brett von 'L' besetzt ist. Ich denke, (3,4) als alternatives Ziel für 'P' ist eine vernünftige Wahl. Auf meinem Brett ist (3,4) derzeit frei, also könnte ich mein 'P' theoretisch dorthin bewegen. Bevor wir fortfahren, sollten wir jedoch sicherstellen, dass wir beide dieselbe Vorstellung davon haben, wie unsere Bretter aussehen und wie die Objekte bewegt werden sollen. Da wir bereits 'L' bewegt haben, sollten wir uns nun auf 'P' konzentrieren. Ich werde mein 'P' von (5,3) nach (3,4) bewegen. Bitte überprüfe, ob auf deinem Brett der Weg für 'P' von (1,7) nach (3,4) frei ist und gib mir ein Update, bevor du deine Bewegung durchführst."} \\
\\ 
\\ 
\texttt{Wie lautet dein nächster Befehl?} \\
            }
        }
    }
    & \\ \\

    \theutterance \stepcounter{utterance}  
    & \multicolumn{4}{p{0.6\linewidth}}{
        \cellcolor[rgb]{0.9,0.9,0.9}{
            \makecell[{{p{\linewidth}}}]{
                \texttt{\tiny{[P1$\rangle$GM]}}
                \texttt{SPRICH: Ich habe mein Brett überprüft und der Weg für 'P' von (1,7) nach (3,4) ist frei. Ich kann mein 'P' ohne Hindernisse dorthin bewegen. Es sieht so aus, als ob wir auf dem richtigen Weg sind, unsere Objekte an die gewünschten Positionen zu bewegen. Bitte bestätige, dass du dein 'P' erfolgreich zu (3,4) bewegt hast, und ich werde mein 'P' von (1,7) nach (3,4) bewegen, um unsere Strategie fortzusetzen.} \\
            }
        }
    }
    & & \\ \\

    \theutterance \stepcounter{utterance}  
    & & & \multicolumn{2}{p{0.3\linewidth}}{
        \cellcolor[rgb]{0.9,0.9,0.9}{
            \makecell[{{p{\linewidth}}}]{
                \texttt{\tiny{[GM$|$GM]}}
                \texttt{SPRICH: Ich habe mein Brett überprüft und der Weg für 'P' von (1,7) nach (3,4) ist frei. Ich kann mein 'P' ohne Hindernisse dorthin bewegen. Es sieht so aus, als ob wir auf dem richtigen Weg sind, unsere Objekte an die gewünschten Positionen zu bewegen. Bitte bestätige, dass du dein 'P' erfolgreich zu (3,4) bewegt hast, und ich werde mein 'P' von (1,7) nach (3,4) bewegen, um unsere Strategie fortzusetzen.} \\
            }
        }
    }
    & & \\ \\

    \theutterance \stepcounter{utterance}  
    & & \multicolumn{4}{p{0.6\linewidth}}{
        \cellcolor[rgb]{0.9,0.9,0.9}{
            \makecell[{{p{\linewidth}}}]{
                \texttt{\tiny{[GM$\rangle$P2]}}
                \texttt{Deine Nachricht wurde an deinen Mitspieler übermittelt.} \\
\\ 
\\ 
\texttt{Ihr seid gerade bei Runde 8 von maximal 12.} \\
\texttt{Ihr habt gemeinsam 3 von maximal 8 Strafpunkten angesammelt.} \\
\texttt{Dein Mitspieler hat dir diese Nachricht geschickt:} \\
\\ 
\\ 
\texttt{"Ich habe mein Brett überprüft und der Weg für 'P' von (1,7) nach (3,4) ist frei. Ich kann mein 'P' ohne Hindernisse dorthin bewegen. Es sieht so aus, als ob wir auf dem richtigen Weg sind, unsere Objekte an die gewünschten Positionen zu bewegen. Bitte bestätige, dass du dein 'P' erfolgreich zu (3,4) bewegt hast, und ich werde mein 'P' von (1,7) nach (3,4) bewegen, um unsere Strategie fortzusetzen."} \\
\\ 
\\ 
\texttt{Wie lautet dein nächster Befehl?} \\
            }
        }
    }
    & \\ \\

    \theutterance \stepcounter{utterance}  
    & & & \multicolumn{4}{p{0.6\linewidth}}{
        \cellcolor[rgb]{0.9,0.9,0.9}{
            \makecell[{{p{\linewidth}}}]{
                \texttt{\tiny{[GM$\langle$P2]}}
                \texttt{VERSCHIEBE: P, (3,4)} \\
            }
        }
    }
     \\ \\

    \theutterance \stepcounter{utterance}  
    & & & \multicolumn{2}{p{0.3\linewidth}}{
        \cellcolor[rgb]{0.9,0.9,0.9}{
            \makecell[{{p{\linewidth}}}]{
                \texttt{\tiny{[GM$|$GM]}}
                \texttt{VERSCHIEBE: P, (3,4)} \\
            }
        }
    }
    & & \\ \\

    \theutterance \stepcounter{utterance}  
    & & & \multicolumn{2}{p{0.3\linewidth}}{
        \cellcolor[rgb]{0.9,0.9,0.9}{
            \makecell[{{p{\linewidth}}}]{
                \texttt{\tiny{[GM$|$GM]}}
                \texttt{Strafpunkt: (3,4) ist nicht leer, sondern enthält '─'.} \\
\\ 
\texttt{Ihr habt gemeinsam 4 von maximal 8 Strafpunkten angesammelt.} \\
\texttt{Bitte versuche es noch mal!} \\
            }
        }
    }
    & & \\ \\

    \theutterance \stepcounter{utterance}  
    & & \multicolumn{4}{p{0.6\linewidth}}{
        \cellcolor[rgb]{0.9,0.9,0.9}{
            \makecell[{{p{\linewidth}}}]{
                \texttt{\tiny{[GM$\rangle$P2]}}
                \texttt{Strafpunkt: (3,4) ist nicht leer, sondern enthält '─'.} \\
\\ 
\texttt{Ihr habt gemeinsam 4 von maximal 8 Strafpunkten angesammelt.} \\
\texttt{Bitte versuche es noch mal!} \\
            }
        }
    }
    & \\ \\

    \theutterance \stepcounter{utterance}  
    & & & \multicolumn{4}{p{0.6\linewidth}}{
        \cellcolor[rgb]{0.9,0.9,0.9}{
            \makecell[{{p{\linewidth}}}]{
                \texttt{\tiny{[GM$\langle$P2]}}
                \texttt{SPRICH: Es sieht so aus, als hätten wir einen kleinen Rückschlag erlitten. Mein Versuch, 'P' zu (3,4) zu bewegen, ist fehlgeschlagen, weil (3,4) nicht leer ist. Lass uns unsere Strategie überdenken und einen neuen Plan entwickeln. Wie wäre es, wenn wir 'P' zu (3,5) bewegen, wie ursprünglich geplant, und dann 'L' anpassen, falls nötig? Oder sollten wir ein alternatives Ziel für 'P' finden? Bitte teile mir deine Gedanken mit, und wir können unsere nächsten Schritte koordinieren.} \\
            }
        }
    }
     \\ \\

    \theutterance \stepcounter{utterance}  
    & & & \multicolumn{2}{p{0.3\linewidth}}{
        \cellcolor[rgb]{0.9,0.9,0.9}{
            \makecell[{{p{\linewidth}}}]{
                \texttt{\tiny{[GM$|$GM]}}
                \texttt{SPRICH: Es sieht so aus, als hätten wir einen kleinen Rückschlag erlitten. Mein Versuch, 'P' zu (3,4) zu bewegen, ist fehlgeschlagen, weil (3,4) nicht leer ist. Lass uns unsere Strategie überdenken und einen neuen Plan entwickeln. Wie wäre es, wenn wir 'P' zu (3,5) bewegen, wie ursprünglich geplant, und dann 'L' anpassen, falls nötig? Oder sollten wir ein alternatives Ziel für 'P' finden? Bitte teile mir deine Gedanken mit, und wir können unsere nächsten Schritte koordinieren.} \\
            }
        }
    }
    & & \\ \\

    \theutterance \stepcounter{utterance}  
    & & \multicolumn{4}{p{0.6\linewidth}}{
        \cellcolor[rgb]{0.9,0.9,0.9}{
            \makecell[{{p{\linewidth}}}]{
                \texttt{\tiny{[P1$\langle$GM]}}
                \texttt{Deine Nachricht wurde an deinen Mitspieler übermittelt.} \\
\\ 
\\ 
\texttt{Ihr seid gerade bei Runde 8 von maximal 12.} \\
\texttt{Ihr habt gemeinsam 4 von maximal 8 Strafpunkten angesammelt.} \\
\texttt{Dein Mitspieler hat dir diese Nachricht geschickt:} \\
\\ 
\\ 
\texttt{"Es sieht so aus, als hätten wir einen kleinen Rückschlag erlitten. Mein Versuch, 'P' zu (3,4) zu bewegen, ist fehlgeschlagen, weil (3,4) nicht leer ist. Lass uns unsere Strategie überdenken und einen neuen Plan entwickeln. Wie wäre es, wenn wir 'P' zu (3,5) bewegen, wie ursprünglich geplant, und dann 'L' anpassen, falls nötig? Oder sollten wir ein alternatives Ziel für 'P' finden? Bitte teile mir deine Gedanken mit, und wir können unsere nächsten Schritte koordinieren."} \\
\\ 
\\ 
\texttt{Wie lautet dein nächster Befehl?} \\
            }
        }
    }
    & \\ \\

    \theutterance \stepcounter{utterance}  
    & \multicolumn{4}{p{0.6\linewidth}}{
        \cellcolor[rgb]{0.9,0.9,0.9}{
            \makecell[{{p{\linewidth}}}]{
                \texttt{\tiny{[P1$\rangle$GM]}}
                \texttt{SPRICH: Es tut mir leid zu hören, dass der Versuch, 'P' zu (3,4) zu bewegen, fehlgeschlagen ist. Ich denke, es wäre vernünftig, unsere ursprüngliche Strategie zu überdenken und alternative Ziele für 'P' zu betrachten. Wie wäre es, wenn wir 'P' zu (2,5) bewegen? Dies könnte eine gute Alternative sein, um 'P' in eine Position zu bringen, die für beide von uns zugänglich ist. Bitte teile mir deine Gedanken mit, und wir können unsere nächsten Schritte koordinieren. Wir sollten auch 'L' und 'C' im Auge behalten, um sicherzustellen, dass unsere Bewegungen koordiniert sind und wir unsere Ziele erreichen.} \\
            }
        }
    }
    & & \\ \\

    \theutterance \stepcounter{utterance}  
    & & & \multicolumn{2}{p{0.3\linewidth}}{
        \cellcolor[rgb]{0.9,0.9,0.9}{
            \makecell[{{p{\linewidth}}}]{
                \texttt{\tiny{[GM$|$GM]}}
                \texttt{SPRICH: Es tut mir leid zu hören, dass der Versuch, 'P' zu (3,4) zu bewegen, fehlgeschlagen ist. Ich denke, es wäre vernünftig, unsere ursprüngliche Strategie zu überdenken und alternative Ziele für 'P' zu betrachten. Wie wäre es, wenn wir 'P' zu (2,5) bewegen? Dies könnte eine gute Alternative sein, um 'P' in eine Position zu bringen, die für beide von uns zugänglich ist. Bitte teile mir deine Gedanken mit, und wir können unsere nächsten Schritte koordinieren. Wir sollten auch 'L' und 'C' im Auge behalten, um sicherzustellen, dass unsere Bewegungen koordiniert sind und wir unsere Ziele erreichen.} \\
            }
        }
    }
    & & \\ \\

    \theutterance \stepcounter{utterance}  
    & & \multicolumn{4}{p{0.6\linewidth}}{
        \cellcolor[rgb]{0.9,0.9,0.9}{
            \makecell[{{p{\linewidth}}}]{
                \texttt{\tiny{[GM$\rangle$P2]}}
                \texttt{Deine Nachricht wurde an deinen Mitspieler übermittelt.} \\
\\ 
\\ 
\texttt{Ihr seid gerade bei Runde 9 von maximal 12.} \\
\texttt{Ihr habt gemeinsam 4 von maximal 8 Strafpunkten angesammelt.} \\
\texttt{Dein Mitspieler hat dir diese Nachricht geschickt:} \\
\\ 
\\ 
\texttt{"Es tut mir leid zu hören, dass der Versuch, 'P' zu (3,4) zu bewegen, fehlgeschlagen ist. Ich denke, es wäre vernünftig, unsere ursprüngliche Strategie zu überdenken und alternative Ziele für 'P' zu betrachten. Wie wäre es, wenn wir 'P' zu (2,5) bewegen? Dies könnte eine gute Alternative sein, um 'P' in eine Position zu bringen, die für beide von uns zugänglich ist. Bitte teile mir deine Gedanken mit, und wir können unsere nächsten Schritte koordinieren. Wir sollten auch 'L' und 'C' im Auge behalten, um sicherzustellen, dass unsere Bewegungen koordiniert sind und wir unsere Ziele erreichen."} \\
\\ 
\\ 
\texttt{Wie lautet dein nächster Befehl?} \\
            }
        }
    }
    & \\ \\

    \theutterance \stepcounter{utterance}  
    & & & \multicolumn{4}{p{0.6\linewidth}}{
        \cellcolor[rgb]{0.9,0.9,0.9}{
            \makecell[{{p{\linewidth}}}]{
                \texttt{\tiny{[GM$\langle$P2]}}
                \texttt{SPRICH: Ich denke, (2,5) könnte tatsächlich eine gute Alternative für 'P' sein. Auf meinem Brett ist (2,5) derzeit leer, also könnte ich mein 'P' dorthin bewegen. Bevor ich jedoch eine Bewegung durchführe, möchte ich sicherstellen, dass der Weg auf deinem Brett frei ist und dass wir beide dieselbe Vorstellung davon haben, wie unsere Bretter aussehen. Bitte bestätige, ob auf deinem Brett der Weg für 'P' von (1,7) nach (2,5) frei ist. Außerdem sollten wir überlegen, wie wir 'L' und 'C' bewegen, um sie an ihre endgültigen Positionen zu bringen, nachdem wir 'P' bewegt haben. Sollten wir uns auf eine feste Reihenfolge für die Bewegung dieser Objekte einigen, um sicherzustellen, dass unsere Strategie koordiniert bleibt?} \\
            }
        }
    }
     \\ \\

    \theutterance \stepcounter{utterance}  
    & & & \multicolumn{2}{p{0.3\linewidth}}{
        \cellcolor[rgb]{0.9,0.9,0.9}{
            \makecell[{{p{\linewidth}}}]{
                \texttt{\tiny{[GM$|$GM]}}
                \texttt{SPRICH: Ich denke, (2,5) könnte tatsächlich eine gute Alternative für 'P' sein. Auf meinem Brett ist (2,5) derzeit leer, also könnte ich mein 'P' dorthin bewegen. Bevor ich jedoch eine Bewegung durchführe, möchte ich sicherstellen, dass der Weg auf deinem Brett frei ist und dass wir beide dieselbe Vorstellung davon haben, wie unsere Bretter aussehen. Bitte bestätige, ob auf deinem Brett der Weg für 'P' von (1,7) nach (2,5) frei ist. Außerdem sollten wir überlegen, wie wir 'L' und 'C' bewegen, um sie an ihre endgültigen Positionen zu bringen, nachdem wir 'P' bewegt haben. Sollten wir uns auf eine feste Reihenfolge für die Bewegung dieser Objekte einigen, um sicherzustellen, dass unsere Strategie koordiniert bleibt?} \\
            }
        }
    }
    & & \\ \\

    \theutterance \stepcounter{utterance}  
    & & \multicolumn{4}{p{0.6\linewidth}}{
        \cellcolor[rgb]{0.9,0.9,0.9}{
            \makecell[{{p{\linewidth}}}]{
                \texttt{\tiny{[P1$\langle$GM]}}
                \texttt{Deine Nachricht wurde an deinen Mitspieler übermittelt.} \\
\\ 
\\ 
\texttt{Ihr seid gerade bei Runde 9 von maximal 12.} \\
\texttt{Ihr habt gemeinsam 4 von maximal 8 Strafpunkten angesammelt.} \\
\texttt{Dein Mitspieler hat dir diese Nachricht geschickt:} \\
\\ 
\\ 
\texttt{"Ich denke, (2,5) könnte tatsächlich eine gute Alternative für 'P' sein. Auf meinem Brett ist (2,5) derzeit leer, also könnte ich mein 'P' dorthin bewegen. Bevor ich jedoch eine Bewegung durchführe, möchte ich sicherstellen, dass der Weg auf deinem Brett frei ist und dass wir beide dieselbe Vorstellung davon haben, wie unsere Bretter aussehen. Bitte bestätige, ob auf deinem Brett der Weg für 'P' von (1,7) nach (2,5) frei ist. Außerdem sollten wir überlegen, wie wir 'L' und 'C' bewegen, um sie an ihre endgültigen Positionen zu bringen, nachdem wir 'P' bewegt haben. Sollten wir uns auf eine feste Reihenfolge für die Bewegung dieser Objekte einigen, um sicherzustellen, dass unsere Strategie koordiniert bleibt?"} \\
\\ 
\\ 
\texttt{Wie lautet dein nächster Befehl?} \\
            }
        }
    }
    & \\ \\

    \theutterance \stepcounter{utterance}  
    & \multicolumn{4}{p{0.6\linewidth}}{
        \cellcolor[rgb]{0.9,0.9,0.9}{
            \makecell[{{p{\linewidth}}}]{
                \texttt{\tiny{[P1$\rangle$GM]}}
                \texttt{SPRICH: Ich habe mein Brett überprüft und der Weg für 'P' von (1,7) nach (2,5) ist frei. Ich kann mein 'P' ohne Hindernisse dorthin bewegen. Bezüglich 'L' und 'C' denke ich, es wäre sinnvoll, eine feste Reihenfolge für ihre Bewegung zu vereinbaren, um unsere Strategie koordiniert zu halten. Wie wäre es, wenn wir nach der Bewegung von 'P' 'C' bewegen und dann 'L' anpassen, falls nötig? Dies könnte uns helfen, unsere Ziele effizienter zu erreichen und potenzielle Konflikte zu vermeiden. Bitte teile mir deine Gedanken mit, und wir können unsere Strategie anpassen.} \\
            }
        }
    }
    & & \\ \\

    \theutterance \stepcounter{utterance}  
    & & & \multicolumn{2}{p{0.3\linewidth}}{
        \cellcolor[rgb]{0.9,0.9,0.9}{
            \makecell[{{p{\linewidth}}}]{
                \texttt{\tiny{[GM$|$GM]}}
                \texttt{SPRICH: Ich habe mein Brett überprüft und der Weg für 'P' von (1,7) nach (2,5) ist frei. Ich kann mein 'P' ohne Hindernisse dorthin bewegen. Bezüglich 'L' und 'C' denke ich, es wäre sinnvoll, eine feste Reihenfolge für ihre Bewegung zu vereinbaren, um unsere Strategie koordiniert zu halten. Wie wäre es, wenn wir nach der Bewegung von 'P' 'C' bewegen und dann 'L' anpassen, falls nötig? Dies könnte uns helfen, unsere Ziele effizienter zu erreichen und potenzielle Konflikte zu vermeiden. Bitte teile mir deine Gedanken mit, und wir können unsere Strategie anpassen.} \\
            }
        }
    }
    & & \\ \\

    \theutterance \stepcounter{utterance}  
    & & \multicolumn{4}{p{0.6\linewidth}}{
        \cellcolor[rgb]{0.9,0.9,0.9}{
            \makecell[{{p{\linewidth}}}]{
                \texttt{\tiny{[GM$\rangle$P2]}}
                \texttt{Deine Nachricht wurde an deinen Mitspieler übermittelt.} \\
\\ 
\\ 
\texttt{Ihr seid gerade bei Runde 10 von maximal 12.} \\
\texttt{Ihr habt gemeinsam 4 von maximal 8 Strafpunkten angesammelt.} \\
\texttt{Dein Mitspieler hat dir diese Nachricht geschickt:} \\
\\ 
\\ 
\texttt{"Ich habe mein Brett überprüft und der Weg für 'P' von (1,7) nach (2,5) ist frei. Ich kann mein 'P' ohne Hindernisse dorthin bewegen. Bezüglich 'L' und 'C' denke ich, es wäre sinnvoll, eine feste Reihenfolge für ihre Bewegung zu vereinbaren, um unsere Strategie koordiniert zu halten. Wie wäre es, wenn wir nach der Bewegung von 'P' 'C' bewegen und dann 'L' anpassen, falls nötig? Dies könnte uns helfen, unsere Ziele effizienter zu erreichen und potenzielle Konflikte zu vermeiden. Bitte teile mir deine Gedanken mit, und wir können unsere Strategie anpassen."} \\
\\ 
\\ 
\texttt{Wie lautet dein nächster Befehl?} \\
            }
        }
    }
    & \\ \\

    \theutterance \stepcounter{utterance}  
    & & & \multicolumn{4}{p{0.6\linewidth}}{
        \cellcolor[rgb]{0.9,0.9,0.9}{
            \makecell[{{p{\linewidth}}}]{
                \texttt{\tiny{[GM$\langle$P2]}}
                \texttt{SPRICH: Ich stimme deiner Vorschlag zu, nach der Bewegung von 'P' 'C' zu bewegen und dann 'L' anzupassen, falls nötig. Diese Reihenfolge scheint sinnvoll, um unsere Ziele zu erreichen und potenzielle Konflikte zu vermeiden. Bevor wir fortfahren, sollte ich jedoch mein 'P' von (5,3) nach (2,5) bewegen, um unsere Strategie umzusetzen. Ich werde mich bemühen, dies in der nächsten Runde zu tun. Bitte bereite dich darauf vor, dein 'P' von (1,7) nach (2,5) zu bewegen, sobald ich meine Bewegung bestätigt habe. Wir sollten auch unsere Bretter überwachen, um sicherzustellen, dass keine unerwarteten Hindernisse auftreten. Wie sieht es bei dir aus? Bist du bereit, dein 'P' zu bewegen, sobald ich meine Aktion durchgeführt habe?} \\
            }
        }
    }
     \\ \\

    \theutterance \stepcounter{utterance}  
    & & & \multicolumn{2}{p{0.3\linewidth}}{
        \cellcolor[rgb]{0.9,0.9,0.9}{
            \makecell[{{p{\linewidth}}}]{
                \texttt{\tiny{[GM$|$GM]}}
                \texttt{SPRICH: Ich stimme deiner Vorschlag zu, nach der Bewegung von 'P' 'C' zu bewegen und dann 'L' anzupassen, falls nötig. Diese Reihenfolge scheint sinnvoll, um unsere Ziele zu erreichen und potenzielle Konflikte zu vermeiden. Bevor wir fortfahren, sollte ich jedoch mein 'P' von (5,3) nach (2,5) bewegen, um unsere Strategie umzusetzen. Ich werde mich bemühen, dies in der nächsten Runde zu tun. Bitte bereite dich darauf vor, dein 'P' von (1,7) nach (2,5) zu bewegen, sobald ich meine Bewegung bestätigt habe. Wir sollten auch unsere Bretter überwachen, um sicherzustellen, dass keine unerwarteten Hindernisse auftreten. Wie sieht es bei dir aus? Bist du bereit, dein 'P' zu bewegen, sobald ich meine Aktion durchgeführt habe?} \\
            }
        }
    }
    & & \\ \\

    \theutterance \stepcounter{utterance}  
    & & \multicolumn{4}{p{0.6\linewidth}}{
        \cellcolor[rgb]{0.9,0.9,0.9}{
            \makecell[{{p{\linewidth}}}]{
                \texttt{\tiny{[P1$\langle$GM]}}
                \texttt{Deine Nachricht wurde an deinen Mitspieler übermittelt.} \\
\\ 
\\ 
\texttt{Ihr seid gerade bei Runde 10 von maximal 12.} \\
\texttt{Ihr habt gemeinsam 4 von maximal 8 Strafpunkten angesammelt.} \\
\texttt{Dein Mitspieler hat dir diese Nachricht geschickt:} \\
\\ 
\\ 
\texttt{"Ich stimme deiner Vorschlag zu, nach der Bewegung von 'P' 'C' zu bewegen und dann 'L' anzupassen, falls nötig. Diese Reihenfolge scheint sinnvoll, um unsere Ziele zu erreichen und potenzielle Konflikte zu vermeiden. Bevor wir fortfahren, sollte ich jedoch mein 'P' von (5,3) nach (2,5) bewegen, um unsere Strategie umzusetzen. Ich werde mich bemühen, dies in der nächsten Runde zu tun. Bitte bereite dich darauf vor, dein 'P' von (1,7) nach (2,5) zu bewegen, sobald ich meine Bewegung bestätigt habe. Wir sollten auch unsere Bretter überwachen, um sicherzustellen, dass keine unerwarteten Hindernisse auftreten. Wie sieht es bei dir aus? Bist du bereit, dein 'P' zu bewegen, sobald ich meine Aktion durchgeführt habe?"} \\
\\ 
\\ 
\texttt{Wie lautet dein nächster Befehl?} \\
            }
        }
    }
    & \\ \\

    \theutterance \stepcounter{utterance}  
    & \multicolumn{4}{p{0.6\linewidth}}{
        \cellcolor[rgb]{0.9,0.9,0.9}{
            \makecell[{{p{\linewidth}}}]{
                \texttt{\tiny{[P1$\rangle$GM]}}
                \texttt{SPRICH: Ja, ich bin bereit, mein 'P' von (1,7) nach (2,5) zu bewegen, sobald du deine Bewegung bestätigt hast. Ich werde mein Brett überwachen, um sicherzustellen, dass keine unerwarteten Hindernisse auftreten. Bitte bestätige, sobald du dein 'P' erfolgreich bewegt hast, und ich werde mein 'P' von (1,7) nach (2,5) bewegen, um unsere Strategie umzusetzen. Ich bin gespannt, wie unsere nächsten Schritte aussehen werden und ob wir unsere Ziele erreichen können.} \\
            }
        }
    }
    & & \\ \\

    \theutterance \stepcounter{utterance}  
    & & & \multicolumn{2}{p{0.3\linewidth}}{
        \cellcolor[rgb]{0.9,0.9,0.9}{
            \makecell[{{p{\linewidth}}}]{
                \texttt{\tiny{[GM$|$GM]}}
                \texttt{SPRICH: Ja, ich bin bereit, mein 'P' von (1,7) nach (2,5) zu bewegen, sobald du deine Bewegung bestätigt hast. Ich werde mein Brett überwachen, um sicherzustellen, dass keine unerwarteten Hindernisse auftreten. Bitte bestätige, sobald du dein 'P' erfolgreich bewegt hast, und ich werde mein 'P' von (1,7) nach (2,5) bewegen, um unsere Strategie umzusetzen. Ich bin gespannt, wie unsere nächsten Schritte aussehen werden und ob wir unsere Ziele erreichen können.} \\
            }
        }
    }
    & & \\ \\

    \theutterance \stepcounter{utterance}  
    & & \multicolumn{4}{p{0.6\linewidth}}{
        \cellcolor[rgb]{0.9,0.9,0.9}{
            \makecell[{{p{\linewidth}}}]{
                \texttt{\tiny{[GM$\rangle$P2]}}
                \texttt{Deine Nachricht wurde an deinen Mitspieler übermittelt.} \\
\\ 
\\ 
\texttt{Ihr seid gerade bei Runde 11 von maximal 12.} \\
\texttt{Ihr habt gemeinsam 4 von maximal 8 Strafpunkten angesammelt.} \\
\texttt{Dein Mitspieler hat dir diese Nachricht geschickt:} \\
\\ 
\\ 
\texttt{"Ja, ich bin bereit, mein 'P' von (1,7) nach (2,5) zu bewegen, sobald du deine Bewegung bestätigt hast. Ich werde mein Brett überwachen, um sicherzustellen, dass keine unerwarteten Hindernisse auftreten. Bitte bestätige, sobald du dein 'P' erfolgreich bewegt hast, und ich werde mein 'P' von (1,7) nach (2,5) bewegen, um unsere Strategie umzusetzen. Ich bin gespannt, wie unsere nächsten Schritte aussehen werden und ob wir unsere Ziele erreichen können."} \\
\\ 
\\ 
\texttt{Wie lautet dein nächster Befehl?} \\
            }
        }
    }
    & \\ \\

    \theutterance \stepcounter{utterance}  
    & & & \multicolumn{4}{p{0.6\linewidth}}{
        \cellcolor[rgb]{0.9,0.9,0.9}{
            \makecell[{{p{\linewidth}}}]{
                \texttt{\tiny{[GM$\langle$P2]}}
                \texttt{VERSCHIEBE: P, (2,5)} \\
            }
        }
    }
     \\ \\

    \theutterance \stepcounter{utterance}  
    & & & \multicolumn{2}{p{0.3\linewidth}}{
        \cellcolor[rgb]{0.9,0.9,0.9}{
            \makecell[{{p{\linewidth}}}]{
                \texttt{\tiny{[GM$|$GM]}}
                \texttt{VERSCHIEBE: P, (2,5)} \\
            }
        }
    }
    & & \\ \\

    \theutterance \stepcounter{utterance}  
    & & & \multicolumn{2}{p{0.3\linewidth}}{
        \cellcolor[rgb]{0.9,0.9,0.9}{
            \makecell[{{p{\linewidth}}}]{
                \texttt{\tiny{[GM$|$GM]}}
                \texttt{Strafpunkt: (2,5) ist nicht leer, sondern enthält '┘'.} \\
\\ 
\texttt{Ihr habt gemeinsam 5 von maximal 8 Strafpunkten angesammelt.} \\
\texttt{Bitte versuche es noch mal!} \\
            }
        }
    }
    & & \\ \\

    \theutterance \stepcounter{utterance}  
    & & \multicolumn{4}{p{0.6\linewidth}}{
        \cellcolor[rgb]{0.9,0.9,0.9}{
            \makecell[{{p{\linewidth}}}]{
                \texttt{\tiny{[GM$\rangle$P2]}}
                \texttt{Strafpunkt: (2,5) ist nicht leer, sondern enthält '┘'.} \\
\\ 
\texttt{Ihr habt gemeinsam 5 von maximal 8 Strafpunkten angesammelt.} \\
\texttt{Bitte versuche es noch mal!} \\
            }
        }
    }
    & \\ \\

    \theutterance \stepcounter{utterance}  
    & & & \multicolumn{4}{p{0.6\linewidth}}{
        \cellcolor[rgb]{0.9,0.9,0.9}{
            \makecell[{{p{\linewidth}}}]{
                \texttt{\tiny{[GM$\langle$P2]}}
                \texttt{SPRICH: Es tut mir leid, dass mein vorheriger Versuch, 'P' zu (2,5) zu bewegen, fehlgeschlagen ist, weil (2,5) nicht leer ist. Ich denke, wir sollten unsere Strategie überdenken und einen neuen Plan entwickeln. Wie wäre es, wenn wir 'P' zu einer anderen Position bewegen, die auf beiden Brettern frei ist? Bitte teile mir deine Gedanken mit, und wir können unsere nächsten Schritte koordinieren. Wir sollten auch unsere verbleibenden Runden im Auge behalten und sicherstellen, dass wir unsere Ziele effizient erreichen. Sollen wir noch einmal versuchen, 'P' zu bewegen, oder haben wir eine bessere Option?} \\
            }
        }
    }
     \\ \\

    \theutterance \stepcounter{utterance}  
    & & & \multicolumn{2}{p{0.3\linewidth}}{
        \cellcolor[rgb]{0.9,0.9,0.9}{
            \makecell[{{p{\linewidth}}}]{
                \texttt{\tiny{[GM$|$GM]}}
                \texttt{SPRICH: Es tut mir leid, dass mein vorheriger Versuch, 'P' zu (2,5) zu bewegen, fehlgeschlagen ist, weil (2,5) nicht leer ist. Ich denke, wir sollten unsere Strategie überdenken und einen neuen Plan entwickeln. Wie wäre es, wenn wir 'P' zu einer anderen Position bewegen, die auf beiden Brettern frei ist? Bitte teile mir deine Gedanken mit, und wir können unsere nächsten Schritte koordinieren. Wir sollten auch unsere verbleibenden Runden im Auge behalten und sicherstellen, dass wir unsere Ziele effizient erreichen. Sollen wir noch einmal versuchen, 'P' zu bewegen, oder haben wir eine bessere Option?} \\
            }
        }
    }
    & & \\ \\

    \theutterance \stepcounter{utterance}  
    & & \multicolumn{4}{p{0.6\linewidth}}{
        \cellcolor[rgb]{0.9,0.9,0.9}{
            \makecell[{{p{\linewidth}}}]{
                \texttt{\tiny{[P1$\langle$GM]}}
                \texttt{Deine Nachricht wurde an deinen Mitspieler übermittelt.} \\
\\ 
\\ 
\texttt{Ihr seid gerade bei Runde 11 von maximal 12.} \\
\texttt{Ihr habt gemeinsam 5 von maximal 8 Strafpunkten angesammelt.} \\
\texttt{Dein Mitspieler hat dir diese Nachricht geschickt:} \\
\\ 
\\ 
\texttt{"Es tut mir leid, dass mein vorheriger Versuch, 'P' zu (2,5) zu bewegen, fehlgeschlagen ist, weil (2,5) nicht leer ist. Ich denke, wir sollten unsere Strategie überdenken und einen neuen Plan entwickeln. Wie wäre es, wenn wir 'P' zu einer anderen Position bewegen, die auf beiden Brettern frei ist? Bitte teile mir deine Gedanken mit, und wir können unsere nächsten Schritte koordinieren. Wir sollten auch unsere verbleibenden Runden im Auge behalten und sicherstellen, dass wir unsere Ziele effizient erreichen. Sollen wir noch einmal versuchen, 'P' zu bewegen, oder haben wir eine bessere Option?"} \\
\\ 
\\ 
\texttt{Wie lautet dein nächster Befehl?} \\
            }
        }
    }
    & \\ \\

    \theutterance \stepcounter{utterance}  
    & \multicolumn{4}{p{0.6\linewidth}}{
        \cellcolor[rgb]{0.9,0.9,0.9}{
            \makecell[{{p{\linewidth}}}]{
                \texttt{\tiny{[P1$\rangle$GM]}}
                \texttt{SPRICH: Ich denke, wir sollten uns auf eine neue Position für 'P' einigen, die auf beiden Brettern frei ist. Wie wäre es, wenn wir 'P' zu (3,3) bewegen? Ich werde mein 'P' von (1,7) nach (3,3) bewegen, wenn du dein 'P' auch dorthin bewegen kannst. Bitte überprüfe, ob (3,3) auf deinem Brett frei ist, und teile mir deine Gedanken mit. Wir sollten auch unsere verbleibenden Runden effizient nutzen, um unsere Ziele zu erreichen. Ich denke, wir können es noch schaffen, wenn wir unsere Strategie anpassen und gemeinsam handeln.} \\
            }
        }
    }
    & & \\ \\

    \theutterance \stepcounter{utterance}  
    & & & \multicolumn{2}{p{0.3\linewidth}}{
        \cellcolor[rgb]{0.9,0.9,0.9}{
            \makecell[{{p{\linewidth}}}]{
                \texttt{\tiny{[GM$|$GM]}}
                \texttt{SPRICH: Ich denke, wir sollten uns auf eine neue Position für 'P' einigen, die auf beiden Brettern frei ist. Wie wäre es, wenn wir 'P' zu (3,3) bewegen? Ich werde mein 'P' von (1,7) nach (3,3) bewegen, wenn du dein 'P' auch dorthin bewegen kannst. Bitte überprüfe, ob (3,3) auf deinem Brett frei ist, und teile mir deine Gedanken mit. Wir sollten auch unsere verbleibenden Runden effizient nutzen, um unsere Ziele zu erreichen. Ich denke, wir können es noch schaffen, wenn wir unsere Strategie anpassen und gemeinsam handeln.} \\
            }
        }
    }
    & & \\ \\

    \theutterance \stepcounter{utterance}  
    & & \multicolumn{4}{p{0.6\linewidth}}{
        \cellcolor[rgb]{0.9,0.9,0.9}{
            \makecell[{{p{\linewidth}}}]{
                \texttt{\tiny{[GM$\rangle$P2]}}
                \texttt{Deine Nachricht wurde an deinen Mitspieler übermittelt.} \\
\\ 
\\ 
\texttt{Ihr seid gerade bei Runde 12 von maximal 12.} \\
\texttt{Ihr habt gemeinsam 5 von maximal 8 Strafpunkten angesammelt.} \\
\texttt{Dein Mitspieler hat dir diese Nachricht geschickt:} \\
\\ 
\\ 
\texttt{"Ich denke, wir sollten uns auf eine neue Position für 'P' einigen, die auf beiden Brettern frei ist. Wie wäre es, wenn wir 'P' zu (3,3) bewegen? Ich werde mein 'P' von (1,7) nach (3,3) bewegen, wenn du dein 'P' auch dorthin bewegen kannst. Bitte überprüfe, ob (3,3) auf deinem Brett frei ist, und teile mir deine Gedanken mit. Wir sollten auch unsere verbleibenden Runden effizient nutzen, um unsere Ziele zu erreichen. Ich denke, wir können es noch schaffen, wenn wir unsere Strategie anpassen und gemeinsam handeln."} \\
\\ 
\\ 
\texttt{Wie lautet dein nächster Befehl?} \\
            }
        }
    }
    & \\ \\

    \theutterance \stepcounter{utterance}  
    & & & \multicolumn{4}{p{0.6\linewidth}}{
        \cellcolor[rgb]{0.9,0.9,0.9}{
            \makecell[{{p{\linewidth}}}]{
                \texttt{\tiny{[GM$\langle$P2]}}
                \texttt{SPRICH: Ich denke, (3,3) könnte eine gute Wahl für 'P' sein, aber ich muss überprüfen, ob diese Position auf meinem Brett frei ist. Auf meinem aktuellen Brett ist (3,3) tatsächlich leer, also könnte ich mein 'P' dorthin bewegen. Bevor ich jedoch eine Bewegung durchführe, möchte ich sicherstellen, dass wir beide dieselbe Vorstellung davon haben, wie unsere Bretter aussehen und wie die Objekte positioniert sind. Da dies unsere letzte Runde ist, sollten wir sicherstellen, dass wir unsere Bewegungen koordinieren, um das beste Ergebnis zu erzielen. Ich werde mein 'P' von (5,3) nach (3,3) bewegen, um unsere Strategie umzusetzen. Bitte bestätige, dass du bereit bist, dein 'P' von (1,7) nach (3,3) zu bewegen, sobald ich meine Aktion durchgeführt habe. Ich denke, wir haben eine gute Chance, unsere Ziele zu erreichen, wenn wir zusammenarbeiten. Sollen wir unsere endgültigen Bewegungen durchführen und hoffen, dass wir gewinnen?} \\
            }
        }
    }
     \\ \\

    \theutterance \stepcounter{utterance}  
    & & & \multicolumn{2}{p{0.3\linewidth}}{
        \cellcolor[rgb]{0.9,0.9,0.9}{
            \makecell[{{p{\linewidth}}}]{
                \texttt{\tiny{[GM$|$GM]}}
                \texttt{SPRICH: Ich denke, (3,3) könnte eine gute Wahl für 'P' sein, aber ich muss überprüfen, ob diese Position auf meinem Brett frei ist. Auf meinem aktuellen Brett ist (3,3) tatsächlich leer, also könnte ich mein 'P' dorthin bewegen. Bevor ich jedoch eine Bewegung durchführe, möchte ich sicherstellen, dass wir beide dieselbe Vorstellung davon haben, wie unsere Bretter aussehen und wie die Objekte positioniert sind. Da dies unsere letzte Runde ist, sollten wir sicherstellen, dass wir unsere Bewegungen koordinieren, um das beste Ergebnis zu erzielen. Ich werde mein 'P' von (5,3) nach (3,3) bewegen, um unsere Strategie umzusetzen. Bitte bestätige, dass du bereit bist, dein 'P' von (1,7) nach (3,3) zu bewegen, sobald ich meine Aktion durchgeführt habe. Ich denke, wir haben eine gute Chance, unsere Ziele zu erreichen, wenn wir zusammenarbeiten. Sollen wir unsere endgültigen Bewegungen durchführen und hoffen, dass wir gewinnen?} \\
            }
        }
    }
    & & \\ \\

    \theutterance \stepcounter{utterance}  
    & & & \multicolumn{2}{p{0.3\linewidth}}{
        \cellcolor[rgb]{0.9,0.9,0.9}{
            \makecell[{{p{\linewidth}}}]{
                \texttt{\tiny{[GM$|$GM]}}
                \texttt{Maximum number of rounds reached} \\
            }
        }
    }
    & & \\ \\

    \theutterance \stepcounter{utterance}  
    & & & \multicolumn{2}{p{0.3\linewidth}}{
        \cellcolor[rgb]{0.9,0.9,0.9}{
            \makecell[{{p{\linewidth}}}]{
                \texttt{\tiny{[GM$|$GM]}}
                \texttt{* success: True} \\
\texttt{* lose: True} \\
\texttt{* aborted: False} \\
\texttt{{-}{-}{-}{-}{-}{-}{-}} \\
\texttt{* Shifts: 0.00} \\
\texttt{* Max Shifts: 4.00} \\
\texttt{* Min Shifts: 2.00} \\
\texttt{* End Distance Sum: 14.23} \\
\texttt{* Init Distance Sum: 15.88} \\
\texttt{* Expected Distance Sum: 12.57} \\
\texttt{* Penalties: 5.00} \\
\texttt{* Max Penalties: 8.00} \\
\texttt{* Rounds: 12.00} \\
\texttt{* Max Rounds: 12.00} \\
\texttt{* Object Count: 3.00} \\
            }
        }
    }
    & & \\ \\

    \theutterance \stepcounter{utterance}  
    & & & \multicolumn{2}{p{0.3\linewidth}}{
        \cellcolor[rgb]{0.9,0.9,0.9}{
            \makecell[{{p{\linewidth}}}]{
                \texttt{\tiny{[GM$|$GM]}}
                \texttt{Initial Grids:} \\
\texttt{```} \\
\texttt{Player 1:     Player 2:} \\
\texttt{ 1234567       1234567} \\
\texttt{╔═══╤═╤═╗     ╔═══╤═╤═╗} \\
\texttt{║◌◌◌│◌│◌║ 1   ║◌◌◌│◌│◌║ 1} \\
\texttt{╟─┬─┼─┴─╢ 2   ╟─┬─┼─┴─╢ 2} \\
\texttt{║◌│◌│C◌◌║ 3   ║◌│L│◌P◌║ 3} \\
\texttt{║◌├─┤◌┌─╢ 4   ║◌├─┤◌┌─╢ 4} \\
\texttt{╟─┘◌│◌│L║ 5   ╟─┘◌│◌│◌║ 5} \\
\texttt{║◌◌◌└┬┴─╢ 6   ║◌◌◌└┬┴─╢ 6} \\
\texttt{║P◌◌◌│◌◌║ 7   ║◌C◌◌│◌◌║ 7} \\
\texttt{╚════╧══╝     ╚════╧══╝} \\
\texttt{```} \\
\\ 
\texttt{Final Grids:} \\
\texttt{```} \\
\texttt{Player 1:     Player 2:} \\
\texttt{ 1234567       1234567} \\
\texttt{╔═══╤═╤═╗     ╔═══╤═╤═╗} \\
\texttt{║◌◌◌│◌│◌║ 1   ║◌◌◌│◌│◌║ 1} \\
\texttt{╟─┬─┼─┴─╢ 2   ╟─┬─┼─┴─╢ 2} \\
\texttt{║◌│◌│C◌◌║ 3   ║◌│◌│LP◌║ 3} \\
\texttt{║◌├─┤◌┌─╢ 4   ║◌├─┤◌┌─╢ 4} \\
\texttt{╟─┘L│◌│◌║ 5   ╟─┘◌│◌│◌║ 5} \\
\texttt{║◌◌◌└┬┴─╢ 6   ║◌◌◌└┬┴─╢ 6} \\
\texttt{║P◌◌◌│◌◌║ 7   ║◌C◌◌│◌◌║ 7} \\
\texttt{╚════╧══╝     ╚════╧══╝} \\
\texttt{```} \\
            }
        }
    }
    & & \\ \\

\end{supertabular}
}

\end{document}
