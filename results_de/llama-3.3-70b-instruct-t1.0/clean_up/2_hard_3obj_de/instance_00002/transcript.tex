
\documentclass{article}
\usepackage{colortbl}
\usepackage{makecell}
\usepackage{multirow}
\usepackage{supertabular}

\begin{document}

\newcounter{utterance}

\centering \large Interaction Transcript for game `clean\_up', experiment `2\_hard\_3obj\_de', episode 2 with llama{-}3.3{-}70b{-}instruct{-}t1.0.
\vspace{24pt}

{ \footnotesize  \setcounter{utterance}{1}
\setlength{\tabcolsep}{0pt}
\begin{supertabular}{c@{$\;$}|p{.15\linewidth}@{}p{.15\linewidth}p{.15\linewidth}p{.15\linewidth}p{.15\linewidth}p{.15\linewidth}}
    \# & $\;$A & \multicolumn{4}{c}{Game Master} & $\;\:$B\\
    \hline

    \theutterance \stepcounter{utterance}  
    & & \multicolumn{4}{p{0.6\linewidth}}{
        \cellcolor[rgb]{0.9,0.9,0.9}{
            \makecell[{{p{\linewidth}}}]{
                \texttt{\tiny{[P1$\langle$GM]}}
                \texttt{Ich bin der Spielleiter, und du spielst ein kooperatives Spiel mit dem folgenden Raster als Spielbrett:} \\
\\ 
\texttt{```} \\
\texttt{ 1234567} \\
\texttt{╔═╤═╤═╤═╗} \\
\texttt{║◌│◌│◌│◌║ 1} \\
\texttt{╟─┼─┤◌│◌║ 2} \\
\texttt{║◌│◌│◌│◌║ 3} \\
\texttt{║L│◌├─┴─╢ 4} \\
\texttt{╟─┴─┤◌◌◌║ 5} \\
\texttt{║◌◌C└─┬─╢ 6} \\
\texttt{║◌◌◌◌P│◌║ 7} \\
\texttt{╚═════╧═╝} \\
\\ 
\texttt{```} \\
\\ 
\texttt{* Am oberen Rand sind die x{-}Koordinaten angegeben, die nach rechts hin zunehmen, und am rechten Rand die y{-}Koordinaten, die nach unten zunehmen.} \\
\texttt{* Die folgenden Objekte sind zufällig auf deinem Raster platziert: 'C', 'L', 'P'.} \\
\\ 
\\ 
\texttt{Dein Mitspieler sieht eine Variation des Spielbretts, auf dem dieselben Objekte an anderen zufälligen Positionen platziert sind. Du kannst das Spielbrett des anderen Spielers nicht sehen, und er kann deins nicht sehen.} \\
\\ 
\\ 
\texttt{**Ziel des Spiels:**} \\
\\ 
\texttt{Beide Spieler müssen die Objekte auf ihrem jeweiligen Spielbrett so bewegen, dass identische Objekte am Ende an denselben Koordinaten liegen. Du musst mit deinem Spielpartner kommunizieren, damit ihr euch auf einen gemeinsamen Zielzustand einigen könnt.} \\
\\ 
\\ 
\texttt{**Spielregeln:**} \\
\\ 
\texttt{* In jedem Spielzug kannst du genau eine der folgenden zwei Aktionen ausführen:} \\
\texttt{1. `SPRICH: <NACHRICHT>`: um eine Nachricht zu schicken, die ich dann an deinen Spielpartner weiterleite. Als Nachricht gilt alles bis zum nächsten Zeilenumbruch.} \\
\texttt{2. `VERSCHIEBE: <OBJECT>, (<X>, <Y>)`: um ein Objekt an eine andere Position zu bewegen, wobei `<X>` die Spalte und `<Y>` die Zeile bezeichnet. Ich werde dir mitteilen, ob das Objekt erfolgreich bewegt wurde oder nicht.} \\
\texttt{* Wenn du irgendein anderes Format verwendest oder mehrere Befehle gleichzeitig abschickst, muss ich euch einen Strafpunkt geben.} \\
\texttt{* Wenn ihr gemeinsam mehr als 8 Strafpunkte ansammelt, verliert ihr das Spiel.} \\
\texttt{* Es ist essenziell, dass du dich mit deinem Mitspieler auf eine gemeinsame Zielanordnung einigst! Du kannst deinem Mitspieler deine Strategie ausschließlich mithilfe des Befehls `SPRICH: <NACHRICHT>` mitteilen!} \\
\\ 
\\ 
\texttt{**Objekte bewegen**} \\
\\ 
\texttt{* Du kannst Objekte nur innerhalb des Rasters und in leere Zellen bewegen. Leere Zellen erkennst am Zeichen '◌'.} \\
\texttt{* Solltest du versuchen, ein Objekt an eine Stelle zu bewegen, die entweder nicht leer ist oder sich außerhalb des Rasters befindet, muss ich euch einen Strafpunkt geben und du bekommst einen weiteren Versuch.} \\
\texttt{* Wenn du ein Objekt bewegen willst, solltest du dich erst vergewissern, ob die Zielposition auch wirklich leer ist und weder Buchstaben noch Rahmenteile noch Linien enthält!} \\
\\ 
\\ 
\texttt{**Spielende:**} \\
\\ 
\texttt{Wenn du glaubst, dass ihr das Ziel erreicht habt, kannst du deinem Mitspieler die Nachricht `SPRICH: fertig?` schicken. Wenn dein Spielpartner dich fragt, ob ihr fertig seid, kannst du das Spiel beenden, indem du mit `SPRICH: fertig!` antwortest.} \\
\\ 
\texttt{Ihr gewinnt gemeinsam, wenn ihr das Spiel innerhalb von \#max\_rounds Runden beendet. Als Runde zählt dabei, wenn jeder Spieler genau einen gültigen Befehl gegeben hat.} \\
\\ 
\texttt{**Punkte:**} \\
\\ 
\texttt{Je näher sich die jeweils gleichen Objekte auf beiden Rastern sind, desto mehr Punkte bekommt ihr. Für Strafpunkte gibt es Punktabzug. Könnt ihr den Rekord brechen?} \\
\\ 
\texttt{Schicke eine Nachricht an deinen Mitspieler um das Spiel zu starten!} \\
            }
        }
    }
    & \\ \\

    \theutterance \stepcounter{utterance}  
    & \multicolumn{4}{p{0.6\linewidth}}{
        \cellcolor[rgb]{0.9,0.9,0.9}{
            \makecell[{{p{\linewidth}}}]{
                \texttt{\tiny{[P1$\rangle$GM]}}
                \texttt{Ich schicke die folgende Nachricht an meinen Mitspieler:} \\
\\ 
\texttt{`SPRICH: Hallo, lass uns das Spiel starten! Wir sollten uns auf eine gemeinsame Zielanordnung einigen. Ich schlage vor, dass wir die Objekte 'C', 'L' und 'P' in der Mitte des Rasters platzieren. Wir können uns auf die Koordinaten (4, 4) für 'C', (3, 4) für 'L' und (5, 4) für 'P' einigen. Was meinst du?`} \\
            }
        }
    }
    & & \\ \\

    \theutterance \stepcounter{utterance}  
    & & & \multicolumn{2}{p{0.3\linewidth}}{
        \cellcolor[rgb]{0.9,0.9,0.9}{
            \makecell[{{p{\linewidth}}}]{
                \texttt{\tiny{[GM$|$GM]}}
                \texttt{Ich schicke die folgende Nachricht an meinen Mitspieler:} \\
\\ 
\texttt{`SPRICH: Hallo, lass uns das Spiel starten! Wir sollten uns auf eine gemeinsame Zielanordnung einigen. Ich schlage vor, dass wir die Objekte 'C', 'L' und 'P' in der Mitte des Rasters platzieren. Wir können uns auf die Koordinaten (4, 4) für 'C', (3, 4) für 'L' und (5, 4) für 'P' einigen. Was meinst du?`} \\
            }
        }
    }
    & & \\ \\

    \theutterance \stepcounter{utterance}  
    & & & \multicolumn{2}{p{0.3\linewidth}}{
        \cellcolor[rgb]{0.9,0.9,0.9}{
            \makecell[{{p{\linewidth}}}]{
                \texttt{\tiny{[GM$|$GM]}}
                \texttt{Invalid format: head is not empty: 'Ich schicke die folgende Nachricht an meinen Mitspieler:} \\
\\ 
\texttt{'} \\
            }
        }
    }
    & & \\ \\

    \theutterance \stepcounter{utterance}  
    & & \multicolumn{4}{p{0.6\linewidth}}{
        \cellcolor[rgb]{0.9,0.9,0.9}{
            \makecell[{{p{\linewidth}}}]{
                \texttt{\tiny{[P1$\langle$GM]}}
                \texttt{Strafpunkt: Deine Nachricht darf nichts vor dem Befehl enthalten! Stelle sicher, dass deine Antwort ausschließlich einen der beiden Befehle `SPRICH: <NACHRICHT>` und `VERSCHIEBE: <OBJECT>, (<X>, <Y>)` und sonst nichts enthält!} \\
\\ 
\texttt{Ihr habt gemeinsam 1 von maximal 8 Strafpunkten angesammelt.} \\
\texttt{Bitte versuche es noch mal!} \\
            }
        }
    }
    & \\ \\

    \theutterance \stepcounter{utterance}  
    & \multicolumn{4}{p{0.6\linewidth}}{
        \cellcolor[rgb]{0.9,0.9,0.9}{
            \makecell[{{p{\linewidth}}}]{
                \texttt{\tiny{[P1$\rangle$GM]}}
                \texttt{SPRICH: Hallo, lass uns das Spiel starten! Wir sollten uns auf eine gemeinsame Zielanordnung einigen. Ich schlage vor, dass wir die Objekte 'C', 'L' und 'P' in der Mitte des Rasters platzieren. Wir können uns auf die Koordinaten (4, 4) für 'C', (3, 4) für 'L' und (5, 4) für 'P' einigen. Was meinst du?} \\
            }
        }
    }
    & & \\ \\

    \theutterance \stepcounter{utterance}  
    & & & \multicolumn{2}{p{0.3\linewidth}}{
        \cellcolor[rgb]{0.9,0.9,0.9}{
            \makecell[{{p{\linewidth}}}]{
                \texttt{\tiny{[GM$|$GM]}}
                \texttt{SPRICH: Hallo, lass uns das Spiel starten! Wir sollten uns auf eine gemeinsame Zielanordnung einigen. Ich schlage vor, dass wir die Objekte 'C', 'L' und 'P' in der Mitte des Rasters platzieren. Wir können uns auf die Koordinaten (4, 4) für 'C', (3, 4) für 'L' und (5, 4) für 'P' einigen. Was meinst du?} \\
            }
        }
    }
    & & \\ \\

    \theutterance \stepcounter{utterance}  
    & & \multicolumn{4}{p{0.6\linewidth}}{
        \cellcolor[rgb]{0.9,0.9,0.9}{
            \makecell[{{p{\linewidth}}}]{
                \texttt{\tiny{[GM$\rangle$P2]}}
                \texttt{Ich bin der Spielleiter, und du spielst ein kooperatives Spiel mit dem folgenden Raster als Spielbrett:} \\
\\ 
\texttt{```} \\
\texttt{ 1234567} \\
\texttt{╔═╤═╤═╤═╗} \\
\texttt{║L│◌│◌│◌║ 1} \\
\texttt{╟─┼─┤◌│◌║ 2} \\
\texttt{║◌│◌│◌│◌║ 3} \\
\texttt{║◌│◌├─┴─╢ 4} \\
\texttt{╟─┴─┤C◌◌║ 5} \\
\texttt{║◌◌◌└─┬─╢ 6} \\
\texttt{║◌P◌◌◌│◌║ 7} \\
\texttt{╚═════╧═╝} \\
\\ 
\texttt{```} \\
\\ 
\texttt{* Am oberen Rand sind die x{-}Koordinaten angegeben, die nach rechts hin zunehmen, und am rechten Rand die y{-}Koordinaten, die nach unten zunehmen.} \\
\texttt{* Die folgenden Objekte sind zufällig auf deinem Raster platziert: 'C', 'L', 'P'.} \\
\\ 
\\ 
\texttt{Dein Mitspieler sieht eine Variation des Spielbretts, auf dem dieselben Objekte an anderen zufälligen Positionen platziert sind. Du kannst das Spielbrett des anderen Spielers nicht sehen, und er kann deins nicht sehen.} \\
\\ 
\\ 
\texttt{**Ziel des Spiels:**} \\
\\ 
\texttt{Beide Spieler müssen die Objekte auf ihrem jeweiligen Spielbrett so bewegen, dass identische Objekte am Ende an denselben Koordinaten liegen. Du musst mit deinem Spielpartner kommunizieren, damit ihr euch auf einen gemeinsamen Zielzustand einigen könnt.} \\
\\ 
\\ 
\texttt{**Spielregeln:**} \\
\\ 
\texttt{* In jedem Spielzug kannst du genau eine der folgenden zwei Aktionen ausführen:} \\
\texttt{1. `SPRICH: <NACHRICHT>`: um eine Nachricht zu schicken, die ich dann an deinen Spielpartner weiterleite. Als Nachricht gilt alles bis zum nächsten Zeilenumbruch.} \\
\texttt{2. `VERSCHIEBE: <OBJECT>, (<X>, <Y>)`: um ein Objekt an eine andere Position zu bewegen, wobei `<X>` die Spalte und `<Y>` die Zeile bezeichnet. Ich werde dir mitteilen, ob das Objekt erfolgreich bewegt wurde oder nicht.} \\
\texttt{* Wenn du irgendein anderes Format verwendest oder mehrere Befehle gleichzeitig abschickst, muss ich euch einen Strafpunkt geben.} \\
\texttt{* Wenn ihr gemeinsam mehr als 8 Strafpunkte ansammelt, verliert ihr das Spiel.} \\
\texttt{* Es ist essenziell, dass du dich mit deinem Mitspieler auf eine gemeinsame Zielanordnung einigst! Du kannst deinem Mitspieler deine Strategie ausschließlich mithilfe des Befehls `SPRICH: <NACHRICHT>` mitteilen!} \\
\\ 
\\ 
\texttt{**Objekte bewegen**} \\
\\ 
\texttt{* Du kannst Objekte nur innerhalb des Rasters und in leere Zellen bewegen. Leere Zellen erkennst am Zeichen '◌'.} \\
\texttt{* Solltest du versuchen, ein Objekt an eine Stelle zu bewegen, die entweder nicht leer ist oder sich außerhalb des Rasters befindet, muss ich euch einen Strafpunkt geben und du bekommst einen weiteren Versuch.} \\
\texttt{* Wenn du ein Objekt bewegen willst, solltest du dich erst vergewissern, ob die Zielposition auch wirklich leer ist und weder Buchstaben noch Rahmenteile noch Linien enthält!} \\
\\ 
\\ 
\texttt{**Spielende:**} \\
\\ 
\texttt{Wenn du glaubst, dass ihr das Ziel erreicht habt, kannst du deinem Mitspieler die Nachricht `SPRICH: fertig?` schicken. Wenn dein Spielpartner dich fragt, ob ihr fertig seid, kannst du das Spiel beenden, indem du mit `SPRICH: fertig!` antwortest.} \\
\\ 
\texttt{Ihr gewinnt gemeinsam, wenn ihr das Spiel innerhalb von \#max\_rounds Runden beendet. Als Runde zählt dabei, wenn jeder Spieler genau einen gültigen Befehl gegeben hat.} \\
\\ 
\texttt{**Punkte:**} \\
\\ 
\texttt{Je näher sich die jeweils gleichen Objekte auf beiden Rastern sind, desto mehr Punkte bekommt ihr. Für Strafpunkte gibt es Punktabzug. Könnt ihr den Rekord brechen?} \\
\\ 
\texttt{Dein Mitspieler hat das Spiel mit dieser Nachricht begonnen:} \\
\\ 
\texttt{"Hallo, lass uns das Spiel starten! Wir sollten uns auf eine gemeinsame Zielanordnung einigen. Ich schlage vor, dass wir die Objekte 'C', 'L' und 'P' in der Mitte des Rasters platzieren. Wir können uns auf die Koordinaten (4, 4) für 'C', (3, 4) für 'L' und (5, 4) für 'P' einigen. Was meinst du?"} \\
\\ 
\texttt{Wie lautet dein erster Befehl?} \\
            }
        }
    }
    & \\ \\

    \theutterance \stepcounter{utterance}  
    & & & \multicolumn{4}{p{0.6\linewidth}}{
        \cellcolor[rgb]{0.9,0.9,0.9}{
            \makecell[{{p{\linewidth}}}]{
                \texttt{\tiny{[GM$\langle$P2]}}
                \texttt{SPRICH: Hallo! Dein Vorschlag klingt gut, aber bevor wir uns auf eine Zielanordnung einigen, möchte ich mich vergewissern, dass wir beide dieselben Objekte auf unserem Raster haben. Können wir uns zunächst austauschen, welche Objekte wir haben und an welchen Positionen sie sich befinden? Ich habe ein 'C' an der Position (2, 5), ein 'L' an der Position (1, 1) und ein 'P' an der Position (3, 7). Wie sieht es bei dir aus?} \\
            }
        }
    }
     \\ \\

    \theutterance \stepcounter{utterance}  
    & & & \multicolumn{2}{p{0.3\linewidth}}{
        \cellcolor[rgb]{0.9,0.9,0.9}{
            \makecell[{{p{\linewidth}}}]{
                \texttt{\tiny{[GM$|$GM]}}
                \texttt{SPRICH: Hallo! Dein Vorschlag klingt gut, aber bevor wir uns auf eine Zielanordnung einigen, möchte ich mich vergewissern, dass wir beide dieselben Objekte auf unserem Raster haben. Können wir uns zunächst austauschen, welche Objekte wir haben und an welchen Positionen sie sich befinden? Ich habe ein 'C' an der Position (2, 5), ein 'L' an der Position (1, 1) und ein 'P' an der Position (3, 7). Wie sieht es bei dir aus?} \\
            }
        }
    }
    & & \\ \\

    \theutterance \stepcounter{utterance}  
    & & \multicolumn{4}{p{0.6\linewidth}}{
        \cellcolor[rgb]{0.9,0.9,0.9}{
            \makecell[{{p{\linewidth}}}]{
                \texttt{\tiny{[P1$\langle$GM]}}
                \texttt{Deine Nachricht wurde an deinen Mitspieler übermittelt.} \\
\\ 
\\ 
\texttt{Ihr seid gerade bei Runde 1 von maximal 12.} \\
\texttt{Ihr habt gemeinsam 1 von maximal 8 Strafpunkten angesammelt.} \\
\texttt{Dein Mitspieler hat dir diese Nachricht geschickt:} \\
\\ 
\\ 
\texttt{"Hallo! Dein Vorschlag klingt gut, aber bevor wir uns auf eine Zielanordnung einigen, möchte ich mich vergewissern, dass wir beide dieselben Objekte auf unserem Raster haben. Können wir uns zunächst austauschen, welche Objekte wir haben und an welchen Positionen sie sich befinden? Ich habe ein 'C' an der Position (2, 5), ein 'L' an der Position (1, 1) und ein 'P' an der Position (3, 7). Wie sieht es bei dir aus?"} \\
\\ 
\\ 
\texttt{Wie lautet dein nächster Befehl?} \\
            }
        }
    }
    & \\ \\

    \theutterance \stepcounter{utterance}  
    & \multicolumn{4}{p{0.6\linewidth}}{
        \cellcolor[rgb]{0.9,0.9,0.9}{
            \makecell[{{p{\linewidth}}}]{
                \texttt{\tiny{[P1$\rangle$GM]}}
                \texttt{SPRICH: Ich habe ein 'C' an der Position (6, 6), ein 'L' an der Position (1, 4) und ein 'P' an der Position (7, 7). Jetzt, da wir unsere Anfangspositionen geteilt haben, können wir unsere Strategie besprechen. Ich denke, wir sollten wie ursprünglich besprochen, die Objekte 'C', 'L' und 'P' in der Mitte des Rasters platzieren, auf den Koordinaten (4, 4) für 'C', (3, 4) für 'L' und (5, 4) für 'P'. Was meinst du dazu? Sollen wir mit dem Bewegen der Objekte beginnen?} \\
            }
        }
    }
    & & \\ \\

    \theutterance \stepcounter{utterance}  
    & & & \multicolumn{2}{p{0.3\linewidth}}{
        \cellcolor[rgb]{0.9,0.9,0.9}{
            \makecell[{{p{\linewidth}}}]{
                \texttt{\tiny{[GM$|$GM]}}
                \texttt{SPRICH: Ich habe ein 'C' an der Position (6, 6), ein 'L' an der Position (1, 4) und ein 'P' an der Position (7, 7). Jetzt, da wir unsere Anfangspositionen geteilt haben, können wir unsere Strategie besprechen. Ich denke, wir sollten wie ursprünglich besprochen, die Objekte 'C', 'L' und 'P' in der Mitte des Rasters platzieren, auf den Koordinaten (4, 4) für 'C', (3, 4) für 'L' und (5, 4) für 'P'. Was meinst du dazu? Sollen wir mit dem Bewegen der Objekte beginnen?} \\
            }
        }
    }
    & & \\ \\

    \theutterance \stepcounter{utterance}  
    & & \multicolumn{4}{p{0.6\linewidth}}{
        \cellcolor[rgb]{0.9,0.9,0.9}{
            \makecell[{{p{\linewidth}}}]{
                \texttt{\tiny{[GM$\rangle$P2]}}
                \texttt{Deine Nachricht wurde an deinen Mitspieler übermittelt.} \\
\\ 
\\ 
\texttt{Ihr seid gerade bei Runde 2 von maximal 12.} \\
\texttt{Ihr habt gemeinsam 1 von maximal 8 Strafpunkten angesammelt.} \\
\texttt{Dein Mitspieler hat dir diese Nachricht geschickt:} \\
\\ 
\\ 
\texttt{"Ich habe ein 'C' an der Position (6, 6), ein 'L' an der Position (1, 4) und ein 'P' an der Position (7, 7). Jetzt, da wir unsere Anfangspositionen geteilt haben, können wir unsere Strategie besprechen. Ich denke, wir sollten wie ursprünglich besprochen, die Objekte 'C', 'L' und 'P' in der Mitte des Rasters platzieren, auf den Koordinaten (4, 4) für 'C', (3, 4) für 'L' und (5, 4) für 'P'. Was meinst du dazu? Sollen wir mit dem Bewegen der Objekte beginnen?"} \\
\\ 
\\ 
\texttt{Wie lautet dein nächster Befehl?} \\
            }
        }
    }
    & \\ \\

    \theutterance \stepcounter{utterance}  
    & & & \multicolumn{4}{p{0.6\linewidth}}{
        \cellcolor[rgb]{0.9,0.9,0.9}{
            \makecell[{{p{\linewidth}}}]{
                \texttt{\tiny{[GM$\langle$P2]}}
                \texttt{SPRICH: Ich denke, das ist ein guter Plan! Bevor wir beginnen, die Objekte zu bewegen, sollten wir uns auf die Reihenfolge einigen, in der wir sie bewegen. Ich denke, es wäre am besten, zuerst das 'L' zu bewegen, da es am kürzesten zum Ziel ist. Dann sollten wir das 'C' und schließlich das 'P' bewegen. Wie soll ich vorgehen, um das 'L' auf die Position (3, 4) zu bringen? Soll ich es direkt dorthin bewegen oder in mehreren Schritten?} \\
            }
        }
    }
     \\ \\

    \theutterance \stepcounter{utterance}  
    & & & \multicolumn{2}{p{0.3\linewidth}}{
        \cellcolor[rgb]{0.9,0.9,0.9}{
            \makecell[{{p{\linewidth}}}]{
                \texttt{\tiny{[GM$|$GM]}}
                \texttt{SPRICH: Ich denke, das ist ein guter Plan! Bevor wir beginnen, die Objekte zu bewegen, sollten wir uns auf die Reihenfolge einigen, in der wir sie bewegen. Ich denke, es wäre am besten, zuerst das 'L' zu bewegen, da es am kürzesten zum Ziel ist. Dann sollten wir das 'C' und schließlich das 'P' bewegen. Wie soll ich vorgehen, um das 'L' auf die Position (3, 4) zu bringen? Soll ich es direkt dorthin bewegen oder in mehreren Schritten?} \\
            }
        }
    }
    & & \\ \\

    \theutterance \stepcounter{utterance}  
    & & \multicolumn{4}{p{0.6\linewidth}}{
        \cellcolor[rgb]{0.9,0.9,0.9}{
            \makecell[{{p{\linewidth}}}]{
                \texttt{\tiny{[P1$\langle$GM]}}
                \texttt{Deine Nachricht wurde an deinen Mitspieler übermittelt.} \\
\\ 
\\ 
\texttt{Ihr seid gerade bei Runde 2 von maximal 12.} \\
\texttt{Ihr habt gemeinsam 1 von maximal 8 Strafpunkten angesammelt.} \\
\texttt{Dein Mitspieler hat dir diese Nachricht geschickt:} \\
\\ 
\\ 
\texttt{"Ich denke, das ist ein guter Plan! Bevor wir beginnen, die Objekte zu bewegen, sollten wir uns auf die Reihenfolge einigen, in der wir sie bewegen. Ich denke, es wäre am besten, zuerst das 'L' zu bewegen, da es am kürzesten zum Ziel ist. Dann sollten wir das 'C' und schließlich das 'P' bewegen. Wie soll ich vorgehen, um das 'L' auf die Position (3, 4) zu bringen? Soll ich es direkt dorthin bewegen oder in mehreren Schritten?"} \\
\\ 
\\ 
\texttt{Wie lautet dein nächster Befehl?} \\
            }
        }
    }
    & \\ \\

    \theutterance \stepcounter{utterance}  
    & \multicolumn{4}{p{0.6\linewidth}}{
        \cellcolor[rgb]{0.9,0.9,0.9}{
            \makecell[{{p{\linewidth}}}]{
                \texttt{\tiny{[P1$\rangle$GM]}}
                \texttt{SPRICH: Ich denke, es wäre am besten, das 'L' in mehreren Schritten zu bewegen, um sicherzustellen, dass wir keine Strafpunkte erhalten, wenn wir es direkt an eine besetzte Position bewegen. Wir sollten zunächst das 'L' nach rechts oder unten bewegen, um es näher an die Zielposition zu bringen, und dann in einem weiteren Schritt direkt auf die Position (3, 4) bewegen. Ich schlage vor, dass du das 'L' zunächst nach rechts auf die Position (2, 1) bewegst, um es aus der Ecke zu bringen. Dann können wir weiter planen, wie wir es auf die Zielposition bringen. Wie sieht es bei dir aus, wenn du das 'L' auf (2, 1) bewegen würdest?} \\
            }
        }
    }
    & & \\ \\

    \theutterance \stepcounter{utterance}  
    & & & \multicolumn{2}{p{0.3\linewidth}}{
        \cellcolor[rgb]{0.9,0.9,0.9}{
            \makecell[{{p{\linewidth}}}]{
                \texttt{\tiny{[GM$|$GM]}}
                \texttt{SPRICH: Ich denke, es wäre am besten, das 'L' in mehreren Schritten zu bewegen, um sicherzustellen, dass wir keine Strafpunkte erhalten, wenn wir es direkt an eine besetzte Position bewegen. Wir sollten zunächst das 'L' nach rechts oder unten bewegen, um es näher an die Zielposition zu bringen, und dann in einem weiteren Schritt direkt auf die Position (3, 4) bewegen. Ich schlage vor, dass du das 'L' zunächst nach rechts auf die Position (2, 1) bewegst, um es aus der Ecke zu bringen. Dann können wir weiter planen, wie wir es auf die Zielposition bringen. Wie sieht es bei dir aus, wenn du das 'L' auf (2, 1) bewegen würdest?} \\
            }
        }
    }
    & & \\ \\

    \theutterance \stepcounter{utterance}  
    & & \multicolumn{4}{p{0.6\linewidth}}{
        \cellcolor[rgb]{0.9,0.9,0.9}{
            \makecell[{{p{\linewidth}}}]{
                \texttt{\tiny{[GM$\rangle$P2]}}
                \texttt{Deine Nachricht wurde an deinen Mitspieler übermittelt.} \\
\\ 
\\ 
\texttt{Ihr seid gerade bei Runde 3 von maximal 12.} \\
\texttt{Ihr habt gemeinsam 1 von maximal 8 Strafpunkten angesammelt.} \\
\texttt{Dein Mitspieler hat dir diese Nachricht geschickt:} \\
\\ 
\\ 
\texttt{"Ich denke, es wäre am besten, das 'L' in mehreren Schritten zu bewegen, um sicherzustellen, dass wir keine Strafpunkte erhalten, wenn wir es direkt an eine besetzte Position bewegen. Wir sollten zunächst das 'L' nach rechts oder unten bewegen, um es näher an die Zielposition zu bringen, und dann in einem weiteren Schritt direkt auf die Position (3, 4) bewegen. Ich schlage vor, dass du das 'L' zunächst nach rechts auf die Position (2, 1) bewegst, um es aus der Ecke zu bringen. Dann können wir weiter planen, wie wir es auf die Zielposition bringen. Wie sieht es bei dir aus, wenn du das 'L' auf (2, 1) bewegen würdest?"} \\
\\ 
\\ 
\texttt{Wie lautet dein nächster Befehl?} \\
            }
        }
    }
    & \\ \\

    \theutterance \stepcounter{utterance}  
    & & & \multicolumn{4}{p{0.6\linewidth}}{
        \cellcolor[rgb]{0.9,0.9,0.9}{
            \makecell[{{p{\linewidth}}}]{
                \texttt{\tiny{[GM$\langle$P2]}}
                \texttt{VERSCHIEBE: L, (2, 1)} \\
            }
        }
    }
     \\ \\

    \theutterance \stepcounter{utterance}  
    & & & \multicolumn{2}{p{0.3\linewidth}}{
        \cellcolor[rgb]{0.9,0.9,0.9}{
            \makecell[{{p{\linewidth}}}]{
                \texttt{\tiny{[GM$|$GM]}}
                \texttt{VERSCHIEBE: L, (2, 1)} \\
            }
        }
    }
    & & \\ \\

    \theutterance \stepcounter{utterance}  
    & & & \multicolumn{2}{p{0.3\linewidth}}{
        \cellcolor[rgb]{0.9,0.9,0.9}{
            \makecell[{{p{\linewidth}}}]{
                \texttt{\tiny{[GM$|$GM]}}
                \texttt{Strafpunkt: (2,1) ist nicht leer, sondern enthält '│'.} \\
\\ 
\texttt{Ihr habt gemeinsam 2 von maximal 8 Strafpunkten angesammelt.} \\
\texttt{Bitte versuche es noch mal!} \\
            }
        }
    }
    & & \\ \\

    \theutterance \stepcounter{utterance}  
    & & \multicolumn{4}{p{0.6\linewidth}}{
        \cellcolor[rgb]{0.9,0.9,0.9}{
            \makecell[{{p{\linewidth}}}]{
                \texttt{\tiny{[GM$\rangle$P2]}}
                \texttt{Strafpunkt: (2,1) ist nicht leer, sondern enthält '│'.} \\
\\ 
\texttt{Ihr habt gemeinsam 2 von maximal 8 Strafpunkten angesammelt.} \\
\texttt{Bitte versuche es noch mal!} \\
            }
        }
    }
    & \\ \\

    \theutterance \stepcounter{utterance}  
    & & & \multicolumn{4}{p{0.6\linewidth}}{
        \cellcolor[rgb]{0.9,0.9,0.9}{
            \makecell[{{p{\linewidth}}}]{
                \texttt{\tiny{[GM$\langle$P2]}}
                \texttt{SPRICH: Ich denke, es wäre besser, wenn ich das 'L' nach unten bewege, anstatt nach rechts. Die Position (1, 2) ist leer, oder? Wenn ja, kann ich das 'L' dorthin bewegen und dann weiter planen, wie wir es auf die Zielposition bringen.} \\
            }
        }
    }
     \\ \\

    \theutterance \stepcounter{utterance}  
    & & & \multicolumn{2}{p{0.3\linewidth}}{
        \cellcolor[rgb]{0.9,0.9,0.9}{
            \makecell[{{p{\linewidth}}}]{
                \texttt{\tiny{[GM$|$GM]}}
                \texttt{SPRICH: Ich denke, es wäre besser, wenn ich das 'L' nach unten bewege, anstatt nach rechts. Die Position (1, 2) ist leer, oder? Wenn ja, kann ich das 'L' dorthin bewegen und dann weiter planen, wie wir es auf die Zielposition bringen.} \\
            }
        }
    }
    & & \\ \\

    \theutterance \stepcounter{utterance}  
    & & \multicolumn{4}{p{0.6\linewidth}}{
        \cellcolor[rgb]{0.9,0.9,0.9}{
            \makecell[{{p{\linewidth}}}]{
                \texttt{\tiny{[P1$\langle$GM]}}
                \texttt{Deine Nachricht wurde an deinen Mitspieler übermittelt.} \\
\\ 
\\ 
\texttt{Ihr seid gerade bei Runde 3 von maximal 12.} \\
\texttt{Ihr habt gemeinsam 2 von maximal 8 Strafpunkten angesammelt.} \\
\texttt{Dein Mitspieler hat dir diese Nachricht geschickt:} \\
\\ 
\\ 
\texttt{"Ich denke, es wäre besser, wenn ich das 'L' nach unten bewege, anstatt nach rechts. Die Position (1, 2) ist leer, oder? Wenn ja, kann ich das 'L' dorthin bewegen und dann weiter planen, wie wir es auf die Zielposition bringen."} \\
\\ 
\\ 
\texttt{Wie lautet dein nächster Befehl?} \\
            }
        }
    }
    & \\ \\

    \theutterance \stepcounter{utterance}  
    & \multicolumn{4}{p{0.6\linewidth}}{
        \cellcolor[rgb]{0.9,0.9,0.9}{
            \makecell[{{p{\linewidth}}}]{
                \texttt{\tiny{[P1$\rangle$GM]}}
                \texttt{SPRICH: Ja, die Position (1, 2) ist leer. Du kannst das 'L' dorthin bewegen. Nachdem du das 'L' auf (1, 2) bewegt hast, sollten wir es weiter nach unten auf die Position (1, 3) und dann nach rechts auf die Position (3, 3) bewegen, bevor wir es auf die Zielposition (3, 4) bringen. Ich denke, das wäre ein guter Plan. Wie sieht es bei dir aus, wenn du so vorgehen würdest?} \\
            }
        }
    }
    & & \\ \\

    \theutterance \stepcounter{utterance}  
    & & & \multicolumn{2}{p{0.3\linewidth}}{
        \cellcolor[rgb]{0.9,0.9,0.9}{
            \makecell[{{p{\linewidth}}}]{
                \texttt{\tiny{[GM$|$GM]}}
                \texttt{SPRICH: Ja, die Position (1, 2) ist leer. Du kannst das 'L' dorthin bewegen. Nachdem du das 'L' auf (1, 2) bewegt hast, sollten wir es weiter nach unten auf die Position (1, 3) und dann nach rechts auf die Position (3, 3) bewegen, bevor wir es auf die Zielposition (3, 4) bringen. Ich denke, das wäre ein guter Plan. Wie sieht es bei dir aus, wenn du so vorgehen würdest?} \\
            }
        }
    }
    & & \\ \\

    \theutterance \stepcounter{utterance}  
    & & \multicolumn{4}{p{0.6\linewidth}}{
        \cellcolor[rgb]{0.9,0.9,0.9}{
            \makecell[{{p{\linewidth}}}]{
                \texttt{\tiny{[GM$\rangle$P2]}}
                \texttt{Deine Nachricht wurde an deinen Mitspieler übermittelt.} \\
\\ 
\\ 
\texttt{Ihr seid gerade bei Runde 4 von maximal 12.} \\
\texttt{Ihr habt gemeinsam 2 von maximal 8 Strafpunkten angesammelt.} \\
\texttt{Dein Mitspieler hat dir diese Nachricht geschickt:} \\
\\ 
\\ 
\texttt{"Ja, die Position (1, 2) ist leer. Du kannst das 'L' dorthin bewegen. Nachdem du das 'L' auf (1, 2) bewegt hast, sollten wir es weiter nach unten auf die Position (1, 3) und dann nach rechts auf die Position (3, 3) bewegen, bevor wir es auf die Zielposition (3, 4) bringen. Ich denke, das wäre ein guter Plan. Wie sieht es bei dir aus, wenn du so vorgehen würdest?"} \\
\\ 
\\ 
\texttt{Wie lautet dein nächster Befehl?} \\
            }
        }
    }
    & \\ \\

    \theutterance \stepcounter{utterance}  
    & & & \multicolumn{4}{p{0.6\linewidth}}{
        \cellcolor[rgb]{0.9,0.9,0.9}{
            \makecell[{{p{\linewidth}}}]{
                \texttt{\tiny{[GM$\langle$P2]}}
                \texttt{VERSCHIEBE: L, (1, 2)} \\
            }
        }
    }
     \\ \\

    \theutterance \stepcounter{utterance}  
    & & & \multicolumn{2}{p{0.3\linewidth}}{
        \cellcolor[rgb]{0.9,0.9,0.9}{
            \makecell[{{p{\linewidth}}}]{
                \texttt{\tiny{[GM$|$GM]}}
                \texttt{VERSCHIEBE: L, (1, 2)} \\
            }
        }
    }
    & & \\ \\

    \theutterance \stepcounter{utterance}  
    & & & \multicolumn{2}{p{0.3\linewidth}}{
        \cellcolor[rgb]{0.9,0.9,0.9}{
            \makecell[{{p{\linewidth}}}]{
                \texttt{\tiny{[GM$|$GM]}}
                \texttt{Strafpunkt: (1,2) ist nicht leer, sondern enthält '─'.} \\
\\ 
\texttt{Ihr habt gemeinsam 3 von maximal 8 Strafpunkten angesammelt.} \\
\texttt{Bitte versuche es noch mal!} \\
            }
        }
    }
    & & \\ \\

    \theutterance \stepcounter{utterance}  
    & & \multicolumn{4}{p{0.6\linewidth}}{
        \cellcolor[rgb]{0.9,0.9,0.9}{
            \makecell[{{p{\linewidth}}}]{
                \texttt{\tiny{[GM$\rangle$P2]}}
                \texttt{Strafpunkt: (1,2) ist nicht leer, sondern enthält '─'.} \\
\\ 
\texttt{Ihr habt gemeinsam 3 von maximal 8 Strafpunkten angesammelt.} \\
\texttt{Bitte versuche es noch mal!} \\
            }
        }
    }
    & \\ \\

    \theutterance \stepcounter{utterance}  
    & & & \multicolumn{4}{p{0.6\linewidth}}{
        \cellcolor[rgb]{0.9,0.9,0.9}{
            \makecell[{{p{\linewidth}}}]{
                \texttt{\tiny{[GM$\langle$P2]}}
                \texttt{SPRICH: Ich denke, ich habe wieder einen Fehler gemacht. Lass uns noch mal überdenken, wie wir das 'L' bewegen können. Da die Position (1, 2) nicht leer ist, müssen wir einen anderen Weg finden. Kannst du mir bitte sagen, welche Positionen um das 'L' herum leer sind? Dann können wir einen neuen Plan machen.} \\
            }
        }
    }
     \\ \\

    \theutterance \stepcounter{utterance}  
    & & & \multicolumn{2}{p{0.3\linewidth}}{
        \cellcolor[rgb]{0.9,0.9,0.9}{
            \makecell[{{p{\linewidth}}}]{
                \texttt{\tiny{[GM$|$GM]}}
                \texttt{SPRICH: Ich denke, ich habe wieder einen Fehler gemacht. Lass uns noch mal überdenken, wie wir das 'L' bewegen können. Da die Position (1, 2) nicht leer ist, müssen wir einen anderen Weg finden. Kannst du mir bitte sagen, welche Positionen um das 'L' herum leer sind? Dann können wir einen neuen Plan machen.} \\
            }
        }
    }
    & & \\ \\

    \theutterance \stepcounter{utterance}  
    & & \multicolumn{4}{p{0.6\linewidth}}{
        \cellcolor[rgb]{0.9,0.9,0.9}{
            \makecell[{{p{\linewidth}}}]{
                \texttt{\tiny{[P1$\langle$GM]}}
                \texttt{Deine Nachricht wurde an deinen Mitspieler übermittelt.} \\
\\ 
\\ 
\texttt{Ihr seid gerade bei Runde 4 von maximal 12.} \\
\texttt{Ihr habt gemeinsam 3 von maximal 8 Strafpunkten angesammelt.} \\
\texttt{Dein Mitspieler hat dir diese Nachricht geschickt:} \\
\\ 
\\ 
\texttt{"Ich denke, ich habe wieder einen Fehler gemacht. Lass uns noch mal überdenken, wie wir das 'L' bewegen können. Da die Position (1, 2) nicht leer ist, müssen wir einen anderen Weg finden. Kannst du mir bitte sagen, welche Positionen um das 'L' herum leer sind? Dann können wir einen neuen Plan machen."} \\
\\ 
\\ 
\texttt{Wie lautet dein nächster Befehl?} \\
            }
        }
    }
    & \\ \\

    \theutterance \stepcounter{utterance}  
    & \multicolumn{4}{p{0.6\linewidth}}{
        \cellcolor[rgb]{0.9,0.9,0.9}{
            \makecell[{{p{\linewidth}}}]{
                \texttt{\tiny{[P1$\rangle$GM]}}
                \texttt{SPRICH: Lass uns die Positionen um das 'L' herum überprüfen. Da das 'L' bei dir an der Position (1, 1) ist, können wir die umliegenden Positionen überprüfen. Die Position (1, 2) ist bei dir besetzt, aber ich denke, die Position (2, 1) könnte leer sein. Bei mir ist die Position (2, 1) leer. Ich schlage vor, dass du das 'L' nach rechts auf die Position (2, 1) bewegst, wenn diese Position bei dir auch leer ist. Dann können wir weiter planen, wie wir es auf die Zielposition bringen. Bitte überprüfe, ob die Position (2, 1) bei dir leer ist.} \\
            }
        }
    }
    & & \\ \\

    \theutterance \stepcounter{utterance}  
    & & & \multicolumn{2}{p{0.3\linewidth}}{
        \cellcolor[rgb]{0.9,0.9,0.9}{
            \makecell[{{p{\linewidth}}}]{
                \texttt{\tiny{[GM$|$GM]}}
                \texttt{SPRICH: Lass uns die Positionen um das 'L' herum überprüfen. Da das 'L' bei dir an der Position (1, 1) ist, können wir die umliegenden Positionen überprüfen. Die Position (1, 2) ist bei dir besetzt, aber ich denke, die Position (2, 1) könnte leer sein. Bei mir ist die Position (2, 1) leer. Ich schlage vor, dass du das 'L' nach rechts auf die Position (2, 1) bewegst, wenn diese Position bei dir auch leer ist. Dann können wir weiter planen, wie wir es auf die Zielposition bringen. Bitte überprüfe, ob die Position (2, 1) bei dir leer ist.} \\
            }
        }
    }
    & & \\ \\

    \theutterance \stepcounter{utterance}  
    & & \multicolumn{4}{p{0.6\linewidth}}{
        \cellcolor[rgb]{0.9,0.9,0.9}{
            \makecell[{{p{\linewidth}}}]{
                \texttt{\tiny{[GM$\rangle$P2]}}
                \texttt{Deine Nachricht wurde an deinen Mitspieler übermittelt.} \\
\\ 
\\ 
\texttt{Ihr seid gerade bei Runde 5 von maximal 12.} \\
\texttt{Ihr habt gemeinsam 3 von maximal 8 Strafpunkten angesammelt.} \\
\texttt{Dein Mitspieler hat dir diese Nachricht geschickt:} \\
\\ 
\\ 
\texttt{"Lass uns die Positionen um das 'L' herum überprüfen. Da das 'L' bei dir an der Position (1, 1) ist, können wir die umliegenden Positionen überprüfen. Die Position (1, 2) ist bei dir besetzt, aber ich denke, die Position (2, 1) könnte leer sein. Bei mir ist die Position (2, 1) leer. Ich schlage vor, dass du das 'L' nach rechts auf die Position (2, 1) bewegst, wenn diese Position bei dir auch leer ist. Dann können wir weiter planen, wie wir es auf die Zielposition bringen. Bitte überprüfe, ob die Position (2, 1) bei dir leer ist."} \\
\\ 
\\ 
\texttt{Wie lautet dein nächster Befehl?} \\
            }
        }
    }
    & \\ \\

    \theutterance \stepcounter{utterance}  
    & & & \multicolumn{4}{p{0.6\linewidth}}{
        \cellcolor[rgb]{0.9,0.9,0.9}{
            \makecell[{{p{\linewidth}}}]{
                \texttt{\tiny{[GM$\langle$P2]}}
                \texttt{SPRICH: Ich habe die Position (2, 1) überprüft und leider ist sie nicht leer, wie wir bereits früher festgestellt haben. Sie enthält das Zeichen '│'. Lass uns weitere Optionen betrachten. Die Position direkt unter dem 'L', also (1, 2), ist bei mir auch nicht leer, da sie ein '─' enthält. Ich denke, wir sollten die Position (2, 1) ignorieren und stattdessen versuchen, das 'L' nach unten zu bewegen. Kannst du mir bitte bestätigen, dass die Position (1, 3) leer ist? Wenn ja, könnte ich das 'L' dorthin bewegen.} \\
            }
        }
    }
     \\ \\

    \theutterance \stepcounter{utterance}  
    & & & \multicolumn{2}{p{0.3\linewidth}}{
        \cellcolor[rgb]{0.9,0.9,0.9}{
            \makecell[{{p{\linewidth}}}]{
                \texttt{\tiny{[GM$|$GM]}}
                \texttt{SPRICH: Ich habe die Position (2, 1) überprüft und leider ist sie nicht leer, wie wir bereits früher festgestellt haben. Sie enthält das Zeichen '│'. Lass uns weitere Optionen betrachten. Die Position direkt unter dem 'L', also (1, 2), ist bei mir auch nicht leer, da sie ein '─' enthält. Ich denke, wir sollten die Position (2, 1) ignorieren und stattdessen versuchen, das 'L' nach unten zu bewegen. Kannst du mir bitte bestätigen, dass die Position (1, 3) leer ist? Wenn ja, könnte ich das 'L' dorthin bewegen.} \\
            }
        }
    }
    & & \\ \\

    \theutterance \stepcounter{utterance}  
    & & \multicolumn{4}{p{0.6\linewidth}}{
        \cellcolor[rgb]{0.9,0.9,0.9}{
            \makecell[{{p{\linewidth}}}]{
                \texttt{\tiny{[P1$\langle$GM]}}
                \texttt{Deine Nachricht wurde an deinen Mitspieler übermittelt.} \\
\\ 
\\ 
\texttt{Ihr seid gerade bei Runde 5 von maximal 12.} \\
\texttt{Ihr habt gemeinsam 3 von maximal 8 Strafpunkten angesammelt.} \\
\texttt{Dein Mitspieler hat dir diese Nachricht geschickt:} \\
\\ 
\\ 
\texttt{"Ich habe die Position (2, 1) überprüft und leider ist sie nicht leer, wie wir bereits früher festgestellt haben. Sie enthält das Zeichen '│'. Lass uns weitere Optionen betrachten. Die Position direkt unter dem 'L', also (1, 2), ist bei mir auch nicht leer, da sie ein '─' enthält. Ich denke, wir sollten die Position (2, 1) ignorieren und stattdessen versuchen, das 'L' nach unten zu bewegen. Kannst du mir bitte bestätigen, dass die Position (1, 3) leer ist? Wenn ja, könnte ich das 'L' dorthin bewegen."} \\
\\ 
\\ 
\texttt{Wie lautet dein nächster Befehl?} \\
            }
        }
    }
    & \\ \\

    \theutterance \stepcounter{utterance}  
    & \multicolumn{4}{p{0.6\linewidth}}{
        \cellcolor[rgb]{0.9,0.9,0.9}{
            \makecell[{{p{\linewidth}}}]{
                \texttt{\tiny{[P1$\rangle$GM]}}
                \texttt{SPRICH: Ja, die Position (1, 3) ist leer. Du kannst das 'L' dorthin bewegen. Das wäre ein guter erster Schritt, um das 'L' in die Nähe der Zielposition zu bringen. Nachdem du das 'L' auf (1, 3) bewegt hast, können wir weiter planen, wie wir es auf die Zielposition (3, 4) bringen. Ich denke, der nächste Schritt wäre, das 'L' nach rechts zu bewegen, um es näher an die Zielposition zu bringen. Bitte teile mir mit, wenn du das 'L' erfolgreich auf (1, 3) bewegt hast.} \\
            }
        }
    }
    & & \\ \\

    \theutterance \stepcounter{utterance}  
    & & & \multicolumn{2}{p{0.3\linewidth}}{
        \cellcolor[rgb]{0.9,0.9,0.9}{
            \makecell[{{p{\linewidth}}}]{
                \texttt{\tiny{[GM$|$GM]}}
                \texttt{SPRICH: Ja, die Position (1, 3) ist leer. Du kannst das 'L' dorthin bewegen. Das wäre ein guter erster Schritt, um das 'L' in die Nähe der Zielposition zu bringen. Nachdem du das 'L' auf (1, 3) bewegt hast, können wir weiter planen, wie wir es auf die Zielposition (3, 4) bringen. Ich denke, der nächste Schritt wäre, das 'L' nach rechts zu bewegen, um es näher an die Zielposition zu bringen. Bitte teile mir mit, wenn du das 'L' erfolgreich auf (1, 3) bewegt hast.} \\
            }
        }
    }
    & & \\ \\

    \theutterance \stepcounter{utterance}  
    & & \multicolumn{4}{p{0.6\linewidth}}{
        \cellcolor[rgb]{0.9,0.9,0.9}{
            \makecell[{{p{\linewidth}}}]{
                \texttt{\tiny{[GM$\rangle$P2]}}
                \texttt{Deine Nachricht wurde an deinen Mitspieler übermittelt.} \\
\\ 
\\ 
\texttt{Ihr seid gerade bei Runde 6 von maximal 12.} \\
\texttt{Ihr habt gemeinsam 3 von maximal 8 Strafpunkten angesammelt.} \\
\texttt{Dein Mitspieler hat dir diese Nachricht geschickt:} \\
\\ 
\\ 
\texttt{"Ja, die Position (1, 3) ist leer. Du kannst das 'L' dorthin bewegen. Das wäre ein guter erster Schritt, um das 'L' in die Nähe der Zielposition zu bringen. Nachdem du das 'L' auf (1, 3) bewegt hast, können wir weiter planen, wie wir es auf die Zielposition (3, 4) bringen. Ich denke, der nächste Schritt wäre, das 'L' nach rechts zu bewegen, um es näher an die Zielposition zu bringen. Bitte teile mir mit, wenn du das 'L' erfolgreich auf (1, 3) bewegt hast."} \\
\\ 
\\ 
\texttt{Wie lautet dein nächster Befehl?} \\
            }
        }
    }
    & \\ \\

    \theutterance \stepcounter{utterance}  
    & & & \multicolumn{4}{p{0.6\linewidth}}{
        \cellcolor[rgb]{0.9,0.9,0.9}{
            \makecell[{{p{\linewidth}}}]{
                \texttt{\tiny{[GM$\langle$P2]}}
                \texttt{VERSCHIEBE: L, (1, 3)} \\
            }
        }
    }
     \\ \\

    \theutterance \stepcounter{utterance}  
    & & & \multicolumn{2}{p{0.3\linewidth}}{
        \cellcolor[rgb]{0.9,0.9,0.9}{
            \makecell[{{p{\linewidth}}}]{
                \texttt{\tiny{[GM$|$GM]}}
                \texttt{VERSCHIEBE: L, (1, 3)} \\
            }
        }
    }
    & & \\ \\

    \theutterance \stepcounter{utterance}  
    & & & \multicolumn{2}{p{0.3\linewidth}}{
        \cellcolor[rgb]{0.9,0.9,0.9}{
            \makecell[{{p{\linewidth}}}]{
                \texttt{\tiny{[GM$|$GM]}}
                \texttt{'L' erfolgreich nach (1,3) verschoben. Dein Raster sieht jetzt so aus:} \\
\texttt{```} \\
\texttt{ 1234567} \\
\texttt{╔═╤═╤═╤═╗} \\
\texttt{║◌│◌│◌│◌║ 1} \\
\texttt{╟─┼─┤◌│◌║ 2} \\
\texttt{║L│◌│◌│◌║ 3} \\
\texttt{║◌│◌├─┴─╢ 4} \\
\texttt{╟─┴─┤C◌◌║ 5} \\
\texttt{║◌◌◌└─┬─╢ 6} \\
\texttt{║◌P◌◌◌│◌║ 7} \\
\texttt{╚═════╧═╝} \\
\texttt{```} \\
            }
        }
    }
    & & \\ \\

    \theutterance \stepcounter{utterance}  
    & & \multicolumn{4}{p{0.6\linewidth}}{
        \cellcolor[rgb]{0.9,0.9,0.9}{
            \makecell[{{p{\linewidth}}}]{
                \texttt{\tiny{[P1$\langle$GM]}}
                \texttt{Deine Nachricht wurde an deinen Mitspieler übermittelt.} \\
\\ 
\\ 
\texttt{Ihr seid gerade bei Runde 6 von maximal 12.} \\
\texttt{Ihr habt gemeinsam 3 von maximal 8 Strafpunkten angesammelt.} \\
\texttt{Dein Mitspieler hat ein Objekt auf seinem Raster bewegt.} \\
\\ 
\\ 
\texttt{Wie lautet dein nächster Befehl?} \\
            }
        }
    }
    & \\ \\

    \theutterance \stepcounter{utterance}  
    & \multicolumn{4}{p{0.6\linewidth}}{
        \cellcolor[rgb]{0.9,0.9,0.9}{
            \makecell[{{p{\linewidth}}}]{
                \texttt{\tiny{[P1$\rangle$GM]}}
                \texttt{SPRICH: Ich denke, es ist nun an der Zeit, dass ich auch ein Objekt auf meinem Raster bewege. Ich werde das 'L' auf (1, 4) auf die Position (2, 4) bewegen, um es näher an die Zielposition zu bringen. Ich werde den Befehl VERSCHIEBE: L, (2, 4) ausführen. Bitte beachte, dass ich auf meinem Raster arbeite und unsere Objekte möglicherweise nicht genau同步 sind. Ich werde weiterhin mit dir kommunizieren, um unsere Strategie abzustimmen. Wie sieht es bei dir aus? Hast du weitere Schritte geplant, um das 'L' auf die Zielposition zu bringen?} \\
            }
        }
    }
    & & \\ \\

    \theutterance \stepcounter{utterance}  
    & & & \multicolumn{2}{p{0.3\linewidth}}{
        \cellcolor[rgb]{0.9,0.9,0.9}{
            \makecell[{{p{\linewidth}}}]{
                \texttt{\tiny{[GM$|$GM]}}
                \texttt{SPRICH: Ich denke, es ist nun an der Zeit, dass ich auch ein Objekt auf meinem Raster bewege. Ich werde das 'L' auf (1, 4) auf die Position (2, 4) bewegen, um es näher an die Zielposition zu bringen. Ich werde den Befehl VERSCHIEBE: L, (2, 4) ausführen. Bitte beachte, dass ich auf meinem Raster arbeite und unsere Objekte möglicherweise nicht genau同步 sind. Ich werde weiterhin mit dir kommunizieren, um unsere Strategie abzustimmen. Wie sieht es bei dir aus? Hast du weitere Schritte geplant, um das 'L' auf die Zielposition zu bringen?} \\
            }
        }
    }
    & & \\ \\

    \theutterance \stepcounter{utterance}  
    & & & \multicolumn{2}{p{0.3\linewidth}}{
        \cellcolor[rgb]{0.9,0.9,0.9}{
            \makecell[{{p{\linewidth}}}]{
                \texttt{\tiny{[GM$|$GM]}}
                \texttt{Invalid response format: SPRICH: Ich denke, es ist nun an der Zeit, dass ich auch ein Objekt auf meinem Raster bewege. Ich werde das 'L' auf (1, 4) auf die Position (2, 4) bewegen, um es näher an die Zielposition zu bringen. Ich werde den Befehl VERSCHIEBE: L, (2, 4) ausführen. Bitte beachte, dass ich auf meinem Raster arbeite und unsere Objekte möglicherweise nicht genau同步 sind. Ich werde weiterhin mit dir kommunizieren, um unsere Strategie abzustimmen. Wie sieht es bei dir aus? Hast du weitere Schritte geplant, um das 'L' auf die Zielposition zu bringen?} \\
            }
        }
    }
    & & \\ \\

    \theutterance \stepcounter{utterance}  
    & & \multicolumn{4}{p{0.6\linewidth}}{
        \cellcolor[rgb]{0.9,0.9,0.9}{
            \makecell[{{p{\linewidth}}}]{
                \texttt{\tiny{[P1$\langle$GM]}}
                \texttt{Strafpunkt: Deine Nachricht enthält mehr als einen Befehl! Stelle sicher, dass deine Antwort ausschließlich einen der beiden Befehle `SPRICH: <NACHRICHT>` und `VERSCHIEBE: <OBJECT>, (<X>, <Y>)` und sonst nichts enthält!} \\
\\ 
\texttt{Ihr habt gemeinsam 4 von maximal 8 Strafpunkten angesammelt.} \\
\texttt{Bitte versuche es noch mal!} \\
            }
        }
    }
    & \\ \\

    \theutterance \stepcounter{utterance}  
    & \multicolumn{4}{p{0.6\linewidth}}{
        \cellcolor[rgb]{0.9,0.9,0.9}{
            \makecell[{{p{\linewidth}}}]{
                \texttt{\tiny{[P1$\rangle$GM]}}
                \texttt{SPRICH: Ich denke, es ist nun an der Zeit, dass ich auch ein Objekt auf meinem Raster bewege. Ich werde das 'L' auf (1, 4) auf die Position (2, 4) bewegen, um es näher an die Zielposition zu bringen. Bitte beachte, dass ich auf meinem Raster arbeite und unsere Objekte möglicherweise nicht genau synchron sind. Ich werde weiterhin mit dir kommunizieren, um unsere Strategie abzustimmen. Wie sieht es bei dir aus? Hast du weitere Schritte geplant, um das 'L' auf die Zielposition zu bringen?} \\
            }
        }
    }
    & & \\ \\

    \theutterance \stepcounter{utterance}  
    & & & \multicolumn{2}{p{0.3\linewidth}}{
        \cellcolor[rgb]{0.9,0.9,0.9}{
            \makecell[{{p{\linewidth}}}]{
                \texttt{\tiny{[GM$|$GM]}}
                \texttt{SPRICH: Ich denke, es ist nun an der Zeit, dass ich auch ein Objekt auf meinem Raster bewege. Ich werde das 'L' auf (1, 4) auf die Position (2, 4) bewegen, um es näher an die Zielposition zu bringen. Bitte beachte, dass ich auf meinem Raster arbeite und unsere Objekte möglicherweise nicht genau synchron sind. Ich werde weiterhin mit dir kommunizieren, um unsere Strategie abzustimmen. Wie sieht es bei dir aus? Hast du weitere Schritte geplant, um das 'L' auf die Zielposition zu bringen?} \\
            }
        }
    }
    & & \\ \\

    \theutterance \stepcounter{utterance}  
    & & \multicolumn{4}{p{0.6\linewidth}}{
        \cellcolor[rgb]{0.9,0.9,0.9}{
            \makecell[{{p{\linewidth}}}]{
                \texttt{\tiny{[GM$\rangle$P2]}}
                \texttt{'L' erfolgreich nach (1,3) verschoben. Dein Raster sieht jetzt so aus:} \\
\texttt{```} \\
\texttt{ 1234567} \\
\texttt{╔═╤═╤═╤═╗} \\
\texttt{║◌│◌│◌│◌║ 1} \\
\texttt{╟─┼─┤◌│◌║ 2} \\
\texttt{║L│◌│◌│◌║ 3} \\
\texttt{║◌│◌├─┴─╢ 4} \\
\texttt{╟─┴─┤C◌◌║ 5} \\
\texttt{║◌◌◌└─┬─╢ 6} \\
\texttt{║◌P◌◌◌│◌║ 7} \\
\texttt{╚═════╧═╝} \\
\texttt{```} \\
\texttt{Ihr seid gerade bei Runde 7 von maximal 12.} \\
\texttt{Ihr habt gemeinsam 4 von maximal 8 Strafpunkten angesammelt.} \\
\texttt{Dein Mitspieler hat dir diese Nachricht geschickt:} \\
\\ 
\\ 
\texttt{"Ich denke, es ist nun an der Zeit, dass ich auch ein Objekt auf meinem Raster bewege. Ich werde das 'L' auf (1, 4) auf die Position (2, 4) bewegen, um es näher an die Zielposition zu bringen. Bitte beachte, dass ich auf meinem Raster arbeite und unsere Objekte möglicherweise nicht genau synchron sind. Ich werde weiterhin mit dir kommunizieren, um unsere Strategie abzustimmen. Wie sieht es bei dir aus? Hast du weitere Schritte geplant, um das 'L' auf die Zielposition zu bringen?"} \\
\\ 
\\ 
\texttt{Wie lautet dein nächster Befehl?} \\
            }
        }
    }
    & \\ \\

    \theutterance \stepcounter{utterance}  
    & & & \multicolumn{4}{p{0.6\linewidth}}{
        \cellcolor[rgb]{0.9,0.9,0.9}{
            \makecell[{{p{\linewidth}}}]{
                \texttt{\tiny{[GM$\langle$P2]}}
                \texttt{SPRICH: Ich denke, es ist großartig, dass du auch ein Objekt bewegst! Ich werde nun mein 'L' auf (1, 3) nach rechts auf die Position (2, 3) bewegen, um es näher an die Zielposition zu bringen. Dann können wir weiter planen, wie wir es auf die Zielposition (3, 4) bringen. Bitte teile mir mit, wenn du das 'L' auf deinem Raster erfolgreich bewegt hast und wie es bei dir weitergeht. Ich bin gespannt, wie unsere Objekte sich annähern!} \\
            }
        }
    }
     \\ \\

    \theutterance \stepcounter{utterance}  
    & & & \multicolumn{2}{p{0.3\linewidth}}{
        \cellcolor[rgb]{0.9,0.9,0.9}{
            \makecell[{{p{\linewidth}}}]{
                \texttt{\tiny{[GM$|$GM]}}
                \texttt{SPRICH: Ich denke, es ist großartig, dass du auch ein Objekt bewegst! Ich werde nun mein 'L' auf (1, 3) nach rechts auf die Position (2, 3) bewegen, um es näher an die Zielposition zu bringen. Dann können wir weiter planen, wie wir es auf die Zielposition (3, 4) bringen. Bitte teile mir mit, wenn du das 'L' auf deinem Raster erfolgreich bewegt hast und wie es bei dir weitergeht. Ich bin gespannt, wie unsere Objekte sich annähern!} \\
            }
        }
    }
    & & \\ \\

    \theutterance \stepcounter{utterance}  
    & & \multicolumn{4}{p{0.6\linewidth}}{
        \cellcolor[rgb]{0.9,0.9,0.9}{
            \makecell[{{p{\linewidth}}}]{
                \texttt{\tiny{[P1$\langle$GM]}}
                \texttt{Deine Nachricht wurde an deinen Mitspieler übermittelt.} \\
\\ 
\\ 
\texttt{Ihr seid gerade bei Runde 7 von maximal 12.} \\
\texttt{Ihr habt gemeinsam 4 von maximal 8 Strafpunkten angesammelt.} \\
\texttt{Dein Mitspieler hat dir diese Nachricht geschickt:} \\
\\ 
\\ 
\texttt{"Ich denke, es ist großartig, dass du auch ein Objekt bewegst! Ich werde nun mein 'L' auf (1, 3) nach rechts auf die Position (2, 3) bewegen, um es näher an die Zielposition zu bringen. Dann können wir weiter planen, wie wir es auf die Zielposition (3, 4) bringen. Bitte teile mir mit, wenn du das 'L' auf deinem Raster erfolgreich bewegt hast und wie es bei dir weitergeht. Ich bin gespannt, wie unsere Objekte sich annähern!"} \\
\\ 
\\ 
\texttt{Wie lautet dein nächster Befehl?} \\
            }
        }
    }
    & \\ \\

    \theutterance \stepcounter{utterance}  
    & \multicolumn{4}{p{0.6\linewidth}}{
        \cellcolor[rgb]{0.9,0.9,0.9}{
            \makecell[{{p{\linewidth}}}]{
                \texttt{\tiny{[P1$\rangle$GM]}}
                \texttt{VERSCHIEBE: L, (2, 4)} \\
            }
        }
    }
    & & \\ \\

    \theutterance \stepcounter{utterance}  
    & & & \multicolumn{2}{p{0.3\linewidth}}{
        \cellcolor[rgb]{0.9,0.9,0.9}{
            \makecell[{{p{\linewidth}}}]{
                \texttt{\tiny{[GM$|$GM]}}
                \texttt{VERSCHIEBE: L, (2, 4)} \\
            }
        }
    }
    & & \\ \\

    \theutterance \stepcounter{utterance}  
    & & & \multicolumn{2}{p{0.3\linewidth}}{
        \cellcolor[rgb]{0.9,0.9,0.9}{
            \makecell[{{p{\linewidth}}}]{
                \texttt{\tiny{[GM$|$GM]}}
                \texttt{Strafpunkt: (2,4) ist nicht leer, sondern enthält '│'.} \\
\\ 
\texttt{Ihr habt gemeinsam 5 von maximal 8 Strafpunkten angesammelt.} \\
\texttt{Bitte versuche es noch mal!} \\
            }
        }
    }
    & & \\ \\

    \theutterance \stepcounter{utterance}  
    & & \multicolumn{4}{p{0.6\linewidth}}{
        \cellcolor[rgb]{0.9,0.9,0.9}{
            \makecell[{{p{\linewidth}}}]{
                \texttt{\tiny{[P1$\langle$GM]}}
                \texttt{Strafpunkt: (2,4) ist nicht leer, sondern enthält '│'.} \\
\\ 
\texttt{Ihr habt gemeinsam 5 von maximal 8 Strafpunkten angesammelt.} \\
\texttt{Bitte versuche es noch mal!} \\
            }
        }
    }
    & \\ \\

    \theutterance \stepcounter{utterance}  
    & \multicolumn{4}{p{0.6\linewidth}}{
        \cellcolor[rgb]{0.9,0.9,0.9}{
            \makecell[{{p{\linewidth}}}]{
                \texttt{\tiny{[P1$\rangle$GM]}}
                \texttt{SPRICH: Ich habe gerade versucht, mein 'L' zu bewegen, aber die Zielposition war nicht leer. Ich muss einen anderen Weg finden. Kannst du mir bitte helfen, eine alternative Route für mein 'L' zu finden, um es näher an die Zielposition zu bringen? Ich denke, ich könnte es nach rechts oder unten bewegen, aber ich muss sicherstellen, dass die Zielposition leer ist. Bitte teile mir mit, wenn du eine Idee hast, wie ich mein 'L' bewegen kann.} \\
            }
        }
    }
    & & \\ \\

    \theutterance \stepcounter{utterance}  
    & & & \multicolumn{2}{p{0.3\linewidth}}{
        \cellcolor[rgb]{0.9,0.9,0.9}{
            \makecell[{{p{\linewidth}}}]{
                \texttt{\tiny{[GM$|$GM]}}
                \texttt{SPRICH: Ich habe gerade versucht, mein 'L' zu bewegen, aber die Zielposition war nicht leer. Ich muss einen anderen Weg finden. Kannst du mir bitte helfen, eine alternative Route für mein 'L' zu finden, um es näher an die Zielposition zu bringen? Ich denke, ich könnte es nach rechts oder unten bewegen, aber ich muss sicherstellen, dass die Zielposition leer ist. Bitte teile mir mit, wenn du eine Idee hast, wie ich mein 'L' bewegen kann.} \\
            }
        }
    }
    & & \\ \\

    \theutterance \stepcounter{utterance}  
    & & \multicolumn{4}{p{0.6\linewidth}}{
        \cellcolor[rgb]{0.9,0.9,0.9}{
            \makecell[{{p{\linewidth}}}]{
                \texttt{\tiny{[GM$\rangle$P2]}}
                \texttt{Deine Nachricht wurde an deinen Mitspieler übermittelt.} \\
\\ 
\\ 
\texttt{Ihr seid gerade bei Runde 8 von maximal 12.} \\
\texttt{Ihr habt gemeinsam 5 von maximal 8 Strafpunkten angesammelt.} \\
\texttt{Dein Mitspieler hat dir diese Nachricht geschickt:} \\
\\ 
\\ 
\texttt{"Ich habe gerade versucht, mein 'L' zu bewegen, aber die Zielposition war nicht leer. Ich muss einen anderen Weg finden. Kannst du mir bitte helfen, eine alternative Route für mein 'L' zu finden, um es näher an die Zielposition zu bringen? Ich denke, ich könnte es nach rechts oder unten bewegen, aber ich muss sicherstellen, dass die Zielposition leer ist. Bitte teile mir mit, wenn du eine Idee hast, wie ich mein 'L' bewegen kann."} \\
\\ 
\\ 
\texttt{Wie lautet dein nächster Befehl?} \\
            }
        }
    }
    & \\ \\

    \theutterance \stepcounter{utterance}  
    & & & \multicolumn{4}{p{0.6\linewidth}}{
        \cellcolor[rgb]{0.9,0.9,0.9}{
            \makecell[{{p{\linewidth}}}]{
                \texttt{\tiny{[GM$\langle$P2]}}
                \texttt{SPRICH: Es tut mir leid, dass du Probleme hast, dein 'L' zu bewegen! Ich denke, es wäre hilfreich, wenn wir uns die Positionen um dein 'L' herum noch einmal ansehen. Da dein 'L' an der Position (1, 4) ist, könntest du es nach rechts auf die Position (2, 4) bewegen, aber du hast bereits erwähnt, dass diese Position nicht leer ist. Eine Alternative könnte sein, das 'L' nach unten auf die Position (1, 5) zu bewegen, wenn diese Position leer ist. Kannst du bitte überprüfen, ob die Position (1, 5) leer ist? Wenn ja, könntest du das 'L' dorthin bewegen und dann weiter planen, wie du es auf die Zielposition bringst. Lass mich wissen, wenn du weitere Fragen hast oder wenn ich dir sonst noch helfen kann!} \\
            }
        }
    }
     \\ \\

    \theutterance \stepcounter{utterance}  
    & & & \multicolumn{2}{p{0.3\linewidth}}{
        \cellcolor[rgb]{0.9,0.9,0.9}{
            \makecell[{{p{\linewidth}}}]{
                \texttt{\tiny{[GM$|$GM]}}
                \texttt{SPRICH: Es tut mir leid, dass du Probleme hast, dein 'L' zu bewegen! Ich denke, es wäre hilfreich, wenn wir uns die Positionen um dein 'L' herum noch einmal ansehen. Da dein 'L' an der Position (1, 4) ist, könntest du es nach rechts auf die Position (2, 4) bewegen, aber du hast bereits erwähnt, dass diese Position nicht leer ist. Eine Alternative könnte sein, das 'L' nach unten auf die Position (1, 5) zu bewegen, wenn diese Position leer ist. Kannst du bitte überprüfen, ob die Position (1, 5) leer ist? Wenn ja, könntest du das 'L' dorthin bewegen und dann weiter planen, wie du es auf die Zielposition bringst. Lass mich wissen, wenn du weitere Fragen hast oder wenn ich dir sonst noch helfen kann!} \\
            }
        }
    }
    & & \\ \\

    \theutterance \stepcounter{utterance}  
    & & \multicolumn{4}{p{0.6\linewidth}}{
        \cellcolor[rgb]{0.9,0.9,0.9}{
            \makecell[{{p{\linewidth}}}]{
                \texttt{\tiny{[P1$\langle$GM]}}
                \texttt{Deine Nachricht wurde an deinen Mitspieler übermittelt.} \\
\\ 
\\ 
\texttt{Ihr seid gerade bei Runde 8 von maximal 12.} \\
\texttt{Ihr habt gemeinsam 5 von maximal 8 Strafpunkten angesammelt.} \\
\texttt{Dein Mitspieler hat dir diese Nachricht geschickt:} \\
\\ 
\\ 
\texttt{"Es tut mir leid, dass du Probleme hast, dein 'L' zu bewegen! Ich denke, es wäre hilfreich, wenn wir uns die Positionen um dein 'L' herum noch einmal ansehen. Da dein 'L' an der Position (1, 4) ist, könntest du es nach rechts auf die Position (2, 4) bewegen, aber du hast bereits erwähnt, dass diese Position nicht leer ist. Eine Alternative könnte sein, das 'L' nach unten auf die Position (1, 5) zu bewegen, wenn diese Position leer ist. Kannst du bitte überprüfen, ob die Position (1, 5) leer ist? Wenn ja, könntest du das 'L' dorthin bewegen und dann weiter planen, wie du es auf die Zielposition bringst. Lass mich wissen, wenn du weitere Fragen hast oder wenn ich dir sonst noch helfen kann!"} \\
\\ 
\\ 
\texttt{Wie lautet dein nächster Befehl?} \\
            }
        }
    }
    & \\ \\

    \theutterance \stepcounter{utterance}  
    & \multicolumn{4}{p{0.6\linewidth}}{
        \cellcolor[rgb]{0.9,0.9,0.9}{
            \makecell[{{p{\linewidth}}}]{
                \texttt{\tiny{[P1$\rangle$GM]}}
                \texttt{VERSCHIEBE: L, (1, 5)} \\
            }
        }
    }
    & & \\ \\

    \theutterance \stepcounter{utterance}  
    & & & \multicolumn{2}{p{0.3\linewidth}}{
        \cellcolor[rgb]{0.9,0.9,0.9}{
            \makecell[{{p{\linewidth}}}]{
                \texttt{\tiny{[GM$|$GM]}}
                \texttt{VERSCHIEBE: L, (1, 5)} \\
            }
        }
    }
    & & \\ \\

    \theutterance \stepcounter{utterance}  
    & & & \multicolumn{2}{p{0.3\linewidth}}{
        \cellcolor[rgb]{0.9,0.9,0.9}{
            \makecell[{{p{\linewidth}}}]{
                \texttt{\tiny{[GM$|$GM]}}
                \texttt{Strafpunkt: (1,5) ist nicht leer, sondern enthält '─'.} \\
\\ 
\texttt{Ihr habt gemeinsam 6 von maximal 8 Strafpunkten angesammelt.} \\
\texttt{Bitte versuche es noch mal!} \\
            }
        }
    }
    & & \\ \\

    \theutterance \stepcounter{utterance}  
    & & \multicolumn{4}{p{0.6\linewidth}}{
        \cellcolor[rgb]{0.9,0.9,0.9}{
            \makecell[{{p{\linewidth}}}]{
                \texttt{\tiny{[P1$\langle$GM]}}
                \texttt{Strafpunkt: (1,5) ist nicht leer, sondern enthält '─'.} \\
\\ 
\texttt{Ihr habt gemeinsam 6 von maximal 8 Strafpunkten angesammelt.} \\
\texttt{Bitte versuche es noch mal!} \\
            }
        }
    }
    & \\ \\

    \theutterance \stepcounter{utterance}  
    & \multicolumn{4}{p{0.6\linewidth}}{
        \cellcolor[rgb]{0.9,0.9,0.9}{
            \makecell[{{p{\linewidth}}}]{
                \texttt{\tiny{[P1$\rangle$GM]}}
                \texttt{SPRICH: Ich habe wieder versucht, mein 'L' zu bewegen, aber die Zielposition war nicht leer. Ich denke, ich brauche Hilfe, um eine alternative Route zu finden. Kannst du mir bitte helfen, eine andere Möglichkeit zu finden, mein 'L' näher an die Zielposition zu bringen? Ich kann meine Raster{-}Positionen noch einmal überprüfen und sehen, ob es eine andere Möglichkeit gibt, mein 'L' zu bewegen. Bitte teile mir mit, wenn du eine Idee hast, wie ich mein 'L' bewegen kann. Ich bin gespannt, ob wir gemeinsam eine Lösung finden können!} \\
            }
        }
    }
    & & \\ \\

    \theutterance \stepcounter{utterance}  
    & & & \multicolumn{2}{p{0.3\linewidth}}{
        \cellcolor[rgb]{0.9,0.9,0.9}{
            \makecell[{{p{\linewidth}}}]{
                \texttt{\tiny{[GM$|$GM]}}
                \texttt{SPRICH: Ich habe wieder versucht, mein 'L' zu bewegen, aber die Zielposition war nicht leer. Ich denke, ich brauche Hilfe, um eine alternative Route zu finden. Kannst du mir bitte helfen, eine andere Möglichkeit zu finden, mein 'L' näher an die Zielposition zu bringen? Ich kann meine Raster{-}Positionen noch einmal überprüfen und sehen, ob es eine andere Möglichkeit gibt, mein 'L' zu bewegen. Bitte teile mir mit, wenn du eine Idee hast, wie ich mein 'L' bewegen kann. Ich bin gespannt, ob wir gemeinsam eine Lösung finden können!} \\
            }
        }
    }
    & & \\ \\

    \theutterance \stepcounter{utterance}  
    & & \multicolumn{4}{p{0.6\linewidth}}{
        \cellcolor[rgb]{0.9,0.9,0.9}{
            \makecell[{{p{\linewidth}}}]{
                \texttt{\tiny{[GM$\rangle$P2]}}
                \texttt{Deine Nachricht wurde an deinen Mitspieler übermittelt.} \\
\\ 
\\ 
\texttt{Ihr seid gerade bei Runde 9 von maximal 12.} \\
\texttt{Ihr habt gemeinsam 6 von maximal 8 Strafpunkten angesammelt.} \\
\texttt{Dein Mitspieler hat dir diese Nachricht geschickt:} \\
\\ 
\\ 
\texttt{"Ich habe wieder versucht, mein 'L' zu bewegen, aber die Zielposition war nicht leer. Ich denke, ich brauche Hilfe, um eine alternative Route zu finden. Kannst du mir bitte helfen, eine andere Möglichkeit zu finden, mein 'L' näher an die Zielposition zu bringen? Ich kann meine Raster{-}Positionen noch einmal überprüfen und sehen, ob es eine andere Möglichkeit gibt, mein 'L' zu bewegen. Bitte teile mir mit, wenn du eine Idee hast, wie ich mein 'L' bewegen kann. Ich bin gespannt, ob wir gemeinsam eine Lösung finden können!"} \\
\\ 
\\ 
\texttt{Wie lautet dein nächster Befehl?} \\
            }
        }
    }
    & \\ \\

    \theutterance \stepcounter{utterance}  
    & & & \multicolumn{4}{p{0.6\linewidth}}{
        \cellcolor[rgb]{0.9,0.9,0.9}{
            \makecell[{{p{\linewidth}}}]{
                \texttt{\tiny{[GM$\langle$P2]}}
                \texttt{SPRICH: Ich denke, es ist wichtig, dass wir gemeinsam nach einer Lösung suchen. Ich schlage vor, dass wir unsere Raster{-}Positionen noch einmal überprüfen und sehen, ob es eine Möglichkeit gibt, das 'L' auf einem anderen Weg zu bewegen. Da wir beide das Ziel haben, das 'L' auf die Position (3, 4) zu bringen, sollten wir uns auf die freien Positionen um die Zielposition herum konzentrieren. Ich denke, wir sollten auch unsere anderen Objekte ('C' und 'P') im Auge behalten, um sicherzustellen, dass wir sie nicht blockieren, wenn wir das 'L' bewegen. Lass uns gemeinsam eine Strategie entwickeln, um unsere Objekte auf die Zielpositionen zu bringen. Wie sieht es bei dir aus? Hast du eine Idee, wie wir vorgehen können?} \\
            }
        }
    }
     \\ \\

    \theutterance \stepcounter{utterance}  
    & & & \multicolumn{2}{p{0.3\linewidth}}{
        \cellcolor[rgb]{0.9,0.9,0.9}{
            \makecell[{{p{\linewidth}}}]{
                \texttt{\tiny{[GM$|$GM]}}
                \texttt{SPRICH: Ich denke, es ist wichtig, dass wir gemeinsam nach einer Lösung suchen. Ich schlage vor, dass wir unsere Raster{-}Positionen noch einmal überprüfen und sehen, ob es eine Möglichkeit gibt, das 'L' auf einem anderen Weg zu bewegen. Da wir beide das Ziel haben, das 'L' auf die Position (3, 4) zu bringen, sollten wir uns auf die freien Positionen um die Zielposition herum konzentrieren. Ich denke, wir sollten auch unsere anderen Objekte ('C' und 'P') im Auge behalten, um sicherzustellen, dass wir sie nicht blockieren, wenn wir das 'L' bewegen. Lass uns gemeinsam eine Strategie entwickeln, um unsere Objekte auf die Zielpositionen zu bringen. Wie sieht es bei dir aus? Hast du eine Idee, wie wir vorgehen können?} \\
            }
        }
    }
    & & \\ \\

    \theutterance \stepcounter{utterance}  
    & & \multicolumn{4}{p{0.6\linewidth}}{
        \cellcolor[rgb]{0.9,0.9,0.9}{
            \makecell[{{p{\linewidth}}}]{
                \texttt{\tiny{[P1$\langle$GM]}}
                \texttt{Deine Nachricht wurde an deinen Mitspieler übermittelt.} \\
\\ 
\\ 
\texttt{Ihr seid gerade bei Runde 9 von maximal 12.} \\
\texttt{Ihr habt gemeinsam 6 von maximal 8 Strafpunkten angesammelt.} \\
\texttt{Dein Mitspieler hat dir diese Nachricht geschickt:} \\
\\ 
\\ 
\texttt{"Ich denke, es ist wichtig, dass wir gemeinsam nach einer Lösung suchen. Ich schlage vor, dass wir unsere Raster{-}Positionen noch einmal überprüfen und sehen, ob es eine Möglichkeit gibt, das 'L' auf einem anderen Weg zu bewegen. Da wir beide das Ziel haben, das 'L' auf die Position (3, 4) zu bringen, sollten wir uns auf die freien Positionen um die Zielposition herum konzentrieren. Ich denke, wir sollten auch unsere anderen Objekte ('C' und 'P') im Auge behalten, um sicherzustellen, dass wir sie nicht blockieren, wenn wir das 'L' bewegen. Lass uns gemeinsam eine Strategie entwickeln, um unsere Objekte auf die Zielpositionen zu bringen. Wie sieht es bei dir aus? Hast du eine Idee, wie wir vorgehen können?"} \\
\\ 
\\ 
\texttt{Wie lautet dein nächster Befehl?} \\
            }
        }
    }
    & \\ \\

    \theutterance \stepcounter{utterance}  
    & \multicolumn{4}{p{0.6\linewidth}}{
        \cellcolor[rgb]{0.9,0.9,0.9}{
            \makecell[{{p{\linewidth}}}]{
                \texttt{\tiny{[P1$\rangle$GM]}}
                \texttt{SPRICH: Ich denke, es ist eine gute Idee, unsere Raster{-}Positionen noch einmal zu überprüfen und eine gemeinsame Strategie zu entwickeln. Ich schlage vor, dass wir unsere Objekte 'C', 'L' und 'P' auf die Positionen (4, 4), (3, 4) und (5, 4) bringen, wie wir ursprünglich geplant hatten. Ich denke, wir sollten unsere Objekte Schritt für Schritt bewegen, um sicherzustellen, dass wir keine Strafpunkte erhalten. Ich kann mein 'L' auf (1, 4) bewegen, aber ich brauche deine Hilfe, um eine neue Route zu finden. Kannst du mir bitte helfen, eine Möglichkeit zu finden, mein 'L' auf die Position (2, 3) oder (2, 4) zu bringen, ohne dass ich Strafpunkte erhalte?} \\
            }
        }
    }
    & & \\ \\

    \theutterance \stepcounter{utterance}  
    & & & \multicolumn{2}{p{0.3\linewidth}}{
        \cellcolor[rgb]{0.9,0.9,0.9}{
            \makecell[{{p{\linewidth}}}]{
                \texttt{\tiny{[GM$|$GM]}}
                \texttt{SPRICH: Ich denke, es ist eine gute Idee, unsere Raster{-}Positionen noch einmal zu überprüfen und eine gemeinsame Strategie zu entwickeln. Ich schlage vor, dass wir unsere Objekte 'C', 'L' und 'P' auf die Positionen (4, 4), (3, 4) und (5, 4) bringen, wie wir ursprünglich geplant hatten. Ich denke, wir sollten unsere Objekte Schritt für Schritt bewegen, um sicherzustellen, dass wir keine Strafpunkte erhalten. Ich kann mein 'L' auf (1, 4) bewegen, aber ich brauche deine Hilfe, um eine neue Route zu finden. Kannst du mir bitte helfen, eine Möglichkeit zu finden, mein 'L' auf die Position (2, 3) oder (2, 4) zu bringen, ohne dass ich Strafpunkte erhalte?} \\
            }
        }
    }
    & & \\ \\

    \theutterance \stepcounter{utterance}  
    & & \multicolumn{4}{p{0.6\linewidth}}{
        \cellcolor[rgb]{0.9,0.9,0.9}{
            \makecell[{{p{\linewidth}}}]{
                \texttt{\tiny{[GM$\rangle$P2]}}
                \texttt{Deine Nachricht wurde an deinen Mitspieler übermittelt.} \\
\\ 
\\ 
\texttt{Ihr seid gerade bei Runde 10 von maximal 12.} \\
\texttt{Ihr habt gemeinsam 6 von maximal 8 Strafpunkten angesammelt.} \\
\texttt{Dein Mitspieler hat dir diese Nachricht geschickt:} \\
\\ 
\\ 
\texttt{"Ich denke, es ist eine gute Idee, unsere Raster{-}Positionen noch einmal zu überprüfen und eine gemeinsame Strategie zu entwickeln. Ich schlage vor, dass wir unsere Objekte 'C', 'L' und 'P' auf die Positionen (4, 4), (3, 4) und (5, 4) bringen, wie wir ursprünglich geplant hatten. Ich denke, wir sollten unsere Objekte Schritt für Schritt bewegen, um sicherzustellen, dass wir keine Strafpunkte erhalten. Ich kann mein 'L' auf (1, 4) bewegen, aber ich brauche deine Hilfe, um eine neue Route zu finden. Kannst du mir bitte helfen, eine Möglichkeit zu finden, mein 'L' auf die Position (2, 3) oder (2, 4) zu bringen, ohne dass ich Strafpunkte erhalte?"} \\
\\ 
\\ 
\texttt{Wie lautet dein nächster Befehl?} \\
            }
        }
    }
    & \\ \\

    \theutterance \stepcounter{utterance}  
    & & & \multicolumn{4}{p{0.6\linewidth}}{
        \cellcolor[rgb]{0.9,0.9,0.9}{
            \makecell[{{p{\linewidth}}}]{
                \texttt{\tiny{[GM$\langle$P2]}}
                \texttt{SPRICH: Ich denke, ich kann dir helfen, eine Route für dein 'L' zu finden. Da dein 'L' an der Position (1, 4) ist, könntest du es nach unten auf die Position (1, 5) bewegen, wenn diese Position leer ist. Von dort aus könntest du es nach rechts auf die Position (2, 5) und dann nach oben auf die Position (2, 4) bewegen, wenn diese Positionen leer sind. Eine andere Möglichkeit wäre, das 'L' nach rechts auf die Position (2, 4) zu bewegen, wenn du sicher bist, dass diese Position leer ist. Ich denke, es wäre hilfreich, wenn du mir sagst, ob diese Positionen bei dir frei sind. Ich werde auch mein 'L' auf (1, 3) nach rechts auf die Position (2, 3) bewegen, um es näher an die Zielposition zu bringen. Lass uns gemeinsam vorgehen und unsere Objekte Schritt für Schritt bewegen, um sicherzustellen, dass wir unsere Ziele erreichen. Wie sieht es bei dir aus?} \\
            }
        }
    }
     \\ \\

    \theutterance \stepcounter{utterance}  
    & & & \multicolumn{2}{p{0.3\linewidth}}{
        \cellcolor[rgb]{0.9,0.9,0.9}{
            \makecell[{{p{\linewidth}}}]{
                \texttt{\tiny{[GM$|$GM]}}
                \texttt{SPRICH: Ich denke, ich kann dir helfen, eine Route für dein 'L' zu finden. Da dein 'L' an der Position (1, 4) ist, könntest du es nach unten auf die Position (1, 5) bewegen, wenn diese Position leer ist. Von dort aus könntest du es nach rechts auf die Position (2, 5) und dann nach oben auf die Position (2, 4) bewegen, wenn diese Positionen leer sind. Eine andere Möglichkeit wäre, das 'L' nach rechts auf die Position (2, 4) zu bewegen, wenn du sicher bist, dass diese Position leer ist. Ich denke, es wäre hilfreich, wenn du mir sagst, ob diese Positionen bei dir frei sind. Ich werde auch mein 'L' auf (1, 3) nach rechts auf die Position (2, 3) bewegen, um es näher an die Zielposition zu bringen. Lass uns gemeinsam vorgehen und unsere Objekte Schritt für Schritt bewegen, um sicherzustellen, dass wir unsere Ziele erreichen. Wie sieht es bei dir aus?} \\
            }
        }
    }
    & & \\ \\

    \theutterance \stepcounter{utterance}  
    & & \multicolumn{4}{p{0.6\linewidth}}{
        \cellcolor[rgb]{0.9,0.9,0.9}{
            \makecell[{{p{\linewidth}}}]{
                \texttt{\tiny{[P1$\langle$GM]}}
                \texttt{Deine Nachricht wurde an deinen Mitspieler übermittelt.} \\
\\ 
\\ 
\texttt{Ihr seid gerade bei Runde 10 von maximal 12.} \\
\texttt{Ihr habt gemeinsam 6 von maximal 8 Strafpunkten angesammelt.} \\
\texttt{Dein Mitspieler hat dir diese Nachricht geschickt:} \\
\\ 
\\ 
\texttt{"Ich denke, ich kann dir helfen, eine Route für dein 'L' zu finden. Da dein 'L' an der Position (1, 4) ist, könntest du es nach unten auf die Position (1, 5) bewegen, wenn diese Position leer ist. Von dort aus könntest du es nach rechts auf die Position (2, 5) und dann nach oben auf die Position (2, 4) bewegen, wenn diese Positionen leer sind. Eine andere Möglichkeit wäre, das 'L' nach rechts auf die Position (2, 4) zu bewegen, wenn du sicher bist, dass diese Position leer ist. Ich denke, es wäre hilfreich, wenn du mir sagst, ob diese Positionen bei dir frei sind. Ich werde auch mein 'L' auf (1, 3) nach rechts auf die Position (2, 3) bewegen, um es näher an die Zielposition zu bringen. Lass uns gemeinsam vorgehen und unsere Objekte Schritt für Schritt bewegen, um sicherzustellen, dass wir unsere Ziele erreichen. Wie sieht es bei dir aus?"} \\
\\ 
\\ 
\texttt{Wie lautet dein nächster Befehl?} \\
            }
        }
    }
    & \\ \\

    \theutterance \stepcounter{utterance}  
    & \multicolumn{4}{p{0.6\linewidth}}{
        \cellcolor[rgb]{0.9,0.9,0.9}{
            \makecell[{{p{\linewidth}}}]{
                \texttt{\tiny{[P1$\rangle$GM]}}
                \texttt{SPRICH: Ich danke dir für deine Hilfe! Leider ist die Position (1, 5) bei mir nicht leer, wie wir bereits früher festgestellt haben. Ich denke, ich werde versuchen, eine andere Route zu finden. Ich werde meine Raster{-}Positionen noch einmal überprüfen und sehen, ob ich eine Möglichkeit finde, mein 'L' auf die Position (2, 3) oder (2, 4) zu bewegen, ohne Strafpunkte zu erhalten. Ich werde auch meine anderen Objekte ('C' und 'P') im Auge behalten, um sicherzustellen, dass ich sie nicht blockiere, wenn ich mein 'L' bewege. Bitte teile mir mit, wenn du weitere Ideen hast oder wenn ich dir sonst noch helfen kann. Ich bin gespannt, ob wir gemeinsam eine Lösung finden können!} \\
            }
        }
    }
    & & \\ \\

    \theutterance \stepcounter{utterance}  
    & & & \multicolumn{2}{p{0.3\linewidth}}{
        \cellcolor[rgb]{0.9,0.9,0.9}{
            \makecell[{{p{\linewidth}}}]{
                \texttt{\tiny{[GM$|$GM]}}
                \texttt{SPRICH: Ich danke dir für deine Hilfe! Leider ist die Position (1, 5) bei mir nicht leer, wie wir bereits früher festgestellt haben. Ich denke, ich werde versuchen, eine andere Route zu finden. Ich werde meine Raster{-}Positionen noch einmal überprüfen und sehen, ob ich eine Möglichkeit finde, mein 'L' auf die Position (2, 3) oder (2, 4) zu bewegen, ohne Strafpunkte zu erhalten. Ich werde auch meine anderen Objekte ('C' und 'P') im Auge behalten, um sicherzustellen, dass ich sie nicht blockiere, wenn ich mein 'L' bewege. Bitte teile mir mit, wenn du weitere Ideen hast oder wenn ich dir sonst noch helfen kann. Ich bin gespannt, ob wir gemeinsam eine Lösung finden können!} \\
            }
        }
    }
    & & \\ \\

    \theutterance \stepcounter{utterance}  
    & & \multicolumn{4}{p{0.6\linewidth}}{
        \cellcolor[rgb]{0.9,0.9,0.9}{
            \makecell[{{p{\linewidth}}}]{
                \texttt{\tiny{[GM$\rangle$P2]}}
                \texttt{Deine Nachricht wurde an deinen Mitspieler übermittelt.} \\
\\ 
\\ 
\texttt{Ihr seid gerade bei Runde 11 von maximal 12.} \\
\texttt{Ihr habt gemeinsam 6 von maximal 8 Strafpunkten angesammelt.} \\
\texttt{Dein Mitspieler hat dir diese Nachricht geschickt:} \\
\\ 
\\ 
\texttt{"Ich danke dir für deine Hilfe! Leider ist die Position (1, 5) bei mir nicht leer, wie wir bereits früher festgestellt haben. Ich denke, ich werde versuchen, eine andere Route zu finden. Ich werde meine Raster{-}Positionen noch einmal überprüfen und sehen, ob ich eine Möglichkeit finde, mein 'L' auf die Position (2, 3) oder (2, 4) zu bewegen, ohne Strafpunkte zu erhalten. Ich werde auch meine anderen Objekte ('C' und 'P') im Auge behalten, um sicherzustellen, dass ich sie nicht blockiere, wenn ich mein 'L' bewege. Bitte teile mir mit, wenn du weitere Ideen hast oder wenn ich dir sonst noch helfen kann. Ich bin gespannt, ob wir gemeinsam eine Lösung finden können!"} \\
\\ 
\\ 
\texttt{Wie lautet dein nächster Befehl?} \\
            }
        }
    }
    & \\ \\

    \theutterance \stepcounter{utterance}  
    & & & \multicolumn{4}{p{0.6\linewidth}}{
        \cellcolor[rgb]{0.9,0.9,0.9}{
            \makecell[{{p{\linewidth}}}]{
                \texttt{\tiny{[GM$\langle$P2]}}
                \texttt{SPRICH: Ich denke, wir sind nah an einer Lösung! Da wir nur noch eine Runde haben, sollten wir uns auf die wichtigsten Schritte konzentrieren. Ich werde mein 'L' auf (1, 3) nach rechts auf die Position (2, 3) bewegen, um es näher an die Zielposition zu bringen. Ich denke, du solltest versuchen, dein 'L' auf eine Position zu bewegen, von der aus du es leicht auf die Zielposition (3, 4) bringen kannst. Wenn du möchtest, können wir auch versuchen, unsere anderen Objekte ('C' und 'P') auf ihre Zielpositionen zu bringen. Lass uns gemeinsam einen letzten Schritt unternehmen, um unsere Ziele zu erreichen. Wie sieht es bei dir aus? Sollen wir versuchen, unsere Objekte auf die Zielpositionen zu bringen und dann "fertig" sagen?} \\
            }
        }
    }
     \\ \\

    \theutterance \stepcounter{utterance}  
    & & & \multicolumn{2}{p{0.3\linewidth}}{
        \cellcolor[rgb]{0.9,0.9,0.9}{
            \makecell[{{p{\linewidth}}}]{
                \texttt{\tiny{[GM$|$GM]}}
                \texttt{SPRICH: Ich denke, wir sind nah an einer Lösung! Da wir nur noch eine Runde haben, sollten wir uns auf die wichtigsten Schritte konzentrieren. Ich werde mein 'L' auf (1, 3) nach rechts auf die Position (2, 3) bewegen, um es näher an die Zielposition zu bringen. Ich denke, du solltest versuchen, dein 'L' auf eine Position zu bewegen, von der aus du es leicht auf die Zielposition (3, 4) bringen kannst. Wenn du möchtest, können wir auch versuchen, unsere anderen Objekte ('C' und 'P') auf ihre Zielpositionen zu bringen. Lass uns gemeinsam einen letzten Schritt unternehmen, um unsere Ziele zu erreichen. Wie sieht es bei dir aus? Sollen wir versuchen, unsere Objekte auf die Zielpositionen zu bringen und dann "fertig" sagen?} \\
            }
        }
    }
    & & \\ \\

    \theutterance \stepcounter{utterance}  
    & & \multicolumn{4}{p{0.6\linewidth}}{
        \cellcolor[rgb]{0.9,0.9,0.9}{
            \makecell[{{p{\linewidth}}}]{
                \texttt{\tiny{[P1$\langle$GM]}}
                \texttt{Deine Nachricht wurde an deinen Mitspieler übermittelt.} \\
\\ 
\\ 
\texttt{Ihr seid gerade bei Runde 11 von maximal 12.} \\
\texttt{Ihr habt gemeinsam 6 von maximal 8 Strafpunkten angesammelt.} \\
\texttt{Dein Mitspieler hat dir diese Nachricht geschickt:} \\
\\ 
\\ 
\texttt{"Ich denke, wir sind nah an einer Lösung! Da wir nur noch eine Runde haben, sollten wir uns auf die wichtigsten Schritte konzentrieren. Ich werde mein 'L' auf (1, 3) nach rechts auf die Position (2, 3) bewegen, um es näher an die Zielposition zu bringen. Ich denke, du solltest versuchen, dein 'L' auf eine Position zu bewegen, von der aus du es leicht auf die Zielposition (3, 4) bringen kannst. Wenn du möchtest, können wir auch versuchen, unsere anderen Objekte ('C' und 'P') auf ihre Zielpositionen zu bringen. Lass uns gemeinsam einen letzten Schritt unternehmen, um unsere Ziele zu erreichen. Wie sieht es bei dir aus? Sollen wir versuchen, unsere Objekte auf die Zielpositionen zu bringen und dann "fertig" sagen?"} \\
\\ 
\\ 
\texttt{Wie lautet dein nächster Befehl?} \\
            }
        }
    }
    & \\ \\

    \theutterance \stepcounter{utterance}  
    & \multicolumn{4}{p{0.6\linewidth}}{
        \cellcolor[rgb]{0.9,0.9,0.9}{
            \makecell[{{p{\linewidth}}}]{
                \texttt{\tiny{[P1$\rangle$GM]}}
                \texttt{SPRICH: Ich denke, wir sind tatsächlich nah an einer Lösung! Ich stimme dir zu, dass wir uns auf die wichtigsten Schritte konzentrieren sollten. Ich werde versuchen, mein 'L' auf eine Position zu bewegen, von der aus ich es leicht auf die Zielposition (3, 4) bringen kann. Ich denke, wir sollten auch unsere anderen Objekte ('C' und 'P') auf ihre Zielpositionen bringen. Lass uns gemeinsam einen letzten Schritt unternehmen, um unsere Ziele zu erreichen. Ich denke, wir sollten nach unserer nächsten Aktion "fertig" sagen, um das Spiel zu beenden. Bist du bereit, das Spiel zu beenden und "fertig" zu sagen, wenn wir unsere Ziele erreicht haben?} \\
            }
        }
    }
    & & \\ \\

    \theutterance \stepcounter{utterance}  
    & & & \multicolumn{2}{p{0.3\linewidth}}{
        \cellcolor[rgb]{0.9,0.9,0.9}{
            \makecell[{{p{\linewidth}}}]{
                \texttt{\tiny{[GM$|$GM]}}
                \texttt{SPRICH: Ich denke, wir sind tatsächlich nah an einer Lösung! Ich stimme dir zu, dass wir uns auf die wichtigsten Schritte konzentrieren sollten. Ich werde versuchen, mein 'L' auf eine Position zu bewegen, von der aus ich es leicht auf die Zielposition (3, 4) bringen kann. Ich denke, wir sollten auch unsere anderen Objekte ('C' und 'P') auf ihre Zielpositionen bringen. Lass uns gemeinsam einen letzten Schritt unternehmen, um unsere Ziele zu erreichen. Ich denke, wir sollten nach unserer nächsten Aktion "fertig" sagen, um das Spiel zu beenden. Bist du bereit, das Spiel zu beenden und "fertig" zu sagen, wenn wir unsere Ziele erreicht haben?} \\
            }
        }
    }
    & & \\ \\

    \theutterance \stepcounter{utterance}  
    & & \multicolumn{4}{p{0.6\linewidth}}{
        \cellcolor[rgb]{0.9,0.9,0.9}{
            \makecell[{{p{\linewidth}}}]{
                \texttt{\tiny{[GM$\rangle$P2]}}
                \texttt{Deine Nachricht wurde an deinen Mitspieler übermittelt.} \\
\\ 
\\ 
\texttt{Ihr seid gerade bei Runde 12 von maximal 12.} \\
\texttt{Ihr habt gemeinsam 6 von maximal 8 Strafpunkten angesammelt.} \\
\texttt{Dein Mitspieler hat dir diese Nachricht geschickt:} \\
\\ 
\\ 
\texttt{"Ich denke, wir sind tatsächlich nah an einer Lösung! Ich stimme dir zu, dass wir uns auf die wichtigsten Schritte konzentrieren sollten. Ich werde versuchen, mein 'L' auf eine Position zu bewegen, von der aus ich es leicht auf die Zielposition (3, 4) bringen kann. Ich denke, wir sollten auch unsere anderen Objekte ('C' und 'P') auf ihre Zielpositionen bringen. Lass uns gemeinsam einen letzten Schritt unternehmen, um unsere Ziele zu erreichen. Ich denke, wir sollten nach unserer nächsten Aktion "fertig" sagen, um das Spiel zu beenden. Bist du bereit, das Spiel zu beenden und "fertig" zu sagen, wenn wir unsere Ziele erreicht haben?"} \\
\\ 
\\ 
\texttt{Wie lautet dein nächster Befehl?} \\
            }
        }
    }
    & \\ \\

    \theutterance \stepcounter{utterance}  
    & & & \multicolumn{4}{p{0.6\linewidth}}{
        \cellcolor[rgb]{0.9,0.9,0.9}{
            \makecell[{{p{\linewidth}}}]{
                \texttt{\tiny{[GM$\langle$P2]}}
                \texttt{SPRICH: Ja, ich bin bereit, das Spiel zu beenden und "fertig" zu sagen, wenn wir unsere Ziele erreicht haben! Ich denke, wir sollten nun unsere Objekte auf die Zielpositionen bringen und dann "fertig" sagen, um das Spiel zu beenden. Ich werde mein 'L' auf (1, 3) nach rechts auf die Position (2, 3) bewegen. Bitte sag mir, wenn du bereit bist, "fertig" zu sagen, und wir können das Spiel beenden. Ich bin gespannt, ob wir unsere Ziele erreicht haben!} \\
            }
        }
    }
     \\ \\

    \theutterance \stepcounter{utterance}  
    & & & \multicolumn{2}{p{0.3\linewidth}}{
        \cellcolor[rgb]{0.9,0.9,0.9}{
            \makecell[{{p{\linewidth}}}]{
                \texttt{\tiny{[GM$|$GM]}}
                \texttt{SPRICH: Ja, ich bin bereit, das Spiel zu beenden und "fertig" zu sagen, wenn wir unsere Ziele erreicht haben! Ich denke, wir sollten nun unsere Objekte auf die Zielpositionen bringen und dann "fertig" sagen, um das Spiel zu beenden. Ich werde mein 'L' auf (1, 3) nach rechts auf die Position (2, 3) bewegen. Bitte sag mir, wenn du bereit bist, "fertig" zu sagen, und wir können das Spiel beenden. Ich bin gespannt, ob wir unsere Ziele erreicht haben!} \\
            }
        }
    }
    & & \\ \\

    \theutterance \stepcounter{utterance}  
    & & & \multicolumn{2}{p{0.3\linewidth}}{
        \cellcolor[rgb]{0.9,0.9,0.9}{
            \makecell[{{p{\linewidth}}}]{
                \texttt{\tiny{[GM$|$GM]}}
                \texttt{Maximum number of rounds reached} \\
            }
        }
    }
    & & \\ \\

    \theutterance \stepcounter{utterance}  
    & & & \multicolumn{2}{p{0.3\linewidth}}{
        \cellcolor[rgb]{0.9,0.9,0.9}{
            \makecell[{{p{\linewidth}}}]{
                \texttt{\tiny{[GM$|$GM]}}
                \texttt{* success: True} \\
\texttt{* lose: False} \\
\texttt{* aborted: False} \\
\texttt{{-}{-}{-}{-}{-}{-}{-}} \\
\texttt{* Shifts: 0.00} \\
\texttt{* Max Shifts: 4.00} \\
\texttt{* Min Shifts: 2.00} \\
\texttt{* End Distance Sum: 6.24} \\
\texttt{* Init Distance Sum: 8.24} \\
\texttt{* Expected Distance Sum: 12.57} \\
\texttt{* Penalties: 6.00} \\
\texttt{* Max Penalties: 8.00} \\
\texttt{* Rounds: 12.00} \\
\texttt{* Max Rounds: 12.00} \\
\texttt{* Object Count: 3.00} \\
            }
        }
    }
    & & \\ \\

    \theutterance \stepcounter{utterance}  
    & & & \multicolumn{2}{p{0.3\linewidth}}{
        \cellcolor[rgb]{0.9,0.9,0.9}{
            \makecell[{{p{\linewidth}}}]{
                \texttt{\tiny{[GM$|$GM]}}
                \texttt{Initial Grids:} \\
\texttt{```} \\
\texttt{Player 1:     Player 2:} \\
\texttt{ 1234567       1234567} \\
\texttt{╔═╤═╤═╤═╗     ╔═╤═╤═╤═╗} \\
\texttt{║◌│◌│◌│◌║ 1   ║L│◌│◌│◌║ 1} \\
\texttt{╟─┼─┤◌│◌║ 2   ╟─┼─┤◌│◌║ 2} \\
\texttt{║◌│◌│◌│◌║ 3   ║◌│◌│◌│◌║ 3} \\
\texttt{║L│◌├─┴─╢ 4   ║◌│◌├─┴─╢ 4} \\
\texttt{╟─┴─┤◌◌◌║ 5   ╟─┴─┤C◌◌║ 5} \\
\texttt{║◌◌C└─┬─╢ 6   ║◌◌◌└─┬─╢ 6} \\
\texttt{║◌◌◌◌P│◌║ 7   ║◌P◌◌◌│◌║ 7} \\
\texttt{╚═════╧═╝     ╚═════╧═╝} \\
\texttt{```} \\
\\ 
\texttt{Final Grids:} \\
\texttt{```} \\
\texttt{Player 1:     Player 2:} \\
\texttt{ 1234567       1234567} \\
\texttt{╔═╤═╤═╤═╗     ╔═╤═╤═╤═╗} \\
\texttt{║◌│◌│◌│◌║ 1   ║◌│◌│◌│◌║ 1} \\
\texttt{╟─┼─┤◌│◌║ 2   ╟─┼─┤◌│◌║ 2} \\
\texttt{║◌│◌│◌│◌║ 3   ║L│◌│◌│◌║ 3} \\
\texttt{║L│◌├─┴─╢ 4   ║◌│◌├─┴─╢ 4} \\
\texttt{╟─┴─┤◌◌◌║ 5   ╟─┴─┤C◌◌║ 5} \\
\texttt{║◌◌C└─┬─╢ 6   ║◌◌◌└─┬─╢ 6} \\
\texttt{║◌◌◌◌P│◌║ 7   ║◌P◌◌◌│◌║ 7} \\
\texttt{╚═════╧═╝     ╚═════╧═╝} \\
\texttt{```} \\
            }
        }
    }
    & & \\ \\

\end{supertabular}
}

\end{document}
