
\documentclass{article}
\usepackage{colortbl}
\usepackage{makecell}
\usepackage{multirow}
\usepackage{supertabular}

\begin{document}

\newcounter{utterance}

\centering \large Interaction Transcript for game `clean\_up', experiment `0\_easy\_5obj\_de', episode 1 with llama{-}3.3{-}70b{-}instruct{-}t1.0.
\vspace{24pt}

{ \footnotesize  \setcounter{utterance}{1}
\setlength{\tabcolsep}{0pt}
\begin{supertabular}{c@{$\;$}|p{.15\linewidth}@{}p{.15\linewidth}p{.15\linewidth}p{.15\linewidth}p{.15\linewidth}p{.15\linewidth}}
    \# & $\;$A & \multicolumn{4}{c}{Game Master} & $\;\:$B\\
    \hline

    \theutterance \stepcounter{utterance}  
    & & \multicolumn{4}{p{0.6\linewidth}}{
        \cellcolor[rgb]{0.9,0.9,0.9}{
            \makecell[{{p{\linewidth}}}]{
                \texttt{\tiny{[P1$\langle$GM]}}
                \texttt{Ich bin der Spielleiter, und du spielst ein kooperatives Spiel mit dem folgenden Raster als Spielbrett:} \\
\\ 
\texttt{```} \\
\texttt{ 1234567} \\
\texttt{╔═══════╗} \\
\texttt{║I◌◌◌◌◌◌║ 1} \\
\texttt{║◌◌◌◌◌◌◌║ 2} \\
\texttt{║◌◌◌T◌┌─╢ 3} \\
\texttt{║◌◌◌┌─┤◌║ 4} \\
\texttt{║◌◌W│C├─╢ 5} \\
\texttt{╟───┼─┘H║ 6} \\
\texttt{║◌◌◌│◌◌◌║ 7} \\
\texttt{╚═══╧═══╝} \\
\\ 
\texttt{```} \\
\\ 
\texttt{* Am oberen Rand sind die x{-}Koordinaten angegeben, die nach rechts hin zunehmen, und am rechten Rand die y{-}Koordinaten, die nach unten zunehmen.} \\
\texttt{* Die folgenden Objekte sind zufällig auf deinem Raster platziert: 'W', 'I', 'T', 'C', 'H'.} \\
\\ 
\\ 
\texttt{Dein Mitspieler sieht eine Variation des Spielbretts, auf dem dieselben Objekte an anderen zufälligen Positionen platziert sind. Du kannst das Spielbrett des anderen Spielers nicht sehen, und er kann deins nicht sehen.} \\
\\ 
\\ 
\texttt{**Ziel des Spiels:**} \\
\\ 
\texttt{Beide Spieler müssen die Objekte auf ihrem jeweiligen Spielbrett so bewegen, dass identische Objekte am Ende an denselben Koordinaten liegen. Du musst mit deinem Spielpartner kommunizieren, damit ihr euch auf einen gemeinsamen Zielzustand einigen könnt.} \\
\\ 
\\ 
\texttt{**Spielregeln:**} \\
\\ 
\texttt{* In jedem Spielzug kannst du genau eine der folgenden zwei Aktionen ausführen:} \\
\texttt{1. `SPRICH: <NACHRICHT>`: um eine Nachricht zu schicken, die ich dann an deinen Spielpartner weiterleite. Als Nachricht gilt alles bis zum nächsten Zeilenumbruch.} \\
\texttt{2. `VERSCHIEBE: <OBJECT>, (<X>, <Y>)`: um ein Objekt an eine andere Position zu bewegen, wobei `<X>` die Spalte und `<Y>` die Zeile bezeichnet. Ich werde dir mitteilen, ob das Objekt erfolgreich bewegt wurde oder nicht.} \\
\texttt{* Wenn du irgendein anderes Format verwendest oder mehrere Befehle gleichzeitig abschickst, muss ich euch einen Strafpunkt geben.} \\
\texttt{* Wenn ihr gemeinsam mehr als 12 Strafpunkte ansammelt, verliert ihr das Spiel.} \\
\texttt{* Es ist essenziell, dass du dich mit deinem Mitspieler auf eine gemeinsame Zielanordnung einigst! Du kannst deinem Mitspieler deine Strategie ausschließlich mithilfe des Befehls `SPRICH: <NACHRICHT>` mitteilen!} \\
\\ 
\\ 
\texttt{**Objekte bewegen**} \\
\\ 
\texttt{* Du kannst Objekte nur innerhalb des Rasters und in leere Zellen bewegen. Leere Zellen erkennst am Zeichen '◌'.} \\
\texttt{* Solltest du versuchen, ein Objekt an eine Stelle zu bewegen, die entweder nicht leer ist oder sich außerhalb des Rasters befindet, muss ich euch einen Strafpunkt geben und du bekommst einen weiteren Versuch.} \\
\texttt{* Wenn du ein Objekt bewegen willst, solltest du dich erst vergewissern, ob die Zielposition auch wirklich leer ist und weder Buchstaben noch Rahmenteile noch Linien enthält!} \\
\\ 
\\ 
\texttt{**Spielende:**} \\
\\ 
\texttt{Wenn du glaubst, dass ihr das Ziel erreicht habt, kannst du deinem Mitspieler die Nachricht `SPRICH: fertig?` schicken. Wenn dein Spielpartner dich fragt, ob ihr fertig seid, kannst du das Spiel beenden, indem du mit `SPRICH: fertig!` antwortest.} \\
\\ 
\texttt{Ihr gewinnt gemeinsam, wenn ihr das Spiel innerhalb von \#max\_rounds Runden beendet. Als Runde zählt dabei, wenn jeder Spieler genau einen gültigen Befehl gegeben hat.} \\
\\ 
\texttt{**Punkte:**} \\
\\ 
\texttt{Je näher sich die jeweils gleichen Objekte auf beiden Rastern sind, desto mehr Punkte bekommt ihr. Für Strafpunkte gibt es Punktabzug. Könnt ihr den Rekord brechen?} \\
\\ 
\texttt{Schicke eine Nachricht an deinen Mitspieler um das Spiel zu starten!} \\
            }
        }
    }
    & \\ \\

    \theutterance \stepcounter{utterance}  
    & \multicolumn{4}{p{0.6\linewidth}}{
        \cellcolor[rgb]{0.9,0.9,0.9}{
            \makecell[{{p{\linewidth}}}]{
                \texttt{\tiny{[P1$\rangle$GM]}}
                \texttt{SPRICH: Hallo, ich schlage vor, dass wir uns auf eine Strategie einigen, um die Objekte anhand ihrer Anfangspositionen zu identifizieren und zu bewegen. Kannst du mir bitte mitteilen, wo auf deinem Brett die Objekte 'W', 'I', 'T', 'C' und 'H' platziert sind, indem du ihre x{-} und y{-}Koordinaten angibst?} \\
            }
        }
    }
    & & \\ \\

    \theutterance \stepcounter{utterance}  
    & & & \multicolumn{2}{p{0.3\linewidth}}{
        \cellcolor[rgb]{0.9,0.9,0.9}{
            \makecell[{{p{\linewidth}}}]{
                \texttt{\tiny{[GM$|$GM]}}
                \texttt{SPRICH: Hallo, ich schlage vor, dass wir uns auf eine Strategie einigen, um die Objekte anhand ihrer Anfangspositionen zu identifizieren und zu bewegen. Kannst du mir bitte mitteilen, wo auf deinem Brett die Objekte 'W', 'I', 'T', 'C' und 'H' platziert sind, indem du ihre x{-} und y{-}Koordinaten angibst?} \\
            }
        }
    }
    & & \\ \\

    \theutterance \stepcounter{utterance}  
    & & \multicolumn{4}{p{0.6\linewidth}}{
        \cellcolor[rgb]{0.9,0.9,0.9}{
            \makecell[{{p{\linewidth}}}]{
                \texttt{\tiny{[GM$\rangle$P2]}}
                \texttt{Ich bin der Spielleiter, und du spielst ein kooperatives Spiel mit dem folgenden Raster als Spielbrett:} \\
\\ 
\texttt{```} \\
\texttt{ 1234567} \\
\texttt{╔═══════╗} \\
\texttt{║◌◌H◌◌◌◌║ 1} \\
\texttt{║◌◌◌C◌◌◌║ 2} \\
\texttt{║◌◌◌◌◌┌─╢ 3} \\
\texttt{║◌◌◌┌─┤◌║ 4} \\
\texttt{║◌IW│◌├─╢ 5} \\
\texttt{╟───┼─┘◌║ 6} \\
\texttt{║T◌◌│◌◌◌║ 7} \\
\texttt{╚═══╧═══╝} \\
\\ 
\texttt{```} \\
\\ 
\texttt{* Am oberen Rand sind die x{-}Koordinaten angegeben, die nach rechts hin zunehmen, und am rechten Rand die y{-}Koordinaten, die nach unten zunehmen.} \\
\texttt{* Die folgenden Objekte sind zufällig auf deinem Raster platziert: 'W', 'I', 'T', 'C', 'H'.} \\
\\ 
\\ 
\texttt{Dein Mitspieler sieht eine Variation des Spielbretts, auf dem dieselben Objekte an anderen zufälligen Positionen platziert sind. Du kannst das Spielbrett des anderen Spielers nicht sehen, und er kann deins nicht sehen.} \\
\\ 
\\ 
\texttt{**Ziel des Spiels:**} \\
\\ 
\texttt{Beide Spieler müssen die Objekte auf ihrem jeweiligen Spielbrett so bewegen, dass identische Objekte am Ende an denselben Koordinaten liegen. Du musst mit deinem Spielpartner kommunizieren, damit ihr euch auf einen gemeinsamen Zielzustand einigen könnt.} \\
\\ 
\\ 
\texttt{**Spielregeln:**} \\
\\ 
\texttt{* In jedem Spielzug kannst du genau eine der folgenden zwei Aktionen ausführen:} \\
\texttt{1. `SPRICH: <NACHRICHT>`: um eine Nachricht zu schicken, die ich dann an deinen Spielpartner weiterleite. Als Nachricht gilt alles bis zum nächsten Zeilenumbruch.} \\
\texttt{2. `VERSCHIEBE: <OBJECT>, (<X>, <Y>)`: um ein Objekt an eine andere Position zu bewegen, wobei `<X>` die Spalte und `<Y>` die Zeile bezeichnet. Ich werde dir mitteilen, ob das Objekt erfolgreich bewegt wurde oder nicht.} \\
\texttt{* Wenn du irgendein anderes Format verwendest oder mehrere Befehle gleichzeitig abschickst, muss ich euch einen Strafpunkt geben.} \\
\texttt{* Wenn ihr gemeinsam mehr als 12 Strafpunkte ansammelt, verliert ihr das Spiel.} \\
\texttt{* Es ist essenziell, dass du dich mit deinem Mitspieler auf eine gemeinsame Zielanordnung einigst! Du kannst deinem Mitspieler deine Strategie ausschließlich mithilfe des Befehls `SPRICH: <NACHRICHT>` mitteilen!} \\
\\ 
\\ 
\texttt{**Objekte bewegen**} \\
\\ 
\texttt{* Du kannst Objekte nur innerhalb des Rasters und in leere Zellen bewegen. Leere Zellen erkennst am Zeichen '◌'.} \\
\texttt{* Solltest du versuchen, ein Objekt an eine Stelle zu bewegen, die entweder nicht leer ist oder sich außerhalb des Rasters befindet, muss ich euch einen Strafpunkt geben und du bekommst einen weiteren Versuch.} \\
\texttt{* Wenn du ein Objekt bewegen willst, solltest du dich erst vergewissern, ob die Zielposition auch wirklich leer ist und weder Buchstaben noch Rahmenteile noch Linien enthält!} \\
\\ 
\\ 
\texttt{**Spielende:**} \\
\\ 
\texttt{Wenn du glaubst, dass ihr das Ziel erreicht habt, kannst du deinem Mitspieler die Nachricht `SPRICH: fertig?` schicken. Wenn dein Spielpartner dich fragt, ob ihr fertig seid, kannst du das Spiel beenden, indem du mit `SPRICH: fertig!` antwortest.} \\
\\ 
\texttt{Ihr gewinnt gemeinsam, wenn ihr das Spiel innerhalb von \#max\_rounds Runden beendet. Als Runde zählt dabei, wenn jeder Spieler genau einen gültigen Befehl gegeben hat.} \\
\\ 
\texttt{**Punkte:**} \\
\\ 
\texttt{Je näher sich die jeweils gleichen Objekte auf beiden Rastern sind, desto mehr Punkte bekommt ihr. Für Strafpunkte gibt es Punktabzug. Könnt ihr den Rekord brechen?} \\
\\ 
\texttt{Dein Mitspieler hat das Spiel mit dieser Nachricht begonnen:} \\
\\ 
\texttt{"Hallo, ich schlage vor, dass wir uns auf eine Strategie einigen, um die Objekte anhand ihrer Anfangspositionen zu identifizieren und zu bewegen. Kannst du mir bitte mitteilen, wo auf deinem Brett die Objekte 'W', 'I', 'T', 'C' und 'H' platziert sind, indem du ihre x{-} und y{-}Koordinaten angibst?"} \\
\\ 
\texttt{Wie lautet dein erster Befehl?} \\
            }
        }
    }
    & \\ \\

    \theutterance \stepcounter{utterance}  
    & & & \multicolumn{4}{p{0.6\linewidth}}{
        \cellcolor[rgb]{0.9,0.9,0.9}{
            \makecell[{{p{\linewidth}}}]{
                \texttt{\tiny{[GM$\langle$P2]}}
                \texttt{SPRICH: Hallo! Auf meinem Brett sind die Objekte wie folgt platziert: 'W' ist nicht sichtbar, 'I' bei (1,5), 'T' bei (1,7), 'C' bei (4,2), 'H' bei (3,1). Ich bin bereit, eine Strategie zu entwickeln und die Objekte zu bewegen. Wie schlägst du vor, dass wir vorgehen?} \\
            }
        }
    }
     \\ \\

    \theutterance \stepcounter{utterance}  
    & & & \multicolumn{2}{p{0.3\linewidth}}{
        \cellcolor[rgb]{0.9,0.9,0.9}{
            \makecell[{{p{\linewidth}}}]{
                \texttt{\tiny{[GM$|$GM]}}
                \texttt{SPRICH: Hallo! Auf meinem Brett sind die Objekte wie folgt platziert: 'W' ist nicht sichtbar, 'I' bei (1,5), 'T' bei (1,7), 'C' bei (4,2), 'H' bei (3,1). Ich bin bereit, eine Strategie zu entwickeln und die Objekte zu bewegen. Wie schlägst du vor, dass wir vorgehen?} \\
            }
        }
    }
    & & \\ \\

    \theutterance \stepcounter{utterance}  
    & & \multicolumn{4}{p{0.6\linewidth}}{
        \cellcolor[rgb]{0.9,0.9,0.9}{
            \makecell[{{p{\linewidth}}}]{
                \texttt{\tiny{[P1$\langle$GM]}}
                \texttt{Deine Nachricht wurde an deinen Mitspieler übermittelt.} \\
\\ 
\\ 
\texttt{Ihr seid gerade bei Runde 1 von maximal 20.} \\
\texttt{Ihr habt gemeinsam 0 von maximal 12 Strafpunkten angesammelt.} \\
\texttt{Dein Mitspieler hat dir diese Nachricht geschickt:} \\
\\ 
\\ 
\texttt{"Hallo! Auf meinem Brett sind die Objekte wie folgt platziert: 'W' ist nicht sichtbar, 'I' bei (1,5), 'T' bei (1,7), 'C' bei (4,2), 'H' bei (3,1). Ich bin bereit, eine Strategie zu entwickeln und die Objekte zu bewegen. Wie schlägst du vor, dass wir vorgehen?"} \\
\\ 
\\ 
\texttt{Wie lautet dein nächster Befehl?} \\
            }
        }
    }
    & \\ \\

    \theutterance \stepcounter{utterance}  
    & \multicolumn{4}{p{0.6\linewidth}}{
        \cellcolor[rgb]{0.9,0.9,0.9}{
            \makecell[{{p{\linewidth}}}]{
                \texttt{\tiny{[P1$\rangle$GM]}}
                \texttt{SPRICH: Danke für die Informationen! Auf meinem Brett sind die Objekte wie folgt platziert: 'W' bei (5,5), 'I' bei (1,1), 'T' bei (4,3), 'C' bei (5,5) und 'H' bei (6,5). Ich denke, wir sollten versuchen, die Objekte anhand ihrer Anfangspositionen zu identifizieren und dann einen gemeinsamen Zielzustand zu definieren. Wie wäre es, wenn wir die Objekte in der Reihenfolge 'I', 'W', 'T', 'C', 'H' von links nach rechts und von oben nach unten anordnen? Wir könnten auch eine feste Position für jedes Objekt festlegen, um die Kommunikation zu erleichtern. Was meinst du?} \\
            }
        }
    }
    & & \\ \\

    \theutterance \stepcounter{utterance}  
    & & & \multicolumn{2}{p{0.3\linewidth}}{
        \cellcolor[rgb]{0.9,0.9,0.9}{
            \makecell[{{p{\linewidth}}}]{
                \texttt{\tiny{[GM$|$GM]}}
                \texttt{SPRICH: Danke für die Informationen! Auf meinem Brett sind die Objekte wie folgt platziert: 'W' bei (5,5), 'I' bei (1,1), 'T' bei (4,3), 'C' bei (5,5) und 'H' bei (6,5). Ich denke, wir sollten versuchen, die Objekte anhand ihrer Anfangspositionen zu identifizieren und dann einen gemeinsamen Zielzustand zu definieren. Wie wäre es, wenn wir die Objekte in der Reihenfolge 'I', 'W', 'T', 'C', 'H' von links nach rechts und von oben nach unten anordnen? Wir könnten auch eine feste Position für jedes Objekt festlegen, um die Kommunikation zu erleichtern. Was meinst du?} \\
            }
        }
    }
    & & \\ \\

    \theutterance \stepcounter{utterance}  
    & & \multicolumn{4}{p{0.6\linewidth}}{
        \cellcolor[rgb]{0.9,0.9,0.9}{
            \makecell[{{p{\linewidth}}}]{
                \texttt{\tiny{[GM$\rangle$P2]}}
                \texttt{Deine Nachricht wurde an deinen Mitspieler übermittelt.} \\
\\ 
\\ 
\texttt{Ihr seid gerade bei Runde 2 von maximal 20.} \\
\texttt{Ihr habt gemeinsam 0 von maximal 12 Strafpunkten angesammelt.} \\
\texttt{Dein Mitspieler hat dir diese Nachricht geschickt:} \\
\\ 
\\ 
\texttt{"Danke für die Informationen! Auf meinem Brett sind die Objekte wie folgt platziert: 'W' bei (5,5), 'I' bei (1,1), 'T' bei (4,3), 'C' bei (5,5) und 'H' bei (6,5). Ich denke, wir sollten versuchen, die Objekte anhand ihrer Anfangspositionen zu identifizieren und dann einen gemeinsamen Zielzustand zu definieren. Wie wäre es, wenn wir die Objekte in der Reihenfolge 'I', 'W', 'T', 'C', 'H' von links nach rechts und von oben nach unten anordnen? Wir könnten auch eine feste Position für jedes Objekt festlegen, um die Kommunikation zu erleichtern. Was meinst du?"} \\
\\ 
\\ 
\texttt{Wie lautet dein nächster Befehl?} \\
            }
        }
    }
    & \\ \\

    \theutterance \stepcounter{utterance}  
    & & & \multicolumn{4}{p{0.6\linewidth}}{
        \cellcolor[rgb]{0.9,0.9,0.9}{
            \makecell[{{p{\linewidth}}}]{
                \texttt{\tiny{[GM$\langle$P2]}}
                \texttt{SPRICH: Ich denke, das ist eine gute Idee! Die Anordnung 'I', 'W', 'T', 'C', 'H' von links nach rechts und von oben nach unten klingt nach einem guten Ansatz. Um die Kommunikation zu erleichtern, schlage ich vor, dass wir eine feste Position für jedes Objekt festlegen. Wie wäre es, wenn wir die Positionsliste wie folgt definieren: } \\
\texttt{{-} 'I' bei (1,1)} \\
\texttt{{-} 'W' bei (2,1)} \\
\texttt{{-} 'T' bei (3,1)} \\
\texttt{{-} 'C' bei (4,1)} \\
\texttt{{-} 'H' bei (5,1)} \\
\texttt{Das würde es uns ermöglichen, leichter zu kommunizieren und die Objekte zu bewegen. Was meinst du? Sollten wir damit beginnen, die Objekte in diese Positionen zu bewegen?} \\
            }
        }
    }
     \\ \\

    \theutterance \stepcounter{utterance}  
    & & & \multicolumn{2}{p{0.3\linewidth}}{
        \cellcolor[rgb]{0.9,0.9,0.9}{
            \makecell[{{p{\linewidth}}}]{
                \texttt{\tiny{[GM$|$GM]}}
                \texttt{SPRICH: Ich denke, das ist eine gute Idee! Die Anordnung 'I', 'W', 'T', 'C', 'H' von links nach rechts und von oben nach unten klingt nach einem guten Ansatz. Um die Kommunikation zu erleichtern, schlage ich vor, dass wir eine feste Position für jedes Objekt festlegen. Wie wäre es, wenn wir die Positionsliste wie folgt definieren: } \\
\texttt{{-} 'I' bei (1,1)} \\
\texttt{{-} 'W' bei (2,1)} \\
\texttt{{-} 'T' bei (3,1)} \\
\texttt{{-} 'C' bei (4,1)} \\
\texttt{{-} 'H' bei (5,1)} \\
\texttt{Das würde es uns ermöglichen, leichter zu kommunizieren und die Objekte zu bewegen. Was meinst du? Sollten wir damit beginnen, die Objekte in diese Positionen zu bewegen?} \\
            }
        }
    }
    & & \\ \\

    \theutterance \stepcounter{utterance}  
    & & & \multicolumn{2}{p{0.3\linewidth}}{
        \cellcolor[rgb]{0.9,0.9,0.9}{
            \makecell[{{p{\linewidth}}}]{
                \texttt{\tiny{[GM$|$GM]}}
                \texttt{Invalid format: tail is not empty: '} \\
\texttt{{-} 'I' bei (1,1)} \\
\texttt{{-} 'W' bei (2,1)} \\
\texttt{{-} 'T' bei (3,1)} \\
\texttt{{-} 'C' bei (4,1)} \\
\texttt{{-} 'H' bei (5,1)} \\
\texttt{Das würde es uns ermöglichen, leichter zu kommunizieren und die Objekte zu bewegen. Was meinst du? Sollten wir damit beginnen, die Objekte in diese Positionen zu bewegen?} \\
            }
        }
    }
    & & \\ \\

    \theutterance \stepcounter{utterance}  
    & & \multicolumn{4}{p{0.6\linewidth}}{
        \cellcolor[rgb]{0.9,0.9,0.9}{
            \makecell[{{p{\linewidth}}}]{
                \texttt{\tiny{[GM$\rangle$P2]}}
                \texttt{Strafpunkt: Deine Nachricht darf nichts nach dem Befehl enthalten! Stelle sicher, dass deine Antwort ausschließlich einen der beiden Befehle `SPRICH: <NACHRICHT>` und `VERSCHIEBE: <OBJECT>, (<X>, <Y>)` und sonst nichts enthält!} \\
\\ 
\texttt{Ihr habt gemeinsam 1 von maximal 12 Strafpunkten angesammelt.} \\
\texttt{Bitte versuche es noch mal!} \\
            }
        }
    }
    & \\ \\

    \theutterance \stepcounter{utterance}  
    & & & \multicolumn{4}{p{0.6\linewidth}}{
        \cellcolor[rgb]{0.9,0.9,0.9}{
            \makecell[{{p{\linewidth}}}]{
                \texttt{\tiny{[GM$\langle$P2]}}
                \texttt{SPRICH: Ich denke, das ist eine gute Idee! Die Anordnung 'I', 'W', 'T', 'C', 'H' von links nach rechts und von oben nach unten klingt nach einem guten Ansatz. Um die Kommunikation zu erleichtern, schlage ich vor, dass wir eine feste Position für jedes Objekt festlegen: 'I' bei (1,1), 'W' bei (2,1), 'T' bei (3,1), 'C' bei (4,1), 'H' bei (5,1)} \\
            }
        }
    }
     \\ \\

    \theutterance \stepcounter{utterance}  
    & & & \multicolumn{2}{p{0.3\linewidth}}{
        \cellcolor[rgb]{0.9,0.9,0.9}{
            \makecell[{{p{\linewidth}}}]{
                \texttt{\tiny{[GM$|$GM]}}
                \texttt{SPRICH: Ich denke, das ist eine gute Idee! Die Anordnung 'I', 'W', 'T', 'C', 'H' von links nach rechts und von oben nach unten klingt nach einem guten Ansatz. Um die Kommunikation zu erleichtern, schlage ich vor, dass wir eine feste Position für jedes Objekt festlegen: 'I' bei (1,1), 'W' bei (2,1), 'T' bei (3,1), 'C' bei (4,1), 'H' bei (5,1)} \\
            }
        }
    }
    & & \\ \\

    \theutterance \stepcounter{utterance}  
    & & \multicolumn{4}{p{0.6\linewidth}}{
        \cellcolor[rgb]{0.9,0.9,0.9}{
            \makecell[{{p{\linewidth}}}]{
                \texttt{\tiny{[P1$\langle$GM]}}
                \texttt{Deine Nachricht wurde an deinen Mitspieler übermittelt.} \\
\\ 
\\ 
\texttt{Ihr seid gerade bei Runde 2 von maximal 20.} \\
\texttt{Ihr habt gemeinsam 1 von maximal 12 Strafpunkten angesammelt.} \\
\texttt{Dein Mitspieler hat dir diese Nachricht geschickt:} \\
\\ 
\\ 
\texttt{"Ich denke, das ist eine gute Idee! Die Anordnung 'I', 'W', 'T', 'C', 'H' von links nach rechts und von oben nach unten klingt nach einem guten Ansatz. Um die Kommunikation zu erleichtern, schlage ich vor, dass wir eine feste Position für jedes Objekt festlegen: 'I' bei (1,1), 'W' bei (2,1), 'T' bei (3,1), 'C' bei (4,1), 'H' bei (5,1)"} \\
\\ 
\\ 
\texttt{Wie lautet dein nächster Befehl?} \\
            }
        }
    }
    & \\ \\

    \theutterance \stepcounter{utterance}  
    & \multicolumn{4}{p{0.6\linewidth}}{
        \cellcolor[rgb]{0.9,0.9,0.9}{
            \makecell[{{p{\linewidth}}}]{
                \texttt{\tiny{[P1$\rangle$GM]}}
                \texttt{SPRICH: Das klingt nach einem großartigen Plan! Ich stimme der Anordnung und den festen Positionen für jedes Objekt zu. Bevor wir beginnen, die Objekte zu bewegen, möchte ich mich vergewissern, dass wir beide das gleiche Verständnis davon haben, wie das Zielbrett aussehen soll. Kannst du mir bitte bestätigen, dass wir uns auf die Anordnung 'I' bei (1,1), 'W' bei (2,1), 'T' bei (3,1), 'C' bei (4,1), 'H' bei (5,1) geeinigt haben? Wenn ja, können wir beginnen, die Objekte zu bewegen, um dieses Ziel zu erreichen.} \\
            }
        }
    }
    & & \\ \\

    \theutterance \stepcounter{utterance}  
    & & & \multicolumn{2}{p{0.3\linewidth}}{
        \cellcolor[rgb]{0.9,0.9,0.9}{
            \makecell[{{p{\linewidth}}}]{
                \texttt{\tiny{[GM$|$GM]}}
                \texttt{SPRICH: Das klingt nach einem großartigen Plan! Ich stimme der Anordnung und den festen Positionen für jedes Objekt zu. Bevor wir beginnen, die Objekte zu bewegen, möchte ich mich vergewissern, dass wir beide das gleiche Verständnis davon haben, wie das Zielbrett aussehen soll. Kannst du mir bitte bestätigen, dass wir uns auf die Anordnung 'I' bei (1,1), 'W' bei (2,1), 'T' bei (3,1), 'C' bei (4,1), 'H' bei (5,1) geeinigt haben? Wenn ja, können wir beginnen, die Objekte zu bewegen, um dieses Ziel zu erreichen.} \\
            }
        }
    }
    & & \\ \\

    \theutterance \stepcounter{utterance}  
    & & \multicolumn{4}{p{0.6\linewidth}}{
        \cellcolor[rgb]{0.9,0.9,0.9}{
            \makecell[{{p{\linewidth}}}]{
                \texttt{\tiny{[GM$\rangle$P2]}}
                \texttt{Deine Nachricht wurde an deinen Mitspieler übermittelt.} \\
\\ 
\\ 
\texttt{Ihr seid gerade bei Runde 3 von maximal 20.} \\
\texttt{Ihr habt gemeinsam 1 von maximal 12 Strafpunkten angesammelt.} \\
\texttt{Dein Mitspieler hat dir diese Nachricht geschickt:} \\
\\ 
\\ 
\texttt{"Das klingt nach einem großartigen Plan! Ich stimme der Anordnung und den festen Positionen für jedes Objekt zu. Bevor wir beginnen, die Objekte zu bewegen, möchte ich mich vergewissern, dass wir beide das gleiche Verständnis davon haben, wie das Zielbrett aussehen soll. Kannst du mir bitte bestätigen, dass wir uns auf die Anordnung 'I' bei (1,1), 'W' bei (2,1), 'T' bei (3,1), 'C' bei (4,1), 'H' bei (5,1) geeinigt haben? Wenn ja, können wir beginnen, die Objekte zu bewegen, um dieses Ziel zu erreichen."} \\
\\ 
\\ 
\texttt{Wie lautet dein nächster Befehl?} \\
            }
        }
    }
    & \\ \\

    \theutterance \stepcounter{utterance}  
    & & & \multicolumn{4}{p{0.6\linewidth}}{
        \cellcolor[rgb]{0.9,0.9,0.9}{
            \makecell[{{p{\linewidth}}}]{
                \texttt{\tiny{[GM$\langle$P2]}}
                \texttt{SPRICH: Ja, das ist korrekt, wir haben uns auf die Anordnung 'I' bei (1,1), 'W' bei (2,1), 'T' bei (3,1), 'C' bei (4,1), 'H' bei (5,1) geeinigt.} \\
            }
        }
    }
     \\ \\

    \theutterance \stepcounter{utterance}  
    & & & \multicolumn{2}{p{0.3\linewidth}}{
        \cellcolor[rgb]{0.9,0.9,0.9}{
            \makecell[{{p{\linewidth}}}]{
                \texttt{\tiny{[GM$|$GM]}}
                \texttt{SPRICH: Ja, das ist korrekt, wir haben uns auf die Anordnung 'I' bei (1,1), 'W' bei (2,1), 'T' bei (3,1), 'C' bei (4,1), 'H' bei (5,1) geeinigt.} \\
            }
        }
    }
    & & \\ \\

    \theutterance \stepcounter{utterance}  
    & & \multicolumn{4}{p{0.6\linewidth}}{
        \cellcolor[rgb]{0.9,0.9,0.9}{
            \makecell[{{p{\linewidth}}}]{
                \texttt{\tiny{[P1$\langle$GM]}}
                \texttt{Deine Nachricht wurde an deinen Mitspieler übermittelt.} \\
\\ 
\\ 
\texttt{Ihr seid gerade bei Runde 3 von maximal 20.} \\
\texttt{Ihr habt gemeinsam 1 von maximal 12 Strafpunkten angesammelt.} \\
\texttt{Dein Mitspieler hat dir diese Nachricht geschickt:} \\
\\ 
\\ 
\texttt{"Ja, das ist korrekt, wir haben uns auf die Anordnung 'I' bei (1,1), 'W' bei (2,1), 'T' bei (3,1), 'C' bei (4,1), 'H' bei (5,1) geeinigt."} \\
\\ 
\\ 
\texttt{Wie lautet dein nächster Befehl?} \\
            }
        }
    }
    & \\ \\

    \theutterance \stepcounter{utterance}  
    & \multicolumn{4}{p{0.6\linewidth}}{
        \cellcolor[rgb]{0.9,0.9,0.9}{
            \makecell[{{p{\linewidth}}}]{
                \texttt{\tiny{[P1$\rangle$GM]}}
                \texttt{VERSCHIEBE: I, (1,1)} \\
            }
        }
    }
    & & \\ \\

    \theutterance \stepcounter{utterance}  
    & & & \multicolumn{2}{p{0.3\linewidth}}{
        \cellcolor[rgb]{0.9,0.9,0.9}{
            \makecell[{{p{\linewidth}}}]{
                \texttt{\tiny{[GM$|$GM]}}
                \texttt{VERSCHIEBE: I, (1,1)} \\
            }
        }
    }
    & & \\ \\

    \theutterance \stepcounter{utterance}  
    & & & \multicolumn{2}{p{0.3\linewidth}}{
        \cellcolor[rgb]{0.9,0.9,0.9}{
            \makecell[{{p{\linewidth}}}]{
                \texttt{\tiny{[GM$|$GM]}}
                \texttt{Strafpunkt: (1,1) ist nicht leer, sondern enthält 'I'.} \\
\\ 
\texttt{Ihr habt gemeinsam 2 von maximal 12 Strafpunkten angesammelt.} \\
\texttt{Bitte versuche es noch mal!} \\
            }
        }
    }
    & & \\ \\

    \theutterance \stepcounter{utterance}  
    & & \multicolumn{4}{p{0.6\linewidth}}{
        \cellcolor[rgb]{0.9,0.9,0.9}{
            \makecell[{{p{\linewidth}}}]{
                \texttt{\tiny{[P1$\langle$GM]}}
                \texttt{Strafpunkt: (1,1) ist nicht leer, sondern enthält 'I'.} \\
\\ 
\texttt{Ihr habt gemeinsam 2 von maximal 12 Strafpunkten angesammelt.} \\
\texttt{Bitte versuche es noch mal!} \\
            }
        }
    }
    & \\ \\

    \theutterance \stepcounter{utterance}  
    & \multicolumn{4}{p{0.6\linewidth}}{
        \cellcolor[rgb]{0.9,0.9,0.9}{
            \makecell[{{p{\linewidth}}}]{
                \texttt{\tiny{[P1$\rangle$GM]}}
                \texttt{VERSCHIEBE: W, (2,1)} \\
            }
        }
    }
    & & \\ \\

    \theutterance \stepcounter{utterance}  
    & & & \multicolumn{2}{p{0.3\linewidth}}{
        \cellcolor[rgb]{0.9,0.9,0.9}{
            \makecell[{{p{\linewidth}}}]{
                \texttt{\tiny{[GM$|$GM]}}
                \texttt{VERSCHIEBE: W, (2,1)} \\
            }
        }
    }
    & & \\ \\

    \theutterance \stepcounter{utterance}  
    & & & \multicolumn{2}{p{0.3\linewidth}}{
        \cellcolor[rgb]{0.9,0.9,0.9}{
            \makecell[{{p{\linewidth}}}]{
                \texttt{\tiny{[GM$|$GM]}}
                \texttt{'W' erfolgreich nach (2,1) verschoben. Dein Raster sieht jetzt so aus:} \\
\texttt{```} \\
\texttt{ 1234567} \\
\texttt{╔═══════╗} \\
\texttt{║IW◌◌◌◌◌║ 1} \\
\texttt{║◌◌◌◌◌◌◌║ 2} \\
\texttt{║◌◌◌T◌┌─╢ 3} \\
\texttt{║◌◌◌┌─┤◌║ 4} \\
\texttt{║◌◌◌│C├─╢ 5} \\
\texttt{╟───┼─┘H║ 6} \\
\texttt{║◌◌◌│◌◌◌║ 7} \\
\texttt{╚═══╧═══╝} \\
\texttt{```} \\
            }
        }
    }
    & & \\ \\

    \theutterance \stepcounter{utterance}  
    & & \multicolumn{4}{p{0.6\linewidth}}{
        \cellcolor[rgb]{0.9,0.9,0.9}{
            \makecell[{{p{\linewidth}}}]{
                \texttt{\tiny{[GM$\rangle$P2]}}
                \texttt{Deine Nachricht wurde an deinen Mitspieler übermittelt.} \\
\\ 
\\ 
\texttt{Ihr seid gerade bei Runde 4 von maximal 20.} \\
\texttt{Ihr habt gemeinsam 2 von maximal 12 Strafpunkten angesammelt.} \\
\texttt{Dein Mitspieler hat ein Objekt auf seinem Raster bewegt.} \\
\\ 
\\ 
\texttt{Wie lautet dein nächster Befehl?} \\
            }
        }
    }
    & \\ \\

    \theutterance \stepcounter{utterance}  
    & & & \multicolumn{4}{p{0.6\linewidth}}{
        \cellcolor[rgb]{0.9,0.9,0.9}{
            \makecell[{{p{\linewidth}}}]{
                \texttt{\tiny{[GM$\langle$P2]}}
                \texttt{SPRICH: Ich werde jetzt beginnen, meine Objekte zu bewegen, um das gemeinsame Ziel zu erreichen. Ich beginne mit dem Objekt 'I', das sich derzeit bei (1,5) befindet und es zur Position (1,1) bewegen werde, dann folgt 'H' zur Position (5,1), 'C' zur Position (4,1) und 'T' zur Position (3,1). Ich melde mich, wenn ich diese Bewegungen durchgeführt habe.} \\
            }
        }
    }
     \\ \\

    \theutterance \stepcounter{utterance}  
    & & & \multicolumn{2}{p{0.3\linewidth}}{
        \cellcolor[rgb]{0.9,0.9,0.9}{
            \makecell[{{p{\linewidth}}}]{
                \texttt{\tiny{[GM$|$GM]}}
                \texttt{SPRICH: Ich werde jetzt beginnen, meine Objekte zu bewegen, um das gemeinsame Ziel zu erreichen. Ich beginne mit dem Objekt 'I', das sich derzeit bei (1,5) befindet und es zur Position (1,1) bewegen werde, dann folgt 'H' zur Position (5,1), 'C' zur Position (4,1) und 'T' zur Position (3,1). Ich melde mich, wenn ich diese Bewegungen durchgeführt habe.} \\
            }
        }
    }
    & & \\ \\

    \theutterance \stepcounter{utterance}  
    & & \multicolumn{4}{p{0.6\linewidth}}{
        \cellcolor[rgb]{0.9,0.9,0.9}{
            \makecell[{{p{\linewidth}}}]{
                \texttt{\tiny{[P1$\langle$GM]}}
                \texttt{'W' erfolgreich nach (2,1) verschoben. Dein Raster sieht jetzt so aus:} \\
\texttt{```} \\
\texttt{ 1234567} \\
\texttt{╔═══════╗} \\
\texttt{║IW◌◌◌◌◌║ 1} \\
\texttt{║◌◌◌◌◌◌◌║ 2} \\
\texttt{║◌◌◌T◌┌─╢ 3} \\
\texttt{║◌◌◌┌─┤◌║ 4} \\
\texttt{║◌◌◌│C├─╢ 5} \\
\texttt{╟───┼─┘H║ 6} \\
\texttt{║◌◌◌│◌◌◌║ 7} \\
\texttt{╚═══╧═══╝} \\
\texttt{```} \\
\texttt{Ihr seid gerade bei Runde 4 von maximal 20.} \\
\texttt{Ihr habt gemeinsam 2 von maximal 12 Strafpunkten angesammelt.} \\
\texttt{Dein Mitspieler hat dir diese Nachricht geschickt:} \\
\\ 
\\ 
\texttt{"Ich werde jetzt beginnen, meine Objekte zu bewegen, um das gemeinsame Ziel zu erreichen. Ich beginne mit dem Objekt 'I', das sich derzeit bei (1,5) befindet und es zur Position (1,1) bewegen werde, dann folgt 'H' zur Position (5,1), 'C' zur Position (4,1) und 'T' zur Position (3,1). Ich melde mich, wenn ich diese Bewegungen durchgeführt habe."} \\
\\ 
\\ 
\texttt{Wie lautet dein nächster Befehl?} \\
            }
        }
    }
    & \\ \\

    \theutterance \stepcounter{utterance}  
    & \multicolumn{4}{p{0.6\linewidth}}{
        \cellcolor[rgb]{0.9,0.9,0.9}{
            \makecell[{{p{\linewidth}}}]{
                \texttt{\tiny{[P1$\rangle$GM]}}
                \texttt{SPRICH: Okay, das klingt nach einem guten Plan! Ich werde abwarten, bis du mir mitteilst, dass du die Bewegungen durchgeführt hast, bevor ich weitere Schritte unternehme. Bitte melde dich, wenn du fertig bist, und wir können dann überprüfen, ob wir das Ziel erreicht haben. Ich werde auch meine Objekte entsprechend anpassen, um das gemeinsame Ziel zu erreichen. Wenn alles korrekt aussieht, können wir dann überlegen, ob wir fertig sind oder noch weitere Anpassungen vornehmen müssen.} \\
            }
        }
    }
    & & \\ \\

    \theutterance \stepcounter{utterance}  
    & & & \multicolumn{2}{p{0.3\linewidth}}{
        \cellcolor[rgb]{0.9,0.9,0.9}{
            \makecell[{{p{\linewidth}}}]{
                \texttt{\tiny{[GM$|$GM]}}
                \texttt{SPRICH: Okay, das klingt nach einem guten Plan! Ich werde abwarten, bis du mir mitteilst, dass du die Bewegungen durchgeführt hast, bevor ich weitere Schritte unternehme. Bitte melde dich, wenn du fertig bist, und wir können dann überprüfen, ob wir das Ziel erreicht haben. Ich werde auch meine Objekte entsprechend anpassen, um das gemeinsame Ziel zu erreichen. Wenn alles korrekt aussieht, können wir dann überlegen, ob wir fertig sind oder noch weitere Anpassungen vornehmen müssen.} \\
            }
        }
    }
    & & \\ \\

    \theutterance \stepcounter{utterance}  
    & & \multicolumn{4}{p{0.6\linewidth}}{
        \cellcolor[rgb]{0.9,0.9,0.9}{
            \makecell[{{p{\linewidth}}}]{
                \texttt{\tiny{[GM$\rangle$P2]}}
                \texttt{Deine Nachricht wurde an deinen Mitspieler übermittelt.} \\
\\ 
\\ 
\texttt{Ihr seid gerade bei Runde 5 von maximal 20.} \\
\texttt{Ihr habt gemeinsam 2 von maximal 12 Strafpunkten angesammelt.} \\
\texttt{Dein Mitspieler hat dir diese Nachricht geschickt:} \\
\\ 
\\ 
\texttt{"Okay, das klingt nach einem guten Plan! Ich werde abwarten, bis du mir mitteilst, dass du die Bewegungen durchgeführt hast, bevor ich weitere Schritte unternehme. Bitte melde dich, wenn du fertig bist, und wir können dann überprüfen, ob wir das Ziel erreicht haben. Ich werde auch meine Objekte entsprechend anpassen, um das gemeinsame Ziel zu erreichen. Wenn alles korrekt aussieht, können wir dann überlegen, ob wir fertig sind oder noch weitere Anpassungen vornehmen müssen."} \\
\\ 
\\ 
\texttt{Wie lautet dein nächster Befehl?} \\
            }
        }
    }
    & \\ \\

    \theutterance \stepcounter{utterance}  
    & & & \multicolumn{4}{p{0.6\linewidth}}{
        \cellcolor[rgb]{0.9,0.9,0.9}{
            \makecell[{{p{\linewidth}}}]{
                \texttt{\tiny{[GM$\langle$P2]}}
                \texttt{VERSCHIEBE: I, (1,1)} \\
            }
        }
    }
     \\ \\

    \theutterance \stepcounter{utterance}  
    & & & \multicolumn{2}{p{0.3\linewidth}}{
        \cellcolor[rgb]{0.9,0.9,0.9}{
            \makecell[{{p{\linewidth}}}]{
                \texttt{\tiny{[GM$|$GM]}}
                \texttt{VERSCHIEBE: I, (1,1)} \\
            }
        }
    }
    & & \\ \\

    \theutterance \stepcounter{utterance}  
    & & & \multicolumn{2}{p{0.3\linewidth}}{
        \cellcolor[rgb]{0.9,0.9,0.9}{
            \makecell[{{p{\linewidth}}}]{
                \texttt{\tiny{[GM$|$GM]}}
                \texttt{'I' erfolgreich nach (1,1) verschoben. Dein Raster sieht jetzt so aus:} \\
\texttt{```} \\
\texttt{ 1234567} \\
\texttt{╔═══════╗} \\
\texttt{║I◌H◌◌◌◌║ 1} \\
\texttt{║◌◌◌C◌◌◌║ 2} \\
\texttt{║◌◌◌◌◌┌─╢ 3} \\
\texttt{║◌◌◌┌─┤◌║ 4} \\
\texttt{║◌◌W│◌├─╢ 5} \\
\texttt{╟───┼─┘◌║ 6} \\
\texttt{║T◌◌│◌◌◌║ 7} \\
\texttt{╚═══╧═══╝} \\
\texttt{```} \\
            }
        }
    }
    & & \\ \\

    \theutterance \stepcounter{utterance}  
    & & \multicolumn{4}{p{0.6\linewidth}}{
        \cellcolor[rgb]{0.9,0.9,0.9}{
            \makecell[{{p{\linewidth}}}]{
                \texttt{\tiny{[P1$\langle$GM]}}
                \texttt{Deine Nachricht wurde an deinen Mitspieler übermittelt.} \\
\\ 
\\ 
\texttt{Ihr seid gerade bei Runde 5 von maximal 20.} \\
\texttt{Ihr habt gemeinsam 2 von maximal 12 Strafpunkten angesammelt.} \\
\texttt{Dein Mitspieler hat ein Objekt auf seinem Raster bewegt.} \\
\\ 
\\ 
\texttt{Wie lautet dein nächster Befehl?} \\
            }
        }
    }
    & \\ \\

    \theutterance \stepcounter{utterance}  
    & \multicolumn{4}{p{0.6\linewidth}}{
        \cellcolor[rgb]{0.9,0.9,0.9}{
            \makecell[{{p{\linewidth}}}]{
                \texttt{\tiny{[P1$\rangle$GM]}}
                \texttt{VERSCHIEBE: T, (3,1)} \\
            }
        }
    }
    & & \\ \\

    \theutterance \stepcounter{utterance}  
    & & & \multicolumn{2}{p{0.3\linewidth}}{
        \cellcolor[rgb]{0.9,0.9,0.9}{
            \makecell[{{p{\linewidth}}}]{
                \texttt{\tiny{[GM$|$GM]}}
                \texttt{VERSCHIEBE: T, (3,1)} \\
            }
        }
    }
    & & \\ \\

    \theutterance \stepcounter{utterance}  
    & & & \multicolumn{2}{p{0.3\linewidth}}{
        \cellcolor[rgb]{0.9,0.9,0.9}{
            \makecell[{{p{\linewidth}}}]{
                \texttt{\tiny{[GM$|$GM]}}
                \texttt{'T' erfolgreich nach (3,1) verschoben. Dein Raster sieht jetzt so aus:} \\
\texttt{```} \\
\texttt{ 1234567} \\
\texttt{╔═══════╗} \\
\texttt{║IWT◌◌◌◌║ 1} \\
\texttt{║◌◌◌◌◌◌◌║ 2} \\
\texttt{║◌◌◌◌◌┌─╢ 3} \\
\texttt{║◌◌◌┌─┤◌║ 4} \\
\texttt{║◌◌◌│C├─╢ 5} \\
\texttt{╟───┼─┘H║ 6} \\
\texttt{║◌◌◌│◌◌◌║ 7} \\
\texttt{╚═══╧═══╝} \\
\texttt{```} \\
            }
        }
    }
    & & \\ \\

    \theutterance \stepcounter{utterance}  
    & & \multicolumn{4}{p{0.6\linewidth}}{
        \cellcolor[rgb]{0.9,0.9,0.9}{
            \makecell[{{p{\linewidth}}}]{
                \texttt{\tiny{[GM$\rangle$P2]}}
                \texttt{'I' erfolgreich nach (1,1) verschoben. Dein Raster sieht jetzt so aus:} \\
\texttt{```} \\
\texttt{ 1234567} \\
\texttt{╔═══════╗} \\
\texttt{║I◌H◌◌◌◌║ 1} \\
\texttt{║◌◌◌C◌◌◌║ 2} \\
\texttt{║◌◌◌◌◌┌─╢ 3} \\
\texttt{║◌◌◌┌─┤◌║ 4} \\
\texttt{║◌◌W│◌├─╢ 5} \\
\texttt{╟───┼─┘◌║ 6} \\
\texttt{║T◌◌│◌◌◌║ 7} \\
\texttt{╚═══╧═══╝} \\
\texttt{```} \\
\texttt{Ihr seid gerade bei Runde 6 von maximal 20.} \\
\texttt{Ihr habt gemeinsam 2 von maximal 12 Strafpunkten angesammelt.} \\
\texttt{Dein Mitspieler hat ein Objekt auf seinem Raster bewegt.} \\
\\ 
\\ 
\texttt{Wie lautet dein nächster Befehl?} \\
            }
        }
    }
    & \\ \\

    \theutterance \stepcounter{utterance}  
    & & & \multicolumn{4}{p{0.6\linewidth}}{
        \cellcolor[rgb]{0.9,0.9,0.9}{
            \makecell[{{p{\linewidth}}}]{
                \texttt{\tiny{[GM$\langle$P2]}}
                \texttt{VERSCHIEBE: H, (5,1)} \\
            }
        }
    }
     \\ \\

    \theutterance \stepcounter{utterance}  
    & & & \multicolumn{2}{p{0.3\linewidth}}{
        \cellcolor[rgb]{0.9,0.9,0.9}{
            \makecell[{{p{\linewidth}}}]{
                \texttt{\tiny{[GM$|$GM]}}
                \texttt{VERSCHIEBE: H, (5,1)} \\
            }
        }
    }
    & & \\ \\

    \theutterance \stepcounter{utterance}  
    & & & \multicolumn{2}{p{0.3\linewidth}}{
        \cellcolor[rgb]{0.9,0.9,0.9}{
            \makecell[{{p{\linewidth}}}]{
                \texttt{\tiny{[GM$|$GM]}}
                \texttt{'H' erfolgreich nach (5,1) verschoben. Dein Raster sieht jetzt so aus:} \\
\texttt{```} \\
\texttt{ 1234567} \\
\texttt{╔═══════╗} \\
\texttt{║I◌◌◌H◌◌║ 1} \\
\texttt{║◌◌◌C◌◌◌║ 2} \\
\texttt{║◌◌◌◌◌┌─╢ 3} \\
\texttt{║◌◌◌┌─┤◌║ 4} \\
\texttt{║◌◌W│◌├─╢ 5} \\
\texttt{╟───┼─┘◌║ 6} \\
\texttt{║T◌◌│◌◌◌║ 7} \\
\texttt{╚═══╧═══╝} \\
\texttt{```} \\
            }
        }
    }
    & & \\ \\

    \theutterance \stepcounter{utterance}  
    & & \multicolumn{4}{p{0.6\linewidth}}{
        \cellcolor[rgb]{0.9,0.9,0.9}{
            \makecell[{{p{\linewidth}}}]{
                \texttt{\tiny{[P1$\langle$GM]}}
                \texttt{'T' erfolgreich nach (3,1) verschoben. Dein Raster sieht jetzt so aus:} \\
\texttt{```} \\
\texttt{ 1234567} \\
\texttt{╔═══════╗} \\
\texttt{║IWT◌◌◌◌║ 1} \\
\texttt{║◌◌◌◌◌◌◌║ 2} \\
\texttt{║◌◌◌◌◌┌─╢ 3} \\
\texttt{║◌◌◌┌─┤◌║ 4} \\
\texttt{║◌◌◌│C├─╢ 5} \\
\texttt{╟───┼─┘H║ 6} \\
\texttt{║◌◌◌│◌◌◌║ 7} \\
\texttt{╚═══╧═══╝} \\
\texttt{```} \\
\texttt{Ihr seid gerade bei Runde 6 von maximal 20.} \\
\texttt{Ihr habt gemeinsam 2 von maximal 12 Strafpunkten angesammelt.} \\
\texttt{Dein Mitspieler hat ein Objekt auf seinem Raster bewegt.} \\
\\ 
\\ 
\texttt{Wie lautet dein nächster Befehl?} \\
            }
        }
    }
    & \\ \\

    \theutterance \stepcounter{utterance}  
    & \multicolumn{4}{p{0.6\linewidth}}{
        \cellcolor[rgb]{0.9,0.9,0.9}{
            \makecell[{{p{\linewidth}}}]{
                \texttt{\tiny{[P1$\rangle$GM]}}
                \texttt{VERSCHIEBE: C, (4,1)} \\
            }
        }
    }
    & & \\ \\

    \theutterance \stepcounter{utterance}  
    & & & \multicolumn{2}{p{0.3\linewidth}}{
        \cellcolor[rgb]{0.9,0.9,0.9}{
            \makecell[{{p{\linewidth}}}]{
                \texttt{\tiny{[GM$|$GM]}}
                \texttt{VERSCHIEBE: C, (4,1)} \\
            }
        }
    }
    & & \\ \\

    \theutterance \stepcounter{utterance}  
    & & & \multicolumn{2}{p{0.3\linewidth}}{
        \cellcolor[rgb]{0.9,0.9,0.9}{
            \makecell[{{p{\linewidth}}}]{
                \texttt{\tiny{[GM$|$GM]}}
                \texttt{'C' erfolgreich nach (4,1) verschoben. Dein Raster sieht jetzt so aus:} \\
\texttt{```} \\
\texttt{ 1234567} \\
\texttt{╔═══════╗} \\
\texttt{║IWTC◌◌◌║ 1} \\
\texttt{║◌◌◌◌◌◌◌║ 2} \\
\texttt{║◌◌◌◌◌┌─╢ 3} \\
\texttt{║◌◌◌┌─┤◌║ 4} \\
\texttt{║◌◌◌│◌├─╢ 5} \\
\texttt{╟───┼─┘H║ 6} \\
\texttt{║◌◌◌│◌◌◌║ 7} \\
\texttt{╚═══╧═══╝} \\
\texttt{```} \\
            }
        }
    }
    & & \\ \\

    \theutterance \stepcounter{utterance}  
    & & \multicolumn{4}{p{0.6\linewidth}}{
        \cellcolor[rgb]{0.9,0.9,0.9}{
            \makecell[{{p{\linewidth}}}]{
                \texttt{\tiny{[GM$\rangle$P2]}}
                \texttt{'H' erfolgreich nach (5,1) verschoben. Dein Raster sieht jetzt so aus:} \\
\texttt{```} \\
\texttt{ 1234567} \\
\texttt{╔═══════╗} \\
\texttt{║I◌◌◌H◌◌║ 1} \\
\texttt{║◌◌◌C◌◌◌║ 2} \\
\texttt{║◌◌◌◌◌┌─╢ 3} \\
\texttt{║◌◌◌┌─┤◌║ 4} \\
\texttt{║◌◌W│◌├─╢ 5} \\
\texttt{╟───┼─┘◌║ 6} \\
\texttt{║T◌◌│◌◌◌║ 7} \\
\texttt{╚═══╧═══╝} \\
\texttt{```} \\
\texttt{Ihr seid gerade bei Runde 7 von maximal 20.} \\
\texttt{Ihr habt gemeinsam 2 von maximal 12 Strafpunkten angesammelt.} \\
\texttt{Dein Mitspieler hat ein Objekt auf seinem Raster bewegt.} \\
\\ 
\\ 
\texttt{Wie lautet dein nächster Befehl?} \\
            }
        }
    }
    & \\ \\

    \theutterance \stepcounter{utterance}  
    & & & \multicolumn{4}{p{0.6\linewidth}}{
        \cellcolor[rgb]{0.9,0.9,0.9}{
            \makecell[{{p{\linewidth}}}]{
                \texttt{\tiny{[GM$\langle$P2]}}
                \texttt{VERSCHIEBE: C, (4,1)} \\
            }
        }
    }
     \\ \\

    \theutterance \stepcounter{utterance}  
    & & & \multicolumn{2}{p{0.3\linewidth}}{
        \cellcolor[rgb]{0.9,0.9,0.9}{
            \makecell[{{p{\linewidth}}}]{
                \texttt{\tiny{[GM$|$GM]}}
                \texttt{VERSCHIEBE: C, (4,1)} \\
            }
        }
    }
    & & \\ \\

    \theutterance \stepcounter{utterance}  
    & & & \multicolumn{2}{p{0.3\linewidth}}{
        \cellcolor[rgb]{0.9,0.9,0.9}{
            \makecell[{{p{\linewidth}}}]{
                \texttt{\tiny{[GM$|$GM]}}
                \texttt{'C' erfolgreich nach (4,1) verschoben. Dein Raster sieht jetzt so aus:} \\
\texttt{```} \\
\texttt{ 1234567} \\
\texttt{╔═══════╗} \\
\texttt{║I◌◌CH◌◌║ 1} \\
\texttt{║◌◌◌◌◌◌◌║ 2} \\
\texttt{║◌◌◌◌◌┌─╢ 3} \\
\texttt{║◌◌◌┌─┤◌║ 4} \\
\texttt{║◌◌W│◌├─╢ 5} \\
\texttt{╟───┼─┘◌║ 6} \\
\texttt{║T◌◌│◌◌◌║ 7} \\
\texttt{╚═══╧═══╝} \\
\texttt{```} \\
            }
        }
    }
    & & \\ \\

    \theutterance \stepcounter{utterance}  
    & & \multicolumn{4}{p{0.6\linewidth}}{
        \cellcolor[rgb]{0.9,0.9,0.9}{
            \makecell[{{p{\linewidth}}}]{
                \texttt{\tiny{[P1$\langle$GM]}}
                \texttt{'C' erfolgreich nach (4,1) verschoben. Dein Raster sieht jetzt so aus:} \\
\texttt{```} \\
\texttt{ 1234567} \\
\texttt{╔═══════╗} \\
\texttt{║IWTC◌◌◌║ 1} \\
\texttt{║◌◌◌◌◌◌◌║ 2} \\
\texttt{║◌◌◌◌◌┌─╢ 3} \\
\texttt{║◌◌◌┌─┤◌║ 4} \\
\texttt{║◌◌◌│◌├─╢ 5} \\
\texttt{╟───┼─┘H║ 6} \\
\texttt{║◌◌◌│◌◌◌║ 7} \\
\texttt{╚═══╧═══╝} \\
\texttt{```} \\
\texttt{Ihr seid gerade bei Runde 7 von maximal 20.} \\
\texttt{Ihr habt gemeinsam 2 von maximal 12 Strafpunkten angesammelt.} \\
\texttt{Dein Mitspieler hat ein Objekt auf seinem Raster bewegt.} \\
\\ 
\\ 
\texttt{Wie lautet dein nächster Befehl?} \\
            }
        }
    }
    & \\ \\

    \theutterance \stepcounter{utterance}  
    & \multicolumn{4}{p{0.6\linewidth}}{
        \cellcolor[rgb]{0.9,0.9,0.9}{
            \makecell[{{p{\linewidth}}}]{
                \texttt{\tiny{[P1$\rangle$GM]}}
                \texttt{VERSCHIEBE: H, (5,1)} \\
            }
        }
    }
    & & \\ \\

    \theutterance \stepcounter{utterance}  
    & & & \multicolumn{2}{p{0.3\linewidth}}{
        \cellcolor[rgb]{0.9,0.9,0.9}{
            \makecell[{{p{\linewidth}}}]{
                \texttt{\tiny{[GM$|$GM]}}
                \texttt{VERSCHIEBE: H, (5,1)} \\
            }
        }
    }
    & & \\ \\

    \theutterance \stepcounter{utterance}  
    & & & \multicolumn{2}{p{0.3\linewidth}}{
        \cellcolor[rgb]{0.9,0.9,0.9}{
            \makecell[{{p{\linewidth}}}]{
                \texttt{\tiny{[GM$|$GM]}}
                \texttt{'H' erfolgreich nach (5,1) verschoben. Dein Raster sieht jetzt so aus:} \\
\texttt{```} \\
\texttt{ 1234567} \\
\texttt{╔═══════╗} \\
\texttt{║IWTCH◌◌║ 1} \\
\texttt{║◌◌◌◌◌◌◌║ 2} \\
\texttt{║◌◌◌◌◌┌─╢ 3} \\
\texttt{║◌◌◌┌─┤◌║ 4} \\
\texttt{║◌◌◌│◌├─╢ 5} \\
\texttt{╟───┼─┘◌║ 6} \\
\texttt{║◌◌◌│◌◌◌║ 7} \\
\texttt{╚═══╧═══╝} \\
\texttt{```} \\
            }
        }
    }
    & & \\ \\

    \theutterance \stepcounter{utterance}  
    & & \multicolumn{4}{p{0.6\linewidth}}{
        \cellcolor[rgb]{0.9,0.9,0.9}{
            \makecell[{{p{\linewidth}}}]{
                \texttt{\tiny{[GM$\rangle$P2]}}
                \texttt{'C' erfolgreich nach (4,1) verschoben. Dein Raster sieht jetzt so aus:} \\
\texttt{```} \\
\texttt{ 1234567} \\
\texttt{╔═══════╗} \\
\texttt{║I◌◌CH◌◌║ 1} \\
\texttt{║◌◌◌◌◌◌◌║ 2} \\
\texttt{║◌◌◌◌◌┌─╢ 3} \\
\texttt{║◌◌◌┌─┤◌║ 4} \\
\texttt{║◌◌W│◌├─╢ 5} \\
\texttt{╟───┼─┘◌║ 6} \\
\texttt{║T◌◌│◌◌◌║ 7} \\
\texttt{╚═══╧═══╝} \\
\texttt{```} \\
\texttt{Ihr seid gerade bei Runde 8 von maximal 20.} \\
\texttt{Ihr habt gemeinsam 2 von maximal 12 Strafpunkten angesammelt.} \\
\texttt{Dein Mitspieler hat ein Objekt auf seinem Raster bewegt.} \\
\\ 
\\ 
\texttt{Wie lautet dein nächster Befehl?} \\
            }
        }
    }
    & \\ \\

    \theutterance \stepcounter{utterance}  
    & & & \multicolumn{4}{p{0.6\linewidth}}{
        \cellcolor[rgb]{0.9,0.9,0.9}{
            \makecell[{{p{\linewidth}}}]{
                \texttt{\tiny{[GM$\langle$P2]}}
                \texttt{VERSCHIEBE: T, (3,1)} \\
            }
        }
    }
     \\ \\

    \theutterance \stepcounter{utterance}  
    & & & \multicolumn{2}{p{0.3\linewidth}}{
        \cellcolor[rgb]{0.9,0.9,0.9}{
            \makecell[{{p{\linewidth}}}]{
                \texttt{\tiny{[GM$|$GM]}}
                \texttt{VERSCHIEBE: T, (3,1)} \\
            }
        }
    }
    & & \\ \\

    \theutterance \stepcounter{utterance}  
    & & & \multicolumn{2}{p{0.3\linewidth}}{
        \cellcolor[rgb]{0.9,0.9,0.9}{
            \makecell[{{p{\linewidth}}}]{
                \texttt{\tiny{[GM$|$GM]}}
                \texttt{'T' erfolgreich nach (3,1) verschoben. Dein Raster sieht jetzt so aus:} \\
\texttt{```} \\
\texttt{ 1234567} \\
\texttt{╔═══════╗} \\
\texttt{║I◌TCH◌◌║ 1} \\
\texttt{║◌◌◌◌◌◌◌║ 2} \\
\texttt{║◌◌◌◌◌┌─╢ 3} \\
\texttt{║◌◌◌┌─┤◌║ 4} \\
\texttt{║◌◌W│◌├─╢ 5} \\
\texttt{╟───┼─┘◌║ 6} \\
\texttt{║◌◌◌│◌◌◌║ 7} \\
\texttt{╚═══╧═══╝} \\
\texttt{```} \\
            }
        }
    }
    & & \\ \\

    \theutterance \stepcounter{utterance}  
    & & \multicolumn{4}{p{0.6\linewidth}}{
        \cellcolor[rgb]{0.9,0.9,0.9}{
            \makecell[{{p{\linewidth}}}]{
                \texttt{\tiny{[P1$\langle$GM]}}
                \texttt{'H' erfolgreich nach (5,1) verschoben. Dein Raster sieht jetzt so aus:} \\
\texttt{```} \\
\texttt{ 1234567} \\
\texttt{╔═══════╗} \\
\texttt{║IWTCH◌◌║ 1} \\
\texttt{║◌◌◌◌◌◌◌║ 2} \\
\texttt{║◌◌◌◌◌┌─╢ 3} \\
\texttt{║◌◌◌┌─┤◌║ 4} \\
\texttt{║◌◌◌│◌├─╢ 5} \\
\texttt{╟───┼─┘◌║ 6} \\
\texttt{║◌◌◌│◌◌◌║ 7} \\
\texttt{╚═══╧═══╝} \\
\texttt{```} \\
\texttt{Ihr seid gerade bei Runde 8 von maximal 20.} \\
\texttt{Ihr habt gemeinsam 2 von maximal 12 Strafpunkten angesammelt.} \\
\texttt{Dein Mitspieler hat ein Objekt auf seinem Raster bewegt.} \\
\\ 
\\ 
\texttt{Wie lautet dein nächster Befehl?} \\
            }
        }
    }
    & \\ \\

    \theutterance \stepcounter{utterance}  
    & \multicolumn{4}{p{0.6\linewidth}}{
        \cellcolor[rgb]{0.9,0.9,0.9}{
            \makecell[{{p{\linewidth}}}]{
                \texttt{\tiny{[P1$\rangle$GM]}}
                \texttt{SPRICH: Fertig?} \\
            }
        }
    }
    & & \\ \\

    \theutterance \stepcounter{utterance}  
    & & & \multicolumn{2}{p{0.3\linewidth}}{
        \cellcolor[rgb]{0.9,0.9,0.9}{
            \makecell[{{p{\linewidth}}}]{
                \texttt{\tiny{[GM$|$GM]}}
                \texttt{SPRICH: Fertig?} \\
            }
        }
    }
    & & \\ \\

    \theutterance \stepcounter{utterance}  
    & & \multicolumn{4}{p{0.6\linewidth}}{
        \cellcolor[rgb]{0.9,0.9,0.9}{
            \makecell[{{p{\linewidth}}}]{
                \texttt{\tiny{[GM$\rangle$P2]}}
                \texttt{'T' erfolgreich nach (3,1) verschoben. Dein Raster sieht jetzt so aus:} \\
\texttt{```} \\
\texttt{ 1234567} \\
\texttt{╔═══════╗} \\
\texttt{║I◌TCH◌◌║ 1} \\
\texttt{║◌◌◌◌◌◌◌║ 2} \\
\texttt{║◌◌◌◌◌┌─╢ 3} \\
\texttt{║◌◌◌┌─┤◌║ 4} \\
\texttt{║◌◌W│◌├─╢ 5} \\
\texttt{╟───┼─┘◌║ 6} \\
\texttt{║◌◌◌│◌◌◌║ 7} \\
\texttt{╚═══╧═══╝} \\
\texttt{```} \\
\texttt{Ihr seid gerade bei Runde 9 von maximal 20.} \\
\texttt{Ihr habt gemeinsam 2 von maximal 12 Strafpunkten angesammelt.} \\
\texttt{Dein Mitspieler hat dir diese Nachricht geschickt:} \\
\\ 
\\ 
\texttt{"Fertig?"} \\
\\ 
\\ 
\texttt{Wie lautet dein nächster Befehl?} \\
            }
        }
    }
    & \\ \\

    \theutterance \stepcounter{utterance}  
    & & & \multicolumn{4}{p{0.6\linewidth}}{
        \cellcolor[rgb]{0.9,0.9,0.9}{
            \makecell[{{p{\linewidth}}}]{
                \texttt{\tiny{[GM$\langle$P2]}}
                \texttt{SPRICH: fertig!} \\
            }
        }
    }
     \\ \\

    \theutterance \stepcounter{utterance}  
    & & & \multicolumn{2}{p{0.3\linewidth}}{
        \cellcolor[rgb]{0.9,0.9,0.9}{
            \makecell[{{p{\linewidth}}}]{
                \texttt{\tiny{[GM$|$GM]}}
                \texttt{SPRICH: fertig!} \\
            }
        }
    }
    & & \\ \\

    \theutterance \stepcounter{utterance}  
    & & \multicolumn{4}{p{0.6\linewidth}}{
        \cellcolor[rgb]{0.9,0.9,0.9}{
            \makecell[{{p{\linewidth}}}]{
                \texttt{\tiny{[P1$\langle$GM]}}
                \texttt{Deine Nachricht wurde an deinen Mitspieler übermittelt.} \\
\\ 
\\ 
\texttt{Ihr seid gerade bei Runde 9 von maximal 20.} \\
\texttt{Ihr habt gemeinsam 2 von maximal 12 Strafpunkten angesammelt.} \\
\texttt{Dein Mitspieler hat dir diese Nachricht geschickt:} \\
\\ 
\\ 
\texttt{"fertig!"} \\
\\ 
\\ 
\texttt{Wie lautet dein nächster Befehl?} \\
            }
        }
    }
    & \\ \\

    \theutterance \stepcounter{utterance}  
    & \multicolumn{4}{p{0.6\linewidth}}{
        \cellcolor[rgb]{0.9,0.9,0.9}{
            \makecell[{{p{\linewidth}}}]{
                \texttt{\tiny{[P1$\rangle$GM]}}
                \texttt{Das Spiel ist beendet. Wir haben das Ziel erreicht! Ich denke, wir haben eine gute Strategie entwickelt und effektiv kommuniziert, um die Objekte an die richtigen Positionen zu bewegen. Ich bin zufrieden mit unserem Ergebnis!} \\
            }
        }
    }
    & & \\ \\

    \theutterance \stepcounter{utterance}  
    & & & \multicolumn{2}{p{0.3\linewidth}}{
        \cellcolor[rgb]{0.9,0.9,0.9}{
            \makecell[{{p{\linewidth}}}]{
                \texttt{\tiny{[GM$|$GM]}}
                \texttt{Das Spiel ist beendet. Wir haben das Ziel erreicht! Ich denke, wir haben eine gute Strategie entwickelt und effektiv kommuniziert, um die Objekte an die richtigen Positionen zu bewegen. Ich bin zufrieden mit unserem Ergebnis!} \\
            }
        }
    }
    & & \\ \\

    \theutterance \stepcounter{utterance}  
    & & & \multicolumn{2}{p{0.3\linewidth}}{
        \cellcolor[rgb]{0.9,0.9,0.9}{
            \makecell[{{p{\linewidth}}}]{
                \texttt{\tiny{[GM$|$GM]}}
                \texttt{Invalid response format} \\
            }
        }
    }
    & & \\ \\

    \theutterance \stepcounter{utterance}  
    & & \multicolumn{4}{p{0.6\linewidth}}{
        \cellcolor[rgb]{0.9,0.9,0.9}{
            \makecell[{{p{\linewidth}}}]{
                \texttt{\tiny{[P1$\langle$GM]}}
                \texttt{Strafpunkt: Deine Nachricht entspricht keinem gültigen Format! Stelle sicher, dass deine Antwort ausschließlich einen der beiden Befehle `SPRICH: <NACHRICHT>` und `VERSCHIEBE: <OBJECT>, (<X>, <Y>)` und sonst nichts enthält!} \\
\\ 
\texttt{Ihr habt gemeinsam 3 von maximal 12 Strafpunkten angesammelt.} \\
\texttt{Bitte versuche es noch mal!} \\
            }
        }
    }
    & \\ \\

    \theutterance \stepcounter{utterance}  
    & \multicolumn{4}{p{0.6\linewidth}}{
        \cellcolor[rgb]{0.9,0.9,0.9}{
            \makecell[{{p{\linewidth}}}]{
                \texttt{\tiny{[P1$\rangle$GM]}}
                \texttt{SPRICH: Wir haben gewonnen!} \\
            }
        }
    }
    & & \\ \\

    \theutterance \stepcounter{utterance}  
    & & & \multicolumn{2}{p{0.3\linewidth}}{
        \cellcolor[rgb]{0.9,0.9,0.9}{
            \makecell[{{p{\linewidth}}}]{
                \texttt{\tiny{[GM$|$GM]}}
                \texttt{SPRICH: Wir haben gewonnen!} \\
            }
        }
    }
    & & \\ \\

    \theutterance \stepcounter{utterance}  
    & & \multicolumn{4}{p{0.6\linewidth}}{
        \cellcolor[rgb]{0.9,0.9,0.9}{
            \makecell[{{p{\linewidth}}}]{
                \texttt{\tiny{[GM$\rangle$P2]}}
                \texttt{Deine Nachricht wurde an deinen Mitspieler übermittelt.} \\
\\ 
\\ 
\texttt{Ihr seid gerade bei Runde 10 von maximal 20.} \\
\texttt{Ihr habt gemeinsam 3 von maximal 12 Strafpunkten angesammelt.} \\
\texttt{Dein Mitspieler hat dir diese Nachricht geschickt:} \\
\\ 
\\ 
\texttt{"Wir haben gewonnen!"} \\
\\ 
\\ 
\texttt{Wie lautet dein nächster Befehl?} \\
            }
        }
    }
    & \\ \\

    \theutterance \stepcounter{utterance}  
    & & & \multicolumn{4}{p{0.6\linewidth}}{
        \cellcolor[rgb]{0.9,0.9,0.9}{
            \makecell[{{p{\linewidth}}}]{
                \texttt{\tiny{[GM$\langle$P2]}}
                \texttt{SPRICH: Gratuliere! Wir haben es gemeinsam geschafft! Ich denke, wir haben eine gute Strategie entwickelt und erfolgreich umgesetzt. Danke für das schöne Spiel!} \\
            }
        }
    }
     \\ \\

    \theutterance \stepcounter{utterance}  
    & & & \multicolumn{2}{p{0.3\linewidth}}{
        \cellcolor[rgb]{0.9,0.9,0.9}{
            \makecell[{{p{\linewidth}}}]{
                \texttt{\tiny{[GM$|$GM]}}
                \texttt{SPRICH: Gratuliere! Wir haben es gemeinsam geschafft! Ich denke, wir haben eine gute Strategie entwickelt und erfolgreich umgesetzt. Danke für das schöne Spiel!} \\
            }
        }
    }
    & & \\ \\

    \theutterance \stepcounter{utterance}  
    & & \multicolumn{4}{p{0.6\linewidth}}{
        \cellcolor[rgb]{0.9,0.9,0.9}{
            \makecell[{{p{\linewidth}}}]{
                \texttt{\tiny{[P1$\langle$GM]}}
                \texttt{Deine Nachricht wurde an deinen Mitspieler übermittelt.} \\
\\ 
\\ 
\texttt{Ihr seid gerade bei Runde 10 von maximal 20.} \\
\texttt{Ihr habt gemeinsam 3 von maximal 12 Strafpunkten angesammelt.} \\
\texttt{Dein Mitspieler hat dir diese Nachricht geschickt:} \\
\\ 
\\ 
\texttt{"Gratuliere! Wir haben es gemeinsam geschafft! Ich denke, wir haben eine gute Strategie entwickelt und erfolgreich umgesetzt. Danke für das schöne Spiel!"} \\
\\ 
\\ 
\texttt{Wie lautet dein nächster Befehl?} \\
            }
        }
    }
    & \\ \\

    \theutterance \stepcounter{utterance}  
    & \multicolumn{4}{p{0.6\linewidth}}{
        \cellcolor[rgb]{0.9,0.9,0.9}{
            \makecell[{{p{\linewidth}}}]{
                \texttt{\tiny{[P1$\rangle$GM]}}
                \texttt{SPRICH: Danke ebenfalls! Es war ein tolles Spiel und ich bin froh, dass wir so gut zusammenarbeiten konnten. Ich wünsche dir ein schönes Spiel beim nächsten Mal!} \\
            }
        }
    }
    & & \\ \\

    \theutterance \stepcounter{utterance}  
    & & & \multicolumn{2}{p{0.3\linewidth}}{
        \cellcolor[rgb]{0.9,0.9,0.9}{
            \makecell[{{p{\linewidth}}}]{
                \texttt{\tiny{[GM$|$GM]}}
                \texttt{SPRICH: Danke ebenfalls! Es war ein tolles Spiel und ich bin froh, dass wir so gut zusammenarbeiten konnten. Ich wünsche dir ein schönes Spiel beim nächsten Mal!} \\
            }
        }
    }
    & & \\ \\

    \theutterance \stepcounter{utterance}  
    & & \multicolumn{4}{p{0.6\linewidth}}{
        \cellcolor[rgb]{0.9,0.9,0.9}{
            \makecell[{{p{\linewidth}}}]{
                \texttt{\tiny{[GM$\rangle$P2]}}
                \texttt{Deine Nachricht wurde an deinen Mitspieler übermittelt.} \\
\\ 
\\ 
\texttt{Ihr seid gerade bei Runde 11 von maximal 20.} \\
\texttt{Ihr habt gemeinsam 3 von maximal 12 Strafpunkten angesammelt.} \\
\texttt{Dein Mitspieler hat dir diese Nachricht geschickt:} \\
\\ 
\\ 
\texttt{"Danke ebenfalls! Es war ein tolles Spiel und ich bin froh, dass wir so gut zusammenarbeiten konnten. Ich wünsche dir ein schönes Spiel beim nächsten Mal!"} \\
\\ 
\\ 
\texttt{Wie lautet dein nächster Befehl?} \\
            }
        }
    }
    & \\ \\

    \theutterance \stepcounter{utterance}  
    & & & \multicolumn{4}{p{0.6\linewidth}}{
        \cellcolor[rgb]{0.9,0.9,0.9}{
            \makecell[{{p{\linewidth}}}]{
                \texttt{\tiny{[GM$\langle$P2]}}
                \texttt{SPRICH: Ich wünsche dir ebenfalls alles Gute und hoffe, dass wir bald wieder zusammen spielen können! Bis zum nächsten Mal!} \\
            }
        }
    }
     \\ \\

    \theutterance \stepcounter{utterance}  
    & & & \multicolumn{2}{p{0.3\linewidth}}{
        \cellcolor[rgb]{0.9,0.9,0.9}{
            \makecell[{{p{\linewidth}}}]{
                \texttt{\tiny{[GM$|$GM]}}
                \texttt{SPRICH: Ich wünsche dir ebenfalls alles Gute und hoffe, dass wir bald wieder zusammen spielen können! Bis zum nächsten Mal!} \\
            }
        }
    }
    & & \\ \\

    \theutterance \stepcounter{utterance}  
    & & \multicolumn{4}{p{0.6\linewidth}}{
        \cellcolor[rgb]{0.9,0.9,0.9}{
            \makecell[{{p{\linewidth}}}]{
                \texttt{\tiny{[P1$\langle$GM]}}
                \texttt{Deine Nachricht wurde an deinen Mitspieler übermittelt.} \\
\\ 
\\ 
\texttt{Ihr seid gerade bei Runde 11 von maximal 20.} \\
\texttt{Ihr habt gemeinsam 3 von maximal 12 Strafpunkten angesammelt.} \\
\texttt{Dein Mitspieler hat dir diese Nachricht geschickt:} \\
\\ 
\\ 
\texttt{"Ich wünsche dir ebenfalls alles Gute und hoffe, dass wir bald wieder zusammen spielen können! Bis zum nächsten Mal!"} \\
\\ 
\\ 
\texttt{Wie lautet dein nächster Befehl?} \\
            }
        }
    }
    & \\ \\

    \theutterance \stepcounter{utterance}  
    & \multicolumn{4}{p{0.6\linewidth}}{
        \cellcolor[rgb]{0.9,0.9,0.9}{
            \makecell[{{p{\linewidth}}}]{
                \texttt{\tiny{[P1$\rangle$GM]}}
                \texttt{SPRICH: Bis zum nächsten Mal! Es war wirklich ein Vergnügen, mit dir zu spielen! Ich hoffe, wir können bald wieder ein Spiel starten und unsere Strategien noch weiter verbessern. Auf Wiedersehen!} \\
            }
        }
    }
    & & \\ \\

    \theutterance \stepcounter{utterance}  
    & & & \multicolumn{2}{p{0.3\linewidth}}{
        \cellcolor[rgb]{0.9,0.9,0.9}{
            \makecell[{{p{\linewidth}}}]{
                \texttt{\tiny{[GM$|$GM]}}
                \texttt{SPRICH: Bis zum nächsten Mal! Es war wirklich ein Vergnügen, mit dir zu spielen! Ich hoffe, wir können bald wieder ein Spiel starten und unsere Strategien noch weiter verbessern. Auf Wiedersehen!} \\
            }
        }
    }
    & & \\ \\

    \theutterance \stepcounter{utterance}  
    & & \multicolumn{4}{p{0.6\linewidth}}{
        \cellcolor[rgb]{0.9,0.9,0.9}{
            \makecell[{{p{\linewidth}}}]{
                \texttt{\tiny{[GM$\rangle$P2]}}
                \texttt{Deine Nachricht wurde an deinen Mitspieler übermittelt.} \\
\\ 
\\ 
\texttt{Ihr seid gerade bei Runde 12 von maximal 20.} \\
\texttt{Ihr habt gemeinsam 3 von maximal 12 Strafpunkten angesammelt.} \\
\texttt{Dein Mitspieler hat dir diese Nachricht geschickt:} \\
\\ 
\\ 
\texttt{"Bis zum nächsten Mal! Es war wirklich ein Vergnügen, mit dir zu spielen! Ich hoffe, wir können bald wieder ein Spiel starten und unsere Strategien noch weiter verbessern. Auf Wiedersehen!"} \\
\\ 
\\ 
\texttt{Wie lautet dein nächster Befehl?} \\
            }
        }
    }
    & \\ \\

    \theutterance \stepcounter{utterance}  
    & & & \multicolumn{4}{p{0.6\linewidth}}{
        \cellcolor[rgb]{0.9,0.9,0.9}{
            \makecell[{{p{\linewidth}}}]{
                \texttt{\tiny{[GM$\langle$P2]}}
                \texttt{SPRICH: Auf Wiedersehen! Es war mir auch ein Vergnügen, mit dir zu spielen! Ich freue mich darauf, bald wieder ein Spiel mit dir zu spielen und unsere Strategien zu verbessern. Bis bald!} \\
            }
        }
    }
     \\ \\

    \theutterance \stepcounter{utterance}  
    & & & \multicolumn{2}{p{0.3\linewidth}}{
        \cellcolor[rgb]{0.9,0.9,0.9}{
            \makecell[{{p{\linewidth}}}]{
                \texttt{\tiny{[GM$|$GM]}}
                \texttt{SPRICH: Auf Wiedersehen! Es war mir auch ein Vergnügen, mit dir zu spielen! Ich freue mich darauf, bald wieder ein Spiel mit dir zu spielen und unsere Strategien zu verbessern. Bis bald!} \\
            }
        }
    }
    & & \\ \\

    \theutterance \stepcounter{utterance}  
    & & \multicolumn{4}{p{0.6\linewidth}}{
        \cellcolor[rgb]{0.9,0.9,0.9}{
            \makecell[{{p{\linewidth}}}]{
                \texttt{\tiny{[P1$\langle$GM]}}
                \texttt{Deine Nachricht wurde an deinen Mitspieler übermittelt.} \\
\\ 
\\ 
\texttt{Ihr seid gerade bei Runde 12 von maximal 20.} \\
\texttt{Ihr habt gemeinsam 3 von maximal 12 Strafpunkten angesammelt.} \\
\texttt{Dein Mitspieler hat dir diese Nachricht geschickt:} \\
\\ 
\\ 
\texttt{"Auf Wiedersehen! Es war mir auch ein Vergnügen, mit dir zu spielen! Ich freue mich darauf, bald wieder ein Spiel mit dir zu spielen und unsere Strategien zu verbessern. Bis bald!"} \\
\\ 
\\ 
\texttt{Wie lautet dein nächster Befehl?} \\
            }
        }
    }
    & \\ \\

    \theutterance \stepcounter{utterance}  
    & \multicolumn{4}{p{0.6\linewidth}}{
        \cellcolor[rgb]{0.9,0.9,0.9}{
            \makecell[{{p{\linewidth}}}]{
                \texttt{\tiny{[P1$\rangle$GM]}}
                \texttt{SPRICH: Bis bald! Ich freue mich auch darauf, bald wieder ein Spiel mit dir zu spielen. Ich denke, wir haben eine gute Basis für unsere Strategie geschaffen und können sie noch weiter verbessern. Ich wünsche dir einen schönen Tag!} \\
            }
        }
    }
    & & \\ \\

    \theutterance \stepcounter{utterance}  
    & & & \multicolumn{2}{p{0.3\linewidth}}{
        \cellcolor[rgb]{0.9,0.9,0.9}{
            \makecell[{{p{\linewidth}}}]{
                \texttt{\tiny{[GM$|$GM]}}
                \texttt{SPRICH: Bis bald! Ich freue mich auch darauf, bald wieder ein Spiel mit dir zu spielen. Ich denke, wir haben eine gute Basis für unsere Strategie geschaffen und können sie noch weiter verbessern. Ich wünsche dir einen schönen Tag!} \\
            }
        }
    }
    & & \\ \\

    \theutterance \stepcounter{utterance}  
    & & \multicolumn{4}{p{0.6\linewidth}}{
        \cellcolor[rgb]{0.9,0.9,0.9}{
            \makecell[{{p{\linewidth}}}]{
                \texttt{\tiny{[GM$\rangle$P2]}}
                \texttt{Deine Nachricht wurde an deinen Mitspieler übermittelt.} \\
\\ 
\\ 
\texttt{Ihr seid gerade bei Runde 13 von maximal 20.} \\
\texttt{Ihr habt gemeinsam 3 von maximal 12 Strafpunkten angesammelt.} \\
\texttt{Dein Mitspieler hat dir diese Nachricht geschickt:} \\
\\ 
\\ 
\texttt{"Bis bald! Ich freue mich auch darauf, bald wieder ein Spiel mit dir zu spielen. Ich denke, wir haben eine gute Basis für unsere Strategie geschaffen und können sie noch weiter verbessern. Ich wünsche dir einen schönen Tag!"} \\
\\ 
\\ 
\texttt{Wie lautet dein nächster Befehl?} \\
            }
        }
    }
    & \\ \\

    \theutterance \stepcounter{utterance}  
    & & & \multicolumn{4}{p{0.6\linewidth}}{
        \cellcolor[rgb]{0.9,0.9,0.9}{
            \makecell[{{p{\linewidth}}}]{
                \texttt{\tiny{[GM$\langle$P2]}}
                \texttt{SPRICH: Ich wünsche dir auch einen schönen Tag! Ich bin gespannt darauf, unsere Strategie weiter zu verbessern und bald wieder ein Spiel zu spielen. Bis zum nächsten Mal!} \\
            }
        }
    }
     \\ \\

    \theutterance \stepcounter{utterance}  
    & & & \multicolumn{2}{p{0.3\linewidth}}{
        \cellcolor[rgb]{0.9,0.9,0.9}{
            \makecell[{{p{\linewidth}}}]{
                \texttt{\tiny{[GM$|$GM]}}
                \texttt{SPRICH: Ich wünsche dir auch einen schönen Tag! Ich bin gespannt darauf, unsere Strategie weiter zu verbessern und bald wieder ein Spiel zu spielen. Bis zum nächsten Mal!} \\
            }
        }
    }
    & & \\ \\

    \theutterance \stepcounter{utterance}  
    & & \multicolumn{4}{p{0.6\linewidth}}{
        \cellcolor[rgb]{0.9,0.9,0.9}{
            \makecell[{{p{\linewidth}}}]{
                \texttt{\tiny{[P1$\langle$GM]}}
                \texttt{Deine Nachricht wurde an deinen Mitspieler übermittelt.} \\
\\ 
\\ 
\texttt{Ihr seid gerade bei Runde 13 von maximal 20.} \\
\texttt{Ihr habt gemeinsam 3 von maximal 12 Strafpunkten angesammelt.} \\
\texttt{Dein Mitspieler hat dir diese Nachricht geschickt:} \\
\\ 
\\ 
\texttt{"Ich wünsche dir auch einen schönen Tag! Ich bin gespannt darauf, unsere Strategie weiter zu verbessern und bald wieder ein Spiel zu spielen. Bis zum nächsten Mal!"} \\
\\ 
\\ 
\texttt{Wie lautet dein nächster Befehl?} \\
            }
        }
    }
    & \\ \\

    \theutterance \stepcounter{utterance}  
    & \multicolumn{4}{p{0.6\linewidth}}{
        \cellcolor[rgb]{0.9,0.9,0.9}{
            \makecell[{{p{\linewidth}}}]{
                \texttt{\tiny{[P1$\rangle$GM]}}
                \texttt{SPRICH: Ebenfalls einen schönen Tag! Ich freue mich darauf, bald wieder von dir zu hören und unser nächstes Spiel zu spielen. Bis zum nächsten Mal!} \\
            }
        }
    }
    & & \\ \\

    \theutterance \stepcounter{utterance}  
    & & & \multicolumn{2}{p{0.3\linewidth}}{
        \cellcolor[rgb]{0.9,0.9,0.9}{
            \makecell[{{p{\linewidth}}}]{
                \texttt{\tiny{[GM$|$GM]}}
                \texttt{SPRICH: Ebenfalls einen schönen Tag! Ich freue mich darauf, bald wieder von dir zu hören und unser nächstes Spiel zu spielen. Bis zum nächsten Mal!} \\
            }
        }
    }
    & & \\ \\

    \theutterance \stepcounter{utterance}  
    & & \multicolumn{4}{p{0.6\linewidth}}{
        \cellcolor[rgb]{0.9,0.9,0.9}{
            \makecell[{{p{\linewidth}}}]{
                \texttt{\tiny{[GM$\rangle$P2]}}
                \texttt{Deine Nachricht wurde an deinen Mitspieler übermittelt.} \\
\\ 
\\ 
\texttt{Ihr seid gerade bei Runde 14 von maximal 20.} \\
\texttt{Ihr habt gemeinsam 3 von maximal 12 Strafpunkten angesammelt.} \\
\texttt{Dein Mitspieler hat dir diese Nachricht geschickt:} \\
\\ 
\\ 
\texttt{"Ebenfalls einen schönen Tag! Ich freue mich darauf, bald wieder von dir zu hören und unser nächstes Spiel zu spielen. Bis zum nächsten Mal!"} \\
\\ 
\\ 
\texttt{Wie lautet dein nächster Befehl?} \\
            }
        }
    }
    & \\ \\

    \theutterance \stepcounter{utterance}  
    & & & \multicolumn{4}{p{0.6\linewidth}}{
        \cellcolor[rgb]{0.9,0.9,0.9}{
            \makecell[{{p{\linewidth}}}]{
                \texttt{\tiny{[GM$\langle$P2]}}
                \texttt{SPRICH: Bis zum nächsten Mal! Ich freue mich auch darauf, bald wieder von dir zu hören und unser nächstes Spiel zu spielen. Einen schönen Tag!} \\
            }
        }
    }
     \\ \\

    \theutterance \stepcounter{utterance}  
    & & & \multicolumn{2}{p{0.3\linewidth}}{
        \cellcolor[rgb]{0.9,0.9,0.9}{
            \makecell[{{p{\linewidth}}}]{
                \texttt{\tiny{[GM$|$GM]}}
                \texttt{SPRICH: Bis zum nächsten Mal! Ich freue mich auch darauf, bald wieder von dir zu hören und unser nächstes Spiel zu spielen. Einen schönen Tag!} \\
            }
        }
    }
    & & \\ \\

    \theutterance \stepcounter{utterance}  
    & & \multicolumn{4}{p{0.6\linewidth}}{
        \cellcolor[rgb]{0.9,0.9,0.9}{
            \makecell[{{p{\linewidth}}}]{
                \texttt{\tiny{[P1$\langle$GM]}}
                \texttt{Deine Nachricht wurde an deinen Mitspieler übermittelt.} \\
\\ 
\\ 
\texttt{Ihr seid gerade bei Runde 14 von maximal 20.} \\
\texttt{Ihr habt gemeinsam 3 von maximal 12 Strafpunkten angesammelt.} \\
\texttt{Dein Mitspieler hat dir diese Nachricht geschickt:} \\
\\ 
\\ 
\texttt{"Bis zum nächsten Mal! Ich freue mich auch darauf, bald wieder von dir zu hören und unser nächstes Spiel zu spielen. Einen schönen Tag!"} \\
\\ 
\\ 
\texttt{Wie lautet dein nächster Befehl?} \\
            }
        }
    }
    & \\ \\

    \theutterance \stepcounter{utterance}  
    & \multicolumn{4}{p{0.6\linewidth}}{
        \cellcolor[rgb]{0.9,0.9,0.9}{
            \makecell[{{p{\linewidth}}}]{
                \texttt{\tiny{[P1$\rangle$GM]}}
                \texttt{SPRICH: Einen schönen Tag und bis bald! Ich wünsche dir alles Gute bis zum nächsten Spiel!} \\
            }
        }
    }
    & & \\ \\

    \theutterance \stepcounter{utterance}  
    & & & \multicolumn{2}{p{0.3\linewidth}}{
        \cellcolor[rgb]{0.9,0.9,0.9}{
            \makecell[{{p{\linewidth}}}]{
                \texttt{\tiny{[GM$|$GM]}}
                \texttt{SPRICH: Einen schönen Tag und bis bald! Ich wünsche dir alles Gute bis zum nächsten Spiel!} \\
            }
        }
    }
    & & \\ \\

    \theutterance \stepcounter{utterance}  
    & & \multicolumn{4}{p{0.6\linewidth}}{
        \cellcolor[rgb]{0.9,0.9,0.9}{
            \makecell[{{p{\linewidth}}}]{
                \texttt{\tiny{[GM$\rangle$P2]}}
                \texttt{Deine Nachricht wurde an deinen Mitspieler übermittelt.} \\
\\ 
\\ 
\texttt{Ihr seid gerade bei Runde 15 von maximal 20.} \\
\texttt{Ihr habt gemeinsam 3 von maximal 12 Strafpunkten angesammelt.} \\
\texttt{Dein Mitspieler hat dir diese Nachricht geschickt:} \\
\\ 
\\ 
\texttt{"Einen schönen Tag und bis bald! Ich wünsche dir alles Gute bis zum nächsten Spiel!"} \\
\\ 
\\ 
\texttt{Wie lautet dein nächster Befehl?} \\
            }
        }
    }
    & \\ \\

    \theutterance \stepcounter{utterance}  
    & & & \multicolumn{4}{p{0.6\linewidth}}{
        \cellcolor[rgb]{0.9,0.9,0.9}{
            \makecell[{{p{\linewidth}}}]{
                \texttt{\tiny{[GM$\langle$P2]}}
                \texttt{SPRICH: Danke für die schönen Worte! Ich wünsche dir auch alles Gute und freue mich auf unser nächstes Spiel!} \\
            }
        }
    }
     \\ \\

    \theutterance \stepcounter{utterance}  
    & & & \multicolumn{2}{p{0.3\linewidth}}{
        \cellcolor[rgb]{0.9,0.9,0.9}{
            \makecell[{{p{\linewidth}}}]{
                \texttt{\tiny{[GM$|$GM]}}
                \texttt{SPRICH: Danke für die schönen Worte! Ich wünsche dir auch alles Gute und freue mich auf unser nächstes Spiel!} \\
            }
        }
    }
    & & \\ \\

    \theutterance \stepcounter{utterance}  
    & & \multicolumn{4}{p{0.6\linewidth}}{
        \cellcolor[rgb]{0.9,0.9,0.9}{
            \makecell[{{p{\linewidth}}}]{
                \texttt{\tiny{[P1$\langle$GM]}}
                \texttt{Deine Nachricht wurde an deinen Mitspieler übermittelt.} \\
\\ 
\\ 
\texttt{Ihr seid gerade bei Runde 15 von maximal 20.} \\
\texttt{Ihr habt gemeinsam 3 von maximal 12 Strafpunkten angesammelt.} \\
\texttt{Dein Mitspieler hat dir diese Nachricht geschickt:} \\
\\ 
\\ 
\texttt{"Danke für die schönen Worte! Ich wünsche dir auch alles Gute und freue mich auf unser nächstes Spiel!"} \\
\\ 
\\ 
\texttt{Wie lautet dein nächster Befehl?} \\
            }
        }
    }
    & \\ \\

    \theutterance \stepcounter{utterance}  
    & \multicolumn{4}{p{0.6\linewidth}}{
        \cellcolor[rgb]{0.9,0.9,0.9}{
            \makecell[{{p{\linewidth}}}]{
                \texttt{\tiny{[P1$\rangle$GM]}}
                \texttt{SPRICH: Gern geschehen! Ich freue mich auch auf unser nächstes Spiel und darauf, unsere Strategie weiter zu verbessern. Bis zum nächsten Mal!} \\
            }
        }
    }
    & & \\ \\

    \theutterance \stepcounter{utterance}  
    & & & \multicolumn{2}{p{0.3\linewidth}}{
        \cellcolor[rgb]{0.9,0.9,0.9}{
            \makecell[{{p{\linewidth}}}]{
                \texttt{\tiny{[GM$|$GM]}}
                \texttt{SPRICH: Gern geschehen! Ich freue mich auch auf unser nächstes Spiel und darauf, unsere Strategie weiter zu verbessern. Bis zum nächsten Mal!} \\
            }
        }
    }
    & & \\ \\

    \theutterance \stepcounter{utterance}  
    & & \multicolumn{4}{p{0.6\linewidth}}{
        \cellcolor[rgb]{0.9,0.9,0.9}{
            \makecell[{{p{\linewidth}}}]{
                \texttt{\tiny{[GM$\rangle$P2]}}
                \texttt{Deine Nachricht wurde an deinen Mitspieler übermittelt.} \\
\\ 
\\ 
\texttt{Ihr seid gerade bei Runde 16 von maximal 20.} \\
\texttt{Ihr habt gemeinsam 3 von maximal 12 Strafpunkten angesammelt.} \\
\texttt{Dein Mitspieler hat dir diese Nachricht geschickt:} \\
\\ 
\\ 
\texttt{"Gern geschehen! Ich freue mich auch auf unser nächstes Spiel und darauf, unsere Strategie weiter zu verbessern. Bis zum nächsten Mal!"} \\
\\ 
\\ 
\texttt{Wie lautet dein nächster Befehl?} \\
            }
        }
    }
    & \\ \\

    \theutterance \stepcounter{utterance}  
    & & & \multicolumn{4}{p{0.6\linewidth}}{
        \cellcolor[rgb]{0.9,0.9,0.9}{
            \makecell[{{p{\linewidth}}}]{
                \texttt{\tiny{[GM$\langle$P2]}}
                \texttt{SPRICH: Bis zum nächsten Mal! Ich freue mich darauf, unsere Strategie zu verbessern und ein weiteres spannendes Spiel zu spielen!} \\
            }
        }
    }
     \\ \\

    \theutterance \stepcounter{utterance}  
    & & & \multicolumn{2}{p{0.3\linewidth}}{
        \cellcolor[rgb]{0.9,0.9,0.9}{
            \makecell[{{p{\linewidth}}}]{
                \texttt{\tiny{[GM$|$GM]}}
                \texttt{SPRICH: Bis zum nächsten Mal! Ich freue mich darauf, unsere Strategie zu verbessern und ein weiteres spannendes Spiel zu spielen!} \\
            }
        }
    }
    & & \\ \\

    \theutterance \stepcounter{utterance}  
    & & \multicolumn{4}{p{0.6\linewidth}}{
        \cellcolor[rgb]{0.9,0.9,0.9}{
            \makecell[{{p{\linewidth}}}]{
                \texttt{\tiny{[P1$\langle$GM]}}
                \texttt{Deine Nachricht wurde an deinen Mitspieler übermittelt.} \\
\\ 
\\ 
\texttt{Ihr seid gerade bei Runde 16 von maximal 20.} \\
\texttt{Ihr habt gemeinsam 3 von maximal 12 Strafpunkten angesammelt.} \\
\texttt{Dein Mitspieler hat dir diese Nachricht geschickt:} \\
\\ 
\\ 
\texttt{"Bis zum nächsten Mal! Ich freue mich darauf, unsere Strategie zu verbessern und ein weiteres spannendes Spiel zu spielen!"} \\
\\ 
\\ 
\texttt{Wie lautet dein nächster Befehl?} \\
            }
        }
    }
    & \\ \\

    \theutterance \stepcounter{utterance}  
    & \multicolumn{4}{p{0.6\linewidth}}{
        \cellcolor[rgb]{0.9,0.9,0.9}{
            \makecell[{{p{\linewidth}}}]{
                \texttt{\tiny{[P1$\rangle$GM]}}
                \texttt{SPRICH: Ich freue mich auch darauf! Ich denke, wir haben eine gute Basis für unsere Strategie geschaffen und können sie noch weiter verbessern. Bis zum nächsten Mal!} \\
            }
        }
    }
    & & \\ \\

    \theutterance \stepcounter{utterance}  
    & & & \multicolumn{2}{p{0.3\linewidth}}{
        \cellcolor[rgb]{0.9,0.9,0.9}{
            \makecell[{{p{\linewidth}}}]{
                \texttt{\tiny{[GM$|$GM]}}
                \texttt{SPRICH: Ich freue mich auch darauf! Ich denke, wir haben eine gute Basis für unsere Strategie geschaffen und können sie noch weiter verbessern. Bis zum nächsten Mal!} \\
            }
        }
    }
    & & \\ \\

    \theutterance \stepcounter{utterance}  
    & & \multicolumn{4}{p{0.6\linewidth}}{
        \cellcolor[rgb]{0.9,0.9,0.9}{
            \makecell[{{p{\linewidth}}}]{
                \texttt{\tiny{[GM$\rangle$P2]}}
                \texttt{Deine Nachricht wurde an deinen Mitspieler übermittelt.} \\
\\ 
\\ 
\texttt{Ihr seid gerade bei Runde 17 von maximal 20.} \\
\texttt{Ihr habt gemeinsam 3 von maximal 12 Strafpunkten angesammelt.} \\
\texttt{Dein Mitspieler hat dir diese Nachricht geschickt:} \\
\\ 
\\ 
\texttt{"Ich freue mich auch darauf! Ich denke, wir haben eine gute Basis für unsere Strategie geschaffen und können sie noch weiter verbessern. Bis zum nächsten Mal!"} \\
\\ 
\\ 
\texttt{Wie lautet dein nächster Befehl?} \\
            }
        }
    }
    & \\ \\

    \theutterance \stepcounter{utterance}  
    & & & \multicolumn{4}{p{0.6\linewidth}}{
        \cellcolor[rgb]{0.9,0.9,0.9}{
            \makecell[{{p{\linewidth}}}]{
                \texttt{\tiny{[GM$\langle$P2]}}
                \texttt{SPRICH: Ich stimme dir zu, wir haben eine gute Basis geschaffen und können sie weiter verbessern. Bis zum nächsten Mal!} \\
            }
        }
    }
     \\ \\

    \theutterance \stepcounter{utterance}  
    & & & \multicolumn{2}{p{0.3\linewidth}}{
        \cellcolor[rgb]{0.9,0.9,0.9}{
            \makecell[{{p{\linewidth}}}]{
                \texttt{\tiny{[GM$|$GM]}}
                \texttt{SPRICH: Ich stimme dir zu, wir haben eine gute Basis geschaffen und können sie weiter verbessern. Bis zum nächsten Mal!} \\
            }
        }
    }
    & & \\ \\

    \theutterance \stepcounter{utterance}  
    & & \multicolumn{4}{p{0.6\linewidth}}{
        \cellcolor[rgb]{0.9,0.9,0.9}{
            \makecell[{{p{\linewidth}}}]{
                \texttt{\tiny{[P1$\langle$GM]}}
                \texttt{Deine Nachricht wurde an deinen Mitspieler übermittelt.} \\
\\ 
\\ 
\texttt{Ihr seid gerade bei Runde 17 von maximal 20.} \\
\texttt{Ihr habt gemeinsam 3 von maximal 12 Strafpunkten angesammelt.} \\
\texttt{Dein Mitspieler hat dir diese Nachricht geschickt:} \\
\\ 
\\ 
\texttt{"Ich stimme dir zu, wir haben eine gute Basis geschaffen und können sie weiter verbessern. Bis zum nächsten Mal!"} \\
\\ 
\\ 
\texttt{Wie lautet dein nächster Befehl?} \\
            }
        }
    }
    & \\ \\

    \theutterance \stepcounter{utterance}  
    & \multicolumn{4}{p{0.6\linewidth}}{
        \cellcolor[rgb]{0.9,0.9,0.9}{
            \makecell[{{p{\linewidth}}}]{
                \texttt{\tiny{[P1$\rangle$GM]}}
                \texttt{SPRICH: Perfekt! Ich bin froh, dass wir uns einig sind. Ich wünsche dir einen schönen Tag und freue mich auf unser nächstes Spiel!} \\
            }
        }
    }
    & & \\ \\

    \theutterance \stepcounter{utterance}  
    & & & \multicolumn{2}{p{0.3\linewidth}}{
        \cellcolor[rgb]{0.9,0.9,0.9}{
            \makecell[{{p{\linewidth}}}]{
                \texttt{\tiny{[GM$|$GM]}}
                \texttt{SPRICH: Perfekt! Ich bin froh, dass wir uns einig sind. Ich wünsche dir einen schönen Tag und freue mich auf unser nächstes Spiel!} \\
            }
        }
    }
    & & \\ \\

    \theutterance \stepcounter{utterance}  
    & & \multicolumn{4}{p{0.6\linewidth}}{
        \cellcolor[rgb]{0.9,0.9,0.9}{
            \makecell[{{p{\linewidth}}}]{
                \texttt{\tiny{[GM$\rangle$P2]}}
                \texttt{Deine Nachricht wurde an deinen Mitspieler übermittelt.} \\
\\ 
\\ 
\texttt{Ihr seid gerade bei Runde 18 von maximal 20.} \\
\texttt{Ihr habt gemeinsam 3 von maximal 12 Strafpunkten angesammelt.} \\
\texttt{Dein Mitspieler hat dir diese Nachricht geschickt:} \\
\\ 
\\ 
\texttt{"Perfekt! Ich bin froh, dass wir uns einig sind. Ich wünsche dir einen schönen Tag und freue mich auf unser nächstes Spiel!"} \\
\\ 
\\ 
\texttt{Wie lautet dein nächster Befehl?} \\
            }
        }
    }
    & \\ \\

    \theutterance \stepcounter{utterance}  
    & & & \multicolumn{4}{p{0.6\linewidth}}{
        \cellcolor[rgb]{0.9,0.9,0.9}{
            \makecell[{{p{\linewidth}}}]{
                \texttt{\tiny{[GM$\langle$P2]}}
                \texttt{SPRICH: Ich wünsche dir auch einen schönen Tag und freue mich auf unser nächstes Spiel! Bis zum nächsten Mal!} \\
            }
        }
    }
     \\ \\

    \theutterance \stepcounter{utterance}  
    & & & \multicolumn{2}{p{0.3\linewidth}}{
        \cellcolor[rgb]{0.9,0.9,0.9}{
            \makecell[{{p{\linewidth}}}]{
                \texttt{\tiny{[GM$|$GM]}}
                \texttt{SPRICH: Ich wünsche dir auch einen schönen Tag und freue mich auf unser nächstes Spiel! Bis zum nächsten Mal!} \\
            }
        }
    }
    & & \\ \\

    \theutterance \stepcounter{utterance}  
    & & \multicolumn{4}{p{0.6\linewidth}}{
        \cellcolor[rgb]{0.9,0.9,0.9}{
            \makecell[{{p{\linewidth}}}]{
                \texttt{\tiny{[P1$\langle$GM]}}
                \texttt{Deine Nachricht wurde an deinen Mitspieler übermittelt.} \\
\\ 
\\ 
\texttt{Ihr seid gerade bei Runde 18 von maximal 20.} \\
\texttt{Ihr habt gemeinsam 3 von maximal 12 Strafpunkten angesammelt.} \\
\texttt{Dein Mitspieler hat dir diese Nachricht geschickt:} \\
\\ 
\\ 
\texttt{"Ich wünsche dir auch einen schönen Tag und freue mich auf unser nächstes Spiel! Bis zum nächsten Mal!"} \\
\\ 
\\ 
\texttt{Wie lautet dein nächster Befehl?} \\
            }
        }
    }
    & \\ \\

    \theutterance \stepcounter{utterance}  
    & \multicolumn{4}{p{0.6\linewidth}}{
        \cellcolor[rgb]{0.9,0.9,0.9}{
            \makecell[{{p{\linewidth}}}]{
                \texttt{\tiny{[P1$\rangle$GM]}}
                \texttt{SPRICH: Bis zum nächsten Mal! Ich wünsche dir alles Gute und freue mich auf unser nächstes Abenteuer!} \\
            }
        }
    }
    & & \\ \\

    \theutterance \stepcounter{utterance}  
    & & & \multicolumn{2}{p{0.3\linewidth}}{
        \cellcolor[rgb]{0.9,0.9,0.9}{
            \makecell[{{p{\linewidth}}}]{
                \texttt{\tiny{[GM$|$GM]}}
                \texttt{SPRICH: Bis zum nächsten Mal! Ich wünsche dir alles Gute und freue mich auf unser nächstes Abenteuer!} \\
            }
        }
    }
    & & \\ \\

    \theutterance \stepcounter{utterance}  
    & & \multicolumn{4}{p{0.6\linewidth}}{
        \cellcolor[rgb]{0.9,0.9,0.9}{
            \makecell[{{p{\linewidth}}}]{
                \texttt{\tiny{[GM$\rangle$P2]}}
                \texttt{Deine Nachricht wurde an deinen Mitspieler übermittelt.} \\
\\ 
\\ 
\texttt{Ihr seid gerade bei Runde 19 von maximal 20.} \\
\texttt{Ihr habt gemeinsam 3 von maximal 12 Strafpunkten angesammelt.} \\
\texttt{Dein Mitspieler hat dir diese Nachricht geschickt:} \\
\\ 
\\ 
\texttt{"Bis zum nächsten Mal! Ich wünsche dir alles Gute und freue mich auf unser nächstes Abenteuer!"} \\
\\ 
\\ 
\texttt{Wie lautet dein nächster Befehl?} \\
            }
        }
    }
    & \\ \\

    \theutterance \stepcounter{utterance}  
    & & & \multicolumn{4}{p{0.6\linewidth}}{
        \cellcolor[rgb]{0.9,0.9,0.9}{
            \makecell[{{p{\linewidth}}}]{
                \texttt{\tiny{[GM$\langle$P2]}}
                \texttt{SPRICH: Ich wünsche dir auch alles Gute und freue mich auf unser nächstes Abenteuer! Bis zum nächsten Mal!} \\
            }
        }
    }
     \\ \\

    \theutterance \stepcounter{utterance}  
    & & & \multicolumn{2}{p{0.3\linewidth}}{
        \cellcolor[rgb]{0.9,0.9,0.9}{
            \makecell[{{p{\linewidth}}}]{
                \texttt{\tiny{[GM$|$GM]}}
                \texttt{SPRICH: Ich wünsche dir auch alles Gute und freue mich auf unser nächstes Abenteuer! Bis zum nächsten Mal!} \\
            }
        }
    }
    & & \\ \\

    \theutterance \stepcounter{utterance}  
    & & \multicolumn{4}{p{0.6\linewidth}}{
        \cellcolor[rgb]{0.9,0.9,0.9}{
            \makecell[{{p{\linewidth}}}]{
                \texttt{\tiny{[P1$\langle$GM]}}
                \texttt{Deine Nachricht wurde an deinen Mitspieler übermittelt.} \\
\\ 
\\ 
\texttt{Ihr seid gerade bei Runde 19 von maximal 20.} \\
\texttt{Ihr habt gemeinsam 3 von maximal 12 Strafpunkten angesammelt.} \\
\texttt{Dein Mitspieler hat dir diese Nachricht geschickt:} \\
\\ 
\\ 
\texttt{"Ich wünsche dir auch alles Gute und freue mich auf unser nächstes Abenteuer! Bis zum nächsten Mal!"} \\
\\ 
\\ 
\texttt{Wie lautet dein nächster Befehl?} \\
            }
        }
    }
    & \\ \\

    \theutterance \stepcounter{utterance}  
    & \multicolumn{4}{p{0.6\linewidth}}{
        \cellcolor[rgb]{0.9,0.9,0.9}{
            \makecell[{{p{\linewidth}}}]{
                \texttt{\tiny{[P1$\rangle$GM]}}
                \texttt{SPRICH: Bis zum nächsten Mal! Ich bin froh, dass wir uns so gut verstanden haben und ich freue mich auf unsere nächste Begegnung!} \\
            }
        }
    }
    & & \\ \\

    \theutterance \stepcounter{utterance}  
    & & & \multicolumn{2}{p{0.3\linewidth}}{
        \cellcolor[rgb]{0.9,0.9,0.9}{
            \makecell[{{p{\linewidth}}}]{
                \texttt{\tiny{[GM$|$GM]}}
                \texttt{SPRICH: Bis zum nächsten Mal! Ich bin froh, dass wir uns so gut verstanden haben und ich freue mich auf unsere nächste Begegnung!} \\
            }
        }
    }
    & & \\ \\

    \theutterance \stepcounter{utterance}  
    & & \multicolumn{4}{p{0.6\linewidth}}{
        \cellcolor[rgb]{0.9,0.9,0.9}{
            \makecell[{{p{\linewidth}}}]{
                \texttt{\tiny{[GM$\rangle$P2]}}
                \texttt{Deine Nachricht wurde an deinen Mitspieler übermittelt.} \\
\\ 
\\ 
\texttt{Ihr seid gerade bei Runde 20 von maximal 20.} \\
\texttt{Ihr habt gemeinsam 3 von maximal 12 Strafpunkten angesammelt.} \\
\texttt{Dein Mitspieler hat dir diese Nachricht geschickt:} \\
\\ 
\\ 
\texttt{"Bis zum nächsten Mal! Ich bin froh, dass wir uns so gut verstanden haben und ich freue mich auf unsere nächste Begegnung!"} \\
\\ 
\\ 
\texttt{Wie lautet dein nächster Befehl?} \\
            }
        }
    }
    & \\ \\

    \theutterance \stepcounter{utterance}  
    & & & \multicolumn{4}{p{0.6\linewidth}}{
        \cellcolor[rgb]{0.9,0.9,0.9}{
            \makecell[{{p{\linewidth}}}]{
                \texttt{\tiny{[GM$\langle$P2]}}
                \texttt{SPRICH: Ebenfalls bis zum nächsten Mal! Ich bin auch froh über unsere gute Zusammenarbeit und freue mich auf unsere nächste Begegnung!} \\
            }
        }
    }
     \\ \\

    \theutterance \stepcounter{utterance}  
    & & & \multicolumn{2}{p{0.3\linewidth}}{
        \cellcolor[rgb]{0.9,0.9,0.9}{
            \makecell[{{p{\linewidth}}}]{
                \texttt{\tiny{[GM$|$GM]}}
                \texttt{SPRICH: Ebenfalls bis zum nächsten Mal! Ich bin auch froh über unsere gute Zusammenarbeit und freue mich auf unsere nächste Begegnung!} \\
            }
        }
    }
    & & \\ \\

    \theutterance \stepcounter{utterance}  
    & & & \multicolumn{2}{p{0.3\linewidth}}{
        \cellcolor[rgb]{0.9,0.9,0.9}{
            \makecell[{{p{\linewidth}}}]{
                \texttt{\tiny{[GM$|$GM]}}
                \texttt{Maximum number of rounds reached} \\
            }
        }
    }
    & & \\ \\

    \theutterance \stepcounter{utterance}  
    & & & \multicolumn{2}{p{0.3\linewidth}}{
        \cellcolor[rgb]{0.9,0.9,0.9}{
            \makecell[{{p{\linewidth}}}]{
                \texttt{\tiny{[GM$|$GM]}}
                \texttt{* success: True} \\
\texttt{* lose: False} \\
\texttt{* aborted: False} \\
\texttt{{-}{-}{-}{-}{-}{-}{-}} \\
\texttt{* Shifts: 6.00} \\
\texttt{* Max Shifts: 8.00} \\
\texttt{* Min Shifts: 4.00} \\
\texttt{* End Distance Sum: 4.12} \\
\texttt{* Init Distance Sum: 18.69} \\
\texttt{* Expected Distance Sum: 20.95} \\
\texttt{* Penalties: 3.00} \\
\texttt{* Max Penalties: 12.00} \\
\texttt{* Rounds: 20.00} \\
\texttt{* Max Rounds: 20.00} \\
\texttt{* Object Count: 5.00} \\
            }
        }
    }
    & & \\ \\

    \theutterance \stepcounter{utterance}  
    & & & \multicolumn{2}{p{0.3\linewidth}}{
        \cellcolor[rgb]{0.9,0.9,0.9}{
            \makecell[{{p{\linewidth}}}]{
                \texttt{\tiny{[GM$|$GM]}}
                \texttt{Initial Grids:} \\
\texttt{```} \\
\texttt{Player 1:     Player 2:} \\
\texttt{ 1234567       1234567} \\
\texttt{╔═══════╗     ╔═══════╗} \\
\texttt{║I◌◌◌◌◌◌║ 1   ║◌◌H◌◌◌◌║ 1} \\
\texttt{║◌◌◌◌◌◌◌║ 2   ║◌◌◌C◌◌◌║ 2} \\
\texttt{║◌◌◌T◌┌─╢ 3   ║◌◌◌◌◌┌─╢ 3} \\
\texttt{║◌◌◌┌─┤◌║ 4   ║◌◌◌┌─┤◌║ 4} \\
\texttt{║◌◌W│C├─╢ 5   ║◌IW│◌├─╢ 5} \\
\texttt{╟───┼─┘H║ 6   ╟───┼─┘◌║ 6} \\
\texttt{║◌◌◌│◌◌◌║ 7   ║T◌◌│◌◌◌║ 7} \\
\texttt{╚═══╧═══╝     ╚═══╧═══╝} \\
\texttt{```} \\
\\ 
\texttt{Final Grids:} \\
\texttt{```} \\
\texttt{Player 1:     Player 2:} \\
\texttt{ 1234567       1234567} \\
\texttt{╔═══════╗     ╔═══════╗} \\
\texttt{║IWTCH◌◌║ 1   ║I◌TCH◌◌║ 1} \\
\texttt{║◌◌◌◌◌◌◌║ 2   ║◌◌◌◌◌◌◌║ 2} \\
\texttt{║◌◌◌◌◌┌─╢ 3   ║◌◌◌◌◌┌─╢ 3} \\
\texttt{║◌◌◌┌─┤◌║ 4   ║◌◌◌┌─┤◌║ 4} \\
\texttt{║◌◌◌│◌├─╢ 5   ║◌◌W│◌├─╢ 5} \\
\texttt{╟───┼─┘◌║ 6   ╟───┼─┘◌║ 6} \\
\texttt{║◌◌◌│◌◌◌║ 7   ║◌◌◌│◌◌◌║ 7} \\
\texttt{╚═══╧═══╝     ╚═══╧═══╝} \\
\texttt{```} \\
            }
        }
    }
    & & \\ \\

\end{supertabular}
}

\end{document}
