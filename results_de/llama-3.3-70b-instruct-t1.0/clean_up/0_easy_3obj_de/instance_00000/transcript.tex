
\documentclass{article}
\usepackage{colortbl}
\usepackage{makecell}
\usepackage{multirow}
\usepackage{supertabular}

\begin{document}

\newcounter{utterance}

\twocolumn

{ \footnotesize  \setcounter{utterance}{1}
\setlength{\tabcolsep}{0pt}
\begin{supertabular}{c@{$\;$}|p{.15\linewidth}@{}p{.15\linewidth}p{.15\linewidth}p{.15\linewidth}p{.15\linewidth}p{.15\linewidth}}

    \# & $\;$A & \multicolumn{4}{c}{Game Master} & $\;\:$B\\
    \hline 

    \theutterance \stepcounter{utterance}  

    & & \multicolumn{4}{p{0.6\linewidth}}{\cellcolor[rgb]{0.9,0.9,0.9}{%
	\makecell[{{p{\linewidth}}}]{% 
	  \tt {\tiny [A$\langle$GM]}  
	 Ich bin der Spielleiter, und du spielst ein kooperatives Spiel mit dem folgenden Raster als Spielbrett:\\ \tt \\ \tt ```\\ \tt  1234567\\ \tt ╔═╤═══╤═╗\\ \tt ║◌│◌◌L│◌║ 1\\ \tt ╟─┼──┬┘◌║ 2\\ \tt ║◌│◌P│◌◌║ 3\\ \tt ╟─┴──┘◌◌║ 4\\ \tt ║◌◌◌◌◌C◌║ 5\\ \tt ║◌◌◌◌◌◌◌║ 6\\ \tt ║◌◌◌◌◌◌◌║ 7\\ \tt ╚═══════╝\\ \tt \\ \tt ```\\ \tt \\ \tt * Am oberen Rand sind die x-Koordinaten angegeben, die nach rechts hin zunehmen, und am rechten Rand die y-Koordinaten, die nach unten zunehmen.\\ \tt * Die folgenden Objekte sind zufällig auf deinem Raster platziert: 'C', 'L', 'P'.\\ \tt \\ \tt \\ \tt Dein Mitspieler sieht eine Variation des Spielbretts, auf dem dieselben Objekte an anderen zufälligen Positionen platziert sind. Du kannst das Spielbrett des anderen Spielers nicht sehen, und er kann deins nicht sehen.\\ \tt \\ \tt \\ \tt **Ziel des Spiels:**\\ \tt \\ \tt Beide Spieler müssen die Objekte auf ihrem jeweiligen Spielbrett so bewegen, dass identische Objekte am Ende an denselben Koordinaten liegen. Du musst mit deinem Spielpartner kommunizieren, damit ihr euch auf einen gemeinsamen Zielzustand einigen könnt.\\ \tt \\ \tt \\ \tt **Spielregeln:**\\ \tt \\ \tt * In jedem Spielzug kannst du genau eine der folgenden zwei Aktionen ausführen:\\ \tt 1. `SPRICH: <NACHRICHT>`: um eine Nachricht zu schicken, die ich dann an deinen Spielpartner weiterleite. Als Nachricht gilt alles bis zum nächsten Zeilenumbruch.\\ \tt 2. `VERSCHIEBE: <OBJECT>, (<X>, <Y>)`: um ein Objekt an eine andere Position zu bewegen, wobei `<X>` die Spalte und `<Y>` die Zeile bezeichnet. Ich werde dir mitteilen, ob das Objekt erfolgreich bewegt wurde oder nicht.\\ \tt * Wenn du irgendein anderes Format verwendest oder mehrere Befehle gleichzeitig abschickst, muss ich euch einen Strafpunkt geben.\\ \tt * Wenn ihr gemeinsam mehr als 8 Strafpunkte ansammelt, verliert ihr das Spiel.\\ \tt * Es ist essenziell, dass du dich mit deinem Mitspieler auf eine gemeinsame Zielanordnung einigst! Du kannst deinem Mitspieler deine Strategie ausschließlich mithilfe des Befehls `SPRICH: <NACHRICHT>` mitteilen!\\ \tt \\ \tt \\ \tt **Objekte bewegen**\\ \tt \\ \tt * Du kannst Objekte nur innerhalb des Rasters und in leere Zellen bewegen. Leere Zellen erkennst am Zeichen '◌'.\\ \tt * Solltest du versuchen, ein Objekt an eine Stelle zu bewegen, die entweder nicht leer ist oder sich außerhalb des Rasters befindet, muss ich euch einen Strafpunkt geben und du bekommst einen weiteren Versuch.\\ \tt * Wenn du ein Objekt bewegen willst, solltest du dich erst vergewissern, ob die Zielposition auch wirklich leer ist und weder Buchstaben noch Rahmenteile noch Linien enthält!\\ \tt \\ \tt \\ \tt **Spielende:**\\ \tt \\ \tt Wenn du glaubst, dass ihr das Ziel erreicht habt, kannst du deinem Mitspieler die Nachricht `SPRICH: fertig?` schicken. Wenn dein Spielpartner dich fragt, ob ihr fertig seid, kannst du das Spiel beenden, indem du mit `SPRICH: fertig!` antwortest.\\ \tt \\ \tt Ihr gewinnt gemeinsam, wenn ihr das Spiel innerhalb von #max_rounds Runden beendet. Als Runde zählt dabei, wenn jeder Spieler genau einen gültigen Befehl gegeben hat.\\ \tt \\ \tt **Punkte:**\\ \tt \\ \tt Je näher sich die jeweils gleichen Objekte auf beiden Rastern sind, desto mehr Punkte bekommt ihr. Für Strafpunkte gibt es Punktabzug. Könnt ihr den Rekord brechen?\\ \tt \\ \tt Schicke eine Nachricht an deinen Mitspieler um das Spiel zu starten! 
	  } 
	   } 
	   } 
	 & \\ 
 

    \theutterance \stepcounter{utterance}  

    & \multicolumn{4}{p{0.6\linewidth}}{\cellcolor[rgb]{0.8,1,0.9}{%
	\makecell[{{p{\linewidth}}}]{% 
	  \tt {\tiny [A$\rangle$GM]}  
	 SPRICH: Hallo! Lass uns das Spiel beginnen. Mein Raster sieht so aus:\\ \tt \\ \tt ```\\ \tt  1234567\\ \tt ╔═╤═══╤═╗\\ \tt ║◌│◌◌L│◌║ 1\\ \tt ╟─┼──┬┘◌║ 2\\ \tt ║◌│◌P│◌◌║ 3\\ \tt ╟─┴──┘◌◌║ 4\\ \tt ║◌◌◌◌◌C◌║ 5\\ \tt ║◌◌◌◌◌◌◌║ 6\\ \tt ║◌◌◌◌◌◌◌║ 7\\ \tt ╚═══════╝\\ \tt ```\\ \tt \\ \tt Ich schlage vor, dass wir uns auf eine gemeinsame Zielanordnung einigen. Wie sieht dein Raster aus? Hast du Vorschläge für die Zielanordnung? 
	  } 
	   } 
	   } 
	 & & \\ 
 

    \theutterance \stepcounter{utterance}  

    & & & \multicolumn{2}{p{0.3\linewidth}}{\cellcolor[rgb]{0.95,0.95,0.95}{%
	\makecell[{{p{\linewidth}}}]{% 
	  \tt {\tiny [GM$|$GM]}  
	 SPRICH: Hallo! Lass uns das Spiel beginnen. Mein Raster sieht so aus:\\ \tt \\ \tt ```\\ \tt  1234567\\ \tt ╔═╤═══╤═╗\\ \tt ║◌│◌◌L│◌║ 1\\ \tt ╟─┼──┬┘◌║ 2\\ \tt ║◌│◌P│◌◌║ 3\\ \tt ╟─┴──┘◌◌║ 4\\ \tt ║◌◌◌◌◌C◌║ 5\\ \tt ║◌◌◌◌◌◌◌║ 6\\ \tt ║◌◌◌◌◌◌◌║ 7\\ \tt ╚═══════╝\\ \tt ```\\ \tt \\ \tt Ich schlage vor, dass wir uns auf eine gemeinsame Zielanordnung einigen. Wie sieht dein Raster aus? Hast du Vorschläge für die Zielanordnung? 
	  } 
	   } 
	   } 
	 & & \\ 
 

    \theutterance \stepcounter{utterance}  

    & & & \multicolumn{2}{p{0.3\linewidth}}{\cellcolor[rgb]{0.95,0.95,0.95}{%
	\makecell[{{p{\linewidth}}}]{% 
	  \tt {\tiny [GM$|$GM]}  
	 Invalid format: tail is not empty: '\\ \tt \\ \tt \\ \tt  1234567\\ \tt ╔═╤═══╤═╗\\ \tt ║◌│◌◌L│◌║ 1\\ \tt ╟─┼──┬┘◌║ 2\\ \tt ║◌│◌P│◌◌║ 3\\ \tt ╟─┴──┘◌◌║ 4\\ \tt ║◌◌◌◌◌C◌║ 5\\ \tt ║◌◌◌◌◌◌◌║ 6\\ \tt ║◌◌◌◌◌◌◌║ 7\\ \tt ╚═══════╝\\ \tt \\ \tt \\ \tt Ich schlage vor, dass wir uns auf eine gemeinsame Zielanordnung einigen. Wie sieht dein Raster aus? Hast du Vorschläge für die Zielanordnung? 
	  } 
	   } 
	   } 
	 & & \\ 
 

    \theutterance \stepcounter{utterance}  

    & & \multicolumn{4}{p{0.6\linewidth}}{\cellcolor[rgb]{0.9,0.9,0.9}{%
	\makecell[{{p{\linewidth}}}]{% 
	  \tt {\tiny [A$\langle$GM]}  
	 Strafpunkt: Deine Nachricht darf nichts nach dem Befehl enthalten! Stelle sicher, dass deine Antwort ausschließlich einen der beiden Befehle `SPRICH: <NACHRICHT>` und `VERSCHIEBE: <OBJECT>, (<X>, <Y>)` und sonst nichts enthält!\\ \tt \\ \tt Ihr habt gemeinsam 1 von maximal 8 Strafpunkten angesammelt.\\ \tt Bitte versuche es noch mal! 
	  } 
	   } 
	   } 
	 & \\ 
 

    \theutterance \stepcounter{utterance}  

    & \multicolumn{4}{p{0.6\linewidth}}{\cellcolor[rgb]{0.8,1,0.9}{%
	\makecell[{{p{\linewidth}}}]{% 
	  \tt {\tiny [A$\rangle$GM]}  
	 SPRICH: Hallo, lass uns das Spiel beginnen und uns auf eine gemeinsame Zielanordnung einigen. 
	  } 
	   } 
	   } 
	 & & \\ 
 

    \theutterance \stepcounter{utterance}  

    & & & \multicolumn{2}{p{0.3\linewidth}}{\cellcolor[rgb]{0.95,0.95,0.95}{%
	\makecell[{{p{\linewidth}}}]{% 
	  \tt {\tiny [GM$|$GM]}  
	 SPRICH: Hallo, lass uns das Spiel beginnen und uns auf eine gemeinsame Zielanordnung einigen. 
	  } 
	   } 
	   } 
	 & & \\ 
 

    \theutterance \stepcounter{utterance}  

    & & \multicolumn{4}{p{0.6\linewidth}}{\cellcolor[rgb]{0.9,0.9,0.9}{%
	\makecell[{{p{\linewidth}}}]{% 
	  \tt {\tiny [GM$\rangle$B]}  
	 Ich bin der Spielleiter, und du spielst ein kooperatives Spiel mit dem folgenden Raster als Spielbrett:\\ \tt \\ \tt ```\\ \tt  1234567\\ \tt ╔═╤═══╤═╗\\ \tt ║◌│◌C◌│◌║ 1\\ \tt ╟─┼──┬┘◌║ 2\\ \tt ║◌│◌◌│◌◌║ 3\\ \tt ╟─┴──┘◌◌║ 4\\ \tt ║◌◌◌◌◌◌◌║ 5\\ \tt ║◌◌◌◌◌◌◌║ 6\\ \tt ║◌◌◌P◌◌L║ 7\\ \tt ╚═══════╝\\ \tt \\ \tt ```\\ \tt \\ \tt * Am oberen Rand sind die x-Koordinaten angegeben, die nach rechts hin zunehmen, und am rechten Rand die y-Koordinaten, die nach unten zunehmen.\\ \tt * Die folgenden Objekte sind zufällig auf deinem Raster platziert: 'C', 'L', 'P'.\\ \tt \\ \tt \\ \tt Dein Mitspieler sieht eine Variation des Spielbretts, auf dem dieselben Objekte an anderen zufälligen Positionen platziert sind. Du kannst das Spielbrett des anderen Spielers nicht sehen, und er kann deins nicht sehen.\\ \tt \\ \tt \\ \tt **Ziel des Spiels:**\\ \tt \\ \tt Beide Spieler müssen die Objekte auf ihrem jeweiligen Spielbrett so bewegen, dass identische Objekte am Ende an denselben Koordinaten liegen. Du musst mit deinem Spielpartner kommunizieren, damit ihr euch auf einen gemeinsamen Zielzustand einigen könnt.\\ \tt \\ \tt \\ \tt **Spielregeln:**\\ \tt \\ \tt * In jedem Spielzug kannst du genau eine der folgenden zwei Aktionen ausführen:\\ \tt 1. `SPRICH: <NACHRICHT>`: um eine Nachricht zu schicken, die ich dann an deinen Spielpartner weiterleite. Als Nachricht gilt alles bis zum nächsten Zeilenumbruch.\\ \tt 2. `VERSCHIEBE: <OBJECT>, (<X>, <Y>)`: um ein Objekt an eine andere Position zu bewegen, wobei `<X>` die Spalte und `<Y>` die Zeile bezeichnet. Ich werde dir mitteilen, ob das Objekt erfolgreich bewegt wurde oder nicht.\\ \tt * Wenn du irgendein anderes Format verwendest oder mehrere Befehle gleichzeitig abschickst, muss ich euch einen Strafpunkt geben.\\ \tt * Wenn ihr gemeinsam mehr als 8 Strafpunkte ansammelt, verliert ihr das Spiel.\\ \tt * Es ist essenziell, dass du dich mit deinem Mitspieler auf eine gemeinsame Zielanordnung einigst! Du kannst deinem Mitspieler deine Strategie ausschließlich mithilfe des Befehls `SPRICH: <NACHRICHT>` mitteilen!\\ \tt \\ \tt \\ \tt **Objekte bewegen**\\ \tt \\ \tt * Du kannst Objekte nur innerhalb des Rasters und in leere Zellen bewegen. Leere Zellen erkennst am Zeichen '◌'.\\ \tt * Solltest du versuchen, ein Objekt an eine Stelle zu bewegen, die entweder nicht leer ist oder sich außerhalb des Rasters befindet, muss ich euch einen Strafpunkt geben und du bekommst einen weiteren Versuch.\\ \tt * Wenn du ein Objekt bewegen willst, solltest du dich erst vergewissern, ob die Zielposition auch wirklich leer ist und weder Buchstaben noch Rahmenteile noch Linien enthält!\\ \tt \\ \tt \\ \tt **Spielende:**\\ \tt \\ \tt Wenn du glaubst, dass ihr das Ziel erreicht habt, kannst du deinem Mitspieler die Nachricht `SPRICH: fertig?` schicken. Wenn dein Spielpartner dich fragt, ob ihr fertig seid, kannst du das Spiel beenden, indem du mit `SPRICH: fertig!` antwortest.\\ \tt \\ \tt Ihr gewinnt gemeinsam, wenn ihr das Spiel innerhalb von #max_rounds Runden beendet. Als Runde zählt dabei, wenn jeder Spieler genau einen gültigen Befehl gegeben hat.\\ \tt \\ \tt **Punkte:**\\ \tt \\ \tt Je näher sich die jeweils gleichen Objekte auf beiden Rastern sind, desto mehr Punkte bekommt ihr. Für Strafpunkte gibt es Punktabzug. Könnt ihr den Rekord brechen?\\ \tt \\ \tt Dein Mitspieler hat das Spiel mit dieser Nachricht begonnen:\\ \tt \\ \tt "Hallo, lass uns das Spiel beginnen und uns auf eine gemeinsame Zielanordnung einigen."\\ \tt \\ \tt Wie lautet dein erster Befehl? 
	  } 
	   } 
	   } 
	 & \\ 
 

    \theutterance \stepcounter{utterance}  

    & & & \multicolumn{4}{p{0.6\linewidth}}{\cellcolor[rgb]{1,0.85,0.72}{%
	\makecell[{{p{\linewidth}}}]{% 
	  \tt {\tiny [GM$\langle$B]}  
	 SPRICH: Hallo, ich schlage vor, dass wir alle Objekte in der mittleren Spalte platzieren, 'C' in der ersten Reihe, 'L' in der mittleren Reihe und 'P' in der letzten Reihe. Wie gefällt dir das? 
	  } 
	   } 
	   } 
	  \\ 
 

    \theutterance \stepcounter{utterance}  

    & & & \multicolumn{2}{p{0.3\linewidth}}{\cellcolor[rgb]{0.95,0.95,0.95}{%
	\makecell[{{p{\linewidth}}}]{% 
	  \tt {\tiny [GM$|$GM]}  
	 SPRICH: Hallo, ich schlage vor, dass wir alle Objekte in der mittleren Spalte platzieren, 'C' in der ersten Reihe, 'L' in der mittleren Reihe und 'P' in der letzten Reihe. Wie gefällt dir das? 
	  } 
	   } 
	   } 
	 & & \\ 
 

    \theutterance \stepcounter{utterance}  

    & & \multicolumn{4}{p{0.6\linewidth}}{\cellcolor[rgb]{0.9,0.9,0.9}{%
	\makecell[{{p{\linewidth}}}]{% 
	  \tt {\tiny [A$\langle$GM]}  
	 Deine Nachricht wurde an deinen Mitspieler übermittelt.\\ \tt \\ \tt \\ \tt Ihr seid gerade bei Runde 1 von maximal 12.\\ \tt Ihr habt gemeinsam 1 von maximal 8 Strafpunkten angesammelt.\\ \tt Dein Mitspieler hat dir diese Nachricht geschickt:\\ \tt \\ \tt \\ \tt "Hallo, ich schlage vor, dass wir alle Objekte in der mittleren Spalte platzieren, 'C' in der ersten Reihe, 'L' in der mittleren Reihe und 'P' in der letzten Reihe. Wie gefällt dir das?"\\ \tt \\ \tt \\ \tt Wie lautet dein nächster Befehl? 
	  } 
	   } 
	   } 
	 & \\ 
 

    \theutterance \stepcounter{utterance}  

    & \multicolumn{4}{p{0.6\linewidth}}{\cellcolor[rgb]{0.8,1,0.9}{%
	\makecell[{{p{\linewidth}}}]{% 
	  \tt {\tiny [A$\rangle$GM]}  
	 SPRICH: Das klingt gut, ich stimme deinem Vorschlag zu, 'C' in der ersten Reihe, 'L' in der mittleren Reihe und 'P' in der letzten Reihe in der mittleren Spalte zu platzieren. Lass uns loslegen! 
	  } 
	   } 
	   } 
	 & & \\ 
 

    \theutterance \stepcounter{utterance}  

    & & & \multicolumn{2}{p{0.3\linewidth}}{\cellcolor[rgb]{0.95,0.95,0.95}{%
	\makecell[{{p{\linewidth}}}]{% 
	  \tt {\tiny [GM$|$GM]}  
	 SPRICH: Das klingt gut, ich stimme deinem Vorschlag zu, 'C' in der ersten Reihe, 'L' in der mittleren Reihe und 'P' in der letzten Reihe in der mittleren Spalte zu platzieren. Lass uns loslegen! 
	  } 
	   } 
	   } 
	 & & \\ 
 

    \theutterance \stepcounter{utterance}  

    & & \multicolumn{4}{p{0.6\linewidth}}{\cellcolor[rgb]{0.9,0.9,0.9}{%
	\makecell[{{p{\linewidth}}}]{% 
	  \tt {\tiny [GM$\rangle$B]}  
	 Deine Nachricht wurde an deinen Mitspieler übermittelt.\\ \tt \\ \tt \\ \tt Ihr seid gerade bei Runde 2 von maximal 12.\\ \tt Ihr habt gemeinsam 1 von maximal 8 Strafpunkten angesammelt.\\ \tt Dein Mitspieler hat dir diese Nachricht geschickt:\\ \tt \\ \tt \\ \tt "Das klingt gut, ich stimme deinem Vorschlag zu, 'C' in der ersten Reihe, 'L' in der mittleren Reihe und 'P' in der letzten Reihe in der mittleren Spalte zu platzieren. Lass uns loslegen!"\\ \tt \\ \tt \\ \tt Wie lautet dein nächster Befehl? 
	  } 
	   } 
	   } 
	 & \\ 
 

    \theutterance \stepcounter{utterance}  

    & & & \multicolumn{4}{p{0.6\linewidth}}{\cellcolor[rgb]{1,0.85,0.72}{%
	\makecell[{{p{\linewidth}}}]{% 
	  \tt {\tiny [GM$\langle$B]}  
	 VERSCHIEBE: C, (4, 1) 
	  } 
	   } 
	   } 
	  \\ 
 

    \theutterance \stepcounter{utterance}  

    & & & \multicolumn{2}{p{0.3\linewidth}}{\cellcolor[rgb]{0.95,0.95,0.95}{%
	\makecell[{{p{\linewidth}}}]{% 
	  \tt {\tiny [GM$|$GM]}  
	 VERSCHIEBE: C, (4, 1) 
	  } 
	   } 
	   } 
	 & & \\ 
 

    \theutterance \stepcounter{utterance}  

    & & & \multicolumn{2}{p{0.3\linewidth}}{\cellcolor[rgb]{0.95,0.95,0.95}{%
	\makecell[{{p{\linewidth}}}]{% 
	  \tt {\tiny [GM$|$GM]}  
	 Strafpunkt: (4,1) ist nicht leer, sondern enthält 'C'.\\ \tt \\ \tt Ihr habt gemeinsam 2 von maximal 8 Strafpunkten angesammelt.\\ \tt Bitte versuche es noch mal! 
	  } 
	   } 
	   } 
	 & & \\ 
 

    \theutterance \stepcounter{utterance}  

    & & \multicolumn{4}{p{0.6\linewidth}}{\cellcolor[rgb]{0.9,0.9,0.9}{%
	\makecell[{{p{\linewidth}}}]{% 
	  \tt {\tiny [GM$\rangle$B]}  
	 Strafpunkt: (4,1) ist nicht leer, sondern enthält 'C'.\\ \tt \\ \tt Ihr habt gemeinsam 2 von maximal 8 Strafpunkten angesammelt.\\ \tt Bitte versuche es noch mal! 
	  } 
	   } 
	   } 
	 & \\ 
 

    \theutterance \stepcounter{utterance}  

    & & & \multicolumn{4}{p{0.6\linewidth}}{\cellcolor[rgb]{1,0.85,0.72}{%
	\makecell[{{p{\linewidth}}}]{% 
	  \tt {\tiny [GM$\langle$B]}  
	 VERSCHIEBE: C, (4, 1) \\ \tt \\ \tt Nein, ich gehe zurück zum vorherigen Versuch, den ich noch nicht geschickt hatte:\\ \tt \\ \tt SPRICH: Bevor wir anfangen, Objekte zu bewegen, sollten wir uns über die aktuelle Position der Objekte auf unseren Spielbrettern im Klaren sein. Kannst du mir bitte deine aktuelle Konfiguration beschreiben? 
	  } 
	   } 
	   } 
	  \\ 
 

    \theutterance \stepcounter{utterance}  

    & & & \multicolumn{2}{p{0.3\linewidth}}{\cellcolor[rgb]{0.95,0.95,0.95}{%
	\makecell[{{p{\linewidth}}}]{% 
	  \tt {\tiny [GM$|$GM]}  
	 VERSCHIEBE: C, (4, 1) \\ \tt \\ \tt Nein, ich gehe zurück zum vorherigen Versuch, den ich noch nicht geschickt hatte:\\ \tt \\ \tt SPRICH: Bevor wir anfangen, Objekte zu bewegen, sollten wir uns über die aktuelle Position der Objekte auf unseren Spielbrettern im Klaren sein. Kannst du mir bitte deine aktuelle Konfiguration beschreiben? 
	  } 
	   } 
	   } 
	 & & \\ 
 

    \theutterance \stepcounter{utterance}  

    & & & \multicolumn{2}{p{0.3\linewidth}}{\cellcolor[rgb]{0.95,0.95,0.95}{%
	\makecell[{{p{\linewidth}}}]{% 
	  \tt {\tiny [GM$|$GM]}  
	 Invalid response format: VERSCHIEBE: C, (4, 1) \\ \tt \\ \tt Nein, ich gehe zurück zum vorherigen Versuch, den ich noch nicht geschickt hatte:\\ \tt \\ \tt SPRICH: Bevor wir anfangen, Objekte zu bewegen, sollten wir uns über die aktuelle Position der Objekte auf unseren Spielbrettern im Klaren sein. Kannst du mir bitte deine aktuelle Konfiguration beschreiben? 
	  } 
	   } 
	   } 
	 & & \\ 
 

    \theutterance \stepcounter{utterance}  

    & & \multicolumn{4}{p{0.6\linewidth}}{\cellcolor[rgb]{0.9,0.9,0.9}{%
	\makecell[{{p{\linewidth}}}]{% 
	  \tt {\tiny [GM$\rangle$B]}  
	 Strafpunkt: Deine Nachricht enthält mehr als einen Befehl! Stelle sicher, dass deine Antwort ausschließlich einen der beiden Befehle `SPRICH: <NACHRICHT>` und `VERSCHIEBE: <OBJECT>, (<X>, <Y>)` und sonst nichts enthält!\\ \tt \\ \tt Ihr habt gemeinsam 3 von maximal 8 Strafpunkten angesammelt.\\ \tt Bitte versuche es noch mal! 
	  } 
	   } 
	   } 
	 & \\ 
 

    \theutterance \stepcounter{utterance}  

    & & & \multicolumn{4}{p{0.6\linewidth}}{\cellcolor[rgb]{1,0.85,0.72}{%
	\makecell[{{p{\linewidth}}}]{% 
	  \tt {\tiny [GM$\langle$B]}  
	 SPRICH: Bevor wir anfangen, Objekte zu bewegen, sollten wir uns über die aktuelle Position der Objekte auf unseren Spielbrettern im Klaren sein. Kannst du mir bitte deine aktuelle Konfiguration beschreiben? 
	  } 
	   } 
	   } 
	  \\ 
 

    \theutterance \stepcounter{utterance}  

    & & & \multicolumn{2}{p{0.3\linewidth}}{\cellcolor[rgb]{0.95,0.95,0.95}{%
	\makecell[{{p{\linewidth}}}]{% 
	  \tt {\tiny [GM$|$GM]}  
	 SPRICH: Bevor wir anfangen, Objekte zu bewegen, sollten wir uns über die aktuelle Position der Objekte auf unseren Spielbrettern im Klaren sein. Kannst du mir bitte deine aktuelle Konfiguration beschreiben? 
	  } 
	   } 
	   } 
	 & & \\ 
 

    \theutterance \stepcounter{utterance}  

    & & \multicolumn{4}{p{0.6\linewidth}}{\cellcolor[rgb]{0.9,0.9,0.9}{%
	\makecell[{{p{\linewidth}}}]{% 
	  \tt {\tiny [A$\langle$GM]}  
	 Deine Nachricht wurde an deinen Mitspieler übermittelt.\\ \tt \\ \tt \\ \tt Ihr seid gerade bei Runde 2 von maximal 12.\\ \tt Ihr habt gemeinsam 3 von maximal 8 Strafpunkten angesammelt.\\ \tt Dein Mitspieler hat dir diese Nachricht geschickt:\\ \tt \\ \tt \\ \tt "Bevor wir anfangen, Objekte zu bewegen, sollten wir uns über die aktuelle Position der Objekte auf unseren Spielbrettern im Klaren sein. Kannst du mir bitte deine aktuelle Konfiguration beschreiben?"\\ \tt \\ \tt \\ \tt Wie lautet dein nächster Befehl? 
	  } 
	   } 
	   } 
	 & \\ 
 

    \theutterance \stepcounter{utterance}  

    & \multicolumn{4}{p{0.6\linewidth}}{\cellcolor[rgb]{0.8,1,0.9}{%
	\makecell[{{p{\linewidth}}}]{% 
	  \tt {\tiny [A$\rangle$GM]}  
	 SPRICH: Meine Objekte sind wie folgt positioniert: 'C' in (5,5), 'L' in (3,1), 'P' in (2,3). 
	  } 
	   } 
	   } 
	 & & \\ 
 

    \theutterance \stepcounter{utterance}  

    & & & \multicolumn{2}{p{0.3\linewidth}}{\cellcolor[rgb]{0.95,0.95,0.95}{%
	\makecell[{{p{\linewidth}}}]{% 
	  \tt {\tiny [GM$|$GM]}  
	 SPRICH: Meine Objekte sind wie folgt positioniert: 'C' in (5,5), 'L' in (3,1), 'P' in (2,3). 
	  } 
	   } 
	   } 
	 & & \\ 
 

    \theutterance \stepcounter{utterance}  

    & & \multicolumn{4}{p{0.6\linewidth}}{\cellcolor[rgb]{0.9,0.9,0.9}{%
	\makecell[{{p{\linewidth}}}]{% 
	  \tt {\tiny [GM$\rangle$B]}  
	 Deine Nachricht wurde an deinen Mitspieler übermittelt.\\ \tt \\ \tt \\ \tt Ihr seid gerade bei Runde 3 von maximal 12.\\ \tt Ihr habt gemeinsam 3 von maximal 8 Strafpunkten angesammelt.\\ \tt Dein Mitspieler hat dir diese Nachricht geschickt:\\ \tt \\ \tt \\ \tt "Meine Objekte sind wie folgt positioniert: 'C' in (5,5), 'L' in (3,1), 'P' in (2,3)."\\ \tt \\ \tt \\ \tt Wie lautet dein nächster Befehl? 
	  } 
	   } 
	   } 
	 & \\ 
 

    \theutterance \stepcounter{utterance}  

    & & & \multicolumn{4}{p{0.6\linewidth}}{\cellcolor[rgb]{1,0.85,0.72}{%
	\makecell[{{p{\linewidth}}}]{% 
	  \tt {\tiny [GM$\langle$B]}  
	 SPRICH: Ich danke dir für die Information. Meine Objekte sind wie folgt positioniert: 'C' in (2,1), 'L' in (7,7), 'P' in (4,7). Jetzt, wo wir wissen, wo jedes Objekt ist, können wir mit dem Bewegen beginnen. Lass uns mit dem Objekt 'C' beginnen und es in die mittlere Spalte bewegen. 
	  } 
	   } 
	   } 
	  \\ 
 

    \theutterance \stepcounter{utterance}  

    & & & \multicolumn{2}{p{0.3\linewidth}}{\cellcolor[rgb]{0.95,0.95,0.95}{%
	\makecell[{{p{\linewidth}}}]{% 
	  \tt {\tiny [GM$|$GM]}  
	 SPRICH: Ich danke dir für die Information. Meine Objekte sind wie folgt positioniert: 'C' in (2,1), 'L' in (7,7), 'P' in (4,7). Jetzt, wo wir wissen, wo jedes Objekt ist, können wir mit dem Bewegen beginnen. Lass uns mit dem Objekt 'C' beginnen und es in die mittlere Spalte bewegen. 
	  } 
	   } 
	   } 
	 & & \\ 
 

    \theutterance \stepcounter{utterance}  

    & & \multicolumn{4}{p{0.6\linewidth}}{\cellcolor[rgb]{0.9,0.9,0.9}{%
	\makecell[{{p{\linewidth}}}]{% 
	  \tt {\tiny [A$\langle$GM]}  
	 Deine Nachricht wurde an deinen Mitspieler übermittelt.\\ \tt \\ \tt \\ \tt Ihr seid gerade bei Runde 3 von maximal 12.\\ \tt Ihr habt gemeinsam 3 von maximal 8 Strafpunkten angesammelt.\\ \tt Dein Mitspieler hat dir diese Nachricht geschickt:\\ \tt \\ \tt \\ \tt "Ich danke dir für die Information. Meine Objekte sind wie folgt positioniert: 'C' in (2,1), 'L' in (7,7), 'P' in (4,7). Jetzt, wo wir wissen, wo jedes Objekt ist, können wir mit dem Bewegen beginnen. Lass uns mit dem Objekt 'C' beginnen und es in die mittlere Spalte bewegen."\\ \tt \\ \tt \\ \tt Wie lautet dein nächster Befehl? 
	  } 
	   } 
	   } 
	 & \\ 
 

    \theutterance \stepcounter{utterance}  

    & \multicolumn{4}{p{0.6\linewidth}}{\cellcolor[rgb]{0.8,1,0.9}{%
	\makecell[{{p{\linewidth}}}]{% 
	  \tt {\tiny [A$\rangle$GM]}  
	 SPRICH: Ich denke, wir sollten 'C' zuerst nach (4,1) verschieben, da dies der erste Schritt in die mittlere Spalte wäre. Wie siehst du das? 
	  } 
	   } 
	   } 
	 & & \\ 
 

    \theutterance \stepcounter{utterance}  

    & & & \multicolumn{2}{p{0.3\linewidth}}{\cellcolor[rgb]{0.95,0.95,0.95}{%
	\makecell[{{p{\linewidth}}}]{% 
	  \tt {\tiny [GM$|$GM]}  
	 SPRICH: Ich denke, wir sollten 'C' zuerst nach (4,1) verschieben, da dies der erste Schritt in die mittlere Spalte wäre. Wie siehst du das? 
	  } 
	   } 
	   } 
	 & & \\ 
 

    \theutterance \stepcounter{utterance}  

    & & \multicolumn{4}{p{0.6\linewidth}}{\cellcolor[rgb]{0.9,0.9,0.9}{%
	\makecell[{{p{\linewidth}}}]{% 
	  \tt {\tiny [GM$\rangle$B]}  
	 Deine Nachricht wurde an deinen Mitspieler übermittelt.\\ \tt \\ \tt \\ \tt Ihr seid gerade bei Runde 4 von maximal 12.\\ \tt Ihr habt gemeinsam 3 von maximal 8 Strafpunkten angesammelt.\\ \tt Dein Mitspieler hat dir diese Nachricht geschickt:\\ \tt \\ \tt \\ \tt "Ich denke, wir sollten 'C' zuerst nach (4,1) verschieben, da dies der erste Schritt in die mittlere Spalte wäre. Wie siehst du das?"\\ \tt \\ \tt \\ \tt Wie lautet dein nächster Befehl? 
	  } 
	   } 
	   } 
	 & \\ 
 

    \theutterance \stepcounter{utterance}  

    & & & \multicolumn{4}{p{0.6\linewidth}}{\cellcolor[rgb]{1,0.85,0.72}{%
	\makecell[{{p{\linewidth}}}]{% 
	  \tt {\tiny [GM$\langle$B]}  
	 VERSCHIEBE: C, (4, 1) 
	  } 
	   } 
	   } 
	  \\ 
 

    \theutterance \stepcounter{utterance}  

    & & & \multicolumn{2}{p{0.3\linewidth}}{\cellcolor[rgb]{0.95,0.95,0.95}{%
	\makecell[{{p{\linewidth}}}]{% 
	  \tt {\tiny [GM$|$GM]}  
	 VERSCHIEBE: C, (4, 1) 
	  } 
	   } 
	   } 
	 & & \\ 
 

    \theutterance \stepcounter{utterance}  

    & & & \multicolumn{2}{p{0.3\linewidth}}{\cellcolor[rgb]{0.95,0.95,0.95}{%
	\makecell[{{p{\linewidth}}}]{% 
	  \tt {\tiny [GM$|$GM]}  
	 Strafpunkt: (4,1) ist nicht leer, sondern enthält 'C'.\\ \tt \\ \tt Ihr habt gemeinsam 4 von maximal 8 Strafpunkten angesammelt.\\ \tt Bitte versuche es noch mal! 
	  } 
	   } 
	   } 
	 & & \\ 
 

    \theutterance \stepcounter{utterance}  

    & & \multicolumn{4}{p{0.6\linewidth}}{\cellcolor[rgb]{0.9,0.9,0.9}{%
	\makecell[{{p{\linewidth}}}]{% 
	  \tt {\tiny [GM$\rangle$B]}  
	 Strafpunkt: (4,1) ist nicht leer, sondern enthält 'C'.\\ \tt \\ \tt Ihr habt gemeinsam 4 von maximal 8 Strafpunkten angesammelt.\\ \tt Bitte versuche es noch mal! 
	  } 
	   } 
	   } 
	 & \\ 
 

    \theutterance \stepcounter{utterance}  

    & & & \multicolumn{4}{p{0.6\linewidth}}{\cellcolor[rgb]{1,0.85,0.72}{%
	\makecell[{{p{\linewidth}}}]{% 
	  \tt {\tiny [GM$\langle$B]}  
	 SPRICH: Da mein 'C' bereits in der Position (2,1) ist, ist es_creation sinnvoll, es direkt nach (4,1) zu bewegen. Allerdings ist (4,1) bereits von meinem 'C' besetzt, daher ist der Vorschlag, 'C' nach (4,1) zu bewegen, nicht durchführbar. Ich schlage vor, ein anderes Objekt zu bewegen. Wie wäre es, wenn wir 'L' nach (4,4) bewegen? 
	  } 
	   } 
	   } 
	  \\ 
 

    \theutterance \stepcounter{utterance}  

    & & & \multicolumn{2}{p{0.3\linewidth}}{\cellcolor[rgb]{0.95,0.95,0.95}{%
	\makecell[{{p{\linewidth}}}]{% 
	  \tt {\tiny [GM$|$GM]}  
	 SPRICH: Da mein 'C' bereits in der Position (2,1) ist, ist es_creation sinnvoll, es direkt nach (4,1) zu bewegen. Allerdings ist (4,1) bereits von meinem 'C' besetzt, daher ist der Vorschlag, 'C' nach (4,1) zu bewegen, nicht durchführbar. Ich schlage vor, ein anderes Objekt zu bewegen. Wie wäre es, wenn wir 'L' nach (4,4) bewegen? 
	  } 
	   } 
	   } 
	 & & \\ 
 

    \theutterance \stepcounter{utterance}  

    & & \multicolumn{4}{p{0.6\linewidth}}{\cellcolor[rgb]{0.9,0.9,0.9}{%
	\makecell[{{p{\linewidth}}}]{% 
	  \tt {\tiny [A$\langle$GM]}  
	 Deine Nachricht wurde an deinen Mitspieler übermittelt.\\ \tt \\ \tt \\ \tt Ihr seid gerade bei Runde 4 von maximal 12.\\ \tt Ihr habt gemeinsam 4 von maximal 8 Strafpunkten angesammelt.\\ \tt Dein Mitspieler hat dir diese Nachricht geschickt:\\ \tt \\ \tt \\ \tt "Da mein 'C' bereits in der Position (2,1) ist, ist es_creation sinnvoll, es direkt nach (4,1) zu bewegen. Allerdings ist (4,1) bereits von meinem 'C' besetzt, daher ist der Vorschlag, 'C' nach (4,1) zu bewegen, nicht durchführbar. Ich schlage vor, ein anderes Objekt zu bewegen. Wie wäre es, wenn wir 'L' nach (4,4) bewegen?"\\ \tt \\ \tt \\ \tt Wie lautet dein nächster Befehl? 
	  } 
	   } 
	   } 
	 & \\ 
 

    \theutterance \stepcounter{utterance}  

    & \multicolumn{4}{p{0.6\linewidth}}{\cellcolor[rgb]{0.8,1,0.9}{%
	\makecell[{{p{\linewidth}}}]{% 
	  \tt {\tiny [A$\rangle$GM]}  
	 SPRICH: Ich stimme zu, 'L' nach (4,4) zu bewegen, das klingt wie ein guter Schritt, um unsere Ziele zu erreichen. Ich werde mein 'L' nach (4,4) bewegen. 
	  } 
	   } 
	   } 
	 & & \\ 
 

    \theutterance \stepcounter{utterance}  

    & & & \multicolumn{2}{p{0.3\linewidth}}{\cellcolor[rgb]{0.95,0.95,0.95}{%
	\makecell[{{p{\linewidth}}}]{% 
	  \tt {\tiny [GM$|$GM]}  
	 SPRICH: Ich stimme zu, 'L' nach (4,4) zu bewegen, das klingt wie ein guter Schritt, um unsere Ziele zu erreichen. Ich werde mein 'L' nach (4,4) bewegen. 
	  } 
	   } 
	   } 
	 & & \\ 
 

    \theutterance \stepcounter{utterance}  

    & & \multicolumn{4}{p{0.6\linewidth}}{\cellcolor[rgb]{0.9,0.9,0.9}{%
	\makecell[{{p{\linewidth}}}]{% 
	  \tt {\tiny [GM$\rangle$B]}  
	 Deine Nachricht wurde an deinen Mitspieler übermittelt.\\ \tt \\ \tt \\ \tt Ihr seid gerade bei Runde 5 von maximal 12.\\ \tt Ihr habt gemeinsam 4 von maximal 8 Strafpunkten angesammelt.\\ \tt Dein Mitspieler hat dir diese Nachricht geschickt:\\ \tt \\ \tt \\ \tt "Ich stimme zu, 'L' nach (4,4) zu bewegen, das klingt wie ein guter Schritt, um unsere Ziele zu erreichen. Ich werde mein 'L' nach (4,4) bewegen."\\ \tt \\ \tt \\ \tt Wie lautet dein nächster Befehl? 
	  } 
	   } 
	   } 
	 & \\ 
 

    \theutterance \stepcounter{utterance}  

    & & & \multicolumn{4}{p{0.6\linewidth}}{\cellcolor[rgb]{1,0.85,0.72}{%
	\makecell[{{p{\linewidth}}}]{% 
	  \tt {\tiny [GM$\langle$B]}  
	 VERSCHIEBE: L, (4, 4) 
	  } 
	   } 
	   } 
	  \\ 
 

    \theutterance \stepcounter{utterance}  

    & & & \multicolumn{2}{p{0.3\linewidth}}{\cellcolor[rgb]{0.95,0.95,0.95}{%
	\makecell[{{p{\linewidth}}}]{% 
	  \tt {\tiny [GM$|$GM]}  
	 VERSCHIEBE: L, (4, 4) 
	  } 
	   } 
	   } 
	 & & \\ 
 

    \theutterance \stepcounter{utterance}  

    & & & \multicolumn{2}{p{0.3\linewidth}}{\cellcolor[rgb]{0.95,0.95,0.95}{%
	\makecell[{{p{\linewidth}}}]{% 
	  \tt {\tiny [GM$|$GM]}  
	 Strafpunkt: (4,4) ist nicht leer, sondern enthält '─'.\\ \tt \\ \tt Ihr habt gemeinsam 5 von maximal 8 Strafpunkten angesammelt.\\ \tt Bitte versuche es noch mal! 
	  } 
	   } 
	   } 
	 & & \\ 
 

    \theutterance \stepcounter{utterance}  

    & & \multicolumn{4}{p{0.6\linewidth}}{\cellcolor[rgb]{0.9,0.9,0.9}{%
	\makecell[{{p{\linewidth}}}]{% 
	  \tt {\tiny [GM$\rangle$B]}  
	 Strafpunkt: (4,4) ist nicht leer, sondern enthält '─'.\\ \tt \\ \tt Ihr habt gemeinsam 5 von maximal 8 Strafpunkten angesammelt.\\ \tt Bitte versuche es noch mal! 
	  } 
	   } 
	   } 
	 & \\ 
 

    \theutterance \stepcounter{utterance}  

    & & & \multicolumn{4}{p{0.6\linewidth}}{\cellcolor[rgb]{1,0.85,0.72}{%
	\makecell[{{p{\linewidth}}}]{% 
	  \tt {\tiny [GM$\langle$B]}  
	 SPRICH: Ich denke, wir sollten unser Ziel für 'L' überdenken, da (4,4) nicht leer ist. Wie wäre es, wenn wir 'L' stattdessen nach (4,5) oder einem anderen leeren Platz in der mittleren Spalte bewegen? 
	  } 
	   } 
	   } 
	  \\ 
 

    \theutterance \stepcounter{utterance}  

    & & & \multicolumn{2}{p{0.3\linewidth}}{\cellcolor[rgb]{0.95,0.95,0.95}{%
	\makecell[{{p{\linewidth}}}]{% 
	  \tt {\tiny [GM$|$GM]}  
	 SPRICH: Ich denke, wir sollten unser Ziel für 'L' überdenken, da (4,4) nicht leer ist. Wie wäre es, wenn wir 'L' stattdessen nach (4,5) oder einem anderen leeren Platz in der mittleren Spalte bewegen? 
	  } 
	   } 
	   } 
	 & & \\ 
 

    \theutterance \stepcounter{utterance}  

    & & \multicolumn{4}{p{0.6\linewidth}}{\cellcolor[rgb]{0.9,0.9,0.9}{%
	\makecell[{{p{\linewidth}}}]{% 
	  \tt {\tiny [A$\langle$GM]}  
	 Deine Nachricht wurde an deinen Mitspieler übermittelt.\\ \tt \\ \tt \\ \tt Ihr seid gerade bei Runde 5 von maximal 12.\\ \tt Ihr habt gemeinsam 5 von maximal 8 Strafpunkten angesammelt.\\ \tt Dein Mitspieler hat dir diese Nachricht geschickt:\\ \tt \\ \tt \\ \tt "Ich denke, wir sollten unser Ziel für 'L' überdenken, da (4,4) nicht leer ist. Wie wäre es, wenn wir 'L' stattdessen nach (4,5) oder einem anderen leeren Platz in der mittleren Spalte bewegen?"\\ \tt \\ \tt \\ \tt Wie lautet dein nächster Befehl? 
	  } 
	   } 
	   } 
	 & \\ 
 

    \theutterance \stepcounter{utterance}  

    & \multicolumn{4}{p{0.6\linewidth}}{\cellcolor[rgb]{0.8,1,0.9}{%
	\makecell[{{p{\linewidth}}}]{% 
	  \tt {\tiny [A$\rangle$GM]}  
	 VERSCHIEBE: L, (4,4) 
	  } 
	   } 
	   } 
	 & & \\ 
 

    \theutterance \stepcounter{utterance}  

    & & & \multicolumn{2}{p{0.3\linewidth}}{\cellcolor[rgb]{0.95,0.95,0.95}{%
	\makecell[{{p{\linewidth}}}]{% 
	  \tt {\tiny [GM$|$GM]}  
	 VERSCHIEBE: L, (4,4) 
	  } 
	   } 
	   } 
	 & & \\ 
 

    \theutterance \stepcounter{utterance}  

    & & & \multicolumn{2}{p{0.3\linewidth}}{\cellcolor[rgb]{0.95,0.95,0.95}{%
	\makecell[{{p{\linewidth}}}]{% 
	  \tt {\tiny [GM$|$GM]}  
	 Strafpunkt: (4,4) ist nicht leer, sondern enthält '─'.\\ \tt \\ \tt Ihr habt gemeinsam 6 von maximal 8 Strafpunkten angesammelt.\\ \tt Bitte versuche es noch mal! 
	  } 
	   } 
	   } 
	 & & \\ 
 

    \theutterance \stepcounter{utterance}  

    & & \multicolumn{4}{p{0.6\linewidth}}{\cellcolor[rgb]{0.9,0.9,0.9}{%
	\makecell[{{p{\linewidth}}}]{% 
	  \tt {\tiny [A$\langle$GM]}  
	 Strafpunkt: (4,4) ist nicht leer, sondern enthält '─'.\\ \tt \\ \tt Ihr habt gemeinsam 6 von maximal 8 Strafpunkten angesammelt.\\ \tt Bitte versuche es noch mal! 
	  } 
	   } 
	   } 
	 & \\ 
 

    \theutterance \stepcounter{utterance}  

    & \multicolumn{4}{p{0.6\linewidth}}{\cellcolor[rgb]{0.8,1,0.9}{%
	\makecell[{{p{\linewidth}}}]{% 
	  \tt {\tiny [A$\rangle$GM]}  
	 VERSCHIEBE: L, (4,3) 
	  } 
	   } 
	   } 
	 & & \\ 
 

    \theutterance \stepcounter{utterance}  

    & & & \multicolumn{2}{p{0.3\linewidth}}{\cellcolor[rgb]{0.95,0.95,0.95}{%
	\makecell[{{p{\linewidth}}}]{% 
	  \tt {\tiny [GM$|$GM]}  
	 VERSCHIEBE: L, (4,3) 
	  } 
	   } 
	   } 
	 & & \\ 
 

    \theutterance \stepcounter{utterance}  

    & & & \multicolumn{2}{p{0.3\linewidth}}{\cellcolor[rgb]{0.95,0.95,0.95}{%
	\makecell[{{p{\linewidth}}}]{% 
	  \tt {\tiny [GM$|$GM]}  
	 Strafpunkt: (4,3) ist nicht leer, sondern enthält 'P'.\\ \tt \\ \tt Ihr habt gemeinsam 7 von maximal 8 Strafpunkten angesammelt.\\ \tt Bitte versuche es noch mal! 
	  } 
	   } 
	   } 
	 & & \\ 
 

    \theutterance \stepcounter{utterance}  

    & & \multicolumn{4}{p{0.6\linewidth}}{\cellcolor[rgb]{0.9,0.9,0.9}{%
	\makecell[{{p{\linewidth}}}]{% 
	  \tt {\tiny [A$\langle$GM]}  
	 Strafpunkt: (4,3) ist nicht leer, sondern enthält 'P'.\\ \tt \\ \tt Ihr habt gemeinsam 7 von maximal 8 Strafpunkten angesammelt.\\ \tt Bitte versuche es noch mal! 
	  } 
	   } 
	   } 
	 & \\ 
 

    \theutterance \stepcounter{utterance}  

    & \multicolumn{4}{p{0.6\linewidth}}{\cellcolor[rgb]{0.8,1,0.9}{%
	\makecell[{{p{\linewidth}}}]{% 
	  \tt {\tiny [A$\rangle$GM]}  
	 VERSCHIEBE: L, (4,5) 
	  } 
	   } 
	   } 
	 & & \\ 
 

    \theutterance \stepcounter{utterance}  

    & & & \multicolumn{2}{p{0.3\linewidth}}{\cellcolor[rgb]{0.95,0.95,0.95}{%
	\makecell[{{p{\linewidth}}}]{% 
	  \tt {\tiny [GM$|$GM]}  
	 VERSCHIEBE: L, (4,5) 
	  } 
	   } 
	   } 
	 & & \\ 
 

    \theutterance \stepcounter{utterance}  

    & & & \multicolumn{2}{p{0.3\linewidth}}{\cellcolor[rgb]{0.95,0.95,0.95}{%
	\makecell[{{p{\linewidth}}}]{% 
	  \tt {\tiny [GM$|$GM]}  
	 'L' erfolgreich nach (4,5) verschoben. Dein Raster sieht jetzt so aus:\\ \tt ```\\ \tt  1234567\\ \tt ╔═╤═══╤═╗\\ \tt ║◌│◌◌◌│◌║ 1\\ \tt ╟─┼──┬┘◌║ 2\\ \tt ║◌│◌P│◌◌║ 3\\ \tt ╟─┴──┘◌◌║ 4\\ \tt ║◌◌◌L◌C◌║ 5\\ \tt ║◌◌◌◌◌◌◌║ 6\\ \tt ║◌◌◌◌◌◌◌║ 7\\ \tt ╚═══════╝\\ \tt ```\\ \tt  
	  } 
	   } 
	   } 
	 & & \\ 
 

    \theutterance \stepcounter{utterance}  

    & & \multicolumn{4}{p{0.6\linewidth}}{\cellcolor[rgb]{0.9,0.9,0.9}{%
	\makecell[{{p{\linewidth}}}]{% 
	  \tt {\tiny [GM$\rangle$B]}  
	 Deine Nachricht wurde an deinen Mitspieler übermittelt.\\ \tt \\ \tt \\ \tt Ihr seid gerade bei Runde 6 von maximal 12.\\ \tt Ihr habt gemeinsam 7 von maximal 8 Strafpunkten angesammelt.\\ \tt Dein Mitspieler hat ein Objekt auf seinem Raster bewegt.\\ \tt \\ \tt \\ \tt Wie lautet dein nächster Befehl? 
	  } 
	   } 
	   } 
	 & \\ 
 

    \theutterance \stepcounter{utterance}  

    & & & \multicolumn{4}{p{0.6\linewidth}}{\cellcolor[rgb]{1,0.85,0.72}{%
	\makecell[{{p{\linewidth}}}]{% 
	  \tt {\tiny [GM$\langle$B]}  
	 SPRICH: Ich hoffe, dein letzter Zug war erfolgreich! Lass uns unseren Fokus wieder auf das gemeinsame Ziel lenken. Wir sollten unsere Objekte in der mittleren Spalte platzieren, wie wir besprochen haben. Wie sieht es bei dir aus? Hast du noch Objekte, die du bewegen musst, um das Ziel zu erreichen? 
	  } 
	   } 
	   } 
	  \\ 
 

    \theutterance \stepcounter{utterance}  

    & & & \multicolumn{2}{p{0.3\linewidth}}{\cellcolor[rgb]{0.95,0.95,0.95}{%
	\makecell[{{p{\linewidth}}}]{% 
	  \tt {\tiny [GM$|$GM]}  
	 SPRICH: Ich hoffe, dein letzter Zug war erfolgreich! Lass uns unseren Fokus wieder auf das gemeinsame Ziel lenken. Wir sollten unsere Objekte in der mittleren Spalte platzieren, wie wir besprochen haben. Wie sieht es bei dir aus? Hast du noch Objekte, die du bewegen musst, um das Ziel zu erreichen? 
	  } 
	   } 
	   } 
	 & & \\ 
 

    \theutterance \stepcounter{utterance}  

    & & \multicolumn{4}{p{0.6\linewidth}}{\cellcolor[rgb]{0.9,0.9,0.9}{%
	\makecell[{{p{\linewidth}}}]{% 
	  \tt {\tiny [A$\langle$GM]}  
	 'L' erfolgreich nach (4,5) verschoben. Dein Raster sieht jetzt so aus:\\ \tt ```\\ \tt  1234567\\ \tt ╔═╤═══╤═╗\\ \tt ║◌│◌◌◌│◌║ 1\\ \tt ╟─┼──┬┘◌║ 2\\ \tt ║◌│◌P│◌◌║ 3\\ \tt ╟─┴──┘◌◌║ 4\\ \tt ║◌◌◌L◌C◌║ 5\\ \tt ║◌◌◌◌◌◌◌║ 6\\ \tt ║◌◌◌◌◌◌◌║ 7\\ \tt ╚═══════╝\\ \tt ```\\ \tt Ihr seid gerade bei Runde 6 von maximal 12.\\ \tt Ihr habt gemeinsam 7 von maximal 8 Strafpunkten angesammelt.\\ \tt Dein Mitspieler hat dir diese Nachricht geschickt:\\ \tt \\ \tt \\ \tt "Ich hoffe, dein letzter Zug war erfolgreich! Lass uns unseren Fokus wieder auf das gemeinsame Ziel lenken. Wir sollten unsere Objekte in der mittleren Spalte platzieren, wie wir besprochen haben. Wie sieht es bei dir aus? Hast du noch Objekte, die du bewegen musst, um das Ziel zu erreichen?"\\ \tt \\ \tt \\ \tt Wie lautet dein nächster Befehl? 
	  } 
	   } 
	   } 
	 & \\ 
 

    \theutterance \stepcounter{utterance}  

    & \multicolumn{4}{p{0.6\linewidth}}{\cellcolor[rgb]{0.8,1,0.9}{%
	\makecell[{{p{\linewidth}}}]{% 
	  \tt {\tiny [A$\rangle$GM]}  
	 SPRICH: Ich denke, wir sollten 'P' nach (4,7) bewegen, um es in die mittlere Spalte zu bringen und unserem Ziel näher zu kommen. Ich habe noch 'P' und 'C', die ich bewegen muss, um unsere ursprüngliche Strategie umzusetzen. 
	  } 
	   } 
	   } 
	 & & \\ 
 

    \theutterance \stepcounter{utterance}  

    & & & \multicolumn{2}{p{0.3\linewidth}}{\cellcolor[rgb]{0.95,0.95,0.95}{%
	\makecell[{{p{\linewidth}}}]{% 
	  \tt {\tiny [GM$|$GM]}  
	 SPRICH: Ich denke, wir sollten 'P' nach (4,7) bewegen, um es in die mittlere Spalte zu bringen und unserem Ziel näher zu kommen. Ich habe noch 'P' und 'C', die ich bewegen muss, um unsere ursprüngliche Strategie umzusetzen. 
	  } 
	   } 
	   } 
	 & & \\ 
 

    \theutterance \stepcounter{utterance}  

    & & \multicolumn{4}{p{0.6\linewidth}}{\cellcolor[rgb]{0.9,0.9,0.9}{%
	\makecell[{{p{\linewidth}}}]{% 
	  \tt {\tiny [GM$\rangle$B]}  
	 Deine Nachricht wurde an deinen Mitspieler übermittelt.\\ \tt \\ \tt \\ \tt Ihr seid gerade bei Runde 7 von maximal 12.\\ \tt Ihr habt gemeinsam 7 von maximal 8 Strafpunkten angesammelt.\\ \tt Dein Mitspieler hat dir diese Nachricht geschickt:\\ \tt \\ \tt \\ \tt "Ich denke, wir sollten 'P' nach (4,7) bewegen, um es in die mittlere Spalte zu bringen und unserem Ziel näher zu kommen. Ich habe noch 'P' und 'C', die ich bewegen muss, um unsere ursprüngliche Strategie umzusetzen."\\ \tt \\ \tt \\ \tt Wie lautet dein nächster Befehl? 
	  } 
	   } 
	   } 
	 & \\ 
 

    \theutterance \stepcounter{utterance}  

    & & & \multicolumn{4}{p{0.6\linewidth}}{\cellcolor[rgb]{1,0.85,0.72}{%
	\makecell[{{p{\linewidth}}}]{% 
	  \tt {\tiny [GM$\langle$B]}  
	 VERSCHIEBE: P, (4, 7) 
	  } 
	   } 
	   } 
	  \\ 
 

    \theutterance \stepcounter{utterance}  

    & & & \multicolumn{2}{p{0.3\linewidth}}{\cellcolor[rgb]{0.95,0.95,0.95}{%
	\makecell[{{p{\linewidth}}}]{% 
	  \tt {\tiny [GM$|$GM]}  
	 VERSCHIEBE: P, (4, 7) 
	  } 
	   } 
	   } 
	 & & \\ 
 

    \theutterance \stepcounter{utterance}  

    & & & \multicolumn{2}{p{0.3\linewidth}}{\cellcolor[rgb]{0.95,0.95,0.95}{%
	\makecell[{{p{\linewidth}}}]{% 
	  \tt {\tiny [GM$|$GM]}  
	 Strafpunkt: (4,7) ist nicht leer, sondern enthält 'P'.\\ \tt \\ \tt Ihr habt gemeinsam 8 von maximal 8 Strafpunkten angesammelt.\\ \tt Bitte versuche es noch mal! 
	  } 
	   } 
	   } 
	 & & \\ 
 

    \theutterance \stepcounter{utterance}  

    & & \multicolumn{4}{p{0.6\linewidth}}{\cellcolor[rgb]{0.9,0.9,0.9}{%
	\makecell[{{p{\linewidth}}}]{% 
	  \tt {\tiny [GM$\rangle$B]}  
	 Strafpunkt: (4,7) ist nicht leer, sondern enthält 'P'.\\ \tt \\ \tt Ihr habt gemeinsam 8 von maximal 8 Strafpunkten angesammelt.\\ \tt Bitte versuche es noch mal! 
	  } 
	   } 
	   } 
	 & \\ 
 

    \theutterance \stepcounter{utterance}  

    & & & \multicolumn{4}{p{0.6\linewidth}}{\cellcolor[rgb]{1,0.85,0.72}{%
	\makecell[{{p{\linewidth}}}]{% 
	  \tt {\tiny [GM$\langle$B]}  
	 SPRICH: Da wir nun alle Strafpunkte erreicht haben, sollten wir vorsichtig sein und keine weitere Strafpunkte sammeln. Ich denke, wir sollten unsere Strategie überdenken und sicherstellen, dass wir keine Objekte in besetzte Felder bewegen. Lass uns unsere aktuellen Positionen überprüfen und einen neuen Plan entwickeln, um unser Ziel zu erreichen. Wie sieht es bei dir aus? 
	  } 
	   } 
	   } 
	  \\ 
 

    \theutterance \stepcounter{utterance}  

    & & & \multicolumn{2}{p{0.3\linewidth}}{\cellcolor[rgb]{0.95,0.95,0.95}{%
	\makecell[{{p{\linewidth}}}]{% 
	  \tt {\tiny [GM$|$GM]}  
	 SPRICH: Da wir nun alle Strafpunkte erreicht haben, sollten wir vorsichtig sein und keine weitere Strafpunkte sammeln. Ich denke, wir sollten unsere Strategie überdenken und sicherstellen, dass wir keine Objekte in besetzte Felder bewegen. Lass uns unsere aktuellen Positionen überprüfen und einen neuen Plan entwickeln, um unser Ziel zu erreichen. Wie sieht es bei dir aus? 
	  } 
	   } 
	   } 
	 & & \\ 
 

    \theutterance \stepcounter{utterance}  

    & & \multicolumn{4}{p{0.6\linewidth}}{\cellcolor[rgb]{0.9,0.9,0.9}{%
	\makecell[{{p{\linewidth}}}]{% 
	  \tt {\tiny [A$\langle$GM]}  
	 Deine Nachricht wurde an deinen Mitspieler übermittelt.\\ \tt \\ \tt \\ \tt Ihr seid gerade bei Runde 7 von maximal 12.\\ \tt Ihr habt gemeinsam 8 von maximal 8 Strafpunkten angesammelt.\\ \tt Dein Mitspieler hat dir diese Nachricht geschickt:\\ \tt \\ \tt \\ \tt "Da wir nun alle Strafpunkte erreicht haben, sollten wir vorsichtig sein und keine weitere Strafpunkte sammeln. Ich denke, wir sollten unsere Strategie überdenken und sicherstellen, dass wir keine Objekte in besetzte Felder bewegen. Lass uns unsere aktuellen Positionen überprüfen und einen neuen Plan entwickeln, um unser Ziel zu erreichen. Wie sieht es bei dir aus?"\\ \tt \\ \tt \\ \tt Wie lautet dein nächster Befehl? 
	  } 
	   } 
	   } 
	 & \\ 
 

    \theutterance \stepcounter{utterance}  

    & \multicolumn{4}{p{0.6\linewidth}}{\cellcolor[rgb]{0.8,1,0.9}{%
	\makecell[{{p{\linewidth}}}]{% 
	  \tt {\tiny [A$\rangle$GM]}  
	 SPRICH: Ich denke, wir sollten 'C' nach (4,1) und 'P' nach (4,7) bewegen, um unsere ursprüngliche Strategie umzusetzen. Ich werde vorsichtig sein und sicherstellen, dass die Felder frei sind, bevor ich etwas bewege. 
	  } 
	   } 
	   } 
	 & & \\ 
 

    \theutterance \stepcounter{utterance}  

    & & & \multicolumn{2}{p{0.3\linewidth}}{\cellcolor[rgb]{0.95,0.95,0.95}{%
	\makecell[{{p{\linewidth}}}]{% 
	  \tt {\tiny [GM$|$GM]}  
	 SPRICH: Ich denke, wir sollten 'C' nach (4,1) und 'P' nach (4,7) bewegen, um unsere ursprüngliche Strategie umzusetzen. Ich werde vorsichtig sein und sicherstellen, dass die Felder frei sind, bevor ich etwas bewege. 
	  } 
	   } 
	   } 
	 & & \\ 
 

    \theutterance \stepcounter{utterance}  

    & & \multicolumn{4}{p{0.6\linewidth}}{\cellcolor[rgb]{0.9,0.9,0.9}{%
	\makecell[{{p{\linewidth}}}]{% 
	  \tt {\tiny [GM$\rangle$B]}  
	 Deine Nachricht wurde an deinen Mitspieler übermittelt.\\ \tt \\ \tt \\ \tt Ihr seid gerade bei Runde 8 von maximal 12.\\ \tt Ihr habt gemeinsam 8 von maximal 8 Strafpunkten angesammelt.\\ \tt Dein Mitspieler hat dir diese Nachricht geschickt:\\ \tt \\ \tt \\ \tt "Ich denke, wir sollten 'C' nach (4,1) und 'P' nach (4,7) bewegen, um unsere ursprüngliche Strategie umzusetzen. Ich werde vorsichtig sein und sicherstellen, dass die Felder frei sind, bevor ich etwas bewege."\\ \tt \\ \tt \\ \tt Wie lautet dein nächster Befehl? 
	  } 
	   } 
	   } 
	 & \\ 
 

    \theutterance \stepcounter{utterance}  

    & & & \multicolumn{4}{p{0.6\linewidth}}{\cellcolor[rgb]{1,0.85,0.72}{%
	\makecell[{{p{\linewidth}}}]{% 
	  \tt {\tiny [GM$\langle$B]}  
	 SPRICH: Ich denke, wir sollten uns vergewissern, dass die Felder (4,1) und (4,7) tatsächlich frei sind, bevor wir 'C' und 'P' dorthin bewegen. Da mein 'C' bereits in (2,1) und mein 'P' in (4,7) ist, muss ich mich auf meine eigenen Objekte konzentrieren. Ich schlage vor, dass wir unsere aktuellen Positionen und die beabsichtigten Ziele noch einmal überprüfen, um sicherzustellen, dass wir keine weiteren Strafpunkte sammeln. Lass uns vorsichtig vorgehen und unsere Strategie anpassen, um unser Ziel zu erreichen. 
	  } 
	   } 
	   } 
	  \\ 
 

    \theutterance \stepcounter{utterance}  

    & & & \multicolumn{2}{p{0.3\linewidth}}{\cellcolor[rgb]{0.95,0.95,0.95}{%
	\makecell[{{p{\linewidth}}}]{% 
	  \tt {\tiny [GM$|$GM]}  
	 SPRICH: Ich denke, wir sollten uns vergewissern, dass die Felder (4,1) und (4,7) tatsächlich frei sind, bevor wir 'C' und 'P' dorthin bewegen. Da mein 'C' bereits in (2,1) und mein 'P' in (4,7) ist, muss ich mich auf meine eigenen Objekte konzentrieren. Ich schlage vor, dass wir unsere aktuellen Positionen und die beabsichtigten Ziele noch einmal überprüfen, um sicherzustellen, dass wir keine weiteren Strafpunkte sammeln. Lass uns vorsichtig vorgehen und unsere Strategie anpassen, um unser Ziel zu erreichen. 
	  } 
	   } 
	   } 
	 & & \\ 
 

    \theutterance \stepcounter{utterance}  

    & & \multicolumn{4}{p{0.6\linewidth}}{\cellcolor[rgb]{0.9,0.9,0.9}{%
	\makecell[{{p{\linewidth}}}]{% 
	  \tt {\tiny [A$\langle$GM]}  
	 Deine Nachricht wurde an deinen Mitspieler übermittelt.\\ \tt \\ \tt \\ \tt Ihr seid gerade bei Runde 8 von maximal 12.\\ \tt Ihr habt gemeinsam 8 von maximal 8 Strafpunkten angesammelt.\\ \tt Dein Mitspieler hat dir diese Nachricht geschickt:\\ \tt \\ \tt \\ \tt "Ich denke, wir sollten uns vergewissern, dass die Felder (4,1) und (4,7) tatsächlich frei sind, bevor wir 'C' und 'P' dorthin bewegen. Da mein 'C' bereits in (2,1) und mein 'P' in (4,7) ist, muss ich mich auf meine eigenen Objekte konzentrieren. Ich schlage vor, dass wir unsere aktuellen Positionen und die beabsichtigten Ziele noch einmal überprüfen, um sicherzustellen, dass wir keine weiteren Strafpunkte sammeln. Lass uns vorsichtig vorgehen und unsere Strategie anpassen, um unser Ziel zu erreichen."\\ \tt \\ \tt \\ \tt Wie lautet dein nächster Befehl? 
	  } 
	   } 
	   } 
	 & \\ 
 

    \theutterance \stepcounter{utterance}  

    & \multicolumn{4}{p{0.6\linewidth}}{\cellcolor[rgb]{0.8,1,0.9}{%
	\makecell[{{p{\linewidth}}}]{% 
	  \tt {\tiny [A$\rangle$GM]}  
	 SPRICH: Ich stimme zu, dass wir vorsichtig sein sollten. Ich werde meine Positionen noch einmal überprüfen und mich auf 'C' und 'P' konzentrieren, um sicherzustellen, dass ich sie an die richtigen Positionen bewege. Ich denke, wir sollten 'C' nach (4,1) und 'P' nach (4,3) bewegen, um sie in die mittlere Spalte zu bringen. 
	  } 
	   } 
	   } 
	 & & \\ 
 

    \theutterance \stepcounter{utterance}  

    & & & \multicolumn{2}{p{0.3\linewidth}}{\cellcolor[rgb]{0.95,0.95,0.95}{%
	\makecell[{{p{\linewidth}}}]{% 
	  \tt {\tiny [GM$|$GM]}  
	 SPRICH: Ich stimme zu, dass wir vorsichtig sein sollten. Ich werde meine Positionen noch einmal überprüfen und mich auf 'C' und 'P' konzentrieren, um sicherzustellen, dass ich sie an die richtigen Positionen bewege. Ich denke, wir sollten 'C' nach (4,1) und 'P' nach (4,3) bewegen, um sie in die mittlere Spalte zu bringen. 
	  } 
	   } 
	   } 
	 & & \\ 
 

    \theutterance \stepcounter{utterance}  

    & & \multicolumn{4}{p{0.6\linewidth}}{\cellcolor[rgb]{0.9,0.9,0.9}{%
	\makecell[{{p{\linewidth}}}]{% 
	  \tt {\tiny [GM$\rangle$B]}  
	 Deine Nachricht wurde an deinen Mitspieler übermittelt.\\ \tt \\ \tt \\ \tt Ihr seid gerade bei Runde 9 von maximal 12.\\ \tt Ihr habt gemeinsam 8 von maximal 8 Strafpunkten angesammelt.\\ \tt Dein Mitspieler hat dir diese Nachricht geschickt:\\ \tt \\ \tt \\ \tt "Ich stimme zu, dass wir vorsichtig sein sollten. Ich werde meine Positionen noch einmal überprüfen und mich auf 'C' und 'P' konzentrieren, um sicherzustellen, dass ich sie an die richtigen Positionen bewege. Ich denke, wir sollten 'C' nach (4,1) und 'P' nach (4,3) bewegen, um sie in die mittlere Spalte zu bringen."\\ \tt \\ \tt \\ \tt Wie lautet dein nächster Befehl? 
	  } 
	   } 
	   } 
	 & \\ 
 

    \theutterance \stepcounter{utterance}  

    & & & \multicolumn{4}{p{0.6\linewidth}}{\cellcolor[rgb]{1,0.85,0.72}{%
	\makecell[{{p{\linewidth}}}]{% 
	  \tt {\tiny [GM$\langle$B]}  
	 SPRICH: Ich denke, wir sind auf dem richtigen Weg, indem wir unsere Positionen überprüfen und vorsichtig vorgehen. Ich stimme zu, dass 'C' nach (4,1) und 'P' nach (4,3) bewegt werden sollten, um sie in die mittlere Spalte zu bringen. Ich werde mich auf meine eigenen Objekte konzentrieren und sicherstellen, dass ich sie an die richtigen Positionen bewege. Lass uns einen letzten Versuch unternehmen, um unser Ziel zu erreichen. Fühlen wir uns bereit, unsere Objekte zu bewegen und zu sehen, ob wir erfolgreich sind? 
	  } 
	   } 
	   } 
	  \\ 
 

    \theutterance \stepcounter{utterance}  

    & & & \multicolumn{2}{p{0.3\linewidth}}{\cellcolor[rgb]{0.95,0.95,0.95}{%
	\makecell[{{p{\linewidth}}}]{% 
	  \tt {\tiny [GM$|$GM]}  
	 SPRICH: Ich denke, wir sind auf dem richtigen Weg, indem wir unsere Positionen überprüfen und vorsichtig vorgehen. Ich stimme zu, dass 'C' nach (4,1) und 'P' nach (4,3) bewegt werden sollten, um sie in die mittlere Spalte zu bringen. Ich werde mich auf meine eigenen Objekte konzentrieren und sicherstellen, dass ich sie an die richtigen Positionen bewege. Lass uns einen letzten Versuch unternehmen, um unser Ziel zu erreichen. Fühlen wir uns bereit, unsere Objekte zu bewegen und zu sehen, ob wir erfolgreich sind? 
	  } 
	   } 
	   } 
	 & & \\ 
 

    \theutterance \stepcounter{utterance}  

    & & \multicolumn{4}{p{0.6\linewidth}}{\cellcolor[rgb]{0.9,0.9,0.9}{%
	\makecell[{{p{\linewidth}}}]{% 
	  \tt {\tiny [A$\langle$GM]}  
	 Deine Nachricht wurde an deinen Mitspieler übermittelt.\\ \tt \\ \tt \\ \tt Ihr seid gerade bei Runde 9 von maximal 12.\\ \tt Ihr habt gemeinsam 8 von maximal 8 Strafpunkten angesammelt.\\ \tt Dein Mitspieler hat dir diese Nachricht geschickt:\\ \tt \\ \tt \\ \tt "Ich denke, wir sind auf dem richtigen Weg, indem wir unsere Positionen überprüfen und vorsichtig vorgehen. Ich stimme zu, dass 'C' nach (4,1) und 'P' nach (4,3) bewegt werden sollten, um sie in die mittlere Spalte zu bringen. Ich werde mich auf meine eigenen Objekte konzentrieren und sicherstellen, dass ich sie an die richtigen Positionen bewege. Lass uns einen letzten Versuch unternehmen, um unser Ziel zu erreichen. Fühlen wir uns bereit, unsere Objekte zu bewegen und zu sehen, ob wir erfolgreich sind?"\\ \tt \\ \tt \\ \tt Wie lautet dein nächster Befehl? 
	  } 
	   } 
	   } 
	 & \\ 
 

    \theutterance \stepcounter{utterance}  

    & \multicolumn{4}{p{0.6\linewidth}}{\cellcolor[rgb]{0.8,1,0.9}{%
	\makecell[{{p{\linewidth}}}]{% 
	  \tt {\tiny [A$\rangle$GM]}  
	 SPRICH: Ja, ich denke, wir sollten es versuchen. Ich bin bereit, meine Objekte zu bewegen und zu sehen, ob wir erfolgreich sind. Lass uns loslegen und unsere Strategie umsetzen. Ich hoffe, wir erreichen unser Ziel und gewinnen das Spiel! 
	  } 
	   } 
	   } 
	 & & \\ 
 

    \theutterance \stepcounter{utterance}  

    & & & \multicolumn{2}{p{0.3\linewidth}}{\cellcolor[rgb]{0.95,0.95,0.95}{%
	\makecell[{{p{\linewidth}}}]{% 
	  \tt {\tiny [GM$|$GM]}  
	 SPRICH: Ja, ich denke, wir sollten es versuchen. Ich bin bereit, meine Objekte zu bewegen und zu sehen, ob wir erfolgreich sind. Lass uns loslegen und unsere Strategie umsetzen. Ich hoffe, wir erreichen unser Ziel und gewinnen das Spiel! 
	  } 
	   } 
	   } 
	 & & \\ 
 

    \theutterance \stepcounter{utterance}  

    & & \multicolumn{4}{p{0.6\linewidth}}{\cellcolor[rgb]{0.9,0.9,0.9}{%
	\makecell[{{p{\linewidth}}}]{% 
	  \tt {\tiny [GM$\rangle$B]}  
	 Deine Nachricht wurde an deinen Mitspieler übermittelt.\\ \tt \\ \tt \\ \tt Ihr seid gerade bei Runde 10 von maximal 12.\\ \tt Ihr habt gemeinsam 8 von maximal 8 Strafpunkten angesammelt.\\ \tt Dein Mitspieler hat dir diese Nachricht geschickt:\\ \tt \\ \tt \\ \tt "Ja, ich denke, wir sollten es versuchen. Ich bin bereit, meine Objekte zu bewegen und zu sehen, ob wir erfolgreich sind. Lass uns loslegen und unsere Strategie umsetzen. Ich hoffe, wir erreichen unser Ziel und gewinnen das Spiel!"\\ \tt \\ \tt \\ \tt Wie lautet dein nächster Befehl? 
	  } 
	   } 
	   } 
	 & \\ 
 

    \theutterance \stepcounter{utterance}  

    & & & \multicolumn{4}{p{0.6\linewidth}}{\cellcolor[rgb]{1,0.85,0.72}{%
	\makecell[{{p{\linewidth}}}]{% 
	  \tt {\tiny [GM$\langle$B]}  
	 SPRICH: Ich bin auch bereit, loszulegen und unsere Strategie umzusetzen. Bevor wir beginnen, möchte ich noch einmal unsere Pläne überprüfen und sicherstellen, dass wir alles richtig machen. Wenn du bereit bist, können wir unsere Objekte bewegen und hoffen, dass wir erfolgreich sind. Ich denke, wir sollten noch einmal unsere Positionen überprüfen und dann loslegen. Bist du sicher, dass du bereit bist, oder möchtest du noch einmal überprüfen? 
	  } 
	   } 
	   } 
	  \\ 
 

    \theutterance \stepcounter{utterance}  

    & & & \multicolumn{2}{p{0.3\linewidth}}{\cellcolor[rgb]{0.95,0.95,0.95}{%
	\makecell[{{p{\linewidth}}}]{% 
	  \tt {\tiny [GM$|$GM]}  
	 SPRICH: Ich bin auch bereit, loszulegen und unsere Strategie umzusetzen. Bevor wir beginnen, möchte ich noch einmal unsere Pläne überprüfen und sicherstellen, dass wir alles richtig machen. Wenn du bereit bist, können wir unsere Objekte bewegen und hoffen, dass wir erfolgreich sind. Ich denke, wir sollten noch einmal unsere Positionen überprüfen und dann loslegen. Bist du sicher, dass du bereit bist, oder möchtest du noch einmal überprüfen? 
	  } 
	   } 
	   } 
	 & & \\ 
 

    \theutterance \stepcounter{utterance}  

    & & \multicolumn{4}{p{0.6\linewidth}}{\cellcolor[rgb]{0.9,0.9,0.9}{%
	\makecell[{{p{\linewidth}}}]{% 
	  \tt {\tiny [A$\langle$GM]}  
	 Deine Nachricht wurde an deinen Mitspieler übermittelt.\\ \tt \\ \tt \\ \tt Ihr seid gerade bei Runde 10 von maximal 12.\\ \tt Ihr habt gemeinsam 8 von maximal 8 Strafpunkten angesammelt.\\ \tt Dein Mitspieler hat dir diese Nachricht geschickt:\\ \tt \\ \tt \\ \tt "Ich bin auch bereit, loszulegen und unsere Strategie umzusetzen. Bevor wir beginnen, möchte ich noch einmal unsere Pläne überprüfen und sicherstellen, dass wir alles richtig machen. Wenn du bereit bist, können wir unsere Objekte bewegen und hoffen, dass wir erfolgreich sind. Ich denke, wir sollten noch einmal unsere Positionen überprüfen und dann loslegen. Bist du sicher, dass du bereit bist, oder möchtest du noch einmal überprüfen?"\\ \tt \\ \tt \\ \tt Wie lautet dein nächster Befehl? 
	  } 
	   } 
	   } 
	 & \\ 
 

    \theutterance \stepcounter{utterance}  

    & \multicolumn{4}{p{0.6\linewidth}}{\cellcolor[rgb]{0.8,1,0.9}{%
	\makecell[{{p{\linewidth}}}]{% 
	  \tt {\tiny [A$\rangle$GM]}  
	 SPRICH: Ja, ich bin sicher, dass ich bereit bin. Ich denke, wir sollten direkt loslegen und unsere Objekte bewegen. Ich bin zuversichtlich, dass wir unser Ziel erreichen können, wenn wir unsere Strategie umsetzen. Lass uns beginnen! 
	  } 
	   } 
	   } 
	 & & \\ 
 

    \theutterance \stepcounter{utterance}  

    & & & \multicolumn{2}{p{0.3\linewidth}}{\cellcolor[rgb]{0.95,0.95,0.95}{%
	\makecell[{{p{\linewidth}}}]{% 
	  \tt {\tiny [GM$|$GM]}  
	 SPRICH: Ja, ich bin sicher, dass ich bereit bin. Ich denke, wir sollten direkt loslegen und unsere Objekte bewegen. Ich bin zuversichtlich, dass wir unser Ziel erreichen können, wenn wir unsere Strategie umsetzen. Lass uns beginnen! 
	  } 
	   } 
	   } 
	 & & \\ 
 

    \theutterance \stepcounter{utterance}  

    & & \multicolumn{4}{p{0.6\linewidth}}{\cellcolor[rgb]{0.9,0.9,0.9}{%
	\makecell[{{p{\linewidth}}}]{% 
	  \tt {\tiny [GM$\rangle$B]}  
	 Deine Nachricht wurde an deinen Mitspieler übermittelt.\\ \tt \\ \tt \\ \tt Ihr seid gerade bei Runde 11 von maximal 12.\\ \tt Ihr habt gemeinsam 8 von maximal 8 Strafpunkten angesammelt.\\ \tt Dein Mitspieler hat dir diese Nachricht geschickt:\\ \tt \\ \tt \\ \tt "Ja, ich bin sicher, dass ich bereit bin. Ich denke, wir sollten direkt loslegen und unsere Objekte bewegen. Ich bin zuversichtlich, dass wir unser Ziel erreichen können, wenn wir unsere Strategie umsetzen. Lass uns beginnen!"\\ \tt \\ \tt \\ \tt Wie lautet dein nächster Befehl? 
	  } 
	   } 
	   } 
	 & \\ 
 

    \theutterance \stepcounter{utterance}  

    & & & \multicolumn{4}{p{0.6\linewidth}}{\cellcolor[rgb]{1,0.85,0.72}{%
	\makecell[{{p{\linewidth}}}]{% 
	  \tt {\tiny [GM$\langle$B]}  
	 SPRICH: Fertig? 
	  } 
	   } 
	   } 
	  \\ 
 

    \theutterance \stepcounter{utterance}  

    & & & \multicolumn{2}{p{0.3\linewidth}}{\cellcolor[rgb]{0.95,0.95,0.95}{%
	\makecell[{{p{\linewidth}}}]{% 
	  \tt {\tiny [GM$|$GM]}  
	 SPRICH: Fertig? 
	  } 
	   } 
	   } 
	 & & \\ 
 

    \theutterance \stepcounter{utterance}  

    & & \multicolumn{4}{p{0.6\linewidth}}{\cellcolor[rgb]{0.9,0.9,0.9}{%
	\makecell[{{p{\linewidth}}}]{% 
	  \tt {\tiny [A$\langle$GM]}  
	 Deine Nachricht wurde an deinen Mitspieler übermittelt.\\ \tt \\ \tt \\ \tt Ihr seid gerade bei Runde 11 von maximal 12.\\ \tt Ihr habt gemeinsam 8 von maximal 8 Strafpunkten angesammelt.\\ \tt Dein Mitspieler hat dir diese Nachricht geschickt:\\ \tt \\ \tt \\ \tt "Fertig?"\\ \tt \\ \tt \\ \tt Wie lautet dein nächster Befehl? 
	  } 
	   } 
	   } 
	 & \\ 
 

    \theutterance \stepcounter{utterance}  

    & \multicolumn{4}{p{0.6\linewidth}}{\cellcolor[rgb]{0.8,1,0.9}{%
	\makecell[{{p{\linewidth}}}]{% 
	  \tt {\tiny [A$\rangle$GM]}  
	 SPRICH: Fertig! 
	  } 
	   } 
	   } 
	 & & \\ 
 

    \theutterance \stepcounter{utterance}  

    & & & \multicolumn{2}{p{0.3\linewidth}}{\cellcolor[rgb]{0.95,0.95,0.95}{%
	\makecell[{{p{\linewidth}}}]{% 
	  \tt {\tiny [GM$|$GM]}  
	 SPRICH: Fertig! 
	  } 
	   } 
	   } 
	 & & \\ 
 

    \theutterance \stepcounter{utterance}  

    & & \multicolumn{4}{p{0.6\linewidth}}{\cellcolor[rgb]{0.9,0.9,0.9}{%
	\makecell[{{p{\linewidth}}}]{% 
	  \tt {\tiny [GM$\rangle$B]}  
	 Deine Nachricht wurde an deinen Mitspieler übermittelt.\\ \tt \\ \tt \\ \tt Ihr seid gerade bei Runde 12 von maximal 12.\\ \tt Ihr habt gemeinsam 8 von maximal 8 Strafpunkten angesammelt.\\ \tt Dein Mitspieler hat dir diese Nachricht geschickt:\\ \tt \\ \tt \\ \tt "Fertig!"\\ \tt \\ \tt \\ \tt Wie lautet dein nächster Befehl? 
	  } 
	   } 
	   } 
	 & \\ 
 

    \theutterance \stepcounter{utterance}  

    & & & \multicolumn{4}{p{0.6\linewidth}}{\cellcolor[rgb]{1,0.85,0.72}{%
	\makecell[{{p{\linewidth}}}]{% 
	  \tt {\tiny [GM$\langle$B]}  
	 SPRICH: fertig! 
	  } 
	   } 
	   } 
	  \\ 
 

    \theutterance \stepcounter{utterance}  

    & & & \multicolumn{2}{p{0.3\linewidth}}{\cellcolor[rgb]{0.95,0.95,0.95}{%
	\makecell[{{p{\linewidth}}}]{% 
	  \tt {\tiny [GM$|$GM]}  
	 SPRICH: fertig! 
	  } 
	   } 
	   } 
	 & & \\ 
 

    \theutterance \stepcounter{utterance}  

    & & & \multicolumn{2}{p{0.3\linewidth}}{\cellcolor[rgb]{0.95,0.95,0.95}{%
	\makecell[{{p{\linewidth}}}]{% 
	  \tt {\tiny [GM$|$GM]}  
	 Maximum number of rounds reached 
	  } 
	   } 
	   } 
	 & & \\ 
 

    \theutterance \stepcounter{utterance}  

    & & & \multicolumn{2}{p{0.3\linewidth}}{\cellcolor[rgb]{0.95,0.95,0.95}{%
	\makecell[{{p{\linewidth}}}]{% 
	  \tt {\tiny [GM$|$GM]}  
	 * success: True\\ \tt * lose: False\\ \tt * aborted: False\\ \tt -------\\ \tt * Shifts: 0.00\\ \tt * Max Shifts: 4.00\\ \tt * Min Shifts: 2.00\\ \tt * End Distance Sum: 12.08\\ \tt * Init Distance Sum: 14.80\\ \tt * Expected Distance Sum: 12.57\\ \tt * Penalties: 8.00\\ \tt * Max Penalties: 8.00\\ \tt * Rounds: 12.00\\ \tt * Max Rounds: 12.00\\ \tt * Object Count: 3.00\\ \tt  
	  } 
	   } 
	   } 
	 & & \\ 
 

    \theutterance \stepcounter{utterance}  

    & & & \multicolumn{2}{p{0.3\linewidth}}{\cellcolor[rgb]{0.95,0.95,0.95}{%
	\makecell[{{p{\linewidth}}}]{% 
	  \tt {\tiny [GM$|$GM]}  
	 * Main Score: 1.68\\ \tt \\ \tt -------\\ \tt * Distance Score: 0.11\\ \tt * Coverage Score: 0.12\\ \tt * Penalty Score: 0.14\\ \tt * Alternative Penalty Score: 0.25\\ \tt * Alternative Main Score: 2.79\\ \tt \\ \tt -------\\ \tt * Shifts: 0.00\\ \tt * Max Shifts: 4.00\\ \tt * Min Shifts: 2.00\\ \tt * End Distance Sum: 12.08\\ \tt * Init Distance Sum: 14.80\\ \tt * Expected Distance Sum: 12.57\\ \tt * Penalties: 8.00\\ \tt * Max Penalties: 8.00\\ \tt * Rounds: 12.00\\ \tt * Max Rounds: 12.00\\ \tt * Object Count: 3.00\\ \tt  
	  } 
	   } 
	   } 
	 & & \\ 
 

\end{supertabular}
}

\end{document}
