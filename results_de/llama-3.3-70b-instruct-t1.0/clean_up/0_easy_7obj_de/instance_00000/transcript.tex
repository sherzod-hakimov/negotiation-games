
\documentclass{article}
\usepackage{colortbl}
\usepackage{makecell}
\usepackage{multirow}
\usepackage{supertabular}

\begin{document}

\newcounter{utterance}

\twocolumn

{ \footnotesize  \setcounter{utterance}{1}
\setlength{\tabcolsep}{0pt}
\begin{supertabular}{c@{$\;$}|p{.15\linewidth}@{}p{.15\linewidth}p{.15\linewidth}p{.15\linewidth}p{.15\linewidth}p{.15\linewidth}}

    \# & $\;$A & \multicolumn{4}{c}{Game Master} & $\;\:$B\\
    \hline 

    \theutterance \stepcounter{utterance}  

    & & \multicolumn{4}{p{0.6\linewidth}}{\cellcolor[rgb]{0.9,0.9,0.9}{%
	\makecell[{{p{\linewidth}}}]{% 
	  \tt {\tiny [A$\langle$GM]}  
	 Ich bin der Spielleiter, und du spielst ein kooperatives Spiel mit dem folgenden Raster als Spielbrett:\\ \tt \\ \tt ```\\ \tt  1234567\\ \tt ╔═╤═══╤═╗\\ \tt ║◌│◌◌◌│◌║ 1\\ \tt ╟─┼──┬┘◌║ 2\\ \tt ║◌│◌◌│S◌║ 3\\ \tt ╟─┴──┘DT║ 4\\ \tt ║◌O◌◌◌◌◌║ 5\\ \tt ║◌◌◌◌◌◌◌║ 6\\ \tt ║◌◌APM◌◌║ 7\\ \tt ╚═══════╝\\ \tt \\ \tt ```\\ \tt \\ \tt * Am oberen Rand sind die x-Koordinaten angegeben, die nach rechts hin zunehmen, und am rechten Rand die y-Koordinaten, die nach unten zunehmen.\\ \tt * Die folgenden Objekte sind zufällig auf deinem Raster platziert: 'P', 'O', 'T', 'S', 'D', 'A', 'M'.\\ \tt \\ \tt \\ \tt Dein Mitspieler sieht eine Variation des Spielbretts, auf dem dieselben Objekte an anderen zufälligen Positionen platziert sind. Du kannst das Spielbrett des anderen Spielers nicht sehen, und er kann deins nicht sehen.\\ \tt \\ \tt \\ \tt **Ziel des Spiels:**\\ \tt \\ \tt Beide Spieler müssen die Objekte auf ihrem jeweiligen Spielbrett so bewegen, dass identische Objekte am Ende an denselben Koordinaten liegen. Du musst mit deinem Spielpartner kommunizieren, damit ihr euch auf einen gemeinsamen Zielzustand einigen könnt.\\ \tt \\ \tt \\ \tt **Spielregeln:**\\ \tt \\ \tt * In jedem Spielzug kannst du genau eine der folgenden zwei Aktionen ausführen:\\ \tt 1. `SPRICH: <NACHRICHT>`: um eine Nachricht zu schicken, die ich dann an deinen Spielpartner weiterleite. Als Nachricht gilt alles bis zum nächsten Zeilenumbruch.\\ \tt 2. `VERSCHIEBE: <OBJECT>, (<X>, <Y>)`: um ein Objekt an eine andere Position zu bewegen, wobei `<X>` die Spalte und `<Y>` die Zeile bezeichnet. Ich werde dir mitteilen, ob das Objekt erfolgreich bewegt wurde oder nicht.\\ \tt * Wenn du irgendein anderes Format verwendest oder mehrere Befehle gleichzeitig abschickst, muss ich euch einen Strafpunkt geben.\\ \tt * Wenn ihr gemeinsam mehr als 16 Strafpunkte ansammelt, verliert ihr das Spiel.\\ \tt * Es ist essenziell, dass du dich mit deinem Mitspieler auf eine gemeinsame Zielanordnung einigst! Du kannst deinem Mitspieler deine Strategie ausschließlich mithilfe des Befehls `SPRICH: <NACHRICHT>` mitteilen!\\ \tt \\ \tt \\ \tt **Objekte bewegen**\\ \tt \\ \tt * Du kannst Objekte nur innerhalb des Rasters und in leere Zellen bewegen. Leere Zellen erkennst am Zeichen '◌'.\\ \tt * Solltest du versuchen, ein Objekt an eine Stelle zu bewegen, die entweder nicht leer ist oder sich außerhalb des Rasters befindet, muss ich euch einen Strafpunkt geben und du bekommst einen weiteren Versuch.\\ \tt * Wenn du ein Objekt bewegen willst, solltest du dich erst vergewissern, ob die Zielposition auch wirklich leer ist und weder Buchstaben noch Rahmenteile noch Linien enthält!\\ \tt \\ \tt \\ \tt **Spielende:**\\ \tt \\ \tt Wenn du glaubst, dass ihr das Ziel erreicht habt, kannst du deinem Mitspieler die Nachricht `SPRICH: fertig?` schicken. Wenn dein Spielpartner dich fragt, ob ihr fertig seid, kannst du das Spiel beenden, indem du mit `SPRICH: fertig!` antwortest.\\ \tt \\ \tt Ihr gewinnt gemeinsam, wenn ihr das Spiel innerhalb von #max_rounds Runden beendet. Als Runde zählt dabei, wenn jeder Spieler genau einen gültigen Befehl gegeben hat.\\ \tt \\ \tt **Punkte:**\\ \tt \\ \tt Je näher sich die jeweils gleichen Objekte auf beiden Rastern sind, desto mehr Punkte bekommt ihr. Für Strafpunkte gibt es Punktabzug. Könnt ihr den Rekord brechen?\\ \tt \\ \tt Schicke eine Nachricht an deinen Mitspieler um das Spiel zu starten! 
	  } 
	   } 
	   } 
	 & \\ 
 

    \theutterance \stepcounter{utterance}  

    & \multicolumn{4}{p{0.6\linewidth}}{\cellcolor[rgb]{0.8,1,0.9}{%
	\makecell[{{p{\linewidth}}}]{% 
	  \tt {\tiny [A$\rangle$GM]}  
	 SPRICH: Hallo, ich bin bereit, loszulegen! Lass uns zunächst unsere Strategie besprechen. Ich schlage vor, dass wir uns auf eine bestimmte Position für jedes Objekt einigen, um sicherzustellen, dass wir beide das gleiche Ziel verfolgen. Ich denke, es wäre hilfreich, wenn wir uns auf den mittleren Punkt des Rasters konzentrieren, also (4, 4). Was meinst du? Sollten wir beginnen, unsere Objekte dorthin zu bewegen oder hast du eine andere Idee? 
	  } 
	   } 
	   } 
	 & & \\ 
 

    \theutterance \stepcounter{utterance}  

    & & & \multicolumn{2}{p{0.3\linewidth}}{\cellcolor[rgb]{0.95,0.95,0.95}{%
	\makecell[{{p{\linewidth}}}]{% 
	  \tt {\tiny [GM$|$GM]}  
	 SPRICH: Hallo, ich bin bereit, loszulegen! Lass uns zunächst unsere Strategie besprechen. Ich schlage vor, dass wir uns auf eine bestimmte Position für jedes Objekt einigen, um sicherzustellen, dass wir beide das gleiche Ziel verfolgen. Ich denke, es wäre hilfreich, wenn wir uns auf den mittleren Punkt des Rasters konzentrieren, also (4, 4). Was meinst du? Sollten wir beginnen, unsere Objekte dorthin zu bewegen oder hast du eine andere Idee? 
	  } 
	   } 
	   } 
	 & & \\ 
 

    \theutterance \stepcounter{utterance}  

    & & \multicolumn{4}{p{0.6\linewidth}}{\cellcolor[rgb]{0.9,0.9,0.9}{%
	\makecell[{{p{\linewidth}}}]{% 
	  \tt {\tiny [GM$\rangle$B]}  
	 Ich bin der Spielleiter, und du spielst ein kooperatives Spiel mit dem folgenden Raster als Spielbrett:\\ \tt \\ \tt ```\\ \tt  1234567\\ \tt ╔═╤═══╤═╗\\ \tt ║◌│◌◌◌│◌║ 1\\ \tt ╟─┼──┬┘◌║ 2\\ \tt ║◌│◌M│◌T║ 3\\ \tt ╟─┴──┘P◌║ 4\\ \tt ║D◌◌◌◌A◌║ 5\\ \tt ║◌◌◌◌O◌◌║ 6\\ \tt ║◌◌S◌◌◌◌║ 7\\ \tt ╚═══════╝\\ \tt \\ \tt ```\\ \tt \\ \tt * Am oberen Rand sind die x-Koordinaten angegeben, die nach rechts hin zunehmen, und am rechten Rand die y-Koordinaten, die nach unten zunehmen.\\ \tt * Die folgenden Objekte sind zufällig auf deinem Raster platziert: 'P', 'O', 'T', 'S', 'D', 'A', 'M'.\\ \tt \\ \tt \\ \tt Dein Mitspieler sieht eine Variation des Spielbretts, auf dem dieselben Objekte an anderen zufälligen Positionen platziert sind. Du kannst das Spielbrett des anderen Spielers nicht sehen, und er kann deins nicht sehen.\\ \tt \\ \tt \\ \tt **Ziel des Spiels:**\\ \tt \\ \tt Beide Spieler müssen die Objekte auf ihrem jeweiligen Spielbrett so bewegen, dass identische Objekte am Ende an denselben Koordinaten liegen. Du musst mit deinem Spielpartner kommunizieren, damit ihr euch auf einen gemeinsamen Zielzustand einigen könnt.\\ \tt \\ \tt \\ \tt **Spielregeln:**\\ \tt \\ \tt * In jedem Spielzug kannst du genau eine der folgenden zwei Aktionen ausführen:\\ \tt 1. `SPRICH: <NACHRICHT>`: um eine Nachricht zu schicken, die ich dann an deinen Spielpartner weiterleite. Als Nachricht gilt alles bis zum nächsten Zeilenumbruch.\\ \tt 2. `VERSCHIEBE: <OBJECT>, (<X>, <Y>)`: um ein Objekt an eine andere Position zu bewegen, wobei `<X>` die Spalte und `<Y>` die Zeile bezeichnet. Ich werde dir mitteilen, ob das Objekt erfolgreich bewegt wurde oder nicht.\\ \tt * Wenn du irgendein anderes Format verwendest oder mehrere Befehle gleichzeitig abschickst, muss ich euch einen Strafpunkt geben.\\ \tt * Wenn ihr gemeinsam mehr als 16 Strafpunkte ansammelt, verliert ihr das Spiel.\\ \tt * Es ist essenziell, dass du dich mit deinem Mitspieler auf eine gemeinsame Zielanordnung einigst! Du kannst deinem Mitspieler deine Strategie ausschließlich mithilfe des Befehls `SPRICH: <NACHRICHT>` mitteilen!\\ \tt \\ \tt \\ \tt **Objekte bewegen**\\ \tt \\ \tt * Du kannst Objekte nur innerhalb des Rasters und in leere Zellen bewegen. Leere Zellen erkennst am Zeichen '◌'.\\ \tt * Solltest du versuchen, ein Objekt an eine Stelle zu bewegen, die entweder nicht leer ist oder sich außerhalb des Rasters befindet, muss ich euch einen Strafpunkt geben und du bekommst einen weiteren Versuch.\\ \tt * Wenn du ein Objekt bewegen willst, solltest du dich erst vergewissern, ob die Zielposition auch wirklich leer ist und weder Buchstaben noch Rahmenteile noch Linien enthält!\\ \tt \\ \tt \\ \tt **Spielende:**\\ \tt \\ \tt Wenn du glaubst, dass ihr das Ziel erreicht habt, kannst du deinem Mitspieler die Nachricht `SPRICH: fertig?` schicken. Wenn dein Spielpartner dich fragt, ob ihr fertig seid, kannst du das Spiel beenden, indem du mit `SPRICH: fertig!` antwortest.\\ \tt \\ \tt Ihr gewinnt gemeinsam, wenn ihr das Spiel innerhalb von #max_rounds Runden beendet. Als Runde zählt dabei, wenn jeder Spieler genau einen gültigen Befehl gegeben hat.\\ \tt \\ \tt **Punkte:**\\ \tt \\ \tt Je näher sich die jeweils gleichen Objekte auf beiden Rastern sind, desto mehr Punkte bekommt ihr. Für Strafpunkte gibt es Punktabzug. Könnt ihr den Rekord brechen?\\ \tt \\ \tt Dein Mitspieler hat das Spiel mit dieser Nachricht begonnen:\\ \tt \\ \tt "Hallo, ich bin bereit, loszulegen! Lass uns zunächst unsere Strategie besprechen. Ich schlage vor, dass wir uns auf eine bestimmte Position für jedes Objekt einigen, um sicherzustellen, dass wir beide das gleiche Ziel verfolgen. Ich denke, es wäre hilfreich, wenn wir uns auf den mittleren Punkt des Rasters konzentrieren, also (4, 4). Was meinst du? Sollten wir beginnen, unsere Objekte dorthin zu bewegen oder hast du eine andere Idee?"\\ \tt \\ \tt Wie lautet dein erster Befehl? 
	  } 
	   } 
	   } 
	 & \\ 
 

    \theutterance \stepcounter{utterance}  

    & & & \multicolumn{4}{p{0.6\linewidth}}{\cellcolor[rgb]{1,0.85,0.72}{%
	\makecell[{{p{\linewidth}}}]{% 
	  \tt {\tiny [GM$\langle$B]}  
	 Ich denke, das ist eine großartige Idee, uns auf den mittleren Punkt des Rasters zu konzentrieren! Die Position (4, 4) scheint mir ein guter Ausgangspunkt zu sein.\\ \tt \\ \tt Mein erster Befehl wäre also: `SPRICH: Ich stimme deiner Idee zu, den mittleren Punkt (4, 4) als Ziel zu verwenden. Lass uns unsere Objekte dorthin bewegen. Wie sollen wir vorgehen, um dies zu erreichen? Sollen wir alle Objekte zuerst an eine bestimmte Position bewegen und dann das Ziel anpassen oder direkt auf (4, 4) zu zielen?` 
	  } 
	   } 
	   } 
	  \\ 
 

    \theutterance \stepcounter{utterance}  

    & & & \multicolumn{2}{p{0.3\linewidth}}{\cellcolor[rgb]{0.95,0.95,0.95}{%
	\makecell[{{p{\linewidth}}}]{% 
	  \tt {\tiny [GM$|$GM]}  
	 Ich denke, das ist eine großartige Idee, uns auf den mittleren Punkt des Rasters zu konzentrieren! Die Position (4, 4) scheint mir ein guter Ausgangspunkt zu sein.\\ \tt \\ \tt Mein erster Befehl wäre also: `SPRICH: Ich stimme deiner Idee zu, den mittleren Punkt (4, 4) als Ziel zu verwenden. Lass uns unsere Objekte dorthin bewegen. Wie sollen wir vorgehen, um dies zu erreichen? Sollen wir alle Objekte zuerst an eine bestimmte Position bewegen und dann das Ziel anpassen oder direkt auf (4, 4) zu zielen?` 
	  } 
	   } 
	   } 
	 & & \\ 
 

    \theutterance \stepcounter{utterance}  

    & & & \multicolumn{2}{p{0.3\linewidth}}{\cellcolor[rgb]{0.95,0.95,0.95}{%
	\makecell[{{p{\linewidth}}}]{% 
	  \tt {\tiny [GM$|$GM]}  
	 Invalid format: head is not empty: 'Ich denke, das ist eine großartige Idee, uns auf den mittleren Punkt des Rasters zu konzentrieren! Die Position (4, 4) scheint mir ein guter Ausgangspunkt zu sein.\\ \tt \\ \tt Mein erster Befehl wäre also: ' 
	  } 
	   } 
	   } 
	 & & \\ 
 

    \theutterance \stepcounter{utterance}  

    & & \multicolumn{4}{p{0.6\linewidth}}{\cellcolor[rgb]{0.9,0.9,0.9}{%
	\makecell[{{p{\linewidth}}}]{% 
	  \tt {\tiny [GM$\rangle$B]}  
	 Strafpunkt: Deine Nachricht darf nichts vor dem Befehl enthalten! Stelle sicher, dass deine Antwort ausschließlich einen der beiden Befehle `SPRICH: <NACHRICHT>` und `VERSCHIEBE: <OBJECT>, (<X>, <Y>)` und sonst nichts enthält!\\ \tt \\ \tt Ihr habt gemeinsam 1 von maximal 16 Strafpunkten angesammelt.\\ \tt Bitte versuche es noch mal! 
	  } 
	   } 
	   } 
	 & \\ 
 

    \theutterance \stepcounter{utterance}  

    & & & \multicolumn{4}{p{0.6\linewidth}}{\cellcolor[rgb]{1,0.85,0.72}{%
	\makecell[{{p{\linewidth}}}]{% 
	  \tt {\tiny [GM$\langle$B]}  
	 SPRICH: Ich stimme deiner Idee zu, den mittleren Punkt (4, 4) als Ziel zu verwenden. Lass uns unsere Objekte dorthin bewegen. Wie sollen wir vorgehen, um dies zu erreichen? Sollen wir alle Objekte zuerst an eine bestimmte Position bewegen und dann das Ziel anpassen oder direkt auf (4, 4) zu zielen? 
	  } 
	   } 
	   } 
	  \\ 
 

    \theutterance \stepcounter{utterance}  

    & & & \multicolumn{2}{p{0.3\linewidth}}{\cellcolor[rgb]{0.95,0.95,0.95}{%
	\makecell[{{p{\linewidth}}}]{% 
	  \tt {\tiny [GM$|$GM]}  
	 SPRICH: Ich stimme deiner Idee zu, den mittleren Punkt (4, 4) als Ziel zu verwenden. Lass uns unsere Objekte dorthin bewegen. Wie sollen wir vorgehen, um dies zu erreichen? Sollen wir alle Objekte zuerst an eine bestimmte Position bewegen und dann das Ziel anpassen oder direkt auf (4, 4) zu zielen? 
	  } 
	   } 
	   } 
	 & & \\ 
 

    \theutterance \stepcounter{utterance}  

    & & \multicolumn{4}{p{0.6\linewidth}}{\cellcolor[rgb]{0.9,0.9,0.9}{%
	\makecell[{{p{\linewidth}}}]{% 
	  \tt {\tiny [A$\langle$GM]}  
	 Deine Nachricht wurde an deinen Mitspieler übermittelt.\\ \tt \\ \tt \\ \tt Ihr seid gerade bei Runde 1 von maximal 28.\\ \tt Ihr habt gemeinsam 1 von maximal 16 Strafpunkten angesammelt.\\ \tt Dein Mitspieler hat dir diese Nachricht geschickt:\\ \tt \\ \tt \\ \tt "Ich stimme deiner Idee zu, den mittleren Punkt (4, 4) als Ziel zu verwenden. Lass uns unsere Objekte dorthin bewegen. Wie sollen wir vorgehen, um dies zu erreichen? Sollen wir alle Objekte zuerst an eine bestimmte Position bewegen und dann das Ziel anpassen oder direkt auf (4, 4) zu zielen?"\\ \tt \\ \tt \\ \tt Wie lautet dein nächster Befehl? 
	  } 
	   } 
	   } 
	 & \\ 
 

    \theutterance \stepcounter{utterance}  

    & \multicolumn{4}{p{0.6\linewidth}}{\cellcolor[rgb]{0.8,1,0.9}{%
	\makecell[{{p{\linewidth}}}]{% 
	  \tt {\tiny [A$\rangle$GM]}  
	 SPRICH: Ich denke, es wäre effizienter, wenn wir zuerst alle Objekte in einer Nähe zum mittleren Punkt sammeln, um dann gemeinsam das endgültige Ziel (4, 4) für jedes Objekt zu erreichen. Wir sollten versuchen, alle Objekte zunächst in der gleichen Zeile oder Spalte wie (4, 4) zu platzieren, bevor wir sie genau auf die Position (4, 4) bewegen. Was hältst du davon, wenn wir zunächst die Objekte in der 4. Spalte oder der 4. Zeile platzieren und dann anpassen? 
	  } 
	   } 
	   } 
	 & & \\ 
 

    \theutterance \stepcounter{utterance}  

    & & & \multicolumn{2}{p{0.3\linewidth}}{\cellcolor[rgb]{0.95,0.95,0.95}{%
	\makecell[{{p{\linewidth}}}]{% 
	  \tt {\tiny [GM$|$GM]}  
	 SPRICH: Ich denke, es wäre effizienter, wenn wir zuerst alle Objekte in einer Nähe zum mittleren Punkt sammeln, um dann gemeinsam das endgültige Ziel (4, 4) für jedes Objekt zu erreichen. Wir sollten versuchen, alle Objekte zunächst in der gleichen Zeile oder Spalte wie (4, 4) zu platzieren, bevor wir sie genau auf die Position (4, 4) bewegen. Was hältst du davon, wenn wir zunächst die Objekte in der 4. Spalte oder der 4. Zeile platzieren und dann anpassen? 
	  } 
	   } 
	   } 
	 & & \\ 
 

    \theutterance \stepcounter{utterance}  

    & & \multicolumn{4}{p{0.6\linewidth}}{\cellcolor[rgb]{0.9,0.9,0.9}{%
	\makecell[{{p{\linewidth}}}]{% 
	  \tt {\tiny [GM$\rangle$B]}  
	 Deine Nachricht wurde an deinen Mitspieler übermittelt.\\ \tt \\ \tt \\ \tt Ihr seid gerade bei Runde 2 von maximal 28.\\ \tt Ihr habt gemeinsam 1 von maximal 16 Strafpunkten angesammelt.\\ \tt Dein Mitspieler hat dir diese Nachricht geschickt:\\ \tt \\ \tt \\ \tt "Ich denke, es wäre effizienter, wenn wir zuerst alle Objekte in einer Nähe zum mittleren Punkt sammeln, um dann gemeinsam das endgültige Ziel (4, 4) für jedes Objekt zu erreichen. Wir sollten versuchen, alle Objekte zunächst in der gleichen Zeile oder Spalte wie (4, 4) zu platzieren, bevor wir sie genau auf die Position (4, 4) bewegen. Was hältst du davon, wenn wir zunächst die Objekte in der 4. Spalte oder der 4. Zeile platzieren und dann anpassen?"\\ \tt \\ \tt \\ \tt Wie lautet dein nächster Befehl? 
	  } 
	   } 
	   } 
	 & \\ 
 

    \theutterance \stepcounter{utterance}  

    & & & \multicolumn{4}{p{0.6\linewidth}}{\cellcolor[rgb]{1,0.85,0.72}{%
	\makecell[{{p{\linewidth}}}]{% 
	  \tt {\tiny [GM$\langle$B]}  
	 SPRICH: Das klingt nach einem guten Plan! Ich stimme zu, die Objekte zunächst in der 4. Spalte oder der 4. Zeile zu platzieren. Lass uns beginnen, unsere Objekte in die 4. Spalte zu bewegen, um dann anzupassen. Sollen wir ein bestimmtes Objekt zuerst verschieben oder alle gleichzeitig? 
	  } 
	   } 
	   } 
	  \\ 
 

    \theutterance \stepcounter{utterance}  

    & & & \multicolumn{2}{p{0.3\linewidth}}{\cellcolor[rgb]{0.95,0.95,0.95}{%
	\makecell[{{p{\linewidth}}}]{% 
	  \tt {\tiny [GM$|$GM]}  
	 SPRICH: Das klingt nach einem guten Plan! Ich stimme zu, die Objekte zunächst in der 4. Spalte oder der 4. Zeile zu platzieren. Lass uns beginnen, unsere Objekte in die 4. Spalte zu bewegen, um dann anzupassen. Sollen wir ein bestimmtes Objekt zuerst verschieben oder alle gleichzeitig? 
	  } 
	   } 
	   } 
	 & & \\ 
 

    \theutterance \stepcounter{utterance}  

    & & \multicolumn{4}{p{0.6\linewidth}}{\cellcolor[rgb]{0.9,0.9,0.9}{%
	\makecell[{{p{\linewidth}}}]{% 
	  \tt {\tiny [A$\langle$GM]}  
	 Deine Nachricht wurde an deinen Mitspieler übermittelt.\\ \tt \\ \tt \\ \tt Ihr seid gerade bei Runde 2 von maximal 28.\\ \tt Ihr habt gemeinsam 1 von maximal 16 Strafpunkten angesammelt.\\ \tt Dein Mitspieler hat dir diese Nachricht geschickt:\\ \tt \\ \tt \\ \tt "Das klingt nach einem guten Plan! Ich stimme zu, die Objekte zunächst in der 4. Spalte oder der 4. Zeile zu platzieren. Lass uns beginnen, unsere Objekte in die 4. Spalte zu bewegen, um dann anzupassen. Sollen wir ein bestimmtes Objekt zuerst verschieben oder alle gleichzeitig?"\\ \tt \\ \tt \\ \tt Wie lautet dein nächster Befehl? 
	  } 
	   } 
	   } 
	 & \\ 
 

    \theutterance \stepcounter{utterance}  

    & \multicolumn{4}{p{0.6\linewidth}}{\cellcolor[rgb]{0.8,1,0.9}{%
	\makecell[{{p{\linewidth}}}]{% 
	  \tt {\tiny [A$\rangle$GM]}  
	 SPRICH: Ich denke, es wäre besser, wenn wir ein Objekt nach dem anderen bewegen, um sicherzustellen, dass wir keine Strafpunkte wegen Überlappung oder ungültiger Ziele sammeln. Lass uns mit dem Objekt 'P' beginnen, da es auf meinem Brett bereits relativ nahe am Ziel ist. Ich werde 'P' nach (4, 4) bewegen, um zu sehen, ob das funktioniert. Bewege bitte auf deinem Brett das Objekt 'P' in die Nähe von (4, 4), wenn möglich. Wir können dann unsere Fortschritte abstimmen und das nächste Objekt auswählen. 
	  } 
	   } 
	   } 
	 & & \\ 
 

    \theutterance \stepcounter{utterance}  

    & & & \multicolumn{2}{p{0.3\linewidth}}{\cellcolor[rgb]{0.95,0.95,0.95}{%
	\makecell[{{p{\linewidth}}}]{% 
	  \tt {\tiny [GM$|$GM]}  
	 SPRICH: Ich denke, es wäre besser, wenn wir ein Objekt nach dem anderen bewegen, um sicherzustellen, dass wir keine Strafpunkte wegen Überlappung oder ungültiger Ziele sammeln. Lass uns mit dem Objekt 'P' beginnen, da es auf meinem Brett bereits relativ nahe am Ziel ist. Ich werde 'P' nach (4, 4) bewegen, um zu sehen, ob das funktioniert. Bewege bitte auf deinem Brett das Objekt 'P' in die Nähe von (4, 4), wenn möglich. Wir können dann unsere Fortschritte abstimmen und das nächste Objekt auswählen. 
	  } 
	   } 
	   } 
	 & & \\ 
 

    \theutterance \stepcounter{utterance}  

    & & \multicolumn{4}{p{0.6\linewidth}}{\cellcolor[rgb]{0.9,0.9,0.9}{%
	\makecell[{{p{\linewidth}}}]{% 
	  \tt {\tiny [GM$\rangle$B]}  
	 Deine Nachricht wurde an deinen Mitspieler übermittelt.\\ \tt \\ \tt \\ \tt Ihr seid gerade bei Runde 3 von maximal 28.\\ \tt Ihr habt gemeinsam 1 von maximal 16 Strafpunkten angesammelt.\\ \tt Dein Mitspieler hat dir diese Nachricht geschickt:\\ \tt \\ \tt \\ \tt "Ich denke, es wäre besser, wenn wir ein Objekt nach dem anderen bewegen, um sicherzustellen, dass wir keine Strafpunkte wegen Überlappung oder ungültiger Ziele sammeln. Lass uns mit dem Objekt 'P' beginnen, da es auf meinem Brett bereits relativ nahe am Ziel ist. Ich werde 'P' nach (4, 4) bewegen, um zu sehen, ob das funktioniert. Bewege bitte auf deinem Brett das Objekt 'P' in die Nähe von (4, 4), wenn möglich. Wir können dann unsere Fortschritte abstimmen und das nächste Objekt auswählen."\\ \tt \\ \tt \\ \tt Wie lautet dein nächster Befehl? 
	  } 
	   } 
	   } 
	 & \\ 
 

    \theutterance \stepcounter{utterance}  

    & & & \multicolumn{4}{p{0.6\linewidth}}{\cellcolor[rgb]{1,0.85,0.72}{%
	\makecell[{{p{\linewidth}}}]{% 
	  \tt {\tiny [GM$\langle$B]}  
	 SPRICH: Ich verstehe deinen Plan und werde das Objekt 'P' auf meinem Brett in die Nähe von (4, 4) bewegen. Ich werde 'P' nach (4, 4) verschieben, da es auf meinem Brett aktuell weit entfernt ist. Ich werde berichten, ob das erfolgreich war. Wie sieht es auf deinem Brett aus, nachdem du 'P' bewegt hast? 
	  } 
	   } 
	   } 
	  \\ 
 

    \theutterance \stepcounter{utterance}  

    & & & \multicolumn{2}{p{0.3\linewidth}}{\cellcolor[rgb]{0.95,0.95,0.95}{%
	\makecell[{{p{\linewidth}}}]{% 
	  \tt {\tiny [GM$|$GM]}  
	 SPRICH: Ich verstehe deinen Plan und werde das Objekt 'P' auf meinem Brett in die Nähe von (4, 4) bewegen. Ich werde 'P' nach (4, 4) verschieben, da es auf meinem Brett aktuell weit entfernt ist. Ich werde berichten, ob das erfolgreich war. Wie sieht es auf deinem Brett aus, nachdem du 'P' bewegt hast? 
	  } 
	   } 
	   } 
	 & & \\ 
 

    \theutterance \stepcounter{utterance}  

    & & \multicolumn{4}{p{0.6\linewidth}}{\cellcolor[rgb]{0.9,0.9,0.9}{%
	\makecell[{{p{\linewidth}}}]{% 
	  \tt {\tiny [A$\langle$GM]}  
	 Deine Nachricht wurde an deinen Mitspieler übermittelt.\\ \tt \\ \tt \\ \tt Ihr seid gerade bei Runde 3 von maximal 28.\\ \tt Ihr habt gemeinsam 1 von maximal 16 Strafpunkten angesammelt.\\ \tt Dein Mitspieler hat dir diese Nachricht geschickt:\\ \tt \\ \tt \\ \tt "Ich verstehe deinen Plan und werde das Objekt 'P' auf meinem Brett in die Nähe von (4, 4) bewegen. Ich werde 'P' nach (4, 4) verschieben, da es auf meinem Brett aktuell weit entfernt ist. Ich werde berichten, ob das erfolgreich war. Wie sieht es auf deinem Brett aus, nachdem du 'P' bewegt hast?"\\ \tt \\ \tt \\ \tt Wie lautet dein nächster Befehl? 
	  } 
	   } 
	   } 
	 & \\ 
 

    \theutterance \stepcounter{utterance}  

    & \multicolumn{4}{p{0.6\linewidth}}{\cellcolor[rgb]{0.8,1,0.9}{%
	\makecell[{{p{\linewidth}}}]{% 
	  \tt {\tiny [A$\rangle$GM]}  
	 SPRICH: Auf meinem Brett habe ich das Objekt 'P' bereits auf (4, 5) positioniert, indem ich den Befehl `VERSCHIEBE: P, (4, 5)` verwendet habe, um zu sehen, wie das Brett aussieht, bevor ich es direkt nach (4, 4) bewege. Mein nächster Schritt wäre, 'P' nach (4, 4) zu bewegen, aber ich warte auf deine Bestätigung, dass dein 'P' erfolgreich verschoben wurde. Bitte teile mir mit, ob dein 'P' jetzt bei (4, 4) ist und wie dein Brett aussieht, damit wir synchron bleiben können. Ich werde dann 'P' auf (4, 4) platzieren, wenn alles okay ist. 
	  } 
	   } 
	   } 
	 & & \\ 
 

    \theutterance \stepcounter{utterance}  

    & & & \multicolumn{2}{p{0.3\linewidth}}{\cellcolor[rgb]{0.95,0.95,0.95}{%
	\makecell[{{p{\linewidth}}}]{% 
	  \tt {\tiny [GM$|$GM]}  
	 SPRICH: Auf meinem Brett habe ich das Objekt 'P' bereits auf (4, 5) positioniert, indem ich den Befehl `VERSCHIEBE: P, (4, 5)` verwendet habe, um zu sehen, wie das Brett aussieht, bevor ich es direkt nach (4, 4) bewege. Mein nächster Schritt wäre, 'P' nach (4, 4) zu bewegen, aber ich warte auf deine Bestätigung, dass dein 'P' erfolgreich verschoben wurde. Bitte teile mir mit, ob dein 'P' jetzt bei (4, 4) ist und wie dein Brett aussieht, damit wir synchron bleiben können. Ich werde dann 'P' auf (4, 4) platzieren, wenn alles okay ist. 
	  } 
	   } 
	   } 
	 & & \\ 
 

    \theutterance \stepcounter{utterance}  

    & & & \multicolumn{2}{p{0.3\linewidth}}{\cellcolor[rgb]{0.95,0.95,0.95}{%
	\makecell[{{p{\linewidth}}}]{% 
	  \tt {\tiny [GM$|$GM]}  
	 Invalid response format: SPRICH: Auf meinem Brett habe ich das Objekt 'P' bereits auf (4, 5) positioniert, indem ich den Befehl VERSCHIEBE: P, (4, 5) verwendet habe, um zu sehen, wie das Brett aussieht, bevor ich es direkt nach (4, 4) bewege. Mein nächster Schritt wäre, 'P' nach (4, 4) zu bewegen, aber ich warte auf deine Bestätigung, dass dein 'P' erfolgreich verschoben wurde. Bitte teile mir mit, ob dein 'P' jetzt bei (4, 4) ist und wie dein Brett aussieht, damit wir synchron bleiben können. Ich werde dann 'P' auf (4, 4) platzieren, wenn alles okay ist. 
	  } 
	   } 
	   } 
	 & & \\ 
 

    \theutterance \stepcounter{utterance}  

    & & \multicolumn{4}{p{0.6\linewidth}}{\cellcolor[rgb]{0.9,0.9,0.9}{%
	\makecell[{{p{\linewidth}}}]{% 
	  \tt {\tiny [A$\langle$GM]}  
	 Strafpunkt: Deine Nachricht enthält mehr als einen Befehl! Stelle sicher, dass deine Antwort ausschließlich einen der beiden Befehle `SPRICH: <NACHRICHT>` und `VERSCHIEBE: <OBJECT>, (<X>, <Y>)` und sonst nichts enthält!\\ \tt \\ \tt Ihr habt gemeinsam 2 von maximal 16 Strafpunkten angesammelt.\\ \tt Bitte versuche es noch mal! 
	  } 
	   } 
	   } 
	 & \\ 
 

    \theutterance \stepcounter{utterance}  

    & \multicolumn{4}{p{0.6\linewidth}}{\cellcolor[rgb]{0.8,1,0.9}{%
	\makecell[{{p{\linewidth}}}]{% 
	  \tt {\tiny [A$\rangle$GM]}  
	 SPRICH: Auf meinem Brett habe ich das Objekt 'P' bereits auf (4, 5) positioniert. Ich warte auf deine Bestätigung, dass dein 'P' erfolgreich verschoben wurde, bevor ich weitere Schritte unternehme. Bitte teile mir mit, ob dein 'P' jetzt bei (4, 4) ist und wie dein Brett aussieht. 
	  } 
	   } 
	   } 
	 & & \\ 
 

    \theutterance \stepcounter{utterance}  

    & & & \multicolumn{2}{p{0.3\linewidth}}{\cellcolor[rgb]{0.95,0.95,0.95}{%
	\makecell[{{p{\linewidth}}}]{% 
	  \tt {\tiny [GM$|$GM]}  
	 SPRICH: Auf meinem Brett habe ich das Objekt 'P' bereits auf (4, 5) positioniert. Ich warte auf deine Bestätigung, dass dein 'P' erfolgreich verschoben wurde, bevor ich weitere Schritte unternehme. Bitte teile mir mit, ob dein 'P' jetzt bei (4, 4) ist und wie dein Brett aussieht. 
	  } 
	   } 
	   } 
	 & & \\ 
 

    \theutterance \stepcounter{utterance}  

    & & \multicolumn{4}{p{0.6\linewidth}}{\cellcolor[rgb]{0.9,0.9,0.9}{%
	\makecell[{{p{\linewidth}}}]{% 
	  \tt {\tiny [GM$\rangle$B]}  
	 Deine Nachricht wurde an deinen Mitspieler übermittelt.\\ \tt \\ \tt \\ \tt Ihr seid gerade bei Runde 4 von maximal 28.\\ \tt Ihr habt gemeinsam 2 von maximal 16 Strafpunkten angesammelt.\\ \tt Dein Mitspieler hat dir diese Nachricht geschickt:\\ \tt \\ \tt \\ \tt "Auf meinem Brett habe ich das Objekt 'P' bereits auf (4, 5) positioniert. Ich warte auf deine Bestätigung, dass dein 'P' erfolgreich verschoben wurde, bevor ich weitere Schritte unternehme. Bitte teile mir mit, ob dein 'P' jetzt bei (4, 4) ist und wie dein Brett aussieht."\\ \tt \\ \tt \\ \tt Wie lautet dein nächster Befehl? 
	  } 
	   } 
	   } 
	 & \\ 
 

    \theutterance \stepcounter{utterance}  

    & & & \multicolumn{4}{p{0.6\linewidth}}{\cellcolor[rgb]{1,0.85,0.72}{%
	\makecell[{{p{\linewidth}}}]{% 
	  \tt {\tiny [GM$\langle$B]}  
	 VERSCHIEBE: P, (4, 4) 
	  } 
	   } 
	   } 
	  \\ 
 

    \theutterance \stepcounter{utterance}  

    & & & \multicolumn{2}{p{0.3\linewidth}}{\cellcolor[rgb]{0.95,0.95,0.95}{%
	\makecell[{{p{\linewidth}}}]{% 
	  \tt {\tiny [GM$|$GM]}  
	 VERSCHIEBE: P, (4, 4) 
	  } 
	   } 
	   } 
	 & & \\ 
 

    \theutterance \stepcounter{utterance}  

    & & & \multicolumn{2}{p{0.3\linewidth}}{\cellcolor[rgb]{0.95,0.95,0.95}{%
	\makecell[{{p{\linewidth}}}]{% 
	  \tt {\tiny [GM$|$GM]}  
	 Strafpunkt: (4,4) ist nicht leer, sondern enthält '─'.\\ \tt \\ \tt Ihr habt gemeinsam 3 von maximal 16 Strafpunkten angesammelt.\\ \tt Bitte versuche es noch mal! 
	  } 
	   } 
	   } 
	 & & \\ 
 

    \theutterance \stepcounter{utterance}  

    & & \multicolumn{4}{p{0.6\linewidth}}{\cellcolor[rgb]{0.9,0.9,0.9}{%
	\makecell[{{p{\linewidth}}}]{% 
	  \tt {\tiny [GM$\rangle$B]}  
	 Strafpunkt: (4,4) ist nicht leer, sondern enthält '─'.\\ \tt \\ \tt Ihr habt gemeinsam 3 von maximal 16 Strafpunkten angesammelt.\\ \tt Bitte versuche es noch mal! 
	  } 
	   } 
	   } 
	 & \\ 
 

    \theutterance \stepcounter{utterance}  

    & & & \multicolumn{4}{p{0.6\linewidth}}{\cellcolor[rgb]{1,0.85,0.72}{%
	\makecell[{{p{\linewidth}}}]{% 
	  \tt {\tiny [GM$\langle$B]}  
	 VERSCHIEBE: P, (3, 4) 
	  } 
	   } 
	   } 
	  \\ 
 

    \theutterance \stepcounter{utterance}  

    & & & \multicolumn{2}{p{0.3\linewidth}}{\cellcolor[rgb]{0.95,0.95,0.95}{%
	\makecell[{{p{\linewidth}}}]{% 
	  \tt {\tiny [GM$|$GM]}  
	 VERSCHIEBE: P, (3, 4) 
	  } 
	   } 
	   } 
	 & & \\ 
 

    \theutterance \stepcounter{utterance}  

    & & & \multicolumn{2}{p{0.3\linewidth}}{\cellcolor[rgb]{0.95,0.95,0.95}{%
	\makecell[{{p{\linewidth}}}]{% 
	  \tt {\tiny [GM$|$GM]}  
	 Strafpunkt: (3,4) ist nicht leer, sondern enthält '─'.\\ \tt \\ \tt Ihr habt gemeinsam 4 von maximal 16 Strafpunkten angesammelt.\\ \tt Bitte versuche es noch mal! 
	  } 
	   } 
	   } 
	 & & \\ 
 

    \theutterance \stepcounter{utterance}  

    & & \multicolumn{4}{p{0.6\linewidth}}{\cellcolor[rgb]{0.9,0.9,0.9}{%
	\makecell[{{p{\linewidth}}}]{% 
	  \tt {\tiny [GM$\rangle$B]}  
	 Strafpunkt: (3,4) ist nicht leer, sondern enthält '─'.\\ \tt \\ \tt Ihr habt gemeinsam 4 von maximal 16 Strafpunkten angesammelt.\\ \tt Bitte versuche es noch mal! 
	  } 
	   } 
	   } 
	 & \\ 
 

    \theutterance \stepcounter{utterance}  

    & & & \multicolumn{4}{p{0.6\linewidth}}{\cellcolor[rgb]{1,0.85,0.72}{%
	\makecell[{{p{\linewidth}}}]{% 
	  \tt {\tiny [GM$\langle$B]}  
	 VERSCHIEBE: P, (5, 4) 
	  } 
	   } 
	   } 
	  \\ 
 

    \theutterance \stepcounter{utterance}  

    & & & \multicolumn{2}{p{0.3\linewidth}}{\cellcolor[rgb]{0.95,0.95,0.95}{%
	\makecell[{{p{\linewidth}}}]{% 
	  \tt {\tiny [GM$|$GM]}  
	 VERSCHIEBE: P, (5, 4) 
	  } 
	   } 
	   } 
	 & & \\ 
 

    \theutterance \stepcounter{utterance}  

    & & & \multicolumn{2}{p{0.3\linewidth}}{\cellcolor[rgb]{0.95,0.95,0.95}{%
	\makecell[{{p{\linewidth}}}]{% 
	  \tt {\tiny [GM$|$GM]}  
	 Strafpunkt: (5,4) ist nicht leer, sondern enthält '┘'.\\ \tt \\ \tt Ihr habt gemeinsam 5 von maximal 16 Strafpunkten angesammelt.\\ \tt Bitte versuche es noch mal! 
	  } 
	   } 
	   } 
	 & & \\ 
 

    \theutterance \stepcounter{utterance}  

    & & \multicolumn{4}{p{0.6\linewidth}}{\cellcolor[rgb]{0.9,0.9,0.9}{%
	\makecell[{{p{\linewidth}}}]{% 
	  \tt {\tiny [GM$\rangle$B]}  
	 Strafpunkt: (5,4) ist nicht leer, sondern enthält '┘'.\\ \tt \\ \tt Ihr habt gemeinsam 5 von maximal 16 Strafpunkten angesammelt.\\ \tt Bitte versuche es noch mal! 
	  } 
	   } 
	   } 
	 & \\ 
 

    \theutterance \stepcounter{utterance}  

    & & & \multicolumn{4}{p{0.6\linewidth}}{\cellcolor[rgb]{1,0.85,0.72}{%
	\makecell[{{p{\linewidth}}}]{% 
	  \tt {\tiny [GM$\langle$B]}  
	 VERSCHIEBE: P, (4, 3) 
	  } 
	   } 
	   } 
	  \\ 
 

    \theutterance \stepcounter{utterance}  

    & & & \multicolumn{2}{p{0.3\linewidth}}{\cellcolor[rgb]{0.95,0.95,0.95}{%
	\makecell[{{p{\linewidth}}}]{% 
	  \tt {\tiny [GM$|$GM]}  
	 VERSCHIEBE: P, (4, 3) 
	  } 
	   } 
	   } 
	 & & \\ 
 

    \theutterance \stepcounter{utterance}  

    & & & \multicolumn{2}{p{0.3\linewidth}}{\cellcolor[rgb]{0.95,0.95,0.95}{%
	\makecell[{{p{\linewidth}}}]{% 
	  \tt {\tiny [GM$|$GM]}  
	 Strafpunkt: (4,3) ist nicht leer, sondern enthält 'M'.\\ \tt \\ \tt Ihr habt gemeinsam 6 von maximal 16 Strafpunkten angesammelt.\\ \tt Bitte versuche es noch mal! 
	  } 
	   } 
	   } 
	 & & \\ 
 

    \theutterance \stepcounter{utterance}  

    & & \multicolumn{4}{p{0.6\linewidth}}{\cellcolor[rgb]{0.9,0.9,0.9}{%
	\makecell[{{p{\linewidth}}}]{% 
	  \tt {\tiny [GM$\rangle$B]}  
	 Strafpunkt: (4,3) ist nicht leer, sondern enthält 'M'.\\ \tt \\ \tt Ihr habt gemeinsam 6 von maximal 16 Strafpunkten angesammelt.\\ \tt Bitte versuche es noch mal! 
	  } 
	   } 
	   } 
	 & \\ 
 

    \theutterance \stepcounter{utterance}  

    & & & \multicolumn{4}{p{0.6\linewidth}}{\cellcolor[rgb]{1,0.85,0.72}{%
	\makecell[{{p{\linewidth}}}]{% 
	  \tt {\tiny [GM$\langle$B]}  
	 VERSCHIEBE: P, (2, 4) 
	  } 
	   } 
	   } 
	  \\ 
 

    \theutterance \stepcounter{utterance}  

    & & & \multicolumn{2}{p{0.3\linewidth}}{\cellcolor[rgb]{0.95,0.95,0.95}{%
	\makecell[{{p{\linewidth}}}]{% 
	  \tt {\tiny [GM$|$GM]}  
	 VERSCHIEBE: P, (2, 4) 
	  } 
	   } 
	   } 
	 & & \\ 
 

    \theutterance \stepcounter{utterance}  

    & & & \multicolumn{2}{p{0.3\linewidth}}{\cellcolor[rgb]{0.95,0.95,0.95}{%
	\makecell[{{p{\linewidth}}}]{% 
	  \tt {\tiny [GM$|$GM]}  
	 Strafpunkt: (2,4) ist nicht leer, sondern enthält '┴'.\\ \tt \\ \tt Ihr habt gemeinsam 7 von maximal 16 Strafpunkten angesammelt.\\ \tt Bitte versuche es noch mal! 
	  } 
	   } 
	   } 
	 & & \\ 
 

    \theutterance \stepcounter{utterance}  

    & & \multicolumn{4}{p{0.6\linewidth}}{\cellcolor[rgb]{0.9,0.9,0.9}{%
	\makecell[{{p{\linewidth}}}]{% 
	  \tt {\tiny [GM$\rangle$B]}  
	 Strafpunkt: (2,4) ist nicht leer, sondern enthält '┴'.\\ \tt \\ \tt Ihr habt gemeinsam 7 von maximal 16 Strafpunkten angesammelt.\\ \tt Bitte versuche es noch mal! 
	  } 
	   } 
	   } 
	 & \\ 
 

    \theutterance \stepcounter{utterance}  

    & & & \multicolumn{4}{p{0.6\linewidth}}{\cellcolor[rgb]{1,0.85,0.72}{%
	\makecell[{{p{\linewidth}}}]{% 
	  \tt {\tiny [GM$\langle$B]}  
	 SPRICH: Ich habe Probleme, das Objekt 'P' zu bewegen, da viele Felder um (4, 4) herum nicht leer sind. Ich denke, wir sollten unsere Strategie anpassen und eine andere Position für 'P' finden. Was denkst du? 
	  } 
	   } 
	   } 
	  \\ 
 

    \theutterance \stepcounter{utterance}  

    & & & \multicolumn{2}{p{0.3\linewidth}}{\cellcolor[rgb]{0.95,0.95,0.95}{%
	\makecell[{{p{\linewidth}}}]{% 
	  \tt {\tiny [GM$|$GM]}  
	 SPRICH: Ich habe Probleme, das Objekt 'P' zu bewegen, da viele Felder um (4, 4) herum nicht leer sind. Ich denke, wir sollten unsere Strategie anpassen und eine andere Position für 'P' finden. Was denkst du? 
	  } 
	   } 
	   } 
	 & & \\ 
 

    \theutterance \stepcounter{utterance}  

    & & \multicolumn{4}{p{0.6\linewidth}}{\cellcolor[rgb]{0.9,0.9,0.9}{%
	\makecell[{{p{\linewidth}}}]{% 
	  \tt {\tiny [A$\langle$GM]}  
	 Deine Nachricht wurde an deinen Mitspieler übermittelt.\\ \tt \\ \tt \\ \tt Ihr seid gerade bei Runde 4 von maximal 28.\\ \tt Ihr habt gemeinsam 7 von maximal 16 Strafpunkten angesammelt.\\ \tt Dein Mitspieler hat dir diese Nachricht geschickt:\\ \tt \\ \tt \\ \tt "Ich habe Probleme, das Objekt 'P' zu bewegen, da viele Felder um (4, 4) herum nicht leer sind. Ich denke, wir sollten unsere Strategie anpassen und eine andere Position für 'P' finden. Was denkst du?"\\ \tt \\ \tt \\ \tt Wie lautet dein nächster Befehl? 
	  } 
	   } 
	   } 
	 & \\ 
 

    \theutterance \stepcounter{utterance}  

    & \multicolumn{4}{p{0.6\linewidth}}{\cellcolor[rgb]{0.8,1,0.9}{%
	\makecell[{{p{\linewidth}}}]{% 
	  \tt {\tiny [A$\rangle$GM]}  
	 SPRICH: Ich verstehe das Problem. Eine Alternative könnte sein, 'P' an eine Position zu bewegen, die leichter zu erreichen ist und dann schrittweise anzupassen. Lass uns eine Liste der Objekte erstellen und entscheiden, welches am einfachsten zu bewegen ist. Auf meinem Brett ist 'O' relativ frei, vielleicht sollten wir mit 'O' beginnen? 
	  } 
	   } 
	   } 
	 & & \\ 
 

    \theutterance \stepcounter{utterance}  

    & & & \multicolumn{2}{p{0.3\linewidth}}{\cellcolor[rgb]{0.95,0.95,0.95}{%
	\makecell[{{p{\linewidth}}}]{% 
	  \tt {\tiny [GM$|$GM]}  
	 SPRICH: Ich verstehe das Problem. Eine Alternative könnte sein, 'P' an eine Position zu bewegen, die leichter zu erreichen ist und dann schrittweise anzupassen. Lass uns eine Liste der Objekte erstellen und entscheiden, welches am einfachsten zu bewegen ist. Auf meinem Brett ist 'O' relativ frei, vielleicht sollten wir mit 'O' beginnen? 
	  } 
	   } 
	   } 
	 & & \\ 
 

    \theutterance \stepcounter{utterance}  

    & & \multicolumn{4}{p{0.6\linewidth}}{\cellcolor[rgb]{0.9,0.9,0.9}{%
	\makecell[{{p{\linewidth}}}]{% 
	  \tt {\tiny [GM$\rangle$B]}  
	 Deine Nachricht wurde an deinen Mitspieler übermittelt.\\ \tt \\ \tt \\ \tt Ihr seid gerade bei Runde 5 von maximal 28.\\ \tt Ihr habt gemeinsam 7 von maximal 16 Strafpunkten angesammelt.\\ \tt Dein Mitspieler hat dir diese Nachricht geschickt:\\ \tt \\ \tt \\ \tt "Ich verstehe das Problem. Eine Alternative könnte sein, 'P' an eine Position zu bewegen, die leichter zu erreichen ist und dann schrittweise anzupassen. Lass uns eine Liste der Objekte erstellen und entscheiden, welches am einfachsten zu bewegen ist. Auf meinem Brett ist 'O' relativ frei, vielleicht sollten wir mit 'O' beginnen?"\\ \tt \\ \tt \\ \tt Wie lautet dein nächster Befehl? 
	  } 
	   } 
	   } 
	 & \\ 
 

    \theutterance \stepcounter{utterance}  

    & & & \multicolumn{4}{p{0.6\linewidth}}{\cellcolor[rgb]{1,0.85,0.72}{%
	\makecell[{{p{\linewidth}}}]{% 
	  \tt {\tiny [GM$\langle$B]}  
	 SPRICH: Ich denke, das ist eine gute Idee, eine Liste der Objekte zu erstellen und zu entscheiden, welches am einfachsten zu bewegen ist. Auf meinem Brett ist 'D' relativ frei, vielleicht sollten wir auch 'D' in Betracht ziehen. Lass uns die Liste erstellen und gemeinsam entscheiden, welches Objekt wir zuerst bewegen sollten. 
	  } 
	   } 
	   } 
	  \\ 
 

    \theutterance \stepcounter{utterance}  

    & & & \multicolumn{2}{p{0.3\linewidth}}{\cellcolor[rgb]{0.95,0.95,0.95}{%
	\makecell[{{p{\linewidth}}}]{% 
	  \tt {\tiny [GM$|$GM]}  
	 SPRICH: Ich denke, das ist eine gute Idee, eine Liste der Objekte zu erstellen und zu entscheiden, welches am einfachsten zu bewegen ist. Auf meinem Brett ist 'D' relativ frei, vielleicht sollten wir auch 'D' in Betracht ziehen. Lass uns die Liste erstellen und gemeinsam entscheiden, welches Objekt wir zuerst bewegen sollten. 
	  } 
	   } 
	   } 
	 & & \\ 
 

    \theutterance \stepcounter{utterance}  

    & & \multicolumn{4}{p{0.6\linewidth}}{\cellcolor[rgb]{0.9,0.9,0.9}{%
	\makecell[{{p{\linewidth}}}]{% 
	  \tt {\tiny [A$\langle$GM]}  
	 Deine Nachricht wurde an deinen Mitspieler übermittelt.\\ \tt \\ \tt \\ \tt Ihr seid gerade bei Runde 5 von maximal 28.\\ \tt Ihr habt gemeinsam 7 von maximal 16 Strafpunkten angesammelt.\\ \tt Dein Mitspieler hat dir diese Nachricht geschickt:\\ \tt \\ \tt \\ \tt "Ich denke, das ist eine gute Idee, eine Liste der Objekte zu erstellen und zu entscheiden, welches am einfachsten zu bewegen ist. Auf meinem Brett ist 'D' relativ frei, vielleicht sollten wir auch 'D' in Betracht ziehen. Lass uns die Liste erstellen und gemeinsam entscheiden, welches Objekt wir zuerst bewegen sollten."\\ \tt \\ \tt \\ \tt Wie lautet dein nächster Befehl? 
	  } 
	   } 
	   } 
	 & \\ 
 

    \theutterance \stepcounter{utterance}  

    & \multicolumn{4}{p{0.6\linewidth}}{\cellcolor[rgb]{0.8,1,0.9}{%
	\makecell[{{p{\linewidth}}}]{% 
	  \tt {\tiny [A$\rangle$GM]}  
	 SPRICH: Das klingt gut! Ich denke, wir sollten auch berücksichtigen, welche Objekte auf beiden Brettern am nächsten bei (4, 4) sind, um den Bewegungsaufwand zu minimieren. Lass uns die Objekte nach ihrer Entfernung zu (4, 4) sortieren und dasjenige auswählen, das am leichtesten zu bewegen ist. Wir sollten auch sicherstellen, dass wir beide die gleichen Objekte priorisieren, um unsere Bemühungen zu koordinieren. Wie sieht deine Liste der Objekte aus, und welches Objekt denkst du, sollten wir zuerst bewegen? 
	  } 
	   } 
	   } 
	 & & \\ 
 

    \theutterance \stepcounter{utterance}  

    & & & \multicolumn{2}{p{0.3\linewidth}}{\cellcolor[rgb]{0.95,0.95,0.95}{%
	\makecell[{{p{\linewidth}}}]{% 
	  \tt {\tiny [GM$|$GM]}  
	 SPRICH: Das klingt gut! Ich denke, wir sollten auch berücksichtigen, welche Objekte auf beiden Brettern am nächsten bei (4, 4) sind, um den Bewegungsaufwand zu minimieren. Lass uns die Objekte nach ihrer Entfernung zu (4, 4) sortieren und dasjenige auswählen, das am leichtesten zu bewegen ist. Wir sollten auch sicherstellen, dass wir beide die gleichen Objekte priorisieren, um unsere Bemühungen zu koordinieren. Wie sieht deine Liste der Objekte aus, und welches Objekt denkst du, sollten wir zuerst bewegen? 
	  } 
	   } 
	   } 
	 & & \\ 
 

    \theutterance \stepcounter{utterance}  

    & & \multicolumn{4}{p{0.6\linewidth}}{\cellcolor[rgb]{0.9,0.9,0.9}{%
	\makecell[{{p{\linewidth}}}]{% 
	  \tt {\tiny [GM$\rangle$B]}  
	 Deine Nachricht wurde an deinen Mitspieler übermittelt.\\ \tt \\ \tt \\ \tt Ihr seid gerade bei Runde 6 von maximal 28.\\ \tt Ihr habt gemeinsam 7 von maximal 16 Strafpunkten angesammelt.\\ \tt Dein Mitspieler hat dir diese Nachricht geschickt:\\ \tt \\ \tt \\ \tt "Das klingt gut! Ich denke, wir sollten auch berücksichtigen, welche Objekte auf beiden Brettern am nächsten bei (4, 4) sind, um den Bewegungsaufwand zu minimieren. Lass uns die Objekte nach ihrer Entfernung zu (4, 4) sortieren und dasjenige auswählen, das am leichtesten zu bewegen ist. Wir sollten auch sicherstellen, dass wir beide die gleichen Objekte priorisieren, um unsere Bemühungen zu koordinieren. Wie sieht deine Liste der Objekte aus, und welches Objekt denkst du, sollten wir zuerst bewegen?"\\ \tt \\ \tt \\ \tt Wie lautet dein nächster Befehl? 
	  } 
	   } 
	   } 
	 & \\ 
 

    \theutterance \stepcounter{utterance}  

    & & & \multicolumn{4}{p{0.6\linewidth}}{\cellcolor[rgb]{1,0.85,0.72}{%
	\makecell[{{p{\linewidth}}}]{% 
	  \tt {\tiny [GM$\langle$B]}  
	 SPRICH: Ich denke, wir sollten die Objekte 'D', 'O', 'P', 'T', 'S', 'A' und 'M' nach ihrer Entfernung zu (4, 4) sortieren und dasjenige auswählen, das am leichtesten zu bewegen ist. Auf meinem Brett ist 'D' in der 1. Spalte und 'A' in der 7. Spalte. 'M' ist in der 3. Zeile und 2. Spalte, 'T' in der 3. Zeile und 4. Spalte, 'S' in der 7. Zeile und 3. Spalte, 'O' in der 6. Zeile und 5. Spalte und 'P' in der 4. Zeile und 1. Spalte. Ich denke, 'T' oder 'M' könnten gute Kandidaten sein, da sie relativ nahe bei (4, 4) sind. Welche Objekte hast du auf deinem Brett und wie sehen ihre Positionen aus? 
	  } 
	   } 
	   } 
	  \\ 
 

    \theutterance \stepcounter{utterance}  

    & & & \multicolumn{2}{p{0.3\linewidth}}{\cellcolor[rgb]{0.95,0.95,0.95}{%
	\makecell[{{p{\linewidth}}}]{% 
	  \tt {\tiny [GM$|$GM]}  
	 SPRICH: Ich denke, wir sollten die Objekte 'D', 'O', 'P', 'T', 'S', 'A' und 'M' nach ihrer Entfernung zu (4, 4) sortieren und dasjenige auswählen, das am leichtesten zu bewegen ist. Auf meinem Brett ist 'D' in der 1. Spalte und 'A' in der 7. Spalte. 'M' ist in der 3. Zeile und 2. Spalte, 'T' in der 3. Zeile und 4. Spalte, 'S' in der 7. Zeile und 3. Spalte, 'O' in der 6. Zeile und 5. Spalte und 'P' in der 4. Zeile und 1. Spalte. Ich denke, 'T' oder 'M' könnten gute Kandidaten sein, da sie relativ nahe bei (4, 4) sind. Welche Objekte hast du auf deinem Brett und wie sehen ihre Positionen aus? 
	  } 
	   } 
	   } 
	 & & \\ 
 

    \theutterance \stepcounter{utterance}  

    & & \multicolumn{4}{p{0.6\linewidth}}{\cellcolor[rgb]{0.9,0.9,0.9}{%
	\makecell[{{p{\linewidth}}}]{% 
	  \tt {\tiny [A$\langle$GM]}  
	 Deine Nachricht wurde an deinen Mitspieler übermittelt.\\ \tt \\ \tt \\ \tt Ihr seid gerade bei Runde 6 von maximal 28.\\ \tt Ihr habt gemeinsam 7 von maximal 16 Strafpunkten angesammelt.\\ \tt Dein Mitspieler hat dir diese Nachricht geschickt:\\ \tt \\ \tt \\ \tt "Ich denke, wir sollten die Objekte 'D', 'O', 'P', 'T', 'S', 'A' und 'M' nach ihrer Entfernung zu (4, 4) sortieren und dasjenige auswählen, das am leichtesten zu bewegen ist. Auf meinem Brett ist 'D' in der 1. Spalte und 'A' in der 7. Spalte. 'M' ist in der 3. Zeile und 2. Spalte, 'T' in der 3. Zeile und 4. Spalte, 'S' in der 7. Zeile und 3. Spalte, 'O' in der 6. Zeile und 5. Spalte und 'P' in der 4. Zeile und 1. Spalte. Ich denke, 'T' oder 'M' könnten gute Kandidaten sein, da sie relativ nahe bei (4, 4) sind. Welche Objekte hast du auf deinem Brett und wie sehen ihre Positionen aus?"\\ \tt \\ \tt \\ \tt Wie lautet dein nächster Befehl? 
	  } 
	   } 
	   } 
	 & \\ 
 

    \theutterance \stepcounter{utterance}  

    & \multicolumn{4}{p{0.6\linewidth}}{\cellcolor[rgb]{0.8,1,0.9}{%
	\makecell[{{p{\linewidth}}}]{% 
	  \tt {\tiny [A$\rangle$GM]}  
	 SPRICH: Auf meinem Brett habe ich 'D' in der 4. Zeile und 7. Spalte, 'A' in der 7. Zeile und 3. Spalte, 'M' in der 7. Zeile und 6. Spalte, 'T' in der 2. Zeile und 3. Spalte, 'S' in der 3. Zeile und 4. Spalte, 'O' in der 5. Zeile und 2. Spalte und 'P' in der 7. Zeile und 1. Spalte. Ich denke, 'T' oder 'S' könnten gute Kandidaten sein, da sie auf meinem Brett auch relativ nahe bei (4, 4) sind. Lass uns 'T' als erstes Ziel auswählen und versuchen, es auf beiden Brettern auf die gleiche Position zu bewegen. 
	  } 
	   } 
	   } 
	 & & \\ 
 

    \theutterance \stepcounter{utterance}  

    & & & \multicolumn{2}{p{0.3\linewidth}}{\cellcolor[rgb]{0.95,0.95,0.95}{%
	\makecell[{{p{\linewidth}}}]{% 
	  \tt {\tiny [GM$|$GM]}  
	 SPRICH: Auf meinem Brett habe ich 'D' in der 4. Zeile und 7. Spalte, 'A' in der 7. Zeile und 3. Spalte, 'M' in der 7. Zeile und 6. Spalte, 'T' in der 2. Zeile und 3. Spalte, 'S' in der 3. Zeile und 4. Spalte, 'O' in der 5. Zeile und 2. Spalte und 'P' in der 7. Zeile und 1. Spalte. Ich denke, 'T' oder 'S' könnten gute Kandidaten sein, da sie auf meinem Brett auch relativ nahe bei (4, 4) sind. Lass uns 'T' als erstes Ziel auswählen und versuchen, es auf beiden Brettern auf die gleiche Position zu bewegen. 
	  } 
	   } 
	   } 
	 & & \\ 
 

    \theutterance \stepcounter{utterance}  

    & & \multicolumn{4}{p{0.6\linewidth}}{\cellcolor[rgb]{0.9,0.9,0.9}{%
	\makecell[{{p{\linewidth}}}]{% 
	  \tt {\tiny [GM$\rangle$B]}  
	 Deine Nachricht wurde an deinen Mitspieler übermittelt.\\ \tt \\ \tt \\ \tt Ihr seid gerade bei Runde 7 von maximal 28.\\ \tt Ihr habt gemeinsam 7 von maximal 16 Strafpunkten angesammelt.\\ \tt Dein Mitspieler hat dir diese Nachricht geschickt:\\ \tt \\ \tt \\ \tt "Auf meinem Brett habe ich 'D' in der 4. Zeile und 7. Spalte, 'A' in der 7. Zeile und 3. Spalte, 'M' in der 7. Zeile und 6. Spalte, 'T' in der 2. Zeile und 3. Spalte, 'S' in der 3. Zeile und 4. Spalte, 'O' in der 5. Zeile und 2. Spalte und 'P' in der 7. Zeile und 1. Spalte. Ich denke, 'T' oder 'S' könnten gute Kandidaten sein, da sie auf meinem Brett auch relativ nahe bei (4, 4) sind. Lass uns 'T' als erstes Ziel auswählen und versuchen, es auf beiden Brettern auf die gleiche Position zu bewegen."\\ \tt \\ \tt \\ \tt Wie lautet dein nächster Befehl? 
	  } 
	   } 
	   } 
	 & \\ 
 

    \theutterance \stepcounter{utterance}  

    & & & \multicolumn{4}{p{0.6\linewidth}}{\cellcolor[rgb]{1,0.85,0.72}{%
	\makecell[{{p{\linewidth}}}]{% 
	  \tt {\tiny [GM$\langle$B]}  
	 VERSCHIEBE: T, (4, 3) 
	  } 
	   } 
	   } 
	  \\ 
 

    \theutterance \stepcounter{utterance}  

    & & & \multicolumn{2}{p{0.3\linewidth}}{\cellcolor[rgb]{0.95,0.95,0.95}{%
	\makecell[{{p{\linewidth}}}]{% 
	  \tt {\tiny [GM$|$GM]}  
	 VERSCHIEBE: T, (4, 3) 
	  } 
	   } 
	   } 
	 & & \\ 
 

    \theutterance \stepcounter{utterance}  

    & & & \multicolumn{2}{p{0.3\linewidth}}{\cellcolor[rgb]{0.95,0.95,0.95}{%
	\makecell[{{p{\linewidth}}}]{% 
	  \tt {\tiny [GM$|$GM]}  
	 Strafpunkt: (4,3) ist nicht leer, sondern enthält 'M'.\\ \tt \\ \tt Ihr habt gemeinsam 8 von maximal 16 Strafpunkten angesammelt.\\ \tt Bitte versuche es noch mal! 
	  } 
	   } 
	   } 
	 & & \\ 
 

    \theutterance \stepcounter{utterance}  

    & & \multicolumn{4}{p{0.6\linewidth}}{\cellcolor[rgb]{0.9,0.9,0.9}{%
	\makecell[{{p{\linewidth}}}]{% 
	  \tt {\tiny [GM$\rangle$B]}  
	 Strafpunkt: (4,3) ist nicht leer, sondern enthält 'M'.\\ \tt \\ \tt Ihr habt gemeinsam 8 von maximal 16 Strafpunkten angesammelt.\\ \tt Bitte versuche es noch mal! 
	  } 
	   } 
	   } 
	 & \\ 
 

    \theutterance \stepcounter{utterance}  

    & & & \multicolumn{4}{p{0.6\linewidth}}{\cellcolor[rgb]{1,0.85,0.72}{%
	\makecell[{{p{\linewidth}}}]{% 
	  \tt {\tiny [GM$\langle$B]}  
	 VERSCHIEBE: T, (5, 3) 
	  } 
	   } 
	   } 
	  \\ 
 

    \theutterance \stepcounter{utterance}  

    & & & \multicolumn{2}{p{0.3\linewidth}}{\cellcolor[rgb]{0.95,0.95,0.95}{%
	\makecell[{{p{\linewidth}}}]{% 
	  \tt {\tiny [GM$|$GM]}  
	 VERSCHIEBE: T, (5, 3) 
	  } 
	   } 
	   } 
	 & & \\ 
 

    \theutterance \stepcounter{utterance}  

    & & & \multicolumn{2}{p{0.3\linewidth}}{\cellcolor[rgb]{0.95,0.95,0.95}{%
	\makecell[{{p{\linewidth}}}]{% 
	  \tt {\tiny [GM$|$GM]}  
	 Strafpunkt: (5,3) ist nicht leer, sondern enthält '│'.\\ \tt \\ \tt Ihr habt gemeinsam 9 von maximal 16 Strafpunkten angesammelt.\\ \tt Bitte versuche es noch mal! 
	  } 
	   } 
	   } 
	 & & \\ 
 

    \theutterance \stepcounter{utterance}  

    & & \multicolumn{4}{p{0.6\linewidth}}{\cellcolor[rgb]{0.9,0.9,0.9}{%
	\makecell[{{p{\linewidth}}}]{% 
	  \tt {\tiny [GM$\rangle$B]}  
	 Strafpunkt: (5,3) ist nicht leer, sondern enthält '│'.\\ \tt \\ \tt Ihr habt gemeinsam 9 von maximal 16 Strafpunkten angesammelt.\\ \tt Bitte versuche es noch mal! 
	  } 
	   } 
	   } 
	 & \\ 
 

    \theutterance \stepcounter{utterance}  

    & & & \multicolumn{4}{p{0.6\linewidth}}{\cellcolor[rgb]{1,0.85,0.72}{%
	\makecell[{{p{\linewidth}}}]{% 
	  \tt {\tiny [GM$\langle$B]}  
	 VERSCHIEBE: T, (4, 2) 
	  } 
	   } 
	   } 
	  \\ 
 

    \theutterance \stepcounter{utterance}  

    & & & \multicolumn{2}{p{0.3\linewidth}}{\cellcolor[rgb]{0.95,0.95,0.95}{%
	\makecell[{{p{\linewidth}}}]{% 
	  \tt {\tiny [GM$|$GM]}  
	 VERSCHIEBE: T, (4, 2) 
	  } 
	   } 
	   } 
	 & & \\ 
 

    \theutterance \stepcounter{utterance}  

    & & & \multicolumn{2}{p{0.3\linewidth}}{\cellcolor[rgb]{0.95,0.95,0.95}{%
	\makecell[{{p{\linewidth}}}]{% 
	  \tt {\tiny [GM$|$GM]}  
	 Strafpunkt: (4,2) ist nicht leer, sondern enthält '─'.\\ \tt \\ \tt Ihr habt gemeinsam 10 von maximal 16 Strafpunkten angesammelt.\\ \tt Bitte versuche es noch mal! 
	  } 
	   } 
	   } 
	 & & \\ 
 

    \theutterance \stepcounter{utterance}  

    & & \multicolumn{4}{p{0.6\linewidth}}{\cellcolor[rgb]{0.9,0.9,0.9}{%
	\makecell[{{p{\linewidth}}}]{% 
	  \tt {\tiny [GM$\rangle$B]}  
	 Strafpunkt: (4,2) ist nicht leer, sondern enthält '─'.\\ \tt \\ \tt Ihr habt gemeinsam 10 von maximal 16 Strafpunkten angesammelt.\\ \tt Bitte versuche es noch mal! 
	  } 
	   } 
	   } 
	 & \\ 
 

    \theutterance \stepcounter{utterance}  

    & & & \multicolumn{4}{p{0.6\linewidth}}{\cellcolor[rgb]{1,0.85,0.72}{%
	\makecell[{{p{\linewidth}}}]{% 
	  \tt {\tiny [GM$\langle$B]}  
	 VERSCHIEBE: T, (3, 2) 
	  } 
	   } 
	   } 
	  \\ 
 

    \theutterance \stepcounter{utterance}  

    & & & \multicolumn{2}{p{0.3\linewidth}}{\cellcolor[rgb]{0.95,0.95,0.95}{%
	\makecell[{{p{\linewidth}}}]{% 
	  \tt {\tiny [GM$|$GM]}  
	 VERSCHIEBE: T, (3, 2) 
	  } 
	   } 
	   } 
	 & & \\ 
 

    \theutterance \stepcounter{utterance}  

    & & & \multicolumn{2}{p{0.3\linewidth}}{\cellcolor[rgb]{0.95,0.95,0.95}{%
	\makecell[{{p{\linewidth}}}]{% 
	  \tt {\tiny [GM$|$GM]}  
	 Strafpunkt: (3,2) ist nicht leer, sondern enthält '─'.\\ \tt \\ \tt Ihr habt gemeinsam 11 von maximal 16 Strafpunkten angesammelt.\\ \tt Bitte versuche es noch mal! 
	  } 
	   } 
	   } 
	 & & \\ 
 

    \theutterance \stepcounter{utterance}  

    & & \multicolumn{4}{p{0.6\linewidth}}{\cellcolor[rgb]{0.9,0.9,0.9}{%
	\makecell[{{p{\linewidth}}}]{% 
	  \tt {\tiny [GM$\rangle$B]}  
	 Strafpunkt: (3,2) ist nicht leer, sondern enthält '─'.\\ \tt \\ \tt Ihr habt gemeinsam 11 von maximal 16 Strafpunkten angesammelt.\\ \tt Bitte versuche es noch mal! 
	  } 
	   } 
	   } 
	 & \\ 
 

    \theutterance \stepcounter{utterance}  

    & & & \multicolumn{4}{p{0.6\linewidth}}{\cellcolor[rgb]{1,0.85,0.72}{%
	\makecell[{{p{\linewidth}}}]{% 
	  \tt {\tiny [GM$\langle$B]}  
	 SPRICH: Ich habe Schwierigkeiten, das Objekt 'T' zu bewegen, da viele Felder um das Ziel herum nicht leer sind. Ich denke, wir sollten unsere Strategie anpassen und eine andere Position für 'T' finden. Welche Alternativen siehst du? 
	  } 
	   } 
	   } 
	  \\ 
 

    \theutterance \stepcounter{utterance}  

    & & & \multicolumn{2}{p{0.3\linewidth}}{\cellcolor[rgb]{0.95,0.95,0.95}{%
	\makecell[{{p{\linewidth}}}]{% 
	  \tt {\tiny [GM$|$GM]}  
	 SPRICH: Ich habe Schwierigkeiten, das Objekt 'T' zu bewegen, da viele Felder um das Ziel herum nicht leer sind. Ich denke, wir sollten unsere Strategie anpassen und eine andere Position für 'T' finden. Welche Alternativen siehst du? 
	  } 
	   } 
	   } 
	 & & \\ 
 

    \theutterance \stepcounter{utterance}  

    & & \multicolumn{4}{p{0.6\linewidth}}{\cellcolor[rgb]{0.9,0.9,0.9}{%
	\makecell[{{p{\linewidth}}}]{% 
	  \tt {\tiny [A$\langle$GM]}  
	 Deine Nachricht wurde an deinen Mitspieler übermittelt.\\ \tt \\ \tt \\ \tt Ihr seid gerade bei Runde 7 von maximal 28.\\ \tt Ihr habt gemeinsam 11 von maximal 16 Strafpunkten angesammelt.\\ \tt Dein Mitspieler hat dir diese Nachricht geschickt:\\ \tt \\ \tt \\ \tt "Ich habe Schwierigkeiten, das Objekt 'T' zu bewegen, da viele Felder um das Ziel herum nicht leer sind. Ich denke, wir sollten unsere Strategie anpassen und eine andere Position für 'T' finden. Welche Alternativen siehst du?"\\ \tt \\ \tt \\ \tt Wie lautet dein nächster Befehl? 
	  } 
	   } 
	   } 
	 & \\ 
 

    \theutterance \stepcounter{utterance}  

    & \multicolumn{4}{p{0.6\linewidth}}{\cellcolor[rgb]{0.8,1,0.9}{%
	\makecell[{{p{\linewidth}}}]{% 
	  \tt {\tiny [A$\rangle$GM]}  
	 SPRICH: Ich verstehe das Problem. Eine Alternative könnte sein, 'S' zu bewegen, da es auf meinem Brett bereits in der 3. Zeile und 4. Spalte steht, also sehr nah am Ziel (4, 4). Vielleicht ist es einfacher, 'S' direkt nach (4, 4) zu bewegen, anstatt 'T' zu versuchen. Was meinst du? Sollten wir 'S' als neues Ziel auswählen? 
	  } 
	   } 
	   } 
	 & & \\ 
 

    \theutterance \stepcounter{utterance}  

    & & & \multicolumn{2}{p{0.3\linewidth}}{\cellcolor[rgb]{0.95,0.95,0.95}{%
	\makecell[{{p{\linewidth}}}]{% 
	  \tt {\tiny [GM$|$GM]}  
	 SPRICH: Ich verstehe das Problem. Eine Alternative könnte sein, 'S' zu bewegen, da es auf meinem Brett bereits in der 3. Zeile und 4. Spalte steht, also sehr nah am Ziel (4, 4). Vielleicht ist es einfacher, 'S' direkt nach (4, 4) zu bewegen, anstatt 'T' zu versuchen. Was meinst du? Sollten wir 'S' als neues Ziel auswählen? 
	  } 
	   } 
	   } 
	 & & \\ 
 

    \theutterance \stepcounter{utterance}  

    & & \multicolumn{4}{p{0.6\linewidth}}{\cellcolor[rgb]{0.9,0.9,0.9}{%
	\makecell[{{p{\linewidth}}}]{% 
	  \tt {\tiny [GM$\rangle$B]}  
	 Deine Nachricht wurde an deinen Mitspieler übermittelt.\\ \tt \\ \tt \\ \tt Ihr seid gerade bei Runde 8 von maximal 28.\\ \tt Ihr habt gemeinsam 11 von maximal 16 Strafpunkten angesammelt.\\ \tt Dein Mitspieler hat dir diese Nachricht geschickt:\\ \tt \\ \tt \\ \tt "Ich verstehe das Problem. Eine Alternative könnte sein, 'S' zu bewegen, da es auf meinem Brett bereits in der 3. Zeile und 4. Spalte steht, also sehr nah am Ziel (4, 4). Vielleicht ist es einfacher, 'S' direkt nach (4, 4) zu bewegen, anstatt 'T' zu versuchen. Was meinst du? Sollten wir 'S' als neues Ziel auswählen?"\\ \tt \\ \tt \\ \tt Wie lautet dein nächster Befehl? 
	  } 
	   } 
	   } 
	 & \\ 
 

    \theutterance \stepcounter{utterance}  

    & & & \multicolumn{4}{p{0.6\linewidth}}{\cellcolor[rgb]{1,0.85,0.72}{%
	\makecell[{{p{\linewidth}}}]{% 
	  \tt {\tiny [GM$\langle$B]}  
	 SPRICH: Das klingt nach einem guten Plan! Ich denke, wir sollten 'S' als neues Ziel auswählen und versuchen, es auf beiden Brettern auf die gleiche Position zu bewegen. Auf meinem Brett ist 'S' in der 7. Zeile und 3. Spalte, also noch ein bisschen entfernt. Lass uns 'S' nach (4, 4) bewegen und sehen, ob wir es schaffen, es auf beiden Brettern an die gleiche Position zu bringen. Sollten wir zuerst mein 'S' oder dein 'S' bewegen? 
	  } 
	   } 
	   } 
	  \\ 
 

    \theutterance \stepcounter{utterance}  

    & & & \multicolumn{2}{p{0.3\linewidth}}{\cellcolor[rgb]{0.95,0.95,0.95}{%
	\makecell[{{p{\linewidth}}}]{% 
	  \tt {\tiny [GM$|$GM]}  
	 SPRICH: Das klingt nach einem guten Plan! Ich denke, wir sollten 'S' als neues Ziel auswählen und versuchen, es auf beiden Brettern auf die gleiche Position zu bewegen. Auf meinem Brett ist 'S' in der 7. Zeile und 3. Spalte, also noch ein bisschen entfernt. Lass uns 'S' nach (4, 4) bewegen und sehen, ob wir es schaffen, es auf beiden Brettern an die gleiche Position zu bringen. Sollten wir zuerst mein 'S' oder dein 'S' bewegen? 
	  } 
	   } 
	   } 
	 & & \\ 
 

    \theutterance \stepcounter{utterance}  

    & & \multicolumn{4}{p{0.6\linewidth}}{\cellcolor[rgb]{0.9,0.9,0.9}{%
	\makecell[{{p{\linewidth}}}]{% 
	  \tt {\tiny [A$\langle$GM]}  
	 Deine Nachricht wurde an deinen Mitspieler übermittelt.\\ \tt \\ \tt \\ \tt Ihr seid gerade bei Runde 8 von maximal 28.\\ \tt Ihr habt gemeinsam 11 von maximal 16 Strafpunkten angesammelt.\\ \tt Dein Mitspieler hat dir diese Nachricht geschickt:\\ \tt \\ \tt \\ \tt "Das klingt nach einem guten Plan! Ich denke, wir sollten 'S' als neues Ziel auswählen und versuchen, es auf beiden Brettern auf die gleiche Position zu bewegen. Auf meinem Brett ist 'S' in der 7. Zeile und 3. Spalte, also noch ein bisschen entfernt. Lass uns 'S' nach (4, 4) bewegen und sehen, ob wir es schaffen, es auf beiden Brettern an die gleiche Position zu bringen. Sollten wir zuerst mein 'S' oder dein 'S' bewegen?"\\ \tt \\ \tt \\ \tt Wie lautet dein nächster Befehl? 
	  } 
	   } 
	   } 
	 & \\ 
 

    \theutterance \stepcounter{utterance}  

    & \multicolumn{4}{p{0.6\linewidth}}{\cellcolor[rgb]{0.8,1,0.9}{%
	\makecell[{{p{\linewidth}}}]{% 
	  \tt {\tiny [A$\rangle$GM]}  
	 SPRICH: Ich denke, wir sollten zuerst mein 'S' bewegen, da es auf meinem Brett bereits in der 3. Zeile und 4. Spalte steht und somit nur eine Zeile nach unten bewegt werden muss, um es auf (4, 4) zu platzieren. Das sollte relativ einfach sein. Bitte versuche, dein 'S' in die Nähe von (4, 4) zu bewegen, während ich mein 'S' direkt nach (4, 4) bewege. 
	  } 
	   } 
	   } 
	 & & \\ 
 

    \theutterance \stepcounter{utterance}  

    & & & \multicolumn{2}{p{0.3\linewidth}}{\cellcolor[rgb]{0.95,0.95,0.95}{%
	\makecell[{{p{\linewidth}}}]{% 
	  \tt {\tiny [GM$|$GM]}  
	 SPRICH: Ich denke, wir sollten zuerst mein 'S' bewegen, da es auf meinem Brett bereits in der 3. Zeile und 4. Spalte steht und somit nur eine Zeile nach unten bewegt werden muss, um es auf (4, 4) zu platzieren. Das sollte relativ einfach sein. Bitte versuche, dein 'S' in die Nähe von (4, 4) zu bewegen, während ich mein 'S' direkt nach (4, 4) bewege. 
	  } 
	   } 
	   } 
	 & & \\ 
 

    \theutterance \stepcounter{utterance}  

    & & \multicolumn{4}{p{0.6\linewidth}}{\cellcolor[rgb]{0.9,0.9,0.9}{%
	\makecell[{{p{\linewidth}}}]{% 
	  \tt {\tiny [GM$\rangle$B]}  
	 Deine Nachricht wurde an deinen Mitspieler übermittelt.\\ \tt \\ \tt \\ \tt Ihr seid gerade bei Runde 9 von maximal 28.\\ \tt Ihr habt gemeinsam 11 von maximal 16 Strafpunkten angesammelt.\\ \tt Dein Mitspieler hat dir diese Nachricht geschickt:\\ \tt \\ \tt \\ \tt "Ich denke, wir sollten zuerst mein 'S' bewegen, da es auf meinem Brett bereits in der 3. Zeile und 4. Spalte steht und somit nur eine Zeile nach unten bewegt werden muss, um es auf (4, 4) zu platzieren. Das sollte relativ einfach sein. Bitte versuche, dein 'S' in die Nähe von (4, 4) zu bewegen, während ich mein 'S' direkt nach (4, 4) bewege."\\ \tt \\ \tt \\ \tt Wie lautet dein nächster Befehl? 
	  } 
	   } 
	   } 
	 & \\ 
 

    \theutterance \stepcounter{utterance}  

    & & & \multicolumn{4}{p{0.6\linewidth}}{\cellcolor[rgb]{1,0.85,0.72}{%
	\makecell[{{p{\linewidth}}}]{% 
	  \tt {\tiny [GM$\langle$B]}  
	 VERSCHIEBE: S, (4, 6) 
	  } 
	   } 
	   } 
	  \\ 
 

    \theutterance \stepcounter{utterance}  

    & & & \multicolumn{2}{p{0.3\linewidth}}{\cellcolor[rgb]{0.95,0.95,0.95}{%
	\makecell[{{p{\linewidth}}}]{% 
	  \tt {\tiny [GM$|$GM]}  
	 VERSCHIEBE: S, (4, 6) 
	  } 
	   } 
	   } 
	 & & \\ 
 

    \theutterance \stepcounter{utterance}  

    & & & \multicolumn{2}{p{0.3\linewidth}}{\cellcolor[rgb]{0.95,0.95,0.95}{%
	\makecell[{{p{\linewidth}}}]{% 
	  \tt {\tiny [GM$|$GM]}  
	 'S' erfolgreich nach (4,6) verschoben. Dein Raster sieht jetzt so aus:\\ \tt ```\\ \tt  1234567\\ \tt ╔═╤═══╤═╗\\ \tt ║◌│◌◌◌│◌║ 1\\ \tt ╟─┼──┬┘◌║ 2\\ \tt ║◌│◌M│◌T║ 3\\ \tt ╟─┴──┘P◌║ 4\\ \tt ║D◌◌◌◌A◌║ 5\\ \tt ║◌◌◌SO◌◌║ 6\\ \tt ║◌◌◌◌◌◌◌║ 7\\ \tt ╚═══════╝\\ \tt ```\\ \tt  
	  } 
	   } 
	   } 
	 & & \\ 
 

    \theutterance \stepcounter{utterance}  

    & & \multicolumn{4}{p{0.6\linewidth}}{\cellcolor[rgb]{0.9,0.9,0.9}{%
	\makecell[{{p{\linewidth}}}]{% 
	  \tt {\tiny [A$\langle$GM]}  
	 Deine Nachricht wurde an deinen Mitspieler übermittelt.\\ \tt \\ \tt \\ \tt Ihr seid gerade bei Runde 9 von maximal 28.\\ \tt Ihr habt gemeinsam 11 von maximal 16 Strafpunkten angesammelt.\\ \tt Dein Mitspieler hat ein Objekt auf seinem Raster bewegt.\\ \tt \\ \tt \\ \tt Wie lautet dein nächster Befehl? 
	  } 
	   } 
	   } 
	 & \\ 
 

    \theutterance \stepcounter{utterance}  

    & \multicolumn{4}{p{0.6\linewidth}}{\cellcolor[rgb]{0.8,1,0.9}{%
	\makecell[{{p{\linewidth}}}]{% 
	  \tt {\tiny [A$\rangle$GM]}  
	 VERSCHIEBE: S, (4, 4) 
	  } 
	   } 
	   } 
	 & & \\ 
 

    \theutterance \stepcounter{utterance}  

    & & & \multicolumn{2}{p{0.3\linewidth}}{\cellcolor[rgb]{0.95,0.95,0.95}{%
	\makecell[{{p{\linewidth}}}]{% 
	  \tt {\tiny [GM$|$GM]}  
	 VERSCHIEBE: S, (4, 4) 
	  } 
	   } 
	   } 
	 & & \\ 
 

    \theutterance \stepcounter{utterance}  

    & & & \multicolumn{2}{p{0.3\linewidth}}{\cellcolor[rgb]{0.95,0.95,0.95}{%
	\makecell[{{p{\linewidth}}}]{% 
	  \tt {\tiny [GM$|$GM]}  
	 Strafpunkt: (4,4) ist nicht leer, sondern enthält '─'.\\ \tt \\ \tt Ihr habt gemeinsam 12 von maximal 16 Strafpunkten angesammelt.\\ \tt Bitte versuche es noch mal! 
	  } 
	   } 
	   } 
	 & & \\ 
 

    \theutterance \stepcounter{utterance}  

    & & \multicolumn{4}{p{0.6\linewidth}}{\cellcolor[rgb]{0.9,0.9,0.9}{%
	\makecell[{{p{\linewidth}}}]{% 
	  \tt {\tiny [A$\langle$GM]}  
	 Strafpunkt: (4,4) ist nicht leer, sondern enthält '─'.\\ \tt \\ \tt Ihr habt gemeinsam 12 von maximal 16 Strafpunkten angesammelt.\\ \tt Bitte versuche es noch mal! 
	  } 
	   } 
	   } 
	 & \\ 
 

    \theutterance \stepcounter{utterance}  

    & \multicolumn{4}{p{0.6\linewidth}}{\cellcolor[rgb]{0.8,1,0.9}{%
	\makecell[{{p{\linewidth}}}]{% 
	  \tt {\tiny [A$\rangle$GM]}  
	 VERSCHIEBE: S, (3, 4) 
	  } 
	   } 
	   } 
	 & & \\ 
 

    \theutterance \stepcounter{utterance}  

    & & & \multicolumn{2}{p{0.3\linewidth}}{\cellcolor[rgb]{0.95,0.95,0.95}{%
	\makecell[{{p{\linewidth}}}]{% 
	  \tt {\tiny [GM$|$GM]}  
	 VERSCHIEBE: S, (3, 4) 
	  } 
	   } 
	   } 
	 & & \\ 
 

    \theutterance \stepcounter{utterance}  

    & & & \multicolumn{2}{p{0.3\linewidth}}{\cellcolor[rgb]{0.95,0.95,0.95}{%
	\makecell[{{p{\linewidth}}}]{% 
	  \tt {\tiny [GM$|$GM]}  
	 Strafpunkt: (3,4) ist nicht leer, sondern enthält '─'.\\ \tt \\ \tt Ihr habt gemeinsam 13 von maximal 16 Strafpunkten angesammelt.\\ \tt Bitte versuche es noch mal! 
	  } 
	   } 
	   } 
	 & & \\ 
 

    \theutterance \stepcounter{utterance}  

    & & \multicolumn{4}{p{0.6\linewidth}}{\cellcolor[rgb]{0.9,0.9,0.9}{%
	\makecell[{{p{\linewidth}}}]{% 
	  \tt {\tiny [A$\langle$GM]}  
	 Strafpunkt: (3,4) ist nicht leer, sondern enthält '─'.\\ \tt \\ \tt Ihr habt gemeinsam 13 von maximal 16 Strafpunkten angesammelt.\\ \tt Bitte versuche es noch mal! 
	  } 
	   } 
	   } 
	 & \\ 
 

    \theutterance \stepcounter{utterance}  

    & \multicolumn{4}{p{0.6\linewidth}}{\cellcolor[rgb]{0.8,1,0.9}{%
	\makecell[{{p{\linewidth}}}]{% 
	  \tt {\tiny [A$\rangle$GM]}  
	 VERSCHIEBE: S, (5, 4) 
	  } 
	   } 
	   } 
	 & & \\ 
 

    \theutterance \stepcounter{utterance}  

    & & & \multicolumn{2}{p{0.3\linewidth}}{\cellcolor[rgb]{0.95,0.95,0.95}{%
	\makecell[{{p{\linewidth}}}]{% 
	  \tt {\tiny [GM$|$GM]}  
	 VERSCHIEBE: S, (5, 4) 
	  } 
	   } 
	   } 
	 & & \\ 
 

    \theutterance \stepcounter{utterance}  

    & & & \multicolumn{2}{p{0.3\linewidth}}{\cellcolor[rgb]{0.95,0.95,0.95}{%
	\makecell[{{p{\linewidth}}}]{% 
	  \tt {\tiny [GM$|$GM]}  
	 Strafpunkt: (5,4) ist nicht leer, sondern enthält '┘'.\\ \tt \\ \tt Ihr habt gemeinsam 14 von maximal 16 Strafpunkten angesammelt.\\ \tt Bitte versuche es noch mal! 
	  } 
	   } 
	   } 
	 & & \\ 
 

    \theutterance \stepcounter{utterance}  

    & & \multicolumn{4}{p{0.6\linewidth}}{\cellcolor[rgb]{0.9,0.9,0.9}{%
	\makecell[{{p{\linewidth}}}]{% 
	  \tt {\tiny [A$\langle$GM]}  
	 Strafpunkt: (5,4) ist nicht leer, sondern enthält '┘'.\\ \tt \\ \tt Ihr habt gemeinsam 14 von maximal 16 Strafpunkten angesammelt.\\ \tt Bitte versuche es noch mal! 
	  } 
	   } 
	   } 
	 & \\ 
 

    \theutterance \stepcounter{utterance}  

    & \multicolumn{4}{p{0.6\linewidth}}{\cellcolor[rgb]{0.8,1,0.9}{%
	\makecell[{{p{\linewidth}}}]{% 
	  \tt {\tiny [A$\rangle$GM]}  
	 VERSCHIEBE: S, (4, 3) 
	  } 
	   } 
	   } 
	 & & \\ 
 

    \theutterance \stepcounter{utterance}  

    & & & \multicolumn{2}{p{0.3\linewidth}}{\cellcolor[rgb]{0.95,0.95,0.95}{%
	\makecell[{{p{\linewidth}}}]{% 
	  \tt {\tiny [GM$|$GM]}  
	 VERSCHIEBE: S, (4, 3) 
	  } 
	   } 
	   } 
	 & & \\ 
 

    \theutterance \stepcounter{utterance}  

    & & & \multicolumn{2}{p{0.3\linewidth}}{\cellcolor[rgb]{0.95,0.95,0.95}{%
	\makecell[{{p{\linewidth}}}]{% 
	  \tt {\tiny [GM$|$GM]}  
	 'S' erfolgreich nach (4,3) verschoben. Dein Raster sieht jetzt so aus:\\ \tt ```\\ \tt  1234567\\ \tt ╔═╤═══╤═╗\\ \tt ║◌│◌◌◌│◌║ 1\\ \tt ╟─┼──┬┘◌║ 2\\ \tt ║◌│◌S│◌◌║ 3\\ \tt ╟─┴──┘DT║ 4\\ \tt ║◌O◌◌◌◌◌║ 5\\ \tt ║◌◌◌◌◌◌◌║ 6\\ \tt ║◌◌APM◌◌║ 7\\ \tt ╚═══════╝\\ \tt ```\\ \tt  
	  } 
	   } 
	   } 
	 & & \\ 
 

    \theutterance \stepcounter{utterance}  

    & & \multicolumn{4}{p{0.6\linewidth}}{\cellcolor[rgb]{0.9,0.9,0.9}{%
	\makecell[{{p{\linewidth}}}]{% 
	  \tt {\tiny [GM$\rangle$B]}  
	 'S' erfolgreich nach (4,6) verschoben. Dein Raster sieht jetzt so aus:\\ \tt ```\\ \tt  1234567\\ \tt ╔═╤═══╤═╗\\ \tt ║◌│◌◌◌│◌║ 1\\ \tt ╟─┼──┬┘◌║ 2\\ \tt ║◌│◌M│◌T║ 3\\ \tt ╟─┴──┘P◌║ 4\\ \tt ║D◌◌◌◌A◌║ 5\\ \tt ║◌◌◌SO◌◌║ 6\\ \tt ║◌◌◌◌◌◌◌║ 7\\ \tt ╚═══════╝\\ \tt ```\\ \tt Ihr seid gerade bei Runde 10 von maximal 28.\\ \tt Ihr habt gemeinsam 14 von maximal 16 Strafpunkten angesammelt.\\ \tt Dein Mitspieler hat ein Objekt auf seinem Raster bewegt.\\ \tt \\ \tt \\ \tt Wie lautet dein nächster Befehl? 
	  } 
	   } 
	   } 
	 & \\ 
 

    \theutterance \stepcounter{utterance}  

    & & & \multicolumn{4}{p{0.6\linewidth}}{\cellcolor[rgb]{1,0.85,0.72}{%
	\makecell[{{p{\linewidth}}}]{% 
	  \tt {\tiny [GM$\langle$B]}  
	 SPRICH: Ich habe mein 'S' nach (4, 6) verschoben. Wie sieht dein Raster aus, nachdem du dein 'S' bewegt hast? Hast du es erfolgreich nach (4, 4) verschoben? 
	  } 
	   } 
	   } 
	  \\ 
 

    \theutterance \stepcounter{utterance}  

    & & & \multicolumn{2}{p{0.3\linewidth}}{\cellcolor[rgb]{0.95,0.95,0.95}{%
	\makecell[{{p{\linewidth}}}]{% 
	  \tt {\tiny [GM$|$GM]}  
	 SPRICH: Ich habe mein 'S' nach (4, 6) verschoben. Wie sieht dein Raster aus, nachdem du dein 'S' bewegt hast? Hast du es erfolgreich nach (4, 4) verschoben? 
	  } 
	   } 
	   } 
	 & & \\ 
 

    \theutterance \stepcounter{utterance}  

    & & \multicolumn{4}{p{0.6\linewidth}}{\cellcolor[rgb]{0.9,0.9,0.9}{%
	\makecell[{{p{\linewidth}}}]{% 
	  \tt {\tiny [A$\langle$GM]}  
	 'S' erfolgreich nach (4,3) verschoben. Dein Raster sieht jetzt so aus:\\ \tt ```\\ \tt  1234567\\ \tt ╔═╤═══╤═╗\\ \tt ║◌│◌◌◌│◌║ 1\\ \tt ╟─┼──┬┘◌║ 2\\ \tt ║◌│◌S│◌◌║ 3\\ \tt ╟─┴──┘DT║ 4\\ \tt ║◌O◌◌◌◌◌║ 5\\ \tt ║◌◌◌◌◌◌◌║ 6\\ \tt ║◌◌APM◌◌║ 7\\ \tt ╚═══════╝\\ \tt ```\\ \tt Ihr seid gerade bei Runde 10 von maximal 28.\\ \tt Ihr habt gemeinsam 14 von maximal 16 Strafpunkten angesammelt.\\ \tt Dein Mitspieler hat dir diese Nachricht geschickt:\\ \tt \\ \tt \\ \tt "Ich habe mein 'S' nach (4, 6) verschoben. Wie sieht dein Raster aus, nachdem du dein 'S' bewegt hast? Hast du es erfolgreich nach (4, 4) verschoben?"\\ \tt \\ \tt \\ \tt Wie lautet dein nächster Befehl? 
	  } 
	   } 
	   } 
	 & \\ 
 

    \theutterance \stepcounter{utterance}  

    & \multicolumn{4}{p{0.6\linewidth}}{\cellcolor[rgb]{0.8,1,0.9}{%
	\makecell[{{p{\linewidth}}}]{% 
	  \tt {\tiny [A$\rangle$GM]}  
	 SPRICH: Mein 'S' ist jetzt bei (4, 3), nicht bei (4, 4), da (4, 4) auf meinem Brett nicht leer ist. Ich denke, wir sollten unsere Strategie anpassen, um sicherzustellen, dass wir unsere Ziele erreichen können. Lass uns unsere nächste Bewegung sorgfältig planen, um Strafpunkte zu vermeiden. Sollten wir versuchen, ein anderes Objekt zu bewegen, das leichter zu erreichen ist? 
	  } 
	   } 
	   } 
	 & & \\ 
 

    \theutterance \stepcounter{utterance}  

    & & & \multicolumn{2}{p{0.3\linewidth}}{\cellcolor[rgb]{0.95,0.95,0.95}{%
	\makecell[{{p{\linewidth}}}]{% 
	  \tt {\tiny [GM$|$GM]}  
	 SPRICH: Mein 'S' ist jetzt bei (4, 3), nicht bei (4, 4), da (4, 4) auf meinem Brett nicht leer ist. Ich denke, wir sollten unsere Strategie anpassen, um sicherzustellen, dass wir unsere Ziele erreichen können. Lass uns unsere nächste Bewegung sorgfältig planen, um Strafpunkte zu vermeiden. Sollten wir versuchen, ein anderes Objekt zu bewegen, das leichter zu erreichen ist? 
	  } 
	   } 
	   } 
	 & & \\ 
 

    \theutterance \stepcounter{utterance}  

    & & \multicolumn{4}{p{0.6\linewidth}}{\cellcolor[rgb]{0.9,0.9,0.9}{%
	\makecell[{{p{\linewidth}}}]{% 
	  \tt {\tiny [GM$\rangle$B]}  
	 Deine Nachricht wurde an deinen Mitspieler übermittelt.\\ \tt \\ \tt \\ \tt Ihr seid gerade bei Runde 11 von maximal 28.\\ \tt Ihr habt gemeinsam 14 von maximal 16 Strafpunkten angesammelt.\\ \tt Dein Mitspieler hat dir diese Nachricht geschickt:\\ \tt \\ \tt \\ \tt "Mein 'S' ist jetzt bei (4, 3), nicht bei (4, 4), da (4, 4) auf meinem Brett nicht leer ist. Ich denke, wir sollten unsere Strategie anpassen, um sicherzustellen, dass wir unsere Ziele erreichen können. Lass uns unsere nächste Bewegung sorgfältig planen, um Strafpunkte zu vermeiden. Sollten wir versuchen, ein anderes Objekt zu bewegen, das leichter zu erreichen ist?"\\ \tt \\ \tt \\ \tt Wie lautet dein nächster Befehl? 
	  } 
	   } 
	   } 
	 & \\ 
 

    \theutterance \stepcounter{utterance}  

    & & & \multicolumn{4}{p{0.6\linewidth}}{\cellcolor[rgb]{1,0.85,0.72}{%
	\makecell[{{p{\linewidth}}}]{% 
	  \tt {\tiny [GM$\langle$B]}  
	 SPRICH: Ich denke, wir sollten unsere Strategie anpassen und ein anderes Objekt auswählen, das leichter zu bewegen ist. Auf meinem Brett ist 'D' in der 1. Spalte und 5. Zeile, also relativ weit entfernt von (4, 4). Ich denke, wir sollten 'D' als nächstes Ziel auswählen und versuchen, es auf beiden Brettern auf die gleiche Position zu bewegen. Was meinst du? Sollten wir 'D' als nächstes Ziel auswählen? 
	  } 
	   } 
	   } 
	  \\ 
 

    \theutterance \stepcounter{utterance}  

    & & & \multicolumn{2}{p{0.3\linewidth}}{\cellcolor[rgb]{0.95,0.95,0.95}{%
	\makecell[{{p{\linewidth}}}]{% 
	  \tt {\tiny [GM$|$GM]}  
	 SPRICH: Ich denke, wir sollten unsere Strategie anpassen und ein anderes Objekt auswählen, das leichter zu bewegen ist. Auf meinem Brett ist 'D' in der 1. Spalte und 5. Zeile, also relativ weit entfernt von (4, 4). Ich denke, wir sollten 'D' als nächstes Ziel auswählen und versuchen, es auf beiden Brettern auf die gleiche Position zu bewegen. Was meinst du? Sollten wir 'D' als nächstes Ziel auswählen? 
	  } 
	   } 
	   } 
	 & & \\ 
 

    \theutterance \stepcounter{utterance}  

    & & \multicolumn{4}{p{0.6\linewidth}}{\cellcolor[rgb]{0.9,0.9,0.9}{%
	\makecell[{{p{\linewidth}}}]{% 
	  \tt {\tiny [A$\langle$GM]}  
	 Deine Nachricht wurde an deinen Mitspieler übermittelt.\\ \tt \\ \tt \\ \tt Ihr seid gerade bei Runde 11 von maximal 28.\\ \tt Ihr habt gemeinsam 14 von maximal 16 Strafpunkten angesammelt.\\ \tt Dein Mitspieler hat dir diese Nachricht geschickt:\\ \tt \\ \tt \\ \tt "Ich denke, wir sollten unsere Strategie anpassen und ein anderes Objekt auswählen, das leichter zu bewegen ist. Auf meinem Brett ist 'D' in der 1. Spalte und 5. Zeile, also relativ weit entfernt von (4, 4). Ich denke, wir sollten 'D' als nächstes Ziel auswählen und versuchen, es auf beiden Brettern auf die gleiche Position zu bewegen. Was meinst du? Sollten wir 'D' als nächstes Ziel auswählen?"\\ \tt \\ \tt \\ \tt Wie lautet dein nächster Befehl? 
	  } 
	   } 
	   } 
	 & \\ 
 

    \theutterance \stepcounter{utterance}  

    & \multicolumn{4}{p{0.6\linewidth}}{\cellcolor[rgb]{0.8,1,0.9}{%
	\makecell[{{p{\linewidth}}}]{% 
	  \tt {\tiny [A$\rangle$GM]}  
	 SPRICH: Ich denke, dass 'D' ein gutes Ziel ist, aber wir sollten sicherstellen, dass wir es auf beiden Brettern an die gleiche Position bewegen können. Auf meinem Brett ist 'D' in der 4. Zeile und 7. Spalte, also auch relativ weit entfernt von (4, 4). Ich denke, wir sollten versuchen, 'D' auf beiden Brettern in die Nähe von (4, 4) zu bewegen, aber wir müssen vorsichtig sein, um keine weiteren Strafpunkte zu sammeln. Lass uns langsam und sorgfältig vorgehen. Sollten wir 'D' nach (4, 5) oder (5, 4) bewegen, um es näher an (4, 4) zu bringen? 
	  } 
	   } 
	   } 
	 & & \\ 
 

    \theutterance \stepcounter{utterance}  

    & & & \multicolumn{2}{p{0.3\linewidth}}{\cellcolor[rgb]{0.95,0.95,0.95}{%
	\makecell[{{p{\linewidth}}}]{% 
	  \tt {\tiny [GM$|$GM]}  
	 SPRICH: Ich denke, dass 'D' ein gutes Ziel ist, aber wir sollten sicherstellen, dass wir es auf beiden Brettern an die gleiche Position bewegen können. Auf meinem Brett ist 'D' in der 4. Zeile und 7. Spalte, also auch relativ weit entfernt von (4, 4). Ich denke, wir sollten versuchen, 'D' auf beiden Brettern in die Nähe von (4, 4) zu bewegen, aber wir müssen vorsichtig sein, um keine weiteren Strafpunkte zu sammeln. Lass uns langsam und sorgfältig vorgehen. Sollten wir 'D' nach (4, 5) oder (5, 4) bewegen, um es näher an (4, 4) zu bringen? 
	  } 
	   } 
	   } 
	 & & \\ 
 

    \theutterance \stepcounter{utterance}  

    & & \multicolumn{4}{p{0.6\linewidth}}{\cellcolor[rgb]{0.9,0.9,0.9}{%
	\makecell[{{p{\linewidth}}}]{% 
	  \tt {\tiny [GM$\rangle$B]}  
	 Deine Nachricht wurde an deinen Mitspieler übermittelt.\\ \tt \\ \tt \\ \tt Ihr seid gerade bei Runde 12 von maximal 28.\\ \tt Ihr habt gemeinsam 14 von maximal 16 Strafpunkten angesammelt.\\ \tt Dein Mitspieler hat dir diese Nachricht geschickt:\\ \tt \\ \tt \\ \tt "Ich denke, dass 'D' ein gutes Ziel ist, aber wir sollten sicherstellen, dass wir es auf beiden Brettern an die gleiche Position bewegen können. Auf meinem Brett ist 'D' in der 4. Zeile und 7. Spalte, also auch relativ weit entfernt von (4, 4). Ich denke, wir sollten versuchen, 'D' auf beiden Brettern in die Nähe von (4, 4) zu bewegen, aber wir müssen vorsichtig sein, um keine weiteren Strafpunkte zu sammeln. Lass uns langsam und sorgfältig vorgehen. Sollten wir 'D' nach (4, 5) oder (5, 4) bewegen, um es näher an (4, 4) zu bringen?"\\ \tt \\ \tt \\ \tt Wie lautet dein nächster Befehl? 
	  } 
	   } 
	   } 
	 & \\ 
 

    \theutterance \stepcounter{utterance}  

    & & & \multicolumn{4}{p{0.6\linewidth}}{\cellcolor[rgb]{1,0.85,0.72}{%
	\makecell[{{p{\linewidth}}}]{% 
	  \tt {\tiny [GM$\langle$B]}  
	 SPRICH: Ich denke, wir sollten 'D' nach (5, 4) bewegen, um es näher an (4, 4) zu bringen. Auf meinem Brett ist die 5. Zeile und 4. Spalte jedoch nicht leer, da sie 'A' enthält. Ich denke, wir sollten 'A' vorher bewegen, um Platz für 'D' zu schaffen. Was meinst du? Sollten wir 'A' bewegen, bevor wir 'D' nach (5, 4) bewegen? 
	  } 
	   } 
	   } 
	  \\ 
 

    \theutterance \stepcounter{utterance}  

    & & & \multicolumn{2}{p{0.3\linewidth}}{\cellcolor[rgb]{0.95,0.95,0.95}{%
	\makecell[{{p{\linewidth}}}]{% 
	  \tt {\tiny [GM$|$GM]}  
	 SPRICH: Ich denke, wir sollten 'D' nach (5, 4) bewegen, um es näher an (4, 4) zu bringen. Auf meinem Brett ist die 5. Zeile und 4. Spalte jedoch nicht leer, da sie 'A' enthält. Ich denke, wir sollten 'A' vorher bewegen, um Platz für 'D' zu schaffen. Was meinst du? Sollten wir 'A' bewegen, bevor wir 'D' nach (5, 4) bewegen? 
	  } 
	   } 
	   } 
	 & & \\ 
 

    \theutterance \stepcounter{utterance}  

    & & \multicolumn{4}{p{0.6\linewidth}}{\cellcolor[rgb]{0.9,0.9,0.9}{%
	\makecell[{{p{\linewidth}}}]{% 
	  \tt {\tiny [A$\langle$GM]}  
	 Deine Nachricht wurde an deinen Mitspieler übermittelt.\\ \tt \\ \tt \\ \tt Ihr seid gerade bei Runde 12 von maximal 28.\\ \tt Ihr habt gemeinsam 14 von maximal 16 Strafpunkten angesammelt.\\ \tt Dein Mitspieler hat dir diese Nachricht geschickt:\\ \tt \\ \tt \\ \tt "Ich denke, wir sollten 'D' nach (5, 4) bewegen, um es näher an (4, 4) zu bringen. Auf meinem Brett ist die 5. Zeile und 4. Spalte jedoch nicht leer, da sie 'A' enthält. Ich denke, wir sollten 'A' vorher bewegen, um Platz für 'D' zu schaffen. Was meinst du? Sollten wir 'A' bewegen, bevor wir 'D' nach (5, 4) bewegen?"\\ \tt \\ \tt \\ \tt Wie lautet dein nächster Befehl? 
	  } 
	   } 
	   } 
	 & \\ 
 

    \theutterance \stepcounter{utterance}  

    & \multicolumn{4}{p{0.6\linewidth}}{\cellcolor[rgb]{0.8,1,0.9}{%
	\makecell[{{p{\linewidth}}}]{% 
	  \tt {\tiny [A$\rangle$GM]}  
	 SPRICH: Das klingt nach einem guten Plan! Ich denke, wir sollten 'A' bewegen, um Platz für 'D' zu schaffen. Auf meinem Brett ist 'A' in der 7. Zeile und 3. Spalte, also könnte ich es möglicherweise nach (6, 3) oder (7, 2) bewegen, um Platz zu schaffen. Lass uns 'A' bewegen und dann 'D' nach (5, 4) bewegen. Wie soll ich 'A' auf meinem Brett bewegen, um sicherzustellen, dass wir genug Platz für 'D' haben? 
	  } 
	   } 
	   } 
	 & & \\ 
 

    \theutterance \stepcounter{utterance}  

    & & & \multicolumn{2}{p{0.3\linewidth}}{\cellcolor[rgb]{0.95,0.95,0.95}{%
	\makecell[{{p{\linewidth}}}]{% 
	  \tt {\tiny [GM$|$GM]}  
	 SPRICH: Das klingt nach einem guten Plan! Ich denke, wir sollten 'A' bewegen, um Platz für 'D' zu schaffen. Auf meinem Brett ist 'A' in der 7. Zeile und 3. Spalte, also könnte ich es möglicherweise nach (6, 3) oder (7, 2) bewegen, um Platz zu schaffen. Lass uns 'A' bewegen und dann 'D' nach (5, 4) bewegen. Wie soll ich 'A' auf meinem Brett bewegen, um sicherzustellen, dass wir genug Platz für 'D' haben? 
	  } 
	   } 
	   } 
	 & & \\ 
 

    \theutterance \stepcounter{utterance}  

    & & \multicolumn{4}{p{0.6\linewidth}}{\cellcolor[rgb]{0.9,0.9,0.9}{%
	\makecell[{{p{\linewidth}}}]{% 
	  \tt {\tiny [GM$\rangle$B]}  
	 Deine Nachricht wurde an deinen Mitspieler übermittelt.\\ \tt \\ \tt \\ \tt Ihr seid gerade bei Runde 13 von maximal 28.\\ \tt Ihr habt gemeinsam 14 von maximal 16 Strafpunkten angesammelt.\\ \tt Dein Mitspieler hat dir diese Nachricht geschickt:\\ \tt \\ \tt \\ \tt "Das klingt nach einem guten Plan! Ich denke, wir sollten 'A' bewegen, um Platz für 'D' zu schaffen. Auf meinem Brett ist 'A' in der 7. Zeile und 3. Spalte, also könnte ich es möglicherweise nach (6, 3) oder (7, 2) bewegen, um Platz zu schaffen. Lass uns 'A' bewegen und dann 'D' nach (5, 4) bewegen. Wie soll ich 'A' auf meinem Brett bewegen, um sicherzustellen, dass wir genug Platz für 'D' haben?"\\ \tt \\ \tt \\ \tt Wie lautet dein nächster Befehl? 
	  } 
	   } 
	   } 
	 & \\ 
 

    \theutterance \stepcounter{utterance}  

    & & & \multicolumn{4}{p{0.6\linewidth}}{\cellcolor[rgb]{1,0.85,0.72}{%
	\makecell[{{p{\linewidth}}}]{% 
	  \tt {\tiny [GM$\langle$B]}  
	 SPRICH: Ich denke, du solltest 'A' auf deinem Brett nach (7, 2) bewegen, um Platz für 'D' zu schaffen. Auf meinem Brett ist 'A' in der 5. Zeile und 7. Spalte, also muss ich es auch bewegen, um Platz für 'D' zu schaffen. Ich denke, ich werde 'A' nach (7, 7) bewegen, um Platz für 'D' zu schaffen. Dann können wir 'D' nach (5, 4) bewegen. Lass uns unsere 'A'-Objekte bewegen und dann 'D' nach (5, 4) bewegen. 
	  } 
	   } 
	   } 
	  \\ 
 

    \theutterance \stepcounter{utterance}  

    & & & \multicolumn{2}{p{0.3\linewidth}}{\cellcolor[rgb]{0.95,0.95,0.95}{%
	\makecell[{{p{\linewidth}}}]{% 
	  \tt {\tiny [GM$|$GM]}  
	 SPRICH: Ich denke, du solltest 'A' auf deinem Brett nach (7, 2) bewegen, um Platz für 'D' zu schaffen. Auf meinem Brett ist 'A' in der 5. Zeile und 7. Spalte, also muss ich es auch bewegen, um Platz für 'D' zu schaffen. Ich denke, ich werde 'A' nach (7, 7) bewegen, um Platz für 'D' zu schaffen. Dann können wir 'D' nach (5, 4) bewegen. Lass uns unsere 'A'-Objekte bewegen und dann 'D' nach (5, 4) bewegen. 
	  } 
	   } 
	   } 
	 & & \\ 
 

    \theutterance \stepcounter{utterance}  

    & & \multicolumn{4}{p{0.6\linewidth}}{\cellcolor[rgb]{0.9,0.9,0.9}{%
	\makecell[{{p{\linewidth}}}]{% 
	  \tt {\tiny [A$\langle$GM]}  
	 Deine Nachricht wurde an deinen Mitspieler übermittelt.\\ \tt \\ \tt \\ \tt Ihr seid gerade bei Runde 13 von maximal 28.\\ \tt Ihr habt gemeinsam 14 von maximal 16 Strafpunkten angesammelt.\\ \tt Dein Mitspieler hat dir diese Nachricht geschickt:\\ \tt \\ \tt \\ \tt "Ich denke, du solltest 'A' auf deinem Brett nach (7, 2) bewegen, um Platz für 'D' zu schaffen. Auf meinem Brett ist 'A' in der 5. Zeile und 7. Spalte, also muss ich es auch bewegen, um Platz für 'D' zu schaffen. Ich denke, ich werde 'A' nach (7, 7) bewegen, um Platz für 'D' zu schaffen. Dann können wir 'D' nach (5, 4) bewegen. Lass uns unsere 'A'-Objekte bewegen und dann 'D' nach (5, 4) bewegen."\\ \tt \\ \tt \\ \tt Wie lautet dein nächster Befehl? 
	  } 
	   } 
	   } 
	 & \\ 
 

    \theutterance \stepcounter{utterance}  

    & \multicolumn{4}{p{0.6\linewidth}}{\cellcolor[rgb]{0.8,1,0.9}{%
	\makecell[{{p{\linewidth}}}]{% 
	  \tt {\tiny [A$\rangle$GM]}  
	 VERSCHIEBE: A, (7, 2) 
	  } 
	   } 
	   } 
	 & & \\ 
 

    \theutterance \stepcounter{utterance}  

    & & & \multicolumn{2}{p{0.3\linewidth}}{\cellcolor[rgb]{0.95,0.95,0.95}{%
	\makecell[{{p{\linewidth}}}]{% 
	  \tt {\tiny [GM$|$GM]}  
	 VERSCHIEBE: A, (7, 2) 
	  } 
	   } 
	   } 
	 & & \\ 
 

    \theutterance \stepcounter{utterance}  

    & & & \multicolumn{2}{p{0.3\linewidth}}{\cellcolor[rgb]{0.95,0.95,0.95}{%
	\makecell[{{p{\linewidth}}}]{% 
	  \tt {\tiny [GM$|$GM]}  
	 'A' erfolgreich nach (7,2) verschoben. Dein Raster sieht jetzt so aus:\\ \tt ```\\ \tt  1234567\\ \tt ╔═╤═══╤═╗\\ \tt ║◌│◌◌◌│◌║ 1\\ \tt ╟─┼──┬┘A║ 2\\ \tt ║◌│◌S│◌◌║ 3\\ \tt ╟─┴──┘DT║ 4\\ \tt ║◌O◌◌◌◌◌║ 5\\ \tt ║◌◌◌◌◌◌◌║ 6\\ \tt ║◌◌◌PM◌◌║ 7\\ \tt ╚═══════╝\\ \tt ```\\ \tt  
	  } 
	   } 
	   } 
	 & & \\ 
 

    \theutterance \stepcounter{utterance}  

    & & \multicolumn{4}{p{0.6\linewidth}}{\cellcolor[rgb]{0.9,0.9,0.9}{%
	\makecell[{{p{\linewidth}}}]{% 
	  \tt {\tiny [GM$\rangle$B]}  
	 Deine Nachricht wurde an deinen Mitspieler übermittelt.\\ \tt \\ \tt \\ \tt Ihr seid gerade bei Runde 14 von maximal 28.\\ \tt Ihr habt gemeinsam 14 von maximal 16 Strafpunkten angesammelt.\\ \tt Dein Mitspieler hat ein Objekt auf seinem Raster bewegt.\\ \tt \\ \tt \\ \tt Wie lautet dein nächster Befehl? 
	  } 
	   } 
	   } 
	 & \\ 
 

    \theutterance \stepcounter{utterance}  

    & & & \multicolumn{4}{p{0.6\linewidth}}{\cellcolor[rgb]{1,0.85,0.72}{%
	\makecell[{{p{\linewidth}}}]{% 
	  \tt {\tiny [GM$\langle$B]}  
	 VERSCHIEBE: A, (7, 7) 
	  } 
	   } 
	   } 
	  \\ 
 

    \theutterance \stepcounter{utterance}  

    & & & \multicolumn{2}{p{0.3\linewidth}}{\cellcolor[rgb]{0.95,0.95,0.95}{%
	\makecell[{{p{\linewidth}}}]{% 
	  \tt {\tiny [GM$|$GM]}  
	 VERSCHIEBE: A, (7, 7) 
	  } 
	   } 
	   } 
	 & & \\ 
 

    \theutterance \stepcounter{utterance}  

    & & & \multicolumn{2}{p{0.3\linewidth}}{\cellcolor[rgb]{0.95,0.95,0.95}{%
	\makecell[{{p{\linewidth}}}]{% 
	  \tt {\tiny [GM$|$GM]}  
	 'A' erfolgreich nach (7,7) verschoben. Dein Raster sieht jetzt so aus:\\ \tt ```\\ \tt  1234567\\ \tt ╔═╤═══╤═╗\\ \tt ║◌│◌◌◌│◌║ 1\\ \tt ╟─┼──┬┘◌║ 2\\ \tt ║◌│◌M│◌T║ 3\\ \tt ╟─┴──┘P◌║ 4\\ \tt ║D◌◌◌◌◌◌║ 5\\ \tt ║◌◌◌SO◌◌║ 6\\ \tt ║◌◌◌◌◌◌A║ 7\\ \tt ╚═══════╝\\ \tt ```\\ \tt  
	  } 
	   } 
	   } 
	 & & \\ 
 

    \theutterance \stepcounter{utterance}  

    & & \multicolumn{4}{p{0.6\linewidth}}{\cellcolor[rgb]{0.9,0.9,0.9}{%
	\makecell[{{p{\linewidth}}}]{% 
	  \tt {\tiny [A$\langle$GM]}  
	 'A' erfolgreich nach (7,2) verschoben. Dein Raster sieht jetzt so aus:\\ \tt ```\\ \tt  1234567\\ \tt ╔═╤═══╤═╗\\ \tt ║◌│◌◌◌│◌║ 1\\ \tt ╟─┼──┬┘A║ 2\\ \tt ║◌│◌S│◌◌║ 3\\ \tt ╟─┴──┘DT║ 4\\ \tt ║◌O◌◌◌◌◌║ 5\\ \tt ║◌◌◌◌◌◌◌║ 6\\ \tt ║◌◌◌PM◌◌║ 7\\ \tt ╚═══════╝\\ \tt ```\\ \tt Ihr seid gerade bei Runde 14 von maximal 28.\\ \tt Ihr habt gemeinsam 14 von maximal 16 Strafpunkten angesammelt.\\ \tt Dein Mitspieler hat ein Objekt auf seinem Raster bewegt.\\ \tt \\ \tt \\ \tt Wie lautet dein nächster Befehl? 
	  } 
	   } 
	   } 
	 & \\ 
 

    \theutterance \stepcounter{utterance}  

    & \multicolumn{4}{p{0.6\linewidth}}{\cellcolor[rgb]{0.8,1,0.9}{%
	\makecell[{{p{\linewidth}}}]{% 
	  \tt {\tiny [A$\rangle$GM]}  
	 VERSCHIEBE: D, (5, 4) 
	  } 
	   } 
	   } 
	 & & \\ 
 

    \theutterance \stepcounter{utterance}  

    & & & \multicolumn{2}{p{0.3\linewidth}}{\cellcolor[rgb]{0.95,0.95,0.95}{%
	\makecell[{{p{\linewidth}}}]{% 
	  \tt {\tiny [GM$|$GM]}  
	 VERSCHIEBE: D, (5, 4) 
	  } 
	   } 
	   } 
	 & & \\ 
 

    \theutterance \stepcounter{utterance}  

    & & & \multicolumn{2}{p{0.3\linewidth}}{\cellcolor[rgb]{0.95,0.95,0.95}{%
	\makecell[{{p{\linewidth}}}]{% 
	  \tt {\tiny [GM$|$GM]}  
	 Strafpunkt: (5,4) ist nicht leer, sondern enthält '┘'.\\ \tt \\ \tt Ihr habt gemeinsam 15 von maximal 16 Strafpunkten angesammelt.\\ \tt Bitte versuche es noch mal! 
	  } 
	   } 
	   } 
	 & & \\ 
 

    \theutterance \stepcounter{utterance}  

    & & \multicolumn{4}{p{0.6\linewidth}}{\cellcolor[rgb]{0.9,0.9,0.9}{%
	\makecell[{{p{\linewidth}}}]{% 
	  \tt {\tiny [A$\langle$GM]}  
	 Strafpunkt: (5,4) ist nicht leer, sondern enthält '┘'.\\ \tt \\ \tt Ihr habt gemeinsam 15 von maximal 16 Strafpunkten angesammelt.\\ \tt Bitte versuche es noch mal! 
	  } 
	   } 
	   } 
	 & \\ 
 

    \theutterance \stepcounter{utterance}  

    & \multicolumn{4}{p{0.6\linewidth}}{\cellcolor[rgb]{0.8,1,0.9}{%
	\makecell[{{p{\linewidth}}}]{% 
	  \tt {\tiny [A$\rangle$GM]}  
	 SPRICH: Ich denke, wir sollten unsere Strategie anpassen, da (5, 4) auf meinem Brett nicht leer ist. Lass uns alternative Positionen für 'D' finden, die frei sind und leichter zu erreichen sind. Welche Positionen würdest du auf deinem Brett für 'D' in Betracht ziehen? 
	  } 
	   } 
	   } 
	 & & \\ 
 

    \theutterance \stepcounter{utterance}  

    & & & \multicolumn{2}{p{0.3\linewidth}}{\cellcolor[rgb]{0.95,0.95,0.95}{%
	\makecell[{{p{\linewidth}}}]{% 
	  \tt {\tiny [GM$|$GM]}  
	 SPRICH: Ich denke, wir sollten unsere Strategie anpassen, da (5, 4) auf meinem Brett nicht leer ist. Lass uns alternative Positionen für 'D' finden, die frei sind und leichter zu erreichen sind. Welche Positionen würdest du auf deinem Brett für 'D' in Betracht ziehen? 
	  } 
	   } 
	   } 
	 & & \\ 
 

    \theutterance \stepcounter{utterance}  

    & & \multicolumn{4}{p{0.6\linewidth}}{\cellcolor[rgb]{0.9,0.9,0.9}{%
	\makecell[{{p{\linewidth}}}]{% 
	  \tt {\tiny [GM$\rangle$B]}  
	 'A' erfolgreich nach (7,7) verschoben. Dein Raster sieht jetzt so aus:\\ \tt ```\\ \tt  1234567\\ \tt ╔═╤═══╤═╗\\ \tt ║◌│◌◌◌│◌║ 1\\ \tt ╟─┼──┬┘◌║ 2\\ \tt ║◌│◌M│◌T║ 3\\ \tt ╟─┴──┘P◌║ 4\\ \tt ║D◌◌◌◌◌◌║ 5\\ \tt ║◌◌◌SO◌◌║ 6\\ \tt ║◌◌◌◌◌◌A║ 7\\ \tt ╚═══════╝\\ \tt ```\\ \tt Ihr seid gerade bei Runde 15 von maximal 28.\\ \tt Ihr habt gemeinsam 15 von maximal 16 Strafpunkten angesammelt.\\ \tt Dein Mitspieler hat dir diese Nachricht geschickt:\\ \tt \\ \tt \\ \tt "Ich denke, wir sollten unsere Strategie anpassen, da (5, 4) auf meinem Brett nicht leer ist. Lass uns alternative Positionen für 'D' finden, die frei sind und leichter zu erreichen sind. Welche Positionen würdest du auf deinem Brett für 'D' in Betracht ziehen?"\\ \tt \\ \tt \\ \tt Wie lautet dein nächster Befehl? 
	  } 
	   } 
	   } 
	 & \\ 
 

    \theutterance \stepcounter{utterance}  

    & & & \multicolumn{4}{p{0.6\linewidth}}{\cellcolor[rgb]{1,0.85,0.72}{%
	\makecell[{{p{\linewidth}}}]{% 
	  \tt {\tiny [GM$\langle$B]}  
	 SPRICH: Ich denke, wir sollten 'D' nach (5, 5) bewegen, da diese Position auf meinem Brett leer ist und 'D' damit näher an das Ziel (4, 4) bringen. Ich denke, dies könnte eine gute Alternative sein, da (5, 4) auf deinem Brett nicht leer ist. Was meinst du? Sollten wir 'D' nach (5, 5) bewegen? 
	  } 
	   } 
	   } 
	  \\ 
 

    \theutterance \stepcounter{utterance}  

    & & & \multicolumn{2}{p{0.3\linewidth}}{\cellcolor[rgb]{0.95,0.95,0.95}{%
	\makecell[{{p{\linewidth}}}]{% 
	  \tt {\tiny [GM$|$GM]}  
	 SPRICH: Ich denke, wir sollten 'D' nach (5, 5) bewegen, da diese Position auf meinem Brett leer ist und 'D' damit näher an das Ziel (4, 4) bringen. Ich denke, dies könnte eine gute Alternative sein, da (5, 4) auf deinem Brett nicht leer ist. Was meinst du? Sollten wir 'D' nach (5, 5) bewegen? 
	  } 
	   } 
	   } 
	 & & \\ 
 

    \theutterance \stepcounter{utterance}  

    & & \multicolumn{4}{p{0.6\linewidth}}{\cellcolor[rgb]{0.9,0.9,0.9}{%
	\makecell[{{p{\linewidth}}}]{% 
	  \tt {\tiny [A$\langle$GM]}  
	 Deine Nachricht wurde an deinen Mitspieler übermittelt.\\ \tt \\ \tt \\ \tt Ihr seid gerade bei Runde 15 von maximal 28.\\ \tt Ihr habt gemeinsam 15 von maximal 16 Strafpunkten angesammelt.\\ \tt Dein Mitspieler hat dir diese Nachricht geschickt:\\ \tt \\ \tt \\ \tt "Ich denke, wir sollten 'D' nach (5, 5) bewegen, da diese Position auf meinem Brett leer ist und 'D' damit näher an das Ziel (4, 4) bringen. Ich denke, dies könnte eine gute Alternative sein, da (5, 4) auf deinem Brett nicht leer ist. Was meinst du? Sollten wir 'D' nach (5, 5) bewegen?"\\ \tt \\ \tt \\ \tt Wie lautet dein nächster Befehl? 
	  } 
	   } 
	   } 
	 & \\ 
 

    \theutterance \stepcounter{utterance}  

    & \multicolumn{4}{p{0.6\linewidth}}{\cellcolor[rgb]{0.8,1,0.9}{%
	\makecell[{{p{\linewidth}}}]{% 
	  \tt {\tiny [A$\rangle$GM]}  
	 VERSCHIEBE: D, (5, 5) 
	  } 
	   } 
	   } 
	 & & \\ 
 

    \theutterance \stepcounter{utterance}  

    & & & \multicolumn{2}{p{0.3\linewidth}}{\cellcolor[rgb]{0.95,0.95,0.95}{%
	\makecell[{{p{\linewidth}}}]{% 
	  \tt {\tiny [GM$|$GM]}  
	 VERSCHIEBE: D, (5, 5) 
	  } 
	   } 
	   } 
	 & & \\ 
 

    \theutterance \stepcounter{utterance}  

    & & & \multicolumn{2}{p{0.3\linewidth}}{\cellcolor[rgb]{0.95,0.95,0.95}{%
	\makecell[{{p{\linewidth}}}]{% 
	  \tt {\tiny [GM$|$GM]}  
	 'D' erfolgreich nach (5,5) verschoben. Dein Raster sieht jetzt so aus:\\ \tt ```\\ \tt  1234567\\ \tt ╔═╤═══╤═╗\\ \tt ║◌│◌◌◌│◌║ 1\\ \tt ╟─┼──┬┘A║ 2\\ \tt ║◌│◌S│◌◌║ 3\\ \tt ╟─┴──┘◌T║ 4\\ \tt ║◌O◌◌D◌◌║ 5\\ \tt ║◌◌◌◌◌◌◌║ 6\\ \tt ║◌◌◌PM◌◌║ 7\\ \tt ╚═══════╝\\ \tt ```\\ \tt  
	  } 
	   } 
	   } 
	 & & \\ 
 

    \theutterance \stepcounter{utterance}  

    & & \multicolumn{4}{p{0.6\linewidth}}{\cellcolor[rgb]{0.9,0.9,0.9}{%
	\makecell[{{p{\linewidth}}}]{% 
	  \tt {\tiny [GM$\rangle$B]}  
	 Deine Nachricht wurde an deinen Mitspieler übermittelt.\\ \tt \\ \tt \\ \tt Ihr seid gerade bei Runde 16 von maximal 28.\\ \tt Ihr habt gemeinsam 15 von maximal 16 Strafpunkten angesammelt.\\ \tt Dein Mitspieler hat ein Objekt auf seinem Raster bewegt.\\ \tt \\ \tt \\ \tt Wie lautet dein nächster Befehl? 
	  } 
	   } 
	   } 
	 & \\ 
 

    \theutterance \stepcounter{utterance}  

    & & & \multicolumn{4}{p{0.6\linewidth}}{\cellcolor[rgb]{1,0.85,0.72}{%
	\makecell[{{p{\linewidth}}}]{% 
	  \tt {\tiny [GM$\langle$B]}  
	 VERSCHIEBE: D, (5, 5) 
	  } 
	   } 
	   } 
	  \\ 
 

    \theutterance \stepcounter{utterance}  

    & & & \multicolumn{2}{p{0.3\linewidth}}{\cellcolor[rgb]{0.95,0.95,0.95}{%
	\makecell[{{p{\linewidth}}}]{% 
	  \tt {\tiny [GM$|$GM]}  
	 VERSCHIEBE: D, (5, 5) 
	  } 
	   } 
	   } 
	 & & \\ 
 

    \theutterance \stepcounter{utterance}  

    & & & \multicolumn{2}{p{0.3\linewidth}}{\cellcolor[rgb]{0.95,0.95,0.95}{%
	\makecell[{{p{\linewidth}}}]{% 
	  \tt {\tiny [GM$|$GM]}  
	 'D' erfolgreich nach (5,5) verschoben. Dein Raster sieht jetzt so aus:\\ \tt ```\\ \tt  1234567\\ \tt ╔═╤═══╤═╗\\ \tt ║◌│◌◌◌│◌║ 1\\ \tt ╟─┼──┬┘◌║ 2\\ \tt ║◌│◌M│◌T║ 3\\ \tt ╟─┴──┘P◌║ 4\\ \tt ║◌◌◌◌D◌◌║ 5\\ \tt ║◌◌◌SO◌◌║ 6\\ \tt ║◌◌◌◌◌◌A║ 7\\ \tt ╚═══════╝\\ \tt ```\\ \tt  
	  } 
	   } 
	   } 
	 & & \\ 
 

    \theutterance \stepcounter{utterance}  

    & & \multicolumn{4}{p{0.6\linewidth}}{\cellcolor[rgb]{0.9,0.9,0.9}{%
	\makecell[{{p{\linewidth}}}]{% 
	  \tt {\tiny [A$\langle$GM]}  
	 'D' erfolgreich nach (5,5) verschoben. Dein Raster sieht jetzt so aus:\\ \tt ```\\ \tt  1234567\\ \tt ╔═╤═══╤═╗\\ \tt ║◌│◌◌◌│◌║ 1\\ \tt ╟─┼──┬┘A║ 2\\ \tt ║◌│◌S│◌◌║ 3\\ \tt ╟─┴──┘◌T║ 4\\ \tt ║◌O◌◌D◌◌║ 5\\ \tt ║◌◌◌◌◌◌◌║ 6\\ \tt ║◌◌◌PM◌◌║ 7\\ \tt ╚═══════╝\\ \tt ```\\ \tt Ihr seid gerade bei Runde 16 von maximal 28.\\ \tt Ihr habt gemeinsam 15 von maximal 16 Strafpunkten angesammelt.\\ \tt Dein Mitspieler hat ein Objekt auf seinem Raster bewegt.\\ \tt \\ \tt \\ \tt Wie lautet dein nächster Befehl? 
	  } 
	   } 
	   } 
	 & \\ 
 

    \theutterance \stepcounter{utterance}  

    & \multicolumn{4}{p{0.6\linewidth}}{\cellcolor[rgb]{0.8,1,0.9}{%
	\makecell[{{p{\linewidth}}}]{% 
	  \tt {\tiny [A$\rangle$GM]}  
	 SPRICH: Ich denke, wir sollten ahora überprüfen, ob wir das Spiel gewinnen können, indem wir unsere Objekte auf beide Brettern an die gleichen Positionen bewegen. Lass uns unsere Bretter vergleichen und entscheiden, ob wir fertig sind oder weitere Bewegungen vornehmen sollten. Sollten wir nach (4, 4) schauen und prüfen, ob wir unsere Objekte dort platzieren können? 
	  } 
	   } 
	   } 
	 & & \\ 
 

    \theutterance \stepcounter{utterance}  

    & & & \multicolumn{2}{p{0.3\linewidth}}{\cellcolor[rgb]{0.95,0.95,0.95}{%
	\makecell[{{p{\linewidth}}}]{% 
	  \tt {\tiny [GM$|$GM]}  
	 SPRICH: Ich denke, wir sollten ahora überprüfen, ob wir das Spiel gewinnen können, indem wir unsere Objekte auf beide Brettern an die gleichen Positionen bewegen. Lass uns unsere Bretter vergleichen und entscheiden, ob wir fertig sind oder weitere Bewegungen vornehmen sollten. Sollten wir nach (4, 4) schauen und prüfen, ob wir unsere Objekte dort platzieren können? 
	  } 
	   } 
	   } 
	 & & \\ 
 

    \theutterance \stepcounter{utterance}  

    & & \multicolumn{4}{p{0.6\linewidth}}{\cellcolor[rgb]{0.9,0.9,0.9}{%
	\makecell[{{p{\linewidth}}}]{% 
	  \tt {\tiny [GM$\rangle$B]}  
	 'D' erfolgreich nach (5,5) verschoben. Dein Raster sieht jetzt so aus:\\ \tt ```\\ \tt  1234567\\ \tt ╔═╤═══╤═╗\\ \tt ║◌│◌◌◌│◌║ 1\\ \tt ╟─┼──┬┘◌║ 2\\ \tt ║◌│◌M│◌T║ 3\\ \tt ╟─┴──┘P◌║ 4\\ \tt ║◌◌◌◌D◌◌║ 5\\ \tt ║◌◌◌SO◌◌║ 6\\ \tt ║◌◌◌◌◌◌A║ 7\\ \tt ╚═══════╝\\ \tt ```\\ \tt Ihr seid gerade bei Runde 17 von maximal 28.\\ \tt Ihr habt gemeinsam 15 von maximal 16 Strafpunkten angesammelt.\\ \tt Dein Mitspieler hat dir diese Nachricht geschickt:\\ \tt \\ \tt \\ \tt "Ich denke, wir sollten ahora überprüfen, ob wir das Spiel gewinnen können, indem wir unsere Objekte auf beide Brettern an die gleichen Positionen bewegen. Lass uns unsere Bretter vergleichen und entscheiden, ob wir fertig sind oder weitere Bewegungen vornehmen sollten. Sollten wir nach (4, 4) schauen und prüfen, ob wir unsere Objekte dort platzieren können?"\\ \tt \\ \tt \\ \tt Wie lautet dein nächster Befehl? 
	  } 
	   } 
	   } 
	 & \\ 
 

    \theutterance \stepcounter{utterance}  

    & & & \multicolumn{4}{p{0.6\linewidth}}{\cellcolor[rgb]{1,0.85,0.72}{%
	\makecell[{{p{\linewidth}}}]{% 
	  \tt {\tiny [GM$\langle$B]}  
	 SPRICH: Ich denke, wir sollten unsere Bretter vergleichen und prüfen, ob wir unsere Objekte auf beide Brettern an die gleichen Positionen bewegen können. Ich denke, wir sollten auch prüfen, ob wir unsere Objekte auf (4, 4) platzieren können, aber ich sehe, dass auf meinem Brett (4, 4) nicht leer ist, da es ein '╟' enthält. Ich denke, wir sollten weitere Bewegungen vornehmen, um unsere Objekte an die gleichen Positionen zu bringen. Wie sieht dein Brett aus? Sind deine Objekte auf den gleichen Positionen wie meine? Sollten wir weitere Bewegungen vornehmen oder sind wir fertig? 
	  } 
	   } 
	   } 
	  \\ 
 

    \theutterance \stepcounter{utterance}  

    & & & \multicolumn{2}{p{0.3\linewidth}}{\cellcolor[rgb]{0.95,0.95,0.95}{%
	\makecell[{{p{\linewidth}}}]{% 
	  \tt {\tiny [GM$|$GM]}  
	 SPRICH: Ich denke, wir sollten unsere Bretter vergleichen und prüfen, ob wir unsere Objekte auf beide Brettern an die gleichen Positionen bewegen können. Ich denke, wir sollten auch prüfen, ob wir unsere Objekte auf (4, 4) platzieren können, aber ich sehe, dass auf meinem Brett (4, 4) nicht leer ist, da es ein '╟' enthält. Ich denke, wir sollten weitere Bewegungen vornehmen, um unsere Objekte an die gleichen Positionen zu bringen. Wie sieht dein Brett aus? Sind deine Objekte auf den gleichen Positionen wie meine? Sollten wir weitere Bewegungen vornehmen oder sind wir fertig? 
	  } 
	   } 
	   } 
	 & & \\ 
 

    \theutterance \stepcounter{utterance}  

    & & \multicolumn{4}{p{0.6\linewidth}}{\cellcolor[rgb]{0.9,0.9,0.9}{%
	\makecell[{{p{\linewidth}}}]{% 
	  \tt {\tiny [A$\langle$GM]}  
	 Deine Nachricht wurde an deinen Mitspieler übermittelt.\\ \tt \\ \tt \\ \tt Ihr seid gerade bei Runde 17 von maximal 28.\\ \tt Ihr habt gemeinsam 15 von maximal 16 Strafpunkten angesammelt.\\ \tt Dein Mitspieler hat dir diese Nachricht geschickt:\\ \tt \\ \tt \\ \tt "Ich denke, wir sollten unsere Bretter vergleichen und prüfen, ob wir unsere Objekte auf beide Brettern an die gleichen Positionen bewegen können. Ich denke, wir sollten auch prüfen, ob wir unsere Objekte auf (4, 4) platzieren können, aber ich sehe, dass auf meinem Brett (4, 4) nicht leer ist, da es ein '╟' enthält. Ich denke, wir sollten weitere Bewegungen vornehmen, um unsere Objekte an die gleichen Positionen zu bringen. Wie sieht dein Brett aus? Sind deine Objekte auf den gleichen Positionen wie meine? Sollten wir weitere Bewegungen vornehmen oder sind wir fertig?"\\ \tt \\ \tt \\ \tt Wie lautet dein nächster Befehl? 
	  } 
	   } 
	   } 
	 & \\ 
 

    \theutterance \stepcounter{utterance}  

    & \multicolumn{4}{p{0.6\linewidth}}{\cellcolor[rgb]{0.8,1,0.9}{%
	\makecell[{{p{\linewidth}}}]{% 
	  \tt {\tiny [A$\rangle$GM]}  
	 SPRICH: Ich denke, wir sollten weitere Bewegungen vornehmen, um unsere Objekte an die gleichen Positionen zu bringen. Auf meinem Brett sind die Objekte 'S' bei (4, 3), 'D' bei (5, 5) und die anderen Objekte sind verteilt. Ich denke, wir sollten unsere Objekte schrittweise an die gleichen Positionen bewegen, um sicherzustellen, dass wir das Spiel gewinnen können. Lass uns unsere nächste Bewegung sorgfältig planen und prüfen, ob wir unsere Objekte an die gleichen Positionen bringen können. Sollten wir ein bestimmtes Objekt als nächstes bewegen oder haben wir bereits ein gutes Ergebnis erreicht? 
	  } 
	   } 
	   } 
	 & & \\ 
 

    \theutterance \stepcounter{utterance}  

    & & & \multicolumn{2}{p{0.3\linewidth}}{\cellcolor[rgb]{0.95,0.95,0.95}{%
	\makecell[{{p{\linewidth}}}]{% 
	  \tt {\tiny [GM$|$GM]}  
	 SPRICH: Ich denke, wir sollten weitere Bewegungen vornehmen, um unsere Objekte an die gleichen Positionen zu bringen. Auf meinem Brett sind die Objekte 'S' bei (4, 3), 'D' bei (5, 5) und die anderen Objekte sind verteilt. Ich denke, wir sollten unsere Objekte schrittweise an die gleichen Positionen bewegen, um sicherzustellen, dass wir das Spiel gewinnen können. Lass uns unsere nächste Bewegung sorgfältig planen und prüfen, ob wir unsere Objekte an die gleichen Positionen bringen können. Sollten wir ein bestimmtes Objekt als nächstes bewegen oder haben wir bereits ein gutes Ergebnis erreicht? 
	  } 
	   } 
	   } 
	 & & \\ 
 

    \theutterance \stepcounter{utterance}  

    & & \multicolumn{4}{p{0.6\linewidth}}{\cellcolor[rgb]{0.9,0.9,0.9}{%
	\makecell[{{p{\linewidth}}}]{% 
	  \tt {\tiny [GM$\rangle$B]}  
	 Deine Nachricht wurde an deinen Mitspieler übermittelt.\\ \tt \\ \tt \\ \tt Ihr seid gerade bei Runde 18 von maximal 28.\\ \tt Ihr habt gemeinsam 15 von maximal 16 Strafpunkten angesammelt.\\ \tt Dein Mitspieler hat dir diese Nachricht geschickt:\\ \tt \\ \tt \\ \tt "Ich denke, wir sollten weitere Bewegungen vornehmen, um unsere Objekte an die gleichen Positionen zu bringen. Auf meinem Brett sind die Objekte 'S' bei (4, 3), 'D' bei (5, 5) und die anderen Objekte sind verteilt. Ich denke, wir sollten unsere Objekte schrittweise an die gleichen Positionen bewegen, um sicherzustellen, dass wir das Spiel gewinnen können. Lass uns unsere nächste Bewegung sorgfältig planen und prüfen, ob wir unsere Objekte an die gleichen Positionen bringen können. Sollten wir ein bestimmtes Objekt als nächstes bewegen oder haben wir bereits ein gutes Ergebnis erreicht?"\\ \tt \\ \tt \\ \tt Wie lautet dein nächster Befehl? 
	  } 
	   } 
	   } 
	 & \\ 
 

    \theutterance \stepcounter{utterance}  

    & & & \multicolumn{4}{p{0.6\linewidth}}{\cellcolor[rgb]{1,0.85,0.72}{%
	\makecell[{{p{\linewidth}}}]{% 
	  \tt {\tiny [GM$\langle$B]}  
	 SPRICH: Ich denke, wir sollten noss nächste Bewegung sorgfältig planen und prüfen, ob wir unsere Objekte an die gleichen Positionen bringen können. Auf meinem Brett sind die Objekte 'S' bei (4, 6), 'D' bei (5, 5) und die anderen Objekte sind verteilt. Ich denke, wir sollten 'S' auf meinem Brett nach (4, 3) bewegen, um es an die gleiche Position wie auf deinem Brett zu bringen. Was meinst du? Sollten wir 'S' als nächstes bewegen? 
	  } 
	   } 
	   } 
	  \\ 
 

    \theutterance \stepcounter{utterance}  

    & & & \multicolumn{2}{p{0.3\linewidth}}{\cellcolor[rgb]{0.95,0.95,0.95}{%
	\makecell[{{p{\linewidth}}}]{% 
	  \tt {\tiny [GM$|$GM]}  
	 SPRICH: Ich denke, wir sollten noss nächste Bewegung sorgfältig planen und prüfen, ob wir unsere Objekte an die gleichen Positionen bringen können. Auf meinem Brett sind die Objekte 'S' bei (4, 6), 'D' bei (5, 5) und die anderen Objekte sind verteilt. Ich denke, wir sollten 'S' auf meinem Brett nach (4, 3) bewegen, um es an die gleiche Position wie auf deinem Brett zu bringen. Was meinst du? Sollten wir 'S' als nächstes bewegen? 
	  } 
	   } 
	   } 
	 & & \\ 
 

    \theutterance \stepcounter{utterance}  

    & & \multicolumn{4}{p{0.6\linewidth}}{\cellcolor[rgb]{0.9,0.9,0.9}{%
	\makecell[{{p{\linewidth}}}]{% 
	  \tt {\tiny [A$\langle$GM]}  
	 Deine Nachricht wurde an deinen Mitspieler übermittelt.\\ \tt \\ \tt \\ \tt Ihr seid gerade bei Runde 18 von maximal 28.\\ \tt Ihr habt gemeinsam 15 von maximal 16 Strafpunkten angesammelt.\\ \tt Dein Mitspieler hat dir diese Nachricht geschickt:\\ \tt \\ \tt \\ \tt "Ich denke, wir sollten noss nächste Bewegung sorgfältig planen und prüfen, ob wir unsere Objekte an die gleichen Positionen bringen können. Auf meinem Brett sind die Objekte 'S' bei (4, 6), 'D' bei (5, 5) und die anderen Objekte sind verteilt. Ich denke, wir sollten 'S' auf meinem Brett nach (4, 3) bewegen, um es an die gleiche Position wie auf deinem Brett zu bringen. Was meinst du? Sollten wir 'S' als nächstes bewegen?"\\ \tt \\ \tt \\ \tt Wie lautet dein nächster Befehl? 
	  } 
	   } 
	   } 
	 & \\ 
 

    \theutterance \stepcounter{utterance}  

    & \multicolumn{4}{p{0.6\linewidth}}{\cellcolor[rgb]{0.8,1,0.9}{%
	\makecell[{{p{\linewidth}}}]{% 
	  \tt {\tiny [A$\rangle$GM]}  
	 SPRICH: Ich denke, das ist eine gute Idee! Wenn wir 'S' auf deinem Brett nach (4, 3) bewegen, wäre es auf beiden Brettern an der gleichen Position. Das wäre ein wichtiger Schritt, um unsere Objekte an die gleichen Positionen zu bringen. Ich denke, wir sollten dies als nächstes tun und dann prüfen, ob wir weitere Bewegungen vornehmen müssen, um das Spiel zu gewinnen. Lass uns 'S' auf deinem Brett nach (4, 3) bewegen und dann unsere nächsten Schritte planen. 
	  } 
	   } 
	   } 
	 & & \\ 
 

    \theutterance \stepcounter{utterance}  

    & & & \multicolumn{2}{p{0.3\linewidth}}{\cellcolor[rgb]{0.95,0.95,0.95}{%
	\makecell[{{p{\linewidth}}}]{% 
	  \tt {\tiny [GM$|$GM]}  
	 SPRICH: Ich denke, das ist eine gute Idee! Wenn wir 'S' auf deinem Brett nach (4, 3) bewegen, wäre es auf beiden Brettern an der gleichen Position. Das wäre ein wichtiger Schritt, um unsere Objekte an die gleichen Positionen zu bringen. Ich denke, wir sollten dies als nächstes tun und dann prüfen, ob wir weitere Bewegungen vornehmen müssen, um das Spiel zu gewinnen. Lass uns 'S' auf deinem Brett nach (4, 3) bewegen und dann unsere nächsten Schritte planen. 
	  } 
	   } 
	   } 
	 & & \\ 
 

    \theutterance \stepcounter{utterance}  

    & & \multicolumn{4}{p{0.6\linewidth}}{\cellcolor[rgb]{0.9,0.9,0.9}{%
	\makecell[{{p{\linewidth}}}]{% 
	  \tt {\tiny [GM$\rangle$B]}  
	 Deine Nachricht wurde an deinen Mitspieler übermittelt.\\ \tt \\ \tt \\ \tt Ihr seid gerade bei Runde 19 von maximal 28.\\ \tt Ihr habt gemeinsam 15 von maximal 16 Strafpunkten angesammelt.\\ \tt Dein Mitspieler hat dir diese Nachricht geschickt:\\ \tt \\ \tt \\ \tt "Ich denke, das ist eine gute Idee! Wenn wir 'S' auf deinem Brett nach (4, 3) bewegen, wäre es auf beiden Brettern an der gleichen Position. Das wäre ein wichtiger Schritt, um unsere Objekte an die gleichen Positionen zu bringen. Ich denke, wir sollten dies als nächstes tun und dann prüfen, ob wir weitere Bewegungen vornehmen müssen, um das Spiel zu gewinnen. Lass uns 'S' auf deinem Brett nach (4, 3) bewegen und dann unsere nächsten Schritte planen."\\ \tt \\ \tt \\ \tt Wie lautet dein nächster Befehl? 
	  } 
	   } 
	   } 
	 & \\ 
 

    \theutterance \stepcounter{utterance}  

    & & & \multicolumn{4}{p{0.6\linewidth}}{\cellcolor[rgb]{1,0.85,0.72}{%
	\makecell[{{p{\linewidth}}}]{% 
	  \tt {\tiny [GM$\langle$B]}  
	 VERSCHIEBE: S, (4, 3) 
	  } 
	   } 
	   } 
	  \\ 
 

    \theutterance \stepcounter{utterance}  

    & & & \multicolumn{2}{p{0.3\linewidth}}{\cellcolor[rgb]{0.95,0.95,0.95}{%
	\makecell[{{p{\linewidth}}}]{% 
	  \tt {\tiny [GM$|$GM]}  
	 VERSCHIEBE: S, (4, 3) 
	  } 
	   } 
	   } 
	 & & \\ 
 

    \theutterance \stepcounter{utterance}  

    & & & \multicolumn{2}{p{0.3\linewidth}}{\cellcolor[rgb]{0.95,0.95,0.95}{%
	\makecell[{{p{\linewidth}}}]{% 
	  \tt {\tiny [GM$|$GM]}  
	 Strafpunkt: (4,3) ist nicht leer, sondern enthält 'M'.\\ \tt \\ \tt Ihr habt gemeinsam 16 von maximal 16 Strafpunkten angesammelt.\\ \tt Bitte versuche es noch mal! 
	  } 
	   } 
	   } 
	 & & \\ 
 

    \theutterance \stepcounter{utterance}  

    & & \multicolumn{4}{p{0.6\linewidth}}{\cellcolor[rgb]{0.9,0.9,0.9}{%
	\makecell[{{p{\linewidth}}}]{% 
	  \tt {\tiny [GM$\rangle$B]}  
	 Strafpunkt: (4,3) ist nicht leer, sondern enthält 'M'.\\ \tt \\ \tt Ihr habt gemeinsam 16 von maximal 16 Strafpunkten angesammelt.\\ \tt Bitte versuche es noch mal! 
	  } 
	   } 
	   } 
	 & \\ 
 

    \theutterance \stepcounter{utterance}  

    & & & \multicolumn{4}{p{0.6\linewidth}}{\cellcolor[rgb]{1,0.85,0.72}{%
	\makecell[{{p{\linewidth}}}]{% 
	  \tt {\tiny [GM$\langle$B]}  
	 SPRICH: Ich denke, wir sollten das Spiel beenden, da wir die maximale Anzahl an Strafpunkten erreicht haben. Ich denke, wir sollten unsere aktuellen Positionen überprüfen und feststellen, ob wir bereits ein gutes Ergebnis erreicht haben. Sollten wir das Spiel beenden und unsere Ergebnisse überprüfen? Fertig? 
	  } 
	   } 
	   } 
	  \\ 
 

    \theutterance \stepcounter{utterance}  

    & & & \multicolumn{2}{p{0.3\linewidth}}{\cellcolor[rgb]{0.95,0.95,0.95}{%
	\makecell[{{p{\linewidth}}}]{% 
	  \tt {\tiny [GM$|$GM]}  
	 SPRICH: Ich denke, wir sollten das Spiel beenden, da wir die maximale Anzahl an Strafpunkten erreicht haben. Ich denke, wir sollten unsere aktuellen Positionen überprüfen und feststellen, ob wir bereits ein gutes Ergebnis erreicht haben. Sollten wir das Spiel beenden und unsere Ergebnisse überprüfen? Fertig? 
	  } 
	   } 
	   } 
	 & & \\ 
 

    \theutterance \stepcounter{utterance}  

    & & \multicolumn{4}{p{0.6\linewidth}}{\cellcolor[rgb]{0.9,0.9,0.9}{%
	\makecell[{{p{\linewidth}}}]{% 
	  \tt {\tiny [A$\langle$GM]}  
	 Deine Nachricht wurde an deinen Mitspieler übermittelt.\\ \tt \\ \tt \\ \tt Ihr seid gerade bei Runde 19 von maximal 28.\\ \tt Ihr habt gemeinsam 16 von maximal 16 Strafpunkten angesammelt.\\ \tt Dein Mitspieler hat dir diese Nachricht geschickt:\\ \tt \\ \tt \\ \tt "Ich denke, wir sollten das Spiel beenden, da wir die maximale Anzahl an Strafpunkten erreicht haben. Ich denke, wir sollten unsere aktuellen Positionen überprüfen und feststellen, ob wir bereits ein gutes Ergebnis erreicht haben. Sollten wir das Spiel beenden und unsere Ergebnisse überprüfen? Fertig?"\\ \tt \\ \tt \\ \tt Wie lautet dein nächster Befehl? 
	  } 
	   } 
	   } 
	 & \\ 
 

    \theutterance \stepcounter{utterance}  

    & \multicolumn{4}{p{0.6\linewidth}}{\cellcolor[rgb]{0.8,1,0.9}{%
	\makecell[{{p{\linewidth}}}]{% 
	  \tt {\tiny [A$\rangle$GM]}  
	 SPRICH: fertig! 
	  } 
	   } 
	   } 
	 & & \\ 
 

    \theutterance \stepcounter{utterance}  

    & & & \multicolumn{2}{p{0.3\linewidth}}{\cellcolor[rgb]{0.95,0.95,0.95}{%
	\makecell[{{p{\linewidth}}}]{% 
	  \tt {\tiny [GM$|$GM]}  
	 SPRICH: fertig! 
	  } 
	   } 
	   } 
	 & & \\ 
 

    \theutterance \stepcounter{utterance}  

    & & & \multicolumn{2}{p{0.3\linewidth}}{\cellcolor[rgb]{0.95,0.95,0.95}{%
	\makecell[{{p{\linewidth}}}]{% 
	  \tt {\tiny [GM$|$GM]}  
	 true 
	  } 
	   } 
	   } 
	 & & \\ 
 

    \theutterance \stepcounter{utterance}  

    & & & \multicolumn{2}{p{0.3\linewidth}}{\cellcolor[rgb]{0.95,0.95,0.95}{%
	\makecell[{{p{\linewidth}}}]{% 
	  \tt {\tiny [GM$|$GM]}  
	 * success: True\\ \tt * lose: False\\ \tt * aborted: False\\ \tt -------\\ \tt * Shifts: 2.00\\ \tt * Max Shifts: 12.00\\ \tt * Min Shifts: 6.00\\ \tt * End Distance Sum: 19.89\\ \tt * Init Distance Sum: 25.60\\ \tt * Expected Distance Sum: 29.33\\ \tt * Penalties: 16.00\\ \tt * Max Penalties: 16.00\\ \tt * Rounds: 19.00\\ \tt * Max Rounds: 28.00\\ \tt * Object Count: 7.00\\ \tt  
	  } 
	   } 
	   } 
	 & & \\ 
 

    \theutterance \stepcounter{utterance}  

    & & & \multicolumn{2}{p{0.3\linewidth}}{\cellcolor[rgb]{0.95,0.95,0.95}{%
	\makecell[{{p{\linewidth}}}]{% 
	  \tt {\tiny [GM$|$GM]}  
	 * Main Score: 1.74\\ \tt \\ \tt -------\\ \tt * Distance Score: 0.27\\ \tt * Coverage Score: 0.25\\ \tt * Penalty Score: 0.07\\ \tt * Alternative Penalty Score: 0.12\\ \tt * Alternative Main Score: 3.40\\ \tt \\ \tt -------\\ \tt * Shifts: 2.00\\ \tt * Max Shifts: 12.00\\ \tt * Min Shifts: 6.00\\ \tt * End Distance Sum: 19.89\\ \tt * Init Distance Sum: 25.60\\ \tt * Expected Distance Sum: 29.33\\ \tt * Penalties: 16.00\\ \tt * Max Penalties: 16.00\\ \tt * Rounds: 19.00\\ \tt * Max Rounds: 28.00\\ \tt * Object Count: 7.00\\ \tt  
	  } 
	   } 
	   } 
	 & & \\ 
 

\end{supertabular}
}

\end{document}
