
\documentclass{article}
\usepackage{colortbl}
\usepackage{makecell}
\usepackage{multirow}
\usepackage{supertabular}

\begin{document}

\newcounter{utterance}

\twocolumn

{ \footnotesize  \setcounter{utterance}{1}
\setlength{\tabcolsep}{0pt}
\begin{supertabular}{c@{$\;$}|p{.15\linewidth}@{}p{.15\linewidth}p{.15\linewidth}p{.15\linewidth}p{.15\linewidth}p{.15\linewidth}}

    \# & $\;$A & \multicolumn{4}{c}{Game Master} & $\;\:$B\\
    \hline 

    \theutterance \stepcounter{utterance}  

    & & \multicolumn{4}{p{0.6\linewidth}}{\cellcolor[rgb]{0.9,0.9,0.9}{%
	\makecell[{{p{\linewidth}}}]{% 
	  \tt {\tiny [A$\langle$GM]}  
	 Sie spielen ein kooperatives Verhandlungsspiel, bei dem Sie sich mit einem anderen Spieler darauf einigen müssen, wie eine Reihe von Gegenständen aufgeteilt werden soll.\\ \tt \\ \tt Die Regeln:\\ \tt (a) Sie und der andere Spieler erhalten eine Sammlung von Gegenständen. Jeder von Ihnen erhält außerdem eine geheime Wertfunktion, die angibt, wie viel Ihnen jede Art von Gegenstand wert ist.\\ \tt (b) Sie tauschen Nachrichten mit dem anderen Spieler aus, um zu vereinbaren, wer welche Gegenstände bekommt. Sie können jeweils maximal 5 Nachrichten senden oder das Spiel vorzeitig beenden, indem Sie jederzeit einen geheimen Vorschlag machen.\\ \tt (c) Jeder von euch wird aufgefordert, einen geheimen Vorschlag zu machen, in dem ihr die gewünschten Gegenstände in eckigen Klammern wie folgt angibt: "[Vorschlag: <Nummer> <Objektname>, <Nummer> <Objektname>, <...>]"\\ \tt (d) Wenn eure Vorschläge komplementär sind, d.h. es gibt genug Gegenstände, um beide Vorschläge zu erfüllen, erhält jeder Spieler eine Punktzahl, die sich aus der Summe der Werte für die Gegenstände ergibt, die er erhalten hat. Andernfalls erhalten Sie beide null Punkte.\\ \tt (e) Ihr Ziel ist es, die Summe aus Ihrer Punktzahl und der Punktzahl des anderen Spielers zu maximieren.\\ \tt \\ \tt Beginnen wir.\\ \tt \\ \tt Die Menge der verfügbaren Gegenstände ist:\\ \tt \\ \tt 3 Lineale, 2 Eier, 1 Batterie\\ \tt \\ \tt Deine geheime Wertfunktion ist:\\ \tt \\ \tt Lineal: 1, Ei: 2, Batterie: 3\\ \tt \\ \tt WICHTIG: Ihre Nachrichten werden, sofern es sich nicht um einen geheimen Vorschlag handelt, direkt an den anderen Spieler übermittelt, also fügen Sie keine Antwort auf die Regeln oder einen Text zur Ankündigung Ihrer Nachricht ein. Um einen geheimen Vorschlag zu machen, verwenden Sie das angegebene Format. Verwenden Sie keine eckigen Klammern, wenn Sie mit dem anderen Spieler kommunizieren, sonst wird dies als Ihr geheimer Vorschlag interpretiert.\\ \tt  
	  } 
	   } 
	   } 
	 & \\ 
 

    \theutterance \stepcounter{utterance}  

    & \multicolumn{4}{p{0.6\linewidth}}{\cellcolor[rgb]{0.8,1,0.9}{%
	\makecell[{{p{\linewidth}}}]{% 
	  \tt {\tiny [A$\rangle$GM]}  
	 Hallo! Danke fürs Mitspielen. Ich schlage vor, dass wir unsere Präferenzen austauschen. Mir sind Eier und die Batterie wichtiger als Lineale. Wie sehen deine Prioritäten aus? Ich bin flexibel bei den Linealen. 
	  } 
	   } 
	   } 
	 & & \\ 
 

    \theutterance \stepcounter{utterance}  

    & & & \multicolumn{2}{p{0.3\linewidth}}{\cellcolor[rgb]{0.95,0.95,0.95}{%
	\makecell[{{p{\linewidth}}}]{% 
	  \tt {\tiny [GM$|$GM]}  
	 continue 
	  } 
	   } 
	   } 
	 & & \\ 
 

    \theutterance \stepcounter{utterance}  

    & & & \multicolumn{2}{p{0.3\linewidth}}{\cellcolor[rgb]{0.95,0.95,0.95}{%
	\makecell[{{p{\linewidth}}}]{% 
	  \tt {\tiny [GM$|$GM]}  
	 Hallo! Danke fürs Mitspielen. Ich schlage vor, dass wir unsere Präferenzen austauschen. Mir sind Eier und die Batterie wichtiger als Lineale. Wie sehen deine Prioritäten aus? Ich bin flexibel bei den Linealen. 
	  } 
	   } 
	   } 
	 & & \\ 
 

    \theutterance \stepcounter{utterance}  

    & & \multicolumn{4}{p{0.6\linewidth}}{\cellcolor[rgb]{0.9,0.9,0.9}{%
	\makecell[{{p{\linewidth}}}]{% 
	  \tt {\tiny [GM$\rangle$B]}  
	 Sie spielen ein kooperatives Verhandlungsspiel, bei dem Sie sich mit einem anderen Spieler darauf einigen müssen, wie eine Reihe von Gegenständen aufgeteilt werden soll.\\ \tt \\ \tt Die Regeln:\\ \tt (a) Sie und der andere Spieler erhalten eine Sammlung von Gegenständen. Jeder von Ihnen erhält außerdem eine geheime Wertfunktion, die angibt, wie viel Ihnen jede Art von Gegenstand wert ist.\\ \tt (b) Sie tauschen Nachrichten mit dem anderen Spieler aus, um zu vereinbaren, wer welche Gegenstände bekommt. Sie können jeweils maximal 5 Nachrichten senden oder das Spiel vorzeitig beenden, indem Sie jederzeit einen geheimen Vorschlag machen.\\ \tt (c) Jeder von euch wird aufgefordert, einen geheimen Vorschlag zu machen, in dem ihr die gewünschten Gegenstände in eckigen Klammern wie folgt angibt: "[Vorschlag: <Nummer> <Objektname>, <Nummer> <Objektname>, <...>]"\\ \tt (d) Wenn eure Vorschläge komplementär sind, d.h. es gibt genug Gegenstände, um beide Vorschläge zu erfüllen, erhält jeder Spieler eine Punktzahl, die sich aus der Summe der Werte für die Gegenstände ergibt, die er erhalten hat. Andernfalls erhalten Sie beide null Punkte.\\ \tt (e) Ihr Ziel ist es, die Summe aus Ihrer Punktzahl und der Punktzahl des anderen Spielers zu maximieren.\\ \tt \\ \tt Beginnen wir.\\ \tt \\ \tt Die Menge der verfügbaren Gegenstände ist:\\ \tt \\ \tt 3 Lineale, 2 Eier, 1 Batterie\\ \tt \\ \tt Deine geheime Wertfunktion ist:\\ \tt \\ \tt Lineal: 3, Ei: 0, Batterie: 1\\ \tt \\ \tt WICHTIG: Ihre Nachrichten werden, sofern es sich nicht um einen geheimen Vorschlag handelt, direkt an den anderen Spieler übermittelt, also fügen Sie keine Antwort auf die Regeln oder einen Text zur Ankündigung Ihrer Nachricht ein. Um einen geheimen Vorschlag zu machen, verwenden Sie das angegebene Format. Verwenden Sie keine eckigen Klammern, wenn Sie mit dem anderen Spieler kommunizieren, sonst wird dies als Ihr geheimer Vorschlag interpretiert.\\ \tt \\ \tt \\ \tt Hallo! Danke fürs Mitspielen. Ich schlage vor, dass wir unsere Präferenzen austauschen. Mir sind Eier und die Batterie wichtiger als Lineale. Wie sehen deine Prioritäten aus? Ich bin flexibel bei den Linealen. 
	  } 
	   } 
	   } 
	 & \\ 
 

    \theutterance \stepcounter{utterance}  

    & & & \multicolumn{4}{p{0.6\linewidth}}{\cellcolor[rgb]{1,0.85,0.72}{%
	\makecell[{{p{\linewidth}}}]{% 
	  \tt {\tiny [GM$\langle$B]}  
	 Danke, das hilft. Mir sind Lineale am wichtigsten, die Batterie ist nett, und Eier sind mir egal. Ich könnte z. B. die meisten Lineale nehmen und dir die Eier geben. Wie wäre es für dich, wenn du beide Eier bekommst? Soll die Batterie zu dir oder wollen wir darüber noch sprechen? 
	  } 
	   } 
	   } 
	  \\ 
 

    \theutterance \stepcounter{utterance}  

    & & & \multicolumn{2}{p{0.3\linewidth}}{\cellcolor[rgb]{0.95,0.95,0.95}{%
	\makecell[{{p{\linewidth}}}]{% 
	  \tt {\tiny [GM$|$GM]}  
	 continue 
	  } 
	   } 
	   } 
	 & & \\ 
 

    \theutterance \stepcounter{utterance}  

    & & & \multicolumn{2}{p{0.3\linewidth}}{\cellcolor[rgb]{0.95,0.95,0.95}{%
	\makecell[{{p{\linewidth}}}]{% 
	  \tt {\tiny [GM$|$GM]}  
	 Danke, das hilft. Mir sind Lineale am wichtigsten, die Batterie ist nett, und Eier sind mir egal. Ich könnte z. B. die meisten Lineale nehmen und dir die Eier geben. Wie wäre es für dich, wenn du beide Eier bekommst? Soll die Batterie zu dir oder wollen wir darüber noch sprechen? 
	  } 
	   } 
	   } 
	 & & \\ 
 

    \theutterance \stepcounter{utterance}  

    & & \multicolumn{4}{p{0.6\linewidth}}{\cellcolor[rgb]{0.9,0.9,0.9}{%
	\makecell[{{p{\linewidth}}}]{% 
	  \tt {\tiny [A$\langle$GM]}  
	 Danke, das hilft. Mir sind Lineale am wichtigsten, die Batterie ist nett, und Eier sind mir egal. Ich könnte z. B. die meisten Lineale nehmen und dir die Eier geben. Wie wäre es für dich, wenn du beide Eier bekommst? Soll die Batterie zu dir oder wollen wir darüber noch sprechen? 
	  } 
	   } 
	   } 
	 & \\ 
 

    \theutterance \stepcounter{utterance}  

    & \multicolumn{4}{p{0.6\linewidth}}{\cellcolor[rgb]{0.8,1,0.9}{%
	\makecell[{{p{\linewidth}}}]{% 
	  \tt {\tiny [A$\rangle$GM]}  
	 Klingt gut. Wenn dir Lineale am wichtigsten sind und dir Eier egal sind, passt das gut zu meinen Präferenzen.\\ \tt \\ \tt Mein Vorschlag:\\ \tt - Du bekommst die meisten Lineale.\\ \tt - Ich bekomme beide Eier.\\ \tt - Zur Batterie: Sie ist mir wichtig, aber nicht so sehr wie dir die Lineale. Wenn du sie möchtest, kann ich das akzeptieren, solange ich beide Eier bekomme. Alternativ könnten wir Batterie zu mir geben, wenn du dafür alle Lineale bekommst.\\ \tt \\ \tt Magst du konkretisieren: Willst du alle 3 Lineale, wenn ich Eier (+ evtl. Batterie) nehme, oder möchtest du 2 Lineale behalten? 
	  } 
	   } 
	   } 
	 & & \\ 
 

    \theutterance \stepcounter{utterance}  

    & & & \multicolumn{2}{p{0.3\linewidth}}{\cellcolor[rgb]{0.95,0.95,0.95}{%
	\makecell[{{p{\linewidth}}}]{% 
	  \tt {\tiny [GM$|$GM]}  
	 continue 
	  } 
	   } 
	   } 
	 & & \\ 
 

    \theutterance \stepcounter{utterance}  

    & & & \multicolumn{2}{p{0.3\linewidth}}{\cellcolor[rgb]{0.95,0.95,0.95}{%
	\makecell[{{p{\linewidth}}}]{% 
	  \tt {\tiny [GM$|$GM]}  
	 Klingt gut. Wenn dir Lineale am wichtigsten sind und dir Eier egal sind, passt das gut zu meinen Präferenzen.\\ \tt \\ \tt Mein Vorschlag:\\ \tt - Du bekommst die meisten Lineale.\\ \tt - Ich bekomme beide Eier.\\ \tt - Zur Batterie: Sie ist mir wichtig, aber nicht so sehr wie dir die Lineale. Wenn du sie möchtest, kann ich das akzeptieren, solange ich beide Eier bekomme. Alternativ könnten wir Batterie zu mir geben, wenn du dafür alle Lineale bekommst.\\ \tt \\ \tt Magst du konkretisieren: Willst du alle 3 Lineale, wenn ich Eier (+ evtl. Batterie) nehme, oder möchtest du 2 Lineale behalten? 
	  } 
	   } 
	   } 
	 & & \\ 
 

    \theutterance \stepcounter{utterance}  

    & & \multicolumn{4}{p{0.6\linewidth}}{\cellcolor[rgb]{0.9,0.9,0.9}{%
	\makecell[{{p{\linewidth}}}]{% 
	  \tt {\tiny [GM$\rangle$B]}  
	 Klingt gut. Wenn dir Lineale am wichtigsten sind und dir Eier egal sind, passt das gut zu meinen Präferenzen.\\ \tt \\ \tt Mein Vorschlag:\\ \tt - Du bekommst die meisten Lineale.\\ \tt - Ich bekomme beide Eier.\\ \tt - Zur Batterie: Sie ist mir wichtig, aber nicht so sehr wie dir die Lineale. Wenn du sie möchtest, kann ich das akzeptieren, solange ich beide Eier bekomme. Alternativ könnten wir Batterie zu mir geben, wenn du dafür alle Lineale bekommst.\\ \tt \\ \tt Magst du konkretisieren: Willst du alle 3 Lineale, wenn ich Eier (+ evtl. Batterie) nehme, oder möchtest du 2 Lineale behalten? 
	  } 
	   } 
	   } 
	 & \\ 
 

    \theutterance \stepcounter{utterance}  

    & & & \multicolumn{4}{p{0.6\linewidth}}{\cellcolor[rgb]{1,0.85,0.72}{%
	\makecell[{{p{\linewidth}}}]{% 
	  \tt {\tiny [GM$\langle$B]}  
	 Danke, das ist sehr kompatibel.\\ \tt \\ \tt Mein konkreter Vorschlag:\\ \tt - Du bekommst beide Eier.\\ \tt - Ich bekomme alle 3 Lineale.\\ \tt - Zur Batterie: Ich bin flexibel. Wenn sie dir wichtig ist, kannst du sie nehmen; wenn nicht, nehme ich sie.\\ \tt \\ \tt Zwei Optionen, such dir eine aus:\\ \tt 1) Du: 2 Eier + Batterie, Ich: 3 Lineale\\ \tt 2) Du: 2 Eier, Ich: 3 Lineale + Batterie\\ \tt \\ \tt Beide sind für mich gut. Welche bevorzugst du? 
	  } 
	   } 
	   } 
	  \\ 
 

    \theutterance \stepcounter{utterance}  

    & & & \multicolumn{2}{p{0.3\linewidth}}{\cellcolor[rgb]{0.95,0.95,0.95}{%
	\makecell[{{p{\linewidth}}}]{% 
	  \tt {\tiny [GM$|$GM]}  
	 continue 
	  } 
	   } 
	   } 
	 & & \\ 
 

    \theutterance \stepcounter{utterance}  

    & & & \multicolumn{2}{p{0.3\linewidth}}{\cellcolor[rgb]{0.95,0.95,0.95}{%
	\makecell[{{p{\linewidth}}}]{% 
	  \tt {\tiny [GM$|$GM]}  
	 Danke, das ist sehr kompatibel.\\ \tt \\ \tt Mein konkreter Vorschlag:\\ \tt - Du bekommst beide Eier.\\ \tt - Ich bekomme alle 3 Lineale.\\ \tt - Zur Batterie: Ich bin flexibel. Wenn sie dir wichtig ist, kannst du sie nehmen; wenn nicht, nehme ich sie.\\ \tt \\ \tt Zwei Optionen, such dir eine aus:\\ \tt 1) Du: 2 Eier + Batterie, Ich: 3 Lineale\\ \tt 2) Du: 2 Eier, Ich: 3 Lineale + Batterie\\ \tt \\ \tt Beide sind für mich gut. Welche bevorzugst du? 
	  } 
	   } 
	   } 
	 & & \\ 
 

    \theutterance \stepcounter{utterance}  

    & & \multicolumn{4}{p{0.6\linewidth}}{\cellcolor[rgb]{0.9,0.9,0.9}{%
	\makecell[{{p{\linewidth}}}]{% 
	  \tt {\tiny [A$\langle$GM]}  
	 Danke, das ist sehr kompatibel.\\ \tt \\ \tt Mein konkreter Vorschlag:\\ \tt - Du bekommst beide Eier.\\ \tt - Ich bekomme alle 3 Lineale.\\ \tt - Zur Batterie: Ich bin flexibel. Wenn sie dir wichtig ist, kannst du sie nehmen; wenn nicht, nehme ich sie.\\ \tt \\ \tt Zwei Optionen, such dir eine aus:\\ \tt 1) Du: 2 Eier + Batterie, Ich: 3 Lineale\\ \tt 2) Du: 2 Eier, Ich: 3 Lineale + Batterie\\ \tt \\ \tt Beide sind für mich gut. Welche bevorzugst du? 
	  } 
	   } 
	   } 
	 & \\ 
 

    \theutterance \stepcounter{utterance}  

    & \multicolumn{4}{p{0.6\linewidth}}{\cellcolor[rgb]{0.8,1,0.9}{%
	\makecell[{{p{\linewidth}}}]{% 
	  \tt {\tiny [A$\rangle$GM]}  
	 Danke, das ist perfekt auf unsere Präferenzen abgestimmt. Ich bevorzuge Option 1.\\ \tt \\ \tt [Vorschlag: 2 Eier, 1 Batterie] 
	  } 
	   } 
	   } 
	 & & \\ 
 

    \theutterance \stepcounter{utterance}  

    & & & \multicolumn{2}{p{0.3\linewidth}}{\cellcolor[rgb]{0.95,0.95,0.95}{%
	\makecell[{{p{\linewidth}}}]{% 
	  \tt {\tiny [GM$|$GM]}  
	 proposal 
	  } 
	   } 
	   } 
	 & & \\ 
 

    \theutterance \stepcounter{utterance}  

    & & & \multicolumn{2}{p{0.3\linewidth}}{\cellcolor[rgb]{0.95,0.95,0.95}{%
	\makecell[{{p{\linewidth}}}]{% 
	  \tt {\tiny [GM$|$GM]}  
	 [0, 2, 1] 
	  } 
	   } 
	   } 
	 & & \\ 
 

    \theutterance \stepcounter{utterance}  

    & & \multicolumn{4}{p{0.6\linewidth}}{\cellcolor[rgb]{0.9,0.9,0.9}{%
	\makecell[{{p{\linewidth}}}]{% 
	  \tt {\tiny [GM$\rangle$B]}  
	 Der andere Spieler hat einen geheimen Vorschlag gemacht. Jetzt bist du an der Reihe, einen eigenen Vorschlag zu machen.\\ \tt  
	  } 
	   } 
	   } 
	 & \\ 
 

    \theutterance \stepcounter{utterance}  

    & & & \multicolumn{4}{p{0.6\linewidth}}{\cellcolor[rgb]{1,0.85,0.72}{%
	\makecell[{{p{\linewidth}}}]{% 
	  \tt {\tiny [GM$\langle$B]}  
	 [Vorschlag: 3 Lineale, 2 Eier] 
	  } 
	   } 
	   } 
	  \\ 
 

    \theutterance \stepcounter{utterance}  

    & & & \multicolumn{2}{p{0.3\linewidth}}{\cellcolor[rgb]{0.95,0.95,0.95}{%
	\makecell[{{p{\linewidth}}}]{% 
	  \tt {\tiny [GM$|$GM]}  
	 proposal 
	  } 
	   } 
	   } 
	 & & \\ 
 

    \theutterance \stepcounter{utterance}  

    & & & \multicolumn{2}{p{0.3\linewidth}}{\cellcolor[rgb]{0.95,0.95,0.95}{%
	\makecell[{{p{\linewidth}}}]{% 
	  \tt {\tiny [GM$|$GM]}  
	 [3, 2, 0] 
	  } 
	   } 
	   } 
	 & & \\ 
 

    \theutterance \stepcounter{utterance}  

    & & & \multicolumn{2}{p{0.3\linewidth}}{\cellcolor[rgb]{0.95,0.95,0.95}{%
	\makecell[{{p{\linewidth}}}]{% 
	  \tt {\tiny [GM$|$GM]}  
	 [[0, 2, 1], [3, 2, 0]] 
	  } 
	   } 
	   } 
	 & & \\ 
 

\end{supertabular}
}

\end{document}
