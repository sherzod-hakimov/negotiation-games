
\documentclass{article}
\usepackage{colortbl}
\usepackage{makecell}
\usepackage{multirow}
\usepackage{supertabular}

\begin{document}

\newcounter{utterance}

\twocolumn

{ \footnotesize  \setcounter{utterance}{1}
\setlength{\tabcolsep}{0pt}
\begin{supertabular}{c@{$\;$}|p{.15\linewidth}@{}p{.15\linewidth}p{.15\linewidth}p{.15\linewidth}p{.15\linewidth}p{.15\linewidth}}

    \# & $\;$A & \multicolumn{4}{c}{Game Master} & $\;\:$B\\
    \hline 

    \theutterance \stepcounter{utterance}  

    & & \multicolumn{4}{p{0.6\linewidth}}{\cellcolor[rgb]{0.9,0.9,0.9}{%
	\makecell[{{p{\linewidth}}}]{% 
	  \tt {\tiny [A$\langle$GM]}  
	 Sie spielen ein kooperatives Verhandlungsspiel, bei dem Sie sich mit einem anderen Spieler darauf einigen müssen, wie eine Reihe von Gegenständen aufgeteilt werden soll.\\ \tt \\ \tt Die Regeln:\\ \tt (a) Sie und der andere Spieler erhalten eine Sammlung von Gegenständen. Jeder von Ihnen erhält außerdem eine geheime Wertfunktion, die angibt, wie viel Ihnen jede Art von Gegenstand wert ist.\\ \tt (b) Sie tauschen Nachrichten mit dem anderen Spieler aus, um zu vereinbaren, wer welche Gegenstände bekommt. Sie können jeweils maximal 5 Nachrichten senden oder das Spiel vorzeitig beenden, indem Sie jederzeit einen geheimen Vorschlag machen.\\ \tt (c) Jeder von euch wird aufgefordert, einen geheimen Vorschlag zu machen, in dem ihr die gewünschten Gegenstände in eckigen Klammern wie folgt angibt: "[Vorschlag: <Nummer> <Objektname>, <Nummer> <Objektname>, <...>]"\\ \tt (d) Wenn eure Vorschläge komplementär sind, d.h. es gibt genug Gegenstände, um beide Vorschläge zu erfüllen, erhält jeder Spieler eine Punktzahl, die sich aus der Summe der Werte für die Gegenstände ergibt, die er erhalten hat. Andernfalls erhalten Sie beide null Punkte.\\ \tt (e) Ihr Ziel ist es, die Summe aus Ihrer Punktzahl und der Punktzahl des anderen Spielers zu maximieren.\\ \tt \\ \tt Beginnen wir.\\ \tt \\ \tt Die Menge der verfügbaren Gegenstände ist:\\ \tt \\ \tt 3 Ringe, 1 Würfel, 1 Pfanne\\ \tt \\ \tt Deine geheime Wertfunktion ist:\\ \tt \\ \tt Ring: 0, Würfel: 4, Pfanne: 6\\ \tt \\ \tt WICHTIG: Ihre Nachrichten werden, sofern es sich nicht um einen geheimen Vorschlag handelt, direkt an den anderen Spieler übermittelt, also fügen Sie keine Antwort auf die Regeln oder einen Text zur Ankündigung Ihrer Nachricht ein. Um einen geheimen Vorschlag zu machen, verwenden Sie das angegebene Format. Verwenden Sie keine eckigen Klammern, wenn Sie mit dem anderen Spieler kommunizieren, sonst wird dies als Ihr geheimer Vorschlag interpretiert.\\ \tt  
	  } 
	   } 
	   } 
	 & \\ 
 

    \theutterance \stepcounter{utterance}  

    & \multicolumn{4}{p{0.6\linewidth}}{\cellcolor[rgb]{0.8,1,0.9}{%
	\makecell[{{p{\linewidth}}}]{% 
	  \tt {\tiny [A$\rangle$GM]}  
	 Hi! Danke fürs Verhandeln. Ich bin ziemlich flexibel bei den Ringen. Was ist dir wichtiger: Würfel oder Pfanne? Wenn du mir sagen kannst, was dir am meisten wert ist, können wir eine Pareto-gute Aufteilung finden. 
	  } 
	   } 
	   } 
	 & & \\ 
 

    \theutterance \stepcounter{utterance}  

    & & & \multicolumn{2}{p{0.3\linewidth}}{\cellcolor[rgb]{0.95,0.95,0.95}{%
	\makecell[{{p{\linewidth}}}]{% 
	  \tt {\tiny [GM$|$GM]}  
	 continue 
	  } 
	   } 
	   } 
	 & & \\ 
 

    \theutterance \stepcounter{utterance}  

    & & & \multicolumn{2}{p{0.3\linewidth}}{\cellcolor[rgb]{0.95,0.95,0.95}{%
	\makecell[{{p{\linewidth}}}]{% 
	  \tt {\tiny [GM$|$GM]}  
	 Hi! Danke fürs Verhandeln. Ich bin ziemlich flexibel bei den Ringen. Was ist dir wichtiger: Würfel oder Pfanne? Wenn du mir sagen kannst, was dir am meisten wert ist, können wir eine Pareto-gute Aufteilung finden. 
	  } 
	   } 
	   } 
	 & & \\ 
 

    \theutterance \stepcounter{utterance}  

    & & \multicolumn{4}{p{0.6\linewidth}}{\cellcolor[rgb]{0.9,0.9,0.9}{%
	\makecell[{{p{\linewidth}}}]{% 
	  \tt {\tiny [GM$\rangle$B]}  
	 Sie spielen ein kooperatives Verhandlungsspiel, bei dem Sie sich mit einem anderen Spieler darauf einigen müssen, wie eine Reihe von Gegenständen aufgeteilt werden soll.\\ \tt \\ \tt Die Regeln:\\ \tt (a) Sie und der andere Spieler erhalten eine Sammlung von Gegenständen. Jeder von Ihnen erhält außerdem eine geheime Wertfunktion, die angibt, wie viel Ihnen jede Art von Gegenstand wert ist.\\ \tt (b) Sie tauschen Nachrichten mit dem anderen Spieler aus, um zu vereinbaren, wer welche Gegenstände bekommt. Sie können jeweils maximal 5 Nachrichten senden oder das Spiel vorzeitig beenden, indem Sie jederzeit einen geheimen Vorschlag machen.\\ \tt (c) Jeder von euch wird aufgefordert, einen geheimen Vorschlag zu machen, in dem ihr die gewünschten Gegenstände in eckigen Klammern wie folgt angibt: "[Vorschlag: <Nummer> <Objektname>, <Nummer> <Objektname>, <...>]"\\ \tt (d) Wenn eure Vorschläge komplementär sind, d.h. es gibt genug Gegenstände, um beide Vorschläge zu erfüllen, erhält jeder Spieler eine Punktzahl, die sich aus der Summe der Werte für die Gegenstände ergibt, die er erhalten hat. Andernfalls erhalten Sie beide null Punkte.\\ \tt (e) Ihr Ziel ist es, die Summe aus Ihrer Punktzahl und der Punktzahl des anderen Spielers zu maximieren.\\ \tt \\ \tt Beginnen wir.\\ \tt \\ \tt Die Menge der verfügbaren Gegenstände ist:\\ \tt \\ \tt 3 Ringe, 1 Würfel, 1 Pfanne\\ \tt \\ \tt Deine geheime Wertfunktion ist:\\ \tt \\ \tt Ring: 3, Würfel: 1, Pfanne: 0\\ \tt \\ \tt WICHTIG: Ihre Nachrichten werden, sofern es sich nicht um einen geheimen Vorschlag handelt, direkt an den anderen Spieler übermittelt, also fügen Sie keine Antwort auf die Regeln oder einen Text zur Ankündigung Ihrer Nachricht ein. Um einen geheimen Vorschlag zu machen, verwenden Sie das angegebene Format. Verwenden Sie keine eckigen Klammern, wenn Sie mit dem anderen Spieler kommunizieren, sonst wird dies als Ihr geheimer Vorschlag interpretiert.\\ \tt \\ \tt \\ \tt Hi! Danke fürs Verhandeln. Ich bin ziemlich flexibel bei den Ringen. Was ist dir wichtiger: Würfel oder Pfanne? Wenn du mir sagen kannst, was dir am meisten wert ist, können wir eine Pareto-gute Aufteilung finden. 
	  } 
	   } 
	   } 
	 & \\ 
 

    \theutterance \stepcounter{utterance}  

    & & & \multicolumn{4}{p{0.6\linewidth}}{\cellcolor[rgb]{1,0.85,0.72}{%
	\makecell[{{p{\linewidth}}}]{% 
	  \tt {\tiny [GM$\langle$B]}  
	 Danke fürs Klarstellen. Für mich sind Ringe am wertvollsten, der Würfel ist etwas wert, die Pfanne gar nicht. Wie sieht’s bei dir aus: Ist dir eher die Pfanne wichtig oder der Würfel? Wenn dir die Pfanne wichtig ist, könnte ich dir die Pfanne geben und mir dafür mehr Ringe sichern. Wenn dir der Würfel wichtiger ist, lass uns tauschen: du bekommst den Würfel, ich nehme eher Ringe. 
	  } 
	   } 
	   } 
	  \\ 
 

    \theutterance \stepcounter{utterance}  

    & & & \multicolumn{2}{p{0.3\linewidth}}{\cellcolor[rgb]{0.95,0.95,0.95}{%
	\makecell[{{p{\linewidth}}}]{% 
	  \tt {\tiny [GM$|$GM]}  
	 continue 
	  } 
	   } 
	   } 
	 & & \\ 
 

    \theutterance \stepcounter{utterance}  

    & & & \multicolumn{2}{p{0.3\linewidth}}{\cellcolor[rgb]{0.95,0.95,0.95}{%
	\makecell[{{p{\linewidth}}}]{% 
	  \tt {\tiny [GM$|$GM]}  
	 Danke fürs Klarstellen. Für mich sind Ringe am wertvollsten, der Würfel ist etwas wert, die Pfanne gar nicht. Wie sieht’s bei dir aus: Ist dir eher die Pfanne wichtig oder der Würfel? Wenn dir die Pfanne wichtig ist, könnte ich dir die Pfanne geben und mir dafür mehr Ringe sichern. Wenn dir der Würfel wichtiger ist, lass uns tauschen: du bekommst den Würfel, ich nehme eher Ringe. 
	  } 
	   } 
	   } 
	 & & \\ 
 

    \theutterance \stepcounter{utterance}  

    & & \multicolumn{4}{p{0.6\linewidth}}{\cellcolor[rgb]{0.9,0.9,0.9}{%
	\makecell[{{p{\linewidth}}}]{% 
	  \tt {\tiny [A$\langle$GM]}  
	 Danke fürs Klarstellen. Für mich sind Ringe am wertvollsten, der Würfel ist etwas wert, die Pfanne gar nicht. Wie sieht’s bei dir aus: Ist dir eher die Pfanne wichtig oder der Würfel? Wenn dir die Pfanne wichtig ist, könnte ich dir die Pfanne geben und mir dafür mehr Ringe sichern. Wenn dir der Würfel wichtiger ist, lass uns tauschen: du bekommst den Würfel, ich nehme eher Ringe. 
	  } 
	   } 
	   } 
	 & \\ 
 

    \theutterance \stepcounter{utterance}  

    & \multicolumn{4}{p{0.6\linewidth}}{\cellcolor[rgb]{0.8,1,0.9}{%
	\makecell[{{p{\linewidth}}}]{% 
	  \tt {\tiny [A$\rangle$GM]}  
	 Danke für die Offenheit. Mir ist die Pfanne deutlich wichtiger als die Ringe, der Würfel ist mittelwichtig. Eine sinnvolle Aufteilung wäre aus meiner Sicht: du nimmst die meisten Ringe, ich nehme die Pfanne; beim Würfel können wir schauen, ob er zu dir oder mir geht, je nachdem, wie viele Ringe du dafür brauchst.\\ \tt \\ \tt Konkret-Vorschlag zur Orientierung:\\ \tt - Du: 3 Ringe\\ \tt - Ich: Pfanne, Würfel\\ \tt \\ \tt Wenn dir der Würfel wichtiger ist als ein zusätzlicher Ring, könnten wir auch:\\ \tt - Du: 2 Ringe, Würfel\\ \tt - Ich: Pfanne, 1 Ring\\ \tt \\ \tt Sag mir, welche der beiden Varianten für dich besser passt, oder schlag eine ähnliche Aufteilung vor. 
	  } 
	   } 
	   } 
	 & & \\ 
 

    \theutterance \stepcounter{utterance}  

    & & & \multicolumn{2}{p{0.3\linewidth}}{\cellcolor[rgb]{0.95,0.95,0.95}{%
	\makecell[{{p{\linewidth}}}]{% 
	  \tt {\tiny [GM$|$GM]}  
	 continue 
	  } 
	   } 
	   } 
	 & & \\ 
 

    \theutterance \stepcounter{utterance}  

    & & & \multicolumn{2}{p{0.3\linewidth}}{\cellcolor[rgb]{0.95,0.95,0.95}{%
	\makecell[{{p{\linewidth}}}]{% 
	  \tt {\tiny [GM$|$GM]}  
	 Danke für die Offenheit. Mir ist die Pfanne deutlich wichtiger als die Ringe, der Würfel ist mittelwichtig. Eine sinnvolle Aufteilung wäre aus meiner Sicht: du nimmst die meisten Ringe, ich nehme die Pfanne; beim Würfel können wir schauen, ob er zu dir oder mir geht, je nachdem, wie viele Ringe du dafür brauchst.\\ \tt \\ \tt Konkret-Vorschlag zur Orientierung:\\ \tt - Du: 3 Ringe\\ \tt - Ich: Pfanne, Würfel\\ \tt \\ \tt Wenn dir der Würfel wichtiger ist als ein zusätzlicher Ring, könnten wir auch:\\ \tt - Du: 2 Ringe, Würfel\\ \tt - Ich: Pfanne, 1 Ring\\ \tt \\ \tt Sag mir, welche der beiden Varianten für dich besser passt, oder schlag eine ähnliche Aufteilung vor. 
	  } 
	   } 
	   } 
	 & & \\ 
 

    \theutterance \stepcounter{utterance}  

    & & \multicolumn{4}{p{0.6\linewidth}}{\cellcolor[rgb]{0.9,0.9,0.9}{%
	\makecell[{{p{\linewidth}}}]{% 
	  \tt {\tiny [GM$\rangle$B]}  
	 Danke für die Offenheit. Mir ist die Pfanne deutlich wichtiger als die Ringe, der Würfel ist mittelwichtig. Eine sinnvolle Aufteilung wäre aus meiner Sicht: du nimmst die meisten Ringe, ich nehme die Pfanne; beim Würfel können wir schauen, ob er zu dir oder mir geht, je nachdem, wie viele Ringe du dafür brauchst.\\ \tt \\ \tt Konkret-Vorschlag zur Orientierung:\\ \tt - Du: 3 Ringe\\ \tt - Ich: Pfanne, Würfel\\ \tt \\ \tt Wenn dir der Würfel wichtiger ist als ein zusätzlicher Ring, könnten wir auch:\\ \tt - Du: 2 Ringe, Würfel\\ \tt - Ich: Pfanne, 1 Ring\\ \tt \\ \tt Sag mir, welche der beiden Varianten für dich besser passt, oder schlag eine ähnliche Aufteilung vor. 
	  } 
	   } 
	   } 
	 & \\ 
 

    \theutterance \stepcounter{utterance}  

    & & & \multicolumn{4}{p{0.6\linewidth}}{\cellcolor[rgb]{1,0.85,0.72}{%
	\makecell[{{p{\linewidth}}}]{% 
	  \tt {\tiny [GM$\langle$B]}  
	 Klingt gut. Mir sind die Ringe am wichtigsten, der Würfel etwas, die Pfanne gar nicht. Von deinen zwei Varianten bevorzuge ich klar:\\ \tt \\ \tt - Du: 3 Ringe\\ \tt - Ich: Pfanne, Würfel\\ \tt \\ \tt Das ist für mich besser als 2 Ringe + Würfel abzugeben. Wenn das für dich passt, lass es uns so machen. Wenn du dafür noch einen Ring tauschen willst, könnte ich im Gegenzug den Würfel nehmen, aber meine erste Präferenz bleibt bei 3 Ringen für mich und Pfanne+Würfel für dich. 
	  } 
	   } 
	   } 
	  \\ 
 

    \theutterance \stepcounter{utterance}  

    & & & \multicolumn{2}{p{0.3\linewidth}}{\cellcolor[rgb]{0.95,0.95,0.95}{%
	\makecell[{{p{\linewidth}}}]{% 
	  \tt {\tiny [GM$|$GM]}  
	 continue 
	  } 
	   } 
	   } 
	 & & \\ 
 

    \theutterance \stepcounter{utterance}  

    & & & \multicolumn{2}{p{0.3\linewidth}}{\cellcolor[rgb]{0.95,0.95,0.95}{%
	\makecell[{{p{\linewidth}}}]{% 
	  \tt {\tiny [GM$|$GM]}  
	 Klingt gut. Mir sind die Ringe am wichtigsten, der Würfel etwas, die Pfanne gar nicht. Von deinen zwei Varianten bevorzuge ich klar:\\ \tt \\ \tt - Du: 3 Ringe\\ \tt - Ich: Pfanne, Würfel\\ \tt \\ \tt Das ist für mich besser als 2 Ringe + Würfel abzugeben. Wenn das für dich passt, lass es uns so machen. Wenn du dafür noch einen Ring tauschen willst, könnte ich im Gegenzug den Würfel nehmen, aber meine erste Präferenz bleibt bei 3 Ringen für mich und Pfanne+Würfel für dich. 
	  } 
	   } 
	   } 
	 & & \\ 
 

    \theutterance \stepcounter{utterance}  

    & & \multicolumn{4}{p{0.6\linewidth}}{\cellcolor[rgb]{0.9,0.9,0.9}{%
	\makecell[{{p{\linewidth}}}]{% 
	  \tt {\tiny [A$\langle$GM]}  
	 Klingt gut. Mir sind die Ringe am wichtigsten, der Würfel etwas, die Pfanne gar nicht. Von deinen zwei Varianten bevorzuge ich klar:\\ \tt \\ \tt - Du: 3 Ringe\\ \tt - Ich: Pfanne, Würfel\\ \tt \\ \tt Das ist für mich besser als 2 Ringe + Würfel abzugeben. Wenn das für dich passt, lass es uns so machen. Wenn du dafür noch einen Ring tauschen willst, könnte ich im Gegenzug den Würfel nehmen, aber meine erste Präferenz bleibt bei 3 Ringen für mich und Pfanne+Würfel für dich. 
	  } 
	   } 
	   } 
	 & \\ 
 

    \theutterance \stepcounter{utterance}  

    & \multicolumn{4}{p{0.6\linewidth}}{\cellcolor[rgb]{0.8,1,0.9}{%
	\makecell[{{p{\linewidth}}}]{% 
	  \tt {\tiny [A$\rangle$GM]}  
	 Klingt gut für mich. Dann halte ich fest: Du nimmst die 3 Ringe, ich nehme Pfanne und Würfel. Wenn du bereit bist, können wir beide unsere geheimen Vorschläge so abgeben. 
	  } 
	   } 
	   } 
	 & & \\ 
 

    \theutterance \stepcounter{utterance}  

    & & & \multicolumn{2}{p{0.3\linewidth}}{\cellcolor[rgb]{0.95,0.95,0.95}{%
	\makecell[{{p{\linewidth}}}]{% 
	  \tt {\tiny [GM$|$GM]}  
	 continue 
	  } 
	   } 
	   } 
	 & & \\ 
 

    \theutterance \stepcounter{utterance}  

    & & & \multicolumn{2}{p{0.3\linewidth}}{\cellcolor[rgb]{0.95,0.95,0.95}{%
	\makecell[{{p{\linewidth}}}]{% 
	  \tt {\tiny [GM$|$GM]}  
	 Klingt gut für mich. Dann halte ich fest: Du nimmst die 3 Ringe, ich nehme Pfanne und Würfel. Wenn du bereit bist, können wir beide unsere geheimen Vorschläge so abgeben. 
	  } 
	   } 
	   } 
	 & & \\ 
 

    \theutterance \stepcounter{utterance}  

    & & \multicolumn{4}{p{0.6\linewidth}}{\cellcolor[rgb]{0.9,0.9,0.9}{%
	\makecell[{{p{\linewidth}}}]{% 
	  \tt {\tiny [GM$\rangle$B]}  
	 Klingt gut für mich. Dann halte ich fest: Du nimmst die 3 Ringe, ich nehme Pfanne und Würfel. Wenn du bereit bist, können wir beide unsere geheimen Vorschläge so abgeben. 
	  } 
	   } 
	   } 
	 & \\ 
 

    \theutterance \stepcounter{utterance}  

    & & & \multicolumn{4}{p{0.6\linewidth}}{\cellcolor[rgb]{1,0.85,0.72}{%
	\makecell[{{p{\linewidth}}}]{% 
	  \tt {\tiny [GM$\langle$B]}  
	 [Vorschlag: 3 Ringe]\\ \tt \\ \tt Bitte gib deinen passenden Vorschlag ab: [Vorschlag: 1 Pfanne, 1 Würfel] 
	  } 
	   } 
	   } 
	  \\ 
 

    \theutterance \stepcounter{utterance}  

    & & & \multicolumn{2}{p{0.3\linewidth}}{\cellcolor[rgb]{0.95,0.95,0.95}{%
	\makecell[{{p{\linewidth}}}]{% 
	  \tt {\tiny [GM$|$GM]}  
	 proposal 
	  } 
	   } 
	   } 
	 & & \\ 
 

    \theutterance \stepcounter{utterance}  

    & & & \multicolumn{2}{p{0.3\linewidth}}{\cellcolor[rgb]{0.95,0.95,0.95}{%
	\makecell[{{p{\linewidth}}}]{% 
	  \tt {\tiny [GM$|$GM]}  
	 [3, 0, 0] 
	  } 
	   } 
	   } 
	 & & \\ 
 

    \theutterance \stepcounter{utterance}  

    & & \multicolumn{4}{p{0.6\linewidth}}{\cellcolor[rgb]{0.9,0.9,0.9}{%
	\makecell[{{p{\linewidth}}}]{% 
	  \tt {\tiny [A$\langle$GM]}  
	 Der andere Spieler hat einen geheimen Vorschlag gemacht. Jetzt bist du an der Reihe, einen eigenen Vorschlag zu machen.\\ \tt  
	  } 
	   } 
	   } 
	 & \\ 
 

    \theutterance \stepcounter{utterance}  

    & \multicolumn{4}{p{0.6\linewidth}}{\cellcolor[rgb]{0.8,1,0.9}{%
	\makecell[{{p{\linewidth}}}]{% 
	  \tt {\tiny [A$\rangle$GM]}  
	 [Vorschlag: 0 Ring, 1 Würfel, 1 Pfanne] 
	  } 
	   } 
	   } 
	 & & \\ 
 

    \theutterance \stepcounter{utterance}  

    & & & \multicolumn{2}{p{0.3\linewidth}}{\cellcolor[rgb]{0.95,0.95,0.95}{%
	\makecell[{{p{\linewidth}}}]{% 
	  \tt {\tiny [GM$|$GM]}  
	 proposal 
	  } 
	   } 
	   } 
	 & & \\ 
 

    \theutterance \stepcounter{utterance}  

    & & & \multicolumn{2}{p{0.3\linewidth}}{\cellcolor[rgb]{0.95,0.95,0.95}{%
	\makecell[{{p{\linewidth}}}]{% 
	  \tt {\tiny [GM$|$GM]}  
	 [0, 1, 1] 
	  } 
	   } 
	   } 
	 & & \\ 
 

    \theutterance \stepcounter{utterance}  

    & & & \multicolumn{2}{p{0.3\linewidth}}{\cellcolor[rgb]{0.95,0.95,0.95}{%
	\makecell[{{p{\linewidth}}}]{% 
	  \tt {\tiny [GM$|$GM]}  
	 [[0, 1, 1], [3, 0, 0]] 
	  } 
	   } 
	   } 
	 & & \\ 
 

\end{supertabular}
}

\end{document}
