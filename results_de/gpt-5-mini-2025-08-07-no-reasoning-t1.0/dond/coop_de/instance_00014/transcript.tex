
\documentclass{article}
\usepackage{colortbl}
\usepackage{makecell}
\usepackage{multirow}
\usepackage{supertabular}

\begin{document}

\newcounter{utterance}

\centering \large Interaction Transcript for game `dond', experiment `coop\_de', episode 14 with gpt{-}5{-}mini{-}2025{-}08{-}07{-}no{-}reasoning{-}t1.0.
\vspace{24pt}

{ \footnotesize  \setcounter{utterance}{1}
\setlength{\tabcolsep}{0pt}
\begin{supertabular}{c@{$\;$}|p{.15\linewidth}@{}p{.15\linewidth}p{.15\linewidth}p{.15\linewidth}p{.15\linewidth}p{.15\linewidth}}
    \# & $\;$A & \multicolumn{4}{c}{Game Master} & $\;\:$B\\
    \hline

    \theutterance \stepcounter{utterance}  
    & & \multicolumn{4}{p{0.6\linewidth}}{
        \cellcolor[rgb]{0.9,0.9,0.9}{
            \makecell[{{p{\linewidth}}}]{
                \texttt{\tiny{[P1$\langle$GM]}}
                \texttt{Sie spielen ein kooperatives Verhandlungsspiel, bei dem Sie sich mit einem anderen Spieler darauf einigen müssen, wie eine Reihe von Gegenständen aufgeteilt werden soll.} \\
\\ 
\texttt{Die Regeln:} \\
\texttt{(a) Sie und der andere Spieler erhalten eine Sammlung von Gegenständen. Jeder von Ihnen erhält außerdem eine geheime Wertfunktion, die angibt, wie viel Ihnen jede Art von Gegenstand wert ist.} \\
\texttt{(b) Sie tauschen Nachrichten mit dem anderen Spieler aus, um zu vereinbaren, wer welche Gegenstände bekommt. Sie können jeweils maximal 5 Nachrichten senden oder das Spiel vorzeitig beenden, indem Sie jederzeit einen geheimen Vorschlag machen.} \\
\texttt{(c) Jeder von euch wird aufgefordert, einen geheimen Vorschlag zu machen, in dem ihr die gewünschten Gegenstände in eckigen Klammern wie folgt angibt: "{[}Vorschlag: <Nummer> <Objektname>, <Nummer> <Objektname>, <...>{]}"} \\
\texttt{(d) Wenn eure Vorschläge komplementär sind, d.h. es gibt genug Gegenstände, um beide Vorschläge zu erfüllen, erhält jeder Spieler eine Punktzahl, die sich aus der Summe der Werte für die Gegenstände ergibt, die er erhalten hat. Andernfalls erhalten Sie beide null Punkte.} \\
\texttt{(e) Ihr Ziel ist es, die Summe aus Ihrer Punktzahl und der Punktzahl des anderen Spielers zu maximieren.} \\
\\ 
\texttt{Beginnen wir.} \\
\\ 
\texttt{Die Menge der verfügbaren Gegenstände ist:} \\
\\ 
\texttt{2 Spiegel, 1 Ziegel, 3 Messer, 1 Kamm} \\
\\ 
\texttt{Deine geheime Wertfunktion ist:} \\
\\ 
\texttt{Spiegel: 1, Ziegel: 4, Messer: 1, Kamm: 1} \\
\\ 
\texttt{WICHTIG: Ihre Nachrichten werden, sofern es sich nicht um einen geheimen Vorschlag handelt, direkt an den anderen Spieler übermittelt, also fügen Sie keine Antwort auf die Regeln oder einen Text zur Ankündigung Ihrer Nachricht ein. Um einen geheimen Vorschlag zu machen, verwenden Sie das angegebene Format. Verwenden Sie keine eckigen Klammern, wenn Sie mit dem anderen Spieler kommunizieren, sonst wird dies als Ihr geheimer Vorschlag interpretiert.} \\
            }
        }
    }
    & \\ \\

    \theutterance \stepcounter{utterance}  
    & \multicolumn{4}{p{0.6\linewidth}}{
        \cellcolor[rgb]{0.9,0.9,0.9}{
            \makecell[{{p{\linewidth}}}]{
                \texttt{\tiny{[P1$\rangle$GM]}}
                \texttt{Hi! Lass uns effizient handeln. Mir sind besonders der Ziegel (4) wichtig, die anderen sind mir jeweils 1 wert. Wie sehen deine Präferenzen aus? Wenn du z. B. hohe Werte auf Messer/Spiegel/Kamm legst, könnte ich gut den Ziegel nehmen und dir den Großteil des Rests überlassen. Welche Gegenstände sind dir am wichtigsten?} \\
            }
        }
    }
    & & \\ \\

    \theutterance \stepcounter{utterance}  
    & & & \multicolumn{2}{p{0.3\linewidth}}{
        \cellcolor[rgb]{0.9,0.9,0.9}{
            \makecell[{{p{\linewidth}}}]{
                \texttt{\tiny{[GM$|$GM]}}
                \texttt{continue} \\
            }
        }
    }
    & & \\ \\

    \theutterance \stepcounter{utterance}  
    & & & \multicolumn{2}{p{0.3\linewidth}}{
        \cellcolor[rgb]{0.9,0.9,0.9}{
            \makecell[{{p{\linewidth}}}]{
                \texttt{\tiny{[GM$|$GM]}}
                \texttt{Hi! Lass uns effizient handeln. Mir sind besonders der Ziegel (4) wichtig, die anderen sind mir jeweils 1 wert. Wie sehen deine Präferenzen aus? Wenn du z. B. hohe Werte auf Messer/Spiegel/Kamm legst, könnte ich gut den Ziegel nehmen und dir den Großteil des Rests überlassen. Welche Gegenstände sind dir am wichtigsten?} \\
            }
        }
    }
    & & \\ \\

    \theutterance \stepcounter{utterance}  
    & & \multicolumn{4}{p{0.6\linewidth}}{
        \cellcolor[rgb]{0.9,0.9,0.9}{
            \makecell[{{p{\linewidth}}}]{
                \texttt{\tiny{[GM$\rangle$P2]}}
                \texttt{Sie spielen ein kooperatives Verhandlungsspiel, bei dem Sie sich mit einem anderen Spieler darauf einigen müssen, wie eine Reihe von Gegenständen aufgeteilt werden soll.} \\
\\ 
\texttt{Die Regeln:} \\
\texttt{(a) Sie und der andere Spieler erhalten eine Sammlung von Gegenständen. Jeder von Ihnen erhält außerdem eine geheime Wertfunktion, die angibt, wie viel Ihnen jede Art von Gegenstand wert ist.} \\
\texttt{(b) Sie tauschen Nachrichten mit dem anderen Spieler aus, um zu vereinbaren, wer welche Gegenstände bekommt. Sie können jeweils maximal 5 Nachrichten senden oder das Spiel vorzeitig beenden, indem Sie jederzeit einen geheimen Vorschlag machen.} \\
\texttt{(c) Jeder von euch wird aufgefordert, einen geheimen Vorschlag zu machen, in dem ihr die gewünschten Gegenstände in eckigen Klammern wie folgt angibt: "{[}Vorschlag: <Nummer> <Objektname>, <Nummer> <Objektname>, <...>{]}"} \\
\texttt{(d) Wenn eure Vorschläge komplementär sind, d.h. es gibt genug Gegenstände, um beide Vorschläge zu erfüllen, erhält jeder Spieler eine Punktzahl, die sich aus der Summe der Werte für die Gegenstände ergibt, die er erhalten hat. Andernfalls erhalten Sie beide null Punkte.} \\
\texttt{(e) Ihr Ziel ist es, die Summe aus Ihrer Punktzahl und der Punktzahl des anderen Spielers zu maximieren.} \\
\\ 
\texttt{Beginnen wir.} \\
\\ 
\texttt{Die Menge der verfügbaren Gegenstände ist:} \\
\\ 
\texttt{2 Spiegel, 1 Ziegel, 3 Messer, 1 Kamm} \\
\\ 
\texttt{Deine geheime Wertfunktion ist:} \\
\\ 
\texttt{Spiegel: 2, Ziegel: 1, Messer: 0, Kamm: 5} \\
\\ 
\texttt{WICHTIG: Ihre Nachrichten werden, sofern es sich nicht um einen geheimen Vorschlag handelt, direkt an den anderen Spieler übermittelt, also fügen Sie keine Antwort auf die Regeln oder einen Text zur Ankündigung Ihrer Nachricht ein. Um einen geheimen Vorschlag zu machen, verwenden Sie das angegebene Format. Verwenden Sie keine eckigen Klammern, wenn Sie mit dem anderen Spieler kommunizieren, sonst wird dies als Ihr geheimer Vorschlag interpretiert.} \\
\\ 
\\ 
\texttt{Hi! Lass uns effizient handeln. Mir sind besonders der Ziegel (4) wichtig, die anderen sind mir jeweils 1 wert. Wie sehen deine Präferenzen aus? Wenn du z. B. hohe Werte auf Messer/Spiegel/Kamm legst, könnte ich gut den Ziegel nehmen und dir den Großteil des Rests überlassen. Welche Gegenstände sind dir am wichtigsten?} \\
            }
        }
    }
    & \\ \\

    \theutterance \stepcounter{utterance}  
    & & & \multicolumn{4}{p{0.6\linewidth}}{
        \cellcolor[rgb]{0.9,0.9,0.9}{
            \makecell[{{p{\linewidth}}}]{
                \texttt{\tiny{[GM$\langle$P2]}}
                \texttt{Danke für die Offenheit! Meine Werte sind: Spiegel 2, Ziegel 1, Messer 0, Kamm 5. Mir ist der Kamm am wichtigsten, Messer sind mir nichts wert, Spiegel moderat, Ziegel gering.} \\
\\ 
\texttt{Vorschlag: Du bekommst den Ziegel. Ich bekomme den Kamm. Die Spiegel und Messer können wir dir großteils zuschlagen, da Messer mir 0 wert sind. Wie klingt: du nimmst den Ziegel, beide Spiegel und 2 Messer; ich nehme den Kamm und 1 Messer?} \\
            }
        }
    }
     \\ \\

    \theutterance \stepcounter{utterance}  
    & & & \multicolumn{2}{p{0.3\linewidth}}{
        \cellcolor[rgb]{0.9,0.9,0.9}{
            \makecell[{{p{\linewidth}}}]{
                \texttt{\tiny{[GM$|$GM]}}
                \texttt{continue} \\
            }
        }
    }
    & & \\ \\

    \theutterance \stepcounter{utterance}  
    & & & \multicolumn{2}{p{0.3\linewidth}}{
        \cellcolor[rgb]{0.9,0.9,0.9}{
            \makecell[{{p{\linewidth}}}]{
                \texttt{\tiny{[GM$|$GM]}}
                \texttt{Danke für die Offenheit! Meine Werte sind: Spiegel 2, Ziegel 1, Messer 0, Kamm 5. Mir ist der Kamm am wichtigsten, Messer sind mir nichts wert, Spiegel moderat, Ziegel gering.} \\
\\ 
\texttt{Vorschlag: Du bekommst den Ziegel. Ich bekomme den Kamm. Die Spiegel und Messer können wir dir großteils zuschlagen, da Messer mir 0 wert sind. Wie klingt: du nimmst den Ziegel, beide Spiegel und 2 Messer; ich nehme den Kamm und 1 Messer?} \\
            }
        }
    }
    & & \\ \\

    \theutterance \stepcounter{utterance}  
    & & \multicolumn{4}{p{0.6\linewidth}}{
        \cellcolor[rgb]{0.9,0.9,0.9}{
            \makecell[{{p{\linewidth}}}]{
                \texttt{\tiny{[P1$\langle$GM]}}
                \texttt{Danke für die Offenheit! Meine Werte sind: Spiegel 2, Ziegel 1, Messer 0, Kamm 5. Mir ist der Kamm am wichtigsten, Messer sind mir nichts wert, Spiegel moderat, Ziegel gering.} \\
\\ 
\texttt{Vorschlag: Du bekommst den Ziegel. Ich bekomme den Kamm. Die Spiegel und Messer können wir dir großteils zuschlagen, da Messer mir 0 wert sind. Wie klingt: du nimmst den Ziegel, beide Spiegel und 2 Messer; ich nehme den Kamm und 1 Messer?} \\
            }
        }
    }
    & \\ \\

    \theutterance \stepcounter{utterance}  
    & \multicolumn{4}{p{0.6\linewidth}}{
        \cellcolor[rgb]{0.9,0.9,0.9}{
            \makecell[{{p{\linewidth}}}]{
                \texttt{\tiny{[P1$\rangle$GM]}}
                \texttt{Klingt sehr gut für mich. Deine Aufteilung ergibt für mich 4 (Ziegel) + 2×1 (Spiegel) + 2×1 (Messer) = 8 Punkte, und für dich 5 (Kamm) + 0 (Messer) = 5 Punkte. Gesamt 13 – das ist vermutlich nahe am Optimum, da Messer für dich 0 sind und der Ziegel für mich am wertvollsten ist.} \\
\\ 
\texttt{Wenn du möchtest, können wir es genau so festhalten. Wenn du keine Anpassung willst, bin ich bereit, den geheimen Vorschlag entsprechend abzugeben.} \\
            }
        }
    }
    & & \\ \\

    \theutterance \stepcounter{utterance}  
    & & & \multicolumn{2}{p{0.3\linewidth}}{
        \cellcolor[rgb]{0.9,0.9,0.9}{
            \makecell[{{p{\linewidth}}}]{
                \texttt{\tiny{[GM$|$GM]}}
                \texttt{continue} \\
            }
        }
    }
    & & \\ \\

    \theutterance \stepcounter{utterance}  
    & & & \multicolumn{2}{p{0.3\linewidth}}{
        \cellcolor[rgb]{0.9,0.9,0.9}{
            \makecell[{{p{\linewidth}}}]{
                \texttt{\tiny{[GM$|$GM]}}
                \texttt{Klingt sehr gut für mich. Deine Aufteilung ergibt für mich 4 (Ziegel) + 2×1 (Spiegel) + 2×1 (Messer) = 8 Punkte, und für dich 5 (Kamm) + 0 (Messer) = 5 Punkte. Gesamt 13 – das ist vermutlich nahe am Optimum, da Messer für dich 0 sind und der Ziegel für mich am wertvollsten ist.} \\
\\ 
\texttt{Wenn du möchtest, können wir es genau so festhalten. Wenn du keine Anpassung willst, bin ich bereit, den geheimen Vorschlag entsprechend abzugeben.} \\
            }
        }
    }
    & & \\ \\

    \theutterance \stepcounter{utterance}  
    & & \multicolumn{4}{p{0.6\linewidth}}{
        \cellcolor[rgb]{0.9,0.9,0.9}{
            \makecell[{{p{\linewidth}}}]{
                \texttt{\tiny{[GM$\rangle$P2]}}
                \texttt{Klingt sehr gut für mich. Deine Aufteilung ergibt für mich 4 (Ziegel) + 2×1 (Spiegel) + 2×1 (Messer) = 8 Punkte, und für dich 5 (Kamm) + 0 (Messer) = 5 Punkte. Gesamt 13 – das ist vermutlich nahe am Optimum, da Messer für dich 0 sind und der Ziegel für mich am wertvollsten ist.} \\
\\ 
\texttt{Wenn du möchtest, können wir es genau so festhalten. Wenn du keine Anpassung willst, bin ich bereit, den geheimen Vorschlag entsprechend abzugeben.} \\
            }
        }
    }
    & \\ \\

    \theutterance \stepcounter{utterance}  
    & & & \multicolumn{4}{p{0.6\linewidth}}{
        \cellcolor[rgb]{0.9,0.9,0.9}{
            \makecell[{{p{\linewidth}}}]{
                \texttt{\tiny{[GM$\langle$P2]}}
                \texttt{Klingt gut, ich bin einverstanden. Lass uns genau diese Aufteilung fixieren.} \\
\\ 
\texttt{{[}Bitte gib deinen geheimen Vorschlag so ab: "{[}Vorschlag: 1 Ziegel, 2 Spiegel, 2 Messer{]}".{]}} \\
\\ 
\texttt{{[}Vorschlag: 1 Kamm, 1 Messer{]}} \\
            }
        }
    }
     \\ \\

    \theutterance \stepcounter{utterance}  
    & & & \multicolumn{2}{p{0.3\linewidth}}{
        \cellcolor[rgb]{0.9,0.9,0.9}{
            \makecell[{{p{\linewidth}}}]{
                \texttt{\tiny{[GM$|$GM]}}
                \texttt{proposal does not start with "vorschlag:"} \\
            }
        }
    }
    & & \\ \\

\end{supertabular}
}

\end{document}
