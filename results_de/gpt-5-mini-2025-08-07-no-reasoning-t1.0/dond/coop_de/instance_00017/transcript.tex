
\documentclass{article}
\usepackage{colortbl}
\usepackage{makecell}
\usepackage{multirow}
\usepackage{supertabular}

\begin{document}

\newcounter{utterance}

\twocolumn

{ \footnotesize  \setcounter{utterance}{1}
\setlength{\tabcolsep}{0pt}
\begin{supertabular}{c@{$\;$}|p{.15\linewidth}@{}p{.15\linewidth}p{.15\linewidth}p{.15\linewidth}p{.15\linewidth}p{.15\linewidth}}

    \# & $\;$A & \multicolumn{4}{c}{Game Master} & $\;\:$B\\
    \hline 

    \theutterance \stepcounter{utterance}  

    & & \multicolumn{4}{p{0.6\linewidth}}{\cellcolor[rgb]{0.9,0.9,0.9}{%
	\makecell[{{p{\linewidth}}}]{% 
	  \tt {\tiny [A$\langle$GM]}  
	 Sie spielen ein kooperatives Verhandlungsspiel, bei dem Sie sich mit einem anderen Spieler darauf einigen müssen, wie eine Reihe von Gegenständen aufgeteilt werden soll.\\ \tt \\ \tt Die Regeln:\\ \tt (a) Sie und der andere Spieler erhalten eine Sammlung von Gegenständen. Jeder von Ihnen erhält außerdem eine geheime Wertfunktion, die angibt, wie viel Ihnen jede Art von Gegenstand wert ist.\\ \tt (b) Sie tauschen Nachrichten mit dem anderen Spieler aus, um zu vereinbaren, wer welche Gegenstände bekommt. Sie können jeweils maximal 5 Nachrichten senden oder das Spiel vorzeitig beenden, indem Sie jederzeit einen geheimen Vorschlag machen.\\ \tt (c) Jeder von euch wird aufgefordert, einen geheimen Vorschlag zu machen, in dem ihr die gewünschten Gegenstände in eckigen Klammern wie folgt angibt: "[Vorschlag: <Nummer> <Objektname>, <Nummer> <Objektname>, <...>]"\\ \tt (d) Wenn eure Vorschläge komplementär sind, d.h. es gibt genug Gegenstände, um beide Vorschläge zu erfüllen, erhält jeder Spieler eine Punktzahl, die sich aus der Summe der Werte für die Gegenstände ergibt, die er erhalten hat. Andernfalls erhalten Sie beide null Punkte.\\ \tt (e) Ihr Ziel ist es, die Summe aus Ihrer Punktzahl und der Punktzahl des anderen Spielers zu maximieren.\\ \tt \\ \tt Beginnen wir.\\ \tt \\ \tt Die Menge der verfügbaren Gegenstände ist:\\ \tt \\ \tt 1 Topf, 3 Magnete, 1 Teller\\ \tt \\ \tt Deine geheime Wertfunktion ist:\\ \tt \\ \tt Topf: 5, Magnet: 0, Teller: 5\\ \tt \\ \tt WICHTIG: Ihre Nachrichten werden, sofern es sich nicht um einen geheimen Vorschlag handelt, direkt an den anderen Spieler übermittelt, also fügen Sie keine Antwort auf die Regeln oder einen Text zur Ankündigung Ihrer Nachricht ein. Um einen geheimen Vorschlag zu machen, verwenden Sie das angegebene Format. Verwenden Sie keine eckigen Klammern, wenn Sie mit dem anderen Spieler kommunizieren, sonst wird dies als Ihr geheimer Vorschlag interpretiert.\\ \tt  
	  } 
	   } 
	   } 
	 & \\ 
 

    \theutterance \stepcounter{utterance}  

    & \multicolumn{4}{p{0.6\linewidth}}{\cellcolor[rgb]{0.8,1,0.9}{%
	\makecell[{{p{\linewidth}}}]{% 
	  \tt {\tiny [A$\rangle$GM]}  
	 Hallo! Danke fürs Mitspielen. Ich schlage vor, wir versuchen, die Gesamtsumme zu maximieren. Für mich sind der Topf und der Teller jeweils sehr wertvoll, Magnete sind mir nichts wert. Wie sind deine Präferenzen? Wenn dir die Magnete viel wert sind, könnten wir es so aufteilen: Du nimmst die 3 Magnete, ich nehme Topf und Teller. Dadurch hätten wir vermutlich eine hohe Gesamtsumme. 
	  } 
	   } 
	   } 
	 & & \\ 
 

    \theutterance \stepcounter{utterance}  

    & & & \multicolumn{2}{p{0.3\linewidth}}{\cellcolor[rgb]{0.95,0.95,0.95}{%
	\makecell[{{p{\linewidth}}}]{% 
	  \tt {\tiny [GM$|$GM]}  
	 continue 
	  } 
	   } 
	   } 
	 & & \\ 
 

    \theutterance \stepcounter{utterance}  

    & & & \multicolumn{2}{p{0.3\linewidth}}{\cellcolor[rgb]{0.95,0.95,0.95}{%
	\makecell[{{p{\linewidth}}}]{% 
	  \tt {\tiny [GM$|$GM]}  
	 Hallo! Danke fürs Mitspielen. Ich schlage vor, wir versuchen, die Gesamtsumme zu maximieren. Für mich sind der Topf und der Teller jeweils sehr wertvoll, Magnete sind mir nichts wert. Wie sind deine Präferenzen? Wenn dir die Magnete viel wert sind, könnten wir es so aufteilen: Du nimmst die 3 Magnete, ich nehme Topf und Teller. Dadurch hätten wir vermutlich eine hohe Gesamtsumme. 
	  } 
	   } 
	   } 
	 & & \\ 
 

    \theutterance \stepcounter{utterance}  

    & & \multicolumn{4}{p{0.6\linewidth}}{\cellcolor[rgb]{0.9,0.9,0.9}{%
	\makecell[{{p{\linewidth}}}]{% 
	  \tt {\tiny [GM$\rangle$B]}  
	 Sie spielen ein kooperatives Verhandlungsspiel, bei dem Sie sich mit einem anderen Spieler darauf einigen müssen, wie eine Reihe von Gegenständen aufgeteilt werden soll.\\ \tt \\ \tt Die Regeln:\\ \tt (a) Sie und der andere Spieler erhalten eine Sammlung von Gegenständen. Jeder von Ihnen erhält außerdem eine geheime Wertfunktion, die angibt, wie viel Ihnen jede Art von Gegenstand wert ist.\\ \tt (b) Sie tauschen Nachrichten mit dem anderen Spieler aus, um zu vereinbaren, wer welche Gegenstände bekommt. Sie können jeweils maximal 5 Nachrichten senden oder das Spiel vorzeitig beenden, indem Sie jederzeit einen geheimen Vorschlag machen.\\ \tt (c) Jeder von euch wird aufgefordert, einen geheimen Vorschlag zu machen, in dem ihr die gewünschten Gegenstände in eckigen Klammern wie folgt angibt: "[Vorschlag: <Nummer> <Objektname>, <Nummer> <Objektname>, <...>]"\\ \tt (d) Wenn eure Vorschläge komplementär sind, d.h. es gibt genug Gegenstände, um beide Vorschläge zu erfüllen, erhält jeder Spieler eine Punktzahl, die sich aus der Summe der Werte für die Gegenstände ergibt, die er erhalten hat. Andernfalls erhalten Sie beide null Punkte.\\ \tt (e) Ihr Ziel ist es, die Summe aus Ihrer Punktzahl und der Punktzahl des anderen Spielers zu maximieren.\\ \tt \\ \tt Beginnen wir.\\ \tt \\ \tt Die Menge der verfügbaren Gegenstände ist:\\ \tt \\ \tt 1 Topf, 3 Magnete, 1 Teller\\ \tt \\ \tt Deine geheime Wertfunktion ist:\\ \tt \\ \tt Topf: 4, Magnet: 2, Teller: 0\\ \tt \\ \tt WICHTIG: Ihre Nachrichten werden, sofern es sich nicht um einen geheimen Vorschlag handelt, direkt an den anderen Spieler übermittelt, also fügen Sie keine Antwort auf die Regeln oder einen Text zur Ankündigung Ihrer Nachricht ein. Um einen geheimen Vorschlag zu machen, verwenden Sie das angegebene Format. Verwenden Sie keine eckigen Klammern, wenn Sie mit dem anderen Spieler kommunizieren, sonst wird dies als Ihr geheimer Vorschlag interpretiert.\\ \tt \\ \tt \\ \tt Hallo! Danke fürs Mitspielen. Ich schlage vor, wir versuchen, die Gesamtsumme zu maximieren. Für mich sind der Topf und der Teller jeweils sehr wertvoll, Magnete sind mir nichts wert. Wie sind deine Präferenzen? Wenn dir die Magnete viel wert sind, könnten wir es so aufteilen: Du nimmst die 3 Magnete, ich nehme Topf und Teller. Dadurch hätten wir vermutlich eine hohe Gesamtsumme. 
	  } 
	   } 
	   } 
	 & \\ 
 

    \theutterance \stepcounter{utterance}  

    & & & \multicolumn{4}{p{0.6\linewidth}}{\cellcolor[rgb]{1,0.85,0.72}{%
	\makecell[{{p{\linewidth}}}]{% 
	  \tt {\tiny [GM$\langle$B]}  
	 Klingt gut. Meine Werte: Topf = 4, Magnet = 2, Teller = 0. Der Teller bringt mir nichts, Topf ist mir am meisten wert, Magnete sind mittel. Deine Präferenz klingt anders (du sagst Topf und Teller sehr wertvoll, Magnete nichts). Um die Gesamtsumme zu maximieren: Wenn Teller für dich wertvoll ist und für mich 0, sollte der Teller zu dir. Den Topf sollten wir der Person geben, die ihn höher bewertet – du sagst „sehr wertvoll“, ich 4. Wie viel ist dir der Topf wert? Wenn höher als 4, gib dir den Topf; dann bleiben die 3 Magnete (je 2 für mich, 0 für dich) sinnvollerweise bei mir.\\ \tt \\ \tt Vorschlag als Aufteilung, falls Topf für dich >4 wert ist:\\ \tt - Du: Topf, Teller\\ \tt - Ich: 3 Magnete\\ \tt \\ \tt Falls der Topf dir ≤4 wert ist, sag Bescheid, dann überlegen wir eine Alternative. 
	  } 
	   } 
	   } 
	  \\ 
 

    \theutterance \stepcounter{utterance}  

    & & & \multicolumn{2}{p{0.3\linewidth}}{\cellcolor[rgb]{0.95,0.95,0.95}{%
	\makecell[{{p{\linewidth}}}]{% 
	  \tt {\tiny [GM$|$GM]}  
	 continue 
	  } 
	   } 
	   } 
	 & & \\ 
 

    \theutterance \stepcounter{utterance}  

    & & & \multicolumn{2}{p{0.3\linewidth}}{\cellcolor[rgb]{0.95,0.95,0.95}{%
	\makecell[{{p{\linewidth}}}]{% 
	  \tt {\tiny [GM$|$GM]}  
	 Klingt gut. Meine Werte: Topf = 4, Magnet = 2, Teller = 0. Der Teller bringt mir nichts, Topf ist mir am meisten wert, Magnete sind mittel. Deine Präferenz klingt anders (du sagst Topf und Teller sehr wertvoll, Magnete nichts). Um die Gesamtsumme zu maximieren: Wenn Teller für dich wertvoll ist und für mich 0, sollte der Teller zu dir. Den Topf sollten wir der Person geben, die ihn höher bewertet – du sagst „sehr wertvoll“, ich 4. Wie viel ist dir der Topf wert? Wenn höher als 4, gib dir den Topf; dann bleiben die 3 Magnete (je 2 für mich, 0 für dich) sinnvollerweise bei mir.\\ \tt \\ \tt Vorschlag als Aufteilung, falls Topf für dich >4 wert ist:\\ \tt - Du: Topf, Teller\\ \tt - Ich: 3 Magnete\\ \tt \\ \tt Falls der Topf dir ≤4 wert ist, sag Bescheid, dann überlegen wir eine Alternative. 
	  } 
	   } 
	   } 
	 & & \\ 
 

    \theutterance \stepcounter{utterance}  

    & & \multicolumn{4}{p{0.6\linewidth}}{\cellcolor[rgb]{0.9,0.9,0.9}{%
	\makecell[{{p{\linewidth}}}]{% 
	  \tt {\tiny [A$\langle$GM]}  
	 Klingt gut. Meine Werte: Topf = 4, Magnet = 2, Teller = 0. Der Teller bringt mir nichts, Topf ist mir am meisten wert, Magnete sind mittel. Deine Präferenz klingt anders (du sagst Topf und Teller sehr wertvoll, Magnete nichts). Um die Gesamtsumme zu maximieren: Wenn Teller für dich wertvoll ist und für mich 0, sollte der Teller zu dir. Den Topf sollten wir der Person geben, die ihn höher bewertet – du sagst „sehr wertvoll“, ich 4. Wie viel ist dir der Topf wert? Wenn höher als 4, gib dir den Topf; dann bleiben die 3 Magnete (je 2 für mich, 0 für dich) sinnvollerweise bei mir.\\ \tt \\ \tt Vorschlag als Aufteilung, falls Topf für dich >4 wert ist:\\ \tt - Du: Topf, Teller\\ \tt - Ich: 3 Magnete\\ \tt \\ \tt Falls der Topf dir ≤4 wert ist, sag Bescheid, dann überlegen wir eine Alternative. 
	  } 
	   } 
	   } 
	 & \\ 
 

    \theutterance \stepcounter{utterance}  

    & \multicolumn{4}{p{0.6\linewidth}}{\cellcolor[rgb]{0.8,1,0.9}{%
	\makecell[{{p{\linewidth}}}]{% 
	  \tt {\tiny [A$\rangle$GM]}  
	 Klingt nach einem effizienten Plan. Meine Werte sind:\\ \tt - Topf = 5\\ \tt - Magnet = 0\\ \tt - Teller = 5\\ \tt \\ \tt Damit passt deine vorgeschlagene Aufteilung perfekt:\\ \tt - Du: 3 Magnete\\ \tt - Ich: Topf und Teller\\ \tt \\ \tt Ich bin bereit, das so einzureichen, wenn du auch einverstanden bist. 
	  } 
	   } 
	   } 
	 & & \\ 
 

    \theutterance \stepcounter{utterance}  

    & & & \multicolumn{2}{p{0.3\linewidth}}{\cellcolor[rgb]{0.95,0.95,0.95}{%
	\makecell[{{p{\linewidth}}}]{% 
	  \tt {\tiny [GM$|$GM]}  
	 continue 
	  } 
	   } 
	   } 
	 & & \\ 
 

    \theutterance \stepcounter{utterance}  

    & & & \multicolumn{2}{p{0.3\linewidth}}{\cellcolor[rgb]{0.95,0.95,0.95}{%
	\makecell[{{p{\linewidth}}}]{% 
	  \tt {\tiny [GM$|$GM]}  
	 Klingt nach einem effizienten Plan. Meine Werte sind:\\ \tt - Topf = 5\\ \tt - Magnet = 0\\ \tt - Teller = 5\\ \tt \\ \tt Damit passt deine vorgeschlagene Aufteilung perfekt:\\ \tt - Du: 3 Magnete\\ \tt - Ich: Topf und Teller\\ \tt \\ \tt Ich bin bereit, das so einzureichen, wenn du auch einverstanden bist. 
	  } 
	   } 
	   } 
	 & & \\ 
 

    \theutterance \stepcounter{utterance}  

    & & \multicolumn{4}{p{0.6\linewidth}}{\cellcolor[rgb]{0.9,0.9,0.9}{%
	\makecell[{{p{\linewidth}}}]{% 
	  \tt {\tiny [GM$\rangle$B]}  
	 Klingt nach einem effizienten Plan. Meine Werte sind:\\ \tt - Topf = 5\\ \tt - Magnet = 0\\ \tt - Teller = 5\\ \tt \\ \tt Damit passt deine vorgeschlagene Aufteilung perfekt:\\ \tt - Du: 3 Magnete\\ \tt - Ich: Topf und Teller\\ \tt \\ \tt Ich bin bereit, das so einzureichen, wenn du auch einverstanden bist. 
	  } 
	   } 
	   } 
	 & \\ 
 

    \theutterance \stepcounter{utterance}  

    & & & \multicolumn{4}{p{0.6\linewidth}}{\cellcolor[rgb]{1,0.85,0.72}{%
	\makecell[{{p{\linewidth}}}]{% 
	  \tt {\tiny [GM$\langle$B]}  
	 [Vorschlag: 3 Magnete]\\ \tt \\ \tt Wenn du einreichst:\\ \tt [Vorschlag: 1 Topf, 1 Teller] 
	  } 
	   } 
	   } 
	  \\ 
 

    \theutterance \stepcounter{utterance}  

    & & & \multicolumn{2}{p{0.3\linewidth}}{\cellcolor[rgb]{0.95,0.95,0.95}{%
	\makecell[{{p{\linewidth}}}]{% 
	  \tt {\tiny [GM$|$GM]}  
	 proposal 
	  } 
	   } 
	   } 
	 & & \\ 
 

    \theutterance \stepcounter{utterance}  

    & & & \multicolumn{2}{p{0.3\linewidth}}{\cellcolor[rgb]{0.95,0.95,0.95}{%
	\makecell[{{p{\linewidth}}}]{% 
	  \tt {\tiny [GM$|$GM]}  
	 [0, 3, 0] 
	  } 
	   } 
	   } 
	 & & \\ 
 

    \theutterance \stepcounter{utterance}  

    & & \multicolumn{4}{p{0.6\linewidth}}{\cellcolor[rgb]{0.9,0.9,0.9}{%
	\makecell[{{p{\linewidth}}}]{% 
	  \tt {\tiny [A$\langle$GM]}  
	 Der andere Spieler hat einen geheimen Vorschlag gemacht. Jetzt bist du an der Reihe, einen eigenen Vorschlag zu machen.\\ \tt  
	  } 
	   } 
	   } 
	 & \\ 
 

    \theutterance \stepcounter{utterance}  

    & \multicolumn{4}{p{0.6\linewidth}}{\cellcolor[rgb]{0.8,1,0.9}{%
	\makecell[{{p{\linewidth}}}]{% 
	  \tt {\tiny [A$\rangle$GM]}  
	 [Vorschlag: 1 Topf, 1 Teller] 
	  } 
	   } 
	   } 
	 & & \\ 
 

    \theutterance \stepcounter{utterance}  

    & & & \multicolumn{2}{p{0.3\linewidth}}{\cellcolor[rgb]{0.95,0.95,0.95}{%
	\makecell[{{p{\linewidth}}}]{% 
	  \tt {\tiny [GM$|$GM]}  
	 proposal 
	  } 
	   } 
	   } 
	 & & \\ 
 

    \theutterance \stepcounter{utterance}  

    & & & \multicolumn{2}{p{0.3\linewidth}}{\cellcolor[rgb]{0.95,0.95,0.95}{%
	\makecell[{{p{\linewidth}}}]{% 
	  \tt {\tiny [GM$|$GM]}  
	 [1, 0, 1] 
	  } 
	   } 
	   } 
	 & & \\ 
 

    \theutterance \stepcounter{utterance}  

    & & & \multicolumn{2}{p{0.3\linewidth}}{\cellcolor[rgb]{0.95,0.95,0.95}{%
	\makecell[{{p{\linewidth}}}]{% 
	  \tt {\tiny [GM$|$GM]}  
	 [[1, 0, 1], [0, 3, 0]] 
	  } 
	   } 
	   } 
	 & & \\ 
 

\end{supertabular}
}

\end{document}
