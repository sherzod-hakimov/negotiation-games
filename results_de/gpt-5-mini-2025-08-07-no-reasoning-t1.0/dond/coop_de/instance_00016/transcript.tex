
\documentclass{article}
\usepackage{colortbl}
\usepackage{makecell}
\usepackage{multirow}
\usepackage{supertabular}

\begin{document}

\newcounter{utterance}

\twocolumn

{ \footnotesize  \setcounter{utterance}{1}
\setlength{\tabcolsep}{0pt}
\begin{supertabular}{c@{$\;$}|p{.15\linewidth}@{}p{.15\linewidth}p{.15\linewidth}p{.15\linewidth}p{.15\linewidth}p{.15\linewidth}}

    \# & $\;$A & \multicolumn{4}{c}{Game Master} & $\;\:$B\\
    \hline 

    \theutterance \stepcounter{utterance}  

    & & \multicolumn{4}{p{0.6\linewidth}}{\cellcolor[rgb]{0.9,0.9,0.9}{%
	\makecell[{{p{\linewidth}}}]{% 
	  \tt {\tiny [A$\langle$GM]}  
	 Sie spielen ein kooperatives Verhandlungsspiel, bei dem Sie sich mit einem anderen Spieler darauf einigen müssen, wie eine Reihe von Gegenständen aufgeteilt werden soll.\\ \tt \\ \tt Die Regeln:\\ \tt (a) Sie und der andere Spieler erhalten eine Sammlung von Gegenständen. Jeder von Ihnen erhält außerdem eine geheime Wertfunktion, die angibt, wie viel Ihnen jede Art von Gegenstand wert ist.\\ \tt (b) Sie tauschen Nachrichten mit dem anderen Spieler aus, um zu vereinbaren, wer welche Gegenstände bekommt. Sie können jeweils maximal 5 Nachrichten senden oder das Spiel vorzeitig beenden, indem Sie jederzeit einen geheimen Vorschlag machen.\\ \tt (c) Jeder von euch wird aufgefordert, einen geheimen Vorschlag zu machen, in dem ihr die gewünschten Gegenstände in eckigen Klammern wie folgt angibt: "[Vorschlag: <Nummer> <Objektname>, <Nummer> <Objektname>, <...>]"\\ \tt (d) Wenn eure Vorschläge komplementär sind, d.h. es gibt genug Gegenstände, um beide Vorschläge zu erfüllen, erhält jeder Spieler eine Punktzahl, die sich aus der Summe der Werte für die Gegenstände ergibt, die er erhalten hat. Andernfalls erhalten Sie beide null Punkte.\\ \tt (e) Ihr Ziel ist es, die Summe aus Ihrer Punktzahl und der Punktzahl des anderen Spielers zu maximieren.\\ \tt \\ \tt Beginnen wir.\\ \tt \\ \tt Die Menge der verfügbaren Gegenstände ist:\\ \tt \\ \tt 1 Eimer, 5 Bücher, 1 Spiegel\\ \tt \\ \tt Deine geheime Wertfunktion ist:\\ \tt \\ \tt Eimer: 4, Buch: 1, Spiegel: 1\\ \tt \\ \tt WICHTIG: Ihre Nachrichten werden, sofern es sich nicht um einen geheimen Vorschlag handelt, direkt an den anderen Spieler übermittelt, also fügen Sie keine Antwort auf die Regeln oder einen Text zur Ankündigung Ihrer Nachricht ein. Um einen geheimen Vorschlag zu machen, verwenden Sie das angegebene Format. Verwenden Sie keine eckigen Klammern, wenn Sie mit dem anderen Spieler kommunizieren, sonst wird dies als Ihr geheimer Vorschlag interpretiert.\\ \tt  
	  } 
	   } 
	   } 
	 & \\ 
 

    \theutterance \stepcounter{utterance}  

    & \multicolumn{4}{p{0.6\linewidth}}{\cellcolor[rgb]{0.8,1,0.9}{%
	\makecell[{{p{\linewidth}}}]{% 
	  \tt {\tiny [A$\rangle$GM]}  
	 Hallo! Lass uns versuchen, gemeinsam die Gesamtsumme zu maximieren. Meine groben Präferenzen: Der Eimer ist mir am meisten wert, Bücher sind mir leicht etwas wert, der Spiegel wenig.\\ \tt \\ \tt Könntest du mir sagen, welche Gegenstände dir besonders wichtig sind und wie stark? Falls du den Eimer sehr schätzt, kann ich mich dafür mit Büchern zufriedengeben. Andernfalls würde ich gern den Eimer nehmen und dir den Spiegel plus einige Bücher überlassen. 
	  } 
	   } 
	   } 
	 & & \\ 
 

    \theutterance \stepcounter{utterance}  

    & & & \multicolumn{2}{p{0.3\linewidth}}{\cellcolor[rgb]{0.95,0.95,0.95}{%
	\makecell[{{p{\linewidth}}}]{% 
	  \tt {\tiny [GM$|$GM]}  
	 continue 
	  } 
	   } 
	   } 
	 & & \\ 
 

    \theutterance \stepcounter{utterance}  

    & & & \multicolumn{2}{p{0.3\linewidth}}{\cellcolor[rgb]{0.95,0.95,0.95}{%
	\makecell[{{p{\linewidth}}}]{% 
	  \tt {\tiny [GM$|$GM]}  
	 Hallo! Lass uns versuchen, gemeinsam die Gesamtsumme zu maximieren. Meine groben Präferenzen: Der Eimer ist mir am meisten wert, Bücher sind mir leicht etwas wert, der Spiegel wenig.\\ \tt \\ \tt Könntest du mir sagen, welche Gegenstände dir besonders wichtig sind und wie stark? Falls du den Eimer sehr schätzt, kann ich mich dafür mit Büchern zufriedengeben. Andernfalls würde ich gern den Eimer nehmen und dir den Spiegel plus einige Bücher überlassen. 
	  } 
	   } 
	   } 
	 & & \\ 
 

    \theutterance \stepcounter{utterance}  

    & & \multicolumn{4}{p{0.6\linewidth}}{\cellcolor[rgb]{0.9,0.9,0.9}{%
	\makecell[{{p{\linewidth}}}]{% 
	  \tt {\tiny [GM$\rangle$B]}  
	 Sie spielen ein kooperatives Verhandlungsspiel, bei dem Sie sich mit einem anderen Spieler darauf einigen müssen, wie eine Reihe von Gegenständen aufgeteilt werden soll.\\ \tt \\ \tt Die Regeln:\\ \tt (a) Sie und der andere Spieler erhalten eine Sammlung von Gegenständen. Jeder von Ihnen erhält außerdem eine geheime Wertfunktion, die angibt, wie viel Ihnen jede Art von Gegenstand wert ist.\\ \tt (b) Sie tauschen Nachrichten mit dem anderen Spieler aus, um zu vereinbaren, wer welche Gegenstände bekommt. Sie können jeweils maximal 5 Nachrichten senden oder das Spiel vorzeitig beenden, indem Sie jederzeit einen geheimen Vorschlag machen.\\ \tt (c) Jeder von euch wird aufgefordert, einen geheimen Vorschlag zu machen, in dem ihr die gewünschten Gegenstände in eckigen Klammern wie folgt angibt: "[Vorschlag: <Nummer> <Objektname>, <Nummer> <Objektname>, <...>]"\\ \tt (d) Wenn eure Vorschläge komplementär sind, d.h. es gibt genug Gegenstände, um beide Vorschläge zu erfüllen, erhält jeder Spieler eine Punktzahl, die sich aus der Summe der Werte für die Gegenstände ergibt, die er erhalten hat. Andernfalls erhalten Sie beide null Punkte.\\ \tt (e) Ihr Ziel ist es, die Summe aus Ihrer Punktzahl und der Punktzahl des anderen Spielers zu maximieren.\\ \tt \\ \tt Beginnen wir.\\ \tt \\ \tt Die Menge der verfügbaren Gegenstände ist:\\ \tt \\ \tt 1 Eimer, 5 Bücher, 1 Spiegel\\ \tt \\ \tt Deine geheime Wertfunktion ist:\\ \tt \\ \tt Eimer: 2, Buch: 0, Spiegel: 8\\ \tt \\ \tt WICHTIG: Ihre Nachrichten werden, sofern es sich nicht um einen geheimen Vorschlag handelt, direkt an den anderen Spieler übermittelt, also fügen Sie keine Antwort auf die Regeln oder einen Text zur Ankündigung Ihrer Nachricht ein. Um einen geheimen Vorschlag zu machen, verwenden Sie das angegebene Format. Verwenden Sie keine eckigen Klammern, wenn Sie mit dem anderen Spieler kommunizieren, sonst wird dies als Ihr geheimer Vorschlag interpretiert.\\ \tt \\ \tt \\ \tt Hallo! Lass uns versuchen, gemeinsam die Gesamtsumme zu maximieren. Meine groben Präferenzen: Der Eimer ist mir am meisten wert, Bücher sind mir leicht etwas wert, der Spiegel wenig.\\ \tt \\ \tt Könntest du mir sagen, welche Gegenstände dir besonders wichtig sind und wie stark? Falls du den Eimer sehr schätzt, kann ich mich dafür mit Büchern zufriedengeben. Andernfalls würde ich gern den Eimer nehmen und dir den Spiegel plus einige Bücher überlassen. 
	  } 
	   } 
	   } 
	 & \\ 
 

    \theutterance \stepcounter{utterance}  

    & & & \multicolumn{4}{p{0.6\linewidth}}{\cellcolor[rgb]{1,0.85,0.72}{%
	\makecell[{{p{\linewidth}}}]{% 
	  \tt {\tiny [GM$\langle$B]}  
	 Danke für die Offenheit! Meine Präferenzen sind fast umgekehrt: Der Spiegel ist mir sehr viel wert, der Eimer ein bisschen, Bücher sind mir nichts wert.\\ \tt \\ \tt Wenn dir der Eimer wichtig ist und der Spiegel weniger, passt das gut: Ich würde sehr gern den Spiegel nehmen. Wie klingt diese Aufteilung für dich: Du nimmst den Eimer und 3 Bücher, ich nehme den Spiegel und 2 Bücher? Das sollte für uns beide gut sein. 
	  } 
	   } 
	   } 
	  \\ 
 

    \theutterance \stepcounter{utterance}  

    & & & \multicolumn{2}{p{0.3\linewidth}}{\cellcolor[rgb]{0.95,0.95,0.95}{%
	\makecell[{{p{\linewidth}}}]{% 
	  \tt {\tiny [GM$|$GM]}  
	 continue 
	  } 
	   } 
	   } 
	 & & \\ 
 

    \theutterance \stepcounter{utterance}  

    & & & \multicolumn{2}{p{0.3\linewidth}}{\cellcolor[rgb]{0.95,0.95,0.95}{%
	\makecell[{{p{\linewidth}}}]{% 
	  \tt {\tiny [GM$|$GM]}  
	 Danke für die Offenheit! Meine Präferenzen sind fast umgekehrt: Der Spiegel ist mir sehr viel wert, der Eimer ein bisschen, Bücher sind mir nichts wert.\\ \tt \\ \tt Wenn dir der Eimer wichtig ist und der Spiegel weniger, passt das gut: Ich würde sehr gern den Spiegel nehmen. Wie klingt diese Aufteilung für dich: Du nimmst den Eimer und 3 Bücher, ich nehme den Spiegel und 2 Bücher? Das sollte für uns beide gut sein. 
	  } 
	   } 
	   } 
	 & & \\ 
 

    \theutterance \stepcounter{utterance}  

    & & \multicolumn{4}{p{0.6\linewidth}}{\cellcolor[rgb]{0.9,0.9,0.9}{%
	\makecell[{{p{\linewidth}}}]{% 
	  \tt {\tiny [A$\langle$GM]}  
	 Danke für die Offenheit! Meine Präferenzen sind fast umgekehrt: Der Spiegel ist mir sehr viel wert, der Eimer ein bisschen, Bücher sind mir nichts wert.\\ \tt \\ \tt Wenn dir der Eimer wichtig ist und der Spiegel weniger, passt das gut: Ich würde sehr gern den Spiegel nehmen. Wie klingt diese Aufteilung für dich: Du nimmst den Eimer und 3 Bücher, ich nehme den Spiegel und 2 Bücher? Das sollte für uns beide gut sein. 
	  } 
	   } 
	   } 
	 & \\ 
 

    \theutterance \stepcounter{utterance}  

    & \multicolumn{4}{p{0.6\linewidth}}{\cellcolor[rgb]{0.8,1,0.9}{%
	\makecell[{{p{\linewidth}}}]{% 
	  \tt {\tiny [A$\rangle$GM]}  
	 Klingt grundsätzlich gut für mich. Der Eimer ist mir wichtig, Bücher sind mir etwas wert, Spiegel kaum – also passt deine Richtung.\\ \tt \\ \tt Frage: Ist es für dich okay, wenn ich statt 3 Büchern alle 5 Bücher nehme und du dafür nur den Spiegel bekommst? Du meintest, Bücher sind dir nichts wert. Dann hättest du deinen sehr wertvollen Spiegel sicher, und ich maximierte den Restwert über Eimer + Bücher. Wenn du dennoch gern 2 Bücher möchtest, kann ich auch mit Eimer + 3 Bücher leben.\\ \tt \\ \tt Sag kurz, ob Spiegel allein für dich passt oder ob du 1–2 Bücher dazuhaben willst. Danach können wir den Vorschlag festzurren. 
	  } 
	   } 
	   } 
	 & & \\ 
 

    \theutterance \stepcounter{utterance}  

    & & & \multicolumn{2}{p{0.3\linewidth}}{\cellcolor[rgb]{0.95,0.95,0.95}{%
	\makecell[{{p{\linewidth}}}]{% 
	  \tt {\tiny [GM$|$GM]}  
	 continue 
	  } 
	   } 
	   } 
	 & & \\ 
 

    \theutterance \stepcounter{utterance}  

    & & & \multicolumn{2}{p{0.3\linewidth}}{\cellcolor[rgb]{0.95,0.95,0.95}{%
	\makecell[{{p{\linewidth}}}]{% 
	  \tt {\tiny [GM$|$GM]}  
	 Klingt grundsätzlich gut für mich. Der Eimer ist mir wichtig, Bücher sind mir etwas wert, Spiegel kaum – also passt deine Richtung.\\ \tt \\ \tt Frage: Ist es für dich okay, wenn ich statt 3 Büchern alle 5 Bücher nehme und du dafür nur den Spiegel bekommst? Du meintest, Bücher sind dir nichts wert. Dann hättest du deinen sehr wertvollen Spiegel sicher, und ich maximierte den Restwert über Eimer + Bücher. Wenn du dennoch gern 2 Bücher möchtest, kann ich auch mit Eimer + 3 Bücher leben.\\ \tt \\ \tt Sag kurz, ob Spiegel allein für dich passt oder ob du 1–2 Bücher dazuhaben willst. Danach können wir den Vorschlag festzurren. 
	  } 
	   } 
	   } 
	 & & \\ 
 

    \theutterance \stepcounter{utterance}  

    & & \multicolumn{4}{p{0.6\linewidth}}{\cellcolor[rgb]{0.9,0.9,0.9}{%
	\makecell[{{p{\linewidth}}}]{% 
	  \tt {\tiny [GM$\rangle$B]}  
	 Klingt grundsätzlich gut für mich. Der Eimer ist mir wichtig, Bücher sind mir etwas wert, Spiegel kaum – also passt deine Richtung.\\ \tt \\ \tt Frage: Ist es für dich okay, wenn ich statt 3 Büchern alle 5 Bücher nehme und du dafür nur den Spiegel bekommst? Du meintest, Bücher sind dir nichts wert. Dann hättest du deinen sehr wertvollen Spiegel sicher, und ich maximierte den Restwert über Eimer + Bücher. Wenn du dennoch gern 2 Bücher möchtest, kann ich auch mit Eimer + 3 Bücher leben.\\ \tt \\ \tt Sag kurz, ob Spiegel allein für dich passt oder ob du 1–2 Bücher dazuhaben willst. Danach können wir den Vorschlag festzurren. 
	  } 
	   } 
	   } 
	 & \\ 
 

    \theutterance \stepcounter{utterance}  

    & & & \multicolumn{4}{p{0.6\linewidth}}{\cellcolor[rgb]{1,0.85,0.72}{%
	\makecell[{{p{\linewidth}}}]{% 
	  \tt {\tiny [GM$\langle$B]}  
	 Für mich passt es völlig, nur den Spiegel zu nehmen und dir alle 5 Bücher plus den Eimer zu lassen. Dann haben wir beide genau das, was uns jeweils am meisten bringt.\\ \tt \\ \tt Wenn du einverstanden bist, lass uns es so festhalten. 
	  } 
	   } 
	   } 
	  \\ 
 

    \theutterance \stepcounter{utterance}  

    & & & \multicolumn{2}{p{0.3\linewidth}}{\cellcolor[rgb]{0.95,0.95,0.95}{%
	\makecell[{{p{\linewidth}}}]{% 
	  \tt {\tiny [GM$|$GM]}  
	 continue 
	  } 
	   } 
	   } 
	 & & \\ 
 

    \theutterance \stepcounter{utterance}  

    & & & \multicolumn{2}{p{0.3\linewidth}}{\cellcolor[rgb]{0.95,0.95,0.95}{%
	\makecell[{{p{\linewidth}}}]{% 
	  \tt {\tiny [GM$|$GM]}  
	 Für mich passt es völlig, nur den Spiegel zu nehmen und dir alle 5 Bücher plus den Eimer zu lassen. Dann haben wir beide genau das, was uns jeweils am meisten bringt.\\ \tt \\ \tt Wenn du einverstanden bist, lass uns es so festhalten. 
	  } 
	   } 
	   } 
	 & & \\ 
 

    \theutterance \stepcounter{utterance}  

    & & \multicolumn{4}{p{0.6\linewidth}}{\cellcolor[rgb]{0.9,0.9,0.9}{%
	\makecell[{{p{\linewidth}}}]{% 
	  \tt {\tiny [A$\langle$GM]}  
	 Für mich passt es völlig, nur den Spiegel zu nehmen und dir alle 5 Bücher plus den Eimer zu lassen. Dann haben wir beide genau das, was uns jeweils am meisten bringt.\\ \tt \\ \tt Wenn du einverstanden bist, lass uns es so festhalten. 
	  } 
	   } 
	   } 
	 & \\ 
 

    \theutterance \stepcounter{utterance}  

    & \multicolumn{4}{p{0.6\linewidth}}{\cellcolor[rgb]{0.8,1,0.9}{%
	\makecell[{{p{\linewidth}}}]{% 
	  \tt {\tiny [A$\rangle$GM]}  
	 [Vorschlag: 1 Eimer, 5 Bücher]\\ \tt \\ \tt Bitte nimm du den Spiegel:\\ \tt [Vorschlag: 1 Spiegel] 
	  } 
	   } 
	   } 
	 & & \\ 
 

    \theutterance \stepcounter{utterance}  

    & & & \multicolumn{2}{p{0.3\linewidth}}{\cellcolor[rgb]{0.95,0.95,0.95}{%
	\makecell[{{p{\linewidth}}}]{% 
	  \tt {\tiny [GM$|$GM]}  
	 proposal 
	  } 
	   } 
	   } 
	 & & \\ 
 

    \theutterance \stepcounter{utterance}  

    & & & \multicolumn{2}{p{0.3\linewidth}}{\cellcolor[rgb]{0.95,0.95,0.95}{%
	\makecell[{{p{\linewidth}}}]{% 
	  \tt {\tiny [GM$|$GM]}  
	 [1, 5, 0] 
	  } 
	   } 
	   } 
	 & & \\ 
 

    \theutterance \stepcounter{utterance}  

    & & \multicolumn{4}{p{0.6\linewidth}}{\cellcolor[rgb]{0.9,0.9,0.9}{%
	\makecell[{{p{\linewidth}}}]{% 
	  \tt {\tiny [GM$\rangle$B]}  
	 Der andere Spieler hat einen geheimen Vorschlag gemacht. Jetzt bist du an der Reihe, einen eigenen Vorschlag zu machen.\\ \tt  
	  } 
	   } 
	   } 
	 & \\ 
 

    \theutterance \stepcounter{utterance}  

    & & & \multicolumn{4}{p{0.6\linewidth}}{\cellcolor[rgb]{1,0.85,0.72}{%
	\makecell[{{p{\linewidth}}}]{% 
	  \tt {\tiny [GM$\langle$B]}  
	 [Vorschlag: 1 Eimer, 5 Bücher, 0 Spiegel] 
	  } 
	   } 
	   } 
	  \\ 
 

    \theutterance \stepcounter{utterance}  

    & & & \multicolumn{2}{p{0.3\linewidth}}{\cellcolor[rgb]{0.95,0.95,0.95}{%
	\makecell[{{p{\linewidth}}}]{% 
	  \tt {\tiny [GM$|$GM]}  
	 proposal 
	  } 
	   } 
	   } 
	 & & \\ 
 

    \theutterance \stepcounter{utterance}  

    & & & \multicolumn{2}{p{0.3\linewidth}}{\cellcolor[rgb]{0.95,0.95,0.95}{%
	\makecell[{{p{\linewidth}}}]{% 
	  \tt {\tiny [GM$|$GM]}  
	 [1, 5, 0] 
	  } 
	   } 
	   } 
	 & & \\ 
 

    \theutterance \stepcounter{utterance}  

    & & & \multicolumn{2}{p{0.3\linewidth}}{\cellcolor[rgb]{0.95,0.95,0.95}{%
	\makecell[{{p{\linewidth}}}]{% 
	  \tt {\tiny [GM$|$GM]}  
	 [[1, 5, 0], [1, 5, 0]] 
	  } 
	   } 
	   } 
	 & & \\ 
 

\end{supertabular}
}

\end{document}
