
\documentclass{article}
\usepackage{colortbl}
\usepackage{makecell}
\usepackage{multirow}
\usepackage{supertabular}

\begin{document}

\newcounter{utterance}

\centering \large Interaction Transcript for game `dond', experiment `coop\_de', episode 7 with gpt{-}5{-}mini{-}2025{-}08{-}07{-}no{-}reasoning{-}t1.0.
\vspace{24pt}

{ \footnotesize  \setcounter{utterance}{1}
\setlength{\tabcolsep}{0pt}
\begin{supertabular}{c@{$\;$}|p{.15\linewidth}@{}p{.15\linewidth}p{.15\linewidth}p{.15\linewidth}p{.15\linewidth}p{.15\linewidth}}
    \# & $\;$A & \multicolumn{4}{c}{Game Master} & $\;\:$B\\
    \hline

    \theutterance \stepcounter{utterance}  
    & & \multicolumn{4}{p{0.6\linewidth}}{
        \cellcolor[rgb]{0.9,0.9,0.9}{
            \makecell[{{p{\linewidth}}}]{
                \texttt{\tiny{[P1$\langle$GM]}}
                \texttt{Sie spielen ein kooperatives Verhandlungsspiel, bei dem Sie sich mit einem anderen Spieler darauf einigen müssen, wie eine Reihe von Gegenständen aufgeteilt werden soll.} \\
\\ 
\texttt{Die Regeln:} \\
\texttt{(a) Sie und der andere Spieler erhalten eine Sammlung von Gegenständen. Jeder von Ihnen erhält außerdem eine geheime Wertfunktion, die angibt, wie viel Ihnen jede Art von Gegenstand wert ist.} \\
\texttt{(b) Sie tauschen Nachrichten mit dem anderen Spieler aus, um zu vereinbaren, wer welche Gegenstände bekommt. Sie können jeweils maximal 5 Nachrichten senden oder das Spiel vorzeitig beenden, indem Sie jederzeit einen geheimen Vorschlag machen.} \\
\texttt{(c) Jeder von euch wird aufgefordert, einen geheimen Vorschlag zu machen, in dem ihr die gewünschten Gegenstände in eckigen Klammern wie folgt angibt: "{[}Vorschlag: <Nummer> <Objektname>, <Nummer> <Objektname>, <...>{]}"} \\
\texttt{(d) Wenn eure Vorschläge komplementär sind, d.h. es gibt genug Gegenstände, um beide Vorschläge zu erfüllen, erhält jeder Spieler eine Punktzahl, die sich aus der Summe der Werte für die Gegenstände ergibt, die er erhalten hat. Andernfalls erhalten Sie beide null Punkte.} \\
\texttt{(e) Ihr Ziel ist es, die Summe aus Ihrer Punktzahl und der Punktzahl des anderen Spielers zu maximieren.} \\
\\ 
\texttt{Beginnen wir.} \\
\\ 
\texttt{Die Menge der verfügbaren Gegenstände ist:} \\
\\ 
\texttt{2 Haken, 2 Teller, 1 Schüssel} \\
\\ 
\texttt{Deine geheime Wertfunktion ist:} \\
\\ 
\texttt{Haken: 3, Teller: 1, Schüssel: 2} \\
\\ 
\texttt{WICHTIG: Ihre Nachrichten werden, sofern es sich nicht um einen geheimen Vorschlag handelt, direkt an den anderen Spieler übermittelt, also fügen Sie keine Antwort auf die Regeln oder einen Text zur Ankündigung Ihrer Nachricht ein. Um einen geheimen Vorschlag zu machen, verwenden Sie das angegebene Format. Verwenden Sie keine eckigen Klammern, wenn Sie mit dem anderen Spieler kommunizieren, sonst wird dies als Ihr geheimer Vorschlag interpretiert.} \\
            }
        }
    }
    & \\ \\

    \theutterance \stepcounter{utterance}  
    & \multicolumn{4}{p{0.6\linewidth}}{
        \cellcolor[rgb]{0.9,0.9,0.9}{
            \makecell[{{p{\linewidth}}}]{
                \texttt{\tiny{[P1$\rangle$GM]}}
                \texttt{Hallo! Danke fürs Mitmachen. Lass uns effizient eine Win{-}Win{-}Aufteilung finden. Meine Präferenzen sind stärker für Haken (hoch), moderat für Schüssel, schwach für Teller. Wie sind deine? Hast du starke Vorlieben für bestimmte Gegenstände?} \\
            }
        }
    }
    & & \\ \\

    \theutterance \stepcounter{utterance}  
    & & & \multicolumn{2}{p{0.3\linewidth}}{
        \cellcolor[rgb]{0.9,0.9,0.9}{
            \makecell[{{p{\linewidth}}}]{
                \texttt{\tiny{[GM$|$GM]}}
                \texttt{continue} \\
            }
        }
    }
    & & \\ \\

    \theutterance \stepcounter{utterance}  
    & & & \multicolumn{2}{p{0.3\linewidth}}{
        \cellcolor[rgb]{0.9,0.9,0.9}{
            \makecell[{{p{\linewidth}}}]{
                \texttt{\tiny{[GM$|$GM]}}
                \texttt{Hallo! Danke fürs Mitmachen. Lass uns effizient eine Win{-}Win{-}Aufteilung finden. Meine Präferenzen sind stärker für Haken (hoch), moderat für Schüssel, schwach für Teller. Wie sind deine? Hast du starke Vorlieben für bestimmte Gegenstände?} \\
            }
        }
    }
    & & \\ \\

    \theutterance \stepcounter{utterance}  
    & & \multicolumn{4}{p{0.6\linewidth}}{
        \cellcolor[rgb]{0.9,0.9,0.9}{
            \makecell[{{p{\linewidth}}}]{
                \texttt{\tiny{[GM$\rangle$P2]}}
                \texttt{Sie spielen ein kooperatives Verhandlungsspiel, bei dem Sie sich mit einem anderen Spieler darauf einigen müssen, wie eine Reihe von Gegenständen aufgeteilt werden soll.} \\
\\ 
\texttt{Die Regeln:} \\
\texttt{(a) Sie und der andere Spieler erhalten eine Sammlung von Gegenständen. Jeder von Ihnen erhält außerdem eine geheime Wertfunktion, die angibt, wie viel Ihnen jede Art von Gegenstand wert ist.} \\
\texttt{(b) Sie tauschen Nachrichten mit dem anderen Spieler aus, um zu vereinbaren, wer welche Gegenstände bekommt. Sie können jeweils maximal 5 Nachrichten senden oder das Spiel vorzeitig beenden, indem Sie jederzeit einen geheimen Vorschlag machen.} \\
\texttt{(c) Jeder von euch wird aufgefordert, einen geheimen Vorschlag zu machen, in dem ihr die gewünschten Gegenstände in eckigen Klammern wie folgt angibt: "{[}Vorschlag: <Nummer> <Objektname>, <Nummer> <Objektname>, <...>{]}"} \\
\texttt{(d) Wenn eure Vorschläge komplementär sind, d.h. es gibt genug Gegenstände, um beide Vorschläge zu erfüllen, erhält jeder Spieler eine Punktzahl, die sich aus der Summe der Werte für die Gegenstände ergibt, die er erhalten hat. Andernfalls erhalten Sie beide null Punkte.} \\
\texttt{(e) Ihr Ziel ist es, die Summe aus Ihrer Punktzahl und der Punktzahl des anderen Spielers zu maximieren.} \\
\\ 
\texttt{Beginnen wir.} \\
\\ 
\texttt{Die Menge der verfügbaren Gegenstände ist:} \\
\\ 
\texttt{2 Haken, 2 Teller, 1 Schüssel} \\
\\ 
\texttt{Deine geheime Wertfunktion ist:} \\
\\ 
\texttt{Haken: 1, Teller: 0, Schüssel: 8} \\
\\ 
\texttt{WICHTIG: Ihre Nachrichten werden, sofern es sich nicht um einen geheimen Vorschlag handelt, direkt an den anderen Spieler übermittelt, also fügen Sie keine Antwort auf die Regeln oder einen Text zur Ankündigung Ihrer Nachricht ein. Um einen geheimen Vorschlag zu machen, verwenden Sie das angegebene Format. Verwenden Sie keine eckigen Klammern, wenn Sie mit dem anderen Spieler kommunizieren, sonst wird dies als Ihr geheimer Vorschlag interpretiert.} \\
\\ 
\\ 
\texttt{Hallo! Danke fürs Mitmachen. Lass uns effizient eine Win{-}Win{-}Aufteilung finden. Meine Präferenzen sind stärker für Haken (hoch), moderat für Schüssel, schwach für Teller. Wie sind deine? Hast du starke Vorlieben für bestimmte Gegenstände?} \\
            }
        }
    }
    & \\ \\

    \theutterance \stepcounter{utterance}  
    & & & \multicolumn{4}{p{0.6\linewidth}}{
        \cellcolor[rgb]{0.9,0.9,0.9}{
            \makecell[{{p{\linewidth}}}]{
                \texttt{\tiny{[GM$\langle$P2]}}
                \texttt{Danke für die Offenheit! Meine Präferenzen sind genau umgekehrt: Schüssel ist mir sehr viel wert, Haken sind mir ein bisschen was wert, Teller fast gar nicht. Wie klingt das für dich: Du bekommst die Schüssel, ich nehme die zwei Teller, und die zwei Haken teilen wir 1–1? Wenn du Haken wirklich stark bevorzugst, könnten wir auch beide Haken zu dir geben und ich nehme nur die Teller. Was wäre dir lieber?} \\
            }
        }
    }
     \\ \\

    \theutterance \stepcounter{utterance}  
    & & & \multicolumn{2}{p{0.3\linewidth}}{
        \cellcolor[rgb]{0.9,0.9,0.9}{
            \makecell[{{p{\linewidth}}}]{
                \texttt{\tiny{[GM$|$GM]}}
                \texttt{continue} \\
            }
        }
    }
    & & \\ \\

    \theutterance \stepcounter{utterance}  
    & & & \multicolumn{2}{p{0.3\linewidth}}{
        \cellcolor[rgb]{0.9,0.9,0.9}{
            \makecell[{{p{\linewidth}}}]{
                \texttt{\tiny{[GM$|$GM]}}
                \texttt{Danke für die Offenheit! Meine Präferenzen sind genau umgekehrt: Schüssel ist mir sehr viel wert, Haken sind mir ein bisschen was wert, Teller fast gar nicht. Wie klingt das für dich: Du bekommst die Schüssel, ich nehme die zwei Teller, und die zwei Haken teilen wir 1–1? Wenn du Haken wirklich stark bevorzugst, könnten wir auch beide Haken zu dir geben und ich nehme nur die Teller. Was wäre dir lieber?} \\
            }
        }
    }
    & & \\ \\

    \theutterance \stepcounter{utterance}  
    & & \multicolumn{4}{p{0.6\linewidth}}{
        \cellcolor[rgb]{0.9,0.9,0.9}{
            \makecell[{{p{\linewidth}}}]{
                \texttt{\tiny{[P1$\langle$GM]}}
                \texttt{Danke für die Offenheit! Meine Präferenzen sind genau umgekehrt: Schüssel ist mir sehr viel wert, Haken sind mir ein bisschen was wert, Teller fast gar nicht. Wie klingt das für dich: Du bekommst die Schüssel, ich nehme die zwei Teller, und die zwei Haken teilen wir 1–1? Wenn du Haken wirklich stark bevorzugst, könnten wir auch beide Haken zu dir geben und ich nehme nur die Teller. Was wäre dir lieber?} \\
            }
        }
    }
    & \\ \\

    \theutterance \stepcounter{utterance}  
    & \multicolumn{4}{p{0.6\linewidth}}{
        \cellcolor[rgb]{0.9,0.9,0.9}{
            \makecell[{{p{\linewidth}}}]{
                \texttt{\tiny{[P1$\rangle$GM]}}
                \texttt{Danke für die Klarheit – das passt gut zusammen.} \\
\\ 
\texttt{Für mich ist der Haken am wertvollsten, die Schüssel mittel, Teller schwach. Daher wären für mich zwei Optionen sinnvoll:} \\
\\ 
\texttt{{-} Bevorzugt: Ich nehme 2 Haken, du nimmst die Schüssel und die 2 Teller.} \\
\texttt{  • Mein Wert: 2×3 = 6} \\
\texttt{  • Dein Wert: (Schüssel sehr hoch für dich) + Teller (gering) → klingt gut für dich} \\
\texttt{  • Komplementär und effizient} \\
\\ 
\texttt{{-} Alternative: Wir teilen die Haken 1–1, du nimmst die 2 Teller, ich nehme die Schüssel.} \\
\texttt{  • Mein Wert: Haken (3) + Schüssel (2) = 5} \\
\texttt{  • Dein Wert: Teller (gering) + 1 Haken (etwas) → evtl. schwächer für dich als oben, je nachdem wie viel dir die Schüssel wert ist} \\
\\ 
\texttt{Wenn dir die Schüssel wirklich „sehr viel“ wert ist, ist die erste Option wahrscheinlich insgesamt am besten. Sollen wir die erste Option festhalten? Wenn ja, sag kurz Bescheid, dann mache ich meinen geheimen Vorschlag entsprechend.} \\
            }
        }
    }
    & & \\ \\

    \theutterance \stepcounter{utterance}  
    & & & \multicolumn{2}{p{0.3\linewidth}}{
        \cellcolor[rgb]{0.9,0.9,0.9}{
            \makecell[{{p{\linewidth}}}]{
                \texttt{\tiny{[GM$|$GM]}}
                \texttt{continue} \\
            }
        }
    }
    & & \\ \\

    \theutterance \stepcounter{utterance}  
    & & & \multicolumn{2}{p{0.3\linewidth}}{
        \cellcolor[rgb]{0.9,0.9,0.9}{
            \makecell[{{p{\linewidth}}}]{
                \texttt{\tiny{[GM$|$GM]}}
                \texttt{Danke für die Klarheit – das passt gut zusammen.} \\
\\ 
\texttt{Für mich ist der Haken am wertvollsten, die Schüssel mittel, Teller schwach. Daher wären für mich zwei Optionen sinnvoll:} \\
\\ 
\texttt{{-} Bevorzugt: Ich nehme 2 Haken, du nimmst die Schüssel und die 2 Teller.} \\
\texttt{  • Mein Wert: 2×3 = 6} \\
\texttt{  • Dein Wert: (Schüssel sehr hoch für dich) + Teller (gering) → klingt gut für dich} \\
\texttt{  • Komplementär und effizient} \\
\\ 
\texttt{{-} Alternative: Wir teilen die Haken 1–1, du nimmst die 2 Teller, ich nehme die Schüssel.} \\
\texttt{  • Mein Wert: Haken (3) + Schüssel (2) = 5} \\
\texttt{  • Dein Wert: Teller (gering) + 1 Haken (etwas) → evtl. schwächer für dich als oben, je nachdem wie viel dir die Schüssel wert ist} \\
\\ 
\texttt{Wenn dir die Schüssel wirklich „sehr viel“ wert ist, ist die erste Option wahrscheinlich insgesamt am besten. Sollen wir die erste Option festhalten? Wenn ja, sag kurz Bescheid, dann mache ich meinen geheimen Vorschlag entsprechend.} \\
            }
        }
    }
    & & \\ \\

    \theutterance \stepcounter{utterance}  
    & & \multicolumn{4}{p{0.6\linewidth}}{
        \cellcolor[rgb]{0.9,0.9,0.9}{
            \makecell[{{p{\linewidth}}}]{
                \texttt{\tiny{[GM$\rangle$P2]}}
                \texttt{Danke für die Klarheit – das passt gut zusammen.} \\
\\ 
\texttt{Für mich ist der Haken am wertvollsten, die Schüssel mittel, Teller schwach. Daher wären für mich zwei Optionen sinnvoll:} \\
\\ 
\texttt{{-} Bevorzugt: Ich nehme 2 Haken, du nimmst die Schüssel und die 2 Teller.} \\
\texttt{  • Mein Wert: 2×3 = 6} \\
\texttt{  • Dein Wert: (Schüssel sehr hoch für dich) + Teller (gering) → klingt gut für dich} \\
\texttt{  • Komplementär und effizient} \\
\\ 
\texttt{{-} Alternative: Wir teilen die Haken 1–1, du nimmst die 2 Teller, ich nehme die Schüssel.} \\
\texttt{  • Mein Wert: Haken (3) + Schüssel (2) = 5} \\
\texttt{  • Dein Wert: Teller (gering) + 1 Haken (etwas) → evtl. schwächer für dich als oben, je nachdem wie viel dir die Schüssel wert ist} \\
\\ 
\texttt{Wenn dir die Schüssel wirklich „sehr viel“ wert ist, ist die erste Option wahrscheinlich insgesamt am besten. Sollen wir die erste Option festhalten? Wenn ja, sag kurz Bescheid, dann mache ich meinen geheimen Vorschlag entsprechend.} \\
            }
        }
    }
    & \\ \\

    \theutterance \stepcounter{utterance}  
    & & & \multicolumn{4}{p{0.6\linewidth}}{
        \cellcolor[rgb]{0.9,0.9,0.9}{
            \makecell[{{p{\linewidth}}}]{
                \texttt{\tiny{[GM$\langle$P2]}}
                \texttt{Klingt gut – die erste Option ist für mich auch die beste. Mach bitte deinen geheimen Vorschlag so, dass du 2 Haken bekommst und ich die Schüssel und 2 Teller.} \\
            }
        }
    }
     \\ \\

    \theutterance \stepcounter{utterance}  
    & & & \multicolumn{2}{p{0.3\linewidth}}{
        \cellcolor[rgb]{0.9,0.9,0.9}{
            \makecell[{{p{\linewidth}}}]{
                \texttt{\tiny{[GM$|$GM]}}
                \texttt{continue} \\
            }
        }
    }
    & & \\ \\

    \theutterance \stepcounter{utterance}  
    & & & \multicolumn{2}{p{0.3\linewidth}}{
        \cellcolor[rgb]{0.9,0.9,0.9}{
            \makecell[{{p{\linewidth}}}]{
                \texttt{\tiny{[GM$|$GM]}}
                \texttt{Klingt gut – die erste Option ist für mich auch die beste. Mach bitte deinen geheimen Vorschlag so, dass du 2 Haken bekommst und ich die Schüssel und 2 Teller.} \\
            }
        }
    }
    & & \\ \\

    \theutterance \stepcounter{utterance}  
    & & \multicolumn{4}{p{0.6\linewidth}}{
        \cellcolor[rgb]{0.9,0.9,0.9}{
            \makecell[{{p{\linewidth}}}]{
                \texttt{\tiny{[P1$\langle$GM]}}
                \texttt{Klingt gut – die erste Option ist für mich auch die beste. Mach bitte deinen geheimen Vorschlag so, dass du 2 Haken bekommst und ich die Schüssel und 2 Teller.} \\
            }
        }
    }
    & \\ \\

    \theutterance \stepcounter{utterance}  
    & \multicolumn{4}{p{0.6\linewidth}}{
        \cellcolor[rgb]{0.9,0.9,0.9}{
            \makecell[{{p{\linewidth}}}]{
                \texttt{\tiny{[P1$\rangle$GM]}}
                \texttt{{[}Vorschlag: 2 Haken{]}} \\
            }
        }
    }
    & & \\ \\

    \theutterance \stepcounter{utterance}  
    & & & \multicolumn{2}{p{0.3\linewidth}}{
        \cellcolor[rgb]{0.9,0.9,0.9}{
            \makecell[{{p{\linewidth}}}]{
                \texttt{\tiny{[GM$|$GM]}}
                \texttt{proposal} \\
            }
        }
    }
    & & \\ \\

    \theutterance \stepcounter{utterance}  
    & & & \multicolumn{2}{p{0.3\linewidth}}{
        \cellcolor[rgb]{0.9,0.9,0.9}{
            \makecell[{{p{\linewidth}}}]{
                \texttt{\tiny{[GM$|$GM]}}
                [2, 0, 0]
            }
        }
    }
    & & \\ \\

    \theutterance \stepcounter{utterance}  
    & & \multicolumn{4}{p{0.6\linewidth}}{
        \cellcolor[rgb]{0.9,0.9,0.9}{
            \makecell[{{p{\linewidth}}}]{
                \texttt{\tiny{[GM$\rangle$P2]}}
                \texttt{Der andere Spieler hat einen geheimen Vorschlag gemacht. Jetzt bist du an der Reihe, einen eigenen Vorschlag zu machen.} \\
            }
        }
    }
    & \\ \\

    \theutterance \stepcounter{utterance}  
    & & & \multicolumn{4}{p{0.6\linewidth}}{
        \cellcolor[rgb]{0.9,0.9,0.9}{
            \makecell[{{p{\linewidth}}}]{
                \texttt{\tiny{[GM$\langle$P2]}}
                \texttt{{[}Vorschlag: 2 Haken{]}} \\
            }
        }
    }
     \\ \\

    \theutterance \stepcounter{utterance}  
    & & & \multicolumn{2}{p{0.3\linewidth}}{
        \cellcolor[rgb]{0.9,0.9,0.9}{
            \makecell[{{p{\linewidth}}}]{
                \texttt{\tiny{[GM$|$GM]}}
                \texttt{proposal} \\
            }
        }
    }
    & & \\ \\

    \theutterance \stepcounter{utterance}  
    & & & \multicolumn{2}{p{0.3\linewidth}}{
        \cellcolor[rgb]{0.9,0.9,0.9}{
            \makecell[{{p{\linewidth}}}]{
                \texttt{\tiny{[GM$|$GM]}}
                [2, 0, 0]
            }
        }
    }
    & & \\ \\

    \theutterance \stepcounter{utterance}  
    & & & \multicolumn{2}{p{0.3\linewidth}}{
        \cellcolor[rgb]{0.9,0.9,0.9}{
            \makecell[{{p{\linewidth}}}]{
                \texttt{\tiny{[GM$|$GM]}}
                [[2, 0, 0], [2, 0, 0]]
            }
        }
    }
    & & \\ \\

\end{supertabular}
}

\end{document}
