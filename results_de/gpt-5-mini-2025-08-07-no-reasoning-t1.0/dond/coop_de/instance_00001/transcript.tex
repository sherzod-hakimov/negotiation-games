
\documentclass{article}
\usepackage{colortbl}
\usepackage{makecell}
\usepackage{multirow}
\usepackage{supertabular}

\begin{document}

\newcounter{utterance}

\centering \large Interaction Transcript for game `dond', experiment `coop\_de', episode 1 with gpt{-}5{-}mini{-}2025{-}08{-}07{-}no{-}reasoning{-}t1.0.
\vspace{24pt}

{ \footnotesize  \setcounter{utterance}{1}
\setlength{\tabcolsep}{0pt}
\begin{supertabular}{c@{$\;$}|p{.15\linewidth}@{}p{.15\linewidth}p{.15\linewidth}p{.15\linewidth}p{.15\linewidth}p{.15\linewidth}}
    \# & $\;$A & \multicolumn{4}{c}{Game Master} & $\;\:$B\\
    \hline

    \theutterance \stepcounter{utterance}  
    & & \multicolumn{4}{p{0.6\linewidth}}{
        \cellcolor[rgb]{0.9,0.9,0.9}{
            \makecell[{{p{\linewidth}}}]{
                \texttt{\tiny{[P1$\langle$GM]}}
                \texttt{Sie spielen ein kooperatives Verhandlungsspiel, bei dem Sie sich mit einem anderen Spieler darauf einigen müssen, wie eine Reihe von Gegenständen aufgeteilt werden soll.} \\
\\ 
\texttt{Die Regeln:} \\
\texttt{(a) Sie und der andere Spieler erhalten eine Sammlung von Gegenständen. Jeder von Ihnen erhält außerdem eine geheime Wertfunktion, die angibt, wie viel Ihnen jede Art von Gegenstand wert ist.} \\
\texttt{(b) Sie tauschen Nachrichten mit dem anderen Spieler aus, um zu vereinbaren, wer welche Gegenstände bekommt. Sie können jeweils maximal 5 Nachrichten senden oder das Spiel vorzeitig beenden, indem Sie jederzeit einen geheimen Vorschlag machen.} \\
\texttt{(c) Jeder von euch wird aufgefordert, einen geheimen Vorschlag zu machen, in dem ihr die gewünschten Gegenstände in eckigen Klammern wie folgt angibt: "{[}Vorschlag: <Nummer> <Objektname>, <Nummer> <Objektname>, <...>{]}"} \\
\texttt{(d) Wenn eure Vorschläge komplementär sind, d.h. es gibt genug Gegenstände, um beide Vorschläge zu erfüllen, erhält jeder Spieler eine Punktzahl, die sich aus der Summe der Werte für die Gegenstände ergibt, die er erhalten hat. Andernfalls erhalten Sie beide null Punkte.} \\
\texttt{(e) Ihr Ziel ist es, die Summe aus Ihrer Punktzahl und der Punktzahl des anderen Spielers zu maximieren.} \\
\\ 
\texttt{Beginnen wir.} \\
\\ 
\texttt{Die Menge der verfügbaren Gegenstände ist:} \\
\\ 
\texttt{2 Glocken, 1 Puppe, 2 Fäden} \\
\\ 
\texttt{Deine geheime Wertfunktion ist:} \\
\\ 
\texttt{Glocke: 1, Puppe: 2, Faden: 3} \\
\\ 
\texttt{WICHTIG: Ihre Nachrichten werden, sofern es sich nicht um einen geheimen Vorschlag handelt, direkt an den anderen Spieler übermittelt, also fügen Sie keine Antwort auf die Regeln oder einen Text zur Ankündigung Ihrer Nachricht ein. Um einen geheimen Vorschlag zu machen, verwenden Sie das angegebene Format. Verwenden Sie keine eckigen Klammern, wenn Sie mit dem anderen Spieler kommunizieren, sonst wird dies als Ihr geheimer Vorschlag interpretiert.} \\
            }
        }
    }
    & \\ \\

    \theutterance \stepcounter{utterance}  
    & \multicolumn{4}{p{0.6\linewidth}}{
        \cellcolor[rgb]{0.9,0.9,0.9}{
            \makecell[{{p{\linewidth}}}]{
                \texttt{\tiny{[P1$\rangle$GM]}}
                \texttt{Hallo! Lass uns effizient zu einer Einigung kommen. Meine Präferenzen sind: Faden ist mir am wichtigsten, dann die Puppe, dann Glocken. Wie sehen deine Werte ungefähr aus? Wenn Faden für dich wenig wert ist, könnte ich die Fäden nehmen und dir die höherwertigen Sachen für dich überlassen.} \\
            }
        }
    }
    & & \\ \\

    \theutterance \stepcounter{utterance}  
    & & & \multicolumn{2}{p{0.3\linewidth}}{
        \cellcolor[rgb]{0.9,0.9,0.9}{
            \makecell[{{p{\linewidth}}}]{
                \texttt{\tiny{[GM$|$GM]}}
                \texttt{continue} \\
            }
        }
    }
    & & \\ \\

    \theutterance \stepcounter{utterance}  
    & & & \multicolumn{2}{p{0.3\linewidth}}{
        \cellcolor[rgb]{0.9,0.9,0.9}{
            \makecell[{{p{\linewidth}}}]{
                \texttt{\tiny{[GM$|$GM]}}
                \texttt{Hallo! Lass uns effizient zu einer Einigung kommen. Meine Präferenzen sind: Faden ist mir am wichtigsten, dann die Puppe, dann Glocken. Wie sehen deine Werte ungefähr aus? Wenn Faden für dich wenig wert ist, könnte ich die Fäden nehmen und dir die höherwertigen Sachen für dich überlassen.} \\
            }
        }
    }
    & & \\ \\

    \theutterance \stepcounter{utterance}  
    & & \multicolumn{4}{p{0.6\linewidth}}{
        \cellcolor[rgb]{0.9,0.9,0.9}{
            \makecell[{{p{\linewidth}}}]{
                \texttt{\tiny{[GM$\rangle$P2]}}
                \texttt{Sie spielen ein kooperatives Verhandlungsspiel, bei dem Sie sich mit einem anderen Spieler darauf einigen müssen, wie eine Reihe von Gegenständen aufgeteilt werden soll.} \\
\\ 
\texttt{Die Regeln:} \\
\texttt{(a) Sie und der andere Spieler erhalten eine Sammlung von Gegenständen. Jeder von Ihnen erhält außerdem eine geheime Wertfunktion, die angibt, wie viel Ihnen jede Art von Gegenstand wert ist.} \\
\texttt{(b) Sie tauschen Nachrichten mit dem anderen Spieler aus, um zu vereinbaren, wer welche Gegenstände bekommt. Sie können jeweils maximal 5 Nachrichten senden oder das Spiel vorzeitig beenden, indem Sie jederzeit einen geheimen Vorschlag machen.} \\
\texttt{(c) Jeder von euch wird aufgefordert, einen geheimen Vorschlag zu machen, in dem ihr die gewünschten Gegenstände in eckigen Klammern wie folgt angibt: "{[}Vorschlag: <Nummer> <Objektname>, <Nummer> <Objektname>, <...>{]}"} \\
\texttt{(d) Wenn eure Vorschläge komplementär sind, d.h. es gibt genug Gegenstände, um beide Vorschläge zu erfüllen, erhält jeder Spieler eine Punktzahl, die sich aus der Summe der Werte für die Gegenstände ergibt, die er erhalten hat. Andernfalls erhalten Sie beide null Punkte.} \\
\texttt{(e) Ihr Ziel ist es, die Summe aus Ihrer Punktzahl und der Punktzahl des anderen Spielers zu maximieren.} \\
\\ 
\texttt{Beginnen wir.} \\
\\ 
\texttt{Die Menge der verfügbaren Gegenstände ist:} \\
\\ 
\texttt{2 Glocken, 1 Puppe, 2 Fäden} \\
\\ 
\texttt{Deine geheime Wertfunktion ist:} \\
\\ 
\texttt{Glocke: 0, Puppe: 2, Faden: 4} \\
\\ 
\texttt{WICHTIG: Ihre Nachrichten werden, sofern es sich nicht um einen geheimen Vorschlag handelt, direkt an den anderen Spieler übermittelt, also fügen Sie keine Antwort auf die Regeln oder einen Text zur Ankündigung Ihrer Nachricht ein. Um einen geheimen Vorschlag zu machen, verwenden Sie das angegebene Format. Verwenden Sie keine eckigen Klammern, wenn Sie mit dem anderen Spieler kommunizieren, sonst wird dies als Ihr geheimer Vorschlag interpretiert.} \\
\\ 
\\ 
\texttt{Hallo! Lass uns effizient zu einer Einigung kommen. Meine Präferenzen sind: Faden ist mir am wichtigsten, dann die Puppe, dann Glocken. Wie sehen deine Werte ungefähr aus? Wenn Faden für dich wenig wert ist, könnte ich die Fäden nehmen und dir die höherwertigen Sachen für dich überlassen.} \\
            }
        }
    }
    & \\ \\

    \theutterance \stepcounter{utterance}  
    & & & \multicolumn{4}{p{0.6\linewidth}}{
        \cellcolor[rgb]{0.9,0.9,0.9}{
            \makecell[{{p{\linewidth}}}]{
                \texttt{\tiny{[GM$\langle$P2]}}
                \texttt{Klingt gut. Ich bin flexibel, aber bevor ich Werte teile: Wie wichtig sind dir Glocken und die Puppe im Vergleich zu Faden? Zum Beispiel: Würdest du lieber die Puppe als einen Faden nehmen, oder sind zwei Glocken für dich mehr wert als ein Faden? Wenn Faden für dich sehr wichtig ist, könnte ich eher Puppe/Glocken nehmen; wenn nicht, nehme ich gern die Fäden.} \\
            }
        }
    }
     \\ \\

    \theutterance \stepcounter{utterance}  
    & & & \multicolumn{2}{p{0.3\linewidth}}{
        \cellcolor[rgb]{0.9,0.9,0.9}{
            \makecell[{{p{\linewidth}}}]{
                \texttt{\tiny{[GM$|$GM]}}
                \texttt{continue} \\
            }
        }
    }
    & & \\ \\

    \theutterance \stepcounter{utterance}  
    & & & \multicolumn{2}{p{0.3\linewidth}}{
        \cellcolor[rgb]{0.9,0.9,0.9}{
            \makecell[{{p{\linewidth}}}]{
                \texttt{\tiny{[GM$|$GM]}}
                \texttt{Klingt gut. Ich bin flexibel, aber bevor ich Werte teile: Wie wichtig sind dir Glocken und die Puppe im Vergleich zu Faden? Zum Beispiel: Würdest du lieber die Puppe als einen Faden nehmen, oder sind zwei Glocken für dich mehr wert als ein Faden? Wenn Faden für dich sehr wichtig ist, könnte ich eher Puppe/Glocken nehmen; wenn nicht, nehme ich gern die Fäden.} \\
            }
        }
    }
    & & \\ \\

    \theutterance \stepcounter{utterance}  
    & & \multicolumn{4}{p{0.6\linewidth}}{
        \cellcolor[rgb]{0.9,0.9,0.9}{
            \makecell[{{p{\linewidth}}}]{
                \texttt{\tiny{[P1$\langle$GM]}}
                \texttt{Klingt gut. Ich bin flexibel, aber bevor ich Werte teile: Wie wichtig sind dir Glocken und die Puppe im Vergleich zu Faden? Zum Beispiel: Würdest du lieber die Puppe als einen Faden nehmen, oder sind zwei Glocken für dich mehr wert als ein Faden? Wenn Faden für dich sehr wichtig ist, könnte ich eher Puppe/Glocken nehmen; wenn nicht, nehme ich gern die Fäden.} \\
            }
        }
    }
    & \\ \\

    \theutterance \stepcounter{utterance}  
    & \multicolumn{4}{p{0.6\linewidth}}{
        \cellcolor[rgb]{0.9,0.9,0.9}{
            \makecell[{{p{\linewidth}}}]{
                \texttt{\tiny{[P1$\rangle$GM]}}
                \texttt{Für mich gilt: Ein Faden > eine Puppe > eine Glocke. Konkret:} \\
\texttt{{-} Ein Faden ist mir mehr wert als eine Puppe.} \\
\texttt{{-} Ein Faden ist mir auch mehr wert als zwei Glocken zusammen.} \\
\\ 
\texttt{Vorschlag für eine faire Aufteilung basierend darauf:} \\
\texttt{{-} Ich nehme die 2 Fäden.} \\
\texttt{{-} Du bekommst die Puppe und die 2 Glocken.} \\
\\ 
\texttt{Wenn das für dich passt, super. Falls du lieber einen Faden haben willst, sag mir, wie wichtig dir Faden ist, dann können wir tauschen (z.B. ich: 1 Faden + 1 Glocke, du: 1 Faden + Puppe + 1 Glocke).} \\
            }
        }
    }
    & & \\ \\

    \theutterance \stepcounter{utterance}  
    & & & \multicolumn{2}{p{0.3\linewidth}}{
        \cellcolor[rgb]{0.9,0.9,0.9}{
            \makecell[{{p{\linewidth}}}]{
                \texttt{\tiny{[GM$|$GM]}}
                \texttt{continue} \\
            }
        }
    }
    & & \\ \\

    \theutterance \stepcounter{utterance}  
    & & & \multicolumn{2}{p{0.3\linewidth}}{
        \cellcolor[rgb]{0.9,0.9,0.9}{
            \makecell[{{p{\linewidth}}}]{
                \texttt{\tiny{[GM$|$GM]}}
                \texttt{Für mich gilt: Ein Faden > eine Puppe > eine Glocke. Konkret:} \\
\texttt{{-} Ein Faden ist mir mehr wert als eine Puppe.} \\
\texttt{{-} Ein Faden ist mir auch mehr wert als zwei Glocken zusammen.} \\
\\ 
\texttt{Vorschlag für eine faire Aufteilung basierend darauf:} \\
\texttt{{-} Ich nehme die 2 Fäden.} \\
\texttt{{-} Du bekommst die Puppe und die 2 Glocken.} \\
\\ 
\texttt{Wenn das für dich passt, super. Falls du lieber einen Faden haben willst, sag mir, wie wichtig dir Faden ist, dann können wir tauschen (z.B. ich: 1 Faden + 1 Glocke, du: 1 Faden + Puppe + 1 Glocke).} \\
            }
        }
    }
    & & \\ \\

    \theutterance \stepcounter{utterance}  
    & & \multicolumn{4}{p{0.6\linewidth}}{
        \cellcolor[rgb]{0.9,0.9,0.9}{
            \makecell[{{p{\linewidth}}}]{
                \texttt{\tiny{[GM$\rangle$P2]}}
                \texttt{Für mich gilt: Ein Faden > eine Puppe > eine Glocke. Konkret:} \\
\texttt{{-} Ein Faden ist mir mehr wert als eine Puppe.} \\
\texttt{{-} Ein Faden ist mir auch mehr wert als zwei Glocken zusammen.} \\
\\ 
\texttt{Vorschlag für eine faire Aufteilung basierend darauf:} \\
\texttt{{-} Ich nehme die 2 Fäden.} \\
\texttt{{-} Du bekommst die Puppe und die 2 Glocken.} \\
\\ 
\texttt{Wenn das für dich passt, super. Falls du lieber einen Faden haben willst, sag mir, wie wichtig dir Faden ist, dann können wir tauschen (z.B. ich: 1 Faden + 1 Glocke, du: 1 Faden + Puppe + 1 Glocke).} \\
            }
        }
    }
    & \\ \\

    \theutterance \stepcounter{utterance}  
    & & & \multicolumn{4}{p{0.6\linewidth}}{
        \cellcolor[rgb]{0.9,0.9,0.9}{
            \makecell[{{p{\linewidth}}}]{
                \texttt{\tiny{[GM$\langle$P2]}}
                \texttt{Danke für die Klarheit. Deine Aufteilung klingt für mich gut: Du nimmst die 2 Fäden, ich nehme die Puppe und die 2 Glocken. Damit sind beide Seiten gut bedient.} \\
\\ 
\texttt{Wenn du bereit bist, können wir das so festhalten. Wenn du doch eine kleine Änderung willst, wäre meine einzige Alternative: ich nehme 1 Faden + Puppe, du nimmst 1 Faden + 2 Glocken. Ansonsten lass uns den ersten Plan bestätigen.} \\
            }
        }
    }
     \\ \\

    \theutterance \stepcounter{utterance}  
    & & & \multicolumn{2}{p{0.3\linewidth}}{
        \cellcolor[rgb]{0.9,0.9,0.9}{
            \makecell[{{p{\linewidth}}}]{
                \texttt{\tiny{[GM$|$GM]}}
                \texttt{continue} \\
            }
        }
    }
    & & \\ \\

    \theutterance \stepcounter{utterance}  
    & & & \multicolumn{2}{p{0.3\linewidth}}{
        \cellcolor[rgb]{0.9,0.9,0.9}{
            \makecell[{{p{\linewidth}}}]{
                \texttt{\tiny{[GM$|$GM]}}
                \texttt{Danke für die Klarheit. Deine Aufteilung klingt für mich gut: Du nimmst die 2 Fäden, ich nehme die Puppe und die 2 Glocken. Damit sind beide Seiten gut bedient.} \\
\\ 
\texttt{Wenn du bereit bist, können wir das so festhalten. Wenn du doch eine kleine Änderung willst, wäre meine einzige Alternative: ich nehme 1 Faden + Puppe, du nimmst 1 Faden + 2 Glocken. Ansonsten lass uns den ersten Plan bestätigen.} \\
            }
        }
    }
    & & \\ \\

    \theutterance \stepcounter{utterance}  
    & & \multicolumn{4}{p{0.6\linewidth}}{
        \cellcolor[rgb]{0.9,0.9,0.9}{
            \makecell[{{p{\linewidth}}}]{
                \texttt{\tiny{[P1$\langle$GM]}}
                \texttt{Danke für die Klarheit. Deine Aufteilung klingt für mich gut: Du nimmst die 2 Fäden, ich nehme die Puppe und die 2 Glocken. Damit sind beide Seiten gut bedient.} \\
\\ 
\texttt{Wenn du bereit bist, können wir das so festhalten. Wenn du doch eine kleine Änderung willst, wäre meine einzige Alternative: ich nehme 1 Faden + Puppe, du nimmst 1 Faden + 2 Glocken. Ansonsten lass uns den ersten Plan bestätigen.} \\
            }
        }
    }
    & \\ \\

    \theutterance \stepcounter{utterance}  
    & \multicolumn{4}{p{0.6\linewidth}}{
        \cellcolor[rgb]{0.9,0.9,0.9}{
            \makecell[{{p{\linewidth}}}]{
                \texttt{\tiny{[P1$\rangle$GM]}}
                \texttt{Klingt perfekt – lass uns den ersten Plan festhalten. } \\
\\ 
\texttt{{[}Proposal: 0 Glocke, 0 Puppe, 2 Faden{]}} \\
            }
        }
    }
    & & \\ \\

    \theutterance \stepcounter{utterance}  
    & & & \multicolumn{2}{p{0.3\linewidth}}{
        \cellcolor[rgb]{0.9,0.9,0.9}{
            \makecell[{{p{\linewidth}}}]{
                \texttt{\tiny{[GM$|$GM]}}
                \texttt{proposal does not start with "vorschlag:"} \\
            }
        }
    }
    & & \\ \\

\end{supertabular}
}

\end{document}
