
\documentclass{article}
\usepackage{colortbl}
\usepackage{makecell}
\usepackage{multirow}
\usepackage{supertabular}

\begin{document}

\newcounter{utterance}

\twocolumn

{ \footnotesize  \setcounter{utterance}{1}
\setlength{\tabcolsep}{0pt}
\begin{supertabular}{c@{$\;$}|p{.15\linewidth}@{}p{.15\linewidth}p{.15\linewidth}p{.15\linewidth}p{.15\linewidth}p{.15\linewidth}}

    \# & $\;$A & \multicolumn{4}{c}{Game Master} & $\;\:$B\\
    \hline 

    \theutterance \stepcounter{utterance}  

    & & \multicolumn{4}{p{0.6\linewidth}}{\cellcolor[rgb]{0.9,0.9,0.9}{%
	\makecell[{{p{\linewidth}}}]{% 
	  \tt {\tiny [A$\langle$GM]}  
	 Sie spielen ein kooperatives Verhandlungsspiel, bei dem Sie sich mit einem anderen Spieler darauf einigen müssen, wie eine Reihe von Gegenständen aufgeteilt werden soll.\\ \tt \\ \tt Die Regeln:\\ \tt (a) Sie und der andere Spieler erhalten eine Sammlung von Gegenständen. Jeder von Ihnen erhält außerdem eine geheime Wertfunktion, die angibt, wie viel Ihnen jede Art von Gegenstand wert ist.\\ \tt (b) Sie tauschen Nachrichten mit dem anderen Spieler aus, um zu vereinbaren, wer welche Gegenstände bekommt. Sie können jeweils maximal 5 Nachrichten senden oder das Spiel vorzeitig beenden, indem Sie jederzeit einen geheimen Vorschlag machen.\\ \tt (c) Jeder von euch wird aufgefordert, einen geheimen Vorschlag zu machen, in dem ihr die gewünschten Gegenstände in eckigen Klammern wie folgt angibt: "[Vorschlag: <Nummer> <Objektname>, <Nummer> <Objektname>, <...>]"\\ \tt (d) Wenn eure Vorschläge komplementär sind, d.h. es gibt genug Gegenstände, um beide Vorschläge zu erfüllen, erhält jeder Spieler eine Punktzahl, die sich aus der Summe der Werte für die Gegenstände ergibt, die er erhalten hat. Andernfalls erhalten Sie beide null Punkte.\\ \tt (e) Ihr Ziel ist es, die Summe aus Ihrer Punktzahl und der Punktzahl des anderen Spielers zu maximieren.\\ \tt \\ \tt Beginnen wir.\\ \tt \\ \tt Die Menge der verfügbaren Gegenstände ist:\\ \tt \\ \tt 1 Telefon, 2 Spiegel, 2 Früchte\\ \tt \\ \tt Deine geheime Wertfunktion ist:\\ \tt \\ \tt Telefon: 4, Spiegel: 2, Frucht: 1\\ \tt \\ \tt WICHTIG: Ihre Nachrichten werden, sofern es sich nicht um einen geheimen Vorschlag handelt, direkt an den anderen Spieler übermittelt, also fügen Sie keine Antwort auf die Regeln oder einen Text zur Ankündigung Ihrer Nachricht ein. Um einen geheimen Vorschlag zu machen, verwenden Sie das angegebene Format. Verwenden Sie keine eckigen Klammern, wenn Sie mit dem anderen Spieler kommunizieren, sonst wird dies als Ihr geheimer Vorschlag interpretiert.\\ \tt  
	  } 
	   } 
	   } 
	 & \\ 
 

    \theutterance \stepcounter{utterance}  

    & \multicolumn{4}{p{0.6\linewidth}}{\cellcolor[rgb]{0.8,1,0.9}{%
	\makecell[{{p{\linewidth}}}]{% 
	  \tt {\tiny [A$\rangle$GM]}  
	 Hallo! Ich schlage vor, wir maximieren gemeinsam die Gesamtpunkte. Meine groben Präferenzen: Telefon ist mir am meisten wert, Spiegel mittel, Früchte gering. Welche Gegenstände sind dir am wichtigsten? Ich kann flexibel bei Spiegeln/Früchten sein, wenn du z. B. das Telefon höher bewertest. 
	  } 
	   } 
	   } 
	 & & \\ 
 

    \theutterance \stepcounter{utterance}  

    & & & \multicolumn{2}{p{0.3\linewidth}}{\cellcolor[rgb]{0.95,0.95,0.95}{%
	\makecell[{{p{\linewidth}}}]{% 
	  \tt {\tiny [GM$|$GM]}  
	 continue 
	  } 
	   } 
	   } 
	 & & \\ 
 

    \theutterance \stepcounter{utterance}  

    & & & \multicolumn{2}{p{0.3\linewidth}}{\cellcolor[rgb]{0.95,0.95,0.95}{%
	\makecell[{{p{\linewidth}}}]{% 
	  \tt {\tiny [GM$|$GM]}  
	 Hallo! Ich schlage vor, wir maximieren gemeinsam die Gesamtpunkte. Meine groben Präferenzen: Telefon ist mir am meisten wert, Spiegel mittel, Früchte gering. Welche Gegenstände sind dir am wichtigsten? Ich kann flexibel bei Spiegeln/Früchten sein, wenn du z. B. das Telefon höher bewertest. 
	  } 
	   } 
	   } 
	 & & \\ 
 

    \theutterance \stepcounter{utterance}  

    & & \multicolumn{4}{p{0.6\linewidth}}{\cellcolor[rgb]{0.9,0.9,0.9}{%
	\makecell[{{p{\linewidth}}}]{% 
	  \tt {\tiny [GM$\rangle$B]}  
	 Sie spielen ein kooperatives Verhandlungsspiel, bei dem Sie sich mit einem anderen Spieler darauf einigen müssen, wie eine Reihe von Gegenständen aufgeteilt werden soll.\\ \tt \\ \tt Die Regeln:\\ \tt (a) Sie und der andere Spieler erhalten eine Sammlung von Gegenständen. Jeder von Ihnen erhält außerdem eine geheime Wertfunktion, die angibt, wie viel Ihnen jede Art von Gegenstand wert ist.\\ \tt (b) Sie tauschen Nachrichten mit dem anderen Spieler aus, um zu vereinbaren, wer welche Gegenstände bekommt. Sie können jeweils maximal 5 Nachrichten senden oder das Spiel vorzeitig beenden, indem Sie jederzeit einen geheimen Vorschlag machen.\\ \tt (c) Jeder von euch wird aufgefordert, einen geheimen Vorschlag zu machen, in dem ihr die gewünschten Gegenstände in eckigen Klammern wie folgt angibt: "[Vorschlag: <Nummer> <Objektname>, <Nummer> <Objektname>, <...>]"\\ \tt (d) Wenn eure Vorschläge komplementär sind, d.h. es gibt genug Gegenstände, um beide Vorschläge zu erfüllen, erhält jeder Spieler eine Punktzahl, die sich aus der Summe der Werte für die Gegenstände ergibt, die er erhalten hat. Andernfalls erhalten Sie beide null Punkte.\\ \tt (e) Ihr Ziel ist es, die Summe aus Ihrer Punktzahl und der Punktzahl des anderen Spielers zu maximieren.\\ \tt \\ \tt Beginnen wir.\\ \tt \\ \tt Die Menge der verfügbaren Gegenstände ist:\\ \tt \\ \tt 1 Telefon, 2 Spiegel, 2 Früchte\\ \tt \\ \tt Deine geheime Wertfunktion ist:\\ \tt \\ \tt Telefon: 6, Spiegel: 0, Frucht: 2\\ \tt \\ \tt WICHTIG: Ihre Nachrichten werden, sofern es sich nicht um einen geheimen Vorschlag handelt, direkt an den anderen Spieler übermittelt, also fügen Sie keine Antwort auf die Regeln oder einen Text zur Ankündigung Ihrer Nachricht ein. Um einen geheimen Vorschlag zu machen, verwenden Sie das angegebene Format. Verwenden Sie keine eckigen Klammern, wenn Sie mit dem anderen Spieler kommunizieren, sonst wird dies als Ihr geheimer Vorschlag interpretiert.\\ \tt \\ \tt \\ \tt Hallo! Ich schlage vor, wir maximieren gemeinsam die Gesamtpunkte. Meine groben Präferenzen: Telefon ist mir am meisten wert, Spiegel mittel, Früchte gering. Welche Gegenstände sind dir am wichtigsten? Ich kann flexibel bei Spiegeln/Früchten sein, wenn du z. B. das Telefon höher bewertest. 
	  } 
	   } 
	   } 
	 & \\ 
 

    \theutterance \stepcounter{utterance}  

    & & & \multicolumn{4}{p{0.6\linewidth}}{\cellcolor[rgb]{1,0.85,0.72}{%
	\makecell[{{p{\linewidth}}}]{% 
	  \tt {\tiny [GM$\langle$B]}  
	 Danke für den Einstieg! Mir ist das Telefon am wichtigsten, Spiegel sind mir praktisch nichts wert, Früchte sind mir etwas wert. Konkreter Vorschlag zur Maximierung der Gesamtsumme: Ich nehme das Telefon und 1 Frucht; du bekommst 2 Spiegel und 1 Frucht. Das lässt dir viel bei den Items, die dir wichtig scheinen, und hält den Gesamtwert hoch. Passt das in etwa zu deinen Prioritäten? 
	  } 
	   } 
	   } 
	  \\ 
 

    \theutterance \stepcounter{utterance}  

    & & & \multicolumn{2}{p{0.3\linewidth}}{\cellcolor[rgb]{0.95,0.95,0.95}{%
	\makecell[{{p{\linewidth}}}]{% 
	  \tt {\tiny [GM$|$GM]}  
	 continue 
	  } 
	   } 
	   } 
	 & & \\ 
 

    \theutterance \stepcounter{utterance}  

    & & & \multicolumn{2}{p{0.3\linewidth}}{\cellcolor[rgb]{0.95,0.95,0.95}{%
	\makecell[{{p{\linewidth}}}]{% 
	  \tt {\tiny [GM$|$GM]}  
	 Danke für den Einstieg! Mir ist das Telefon am wichtigsten, Spiegel sind mir praktisch nichts wert, Früchte sind mir etwas wert. Konkreter Vorschlag zur Maximierung der Gesamtsumme: Ich nehme das Telefon und 1 Frucht; du bekommst 2 Spiegel und 1 Frucht. Das lässt dir viel bei den Items, die dir wichtig scheinen, und hält den Gesamtwert hoch. Passt das in etwa zu deinen Prioritäten? 
	  } 
	   } 
	   } 
	 & & \\ 
 

    \theutterance \stepcounter{utterance}  

    & & \multicolumn{4}{p{0.6\linewidth}}{\cellcolor[rgb]{0.9,0.9,0.9}{%
	\makecell[{{p{\linewidth}}}]{% 
	  \tt {\tiny [A$\langle$GM]}  
	 Danke für den Einstieg! Mir ist das Telefon am wichtigsten, Spiegel sind mir praktisch nichts wert, Früchte sind mir etwas wert. Konkreter Vorschlag zur Maximierung der Gesamtsumme: Ich nehme das Telefon und 1 Frucht; du bekommst 2 Spiegel und 1 Frucht. Das lässt dir viel bei den Items, die dir wichtig scheinen, und hält den Gesamtwert hoch. Passt das in etwa zu deinen Prioritäten? 
	  } 
	   } 
	   } 
	 & \\ 
 

    \theutterance \stepcounter{utterance}  

    & \multicolumn{4}{p{0.6\linewidth}}{\cellcolor[rgb]{0.8,1,0.9}{%
	\makecell[{{p{\linewidth}}}]{% 
	  \tt {\tiny [A$\rangle$GM]}  
	 Klingt grundsätzlich gut und ist nahe an meinen Präferenzen. Für mich wäre ein klein wenig Tauschwert drin, weil mir Spiegel je 2 wert sind und Früchte nur 1.\\ \tt \\ \tt Zwei Varianten, die beide die Gesamtsumme maximieren:\\ \tt - Dein Vorschlag: Du nimmst Telefon + 1 Frucht; ich nehme 2 Spiegel + 1 Frucht. (Gesamtwert = 4+1 für mich + unbekannt für dich)\\ \tt - Alternative: Du nimmst Telefon + 2 Früchte; ich nehme 2 Spiegel. So behältst du dein wichtigstes Item und bekommst sogar beide Früchte; ich erhalte die für mich wertvolleren Spiegel.\\ \tt \\ \tt Wenn dir das Telefon wirklich zentral ist, wähle gern eine der beiden Optionen – mir sind beide recht. 
	  } 
	   } 
	   } 
	 & & \\ 
 

    \theutterance \stepcounter{utterance}  

    & & & \multicolumn{2}{p{0.3\linewidth}}{\cellcolor[rgb]{0.95,0.95,0.95}{%
	\makecell[{{p{\linewidth}}}]{% 
	  \tt {\tiny [GM$|$GM]}  
	 continue 
	  } 
	   } 
	   } 
	 & & \\ 
 

    \theutterance \stepcounter{utterance}  

    & & & \multicolumn{2}{p{0.3\linewidth}}{\cellcolor[rgb]{0.95,0.95,0.95}{%
	\makecell[{{p{\linewidth}}}]{% 
	  \tt {\tiny [GM$|$GM]}  
	 Klingt grundsätzlich gut und ist nahe an meinen Präferenzen. Für mich wäre ein klein wenig Tauschwert drin, weil mir Spiegel je 2 wert sind und Früchte nur 1.\\ \tt \\ \tt Zwei Varianten, die beide die Gesamtsumme maximieren:\\ \tt - Dein Vorschlag: Du nimmst Telefon + 1 Frucht; ich nehme 2 Spiegel + 1 Frucht. (Gesamtwert = 4+1 für mich + unbekannt für dich)\\ \tt - Alternative: Du nimmst Telefon + 2 Früchte; ich nehme 2 Spiegel. So behältst du dein wichtigstes Item und bekommst sogar beide Früchte; ich erhalte die für mich wertvolleren Spiegel.\\ \tt \\ \tt Wenn dir das Telefon wirklich zentral ist, wähle gern eine der beiden Optionen – mir sind beide recht. 
	  } 
	   } 
	   } 
	 & & \\ 
 

    \theutterance \stepcounter{utterance}  

    & & \multicolumn{4}{p{0.6\linewidth}}{\cellcolor[rgb]{0.9,0.9,0.9}{%
	\makecell[{{p{\linewidth}}}]{% 
	  \tt {\tiny [GM$\rangle$B]}  
	 Klingt grundsätzlich gut und ist nahe an meinen Präferenzen. Für mich wäre ein klein wenig Tauschwert drin, weil mir Spiegel je 2 wert sind und Früchte nur 1.\\ \tt \\ \tt Zwei Varianten, die beide die Gesamtsumme maximieren:\\ \tt - Dein Vorschlag: Du nimmst Telefon + 1 Frucht; ich nehme 2 Spiegel + 1 Frucht. (Gesamtwert = 4+1 für mich + unbekannt für dich)\\ \tt - Alternative: Du nimmst Telefon + 2 Früchte; ich nehme 2 Spiegel. So behältst du dein wichtigstes Item und bekommst sogar beide Früchte; ich erhalte die für mich wertvolleren Spiegel.\\ \tt \\ \tt Wenn dir das Telefon wirklich zentral ist, wähle gern eine der beiden Optionen – mir sind beide recht. 
	  } 
	   } 
	   } 
	 & \\ 
 

    \theutterance \stepcounter{utterance}  

    & & & \multicolumn{4}{p{0.6\linewidth}}{\cellcolor[rgb]{1,0.85,0.72}{%
	\makecell[{{p{\linewidth}}}]{% 
	  \tt {\tiny [GM$\langle$B]}  
	 Klingt gut. Mir sind Spiegel nichts wert und Früchte etwas, das Telefon ist mir am wichtigsten. Dann wähle ich die Alternative: Ich nehme das Telefon und 2 Früchte, du bekommst die 2 Spiegel. \\ \tt \\ \tt [Vorschlag: 1 Telefon, 2 Frucht] 
	  } 
	   } 
	   } 
	  \\ 
 

    \theutterance \stepcounter{utterance}  

    & & & \multicolumn{2}{p{0.3\linewidth}}{\cellcolor[rgb]{0.95,0.95,0.95}{%
	\makecell[{{p{\linewidth}}}]{% 
	  \tt {\tiny [GM$|$GM]}  
	 proposal 
	  } 
	   } 
	   } 
	 & & \\ 
 

    \theutterance \stepcounter{utterance}  

    & & & \multicolumn{2}{p{0.3\linewidth}}{\cellcolor[rgb]{0.95,0.95,0.95}{%
	\makecell[{{p{\linewidth}}}]{% 
	  \tt {\tiny [GM$|$GM]}  
	 [1, 0, 2] 
	  } 
	   } 
	   } 
	 & & \\ 
 

    \theutterance \stepcounter{utterance}  

    & & \multicolumn{4}{p{0.6\linewidth}}{\cellcolor[rgb]{0.9,0.9,0.9}{%
	\makecell[{{p{\linewidth}}}]{% 
	  \tt {\tiny [A$\langle$GM]}  
	 Der andere Spieler hat einen geheimen Vorschlag gemacht. Jetzt bist du an der Reihe, einen eigenen Vorschlag zu machen.\\ \tt  
	  } 
	   } 
	   } 
	 & \\ 
 

    \theutterance \stepcounter{utterance}  

    & \multicolumn{4}{p{0.6\linewidth}}{\cellcolor[rgb]{0.8,1,0.9}{%
	\makecell[{{p{\linewidth}}}]{% 
	  \tt {\tiny [A$\rangle$GM]}  
	 [Vorschlag: 0 Telefon, 2 Spiegel, 0 Frucht] 
	  } 
	   } 
	   } 
	 & & \\ 
 

    \theutterance \stepcounter{utterance}  

    & & & \multicolumn{2}{p{0.3\linewidth}}{\cellcolor[rgb]{0.95,0.95,0.95}{%
	\makecell[{{p{\linewidth}}}]{% 
	  \tt {\tiny [GM$|$GM]}  
	 proposal 
	  } 
	   } 
	   } 
	 & & \\ 
 

    \theutterance \stepcounter{utterance}  

    & & & \multicolumn{2}{p{0.3\linewidth}}{\cellcolor[rgb]{0.95,0.95,0.95}{%
	\makecell[{{p{\linewidth}}}]{% 
	  \tt {\tiny [GM$|$GM]}  
	 [0, 2, 0] 
	  } 
	   } 
	   } 
	 & & \\ 
 

    \theutterance \stepcounter{utterance}  

    & & & \multicolumn{2}{p{0.3\linewidth}}{\cellcolor[rgb]{0.95,0.95,0.95}{%
	\makecell[{{p{\linewidth}}}]{% 
	  \tt {\tiny [GM$|$GM]}  
	 [[0, 2, 0], [1, 0, 2]] 
	  } 
	   } 
	   } 
	 & & \\ 
 

\end{supertabular}
}

\end{document}
