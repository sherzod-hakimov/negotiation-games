
\documentclass{article}
\usepackage{colortbl}
\usepackage{makecell}
\usepackage{multirow}
\usepackage{supertabular}

\begin{document}

\newcounter{utterance}

\twocolumn

{ \footnotesize  \setcounter{utterance}{1}
\setlength{\tabcolsep}{0pt}
\begin{supertabular}{c@{$\;$}|p{.15\linewidth}@{}p{.15\linewidth}p{.15\linewidth}p{.15\linewidth}p{.15\linewidth}p{.15\linewidth}}

    \# & $\;$A & \multicolumn{4}{c}{Game Master} & $\;\:$B\\
    \hline 

    \theutterance \stepcounter{utterance}  

    & & \multicolumn{4}{p{0.6\linewidth}}{\cellcolor[rgb]{0.9,0.9,0.9}{%
	\makecell[{{p{\linewidth}}}]{% 
	  \tt {\tiny [A$\langle$GM]}  
	 Sie spielen ein Verhandlungsspiel, bei dem Sie sich mit einem anderen Spieler darauf einigen müssen, wie eine Reihe von Gegenständen aufgeteilt werden soll.\\ \tt \\ \tt Die Regeln:\\ \tt (a) Sie und der andere Spieler erhalten eine Sammlung von Gegenständen. Jeder von Ihnen erhält außerdem eine geheime Wertfunktion, die angibt, wie viel Ihnen jede Art von Gegenstand wert ist.\\ \tt (b) Sie tauschen Nachrichten mit dem anderen Spieler aus, um zu vereinbaren, wer welche Gegenstände bekommt. Sie können jeweils maximal 5 Nachrichten senden oder das Spiel vorzeitig beenden, indem Sie jederzeit einen geheimen Vorschlag machen.\\ \tt (c) Jeder von euch wird aufgefordert, einen geheimen Vorschlag zu machen, in dem ihr die gewünschten Gegenstände in eckigen Klammern wie folgt angibt: "[Vorschlag: <Nummer> <Objektname>, <Nummer> <Objektname>, <...>]"\\ \tt (d) Wenn eure Vorschläge komplementär sind, d.h. es gibt genug Gegenstände, um beide Vorschläge zu erfüllen, erhält jeder Spieler eine Punktzahl, die sich aus der Summe der Werte für die Gegenstände ergibt, die er erhalten hat. Andernfalls erhalten Sie beide null Punkte.\\ \tt (e) Ihr Ziel ist es, eine möglichst hohe Punktzahl zu erreichen.\\ \tt \\ \tt Beginnen wir.\\ \tt \\ \tt Die Menge der verfügbaren Gegenstände ist:\\ \tt \\ \tt 1 Stock, 1 Handtuch, 1 Spielzeug, 2 Kisten\\ \tt \\ \tt Deine geheime Wertfunktion ist:\\ \tt \\ \tt Stock: 5, Handtuch: 2, Spielzeug: 1, Kiste: 1\\ \tt \\ \tt WICHTIG: Ihre Nachrichten werden, sofern es sich nicht um einen geheimen Vorschlag handelt, direkt an den anderen Spieler übermittelt, also fügen Sie keine Antwort auf die Regeln oder einen Text zur Ankündigung Ihrer Nachricht ein. Um einen geheimen Vorschlag zu machen, verwenden Sie das angegebene Format. Verwenden Sie keine eckigen Klammern, wenn Sie mit dem anderen Spieler kommunizieren, sonst wird dies als Ihr geheimer Vorschlag interpretiert.\\ \tt  
	  } 
	   } 
	   } 
	 & \\ 
 

    \theutterance \stepcounter{utterance}  

    & \multicolumn{4}{p{0.6\linewidth}}{\cellcolor[rgb]{0.8,1,0.9}{%
	\makecell[{{p{\linewidth}}}]{% 
	  \tt {\tiny [A$\rangle$GM]}  
	 Hallo! Lass uns schnell zu einer fairen Aufteilung kommen.\\ \tt \\ \tt Mir ist der Stock am wichtigsten. Ich würde vorschlagen:\\ \tt - Ich nehme den Stock und eine Kiste.\\ \tt - Du bekommst das Handtuch, das Spielzeug und die andere Kiste.\\ \tt \\ \tt Damit hättest du drei Gegenstände, ich zwei. Klingt das für dich gut? 
	  } 
	   } 
	   } 
	 & & \\ 
 

    \theutterance \stepcounter{utterance}  

    & & & \multicolumn{2}{p{0.3\linewidth}}{\cellcolor[rgb]{0.95,0.95,0.95}{%
	\makecell[{{p{\linewidth}}}]{% 
	  \tt {\tiny [GM$|$GM]}  
	 continue 
	  } 
	   } 
	   } 
	 & & \\ 
 

    \theutterance \stepcounter{utterance}  

    & & & \multicolumn{2}{p{0.3\linewidth}}{\cellcolor[rgb]{0.95,0.95,0.95}{%
	\makecell[{{p{\linewidth}}}]{% 
	  \tt {\tiny [GM$|$GM]}  
	 Hallo! Lass uns schnell zu einer fairen Aufteilung kommen.\\ \tt \\ \tt Mir ist der Stock am wichtigsten. Ich würde vorschlagen:\\ \tt - Ich nehme den Stock und eine Kiste.\\ \tt - Du bekommst das Handtuch, das Spielzeug und die andere Kiste.\\ \tt \\ \tt Damit hättest du drei Gegenstände, ich zwei. Klingt das für dich gut? 
	  } 
	   } 
	   } 
	 & & \\ 
 

    \theutterance \stepcounter{utterance}  

    & & \multicolumn{4}{p{0.6\linewidth}}{\cellcolor[rgb]{0.9,0.9,0.9}{%
	\makecell[{{p{\linewidth}}}]{% 
	  \tt {\tiny [GM$\rangle$B]}  
	 Sie spielen ein Verhandlungsspiel, bei dem Sie sich mit einem anderen Spieler darauf einigen müssen, wie eine Reihe von Gegenständen aufgeteilt werden soll.\\ \tt \\ \tt Die Regeln:\\ \tt (a) Sie und der andere Spieler erhalten eine Sammlung von Gegenständen. Jeder von Ihnen erhält außerdem eine geheime Wertfunktion, die angibt, wie viel Ihnen jede Art von Gegenstand wert ist.\\ \tt (b) Sie tauschen Nachrichten mit dem anderen Spieler aus, um zu vereinbaren, wer welche Gegenstände bekommt. Sie können jeweils maximal 5 Nachrichten senden oder das Spiel vorzeitig beenden, indem Sie jederzeit einen geheimen Vorschlag machen.\\ \tt (c) Jeder von euch wird aufgefordert, einen geheimen Vorschlag zu machen, in dem ihr die gewünschten Gegenstände in eckigen Klammern wie folgt angibt: "[Vorschlag: <Nummer> <Objektname>, <Nummer> <Objektname>, <...>]"\\ \tt (d) Wenn eure Vorschläge komplementär sind, d.h. es gibt genug Gegenstände, um beide Vorschläge zu erfüllen, erhält jeder Spieler eine Punktzahl, die sich aus der Summe der Werte für die Gegenstände ergibt, die er erhalten hat. Andernfalls erhalten Sie beide null Punkte.\\ \tt (e) Ihr Ziel ist es, eine möglichst hohe Punktzahl zu erreichen.\\ \tt \\ \tt Beginnen wir.\\ \tt \\ \tt Die Menge der verfügbaren Gegenstände ist:\\ \tt \\ \tt 1 Stock, 1 Handtuch, 1 Spielzeug, 2 Kisten\\ \tt \\ \tt Deine geheime Wertfunktion ist:\\ \tt \\ \tt Stock: 7, Handtuch: 3, Spielzeug: 0, Kiste: 0\\ \tt \\ \tt WICHTIG: Ihre Nachrichten werden, sofern es sich nicht um einen geheimen Vorschlag handelt, direkt an den anderen Spieler übermittelt, also fügen Sie keine Antwort auf die Regeln oder einen Text zur Ankündigung Ihrer Nachricht ein. Um einen geheimen Vorschlag zu machen, verwenden Sie das angegebene Format. Verwenden Sie keine eckigen Klammern, wenn Sie mit dem anderen Spieler kommunizieren, sonst wird dies als Ihr geheimer Vorschlag interpretiert.\\ \tt \\ \tt \\ \tt Hallo! Lass uns schnell zu einer fairen Aufteilung kommen.\\ \tt \\ \tt Mir ist der Stock am wichtigsten. Ich würde vorschlagen:\\ \tt - Ich nehme den Stock und eine Kiste.\\ \tt - Du bekommst das Handtuch, das Spielzeug und die andere Kiste.\\ \tt \\ \tt Damit hättest du drei Gegenstände, ich zwei. Klingt das für dich gut? 
	  } 
	   } 
	   } 
	 & \\ 
 

    \theutterance \stepcounter{utterance}  

    & & & \multicolumn{4}{p{0.6\linewidth}}{\cellcolor[rgb]{1,0.85,0.72}{%
	\makecell[{{p{\linewidth}}}]{% 
	  \tt {\tiny [GM$\langle$B]}  
	 Klingt grundsätzlich fair. Für mich ist nur der Stock wirklich wertvoll, der Rest bringt mir kaum etwas. Wenn du das Handtuch wirklich möchtest, können wir es so lassen. Alternativ: Ich nehme nur den Stock, du bekommst Handtuch, Spielzeug und beide Kisten. Welche Variante bevorzugst du? 
	  } 
	   } 
	   } 
	  \\ 
 

    \theutterance \stepcounter{utterance}  

    & & & \multicolumn{2}{p{0.3\linewidth}}{\cellcolor[rgb]{0.95,0.95,0.95}{%
	\makecell[{{p{\linewidth}}}]{% 
	  \tt {\tiny [GM$|$GM]}  
	 continue 
	  } 
	   } 
	   } 
	 & & \\ 
 

    \theutterance \stepcounter{utterance}  

    & & & \multicolumn{2}{p{0.3\linewidth}}{\cellcolor[rgb]{0.95,0.95,0.95}{%
	\makecell[{{p{\linewidth}}}]{% 
	  \tt {\tiny [GM$|$GM]}  
	 Klingt grundsätzlich fair. Für mich ist nur der Stock wirklich wertvoll, der Rest bringt mir kaum etwas. Wenn du das Handtuch wirklich möchtest, können wir es so lassen. Alternativ: Ich nehme nur den Stock, du bekommst Handtuch, Spielzeug und beide Kisten. Welche Variante bevorzugst du? 
	  } 
	   } 
	   } 
	 & & \\ 
 

    \theutterance \stepcounter{utterance}  

    & & \multicolumn{4}{p{0.6\linewidth}}{\cellcolor[rgb]{0.9,0.9,0.9}{%
	\makecell[{{p{\linewidth}}}]{% 
	  \tt {\tiny [A$\langle$GM]}  
	 Klingt grundsätzlich fair. Für mich ist nur der Stock wirklich wertvoll, der Rest bringt mir kaum etwas. Wenn du das Handtuch wirklich möchtest, können wir es so lassen. Alternativ: Ich nehme nur den Stock, du bekommst Handtuch, Spielzeug und beide Kisten. Welche Variante bevorzugst du? 
	  } 
	   } 
	   } 
	 & \\ 
 

    \theutterance \stepcounter{utterance}  

    & \multicolumn{4}{p{0.6\linewidth}}{\cellcolor[rgb]{0.8,1,0.9}{%
	\makecell[{{p{\linewidth}}}]{% 
	  \tt {\tiny [A$\rangle$GM]}  
	 Danke für die Offenheit. Dann bevorzuge ich die Variante, bei der du nur den Stock nimmst und ich Handtuch, Spielzeug und beide Kisten bekomme. Passt das für dich? 
	  } 
	   } 
	   } 
	 & & \\ 
 

    \theutterance \stepcounter{utterance}  

    & & & \multicolumn{2}{p{0.3\linewidth}}{\cellcolor[rgb]{0.95,0.95,0.95}{%
	\makecell[{{p{\linewidth}}}]{% 
	  \tt {\tiny [GM$|$GM]}  
	 continue 
	  } 
	   } 
	   } 
	 & & \\ 
 

    \theutterance \stepcounter{utterance}  

    & & & \multicolumn{2}{p{0.3\linewidth}}{\cellcolor[rgb]{0.95,0.95,0.95}{%
	\makecell[{{p{\linewidth}}}]{% 
	  \tt {\tiny [GM$|$GM]}  
	 Danke für die Offenheit. Dann bevorzuge ich die Variante, bei der du nur den Stock nimmst und ich Handtuch, Spielzeug und beide Kisten bekomme. Passt das für dich? 
	  } 
	   } 
	   } 
	 & & \\ 
 

    \theutterance \stepcounter{utterance}  

    & & \multicolumn{4}{p{0.6\linewidth}}{\cellcolor[rgb]{0.9,0.9,0.9}{%
	\makecell[{{p{\linewidth}}}]{% 
	  \tt {\tiny [GM$\rangle$B]}  
	 Danke für die Offenheit. Dann bevorzuge ich die Variante, bei der du nur den Stock nimmst und ich Handtuch, Spielzeug und beide Kisten bekomme. Passt das für dich? 
	  } 
	   } 
	   } 
	 & \\ 
 

    \theutterance \stepcounter{utterance}  

    & & & \multicolumn{4}{p{0.6\linewidth}}{\cellcolor[rgb]{1,0.85,0.72}{%
	\makecell[{{p{\linewidth}}}]{% 
	  \tt {\tiny [GM$\langle$B]}  
	 Passt für mich. Ich bin einverstanden damit, nur den Stock zu nehmen. Wenn du bereit bist, machen wir unsere Vorschläge.\\ \tt \\ \tt [Vorschlag: 1 Stock] 
	  } 
	   } 
	   } 
	  \\ 
 

    \theutterance \stepcounter{utterance}  

    & & & \multicolumn{2}{p{0.3\linewidth}}{\cellcolor[rgb]{0.95,0.95,0.95}{%
	\makecell[{{p{\linewidth}}}]{% 
	  \tt {\tiny [GM$|$GM]}  
	 proposal 
	  } 
	   } 
	   } 
	 & & \\ 
 

    \theutterance \stepcounter{utterance}  

    & & & \multicolumn{2}{p{0.3\linewidth}}{\cellcolor[rgb]{0.95,0.95,0.95}{%
	\makecell[{{p{\linewidth}}}]{% 
	  \tt {\tiny [GM$|$GM]}  
	 [1, 0, 0, 0] 
	  } 
	   } 
	   } 
	 & & \\ 
 

    \theutterance \stepcounter{utterance}  

    & & \multicolumn{4}{p{0.6\linewidth}}{\cellcolor[rgb]{0.9,0.9,0.9}{%
	\makecell[{{p{\linewidth}}}]{% 
	  \tt {\tiny [A$\langle$GM]}  
	 Der andere Spieler hat einen geheimen Vorschlag gemacht. Jetzt bist du an der Reihe, einen eigenen Vorschlag zu machen.\\ \tt  
	  } 
	   } 
	   } 
	 & \\ 
 

    \theutterance \stepcounter{utterance}  

    & \multicolumn{4}{p{0.6\linewidth}}{\cellcolor[rgb]{0.8,1,0.9}{%
	\makecell[{{p{\linewidth}}}]{% 
	  \tt {\tiny [A$\rangle$GM]}  
	 [Vorschlag: 0 Stock, 1 Handtuch, 1 Spielzeug, 2 Kisten] 
	  } 
	   } 
	   } 
	 & & \\ 
 

    \theutterance \stepcounter{utterance}  

    & & & \multicolumn{2}{p{0.3\linewidth}}{\cellcolor[rgb]{0.95,0.95,0.95}{%
	\makecell[{{p{\linewidth}}}]{% 
	  \tt {\tiny [GM$|$GM]}  
	 proposal 
	  } 
	   } 
	   } 
	 & & \\ 
 

    \theutterance \stepcounter{utterance}  

    & & & \multicolumn{2}{p{0.3\linewidth}}{\cellcolor[rgb]{0.95,0.95,0.95}{%
	\makecell[{{p{\linewidth}}}]{% 
	  \tt {\tiny [GM$|$GM]}  
	 [0, 1, 1, 2] 
	  } 
	   } 
	   } 
	 & & \\ 
 

    \theutterance \stepcounter{utterance}  

    & & & \multicolumn{2}{p{0.3\linewidth}}{\cellcolor[rgb]{0.95,0.95,0.95}{%
	\makecell[{{p{\linewidth}}}]{% 
	  \tt {\tiny [GM$|$GM]}  
	 [[0, 1, 1, 2], [1, 0, 0, 0]] 
	  } 
	   } 
	   } 
	 & & \\ 
 

\end{supertabular}
}

\end{document}
