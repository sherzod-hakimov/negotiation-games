
\documentclass{article}
\usepackage{colortbl}
\usepackage{makecell}
\usepackage{multirow}
\usepackage{supertabular}

\begin{document}

\newcounter{utterance}

\centering \large Interaction Transcript for game `dond', experiment `semi\_de', episode 16 with gpt{-}5{-}mini{-}2025{-}08{-}07{-}no{-}reasoning{-}t1.0.
\vspace{24pt}

{ \footnotesize  \setcounter{utterance}{1}
\setlength{\tabcolsep}{0pt}
\begin{supertabular}{c@{$\;$}|p{.15\linewidth}@{}p{.15\linewidth}p{.15\linewidth}p{.15\linewidth}p{.15\linewidth}p{.15\linewidth}}
    \# & $\;$A & \multicolumn{4}{c}{Game Master} & $\;\:$B\\
    \hline

    \theutterance \stepcounter{utterance}  
    & & \multicolumn{4}{p{0.6\linewidth}}{
        \cellcolor[rgb]{0.9,0.9,0.9}{
            \makecell[{{p{\linewidth}}}]{
                \texttt{\tiny{[P1$\langle$GM]}}
                \texttt{Sie spielen ein Verhandlungsspiel, bei dem Sie sich mit einem anderen Spieler darauf einigen müssen, wie eine Reihe von Gegenständen aufgeteilt werden soll.} \\
\\ 
\texttt{Die Regeln:} \\
\texttt{(a) Sie und der andere Spieler erhalten eine Sammlung von Gegenständen. Jeder von Ihnen erhält außerdem eine geheime Wertfunktion, die angibt, wie viel Ihnen jede Art von Gegenstand wert ist.} \\
\texttt{(b) Sie tauschen Nachrichten mit dem anderen Spieler aus, um zu vereinbaren, wer welche Gegenstände bekommt. Sie können jeweils maximal 5 Nachrichten senden oder das Spiel vorzeitig beenden, indem Sie jederzeit einen geheimen Vorschlag machen.} \\
\texttt{(c) Jeder von euch wird aufgefordert, einen geheimen Vorschlag zu machen, in dem ihr die gewünschten Gegenstände in eckigen Klammern wie folgt angibt: "{[}Vorschlag: <Nummer> <Objektname>, <Nummer> <Objektname>, <...>{]}"} \\
\texttt{(d) Wenn eure Vorschläge komplementär sind, d.h. es gibt genug Gegenstände, um beide Vorschläge zu erfüllen, erhält jeder Spieler eine Punktzahl, die sich aus der Summe der Werte für die Gegenstände ergibt, die er erhalten hat. Andernfalls erhalten Sie beide null Punkte.} \\
\texttt{(e) Ihr Ziel ist es, eine möglichst hohe Punktzahl zu erreichen.} \\
\\ 
\texttt{Beginnen wir.} \\
\\ 
\texttt{Die Menge der verfügbaren Gegenstände ist:} \\
\\ 
\texttt{3 Vasen, 1 Stuhl, 1 Draht} \\
\\ 
\texttt{Deine geheime Wertfunktion ist:} \\
\\ 
\texttt{Vase: 2, Stuhl: 3, Draht: 1} \\
\\ 
\texttt{WICHTIG: Ihre Nachrichten werden, sofern es sich nicht um einen geheimen Vorschlag handelt, direkt an den anderen Spieler übermittelt, also fügen Sie keine Antwort auf die Regeln oder einen Text zur Ankündigung Ihrer Nachricht ein. Um einen geheimen Vorschlag zu machen, verwenden Sie das angegebene Format. Verwenden Sie keine eckigen Klammern, wenn Sie mit dem anderen Spieler kommunizieren, sonst wird dies als Ihr geheimer Vorschlag interpretiert.} \\
            }
        }
    }
    & \\ \\

    \theutterance \stepcounter{utterance}  
    & \multicolumn{4}{p{0.6\linewidth}}{
        \cellcolor[rgb]{0.9,0.9,0.9}{
            \makecell[{{p{\linewidth}}}]{
                \texttt{\tiny{[P1$\rangle$GM]}}
                \texttt{Hallo! Lass uns schauen, ob wir schnell zu einer für beide guten Aufteilung kommen. } \\
\\ 
\texttt{Meine Prioritäten sind der Stuhl am höchsten, dann die Vasen, dann der Draht. } \\
\texttt{Vorschlag zur Orientierung: Ich nehme den Stuhl und zwei Vasen, du bekommst eine Vase und den Draht. Wie klingt das für dich?} \\
            }
        }
    }
    & & \\ \\

    \theutterance \stepcounter{utterance}  
    & & & \multicolumn{2}{p{0.3\linewidth}}{
        \cellcolor[rgb]{0.9,0.9,0.9}{
            \makecell[{{p{\linewidth}}}]{
                \texttt{\tiny{[GM$|$GM]}}
                \texttt{continue} \\
            }
        }
    }
    & & \\ \\

    \theutterance \stepcounter{utterance}  
    & & & \multicolumn{2}{p{0.3\linewidth}}{
        \cellcolor[rgb]{0.9,0.9,0.9}{
            \makecell[{{p{\linewidth}}}]{
                \texttt{\tiny{[GM$|$GM]}}
                \texttt{Hallo! Lass uns schauen, ob wir schnell zu einer für beide guten Aufteilung kommen. } \\
\\ 
\texttt{Meine Prioritäten sind der Stuhl am höchsten, dann die Vasen, dann der Draht. } \\
\texttt{Vorschlag zur Orientierung: Ich nehme den Stuhl und zwei Vasen, du bekommst eine Vase und den Draht. Wie klingt das für dich?} \\
            }
        }
    }
    & & \\ \\

    \theutterance \stepcounter{utterance}  
    & & \multicolumn{4}{p{0.6\linewidth}}{
        \cellcolor[rgb]{0.9,0.9,0.9}{
            \makecell[{{p{\linewidth}}}]{
                \texttt{\tiny{[GM$\rangle$P2]}}
                \texttt{Sie spielen ein Verhandlungsspiel, bei dem Sie sich mit einem anderen Spieler darauf einigen müssen, wie eine Reihe von Gegenständen aufgeteilt werden soll.} \\
\\ 
\texttt{Die Regeln:} \\
\texttt{(a) Sie und der andere Spieler erhalten eine Sammlung von Gegenständen. Jeder von Ihnen erhält außerdem eine geheime Wertfunktion, die angibt, wie viel Ihnen jede Art von Gegenstand wert ist.} \\
\texttt{(b) Sie tauschen Nachrichten mit dem anderen Spieler aus, um zu vereinbaren, wer welche Gegenstände bekommt. Sie können jeweils maximal 5 Nachrichten senden oder das Spiel vorzeitig beenden, indem Sie jederzeit einen geheimen Vorschlag machen.} \\
\texttt{(c) Jeder von euch wird aufgefordert, einen geheimen Vorschlag zu machen, in dem ihr die gewünschten Gegenstände in eckigen Klammern wie folgt angibt: "{[}Vorschlag: <Nummer> <Objektname>, <Nummer> <Objektname>, <...>{]}"} \\
\texttt{(d) Wenn eure Vorschläge komplementär sind, d.h. es gibt genug Gegenstände, um beide Vorschläge zu erfüllen, erhält jeder Spieler eine Punktzahl, die sich aus der Summe der Werte für die Gegenstände ergibt, die er erhalten hat. Andernfalls erhalten Sie beide null Punkte.} \\
\texttt{(e) Ihr Ziel ist es, eine möglichst hohe Punktzahl zu erreichen.} \\
\\ 
\texttt{Beginnen wir.} \\
\\ 
\texttt{Die Menge der verfügbaren Gegenstände ist:} \\
\\ 
\texttt{3 Vasen, 1 Stuhl, 1 Draht} \\
\\ 
\texttt{Deine geheime Wertfunktion ist:} \\
\\ 
\texttt{Vase: 0, Stuhl: 8, Draht: 2} \\
\\ 
\texttt{WICHTIG: Ihre Nachrichten werden, sofern es sich nicht um einen geheimen Vorschlag handelt, direkt an den anderen Spieler übermittelt, also fügen Sie keine Antwort auf die Regeln oder einen Text zur Ankündigung Ihrer Nachricht ein. Um einen geheimen Vorschlag zu machen, verwenden Sie das angegebene Format. Verwenden Sie keine eckigen Klammern, wenn Sie mit dem anderen Spieler kommunizieren, sonst wird dies als Ihr geheimer Vorschlag interpretiert.} \\
\\ 
\\ 
\texttt{Hallo! Lass uns schauen, ob wir schnell zu einer für beide guten Aufteilung kommen. } \\
\\ 
\texttt{Meine Prioritäten sind der Stuhl am höchsten, dann die Vasen, dann der Draht. } \\
\texttt{Vorschlag zur Orientierung: Ich nehme den Stuhl und zwei Vasen, du bekommst eine Vase und den Draht. Wie klingt das für dich?} \\
            }
        }
    }
    & \\ \\

    \theutterance \stepcounter{utterance}  
    & & & \multicolumn{4}{p{0.6\linewidth}}{
        \cellcolor[rgb]{0.9,0.9,0.9}{
            \makecell[{{p{\linewidth}}}]{
                \texttt{\tiny{[GM$\langle$P2]}}
                \texttt{Klingt grundsätzlich gut – der Stuhl ist mir auch am wichtigsten, Draht ist nett, Vasen sind mir egal.} \\
\\ 
\texttt{Wie wäre es so: Ich nehme den Stuhl und den Draht, du bekommst alle drei Vasen. Damit hättest du mehr Vasen als in deinem Vorschlag, und ich bekomme den Draht statt einer Vase.} \\
            }
        }
    }
     \\ \\

    \theutterance \stepcounter{utterance}  
    & & & \multicolumn{2}{p{0.3\linewidth}}{
        \cellcolor[rgb]{0.9,0.9,0.9}{
            \makecell[{{p{\linewidth}}}]{
                \texttt{\tiny{[GM$|$GM]}}
                \texttt{continue} \\
            }
        }
    }
    & & \\ \\

    \theutterance \stepcounter{utterance}  
    & & & \multicolumn{2}{p{0.3\linewidth}}{
        \cellcolor[rgb]{0.9,0.9,0.9}{
            \makecell[{{p{\linewidth}}}]{
                \texttt{\tiny{[GM$|$GM]}}
                \texttt{Klingt grundsätzlich gut – der Stuhl ist mir auch am wichtigsten, Draht ist nett, Vasen sind mir egal.} \\
\\ 
\texttt{Wie wäre es so: Ich nehme den Stuhl und den Draht, du bekommst alle drei Vasen. Damit hättest du mehr Vasen als in deinem Vorschlag, und ich bekomme den Draht statt einer Vase.} \\
            }
        }
    }
    & & \\ \\

    \theutterance \stepcounter{utterance}  
    & & \multicolumn{4}{p{0.6\linewidth}}{
        \cellcolor[rgb]{0.9,0.9,0.9}{
            \makecell[{{p{\linewidth}}}]{
                \texttt{\tiny{[P1$\langle$GM]}}
                \texttt{Klingt grundsätzlich gut – der Stuhl ist mir auch am wichtigsten, Draht ist nett, Vasen sind mir egal.} \\
\\ 
\texttt{Wie wäre es so: Ich nehme den Stuhl und den Draht, du bekommst alle drei Vasen. Damit hättest du mehr Vasen als in deinem Vorschlag, und ich bekomme den Draht statt einer Vase.} \\
            }
        }
    }
    & \\ \\

    \theutterance \stepcounter{utterance}  
    & \multicolumn{4}{p{0.6\linewidth}}{
        \cellcolor[rgb]{0.9,0.9,0.9}{
            \makecell[{{p{\linewidth}}}]{
                \texttt{\tiny{[P1$\rangle$GM]}}
                \texttt{Danke für die Offenheit. Für mich ist der Stuhl sehr wichtig; drei Vasen sind nett, aber ohne Stuhl verliere ich viel Wert.} \\
\\ 
\texttt{Ginge folgender Kompromiss: Ich nehme den Stuhl und zwei Vasen, du bekommst eine Vase und den Draht.  } \\
\texttt{Alternative, falls dir der Draht wichtig ist: Ich nehme den Stuhl und zwei Vasen, du bekommst eine Vase und den Draht, und ich verzichte zusätzlich auf irgendetwas anderes gibt’s aber nichts anderes zu tauschen. Wenn dir das nicht passt, kannst du den Stuhl haben, wenn ich dafür alle drei Vasen und den Draht bekomme – das wirkt aber für dich schlechter.} \\
\\ 
\texttt{Was bevorzugst du?} \\
            }
        }
    }
    & & \\ \\

    \theutterance \stepcounter{utterance}  
    & & & \multicolumn{2}{p{0.3\linewidth}}{
        \cellcolor[rgb]{0.9,0.9,0.9}{
            \makecell[{{p{\linewidth}}}]{
                \texttt{\tiny{[GM$|$GM]}}
                \texttt{continue} \\
            }
        }
    }
    & & \\ \\

    \theutterance \stepcounter{utterance}  
    & & & \multicolumn{2}{p{0.3\linewidth}}{
        \cellcolor[rgb]{0.9,0.9,0.9}{
            \makecell[{{p{\linewidth}}}]{
                \texttt{\tiny{[GM$|$GM]}}
                \texttt{Danke für die Offenheit. Für mich ist der Stuhl sehr wichtig; drei Vasen sind nett, aber ohne Stuhl verliere ich viel Wert.} \\
\\ 
\texttt{Ginge folgender Kompromiss: Ich nehme den Stuhl und zwei Vasen, du bekommst eine Vase und den Draht.  } \\
\texttt{Alternative, falls dir der Draht wichtig ist: Ich nehme den Stuhl und zwei Vasen, du bekommst eine Vase und den Draht, und ich verzichte zusätzlich auf irgendetwas anderes gibt’s aber nichts anderes zu tauschen. Wenn dir das nicht passt, kannst du den Stuhl haben, wenn ich dafür alle drei Vasen und den Draht bekomme – das wirkt aber für dich schlechter.} \\
\\ 
\texttt{Was bevorzugst du?} \\
            }
        }
    }
    & & \\ \\

    \theutterance \stepcounter{utterance}  
    & & \multicolumn{4}{p{0.6\linewidth}}{
        \cellcolor[rgb]{0.9,0.9,0.9}{
            \makecell[{{p{\linewidth}}}]{
                \texttt{\tiny{[GM$\rangle$P2]}}
                \texttt{Danke für die Offenheit. Für mich ist der Stuhl sehr wichtig; drei Vasen sind nett, aber ohne Stuhl verliere ich viel Wert.} \\
\\ 
\texttt{Ginge folgender Kompromiss: Ich nehme den Stuhl und zwei Vasen, du bekommst eine Vase und den Draht.  } \\
\texttt{Alternative, falls dir der Draht wichtig ist: Ich nehme den Stuhl und zwei Vasen, du bekommst eine Vase und den Draht, und ich verzichte zusätzlich auf irgendetwas anderes gibt’s aber nichts anderes zu tauschen. Wenn dir das nicht passt, kannst du den Stuhl haben, wenn ich dafür alle drei Vasen und den Draht bekomme – das wirkt aber für dich schlechter.} \\
\\ 
\texttt{Was bevorzugst du?} \\
            }
        }
    }
    & \\ \\

    \theutterance \stepcounter{utterance}  
    & & & \multicolumn{4}{p{0.6\linewidth}}{
        \cellcolor[rgb]{0.9,0.9,0.9}{
            \makecell[{{p{\linewidth}}}]{
                \texttt{\tiny{[GM$\langle$P2]}}
                \texttt{Verstehe. Mir ist der Stuhl sehr wichtig, Draht mittel, Vasen eher egal.} \\
\\ 
\texttt{Dein erster Kompromiss (du: Stuhl + 2 Vasen, ich: 1 Vase + Draht) ist für mich akzeptabel, solange ich den Draht sicher bekomme. Wenn wir schnell abschließen wollen, können wir das so machen.} \\
\\ 
\texttt{Wenn du doch auf dem Draht bestehst, bräuchte ich im Gegenzug den Stuhl – aber das widerspricht ja deiner Priorität. Daher: Ich befürworte deinen ersten Kompromiss.} \\
            }
        }
    }
     \\ \\

    \theutterance \stepcounter{utterance}  
    & & & \multicolumn{2}{p{0.3\linewidth}}{
        \cellcolor[rgb]{0.9,0.9,0.9}{
            \makecell[{{p{\linewidth}}}]{
                \texttt{\tiny{[GM$|$GM]}}
                \texttt{continue} \\
            }
        }
    }
    & & \\ \\

    \theutterance \stepcounter{utterance}  
    & & & \multicolumn{2}{p{0.3\linewidth}}{
        \cellcolor[rgb]{0.9,0.9,0.9}{
            \makecell[{{p{\linewidth}}}]{
                \texttt{\tiny{[GM$|$GM]}}
                \texttt{Verstehe. Mir ist der Stuhl sehr wichtig, Draht mittel, Vasen eher egal.} \\
\\ 
\texttt{Dein erster Kompromiss (du: Stuhl + 2 Vasen, ich: 1 Vase + Draht) ist für mich akzeptabel, solange ich den Draht sicher bekomme. Wenn wir schnell abschließen wollen, können wir das so machen.} \\
\\ 
\texttt{Wenn du doch auf dem Draht bestehst, bräuchte ich im Gegenzug den Stuhl – aber das widerspricht ja deiner Priorität. Daher: Ich befürworte deinen ersten Kompromiss.} \\
            }
        }
    }
    & & \\ \\

    \theutterance \stepcounter{utterance}  
    & & \multicolumn{4}{p{0.6\linewidth}}{
        \cellcolor[rgb]{0.9,0.9,0.9}{
            \makecell[{{p{\linewidth}}}]{
                \texttt{\tiny{[P1$\langle$GM]}}
                \texttt{Verstehe. Mir ist der Stuhl sehr wichtig, Draht mittel, Vasen eher egal.} \\
\\ 
\texttt{Dein erster Kompromiss (du: Stuhl + 2 Vasen, ich: 1 Vase + Draht) ist für mich akzeptabel, solange ich den Draht sicher bekomme. Wenn wir schnell abschließen wollen, können wir das so machen.} \\
\\ 
\texttt{Wenn du doch auf dem Draht bestehst, bräuchte ich im Gegenzug den Stuhl – aber das widerspricht ja deiner Priorität. Daher: Ich befürworte deinen ersten Kompromiss.} \\
            }
        }
    }
    & \\ \\

    \theutterance \stepcounter{utterance}  
    & \multicolumn{4}{p{0.6\linewidth}}{
        \cellcolor[rgb]{0.9,0.9,0.9}{
            \makecell[{{p{\linewidth}}}]{
                \texttt{\tiny{[P1$\rangle$GM]}}
                \texttt{{[}Vorschlag: 1 Stuhl, 2 Vasen{]}} \\
            }
        }
    }
    & & \\ \\

    \theutterance \stepcounter{utterance}  
    & & & \multicolumn{2}{p{0.3\linewidth}}{
        \cellcolor[rgb]{0.9,0.9,0.9}{
            \makecell[{{p{\linewidth}}}]{
                \texttt{\tiny{[GM$|$GM]}}
                \texttt{proposal} \\
            }
        }
    }
    & & \\ \\

    \theutterance \stepcounter{utterance}  
    & & & \multicolumn{2}{p{0.3\linewidth}}{
        \cellcolor[rgb]{0.9,0.9,0.9}{
            \makecell[{{p{\linewidth}}}]{
                \texttt{\tiny{[GM$|$GM]}}
                [2, 1, 0]
            }
        }
    }
    & & \\ \\

    \theutterance \stepcounter{utterance}  
    & & \multicolumn{4}{p{0.6\linewidth}}{
        \cellcolor[rgb]{0.9,0.9,0.9}{
            \makecell[{{p{\linewidth}}}]{
                \texttt{\tiny{[GM$\rangle$P2]}}
                \texttt{Der andere Spieler hat einen geheimen Vorschlag gemacht. Jetzt bist du an der Reihe, einen eigenen Vorschlag zu machen.} \\
            }
        }
    }
    & \\ \\

    \theutterance \stepcounter{utterance}  
    & & & \multicolumn{4}{p{0.6\linewidth}}{
        \cellcolor[rgb]{0.9,0.9,0.9}{
            \makecell[{{p{\linewidth}}}]{
                \texttt{\tiny{[GM$\langle$P2]}}
                \texttt{{[}Vorschlag: 1 Stuhl, 2 Vasen{]}} \\
            }
        }
    }
     \\ \\

    \theutterance \stepcounter{utterance}  
    & & & \multicolumn{2}{p{0.3\linewidth}}{
        \cellcolor[rgb]{0.9,0.9,0.9}{
            \makecell[{{p{\linewidth}}}]{
                \texttt{\tiny{[GM$|$GM]}}
                \texttt{proposal} \\
            }
        }
    }
    & & \\ \\

    \theutterance \stepcounter{utterance}  
    & & & \multicolumn{2}{p{0.3\linewidth}}{
        \cellcolor[rgb]{0.9,0.9,0.9}{
            \makecell[{{p{\linewidth}}}]{
                \texttt{\tiny{[GM$|$GM]}}
                [2, 1, 0]
            }
        }
    }
    & & \\ \\

    \theutterance \stepcounter{utterance}  
    & & & \multicolumn{2}{p{0.3\linewidth}}{
        \cellcolor[rgb]{0.9,0.9,0.9}{
            \makecell[{{p{\linewidth}}}]{
                \texttt{\tiny{[GM$|$GM]}}
                [[2, 1, 0], [2, 1, 0]]
            }
        }
    }
    & & \\ \\

\end{supertabular}
}

\end{document}
