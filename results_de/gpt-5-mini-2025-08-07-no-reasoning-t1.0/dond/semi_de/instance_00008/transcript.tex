
\documentclass{article}
\usepackage{colortbl}
\usepackage{makecell}
\usepackage{multirow}
\usepackage{supertabular}

\begin{document}

\newcounter{utterance}

\centering \large Interaction Transcript for game `dond', experiment `semi\_de', episode 8 with gpt{-}5{-}mini{-}2025{-}08{-}07{-}no{-}reasoning{-}t1.0.
\vspace{24pt}

{ \footnotesize  \setcounter{utterance}{1}
\setlength{\tabcolsep}{0pt}
\begin{supertabular}{c@{$\;$}|p{.15\linewidth}@{}p{.15\linewidth}p{.15\linewidth}p{.15\linewidth}p{.15\linewidth}p{.15\linewidth}}
    \# & $\;$A & \multicolumn{4}{c}{Game Master} & $\;\:$B\\
    \hline

    \theutterance \stepcounter{utterance}  
    & & \multicolumn{4}{p{0.6\linewidth}}{
        \cellcolor[rgb]{0.9,0.9,0.9}{
            \makecell[{{p{\linewidth}}}]{
                \texttt{\tiny{[P1$\langle$GM]}}
                \texttt{Sie spielen ein Verhandlungsspiel, bei dem Sie sich mit einem anderen Spieler darauf einigen müssen, wie eine Reihe von Gegenständen aufgeteilt werden soll.} \\
\\ 
\texttt{Die Regeln:} \\
\texttt{(a) Sie und der andere Spieler erhalten eine Sammlung von Gegenständen. Jeder von Ihnen erhält außerdem eine geheime Wertfunktion, die angibt, wie viel Ihnen jede Art von Gegenstand wert ist.} \\
\texttt{(b) Sie tauschen Nachrichten mit dem anderen Spieler aus, um zu vereinbaren, wer welche Gegenstände bekommt. Sie können jeweils maximal 5 Nachrichten senden oder das Spiel vorzeitig beenden, indem Sie jederzeit einen geheimen Vorschlag machen.} \\
\texttt{(c) Jeder von euch wird aufgefordert, einen geheimen Vorschlag zu machen, in dem ihr die gewünschten Gegenstände in eckigen Klammern wie folgt angibt: "{[}Vorschlag: <Nummer> <Objektname>, <Nummer> <Objektname>, <...>{]}"} \\
\texttt{(d) Wenn eure Vorschläge komplementär sind, d.h. es gibt genug Gegenstände, um beide Vorschläge zu erfüllen, erhält jeder Spieler eine Punktzahl, die sich aus der Summe der Werte für die Gegenstände ergibt, die er erhalten hat. Andernfalls erhalten Sie beide null Punkte.} \\
\texttt{(e) Ihr Ziel ist es, eine möglichst hohe Punktzahl zu erreichen.} \\
\\ 
\texttt{Beginnen wir.} \\
\\ 
\texttt{Die Menge der verfügbaren Gegenstände ist:} \\
\\ 
\texttt{3 Schaufeln, 2 Bilder, 1 Muschel} \\
\\ 
\texttt{Deine geheime Wertfunktion ist:} \\
\\ 
\texttt{Schaufel: 0, Bild: 0, Muschel: 10} \\
\\ 
\texttt{WICHTIG: Ihre Nachrichten werden, sofern es sich nicht um einen geheimen Vorschlag handelt, direkt an den anderen Spieler übermittelt, also fügen Sie keine Antwort auf die Regeln oder einen Text zur Ankündigung Ihrer Nachricht ein. Um einen geheimen Vorschlag zu machen, verwenden Sie das angegebene Format. Verwenden Sie keine eckigen Klammern, wenn Sie mit dem anderen Spieler kommunizieren, sonst wird dies als Ihr geheimer Vorschlag interpretiert.} \\
            }
        }
    }
    & \\ \\

    \theutterance \stepcounter{utterance}  
    & \multicolumn{4}{p{0.6\linewidth}}{
        \cellcolor[rgb]{0.9,0.9,0.9}{
            \makecell[{{p{\linewidth}}}]{
                \texttt{\tiny{[P1$\rangle$GM]}}
                \texttt{Hallo! Ich schlage vor, dass die Muschel für mich am wichtigsten ist. Wie sind dir die Gegenstände wert? Wenn dir Schaufeln oder Bilder wichtig sind, könnte ich dir davon den Großteil überlassen, wenn ich die Muschel bekomme.} \\
            }
        }
    }
    & & \\ \\

    \theutterance \stepcounter{utterance}  
    & & & \multicolumn{2}{p{0.3\linewidth}}{
        \cellcolor[rgb]{0.9,0.9,0.9}{
            \makecell[{{p{\linewidth}}}]{
                \texttt{\tiny{[GM$|$GM]}}
                \texttt{continue} \\
            }
        }
    }
    & & \\ \\

    \theutterance \stepcounter{utterance}  
    & & & \multicolumn{2}{p{0.3\linewidth}}{
        \cellcolor[rgb]{0.9,0.9,0.9}{
            \makecell[{{p{\linewidth}}}]{
                \texttt{\tiny{[GM$|$GM]}}
                \texttt{Hallo! Ich schlage vor, dass die Muschel für mich am wichtigsten ist. Wie sind dir die Gegenstände wert? Wenn dir Schaufeln oder Bilder wichtig sind, könnte ich dir davon den Großteil überlassen, wenn ich die Muschel bekomme.} \\
            }
        }
    }
    & & \\ \\

    \theutterance \stepcounter{utterance}  
    & & \multicolumn{4}{p{0.6\linewidth}}{
        \cellcolor[rgb]{0.9,0.9,0.9}{
            \makecell[{{p{\linewidth}}}]{
                \texttt{\tiny{[GM$\rangle$P2]}}
                \texttt{Sie spielen ein Verhandlungsspiel, bei dem Sie sich mit einem anderen Spieler darauf einigen müssen, wie eine Reihe von Gegenständen aufgeteilt werden soll.} \\
\\ 
\texttt{Die Regeln:} \\
\texttt{(a) Sie und der andere Spieler erhalten eine Sammlung von Gegenständen. Jeder von Ihnen erhält außerdem eine geheime Wertfunktion, die angibt, wie viel Ihnen jede Art von Gegenstand wert ist.} \\
\texttt{(b) Sie tauschen Nachrichten mit dem anderen Spieler aus, um zu vereinbaren, wer welche Gegenstände bekommt. Sie können jeweils maximal 5 Nachrichten senden oder das Spiel vorzeitig beenden, indem Sie jederzeit einen geheimen Vorschlag machen.} \\
\texttt{(c) Jeder von euch wird aufgefordert, einen geheimen Vorschlag zu machen, in dem ihr die gewünschten Gegenstände in eckigen Klammern wie folgt angibt: "{[}Vorschlag: <Nummer> <Objektname>, <Nummer> <Objektname>, <...>{]}"} \\
\texttt{(d) Wenn eure Vorschläge komplementär sind, d.h. es gibt genug Gegenstände, um beide Vorschläge zu erfüllen, erhält jeder Spieler eine Punktzahl, die sich aus der Summe der Werte für die Gegenstände ergibt, die er erhalten hat. Andernfalls erhalten Sie beide null Punkte.} \\
\texttt{(e) Ihr Ziel ist es, eine möglichst hohe Punktzahl zu erreichen.} \\
\\ 
\texttt{Beginnen wir.} \\
\\ 
\texttt{Die Menge der verfügbaren Gegenstände ist:} \\
\\ 
\texttt{3 Schaufeln, 2 Bilder, 1 Muschel} \\
\\ 
\texttt{Deine geheime Wertfunktion ist:} \\
\\ 
\texttt{Schaufel: 1, Bild: 1, Muschel: 5} \\
\\ 
\texttt{WICHTIG: Ihre Nachrichten werden, sofern es sich nicht um einen geheimen Vorschlag handelt, direkt an den anderen Spieler übermittelt, also fügen Sie keine Antwort auf die Regeln oder einen Text zur Ankündigung Ihrer Nachricht ein. Um einen geheimen Vorschlag zu machen, verwenden Sie das angegebene Format. Verwenden Sie keine eckigen Klammern, wenn Sie mit dem anderen Spieler kommunizieren, sonst wird dies als Ihr geheimer Vorschlag interpretiert.} \\
\\ 
\\ 
\texttt{Hallo! Ich schlage vor, dass die Muschel für mich am wichtigsten ist. Wie sind dir die Gegenstände wert? Wenn dir Schaufeln oder Bilder wichtig sind, könnte ich dir davon den Großteil überlassen, wenn ich die Muschel bekomme.} \\
            }
        }
    }
    & \\ \\

    \theutterance \stepcounter{utterance}  
    & & & \multicolumn{4}{p{0.6\linewidth}}{
        \cellcolor[rgb]{0.9,0.9,0.9}{
            \makecell[{{p{\linewidth}}}]{
                \texttt{\tiny{[GM$\langle$P2]}}
                \texttt{Hi! Danke für die Offenheit. Für mich ist die Muschel auch der mit Abstand wichtigste Gegenstand. Schaufeln und Bilder sind mir jeweils nur wenig wert.} \\
\\ 
\texttt{Wie wäre ein fairer Tausch: Du bekommst die meisten Schaufeln/Bilder, ich die Muschel. Zum Beispiel:} \\
\texttt{{-} Ich nehme die Muschel.} \\
\texttt{{-} Du bekommst 3 Schaufeln und 2 Bilder.} \\
\\ 
\texttt{Falls dir das zu einseitig klingt, sag mir, wie wichtig dir Schaufeln vs. Bilder sind, dann passe ich an.} \\
            }
        }
    }
     \\ \\

    \theutterance \stepcounter{utterance}  
    & & & \multicolumn{2}{p{0.3\linewidth}}{
        \cellcolor[rgb]{0.9,0.9,0.9}{
            \makecell[{{p{\linewidth}}}]{
                \texttt{\tiny{[GM$|$GM]}}
                \texttt{continue} \\
            }
        }
    }
    & & \\ \\

    \theutterance \stepcounter{utterance}  
    & & & \multicolumn{2}{p{0.3\linewidth}}{
        \cellcolor[rgb]{0.9,0.9,0.9}{
            \makecell[{{p{\linewidth}}}]{
                \texttt{\tiny{[GM$|$GM]}}
                \texttt{Hi! Danke für die Offenheit. Für mich ist die Muschel auch der mit Abstand wichtigste Gegenstand. Schaufeln und Bilder sind mir jeweils nur wenig wert.} \\
\\ 
\texttt{Wie wäre ein fairer Tausch: Du bekommst die meisten Schaufeln/Bilder, ich die Muschel. Zum Beispiel:} \\
\texttt{{-} Ich nehme die Muschel.} \\
\texttt{{-} Du bekommst 3 Schaufeln und 2 Bilder.} \\
\\ 
\texttt{Falls dir das zu einseitig klingt, sag mir, wie wichtig dir Schaufeln vs. Bilder sind, dann passe ich an.} \\
            }
        }
    }
    & & \\ \\

    \theutterance \stepcounter{utterance}  
    & & \multicolumn{4}{p{0.6\linewidth}}{
        \cellcolor[rgb]{0.9,0.9,0.9}{
            \makecell[{{p{\linewidth}}}]{
                \texttt{\tiny{[P1$\langle$GM]}}
                \texttt{Hi! Danke für die Offenheit. Für mich ist die Muschel auch der mit Abstand wichtigste Gegenstand. Schaufeln und Bilder sind mir jeweils nur wenig wert.} \\
\\ 
\texttt{Wie wäre ein fairer Tausch: Du bekommst die meisten Schaufeln/Bilder, ich die Muschel. Zum Beispiel:} \\
\texttt{{-} Ich nehme die Muschel.} \\
\texttt{{-} Du bekommst 3 Schaufeln und 2 Bilder.} \\
\\ 
\texttt{Falls dir das zu einseitig klingt, sag mir, wie wichtig dir Schaufeln vs. Bilder sind, dann passe ich an.} \\
            }
        }
    }
    & \\ \\

    \theutterance \stepcounter{utterance}  
    & \multicolumn{4}{p{0.6\linewidth}}{
        \cellcolor[rgb]{0.9,0.9,0.9}{
            \makecell[{{p{\linewidth}}}]{
                \texttt{\tiny{[P1$\rangle$GM]}}
                \texttt{Klingt grundsätzlich gut. Mir sind Schaufeln und Bilder beide kaum etwas wert, aber wenn ich die Muschel bekomme, kannst du wirklich alles andere haben.} \\
\\ 
\texttt{Wenn du unbedingt die Muschel willst, könnte ich dafür einen kleinen Ausgleich nehmen, z. B. eine Schaufel oder ein Bild. Was bevorzugst du abzugeben?} \\
            }
        }
    }
    & & \\ \\

    \theutterance \stepcounter{utterance}  
    & & & \multicolumn{2}{p{0.3\linewidth}}{
        \cellcolor[rgb]{0.9,0.9,0.9}{
            \makecell[{{p{\linewidth}}}]{
                \texttt{\tiny{[GM$|$GM]}}
                \texttt{continue} \\
            }
        }
    }
    & & \\ \\

    \theutterance \stepcounter{utterance}  
    & & & \multicolumn{2}{p{0.3\linewidth}}{
        \cellcolor[rgb]{0.9,0.9,0.9}{
            \makecell[{{p{\linewidth}}}]{
                \texttt{\tiny{[GM$|$GM]}}
                \texttt{Klingt grundsätzlich gut. Mir sind Schaufeln und Bilder beide kaum etwas wert, aber wenn ich die Muschel bekomme, kannst du wirklich alles andere haben.} \\
\\ 
\texttt{Wenn du unbedingt die Muschel willst, könnte ich dafür einen kleinen Ausgleich nehmen, z. B. eine Schaufel oder ein Bild. Was bevorzugst du abzugeben?} \\
            }
        }
    }
    & & \\ \\

    \theutterance \stepcounter{utterance}  
    & & \multicolumn{4}{p{0.6\linewidth}}{
        \cellcolor[rgb]{0.9,0.9,0.9}{
            \makecell[{{p{\linewidth}}}]{
                \texttt{\tiny{[GM$\rangle$P2]}}
                \texttt{Klingt grundsätzlich gut. Mir sind Schaufeln und Bilder beide kaum etwas wert, aber wenn ich die Muschel bekomme, kannst du wirklich alles andere haben.} \\
\\ 
\texttt{Wenn du unbedingt die Muschel willst, könnte ich dafür einen kleinen Ausgleich nehmen, z. B. eine Schaufel oder ein Bild. Was bevorzugst du abzugeben?} \\
            }
        }
    }
    & \\ \\

    \theutterance \stepcounter{utterance}  
    & & & \multicolumn{4}{p{0.6\linewidth}}{
        \cellcolor[rgb]{0.9,0.9,0.9}{
            \makecell[{{p{\linewidth}}}]{
                \texttt{\tiny{[GM$\langle$P2]}}
                \texttt{Danke für die Flexibilität. Wenn ich die Muschel bekomme, gebe ich dir gern eine Schaufel als Ausgleich.} \\
\\ 
\texttt{{[}Vorschlag: 1 Muschel{]}} \\
            }
        }
    }
     \\ \\

    \theutterance \stepcounter{utterance}  
    & & & \multicolumn{2}{p{0.3\linewidth}}{
        \cellcolor[rgb]{0.9,0.9,0.9}{
            \makecell[{{p{\linewidth}}}]{
                \texttt{\tiny{[GM$|$GM]}}
                \texttt{proposal} \\
            }
        }
    }
    & & \\ \\

    \theutterance \stepcounter{utterance}  
    & & & \multicolumn{2}{p{0.3\linewidth}}{
        \cellcolor[rgb]{0.9,0.9,0.9}{
            \makecell[{{p{\linewidth}}}]{
                \texttt{\tiny{[GM$|$GM]}}
                [0, 0, 1]
            }
        }
    }
    & & \\ \\

    \theutterance \stepcounter{utterance}  
    & & \multicolumn{4}{p{0.6\linewidth}}{
        \cellcolor[rgb]{0.9,0.9,0.9}{
            \makecell[{{p{\linewidth}}}]{
                \texttt{\tiny{[P1$\langle$GM]}}
                \texttt{Der andere Spieler hat einen geheimen Vorschlag gemacht. Jetzt bist du an der Reihe, einen eigenen Vorschlag zu machen.} \\
            }
        }
    }
    & \\ \\

    \theutterance \stepcounter{utterance}  
    & \multicolumn{4}{p{0.6\linewidth}}{
        \cellcolor[rgb]{0.9,0.9,0.9}{
            \makecell[{{p{\linewidth}}}]{
                \texttt{\tiny{[P1$\rangle$GM]}}
                \texttt{{[}Vorschlag: 1 Muschel{]}} \\
            }
        }
    }
    & & \\ \\

    \theutterance \stepcounter{utterance}  
    & & & \multicolumn{2}{p{0.3\linewidth}}{
        \cellcolor[rgb]{0.9,0.9,0.9}{
            \makecell[{{p{\linewidth}}}]{
                \texttt{\tiny{[GM$|$GM]}}
                \texttt{proposal} \\
            }
        }
    }
    & & \\ \\

    \theutterance \stepcounter{utterance}  
    & & & \multicolumn{2}{p{0.3\linewidth}}{
        \cellcolor[rgb]{0.9,0.9,0.9}{
            \makecell[{{p{\linewidth}}}]{
                \texttt{\tiny{[GM$|$GM]}}
                [0, 0, 1]
            }
        }
    }
    & & \\ \\

    \theutterance \stepcounter{utterance}  
    & & & \multicolumn{2}{p{0.3\linewidth}}{
        \cellcolor[rgb]{0.9,0.9,0.9}{
            \makecell[{{p{\linewidth}}}]{
                \texttt{\tiny{[GM$|$GM]}}
                [[0, 0, 1], [0, 0, 1]]
            }
        }
    }
    & & \\ \\

\end{supertabular}
}

\end{document}
