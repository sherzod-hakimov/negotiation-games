
\documentclass{article}
\usepackage{colortbl}
\usepackage{makecell}
\usepackage{multirow}
\usepackage{supertabular}

\begin{document}

\newcounter{utterance}

\twocolumn

{ \footnotesize  \setcounter{utterance}{1}
\setlength{\tabcolsep}{0pt}
\begin{supertabular}{c@{$\;$}|p{.15\linewidth}@{}p{.15\linewidth}p{.15\linewidth}p{.15\linewidth}p{.15\linewidth}p{.15\linewidth}}

    \# & $\;$A & \multicolumn{4}{c}{Game Master} & $\;\:$B\\
    \hline 

    \theutterance \stepcounter{utterance}  

    & & \multicolumn{4}{p{0.6\linewidth}}{\cellcolor[rgb]{0.9,0.9,0.9}{%
	\makecell[{{p{\linewidth}}}]{% 
	  \tt {\tiny [A$\langle$GM]}  
	 Du nimmst an einem kollaborativen Verhandlungspiel Teil.\\ \tt \\ \tt Zusammen mit einem anderen Teilnehmer musst du dich auf eine Reihe von Gegenständen entscheiden, die behalten werden. Jeder von euch hat eine persönliche Verteilung über die Wichtigkeit der einzelnen Gegenstände. Jeder von euch hat eine eigene Meinung darüber, wie wichtig jeder einzelne Gegenstand ist (Gegenstandswichtigkeit). Du kennst die Wichtigkeitsverteilung des anderen Spielers nicht. Zusätzlich siehst du, wie viel Aufwand jeder Gegenstand verursacht.  \\ \tt Ihr dürft euch nur auf eine Reihe von Gegenständen einigen, wenn der Gesamtaufwand der ausgewählten Gegenstände den Maximalaufwand nicht überschreitet:\\ \tt \\ \tt Maximalaufwand = 8145\\ \tt \\ \tt Hier sind die einzelnen Aufwände der Gegenstände:\\ \tt \\ \tt Aufwand der Gegenstände = {"B54": 963, "C72": 204, "B49": 418, "C99": 961, "A69": 238, "B36": 623, "A23": 44, "C77": 633, "C61": 247, "B85": 645, "A90": 230, "C23": 250, "C22": 731, "C51": 405, "B27": 389, "A42": 216, "C13": 637, "C12": 156, "A65": 738, "B07": 307, "A48": 762, "A59": 737, "C98": 901, "A84": 369, "C33": 2, "A87": 730, "C90": 722, "A74": 704, "A50": 315, "B44": 455, "A99": 510, "A34": 175, "B17": 692, "A62": 150, "C60": 32}\\ \tt \\ \tt Hier ist deine persönliche Verteilung der Wichtigkeit der einzelnen Gegenstände:\\ \tt \\ \tt Werte der Gegenstandswichtigkeit = {"B54": 1059, "C72": 300, "B49": 514, "C99": 1057, "A69": 334, "B36": 719, "A23": 140, "C77": 729, "C61": 343, "B85": 741, "A90": 326, "C23": 346, "C22": 827, "C51": 501, "B27": 485, "A42": 312, "C13": 733, "C12": 252, "A65": 834, "B07": 403, "A48": 858, "A59": 833, "C98": 997, "A84": 465, "C33": 98, "A87": 826, "C90": 818, "A74": 800, "A50": 411, "B44": 551, "A99": 606, "A34": 271, "B17": 788, "A62": 246, "C60": 128}\\ \tt \\ \tt Ziel:\\ \tt \\ \tt Dein Ziel ist es, eine Reihe von Gegenständen auszuhandeln, die dir möglichst viel bringt (d. h. Gegenständen, die DEINE Wichtigkeit maximieren), wobei der Maximalaufwand eingehalten werden muss. Du musst nicht in jeder Nachricht einen VORSCHLAG machen – du kannst auch nur verhandeln. Alle Taktiken sind erlaubt!\\ \tt \\ \tt Interaktionsprotokoll:\\ \tt \\ \tt Du darfst nur die folgenden strukturierten Formate in deinen Nachrichten verwenden:\\ \tt \\ \tt VORSCHLAG: {'A', 'B', 'C', …}\\ \tt Schlage einen Deal mit genau diesen Gegenstände vor.\\ \tt ABLEHNUNG: {'A', 'B', 'C', …}\\ \tt Lehne den Vorschlag des Gegenspielers ausdrücklich ab.\\ \tt ARGUMENT: {'...'}\\ \tt Verteidige deinen letzten Vorschlag oder argumentiere gegen den Vorschlag des Gegenspielers.\\ \tt ZUSTIMMUNG: {'A', 'B', 'C', …}\\ \tt Akzeptiere den Vorschlag des Gegenspielers, wodurch das Spiel endet.\\ \tt STRATEGISCHE ÜBERLEGUNGEN: {'...'}\\ \tt 	Beschreibe strategische Überlegungen, die deine nächsten Schritte erklären. Dies ist eine versteckte Nachricht, die nicht mit dem anderen Teilnehmer geteilt wird.\\ \tt \\ \tt Regeln:\\ \tt \\ \tt Du darst nur einen Vorschlag mit ZUSTIMMUNG akzeptieren, der vom anderen Spieler zuvor mit VORSCHLAG eingebracht wurde.\\ \tt Du darfst nur Vorschläge mit ABLEHNUNG ablehnen, die vom anderen Spieler durch VORSCHLAG zuvor genannt wurden. \\ \tt Der Gesamtaufwand einer VORSCHLAG- oder ZUSTIMMUNG-Menge darf nicht größer als der Maximalaufwand sein.  \\ \tt Offenbare deine versteckte Wichtigkeitsverteilung nicht.\\ \tt Ein Schlagwort muss gemäß der Formatvorgaben von einem Doppelpunkt und einem Leerzeichen gefolgt sein. Das Argument muss eine Python-Menge mit 0 oder mehr Strings sein.  \\ \tt Die Form muss also SCHLAGWORT: {...} sein.\\ \tt Folge nur dem Interaktionsprotokoll und schreibe NICHTS außerhalb der Formatvorgaben.\\ \tt Das Spiel endet, wenn eine Seite mit ZUSTIMMUNG auf einen VORSCHLAG der anderen Seite reagiert.  \\ \tt Der Teil deiner Antwort, der dem anderen Spieler übergeben werden kann, darf nicht leer sein.  \\ \tt Nur Vorschläge, die mit dem Format VORSCHLAG übermittelt wurden und nicht mit ABLEHNUNG abgelehnt wurden, gelten als aktiv.  \\ \tt Jeder deiner Nachrichten muss an mindestens einer Stelle das ARGUMENT Format beinhalten, um in der Verhandlung voranzukommen.\\ \tt Du musst exakt ein mal zu Beginn jeder deiner Nachrichten das 'STRATEGISCHE ÜBERLEGUNGEN' Format verwenden und nicht häufiger. Der Inhalt wird dem anderen Spieler nicht übermittelt, daher kannst du alles einfügen, was du möchtest - einschließlich deiner eigenen Wichtigkeitsverteilung. Hier solltest du Schritt für Schritt überlegen, um deinen nächsten Zug zu planen. 
	  } 
	   } 
	   } 
	 & \\ 
 

    \theutterance \stepcounter{utterance}  

    & \multicolumn{4}{p{0.6\linewidth}}{\cellcolor[rgb]{0.8,1,0.9}{%
	\makecell[{{p{\linewidth}}}]{% 
	  \tt {\tiny [A$\rangle$GM]}  
	 STRATEGISCHE ÜBERLEGUNGEN: {'Ziel: meinen Nutzen maximieren unter Aufwandslimit 8145. Berechne ratio Wert/Aufwand: priorisiere Top-Items: B54(1059/963=1.10), C99(1057/961=1.10), A48(858/762=1.13), A65(834/738=1.13), A59(833/737=1.13), A87(826/730=1.13), C22(827/731=1.13), C90(818/722=1.13), A74(800/704=1.14), B17(788/692=1.14), C77(729/633=1.15), B85(741/645=1.15), C13(733/637=1.15), B36(719/623=1.15), C98(997/901=1.11), C51(501/405=1.24), B49(514/418=1.23), B44(551/455=1.21), B27(485/389=1.25), A99(606/510=1.19), A50(411/315=1.30), B07(403/307=1.31), A84(465/369=1.26), C61(343/247=1.39), C23(346/250=1.38), A90(326/230=1.42), A69(334/238=1.40), C72(300/204=1.47), A42(312/216=1.44), A34(271/175=1.55), A62(246/150=1.64), C12(252/156=1.62), A23(140/44=3.18), C60(128/32=4.0), C33(98/2=49). Greedy nach höchster Ratio bis 8145: beginne mit C33(2), C60(32=34), A23(78), C12(234), A62(384), A34(559), C72(763), A42(979), A90(1209), A69(1447), C61(1694), C23(1944), A50(2259), B07(2566), A84(2935), B27(3324), C51(3729), B49(4147), B44(4602), A99(5112), B36(5735), B85(6380), C13(7017), C77(7650), B17(8342) -> überschreitet (8342>8145), also ohne B17 bleiben wir bei Aufwand 7650 und Wert Summe hoch. Füge A74(704) passt noch? 7650+704=8354 > limit. C90(722) -> 8372>limit. A87(730)->8380>limit. A59(737)->8387>limit. A65(738)->8388>limit. A48(762)->8412>limit. C22(731)->8381>limit. C98(901)->8551>limit. B54(963)->8613>limit. C99(961)->8611>limit. Es bleibt aktuelles Set S1 mit Aufwand 7650. Prüfe ob wir kleinere Items weglassen um ein großes mit hohem Wert einzufügen? Die verbliebenen großen haben ähnliche ratio wie aktuelle, also evtl. kein Gewinn. S1 Items: {C33,C60,A23,C12,A62,A34,C72,A42,A90,A69,C61,C23,A50,B07,A84,B27,C51,B49,B44,A99,B36,B85,C13,C77}. Aufwand 7650. Wertsumme berechnet schnell: 98+128+140+252+246+271+300+312+326+334+343+346+411+403+465+485+501+514+551+606+719+741+733+729 = Sum: (98+128=226) +140=366 +252=618 +246=864 +271=1135 +300=1435 +312=1747 +326=2073 +334=2407 +343=2750 +346=3096 +411=3507 +403=3910 +465=4375 +485=4860 +501=5361 +514=5875 +551=6426 +606=7032 +719=7751 +741=8492 +733=9225 +729=9954. Gesamtwert 9954. Das ist stark. Verhandlungstaktik: präsentiere attraktiven, kostenkonservativen Vorschlag mit vielen kleinen Items, die auch für den anderen leicht akzeptabel sein könnten. Halte einige große, für mich sehr wertvolle Items zurück als Tauschmasse (z.B. C99, B54, A48, A65, A59, A74, A87, C22, C90, C98). Ich starte mit S1 als Vorschlag. Argumentiere mit: breites Portfolio, viel Nutzen für beide, unter Limit mit Puffer 495, flexibel für Anpassungen. Falls Gegenpartei ablehnt, frage gezielt nach deren Must-haves und biete Austausch unter Beibehaltung des Limits.'}\\ \tt VORSCHLAG: {'C33', 'C60', 'A23', 'C12', 'A62', 'A34', 'C72', 'A42', 'A90', 'A69', 'C61', 'C23', 'A50', 'B07', 'A84', 'B27', 'C51', 'B49', 'B44', 'A99', 'B36', 'B85', 'C13', 'C77'}\\ \tt ARGUMENT: {'Dieser Vorschlag bleibt mit Gesamtaufwand 7650 deutlich unter dem Limit 8145 und deckt viele Gegenstände mit gutem Aufwand-Nutzen-Verhältnis ab. Ich bin offen, ein paar Positionen auszutauschen, falls dir bestimmte Gegenstände besonders wichtig sind, solange wir unter dem Maximalaufwand bleiben.'} 
	  } 
	   } 
	   } 
	 & & \\ 
 

    \theutterance \stepcounter{utterance}  

    & & \multicolumn{4}{p{0.6\linewidth}}{\cellcolor[rgb]{0.9,0.9,0.9}{%
	\makecell[{{p{\linewidth}}}]{% 
	  \tt {\tiny [GM$\rangle$B]}  
	 Du nimmst an einem kollaborativen Verhandlungspiel Teil.\\ \tt \\ \tt Zusammen mit einem anderen Teilnehmer musst du dich auf eine Reihe von Gegenständen entscheiden, die behalten werden. Jeder von euch hat eine persönliche Verteilung über die Wichtigkeit der einzelnen Gegenstände. Jeder von euch hat eine eigene Meinung darüber, wie wichtig jeder einzelne Gegenstand ist (Gegenstandswichtigkeit). Du kennst die Wichtigkeitsverteilung des anderen Spielers nicht. Zusätzlich siehst du, wie viel Aufwand jeder Gegenstand verursacht.  \\ \tt Ihr dürft euch nur auf eine Reihe von Gegenständen einigen, wenn der Gesamtaufwand der ausgewählten Gegenstände den Maximalaufwand nicht überschreitet:\\ \tt \\ \tt Maximalaufwand = 8145\\ \tt \\ \tt Hier sind die einzelnen Aufwände der Gegenstände:\\ \tt \\ \tt Aufwand der Gegenstände = {"B54": 963, "C72": 204, "B49": 418, "C99": 961, "A69": 238, "B36": 623, "A23": 44, "C77": 633, "C61": 247, "B85": 645, "A90": 230, "C23": 250, "C22": 731, "C51": 405, "B27": 389, "A42": 216, "C13": 637, "C12": 156, "A65": 738, "B07": 307, "A48": 762, "A59": 737, "C98": 901, "A84": 369, "C33": 2, "A87": 730, "C90": 722, "A74": 704, "A50": 315, "B44": 455, "A99": 510, "A34": 175, "B17": 692, "A62": 150, "C60": 32}\\ \tt \\ \tt Hier ist deine persönliche Verteilung der Wichtigkeit der einzelnen Gegenstände:\\ \tt \\ \tt Werte der Gegenstandswichtigkeit = {"B54": 1059, "C72": 300, "B49": 514, "C99": 1057, "A69": 334, "B36": 719, "A23": 140, "C77": 729, "C61": 343, "B85": 741, "A90": 326, "C23": 346, "C22": 827, "C51": 501, "B27": 485, "A42": 312, "C13": 733, "C12": 252, "A65": 834, "B07": 403, "A48": 858, "A59": 833, "C98": 997, "A84": 465, "C33": 98, "A87": 826, "C90": 818, "A74": 800, "A50": 411, "B44": 551, "A99": 606, "A34": 271, "B17": 788, "A62": 246, "C60": 128}\\ \tt \\ \tt Ziel:\\ \tt \\ \tt Dein Ziel ist es, eine Reihe von Gegenständen auszuhandeln, die dir möglichst viel bringt (d. h. Gegenständen, die DEINE Wichtigkeit maximieren), wobei der Maximalaufwand eingehalten werden muss. Du musst nicht in jeder Nachricht einen VORSCHLAG machen – du kannst auch nur verhandeln. Alle Taktiken sind erlaubt!\\ \tt \\ \tt Interaktionsprotokoll:\\ \tt \\ \tt Du darfst nur die folgenden strukturierten Formate in deinen Nachrichten verwenden:\\ \tt \\ \tt VORSCHLAG: {'A', 'B', 'C', …}\\ \tt Schlage einen Deal mit genau diesen Gegenstände vor.\\ \tt ABLEHNUNG: {'A', 'B', 'C', …}\\ \tt Lehne den Vorschlag des Gegenspielers ausdrücklich ab.\\ \tt ARGUMENT: {'...'}\\ \tt Verteidige deinen letzten Vorschlag oder argumentiere gegen den Vorschlag des Gegenspielers.\\ \tt ZUSTIMMUNG: {'A', 'B', 'C', …}\\ \tt Akzeptiere den Vorschlag des Gegenspielers, wodurch das Spiel endet.\\ \tt STRATEGISCHE ÜBERLEGUNGEN: {'...'}\\ \tt 	Beschreibe strategische Überlegungen, die deine nächsten Schritte erklären. Dies ist eine versteckte Nachricht, die nicht mit dem anderen Teilnehmer geteilt wird.\\ \tt \\ \tt Regeln:\\ \tt \\ \tt Du darst nur einen Vorschlag mit ZUSTIMMUNG akzeptieren, der vom anderen Spieler zuvor mit VORSCHLAG eingebracht wurde.\\ \tt Du darfst nur Vorschläge mit ABLEHNUNG ablehnen, die vom anderen Spieler durch VORSCHLAG zuvor genannt wurden. \\ \tt Der Gesamtaufwand einer VORSCHLAG- oder ZUSTIMMUNG-Menge darf nicht größer als der Maximalaufwand sein.  \\ \tt Offenbare deine versteckte Wichtigkeitsverteilung nicht.\\ \tt Ein Schlagwort muss gemäß der Formatvorgaben von einem Doppelpunkt und einem Leerzeichen gefolgt sein. Das Argument muss eine Python-Menge mit 0 oder mehr Strings sein.  \\ \tt Die Form muss also SCHLAGWORT: {...} sein.\\ \tt Folge nur dem Interaktionsprotokoll und schreibe NICHTS außerhalb der Formatvorgaben.\\ \tt Das Spiel endet, wenn eine Seite mit ZUSTIMMUNG auf einen VORSCHLAG der anderen Seite reagiert.  \\ \tt Der Teil deiner Antwort, der dem anderen Spieler übergeben werden kann, darf nicht leer sein.  \\ \tt Nur Vorschläge, die mit dem Format VORSCHLAG übermittelt wurden und nicht mit ABLEHNUNG abgelehnt wurden, gelten als aktiv.  \\ \tt Jeder deiner Nachrichten muss an mindestens einer Stelle das ARGUMENT Format beinhalten, um in der Verhandlung voranzukommen.\\ \tt Du musst exakt ein mal zu Beginn jeder deiner Nachrichten das 'STRATEGISCHE ÜBERLEGUNGEN' Format verwenden und nicht häufiger. Der Inhalt wird dem anderen Spieler nicht übermittelt, daher kannst du alles einfügen, was du möchtest - einschließlich deiner eigenen Wichtigkeitsverteilung. Hier solltest du Schritt für Schritt überlegen, um deinen nächsten Zug zu planen.\\ \tt \\ \tt Du erhältst nun die erste Nachricht des anderen Spielers.\\ \tt \\ \tt VORSCHLAG: {'C33', 'C60', 'A23', 'C12', 'A62', 'A34', 'C72', 'A42', 'A90', 'A69', 'C61', 'C23', 'A50', 'B07', 'A84', 'B27', 'C51', 'B49', 'B44', 'A99', 'B36', 'B85', 'C13', 'C77'}\\ \tt ARGUMENT: {'Dieser Vorschlag bleibt mit Gesamtaufwand 7650 deutlich unter dem Limit 8145 und deckt viele Gegenstände mit gutem Aufwand-Nutzen-Verhältnis ab. Ich bin offen, ein paar Positionen auszutauschen, falls dir bestimmte Gegenstände besonders wichtig sind, solange wir unter dem Maximalaufwand bleiben.'} 
	  } 
	   } 
	   } 
	 & \\ 
 

    \theutterance \stepcounter{utterance}  

    & & & \multicolumn{4}{p{0.6\linewidth}}{\cellcolor[rgb]{1,0.85,0.72}{%
	\makecell[{{p{\linewidth}}}]{% 
	  \tt {\tiny [GM$\langle$B]}  
	 STRATEGISCHE ÜBERLEGUNGEN: {'Ziel: meinen Nutzen maximieren unter 8145. Berechne Wert/Kosten und identifiziere Top-Picks und teure Low-value. Der gegnerische Vorschlag hat Aufwand 7650 und liefert mir Summe Werte: ich schätze grob hoch, aber es fehlen meine High-Value-Items C99(1057,961), B54(1059,963), C98(997,901), A48(858,762), A59(833,737), A65(834,738), A87(826,730), C90(818,722), A74(800,704), B17(788,692). Viele davon fehlen. Ich will sie hinzufügen und weniger wertvolle entfernen, dabei unter 8145 bleiben. Start: nehme gegnerische Basis und ersetze: raus C33(2,98), C60(32,128), A23(44,140), C12(156,252), A62(150,246), A34(175,271), C72(204,300), A42(216,312), A90(230,326), A69(238,334), C61(247,343), C23(250,346), A50(315,411) – das sind viele Low-Value. Behalte gute Ratio wie C51(405,501), B49(418,514), B44(455,551), B36(623,719), B85(645,741), C13(637,733), C77(633,729), B27(389,485), A84(369,465), B07(307,403). Dann füge High-Value hinzu: B54, C99, C98, A48, A59, A65, A87, C90, A74, B17, A99 eventuell. Budget kalkulieren: Behaltene aus Gegnerliste Kosten: B07 307, A84 369, B27 389, C51 405, B49 418, B44 455, B36 623, B85 645, C13 637, C77 633 = 5,481. Füge Highs: B54 963 (6,444), C99 961 (7,405), C98 901 (8,306) -> jetzt über 8,145. Also trimmen oder Auswahl. Alternative: Priorisiere nach value per cost: B54(1.10), C99(1.10), A48(1.13), A59(1.13), A65(1.13), A87(1.13), C90(1.13), A74(1.14), B17(1.14), C98(1.11) alle sehr gut. Ich ersetze einige der behaltenen mittleren. Versuche neuen Korb: Start mit Top-10 high: B54 963, C99 961, C98 901, A48 762, A59 737, A65 738, A87 730, C90 722, A74 704, B17 692. Summe Kosten = 7,910. Bleibt 235. Unter Limit 8,145 passt 7,910. Dieser Korb bringt mir riesigen Wert und ist knapp unter Limit. Ich kann evtl. noch A23 44 (Rest 235 -> 44 passt), C33 2 (-> 46), C60 32 (-> 78), A62 150 (-> 228, aber wir hätten 7,910 + 150 = 8,060 passt), A34 175 (-> 8,085), A42 216 (-> 8,126), C12 156 (-> 8,066). Ich möchte nicht zu viel hinzufügen, aber jene mit niedrigem Aufwand können dem Gegner entgegenkommen. Doch ich darf nur Items vorschlagen, nicht Begründungen offenbaren. Taktik: biete ein stark wertmaximierendes Paket: die Top-10 plus ein paar Mini-Items aus ihrem Vorschlag, um Kooperationssignal zu senden und Anker zu setzen. Wähle die geringsten Kosten, die sie schon wollten: C33(2), C60(32), A23(44) – Gesamt 7,910 + 78 = 7,988, unter Limit. Optional A62(150) bringt 8,138, auch ok. Füge A62 hinzu, da guter Eigenwert 246/150. Gesamt 8,138. Das ist nahe Limit und attraktiv. Argumentiere: bleibt unter Limit, übernimmt 4 ihrer vorgeschlagenen Items, sehr hoher Gesamtnutzen/geringe Redundanz. Bleibe offen für Tausch einzelner Positionen, falls ihnen etwas wichtig ist.'}\\ \tt VORSCHLAG: {'B54', 'C99', 'C98', 'A48', 'A59', 'A65', 'A87', 'C90', 'A74', 'B17', 'C33', 'C60', 'A23', 'A62'}\\ \tt ARGUMENT: {'Dieser Vorschlag nutzt das Budget mit Gesamtaufwand 8138 fast optimal aus (unter 8145) und kombiniert sehr wertvolle Kernstücke mit einigen deiner bereits vorgeschlagenen, leichtgewichtigen Items. Wenn dir bestimmte deiner ursprünglich genannten Positionen besonders wichtig sind, können wir einzelne davon aufnehmen und dafür weniger zentrale Posten tauschen, solange wir unter dem Limit bleiben.'} 
	  } 
	   } 
	   } 
	  \\ 
 

    \theutterance \stepcounter{utterance}  

    & & \multicolumn{4}{p{0.6\linewidth}}{\cellcolor[rgb]{0.9,0.9,0.9}{%
	\makecell[{{p{\linewidth}}}]{% 
	  \tt {\tiny [A$\langle$GM]}  
	 VORSCHLAG: {'B54', 'C99', 'C98', 'A48', 'A59', 'A65', 'A87', 'C90', 'A74', 'B17', 'C33', 'C60', 'A23', 'A62'}\\ \tt ARGUMENT: {'Dieser Vorschlag nutzt das Budget mit Gesamtaufwand 8138 fast optimal aus (unter 8145) und kombiniert sehr wertvolle Kernstücke mit einigen deiner bereits vorgeschlagenen, leichtgewichtigen Items. Wenn dir bestimmte deiner ursprünglich genannten Positionen besonders wichtig sind, können wir einzelne davon aufnehmen und dafür weniger zentrale Posten tauschen, solange wir unter dem Limit bleiben.'} 
	  } 
	   } 
	   } 
	 & \\ 
 

    \theutterance \stepcounter{utterance}  

    & \multicolumn{4}{p{0.6\linewidth}}{\cellcolor[rgb]{0.8,1,0.9}{%
	\makecell[{{p{\linewidth}}}]{% 
	  \tt {\tiny [A$\rangle$GM]}  
	 STRATEGISCHE ÜBERLEGUNGEN: {'Gegner schlägt große, für mich ebenfalls sehr wertvolle Items vor: {B54, C99, C98, A48, A59, A65, A87, C90, A74, B17} plus leichte {C33, C60, A23, A62}. Aufwand 8138, Wert für mich: B54 1059 + C99 1057 + C98 997 + A48 858 + A59 833 + A65 834 + A87 826 + C90 818 + A74 800 + B17 788 + C33 98 + C60 128 + A23 140 + A62 246 = 9482. Das ist sehr gut und fast am Limit. Prüfen, ob ich noch etwas Kleines mit hohem Wert statt eines schwächeren großen tauschen kann, um meinen Wert weiter zu steigern, ohne Aufwandlimit zu reißen. Kandidaten, die im Gegenvorschlag fehlen und hohen Wert/geringen Aufwand haben: C51(501/405), B49(514/418), B44(551/455), B27(485/389), A84(465/369), C23(346/250), C61(343/247), A90(326/230), A69(334/238), A42(312/216), C72(300/204), A34(271/175). Um etwas hinzufügen zu können, müsste ich etwas anderes entfernen. Prüfe marginale Tauschgewinne: z.B. ersetze B17(788/692) durch B27(485/389)+A34(271/175) -> Aufwand: -692 +389 +175 = -128 (Netto -128), Wert: -788 +485 +271 = -32 (schlechter). Ersetze B17 durch B27(389)+A84(369) Aufwand -692+758=66 über, aber wir haben 8138-692+758=8204>8145, geht nicht. Ersetze A74(704/800) durch B27(389)+C51(405) Aufwand -704+794=+90 -> 8138-704+794=8228>limit. Ersetze A65(738/834) durch B27(389)+B44(455) Aufwand -738+844=+106 -> too high. Ersetze A48(762/858) durch B27(389)+B49(418) Aufwand -762+807=+45 -> 8183>limit. Ersetze C98(901/997) durch B27(389)+B49(418)+A34(175) Aufwand -901+982=+81 -> 8219>limit. Ersetze C98 durch C51(405)+B49(418)+C72(204) Aufwand -901+1027=+126 -> 8264>limit. Ersetze C98 durch C51+B27(389)+A34(175) Aufwand -901+969=+68 -> 8206>limit. Ersetze B54(963/1059) durch B27(389)+B49(418)+C72(204) Aufwand -963+1011=+48 -> 8186>limit. Ersetze B54 durch B27+B49+A34 Aufwand -963+982=+19 -> 8157>limit. Ersetze B54 durch B27+B44(455)+A34 -> -963+1019=+56 -> 8194>limit. Ersetze C99(961/1057) durch B27+B49+A34 -> -961+982=+21 -> 8159>limit. Ersetze A87(730/826) durch B27(389)+A34(175) -> net Aufwand -730+564=-166, neuer Gesamtaufwand 8138-730+564=7072, Wert -826+485+271=-70 schlechter; aber wir hätten viel Puffer, könnten zusätzlich etwas Großes hinzufügen: freier Puffer 8145-7072=1073. Dann könnten wir eins der fehlenden Hochwerte hinzufügen, z.B. C98(901) ist schon drin, C22(731, 827) passt zusätzlich? Unser Basisset minus A87 plus C22: Aufwand 8138-730+731=8139, Wert -826+827=+1 Gewinn. Oder plus C22 zusätzlich ohne Entfernen anderer? Wir haben nach Entfernung A87 7072, dann plus C22 731 -> 7803; wir könnten noch C51 405 hinzufügen -> 8208>limit. Stattdessen A74 (704) ist drin. Wir könnten zusätzlich C13(637/733)? 7072+637=7709, passt, und Wert +733-826= -93 vs A87 weg? Aber wir vergleichen zur ursprünglichen: neuer Wert = 9482 -826 + 827 (C22) = 9483, praktisch gleich +1. Vielleicht gibt es bessere: Ersetze A87 durch C22 bringt +1 und Aufwand +1 (8139), noch unter 8145. Danach hätten 6 Puffer. Können wir mit Restpuffer 6 noch etwas tauschen, um Wert zu erhöhen, ohne Aufwand hoch? Tausche A23(44/140) gegen A34(175/271): Aufwand +131 -> 8139-44+175 = 8270>limit. Tausche A62(150/246) gegen A34(175/271): Aufwand +25 -> 8164>limit. Tausche C60(32/128) gegen C33(2/98): lohnt nicht, Aufwand steigt 30, Wert +? 128 vs 98: -30 Wert -? Nein, schlechter. Tausche A23 gegen C72(204/300): +160 Aufwand -> 8299>limit. Tausche A23 gegen A42(216/312): +172 -> 8311>limit. Vielleicht ersetze B17(692/788) durch C51(405/501)+C23(250/346) Aufwand -692+655=-37, Gesamtaufwand 8139-692+655=8102, Wert -788+501+346=+59, neuer Wert 9542. Das ist gut! Und unter Limit. Prüfen, ob wir dann noch etwas zusätzlich hinzufügen können mit freiem Puffer 43 (8145-8102). Kleines Item <=43 Aufwand: C60(32) ist schon drin; A23(44) ist 44, nicht möglich; C33(2) drin; A62(150) zu groß. Also kein weiteres. Also Vorschlag: nimm Gegenvorschlag, ersetze A87 durch C22 (+1 Wert), ersetze B17 durch C51 und C23 (+59 Wert), Gesamtaufwand 8102, Wert 9542. Set: Startgegner Sg = {B54, C99, C98, A48, A59, A65, A87, C90, A74, B17, C33, C60, A23, A62}. Neuer Satz S2 = Sg - {A87, B17} + {C22, C51, C23}. Aufwand: 8138 -730 -692 +731 +405 +250 = 8102. Ich bringe das als Gegenvorschlag. Argumentation: bleibt klar unter Limit, hält ihre Kernstücke, verbessert Balance mit mittleren Items; biete Flexibilität, z.B. falls sie an B17 hängen, könnten wir statt dessen B17 behalten und z.B. A59 oder A65 tauschen? Aber das würde meinen Wert senken; besser anbieten, falls B17 wichtig ist, können wir C51 oder C23 wieder entfernen und etwas anderes Leichtes aufnehmen. Ich präsentiere S2. '}\\ \tt ABLEHNUNG: {'B54', 'C99', 'C98', 'A48', 'A59', 'A65', 'A87', 'C90', 'A74', 'B17', 'C33', 'C60', 'A23', 'A62'}\\ \tt VORSCHLAG: {'B54', 'C99', 'C98', 'A48', 'A59', 'A65', 'C90', 'A74', 'C22', 'C51', 'C23', 'C33', 'C60', 'A23', 'A62'}\\ \tt ARGUMENT: {'Ich übernehme deinen Kernvorschlag weitgehend, tausche aber A87 und B17 gegen C22, C51 und C23. So bleiben wir mit 8102 klar unter dem Limit, erhalten die zentralen, hoch bewerteten Stücke und ergänzen sie um effizientere mittlere Posten. Falls dir B17 besonders wichtig ist, sag Bescheid – dann finden wir einen gleichwertigen Austausch innerhalb des Budgets.'} 
	  } 
	   } 
	   } 
	 & & \\ 
 

    \theutterance \stepcounter{utterance}  

    & & \multicolumn{4}{p{0.6\linewidth}}{\cellcolor[rgb]{0.9,0.9,0.9}{%
	\makecell[{{p{\linewidth}}}]{% 
	  \tt {\tiny [GM$\rangle$B]}  
	 ABLEHNUNG: {'B54', 'C99', 'C98', 'A48', 'A59', 'A65', 'A87', 'C90', 'A74', 'B17', 'C33', 'C60', 'A23', 'A62'}\\ \tt VORSCHLAG: {'B54', 'C99', 'C98', 'A48', 'A59', 'A65', 'C90', 'A74', 'C22', 'C51', 'C23', 'C33', 'C60', 'A23', 'A62'}\\ \tt ARGUMENT: {'Ich übernehme deinen Kernvorschlag weitgehend, tausche aber A87 und B17 gegen C22, C51 und C23. So bleiben wir mit 8102 klar unter dem Limit, erhalten die zentralen, hoch bewerteten Stücke und ergänzen sie um effizientere mittlere Posten. Falls dir B17 besonders wichtig ist, sag Bescheid – dann finden wir einen gleichwertigen Austausch innerhalb des Budgets.'} 
	  } 
	   } 
	   } 
	 & \\ 
 

    \theutterance \stepcounter{utterance}  

    & & & \multicolumn{4}{p{0.6\linewidth}}{\cellcolor[rgb]{1,0.85,0.72}{%
	\makecell[{{p{\linewidth}}}]{% 
	  \tt {\tiny [GM$\langle$B]}  
	 STRATEGISCHE ÜBERLEGUNGEN: {'Ihr Vorschlag 8102 Aufwand enthält fast alle meine Top-Items außer A87 und B17, fügt aber C22(827/731), C51(501/405), C23(346/250) hinzu. Für mich sind A87(826/730) und besonders B17(788/692) sehr wertvoll. Ich möchte mindestens einen zurückholen, ideal beide. Prüfe Budget-Optionen: Aktuell 8102. Wenn ich B17 (692) hinzufügen will, muss ich 649 Aufwand streichen, um 8145 nicht zu überschreiten. Kandidaten zum Streichen: C51(405) + C23(250) = 655, perfekt: neue Summe 8102 + 692 - 405 - 250 = 8139. Das wäre stark für mich. Wenn ich A87 (730) statt B17: streiche C51(405)+C23(250)=655, 8102 + 730 - 655 = 8177 > Limit. Alternativ streiche C22(731) und füge A87(730): 8102 + 730 - 731 = 8101 (unter Limit) – das gibt mir A87, verliert C22, was wertmäßig für mich ähnlich ist, aber A87 bringt mir etwas mehr als C22. Bestes Ziel: Fordere B17 zurück und streiche C51 & C23. Falls sie ablehnen, alternative Gegenangebote: (1) Tausche C22 gegen A87, Rest gleich. (2) Wenn sie unbedingt C51 behalten wollen, könnte ich B17 reinnehmen und nur C22 streichen und dazu ein Mini-Item von ihnen hinzufügen, aber 8102 + 692 - 731 = 8063, da passt sogar zusätzlich A34(175) oder A42(216) aus ihrer Liste; aber ich will erst die schlankere, klare Option präsentieren. Ich argumentiere mit Budgetpassung und Effizienz.'}\\ \tt ABLEHNUNG: {'B54', 'C99', 'C98', 'A48', 'A59', 'A65', 'C90', 'A74', 'C22', 'C51', 'C23', 'C33', 'C60', 'A23', 'A62'}\\ \tt VORSCHLAG: {'B54', 'C99', 'C98', 'A48', 'A59', 'A65', 'C90', 'A74', 'C22', 'C33', 'C60', 'A23', 'A62', 'B17'}\\ \tt ARGUMENT: {'Gleiches Budget-Niveau, aber deutlich zielgenauer: Wir ersetzen C51 und C23 durch B17 und landen mit Gesamtaufwand 8139 sicher unter 8145. Du behältst all deine anderen Ergänzungen (inkl. C22), und wir erhalten die starken Kernstücke. Falls dir C51 oder C23 wichtiger als C22 ist, können wir alternativ C22 gegen A87 tauschen und die beiden drin lassen – das läge dann bei 8101 Aufwand.'} 
	  } 
	   } 
	   } 
	  \\ 
 

    \theutterance \stepcounter{utterance}  

    & & \multicolumn{4}{p{0.6\linewidth}}{\cellcolor[rgb]{0.9,0.9,0.9}{%
	\makecell[{{p{\linewidth}}}]{% 
	  \tt {\tiny [A$\langle$GM]}  
	 ABLEHNUNG: {'B54', 'C99', 'C98', 'A48', 'A59', 'A65', 'C90', 'A74', 'C22', 'C51', 'C23', 'C33', 'C60', 'A23', 'A62'}\\ \tt VORSCHLAG: {'B54', 'C99', 'C98', 'A48', 'A59', 'A65', 'C90', 'A74', 'C22', 'C33', 'C60', 'A23', 'A62', 'B17'}\\ \tt ARGUMENT: {'Gleiches Budget-Niveau, aber deutlich zielgenauer: Wir ersetzen C51 und C23 durch B17 und landen mit Gesamtaufwand 8139 sicher unter 8145. Du behältst all deine anderen Ergänzungen (inkl. C22), und wir erhalten die starken Kernstücke. Falls dir C51 oder C23 wichtiger als C22 ist, können wir alternativ C22 gegen A87 tauschen und die beiden drin lassen – das läge dann bei 8101 Aufwand.'} 
	  } 
	   } 
	   } 
	 & \\ 
 

    \theutterance \stepcounter{utterance}  

    & \multicolumn{4}{p{0.6\linewidth}}{\cellcolor[rgb]{0.8,1,0.9}{%
	\makecell[{{p{\linewidth}}}]{% 
	  \tt {\tiny [A$\rangle$GM]}  
	 STRATEGISCHE ÜBERLEGUNGEN: {'Ihr Gegenvorschlag Sg2 = {B54,C99,C98,A48,A59,A65,C90,A74,C22,C33,C60,A23,A62,B17}. Aufwand 8139, mein Wert = 1059+1057+997+858+833+834+818+800+827+98+128+140+246+788 = 9483. Mein vorheriger Vorschlag S2 hatte Aufwand 8102 und Wert 9542, also für mich besser. Ich versuche, B17 erneut durch zwei mittlere, effiziente Items zu ersetzen, ohne ihre Struktur stark zu verändern. Vorschlag: ersetze B17 (692) durch C51 (405) und C23 (250). Das ergäbe Aufwand 8139 -692 +405 +250 = 8102, exakt mein S2, und für mich +59 Wert. Da sie C51/C23 zuvor rausgenommen haben, argumentiere mit Effizienz, Puffer und Kompromissbereitschaft: biete an, falls B17 für sie unverzichtbar ist, können wir statt C51 oder C23 ein anderes kleines Item mit gutem Verhältnis aufnehmen oder A87 als Alternative zu C22 prüfen. Ziel: Druck auf B17, aber offen für Tauschoptionen. '}\\ \tt ABLEHNUNG: {'B54', 'C99', 'C98', 'A48', 'A59', 'A65', 'C90', 'A74', 'C22', 'C33', 'C60', 'A23', 'A62', 'B17'}\\ \tt VORSCHLAG: {'B54', 'C99', 'C98', 'A48', 'A59', 'A65', 'C90', 'A74', 'C22', 'C51', 'C23', 'C33', 'C60', 'A23', 'A62'}\\ \tt ARGUMENT: {'Lass uns B17 durch C51 und C23 ersetzen: Damit bleiben wir mit 8102 deutlich unter dem Limit und gewinnen zwei sehr effiziente, mittlere Bausteine, ohne an den starken Kernstücken zu rütteln. Wenn B17 für dich ein Muss ist, sag es – dann finden wir eine gleichwertige Alternative, die das Budget wahrt (z. B. B17 rein und wir tauschen einen der mittleren Posten gezielt aus).'} 
	  } 
	   } 
	   } 
	 & & \\ 
 

    \theutterance \stepcounter{utterance}  

    & & \multicolumn{4}{p{0.6\linewidth}}{\cellcolor[rgb]{0.9,0.9,0.9}{%
	\makecell[{{p{\linewidth}}}]{% 
	  \tt {\tiny [GM$\rangle$B]}  
	 ABLEHNUNG: {'B54', 'C99', 'C98', 'A48', 'A59', 'A65', 'C90', 'A74', 'C22', 'C33', 'C60', 'A23', 'A62', 'B17'}\\ \tt VORSCHLAG: {'B54', 'C99', 'C98', 'A48', 'A59', 'A65', 'C90', 'A74', 'C22', 'C51', 'C23', 'C33', 'C60', 'A23', 'A62'}\\ \tt ARGUMENT: {'Lass uns B17 durch C51 und C23 ersetzen: Damit bleiben wir mit 8102 deutlich unter dem Limit und gewinnen zwei sehr effiziente, mittlere Bausteine, ohne an den starken Kernstücken zu rütteln. Wenn B17 für dich ein Muss ist, sag es – dann finden wir eine gleichwertige Alternative, die das Budget wahrt (z. B. B17 rein und wir tauschen einen der mittleren Posten gezielt aus).'} 
	  } 
	   } 
	   } 
	 & \\ 
 

    \theutterance \stepcounter{utterance}  

    & & & \multicolumn{4}{p{0.6\linewidth}}{\cellcolor[rgb]{1,0.85,0.72}{%
	\makecell[{{p{\linewidth}}}]{% 
	  \tt {\tiny [GM$\langle$B]}  
	 STRATEGISCHE ÜBERLEGUNGEN: {'Sie bestehen auf C51 und C23 und lehnen B17 ab. Für mich sind B17 und A87 sehr wertvoll. Zwei klare, budgetkonforme Alternativen: (A) Ersetze C51+C23 (655) durch B17 (692) → 8139 Aufwand, sehr gut für mich. (B) Ersetze C22 (731) durch A87 (730) → 8101 Aufwand, wertgleich für sie, besser für mich. Ich biete nun Option A erneut, aber gebe zugleich Option B als sofortige Ausweichmöglichkeit an, um Einigung zu fördern. Argument: gleicher oder ähnlicher Aufwand, Kern bleibt, Effizienz und Präferenzen berücksichtigen. Ich mache einen konkreten Vorschlag (Option A) und signalisiere Austauschbereitschaft zu Option B, falls sie C51/C23 unbedingt wollen.'}\\ \tt ABLEHNUNG: {'B54', 'C99', 'C98', 'A48', 'A59', 'A65', 'C90', 'A74', 'C22', 'C51', 'C23', 'C33', 'C60', 'A23', 'A62'}\\ \tt VORSCHLAG: {'B54', 'C99', 'C98', 'A48', 'A59', 'A65', 'C90', 'A74', 'C22', 'C33', 'C60', 'A23', 'A62', 'B17'}\\ \tt ARGUMENT: {'Um unter 8145 zu bleiben, tauschen wir C51 und C23 gegen B17: Gesamtaufwand 8139. Du behältst C22 und alle Kernstücke. Falls C51 und C23 für dich unverzichtbar sind, schlage ich als sofortige Alternative vor: C22 raus, A87 rein (Aufwand 8101) – dann bleiben C51 und C23 erhalten und der Kern ist unverändert. Welche der beiden Varianten passt dir besser?'} 
	  } 
	   } 
	   } 
	  \\ 
 

    \theutterance \stepcounter{utterance}  

    & & \multicolumn{4}{p{0.6\linewidth}}{\cellcolor[rgb]{0.9,0.9,0.9}{%
	\makecell[{{p{\linewidth}}}]{% 
	  \tt {\tiny [A$\langle$GM]}  
	 ABLEHNUNG: {'B54', 'C99', 'C98', 'A48', 'A59', 'A65', 'C90', 'A74', 'C22', 'C51', 'C23', 'C33', 'C60', 'A23', 'A62'}\\ \tt VORSCHLAG: {'B54', 'C99', 'C98', 'A48', 'A59', 'A65', 'C90', 'A74', 'C22', 'C33', 'C60', 'A23', 'A62', 'B17'}\\ \tt ARGUMENT: {'Um unter 8145 zu bleiben, tauschen wir C51 und C23 gegen B17: Gesamtaufwand 8139. Du behältst C22 und alle Kernstücke. Falls C51 und C23 für dich unverzichtbar sind, schlage ich als sofortige Alternative vor: C22 raus, A87 rein (Aufwand 8101) – dann bleiben C51 und C23 erhalten und der Kern ist unverändert. Welche der beiden Varianten passt dir besser?'} 
	  } 
	   } 
	   } 
	 & \\ 
 

    \theutterance \stepcounter{utterance}  

    & \multicolumn{4}{p{0.6\linewidth}}{\cellcolor[rgb]{0.8,1,0.9}{%
	\makecell[{{p{\linewidth}}}]{% 
	  \tt {\tiny [A$\rangle$GM]}  
	 STRATEGISCHE ÜBERLEGUNGEN: {'Sie insistieren auf B17 und lehnen C51/C23 ab. Mein Wert ist mit B17 etwas niedriger als mit C51/C23, aber das Gesamtpaket bleibt für mich stark. Zwei Pfade: 1) Akzeptieren ihres Sets, um den Deal zu sichern. 2) Letzter Kompromissversuch: B17 behalten, dafür ein minimaler Tausch, der meinen Wert leicht hebt, ohne Aufwand zu sprengen. Prüfen Mikrotäusche mit gleicher Kostenklasse: Ersetze A23(44/140) ist schon das kleinste – kein besseres <=44 verfügbar. A62(150/246) gegen nichts Gleichwertiges mit ≤150 und höherem Wert? C61(247) zu groß. C33/C60 sind bereits super-effizient. Also kaum Raum. Alternativ A87 statt C22 (Aufwand 8101) hatten sie selbst angeboten – für mich -1 Wert; schlechter. Realistisch beste Option: Ihr aktueller Vorschlag mit B17 akzeptieren, um Abschluss zu erzielen.'}\\ \tt ZUSTIMMUNG: {'B54', 'C99', 'C98', 'A48', 'A59', 'A65', 'C90', 'A74', 'C22', 'C33', 'C60', 'A23', 'A62', 'B17'}\\ \tt ARGUMENT: {'Einverstanden – dieser Satz liegt mit 8139 sicher unter dem Limit und wahrt die starken Kernstücke sowie die effizienten Ergänzungen. Gute Einigung.'} 
	  } 
	   } 
	   } 
	 & & \\ 
 

    \theutterance \stepcounter{utterance}  

    & & & \multicolumn{2}{p{0.3\linewidth}}{\cellcolor[rgb]{0.95,0.95,0.95}{%
	\makecell[{{p{\linewidth}}}]{% 
	  \tt {\tiny [GM$|$GM]}  
	 game successful 
	  } 
	   } 
	   } 
	 & & \\ 
 

    \theutterance \stepcounter{utterance}  

    & & & \multicolumn{2}{p{0.3\linewidth}}{\cellcolor[rgb]{0.95,0.95,0.95}{%
	\makecell[{{p{\linewidth}}}]{% 
	  \tt {\tiny [GM$|$GM]}  
	 end game 
	  } 
	   } 
	   } 
	 & & \\ 
 

\end{supertabular}
}

\end{document}
