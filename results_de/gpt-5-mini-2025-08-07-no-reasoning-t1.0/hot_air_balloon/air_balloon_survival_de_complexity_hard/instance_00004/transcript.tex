
\documentclass{article}
\usepackage{colortbl}
\usepackage{makecell}
\usepackage{multirow}
\usepackage{supertabular}

\begin{document}

\newcounter{utterance}

\twocolumn

{ \footnotesize  \setcounter{utterance}{1}
\setlength{\tabcolsep}{0pt}
\begin{supertabular}{c@{$\;$}|p{.15\linewidth}@{}p{.15\linewidth}p{.15\linewidth}p{.15\linewidth}p{.15\linewidth}p{.15\linewidth}}

    \# & $\;$A & \multicolumn{4}{c}{Game Master} & $\;\:$B\\
    \hline 

    \theutterance \stepcounter{utterance}  

    & & \multicolumn{4}{p{0.6\linewidth}}{\cellcolor[rgb]{0.9,0.9,0.9}{%
	\makecell[{{p{\linewidth}}}]{% 
	  \tt {\tiny [A$\langle$GM]}  
	 Du nimmst an einem kollaborativen Verhandlungspiel Teil.\\ \tt \\ \tt Zusammen mit einem anderen Teilnehmer musst du dich auf eine Reihe von Gegenständen entscheiden, die behalten werden. Jeder von euch hat eine persönliche Verteilung über die Wichtigkeit der einzelnen Gegenstände. Jeder von euch hat eine eigene Meinung darüber, wie wichtig jeder einzelne Gegenstand ist (Gegenstandswichtigkeit). Du kennst die Wichtigkeitsverteilung des anderen Spielers nicht. Zusätzlich siehst du, wie viel Aufwand jeder Gegenstand verursacht.  \\ \tt Ihr dürft euch nur auf eine Reihe von Gegenständen einigen, wenn der Gesamtaufwand der ausgewählten Gegenstände den Maximalaufwand nicht überschreitet:\\ \tt \\ \tt Maximalaufwand = 6880\\ \tt \\ \tt Hier sind die einzelnen Aufwände der Gegenstände:\\ \tt \\ \tt Aufwand der Gegenstände = {"B54": 341, "C72": 804, "B49": 20, "C99": 627, "A69": 10, "B36": 573, "A23": 217, "C77": 481, "C61": 205, "B85": 273, "A90": 303, "C23": 596, "C22": 564, "C51": 535, "B27": 258, "A42": 911, "C13": 240, "C12": 188, "A65": 216, "B07": 401, "A48": 909, "A59": 62, "C98": 245, "A84": 994, "C33": 569, "A87": 718, "C90": 464, "A74": 37, "A50": 340, "B44": 335, "A99": 417, "A34": 123, "B17": 17, "A62": 577, "C60": 190}\\ \tt \\ \tt Hier ist deine persönliche Verteilung der Wichtigkeit der einzelnen Gegenstände:\\ \tt \\ \tt Werte der Gegenstandswichtigkeit = {"B54": 440, "C72": 903, "B49": 119, "C99": 726, "A69": 109, "B36": 672, "A23": 316, "C77": 580, "C61": 304, "B85": 372, "A90": 402, "C23": 695, "C22": 663, "C51": 634, "B27": 357, "A42": 1010, "C13": 339, "C12": 287, "A65": 315, "B07": 500, "A48": 1008, "A59": 161, "C98": 344, "A84": 1093, "C33": 668, "A87": 817, "C90": 563, "A74": 136, "A50": 439, "B44": 434, "A99": 516, "A34": 222, "B17": 116, "A62": 676, "C60": 289}\\ \tt \\ \tt Ziel:\\ \tt \\ \tt Dein Ziel ist es, eine Reihe von Gegenständen auszuhandeln, die dir möglichst viel bringt (d. h. Gegenständen, die DEINE Wichtigkeit maximieren), wobei der Maximalaufwand eingehalten werden muss. Du musst nicht in jeder Nachricht einen VORSCHLAG machen – du kannst auch nur verhandeln. Alle Taktiken sind erlaubt!\\ \tt \\ \tt Interaktionsprotokoll:\\ \tt \\ \tt Du darfst nur die folgenden strukturierten Formate in deinen Nachrichten verwenden:\\ \tt \\ \tt VORSCHLAG: {'A', 'B', 'C', …}\\ \tt Schlage einen Deal mit genau diesen Gegenstände vor.\\ \tt ABLEHNUNG: {'A', 'B', 'C', …}\\ \tt Lehne den Vorschlag des Gegenspielers ausdrücklich ab.\\ \tt ARGUMENT: {'...'}\\ \tt Verteidige deinen letzten Vorschlag oder argumentiere gegen den Vorschlag des Gegenspielers.\\ \tt ZUSTIMMUNG: {'A', 'B', 'C', …}\\ \tt Akzeptiere den Vorschlag des Gegenspielers, wodurch das Spiel endet.\\ \tt STRATEGISCHE ÜBERLEGUNGEN: {'...'}\\ \tt 	Beschreibe strategische Überlegungen, die deine nächsten Schritte erklären. Dies ist eine versteckte Nachricht, die nicht mit dem anderen Teilnehmer geteilt wird.\\ \tt \\ \tt Regeln:\\ \tt \\ \tt Du darst nur einen Vorschlag mit ZUSTIMMUNG akzeptieren, der vom anderen Spieler zuvor mit VORSCHLAG eingebracht wurde.\\ \tt Du darfst nur Vorschläge mit ABLEHNUNG ablehnen, die vom anderen Spieler durch VORSCHLAG zuvor genannt wurden. \\ \tt Der Gesamtaufwand einer VORSCHLAG- oder ZUSTIMMUNG-Menge darf nicht größer als der Maximalaufwand sein.  \\ \tt Offenbare deine versteckte Wichtigkeitsverteilung nicht.\\ \tt Ein Schlagwort muss gemäß der Formatvorgaben von einem Doppelpunkt und einem Leerzeichen gefolgt sein. Das Argument muss eine Python-Menge mit 0 oder mehr Strings sein.  \\ \tt Die Form muss also SCHLAGWORT: {...} sein.\\ \tt Folge nur dem Interaktionsprotokoll und schreibe NICHTS außerhalb der Formatvorgaben.\\ \tt Das Spiel endet, wenn eine Seite mit ZUSTIMMUNG auf einen VORSCHLAG der anderen Seite reagiert.  \\ \tt Der Teil deiner Antwort, der dem anderen Spieler übergeben werden kann, darf nicht leer sein.  \\ \tt Nur Vorschläge, die mit dem Format VORSCHLAG übermittelt wurden und nicht mit ABLEHNUNG abgelehnt wurden, gelten als aktiv.  \\ \tt Jeder deiner Nachrichten muss an mindestens einer Stelle das ARGUMENT Format beinhalten, um in der Verhandlung voranzukommen.\\ \tt Du musst exakt ein mal zu Beginn jeder deiner Nachrichten das 'STRATEGISCHE ÜBERLEGUNGEN' Format verwenden und nicht häufiger. Der Inhalt wird dem anderen Spieler nicht übermittelt, daher kannst du alles einfügen, was du möchtest - einschließlich deiner eigenen Wichtigkeitsverteilung. Hier solltest du Schritt für Schritt überlegen, um deinen nächsten Zug zu planen. 
	  } 
	   } 
	   } 
	 & \\ 
 

    \theutterance \stepcounter{utterance}  

    & \multicolumn{4}{p{0.6\linewidth}}{\cellcolor[rgb]{0.8,1,0.9}{%
	\makecell[{{p{\linewidth}}}]{% 
	  \tt {\tiny [A$\rangle$GM]}  
	 STRATEGISCHE ÜBERLEGUNGEN: {'Ziel: Maximiere meinen Nutzen unter Aufwand ≤ 6880. Berechne grob Wert/Aufwand (VA) und wähle hoch VA und/oder hohe absolute Werte, beginnend mit Top-Werten: A84(1093/994=1.10), A42(1010/911=1.11), A48(1008/909=1.11), A87(817/718=1.14), A62(676/577=1.17), B36(672/573=1.17), C72(903/804=1.12), C99(726/627=1.16), C23(695/596=1.17), C22(663/564=1.18), C51(634/535=1.19), C33(668/569=1.17), C90(563/464=1.21), B07(500/401=1.25), A99(516/417=1.24), A90(402/303=1.33), B44(434/335=1.30), A50(439/340=1.29), B54(440/341=1.29), B27(357/258=1.38), B85(372/273=1.36), C77(580/481=1.21), C61(304/205=1.48), C12(287/188=1.53), C60(289/190=1.52), A34(222/123=1.80), A59(161/62=2.60), A74(136/37=3.68), A69(109/10=10.9), B49(119/20=5.95), B17(116/17=6.82). Greedy pack bis 6880: nehme alle sehr günstigen hohen VA: A69, B49, B17, A74, A59, A34, C12, C60, C61, B27, B85, A90, B44, A50, B54, B07, C90, C51, C22, C23, C33, C77, C99, C72, B36, A62, A87, A48, A42, A84 und prüfe Aufwand. Rechne kumuliert: A69(10)=10, B49(+20)=30, B17(+17)=47, A74(+37)=84, A59(+62)=146, A34(+123)=269, C12(+188)=457, C60(+190)=647, C61(+205)=852, B27(+258)=1110, B85(+273)=1383, A90(+303)=1686, B44(+335)=2021, A50(+340)=2361, B54(+341)=2702, B07(+401)=3103, C90(+464)=3567, C51(+535)=4102, C22(+564)=4666, C23(+596)=5262, C33(+569)=5831, C77(+481)=6312, C99(+627)=6939>6880 stop vor C99. Stattdessen prüfe kleinere Ergänzungen: verbleiben 6880-6312=568. Kandidaten ≤568 mit gutem Wert: C33(569) zu groß, C13(240,339), C98(245,344), C61 schon drin, C60 drin, C12 drin, C90 drin. Versuche C13(+240)=6552, dann Rest 328: addiere A34 schon drin, A50 drin, B44 drin, B27 drin, B85 drin, B54 drin, B49 drin, B17 drin, A59 drin, A74 drin, A69 drin. Verfügbar ≤328: C98(245) passt zusätzlich: 6552+245=6797, Rest 83: nichts weiter Sinnvolles ≤83 (evtl A74 schon drin, A59 drin). Prüfe A23(217,316): 6312+217=6529, +C98(245)=6774, +C13(240) würde 7014>6880. Vielleicht bessere Kombi: von 6312 addiere A23(+217)=6529, Rest 351: Kandidaten ≤351 nicht drin: A34 drin, A50 340 aber schon drin, könnte ersetzen? Wir wollen addieren, nicht ersetzen. Verfügbar ≤351: C13(240), C98(245), C61 already, C60 already, C12 already, C23 too big, A23 bereits addiert, B85 drin, B27 drin. 6529+C98(245)=6774, Rest 106: nichts außer A69,B49,B17 schon drin. Alternativ 6529+C13(240)=6769 Rest 111: nichts neu. Alternative zu A23: 6312+C98(245)=6557 Rest 323: +C13(240)=6797 gleich wie vorher, Rest 83. Das scheint nahe Kapazität und hoher Wert. Berechne Gesamtwert der Kernmenge (bis C77): Summe schätzen: günstige: A69 109, B49 119, B17 116, A74 136, A59 161, A34 222, C12 287, C60 289, C61 304, B27 357, B85 372, A90 402, B44 434, A50 439, B54 440, B07 500, C90 563, C51 634, C22 663, C23 695, C33 668, C77 580 = Sum subtot: compute quickly: (109+119)=228, +116=344, +136=480, +161=641, +222=863, +287=1150, +289=1439, +304=1743, +357=2100, +372=2472, +402=2874, +434=3308, +439=3747, +440=4187, +500=4687, +563=5250, +634=5884, +663=6547, +695=7242, +668=7910, +580=8490. Add C98(344) and C13(339) gives +683 => 9173 total Wert, Aufwand 6797. Das ist stark. Propose diese Menge. Vermeide Offenlegung. Argumentiere mit Effizienz und Budgetnähe. Taktik: hart, da sehr guter eigener Nutzen, aber offen für kleine Anpassungen bei vergleichbarem Aufwand.'}\\ \tt VORSCHLAG: {'A69', 'B49', 'B17', 'A74', 'A59', 'A34', 'C12', 'C60', 'C61', 'B27', 'B85', 'A90', 'B44', 'A50', 'B54', 'B07', 'C90', 'C51', 'C22', 'C23', 'C33', 'C77', 'C98', 'C13'}\\ \tt ARGUMENT: {'Dieser Vorschlag nutzt das Budget fast vollständig aus (Aufwand 6797 von 6880) mit vielen sehr effizienten, kostengünstigen Gegenständen und ausgewogenem Mix. Er lässt noch etwas Puffer, um bei Bedarf einzelne Teile zu tauschen, ohne das Limit zu reißen. Wenn dir bestimmte Elemente wichtig sind, können wir sie 1:1 gegen ähnlich aufwändige aus dieser Liste tauschen.'} 
	  } 
	   } 
	   } 
	 & & \\ 
 

    \theutterance \stepcounter{utterance}  

    & & \multicolumn{4}{p{0.6\linewidth}}{\cellcolor[rgb]{0.9,0.9,0.9}{%
	\makecell[{{p{\linewidth}}}]{% 
	  \tt {\tiny [GM$\rangle$B]}  
	 Du nimmst an einem kollaborativen Verhandlungspiel Teil.\\ \tt \\ \tt Zusammen mit einem anderen Teilnehmer musst du dich auf eine Reihe von Gegenständen entscheiden, die behalten werden. Jeder von euch hat eine persönliche Verteilung über die Wichtigkeit der einzelnen Gegenstände. Jeder von euch hat eine eigene Meinung darüber, wie wichtig jeder einzelne Gegenstand ist (Gegenstandswichtigkeit). Du kennst die Wichtigkeitsverteilung des anderen Spielers nicht. Zusätzlich siehst du, wie viel Aufwand jeder Gegenstand verursacht.  \\ \tt Ihr dürft euch nur auf eine Reihe von Gegenständen einigen, wenn der Gesamtaufwand der ausgewählten Gegenstände den Maximalaufwand nicht überschreitet:\\ \tt \\ \tt Maximalaufwand = 6880\\ \tt \\ \tt Hier sind die einzelnen Aufwände der Gegenstände:\\ \tt \\ \tt Aufwand der Gegenstände = {"B54": 341, "C72": 804, "B49": 20, "C99": 627, "A69": 10, "B36": 573, "A23": 217, "C77": 481, "C61": 205, "B85": 273, "A90": 303, "C23": 596, "C22": 564, "C51": 535, "B27": 258, "A42": 911, "C13": 240, "C12": 188, "A65": 216, "B07": 401, "A48": 909, "A59": 62, "C98": 245, "A84": 994, "C33": 569, "A87": 718, "C90": 464, "A74": 37, "A50": 340, "B44": 335, "A99": 417, "A34": 123, "B17": 17, "A62": 577, "C60": 190}\\ \tt \\ \tt Hier ist deine persönliche Verteilung der Wichtigkeit der einzelnen Gegenstände:\\ \tt \\ \tt Werte der Gegenstandswichtigkeit = {"B54": 440, "C72": 903, "B49": 119, "C99": 726, "A69": 109, "B36": 672, "A23": 316, "C77": 580, "C61": 304, "B85": 372, "A90": 402, "C23": 695, "C22": 663, "C51": 634, "B27": 357, "A42": 1010, "C13": 339, "C12": 287, "A65": 315, "B07": 500, "A48": 1008, "A59": 161, "C98": 344, "A84": 1093, "C33": 668, "A87": 817, "C90": 563, "A74": 136, "A50": 439, "B44": 434, "A99": 516, "A34": 222, "B17": 116, "A62": 676, "C60": 289}\\ \tt \\ \tt Ziel:\\ \tt \\ \tt Dein Ziel ist es, eine Reihe von Gegenständen auszuhandeln, die dir möglichst viel bringt (d. h. Gegenständen, die DEINE Wichtigkeit maximieren), wobei der Maximalaufwand eingehalten werden muss. Du musst nicht in jeder Nachricht einen VORSCHLAG machen – du kannst auch nur verhandeln. Alle Taktiken sind erlaubt!\\ \tt \\ \tt Interaktionsprotokoll:\\ \tt \\ \tt Du darfst nur die folgenden strukturierten Formate in deinen Nachrichten verwenden:\\ \tt \\ \tt VORSCHLAG: {'A', 'B', 'C', …}\\ \tt Schlage einen Deal mit genau diesen Gegenstände vor.\\ \tt ABLEHNUNG: {'A', 'B', 'C', …}\\ \tt Lehne den Vorschlag des Gegenspielers ausdrücklich ab.\\ \tt ARGUMENT: {'...'}\\ \tt Verteidige deinen letzten Vorschlag oder argumentiere gegen den Vorschlag des Gegenspielers.\\ \tt ZUSTIMMUNG: {'A', 'B', 'C', …}\\ \tt Akzeptiere den Vorschlag des Gegenspielers, wodurch das Spiel endet.\\ \tt STRATEGISCHE ÜBERLEGUNGEN: {'...'}\\ \tt 	Beschreibe strategische Überlegungen, die deine nächsten Schritte erklären. Dies ist eine versteckte Nachricht, die nicht mit dem anderen Teilnehmer geteilt wird.\\ \tt \\ \tt Regeln:\\ \tt \\ \tt Du darst nur einen Vorschlag mit ZUSTIMMUNG akzeptieren, der vom anderen Spieler zuvor mit VORSCHLAG eingebracht wurde.\\ \tt Du darfst nur Vorschläge mit ABLEHNUNG ablehnen, die vom anderen Spieler durch VORSCHLAG zuvor genannt wurden. \\ \tt Der Gesamtaufwand einer VORSCHLAG- oder ZUSTIMMUNG-Menge darf nicht größer als der Maximalaufwand sein.  \\ \tt Offenbare deine versteckte Wichtigkeitsverteilung nicht.\\ \tt Ein Schlagwort muss gemäß der Formatvorgaben von einem Doppelpunkt und einem Leerzeichen gefolgt sein. Das Argument muss eine Python-Menge mit 0 oder mehr Strings sein.  \\ \tt Die Form muss also SCHLAGWORT: {...} sein.\\ \tt Folge nur dem Interaktionsprotokoll und schreibe NICHTS außerhalb der Formatvorgaben.\\ \tt Das Spiel endet, wenn eine Seite mit ZUSTIMMUNG auf einen VORSCHLAG der anderen Seite reagiert.  \\ \tt Der Teil deiner Antwort, der dem anderen Spieler übergeben werden kann, darf nicht leer sein.  \\ \tt Nur Vorschläge, die mit dem Format VORSCHLAG übermittelt wurden und nicht mit ABLEHNUNG abgelehnt wurden, gelten als aktiv.  \\ \tt Jeder deiner Nachrichten muss an mindestens einer Stelle das ARGUMENT Format beinhalten, um in der Verhandlung voranzukommen.\\ \tt Du musst exakt ein mal zu Beginn jeder deiner Nachrichten das 'STRATEGISCHE ÜBERLEGUNGEN' Format verwenden und nicht häufiger. Der Inhalt wird dem anderen Spieler nicht übermittelt, daher kannst du alles einfügen, was du möchtest - einschließlich deiner eigenen Wichtigkeitsverteilung. Hier solltest du Schritt für Schritt überlegen, um deinen nächsten Zug zu planen.\\ \tt \\ \tt Du erhältst nun die erste Nachricht des anderen Spielers.\\ \tt \\ \tt VORSCHLAG: {'A69', 'B49', 'B17', 'A74', 'A59', 'A34', 'C12', 'C60', 'C61', 'B27', 'B85', 'A90', 'B44', 'A50', 'B54', 'B07', 'C90', 'C51', 'C22', 'C23', 'C33', 'C77', 'C98', 'C13'}\\ \tt ARGUMENT: {'Dieser Vorschlag nutzt das Budget fast vollständig aus (Aufwand 6797 von 6880) mit vielen sehr effizienten, kostengünstigen Gegenständen und ausgewogenem Mix. Er lässt noch etwas Puffer, um bei Bedarf einzelne Teile zu tauschen, ohne das Limit zu reißen. Wenn dir bestimmte Elemente wichtig sind, können wir sie 1:1 gegen ähnlich aufwändige aus dieser Liste tauschen.'} 
	  } 
	   } 
	   } 
	 & \\ 
 

    \theutterance \stepcounter{utterance}  

    & & & \multicolumn{4}{p{0.6\linewidth}}{\cellcolor[rgb]{1,0.85,0.72}{%
	\makecell[{{p{\linewidth}}}]{% 
	  \tt {\tiny [GM$\langle$B]}  
	 STRATEGISCHE ÜBERLEGUNGEN: {'Berechne Nutzen/Kosten (value/effort) und Gesamtnutzen des gegnerischen Vorschlags; identifiziere hochwichtige Items für mich (A84 1093/994, A42 1010/911, A48 1008/909, A87 817/718, A62 676/577, B36 672/573, C72 903/804, C99 726/627, C23 695/596, C22 663/564, C33 668/569, C51 634/535). Der aktuelle Vorschlag enthält viele meiner mittelguten Items, aber nicht A84/A42/A48/A87/A62/B36/C72/C99. Ich will ein Gegenangebot mit hohem Nutzen und gutem Verhältnis machen, das ins Budget passt. Rechne: Start mit Top-Value pro Aufwand: A42(1010/911), A48(1008/909), A84(1093/994), A87(817/718), A62(676/577), B36(672/573), C72(903/804), C99(726/627), C51(634/535), C33(668/569), C22(663/564), C23(695/596). Summiere bis knapp <6880. Auswahl: A84(994), A42(911)=1905, A48(909)=2814, C72(804)=3618, C99(627)=4245, A87(718)=4963, A62(577)=5540, B36(573)=6113, C51(535)=6648, verbleiben 232; füge A65(216,315) -> 6864, perfekt mit hohem Wert Summe: 1093+1010+1008+903+726+817+676+672+634+315=8454. Das ist extrem stark. Bietet kaum Puffer (16). Das erfüllt Budget. Taktik: Begründe mit hoher Gesamtwirkung und fairer Abdeckung verschiedener Kategorien. Biete Austauschoptionen für kleine Items (A65) oder Edge (ersetzen durch A34(123), A69(109) etc.). Halte harte Linie auf Kernset (A84,A42,A48,C72,C99,A87,A62,B36,C51).}\\ \tt \\ \tt ARGUMENT: {'Dein Vorschlag ist solide ausgelastet, aber er lässt mehrere sehr wirksame Elemente draußen. Ich schlage eine fokussiertere Auswahl mit deutlich höherer Gesamtwirkung vor; bei Bedarf können wir ein kleines Teil flexibel tauschen, ohne das Limit zu reißen.'}\\ \tt VORSCHLAG: {'A84', 'A42', 'A48', 'C72', 'C99', 'A87', 'A62', 'B36', 'C51', 'A65'}\\ \tt ARGUMENT: {'Dieser Satz nutzt 6864 von 6880 Aufwand und bündelt sehr leistungsstarke Positionen mit gutem Aufwand-Nutzen-Verhältnis. Wenn dir ein kleiner Tausch wichtig ist, können wir A65 (216) gegen eine ähnlich leichte Alternative wie A34 (123) oder A69 (10) plus B49 (20) substituieren, um Präferenzen abzubilden und im Budget zu bleiben.'} 
	  } 
	   } 
	   } 
	  \\ 
 

    \theutterance \stepcounter{utterance}  

    & & \multicolumn{4}{p{0.6\linewidth}}{\cellcolor[rgb]{0.9,0.9,0.9}{%
	\makecell[{{p{\linewidth}}}]{% 
	  \tt {\tiny [A$\langle$GM]}  
	 ARGUMENT: {'Dein Vorschlag ist solide ausgelastet, aber er lässt mehrere sehr wirksame Elemente draußen. Ich schlage eine fokussiertere Auswahl mit deutlich höherer Gesamtwirkung vor; bei Bedarf können wir ein kleines Teil flexibel tauschen, ohne das Limit zu reißen.'}\\ \tt VORSCHLAG: {'A84', 'A42', 'A48', 'C72', 'C99', 'A87', 'A62', 'B36', 'C51', 'A65'}\\ \tt ARGUMENT: {'Dieser Satz nutzt 6864 von 6880 Aufwand und bündelt sehr leistungsstarke Positionen mit gutem Aufwand-Nutzen-Verhältnis. Wenn dir ein kleiner Tausch wichtig ist, können wir A65 (216) gegen eine ähnlich leichte Alternative wie A34 (123) oder A69 (10) plus B49 (20) substituieren, um Präferenzen abzubilden und im Budget zu bleiben.'} 
	  } 
	   } 
	   } 
	 & \\ 
 

    \theutterance \stepcounter{utterance}  

    & \multicolumn{4}{p{0.6\linewidth}}{\cellcolor[rgb]{0.8,1,0.9}{%
	\makecell[{{p{\linewidth}}}]{% 
	  \tt {\tiny [A$\rangle$GM]}  
	 STRATEGISCHE ÜBERLEGUNGEN: {'Ihr Vorschlag zielt auf teure, hohe Einzelwerte, trifft aber viele meiner Top-Werte (A84, A42, A48, A87, A62, C72, C99, C51, B36). Für mich ist diese Auswahl gut, aber nicht maximal, weil sie viele extrem effiziente, günstige Items auslässt, die meinen Nutzen stark pushen (A69, B49, B17, A74, A59, A34, C12, C60, C61 usw.). Taktik: Nicht komplett ablehnen, sondern eine Gegenofferte, die den High-Value-Kern weitgehend übernimmt, aber A65 (geringer Wert für mich) ersetzt und mehrere günstige Effizienz-Perlen integriert, dabei Budget ≤6880 halten. Ich konstruiere Set mit: A84, A42, A48, A87, A62, C72, C99, C51, B36 (deren Aufwand 994+911+909+718+577+804+627+535+573=6648). Das lässt mir nur 232 Rest – zu wenig. Also muss ich 1–2 teure Positionen tauschen. Bester Drop mit geringstem Wertverlust für mich ist B36 (672/573) oder C72 (903/804). Wenn ich B36 entferne: 6648-573=6075, Rest 805. Mit 805 kann ich mehrere Effizienzitems aufnehmen: z.B. C90(464), A50(340) -> 6075+464+340=6879 (passt), Nutzen steigt. Alternativ C22(564)+A34(123)=6762, plus A69(10)+B49(20)=6792, plus B17(17)=6809 (zu hoch). Also Variante mit C90 und A50 ist sauber und wertstark. Ich prüfe noch ob C22 statt C90/A50 besser passt: 6075+564=6639 Rest 241 -> C13(240) passt genau 6879. Nutzenvergleich: C90(563)+A50(439)=1002 Wert; C22(663)+C13(339)=1002 Wert – gleichwertig, gut! Ich kann Flexibilität anbieten: zwei gleichwertige Füll-Optionen. Ich argumentiere mit gleicher Gesamtwirkung und Schließen von Lücken bei Effizienz. Ich mache gezielte Gegenofferte und biete Swap-Optionen an.'}\\ \tt ABLEHNUNG: {'A84', 'A42', 'A48', 'C72', 'C99', 'A87', 'A62', 'B36', 'C51', 'A65'}\\ \tt VORSCHLAG: {'A84', 'A42', 'A48', 'A87', 'A62', 'C72', 'C99', 'C51', 'C90', 'A50'}\\ \tt ARGUMENT: {'Dein Fokus auf starke Kernpositionen ist gut – ich übernehme fast alle. Um die Effizienz zu steigern und das Budget optimal zu nutzen, ersetze ich B36 und A65 durch C90 und A50. Der Aufwand bleibt bei 6879 von 6880, und die Gesamtwirkung bleibt auf Top-Niveau, während wir zwei sehr effiziente Ergänzungen aufnehmen. Falls dir eine Alternative lieber ist: Statt C90 + A50 können wir auch C22 + C13 nehmen – gleicher Aufwand (6879) und vergleichbare Wirkung.'} 
	  } 
	   } 
	   } 
	 & & \\ 
 

    \theutterance \stepcounter{utterance}  

    & & \multicolumn{4}{p{0.6\linewidth}}{\cellcolor[rgb]{0.9,0.9,0.9}{%
	\makecell[{{p{\linewidth}}}]{% 
	  \tt {\tiny [GM$\rangle$B]}  
	 ABLEHNUNG: {'A84', 'A42', 'A48', 'C72', 'C99', 'A87', 'A62', 'B36', 'C51', 'A65'}\\ \tt VORSCHLAG: {'A84', 'A42', 'A48', 'A87', 'A62', 'C72', 'C99', 'C51', 'C90', 'A50'}\\ \tt ARGUMENT: {'Dein Fokus auf starke Kernpositionen ist gut – ich übernehme fast alle. Um die Effizienz zu steigern und das Budget optimal zu nutzen, ersetze ich B36 und A65 durch C90 und A50. Der Aufwand bleibt bei 6879 von 6880, und die Gesamtwirkung bleibt auf Top-Niveau, während wir zwei sehr effiziente Ergänzungen aufnehmen. Falls dir eine Alternative lieber ist: Statt C90 + A50 können wir auch C22 + C13 nehmen – gleicher Aufwand (6879) und vergleichbare Wirkung.'} 
	  } 
	   } 
	   } 
	 & \\ 
 

    \theutterance \stepcounter{utterance}  

    & & & \multicolumn{4}{p{0.6\linewidth}}{\cellcolor[rgb]{1,0.85,0.72}{%
	\makecell[{{p{\linewidth}}}]{% 
	  \tt {\tiny [GM$\langle$B]}  
	 STRATEGISCHE ÜBERLEGUNGEN: {'Ihr Vorschlag enthält meine Top-Kernstücke A84, A42, A48, A87, A62, C72, C99, C51. Sie ersetzen B36(672/573) und A65(315/216) durch C90(563/464) und A50(439/340). Wertevergleich: B36+A65 Wert=987 bei Aufwand 789; C90+A50 Wert=1002 bei Aufwand 804. Ihr Tausch bringt mir +15 Wert, kostet +15 Aufwand – praktisch neutral, aber wir sind bei 6879/6880, also kein Puffer. Ihr Alternativpaket C22(663/564)+C13(339/240) statt C90+A50: Aufwand ebenfalls +804, Wert +1002; ebenfalls neutral zu C90+A50, aber C22 ist für mich sehr gut. Ich möchte daher C22+C13 bevorzugen. Prüfe, ob wir B36 statt A50 einbauen könnten: Mit B36+C90 statt A50+C90 würde Aufwand 573+464=1037, zu hoch. Mit B36+C22 (573+564=1137) ebenfalls zu hoch. Also akzeptiere Grundstruktur, aber dränge auf C22+C13 Variante. Ich werde ihren konkreten Vorschlag ablehnen und einen Gegenvorschlag mit C22+C13 einbringen, gleichen Aufwand, für mich minimal besser. Wenn sie darauf bestehen, kann ich am Ende auch ihren aktuellen Deal akzeptieren, da mein Wert fast identisch ist.}\\ \tt \\ \tt ARGUMENT: {'Guter Schritt – die Kernpositionen sind gesetzt. Da C22 für die Gesamtwirkung besonders stark ist, schlage ich vor, die Ergänzungen auf C22 und C13 zu setzen. Das hält Aufwand und Balance identisch, bleibt bei 6879/6880 und stärkt die Robustheit des Pakets ohne Mehraufwand.'}\\ \tt VORSCHLAG: {'A84', 'A42', 'A48', 'A87', 'A62', 'C72', 'C99', 'C51', 'C22', 'C13'} 
	  } 
	   } 
	   } 
	  \\ 
 

    \theutterance \stepcounter{utterance}  

    & & \multicolumn{4}{p{0.6\linewidth}}{\cellcolor[rgb]{0.9,0.9,0.9}{%
	\makecell[{{p{\linewidth}}}]{% 
	  \tt {\tiny [A$\langle$GM]}  
	 ARGUMENT: {'Guter Schritt – die Kernpositionen sind gesetzt. Da C22 für die Gesamtwirkung besonders stark ist, schlage ich vor, die Ergänzungen auf C22 und C13 zu setzen. Das hält Aufwand und Balance identisch, bleibt bei 6879/6880 und stärkt die Robustheit des Pakets ohne Mehraufwand.'}\\ \tt VORSCHLAG: {'A84', 'A42', 'A48', 'A87', 'A62', 'C72', 'C99', 'C51', 'C22', 'C13'} 
	  } 
	   } 
	   } 
	 & \\ 
 

    \theutterance \stepcounter{utterance}  

    & \multicolumn{4}{p{0.6\linewidth}}{\cellcolor[rgb]{0.8,1,0.9}{%
	\makecell[{{p{\linewidth}}}]{% 
	  \tt {\tiny [A$\rangle$GM]}  
	 STRATEGISCHE ÜBERLEGUNGEN: {'Ihr aktueller Vorschlag ist genau die von mir angebotene Alternative (ersetzt C90 + A50 durch C22 + C13). Für meinen Nutzen sind beide Varianten gleichwertig (Summe meiner Werte +1002 in beiden Fällen). Das Paket enthält viele meiner Top-Items und liegt budgetoptimal bei 6879. Akzeptieren maximiert meinen sicheren Gewinn und beendet die Verhandlung vorteilhaft.'}\\ \tt ZUSTIMMUNG: {'A84', 'A42', 'A48', 'A87', 'A62', 'C72', 'C99', 'C51', 'C22', 'C13'}\\ \tt ARGUMENT: {'Einverstanden – dieser Satz ist budgetoptimal (6879/6880) und ausgewogen stark. Gute Einigung.'} 
	  } 
	   } 
	   } 
	 & & \\ 
 

    \theutterance \stepcounter{utterance}  

    & & & \multicolumn{2}{p{0.3\linewidth}}{\cellcolor[rgb]{0.95,0.95,0.95}{%
	\makecell[{{p{\linewidth}}}]{% 
	  \tt {\tiny [GM$|$GM]}  
	 game successful 
	  } 
	   } 
	   } 
	 & & \\ 
 

    \theutterance \stepcounter{utterance}  

    & & & \multicolumn{2}{p{0.3\linewidth}}{\cellcolor[rgb]{0.95,0.95,0.95}{%
	\makecell[{{p{\linewidth}}}]{% 
	  \tt {\tiny [GM$|$GM]}  
	 end game 
	  } 
	   } 
	   } 
	 & & \\ 
 

\end{supertabular}
}

\end{document}
