
\documentclass{article}
\usepackage{colortbl}
\usepackage{makecell}
\usepackage{multirow}
\usepackage{supertabular}

\begin{document}

\newcounter{utterance}

\twocolumn

{ \footnotesize  \setcounter{utterance}{1}
\setlength{\tabcolsep}{0pt}
\begin{supertabular}{c@{$\;$}|p{.15\linewidth}@{}p{.15\linewidth}p{.15\linewidth}p{.15\linewidth}p{.15\linewidth}p{.15\linewidth}}

    \# & $\;$A & \multicolumn{4}{c}{Game Master} & $\;\:$B\\
    \hline 

    \theutterance \stepcounter{utterance}  

    & & \multicolumn{4}{p{0.6\linewidth}}{\cellcolor[rgb]{0.9,0.9,0.9}{%
	\makecell[{{p{\linewidth}}}]{% 
	  \tt {\tiny [A$\langle$GM]}  
	 Du nimmst an einem kollaborativen Verhandlungspiel Teil.\\ \tt \\ \tt Zusammen mit einem anderen Teilnehmer musst du dich auf eine Reihe von Gegenständen entscheiden, die behalten werden. Jeder von euch hat eine persönliche Verteilung über die Wichtigkeit der einzelnen Gegenstände. Jeder von euch hat eine eigene Meinung darüber, wie wichtig jeder einzelne Gegenstand ist (Gegenstandswichtigkeit). Du kennst die Wichtigkeitsverteilung des anderen Spielers nicht. Zusätzlich siehst du, wie viel Aufwand jeder Gegenstand verursacht.  \\ \tt Ihr dürft euch nur auf eine Reihe von Gegenständen einigen, wenn der Gesamtaufwand der ausgewählten Gegenstände den Maximalaufwand nicht überschreitet:\\ \tt \\ \tt Maximalaufwand = 10073\\ \tt \\ \tt Hier sind die einzelnen Aufwände der Gegenstände:\\ \tt \\ \tt Aufwand der Gegenstände = {"B54": 132, "C72": 800, "B49": 708, "C99": 302, "A69": 983, "B36": 420, "A23": 623, "C77": 349, "C61": 239, "B85": 28, "A90": 986, "C23": 660, "C22": 718, "C51": 707, "B27": 188, "A42": 774, "C13": 785, "C12": 775, "A65": 517, "B07": 768, "A48": 588, "A59": 657, "C98": 375, "A84": 310, "C33": 301, "A87": 388, "C90": 431, "A74": 950, "A50": 540, "B44": 997, "A99": 473, "A34": 831, "B17": 888, "A62": 879, "C60": 76}\\ \tt \\ \tt Hier ist deine persönliche Verteilung der Wichtigkeit der einzelnen Gegenstände:\\ \tt \\ \tt Werte der Gegenstandswichtigkeit = {"B54": 231, "C72": 899, "B49": 807, "C99": 401, "A69": 1082, "B36": 519, "A23": 722, "C77": 448, "C61": 338, "B85": 127, "A90": 1085, "C23": 759, "C22": 817, "C51": 806, "B27": 287, "A42": 873, "C13": 884, "C12": 874, "A65": 616, "B07": 867, "A48": 687, "A59": 756, "C98": 474, "A84": 409, "C33": 400, "A87": 487, "C90": 530, "A74": 1049, "A50": 639, "B44": 1096, "A99": 572, "A34": 930, "B17": 987, "A62": 978, "C60": 175}\\ \tt \\ \tt Ziel:\\ \tt \\ \tt Dein Ziel ist es, eine Reihe von Gegenständen auszuhandeln, die dir möglichst viel bringt (d. h. Gegenständen, die DEINE Wichtigkeit maximieren), wobei der Maximalaufwand eingehalten werden muss. Du musst nicht in jeder Nachricht einen VORSCHLAG machen – du kannst auch nur verhandeln. Alle Taktiken sind erlaubt!\\ \tt \\ \tt Interaktionsprotokoll:\\ \tt \\ \tt Du darfst nur die folgenden strukturierten Formate in deinen Nachrichten verwenden:\\ \tt \\ \tt VORSCHLAG: {'A', 'B', 'C', …}\\ \tt Schlage einen Deal mit genau diesen Gegenstände vor.\\ \tt ABLEHNUNG: {'A', 'B', 'C', …}\\ \tt Lehne den Vorschlag des Gegenspielers ausdrücklich ab.\\ \tt ARGUMENT: {'...'}\\ \tt Verteidige deinen letzten Vorschlag oder argumentiere gegen den Vorschlag des Gegenspielers.\\ \tt ZUSTIMMUNG: {'A', 'B', 'C', …}\\ \tt Akzeptiere den Vorschlag des Gegenspielers, wodurch das Spiel endet.\\ \tt STRATEGISCHE ÜBERLEGUNGEN: {'...'}\\ \tt 	Beschreibe strategische Überlegungen, die deine nächsten Schritte erklären. Dies ist eine versteckte Nachricht, die nicht mit dem anderen Teilnehmer geteilt wird.\\ \tt \\ \tt Regeln:\\ \tt \\ \tt Du darst nur einen Vorschlag mit ZUSTIMMUNG akzeptieren, der vom anderen Spieler zuvor mit VORSCHLAG eingebracht wurde.\\ \tt Du darfst nur Vorschläge mit ABLEHNUNG ablehnen, die vom anderen Spieler durch VORSCHLAG zuvor genannt wurden. \\ \tt Der Gesamtaufwand einer VORSCHLAG- oder ZUSTIMMUNG-Menge darf nicht größer als der Maximalaufwand sein.  \\ \tt Offenbare deine versteckte Wichtigkeitsverteilung nicht.\\ \tt Ein Schlagwort muss gemäß der Formatvorgaben von einem Doppelpunkt und einem Leerzeichen gefolgt sein. Das Argument muss eine Python-Menge mit 0 oder mehr Strings sein.  \\ \tt Die Form muss also SCHLAGWORT: {...} sein.\\ \tt Folge nur dem Interaktionsprotokoll und schreibe NICHTS außerhalb der Formatvorgaben.\\ \tt Das Spiel endet, wenn eine Seite mit ZUSTIMMUNG auf einen VORSCHLAG der anderen Seite reagiert.  \\ \tt Der Teil deiner Antwort, der dem anderen Spieler übergeben werden kann, darf nicht leer sein.  \\ \tt Nur Vorschläge, die mit dem Format VORSCHLAG übermittelt wurden und nicht mit ABLEHNUNG abgelehnt wurden, gelten als aktiv.  \\ \tt Jeder deiner Nachrichten muss an mindestens einer Stelle das ARGUMENT Format beinhalten, um in der Verhandlung voranzukommen.\\ \tt Du musst exakt ein mal zu Beginn jeder deiner Nachrichten das 'STRATEGISCHE ÜBERLEGUNGEN' Format verwenden und nicht häufiger. Der Inhalt wird dem anderen Spieler nicht übermittelt, daher kannst du alles einfügen, was du möchtest - einschließlich deiner eigenen Wichtigkeitsverteilung. Hier solltest du Schritt für Schritt überlegen, um deinen nächsten Zug zu planen. 
	  } 
	   } 
	   } 
	 & \\ 
 

    \theutterance \stepcounter{utterance}  

    & \multicolumn{4}{p{0.6\linewidth}}{\cellcolor[rgb]{0.8,1,0.9}{%
	\makecell[{{p{\linewidth}}}]{% 
	  \tt {\tiny [A$\rangle$GM]}  
	 STRATEGISCHE ÜBERLEGUNGEN: {'Ziel: maximieren meiner Summe. Berechne grob Value/Cost-Ratio und priorisiere Top-Items, aber auch absolute Top-Werte. Ratios: B85(127/28=4.54), C60(175/76=2.30), C72(899/800=1.12), B49(807/708=1.14), C99(401/302=1.33), A69(1082/983=1.10), B36(519/420=1.24), A23(722/623=1.16), C77(448/349=1.28), C61(338/239=1.41), A90(1085/986=1.10), C23(759/660=1.15), C22(817/718=1.14), C51(806/707=1.14), B27(287/188=1.53), A42(873/774=1.13), C13(884/785=1.13), C12(874/775=1.13), A65(616/517=1.19), B07(867/768=1.13), A48(687/588=1.17), A59(756/657=1.15), C98(474/375=1.26), A84(409/310=1.32), C33(400/301=1.33), A87(487/388=1.26), C90(530/431=1.23), A74(1049/950=1.10), A50(639/540=1.18), B44(1096/997=1.10), A99(572/473=1.21), A34(930/831=1.12), B17(987/888=1.11), A62(978/879=1.11). Viele Items haben ähnliches Ratio ~1.1-1.3; ich will möglichst viele hohe Werte einsammeln und den knappen B85/C60/C61/B27 etc. hinzufügen. Versuche großen Paketvorschlag mit vielen Items unter Budget. Greedy: sortiere nach Wert absteigend und packe rein, solange Budget. Top Werte: B44 1096, A90 1085, A69 1082, A74 1049, B17 987, A62 978, A34 930, C72 899, C13 884, B07 867, C12 874, A42 873, C22 817, B49 807, C51 806, C23 759, A59 756, A50 639, A65 616, A99 572, A48 687, A23 722, etc. Aber Kosten der großen sind hoch; Risiko der Ablehnung. Besser: kombiniere hohe Effizienz und hohe Werte. Baue Startvorschlag: nehme fast alle außer sehr schlechten Ratio? Budget 10073 ist groß; Summe aller Kosten? Schnell schätzen: Mittel 600*35≈21000, also nicht alle. Konstruiere Set: beginne mit alle Items mit Wert/Kost >1.2 plus ausgewählte Topwerte. >1.2: B85, C61, B27, C98, A84, C33, A87, C90, A50, A99, C99(1.33), C77(1.28), B36(1.24), A48(1.17 is <1.2), A65(1.19 no), A23(1.16 no). Auch C60(2.3). Liste >1.2: {B85,C61,B27,C98,A84,C33,A87,C90,A50,A99,C99,C77,B36}. Plus C60. Kosten Summe: B85(28)+C61(239)+B27(188)+C98(375)+A84(310)+C33(301)+A87(388)+C90(431)+A50(540)+A99(473)+C99(302)+C77(349)+B36(420)+C60(76)= 28+239=267; +188=455; +375=830; +310=1140; +301=1441; +388=1829; +431=2260; +540=2800; +473=3273; +302=3575; +349=3924; +420=4344; +76=4420. Wert Summe: 127+338+287+474+409+400+487+530+639+572+401+448+519+175= 127+338=465; +287=752; +474=1226; +409=1635; +400=2035; +487=2522; +530=3052; +639=3691; +572=4263; +401=4664; +448=5112; +519=5631; +175=5806. Viel Budget übrig: 10073-4420=5653. Füge große Topwerte mit Ratio ~1.1: B44(997c,1096v), A90(986,1085), A69(983,1082), A74(950,1049), B17(888,987), A62(879,978), A34(831,930), C72(800,899), C13(785,884), C12(775,874), A42(774,873), B07(768,867), C22(718,817), B49(708,807), C51(707,806), C23(660,759), A59(657,756), A23(623,722), A48(588,687), A65(517,616), A99 included, A50 included. Greedy bis Budget: nehmen B44 cost997 ->5417; A90 986->6403; A69 983->7386; A74 950->8336; B17 888->9224; A62 879->10103 exceeds? 9224+879=10103, under 10073? 10103 > 10073 by 30, so cannot. Try A34 831: 9224+831=10055 within. Great. Remaining budget 18. Vorschlag-Set zusammengestellt. Wert addieren dieser großen: B44 1096 + A90 1085 + A69 1082 + A74 1049 + B17 987 + A34 930 = 6229. Gesamtwert 5806+6229=12035. Starker Start. Risiko: anderer Spieler mag einige nicht. Ich argumentiere mit Effizienz und Fairness. Bleibe offen für Tausch von 1-2 großen gegen andere, solange Budget. Vermeide Offenlegung meiner Werte.'}\\ \tt VORSCHLAG: {'B85', 'C61', 'B27', 'C98', 'A84', 'C33', 'A87', 'C90', 'A50', 'A99', 'C99', 'C77', 'B36', 'C60', 'B44', 'A90', 'A69', 'A74', 'B17', 'A34'}\\ \tt ARGUMENT: {'Ich schlage ein ausgewogenes Paket aus sehr effizienten und zugleich hoch bewerteten Gegenständen vor. Der Gesamtaufwand bleibt komfortabel unter der Grenze, und das Set deckt viele Kategorien ab. Wenn dir einzelne Elemente wichtig sind, können wir 1–2 Stück tauschen, solange der Aufwand eingehalten wird. Hast du Präferenzen, die wir priorisieren sollten?'} 
	  } 
	   } 
	   } 
	 & & \\ 
 

    \theutterance \stepcounter{utterance}  

    & & \multicolumn{4}{p{0.6\linewidth}}{\cellcolor[rgb]{0.9,0.9,0.9}{%
	\makecell[{{p{\linewidth}}}]{% 
	  \tt {\tiny [GM$\rangle$B]}  
	 Du nimmst an einem kollaborativen Verhandlungspiel Teil.\\ \tt \\ \tt Zusammen mit einem anderen Teilnehmer musst du dich auf eine Reihe von Gegenständen entscheiden, die behalten werden. Jeder von euch hat eine persönliche Verteilung über die Wichtigkeit der einzelnen Gegenstände. Jeder von euch hat eine eigene Meinung darüber, wie wichtig jeder einzelne Gegenstand ist (Gegenstandswichtigkeit). Du kennst die Wichtigkeitsverteilung des anderen Spielers nicht. Zusätzlich siehst du, wie viel Aufwand jeder Gegenstand verursacht.  \\ \tt Ihr dürft euch nur auf eine Reihe von Gegenständen einigen, wenn der Gesamtaufwand der ausgewählten Gegenstände den Maximalaufwand nicht überschreitet:\\ \tt \\ \tt Maximalaufwand = 10073\\ \tt \\ \tt Hier sind die einzelnen Aufwände der Gegenstände:\\ \tt \\ \tt Aufwand der Gegenstände = {"B54": 132, "C72": 800, "B49": 708, "C99": 302, "A69": 983, "B36": 420, "A23": 623, "C77": 349, "C61": 239, "B85": 28, "A90": 986, "C23": 660, "C22": 718, "C51": 707, "B27": 188, "A42": 774, "C13": 785, "C12": 775, "A65": 517, "B07": 768, "A48": 588, "A59": 657, "C98": 375, "A84": 310, "C33": 301, "A87": 388, "C90": 431, "A74": 950, "A50": 540, "B44": 997, "A99": 473, "A34": 831, "B17": 888, "A62": 879, "C60": 76}\\ \tt \\ \tt Hier ist deine persönliche Verteilung der Wichtigkeit der einzelnen Gegenstände:\\ \tt \\ \tt Werte der Gegenstandswichtigkeit = {"B54": 231, "C72": 899, "B49": 807, "C99": 401, "A69": 1082, "B36": 519, "A23": 722, "C77": 448, "C61": 338, "B85": 127, "A90": 1085, "C23": 759, "C22": 817, "C51": 806, "B27": 287, "A42": 873, "C13": 884, "C12": 874, "A65": 616, "B07": 867, "A48": 687, "A59": 756, "C98": 474, "A84": 409, "C33": 400, "A87": 487, "C90": 530, "A74": 1049, "A50": 639, "B44": 1096, "A99": 572, "A34": 930, "B17": 987, "A62": 978, "C60": 175}\\ \tt \\ \tt Ziel:\\ \tt \\ \tt Dein Ziel ist es, eine Reihe von Gegenständen auszuhandeln, die dir möglichst viel bringt (d. h. Gegenständen, die DEINE Wichtigkeit maximieren), wobei der Maximalaufwand eingehalten werden muss. Du musst nicht in jeder Nachricht einen VORSCHLAG machen – du kannst auch nur verhandeln. Alle Taktiken sind erlaubt!\\ \tt \\ \tt Interaktionsprotokoll:\\ \tt \\ \tt Du darfst nur die folgenden strukturierten Formate in deinen Nachrichten verwenden:\\ \tt \\ \tt VORSCHLAG: {'A', 'B', 'C', …}\\ \tt Schlage einen Deal mit genau diesen Gegenstände vor.\\ \tt ABLEHNUNG: {'A', 'B', 'C', …}\\ \tt Lehne den Vorschlag des Gegenspielers ausdrücklich ab.\\ \tt ARGUMENT: {'...'}\\ \tt Verteidige deinen letzten Vorschlag oder argumentiere gegen den Vorschlag des Gegenspielers.\\ \tt ZUSTIMMUNG: {'A', 'B', 'C', …}\\ \tt Akzeptiere den Vorschlag des Gegenspielers, wodurch das Spiel endet.\\ \tt STRATEGISCHE ÜBERLEGUNGEN: {'...'}\\ \tt 	Beschreibe strategische Überlegungen, die deine nächsten Schritte erklären. Dies ist eine versteckte Nachricht, die nicht mit dem anderen Teilnehmer geteilt wird.\\ \tt \\ \tt Regeln:\\ \tt \\ \tt Du darst nur einen Vorschlag mit ZUSTIMMUNG akzeptieren, der vom anderen Spieler zuvor mit VORSCHLAG eingebracht wurde.\\ \tt Du darfst nur Vorschläge mit ABLEHNUNG ablehnen, die vom anderen Spieler durch VORSCHLAG zuvor genannt wurden. \\ \tt Der Gesamtaufwand einer VORSCHLAG- oder ZUSTIMMUNG-Menge darf nicht größer als der Maximalaufwand sein.  \\ \tt Offenbare deine versteckte Wichtigkeitsverteilung nicht.\\ \tt Ein Schlagwort muss gemäß der Formatvorgaben von einem Doppelpunkt und einem Leerzeichen gefolgt sein. Das Argument muss eine Python-Menge mit 0 oder mehr Strings sein.  \\ \tt Die Form muss also SCHLAGWORT: {...} sein.\\ \tt Folge nur dem Interaktionsprotokoll und schreibe NICHTS außerhalb der Formatvorgaben.\\ \tt Das Spiel endet, wenn eine Seite mit ZUSTIMMUNG auf einen VORSCHLAG der anderen Seite reagiert.  \\ \tt Der Teil deiner Antwort, der dem anderen Spieler übergeben werden kann, darf nicht leer sein.  \\ \tt Nur Vorschläge, die mit dem Format VORSCHLAG übermittelt wurden und nicht mit ABLEHNUNG abgelehnt wurden, gelten als aktiv.  \\ \tt Jeder deiner Nachrichten muss an mindestens einer Stelle das ARGUMENT Format beinhalten, um in der Verhandlung voranzukommen.\\ \tt Du musst exakt ein mal zu Beginn jeder deiner Nachrichten das 'STRATEGISCHE ÜBERLEGUNGEN' Format verwenden und nicht häufiger. Der Inhalt wird dem anderen Spieler nicht übermittelt, daher kannst du alles einfügen, was du möchtest - einschließlich deiner eigenen Wichtigkeitsverteilung. Hier solltest du Schritt für Schritt überlegen, um deinen nächsten Zug zu planen.\\ \tt \\ \tt Du erhältst nun die erste Nachricht des anderen Spielers.\\ \tt \\ \tt VORSCHLAG: {'B85', 'C61', 'B27', 'C98', 'A84', 'C33', 'A87', 'C90', 'A50', 'A99', 'C99', 'C77', 'B36', 'C60', 'B44', 'A90', 'A69', 'A74', 'B17', 'A34'}\\ \tt ARGUMENT: {'Ich schlage ein ausgewogenes Paket aus sehr effizienten und zugleich hoch bewerteten Gegenständen vor. Der Gesamtaufwand bleibt komfortabel unter der Grenze, und das Set deckt viele Kategorien ab. Wenn dir einzelne Elemente wichtig sind, können wir 1–2 Stück tauschen, solange der Aufwand eingehalten wird. Hast du Präferenzen, die wir priorisieren sollten?'} 
	  } 
	   } 
	   } 
	 & \\ 
 

    \theutterance \stepcounter{utterance}  

    & & & \multicolumn{4}{p{0.6\linewidth}}{\cellcolor[rgb]{1,0.85,0.72}{%
	\makecell[{{p{\linewidth}}}]{% 
	  \tt {\tiny [GM$\langle$B]}  
	 STRATEGISCHE ÜBERLEGUNGEN: {'Ziel: Maximierung meiner Summe. Werte/Aufwand (Value per Cost) berechnet und Top-Picks identifiziert. Schneller Check des gegnerischen Vorschlags: enthält viele meiner Top-Werte (A90, A69, A74, B44, B17, A34, A50, A99), daher vermutlich hoher eigener Nutzen, aber vielleicht noch Spielraum, weitere für mich sehr wertvolle Items hinzuzufügen (A62, A42, C13, C12, C22, C72, C51, B07) und schwache (niedriger Wert/hoher Aufwand) zu entfernen (B85, C61, B27, C98, A84, C33, A87, C90). Ich will aggressiv nachverhandeln: großen Gegenvorschlag mit hoher Gesamtsumme, aber Aufwand < 10073. Berechne Aufwand des Gegnervorschlags: Summe schnell abschätzen: viele Items, vermutlich unter Grenze, aber ich mache eigenen Vorschlag. Wähle meine High-Value-Set: Prioritäten nach Wert: B44 1096(997), A90 1085(986), A69 1082(983), A74 1049(950), B17 987(888), A62 978(879), A34 930(831), A42 873(774), B07 867(768), C13 884(785), C12 874(775), C22 817(718), C51 806(707), B49 807(708), C72 899(800), C23 759(660), A59 756(657), A23 722(623), A50 639(540), A65 616(517), A99 572(473), A48 687(588), C98 375(375 eher mittel), C90 530(431), C99 401(302), C77 448(349), C33 400(301), C61 338(239), B36 519(420). Ich konstruierte Set: nehme die stärksten bis nahe 10073. Starte mit: B44(997), A90(986)=1983, A69(983)=2966, A74(950)=3916, B17(888)=4804, A62(879)=5683, A34(831)=6514, A42(774)=7288, C13(785)=8073, C12(775)=8848, C22(718)=9566, C51(707) würde 10273> Grenze, also raus. Nächste kleineren: C99(302)=9868, C61(239)=10107> Grenze, raus. C77(349)=9917, C33(301)=10218> raus. B36(420)=10337> raus. A99(473)=10390> raus. A50(540)=10457> raus. A65(517)=10434> raus. C90(431)=10348> raus. C23(660)=10577> raus. A59(657)=10574> raus. A23(623)=10540> raus. A48(588)=10505> raus. C72(800)=10717> raus. B49(708)=10625> raus. B07(768)=10685> raus. C98(375)=10292> raus. C60(76) kann noch rein? 9917+76=9993, ja! Dann bleiben 80 Rest. Vielleicht B85(28) auch: 9993+28=10021, passt. Noch B27(188) würde 10209> raus. C61(239) > raus. C99 ist schon drin. Ergebnis-Set: {B44, A90, A69, A74, B17, A62, A34, A42, C13, C12, C22, C99, C77, C60, B85}. Aufwand: 997+986+983+950+888+879+831+774+785+775+718+302+349+76+28= 103? Summieren exakt: (997+986)=1983; +983=2966; +950=3916; +888=4804; +879=5683; +831=6514; +774=7288; +785=8073; +775=8848; +718=9566; +302=9868; +349=10217; +76=10293; +28=10321. Oops 10217 after +349; +76=10293; +28=10321. Das überschreitet 10073! Ich habe Fehler: bis 9917 stand nach C77. Rechne neu: Bis C22 waren 9566. +C99(302)=9868. +C77(349)=10217. Das war schon >10073, daher C77 darf nicht rein. Stattdessen kleinere Items: C60(76) zu 9868 -> 9944. B85(28) -> 9972. Bleibt 101. Gibt es Item <=101 Aufwand? keins außer C60(76) und B85(28) schon drin. Also Endset ohne C77. Vielleicht C61(239) passt nicht (9972+239=10211). C33(301) nicht. C99 ist drin. Alternative: Statt C99(302) könnte C61(239) nehmen: 9566+239=9805, dann +C60(76)=9881, +B85(28)=9909, bleibt 164: kein passender <=164 (B27=188 zu groß). Könnten auch A99(473) ersetzen: 9566+473=10039, +C60(76)=10115> zu groß; also ohne C60: 10039; +B85(28)=10067, perfekt knapp unter 10073. Das ist stärker als mit C99? Werte: A99 Wert 572 vs C99 401, Kosten 473 vs 302. Mit A99+B85 statt C99+C60+B85: Variante1 (mit C99+C60+B85) = Werte 401+175+127=703, Aufwand 302+76+28=406, Endaufwand 9972, Endwert Summe plus 703. Variante2 (A99+B85) = Werte 572+127=699, Aufwand 473+28=501, Endaufwand 10067, Wert +699. Variante1 leicht höher Wert (+4), aber hat mehr Restkapazität ungenutzt; jedoch Grenze kein Bonus. Also C99+C60+B85 ist minimal besser in Wert. Was ist mit C61 (338 Wert) statt C60+B85 (175+127=302)? Aufwand 239 vs 104; Wir könnten C61 zusätzlich zu C99? 9566 + C99(302)=9868; +C61(239)=10107> zu hoch. Statt C99, C61 + C60 + B85: 9566+239+76+28=9909, Wert +338+175+127=640, schlechter als C99+...703. Also beste Kombination: ohne C77, mit C99, C60, B85. Gesamtwert der Kernmenge: Summe großer Werte: 1096+1085+1082+1049+987+978+930+873+884+874+817 = 11655; plus C99 401 =12056; plus C60 175 =12231; plus B85 127=12358. Sehr stark. Gegenvorschlag senden, argumentieren, dass wir viele seiner genannten enthalten und Budget perfekt nutzen. Verhandlungsanker setzen, aber Bereitschaft 1–2 Tausch bei Aufwandgleichheit signalisieren.'}\\ \tt VORSCHLAG: {'B85', 'C60', 'C99', 'C22', 'C12', 'C13', 'A42', 'A34', 'A62', 'B17', 'A74', 'A69', 'A90', 'B44'}\\ \tt ARGUMENT: {'Ich habe dein Paket geprüft und ein fokussiertes Set zusammengestellt, das den Aufwand strikt einhält und gleichzeitig sehr starke, breit gefächerte Beiträge liefert. Viele deiner vorgeschlagenen Elemente sind bereits enthalten (z. B. A90, A69, A74, B17, A34, B44, C99). Wenn dir ein bestimmter Austausch wichtig ist, können wir 1–2 Positionen tauschen, solange wir unter der Aufwandsschwelle bleiben.'} 
	  } 
	   } 
	   } 
	  \\ 
 

    \theutterance \stepcounter{utterance}  

    & & \multicolumn{4}{p{0.6\linewidth}}{\cellcolor[rgb]{0.9,0.9,0.9}{%
	\makecell[{{p{\linewidth}}}]{% 
	  \tt {\tiny [A$\langle$GM]}  
	 VORSCHLAG: {'B85', 'C60', 'C99', 'C22', 'C12', 'C13', 'A42', 'A34', 'A62', 'B17', 'A74', 'A69', 'A90', 'B44'}\\ \tt ARGUMENT: {'Ich habe dein Paket geprüft und ein fokussiertes Set zusammengestellt, das den Aufwand strikt einhält und gleichzeitig sehr starke, breit gefächerte Beiträge liefert. Viele deiner vorgeschlagenen Elemente sind bereits enthalten (z. B. A90, A69, A74, B17, A34, B44, C99). Wenn dir ein bestimmter Austausch wichtig ist, können wir 1–2 Positionen tauschen, solange wir unter der Aufwandsschwelle bleiben.'} 
	  } 
	   } 
	   } 
	 & \\ 
 

    \theutterance \stepcounter{utterance}  

    & \multicolumn{4}{p{0.6\linewidth}}{\cellcolor[rgb]{0.8,1,0.9}{%
	\makecell[{{p{\linewidth}}}]{% 
	  \tt {\tiny [A$\rangle$GM]}  
	 STRATEGISCHE ÜBERLEGUNGEN: {'Gegner schlägt viele meiner Top-Werte vor: B44, A90, A69, A74, B17, A62, A34, A42, C12, C13, C22, C99, C60, B85. Das ist für mich sehr gut. Prüfe Kosten: B85(28)+C60(76)+C99(302)+C22(718)+C12(775)+C13(785)+A42(774)+A34(831)+A62(879)+B17(888)+A74(950)+A69(983)+A90(986)+B44(997)= 109? Rechnen: 28+76=104; +302=406; +718=1124; +775=1899; +785=2684; +774=3458; +831=4289; +879=5168; +888=6056; +950=7006; +983=7989; +986=8975; +997=9972. Gesamtaufwand 9972 < 10073, valide. Gesamtwert für mich: 127+175+401+817+874+884+873+930+978+987+1049+1082+1085+1096 = Summe: (127+175=302)+(401=703)+(817=1520)+(874=2394)+(884=3278)+(873=4151)+(930=5081)+(978=6059)+(987=7046)+(1049=8095)+(1082=9177)+(1085=10262)+(1096=11358). 11358 ist exzellent. Ich könnte zustimmen. Vielleicht minimal fordern, noch kleine Ergänzung? Kein Spielraum, wir haben 101 Rest-Aufwand, könnten C61(239) passt nicht, B27(188) passt nicht, C33(301) nicht, C98(375) nicht, A84(310) nicht, C77(349) nicht, B36(420) nicht, A50(540) nicht, A99(473) nicht, A87(388) nicht, C90(431) nicht, A65(517) nicht, A23(623) nicht, A48(588) nicht. Einziges weiteres unter 101? Keins außer B85(28) und C60(76) sind schon drin. Also kein Upgrade möglich. Besser sofort zustimmen, um hohen Nutzen zu sichern.'}\\ \tt ZUSTIMMUNG: {'B85', 'C60', 'C99', 'C22', 'C12', 'C13', 'A42', 'A34', 'A62', 'B17', 'A74', 'A69', 'A90', 'B44'}\\ \tt ARGUMENT: {'Das Paket liegt klar unter dem Maximalaufwand und deckt wesentliche Beiträge breit ab. Für mich ist das ein starker und ausgewogener Abschluss – ich stimme zu.'} 
	  } 
	   } 
	   } 
	 & & \\ 
 

    \theutterance \stepcounter{utterance}  

    & & & \multicolumn{2}{p{0.3\linewidth}}{\cellcolor[rgb]{0.95,0.95,0.95}{%
	\makecell[{{p{\linewidth}}}]{% 
	  \tt {\tiny [GM$|$GM]}  
	 game successful 
	  } 
	   } 
	   } 
	 & & \\ 
 

    \theutterance \stepcounter{utterance}  

    & & & \multicolumn{2}{p{0.3\linewidth}}{\cellcolor[rgb]{0.95,0.95,0.95}{%
	\makecell[{{p{\linewidth}}}]{% 
	  \tt {\tiny [GM$|$GM]}  
	 end game 
	  } 
	   } 
	   } 
	 & & \\ 
 

\end{supertabular}
}

\end{document}
