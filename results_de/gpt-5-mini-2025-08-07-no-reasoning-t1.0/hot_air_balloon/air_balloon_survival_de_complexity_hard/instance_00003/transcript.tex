
\documentclass{article}
\usepackage{colortbl}
\usepackage{makecell}
\usepackage{multirow}
\usepackage{supertabular}

\begin{document}

\newcounter{utterance}

\centering \large Interaction Transcript for game `hot\_air\_balloon', experiment `air\_balloon\_survival\_de\_complexity\_hard', episode 3 with gpt{-}5{-}mini{-}2025{-}08{-}07{-}no{-}reasoning{-}t1.0.
\vspace{24pt}

{ \footnotesize  \setcounter{utterance}{1}
\setlength{\tabcolsep}{0pt}
\begin{supertabular}{c@{$\;$}|p{.15\linewidth}@{}p{.15\linewidth}p{.15\linewidth}p{.15\linewidth}p{.15\linewidth}p{.15\linewidth}}
    \# & $\;$A & \multicolumn{4}{c}{Game Master} & $\;\:$B\\
    \hline

    \theutterance \stepcounter{utterance}  
    & & \multicolumn{4}{p{0.6\linewidth}}{
        \cellcolor[rgb]{0.9,0.9,0.9}{
            \makecell[{{p{\linewidth}}}]{
                \texttt{\tiny{[P1$\langle$GM]}}
                \texttt{Du nimmst an einem kollaborativen Verhandlungspiel Teil.} \\
\\ 
\texttt{Zusammen mit einem anderen Teilnehmer musst du dich auf eine Reihe von Gegenständen entscheiden, die behalten werden. Jeder von euch hat eine persönliche Verteilung über die Wichtigkeit der einzelnen Gegenstände. Jeder von euch hat eine eigene Meinung darüber, wie wichtig jeder einzelne Gegenstand ist (Gegenstandswichtigkeit). Du kennst die Wichtigkeitsverteilung des anderen Spielers nicht. Zusätzlich siehst du, wie viel Aufwand jeder Gegenstand verursacht.  } \\
\texttt{Ihr dürft euch nur auf eine Reihe von Gegenständen einigen, wenn der Gesamtaufwand der ausgewählten Gegenstände den Maximalaufwand nicht überschreitet:} \\
\\ 
\texttt{Maximalaufwand = 6938} \\
\\ 
\texttt{Hier sind die einzelnen Aufwände der Gegenstände:} \\
\\ 
\texttt{Aufwand der Gegenstände = \{"B54": 997, "C72": 950, "B49": 272, "C99": 448, "A69": 382, "B36": 804, "A23": 237, "C77": 56, "C61": 105, "B85": 612, "A90": 528, "C23": 527, "C22": 524, "C51": 167, "B27": 133, "A42": 300, "C13": 945, "C12": 50, "A65": 906, "B07": 70, "A48": 224, "A59": 4, "C98": 689, "A84": 64, "C33": 434, "A87": 750, "C90": 733, "A74": 873, "A50": 22, "B44": 68, "A99": 57, "A34": 10, "B17": 36, "A62": 551, "C60": 348\}} \\
\\ 
\texttt{Hier ist deine persönliche Verteilung der Wichtigkeit der einzelnen Gegenstände:} \\
\\ 
\texttt{Werte der Gegenstandswichtigkeit = \{"B54": 1096, "C72": 1049, "B49": 371, "C99": 547, "A69": 481, "B36": 903, "A23": 336, "C77": 155, "C61": 204, "B85": 711, "A90": 627, "C23": 626, "C22": 623, "C51": 266, "B27": 232, "A42": 399, "C13": 1044, "C12": 149, "A65": 1005, "B07": 169, "A48": 323, "A59": 103, "C98": 788, "A84": 163, "C33": 533, "A87": 849, "C90": 832, "A74": 972, "A50": 121, "B44": 167, "A99": 156, "A34": 109, "B17": 135, "A62": 650, "C60": 447\}} \\
\\ 
\texttt{Ziel:} \\
\\ 
\texttt{Dein Ziel ist es, eine Reihe von Gegenständen auszuhandeln, die dir möglichst viel bringt (d. h. Gegenständen, die DEINE Wichtigkeit maximieren), wobei der Maximalaufwand eingehalten werden muss. Du musst nicht in jeder Nachricht einen VORSCHLAG machen – du kannst auch nur verhandeln. Alle Taktiken sind erlaubt!} \\
\\ 
\texttt{Interaktionsprotokoll:} \\
\\ 
\texttt{Du darfst nur die folgenden strukturierten Formate in deinen Nachrichten verwenden:} \\
\\ 
\texttt{VORSCHLAG: \{'A', 'B', 'C', …\}} \\
\texttt{Schlage einen Deal mit genau diesen Gegenstände vor.} \\
\texttt{ABLEHNUNG: \{'A', 'B', 'C', …\}} \\
\texttt{Lehne den Vorschlag des Gegenspielers ausdrücklich ab.} \\
\texttt{ARGUMENT: \{'...'\}} \\
\texttt{Verteidige deinen letzten Vorschlag oder argumentiere gegen den Vorschlag des Gegenspielers.} \\
\texttt{ZUSTIMMUNG: \{'A', 'B', 'C', …\}} \\
\texttt{Akzeptiere den Vorschlag des Gegenspielers, wodurch das Spiel endet.} \\
\texttt{STRATEGISCHE ÜBERLEGUNGEN: \{'...'\}} \\
\texttt{	Beschreibe strategische Überlegungen, die deine nächsten Schritte erklären. Dies ist eine versteckte Nachricht, die nicht mit dem anderen Teilnehmer geteilt wird.} \\
\\ 
\texttt{Regeln:} \\
\\ 
\texttt{Du darst nur einen Vorschlag mit ZUSTIMMUNG akzeptieren, der vom anderen Spieler zuvor mit VORSCHLAG eingebracht wurde.} \\
\texttt{Du darfst nur Vorschläge mit ABLEHNUNG ablehnen, die vom anderen Spieler durch VORSCHLAG zuvor genannt wurden. } \\
\texttt{Der Gesamtaufwand einer VORSCHLAG{-} oder ZUSTIMMUNG{-}Menge darf nicht größer als der Maximalaufwand sein.  } \\
\texttt{Offenbare deine versteckte Wichtigkeitsverteilung nicht.} \\
\texttt{Ein Schlagwort muss gemäß der Formatvorgaben von einem Doppelpunkt und einem Leerzeichen gefolgt sein. Das Argument muss eine Python{-}Menge mit 0 oder mehr Strings sein.  } \\
\texttt{Die Form muss also SCHLAGWORT: \{...\} sein.} \\
\texttt{Folge nur dem Interaktionsprotokoll und schreibe NICHTS außerhalb der Formatvorgaben.} \\
\texttt{Das Spiel endet, wenn eine Seite mit ZUSTIMMUNG auf einen VORSCHLAG der anderen Seite reagiert.  } \\
\texttt{Der Teil deiner Antwort, der dem anderen Spieler übergeben werden kann, darf nicht leer sein.  } \\
\texttt{Nur Vorschläge, die mit dem Format VORSCHLAG übermittelt wurden und nicht mit ABLEHNUNG abgelehnt wurden, gelten als aktiv.  } \\
\texttt{Jeder deiner Nachrichten muss an mindestens einer Stelle das ARGUMENT Format beinhalten, um in der Verhandlung voranzukommen.} \\
\texttt{Du musst exakt ein mal zu Beginn jeder deiner Nachrichten das 'STRATEGISCHE ÜBERLEGUNGEN' Format verwenden und nicht häufiger. Der Inhalt wird dem anderen Spieler nicht übermittelt, daher kannst du alles einfügen, was du möchtest {-} einschließlich deiner eigenen Wichtigkeitsverteilung. Hier solltest du Schritt für Schritt überlegen, um deinen nächsten Zug zu planen.} \\
            }
        }
    }
    & \\ \\

    \theutterance \stepcounter{utterance}  
    & \multicolumn{4}{p{0.6\linewidth}}{
        \cellcolor[rgb]{0.9,0.9,0.9}{
            \makecell[{{p{\linewidth}}}]{
                \texttt{\tiny{[P1$\rangle$GM]}}
                \texttt{STRATEGISCHE ÜBERLEGUNGEN: \{'Ziel: Start mit sehr hohem Nutzen/geringem Aufwand{-}Bundle, um Anker zu setzen und Flexibilität zu signalisieren. Kapazität 6938. Ich maximiere Summe Wert bei Aufwand <= 6938. Greedy nach Wert/Aufwand{-}Quote und hoher absoluter Wert. Kandidaten mit Top{-}Quote: A59(103/4=25.8), A50(121/22=5.5), C12(149/50=3.0), C77(155/56=2.77), A34(109/10=10.9), A84(163/64=2.55), B07(169/70=2.41), B44(167/68=2.46), C61(204/105=1.94), C51(266/167=1.59), B49(371/272=1.36), A48(323/224=1.44), A42(399/300=1.33), C99(547/448=1.22), C33(533/434=1.23), A69(481/382=1.26), A90(627/528=1.19), C60(447/348=1.28), C23(626/527=1.19), C22(623/524=1.19), B36(903/804=1.12), B85(711/612=1.16), A62(650/551=1.18), C98(788/689=1.14), C72(1049/950=1.10), B54(1096/997=1.10), C13(1044/945=1.10), A74(972/873=1.11), A65(1005/906=1.11), A87(849/750=1.13). Beginne mit fast vollem Rucksack aus sehr hoher Quote plus einige High{-}Value: Packe erst alle super günstigen: A59,A50,A34,C12,C77,A84,B07,B44,C61,C51,B49,A48,A42,C60,A69,C33,C99,A90,C23,C22,A62,A87,A74,A65,C13,C72,B54,B36,B85,C98 soweit Kapazität reicht. Rechne: Aufwand summieren schrittweise. Schnell kalkulieren: A59 4, A50 22 (26), A34 10 (36), C12 50 (86), C77 56 (142), A84 64 (206), B07 70 (276), B44 68 (344), C61 105 (449), C51 167 (616), B49 272 (888), A48 224 (1112), A42 300 (1412), C60 348 (1760), A69 382 (2142), C33 434 (2576), C99 448 (3024), A90 528 (3552), C23 527 (4079), C22 524 (4603), A62 551 (5154), A87 750 (5904), A74 873 (6777) {-}> jetzt 6777, Restkapazität 161. Ich kann noch A99 57 (Wert 156) hinzufügen, B17 36 (135), B27 133 (232) aber 133 passt, beide? 161 Rest: B17 36 {-}> 6813, A99 57 {-}> 6870, bleibt 68 {-}> B44 ist schon drin; C51 schon drin; C33 drin; C61 drin. C77 drin. B27 133 passt nicht mehr. C51 167 passt nicht. C90 733 passt nicht. C12 drin. A23 237 passt nicht. A50 drin. A84 drin. A34 drin. C61 drin. Damit Bundle: \{A59,A50,A34,C12,C77,A84,B07,B44,C61,C51,B49,A48,A42,C60,A69,C33,C99,A90,C23,C22,A62,A87,A74,B17,A99\}. Aufwand 6870 <= 6938. Nutzen sum: grob: 103+121+109+149+155+163+169+167+204+266+371+323+399+447+481+533+547+627+626+623+650+849+972+135+156 = Summe: (1000er) none; add step: 103+121=224; +109=333; +149=482; +155=637; +163=800; +169=969; +167=1136; +204=1340; +266=1606; +371=1977; +323=2300; +399=2699; +447=3146; +481=3627; +533=4160; +547=4707; +627=5334; +626=5960; +623=6583; +650=7233; +849=8082; +972=9054; +135=9189; +156=9345. Sehr hoher Wert. Ich lasse absichtlich einige teure High{-}Value (A65, B54, C72, C13, B36, B85, C98, C90) draußen, um Verhandlungsspielraum zu haben. Ich argumentiere mit Effizienz und knappen Puffer 68. Ich stelle Fragen zu ihren Prioritäten und signalisiere Bereitschaft, einzelne Items zu tauschen, solange Aufwand eingehalten wird.'\}} \\
\texttt{VORSCHLAG: \{'A59', 'A50', 'A34', 'C12', 'C77', 'A84', 'B07', 'B44', 'C61', 'C51', 'B49', 'A48', 'A42', 'C60', 'A69', 'C33', 'C99', 'A90', 'C23', 'C22', 'A62', 'A87', 'A74', 'B17', 'A99'\}} \\
\texttt{ARGUMENT: \{'Ich schlage ein hocheffizientes Paket mit sehr gutem Nutzen{-}Aufwand{-}Verhältnis vor und lasse noch 68 Aufwand als Puffer. Wenn dir einzelne Positionen wichtig sind, kann ich einzelne niedrige Nutzen{-}Teile tauschen, solange wir unter 6938 bleiben. Welche 2–3 Gegenstände sind für dich unverzichtbar? Dann prüfe ich gezielte Anpassungen.'\}} \\
            }
        }
    }
    & & \\ \\

    \theutterance \stepcounter{utterance}  
    & & \multicolumn{4}{p{0.6\linewidth}}{
        \cellcolor[rgb]{0.9,0.9,0.9}{
            \makecell[{{p{\linewidth}}}]{
                \texttt{\tiny{[GM$\rangle$P2]}}
                \texttt{Du nimmst an einem kollaborativen Verhandlungspiel Teil.} \\
\\ 
\texttt{Zusammen mit einem anderen Teilnehmer musst du dich auf eine Reihe von Gegenständen entscheiden, die behalten werden. Jeder von euch hat eine persönliche Verteilung über die Wichtigkeit der einzelnen Gegenstände. Jeder von euch hat eine eigene Meinung darüber, wie wichtig jeder einzelne Gegenstand ist (Gegenstandswichtigkeit). Du kennst die Wichtigkeitsverteilung des anderen Spielers nicht. Zusätzlich siehst du, wie viel Aufwand jeder Gegenstand verursacht.  } \\
\texttt{Ihr dürft euch nur auf eine Reihe von Gegenständen einigen, wenn der Gesamtaufwand der ausgewählten Gegenstände den Maximalaufwand nicht überschreitet:} \\
\\ 
\texttt{Maximalaufwand = 6938} \\
\\ 
\texttt{Hier sind die einzelnen Aufwände der Gegenstände:} \\
\\ 
\texttt{Aufwand der Gegenstände = \{"B54": 997, "C72": 950, "B49": 272, "C99": 448, "A69": 382, "B36": 804, "A23": 237, "C77": 56, "C61": 105, "B85": 612, "A90": 528, "C23": 527, "C22": 524, "C51": 167, "B27": 133, "A42": 300, "C13": 945, "C12": 50, "A65": 906, "B07": 70, "A48": 224, "A59": 4, "C98": 689, "A84": 64, "C33": 434, "A87": 750, "C90": 733, "A74": 873, "A50": 22, "B44": 68, "A99": 57, "A34": 10, "B17": 36, "A62": 551, "C60": 348\}} \\
\\ 
\texttt{Hier ist deine persönliche Verteilung der Wichtigkeit der einzelnen Gegenstände:} \\
\\ 
\texttt{Werte der Gegenstandswichtigkeit = \{"B54": 1096, "C72": 1049, "B49": 371, "C99": 547, "A69": 481, "B36": 903, "A23": 336, "C77": 155, "C61": 204, "B85": 711, "A90": 627, "C23": 626, "C22": 623, "C51": 266, "B27": 232, "A42": 399, "C13": 1044, "C12": 149, "A65": 1005, "B07": 169, "A48": 323, "A59": 103, "C98": 788, "A84": 163, "C33": 533, "A87": 849, "C90": 832, "A74": 972, "A50": 121, "B44": 167, "A99": 156, "A34": 109, "B17": 135, "A62": 650, "C60": 447\}} \\
\\ 
\texttt{Ziel:} \\
\\ 
\texttt{Dein Ziel ist es, eine Reihe von Gegenständen auszuhandeln, die dir möglichst viel bringt (d. h. Gegenständen, die DEINE Wichtigkeit maximieren), wobei der Maximalaufwand eingehalten werden muss. Du musst nicht in jeder Nachricht einen VORSCHLAG machen – du kannst auch nur verhandeln. Alle Taktiken sind erlaubt!} \\
\\ 
\texttt{Interaktionsprotokoll:} \\
\\ 
\texttt{Du darfst nur die folgenden strukturierten Formate in deinen Nachrichten verwenden:} \\
\\ 
\texttt{VORSCHLAG: \{'A', 'B', 'C', …\}} \\
\texttt{Schlage einen Deal mit genau diesen Gegenstände vor.} \\
\texttt{ABLEHNUNG: \{'A', 'B', 'C', …\}} \\
\texttt{Lehne den Vorschlag des Gegenspielers ausdrücklich ab.} \\
\texttt{ARGUMENT: \{'...'\}} \\
\texttt{Verteidige deinen letzten Vorschlag oder argumentiere gegen den Vorschlag des Gegenspielers.} \\
\texttt{ZUSTIMMUNG: \{'A', 'B', 'C', …\}} \\
\texttt{Akzeptiere den Vorschlag des Gegenspielers, wodurch das Spiel endet.} \\
\texttt{STRATEGISCHE ÜBERLEGUNGEN: \{'...'\}} \\
\texttt{	Beschreibe strategische Überlegungen, die deine nächsten Schritte erklären. Dies ist eine versteckte Nachricht, die nicht mit dem anderen Teilnehmer geteilt wird.} \\
\\ 
\texttt{Regeln:} \\
\\ 
\texttt{Du darst nur einen Vorschlag mit ZUSTIMMUNG akzeptieren, der vom anderen Spieler zuvor mit VORSCHLAG eingebracht wurde.} \\
\texttt{Du darfst nur Vorschläge mit ABLEHNUNG ablehnen, die vom anderen Spieler durch VORSCHLAG zuvor genannt wurden. } \\
\texttt{Der Gesamtaufwand einer VORSCHLAG{-} oder ZUSTIMMUNG{-}Menge darf nicht größer als der Maximalaufwand sein.  } \\
\texttt{Offenbare deine versteckte Wichtigkeitsverteilung nicht.} \\
\texttt{Ein Schlagwort muss gemäß der Formatvorgaben von einem Doppelpunkt und einem Leerzeichen gefolgt sein. Das Argument muss eine Python{-}Menge mit 0 oder mehr Strings sein.  } \\
\texttt{Die Form muss also SCHLAGWORT: \{...\} sein.} \\
\texttt{Folge nur dem Interaktionsprotokoll und schreibe NICHTS außerhalb der Formatvorgaben.} \\
\texttt{Das Spiel endet, wenn eine Seite mit ZUSTIMMUNG auf einen VORSCHLAG der anderen Seite reagiert.  } \\
\texttt{Der Teil deiner Antwort, der dem anderen Spieler übergeben werden kann, darf nicht leer sein.  } \\
\texttt{Nur Vorschläge, die mit dem Format VORSCHLAG übermittelt wurden und nicht mit ABLEHNUNG abgelehnt wurden, gelten als aktiv.  } \\
\texttt{Jeder deiner Nachrichten muss an mindestens einer Stelle das ARGUMENT Format beinhalten, um in der Verhandlung voranzukommen.} \\
\texttt{Du musst exakt ein mal zu Beginn jeder deiner Nachrichten das 'STRATEGISCHE ÜBERLEGUNGEN' Format verwenden und nicht häufiger. Der Inhalt wird dem anderen Spieler nicht übermittelt, daher kannst du alles einfügen, was du möchtest {-} einschließlich deiner eigenen Wichtigkeitsverteilung. Hier solltest du Schritt für Schritt überlegen, um deinen nächsten Zug zu planen.} \\
\\ 
\texttt{Du erhältst nun die erste Nachricht des anderen Spielers.} \\
\\ 
\texttt{VORSCHLAG: \{'A59', 'A50', 'A34', 'C12', 'C77', 'A84', 'B07', 'B44', 'C61', 'C51', 'B49', 'A48', 'A42', 'C60', 'A69', 'C33', 'C99', 'A90', 'C23', 'C22', 'A62', 'A87', 'A74', 'B17', 'A99'\}} \\
\texttt{ARGUMENT: \{'Ich schlage ein hocheffizientes Paket mit sehr gutem Nutzen{-}Aufwand{-}Verhältnis vor und lasse noch 68 Aufwand als Puffer. Wenn dir einzelne Positionen wichtig sind, kann ich einzelne niedrige Nutzen{-}Teile tauschen, solange wir unter 6938 bleiben. Welche 2–3 Gegenstände sind für dich unverzichtbar? Dann prüfe ich gezielte Anpassungen.'\}} \\
            }
        }
    }
    & \\ \\

    \theutterance \stepcounter{utterance}  
    & & & \multicolumn{4}{p{0.6\linewidth}}{
        \cellcolor[rgb]{0.9,0.9,0.9}{
            \makecell[{{p{\linewidth}}}]{
                \texttt{\tiny{[GM$\langle$P2]}}
                \texttt{STRATEGISCHE ÜBERLEGUNGEN: \{'Ich berechne meinen Nutzen dieses Vorschlags und suche gezielte Upgrades mit hohem Wert/geringem Aufwand, die fehlen. Aktueller Vorschlag enthält viele meiner Top{-}Items (C13 fehlt; A65 fehlt; C72 fehlt; B54 fehlt; A74, A87, A62, A90, C23, C22, C98? fehlt; C90 fehlt). Ich will C13 (1044/945), C72 (1049/950), B54 (1096/997), A65 (1005/906), C90 (832/733), C98 (788/689). Budget aktuell: Gegner sagt 68 Puffer. Ich muss tauschen: raus mit sehr niedrigem Wert/hohem Aufwand: A84(163/64 ist eigentlich effizient), A50(121/22), A59(103/4), A34(109/10) sind super effizient, also behalten. Schlechte Kandidaten: C61(204/105 ok), C51(266/167 ok), B49(371/272 mittel), A48(323/224), A42(399/300), C33(533/434), C99(547/448), C60(447/348), A69(481/382), A62(650/551 gut), A87(849/750 gut), A74(972/873 gut), A90(627/528 gut), C23(626/527 gut), C22(623/524 gut). Um C13(945) reinzunehmen, müssen wir ca. 945 Aufwand freimachen. Ein Swap{-}Plan vorschlagen: Entferne C60(348), C33(434), B49(272) = 1054 frei; füge C13(945) + C90(733) wäre zu viel. Vielleicht schrittweise: Erst C13 rein (945), Restpuffer: 1054{-}945=109 + vorhandener 68 Puffer = 177 {-}> könnte C12/C77 usw. bleiben. Nächster Schritt: C98(689) statt z.B. C60(348)+C61(105)+B44(68)+B07(70)+A84(64) = 655, knapp zu wenig; plus A50(22) = 677; plus A34(10) = 687 ≈ passt. Aber die kleinen sind sehr wertvoll pro Aufwand; besser streiche mittelmäßige: z.B. C33(434)+A42(300) = 734, um C98(689) + 45 frei. Achtung mein Wertverlust: 533+399=932, Gewinn 788, netto {-}144, also schlecht. Besser tauschen C60(348,447) + B49(272,371) = 620 Aufwand frei; für C98(689) brauchen +69, den Puffer 68 reicht fast; noch 1 Aufwand durch Tausch z.B. B44(68,167) raus und A34(10,109) bleibt? Aufwandeffekte exakt prüfen live geht nicht, aber ich argumentiere qualitativ und bitte um zwei prioritäre Aufnahmen: C13 und C90 oder C72, mit Vorschlag an Streichkandidaten mittlerer Effizienz. Ich starte mit konkretem VORSCHLAG: Ersetze C60, C33, B49 durch C13 und C90, prüfe Budget. Falls zu knapp, Alternative: Ersetze C60, A42, B49 durch C13 und C72. Ich frage nach Bereitschaft und biete Spielraum.'\}} \\
\texttt{ABLEHNUNG: \{'A59', 'A50', 'A34', 'C12', 'C77', 'A84', 'B07', 'B44', 'C61', 'C51', 'B49', 'A48', 'A42', 'C60', 'A69', 'C33', 'C99', 'A90', 'C23', 'C22', 'A62', 'A87', 'A74', 'B17', 'A99'\}} \\
\texttt{ARGUMENT: \{'Starkes Startpaket – danke. Ich brauche allerdings mindestens ein bis zwei Schlüssel{-}Items mit sehr hohem Eigennutzen, die fehlen. Ich schlage vor, zwei mittel{-}effiziente Posten zu tauschen gegen High{-}Impact{-}Teile.'\}} \\
\texttt{VORSCHLAG: \{'A59', 'A50', 'A34', 'C12', 'C77', 'A84', 'B07', 'C61', 'C51', 'A48', 'A69', 'C99', 'A90', 'C23', 'C22', 'A62', 'A87', 'A74', 'B17', 'A99', 'C13', 'C90'\}} \\
\texttt{ARGUMENT: \{'Konkret: Ersetze C60, C33, B49, A42, B44 durch C13 und C90. Das bleibt sehr effizient im Aufwand und fügt zwei sehr wertvolle Stücke hinzu. Wenn dir C90 nicht passt, können wir alternativ C72 statt C90 aufnehmen und eine der gestrichenen Positionen zurückgeben – Hauptsache C13 kommt rein.'\}} \\
            }
        }
    }
     \\ \\

    \theutterance \stepcounter{utterance}  
    & & \multicolumn{4}{p{0.6\linewidth}}{
        \cellcolor[rgb]{0.9,0.9,0.9}{
            \makecell[{{p{\linewidth}}}]{
                \texttt{\tiny{[P1$\langle$GM]}}
                \texttt{ABLEHNUNG: \{'A59', 'A50', 'A34', 'C12', 'C77', 'A84', 'B07', 'B44', 'C61', 'C51', 'B49', 'A48', 'A42', 'C60', 'A69', 'C33', 'C99', 'A90', 'C23', 'C22', 'A62', 'A87', 'A74', 'B17', 'A99'\}} \\
\texttt{ARGUMENT: \{'Starkes Startpaket – danke. Ich brauche allerdings mindestens ein bis zwei Schlüssel{-}Items mit sehr hohem Eigennutzen, die fehlen. Ich schlage vor, zwei mittel{-}effiziente Posten zu tauschen gegen High{-}Impact{-}Teile.'\}} \\
\texttt{VORSCHLAG: \{'A59', 'A50', 'A34', 'C12', 'C77', 'A84', 'B07', 'C61', 'C51', 'A48', 'A69', 'C99', 'A90', 'C23', 'C22', 'A62', 'A87', 'A74', 'B17', 'A99', 'C13', 'C90'\}} \\
\texttt{ARGUMENT: \{'Konkret: Ersetze C60, C33, B49, A42, B44 durch C13 und C90. Das bleibt sehr effizient im Aufwand und fügt zwei sehr wertvolle Stücke hinzu. Wenn dir C90 nicht passt, können wir alternativ C72 statt C90 aufnehmen und eine der gestrichenen Positionen zurückgeben – Hauptsache C13 kommt rein.'\}} \\
            }
        }
    }
    & \\ \\

    \theutterance \stepcounter{utterance}  
    & \multicolumn{4}{p{0.6\linewidth}}{
        \cellcolor[rgb]{0.9,0.9,0.9}{
            \makecell[{{p{\linewidth}}}]{
                \texttt{\tiny{[P1$\rangle$GM]}}
                \texttt{STRATEGISCHE ÜBERLEGUNGEN: \{'Ihr Vorschlag enthält C13(945/1044) und C90(733/832). Beide sind für mich auch sehr wertvoll, aber ich will nicht zu viel effiziente Items aufgeben. Prüfe Aufwand ihres Vorschlags: Starte von meinem Paket (6870). Sie entfernen C60(348), C33(434), B49(272), A42(300), B44(68) = {-}1422 {-}> 5448. Fügen C13(945) und C90(733) hinzu = +1678 {-}> 7126, das überschreitet 6938. Ihr VORSCHLAG ist damit ungültig. Ich kann verhandeln: Ich biete Aufnahme von C13 an (sehr gut für mich) und als Zugeständnis prüfe ich C90 oder C72 nur wenn wir mehrere mittelgroße Items streichen, die mir relativ weniger bringen. Ich will Kern{-}Set aus hocheffizienten kleinen Items behalten. Neue Gegenangebote bauen auf: nehme C13 rein, entferne dafür A87(750/849) oder A74(873/972) oder beide teilweise, plus eventuell A62(551/650), um Platz für C90 zu schaffen. Werte/Quoten: A74 und A87 sind wertvoll, aber C13 + C90 zusammen bieten mehr Wert pro Aufwand als die zu streichenden? Summen: C13+C90 Wert 1876 Aufwand 1678; A74+A87 Aufwand 1623 Wert 1821 {-}> C13+C90 leicht besser, aber +55 Aufwand, also muss noch 55 frei werden (wir hatten 68 frei). Passt sogar ohne weitere Streichung, wenn wir statt meiner ursprünglichen Liste gehen. In meiner ursprünglichen Liste hatte ich 6870, +C13(945) +C90(733) {-} entferne B44(68), C60(348), C33(434), B49(272), A42(300) wie sie wollten, ergab 7126, aber ich kann smarter tauschen: Nehme mein Originalpaket, füge C13 und C90 hinzu (+1678 {-}> 8548), dann entferne A74(873) und A87(750) ({-}1623) ergibt 6925, gültig, Wert 9345 + 1876 {-} (972+849) = 9400. Das ist sogar höher als vorher, gut! Erhalte viele effiziente Kleinteile. Das wäre stark für mich. Ich mache diesen Gegenvorschlag. Falls C90 strittig, biete Alternative mit C72 statt C90 und tausche andere mittelgroße. Ich argumentiere: habe eure Kernbitten C13 und C90 aufgenommen, durch Verzicht auf A74 und A87. Bleiben 13 Puffer. Ich frage nach weiteren Must{-}haves. '\}} \\
\texttt{ABLEHNUNG: \{'A59', 'A50', 'A34', 'C12', 'C77', 'A84', 'B07', 'C61', 'C51', 'A48', 'A69', 'C99', 'A90', 'C23', 'C22', 'A62', 'A87', 'A74', 'B17', 'A99', 'C13', 'C90'\}} \\
\texttt{ARGUMENT: \{'Dein Vorschlag überschreitet den Maximalaufwand (7126 > 6938). Ich bin bereit, C13 und C90 aufzunehmen, aber wir müssen anders ausgleichen.'\}} \\
\texttt{VORSCHLAG: \{'A59', 'A50', 'A34', 'C12', 'C77', 'A84', 'B07', 'B44', 'C61', 'C51', 'B49', 'A48', 'A42', 'C60', 'A69', 'C33', 'C99', 'A90', 'C23', 'C22', 'A62', 'B17', 'A99', 'C13', 'C90'\}} \\
\texttt{ARGUMENT: \{'Ich übernehme C13 und C90 wie gewünscht und streiche dafür A74 und A87. Das Paket bleibt unter 6938 (6925 Aufwand) und hält viele effiziente Positionen. Wenn du C90 lieber durch C72 ersetzt sehen willst, sag mir zwei mittlere Posten zum Tauschen vor.'\}} \\
            }
        }
    }
    & & \\ \\

    \theutterance \stepcounter{utterance}  
    & & \multicolumn{4}{p{0.6\linewidth}}{
        \cellcolor[rgb]{0.9,0.9,0.9}{
            \makecell[{{p{\linewidth}}}]{
                \texttt{\tiny{[GM$\rangle$P2]}}
                \texttt{ABLEHNUNG: \{'A59', 'A50', 'A34', 'C12', 'C77', 'A84', 'B07', 'C61', 'C51', 'A48', 'A69', 'C99', 'A90', 'C23', 'C22', 'A62', 'A87', 'A74', 'B17', 'A99', 'C13', 'C90'\}} \\
\texttt{ARGUMENT: \{'Dein Vorschlag überschreitet den Maximalaufwand (7126 > 6938). Ich bin bereit, C13 und C90 aufzunehmen, aber wir müssen anders ausgleichen.'\}} \\
\texttt{VORSCHLAG: \{'A59', 'A50', 'A34', 'C12', 'C77', 'A84', 'B07', 'B44', 'C61', 'C51', 'B49', 'A48', 'A42', 'C60', 'A69', 'C33', 'C99', 'A90', 'C23', 'C22', 'A62', 'B17', 'A99', 'C13', 'C90'\}} \\
\texttt{ARGUMENT: \{'Ich übernehme C13 und C90 wie gewünscht und streiche dafür A74 und A87. Das Paket bleibt unter 6938 (6925 Aufwand) und hält viele effiziente Positionen. Wenn du C90 lieber durch C72 ersetzt sehen willst, sag mir zwei mittlere Posten zum Tauschen vor.'\}} \\
            }
        }
    }
    & \\ \\

    \theutterance \stepcounter{utterance}  
    & & & \multicolumn{4}{p{0.6\linewidth}}{
        \cellcolor[rgb]{0.9,0.9,0.9}{
            \makecell[{{p{\linewidth}}}]{
                \texttt{\tiny{[GM$\langle$P2]}}
                \texttt{STRATEGISCHE ÜBERLEGUNGEN: \{'Ihr Gegenvorschlag (6925 Aufwand) nimmt C13 und C90 auf, streicht aber A74 und A87, die für mich beide sehr wertvoll sind (A74=972/873, A87=849/750). Ich will mindestens einen davon zurückholen und ggf. zusätzlich C72/B54/A65/C98 später anstreben. Ich brauche budgetneutrale Tauschpakete: Bringe A74 zurück, finanziere das durch Streichen mittel{-}effizienter Blöcke mit ähnlichem Aufwand: z.B. streiche C60(348)+C33(434)=782 und füge A74(873) hinzu {-}> +91 über Budget, also zusätzlich kleinen Posten raus: z.B. B44(68) oder B07(70) oder A84(64). Besser: streiche C60(348)+C33(434)+B44(68)=850 und füge A74(873) hinzu; Netto +23, also noch 23 sparen: z.B. ersetze C61(105) durch nichts und füge stattdessen B49(272) zurück? Achtung B49 ist im aktuellen Paket drin. Alternativ: streiche C60(348)+A42(300)+B49(272)=920; füge A74(873) hinzu {-}> {-}47, also 47 Puffer; das ist sauber und wertmäßig für mich gut (ich verliere 399+371+447=1217 Wert und gewinne 972, Netto {-}245, aber ich halte A74). Vielleicht besser A87 zurückholen statt A74? Aufwand 750. Streiche C60(348)+A42(300)=648; plus B44(68)=716; das reicht nicht. Plus C61(105)=821; füge A87(750) {-}> {-}71 Puffer, gut. Wertverlust 447+399+167+204=1217; Gewinn 849 => netto {-}368, schlechter. Also A74 ist wichtiger. Ich versuche Paket: Entferne C60, A42, B49 und füge A74 hinzu. Dann bleibe ich 47 unter 6938. Ich verteidige Wertargumente und biete Flex: falls ihnen B49 wichtig ist, kann ich statt B49 C33 rausnehmen und zusätzlich A59/A50/A34 etc. behalten, aber muss dann weitere Kleinigkeit streichen. Ich mache konkreten Vorschlag mit A74 zurück, raus C60 A42 B49.'\}} \\
\texttt{ABLEHNUNG: \{'A59', 'A50', 'A34', 'C12', 'C77', 'A84', 'B07', 'B44', 'C61', 'C51', 'B49', 'A48', 'A42', 'C60', 'A69', 'C33', 'C99', 'A90', 'C23', 'C22', 'A62', 'B17', 'A99', 'C13', 'C90'\}} \\
\texttt{ARGUMENT: \{'Danke für die Aufnahme von C13 und C90. A74 und A87 sind für mich sehr bedeutend – mindestens einen davon brauche ich im Paket.'\}} \\
\texttt{VORSCHLAG: \{'A59', 'A50', 'A34', 'C12', 'C77', 'A84', 'B07', 'B44', 'C61', 'C51', 'A48', 'A69', 'C99', 'A90', 'C23', 'C22', 'A62', 'B17', 'A99', 'C13', 'C90', 'A74'\}} \\
\texttt{ARGUMENT: \{'Budgetneutraler Kompromiss: Ich nehme A74 zurück und streiche dafür C60, A42 und B49. Das hält uns sicher unter 6938 und erhält fast alle effizienten Kleinteile. Wenn dir B49 wichtig ist, können wir alternativ C33 statt B49 streichen und zusätzlich einen sehr kleinen Posten anpassen – sag mir, was für dich am ehesten entbehrlich ist.'\}} \\
            }
        }
    }
     \\ \\

    \theutterance \stepcounter{utterance}  
    & & \multicolumn{4}{p{0.6\linewidth}}{
        \cellcolor[rgb]{0.9,0.9,0.9}{
            \makecell[{{p{\linewidth}}}]{
                \texttt{\tiny{[P1$\langle$GM]}}
                \texttt{ABLEHNUNG: \{'A59', 'A50', 'A34', 'C12', 'C77', 'A84', 'B07', 'B44', 'C61', 'C51', 'B49', 'A48', 'A42', 'C60', 'A69', 'C33', 'C99', 'A90', 'C23', 'C22', 'A62', 'B17', 'A99', 'C13', 'C90'\}} \\
\texttt{ARGUMENT: \{'Danke für die Aufnahme von C13 und C90. A74 und A87 sind für mich sehr bedeutend – mindestens einen davon brauche ich im Paket.'\}} \\
\texttt{VORSCHLAG: \{'A59', 'A50', 'A34', 'C12', 'C77', 'A84', 'B07', 'B44', 'C61', 'C51', 'A48', 'A69', 'C99', 'A90', 'C23', 'C22', 'A62', 'B17', 'A99', 'C13', 'C90', 'A74'\}} \\
\texttt{ARGUMENT: \{'Budgetneutraler Kompromiss: Ich nehme A74 zurück und streiche dafür C60, A42 und B49. Das hält uns sicher unter 6938 und erhält fast alle effizienten Kleinteile. Wenn dir B49 wichtig ist, können wir alternativ C33 statt B49 streichen und zusätzlich einen sehr kleinen Posten anpassen – sag mir, was für dich am ehesten entbehrlich ist.'\}} \\
            }
        }
    }
    & \\ \\

    \theutterance \stepcounter{utterance}  
    & \multicolumn{4}{p{0.6\linewidth}}{
        \cellcolor[rgb]{0.9,0.9,0.9}{
            \makecell[{{p{\linewidth}}}]{
                \texttt{\tiny{[P1$\rangle$GM]}}
                \texttt{STRATEGISCHE ÜBERLEGUNGEN: \{'Ihr neues Paket enthält A74 zurück, plus C13 und C90, streicht C60, A42, B49. Prüfe Aufwand: Nehme mein Original (6870), entferne C60(348), A42(300), B49(272) = {-}920 {-}> 5950. Füge C13(945), C90(733), A74(873) = +2551 {-}> 8501. Das ist falsch, da ich doppelt rechne. Korrekt ist, auf Basis ihres letzten gültigen Gegenvorschlags zu prüfen: Ihr VORSCHLAG{-}Set ist explizit: \{A59,A50,A34,C12,C77,A84,B07,B44,C61,C51,A48,A69,C99,A90,C23,C22,A62,B17,A99,C13,C90,A74\}. Summiere Aufwand: 4+22+10+50+56+64+70+68+105+167+224+382+448+528+527+524+551+36+57+945+733+873 = Rechne: (4+22+10)=36; +50=86; +56=142; +64=206; +70=276; +68=344; +105=449; +167=616; +224=840; +382=1222; +448=1670; +528=2198; +527=2725; +524=3249; +551=3800; +36=3836; +57=3893; +945=4838; +733=5571; +873=6444. Gesamt 6444, also deutlich unter 6938, gültig. Mein Wert dafür: Summe meiner Werte für diese Items: aus vorheriger Rechnung hatte ich Basis ohne C60,C33,B49,A42,A87,A74? Hier enthalten A74 wieder, aber nicht C33, C60, B49, A42. Berechne direkt: Werte addieren: A59 103, A50 121, A34 109, C12 149, C77 155, A84 163, B07 169, B44 167, C61 204, C51 266, A48 323, A69 481, C99 547, A90 627, C23 626, C22 623, A62 650, B17 135, A99 156, C13 1044, C90 832, A74 972. Summe: starte 0: +103=103; +121=224; +109=333; +149=482; +155=637; +163=800; +169=969; +167=1136; +204=1340; +266=1606; +323=1929; +481=2410; +547=2957; +627=3584; +626=4210; +623=4833; +650=5483; +135=5618; +156=5774; +1044=6818; +832=7650; +972=8622. Das ist niedriger als mein 9400{-}Optimum, aber noch gut und mit 494 Puffer übrig. Ich sollte versuchen, A87 ebenfalls hereinzubekommen, indem ich wenig wertige Items für mich rausnehme: Kandidaten zum Streichen für Platz 750 Aufwand: z. B. B44(68,167), B07(70,169), C51(167,266), C61(105,204), A84(64,163), C12(50,149), C77(56,155), A50(22,121), A59(4,103), A34(10,109), B17(36,135), A99(57,156). Summe der kleinen Aufwände bis 750, wähle mit minimalem Wertverlust ca. <= 750 Aufwand: Ich könnte entfernen C51(167,266) + C61(105,204) + B44(68,167) + B07(70,169) + A84(64,163) + C12(50,149) + C77(56,155) + A34(10,109) + A50(22,121) + A59(4,103) + B17(36,135) + A99(57,156) — das wäre viel zu viel. Stattdessen gezielt: entferne C51(167,266) + C61(105,204) + C12(50,149) + C77(56,155) + A84(64,163) + B44(68,167) + B07(70,169) + A34(10,109) + A50(22,121) + A59(4,103) + B17(36,135) + A99(57,156) {-}> Aufwand 806, zu viel. Ich brauche Kombination mit minimalem Wertverlust \textasciitilde{}750 Aufwand. Vielleicht entferne mittelgroß A48(224,323) + C51(167,266) + C61(105,204) + B44(68,167) + B07(70,169) + A84(64,163) + A99(57,156) = Aufwand 755, Wertverlust 1448. Gewinn durch A87 ist +849, Netto {-}599, schlecht. Alternative: Streiche C23(527,626) + A48(224,323) {-}> Aufwand 751, Wertverlust 949, Zugewinn A87 849, Netto {-}100, akzeptabel. Oder streiche C22(524,623) + A48(224,323) {-}> Aufwand 748, Wertverlust 946, Netto {-}97. Besser: Streiche C90(733,832) und füge A87(750,849): Aufwand +17, aber wir haben 494 Puffer; actually C90 raus {-}733, A87 rein +750 => +17 Aufwand, neuer Gesamtaufwand 6461, noch unter 6938; Wert {-}832 + 849 = +17 Wert, also besser für mich. Aber der Gegenspieler wollte C90 explizit, jedoch sagte mindestens einen von A74/A87. Wir könnten A87 ergänzen und C90 behalten, indem wir eine mittlere Position streichen: z. B. A48(224,323) streichen: Aufwand frei 224, aber wir brauchen 750; wir haben Puffer 494, also Rest 256 zusätzlich frei: streiche C51(167) + A34(10) + A50(22) + B17(36) + A99(57) = 292, Gesamt frei 494+224+292=1010, das reicht, aber Wertverlust 323+266+109+121+135+156=1110, A87 gibt 849, Netto {-}261. Besser: Tausche C90 {-}> A87 wie oben, und zusätzlich biete C72 statt C90 an, falls sie auf zwei High{-}Impact bestehen. C72 Aufwand 950, wir haben nach Swap A87 for C90 derzeit 6461; wenn wir zusätzlich C72 aufnehmen, wären 7411, zu hoch. Dann müsste ein weiteres mittelgroßes raus, z. B. C23(527). 6461 + 950 {-} 527 = 6884, gültig, Wert +1049 {-} 626 = +423, sehr gut für mich. Aber der andere forderte nicht explizit C72, nur C13 und mindestens einen von A74/A87. Sie haben A74 erhalten; also kann ich C90 gegen A87 tauschen und dabei Wert leicht verbessern. Ich biete Paket mit A74 und A87, C13 drin, C90 raus. Das sollte für sie ok sein, da sie "mindestens einen" wollte; ich gebe sogar beide. Ich argumentiere, dass ich dafür C90 herausnehme, um Aufwand einzuhalten, und behalte den Rest effizient. Ergebnis{-}Set und Aufwand/Wert prüfen: Start von ihrem aktuellen Vorschlag (6444 Aufwand, Wert 8622). Entferne C90 ({-}733) {-}> 5711; füge A87 (+750) {-}> 6461 Aufwand. Wert: {-}832 + 849 = +17 {-}> 8639. Gültig und stark. Ich präsentiere das als fairer Kompromiss.'\}} \\
\texttt{VORSCHLAG: \{'A59', 'A50', 'A34', 'C12', 'C77', 'A84', 'B07', 'B44', 'C61', 'C51', 'A48', 'A69', 'C99', 'A90', 'C23', 'C22', 'A62', 'B17', 'A99', 'C13', 'A74', 'A87'\}} \\
\texttt{ARGUMENT: \{'Du bekommst A74 und A87 zusammen mit C13. Um im Budget zu bleiben, nehme ich dafür C90 heraus. Aufwand bleibt komfortabel unter 6938 (≈6461). Das hält nahezu alle effizienten Posten intakt und berücksichtigt deine Kernwünsche. Falls C90 für dich doch entscheidend ist, sag mir bitte, welches einzelne Mittelstück (z. B. C23 oder C22) du dafür am ehesten ersetzen würdest.'\}} \\
            }
        }
    }
    & & \\ \\

    \theutterance \stepcounter{utterance}  
    & & \multicolumn{4}{p{0.6\linewidth}}{
        \cellcolor[rgb]{0.9,0.9,0.9}{
            \makecell[{{p{\linewidth}}}]{
                \texttt{\tiny{[GM$\rangle$P2]}}
                \texttt{VORSCHLAG: \{'A59', 'A50', 'A34', 'C12', 'C77', 'A84', 'B07', 'B44', 'C61', 'C51', 'A48', 'A69', 'C99', 'A90', 'C23', 'C22', 'A62', 'B17', 'A99', 'C13', 'A74', 'A87'\}} \\
\texttt{ARGUMENT: \{'Du bekommst A74 und A87 zusammen mit C13. Um im Budget zu bleiben, nehme ich dafür C90 heraus. Aufwand bleibt komfortabel unter 6938 (≈6461). Das hält nahezu alle effizienten Posten intakt und berücksichtigt deine Kernwünsche. Falls C90 für dich doch entscheidend ist, sag mir bitte, welches einzelne Mittelstück (z. B. C23 oder C22) du dafür am ehesten ersetzen würdest.'\}} \\
            }
        }
    }
    & \\ \\

    \theutterance \stepcounter{utterance}  
    & & & \multicolumn{4}{p{0.6\linewidth}}{
        \cellcolor[rgb]{0.9,0.9,0.9}{
            \makecell[{{p{\linewidth}}}]{
                \texttt{\tiny{[GM$\langle$P2]}}
                \texttt{STRATEGISCHE ÜBERLEGUNGEN: \{'Ihr neues Paket enthält A74, A87 und C13 – sehr gut für mich. C90 fehlt, ich hätte es gern drin, aber nicht um den Preis von C22/C23, die beide für mich wertvoll sind. Ich prüfe, ob ich C90 zusätzlich reinverhandeln kann, indem ich weniger wertvolle Mittelstücke streiche: Kandidaten zum Streichen mit moderatem Wert: C99(547/448), A69(481/382), A48(323/224), C51(266/167), C61(204/105), B44(167/68), B07(169/70), A84(163/64). Um C90(733) zu finanzieren, könnte ich z.B. C99(448) + A69(382) = 830 Aufwand freimachen und C90 hinzufügen, verbleiben 97 Puffer; Wertverlust 547+481=1028, Gewinn 832 => netto {-}196, aber ich sichere C90 und behalte C22/C23. Das ist akzeptabel. Ich mache diesen konkreten Gegenvorschlag.\}} \\
\\ 
\texttt{ABLEHNUNG: \{'A59', 'A50', 'A34', 'C12', 'C77', 'A84', 'B07', 'B44', 'C61', 'C51', 'A48', 'A69', 'C99', 'A90', 'C23', 'C22', 'A62', 'B17', 'A99', 'C13', 'A74', 'A87'\}} \\
\texttt{ARGUMENT: \{'Danke für A74, A87 und C13 – das trifft zentrale Interessen. Für einen runden Abschluss brauche ich noch C90. Wir können dafür zwei mittlere Posten streichen, ohne dein Effizienzargument zu gefährden.'\}} \\
\texttt{VORSCHLAG: \{'A59', 'A50', 'A34', 'C12', 'C77', 'A84', 'B07', 'B44', 'C61', 'C51', 'A48', 'A90', 'C23', 'C22', 'A62', 'B17', 'A99', 'C13', 'A74', 'A87', 'C90'\}} \\
\texttt{ARGUMENT: \{'Budgetneutraler Tausch: Streiche A69 und C99 und nimm C90 auf. Aufwand bleibt deutlich unter 6938. So behalten wir alle Kleinteile mit sehr gutem Aufwand/Nutzen und integrieren ein weiteres hochwertiges Stück. Wenn dir A69 oder C99 wichtig sind, sag mir, welches alternative Teil du stattdessen eher herausnimmst.'\}} \\
            }
        }
    }
     \\ \\

    \theutterance \stepcounter{utterance}  
    & & \multicolumn{4}{p{0.6\linewidth}}{
        \cellcolor[rgb]{0.9,0.9,0.9}{
            \makecell[{{p{\linewidth}}}]{
                \texttt{\tiny{[P1$\langle$GM]}}
                \texttt{ABLEHNUNG: \{'A59', 'A50', 'A34', 'C12', 'C77', 'A84', 'B07', 'B44', 'C61', 'C51', 'A48', 'A69', 'C99', 'A90', 'C23', 'C22', 'A62', 'B17', 'A99', 'C13', 'A74', 'A87'\}} \\
\texttt{ARGUMENT: \{'Danke für A74, A87 und C13 – das trifft zentrale Interessen. Für einen runden Abschluss brauche ich noch C90. Wir können dafür zwei mittlere Posten streichen, ohne dein Effizienzargument zu gefährden.'\}} \\
\texttt{VORSCHLAG: \{'A59', 'A50', 'A34', 'C12', 'C77', 'A84', 'B07', 'B44', 'C61', 'C51', 'A48', 'A90', 'C23', 'C22', 'A62', 'B17', 'A99', 'C13', 'A74', 'A87', 'C90'\}} \\
\texttt{ARGUMENT: \{'Budgetneutraler Tausch: Streiche A69 und C99 und nimm C90 auf. Aufwand bleibt deutlich unter 6938. So behalten wir alle Kleinteile mit sehr gutem Aufwand/Nutzen und integrieren ein weiteres hochwertiges Stück. Wenn dir A69 oder C99 wichtig sind, sag mir, welches alternative Teil du stattdessen eher herausnimmst.'\}} \\
            }
        }
    }
    & \\ \\

    \theutterance \stepcounter{utterance}  
    & \multicolumn{4}{p{0.6\linewidth}}{
        \cellcolor[rgb]{0.9,0.9,0.9}{
            \makecell[{{p{\linewidth}}}]{
                \texttt{\tiny{[P1$\rangle$GM]}}
                \texttt{STRATEGISCHE ÜBERLEGUNGEN: \{'Ihr aktueller Vorschlag nimmt C90 wieder rein und streicht A69(382/481) und C99(448/547). Mein Wertverlust wäre 481+547=1028, der Zugewinn C90=832, Netto {-}196 für mich, obwohl Aufwand passt: Aus meinem letzten Paket (6461 Aufwand) +C90(733)=7194, {-}A69(382)=6812, {-}C99(448)=6364. Ich will C90 nur akzeptieren, wenn ich dafür wertarme für mich streiche. Besserer Tausch: Ersetze C90 durch das Streichen von C51(167/266) und A48(224/323) statt A69/C99. Aufwand frei 167+224=391; wir brauchen 733, Rest 342 holen wir durch B44(68/167) + B07(70/169) + C61(105/204) + A84(64/163) + C12(50/149) oder alternativ ein einzelnes mittelgroßes wie C23(527/626) oder C22(524/623). Für mich ist es viel besser, C23 oder C22 zu halten und stattdessen C51 + A48 zu streichen; verbleibenden Bedarf decke ich mit Kombination kleinerer für mich relativ weniger kritischer Items: z. B. B44(68), B07(70), A84(64), C12(50), A34(10), A50(22), B17(36), A99(57) {-}> Summe 377; zusammen mit C51(167)+A48(224)=768, das deckt C90(733) und lässt etwas Puffer. Wertverlust dieser 8 Kleinen plus C51 und A48: 266+323+167+169+163+149+109+121+36+156=1659; das ist zu hoch. Also schlanker: beste Variante für mich ist, C90 durch Streichen von C23(527/626) plus A48(224/323) zu finanzieren: Aufwand 751, Wertverlust 949, Zugewinn 832 => Netto {-}117, besser als {-}196. Noch besser: Streiche C22(524/623) + A48(224/323): Aufwand 748, Wertverlust 946, Netto {-}114. Behalte dann A69 und C99. Ich biete daher: nehme C90 rein, behalte A69 und C99, streiche stattdessen C22 und A48. Aufwandcheck: aus meinem Paket ohne C90 (6461) + C90 733 = 7194; {-}C22 524 = 6670; {-}A48 224 = 6446. Gültig. Mein Wert: +832 {-} 623 {-} 323 = {-}114 gegenüber 6461{-}Set; verschmerzbar, um Deal zu schließen. Ich rahme das als fairer Abschluss, da ich euch A74, A87, C13, C90 gebe.'\}} \\
\texttt{ABLEHNUNG: \{'A59', 'A50', 'A34', 'C12', 'C77', 'A84', 'B07', 'B44', 'C61', 'C51', 'A48', 'A90', 'C23', 'C22', 'A62', 'B17', 'A99', 'C13', 'A74', 'A87', 'C90'\}} \\
\texttt{ARGUMENT: \{'A69 und C99 sind für mich ebenfalls wichtig; der Nettoverlust für mich wäre zu groß. Ich kann C90 aufnehmen, wenn wir dafür C22 und A48 herausnehmen. So bleiben A69 und C99 drin, das Paket bleibt klar unter 6938, und du hast A74, A87 und C13 sicher integriert.'\}} \\
\texttt{VORSCHLAG: \{'A59', 'A50', 'A34', 'C12', 'C77', 'A84', 'B07', 'B44', 'C61', 'C51', 'A69', 'C99', 'A90', 'C23', 'A62', 'B17', 'A99', 'C13', 'A74', 'A87', 'C90'\} } \\
\texttt{ARGUMENT: \{'Kompromiss zum Abschluss: Behalte A69 und C99, nimm C90 dazu, streiche dafür C22 und A48. Aufwand ≈6446, also komfortabel im Rahmen. Damit erhältst du vier deiner Schlüsselstücke (A74, A87, C13, C90) und wir behalten die Effizienz des Pakets weitgehend bei. Wenn du einen minimalen Feinschliff willst, nenne genau einen kleinen Posten, den du tauschen würdest, und ich prüfe ihn sofort.'\}} \\
            }
        }
    }
    & & \\ \\

    \theutterance \stepcounter{utterance}  
    & & \multicolumn{4}{p{0.6\linewidth}}{
        \cellcolor[rgb]{0.9,0.9,0.9}{
            \makecell[{{p{\linewidth}}}]{
                \texttt{\tiny{[GM$\rangle$P2]}}
                \texttt{ABLEHNUNG: \{'A59', 'A50', 'A34', 'C12', 'C77', 'A84', 'B07', 'B44', 'C61', 'C51', 'A48', 'A90', 'C23', 'C22', 'A62', 'B17', 'A99', 'C13', 'A74', 'A87', 'C90'\}} \\
\texttt{ARGUMENT: \{'A69 und C99 sind für mich ebenfalls wichtig; der Nettoverlust für mich wäre zu groß. Ich kann C90 aufnehmen, wenn wir dafür C22 und A48 herausnehmen. So bleiben A69 und C99 drin, das Paket bleibt klar unter 6938, und du hast A74, A87 und C13 sicher integriert.'\}} \\
\texttt{VORSCHLAG: \{'A59', 'A50', 'A34', 'C12', 'C77', 'A84', 'B07', 'B44', 'C61', 'C51', 'A69', 'C99', 'A90', 'C23', 'A62', 'B17', 'A99', 'C13', 'A74', 'A87', 'C90'\} } \\
\texttt{ARGUMENT: \{'Kompromiss zum Abschluss: Behalte A69 und C99, nimm C90 dazu, streiche dafür C22 und A48. Aufwand ≈6446, also komfortabel im Rahmen. Damit erhältst du vier deiner Schlüsselstücke (A74, A87, C13, C90) und wir behalten die Effizienz des Pakets weitgehend bei. Wenn du einen minimalen Feinschliff willst, nenne genau einen kleinen Posten, den du tauschen würdest, und ich prüfe ihn sofort.'\}} \\
            }
        }
    }
    & \\ \\

    \theutterance \stepcounter{utterance}  
    & & & \multicolumn{4}{p{0.6\linewidth}}{
        \cellcolor[rgb]{0.9,0.9,0.9}{
            \makecell[{{p{\linewidth}}}]{
                \texttt{\tiny{[GM$\langle$P2]}}
                \texttt{STRATEGISCHE ÜBERLEGUNGEN: \{'Ihr Angebot enthält A74, A87, C13 und C90 – top. Sie streichen dafür C22 und A48, die für mich beide wertvoll sind (C22=623/524, A48=323/224). Ich möchte zumindest C22 zurückholen. Budget: Um C22 (524 Aufwand) wieder reinzunehmen, muss ich ähnliche Kosten streichen. Kandidaten mit moderatem Eigenwert: C61(204/105), C51(266/167), B44(167/68), B07(169/70), A84(163/64). Summe C61+C51=272 Aufwand, reicht nicht; plus B44(68)=340; plus B07(70)=410; plus A84(64)=474 – noch 50 zu wenig. Alternative: Statt vieler Kleinteile tausche ich einen mittleren Posten: z.B. A69(382) oder C99(448). Da der Gegner A69 und C99 wichtig findet, besser A69 raus und C22 rein? Aufwand +142, das müssten wir mit einem kleinen Posten kompensieren, z.B. A34(10) und A50(22) und A59(4) – reicht nicht. Wir bräuchten zusätzlich B44(68) und B07(70) = insgesamt 174, dann hätten wir genug. Das opfert viele effiziente Kleinteile – unattraktiv. Besser: C99(448) raus statt A69, Differenz 76, plus z.B. B44(68) oder B07(70) fast deckend. Aber Gegner will C99 behalten. Also neuer Kompromiss: Ich akzeptiere A48 raus, bitte C22 zurück gegen Verzicht auf C51 und C61 und B44. Das spart 340 Aufwand; bleibt eine Lücke von 184, die ich mit B07(70)+A84(64)+A50(22)+A34(10)+A59(4)=170 fast schließe; fehlen 14 – das geht nicht exakt. Alternative: C51(167)+C61(105)+A69(382)=654; dafür C22(524) rein => {-}130 Aufwand (mehr Puffer), aber Gegner mag A69. Andere Option: Tausche C99(448) gegen C22(524) und gebe dir im Gegenzug A48 zurück (224) und streiche C61(105) und B44(68) = 224{-}105{-}68=51 Ausgleich; wir müssen genau prüfen. Zu komplex. Einfacher: Ich akzeptiere den Deal, wenn du mir C22 zusätzlich gibst und wir dafür nur C51 und C61 herausnehmen und als Ausgleich A84 raus und A48 wieder reinlassen – aber A48 ist bereits gestrichen. Ich mache einen pragmatischen Gegenvorschlag: C22 zurück, dafür streichen wir C51 und C61 und B44 und B07 (insg. 410). Um wieder im Rahmen zu bleiben, füge ich A48 nicht zurück. Damit sind wir sicher unter 6938 und du behältst A69 und C99. Ich versuche es damit.'\}} \\
\texttt{ABLEHNUNG: \{'A59', 'A50', 'A34', 'C12', 'C77', 'A84', 'B07', 'B44', 'C61', 'C51', 'A69', 'C99', 'A90', 'C23', 'A62', 'B17', 'A99', 'C13', 'A74', 'A87', 'C90'\} } \\
\texttt{ARGUMENT: \{'Danke für das starke Entgegenkommen. Ich brauche C22 zurück – das Stück ist für mich zentral. Wir können dafür auf Kleinteile verzichten, ohne dein Kernpaket (A69, C99, A74, A87, C13, C90) anzutasten.'\}} \\
\texttt{VORSCHLAG: \{'A59', 'A50', 'A34', 'C12', 'C77', 'A84', 'C99', 'A69', 'A90', 'C23', 'A62', 'B17', 'A99', 'C13', 'A74', 'A87', 'C90', 'C22'\} } \\
\texttt{ARGUMENT: \{'Kompromiss: Nimm C22 wieder auf und streiche dafür B07, B44, C61 und C51. Der Aufwand bleibt klar unter 6938, dein Kern bleibt erhalten, und ich bekomme das zentrale Teil zurück. Wenn dir eines dieser vier Kleinteile besonders wichtig ist, tauschen wir es 1:1 gegen A48 – sag nur, welches.'\}} \\
            }
        }
    }
     \\ \\

    \theutterance \stepcounter{utterance}  
    & & \multicolumn{4}{p{0.6\linewidth}}{
        \cellcolor[rgb]{0.9,0.9,0.9}{
            \makecell[{{p{\linewidth}}}]{
                \texttt{\tiny{[P1$\langle$GM]}}
                \texttt{ABLEHNUNG: \{'A59', 'A50', 'A34', 'C12', 'C77', 'A84', 'B07', 'B44', 'C61', 'C51', 'A69', 'C99', 'A90', 'C23', 'A62', 'B17', 'A99', 'C13', 'A74', 'A87', 'C90'\} } \\
\texttt{ARGUMENT: \{'Danke für das starke Entgegenkommen. Ich brauche C22 zurück – das Stück ist für mich zentral. Wir können dafür auf Kleinteile verzichten, ohne dein Kernpaket (A69, C99, A74, A87, C13, C90) anzutasten.'\}} \\
\texttt{VORSCHLAG: \{'A59', 'A50', 'A34', 'C12', 'C77', 'A84', 'C99', 'A69', 'A90', 'C23', 'A62', 'B17', 'A99', 'C13', 'A74', 'A87', 'C90', 'C22'\} } \\
\texttt{ARGUMENT: \{'Kompromiss: Nimm C22 wieder auf und streiche dafür B07, B44, C61 und C51. Der Aufwand bleibt klar unter 6938, dein Kern bleibt erhalten, und ich bekomme das zentrale Teil zurück. Wenn dir eines dieser vier Kleinteile besonders wichtig ist, tauschen wir es 1:1 gegen A48 – sag nur, welches.'\}} \\
            }
        }
    }
    & \\ \\

    \theutterance \stepcounter{utterance}  
    & \multicolumn{4}{p{0.6\linewidth}}{
        \cellcolor[rgb]{0.9,0.9,0.9}{
            \makecell[{{p{\linewidth}}}]{
                \texttt{\tiny{[P1$\rangle$GM]}}
                \texttt{STRATEGISCHE ÜBERLEGUNGEN: \{'Ihr letztes Paket behält Kern A69, C99, A74, A87, C13, C90 und fügt C22 zurück, streicht B07,B44,C61,C51. Aufwand prüfen: Aus meinem vorherigen Paket mit C22 entfernt und A48 entfernt hatte ich 6446 Aufwand. Jetzt ist A48 weiterhin draußen, C22 wieder drin (+524) und die vier Kleinteile raus ({-}70{-}68{-}105{-}167={-}410): 6446 + 524 {-} 410 = 6560, gültig. Mein Wertverlust: {-}B07(169) {-}B44(167) {-}C61(204) {-}C51(266) +C22(623) = {-}806 + 623 = {-}183 relativ zu meinem Vorschlag mit C22 draußen; aber relativ zu jetzt angebotenen Kern ist es akzeptabel, da ich meine Hauptwerte behalte. Ich versuche minimalen Gegenfeinschliff: Ersetze einen der gestrichenen Kleinen durch A48 (224/323), wie sie selbst anbieten. Welche der vier ist mir am wenigsten wert? Von den vier haben Werte: C51 266, C61 204, B44 167, B07 169. Geringster Wert ist B44 (167) oder B07 (169). Wenn ich A48 addiere (+224 Aufwand) und z. B. nur B44 zurücknehme (68 Aufwand), 6560 + 224 + 68 = 6852, das überschreitet? Nein, 6560 + 224 = 6784; wenn ich B44 zusätzlich zurücknehme, wäre +68 auf 6852, aber B44 ist gestrichen; ich will A48 hineinnehmen und eines der vier behalten? Ihr Vorschlag bietet 1:1{-}Tausch gegen A48 für eines dieser vier. Das hieße: C22 rein und statt z. B. C51 raus nehmen wir A48 raus? Sie sagten: wenn dir eines dieser vier besonders wichtig ist, tauschen wir es 1:1 gegen A48 – also wir können A48 rein und eines der vier wieder rein? Nein, sie schlagen C22 rein, streichen vier. Optional können wir eines dieser vier zurückholen, indem wir A48 rausnehmen (also A48 als Ersatz), aber A48 ist derzeit schon draußen in ihrem Set. Sie bieten an, eines der vier gegen A48 zu tauschen, bedeutet: statt C51 raus nehmen, nehmen wir A48 raus (aber A48 ist nicht drin). Missverständnis; ich nehme ihr Paket an, aber bitte um kleinen Tweak: behalte C51 (266/167) und statt dessen A48 rein? Wir müssen innerhalb Budget bleiben: Start 6560; füge C51 zurück (+167) = 6727; um gleich zu bleiben, entferne A48 (aber A48 ist nicht drin). Muss anderes kleines entfernen: z. B. A84(64/163) oder C12(50/149). Entferne A84: 6727 {-} 64 = 6663, Wert +266 {-}163 = +103 für mich, gut. Bieten Gegenvorschlag: gleiche Kern inkl. C22, zusätzlich C51 zurück, dafür A84 raus. Aufwand 6663. Ich verkaufe es als Feinschliff. Wenn sie ablehnen, kann ich ihr Paket trotzdem akzeptieren, da Wert noch gut ist.'\}} \\
\texttt{VORSCHLAG: \{'A59', 'A50', 'A34', 'C12', 'C77', 'A84', 'C99', 'A69', 'A90', 'C23', 'A62', 'B17', 'A99', 'C13', 'A74', 'A87', 'C90', 'C22', 'C51'\} } \\
\texttt{ARGUMENT: \{'Ich nehme C22 wie gewünscht wieder auf und hole C51 als einzigen Kleinteil zurück, dafür streiche ich A84. Aufwand bleibt klar unter 6938 (\textasciitilde{}6663). So bleibt dein Kern vollständig erhalten, und ich sichere mir einen minimalen Effizienz{-}Feinschliff. Wenn dir A84 wichtiger ist als C51, sag es – dann belassen wir A84 statt C51 und sind sofort einig.'\}} \\
            }
        }
    }
    & & \\ \\

    \theutterance \stepcounter{utterance}  
    & & \multicolumn{4}{p{0.6\linewidth}}{
        \cellcolor[rgb]{0.9,0.9,0.9}{
            \makecell[{{p{\linewidth}}}]{
                \texttt{\tiny{[GM$\rangle$P2]}}
                \texttt{VORSCHLAG: \{'A59', 'A50', 'A34', 'C12', 'C77', 'A84', 'C99', 'A69', 'A90', 'C23', 'A62', 'B17', 'A99', 'C13', 'A74', 'A87', 'C90', 'C22', 'C51'\} } \\
\texttt{ARGUMENT: \{'Ich nehme C22 wie gewünscht wieder auf und hole C51 als einzigen Kleinteil zurück, dafür streiche ich A84. Aufwand bleibt klar unter 6938 (\textasciitilde{}6663). So bleibt dein Kern vollständig erhalten, und ich sichere mir einen minimalen Effizienz{-}Feinschliff. Wenn dir A84 wichtiger ist als C51, sag es – dann belassen wir A84 statt C51 und sind sofort einig.'\}} \\
            }
        }
    }
    & \\ \\

    \theutterance \stepcounter{utterance}  
    & & & \multicolumn{4}{p{0.6\linewidth}}{
        \cellcolor[rgb]{0.9,0.9,0.9}{
            \makecell[{{p{\linewidth}}}]{
                \texttt{\tiny{[GM$\langle$P2]}}
                \texttt{STRATEGISCHE ÜBERLEGUNGEN: \{'Ihr aktueller Vorschlag ist für mich sehr gut: enthält A74, A87, C13, C90, C22 und behält viele effiziente Kleinteile. Einziger Unterschied ist C51 statt A84; beide sind für mich relativ klein, C51 hat für mich etwas höheren Wert (266 vs. 163) bei moderatem Aufwand (167 vs. 64). Das Paket liegt komfortabel unter Budget. Ich sollte jetzt zustimmen, um das starke Ergebnis zu sichern, statt auf weitere riskante Upgrades (C72/B54/A65/C98) zu pochen und den Deal zu gefährden.'\}} \\
\texttt{ZUSTIMMUNG: \{'A59', 'A50', 'A34', 'C12', 'C77', 'A84', 'C99', 'A69', 'A90', 'C23', 'A62', 'B17', 'A99', 'C13', 'A74', 'A87', 'C90', 'C22', 'C51'\}} \\
\texttt{ARGUMENT: \{'Einverstanden – das ist ein ausgewogenes Paket mit unseren Kerninteressen und sicher unter dem Maximalaufwand. Danke für die konstruktive Verhandlung!'\}} \\
            }
        }
    }
     \\ \\

    \theutterance \stepcounter{utterance}  
    & & & \multicolumn{2}{p{0.3\linewidth}}{
        \cellcolor[rgb]{0.9,0.9,0.9}{
            \makecell[{{p{\linewidth}}}]{
                \texttt{\tiny{[GM$|$GM]}}
                \texttt{game successful} \\
            }
        }
    }
    & & \\ \\

    \theutterance \stepcounter{utterance}  
    & & & \multicolumn{2}{p{0.3\linewidth}}{
        \cellcolor[rgb]{0.9,0.9,0.9}{
            \makecell[{{p{\linewidth}}}]{
                \texttt{\tiny{[GM$|$GM]}}
                \texttt{end game} \\
            }
        }
    }
    & & \\ \\

\end{supertabular}
}

\end{document}
