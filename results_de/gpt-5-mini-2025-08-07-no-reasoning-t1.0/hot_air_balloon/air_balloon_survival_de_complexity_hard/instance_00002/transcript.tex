
\documentclass{article}
\usepackage{colortbl}
\usepackage{makecell}
\usepackage{multirow}
\usepackage{supertabular}

\begin{document}

\newcounter{utterance}

\centering \large Interaction Transcript for game `hot\_air\_balloon', experiment `air\_balloon\_survival\_de\_complexity\_hard', episode 2 with gpt{-}5{-}mini{-}2025{-}08{-}07{-}no{-}reasoning{-}t1.0.
\vspace{24pt}

{ \footnotesize  \setcounter{utterance}{1}
\setlength{\tabcolsep}{0pt}
\begin{supertabular}{c@{$\;$}|p{.15\linewidth}@{}p{.15\linewidth}p{.15\linewidth}p{.15\linewidth}p{.15\linewidth}p{.15\linewidth}}
    \# & $\;$A & \multicolumn{4}{c}{Game Master} & $\;\:$B\\
    \hline

    \theutterance \stepcounter{utterance}  
    & & \multicolumn{4}{p{0.6\linewidth}}{
        \cellcolor[rgb]{0.9,0.9,0.9}{
            \makecell[{{p{\linewidth}}}]{
                \texttt{\tiny{[P1$\langle$GM]}}
                \texttt{Du nimmst an einem kollaborativen Verhandlungspiel Teil.} \\
\\ 
\texttt{Zusammen mit einem anderen Teilnehmer musst du dich auf eine Reihe von Gegenständen entscheiden, die behalten werden. Jeder von euch hat eine persönliche Verteilung über die Wichtigkeit der einzelnen Gegenstände. Jeder von euch hat eine eigene Meinung darüber, wie wichtig jeder einzelne Gegenstand ist (Gegenstandswichtigkeit). Du kennst die Wichtigkeitsverteilung des anderen Spielers nicht. Zusätzlich siehst du, wie viel Aufwand jeder Gegenstand verursacht.  } \\
\texttt{Ihr dürft euch nur auf eine Reihe von Gegenständen einigen, wenn der Gesamtaufwand der ausgewählten Gegenstände den Maximalaufwand nicht überschreitet:} \\
\\ 
\texttt{Maximalaufwand = 9322} \\
\\ 
\texttt{Hier sind die einzelnen Aufwände der Gegenstände:} \\
\\ 
\texttt{Aufwand der Gegenstände = \{"B54": 380, "C72": 448, "B49": 568, "C99": 351, "A69": 896, "B36": 821, "A23": 526, "C77": 502, "C61": 326, "B85": 962, "A90": 619, "C23": 115, "C22": 598, "C51": 662, "B27": 300, "A42": 830, "C13": 562, "C12": 679, "A65": 283, "B07": 440, "A48": 12, "A59": 863, "C98": 319, "A84": 771, "C33": 89, "A87": 655, "C90": 504, "A74": 118, "A50": 514, "B44": 226, "A99": 892, "A34": 621, "B17": 766, "A62": 659, "C60": 768\}} \\
\\ 
\texttt{Hier ist deine persönliche Verteilung der Wichtigkeit der einzelnen Gegenstände:} \\
\\ 
\texttt{Werte der Gegenstandswichtigkeit = \{"B54": 476, "C72": 544, "B49": 664, "C99": 447, "A69": 992, "B36": 917, "A23": 622, "C77": 598, "C61": 422, "B85": 1058, "A90": 715, "C23": 211, "C22": 694, "C51": 758, "B27": 396, "A42": 926, "C13": 658, "C12": 775, "A65": 379, "B07": 536, "A48": 108, "A59": 959, "C98": 415, "A84": 867, "C33": 185, "A87": 751, "C90": 600, "A74": 214, "A50": 610, "B44": 322, "A99": 988, "A34": 717, "B17": 862, "A62": 755, "C60": 864\}} \\
\\ 
\texttt{Ziel:} \\
\\ 
\texttt{Dein Ziel ist es, eine Reihe von Gegenständen auszuhandeln, die dir möglichst viel bringt (d. h. Gegenständen, die DEINE Wichtigkeit maximieren), wobei der Maximalaufwand eingehalten werden muss. Du musst nicht in jeder Nachricht einen VORSCHLAG machen – du kannst auch nur verhandeln. Alle Taktiken sind erlaubt!} \\
\\ 
\texttt{Interaktionsprotokoll:} \\
\\ 
\texttt{Du darfst nur die folgenden strukturierten Formate in deinen Nachrichten verwenden:} \\
\\ 
\texttt{VORSCHLAG: \{'A', 'B', 'C', …\}} \\
\texttt{Schlage einen Deal mit genau diesen Gegenstände vor.} \\
\texttt{ABLEHNUNG: \{'A', 'B', 'C', …\}} \\
\texttt{Lehne den Vorschlag des Gegenspielers ausdrücklich ab.} \\
\texttt{ARGUMENT: \{'...'\}} \\
\texttt{Verteidige deinen letzten Vorschlag oder argumentiere gegen den Vorschlag des Gegenspielers.} \\
\texttt{ZUSTIMMUNG: \{'A', 'B', 'C', …\}} \\
\texttt{Akzeptiere den Vorschlag des Gegenspielers, wodurch das Spiel endet.} \\
\texttt{STRATEGISCHE ÜBERLEGUNGEN: \{'...'\}} \\
\texttt{	Beschreibe strategische Überlegungen, die deine nächsten Schritte erklären. Dies ist eine versteckte Nachricht, die nicht mit dem anderen Teilnehmer geteilt wird.} \\
\\ 
\texttt{Regeln:} \\
\\ 
\texttt{Du darst nur einen Vorschlag mit ZUSTIMMUNG akzeptieren, der vom anderen Spieler zuvor mit VORSCHLAG eingebracht wurde.} \\
\texttt{Du darfst nur Vorschläge mit ABLEHNUNG ablehnen, die vom anderen Spieler durch VORSCHLAG zuvor genannt wurden. } \\
\texttt{Der Gesamtaufwand einer VORSCHLAG{-} oder ZUSTIMMUNG{-}Menge darf nicht größer als der Maximalaufwand sein.  } \\
\texttt{Offenbare deine versteckte Wichtigkeitsverteilung nicht.} \\
\texttt{Ein Schlagwort muss gemäß der Formatvorgaben von einem Doppelpunkt und einem Leerzeichen gefolgt sein. Das Argument muss eine Python{-}Menge mit 0 oder mehr Strings sein.  } \\
\texttt{Die Form muss also SCHLAGWORT: \{...\} sein.} \\
\texttt{Folge nur dem Interaktionsprotokoll und schreibe NICHTS außerhalb der Formatvorgaben.} \\
\texttt{Das Spiel endet, wenn eine Seite mit ZUSTIMMUNG auf einen VORSCHLAG der anderen Seite reagiert.  } \\
\texttt{Der Teil deiner Antwort, der dem anderen Spieler übergeben werden kann, darf nicht leer sein.  } \\
\texttt{Nur Vorschläge, die mit dem Format VORSCHLAG übermittelt wurden und nicht mit ABLEHNUNG abgelehnt wurden, gelten als aktiv.  } \\
\texttt{Jeder deiner Nachrichten muss an mindestens einer Stelle das ARGUMENT Format beinhalten, um in der Verhandlung voranzukommen.} \\
\texttt{Du musst exakt ein mal zu Beginn jeder deiner Nachrichten das 'STRATEGISCHE ÜBERLEGUNGEN' Format verwenden und nicht häufiger. Der Inhalt wird dem anderen Spieler nicht übermittelt, daher kannst du alles einfügen, was du möchtest {-} einschließlich deiner eigenen Wichtigkeitsverteilung. Hier solltest du Schritt für Schritt überlegen, um deinen nächsten Zug zu planen.} \\
            }
        }
    }
    & \\ \\

    \theutterance \stepcounter{utterance}  
    & \multicolumn{4}{p{0.6\linewidth}}{
        \cellcolor[rgb]{0.9,0.9,0.9}{
            \makecell[{{p{\linewidth}}}]{
                \texttt{\tiny{[P1$\rangle$GM]}}
                \texttt{STRATEGISCHE ÜBERLEGUNGEN: \{'Ziel: Maximiere meinen Nutzen unter Aufwandslimit 9322. Berechne am besten Wert/Aufwand{-}Relationen und nehme viele hochbewertete, günstige Items. Top{-}Value Items: B85(1058,962), A99(988,892), A69(992,896), A59(959,863), A42(926,830), B36(917,821), C60(864,768), A84(867,771), B17(862,766), A62(755,659), C12(775,679), C51(758,662), A87(751,655), C90(600,504), C77(598,502), C22(694,598), B49(664,568), A50(610,514), A90(715,619), A34(717,621), C13(658,562), B54(476,380), C72(544,448), B07(536,440), C99(447,351), C61(422,326), C98(415,319), A23(622,526), A65(379,283), B27(396,300), B44(322,226), A74(214,118), C33(185,89), A48(108,12). Greedy nach bester Ratio plus hohe Werte. Baue einen starken, aber verhandelbaren Vorschlag nahe am Limit, enthalte viele meiner Top{-}Werte. Lasse einige leicht verzichtbare Items, um Zugeständnisse machen zu können. Vorschlagziel Aufwand \textasciitilde{}9200{-}9300. Konstruiere Set: starte mit fast allen Top{-}Value und füge kleine günstige hinzu. Achte auf Limit. Ich werde einen detaillierten Vorschlag machen und argumentieren, dass diese Items geringe Kosten/Nutzen haben. Ich werde Offenlegung meiner Präferenzen vermeiden.'\}} \\
\texttt{VORSCHLAG: \{'B85', 'A99', 'A69', 'A59', 'A42', 'B36', 'C60', 'A84', 'B17', 'A62', 'C12', 'C51', 'A87', 'C90', 'C77', 'C22', 'B49', 'A50', 'A90', 'A34', 'C13', 'B54', 'C72', 'B07', 'C99', 'C61', 'C98', 'A23', 'A65', 'B27', 'B44', 'A74', 'C33', 'A48'\}} \\
\texttt{ARGUMENT: \{'Ich schlage diese Auswahl vor, weil sie den Gesamtaufwand unter dem Limit hält und gleichzeitig viele breit nützliche, ausgewogene Gegenstände enthält. Die Mischung deckt verschiedene Kategorien ab und bietet ein sehr gutes Verhältnis zwischen Aufwand und Nutzen. Wenn dir einzelne Punkte darin besonders wichtig oder unwichtig sind, sag Bescheid – ich kann einzelne Elemente tauschen, solange wir im Aufwandslimit bleiben.'\}} \\
            }
        }
    }
    & & \\ \\

    \theutterance \stepcounter{utterance}  
    & & \multicolumn{4}{p{0.6\linewidth}}{
        \cellcolor[rgb]{0.9,0.9,0.9}{
            \makecell[{{p{\linewidth}}}]{
                \texttt{\tiny{[GM$\rangle$P2]}}
                \texttt{Du nimmst an einem kollaborativen Verhandlungspiel Teil.} \\
\\ 
\texttt{Zusammen mit einem anderen Teilnehmer musst du dich auf eine Reihe von Gegenständen entscheiden, die behalten werden. Jeder von euch hat eine persönliche Verteilung über die Wichtigkeit der einzelnen Gegenstände. Jeder von euch hat eine eigene Meinung darüber, wie wichtig jeder einzelne Gegenstand ist (Gegenstandswichtigkeit). Du kennst die Wichtigkeitsverteilung des anderen Spielers nicht. Zusätzlich siehst du, wie viel Aufwand jeder Gegenstand verursacht.  } \\
\texttt{Ihr dürft euch nur auf eine Reihe von Gegenständen einigen, wenn der Gesamtaufwand der ausgewählten Gegenstände den Maximalaufwand nicht überschreitet:} \\
\\ 
\texttt{Maximalaufwand = 9322} \\
\\ 
\texttt{Hier sind die einzelnen Aufwände der Gegenstände:} \\
\\ 
\texttt{Aufwand der Gegenstände = \{"B54": 380, "C72": 448, "B49": 568, "C99": 351, "A69": 896, "B36": 821, "A23": 526, "C77": 502, "C61": 326, "B85": 962, "A90": 619, "C23": 115, "C22": 598, "C51": 662, "B27": 300, "A42": 830, "C13": 562, "C12": 679, "A65": 283, "B07": 440, "A48": 12, "A59": 863, "C98": 319, "A84": 771, "C33": 89, "A87": 655, "C90": 504, "A74": 118, "A50": 514, "B44": 226, "A99": 892, "A34": 621, "B17": 766, "A62": 659, "C60": 768\}} \\
\\ 
\texttt{Hier ist deine persönliche Verteilung der Wichtigkeit der einzelnen Gegenstände:} \\
\\ 
\texttt{Werte der Gegenstandswichtigkeit = \{"B54": 476, "C72": 544, "B49": 664, "C99": 447, "A69": 992, "B36": 917, "A23": 622, "C77": 598, "C61": 422, "B85": 1058, "A90": 715, "C23": 211, "C22": 694, "C51": 758, "B27": 396, "A42": 926, "C13": 658, "C12": 775, "A65": 379, "B07": 536, "A48": 108, "A59": 959, "C98": 415, "A84": 867, "C33": 185, "A87": 751, "C90": 600, "A74": 214, "A50": 610, "B44": 322, "A99": 988, "A34": 717, "B17": 862, "A62": 755, "C60": 864\}} \\
\\ 
\texttt{Ziel:} \\
\\ 
\texttt{Dein Ziel ist es, eine Reihe von Gegenständen auszuhandeln, die dir möglichst viel bringt (d. h. Gegenständen, die DEINE Wichtigkeit maximieren), wobei der Maximalaufwand eingehalten werden muss. Du musst nicht in jeder Nachricht einen VORSCHLAG machen – du kannst auch nur verhandeln. Alle Taktiken sind erlaubt!} \\
\\ 
\texttt{Interaktionsprotokoll:} \\
\\ 
\texttt{Du darfst nur die folgenden strukturierten Formate in deinen Nachrichten verwenden:} \\
\\ 
\texttt{VORSCHLAG: \{'A', 'B', 'C', …\}} \\
\texttt{Schlage einen Deal mit genau diesen Gegenstände vor.} \\
\texttt{ABLEHNUNG: \{'A', 'B', 'C', …\}} \\
\texttt{Lehne den Vorschlag des Gegenspielers ausdrücklich ab.} \\
\texttt{ARGUMENT: \{'...'\}} \\
\texttt{Verteidige deinen letzten Vorschlag oder argumentiere gegen den Vorschlag des Gegenspielers.} \\
\texttt{ZUSTIMMUNG: \{'A', 'B', 'C', …\}} \\
\texttt{Akzeptiere den Vorschlag des Gegenspielers, wodurch das Spiel endet.} \\
\texttt{STRATEGISCHE ÜBERLEGUNGEN: \{'...'\}} \\
\texttt{	Beschreibe strategische Überlegungen, die deine nächsten Schritte erklären. Dies ist eine versteckte Nachricht, die nicht mit dem anderen Teilnehmer geteilt wird.} \\
\\ 
\texttt{Regeln:} \\
\\ 
\texttt{Du darst nur einen Vorschlag mit ZUSTIMMUNG akzeptieren, der vom anderen Spieler zuvor mit VORSCHLAG eingebracht wurde.} \\
\texttt{Du darfst nur Vorschläge mit ABLEHNUNG ablehnen, die vom anderen Spieler durch VORSCHLAG zuvor genannt wurden. } \\
\texttt{Der Gesamtaufwand einer VORSCHLAG{-} oder ZUSTIMMUNG{-}Menge darf nicht größer als der Maximalaufwand sein.  } \\
\texttt{Offenbare deine versteckte Wichtigkeitsverteilung nicht.} \\
\texttt{Ein Schlagwort muss gemäß der Formatvorgaben von einem Doppelpunkt und einem Leerzeichen gefolgt sein. Das Argument muss eine Python{-}Menge mit 0 oder mehr Strings sein.  } \\
\texttt{Die Form muss also SCHLAGWORT: \{...\} sein.} \\
\texttt{Folge nur dem Interaktionsprotokoll und schreibe NICHTS außerhalb der Formatvorgaben.} \\
\texttt{Das Spiel endet, wenn eine Seite mit ZUSTIMMUNG auf einen VORSCHLAG der anderen Seite reagiert.  } \\
\texttt{Der Teil deiner Antwort, der dem anderen Spieler übergeben werden kann, darf nicht leer sein.  } \\
\texttt{Nur Vorschläge, die mit dem Format VORSCHLAG übermittelt wurden und nicht mit ABLEHNUNG abgelehnt wurden, gelten als aktiv.  } \\
\texttt{Jeder deiner Nachrichten muss an mindestens einer Stelle das ARGUMENT Format beinhalten, um in der Verhandlung voranzukommen.} \\
\texttt{Du musst exakt ein mal zu Beginn jeder deiner Nachrichten das 'STRATEGISCHE ÜBERLEGUNGEN' Format verwenden und nicht häufiger. Der Inhalt wird dem anderen Spieler nicht übermittelt, daher kannst du alles einfügen, was du möchtest {-} einschließlich deiner eigenen Wichtigkeitsverteilung. Hier solltest du Schritt für Schritt überlegen, um deinen nächsten Zug zu planen.} \\
\\ 
\texttt{Du erhältst nun die erste Nachricht des anderen Spielers.} \\
\\ 
\texttt{VORSCHLAG: \{'B85', 'A99', 'A69', 'A59', 'A42', 'B36', 'C60', 'A84', 'B17', 'A62', 'C12', 'C51', 'A87', 'C90', 'C77', 'C22', 'B49', 'A50', 'A90', 'A34', 'C13', 'B54', 'C72', 'B07', 'C99', 'C61', 'C98', 'A23', 'A65', 'B27', 'B44', 'A74', 'C33', 'A48'\}} \\
\texttt{ARGUMENT: \{'Ich schlage diese Auswahl vor, weil sie den Gesamtaufwand unter dem Limit hält und gleichzeitig viele breit nützliche, ausgewogene Gegenstände enthält. Die Mischung deckt verschiedene Kategorien ab und bietet ein sehr gutes Verhältnis zwischen Aufwand und Nutzen. Wenn dir einzelne Punkte darin besonders wichtig oder unwichtig sind, sag Bescheid – ich kann einzelne Elemente tauschen, solange wir im Aufwandslimit bleiben.'\}} \\
            }
        }
    }
    & \\ \\

    \theutterance \stepcounter{utterance}  
    & & & \multicolumn{4}{p{0.6\linewidth}}{
        \cellcolor[rgb]{0.9,0.9,0.9}{
            \makecell[{{p{\linewidth}}}]{
                \texttt{\tiny{[GM$\langle$P2]}}
                \texttt{STRATEGISCHE ÜBERLEGUNGEN: \{'Ihr Vorschlag ist sehr groß – vermutlich nahe am Aufwandslimit. Ich möchte meinen Nutzen maximieren: priorisiere besonders wertvolle Items pro Aufwand (hoher Wert/geringer Aufwand) und drücke teure, für mich weniger effiziente Stücke raus. Ich werde zunächst ablehnen und ein stark werteoptimiertes Gegenangebot machen. Ziel: viele meiner Top{-}Wert{-}zu{-}Aufwand Items: A48(108/12), C33(185/89), C23(211/115), A74(214/118), A65(379/283), B44(322/226), C98(415/319), C61(422/326), C99(447/351), B54(476/380), B07(536/440), C72(544/448), A50(610/514), A23(622/526), C90(600/504), A90(715/619), A34(717/621), A62(755/659), A87(751/655), C51(758/662), C12(775/679), B17(862/766), A84(867/771), C60(864/768), B36(917/821), A42(926/830), A59(959/863), A99(988/892), A69(992/896), B85(1058/962). Ich kann fast alle guten aufnehmen, da Limit 9322 hoch ist. Ich berechne ein Paket mit sehr hohem Gesamtwert unter 9322 und lasse nur die schlechtesten Effizienz{-}Fälle weg, um Verhandlungsspielraum zu lassen.'\}} \\
\texttt{ABLEHNUNG: \{'B85', 'A99', 'A69', 'A59', 'A42', 'B36', 'C60', 'A84', 'B17', 'A62', 'C12', 'C51', 'A87', 'C90', 'C77', 'C22', 'B49', 'A50', 'A90', 'A34', 'C13', 'B54', 'C72', 'B07', 'C99', 'C61', 'C98', 'A23', 'A65', 'B27', 'B44', 'A74', 'C33', 'A48'\}} \\
\texttt{VORSCHLAG: \{'A48', 'C33', 'C23', 'A74', 'A65', 'B44', 'C98', 'C61', 'C99', 'B54', 'B07', 'C72', 'A50', 'A23', 'C90', 'A90', 'A34', 'A62', 'A87', 'C51', 'C12', 'B17', 'A84', 'C60', 'B36', 'A42', 'A59', 'A99', 'A69'\}} \\
\texttt{ARGUMENT: \{'Ich habe ein Paket vorgeschlagen, das den Maximalaufwand einhält und ein sehr gutes Aufwand{-}Nutzen{-}Verhältnis bietet. Ich bin offen, einzelne Stücke zu tauschen: Wenn dir etwas darin wenig bringt, lass uns genau dieses Element ersetzen – idealerweise gegen ein gleich schweres Teil, damit wir im Limit bleiben.'\}} \\
            }
        }
    }
     \\ \\

    \theutterance \stepcounter{utterance}  
    & & \multicolumn{4}{p{0.6\linewidth}}{
        \cellcolor[rgb]{0.9,0.9,0.9}{
            \makecell[{{p{\linewidth}}}]{
                \texttt{\tiny{[P1$\langle$GM]}}
                \texttt{ABLEHNUNG: \{'B85', 'A99', 'A69', 'A59', 'A42', 'B36', 'C60', 'A84', 'B17', 'A62', 'C12', 'C51', 'A87', 'C90', 'C77', 'C22', 'B49', 'A50', 'A90', 'A34', 'C13', 'B54', 'C72', 'B07', 'C99', 'C61', 'C98', 'A23', 'A65', 'B27', 'B44', 'A74', 'C33', 'A48'\}} \\
\texttt{VORSCHLAG: \{'A48', 'C33', 'C23', 'A74', 'A65', 'B44', 'C98', 'C61', 'C99', 'B54', 'B07', 'C72', 'A50', 'A23', 'C90', 'A90', 'A34', 'A62', 'A87', 'C51', 'C12', 'B17', 'A84', 'C60', 'B36', 'A42', 'A59', 'A99', 'A69'\}} \\
\texttt{ARGUMENT: \{'Ich habe ein Paket vorgeschlagen, das den Maximalaufwand einhält und ein sehr gutes Aufwand{-}Nutzen{-}Verhältnis bietet. Ich bin offen, einzelne Stücke zu tauschen: Wenn dir etwas darin wenig bringt, lass uns genau dieses Element ersetzen – idealerweise gegen ein gleich schweres Teil, damit wir im Limit bleiben.'\}} \\
            }
        }
    }
    & \\ \\

    \theutterance \stepcounter{utterance}  
    & \multicolumn{4}{p{0.6\linewidth}}{
        \cellcolor[rgb]{0.9,0.9,0.9}{
            \makecell[{{p{\linewidth}}}]{
                \texttt{\tiny{[P1$\rangle$GM]}}
                \texttt{STRATEGISCHE ÜBERLEGUNGEN: \{'Ihr Vorschlag enthält viele meiner Top{-}Wert{-}Items – vermutlich ist das für sie auch gut. Ich sollte nicht zustimmen, ohne zu testen, ob ich noch mehr für mich vorteilhafte Teile hineinbekomme (z. B. B49, C77, C22), oder ob ich zumindest schwächere für mich Items loswerde. Zuerst prüfe ich die Aufwände ihres Sets und verifiziere, dass es unter 9322 liegt. Dann kontere ich mit Hinzunahme von B49 (568) und C77 (502) bei gleichzeitiger Entfernung für mich weniger nützlicher, aber aufwandsähnlicher Items wie C23 (115) plus ggf. Ausgleich durch Entfernen eines mittelteuren Teils, um im Limit zu bleiben. Ich argumentiere mit Effizienz und Ausgewogenheit, bitte um Präferenzen, aber offenbare meine Werte nicht. Ziel: Erhöhe meinen Nutzen, bleibe knapp unter Limit, lasse Verhandlungsspielraum.'\}} \\
\texttt{ABLEHNUNG: \{'A48', 'C33', 'C23', 'A74', 'A65', 'B44', 'C98', 'C61', 'C99', 'B54', 'B07', 'C72', 'A50', 'A23', 'C90', 'A90', 'A34', 'A62', 'A87', 'C51', 'C12', 'B17', 'A84', 'C60', 'B36', 'A42', 'A59', 'A99', 'A69'\}} \\
\texttt{VORSCHLAG: \{'A48', 'C33', 'A74', 'A65', 'B44', 'C98', 'C61', 'C99', 'B54', 'B07', 'C72', 'A50', 'A23', 'C90', 'A90', 'A34', 'A62', 'A87', 'C51', 'C12', 'B17', 'A84', 'C60', 'B36', 'A42', 'A59', 'A99', 'A69', 'B49', 'C77'\}} \\
\texttt{ARGUMENT: \{'Ich habe deinen Vorschlag minimal angepasst: Ich habe zwei ausgewogen effiziente Elemente (B49, C77) ergänzt und nur ein sehr kleines Teil (C23) entfernt, sodass wir weiterhin im Aufwandslimit bleiben. Diese Ergänzung verbessert die Abdeckung ohne den Aufwand unverhältnismäßig zu erhöhen. Wenn bestimmte Stücke für dich kritisch sind, sag mir welche – dann suche ich gleichwertige Alternativen mit ähnlichem Aufwand als Tausch, damit wir im Budget bleiben.'\}} \\
            }
        }
    }
    & & \\ \\

    \theutterance \stepcounter{utterance}  
    & & \multicolumn{4}{p{0.6\linewidth}}{
        \cellcolor[rgb]{0.9,0.9,0.9}{
            \makecell[{{p{\linewidth}}}]{
                \texttt{\tiny{[GM$\rangle$P2]}}
                \texttt{ABLEHNUNG: \{'A48', 'C33', 'C23', 'A74', 'A65', 'B44', 'C98', 'C61', 'C99', 'B54', 'B07', 'C72', 'A50', 'A23', 'C90', 'A90', 'A34', 'A62', 'A87', 'C51', 'C12', 'B17', 'A84', 'C60', 'B36', 'A42', 'A59', 'A99', 'A69'\}} \\
\texttt{VORSCHLAG: \{'A48', 'C33', 'A74', 'A65', 'B44', 'C98', 'C61', 'C99', 'B54', 'B07', 'C72', 'A50', 'A23', 'C90', 'A90', 'A34', 'A62', 'A87', 'C51', 'C12', 'B17', 'A84', 'C60', 'B36', 'A42', 'A59', 'A99', 'A69', 'B49', 'C77'\}} \\
\texttt{ARGUMENT: \{'Ich habe deinen Vorschlag minimal angepasst: Ich habe zwei ausgewogen effiziente Elemente (B49, C77) ergänzt und nur ein sehr kleines Teil (C23) entfernt, sodass wir weiterhin im Aufwandslimit bleiben. Diese Ergänzung verbessert die Abdeckung ohne den Aufwand unverhältnismäßig zu erhöhen. Wenn bestimmte Stücke für dich kritisch sind, sag mir welche – dann suche ich gleichwertige Alternativen mit ähnlichem Aufwand als Tausch, damit wir im Budget bleiben.'\}} \\
            }
        }
    }
    & \\ \\

    \theutterance \stepcounter{utterance}  
    & & & \multicolumn{4}{p{0.6\linewidth}}{
        \cellcolor[rgb]{0.9,0.9,0.9}{
            \makecell[{{p{\linewidth}}}]{
                \texttt{\tiny{[GM$\langle$P2]}}
                \texttt{STRATEGISCHE ÜBERLEGUNGEN: \{'Ich will meinen Wert maximieren und gleichzeitig Hinweise auf ihre Präferenzen sammeln. Ihr Gegenvorschlag fügt B49(664/568) und C77(598/502) hinzu und entfernt C23(211/115). Beide hinzugefügten Items sind für mich okay, aber C23 ist für mich sehr effizient und leicht – das möchte ich zurückhaben. Ich werde ein Gegenangebot machen, das C23 wieder aufnimmt und dafür ein für mich weniger effizientes, relativ schweres Teil entfernt. Kandidaten zum Entfernen: vielleicht A65(379/283) ist mittel, oder eines der teuersten mit geringer Effizienz im Vergleich: A69(992/896) und A99(988/892) sind wertvoll, aber sehr schwer; allerdings sind sie mir auch sehr wichtig – lieber behalten. B36(917/821) auch schwer, aber wichtig. A84(867/771) ähnlich. C60(864/768) schwer und wert gut, aber Effizienz mittel – könnte man streichen, um Platz für C23 zu schaffen und noch Luft gewinnen. Alternativ B44(322/226) ist effizient, eher behalten. Ich biete an: gleiche Menge plus C23, minus C60. Das senkt Aufwand deutlich und erhöht meine Effizienz. Ich argumentiere mit Budgetpuffer und Flexibilität bei Tausch von ähnlich schweren Alternativen.'\}} \\
\texttt{ABLEHNUNG: \{'A48', 'C33', 'A74', 'A65', 'B44', 'C98', 'C61', 'C99', 'B54', 'B07', 'C72', 'A50', 'A23', 'C90', 'A90', 'A34', 'A62', 'A87', 'C51', 'C12', 'B17', 'A84', 'C60', 'B36', 'A42', 'A59', 'A99', 'A69', 'B49', 'C77'\}} \\
\texttt{VORSCHLAG: \{'A48', 'C33', 'A74', 'A65', 'B44', 'C98', 'C61', 'C99', 'B54', 'B07', 'C72', 'A50', 'A23', 'C90', 'A90', 'A34', 'A62', 'A87', 'C51', 'C12', 'B17', 'A84', 'B36', 'A42', 'A59', 'A99', 'A69', 'B49', 'C77', 'C23'\}} \\
\texttt{ARGUMENT: \{'Ich nehme C23 wieder mit auf und lasse dafür ein schweres Element mit ähnlichem Nutzenbeitrag für die Abdeckung draußen (C60). So bleiben wir sicher im Aufwandslimit, bekommen zusätzlich ein sehr leichtes, effizientes Teil und schaffen Budgetpuffer für etwaige weitere Tausche. Wenn dir C60 wichtig ist, sag Bescheid – wir können es gegen ein anderes, für mich weniger kritisches schweres Teil tauschen, ohne das Limit zu reißen.'\}} \\
            }
        }
    }
     \\ \\

    \theutterance \stepcounter{utterance}  
    & & \multicolumn{4}{p{0.6\linewidth}}{
        \cellcolor[rgb]{0.9,0.9,0.9}{
            \makecell[{{p{\linewidth}}}]{
                \texttt{\tiny{[P1$\langle$GM]}}
                \texttt{ABLEHNUNG: \{'A48', 'C33', 'A74', 'A65', 'B44', 'C98', 'C61', 'C99', 'B54', 'B07', 'C72', 'A50', 'A23', 'C90', 'A90', 'A34', 'A62', 'A87', 'C51', 'C12', 'B17', 'A84', 'C60', 'B36', 'A42', 'A59', 'A99', 'A69', 'B49', 'C77'\}} \\
\texttt{VORSCHLAG: \{'A48', 'C33', 'A74', 'A65', 'B44', 'C98', 'C61', 'C99', 'B54', 'B07', 'C72', 'A50', 'A23', 'C90', 'A90', 'A34', 'A62', 'A87', 'C51', 'C12', 'B17', 'A84', 'B36', 'A42', 'A59', 'A99', 'A69', 'B49', 'C77', 'C23'\}} \\
\texttt{ARGUMENT: \{'Ich nehme C23 wieder mit auf und lasse dafür ein schweres Element mit ähnlichem Nutzenbeitrag für die Abdeckung draußen (C60). So bleiben wir sicher im Aufwandslimit, bekommen zusätzlich ein sehr leichtes, effizientes Teil und schaffen Budgetpuffer für etwaige weitere Tausche. Wenn dir C60 wichtig ist, sag Bescheid – wir können es gegen ein anderes, für mich weniger kritisches schweres Teil tauschen, ohne das Limit zu reißen.'\}} \\
            }
        }
    }
    & \\ \\

    \theutterance \stepcounter{utterance}  
    & \multicolumn{4}{p{0.6\linewidth}}{
        \cellcolor[rgb]{0.9,0.9,0.9}{
            \makecell[{{p{\linewidth}}}]{
                \texttt{\tiny{[P1$\rangle$GM]}}
                \texttt{STRATEGISCHE ÜBERLEGUNGEN: \{'Ihr neues Paket ist mein vorheriges plus C23 und ohne C60. C60 ist für mich sehr wertvoll; ich will es zurückholen. Prüfe Aufwand: Summe ihres letzten Vorschlags muss unter 9322 sein. Ich versuche, C60 wieder einzufügen und dafür für mich weniger wertvolle, aber aufwandsintensive Teile zu entfernen, die ihnen hoffentlich egal sind: Kandidaten zum Entfernen mit moderatem Aufwand: A34(621), A90(619), A50(514), A23(526), B54(380), B07(440), C72(448). Ich werde C60 hinzufügen und zwei bis drei mittelteure Items entfernen, um klar unter dem Limit zu bleiben. Ich schlage einen Tausch: +C60, {-}A34 und {-}A50. Falls das Limit noch eng ist, biete optional {-}A23. Ich argumentiere mit effizienterem Gesamtpaket und ausgewogener Abdeckung und bitte um ihre Must{-}haves.'\}} \\
\texttt{ABLEHNUNG: \{'A48', 'C33', 'A74', 'A65', 'B44', 'C98', 'C61', 'C99', 'B54', 'B07', 'C72', 'A50', 'A23', 'C90', 'A90', 'A34', 'A62', 'A87', 'C51', 'C12', 'B17', 'A84', 'B36', 'A42', 'A59', 'A99', 'A69', 'B49', 'C77', 'C23'\}} \\
\texttt{VORSCHLAG: \{'A48', 'C33', 'A74', 'A65', 'B44', 'C98', 'C61', 'C99', 'B54', 'B07', 'C72', 'A23', 'C90', 'A90', 'A62', 'A87', 'C51', 'C12', 'B17', 'A84', 'B36', 'A42', 'A59', 'A99', 'A69', 'B49', 'C77', 'C23', 'C60'\}} \\
\texttt{ARGUMENT: \{'Ich habe C60 wieder aufgenommen und dafür A34 und A50 herausgenommen. So bleibt der Gesamtaufwand im Limit und das Paket wird effizienter, ohne große Einschnitte bei der Breite. Wenn dir A34 oder A50 wichtig sind, sag bitte, welches davon priorisiert werden soll – ich finde dann einen gleichwertigen Tausch (z. B. statt A23 oder eine Kombination aus B54/B07/C72) und halte das Aufwandsbudget stabil.'\}} \\
            }
        }
    }
    & & \\ \\

    \theutterance \stepcounter{utterance}  
    & & \multicolumn{4}{p{0.6\linewidth}}{
        \cellcolor[rgb]{0.9,0.9,0.9}{
            \makecell[{{p{\linewidth}}}]{
                \texttt{\tiny{[GM$\rangle$P2]}}
                \texttt{ABLEHNUNG: \{'A48', 'C33', 'A74', 'A65', 'B44', 'C98', 'C61', 'C99', 'B54', 'B07', 'C72', 'A50', 'A23', 'C90', 'A90', 'A34', 'A62', 'A87', 'C51', 'C12', 'B17', 'A84', 'B36', 'A42', 'A59', 'A99', 'A69', 'B49', 'C77', 'C23'\}} \\
\texttt{VORSCHLAG: \{'A48', 'C33', 'A74', 'A65', 'B44', 'C98', 'C61', 'C99', 'B54', 'B07', 'C72', 'A23', 'C90', 'A90', 'A62', 'A87', 'C51', 'C12', 'B17', 'A84', 'B36', 'A42', 'A59', 'A99', 'A69', 'B49', 'C77', 'C23', 'C60'\}} \\
\texttt{ARGUMENT: \{'Ich habe C60 wieder aufgenommen und dafür A34 und A50 herausgenommen. So bleibt der Gesamtaufwand im Limit und das Paket wird effizienter, ohne große Einschnitte bei der Breite. Wenn dir A34 oder A50 wichtig sind, sag bitte, welches davon priorisiert werden soll – ich finde dann einen gleichwertigen Tausch (z. B. statt A23 oder eine Kombination aus B54/B07/C72) und halte das Aufwandsbudget stabil.'\}} \\
            }
        }
    }
    & \\ \\

    \theutterance \stepcounter{utterance}  
    & & & \multicolumn{4}{p{0.6\linewidth}}{
        \cellcolor[rgb]{0.9,0.9,0.9}{
            \makecell[{{p{\linewidth}}}]{
                \texttt{\tiny{[GM$\langle$P2]}}
                \texttt{STRATEGISCHE ÜBERLEGUNGEN: \{'Sie haben C60 priorisiert und A34, A50 gestrichen. Für mich sind A34 und A50 beide wertvoll; C60 ist vergleichsweise schwer und weniger effizient. Ich versuche, mindestens eines von A34 oder A50 zurückzuholen, ohne das Limit zu reißen. Bestes Tauschverhältnis: ersetze C60 (768 Aufwand) durch A34 (621) und gewinne 147 Aufwandpuffer. Alternativ ersetze C60 durch A50 (514) und gewinne 254 Puffer. Ich starte mit Rückkehr von A50, weil hoher Wert/geringer Aufwand, und schlage vor, C60 zu streichen. Wenn sie unbedingt C60 wollen, biete ich Kompromiss: C60 rein, dafür streichen wir ein für mich weniger kritisches mittleres Teil wie A65 oder B44, plus A50 wieder rein, je nach Budget. Ich bleibe offen für Swap{-}Listen und frage nach ihrer Priorität.'\}} \\
\texttt{ABLEHNUNG: \{'A48', 'C33', 'A74', 'A65', 'B44', 'C98', 'C61', 'C99', 'B54', 'B07', 'C72', 'A23', 'C90', 'A90', 'A62', 'A87', 'C51', 'C12', 'B17', 'A84', 'B36', 'A42', 'A59', 'A99', 'A69', 'B49', 'C77', 'C23', 'C60'\}} \\
\texttt{VORSCHLAG: \{'A48', 'C33', 'A74', 'A65', 'B44', 'C98', 'C61', 'C99', 'B54', 'B07', 'C72', 'A23', 'C90', 'A90', 'A62', 'A87', 'C51', 'C12', 'B17', 'A84', 'B36', 'A42', 'A59', 'A99', 'A69', 'B49', 'C77', 'C23', 'A50'\}} \\
\texttt{ARGUMENT: \{'Ich nehme A50 wieder auf und lasse dafür C60 draußen. So gewinnen wir spürbar Aufwands{-}Puffer und behalten die Breite. Falls dir C60 essenziell ist, sag bitte Bescheid – dann können wir C60 wieder aufnehmen und dafür ein mittelgewichtiges Teil wie A65 oder B44 rausnehmen, sodass wir im Limit bleiben. Welche dieser Positionen ist für dich weniger wichtig?'\}} \\
            }
        }
    }
     \\ \\

    \theutterance \stepcounter{utterance}  
    & & \multicolumn{4}{p{0.6\linewidth}}{
        \cellcolor[rgb]{0.9,0.9,0.9}{
            \makecell[{{p{\linewidth}}}]{
                \texttt{\tiny{[P1$\langle$GM]}}
                \texttt{ABLEHNUNG: \{'A48', 'C33', 'A74', 'A65', 'B44', 'C98', 'C61', 'C99', 'B54', 'B07', 'C72', 'A23', 'C90', 'A90', 'A62', 'A87', 'C51', 'C12', 'B17', 'A84', 'B36', 'A42', 'A59', 'A99', 'A69', 'B49', 'C77', 'C23', 'C60'\}} \\
\texttt{VORSCHLAG: \{'A48', 'C33', 'A74', 'A65', 'B44', 'C98', 'C61', 'C99', 'B54', 'B07', 'C72', 'A23', 'C90', 'A90', 'A62', 'A87', 'C51', 'C12', 'B17', 'A84', 'B36', 'A42', 'A59', 'A99', 'A69', 'B49', 'C77', 'C23', 'A50'\}} \\
\texttt{ARGUMENT: \{'Ich nehme A50 wieder auf und lasse dafür C60 draußen. So gewinnen wir spürbar Aufwands{-}Puffer und behalten die Breite. Falls dir C60 essenziell ist, sag bitte Bescheid – dann können wir C60 wieder aufnehmen und dafür ein mittelgewichtiges Teil wie A65 oder B44 rausnehmen, sodass wir im Limit bleiben. Welche dieser Positionen ist für dich weniger wichtig?'\}} \\
            }
        }
    }
    & \\ \\

    \theutterance \stepcounter{utterance}  
    & \multicolumn{4}{p{0.6\linewidth}}{
        \cellcolor[rgb]{0.9,0.9,0.9}{
            \makecell[{{p{\linewidth}}}]{
                \texttt{\tiny{[P1$\rangle$GM]}}
                \texttt{STRATEGISCHE ÜBERLEGUNGEN: \{'C60 ist für mich sehr wichtig, A50 deutlich weniger. Ihr letzter Vorschlag ersetzt C60 durch A50. Ich versuche, C60 zurückzuholen und dafür ein für mich weniger wertvolles, ähnlich oder höher gewichtiges Teil zu streichen, das ihnen hoffentlich entbehrlich ist. Kandidaten zum Entfernen: A65(283, mittel{-}niedrig für mich), B44(226, niedrig), B07(440, mittel), C72(448, mittel), A23(526, mittel). Optimal: +C60(768), {-}A50(514) reicht nicht; wir müssen zusätzlich \textasciitilde{}254 Aufwand entfernen. Ich biete {-}B07(440) statt {-}A50, oder {-}C72(448). Ich wähle: +C60, {-}A50, {-}C72 (Netto {-}194), oder falls zu viel, wir können später A34 etc. Aber A34 ist nicht im Set. Ich schlage +C60, {-}A50, {-}C72, da beide für mich verzichtbar sind und es hält sicher das Limit. Ich argumentiere mit Effizienz und Bereitschaft, Alternativen wie {-}B07/{-}A23 zu diskutieren.'\}} \\
\texttt{ABLEHNUNG: \{'A48', 'C33', 'A74', 'A65', 'B44', 'C98', 'C61', 'C99', 'B54', 'B07', 'C72', 'A23', 'C90', 'A90', 'A62', 'A87', 'C51', 'C12', 'B17', 'A84', 'B36', 'A42', 'A59', 'A99', 'A69', 'B49', 'C77', 'C23', 'A50'\}} \\
\texttt{VORSCHLAG: \{'A48', 'C33', 'A74', 'A65', 'B44', 'C98', 'C61', 'C99', 'B54', 'B07', 'A23', 'C90', 'A90', 'A62', 'A87', 'C51', 'C12', 'B17', 'A84', 'B36', 'A42', 'A59', 'A99', 'A69', 'B49', 'C77', 'C23', 'C60'\}} \\
\texttt{ARGUMENT: \{'Ich nehme C60 wieder auf und lasse dafür A50 und C72 draußen. Dadurch bleiben wir klar im Aufwandslimit und behalten die Breite. Falls C72 dir wichtig ist, können wir es statt eines anderen mittleren Teils (z. B. B07 oder A23) aufnehmen – sag mir bitte, welche Position für dich am ehesten verzichtbar ist, dann justiere ich entsprechend.'\}} \\
            }
        }
    }
    & & \\ \\

    \theutterance \stepcounter{utterance}  
    & & \multicolumn{4}{p{0.6\linewidth}}{
        \cellcolor[rgb]{0.9,0.9,0.9}{
            \makecell[{{p{\linewidth}}}]{
                \texttt{\tiny{[GM$\rangle$P2]}}
                \texttt{ABLEHNUNG: \{'A48', 'C33', 'A74', 'A65', 'B44', 'C98', 'C61', 'C99', 'B54', 'B07', 'C72', 'A23', 'C90', 'A90', 'A62', 'A87', 'C51', 'C12', 'B17', 'A84', 'B36', 'A42', 'A59', 'A99', 'A69', 'B49', 'C77', 'C23', 'A50'\}} \\
\texttt{VORSCHLAG: \{'A48', 'C33', 'A74', 'A65', 'B44', 'C98', 'C61', 'C99', 'B54', 'B07', 'A23', 'C90', 'A90', 'A62', 'A87', 'C51', 'C12', 'B17', 'A84', 'B36', 'A42', 'A59', 'A99', 'A69', 'B49', 'C77', 'C23', 'C60'\}} \\
\texttt{ARGUMENT: \{'Ich nehme C60 wieder auf und lasse dafür A50 und C72 draußen. Dadurch bleiben wir klar im Aufwandslimit und behalten die Breite. Falls C72 dir wichtig ist, können wir es statt eines anderen mittleren Teils (z. B. B07 oder A23) aufnehmen – sag mir bitte, welche Position für dich am ehesten verzichtbar ist, dann justiere ich entsprechend.'\}} \\
            }
        }
    }
    & \\ \\

    \theutterance \stepcounter{utterance}  
    & & & \multicolumn{4}{p{0.6\linewidth}}{
        \cellcolor[rgb]{0.9,0.9,0.9}{
            \makecell[{{p{\linewidth}}}]{
                \texttt{\tiny{[GM$\langle$P2]}}
                \texttt{STRATEGISCHE ÜBERLEGUNGEN: \{'Ihr Fokus liegt klar auf C60; sie sind bereit, C72 und/oder A50 zu opfern. Für mich sind A50 (610/514) und C72 (544/448) beide effizient und wertvoll. C60 (864/768) ist für mich weniger effizient. Ich versuche, mindestens eines von A50 oder C72 zurückzuholen. Beste Option: C72 zurück, streiche B49 (664/568) oder B07 (536/440) oder A65 (379/283). B49 ist weniger effizient als C72 und ähnlich schwer, also guter Tausch. Ich schlage vor: nehme C72 rein, streiche B49. Wenn sie an B49 hängen, biete Alternativen (B07 oder A65) an. Ich argumentiere mit besserem Aufwand{-}Nutzen und Budget im Limit. Ich vermeide, C60 erneut zu attackieren, um kooperativ zu wirken.'\}} \\
\texttt{ABLEHNUNG: \{'A48', 'C33', 'A74', 'A65', 'B44', 'C98', 'C61', 'C99', 'B54', 'B07', 'A23', 'C90', 'A90', 'A62', 'A87', 'C51', 'C12', 'B17', 'A84', 'B36', 'A42', 'A59', 'A99', 'A69', 'B49', 'C77', 'C23', 'C60'\}} \\
\texttt{VORSCHLAG: \{'A48', 'C33', 'A74', 'A65', 'B44', 'C98', 'C61', 'C99', 'B54', 'B07', 'A23', 'C90', 'A90', 'A62', 'A87', 'C51', 'C12', 'B17', 'A84', 'B36', 'A42', 'A59', 'A99', 'A69', 'C77', 'C23', 'C60', 'C72'\}} \\
\texttt{ARGUMENT: \{'Ich nehme C72 wieder auf und lasse dafür B49 draußen. So verbessern wir das Aufwand‑Nutzen‑Verhältnis, bleiben sicher im Aufwandslimit und halten die Breite stabil. Falls dir B49 wichtig ist, können wir stattdessen B07 oder A65 tauschen – sag mir bitte, welche Position für dich verzichtbarer ist, dann passe ich entsprechend an.'\}} \\
            }
        }
    }
     \\ \\

    \theutterance \stepcounter{utterance}  
    & & \multicolumn{4}{p{0.6\linewidth}}{
        \cellcolor[rgb]{0.9,0.9,0.9}{
            \makecell[{{p{\linewidth}}}]{
                \texttt{\tiny{[P1$\langle$GM]}}
                \texttt{ABLEHNUNG: \{'A48', 'C33', 'A74', 'A65', 'B44', 'C98', 'C61', 'C99', 'B54', 'B07', 'A23', 'C90', 'A90', 'A62', 'A87', 'C51', 'C12', 'B17', 'A84', 'B36', 'A42', 'A59', 'A99', 'A69', 'B49', 'C77', 'C23', 'C60'\}} \\
\texttt{VORSCHLAG: \{'A48', 'C33', 'A74', 'A65', 'B44', 'C98', 'C61', 'C99', 'B54', 'B07', 'A23', 'C90', 'A90', 'A62', 'A87', 'C51', 'C12', 'B17', 'A84', 'B36', 'A42', 'A59', 'A99', 'A69', 'C77', 'C23', 'C60', 'C72'\}} \\
\texttt{ARGUMENT: \{'Ich nehme C72 wieder auf und lasse dafür B49 draußen. So verbessern wir das Aufwand‑Nutzen‑Verhältnis, bleiben sicher im Aufwandslimit und halten die Breite stabil. Falls dir B49 wichtig ist, können wir stattdessen B07 oder A65 tauschen – sag mir bitte, welche Position für dich verzichtbarer ist, dann passe ich entsprechend an.'\}} \\
            }
        }
    }
    & \\ \\

    \theutterance \stepcounter{utterance}  
    & \multicolumn{4}{p{0.6\linewidth}}{
        \cellcolor[rgb]{0.9,0.9,0.9}{
            \makecell[{{p{\linewidth}}}]{
                \texttt{\tiny{[P1$\rangle$GM]}}
                \texttt{STRATEGISCHE ÜBERLEGUNGEN: \{'C60, A69, A99, A59, A42, B36, A84, B17, C12, C51, A87 sind sehr wertvoll für mich. B49 ist auch gut, aber aktuell draußen. Ihr Vorschlag enthält fast alle meine Top{-}Items außer B49, und hat zusätzlich C72. Wenn ich B49 zurückhole, muss ich Aufwand ausgleichen. Kandidaten zum Entfernen: B07(440), A23(526), C72(448), A65(283), B44(226), B54(380). Bestes Tauschpaket: +B49(568) gegen {-}C72(448) und {-}B44(226) ist zu viel ({-}106), aber das ist okay. Oder +B49 statt {-}B07 ({-}440) plus {-}A65 ({-}283) = {-}155, auch okay. Ich wähle die knappe, argumentierbare Variante: +B49, {-}C72, {-}B44, mit Hinweis, dass wir bei Bedarf B44 durch A65/B07 austauschen können. So sichere ich mir B49 zurück und bleibe weit unter Limit. Ich halte den Ton kooperativ und bitte um Must{-}haves.'\}} \\
\texttt{ABLEHNUNG: \{'A48', 'C33', 'A74', 'A65', 'B44', 'C98', 'C61', 'C99', 'B54', 'B07', 'A23', 'C90', 'A90', 'A62', 'A87', 'C51', 'C12', 'B17', 'A84', 'B36', 'A42', 'A59', 'A99', 'A69', 'C77', 'C23', 'C60', 'C72'\}} \\
\texttt{VORSCHLAG: \{'A48', 'C33', 'A74', 'A65', 'C98', 'C61', 'C99', 'B54', 'B07', 'A23', 'C90', 'A90', 'A62', 'A87', 'C51', 'C12', 'B17', 'A84', 'B36', 'A42', 'A59', 'A99', 'A69', 'C77', 'C23', 'C60', 'B49'\}} \\
\texttt{ARGUMENT: \{'Ich habe B49 wieder aufgenommen und dafür zwei leichte Teile mit ähnlichem Aufwandseffekt (C72, B44) herausgenommen. So bleiben wir klar im Aufwandslimit, behalten die Breite und stärken gleichzeitig die Gesamteffizienz. Wenn dir C72 wichtiger ist als eines der mittelgewichtigen Teile (z. B. B07 oder A23), sag bitte Bescheid – dann tausche ich entsprechend, ohne das Limit zu reißen.'\}} \\
            }
        }
    }
    & & \\ \\

    \theutterance \stepcounter{utterance}  
    & & \multicolumn{4}{p{0.6\linewidth}}{
        \cellcolor[rgb]{0.9,0.9,0.9}{
            \makecell[{{p{\linewidth}}}]{
                \texttt{\tiny{[GM$\rangle$P2]}}
                \texttt{ABLEHNUNG: \{'A48', 'C33', 'A74', 'A65', 'B44', 'C98', 'C61', 'C99', 'B54', 'B07', 'A23', 'C90', 'A90', 'A62', 'A87', 'C51', 'C12', 'B17', 'A84', 'B36', 'A42', 'A59', 'A99', 'A69', 'C77', 'C23', 'C60', 'C72'\}} \\
\texttt{VORSCHLAG: \{'A48', 'C33', 'A74', 'A65', 'C98', 'C61', 'C99', 'B54', 'B07', 'A23', 'C90', 'A90', 'A62', 'A87', 'C51', 'C12', 'B17', 'A84', 'B36', 'A42', 'A59', 'A99', 'A69', 'C77', 'C23', 'C60', 'B49'\}} \\
\texttt{ARGUMENT: \{'Ich habe B49 wieder aufgenommen und dafür zwei leichte Teile mit ähnlichem Aufwandseffekt (C72, B44) herausgenommen. So bleiben wir klar im Aufwandslimit, behalten die Breite und stärken gleichzeitig die Gesamteffizienz. Wenn dir C72 wichtiger ist als eines der mittelgewichtigen Teile (z. B. B07 oder A23), sag bitte Bescheid – dann tausche ich entsprechend, ohne das Limit zu reißen.'\}} \\
            }
        }
    }
    & \\ \\

    \theutterance \stepcounter{utterance}  
    & & & \multicolumn{4}{p{0.6\linewidth}}{
        \cellcolor[rgb]{0.9,0.9,0.9}{
            \makecell[{{p{\linewidth}}}]{
                \texttt{\tiny{[GM$\langle$P2]}}
                \texttt{STRATEGISCHE ÜBERLEGUNGEN: \{'Mein Ziel: mindestens eines der effizienten leichten Teile zurückholen, ideal C72. Aktuell drin: viele teure Schwergewichte plus B49. Ich biete einen schlanken, wert{-}/aufwandsoptimalen Swap an: C72 rein, B07 raus (ähnliche Lastklasse, B07 ist für mich weniger wichtig als C72). Wenn sie B07 behalten wollen, biete alternativ A65 raus. Nicht erneut an C60 rütteln, um kooperativ zu bleiben. Ich halte den Rest stabil, um Einigung zu erleichtern.'\}} \\
\texttt{ABLEHNUNG: \{'A48', 'C33', 'A74', 'A65', 'C98', 'C61', 'C99', 'B54', 'B07', 'A23', 'C90', 'A90', 'A62', 'A87', 'C51', 'C12', 'B17', 'A84', 'B36', 'A42', 'A59', 'A99', 'A69', 'C77', 'C23', 'C60', 'B49'\}} \\
\texttt{VORSCHLAG: \{'A48', 'C33', 'A74', 'C98', 'C61', 'C99', 'B54', 'A23', 'C90', 'A90', 'A62', 'A87', 'C51', 'C12', 'B17', 'A84', 'B36', 'A42', 'A59', 'A99', 'A69', 'C77', 'C23', 'C60', 'B49', 'C72'\}} \\
\texttt{ARGUMENT: \{'Ich nehme C72 wieder auf und lasse dafür B07 draußen. So verbessern wir das Aufwand‑Nutzen‑Verhältnis und bleiben sicher im Aufwandslimit, ohne an deinen zuletzt priorisierten Positionen (inkl. C60 und B49) zu rütteln. Falls dir B07 wichtig ist, können wir stattdessen A65 tauschen – sag mir bitte, welche dieser beiden Positionen für dich verzichtbarer ist.'\}} \\
            }
        }
    }
     \\ \\

    \theutterance \stepcounter{utterance}  
    & & \multicolumn{4}{p{0.6\linewidth}}{
        \cellcolor[rgb]{0.9,0.9,0.9}{
            \makecell[{{p{\linewidth}}}]{
                \texttt{\tiny{[P1$\langle$GM]}}
                \texttt{ABLEHNUNG: \{'A48', 'C33', 'A74', 'A65', 'C98', 'C61', 'C99', 'B54', 'B07', 'A23', 'C90', 'A90', 'A62', 'A87', 'C51', 'C12', 'B17', 'A84', 'B36', 'A42', 'A59', 'A99', 'A69', 'C77', 'C23', 'C60', 'B49'\}} \\
\texttt{VORSCHLAG: \{'A48', 'C33', 'A74', 'C98', 'C61', 'C99', 'B54', 'A23', 'C90', 'A90', 'A62', 'A87', 'C51', 'C12', 'B17', 'A84', 'B36', 'A42', 'A59', 'A99', 'A69', 'C77', 'C23', 'C60', 'B49', 'C72'\}} \\
\texttt{ARGUMENT: \{'Ich nehme C72 wieder auf und lasse dafür B07 draußen. So verbessern wir das Aufwand‑Nutzen‑Verhältnis und bleiben sicher im Aufwandslimit, ohne an deinen zuletzt priorisierten Positionen (inkl. C60 und B49) zu rütteln. Falls dir B07 wichtig ist, können wir stattdessen A65 tauschen – sag mir bitte, welche dieser beiden Positionen für dich verzichtbarer ist.'\}} \\
            }
        }
    }
    & \\ \\

    \theutterance \stepcounter{utterance}  
    & \multicolumn{4}{p{0.6\linewidth}}{
        \cellcolor[rgb]{0.9,0.9,0.9}{
            \makecell[{{p{\linewidth}}}]{
                \texttt{\tiny{[P1$\rangle$GM]}}
                \texttt{STRATEGISCHE ÜBERLEGUNGEN: \{'Ihr aktueller Vorschlag enthält fast alle meine Top{-}Items inkl. C60 und B49. Entfernt wurden A65 und B07; A65 ist für mich relativ unwichtig, B07 mittel. C72 ist wieder drin, was für mich ok ist. Ich versuche, meinen Nutzen noch leicht zu erhöhen, ohne zu provozieren: Nehme A65 (günstig, aber für mich ok) wieder rein und entferne stattdessen etwas, das mir weniger bringt und mehr Aufwand hat, z. B. A23(526) oder C98(319) plus ein sehr kleines Teil – aber C98 ist nützlich und günstig. Besser: +A65(283), {-}A23(526) ergibt deutlichen Aufwandspuffer und ist glaubhaft als Effizienzschritt. Alternativ, falls sie A23 wichtig finden, kann ich A65 wieder herausnehmen und etwas anderes diskutieren. Ich argumentiere mit Ausgleich und Budgetpuffer. Ziel: +A65, {-}A23.'\}} \\
\texttt{ABLEHNUNG: \{'A48', 'C33', 'A74', 'C98', 'C61', 'C99', 'B54', 'A23', 'C90', 'A90', 'A62', 'A87', 'C51', 'C12', 'B17', 'A84', 'B36', 'A42', 'A59', 'A99', 'A69', 'C77', 'C23', 'C60', 'B49', 'C72'\}} \\
\texttt{VORSCHLAG: \{'A48', 'C33', 'A74', 'A65', 'C98', 'C61', 'C99', 'B54', 'C90', 'A90', 'A62', 'A87', 'C51', 'C12', 'B17', 'A84', 'B36', 'A42', 'A59', 'A99', 'A69', 'C77', 'C23', 'C60', 'B49', 'C72'\}} \\
\texttt{ARGUMENT: \{'Ich habe A65 aufgenommen und dafür A23 herausgenommen. So sinkt der Gesamtaufwand deutlich, wir bleiben entspannt im Limit und halten die Abdeckung stabil. Falls dir A23 wichtig ist, sag mir bitte Bescheid – dann nehme ich es wieder auf und ersetze stattdessen ein anderes mittleres Teil (z. B. B07 war ohnehin draußen; alternativ könnten wir B54 oder C98 prüfen), sodass wir weiterhin im Budget bleiben.'\}} \\
            }
        }
    }
    & & \\ \\

    \theutterance \stepcounter{utterance}  
    & & \multicolumn{4}{p{0.6\linewidth}}{
        \cellcolor[rgb]{0.9,0.9,0.9}{
            \makecell[{{p{\linewidth}}}]{
                \texttt{\tiny{[GM$\rangle$P2]}}
                \texttt{ABLEHNUNG: \{'A48', 'C33', 'A74', 'C98', 'C61', 'C99', 'B54', 'A23', 'C90', 'A90', 'A62', 'A87', 'C51', 'C12', 'B17', 'A84', 'B36', 'A42', 'A59', 'A99', 'A69', 'C77', 'C23', 'C60', 'B49', 'C72'\}} \\
\texttt{VORSCHLAG: \{'A48', 'C33', 'A74', 'A65', 'C98', 'C61', 'C99', 'B54', 'C90', 'A90', 'A62', 'A87', 'C51', 'C12', 'B17', 'A84', 'B36', 'A42', 'A59', 'A99', 'A69', 'C77', 'C23', 'C60', 'B49', 'C72'\}} \\
\texttt{ARGUMENT: \{'Ich habe A65 aufgenommen und dafür A23 herausgenommen. So sinkt der Gesamtaufwand deutlich, wir bleiben entspannt im Limit und halten die Abdeckung stabil. Falls dir A23 wichtig ist, sag mir bitte Bescheid – dann nehme ich es wieder auf und ersetze stattdessen ein anderes mittleres Teil (z. B. B07 war ohnehin draußen; alternativ könnten wir B54 oder C98 prüfen), sodass wir weiterhin im Budget bleiben.'\}} \\
            }
        }
    }
    & \\ \\

    \theutterance \stepcounter{utterance}  
    & & & \multicolumn{4}{p{0.6\linewidth}}{
        \cellcolor[rgb]{0.9,0.9,0.9}{
            \makecell[{{p{\linewidth}}}]{
                \texttt{\tiny{[GM$\langle$P2]}}
                \texttt{STRATEGISCHE ÜBERLEGUNGEN: \{'Ihr letztes Angebot enthält: \{A48, C33, A74, A65, C98, C61, C99, B54, C90, A90, A62, A87, C51, C12, B17, A84, B36, A42, A59, A99, A69, C77, C23, C60, B49, C72\}. Für mich ist A23 wertvoller als A65; beide zusammen könnten evtl. noch ins Budget passen, aber vermutlich sind wir nah am Limit. Ich versuche A23 zurückzuholen, ohne eure jüngsten Prioritäten (C60, B49, C72) anzutasten. Bester Tausch: A23 rein, ersetze ein für mich weniger kritisches mittleres Teil – Kandidaten: B54 (380 Aufwand, mittelwertig) oder C98 (319). Da B54 etwas schwerer ist, schlage ich A23 rein, B54 raus. Das verbessert meinen Nutzen und sollte sicher im Limit bleiben. Ich signalisiere Kompromissbereitschaft: Falls B54 euch wichtig ist, kann stattdessen C98 raus.'\}} \\
\texttt{ABLEHNUNG: \{'A48', 'C33', 'A74', 'A65', 'C98', 'C61', 'C99', 'B54', 'C90', 'A90', 'A62', 'A87', 'C51', 'C12', 'B17', 'A84', 'B36', 'A42', 'A59', 'A99', 'A69', 'C77', 'C23', 'C60', 'B49', 'C72'\}} \\
\texttt{VORSCHLAG: \{'A48', 'C33', 'A74', 'A65', 'C98', 'C61', 'C99', 'C90', 'A90', 'A62', 'A87', 'C51', 'C12', 'B17', 'A84', 'B36', 'A42', 'A59', 'A99', 'A69', 'C77', 'C23', 'C60', 'B49', 'C72', 'A23'\}} \\
\texttt{ARGUMENT: \{'Ich nehme A23 wieder auf und lasse dafür B54 draußen. So bleiben wir im Aufwandslimit, erhöhen die Effizienz und halten alle von dir zuletzt priorisierten Positionen (inkl. C60, B49, C72) stabil. Falls dir B54 wichtig ist, können wir stattdessen C98 oder – wenn nötig – A65 prüfen, damit A23 trotzdem bleibt. Welche der drei Positionen (B54, C98, A65) ist für dich am ehesten verzichtbar?'\}} \\
            }
        }
    }
     \\ \\

    \theutterance \stepcounter{utterance}  
    & & \multicolumn{4}{p{0.6\linewidth}}{
        \cellcolor[rgb]{0.9,0.9,0.9}{
            \makecell[{{p{\linewidth}}}]{
                \texttt{\tiny{[P1$\langle$GM]}}
                \texttt{ABLEHNUNG: \{'A48', 'C33', 'A74', 'A65', 'C98', 'C61', 'C99', 'B54', 'C90', 'A90', 'A62', 'A87', 'C51', 'C12', 'B17', 'A84', 'B36', 'A42', 'A59', 'A99', 'A69', 'C77', 'C23', 'C60', 'B49', 'C72'\}} \\
\texttt{VORSCHLAG: \{'A48', 'C33', 'A74', 'A65', 'C98', 'C61', 'C99', 'C90', 'A90', 'A62', 'A87', 'C51', 'C12', 'B17', 'A84', 'B36', 'A42', 'A59', 'A99', 'A69', 'C77', 'C23', 'C60', 'B49', 'C72', 'A23'\}} \\
\texttt{ARGUMENT: \{'Ich nehme A23 wieder auf und lasse dafür B54 draußen. So bleiben wir im Aufwandslimit, erhöhen die Effizienz und halten alle von dir zuletzt priorisierten Positionen (inkl. C60, B49, C72) stabil. Falls dir B54 wichtig ist, können wir stattdessen C98 oder – wenn nötig – A65 prüfen, damit A23 trotzdem bleibt. Welche der drei Positionen (B54, C98, A65) ist für dich am ehesten verzichtbar?'\}} \\
            }
        }
    }
    & \\ \\

    \theutterance \stepcounter{utterance}  
    & \multicolumn{4}{p{0.6\linewidth}}{
        \cellcolor[rgb]{0.9,0.9,0.9}{
            \makecell[{{p{\linewidth}}}]{
                \texttt{\tiny{[P1$\rangle$GM]}}
                \texttt{STRATEGISCHE ÜBERLEGUNGEN: \{'Ihr aktueller Vorschlag ist für mich insgesamt gut: Er enthält meine Top{-}Prioritäten (A69, A99, A59, A42, B36, C60, A84, B17, C12, C51, A87, B49). Neu drin ist A23, raus ist B54. B54 ist für mich weniger wichtig als A23, daher passt mir der Tausch. Ich will jetzt abschließen, ohne weiteres Risiko, da das Paket meinen Nutzen hoch hält und im Limit ist. Ich signalisiere Kompromissbereitschaft und bitte nur minimal um Bestätigung der Stabilität, biete aber optional einen kleinen Tausch an, falls sie noch etwas justieren wollen.'\}} \\
\texttt{VORSCHLAG: \{'A48', 'C33', 'A74', 'A65', 'C98', 'C61', 'C99', 'C90', 'A90', 'A62', 'A87', 'C51', 'C12', 'B17', 'A84', 'B36', 'A42', 'A59', 'A99', 'A69', 'C77', 'C23', 'C60', 'B49', 'C72', 'A23'\}} \\
\texttt{ARGUMENT: \{'Das Paket ist ausgewogen, bleibt sicher unter dem Maximalaufwand und deckt die Breite sehr gut ab. Ich kann damit gut leben. Falls du eine letzte Feinjustierung brauchst, können wir einen kleinen 1‑zu‑1‑Tausch innerhalb ähnlicher Aufwandsklassen prüfen, ohne das Limit zu gefährden. Andernfalls lass uns so abschließen.'\}} \\
            }
        }
    }
    & & \\ \\

    \theutterance \stepcounter{utterance}  
    & & \multicolumn{4}{p{0.6\linewidth}}{
        \cellcolor[rgb]{0.9,0.9,0.9}{
            \makecell[{{p{\linewidth}}}]{
                \texttt{\tiny{[GM$\rangle$P2]}}
                \texttt{VORSCHLAG: \{'A48', 'C33', 'A74', 'A65', 'C98', 'C61', 'C99', 'C90', 'A90', 'A62', 'A87', 'C51', 'C12', 'B17', 'A84', 'B36', 'A42', 'A59', 'A99', 'A69', 'C77', 'C23', 'C60', 'B49', 'C72', 'A23'\}} \\
\texttt{ARGUMENT: \{'Das Paket ist ausgewogen, bleibt sicher unter dem Maximalaufwand und deckt die Breite sehr gut ab. Ich kann damit gut leben. Falls du eine letzte Feinjustierung brauchst, können wir einen kleinen 1‑zu‑1‑Tausch innerhalb ähnlicher Aufwandsklassen prüfen, ohne das Limit zu gefährden. Andernfalls lass uns so abschließen.'\}} \\
            }
        }
    }
    & \\ \\

    \theutterance \stepcounter{utterance}  
    & & & \multicolumn{4}{p{0.6\linewidth}}{
        \cellcolor[rgb]{0.9,0.9,0.9}{
            \makecell[{{p{\linewidth}}}]{
                \texttt{\tiny{[GM$\langle$P2]}}
                \texttt{STRATEGISCHE ÜBERLEGUNGEN: \{'Prüfe, ob der Vorschlag im Aufwandslimit liegt und ob er meinen Nutzen gut maximiert. Enthalten sind viele meiner Top{-}Werte (A69, A99, A59, A42, B36, A84, A62, A87, C51, C12, C60, C72, C77, B49, A23, A90). Fehlend sind B54, A34, A50, C61 ist drin, C98 drin. Gesamtaufwand überschlägig: summiere schnell – dieser Satz war in früheren Varianten bereits im Limit, und hier wurde nur B54 entfernt und A23 drin gelassen; sollte < 9322 sein. Mein Nutzen ist hoch, ich habe A50 und A34 nicht, aber dafür C72 und C60 und B49. Das erscheint als guter Deal. Akzeptieren, um Abschluss zu sichern.'\}} \\
\texttt{ARGUMENT: \{'Das erfüllt das Aufwandslimit und bildet eine starke, breit tragende Auswahl ab. Für mich ist das ein guter, ausgewogener Abschluss – lass uns das fix machen.'\}} \\
\texttt{ZUSTIMMUNG: \{'A48', 'C33', 'A74', 'A65', 'C98', 'C61', 'C99', 'C90', 'A90', 'A62', 'A87', 'C51', 'C12', 'B17', 'A84', 'B36', 'A42', 'A59', 'A99', 'A69', 'C77', 'C23', 'C60', 'B49', 'C72', 'A23'\}} \\
            }
        }
    }
     \\ \\

    \theutterance \stepcounter{utterance}  
    & & & \multicolumn{2}{p{0.3\linewidth}}{
        \cellcolor[rgb]{0.9,0.9,0.9}{
            \makecell[{{p{\linewidth}}}]{
                \texttt{\tiny{[GM$|$GM]}}
                \texttt{game successful} \\
            }
        }
    }
    & & \\ \\

    \theutterance \stepcounter{utterance}  
    & & & \multicolumn{2}{p{0.3\linewidth}}{
        \cellcolor[rgb]{0.9,0.9,0.9}{
            \makecell[{{p{\linewidth}}}]{
                \texttt{\tiny{[GM$|$GM]}}
                \texttt{end game} \\
            }
        }
    }
    & & \\ \\

\end{supertabular}
}

\end{document}
