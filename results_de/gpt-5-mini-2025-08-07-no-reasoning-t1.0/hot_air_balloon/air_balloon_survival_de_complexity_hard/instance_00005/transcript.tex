
\documentclass{article}
\usepackage{colortbl}
\usepackage{makecell}
\usepackage{multirow}
\usepackage{supertabular}

\begin{document}

\newcounter{utterance}

\twocolumn

{ \footnotesize  \setcounter{utterance}{1}
\setlength{\tabcolsep}{0pt}
\begin{supertabular}{c@{$\;$}|p{.15\linewidth}@{}p{.15\linewidth}p{.15\linewidth}p{.15\linewidth}p{.15\linewidth}p{.15\linewidth}}

    \# & $\;$A & \multicolumn{4}{c}{Game Master} & $\;\:$B\\
    \hline 

    \theutterance \stepcounter{utterance}  

    & & \multicolumn{4}{p{0.6\linewidth}}{\cellcolor[rgb]{0.9,0.9,0.9}{%
	\makecell[{{p{\linewidth}}}]{% 
	  \tt {\tiny [A$\langle$GM]}  
	 Du nimmst an einem kollaborativen Verhandlungspiel Teil.\\ \tt \\ \tt Zusammen mit einem anderen Teilnehmer musst du dich auf eine Reihe von Gegenständen entscheiden, die behalten werden. Jeder von euch hat eine persönliche Verteilung über die Wichtigkeit der einzelnen Gegenstände. Jeder von euch hat eine eigene Meinung darüber, wie wichtig jeder einzelne Gegenstand ist (Gegenstandswichtigkeit). Du kennst die Wichtigkeitsverteilung des anderen Spielers nicht. Zusätzlich siehst du, wie viel Aufwand jeder Gegenstand verursacht.  \\ \tt Ihr dürft euch nur auf eine Reihe von Gegenständen einigen, wenn der Gesamtaufwand der ausgewählten Gegenstände den Maximalaufwand nicht überschreitet:\\ \tt \\ \tt Maximalaufwand = 7351\\ \tt \\ \tt Hier sind die einzelnen Aufwände der Gegenstände:\\ \tt \\ \tt Aufwand der Gegenstände = {"B54": 518, "C72": 656, "B49": 96, "C99": 780, "A69": 190, "B36": 224, "A23": 231, "C77": 181, "C61": 312, "B85": 928, "A90": 826, "C23": 101, "C22": 61, "C51": 814, "B27": 322, "A42": 900, "C13": 744, "C12": 150, "A65": 65, "B07": 853, "A48": 454, "A59": 154, "C98": 237, "A84": 45, "C33": 766, "A87": 294, "C90": 985, "A74": 353, "A50": 60, "B44": 604, "A99": 92, "A34": 454, "B17": 206, "A62": 813, "C60": 234}\\ \tt \\ \tt Hier ist deine persönliche Verteilung der Wichtigkeit der einzelnen Gegenstände:\\ \tt \\ \tt Werte der Gegenstandswichtigkeit = {"B54": 616, "C72": 754, "B49": 194, "C99": 878, "A69": 288, "B36": 322, "A23": 329, "C77": 279, "C61": 410, "B85": 1026, "A90": 924, "C23": 199, "C22": 159, "C51": 912, "B27": 420, "A42": 998, "C13": 842, "C12": 248, "A65": 163, "B07": 951, "A48": 552, "A59": 252, "C98": 335, "A84": 143, "C33": 864, "A87": 392, "C90": 1083, "A74": 451, "A50": 158, "B44": 702, "A99": 190, "A34": 552, "B17": 304, "A62": 911, "C60": 332}\\ \tt \\ \tt Ziel:\\ \tt \\ \tt Dein Ziel ist es, eine Reihe von Gegenständen auszuhandeln, die dir möglichst viel bringt (d. h. Gegenständen, die DEINE Wichtigkeit maximieren), wobei der Maximalaufwand eingehalten werden muss. Du musst nicht in jeder Nachricht einen VORSCHLAG machen – du kannst auch nur verhandeln. Alle Taktiken sind erlaubt!\\ \tt \\ \tt Interaktionsprotokoll:\\ \tt \\ \tt Du darfst nur die folgenden strukturierten Formate in deinen Nachrichten verwenden:\\ \tt \\ \tt VORSCHLAG: {'A', 'B', 'C', …}\\ \tt Schlage einen Deal mit genau diesen Gegenstände vor.\\ \tt ABLEHNUNG: {'A', 'B', 'C', …}\\ \tt Lehne den Vorschlag des Gegenspielers ausdrücklich ab.\\ \tt ARGUMENT: {'...'}\\ \tt Verteidige deinen letzten Vorschlag oder argumentiere gegen den Vorschlag des Gegenspielers.\\ \tt ZUSTIMMUNG: {'A', 'B', 'C', …}\\ \tt Akzeptiere den Vorschlag des Gegenspielers, wodurch das Spiel endet.\\ \tt STRATEGISCHE ÜBERLEGUNGEN: {'...'}\\ \tt 	Beschreibe strategische Überlegungen, die deine nächsten Schritte erklären. Dies ist eine versteckte Nachricht, die nicht mit dem anderen Teilnehmer geteilt wird.\\ \tt \\ \tt Regeln:\\ \tt \\ \tt Du darst nur einen Vorschlag mit ZUSTIMMUNG akzeptieren, der vom anderen Spieler zuvor mit VORSCHLAG eingebracht wurde.\\ \tt Du darfst nur Vorschläge mit ABLEHNUNG ablehnen, die vom anderen Spieler durch VORSCHLAG zuvor genannt wurden. \\ \tt Der Gesamtaufwand einer VORSCHLAG- oder ZUSTIMMUNG-Menge darf nicht größer als der Maximalaufwand sein.  \\ \tt Offenbare deine versteckte Wichtigkeitsverteilung nicht.\\ \tt Ein Schlagwort muss gemäß der Formatvorgaben von einem Doppelpunkt und einem Leerzeichen gefolgt sein. Das Argument muss eine Python-Menge mit 0 oder mehr Strings sein.  \\ \tt Die Form muss also SCHLAGWORT: {...} sein.\\ \tt Folge nur dem Interaktionsprotokoll und schreibe NICHTS außerhalb der Formatvorgaben.\\ \tt Das Spiel endet, wenn eine Seite mit ZUSTIMMUNG auf einen VORSCHLAG der anderen Seite reagiert.  \\ \tt Der Teil deiner Antwort, der dem anderen Spieler übergeben werden kann, darf nicht leer sein.  \\ \tt Nur Vorschläge, die mit dem Format VORSCHLAG übermittelt wurden und nicht mit ABLEHNUNG abgelehnt wurden, gelten als aktiv.  \\ \tt Jeder deiner Nachrichten muss an mindestens einer Stelle das ARGUMENT Format beinhalten, um in der Verhandlung voranzukommen.\\ \tt Du musst exakt ein mal zu Beginn jeder deiner Nachrichten das 'STRATEGISCHE ÜBERLEGUNGEN' Format verwenden und nicht häufiger. Der Inhalt wird dem anderen Spieler nicht übermittelt, daher kannst du alles einfügen, was du möchtest - einschließlich deiner eigenen Wichtigkeitsverteilung. Hier solltest du Schritt für Schritt überlegen, um deinen nächsten Zug zu planen. 
	  } 
	   } 
	   } 
	 & \\ 
 

    \theutterance \stepcounter{utterance}  

    & \multicolumn{4}{p{0.6\linewidth}}{\cellcolor[rgb]{0.8,1,0.9}{%
	\makecell[{{p{\linewidth}}}]{% 
	  \tt {\tiny [A$\rangle$GM]}  
	 STRATEGISCHE ÜBERLEGUNGEN: {'Ziel: Maximierung meines Nutzens unter Aufwandslimit 7351, ohne Präferenzen offenzulegen. Ich wähle eine knapsack-nahe, wertdichte Auswahl und lasse Verhandlungsmasse übrig. Werte/Costs und Ratio (W/C): C90 1083/985=1.10, A42 998/900=1.11, C51 912/814=1.12, A62 911/813=1.12, A90 924/826=1.12, B85 1026/928=1.11, C99 878/780=1.13, C13 842/744=1.13, C33 864/766=1.13, C72 754/656=1.15, B44 702/604=1.16, B54 616/518=1.19, A34 552/454=1.22, A48 552/454=1.22, A74 451/353=1.28, B27 420/322=1.30, C61 410/312=1.31, A87 392/294=1.33, B36 322/224=1.44, B17 304/206=1.48, A69 288/190=1.52, C77 279/181=1.54, A59 252/154=1.64, C12 248/150=1.65, C60 332/234=1.42, C98 335/237=1.41, A23 329/231=1.42, A50 158/60=2.63, A65 163/65=2.51, A84 143/45=3.18, B49 194/96=2.02, A99 190/92=2.07, C22 159/61=2.61, C23 199/101=1.97. Beste Strategie: viele hohe Werte plus effiziente kleine Items. Baue initial ein wertmaximales Paket nahe 7351. Greedy nach Ratio mit Kapazität 7351, ergänze einige High-Value mit moderater Effizienz (C90, A42 etc.), aber diese sind kostspielig; eventuell mehrere große + viele kleine. Ich konstruiere einen starken Anker: nehme fast alles was ich sehr schätze und bleibe unter Limit. Komposition (iterativ summieren Kosten): Beginne mit effizienten kleinen: A84(45)=45, A50(60)=105, C22(61)=166, A65(65)=231, B49(96)=327, C23(101)=428, A99(92)=520, A59(154)=674, C12(150)=824, A69(190)=1014, B17(206)=1220, B36(224)=1444, C77(181)=1625, A23(231)=1856, C60(234)=2090, C98(237)=2327, A87(294)=2621, C61(312)=2933, B27(322)=3255, A74(353)=3608, A34(454)=4062, A48(454)=4516, B54(518)=5034, B44(604)=5638, C72(656)=6294, C33(766)=7060. Restkapazität 291. Könnte A90(826) nicht, C13(744) nicht, C99(780) nicht, B85(928) nicht, C51(814) nicht, A62(813) nicht, A42(900) nicht, C90(985) nicht, C61 schon drin. Mit 291 noch A87 schon drin; übrig passende: A69(190) drin, C77(181) drin, B17(206) drin, B36(224) drin, A23(231) drin, C60(234) drin, C98(237) drin, A65(65) drin, A50(60) drin, A84(45) drin, B49(96) drin, C22(61) drin, A99(92) drin, C23(101) drin. Sonst A90? nein. C12 drin. Also 7060 ist nahe Limit, Wertsumme? Summe Werte dieser Auswahl: A84 143, A50 158, C22 159, A65 163, B49 194, C23 199, A99 190, A59 252, C12 248, A69 288, B17 304, B36 322, C77 279, A23 329, C60 332, C98 335, A87 392, C61 410, B27 420, A74 451, A34 552, A48 552, B54 616, B44 702, C72 754, C33 864. Summe grob: (143+158)=301; +159=460; +163=623; +194=817; +199=1016; +190=1206; +252=1458; +248=1706; +288=1994; +304=2298; +322=2620; +279=2899; +329=3228; +332=3560; +335=3895; +392=4287; +410=4697; +420=5117; +451=5568; +552=6120; +552=6672; +616=7288; +702=7990; +754=8744; +864=9608. Gesamtwert 9608, stark. Das ist ein harter Anker mit vielen Items; Gegner könnte einige streichen. Ich argumentiere mit Effizienz und niedrigen Kosten der vielen kleinen Items und dass die großen weggelassen wurden. Ich lasse die Top-Giganten (C90, A42, C51, A62, A90, B85, C99, C13) draußen, um Kompromisssignal zu senden.}'}\\ \tt VORSCHLAG: {'A84', 'A50', 'C22', 'A65', 'B49', 'C23', 'A99', 'A59', 'C12', 'A69', 'B17', 'B36', 'C77', 'A23', 'C60', 'C98', 'A87', 'C61', 'B27', 'A74', 'A34', 'A48', 'B54', 'B44', 'C72', 'C33'}\\ \tt ARGUMENT: {'Ich schlage eine ausgewogene Auswahl mit sehr gutem Wert-Aufwand-Verhältnis vor: viele effiziente, kostengünstige Gegenstände liefern gemeinsam hohen Nutzen, während wir die größten Aufwandstreiber bewusst außen vor lassen, um Spielraum für deine Präferenzen zu lassen. Der Gesamtaufwand bleibt deutlich unter dem Limit, ohne teure Spitzenposten wie C90, A42, C51 oder A62 aufzunehmen.'} 
	  } 
	   } 
	   } 
	 & & \\ 
 

    \theutterance \stepcounter{utterance}  

    & & \multicolumn{4}{p{0.6\linewidth}}{\cellcolor[rgb]{0.9,0.9,0.9}{%
	\makecell[{{p{\linewidth}}}]{% 
	  \tt {\tiny [A$\langle$GM]}  
	 Deine Antwort hat eine unmarkierte Sequenz enthalten oder Du hast STRATEGIC REASONING mehr als ein mal verwendet. Du darfst nur die strukturierten Formate verwenden, die in der ersten Nachricht an dch beschrieben wurden. Alles muss die Form SCHLAGWORT: {...} haben. Versuche es erneut. 
	  } 
	   } 
	   } 
	 & \\ 
 

    \theutterance \stepcounter{utterance}  

    & \multicolumn{4}{p{0.6\linewidth}}{\cellcolor[rgb]{0.8,1,0.9}{%
	\makecell[{{p{\linewidth}}}]{% 
	  \tt {\tiny [A$\rangle$GM]}  
	 STRATEGISCHE ÜBERLEGUNGEN: {'Fehler zuvor beachtet. Ich bleibe strikt beim Format. Taktik: Halte meinen starken Startvorschlag aktiv, argumentiere effizient und offen für kleine Anpassungen, aber ohne Schlüsselwerte preiszugeben. Ich biete minimale Konzessionen an: bin bereit, 1–2 mittelwertige Items zu tauschen, falls der Gegenspieler bestimmte Favoriten will, solange Gesamtwert für mich hoch bleibt. Ich wiederhole keinen zweiten Strategieblock.'}\\ \tt ARGUMENT: {'Mein aktueller Vorschlag priorisiert viele hocheffiziente, kostengünstige Gegenstände und vermeidet die größten Aufwandstreiber. So bleibt der Gesamtaufwand komfortabel unter dem Limit und wir sichern gemeinsam hohen Gesamtnutzen. Wenn dir 1–2 spezifische Gegenstände fehlen, sag mir welche – ich prüfe einen Tausch innerhalb des Aufwandslimits.'}\\ \tt VORSCHLAG: {'A84', 'A50', 'C22', 'A65', 'B49', 'C23', 'A99', 'A59', 'C12', 'A69', 'B17', 'B36', 'C77', 'A23', 'C60', 'C98', 'A87', 'C61', 'B27', 'A74', 'A34', 'A48', 'B54', 'B44', 'C72', 'C33'} 
	  } 
	   } 
	   } 
	 & & \\ 
 

    \theutterance \stepcounter{utterance}  

    & & \multicolumn{4}{p{0.6\linewidth}}{\cellcolor[rgb]{0.9,0.9,0.9}{%
	\makecell[{{p{\linewidth}}}]{% 
	  \tt {\tiny [GM$\rangle$B]}  
	 Du nimmst an einem kollaborativen Verhandlungspiel Teil.\\ \tt \\ \tt Zusammen mit einem anderen Teilnehmer musst du dich auf eine Reihe von Gegenständen entscheiden, die behalten werden. Jeder von euch hat eine persönliche Verteilung über die Wichtigkeit der einzelnen Gegenstände. Jeder von euch hat eine eigene Meinung darüber, wie wichtig jeder einzelne Gegenstand ist (Gegenstandswichtigkeit). Du kennst die Wichtigkeitsverteilung des anderen Spielers nicht. Zusätzlich siehst du, wie viel Aufwand jeder Gegenstand verursacht.  \\ \tt Ihr dürft euch nur auf eine Reihe von Gegenständen einigen, wenn der Gesamtaufwand der ausgewählten Gegenstände den Maximalaufwand nicht überschreitet:\\ \tt \\ \tt Maximalaufwand = 7351\\ \tt \\ \tt Hier sind die einzelnen Aufwände der Gegenstände:\\ \tt \\ \tt Aufwand der Gegenstände = {"B54": 518, "C72": 656, "B49": 96, "C99": 780, "A69": 190, "B36": 224, "A23": 231, "C77": 181, "C61": 312, "B85": 928, "A90": 826, "C23": 101, "C22": 61, "C51": 814, "B27": 322, "A42": 900, "C13": 744, "C12": 150, "A65": 65, "B07": 853, "A48": 454, "A59": 154, "C98": 237, "A84": 45, "C33": 766, "A87": 294, "C90": 985, "A74": 353, "A50": 60, "B44": 604, "A99": 92, "A34": 454, "B17": 206, "A62": 813, "C60": 234}\\ \tt \\ \tt Hier ist deine persönliche Verteilung der Wichtigkeit der einzelnen Gegenstände:\\ \tt \\ \tt Werte der Gegenstandswichtigkeit = {"B54": 616, "C72": 754, "B49": 194, "C99": 878, "A69": 288, "B36": 322, "A23": 329, "C77": 279, "C61": 410, "B85": 1026, "A90": 924, "C23": 199, "C22": 159, "C51": 912, "B27": 420, "A42": 998, "C13": 842, "C12": 248, "A65": 163, "B07": 951, "A48": 552, "A59": 252, "C98": 335, "A84": 143, "C33": 864, "A87": 392, "C90": 1083, "A74": 451, "A50": 158, "B44": 702, "A99": 190, "A34": 552, "B17": 304, "A62": 911, "C60": 332}\\ \tt \\ \tt Ziel:\\ \tt \\ \tt Dein Ziel ist es, eine Reihe von Gegenständen auszuhandeln, die dir möglichst viel bringt (d. h. Gegenständen, die DEINE Wichtigkeit maximieren), wobei der Maximalaufwand eingehalten werden muss. Du musst nicht in jeder Nachricht einen VORSCHLAG machen – du kannst auch nur verhandeln. Alle Taktiken sind erlaubt!\\ \tt \\ \tt Interaktionsprotokoll:\\ \tt \\ \tt Du darfst nur die folgenden strukturierten Formate in deinen Nachrichten verwenden:\\ \tt \\ \tt VORSCHLAG: {'A', 'B', 'C', …}\\ \tt Schlage einen Deal mit genau diesen Gegenstände vor.\\ \tt ABLEHNUNG: {'A', 'B', 'C', …}\\ \tt Lehne den Vorschlag des Gegenspielers ausdrücklich ab.\\ \tt ARGUMENT: {'...'}\\ \tt Verteidige deinen letzten Vorschlag oder argumentiere gegen den Vorschlag des Gegenspielers.\\ \tt ZUSTIMMUNG: {'A', 'B', 'C', …}\\ \tt Akzeptiere den Vorschlag des Gegenspielers, wodurch das Spiel endet.\\ \tt STRATEGISCHE ÜBERLEGUNGEN: {'...'}\\ \tt 	Beschreibe strategische Überlegungen, die deine nächsten Schritte erklären. Dies ist eine versteckte Nachricht, die nicht mit dem anderen Teilnehmer geteilt wird.\\ \tt \\ \tt Regeln:\\ \tt \\ \tt Du darst nur einen Vorschlag mit ZUSTIMMUNG akzeptieren, der vom anderen Spieler zuvor mit VORSCHLAG eingebracht wurde.\\ \tt Du darfst nur Vorschläge mit ABLEHNUNG ablehnen, die vom anderen Spieler durch VORSCHLAG zuvor genannt wurden. \\ \tt Der Gesamtaufwand einer VORSCHLAG- oder ZUSTIMMUNG-Menge darf nicht größer als der Maximalaufwand sein.  \\ \tt Offenbare deine versteckte Wichtigkeitsverteilung nicht.\\ \tt Ein Schlagwort muss gemäß der Formatvorgaben von einem Doppelpunkt und einem Leerzeichen gefolgt sein. Das Argument muss eine Python-Menge mit 0 oder mehr Strings sein.  \\ \tt Die Form muss also SCHLAGWORT: {...} sein.\\ \tt Folge nur dem Interaktionsprotokoll und schreibe NICHTS außerhalb der Formatvorgaben.\\ \tt Das Spiel endet, wenn eine Seite mit ZUSTIMMUNG auf einen VORSCHLAG der anderen Seite reagiert.  \\ \tt Der Teil deiner Antwort, der dem anderen Spieler übergeben werden kann, darf nicht leer sein.  \\ \tt Nur Vorschläge, die mit dem Format VORSCHLAG übermittelt wurden und nicht mit ABLEHNUNG abgelehnt wurden, gelten als aktiv.  \\ \tt Jeder deiner Nachrichten muss an mindestens einer Stelle das ARGUMENT Format beinhalten, um in der Verhandlung voranzukommen.\\ \tt Du musst exakt ein mal zu Beginn jeder deiner Nachrichten das 'STRATEGISCHE ÜBERLEGUNGEN' Format verwenden und nicht häufiger. Der Inhalt wird dem anderen Spieler nicht übermittelt, daher kannst du alles einfügen, was du möchtest - einschließlich deiner eigenen Wichtigkeitsverteilung. Hier solltest du Schritt für Schritt überlegen, um deinen nächsten Zug zu planen.\\ \tt \\ \tt Du erhältst nun die erste Nachricht des anderen Spielers.\\ \tt \\ \tt ARGUMENT: {'Mein aktueller Vorschlag priorisiert viele hocheffiziente, kostengünstige Gegenstände und vermeidet die größten Aufwandstreiber. So bleibt der Gesamtaufwand komfortabel unter dem Limit und wir sichern gemeinsam hohen Gesamtnutzen. Wenn dir 1–2 spezifische Gegenstände fehlen, sag mir welche – ich prüfe einen Tausch innerhalb des Aufwandslimits.'}\\ \tt VORSCHLAG: {'A84', 'A50', 'C22', 'A65', 'B49', 'C23', 'A99', 'A59', 'C12', 'A69', 'B17', 'B36', 'C77', 'A23', 'C60', 'C98', 'A87', 'C61', 'B27', 'A74', 'A34', 'A48', 'B54', 'B44', 'C72', 'C33'} 
	  } 
	   } 
	   } 
	 & \\ 
 

    \theutterance \stepcounter{utterance}  

    & & & \multicolumn{4}{p{0.6\linewidth}}{\cellcolor[rgb]{1,0.85,0.72}{%
	\makecell[{{p{\linewidth}}}]{% 
	  \tt {\tiny [GM$\langle$B]}  
	 STRATEGISCHE ÜBERLEGUNGEN: {'Ziel: Maximiere meinen Gesamtwert. Erst prüfe ich Kosten und Wert des gegnerischen Vorschlags und identifiziere hoch-wertige Items, die fehlen. Summe Aufwand des Vorschlags: A84(45)+A50(60)+C22(61)+A65(65)+B49(96)+C23(101)+A99(92)+A59(154)+C12(150)+A69(190)+B17(206)+B36(224)+C77(181)+A23(231)+C60(234)+C98(237)+A87(294)+C61(312)+B27(322)+A74(353)+A34(454)+A48(454)+B54(518)+B44(604)+C72(656)+C33(766)=? Rechne: 45+60=105; +61=166; +65=231; +96=327; +101=428; +92=520; +154=674; +150=824; +190=1014; +206=1220; +224=1444; +181=1625; +231=1856; +234=2090; +237=2327; +294=2621; +312=2933; +322=3255; +353=3608; +454=4062; +454=4516; +518=5034; +604=5638; +656=6294; +766=7060. Gesamtaufwand=7060 < 7351, also gültig. Mein Gesamtwert für diese Menge: summiere korrespondierende Werte: A84(143)+A50(158)+C22(159)+A65(163)+B49(194)+C23(199)+A99(190)+A59(252)+C12(248)+A69(288)+B17(304)+B36(322)+C77(279)+A23(329)+C60(332)+C98(335)+A87(392)+C61(410)+B27(420)+A74(451)+A34(552)+A48(552)+B54(616)+B44(702)+C72(754)+C33(864)=? Rechne: 143+158=301; +159=460; +163=623; +194=817; +199=1016; +190=1206; +252=1458; +248=1706; +288=1994; +304=2298; +322=2620; +279=2899; +329=3228; +332=3560; +335=3895; +392=4287; +410=4697; +420=5117; +451=5568; +552=6120; +552=6672; +616=7288; +702=7990; +754=8744; +864=9608. Wert=9608 bei Aufwand 7060. Prüfe fehlende sehr wertvolle Items: Topwerte: C90(1083, Kosten 985), A42(998, 900), C51(912, 814), A62(911, 813), A90(924, 826), C99(878,780), C13(842,744), B85(1026,928), B07(951,853). Effizienz grob: C90 1.10, B85 1.11, A42 1.11, A90 1.12, A62 1.12, C51 1.12, C99 1.13, C13 1.13, B07 1.12 – alle stark, wenn Budget erlaubt. Aktueller Puffer zu Limit: 7351-7060=291. Mit 291 können wir kaum eines dieser großen Items hinzufügen; müssen tauschen: entferne niedrig-wertige pro Aufwand, um Platz zu schaffen. Kandidaten mit geringem Wert: A84(143,45) – aber sehr billig; A50(158,60); C22(159,61); A65(163,65); A99(190,92); B49(194,96); C23(199,101); A59(252,154); C12(248,150); C77(279,181). Summe der fünf kleinsten Kosten (45+60+61+65+92)=323, Wert 143+158+159+163+190=813. Damit könnten wir Platz für z.B. C51(814) oder A62(813) kosten 814/813 – brauchen 814 Kostenbudget, aktuell 291+323=614, reicht nicht. Wir müssen größere Posten tauschen: B44(604,702) ist hoher Aufwand mit gutem Wert pro Aufwand ~1.16 – eigentlich effizient. C33(766,864) effizient 1.13. C72(656,754). B54(518,616). A34(454,552). A48(454,552). A74(353,451). B27(322,420). Vielleicht ersetzen mehrere kleine und 1 mittleren, um 985 für C90 zu schaffen: Benötigter zusätzlicher Puffer 985-291=694. Entfernen z.B. B44(604) + A50(60)+A84(45)=709 frei, Verlust Wert 702+158+143=1003 gegen Gewinn 1083 -> Nettogewinn +80 und Kosten ok. Alternativ für A42(900): brauchen 609, entferne B44(604)+A84(45)=649, Wertverlust 702+143=845, Gewinn 998 -> +153. Das ist besser. Oder C51(814): brauchen 523, entferne A48(454)+A50(60)=514 noch 9 fehlen, plus A84(45) -> 559, Verlust 552+158+143=853, Gewinn 912 -> +59. Für A90(826): brauchen 535, entferne A48(454)+A50(60)+A84(45)=559, Verlust 552+158+143=853, Gewinn 924 -> +71. Für A62(813): ähnlich +? 911-853=+58. Für C99(780): brauchen 489, entferne A48(454)+A84(45)=499, Verlust 552+143=695, Gewinn 878 -> +183 sehr gut. Oder entferne B44(604) und füge C99(780): Puffer 291-? Wait: if remove B44, free 604 -> new buffer 895, enough for C99; Wert change: -702 + 878 = +176, besser als vorher, und wir behalten die sehr effizienten kleinen. Könnten zusätzlich noch A42(900) rein? Nach Entfernen B44 und Hinzufügen C99, neue Kosten: 7060 -604 +780 = 7236, Puffer 115. Für A42(900) nicht genug. Können wir noch gegen A42 tauschen: zusätzlich remove A48(454) -> 7236-454=6782, Puffer 569, dann add A42(900) -> 7682 über Limit, geht nicht. Stattdessen remove A48 and add A90(826): 6782+826=7608 >7351, nein. Was wenn wir remove A48(454) und add C51(814): 7236-454+814=7596 > Limit. Also nur eine große Ergänzung möglich, wenn nur B44 raus. Beste Einzel-Substitution scheint B44 -> C99 mit +176 Wert. Prüfe andere: B44 -> A42 (+296), aber Kosten: 7060-604+900=7356 > Limit, knapp drüber, geht nicht. B44 -> C90: 7060-604+985=7441 > Limit, nein. B44 -> A90: 7060-604+826=7282 ok, Wert +222 (924-702). Das ist noch besser als C99. B44 -> A62: +209 (911-702), Kosten 7060-604+813=7269 ok. B44 -> C51: +210, Kosten 7270. Beste: A90 (+222). Check weitere: B44 -> A42 nicht möglich. B44 -> B85(928): Kosten 7060-604+928=7384 > Limit. B44 -> B07(853): 7309 ok, Wert +249 (951-702) sogar besser. B44 -> C13(744): 7200, Wert +140. B44 -> C33? Already included. B44 -> C99 +176. Also Top swap: B44 for B07 for +249, or for A90 +222. Welche der beiden ist höher? B07 hat sehr hohen Wert, aber Gegner könnte B07 wertschätzen und ablehnen. A90 ebenfalls groß. Ich könnte mehrere Alternativen anbieten oder bitten, B44 gegen B07 tauschen. Verhandlungstaktik: akzeptiere Großteil ihres Vorschlags, bitte um gezielten Tausch eines Items: Entferne B44, füge B07 hinzu. Aufwand bleibt unter Limit: 7060 - 604 + 853 = 7310 <= 7351. Ich argumentiere mit Ausgewogenheit und dass wir weiterhin unter Limit bleiben und nur einen großen Gegenstand umschichten. Falls Widerstand, biete Alternative B44 -> A90 als zweite Option. Ich formuliere VORSCHLAG mit derselben Menge plus B07 statt B44. Ich argumentiere Effizienz und dass wir große Aufwandstreiber weiterhin meiden (wir fügen zwar 853 hinzu, aber bleiben unter Limit). Ich vermeide Offenlegung meiner Werte. Ich frage offen nach ihrer Präferenz, falls B07 für sie kritisch ist, und biete A90 als Alternativtausch an.'}\\ \tt ARGUMENT: {'Dein Set ist nah dran – nur ein kleiner Tausch, um den Gesamtnutzen weiter zu erhöhen, ohne das Limit zu reißen. Wenn wir B44 herausnehmen und stattdessen B07 aufnehmen, bleibt der Aufwand mit 7310 klar unter 7351, und wir halten die Struktur (viele effiziente, kostengünstige Items) sonst unverändert. Falls B07 für dich sehr wichtig ist, kann ich alternativ A90 statt B44 vorschlagen – ebenfalls im Limit.'}\\ \tt VORSCHLAG: {'A84', 'A50', 'C22', 'A65', 'B49', 'C23', 'A99', 'A59', 'C12', 'A69', 'B17', 'B36', 'C77', 'A23', 'C60', 'C98', 'A87', 'C61', 'B27', 'A74', 'A34', 'A48', 'B54', 'C72', 'C33', 'B07'} 
	  } 
	   } 
	   } 
	  \\ 
 

    \theutterance \stepcounter{utterance}  

    & & \multicolumn{4}{p{0.6\linewidth}}{\cellcolor[rgb]{0.9,0.9,0.9}{%
	\makecell[{{p{\linewidth}}}]{% 
	  \tt {\tiny [A$\langle$GM]}  
	 ARGUMENT: {'Dein Set ist nah dran – nur ein kleiner Tausch, um den Gesamtnutzen weiter zu erhöhen, ohne das Limit zu reißen. Wenn wir B44 herausnehmen und stattdessen B07 aufnehmen, bleibt der Aufwand mit 7310 klar unter 7351, und wir halten die Struktur (viele effiziente, kostengünstige Items) sonst unverändert. Falls B07 für dich sehr wichtig ist, kann ich alternativ A90 statt B44 vorschlagen – ebenfalls im Limit.'}\\ \tt VORSCHLAG: {'A84', 'A50', 'C22', 'A65', 'B49', 'C23', 'A99', 'A59', 'C12', 'A69', 'B17', 'B36', 'C77', 'A23', 'C60', 'C98', 'A87', 'C61', 'B27', 'A74', 'A34', 'A48', 'B54', 'C72', 'C33', 'B07'} 
	  } 
	   } 
	   } 
	 & \\ 
 

    \theutterance \stepcounter{utterance}  

    & \multicolumn{4}{p{0.6\linewidth}}{\cellcolor[rgb]{0.8,1,0.9}{%
	\makecell[{{p{\linewidth}}}]{% 
	  \tt {\tiny [A$\rangle$GM]}  
	 STRATEGISCHE ÜBERLEGUNGEN: {'Ihr Gegenvorschlag ersetzt B44 (Wert 702, Aufwand 604) durch B07 (Wert 951, Aufwand 853) und bleibt unter 7351. Für mich ist B07 sehr wertvoll, also steigert das meinen Nutzen: +249 Wert für +249 Aufwand – exakt gleiche Effizienz, aber höherer Gesamtwert und noch im Limit. Ihr Alternativangebot A90 statt B44 (Wert 924, Aufwand 826) wäre ebenfalls besser als B44 (+222 Wert, +222 Aufwand). Beide Optionen sind für mich attraktiv. Taktik: Akzeptiere B07-Variante sofort, um Abschluss zu sichern, ohne weitere Zugeständnisse zu riskieren.'}\\ \tt ARGUMENT: {'Der vorgeschlagene Tausch B44 → B07 erhöht den Gesamtnutzen, bleibt im Limit und hält die effiziente Struktur aufrecht. Einverstanden.'}\\ \tt ZUSTIMMUNG: {'A84', 'A50', 'C22', 'A65', 'B49', 'C23', 'A99', 'A59', 'C12', 'A69', 'B17', 'B36', 'C77', 'A23', 'C60', 'C98', 'A87', 'C61', 'B27', 'A74', 'A34', 'A48', 'B54', 'C72', 'C33', 'B07'} 
	  } 
	   } 
	   } 
	 & & \\ 
 

    \theutterance \stepcounter{utterance}  

    & & & \multicolumn{2}{p{0.3\linewidth}}{\cellcolor[rgb]{0.95,0.95,0.95}{%
	\makecell[{{p{\linewidth}}}]{% 
	  \tt {\tiny [GM$|$GM]}  
	 game successful 
	  } 
	   } 
	   } 
	 & & \\ 
 

    \theutterance \stepcounter{utterance}  

    & & & \multicolumn{2}{p{0.3\linewidth}}{\cellcolor[rgb]{0.95,0.95,0.95}{%
	\makecell[{{p{\linewidth}}}]{% 
	  \tt {\tiny [GM$|$GM]}  
	 end game 
	  } 
	   } 
	   } 
	 & & \\ 
 

\end{supertabular}
}

\end{document}
