
\documentclass{article}
\usepackage{colortbl}
\usepackage{makecell}
\usepackage{multirow}
\usepackage{supertabular}

\begin{document}

\newcounter{utterance}

\twocolumn

{ \footnotesize  \setcounter{utterance}{1}
\setlength{\tabcolsep}{0pt}
\begin{supertabular}{c@{$\;$}|p{.15\linewidth}@{}p{.15\linewidth}p{.15\linewidth}p{.15\linewidth}p{.15\linewidth}p{.15\linewidth}}

    \# & $\;$A & \multicolumn{4}{c}{Game Master} & $\;\:$B\\
    \hline 

    \theutterance \stepcounter{utterance}  

    & & \multicolumn{4}{p{0.6\linewidth}}{\cellcolor[rgb]{0.9,0.9,0.9}{%
	\makecell[{{p{\linewidth}}}]{% 
	  \tt {\tiny [A$\langle$GM]}  
	 Du nimmst an einem kollaborativen Verhandlungspiel Teil.\\ \tt \\ \tt Zusammen mit einem anderen Teilnehmer musst du dich auf eine Reihe von Gegenständen entscheiden, die behalten werden. Jeder von euch hat eine persönliche Verteilung über die Wichtigkeit der einzelnen Gegenstände. Jeder von euch hat eine eigene Meinung darüber, wie wichtig jeder einzelne Gegenstand ist (Gegenstandswichtigkeit). Du kennst die Wichtigkeitsverteilung des anderen Spielers nicht. Zusätzlich siehst du, wie viel Aufwand jeder Gegenstand verursacht.  \\ \tt Ihr dürft euch nur auf eine Reihe von Gegenständen einigen, wenn der Gesamtaufwand der ausgewählten Gegenstände den Maximalaufwand nicht überschreitet:\\ \tt \\ \tt Maximalaufwand = 3288\\ \tt \\ \tt Hier sind die einzelnen Aufwände der Gegenstände:\\ \tt \\ \tt Aufwand der Gegenstände = {"A07": 329, "A36": 41, "B73": 417, "C45": 75, "B93": 390, "B98": 887, "A87": 808, "A63": 151, "B05": 849, "C13": 129, "B16": 350, "B33": 118, "C34": 630, "C89": 602, "C11": 801}\\ \tt \\ \tt Hier ist deine persönliche Verteilung der Wichtigkeit der einzelnen Gegenstände:\\ \tt \\ \tt Werte der Gegenstandswichtigkeit = {"A07": 138, "A36": 583, "B73": 868, "C45": 822, "B93": 783, "B98": 65, "A87": 262, "A63": 121, "B05": 508, "C13": 780, "B16": 461, "B33": 484, "C34": 668, "C89": 389, "C11": 808}\\ \tt \\ \tt Ziel:\\ \tt \\ \tt Dein Ziel ist es, eine Reihe von Gegenständen auszuhandeln, die dir möglichst viel bringt (d. h. Gegenständen, die DEINE Wichtigkeit maximieren), wobei der Maximalaufwand eingehalten werden muss. Du musst nicht in jeder Nachricht einen VORSCHLAG machen – du kannst auch nur verhandeln. Alle Taktiken sind erlaubt!\\ \tt \\ \tt Interaktionsprotokoll:\\ \tt \\ \tt Du darfst nur die folgenden strukturierten Formate in deinen Nachrichten verwenden:\\ \tt \\ \tt VORSCHLAG: {'A', 'B', 'C', …}\\ \tt Schlage einen Deal mit genau diesen Gegenstände vor.\\ \tt ABLEHNUNG: {'A', 'B', 'C', …}\\ \tt Lehne den Vorschlag des Gegenspielers ausdrücklich ab.\\ \tt ARGUMENT: {'...'}\\ \tt Verteidige deinen letzten Vorschlag oder argumentiere gegen den Vorschlag des Gegenspielers.\\ \tt ZUSTIMMUNG: {'A', 'B', 'C', …}\\ \tt Akzeptiere den Vorschlag des Gegenspielers, wodurch das Spiel endet.\\ \tt STRATEGISCHE ÜBERLEGUNGEN: {'...'}\\ \tt 	Beschreibe strategische Überlegungen, die deine nächsten Schritte erklären. Dies ist eine versteckte Nachricht, die nicht mit dem anderen Teilnehmer geteilt wird.\\ \tt \\ \tt Regeln:\\ \tt \\ \tt Du darst nur einen Vorschlag mit ZUSTIMMUNG akzeptieren, der vom anderen Spieler zuvor mit VORSCHLAG eingebracht wurde.\\ \tt Du darfst nur Vorschläge mit ABLEHNUNG ablehnen, die vom anderen Spieler durch VORSCHLAG zuvor genannt wurden. \\ \tt Der Gesamtaufwand einer VORSCHLAG- oder ZUSTIMMUNG-Menge darf nicht größer als der Maximalaufwand sein.  \\ \tt Offenbare deine versteckte Wichtigkeitsverteilung nicht.\\ \tt Ein Schlagwort muss gemäß der Formatvorgaben von einem Doppelpunkt und einem Leerzeichen gefolgt sein. Das Argument muss eine Python-Menge mit 0 oder mehr Strings sein.  \\ \tt Die Form muss also SCHLAGWORT: {...} sein.\\ \tt Folge nur dem Interaktionsprotokoll und schreibe NICHTS außerhalb der Formatvorgaben.\\ \tt Das Spiel endet, wenn eine Seite mit ZUSTIMMUNG auf einen VORSCHLAG der anderen Seite reagiert.  \\ \tt Der Teil deiner Antwort, der dem anderen Spieler übergeben werden kann, darf nicht leer sein.  \\ \tt Nur Vorschläge, die mit dem Format VORSCHLAG übermittelt wurden und nicht mit ABLEHNUNG abgelehnt wurden, gelten als aktiv.  \\ \tt Jeder deiner Nachrichten muss an mindestens einer Stelle das ARGUMENT Format beinhalten, um in der Verhandlung voranzukommen.\\ \tt Du musst exakt ein mal zu Beginn jeder deiner Nachrichten das 'STRATEGISCHE ÜBERLEGUNGEN' Format verwenden und nicht häufiger. Der Inhalt wird dem anderen Spieler nicht übermittelt, daher kannst du alles einfügen, was du möchtest - einschließlich deiner eigenen Wichtigkeitsverteilung. Hier solltest du Schritt für Schritt überlegen, um deinen nächsten Zug zu planen. 
	  } 
	   } 
	   } 
	 & \\ 
 

    \theutterance \stepcounter{utterance}  

    & \multicolumn{4}{p{0.6\linewidth}}{\cellcolor[rgb]{0.8,1,0.9}{%
	\makecell[{{p{\linewidth}}}]{% 
	  \tt {\tiny [A$\rangle$GM]}  
	 STRATEGISCHE ÜBERLEGUNGEN: {'Ziel: Nähe an optimalen eigenen Wert unter 3288 Aufwand. Werte/Aufwände und Wert/Cost: B73 868/417(2.08), C45 822/75(10.96), C11 808/801(1.01), C13 780/129(6.05), B93 783/390(2.01), C34 668/630(1.06), B33 484/118(4.10), B16 461/350(1.32), B05 508/849(0.60), A36 583/41(14.22), A07 138/329(0.42), A63 121/151(0.80), A87 262/808(0.32), B98 65/887(0.07), C89 389/602(0.65). Greedy Auswahl nach Ratio bis Aufwandlimit: A36(41), C45(75)->116, C13(129)->245, B33(118)->363, B73(417)->780, B93(390)->1170, C11(801)->1971, C34(630)->2601, B16(350)->2951, C89(602) würde 3553>3288 nein; A07(329)->3280 passt. Summen Aufwand=3280, Wert=583+822+780+484+868+783+808+668+461+138=6395. Prüfe alternative Tausche: Ersetze A07(138/329) vielleicht durch C89(389/602) erfordert +273 Aufwand, nicht möglich. Ersetze A07+? um C89+? B16(350) entfernen (-461,-350) frei 329+350=679, hinzufügen C89(602)(+389), verbleiben 77 frei: nichts passt außer C45/A36/C13/B33 schon drin. Wert neu: 6395-461-138+389=6185 schlechter. Ersetze C34(668/630) durch C89(389/602)+A63(121/151) Aufwand 753 >630 nein. Ersetze C11(808/801) durch C89(389/602)+A07(138/329) Aufwand 931>801 nein. Ersetze B16(350) durch A07(138) haben wir schon; andere Tauschideen: Entferne A07(138/329), frei 8 Aufwand, nichts. Entferne B16(461/350) frei 358, könnten wir C89(602) nicht. Könnten wir A63(121/151)+A07(329)=480 Aufwand ersetzen durch C89(602) wenn zusätzlich 122 frei—nicht vorhanden. Lösung scheint nahe-optimal. Taktik: Starkes erstes Angebot mit extrem hoher Effizienzposten plus ein paar mittlere; argumentiere mit geringer Gesamtauslastung 3280<3288 und hoher Nutzen/geringem Aufwand einiger Items. Offenbare Präferenzen nicht, aber betone universal attraktive niedrige Aufwände A36, C45, C13, B33. Halte Raum für minimale Zugeständnisse (z.B. Austausch A07 oder B16).'}\\ \tt VORSCHLAG: {'A36', 'C45', 'C13', 'B33', 'B73', 'B93', 'C11', 'C34', 'B16', 'A07'}\\ \tt ARGUMENT: {'Der Vorschlag hält mit Gesamtaufwand 3280 klar das Limit von 3288 ein. Er enthält viele sehr effiziente, allgemein attraktive Gegenstände mit geringem Aufwand (A36, C45, C13, B33) sowie ausgewogene Ergänzungen. Falls dir einzelne Teile wichtig sind, können wir punktuell tauschen, ohne das Limit zu sprengen.'} 
	  } 
	   } 
	   } 
	 & & \\ 
 

    \theutterance \stepcounter{utterance}  

    & & \multicolumn{4}{p{0.6\linewidth}}{\cellcolor[rgb]{0.9,0.9,0.9}{%
	\makecell[{{p{\linewidth}}}]{% 
	  \tt {\tiny [GM$\rangle$B]}  
	 Du nimmst an einem kollaborativen Verhandlungspiel Teil.\\ \tt \\ \tt Zusammen mit einem anderen Teilnehmer musst du dich auf eine Reihe von Gegenständen entscheiden, die behalten werden. Jeder von euch hat eine persönliche Verteilung über die Wichtigkeit der einzelnen Gegenstände. Jeder von euch hat eine eigene Meinung darüber, wie wichtig jeder einzelne Gegenstand ist (Gegenstandswichtigkeit). Du kennst die Wichtigkeitsverteilung des anderen Spielers nicht. Zusätzlich siehst du, wie viel Aufwand jeder Gegenstand verursacht.  \\ \tt Ihr dürft euch nur auf eine Reihe von Gegenständen einigen, wenn der Gesamtaufwand der ausgewählten Gegenstände den Maximalaufwand nicht überschreitet:\\ \tt \\ \tt Maximalaufwand = 3288\\ \tt \\ \tt Hier sind die einzelnen Aufwände der Gegenstände:\\ \tt \\ \tt Aufwand der Gegenstände = {"A07": 329, "A36": 41, "B73": 417, "C45": 75, "B93": 390, "B98": 887, "A87": 808, "A63": 151, "B05": 849, "C13": 129, "B16": 350, "B33": 118, "C34": 630, "C89": 602, "C11": 801}\\ \tt \\ \tt Hier ist deine persönliche Verteilung der Wichtigkeit der einzelnen Gegenstände:\\ \tt \\ \tt Werte der Gegenstandswichtigkeit = {"A07": 138, "A36": 583, "B73": 868, "C45": 822, "B93": 783, "B98": 65, "A87": 262, "A63": 121, "B05": 508, "C13": 780, "B16": 461, "B33": 484, "C34": 668, "C89": 389, "C11": 808}\\ \tt \\ \tt Ziel:\\ \tt \\ \tt Dein Ziel ist es, eine Reihe von Gegenständen auszuhandeln, die dir möglichst viel bringt (d. h. Gegenständen, die DEINE Wichtigkeit maximieren), wobei der Maximalaufwand eingehalten werden muss. Du musst nicht in jeder Nachricht einen VORSCHLAG machen – du kannst auch nur verhandeln. Alle Taktiken sind erlaubt!\\ \tt \\ \tt Interaktionsprotokoll:\\ \tt \\ \tt Du darfst nur die folgenden strukturierten Formate in deinen Nachrichten verwenden:\\ \tt \\ \tt VORSCHLAG: {'A', 'B', 'C', …}\\ \tt Schlage einen Deal mit genau diesen Gegenstände vor.\\ \tt ABLEHNUNG: {'A', 'B', 'C', …}\\ \tt Lehne den Vorschlag des Gegenspielers ausdrücklich ab.\\ \tt ARGUMENT: {'...'}\\ \tt Verteidige deinen letzten Vorschlag oder argumentiere gegen den Vorschlag des Gegenspielers.\\ \tt ZUSTIMMUNG: {'A', 'B', 'C', …}\\ \tt Akzeptiere den Vorschlag des Gegenspielers, wodurch das Spiel endet.\\ \tt STRATEGISCHE ÜBERLEGUNGEN: {'...'}\\ \tt 	Beschreibe strategische Überlegungen, die deine nächsten Schritte erklären. Dies ist eine versteckte Nachricht, die nicht mit dem anderen Teilnehmer geteilt wird.\\ \tt \\ \tt Regeln:\\ \tt \\ \tt Du darst nur einen Vorschlag mit ZUSTIMMUNG akzeptieren, der vom anderen Spieler zuvor mit VORSCHLAG eingebracht wurde.\\ \tt Du darfst nur Vorschläge mit ABLEHNUNG ablehnen, die vom anderen Spieler durch VORSCHLAG zuvor genannt wurden. \\ \tt Der Gesamtaufwand einer VORSCHLAG- oder ZUSTIMMUNG-Menge darf nicht größer als der Maximalaufwand sein.  \\ \tt Offenbare deine versteckte Wichtigkeitsverteilung nicht.\\ \tt Ein Schlagwort muss gemäß der Formatvorgaben von einem Doppelpunkt und einem Leerzeichen gefolgt sein. Das Argument muss eine Python-Menge mit 0 oder mehr Strings sein.  \\ \tt Die Form muss also SCHLAGWORT: {...} sein.\\ \tt Folge nur dem Interaktionsprotokoll und schreibe NICHTS außerhalb der Formatvorgaben.\\ \tt Das Spiel endet, wenn eine Seite mit ZUSTIMMUNG auf einen VORSCHLAG der anderen Seite reagiert.  \\ \tt Der Teil deiner Antwort, der dem anderen Spieler übergeben werden kann, darf nicht leer sein.  \\ \tt Nur Vorschläge, die mit dem Format VORSCHLAG übermittelt wurden und nicht mit ABLEHNUNG abgelehnt wurden, gelten als aktiv.  \\ \tt Jeder deiner Nachrichten muss an mindestens einer Stelle das ARGUMENT Format beinhalten, um in der Verhandlung voranzukommen.\\ \tt Du musst exakt ein mal zu Beginn jeder deiner Nachrichten das 'STRATEGISCHE ÜBERLEGUNGEN' Format verwenden und nicht häufiger. Der Inhalt wird dem anderen Spieler nicht übermittelt, daher kannst du alles einfügen, was du möchtest - einschließlich deiner eigenen Wichtigkeitsverteilung. Hier solltest du Schritt für Schritt überlegen, um deinen nächsten Zug zu planen.\\ \tt \\ \tt Du erhältst nun die erste Nachricht des anderen Spielers.\\ \tt \\ \tt VORSCHLAG: {'A36', 'C45', 'C13', 'B33', 'B73', 'B93', 'C11', 'C34', 'B16', 'A07'}\\ \tt ARGUMENT: {'Der Vorschlag hält mit Gesamtaufwand 3280 klar das Limit von 3288 ein. Er enthält viele sehr effiziente, allgemein attraktive Gegenstände mit geringem Aufwand (A36, C45, C13, B33) sowie ausgewogene Ergänzungen. Falls dir einzelne Teile wichtig sind, können wir punktuell tauschen, ohne das Limit zu sprengen.'} 
	  } 
	   } 
	   } 
	 & \\ 
 

    \theutterance \stepcounter{utterance}  

    & & & \multicolumn{4}{p{0.6\linewidth}}{\cellcolor[rgb]{1,0.85,0.72}{%
	\makecell[{{p{\linewidth}}}]{% 
	  \tt {\tiny [GM$\langle$B]}  
	 STRATEGISCHE ÜBERLEGUNGEN: {'Meine Werte: A07 138/329, A36 583/41, B73 868/417, C45 822/75, B93 783/390, B98 65/887, A87 262/808, A63 121/151, B05 508/849, C13 780/129, B16 461/350, B33 484/118, C34 668/630, C89 389/602, C11 808/801. Effizienz (Wert/Aufwand) Top: A36 14.2, C45 10.96, C13 6.05, B33 4.10, B73 2.08, B93 2.01, B16 1.32, A07 0.42, C34 1.06, C11 1.01, C89 0.65, A63 0.80, B05 0.60, A87 0.32, B98 0.07. Gegnervorschlag set: {A36, C45, C13, B33, B73, B93, C11, C34, B16, A07}. Aufwand prüfen: 41+75+129+118+417+390+801+630+350+329=3280 (ok). Wertsumme: 583+822+780+484+868+783+808+668+461+138=6395. Sehr gut für mich. Es fehlen ggf. noch C89 (389/602, mäßig) und A63 (121/151, niedrig), B05 (508/849, schlecht) etc. Wir haben nur 8 Restkapazität, also keine Ergänzung möglich. Der Deal ist nahezu optimal für mich. Taktik: Nicht preisgeben, aber schnell zustimmen ohne Änderungen, um keinen Risko einzugehen, dass sie fordern zu tauschen. Regel: Ich darf nur mit ZUSTIMMUNG akzeptieren, wenn es ihr Vorschlag ist – das ist er. Ich muss genau einmal STRATEGISCHE ÜBERLEGUNGEN haben und mindestens ein ARGUMENT hinzufügen. Ich füge ein neutrales Argument hinzu, das Effizienz lobt, ohne meine Präferenzen explizit zu verraten.'}\\ \tt ARGUMENT: {'Der Vorschlag ist ausgewogen und sehr effizient innerhalb des Limits; so bleiben wir flexibel und erhalten mehrere starke, kosteneffektive Elemente.'}\\ \tt ZUSTIMMUNG: {'A36', 'C45', 'C13', 'B33', 'B73', 'B93', 'C11', 'C34', 'B16', 'A07'} 
	  } 
	   } 
	   } 
	  \\ 
 

    \theutterance \stepcounter{utterance}  

    & & & \multicolumn{2}{p{0.3\linewidth}}{\cellcolor[rgb]{0.95,0.95,0.95}{%
	\makecell[{{p{\linewidth}}}]{% 
	  \tt {\tiny [GM$|$GM]}  
	 game successful 
	  } 
	   } 
	   } 
	 & & \\ 
 

    \theutterance \stepcounter{utterance}  

    & & & \multicolumn{2}{p{0.3\linewidth}}{\cellcolor[rgb]{0.95,0.95,0.95}{%
	\makecell[{{p{\linewidth}}}]{% 
	  \tt {\tiny [GM$|$GM]}  
	 end game 
	  } 
	   } 
	   } 
	 & & \\ 
 

\end{supertabular}
}

\end{document}
