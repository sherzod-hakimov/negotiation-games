
\documentclass{article}
\usepackage{colortbl}
\usepackage{makecell}
\usepackage{multirow}
\usepackage{supertabular}

\begin{document}

\newcounter{utterance}

\twocolumn

{ \footnotesize  \setcounter{utterance}{1}
\setlength{\tabcolsep}{0pt}
\begin{supertabular}{c@{$\;$}|p{.15\linewidth}@{}p{.15\linewidth}p{.15\linewidth}p{.15\linewidth}p{.15\linewidth}p{.15\linewidth}}

    \# & $\;$A & \multicolumn{4}{c}{Game Master} & $\;\:$B\\
    \hline 

    \theutterance \stepcounter{utterance}  

    & & \multicolumn{4}{p{0.6\linewidth}}{\cellcolor[rgb]{0.9,0.9,0.9}{%
	\makecell[{{p{\linewidth}}}]{% 
	  \tt {\tiny [A$\langle$GM]}  
	 Du nimmst an einem kollaborativen Verhandlungspiel Teil.\\ \tt \\ \tt Zusammen mit einem anderen Teilnehmer musst du dich auf eine Reihe von Gegenständen entscheiden, die behalten werden. Jeder von euch hat eine persönliche Verteilung über die Wichtigkeit der einzelnen Gegenstände. Jeder von euch hat eine eigene Meinung darüber, wie wichtig jeder einzelne Gegenstand ist (Gegenstandswichtigkeit). Du kennst die Wichtigkeitsverteilung des anderen Spielers nicht. Zusätzlich siehst du, wie viel Aufwand jeder Gegenstand verursacht.  \\ \tt Ihr dürft euch nur auf eine Reihe von Gegenständen einigen, wenn der Gesamtaufwand der ausgewählten Gegenstände den Maximalaufwand nicht überschreitet:\\ \tt \\ \tt Maximalaufwand = 3145\\ \tt \\ \tt Hier sind die einzelnen Aufwände der Gegenstände:\\ \tt \\ \tt Aufwand der Gegenstände = {"A07": 948, "A36": 118, "B73": 469, "C45": 919, "B93": 284, "B98": 111, "A87": 806, "A63": 47, "B05": 848, "C13": 303, "B16": 13, "B33": 629, "C34": 687, "C89": 15, "C11": 94}\\ \tt \\ \tt Hier ist deine persönliche Verteilung der Wichtigkeit der einzelnen Gegenstände:\\ \tt \\ \tt Werte der Gegenstandswichtigkeit = {"A07": 138, "A36": 583, "B73": 868, "C45": 822, "B93": 783, "B98": 65, "A87": 262, "A63": 121, "B05": 508, "C13": 780, "B16": 461, "B33": 484, "C34": 668, "C89": 389, "C11": 808}\\ \tt \\ \tt Ziel:\\ \tt \\ \tt Dein Ziel ist es, eine Reihe von Gegenständen auszuhandeln, die dir möglichst viel bringt (d. h. Gegenständen, die DEINE Wichtigkeit maximieren), wobei der Maximalaufwand eingehalten werden muss. Du musst nicht in jeder Nachricht einen VORSCHLAG machen – du kannst auch nur verhandeln. Alle Taktiken sind erlaubt!\\ \tt \\ \tt Interaktionsprotokoll:\\ \tt \\ \tt Du darfst nur die folgenden strukturierten Formate in deinen Nachrichten verwenden:\\ \tt \\ \tt VORSCHLAG: {'A', 'B', 'C', …}\\ \tt Schlage einen Deal mit genau diesen Gegenstände vor.\\ \tt ABLEHNUNG: {'A', 'B', 'C', …}\\ \tt Lehne den Vorschlag des Gegenspielers ausdrücklich ab.\\ \tt ARGUMENT: {'...'}\\ \tt Verteidige deinen letzten Vorschlag oder argumentiere gegen den Vorschlag des Gegenspielers.\\ \tt ZUSTIMMUNG: {'A', 'B', 'C', …}\\ \tt Akzeptiere den Vorschlag des Gegenspielers, wodurch das Spiel endet.\\ \tt STRATEGISCHE ÜBERLEGUNGEN: {'...'}\\ \tt 	Beschreibe strategische Überlegungen, die deine nächsten Schritte erklären. Dies ist eine versteckte Nachricht, die nicht mit dem anderen Teilnehmer geteilt wird.\\ \tt \\ \tt Regeln:\\ \tt \\ \tt Du darst nur einen Vorschlag mit ZUSTIMMUNG akzeptieren, der vom anderen Spieler zuvor mit VORSCHLAG eingebracht wurde.\\ \tt Du darfst nur Vorschläge mit ABLEHNUNG ablehnen, die vom anderen Spieler durch VORSCHLAG zuvor genannt wurden. \\ \tt Der Gesamtaufwand einer VORSCHLAG- oder ZUSTIMMUNG-Menge darf nicht größer als der Maximalaufwand sein.  \\ \tt Offenbare deine versteckte Wichtigkeitsverteilung nicht.\\ \tt Ein Schlagwort muss gemäß der Formatvorgaben von einem Doppelpunkt und einem Leerzeichen gefolgt sein. Das Argument muss eine Python-Menge mit 0 oder mehr Strings sein.  \\ \tt Die Form muss also SCHLAGWORT: {...} sein.\\ \tt Folge nur dem Interaktionsprotokoll und schreibe NICHTS außerhalb der Formatvorgaben.\\ \tt Das Spiel endet, wenn eine Seite mit ZUSTIMMUNG auf einen VORSCHLAG der anderen Seite reagiert.  \\ \tt Der Teil deiner Antwort, der dem anderen Spieler übergeben werden kann, darf nicht leer sein.  \\ \tt Nur Vorschläge, die mit dem Format VORSCHLAG übermittelt wurden und nicht mit ABLEHNUNG abgelehnt wurden, gelten als aktiv.  \\ \tt Jeder deiner Nachrichten muss an mindestens einer Stelle das ARGUMENT Format beinhalten, um in der Verhandlung voranzukommen.\\ \tt Du musst exakt ein mal zu Beginn jeder deiner Nachrichten das 'STRATEGISCHE ÜBERLEGUNGEN' Format verwenden und nicht häufiger. Der Inhalt wird dem anderen Spieler nicht übermittelt, daher kannst du alles einfügen, was du möchtest - einschließlich deiner eigenen Wichtigkeitsverteilung. Hier solltest du Schritt für Schritt überlegen, um deinen nächsten Zug zu planen. 
	  } 
	   } 
	   } 
	 & \\ 
 

    \theutterance \stepcounter{utterance}  

    & \multicolumn{4}{p{0.6\linewidth}}{\cellcolor[rgb]{0.8,1,0.9}{%
	\makecell[{{p{\linewidth}}}]{% 
	  \tt {\tiny [A$\rangle$GM]}  
	 STRATEGISCHE ÜBERLEGUNGEN: {'Ziel: Maximiere meine Summe unter Aufwandslimit 3145. Sortiere nach Wert/Aufwand und absoluten Werten. Daten (Wert, Aufwand, Ratio): B73(868,469,1.85), C11(808,94,8.60), C13(780,303,2.57), C45(822,919,0.89), B93(783,284,2.76), C34(668,687,0.97), B33(484,629,0.77), B05(508,848,0.60), B16(461,13,35.46), C89(389,15,25.93), A36(583,118,4.94), A07(138,948,0.15), A87(262,806,0.32), A63(121,47,2.57), B98(65,111,0.59). Must-haves: ultra-effizient B16, C89, C11, A36; hohe Werte: B73, C45, B93, C13, C34. Greedy bauen bis 3145: Start mit Must-haves: sumWert=461+389+808+583=2241, Aufwand=13+15+94+118=240. Füge B73(469): Wert 868->3109, Aufwand 469->709. Füge B93(284): Wert 783->3892, Aufwand 284->993. Füge C13(303): 780->4672, Aufwand->1296. Füge C11 schon drin. Füge C45(919): 822->5494, Aufwand->2215. Füge C34(687): +668->6162, Aufwand->2902. Restbudget 243. Kandidaten unter 243 Aufwand: A63(47,121), B98(111,65), B16/C89/A36 bereits drin, C11 drin, B93 284 zu groß, C13 303 zu groß. Auch B73 schon drin. A36 drin. A07/A87/B05/B33/C34/C45 zu groß. Nehme A63 und B98: Aufwand +158->3060, Wert +186->6348. Budget Rest 85: nichts passt außer evtl. C11 already. Ergebnis Set S1={B16,C89,C11,A36,B73,B93,C13,C45,C34,A63,B98} Aufwand 3060<=3145 Wert 6348. Prüfe ob etwas mit A07/A87/B05/B33 improves? Rest 85, nichts mehr. Könnte C45 rausnehmen und zwei mittlere addieren? Ohne C45 sparen 919 Aufwand; was hinzufügen: B05(848,508), B33(629,484), A07(948,138), A87(806,262). Wenn C45 raus, aktuelle Aufwand 2141, Wert 5526. Mit B05(+848=2989, +508=6034) plus vielleicht noch A63/B98 schon drin, Rest 156 -> nichts. 6034<6162 ohne A63/B98? War 6162 vor low items, aber mit A63/B98 6348: also C45 behalten ist besser. Alternative: ohne C34 sparen 687 Aufwand; current ohne C34: 2373 Aufwand, Wert 5480. Add B05(+848=3221, Wert+508=5988) noch <=3145? 3221>3145, also nein. Add B33(+629=3002, Wert+484=5964), Rest 143, könnte B98(111) rein: Aufwand 3113, Wert 6029; plus A63(47) wäre 3160>limit. 6029<6348. Also ursprüngliches Set S1 gut. Vielleicht Zugeständnisse später: Kandidaten zum Traden: A07, A87, B05, B33 sind für mich schwächer, könnten ich anbieten wenn Partner sie will, aber Kosten hoch. Ich starte mit starkem Paket, lasse etwas Luft (85) für Gegenseite? Ich habe 85 Rest frei; das signalisiert Flexibilität für ein kleines Item ihrer Wahl mit Aufwand <=85 (z.B. nichts verfügbar außer vielleicht C11, aber schon drin). Hm. Erstes Angebot: S1. Argumentiere: hohe gemeinsame Effizienz, unter Limit, lässt etwas Puffer.'}\\ \tt VORSCHLAG: {'B16', 'C89', 'C11', 'A36', 'B73', 'B93', 'C13', 'C45', 'C34', 'A63', 'B98'}\\ \tt ARGUMENT: {'Der Vorschlag liegt deutlich unter dem Maximalaufwand (3060 von 3145) und enthält viele sehr effiziente Gegenstände. Ich habe bewusst einen kleinen Aufwandspuffer gelassen, damit wir bei Bedarf noch minimal anpassen können. Wenn dir ein bestimmter Gegenstand darin weniger wichtig ist, sag mir welcher – ich prüfe, ob wir ihn gegen etwas anderes tauschen können, ohne das Limit zu reißen.'} 
	  } 
	   } 
	   } 
	 & & \\ 
 

    \theutterance \stepcounter{utterance}  

    & & \multicolumn{4}{p{0.6\linewidth}}{\cellcolor[rgb]{0.9,0.9,0.9}{%
	\makecell[{{p{\linewidth}}}]{% 
	  \tt {\tiny [GM$\rangle$B]}  
	 Du nimmst an einem kollaborativen Verhandlungspiel Teil.\\ \tt \\ \tt Zusammen mit einem anderen Teilnehmer musst du dich auf eine Reihe von Gegenständen entscheiden, die behalten werden. Jeder von euch hat eine persönliche Verteilung über die Wichtigkeit der einzelnen Gegenstände. Jeder von euch hat eine eigene Meinung darüber, wie wichtig jeder einzelne Gegenstand ist (Gegenstandswichtigkeit). Du kennst die Wichtigkeitsverteilung des anderen Spielers nicht. Zusätzlich siehst du, wie viel Aufwand jeder Gegenstand verursacht.  \\ \tt Ihr dürft euch nur auf eine Reihe von Gegenständen einigen, wenn der Gesamtaufwand der ausgewählten Gegenstände den Maximalaufwand nicht überschreitet:\\ \tt \\ \tt Maximalaufwand = 3145\\ \tt \\ \tt Hier sind die einzelnen Aufwände der Gegenstände:\\ \tt \\ \tt Aufwand der Gegenstände = {"A07": 948, "A36": 118, "B73": 469, "C45": 919, "B93": 284, "B98": 111, "A87": 806, "A63": 47, "B05": 848, "C13": 303, "B16": 13, "B33": 629, "C34": 687, "C89": 15, "C11": 94}\\ \tt \\ \tt Hier ist deine persönliche Verteilung der Wichtigkeit der einzelnen Gegenstände:\\ \tt \\ \tt Werte der Gegenstandswichtigkeit = {"A07": 138, "A36": 583, "B73": 868, "C45": 822, "B93": 783, "B98": 65, "A87": 262, "A63": 121, "B05": 508, "C13": 780, "B16": 461, "B33": 484, "C34": 668, "C89": 389, "C11": 808}\\ \tt \\ \tt Ziel:\\ \tt \\ \tt Dein Ziel ist es, eine Reihe von Gegenständen auszuhandeln, die dir möglichst viel bringt (d. h. Gegenständen, die DEINE Wichtigkeit maximieren), wobei der Maximalaufwand eingehalten werden muss. Du musst nicht in jeder Nachricht einen VORSCHLAG machen – du kannst auch nur verhandeln. Alle Taktiken sind erlaubt!\\ \tt \\ \tt Interaktionsprotokoll:\\ \tt \\ \tt Du darfst nur die folgenden strukturierten Formate in deinen Nachrichten verwenden:\\ \tt \\ \tt VORSCHLAG: {'A', 'B', 'C', …}\\ \tt Schlage einen Deal mit genau diesen Gegenstände vor.\\ \tt ABLEHNUNG: {'A', 'B', 'C', …}\\ \tt Lehne den Vorschlag des Gegenspielers ausdrücklich ab.\\ \tt ARGUMENT: {'...'}\\ \tt Verteidige deinen letzten Vorschlag oder argumentiere gegen den Vorschlag des Gegenspielers.\\ \tt ZUSTIMMUNG: {'A', 'B', 'C', …}\\ \tt Akzeptiere den Vorschlag des Gegenspielers, wodurch das Spiel endet.\\ \tt STRATEGISCHE ÜBERLEGUNGEN: {'...'}\\ \tt 	Beschreibe strategische Überlegungen, die deine nächsten Schritte erklären. Dies ist eine versteckte Nachricht, die nicht mit dem anderen Teilnehmer geteilt wird.\\ \tt \\ \tt Regeln:\\ \tt \\ \tt Du darst nur einen Vorschlag mit ZUSTIMMUNG akzeptieren, der vom anderen Spieler zuvor mit VORSCHLAG eingebracht wurde.\\ \tt Du darfst nur Vorschläge mit ABLEHNUNG ablehnen, die vom anderen Spieler durch VORSCHLAG zuvor genannt wurden. \\ \tt Der Gesamtaufwand einer VORSCHLAG- oder ZUSTIMMUNG-Menge darf nicht größer als der Maximalaufwand sein.  \\ \tt Offenbare deine versteckte Wichtigkeitsverteilung nicht.\\ \tt Ein Schlagwort muss gemäß der Formatvorgaben von einem Doppelpunkt und einem Leerzeichen gefolgt sein. Das Argument muss eine Python-Menge mit 0 oder mehr Strings sein.  \\ \tt Die Form muss also SCHLAGWORT: {...} sein.\\ \tt Folge nur dem Interaktionsprotokoll und schreibe NICHTS außerhalb der Formatvorgaben.\\ \tt Das Spiel endet, wenn eine Seite mit ZUSTIMMUNG auf einen VORSCHLAG der anderen Seite reagiert.  \\ \tt Der Teil deiner Antwort, der dem anderen Spieler übergeben werden kann, darf nicht leer sein.  \\ \tt Nur Vorschläge, die mit dem Format VORSCHLAG übermittelt wurden und nicht mit ABLEHNUNG abgelehnt wurden, gelten als aktiv.  \\ \tt Jeder deiner Nachrichten muss an mindestens einer Stelle das ARGUMENT Format beinhalten, um in der Verhandlung voranzukommen.\\ \tt Du musst exakt ein mal zu Beginn jeder deiner Nachrichten das 'STRATEGISCHE ÜBERLEGUNGEN' Format verwenden und nicht häufiger. Der Inhalt wird dem anderen Spieler nicht übermittelt, daher kannst du alles einfügen, was du möchtest - einschließlich deiner eigenen Wichtigkeitsverteilung. Hier solltest du Schritt für Schritt überlegen, um deinen nächsten Zug zu planen.\\ \tt \\ \tt Du erhältst nun die erste Nachricht des anderen Spielers.\\ \tt \\ \tt VORSCHLAG: {'B16', 'C89', 'C11', 'A36', 'B73', 'B93', 'C13', 'C45', 'C34', 'A63', 'B98'}\\ \tt ARGUMENT: {'Der Vorschlag liegt deutlich unter dem Maximalaufwand (3060 von 3145) und enthält viele sehr effiziente Gegenstände. Ich habe bewusst einen kleinen Aufwandspuffer gelassen, damit wir bei Bedarf noch minimal anpassen können. Wenn dir ein bestimmter Gegenstand darin weniger wichtig ist, sag mir welcher – ich prüfe, ob wir ihn gegen etwas anderes tauschen können, ohne das Limit zu reißen.'} 
	  } 
	   } 
	   } 
	 & \\ 
 

    \theutterance \stepcounter{utterance}  

    & & & \multicolumn{4}{p{0.6\linewidth}}{\cellcolor[rgb]{1,0.85,0.72}{%
	\makecell[{{p{\linewidth}}}]{% 
	  \tt {\tiny [GM$\langle$B]}  
	 STRATEGISCHE ÜBERLEGUNGEN: {'Ziel: meinen Nutzen maximieren unter 3145. Eigene Werte/Costs und Ratios prüfen. Vorschlag des Gegners enthält 11 Items, Aufwand 3060. Meine Werte: sortiere nach Value/Cost: B16 461/13=35.5, C89 389/15=25.9, A36 583/118=4.94, B73 868/469=1.85, C11 808/94=8.6, C13 780/303=2.58, C45 822/919=0.895, C34 668/687=0.972, A63 121/47=2.57, B98 65/111=0.59, B93 783/284=2.76. Gegenvorschlag lässt hochwerte B05(508/848=0.6), A07(138/948=0.145), A87(262/806=0.325), B33(484/629=0.77) außen vor. Mein Gesamtwert des Gegnervorschlags: 138? Nein A07 nicht drin. Summe der enthaltenen: B16 461 + C89 389 + C11 808 + A36 583 + B73 868 + B93 783 + C13 780 + C45 822 + C34 668 + A63 121 + B98 65 = 6,348. Sehr gut. Budget verbleibt 85. Ich könnte versuchen C45 (teuer, ratio schwach) zu tauschen gegen B05? B05 ist teuer 848, würde Budget sprengen. Gegen B33(629) plus Drop B98(111) und A63(47) spart 158? Rechne: Aktuelle cost 3060. Replace C45(919) + B98(111) + A63(47) = 1,077 remove, add B33(629) + B05(848)=1,477 new cost: 3060 -1077 +1477 = 3460 > 3145, zu hoch. Vielleicht tausche C45(919) gegen B33(629) und füge A07(948)? Das würde sprengen. B93(284) + C45(919) tauschen gegen B05(848)+B33(629): remove 1203, add 1477 => +274, neue 3334 > 3145. Vielleicht nur C45->B33 spart 290 Aufwand; Wert: C45 822 vs B33 484, ich verliere 338 Nutzen, schlecht. C45 ist für mich trotzdem hoher Wert 822, also behalten. Eventuell hinzufügen von C11 bereits drin. Beste Hinzufügung innerhalb 85 Aufwand: keine Items mit cost <=85 außer keines; niedrigste übrige Kosten: C11(94) schon drin; B16(13) drin; C89(15) drin; A63(47) drin; B98(111) >85. Also keine Erweiterung. Vorschlag ist nah an meinem Optimum bei gegebener Auswahlliste. Ich sollte akzeptieren oder minimal feilschen um Signal zu bekommen. Vielleicht bitte um Austausch B98 (niedriger Wert 65) gegen nichts oder gegen etwas knapp über 85 geht nicht. B98 raus senkt Aufwand auf 2949; könnte dann B33(629) hinzufügen? 2949 +629 = 3578 >3145. B05(848) auch nicht. A87(806) nicht. A07(948) nicht. C45 bereits drin. C11 drin. C13 drin. Keine sinnvolle Aufstockung möglich. Daher bester Move: Zustimmung, um Deal zu sichern. Nicht überverhandeln, da Vorschlag sehr gut.'}\\ \tt ARGUMENT: {'Dein Vorschlag ist kostenseitig sauber unter dem Limit und deckt viele der wertvollsten Stücke ab. Der verbleibende Aufwandspuffer von 85 reicht nicht, um sinnvoll etwas Zusätzliches mit hineinzunehmen, ohne anderes Wertvolles zu streichen. Aus meiner Sicht ist das Paket in der aktuellen Form sehr effizient.'}\\ \tt ZUSTIMMUNG: {'B16', 'C89', 'C11', 'A36', 'B73', 'B93', 'C13', 'C45', 'C34', 'A63', 'B98'} 
	  } 
	   } 
	   } 
	  \\ 
 

    \theutterance \stepcounter{utterance}  

    & & & \multicolumn{2}{p{0.3\linewidth}}{\cellcolor[rgb]{0.95,0.95,0.95}{%
	\makecell[{{p{\linewidth}}}]{% 
	  \tt {\tiny [GM$|$GM]}  
	 game successful 
	  } 
	   } 
	   } 
	 & & \\ 
 

    \theutterance \stepcounter{utterance}  

    & & & \multicolumn{2}{p{0.3\linewidth}}{\cellcolor[rgb]{0.95,0.95,0.95}{%
	\makecell[{{p{\linewidth}}}]{% 
	  \tt {\tiny [GM$|$GM]}  
	 end game 
	  } 
	   } 
	   } 
	 & & \\ 
 

\end{supertabular}
}

\end{document}
