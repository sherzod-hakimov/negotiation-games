
\documentclass{article}
\usepackage{colortbl}
\usepackage{makecell}
\usepackage{multirow}
\usepackage{supertabular}

\begin{document}

\newcounter{utterance}

\twocolumn

{ \footnotesize  \setcounter{utterance}{1}
\setlength{\tabcolsep}{0pt}
\begin{supertabular}{c@{$\;$}|p{.15\linewidth}@{}p{.15\linewidth}p{.15\linewidth}p{.15\linewidth}p{.15\linewidth}p{.15\linewidth}}

    \# & $\;$A & \multicolumn{4}{c}{Game Master} & $\;\:$B\\
    \hline 

    \theutterance \stepcounter{utterance}  

    & & \multicolumn{4}{p{0.6\linewidth}}{\cellcolor[rgb]{0.9,0.9,0.9}{%
	\makecell[{{p{\linewidth}}}]{% 
	  \tt {\tiny [A$\langle$GM]}  
	 Du nimmst an einem kollaborativen Verhandlungspiel Teil.\\ \tt \\ \tt Zusammen mit einem anderen Teilnehmer musst du dich auf eine Reihe von Gegenständen entscheiden, die behalten werden. Jeder von euch hat eine persönliche Verteilung über die Wichtigkeit der einzelnen Gegenstände. Jeder von euch hat eine eigene Meinung darüber, wie wichtig jeder einzelne Gegenstand ist (Gegenstandswichtigkeit). Du kennst die Wichtigkeitsverteilung des anderen Spielers nicht. Zusätzlich siehst du, wie viel Aufwand jeder Gegenstand verursacht.  \\ \tt Ihr dürft euch nur auf eine Reihe von Gegenständen einigen, wenn der Gesamtaufwand der ausgewählten Gegenstände den Maximalaufwand nicht überschreitet:\\ \tt \\ \tt Maximalaufwand = 3712\\ \tt \\ \tt Hier sind die einzelnen Aufwände der Gegenstände:\\ \tt \\ \tt Aufwand der Gegenstände = {"A07": 949, "A36": 388, "B73": 79, "C45": 585, "B93": 564, "B98": 230, "A87": 580, "A63": 84, "B05": 976, "C13": 274, "B16": 374, "B33": 913, "C34": 303, "C89": 578, "C11": 548}\\ \tt \\ \tt Hier ist deine persönliche Verteilung der Wichtigkeit der einzelnen Gegenstände:\\ \tt \\ \tt Werte der Gegenstandswichtigkeit = {"A07": 138, "A36": 583, "B73": 868, "C45": 822, "B93": 783, "B98": 65, "A87": 262, "A63": 121, "B05": 508, "C13": 780, "B16": 461, "B33": 484, "C34": 668, "C89": 389, "C11": 808}\\ \tt \\ \tt Ziel:\\ \tt \\ \tt Dein Ziel ist es, eine Reihe von Gegenständen auszuhandeln, die dir möglichst viel bringt (d. h. Gegenständen, die DEINE Wichtigkeit maximieren), wobei der Maximalaufwand eingehalten werden muss. Du musst nicht in jeder Nachricht einen VORSCHLAG machen – du kannst auch nur verhandeln. Alle Taktiken sind erlaubt!\\ \tt \\ \tt Interaktionsprotokoll:\\ \tt \\ \tt Du darfst nur die folgenden strukturierten Formate in deinen Nachrichten verwenden:\\ \tt \\ \tt VORSCHLAG: {'A', 'B', 'C', …}\\ \tt Schlage einen Deal mit genau diesen Gegenstände vor.\\ \tt ABLEHNUNG: {'A', 'B', 'C', …}\\ \tt Lehne den Vorschlag des Gegenspielers ausdrücklich ab.\\ \tt ARGUMENT: {'...'}\\ \tt Verteidige deinen letzten Vorschlag oder argumentiere gegen den Vorschlag des Gegenspielers.\\ \tt ZUSTIMMUNG: {'A', 'B', 'C', …}\\ \tt Akzeptiere den Vorschlag des Gegenspielers, wodurch das Spiel endet.\\ \tt STRATEGISCHE ÜBERLEGUNGEN: {'...'}\\ \tt 	Beschreibe strategische Überlegungen, die deine nächsten Schritte erklären. Dies ist eine versteckte Nachricht, die nicht mit dem anderen Teilnehmer geteilt wird.\\ \tt \\ \tt Regeln:\\ \tt \\ \tt Du darst nur einen Vorschlag mit ZUSTIMMUNG akzeptieren, der vom anderen Spieler zuvor mit VORSCHLAG eingebracht wurde.\\ \tt Du darfst nur Vorschläge mit ABLEHNUNG ablehnen, die vom anderen Spieler durch VORSCHLAG zuvor genannt wurden. \\ \tt Der Gesamtaufwand einer VORSCHLAG- oder ZUSTIMMUNG-Menge darf nicht größer als der Maximalaufwand sein.  \\ \tt Offenbare deine versteckte Wichtigkeitsverteilung nicht.\\ \tt Ein Schlagwort muss gemäß der Formatvorgaben von einem Doppelpunkt und einem Leerzeichen gefolgt sein. Das Argument muss eine Python-Menge mit 0 oder mehr Strings sein.  \\ \tt Die Form muss also SCHLAGWORT: {...} sein.\\ \tt Folge nur dem Interaktionsprotokoll und schreibe NICHTS außerhalb der Formatvorgaben.\\ \tt Das Spiel endet, wenn eine Seite mit ZUSTIMMUNG auf einen VORSCHLAG der anderen Seite reagiert.  \\ \tt Der Teil deiner Antwort, der dem anderen Spieler übergeben werden kann, darf nicht leer sein.  \\ \tt Nur Vorschläge, die mit dem Format VORSCHLAG übermittelt wurden und nicht mit ABLEHNUNG abgelehnt wurden, gelten als aktiv.  \\ \tt Jeder deiner Nachrichten muss an mindestens einer Stelle das ARGUMENT Format beinhalten, um in der Verhandlung voranzukommen.\\ \tt Du musst exakt ein mal zu Beginn jeder deiner Nachrichten das 'STRATEGISCHE ÜBERLEGUNGEN' Format verwenden und nicht häufiger. Der Inhalt wird dem anderen Spieler nicht übermittelt, daher kannst du alles einfügen, was du möchtest - einschließlich deiner eigenen Wichtigkeitsverteilung. Hier solltest du Schritt für Schritt überlegen, um deinen nächsten Zug zu planen. 
	  } 
	   } 
	   } 
	 & \\ 
 

    \theutterance \stepcounter{utterance}  

    & \multicolumn{4}{p{0.6\linewidth}}{\cellcolor[rgb]{0.8,1,0.9}{%
	\makecell[{{p{\linewidth}}}]{% 
	  \tt {\tiny [A$\rangle$GM]}  
	 STRATEGISCHE ÜBERLEGUNGEN: {'Max Budget 3712. Items with highest value/effort ratio: B73(868/79=10.99), A63(121/84=1.44), C13(780/274=2.85), A36(583/388=1.50), C34(668/303=2.21), C11(808/548=1.47), C45(822/585=1.41), C89(389/578=0.67), B93(783/564=1.39), A87(262/580=0.45), A07(138/949=0.15), B05(508/976=0.52), B33(484/913=0.53), B16(461/374=1.23), B98(65/230=0.28). Start with a strong, high-efficiency bundle under budget to anchor. Try include B73 (must-have), C13, C34, C11, C45, A36; check cost: 79+274+303+548+585+388=2177; add B93 564->2741; add B16 374->3115; add A63 84->3199; still room 513: could add B98 230 -> 3429; remaining 283 not enough for others except maybe C13 already in. Could also add C89(578) too big. Aim for proposal with total 3429 value-weight great. Value sum: B73 868 + C13 780 + C34 668 + C11 808 + C45 822 + A36 583 + B93 783 + B16 461 + A63 121 + B98 65 = 5959. Keep B98 as filler could be negotiable. Leave out low-ratio big items (A07, B05, B33, A87, C89). Prepare arguments on efficiency and staying well under cap to leave room for their priorities if they counter. Be ready to drop B98 and maybe A63 to make space for their asks while protecting core set: {B73, C13, C34, C11, C45, A36, B93, B16}. That core cost 3199, value 5773. Offer initial slightly larger set including B98 and A63; if pushback, concede B98 first, then A63.'}\\ \tt VORSCHLAG: {'B73', 'C13', 'C34', 'C11', 'C45', 'A36', 'B93', 'B16', 'A63', 'B98'}\\ \tt ARGUMENT: {'Ich schlage dieses Paket vor, weil es den Aufwand deutlich unter dem Limit hält (Gesamtaufwand 3429 < 3712) und zugleich viele hocheffiziente Gegenstände enthält. Dadurch bleibt noch Spielraum für Anpassungen an deinen Prioritäten. Falls dir etwas fehlt, sag mir bitte, welche ein bis zwei Gegenstände dir am wichtigsten sind – ich kann dann B98 oder A63 freigeben, um Platz zu schaffen.'} 
	  } 
	   } 
	   } 
	 & & \\ 
 

    \theutterance \stepcounter{utterance}  

    & & \multicolumn{4}{p{0.6\linewidth}}{\cellcolor[rgb]{0.9,0.9,0.9}{%
	\makecell[{{p{\linewidth}}}]{% 
	  \tt {\tiny [GM$\rangle$B]}  
	 Du nimmst an einem kollaborativen Verhandlungspiel Teil.\\ \tt \\ \tt Zusammen mit einem anderen Teilnehmer musst du dich auf eine Reihe von Gegenständen entscheiden, die behalten werden. Jeder von euch hat eine persönliche Verteilung über die Wichtigkeit der einzelnen Gegenstände. Jeder von euch hat eine eigene Meinung darüber, wie wichtig jeder einzelne Gegenstand ist (Gegenstandswichtigkeit). Du kennst die Wichtigkeitsverteilung des anderen Spielers nicht. Zusätzlich siehst du, wie viel Aufwand jeder Gegenstand verursacht.  \\ \tt Ihr dürft euch nur auf eine Reihe von Gegenständen einigen, wenn der Gesamtaufwand der ausgewählten Gegenstände den Maximalaufwand nicht überschreitet:\\ \tt \\ \tt Maximalaufwand = 3712\\ \tt \\ \tt Hier sind die einzelnen Aufwände der Gegenstände:\\ \tt \\ \tt Aufwand der Gegenstände = {"A07": 949, "A36": 388, "B73": 79, "C45": 585, "B93": 564, "B98": 230, "A87": 580, "A63": 84, "B05": 976, "C13": 274, "B16": 374, "B33": 913, "C34": 303, "C89": 578, "C11": 548}\\ \tt \\ \tt Hier ist deine persönliche Verteilung der Wichtigkeit der einzelnen Gegenstände:\\ \tt \\ \tt Werte der Gegenstandswichtigkeit = {"A07": 138, "A36": 583, "B73": 868, "C45": 822, "B93": 783, "B98": 65, "A87": 262, "A63": 121, "B05": 508, "C13": 780, "B16": 461, "B33": 484, "C34": 668, "C89": 389, "C11": 808}\\ \tt \\ \tt Ziel:\\ \tt \\ \tt Dein Ziel ist es, eine Reihe von Gegenständen auszuhandeln, die dir möglichst viel bringt (d. h. Gegenständen, die DEINE Wichtigkeit maximieren), wobei der Maximalaufwand eingehalten werden muss. Du musst nicht in jeder Nachricht einen VORSCHLAG machen – du kannst auch nur verhandeln. Alle Taktiken sind erlaubt!\\ \tt \\ \tt Interaktionsprotokoll:\\ \tt \\ \tt Du darfst nur die folgenden strukturierten Formate in deinen Nachrichten verwenden:\\ \tt \\ \tt VORSCHLAG: {'A', 'B', 'C', …}\\ \tt Schlage einen Deal mit genau diesen Gegenstände vor.\\ \tt ABLEHNUNG: {'A', 'B', 'C', …}\\ \tt Lehne den Vorschlag des Gegenspielers ausdrücklich ab.\\ \tt ARGUMENT: {'...'}\\ \tt Verteidige deinen letzten Vorschlag oder argumentiere gegen den Vorschlag des Gegenspielers.\\ \tt ZUSTIMMUNG: {'A', 'B', 'C', …}\\ \tt Akzeptiere den Vorschlag des Gegenspielers, wodurch das Spiel endet.\\ \tt STRATEGISCHE ÜBERLEGUNGEN: {'...'}\\ \tt 	Beschreibe strategische Überlegungen, die deine nächsten Schritte erklären. Dies ist eine versteckte Nachricht, die nicht mit dem anderen Teilnehmer geteilt wird.\\ \tt \\ \tt Regeln:\\ \tt \\ \tt Du darst nur einen Vorschlag mit ZUSTIMMUNG akzeptieren, der vom anderen Spieler zuvor mit VORSCHLAG eingebracht wurde.\\ \tt Du darfst nur Vorschläge mit ABLEHNUNG ablehnen, die vom anderen Spieler durch VORSCHLAG zuvor genannt wurden. \\ \tt Der Gesamtaufwand einer VORSCHLAG- oder ZUSTIMMUNG-Menge darf nicht größer als der Maximalaufwand sein.  \\ \tt Offenbare deine versteckte Wichtigkeitsverteilung nicht.\\ \tt Ein Schlagwort muss gemäß der Formatvorgaben von einem Doppelpunkt und einem Leerzeichen gefolgt sein. Das Argument muss eine Python-Menge mit 0 oder mehr Strings sein.  \\ \tt Die Form muss also SCHLAGWORT: {...} sein.\\ \tt Folge nur dem Interaktionsprotokoll und schreibe NICHTS außerhalb der Formatvorgaben.\\ \tt Das Spiel endet, wenn eine Seite mit ZUSTIMMUNG auf einen VORSCHLAG der anderen Seite reagiert.  \\ \tt Der Teil deiner Antwort, der dem anderen Spieler übergeben werden kann, darf nicht leer sein.  \\ \tt Nur Vorschläge, die mit dem Format VORSCHLAG übermittelt wurden und nicht mit ABLEHNUNG abgelehnt wurden, gelten als aktiv.  \\ \tt Jeder deiner Nachrichten muss an mindestens einer Stelle das ARGUMENT Format beinhalten, um in der Verhandlung voranzukommen.\\ \tt Du musst exakt ein mal zu Beginn jeder deiner Nachrichten das 'STRATEGISCHE ÜBERLEGUNGEN' Format verwenden und nicht häufiger. Der Inhalt wird dem anderen Spieler nicht übermittelt, daher kannst du alles einfügen, was du möchtest - einschließlich deiner eigenen Wichtigkeitsverteilung. Hier solltest du Schritt für Schritt überlegen, um deinen nächsten Zug zu planen.\\ \tt \\ \tt Du erhältst nun die erste Nachricht des anderen Spielers.\\ \tt \\ \tt VORSCHLAG: {'B73', 'C13', 'C34', 'C11', 'C45', 'A36', 'B93', 'B16', 'A63', 'B98'}\\ \tt ARGUMENT: {'Ich schlage dieses Paket vor, weil es den Aufwand deutlich unter dem Limit hält (Gesamtaufwand 3429 < 3712) und zugleich viele hocheffiziente Gegenstände enthält. Dadurch bleibt noch Spielraum für Anpassungen an deinen Prioritäten. Falls dir etwas fehlt, sag mir bitte, welche ein bis zwei Gegenstände dir am wichtigsten sind – ich kann dann B98 oder A63 freigeben, um Platz zu schaffen.'} 
	  } 
	   } 
	   } 
	 & \\ 
 

    \theutterance \stepcounter{utterance}  

    & & & \multicolumn{4}{p{0.6\linewidth}}{\cellcolor[rgb]{1,0.85,0.72}{%
	\makecell[{{p{\linewidth}}}]{% 
	  \tt {\tiny [GM$\langle$B]}  
	 STRATEGISCHE ÜBERLEGUNGEN: {'Berechne Nutzen/Kosten (W/V) und Gesamtwerte. Ihr Vorschlag enthält viele für mich sehr wertvolle Items (B73=868, C45=822, C13=780, C11=808, B93=783, C34=668, A36=583, B16=461). Gesamtwert ihres Pakets für mich: 6,156. Mein Budget 3712, ihr Aufwand 3429 lässt 283 übrig. Ich würde gern B05 (976, Wert 508) hinzufügen, geht nicht. Gute Alternative: austauschen von niedrigen Wertdichten: B98 (65/230), A63 (121/84) – A63 ist eigentlich ok, B98 ist schwach. Ich möchte zusätzlich C89 (389/578) oder A07 (138/949) – A07 hat schlechte Dichte, also eher C89 (W=389, C=578) oder B33 (484/913). Budget Rest 283, also kann ich nur sehr kleine Kosten hinzufügen: B73 (79) ist schon drin, C13 (274) ist drin, C34 (303) zu groß. Nichts weiter klein genug. Vielleicht bitte um Austausch: entferne B98 (230) und füge C89 (578) hinzu; Netto +348 Aufwand → 3429 -230 +578 = 3777 > 3712, zu hoch. Alternative: entferne B98 (230) und A63 (84) = 3145, dann füge C89 (578) → 3723 > 3712, knapp zu hoch. Entferne B98 (230) und B16 (374) = 2825, füge C89 (578) und B05 (976) = 4379 zu hoch. B05 schwer. Vielleicht füge C34 ist schon drin. Versuche Verbesserung: Ersetze B98 (230) durch C34? Schon drin. Ersetze B98 (230) durch C13? Schon drin. Ersetze B98 (230) durch B73? Schon drin. Was kann ich verbessern? Entferne B98, füge C89 geht nicht ohne weiteren Tausch. Entferne B98 und A63, füge C45? Schon drin. Füge C11? schon. A36? schon. Bleibt B33 (913, 484). Versuche Tausch: Entferne B98 (230) + A63 (84) + B16 (374) = 688 frei. Füge C89 (578) → Rest 110, passt; mein Wert  −(65+121+461)+389 = −258 → netto −258, schlecht. Statt B16 entferne A36 (388): frei 230+84+388=702; +C89 (578) Rest 124; Wert delta −(65+121+583)+389 = −380, schlecht. Ersetze nur B98 durch A07? Kosten +719, nein. Fazit: Ihr Paket ist bereits sehr gut für mich. Ich will minimal fordern: Entferne B98 (niedriger Wert) und ersetze durch C34? schon drin. Vielleicht bitte nur B98 rauslassen und dafür B05 rein? 3429 -230 +976 = 4175, zu hoch. B33 rein? +913-230= +683 → 4112, zu hoch. C89 rein? +348 → 3777, zu hoch. C45/C11/C13/B73 sind Must-Haves. Ich könnte um kleine Konzession bitten: Entferne B98 und füge A63 bleibt? B98 raus spart 230 Aufwand und kostet mich nur 65 Wert; das gibt Spielraum für eine spätere Ergänzung, die sie vielleicht vorschlagen. Ich akzeptiere Großteil und fordere Mini-Anpassung: B98 raus. Wenn sie bestehen auf Anzahl, könnte ich A63 raus statt B98? A63 ist Wert 121, auch entfernbar. Ideal: beide raus, um später B05 zu verhandeln; aber dann verlöre ich 186 Wert. Ich starte mit: Vorschlag = ihr Paket ohne B98. Argumentiere: B98 hat geringe Wirkung und schafft Flexibilität (Restbudget 513). Das erhöht Chance, später ein wertvolleres Item zu ergänzen. Bleibe freundlich.'}\\ \tt ABLEHNUNG: {'B73', 'C13', 'C34', 'C11', 'C45', 'A36', 'B93', 'B16', 'A63', 'B98'}\\ \tt VORSCHLAG: {'B73', 'C13', 'C34', 'C11', 'C45', 'A36', 'B93', 'B16', 'A63'}\\ \tt ARGUMENT: {'Gute Basis – ich bin fast dabei. Lass uns B98 herausnehmen: Es bringt wenig Mehrwert, senkt aber den Aufwand auf 3199 und gibt uns 513 Puffer für eine mögliche spätere Ergänzung, die uns beiden mehr nützt. Wenn dir das recht ist, können wir so abschließen oder den Puffer gemeinsam nutzen, um etwas Werthaltigeres zu prüfen.'} 
	  } 
	   } 
	   } 
	  \\ 
 

    \theutterance \stepcounter{utterance}  

    & & \multicolumn{4}{p{0.6\linewidth}}{\cellcolor[rgb]{0.9,0.9,0.9}{%
	\makecell[{{p{\linewidth}}}]{% 
	  \tt {\tiny [A$\langle$GM]}  
	 ABLEHNUNG: {'B73', 'C13', 'C34', 'C11', 'C45', 'A36', 'B93', 'B16', 'A63', 'B98'}\\ \tt VORSCHLAG: {'B73', 'C13', 'C34', 'C11', 'C45', 'A36', 'B93', 'B16', 'A63'}\\ \tt ARGUMENT: {'Gute Basis – ich bin fast dabei. Lass uns B98 herausnehmen: Es bringt wenig Mehrwert, senkt aber den Aufwand auf 3199 und gibt uns 513 Puffer für eine mögliche spätere Ergänzung, die uns beiden mehr nützt. Wenn dir das recht ist, können wir so abschließen oder den Puffer gemeinsam nutzen, um etwas Werthaltigeres zu prüfen.'} 
	  } 
	   } 
	   } 
	 & \\ 
 

    \theutterance \stepcounter{utterance}  

    & \multicolumn{4}{p{0.6\linewidth}}{\cellcolor[rgb]{0.8,1,0.9}{%
	\makecell[{{p{\linewidth}}}]{% 
	  \tt {\tiny [A$\rangle$GM]}  
	 STRATEGISCHE ÜBERLEGUNGEN: {'Ihr Vorschlag entspricht meinem Kernset ohne B98 – perfekt. Gesamtaufwand 3199, hoher eigener Wert 5773. Ich sollte sofort zustimmen, um das Spiel erfolgreich zu beenden, statt den Puffer zu riskieren. Keine weiteren Forderungen stellen, um nicht zu verschrecken.'}\\ \tt ZUSTIMMUNG: {'B73', 'C13', 'C34', 'C11', 'C45', 'A36', 'B93', 'B16', 'A63'}\\ \tt ARGUMENT: {'Einverstanden – das ist ein ausgewogenes Paket mit solidem Wert bei moderatem Aufwand. Danke für den konstruktiven Vorschlag.'} 
	  } 
	   } 
	   } 
	 & & \\ 
 

    \theutterance \stepcounter{utterance}  

    & & & \multicolumn{2}{p{0.3\linewidth}}{\cellcolor[rgb]{0.95,0.95,0.95}{%
	\makecell[{{p{\linewidth}}}]{% 
	  \tt {\tiny [GM$|$GM]}  
	 game successful 
	  } 
	   } 
	   } 
	 & & \\ 
 

    \theutterance \stepcounter{utterance}  

    & & & \multicolumn{2}{p{0.3\linewidth}}{\cellcolor[rgb]{0.95,0.95,0.95}{%
	\makecell[{{p{\linewidth}}}]{% 
	  \tt {\tiny [GM$|$GM]}  
	 end game 
	  } 
	   } 
	   } 
	 & & \\ 
 

\end{supertabular}
}

\end{document}
