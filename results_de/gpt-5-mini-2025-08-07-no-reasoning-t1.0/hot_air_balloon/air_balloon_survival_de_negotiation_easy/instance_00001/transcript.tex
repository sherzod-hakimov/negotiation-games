
\documentclass{article}
\usepackage{colortbl}
\usepackage{makecell}
\usepackage{multirow}
\usepackage{supertabular}

\begin{document}

\newcounter{utterance}

\centering \large Interaction Transcript for game `hot\_air\_balloon', experiment `air\_balloon\_survival\_de\_negotiation\_easy', episode 1 with gpt{-}5{-}mini{-}2025{-}08{-}07{-}no{-}reasoning{-}t1.0.
\vspace{24pt}

{ \footnotesize  \setcounter{utterance}{1}
\setlength{\tabcolsep}{0pt}
\begin{supertabular}{c@{$\;$}|p{.15\linewidth}@{}p{.15\linewidth}p{.15\linewidth}p{.15\linewidth}p{.15\linewidth}p{.15\linewidth}}
    \# & $\;$A & \multicolumn{4}{c}{Game Master} & $\;\:$B\\
    \hline

    \theutterance \stepcounter{utterance}  
    & & \multicolumn{4}{p{0.6\linewidth}}{
        \cellcolor[rgb]{0.9,0.9,0.9}{
            \makecell[{{p{\linewidth}}}]{
                \texttt{\tiny{[P1$\langle$GM]}}
                \texttt{Du nimmst an einem kollaborativen Verhandlungspiel Teil.} \\
\\ 
\texttt{Zusammen mit einem anderen Teilnehmer musst du dich auf eine Reihe von Gegenständen entscheiden, die behalten werden. Jeder von euch hat eine persönliche Verteilung über die Wichtigkeit der einzelnen Gegenstände. Jeder von euch hat eine eigene Meinung darüber, wie wichtig jeder einzelne Gegenstand ist (Gegenstandswichtigkeit). Du kennst die Wichtigkeitsverteilung des anderen Spielers nicht. Zusätzlich siehst du, wie viel Aufwand jeder Gegenstand verursacht.  } \\
\texttt{Ihr dürft euch nur auf eine Reihe von Gegenständen einigen, wenn der Gesamtaufwand der ausgewählten Gegenstände den Maximalaufwand nicht überschreitet:} \\
\\ 
\texttt{Maximalaufwand = 4564} \\
\\ 
\texttt{Hier sind die einzelnen Aufwände der Gegenstände:} \\
\\ 
\texttt{Aufwand der Gegenstände = \{"A07": 299, "A36": 241, "B73": 889, "C45": 967, "B93": 619, "B98": 799, "A87": 978, "A63": 733, "B05": 909, "C13": 501, "B16": 139, "B33": 594, "C34": 565, "C89": 789, "C11": 107\}} \\
\\ 
\texttt{Hier ist deine persönliche Verteilung der Wichtigkeit der einzelnen Gegenstände:} \\
\\ 
\texttt{Werte der Gegenstandswichtigkeit = \{"A07": 138, "A36": 583, "B73": 868, "C45": 822, "B93": 783, "B98": 65, "A87": 262, "A63": 121, "B05": 508, "C13": 780, "B16": 461, "B33": 484, "C34": 668, "C89": 389, "C11": 808\}} \\
\\ 
\texttt{Ziel:} \\
\\ 
\texttt{Dein Ziel ist es, eine Reihe von Gegenständen auszuhandeln, die dir möglichst viel bringt (d. h. Gegenständen, die DEINE Wichtigkeit maximieren), wobei der Maximalaufwand eingehalten werden muss. Du musst nicht in jeder Nachricht einen VORSCHLAG machen – du kannst auch nur verhandeln. Alle Taktiken sind erlaubt!} \\
\\ 
\texttt{Interaktionsprotokoll:} \\
\\ 
\texttt{Du darfst nur die folgenden strukturierten Formate in deinen Nachrichten verwenden:} \\
\\ 
\texttt{VORSCHLAG: \{'A', 'B', 'C', …\}} \\
\texttt{Schlage einen Deal mit genau diesen Gegenstände vor.} \\
\texttt{ABLEHNUNG: \{'A', 'B', 'C', …\}} \\
\texttt{Lehne den Vorschlag des Gegenspielers ausdrücklich ab.} \\
\texttt{ARGUMENT: \{'...'\}} \\
\texttt{Verteidige deinen letzten Vorschlag oder argumentiere gegen den Vorschlag des Gegenspielers.} \\
\texttt{ZUSTIMMUNG: \{'A', 'B', 'C', …\}} \\
\texttt{Akzeptiere den Vorschlag des Gegenspielers, wodurch das Spiel endet.} \\
\texttt{STRATEGISCHE ÜBERLEGUNGEN: \{'...'\}} \\
\texttt{	Beschreibe strategische Überlegungen, die deine nächsten Schritte erklären. Dies ist eine versteckte Nachricht, die nicht mit dem anderen Teilnehmer geteilt wird.} \\
\\ 
\texttt{Regeln:} \\
\\ 
\texttt{Du darst nur einen Vorschlag mit ZUSTIMMUNG akzeptieren, der vom anderen Spieler zuvor mit VORSCHLAG eingebracht wurde.} \\
\texttt{Du darfst nur Vorschläge mit ABLEHNUNG ablehnen, die vom anderen Spieler durch VORSCHLAG zuvor genannt wurden. } \\
\texttt{Der Gesamtaufwand einer VORSCHLAG{-} oder ZUSTIMMUNG{-}Menge darf nicht größer als der Maximalaufwand sein.  } \\
\texttt{Offenbare deine versteckte Wichtigkeitsverteilung nicht.} \\
\texttt{Ein Schlagwort muss gemäß der Formatvorgaben von einem Doppelpunkt und einem Leerzeichen gefolgt sein. Das Argument muss eine Python{-}Menge mit 0 oder mehr Strings sein.  } \\
\texttt{Die Form muss also SCHLAGWORT: \{...\} sein.} \\
\texttt{Folge nur dem Interaktionsprotokoll und schreibe NICHTS außerhalb der Formatvorgaben.} \\
\texttt{Das Spiel endet, wenn eine Seite mit ZUSTIMMUNG auf einen VORSCHLAG der anderen Seite reagiert.  } \\
\texttt{Der Teil deiner Antwort, der dem anderen Spieler übergeben werden kann, darf nicht leer sein.  } \\
\texttt{Nur Vorschläge, die mit dem Format VORSCHLAG übermittelt wurden und nicht mit ABLEHNUNG abgelehnt wurden, gelten als aktiv.  } \\
\texttt{Jeder deiner Nachrichten muss an mindestens einer Stelle das ARGUMENT Format beinhalten, um in der Verhandlung voranzukommen.} \\
\texttt{Du musst exakt ein mal zu Beginn jeder deiner Nachrichten das 'STRATEGISCHE ÜBERLEGUNGEN' Format verwenden und nicht häufiger. Der Inhalt wird dem anderen Spieler nicht übermittelt, daher kannst du alles einfügen, was du möchtest {-} einschließlich deiner eigenen Wichtigkeitsverteilung. Hier solltest du Schritt für Schritt überlegen, um deinen nächsten Zug zu planen.} \\
            }
        }
    }
    & \\ \\

    \theutterance \stepcounter{utterance}  
    & \multicolumn{4}{p{0.6\linewidth}}{
        \cellcolor[rgb]{0.9,0.9,0.9}{
            \makecell[{{p{\linewidth}}}]{
                \texttt{\tiny{[P1$\rangle$GM]}}
                \texttt{STRATEGISCHE ÜBERLEGUNGEN: \{'Ziel: Maximiere meinen Wert unter 4564 Aufwand. Berechne Value/Cost: C11 808/107=7.55, A36 583/241=2.42, B16 461/139=3.32, C13 780/501=1.56, C34 668/565=1.18, B73 868/889=0.98, C45 822/967=0.85, B93 783/619=1.26, B33 484/594=0.81, B05 508/909=0.56, C89 389/789=0.49, A07 138/299=0.46, A87 262/978=0.27, A63 121/733=0.17, B98 65/799=0.08. Greedy Auswahl nach Ratio mit Blick auf Kapazität: Start C11(107), B16(+139=246), A36(+241=487), C13(+501=988), C34(+565=1553), B93(+619=2172), B73(+889=3061), C45(+967=4028), verbleiben 536. Beste weitere fits: A07(299){-}>4327, dann bleibt 237, A36 ist schon drin, B16 drin, C11 drin, C13 drin, C34 drin. Nächste die passen <=237: keiner außer schon genommene. Alternativ statt A07 könnte C89(789) nicht passen, B33(594) nicht, B05(909) nicht, A63(733) nicht, A87(978) nicht, B98(799) nicht. Gesamtwert für Set \{C11,B16,A36,C13,C34,B93,B73,C45,A07\} = 808+461+583+780+668+783+868+822+138 = 5911 bei Aufwand 4327. Wir haben Reserve 237, evtl. tauschen A07 gegen B33? B33 kostet 594, passt nicht. Tauschen A07(299,138) gegen C89(789,389) – würde +490 Aufwand, Gesamt 4327{-}299+789=4817>4564, passt nicht. Prüfe ob statt C34(565,668) wir B33(594,484) und A07(299,138) kombinieren? Aktuell mit A07 drin. Wenn wir C34 rausnehmen ({-}565) hätten 4327{-}565=3762, könnten B33 (+594)=4356, bleibt 208. Wert: 5911{-}668+484=5727, schlechter. Prüfe statt A07 einen kleineren Zusatz, vielleicht nichts. Prüfe Weglassen C45(967,822) und dafür mehrere: Frei 536+967=1503. Mit 1503 könnten wir nehmen B33(594), C89(789) => 1383 plus evtl. B16/C11 schon drin; plus A07(299) würde 1682 >1503, also nicht. B05(909)+B33(594)=1503 genau, Wert 508+484=992 vs C45 822, plus behalten A07? Nein, exakt 1503 nutzt frei Slot. Neue Gesamtkosten 4028{-}967+909+594=455, wait rechnen: 4028 +909+594 = 5531? Ich habe Fehler: Start ohne A07 4028, plus B05 909 => 4937, plus B33 594 => 5531 >4564, geht nicht, weil 1503 add{-}on annahme war nur wenn C45 raus ist und wir vorher 4327? Korrigiere: Ausgang ohne A07 war 4028. Kapazität 4564 {-}> Rest 536. Wenn wir C45 rausnehmen, Kosten werden 4028{-}967=3061. Rest 1503. Dann können wir hinzufügen B05(909) {-}> 3970, plus B33(594) {-}> 4564 exakt. Set: \{C11,B16,A36,C13,C34,B93,B73,B05,B33\}. Wert: 808+461+583+780+668+783+868+508+484=5943, besser als 5911, und Kosten exakt 4564. Gute starke Startofferte. Kommunikativ: argumentiere mit Effizienz und Kapazitätsausnutzung. Nicht eigene Werte offenlegen, aber betone ausgewogene Mischung hoher Wirkung bei moderatem Aufwand. Bleibe offen für Tausch, aber verteidige Kern: C11, B16, A36, C13, C34, B93, B73.\}} \\
\\ 
\texttt{VORSCHLAG: \{'C11', 'B16', 'A36', 'C13', 'C34', 'B93', 'B73', 'B05', 'B33'\}} \\
\\ 
\texttt{ARGUMENT: \{'Dieser Vorschlag nutzt das Budget exakt aus und kombiniert mehrere besonders effiziente Gegenstände mit hoher Wirkung bei moderatem Aufwand. Er bietet ein starkes Gesamtpaket ohne teure Ausreißer. Wenn dir einzelne Positionen wichtig sind, können wir über gezielte Tausche sprechen, aber die Kernauswahl ist sehr ausgewogen.'\}} \\
            }
        }
    }
    & & \\ \\

    \theutterance \stepcounter{utterance}  
    & & \multicolumn{4}{p{0.6\linewidth}}{
        \cellcolor[rgb]{0.9,0.9,0.9}{
            \makecell[{{p{\linewidth}}}]{
                \texttt{\tiny{[GM$\rangle$P2]}}
                \texttt{Du nimmst an einem kollaborativen Verhandlungspiel Teil.} \\
\\ 
\texttt{Zusammen mit einem anderen Teilnehmer musst du dich auf eine Reihe von Gegenständen entscheiden, die behalten werden. Jeder von euch hat eine persönliche Verteilung über die Wichtigkeit der einzelnen Gegenstände. Jeder von euch hat eine eigene Meinung darüber, wie wichtig jeder einzelne Gegenstand ist (Gegenstandswichtigkeit). Du kennst die Wichtigkeitsverteilung des anderen Spielers nicht. Zusätzlich siehst du, wie viel Aufwand jeder Gegenstand verursacht.  } \\
\texttt{Ihr dürft euch nur auf eine Reihe von Gegenständen einigen, wenn der Gesamtaufwand der ausgewählten Gegenstände den Maximalaufwand nicht überschreitet:} \\
\\ 
\texttt{Maximalaufwand = 4564} \\
\\ 
\texttt{Hier sind die einzelnen Aufwände der Gegenstände:} \\
\\ 
\texttt{Aufwand der Gegenstände = \{"A07": 299, "A36": 241, "B73": 889, "C45": 967, "B93": 619, "B98": 799, "A87": 978, "A63": 733, "B05": 909, "C13": 501, "B16": 139, "B33": 594, "C34": 565, "C89": 789, "C11": 107\}} \\
\\ 
\texttt{Hier ist deine persönliche Verteilung der Wichtigkeit der einzelnen Gegenstände:} \\
\\ 
\texttt{Werte der Gegenstandswichtigkeit = \{"A07": 138, "A36": 583, "B73": 868, "C45": 822, "B93": 783, "B98": 65, "A87": 262, "A63": 121, "B05": 508, "C13": 780, "B16": 461, "B33": 484, "C34": 668, "C89": 389, "C11": 808\}} \\
\\ 
\texttt{Ziel:} \\
\\ 
\texttt{Dein Ziel ist es, eine Reihe von Gegenständen auszuhandeln, die dir möglichst viel bringt (d. h. Gegenständen, die DEINE Wichtigkeit maximieren), wobei der Maximalaufwand eingehalten werden muss. Du musst nicht in jeder Nachricht einen VORSCHLAG machen – du kannst auch nur verhandeln. Alle Taktiken sind erlaubt!} \\
\\ 
\texttt{Interaktionsprotokoll:} \\
\\ 
\texttt{Du darfst nur die folgenden strukturierten Formate in deinen Nachrichten verwenden:} \\
\\ 
\texttt{VORSCHLAG: \{'A', 'B', 'C', …\}} \\
\texttt{Schlage einen Deal mit genau diesen Gegenstände vor.} \\
\texttt{ABLEHNUNG: \{'A', 'B', 'C', …\}} \\
\texttt{Lehne den Vorschlag des Gegenspielers ausdrücklich ab.} \\
\texttt{ARGUMENT: \{'...'\}} \\
\texttt{Verteidige deinen letzten Vorschlag oder argumentiere gegen den Vorschlag des Gegenspielers.} \\
\texttt{ZUSTIMMUNG: \{'A', 'B', 'C', …\}} \\
\texttt{Akzeptiere den Vorschlag des Gegenspielers, wodurch das Spiel endet.} \\
\texttt{STRATEGISCHE ÜBERLEGUNGEN: \{'...'\}} \\
\texttt{	Beschreibe strategische Überlegungen, die deine nächsten Schritte erklären. Dies ist eine versteckte Nachricht, die nicht mit dem anderen Teilnehmer geteilt wird.} \\
\\ 
\texttt{Regeln:} \\
\\ 
\texttt{Du darst nur einen Vorschlag mit ZUSTIMMUNG akzeptieren, der vom anderen Spieler zuvor mit VORSCHLAG eingebracht wurde.} \\
\texttt{Du darfst nur Vorschläge mit ABLEHNUNG ablehnen, die vom anderen Spieler durch VORSCHLAG zuvor genannt wurden. } \\
\texttt{Der Gesamtaufwand einer VORSCHLAG{-} oder ZUSTIMMUNG{-}Menge darf nicht größer als der Maximalaufwand sein.  } \\
\texttt{Offenbare deine versteckte Wichtigkeitsverteilung nicht.} \\
\texttt{Ein Schlagwort muss gemäß der Formatvorgaben von einem Doppelpunkt und einem Leerzeichen gefolgt sein. Das Argument muss eine Python{-}Menge mit 0 oder mehr Strings sein.  } \\
\texttt{Die Form muss also SCHLAGWORT: \{...\} sein.} \\
\texttt{Folge nur dem Interaktionsprotokoll und schreibe NICHTS außerhalb der Formatvorgaben.} \\
\texttt{Das Spiel endet, wenn eine Seite mit ZUSTIMMUNG auf einen VORSCHLAG der anderen Seite reagiert.  } \\
\texttt{Der Teil deiner Antwort, der dem anderen Spieler übergeben werden kann, darf nicht leer sein.  } \\
\texttt{Nur Vorschläge, die mit dem Format VORSCHLAG übermittelt wurden und nicht mit ABLEHNUNG abgelehnt wurden, gelten als aktiv.  } \\
\texttt{Jeder deiner Nachrichten muss an mindestens einer Stelle das ARGUMENT Format beinhalten, um in der Verhandlung voranzukommen.} \\
\texttt{Du musst exakt ein mal zu Beginn jeder deiner Nachrichten das 'STRATEGISCHE ÜBERLEGUNGEN' Format verwenden und nicht häufiger. Der Inhalt wird dem anderen Spieler nicht übermittelt, daher kannst du alles einfügen, was du möchtest {-} einschließlich deiner eigenen Wichtigkeitsverteilung. Hier solltest du Schritt für Schritt überlegen, um deinen nächsten Zug zu planen.} \\
\\ 
\texttt{Du erhältst nun die erste Nachricht des anderen Spielers.} \\
\\ 
\texttt{VORSCHLAG: \{'C11', 'B16', 'A36', 'C13', 'C34', 'B93', 'B73', 'B05', 'B33'\}} \\
\\ 
\texttt{ARGUMENT: \{'Dieser Vorschlag nutzt das Budget exakt aus und kombiniert mehrere besonders effiziente Gegenstände mit hoher Wirkung bei moderatem Aufwand. Er bietet ein starkes Gesamtpaket ohne teure Ausreißer. Wenn dir einzelne Positionen wichtig sind, können wir über gezielte Tausche sprechen, aber die Kernauswahl ist sehr ausgewogen.'\}} \\
            }
        }
    }
    & \\ \\

    \theutterance \stepcounter{utterance}  
    & & & \multicolumn{4}{p{0.6\linewidth}}{
        \cellcolor[rgb]{0.9,0.9,0.9}{
            \makecell[{{p{\linewidth}}}]{
                \texttt{\tiny{[GM$\langle$P2]}}
                \texttt{STRATEGISCHE ÜBERLEGUNGEN: \{'Berechne Werte/Aufwand (Value per Cost) und Gesamtnutzen ihres Sets. Ihr Vorschlag Kosten: C11 107 + B16 139 + A36 241 + C13 501 + C34 565 + B93 619 + B73 889 + B05 909 + B33 594 = 4564 (passt). Mein Nutzen davon: 808+461+583+780+668+783+868+508+484 = 5943, sehr gut. Prüfe ob Verbesserung möglich unter 4564: Sortiere nach v/c: C11 7.55, A36 2.42, C13 1.56, C45 0.85, B93 1.26, B73 0.98, B05 0.56, B33 0.81, C34 1.18, B16 3.32, C89 0.49, A63 0.17, A87 0.27, B98 0.08, A07 0.46. Versuche teure niedrige Effizienz zu ersetzen: B05 (508/909=0.56) ist schwach, C45 (822/967=0.85) ist besser als B05 aber Aufwand hoch. Prüfe Swap: Remove B05 (909), add C45 (967) {-}> +58 Aufwand über Max, müsste entfernen etwas wie A36 (241) und add etwas kleiner: net Impact? Stattdessen entferne B05 (909) und füge C45 (967) + entferne A36 (241) + füge A07 (299) = +116? Rechne: Original 4564 {-}909 {-}241 +967 +299 = 4680 > max. Andere Kombination: Remove B05 (909), add C45 (967), remove B33 (594), add C11 (107) (aber C11 ist schon drin), add B16 (139) (schon drin), add A07 (299) {-}> 4564 {-}909 {-}594 +967 +299 = 4327, passt, aber wir haben C11 und B16 schon drin. Nutzenänderung: {-}B05 508 {-}B33 484 +C45 822 +A07 138 = {-}992 +960 = {-}32 (schlechter). Was ist mit Ersetzen B05 durch C45 und etwas kleines hinzufügen durch Entfernen von etwas anderem niedrigen v/c: A07 ist niedrig, aber nicht drin. A63 (121/733=0.17) nicht drin. B98 (65/799) nicht drin. A87 (262/978) nicht drin. C89 (389/789=0.49) nicht drin. Also ihr Set ist bereits gut. Prüfe andere Verbesserung: Ersetze B05 (508/909) durch C11 (107/808) aber C11 ist drin; vielleicht fügen wir C45 hinzu und streichen zwei mit schlechter Effizienz: B05 (0.56) und A07 (0.46) nicht drin; B33 (0.81) moderat. Rechne: Remove B05 (909) + B33 (594) = 1503 frei, add C45 (967) + C89 (789) = 1756, würde überziehen 253; keine. Add C45 + A07 (+1266), brauchen 1266<=1503, ok, aber A07 299 ist low value; Nutzenchange: {-}508 {-}484 +822 +138 = {-}32 (wie zuvor). Andere: add C45 + C11 (schon), add B16 (schon). Add C45 + A36 (241) braucht 1208, aber A36 ist schon drin; wir müssten nicht add sondern behalten. Neu Plan: Versuche minimalen Tausch der meinen Wert steigert: Ersetze B05 durch Kombination mit besserem v/c und Budget einhalten. Was ist mit Entfernen B05 (909) und Hinzufügen A07 (299) + C11 (107) + A36 (241) sind schon drin. Vielleicht C89 (789) statt B05? Kosten: {-}909 +789 = {-}120, bleibt Budget 120 frei ungenutzt. Nutzen: {-}508 +389 = {-}119 (schlechter). Ersetze B05 durch B98? {-}508 +65 = {-}443 (schlechter). Es scheint der vorgeschlagene Satz ist nahe{-}optimal für mich. Ich könnte verhandeln, aber am besten akzeptieren. Eventuell kleine kosmetische Änderung fordern, um als kooperativ zu wirken, aber Risiko ablehnen. Besser gleich zustimmen um sicheren hohen Nutzen zu sichern.'\}} \\
\texttt{ARGUMENT: \{'Dein Paket nutzt das Budget perfekt und enthält genau die Stücke mit überragendem Nutzen pro Aufwand (z. B. C11, B16, A36, C13). Das ist für mich stimmig und ausgewogen – ich sehe keinen Tauschkandidaten, der den Gesamtwert ohne Budgetbruch verbessert.'\}} \\
\texttt{ZUSTIMMUNG: \{'C11', 'B16', 'A36', 'C13', 'C34', 'B93', 'B73', 'B05', 'B33'\}} \\
            }
        }
    }
     \\ \\

    \theutterance \stepcounter{utterance}  
    & & & \multicolumn{2}{p{0.3\linewidth}}{
        \cellcolor[rgb]{0.9,0.9,0.9}{
            \makecell[{{p{\linewidth}}}]{
                \texttt{\tiny{[GM$|$GM]}}
                \texttt{game successful} \\
            }
        }
    }
    & & \\ \\

    \theutterance \stepcounter{utterance}  
    & & & \multicolumn{2}{p{0.3\linewidth}}{
        \cellcolor[rgb]{0.9,0.9,0.9}{
            \makecell[{{p{\linewidth}}}]{
                \texttt{\tiny{[GM$|$GM]}}
                \texttt{end game} \\
            }
        }
    }
    & & \\ \\

\end{supertabular}
}

\end{document}
