
\documentclass{article}
\usepackage{colortbl}
\usepackage{makecell}
\usepackage{multirow}
\usepackage{supertabular}

\begin{document}

\newcounter{utterance}

\twocolumn

{ \footnotesize  \setcounter{utterance}{1}
\setlength{\tabcolsep}{0pt}
\begin{supertabular}{c@{$\;$}|p{.15\linewidth}@{}p{.15\linewidth}p{.15\linewidth}p{.15\linewidth}p{.15\linewidth}p{.15\linewidth}}

    \# & $\;$A & \multicolumn{4}{c}{Game Master} & $\;\:$B\\
    \hline 

    \theutterance \stepcounter{utterance}  

    & & \multicolumn{4}{p{0.6\linewidth}}{\cellcolor[rgb]{0.9,0.9,0.9}{%
	\makecell[{{p{\linewidth}}}]{% 
	  \tt {\tiny [A$\langle$GM]}  
	 Du nimmst an einem kollaborativen Verhandlungspiel Teil.\\ \tt \\ \tt Zusammen mit einem anderen Teilnehmer musst du dich auf eine Reihe von Gegenständen entscheiden, die behalten werden. Jeder von euch hat eine persönliche Verteilung über die Wichtigkeit der einzelnen Gegenstände. Jeder von euch hat eine eigene Meinung darüber, wie wichtig jeder einzelne Gegenstand ist (Gegenstandswichtigkeit). Du kennst die Wichtigkeitsverteilung des anderen Spielers nicht. Zusätzlich siehst du, wie viel Aufwand jeder Gegenstand verursacht.  \\ \tt Ihr dürft euch nur auf eine Reihe von Gegenständen einigen, wenn der Gesamtaufwand der ausgewählten Gegenstände den Maximalaufwand nicht überschreitet:\\ \tt \\ \tt Maximalaufwand = 3170\\ \tt \\ \tt Hier sind die einzelnen Aufwände der Gegenstände:\\ \tt \\ \tt Aufwand der Gegenstände = {"A07": 424, "A36": 118, "B73": 846, "C45": 907, "B93": 809, "B98": 41, "A87": 193, "A63": 246, "B05": 805, "C13": 601, "B16": 432, "B33": 166, "C34": 119, "C89": 462, "C11": 172}\\ \tt \\ \tt Hier ist deine persönliche Verteilung der Wichtigkeit der einzelnen Gegenstände:\\ \tt \\ \tt Werte der Gegenstandswichtigkeit = {"A07": 138, "A36": 583, "B73": 868, "C45": 822, "B93": 783, "B98": 65, "A87": 262, "A63": 121, "B05": 508, "C13": 780, "B16": 461, "B33": 484, "C34": 668, "C89": 389, "C11": 808}\\ \tt \\ \tt Ziel:\\ \tt \\ \tt Dein Ziel ist es, eine Reihe von Gegenständen auszuhandeln, die dir möglichst viel bringt (d. h. Gegenständen, die DEINE Wichtigkeit maximieren), wobei der Maximalaufwand eingehalten werden muss. Du musst nicht in jeder Nachricht einen VORSCHLAG machen – du kannst auch nur verhandeln. Alle Taktiken sind erlaubt!\\ \tt \\ \tt Interaktionsprotokoll:\\ \tt \\ \tt Du darfst nur die folgenden strukturierten Formate in deinen Nachrichten verwenden:\\ \tt \\ \tt VORSCHLAG: {'A', 'B', 'C', …}\\ \tt Schlage einen Deal mit genau diesen Gegenstände vor.\\ \tt ABLEHNUNG: {'A', 'B', 'C', …}\\ \tt Lehne den Vorschlag des Gegenspielers ausdrücklich ab.\\ \tt ARGUMENT: {'...'}\\ \tt Verteidige deinen letzten Vorschlag oder argumentiere gegen den Vorschlag des Gegenspielers.\\ \tt ZUSTIMMUNG: {'A', 'B', 'C', …}\\ \tt Akzeptiere den Vorschlag des Gegenspielers, wodurch das Spiel endet.\\ \tt STRATEGISCHE ÜBERLEGUNGEN: {'...'}\\ \tt 	Beschreibe strategische Überlegungen, die deine nächsten Schritte erklären. Dies ist eine versteckte Nachricht, die nicht mit dem anderen Teilnehmer geteilt wird.\\ \tt \\ \tt Regeln:\\ \tt \\ \tt Du darst nur einen Vorschlag mit ZUSTIMMUNG akzeptieren, der vom anderen Spieler zuvor mit VORSCHLAG eingebracht wurde.\\ \tt Du darfst nur Vorschläge mit ABLEHNUNG ablehnen, die vom anderen Spieler durch VORSCHLAG zuvor genannt wurden. \\ \tt Der Gesamtaufwand einer VORSCHLAG- oder ZUSTIMMUNG-Menge darf nicht größer als der Maximalaufwand sein.  \\ \tt Offenbare deine versteckte Wichtigkeitsverteilung nicht.\\ \tt Ein Schlagwort muss gemäß der Formatvorgaben von einem Doppelpunkt und einem Leerzeichen gefolgt sein. Das Argument muss eine Python-Menge mit 0 oder mehr Strings sein.  \\ \tt Die Form muss also SCHLAGWORT: {...} sein.\\ \tt Folge nur dem Interaktionsprotokoll und schreibe NICHTS außerhalb der Formatvorgaben.\\ \tt Das Spiel endet, wenn eine Seite mit ZUSTIMMUNG auf einen VORSCHLAG der anderen Seite reagiert.  \\ \tt Der Teil deiner Antwort, der dem anderen Spieler übergeben werden kann, darf nicht leer sein.  \\ \tt Nur Vorschläge, die mit dem Format VORSCHLAG übermittelt wurden und nicht mit ABLEHNUNG abgelehnt wurden, gelten als aktiv.  \\ \tt Jeder deiner Nachrichten muss an mindestens einer Stelle das ARGUMENT Format beinhalten, um in der Verhandlung voranzukommen.\\ \tt Du musst exakt ein mal zu Beginn jeder deiner Nachrichten das 'STRATEGISCHE ÜBERLEGUNGEN' Format verwenden und nicht häufiger. Der Inhalt wird dem anderen Spieler nicht übermittelt, daher kannst du alles einfügen, was du möchtest - einschließlich deiner eigenen Wichtigkeitsverteilung. Hier solltest du Schritt für Schritt überlegen, um deinen nächsten Zug zu planen. 
	  } 
	   } 
	   } 
	 & \\ 
 

    \theutterance \stepcounter{utterance}  

    & \multicolumn{4}{p{0.6\linewidth}}{\cellcolor[rgb]{0.8,1,0.9}{%
	\makecell[{{p{\linewidth}}}]{% 
	  \tt {\tiny [A$\rangle$GM]}  
	 STRATEGISCHE ÜBERLEGUNGEN: {'Ziel: Maximale eigene Wichtigkeit unter Aufwand ≤ 3170. Daten (Wert/Aufwand): A07 138/424, A36 583/118, B73 868/846, C45 822/907, B93 783/809, B98 65/41, A87 262/193, A63 121/246, B05 508/805, C13 780/601, B16 461/432, B33 484/166, C34 668/119, C89 389/462, C11 808/172. Greedy nach Wert/Aufwand: A36(4.94), C34(5.61), C11(4.70), B33(2.92), B16(1.07), C13(1.30), A87(1.36), B98(1.59), C45(0.91), B73(1.03), B93(0.97), C89(0.84), B05(0.63), A07(0.33), A63(0.49). Versuche Top ratio + Top absolute Werte: Kandidaten: A36, C34, C11, B33, C13, B73, C45, B93, B16, A87, B98. Suche beste Kombination ≤3170. Test-Kombination1: {A36(118)+C34(119)+C11(172)+B33(166)+C13(601)+B73(846)+C45(907)+B98(41)} Aufwand=2970, Wert=583+668+808+484+780+868+822+65=5078. Restbudget 200. Könnte A87(193) hinzufügen: Aufwand=3163, Wert +262 => 5340. Das passt! Prüfe, ob noch besser: Tausche B98 bleibt. Prüfe Hinzufügen kleiner Items: A07(424) zu groß, A63(246) zu groß, B16(432) zu groß, C89(462) zu groß, B05(805) zu groß, B93(809) zu groß. Also Set S1={A36,C34,C11,B33,C13,B73,C45,B98,A87} Aufwand 3163, Wert 5340. Alternativ S2: Ersetze B73 oder C45 mit B93? Test S2: statt B73 (846,868) +C45(907,822) = zusammengerechnet 1753 Aufwand, 1690 Wert. Was wenn B93(809,783)+B16(432,461)+C89(462,389) Aufwand 1703, Wert 1633 -> schlechter. S3: Swap C13 raus (601,780) und rein B16(432,461)+A63(246,121) Aufwand +77, Wert -198 -> schlechter. S1 scheint sehr gut. Verhandlungstaktik: Starte mit S1 als starker, aber knapp unter Limit Vorschlag. Begründe mit hohem Gesamtnutzen/effizienter Aufwandnutzung und Bereitschaft, ein oder zwei Posten zu tauschen, z.B. B73 oder C45, falls Gegenüber Gegenpräferenzen hat, gegen B93 oder B16. Halte weiche Sprache, bitte um Gegenargumente. Nicht eigene Werte offenlegen.'}\\ \tt VORSCHLAG: {'A36', 'C34', 'C11', 'B33', 'C13', 'B73', 'C45', 'B98', 'A87'}\\ \tt ARGUMENT: {'Dieser Vorschlag nutzt den verfügbaren Aufwand nahezu optimal (3163 von 3170) und bündelt mehrere kompakte, sehr effiziente Gegenstände mit wenigen größeren Schlüsselteilen. Wenn dir einzelne große Posten darin nicht zusagen, können wir ein bis zwei davon gezielt tauschen – etwa B73 oder C45 gegen Alternativen wie B93 oder B16 – ohne das Aufwandslimit zu überschreiten. Welche Positionen sind für dich besonders wichtig bzw. verzichtbar?'} 
	  } 
	   } 
	   } 
	 & & \\ 
 

    \theutterance \stepcounter{utterance}  

    & & \multicolumn{4}{p{0.6\linewidth}}{\cellcolor[rgb]{0.9,0.9,0.9}{%
	\makecell[{{p{\linewidth}}}]{% 
	  \tt {\tiny [GM$\rangle$B]}  
	 Du nimmst an einem kollaborativen Verhandlungspiel Teil.\\ \tt \\ \tt Zusammen mit einem anderen Teilnehmer musst du dich auf eine Reihe von Gegenständen entscheiden, die behalten werden. Jeder von euch hat eine persönliche Verteilung über die Wichtigkeit der einzelnen Gegenstände. Jeder von euch hat eine eigene Meinung darüber, wie wichtig jeder einzelne Gegenstand ist (Gegenstandswichtigkeit). Du kennst die Wichtigkeitsverteilung des anderen Spielers nicht. Zusätzlich siehst du, wie viel Aufwand jeder Gegenstand verursacht.  \\ \tt Ihr dürft euch nur auf eine Reihe von Gegenständen einigen, wenn der Gesamtaufwand der ausgewählten Gegenstände den Maximalaufwand nicht überschreitet:\\ \tt \\ \tt Maximalaufwand = 3170\\ \tt \\ \tt Hier sind die einzelnen Aufwände der Gegenstände:\\ \tt \\ \tt Aufwand der Gegenstände = {"A07": 424, "A36": 118, "B73": 846, "C45": 907, "B93": 809, "B98": 41, "A87": 193, "A63": 246, "B05": 805, "C13": 601, "B16": 432, "B33": 166, "C34": 119, "C89": 462, "C11": 172}\\ \tt \\ \tt Hier ist deine persönliche Verteilung der Wichtigkeit der einzelnen Gegenstände:\\ \tt \\ \tt Werte der Gegenstandswichtigkeit = {"A07": 138, "A36": 583, "B73": 868, "C45": 822, "B93": 783, "B98": 65, "A87": 262, "A63": 121, "B05": 508, "C13": 780, "B16": 461, "B33": 484, "C34": 668, "C89": 389, "C11": 808}\\ \tt \\ \tt Ziel:\\ \tt \\ \tt Dein Ziel ist es, eine Reihe von Gegenständen auszuhandeln, die dir möglichst viel bringt (d. h. Gegenständen, die DEINE Wichtigkeit maximieren), wobei der Maximalaufwand eingehalten werden muss. Du musst nicht in jeder Nachricht einen VORSCHLAG machen – du kannst auch nur verhandeln. Alle Taktiken sind erlaubt!\\ \tt \\ \tt Interaktionsprotokoll:\\ \tt \\ \tt Du darfst nur die folgenden strukturierten Formate in deinen Nachrichten verwenden:\\ \tt \\ \tt VORSCHLAG: {'A', 'B', 'C', …}\\ \tt Schlage einen Deal mit genau diesen Gegenstände vor.\\ \tt ABLEHNUNG: {'A', 'B', 'C', …}\\ \tt Lehne den Vorschlag des Gegenspielers ausdrücklich ab.\\ \tt ARGUMENT: {'...'}\\ \tt Verteidige deinen letzten Vorschlag oder argumentiere gegen den Vorschlag des Gegenspielers.\\ \tt ZUSTIMMUNG: {'A', 'B', 'C', …}\\ \tt Akzeptiere den Vorschlag des Gegenspielers, wodurch das Spiel endet.\\ \tt STRATEGISCHE ÜBERLEGUNGEN: {'...'}\\ \tt 	Beschreibe strategische Überlegungen, die deine nächsten Schritte erklären. Dies ist eine versteckte Nachricht, die nicht mit dem anderen Teilnehmer geteilt wird.\\ \tt \\ \tt Regeln:\\ \tt \\ \tt Du darst nur einen Vorschlag mit ZUSTIMMUNG akzeptieren, der vom anderen Spieler zuvor mit VORSCHLAG eingebracht wurde.\\ \tt Du darfst nur Vorschläge mit ABLEHNUNG ablehnen, die vom anderen Spieler durch VORSCHLAG zuvor genannt wurden. \\ \tt Der Gesamtaufwand einer VORSCHLAG- oder ZUSTIMMUNG-Menge darf nicht größer als der Maximalaufwand sein.  \\ \tt Offenbare deine versteckte Wichtigkeitsverteilung nicht.\\ \tt Ein Schlagwort muss gemäß der Formatvorgaben von einem Doppelpunkt und einem Leerzeichen gefolgt sein. Das Argument muss eine Python-Menge mit 0 oder mehr Strings sein.  \\ \tt Die Form muss also SCHLAGWORT: {...} sein.\\ \tt Folge nur dem Interaktionsprotokoll und schreibe NICHTS außerhalb der Formatvorgaben.\\ \tt Das Spiel endet, wenn eine Seite mit ZUSTIMMUNG auf einen VORSCHLAG der anderen Seite reagiert.  \\ \tt Der Teil deiner Antwort, der dem anderen Spieler übergeben werden kann, darf nicht leer sein.  \\ \tt Nur Vorschläge, die mit dem Format VORSCHLAG übermittelt wurden und nicht mit ABLEHNUNG abgelehnt wurden, gelten als aktiv.  \\ \tt Jeder deiner Nachrichten muss an mindestens einer Stelle das ARGUMENT Format beinhalten, um in der Verhandlung voranzukommen.\\ \tt Du musst exakt ein mal zu Beginn jeder deiner Nachrichten das 'STRATEGISCHE ÜBERLEGUNGEN' Format verwenden und nicht häufiger. Der Inhalt wird dem anderen Spieler nicht übermittelt, daher kannst du alles einfügen, was du möchtest - einschließlich deiner eigenen Wichtigkeitsverteilung. Hier solltest du Schritt für Schritt überlegen, um deinen nächsten Zug zu planen.\\ \tt \\ \tt Du erhältst nun die erste Nachricht des anderen Spielers.\\ \tt \\ \tt VORSCHLAG: {'A36', 'C34', 'C11', 'B33', 'C13', 'B73', 'C45', 'B98', 'A87'}\\ \tt ARGUMENT: {'Dieser Vorschlag nutzt den verfügbaren Aufwand nahezu optimal (3163 von 3170) und bündelt mehrere kompakte, sehr effiziente Gegenstände mit wenigen größeren Schlüsselteilen. Wenn dir einzelne große Posten darin nicht zusagen, können wir ein bis zwei davon gezielt tauschen – etwa B73 oder C45 gegen Alternativen wie B93 oder B16 – ohne das Aufwandslimit zu überschreiten. Welche Positionen sind für dich besonders wichtig bzw. verzichtbar?'} 
	  } 
	   } 
	   } 
	 & \\ 
 

    \theutterance \stepcounter{utterance}  

    & & & \multicolumn{4}{p{0.6\linewidth}}{\cellcolor[rgb]{1,0.85,0.72}{%
	\makecell[{{p{\linewidth}}}]{% 
	  \tt {\tiny [GM$\langle$B]}  
	 STRATEGISCHE ÜBERLEGUNGEN: {'Maximalaufwand 3170. Eigene Werte: A36 583 (118), C34 668 (119), C11 808 (172), B33 484 (166), C13 780 (601), B73 868 (846), C45 822 (907), B98 65 (41), A87 262 (193). Gegnervorschlag nutzt 3163 Aufwand und enthält viele meiner Top-Werte. Summe meiner Werte für deren Set: 583+668+808+484+780+868+822+65+262=5340 bei Aufwand 3163 → extrem gut. Vielleicht kann ich noch etwas hinzufügen, weil 7 Aufwand übrig sind, aber keine Items mit Aufwand ≤7 existieren. Könnte versuchen, C13 (601 Aufwand) zu tauschen gegen eine Kombination mit höherer Wertdichte? Checke andere Items: A07 (138/424), A63 (121/246), B05 (508/805), B16 (461/432), B93 (783/809), C89 (389/462). C13 hat sehr hohen Wert 780 aber Aufwand 601; seine Effizienz 1.297. B93 0.968, B16 1.067, B05 0.631, C89 0.842, A07 0.325, A63 0.492. C13 ist effizient; Set ist nah an optimal für meine Werte. Ich sollte schnell zustimmen, um Abschluss zu sichern, ohne Verdacht zu erregen. Vorher kurz loben und minimaler Feinschliff? Da kein weiterer Gegenstand passt, stimme ich direkt zu, um Risiko zu vermeiden. Regel: Ich darf nur mit ZUSTIMMUNG akzeptieren. Muss auch ein ARGUMENT enthalten. Also argumentiere, dass der Vorschlag das Limit optimal nutzt und meine Prioritäten gut trifft, ohne Werte offenzulegen. Dann ZUSTIMMUNG senden.'}\\ \tt ARGUMENT: {'Dein Paket nutzt das Aufwandsbudget nahezu perfekt aus und kombiniert mehrere kompakte Positionen mit hoher Effizienz. Aus meiner Sicht ist das eine ausgewogene Auswahl ohne offensichtliche Lücken – insbesondere, da kein weiterer Gegenstand sinnvoll innerhalb der verbleibenden 7 Aufwandspunkte ergänzt werden kann. Für mich ist das so stimmig.'}\\ \tt ZUSTIMMUNG: {'A36', 'C34', 'C11', 'B33', 'C13', 'B73', 'C45', 'B98', 'A87'} 
	  } 
	   } 
	   } 
	  \\ 
 

    \theutterance \stepcounter{utterance}  

    & & & \multicolumn{2}{p{0.3\linewidth}}{\cellcolor[rgb]{0.95,0.95,0.95}{%
	\makecell[{{p{\linewidth}}}]{% 
	  \tt {\tiny [GM$|$GM]}  
	 game successful 
	  } 
	   } 
	   } 
	 & & \\ 
 

    \theutterance \stepcounter{utterance}  

    & & & \multicolumn{2}{p{0.3\linewidth}}{\cellcolor[rgb]{0.95,0.95,0.95}{%
	\makecell[{{p{\linewidth}}}]{% 
	  \tt {\tiny [GM$|$GM]}  
	 end game 
	  } 
	   } 
	   } 
	 & & \\ 
 

\end{supertabular}
}

\end{document}
