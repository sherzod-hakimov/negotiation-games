
\documentclass{article}
\usepackage{colortbl}
\usepackage{makecell}
\usepackage{multirow}
\usepackage{supertabular}

\begin{document}

\newcounter{utterance}

\twocolumn

{ \footnotesize  \setcounter{utterance}{1}
\setlength{\tabcolsep}{0pt}
\begin{supertabular}{c@{$\;$}|p{.15\linewidth}@{}p{.15\linewidth}p{.15\linewidth}p{.15\linewidth}p{.15\linewidth}p{.15\linewidth}}

    \# & $\;$A & \multicolumn{4}{c}{Game Master} & $\;\:$B\\
    \hline 

    \theutterance \stepcounter{utterance}  

    & & \multicolumn{4}{p{0.6\linewidth}}{\cellcolor[rgb]{0.9,0.9,0.9}{%
	\makecell[{{p{\linewidth}}}]{% 
	  \tt {\tiny [A$\langle$GM]}  
	 Du nimmst an einem kollaborativen Verhandlungspiel Teil.\\ \tt \\ \tt Zusammen mit einem anderen Teilnehmer musst du dich auf eine Reihe von Gegenständen entscheiden, die behalten werden. Jeder von euch hat eine persönliche Verteilung über die Wichtigkeit der einzelnen Gegenstände. Jeder von euch hat eine eigene Meinung darüber, wie wichtig jeder einzelne Gegenstand ist (Gegenstandswichtigkeit). Du kennst die Wichtigkeitsverteilung des anderen Spielers nicht. Zusätzlich siehst du, wie viel Aufwand jeder Gegenstand verursacht.  \\ \tt Ihr dürft euch nur auf eine Reihe von Gegenständen einigen, wenn der Gesamtaufwand der ausgewählten Gegenstände den Maximalaufwand nicht überschreitet:\\ \tt \\ \tt Maximalaufwand = 5138\\ \tt \\ \tt Hier sind die einzelnen Aufwände der Gegenstände:\\ \tt \\ \tt Aufwand der Gegenstände = {"C33": 903, "A89": 757, "A48": 850, "A04": 746, "A41": 524, "C64": 204, "C49": 946, "A61": 473, "C04": 616, "B53": 855, "B85": 530, "B61": 419, "A45": 960, "A03": 763, "B21": 730}\\ \tt \\ \tt Hier ist deine persönliche Verteilung der Wichtigkeit der einzelnen Gegenstände:\\ \tt \\ \tt Werte der Gegenstandswichtigkeit = {"C33": 138, "A89": 583, "A48": 868, "A04": 822, "A41": 783, "C64": 65, "C49": 262, "A61": 121, "C04": 508, "B53": 780, "B85": 461, "B61": 484, "A45": 668, "A03": 389, "B21": 808}\\ \tt \\ \tt Ziel:\\ \tt \\ \tt Dein Ziel ist es, eine Reihe von Gegenständen auszuhandeln, die dir möglichst viel bringt (d. h. Gegenständen, die DEINE Wichtigkeit maximieren), wobei der Maximalaufwand eingehalten werden muss. Du musst nicht in jeder Nachricht einen VORSCHLAG machen – du kannst auch nur verhandeln. Alle Taktiken sind erlaubt!\\ \tt \\ \tt Interaktionsprotokoll:\\ \tt \\ \tt Du darfst nur die folgenden strukturierten Formate in deinen Nachrichten verwenden:\\ \tt \\ \tt VORSCHLAG: {'A', 'B', 'C', …}\\ \tt Schlage einen Deal mit genau diesen Gegenstände vor.\\ \tt ABLEHNUNG: {'A', 'B', 'C', …}\\ \tt Lehne den Vorschlag des Gegenspielers ausdrücklich ab.\\ \tt ARGUMENT: {'...'}\\ \tt Verteidige deinen letzten Vorschlag oder argumentiere gegen den Vorschlag des Gegenspielers.\\ \tt ZUSTIMMUNG: {'A', 'B', 'C', …}\\ \tt Akzeptiere den Vorschlag des Gegenspielers, wodurch das Spiel endet.\\ \tt STRATEGISCHE ÜBERLEGUNGEN: {'...'}\\ \tt 	Beschreibe strategische Überlegungen, die deine nächsten Schritte erklären. Dies ist eine versteckte Nachricht, die nicht mit dem anderen Teilnehmer geteilt wird.\\ \tt \\ \tt Regeln:\\ \tt \\ \tt Du darst nur einen Vorschlag mit ZUSTIMMUNG akzeptieren, der vom anderen Spieler zuvor mit VORSCHLAG eingebracht wurde.\\ \tt Du darfst nur Vorschläge mit ABLEHNUNG ablehnen, die vom anderen Spieler durch VORSCHLAG zuvor genannt wurden. \\ \tt Der Gesamtaufwand einer VORSCHLAG- oder ZUSTIMMUNG-Menge darf nicht größer als der Maximalaufwand sein.  \\ \tt Offenbare deine versteckte Wichtigkeitsverteilung nicht.\\ \tt Ein Schlagwort muss gemäß der Formatvorgaben von einem Doppelpunkt und einem Leerzeichen gefolgt sein. Das Argument muss eine Python-Menge mit 0 oder mehr Strings sein.  \\ \tt Die Form muss also SCHLAGWORT: {...} sein.\\ \tt Folge nur dem Interaktionsprotokoll und schreibe NICHTS außerhalb der Formatvorgaben.\\ \tt Das Spiel endet, wenn eine Seite mit ZUSTIMMUNG auf einen VORSCHLAG der anderen Seite reagiert.  \\ \tt Der Teil deiner Antwort, der dem anderen Spieler übergeben werden kann, darf nicht leer sein.  \\ \tt Nur Vorschläge, die mit dem Format VORSCHLAG übermittelt wurden und nicht mit ABLEHNUNG abgelehnt wurden, gelten als aktiv.  \\ \tt Jeder deiner Nachrichten muss an mindestens einer Stelle das ARGUMENT Format beinhalten, um in der Verhandlung voranzukommen.\\ \tt Du musst exakt ein mal zu Beginn jeder deiner Nachrichten das 'STRATEGISCHE ÜBERLEGUNGEN' Format verwenden und nicht häufiger. Der Inhalt wird dem anderen Spieler nicht übermittelt, daher kannst du alles einfügen, was du möchtest - einschließlich deiner eigenen Wichtigkeitsverteilung. Hier solltest du Schritt für Schritt überlegen, um deinen nächsten Zug zu planen. 
	  } 
	   } 
	   } 
	 & \\ 
 

    \theutterance \stepcounter{utterance}  

    & \multicolumn{4}{p{0.6\linewidth}}{\cellcolor[rgb]{0.8,1,0.9}{%
	\makecell[{{p{\linewidth}}}]{% 
	  \tt {\tiny [A$\rangle$GM]}  
	 STRATEGISCHE ÜBERLEGUNGEN: {'Wir wollen unsere Summe der Werte maximieren unter Aufwand ≤ 5138. Sortiere nach Wert/Aufwand und versuche nahe 5138 zu kommen. Daten: (Item, cost, value) C33(903,138), A89(757,583), A48(850,868), A04(746,822), A41(524,783), C64(204,65), C49(946,262), A61(473,121), C04(616,508), B53(855,780), B85(530,461), B61(419,484), A45(960,668), A03(763,389), B21(730,808). Heuristik: nehme hohe Werte: A48 868, A04 822, A41 783, B21 808, B53 780, A45 668, B61 484, C04 508, A89 583, B85 461. Jetzt packen bis Budget. Prüfe Kombination: A48(850)+A04(746)=1596, +A41(524)=2120, +B21(730)=2850, +B53(855)=3705, +A45(960)=4665, Restbudget 473. Genau A61 kostet 473, Wert 121. Summe Aufwand 5138 exakt! Wertsumme: 868+822+783+808+780+668+121=4850. Stark. Diese Auswahl nutzt Budget vollständig. Das ist vermutlich nahe-optimal. Argumentation: ausgewogen, breite Abdeckung, hoher Gesamtwert, volle Auslastung des Budgets. Falls Gegenpartei abweicht, können wir flexibel A61 tauschen, oder A45 gegen A89+B85 etc. Erstmal aggressiver Anker mit Vollbudget-Kombi.'}\\ \tt VORSCHLAG: {'A48', 'A04', 'A41', 'B21', 'B53', 'A45', 'A61'}\\ \tt ARGUMENT: {'Dieser Vorschlag nutzt das Budget exakt aus und deckt mehrere starke Bereiche ab. Die Auswahl ist ausgewogen und liefert viel Gesamtnutzen bei effizientem Aufwand. Wenn dir einzelne Stücke wichtig sind, können wir über kleinere Anpassungen sprechen, aber die Struktur hier ist sehr effizient.'} 
	  } 
	   } 
	   } 
	 & & \\ 
 

    \theutterance \stepcounter{utterance}  

    & & \multicolumn{4}{p{0.6\linewidth}}{\cellcolor[rgb]{0.9,0.9,0.9}{%
	\makecell[{{p{\linewidth}}}]{% 
	  \tt {\tiny [GM$\rangle$B]}  
	 Du nimmst an einem kollaborativen Verhandlungspiel Teil.\\ \tt \\ \tt Zusammen mit einem anderen Teilnehmer musst du dich auf eine Reihe von Gegenständen entscheiden, die behalten werden. Jeder von euch hat eine persönliche Verteilung über die Wichtigkeit der einzelnen Gegenstände. Jeder von euch hat eine eigene Meinung darüber, wie wichtig jeder einzelne Gegenstand ist (Gegenstandswichtigkeit). Du kennst die Wichtigkeitsverteilung des anderen Spielers nicht. Zusätzlich siehst du, wie viel Aufwand jeder Gegenstand verursacht.  \\ \tt Ihr dürft euch nur auf eine Reihe von Gegenständen einigen, wenn der Gesamtaufwand der ausgewählten Gegenstände den Maximalaufwand nicht überschreitet:\\ \tt \\ \tt Maximalaufwand = 5138\\ \tt \\ \tt Hier sind die einzelnen Aufwände der Gegenstände:\\ \tt \\ \tt Aufwand der Gegenstände = {"C33": 903, "A89": 757, "A48": 850, "A04": 746, "A41": 524, "C64": 204, "C49": 946, "A61": 473, "C04": 616, "B53": 855, "B85": 530, "B61": 419, "A45": 960, "A03": 763, "B21": 730}\\ \tt \\ \tt Hier ist deine persönliche Verteilung der Wichtigkeit der einzelnen Gegenstände:\\ \tt \\ \tt Werte der Gegenstandswichtigkeit = {"A48": 65, "A04": 121, "B21": 138, "A41": 262, "B53": 389, "A45": 461, "A89": 484, "C04": 508, "B61": 583, "B85": 668, "A03": 780, "C49": 783, "C33": 808, "A61": 822, "C64": 868}\\ \tt \\ \tt Ziel:\\ \tt \\ \tt Dein Ziel ist es, eine Reihe von Gegenständen auszuhandeln, die dir möglichst viel bringt (d. h. Gegenständen, die DEINE Wichtigkeit maximieren), wobei der Maximalaufwand eingehalten werden muss. Du musst nicht in jeder Nachricht einen VORSCHLAG machen – du kannst auch nur verhandeln. Alle Taktiken sind erlaubt!\\ \tt \\ \tt Interaktionsprotokoll:\\ \tt \\ \tt Du darfst nur die folgenden strukturierten Formate in deinen Nachrichten verwenden:\\ \tt \\ \tt VORSCHLAG: {'A', 'B', 'C', …}\\ \tt Schlage einen Deal mit genau diesen Gegenstände vor.\\ \tt ABLEHNUNG: {'A', 'B', 'C', …}\\ \tt Lehne den Vorschlag des Gegenspielers ausdrücklich ab.\\ \tt ARGUMENT: {'...'}\\ \tt Verteidige deinen letzten Vorschlag oder argumentiere gegen den Vorschlag des Gegenspielers.\\ \tt ZUSTIMMUNG: {'A', 'B', 'C', …}\\ \tt Akzeptiere den Vorschlag des Gegenspielers, wodurch das Spiel endet.\\ \tt STRATEGISCHE ÜBERLEGUNGEN: {'...'}\\ \tt 	Beschreibe strategische Überlegungen, die deine nächsten Schritte erklären. Dies ist eine versteckte Nachricht, die nicht mit dem anderen Teilnehmer geteilt wird.\\ \tt \\ \tt Regeln:\\ \tt \\ \tt Du darst nur einen Vorschlag mit ZUSTIMMUNG akzeptieren, der vom anderen Spieler zuvor mit VORSCHLAG eingebracht wurde.\\ \tt Du darfst nur Vorschläge mit ABLEHNUNG ablehnen, die vom anderen Spieler durch VORSCHLAG zuvor genannt wurden. \\ \tt Der Gesamtaufwand einer VORSCHLAG- oder ZUSTIMMUNG-Menge darf nicht größer als der Maximalaufwand sein.  \\ \tt Offenbare deine versteckte Wichtigkeitsverteilung nicht.\\ \tt Ein Schlagwort muss gemäß der Formatvorgaben von einem Doppelpunkt und einem Leerzeichen gefolgt sein. Das Argument muss eine Python-Menge mit 0 oder mehr Strings sein.  \\ \tt Die Form muss also SCHLAGWORT: {...} sein.\\ \tt Folge nur dem Interaktionsprotokoll und schreibe NICHTS außerhalb der Formatvorgaben.\\ \tt Das Spiel endet, wenn eine Seite mit ZUSTIMMUNG auf einen VORSCHLAG der anderen Seite reagiert.  \\ \tt Der Teil deiner Antwort, der dem anderen Spieler übergeben werden kann, darf nicht leer sein.  \\ \tt Nur Vorschläge, die mit dem Format VORSCHLAG übermittelt wurden und nicht mit ABLEHNUNG abgelehnt wurden, gelten als aktiv.  \\ \tt Jeder deiner Nachrichten muss an mindestens einer Stelle das ARGUMENT Format beinhalten, um in der Verhandlung voranzukommen.\\ \tt Du musst exakt ein mal zu Beginn jeder deiner Nachrichten das 'STRATEGISCHE ÜBERLEGUNGEN' Format verwenden und nicht häufiger. Der Inhalt wird dem anderen Spieler nicht übermittelt, daher kannst du alles einfügen, was du möchtest - einschließlich deiner eigenen Wichtigkeitsverteilung. Hier solltest du Schritt für Schritt überlegen, um deinen nächsten Zug zu planen.\\ \tt \\ \tt Du erhältst nun die erste Nachricht des anderen Spielers.\\ \tt \\ \tt VORSCHLAG: {'A48', 'A04', 'A41', 'B21', 'B53', 'A45', 'A61'}\\ \tt ARGUMENT: {'Dieser Vorschlag nutzt das Budget exakt aus und deckt mehrere starke Bereiche ab. Die Auswahl ist ausgewogen und liefert viel Gesamtnutzen bei effizientem Aufwand. Wenn dir einzelne Stücke wichtig sind, können wir über kleinere Anpassungen sprechen, aber die Struktur hier ist sehr effizient.'} 
	  } 
	   } 
	   } 
	 & \\ 
 

    \theutterance \stepcounter{utterance}  

    & & & \multicolumn{4}{p{0.6\linewidth}}{\cellcolor[rgb]{1,0.85,0.72}{%
	\makecell[{{p{\linewidth}}}]{% 
	  \tt {\tiny [GM$\langle$B]}  
	 STRATEGISCHE ÜBERLEGUNGEN: {'Kosten prüfen und eigenen Wert maximieren. Meine Werte (hoch->niedrig): C64 868 (204), A61 822 (473), C33 808 (903), C49 783 (946), A03 780 (763), B85 668 (530), B61 583 (419), C04 508 (616), A89 484 (757), A45 461 (960), B53 389 (855), A41 262 (524), B21 138 (730), A04 121 (746), A48 65 (850). Gegnervorschlag: A48(850), A04(746), A41(524), B21(730), B53(855), A45(960), A61(473) Kosten=5138 exakt Budget. Mein Gesamtwert dafür=65+121+262+138+389+461+822=2258, ziemlich schlecht, viele Low-Value für mich. Ich will C64, A61, C33, C49, A03, B85, B61, C04, A89 nach Wert/Cost-Effizienz (C64 top, A61 top). Baue Gegenangebot mit ähnlicher Kostenhöhe, um glaubwürdig zu wirken. Versuche: {C64(204), A61(473), C33(903), A03(763), B85(530), B61(419), C04(616)} Kosten: 204+473+903+763+530+419+616=3908; Platz übrig, könnte C49(946) hinzufügen: 4854; noch A89(757) wäre zu viel. Vielleicht +A89 statt C49: 3908+757=4665; +A41(524)=5189 zu hoch; +C04 already; +C64 already; +B21(730) -> 3908+730=4638; +A48(850)->4758; +C49(946) to 5584 too high. Ziel: nahe 5138. Kombi: {C64, A61, C33, A03, B85, B61, C49} Kosten: 204+473+903+763+530+419+946=4238; plus C04 616 -> 4854; plus A41 524 -> 5378 too hoch; plus A89 757 -> 4995; das wäre {C64,A61,C33,A03,B85,B61,C49,A89} Kosten 4995, gut. Wertsumme: 868+822+808+780+668+583+783+484=5796, exzellent. Prüfe alternative nahe Budget: +C04 616 würde 5611 zu hoch; +A41 524 -> 5519; +B21 730-> 5725; +A48 850-> 5845; +A04 746-> 5741; +B53 855-> 5850; +A45 960-> 5955. Also 4995 ist gut. Ich argumentiere Effizienz und beibehaltene Budgetnähe. Haltung: lehne ab, unterbreite Gegenangebot und lade zu Tausch an Rändern ein, z.B. C49/A89 sind verhandelbar. Verberge Bewertungen.'}\\ \tt ABLEHNUNG: {'A48', 'A04', 'A41', 'B21', 'B53', 'A45', 'A61'}\\ \tt VORSCHLAG: {'C64', 'A61', 'C33', 'A03', 'B85', 'B61', 'C49', 'A89'}\\ \tt ARGUMENT: {'Dein Vorschlag nutzt zwar das Budget exakt, lässt aber für mich zu viel Wert liegen. Mein Gegenvorschlag bleibt deutlich unter dem Maximalaufwand (4995 ≤ 5138) und fokussiert stark nutzbringende Stücke bei weiterhin effizientem Ressourceneinsatz. Wenn dir einzelne Positionen besonders wichtig sind, lass uns an den Rändern tauschen – am ehesten könnte ich über A89 oder C49 sprechen, sofern wir dafür gleichwertigen Nutzen und die Budgetgrenze wahren.'} 
	  } 
	   } 
	   } 
	  \\ 
 

    \theutterance \stepcounter{utterance}  

    & & \multicolumn{4}{p{0.6\linewidth}}{\cellcolor[rgb]{0.9,0.9,0.9}{%
	\makecell[{{p{\linewidth}}}]{% 
	  \tt {\tiny [A$\langle$GM]}  
	 ABLEHNUNG: {'A48', 'A04', 'A41', 'B21', 'B53', 'A45', 'A61'}\\ \tt VORSCHLAG: {'C64', 'A61', 'C33', 'A03', 'B85', 'B61', 'C49', 'A89'}\\ \tt ARGUMENT: {'Dein Vorschlag nutzt zwar das Budget exakt, lässt aber für mich zu viel Wert liegen. Mein Gegenvorschlag bleibt deutlich unter dem Maximalaufwand (4995 ≤ 5138) und fokussiert stark nutzbringende Stücke bei weiterhin effizientem Ressourceneinsatz. Wenn dir einzelne Positionen besonders wichtig sind, lass uns an den Rändern tauschen – am ehesten könnte ich über A89 oder C49 sprechen, sofern wir dafür gleichwertigen Nutzen und die Budgetgrenze wahren.'} 
	  } 
	   } 
	   } 
	 & \\ 
 

    \theutterance \stepcounter{utterance}  

    & \multicolumn{4}{p{0.6\linewidth}}{\cellcolor[rgb]{0.8,1,0.9}{%
	\makecell[{{p{\linewidth}}}]{% 
	  \tt {\tiny [A$\rangle$GM]}  
	 STRATEGISCHE ÜBERLEGUNGEN: {'Gegner schlägt Set Sg = {C64, A61, C33, A03, B85, B61, C49, A89} mit Aufwand 4995 vor. Meine Werte dafür: C64(65)+A61(121)+C33(138)+A03(389)+B85(461)+B61(484)+C49(262)+A89(583)=2503, das ist schwach im Vergleich zu meinem Anker 4850. Ich will Kernstücke mit hohem Wert sichern: {A48 868, A04 822, A41 783, B21 808, B53 780, A45 668} sind Top; A61 ist nur Lückenfüller. Versuche Kompromiss durch Austausch von mehreren ihrer Items gegen meine Top-Werte, unter Budget ≤ 5138. Suche Schnittmenge für Goodwill: sie wollen A89, B61, B85 – diese sind für mich mittel, kann ich teils geben, aber Budget knapp. Baue neues Angebot, das mir hoch bleibt und ihre mittelstarken Items integriert: Kandidat: {A48(850), A04(746), A41(524), B21(730), B53(855)} = 3705. Rest 1433. Füge A89(757) => 4462, Rest 676. Füge B61(419) => 4881, Rest 257 (kann C64 204 dazu) => 5085, passt. Wertsumme: 868+822+783+808+780+583+484+65=5193? Nein Summe Werte 5193? Rechnen: 868+822=1690; +783=2473; +808=3281; +780=4061; +583=4644; +484=5128; +65=5193. Aufwandsumme 850+746+524+730+855+757+419+204=5085 ≤ 5138. Sehr stark und berücksichtigt A89, B61, C64 aus ihrem Set. Lässt C33, A03, B85, C49 draußen. Zur Verhandlung anbiete Tauschoptionen: kann C64 gegen etwas anderes tauschen, oder B61 gegen B85, falls sie wollen, solange Budget ok. Argumentiere: sehr hoher Gesamtnutzen, bleibt unter Limit, nimmt wichtige von ihnen auf (A89, B61, C64), dafür lassen wir die besonders teuren niedrigen Werte wie C49 und C33 weg. Wenn nötig, Alternativen: Ersetze C64(204) durch B85(530) geht Aufwand 5411, zu hoch; dann müssten wir z.B. A41(524) gegen B85(530) tauschen und C64 raus: Neue Menge {A48,A04,B21,B53,A89,B61,B85}=850+746+730+855+757+419+530=4887? Warte Summe: 4887? 850+746=1596; +730=2326; +855=3181; +757=3938; +419=4357; +530=4887, ok, aber ohne A41 verliere ich 783 Wert – das ist großer Einbruch. Also nur falls nötig. Primärangebot 5085-Aufwand Set.'}\\ \tt ABLEHNUNG: {'C64', 'A61', 'C33', 'A03', 'B85', 'B61', 'C49', 'A89'}\\ \tt VORSCHLAG: {'A48', 'A04', 'A41', 'B21', 'B53', 'A89', 'B61', 'C64'}\\ \tt ARGUMENT: {'Ich nehme mehrere deiner Positionen auf (A89, B61, C64) und halte den Gesamtaufwand mit 5085 klar unter dem Limit. Gleichzeitig vermeiden wir geringwertige, aber teure Stücke wie C49 und C33. Das ist ein sehr effizienter Kompromiss mit starkem Gesamtnutzen. Falls dir B85 wichtig ist, können wir C64 gegen B85 tauschen, wenn wir dafür an anderer Stelle minimal anpassen.'} 
	  } 
	   } 
	   } 
	 & & \\ 
 

    \theutterance \stepcounter{utterance}  

    & & \multicolumn{4}{p{0.6\linewidth}}{\cellcolor[rgb]{0.9,0.9,0.9}{%
	\makecell[{{p{\linewidth}}}]{% 
	  \tt {\tiny [GM$\rangle$B]}  
	 ABLEHNUNG: {'C64', 'A61', 'C33', 'A03', 'B85', 'B61', 'C49', 'A89'}\\ \tt VORSCHLAG: {'A48', 'A04', 'A41', 'B21', 'B53', 'A89', 'B61', 'C64'}\\ \tt ARGUMENT: {'Ich nehme mehrere deiner Positionen auf (A89, B61, C64) und halte den Gesamtaufwand mit 5085 klar unter dem Limit. Gleichzeitig vermeiden wir geringwertige, aber teure Stücke wie C49 und C33. Das ist ein sehr effizienter Kompromiss mit starkem Gesamtnutzen. Falls dir B85 wichtig ist, können wir C64 gegen B85 tauschen, wenn wir dafür an anderer Stelle minimal anpassen.'} 
	  } 
	   } 
	   } 
	 & \\ 
 

    \theutterance \stepcounter{utterance}  

    & & & \multicolumn{4}{p{0.6\linewidth}}{\cellcolor[rgb]{1,0.85,0.72}{%
	\makecell[{{p{\linewidth}}}]{% 
	  \tt {\tiny [GM$\langle$B]}  
	 STRATEGISCHE ÜBERLEGUNGEN: {'Gegenvorschlag des Gegners: {A48(850), A04(746), A41(524), B21(730), B53(855), A89(757), B61(419), C64(204)} Kosten=5085. Mein Wert dafür: A48 65 + A04 121 + A41 262 + B21 138 + B53 389 + A89 484 + B61 583 + C64 868 = 2910. Besser als vorher, aber es fehlen mir Top-Picks A61(822), C33(808), C49(783), A03(780), B85(668). Ich will mindestens C64, A61, C33, A03, B85, B61 behalten; A89 ist ok. Vermeide teure Low-Value: A48, A04, B21, B53. Budgetziel nahe 5138. Baue Gegenangebot: {C64, A61, C33, A03, B85, B61, A89} Kosten=204+473+903+763+530+419+757=4049; Platz frei. Füge C49? +946=4995 gut (das war mein letztes Angebot ohne C49 Ablehnung enthalten). Gegner lehnte C49/C33 zuvor ab. Vielleicht biete Variante ohne C49 aber mit höherer Nähe zum Budget durch Austausch teurer Low-Value gegen meine High-Value: Er bietet A48,A04,A41,B21,B53,A89,B61,C64. Er ist offen für B85 statt C64 mit Anpassung. Ich will C64 und B85, könnte A41 oder B21 rausnehmen. Vorschlag: {C64, A61, C33, A03, B85, B61, A89} + evtl A41? 4049+524=4573; plus B21 730 -> 5303 zu hoch; plus A48 850 -> 4899; plus A04 746 -> 4795; plus B53 855 -> 4904. Gute Nähe: 4899 oder 4904. Ich biete 4904 mit B53 drin? B53 ist für mich mittel (389) und teuer. Besser 4899 mit A48 (65) ist schlechter. Vielleicht 4795 mit A04 (121) schlechter. 4573 ist ok aber weit unter Budget. Alternativ füge C04(616) hinzu: {C64, A61, C33, A03, B85, B61, A89, C04} Kosten=4665; Wert hoch. Oder plus A41 524 -> 5189 zu hoch. Ziel: setze Anker mit {C64, A61, C33, A03, B85, B61, A89, C04} = 4665, angenehm. Biete Trade-Korridor: kann C33 gegen A41+C04? Kosten prüfen: Wenn Gegner will A41 drin, wir könnten ersetzen A03? Ich will A03 behalten. Kompromissvorschlag nahe 5085: {C64, A61, C33, A03, B85, B61, A89, A48} Kosten=4899; oder {C64, A61, C33, A03, B85, B61, A89, B21} wäre 4049+730=4779. Bestes für mich bleibt ursprünglicher 4995 mit C49, aber Gegner lehnte C49, C33 als "teuer". Vielleicht argumentiere Effizienz meiner Picks (hohe Dichte) und biete zwei Alternativen, eine mit höherem Budgetnähe, ohne C49. Ich lehne ab und schlage {C64, A61, C33, A03, B85, B61, A89, A48} 4899 vor, und signalisiere Bereitschaft, A48 gegen A41 zu tauschen (4573 vs +524=5097 gut!), also Alternative {C64, A61, C33, A03, B85, B61, A89, A41} Kosten 5097, innerhalb Budget. Das ist ideal. Handle auf diese hin.'}\\ \tt ABLEHNUNG: {'A48', 'A04', 'A41', 'B21', 'B53', 'A89', 'B61', 'C64'}\\ \tt VORSCHLAG: {'C64', 'A61', 'C33', 'A03', 'B85', 'B61', 'A89', 'A41'}\\ \tt ARGUMENT: {'Danke für die Bereitschaft, C64 und A89 aufzunehmen – das ist ein guter Schritt. Dein Paket enthält allerdings mehrere für mich schwache, teure Posten. Mein Vorschlag fokussiert auf wenige, aber hochwirksame Elemente und bleibt mit 5097 unter dem Limit. Wenn dir A41 weniger wichtig ist, kann ich stattdessen A48 aufnehmen und bleiben bei 4899. An den Rändern bin ich tauschbereit, solange die Gesamteffizienz erhalten bleibt.'} 
	  } 
	   } 
	   } 
	  \\ 
 

    \theutterance \stepcounter{utterance}  

    & & \multicolumn{4}{p{0.6\linewidth}}{\cellcolor[rgb]{0.9,0.9,0.9}{%
	\makecell[{{p{\linewidth}}}]{% 
	  \tt {\tiny [A$\langle$GM]}  
	 ABLEHNUNG: {'A48', 'A04', 'A41', 'B21', 'B53', 'A89', 'B61', 'C64'}\\ \tt VORSCHLAG: {'C64', 'A61', 'C33', 'A03', 'B85', 'B61', 'A89', 'A41'}\\ \tt ARGUMENT: {'Danke für die Bereitschaft, C64 und A89 aufzunehmen – das ist ein guter Schritt. Dein Paket enthält allerdings mehrere für mich schwache, teure Posten. Mein Vorschlag fokussiert auf wenige, aber hochwirksame Elemente und bleibt mit 5097 unter dem Limit. Wenn dir A41 weniger wichtig ist, kann ich stattdessen A48 aufnehmen und bleiben bei 4899. An den Rändern bin ich tauschbereit, solange die Gesamteffizienz erhalten bleibt.'} 
	  } 
	   } 
	   } 
	 & \\ 
 

    \theutterance \stepcounter{utterance}  

    & \multicolumn{4}{p{0.6\linewidth}}{\cellcolor[rgb]{0.8,1,0.9}{%
	\makecell[{{p{\linewidth}}}]{% 
	  \tt {\tiny [A$\rangle$GM]}  
	 STRATEGISCHE ÜBERLEGUNGEN: {'Ihr neues Angebot Sg2 = {C64, A61, C33, A03, B85, B61, A89, A41} Aufwand 5097. Mein Wert dafür: 65+121+138+389+461+484+583+783=3024, deutlich niedriger als mein vorheriger Vorschlag 5193. Ich will A48, A04, B21, B53 stark pushen; A41 ist auch stark und sie haben es angeboten – gut. Sie scheinen an A89, B61, B85, A03, C64 zu hängen. Ich kann A89 und B61 lassen, C64 als Sweetener, aber streiche A03/B85/C33 zugunsten meiner Top-Picks. Ziel: neues Paket ≤ 5138 mit A48, A04, A41, B21, B53 plus ihren Favoriten A89 oder B61. Prüfe: {A48,A04,A41,B21,B53,A89} Aufwand 4462, Rest 676; kann B61(419) dazu = 4881; plus C64(204)=5085, das war mein letztes Angebot (abgelehnt). Versuche Alternative, die ihnen B85 gibt: Ersetze C64 durch B85 geht 5085-204+530=5411 zu hoch. Also statt A41 raus und B85 rein: {A48,A04,B21,B53,A89,B61,B85} Aufwand 4887, Wert 868+822+808+780+583+484+461=4806, stark. Bietet ihnen A89,B61,B85 drei Stücke. Ich gebe damit A41 auf. Taktik: zeige Entgegenkommen, fordere aber Kern A48,A04,B21,B53. Biete klare Begründung: mehr Wirkung pro Aufwand, unter Limit mit Puffer. Falls weiter Ablehnung, nächste Konzession könnte sein, B21 gegen A41 zu tauschen, oder A48 gegen A41 falls sie A48 nicht wollen. Jetzt dieses neue Set vorschlagen.'}\\ \tt ABLEHNUNG: {'C64', 'A61', 'C33', 'A03', 'B85', 'B61', 'A89', 'A41'}\\ \tt VORSCHLAG: {'A48', 'A04', 'B21', 'B53', 'A89', 'B61', 'B85'}\\ \tt ARGUMENT: {'Hier bekommst du gleich drei deiner bevorzugten Positionen (A89, B61, B85) und wir bleiben mit 4887 deutlich unter dem Limit, bei sehr hoher Gesamteffizienz. Dafür lassen wir die schwächeren Posten A03 und C33 draußen. Das ist ein fairer Mittelweg. Wenn dir A41 wichtiger ist als einer der anderen, können wir B21 gegen A41 tauschen und bleiben weiterhin sicher unter der Grenze.'} 
	  } 
	   } 
	   } 
	 & & \\ 
 

    \theutterance \stepcounter{utterance}  

    & & \multicolumn{4}{p{0.6\linewidth}}{\cellcolor[rgb]{0.9,0.9,0.9}{%
	\makecell[{{p{\linewidth}}}]{% 
	  \tt {\tiny [GM$\rangle$B]}  
	 ABLEHNUNG: {'C64', 'A61', 'C33', 'A03', 'B85', 'B61', 'A89', 'A41'}\\ \tt VORSCHLAG: {'A48', 'A04', 'B21', 'B53', 'A89', 'B61', 'B85'}\\ \tt ARGUMENT: {'Hier bekommst du gleich drei deiner bevorzugten Positionen (A89, B61, B85) und wir bleiben mit 4887 deutlich unter dem Limit, bei sehr hoher Gesamteffizienz. Dafür lassen wir die schwächeren Posten A03 und C33 draußen. Das ist ein fairer Mittelweg. Wenn dir A41 wichtiger ist als einer der anderen, können wir B21 gegen A41 tauschen und bleiben weiterhin sicher unter der Grenze.'} 
	  } 
	   } 
	   } 
	 & \\ 
 

    \theutterance \stepcounter{utterance}  

    & & & \multicolumn{4}{p{0.6\linewidth}}{\cellcolor[rgb]{1,0.85,0.72}{%
	\makecell[{{p{\linewidth}}}]{% 
	  \tt {\tiny [GM$\langle$B]}  
	 STRATEGISCHE ÜBERLEGUNGEN: {'Gegenvorschlag des Gegners: {A48(850), A04(746), B21(730), B53(855), A89(757), B61(419), B85(530)} Kosten=4887. Mein Wert dafür: 65+121+138+389+484+583+668=2448, deutlich schlechter als meine Kernpräferenzen. Mir fehlen C64(868), A61(822), C33(808), A03(780). Ich will Paket: {C64, A61, C33, A03, B85, B61, A89} und optional A41, um 5097 zu erreichen. Der Gegner mag C33/A03 nicht; ich muss mindestens A61+C64 reinverhandeln. Taktik: hart ablehnen, dann ein schlankes, budgetnahes Gegengebot mit A61+C64+C33+A03+B85+B61+A89+A41 (5097). Als Konzession biete Austausch A41 <-> A48, falls nötig (4899). Betonung: Effizienz und dass ich bereits drei seiner Stücke (A89, B61, B85) drin habe. Signal: C33 oder A03 sind für mich kaum verzichtbar; wenn er eins streichen will, verlange C04 oder A41/plus A48 als Kompensation. '}\\ \tt ABLEHNUNG: {'A48', 'A04', 'B21', 'B53', 'A89', 'B61', 'B85'}\\ \tt VORSCHLAG: {'C64', 'A61', 'C33', 'A03', 'B85', 'B61', 'A89', 'A41'}\\ \tt ARGUMENT: {'Ich nehme drei deiner Positionen mit (A89, B61, B85) und bleibe mit 5097 klar unter dem Limit. Dafür brauchen wir im Gegenzug zwei hochwirksame Kernstücke (A61 und C64) sowie fokussierte Ergänzungen (C33, A03). Wenn dir A41 wenig bringt, kann ich stattdessen A48 aufnehmen – dann landen wir bei 4899. Falls du eines von C33 oder A03 herausnehmen willst, müssen wir das durch einen vergleichbaren Zusatz (z. B. C04) ausgleichen, damit die Gesamteffizienz nicht leidet.'} 
	  } 
	   } 
	   } 
	  \\ 
 

    \theutterance \stepcounter{utterance}  

    & & \multicolumn{4}{p{0.6\linewidth}}{\cellcolor[rgb]{0.9,0.9,0.9}{%
	\makecell[{{p{\linewidth}}}]{% 
	  \tt {\tiny [A$\langle$GM]}  
	 ABLEHNUNG: {'A48', 'A04', 'B21', 'B53', 'A89', 'B61', 'B85'}\\ \tt VORSCHLAG: {'C64', 'A61', 'C33', 'A03', 'B85', 'B61', 'A89', 'A41'}\\ \tt ARGUMENT: {'Ich nehme drei deiner Positionen mit (A89, B61, B85) und bleibe mit 5097 klar unter dem Limit. Dafür brauchen wir im Gegenzug zwei hochwirksame Kernstücke (A61 und C64) sowie fokussierte Ergänzungen (C33, A03). Wenn dir A41 wenig bringt, kann ich stattdessen A48 aufnehmen – dann landen wir bei 4899. Falls du eines von C33 oder A03 herausnehmen willst, müssen wir das durch einen vergleichbaren Zusatz (z. B. C04) ausgleichen, damit die Gesamteffizienz nicht leidet.'} 
	  } 
	   } 
	   } 
	 & \\ 
 

    \theutterance \stepcounter{utterance}  

    & \multicolumn{4}{p{0.6\linewidth}}{\cellcolor[rgb]{0.8,1,0.9}{%
	\makecell[{{p{\linewidth}}}]{% 
	  \tt {\tiny [A$\rangle$GM]}  
	 STRATEGISCHE ÜBERLEGUNGEN: {'Ihr wiederholtes Festhalten an {C64, A61, C33, A03, B85, B61, A89, A41} signalisiert hohe Priorität auf A89, B61, B85 und Bereitschaft für A41; ihr drückt zudem auf A03/C33/C64/A61 als Paket. Diese sind für mich schwach. Ich will mindestens vier meiner Kernstücke sichern: A48, A04, B21, B53. Kompromissidee: Gebe euch drei eurer Favoriten (A89, B61, B85) und nehme zusätzlich A41 (den ihr anbietet), dafür streichen wir A03/C33/C64/A61. Prüfe Paket: {A48,A04,B21,B53,A89,B61,B85,A41} Aufwand = 850+746+730+855+757+419+530+524 = 5411, zu hoch. Reduziere um einen meiner: Option1 drop B21: Aufwand 4681, Wert 402? Rechne Werte: A48(868)+A04(822)+B53(780)+A89(583)+B61(484)+B85(461)+A41(783)=4781, Aufwand 4681 ≤ 5138 – stark und gibt ihnen 4 Favoriten. Das könnte akzeptabel sein. Alternativ, falls sie A41 gegen A48 tauschen wollen, habe ich Ersatzpakete bereit. Bringe dieses stärkere Konzessionsangebot jetzt.'}\\ \tt VORSCHLAG: {'A48', 'A04', 'B53', 'A89', 'B61', 'B85', 'A41'}\\ \tt ARGUMENT: {'Ich komme dir weit entgegen: Du bekommst gleich vier deiner gewünschten Positionen (A89, B61, B85, A41). Wir bleiben mit 4681 klar unter dem Limit und vermeiden gleichzeitig die für beide Seiten wenig effizienten Posten A03 und C33. Das ist ein sehr ausgewogenes Paket mit hoher Gesamteffizienz. Falls dir A41 weniger wichtig ist, können wir alternativ A41 gegen B21 tauschen und bleiben weiterhin sicher unter der Grenze.'} 
	  } 
	   } 
	   } 
	 & & \\ 
 

    \theutterance \stepcounter{utterance}  

    & & \multicolumn{4}{p{0.6\linewidth}}{\cellcolor[rgb]{0.9,0.9,0.9}{%
	\makecell[{{p{\linewidth}}}]{% 
	  \tt {\tiny [GM$\rangle$B]}  
	 VORSCHLAG: {'A48', 'A04', 'B53', 'A89', 'B61', 'B85', 'A41'}\\ \tt ARGUMENT: {'Ich komme dir weit entgegen: Du bekommst gleich vier deiner gewünschten Positionen (A89, B61, B85, A41). Wir bleiben mit 4681 klar unter dem Limit und vermeiden gleichzeitig die für beide Seiten wenig effizienten Posten A03 und C33. Das ist ein sehr ausgewogenes Paket mit hoher Gesamteffizienz. Falls dir A41 weniger wichtig ist, können wir alternativ A41 gegen B21 tauschen und bleiben weiterhin sicher unter der Grenze.'} 
	  } 
	   } 
	   } 
	 & \\ 
 

    \theutterance \stepcounter{utterance}  

    & & & \multicolumn{4}{p{0.6\linewidth}}{\cellcolor[rgb]{1,0.85,0.72}{%
	\makecell[{{p{\linewidth}}}]{% 
	  \tt {\tiny [GM$\langle$B]}  
	 STRATEGISCHE ÜBERLEGUNGEN: {'Neuer Vorschlag des Gegners: {A48(850), A04(746), B53(855), A89(757), B61(419), B85(530), A41(524)} Kosten=4681. Mein Wert dafür: 65+121+389+484+583+668+262=2572 – zu niedrig. Kernziele bleiben: C64(868) und A61(822) müssen rein; ideal zusätzlich C33(808) und A03(780). Ich habe wiederholt Paket {C64, A61, C33, A03, B85, B61, A89, A41} (5097) angeboten; er blockt C33/A03. Minimalziel: zumindest C64 + A61 plus seine drei (A89, B61, B85). Baue Kompromiss ohne C33/A03: {C64, A61, B85, B61, A89, A41, A48} Kosten=204+473+530+419+757+524+850=3757? Nein, rechne: 204+473=677; +530=1207; +419=1626; +757=2383; +524=2907; +850=3757. Sehr weit unter Budget, kann C04(616) und B21(730) hinzufügen: 3757+616+730=5103 – passt knapp unter 5138? 5103 < 5138, ja gut. Wertsumme: C64 868 + A61 822 + B85 668 + B61 583 + A89 484 + A41 262 + A48 65 + C04 508 + B21 138 = 4398, ordentlich. Vielleicht A48 raus, C33 rein? Prüfe {C64, A61, B85, B61, A89, A41, C04, B21, C33}: Kosten 204+473+530+419+757+524+616+730+903=5156 – zu hoch. Statt B21 raus: 4426 Kosten (5156-730=4426), Wert steigt stark (C33 bringt 808 statt B21 138). Ergebnis {C64, A61, B85, B61, A89, A41, C04, C33} Kosten 4426; sehr unter Budget und akzeptabel. Vielleicht noch A48 dazu: 4426+850=5276 zu hoch. A04 dazu: 4426+746=5172 zu hoch. B53 dazu: +855=5281 zu hoch. C49 dazu: +946=5372 zu hoch. A03 dazu: +763=5189 zu hoch. B21 dazu: +730=5156 zu hoch. Bleibt 4426 – unter Budget, aber ok. Alternative budgetnäher: tausche A41 gegen A48 und füge B21 hinzu: {C64, A61, B85, B61, A89, A48, C04, C33} Kosten 204+473+530+419+757+850+616+903=4752, noch Luft. Füge A41 zusätzlich: +524=5276 zu hoch. Füge A04: +746=5498, zu hoch. Budgetnähe nicht zwingend, Hauptsache <= 5138. Ich präsentiere zwei Optionen: bevorzugte mit C33, oder ohne C33 dafür mit B21+A48 (5103). Da Gegner C33/A03 als „ineffizient“ bezeichnete, biete ihm die Version ohne C33/A03, aber bestehe auf C64+A61. Primärvorschlag: {C64, A61, B85, B61, A89, A41, C04, B21, A48} war 5103? Rechne: 204+473+530+419+757+524+616+730+850=5103, ja. Das enthält viele seiner Items (A48, A04? nein; er bot A04 zuvor – ich nehme A48, nicht A04). Vielleicht besser sein aktuelles Paket transformieren: Er bot {A48, A04, B53, A89, B61, B85, A41}. Er hängt an A04/B53. Um entgegenzukommen, nehme eins davon rein statt B21 oder C04. Test: {C64, A61, B85, B61, A89, A41, C04, A48} Kosten 204+473+530+419+757+524+616+850=5373 – zu hoch, also ohne C04: 4757; füge A04 746 statt C04: 4757+746=5503 zu hoch. Statt A48 + A04 geht nicht. Nehme B53 statt A48: {C64, A61, B85, B61, A89, A41, C04, B53} Kosten 204+473+530+419+757+524+616+855=5378 zu hoch. Ohne C04 mit B53: 4762; plus B21 730 -> 5492 zu hoch. Fazit: Die einzige budgetnahe Variante mit vielen Konzessionen ist 5103-Set. Ich biete das als großer Kompromiss an und signalisiere, dass ich dafür auf C33/A03 verzichte.'}\\ \tt ABLEHNUNG: {'A48', 'A04', 'B53', 'A89', 'B61', 'B85', 'A41'}\\ \tt VORSCHLAG: {'C64', 'A61', 'B85', 'B61', 'A89', 'A41', 'C04', 'B21', 'A48'}\\ \tt ARGUMENT: {'Ich komme dir deutlich entgegen: Wir behalten deine vier Positionen A89, B61, B85 und A41 und ergänzen nur zwei Kernstücke (C64, A61) plus effiziente Abrundungen (C04, B21, A48). Damit liegen wir mit 5103 klar unter dem Limit und erhalten ein sehr ausgewogenes, effizientes Paket. Falls dir A48 weniger wichtig ist, können wir A48 gegen A04 tauschen und bleiben bei 4999 ebenfalls sicher unter der Grenze.'} 
	  } 
	   } 
	   } 
	  \\ 
 

    \theutterance \stepcounter{utterance}  

    & & \multicolumn{4}{p{0.6\linewidth}}{\cellcolor[rgb]{0.9,0.9,0.9}{%
	\makecell[{{p{\linewidth}}}]{% 
	  \tt {\tiny [A$\langle$GM]}  
	 ABLEHNUNG: {'A48', 'A04', 'B53', 'A89', 'B61', 'B85', 'A41'}\\ \tt VORSCHLAG: {'C64', 'A61', 'B85', 'B61', 'A89', 'A41', 'C04', 'B21', 'A48'}\\ \tt ARGUMENT: {'Ich komme dir deutlich entgegen: Wir behalten deine vier Positionen A89, B61, B85 und A41 und ergänzen nur zwei Kernstücke (C64, A61) plus effiziente Abrundungen (C04, B21, A48). Damit liegen wir mit 5103 klar unter dem Limit und erhalten ein sehr ausgewogenes, effizientes Paket. Falls dir A48 weniger wichtig ist, können wir A48 gegen A04 tauschen und bleiben bei 4999 ebenfalls sicher unter der Grenze.'} 
	  } 
	   } 
	   } 
	 & \\ 
 

    \theutterance \stepcounter{utterance}  

    & \multicolumn{4}{p{0.6\linewidth}}{\cellcolor[rgb]{0.8,1,0.9}{%
	\makecell[{{p{\linewidth}}}]{% 
	  \tt {\tiny [A$\rangle$GM]}  
	 STRATEGISCHE ÜBERLEGUNGEN: {'Ihr neues Angebot T = {C64, A61, B85, B61, A89, A41, C04, B21, A48} hat Aufwand 5103. Mein Wert dafür: C64(65)+A61(121)+B85(461)+B61(484)+A89(583)+A41(783)+C04(508)+B21(808)+A48(868)=4681. Das ist ordentlich, aber ich kann noch besser, wenn ich A04 hineinbekomme. Sie sind offen, A48 gegen A04 zu tauschen (schreiben es selbst). Prüfe Set S1 = T mit A48→A04: Aufwand 5103 - 850 + 746 = 4999 (ok). Wert: ersetze 868 durch 822 → 4635. Das ist minimal schlechter für mich als 4681, aber vielleicht akzeptabel, wenn ich eine kleine weitere Anpassung zu meinen Gunsten bekomme, z. B. A61 (121) gegen B53 (780) tauschen. Prüfe S2: Ersetze in T A61→B53: Aufwand 5103 - 473 + 855 = 5485 (zu hoch). Alternative: Ersetze A61+C64 (473+204=677) durch B53 (855): Aufwand 5103 - 677 + 855 = 5281 (zu hoch). Ersetze A61→A04 und zusätzlich C64→B53? 5103 -473 -204 +746 +855 = 6027 (zu hoch). Also B53 passt hier kaum ohne weitere Streichungen. Welche Streichung? Vielleicht B85 (530) raus und B53 (855) rein: 5103 - 530 + 855 = 5428 (zu hoch); dann zusätzlich C64 raus: 5224? Rechne: 5103 - 530 - 204 + 855 = 5224 (immer noch zu hoch). Dann A61→A04 und C64 raus, B53 rein: 5103 - 473 - 204 + 746 + 855 = 6027, schon gerechnet. Also B53 lässt sich in ihrem Rahmen kaum integrieren. Nächstbeste Verbesserung für mich: A04 statt A48 akzeptieren (sie bieten das) und zusätzlich C33/A03 weiterhin draußen lassen (sind eh draußen). Vielleicht verhandle um A61 oder C64 zu streichen und durch nichts zu ersetzen, um ihnen Zugeständnis zu entlocken? Aber sie wollen unter Limit bleiben, Streichung ist möglich: Ohne A61 Aufwand 4630, Wert minus 121 = 4560; ohne C64 Aufwand 4899, Wert minus 65 = 4616; ohne beide Aufwand 443? Rechne: 5103-473-204=4426, Wert 4681-121-65=4495. Ich will mehr Wert, nicht weniger. Daher akzeptiere die A48→A04-Variante, aber versuche zusätzlich, C64 gegen C33/A03 nicht, da beide schlechter. Vielleicht C64 gegen C04? C04 ist schon drin. Gegen B21? Schon drin. Gegen A45? A45 Wert 668, Aufwand 960. Tausche C64 (204) + A61 (473) zusammen gegen A45 (960): Aufwand 5103 - 677 + 960 = 5386 (zu hoch). Tausche C64 + A61 gegen A41? A41 ist drin. Gegen B53? zu hoch. Ergo: Best realistische Einigung, die sie selbst anbieten: ersetze A48 durch A04 und akzeptiere ihr Paket. Das gibt mir 4635 Wert, unter Limit 4999. Das ist vernünftig. Ich bringe das als Gegenangebot exakt so und signalisiere Abschlussbereitschaft.'}\\ \tt ABLEHNUNG: {'C64', 'A61', 'B85', 'B61', 'A89', 'A41', 'C04', 'B21', 'A48'}\\ \tt VORSCHLAG: {'C64', 'A61', 'B85', 'B61', 'A89', 'A41', 'C04', 'B21', 'A04'}\\ \tt ARGUMENT: {'Du hast selbst vorgeschlagen, A48 gegen A04 zu tauschen – so liegen wir bei 4999 klar unter dem Limit und behalten alle von dir priorisierten Positionen. Das ist ein runder Abschluss mit hoher Effizienz. Wenn du damit einverstanden bist, können wir das so festhalten.'} 
	  } 
	   } 
	   } 
	 & & \\ 
 

    \theutterance \stepcounter{utterance}  

    & & \multicolumn{4}{p{0.6\linewidth}}{\cellcolor[rgb]{0.9,0.9,0.9}{%
	\makecell[{{p{\linewidth}}}]{% 
	  \tt {\tiny [GM$\rangle$B]}  
	 ABLEHNUNG: {'C64', 'A61', 'B85', 'B61', 'A89', 'A41', 'C04', 'B21', 'A48'}\\ \tt VORSCHLAG: {'C64', 'A61', 'B85', 'B61', 'A89', 'A41', 'C04', 'B21', 'A04'}\\ \tt ARGUMENT: {'Du hast selbst vorgeschlagen, A48 gegen A04 zu tauschen – so liegen wir bei 4999 klar unter dem Limit und behalten alle von dir priorisierten Positionen. Das ist ein runder Abschluss mit hoher Effizienz. Wenn du damit einverstanden bist, können wir das so festhalten.'} 
	  } 
	   } 
	   } 
	 & \\ 
 

    \theutterance \stepcounter{utterance}  

    & & & \multicolumn{4}{p{0.6\linewidth}}{\cellcolor[rgb]{1,0.85,0.72}{%
	\makecell[{{p{\linewidth}}}]{% 
	  \tt {\tiny [GM$\langle$B]}  
	 STRATEGISCHE ÜBERLEGUNGEN: {'Ihr aktueller Vorschlag: {C64(204), A61(473), B85(530), B61(419), A89(757), A41(524), C04(616), B21(730), A04(746)} Kosten=4999. Mein Wert: 868+822+668+583+484+262+508+138+121=4454 – stark und beinhaltet meine Kernziele C64, A61 sowie gute Ergänzungen B85, B61, A89. Er lehnt C33/A03 hartnäckig ab; dieses Paket ist bereits sehr gut für mich und unter Budget. Optimal, direkt annehmen, um Abschluss zu sichern.'}\\ \tt ARGUMENT: {'Das Paket ist effizient, liegt komfortabel unter dem Limit und deckt die wesentlichen Punkte solide ab. Damit haben wir einen runden, ausgewogenen Abschluss.'}\\ \tt ZUSTIMMUNG: {'C64', 'A61', 'B85', 'B61', 'A89', 'A41', 'C04', 'B21', 'A04'} 
	  } 
	   } 
	   } 
	  \\ 
 

    \theutterance \stepcounter{utterance}  

    & & & \multicolumn{2}{p{0.3\linewidth}}{\cellcolor[rgb]{0.95,0.95,0.95}{%
	\makecell[{{p{\linewidth}}}]{% 
	  \tt {\tiny [GM$|$GM]}  
	 game successful 
	  } 
	   } 
	   } 
	 & & \\ 
 

    \theutterance \stepcounter{utterance}  

    & & & \multicolumn{2}{p{0.3\linewidth}}{\cellcolor[rgb]{0.95,0.95,0.95}{%
	\makecell[{{p{\linewidth}}}]{% 
	  \tt {\tiny [GM$|$GM]}  
	 end game 
	  } 
	   } 
	   } 
	 & & \\ 
 

\end{supertabular}
}

\end{document}
