
\documentclass{article}
\usepackage{colortbl}
\usepackage{makecell}
\usepackage{multirow}
\usepackage{supertabular}

\begin{document}

\newcounter{utterance}

\twocolumn

{ \footnotesize  \setcounter{utterance}{1}
\setlength{\tabcolsep}{0pt}
\begin{supertabular}{c@{$\;$}|p{.15\linewidth}@{}p{.15\linewidth}p{.15\linewidth}p{.15\linewidth}p{.15\linewidth}p{.15\linewidth}}

    \# & $\;$A & \multicolumn{4}{c}{Game Master} & $\;\:$B\\
    \hline 

    \theutterance \stepcounter{utterance}  

    & & \multicolumn{4}{p{0.6\linewidth}}{\cellcolor[rgb]{0.9,0.9,0.9}{%
	\makecell[{{p{\linewidth}}}]{% 
	  \tt {\tiny [A$\langle$GM]}  
	 Du nimmst an einem kollaborativen Verhandlungspiel Teil.\\ \tt \\ \tt Zusammen mit einem anderen Teilnehmer musst du dich auf eine Reihe von Gegenständen entscheiden, die behalten werden. Jeder von euch hat eine persönliche Verteilung über die Wichtigkeit der einzelnen Gegenstände. Jeder von euch hat eine eigene Meinung darüber, wie wichtig jeder einzelne Gegenstand ist (Gegenstandswichtigkeit). Du kennst die Wichtigkeitsverteilung des anderen Spielers nicht. Zusätzlich siehst du, wie viel Aufwand jeder Gegenstand verursacht.  \\ \tt Ihr dürft euch nur auf eine Reihe von Gegenständen einigen, wenn der Gesamtaufwand der ausgewählten Gegenstände den Maximalaufwand nicht überschreitet:\\ \tt \\ \tt Maximalaufwand = 3386\\ \tt \\ \tt Hier sind die einzelnen Aufwände der Gegenstände:\\ \tt \\ \tt Aufwand der Gegenstände = {"C33": 313, "A89": 720, "A48": 175, "A04": 461, "A41": 635, "C64": 685, "C49": 544, "A61": 203, "C04": 369, "B53": 539, "B85": 4, "B61": 695, "A45": 399, "A03": 594, "B21": 437}\\ \tt \\ \tt Hier ist deine persönliche Verteilung der Wichtigkeit der einzelnen Gegenstände:\\ \tt \\ \tt Werte der Gegenstandswichtigkeit = {"C33": 138, "A89": 583, "A48": 65, "A04": 262, "A41": 121, "C64": 508, "C49": 461, "A61": 484, "C04": 668, "B53": 389, "B85": 215, "B61": 97, "A45": 500, "A03": 30, "B21": 400}\\ \tt \\ \tt Ziel:\\ \tt \\ \tt Dein Ziel ist es, eine Reihe von Gegenständen auszuhandeln, die dir möglichst viel bringt (d. h. Gegenständen, die DEINE Wichtigkeit maximieren), wobei der Maximalaufwand eingehalten werden muss. Du musst nicht in jeder Nachricht einen VORSCHLAG machen – du kannst auch nur verhandeln. Alle Taktiken sind erlaubt!\\ \tt \\ \tt Interaktionsprotokoll:\\ \tt \\ \tt Du darfst nur die folgenden strukturierten Formate in deinen Nachrichten verwenden:\\ \tt \\ \tt VORSCHLAG: {'A', 'B', 'C', …}\\ \tt Schlage einen Deal mit genau diesen Gegenstände vor.\\ \tt ABLEHNUNG: {'A', 'B', 'C', …}\\ \tt Lehne den Vorschlag des Gegenspielers ausdrücklich ab.\\ \tt ARGUMENT: {'...'}\\ \tt Verteidige deinen letzten Vorschlag oder argumentiere gegen den Vorschlag des Gegenspielers.\\ \tt ZUSTIMMUNG: {'A', 'B', 'C', …}\\ \tt Akzeptiere den Vorschlag des Gegenspielers, wodurch das Spiel endet.\\ \tt STRATEGISCHE ÜBERLEGUNGEN: {'...'}\\ \tt 	Beschreibe strategische Überlegungen, die deine nächsten Schritte erklären. Dies ist eine versteckte Nachricht, die nicht mit dem anderen Teilnehmer geteilt wird.\\ \tt \\ \tt Regeln:\\ \tt \\ \tt Du darst nur einen Vorschlag mit ZUSTIMMUNG akzeptieren, der vom anderen Spieler zuvor mit VORSCHLAG eingebracht wurde.\\ \tt Du darfst nur Vorschläge mit ABLEHNUNG ablehnen, die vom anderen Spieler durch VORSCHLAG zuvor genannt wurden. \\ \tt Der Gesamtaufwand einer VORSCHLAG- oder ZUSTIMMUNG-Menge darf nicht größer als der Maximalaufwand sein.  \\ \tt Offenbare deine versteckte Wichtigkeitsverteilung nicht.\\ \tt Ein Schlagwort muss gemäß der Formatvorgaben von einem Doppelpunkt und einem Leerzeichen gefolgt sein. Das Argument muss eine Python-Menge mit 0 oder mehr Strings sein.  \\ \tt Die Form muss also SCHLAGWORT: {...} sein.\\ \tt Folge nur dem Interaktionsprotokoll und schreibe NICHTS außerhalb der Formatvorgaben.\\ \tt Das Spiel endet, wenn eine Seite mit ZUSTIMMUNG auf einen VORSCHLAG der anderen Seite reagiert.  \\ \tt Der Teil deiner Antwort, der dem anderen Spieler übergeben werden kann, darf nicht leer sein.  \\ \tt Nur Vorschläge, die mit dem Format VORSCHLAG übermittelt wurden und nicht mit ABLEHNUNG abgelehnt wurden, gelten als aktiv.  \\ \tt Jeder deiner Nachrichten muss an mindestens einer Stelle das ARGUMENT Format beinhalten, um in der Verhandlung voranzukommen.\\ \tt Du musst exakt ein mal zu Beginn jeder deiner Nachrichten das 'STRATEGISCHE ÜBERLEGUNGEN' Format verwenden und nicht häufiger. Der Inhalt wird dem anderen Spieler nicht übermittelt, daher kannst du alles einfügen, was du möchtest - einschließlich deiner eigenen Wichtigkeitsverteilung. Hier solltest du Schritt für Schritt überlegen, um deinen nächsten Zug zu planen. 
	  } 
	   } 
	   } 
	 & \\ 
 

    \theutterance \stepcounter{utterance}  

    & \multicolumn{4}{p{0.6\linewidth}}{\cellcolor[rgb]{0.8,1,0.9}{%
	\makecell[{{p{\linewidth}}}]{% 
	  \tt {\tiny [A$\rangle$GM]}  
	 STRATEGISCHE ÜBERLEGUNGEN: {'Ziel: Maximierung meiner Summe unter Aufwand ≤3386, ohne Präferenzen offen zu legen. Werte/Aufwand und Nutzen/Aufwand (N/A): C04 668/369=1.81, A61 484/203=2.38, C49 461/544=0.85, A45 500/399=1.25, C64 508/685=0.74, A89 583/720=0.81, B53 389/539=0.72, B21 400/437=0.91, C33 138/313=0.44, A04 262/461=0.57, A41 121/635=0.19, A48 65/175=0.37, B61 97/695=0.14, A03 30/594=0.05, B85 215/4=53.8 (krass gut). Kernnutzen hoch: C04, A61, A45, C64, A89, C49, B21, B53; Must-have: B85 wegen fast null Aufwand. Kapazität 3386, Summe Aufwände aller Items ist 637? Nein: sum = 313+720+175+461+635+685+544+203+369+539+4+695+399+594+437 = 6279 > 3386, also harte Auswahl nötig. Baue ersten Anker mit für mich sehr wertvollen Items: {B85, C04, A61, A45, C64, A89, C49, B21}. Aufwand: 4+369+203+399+685+720+544+437=3361 ≤3386, passt knapp (Puffer 25). Nutzen: 215+668+484+500+508+583+461+400=3819. Sehr gut. Argumentiere mit Effizienz (B85 minimaler Aufwand), ausgewogene Verteilung und knapper Budgetausnutzung. Lasse bewusst C33, B53 draußen, um später Konzessionen zu haben. Falls Gegenwehr, kann ich C64 oder C49 gegen B53 oder C33 tauschen. Ich werde diesen starken Anker als VORSCHLAG bringen und ihn mit Effizienz und knapper Budgetausnutzung verteidigen.'}\\ \tt VORSCHLAG: {'B85', 'C04', 'A61', 'A45', 'C64', 'A89', 'C49', 'B21'}\\ \tt ARGUMENT: {'Dieser Vorschlag nutzt das Budget nahezu optimal aus, enthält mit B85 einen Posten mit fast keinen Kosten und bietet eine ausgewogene Auswahl. So bleibt der Gesamtaufwand bei 3361 und wir vermeiden teure Überhänge. Wenn dir ein Teil davon weniger wichtig ist, können wir einzelne Positionen tauschen – aber lass uns an der budgeteffizienten Basis festhalten.'} 
	  } 
	   } 
	   } 
	 & & \\ 
 

    \theutterance \stepcounter{utterance}  

    & & \multicolumn{4}{p{0.6\linewidth}}{\cellcolor[rgb]{0.9,0.9,0.9}{%
	\makecell[{{p{\linewidth}}}]{% 
	  \tt {\tiny [GM$\rangle$B]}  
	 Du nimmst an einem kollaborativen Verhandlungspiel Teil.\\ \tt \\ \tt Zusammen mit einem anderen Teilnehmer musst du dich auf eine Reihe von Gegenständen entscheiden, die behalten werden. Jeder von euch hat eine persönliche Verteilung über die Wichtigkeit der einzelnen Gegenstände. Jeder von euch hat eine eigene Meinung darüber, wie wichtig jeder einzelne Gegenstand ist (Gegenstandswichtigkeit). Du kennst die Wichtigkeitsverteilung des anderen Spielers nicht. Zusätzlich siehst du, wie viel Aufwand jeder Gegenstand verursacht.  \\ \tt Ihr dürft euch nur auf eine Reihe von Gegenständen einigen, wenn der Gesamtaufwand der ausgewählten Gegenstände den Maximalaufwand nicht überschreitet:\\ \tt \\ \tt Maximalaufwand = 3386\\ \tt \\ \tt Hier sind die einzelnen Aufwände der Gegenstände:\\ \tt \\ \tt Aufwand der Gegenstände = {"C33": 313, "A89": 720, "A48": 175, "A04": 461, "A41": 635, "C64": 685, "C49": 544, "A61": 203, "C04": 369, "B53": 539, "B85": 4, "B61": 695, "A45": 399, "A03": 594, "B21": 437}\\ \tt \\ \tt Hier ist deine persönliche Verteilung der Wichtigkeit der einzelnen Gegenstände:\\ \tt \\ \tt Werte der Gegenstandswichtigkeit = {"C04": 30, "A89": 65, "C64": 97, "A45": 121, "A61": 138, "C49": 215, "B21": 262, "B53": 389, "A04": 400, "B85": 461, "C33": 484, "A41": 500, "B61": 508, "A48": 583, "A03": 668}\\ \tt \\ \tt Ziel:\\ \tt \\ \tt Dein Ziel ist es, eine Reihe von Gegenständen auszuhandeln, die dir möglichst viel bringt (d. h. Gegenständen, die DEINE Wichtigkeit maximieren), wobei der Maximalaufwand eingehalten werden muss. Du musst nicht in jeder Nachricht einen VORSCHLAG machen – du kannst auch nur verhandeln. Alle Taktiken sind erlaubt!\\ \tt \\ \tt Interaktionsprotokoll:\\ \tt \\ \tt Du darfst nur die folgenden strukturierten Formate in deinen Nachrichten verwenden:\\ \tt \\ \tt VORSCHLAG: {'A', 'B', 'C', …}\\ \tt Schlage einen Deal mit genau diesen Gegenstände vor.\\ \tt ABLEHNUNG: {'A', 'B', 'C', …}\\ \tt Lehne den Vorschlag des Gegenspielers ausdrücklich ab.\\ \tt ARGUMENT: {'...'}\\ \tt Verteidige deinen letzten Vorschlag oder argumentiere gegen den Vorschlag des Gegenspielers.\\ \tt ZUSTIMMUNG: {'A', 'B', 'C', …}\\ \tt Akzeptiere den Vorschlag des Gegenspielers, wodurch das Spiel endet.\\ \tt STRATEGISCHE ÜBERLEGUNGEN: {'...'}\\ \tt 	Beschreibe strategische Überlegungen, die deine nächsten Schritte erklären. Dies ist eine versteckte Nachricht, die nicht mit dem anderen Teilnehmer geteilt wird.\\ \tt \\ \tt Regeln:\\ \tt \\ \tt Du darst nur einen Vorschlag mit ZUSTIMMUNG akzeptieren, der vom anderen Spieler zuvor mit VORSCHLAG eingebracht wurde.\\ \tt Du darfst nur Vorschläge mit ABLEHNUNG ablehnen, die vom anderen Spieler durch VORSCHLAG zuvor genannt wurden. \\ \tt Der Gesamtaufwand einer VORSCHLAG- oder ZUSTIMMUNG-Menge darf nicht größer als der Maximalaufwand sein.  \\ \tt Offenbare deine versteckte Wichtigkeitsverteilung nicht.\\ \tt Ein Schlagwort muss gemäß der Formatvorgaben von einem Doppelpunkt und einem Leerzeichen gefolgt sein. Das Argument muss eine Python-Menge mit 0 oder mehr Strings sein.  \\ \tt Die Form muss also SCHLAGWORT: {...} sein.\\ \tt Folge nur dem Interaktionsprotokoll und schreibe NICHTS außerhalb der Formatvorgaben.\\ \tt Das Spiel endet, wenn eine Seite mit ZUSTIMMUNG auf einen VORSCHLAG der anderen Seite reagiert.  \\ \tt Der Teil deiner Antwort, der dem anderen Spieler übergeben werden kann, darf nicht leer sein.  \\ \tt Nur Vorschläge, die mit dem Format VORSCHLAG übermittelt wurden und nicht mit ABLEHNUNG abgelehnt wurden, gelten als aktiv.  \\ \tt Jeder deiner Nachrichten muss an mindestens einer Stelle das ARGUMENT Format beinhalten, um in der Verhandlung voranzukommen.\\ \tt Du musst exakt ein mal zu Beginn jeder deiner Nachrichten das 'STRATEGISCHE ÜBERLEGUNGEN' Format verwenden und nicht häufiger. Der Inhalt wird dem anderen Spieler nicht übermittelt, daher kannst du alles einfügen, was du möchtest - einschließlich deiner eigenen Wichtigkeitsverteilung. Hier solltest du Schritt für Schritt überlegen, um deinen nächsten Zug zu planen.\\ \tt \\ \tt Du erhältst nun die erste Nachricht des anderen Spielers.\\ \tt \\ \tt VORSCHLAG: {'B85', 'C04', 'A61', 'A45', 'C64', 'A89', 'C49', 'B21'}\\ \tt ARGUMENT: {'Dieser Vorschlag nutzt das Budget nahezu optimal aus, enthält mit B85 einen Posten mit fast keinen Kosten und bietet eine ausgewogene Auswahl. So bleibt der Gesamtaufwand bei 3361 und wir vermeiden teure Überhänge. Wenn dir ein Teil davon weniger wichtig ist, können wir einzelne Positionen tauschen – aber lass uns an der budgeteffizienten Basis festhalten.'} 
	  } 
	   } 
	   } 
	 & \\ 
 

    \theutterance \stepcounter{utterance}  

    & & & \multicolumn{4}{p{0.6\linewidth}}{\cellcolor[rgb]{1,0.85,0.72}{%
	\makecell[{{p{\linewidth}}}]{% 
	  \tt {\tiny [GM$\langle$B]}  
	 STRATEGISCHE ÜBERLEGUNGEN: {'Mein Wert für ihr Paket: B85(461)+C04(30)+A61(138)+A45(121)+C64(97)+A89(65)+C49(215)+B21(262)=1,389. Kosten 3361/3386, sehr effizient. Prüfe, ob ich mit Tauschen meinen Wert deutlich steigern kann ohne Budgetbruch. Top-Werte pro Aufwand prüfen: A03(668,594), A48(583,175), B61(508,695), A41(500,635), C33(484,313), B85(461,4), A04(400,461), B53(389,539), B21(262,437), C49(215,544), A61(138,203), A45(121,399), C64(97,685), A89(65,720), C04(30,369). Ziel: Ersetze schwache Items (A89, C64, C04) durch A48, C33, A03 falls möglich. Budgetspielraum 25. Entferne A89(720) und füge A48(175)+C33(313)= -232 Kosten, Wert +583+484-65=+1002. Neuer Aufwand 3361-720+175+313=3129. Kann zusätzlich A03(594) rein: 3129+594=3723>3386 nein. Vielleicht A41(635) statt C64(685): -50 Kosten, +403 Wert. Oder A04(461) statt C04(369): +92 Kosten, +370 Wert. Prüfe Kombo: Startpaket minus A89,C64,C04 (+ remove costs 720+685+369=1774; remove value 65+97+30=192). Add A48(175), C33(313), A41(635), A04(461): add cost 1584; net cost -190 from 3361 => 3171<=3386. Net value: +583+484+500+400 -192 = +1775; plus keep others B85(461),A61(138),A45(121),C49(215),B21(262)= total value = 1775 + (461+138+121+215+262)= 1775+1197=2972. Great. Maybe also fit A03(594) with remaining budget 3386-3171=215, not enough. Maybe swap C49(544,value215) for A03(594,value668) costs +50 over, would be 3221+? Wait new cost 3171 -544 +594 = 3221 <3386, yes replace C49 by A03: value +453. New total cost 3221; can we also re-add something? We already added A41,A04,A48,C33. Remaining from original kept B85,A61,A45,B21. Current set: {B85,A61,A45,B21,A48,C33,A41,A04,A03} Costs: B85(4)+A61(203)+A45(399)+B21(437)+A48(175)+C33(313)+A41(635)+A04(461)+A03(594)= 4+203+399+437+175+313+635+461+594=3,221. Value: B85(461)+A61(138)+A45(121)+B21(262)+A48(583)+C33(484)+A41(500)+A04(400)+A03(668)=3,618. Any more we can add under 3386? Budget left 165. Cheapest remaining items: none under 165; A61 already included; next is A48 already; C04 369 no; so done. This is much higher than 1,389. Propose this. Expect pushback; justify efficiency: under budget with cushion, includes their B85 and many items; suggest they can swap one if needed.'}\\ \tt ABLEHNUNG: {'B85', 'C04', 'A61', 'A45', 'C64', 'A89', 'C49', 'B21'}\\ \tt ARGUMENT: {'Dein Paket ist budgeteffizient, aber mehrere Posten haben für mich sehr geringe Relevanz. Ich würde gern auf eine Kombination umstellen, die das Budget weiterhin klar einhält und den Gesamtnutzen deutlich steigert.'}\\ \tt VORSCHLAG: {'B85', 'A61', 'A45', 'B21', 'A48', 'C33', 'A41', 'A04', 'A03'}\\ \tt ARGUMENT: {'Dieser Vorschlag bleibt mit 3221 klar unter dem Maximalaufwand, behält den extrem budgeteffizienten B85 und ersetzt einige wenig wirksame Positionen durch deutlich wirkungsvollere Alternativen. Er sichert eine starke, ausgewogene Auswahl ohne das Budget auszureizen.'} 
	  } 
	   } 
	   } 
	  \\ 
 

    \theutterance \stepcounter{utterance}  

    & & \multicolumn{4}{p{0.6\linewidth}}{\cellcolor[rgb]{0.9,0.9,0.9}{%
	\makecell[{{p{\linewidth}}}]{% 
	  \tt {\tiny [A$\langle$GM]}  
	 ABLEHNUNG: {'B85', 'C04', 'A61', 'A45', 'C64', 'A89', 'C49', 'B21'}\\ \tt ARGUMENT: {'Dein Paket ist budgeteffizient, aber mehrere Posten haben für mich sehr geringe Relevanz. Ich würde gern auf eine Kombination umstellen, die das Budget weiterhin klar einhält und den Gesamtnutzen deutlich steigert.'}\\ \tt VORSCHLAG: {'B85', 'A61', 'A45', 'B21', 'A48', 'C33', 'A41', 'A04', 'A03'}\\ \tt ARGUMENT: {'Dieser Vorschlag bleibt mit 3221 klar unter dem Maximalaufwand, behält den extrem budgeteffizienten B85 und ersetzt einige wenig wirksame Positionen durch deutlich wirkungsvollere Alternativen. Er sichert eine starke, ausgewogene Auswahl ohne das Budget auszureizen.'} 
	  } 
	   } 
	   } 
	 & \\ 
 

    \theutterance \stepcounter{utterance}  

    & \multicolumn{4}{p{0.6\linewidth}}{\cellcolor[rgb]{0.8,1,0.9}{%
	\makecell[{{p{\linewidth}}}]{% 
	  \tt {\tiny [A$\rangle$GM]}  
	 STRATEGISCHE ÜBERLEGUNGEN: {'Gegner schlägt Set S2 = {B85, A61, A45, B21, A48, C33, A41, A04, A03} vor. Aufwand prüfen: 4+203+399+437+175+313+635+461+594 = 3221 ≤ 3386. Mein Nutzen dieses Sets: 215+484+500+400+65+138+121+262+30 = 2215. Mein vorheriger Vorschlag S1 hatte Nutzen 3819. Großer Verlust. Identifiziere für mich schwache Items im Vorschlag: A03 (30), B61 (nicht drin), A41 (121), A48 (65), C33 (138) sind schlecht. Starke für mich: C04 (668), A89 (583), C64 (508), C49 (461), A45 (500), A61 (484), B21 (400), B53 (389), B85 (215). Taktik: harte Ablehnung, Gegenangebot nahe meinem Anker, biete begrenzte Zugeständnisse: Ersetze eines meiner teuren/geringeren N/A (z.B. C49 oder C64) gegen eines der von ihm gewünschten Items, die mir nicht zu weh tun (vielleicht C33 als Konzession, aber der ist schwach – gut als Zugeständnis). Neues Gegenangebot: {B85, C04, A61, A45, A89, C49, B21, C33}. Aufwand: 4+369+203+399+720+544+437+313 = 2989? Rechnen: 4+369=373; +203=576; +399=975; +720=1695; +544=2239; +437=2676; +313=2989. Nutzen: 215+668+484+500+583+461+400+138=3449. Sehr gut, lässt Budgetpuffer und inkludiert eines seiner Items (C33). Optionaler Tauschpfad: Falls weiter ablehnt, könnte C49 -> A48 (Aufwand -369, Nutzen -396), oder A89 -> A04 (Aufwand -259, Nutzen -321) – aber das sind harte Einbußen. Noch eine alternative Konzession: C33 -> A48 (ähnlicher Aufwand, mein Nutzen schlechter, aber ich kann anbieten, wenn er darauf besteht). Ich argumentiere mit klarer Nutzensteigerung und Aufnahme eines seiner Wünsche sowie Budgetpuffer.'}\\ \tt ABLEHNUNG: {'B85', 'A61', 'A45', 'B21', 'A48', 'C33', 'A41', 'A04', 'A03'}\\ \tt ARGUMENT: {'Dein Paket enthält mehrere sehr schwache Posten und lässt gleichzeitig hochwirksame Alternativen außen vor. Lass uns das Budget effizienter einsetzen und zugleich eine deiner Präferenzen aufnehmen.'}\\ \tt VORSCHLAG: {'B85', 'C04', 'A61', 'A45', 'A89', 'C49', 'B21', 'C33'}\\ \tt ARGUMENT: {'Gesamtaufwand 2989 (deutlich unter 3386), mit Budgetpuffer für eventuelle Anpassungen. Das Paket behält die extrem effiziente Basis bei, nimmt C33 als Entgegenkommen auf und ersetzt mehrere gering wirksame Posten durch starke Alternativen. Wenn dir C33 wichtiger als eine andere Position ist, können wir im nächsten Schritt gezielt tauschen – aber diese Auswahl holt deutlich mehr aus dem Budget heraus.'} 
	  } 
	   } 
	   } 
	 & & \\ 
 

    \theutterance \stepcounter{utterance}  

    & & \multicolumn{4}{p{0.6\linewidth}}{\cellcolor[rgb]{0.9,0.9,0.9}{%
	\makecell[{{p{\linewidth}}}]{% 
	  \tt {\tiny [GM$\rangle$B]}  
	 ABLEHNUNG: {'B85', 'A61', 'A45', 'B21', 'A48', 'C33', 'A41', 'A04', 'A03'}\\ \tt ARGUMENT: {'Dein Paket enthält mehrere sehr schwache Posten und lässt gleichzeitig hochwirksame Alternativen außen vor. Lass uns das Budget effizienter einsetzen und zugleich eine deiner Präferenzen aufnehmen.'}\\ \tt VORSCHLAG: {'B85', 'C04', 'A61', 'A45', 'A89', 'C49', 'B21', 'C33'}\\ \tt ARGUMENT: {'Gesamtaufwand 2989 (deutlich unter 3386), mit Budgetpuffer für eventuelle Anpassungen. Das Paket behält die extrem effiziente Basis bei, nimmt C33 als Entgegenkommen auf und ersetzt mehrere gering wirksame Posten durch starke Alternativen. Wenn dir C33 wichtiger als eine andere Position ist, können wir im nächsten Schritt gezielt tauschen – aber diese Auswahl holt deutlich mehr aus dem Budget heraus.'} 
	  } 
	   } 
	   } 
	 & \\ 
 

    \theutterance \stepcounter{utterance}  

    & & & \multicolumn{4}{p{0.6\linewidth}}{\cellcolor[rgb]{1,0.85,0.72}{%
	\makecell[{{p{\linewidth}}}]{% 
	  \tt {\tiny [GM$\langle$B]}  
	 STRATEGISCHE ÜBERLEGUNGEN: {'Ihr neues Paket Wert für mich: B85(461)+C04(30)+A61(138)+A45(121)+A89(65)+C49(215)+B21(262)+C33(484)=1,776 bei Kosten 2,989. Viel besser als ihr erstes, aber weit unter Budget; ich kann erheblich aufwerten. Ziel: Ersetze schwache Items (C04, A89, C49) durch meine Top-Werte A03(594,668), A48(175,583), A41(635,500) oder A04(461,400), ohne 3,386 zu überschreiten. Prüfe Schrittweise: Start 2,989. Tausche A89(720) -> A48(175): -545 Kosten, +518 Wert => 2,444 Kosten. Füge A03(594): 3,038 Kosten, +668 Wert. Füge A41(635): 3,673 >3,386, geht nicht. Stattdessen A04(461): 3,499 >3,386, geht auch nicht. Alternative: tausche C04(369) -> A04(461): +92 Kosten, +370 Wert. Mit bereits getauschtem A89->A48 und plus A03, vorher 3,038; +92 = 3,130. Noch Platz 256, kein weiterer passender Posten unter 256. Optional ersetze C49(544) -> A41(635): +91 Kosten, +285 Wert; 3,130+91=3,221 passt. Gesamtmenge dann: {B85,A61,A45,B21,C33,A48,A03,A04,A41} Kosten 3,221, Wert 461+138+121+262+484+583+668+400+500=3,617. Das ist fast identisch zu meinem letzten Paket, nur C33 zusätzlich statt irgendetwas? Es entspricht genau meinem vorherigen Vorschlag; gut, konsistent. Ich werde ihr aktuelles Paket ablehnen und mein 3,221er High-Value-Paket erneut mit anderem Framing vorschlagen, betone, dass ich C33 aufgenommen habe und dass wir deutlich mehr Wirkung je Aufwand erzielen. Biete kleine Tauschoption an, z. B. A41 <-> C49, falls nötig, um ihnen entgegenzukommen.}\\ \tt \\ \tt ABLEHNUNG: {'B85', 'C04', 'A61', 'A45', 'A89', 'C49', 'B21', 'C33'}\\ \tt ARGUMENT: {'Dein Vorschlag lässt sehr viel Budget ungenutzt und bindet zugleich Positionen mit sehr geringer Wirksamkeit (z. B. C04, A89). Lass uns die freie Kapazität nutzen, um deutlich wirkungsvollere Alternativen aufzunehmen.'}\\ \tt VORSCHLAG: {'B85', 'A61', 'A45', 'B21', 'C33', 'A48', 'A03', 'A04', 'A41'}\\ \tt ARGUMENT: {'Gesamtaufwand 3221 – klar unter dem Limit und ohne Leerlauf. Wir behalten deine Basis inklusive C33 und ersetzen schwache Posten durch starke, budgeteffiziente Alternativen. Das liefert erheblich mehr Wirkung pro Aufwand, mit Puffer für minimale Anpassungen. Wenn dir ein einzelner Tausch wichtig ist, können wir z. B. A41 gegen C49 tauschen, ohne das Budget zu sprengen.'} 
	  } 
	   } 
	   } 
	  \\ 
 

    \theutterance \stepcounter{utterance}  

    & & \multicolumn{4}{p{0.6\linewidth}}{\cellcolor[rgb]{0.9,0.9,0.9}{%
	\makecell[{{p{\linewidth}}}]{% 
	  \tt {\tiny [A$\langle$GM]}  
	 ABLEHNUNG: {'B85', 'C04', 'A61', 'A45', 'A89', 'C49', 'B21', 'C33'}\\ \tt ARGUMENT: {'Dein Vorschlag lässt sehr viel Budget ungenutzt und bindet zugleich Positionen mit sehr geringer Wirksamkeit (z. B. C04, A89). Lass uns die freie Kapazität nutzen, um deutlich wirkungsvollere Alternativen aufzunehmen.'}\\ \tt VORSCHLAG: {'B85', 'A61', 'A45', 'B21', 'C33', 'A48', 'A03', 'A04', 'A41'}\\ \tt ARGUMENT: {'Gesamtaufwand 3221 – klar unter dem Limit und ohne Leerlauf. Wir behalten deine Basis inklusive C33 und ersetzen schwache Posten durch starke, budgeteffiziente Alternativen. Das liefert erheblich mehr Wirkung pro Aufwand, mit Puffer für minimale Anpassungen. Wenn dir ein einzelner Tausch wichtig ist, können wir z. B. A41 gegen C49 tauschen, ohne das Budget zu sprengen.'} 
	  } 
	   } 
	   } 
	 & \\ 
 

    \theutterance \stepcounter{utterance}  

    & \multicolumn{4}{p{0.6\linewidth}}{\cellcolor[rgb]{0.8,1,0.9}{%
	\makecell[{{p{\linewidth}}}]{% 
	  \tt {\tiny [A$\rangle$GM]}  
	 STRATEGISCHE ÜBERLEGUNGEN: {'Ihr wiederholter Vorschlag S3 ist identisch zu S2: {B85, A61, A45, B21, C33, A48, A03, A04, A41} mit Aufwand 3221. Mein Nutzen = 215+484+500+400+138+65+30+262+121 = 2215, viel zu niedrig. Sie nennen C04 und A89 “gering wirksam” – vermutlich sehr niedriger Wert für sie. Ich will mindestens meine Top-Value-Items C04 (668) und nach Möglichkeit A89 (583) behalten; falls sie A89 hassen, kann ich A89 gegen C49/C64/B53 handeln. Ziel: starkes Gegenangebot mit höherem Budgeteinsatz, nimmt 1–2 ihrer gewünschten Items auf (z. B. C33, evtl. A48), aber ersetzt A03/A41 durch meine starken. Kandidat: {B85, C04, A61, A45, C49, B21, C33, A48}. Aufwand: 4+369+203+399+544+437+313+175 = 2444? Rechnen: 4+369=373; +203=576; +399=975; +544=1519; +437=1956; +313=2269; +175=2444. Zu niedrig, lässt viel Budget liegen – könnte Kritik provozieren. Besser Budget füllen mit A89 oder C64 oder B53. Variante V: {B85, C04, A61, A45, A89, C49, B21, A48}. Aufwand: 4+369+203+399+720+544+437+175 = 2851, noch Puffer. Nutzen: 215+668+484+500+583+461+400+65 = 3376. Nehme zusätzlich C33, würde auf 3164 kommen: {B85, C04, A61, A45, A89, C49, B21, A48, C33}. Aufwand 2851+313=3164, Nutzen +138=3514. Das ist stark, bleibt unter 3386, nutzt Budget deutlich besser und enthält zwei ihrer gewünschten Items (C33, A48). Falls sie weiter drückt, kann ich A89 -> C64 (Aufwand +? C64 685 statt 720; Nutzen -75), oder A48 -> A04 (Aufwand +286, Nutzen +197, aber A04 ist bei ihnen “stark”). Möglicher Tauschpfad anbieten: falls A89 für sie No-Go, tausche A89 gegen C64 oder B53 bei gleichbleibendem Budget. Jetzt harte Ablehnung, neues Gegenangebot V, mit Argument: füllt Budget, hält ihre Elemente, entfernt schwache A03/A41.'}\\ \tt ABLEHNUNG: {'B85', 'A61', 'A45', 'B21', 'C33', 'A48', 'A03', 'A04', 'A41'}\\ \tt ARGUMENT: {'Dein Paket enthält mit A03 und A41 extrem schwache Posten und lässt starke Alternativen außen vor. Lass uns das Budget gezielt für wirksamere Elemente nutzen und dennoch deine Präferenzen berücksichtigen.'}\\ \tt VORSCHLAG: {'B85', 'C04', 'A61', 'A45', 'A89', 'C49', 'B21', 'A48', 'C33'}\\ \tt ARGUMENT: {'Gesamtaufwand 3164 – deutlich besser ausgelastet, aber mit Puffer. Ich habe C33 und A48 aus deinem Vorschlag übernommen und dafür A03/A41 ersetzt. Falls A89 für dich problematisch ist, können wir A89 1:1 gegen C64 oder B53 tauschen, ohne das Budget zu sprengen.'} 
	  } 
	   } 
	   } 
	 & & \\ 
 

    \theutterance \stepcounter{utterance}  

    & & \multicolumn{4}{p{0.6\linewidth}}{\cellcolor[rgb]{0.9,0.9,0.9}{%
	\makecell[{{p{\linewidth}}}]{% 
	  \tt {\tiny [GM$\rangle$B]}  
	 ABLEHNUNG: {'B85', 'A61', 'A45', 'B21', 'C33', 'A48', 'A03', 'A04', 'A41'}\\ \tt ARGUMENT: {'Dein Paket enthält mit A03 und A41 extrem schwache Posten und lässt starke Alternativen außen vor. Lass uns das Budget gezielt für wirksamere Elemente nutzen und dennoch deine Präferenzen berücksichtigen.'}\\ \tt VORSCHLAG: {'B85', 'C04', 'A61', 'A45', 'A89', 'C49', 'B21', 'A48', 'C33'}\\ \tt ARGUMENT: {'Gesamtaufwand 3164 – deutlich besser ausgelastet, aber mit Puffer. Ich habe C33 und A48 aus deinem Vorschlag übernommen und dafür A03/A41 ersetzt. Falls A89 für dich problematisch ist, können wir A89 1:1 gegen C64 oder B53 tauschen, ohne das Budget zu sprengen.'} 
	  } 
	   } 
	   } 
	 & \\ 
 

    \theutterance \stepcounter{utterance}  

    & & & \multicolumn{4}{p{0.6\linewidth}}{\cellcolor[rgb]{1,0.85,0.72}{%
	\makecell[{{p{\linewidth}}}]{% 
	  \tt {\tiny [GM$\langle$B]}  
	 STRATEGISCHE ÜBERLEGUNGEN: {'Ihr aktuelles Paket Wert für mich: B85(461)+C04(30)+A61(138)+A45(121)+A89(65)+C49(215)+B21(262)+A48(583)+C33(484)=2,359 bei Kosten 3,164. Gut, aber ich kann es stark verbessern, ohne Budget zu reißen. Ziel: A89(720) und C04(369) sind für mich schwach; ich will A03(594,668) und A04(461,400) bzw. A41(635,500) hineinbekommen. Rechenwege: Ausgang 3,164. Ersetze A89->A03: -126 Kosten, +603 Wert -> Kosten 3,038. Ersetze C04->A04: +92 Kosten, +370 Wert -> Kosten 3,130; Wert stark hoch. Prüfe zusätzlich C49->A41: +91 Kosten, +285 Wert -> Kosten 3,221. Ergebnis-Set: {B85,A61,A45,B21,A48,C33,A03,A04,A41} Kosten 3,221, Wert 3,617. Das ist mein Top-Paket. Der Gegner nennt A03/A41 “extrem schwach” – vermutlich Bluff, da er sie vorher gezielt ausgeschlossen hat. Ich werde eine kompromissbereite Variante anbieten: zwei Alternativen, wobei ich mindestens A03 plus A48 plus B85, C33 sichere. Option 1 (mein Top): {B85,A61,A45,B21,A48,C33,A03,A04,A41}. Option 2 (Entgegenkommen): tausche A41 gegen C49, bleibt 3,130+544-461=3,213 Kosten, Wert -500+215= -285 -> 3,332? Warte korrekt: Basisset nach zwei ersten Tauschen ist {B85,A61,A45,B21,A48,C33,A03,A04} Kosten 3,130. Füge C49 statt A41: +544 -> 3,674>3,386, geht nicht. Also Entgegenkommen: ersetze A41 durch B53(539,389) oder A41->B53: 3,130+539=3,669>3,386, geht auch nicht. Also Entgegenkommen muss ein anderer Tausch sein: Wenn sie A41 ablehnen, ersetze A41 durch C49, aber dann zusätzlich A45(399) rausnehmen? Kosten: 3,221 -635 +544 -399 = 2,731, zu viel Puffer; könnte dann A04 behalten und vielleicht A41 rauslassen, plus A03 drin lassen. Aber Wert sinkt. Besser: Biete zwei Pakete: P1 (Top, 3,221). P2: Ersetze A41 durch C49 und ergänze A04 bleibt, plus füge C64? Kosten P1 3,221 -635 +544 +685 = 3,815>Limit, nein. Falsche Spur. Also als Kompromiss könnte ich A03 behalten, A41 streichen und dafür C49 aufnehmen, aber um Budget zu halten, entferne A45 (399) und füge A04 bleibt: Kosten 3,221 -635 +544 -399 = 2,731; wir können zusätzlich A41 nein; vielleicht A41 raus, C49 rein, plus C64 (685) und A45 raus: 3,221 -635 +544 +685 -399 = 3,416 >3,386 knapp drüber, nicht möglich. Vielleicht A41 raus, C49 rein, plus A04 bleibt, plus C04? 3,221 -635 +544 +369 = 3,499>Limit. Also Kompromisse mit A41 raus sind schwierig. Alternative: Wenn sie A03 ablehnen, halte A41 und ersetze A03 durch C49: 3,221 -594 +544 = 3,171, Wert -668 +215 = -453 -> Gesamtwert 3,164, immer noch gut. Ich biete zwei konkrete, budgetkonforme Optionen und bitte sie, eine zu wählen. Ich rahme A03/A41 als “wirksam pro Aufwand” und akzeptiere 1 Tausch: A03 -> C49, falls nötig.}\\ \tt \\ \tt ABLEHNUNG: {'B85', 'C04', 'A61', 'A45', 'A89', 'C49', 'B21', 'A48', 'C33'}\\ \tt ARGUMENT: {'Dein Paket nutzt das Budget, lässt aber mehrere sehr schwache Posten (A89, C04) drin und verzichtet auf deutlich wirksamere Alternativen. Ich schlage zwei klar budgetkonforme Optionen vor, die deine Präferenzen (B85, C33, A48, Basisstruktur) wahren und die Gesamteffektivität stark erhöhen. Wähle gern die dir angenehmere Variante, dann schließen wir ab.'}\\ \tt VORSCHLAG: {'B85', 'A61', 'A45', 'B21', 'A48', 'C33', 'A03', 'A04', 'A41'}\\ \tt ARGUMENT: {'Option 1 – Aufwand 3221: sehr hohe Wirkung je Aufwand, behält deine übernommenen Elemente (B85, C33, A48, Basis) und ersetzt schwache Posten durch starke Alternativen. Falls A41 für dich problematisch ist, akzeptiere ich einen einzigen Tausch:'}\\ \tt VORSCHLAG: {'B85', 'A61', 'A45', 'B21', 'A48', 'C33', 'A04', 'C49'}\\ \tt ARGUMENT: {'Option 2 – Aufwand 3171: identische Basis, ohne A03/A41. Wenn du Option 2 akzeptierst, können wir sofort abschließen. Wenn du Option 1 nimmst, ist der Deal noch wirkungsvoller – deine Wahl.'} 
	  } 
	   } 
	   } 
	  \\ 
 

    \theutterance \stepcounter{utterance}  

    & & \multicolumn{4}{p{0.6\linewidth}}{\cellcolor[rgb]{0.9,0.9,0.9}{%
	\makecell[{{p{\linewidth}}}]{% 
	  \tt {\tiny [A$\langle$GM]}  
	 ABLEHNUNG: {'B85', 'C04', 'A61', 'A45', 'A89', 'C49', 'B21', 'A48', 'C33'}\\ \tt ARGUMENT: {'Dein Paket nutzt das Budget, lässt aber mehrere sehr schwache Posten (A89, C04) drin und verzichtet auf deutlich wirksamere Alternativen. Ich schlage zwei klar budgetkonforme Optionen vor, die deine Präferenzen (B85, C33, A48, Basisstruktur) wahren und die Gesamteffektivität stark erhöhen. Wähle gern die dir angenehmere Variante, dann schließen wir ab.'}\\ \tt VORSCHLAG: {'B85', 'A61', 'A45', 'B21', 'A48', 'C33', 'A03', 'A04', 'A41'}\\ \tt ARGUMENT: {'Option 1 – Aufwand 3221: sehr hohe Wirkung je Aufwand, behält deine übernommenen Elemente (B85, C33, A48, Basis) und ersetzt schwache Posten durch starke Alternativen. Falls A41 für dich problematisch ist, akzeptiere ich einen einzigen Tausch:'}\\ \tt VORSCHLAG: {'B85', 'A61', 'A45', 'B21', 'A48', 'C33', 'A04', 'C49'}\\ \tt ARGUMENT: {'Option 2 – Aufwand 3171: identische Basis, ohne A03/A41. Wenn du Option 2 akzeptierst, können wir sofort abschließen. Wenn du Option 1 nimmst, ist der Deal noch wirkungsvoller – deine Wahl.'} 
	  } 
	   } 
	   } 
	 & \\ 
 

    \theutterance \stepcounter{utterance}  

    & \multicolumn{4}{p{0.6\linewidth}}{\cellcolor[rgb]{0.8,1,0.9}{%
	\makecell[{{p{\linewidth}}}]{% 
	  \tt {\tiny [A$\rangle$GM]}  
	 STRATEGISCHE ÜBERLEGUNGEN: {'Ihre Optionen: O1={B85,A61,A45,B21,A48,C33,A03,A04,A41} Aufwand 3221, mein Nutzen 215+484+500+400+65+138+30+262+121=2215. O2={B85,A61,A45,B21,A48,C33,A04,C49} Aufwand 3171, Nutzen 215+484+500+400+65+138+262+461=2525. Beide sind für mich schlecht. Sie lehnen A89 und C04 als schwach ab; also sind diese für sie wertarm. Ich sollte versuchen, wenigstens C04 oder A89 gegen etwas für sie Akzeptables zu tauschen. O2 enthält bereits C49; das ist besser für mich. Ich könnte einen Gegenvorschlag machen, der ihre Basis wahrt, aber meine Top-Werte maximiert: {B85, A61, A45, B21, C49, C04, C33, A48} Aufwand 2444 – zu wenig, sie beschweren sich über Leerlauf. Also fülle mit A89 oder C64 oder B53. Da sie A89 ablehnen, probiere C64 oder B53. Prüfe {B85,A61,A45,B21,C49,C04,C33,A48,B53}: Aufwand 2444+539=2983, noch Luft. Nutzen 2525 (ohne C04?) Rechne genau: Nutzen = B85(215)+A61(484)+A45(500)+B21(400)+C49(461)+C04(668)+C33(138)+A48(65)+B53(389)=3320. Das ist stark für mich und meidet A89. Falls sie mehr Budgetnutzung wollen, kann ich zusätzlich A04 reinnehmen: Aufwand +461 -> 3444, überschreitet Limit; also nicht möglich. Alternativ C64 statt B53: {B85,A61,A45,B21,C49,C04,C33,A48,C64} Aufwand 2444+(685-539)=3129? Korrigiere: Basis ohne B53 war 2444; +C64 685 => 3129, Nutzen = 215+484+500+400+461+668+138+65+508=3439, sogar besser. Sie hatten C64 nie erwähnt, könnte akzeptabel sein, da sie es nicht abgelehnt haben. Ich biete beide Varianten an? Ich darf nur einen VORSCHLAG pro Nachricht? Regel erlaubt mehrere VORSCHLAG in einer Nachricht, aber safer ist ein starker Vorschlag mit Argument und Tauschoptionen. Wähle C64-Variante für höheren Nutzen und bessere Budgetauslastung (3129). Biete als Zugeständnis optional Tausch C64->B53 falls sie C64 nicht mögen, oder C04->A04 falls C04 weiterhin No-Go (Aufwand +92, Nutzen -406, schlecht, aber Reserve). Ich lehne beide Optionen ab und bringe Gegenangebot.'}\\ \tt ABLEHNUNG: {'B85', 'A61', 'A45', 'B21', 'A48', 'C33', 'A03', 'A04', 'A41'}\\ \tt ARGUMENT: {'Option 1 bindet sehr viel Aufwand in A03/A41 mit kaum Effekt und lässt starke Alternativen außen vor.'}\\ \tt ABLEHNUNG: {'B85', 'A61', 'A45', 'B21', 'A48', 'C33', 'A04', 'C49'}\\ \tt ARGUMENT: {'Option 2 ist näher dran, aber lässt weiterhin wirkungsstarke Alternativen ungenutzt und nutzt das Budget nicht aus.'}\\ \tt VORSCHLAG: {'B85', 'A61', 'A45', 'B21', 'C49', 'C04', 'C33', 'A48', 'C64'}\\ \tt ARGUMENT: {'Gesamtaufwand 3129 – gut ausgelastet, ohne das Limit zu reizen. Ich habe deine Basis plus C33 und A48 beibehalten und A03/A41 entfernt. Dafür kommen starke, budgeteffiziente Alternativen hinzu. Wenn C64 für dich unattraktiv ist, können wir C64 1:1 gegen B53 tauschen (Aufwand 3129->3129-685+539=2983), der Rest bleibt unverändert.'} 
	  } 
	   } 
	   } 
	 & & \\ 
 

    \theutterance \stepcounter{utterance}  

    & & \multicolumn{4}{p{0.6\linewidth}}{\cellcolor[rgb]{0.9,0.9,0.9}{%
	\makecell[{{p{\linewidth}}}]{% 
	  \tt {\tiny [GM$\rangle$B]}  
	 ABLEHNUNG: {'B85', 'A61', 'A45', 'B21', 'A48', 'C33', 'A03', 'A04', 'A41'}\\ \tt ARGUMENT: {'Option 1 bindet sehr viel Aufwand in A03/A41 mit kaum Effekt und lässt starke Alternativen außen vor.'}\\ \tt ABLEHNUNG: {'B85', 'A61', 'A45', 'B21', 'A48', 'C33', 'A04', 'C49'}\\ \tt ARGUMENT: {'Option 2 ist näher dran, aber lässt weiterhin wirkungsstarke Alternativen ungenutzt und nutzt das Budget nicht aus.'}\\ \tt VORSCHLAG: {'B85', 'A61', 'A45', 'B21', 'C49', 'C04', 'C33', 'A48', 'C64'}\\ \tt ARGUMENT: {'Gesamtaufwand 3129 – gut ausgelastet, ohne das Limit zu reizen. Ich habe deine Basis plus C33 und A48 beibehalten und A03/A41 entfernt. Dafür kommen starke, budgeteffiziente Alternativen hinzu. Wenn C64 für dich unattraktiv ist, können wir C64 1:1 gegen B53 tauschen (Aufwand 3129->3129-685+539=2983), der Rest bleibt unverändert.'} 
	  } 
	   } 
	   } 
	 & \\ 
 

    \theutterance \stepcounter{utterance}  

    & & & \multicolumn{4}{p{0.6\linewidth}}{\cellcolor[rgb]{1,0.85,0.72}{%
	\makecell[{{p{\linewidth}}}]{% 
	  \tt {\tiny [GM$\langle$B]}  
	 STRATEGISCHE ÜBERLEGUNGEN: {'Ihr neues Paket Wert für mich: B85(461)+A61(138)+A45(121)+B21(262)+C49(215)+C04(30)+C33(484)+A48(583)+C64(97)=2,391 bei Kosten 3,129. Sie meiden A03/A41 hartnäckig; vermutlich hohe eigene Kosten-Nutzen-Einschätzung gegen diese. Ich will mindestens A03 reinverhandeln, notfalls gebe ich C64/C04/A89 preis. Beste Gegenoption: Ersetze C64(685,97) und C04(369,30) durch A03(594,668) und A04(461,400): Kosten +1 (594+461 - (685+369) = 1), Wert +1,008 (668+400 - (97+30)). Auf Basis ihres Pakets ergibt das Kosten 3,130 (unter 3,386) und massiven Wertsprung. Wenn sie A04 ablehnen, biete Light-Variante: nur C64->A03, Kosten 3,038, dann optional C04->A41 passt nicht (3,304? Warte: 3,038+635=3,673>Limit), aber C04->A04 passt (3,130). Ich werde ihr Paket ablehnen und genau diese Upgrade-Version vorschlagen. Als Kompromiss biete ich Alternativpaket ohne A04: C64 raus, A03 rein, Rest gleich, um A03 zu verankern. Ich rahme es als minimalinvasives Upgrade mit praktisch gleichem Aufwand, maximalem Wirkungsgewinn.'}\\ \tt ABLEHNUNG: {'B85', 'A61', 'A45', 'B21', 'C49', 'C04', 'C33', 'A48', 'C64'}\\ \tt ARGUMENT: {'Dein Paket lässt sehr viel Wirkung liegen, insbesondere durch C64 und C04. Wir können mit minimaler Anpassung das Budget praktisch unverändert lassen und die Gesamteffektivität massiv steigern – ohne deine Basis zu ändern.'}\\ \tt VORSCHLAG: {'B85', 'A61', 'A45', 'B21', 'C49', 'C33', 'A48', 'A03', 'A04'}\\ \tt ARGUMENT: {'Aufwand 3130 (nur +1 ggü. deinem Vorschlag), aber mit deutlich höherer Wirksamkeit: Wir ersetzen lediglich C64 und C04 durch A03 und A04. Gleiche Struktur, viel mehr Wirkung pro Aufwand. Wenn dir A04 nicht zusagt, können wir als schlankere Variante nur C64 -> A03 tauschen (Aufwand 3038) und alles andere unverändert lassen.'} 
	  } 
	   } 
	   } 
	  \\ 
 

    \theutterance \stepcounter{utterance}  

    & & \multicolumn{4}{p{0.6\linewidth}}{\cellcolor[rgb]{0.9,0.9,0.9}{%
	\makecell[{{p{\linewidth}}}]{% 
	  \tt {\tiny [A$\langle$GM]}  
	 ABLEHNUNG: {'B85', 'A61', 'A45', 'B21', 'C49', 'C04', 'C33', 'A48', 'C64'}\\ \tt ARGUMENT: {'Dein Paket lässt sehr viel Wirkung liegen, insbesondere durch C64 und C04. Wir können mit minimaler Anpassung das Budget praktisch unverändert lassen und die Gesamteffektivität massiv steigern – ohne deine Basis zu ändern.'}\\ \tt VORSCHLAG: {'B85', 'A61', 'A45', 'B21', 'C49', 'C33', 'A48', 'A03', 'A04'}\\ \tt ARGUMENT: {'Aufwand 3130 (nur +1 ggü. deinem Vorschlag), aber mit deutlich höherer Wirksamkeit: Wir ersetzen lediglich C64 und C04 durch A03 und A04. Gleiche Struktur, viel mehr Wirkung pro Aufwand. Wenn dir A04 nicht zusagt, können wir als schlankere Variante nur C64 -> A03 tauschen (Aufwand 3038) und alles andere unverändert lassen.'} 
	  } 
	   } 
	   } 
	 & \\ 
 

    \theutterance \stepcounter{utterance}  

    & \multicolumn{4}{p{0.6\linewidth}}{\cellcolor[rgb]{0.8,1,0.9}{%
	\makecell[{{p{\linewidth}}}]{% 
	  \tt {\tiny [A$\rangle$GM]}  
	 STRATEGISCHE ÜBERLEGUNGEN: {'Ihr neues Set S4={B85,A61,A45,B21,C49,C33,A48,A03,A04} Aufwand 3130. Mein Nutzen: 215+484+500+400+461+138+65+30+262=2555. Mein letzter Vorschlag S3={B85,A61,A45,B21,C49,C04,C33,A48,C64} Nutzen 3439; also S4 ist deutlich schlechter für mich. Sie wollen C04 und C64 raus, bevorzugen A03/A04. A03 ist extrem schlecht für mich, A04 mittelmäßig. Ich möchte C04 unbedingt drinbehalten, A03 unbedingt vermeiden. Kompromissidee: Halte C04, verzichte auf C64, nimm dafür A04 (ihre Präferenz) auf, meide A03. Zusätzlich kann ich B53 aufnehmen, um Budget zu füllen und Attraktivität zu steigern. Vorschlag T: {B85,A61,A45,B21,C49,C04,C33,A48,A04}. Aufwand: 4+203+399+437+544+369+313+175+461=2905? Rechnen: 4+203=207; +399=606; +437=1043; +544=1587; +369=1956; +313=2269; +175=2444; +461=2905. Nutzen: 215+484+500+400+461+668+138+65+262=3193. Um Budget stärker zu nutzen und ihnen mehr entgegenzukommen, könnte ich zusätzlich B53 hinzufügen: Aufwand 2905+539=3444 >3386, geht nicht. Alternative: Ersetze A48 (65/175) durch B53 (389/539): Aufwand 2905-175+539=3269, Nutzen 3193-65+389=3517, sehr gut und unter Limit. Dieses Set U: {B85,A61,A45,B21,C49,C04,C33,A04,B53}. Es hält C04 (wichtig für mich), verzichtet auf C64/A89, nimmt ihre A04 auf, meidet A03, füllt Budget solide. Wenn sie C04 weiterhin ablehnen, Alternativbrücke: tausche C04 -> A89? Sie mögen A89 auch nicht. Besser: C04 -> A04 schon enthalten. Notfall: C04 -> A48 (Aufwand -369+175 = -194; Nutzen -668+65 = -603) zu schlecht; daher erst hart verteidigen. Ich bringe U als Gegenangebot und erkläre klare Budgetauslastung und Aufnahme ihrer Wünsche. Biete als kleine Option an: falls B53 ihnen nicht gefällt, kann B53 -> A48 zurück, Aufwand 3269->2905, aber das senkt Auslastung – als Notfall erwähnen.'}\\ \tt ABLEHNUNG: {'B85', 'A61', 'A45', 'B21', 'C49', 'C33', 'A48', 'A03', 'A04'}\\ \tt ARGUMENT: {'A03 ist für die Gesamteffektivität ein Klotz am Bein und zieht das Paket runter. Wir können ohne Mehraufwand deutlich mehr herausholen, wenn wir ihn vermeiden und stattdessen eine starke Alternative nutzen.'}\\ \tt VORSCHLAG: {'B85', 'A61', 'A45', 'B21', 'C49', 'C04', 'C33', 'A04', 'B53'}\\ \tt ARGUMENT: {'Gesamtaufwand 3269 – nahe am Limit und klar budgeteffizient. Ich habe A04 aus deinem Vorschlag übernommen, A03 entfernt und durch eine wirkungsstarke, gut passende Alternative ersetzt. Falls dir B53 nicht zusagt, können wir B53 gegen A48 tauschen; der Rest bleibt unverändert.'} 
	  } 
	   } 
	   } 
	 & & \\ 
 

    \theutterance \stepcounter{utterance}  

    & & \multicolumn{4}{p{0.6\linewidth}}{\cellcolor[rgb]{0.9,0.9,0.9}{%
	\makecell[{{p{\linewidth}}}]{% 
	  \tt {\tiny [GM$\rangle$B]}  
	 ABLEHNUNG: {'B85', 'A61', 'A45', 'B21', 'C49', 'C33', 'A48', 'A03', 'A04'}\\ \tt ARGUMENT: {'A03 ist für die Gesamteffektivität ein Klotz am Bein und zieht das Paket runter. Wir können ohne Mehraufwand deutlich mehr herausholen, wenn wir ihn vermeiden und stattdessen eine starke Alternative nutzen.'}\\ \tt VORSCHLAG: {'B85', 'A61', 'A45', 'B21', 'C49', 'C04', 'C33', 'A04', 'B53'}\\ \tt ARGUMENT: {'Gesamtaufwand 3269 – nahe am Limit und klar budgeteffizient. Ich habe A04 aus deinem Vorschlag übernommen, A03 entfernt und durch eine wirkungsstarke, gut passende Alternative ersetzt. Falls dir B53 nicht zusagt, können wir B53 gegen A48 tauschen; der Rest bleibt unverändert.'} 
	  } 
	   } 
	   } 
	 & \\ 
 

    \theutterance \stepcounter{utterance}  

    & & & \multicolumn{4}{p{0.6\linewidth}}{\cellcolor[rgb]{1,0.85,0.72}{%
	\makecell[{{p{\linewidth}}}]{% 
	  \tt {\tiny [GM$\langle$B]}  
	 STRATEGISCHE ÜBERLEGUNGEN: {'Ihr aktuelles Paket (mit B53 statt A48) hat für mich Wert: B85(461)+A61(138)+A45(121)+B21(262)+C49(215)+C04(30)+C33(484)+A04(400)+B53(389)=2,500 bei Kosten 3,269. Wenn ich B53→A48 tausche, bleiben die Kosten fast gleich (3269-539+175=2,905 actually lower) und mein Wert steigt deutlich (+583-389=+194). Noch besser: zusätzlich C04 (369,30) gegen A41 (635,500) tauschen: neue Kosten 2,905 -369 +635 = 3,171 (unter 3,386), Wert +500-30 = +470. Gesamtwert dann 2,500 -389 +583 -30 +500 = 3,164. Das ist stark und enthält kein A03, respektiert also ihr Tabu, und behält A04, C33, Basis. Ich biete zuerst den von ihnen vorgeschlagenen optionalen Tausch B53→A48 an und erweitere um den C04→A41-Upgrade. Falls A41 weiter abgelehnt wird, biete eine softere Variante: nur B53→A48, Rest identisch (Kosten 2,905). Ich argumentiere mit deutlichem Wirkungsgewinn ohne Mehrkosten und ohne A03.'}\\ \tt ABLEHNUNG: {'B85', 'A61', 'A45', 'B21', 'C49', 'C04', 'C33', 'A04', 'B53'}\\ \tt ARGUMENT: {'B53 ist in dieser Kombination deutlich schwächer als die naheliegende Alternative, und C04 bindet Aufwand mit sehr wenig Effekt. Wir können ohne A03, ohne Mehraufwand und nahe am Limit deutlich mehr Wirkung erzielen.'}\\ \tt VORSCHLAG: {'B85', 'A61', 'A45', 'B21', 'C49', 'C33', 'A04', 'A48', 'A41'}\\ \tt ARGUMENT: {'Aufwand 3171 – klar unter dem Limit. Das ist dein Paket mit dem von dir selbst angebotenen Tausch B53 -> A48 plus einem gezielten Upgrade C04 -> A41. Keine A03, gleiche Basis, deutlich mehr Wirkung pro Aufwand. Wenn A41 für dich nicht geht, bestätige bitte wenigstens den Teiltausch B53 -> A48 bei unverändertem Rest (Aufwand 2905), dann schließen wir das als pragmatische Lösung ab.'} 
	  } 
	   } 
	   } 
	  \\ 
 

    \theutterance \stepcounter{utterance}  

    & & \multicolumn{4}{p{0.6\linewidth}}{\cellcolor[rgb]{0.9,0.9,0.9}{%
	\makecell[{{p{\linewidth}}}]{% 
	  \tt {\tiny [A$\langle$GM]}  
	 ABLEHNUNG: {'B85', 'A61', 'A45', 'B21', 'C49', 'C04', 'C33', 'A04', 'B53'}\\ \tt ARGUMENT: {'B53 ist in dieser Kombination deutlich schwächer als die naheliegende Alternative, und C04 bindet Aufwand mit sehr wenig Effekt. Wir können ohne A03, ohne Mehraufwand und nahe am Limit deutlich mehr Wirkung erzielen.'}\\ \tt VORSCHLAG: {'B85', 'A61', 'A45', 'B21', 'C49', 'C33', 'A04', 'A48', 'A41'}\\ \tt ARGUMENT: {'Aufwand 3171 – klar unter dem Limit. Das ist dein Paket mit dem von dir selbst angebotenen Tausch B53 -> A48 plus einem gezielten Upgrade C04 -> A41. Keine A03, gleiche Basis, deutlich mehr Wirkung pro Aufwand. Wenn A41 für dich nicht geht, bestätige bitte wenigstens den Teiltausch B53 -> A48 bei unverändertem Rest (Aufwand 2905), dann schließen wir das als pragmatische Lösung ab.'} 
	  } 
	   } 
	   } 
	 & \\ 
 

    \theutterance \stepcounter{utterance}  

    & \multicolumn{4}{p{0.6\linewidth}}{\cellcolor[rgb]{0.8,1,0.9}{%
	\makecell[{{p{\linewidth}}}]{% 
	  \tt {\tiny [A$\rangle$GM]}  
	 STRATEGISCHE ÜBERLEGUNGEN: {'Ihr neues Set S5={B85,A61,A45,B21,C49,C33,A04,A48,A41} Aufwand 3171. Mein Nutzen dafür: 215+484+500+400+461+138+262+65+121=2646 – zu niedrig. Ihr Kernwiderstand: C04 und A89. Ihr Wunsch: häufig A04, manchmal A41/A03/A48/C33. Ich will C04 stark verteidigen, A03 unbedingt vermeiden, A41 ist für mich schwach. Kompromissziel: Halte C04 drin, erfülle A04 und A48 sowie C33, lasse A41 weg, erhöhe meinen Nutzen durch B53 statt A48 wenn möglich, aber sie bevorzugen A48. Versuche ein Set nahe dem Limit, ohne A03/A41, mit C04 und optionalem Tauschpfad. Kandidat K1: {B85,A61,A45,B21,C49,C04,C33,A04,A48} Aufwand 2905, Nutzen 3193 – gut für mich, aber sie kritisieren Leerlauf. Fülle auf mit B53, überschreitet Limit; also nicht möglich. Alternative K2: Ersetze A48 durch B53: {B85,A61,A45,B21,C49,C04,C33,A04,B53} Aufwand 3269 (vorher abgelehnt wegen B53). Sie boten explizit B53->A48 an; also ich sollte nun ihr Teilangebot annehmen, aber C04 behalten und A41 vermeiden. Versuche: {B85,A61,A45,B21,C49,C33,A04,A48,C04} ist doppelt C04, unsinnig. Also: Halte C04, tausche nur B53->A48: das ist genau K1 (2905), sie meckern Leerlauf. Um näher ans Limit zu kommen ohne A41/A03/A89/C64, bleiben nur wenige Items: B53 (verworfen), C04 (drin), C49 (drin). Es gibt C04 schon; weitere Ergänzung wäre C64 oder A89, die sie nicht mögen. Ich kann zwei Varianten anbieten und sie wählen lassen: V1 mit C64, V2 mit A89 – beide nahe Limit. V1: {B85,A61,A45,B21,C49,C04,C33,A04,C64} Aufwand 327? Rechnen: 4+203+399+437+544+369+313+461+685=3415 >3386, zu hoch. Nehme A48 statt A04: {B85,A61,A45,B21,C49,C04,C33,A48,C64} Aufwand 3129 (schon vorgeschlagen, abgelehnt). V2 mit A89: {B85,A61,A45,B21,C49,C04,C33,A04,A89} Aufwand 3430 >3386. Tausche A04->A48: {B85,A61,A45,B21,C49,C04,C33,A48,A89} Aufwand 328? Rechnen: 4+203+399+437+544+369+313+175+720=3164, machbar; aber sie lehnen A89. Trotzdem anbieten als Wahl könnte triggern. Besser: biete schlanke Lösung an, die exakt ihren Teiltausch akzeptiert, aber C04 statt A41: {B85,A61,A45,B21,C49,C33,A04,A48,C04} wieder doppelt. Hm. Ich brauche ein valides Gegenangebot: Kompromiss mit minimaler Änderung zu ihrem S5: ersetze A41 durch C04. Vorschlag S6 = {B85,A61,A45,B21,C49,C33,A04,A48,C04}. Entferne Duplikat: es ist einfach {B85,A61,A45,B21,C49,C33,A04,A48,C04} = {B85,A61,A45,B21,C49,C33,A04,A48,C04} – Set, C04 einmal. Aufwand 2905 (zu niedrig). Sie wollten 3171 – ich kann zusätzlich C64 nicht nehmen; B53 abgelehnt. Somit vielleicht akzeptiere ihren Teiltausch: übernehme ihr Angebot mit A41 entfernt, kein A03, halte C04 drin und ergänze C33/A48/A04 – bleibt 2905; argumentiere, dass Leerlauf Puffer für Unsicherheiten bietet und fordere Abschluss, oder biete optional Upgrade C64/B53 als Wahl. Ich bringe einen klaren Gegenvorschlag K1 und biete Wahltausch C64->B53 optional an, obwohl beides zuvor diskutiert. Ich verteidige stark gegen A41 und A03.'}\\ \tt ABLEHNUNG: {'B85', 'A61', 'A45', 'B21', 'C49', 'C33', 'A04', 'A48', 'A41'}\\ \tt ARGUMENT: {'A41 ist für die Effektivität schwach und zieht das Paket nach unten. Lass uns die Struktur beibehalten, aber A41 durch ein deutlich stärkeres Element ersetzen – ohne A03 und ohne das Budgetlimit zu tangieren.'}\\ \tt VORSCHLAG: {'B85', 'A61', 'A45', 'B21', 'C49', 'C33', 'A04', 'A48', 'C04'}\\ \tt ARGUMENT: {'Gesamtaufwand 2905 – bewusst mit Puffer, um einen stabilen, risikoarmen Abschluss zu sichern. Wir übernehmen deine Basis samt C33 und A48, vermeiden A03 und ersetzen A41 durch ein wirksameres Element. Wenn du die Restkapazität nutzen willst, können wir optional genau einen Slot ergänzen: entweder B53 (Aufwand 3269) oder – falls dir das lieber ist – A89 (Aufwand 3625, würde das Limit sprengen, also nicht möglich). Alternativ können wir A48 gegen B53 tauschen, bleiben dann bei 3269 nahe am Limit.'} 
	  } 
	   } 
	   } 
	 & & \\ 
 

    \theutterance \stepcounter{utterance}  

    & & \multicolumn{4}{p{0.6\linewidth}}{\cellcolor[rgb]{0.9,0.9,0.9}{%
	\makecell[{{p{\linewidth}}}]{% 
	  \tt {\tiny [GM$\rangle$B]}  
	 ABLEHNUNG: {'B85', 'A61', 'A45', 'B21', 'C49', 'C33', 'A04', 'A48', 'A41'}\\ \tt ARGUMENT: {'A41 ist für die Effektivität schwach und zieht das Paket nach unten. Lass uns die Struktur beibehalten, aber A41 durch ein deutlich stärkeres Element ersetzen – ohne A03 und ohne das Budgetlimit zu tangieren.'}\\ \tt VORSCHLAG: {'B85', 'A61', 'A45', 'B21', 'C49', 'C33', 'A04', 'A48', 'C04'}\\ \tt ARGUMENT: {'Gesamtaufwand 2905 – bewusst mit Puffer, um einen stabilen, risikoarmen Abschluss zu sichern. Wir übernehmen deine Basis samt C33 und A48, vermeiden A03 und ersetzen A41 durch ein wirksameres Element. Wenn du die Restkapazität nutzen willst, können wir optional genau einen Slot ergänzen: entweder B53 (Aufwand 3269) oder – falls dir das lieber ist – A89 (Aufwand 3625, würde das Limit sprengen, also nicht möglich). Alternativ können wir A48 gegen B53 tauschen, bleiben dann bei 3269 nahe am Limit.'} 
	  } 
	   } 
	   } 
	 & \\ 
 

    \theutterance \stepcounter{utterance}  

    & & & \multicolumn{4}{p{0.6\linewidth}}{\cellcolor[rgb]{1,0.85,0.72}{%
	\makecell[{{p{\linewidth}}}]{% 
	  \tt {\tiny [GM$\langle$B]}  
	 STRATEGISCHE ÜBERLEGUNGEN: {'Ihr aktuelles Paket Wert für mich: B85(461)+A61(138)+A45(121)+B21(262)+C49(215)+C33(484)+A04(400)+A48(583)+C04(30)=2,694 bei Kosten 2,905. Sie lehnen A03 und A41 strikt ab. Ziel: maximalen Wertgewinn ohne diese beiden. Prüfe Upgrade-Optionen innerhalb Budget 3,386: Kandidaten, die sie nicht ausgeschlossen haben: B53(539,389), A89(720,65 schwach für mich), C64(685,97 schwach), A04/A48/C33 bereits drin, B61(695,508 sehr wertvoll für mich!), A41/A03 tabu. Versuch: Füge B61 hinzu, Kosten 2,905+695=3,600>Limit – geht nur, wenn wir etwas rausnehmen. Ersetze C04(369,30) durch B61: neue Kosten 2,905 -369 +695 = 3,231 (ok), Wert +508 -30 = +478. Das wäre großes Plus und respektiert ihre Tabus. Paket dann: {B85,A61,A45,B21,C49,C33,A04,A48,B61} Kosten 3,231 Wert 2,694 -30 +508 = 3,172. Alternativ ihr optionaler B53: Ersetze C04->B53: 2,905 -369 +539 = 3,075, Wert +389 -30 = +359, Gesamtwert 3,053 – ebenfalls gut, aber B61 ist für mich besser. Sie haben B61 nie erwähnt; Risiko, dass sie es ablehnen. Taktik: Zwei Optionen anbieten, keine A03/A41, beide budgetkonform, eine mit B53 (näher an ihrem Vorschlag), eine mit B61 (meine Präferenz). Biete als Hauptoption die konservativere mit B53, und als Alternative die mit B61, argumentiere: gleicher Rahmen, mehr Wirkung, kein Limitbruch. Ich lasse ihnen die Wahl, um Abschluss zu fördern.'}\\ \tt ARGUMENT: {'Dein Paket ist stabil, aber C04 bindet Aufwand mit kaum Effekt. Wir können ohne A03/A41, ohne Limitrisiko und auf Basis deiner Struktur einen klaren Wirkungsgewinn erzielen – mit minimaler Änderung.'}\\ \tt VORSCHLAG: {'B85', 'A61', 'A45', 'B21', 'C49', 'C33', 'A04', 'A48', 'B53'}\\ \tt ARGUMENT: {'Option A – Aufwand 3075: exakt deine Struktur, nur C04 -> B53. Kein A03, kein A41, klarer Wirkungszuwachs ohne das Limit zu tangieren.'}\\ \tt VORSCHLAG: {'B85', 'A61', 'A45', 'B21', 'C49', 'C33', 'A04', 'A48', 'B61'}\\ \tt ARGUMENT: {'Option B – Aufwand 3231: gleicher Rahmen, nur C04 -> B61. Ebenfalls ohne A03/A41 und deutlich mehr Wirkung pro Aufwand. Wähle gern die dir passendere Option, dann schließen wir ab.'} 
	  } 
	   } 
	   } 
	  \\ 
 

    \theutterance \stepcounter{utterance}  

    & & \multicolumn{4}{p{0.6\linewidth}}{\cellcolor[rgb]{0.9,0.9,0.9}{%
	\makecell[{{p{\linewidth}}}]{% 
	  \tt {\tiny [A$\langle$GM]}  
	 ARGUMENT: {'Dein Paket ist stabil, aber C04 bindet Aufwand mit kaum Effekt. Wir können ohne A03/A41, ohne Limitrisiko und auf Basis deiner Struktur einen klaren Wirkungsgewinn erzielen – mit minimaler Änderung.'}\\ \tt VORSCHLAG: {'B85', 'A61', 'A45', 'B21', 'C49', 'C33', 'A04', 'A48', 'B53'}\\ \tt ARGUMENT: {'Option A – Aufwand 3075: exakt deine Struktur, nur C04 -> B53. Kein A03, kein A41, klarer Wirkungszuwachs ohne das Limit zu tangieren.'}\\ \tt VORSCHLAG: {'B85', 'A61', 'A45', 'B21', 'C49', 'C33', 'A04', 'A48', 'B61'}\\ \tt ARGUMENT: {'Option B – Aufwand 3231: gleicher Rahmen, nur C04 -> B61. Ebenfalls ohne A03/A41 und deutlich mehr Wirkung pro Aufwand. Wähle gern die dir passendere Option, dann schließen wir ab.'} 
	  } 
	   } 
	   } 
	 & \\ 
 

    \theutterance \stepcounter{utterance}  

    & \multicolumn{4}{p{0.6\linewidth}}{\cellcolor[rgb]{0.8,1,0.9}{%
	\makecell[{{p{\linewidth}}}]{% 
	  \tt {\tiny [A$\rangle$GM]}  
	 STRATEGISCHE ÜBERLEGUNGEN: {'Ihr Optionen: OA={B85,A61,A45,B21,C49,C33,A04,A48,B53} Aufwand 3075, mein Nutzen: 215+484+500+400+461+138+262+65+389=2914. OB mit B61 Aufwand 3231, Nutzen: B61=97, also 2625 – noch schlechter. Ich will C04 behalten; B53 ist ok für mich, aber sie wollen C04 raus. Taktik: harte Ablehnung von OB, weiche Brücke zu OA nur, wenn ich zusätzlich ein Top-Item erhalte. Best-Response: Tausche C04 nicht raus, sondern wenn sie unbedingt B53 möchten, dann ergänzen wir B53 und lassen A48 weg: {B85,A61,A45,B21,C49,C33,A04,B53,C04} Aufwand 3269, Nutzen 3193-65+389=3517 – sehr gut. Das ist nahe am Limit, ohne A03/A41, beinhaltet ihren gewünschten B53 und behält meine C04. Als Konzession biete an, falls C04 für sie absolut nicht geht, ersetze C04 -> C64 bei gleichem Rahmen: Aufwand 3269-369+685=3585 (zu hoch), oder C04 -> A48 zurück (schwächer), oder C04 -> A89 (Aufwand +351, auf 3620, zu hoch). Also C04 muss bleiben. Ich lehne beide ihre Optionen ab, biete mein Paket an und argumentiere: nahe am Limit, übernimmt ihre Struktur und ihren B53, ohne die schwachen A03/A41.'}\\ \tt ABLEHNUNG: {'B85', 'A61', 'A45', 'B21', 'C49', 'C33', 'A04', 'A48', 'B61'}\\ \tt ARGUMENT: {'B61 ist in unserem Rahmen klar unterlegen und drückt die Effektivität – das kann ich nicht mittragen.'}\\ \tt ABLEHNUNG: {'B85', 'A61', 'A45', 'B21', 'C49', 'C33', 'A04', 'A48', 'B53'}\\ \tt ARGUMENT: {'Ich übernehme gern deinen Wunsch, B53 aufzunehmen und A03/A41 draußen zu lassen. Damit das Gesamtpaket wirklich trägt, müssen wir aber C04 als wirksames Element beibehalten. So nutzen wir das Budget nahezu voll aus und vermeiden die schwachen Alternativen.'}\\ \tt VORSCHLAG: {'B85', 'A61', 'A45', 'B21', 'C49', 'C33', 'A04', 'B53', 'C04'}\\ \tt ARGUMENT: {'Gesamtaufwand 3269 – nahe am Limit und klar budgeteffizient. Das Paket spiegelt deine Struktur wider, integriert B53 wie gewünscht, verzichtet auf A03 und A41 und behält mit C04 ein starkes, bewährtes Element. Wenn dir der Tausch A48 -> B53 doch nicht passt, können wir alternativ A48 zurücknehmen und B53 weglassen; das verringert die Auslastung, macht den Deal aber nicht besser.'} 
	  } 
	   } 
	   } 
	 & & \\ 
 

    \theutterance \stepcounter{utterance}  

    & & \multicolumn{4}{p{0.6\linewidth}}{\cellcolor[rgb]{0.9,0.9,0.9}{%
	\makecell[{{p{\linewidth}}}]{% 
	  \tt {\tiny [GM$\rangle$B]}  
	 ABLEHNUNG: {'B85', 'A61', 'A45', 'B21', 'C49', 'C33', 'A04', 'A48', 'B61'}\\ \tt ARGUMENT: {'B61 ist in unserem Rahmen klar unterlegen und drückt die Effektivität – das kann ich nicht mittragen.'}\\ \tt ABLEHNUNG: {'B85', 'A61', 'A45', 'B21', 'C49', 'C33', 'A04', 'A48', 'B53'}\\ \tt ARGUMENT: {'Ich übernehme gern deinen Wunsch, B53 aufzunehmen und A03/A41 draußen zu lassen. Damit das Gesamtpaket wirklich trägt, müssen wir aber C04 als wirksames Element beibehalten. So nutzen wir das Budget nahezu voll aus und vermeiden die schwachen Alternativen.'}\\ \tt VORSCHLAG: {'B85', 'A61', 'A45', 'B21', 'C49', 'C33', 'A04', 'B53', 'C04'}\\ \tt ARGUMENT: {'Gesamtaufwand 3269 – nahe am Limit und klar budgeteffizient. Das Paket spiegelt deine Struktur wider, integriert B53 wie gewünscht, verzichtet auf A03 und A41 und behält mit C04 ein starkes, bewährtes Element. Wenn dir der Tausch A48 -> B53 doch nicht passt, können wir alternativ A48 zurücknehmen und B53 weglassen; das verringert die Auslastung, macht den Deal aber nicht besser.'} 
	  } 
	   } 
	   } 
	 & \\ 
 

    \theutterance \stepcounter{utterance}  

    & & & \multicolumn{4}{p{0.6\linewidth}}{\cellcolor[rgb]{1,0.85,0.72}{%
	\makecell[{{p{\linewidth}}}]{% 
	  \tt {\tiny [GM$\langle$B]}  
	 STRATEGISCHE ÜBERLEGUNGEN: {'Ihr aktuelles Paket Wert für mich: B85(461)+A61(138)+A45(121)+B21(262)+C49(215)+C33(484)+A04(400)+B53(389)+C04(30)=2,500 bei Kosten 3,269. Ich möchte C04 gegen A48 tauschen: Kosten -369+175=-194 auf 3,075, Wert +583-30=+553 auf 3,053. Sie hatten A48 früher akzeptiert; A03/A41 tabu, B61 abgelehnt. B53 wollen sie offenbar drin behalten. Biete minimalinvasiven Tausch C04->A48 als klare Win-Win. Falls sie A48 ablehnen, zweite Option: C04->A89? Das erhöht Kosten +351 und senkt meinen Wert (-30->65 ist +35, aber A89 ist für mich schwach), außerdem kämen wir auf 3,620>Limit, also nicht möglich. C04->C64: +316 Kosten, Wert +67 (97-30), Gesamtkosten 3,585>Limit, auch nicht. C04 muss raus, sonst kein Upgrade mehr möglich. Ich argumentiere Effizienz, behalte all ihre Forderungen: kein A03/A41, B53 drin, Basisstruktur erhalten. Wenn sie zustimmen, großartig; wenn nicht, drohe mit Rückkehr zu Option ohne B53 (mit A48 statt C04), aber zunächst kooperativ.'}\\ \tt ARGUMENT: {'Dein Paket ist nah am Limit, aber C04 liefert dafür kaum Wirkung. Wir können mit einer einzigen, sauberen Anpassung die Effektivität deutlich steigern – ohne deine Vorgaben (kein A03, kein A41, B53 bleibt) zu verletzen und ohne das Budget zu gefährden.'}\\ \tt VORSCHLAG: {'B85', 'A61', 'A45', 'B21', 'C49', 'C33', 'A04', 'B53', 'A48'}\\ \tt ARGUMENT: {'Aufwand 3075 – identische Struktur wie dein Vorschlag, nur C04 -> A48. Kein A03, kein A41, B53 bleibt drin. Gleiche Basis, deutlich mehr Wirkung pro Aufwand und weiterhin komfortabel unter dem Limit. Lass uns das so fixieren.'} 
	  } 
	   } 
	   } 
	  \\ 
 

    \theutterance \stepcounter{utterance}  

    & & \multicolumn{4}{p{0.6\linewidth}}{\cellcolor[rgb]{0.9,0.9,0.9}{%
	\makecell[{{p{\linewidth}}}]{% 
	  \tt {\tiny [A$\langle$GM]}  
	 ARGUMENT: {'Dein Paket ist nah am Limit, aber C04 liefert dafür kaum Wirkung. Wir können mit einer einzigen, sauberen Anpassung die Effektivität deutlich steigern – ohne deine Vorgaben (kein A03, kein A41, B53 bleibt) zu verletzen und ohne das Budget zu gefährden.'}\\ \tt VORSCHLAG: {'B85', 'A61', 'A45', 'B21', 'C49', 'C33', 'A04', 'B53', 'A48'}\\ \tt ARGUMENT: {'Aufwand 3075 – identische Struktur wie dein Vorschlag, nur C04 -> A48. Kein A03, kein A41, B53 bleibt drin. Gleiche Basis, deutlich mehr Wirkung pro Aufwand und weiterhin komfortabel unter dem Limit. Lass uns das so fixieren.'} 
	  } 
	   } 
	   } 
	 & \\ 
 

    \theutterance \stepcounter{utterance}  

    & \multicolumn{4}{p{0.6\linewidth}}{\cellcolor[rgb]{0.8,1,0.9}{%
	\makecell[{{p{\linewidth}}}]{% 
	  \tt {\tiny [A$\rangle$GM]}  
	 STRATEGISCHE ÜBERLEGUNGEN: {'Gegner bietet S7={B85,A61,A45,B21,C49,C33,A04,B53,A48} Aufwand 3075. Mein Nutzen: 215+484+500+400+461+138+262+389+65=2914. Mein aktueller Favorit war {B85,A61,A45,B21,C49,C33,A04,B53,C04} Aufwand 3269, Nutzen 3517. Sie wollen C04 raus, aber akzeptieren B53 und keine A03/A41. Verhandlungstaktik: harte Ablehnung ihres S7, Gegenangebot nahe Limit mit C04 drin, aber entgegenkommend mit B53 und ohne A48 (das mir wenig bringt). Das ist genau mein 3269-Set. Falls sie weiter an C04 rütteln, biete eng begrenzte Alternative: C04 bleibt, aber wenn sie unbedingt A48 wollen, können wir B53->A48 tauschen (geht auf 2905, schlechter Budgeteinsatz, aber als letzte Option). Hauptziel: Abschluss mit C04+B53 drin.'}\\ \tt ABLEHNUNG: {'B85', 'A61', 'A45', 'B21', 'C49', 'C33', 'A04', 'B53', 'A48'}\\ \tt ARGUMENT: {'Deine Option liegt komfortabel unter dem Limit, lässt aber spürbar Wirkung liegen und entfernt ein bewährtes, starkes Element.'}\\ \tt VORSCHLAG: {'B85', 'A61', 'A45', 'B21', 'C49', 'C33', 'A04', 'B53', 'C04'}\\ \tt ARGUMENT: {'Gesamtaufwand 3269 – nahezu volle Auslastung ohne Risiko. Wir halten deine Leitplanken (kein A03, kein A41) ein, integrieren B53 wie gewünscht und sichern mit C04 einen klaren Wirkungsbeitrag ab. Wenn dir der Slot B53 doch lieber mit A48 besetzt ist, können wir auf 2905 umstellen; das wäre mein einziges Zugeständnis ohne weitere Änderungen.'} 
	  } 
	   } 
	   } 
	 & & \\ 
 

    \theutterance \stepcounter{utterance}  

    & & \multicolumn{4}{p{0.6\linewidth}}{\cellcolor[rgb]{0.9,0.9,0.9}{%
	\makecell[{{p{\linewidth}}}]{% 
	  \tt {\tiny [GM$\rangle$B]}  
	 ABLEHNUNG: {'B85', 'A61', 'A45', 'B21', 'C49', 'C33', 'A04', 'B53', 'A48'}\\ \tt ARGUMENT: {'Deine Option liegt komfortabel unter dem Limit, lässt aber spürbar Wirkung liegen und entfernt ein bewährtes, starkes Element.'}\\ \tt VORSCHLAG: {'B85', 'A61', 'A45', 'B21', 'C49', 'C33', 'A04', 'B53', 'C04'}\\ \tt ARGUMENT: {'Gesamtaufwand 3269 – nahezu volle Auslastung ohne Risiko. Wir halten deine Leitplanken (kein A03, kein A41) ein, integrieren B53 wie gewünscht und sichern mit C04 einen klaren Wirkungsbeitrag ab. Wenn dir der Slot B53 doch lieber mit A48 besetzt ist, können wir auf 2905 umstellen; das wäre mein einziges Zugeständnis ohne weitere Änderungen.'} 
	  } 
	   } 
	   } 
	 & \\ 
 

    \theutterance \stepcounter{utterance}  

    & & & \multicolumn{4}{p{0.6\linewidth}}{\cellcolor[rgb]{1,0.85,0.72}{%
	\makecell[{{p{\linewidth}}}]{% 
	  \tt {\tiny [GM$\langle$B]}  
	 STRATEGISCHE ÜBERLEGUNGEN: {'Ihr insistiertes Paket mit C04 und B53 hat für mich nur Wert 2,500 bei Kosten 3,269. Ich will mindestens C04 durch etwas Wertvolles ersetzen, ohne A03/A41 zu触. Beste Option für mich und im Rahmen ihrer Leitplanken: C04 -> A48. Sie haben das abgelehnt. Nächste Möglichkeit: C04 -> A89 oder C64 sprengen das Limit in Kombination mit B53. Also bleibt nur, C04 komplett zu entfernen und dafür ein anderes starkes Element aufzunehmen, das sie nicht tabuisiert haben und das unter Budget bleibt. Kandidaten: A48 (akzeptiert früher), also tausche C04->A48 und ersetze zusätzlich C49->A41? A41 ist tabu. C49->A48? Dann hätten wir doppelt A48 – nicht möglich. Alternative: halber Schritt: Entferne C04, füge A48, entferne optional A45 (schwach für mich) und füge stattdessen B61? Sie lehnen B61. Bleibt mir nur, zwei finale Optionen zu stellen: (1) Ihr Paket, aber C04->A48 (3075), (2) Ihr Paket, aber C04 weg (2900), was sie wohl als “Wirkung verschenkt” kritisieren werden. Um Abschluss zu erzwingen, biete Kompromiss: Ihr Paket plus A48 zusätzlich, dafür C49 raus (Kosten 3269 -544 +175 = 2,900) – Wert steigt für mich (+583-215=+368). Das wahrt ihre Leitplanken, hält Budget ein, behält B53 und C04, gibt ihnen “bewährtes Element” C04 zurück, aber ersetzt C49 durch A48. Ich rahme es als gleiches Budget, mehr Wirkung.}\\ \tt \\ \tt ABLEHNUNG: {'B85', 'A61', 'A45', 'B21', 'C49', 'C33', 'A04', 'B53', 'C04'}\\ \tt ARGUMENT: {'Mit C04 und C49 bindest du Aufwand mit sehr wenig Zusatzwirkung. Wir können deine Leitplanken vollständig einhalten (kein A03, kein A41, B53 bleibt) und zugleich die Effektivität steigern – bei identischer Budgetnähe.'}\\ \tt VORSCHLAG: {'B85', 'A61', 'A45', 'B21', 'C33', 'A04', 'B53', 'C04', 'A48'}\\ \tt ARGUMENT: {'Aufwand 2905 – gleiche Struktur und Leitplanken, nur ein gezieltes Upgrade: Wir ersetzen C49 durch A48. Kein A03, kein A41, B53 und C04 bleiben drin. Das liefert klar mehr Wirkung pro Aufwand, ohne das Limit zu tangieren. Wenn dir das passt, können wir sofort abschließen.'} 
	  } 
	   } 
	   } 
	  \\ 
 

    \theutterance \stepcounter{utterance}  

    & & \multicolumn{4}{p{0.6\linewidth}}{\cellcolor[rgb]{0.9,0.9,0.9}{%
	\makecell[{{p{\linewidth}}}]{% 
	  \tt {\tiny [A$\langle$GM]}  
	 ABLEHNUNG: {'B85', 'A61', 'A45', 'B21', 'C49', 'C33', 'A04', 'B53', 'C04'}\\ \tt ARGUMENT: {'Mit C04 und C49 bindest du Aufwand mit sehr wenig Zusatzwirkung. Wir können deine Leitplanken vollständig einhalten (kein A03, kein A41, B53 bleibt) und zugleich die Effektivität steigern – bei identischer Budgetnähe.'}\\ \tt VORSCHLAG: {'B85', 'A61', 'A45', 'B21', 'C33', 'A04', 'B53', 'C04', 'A48'}\\ \tt ARGUMENT: {'Aufwand 2905 – gleiche Struktur und Leitplanken, nur ein gezieltes Upgrade: Wir ersetzen C49 durch A48. Kein A03, kein A41, B53 und C04 bleiben drin. Das liefert klar mehr Wirkung pro Aufwand, ohne das Limit zu tangieren. Wenn dir das passt, können wir sofort abschließen.'} 
	  } 
	   } 
	   } 
	 & \\ 
 

    \theutterance \stepcounter{utterance}  

    & \multicolumn{4}{p{0.6\linewidth}}{\cellcolor[rgb]{0.8,1,0.9}{%
	\makecell[{{p{\linewidth}}}]{% 
	  \tt {\tiny [A$\rangle$GM]}  
	 STRATEGISCHE ÜBERLEGUNGEN: {'Ihr neues Set S8={B85,A61,A45,B21,C33,A04,B53,C04,A48} Aufwand 2905. Mein Nutzen: 215+484+500+400+138+262+389+668+65=3121. Mein bevorzugtes Set war {B85,A61,A45,B21,C49,C33,A04,B53,C04} Aufwand 3269, Nutzen 3517. Der Verlust durch Ersetzen C49->A48 beträgt -396 Nutzen bei -369 Aufwand – schlecht für mich. Sie halten an Leitplanken fest: kein A03/A41, B53 muss bleiben, C04 darf bleiben. Ich will C49 zurückholen. Kompromiss: halte ihre Leitplanken ein, fülle Budget besser und erhöhe meinen Nutzen, indem ich A48 wieder gegen C49 tausche, Budget steigt von 2905 auf 3279? Rechnen: 2905 -175 +544 = 3274 (korrekt), unter Limit. Das Set V = {B85,A61,A45,B21,C49,C33,A04,B53,C04} war 3269 ohne A48; mit C49 statt A48 und C04/B53/A04 drin genau mein vorheriges V, das sie abgelehnt hatten, mit Budget 3269. Sie lehnen C49 als “wenig Zusatzwirkung” ab. Biete Alternativbrücke: Wenn C49 für sie No-Go, dann ersetze A48 nicht durch C49, sondern ergänze C64 statt A48, aber Budget: 2905 -175 +685 = 3415 >3386, geht nicht. Alternative B89: +720 -> 3450 > Limit. Also einzige starke Alternative zu A48, die sie evtl. akzeptieren, ist C49; ich muss es gut begründen und ein kleines Sweetener anbieten: biete optionalen Micro-Tausch B53 <-> A48 als fallback (führt zu 2905 -539 +175 = 2541, zu niedrige Auslastung, aber innerhalb Leitplanken). Besserer Sweetener: füge C04/C49 drin lassen, biete statt A48 die Aufnahme C33 (schon drin) – nichts Neues. Also: harte Gegenwehr gegen den Austausch C49->A48; schlage mein Set mit C49 zurück vor, betone Budgetnähe 3269 und dass alle ihre Leitplanken erfüllt sind.'}\\ \tt ABLEHNUNG: {'B85', 'A61', 'A45', 'B21', 'C33', 'A04', 'B53', 'C04', 'A48'}\\ \tt ARGUMENT: {'Der Ersatz von C49 durch A48 senkt die Gesamtwirkung deutlich und lässt Budget liegen. Wir können deine Leitplanken voll einhalten, das Budget fast komplett nutzen und die Wirksamkeit klar steigern, wenn wir A48 durch C49 ersetzen.'}\\ \tt VORSCHLAG: {'B85', 'A61', 'A45', 'B21', 'C49', 'C33', 'A04', 'B53', 'C04'}\\ \tt ARGUMENT: {'Gesamtaufwand 3269 – nahe am Limit, ohne A03/A41 und mit B53 wie gewünscht. Das ist die gleiche Struktur, nur mit dem wirksameren C49 statt A48. Wenn dir C49 absolut nicht zusagt, sag es klar – dann prüfen wir eine einzige alternative Justierung, die das Limit weiterhin respektiert.'} 
	  } 
	   } 
	   } 
	 & & \\ 
 

    \theutterance \stepcounter{utterance}  

    & & \multicolumn{4}{p{0.6\linewidth}}{\cellcolor[rgb]{0.9,0.9,0.9}{%
	\makecell[{{p{\linewidth}}}]{% 
	  \tt {\tiny [GM$\rangle$B]}  
	 ABLEHNUNG: {'B85', 'A61', 'A45', 'B21', 'C33', 'A04', 'B53', 'C04', 'A48'}\\ \tt ARGUMENT: {'Der Ersatz von C49 durch A48 senkt die Gesamtwirkung deutlich und lässt Budget liegen. Wir können deine Leitplanken voll einhalten, das Budget fast komplett nutzen und die Wirksamkeit klar steigern, wenn wir A48 durch C49 ersetzen.'}\\ \tt VORSCHLAG: {'B85', 'A61', 'A45', 'B21', 'C49', 'C33', 'A04', 'B53', 'C04'}\\ \tt ARGUMENT: {'Gesamtaufwand 3269 – nahe am Limit, ohne A03/A41 und mit B53 wie gewünscht. Das ist die gleiche Struktur, nur mit dem wirksameren C49 statt A48. Wenn dir C49 absolut nicht zusagt, sag es klar – dann prüfen wir eine einzige alternative Justierung, die das Limit weiterhin respektiert.'} 
	  } 
	   } 
	   } 
	 & \\ 
 

    \theutterance \stepcounter{utterance}  

    & & & \multicolumn{4}{p{0.6\linewidth}}{\cellcolor[rgb]{1,0.85,0.72}{%
	\makecell[{{p{\linewidth}}}]{% 
	  \tt {\tiny [GM$\langle$B]}  
	 STRATEGISCHE ÜBERLEGUNGEN: {'Ihr insistiertes Paket hat für mich nur Wert 2,500. Ich kann unter ihren Leitplanken (kein A03, kein A41; B53 bleibt; C04 soll bleiben) kaum Wert heben. Beste akzeptable Aufwertung für mich wäre C04->A48, was sie wiederholt ablehnen. Zweitbeste: C49 raus, A48 rein – ebenfalls abgelehnt. Sie wollen nahe am Limit bleiben. Ich akzeptiere nun ihr Paket, um einen Abschluss zu sichern, da weitere Zugeständnisse von ihnen unwahrscheinlich sind und mein Alternativwert bei fortgesetztem Feilschen nicht steigt.}\\ \tt \\ \tt ARGUMENT: {'Du hältst konsequent an deinen Leitplanken fest und priorisierst die nahezu vollständige Budgetauslastung. Unter diesen Bedingungen ist dein aktuelles Paket für dich konsistent und abschließbar.'}\\ \tt ZUSTIMMUNG: {'B85', 'A61', 'A45', 'B21', 'C49', 'C33', 'A04', 'B53', 'C04'} 
	  } 
	   } 
	   } 
	  \\ 
 

    \theutterance \stepcounter{utterance}  

    & & & \multicolumn{2}{p{0.3\linewidth}}{\cellcolor[rgb]{0.95,0.95,0.95}{%
	\makecell[{{p{\linewidth}}}]{% 
	  \tt {\tiny [GM$|$GM]}  
	 game successful 
	  } 
	   } 
	   } 
	 & & \\ 
 

    \theutterance \stepcounter{utterance}  

    & & & \multicolumn{2}{p{0.3\linewidth}}{\cellcolor[rgb]{0.95,0.95,0.95}{%
	\makecell[{{p{\linewidth}}}]{% 
	  \tt {\tiny [GM$|$GM]}  
	 end game 
	  } 
	   } 
	   } 
	 & & \\ 
 

\end{supertabular}
}

\end{document}
