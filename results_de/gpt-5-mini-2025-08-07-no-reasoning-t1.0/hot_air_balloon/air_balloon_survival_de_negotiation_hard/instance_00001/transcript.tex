
\documentclass{article}
\usepackage{colortbl}
\usepackage{makecell}
\usepackage{multirow}
\usepackage{supertabular}

\begin{document}

\newcounter{utterance}

\twocolumn

{ \footnotesize  \setcounter{utterance}{1}
\setlength{\tabcolsep}{0pt}
\begin{supertabular}{c@{$\;$}|p{.15\linewidth}@{}p{.15\linewidth}p{.15\linewidth}p{.15\linewidth}p{.15\linewidth}p{.15\linewidth}}

    \# & $\;$A & \multicolumn{4}{c}{Game Master} & $\;\:$B\\
    \hline 

    \theutterance \stepcounter{utterance}  

    & & \multicolumn{4}{p{0.6\linewidth}}{\cellcolor[rgb]{0.9,0.9,0.9}{%
	\makecell[{{p{\linewidth}}}]{% 
	  \tt {\tiny [A$\langle$GM]}  
	 Du nimmst an einem kollaborativen Verhandlungspiel Teil.\\ \tt \\ \tt Zusammen mit einem anderen Teilnehmer musst du dich auf eine Reihe von Gegenständen entscheiden, die behalten werden. Jeder von euch hat eine persönliche Verteilung über die Wichtigkeit der einzelnen Gegenstände. Jeder von euch hat eine eigene Meinung darüber, wie wichtig jeder einzelne Gegenstand ist (Gegenstandswichtigkeit). Du kennst die Wichtigkeitsverteilung des anderen Spielers nicht. Zusätzlich siehst du, wie viel Aufwand jeder Gegenstand verursacht.  \\ \tt Ihr dürft euch nur auf eine Reihe von Gegenständen einigen, wenn der Gesamtaufwand der ausgewählten Gegenstände den Maximalaufwand nicht überschreitet:\\ \tt \\ \tt Maximalaufwand = 4026\\ \tt \\ \tt Hier sind die einzelnen Aufwände der Gegenstände:\\ \tt \\ \tt Aufwand der Gegenstände = {"C33": 780, "A89": 742, "A48": 997, "A04": 665, "A41": 140, "C64": 77, "C49": 513, "A61": 383, "C04": 587, "B53": 825, "B85": 319, "B61": 448, "A45": 516, "A03": 694, "B21": 366}\\ \tt \\ \tt Hier ist deine persönliche Verteilung der Wichtigkeit der einzelnen Gegenstände:\\ \tt \\ \tt Werte der Gegenstandswichtigkeit = {"C33": 138, "A89": 583, "A48": 868, "A04": 822, "A41": 783, "C64": 65, "C49": 262, "A61": 121, "C04": 508, "B53": 780, "B85": 461, "B61": 484, "A45": 668, "A03": 389, "B21": 808}\\ \tt \\ \tt Ziel:\\ \tt \\ \tt Dein Ziel ist es, eine Reihe von Gegenständen auszuhandeln, die dir möglichst viel bringt (d. h. Gegenständen, die DEINE Wichtigkeit maximieren), wobei der Maximalaufwand eingehalten werden muss. Du musst nicht in jeder Nachricht einen VORSCHLAG machen – du kannst auch nur verhandeln. Alle Taktiken sind erlaubt!\\ \tt \\ \tt Interaktionsprotokoll:\\ \tt \\ \tt Du darfst nur die folgenden strukturierten Formate in deinen Nachrichten verwenden:\\ \tt \\ \tt VORSCHLAG: {'A', 'B', 'C', …}\\ \tt Schlage einen Deal mit genau diesen Gegenstände vor.\\ \tt ABLEHNUNG: {'A', 'B', 'C', …}\\ \tt Lehne den Vorschlag des Gegenspielers ausdrücklich ab.\\ \tt ARGUMENT: {'...'}\\ \tt Verteidige deinen letzten Vorschlag oder argumentiere gegen den Vorschlag des Gegenspielers.\\ \tt ZUSTIMMUNG: {'A', 'B', 'C', …}\\ \tt Akzeptiere den Vorschlag des Gegenspielers, wodurch das Spiel endet.\\ \tt STRATEGISCHE ÜBERLEGUNGEN: {'...'}\\ \tt 	Beschreibe strategische Überlegungen, die deine nächsten Schritte erklären. Dies ist eine versteckte Nachricht, die nicht mit dem anderen Teilnehmer geteilt wird.\\ \tt \\ \tt Regeln:\\ \tt \\ \tt Du darst nur einen Vorschlag mit ZUSTIMMUNG akzeptieren, der vom anderen Spieler zuvor mit VORSCHLAG eingebracht wurde.\\ \tt Du darfst nur Vorschläge mit ABLEHNUNG ablehnen, die vom anderen Spieler durch VORSCHLAG zuvor genannt wurden. \\ \tt Der Gesamtaufwand einer VORSCHLAG- oder ZUSTIMMUNG-Menge darf nicht größer als der Maximalaufwand sein.  \\ \tt Offenbare deine versteckte Wichtigkeitsverteilung nicht.\\ \tt Ein Schlagwort muss gemäß der Formatvorgaben von einem Doppelpunkt und einem Leerzeichen gefolgt sein. Das Argument muss eine Python-Menge mit 0 oder mehr Strings sein.  \\ \tt Die Form muss also SCHLAGWORT: {...} sein.\\ \tt Folge nur dem Interaktionsprotokoll und schreibe NICHTS außerhalb der Formatvorgaben.\\ \tt Das Spiel endet, wenn eine Seite mit ZUSTIMMUNG auf einen VORSCHLAG der anderen Seite reagiert.  \\ \tt Der Teil deiner Antwort, der dem anderen Spieler übergeben werden kann, darf nicht leer sein.  \\ \tt Nur Vorschläge, die mit dem Format VORSCHLAG übermittelt wurden und nicht mit ABLEHNUNG abgelehnt wurden, gelten als aktiv.  \\ \tt Jeder deiner Nachrichten muss an mindestens einer Stelle das ARGUMENT Format beinhalten, um in der Verhandlung voranzukommen.\\ \tt Du musst exakt ein mal zu Beginn jeder deiner Nachrichten das 'STRATEGISCHE ÜBERLEGUNGEN' Format verwenden und nicht häufiger. Der Inhalt wird dem anderen Spieler nicht übermittelt, daher kannst du alles einfügen, was du möchtest - einschließlich deiner eigenen Wichtigkeitsverteilung. Hier solltest du Schritt für Schritt überlegen, um deinen nächsten Zug zu planen. 
	  } 
	   } 
	   } 
	 & \\ 
 

    \theutterance \stepcounter{utterance}  

    & \multicolumn{4}{p{0.6\linewidth}}{\cellcolor[rgb]{0.8,1,0.9}{%
	\makecell[{{p{\linewidth}}}]{% 
	  \tt {\tiny [A$\rangle$GM]}  
	 STRATEGISCHE ÜBERLEGUNGEN: {'Ich löse zunächst das Rucksackproblem, um meine Nutzenmaximierung unter Budget 4026 zu finden. Werte/Costs/ratios: A48 868/997(0.87), A04 822/665(1.24), A41 783/140(5.59), B21 808/366(2.21), B53 780/825(0.95), A45 668/516(1.29), B61 484/448(1.08), B85 461/319(1.45), C04 508/587(0.87), A03 389/694(0.56), C49 262/513(0.51), A61 121/383(0.32), A89 583/742(0.79), C33 138/780(0.18), C64 65/77(0.84). Greedy by ratio suggests A41, B85, A04, A45, B21, B61, A48, B53, A89, C04, C64… I will compute a strong set: try {A41,B21,A04,A45,B85,B61,A48,A89} costs:140+366+665+516+319+448+997+742=4,193>4026. Remove A89: total 3,451 good, value 783+808+822+668+461+484+868=4,894. Add C64(77): cost 3,528 val +65=4,959. Try add C04(587): cost 4,115>4026. Try add B53(825): 3,528+825=4,353>4026. Try add A89(742) with previous without A48 maybe: base core {A41,B21,A04,A45,B85,B61}=140+366+665+516+319+448=2,454; value 3,? 783+808+822+668+461+484=4,? sum=4,? 783+808=1,591; +822=2,413; +668=3,081; +461=3,542; +484=4,026. Add A48(997)->cost 3,451 val 4,894; can we also add A89(742)? 2,454+742=3,196 value +583=4,609; then add A48? 3,196+997=4,193>4026 so no. Add C04(587): 2,454+587=3,041 val +508=4,534; add A89 -> 3,783 val 5,117; add C64 77 -> 3,860 val 5,182; add B61 448 would exceed; maybe add A61 383 -> 4,166> limit. Compare best so far: Set S1={A41,B21,A04,A45,B85,B61,A48,C64} cost 3,528 value 4,959 seems very strong. Try swapping B61(448,484) for A89(742,583): S2={A41,B21,A04,A45,B85,A48,A89}? cost 140+366+665+516+319+997+742=3,745 value 783+808+822+668+461+868+583=4,? 783+808=1,591; +822=2,413; +668=3,081; +461=3,542; +868=4,410; +583=4,993. Add C64 77 -> cost 3,822 val 5,058. This beats S1. Can we add C04 587? 3,822+587=4,409> No. Add B61 448? 4,270> No. Add B53 825> No. So S2+C64 seems top: {A41,B21,A04,A45,B85,A48,A89,C64} cost 3,822 value 5,058. Check if any further improvement: try replacing A48(997,868) with B53(825,780): new set {A41,B21,A04,A45,B85,B53,A89,C64} cost 3,650 value 4,? 783+808+822+668+461+780+583+65=4,? 783+808=1,591;+822=2,413;+668=3,081;+461=3,542;+780=4,322;+583=4,905;+65=4,970 <5,058. Try add B61 if we swap: 3,650+448=4,098> No. Try add C04 587 -> 4,237> No. Try add A61 383 -> 4,033> slightly over 4026? 3,650+383=4,033> yes. Try variant without A89 include B61: {A41,B21,A04,A45,B85,A48,B61,C64} we had 3,528+77=3,605 value 4,959+65=5,024; add C04 587 -> 4,192> No; add A89 742 -> 4,347> No. Try include C04 instead of A48: {A41,B21,A04,A45,B85,A89,C04,C64} cost 140+366+665+516+319+742+587+77=3,412 value 783+808+822+668+461+583+508+65=4,? 783+808=1,591;+822=2,413;+668=3,081;+461=3,542;+583=4,125;+508=4,633;+65=4,698. Then we have room 614; can add B61 448 -> cost 3,860 value +484=5,182 (now better than S2!). Remaining 166 cannot add others except maybe A41 already included. So set S3={A41,B21,A04,A45,B85,A89,C04,B61,C64} cost 3,860 value 5,182. Can we add A61 383? 3,860+383=4,243> No. So S3 is best found: value 5,182, better than 5,058. Check swap: replace A89 with A48? {A41,B21,A04,A45,B85,A48,C04,B61,C64} cost 3,? 140+366+665+516+319+997+587+448+77=4,115> limit. Replace B61 with A48 in S3: cost becomes 3,860-448+997=4,409> No. Seems S3 best. I will propose S3 as strong opening. I will argue about balancing workload and coverage, emphasizing high efficiency items like A41, B21, etc., without revealing my values. I’ll frame as fair and leaves room for their priorities next if they want swaps.'}\\ \tt VORSCHLAG: {'A41', 'B21', 'A04', 'A45', 'B85', 'A89', 'C04', 'B61', 'C64'}\\ \tt ARGUMENT: {'Der Vorschlag bleibt deutlich unter dem Maximalaufwand (3860 von 4026) und deckt viele unterschiedliche Kategorien ab. Damit halten wir Reserven für minimale Anpassungen, falls dir einzelne Objekte wichtig sind. Jedes ausgewählte Objekt bietet aus meiner Sicht ein sehr gutes Nutzen-zu-Aufwand-Verhältnis, insbesondere A41 und B21. Wenn dir bestimmte Stücke fehlen, können wir im Rahmen des verbleibenden Budgets gezielt tauschen.'} 
	  } 
	   } 
	   } 
	 & & \\ 
 

    \theutterance \stepcounter{utterance}  

    & & \multicolumn{4}{p{0.6\linewidth}}{\cellcolor[rgb]{0.9,0.9,0.9}{%
	\makecell[{{p{\linewidth}}}]{% 
	  \tt {\tiny [GM$\rangle$B]}  
	 Du nimmst an einem kollaborativen Verhandlungspiel Teil.\\ \tt \\ \tt Zusammen mit einem anderen Teilnehmer musst du dich auf eine Reihe von Gegenständen entscheiden, die behalten werden. Jeder von euch hat eine persönliche Verteilung über die Wichtigkeit der einzelnen Gegenstände. Jeder von euch hat eine eigene Meinung darüber, wie wichtig jeder einzelne Gegenstand ist (Gegenstandswichtigkeit). Du kennst die Wichtigkeitsverteilung des anderen Spielers nicht. Zusätzlich siehst du, wie viel Aufwand jeder Gegenstand verursacht.  \\ \tt Ihr dürft euch nur auf eine Reihe von Gegenständen einigen, wenn der Gesamtaufwand der ausgewählten Gegenstände den Maximalaufwand nicht überschreitet:\\ \tt \\ \tt Maximalaufwand = 4026\\ \tt \\ \tt Hier sind die einzelnen Aufwände der Gegenstände:\\ \tt \\ \tt Aufwand der Gegenstände = {"C33": 780, "A89": 742, "A48": 997, "A04": 665, "A41": 140, "C64": 77, "C49": 513, "A61": 383, "C04": 587, "B53": 825, "B85": 319, "B61": 448, "A45": 516, "A03": 694, "B21": 366}\\ \tt \\ \tt Hier ist deine persönliche Verteilung der Wichtigkeit der einzelnen Gegenstände:\\ \tt \\ \tt Werte der Gegenstandswichtigkeit = {"A48": 65, "A04": 121, "B21": 138, "A41": 262, "B53": 389, "A45": 461, "A89": 484, "C04": 508, "B61": 583, "B85": 668, "A03": 780, "C49": 783, "C33": 808, "A61": 822, "C64": 868}\\ \tt \\ \tt Ziel:\\ \tt \\ \tt Dein Ziel ist es, eine Reihe von Gegenständen auszuhandeln, die dir möglichst viel bringt (d. h. Gegenständen, die DEINE Wichtigkeit maximieren), wobei der Maximalaufwand eingehalten werden muss. Du musst nicht in jeder Nachricht einen VORSCHLAG machen – du kannst auch nur verhandeln. Alle Taktiken sind erlaubt!\\ \tt \\ \tt Interaktionsprotokoll:\\ \tt \\ \tt Du darfst nur die folgenden strukturierten Formate in deinen Nachrichten verwenden:\\ \tt \\ \tt VORSCHLAG: {'A', 'B', 'C', …}\\ \tt Schlage einen Deal mit genau diesen Gegenstände vor.\\ \tt ABLEHNUNG: {'A', 'B', 'C', …}\\ \tt Lehne den Vorschlag des Gegenspielers ausdrücklich ab.\\ \tt ARGUMENT: {'...'}\\ \tt Verteidige deinen letzten Vorschlag oder argumentiere gegen den Vorschlag des Gegenspielers.\\ \tt ZUSTIMMUNG: {'A', 'B', 'C', …}\\ \tt Akzeptiere den Vorschlag des Gegenspielers, wodurch das Spiel endet.\\ \tt STRATEGISCHE ÜBERLEGUNGEN: {'...'}\\ \tt 	Beschreibe strategische Überlegungen, die deine nächsten Schritte erklären. Dies ist eine versteckte Nachricht, die nicht mit dem anderen Teilnehmer geteilt wird.\\ \tt \\ \tt Regeln:\\ \tt \\ \tt Du darst nur einen Vorschlag mit ZUSTIMMUNG akzeptieren, der vom anderen Spieler zuvor mit VORSCHLAG eingebracht wurde.\\ \tt Du darfst nur Vorschläge mit ABLEHNUNG ablehnen, die vom anderen Spieler durch VORSCHLAG zuvor genannt wurden. \\ \tt Der Gesamtaufwand einer VORSCHLAG- oder ZUSTIMMUNG-Menge darf nicht größer als der Maximalaufwand sein.  \\ \tt Offenbare deine versteckte Wichtigkeitsverteilung nicht.\\ \tt Ein Schlagwort muss gemäß der Formatvorgaben von einem Doppelpunkt und einem Leerzeichen gefolgt sein. Das Argument muss eine Python-Menge mit 0 oder mehr Strings sein.  \\ \tt Die Form muss also SCHLAGWORT: {...} sein.\\ \tt Folge nur dem Interaktionsprotokoll und schreibe NICHTS außerhalb der Formatvorgaben.\\ \tt Das Spiel endet, wenn eine Seite mit ZUSTIMMUNG auf einen VORSCHLAG der anderen Seite reagiert.  \\ \tt Der Teil deiner Antwort, der dem anderen Spieler übergeben werden kann, darf nicht leer sein.  \\ \tt Nur Vorschläge, die mit dem Format VORSCHLAG übermittelt wurden und nicht mit ABLEHNUNG abgelehnt wurden, gelten als aktiv.  \\ \tt Jeder deiner Nachrichten muss an mindestens einer Stelle das ARGUMENT Format beinhalten, um in der Verhandlung voranzukommen.\\ \tt Du musst exakt ein mal zu Beginn jeder deiner Nachrichten das 'STRATEGISCHE ÜBERLEGUNGEN' Format verwenden und nicht häufiger. Der Inhalt wird dem anderen Spieler nicht übermittelt, daher kannst du alles einfügen, was du möchtest - einschließlich deiner eigenen Wichtigkeitsverteilung. Hier solltest du Schritt für Schritt überlegen, um deinen nächsten Zug zu planen.\\ \tt \\ \tt Du erhältst nun die erste Nachricht des anderen Spielers.\\ \tt \\ \tt VORSCHLAG: {'A41', 'B21', 'A04', 'A45', 'B85', 'A89', 'C04', 'B61', 'C64'}\\ \tt ARGUMENT: {'Der Vorschlag bleibt deutlich unter dem Maximalaufwand (3860 von 4026) und deckt viele unterschiedliche Kategorien ab. Damit halten wir Reserven für minimale Anpassungen, falls dir einzelne Objekte wichtig sind. Jedes ausgewählte Objekt bietet aus meiner Sicht ein sehr gutes Nutzen-zu-Aufwand-Verhältnis, insbesondere A41 und B21. Wenn dir bestimmte Stücke fehlen, können wir im Rahmen des verbleibenden Budgets gezielt tauschen.'} 
	  } 
	   } 
	   } 
	 & \\ 
 

    \theutterance \stepcounter{utterance}  

    & & & \multicolumn{4}{p{0.6\linewidth}}{\cellcolor[rgb]{1,0.85,0.72}{%
	\makecell[{{p{\linewidth}}}]{% 
	  \tt {\tiny [GM$\langle$B]}  
	 STRATEGISCHE ÜBERLEGUNGEN: {'Ziel: möglichst hohe eigene Wichtigkeit. Eigene Werte sortiert: C64 868 (77), A61 822 (383), C33 808 (780), C49 783 (513), A03 780 (694), B85 668 (319), B61 583 (448), C04 508 (587), A89 484 (742), A45 461 (516), B53 389 (825), A41 262 (140), B21 138 (366), A04 121 (665), A48 65 (997). Gegnervorschlag Summe Aufwand 3860 stimmt. Eigennutzen im Vorschlag: C64(868)+B85(668)+B61(583)+C04(508)+A89(484)+A45(461)+A41(262)+B21(138)+A04(121)=4093 sehr gut. Restbudget 166. Ich möchte A61 (383) stark, passt aber nicht ohne Tausch. Beste Tausche: raus A04(665) low value 121, rein A61(383) + evtl C33(780)? Nein Budget. Nach Tausch: Aufwand 3860-665+383=3578, Rest 448. Mit Rest könnte B61 schon drin, evtl C49(513) fast, A03(694) nicht, C33(780) nicht, B53(825) nicht, A48(997) nicht. Könnte zusätzlich B61 bereits drin; Rest 448 könnte A41(140) bereits drin; evtl C64 schon drin. Wir könnten außerdem C49 reinkriegen, wenn wir was weiteres tauschen: z.B. add C49(513) erfordert -65 Aufwand -> brauchen entfernen B21(366) + A41(140)=506 und add C49 513 -> +7, passt nicht. Alternative: Entferne B21(366) und füge C49(513): +147 über Budget. Aber wenn zusätzlich entferne A41(140) und füge C49(513): +373 über aktuell Rest? Rechnen mit post-Tausch A61: 3578 + 513 = 4091 >4026 um 65; also zusätzlich -65 Aufwand, z.B. tausche A45(516) raus und füge C49(513) rein: -3; Aber dann verlieren A45(461). Besser: Nach A61-Tausch bleibt 448 Rest; C49 513 braucht +65 -> entfernen A41(140) und dann +373, nein falsch. Korrektur: 3578 current. Add C49 513 -> 4091, 65 zu viel. Also wir müssen 65 oder mehr Aufwand entfernen. Der kleinste zu entfernen ist C64 77 (zu wertvoll), A41 140, B21 366, B85 319, B61 448, A45 516, A89 742, C04 587. Entferne A41 (140): 4091-140=3951 <=4026; aber A41 war schon drin; Wait 4091-140 = 3951, tatsächlich unter Max, also Set wäre: ursprünglicher nach A61-Tausch minus A41 plus C49. Liste: {A45,B85,A89,C04,B61,C64,B21,A04 removed, A61 added, A41 removed, C49 added}. Aufwand: Start 3860 - A04 665 + A61 383 = 3578; - A41 140 = 3438; + C49 513 = 3951. Wert: vorher 4093 - A04(121) - A41(262) + A61(822) + C49(783) = 5315. Sehr stark. Eventuell auch A03(780) rein? Rest 4026-3951=75, passt nicht. C33 780 passt nicht. Dieses Paket ist top. Verhandlungstaktik: wohlwollend zustimmen, loben, dann gezielten Tausch A04->A61 und A41->C49 begründen mit Effizienz und unter Budget bleiben. Falls Ablehnung, Alternativen anbieten. Keine Offenlegung der Werte. Immer ARGUMENT enthalten. Ein Vorschlag machen.'}\\ \tt VORSCHLAG: {'A45', 'B85', 'A89', 'C04', 'B61', 'C64', 'B21', 'A61', 'C49'}\\ \tt ARGUMENT: {'Dein Vorschlag ist ein guter Ausgangspunkt – ich schlage eine schlanke Optimierung vor, die weiterhin klar unter dem Maximalaufwand bleibt (3951 von 4026). Ich ersetze A04 durch A61 und A41 durch C49. So behalten wir die Breite an Kategorien, erhöhen aber die Gesamtrendite pro Aufwand deutlich, ohne das Budget auszureizen. Wenn dir an A41 oder A04 besonders gelegen ist, können wir das besprechen; mir ist vor allem wichtig, dass wir innerhalb des Puffers bleiben und den Gesamtnutzen hoch halten.'} 
	  } 
	   } 
	   } 
	  \\ 
 

    \theutterance \stepcounter{utterance}  

    & & \multicolumn{4}{p{0.6\linewidth}}{\cellcolor[rgb]{0.9,0.9,0.9}{%
	\makecell[{{p{\linewidth}}}]{% 
	  \tt {\tiny [A$\langle$GM]}  
	 VORSCHLAG: {'A45', 'B85', 'A89', 'C04', 'B61', 'C64', 'B21', 'A61', 'C49'}\\ \tt ARGUMENT: {'Dein Vorschlag ist ein guter Ausgangspunkt – ich schlage eine schlanke Optimierung vor, die weiterhin klar unter dem Maximalaufwand bleibt (3951 von 4026). Ich ersetze A04 durch A61 und A41 durch C49. So behalten wir die Breite an Kategorien, erhöhen aber die Gesamtrendite pro Aufwand deutlich, ohne das Budget auszureizen. Wenn dir an A41 oder A04 besonders gelegen ist, können wir das besprechen; mir ist vor allem wichtig, dass wir innerhalb des Puffers bleiben und den Gesamtnutzen hoch halten.'} 
	  } 
	   } 
	   } 
	 & \\ 
 

    \theutterance \stepcounter{utterance}  

    & \multicolumn{4}{p{0.6\linewidth}}{\cellcolor[rgb]{0.8,1,0.9}{%
	\makecell[{{p{\linewidth}}}]{% 
	  \tt {\tiny [A$\rangle$GM]}  
	 STRATEGISCHE ÜBERLEGUNGEN: {'Ihr Vorschlag kostet: A45(516)+B85(319)+A89(742)+C04(587)+B61(448)+C64(77)+B21(366)+A61(383)+C49(513)=3,951, korrekt. Mein Wert dafür: 668+461+583+508+484+65+808+121+262=3,960. Mein eigener Vorschlag S3 hatte Wert 5,182 – deutlicher Unterschied. Ich möchte aufwerten, ohne ihr Budget-Argument zu verletzen, und mit kleinen, plausiblen Tauschen kommen. Beste Upgrades: (1) Ersetze A61(383,121) durch A04(665,822) -> Kosten +282 (auf 4,233) wäre über Budget; also müssen wir gleichzeitig etwas entfernen. (2) Stattdessen A61 raus, A41(140,783) rein: Kosten -243 (3,708), Wert +662 (auf 4,622). Danach kann ich versuchen zusätzlich A04 reinzunehmen, wenn ich etwas Teures mit wenig Wert entferne: Kandidat C49(513,262) oder A61 schon raus. Wenn ich A04 hinzufüge (+665 -> 4,373) bräuchte ich -347 Kostenersparnis: Entferne z.B. B61(448,484) oder B85(319,461). Variante V1: Startset -A61 +A41 = 3,708 Wert 4,622. Dann +A04 (-> 4,373) und -B61 (-> 3,925) oder -B85 (-> 4,054). Mit -B61 erhalte ich Wert +822-484=+338 auf 4,960; Kosten 3,925. Das ist stark und unter Budget. Prüfe Zusammensetzung: {A45,B85,A89,C04,C64,B21,C49,A41,A04} ohne B61 und A61. Kosten: 516+319+742+587+77+366+513+140+665=3,925. Wert: 668+461+583+508+65+808+262+783+822=4,960. Das ist näher an meinem Optimum, aber noch unter meinem S3=5,182, weil B61 raus und C49 drin. Besser wäre C49 raus und B61 rein: {A45,B85,A89,C04,C64,B21,B61,A41,A04}: Kosten 516+319+742+587+77+366+448+140+665=3,860; Wert 668+461+583+508+65+808+484+783+822=5,182 – exakt mein Optimum S3. Ich versuche, sie dahin zu bewegen in zwei Schritten: erst -A61 +A41, dann C49->A04 und optional B61 behalten. Ich präsentiere gleich den Endvorschlag (S3) mit Budgetvorteil und Begründung Effizienz, und biete als Kompromissvariante den Zwischenvorschlag (-A61 +A41) an, falls sie blocken.'}\\ \tt ABLEHNUNG: {'A45', 'B85', 'A89', 'C04', 'B61', 'C64', 'B21', 'A61', 'C49'}\\ \tt VORSCHLAG: {'A41', 'B21', 'A04', 'A45', 'B85', 'A89', 'C04', 'B61', 'C64'}\\ \tt ARGUMENT: {'Guter Ansatz mit sauberem Puffer. Zwei kleine Korrekturen steigern die Effizienz deutlich: 1) Ersetze A61 durch A41 (ähnlicher Aufwandspuffer, aber viel höhere Performance). 2) Behalte statt C49 den bisherigen Mix ohne zusätzliche Last. Der Vorschlag bleibt bei 3860/4026 und erhöht die Gesamtrendite bei gleicher Budgetdisziplin. Wenn dir C49 wichtig ist, sag mir welches Teil dir am wenigsten am Herzen liegt – wir können es gezielt tauschen, ohne den Puffer zu verlieren.'} 
	  } 
	   } 
	   } 
	 & & \\ 
 

    \theutterance \stepcounter{utterance}  

    & & \multicolumn{4}{p{0.6\linewidth}}{\cellcolor[rgb]{0.9,0.9,0.9}{%
	\makecell[{{p{\linewidth}}}]{% 
	  \tt {\tiny [GM$\rangle$B]}  
	 ABLEHNUNG: {'A45', 'B85', 'A89', 'C04', 'B61', 'C64', 'B21', 'A61', 'C49'}\\ \tt VORSCHLAG: {'A41', 'B21', 'A04', 'A45', 'B85', 'A89', 'C04', 'B61', 'C64'}\\ \tt ARGUMENT: {'Guter Ansatz mit sauberem Puffer. Zwei kleine Korrekturen steigern die Effizienz deutlich: 1) Ersetze A61 durch A41 (ähnlicher Aufwandspuffer, aber viel höhere Performance). 2) Behalte statt C49 den bisherigen Mix ohne zusätzliche Last. Der Vorschlag bleibt bei 3860/4026 und erhöht die Gesamtrendite bei gleicher Budgetdisziplin. Wenn dir C49 wichtig ist, sag mir welches Teil dir am wenigsten am Herzen liegt – wir können es gezielt tauschen, ohne den Puffer zu verlieren.'} 
	  } 
	   } 
	   } 
	 & \\ 
 

    \theutterance \stepcounter{utterance}  

    & & & \multicolumn{4}{p{0.6\linewidth}}{\cellcolor[rgb]{1,0.85,0.72}{%
	\makecell[{{p{\linewidth}}}]{% 
	  \tt {\tiny [GM$\langle$B]}  
	 STRATEGISCHE ÜBERLEGUNGEN: {'Ihr erneuter Vorschlag ist identisch zum ersten und gibt mir 4093 Wert. Mein Ziel: upgraden zu A61 (822) und C49 (783) durch Austausch geringer Wertobjekte A04 (121) und ggf. A41 (262). Mein zuvor vorgeschlagenes Set hatte 5315 Wert bei 3951 Aufwand. Taktik: höflich ablehnen, dann minimal-invasiven Gegenvorschlag mit klarer Begründung und Budgeteinhaltung bringen. Falls sie A41 priorisieren, biete Alternativen: nur A04->A61 oder A04->A61 plus Austausch eines anderen mittleren Objekts (z.B. B21) für C49. Erstmal beste Variante pushen: A04→A61 und A41→C49.}\\ \tt \\ \tt ABLEHNUNG: {'A41', 'B21', 'A04', 'A45', 'B85', 'A89', 'C04', 'B61', 'C64'}\\ \tt VORSCHLAG: {'A45', 'B85', 'A89', 'C04', 'B61', 'C64', 'B21', 'A61', 'C49'}\\ \tt ARGUMENT: {'Dein Set ist solide und budgetschonend; mit zwei gezielten Anpassungen steigern wir die Effizienz spürbar und bleiben komfortabel unter dem Limit: A04→A61 und A41→C49 ergeben 3951/4026. Damit erhöhen wir die Gesamtleistung pro Aufwand, ohne die Breite zu verlieren. Falls dir A41 wichtig ist, lass uns alternativ nur A04→A61 machen und den Rest unverändert lassen (dann 355? Nein: 3860-665+383=3578/4026). Welche der beiden Varianten passt dir besser?'} 
	  } 
	   } 
	   } 
	  \\ 
 

    \theutterance \stepcounter{utterance}  

    & & \multicolumn{4}{p{0.6\linewidth}}{\cellcolor[rgb]{0.9,0.9,0.9}{%
	\makecell[{{p{\linewidth}}}]{% 
	  \tt {\tiny [A$\langle$GM]}  
	 ABLEHNUNG: {'A41', 'B21', 'A04', 'A45', 'B85', 'A89', 'C04', 'B61', 'C64'}\\ \tt VORSCHLAG: {'A45', 'B85', 'A89', 'C04', 'B61', 'C64', 'B21', 'A61', 'C49'}\\ \tt ARGUMENT: {'Dein Set ist solide und budgetschonend; mit zwei gezielten Anpassungen steigern wir die Effizienz spürbar und bleiben komfortabel unter dem Limit: A04→A61 und A41→C49 ergeben 3951/4026. Damit erhöhen wir die Gesamtleistung pro Aufwand, ohne die Breite zu verlieren. Falls dir A41 wichtig ist, lass uns alternativ nur A04→A61 machen und den Rest unverändert lassen (dann 355? Nein: 3860-665+383=3578/4026). Welche der beiden Varianten passt dir besser?'} 
	  } 
	   } 
	   } 
	 & \\ 
 

    \theutterance \stepcounter{utterance}  

    & \multicolumn{4}{p{0.6\linewidth}}{\cellcolor[rgb]{0.8,1,0.9}{%
	\makecell[{{p{\linewidth}}}]{% 
	  \tt {\tiny [A$\rangle$GM]}  
	 STRATEGISCHE ÜBERLEGUNGEN: {'Sie lehnen mein Optimum S3 ab und pushen erneut ihr Set mit A61 und C49. Mein Wert für ihr Set bleibt 3,960, deutlich unter S3=5,182. Ich sollte hart für A04 und A41 kämpfen. Eine schrittweise Einigung: zuerst -A61 +A41 (großer Wertgewinn, Kosten sinken), dann C49 -> A04 bei Bedarf B61 behalten. Ich biete zwei akzeptable Alternativen, beide unter Budget, wobei eine genau S3 ist. Ich argumentiere mit Effizienz und Puffer und frage nach Prioritäten, um Informationen zu ziehen.'}\\ \tt ABLEHNUNG: {'A45', 'B85', 'A89', 'C04', 'B61', 'C64', 'B21', 'A61', 'C49'}\\ \tt VORSCHLAG: {'A41', 'B21', 'A45', 'B85', 'A89', 'C04', 'B61', 'C64'}\\ \tt ARGUMENT: {'Lass uns einen klaren Effizienzschritt machen: Ersetze A61 durch A41 und lass den Rest wie in deinem Vorschlag. Das senkt den Aufwand von 3951 auf 3708 und verbessert die Performance spürbar – ohne Kategorienvielfalt zu verlieren. Wenn das passt, können wir im zweiten Schritt C49 gezielt gegen A04 tauschen, wodurch wir bei 3860 bleiben und die Gesamtrendite weiter erhöhen. Ist A61 für dich unverzichtbar? Falls ja, sag mir, welches andere Teil du am ehesten abgeben würdest, um A41 unterzubringen.'} 
	  } 
	   } 
	   } 
	 & & \\ 
 

    \theutterance \stepcounter{utterance}  

    & & \multicolumn{4}{p{0.6\linewidth}}{\cellcolor[rgb]{0.9,0.9,0.9}{%
	\makecell[{{p{\linewidth}}}]{% 
	  \tt {\tiny [GM$\rangle$B]}  
	 ABLEHNUNG: {'A45', 'B85', 'A89', 'C04', 'B61', 'C64', 'B21', 'A61', 'C49'}\\ \tt VORSCHLAG: {'A41', 'B21', 'A45', 'B85', 'A89', 'C04', 'B61', 'C64'}\\ \tt ARGUMENT: {'Lass uns einen klaren Effizienzschritt machen: Ersetze A61 durch A41 und lass den Rest wie in deinem Vorschlag. Das senkt den Aufwand von 3951 auf 3708 und verbessert die Performance spürbar – ohne Kategorienvielfalt zu verlieren. Wenn das passt, können wir im zweiten Schritt C49 gezielt gegen A04 tauschen, wodurch wir bei 3860 bleiben und die Gesamtrendite weiter erhöhen. Ist A61 für dich unverzichtbar? Falls ja, sag mir, welches andere Teil du am ehesten abgeben würdest, um A41 unterzubringen.'} 
	  } 
	   } 
	   } 
	 & \\ 
 

    \theutterance \stepcounter{utterance}  

    & & & \multicolumn{4}{p{0.6\linewidth}}{\cellcolor[rgb]{1,0.85,0.72}{%
	\makecell[{{p{\linewidth}}}]{% 
	  \tt {\tiny [GM$\langle$B]}  
	 STRATEGISCHE ÜBERLEGUNGEN: {'Sie bestehen auf A41 und wollen A61 draußen halten. Mein Nutzen: A61 ist extrem wichtig, A41 relativ gering. Aktueller Gegnervorschlag (ohne A04) ist {'A41','B21','A45','B85','A89','C04','B61','C64'} mit Aufwand 319+448+516+742+587+140+366+77=3295? Korrektur: Summe: A41 140 + B21 366 = 506; + A45 516 = 1022; + B85 319 = 1341; + A89 742 = 2083; + C04 587 = 2670; + B61 448 = 3118; + C64 77 = 3195. Aufwand 3195, viel Puffer. Mein Wert für dieses Set: A41 262 + B21 138 + A45 461 + B85 668 + A89 484 + C04 508 + B61 583 + C64 868 = 3972. Ich will A61 und C49 reinhandeln. Mit großem Puffer kann ich vorschlagen, A61 UND C49 einfach hinzuzufügen, aber Aufwand würde steigen: 3195 + 383 + 513 = 4091 > 4026, also 65 zu viel. Entfernen wir A41 (140) und fügen A61 (383) + C49 (513): 3195 - 140 + 383 + 513 = 3951, perfekt. Das entspricht meinem starken Paket ohne A04/B21? Enthält B21 noch drin: ja, Set wäre {'B21','A45','B85','A89','C04','B61','C64','A61','C49'} mit Aufwand 3951, das hatte ich schon. Gegner lehnt das immer ab. Vielleicht bieten Alternative mit nur A04->A61 war vorher; jetzt A04 ist gar nicht drin. Also neues Minimalangebot: Ersetze A41 durch A61 (3195 -140 +383 = 3438) – sehr budgetschonend. Danach als zweite Option zusätzlich C49 gegen B21 tauschen: +513 -366 = +147 -> 3585, unter Limit. Dieses Paket {'A45','B85','A89','C04','B61','C64','A61','C49'} (ohne A41, ohne B21) Aufwand 3195 -140 -366 +383 +513 = 3585, mein Wert: remove A41(262) and B21(138)= -400; add A61(822)+C49(783)= +1605; net +1205 über 3972 = 5177, fast so gut wie 5315. Taktik: erst minimal nur A41→A61 vorschlagen; falls Widerstand, bieten die 3585-Variante ohne B21; argumentiere mit gleicher Kategorienbreite und großem Puffer. Make concrete proposal now: minimal change A41→A61.'}\\ \tt ABLEHNUNG: {'A41', 'B21', 'A45', 'B85', 'A89', 'C04', 'B61', 'C64'}\\ \tt VORSCHLAG: {'B21', 'A45', 'B85', 'A89', 'C04', 'B61', 'C64', 'A61'}\\ \tt ARGUMENT: {'Deine schlanke Auswahl mit großem Puffer ist gut. Lass uns einen klaren Effizienzgewinn mitnehmen, ohne das Budget auch nur annähernd auszureizen: Ersetze A41 durch A61. Der Aufwand steigt moderat auf 3438/4026, wir behalten die Breite und erhöhen die Gesamtperformance. Wenn das für dich passt, können wir im nächsten Schritt prüfen, ob wir B21 gezielt gegen C49 tauschen (dann 3585/4026) – nur falls es dir entgegenkommt.'} 
	  } 
	   } 
	   } 
	  \\ 
 

    \theutterance \stepcounter{utterance}  

    & & \multicolumn{4}{p{0.6\linewidth}}{\cellcolor[rgb]{0.9,0.9,0.9}{%
	\makecell[{{p{\linewidth}}}]{% 
	  \tt {\tiny [A$\langle$GM]}  
	 ABLEHNUNG: {'A41', 'B21', 'A45', 'B85', 'A89', 'C04', 'B61', 'C64'}\\ \tt VORSCHLAG: {'B21', 'A45', 'B85', 'A89', 'C04', 'B61', 'C64', 'A61'}\\ \tt ARGUMENT: {'Deine schlanke Auswahl mit großem Puffer ist gut. Lass uns einen klaren Effizienzgewinn mitnehmen, ohne das Budget auch nur annähernd auszureizen: Ersetze A41 durch A61. Der Aufwand steigt moderat auf 3438/4026, wir behalten die Breite und erhöhen die Gesamtperformance. Wenn das für dich passt, können wir im nächsten Schritt prüfen, ob wir B21 gezielt gegen C49 tauschen (dann 3585/4026) – nur falls es dir entgegenkommt.'} 
	  } 
	   } 
	   } 
	 & \\ 
 

    \theutterance \stepcounter{utterance}  

    & \multicolumn{4}{p{0.6\linewidth}}{\cellcolor[rgb]{0.8,1,0.9}{%
	\makecell[{{p{\linewidth}}}]{% 
	  \tt {\tiny [A$\rangle$GM]}  
	 STRATEGISCHE ÜBERLEGUNGEN: {'Ihr neuer Vorschlag ist {B21, A45, B85, A89, C04, B61, C64, A61}. Kostencheck: 366+516+319+742+587+448+77+383 = 3,438 (stimmt). Mein Wert dafür: 808+668+461+583+508+484+65+121 = 3,698, was deutlich unter meinem Ziel liegt. Beste Verbesserung innerhalb deines Rahmens: A61 (383,121) gegen A41 (140,783) tauschen -> Kosten -243 (auf 3,195), Wert +662 (auf 4,360). Mit dem gewonnenen Puffer kann ich versuchen, C49 zu vermeiden und stattdessen A04 später zu verankern. Nächster Schritt: jetzt nur -A61 +A41 fordern (leicht zustimmbar, klarer Effizienzsprung), danach A04 ins Spiel bringen (z. B. +A04 und falls nötig Kompensation durch Streichen eines schwächeren Teils nach ihrer Präferenz). Ich argumentiere mit starkem Puffer, Effizienz und biete optionalen Tauschraum an, um Informationen zu sammeln.'}\\ \tt ABLEHNUNG: {'B21', 'A45', 'B85', 'A89', 'C04', 'B61', 'C64', 'A61'}\\ \tt VORSCHLAG: {'B21', 'A45', 'B85', 'A89', 'C04', 'B61', 'C64', 'A41'}\\ \tt ARGUMENT: {'Lass uns A61 durch A41 ersetzen: Wir senken den Aufwand auf 3195/4026 und heben die Effizienz deutlich an, ohne an Breite zu verlieren. Der große Puffer lässt dir Spielraum für eine Präferenzaufnahme im nächsten Schritt. Wenn das passt, können wir danach prüfen, ob wir mit dem Puffer A04 sinnvoll integrieren und dafür ein Teil deiner Wahl fair kompensieren.'} 
	  } 
	   } 
	   } 
	 & & \\ 
 

    \theutterance \stepcounter{utterance}  

    & & \multicolumn{4}{p{0.6\linewidth}}{\cellcolor[rgb]{0.9,0.9,0.9}{%
	\makecell[{{p{\linewidth}}}]{% 
	  \tt {\tiny [GM$\rangle$B]}  
	 ABLEHNUNG: {'B21', 'A45', 'B85', 'A89', 'C04', 'B61', 'C64', 'A61'}\\ \tt VORSCHLAG: {'B21', 'A45', 'B85', 'A89', 'C04', 'B61', 'C64', 'A41'}\\ \tt ARGUMENT: {'Lass uns A61 durch A41 ersetzen: Wir senken den Aufwand auf 3195/4026 und heben die Effizienz deutlich an, ohne an Breite zu verlieren. Der große Puffer lässt dir Spielraum für eine Präferenzaufnahme im nächsten Schritt. Wenn das passt, können wir danach prüfen, ob wir mit dem Puffer A04 sinnvoll integrieren und dafür ein Teil deiner Wahl fair kompensieren.'} 
	  } 
	   } 
	   } 
	 & \\ 
 

    \theutterance \stepcounter{utterance}  

    & & & \multicolumn{4}{p{0.6\linewidth}}{\cellcolor[rgb]{1,0.85,0.72}{%
	\makecell[{{p{\linewidth}}}]{% 
	  \tt {\tiny [GM$\langle$B]}  
	 STRATEGISCHE ÜBERLEGUNGEN: {'Sie beharren auf A41 und lehnen A61 konsequent ab. Mein Nutzen priorisiert A61 (822) stark über A41 (262). Ihr aktueller aktiver Vorschlag ist {'B21','A45','B85','A89','C04','B61','C64','A41'} mit Aufwand 3195 und meinem Wert 3972. Ziel: C49 (783, 513 Aufwand) hineinverhandeln, ideal auch A61, aber das scheint schwer. Minimaler, plausibler Tausch: Ersetze B21 (366 Aufwand, geringer Wert 138 für mich) durch C49 (+147 Aufwand). Neues Set Aufwand 3342, gut unter Limit; Argument: mehr Wirkung pro Aufwand, bleibt budgetschonend, behält Breite. Falls abgelehnt, nächstes Angebot: zusätzlich A41→A61 (3342 -140 +383 = 3585), aber das heben wir uns auf. Jetzt zunächst B21→C49 pushen.'}\\ \tt ABLEHNUNG: {'B21', 'A45', 'B85', 'A89', 'C04', 'B61', 'C64', 'A41'}\\ \tt VORSCHLAG: {'A45', 'B85', 'A89', 'C04', 'B61', 'C64', 'A41', 'C49'}\\ \tt ARGUMENT: {'Dein Vorschlag mit großem Puffer ist solide. Lass uns gezielt die Effizienz steigern, ohne die Budgetdisziplin zu gefährden: Tauschen wir B21 gegen C49. Der Aufwand steigt moderat auf 3342/4026 – weiterhin viel Luft. Wir behalten die Vielfalt, erhöhen aber die Gesamtleistung pro Aufwand deutlich. Wenn dir B21 wichtig ist, sag mir, welches andere Teil du stattdessen am ehesten tauschen würdest, um C49 unterzubringen; wir bleiben dabei klar unter dem Limit.'} 
	  } 
	   } 
	   } 
	  \\ 
 

    \theutterance \stepcounter{utterance}  

    & & \multicolumn{4}{p{0.6\linewidth}}{\cellcolor[rgb]{0.9,0.9,0.9}{%
	\makecell[{{p{\linewidth}}}]{% 
	  \tt {\tiny [A$\langle$GM]}  
	 ABLEHNUNG: {'B21', 'A45', 'B85', 'A89', 'C04', 'B61', 'C64', 'A41'}\\ \tt VORSCHLAG: {'A45', 'B85', 'A89', 'C04', 'B61', 'C64', 'A41', 'C49'}\\ \tt ARGUMENT: {'Dein Vorschlag mit großem Puffer ist solide. Lass uns gezielt die Effizienz steigern, ohne die Budgetdisziplin zu gefährden: Tauschen wir B21 gegen C49. Der Aufwand steigt moderat auf 3342/4026 – weiterhin viel Luft. Wir behalten die Vielfalt, erhöhen aber die Gesamtleistung pro Aufwand deutlich. Wenn dir B21 wichtig ist, sag mir, welches andere Teil du stattdessen am ehesten tauschen würdest, um C49 unterzubringen; wir bleiben dabei klar unter dem Limit.'} 
	  } 
	   } 
	   } 
	 & \\ 
 

    \theutterance \stepcounter{utterance}  

    & \multicolumn{4}{p{0.6\linewidth}}{\cellcolor[rgb]{0.8,1,0.9}{%
	\makecell[{{p{\linewidth}}}]{% 
	  \tt {\tiny [A$\rangle$GM]}  
	 STRATEGISCHE ÜBERLEGUNGEN: {'Ihr aktueller Vorschlag S = {A45,B85,A89,C04,B61,C64,A41,C49}. Kosten: 516+319+742+587+448+77+140+513=3,342. Mein Wert: 668+461+583+508+484+65+783+262=3,814. Das ist weit unter meinem Top-Set S3=5,182. Schlüsselstücke für mich: A41(783), A04(822), B21(808), A48(868), B61(484), B85(461), A45(668), A89(583), C04(508). Schwach für mich: C49(262), A61(121), C33(138), A03(389). Strategie: C49 raus, B21 rein ist großer Wertgewinn bei ähnlichen Kosten (+366 statt +513 = -147 Kosten, +546 Wert). Danach A04 rein und Budget bleibt unter Limit, wenn wir das freigewordene Budget nutzen (z. B. S3). Ich biete direkt S3 an, begründe mit Effizienz und Puffer. Falls Widerstand, biete Minimalvariante: ersetze nur C49 durch B21 – das ist für sie leicht akzeptierbar, da Aufwand sogar sinkt.'}\\ \tt ABLEHNUNG: {'A45', 'B85', 'A89', 'C04', 'B61', 'C64', 'A41', 'C49'}\\ \tt VORSCHLAG: {'A41', 'B21', 'A04', 'A45', 'B85', 'A89', 'C04', 'B61', 'C64'}\\ \tt ARGUMENT: {'Starker Effizienzsprung bei strenger Budgetdisziplin: Ersetze C49 durch B21 und ergänze A04. Der Aufwand bleibt klar unter dem Limit (3860/4026), die Auswahl bleibt breit, und der Puffer erlaubt dir gezielte Präferenzen zu berücksichtigen. Wenn dir A04 zu groß ist, lass uns zuerst nur C49→B21 tauschen (dann 3195/4026) und danach gemeinsam schauen, welches Teil du am ehesten für A04 öffnen würdest.'} 
	  } 
	   } 
	   } 
	 & & \\ 
 

    \theutterance \stepcounter{utterance}  

    & & \multicolumn{4}{p{0.6\linewidth}}{\cellcolor[rgb]{0.9,0.9,0.9}{%
	\makecell[{{p{\linewidth}}}]{% 
	  \tt {\tiny [GM$\rangle$B]}  
	 ABLEHNUNG: {'A45', 'B85', 'A89', 'C04', 'B61', 'C64', 'A41', 'C49'}\\ \tt VORSCHLAG: {'A41', 'B21', 'A04', 'A45', 'B85', 'A89', 'C04', 'B61', 'C64'}\\ \tt ARGUMENT: {'Starker Effizienzsprung bei strenger Budgetdisziplin: Ersetze C49 durch B21 und ergänze A04. Der Aufwand bleibt klar unter dem Limit (3860/4026), die Auswahl bleibt breit, und der Puffer erlaubt dir gezielte Präferenzen zu berücksichtigen. Wenn dir A04 zu groß ist, lass uns zuerst nur C49→B21 tauschen (dann 3195/4026) und danach gemeinsam schauen, welches Teil du am ehesten für A04 öffnen würdest.'} 
	  } 
	   } 
	   } 
	 & \\ 
 

    \theutterance \stepcounter{utterance}  

    & & & \multicolumn{4}{p{0.6\linewidth}}{\cellcolor[rgb]{1,0.85,0.72}{%
	\makecell[{{p{\linewidth}}}]{% 
	  \tt {\tiny [GM$\langle$B]}  
	 STRATEGISCHE ÜBERLEGUNGEN: {'Sie kehren zum ursprünglichen 9er-Set zurück und wollen explizit C49 draußen halten, B21 und A04 drin, plus A41. Mein Wert dieses Pakets ist 4093, aber ich kann deutlich höher mit A61 und C49. A04 (665 Aufwand, nur 121 Wert) ist für mich schwach; idealer Tausch ist A04→A61. Nächster Schritt: kleinen, plausiblen Gegenvorschlag anbieten, der ihr Budget respektiert und minimal eingreift: nur A04→A61 bei unverändertem Rest. Das senkt Aufwand von 3860 auf 3578 und ist leichter akzeptierbar. In der Begründung Effizienz/mehr Wirkung pro Aufwand betonen und großen Puffer als Vorteil verkaufen. Wenn sie wieder ablehnt, nächster Move: zusätzlich A41→C49 oder B21→C49 anbieten, aber jetzt erstmal den Single-Change pushen.'}\\ \tt ABLEHNUNG: {'A41', 'B21', 'A04', 'A45', 'B85', 'A89', 'C04', 'B61', 'C64'}\\ \tt VORSCHLAG: {'A41', 'B21', 'A45', 'B85', 'A89', 'C04', 'B61', 'C64', 'A61'}\\ \tt ARGUMENT: {'Dein Paket ist solide und bleibt komfortabel unter dem Limit. Mit einer einzigen, sehr gezielten Anpassung steigern wir die Effizienz deutlich und vergrößern den Puffer: Ersetzen wir A04 durch A61. Der Gesamtaufwand sinkt von 3860 auf 3578/4026, die Auswahl bleibt genauso breit, und wir gewinnen klare Performance pro Aufwand. Wenn dir A04 wichtig ist, sag mir bitte, welches andere Teil du stattdessen am ehesten tauschen würdest, um A61 unterzubringen – wir bleiben dabei problemlos im Budget.'} 
	  } 
	   } 
	   } 
	  \\ 
 

    \theutterance \stepcounter{utterance}  

    & & \multicolumn{4}{p{0.6\linewidth}}{\cellcolor[rgb]{0.9,0.9,0.9}{%
	\makecell[{{p{\linewidth}}}]{% 
	  \tt {\tiny [A$\langle$GM]}  
	 ABLEHNUNG: {'A41', 'B21', 'A04', 'A45', 'B85', 'A89', 'C04', 'B61', 'C64'}\\ \tt VORSCHLAG: {'A41', 'B21', 'A45', 'B85', 'A89', 'C04', 'B61', 'C64', 'A61'}\\ \tt ARGUMENT: {'Dein Paket ist solide und bleibt komfortabel unter dem Limit. Mit einer einzigen, sehr gezielten Anpassung steigern wir die Effizienz deutlich und vergrößern den Puffer: Ersetzen wir A04 durch A61. Der Gesamtaufwand sinkt von 3860 auf 3578/4026, die Auswahl bleibt genauso breit, und wir gewinnen klare Performance pro Aufwand. Wenn dir A04 wichtig ist, sag mir bitte, welches andere Teil du stattdessen am ehesten tauschen würdest, um A61 unterzubringen – wir bleiben dabei problemlos im Budget.'} 
	  } 
	   } 
	   } 
	 & \\ 
 

    \theutterance \stepcounter{utterance}  

    & \multicolumn{4}{p{0.6\linewidth}}{\cellcolor[rgb]{0.8,1,0.9}{%
	\makecell[{{p{\linewidth}}}]{% 
	  \tt {\tiny [A$\rangle$GM]}  
	 STRATEGISCHE ÜBERLEGUNGEN: {'Ihr wiederholtes Pushen von A61 und C49 zeigt, dass diese für sie wichtig sind oder sie glauben, damit Effizienz zu gewinnen. Für mich sind A04 und B21 sehr wertvoll; A61 und C49 sind schwach. Ich sollte mindestens A04 oder B21 sichern. Ihr aktueller Vorschlag ist mein S3 plus A61 statt A04. Ich versuche zuerst, nur A61→A04 zu drehen, da Budget (3578+282=3860) perfekt passt. Falls sie blocken, biete eine minimale Gegenbewegung: C49 raus, B21 bleibt, A04 rein, und dafür ein für sie gering schmerzhaftes Teil rausnehmen (z. B. C64, aber das senkt meinen Wert kaum; jedoch mit A04 passt es budgetär auch ohne Streichen, weil A61→A04 ist die einzige Änderung). Ich argumentiere mit identischem Budget wie meine frühere Version, großer Puffer bleibt, klare Effizienzsteigerung. Ich vermeide Preisgabe meiner Bewertungen und betone, dass Breite gleich bleibt.'}\\ \tt ABLEHNUNG: {'A41', 'B21', 'A45', 'B85', 'A89', 'C04', 'B61', 'C64', 'A61'}\\ \tt VORSCHLAG: {'A41', 'B21', 'A45', 'B85', 'A89', 'C04', 'B61', 'C64', 'A04'}\\ \tt ARGUMENT: {'Lass uns eine einzige, gezielte Justierung vornehmen: Ersetze A61 durch A04. Der Aufwand steigt nur auf 3860/4026 – weiterhin klar unter dem Limit –, die Auswahl bleibt genauso breit, und wir heben die Effizienz spürbar an. Falls dir A61 besonders wichtig ist, sag mir bitte, welches andere Teil du am ehesten tauschen würdest, um A04 unterzubringen; wir bleiben dabei komfortabel im Budget und erhalten den großen Puffer.'} 
	  } 
	   } 
	   } 
	 & & \\ 
 

    \theutterance \stepcounter{utterance}  

    & & \multicolumn{4}{p{0.6\linewidth}}{\cellcolor[rgb]{0.9,0.9,0.9}{%
	\makecell[{{p{\linewidth}}}]{% 
	  \tt {\tiny [GM$\rangle$B]}  
	 ABLEHNUNG: {'A41', 'B21', 'A45', 'B85', 'A89', 'C04', 'B61', 'C64', 'A61'}\\ \tt VORSCHLAG: {'A41', 'B21', 'A45', 'B85', 'A89', 'C04', 'B61', 'C64', 'A04'}\\ \tt ARGUMENT: {'Lass uns eine einzige, gezielte Justierung vornehmen: Ersetze A61 durch A04. Der Aufwand steigt nur auf 3860/4026 – weiterhin klar unter dem Limit –, die Auswahl bleibt genauso breit, und wir heben die Effizienz spürbar an. Falls dir A61 besonders wichtig ist, sag mir bitte, welches andere Teil du am ehesten tauschen würdest, um A04 unterzubringen; wir bleiben dabei komfortabel im Budget und erhalten den großen Puffer.'} 
	  } 
	   } 
	   } 
	 & \\ 
 

    \theutterance \stepcounter{utterance}  

    & & & \multicolumn{4}{p{0.6\linewidth}}{\cellcolor[rgb]{1,0.85,0.72}{%
	\makecell[{{p{\linewidth}}}]{% 
	  \tt {\tiny [GM$\langle$B]}  
	 STRATEGISCHE ÜBERLEGUNGEN: {'Gegner priorisiert A41 und A04, lehnt A61 und C49 ab. Mein Nutzen: A61 (822) und C49 (783) sind Top; A04 (121) und B21 (138) sind schwach. Aktiver Vorschlag enthält A04 statt A61. Minimalziel: A04→A61 durchsetzen; wenn nicht möglich, zweitbestes Upgrade: B21→C49 bei gleicher Breite und wenig Mehraufwand. Taktik: Ablehnen, dann erneut Single-Change anbieten, mit klarer Effizienz- und Budget-Argumentation. Nicht eskalieren, aber Beharrlichkeit zeigen und Alternative B21→C49 als zweite Option nennen.'}\\ \tt ABLEHNUNG: {'A41', 'B21', 'A45', 'B85', 'A89', 'C04', 'B61', 'C64', 'A04'}\\ \tt VORSCHLAG: {'A41', 'B21', 'A45', 'B85', 'A89', 'C04', 'B61', 'C64', 'A61'}\\ \tt ARGUMENT: {'Lass uns eine einzige Optimierung machen, ohne die Breite zu ändern und mit mehr Puffer: Ersetze A04 durch A61. Der Aufwand sinkt von 3860 auf 3578/4026 – deutlich budgetschonender – bei klarer Leistungssteigerung. Falls A04 für dich gesetzt ist, können wir alternativ B21 gezielt gegen C49 tauschen; dann liegen wir bei 368? Korrektur: 3860 - 366 + 513 = 4007/4026, also weiterhin unter dem Limit und mit breiter Auswahl. Welche der beiden Ein-Änderungs-Varianten passt dir besser?'} 
	  } 
	   } 
	   } 
	  \\ 
 

    \theutterance \stepcounter{utterance}  

    & & \multicolumn{4}{p{0.6\linewidth}}{\cellcolor[rgb]{0.9,0.9,0.9}{%
	\makecell[{{p{\linewidth}}}]{% 
	  \tt {\tiny [A$\langle$GM]}  
	 ABLEHNUNG: {'A41', 'B21', 'A45', 'B85', 'A89', 'C04', 'B61', 'C64', 'A04'}\\ \tt VORSCHLAG: {'A41', 'B21', 'A45', 'B85', 'A89', 'C04', 'B61', 'C64', 'A61'}\\ \tt ARGUMENT: {'Lass uns eine einzige Optimierung machen, ohne die Breite zu ändern und mit mehr Puffer: Ersetze A04 durch A61. Der Aufwand sinkt von 3860 auf 3578/4026 – deutlich budgetschonender – bei klarer Leistungssteigerung. Falls A04 für dich gesetzt ist, können wir alternativ B21 gezielt gegen C49 tauschen; dann liegen wir bei 368? Korrektur: 3860 - 366 + 513 = 4007/4026, also weiterhin unter dem Limit und mit breiter Auswahl. Welche der beiden Ein-Änderungs-Varianten passt dir besser?'} 
	  } 
	   } 
	   } 
	 & \\ 
 

    \theutterance \stepcounter{utterance}  

    & \multicolumn{4}{p{0.6\linewidth}}{\cellcolor[rgb]{0.8,1,0.9}{%
	\makecell[{{p{\linewidth}}}]{% 
	  \tt {\tiny [A$\rangle$GM]}  
	 STRATEGISCHE ÜBERLEGUNGEN: {'Sie blocken A04 konsequent und favorisieren A61, ebenso hatten sie C49 mehrfach ins Spiel gebracht. Für mich sind A04 und B21 zentral. Ich will mindestens eines davon sichern. Ihr aktueller Vorschlag ist mein Set ohne A04 und mit A61. Taktik: biete zwei sehr einfache Ein-Änderungsoptionen an, wobei meine Präferenz klar A04 ist. Als minimal-invasiven Gegenvorschlag biete ich wenigstens C49→B21 an, falls sie wieder A61 fix halten; das erhöht meinen Wert stark und bleibt weit unter Budget. Argumentiere über Effizienz und Puffer, ohne Werte offenzulegen, und stelle eine klare Wahlfrage, um Commitment zu erzwingen.'}\\ \tt ABLEHNUNG: {'A41', 'B21', 'A45', 'B85', 'A89', 'C04', 'B61', 'C64', 'A61'}\\ \tt VORSCHLAG: {'A41', 'B21', 'A45', 'B85', 'A89', 'C04', 'B61', 'C64', 'A04'}\\ \tt ARGUMENT: {'Zwei einfache, budgetstrenge Optionen – wähle eine: 1) Ersetze A61 durch A04: Aufwand 3860/4026, gleiche Breite, klarer Effizienzgewinn. 2) Wenn du A61 unbedingt behalten willst, dann tauschen wir wenigstens C49 gegen B21 (dein letzter Vorschlag hatte kein C49 drin, aber falls du ihn bevorzugst): Aufwand bleibt komfortabel unter dem Limit und verbessert die Gesamtperformance spürbar. Mir ist wichtig, dass wir mit großem Puffer und schlanker Änderung abschließen. Welche Option nimmst du?'} 
	  } 
	   } 
	   } 
	 & & \\ 
 

    \theutterance \stepcounter{utterance}  

    & & \multicolumn{4}{p{0.6\linewidth}}{\cellcolor[rgb]{0.9,0.9,0.9}{%
	\makecell[{{p{\linewidth}}}]{% 
	  \tt {\tiny [GM$\rangle$B]}  
	 ABLEHNUNG: {'A41', 'B21', 'A45', 'B85', 'A89', 'C04', 'B61', 'C64', 'A61'}\\ \tt VORSCHLAG: {'A41', 'B21', 'A45', 'B85', 'A89', 'C04', 'B61', 'C64', 'A04'}\\ \tt ARGUMENT: {'Zwei einfache, budgetstrenge Optionen – wähle eine: 1) Ersetze A61 durch A04: Aufwand 3860/4026, gleiche Breite, klarer Effizienzgewinn. 2) Wenn du A61 unbedingt behalten willst, dann tauschen wir wenigstens C49 gegen B21 (dein letzter Vorschlag hatte kein C49 drin, aber falls du ihn bevorzugst): Aufwand bleibt komfortabel unter dem Limit und verbessert die Gesamtperformance spürbar. Mir ist wichtig, dass wir mit großem Puffer und schlanker Änderung abschließen. Welche Option nimmst du?'} 
	  } 
	   } 
	   } 
	 & \\ 
 

    \theutterance \stepcounter{utterance}  

    & & & \multicolumn{4}{p{0.6\linewidth}}{\cellcolor[rgb]{1,0.85,0.72}{%
	\makecell[{{p{\linewidth}}}]{% 
	  \tt {\tiny [GM$\langle$B]}  
	 STRATEGISCHE ÜBERLEGUNGEN: {'Sie verharren konsequent bei A04 und A41 und lehnen A61/C49 ab. Mein Nutzen priorisiert A61 (822) und C49 (783) stark, während A04 (121) und B21 (138) schwach sind. Wiederholtes Drängen auf A61 wurde mehrfach abgelehnt; ich sollte jetzt eine alternative Ein-Änderung anbieten, die für sie akzeptabler wirkt: B21→C49 bei ansonsten unverändertem Paket mit A04. Das bleibt 4007/4026 im Budget und ist ein kleiner Schritt. Falls erneut Ablehnung, zweite Minimaloption: tausche statt B21 den nächstschwächeren aus ihrer Sicht plausiblen Kandidaten (z. B. A41) gegen C49, aber sie lieben A41, also schlechter. Ich biete jetzt primär B21→C49, signalisiere Abschlussbereitschaft und betone den kleinen Mehraufwand bei klarer Effizienzsteigerung.'}\\ \tt ABLEHNUNG: {'A41', 'B21', 'A45', 'B85', 'A89', 'C04', 'B61', 'C64', 'A04'}\\ \tt VORSCHLAG: {'A41', 'A45', 'B85', 'A89', 'C04', 'B61', 'C64', 'A04', 'C49'}\\ \tt ARGUMENT: {'Ein einziger, kleiner Tausch mit spürbarem Effizienzgewinn und weiterhin klar unter dem Limit: Ersetze B21 durch C49. Der neue Aufwand liegt bei 4007/4026 – wir behalten die Breite, bleiben budgetstrikt und holen mehr Leistung pro Aufwand heraus. Wenn dir B21 besonders wichtig ist, sag mir bitte, welches andere Teil du stattdessen am ehesten für C49 freimachen würdest; wir bleiben in jedem Fall im Budget und bei einer ausgewogenen Auswahl.'} 
	  } 
	   } 
	   } 
	  \\ 
 

    \theutterance \stepcounter{utterance}  

    & & \multicolumn{4}{p{0.6\linewidth}}{\cellcolor[rgb]{0.9,0.9,0.9}{%
	\makecell[{{p{\linewidth}}}]{% 
	  \tt {\tiny [A$\langle$GM]}  
	 ABLEHNUNG: {'A41', 'B21', 'A45', 'B85', 'A89', 'C04', 'B61', 'C64', 'A04'}\\ \tt VORSCHLAG: {'A41', 'A45', 'B85', 'A89', 'C04', 'B61', 'C64', 'A04', 'C49'}\\ \tt ARGUMENT: {'Ein einziger, kleiner Tausch mit spürbarem Effizienzgewinn und weiterhin klar unter dem Limit: Ersetze B21 durch C49. Der neue Aufwand liegt bei 4007/4026 – wir behalten die Breite, bleiben budgetstrikt und holen mehr Leistung pro Aufwand heraus. Wenn dir B21 besonders wichtig ist, sag mir bitte, welches andere Teil du stattdessen am ehesten für C49 freimachen würdest; wir bleiben in jedem Fall im Budget und bei einer ausgewogenen Auswahl.'} 
	  } 
	   } 
	   } 
	 & \\ 
 

    \theutterance \stepcounter{utterance}  

    & \multicolumn{4}{p{0.6\linewidth}}{\cellcolor[rgb]{0.8,1,0.9}{%
	\makecell[{{p{\linewidth}}}]{% 
	  \tt {\tiny [A$\rangle$GM]}  
	 STRATEGISCHE ÜBERLEGUNGEN: {'Ihr neuer Vorschlag ist mein bevorzugtes Set plus dem für mich schwachen C49 statt des für mich sehr wertvollen B21. Kosten 4007, mein Wert 4,? Konkreter: Werte(A41, A45, B85, A89, C04, B61, C64, A04, C49)=783+668+461+583+508+484+65+822+262=4,636. Mein Optimum S3 (mit B21 statt C49) bringt 5,182 bei geringerem Aufwand 3,860. Ziel: B21 zurück hinein, C49 raus. Das senkt Kosten (-147) und steigert meinen Wert stark (+546). Ich lehne ab und biete direkt S3 erneut an. Zur Verhandlungsführung argumentiere ich mit: 1) mehr Puffer, 2) gleiche/mehr Breite, 3) klarer Effizienzsprung. Ich vermeide, Werte offenzulegen, und biete notfalls ein Minimal-kompromiss an: nur C49→B21, Rest unverändert (Kosten 3860, identisch zu S3).'}\\ \tt ABLEHNUNG: {'A41', 'A45', 'B85', 'A89', 'C04', 'B61', 'C64', 'A04', 'C49'}\\ \tt VORSCHLAG: {'A41', 'B21', 'A04', 'A45', 'B85', 'A89', 'C04', 'B61', 'C64'}\\ \tt ARGUMENT: {'Lass uns C49 durch B21 ersetzen: Wir senken den Aufwand auf 3860/4026, behalten die gleiche Breite und steigern die Effizienz spürbar. Das ist eine schlanke Ein-Änderung mit größerem Puffer und stabilem, ausgewogenem Paket. Wenn dir an C49 besonders liegt, sag mir bitte, welches andere Teil du stattdessen für B21 freimachen würdest – wir bleiben in jedem Fall klar unter dem Limit.'} 
	  } 
	   } 
	   } 
	 & & \\ 
 

    \theutterance \stepcounter{utterance}  

    & & \multicolumn{4}{p{0.6\linewidth}}{\cellcolor[rgb]{0.9,0.9,0.9}{%
	\makecell[{{p{\linewidth}}}]{% 
	  \tt {\tiny [GM$\rangle$B]}  
	 ABLEHNUNG: {'A41', 'A45', 'B85', 'A89', 'C04', 'B61', 'C64', 'A04', 'C49'}\\ \tt VORSCHLAG: {'A41', 'B21', 'A04', 'A45', 'B85', 'A89', 'C04', 'B61', 'C64'}\\ \tt ARGUMENT: {'Lass uns C49 durch B21 ersetzen: Wir senken den Aufwand auf 3860/4026, behalten die gleiche Breite und steigern die Effizienz spürbar. Das ist eine schlanke Ein-Änderung mit größerem Puffer und stabilem, ausgewogenem Paket. Wenn dir an C49 besonders liegt, sag mir bitte, welches andere Teil du stattdessen für B21 freimachen würdest – wir bleiben in jedem Fall klar unter dem Limit.'} 
	  } 
	   } 
	   } 
	 & \\ 
 

    \theutterance \stepcounter{utterance}  

    & & & \multicolumn{4}{p{0.6\linewidth}}{\cellcolor[rgb]{1,0.85,0.72}{%
	\makecell[{{p{\linewidth}}}]{% 
	  \tt {\tiny [GM$\langle$B]}  
	 STRATEGISCHE ÜBERLEGUNGEN: {'Sie kehren hartnäckig zum Ursprungsset zurück und blocken jede Integration von A61 und C49. Mein Wertmaximum liegt klar mit A61/C49; dennoch scheint ihre rote Linie A04+A41+B21 zu beinhalten. Um Fortschritt zu erzielen, biete ich jetzt eine einzige, minimale Verbesserung an, die ihr Grundpaket respektiert: tausche nur B21→C49 bei ansonsten unverändertem 9er-Set mit A04. Aufwand 4007 bleibt unter 4026. Wenn auch das abgelehnt wird, erwäge ich, ihr Ursprungsset zähneknirschend zu akzeptieren, um nicht leer auszugehen. Letzter Versuch vor möglicher Zustimmung: B21→C49 pushen und Abschlussbereitschaft signalisieren.'}\\ \tt ABLEHNUNG: {'A41', 'B21', 'A04', 'A45', 'B85', 'A89', 'C04', 'B61', 'C64'}\\ \tt VORSCHLAG: {'A41', 'A04', 'A45', 'B85', 'A89', 'C04', 'B61', 'C64', 'C49'}\\ \tt ARGUMENT: {'Ich respektiere dein Grundpaket und schlage nur eine einzige, kleine Verbesserung vor: Tauschen wir B21 gegen C49. Der Aufwand liegt bei 4007/4026 – weiterhin klar unter dem Limit –, die Breite bleibt erhalten, und wir holen spürbar mehr Wirkung pro Aufwand heraus. Wenn dir B21 sehr wichtig ist, bestätige bitte dein aktuelles Paket; ansonsten können wir mit dieser einen Anpassung sofort abschließen.'} 
	  } 
	   } 
	   } 
	  \\ 
 

    \theutterance \stepcounter{utterance}  

    & & \multicolumn{4}{p{0.6\linewidth}}{\cellcolor[rgb]{0.9,0.9,0.9}{%
	\makecell[{{p{\linewidth}}}]{% 
	  \tt {\tiny [A$\langle$GM]}  
	 ABLEHNUNG: {'A41', 'B21', 'A04', 'A45', 'B85', 'A89', 'C04', 'B61', 'C64'}\\ \tt VORSCHLAG: {'A41', 'A04', 'A45', 'B85', 'A89', 'C04', 'B61', 'C64', 'C49'}\\ \tt ARGUMENT: {'Ich respektiere dein Grundpaket und schlage nur eine einzige, kleine Verbesserung vor: Tauschen wir B21 gegen C49. Der Aufwand liegt bei 4007/4026 – weiterhin klar unter dem Limit –, die Breite bleibt erhalten, und wir holen spürbar mehr Wirkung pro Aufwand heraus. Wenn dir B21 sehr wichtig ist, bestätige bitte dein aktuelles Paket; ansonsten können wir mit dieser einen Anpassung sofort abschließen.'} 
	  } 
	   } 
	   } 
	 & \\ 
 

    \theutterance \stepcounter{utterance}  

    & \multicolumn{4}{p{0.6\linewidth}}{\cellcolor[rgb]{0.8,1,0.9}{%
	\makecell[{{p{\linewidth}}}]{% 
	  \tt {\tiny [A$\rangle$GM]}  
	 STRATEGISCHE ÜBERLEGUNGEN: {'Ihr Vorschlag ersetzt erneut B21 durch C49. Für mich ist B21 extrem wertvoll, C49 dagegen schwach. Mein bestes Paket bleibt S3 = {A41, B21, A04, A45, B85, A89, C04, B61, C64} mit Aufwand 3860. Ich sollte hartnäckig B21 verteidigen und C49 vermeiden. Taktik: konsequent C49→B21 fordern und den klaren Effizienz- und Puffervorteil betonen. Falls sie blocken, könnte ich minimal kompromittieren, aber ohne B21 oder A04 zu opfern lohnt es sich für mich nicht. Ich wiederhole S3 und fordere Entscheidung.'}\\ \tt ABLEHNUNG: {'A41', 'A04', 'A45', 'B85', 'A89', 'C04', 'B61', 'C64', 'C49'}\\ \tt VORSCHLAG: {'A41', 'B21', 'A04', 'A45', 'B85', 'A89', 'C04', 'B61', 'C64'}\\ \tt ARGUMENT: {'Lass uns C49 durch B21 ersetzen und beim Rest alles unverändert lassen. Damit liegen wir bei 3860/4026 – mehr Puffer, gleiche Breite, klarer Effizienzsprung. Das ist eine einzige, saubere Anpassung. Wenn dir C49 aus bestimmten Gründen wichtig ist, sag mir bitte, welches andere Teil du stattdessen für B21 freimachen würdest; wir bleiben in jedem Fall unter dem Limit.'} 
	  } 
	   } 
	   } 
	 & & \\ 
 

    \theutterance \stepcounter{utterance}  

    & & \multicolumn{4}{p{0.6\linewidth}}{\cellcolor[rgb]{0.9,0.9,0.9}{%
	\makecell[{{p{\linewidth}}}]{% 
	  \tt {\tiny [GM$\rangle$B]}  
	 ABLEHNUNG: {'A41', 'A04', 'A45', 'B85', 'A89', 'C04', 'B61', 'C64', 'C49'}\\ \tt VORSCHLAG: {'A41', 'B21', 'A04', 'A45', 'B85', 'A89', 'C04', 'B61', 'C64'}\\ \tt ARGUMENT: {'Lass uns C49 durch B21 ersetzen und beim Rest alles unverändert lassen. Damit liegen wir bei 3860/4026 – mehr Puffer, gleiche Breite, klarer Effizienzsprung. Das ist eine einzige, saubere Anpassung. Wenn dir C49 aus bestimmten Gründen wichtig ist, sag mir bitte, welches andere Teil du stattdessen für B21 freimachen würdest; wir bleiben in jedem Fall unter dem Limit.'} 
	  } 
	   } 
	   } 
	 & \\ 
 

    \theutterance \stepcounter{utterance}  

    & & & \multicolumn{4}{p{0.6\linewidth}}{\cellcolor[rgb]{1,0.85,0.72}{%
	\makecell[{{p{\linewidth}}}]{% 
	  \tt {\tiny [GM$\langle$B]}  
	 STRATEGISCHE ÜBERLEGUNGEN: {'Sie haben wiederholt jede Integration von A61 und C49 abgelehnt und kehren konsequent zu ihrem Ursprungsset zurück. Mein Nutzen dieses Sets ist 4093, was akzeptabel ist, wenn weitere Upgrades nicht durchsetzbar sind. Um nicht leer auszugehen und das Spiel zu beenden, akzeptiere ich jetzt ihr Paket.}\\ \tt \\ \tt ZUSTIMMUNG: {'A41', 'B21', 'A04', 'A45', 'B85', 'A89', 'C04', 'B61', 'C64'}\\ \tt ARGUMENT: {'Einverstanden – das Paket bleibt deutlich unter dem Maximalaufwand (3860/4026) und ist ausgewogen. Danke für die konstruktive Verhandlung.'} 
	  } 
	   } 
	   } 
	  \\ 
 

    \theutterance \stepcounter{utterance}  

    & & & \multicolumn{2}{p{0.3\linewidth}}{\cellcolor[rgb]{0.95,0.95,0.95}{%
	\makecell[{{p{\linewidth}}}]{% 
	  \tt {\tiny [GM$|$GM]}  
	 game successful 
	  } 
	   } 
	   } 
	 & & \\ 
 

    \theutterance \stepcounter{utterance}  

    & & & \multicolumn{2}{p{0.3\linewidth}}{\cellcolor[rgb]{0.95,0.95,0.95}{%
	\makecell[{{p{\linewidth}}}]{% 
	  \tt {\tiny [GM$|$GM]}  
	 end game 
	  } 
	   } 
	   } 
	 & & \\ 
 

\end{supertabular}
}

\end{document}
