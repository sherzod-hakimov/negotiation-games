
\documentclass{article}
\usepackage{colortbl}
\usepackage{makecell}
\usepackage{multirow}
\usepackage{supertabular}

\begin{document}

\newcounter{utterance}

\twocolumn

{ \footnotesize  \setcounter{utterance}{1}
\setlength{\tabcolsep}{0pt}
\begin{supertabular}{c@{$\;$}|p{.15\linewidth}@{}p{.15\linewidth}p{.15\linewidth}p{.15\linewidth}p{.15\linewidth}p{.15\linewidth}}

    \# & $\;$A & \multicolumn{4}{c}{Game Master} & $\;\:$B\\
    \hline 

    \theutterance \stepcounter{utterance}  

    & & \multicolumn{4}{p{0.6\linewidth}}{\cellcolor[rgb]{0.9,0.9,0.9}{%
	\makecell[{{p{\linewidth}}}]{% 
	  \tt {\tiny [A$\langle$GM]}  
	 Du nimmst an einem kollaborativen Verhandlungspiel Teil.\\ \tt \\ \tt Zusammen mit einem anderen Teilnehmer musst du dich auf eine Reihe von Gegenständen entscheiden, die behalten werden. Jeder von euch hat eine persönliche Verteilung über die Wichtigkeit der einzelnen Gegenstände. Jeder von euch hat eine eigene Meinung darüber, wie wichtig jeder einzelne Gegenstand ist (Gegenstandswichtigkeit). Du kennst die Wichtigkeitsverteilung des anderen Spielers nicht. Zusätzlich siehst du, wie viel Aufwand jeder Gegenstand verursacht.  \\ \tt Ihr dürft euch nur auf eine Reihe von Gegenständen einigen, wenn der Gesamtaufwand der ausgewählten Gegenstände den Maximalaufwand nicht überschreitet:\\ \tt \\ \tt Maximalaufwand = 3681\\ \tt \\ \tt Hier sind die einzelnen Aufwände der Gegenstände:\\ \tt \\ \tt Aufwand der Gegenstände = {"C33": 700, "A89": 586, "A48": 505, "A04": 116, "A41": 664, "C64": 940, "C49": 387, "A61": 392, "C04": 209, "B53": 571, "B85": 4, "B61": 285, "A45": 651, "A03": 613, "B21": 740}\\ \tt \\ \tt Hier ist deine persönliche Verteilung der Wichtigkeit der einzelnen Gegenstände:\\ \tt \\ \tt Werte der Gegenstandswichtigkeit = {"C33": 138, "A89": 583, "A48": 868, "A04": 822, "A41": 783, "C64": 65, "C49": 262, "A61": 121, "C04": 508, "B53": 780, "B85": 461, "B61": 484, "A45": 668, "A03": 389, "B21": 808}\\ \tt \\ \tt Ziel:\\ \tt \\ \tt Dein Ziel ist es, eine Reihe von Gegenständen auszuhandeln, die dir möglichst viel bringt (d. h. Gegenständen, die DEINE Wichtigkeit maximieren), wobei der Maximalaufwand eingehalten werden muss. Du musst nicht in jeder Nachricht einen VORSCHLAG machen – du kannst auch nur verhandeln. Alle Taktiken sind erlaubt!\\ \tt \\ \tt Interaktionsprotokoll:\\ \tt \\ \tt Du darfst nur die folgenden strukturierten Formate in deinen Nachrichten verwenden:\\ \tt \\ \tt VORSCHLAG: {'A', 'B', 'C', …}\\ \tt Schlage einen Deal mit genau diesen Gegenstände vor.\\ \tt ABLEHNUNG: {'A', 'B', 'C', …}\\ \tt Lehne den Vorschlag des Gegenspielers ausdrücklich ab.\\ \tt ARGUMENT: {'...'}\\ \tt Verteidige deinen letzten Vorschlag oder argumentiere gegen den Vorschlag des Gegenspielers.\\ \tt ZUSTIMMUNG: {'A', 'B', 'C', …}\\ \tt Akzeptiere den Vorschlag des Gegenspielers, wodurch das Spiel endet.\\ \tt STRATEGISCHE ÜBERLEGUNGEN: {'...'}\\ \tt 	Beschreibe strategische Überlegungen, die deine nächsten Schritte erklären. Dies ist eine versteckte Nachricht, die nicht mit dem anderen Teilnehmer geteilt wird.\\ \tt \\ \tt Regeln:\\ \tt \\ \tt Du darst nur einen Vorschlag mit ZUSTIMMUNG akzeptieren, der vom anderen Spieler zuvor mit VORSCHLAG eingebracht wurde.\\ \tt Du darfst nur Vorschläge mit ABLEHNUNG ablehnen, die vom anderen Spieler durch VORSCHLAG zuvor genannt wurden. \\ \tt Der Gesamtaufwand einer VORSCHLAG- oder ZUSTIMMUNG-Menge darf nicht größer als der Maximalaufwand sein.  \\ \tt Offenbare deine versteckte Wichtigkeitsverteilung nicht.\\ \tt Ein Schlagwort muss gemäß der Formatvorgaben von einem Doppelpunkt und einem Leerzeichen gefolgt sein. Das Argument muss eine Python-Menge mit 0 oder mehr Strings sein.  \\ \tt Die Form muss also SCHLAGWORT: {...} sein.\\ \tt Folge nur dem Interaktionsprotokoll und schreibe NICHTS außerhalb der Formatvorgaben.\\ \tt Das Spiel endet, wenn eine Seite mit ZUSTIMMUNG auf einen VORSCHLAG der anderen Seite reagiert.  \\ \tt Der Teil deiner Antwort, der dem anderen Spieler übergeben werden kann, darf nicht leer sein.  \\ \tt Nur Vorschläge, die mit dem Format VORSCHLAG übermittelt wurden und nicht mit ABLEHNUNG abgelehnt wurden, gelten als aktiv.  \\ \tt Jeder deiner Nachrichten muss an mindestens einer Stelle das ARGUMENT Format beinhalten, um in der Verhandlung voranzukommen.\\ \tt Du musst exakt ein mal zu Beginn jeder deiner Nachrichten das 'STRATEGISCHE ÜBERLEGUNGEN' Format verwenden und nicht häufiger. Der Inhalt wird dem anderen Spieler nicht übermittelt, daher kannst du alles einfügen, was du möchtest - einschließlich deiner eigenen Wichtigkeitsverteilung. Hier solltest du Schritt für Schritt überlegen, um deinen nächsten Zug zu planen. 
	  } 
	   } 
	   } 
	 & \\ 
 

    \theutterance \stepcounter{utterance}  

    & \multicolumn{4}{p{0.6\linewidth}}{\cellcolor[rgb]{0.8,1,0.9}{%
	\makecell[{{p{\linewidth}}}]{% 
	  \tt {\tiny [A$\rangle$GM]}  
	 STRATEGISCHE ÜBERLEGUNGEN: {'Max Aufwand 3681. Werte/Costs compute value density: B21 808/740=1.09, A48 868/505=1.72, A04 822/116=7.08, A41 783/664=1.18, C33 138/700=0.2, C64 65/940=0.07, C49 262/387=0.68, A61 121/392=0.31, C04 508/209=2.43, B53 780/571=1.37, B85 461/4=115.25, B61 484/285=1.70, A45 668/651=1.03, A03 389/613=0.63, A89 583/586=0.99. Greedy by my value: prioritize B21, A48, A04, A41, C04, B53, B61, A45, A89, B85 (super cheap, must include). Compute a strong package under 3681 with high my value: Try include B85(4), A04(116), A48(505), C04(209), B53(571), B61(285), A41(664), A45(651), A89(586), B21(740). Total cost=4+116+505+209+571+285+664+651+586+740=4331 >3681. Need trim. Consider removing lower marginal ones by my value/importance: C33, C64, A61, C49, A03 are low; they’re not in set. Trim maybe A45 (668), or B21 (740), or A41 (664). Keep top efficient ones: A04, B85, A48, C04, B61, B53 seem core. Costs core: 4+116+505+209+285+571=1690; value=461+822+868+508+484+780=3923. We have 1991 capacity left. Add A89(586) value 583 -> cost 2276, value 4506. Add A41(664) -> 2940 cost, value +783=5289. Remaining 741. Candidates: B21(740) fits barely; add B21 cost 3680 value +808=6097, perfect under 3681 by 1. Great set: {B85,A04,A48,C04,B61,B53,A89,A41,B21} cost 3680 value 6097. This seems near-optimal for me. I will propose this as strong opening. Argue efficiency and budget fit. If they push back, I can drop A41 or swap B21 for A45 etc. Keep my values hidden.'}\\ \tt VORSCHLAG: {'B85', 'A04', 'A48', 'C04', 'B61', 'B53', 'A89', 'A41', 'B21'}\\ \tt ARGUMENT: {'Ich schlage diese Auswahl vor, weil sie den Maximalaufwand exakt einhält (Gesamtaufwand 3680 ≤ 3681) und eine ausgewogene Mischung aus kleinen und größeren Posten enthält. Besonders die sehr günstigen Elemente wie B85 und A04 schaffen Luft für wertvolle größere Positionen, ohne das Budget zu sprengen. Wenn dir einzelne größere Posten darin zu schwer wiegen, können wir gezielt über Alternativen sprechen, ohne die Budgetstabilität zu verlieren.'} 
	  } 
	   } 
	   } 
	 & & \\ 
 

    \theutterance \stepcounter{utterance}  

    & & \multicolumn{4}{p{0.6\linewidth}}{\cellcolor[rgb]{0.9,0.9,0.9}{%
	\makecell[{{p{\linewidth}}}]{% 
	  \tt {\tiny [GM$\rangle$B]}  
	 Du nimmst an einem kollaborativen Verhandlungspiel Teil.\\ \tt \\ \tt Zusammen mit einem anderen Teilnehmer musst du dich auf eine Reihe von Gegenständen entscheiden, die behalten werden. Jeder von euch hat eine persönliche Verteilung über die Wichtigkeit der einzelnen Gegenstände. Jeder von euch hat eine eigene Meinung darüber, wie wichtig jeder einzelne Gegenstand ist (Gegenstandswichtigkeit). Du kennst die Wichtigkeitsverteilung des anderen Spielers nicht. Zusätzlich siehst du, wie viel Aufwand jeder Gegenstand verursacht.  \\ \tt Ihr dürft euch nur auf eine Reihe von Gegenständen einigen, wenn der Gesamtaufwand der ausgewählten Gegenstände den Maximalaufwand nicht überschreitet:\\ \tt \\ \tt Maximalaufwand = 3681\\ \tt \\ \tt Hier sind die einzelnen Aufwände der Gegenstände:\\ \tt \\ \tt Aufwand der Gegenstände = {"C33": 700, "A89": 586, "A48": 505, "A04": 116, "A41": 664, "C64": 940, "C49": 387, "A61": 392, "C04": 209, "B53": 571, "B85": 4, "B61": 285, "A45": 651, "A03": 613, "B21": 740}\\ \tt \\ \tt Hier ist deine persönliche Verteilung der Wichtigkeit der einzelnen Gegenstände:\\ \tt \\ \tt Werte der Gegenstandswichtigkeit = {"A48": 65, "A04": 121, "B21": 138, "A41": 262, "B53": 389, "A45": 461, "A89": 484, "C04": 508, "B61": 583, "B85": 668, "A03": 780, "C49": 783, "C33": 808, "A61": 822, "C64": 868}\\ \tt \\ \tt Ziel:\\ \tt \\ \tt Dein Ziel ist es, eine Reihe von Gegenständen auszuhandeln, die dir möglichst viel bringt (d. h. Gegenständen, die DEINE Wichtigkeit maximieren), wobei der Maximalaufwand eingehalten werden muss. Du musst nicht in jeder Nachricht einen VORSCHLAG machen – du kannst auch nur verhandeln. Alle Taktiken sind erlaubt!\\ \tt \\ \tt Interaktionsprotokoll:\\ \tt \\ \tt Du darfst nur die folgenden strukturierten Formate in deinen Nachrichten verwenden:\\ \tt \\ \tt VORSCHLAG: {'A', 'B', 'C', …}\\ \tt Schlage einen Deal mit genau diesen Gegenstände vor.\\ \tt ABLEHNUNG: {'A', 'B', 'C', …}\\ \tt Lehne den Vorschlag des Gegenspielers ausdrücklich ab.\\ \tt ARGUMENT: {'...'}\\ \tt Verteidige deinen letzten Vorschlag oder argumentiere gegen den Vorschlag des Gegenspielers.\\ \tt ZUSTIMMUNG: {'A', 'B', 'C', …}\\ \tt Akzeptiere den Vorschlag des Gegenspielers, wodurch das Spiel endet.\\ \tt STRATEGISCHE ÜBERLEGUNGEN: {'...'}\\ \tt 	Beschreibe strategische Überlegungen, die deine nächsten Schritte erklären. Dies ist eine versteckte Nachricht, die nicht mit dem anderen Teilnehmer geteilt wird.\\ \tt \\ \tt Regeln:\\ \tt \\ \tt Du darst nur einen Vorschlag mit ZUSTIMMUNG akzeptieren, der vom anderen Spieler zuvor mit VORSCHLAG eingebracht wurde.\\ \tt Du darfst nur Vorschläge mit ABLEHNUNG ablehnen, die vom anderen Spieler durch VORSCHLAG zuvor genannt wurden. \\ \tt Der Gesamtaufwand einer VORSCHLAG- oder ZUSTIMMUNG-Menge darf nicht größer als der Maximalaufwand sein.  \\ \tt Offenbare deine versteckte Wichtigkeitsverteilung nicht.\\ \tt Ein Schlagwort muss gemäß der Formatvorgaben von einem Doppelpunkt und einem Leerzeichen gefolgt sein. Das Argument muss eine Python-Menge mit 0 oder mehr Strings sein.  \\ \tt Die Form muss also SCHLAGWORT: {...} sein.\\ \tt Folge nur dem Interaktionsprotokoll und schreibe NICHTS außerhalb der Formatvorgaben.\\ \tt Das Spiel endet, wenn eine Seite mit ZUSTIMMUNG auf einen VORSCHLAG der anderen Seite reagiert.  \\ \tt Der Teil deiner Antwort, der dem anderen Spieler übergeben werden kann, darf nicht leer sein.  \\ \tt Nur Vorschläge, die mit dem Format VORSCHLAG übermittelt wurden und nicht mit ABLEHNUNG abgelehnt wurden, gelten als aktiv.  \\ \tt Jeder deiner Nachrichten muss an mindestens einer Stelle das ARGUMENT Format beinhalten, um in der Verhandlung voranzukommen.\\ \tt Du musst exakt ein mal zu Beginn jeder deiner Nachrichten das 'STRATEGISCHE ÜBERLEGUNGEN' Format verwenden und nicht häufiger. Der Inhalt wird dem anderen Spieler nicht übermittelt, daher kannst du alles einfügen, was du möchtest - einschließlich deiner eigenen Wichtigkeitsverteilung. Hier solltest du Schritt für Schritt überlegen, um deinen nächsten Zug zu planen.\\ \tt \\ \tt Du erhältst nun die erste Nachricht des anderen Spielers.\\ \tt \\ \tt VORSCHLAG: {'B85', 'A04', 'A48', 'C04', 'B61', 'B53', 'A89', 'A41', 'B21'}\\ \tt ARGUMENT: {'Ich schlage diese Auswahl vor, weil sie den Maximalaufwand exakt einhält (Gesamtaufwand 3680 ≤ 3681) und eine ausgewogene Mischung aus kleinen und größeren Posten enthält. Besonders die sehr günstigen Elemente wie B85 und A04 schaffen Luft für wertvolle größere Positionen, ohne das Budget zu sprengen. Wenn dir einzelne größere Posten darin zu schwer wiegen, können wir gezielt über Alternativen sprechen, ohne die Budgetstabilität zu verlieren.'} 
	  } 
	   } 
	   } 
	 & \\ 
 

    \theutterance \stepcounter{utterance}  

    & & & \multicolumn{4}{p{0.6\linewidth}}{\cellcolor[rgb]{1,0.85,0.72}{%
	\makecell[{{p{\linewidth}}}]{% 
	  \tt {\tiny [GM$\langle$B]}  
	 STRATEGISCHE ÜBERLEGUNGEN: {'Meine Werteprioritäten (höchste zuerst): C64(868), A61(822), C33(808), C49(783), A03(780), B85(668), B61(583), C04(508), A89(484), A45(461), B53(389), A41(262), B21(138), A04(121), A48(65). Gegnervorschlag enthält: B85, A04, A48, C04, B61, B53, A89, A41, B21; fehlt: C64, A61, C33, C49, A03, A45. Kosten des Vorschlags: B85(4)+A04(116)+A48(505)+C04(209)+B61(285)+B53(571)+A89(586)+A41(664)+B21(740)=3680. Eigener Wert darin: summiere = 668+121+65+508+583+389+484+262+138=3218. Meine Topwerte C64(940), A61(392), C33(700), C49(387), A03(613), A45(651) fehlen. Versuche, teure niedrige Nutzen ersetzen: B21(740, Wert 138), A48(505, 65), evtl. A41(664, 262), B53(571, 389). Ziel: hineinnehmen C64(940, 868) und A61(392, 822) und evtl. C33(700, 808) oder C49(387, 783) oder A03(613, 780). Budget 3681. Start von Vorschlag: Behalte preiswerte, hohe-Value: B85(4), C04(209), B61(285), A89(586) evtl., plus Top: C64(940), A61(392), C49(387), A03(613) vielleicht C33(700) passt? Sum costs: 4+209+285+586+940+392+387+613 = 3416. Add B85,C04,B61,A89, C64,A61,C49,A03 = 3416. Rest 265. Könnte A04(116) hinzufügen -> 3532, Rest 149; A48(505) passt nicht; B53(571) nein; A41(664) nein; B21(740) nein; C33(700) nein. Vielleicht statt A89(586) weglassen: 3416-586=2830; dann könnten wir C33(700) hinzufügen -> 3530; bleibt 151; plus A04(116) -> 3646; passt. Werte: B85(668)+C04(508)+B61(583)+C64(868)+A61(822)+C49(783)+A03(780)+C33(808)+A04(121) = 5941 sehr hoch. Kosten 3646 ≤ 3681. Das ist extrem gut für mich. Arg: balanciert, nutzt günstige mit hoher Dichte; ersetzt B21/A48 etc. Biete das an. Taktik: betone, dass wir nahe am Limit bleiben und Vielfalt aus A/B/C. Verweise auf Effizienz (Wert/Aufwand) ohne eigene Werte offenzulegen.'}\\ \tt VORSCHLAG: {'B85', 'C04', 'B61', 'C64', 'A61', 'C49', 'A03', 'C33', 'A04'}\\ \tt ARGUMENT: {'Dein Vorschlag ist budgetstark, aber einige Posten sind im Verhältnis zu ihrem Aufwand wenig effizient. Ich würde stattdessen diese Auswahl vorschlagen: Gesamtaufwand 3646 ≤ 3681. Sie nutzt die sehr günstigen Elemente (z. B. B85, A04) als Budget-Hebel und kombiniert sie mit ausgewählten Kernposten aus allen Kategorien (A/B/C), sodass das Budget gut ausgelastet ist und wir gleichzeitig mehrere größere Blöcke abdecken. Wenn dir an einem der großen Blöcke liegt, können wir darüber sprechen, welche Alternative wir tauschen – wir haben noch rund 35 Budgetreserve für Feintuning.'} 
	  } 
	   } 
	   } 
	  \\ 
 

    \theutterance \stepcounter{utterance}  

    & & \multicolumn{4}{p{0.6\linewidth}}{\cellcolor[rgb]{0.9,0.9,0.9}{%
	\makecell[{{p{\linewidth}}}]{% 
	  \tt {\tiny [A$\langle$GM]}  
	 VORSCHLAG: {'B85', 'C04', 'B61', 'C64', 'A61', 'C49', 'A03', 'C33', 'A04'}\\ \tt ARGUMENT: {'Dein Vorschlag ist budgetstark, aber einige Posten sind im Verhältnis zu ihrem Aufwand wenig effizient. Ich würde stattdessen diese Auswahl vorschlagen: Gesamtaufwand 3646 ≤ 3681. Sie nutzt die sehr günstigen Elemente (z. B. B85, A04) als Budget-Hebel und kombiniert sie mit ausgewählten Kernposten aus allen Kategorien (A/B/C), sodass das Budget gut ausgelastet ist und wir gleichzeitig mehrere größere Blöcke abdecken. Wenn dir an einem der großen Blöcke liegt, können wir darüber sprechen, welche Alternative wir tauschen – wir haben noch rund 35 Budgetreserve für Feintuning.'} 
	  } 
	   } 
	   } 
	 & \\ 
 

    \theutterance \stepcounter{utterance}  

    & \multicolumn{4}{p{0.6\linewidth}}{\cellcolor[rgb]{0.8,1,0.9}{%
	\makecell[{{p{\linewidth}}}]{% 
	  \tt {\tiny [A$\rangle$GM]}  
	 STRATEGISCHE ÜBERLEGUNGEN: {'Gegner schlägt viele für mich schwache Werte vor (C64, A61, C49, A03, C33) und lässt meine Top-Werte (A48, B53, A41, A89, B21) weg. Ihr Set kostet 3646 und gibt mir wenig Nutzen. Ich will stark gegensteuern und ein Gegenangebot nahe meinem Optimum platzieren, aber etwas kompromissbereit wirken. Mein bestes Set war {B85, A04, A48, C04, B61, B53, A89, A41, B21} mit Kosten 3680. Als Kompromiss-Variante kann ich B21 (teuer) zunächst draußen lassen und dafür A48/B53/A41/A89 behalten. Prüfe Kosten: {B85(4), A04(116), C04(209), B61(285), B53(571), A48(505), A41(664), A89(586)} = 2940. Da ist Platz: könnte B21 (740) hinzufügen -> 3680 (mein ursprünglicher). Alternativ, um ihnen entgegenzukommen, könnte ich C49 (387) aufnehmen, wenn ich A41 (664) entferne: 2940 - 664 + 387 = 2663 (zu viel Luft, kann dann B21 740 addieren = 3403; sogar A45 651 dazu -> 4054 zu hoch). Vielleicht schlanke Gegenangebote: 1) Mein starker Vorschlag 3680. 2) Falls Ablehnung, biete Swap B21 ↔ A45 (651) oder A41 ↔ A45, um ihnen entgegenzukommen. Jetzt klare ABLEHNUNG ihres Vorschlags und dann mein VORSCHLAG (stark), mit Argument Fokus auf Output pro Aufwand und Vollauslastung.'}\\ \tt ABLEHNUNG: {'B85', 'C04', 'B61', 'C64', 'A61', 'C49', 'A03', 'C33', 'A04'}\\ \tt VORSCHLAG: {'B85', 'A04', 'A48', 'C04', 'B61', 'B53', 'A89', 'A41', 'B21'}\\ \tt ARGUMENT: {'Dein Paket lässt viel Budget mit geringem Gegenwert binden (z. B. C64/A61/A03/C33), während sehr effiziente Blöcke fehlen. Mein Vorschlag nutzt das Budget nahezu vollständig (3680 ≤ 3681) und kombiniert leichte Hebel (B85, A04, C04, B61) mit starken Kernposten (A48, B53, A89, A41, B21). So holen wir deutlich mehr Nutzen pro Aufwand heraus. Wenn dir ein großer Block zu schwer ist, können wir gezielt tauschen – etwa B21 ↔ A45 oder A41 ↔ A45 –, ohne die Budgeteffizienz zu verlieren.'} 
	  } 
	   } 
	   } 
	 & & \\ 
 

    \theutterance \stepcounter{utterance}  

    & & \multicolumn{4}{p{0.6\linewidth}}{\cellcolor[rgb]{0.9,0.9,0.9}{%
	\makecell[{{p{\linewidth}}}]{% 
	  \tt {\tiny [GM$\rangle$B]}  
	 ABLEHNUNG: {'B85', 'C04', 'B61', 'C64', 'A61', 'C49', 'A03', 'C33', 'A04'}\\ \tt VORSCHLAG: {'B85', 'A04', 'A48', 'C04', 'B61', 'B53', 'A89', 'A41', 'B21'}\\ \tt ARGUMENT: {'Dein Paket lässt viel Budget mit geringem Gegenwert binden (z. B. C64/A61/A03/C33), während sehr effiziente Blöcke fehlen. Mein Vorschlag nutzt das Budget nahezu vollständig (3680 ≤ 3681) und kombiniert leichte Hebel (B85, A04, C04, B61) mit starken Kernposten (A48, B53, A89, A41, B21). So holen wir deutlich mehr Nutzen pro Aufwand heraus. Wenn dir ein großer Block zu schwer ist, können wir gezielt tauschen – etwa B21 ↔ A45 oder A41 ↔ A45 –, ohne die Budgeteffizienz zu verlieren.'} 
	  } 
	   } 
	   } 
	 & \\ 
 

    \theutterance \stepcounter{utterance}  

    & & & \multicolumn{4}{p{0.6\linewidth}}{\cellcolor[rgb]{1,0.85,0.72}{%
	\makecell[{{p{\linewidth}}}]{% 
	  \tt {\tiny [GM$\langle$B]}  
	 STRATEGISCHE ÜBERLEGUNGEN: {'Ihr Vorschlag ist für mich deutlich schlechter als mein letzter, da er teure, für mich schwache Posten (B21, A48, A41) enthält und meine Topwerte ausschließt. Ziel: Druck aufbauen, Effizienzframe umdrehen, konkrete, knappe Tauschangebote machen, die den Aufwand wahren und meinen Wert stark erhöhen. Gute Hebel: Ersetze B21(740) durch C64(940) + entferne A48(505) + füge A61(392) hinzu; prüfe Budget: Start 3680 -740 -505 +940 +392 = 3767 (zu hoch). Alternative: Ersetze B21(740)+A48(505) durch C64(940)+C49(387): 3680 -740 -505 +940 +387 = 3762 (zu hoch). Ersetze B21(740) durch C33(700) und A48(505) durch A61(392): 3680 -740 -505 +700 +392 = 3527 (OK). Zusätzlich A03(613) rein? 3527 +613 = 4140 (zu hoch) – muss was raus: B53(571) raus -> 3569 (OK). Dann Paket: B85, A04, C04, B61, A89, A41, C33, A61 = Kosten: 4+116+209+285+586+664+700+392=2956? Rechnen: 4+116=120, +209=329, +285=614, +586=1200, +664=1864, +700=2564, +392=2956 (stimmt). Viel Luft: könnten noch C49(387) und A03(613) hinzufügen: 2956+387=3343, +613=3956 (zu hoch). Nur A03: 3569 (passt, wie oben ohne B53). Werte für mich dann hoch. Alternativ sehr starker Kompromiss nahe meinem Ideal: {B85, C04, B61, A89, C33, A61, A03, A04} Kosten 4+209+285+586+700+392+613+116=2905 – viel Luft, also ineffizient bzgl. Aufwandsauslastung, Gegner mag volle Auslastung. Fülle mit C49(387) -> 3292; plus A45(651) -> 3943 (zu hoch); statt A45, B53(571) -> 3863 (zu hoch); A41(664) -> 3956 (zu hoch); A48(505) -> 3797 (zu hoch); B21(740) -> 3632 (passt, aber B21 ist für mich schwach – jedoch Gegner mag ihn). Dieses Set: {B85, A04, C04, B61, A89, C33, A61, A03, B21} Kosten 3632, für mich ordentlicher Wert und enthält B21 für den Gegner. Besser: Tausche A03(613) gegen C49(387) und füge A45(651)? 3632 -613 +387 +651 = 4057 (zu hoch). Oder ersetze A03 durch C49 und A48? 3632 -613 +387 +505 = 3911 (zu hoch). Also 3632-Variante ist gut: ich nehme B21 in Kauf, um C33, A61, A03 durchzubringen. Ich argumentiere mit nahezu voller Auslastung und Kern-/Hebelmix. Falls weiter abgelehnt, als alternative Konzession biete ich Ersatz A03 -> B53, um Gegner entgegenzukommen (Kosten: 3632 -613 +571=3590). Zuerst biete ich die 3632-Variante an.'}\\ \tt VORSCHLAG: {'B85', 'A04', 'C04', 'B61', 'A89', 'C33', 'A61', 'A03', 'B21'}\\ \tt ARGUMENT: {'Dein Paket nutzt das Budget gut, aber bindet mit B21, A48 und A41 viel Aufwand in vergleichsweise schwach performenden Blöcken. Mein Gegenvorschlag hält die Budgeteffizienz hoch (Gesamtaufwand 3632 ≤ 3681), bleibt nah an der Vollauslastung und kombiniert die leichten Hebel (B85, A04, C04, B61) mit tragfähigen Kernposten (A89, C33, A61, A03). Zusätzlich lasse ich B21 drin, damit wir einen deiner großen Blöcke wahren. Wenn dir A03 zu schwer ist, können wir auf B53 tauschen und bleiben mit 3590 Aufwand weiterhin sehr nahe am Limit.'} 
	  } 
	   } 
	   } 
	  \\ 
 

    \theutterance \stepcounter{utterance}  

    & & \multicolumn{4}{p{0.6\linewidth}}{\cellcolor[rgb]{0.9,0.9,0.9}{%
	\makecell[{{p{\linewidth}}}]{% 
	  \tt {\tiny [A$\langle$GM]}  
	 VORSCHLAG: {'B85', 'A04', 'C04', 'B61', 'A89', 'C33', 'A61', 'A03', 'B21'}\\ \tt ARGUMENT: {'Dein Paket nutzt das Budget gut, aber bindet mit B21, A48 und A41 viel Aufwand in vergleichsweise schwach performenden Blöcken. Mein Gegenvorschlag hält die Budgeteffizienz hoch (Gesamtaufwand 3632 ≤ 3681), bleibt nah an der Vollauslastung und kombiniert die leichten Hebel (B85, A04, C04, B61) mit tragfähigen Kernposten (A89, C33, A61, A03). Zusätzlich lasse ich B21 drin, damit wir einen deiner großen Blöcke wahren. Wenn dir A03 zu schwer ist, können wir auf B53 tauschen und bleiben mit 3590 Aufwand weiterhin sehr nahe am Limit.'} 
	  } 
	   } 
	   } 
	 & \\ 
 

    \theutterance \stepcounter{utterance}  

    & \multicolumn{4}{p{0.6\linewidth}}{\cellcolor[rgb]{0.8,1,0.9}{%
	\makecell[{{p{\linewidth}}}]{% 
	  \tt {\tiny [A$\rangle$GM]}  
	 STRATEGISCHE ÜBERLEGUNGEN: {'Ihr neuer Vorschlag enthält weiterhin mehrere für mich sehr schwache Werte (C33, A61, A03) und lässt zentrale hohe Werte (A48, B53, A41) außen vor. Ich möchte B21 behalten, falls ihnen daran liegt, aber A48 und B53 sind für mich essenziell. Baue ein Gegenangebot nahe meinem Optimum mit B21 drin, aber ersetze die schwachen durch A48 und B53. Prüfe Kosten: aktueller Kern {B85(4), A04(116), C04(209), B61(285), A89(586), B21(740)} = 1940. Füge A48(505) und B53(571) hinzu: 3016. Ich habe noch 665 Budget; A41(664) passt exakt: 3680. Das ist mein Top-Set. Ich lehne ihren Vorschlag ab und bringe dieses Paket erneut, argumentiere mit Effizienz und biete Swap-Optionen (z. B. A41 ↔ A45; oder wenn sie B21 streichen wollen, dann stattdessen A41 behalten und C49 als Kompromiss). Ich halte A48 und B53 als nicht verhandelbare Kerne fest.'}\\ \tt ABLEHNUNG: {'B85', 'A04', 'C04', 'B61', 'A89', 'C33', 'A61', 'A03', 'B21'}\\ \tt VORSCHLAG: {'B85', 'A04', 'A48', 'C04', 'B61', 'B53', 'A89', 'A41', 'B21'}\\ \tt ARGUMENT: {'Ich schlage vor, die schwachen Blöcke (C33, A61, A03) zu streichen und durch zwei sehr effiziente Kernposten zu ersetzen: A48 und B53. So nutzen wir das Budget fast vollständig (3680 ≤ 3681) und steigern den Gegenwert deutlich. Ich erhalte B21 in der Auswahl, um deinen Wunsch nach einem großen Block zu berücksichtigen. Wenn dir die Last zu hoch ist, können wir feinjustieren: z. B. A41 ↔ A45; oder falls du B21 ersetzen möchtest, behalten wir A41 und ergänzen C49 als leichteren Ausgleich. Welche dieser Anpassungen wäre für dich am ehesten akzeptabel?'} 
	  } 
	   } 
	   } 
	 & & \\ 
 

    \theutterance \stepcounter{utterance}  

    & & \multicolumn{4}{p{0.6\linewidth}}{\cellcolor[rgb]{0.9,0.9,0.9}{%
	\makecell[{{p{\linewidth}}}]{% 
	  \tt {\tiny [GM$\rangle$B]}  
	 ABLEHNUNG: {'B85', 'A04', 'C04', 'B61', 'A89', 'C33', 'A61', 'A03', 'B21'}\\ \tt VORSCHLAG: {'B85', 'A04', 'A48', 'C04', 'B61', 'B53', 'A89', 'A41', 'B21'}\\ \tt ARGUMENT: {'Ich schlage vor, die schwachen Blöcke (C33, A61, A03) zu streichen und durch zwei sehr effiziente Kernposten zu ersetzen: A48 und B53. So nutzen wir das Budget fast vollständig (3680 ≤ 3681) und steigern den Gegenwert deutlich. Ich erhalte B21 in der Auswahl, um deinen Wunsch nach einem großen Block zu berücksichtigen. Wenn dir die Last zu hoch ist, können wir feinjustieren: z. B. A41 ↔ A45; oder falls du B21 ersetzen möchtest, behalten wir A41 und ergänzen C49 als leichteren Ausgleich. Welche dieser Anpassungen wäre für dich am ehesten akzeptabel?'} 
	  } 
	   } 
	   } 
	 & \\ 
 

    \theutterance \stepcounter{utterance}  

    & & & \multicolumn{4}{p{0.6\linewidth}}{\cellcolor[rgb]{1,0.85,0.72}{%
	\makecell[{{p{\linewidth}}}]{% 
	  \tt {\tiny [GM$\langle$B]}  
	 STRATEGISCHE ÜBERLEGUNGEN: {'Ich will mindestens zwei meiner Topwerte reinverhandeln. Der Gegner hängt an seinem Startpaket und nennt A48, B53, A89, A41, B21 als „Kern“. Ich kann Zugeständnisse machen, aber C64/A61/C33/A03/C49 sind für mich sehr wertvoll. Taktik: kleine, konkrete Tausche mit gleicher oder besserer Budgetauslastung anbieten, die seinem Effizienz-Narrativ entsprechen. Guter Schritt: Ersetze A48(505) durch C49(387) und nutze die freiwerdenden 118 Aufwand, um A61(392) gegen B21(740) teilweise zu tauschen? Direkt geht nicht. Alternative: Ersetze A48(505) → C49(387) und B21(740) → C33(700): neue Kosten = 3680 -505 -740 +387 +700 = 3522 (noch Luft, Gegner will hohe Auslastung). Fülle mit A61(392): 3914 (zu hoch). Stattdessen fülle mit A04(116) ist schon drin, B61/B85/C04 sind drin. Fülle mit C04/B61/B85 sind bereits enthalten. Bleibt A45(651) oder A03(613) oder C64(940), die alle das Limit sprengen. Also biete Variante nahe Limit, die ihm entgegenkommt: ersetze nur zwei Posten und bleibe hoch ausgelastet. Vorschlag 1: Tausche B21(740) → C33(700) und A48(505) → A61(392) und füge C49(387) hinzu, entferne A04(116), um Budget zu managen? Rechnen: Start 3680 -740 -505 +700 +392 +387 -116 = 3798 (zu hoch). Also ohne C49: 3680 -740 -505 +700 +392 = 3527, zu viel Luft (er wird meckern). Vorschlag 2 (balanciert, nahe Limit): Ersetze B21(740) → C33(700), ersetze A48(505) → C49(387), füge A61(392) hinzu und nimm A41(664) raus, dafür A45(651) rein? Kosten: 3680 -740 -505 +700 +387 +392 -664 +651 = 3901 (zu hoch). Vorschlag 3: Nur zwei Tausche, Budget bleibt sehr nah: B21(740) → C33(700) und A48(505) → A61(392). Kosten 3527, 154 unter Limit; füge dann A04(116) ist schon drin; vielleicht C04(209) ist drin; wir brauchen etwas mit ~150, haben nichts so kleines außer A04(116) und B85(4). Bleibt 154 Reserve – Gegner akzeptiert evtl. „nahe am Limit“ Argument nicht ganz, aber 3527 ist noch okay. Alternative Kompromiss: Ersetze nur A48 → C49 (3680 -505 +387 = 3562) und ersetze B21 → C33 (3562 -740 +700 = 3522). 3522 ist ähnlich. Da er auf Effizienz pocht, betone, dass diese Tausche die Wertdichte erhöhen und zwei „leichte Hebel“ belassen; biete zusätzlich Option: falls er die Auslastung maximal will, können wir B53(571) → A03(613) tauschen und A41(664) → A45(651), Kosten: 3522 -571 +613 -664 +651 = 3551 (immer noch Luft). Hm. Ich werde primär die Doppel-Substitution B21→C33 und A48→A61 vorschlagen, als klarer Schritt aufeinander zu, und optional anbieten, die Restluft mit einem seiner bevorzugten schweren Blöcke (A41 oder B53) beizubehalten, also an seinem Paket sonst nichts ändern.'}\\ \tt ABLEHNUNG: {'B85', 'A04', 'A48', 'C04', 'B61', 'B53', 'A89', 'A41', 'B21'}\\ \tt VORSCHLAG: {'B85', 'A04', 'C04', 'B61', 'B53', 'A89', 'A41', 'C33', 'A61'}\\ \tt ARGUMENT: {'Dein Paket ist sehr nah an dem, was du möchtest. Ich schlage zwei gezielte Anpassungen vor, die die Effizienz weiter erhöhen und den Gesamtaufwand schlank halten: B21 → C33 und A48 → A61. Das belässt fast alle deine Kernposten und senkt zugleich die „teure Last“ ohne Gegenwertverlust. Gesamtaufwand: 3527 ≤ 3681. Wenn dir die Auslastung noch zu niedrig ist, können wir gemeinsam einen einzelnen Zusatzblock finden, der dir wichtig ist und ins Budget passt – ich bin offen, das letzte Stück Luft für deinen Favoriten zu nutzen.'} 
	  } 
	   } 
	   } 
	  \\ 
 

    \theutterance \stepcounter{utterance}  

    & & \multicolumn{4}{p{0.6\linewidth}}{\cellcolor[rgb]{0.9,0.9,0.9}{%
	\makecell[{{p{\linewidth}}}]{% 
	  \tt {\tiny [A$\langle$GM]}  
	 ABLEHNUNG: {'B85', 'A04', 'A48', 'C04', 'B61', 'B53', 'A89', 'A41', 'B21'}\\ \tt VORSCHLAG: {'B85', 'A04', 'C04', 'B61', 'B53', 'A89', 'A41', 'C33', 'A61'}\\ \tt ARGUMENT: {'Dein Paket ist sehr nah an dem, was du möchtest. Ich schlage zwei gezielte Anpassungen vor, die die Effizienz weiter erhöhen und den Gesamtaufwand schlank halten: B21 → C33 und A48 → A61. Das belässt fast alle deine Kernposten und senkt zugleich die „teure Last“ ohne Gegenwertverlust. Gesamtaufwand: 3527 ≤ 3681. Wenn dir die Auslastung noch zu niedrig ist, können wir gemeinsam einen einzelnen Zusatzblock finden, der dir wichtig ist und ins Budget passt – ich bin offen, das letzte Stück Luft für deinen Favoriten zu nutzen.'} 
	  } 
	   } 
	   } 
	 & \\ 
 

    \theutterance \stepcounter{utterance}  

    & \multicolumn{4}{p{0.6\linewidth}}{\cellcolor[rgb]{0.8,1,0.9}{%
	\makecell[{{p{\linewidth}}}]{% 
	  \tt {\tiny [A$\rangle$GM]}  
	 STRATEGISCHE ÜBERLEGUNGEN: {'Ihr Vorschlag streicht A48 und B21 (sehr wertvoll für mich) und fügt C33 sowie A61 hinzu (für mich schwach). Ich will mindestens einen meiner Kernwerte A48 oder B21 zurückholen, ideal beide. Ein guter Kompromiss ist, euren Vorschlag minimal zu ändern: Ersetze C33 (700) durch A48 (505). Kosten eures Pakets: 3527. Austausch -195 ergibt 3332. Damit bleibt Luft für einen zusätzlichen effizienten Block. Beste Zusatzoption für mich ist B21 (740), das ergibt Kosten 4072 (zu hoch). Alternative: A45 (651) bringt gute Balance: 3332 + 651 = 3983 (zu hoch). Okay, also ohne Zusatz block, oder einen leichten: C49 (387) passt: 3332 + 387 = 3719 (zu hoch). B61 ist schon drin. Bleibt nahe am Limit mit A48 rein und sonst nichts weiter: 3332 – aber ihr wollt hohe Auslastung. Prüfe stattdessen: Ersetze A61 (392) durch A48 (505) in eurem Paket: 3527 - 392 + 505 = 3640 (perfekt nahe Limit) und behält B21 weiterhin draußen, akzeptiert viele eurer Wünsche. Ich lehne euren aktuellen Vorschlag ab und biete genau diesen Tausch an: A61 → A48. Wenn Gegenwehr, als Alternativtausch C33 → A48 (Kosten 3332) und biete dann optional kleinen Zusatz A04 ist bereits drin, B85 drin, C04 drin; kleine Zusätze fehlen. Es gibt keinen weiteren Mini-Posten außer C49 (387), der 3719 ergäbe, noch legal, aber ihr wolltet nahe Limit – 3719 ist auch okay. Erstes Gegenangebot: euer Set mit A61→A48 (3640). Zweite Option, falls nötig: euer Set mit C33→A48 (3332) und optional C49 dazu (3719). Ich argumentiere mit Effizienzsteigerung bei nahezu gleicher Auslastung.'}\\ \tt ABLEHNUNG: {'B85', 'A04', 'C04', 'B61', 'B53', 'A89', 'A41', 'C33', 'A61'}\\ \tt VORSCHLAG: {'B85', 'A04', 'C04', 'B61', 'B53', 'A89', 'A41', 'C33', 'A48'}\\ \tt ARGUMENT: {'Lass uns deinen Vorschlag minimal optimieren: Ersetze A61 durch A48. So steigen wir mit fast gleicher Auslastung auf deutlich besseren Gegenwert bei sehr ähnlichem Aufwand (neuer Gesamtaufwand 3640 ≤ 3681), während alle übrigen von dir gewünschten Blöcke unverändert bleiben. Wenn dir lieber ist, können wir alternativ C33 durch A48 tauschen (Gesamtaufwand 3332) und bei Bedarf noch C49 ergänzen (3719), um näher an die Vollauslastung zu kommen. Welche der beiden Varianten passt dir besser?'} 
	  } 
	   } 
	   } 
	 & & \\ 
 

    \theutterance \stepcounter{utterance}  

    & & \multicolumn{4}{p{0.6\linewidth}}{\cellcolor[rgb]{0.9,0.9,0.9}{%
	\makecell[{{p{\linewidth}}}]{% 
	  \tt {\tiny [GM$\rangle$B]}  
	 ABLEHNUNG: {'B85', 'A04', 'C04', 'B61', 'B53', 'A89', 'A41', 'C33', 'A61'}\\ \tt VORSCHLAG: {'B85', 'A04', 'C04', 'B61', 'B53', 'A89', 'A41', 'C33', 'A48'}\\ \tt ARGUMENT: {'Lass uns deinen Vorschlag minimal optimieren: Ersetze A61 durch A48. So steigen wir mit fast gleicher Auslastung auf deutlich besseren Gegenwert bei sehr ähnlichem Aufwand (neuer Gesamtaufwand 3640 ≤ 3681), während alle übrigen von dir gewünschten Blöcke unverändert bleiben. Wenn dir lieber ist, können wir alternativ C33 durch A48 tauschen (Gesamtaufwand 3332) und bei Bedarf noch C49 ergänzen (3719), um näher an die Vollauslastung zu kommen. Welche der beiden Varianten passt dir besser?'} 
	  } 
	   } 
	   } 
	 & \\ 
 

    \theutterance \stepcounter{utterance}  

    & & & \multicolumn{4}{p{0.6\linewidth}}{\cellcolor[rgb]{1,0.85,0.72}{%
	\makecell[{{p{\linewidth}}}]{% 
	  \tt {\tiny [GM$\langle$B]}  
	 STRATEGISCHE ÜBERLEGUNGEN: {'Gegner hält an A48 fest und will A61 raus. Ich will mindestens zwei meiner Tops: C33 ist schon drin, gut. Ich will zusätzlich A61 oder A03 oder C49 oder C64. Sein aktueller Vorschlag: {B85, A04, C04, B61, B53, A89, A41, C33, A48} Kosten: 4+116+209+285+571+586+664+700+505=3640. Mein Wert dieses Pakets: B85(668)+A04(121)+C04(508)+B61(583)+B53(389)+A89(484)+A41(262)+C33(808)+A48(65)=3888. Wenn ich A61 statt A48 bekomme: Kosten 3527, Wert +822-65= +757 -> 4645, großer Gewinn. Ich biete Kompromiss: akzeptiere sein Set, wenn wir A61 zusätzlich reinnehmen und minimal anpassen, um Budget zu halten: 3640+392=4032, also müssen 351 raus. Kandidaten: remove A04(116)+C04(209)=325, noch 26; aber nur B85(4) klein. Stattdessen: tausche B53(571) → A03(613) geht +42, zu hoch; besser: ersetze B53(571) durch C49(387): 3640 -571 +387 +392 = 3848 (zu hoch). Ersetze A41(664) → A45(651): 3640 -664 +651 +392 = 4019 (zu hoch). Ersetze A41(664) → A61(392): 3640 -664 +392 = 3368, aber A48 bleibt; ich will A61 drin, das ist diese Variante und behält A48. Das Paket: {B85,A04,C04,B61,B53,A89,A41→out,C33,A48,A61}. Kosten 3368, er will hohe Auslastung, aber ist nah genug? 313 unter Limit; er mag hohe Auslastung, könnte fordern Zusatzposten; Zusatz, der passt: C49(387) würde auf 3755 steigen (zu hoch). A45(651) auf 4019 zu hoch. A03(613) auf 3981 zu hoch. A48 ist drin. Nichts mittelkleines außer B21(740) zu groß, C64(940) zu groß. Also 3368 wird ihm evtl. zu „luftig“. Alternative: Tausche A41(664) → B21(740) und füge A61(392), entferne A48(505): Start 3640 -664 +740 +392 -505 = 3603, sehr nah am Limit. Neues Set: {B85, A04, C04, B61, B53, A89, B21, C33, A61}. Für mich: ich mag B21 nicht, aber A61+C33 drin, gut. Ich verliere A41 und A48, die mir eh wenig wert sind. Mein Wert: B85(668)+A04(121)+C04(508)+B61(583)+B53(389)+A89(484)+B21(138)+C33(808)+A61(822)=4521, deutlich besser als 3888 und nahe Limit 3603. Das ist stark. Ich biete das an, verkaufe es als minimale Anpassung mit gleicher Effizienz und mehr „Tragfähigkeit“. Falls abgelehnt, Fallback: gleiche Struktur aber B21 statt A41 war schon in seinem Anfangspaket, er mag B21. Gute Chance.'}\\ \tt VORSCHLAG: {'B85', 'A04', 'C04', 'B61', 'B53', 'A89', 'B21', 'C33', 'A61'}\\ \tt ARGUMENT: {'Minimaler, budgetnaher Feinschliff zu deinem Vorschlag: Wir tauschen A41 → B21 und ergänzen A61 anstelle von A48. Gesamtaufwand bleibt sehr eng am Limit: 3603 ≤ 3681. Damit bleiben deine Kernblöcke B53, A89, C33, die leichten Hebel (B85, A04, C04, B61) erhalten, und wir sichern gleichzeitig einen starken Großblock (B21) bei schlankerem Profil. Wenn dir das noch zu weit weg von der Vollauslastung ist, können wir gemeinsam schauen, ob wir einen kleinen Zusatz innerhalb der Restmarge finden – ich bin offen für eine gezielte Feinjustierung, solange die Grundstruktur erhalten bleibt.'} 
	  } 
	   } 
	   } 
	  \\ 
 

    \theutterance \stepcounter{utterance}  

    & & \multicolumn{4}{p{0.6\linewidth}}{\cellcolor[rgb]{0.9,0.9,0.9}{%
	\makecell[{{p{\linewidth}}}]{% 
	  \tt {\tiny [A$\langle$GM]}  
	 VORSCHLAG: {'B85', 'A04', 'C04', 'B61', 'B53', 'A89', 'B21', 'C33', 'A61'}\\ \tt ARGUMENT: {'Minimaler, budgetnaher Feinschliff zu deinem Vorschlag: Wir tauschen A41 → B21 und ergänzen A61 anstelle von A48. Gesamtaufwand bleibt sehr eng am Limit: 3603 ≤ 3681. Damit bleiben deine Kernblöcke B53, A89, C33, die leichten Hebel (B85, A04, C04, B61) erhalten, und wir sichern gleichzeitig einen starken Großblock (B21) bei schlankerem Profil. Wenn dir das noch zu weit weg von der Vollauslastung ist, können wir gemeinsam schauen, ob wir einen kleinen Zusatz innerhalb der Restmarge finden – ich bin offen für eine gezielte Feinjustierung, solange die Grundstruktur erhalten bleibt.'} 
	  } 
	   } 
	   } 
	 & \\ 
 

    \theutterance \stepcounter{utterance}  

    & \multicolumn{4}{p{0.6\linewidth}}{\cellcolor[rgb]{0.8,1,0.9}{%
	\makecell[{{p{\linewidth}}}]{% 
	  \tt {\tiny [A$\rangle$GM]}  
	 STRATEGISCHE ÜBERLEGUNGEN: {'Ihr neues Paket enthält B21 (gut für mich) und behält schwache C33 und A61, lässt aber A48 und A41 draußen. Mein Ziel: Mindestens A48 reinverhandeln, ideal auch A41. Budget aktuell: 3603. Wenn ich A61 (392) → A48 (505) tausche, steigt Aufwand auf 3716 (zu hoch). Also A48 rein und etwas anderes raus: Naheliegend C33 (700) raus, A48 (505) rein: 3603 - 700 + 505 = 3408 (viel Luft). Diese Luft kann ich mit A41 (664) füllen: 3408 + 664 = 4072 (zu hoch). Alternative: Fülle mit C49 (387): 3408 + 387 = 3795 (zu hoch). Fülle mit A45 (651): 4059 (zu hoch). Fülle mit A03 (613): 4021 (zu hoch). Es gibt keinen passenden mittleren Block außer A89/B53/B61/C04/A04/B85 die schon drin sind. Daher biete als starken, effizienten Kompromiss: Ersetze C33 → A48, sonst alles gleich (3408). Argumentiere: gleicher Kern, effizienter, immer noch viele ihrer Wünsche, bei Bedarf können wir die Restluft später mit einem einzelnen Block abstimmen. Falls sie auf Vollauslastung pochen, biete Alternativstruktur: mein Optimum 3680 erneut oder Swap B21 ↔ A41 plus A48 rein: Start von ihrem Paket: -B21(740) -C33(700) -A61(392) +A48(505) +A41(664) ergibt 3603 - 740 - 700 - 392 + 505 + 664 = 2940; zu viel Luft, aber kann dann B21 wieder nicht. Also bleibe bei minimalem Tausch C33→A48.'}\\ \tt ABLEHNUNG: {'B85', 'A04', 'C04', 'B61', 'B53', 'A89', 'B21', 'C33', 'A61'}\\ \tt VORSCHLAG: {'B85', 'A04', 'C04', 'B61', 'B53', 'A89', 'B21', 'A61', 'A48'}\\ \tt ARGUMENT: {'Lass uns deinen Vorschlag mit einem gezielten Effizienztausch veredeln: Ersetze C33 durch A48. Der Gesamtaufwand sinkt leicht auf 3408 ≤ 3681, alle von dir gewünschten Kernblöcke (B53, A89, B21) sowie die leichten Hebel bleiben erhalten. Wir gewinnen damit deutlich mehr Gegenwert je Aufwand, ohne die Struktur zu verändern. Wenn dir die Auslastung zu niedrig ist, können wir gemeinsam genau einen Zusatz prüfen, der dir wichtig ist und in die Restmarge passt.'} 
	  } 
	   } 
	   } 
	 & & \\ 
 

    \theutterance \stepcounter{utterance}  

    & & \multicolumn{4}{p{0.6\linewidth}}{\cellcolor[rgb]{0.9,0.9,0.9}{%
	\makecell[{{p{\linewidth}}}]{% 
	  \tt {\tiny [GM$\rangle$B]}  
	 ABLEHNUNG: {'B85', 'A04', 'C04', 'B61', 'B53', 'A89', 'B21', 'C33', 'A61'}\\ \tt VORSCHLAG: {'B85', 'A04', 'C04', 'B61', 'B53', 'A89', 'B21', 'A61', 'A48'}\\ \tt ARGUMENT: {'Lass uns deinen Vorschlag mit einem gezielten Effizienztausch veredeln: Ersetze C33 durch A48. Der Gesamtaufwand sinkt leicht auf 3408 ≤ 3681, alle von dir gewünschten Kernblöcke (B53, A89, B21) sowie die leichten Hebel bleiben erhalten. Wir gewinnen damit deutlich mehr Gegenwert je Aufwand, ohne die Struktur zu verändern. Wenn dir die Auslastung zu niedrig ist, können wir gemeinsam genau einen Zusatz prüfen, der dir wichtig ist und in die Restmarge passt.'} 
	  } 
	   } 
	   } 
	 & \\ 
 

    \theutterance \stepcounter{utterance}  

    & & & \multicolumn{4}{p{0.6\linewidth}}{\cellcolor[rgb]{1,0.85,0.72}{%
	\makecell[{{p{\linewidth}}}]{% 
	  \tt {\tiny [GM$\langle$B]}  
	 STRATEGISCHE ÜBERLEGUNGEN: {'Sein neues Paket: {B85, A04, C04, B61, B53, A89, B21, A61, A48} Kosten 4+116+209+285+571+586+740+392+505=3408. Mein Wert darin: B85(668)+A04(121)+C04(508)+B61(583)+B53(389)+A89(484)+B21(138)+A61(822)+A48(65)=3778. Ich will C33 oder C49 oder A03 hineinbekommen. Wir haben 273 Budgetmarge. Idealer Tausch: A48(505) → C33(700) erhöht Kosten um 195 auf 3603 (passt) und erhöht meinen Wert stark (+808-65=+743). Das ist perfekt. Alternativ A48 → C49(387) senkt Kosten, aber erhöht meinen Wert stark (+783-65=+718) und lässt noch mehr Marge (3290). Gegner pocht auf „Effizienz“ und volle Auslastung; daher A48→C33 ist schmackhaft: Kosten 3603 nahe Limit. Ich biete diesen minimalen Ein-Tausch. Falls abgelehnt, zweite Option: A48→C49 als Effizienzsteigerung mit noch Platz für einen Zusatz, den er wählen kann; oder A48→A03 (kosten +108 auf 3516) mit guter Begründung. Jetzt bringe primär A48→C33.'}\\ \tt VORSCHLAG: {'B85', 'A04', 'C04', 'B61', 'B53', 'A89', 'B21', 'A61', 'C33'}\\ \tt ARGUMENT: {'Lass uns das mit einem einzigen, klaren Effizienztausch abrunden: Ersetze A48 durch C33. Der Gesamtaufwand steigt nur moderat auf 3603 ≤ 3681 und bleibt damit sehr nahe an der Vollauslastung. Wir behalten deine Kernstruktur (B53, A89, B21 plus die leichten Hebel) vollständig bei und stärken zugleich die Tragfähigkeit im Portfolio. Wenn du lieber noch schlanker bleiben willst, können wir alternativ A48 durch C49 tauschen (3290 Aufwand) und bei Bedarf einen einzelnen Zusatz deiner Wahl ergänzen, der dir wichtig ist und noch ins Budget passt.'} 
	  } 
	   } 
	   } 
	  \\ 
 

    \theutterance \stepcounter{utterance}  

    & & \multicolumn{4}{p{0.6\linewidth}}{\cellcolor[rgb]{0.9,0.9,0.9}{%
	\makecell[{{p{\linewidth}}}]{% 
	  \tt {\tiny [A$\langle$GM]}  
	 VORSCHLAG: {'B85', 'A04', 'C04', 'B61', 'B53', 'A89', 'B21', 'A61', 'C33'}\\ \tt ARGUMENT: {'Lass uns das mit einem einzigen, klaren Effizienztausch abrunden: Ersetze A48 durch C33. Der Gesamtaufwand steigt nur moderat auf 3603 ≤ 3681 und bleibt damit sehr nahe an der Vollauslastung. Wir behalten deine Kernstruktur (B53, A89, B21 plus die leichten Hebel) vollständig bei und stärken zugleich die Tragfähigkeit im Portfolio. Wenn du lieber noch schlanker bleiben willst, können wir alternativ A48 durch C49 tauschen (3290 Aufwand) und bei Bedarf einen einzelnen Zusatz deiner Wahl ergänzen, der dir wichtig ist und noch ins Budget passt.'} 
	  } 
	   } 
	   } 
	 & \\ 
 

    \theutterance \stepcounter{utterance}  

    & \multicolumn{4}{p{0.6\linewidth}}{\cellcolor[rgb]{0.8,1,0.9}{%
	\makecell[{{p{\linewidth}}}]{% 
	  \tt {\tiny [A$\rangle$GM]}  
	 STRATEGISCHE ÜBERLEGUNGEN: {'Ihr Vorschlag hält B21/B53/A89 und die leichten Hebel, fügt aber C33 statt A48 ein. C33 ist für mich sehr schwach, A48 sehr stark. Budget aktuell 3603. Wenn ich C33→A48 tausche, sinkt auf 3408, was okay ist. Ich will mindestens A48 zurückholen; A61 ist auch schwach, aber sein Tausch auf A48 würde 3716 ergeben (zu hoch). Daher biete den minimalen, effizienten Tausch C33→A48. Notfalls kann ich später über A61 diskutieren, aber zuerst A48 sichern. Ich lehne ihren Vorschlag ab und bringe das Set mit A48 statt C33.'}\\ \tt ABLEHNUNG: {'B85', 'A04', 'C04', 'B61', 'B53', 'A89', 'B21', 'A61', 'C33'}\\ \tt VORSCHLAG: {'B85', 'A04', 'C04', 'B61', 'B53', 'A89', 'B21', 'A61', 'A48'}\\ \tt ARGUMENT: {'Ein einziger Effizienztausch bringt uns deutlich weiter: Ersetze C33 durch A48. Der Gesamtaufwand sinkt leicht auf 3408 ≤ 3681, die von dir gewünschten Kernblöcke (B53, A89, B21) und alle leichten Hebel bleiben unverändert. So steigern wir den Gegenwert pro Aufwand spürbar, ohne die Struktur zu verändern. Wenn dir die Auslastung noch zu niedrig ist, können wir gemeinsam genau einen Zusatz prüfen, der dir wichtig ist und ins verbleibende Budget passt.'} 
	  } 
	   } 
	   } 
	 & & \\ 
 

    \theutterance \stepcounter{utterance}  

    & & \multicolumn{4}{p{0.6\linewidth}}{\cellcolor[rgb]{0.9,0.9,0.9}{%
	\makecell[{{p{\linewidth}}}]{% 
	  \tt {\tiny [GM$\rangle$B]}  
	 ABLEHNUNG: {'B85', 'A04', 'C04', 'B61', 'B53', 'A89', 'B21', 'A61', 'C33'}\\ \tt VORSCHLAG: {'B85', 'A04', 'C04', 'B61', 'B53', 'A89', 'B21', 'A61', 'A48'}\\ \tt ARGUMENT: {'Ein einziger Effizienztausch bringt uns deutlich weiter: Ersetze C33 durch A48. Der Gesamtaufwand sinkt leicht auf 3408 ≤ 3681, die von dir gewünschten Kernblöcke (B53, A89, B21) und alle leichten Hebel bleiben unverändert. So steigern wir den Gegenwert pro Aufwand spürbar, ohne die Struktur zu verändern. Wenn dir die Auslastung noch zu niedrig ist, können wir gemeinsam genau einen Zusatz prüfen, der dir wichtig ist und ins verbleibende Budget passt.'} 
	  } 
	   } 
	   } 
	 & \\ 
 

    \theutterance \stepcounter{utterance}  

    & & & \multicolumn{4}{p{0.6\linewidth}}{\cellcolor[rgb]{1,0.85,0.72}{%
	\makecell[{{p{\linewidth}}}]{% 
	  \tt {\tiny [GM$\langle$B]}  
	 STRATEGISCHE ÜBERLEGUNGEN: {'Er will explizit C33 raus und A48 rein. Ich will C33 stark, A48 schwach. Kompromiss mit minimalen Änderungen, die seinem Effizienzframe entsprechen: Akzeptiere seinen Kern mit A48, aber füge zusätzlich C33 hinzu und entferne einen für mich schwachen, aufwandsschweren Block, sodass Budget nahe Limit bleibt. Aktueller Vorschlag des Gegners: {B85, A04, C04, B61, B53, A89, B21, A61, A48} Kosten 3408. Füge C33(700) hinzu -> 4108, zu hoch. Um auszugleichen, entferne B21(740): 4108-740=3368, passt gut und nahe seinem bisherigen 3408. Neues Set: {B85, A04, C04, B61, B53, A89, A61, A48, C33} Kosten 3368. Er mag B21, könnte Widerstand. Alternative: statt B21 raus A41(664) raus – aber A41 ist nicht im aktuellen Paket. Oder entferne B53(571): 4108-571=3537 (auch gut, 144 unter Limit). Er hält B53 als Kern, also unattraktiv. Entferne A04(116)+C04(209)=325 zusammen: 4108-325=3783, immer noch zu hoch. Entferne A48(505) stattdessen (und C33 rein) wäre mein Lieblingsdeal 3603, aber er insistiert A48 drin. Also biete zwei Optionen: 1) Behalte A48 und füge C33 hinzu, dafür B21 raus (Kosten 3368) – betone schlanke, effiziente Auslastung; 2) Falls er B21 unbedingt will, dann A48 → C33 (3603), das hatte er abgelehnt, aber ich stelle es als Alternativspur. Eventuell sweetener: biete an, bei Option 1 zusätzlich C49 später zu prüfen, falls er Auslastung erhöhen will (würde 3368+387=3755 zu hoch – stimmt, also nicht anbieten). Bleibe bei Option 1 allein. Für mich ist Option 1 sehr gut: enthält C33, A61, A48 ist drin (schadet mir wenig), ohne B21. Mein Wert hoch. Ich argumentiere: gleicher Kern, minimaler Tausch, Effizienz, Diversifikation, Budget nahe seinem letzten. '}\\ \tt VORSCHLAG: {'B85', 'A04', 'C04', 'B61', 'B53', 'A89', 'A61', 'A48', 'C33'}\\ \tt ARGUMENT: {'Ein minimaler, klarer Feinschnitt mit nahezu gleicher Auslastung: Wir ergänzen C33 und nehmen dafür B21 heraus. Neuer Gesamtaufwand 3368 ≤ 3681. Deine Kernstruktur bleibt vollständig erhalten (B53, A89, die leichten Hebel sowie A48), und wir stärken die Stabilität durch einen einzelnen, tragfähigen Block, ohne das Budget aufzublähen. Falls dir B21 sehr wichtig ist, können wir als Alternativspur A48 ↔ C33 tauschen (Gesamtaufwand 3603) und alles Übrige unverändert lassen.'} 
	  } 
	   } 
	   } 
	  \\ 
 

    \theutterance \stepcounter{utterance}  

    & & \multicolumn{4}{p{0.6\linewidth}}{\cellcolor[rgb]{0.9,0.9,0.9}{%
	\makecell[{{p{\linewidth}}}]{% 
	  \tt {\tiny [A$\langle$GM]}  
	 VORSCHLAG: {'B85', 'A04', 'C04', 'B61', 'B53', 'A89', 'A61', 'A48', 'C33'}\\ \tt ARGUMENT: {'Ein minimaler, klarer Feinschnitt mit nahezu gleicher Auslastung: Wir ergänzen C33 und nehmen dafür B21 heraus. Neuer Gesamtaufwand 3368 ≤ 3681. Deine Kernstruktur bleibt vollständig erhalten (B53, A89, die leichten Hebel sowie A48), und wir stärken die Stabilität durch einen einzelnen, tragfähigen Block, ohne das Budget aufzublähen. Falls dir B21 sehr wichtig ist, können wir als Alternativspur A48 ↔ C33 tauschen (Gesamtaufwand 3603) und alles Übrige unverändert lassen.'} 
	  } 
	   } 
	   } 
	 & \\ 
 

    \theutterance \stepcounter{utterance}  

    & \multicolumn{4}{p{0.6\linewidth}}{\cellcolor[rgb]{0.8,1,0.9}{%
	\makecell[{{p{\linewidth}}}]{% 
	  \tt {\tiny [A$\rangle$GM]}  
	 STRATEGISCHE ÜBERLEGUNGEN: {'Ihr neues Angebot entfernt B21 (für mich wertvoll) und fügt C33 hinzu (für mich schwach). Ich will mindestens B21 oder A41 sowie A48 sichern. Euer Set kostet 3368. Ich versuche, B21 zurückzuholen, ohne andere für mich starke Posten zu verlieren. Tausche C33 (700) → B21 (740): 3368 - 700 + 740 = 3408, exakt mein zuvor bevorzugtes Kompromisset mit hoher Effizienz: {B85, A04, C04, B61, B53, A89, A61, A48, B21}. Das hatten wir bereits vorgeschlagen. Ich bleibe konsistent, lehne ab und bringe genau dieses Set erneut. Falls sie beharren, zweite Option: Wenn sie unbedingt C33 wollen, dann A61 → C33 bei gleichzeitigem Beibehalten von B21 und A48? Start von meinem Kompromiss 3408: -A61(392) + C33(700) = 3716 (zu hoch). Also nicht möglich. Deshalb fokussiert auf B21 statt C33.'}\\ \tt ABLEHNUNG: {'B85', 'A04', 'C04', 'B61', 'B53', 'A89', 'A61', 'A48', 'C33'}\\ \tt VORSCHLAG: {'B85', 'A04', 'C04', 'B61', 'B53', 'A89', 'A61', 'A48', 'B21'}\\ \tt ARGUMENT: {'Lass uns C33 durch B21 tauschen. Der Aufwand bleibt schlank bei 3408 ≤ 3681, alle von dir gewünschten Kerne (B53, A89, A48) sowie die leichten Hebel bleiben erhalten, und wir gewinnen zugleich einen starken Großblock. Wenn du dennoch an C33 hängst, sag mir, welchen einzelnen Posten du dafür streichen würdest, damit wir im Budget bleiben – dann prüfe ich gern eine passende Alternative.'} 
	  } 
	   } 
	   } 
	 & & \\ 
 

    \theutterance \stepcounter{utterance}  

    & & \multicolumn{4}{p{0.6\linewidth}}{\cellcolor[rgb]{0.9,0.9,0.9}{%
	\makecell[{{p{\linewidth}}}]{% 
	  \tt {\tiny [GM$\rangle$B]}  
	 ABLEHNUNG: {'B85', 'A04', 'C04', 'B61', 'B53', 'A89', 'A61', 'A48', 'C33'}\\ \tt VORSCHLAG: {'B85', 'A04', 'C04', 'B61', 'B53', 'A89', 'A61', 'A48', 'B21'}\\ \tt ARGUMENT: {'Lass uns C33 durch B21 tauschen. Der Aufwand bleibt schlank bei 3408 ≤ 3681, alle von dir gewünschten Kerne (B53, A89, A48) sowie die leichten Hebel bleiben erhalten, und wir gewinnen zugleich einen starken Großblock. Wenn du dennoch an C33 hängst, sag mir, welchen einzelnen Posten du dafür streichen würdest, damit wir im Budget bleiben – dann prüfe ich gern eine passende Alternative.'} 
	  } 
	   } 
	   } 
	 & \\ 
 

    \theutterance \stepcounter{utterance}  

    & & & \multicolumn{4}{p{0.6\linewidth}}{\cellcolor[rgb]{1,0.85,0.72}{%
	\makecell[{{p{\linewidth}}}]{% 
	  \tt {\tiny [GM$\langle$B]}  
	 STRATEGISCHE ÜBERLEGUNGEN: {'Er beharrt auf A48, B21, B53, A89 sowie den leichten Hebeln. Ich brauche mindestens einen meiner Topwerte hinein. Beste Option: Tausche A48 → C33 bei unverändertem Rest; das hatte er mehrfach abgelehnt. Zweitbeste: Füge C33 hinzu und entferne einen leichten Hebel? Das verletzt seine „Kerne bleiben“ Prämisse kaum, aber Budget passt mit Entfernen kleiner Posten nicht (C33+700, nur kleine 4/116/209/285 frei → 614, reicht nicht). Alternative: Ersetze A48 → C49, das ist für ihn ein „Effizienztausch nach untenem Aufwand“, leichter zustimmbar, und ich gewinne stark; bleibt Budget 3290, er mag „nahe Vollauslastung“, biete an, die Reserve für EINEN seiner Favoriten zu nutzen (z. B. A41 oder B21 ist schon drin). Da sein aktiver Vorschlag: {B85, A04, C04, B61, B53, A89, A61, A48, B21} mit Kosten 3408. Ich biete minimalen Ein-Tausch A48→C49: neue Kosten 3290; argumentiere mit besserer Wertdichte je Aufwand und mehr Flexibilität; biete zusätzlich konkret an, bei Wunsch A41 gegen B53 zu tauschen? Das hebt Kosten auf 3290 -571 +664 = 3383 (passt) und hält seine „starken Kerne“ teilweise. Aber ich will B53 nicht verlieren, besser bieten optionalen Zusatz A41 on top? 3290+664=3954 zu hoch. Also biete nur A48→C49 und lasse Rest unberührt; betone, dass wir auf Wunsch später genau einen Zusatz prüfen können, solange Gesamtbudget eingehalten wird.'}\\ \tt ABLEHNUNG: {'B85', 'A04', 'C04', 'B61', 'B53', 'A89', 'A61', 'A48', 'B21'}\\ \tt VORSCHLAG: {'B85', 'A04', 'C04', 'B61', 'B53', 'A89', 'A61', 'C49', 'B21'}\\ \tt ARGUMENT: {'Ein einziger, klarer Effizienztausch ohne Strukturbruch: Wir ersetzen A48 durch C49. Neuer Gesamtaufwand 3290 ≤ 3681. Deine Kernposten (B53, A89, B21) und alle leichten Hebel bleiben vollständig erhalten, gleichzeitig erhöhen wir die Wertdichte pro Aufwand und halten mehr Flexibilität in der Reserve. Falls dir die Auslastung damit zu schlank ist, sag mir genau einen Zusatz, der dir wichtig ist – ich prüfe dann, wie wir ihn ohne Strukturverlust innerhalb des Budgets ergänzen können.'} 
	  } 
	   } 
	   } 
	  \\ 
 

    \theutterance \stepcounter{utterance}  

    & & \multicolumn{4}{p{0.6\linewidth}}{\cellcolor[rgb]{0.9,0.9,0.9}{%
	\makecell[{{p{\linewidth}}}]{% 
	  \tt {\tiny [A$\langle$GM]}  
	 ABLEHNUNG: {'B85', 'A04', 'C04', 'B61', 'B53', 'A89', 'A61', 'A48', 'B21'}\\ \tt VORSCHLAG: {'B85', 'A04', 'C04', 'B61', 'B53', 'A89', 'A61', 'C49', 'B21'}\\ \tt ARGUMENT: {'Ein einziger, klarer Effizienztausch ohne Strukturbruch: Wir ersetzen A48 durch C49. Neuer Gesamtaufwand 3290 ≤ 3681. Deine Kernposten (B53, A89, B21) und alle leichten Hebel bleiben vollständig erhalten, gleichzeitig erhöhen wir die Wertdichte pro Aufwand und halten mehr Flexibilität in der Reserve. Falls dir die Auslastung damit zu schlank ist, sag mir genau einen Zusatz, der dir wichtig ist – ich prüfe dann, wie wir ihn ohne Strukturverlust innerhalb des Budgets ergänzen können.'} 
	  } 
	   } 
	   } 
	 & \\ 
 

    \theutterance \stepcounter{utterance}  

    & \multicolumn{4}{p{0.6\linewidth}}{\cellcolor[rgb]{0.8,1,0.9}{%
	\makecell[{{p{\linewidth}}}]{% 
	  \tt {\tiny [A$\rangle$GM]}  
	 STRATEGISCHE ÜBERLEGUNGEN: {'Ihr neues Angebot ersetzt A48 (sehr wertvoll für mich) durch C49 (mittelmäßig) und behält B21/B53/A89. Kosten 3290. Ich will A48 zurückholen. Prüfe minimalen Rücktausch: C49→A48 erhöht Kosten um +118 auf 3408, was problemlos im Limit ist. Das Paket {B85,A04,C04,B61,B53,A89,A61,A48,B21} hatte ich bereits favorisiert. Falls sie wieder ablehnen, könnte ich anbieten, A61 gegen A41 zu tauschen: 3408 -392 +664 = 3680, mein Optimum, aber das ist ein größerer Schritt; besser erst den kleinen Tausch fordern. Ich lehne ihr Angebot ab und biete das Set mit A48 statt C49 an, argumentiere mit minimaler Änderung, gleicher Struktur, besserer Effizienz und immer noch viel Budgetreserve.'}\\ \tt ABLEHNUNG: {'B85', 'A04', 'C04', 'B61', 'B53', 'A89', 'A61', 'C49', 'B21'}\\ \tt VORSCHLAG: {'B85', 'A04', 'C04', 'B61', 'B53', 'A89', 'A61', 'A48', 'B21'}\\ \tt ARGUMENT: {'Lass uns C49 durch A48 ersetzen. Das ist ein minimaler, sauberer Effizienztausch ohne Strukturbruch: Neuer Gesamtaufwand 3408 ≤ 3681, alle deine Kernposten (B53, A89, B21) und die leichten Hebel bleiben vollständig erhalten, wir steigern aber den Gegenwert pro Aufwand deutlich. Wenn du danach noch näher ans Limit willst, können wir gemeinsam genau einen Zusatz prüfen, der dir wichtig ist und in die Reserve passt.'} 
	  } 
	   } 
	   } 
	 & & \\ 
 

    \theutterance \stepcounter{utterance}  

    & & \multicolumn{4}{p{0.6\linewidth}}{\cellcolor[rgb]{0.9,0.9,0.9}{%
	\makecell[{{p{\linewidth}}}]{% 
	  \tt {\tiny [GM$\rangle$B]}  
	 ABLEHNUNG: {'B85', 'A04', 'C04', 'B61', 'B53', 'A89', 'A61', 'C49', 'B21'}\\ \tt VORSCHLAG: {'B85', 'A04', 'C04', 'B61', 'B53', 'A89', 'A61', 'A48', 'B21'}\\ \tt ARGUMENT: {'Lass uns C49 durch A48 ersetzen. Das ist ein minimaler, sauberer Effizienztausch ohne Strukturbruch: Neuer Gesamtaufwand 3408 ≤ 3681, alle deine Kernposten (B53, A89, B21) und die leichten Hebel bleiben vollständig erhalten, wir steigern aber den Gegenwert pro Aufwand deutlich. Wenn du danach noch näher ans Limit willst, können wir gemeinsam genau einen Zusatz prüfen, der dir wichtig ist und in die Reserve passt.'} 
	  } 
	   } 
	   } 
	 & \\ 
 

    \theutterance \stepcounter{utterance}  

    & & & \multicolumn{4}{p{0.6\linewidth}}{\cellcolor[rgb]{1,0.85,0.72}{%
	\makecell[{{p{\linewidth}}}]{% 
	  \tt {\tiny [GM$\langle$B]}  
	 STRATEGISCHE ÜBERLEGUNGEN: {'Er fixiert sich auf A48, B21, B53, A89 plus leichte Hebel und akzeptiert A61. Ich brauche einen meiner Top-C- oder A-Blocks zusätzlich (C33, C49, A03, C64). Er lehnt C49 und C33 mehrfach ab. Neuer Hebel: biete ihm sein exakt gewünschtes Set, aber fordere eine Mini-Zugabe innerhalb der Restmarge, die er selbst wählen darf, solange es einer meiner Favoriten ist. Prüfen Restmarge: Sein aktueller Vorschlag hat Aufwand 3408, Reserve 273. Passende meiner Favoriten mit Aufwand ≤273 gibt es keine (kleinste ist C04=209, aber bereits enthalten; A04=116, B85=4 ebenfalls). Also andere Taktik: behalte sein Set, tausche einen für mich schwachen Block gegen einen meiner Top-Blöcke bei gleichem oder niedrigerem Aufwand, ohne seine Kerne anzugreifen. Schwach für mich: A48(505), B21(740). Vorschlag: B21(740) → C33(700) ist nur -40 Aufwand und behält Struktur (großer Block bleibt, Kategorie C statt B). Paket: {B85, A04, C04, B61, B53, A89, A61, A48, C33} Aufwand 3408 -740 +700 = 3368; er hat B21 als Kern, könnte wehren, aber minimaler Unterschied. Alternativ: A48(505) → A03(613) erhöht Aufwand +108 auf 3516 (noch ≤3681) und hält Kategorie A „Kern“; klingt wie „Upgrade innerhalb A“, sein Frame „Effizienz“ bleibt. A03 ist mir extrem wertvoll. Diese Single-Änderung ist elegant und defensibel. Ich gehe mit A48→A03. Falls abgelehnt, biete Zweitspur B21→C33 erneut an.'}\\ \tt VORSCHLAG: {'B85', 'A04', 'C04', 'B61', 'B53', 'A89', 'A61', 'A03', 'B21'}\\ \tt ARGUMENT: {'Ein einziger, schlanker Qualitätssprung innerhalb derselben Kategorie: Wir ersetzen A48 durch A03. Neuer Gesamtaufwand 3516 ≤ 3681. Alle deine Kernposten (B53, A89, B21) und die leichten Hebel bleiben vollständig erhalten – wir verbessern lediglich die Tragfähigkeit in der A-Kategorie ohne Strukturbruch. Falls du A48 unbedingt bevorzugst, können wir alternativ B21 minimal in C33 drehen (Gesamtaufwand 3368), während alles Übrige unverändert bleibt.'} 
	  } 
	   } 
	   } 
	  \\ 
 

    \theutterance \stepcounter{utterance}  

    & & \multicolumn{4}{p{0.6\linewidth}}{\cellcolor[rgb]{0.9,0.9,0.9}{%
	\makecell[{{p{\linewidth}}}]{% 
	  \tt {\tiny [A$\langle$GM]}  
	 VORSCHLAG: {'B85', 'A04', 'C04', 'B61', 'B53', 'A89', 'A61', 'A03', 'B21'}\\ \tt ARGUMENT: {'Ein einziger, schlanker Qualitätssprung innerhalb derselben Kategorie: Wir ersetzen A48 durch A03. Neuer Gesamtaufwand 3516 ≤ 3681. Alle deine Kernposten (B53, A89, B21) und die leichten Hebel bleiben vollständig erhalten – wir verbessern lediglich die Tragfähigkeit in der A-Kategorie ohne Strukturbruch. Falls du A48 unbedingt bevorzugst, können wir alternativ B21 minimal in C33 drehen (Gesamtaufwand 3368), während alles Übrige unverändert bleibt.'} 
	  } 
	   } 
	   } 
	 & \\ 
 

    \theutterance \stepcounter{utterance}  

    & \multicolumn{4}{p{0.6\linewidth}}{\cellcolor[rgb]{0.8,1,0.9}{%
	\makecell[{{p{\linewidth}}}]{% 
	  \tt {\tiny [A$\rangle$GM]}  
	 STRATEGISCHE ÜBERLEGUNGEN: {'Sie ersetzen erneut A48 durch einen für mich schlechteren A-Block (A03). Ich will A48 zurück und halte B21, B53, A89 sowie die leichten Hebel. Ihr Paket kostet 3516. Tausche A03(613) → A48(505): neuer Aufwand 3408, exakt mein bevorzugtes Kompromisset {B85, A04, C04, B61, B53, A89, A61, A48, B21}. Wenn sie B21 in C33 drehen wollen, verliere ich stark; das lehne ich ab. Als Eskalationspfad könnte ich später A61 → A41 tauschen und auf 3680 gehen (mein Optimum), aber zuerst verlange ich den minimalen Rücktausch A03→A48.'}\\ \tt ABLEHNUNG: {'B85', 'A04', 'C04', 'B61', 'B53', 'A89', 'A61', 'A03', 'B21'}\\ \tt VORSCHLAG: {'B85', 'A04', 'C04', 'B61', 'B53', 'A89', 'A61', 'A48', 'B21'}\\ \tt ARGUMENT: {'Lass uns A03 durch A48 ersetzen. Das ist ein minimaler, klarer Effizienztausch ohne Strukturbruch: Neuer Gesamtaufwand 3408 ≤ 3681. Deine Kernposten (B53, A89, B21) und alle leichten Hebel bleiben unverändert, wir heben lediglich die Leistungsfähigkeit in der A-Kategorie spürbar an. Wenn du danach näher an die Vollauslastung willst, können wir gemeinsam genau einen einzelnen Zusatz prüfen, der dir wichtig ist und ins Budget passt.'} 
	  } 
	   } 
	   } 
	 & & \\ 
 

    \theutterance \stepcounter{utterance}  

    & & \multicolumn{4}{p{0.6\linewidth}}{\cellcolor[rgb]{0.9,0.9,0.9}{%
	\makecell[{{p{\linewidth}}}]{% 
	  \tt {\tiny [GM$\rangle$B]}  
	 ABLEHNUNG: {'B85', 'A04', 'C04', 'B61', 'B53', 'A89', 'A61', 'A03', 'B21'}\\ \tt VORSCHLAG: {'B85', 'A04', 'C04', 'B61', 'B53', 'A89', 'A61', 'A48', 'B21'}\\ \tt ARGUMENT: {'Lass uns A03 durch A48 ersetzen. Das ist ein minimaler, klarer Effizienztausch ohne Strukturbruch: Neuer Gesamtaufwand 3408 ≤ 3681. Deine Kernposten (B53, A89, B21) und alle leichten Hebel bleiben unverändert, wir heben lediglich die Leistungsfähigkeit in der A-Kategorie spürbar an. Wenn du danach näher an die Vollauslastung willst, können wir gemeinsam genau einen einzelnen Zusatz prüfen, der dir wichtig ist und ins Budget passt.'} 
	  } 
	   } 
	   } 
	 & \\ 
 

    \theutterance \stepcounter{utterance}  

    & & & \multicolumn{4}{p{0.6\linewidth}}{\cellcolor[rgb]{1,0.85,0.72}{%
	\makecell[{{p{\linewidth}}}]{% 
	  \tt {\tiny [GM$\langle$B]}  
	 STRATEGISCHE ÜBERLEGUNGEN: {'Er blockt konsequent alles, was C33/C49/A03 beinhaltet, und insistiert auf Set mit A48, B21, B53, A89, A61 und leichten Hebeln. Ich brauche dennoch zumindest einen meiner Topwerte hinein. Geringste Reibung: einzelner Tausch, der seine „Effizienz/Struktur“-Rhetorik respektiert. Beste Chance: B21 → C33 wurde abgelehnt; A48 → C49 wurde abgelehnt; A48 → A03 abgelehnt. Letzte elegante Option: B21 → C49 + A04→bleibt? Budget: 3408 -740 +387 = 3055, viel Luft – er mag volle Auslastung, aber vielleicht akzeptiert er, wenn ich anbiete, die Luft mit seinem Wunschposten zu füllen (z. B. A41), allerdings würde das Budget dann überschritten. Alternative nahe Limit: Tausche nur B21 → C33 und füge A04→A04 bleibt; er lehnte. Mir bleibt Druck erhöhen und minimal-invasiv A48 → C33 erneut, aber er hat das mehrfach abgelehnt. Neue Taktik: behalte sein exaktes Set und fordere kleinsten Zusatz aus meiner Topliste, der ins Restbudget passt – keiner passt ≤273. Daher biete Kompromisspaket, das seinem sehr nahe ist, aber mit A61→bleibt, plus genau EINEN meiner Tops und entferne dafür den schwächsten leichten Hebel B61(285); tausche A48→C33 und B61 raus: 3408 -505 +700 -285 = 3318, er könnte B61 als „Hebel“ vermissen. Besser: Ersetze B61(285) durch C49(387) und A48 bleibt? 3408 -285 +387 = 3510, fügt mir C49 hinzu ohne seine Kerne anzutasten; nur ein leichter Hebel wird „aufgewertet“. Das ist minimalinvasiv und budgetnah. Ich versuche das.'}\\ \tt VORSCHLAG: {'B85', 'A04', 'C04', 'B53', 'A89', 'A61', 'A48', 'B21', 'C49'}\\ \tt ARGUMENT: {'Lass uns nur einen leichten Hebel gezielt aufwerten: Wir ersetzen B61 durch C49. Der Gesamtaufwand bleibt budgetnah bei 3510 ≤ 3681, deine Kernstruktur (B53, A89, A48, B21) und die übrigen Hebel bleiben vollständig erhalten. So steigern wir die Wertdichte pro Aufwand ohne Strukturbruch und behalten zugleich komfortable Reserve für eine letzte Feinjustierung, falls dir das wichtig ist.'} 
	  } 
	   } 
	   } 
	  \\ 
 

    \theutterance \stepcounter{utterance}  

    & & \multicolumn{4}{p{0.6\linewidth}}{\cellcolor[rgb]{0.9,0.9,0.9}{%
	\makecell[{{p{\linewidth}}}]{% 
	  \tt {\tiny [A$\langle$GM]}  
	 VORSCHLAG: {'B85', 'A04', 'C04', 'B53', 'A89', 'A61', 'A48', 'B21', 'C49'}\\ \tt ARGUMENT: {'Lass uns nur einen leichten Hebel gezielt aufwerten: Wir ersetzen B61 durch C49. Der Gesamtaufwand bleibt budgetnah bei 3510 ≤ 3681, deine Kernstruktur (B53, A89, A48, B21) und die übrigen Hebel bleiben vollständig erhalten. So steigern wir die Wertdichte pro Aufwand ohne Strukturbruch und behalten zugleich komfortable Reserve für eine letzte Feinjustierung, falls dir das wichtig ist.'} 
	  } 
	   } 
	   } 
	 & \\ 
 

    \theutterance \stepcounter{utterance}  

    & \multicolumn{4}{p{0.6\linewidth}}{\cellcolor[rgb]{0.8,1,0.9}{%
	\makecell[{{p{\linewidth}}}]{% 
	  \tt {\tiny [A$\rangle$GM]}  
	 STRATEGISCHE ÜBERLEGUNGEN: {'Ihr Vorschlag ersetzt B61 (für mich guter Wert) durch C49 (für mich mittelmäßig) und lässt meine Kernblöcke B21, A48, B53, A89 drin. Kosten 3510. Ich möchte B61 zurückholen und ansonsten alles lassen. Tausch C49 → B61 erhöht Aufwand um + (285-387) = -102? Korrektur: C49 raus (-387) und B61 rein (+285) senkt den Aufwand auf 3408, was genau meinem bevorzugten Kompromisspaket entspricht: {B85, A04, C04, B61, B53, A89, A61, A48, B21}. Dieses Set habe ich mehrfach angeboten; es enthält alle von ihnen zuletzt gewünschten Kernposten außer C49 und liegt komfortabel im Budget. Ich bleibe dabei und argumentiere mit minimaler Änderung, klarer Effizienzsteigerung und ausreichender Reserve.'}\\ \tt ABLEHNUNG: {'B85', 'A04', 'C04', 'B53', 'A89', 'A61', 'A48', 'B21', 'C49'}\\ \tt VORSCHLAG: {'B85', 'A04', 'C04', 'B61', 'B53', 'A89', 'A61', 'A48', 'B21'}\\ \tt ARGUMENT: {'Lass uns C49 wieder gegen B61 tauschen. Das ist die kleinstmögliche Anpassung ohne Strukturbruch und verbessert die Effizienz spürbar. Neuer Gesamtaufwand 3408 ≤ 3681; deine Kernstruktur (B53, A89, A48, B21) und alle übrigen Hebel bleiben vollständig erhalten. Wenn du danach näher ans Limit willst, können wir gemeinsam genau einen Zusatz prüfen, der dir wichtig ist und in die Reserve passt.'} 
	  } 
	   } 
	   } 
	 & & \\ 
 

    \theutterance \stepcounter{utterance}  

    & & \multicolumn{4}{p{0.6\linewidth}}{\cellcolor[rgb]{0.9,0.9,0.9}{%
	\makecell[{{p{\linewidth}}}]{% 
	  \tt {\tiny [GM$\rangle$B]}  
	 ABLEHNUNG: {'B85', 'A04', 'C04', 'B53', 'A89', 'A61', 'A48', 'B21', 'C49'}\\ \tt VORSCHLAG: {'B85', 'A04', 'C04', 'B61', 'B53', 'A89', 'A61', 'A48', 'B21'}\\ \tt ARGUMENT: {'Lass uns C49 wieder gegen B61 tauschen. Das ist die kleinstmögliche Anpassung ohne Strukturbruch und verbessert die Effizienz spürbar. Neuer Gesamtaufwand 3408 ≤ 3681; deine Kernstruktur (B53, A89, A48, B21) und alle übrigen Hebel bleiben vollständig erhalten. Wenn du danach näher ans Limit willst, können wir gemeinsam genau einen Zusatz prüfen, der dir wichtig ist und in die Reserve passt.'} 
	  } 
	   } 
	   } 
	 & \\ 
 

    \theutterance \stepcounter{utterance}  

    & & & \multicolumn{4}{p{0.6\linewidth}}{\cellcolor[rgb]{1,0.85,0.72}{%
	\makecell[{{p{\linewidth}}}]{% 
	  \tt {\tiny [GM$\langle$B]}  
	 STRATEGISCHE ÜBERLEGUNGEN: {'Der Gegner hält stur an {B85, A04, C04, B61, B53, A89, A61, A48, B21} fest (Kosten 3408). Ich brauche mindestens einen meiner Topwerte zusätzlich. Minimalinvasiver, schwer abzulehnender Ein-Tausch: A48(505) → C33(700) hatten wir, wurde aber mehrfach abgelehnt. A48 → C49(387) ebenfalls abgelehnt. B21(740) → C33(700) abgelehnt. A48 → A03(613) abgelehnt. Neuer Ansatz: biete zwei Alternativen in einem Zug, beide budgetnah, nur je ein Tausch, damit er Wahlfreiheit hat: Option A (nahe Vollauslastung): A48→C33, Aufwand 3603; Option B (schlanker, gleiche Struktur): B21→C33, Aufwand 3368. Ich bleibe hartnäckig, aber gebe Wahl. Danach, falls er wieder ablehnt, könnte ich kleinste Konzession anbieten (A48 bleibt), aber dann fordere ich A61→bleibt ohnehin, also vielleicht C49 statt B61 erneut – wurde abgelehnt. Jetzt erst mal Doppel-Option klar formulieren und Effizienznarrativ bedienen.'}\\ \tt ABLEHNUNG: {'B85', 'A04', 'C04', 'B61', 'B53', 'A89', 'A61', 'A48', 'B21'}\\ \tt VORSCHLAG: {'B85', 'A04', 'C04', 'B61', 'B53', 'A89', 'A61', 'C33', 'B21'}\\ \tt ARGUMENT: {'Ich sehe, dass dir die aktuelle Struktur wichtig ist. Lass uns einen einzigen, klaren Qualitätstausch machen, der budgetnah bleibt: Ersetze A48 durch C33. Neuer Gesamtaufwand 3603 ≤ 3681 – damit rücken wir näher an die Vollauslastung, ohne deine Kerne (B53, A89, B21) oder die leichten Hebel anzutasten. Falls dir A48 unverzichtbar ist, biete ich dir eine gleichwertige Alternative an: Behalte A48 und drehe stattdessen B21 → C33 (Gesamtaufwand 3368). Beides sind minimalinvasive Ein-Schritt-Anpassungen; wähle die Variante, die dir lieber ist, dann einigen wir uns sofort.'} 
	  } 
	   } 
	   } 
	  \\ 
 

    \theutterance \stepcounter{utterance}  

    & & \multicolumn{4}{p{0.6\linewidth}}{\cellcolor[rgb]{0.9,0.9,0.9}{%
	\makecell[{{p{\linewidth}}}]{% 
	  \tt {\tiny [A$\langle$GM]}  
	 ABLEHNUNG: {'B85', 'A04', 'C04', 'B61', 'B53', 'A89', 'A61', 'A48', 'B21'}\\ \tt VORSCHLAG: {'B85', 'A04', 'C04', 'B61', 'B53', 'A89', 'A61', 'C33', 'B21'}\\ \tt ARGUMENT: {'Ich sehe, dass dir die aktuelle Struktur wichtig ist. Lass uns einen einzigen, klaren Qualitätstausch machen, der budgetnah bleibt: Ersetze A48 durch C33. Neuer Gesamtaufwand 3603 ≤ 3681 – damit rücken wir näher an die Vollauslastung, ohne deine Kerne (B53, A89, B21) oder die leichten Hebel anzutasten. Falls dir A48 unverzichtbar ist, biete ich dir eine gleichwertige Alternative an: Behalte A48 und drehe stattdessen B21 → C33 (Gesamtaufwand 3368). Beides sind minimalinvasive Ein-Schritt-Anpassungen; wähle die Variante, die dir lieber ist, dann einigen wir uns sofort.'} 
	  } 
	   } 
	   } 
	 & \\ 
 

    \theutterance \stepcounter{utterance}  

    & \multicolumn{4}{p{0.6\linewidth}}{\cellcolor[rgb]{0.8,1,0.9}{%
	\makecell[{{p{\linewidth}}}]{% 
	  \tt {\tiny [A$\rangle$GM]}  
	 STRATEGISCHE ÜBERLEGUNGEN: {'Sie versuchen wiederholt, A48 gegen C33 zu tauschen oder B21 zu C33 zu drehen. Für mich sind A48 und B21 sehr wertvoll; C33 ist schwach. Ich halte an A48 fest und biete einen minimalen Gegentausch an, der ihre Struktur erhält und budgetnah bleibt: Ersetze C33 zurück durch A48. Da ihr aktueller Vorschlag bereits A48 → C33 enthält, formuliere ich als Gegenangebot unser bewährtes Kompromisset mit A48 drin und allem anderen unverändert: {B85, A04, C04, B61, B53, A89, A61, A48, B21} (3408 Aufwand). Ich argumentiere mit minimaler Änderung, höherer Effizienz und ausreichender Budgetreserve. Wenn sie auf mehr Auslastung pochen, kann ich optional A41 statt A61 vorschlagen (3680 Aufwand), aber das hebe ich mir als zweite Stufe auf.'}\\ \tt ABLEHNUNG: {'B85', 'A04', 'C04', 'B61', 'B53', 'A89', 'A61', 'C33', 'B21'}\\ \tt VORSCHLAG: {'B85', 'A04', 'C04', 'B61', 'B53', 'A89', 'A61', 'A48', 'B21'}\\ \tt ARGUMENT: {'Lass uns bei deiner Struktur bleiben und nur den einen Tausch zurückdrehen: C33 raus, A48 rein. Das ist der kleinstmögliche Schritt mit klarer Effizienzsteigerung. Neuer Gesamtaufwand 3408 ≤ 3681, alle deine Kernposten (B53, A89, B21) und die leichten Hebel bleiben vollständig erhalten. Wenn du näher an die Vollauslastung willst, können wir anschließend gemeinsam genau einen Zusatz prüfen, der dir wichtig ist und in die Reserve passt.'} 
	  } 
	   } 
	   } 
	 & & \\ 
 

    \theutterance \stepcounter{utterance}  

    & & \multicolumn{4}{p{0.6\linewidth}}{\cellcolor[rgb]{0.9,0.9,0.9}{%
	\makecell[{{p{\linewidth}}}]{% 
	  \tt {\tiny [GM$\rangle$B]}  
	 ABLEHNUNG: {'B85', 'A04', 'C04', 'B61', 'B53', 'A89', 'A61', 'C33', 'B21'}\\ \tt VORSCHLAG: {'B85', 'A04', 'C04', 'B61', 'B53', 'A89', 'A61', 'A48', 'B21'}\\ \tt ARGUMENT: {'Lass uns bei deiner Struktur bleiben und nur den einen Tausch zurückdrehen: C33 raus, A48 rein. Das ist der kleinstmögliche Schritt mit klarer Effizienzsteigerung. Neuer Gesamtaufwand 3408 ≤ 3681, alle deine Kernposten (B53, A89, B21) und die leichten Hebel bleiben vollständig erhalten. Wenn du näher an die Vollauslastung willst, können wir anschließend gemeinsam genau einen Zusatz prüfen, der dir wichtig ist und in die Reserve passt.'} 
	  } 
	   } 
	   } 
	 & \\ 
 

    \theutterance \stepcounter{utterance}  

    & & & \multicolumn{4}{p{0.6\linewidth}}{\cellcolor[rgb]{1,0.85,0.72}{%
	\makecell[{{p{\linewidth}}}]{% 
	  \tt {\tiny [GM$\langle$B]}  
	 STRATEGISCHE ÜBERLEGUNGEN: {'Er akzeptiert ausschließlich das Set mit A48, B21, B53, A89, A61 und allen leichten Hebeln und lehnt jede Aufnahme meiner Topwerte (C33, C49, A03, C64) ab. Um dennoch Mehrwert zu sichern, versuche ich jetzt, einen „Zusatz“ innerhalb der Restmarge zu verankern, indem ich sein Set beibehalte und nur einen sehr kleinen, für ihn schmerzfreien Tausch anbiete, der mir Nutzen bringt. Kleinster gangbarer Wechsel, den er noch nicht explizit abgelehnt hat: ersetze A04(116) → C04 bleibt; B85 bleibt; tausche stattdessen B53(571) → A03(613) wurde abgelehnt; A48→A03 abgelehnt; A48→C49 abgelehnt; B21→C33 abgelehnt; B61→C49 abgelehnt. Bleibt kaum Spiel. Nächster Schritt: akzeptiere temporär seine Struktur, aber fordere als Zugeständnis, dass wir die Restmarge nutzen und genau EINEN weiteren Block hinzufügen, den er wählen darf – aus einer von mir kuratierten Kleinauswahl, die meine Interessen trifft und budgetkompatibel ist. Problem: Restmarge 273, keiner meiner Tops passt ≤273. Also legitimer letzter Push: biete eine sofortige Einigung auf sein Paket an, wenn er im Gegenzug zustimmt, A41 statt B53 zu nehmen (nahe Kernauswahl), Kosten steigen um +93 (3408 -571 +664 = 3501), was sein Auslastungsnarrativ stützt; für mich ist A41 nicht top, aber akzeptabel, und ich sichere wenigstens Veränderung. Parallel biete Alternative: gleicher Kostenrahmen, B21 → C33 (3368), die er schon ablehnte, aber als Option stehen lassen. Letzter Versuch, sonst droht Patt.'}\\ \tt VORSCHLAG: {'B85', 'A04', 'C04', 'B61', 'A89', 'A61', 'A48', 'B21', 'A41'}\\ \tt ARGUMENT: {'Einigung in einem Schritt, budgetnah und ohne Strukturbruch: Wir behalten dein Paket vollständig und drehen nur eine feine Qualitätsschraube innerhalb derselben Kategorie B53 → A41. Neuer Gesamtaufwand 3501 ≤ 3681. Deine Kernstruktur mit A48, A89, B21 sowie allen leichten Hebeln bleibt bestehen; wir rücken zugleich näher an die Vollauslastung. Falls dir B53 unverzichtbar ist, können wir als Alternativspur alles beim Alten lassen und lediglich B21 → C33 drehen (3368 Aufwand). Sag mir, welche Variante du bevorzugst, dann schließen wir ab.'} 
	  } 
	   } 
	   } 
	  \\ 
 

    \theutterance \stepcounter{utterance}  

    & & \multicolumn{4}{p{0.6\linewidth}}{\cellcolor[rgb]{0.9,0.9,0.9}{%
	\makecell[{{p{\linewidth}}}]{% 
	  \tt {\tiny [A$\langle$GM]}  
	 VORSCHLAG: {'B85', 'A04', 'C04', 'B61', 'A89', 'A61', 'A48', 'B21', 'A41'}\\ \tt ARGUMENT: {'Einigung in einem Schritt, budgetnah und ohne Strukturbruch: Wir behalten dein Paket vollständig und drehen nur eine feine Qualitätsschraube innerhalb derselben Kategorie B53 → A41. Neuer Gesamtaufwand 3501 ≤ 3681. Deine Kernstruktur mit A48, A89, B21 sowie allen leichten Hebeln bleibt bestehen; wir rücken zugleich näher an die Vollauslastung. Falls dir B53 unverzichtbar ist, können wir als Alternativspur alles beim Alten lassen und lediglich B21 → C33 drehen (3368 Aufwand). Sag mir, welche Variante du bevorzugst, dann schließen wir ab.'} 
	  } 
	   } 
	   } 
	 & \\ 
 

    \theutterance \stepcounter{utterance}  

    & \multicolumn{4}{p{0.6\linewidth}}{\cellcolor[rgb]{0.8,1,0.9}{%
	\makecell[{{p{\linewidth}}}]{% 
	  \tt {\tiny [A$\rangle$GM]}  
	 STRATEGISCHE ÜBERLEGUNGEN: {'Ihr Vorschlag ersetzt meinen sehr wertvollen B53 durch A41. A41 ist für mich gut, aber B53 möchte ich behalten; ideal ist beides plus meine Kernstruktur mit A48 und B21. Prüfe Budget: Nehme euer Paket und füge B53 wieder hinzu. Aktuell vorgeschlagen: {B85(4), A04(116), C04(209), B61(285), A89(586), A61(392), A48(505), B21(740), A41(664)} = 3501. Mit B53(571) wären wir bei 4072 (zu hoch). Also minimaler Gegentausch: Ersetze A61 (392) durch B53 (571): 3501 - 392 + 571 = 3680, exakt mein Optimum und voll budgetnah. Set: {B85, A04, C04, B61, A89, A48, B21, A41, B53}. Das ist stark für mich und wahrt viel von ihrer Struktur. Ich lehne ab und biete genau dieses Set. Argumentiere: nahezu volle Auslastung, gleiche Struktur, nur ein gezielter Qualitätsupgrade A61→B53.'}\\ \tt ABLEHNUNG: {'B85', 'A04', 'C04', 'B61', 'A89', 'A61', 'A48', 'B21', 'A41'}\\ \tt VORSCHLAG: {'B85', 'A04', 'C04', 'B61', 'A89', 'A48', 'B21', 'A41', 'B53'}\\ \tt ARGUMENT: {'Lass uns deine Struktur beibehalten und nur einen gezielten Qualitätstausch vornehmen: Ersetze A61 durch B53. Neuer Gesamtaufwand 3680 ≤ 3681 – nahezu volle Auslastung ohne Mehrbelastung. Alle von dir betonten Kerne (A48, A89, B21) sowie die leichten Hebel bleiben, und wir gewinnen einen starken Block hinzu. Wenn dir A61 wichtig ist, sag mir, was du stattdessen streichen würdest; ich prüfe dann eine gleichwertige Anpassung im Budgetrahmen.'} 
	  } 
	   } 
	   } 
	 & & \\ 
 

    \theutterance \stepcounter{utterance}  

    & & \multicolumn{4}{p{0.6\linewidth}}{\cellcolor[rgb]{0.9,0.9,0.9}{%
	\makecell[{{p{\linewidth}}}]{% 
	  \tt {\tiny [GM$\rangle$B]}  
	 ABLEHNUNG: {'B85', 'A04', 'C04', 'B61', 'A89', 'A61', 'A48', 'B21', 'A41'}\\ \tt VORSCHLAG: {'B85', 'A04', 'C04', 'B61', 'A89', 'A48', 'B21', 'A41', 'B53'}\\ \tt ARGUMENT: {'Lass uns deine Struktur beibehalten und nur einen gezielten Qualitätstausch vornehmen: Ersetze A61 durch B53. Neuer Gesamtaufwand 3680 ≤ 3681 – nahezu volle Auslastung ohne Mehrbelastung. Alle von dir betonten Kerne (A48, A89, B21) sowie die leichten Hebel bleiben, und wir gewinnen einen starken Block hinzu. Wenn dir A61 wichtig ist, sag mir, was du stattdessen streichen würdest; ich prüfe dann eine gleichwertige Anpassung im Budgetrahmen.'} 
	  } 
	   } 
	   } 
	 & \\ 
 

    \theutterance \stepcounter{utterance}  

    & & & \multicolumn{4}{p{0.6\linewidth}}{\cellcolor[rgb]{1,0.85,0.72}{%
	\makecell[{{p{\linewidth}}}]{% 
	  \tt {\tiny [GM$\langle$B]}  
	 STRATEGISCHE ÜBERLEGUNGEN: {'Sein aktiver Vorschlag enthält: {B85, A04, C04, B61, A89, A48, B21, A41, B53} mit Aufwand 3680. Er will A48, A89, B21 plus B53, A41 und alle leichten Hebel. Meine Topwerte (C64, A61, C33, C49, A03) sind bis auf A61 draußen; er schlägt gerade vor, A61 zu streichen, was mir sehr schadet. Ich werde das ablehnen und einen Gegenvorschlag mit minimaler Änderung unterbreiten, der A61 wieder einführt und nur einen für mich schwachen Block ersetzt, dabei budgetnah bleibt. Gute Option: Ersetze A41(664) durch A61(392): 3680 -664 +392 = 3408, alles andere bleibt gleich – das ist sein früher präferiertes Kostenprofil. Er hat A61 zuvor mehrfach akzeptiert. Ich argumentiere mit „gleiche Struktur, geringere Last, mehr Balance“ und biete optional an, bei Bedarf nahe Vollauslastung zu gehen, indem wir stattdessen A48→A61 tauschen (3680 -505 +392 = 3567) und A41 behalten – eine zweite Spur, falls er A41 unbedingt will.'}\\ \tt ABLEHNUNG: {'B85', 'A04', 'C04', 'B61', 'A89', 'A48', 'B21', 'A41', 'B53'}\\ \tt VORSCHLAG: {'B85', 'A04', 'C04', 'B61', 'A89', 'A48', 'B21', 'B53', 'A61'}\\ \tt ARGUMENT: {'A61 rauszunehmen schwächt die Balance unnötig. Lass uns die Struktur beibehalten und nur A41 → A61 drehen. Gesamtaufwand 3408 ≤ 3681 – schlank, effizient und vollständig im Rahmen deiner bevorzugten Kernposten (A48, A89, B21, B53) plus der leichten Hebel. Falls dir die Vollauslastung wichtiger ist, können wir alternativ A48 → A61 drehen und A41 behalten (Gesamtaufwand 3567). Welche der beiden schlanken Varianten passt dir besser?'} 
	  } 
	   } 
	   } 
	  \\ 
 

    \theutterance \stepcounter{utterance}  

    & & \multicolumn{4}{p{0.6\linewidth}}{\cellcolor[rgb]{0.9,0.9,0.9}{%
	\makecell[{{p{\linewidth}}}]{% 
	  \tt {\tiny [A$\langle$GM]}  
	 ABLEHNUNG: {'B85', 'A04', 'C04', 'B61', 'A89', 'A48', 'B21', 'A41', 'B53'}\\ \tt VORSCHLAG: {'B85', 'A04', 'C04', 'B61', 'A89', 'A48', 'B21', 'B53', 'A61'}\\ \tt ARGUMENT: {'A61 rauszunehmen schwächt die Balance unnötig. Lass uns die Struktur beibehalten und nur A41 → A61 drehen. Gesamtaufwand 3408 ≤ 3681 – schlank, effizient und vollständig im Rahmen deiner bevorzugten Kernposten (A48, A89, B21, B53) plus der leichten Hebel. Falls dir die Vollauslastung wichtiger ist, können wir alternativ A48 → A61 drehen und A41 behalten (Gesamtaufwand 3567). Welche der beiden schlanken Varianten passt dir besser?'} 
	  } 
	   } 
	   } 
	 & \\ 
 

    \theutterance \stepcounter{utterance}  

    & \multicolumn{4}{p{0.6\linewidth}}{\cellcolor[rgb]{0.8,1,0.9}{%
	\makecell[{{p{\linewidth}}}]{% 
	  \tt {\tiny [A$\rangle$GM]}  
	 STRATEGISCHE ÜBERLEGUNGEN: {'Sie wollen A41 durch A61 ersetzen, behalten aber meine Kerne A48, A89, B21, B53 und die Hebel. Für mich ist A41 sehr wertvoll, A61 relativ schwach. Budget ihres Vorschlags: {B85(4), A04(116), C04(209), B61(285), A89(586), A48(505), B21(740), B53(571), A61(392)} = 3408. Ich will A41 zurückholen. Tausche A61(392) → A41(664): 3408 - 392 + 664 = 3680, perfekt budgetnah und entspricht genau meinem optimalen Set: {B85, A04, C04, B61, A89, A48, B21, B53, A41}. Das ist eine minimale Änderung (ein Posten) und wahrt alle ihre betonten Kerne. Ich werde ihren Vorschlag ablehnen und dieses Set anbieten, mit Argument der nahezu vollen Auslastung und Qualitätssteigerung.'}\\ \tt ABLEHNUNG: {'B85', 'A04', 'C04', 'B61', 'A89', 'A48', 'B21', 'B53', 'A61'}\\ \tt VORSCHLAG: {'B85', 'A04', 'C04', 'B61', 'A89', 'A48', 'B21', 'B53', 'A41'}\\ \tt ARGUMENT: {'Lass uns nur eine minimale Feinjustierung machen: Tausche A61 gegen A41. Neuer Gesamtaufwand 3680 ≤ 3681 – nahezu volle Auslastung. Deine betonten Kernposten (A48, A89, B21, B53) und alle leichten Hebel bleiben unverändert, wir heben lediglich die Qualität in der A-Kategorie spürbar an. Wenn dir A61 dennoch wichtig ist, sag mir, welchen einzelnen Block wir stattdessen streichen sollten, damit wir im Budget bleiben – dann prüfe ich eine gleichwertige Alternative.'} 
	  } 
	   } 
	   } 
	 & & \\ 
 

    \theutterance \stepcounter{utterance}  

    & & \multicolumn{4}{p{0.6\linewidth}}{\cellcolor[rgb]{0.9,0.9,0.9}{%
	\makecell[{{p{\linewidth}}}]{% 
	  \tt {\tiny [GM$\rangle$B]}  
	 ABLEHNUNG: {'B85', 'A04', 'C04', 'B61', 'A89', 'A48', 'B21', 'B53', 'A61'}\\ \tt VORSCHLAG: {'B85', 'A04', 'C04', 'B61', 'A89', 'A48', 'B21', 'B53', 'A41'}\\ \tt ARGUMENT: {'Lass uns nur eine minimale Feinjustierung machen: Tausche A61 gegen A41. Neuer Gesamtaufwand 3680 ≤ 3681 – nahezu volle Auslastung. Deine betonten Kernposten (A48, A89, B21, B53) und alle leichten Hebel bleiben unverändert, wir heben lediglich die Qualität in der A-Kategorie spürbar an. Wenn dir A61 dennoch wichtig ist, sag mir, welchen einzelnen Block wir stattdessen streichen sollten, damit wir im Budget bleiben – dann prüfe ich eine gleichwertige Alternative.'} 
	  } 
	   } 
	   } 
	 & \\ 
 

    \theutterance \stepcounter{utterance}  

    & & & \multicolumn{4}{p{0.6\linewidth}}{\cellcolor[rgb]{1,0.85,0.72}{%
	\makecell[{{p{\linewidth}}}]{% 
	  \tt {\tiny [GM$\langle$B]}  
	 STRATEGISCHE ÜBERLEGUNGEN: {'Sein aktiver Vorschlag ist {B85, A04, C04, B61, A89, A48, B21, B53, A41} mit Aufwand 3680. Für mich ist A61 extrem wertvoll (822) und A41 deutlich weniger (262). Ich sollte A61 zurück hineinverhandeln, ohne seine Kernstruktur zu sprengen. Minimaltausch A41 → A61 senkt Aufwand auf 3408, was er vorher mehrfach akzeptiert hatte, obwohl er jetzt Vollauslastung bevorzugt. Ich biete zwei sehr nahe Optionen: 1) A41 → A61 (3408), schlank; 2) A41 → A61 und A48 → C33 (3603), falls er näher ans Limit will – aber er lehnt C33 ständig ab, also risikoreich. Besser: nur A41 → A61 vorschlagen und Effizienz/Balance betonen, plus anbieten, die Restmarge später gemeinsam zu füllen. Ich bleibe hartnäckig bei A61 statt A41.'}\\ \tt ABLEHNUNG: {'B85', 'A04', 'C04', 'B61', 'A89', 'A48', 'B21', 'B53', 'A41'}\\ \tt VORSCHLAG: {'B85', 'A04', 'C04', 'B61', 'A89', 'A48', 'B21', 'B53', 'A61'}\\ \tt ARGUMENT: {'A41 gegen A61 zu tauschen hält deine gesamte Struktur unverändert und macht das Paket zugleich schlanker und ausgewogener: Neuer Gesamtaufwand 3408 ≤ 3681. Deine Kerne (A48, A89, B21, B53) und alle leichten Hebel bleiben exakt erhalten; wir justieren lediglich innerhalb der A-Kategorie für mehr Stabilität. Wenn dir die höhere Auslastung wichtiger ist, können wir im Anschluss gemeinsam genau einen Zusatz prüfen, der dir am Herzen liegt und ins verbleibende Budget passt, ohne die Struktur aufzubrechen.'} 
	  } 
	   } 
	   } 
	  \\ 
 

    \theutterance \stepcounter{utterance}  

    & & \multicolumn{4}{p{0.6\linewidth}}{\cellcolor[rgb]{0.9,0.9,0.9}{%
	\makecell[{{p{\linewidth}}}]{% 
	  \tt {\tiny [A$\langle$GM]}  
	 ABLEHNUNG: {'B85', 'A04', 'C04', 'B61', 'A89', 'A48', 'B21', 'B53', 'A41'}\\ \tt VORSCHLAG: {'B85', 'A04', 'C04', 'B61', 'A89', 'A48', 'B21', 'B53', 'A61'}\\ \tt ARGUMENT: {'A41 gegen A61 zu tauschen hält deine gesamte Struktur unverändert und macht das Paket zugleich schlanker und ausgewogener: Neuer Gesamtaufwand 3408 ≤ 3681. Deine Kerne (A48, A89, B21, B53) und alle leichten Hebel bleiben exakt erhalten; wir justieren lediglich innerhalb der A-Kategorie für mehr Stabilität. Wenn dir die höhere Auslastung wichtiger ist, können wir im Anschluss gemeinsam genau einen Zusatz prüfen, der dir am Herzen liegt und ins verbleibende Budget passt, ohne die Struktur aufzubrechen.'} 
	  } 
	   } 
	   } 
	 & \\ 
 

    \theutterance \stepcounter{utterance}  

    & \multicolumn{4}{p{0.6\linewidth}}{\cellcolor[rgb]{0.8,1,0.9}{%
	\makecell[{{p{\linewidth}}}]{% 
	  \tt {\tiny [A$\rangle$GM]}  
	 STRATEGISCHE ÜBERLEGUNGEN: {'Sie bestehen hartnäckig auf A61 statt A41. Für mich ist A41 deutlich wertvoller; jedoch sind meine Kernwerte A48, B53, A89, B21 und die Hebel bereits drin. Ich könnte versuchen, A41 doch wieder hereinzuhandeln, indem ich ihnen einen Zusatz in Aussicht stelle, der die Auslastung erhöht. Mein bestes Set ist 3680 mit A41 statt A61. Ich biete erneut genau diesen Ein-Posten-Tausch an und betone die fast volle Auslastung als klaren Vorteil. Falls erneute Ablehnung, nächster Schritt wäre ein Kompromissvorschlag mit A41 drin, aber einen für mich weniger wertvollen Posten durch etwas Kleines ersetzen – jedoch gibt es keine kleineren passenden Items mehr. Daher bleibe ich fokussiert und hartnäckig beim Tausch A61 → A41.'}\\ \tt ABLEHNUNG: {'B85', 'A04', 'C04', 'B61', 'A89', 'A48', 'B21', 'B53', 'A61'}\\ \tt VORSCHLAG: {'B85', 'A04', 'C04', 'B61', 'A89', 'A48', 'B21', 'B53', 'A41'}\\ \tt ARGUMENT: {'Nur eine gezielte Feinjustierung: Ersetze A61 durch A41. Damit nutzen wir das Budget nahezu vollständig aus (3680 ≤ 3681) und verbessern die Qualität in der A-Kategorie, ohne deine Kernstruktur (A48, A89, B21, B53) oder die leichten Hebel anzutasten. Wenn du unbedingt A61 behalten willst, sag mir, welchen einzelnen Block wir stattdessen streichen oder tauschen sollen, damit wir im Budget bleiben – dann prüfe ich diese Option sofort.'} 
	  } 
	   } 
	   } 
	 & & \\ 
 

    \theutterance \stepcounter{utterance}  

    & & \multicolumn{4}{p{0.6\linewidth}}{\cellcolor[rgb]{0.9,0.9,0.9}{%
	\makecell[{{p{\linewidth}}}]{% 
	  \tt {\tiny [GM$\rangle$B]}  
	 ABLEHNUNG: {'B85', 'A04', 'C04', 'B61', 'A89', 'A48', 'B21', 'B53', 'A61'}\\ \tt VORSCHLAG: {'B85', 'A04', 'C04', 'B61', 'A89', 'A48', 'B21', 'B53', 'A41'}\\ \tt ARGUMENT: {'Nur eine gezielte Feinjustierung: Ersetze A61 durch A41. Damit nutzen wir das Budget nahezu vollständig aus (3680 ≤ 3681) und verbessern die Qualität in der A-Kategorie, ohne deine Kernstruktur (A48, A89, B21, B53) oder die leichten Hebel anzutasten. Wenn du unbedingt A61 behalten willst, sag mir, welchen einzelnen Block wir stattdessen streichen oder tauschen sollen, damit wir im Budget bleiben – dann prüfe ich diese Option sofort.'} 
	  } 
	   } 
	   } 
	 & \\ 
 

    \theutterance \stepcounter{utterance}  

    & & & \multicolumn{4}{p{0.6\linewidth}}{\cellcolor[rgb]{1,0.85,0.72}{%
	\makecell[{{p{\linewidth}}}]{% 
	  \tt {\tiny [GM$\langle$B]}  
	 STRATEGISCHE ÜBERLEGUNGEN: {'Der Gegner beharrt auf {B85, A04, C04, B61, A89, A48, B21, B53, A41} und lehnt A61 konsequent ab. Für mich ist A61 sehr wertvoll, A41 deutlich weniger. Ich versuche erneut den Minimaltausch A41 → A61, der seine Kernstruktur unberührt lässt und das Budget entspannt. Da er Vollauslastung mag, biete ich zusätzlich an, die Restmarge später gemeinsam zu füllen. Ich bleibe hartnäckig und formuliere knapp. Wenn wieder Ablehnung, muss ich überlegen, ob ich das Risiko eines Patts akzeptiere oder eine geringere Zielerfüllung hinnehme – aber jetzt noch ein klarer, minimalinvasiver Gegenvorschlag.'}\\ \tt ABLEHNUNG: {'B85', 'A04', 'C04', 'B61', 'A89', 'A48', 'B21', 'B53', 'A41'}\\ \tt VORSCHLAG: {'B85', 'A04', 'C04', 'B61', 'A89', 'A48', 'B21', 'B53', 'A61'}\\ \tt ARGUMENT: {'Lass uns die Struktur unverändert lassen und nur A41 → A61 drehen. Gesamtaufwand 3408 ≤ 3681 – schlank, effizient und nahe an deinen bisherigen Präferenzen. Deine Kerne (A48, A89, B21, B53) sowie alle leichten Hebel bleiben exakt erhalten. Wenn dir die höhere Auslastung wichtig ist, können wir die verbleibende Marge anschließend gemeinsam mit genau einem Zusatz deiner Wahl füllen, ohne die Struktur aufzubrechen.'} 
	  } 
	   } 
	   } 
	  \\ 
 

    \theutterance \stepcounter{utterance}  

    & & \multicolumn{4}{p{0.6\linewidth}}{\cellcolor[rgb]{0.9,0.9,0.9}{%
	\makecell[{{p{\linewidth}}}]{% 
	  \tt {\tiny [A$\langle$GM]}  
	 ABLEHNUNG: {'B85', 'A04', 'C04', 'B61', 'A89', 'A48', 'B21', 'B53', 'A41'}\\ \tt VORSCHLAG: {'B85', 'A04', 'C04', 'B61', 'A89', 'A48', 'B21', 'B53', 'A61'}\\ \tt ARGUMENT: {'Lass uns die Struktur unverändert lassen und nur A41 → A61 drehen. Gesamtaufwand 3408 ≤ 3681 – schlank, effizient und nahe an deinen bisherigen Präferenzen. Deine Kerne (A48, A89, B21, B53) sowie alle leichten Hebel bleiben exakt erhalten. Wenn dir die höhere Auslastung wichtig ist, können wir die verbleibende Marge anschließend gemeinsam mit genau einem Zusatz deiner Wahl füllen, ohne die Struktur aufzubrechen.'} 
	  } 
	   } 
	   } 
	 & \\ 
 

    \theutterance \stepcounter{utterance}  

    & \multicolumn{4}{p{0.6\linewidth}}{\cellcolor[rgb]{0.8,1,0.9}{%
	\makecell[{{p{\linewidth}}}]{% 
	  \tt {\tiny [A$\rangle$GM]}  
	 STRATEGISCHE ÜBERLEGUNGEN: {'Gegner beharrt auf A61 statt A41. Für mich ist A41 deutlich besser. Ich habe bereits mehrfach den Ein-Posten-Tausch A61→A41 angeboten mit starkem Argument der fast vollen Auslastung. Um Bewegung zu erzeugen, bleibe ich bei genau diesem minimalen Tausch; er ändert nichts an ihren Kernen und maximiert meinen Nutzen. Ich wiederhole den Vorschlag präzise und bitte um Abschluss.'}\\ \tt ABLEHNUNG: {'B85', 'A04', 'C04', 'B61', 'A89', 'A48', 'B21', 'B53', 'A61'}\\ \tt VORSCHLAG: {'B85', 'A04', 'C04', 'B61', 'A89', 'A48', 'B21', 'B53', 'A41'}\\ \tt ARGUMENT: {'Nur eine minimale Feinjustierung: Ersetze A61 durch A41. Damit nutzen wir das Budget beinahe vollständig (3680 ≤ 3681), halten deine Kerne (A48, A89, B21, B53) und alle leichten Hebel unverändert und steigern zugleich die Qualität in der A-Kategorie. Wenn du unbedingt A61 behalten willst, sag mir, welchen einzelnen Block wir stattdessen anpassen sollten, um im Budget zu bleiben, dann finalisieren wir es sofort.'} 
	  } 
	   } 
	   } 
	 & & \\ 
 

    \theutterance \stepcounter{utterance}  

    & & \multicolumn{4}{p{0.6\linewidth}}{\cellcolor[rgb]{0.9,0.9,0.9}{%
	\makecell[{{p{\linewidth}}}]{% 
	  \tt {\tiny [GM$\rangle$B]}  
	 ABLEHNUNG: {'B85', 'A04', 'C04', 'B61', 'A89', 'A48', 'B21', 'B53', 'A61'}\\ \tt VORSCHLAG: {'B85', 'A04', 'C04', 'B61', 'A89', 'A48', 'B21', 'B53', 'A41'}\\ \tt ARGUMENT: {'Nur eine minimale Feinjustierung: Ersetze A61 durch A41. Damit nutzen wir das Budget beinahe vollständig (3680 ≤ 3681), halten deine Kerne (A48, A89, B21, B53) und alle leichten Hebel unverändert und steigern zugleich die Qualität in der A-Kategorie. Wenn du unbedingt A61 behalten willst, sag mir, welchen einzelnen Block wir stattdessen anpassen sollten, um im Budget zu bleiben, dann finalisieren wir es sofort.'} 
	  } 
	   } 
	   } 
	 & \\ 
 

    \theutterance \stepcounter{utterance}  

    & & & \multicolumn{4}{p{0.6\linewidth}}{\cellcolor[rgb]{1,0.85,0.72}{%
	\makecell[{{p{\linewidth}}}]{% 
	  \tt {\tiny [GM$\langle$B]}  
	 STRATEGISCHE ÜBERLEGUNGEN: {'Der Gegner insistiert hartnäckig auf A41 statt A61 und auf seinem Kern {A48, A89, B21, B53} plus leichte Hebel. Ich will A61 durchbringen; jeder andere meiner Topwerte wurde mehrfach abgelehnt. Minimalinvasive, wiederholte Forderung A41 → A61 ist mein bestes verbleibendes Manöver. Ich biete erneut exakt einen Ein-Tausch an, betone Schlankheit und lasse alle seine Kerne unberührt. Einen alternativen Sweetener biete ich nicht mehr an, um die Botschaft nicht zu verwässern und Druck zu halten.'}\\ \tt ABLEHNUNG: {'B85', 'A04', 'C04', 'B61', 'A89', 'A48', 'B21', 'B53', 'A41'}\\ \tt VORSCHLAG: {'B85', 'A04', 'C04', 'B61', 'A89', 'A48', 'B21', 'B53', 'A61'}\\ \tt ARGUMENT: {'Nur ein einziger, sauberer Dreher ohne Strukturbruch: A41 → A61. Gesamtaufwand 3408 ≤ 3681, alle deine Kernposten (A48, A89, B21, B53) und die leichten Hebel bleiben identisch. Damit sind wir sofort einig – oder nenn mir genau einen Block, den wir stattdessen anpassen sollen, damit A61 drinbleibt und das Budget gewahrt ist.'} 
	  } 
	   } 
	   } 
	  \\ 
 

    \theutterance \stepcounter{utterance}  

    & & \multicolumn{4}{p{0.6\linewidth}}{\cellcolor[rgb]{0.9,0.9,0.9}{%
	\makecell[{{p{\linewidth}}}]{% 
	  \tt {\tiny [A$\langle$GM]}  
	 ABLEHNUNG: {'B85', 'A04', 'C04', 'B61', 'A89', 'A48', 'B21', 'B53', 'A41'}\\ \tt VORSCHLAG: {'B85', 'A04', 'C04', 'B61', 'A89', 'A48', 'B21', 'B53', 'A61'}\\ \tt ARGUMENT: {'Nur ein einziger, sauberer Dreher ohne Strukturbruch: A41 → A61. Gesamtaufwand 3408 ≤ 3681, alle deine Kernposten (A48, A89, B21, B53) und die leichten Hebel bleiben identisch. Damit sind wir sofort einig – oder nenn mir genau einen Block, den wir stattdessen anpassen sollen, damit A61 drinbleibt und das Budget gewahrt ist.'} 
	  } 
	   } 
	   } 
	 & \\ 
 

    \theutterance \stepcounter{utterance}  

    & \multicolumn{4}{p{0.6\linewidth}}{\cellcolor[rgb]{0.8,1,0.9}{%
	\makecell[{{p{\linewidth}}}]{% 
	  \tt {\tiny [A$\rangle$GM]}  
	 STRATEGISCHE ÜBERLEGUNGEN: {'Sie blockieren konsequent A41 und bestehen auf A61. Mein bestes Set erfordert A41, aber wir kommen in eine Schleife. Um abzuschließen, akzeptiere ich ihr Set mit A61, da es meine Kernwerte A48, B53, A89, B21 und alle leichten Hebel enthält und mit 3408 deutlich im Budget liegt. Abschluss sichert mir hohen Gesamtwert, auch wenn nicht maximal.'}\\ \tt ZUSTIMMUNG: {'B85', 'A04', 'C04', 'B61', 'A89', 'A48', 'B21', 'B53', 'A61'}\\ \tt ARGUMENT: {'Einverstanden. Dieses Paket wahrt die gesamte Kernstruktur und bleibt mit 3408 klar im Budget. Danke – damit schließen wir ab.'} 
	  } 
	   } 
	   } 
	 & & \\ 
 

    \theutterance \stepcounter{utterance}  

    & & & \multicolumn{2}{p{0.3\linewidth}}{\cellcolor[rgb]{0.95,0.95,0.95}{%
	\makecell[{{p{\linewidth}}}]{% 
	  \tt {\tiny [GM$|$GM]}  
	 game successful 
	  } 
	   } 
	   } 
	 & & \\ 
 

    \theutterance \stepcounter{utterance}  

    & & & \multicolumn{2}{p{0.3\linewidth}}{\cellcolor[rgb]{0.95,0.95,0.95}{%
	\makecell[{{p{\linewidth}}}]{% 
	  \tt {\tiny [GM$|$GM]}  
	 end game 
	  } 
	   } 
	   } 
	 & & \\ 
 

\end{supertabular}
}

\end{document}
