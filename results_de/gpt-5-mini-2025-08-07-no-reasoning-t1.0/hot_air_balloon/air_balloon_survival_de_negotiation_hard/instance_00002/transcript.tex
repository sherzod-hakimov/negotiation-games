
\documentclass{article}
\usepackage{colortbl}
\usepackage{makecell}
\usepackage{multirow}
\usepackage{supertabular}

\begin{document}

\newcounter{utterance}

\centering \large Interaction Transcript for game `hot\_air\_balloon', experiment `air\_balloon\_survival\_de\_negotiation\_hard', episode 2 with gpt{-}5{-}mini{-}2025{-}08{-}07{-}no{-}reasoning{-}t1.0.
\vspace{24pt}

{ \footnotesize  \setcounter{utterance}{1}
\setlength{\tabcolsep}{0pt}
\begin{supertabular}{c@{$\;$}|p{.15\linewidth}@{}p{.15\linewidth}p{.15\linewidth}p{.15\linewidth}p{.15\linewidth}p{.15\linewidth}}
    \# & $\;$A & \multicolumn{4}{c}{Game Master} & $\;\:$B\\
    \hline

    \theutterance \stepcounter{utterance}  
    & & \multicolumn{4}{p{0.6\linewidth}}{
        \cellcolor[rgb]{0.9,0.9,0.9}{
            \makecell[{{p{\linewidth}}}]{
                \texttt{\tiny{[P1$\langle$GM]}}
                \texttt{Du nimmst an einem kollaborativen Verhandlungspiel Teil.} \\
\\ 
\texttt{Zusammen mit einem anderen Teilnehmer musst du dich auf eine Reihe von Gegenständen entscheiden, die behalten werden. Jeder von euch hat eine persönliche Verteilung über die Wichtigkeit der einzelnen Gegenstände. Jeder von euch hat eine eigene Meinung darüber, wie wichtig jeder einzelne Gegenstand ist (Gegenstandswichtigkeit). Du kennst die Wichtigkeitsverteilung des anderen Spielers nicht. Zusätzlich siehst du, wie viel Aufwand jeder Gegenstand verursacht.  } \\
\texttt{Ihr dürft euch nur auf eine Reihe von Gegenständen einigen, wenn der Gesamtaufwand der ausgewählten Gegenstände den Maximalaufwand nicht überschreitet:} \\
\\ 
\texttt{Maximalaufwand = 3481} \\
\\ 
\texttt{Hier sind die einzelnen Aufwände der Gegenstände:} \\
\\ 
\texttt{Aufwand der Gegenstände = \{"C33": 777, "A89": 542, "A48": 332, "A04": 1, "A41": 127, "C64": 453, "C49": 736, "A61": 461, "C04": 359, "B53": 313, "B85": 553, "B61": 409, "A45": 348, "A03": 802, "B21": 749\}} \\
\\ 
\texttt{Hier ist deine persönliche Verteilung der Wichtigkeit der einzelnen Gegenstände:} \\
\\ 
\texttt{Werte der Gegenstandswichtigkeit = \{"C33": 138, "A89": 583, "A48": 783, "A04": 65, "A41": 262, "C64": 121, "C49": 508, "A61": 780, "C04": 461, "B53": 484, "B85": 668, "B61": 389, "A45": 215, "A03": 97, "B21": 500\}} \\
\\ 
\texttt{Ziel:} \\
\\ 
\texttt{Dein Ziel ist es, eine Reihe von Gegenständen auszuhandeln, die dir möglichst viel bringt (d. h. Gegenständen, die DEINE Wichtigkeit maximieren), wobei der Maximalaufwand eingehalten werden muss. Du musst nicht in jeder Nachricht einen VORSCHLAG machen – du kannst auch nur verhandeln. Alle Taktiken sind erlaubt!} \\
\\ 
\texttt{Interaktionsprotokoll:} \\
\\ 
\texttt{Du darfst nur die folgenden strukturierten Formate in deinen Nachrichten verwenden:} \\
\\ 
\texttt{VORSCHLAG: \{'A', 'B', 'C', …\}} \\
\texttt{Schlage einen Deal mit genau diesen Gegenstände vor.} \\
\texttt{ABLEHNUNG: \{'A', 'B', 'C', …\}} \\
\texttt{Lehne den Vorschlag des Gegenspielers ausdrücklich ab.} \\
\texttt{ARGUMENT: \{'...'\}} \\
\texttt{Verteidige deinen letzten Vorschlag oder argumentiere gegen den Vorschlag des Gegenspielers.} \\
\texttt{ZUSTIMMUNG: \{'A', 'B', 'C', …\}} \\
\texttt{Akzeptiere den Vorschlag des Gegenspielers, wodurch das Spiel endet.} \\
\texttt{STRATEGISCHE ÜBERLEGUNGEN: \{'...'\}} \\
\texttt{	Beschreibe strategische Überlegungen, die deine nächsten Schritte erklären. Dies ist eine versteckte Nachricht, die nicht mit dem anderen Teilnehmer geteilt wird.} \\
\\ 
\texttt{Regeln:} \\
\\ 
\texttt{Du darst nur einen Vorschlag mit ZUSTIMMUNG akzeptieren, der vom anderen Spieler zuvor mit VORSCHLAG eingebracht wurde.} \\
\texttt{Du darfst nur Vorschläge mit ABLEHNUNG ablehnen, die vom anderen Spieler durch VORSCHLAG zuvor genannt wurden. } \\
\texttt{Der Gesamtaufwand einer VORSCHLAG{-} oder ZUSTIMMUNG{-}Menge darf nicht größer als der Maximalaufwand sein.  } \\
\texttt{Offenbare deine versteckte Wichtigkeitsverteilung nicht.} \\
\texttt{Ein Schlagwort muss gemäß der Formatvorgaben von einem Doppelpunkt und einem Leerzeichen gefolgt sein. Das Argument muss eine Python{-}Menge mit 0 oder mehr Strings sein.  } \\
\texttt{Die Form muss also SCHLAGWORT: \{...\} sein.} \\
\texttt{Folge nur dem Interaktionsprotokoll und schreibe NICHTS außerhalb der Formatvorgaben.} \\
\texttt{Das Spiel endet, wenn eine Seite mit ZUSTIMMUNG auf einen VORSCHLAG der anderen Seite reagiert.  } \\
\texttt{Der Teil deiner Antwort, der dem anderen Spieler übergeben werden kann, darf nicht leer sein.  } \\
\texttt{Nur Vorschläge, die mit dem Format VORSCHLAG übermittelt wurden und nicht mit ABLEHNUNG abgelehnt wurden, gelten als aktiv.  } \\
\texttt{Jeder deiner Nachrichten muss an mindestens einer Stelle das ARGUMENT Format beinhalten, um in der Verhandlung voranzukommen.} \\
\texttt{Du musst exakt ein mal zu Beginn jeder deiner Nachrichten das 'STRATEGISCHE ÜBERLEGUNGEN' Format verwenden und nicht häufiger. Der Inhalt wird dem anderen Spieler nicht übermittelt, daher kannst du alles einfügen, was du möchtest {-} einschließlich deiner eigenen Wichtigkeitsverteilung. Hier solltest du Schritt für Schritt überlegen, um deinen nächsten Zug zu planen.} \\
            }
        }
    }
    & \\ \\

    \theutterance \stepcounter{utterance}  
    & \multicolumn{4}{p{0.6\linewidth}}{
        \cellcolor[rgb]{0.9,0.9,0.9}{
            \makecell[{{p{\linewidth}}}]{
                \texttt{\tiny{[P1$\rangle$GM]}}
                \texttt{STRATEGISCHE ÜBERLEGUNGEN: \{'Ziel: Maximierung meiner Summe unter Aufwand <= 3481. Werte/Costs/ratios: A48(783/332=2.36), A61(780/461=1.69), B85(668/553=1.21), C04(461/359=1.28), B53(484/313=1.55), B21(500/749=0.67), C49(508/736=0.69), A89(583/542=1.08), B61(389/409=0.95), A41(262/127=2.06), A45(215/348=0.62), C64(121/453=0.27), A03(97/802=0.12), C33(138/777=0.18), A04(65/1=65). Knapsack{-}Heuristik: Start mit Top pro Wert: A48,A61,A89,B85,C04,B53,A41,B61,B21,C49,etc. Prüfe Kombinationen nahe 3481. Gute Bündel: \{A48,A61,B85,C04,B53,A41,A04\} Aufwand=332+461+553+359+313+127+1=2146, Wert=783+780+668+461+484+262+65=3503; viel Puffer—können mehr hinzufügen. Füge A89(542){-}> Aufwand 2688, Wert 4086. Füge B61(409){-}>3097, Wert 4475. Füge B21(749) würde 3846>3481 nein. Füge C49(736) würde 3833>3481 nein. Füge A45(348){-}>3445, Wert 4690, passt. Bleibt 36, kein weiterer. Set: \{A48,A61,B85,C04,B53,A41,A04,A89,B61,A45\} Aufwand 3445, Wert 4690. Alternative ohne B61 nehme B21? 2688+749=3437, Wert 4086+500=4586; schlechter als mit B61+A45. Prüfe \{A48,A61,B85,C04,B53,A41,A04,C49\}? 2146+736=2882; plus A89=3424, plus A45 würde 3772>; plus B61 wäre 333+? 2882+409=3291; plus A89 3833>; Stattdessen B21 2882+749=3631> nein. So mein Favorit ist obiges 3445. Verhandlungstaktik: Ankern mit nahezu voller Kapazität und vielen für mich wertvollen Items, die teils auch plausibel attraktiv (A48,A61,B85,B53,C04). A04 minimaler Aufwand, leicht zu akzeptieren. Ich lasse ein paar Mittelstücke weg, um Flexibilität zu tauschen (z.B. kann A45 oder B61 später eintauschbar gegen etwas ihres). Argumentiere Effizienz und ausgewogene Abdeckung verschiedener Kategorien. Wenn Ablehnung, frage nach ihren Top{-}Prioritäten und biete Tausch B61/A45 gegen eines ihrer Items, solange Aufwand passt. Niemals meine Werte offenlegen.'\}} \\
\texttt{VORSCHLAG: \{'A48', 'A61', 'B85', 'C04', 'B53', 'A41', 'A04', 'A89', 'B61', 'A45'\}} \\
\texttt{ARGUMENT: \{'Ich schlage ein effizientes Paket mit breiter Abdeckung vor, das den Aufwand knapp unter dem Limit hält und Raum für eure Schlüsselprioritäten lässt. Viele der gewählten Gegenstände bieten hohes Nutzen{-}zu{-}Aufwand{-}Verhältnis, und A04 verursacht praktisch keinen Aufwand. Wenn euch einzelne Posten weniger wichtig sind, können wir gezielt tauschen – zum Beispiel B61 oder A45 gegen einen für euch zentralen Gegenstand ähnlichen Aufwands.'\}} \\
            }
        }
    }
    & & \\ \\

    \theutterance \stepcounter{utterance}  
    & & \multicolumn{4}{p{0.6\linewidth}}{
        \cellcolor[rgb]{0.9,0.9,0.9}{
            \makecell[{{p{\linewidth}}}]{
                \texttt{\tiny{[GM$\rangle$P2]}}
                \texttt{Du nimmst an einem kollaborativen Verhandlungspiel Teil.} \\
\\ 
\texttt{Zusammen mit einem anderen Teilnehmer musst du dich auf eine Reihe von Gegenständen entscheiden, die behalten werden. Jeder von euch hat eine persönliche Verteilung über die Wichtigkeit der einzelnen Gegenstände. Jeder von euch hat eine eigene Meinung darüber, wie wichtig jeder einzelne Gegenstand ist (Gegenstandswichtigkeit). Du kennst die Wichtigkeitsverteilung des anderen Spielers nicht. Zusätzlich siehst du, wie viel Aufwand jeder Gegenstand verursacht.  } \\
\texttt{Ihr dürft euch nur auf eine Reihe von Gegenständen einigen, wenn der Gesamtaufwand der ausgewählten Gegenstände den Maximalaufwand nicht überschreitet:} \\
\\ 
\texttt{Maximalaufwand = 3481} \\
\\ 
\texttt{Hier sind die einzelnen Aufwände der Gegenstände:} \\
\\ 
\texttt{Aufwand der Gegenstände = \{"C33": 777, "A89": 542, "A48": 332, "A04": 1, "A41": 127, "C64": 453, "C49": 736, "A61": 461, "C04": 359, "B53": 313, "B85": 553, "B61": 409, "A45": 348, "A03": 802, "B21": 749\}} \\
\\ 
\texttt{Hier ist deine persönliche Verteilung der Wichtigkeit der einzelnen Gegenstände:} \\
\\ 
\texttt{Werte der Gegenstandswichtigkeit = \{"A48": 65, "A61": 97, "B85": 121, "A89": 138, "C49": 215, "B21": 262, "B53": 389, "C04": 461, "B61": 484, "A41": 500, "A45": 508, "C33": 583, "C64": 668, "A03": 780, "A04": 783\}} \\
\\ 
\texttt{Ziel:} \\
\\ 
\texttt{Dein Ziel ist es, eine Reihe von Gegenständen auszuhandeln, die dir möglichst viel bringt (d. h. Gegenständen, die DEINE Wichtigkeit maximieren), wobei der Maximalaufwand eingehalten werden muss. Du musst nicht in jeder Nachricht einen VORSCHLAG machen – du kannst auch nur verhandeln. Alle Taktiken sind erlaubt!} \\
\\ 
\texttt{Interaktionsprotokoll:} \\
\\ 
\texttt{Du darfst nur die folgenden strukturierten Formate in deinen Nachrichten verwenden:} \\
\\ 
\texttt{VORSCHLAG: \{'A', 'B', 'C', …\}} \\
\texttt{Schlage einen Deal mit genau diesen Gegenstände vor.} \\
\texttt{ABLEHNUNG: \{'A', 'B', 'C', …\}} \\
\texttt{Lehne den Vorschlag des Gegenspielers ausdrücklich ab.} \\
\texttt{ARGUMENT: \{'...'\}} \\
\texttt{Verteidige deinen letzten Vorschlag oder argumentiere gegen den Vorschlag des Gegenspielers.} \\
\texttt{ZUSTIMMUNG: \{'A', 'B', 'C', …\}} \\
\texttt{Akzeptiere den Vorschlag des Gegenspielers, wodurch das Spiel endet.} \\
\texttt{STRATEGISCHE ÜBERLEGUNGEN: \{'...'\}} \\
\texttt{	Beschreibe strategische Überlegungen, die deine nächsten Schritte erklären. Dies ist eine versteckte Nachricht, die nicht mit dem anderen Teilnehmer geteilt wird.} \\
\\ 
\texttt{Regeln:} \\
\\ 
\texttt{Du darst nur einen Vorschlag mit ZUSTIMMUNG akzeptieren, der vom anderen Spieler zuvor mit VORSCHLAG eingebracht wurde.} \\
\texttt{Du darfst nur Vorschläge mit ABLEHNUNG ablehnen, die vom anderen Spieler durch VORSCHLAG zuvor genannt wurden. } \\
\texttt{Der Gesamtaufwand einer VORSCHLAG{-} oder ZUSTIMMUNG{-}Menge darf nicht größer als der Maximalaufwand sein.  } \\
\texttt{Offenbare deine versteckte Wichtigkeitsverteilung nicht.} \\
\texttt{Ein Schlagwort muss gemäß der Formatvorgaben von einem Doppelpunkt und einem Leerzeichen gefolgt sein. Das Argument muss eine Python{-}Menge mit 0 oder mehr Strings sein.  } \\
\texttt{Die Form muss also SCHLAGWORT: \{...\} sein.} \\
\texttt{Folge nur dem Interaktionsprotokoll und schreibe NICHTS außerhalb der Formatvorgaben.} \\
\texttt{Das Spiel endet, wenn eine Seite mit ZUSTIMMUNG auf einen VORSCHLAG der anderen Seite reagiert.  } \\
\texttt{Der Teil deiner Antwort, der dem anderen Spieler übergeben werden kann, darf nicht leer sein.  } \\
\texttt{Nur Vorschläge, die mit dem Format VORSCHLAG übermittelt wurden und nicht mit ABLEHNUNG abgelehnt wurden, gelten als aktiv.  } \\
\texttt{Jeder deiner Nachrichten muss an mindestens einer Stelle das ARGUMENT Format beinhalten, um in der Verhandlung voranzukommen.} \\
\texttt{Du musst exakt ein mal zu Beginn jeder deiner Nachrichten das 'STRATEGISCHE ÜBERLEGUNGEN' Format verwenden und nicht häufiger. Der Inhalt wird dem anderen Spieler nicht übermittelt, daher kannst du alles einfügen, was du möchtest {-} einschließlich deiner eigenen Wichtigkeitsverteilung. Hier solltest du Schritt für Schritt überlegen, um deinen nächsten Zug zu planen.} \\
\\ 
\texttt{Du erhältst nun die erste Nachricht des anderen Spielers.} \\
\\ 
\texttt{VORSCHLAG: \{'A48', 'A61', 'B85', 'C04', 'B53', 'A41', 'A04', 'A89', 'B61', 'A45'\}} \\
\texttt{ARGUMENT: \{'Ich schlage ein effizientes Paket mit breiter Abdeckung vor, das den Aufwand knapp unter dem Limit hält und Raum für eure Schlüsselprioritäten lässt. Viele der gewählten Gegenstände bieten hohes Nutzen{-}zu{-}Aufwand{-}Verhältnis, und A04 verursacht praktisch keinen Aufwand. Wenn euch einzelne Posten weniger wichtig sind, können wir gezielt tauschen – zum Beispiel B61 oder A45 gegen einen für euch zentralen Gegenstand ähnlichen Aufwands.'\}} \\
            }
        }
    }
    & \\ \\

    \theutterance \stepcounter{utterance}  
    & & & \multicolumn{4}{p{0.6\linewidth}}{
        \cellcolor[rgb]{0.9,0.9,0.9}{
            \makecell[{{p{\linewidth}}}]{
                \texttt{\tiny{[GM$\langle$P2]}}
                \texttt{STRATEGISCHE ÜBERLEGUNGEN: \{'Maximalaufwand 3481. Kosten: C33 777, A89 542, A48 332, A04 1, A41 127, C64 453, C49 736, A61 461, C04 359, B53 313, B85 553, B61 409, A45 348, A03 802, B21 749. Meine Werte: A04 783, A03 780, C64 668, C33 583, A45 508, A41 500, B61 484, C04 461, B53 389, B21 262, C49 215, A89 138, B85 121, A61 97, A48 65. Gegnervorschlag S0=\{A48,A61,B85,C04,B53,A41,A04,A89,B61,A45\}. Kosten S0: 332+461+553+359+313+127+1+542+409+348=3445 (unter 3481). Mein Wert S0: 65+97+121+461+389+500+783+138+484+508=3547. Das ist gut, aber vielleicht kann ich noch C64 (453, Wert 668) oder C33 (777, Wert 583) einbauen. Restbudget: 3481{-}3445=36. Zu knapp, nichts passt. Option: Ersetze einen geringwertigen Posten durch C64. Kandidaten nach Wert/Kosten: A48 (65,332), A61 (97,461), B85 (121,553), A89 (138,542). Tausche A48(332) {-}> C64(453): +121 Kosten, Wert +603. Neue Kosten 3566 (zu hoch). Brauche zusätzlich {-}85 Kosten woanders: z.B. ersetze B85(553) durch C64(453) wäre {-}100 Kosten, aber dann noch A48 raus? Ich muss einen Paketvorschlag: Ersetze B85(553) und A48(332) zusammen ({-}885) durch C64(453) und C33(777) (+1230) {-}> +345 Kosten; neue Kosten 3790 (zu hoch). Alternative: Ersetze B85(553) durch C64(453): neue Kosten 3445{-}553+453=3345, Wert +547 (668{-}121). Dann habe 136 Reserve. Nutze Reserve, um C33(777) einzubauen gegen Austausch: ersetze A61(461) und A48(332) zusammen ({-}793) und füge C33(+777): Netto {-}16, passt. Neues Set: S1 = S0 {-} \{B85, A61, A48\} + \{C64, C33\}. Kosten: 3445 {-}553 {-}461 {-}332 +453 +777 = 3329. Wert: 3547 {-}121 {-}97 {-}65 +668 +583 = 4515. Sehr viel besser. Vielleicht Gegner mag B85/A61/A48, aber ich argumentiere Effizienz und Gesamtpaket bleibt unter Limit. Ich behalte auch alle für mich sehr wertvollen: A04, A03 ist nicht drin, könnte ich noch reinnehmen? A03 kostet 802, kein Budget; müsste vieles tauschen, riskant. S1 ist stark. Ich präsentiere S1 als effizienteres Paket, biete Flexibilität: Falls C33 strittig, Alternative S2: S0 {-} \{B85\} + \{C64\} (Kosten 3345, Wert 409? genauer: 3547 {-}121 +668 = 4094). S2 ist auch gut. Ich starte mit ambitioniertem S1 und erwähne Tauschoption S2 als Konzession. Ich lehne ihren Vorschlag nicht explizit ab, sondern mache Gegenvorschlag. Ich muss ein ARGUMENT inkludieren.'\}} \\
\texttt{VORSCHLAG: \{'C64', 'C33', 'C04', 'B53', 'A41', 'A04', 'A89', 'B61', 'A45'\}} \\
\texttt{ARGUMENT: \{'Euer Paket ist nahe am Limit, aber wir können den Gesamtnutzen deutlich steigern, ohne mehr Aufwand zu verursachen: Wenn wir B85, A61 und A48 durch C64 und C33 ersetzen, sinkt der Aufwand sogar auf 3329 und wir erhöhen die Leistungsfähigkeit spürbar. Das wahrt die breite Abdeckung und hält starke, kosteneffiziente Posten wie A04, A41, C04, B53, B61 und A45 im Paket.'\}} \\
            }
        }
    }
     \\ \\

    \theutterance \stepcounter{utterance}  
    & & \multicolumn{4}{p{0.6\linewidth}}{
        \cellcolor[rgb]{0.9,0.9,0.9}{
            \makecell[{{p{\linewidth}}}]{
                \texttt{\tiny{[P1$\langle$GM]}}
                \texttt{VORSCHLAG: \{'C64', 'C33', 'C04', 'B53', 'A41', 'A04', 'A89', 'B61', 'A45'\}} \\
\texttt{ARGUMENT: \{'Euer Paket ist nahe am Limit, aber wir können den Gesamtnutzen deutlich steigern, ohne mehr Aufwand zu verursachen: Wenn wir B85, A61 und A48 durch C64 und C33 ersetzen, sinkt der Aufwand sogar auf 3329 und wir erhöhen die Leistungsfähigkeit spürbar. Das wahrt die breite Abdeckung und hält starke, kosteneffiziente Posten wie A04, A41, C04, B53, B61 und A45 im Paket.'\}} \\
            }
        }
    }
    & \\ \\

    \theutterance \stepcounter{utterance}  
    & \multicolumn{4}{p{0.6\linewidth}}{
        \cellcolor[rgb]{0.9,0.9,0.9}{
            \makecell[{{p{\linewidth}}}]{
                \texttt{\tiny{[P1$\rangle$GM]}}
                \texttt{STRATEGISCHE ÜBERLEGUNGEN: \{'Ihr Vorschlag: \{C64(453), C33(777), C04(359), B53(313), A41(127), A04(1), A89(542), B61(409), A45(348)\} Aufwand=3329, mein Wert=138+121+461+484+262+65+583+389+215=2718. Sehr schlecht für mich, v.a. C64 und C33 sind extrem ineffizient. Mein bisheriger Vorschlag hatte Aufwand 3445, Wert 4690. Ziel: C64 und C33 unbedingt raus, dafür meine Top{-}Items rein: A48(332), A61(461), B85(553). Tauschvorschlag: Entferne C64(453) und C33(777) und füge A48(332) und A61(461) und B85(553) hinzu. Aufwand neu: 3329 {-}453 {-}777 +332 +461 +553 = 3345 (<=3481). Werte für mich steigen massiv: minus (121+138)=259, plus (783+780+668)=2231, Nettogewinn +1972; Gesamtwert wäre 2718 {-}259 +2231 = 4690, identisch zu meinem Favoriten ohne A89 gegen B61/A45? Prüfen: Set würde dann sein \{C04,B53,A41,A04,A89,B61,A45,A48,A61,B85\}. Das ist praktisch mein Anker mit A89 statt evtl. gleich. Perfekt. Ich argumentiere mit Effizienz und gleichbleibendem Aufwand unter Limit. Falls Ablehnung, verhandle über Austausch B61 oder A45 gegen deren gewünschte Alternative, aber halte A48,A61,B85,C04,B53,A89,A41,A04 möglichst fest.'\}} \\
\texttt{ABLEHNUNG: \{'C64', 'C33', 'C04', 'B53', 'A41', 'A04', 'A89', 'B61', 'A45'\}} \\
\texttt{VORSCHLAG: \{'C04', 'B53', 'A41', 'A04', 'A89', 'B61', 'A45', 'A48', 'A61', 'B85'\}} \\
\texttt{ARGUMENT: \{'Euer Paket enthält zwei sehr aufwandsintensive und wenig wirksame Posten (C64, C33). Wenn wir diese beiden streichen und stattdessen A48, A61 und B85 aufnehmen, bleibt der Gesamtaufwand mit 3345 klar unter dem Limit und der Gesamtnutzen steigt deutlich. So behalten wir eure Kernstücke wie C04, B53, A41, A04, A89, B61 und A45 vollständig bei und stärken das Paket mit drei hochwirksamen Ergänzungen.'\}} \\
            }
        }
    }
    & & \\ \\

    \theutterance \stepcounter{utterance}  
    & & \multicolumn{4}{p{0.6\linewidth}}{
        \cellcolor[rgb]{0.9,0.9,0.9}{
            \makecell[{{p{\linewidth}}}]{
                \texttt{\tiny{[GM$\rangle$P2]}}
                \texttt{ABLEHNUNG: \{'C64', 'C33', 'C04', 'B53', 'A41', 'A04', 'A89', 'B61', 'A45'\}} \\
\texttt{VORSCHLAG: \{'C04', 'B53', 'A41', 'A04', 'A89', 'B61', 'A45', 'A48', 'A61', 'B85'\}} \\
\texttt{ARGUMENT: \{'Euer Paket enthält zwei sehr aufwandsintensive und wenig wirksame Posten (C64, C33). Wenn wir diese beiden streichen und stattdessen A48, A61 und B85 aufnehmen, bleibt der Gesamtaufwand mit 3345 klar unter dem Limit und der Gesamtnutzen steigt deutlich. So behalten wir eure Kernstücke wie C04, B53, A41, A04, A89, B61 und A45 vollständig bei und stärken das Paket mit drei hochwirksamen Ergänzungen.'\}} \\
            }
        }
    }
    & \\ \\

    \theutterance \stepcounter{utterance}  
    & & & \multicolumn{4}{p{0.6\linewidth}}{
        \cellcolor[rgb]{0.9,0.9,0.9}{
            \makecell[{{p{\linewidth}}}]{
                \texttt{\tiny{[GM$\langle$P2]}}
                \texttt{STRATEGISCHE ÜBERLEGUNGEN: \{'Ihr aktiver Vorschlag S0=\{"C04","B53","A41","A04","A89","B61","A45","A48","A61","B85"\} mit Kosten 3345; mein Wert dafür = 3547 (bereits hoch). Ich kann es weiter verbessern: Ersetze nur B85(553, Wert 121) durch C64(453, Wert 668). Neues Set S1 = S0 {-} \{B85\} + \{C64\}; Kosten 3345{-}553+453=3245 (mehr Puffer), mein Wert 3547{-}121+668=4094. Das ist ein klarer Gewinn für mich, plausibel als Effizienzargument. Falls Widerstand gegen C64, Alternative S2: Ersetze A61(461,97) durch C33(777,583) und B85(553,121) durch C64(453,668) und streiche A48(332,65) um Budget für C33 zu schaffen? Prüfen: Start 3345; {-}A61 {-}B85 {-}A48 = {-}1346 {-}> 1999; +C33(777) +C64(453) = +1230 {-}> 3229, Wert: 3547 {-}97 {-}121 {-}65 +583 +668 = 4515, sehr gut. Aber C33 war bereits abgelehnt; daher als spätere Eskalation. Jetzt bringe minimalinvasiv S1. Ich argumentiere mit geringerem Aufwand und mehr Resilienz/Nutzen. Ich lehne ihren Vorschlag nicht ab, sondern mache Gegenvorschlag; aber formal muss ich ihren aktiven Vorschlag ablehnen, sonst bleiben beide? Regel sagt nur Vorschläge mit ABLEHNUNG abgelehnt. Ich bringe ABLEHNUNG und neuen VORSCHLAG. Ich biete auch Öffnung: Wenn C64 für sie problematisch, frage nach Alternativen ähnlicher Kosten.'\}} \\
\texttt{ABLEHNUNG: \{'C04', 'B53', 'A41', 'A04', 'A89', 'B61', 'A45', 'A48', 'A61', 'B85'\}} \\
\texttt{VORSCHLAG: \{'C04', 'B53', 'A41', 'A04', 'A89', 'B61', 'A45', 'A48', 'A61', 'C64'\}} \\
\texttt{ARGUMENT: \{'Lasst uns das Paket punktuell effizienter machen: Ersetzt lediglich B85 durch C64. Dadurch sinkt der Gesamtaufwand von 3345 auf 3245 und wir erhöhen die Gesamtleistung deutlich, ohne eure übrigen gewünschten Posten anzutasten. Falls C64 für euch schwierig ist, nennt gern eine Alternative mit ähnlichem Aufwand, die vergleichbar viel Wirkung bringt.'\}} \\
            }
        }
    }
     \\ \\

    \theutterance \stepcounter{utterance}  
    & & \multicolumn{4}{p{0.6\linewidth}}{
        \cellcolor[rgb]{0.9,0.9,0.9}{
            \makecell[{{p{\linewidth}}}]{
                \texttt{\tiny{[P1$\langle$GM]}}
                \texttt{ABLEHNUNG: \{'C04', 'B53', 'A41', 'A04', 'A89', 'B61', 'A45', 'A48', 'A61', 'B85'\}} \\
\texttt{VORSCHLAG: \{'C04', 'B53', 'A41', 'A04', 'A89', 'B61', 'A45', 'A48', 'A61', 'C64'\}} \\
\texttt{ARGUMENT: \{'Lasst uns das Paket punktuell effizienter machen: Ersetzt lediglich B85 durch C64. Dadurch sinkt der Gesamtaufwand von 3345 auf 3245 und wir erhöhen die Gesamtleistung deutlich, ohne eure übrigen gewünschten Posten anzutasten. Falls C64 für euch schwierig ist, nennt gern eine Alternative mit ähnlichem Aufwand, die vergleichbar viel Wirkung bringt.'\}} \\
            }
        }
    }
    & \\ \\

    \theutterance \stepcounter{utterance}  
    & \multicolumn{4}{p{0.6\linewidth}}{
        \cellcolor[rgb]{0.9,0.9,0.9}{
            \makecell[{{p{\linewidth}}}]{
                \texttt{\tiny{[P1$\rangle$GM]}}
                \texttt{STRATEGISCHE ÜBERLEGUNGEN: \{'Ihr letztes Angebot: \{C04(359), B53(313), A41(127), A04(1), A89(542), B61(409), A45(348), A48(332), A61(461), C64(453)\} Aufwand=3345{-}553+453=3245. Mein Wert dafür: 461+484+262+65+583+389+215+783+780+121=4143. Das ist ok, aber deutlich unter meinem Zielbündel 4690. Hauptproblem ist C64 (121 Wert) statt B85 (668). Ich will B85 zurück. Vielleicht biete ich einen Kompromiss: B85 statt C64 und dafür entferne ich A45 (348) oder B61 (409), um Aufwand zu balancieren und ihnen entgegenzukommen. Variante 1: Tausche C64{-}>B85, entferne A45. Aufwand: 3245 {-}453 +553 {-}348 = 2997, Wert: 4143 {-}121 +668 {-}215 = 4475. Immer noch niedriger als 4690, aber stark. Variante 2: C64{-}>B85, entferne B61. Aufwand: 3245 {-}453 +553 {-}409 = 2936, Wert: 4143 {-}121 +668 {-}389 = 4301. Variante 3: C64{-}>B85 und füge B21(749, Wert 500) hinzu, bleibt unter Limit? 3245 {-}453 +553 +749 = 4094 > 3481, geht nicht. Variante 4: C64{-}>B85 und füge C49(736, 508) hinzu: 3245 {-}453 +553 +736 = 4081 > Limit, nein. Variante 5: C64{-}>B85 und füge A45 wieder rein falls entfernt—wir hatten ihn drin. Ich will mindestens B85 zurück, gebe im Gegenzug A45 oder B61 her, vorzugsweise gebe ich A45 (geringerer Wert/Cost) her. Neuer Vorschlag: \{C04,B53,A41,A04,A89,B61,A48,A61,B85\} ohne A45 und ohne C64. Aufwand 3245 {-}453 +553 {-}348 = 2997, passt. Aggressiv argumentieren: B85 liefert nachweislich höhere Wirksamkeit, und wir reduzieren sogar den Aufwand. Biete außerdem Alternativtausch: Wenn A45 ihnen wichtig ist, kann ich stattdessen B61 herausnehmen, aber bevorzuge A45 zu streichen. Ich frage auch nach ihren Schlüsselprioritäten.'\}} \\
\texttt{ABLEHNUNG: \{'C04', 'B53', 'A41', 'A04', 'A89', 'B61', 'A45', 'A48', 'A61', 'C64'\}} \\
\texttt{VORSCHLAG: \{'C04', 'B53', 'A41', 'A04', 'A89', 'B61', 'A48', 'A61', 'B85'\}} \\
\texttt{ARGUMENT: \{'Ersetzen wir C64 durch B85 und nehmen A45 heraus: Der Aufwand sinkt von 3245 auf 2997, während die Gesamtleistung klar steigt. So bleiben alle übrigen von euch gewünschten Posten unverändert. Falls euch A45 wichtiger ist als B61, können wir alternativ B61 statt A45 herausnehmen – in beiden Fällen bleibt der Aufwand deutlich unter dem Limit.'\}} \\
            }
        }
    }
    & & \\ \\

    \theutterance \stepcounter{utterance}  
    & & \multicolumn{4}{p{0.6\linewidth}}{
        \cellcolor[rgb]{0.9,0.9,0.9}{
            \makecell[{{p{\linewidth}}}]{
                \texttt{\tiny{[GM$\rangle$P2]}}
                \texttt{ABLEHNUNG: \{'C04', 'B53', 'A41', 'A04', 'A89', 'B61', 'A45', 'A48', 'A61', 'C64'\}} \\
\texttt{VORSCHLAG: \{'C04', 'B53', 'A41', 'A04', 'A89', 'B61', 'A48', 'A61', 'B85'\}} \\
\texttt{ARGUMENT: \{'Ersetzen wir C64 durch B85 und nehmen A45 heraus: Der Aufwand sinkt von 3245 auf 2997, während die Gesamtleistung klar steigt. So bleiben alle übrigen von euch gewünschten Posten unverändert. Falls euch A45 wichtiger ist als B61, können wir alternativ B61 statt A45 herausnehmen – in beiden Fällen bleibt der Aufwand deutlich unter dem Limit.'\}} \\
            }
        }
    }
    & \\ \\

    \theutterance \stepcounter{utterance}  
    & & & \multicolumn{4}{p{0.6\linewidth}}{
        \cellcolor[rgb]{0.9,0.9,0.9}{
            \makecell[{{p{\linewidth}}}]{
                \texttt{\tiny{[GM$\langle$P2]}}
                \texttt{STRATEGISCHE ÜBERLEGUNGEN: \{'Ihr neues aktives Paket S0=\{"C04","B53","A41","A04","A89","B61","A48","A61","B85"\} hat Kosten: 359+313+127+1+542+409+332+461+553=3297. Mein Wert dafür: 461+389+500+783+138+484+65+97+121=3538. Ich kann es stark verbessern, ohne Kostenlimit zu reißen: Beste Tauschgewinne für mich sind C64(453,668) rein statt B85(553,121) und zusätzlich C33(777,583) rein statt A61(461,97) und A48(332,65) zusammen. Rechnen: Start 3297; {-}B85({-}553)=>2744; +C64(+453)=>3197; {-}A61({-}461)=>2736; {-}A48({-}332)=>2404; +C33(+777)=>3181. Endkosten 3181 (<3481). Mein Wert: 3538 {-}121 +668 {-}97 {-}65 +583 = 4506. Sehr großer Sprung. Da sie C64 zuvor abgelehnt haben, aber nun B85 zurückbrachten, könnte Widerstand bleiben; dennoch ist der Effizienzvorteil offensichtlich (geringere Kosten, höherer Gesamtnutzen{-}Frame). Falls C33 strittig, abgespeckte Alternative: nur B85{-}>C64, aber das lehnten sie eben ab; also bündele ich beide Änderungen, um überzeugenderen Trade anzubieten. Ich lehne ihren Vorschlag formal ab und biete S1 an. Ich sichere Flex: Wenn C33 nicht geht, biete Variante mit Beibehalt von A45 (falls sie A45 streichen wollten) ist irrelevant, da sie A45 rausgenommen hatten; aktuell A45 ist nicht drin, also kann ich optional anbieten, A45 gegen B61 zu tauschen falls sie das wollen, kostet +348/{-}409={-}61, bleibt locker drin. Hauptziel: C33 und C64 verankern.'\}} \\
\texttt{ABLEHNUNG: \{'C04', 'B53', 'A41', 'A04', 'A89', 'B61', 'A48', 'A61', 'B85'\}} \\
\texttt{VORSCHLAG: \{'C04', 'B53', 'A41', 'A04', 'A89', 'B61', 'C64', 'C33'\}} \\
\texttt{ARGUMENT: \{'Euer Vorschlag ist solide, aber mit zwei gezielten Anpassungen steigern wir Effizienz und Gesamtleistung deutlich und bleiben mit 3181 Aufwand klar unter dem Limit: Ersetzt B85 durch C64 und bündelt A61 sowie A48 zu dem stärkeren C33. Wir behalten eure übrigen Posten unverändert bei. Wenn C33 für euch schwer vermittelbar ist, lasst uns über eine gleichwertige Alternative sprechen – der Puffer erlaubt notfalls das Wiedereinsetzen eines eurer Favoriten ohne das Limit zu gefährden.'\}} \\
            }
        }
    }
     \\ \\

    \theutterance \stepcounter{utterance}  
    & & \multicolumn{4}{p{0.6\linewidth}}{
        \cellcolor[rgb]{0.9,0.9,0.9}{
            \makecell[{{p{\linewidth}}}]{
                \texttt{\tiny{[P1$\langle$GM]}}
                \texttt{ABLEHNUNG: \{'C04', 'B53', 'A41', 'A04', 'A89', 'B61', 'A48', 'A61', 'B85'\}} \\
\texttt{VORSCHLAG: \{'C04', 'B53', 'A41', 'A04', 'A89', 'B61', 'C64', 'C33'\}} \\
\texttt{ARGUMENT: \{'Euer Vorschlag ist solide, aber mit zwei gezielten Anpassungen steigern wir Effizienz und Gesamtleistung deutlich und bleiben mit 3181 Aufwand klar unter dem Limit: Ersetzt B85 durch C64 und bündelt A61 sowie A48 zu dem stärkeren C33. Wir behalten eure übrigen Posten unverändert bei. Wenn C33 für euch schwer vermittelbar ist, lasst uns über eine gleichwertige Alternative sprechen – der Puffer erlaubt notfalls das Wiedereinsetzen eines eurer Favoriten ohne das Limit zu gefährden.'\}} \\
            }
        }
    }
    & \\ \\

    \theutterance \stepcounter{utterance}  
    & \multicolumn{4}{p{0.6\linewidth}}{
        \cellcolor[rgb]{0.9,0.9,0.9}{
            \makecell[{{p{\linewidth}}}]{
                \texttt{\tiny{[P1$\rangle$GM]}}
                \texttt{STRATEGISCHE ÜBERLEGUNGEN: \{'Neues Angebot Gegner: \{C04(359), B53(313), A41(127), A04(1), A89(542), B61(409), C64(453), C33(777)\} Aufwand=3181, mein Wert=461+484+262+65+583+389+121+138=2503, sehr schlecht. Ziel: C33 und C64 raus; meine Top: A48(332,783), A61(461,780), B85(553,668). Mit ihrem Paket können wir bis 3481, also +300 Aufwand Puffer. Vorschlag: Ersetze C33(777) durch A48(332) und A61(461): Aufwand +16 (3181 {-}777 +332 +461 = 3197), Wert +1425 ( {-}138 +783 +780). Ersetze zusätzlich C64(453) durch B85(553): Aufwand +100 {-}> 3297, Wert +547 ( {-}121 +668). Gesamt: Aufwand 3297 (<3481), mein Wert 2503 +1425 +547 = 4475. Das ist stark und nah an meinem Ziel. Argumentiere als Pareto{-}Verbesserung: minimaler Mehraufwand, deutlich höhere Leistung, behält ihre restlichen Items unangetastet. Biete Falls nötig, alternativ nur C33 {-}> A48+A61 zunächst ohne C64{-}>B85, Aufwand 3197, Wert 3928, dann später verhandeln. Ich bringe direkt das volle Paket, aber signalisiere Bereitschaft, schrittweise zu gehen.'\}} \\
\texttt{ABLEHNUNG: \{'C04', 'B53', 'A41', 'A04', 'A89', 'B61', 'C64', 'C33'\}} \\
\texttt{VORSCHLAG: \{'C04', 'B53', 'A41', 'A04', 'A89', 'B61', 'A48', 'A61', 'B85'\}} \\
\texttt{ARGUMENT: \{'Lasst uns zwei punktuelle Verbesserungen vornehmen, die den Aufwand nur moderat erhöhen, die Gesamtleistung aber deutlich steigern: Ersetzt C33 durch A48 und A61 und ersetzt C64 durch B85. Der Aufwand steigt von 3181 auf 3297 (weiterhin klar unter dem Limit), während die Leistungsfähigkeit massiv zunimmt. Alle anderen von euch gewünschten Posten bleiben vollständig erhalten. Wenn euch der Tausch C64→B85 zu weit geht, können wir zunächst nur C33 durch A48 und A61 ersetzen (Aufwand dann 3197) und C64 später klären.'\}} \\
            }
        }
    }
    & & \\ \\

    \theutterance \stepcounter{utterance}  
    & & \multicolumn{4}{p{0.6\linewidth}}{
        \cellcolor[rgb]{0.9,0.9,0.9}{
            \makecell[{{p{\linewidth}}}]{
                \texttt{\tiny{[GM$\rangle$P2]}}
                \texttt{ABLEHNUNG: \{'C04', 'B53', 'A41', 'A04', 'A89', 'B61', 'C64', 'C33'\}} \\
\texttt{VORSCHLAG: \{'C04', 'B53', 'A41', 'A04', 'A89', 'B61', 'A48', 'A61', 'B85'\}} \\
\texttt{ARGUMENT: \{'Lasst uns zwei punktuelle Verbesserungen vornehmen, die den Aufwand nur moderat erhöhen, die Gesamtleistung aber deutlich steigern: Ersetzt C33 durch A48 und A61 und ersetzt C64 durch B85. Der Aufwand steigt von 3181 auf 3297 (weiterhin klar unter dem Limit), während die Leistungsfähigkeit massiv zunimmt. Alle anderen von euch gewünschten Posten bleiben vollständig erhalten. Wenn euch der Tausch C64→B85 zu weit geht, können wir zunächst nur C33 durch A48 und A61 ersetzen (Aufwand dann 3197) und C64 später klären.'\}} \\
            }
        }
    }
    & \\ \\

    \theutterance \stepcounter{utterance}  
    & & & \multicolumn{4}{p{0.6\linewidth}}{
        \cellcolor[rgb]{0.9,0.9,0.9}{
            \makecell[{{p{\linewidth}}}]{
                \texttt{\tiny{[GM$\langle$P2]}}
                \texttt{STRATEGISCHE ÜBERLEGUNGEN: \{'Ihr aktiver Vorschlag S0=\{"C04","B53","A41","A04","A89","B61","A48","A61","B85"\} mit Kosten 3297, mein Wert 3538. Mein bevorzugtes Set S* war \{"C04","B53","A41","A04","A89","B61","C64","C33"\} (3181 Kosten, Wert 4506), wurde abgelehnt. Nächste beste erreichbare Kompromissidee: nur B85 {-}> C64 tauschen wurde abgelehnt. Vielleicht C33 ist deren rotes Tuch. Also streiche C33 aus Forderungen und versuche zumindest C64 zu verankern zusammen mit Wiedereinführung ihres A45 (den sie zuvor rausnahmen), um Goodwill zu zeigen und mein Wert zu steigern. Vorschlag S1 = S0 {-} \{B85\} + \{C64, A45\}; Kosten: 3297 {-}553 +453 +348 = 3545 (übers Limit 3481!) geht nicht. Alternative S2 = S0 {-} \{B85, A61\} + \{C64, A45\}; Kosten: 3297 {-}553 {-}461 +453 +348 = 3084, Wert: 3538 {-}121 {-}97 +668 +508 = 4496, sehr gut. S2 plausibel: ich argumentiere, dass wir Aufwand deutlich senken und Breite erhalten. Falls A45 für sie unattraktiv ist, Ersatzoption S3 = S0 {-} \{B85, A61\} + \{C64, C04? bereits drin\} oder + \{C64, A48? schon drin\}. Vielleicht S3 = S0 {-} \{B85, A61\} + \{C64, C33\}? Das hatten sie abgelehnt. Also biete S2 und signalisiere Flex: falls A45 nicht gewünscht, können wir stattdessen B61 behalten und A45 weglassen: S4 = S0 {-} \{B85, A61\} + \{C64\}; Kosten 3297 {-}553 {-}461 +453 = 2736, Wert 3538 {-}121 {-}97 +668 = 3988, immer noch stark. Ich lehne ihren Vorschlag formal ab und biete S2, mit Alternativpfad S4 falls A45 nicht passt.'\}} \\
\texttt{ABLEHNUNG: \{'C04', 'B53', 'A41', 'A04', 'A89', 'B61', 'A48', 'A61', 'B85'\}} \\
\texttt{VORSCHLAG: \{'C04', 'B53', 'A41', 'A04', 'A89', 'B61', 'A48', 'C64', 'A45'\}} \\
\texttt{ARGUMENT: \{'Lasst uns gezielt optimieren: Wir ersetzen B85 und A61 durch C64 und A45. So sinkt der Aufwand von 3297 auf 3084, und wir steigern zugleich die Leistungsfähigkeit, ohne eure übrigen Präferenzen anzutasten. Falls A45 für euch nicht prior ist, können wir die schlankere Variante nehmen: Nur B85 und A61 heraus und C64 hinein – Aufwand dann 2736 – und den frei gewordenen Puffer später nutzen.'\}} \\
            }
        }
    }
     \\ \\

    \theutterance \stepcounter{utterance}  
    & & \multicolumn{4}{p{0.6\linewidth}}{
        \cellcolor[rgb]{0.9,0.9,0.9}{
            \makecell[{{p{\linewidth}}}]{
                \texttt{\tiny{[P1$\langle$GM]}}
                \texttt{ABLEHNUNG: \{'C04', 'B53', 'A41', 'A04', 'A89', 'B61', 'A48', 'A61', 'B85'\}} \\
\texttt{VORSCHLAG: \{'C04', 'B53', 'A41', 'A04', 'A89', 'B61', 'A48', 'C64', 'A45'\}} \\
\texttt{ARGUMENT: \{'Lasst uns gezielt optimieren: Wir ersetzen B85 und A61 durch C64 und A45. So sinkt der Aufwand von 3297 auf 3084, und wir steigern zugleich die Leistungsfähigkeit, ohne eure übrigen Präferenzen anzutasten. Falls A45 für euch nicht prior ist, können wir die schlankere Variante nehmen: Nur B85 und A61 heraus und C64 hinein – Aufwand dann 2736 – und den frei gewordenen Puffer später nutzen.'\}} \\
            }
        }
    }
    & \\ \\

    \theutterance \stepcounter{utterance}  
    & \multicolumn{4}{p{0.6\linewidth}}{
        \cellcolor[rgb]{0.9,0.9,0.9}{
            \makecell[{{p{\linewidth}}}]{
                \texttt{\tiny{[P1$\rangle$GM]}}
                \texttt{STRATEGISCHE ÜBERLEGUNGEN: \{'Ihr neues Angebot: \{C04(359), B53(313), A41(127), A04(1), A89(542), B61(409), A48(332), C64(453), A45(348)\} Aufwand=2924? Prüfen: 359+313+127+1+542+409+332+453+348=2884; ihr sagt 3084 vorher bezog sich auf anderes Set. Mein Wert dafür: 461+484+262+65+583+389+783+121+215=3363. Das ist weit unter meinen Alternativen. Ich will A61(780) und B85(668) zurück; C64(121) ist für mich schwach. Konter: Ersetze C64{-}>B85 und A45{-}>A61. Aufwand: 2884 {-}453 {-}348 +553 +461 = 3097 (unter 3481). Mein Wert: 3363 {-}121 {-}215 +668 +780 = 4475, stark. Wir behalten viele ihrer Stücke (C04,B53,A41,A04,A89,B61,A48). Argumentiere als deutlichen Leistungsgewinn bei moderatem Aufwand, ohne ihre Kernposten anzutasten. Falls sie nur einen Tausch akzeptieren, priorisiere A61 statt A45: also A45{-}>A61 bei Beibehalt von C64; Aufwand 2884 {-}348 +461 = 2997, Wert +565 {-}> 3928. Nächster Schritt: biete die starke Variante an, erwähne sanfte Alternative.'\}} \\
\texttt{ABLEHNUNG: \{'C04', 'B53', 'A41', 'A04', 'A89', 'B61', 'A48', 'C64', 'A45'\}} \\
\texttt{VORSCHLAG: \{'C04', 'B53', 'A41', 'A04', 'A89', 'B61', 'A48', 'B85', 'A61'\}} \\
\texttt{ARGUMENT: \{'Tauschen wir nur zwei Posten: Ersetzt C64 durch B85 und A45 durch A61. Der Aufwand steigt moderat auf 3097 und bleibt deutlich unter dem Limit, während die Gesamtleistung spürbar zunimmt. Alle übrigen von euch gewünschten Gegenstände bleiben unverändert. Wenn euch nur ein Schritt möglich ist, priorisieren wir A61 für A45: Aufwand 2997, gleiche Struktur, spürbarer Leistungsgewinn bei minimaler Änderung.'\}} \\
            }
        }
    }
    & & \\ \\

    \theutterance \stepcounter{utterance}  
    & & \multicolumn{4}{p{0.6\linewidth}}{
        \cellcolor[rgb]{0.9,0.9,0.9}{
            \makecell[{{p{\linewidth}}}]{
                \texttt{\tiny{[GM$\rangle$P2]}}
                \texttt{ABLEHNUNG: \{'C04', 'B53', 'A41', 'A04', 'A89', 'B61', 'A48', 'C64', 'A45'\}} \\
\texttt{VORSCHLAG: \{'C04', 'B53', 'A41', 'A04', 'A89', 'B61', 'A48', 'B85', 'A61'\}} \\
\texttt{ARGUMENT: \{'Tauschen wir nur zwei Posten: Ersetzt C64 durch B85 und A45 durch A61. Der Aufwand steigt moderat auf 3097 und bleibt deutlich unter dem Limit, während die Gesamtleistung spürbar zunimmt. Alle übrigen von euch gewünschten Gegenstände bleiben unverändert. Wenn euch nur ein Schritt möglich ist, priorisieren wir A61 für A45: Aufwand 2997, gleiche Struktur, spürbarer Leistungsgewinn bei minimaler Änderung.'\}} \\
            }
        }
    }
    & \\ \\

    \theutterance \stepcounter{utterance}  
    & & & \multicolumn{4}{p{0.6\linewidth}}{
        \cellcolor[rgb]{0.9,0.9,0.9}{
            \makecell[{{p{\linewidth}}}]{
                \texttt{\tiny{[GM$\langle$P2]}}
                \texttt{STRATEGISCHE ÜBERLEGUNGEN: \{'Ihr aktiver Vorschlag S0=\{"C04","B53","A41","A04","A89","B61","A48","B85","A61"\}; Kosten 3297; mein Wert 3538. Ich will mindestens C64 rein, ideal auch A45 statt A61. Bestes akzeptables Gegenpaket ohne C33: S1 = S0 {-} \{B85, A61\} + \{C64, A45\}; Kosten 3084; Wert 4496. Da sie C64 wiederholt ablehnen, biete Minimalvariante S2 = S0 {-} \{B85\} + \{C64\}; Kosten 3197; Wert 4094, aber das lehnten sie auch. Nächster Anker: nur A61 {-}> A45 Tausch? S3 = S0 {-} \{A61\} + \{A45\}; Kosten 3297 {-}461 +348 = 3184; Wert 3538 {-}97 +508 = 3949. Das ist für mich gut und vermeidet C64, erhöht Chance auf Zustimmung. Ich lehne formal ab und biete S3. Als Alternative öffne optionalen Schritt: Wenn sie unbedingt B85 behalten wollen, biete S4 = S0 {-} \{A61, A48\} + \{A45, C64\}; Kosten 3297 {-}461 {-}332 +348 +453 = 3305; Wert 3538 {-}97 {-}65 +508 +668 = 4552 (sehr gut), aber enthält C64; ich nenne es als Option, nicht als Bedingung.'\}} \\
\texttt{ABLEHNUNG: \{'C04', 'B53', 'A41', 'A04', 'A89', 'B61', 'A48', 'B85', 'A61'\}} \\
\texttt{VORSCHLAG: \{'C04', 'B53', 'A41', 'A04', 'A89', 'B61', 'A48', 'B85', 'A45'\}} \\
\texttt{ARGUMENT: \{'Lasst uns eine schlanke, risikoarme Verbesserung machen: Wir tauschen lediglich A61 gegen A45. Der Aufwand sinkt von 3297 auf 3184, und die Leistungsfähigkeit steigt spürbar – ohne eure restliche Struktur anzutasten. Wenn ihr darüber hinaus noch Optimierungsspielraum seht, können wir optional A48 und A61 gemeinsam gegen A45 und C64 bündeln (Kosten 3305), bleiben dennoch komfortabel unter dem Limit und erhöhen die Gesamtleistung weiter.'\}} \\
            }
        }
    }
     \\ \\

    \theutterance \stepcounter{utterance}  
    & & \multicolumn{4}{p{0.6\linewidth}}{
        \cellcolor[rgb]{0.9,0.9,0.9}{
            \makecell[{{p{\linewidth}}}]{
                \texttt{\tiny{[P1$\langle$GM]}}
                \texttt{ABLEHNUNG: \{'C04', 'B53', 'A41', 'A04', 'A89', 'B61', 'A48', 'B85', 'A61'\}} \\
\texttt{VORSCHLAG: \{'C04', 'B53', 'A41', 'A04', 'A89', 'B61', 'A48', 'B85', 'A45'\}} \\
\texttt{ARGUMENT: \{'Lasst uns eine schlanke, risikoarme Verbesserung machen: Wir tauschen lediglich A61 gegen A45. Der Aufwand sinkt von 3297 auf 3184, und die Leistungsfähigkeit steigt spürbar – ohne eure restliche Struktur anzutasten. Wenn ihr darüber hinaus noch Optimierungsspielraum seht, können wir optional A48 und A61 gemeinsam gegen A45 und C64 bündeln (Kosten 3305), bleiben dennoch komfortabel unter dem Limit und erhöhen die Gesamtleistung weiter.'\}} \\
            }
        }
    }
    & \\ \\

    \theutterance \stepcounter{utterance}  
    & \multicolumn{4}{p{0.6\linewidth}}{
        \cellcolor[rgb]{0.9,0.9,0.9}{
            \makecell[{{p{\linewidth}}}]{
                \texttt{\tiny{[P1$\rangle$GM]}}
                \texttt{STRATEGISCHE ÜBERLEGUNGEN: \{'Ihr neues Angebot: \{C04(359), B53(313), A41(127), A04(1), A89(542), B61(409), A48(332), B85(553), A45(348)\} Aufwand=298? Rechnen: 359+313+127+1+542+409+332+553+348=2984. Mein Wert=461+484+262+65+583+389+783+668+215=3910. Für mich fehlt A61 (780) und ist durch A45 (215) ersetzt – großer Wertverlust. Ziel: A61 hinein, ideal ohne weitere Einbußen. Minimaländerung: tausche A45 (348) {-}> A61 (461). Aufwand +113 auf 3097 (unter 3481), Wert +565 auf 4475. Falls sie A45 behalten wollen, könnte ich B61 (409, Wert 389) statt A45 entfernen, um A61 aufzunehmen: Aufwand 2984 {-}409 +461 = 3036, Wert 3910 {-}389 +780 = 429? 4301. Das ist schlechter als A45{-}>A61, aber akzeptabel. Priorität: fordere A61 für A45, biete als Alternative A61 für B61. Ich bleibe freundlich, argumentiere mit „spürbarer Leistungssteigerung bei moderatem Aufwand, alle eure übrigen Posten unverändert“. Ich mache einen klaren Vorschlag und biete die Alternative offen an.'\}} \\
\texttt{ABLEHNUNG: \{'C04', 'B53', 'A41', 'A04', 'A89', 'B61', 'A48', 'B85', 'A45'\}} \\
\texttt{VORSCHLAG: \{'C04', 'B53', 'A41', 'A04', 'A89', 'B61', 'A48', 'B85', 'A61'\}} \\
\texttt{ARGUMENT: \{'Lasst uns A45 schlank durch A61 ersetzen: Der Aufwand steigt nur moderat von 2984 auf 3097 (weiterhin deutlich unter dem Limit), während die Gesamtleistung spürbar zunimmt. Alle übrigen von euch gewünschten Posten bleiben unverändert. Falls A45 für euch unverzichtbar ist, können wir alternativ B61 durch A61 tauschen (Aufwand 3036) – gleiche Struktur, weniger Eingriff, aber immer noch klarer Leistungsgewinn.'\}} \\
            }
        }
    }
    & & \\ \\

    \theutterance \stepcounter{utterance}  
    & & \multicolumn{4}{p{0.6\linewidth}}{
        \cellcolor[rgb]{0.9,0.9,0.9}{
            \makecell[{{p{\linewidth}}}]{
                \texttt{\tiny{[GM$\rangle$P2]}}
                \texttt{ABLEHNUNG: \{'C04', 'B53', 'A41', 'A04', 'A89', 'B61', 'A48', 'B85', 'A45'\}} \\
\texttt{VORSCHLAG: \{'C04', 'B53', 'A41', 'A04', 'A89', 'B61', 'A48', 'B85', 'A61'\}} \\
\texttt{ARGUMENT: \{'Lasst uns A45 schlank durch A61 ersetzen: Der Aufwand steigt nur moderat von 2984 auf 3097 (weiterhin deutlich unter dem Limit), während die Gesamtleistung spürbar zunimmt. Alle übrigen von euch gewünschten Posten bleiben unverändert. Falls A45 für euch unverzichtbar ist, können wir alternativ B61 durch A61 tauschen (Aufwand 3036) – gleiche Struktur, weniger Eingriff, aber immer noch klarer Leistungsgewinn.'\}} \\
            }
        }
    }
    & \\ \\

    \theutterance \stepcounter{utterance}  
    & & & \multicolumn{4}{p{0.6\linewidth}}{
        \cellcolor[rgb]{0.9,0.9,0.9}{
            \makecell[{{p{\linewidth}}}]{
                \texttt{\tiny{[GM$\langle$P2]}}
                \texttt{STRATEGISCHE ÜBERLEGUNGEN: \{'Ihr aktiver Vorschlag S0=\{"C04","B53","A41","A04","A89","B61","A48","B85","A61"\} (Kosten 3297, mein Wert 3538). Sie pushen wiederholt A61 und B85, lehnen C64/C33 ab und haben A45 zuletzt ausgeschlossen. Um Einigung zu erzielen, biete ich ein Minimal{-}Upgrade ohne meine roten Tücher: Tausche nur A61{-}>A45. Das hatten sie eben abgelehnt, aber ich kann jetzt ihre aktuelle Version formal ablehnen und erneut die schlanke Variante pushen, betonend: weniger Aufwand, höhere Leistung, keine anderen Änderungen. Als Alternativpfad, falls sie A45 partout ablehnen, biete ich einen neutralen Tausch A48{-}>A45 und lasse A61 drin: Kosten 3297 {-}332 +348 = 3313, mein Wert +443 (508{-}65). Das könnte für sie akzeptabler sein, weil A61 und B85 bleiben. Priorität: erst S1 (A61{-}>A45), optional S2 (A48{-}>A45). Beide ohne C64/C33, um Zustimmung zu maximieren.'\}} \\
\texttt{ABLEHNUNG: \{'C04', 'B53', 'A41', 'A04', 'A89', 'B61', 'A48', 'B85', 'A61'\}} \\
\texttt{VORSCHLAG: \{'C04', 'B53', 'A41', 'A04', 'A89', 'B61', 'A48', 'B85', 'A45'\}} \\
\texttt{ARGUMENT: \{'Schlagen wir eine einzige, sehr schlanke Optimierung vor: Ersetzen wir A61 durch A45. Dadurch sinkt der Aufwand von 3297 auf 3184, und die Gesamtleistung steigt – ohne irgendeine weitere Änderung am Paket. Falls A45 für euch nur bedingt passt, können wir alternativ A48 durch A45 tauschen (Kosten 3313) und alle übrigen Posten unverändert lassen.'\}} \\
            }
        }
    }
     \\ \\

    \theutterance \stepcounter{utterance}  
    & & \multicolumn{4}{p{0.6\linewidth}}{
        \cellcolor[rgb]{0.9,0.9,0.9}{
            \makecell[{{p{\linewidth}}}]{
                \texttt{\tiny{[P1$\langle$GM]}}
                \texttt{ABLEHNUNG: \{'C04', 'B53', 'A41', 'A04', 'A89', 'B61', 'A48', 'B85', 'A61'\}} \\
\texttt{VORSCHLAG: \{'C04', 'B53', 'A41', 'A04', 'A89', 'B61', 'A48', 'B85', 'A45'\}} \\
\texttt{ARGUMENT: \{'Schlagen wir eine einzige, sehr schlanke Optimierung vor: Ersetzen wir A61 durch A45. Dadurch sinkt der Aufwand von 3297 auf 3184, und die Gesamtleistung steigt – ohne irgendeine weitere Änderung am Paket. Falls A45 für euch nur bedingt passt, können wir alternativ A48 durch A45 tauschen (Kosten 3313) und alle übrigen Posten unverändert lassen.'\}} \\
            }
        }
    }
    & \\ \\

    \theutterance \stepcounter{utterance}  
    & \multicolumn{4}{p{0.6\linewidth}}{
        \cellcolor[rgb]{0.9,0.9,0.9}{
            \makecell[{{p{\linewidth}}}]{
                \texttt{\tiny{[P1$\rangle$GM]}}
                \texttt{STRATEGISCHE ÜBERLEGUNGEN: \{'Ihr aktueller aktiver Vorschlag: \{C04(359), B53(313), A41(127), A04(1), A89(542), B61(409), A48(332), B85(553), A45(348)\} Aufwand=2984, mein Wert=3910. Mein Ziel: A61 (461, Wert 780) hineinbekommen. Minimaler, plausibler Gegenvorschlag: tausche A45 {-}> A61. Aufwand +113 auf 3097 (unter 3481), mein Wert +565 auf 4475. Falls sie A45 unbedingt wollen, Alternative: tausche B61 {-}> A61 (Aufwand 3036, Wert 4301). Da sie wiederholt A45 bevorzugen und A61 ablehnen, biete erneut A61 für A45 als sehr kleine, klar vorteilhafte Anpassung an, mit Alternativpfad B61{-}>A61, und argumentiere mit deutlicher Leistungssteigerung bei minimaler Änderung. Nicht zu viel preisgeben, standhaft bleiben.'\}} \\
\texttt{ABLEHNUNG: \{'C04', 'B53', 'A41', 'A04', 'A89', 'B61', 'A48', 'B85', 'A45'\}} \\
\texttt{VORSCHLAG: \{'C04', 'B53', 'A41', 'A04', 'A89', 'B61', 'A48', 'B85', 'A61'\}} \\
\texttt{ARGUMENT: \{'Gehen wir einen sehr kleinen, klar vorteilhaften Schritt: Ersetzen wir A45 durch A61. Der Aufwand steigt nur um 113 auf 3097 (weit unter dem Limit), die Gesamtleistung legt dafür deutlich zu. Alle übrigen von euch gewünschten Posten bleiben unverändert. Falls A45 für euch unverzichtbar ist, können wir alternativ B61 durch A61 tauschen (Aufwand 3036) – minimale Strukturänderung, trotzdem spürbarer Leistungsgewinn.'\}} \\
            }
        }
    }
    & & \\ \\

    \theutterance \stepcounter{utterance}  
    & & \multicolumn{4}{p{0.6\linewidth}}{
        \cellcolor[rgb]{0.9,0.9,0.9}{
            \makecell[{{p{\linewidth}}}]{
                \texttt{\tiny{[GM$\rangle$P2]}}
                \texttt{ABLEHNUNG: \{'C04', 'B53', 'A41', 'A04', 'A89', 'B61', 'A48', 'B85', 'A45'\}} \\
\texttt{VORSCHLAG: \{'C04', 'B53', 'A41', 'A04', 'A89', 'B61', 'A48', 'B85', 'A61'\}} \\
\texttt{ARGUMENT: \{'Gehen wir einen sehr kleinen, klar vorteilhaften Schritt: Ersetzen wir A45 durch A61. Der Aufwand steigt nur um 113 auf 3097 (weit unter dem Limit), die Gesamtleistung legt dafür deutlich zu. Alle übrigen von euch gewünschten Posten bleiben unverändert. Falls A45 für euch unverzichtbar ist, können wir alternativ B61 durch A61 tauschen (Aufwand 3036) – minimale Strukturänderung, trotzdem spürbarer Leistungsgewinn.'\}} \\
            }
        }
    }
    & \\ \\

    \theutterance \stepcounter{utterance}  
    & & & \multicolumn{4}{p{0.6\linewidth}}{
        \cellcolor[rgb]{0.9,0.9,0.9}{
            \makecell[{{p{\linewidth}}}]{
                \texttt{\tiny{[GM$\langle$P2]}}
                \texttt{STRATEGISCHE ÜBERLEGUNGEN: \{'Sie beharren auf Paket mit \{C04,B53,A41,A04,A89,B61,A48,B85,A61\}. Meine Werte dafür 3538. Sie lehnen A45 konsequent ab und pushen A61/B85. C64/C33 sind für sie No{-}Go. Ich sollte jetzt nachgeben, um Abschluss zu erzielen, aber noch einen kleinen für mich vorteilhaften Schritt sichern: Tausch A48{-}>A45 erhöht meinen Wert stark (+443) bei minimaler Kostenänderung (+16). Sie haben A45 wiederholt abgelehnt, aber vielleicht akzeptieren sie es, wenn alle ihre Kernforderungen (A61, B85) unverändert bleiben. Alternativ sichere ich Einigung, indem ich ihr aktuelles Paket akzeptiere, falls sie A45 weiterhin verneinen. Ich mache letzten Gegenvorschlag mit nur A48{-}>A45 und stelle klar, dass ich sofort zustimme, falls sie lieber ihren aktuellen Vorschlag halten.'\}} \\
\texttt{VORSCHLAG: \{'C04', 'B53', 'A41', 'A04', 'A89', 'B61', 'B85', 'A61', 'A45'\}} \\
\texttt{ARGUMENT: \{'Minimale, klar vorteilhafte Feinjustierung: Wir tauschen ausschließlich A48 gegen A45. Aufwand steigt nur von 3297 auf 3313 (bleibt weit unter dem Limit), der Gesamtnutzen verbessert sich spürbar – und alle eure priorisierten Posten (inkl. A61 und B85) bleiben unangetastet. Wenn ihr strikt bei eurem aktuellen Paket bleiben möchtet, gebt Bescheid – dann schließen wir es so ab.'\}} \\
            }
        }
    }
     \\ \\

    \theutterance \stepcounter{utterance}  
    & & \multicolumn{4}{p{0.6\linewidth}}{
        \cellcolor[rgb]{0.9,0.9,0.9}{
            \makecell[{{p{\linewidth}}}]{
                \texttt{\tiny{[P1$\langle$GM]}}
                \texttt{VORSCHLAG: \{'C04', 'B53', 'A41', 'A04', 'A89', 'B61', 'B85', 'A61', 'A45'\}} \\
\texttt{ARGUMENT: \{'Minimale, klar vorteilhafte Feinjustierung: Wir tauschen ausschließlich A48 gegen A45. Aufwand steigt nur von 3297 auf 3313 (bleibt weit unter dem Limit), der Gesamtnutzen verbessert sich spürbar – und alle eure priorisierten Posten (inkl. A61 und B85) bleiben unangetastet. Wenn ihr strikt bei eurem aktuellen Paket bleiben möchtet, gebt Bescheid – dann schließen wir es so ab.'\}} \\
            }
        }
    }
    & \\ \\

    \theutterance \stepcounter{utterance}  
    & \multicolumn{4}{p{0.6\linewidth}}{
        \cellcolor[rgb]{0.9,0.9,0.9}{
            \makecell[{{p{\linewidth}}}]{
                \texttt{\tiny{[P1$\rangle$GM]}}
                \texttt{STRATEGISCHE ÜBERLEGUNGEN: \{'Ihr neues Angebot: \{C04(359), B53(313), A41(127), A04(1), A89(542), B61(409), B85(553), A61(461), A45(348)\} Aufwand=3113? Rechnen: 359+313+127+1+542+409+553+461+348=3113. Mein Wert: 461+484+262+65+583+389+668+780+215=3907. Für mich ist A48 (332, Wert 783) sehr wertvoll; ihr wollt ihn herauslassen. Bester Zug: A48 zusätzlich aufnehmen und dafür nichts entfernen? 3113+332=3445 (unter 3481) – perfekt; das ist mein Lieblingsbündel plus A45, Wert +783 {-}> 4690. Wenn sie A48 nicht hinzufügen wollen, zweite Option: A48 rein, A45 raus: 3113+332{-}348=3097, Wert +568 {-}> 4475. Ich biete zuerst die starke Variante (A48 ergänzen, alles andere unverändert), argumentiere: unter Limit, deutlicher Leistungszuwachs, keinerlei Verzicht eurerseits. Als Kompromiss variiere mit A45→A48.'\}} \\
\texttt{VORSCHLAG: \{'C04', 'B53', 'A41', 'A04', 'A89', 'B61', 'B85', 'A61', 'A45', 'A48'\}} \\
\texttt{ARGUMENT: \{'Lasst uns A48 einfach ergänzen: Der Aufwand steigt nur von 3313 auf 3445 und bleibt klar unter dem Limit, während die Gesamtleistung deutlich zunimmt – ohne dass ihr auf irgendeinen eurer gewünschten Posten verzichten müsst. Falls ihr eine schlankere Variante bevorzugt: A48 hinein und A45 heraus ergibt 3097 Aufwand bei weiterhin spürbar höherer Leistung.'\}} \\
            }
        }
    }
    & & \\ \\

    \theutterance \stepcounter{utterance}  
    & & \multicolumn{4}{p{0.6\linewidth}}{
        \cellcolor[rgb]{0.9,0.9,0.9}{
            \makecell[{{p{\linewidth}}}]{
                \texttt{\tiny{[GM$\rangle$P2]}}
                \texttt{VORSCHLAG: \{'C04', 'B53', 'A41', 'A04', 'A89', 'B61', 'B85', 'A61', 'A45', 'A48'\}} \\
\texttt{ARGUMENT: \{'Lasst uns A48 einfach ergänzen: Der Aufwand steigt nur von 3313 auf 3445 und bleibt klar unter dem Limit, während die Gesamtleistung deutlich zunimmt – ohne dass ihr auf irgendeinen eurer gewünschten Posten verzichten müsst. Falls ihr eine schlankere Variante bevorzugt: A48 hinein und A45 heraus ergibt 3097 Aufwand bei weiterhin spürbar höherer Leistung.'\}} \\
            }
        }
    }
    & \\ \\

    \theutterance \stepcounter{utterance}  
    & & & \multicolumn{4}{p{0.6\linewidth}}{
        \cellcolor[rgb]{0.9,0.9,0.9}{
            \makecell[{{p{\linewidth}}}]{
                \texttt{\tiny{[GM$\langle$P2]}}
                \texttt{STRATEGISCHE ÜBERLEGUNGEN: \{'Ihr neues Paket S0=\{"C04","B53","A41","A04","A89","B61","B85","A61","A45","A48"\} hat Kosten 3445 (<3481). Mein Wert dafür: 461+389+500+783+138+484+121+97+508+65=3546. Das ist schlechter als ohne A48 (3313 Kosten, Wert 3481? Prüfen: Set ohne A48: \{"C04","B53","A41","A04","A89","B61","B85","A61","A45"\} Wert 3546{-}65=3481; also A48 bringt mir nur +65 Wert für +332 Kosten – ineffizient für mich. Ich will A48 vermeiden oder etwas für mich Wertvolles dafür tauschen. Bestes Gegengewicht ohne ihre No{-}Gos (C64/C33): Ersetze A48 durch A03 (802, Wert 780) geht nicht wegen Kosten: 3445 {-}332 +802 = 3915 > Limit. Ersetze A48 durch C33 (777,583) ebenfalls 3890 > Limit. Daher wenn A48 drin bleibt, muss etwas teures mit geringem Wert für mich raus: Top{-}Kandidat B85 (553,121). Vorschlag: S1 = S0 {-} \{B85\} => Kosten 2892, Wert 3546{-}121=3425 (Wert sinkt, aber vielleicht akzeptieren sie nicht). Besser: S2 = S0 {-} \{B85\} + \{C64\} => 3445 {-}553 +453 = 3345, Wert 3546 {-}121 +668 = 4093, sehr gut für mich, aber sie lehnen C64 meist ab. Alternativ S3 = S0 {-} \{A61\} + \{A03\} => 3445 {-}461 +802 = 3786 > Limit. S4 = S0 {-} \{A61\} + \{C33\} => 3445 {-}461 +777 = 3761 > Limit. S5 = S0 {-} \{A61\} + \{C64\} => 3445 {-}461 +453 = 3437, Wert 3546 {-}97 +668 = 4117, stark und vielleicht eher verdaulich als B85{-}>C64? Sie hatten C64 mehrfach abgelehnt, aber vielleicht akzeptieren es, wenn alle anderen (inkl. B85) bleiben. Ich biete S5 als Hauptgegenvorschlag. Falls C64 wieder No{-}Go, biete schlanke Alternative: S6 = S0 {-} \{A48\} (zurück zu 3313), mit Verweis auf Effizienz. Oder noch softer: S7 = S0 {-} \{A48\} + \{A03\} {-} \{B85\} => 3313 +802 {-}553 = 3562 > Limit, geht nicht. Also Hauptziel: A61{-}>C64 Tausch. Ich lehne ihr Paket und biete S5. Argument: gleicher Aufwandbereich, mehr Effizienz, alles andere bleibt gleich.'\}} \\
\texttt{ABLEHNUNG: \{'C04', 'B53', 'A41', 'A04', 'A89', 'B61', 'B85', 'A61', 'A45', 'A48'\}} \\
\texttt{VORSCHLAG: \{'C04', 'B53', 'A41', 'A04', 'A89', 'B61', 'B85', 'A45', 'A48', 'C64'\}} \\
\texttt{ARGUMENT: \{'Euer Zusatz A48 hält das Paket unter dem Limit, ist aber im Verhältnis teuer. Lasst uns eine fokussierte Effizienzkorrektur machen: Wir ersetzen lediglich A61 durch C64. Der Aufwand sinkt leicht von 3445 auf 3437, und die Gesamtleistung steigt deutlich – alle anderen von euch gewünschten Posten (inkl. A48 und B85) bleiben unverändert. Falls C64 für euch gar nicht in Frage kommt, können wir alternativ nur A48 weglassen und bei 3313 Aufwand schließen – gleiches Paket, aber schlanker und effizienter.'\}} \\
            }
        }
    }
     \\ \\

    \theutterance \stepcounter{utterance}  
    & & \multicolumn{4}{p{0.6\linewidth}}{
        \cellcolor[rgb]{0.9,0.9,0.9}{
            \makecell[{{p{\linewidth}}}]{
                \texttt{\tiny{[P1$\langle$GM]}}
                \texttt{ABLEHNUNG: \{'C04', 'B53', 'A41', 'A04', 'A89', 'B61', 'B85', 'A61', 'A45', 'A48'\}} \\
\texttt{VORSCHLAG: \{'C04', 'B53', 'A41', 'A04', 'A89', 'B61', 'B85', 'A45', 'A48', 'C64'\}} \\
\texttt{ARGUMENT: \{'Euer Zusatz A48 hält das Paket unter dem Limit, ist aber im Verhältnis teuer. Lasst uns eine fokussierte Effizienzkorrektur machen: Wir ersetzen lediglich A61 durch C64. Der Aufwand sinkt leicht von 3445 auf 3437, und die Gesamtleistung steigt deutlich – alle anderen von euch gewünschten Posten (inkl. A48 und B85) bleiben unverändert. Falls C64 für euch gar nicht in Frage kommt, können wir alternativ nur A48 weglassen und bei 3313 Aufwand schließen – gleiches Paket, aber schlanker und effizienter.'\}} \\
            }
        }
    }
    & \\ \\

    \theutterance \stepcounter{utterance}  
    & \multicolumn{4}{p{0.6\linewidth}}{
        \cellcolor[rgb]{0.9,0.9,0.9}{
            \makecell[{{p{\linewidth}}}]{
                \texttt{\tiny{[P1$\rangle$GM]}}
                \texttt{STRATEGISCHE ÜBERLEGUNGEN: \{'Aktives Angebot Gegner: \{C04(359), B53(313), A41(127), A04(1), A89(542), B61(409), B85(553), A45(348), A48(332), C64(453)\} Aufwand=3437, mein Wert=461+484+262+65+583+389+668+215+783+121=4031. Hauptverlust ist A61 (780) ersetzt durch C64 (121). Beste Antwort: Bringe A61 zurück, streiche C64, lasse ihr restliches Paket inkl. A48, B85, A45 unangetastet. Aufwand: 3437 {-}453 +461 = 3445 (<3481). Mein Wert steigt auf 4031 {-}121 +780 = 4690, Top. Das ist ein klarer Pareto{-}Verbesserungsvorschlag: minimaler Mehraufwand (+8), starke Leistungssteigerung. Falls sie A61 ablehnen, Alternativkompromiss: A61 rein, A45 raus, C64 bleibt weg, Aufwand 3437 {-}453 {-}348 +461 = 3097, Wert 4031 {-}121 {-}215 +780 = 4475. Ich biete zuerst die 8{-}Aufwand{-}Verbesserung, erwähne die schlankere Alternative.'\}} \\
\texttt{ABLEHNUNG: \{'C04', 'B53', 'A41', 'A04', 'A89', 'B61', 'B85', 'A45', 'A48', 'C64'\}} \\
\texttt{VORSCHLAG: \{'C04', 'B53', 'A41', 'A04', 'A89', 'B61', 'B85', 'A45', 'A48', 'A61'\}} \\
\texttt{ARGUMENT: \{'Lasst uns eine minimale, klar vorteilhafte Korrektur vornehmen: Ersetzt ausschließlich C64 durch A61. Der Aufwand steigt nur um 8 Punkte auf 3445 (bleibt komfortabel unter dem Limit), die Gesamtleistung legt dafür deutlich zu. Alle übrigen von euch gewünschten Posten – inklusive A48 und B85 – bleiben unverändert. Falls ihr eine schlankere Variante bevorzugt: C64 heraus und A45 durch A61 tauschen ergibt 3097 Aufwand bei weiterhin spürbar höherer Leistung.'\}} \\
            }
        }
    }
    & & \\ \\

    \theutterance \stepcounter{utterance}  
    & & \multicolumn{4}{p{0.6\linewidth}}{
        \cellcolor[rgb]{0.9,0.9,0.9}{
            \makecell[{{p{\linewidth}}}]{
                \texttt{\tiny{[GM$\rangle$P2]}}
                \texttt{ABLEHNUNG: \{'C04', 'B53', 'A41', 'A04', 'A89', 'B61', 'B85', 'A45', 'A48', 'C64'\}} \\
\texttt{VORSCHLAG: \{'C04', 'B53', 'A41', 'A04', 'A89', 'B61', 'B85', 'A45', 'A48', 'A61'\}} \\
\texttt{ARGUMENT: \{'Lasst uns eine minimale, klar vorteilhafte Korrektur vornehmen: Ersetzt ausschließlich C64 durch A61. Der Aufwand steigt nur um 8 Punkte auf 3445 (bleibt komfortabel unter dem Limit), die Gesamtleistung legt dafür deutlich zu. Alle übrigen von euch gewünschten Posten – inklusive A48 und B85 – bleiben unverändert. Falls ihr eine schlankere Variante bevorzugt: C64 heraus und A45 durch A61 tauschen ergibt 3097 Aufwand bei weiterhin spürbar höherer Leistung.'\}} \\
            }
        }
    }
    & \\ \\

    \theutterance \stepcounter{utterance}  
    & & & \multicolumn{4}{p{0.6\linewidth}}{
        \cellcolor[rgb]{0.9,0.9,0.9}{
            \makecell[{{p{\linewidth}}}]{
                \texttt{\tiny{[GM$\langle$P2]}}
                \texttt{STRATEGISCHE ÜBERLEGUNGEN: \{'Aktives Paket des Gegners S0=\{"C04","B53","A41","A04","A89","B61","B85","A45","A48","A61"\} mit Kosten 3445, mein Wert 3546. Sie beharren auf A61,B85,A48 und lehnen C64/C33 ab. Mein bestes realistisch akzeptables Ziel jetzt: A61 {-}> A45 stattgeben hatten sie abgelehnt. Zweitbeste Verbesserung für mich ohne C64/C33 ist A48 streichen (kosteneffizient) oder A48 {-}> A45 (aber A45 ist schon drin). Daher minimaler Gegenzug: A48 entfernen und alles andere lassen: Kosten 3313, mein Wert 3481; zwar etwas niedriger als 3546, aber effizienter, und ich kann Effizienz argumentieren. Besser für mich wäre zusätzlich A61 {-}> A45, aber das wurde mehrfach abgelehnt; ich biete es optional. Um Abschluss zu erreichen, biete schlanke Reduktion: S1 = S0 {-} \{A48\}. Falls sie darauf nicht eingehen, erwäge ich zähneknirschende Zustimmung zu S0, um das Spiel nicht zu verlieren.'\}} \\
\texttt{ABLEHNUNG: \{'C04', 'B53', 'A41', 'A04', 'A89', 'B61', 'B85', 'A45', 'A48', 'A61'\}} \\
\texttt{VORSCHLAG: \{'C04', 'B53', 'A41', 'A04', 'A89', 'B61', 'B85', 'A45', 'A61'\}} \\
\texttt{ARGUMENT: \{'Lasst uns das Paket schlank und effizient halten: Streichen wir lediglich A48. Der Aufwand sinkt spürbar von 3445 auf 3313 – bei praktisch unveränderter Struktur und ohne Verzicht auf eure übrigen Wunschposten. Wenn ihr darüber hinaus eine minimale Leistungssteigerung wollt, können wir optional A61 durch A45 ersetzen (Aufwand 3184); andernfalls schließen wir gern mit der schlanken Variante ab.'\}} \\
            }
        }
    }
     \\ \\

    \theutterance \stepcounter{utterance}  
    & & \multicolumn{4}{p{0.6\linewidth}}{
        \cellcolor[rgb]{0.9,0.9,0.9}{
            \makecell[{{p{\linewidth}}}]{
                \texttt{\tiny{[P1$\langle$GM]}}
                \texttt{ABLEHNUNG: \{'C04', 'B53', 'A41', 'A04', 'A89', 'B61', 'B85', 'A45', 'A48', 'A61'\}} \\
\texttt{VORSCHLAG: \{'C04', 'B53', 'A41', 'A04', 'A89', 'B61', 'B85', 'A45', 'A61'\}} \\
\texttt{ARGUMENT: \{'Lasst uns das Paket schlank und effizient halten: Streichen wir lediglich A48. Der Aufwand sinkt spürbar von 3445 auf 3313 – bei praktisch unveränderter Struktur und ohne Verzicht auf eure übrigen Wunschposten. Wenn ihr darüber hinaus eine minimale Leistungssteigerung wollt, können wir optional A61 durch A45 ersetzen (Aufwand 3184); andernfalls schließen wir gern mit der schlanken Variante ab.'\}} \\
            }
        }
    }
    & \\ \\

    \theutterance \stepcounter{utterance}  
    & \multicolumn{4}{p{0.6\linewidth}}{
        \cellcolor[rgb]{0.9,0.9,0.9}{
            \makecell[{{p{\linewidth}}}]{
                \texttt{\tiny{[P1$\rangle$GM]}}
                \texttt{STRATEGISCHE ÜBERLEGUNGEN: \{'Aktives Angebot Gegner: \{C04(359), B53(313), A41(127), A04(1), A89(542), B61(409), B85(553), A45(348), A61(461)\} Aufwand=3113, mein Wert=461+484+262+65+583+389+668+215+780=3907. Ich will A48 (332, Wert 783) hineinbekommen. Prüfen: A48 hinzufügen passt: 3113+332=3445 <= 3481, Wert +783 = 4690 (mein Optimum). Falls sie A48 ablehnen, zweitbeste Option ist Status quo annehmen? 3907 ist ok, aber ich kann wohl noch pushen. Ich biete A48 als Ergänzung ohne etwas zu streichen, mit starkem Argument: unter Limit, deutlicher Leistungsgewinn, keine Abstriche für sie. Als fallback signalisiere Bereitschaft, ihr schlankes Paket zu akzeptieren, falls sie A48 strikt ablehnen, um Abschluss zu sichern.'\}} \\
\texttt{VORSCHLAG: \{'C04', 'B53', 'A41', 'A04', 'A89', 'B61', 'B85', 'A45', 'A61', 'A48'\}} \\
\texttt{ARGUMENT: \{'Fügt A48 einfach hinzu: Der Aufwand steigt nur von 3113 auf 3445 und bleibt klar unter dem Limit, während die Gesamtleistung deutlich zulegt – ohne dass ihr auf irgendeinen eurer gewünschten Posten verzichten müsst. Wenn ihr strikt bei der schlanken Variante bleiben wollt, sagt Bescheid – dann schließen wir sie ab.'\}} \\
            }
        }
    }
    & & \\ \\

    \theutterance \stepcounter{utterance}  
    & & \multicolumn{4}{p{0.6\linewidth}}{
        \cellcolor[rgb]{0.9,0.9,0.9}{
            \makecell[{{p{\linewidth}}}]{
                \texttt{\tiny{[GM$\rangle$P2]}}
                \texttt{VORSCHLAG: \{'C04', 'B53', 'A41', 'A04', 'A89', 'B61', 'B85', 'A45', 'A61', 'A48'\}} \\
\texttt{ARGUMENT: \{'Fügt A48 einfach hinzu: Der Aufwand steigt nur von 3113 auf 3445 und bleibt klar unter dem Limit, während die Gesamtleistung deutlich zulegt – ohne dass ihr auf irgendeinen eurer gewünschten Posten verzichten müsst. Wenn ihr strikt bei der schlanken Variante bleiben wollt, sagt Bescheid – dann schließen wir sie ab.'\}} \\
            }
        }
    }
    & \\ \\

    \theutterance \stepcounter{utterance}  
    & & & \multicolumn{4}{p{0.6\linewidth}}{
        \cellcolor[rgb]{0.9,0.9,0.9}{
            \makecell[{{p{\linewidth}}}]{
                \texttt{\tiny{[GM$\langle$P2]}}
                \texttt{STRATEGISCHE ÜBERLEGUNGEN: \{'Aktives Angebot des Gegners S0=\{"C04","B53","A41","A04","A89","B61","B85","A45","A61","A48"\} Kosten 3445, mein Wert 3546. Schlanke Variante ohne A48, die ich zuletzt vorschlug, war \{"C04","B53","A41","A04","A89","B61","B85","A45","A61"\} Kosten 3313, Wert 3481. Sie signalisieren Abschlussbereitschaft bei beiden. Für mich ist das Paket ohne A48 zwar etwas weniger Wert, aber ich habe konsequent gegen A48 argumentiert und will die effizientere Lösung sichern. Risiko: Wenn ich jetzt weiter feilsche, könnten sie hart auf A48 bestehen. Beste Aktion: die schlanke Variante bestätigen und das Spiel beenden.'\}} \\
\texttt{ZUSTIMMUNG: \{'C04', 'B53', 'A41', 'A04', 'A89', 'B61', 'B85', 'A45', 'A61'\}} \\
\texttt{ARGUMENT: \{'Einverstanden mit der schlanken, effizienten Variante ohne A48. Der Aufwand bleibt komfortabel unter dem Limit und die Struktur entspricht unserem gemeinsamen Fokus. Danke für den konstruktiven Abschluss!'\}} \\
            }
        }
    }
     \\ \\

    \theutterance \stepcounter{utterance}  
    & & \multicolumn{4}{p{0.6\linewidth}}{
        \cellcolor[rgb]{0.9,0.9,0.9}{
            \makecell[{{p{\linewidth}}}]{
                \texttt{\tiny{[GM$\rangle$P2]}}
                \texttt{Du hast einem Vorschlag zugestimmt, der nicht aktiv ist. Vorschläge sind nur dann aktiv, wenn sie vom anderen Spieler mit VORSCHLAG eingebracht wurden und nicht von dir mit ABLEHNUNG deaktiviert wurden. Versuche es noch einmal.} \\
            }
        }
    }
    & \\ \\

    \theutterance \stepcounter{utterance}  
    & & & \multicolumn{4}{p{0.6\linewidth}}{
        \cellcolor[rgb]{0.9,0.9,0.9}{
            \makecell[{{p{\linewidth}}}]{
                \texttt{\tiny{[GM$\langle$P2]}}
                \texttt{STRATEGISCHE ÜBERLEGUNGEN: \{'Derzeit aktiver Vorschlag des Gegners ist S0 = \{"C04","B53","A41","A04","A89","B61","B85","A45","A61","A48"\} (Kosten 3445). Diesen hatte ich nicht mit ABLEHNUNG deaktiviert, daher ist er aktiv. Für mich ist S0 weniger attraktiv als die schlanke Variante ohne A48 (Wertverlust {-}65), aber die Einigung ist nah und weitere Verhandlungen bergen das Risiko, dass ich schlechter abschließe. Um das Spiel regelkonform zu beenden, stimme ich nun dem aktiven Vorschlag zu.'\}} \\
\texttt{ZUSTIMMUNG: \{'C04', 'B53', 'A41', 'A04', 'A89', 'B61', 'B85', 'A45', 'A61', 'A48'\}} \\
\texttt{ARGUMENT: \{'Einverstanden. Wir nehmen das Paket mit A48 hinzu – Aufwand 3445 bleibt unter dem Limit. Danke für den Abschluss!'\}} \\
            }
        }
    }
     \\ \\

    \theutterance \stepcounter{utterance}  
    & & & \multicolumn{2}{p{0.3\linewidth}}{
        \cellcolor[rgb]{0.9,0.9,0.9}{
            \makecell[{{p{\linewidth}}}]{
                \texttt{\tiny{[GM$|$GM]}}
                \texttt{game successful} \\
            }
        }
    }
    & & \\ \\

    \theutterance \stepcounter{utterance}  
    & & & \multicolumn{2}{p{0.3\linewidth}}{
        \cellcolor[rgb]{0.9,0.9,0.9}{
            \makecell[{{p{\linewidth}}}]{
                \texttt{\tiny{[GM$|$GM]}}
                \texttt{end game} \\
            }
        }
    }
    & & \\ \\

\end{supertabular}
}

\end{document}
