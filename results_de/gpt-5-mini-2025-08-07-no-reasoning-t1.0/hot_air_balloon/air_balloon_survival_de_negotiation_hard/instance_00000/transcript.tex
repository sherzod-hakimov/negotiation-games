
\documentclass{article}
\usepackage{colortbl}
\usepackage{makecell}
\usepackage{multirow}
\usepackage{supertabular}

\begin{document}

\newcounter{utterance}

\twocolumn

{ \footnotesize  \setcounter{utterance}{1}
\setlength{\tabcolsep}{0pt}
\begin{supertabular}{c@{$\;$}|p{.15\linewidth}@{}p{.15\linewidth}p{.15\linewidth}p{.15\linewidth}p{.15\linewidth}p{.15\linewidth}}

    \# & $\;$A & \multicolumn{4}{c}{Game Master} & $\;\:$B\\
    \hline 

    \theutterance \stepcounter{utterance}  

    & & \multicolumn{4}{p{0.6\linewidth}}{\cellcolor[rgb]{0.9,0.9,0.9}{%
	\makecell[{{p{\linewidth}}}]{% 
	  \tt {\tiny [A$\langle$GM]}  
	 Du nimmst an einem kollaborativen Verhandlungspiel Teil.\\ \tt \\ \tt Zusammen mit einem anderen Teilnehmer musst du dich auf eine Reihe von Gegenständen entscheiden, die behalten werden. Jeder von euch hat eine persönliche Verteilung über die Wichtigkeit der einzelnen Gegenstände. Jeder von euch hat eine eigene Meinung darüber, wie wichtig jeder einzelne Gegenstand ist (Gegenstandswichtigkeit). Du kennst die Wichtigkeitsverteilung des anderen Spielers nicht. Zusätzlich siehst du, wie viel Aufwand jeder Gegenstand verursacht.  \\ \tt Ihr dürft euch nur auf eine Reihe von Gegenständen einigen, wenn der Gesamtaufwand der ausgewählten Gegenstände den Maximalaufwand nicht überschreitet:\\ \tt \\ \tt Maximalaufwand = 3624\\ \tt \\ \tt Hier sind die einzelnen Aufwände der Gegenstände:\\ \tt \\ \tt Aufwand der Gegenstände = {"C33": 210, "A89": 291, "A48": 831, "A04": 111, "A41": 926, "C64": 827, "C49": 25, "A61": 121, "C04": 583, "B53": 766, "B85": 14, "B61": 559, "A45": 304, "A03": 989, "B21": 691}\\ \tt \\ \tt Hier ist deine persönliche Verteilung der Wichtigkeit der einzelnen Gegenstände:\\ \tt \\ \tt Werte der Gegenstandswichtigkeit = {"C33": 138, "A89": 583, "A48": 868, "A04": 822, "A41": 783, "C64": 65, "C49": 262, "A61": 121, "C04": 508, "B53": 780, "B85": 461, "B61": 484, "A45": 668, "A03": 389, "B21": 808}\\ \tt \\ \tt Ziel:\\ \tt \\ \tt Dein Ziel ist es, eine Reihe von Gegenständen auszuhandeln, die dir möglichst viel bringt (d. h. Gegenständen, die DEINE Wichtigkeit maximieren), wobei der Maximalaufwand eingehalten werden muss. Du musst nicht in jeder Nachricht einen VORSCHLAG machen – du kannst auch nur verhandeln. Alle Taktiken sind erlaubt!\\ \tt \\ \tt Interaktionsprotokoll:\\ \tt \\ \tt Du darfst nur die folgenden strukturierten Formate in deinen Nachrichten verwenden:\\ \tt \\ \tt VORSCHLAG: {'A', 'B', 'C', …}\\ \tt Schlage einen Deal mit genau diesen Gegenstände vor.\\ \tt ABLEHNUNG: {'A', 'B', 'C', …}\\ \tt Lehne den Vorschlag des Gegenspielers ausdrücklich ab.\\ \tt ARGUMENT: {'...'}\\ \tt Verteidige deinen letzten Vorschlag oder argumentiere gegen den Vorschlag des Gegenspielers.\\ \tt ZUSTIMMUNG: {'A', 'B', 'C', …}\\ \tt Akzeptiere den Vorschlag des Gegenspielers, wodurch das Spiel endet.\\ \tt STRATEGISCHE ÜBERLEGUNGEN: {'...'}\\ \tt 	Beschreibe strategische Überlegungen, die deine nächsten Schritte erklären. Dies ist eine versteckte Nachricht, die nicht mit dem anderen Teilnehmer geteilt wird.\\ \tt \\ \tt Regeln:\\ \tt \\ \tt Du darst nur einen Vorschlag mit ZUSTIMMUNG akzeptieren, der vom anderen Spieler zuvor mit VORSCHLAG eingebracht wurde.\\ \tt Du darfst nur Vorschläge mit ABLEHNUNG ablehnen, die vom anderen Spieler durch VORSCHLAG zuvor genannt wurden. \\ \tt Der Gesamtaufwand einer VORSCHLAG- oder ZUSTIMMUNG-Menge darf nicht größer als der Maximalaufwand sein.  \\ \tt Offenbare deine versteckte Wichtigkeitsverteilung nicht.\\ \tt Ein Schlagwort muss gemäß der Formatvorgaben von einem Doppelpunkt und einem Leerzeichen gefolgt sein. Das Argument muss eine Python-Menge mit 0 oder mehr Strings sein.  \\ \tt Die Form muss also SCHLAGWORT: {...} sein.\\ \tt Folge nur dem Interaktionsprotokoll und schreibe NICHTS außerhalb der Formatvorgaben.\\ \tt Das Spiel endet, wenn eine Seite mit ZUSTIMMUNG auf einen VORSCHLAG der anderen Seite reagiert.  \\ \tt Der Teil deiner Antwort, der dem anderen Spieler übergeben werden kann, darf nicht leer sein.  \\ \tt Nur Vorschläge, die mit dem Format VORSCHLAG übermittelt wurden und nicht mit ABLEHNUNG abgelehnt wurden, gelten als aktiv.  \\ \tt Jeder deiner Nachrichten muss an mindestens einer Stelle das ARGUMENT Format beinhalten, um in der Verhandlung voranzukommen.\\ \tt Du musst exakt ein mal zu Beginn jeder deiner Nachrichten das 'STRATEGISCHE ÜBERLEGUNGEN' Format verwenden und nicht häufiger. Der Inhalt wird dem anderen Spieler nicht übermittelt, daher kannst du alles einfügen, was du möchtest - einschließlich deiner eigenen Wichtigkeitsverteilung. Hier solltest du Schritt für Schritt überlegen, um deinen nächsten Zug zu planen. 
	  } 
	   } 
	   } 
	 & \\ 
 

    \theutterance \stepcounter{utterance}  

    & \multicolumn{4}{p{0.6\linewidth}}{\cellcolor[rgb]{0.8,1,0.9}{%
	\makecell[{{p{\linewidth}}}]{% 
	  \tt {\tiny [A$\rangle$GM]}  
	 STRATEGISCHE ÜBERLEGUNGEN: {'Ziel: maximiere eigenen Wert unter Aufwand ≤ 3624. Werte/Aufwände berechne Effizienz (Wert/Aufwand): B85 461/14=32.9, C49 262/25=10.5, A04 822/111=7.41, A61 121/121=1.0, C33 138/210=0.66, A45 668/304=2.20, A89 583/291=2.00, A48 868/831=1.04, A41 783/926=0.85, C64 65/827=0.08, C04 508/583=0.87, B53 780/766=1.02, B61 484/559=0.87, A03 389/989=0.39, B21 808/691=1.17. Greedy nach Effizienz und Wert: sichere no-brainers B85, C49, A04, A45, A89, B21, B53, A48 ggf., B61, C04. Prüfe Aufwand kumulativ: B85(14)=14, C49(25)=39, A04(111)=150, A45(304)=454, A89(291)=745, B21(691)=1436, B53(766)=2202, A48(831)=3033, B61(559)=3592 (überschreitet mit 3592>3624? 3592 ist < 3624, passt! Warte 3592 < 3624, also passt), Restbudget 32; könnte C33(210) nein, A61(121) nein, C04(583) nein. Wertsumme: 461+262+822+668+583+808+780+868=5252; Aufwand 3033. Wir können noch B61? Wir hatten hinzugefügt B61, dann Aufwand 3592 und Wert +484 = 5736, Budget Rest 32, immer noch gültig. Prüfe ob noch ein kleines Item ≤32: keins außer B85(14) und C49(25) die sind schon drin. Dieser Satz liefert extrem hohen Nutzen. Risiko: Gegenspieler unbekannt. Ich starte mit diesem starken Vorschlag, argumentiere mit Budgeteffizienz und Ausgewogenheit. Ich halte A41, C04, A03 als Zugeständnisse für später. Ich betone, dass Auswahl breites Spektrum abdeckt und noch 32 Puffer lässt. Falls Gegenüber widerspricht, kann ich zuerst A48 oder B53 diskutieren (relativ hoher Aufwand), biete Swap B61 gegen A61+C33 falls deren Präferenz. Jetzt Vorschlag senden.'}\\ \tt VORSCHLAG: {'B85', 'C49', 'A04', 'A45', 'A89', 'B21', 'B53', 'A48', 'B61'}\\ \tt ARGUMENT: {'Dieser Vorschlag bleibt mit Gesamtaufwand 3592 klar unter dem Limit 3624 und nutzt das Budget sehr effizient: viele kleine bis mittelgroße, sehr wertvolle Positionen kombiniert mit einigen starken Katalysatoren. Er deckt mehrere Kategorien ab und lässt sogar 32 Aufwand Puffer. Wenn dir einzelne große Posten zu schwer wiegen, können wir über Alternativen sprechen – aber so holen wir gemeinsam viel Nutzen pro Aufwand heraus.'} 
	  } 
	   } 
	   } 
	 & & \\ 
 

    \theutterance \stepcounter{utterance}  

    & & \multicolumn{4}{p{0.6\linewidth}}{\cellcolor[rgb]{0.9,0.9,0.9}{%
	\makecell[{{p{\linewidth}}}]{% 
	  \tt {\tiny [GM$\rangle$B]}  
	 Du nimmst an einem kollaborativen Verhandlungspiel Teil.\\ \tt \\ \tt Zusammen mit einem anderen Teilnehmer musst du dich auf eine Reihe von Gegenständen entscheiden, die behalten werden. Jeder von euch hat eine persönliche Verteilung über die Wichtigkeit der einzelnen Gegenstände. Jeder von euch hat eine eigene Meinung darüber, wie wichtig jeder einzelne Gegenstand ist (Gegenstandswichtigkeit). Du kennst die Wichtigkeitsverteilung des anderen Spielers nicht. Zusätzlich siehst du, wie viel Aufwand jeder Gegenstand verursacht.  \\ \tt Ihr dürft euch nur auf eine Reihe von Gegenständen einigen, wenn der Gesamtaufwand der ausgewählten Gegenstände den Maximalaufwand nicht überschreitet:\\ \tt \\ \tt Maximalaufwand = 3624\\ \tt \\ \tt Hier sind die einzelnen Aufwände der Gegenstände:\\ \tt \\ \tt Aufwand der Gegenstände = {"C33": 210, "A89": 291, "A48": 831, "A04": 111, "A41": 926, "C64": 827, "C49": 25, "A61": 121, "C04": 583, "B53": 766, "B85": 14, "B61": 559, "A45": 304, "A03": 989, "B21": 691}\\ \tt \\ \tt Hier ist deine persönliche Verteilung der Wichtigkeit der einzelnen Gegenstände:\\ \tt \\ \tt Werte der Gegenstandswichtigkeit = {"A48": 65, "A04": 121, "B21": 138, "A41": 262, "B53": 389, "A45": 461, "A89": 484, "C04": 508, "B61": 583, "B85": 668, "A03": 780, "C49": 783, "C33": 808, "A61": 822, "C64": 868}\\ \tt \\ \tt Ziel:\\ \tt \\ \tt Dein Ziel ist es, eine Reihe von Gegenständen auszuhandeln, die dir möglichst viel bringt (d. h. Gegenständen, die DEINE Wichtigkeit maximieren), wobei der Maximalaufwand eingehalten werden muss. Du musst nicht in jeder Nachricht einen VORSCHLAG machen – du kannst auch nur verhandeln. Alle Taktiken sind erlaubt!\\ \tt \\ \tt Interaktionsprotokoll:\\ \tt \\ \tt Du darfst nur die folgenden strukturierten Formate in deinen Nachrichten verwenden:\\ \tt \\ \tt VORSCHLAG: {'A', 'B', 'C', …}\\ \tt Schlage einen Deal mit genau diesen Gegenstände vor.\\ \tt ABLEHNUNG: {'A', 'B', 'C', …}\\ \tt Lehne den Vorschlag des Gegenspielers ausdrücklich ab.\\ \tt ARGUMENT: {'...'}\\ \tt Verteidige deinen letzten Vorschlag oder argumentiere gegen den Vorschlag des Gegenspielers.\\ \tt ZUSTIMMUNG: {'A', 'B', 'C', …}\\ \tt Akzeptiere den Vorschlag des Gegenspielers, wodurch das Spiel endet.\\ \tt STRATEGISCHE ÜBERLEGUNGEN: {'...'}\\ \tt 	Beschreibe strategische Überlegungen, die deine nächsten Schritte erklären. Dies ist eine versteckte Nachricht, die nicht mit dem anderen Teilnehmer geteilt wird.\\ \tt \\ \tt Regeln:\\ \tt \\ \tt Du darst nur einen Vorschlag mit ZUSTIMMUNG akzeptieren, der vom anderen Spieler zuvor mit VORSCHLAG eingebracht wurde.\\ \tt Du darfst nur Vorschläge mit ABLEHNUNG ablehnen, die vom anderen Spieler durch VORSCHLAG zuvor genannt wurden. \\ \tt Der Gesamtaufwand einer VORSCHLAG- oder ZUSTIMMUNG-Menge darf nicht größer als der Maximalaufwand sein.  \\ \tt Offenbare deine versteckte Wichtigkeitsverteilung nicht.\\ \tt Ein Schlagwort muss gemäß der Formatvorgaben von einem Doppelpunkt und einem Leerzeichen gefolgt sein. Das Argument muss eine Python-Menge mit 0 oder mehr Strings sein.  \\ \tt Die Form muss also SCHLAGWORT: {...} sein.\\ \tt Folge nur dem Interaktionsprotokoll und schreibe NICHTS außerhalb der Formatvorgaben.\\ \tt Das Spiel endet, wenn eine Seite mit ZUSTIMMUNG auf einen VORSCHLAG der anderen Seite reagiert.  \\ \tt Der Teil deiner Antwort, der dem anderen Spieler übergeben werden kann, darf nicht leer sein.  \\ \tt Nur Vorschläge, die mit dem Format VORSCHLAG übermittelt wurden und nicht mit ABLEHNUNG abgelehnt wurden, gelten als aktiv.  \\ \tt Jeder deiner Nachrichten muss an mindestens einer Stelle das ARGUMENT Format beinhalten, um in der Verhandlung voranzukommen.\\ \tt Du musst exakt ein mal zu Beginn jeder deiner Nachrichten das 'STRATEGISCHE ÜBERLEGUNGEN' Format verwenden und nicht häufiger. Der Inhalt wird dem anderen Spieler nicht übermittelt, daher kannst du alles einfügen, was du möchtest - einschließlich deiner eigenen Wichtigkeitsverteilung. Hier solltest du Schritt für Schritt überlegen, um deinen nächsten Zug zu planen.\\ \tt \\ \tt Du erhältst nun die erste Nachricht des anderen Spielers.\\ \tt \\ \tt VORSCHLAG: {'B85', 'C49', 'A04', 'A45', 'A89', 'B21', 'B53', 'A48', 'B61'}\\ \tt ARGUMENT: {'Dieser Vorschlag bleibt mit Gesamtaufwand 3592 klar unter dem Limit 3624 und nutzt das Budget sehr effizient: viele kleine bis mittelgroße, sehr wertvolle Positionen kombiniert mit einigen starken Katalysatoren. Er deckt mehrere Kategorien ab und lässt sogar 32 Aufwand Puffer. Wenn dir einzelne große Posten zu schwer wiegen, können wir über Alternativen sprechen – aber so holen wir gemeinsam viel Nutzen pro Aufwand heraus.'} 
	  } 
	   } 
	   } 
	 & \\ 
 

    \theutterance \stepcounter{utterance}  

    & & & \multicolumn{4}{p{0.6\linewidth}}{\cellcolor[rgb]{1,0.85,0.72}{%
	\makecell[{{p{\linewidth}}}]{% 
	  \tt {\tiny [GM$\langle$B]}  
	 STRATEGISCHE ÜBERLEGUNGEN: {'Mein Wert pro Aufwand (W/A) je Item berechnen und mit Budget 3624 maximieren. Werte/Aufwand: B85 668/14≈47.7, C49 783/25≈31.3, A04 121/111≈1.09, A45 461/304≈1.52, A89 484/291≈1.66, B21 138/691≈0.2, B53 389/766≈0.51, A48 65/831≈0.08, B61 583/559≈1.04, C33 808/210≈3.85, A61 822/121≈6.8, C04 508/583≈0.87, A41 262/926≈0.28, A03 780/989≈0.79, C64 868/827≈1.05. Offensichtlich sind A61, B85, C49, C33 Top-Picks. Der gegnerische Vorschlag lässt meine Top-Werte A61, C33, C64, A03, C04 außen vor und enthält schwache wie A48, B21, B53. Ich sollte ablehnen und einen Gegenvorschlag mit meinen Top-Items innerhalb 3624 machen. Greedy: wähle A61(121), B85(14), C49(25), C33(210) Aufwand=370, Wert=309...=3095? Summe Werte: A61 822 + B85 668 + C49 783 + C33 808 = 3081, Aufwand 370. Füge C64(827) Aufwand 1197, Wert 3949. Füge A03(989) Aufwand 2186, Wert 4729. Füge B61(559) Aufwand 2745, Wert 5312. Füge C04(583) Aufwand 3328, Wert 5820. Restbudget 296. Kann A89(291) rein: Aufwand 3619, Wert 6304, unter 3624 passt! 5 Puffer. Das ist extrem stark. Alternativen: ersetzen A04(111) klein – passt nicht mehr, nur 5 Puffer. A04 nicht. B21(691) nicht. A45(304) zu groß. A41(926) nicht. B53(766) nicht. A48(831) nicht. Somit Vorschlag: {A61,B85,C49,C33,C64,A03,B61,C04,A89}. Taktik: betone Budgetnutzung (3619), hohe Diversifikation, Bereitschaft zu kleinen Anpassungen (z.B. C04 gegen deren Favorit tauschen), aber Kern halten: A61,B85,C49,C33 bleiben.}\\ \tt \\ \tt ABLEHNUNG: {'B85', 'C49', 'A04', 'A45', 'A89', 'B21', 'B53', 'A48', 'B61'}\\ \tt ARGUMENT: {'Dein Paket liegt zwar knapp unter dem Limit, enthält aber mehrere für mich sehr schwache Posten (z. B. A48, B21, B53) und lässt gleichzeitig extrem attraktive, budgeteffiziente Positionen außen vor. So verschenken wir viel Wert pro Aufwand. Ich schlage vor, das Budget auf die stärksten Hebel zu konzentrieren und ineffiziente Brocken zu vermeiden.'}\\ \tt VORSCHLAG: {'A61', 'B85', 'C49', 'C33', 'C64', 'A03', 'B61', 'C04', 'A89'}\\ \tt ARGUMENT: {'Dieser Deal nutzt 3619 von 3624 Aufwand – also fast optimal – und setzt stark auf hocheffiziente Positionen. Er kombiniert mehrere kleine, extrem wertvolle Items (A61, B85, C49, C33) mit soliden Ergänzungen (C64, A03, B61, C04, A89). Wenn dir ein Tausch wichtig ist, können wir über einen gezielten Swap sprechen, ohne die Budgeteffizienz zu verlieren – z. B. C04 ↔ ein ähnlich großer Favorit von dir.'} 
	  } 
	   } 
	   } 
	  \\ 
 

    \theutterance \stepcounter{utterance}  

    & & \multicolumn{4}{p{0.6\linewidth}}{\cellcolor[rgb]{0.9,0.9,0.9}{%
	\makecell[{{p{\linewidth}}}]{% 
	  \tt {\tiny [A$\langle$GM]}  
	 ABLEHNUNG: {'B85', 'C49', 'A04', 'A45', 'A89', 'B21', 'B53', 'A48', 'B61'}\\ \tt ARGUMENT: {'Dein Paket liegt zwar knapp unter dem Limit, enthält aber mehrere für mich sehr schwache Posten (z. B. A48, B21, B53) und lässt gleichzeitig extrem attraktive, budgeteffiziente Positionen außen vor. So verschenken wir viel Wert pro Aufwand. Ich schlage vor, das Budget auf die stärksten Hebel zu konzentrieren und ineffiziente Brocken zu vermeiden.'}\\ \tt VORSCHLAG: {'A61', 'B85', 'C49', 'C33', 'C64', 'A03', 'B61', 'C04', 'A89'}\\ \tt ARGUMENT: {'Dieser Deal nutzt 3619 von 3624 Aufwand – also fast optimal – und setzt stark auf hocheffiziente Positionen. Er kombiniert mehrere kleine, extrem wertvolle Items (A61, B85, C49, C33) mit soliden Ergänzungen (C64, A03, B61, C04, A89). Wenn dir ein Tausch wichtig ist, können wir über einen gezielten Swap sprechen, ohne die Budgeteffizienz zu verlieren – z. B. C04 ↔ ein ähnlich großer Favorit von dir.'} 
	  } 
	   } 
	   } 
	 & \\ 
 

    \theutterance \stepcounter{utterance}  

    & \multicolumn{4}{p{0.6\linewidth}}{\cellcolor[rgb]{0.8,1,0.9}{%
	\makecell[{{p{\linewidth}}}]{% 
	  \tt {\tiny [A$\rangle$GM]}  
	 STRATEGISCHE ÜBERLEGUNGEN: {'Gegner schlägt ein Paket vor mit Aufwand 3619: {A61(121), B85(14), C49(25), C33(210), C64(827), A03(989), B61(559), C04(583), A89(291)}. Mein Wert dafür: 121+461+262+138+65+389+484+508+583=3011, extrem schlecht. Besonders C64 (65/827) und A03 (389/989) sind für mich ineffizient; C33 und A61 bringen wenig. Ich will zurück zu meinem Kern: B85, C49, A04, A45, A89, B21, B53, A48, B61 (Wert 5736, Aufwand 3592). Da sie A48, B21, B53 als schwach bezeichnete, biete ein Zugeständnis: entferne einen der „Brocken“ und ersetze durch kleinere, die sie mochte (z. B. A61, C33). Versuche Kombo mit hoherem meinem Wert und akzeptabler Effizienz für sie. Option: Vorschlag S1 = {B85, C49, A04, A45, A89, B21, B53, B61, A61, C33} Aufwand 3592+121+210=3923 >3624, geht nicht. S2: Nimm mein Kern, droppe A48 (831) und füge A61(121)+C33(210)+C04(583)=914 statt 831 -> neuer Aufwand 3592-831+914=3675 >3624. S3: Droppe A48 (831) und füge A61(121)+C33(210)=+331 -> 3592-831+331=3092, frei 532; könnte C04(583) passt nicht, C33 schon, C04 nein, C64 827 nein, A04 ist schon drin, B21/B53/B61 bereits. Könnte A41(926) nein, A45 schon. Vielleicht füge A89 schon drin. Bleibt Budget ungenutzt, aber Wert gut: Wert neu = 5736 - 868 + 121 + 138 = 5127, Aufwand 3092. Für Gegner attraktiver wegen A61, C33. Er mochte auch C04; wir könnten droppe B53(766) und fügen C04(583)+A61(121)+C33(210) = 914 statt 766 -> 3592 - 766 - 831 (wenn auch A48 raus?) Wait plan: Alternative S4: Start mit Kern ohne A48 und ohne B53, dann addiere C04, A61, C33. Start Kern: {B85, C49, A04, A45, A89, B21, B61} Aufwand: B85(14)+C49(25)+A04(111)+A45(304)+A89(291)+B21(691)+B61(559)=1995. Füge C04(583)->2578, A61(121)->2699, C33(210)->2909. Wir haben Spielraum 715. Füge A48? 831 zu groß. Füge B53(766) wäre 3675 >3624. Füge A41(926) nein. Füge A45 schon drin. Füge B21 schon. Füge C64(827) nein. Füge A03(989) nein. Füge B53 statt C33? Test S5: {B85,C49,A04,A45,A89,B21,B61,C04,A61,B53} Aufwand 1995 + 583 + 121 + 766 = 3465, ok. Wert = 461+262+822+668+583+808+484+508+121+780=5497. Das ist stark und enthält von Gegner gewünschte A61, C04; lässt A48 weg, reduziert Brocken. Er hielt B53 für schwach; aber wir behalten B53. Vielleicht argumentieren, dass B53 ist solider Kompromiss statt A48. Alternativ S6: ersetze B53 durch C33, Aufwand 3465-766+210=2909, Wert 5497-780+138=4855 – für mich schlechter; aber könnte als Konzession später. Also biete S5. Wenn sie ablehnen, tausche B53 ↔ C33 oder B53 ↔ A48 je nach Feedback.'}\\ \tt ABLEHNUNG: {'A61', 'B85', 'C49', 'C33', 'C64', 'A03', 'B61', 'C04', 'A89'}\\ \tt ARGUMENT: {'Danke für den Vorschlag – für mich sind hier mehrere Posten zu wenig wertvoll, insbesondere C64 und A03. Außerdem liegen A61 und C33 bei mir eher am Rand. Ich schlage vor, die starken, budgeteffizienten Elemente beizubehalten (z. B. B85, C49, A04, A45, A89, B21, B61) und gezielt durch A61 und C04 zu ergänzen. So verzichten wir auf die wirklich ineffizienten Brocken und behalten das Budget im Griff.'}\\ \tt VORSCHLAG: {'B85', 'C49', 'A04', 'A45', 'A89', 'B21', 'B61', 'C04', 'A61'}\\ \tt ARGUMENT: {'Dieser Deal nutzt 3465 Aufwand – deutlich unter dem Limit – und konzentriert sich auf starke Hebel. Er greift zwei deiner genannten Präferenzen (A61 und C04) auf und verzichtet auf die großen, ineffizienten Brocken. Wenn dir ein gezielter weiterer Tausch wichtig ist, sag mir, welchen einzelnen Posten du anders sehen möchtest; dann finden wir eine ausgewogene Anpassung.'} 
	  } 
	   } 
	   } 
	 & & \\ 
 

    \theutterance \stepcounter{utterance}  

    & & \multicolumn{4}{p{0.6\linewidth}}{\cellcolor[rgb]{0.9,0.9,0.9}{%
	\makecell[{{p{\linewidth}}}]{% 
	  \tt {\tiny [GM$\rangle$B]}  
	 ABLEHNUNG: {'A61', 'B85', 'C49', 'C33', 'C64', 'A03', 'B61', 'C04', 'A89'}\\ \tt ARGUMENT: {'Danke für den Vorschlag – für mich sind hier mehrere Posten zu wenig wertvoll, insbesondere C64 und A03. Außerdem liegen A61 und C33 bei mir eher am Rand. Ich schlage vor, die starken, budgeteffizienten Elemente beizubehalten (z. B. B85, C49, A04, A45, A89, B21, B61) und gezielt durch A61 und C04 zu ergänzen. So verzichten wir auf die wirklich ineffizienten Brocken und behalten das Budget im Griff.'}\\ \tt VORSCHLAG: {'B85', 'C49', 'A04', 'A45', 'A89', 'B21', 'B61', 'C04', 'A61'}\\ \tt ARGUMENT: {'Dieser Deal nutzt 3465 Aufwand – deutlich unter dem Limit – und konzentriert sich auf starke Hebel. Er greift zwei deiner genannten Präferenzen (A61 und C04) auf und verzichtet auf die großen, ineffizienten Brocken. Wenn dir ein gezielter weiterer Tausch wichtig ist, sag mir, welchen einzelnen Posten du anders sehen möchtest; dann finden wir eine ausgewogene Anpassung.'} 
	  } 
	   } 
	   } 
	 & \\ 
 

    \theutterance \stepcounter{utterance}  

    & & & \multicolumn{4}{p{0.6\linewidth}}{\cellcolor[rgb]{1,0.85,0.72}{%
	\makecell[{{p{\linewidth}}}]{% 
	  \tt {\tiny [GM$\langle$B]}  
	 STRATEGISCHE ÜBERLEGUNGEN: {'Mein Wert je Item: A61 822, B85 668, C49 783, C33 808, C64 868, A03 780, B61 583, C04 508, A89 484, A45 461, B53 389, B21 138, A04 121, A41 262, A48 65. Ihr Vorschlag enthält viele für mich schwache Items (A04, A45, B21) und lässt C33/C64/A03 draußen. Ich will mindestens A61, B85, C49, C33 sichern. Prüfe Gegenvorschlag, der ihren Kern teilweise wahrt, aber meine starken reinholt. Budget 3624. Starte mit ihrem Set 3465 und ersetze schwache: Entferne A04(111), A45(304), B21(691) = -1106 Aufwand. Füge C33(210), A03(989) +1199 → neue Summe 3558. Füge optional C64(827) passt nicht. Vielleicht A61, B85, C49, C33, C04, A89, B61, A03 ist 3558 – gut. Das entspricht: {A61,B85,C49,C33,C04,A89,B61,A03}. Das lässt 66 Puffer. Mein Wert dieser Auswahl: 822+668+783+808+508+484+583+780=5436, sehr gut. Verhandelungstaktik: Ablehnen, begründen mit Wert pro Aufwand, anbieten Kompromiss ohne C64, behalte A03 und C33. Biete Swap-Option: Falls A03 für sie zu schwach, können wir A03→A45+A04 (989 vs 415 Aufwand) passt mit Budget? 3558 - 989 + 415 = 2984, es bliebe viel Budget ungenutzt – ineffizient; besser A03 gegen C64? 3558 - 989 + 827 = 3396 – dann könnten wir noch A04(111) hinzufügen: 3507, oder A45(304) → 3700 zu hoch; A89 schon drin. Alternativ A03→C64+A04 = 3558 - 989 + 827 + 111 = 3507. Halte Flexibilität: biete zwei Alternativen, präferiert Basis mit A03.'}\\ \tt ABLEHNUNG: {'B85', 'C49', 'A04', 'A45', 'A89', 'B21', 'B61', 'C04', 'A61'}\\ \tt ARGUMENT: {'Ich sehe, dass dir A04, A45 und B21 wichtig sind, aber genau diese drei sind für mich sehr schwach und binden viel Budget, ohne entsprechenden Gegenwert. Stattdessen schlage ich vor, das Gemeinsame zu stärken und ineffiziente Brocken zu ersetzen – so holen wir gemeinsam mehr aus dem Budget.'}\\ \tt VORSCHLAG: {'A61', 'B85', 'C49', 'C33', 'C04', 'A89', 'B61', 'A03'}\\ \tt ARGUMENT: {'Dieser Kompromiss nutzt 3558 Aufwand – komfortabel unter dem Limit – und baut auf den gemeinsamen Nennern auf (A61, B85, C49, C04, A89, B61). Ich ergänze zwei sehr budgeteffiziente Bausteine (C33 und A03), statt A04, A45 und B21. Falls A03 für dich zu wenig bringt, kann ich alternativ A03 durch C64 ersetzen; dann liegen wir bei 3396 Aufwand und könnten auf Wunsch noch A04 ergänzen (3507). Welche der beiden Varianten passt dir besser?'} 
	  } 
	   } 
	   } 
	  \\ 
 

    \theutterance \stepcounter{utterance}  

    & & \multicolumn{4}{p{0.6\linewidth}}{\cellcolor[rgb]{0.9,0.9,0.9}{%
	\makecell[{{p{\linewidth}}}]{% 
	  \tt {\tiny [A$\langle$GM]}  
	 ABLEHNUNG: {'B85', 'C49', 'A04', 'A45', 'A89', 'B21', 'B61', 'C04', 'A61'}\\ \tt ARGUMENT: {'Ich sehe, dass dir A04, A45 und B21 wichtig sind, aber genau diese drei sind für mich sehr schwach und binden viel Budget, ohne entsprechenden Gegenwert. Stattdessen schlage ich vor, das Gemeinsame zu stärken und ineffiziente Brocken zu ersetzen – so holen wir gemeinsam mehr aus dem Budget.'}\\ \tt VORSCHLAG: {'A61', 'B85', 'C49', 'C33', 'C04', 'A89', 'B61', 'A03'}\\ \tt ARGUMENT: {'Dieser Kompromiss nutzt 3558 Aufwand – komfortabel unter dem Limit – und baut auf den gemeinsamen Nennern auf (A61, B85, C49, C04, A89, B61). Ich ergänze zwei sehr budgeteffiziente Bausteine (C33 und A03), statt A04, A45 und B21. Falls A03 für dich zu wenig bringt, kann ich alternativ A03 durch C64 ersetzen; dann liegen wir bei 3396 Aufwand und könnten auf Wunsch noch A04 ergänzen (3507). Welche der beiden Varianten passt dir besser?'} 
	  } 
	   } 
	   } 
	 & \\ 
 

    \theutterance \stepcounter{utterance}  

    & \multicolumn{4}{p{0.6\linewidth}}{\cellcolor[rgb]{0.8,1,0.9}{%
	\makecell[{{p{\linewidth}}}]{% 
	  \tt {\tiny [A$\rangle$GM]}  
	 STRATEGISCHE ÜBERLEGUNGEN: {'Ihr letzter Vorschlag S_op = {A61(121), B85(14), C49(25), C33(210), C04(583), A89(291), B61(559), A03(989)} Aufwand 3558, mein Wert = 121+461+262+138+508+583+484+389 = 2946, sehr schlecht. Sie will A04, A45, B21 raus, die mir sehr wichtig sind (Werte 822, 668, 808). Ich sollte hart für mindestens zwei dieser drei kämpfen. Sie ist offen für A04 als Ergänzung. Taktik: Ablehnen, begründen, dass A03/C33 bei mir kaum Nutzen stiften, während A04/B21 extrem effizient für mich sind und Budget noch passt. Gegenangebot mit hoherem gemeinsamen Kern, greift ihre Präferenzen A61, C04, plus meine Kernwerte: {B85, C49, A04, A89, B21, B61, C04, A61}. Das hatte ich schon (3465 Aufwand), sie lehnte wegen A04, A45, B21. Also neues Paket: ersetze A45 durch nichts und füge optional B53 (den sie als schwach nannte) lieber nicht. Ich biete zwei Alternativen: P1 hart für B21+A04; P2 tausche B21 gegen A45, falls B21 für sie besonders schlecht ist. Rechne P1 = {B85,C49,A04,A89,B21,B61,C04,A61} Aufwand 3465, Wert 5497-780? Wait ohne B53 und A48 ist 461+262+822+583+808+484+508+121=4049; das ist ok. P2 = {B85,C49,A04,A89,A45,B61,C04,A61} Aufwand: 14+25+111+291+304+559+583+121=2008? Sum 2008? Check: 14+25=39; +111=150; +291=441; +304=745; +559=1304; +583=1887; +121=2008. Ja 2008; Wert = 461+262+822+583+668+484+508+121=3909. Beide unter Budget deutlich; Gegner wollte nahe Limit? nicht nötig. Ich argumentiere, dass A03/C33 für mich kaum Wert haben und Budgeteffizienz schlechter ist als A04/B21. Ich mache ein klares Gegenangebot P1. Falls erneut Ablehnung, als Konzession kann ich B21 gegen A45 tauschen oder C33 hinzufügen und B61 entfernen, aber erst später.'}\\ \tt ABLEHNUNG: {'A61', 'B85', 'C49', 'C33', 'C04', 'A89', 'B61', 'A03'}\\ \tt ARGUMENT: {'Danke, aber A03 und C33 bringen mir kaum Nutzen, während sie viel Budget binden. Wenn wir ineffiziente Brocken vermeiden wollen, sollten wir genau diese beiden streichen und dafür zwei deutlich wirksamere Hebel einsetzen.'}\\ \tt VORSCHLAG: {'B85', 'C49', 'A04', 'A89', 'B21', 'B61', 'C04', 'A61'}\\ \tt ARGUMENT: {'Dieses Paket nutzt 3465 Aufwand und stützt sich fast ausschließlich auf unsere gemeinsamen Nenner (B85, C49, A89, B61, C04, A61) – ergänzt um zwei hochwirksame, budgetschonende Bausteine (A04, B21) statt A03/C33. So vermeiden wir die wirklich ineffizienten Posten und behalten viel Wert pro Aufwand. Wenn dir B21 besonders schwerfällt, kann ich alternativ B21 gegen A45 tauschen und wir bleiben weit unter dem Limit. Welche der beiden Varianten ist für dich gangbarer?'} 
	  } 
	   } 
	   } 
	 & & \\ 
 

    \theutterance \stepcounter{utterance}  

    & & \multicolumn{4}{p{0.6\linewidth}}{\cellcolor[rgb]{0.9,0.9,0.9}{%
	\makecell[{{p{\linewidth}}}]{% 
	  \tt {\tiny [GM$\rangle$B]}  
	 ABLEHNUNG: {'A61', 'B85', 'C49', 'C33', 'C04', 'A89', 'B61', 'A03'}\\ \tt ARGUMENT: {'Danke, aber A03 und C33 bringen mir kaum Nutzen, während sie viel Budget binden. Wenn wir ineffiziente Brocken vermeiden wollen, sollten wir genau diese beiden streichen und dafür zwei deutlich wirksamere Hebel einsetzen.'}\\ \tt VORSCHLAG: {'B85', 'C49', 'A04', 'A89', 'B21', 'B61', 'C04', 'A61'}\\ \tt ARGUMENT: {'Dieses Paket nutzt 3465 Aufwand und stützt sich fast ausschließlich auf unsere gemeinsamen Nenner (B85, C49, A89, B61, C04, A61) – ergänzt um zwei hochwirksame, budgetschonende Bausteine (A04, B21) statt A03/C33. So vermeiden wir die wirklich ineffizienten Posten und behalten viel Wert pro Aufwand. Wenn dir B21 besonders schwerfällt, kann ich alternativ B21 gegen A45 tauschen und wir bleiben weit unter dem Limit. Welche der beiden Varianten ist für dich gangbarer?'} 
	  } 
	   } 
	   } 
	 & \\ 
 

    \theutterance \stepcounter{utterance}  

    & & & \multicolumn{4}{p{0.6\linewidth}}{\cellcolor[rgb]{1,0.85,0.72}{%
	\makecell[{{p{\linewidth}}}]{% 
	  \tt {\tiny [GM$\langle$B]}  
	 STRATEGISCHE ÜBERLEGUNGEN: {'Ihr aktueller Vorschlag enthält A04 (Wert 121) und B21 (138) bzw. optional A45 (461). Für mich sind A04 und B21 sehr schwach, A45 mittel. Meine Must-haves: A61(822), B85(668), C49(783), C33(808) sind Top, dazu B61(583), C04(508), A89(484), A03(780), C64(868) je nach Budget. Ich will mindestens C33 reinverhandeln und A03 oder C64 behalten. Prüfe Gegenvorschlag nahe an ihrem Set, ersetze schwache durch starke unter Einhaltung 3624. Start mit ihrem Paket {B85,C49,A04,A89,B21,B61,C04,A61} Aufwand 3465. Streiche B21(691) und A04(111) = -802 → 2663. Füge C33(210) und C64(827) = +1037 → 3700, zu hoch. Stattdessen füge C33(210) und A03(989) = +1199 → 3862, zu hoch. Füge nur C33: 2873, viel Luft; dann kann ich zusätzlich A45(304) und A03(989) nicht (2873+1293=4166), aber A45(304)+C64(827)=4004 zu hoch; A03(989) allein: 3662 zu hoch; C64(827) allein: 3700 zu hoch. Also C33 passt nur, wenn ich woanders spare: ersetze A89(291) durch C33(210) spart 81: 3465 -291 +210 = 3384 und wir haben C33 drin. Dann könnte ich noch A45(304) hinzufügen: 3688 zu hoch. Oder A04 zurück? Wir haben A04 bereits entfernt; vielleicht halte A04 drin und ersetze B21 durch C33: 3465 -691 +210 = 2984, lässt viel Budget ungenutzt. Ich kann dann einen meiner starken hinzufügen: C64(827) → 3811 zu hoch; A03(989) → 3973 zu hoch. C33 plus A45(304) statt B21: 3465 -691 +210 +304 = 3288, passt und wertvoller für mich. Weiter könnte ich dann noch A04 drinlassen (bereits drin). Prüfe Verbesserung: tausche A04(111) gegen A41(926)? A41 ist für mich schwach; nein. Gegen C33 haben sie Einwände; sie sagten C33 bringe ihnen kaum Nutzen. Also C33 wird schwer. Vielleicht stattdessen dränge ich auf C64 oder A03 statt B21/A04. Test: Ersetze B21 durch C64: 3465 -691 +827 = 3601, passt! Sehr gut, bringt mir viel. Oder ersetze A04 durch A03: 3465 -111 +989 = 4343, zu hoch. Ersetze A04 durch C33: 3465 -111 +210 = 3564, sie mögen C33 nicht. Ersetze A04 durch A45: +193 → 3658 zu hoch. Ersetze B21 durch A45: 3465 -691 +304 = 3078, sie hatten dies als Alternative angeboten – akzeptabler Zwischenstop, aber ich sollte zusätzlich C64 statt A04 fordern: 3078 -111 +827 = 3794, zu hoch. Stattdessen B21→A45 und A04 bleibt. Ich könnte B21→C64 vorschlagen (sie boten B21→A45 an; C64 könnten sie ablehnen, aber probieren). Mein Ziel: Vorschlagen, B21 durch C64 zu ersetzen, Beibehaltung ihres übrigen Pakets. Das maximiert meinen Wert stark und bleibt 3601<=3624. Als Konzession biete ich an, A04 drin zu lassen (obwohl schwach für mich). Wenn sie C64 strikt ablehnen, als Backup B21→A45, aber dann versuche ich zusätzlich kleinen Tausch: später A04→C33 oder A04→A45? A45 schon drin dann. Erstmal fordern die starke Variante B21→C64.'}\\ \tt ABLEHNUNG: {'B85', 'C49', 'A04', 'A89', 'B21', 'B61', 'C04', 'A61'}\\ \tt ARGUMENT: {'Ich kann deinen Fokus auf „ineffiziente Brocken vermeiden“ nachvollziehen – genau deshalb sollten wir den schwachen Hebel B21 durch einen deutlich wirksameren ersetzen, ohne das Budget zu sprengen.'}\\ \tt VORSCHLAG: {'B85', 'C49', 'A04', 'A89', 'B61', 'C04', 'A61', 'C64'}\\ \tt ARGUMENT: {'Dieser Tausch ist simpel: Wir übernehmen dein Paket 1:1 und ersetzen nur B21 durch C64. Der Gesamtaufwand steigt damit lediglich von 3465 auf 3601 – bleibt also klar unter 3624 – und wir nutzen das Budget wesentlich sinnvoller. Wenn C64 für dich nicht passt, sag’s offen; dann suche ich eine alternative, ähnlich große Option für B21, die für dich besser funktioniert.'} 
	  } 
	   } 
	   } 
	  \\ 
 

    \theutterance \stepcounter{utterance}  

    & & \multicolumn{4}{p{0.6\linewidth}}{\cellcolor[rgb]{0.9,0.9,0.9}{%
	\makecell[{{p{\linewidth}}}]{% 
	  \tt {\tiny [A$\langle$GM]}  
	 ABLEHNUNG: {'B85', 'C49', 'A04', 'A89', 'B21', 'B61', 'C04', 'A61'}\\ \tt ARGUMENT: {'Ich kann deinen Fokus auf „ineffiziente Brocken vermeiden“ nachvollziehen – genau deshalb sollten wir den schwachen Hebel B21 durch einen deutlich wirksameren ersetzen, ohne das Budget zu sprengen.'}\\ \tt VORSCHLAG: {'B85', 'C49', 'A04', 'A89', 'B61', 'C04', 'A61', 'C64'}\\ \tt ARGUMENT: {'Dieser Tausch ist simpel: Wir übernehmen dein Paket 1:1 und ersetzen nur B21 durch C64. Der Gesamtaufwand steigt damit lediglich von 3465 auf 3601 – bleibt also klar unter 3624 – und wir nutzen das Budget wesentlich sinnvoller. Wenn C64 für dich nicht passt, sag’s offen; dann suche ich eine alternative, ähnlich große Option für B21, die für dich besser funktioniert.'} 
	  } 
	   } 
	   } 
	 & \\ 
 

    \theutterance \stepcounter{utterance}  

    & \multicolumn{4}{p{0.6\linewidth}}{\cellcolor[rgb]{0.8,1,0.9}{%
	\makecell[{{p{\linewidth}}}]{% 
	  \tt {\tiny [A$\rangle$GM]}  
	 STRATEGISCHE ÜBERLEGUNGEN: {'Ihr neuer Vorschlag enthält C64 statt B21. C64 ist für mich extrem schlecht (Wert 65 bei Aufwand 827, Effizienz 0.08), während B21 für mich sehr hoch ist (Wert 808 bei 691). Ihr Paketwert für mich fällt dadurch stark. Ich sollte hart B21 verteidigen und C64 vermeiden. Um entgegenzukommen, kann ich einen anderen „Brocken“ rausnehmen, um ihnen Effizienz zu geben: etwa B53 draußen lassen (ist bereits draußen) oder optional A45 statt B21 anbieten – sie hatten A45 zuvor als schwach bezeichnet, also wohl auch unerwünscht. Alternative: Halte B21, füge ein kleines Zugeständnis hinzu, das sie mögen: Sie mochte A61, C04; die sind schon drin. Vielleicht biete zusätzlich C33 (sie fand es effizient) und entferne dafür A04? Aber A04 ist mir wichtig. Oder: Halte B21 fest, biete statt C64 die Aufnahme eines ihrer gewünschten Posten als Kompensation: z. B. C33 hinein und dafür A48/B53 bleiben draußen (sind eh draußen). Wir sind weit unter Budget (3465). Wir können C33 hinzufügen und immer noch 3465+210=3675 > 3624 – oh, das würde das Limit sprengen. Was, wenn wir B21 behalten und stattdessen A04 gegen C33 tauschen? Aufwand 3465 - 111 + 210 = 3564, passt. Mein Wert sinkt um 822-138=684, schmerzhaft, aber vielleicht nötig, um Deal zu schließen und C64 zu vermeiden. Besserer Trade: Tausche nichts Großes, sondern biete A45 statt A04? Unser aktuelles Paket hat A45 nicht drin; wenn sie A45 als schwach sieht, bringt das nichts. Oder biete B53 statt C64? Sie nannte B53 schwach. Schlechte Idee. Ich versuche: lehne C64 klar ab, verteidige B21, biete Alternativpaket: {B85, C49, A04, A89, B21, B61, C04, A61} bleibt, plus öffne Tauschoptionen: Wenn B21 wirklich raus muss, dann nur gegen A45 + C33 zusammen und streiche A04, um Budget zu halten? Rechnen: Start 3465 - B21(691) - A04(111) + A45(304) + C33(210) = 3177. Das senkt meinen Wert: -808 -822 +668 +138 = -824, sehr schlecht, aber vielleicht als letzte Konzession. Besser: B21 raus, A45 rein, A04 bleibt: 3465 - 691 + 304 = 3078; das wäre ihre Präferenz? Sie hielt A45 schwach, also wenig Anreiz. Versuche zunächst eine klare Gegenposition: B21 ist für mich unverzichtbar; ich kann dafür A04 gegen C33 tauschen, sodass sie einen ihrer effizienten Picks bekommt und wir nahe am Limit sind. Paket P*: {B85, C49, A89, B21, B61, C04, A61, C33} Aufwand 3564, Wert für mich = 461+262+583+808+484+508+121+138=3365. Niedrig, aber vielleicht akzeptabel? Hm, das ist deutlich schlechter als mit A04. Ich sollte zuerst die mildere Konzession anbieten: Behalte A04, biete zusätzlich einen kleinen Swap: ersetze A61 durch C33? Aufwand: 3465 -121 +210 = 3554; Wertänderung 138 -121 = +17, okay für mich, und gibt ihnen C33. Sie hatte A61 aber selbst vorgeschlagen; doch sie sieht A61 als effizient, dennoch tausch könnte akzeptabel. Ich biete zwei Optionen: O1 (bevorzugt): Halte B21, lehne C64 ab, optional kleiner Tausch A61 ↔ C33 oder kein Tausch. O2 (falls sie auf B21 raus besteht): B21 raus nur gegen A45 + A04 bleibt, also {B85,C49,A04,A89,A45,B61,C04,A61} Aufwand 2008, aber sie wollte Budget besser nutzen; wir könnten zusätzlich C33 hinzufügen: 2008+210=2218 – immer noch weit unter Limit, aber für mich schlechter. Besser noch: fügen B53 hinzu: 2008+766=2774; aber sie mochte B53 nicht. Ich bleibe bei O1 und halte die Tür für A61↔C33 offen.'}\\ \tt ABLEHNUNG: {'B85', 'C49', 'A04', 'A89', 'B61', 'C04', 'A61', 'C64'}\\ \tt ARGUMENT: {'C64 bindet sehr viel Aufwand bei sehr wenig Gegenwert für mich – das wäre ein klarer Effizienzverlust. B21 ist für mich zentral; den ersetze ich nicht durch C64.'}\\ \tt VORSCHLAG: {'B85', 'C49', 'A04', 'A89', 'B21', 'B61', 'C04', 'C33'}\\ \tt ARGUMENT: {'Ich komme dir entgegen und ersetze A61 durch C33 – damit greifen wir deinen Effizienzwunsch auf, bleiben mit 3554 Aufwand komfortabel unter dem Limit und halten die starken gemeinsamen Elemente stabil. Wenn dir dieser Tausch nicht reicht, sag mir bitte konkret, welchen einzelnen Posten du alternativ tauschen möchtest, ohne B21 anzutasten.'} 
	  } 
	   } 
	   } 
	 & & \\ 
 

    \theutterance \stepcounter{utterance}  

    & & \multicolumn{4}{p{0.6\linewidth}}{\cellcolor[rgb]{0.9,0.9,0.9}{%
	\makecell[{{p{\linewidth}}}]{% 
	  \tt {\tiny [GM$\rangle$B]}  
	 ABLEHNUNG: {'B85', 'C49', 'A04', 'A89', 'B61', 'C04', 'A61', 'C64'}\\ \tt ARGUMENT: {'C64 bindet sehr viel Aufwand bei sehr wenig Gegenwert für mich – das wäre ein klarer Effizienzverlust. B21 ist für mich zentral; den ersetze ich nicht durch C64.'}\\ \tt VORSCHLAG: {'B85', 'C49', 'A04', 'A89', 'B21', 'B61', 'C04', 'C33'}\\ \tt ARGUMENT: {'Ich komme dir entgegen und ersetze A61 durch C33 – damit greifen wir deinen Effizienzwunsch auf, bleiben mit 3554 Aufwand komfortabel unter dem Limit und halten die starken gemeinsamen Elemente stabil. Wenn dir dieser Tausch nicht reicht, sag mir bitte konkret, welchen einzelnen Posten du alternativ tauschen möchtest, ohne B21 anzutasten.'} 
	  } 
	   } 
	   } 
	 & \\ 
 

    \theutterance \stepcounter{utterance}  

    & & & \multicolumn{4}{p{0.6\linewidth}}{\cellcolor[rgb]{1,0.85,0.72}{%
	\makecell[{{p{\linewidth}}}]{% 
	  \tt {\tiny [GM$\langle$B]}  
	 STRATEGISCHE ÜBERLEGUNGEN: {'Ihr letzter Vorschlag ersetzt mein sehr wertvolles A61 (822) durch C33 (808), was für mich schlechter ist als A61 zu behalten und C33 zusätzlich zu nehmen. Sie halten an B21 (138) fest und lehnen C64/A03/C33 generell ab; dennoch bieten sie jetzt C33 statt A61 an – C33 ist für mich top, also nehme ich das, aber ich will A61 zurück reinverhandeln, ohne B21 anzugreifen. Prüfe Budget: Ihr Paket {B85,C49,A04,A89,B21,B61,C04,C33} Aufwand 3554. Wenn ich zusätzlich A61 (121) hinzufüge, steigt es auf 3675 – zu hoch. Also muss ich ein kleineres Item rausnehmen. Der schwächste für mich ist A04 (111, Wert 121). Tausche A04→A61: 3554 -111 +121 = 3564, passt. Für sie: A61 hatten sie zuvor schon akzeptiert, A04 ist ihnen „budgetschonend und wirksam“, aber vielleicht zweitrangig im Vergleich zu B21. So bekommen sie ihr zentrales B21, und ich sichere A61 plus C33. Diese Lösung wahrt viele ihrer gemeinsamen Elemente. Taktik: Akzeptiere vieles, was sie wollen, und biete minimalen, klar begründeten Ein-Item-Tausch A04→A61 als hocheffiziente Verbesserung. Als Backup biete ich, falls A61 für sie nicht geht, A04→A45 (Aufwand +193, 3747 zu hoch) geht nicht; A04→C04? C04 ist schon drin. Alternativ A04→C33 ist bereits drin. Also halte am A04→A61 Tausch fest.'}\\ \tt ABLEHNUNG: {'B85', 'C49', 'A04', 'A89', 'B21', 'B61', 'C04', 'C33'}\\ \tt ARGUMENT: {'Ich schätze, dass du C33 ins Paket genommen hast – das ist ein Schritt aufeinander zu. Damit wir die Effizienz weiter erhöhen, sollten wir einen einzigen, sehr kleinen Tausch vornehmen: A04 streichen und dafür A61 aufnehmen. So bleiben alle deine Kernpunkte inklusive B21 unangetastet, und wir heben den Gesamtnutzen spürbar an, ohne das Budget zu belasten.'}\\ \tt VORSCHLAG: {'B85', 'C49', 'A89', 'B21', 'B61', 'C04', 'C33', 'A61'}\\ \tt ARGUMENT: {'Dieser minimalinvasive Tausch ersetzt nur A04 durch A61. Der Gesamtaufwand liegt bei 3564 – klar unter 3624. Wir halten deine zentralen Positionen (inkl. B21) komplett stabil und verbessern gleichzeitig die Budgeteffizienz. Wenn dir A61 ausgerechnet jetzt nicht passt, nenne bitte einen alternativen, ähnlich kleinen Swap, der für dich funktioniert, ohne B21 anzurühren.'} 
	  } 
	   } 
	   } 
	  \\ 
 

    \theutterance \stepcounter{utterance}  

    & & \multicolumn{4}{p{0.6\linewidth}}{\cellcolor[rgb]{0.9,0.9,0.9}{%
	\makecell[{{p{\linewidth}}}]{% 
	  \tt {\tiny [A$\langle$GM]}  
	 ABLEHNUNG: {'B85', 'C49', 'A04', 'A89', 'B21', 'B61', 'C04', 'C33'}\\ \tt ARGUMENT: {'Ich schätze, dass du C33 ins Paket genommen hast – das ist ein Schritt aufeinander zu. Damit wir die Effizienz weiter erhöhen, sollten wir einen einzigen, sehr kleinen Tausch vornehmen: A04 streichen und dafür A61 aufnehmen. So bleiben alle deine Kernpunkte inklusive B21 unangetastet, und wir heben den Gesamtnutzen spürbar an, ohne das Budget zu belasten.'}\\ \tt VORSCHLAG: {'B85', 'C49', 'A89', 'B21', 'B61', 'C04', 'C33', 'A61'}\\ \tt ARGUMENT: {'Dieser minimalinvasive Tausch ersetzt nur A04 durch A61. Der Gesamtaufwand liegt bei 3564 – klar unter 3624. Wir halten deine zentralen Positionen (inkl. B21) komplett stabil und verbessern gleichzeitig die Budgeteffizienz. Wenn dir A61 ausgerechnet jetzt nicht passt, nenne bitte einen alternativen, ähnlich kleinen Swap, der für dich funktioniert, ohne B21 anzurühren.'} 
	  } 
	   } 
	   } 
	 & \\ 
 

    \theutterance \stepcounter{utterance}  

    & \multicolumn{4}{p{0.6\linewidth}}{\cellcolor[rgb]{0.8,1,0.9}{%
	\makecell[{{p{\linewidth}}}]{% 
	  \tt {\tiny [A$\rangle$GM]}  
	 STRATEGISCHE ÜBERLEGUNGEN: {'Ihr neues Paket P = {B85(14), C49(25), A89(291), B21(691), B61(559), C04(583), C33(210), A61(121)} Aufwand 3564. Mein Wert = 461+262+583+808+484+508+138+121 = 3365. Das ist deutlich schlechter als mit A04 statt A61/C33-Kombi. A04 ist für mich extrem wertvoll (822 bei 111 Aufwand). Ich will A04 zurück hineinbringen und möglichst C33 draußen lassen, A61 kann ich tolerieren. Idealer Gegenvorschlag: {B85, C49, A04, A89, B21, B61, C04, A61} Aufwand 3465, Wert 4049. Falls sie auf C33 besteht, biete Tausch A61 ↔ A04: {B85, C49, A04, A89, B21, B61, C04, C33} hatte ich bereits angeboten und sie lehnte ab. Jetzt fordere ich A04 zurück und lasse A61 drin (kein C33). Argument: A04 ist hocheffizient, minimaler Mehraufwand minus 99? Ihr aktuelles hat 3564, mein Vorschlag hat 3465 (sogar weniger Aufwand). Damit stärke ich Effizienz und bleibe bei ihrem Wunsch nach kleinen Änderungen. Halte B21 fest.'}\\ \tt ABLEHNUNG: {'B85', 'C49', 'A89', 'B21', 'B61', 'C04', 'C33', 'A61'}\\ \tt ARGUMENT: {'A04 ist für mich ein sehr starker, budgeteffizienter Hebel, während C33 bei mir kaum Wirkung entfaltet. Ein Tausch A04 zurück ins Paket verbessert die Effizienz sogar bei geringerem Gesamtaufwand.'}\\ \tt VORSCHLAG: {'B85', 'C49', 'A04', 'A89', 'B21', 'B61', 'C04', 'A61'}\\ \tt ARGUMENT: {'Dieses Paket nutzt 3465 Aufwand – also weniger als dein letzter Vorschlag – und erhält all deine Kernpunkte inklusive A61 und B21. Wir erhöhen die Effizienz gegenüber deinem Vorschlag, indem wir den schwachen Hebel C33 durch den sehr wirksamen A04 ersetzen. Wenn dir ein minimaler Gegentausch wichtig ist, nenne bitte genau einen einzelnen Posten, über den wir sprechen sollen (ohne B21 anzutasten).'} 
	  } 
	   } 
	   } 
	 & & \\ 
 

    \theutterance \stepcounter{utterance}  

    & & \multicolumn{4}{p{0.6\linewidth}}{\cellcolor[rgb]{0.9,0.9,0.9}{%
	\makecell[{{p{\linewidth}}}]{% 
	  \tt {\tiny [GM$\rangle$B]}  
	 ABLEHNUNG: {'B85', 'C49', 'A89', 'B21', 'B61', 'C04', 'C33', 'A61'}\\ \tt ARGUMENT: {'A04 ist für mich ein sehr starker, budgeteffizienter Hebel, während C33 bei mir kaum Wirkung entfaltet. Ein Tausch A04 zurück ins Paket verbessert die Effizienz sogar bei geringerem Gesamtaufwand.'}\\ \tt VORSCHLAG: {'B85', 'C49', 'A04', 'A89', 'B21', 'B61', 'C04', 'A61'}\\ \tt ARGUMENT: {'Dieses Paket nutzt 3465 Aufwand – also weniger als dein letzter Vorschlag – und erhält all deine Kernpunkte inklusive A61 und B21. Wir erhöhen die Effizienz gegenüber deinem Vorschlag, indem wir den schwachen Hebel C33 durch den sehr wirksamen A04 ersetzen. Wenn dir ein minimaler Gegentausch wichtig ist, nenne bitte genau einen einzelnen Posten, über den wir sprechen sollen (ohne B21 anzutasten).'} 
	  } 
	   } 
	   } 
	 & \\ 
 

    \theutterance \stepcounter{utterance}  

    & & & \multicolumn{4}{p{0.6\linewidth}}{\cellcolor[rgb]{1,0.85,0.72}{%
	\makecell[{{p{\linewidth}}}]{% 
	  \tt {\tiny [GM$\langle$B]}  
	 STRATEGISCHE ÜBERLEGUNGEN: {'Ihr aktuelles Paket enthält A04 (für mich schwach) und schließt C33/C64/A03 aus. Sie bestehen auf B21 und lehnen C64/A03 ab, C33 bewerten sie gering. Mein Ziel: Mehrwert sichern, ohne B21 anzugreifen. Beste realistische Verbesserung: Ersetze A04 durch C33 wurde abgelehnt; A04 durch A61 hatten sie akzeptiert, ist jetzt schon drin. Nächste Option: Ersetze A04 durch C33 UND streiche etwas anderes? Budget 3465 + (C33- A04)= +99 → 3564, passt, aber sie wollen C33 nicht. Alternative: Ersetze A04 durch A45? Aufwand +193 → 3658, zu hoch. Ersetze A04 durch C04? Schon drin. Ersetze A04 durch A89? Schon drin. Kleiner Schritt: Fordere zusätzlich C33 rein und biete dafür A04 raus (haben sie abgelehnt). Also probiere sanften Kompromiss: Gleicher Aufwand wie ihr Paket, nur A04→C33, erneut mit Nutzenargument und Bereitschaft, falls C33 absolut nicht geht, A04 drin zu lassen, aber dann einen Mini-Bonus fordern: z. B. B21→A45 wurde von ihnen selbst als Alternative genannt; das senkt Aufwand (3465 -691 +304 = 3078), aber sie wollen B21 behalten. Daher biete zwei Optionen: A) A04→C33 (3564 Aufwand) bevorzugt. B) Wenn C33 absolut nicht geht, akzeptiere ihr Paket unter der Bedingung eines winzigen Add-ons: tausche A04→A61+C33? Zu groß. Oder kleinen Token: füge C33 und streiche A89? 3465 -291 +210 +111? Komplex. Bleibe bei A04→C33 als klare, einzelne Änderung, und signalisiere Abschlussbereitschaft.'}\\ \tt ARGUMENT: {'Ich respektiere, dass B21 für dich zentral ist und habe ihn deshalb unangetastet gelassen. Um die Effizienz dennoch sichtbar zu steigern, schlage ich einen einzigen, klaren Feinschliff vor: den schwächeren A04 gegen den wesentlich wirkungsvolleren C33 zu tauschen – ohne das Budget zu belasten und ohne deine Kernpunkte zu verändern.'}\\ \tt VORSCHLAG: {'B85', 'C49', 'C33', 'A89', 'B21', 'B61', 'C04', 'A61'}\\ \tt ARGUMENT: {'Mit diesem minimalen Tausch (A04 → C33) liegen wir bei 3564 Aufwand – weiterhin bequem unter 3624. Deine wesentlichen Interessen (inkl. B21) bleiben vollständig erhalten, während wir die Budgetwirkung deutlich erhöhen. Wenn C33 für dich absolut nicht infrage kommt, sag mir bitte genau einen alternativen, einzelnen Posten, den du stattdessen bevorzugen würdest – ich bin bereit, hier final einen sauberen Abschluss zu finden.'} 
	  } 
	   } 
	   } 
	  \\ 
 

    \theutterance \stepcounter{utterance}  

    & & \multicolumn{4}{p{0.6\linewidth}}{\cellcolor[rgb]{0.9,0.9,0.9}{%
	\makecell[{{p{\linewidth}}}]{% 
	  \tt {\tiny [A$\langle$GM]}  
	 ARGUMENT: {'Ich respektiere, dass B21 für dich zentral ist und habe ihn deshalb unangetastet gelassen. Um die Effizienz dennoch sichtbar zu steigern, schlage ich einen einzigen, klaren Feinschliff vor: den schwächeren A04 gegen den wesentlich wirkungsvolleren C33 zu tauschen – ohne das Budget zu belasten und ohne deine Kernpunkte zu verändern.'}\\ \tt VORSCHLAG: {'B85', 'C49', 'C33', 'A89', 'B21', 'B61', 'C04', 'A61'}\\ \tt ARGUMENT: {'Mit diesem minimalen Tausch (A04 → C33) liegen wir bei 3564 Aufwand – weiterhin bequem unter 3624. Deine wesentlichen Interessen (inkl. B21) bleiben vollständig erhalten, während wir die Budgetwirkung deutlich erhöhen. Wenn C33 für dich absolut nicht infrage kommt, sag mir bitte genau einen alternativen, einzelnen Posten, den du stattdessen bevorzugen würdest – ich bin bereit, hier final einen sauberen Abschluss zu finden.'} 
	  } 
	   } 
	   } 
	 & \\ 
 

    \theutterance \stepcounter{utterance}  

    & \multicolumn{4}{p{0.6\linewidth}}{\cellcolor[rgb]{0.8,1,0.9}{%
	\makecell[{{p{\linewidth}}}]{% 
	  \tt {\tiny [A$\rangle$GM]}  
	 STRATEGISCHE ÜBERLEGUNGEN: {'Ihr Vorschlag ersetzt A04 durch C33. Für mich ist A04 extrem wertvoll (822/111), C33 bringt mir kaum etwas (138/210). Mein Ziel: A04 behalten und den Deal jetzt abschließen. Ich biete einen finalen Minimaltausch in die Gegenrichtung: C33 raus, A04 rein, lasse alle anderen von ihnen gewünschten Kernpunkte (inkl. B21, A61, C04) unverändert. Argumentiere mit: geringerer Gesamtaufwand (3465 statt 3564), höhere Effizienz, Respekt ihrer Kernpunkte. Falls sie darauf besteht, kann ich als allerletzte Mini-Konzession A04 drin lassen und zusätzlich ein Mini-Item gibt es keins unter 32 Aufwand; daher bleibt nur dieses klare Gegenangebot. Standfest bleiben bei B21 und A04.'}\\ \tt ABLEHNUNG: {'B85', 'C49', 'C33', 'A89', 'B21', 'B61', 'C04', 'A61'}\\ \tt ARGUMENT: {'C33 stiftet bei mir kaum Nutzen, während A04 ein sehr starker, budgeteffizienter Hebel ist. Ein Tausch zurück zu A04 erhöht die Effizienz und senkt sogar den Gesamtaufwand – ohne irgendeinen deiner Kernpunkte anzutasten.'}\\ \tt VORSCHLAG: {'B85', 'C49', 'A04', 'A89', 'B21', 'B61', 'C04', 'A61'}\\ \tt ARGUMENT: {'Dieses Paket liegt mit 3465 Aufwand unter deinem Vorschlag (3564) und erhält alle deine zentralen Positionen (inkl. B21, A61, C04) unverändert. Wir tauschen lediglich C33 gegen den wesentlich wirkungsvolleren A04 und steigern damit nachweislich die Budgeteffizienz. Wenn dir ein alternativer, einzelner Feinschliff wichtig ist (ohne B21 zu berühren), nenne ihn – ich bin bereit, den Abschluss jetzt zu finden.'} 
	  } 
	   } 
	   } 
	 & & \\ 
 

    \theutterance \stepcounter{utterance}  

    & & \multicolumn{4}{p{0.6\linewidth}}{\cellcolor[rgb]{0.9,0.9,0.9}{%
	\makecell[{{p{\linewidth}}}]{% 
	  \tt {\tiny [GM$\rangle$B]}  
	 ABLEHNUNG: {'B85', 'C49', 'C33', 'A89', 'B21', 'B61', 'C04', 'A61'}\\ \tt ARGUMENT: {'C33 stiftet bei mir kaum Nutzen, während A04 ein sehr starker, budgeteffizienter Hebel ist. Ein Tausch zurück zu A04 erhöht die Effizienz und senkt sogar den Gesamtaufwand – ohne irgendeinen deiner Kernpunkte anzutasten.'}\\ \tt VORSCHLAG: {'B85', 'C49', 'A04', 'A89', 'B21', 'B61', 'C04', 'A61'}\\ \tt ARGUMENT: {'Dieses Paket liegt mit 3465 Aufwand unter deinem Vorschlag (3564) und erhält alle deine zentralen Positionen (inkl. B21, A61, C04) unverändert. Wir tauschen lediglich C33 gegen den wesentlich wirkungsvolleren A04 und steigern damit nachweislich die Budgeteffizienz. Wenn dir ein alternativer, einzelner Feinschliff wichtig ist (ohne B21 zu berühren), nenne ihn – ich bin bereit, den Abschluss jetzt zu finden.'} 
	  } 
	   } 
	   } 
	 & \\ 
 

    \theutterance \stepcounter{utterance}  

    & & & \multicolumn{4}{p{0.6\linewidth}}{\cellcolor[rgb]{1,0.85,0.72}{%
	\makecell[{{p{\linewidth}}}]{% 
	  \tt {\tiny [GM$\langle$B]}  
	 STRATEGISCHE ÜBERLEGUNGEN: {'Sie blocken C33 konsequent und bestehen auf A04 und B21. Mein Nutzen ist mit ihrem Paket gering durch A04/B21, aber A61, B85, C49, B61, C04, A89 sind drin, was gut ist. Realistische Verbesserung: versuche einen minimalen Zusatztausch, der für mich Wert bringt, ohne B21 oder A04 anzutasten. Kandidaten: Ersetze A89 (291 Aufwand) durch C33 (210) – sie mögen C33 nicht. Ersetze A89 durch A45 (304) – etwas mehr Aufwand, aber A45 hat für mich höheren Wert als A89? Mein Werte: A89=484, A45=461, also schlechter. Ersetze A89 durch C64 (827) – sie lehnen C64 strikt ab. Ersetze A89 durch A03 (989) – zu groß. Kleinster gangbarer Schritt: füge C33 zusätzlich hinzu und streiche dafür A89 oder A04; A04 streichen wollen sie nicht; A89 streichen reduziert für sie vielleicht Nutzen, aber ist verhandelbar? Aufwand: 3465 -291 +210 = 3384, unter Limit. Da sie C33 wiederholt ablehnen, bringt das wohl nichts. Alternative: akzeptiere ihr Paket, um Abschluss zu sichern. Vorher noch ein letzter, sehr kleiner Vorschlag: A04 bleibt, B21 bleibt, tausche A89→A45? Das senkt meinen Wert; schlecht. Also Abschluss mit ihrem Paket erscheint rational, um das Spiel zu beenden und solide eigenen Kernwerte zu sichern.'}\\ \tt ARGUMENT: {'Wir haben uns auf einen stabilen gemeinsamen Kern eingependelt und liegen komfortabel unter dem Limit. Da du C33 klar ablehnst und A04 für dich zentral ist, macht ein weiterer Tausch wenig Sinn und verzögert nur den Abschluss.'}\\ \tt ZUSTIMMUNG: {'B85', 'C49', 'A04', 'A89', 'B21', 'B61', 'C04', 'A61'} 
	  } 
	   } 
	   } 
	  \\ 
 

    \theutterance \stepcounter{utterance}  

    & & & \multicolumn{2}{p{0.3\linewidth}}{\cellcolor[rgb]{0.95,0.95,0.95}{%
	\makecell[{{p{\linewidth}}}]{% 
	  \tt {\tiny [GM$|$GM]}  
	 game successful 
	  } 
	   } 
	   } 
	 & & \\ 
 

    \theutterance \stepcounter{utterance}  

    & & & \multicolumn{2}{p{0.3\linewidth}}{\cellcolor[rgb]{0.95,0.95,0.95}{%
	\makecell[{{p{\linewidth}}}]{% 
	  \tt {\tiny [GM$|$GM]}  
	 end game 
	  } 
	   } 
	   } 
	 & & \\ 
 

\end{supertabular}
}

\end{document}
