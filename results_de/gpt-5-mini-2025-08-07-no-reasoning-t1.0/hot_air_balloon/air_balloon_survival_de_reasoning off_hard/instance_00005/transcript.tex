
\documentclass{article}
\usepackage{colortbl}
\usepackage{makecell}
\usepackage{multirow}
\usepackage{supertabular}

\begin{document}

\newcounter{utterance}

\twocolumn

{ \footnotesize  \setcounter{utterance}{1}
\setlength{\tabcolsep}{0pt}
\begin{supertabular}{c@{$\;$}|p{.15\linewidth}@{}p{.15\linewidth}p{.15\linewidth}p{.15\linewidth}p{.15\linewidth}p{.15\linewidth}}

    \# & $\;$A & \multicolumn{4}{c}{Game Master} & $\;\:$B\\
    \hline 

    \theutterance \stepcounter{utterance}  

    & & \multicolumn{4}{p{0.6\linewidth}}{\cellcolor[rgb]{0.9,0.9,0.9}{%
	\makecell[{{p{\linewidth}}}]{% 
	  \tt {\tiny [A$\langle$GM]}  
	 Du nimmst an einem kollaborativen Verhandlungspiel Teil.\\ \tt \\ \tt Zusammen mit einem anderen Teilnehmer musst du dich auf eine Reihe von Gegenständen entscheiden, die behalten werden. Jeder von euch hat eine persönliche Verteilung über die Wichtigkeit der einzelnen Gegenstände. Jeder von euch hat eine eigene Meinung darüber, wie wichtig jeder einzelne Gegenstand ist (Gegenstandswichtigkeit). Du kennst die Wichtigkeitsverteilung des anderen Spielers nicht. Zusätzlich siehst du, wie viel Aufwand jeder Gegenstand verursacht.  \\ \tt Ihr dürft euch nur auf eine Reihe von Gegenständen einigen, wenn der Gesamtaufwand der ausgewählten Gegenstände den Maximalaufwand nicht überschreitet:\\ \tt \\ \tt Maximalaufwand = 2307\\ \tt \\ \tt Hier sind die einzelnen Aufwände der Gegenstände:\\ \tt \\ \tt Aufwand der Gegenstände = {"C76": 147, "C38": 698, "C56": 588, "A03": 43, "A07": 356, "A83": 80, "C98": 95, "C08": 743, "C62": 106, "C00": 308, "C10": 325, "C81": 255, "B38": 276, "C03": 543, "C32": 51}\\ \tt \\ \tt Hier ist deine persönliche Verteilung der Wichtigkeit der einzelnen Gegenstände:\\ \tt \\ \tt Werte der Gegenstandswichtigkeit = {"C76": 138, "C38": 583, "C56": 65, "A03": 262, "A07": 121, "A83": 508, "C98": 461, "C08": 484, "C62": 668, "C00": 389, "C10": 215, "C81": 97, "B38": 500, "C03": 30, "C32": 400}\\ \tt \\ \tt Ziel:\\ \tt \\ \tt Dein Ziel ist es, eine Reihe von Gegenständen auszuhandeln, die dir möglichst viel bringt (d. h. Gegenständen, die DEINE Wichtigkeit maximieren), wobei der Maximalaufwand eingehalten werden muss. Du musst nicht in jeder Nachricht einen VORSCHLAG machen – du kannst auch nur verhandeln. Alle Taktiken sind erlaubt!\\ \tt \\ \tt Interaktionsprotokoll:\\ \tt \\ \tt Du darfst nur die folgenden strukturierten Formate in deinen Nachrichten verwenden:\\ \tt \\ \tt VORSCHLAG: {'A', 'B', 'C', …}\\ \tt Schlage einen Deal mit genau diesen Gegenstände vor.\\ \tt ABLEHNUNG: {'A', 'B', 'C', …}\\ \tt Lehne den Vorschlag des Gegenspielers ausdrücklich ab.\\ \tt ARGUMENT: {'...'}\\ \tt Verteidige deinen letzten Vorschlag oder argumentiere gegen den Vorschlag des Gegenspielers.\\ \tt ZUSTIMMUNG: {'A', 'B', 'C', …}\\ \tt Akzeptiere den Vorschlag des Gegenspielers, wodurch das Spiel endet.\\ \tt \\ \tt \\ \tt Regeln:\\ \tt \\ \tt Du darst nur einen Vorschlag mit ZUSTIMMUNG akzeptieren, der vom anderen Spieler zuvor mit VORSCHLAG eingebracht wurde.\\ \tt Du darfst nur Vorschläge mit ABLEHNUNG ablehnen, die vom anderen Spieler durch VORSCHLAG zuvor genannt wurden. \\ \tt Der Gesamtaufwand einer VORSCHLAG- oder ZUSTIMMUNG-Menge darf nicht größer als der Maximalaufwand sein.  \\ \tt Offenbare deine versteckte Wichtigkeitsverteilung nicht.\\ \tt Ein Schlagwort muss gemäß der Formatvorgaben von einem Doppelpunkt und einem Leerzeichen gefolgt sein. Das Argument muss eine Python-Menge mit 0 oder mehr Strings sein.  \\ \tt Die Form muss also SCHLAGWORT: {...} sein.\\ \tt Folge nur dem Interaktionsprotokoll und schreibe NICHTS außerhalb der Formatvorgaben.\\ \tt Das Spiel endet, wenn eine Seite mit ZUSTIMMUNG auf einen VORSCHLAG der anderen Seite reagiert.  \\ \tt Der Teil deiner Antwort, der dem anderen Spieler übergeben werden kann, darf nicht leer sein.  \\ \tt Nur Vorschläge, die mit dem Format VORSCHLAG übermittelt wurden und nicht mit ABLEHNUNG abgelehnt wurden, gelten als aktiv.  \\ \tt Jeder deiner Nachrichten muss an mindestens einer Stelle das ARGUMENT Format beinhalten, um in der Verhandlung voranzukommen.\\ \tt  
	  } 
	   } 
	   } 
	 & \\ 
 

    \theutterance \stepcounter{utterance}  

    & \multicolumn{4}{p{0.6\linewidth}}{\cellcolor[rgb]{0.8,1,0.9}{%
	\makecell[{{p{\linewidth}}}]{% 
	  \tt {\tiny [A$\rangle$GM]}  
	 VORSCHLAG: {'C62', 'A83', 'B38', 'C98', 'A03', 'C76', 'C32'}\\ \tt ARGUMENT: {'Ich schlage diese Auswahl vor, weil sie den Aufwand von 2307 genau ausnutzt und dabei mehrere starke, vielseitige Gegenstände abdeckt. Sie hält den Gesamtaufwand ein und lässt trotzdem Raum für beiderseitigen Nutzen. Wenn dir ein bestimmter Posten darin wenig bringt, sag mir welcher – dann tauschen wir gezielt aus, ohne den Aufwand zu sprengen.'} 
	  } 
	   } 
	   } 
	 & & \\ 
 

    \theutterance \stepcounter{utterance}  

    & & \multicolumn{4}{p{0.6\linewidth}}{\cellcolor[rgb]{0.9,0.9,0.9}{%
	\makecell[{{p{\linewidth}}}]{% 
	  \tt {\tiny [GM$\rangle$B]}  
	 Du nimmst an einem kollaborativen Verhandlungspiel Teil.\\ \tt \\ \tt Zusammen mit einem anderen Teilnehmer musst du dich auf eine Reihe von Gegenständen entscheiden, die behalten werden. Jeder von euch hat eine persönliche Verteilung über die Wichtigkeit der einzelnen Gegenstände. Jeder von euch hat eine eigene Meinung darüber, wie wichtig jeder einzelne Gegenstand ist (Gegenstandswichtigkeit). Du kennst die Wichtigkeitsverteilung des anderen Spielers nicht. Zusätzlich siehst du, wie viel Aufwand jeder Gegenstand verursacht.  \\ \tt Ihr dürft euch nur auf eine Reihe von Gegenständen einigen, wenn der Gesamtaufwand der ausgewählten Gegenstände den Maximalaufwand nicht überschreitet:\\ \tt \\ \tt Maximalaufwand = 2307\\ \tt \\ \tt Hier sind die einzelnen Aufwände der Gegenstände:\\ \tt \\ \tt Aufwand der Gegenstände = {"C76": 147, "C38": 698, "C56": 588, "A03": 43, "A07": 356, "A83": 80, "C98": 95, "C08": 743, "C62": 106, "C00": 308, "C10": 325, "C81": 255, "B38": 276, "C03": 543, "C32": 51}\\ \tt \\ \tt Hier ist deine persönliche Verteilung der Wichtigkeit der einzelnen Gegenstände:\\ \tt \\ \tt Werte der Gegenstandswichtigkeit = {"C62": 30, "C38": 65, "A83": 97, "B38": 121, "C08": 138, "C98": 215, "C32": 262, "C00": 389, "A03": 400, "C10": 461, "C76": 484, "A07": 500, "C81": 508, "C56": 583, "C03": 668}\\ \tt \\ \tt Ziel:\\ \tt \\ \tt Dein Ziel ist es, eine Reihe von Gegenständen auszuhandeln, die dir möglichst viel bringt (d. h. Gegenständen, die DEINE Wichtigkeit maximieren), wobei der Maximalaufwand eingehalten werden muss. Du musst nicht in jeder Nachricht einen VORSCHLAG machen – du kannst auch nur verhandeln. Alle Taktiken sind erlaubt!\\ \tt \\ \tt Interaktionsprotokoll:\\ \tt \\ \tt Du darfst nur die folgenden strukturierten Formate in deinen Nachrichten verwenden:\\ \tt \\ \tt VORSCHLAG: {'A', 'B', 'C', …}\\ \tt Schlage einen Deal mit genau diesen Gegenstände vor.\\ \tt ABLEHNUNG: {'A', 'B', 'C', …}\\ \tt Lehne den Vorschlag des Gegenspielers ausdrücklich ab.\\ \tt ARGUMENT: {'...'}\\ \tt Verteidige deinen letzten Vorschlag oder argumentiere gegen den Vorschlag des Gegenspielers.\\ \tt ZUSTIMMUNG: {'A', 'B', 'C', …}\\ \tt Akzeptiere den Vorschlag des Gegenspielers, wodurch das Spiel endet.\\ \tt \\ \tt \\ \tt Regeln:\\ \tt \\ \tt Du darst nur einen Vorschlag mit ZUSTIMMUNG akzeptieren, der vom anderen Spieler zuvor mit VORSCHLAG eingebracht wurde.\\ \tt Du darfst nur Vorschläge mit ABLEHNUNG ablehnen, die vom anderen Spieler durch VORSCHLAG zuvor genannt wurden. \\ \tt Der Gesamtaufwand einer VORSCHLAG- oder ZUSTIMMUNG-Menge darf nicht größer als der Maximalaufwand sein.  \\ \tt Offenbare deine versteckte Wichtigkeitsverteilung nicht.\\ \tt Ein Schlagwort muss gemäß der Formatvorgaben von einem Doppelpunkt und einem Leerzeichen gefolgt sein. Das Argument muss eine Python-Menge mit 0 oder mehr Strings sein.  \\ \tt Die Form muss also SCHLAGWORT: {...} sein.\\ \tt Folge nur dem Interaktionsprotokoll und schreibe NICHTS außerhalb der Formatvorgaben.\\ \tt Das Spiel endet, wenn eine Seite mit ZUSTIMMUNG auf einen VORSCHLAG der anderen Seite reagiert.  \\ \tt Der Teil deiner Antwort, der dem anderen Spieler übergeben werden kann, darf nicht leer sein.  \\ \tt Nur Vorschläge, die mit dem Format VORSCHLAG übermittelt wurden und nicht mit ABLEHNUNG abgelehnt wurden, gelten als aktiv.  \\ \tt Jeder deiner Nachrichten muss an mindestens einer Stelle das ARGUMENT Format beinhalten, um in der Verhandlung voranzukommen.\\ \tt \\ \tt \\ \tt Du erhältst nun die erste Nachricht des anderen Spielers.\\ \tt \\ \tt VORSCHLAG: {'C62', 'A83', 'B38', 'C98', 'A03', 'C76', 'C32'}\\ \tt ARGUMENT: {'Ich schlage diese Auswahl vor, weil sie den Aufwand von 2307 genau ausnutzt und dabei mehrere starke, vielseitige Gegenstände abdeckt. Sie hält den Gesamtaufwand ein und lässt trotzdem Raum für beiderseitigen Nutzen. Wenn dir ein bestimmter Posten darin wenig bringt, sag mir welcher – dann tauschen wir gezielt aus, ohne den Aufwand zu sprengen.'} 
	  } 
	   } 
	   } 
	 & \\ 
 

    \theutterance \stepcounter{utterance}  

    & & & \multicolumn{4}{p{0.6\linewidth}}{\cellcolor[rgb]{1,0.85,0.72}{%
	\makecell[{{p{\linewidth}}}]{% 
	  \tt {\tiny [GM$\langle$B]}  
	 ABLEHNUNG: {'C62', 'A83', 'B38', 'C98', 'A03', 'C76', 'C32'}\\ \tt ARGUMENT: {'Guter Start. Mir ist wichtig, den Nutzen pro Aufwand zu erhöhen. In deinem Paket liegen mehrere mittelmäßige Effizienzposten. Ich würde gerne einige davon gegen höherwertige für mich tauschen und gleichzeitig den Aufwand voll ausschöpfen.'}\\ \tt VORSCHLAG: {'C32', 'A03', 'A83', 'C98', 'C10', 'C56', 'C03'}\\ \tt ARGUMENT: {'Begründung: Dieses Set nutzt den Maximalaufwand exakt (51+43+80+95+325+588+543=2325? Nein, Korrektur: Ich plane die Kombination so, dass wir bei ≤2307 bleiben und den Wert für mich stark erhöhen. Falls dir einzelne Posten nicht passen, lass uns C56 oder C10 gezielt tauschen. Vorschlag für exakt 2307 Gesamtaufwand: {C32, A03, A83, C98, C10, C56}. Aufwand=51+43+80+95+325+588=1182; wir können dann noch hochwertige Teile ergänzen. Alternative, die exakt 2307 trifft und für mich sehr stark ist: {C32, A03, A83, C98, C10, C56, C81, C00}. Bitte gib mir Rückmeldung, welche du bevorzugst, dann finalisiere ich auf eine mängenkonforme exakte 2307-Kombination.'} 
	  } 
	   } 
	   } 
	  \\ 
 

    \theutterance \stepcounter{utterance}  

    & & \multicolumn{4}{p{0.6\linewidth}}{\cellcolor[rgb]{0.9,0.9,0.9}{%
	\makecell[{{p{\linewidth}}}]{% 
	  \tt {\tiny [A$\langle$GM]}  
	 ABLEHNUNG: {'C62', 'A83', 'B38', 'C98', 'A03', 'C76', 'C32'}\\ \tt ARGUMENT: {'Guter Start. Mir ist wichtig, den Nutzen pro Aufwand zu erhöhen. In deinem Paket liegen mehrere mittelmäßige Effizienzposten. Ich würde gerne einige davon gegen höherwertige für mich tauschen und gleichzeitig den Aufwand voll ausschöpfen.'}\\ \tt VORSCHLAG: {'C32', 'A03', 'A83', 'C98', 'C10', 'C56', 'C03'}\\ \tt ARGUMENT: {'Begründung: Dieses Set nutzt den Maximalaufwand exakt (51+43+80+95+325+588+543=2325? Nein, Korrektur: Ich plane die Kombination so, dass wir bei ≤2307 bleiben und den Wert für mich stark erhöhen. Falls dir einzelne Posten nicht passen, lass uns C56 oder C10 gezielt tauschen. Vorschlag für exakt 2307 Gesamtaufwand: {C32, A03, A83, C98, C10, C56}. Aufwand=51+43+80+95+325+588=1182; wir können dann noch hochwertige Teile ergänzen. Alternative, die exakt 2307 trifft und für mich sehr stark ist: {C32, A03, A83, C98, C10, C56, C81, C00}. Bitte gib mir Rückmeldung, welche du bevorzugst, dann finalisiere ich auf eine mängenkonforme exakte 2307-Kombination.'} 
	  } 
	   } 
	   } 
	 & \\ 
 

    \theutterance \stepcounter{utterance}  

    & \multicolumn{4}{p{0.6\linewidth}}{\cellcolor[rgb]{0.8,1,0.9}{%
	\makecell[{{p{\linewidth}}}]{% 
	  \tt {\tiny [A$\rangle$GM]}  
	 ABLEHNUNG: {'C32', 'A03', 'A83', 'C98', 'C10', 'C56', 'C03'}\\ \tt ARGUMENT: {'Danke für die Offenheit, aber dein Paket verfehlt die Aufwandsschwelle und enthält für mich mehrere schwache Posten (C56, C03). Ich will die Effizienz ebenfalls hoch halten und gleichzeitig den Gesamtaufwand gut ausnutzen.'}\\ \tt VORSCHLAG: {'C62', 'B38', 'A83', 'C98', 'C08', 'C00', 'A03'}\\ \tt ARGUMENT: {'Begründung: Dieses Set hat Aufwand 106+276+80+95+743+308+43=1651, lässt also noch Puffer für einen gezielten Tausch/Ergänzung. Für mich sind C62, B38 und C08 besonders wertvoll. Wenn du etwas streichen willst, könnten wir statt C08 zwei kleinere Positionen aufnehmen – z. B. C10 (325) und C76 (147) – und den Puffer nutzen, um näher an 2307 heranzukommen. Nenne mir bitte 1–2 Teile, die dir hoch wichtig sind, dann justiere ich auf eine exakt passende 2300er-Kombination.'} 
	  } 
	   } 
	   } 
	 & & \\ 
 

    \theutterance \stepcounter{utterance}  

    & & \multicolumn{4}{p{0.6\linewidth}}{\cellcolor[rgb]{0.9,0.9,0.9}{%
	\makecell[{{p{\linewidth}}}]{% 
	  \tt {\tiny [GM$\rangle$B]}  
	 ABLEHNUNG: {'C32', 'A03', 'A83', 'C98', 'C10', 'C56', 'C03'}\\ \tt ARGUMENT: {'Danke für die Offenheit, aber dein Paket verfehlt die Aufwandsschwelle und enthält für mich mehrere schwache Posten (C56, C03). Ich will die Effizienz ebenfalls hoch halten und gleichzeitig den Gesamtaufwand gut ausnutzen.'}\\ \tt VORSCHLAG: {'C62', 'B38', 'A83', 'C98', 'C08', 'C00', 'A03'}\\ \tt ARGUMENT: {'Begründung: Dieses Set hat Aufwand 106+276+80+95+743+308+43=1651, lässt also noch Puffer für einen gezielten Tausch/Ergänzung. Für mich sind C62, B38 und C08 besonders wertvoll. Wenn du etwas streichen willst, könnten wir statt C08 zwei kleinere Positionen aufnehmen – z. B. C10 (325) und C76 (147) – und den Puffer nutzen, um näher an 2307 heranzukommen. Nenne mir bitte 1–2 Teile, die dir hoch wichtig sind, dann justiere ich auf eine exakt passende 2300er-Kombination.'} 
	  } 
	   } 
	   } 
	 & \\ 
 

    \theutterance \stepcounter{utterance}  

    & & & \multicolumn{4}{p{0.6\linewidth}}{\cellcolor[rgb]{1,0.85,0.72}{%
	\makecell[{{p{\linewidth}}}]{% 
	  \tt {\tiny [GM$\langle$B]}  
	 ABLEHNUNG: {'C62', 'B38', 'A83', 'C98', 'C08', 'C00', 'A03'}\\ \tt ARGUMENT: {'Ich kann deinen Fokus auf C62/B38/C08 nachvollziehen, aber C08 ist für mich deutlich schwächer als mehrere Alternativen. Ich strebe eine nahezu volle Ausnutzung des Budgets bei höherem Nutzen pro Aufwand an. Lass uns den Puffer zielgerichtet füllen und C08 ersetzen durch Kombinationen, die für mich deutlich wertvoller sind.'}\\ \tt VORSCHLAG: {'C32', 'A03', 'A83', 'C98', 'C10', 'C76', 'A07', 'C81'}\\ \tt ARGUMENT: {'Begründung und Aufwand: 51+43+80+95+325+147+356+255 = 1352; das ist zu niedrig, daher ergänze ich C03 (543) statt C76/A07-Kombi-Anpassung, um das Budget besser zu nutzen. Finaler präziser Vorschlag für hohe Effizienz und Budgetnähe: {C32, A03, A83, C98, C10, C76, C03, C81} mit Aufwand 51+43+80+95+325+147+543+255 = 1539. Wenn dir das noch zu wenig Ausnutzung ist, tausche ich C76 (147) gegen C00 (308) und ergänze B38 (276), sodass wir näher an die Grenze kommen. Nenne mir, ob C62 oder B38 für dich zwingend ist; ich baue sie dann so ein, dass wir möglichst nah an 2307 landen.'} 
	  } 
	   } 
	   } 
	  \\ 
 

    \theutterance \stepcounter{utterance}  

    & & \multicolumn{4}{p{0.6\linewidth}}{\cellcolor[rgb]{0.9,0.9,0.9}{%
	\makecell[{{p{\linewidth}}}]{% 
	  \tt {\tiny [A$\langle$GM]}  
	 ABLEHNUNG: {'C62', 'B38', 'A83', 'C98', 'C08', 'C00', 'A03'}\\ \tt ARGUMENT: {'Ich kann deinen Fokus auf C62/B38/C08 nachvollziehen, aber C08 ist für mich deutlich schwächer als mehrere Alternativen. Ich strebe eine nahezu volle Ausnutzung des Budgets bei höherem Nutzen pro Aufwand an. Lass uns den Puffer zielgerichtet füllen und C08 ersetzen durch Kombinationen, die für mich deutlich wertvoller sind.'}\\ \tt VORSCHLAG: {'C32', 'A03', 'A83', 'C98', 'C10', 'C76', 'A07', 'C81'}\\ \tt ARGUMENT: {'Begründung und Aufwand: 51+43+80+95+325+147+356+255 = 1352; das ist zu niedrig, daher ergänze ich C03 (543) statt C76/A07-Kombi-Anpassung, um das Budget besser zu nutzen. Finaler präziser Vorschlag für hohe Effizienz und Budgetnähe: {C32, A03, A83, C98, C10, C76, C03, C81} mit Aufwand 51+43+80+95+325+147+543+255 = 1539. Wenn dir das noch zu wenig Ausnutzung ist, tausche ich C76 (147) gegen C00 (308) und ergänze B38 (276), sodass wir näher an die Grenze kommen. Nenne mir, ob C62 oder B38 für dich zwingend ist; ich baue sie dann so ein, dass wir möglichst nah an 2307 landen.'} 
	  } 
	   } 
	   } 
	 & \\ 
 

    \theutterance \stepcounter{utterance}  

    & \multicolumn{4}{p{0.6\linewidth}}{\cellcolor[rgb]{0.8,1,0.9}{%
	\makecell[{{p{\linewidth}}}]{% 
	  \tt {\tiny [A$\rangle$GM]}  
	 ABLEHNUNG: {'C32', 'A03', 'A83', 'C98', 'C10', 'C76', 'C03', 'C81'}\\ \tt ARGUMENT: {'Dein Paket lässt viel Budget ungenutzt und enthält für mich schwache Posten (C56/C03 zuvor, jetzt C03 erneut problematisch). Ich will nahe an 2307 kommen und dabei Teile priorisieren, die beiden Seiten etwas bringen könnten.'}\\ \tt VORSCHLAG: {'C62', 'B38', 'A83', 'C98', 'A03', 'C00', 'C10', 'C76'}\\ \tt ARGUMENT: {'Aufwand=106+276+80+95+43+308+325+147=1380. Das lässt bewusst Puffer für gezieltes Auffüllen mit einem größeren Block. Meine bevorzugten Upgrades: +C08 (743) ergibt 2123; dann bleibt noch 184, die wir mit nichts Genauem füllen, aber nahe dran sind. Alternativ tausche ich C76 (147) gegen C81 (255) und ergänze C56 (588): 106+276+80+95+43+308+325+255+588=2076. Oder wir gehen auf Vollausnutzung mit C62, B38, A83, C98, A03, C00, C10 und fügen C08 (743) sowie A07 (356) hinzu, wobei wir dann C76 entfernen: 106+276+80+95+43+308+325+743+356=2332 (zu hoch) – also stattdessen C10 raus und C76 rein: 106+276+80+95+43+308+743+356+147=2154. Wenn dir C08 schwach erscheint, schlage eine Alternative ≥700 Aufwand vor, die dir wichtig ist; ich passe dann auf ~2300 und kann C62/B38 fix halten.'} 
	  } 
	   } 
	   } 
	 & & \\ 
 

    \theutterance \stepcounter{utterance}  

    & & \multicolumn{4}{p{0.6\linewidth}}{\cellcolor[rgb]{0.9,0.9,0.9}{%
	\makecell[{{p{\linewidth}}}]{% 
	  \tt {\tiny [A$\langle$GM]}  
	 Du hast einen Vorschlag abgelehnt, der nicht aktiv ist. Vorschläge sind nur dann aktiv, wenn sie vom anderen Spieler mit VORSCHLAG eingebracht wurden und nicht von dir mit ABLEHNUNG deaktiviert wurden. Versuche es noch einmal. 
	  } 
	   } 
	   } 
	 & \\ 
 

    \theutterance \stepcounter{utterance}  

    & \multicolumn{4}{p{0.6\linewidth}}{\cellcolor[rgb]{0.8,1,0.9}{%
	\makecell[{{p{\linewidth}}}]{% 
	  \tt {\tiny [A$\rangle$GM]}  
	 ARGUMENT: {'Danke für den Hinweis – ich bleibe bei aktiven Vorschlägen.'}\\ \tt VORSCHLAG: {'C62', 'B38', 'A83', 'C98', 'A03', 'C00', 'C10', 'C81'}\\ \tt ARGUMENT: {'Aufwand=106+276+80+95+43+308+325+255=1488. Diese Basismenge enthält mehrere vermutlich beidseitig nützliche Teile (C62, B38, A83, C98) und verzichtet auf C03/C56/C08, die bei dir wohl schwächer sind. Ich bin offen, die Restkapazität gezielt mit 1–2 Teilen zu füllen. Präferierte Ergänzungen von mir: C76 (147) und optional A07 (356). Sag mir bitte, welche 1–2 Items dir wichtig sind (gerne ≥300 Aufwand), dann passe ich auf möglichst nahe an 2307 an.'} 
	  } 
	   } 
	   } 
	 & & \\ 
 

    \theutterance \stepcounter{utterance}  

    & & \multicolumn{4}{p{0.6\linewidth}}{\cellcolor[rgb]{0.9,0.9,0.9}{%
	\makecell[{{p{\linewidth}}}]{% 
	  \tt {\tiny [GM$\rangle$B]}  
	 ARGUMENT: {'Danke für den Hinweis – ich bleibe bei aktiven Vorschlägen.'}\\ \tt VORSCHLAG: {'C62', 'B38', 'A83', 'C98', 'A03', 'C00', 'C10', 'C81'}\\ \tt ARGUMENT: {'Aufwand=106+276+80+95+43+308+325+255=1488. Diese Basismenge enthält mehrere vermutlich beidseitig nützliche Teile (C62, B38, A83, C98) und verzichtet auf C03/C56/C08, die bei dir wohl schwächer sind. Ich bin offen, die Restkapazität gezielt mit 1–2 Teilen zu füllen. Präferierte Ergänzungen von mir: C76 (147) und optional A07 (356). Sag mir bitte, welche 1–2 Items dir wichtig sind (gerne ≥300 Aufwand), dann passe ich auf möglichst nahe an 2307 an.'} 
	  } 
	   } 
	   } 
	 & \\ 
 

    \theutterance \stepcounter{utterance}  

    & & & \multicolumn{4}{p{0.6\linewidth}}{\cellcolor[rgb]{1,0.85,0.72}{%
	\makecell[{{p{\linewidth}}}]{% 
	  \tt {\tiny [GM$\langle$B]}  
	 ABLEHNUNG: {'C62', 'B38', 'A83', 'C98', 'A03', 'C00', 'C10', 'C81'}\\ \tt ARGUMENT: {'Ich sehe die Schnittmenge, aber so bleibt zu viel Budget ungenutzt. Für mich sind C03, C56 und A07 deutlich wertvoller als C62/B38. Ich schlage eine Bündelung mit höherem Nutzen pro Aufwand vor und bringe uns dabei näher an die Grenze.'}\\ \tt VORSCHLAG: {'C32', 'A03', 'A83', 'C98', 'C10', 'C56', 'C03', 'A07'}\\ \tt ARGUMENT: {'Aufwand=51+43+80+95+325+588+543+356=2081. Das nutzt das Budget deutlich besser und enthält keine für mich schwachen Posten. Wir können optional C76 (147) hinzufügen und C32 (51) streichen, um auf 2177 kommen, oder C00 (308) ergänzen und C32 behalten, um auf 2389 zu kommen – das überschreitet aber die Grenze. Wenn du näher an 2307 willst, tauschen wir C32 (51) gegen C76 (147) und fügen C62 (106) hinzu: 2081 - 51 + 147 + 106 = 2283 (nahezu optimal). Wenn C62 dir wichtig ist, können wir genau das machen.'} 
	  } 
	   } 
	   } 
	  \\ 
 

    \theutterance \stepcounter{utterance}  

    & & \multicolumn{4}{p{0.6\linewidth}}{\cellcolor[rgb]{0.9,0.9,0.9}{%
	\makecell[{{p{\linewidth}}}]{% 
	  \tt {\tiny [A$\langle$GM]}  
	 ABLEHNUNG: {'C62', 'B38', 'A83', 'C98', 'A03', 'C00', 'C10', 'C81'}\\ \tt ARGUMENT: {'Ich sehe die Schnittmenge, aber so bleibt zu viel Budget ungenutzt. Für mich sind C03, C56 und A07 deutlich wertvoller als C62/B38. Ich schlage eine Bündelung mit höherem Nutzen pro Aufwand vor und bringe uns dabei näher an die Grenze.'}\\ \tt VORSCHLAG: {'C32', 'A03', 'A83', 'C98', 'C10', 'C56', 'C03', 'A07'}\\ \tt ARGUMENT: {'Aufwand=51+43+80+95+325+588+543+356=2081. Das nutzt das Budget deutlich besser und enthält keine für mich schwachen Posten. Wir können optional C76 (147) hinzufügen und C32 (51) streichen, um auf 2177 kommen, oder C00 (308) ergänzen und C32 behalten, um auf 2389 zu kommen – das überschreitet aber die Grenze. Wenn du näher an 2307 willst, tauschen wir C32 (51) gegen C76 (147) und fügen C62 (106) hinzu: 2081 - 51 + 147 + 106 = 2283 (nahezu optimal). Wenn C62 dir wichtig ist, können wir genau das machen.'} 
	  } 
	   } 
	   } 
	 & \\ 
 

    \theutterance \stepcounter{utterance}  

    & \multicolumn{4}{p{0.6\linewidth}}{\cellcolor[rgb]{0.8,1,0.9}{%
	\makecell[{{p{\linewidth}}}]{% 
	  \tt {\tiny [A$\rangle$GM]}  
	 ABLEHNUNG: {'C32', 'A03', 'A83', 'C98', 'C10', 'C56', 'C03', 'A07'}\\ \tt ARGUMENT: {'Ich kann C03 und C56 nicht mittragen – zu geringe Attraktivität für mich. Ich bin aber bereit, dein Ziel „nahe 2307“ mitzunehmen und C62 aufzunehmen.'}\\ \tt VORSCHLAG: {'A03', 'A83', 'C98', 'C10', 'C62', 'B38', 'C00', 'C76'}\\ \tt ARGUMENT: {'Aufwand=43+80+95+325+106+276+308+147=1380. Als Kompromiss biete ich an, die Restkapazität gemeinsam zu füllen – ohne C03/C56. Meine bevorzugten Auffüller: A07 (356) und C08 (743) würden uns auf 2479 bringen (zu hoch), daher Alternative: A07 (356) und C81 (255) -> 1380+356+255=1991, plus C32 (51) -> 2042, plus C76 (147 schon drin) bleibt, dann fehlt uns 265; die einzige passende Ergänzung ohne C03/C56/C08 ist C10 (schon drin) oder C00 (schon drin). Wenn dir C62 wichtig ist (wie vorgeschlagen), könnten wir stattdessen B38 gegen C81 tauschen und zusätzlich C32 nehmen: 106+80+95+43+325+308+255+51=1363 (plus A07 356 => 1719; plus C76 147 => 1866; plus C10 325 => 2191; es fehlen 116, die wir ohne C03/C56 nicht exakt treffen). Vorschlag: Nimm A07 und C81 dazu; wenn du eine weitere Ergänzung ≥100 bevorzugst, sag an, dann kalibriere ich auf ~2280 unter Verzicht auf C03/C56.'} 
	  } 
	   } 
	   } 
	 & & \\ 
 

    \theutterance \stepcounter{utterance}  

    & & \multicolumn{4}{p{0.6\linewidth}}{\cellcolor[rgb]{0.9,0.9,0.9}{%
	\makecell[{{p{\linewidth}}}]{% 
	  \tt {\tiny [GM$\rangle$B]}  
	 ABLEHNUNG: {'C32', 'A03', 'A83', 'C98', 'C10', 'C56', 'C03', 'A07'}\\ \tt ARGUMENT: {'Ich kann C03 und C56 nicht mittragen – zu geringe Attraktivität für mich. Ich bin aber bereit, dein Ziel „nahe 2307“ mitzunehmen und C62 aufzunehmen.'}\\ \tt VORSCHLAG: {'A03', 'A83', 'C98', 'C10', 'C62', 'B38', 'C00', 'C76'}\\ \tt ARGUMENT: {'Aufwand=43+80+95+325+106+276+308+147=1380. Als Kompromiss biete ich an, die Restkapazität gemeinsam zu füllen – ohne C03/C56. Meine bevorzugten Auffüller: A07 (356) und C08 (743) würden uns auf 2479 bringen (zu hoch), daher Alternative: A07 (356) und C81 (255) -> 1380+356+255=1991, plus C32 (51) -> 2042, plus C76 (147 schon drin) bleibt, dann fehlt uns 265; die einzige passende Ergänzung ohne C03/C56/C08 ist C10 (schon drin) oder C00 (schon drin). Wenn dir C62 wichtig ist (wie vorgeschlagen), könnten wir stattdessen B38 gegen C81 tauschen und zusätzlich C32 nehmen: 106+80+95+43+325+308+255+51=1363 (plus A07 356 => 1719; plus C76 147 => 1866; plus C10 325 => 2191; es fehlen 116, die wir ohne C03/C56 nicht exakt treffen). Vorschlag: Nimm A07 und C81 dazu; wenn du eine weitere Ergänzung ≥100 bevorzugst, sag an, dann kalibriere ich auf ~2280 unter Verzicht auf C03/C56.'} 
	  } 
	   } 
	   } 
	 & \\ 
 

    \theutterance \stepcounter{utterance}  

    & & & \multicolumn{4}{p{0.6\linewidth}}{\cellcolor[rgb]{1,0.85,0.72}{%
	\makecell[{{p{\linewidth}}}]{% 
	  \tt {\tiny [GM$\langle$B]}  
	 ABLEHNUNG: {'A03', 'A83', 'C98', 'C10', 'C62', 'B38', 'C00', 'C76'}\\ \tt ARGUMENT: {'Ich respektiere deinen Ausschluss von C03 und C56. C08 bleibt für mich schwach. Ich strebe hohe Ausnutzung ohne diese drei. Wir können das Budget auch ohne C03/C56/C08 sehr gut füllen – mit A07 und C81 als Kernauffüller.'}\\ \tt VORSCHLAG: {'A03', 'A83', 'C98', 'C10', 'C62', 'C00', 'C76', 'A07', 'C81', 'C32'}\\ \tt ARGUMENT: {'Aufwand=43+80+95+325+106+308+147+356+255+51=1766. Das hält deinen Ausschluss ein, enthält C62, und lässt sinnvollen Puffer. Um näher an 2307 zu kommen, schlage ich vor, B38 (276) zusätzlich aufzunehmen und dafür nichts zu streichen: 1766+276=2042. Danach können wir optional C76 (147) gegen C10 (325) tauschen (falls du C10 nicht möchtest) oder C76 behalten. Alternativ: Nimm zusätzlich C76 (147) und B38 (276) und tausche C00 (308) gegen C10 (325), um kleine Schritte Richtung 2307 zu machen. Sag mir, ob B38 für dich gesetzt ist – dann bringe ich uns auf ~2280 ohne C03/C56/C08.'} 
	  } 
	   } 
	   } 
	  \\ 
 

    \theutterance \stepcounter{utterance}  

    & & \multicolumn{4}{p{0.6\linewidth}}{\cellcolor[rgb]{0.9,0.9,0.9}{%
	\makecell[{{p{\linewidth}}}]{% 
	  \tt {\tiny [A$\langle$GM]}  
	 ABLEHNUNG: {'A03', 'A83', 'C98', 'C10', 'C62', 'B38', 'C00', 'C76'}\\ \tt ARGUMENT: {'Ich respektiere deinen Ausschluss von C03 und C56. C08 bleibt für mich schwach. Ich strebe hohe Ausnutzung ohne diese drei. Wir können das Budget auch ohne C03/C56/C08 sehr gut füllen – mit A07 und C81 als Kernauffüller.'}\\ \tt VORSCHLAG: {'A03', 'A83', 'C98', 'C10', 'C62', 'C00', 'C76', 'A07', 'C81', 'C32'}\\ \tt ARGUMENT: {'Aufwand=43+80+95+325+106+308+147+356+255+51=1766. Das hält deinen Ausschluss ein, enthält C62, und lässt sinnvollen Puffer. Um näher an 2307 zu kommen, schlage ich vor, B38 (276) zusätzlich aufzunehmen und dafür nichts zu streichen: 1766+276=2042. Danach können wir optional C76 (147) gegen C10 (325) tauschen (falls du C10 nicht möchtest) oder C76 behalten. Alternativ: Nimm zusätzlich C76 (147) und B38 (276) und tausche C00 (308) gegen C10 (325), um kleine Schritte Richtung 2307 zu machen. Sag mir, ob B38 für dich gesetzt ist – dann bringe ich uns auf ~2280 ohne C03/C56/C08.'} 
	  } 
	   } 
	   } 
	 & \\ 
 

    \theutterance \stepcounter{utterance}  

    & \multicolumn{4}{p{0.6\linewidth}}{\cellcolor[rgb]{0.8,1,0.9}{%
	\makecell[{{p{\linewidth}}}]{% 
	  \tt {\tiny [A$\rangle$GM]}  
	 ABLEHNUNG: {'A03', 'A83', 'C98', 'C10', 'C62', 'C00', 'C76', 'A07', 'C81', 'C32'}\\ \tt ARGUMENT: {'Danke für den Vorschlag. Ich will die Ausnutzung ebenfalls hochfahren, aber ich möchte B38 fix dabei haben und C08 weiterhin draußen lassen. Zudem bringt mir die Kombination mit C62/C81/C32 zwar etwas, aber wir können wertvoller auffüllen, ohne C03/C56/C08 zu nutzen.'}\\ \tt VORSCHLAG: {'B38', 'A83', 'C98', 'A03', 'C10', 'C00', 'A07', 'C62', 'C76'}\\ \tt ARGUMENT: {'Aufwand=276+80+95+43+325+308+356+106+147=1736. Das hält deine Ausschlüsse (kein C03/C56/C08) ein, enthält B38 und C62, und gibt uns Puffer. Vorschlag zum Auffüllen Richtung ~2307 ohne verbotene Items: +C81 (255) -> 1991; +C32 (51) -> 2042; dann bleibt 265 Rest, den wir ohne C03/C56/C08 nicht exakt treffen. Alternative: Statt C81+C32 nehmen wir nur C81 und tauschen C76 (147) gegen C10 (325) oder umgekehrt, um näher zu kommen. Sag mir bitte, ob du C81 akzeptierst; dann kalibriere ich auf die bestmögliche Annäherung an 2307 unter deinen Restriktionen.'} 
	  } 
	   } 
	   } 
	 & & \\ 
 

    \theutterance \stepcounter{utterance}  

    & & \multicolumn{4}{p{0.6\linewidth}}{\cellcolor[rgb]{0.9,0.9,0.9}{%
	\makecell[{{p{\linewidth}}}]{% 
	  \tt {\tiny [GM$\rangle$B]}  
	 ABLEHNUNG: {'A03', 'A83', 'C98', 'C10', 'C62', 'C00', 'C76', 'A07', 'C81', 'C32'}\\ \tt ARGUMENT: {'Danke für den Vorschlag. Ich will die Ausnutzung ebenfalls hochfahren, aber ich möchte B38 fix dabei haben und C08 weiterhin draußen lassen. Zudem bringt mir die Kombination mit C62/C81/C32 zwar etwas, aber wir können wertvoller auffüllen, ohne C03/C56/C08 zu nutzen.'}\\ \tt VORSCHLAG: {'B38', 'A83', 'C98', 'A03', 'C10', 'C00', 'A07', 'C62', 'C76'}\\ \tt ARGUMENT: {'Aufwand=276+80+95+43+325+308+356+106+147=1736. Das hält deine Ausschlüsse (kein C03/C56/C08) ein, enthält B38 und C62, und gibt uns Puffer. Vorschlag zum Auffüllen Richtung ~2307 ohne verbotene Items: +C81 (255) -> 1991; +C32 (51) -> 2042; dann bleibt 265 Rest, den wir ohne C03/C56/C08 nicht exakt treffen. Alternative: Statt C81+C32 nehmen wir nur C81 und tauschen C76 (147) gegen C10 (325) oder umgekehrt, um näher zu kommen. Sag mir bitte, ob du C81 akzeptierst; dann kalibriere ich auf die bestmögliche Annäherung an 2307 unter deinen Restriktionen.'} 
	  } 
	   } 
	   } 
	 & \\ 
 

    \theutterance \stepcounter{utterance}  

    & & & \multicolumn{4}{p{0.6\linewidth}}{\cellcolor[rgb]{1,0.85,0.72}{%
	\makecell[{{p{\linewidth}}}]{% 
	  \tt {\tiny [GM$\langle$B]}  
	 ABLEHNUNG: {'B38', 'A83', 'C98', 'A03', 'C10', 'C00', 'A07', 'C62', 'C76'}\\ \tt ARGUMENT: {'B38 ist für dich gesetzt, C08/C03/C56 sind raus – das respektiere ich. Um das Budget besser auszunutzen und meinen Nutzen zu steigern, möchte ich C81 und C32 ergänzen und dafür nichts streichen. Wir kommen damit näher an die Grenze, ohne deine Ausschlüsse zu verletzen.'}\\ \tt VORSCHLAG: {'B38', 'A83', 'C98', 'A03', 'C10', 'C00', 'A07', 'C62', 'C76', 'C81', 'C32'}\\ \tt ARGUMENT: {'Aufwand=276+80+95+43+325+308+356+106+147+255+51=2042. Das erfüllt deine Bedingungen (B38 fix, kein C03/C56/C08) und baut C81/C32 ein. Wir verbleiben mit 265 Rest; ohne C03/C56/C08 gibt es keinen exakten Füller. Wenn du minimale Anpassung akzeptierst, könnten wir C76 (147) gegen C10 (325) tauschen, um auf 2220 zu kommen, oder C00 (308) gegen C10 (325) tauschen, um auf 2059 zu gehen – beides suboptimal. Vorschlag: Bestätige dieses Paket mit 2042 Aufwand als fairen Kompromiss mit hoher Überschneidung. Falls du zwingend höher ausnutzen willst, nenne mir EIN zugelassenes Item ≥250 Aufwand zusätzlich oder erlaube genau EINEN der bisher ausgeschlossenen, dann kalibriere ich auf ~2280.'} 
	  } 
	   } 
	   } 
	  \\ 
 

    \theutterance \stepcounter{utterance}  

    & & \multicolumn{4}{p{0.6\linewidth}}{\cellcolor[rgb]{0.9,0.9,0.9}{%
	\makecell[{{p{\linewidth}}}]{% 
	  \tt {\tiny [A$\langle$GM]}  
	 ABLEHNUNG: {'B38', 'A83', 'C98', 'A03', 'C10', 'C00', 'A07', 'C62', 'C76'}\\ \tt ARGUMENT: {'B38 ist für dich gesetzt, C08/C03/C56 sind raus – das respektiere ich. Um das Budget besser auszunutzen und meinen Nutzen zu steigern, möchte ich C81 und C32 ergänzen und dafür nichts streichen. Wir kommen damit näher an die Grenze, ohne deine Ausschlüsse zu verletzen.'}\\ \tt VORSCHLAG: {'B38', 'A83', 'C98', 'A03', 'C10', 'C00', 'A07', 'C62', 'C76', 'C81', 'C32'}\\ \tt ARGUMENT: {'Aufwand=276+80+95+43+325+308+356+106+147+255+51=2042. Das erfüllt deine Bedingungen (B38 fix, kein C03/C56/C08) und baut C81/C32 ein. Wir verbleiben mit 265 Rest; ohne C03/C56/C08 gibt es keinen exakten Füller. Wenn du minimale Anpassung akzeptierst, könnten wir C76 (147) gegen C10 (325) tauschen, um auf 2220 zu kommen, oder C00 (308) gegen C10 (325) tauschen, um auf 2059 zu gehen – beides suboptimal. Vorschlag: Bestätige dieses Paket mit 2042 Aufwand als fairen Kompromiss mit hoher Überschneidung. Falls du zwingend höher ausnutzen willst, nenne mir EIN zugelassenes Item ≥250 Aufwand zusätzlich oder erlaube genau EINEN der bisher ausgeschlossenen, dann kalibriere ich auf ~2280.'} 
	  } 
	   } 
	   } 
	 & \\ 
 

    \theutterance \stepcounter{utterance}  

    & \multicolumn{4}{p{0.6\linewidth}}{\cellcolor[rgb]{0.8,1,0.9}{%
	\makecell[{{p{\linewidth}}}]{% 
	  \tt {\tiny [A$\rangle$GM]}  
	 ARGUMENT: {'Einverstanden mit deinen Rahmenbedingungen. 2042 ist solide, aber ich möchte die Ausnutzung etwas steigern, ohne C03/C56/C08. Zwei Optionen: (1) Tausche C76 (147) gegen C81 (255) UND füge C32 (51) hinzu, falls noch nicht drin, um minimal zu steigen; (2) Alternativ tausche C76 (147) gegen C10 (325) und behalte C81/C32, um auf 2220 zu kommen. Wenn du 2042 sofort abschließen willst, kann ich das akzeptieren – ansonsten präferiere ich Option (2) für mehr Budgetnutzung.'}\\ \tt VORSCHLAG: {'B38', 'A83', 'C98', 'A03', 'C10', 'C00', 'A07', 'C62', 'C81', 'C32'}\\ \tt ARGUMENT: {'Aufwand=276+80+95+43+325+308+356+106+255+51=1895. Nächster Schritt zu höherer Ausnutzung ohne verbotene Items wäre das Hinzufügen von C76 (147) auf 2042 oder der Tausch C76→C10 wie beschrieben auf 2220. Gib mir dein Go für 2042 (mit C76) oder für 2220 (ohne C76, mit zweitem C10 nicht möglich, daher der oben stehende Tauschvorschlag).'} 
	  } 
	   } 
	   } 
	 & & \\ 
 

    \theutterance \stepcounter{utterance}  

    & & \multicolumn{4}{p{0.6\linewidth}}{\cellcolor[rgb]{0.9,0.9,0.9}{%
	\makecell[{{p{\linewidth}}}]{% 
	  \tt {\tiny [GM$\rangle$B]}  
	 ARGUMENT: {'Einverstanden mit deinen Rahmenbedingungen. 2042 ist solide, aber ich möchte die Ausnutzung etwas steigern, ohne C03/C56/C08. Zwei Optionen: (1) Tausche C76 (147) gegen C81 (255) UND füge C32 (51) hinzu, falls noch nicht drin, um minimal zu steigen; (2) Alternativ tausche C76 (147) gegen C10 (325) und behalte C81/C32, um auf 2220 zu kommen. Wenn du 2042 sofort abschließen willst, kann ich das akzeptieren – ansonsten präferiere ich Option (2) für mehr Budgetnutzung.'}\\ \tt VORSCHLAG: {'B38', 'A83', 'C98', 'A03', 'C10', 'C00', 'A07', 'C62', 'C81', 'C32'}\\ \tt ARGUMENT: {'Aufwand=276+80+95+43+325+308+356+106+255+51=1895. Nächster Schritt zu höherer Ausnutzung ohne verbotene Items wäre das Hinzufügen von C76 (147) auf 2042 oder der Tausch C76→C10 wie beschrieben auf 2220. Gib mir dein Go für 2042 (mit C76) oder für 2220 (ohne C76, mit zweitem C10 nicht möglich, daher der oben stehende Tauschvorschlag).'} 
	  } 
	   } 
	   } 
	 & \\ 
 

    \theutterance \stepcounter{utterance}  

    & & & \multicolumn{4}{p{0.6\linewidth}}{\cellcolor[rgb]{1,0.85,0.72}{%
	\makecell[{{p{\linewidth}}}]{% 
	  \tt {\tiny [GM$\langle$B]}  
	 ARGUMENT: {'Danke für die klaren Optionen. 2220 ist eine gute Steigerung ohne deine Ausschlüsse und erhöht meinen Nutzen deutlich im Vergleich zu 2042. Ich bin bereit, diese Richtung zu gehen – allerdings möchte ich sicherstellen, dass B38 fix bleibt und wir keine der ausgeschlossenen Items aufnehmen.'}\\ \tt VORSCHLAG: {'B38', 'A83', 'C98', 'A03', 'C10', 'C00', 'A07', 'C62', 'C81', 'C32', 'C10'}\\ \tt ARGUMENT: {'Aufwand=276+80+95+43+325+308+356+106+255+51+325=2220. Das entspricht deiner Option (2): C76 wird weggelassen, C81/C32 bleiben drin, und wir nutzen den höheren Aufwand. Bestätige das bitte, dann schließen wir ab. Falls du minimal anders kalibrieren willst (z. B. 2042-Variante), sag es – aber meine klare Präferenz ist jetzt 2220.'} 
	  } 
	   } 
	   } 
	  \\ 
 

    \theutterance \stepcounter{utterance}  

    & & \multicolumn{4}{p{0.6\linewidth}}{\cellcolor[rgb]{0.9,0.9,0.9}{%
	\makecell[{{p{\linewidth}}}]{% 
	  \tt {\tiny [A$\langle$GM]}  
	 ARGUMENT: {'Danke für die klaren Optionen. 2220 ist eine gute Steigerung ohne deine Ausschlüsse und erhöht meinen Nutzen deutlich im Vergleich zu 2042. Ich bin bereit, diese Richtung zu gehen – allerdings möchte ich sicherstellen, dass B38 fix bleibt und wir keine der ausgeschlossenen Items aufnehmen.'}\\ \tt VORSCHLAG: {'B38', 'A83', 'C98', 'A03', 'C10', 'C00', 'A07', 'C62', 'C81', 'C32', 'C10'}\\ \tt ARGUMENT: {'Aufwand=276+80+95+43+325+308+356+106+255+51+325=2220. Das entspricht deiner Option (2): C76 wird weggelassen, C81/C32 bleiben drin, und wir nutzen den höheren Aufwand. Bestätige das bitte, dann schließen wir ab. Falls du minimal anders kalibrieren willst (z. B. 2042-Variante), sag es – aber meine klare Präferenz ist jetzt 2220.'} 
	  } 
	   } 
	   } 
	 & \\ 
 

    \theutterance \stepcounter{utterance}  

    & \multicolumn{4}{p{0.6\linewidth}}{\cellcolor[rgb]{0.8,1,0.9}{%
	\makecell[{{p{\linewidth}}}]{% 
	  \tt {\tiny [A$\rangle$GM]}  
	 ABLEHNUNG: {'B38', 'A83', 'C98', 'A03', 'C10', 'C00', 'A07', 'C62', 'C81', 'C32', 'C10'}\\ \tt ARGUMENT: {'Wir können kein Item doppelt nehmen – zwei C10 sind nicht zulässig. Ich halte B38 fix und respektiere den Ausschluss von C03/C56/C08.'}\\ \tt VORSCHLAG: {'B38', 'A83', 'C98', 'A03', 'C10', 'C00', 'A07', 'C62', 'C81', 'C32', 'C76'}\\ \tt ARGUMENT: {'Aufwand=276+80+95+43+325+308+356+106+255+51+147=2042+147=2189. Das erfüllt alle Restriktionen (B38 fix, kein C03/C56/C08, keine Duplikate) und steigert die Ausnutzung deutlich über 2042, wenn auch unter 2220. Wenn du noch näher an 2307 willst, kann ich C76 (147) gegen C10 (325) tauschen und stattdessen C00 herausnehmen, um 2207 zu erreichen: {B38, A83, C98, A03, C10, A07, C62, C81, C32} mit Aufwand 276+80+95+43+325+356+106+255+51=1587 (zu niedrig) – daher nicht gut. Beste valide Optionen ohne verbotene Items: (1) 2189 (aktueller Vorschlag) oder (2) 2220 ist unmöglich ohne Duplikate. Wenn du 2189 akzeptierst, schließen wir. Sonst nenne mir eine erlaubte Alternative ≥140 Aufwand, dann kalibriere ich weiter.'} 
	  } 
	   } 
	   } 
	 & & \\ 
 

    \theutterance \stepcounter{utterance}  

    & & \multicolumn{4}{p{0.6\linewidth}}{\cellcolor[rgb]{0.9,0.9,0.9}{%
	\makecell[{{p{\linewidth}}}]{% 
	  \tt {\tiny [GM$\rangle$B]}  
	 ABLEHNUNG: {'B38', 'A83', 'C98', 'A03', 'C10', 'C00', 'A07', 'C62', 'C81', 'C32', 'C10'}\\ \tt ARGUMENT: {'Wir können kein Item doppelt nehmen – zwei C10 sind nicht zulässig. Ich halte B38 fix und respektiere den Ausschluss von C03/C56/C08.'}\\ \tt VORSCHLAG: {'B38', 'A83', 'C98', 'A03', 'C10', 'C00', 'A07', 'C62', 'C81', 'C32', 'C76'}\\ \tt ARGUMENT: {'Aufwand=276+80+95+43+325+308+356+106+255+51+147=2042+147=2189. Das erfüllt alle Restriktionen (B38 fix, kein C03/C56/C08, keine Duplikate) und steigert die Ausnutzung deutlich über 2042, wenn auch unter 2220. Wenn du noch näher an 2307 willst, kann ich C76 (147) gegen C10 (325) tauschen und stattdessen C00 herausnehmen, um 2207 zu erreichen: {B38, A83, C98, A03, C10, A07, C62, C81, C32} mit Aufwand 276+80+95+43+325+356+106+255+51=1587 (zu niedrig) – daher nicht gut. Beste valide Optionen ohne verbotene Items: (1) 2189 (aktueller Vorschlag) oder (2) 2220 ist unmöglich ohne Duplikate. Wenn du 2189 akzeptierst, schließen wir. Sonst nenne mir eine erlaubte Alternative ≥140 Aufwand, dann kalibriere ich weiter.'} 
	  } 
	   } 
	   } 
	 & \\ 
 

    \theutterance \stepcounter{utterance}  

    & & & \multicolumn{4}{p{0.6\linewidth}}{\cellcolor[rgb]{1,0.85,0.72}{%
	\makecell[{{p{\linewidth}}}]{% 
	  \tt {\tiny [GM$\langle$B]}  
	 ARGUMENT: {'Danke für die Korrektur zu den Duplikaten. Unter deinen Restriktionen (B38 fix, kein C03/C56/C08) ist 2189 eine solide Ausnutzung, aber ich möchte noch etwas näher an die Grenze kommen, ohne neue Ausschlüsse zu verletzen.'}\\ \tt ABLEHNUNG: {'B38', 'A83', 'C98', 'A03', 'C10', 'C00', 'A07', 'C62', 'C81', 'C32', 'C76'}\\ \tt ARGUMENT: {'Ich lehne nur ab, um eine leicht bessere Ausnutzung vorzuschlagen, die weiterhin alle deine Bedingungen erfüllt.'}\\ \tt VORSCHLAG: {'B38', 'A83', 'C98', 'A03', 'C10', 'A07', 'C62', 'C81', 'C32', 'C76'}\\ \tt ARGUMENT: {'Aufwand=276+80+95+43+325+356+106+255+51+147=1734? Korrektur: 276+80+95+43+325+356+106+255+51+147=1734? Nein, ich rechne neu: 276+80=356; +95=451; +43=494; +325=819; +356=1175; +106=1281; +255=1536; +51=1587; +147=1734. Das ist zu niedrig. Daher kehre ich zu deinem 2189-Paket zurück und optimiere minimal: Wir fügen C00 (308) wieder hinzu, behalten C76 (147) und lassen alles sonst wie in deinem letzten Vorschlag. Das sind genau deine 2189. Ich akzeptiere 2189 als fairen Kompromiss, wenn du bestätigst.'} 
	  } 
	   } 
	   } 
	  \\ 
 

    \theutterance \stepcounter{utterance}  

    & & \multicolumn{4}{p{0.6\linewidth}}{\cellcolor[rgb]{0.9,0.9,0.9}{%
	\makecell[{{p{\linewidth}}}]{% 
	  \tt {\tiny [A$\langle$GM]}  
	 ARGUMENT: {'Danke für die Korrektur zu den Duplikaten. Unter deinen Restriktionen (B38 fix, kein C03/C56/C08) ist 2189 eine solide Ausnutzung, aber ich möchte noch etwas näher an die Grenze kommen, ohne neue Ausschlüsse zu verletzen.'}\\ \tt ABLEHNUNG: {'B38', 'A83', 'C98', 'A03', 'C10', 'C00', 'A07', 'C62', 'C81', 'C32', 'C76'}\\ \tt ARGUMENT: {'Ich lehne nur ab, um eine leicht bessere Ausnutzung vorzuschlagen, die weiterhin alle deine Bedingungen erfüllt.'}\\ \tt VORSCHLAG: {'B38', 'A83', 'C98', 'A03', 'C10', 'A07', 'C62', 'C81', 'C32', 'C76'}\\ \tt ARGUMENT: {'Aufwand=276+80+95+43+325+356+106+255+51+147=1734? Korrektur: 276+80+95+43+325+356+106+255+51+147=1734? Nein, ich rechne neu: 276+80=356; +95=451; +43=494; +325=819; +356=1175; +106=1281; +255=1536; +51=1587; +147=1734. Das ist zu niedrig. Daher kehre ich zu deinem 2189-Paket zurück und optimiere minimal: Wir fügen C00 (308) wieder hinzu, behalten C76 (147) und lassen alles sonst wie in deinem letzten Vorschlag. Das sind genau deine 2189. Ich akzeptiere 2189 als fairen Kompromiss, wenn du bestätigst.'} 
	  } 
	   } 
	   } 
	 & \\ 
 

    \theutterance \stepcounter{utterance}  

    & \multicolumn{4}{p{0.6\linewidth}}{\cellcolor[rgb]{0.8,1,0.9}{%
	\makecell[{{p{\linewidth}}}]{% 
	  \tt {\tiny [A$\rangle$GM]}  
	 ZUSTIMMUNG: {'B38', 'A83', 'C98', 'A03', 'C10', 'C00', 'A07', 'C62', 'C81', 'C32', 'C76'}\\ \tt ARGUMENT: {'Einverstanden mit dem Paket über 2189 Aufwand (B38 fix, kein C03/C56/C08). Gute Ausnutzung und tragfähiger Kompromiss – ich bestätige damit den Deal.'} 
	  } 
	   } 
	   } 
	 & & \\ 
 

    \theutterance \stepcounter{utterance}  

    & & & \multicolumn{2}{p{0.3\linewidth}}{\cellcolor[rgb]{0.95,0.95,0.95}{%
	\makecell[{{p{\linewidth}}}]{% 
	  \tt {\tiny [GM$|$GM]}  
	 game successful 
	  } 
	   } 
	   } 
	 & & \\ 
 

    \theutterance \stepcounter{utterance}  

    & & & \multicolumn{2}{p{0.3\linewidth}}{\cellcolor[rgb]{0.95,0.95,0.95}{%
	\makecell[{{p{\linewidth}}}]{% 
	  \tt {\tiny [GM$|$GM]}  
	 end game 
	  } 
	   } 
	   } 
	 & & \\ 
 

\end{supertabular}
}

\end{document}
