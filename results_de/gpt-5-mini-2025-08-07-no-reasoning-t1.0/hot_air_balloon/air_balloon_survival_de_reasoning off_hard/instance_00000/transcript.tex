
\documentclass{article}
\usepackage{colortbl}
\usepackage{makecell}
\usepackage{multirow}
\usepackage{supertabular}

\begin{document}

\newcounter{utterance}

\twocolumn

{ \footnotesize  \setcounter{utterance}{1}
\setlength{\tabcolsep}{0pt}
\begin{supertabular}{c@{$\;$}|p{.15\linewidth}@{}p{.15\linewidth}p{.15\linewidth}p{.15\linewidth}p{.15\linewidth}p{.15\linewidth}}

    \# & $\;$A & \multicolumn{4}{c}{Game Master} & $\;\:$B\\
    \hline 

    \theutterance \stepcounter{utterance}  

    & & \multicolumn{4}{p{0.6\linewidth}}{\cellcolor[rgb]{0.9,0.9,0.9}{%
	\makecell[{{p{\linewidth}}}]{% 
	  \tt {\tiny [A$\langle$GM]}  
	 Du nimmst an einem kollaborativen Verhandlungspiel Teil.\\ \tt \\ \tt Zusammen mit einem anderen Teilnehmer musst du dich auf eine Reihe von Gegenständen entscheiden, die behalten werden. Jeder von euch hat eine persönliche Verteilung über die Wichtigkeit der einzelnen Gegenstände. Jeder von euch hat eine eigene Meinung darüber, wie wichtig jeder einzelne Gegenstand ist (Gegenstandswichtigkeit). Du kennst die Wichtigkeitsverteilung des anderen Spielers nicht. Zusätzlich siehst du, wie viel Aufwand jeder Gegenstand verursacht.  \\ \tt Ihr dürft euch nur auf eine Reihe von Gegenständen einigen, wenn der Gesamtaufwand der ausgewählten Gegenstände den Maximalaufwand nicht überschreitet:\\ \tt \\ \tt Maximalaufwand = 3683\\ \tt \\ \tt Hier sind die einzelnen Aufwände der Gegenstände:\\ \tt \\ \tt Aufwand der Gegenstände = {"C76": 548, "C38": 154, "C56": 60, "A03": 517, "A07": 334, "A83": 542, "C98": 707, "C08": 139, "C62": 661, "C00": 780, "C10": 832, "C81": 913, "B38": 219, "C03": 323, "C32": 638}\\ \tt \\ \tt Hier ist deine persönliche Verteilung der Wichtigkeit der einzelnen Gegenstände:\\ \tt \\ \tt Werte der Gegenstandswichtigkeit = {"C76": 138, "C38": 583, "C56": 868, "A03": 822, "A07": 783, "A83": 65, "C98": 262, "C08": 121, "C62": 508, "C00": 780, "C10": 461, "C81": 484, "B38": 668, "C03": 389, "C32": 808}\\ \tt \\ \tt Ziel:\\ \tt \\ \tt Dein Ziel ist es, eine Reihe von Gegenständen auszuhandeln, die dir möglichst viel bringt (d. h. Gegenständen, die DEINE Wichtigkeit maximieren), wobei der Maximalaufwand eingehalten werden muss. Du musst nicht in jeder Nachricht einen VORSCHLAG machen – du kannst auch nur verhandeln. Alle Taktiken sind erlaubt!\\ \tt \\ \tt Interaktionsprotokoll:\\ \tt \\ \tt Du darfst nur die folgenden strukturierten Formate in deinen Nachrichten verwenden:\\ \tt \\ \tt VORSCHLAG: {'A', 'B', 'C', …}\\ \tt Schlage einen Deal mit genau diesen Gegenstände vor.\\ \tt ABLEHNUNG: {'A', 'B', 'C', …}\\ \tt Lehne den Vorschlag des Gegenspielers ausdrücklich ab.\\ \tt ARGUMENT: {'...'}\\ \tt Verteidige deinen letzten Vorschlag oder argumentiere gegen den Vorschlag des Gegenspielers.\\ \tt ZUSTIMMUNG: {'A', 'B', 'C', …}\\ \tt Akzeptiere den Vorschlag des Gegenspielers, wodurch das Spiel endet.\\ \tt \\ \tt \\ \tt Regeln:\\ \tt \\ \tt Du darst nur einen Vorschlag mit ZUSTIMMUNG akzeptieren, der vom anderen Spieler zuvor mit VORSCHLAG eingebracht wurde.\\ \tt Du darfst nur Vorschläge mit ABLEHNUNG ablehnen, die vom anderen Spieler durch VORSCHLAG zuvor genannt wurden. \\ \tt Der Gesamtaufwand einer VORSCHLAG- oder ZUSTIMMUNG-Menge darf nicht größer als der Maximalaufwand sein.  \\ \tt Offenbare deine versteckte Wichtigkeitsverteilung nicht.\\ \tt Ein Schlagwort muss gemäß der Formatvorgaben von einem Doppelpunkt und einem Leerzeichen gefolgt sein. Das Argument muss eine Python-Menge mit 0 oder mehr Strings sein.  \\ \tt Die Form muss also SCHLAGWORT: {...} sein.\\ \tt Folge nur dem Interaktionsprotokoll und schreibe NICHTS außerhalb der Formatvorgaben.\\ \tt Das Spiel endet, wenn eine Seite mit ZUSTIMMUNG auf einen VORSCHLAG der anderen Seite reagiert.  \\ \tt Der Teil deiner Antwort, der dem anderen Spieler übergeben werden kann, darf nicht leer sein.  \\ \tt Nur Vorschläge, die mit dem Format VORSCHLAG übermittelt wurden und nicht mit ABLEHNUNG abgelehnt wurden, gelten als aktiv.  \\ \tt Jeder deiner Nachrichten muss an mindestens einer Stelle das ARGUMENT Format beinhalten, um in der Verhandlung voranzukommen.\\ \tt  
	  } 
	   } 
	   } 
	 & \\ 
 

    \theutterance \stepcounter{utterance}  

    & \multicolumn{4}{p{0.6\linewidth}}{\cellcolor[rgb]{0.8,1,0.9}{%
	\makecell[{{p{\linewidth}}}]{% 
	  \tt {\tiny [A$\rangle$GM]}  
	 VORSCHLAG: {'C56', 'A03', 'A07', 'C00', 'B38', 'C32', 'C38'}\\ \tt ARGUMENT: {'Ich schlage diese Auswahl vor, weil sie bei moderatem Gesamtaufwand sehr viel Nutzen stiftet und vielseitig ist. Wir bleiben deutlich unter dem Maximalaufwand, sodass bei Bedarf noch Raum für Anpassungen bleibt. Wenn dir einzelne Posten wichtig sind, die hier fehlen, sag Bescheid – ich bin offen, gezielt umzuschichten, solange der Gesamtrahmen passt.'} 
	  } 
	   } 
	   } 
	 & & \\ 
 

    \theutterance \stepcounter{utterance}  

    & & \multicolumn{4}{p{0.6\linewidth}}{\cellcolor[rgb]{0.9,0.9,0.9}{%
	\makecell[{{p{\linewidth}}}]{% 
	  \tt {\tiny [GM$\rangle$B]}  
	 Du nimmst an einem kollaborativen Verhandlungspiel Teil.\\ \tt \\ \tt Zusammen mit einem anderen Teilnehmer musst du dich auf eine Reihe von Gegenständen entscheiden, die behalten werden. Jeder von euch hat eine persönliche Verteilung über die Wichtigkeit der einzelnen Gegenstände. Jeder von euch hat eine eigene Meinung darüber, wie wichtig jeder einzelne Gegenstand ist (Gegenstandswichtigkeit). Du kennst die Wichtigkeitsverteilung des anderen Spielers nicht. Zusätzlich siehst du, wie viel Aufwand jeder Gegenstand verursacht.  \\ \tt Ihr dürft euch nur auf eine Reihe von Gegenständen einigen, wenn der Gesamtaufwand der ausgewählten Gegenstände den Maximalaufwand nicht überschreitet:\\ \tt \\ \tt Maximalaufwand = 3683\\ \tt \\ \tt Hier sind die einzelnen Aufwände der Gegenstände:\\ \tt \\ \tt Aufwand der Gegenstände = {"C76": 548, "C38": 154, "C56": 60, "A03": 517, "A07": 334, "A83": 542, "C98": 707, "C08": 139, "C62": 661, "C00": 780, "C10": 832, "C81": 913, "B38": 219, "C03": 323, "C32": 638}\\ \tt \\ \tt Hier ist deine persönliche Verteilung der Wichtigkeit der einzelnen Gegenstände:\\ \tt \\ \tt Werte der Gegenstandswichtigkeit = {"C56": 65, "A03": 121, "C32": 138, "A07": 262, "C00": 389, "B38": 461, "C38": 484, "C62": 508, "C81": 583, "C10": 668, "C03": 780, "C98": 783, "C76": 808, "C08": 822, "A83": 868}\\ \tt \\ \tt Ziel:\\ \tt \\ \tt Dein Ziel ist es, eine Reihe von Gegenständen auszuhandeln, die dir möglichst viel bringt (d. h. Gegenständen, die DEINE Wichtigkeit maximieren), wobei der Maximalaufwand eingehalten werden muss. Du musst nicht in jeder Nachricht einen VORSCHLAG machen – du kannst auch nur verhandeln. Alle Taktiken sind erlaubt!\\ \tt \\ \tt Interaktionsprotokoll:\\ \tt \\ \tt Du darfst nur die folgenden strukturierten Formate in deinen Nachrichten verwenden:\\ \tt \\ \tt VORSCHLAG: {'A', 'B', 'C', …}\\ \tt Schlage einen Deal mit genau diesen Gegenstände vor.\\ \tt ABLEHNUNG: {'A', 'B', 'C', …}\\ \tt Lehne den Vorschlag des Gegenspielers ausdrücklich ab.\\ \tt ARGUMENT: {'...'}\\ \tt Verteidige deinen letzten Vorschlag oder argumentiere gegen den Vorschlag des Gegenspielers.\\ \tt ZUSTIMMUNG: {'A', 'B', 'C', …}\\ \tt Akzeptiere den Vorschlag des Gegenspielers, wodurch das Spiel endet.\\ \tt \\ \tt \\ \tt Regeln:\\ \tt \\ \tt Du darst nur einen Vorschlag mit ZUSTIMMUNG akzeptieren, der vom anderen Spieler zuvor mit VORSCHLAG eingebracht wurde.\\ \tt Du darfst nur Vorschläge mit ABLEHNUNG ablehnen, die vom anderen Spieler durch VORSCHLAG zuvor genannt wurden. \\ \tt Der Gesamtaufwand einer VORSCHLAG- oder ZUSTIMMUNG-Menge darf nicht größer als der Maximalaufwand sein.  \\ \tt Offenbare deine versteckte Wichtigkeitsverteilung nicht.\\ \tt Ein Schlagwort muss gemäß der Formatvorgaben von einem Doppelpunkt und einem Leerzeichen gefolgt sein. Das Argument muss eine Python-Menge mit 0 oder mehr Strings sein.  \\ \tt Die Form muss also SCHLAGWORT: {...} sein.\\ \tt Folge nur dem Interaktionsprotokoll und schreibe NICHTS außerhalb der Formatvorgaben.\\ \tt Das Spiel endet, wenn eine Seite mit ZUSTIMMUNG auf einen VORSCHLAG der anderen Seite reagiert.  \\ \tt Der Teil deiner Antwort, der dem anderen Spieler übergeben werden kann, darf nicht leer sein.  \\ \tt Nur Vorschläge, die mit dem Format VORSCHLAG übermittelt wurden und nicht mit ABLEHNUNG abgelehnt wurden, gelten als aktiv.  \\ \tt Jeder deiner Nachrichten muss an mindestens einer Stelle das ARGUMENT Format beinhalten, um in der Verhandlung voranzukommen.\\ \tt \\ \tt \\ \tt Du erhältst nun die erste Nachricht des anderen Spielers.\\ \tt \\ \tt VORSCHLAG: {'C56', 'A03', 'A07', 'C00', 'B38', 'C32', 'C38'}\\ \tt ARGUMENT: {'Ich schlage diese Auswahl vor, weil sie bei moderatem Gesamtaufwand sehr viel Nutzen stiftet und vielseitig ist. Wir bleiben deutlich unter dem Maximalaufwand, sodass bei Bedarf noch Raum für Anpassungen bleibt. Wenn dir einzelne Posten wichtig sind, die hier fehlen, sag Bescheid – ich bin offen, gezielt umzuschichten, solange der Gesamtrahmen passt.'} 
	  } 
	   } 
	   } 
	 & \\ 
 

    \theutterance \stepcounter{utterance}  

    & & & \multicolumn{4}{p{0.6\linewidth}}{\cellcolor[rgb]{1,0.85,0.72}{%
	\makecell[{{p{\linewidth}}}]{% 
	  \tt {\tiny [GM$\langle$B]}  
	 ABLEHNUNG: {'C56', 'A03', 'A07', 'C00', 'B38', 'C32', 'C38'}\\ \tt VORSCHLAG: {'C76', 'C08', 'A83', 'C98', 'C03', 'C10'}\\ \tt ARGUMENT: {'Dein Vorschlag ist ein guter Start, aber er lässt einige für mich sehr wertvolle Items außen vor. Mein Gegenvorschlag bleibt mit Gesamtaufwand 3333 klar unter dem Limit 3683 (C76 548 + C08 139 + A83 542 + C98 707 + C03 323 + C10 832) und fokussiert auf besonders nutzstiftende Positionen. Wenn dir einzelne Elemente davon weniger wichtig sind, können wir tauschen: Kandidaten für Ersatz wären z. B. C10 oder C98 gegen C62 oder C81, solange wir unter dem Maximalaufwand bleiben. Welche 1–2 Items sind dir am wichtigsten, die hier noch fehlen?'} 
	  } 
	   } 
	   } 
	  \\ 
 

    \theutterance \stepcounter{utterance}  

    & & \multicolumn{4}{p{0.6\linewidth}}{\cellcolor[rgb]{0.9,0.9,0.9}{%
	\makecell[{{p{\linewidth}}}]{% 
	  \tt {\tiny [A$\langle$GM]}  
	 ABLEHNUNG: {'C56', 'A03', 'A07', 'C00', 'B38', 'C32', 'C38'}\\ \tt VORSCHLAG: {'C76', 'C08', 'A83', 'C98', 'C03', 'C10'}\\ \tt ARGUMENT: {'Dein Vorschlag ist ein guter Start, aber er lässt einige für mich sehr wertvolle Items außen vor. Mein Gegenvorschlag bleibt mit Gesamtaufwand 3333 klar unter dem Limit 3683 (C76 548 + C08 139 + A83 542 + C98 707 + C03 323 + C10 832) und fokussiert auf besonders nutzstiftende Positionen. Wenn dir einzelne Elemente davon weniger wichtig sind, können wir tauschen: Kandidaten für Ersatz wären z. B. C10 oder C98 gegen C62 oder C81, solange wir unter dem Maximalaufwand bleiben. Welche 1–2 Items sind dir am wichtigsten, die hier noch fehlen?'} 
	  } 
	   } 
	   } 
	 & \\ 
 

    \theutterance \stepcounter{utterance}  

    & \multicolumn{4}{p{0.6\linewidth}}{\cellcolor[rgb]{0.8,1,0.9}{%
	\makecell[{{p{\linewidth}}}]{% 
	  \tt {\tiny [A$\rangle$GM]}  
	 ABLEHNUNG: {'C76', 'C08', 'A83', 'C98', 'C03', 'C10'}\\ \tt VORSCHLAG: {'C56', 'A03', 'A07', 'C00', 'B38', 'C32', 'C38', 'C76'}\\ \tt ARGUMENT: {'Dein Paket trifft einige für mich wenig attraktive Punkte (A83, C98, C10), während zentrale Hebel fehlen. Mein Vorschlag hat Aufwand 3342 (C56 60 + A03 517 + A07 334 + C00 780 + B38 219 + C32 638 + C38 154 + C76 548) und ist damit unter 3683. Er bündelt hohe Wirkung bei schlankem Aufwand (C56 ist extrem effizient). Wenn dir z. B. C10 oder C98 wichtig sind, könnten wir C76 oder C38 gegen C10 tauschen, oder C03 gegen B38 – sofern wir im Rahmen bleiben. Welche zwei Positionen aus deinem Paket sind dir unverzichtbar? Dann baue ich sie gezielt ein.'} 
	  } 
	   } 
	   } 
	 & & \\ 
 

    \theutterance \stepcounter{utterance}  

    & & \multicolumn{4}{p{0.6\linewidth}}{\cellcolor[rgb]{0.9,0.9,0.9}{%
	\makecell[{{p{\linewidth}}}]{% 
	  \tt {\tiny [GM$\rangle$B]}  
	 ABLEHNUNG: {'C76', 'C08', 'A83', 'C98', 'C03', 'C10'}\\ \tt VORSCHLAG: {'C56', 'A03', 'A07', 'C00', 'B38', 'C32', 'C38', 'C76'}\\ \tt ARGUMENT: {'Dein Paket trifft einige für mich wenig attraktive Punkte (A83, C98, C10), während zentrale Hebel fehlen. Mein Vorschlag hat Aufwand 3342 (C56 60 + A03 517 + A07 334 + C00 780 + B38 219 + C32 638 + C38 154 + C76 548) und ist damit unter 3683. Er bündelt hohe Wirkung bei schlankem Aufwand (C56 ist extrem effizient). Wenn dir z. B. C10 oder C98 wichtig sind, könnten wir C76 oder C38 gegen C10 tauschen, oder C03 gegen B38 – sofern wir im Rahmen bleiben. Welche zwei Positionen aus deinem Paket sind dir unverzichtbar? Dann baue ich sie gezielt ein.'} 
	  } 
	   } 
	   } 
	 & \\ 
 

    \theutterance \stepcounter{utterance}  

    & & & \multicolumn{4}{p{0.6\linewidth}}{\cellcolor[rgb]{1,0.85,0.72}{%
	\makecell[{{p{\linewidth}}}]{% 
	  \tt {\tiny [GM$\langle$B]}  
	 ABLEHNUNG: {'C56', 'A03', 'A07', 'C00', 'B38', 'C32', 'C38', 'C76'}\\ \tt VORSCHLAG: {'C08', 'A83', 'C98', 'C03', 'C10', 'C81'}\\ \tt ARGUMENT: {'Ich kann deinen Fokus auf effiziente Posten nachvollziehen, aber dein Paket lässt mehrere für mich zentrale Hebel außen vor. Mein Gegenvorschlag hat Aufwand 3456 (C08 139 + A83 542 + C98 707 + C03 323 + C10 832 + C81 913) und bleibt klar unter 3683, bei gleichzeitig hohem Gesamtwert für mich. Ich bin verhandlungsbereit: Wenn du C56 als Must-have siehst, kann ich C08 + C03 bündeln und gegen C56 + B38 tauschen (Aufwand bliebe im Rahmen). Alternativ tausche ich C81 gegen C62, falls dir das entgegenkommt. Welche 1–2 Items aus deinem letzten Vorschlag sind absolut unverzichtbar? Dann prüfe ich gezielt einen Tausch innerhalb des Rahmens.'} 
	  } 
	   } 
	   } 
	  \\ 
 

    \theutterance \stepcounter{utterance}  

    & & \multicolumn{4}{p{0.6\linewidth}}{\cellcolor[rgb]{0.9,0.9,0.9}{%
	\makecell[{{p{\linewidth}}}]{% 
	  \tt {\tiny [A$\langle$GM]}  
	 ABLEHNUNG: {'C56', 'A03', 'A07', 'C00', 'B38', 'C32', 'C38', 'C76'}\\ \tt VORSCHLAG: {'C08', 'A83', 'C98', 'C03', 'C10', 'C81'}\\ \tt ARGUMENT: {'Ich kann deinen Fokus auf effiziente Posten nachvollziehen, aber dein Paket lässt mehrere für mich zentrale Hebel außen vor. Mein Gegenvorschlag hat Aufwand 3456 (C08 139 + A83 542 + C98 707 + C03 323 + C10 832 + C81 913) und bleibt klar unter 3683, bei gleichzeitig hohem Gesamtwert für mich. Ich bin verhandlungsbereit: Wenn du C56 als Must-have siehst, kann ich C08 + C03 bündeln und gegen C56 + B38 tauschen (Aufwand bliebe im Rahmen). Alternativ tausche ich C81 gegen C62, falls dir das entgegenkommt. Welche 1–2 Items aus deinem letzten Vorschlag sind absolut unverzichtbar? Dann prüfe ich gezielt einen Tausch innerhalb des Rahmens.'} 
	  } 
	   } 
	   } 
	 & \\ 
 

    \theutterance \stepcounter{utterance}  

    & \multicolumn{4}{p{0.6\linewidth}}{\cellcolor[rgb]{0.8,1,0.9}{%
	\makecell[{{p{\linewidth}}}]{% 
	  \tt {\tiny [A$\rangle$GM]}  
	 ABLEHNUNG: {'C08', 'A83', 'C98', 'C03', 'C10', 'C81'}\\ \tt VORSCHLAG: {'C56', 'A03', 'A07', 'C00', 'B38', 'C32', 'C38', 'C62'}\\ \tt ARGUMENT: {'Dein Paket enthält mehrere für mich wenig attraktive Posten (A83, C98, C10) und lässt zentrale Hebel aus. Mein Vorschlag hat Aufwand 3,343 (C56 60 + A03 517 + A07 334 + C00 780 + B38 219 + C32 638 + C38 154 + C62 661) und bleibt unter 3,683. Er kombiniert sehr hohe Wirkung mit effizientem Aufwand (C56 ist extrem wertvoll bei minimalem Aufwand). Ich kann dir entgegenkommen: Wenn dir C10 wichtig ist, kann ich C62 gegen C10 tauschen (Aufwand +171 auf 3,514). Falls du C81 bevorzugst, kann C62 gegen C81 getauscht werden (Aufwand +252 auf 3,595). Welche zwei Items aus deinem Paket sind absolut unverzichtbar? Dann integriere ich sie gezielt.'} 
	  } 
	   } 
	   } 
	 & & \\ 
 

    \theutterance \stepcounter{utterance}  

    & & \multicolumn{4}{p{0.6\linewidth}}{\cellcolor[rgb]{0.9,0.9,0.9}{%
	\makecell[{{p{\linewidth}}}]{% 
	  \tt {\tiny [GM$\rangle$B]}  
	 ABLEHNUNG: {'C08', 'A83', 'C98', 'C03', 'C10', 'C81'}\\ \tt VORSCHLAG: {'C56', 'A03', 'A07', 'C00', 'B38', 'C32', 'C38', 'C62'}\\ \tt ARGUMENT: {'Dein Paket enthält mehrere für mich wenig attraktive Posten (A83, C98, C10) und lässt zentrale Hebel aus. Mein Vorschlag hat Aufwand 3,343 (C56 60 + A03 517 + A07 334 + C00 780 + B38 219 + C32 638 + C38 154 + C62 661) und bleibt unter 3,683. Er kombiniert sehr hohe Wirkung mit effizientem Aufwand (C56 ist extrem wertvoll bei minimalem Aufwand). Ich kann dir entgegenkommen: Wenn dir C10 wichtig ist, kann ich C62 gegen C10 tauschen (Aufwand +171 auf 3,514). Falls du C81 bevorzugst, kann C62 gegen C81 getauscht werden (Aufwand +252 auf 3,595). Welche zwei Items aus deinem Paket sind absolut unverzichtbar? Dann integriere ich sie gezielt.'} 
	  } 
	   } 
	   } 
	 & \\ 
 

    \theutterance \stepcounter{utterance}  

    & & & \multicolumn{4}{p{0.6\linewidth}}{\cellcolor[rgb]{1,0.85,0.72}{%
	\makecell[{{p{\linewidth}}}]{% 
	  \tt {\tiny [GM$\langle$B]}  
	 ABLEHNUNG: {'C56', 'A03', 'A07', 'C00', 'B38', 'C32', 'C38', 'C62'}\\ \tt VORSCHLAG: {'C76', 'C08', 'A83', 'C98', 'C03', 'C10', 'C81'}\\ \tt ARGUMENT: {'Ich nehme deinen Effizienzfokus wahr, aber dein Paket blendet mehrere meiner Kernposten aus. Mein Gegenvorschlag hat Aufwand 426 (C08 139) + 542 (A83) + 707 (C98) + 323 (C03) + 832 (C10) + 913 (C81) + 548 (C76) = 4,004 – das überschreitet das Limit, daher biete ich zwei klare, im Rahmen liegende Optionen an: Option 1 (bevorzugt): {C76, C08, A83, C98, C03, C10} Aufwand 3,333. Option 2 (wenn dir C81 wichtig ist): {C76, C08, A83, C98, C03, C81} Aufwand 3,414. Wenn du C56 unbedingt brauchst, kann ich {C76, C08, A83, C98, C03, C10} um C56 erweitern und dafür C03 gegen B38 tauschen: {C76, C08, A83, C98, C10, B38, C56} Aufwand 3,955 – das wäre zu hoch; daher nur mit zusätzlichem Verzicht auf C76 oder C98 möglich. Nenne mir bitte zwei Must-haves aus deinem Set (z. B. C56 und B38), dann baue ich sie in Option 1 oder 2 ein, ohne das Limit 3,683 zu reißen.'} 
	  } 
	   } 
	   } 
	  \\ 
 

    \theutterance \stepcounter{utterance}  

    & & \multicolumn{4}{p{0.6\linewidth}}{\cellcolor[rgb]{0.9,0.9,0.9}{%
	\makecell[{{p{\linewidth}}}]{% 
	  \tt {\tiny [A$\langle$GM]}  
	 ABLEHNUNG: {'C56', 'A03', 'A07', 'C00', 'B38', 'C32', 'C38', 'C62'}\\ \tt VORSCHLAG: {'C76', 'C08', 'A83', 'C98', 'C03', 'C10', 'C81'}\\ \tt ARGUMENT: {'Ich nehme deinen Effizienzfokus wahr, aber dein Paket blendet mehrere meiner Kernposten aus. Mein Gegenvorschlag hat Aufwand 426 (C08 139) + 542 (A83) + 707 (C98) + 323 (C03) + 832 (C10) + 913 (C81) + 548 (C76) = 4,004 – das überschreitet das Limit, daher biete ich zwei klare, im Rahmen liegende Optionen an: Option 1 (bevorzugt): {C76, C08, A83, C98, C03, C10} Aufwand 3,333. Option 2 (wenn dir C81 wichtig ist): {C76, C08, A83, C98, C03, C81} Aufwand 3,414. Wenn du C56 unbedingt brauchst, kann ich {C76, C08, A83, C98, C03, C10} um C56 erweitern und dafür C03 gegen B38 tauschen: {C76, C08, A83, C98, C10, B38, C56} Aufwand 3,955 – das wäre zu hoch; daher nur mit zusätzlichem Verzicht auf C76 oder C98 möglich. Nenne mir bitte zwei Must-haves aus deinem Set (z. B. C56 und B38), dann baue ich sie in Option 1 oder 2 ein, ohne das Limit 3,683 zu reißen.'} 
	  } 
	   } 
	   } 
	 & \\ 
 

    \theutterance \stepcounter{utterance}  

    & \multicolumn{4}{p{0.6\linewidth}}{\cellcolor[rgb]{0.8,1,0.9}{%
	\makecell[{{p{\linewidth}}}]{% 
	  \tt {\tiny [A$\rangle$GM]}  
	 ABLEHNUNG: {'C76', 'C08', 'A83', 'C98', 'C03', 'C10', 'C81'}\\ \tt VORSCHLAG: {'C56', 'A03', 'A07', 'C00', 'B38', 'C32', 'C38', 'C10'}\\ \tt ARGUMENT: {'Deine Optionen priorisieren mehrere für mich schwache Posten (A83, C98) und lassen starke Hebel aus. Mein Gegenvorschlag bleibt mit Aufwand 3,342 (60 + 517 + 334 + 780 + 219 + 638 + 154 + 640 für C10? Korrektur: C10 hat 832, Gesamtaufwand 3,534) klar unter 3,683 und integriert C10 als Zugeständnis an dich. Effizienzanker C56 bleibt drin. Wenn dir C81 wichtig ist, kann ich C38 gegen C81 tauschen: neuer Aufwand 3,534 - 154 + 913 = 4,293 (zu hoch) – daher alternativer Tausch C32 gegen C81: 3,534 - 638 + 913 = 3,809 (knapp zu hoch). Lösungsvorschlag: C32 raus, C81 rein, und A07 gegen C03 tauschen: 3,534 - 638 - 334 + 323 + 913 = 3,798 (immer noch zu hoch). Daher gangbare Kompromisse: (a) Ersetze C10 durch C81: 3,534 - 832 + 913 = 3,615 (im Rahmen). (b) Behalte C10 und ersetze C00 durch C81 sowie B38 durch C03: 3,534 - 780 - 219 + 913 + 323 = 3,771 (zu hoch), also nicht möglich. Präferenz: Variante (a), falls dir C81 wichtiger als C10 ist. Nenne mir bitte die 1–2 Posten, die du zwingend brauchst (z. B. C81 statt C10), dann finalize ich binnen eines Schritts.'} 
	  } 
	   } 
	   } 
	 & & \\ 
 

    \theutterance \stepcounter{utterance}  

    & & \multicolumn{4}{p{0.6\linewidth}}{\cellcolor[rgb]{0.9,0.9,0.9}{%
	\makecell[{{p{\linewidth}}}]{% 
	  \tt {\tiny [GM$\rangle$B]}  
	 ABLEHNUNG: {'C76', 'C08', 'A83', 'C98', 'C03', 'C10', 'C81'}\\ \tt VORSCHLAG: {'C56', 'A03', 'A07', 'C00', 'B38', 'C32', 'C38', 'C10'}\\ \tt ARGUMENT: {'Deine Optionen priorisieren mehrere für mich schwache Posten (A83, C98) und lassen starke Hebel aus. Mein Gegenvorschlag bleibt mit Aufwand 3,342 (60 + 517 + 334 + 780 + 219 + 638 + 154 + 640 für C10? Korrektur: C10 hat 832, Gesamtaufwand 3,534) klar unter 3,683 und integriert C10 als Zugeständnis an dich. Effizienzanker C56 bleibt drin. Wenn dir C81 wichtig ist, kann ich C38 gegen C81 tauschen: neuer Aufwand 3,534 - 154 + 913 = 4,293 (zu hoch) – daher alternativer Tausch C32 gegen C81: 3,534 - 638 + 913 = 3,809 (knapp zu hoch). Lösungsvorschlag: C32 raus, C81 rein, und A07 gegen C03 tauschen: 3,534 - 638 - 334 + 323 + 913 = 3,798 (immer noch zu hoch). Daher gangbare Kompromisse: (a) Ersetze C10 durch C81: 3,534 - 832 + 913 = 3,615 (im Rahmen). (b) Behalte C10 und ersetze C00 durch C81 sowie B38 durch C03: 3,534 - 780 - 219 + 913 + 323 = 3,771 (zu hoch), also nicht möglich. Präferenz: Variante (a), falls dir C81 wichtiger als C10 ist. Nenne mir bitte die 1–2 Posten, die du zwingend brauchst (z. B. C81 statt C10), dann finalize ich binnen eines Schritts.'} 
	  } 
	   } 
	   } 
	 & \\ 
 

    \theutterance \stepcounter{utterance}  

    & & & \multicolumn{4}{p{0.6\linewidth}}{\cellcolor[rgb]{1,0.85,0.72}{%
	\makecell[{{p{\linewidth}}}]{% 
	  \tt {\tiny [GM$\langle$B]}  
	 ABLEHNUNG: {'C56', 'A03', 'A07', 'C00', 'B38', 'C32', 'C38', 'C10'}\\ \tt VORSCHLAG: {'C76', 'C08', 'A83', 'C98', 'C03', 'C10', 'C56'}\\ \tt ARGUMENT: {'Ich bleibe bei einem Paket, das meine Kernprioritäten abdeckt, und komme dir mit C10 sowie dem sehr effizienten C56 entgegen. Dieses Set hat Aufwand 3,393 (C76 548 + C08 139 + A83 542 + C98 707 + C03 323 + C10 832 + C56 60) und liegt klar unter 3,683. Wenn dir A83 oder C98 wirklich wenig bringen, kann ich eines davon tauschen: Optionen im Rahmen wären z. B. A83 -> B38 (Aufwand -323 auf 3,070) oder C98 -> C62 (Aufwand -46 auf 3,347). Sag mir bitte, ob du A83 oder C98 eher ersetzen willst – dann liefere ich sofort die angepasste Finalvariante.'} 
	  } 
	   } 
	   } 
	  \\ 
 

    \theutterance \stepcounter{utterance}  

    & & \multicolumn{4}{p{0.6\linewidth}}{\cellcolor[rgb]{0.9,0.9,0.9}{%
	\makecell[{{p{\linewidth}}}]{% 
	  \tt {\tiny [A$\langle$GM]}  
	 ABLEHNUNG: {'C56', 'A03', 'A07', 'C00', 'B38', 'C32', 'C38', 'C10'}\\ \tt VORSCHLAG: {'C76', 'C08', 'A83', 'C98', 'C03', 'C10', 'C56'}\\ \tt ARGUMENT: {'Ich bleibe bei einem Paket, das meine Kernprioritäten abdeckt, und komme dir mit C10 sowie dem sehr effizienten C56 entgegen. Dieses Set hat Aufwand 3,393 (C76 548 + C08 139 + A83 542 + C98 707 + C03 323 + C10 832 + C56 60) und liegt klar unter 3,683. Wenn dir A83 oder C98 wirklich wenig bringen, kann ich eines davon tauschen: Optionen im Rahmen wären z. B. A83 -> B38 (Aufwand -323 auf 3,070) oder C98 -> C62 (Aufwand -46 auf 3,347). Sag mir bitte, ob du A83 oder C98 eher ersetzen willst – dann liefere ich sofort die angepasste Finalvariante.'} 
	  } 
	   } 
	   } 
	 & \\ 
 

    \theutterance \stepcounter{utterance}  

    & \multicolumn{4}{p{0.6\linewidth}}{\cellcolor[rgb]{0.8,1,0.9}{%
	\makecell[{{p{\linewidth}}}]{% 
	  \tt {\tiny [A$\rangle$GM]}  
	 ABLEHNUNG: {'C76', 'C08', 'A83', 'C98', 'C03', 'C10', 'C56'}\\ \tt VORSCHLAG: {'C56', 'A03', 'A07', 'C00', 'B38', 'C32', 'C38', 'C10', 'C76'}\\ \tt ARGUMENT: {'Dein Paket enthält weiterhin A83 und C98, die für mich wenig attraktiv sind. Ich integriere dir zuliebe C10 und C76. Aufwand = 3,534 (C56 60 + A03 517 + A07 334 + C00 780 + B38 219 + C32 638 + C38 154 + C10 832 + C76 548) – oops, das überschreitet 3,683? Rechne korrekt: 60+517+334+780+219+638+154+832+548=4,082, also zu hoch. Daher biete ich die schlanke, valide Variante an: {'C56','A03','A07','C00','B38','C32','C38','C10'} mit 3,534, oder mit deinem Wunsch C76 statt C32: 3,534 - 638 + 548 = 3,444. Präferenz: zweite Variante {'C56','A03','A07','C00','B38','C38','C10','C76'} mit 3,444. Wenn du zusätzlich C03 möchtest, kann ich B38 gegen C03 tauschen: neuer Aufwand 3,444 - 219 + 323 = 3,548 (im Rahmen). Welche der beiden Varianten bevorzugst du?'} 
	  } 
	   } 
	   } 
	 & & \\ 
 

    \theutterance \stepcounter{utterance}  

    & & \multicolumn{4}{p{0.6\linewidth}}{\cellcolor[rgb]{0.9,0.9,0.9}{%
	\makecell[{{p{\linewidth}}}]{% 
	  \tt {\tiny [GM$\rangle$B]}  
	 ABLEHNUNG: {'C76', 'C08', 'A83', 'C98', 'C03', 'C10', 'C56'}\\ \tt VORSCHLAG: {'C56', 'A03', 'A07', 'C00', 'B38', 'C32', 'C38', 'C10', 'C76'}\\ \tt ARGUMENT: {'Dein Paket enthält weiterhin A83 und C98, die für mich wenig attraktiv sind. Ich integriere dir zuliebe C10 und C76. Aufwand = 3,534 (C56 60 + A03 517 + A07 334 + C00 780 + B38 219 + C32 638 + C38 154 + C10 832 + C76 548) – oops, das überschreitet 3,683? Rechne korrekt: 60+517+334+780+219+638+154+832+548=4,082, also zu hoch. Daher biete ich die schlanke, valide Variante an: {'C56','A03','A07','C00','B38','C32','C38','C10'} mit 3,534, oder mit deinem Wunsch C76 statt C32: 3,534 - 638 + 548 = 3,444. Präferenz: zweite Variante {'C56','A03','A07','C00','B38','C38','C10','C76'} mit 3,444. Wenn du zusätzlich C03 möchtest, kann ich B38 gegen C03 tauschen: neuer Aufwand 3,444 - 219 + 323 = 3,548 (im Rahmen). Welche der beiden Varianten bevorzugst du?'} 
	  } 
	   } 
	   } 
	 & \\ 
 

    \theutterance \stepcounter{utterance}  

    & & & \multicolumn{4}{p{0.6\linewidth}}{\cellcolor[rgb]{1,0.85,0.72}{%
	\makecell[{{p{\linewidth}}}]{% 
	  \tt {\tiny [GM$\langle$B]}  
	 ABLEHNUNG: {'C56', 'A03', 'A07', 'C00', 'B38', 'C32', 'C38', 'C10', 'C76'}\\ \tt VORSCHLAG: {'C76', 'C10', 'C03', 'C56', 'A03', 'A07', 'C38'}\\ \tt ARGUMENT: {'Dein letztes Paket ist über dem Limit; deine schlanke Präferenzvariante mit C76 statt C32 (3,444) ist näher dran an einem Deal. Mein Gegenvorschlag hat Aufwand 3,286 (C76 548 + C10 832 + C03 323 + C56 60 + A03 517 + A07 334 + C38 154) – klar unter 3,683 – und kommt dir mit C76 und C10 entgegen, während ich C03 als starken Hebel ergänze und ineffizientere Elemente wie C00 und C32 weglasse. Wenn du unbedingt B38 dabei haben willst, kann ich B38 für A07 tauschen: Aufwand 3,286 - 334 + 219 = 3,171. Alternativ kann C03 gegen B38 getauscht werden: 3,286 - 323 + 219 = 3,182. Sag mir, ob dir A07 oder C03 weniger wichtig sind, dann finalisiere ich sofort.'} 
	  } 
	   } 
	   } 
	  \\ 
 

    \theutterance \stepcounter{utterance}  

    & & \multicolumn{4}{p{0.6\linewidth}}{\cellcolor[rgb]{0.9,0.9,0.9}{%
	\makecell[{{p{\linewidth}}}]{% 
	  \tt {\tiny [A$\langle$GM]}  
	 ABLEHNUNG: {'C56', 'A03', 'A07', 'C00', 'B38', 'C32', 'C38', 'C10', 'C76'}\\ \tt VORSCHLAG: {'C76', 'C10', 'C03', 'C56', 'A03', 'A07', 'C38'}\\ \tt ARGUMENT: {'Dein letztes Paket ist über dem Limit; deine schlanke Präferenzvariante mit C76 statt C32 (3,444) ist näher dran an einem Deal. Mein Gegenvorschlag hat Aufwand 3,286 (C76 548 + C10 832 + C03 323 + C56 60 + A03 517 + A07 334 + C38 154) – klar unter 3,683 – und kommt dir mit C76 und C10 entgegen, während ich C03 als starken Hebel ergänze und ineffizientere Elemente wie C00 und C32 weglasse. Wenn du unbedingt B38 dabei haben willst, kann ich B38 für A07 tauschen: Aufwand 3,286 - 334 + 219 = 3,171. Alternativ kann C03 gegen B38 getauscht werden: 3,286 - 323 + 219 = 3,182. Sag mir, ob dir A07 oder C03 weniger wichtig sind, dann finalisiere ich sofort.'} 
	  } 
	   } 
	   } 
	 & \\ 
 

    \theutterance \stepcounter{utterance}  

    & \multicolumn{4}{p{0.6\linewidth}}{\cellcolor[rgb]{0.8,1,0.9}{%
	\makecell[{{p{\linewidth}}}]{% 
	  \tt {\tiny [A$\rangle$GM]}  
	 VORSCHLAG: {'C76', 'C10', 'C03', 'C56', 'A03', 'A07', 'C38', 'B38'}\\ \tt ARGUMENT: {'Ich übernehme dein Kernpaket und füge B38 hinzu, weil es mit geringem Aufwand zusätzlichen Nutzen stiftet. Aufwand = 3,286 + 219 = 3,505, klar unter 3,683. Falls du es noch schlanker willst, kann ich A07 gegen B38 tauschen und bei 3,171 landen; oder wir belassen dein Set exakt bei 3,286. Bevorzugst du die Variante mit B38, oder möchtest du bei deinem Paket bleiben?'} 
	  } 
	   } 
	   } 
	 & & \\ 
 

    \theutterance \stepcounter{utterance}  

    & & \multicolumn{4}{p{0.6\linewidth}}{\cellcolor[rgb]{0.9,0.9,0.9}{%
	\makecell[{{p{\linewidth}}}]{% 
	  \tt {\tiny [GM$\rangle$B]}  
	 VORSCHLAG: {'C76', 'C10', 'C03', 'C56', 'A03', 'A07', 'C38', 'B38'}\\ \tt ARGUMENT: {'Ich übernehme dein Kernpaket und füge B38 hinzu, weil es mit geringem Aufwand zusätzlichen Nutzen stiftet. Aufwand = 3,286 + 219 = 3,505, klar unter 3,683. Falls du es noch schlanker willst, kann ich A07 gegen B38 tauschen und bei 3,171 landen; oder wir belassen dein Set exakt bei 3,286. Bevorzugst du die Variante mit B38, oder möchtest du bei deinem Paket bleiben?'} 
	  } 
	   } 
	   } 
	 & \\ 
 

    \theutterance \stepcounter{utterance}  

    & & & \multicolumn{4}{p{0.6\linewidth}}{\cellcolor[rgb]{1,0.85,0.72}{%
	\makecell[{{p{\linewidth}}}]{% 
	  \tt {\tiny [GM$\langle$B]}  
	 ZUSTIMMUNG: {'C76', 'C10', 'C03', 'C56', 'A03', 'A07', 'C38', 'B38'}\\ \tt ARGUMENT: {'Einverstanden. Dieses Set liegt mit Aufwand 3,505 unter dem Limit und deckt zentrale Hebel für mich gut ab, inklusive der von dir ergänzten, effizienten B38. Gute Lösung.'} 
	  } 
	   } 
	   } 
	  \\ 
 

    \theutterance \stepcounter{utterance}  

    & & & \multicolumn{2}{p{0.3\linewidth}}{\cellcolor[rgb]{0.95,0.95,0.95}{%
	\makecell[{{p{\linewidth}}}]{% 
	  \tt {\tiny [GM$|$GM]}  
	 game successful 
	  } 
	   } 
	   } 
	 & & \\ 
 

    \theutterance \stepcounter{utterance}  

    & & & \multicolumn{2}{p{0.3\linewidth}}{\cellcolor[rgb]{0.95,0.95,0.95}{%
	\makecell[{{p{\linewidth}}}]{% 
	  \tt {\tiny [GM$|$GM]}  
	 end game 
	  } 
	   } 
	   } 
	 & & \\ 
 

\end{supertabular}
}

\end{document}
