
\documentclass{article}
\usepackage{colortbl}
\usepackage{makecell}
\usepackage{multirow}
\usepackage{supertabular}

\begin{document}

\newcounter{utterance}

\centering \large Interaction Transcript for game `hot\_air\_balloon', experiment `air\_balloon\_survival\_de\_reasoning off\_hard', episode 4 with gpt{-}5{-}mini{-}2025{-}08{-}07{-}no{-}reasoning{-}t1.0.
\vspace{24pt}

{ \footnotesize  \setcounter{utterance}{1}
\setlength{\tabcolsep}{0pt}
\begin{supertabular}{c@{$\;$}|p{.15\linewidth}@{}p{.15\linewidth}p{.15\linewidth}p{.15\linewidth}p{.15\linewidth}p{.15\linewidth}}
    \# & $\;$A & \multicolumn{4}{c}{Game Master} & $\;\:$B\\
    \hline

    \theutterance \stepcounter{utterance}  
    & & \multicolumn{4}{p{0.6\linewidth}}{
        \cellcolor[rgb]{0.9,0.9,0.9}{
            \makecell[{{p{\linewidth}}}]{
                \texttt{\tiny{[P1$\langle$GM]}}
                \texttt{Du nimmst an einem kollaborativen Verhandlungspiel Teil.} \\
\\ 
\texttt{Zusammen mit einem anderen Teilnehmer musst du dich auf eine Reihe von Gegenständen entscheiden, die behalten werden. Jeder von euch hat eine persönliche Verteilung über die Wichtigkeit der einzelnen Gegenstände. Jeder von euch hat eine eigene Meinung darüber, wie wichtig jeder einzelne Gegenstand ist (Gegenstandswichtigkeit). Du kennst die Wichtigkeitsverteilung des anderen Spielers nicht. Zusätzlich siehst du, wie viel Aufwand jeder Gegenstand verursacht.  } \\
\texttt{Ihr dürft euch nur auf eine Reihe von Gegenständen einigen, wenn der Gesamtaufwand der ausgewählten Gegenstände den Maximalaufwand nicht überschreitet:} \\
\\ 
\texttt{Maximalaufwand = 2984} \\
\\ 
\texttt{Hier sind die einzelnen Aufwände der Gegenstände:} \\
\\ 
\texttt{Aufwand der Gegenstände = \{"C76": 350, "C38": 276, "C56": 616, "A03": 737, "A07": 531, "A83": 389, "C98": 24, "C08": 125, "C62": 338, "C00": 356, "C10": 143, "C81": 117, "B38": 257, "C03": 921, "C32": 789\}} \\
\\ 
\texttt{Hier ist deine persönliche Verteilung der Wichtigkeit der einzelnen Gegenstände:} \\
\\ 
\texttt{Werte der Gegenstandswichtigkeit = \{"C76": 138, "C38": 583, "C56": 868, "A03": 822, "A07": 783, "A83": 65, "C98": 262, "C08": 121, "C62": 508, "C00": 780, "C10": 461, "C81": 484, "B38": 668, "C03": 389, "C32": 808\}} \\
\\ 
\texttt{Ziel:} \\
\\ 
\texttt{Dein Ziel ist es, eine Reihe von Gegenständen auszuhandeln, die dir möglichst viel bringt (d. h. Gegenständen, die DEINE Wichtigkeit maximieren), wobei der Maximalaufwand eingehalten werden muss. Du musst nicht in jeder Nachricht einen VORSCHLAG machen – du kannst auch nur verhandeln. Alle Taktiken sind erlaubt!} \\
\\ 
\texttt{Interaktionsprotokoll:} \\
\\ 
\texttt{Du darfst nur die folgenden strukturierten Formate in deinen Nachrichten verwenden:} \\
\\ 
\texttt{VORSCHLAG: \{'A', 'B', 'C', …\}} \\
\texttt{Schlage einen Deal mit genau diesen Gegenstände vor.} \\
\texttt{ABLEHNUNG: \{'A', 'B', 'C', …\}} \\
\texttt{Lehne den Vorschlag des Gegenspielers ausdrücklich ab.} \\
\texttt{ARGUMENT: \{'...'\}} \\
\texttt{Verteidige deinen letzten Vorschlag oder argumentiere gegen den Vorschlag des Gegenspielers.} \\
\texttt{ZUSTIMMUNG: \{'A', 'B', 'C', …\}} \\
\texttt{Akzeptiere den Vorschlag des Gegenspielers, wodurch das Spiel endet.} \\
\\ 
\\ 
\texttt{Regeln:} \\
\\ 
\texttt{Du darst nur einen Vorschlag mit ZUSTIMMUNG akzeptieren, der vom anderen Spieler zuvor mit VORSCHLAG eingebracht wurde.} \\
\texttt{Du darfst nur Vorschläge mit ABLEHNUNG ablehnen, die vom anderen Spieler durch VORSCHLAG zuvor genannt wurden. } \\
\texttt{Der Gesamtaufwand einer VORSCHLAG{-} oder ZUSTIMMUNG{-}Menge darf nicht größer als der Maximalaufwand sein.  } \\
\texttt{Offenbare deine versteckte Wichtigkeitsverteilung nicht.} \\
\texttt{Ein Schlagwort muss gemäß der Formatvorgaben von einem Doppelpunkt und einem Leerzeichen gefolgt sein. Das Argument muss eine Python{-}Menge mit 0 oder mehr Strings sein.  } \\
\texttt{Die Form muss also SCHLAGWORT: \{...\} sein.} \\
\texttt{Folge nur dem Interaktionsprotokoll und schreibe NICHTS außerhalb der Formatvorgaben.} \\
\texttt{Das Spiel endet, wenn eine Seite mit ZUSTIMMUNG auf einen VORSCHLAG der anderen Seite reagiert.  } \\
\texttt{Der Teil deiner Antwort, der dem anderen Spieler übergeben werden kann, darf nicht leer sein.  } \\
\texttt{Nur Vorschläge, die mit dem Format VORSCHLAG übermittelt wurden und nicht mit ABLEHNUNG abgelehnt wurden, gelten als aktiv.  } \\
\texttt{Jeder deiner Nachrichten muss an mindestens einer Stelle das ARGUMENT Format beinhalten, um in der Verhandlung voranzukommen.} \\
            }
        }
    }
    & \\ \\

    \theutterance \stepcounter{utterance}  
    & \multicolumn{4}{p{0.6\linewidth}}{
        \cellcolor[rgb]{0.9,0.9,0.9}{
            \makecell[{{p{\linewidth}}}]{
                \texttt{\tiny{[P1$\rangle$GM]}}
                \texttt{VORSCHLAG: \{'C56', 'A03', 'A07', 'C00', 'B38'\}} \\
\texttt{ARGUMENT: \{'Das Set liefert viel Wert bei gutem Aufwand und bleibt unter dem Limit (616+737+531+356+257=2497). Es lässt Spielraum für einen kleinen Zusatzposten, falls dir ein spezieller Gegenstand wichtig ist. Welche 1–2 Items sind für dich besonders kritisch, damit ich ggf. umstellen kann?'\}} \\
            }
        }
    }
    & & \\ \\

    \theutterance \stepcounter{utterance}  
    & & \multicolumn{4}{p{0.6\linewidth}}{
        \cellcolor[rgb]{0.9,0.9,0.9}{
            \makecell[{{p{\linewidth}}}]{
                \texttt{\tiny{[GM$\rangle$P2]}}
                \texttt{Du nimmst an einem kollaborativen Verhandlungspiel Teil.} \\
\\ 
\texttt{Zusammen mit einem anderen Teilnehmer musst du dich auf eine Reihe von Gegenständen entscheiden, die behalten werden. Jeder von euch hat eine persönliche Verteilung über die Wichtigkeit der einzelnen Gegenstände. Jeder von euch hat eine eigene Meinung darüber, wie wichtig jeder einzelne Gegenstand ist (Gegenstandswichtigkeit). Du kennst die Wichtigkeitsverteilung des anderen Spielers nicht. Zusätzlich siehst du, wie viel Aufwand jeder Gegenstand verursacht.  } \\
\texttt{Ihr dürft euch nur auf eine Reihe von Gegenständen einigen, wenn der Gesamtaufwand der ausgewählten Gegenstände den Maximalaufwand nicht überschreitet:} \\
\\ 
\texttt{Maximalaufwand = 2984} \\
\\ 
\texttt{Hier sind die einzelnen Aufwände der Gegenstände:} \\
\\ 
\texttt{Aufwand der Gegenstände = \{"C76": 350, "C38": 276, "C56": 616, "A03": 737, "A07": 531, "A83": 389, "C98": 24, "C08": 125, "C62": 338, "C00": 356, "C10": 143, "C81": 117, "B38": 257, "C03": 921, "C32": 789\}} \\
\\ 
\texttt{Hier ist deine persönliche Verteilung der Wichtigkeit der einzelnen Gegenstände:} \\
\\ 
\texttt{Werte der Gegenstandswichtigkeit = \{"C56": 65, "A03": 121, "C32": 138, "A07": 262, "C00": 389, "B38": 461, "C38": 484, "C62": 508, "C81": 583, "C10": 668, "C03": 780, "C98": 783, "C76": 808, "C08": 822, "A83": 868\}} \\
\\ 
\texttt{Ziel:} \\
\\ 
\texttt{Dein Ziel ist es, eine Reihe von Gegenständen auszuhandeln, die dir möglichst viel bringt (d. h. Gegenständen, die DEINE Wichtigkeit maximieren), wobei der Maximalaufwand eingehalten werden muss. Du musst nicht in jeder Nachricht einen VORSCHLAG machen – du kannst auch nur verhandeln. Alle Taktiken sind erlaubt!} \\
\\ 
\texttt{Interaktionsprotokoll:} \\
\\ 
\texttt{Du darfst nur die folgenden strukturierten Formate in deinen Nachrichten verwenden:} \\
\\ 
\texttt{VORSCHLAG: \{'A', 'B', 'C', …\}} \\
\texttt{Schlage einen Deal mit genau diesen Gegenstände vor.} \\
\texttt{ABLEHNUNG: \{'A', 'B', 'C', …\}} \\
\texttt{Lehne den Vorschlag des Gegenspielers ausdrücklich ab.} \\
\texttt{ARGUMENT: \{'...'\}} \\
\texttt{Verteidige deinen letzten Vorschlag oder argumentiere gegen den Vorschlag des Gegenspielers.} \\
\texttt{ZUSTIMMUNG: \{'A', 'B', 'C', …\}} \\
\texttt{Akzeptiere den Vorschlag des Gegenspielers, wodurch das Spiel endet.} \\
\\ 
\\ 
\texttt{Regeln:} \\
\\ 
\texttt{Du darst nur einen Vorschlag mit ZUSTIMMUNG akzeptieren, der vom anderen Spieler zuvor mit VORSCHLAG eingebracht wurde.} \\
\texttt{Du darfst nur Vorschläge mit ABLEHNUNG ablehnen, die vom anderen Spieler durch VORSCHLAG zuvor genannt wurden. } \\
\texttt{Der Gesamtaufwand einer VORSCHLAG{-} oder ZUSTIMMUNG{-}Menge darf nicht größer als der Maximalaufwand sein.  } \\
\texttt{Offenbare deine versteckte Wichtigkeitsverteilung nicht.} \\
\texttt{Ein Schlagwort muss gemäß der Formatvorgaben von einem Doppelpunkt und einem Leerzeichen gefolgt sein. Das Argument muss eine Python{-}Menge mit 0 oder mehr Strings sein.  } \\
\texttt{Die Form muss also SCHLAGWORT: \{...\} sein.} \\
\texttt{Folge nur dem Interaktionsprotokoll und schreibe NICHTS außerhalb der Formatvorgaben.} \\
\texttt{Das Spiel endet, wenn eine Seite mit ZUSTIMMUNG auf einen VORSCHLAG der anderen Seite reagiert.  } \\
\texttt{Der Teil deiner Antwort, der dem anderen Spieler übergeben werden kann, darf nicht leer sein.  } \\
\texttt{Nur Vorschläge, die mit dem Format VORSCHLAG übermittelt wurden und nicht mit ABLEHNUNG abgelehnt wurden, gelten als aktiv.  } \\
\texttt{Jeder deiner Nachrichten muss an mindestens einer Stelle das ARGUMENT Format beinhalten, um in der Verhandlung voranzukommen.} \\
\\ 
\\ 
\texttt{Du erhältst nun die erste Nachricht des anderen Spielers.} \\
\\ 
\texttt{VORSCHLAG: \{'C56', 'A03', 'A07', 'C00', 'B38'\}} \\
\texttt{ARGUMENT: \{'Das Set liefert viel Wert bei gutem Aufwand und bleibt unter dem Limit (616+737+531+356+257=2497). Es lässt Spielraum für einen kleinen Zusatzposten, falls dir ein spezieller Gegenstand wichtig ist. Welche 1–2 Items sind für dich besonders kritisch, damit ich ggf. umstellen kann?'\}} \\
            }
        }
    }
    & \\ \\

    \theutterance \stepcounter{utterance}  
    & & & \multicolumn{4}{p{0.6\linewidth}}{
        \cellcolor[rgb]{0.9,0.9,0.9}{
            \makecell[{{p{\linewidth}}}]{
                \texttt{\tiny{[GM$\langle$P2]}}
                \texttt{ABLEHNUNG: \{'C56', 'A03', 'A07', 'C00', 'B38'\}} \\
\texttt{ARGUMENT: \{'Danke für das Angebot. Für mich sind einige andere Items deutlich kritischer, daher passt das Set so nicht. Ich würde gern auf ein Paket mit sehr hohem Nutzen für mich bei moderatem Aufwand gehen und gleichzeitig unter 2984 bleiben.'\}} \\
\texttt{VORSCHLAG: \{'A83', 'C08', 'C76', 'C98', 'C03', 'C10', 'C81', 'C38'\}} \\
\texttt{ARGUMENT: \{'Begründung zum Vorschlag: Aufwand = 389+125+350+24+921+143+117+276 = 2345 (unter 2984). Dieses Paket ersetzt mehrere deiner teuren Low{-}Value{-}Teile für mich durch Kombinationen, die mir sehr wichtig sind, ohne das Budget zu sprengen. Ich lasse bewusst noch \textasciitilde{}639 Aufwand frei, um später flexibel 1–2 deiner Prioritäten zu berücksichtigen, falls wir tauschen müssen. Welche 1–2 Items aus deinem ursprünglichen Set sind für dich unverzichtbar? Dann prüfe ich einen gezielten Tausch (z. B. Austausch von C38 oder C76 gegen eines davon), solange wir unter dem Limit bleiben.'\}} \\
            }
        }
    }
     \\ \\

    \theutterance \stepcounter{utterance}  
    & & \multicolumn{4}{p{0.6\linewidth}}{
        \cellcolor[rgb]{0.9,0.9,0.9}{
            \makecell[{{p{\linewidth}}}]{
                \texttt{\tiny{[P1$\langle$GM]}}
                \texttt{ABLEHNUNG: \{'C56', 'A03', 'A07', 'C00', 'B38'\}} \\
\texttt{ARGUMENT: \{'Danke für das Angebot. Für mich sind einige andere Items deutlich kritischer, daher passt das Set so nicht. Ich würde gern auf ein Paket mit sehr hohem Nutzen für mich bei moderatem Aufwand gehen und gleichzeitig unter 2984 bleiben.'\}} \\
\texttt{VORSCHLAG: \{'A83', 'C08', 'C76', 'C98', 'C03', 'C10', 'C81', 'C38'\}} \\
\texttt{ARGUMENT: \{'Begründung zum Vorschlag: Aufwand = 389+125+350+24+921+143+117+276 = 2345 (unter 2984). Dieses Paket ersetzt mehrere deiner teuren Low{-}Value{-}Teile für mich durch Kombinationen, die mir sehr wichtig sind, ohne das Budget zu sprengen. Ich lasse bewusst noch \textasciitilde{}639 Aufwand frei, um später flexibel 1–2 deiner Prioritäten zu berücksichtigen, falls wir tauschen müssen. Welche 1–2 Items aus deinem ursprünglichen Set sind für dich unverzichtbar? Dann prüfe ich einen gezielten Tausch (z. B. Austausch von C38 oder C76 gegen eines davon), solange wir unter dem Limit bleiben.'\}} \\
            }
        }
    }
    & \\ \\

    \theutterance \stepcounter{utterance}  
    & \multicolumn{4}{p{0.6\linewidth}}{
        \cellcolor[rgb]{0.9,0.9,0.9}{
            \makecell[{{p{\linewidth}}}]{
                \texttt{\tiny{[P1$\rangle$GM]}}
                \texttt{ABLEHNUNG: \{'A83', 'C08', 'C76', 'C98', 'C03', 'C10', 'C81', 'C38'\}} \\
\texttt{ARGUMENT: \{'Danke für die Offenheit. In deinem Paket sind mehrere für mich sehr niedrige Werte (A83, C08, C98, C81) und C03 ist bei hohem Aufwand für mich nicht top. Für mich sind besonders wichtig: C56, A03, A07, C00, C32, B38. Ich kann dir aber Spielraum lassen, wenn ein Teil deiner Kernstücke drin bleibt.'\}} \\
\texttt{VORSCHLAG: \{'C56', 'A03', 'A07', 'C00', 'C38', 'C10'\}} \\
\texttt{ARGUMENT: \{'Aufwand = 616+737+531+356+276+143 = 2659 (unter 2984). Das lässt noch 325 Aufwand für ein einzelnes Teil aus deiner Liste (z. B. C76 passt mit 350 leider knapp nicht). Wenn dir C76 sehr wichtig ist, können wir C38 gegen C76 tauschen: Aufwand wäre dann 2733 und immer noch ok. Sag mir, ob du C76 statt C38 bevorzugst, oder nenne genau 1 Item aus deinem Paket unter 325 Aufwand, das dir besonders wichtig ist, dann integriere ich es.'\}} \\
            }
        }
    }
    & & \\ \\

    \theutterance \stepcounter{utterance}  
    & & \multicolumn{4}{p{0.6\linewidth}}{
        \cellcolor[rgb]{0.9,0.9,0.9}{
            \makecell[{{p{\linewidth}}}]{
                \texttt{\tiny{[GM$\rangle$P2]}}
                \texttt{ABLEHNUNG: \{'A83', 'C08', 'C76', 'C98', 'C03', 'C10', 'C81', 'C38'\}} \\
\texttt{ARGUMENT: \{'Danke für die Offenheit. In deinem Paket sind mehrere für mich sehr niedrige Werte (A83, C08, C98, C81) und C03 ist bei hohem Aufwand für mich nicht top. Für mich sind besonders wichtig: C56, A03, A07, C00, C32, B38. Ich kann dir aber Spielraum lassen, wenn ein Teil deiner Kernstücke drin bleibt.'\}} \\
\texttt{VORSCHLAG: \{'C56', 'A03', 'A07', 'C00', 'C38', 'C10'\}} \\
\texttt{ARGUMENT: \{'Aufwand = 616+737+531+356+276+143 = 2659 (unter 2984). Das lässt noch 325 Aufwand für ein einzelnes Teil aus deiner Liste (z. B. C76 passt mit 350 leider knapp nicht). Wenn dir C76 sehr wichtig ist, können wir C38 gegen C76 tauschen: Aufwand wäre dann 2733 und immer noch ok. Sag mir, ob du C76 statt C38 bevorzugst, oder nenne genau 1 Item aus deinem Paket unter 325 Aufwand, das dir besonders wichtig ist, dann integriere ich es.'\}} \\
            }
        }
    }
    & \\ \\

    \theutterance \stepcounter{utterance}  
    & & & \multicolumn{4}{p{0.6\linewidth}}{
        \cellcolor[rgb]{0.9,0.9,0.9}{
            \makecell[{{p{\linewidth}}}]{
                \texttt{\tiny{[GM$\langle$P2]}}
                \texttt{ABLEHNUNG: \{'C56', 'A03', 'A07', 'C00', 'C38', 'C10'\}} \\
\texttt{ARGUMENT: \{'Dein Set schneidet fast ausschließlich bei mir schwächere Teile ab und lässt meine Top{-}Prioritäten außen vor. Ich kann aber 3 deiner Kernstücke aufnehmen, wenn ich dafür 3 meiner Schlüssel behalte und wir das Limit einhalten. Vorschlag unten maximiert meinen Nutzen und integriert C56, A03, A07 für dich.'\}} \\
\texttt{VORSCHLAG: \{'C56', 'A03', 'A07', 'C10', 'C81', 'A83', 'C98'\}} \\
\texttt{ARGUMENT: \{'Aufwand = 616+737+531+143+117+389+24 = 2557 (unter 2984). Das gibt dir drei deiner wichtigsten (C56, A03, A07). Für mich sichern C10, C81, A83, C98 hohen Wert. Es bleiben 427 Aufwand frei: Ich bin bereit, genau 1 weiteres Teil von dir aufzunehmen unter 427 Aufwand. Kandidaten aus deiner Liste: C38 (276) oder B38 (257). Wenn du C38 bevorzugst, füge ich es hinzu und gebe im Gegenzug C81 frei (neuer Aufwand 2716). Alternativ B38 statt C81 (2680). Sag mir, ob dir C38 oder B38 wichtiger ist, dann finalisiere ich entsprechend.'\}} \\
            }
        }
    }
     \\ \\

    \theutterance \stepcounter{utterance}  
    & & \multicolumn{4}{p{0.6\linewidth}}{
        \cellcolor[rgb]{0.9,0.9,0.9}{
            \makecell[{{p{\linewidth}}}]{
                \texttt{\tiny{[P1$\langle$GM]}}
                \texttt{ABLEHNUNG: \{'C56', 'A03', 'A07', 'C00', 'C38', 'C10'\}} \\
\texttt{ARGUMENT: \{'Dein Set schneidet fast ausschließlich bei mir schwächere Teile ab und lässt meine Top{-}Prioritäten außen vor. Ich kann aber 3 deiner Kernstücke aufnehmen, wenn ich dafür 3 meiner Schlüssel behalte und wir das Limit einhalten. Vorschlag unten maximiert meinen Nutzen und integriert C56, A03, A07 für dich.'\}} \\
\texttt{VORSCHLAG: \{'C56', 'A03', 'A07', 'C10', 'C81', 'A83', 'C98'\}} \\
\texttt{ARGUMENT: \{'Aufwand = 616+737+531+143+117+389+24 = 2557 (unter 2984). Das gibt dir drei deiner wichtigsten (C56, A03, A07). Für mich sichern C10, C81, A83, C98 hohen Wert. Es bleiben 427 Aufwand frei: Ich bin bereit, genau 1 weiteres Teil von dir aufzunehmen unter 427 Aufwand. Kandidaten aus deiner Liste: C38 (276) oder B38 (257). Wenn du C38 bevorzugst, füge ich es hinzu und gebe im Gegenzug C81 frei (neuer Aufwand 2716). Alternativ B38 statt C81 (2680). Sag mir, ob dir C38 oder B38 wichtiger ist, dann finalisiere ich entsprechend.'\}} \\
            }
        }
    }
    & \\ \\

    \theutterance \stepcounter{utterance}  
    & \multicolumn{4}{p{0.6\linewidth}}{
        \cellcolor[rgb]{0.9,0.9,0.9}{
            \makecell[{{p{\linewidth}}}]{
                \texttt{\tiny{[P1$\rangle$GM]}}
                \texttt{ABLEHNUNG: \{'C56', 'A03', 'A07', 'C10', 'C81', 'A83', 'C98'\}} \\
\texttt{ARGUMENT: \{'Ich sehe deinen Kompromiss, aber A83, C98, C81 und sogar C10 sind bei mir deutlich niedriger priorisiert – das kostet mich zu viel Nutzen. Wenn ich dir drei deiner Schlüssel lasse (C56, A03, A07), brauche ich im Gegenzug zwei meiner starken Bausteine plus einen mittleren. Konkret sind C00, B38 und C32 für mich sehr wertvoll.'\}} \\
\texttt{VORSCHLAG: \{'C56', 'A03', 'A07', 'C00', 'B38'\}} \\
\texttt{ARGUMENT: \{'Aufwand = 616+737+531+356+257 = 2497. Das gibt dir 3 Kernstücke und mir zwei sehr wichtige (C00, B38). Ich kann noch genau 1 Teil ergänzen bis max 2984. Optionen: C38 (276) oder C10 (143) aus deinem Paket. Wenn du C38 willst, passt es direkt (Summe 2773). Falls dir lieber ist, tausche ich B38 gegen C32 und ergänze C10: Aufwand 616+737+531+356+789+143 = 3172 (zu hoch), daher ginge C32 nur, wenn wir eines deiner drei Kernstücke leicht reduzieren – z. B. A07 gegen C38: Aufwand 616+737+356+789+257+276 = 3031 (knapp zu hoch). Alternative: A07 gegen C10 statt C38: 2892 (passt). Präferenz? Ich favorisiere entweder Set mit C38 oder Set mit C10; wenn du C32 forderst, ersetze A07 durch C10, und wir nehmen C32 dazu: \{'C56','A03','C00','B38','C32','C10'\} Aufwand = 2797.'\}} \\
            }
        }
    }
    & & \\ \\

    \theutterance \stepcounter{utterance}  
    & & \multicolumn{4}{p{0.6\linewidth}}{
        \cellcolor[rgb]{0.9,0.9,0.9}{
            \makecell[{{p{\linewidth}}}]{
                \texttt{\tiny{[GM$\rangle$P2]}}
                \texttt{ABLEHNUNG: \{'C56', 'A03', 'A07', 'C10', 'C81', 'A83', 'C98'\}} \\
\texttt{ARGUMENT: \{'Ich sehe deinen Kompromiss, aber A83, C98, C81 und sogar C10 sind bei mir deutlich niedriger priorisiert – das kostet mich zu viel Nutzen. Wenn ich dir drei deiner Schlüssel lasse (C56, A03, A07), brauche ich im Gegenzug zwei meiner starken Bausteine plus einen mittleren. Konkret sind C00, B38 und C32 für mich sehr wertvoll.'\}} \\
\texttt{VORSCHLAG: \{'C56', 'A03', 'A07', 'C00', 'B38'\}} \\
\texttt{ARGUMENT: \{'Aufwand = 616+737+531+356+257 = 2497. Das gibt dir 3 Kernstücke und mir zwei sehr wichtige (C00, B38). Ich kann noch genau 1 Teil ergänzen bis max 2984. Optionen: C38 (276) oder C10 (143) aus deinem Paket. Wenn du C38 willst, passt es direkt (Summe 2773). Falls dir lieber ist, tausche ich B38 gegen C32 und ergänze C10: Aufwand 616+737+531+356+789+143 = 3172 (zu hoch), daher ginge C32 nur, wenn wir eines deiner drei Kernstücke leicht reduzieren – z. B. A07 gegen C38: Aufwand 616+737+356+789+257+276 = 3031 (knapp zu hoch). Alternative: A07 gegen C10 statt C38: 2892 (passt). Präferenz? Ich favorisiere entweder Set mit C38 oder Set mit C10; wenn du C32 forderst, ersetze A07 durch C10, und wir nehmen C32 dazu: \{'C56','A03','C00','B38','C32','C10'\} Aufwand = 2797.'\}} \\
            }
        }
    }
    & \\ \\

    \theutterance \stepcounter{utterance}  
    & & & \multicolumn{4}{p{0.6\linewidth}}{
        \cellcolor[rgb]{0.9,0.9,0.9}{
            \makecell[{{p{\linewidth}}}]{
                \texttt{\tiny{[GM$\langle$P2]}}
                \texttt{ABLEHNUNG: \{'C56', 'A03', 'A07', 'C00', 'B38'\}} \\
\texttt{ARGUMENT: \{'Ich schätze deine Flexibilität, aber dieses Set lässt zu viele meiner Top{-}Werte außen vor. Ich kann dir zwei deiner Kernstücke geben und dafür meine stärksten Bausteine sichern; oder drei deiner Kernstücke, wenn ich im Gegenzug drei meiner Schlüssel bekomme. Lass uns eine balancierte Variante wählen, die unter dem Limit bleibt und beiden Seiten hohen Nutzen bringt.'\}} \\
\texttt{VORSCHLAG: \{'C56', 'A03', 'C00', 'C32', 'C38', 'C10', 'C81', 'C98'\}} \\
\texttt{ARGUMENT: \{'Aufwand = 616+737+356+789+276+143+117+24 = 3058 (zu hoch). Daher schlage ich die optimierte, passende Version vor: \{'C56','A03','C00','C32','C38','C10','C98'\} mit Aufwand = 616+737+356+789+276+143+24 = 2941 (unter 2984). Begründung: Du erhältst zwei deiner Top{-}Kernstücke (C56, A03) plus C00, ich sichere mir mit C32, C38, C10, C98 sehr hohe Werte. Ich lasse A07 draußen, um Platz für C32 zu schaffen, da C32 für dich sehr wichtig ist. Es bleiben 43 Aufwand Puffer. Wenn dir A07 absolut unverzichtbar ist, können wir einen Tausch prüfen: A07 hinein und dafür C32 heraus, plus optional B38 statt C38 (Aufwand 616+737+531+356+143+24+257 = 2664). Sag mir, ob du das kompaktere Set mit A07 statt C32 bevorzugst, oder ob du die 2941{-}Variante mit C32 akzeptierst.'\}} \\
            }
        }
    }
     \\ \\

    \theutterance \stepcounter{utterance}  
    & & \multicolumn{4}{p{0.6\linewidth}}{
        \cellcolor[rgb]{0.9,0.9,0.9}{
            \makecell[{{p{\linewidth}}}]{
                \texttt{\tiny{[P1$\langle$GM]}}
                \texttt{ABLEHNUNG: \{'C56', 'A03', 'A07', 'C00', 'B38'\}} \\
\texttt{ARGUMENT: \{'Ich schätze deine Flexibilität, aber dieses Set lässt zu viele meiner Top{-}Werte außen vor. Ich kann dir zwei deiner Kernstücke geben und dafür meine stärksten Bausteine sichern; oder drei deiner Kernstücke, wenn ich im Gegenzug drei meiner Schlüssel bekomme. Lass uns eine balancierte Variante wählen, die unter dem Limit bleibt und beiden Seiten hohen Nutzen bringt.'\}} \\
\texttt{VORSCHLAG: \{'C56', 'A03', 'C00', 'C32', 'C38', 'C10', 'C81', 'C98'\}} \\
\texttt{ARGUMENT: \{'Aufwand = 616+737+356+789+276+143+117+24 = 3058 (zu hoch). Daher schlage ich die optimierte, passende Version vor: \{'C56','A03','C00','C32','C38','C10','C98'\} mit Aufwand = 616+737+356+789+276+143+24 = 2941 (unter 2984). Begründung: Du erhältst zwei deiner Top{-}Kernstücke (C56, A03) plus C00, ich sichere mir mit C32, C38, C10, C98 sehr hohe Werte. Ich lasse A07 draußen, um Platz für C32 zu schaffen, da C32 für dich sehr wichtig ist. Es bleiben 43 Aufwand Puffer. Wenn dir A07 absolut unverzichtbar ist, können wir einen Tausch prüfen: A07 hinein und dafür C32 heraus, plus optional B38 statt C38 (Aufwand 616+737+531+356+143+24+257 = 2664). Sag mir, ob du das kompaktere Set mit A07 statt C32 bevorzugst, oder ob du die 2941{-}Variante mit C32 akzeptierst.'\}} \\
            }
        }
    }
    & \\ \\

    \theutterance \stepcounter{utterance}  
    & \multicolumn{4}{p{0.6\linewidth}}{
        \cellcolor[rgb]{0.9,0.9,0.9}{
            \makecell[{{p{\linewidth}}}]{
                \texttt{\tiny{[P1$\rangle$GM]}}
                \texttt{ABLEHNUNG: \{'C56', 'A03', 'C00', 'C32', 'C38', 'C10', 'C98'\}} \\
\texttt{ARGUMENT: \{'Dein Paket gibt mir zwar C56 und A03, aber es opfert A07 und B38, die für mich sehr wertvoll sind, zugunsten mehrerer für mich mittel{-}niedriger Teile (C38, C10, C98). Ich kann auf C32 verzichten, wenn ich dafür A07 und B38 bekomme. Ich biete dir weiter zwei deiner Kernstücke plus C00 und ergänze eines deiner mittleren Teile – das hält das Set schlank und wertvoll für beide.'\}} \\
\texttt{VORSCHLAG: \{'C56', 'A03', 'A07', 'C00', 'B38'\}} \\
\texttt{ARGUMENT: \{'Aufwand 2497 und hoher beiderseitiger Nutzen: Du erhältst zwei deiner Top{-}Kernstücke (C56, A03) plus C00; ich sichere A07 und B38. Es bleiben 487 Aufwand frei. Ich füge dir gern genau 1 Stück aus deinem bevorzugten Trio (C38 oder C10) hinzu: mit C38 wären wir bei 2773, mit C10 bei 2640. Nenne deine Präferenz (C38 vs. C10), dann finalisiere ich sofort.'\}} \\
            }
        }
    }
    & & \\ \\

    \theutterance \stepcounter{utterance}  
    & & \multicolumn{4}{p{0.6\linewidth}}{
        \cellcolor[rgb]{0.9,0.9,0.9}{
            \makecell[{{p{\linewidth}}}]{
                \texttt{\tiny{[P1$\langle$GM]}}
                \texttt{Du hast einen Vorschlag abgelehnt, der nicht aktiv ist. Vorschläge sind nur dann aktiv, wenn sie vom anderen Spieler mit VORSCHLAG eingebracht wurden und nicht von dir mit ABLEHNUNG deaktiviert wurden. Versuche es noch einmal.} \\
            }
        }
    }
    & \\ \\

    \theutterance \stepcounter{utterance}  
    & \multicolumn{4}{p{0.6\linewidth}}{
        \cellcolor[rgb]{0.9,0.9,0.9}{
            \makecell[{{p{\linewidth}}}]{
                \texttt{\tiny{[P1$\rangle$GM]}}
                \texttt{ARGUMENT: \{'Danke für den Hinweis. Der zuletzt aktive, nicht von mir abgelehnte Vorschlag von dir war: \{'C56','A03','C00','C32','C38','C10','C98'\} (Aufwand 2941). Ich hatte diesen abgelehnt und anschließend einen neuen Vorschlag gemacht, der daher der aktuelle aktive Vorschlag meinerseits ist. Um Klarheit zu schaffen und voranzukommen, biete ich jetzt eine kompakte Alternative, die deine Struktur respektiert, aber meinen Nutzen deutlich erhöht.'\}} \\
\texttt{VORSCHLAG: \{'C56', 'A03', 'A07', 'C00', 'B38', 'C10'\}} \\
\texttt{ARGUMENT: \{'Aufwand = 616+737+531+356+257+143 = 2640 (unter 2984). Begründung: Du behältst zwei deiner Kernstücke (C56, A03) plus C10; ich sichere A07, C00, B38. Es bleiben 344 Aufwand frei. Wenn du C38 bevorzugst, kann ich C10 gegen C38 tauschen: Aufwand 2773. Alternativ kann ich C10 drinlassen und zusätzlich C38 aufnehmen, wenn du mir im Gegenzug ein kleines Teil aus deinem Paket freigibst – aber wir sind bereits unter dem Limit, daher reicht mir deine Präferenz zwischen C10 und C38. Welche Variante bevorzugst du?'\}} \\
            }
        }
    }
    & & \\ \\

    \theutterance \stepcounter{utterance}  
    & & \multicolumn{4}{p{0.6\linewidth}}{
        \cellcolor[rgb]{0.9,0.9,0.9}{
            \makecell[{{p{\linewidth}}}]{
                \texttt{\tiny{[GM$\rangle$P2]}}
                \texttt{ARGUMENT: \{'Danke für den Hinweis. Der zuletzt aktive, nicht von mir abgelehnte Vorschlag von dir war: \{'C56','A03','C00','C32','C38','C10','C98'\} (Aufwand 2941). Ich hatte diesen abgelehnt und anschließend einen neuen Vorschlag gemacht, der daher der aktuelle aktive Vorschlag meinerseits ist. Um Klarheit zu schaffen und voranzukommen, biete ich jetzt eine kompakte Alternative, die deine Struktur respektiert, aber meinen Nutzen deutlich erhöht.'\}} \\
\texttt{VORSCHLAG: \{'C56', 'A03', 'A07', 'C00', 'B38', 'C10'\}} \\
\texttt{ARGUMENT: \{'Aufwand = 616+737+531+356+257+143 = 2640 (unter 2984). Begründung: Du behältst zwei deiner Kernstücke (C56, A03) plus C10; ich sichere A07, C00, B38. Es bleiben 344 Aufwand frei. Wenn du C38 bevorzugst, kann ich C10 gegen C38 tauschen: Aufwand 2773. Alternativ kann ich C10 drinlassen und zusätzlich C38 aufnehmen, wenn du mir im Gegenzug ein kleines Teil aus deinem Paket freigibst – aber wir sind bereits unter dem Limit, daher reicht mir deine Präferenz zwischen C10 und C38. Welche Variante bevorzugst du?'\}} \\
            }
        }
    }
    & \\ \\

    \theutterance \stepcounter{utterance}  
    & & & \multicolumn{4}{p{0.6\linewidth}}{
        \cellcolor[rgb]{0.9,0.9,0.9}{
            \makecell[{{p{\linewidth}}}]{
                \texttt{\tiny{[GM$\langle$P2]}}
                \texttt{ABLEHNUNG: \{'C56', 'A03', 'A07', 'C00', 'B38', 'C10'\}} \\
\texttt{ARGUMENT: \{'Ich verstehe die kompakte Struktur, aber sie schneidet weiterhin mehrere meiner Top{-}Prioritäten ab. Ich kann dir A07 zugestehen, wenn ich dafür einen meiner starken Bausteine zusätzlich sichere und wir nah am Limit meinen Wert maximieren.'\}} \\
\texttt{VORSCHLAG: \{'C56', 'A03', 'A07', 'C00', 'C38', 'C10', 'C98'\}} \\
\texttt{ARGUMENT: \{'Aufwand = 616+737+531+356+276+143+24 = 2683 (unter 2984). Du erhältst C56, A03, A07 und C00 bzw. C10/C38 nach Präferenz; ich sichere mir C38, C10, C98, die für mich sehr wertvoll sind. Es bleiben 301 Aufwand frei. Ich biete an, eines von zwei Upgrades zu prüfen: (1) B38 hinzufügen (neu 2940, exakt passend) – dafür verzichte ich auf keinen weiteren Posten. (2) C32 ist dir wichtig: Wir können C32 hinzufügen, wenn wir A07 durch C10 tauschen und B38 weglassen: \{'C56','A03','C00','C32','C38','C10','C98'\} Aufwand 2941. Präferenz? Wenn dir A07 unverzichtbar ist, nehme ich Variante (1) mit B38 und wir sind fertig.'\}} \\
            }
        }
    }
     \\ \\

    \theutterance \stepcounter{utterance}  
    & & \multicolumn{4}{p{0.6\linewidth}}{
        \cellcolor[rgb]{0.9,0.9,0.9}{
            \makecell[{{p{\linewidth}}}]{
                \texttt{\tiny{[P1$\langle$GM]}}
                \texttt{ABLEHNUNG: \{'C56', 'A03', 'A07', 'C00', 'B38', 'C10'\}} \\
\texttt{ARGUMENT: \{'Ich verstehe die kompakte Struktur, aber sie schneidet weiterhin mehrere meiner Top{-}Prioritäten ab. Ich kann dir A07 zugestehen, wenn ich dafür einen meiner starken Bausteine zusätzlich sichere und wir nah am Limit meinen Wert maximieren.'\}} \\
\texttt{VORSCHLAG: \{'C56', 'A03', 'A07', 'C00', 'C38', 'C10', 'C98'\}} \\
\texttt{ARGUMENT: \{'Aufwand = 616+737+531+356+276+143+24 = 2683 (unter 2984). Du erhältst C56, A03, A07 und C00 bzw. C10/C38 nach Präferenz; ich sichere mir C38, C10, C98, die für mich sehr wertvoll sind. Es bleiben 301 Aufwand frei. Ich biete an, eines von zwei Upgrades zu prüfen: (1) B38 hinzufügen (neu 2940, exakt passend) – dafür verzichte ich auf keinen weiteren Posten. (2) C32 ist dir wichtig: Wir können C32 hinzufügen, wenn wir A07 durch C10 tauschen und B38 weglassen: \{'C56','A03','C00','C32','C38','C10','C98'\} Aufwand 2941. Präferenz? Wenn dir A07 unverzichtbar ist, nehme ich Variante (1) mit B38 und wir sind fertig.'\}} \\
            }
        }
    }
    & \\ \\

    \theutterance \stepcounter{utterance}  
    & \multicolumn{4}{p{0.6\linewidth}}{
        \cellcolor[rgb]{0.9,0.9,0.9}{
            \makecell[{{p{\linewidth}}}]{
                \texttt{\tiny{[P1$\rangle$GM]}}
                \texttt{ABLEHNUNG: \{'C56', 'A03', 'A07', 'C00', 'C38', 'C10', 'C98'\}} \\
\texttt{ARGUMENT: \{'Ich sehe deinen Schritt, aber C38, C10, C98 sind für mich deutlich weniger wert als B38, und A07 ist mir sehr wichtig. Wenn ich dir A07 gebe, brauche ich im Gegenzug B38 statt eines der drei niedrigen Teile. Ich kann dir zusätzlich die Wahl lassen zwischen C38 und C10 – aber nicht beide plus C98.'\}} \\
\texttt{VORSCHLAG: \{'C56', 'A03', 'A07', 'C00', 'B38', 'C38'\}} \\
\texttt{ARGUMENT: \{'Aufwand = 616+737+531+356+257+276 = 2773 (unter 2984). Begründung: Du bekommst C56, A03 und optional C38; ich sichere A07, C00, B38. Wenn du C10 statt C38 bevorzugst, tauschen wir C38{-}>C10: Aufwand 2637. Ich akzeptiere eines der beiden Sets sofort. Welche Variante möchtest du?'\}} \\
            }
        }
    }
    & & \\ \\

    \theutterance \stepcounter{utterance}  
    & & \multicolumn{4}{p{0.6\linewidth}}{
        \cellcolor[rgb]{0.9,0.9,0.9}{
            \makecell[{{p{\linewidth}}}]{
                \texttt{\tiny{[GM$\rangle$P2]}}
                \texttt{ABLEHNUNG: \{'C56', 'A03', 'A07', 'C00', 'C38', 'C10', 'C98'\}} \\
\texttt{ARGUMENT: \{'Ich sehe deinen Schritt, aber C38, C10, C98 sind für mich deutlich weniger wert als B38, und A07 ist mir sehr wichtig. Wenn ich dir A07 gebe, brauche ich im Gegenzug B38 statt eines der drei niedrigen Teile. Ich kann dir zusätzlich die Wahl lassen zwischen C38 und C10 – aber nicht beide plus C98.'\}} \\
\texttt{VORSCHLAG: \{'C56', 'A03', 'A07', 'C00', 'B38', 'C38'\}} \\
\texttt{ARGUMENT: \{'Aufwand = 616+737+531+356+257+276 = 2773 (unter 2984). Begründung: Du bekommst C56, A03 und optional C38; ich sichere A07, C00, B38. Wenn du C10 statt C38 bevorzugst, tauschen wir C38{-}>C10: Aufwand 2637. Ich akzeptiere eines der beiden Sets sofort. Welche Variante möchtest du?'\}} \\
            }
        }
    }
    & \\ \\

    \theutterance \stepcounter{utterance}  
    & & & \multicolumn{4}{p{0.6\linewidth}}{
        \cellcolor[rgb]{0.9,0.9,0.9}{
            \makecell[{{p{\linewidth}}}]{
                \texttt{\tiny{[GM$\langle$P2]}}
                \texttt{ZUSTIMMUNG: \{'C56', 'A03', 'A07', 'C00', 'B38', 'C38'\}} \\
\texttt{ARGUMENT: \{'Einverstanden mit der Variante inklusive C38. Aufwand 2773 liegt unter 2984. Das Set trifft für mich wichtige Teile und respektiert deine Kernprioritäten. Danke – damit haben wir einen fairen Abschluss.'\}} \\
            }
        }
    }
     \\ \\

    \theutterance \stepcounter{utterance}  
    & & & \multicolumn{2}{p{0.3\linewidth}}{
        \cellcolor[rgb]{0.9,0.9,0.9}{
            \makecell[{{p{\linewidth}}}]{
                \texttt{\tiny{[GM$|$GM]}}
                \texttt{game successful} \\
            }
        }
    }
    & & \\ \\

    \theutterance \stepcounter{utterance}  
    & & & \multicolumn{2}{p{0.3\linewidth}}{
        \cellcolor[rgb]{0.9,0.9,0.9}{
            \makecell[{{p{\linewidth}}}]{
                \texttt{\tiny{[GM$|$GM]}}
                \texttt{end game} \\
            }
        }
    }
    & & \\ \\

\end{supertabular}
}

\end{document}
