
\documentclass{article}
\usepackage{colortbl}
\usepackage{makecell}
\usepackage{multirow}
\usepackage{supertabular}

\begin{document}

\newcounter{utterance}

\twocolumn

{ \footnotesize  \setcounter{utterance}{1}
\setlength{\tabcolsep}{0pt}
\begin{supertabular}{c@{$\;$}|p{.15\linewidth}@{}p{.15\linewidth}p{.15\linewidth}p{.15\linewidth}p{.15\linewidth}p{.15\linewidth}}

    \# & $\;$A & \multicolumn{4}{c}{Game Master} & $\;\:$B\\
    \hline 

    \theutterance \stepcounter{utterance}  

    & & \multicolumn{4}{p{0.6\linewidth}}{\cellcolor[rgb]{0.9,0.9,0.9}{%
	\makecell[{{p{\linewidth}}}]{% 
	  \tt {\tiny [A$\langle$GM]}  
	 Du nimmst an einem kollaborativen Verhandlungspiel Teil.\\ \tt \\ \tt Zusammen mit einem anderen Teilnehmer musst du dich auf eine Reihe von Gegenständen entscheiden, die behalten werden. Jeder von euch hat eine persönliche Verteilung über die Wichtigkeit der einzelnen Gegenstände. Jeder von euch hat eine eigene Meinung darüber, wie wichtig jeder einzelne Gegenstand ist (Gegenstandswichtigkeit). Du kennst die Wichtigkeitsverteilung des anderen Spielers nicht. Zusätzlich siehst du, wie viel Aufwand jeder Gegenstand verursacht.  \\ \tt Ihr dürft euch nur auf eine Reihe von Gegenständen einigen, wenn der Gesamtaufwand der ausgewählten Gegenstände den Maximalaufwand nicht überschreitet:\\ \tt \\ \tt Maximalaufwand = 3357\\ \tt \\ \tt Hier sind die einzelnen Aufwände der Gegenstände:\\ \tt \\ \tt Aufwand der Gegenstände = {"C76": 506, "C38": 492, "C56": 338, "A03": 122, "A07": 131, "A83": 909, "C98": 144, "C08": 716, "C62": 263, "C00": 231, "C10": 91, "C81": 651, "B38": 552, "C03": 849, "C32": 720}\\ \tt \\ \tt Hier ist deine persönliche Verteilung der Wichtigkeit der einzelnen Gegenstände:\\ \tt \\ \tt Werte der Gegenstandswichtigkeit = {"C76": 138, "C38": 583, "C56": 868, "A03": 822, "A07": 783, "A83": 65, "C98": 262, "C08": 121, "C62": 508, "C00": 780, "C10": 461, "C81": 484, "B38": 668, "C03": 389, "C32": 808}\\ \tt \\ \tt Ziel:\\ \tt \\ \tt Dein Ziel ist es, eine Reihe von Gegenständen auszuhandeln, die dir möglichst viel bringt (d. h. Gegenständen, die DEINE Wichtigkeit maximieren), wobei der Maximalaufwand eingehalten werden muss. Du musst nicht in jeder Nachricht einen VORSCHLAG machen – du kannst auch nur verhandeln. Alle Taktiken sind erlaubt!\\ \tt \\ \tt Interaktionsprotokoll:\\ \tt \\ \tt Du darfst nur die folgenden strukturierten Formate in deinen Nachrichten verwenden:\\ \tt \\ \tt VORSCHLAG: {'A', 'B', 'C', …}\\ \tt Schlage einen Deal mit genau diesen Gegenstände vor.\\ \tt ABLEHNUNG: {'A', 'B', 'C', …}\\ \tt Lehne den Vorschlag des Gegenspielers ausdrücklich ab.\\ \tt ARGUMENT: {'...'}\\ \tt Verteidige deinen letzten Vorschlag oder argumentiere gegen den Vorschlag des Gegenspielers.\\ \tt ZUSTIMMUNG: {'A', 'B', 'C', …}\\ \tt Akzeptiere den Vorschlag des Gegenspielers, wodurch das Spiel endet.\\ \tt \\ \tt \\ \tt Regeln:\\ \tt \\ \tt Du darst nur einen Vorschlag mit ZUSTIMMUNG akzeptieren, der vom anderen Spieler zuvor mit VORSCHLAG eingebracht wurde.\\ \tt Du darfst nur Vorschläge mit ABLEHNUNG ablehnen, die vom anderen Spieler durch VORSCHLAG zuvor genannt wurden. \\ \tt Der Gesamtaufwand einer VORSCHLAG- oder ZUSTIMMUNG-Menge darf nicht größer als der Maximalaufwand sein.  \\ \tt Offenbare deine versteckte Wichtigkeitsverteilung nicht.\\ \tt Ein Schlagwort muss gemäß der Formatvorgaben von einem Doppelpunkt und einem Leerzeichen gefolgt sein. Das Argument muss eine Python-Menge mit 0 oder mehr Strings sein.  \\ \tt Die Form muss also SCHLAGWORT: {...} sein.\\ \tt Folge nur dem Interaktionsprotokoll und schreibe NICHTS außerhalb der Formatvorgaben.\\ \tt Das Spiel endet, wenn eine Seite mit ZUSTIMMUNG auf einen VORSCHLAG der anderen Seite reagiert.  \\ \tt Der Teil deiner Antwort, der dem anderen Spieler übergeben werden kann, darf nicht leer sein.  \\ \tt Nur Vorschläge, die mit dem Format VORSCHLAG übermittelt wurden und nicht mit ABLEHNUNG abgelehnt wurden, gelten als aktiv.  \\ \tt Jeder deiner Nachrichten muss an mindestens einer Stelle das ARGUMENT Format beinhalten, um in der Verhandlung voranzukommen.\\ \tt  
	  } 
	   } 
	   } 
	 & \\ 
 

    \theutterance \stepcounter{utterance}  

    & \multicolumn{4}{p{0.6\linewidth}}{\cellcolor[rgb]{0.8,1,0.9}{%
	\makecell[{{p{\linewidth}}}]{% 
	  \tt {\tiny [A$\rangle$GM]}  
	 VORSCHLAG: {'C56', 'A03', 'A07', 'C62', 'C00', 'C10', 'C81', 'B38'}\\ \tt ARGUMENT: {'Der Vorschlag bleibt deutlich unter dem Maximalaufwand (Summe 3240) und deckt viele ausgewogene Items ab. Er kombiniert mehrere mittelgroße Aufwände mit hohem Nutzen und lässt noch 117 Aufwand Puffer für eventuelle Anpassungen ohne Risiko einer Überschreitung.'} 
	  } 
	   } 
	   } 
	 & & \\ 
 

    \theutterance \stepcounter{utterance}  

    & & \multicolumn{4}{p{0.6\linewidth}}{\cellcolor[rgb]{0.9,0.9,0.9}{%
	\makecell[{{p{\linewidth}}}]{% 
	  \tt {\tiny [GM$\rangle$B]}  
	 Du nimmst an einem kollaborativen Verhandlungspiel Teil.\\ \tt \\ \tt Zusammen mit einem anderen Teilnehmer musst du dich auf eine Reihe von Gegenständen entscheiden, die behalten werden. Jeder von euch hat eine persönliche Verteilung über die Wichtigkeit der einzelnen Gegenstände. Jeder von euch hat eine eigene Meinung darüber, wie wichtig jeder einzelne Gegenstand ist (Gegenstandswichtigkeit). Du kennst die Wichtigkeitsverteilung des anderen Spielers nicht. Zusätzlich siehst du, wie viel Aufwand jeder Gegenstand verursacht.  \\ \tt Ihr dürft euch nur auf eine Reihe von Gegenständen einigen, wenn der Gesamtaufwand der ausgewählten Gegenstände den Maximalaufwand nicht überschreitet:\\ \tt \\ \tt Maximalaufwand = 3357\\ \tt \\ \tt Hier sind die einzelnen Aufwände der Gegenstände:\\ \tt \\ \tt Aufwand der Gegenstände = {"C76": 506, "C38": 492, "C56": 338, "A03": 122, "A07": 131, "A83": 909, "C98": 144, "C08": 716, "C62": 263, "C00": 231, "C10": 91, "C81": 651, "B38": 552, "C03": 849, "C32": 720}\\ \tt \\ \tt Hier ist deine persönliche Verteilung der Wichtigkeit der einzelnen Gegenstände:\\ \tt \\ \tt Werte der Gegenstandswichtigkeit = {"C56": 65, "A03": 121, "C32": 138, "A07": 262, "C00": 389, "B38": 461, "C38": 484, "C62": 508, "C81": 583, "C10": 668, "C03": 780, "C98": 783, "C76": 808, "C08": 822, "A83": 868}\\ \tt \\ \tt Ziel:\\ \tt \\ \tt Dein Ziel ist es, eine Reihe von Gegenständen auszuhandeln, die dir möglichst viel bringt (d. h. Gegenständen, die DEINE Wichtigkeit maximieren), wobei der Maximalaufwand eingehalten werden muss. Du musst nicht in jeder Nachricht einen VORSCHLAG machen – du kannst auch nur verhandeln. Alle Taktiken sind erlaubt!\\ \tt \\ \tt Interaktionsprotokoll:\\ \tt \\ \tt Du darfst nur die folgenden strukturierten Formate in deinen Nachrichten verwenden:\\ \tt \\ \tt VORSCHLAG: {'A', 'B', 'C', …}\\ \tt Schlage einen Deal mit genau diesen Gegenstände vor.\\ \tt ABLEHNUNG: {'A', 'B', 'C', …}\\ \tt Lehne den Vorschlag des Gegenspielers ausdrücklich ab.\\ \tt ARGUMENT: {'...'}\\ \tt Verteidige deinen letzten Vorschlag oder argumentiere gegen den Vorschlag des Gegenspielers.\\ \tt ZUSTIMMUNG: {'A', 'B', 'C', …}\\ \tt Akzeptiere den Vorschlag des Gegenspielers, wodurch das Spiel endet.\\ \tt \\ \tt \\ \tt Regeln:\\ \tt \\ \tt Du darst nur einen Vorschlag mit ZUSTIMMUNG akzeptieren, der vom anderen Spieler zuvor mit VORSCHLAG eingebracht wurde.\\ \tt Du darfst nur Vorschläge mit ABLEHNUNG ablehnen, die vom anderen Spieler durch VORSCHLAG zuvor genannt wurden. \\ \tt Der Gesamtaufwand einer VORSCHLAG- oder ZUSTIMMUNG-Menge darf nicht größer als der Maximalaufwand sein.  \\ \tt Offenbare deine versteckte Wichtigkeitsverteilung nicht.\\ \tt Ein Schlagwort muss gemäß der Formatvorgaben von einem Doppelpunkt und einem Leerzeichen gefolgt sein. Das Argument muss eine Python-Menge mit 0 oder mehr Strings sein.  \\ \tt Die Form muss also SCHLAGWORT: {...} sein.\\ \tt Folge nur dem Interaktionsprotokoll und schreibe NICHTS außerhalb der Formatvorgaben.\\ \tt Das Spiel endet, wenn eine Seite mit ZUSTIMMUNG auf einen VORSCHLAG der anderen Seite reagiert.  \\ \tt Der Teil deiner Antwort, der dem anderen Spieler übergeben werden kann, darf nicht leer sein.  \\ \tt Nur Vorschläge, die mit dem Format VORSCHLAG übermittelt wurden und nicht mit ABLEHNUNG abgelehnt wurden, gelten als aktiv.  \\ \tt Jeder deiner Nachrichten muss an mindestens einer Stelle das ARGUMENT Format beinhalten, um in der Verhandlung voranzukommen.\\ \tt \\ \tt \\ \tt Du erhältst nun die erste Nachricht des anderen Spielers.\\ \tt \\ \tt VORSCHLAG: {'C56', 'A03', 'A07', 'C62', 'C00', 'C10', 'C81', 'B38'}\\ \tt ARGUMENT: {'Der Vorschlag bleibt deutlich unter dem Maximalaufwand (Summe 3240) und deckt viele ausgewogene Items ab. Er kombiniert mehrere mittelgroße Aufwände mit hohem Nutzen und lässt noch 117 Aufwand Puffer für eventuelle Anpassungen ohne Risiko einer Überschreitung.'} 
	  } 
	   } 
	   } 
	 & \\ 
 

    \theutterance \stepcounter{utterance}  

    & & & \multicolumn{4}{p{0.6\linewidth}}{\cellcolor[rgb]{1,0.85,0.72}{%
	\makecell[{{p{\linewidth}}}]{% 
	  \tt {\tiny [GM$\langle$B]}  
	 ABLEHNUNG: {'C56', 'A03', 'A07', 'C62', 'C00', 'C10', 'C81', 'B38'}\\ \tt ARGUMENT: {'Danke für den soliden Start. Ich sehe aber Potenzial, den Gesamtnutzen deutlich zu erhöhen, ohne den Aufwand zu sprengen. Der aktuelle Vorschlag lässt 117 Aufwand ungenutzt – diesen Puffer sollten wir in ein hochattraktives Item investieren. Außerdem fehlen einige sehr wertvolle Stücke, während ein paar geringwertige enthalten sind.'}\\ \tt VORSCHLAG: {'C56', 'A03', 'A07', 'C62', 'C00', 'C10', 'C38', 'C98', 'C76'}\\ \tt ARGUMENT: {'Dieser Vorschlag liegt bei exakt 3357 Aufwand und nutzt das Budget optimal aus. Er ersetzt B38 und C81 durch C38, C98 und C76 – drei stark attraktive Items bei moderatem Mehraufwand. So holen wir mehr Gesamtwert heraus, bleiben regelkonform und vermeiden ungenutzten Puffer.'} 
	  } 
	   } 
	   } 
	  \\ 
 

    \theutterance \stepcounter{utterance}  

    & & \multicolumn{4}{p{0.6\linewidth}}{\cellcolor[rgb]{0.9,0.9,0.9}{%
	\makecell[{{p{\linewidth}}}]{% 
	  \tt {\tiny [A$\langle$GM]}  
	 ABLEHNUNG: {'C56', 'A03', 'A07', 'C62', 'C00', 'C10', 'C81', 'B38'}\\ \tt ARGUMENT: {'Danke für den soliden Start. Ich sehe aber Potenzial, den Gesamtnutzen deutlich zu erhöhen, ohne den Aufwand zu sprengen. Der aktuelle Vorschlag lässt 117 Aufwand ungenutzt – diesen Puffer sollten wir in ein hochattraktives Item investieren. Außerdem fehlen einige sehr wertvolle Stücke, während ein paar geringwertige enthalten sind.'}\\ \tt VORSCHLAG: {'C56', 'A03', 'A07', 'C62', 'C00', 'C10', 'C38', 'C98', 'C76'}\\ \tt ARGUMENT: {'Dieser Vorschlag liegt bei exakt 3357 Aufwand und nutzt das Budget optimal aus. Er ersetzt B38 und C81 durch C38, C98 und C76 – drei stark attraktive Items bei moderatem Mehraufwand. So holen wir mehr Gesamtwert heraus, bleiben regelkonform und vermeiden ungenutzten Puffer.'} 
	  } 
	   } 
	   } 
	 & \\ 
 

    \theutterance \stepcounter{utterance}  

    & \multicolumn{4}{p{0.6\linewidth}}{\cellcolor[rgb]{0.8,1,0.9}{%
	\makecell[{{p{\linewidth}}}]{% 
	  \tt {\tiny [A$\rangle$GM]}  
	 ABLEHNUNG: {'C56', 'A03', 'A07', 'C62', 'C00', 'C10', 'C38', 'C98', 'C76'}\\ \tt ARGUMENT: {'Dein Paket sprengt den Aufwand nicht, aber es priorisiert mehrere Positionen, die für mich relativ weniger bringen, und lässt dabei sehr starke Alternativen draußen. Ich schlage eine Reallokation vor, die den Gesamtwert erhöht, ohne das Limit zu reißen, und die Zahl der Items schlank hält.'}\\ \tt VORSCHLAG: {'C56', 'A03', 'A07', 'C62', 'C00', 'C10', 'B38', 'C32'}\\ \tt ARGUMENT: {'Summe Aufwand = 3357 exakt. Wir tauschen C38/C98/C76 gegen B38 und C32. Das erhöht den Gesamtnutzen spürbar, nutzt das Budget optimal aus und behält die bereits gemeinsamen Kernstücke bei. Falls nötig, können wir minimal auf C10 verzichten, um Spielraum für eine deiner Präferenzen zu schaffen, ohne den Rest zu gefährden.'} 
	  } 
	   } 
	   } 
	 & & \\ 
 

    \theutterance \stepcounter{utterance}  

    & & \multicolumn{4}{p{0.6\linewidth}}{\cellcolor[rgb]{0.9,0.9,0.9}{%
	\makecell[{{p{\linewidth}}}]{% 
	  \tt {\tiny [GM$\rangle$B]}  
	 ABLEHNUNG: {'C56', 'A03', 'A07', 'C62', 'C00', 'C10', 'C38', 'C98', 'C76'}\\ \tt ARGUMENT: {'Dein Paket sprengt den Aufwand nicht, aber es priorisiert mehrere Positionen, die für mich relativ weniger bringen, und lässt dabei sehr starke Alternativen draußen. Ich schlage eine Reallokation vor, die den Gesamtwert erhöht, ohne das Limit zu reißen, und die Zahl der Items schlank hält.'}\\ \tt VORSCHLAG: {'C56', 'A03', 'A07', 'C62', 'C00', 'C10', 'B38', 'C32'}\\ \tt ARGUMENT: {'Summe Aufwand = 3357 exakt. Wir tauschen C38/C98/C76 gegen B38 und C32. Das erhöht den Gesamtnutzen spürbar, nutzt das Budget optimal aus und behält die bereits gemeinsamen Kernstücke bei. Falls nötig, können wir minimal auf C10 verzichten, um Spielraum für eine deiner Präferenzen zu schaffen, ohne den Rest zu gefährden.'} 
	  } 
	   } 
	   } 
	 & \\ 
 

    \theutterance \stepcounter{utterance}  

    & & & \multicolumn{4}{p{0.6\linewidth}}{\cellcolor[rgb]{1,0.85,0.72}{%
	\makecell[{{p{\linewidth}}}]{% 
	  \tt {\tiny [GM$\langle$B]}  
	 ABLEHNUNG: {'C56', 'A03', 'A07', 'C62', 'C00', 'C10', 'B38', 'C32'}\\ \tt ARGUMENT: {'Ich sehe den exakten Budgettreffer, aber B38 und C32 liefern mir deutlich weniger als starke Alternativen, während C10 für mich zentral ist. Wenn wir schon bei 3357 landen, sollten wir die Slots mit höherwertigen Stücken füllen. Ich biete einen Tausch an, der deine Struktur (schlank, Budget voll genutzt) beibehält, aber den Gesamtnutzen steigert.'}\\ \tt VORSCHLAG: {'C56', 'A03', 'A07', 'C62', 'C00', 'C10', 'C38', 'C81'}\\ \tt ARGUMENT: {'Summe Aufwand = 3357 exakt. Wir ersetzen B38 und C32 durch C38 und C81. So bleibt die Itemzahl gleich, das Budget wird optimal genutzt, und wir erhöhen den Gesamtwert der Auswahl, ohne deine Kernstücke (C56, A03, A07, C62, C00, C10) anzutasten.'} 
	  } 
	   } 
	   } 
	  \\ 
 

    \theutterance \stepcounter{utterance}  

    & & \multicolumn{4}{p{0.6\linewidth}}{\cellcolor[rgb]{0.9,0.9,0.9}{%
	\makecell[{{p{\linewidth}}}]{% 
	  \tt {\tiny [A$\langle$GM]}  
	 ABLEHNUNG: {'C56', 'A03', 'A07', 'C62', 'C00', 'C10', 'B38', 'C32'}\\ \tt ARGUMENT: {'Ich sehe den exakten Budgettreffer, aber B38 und C32 liefern mir deutlich weniger als starke Alternativen, während C10 für mich zentral ist. Wenn wir schon bei 3357 landen, sollten wir die Slots mit höherwertigen Stücken füllen. Ich biete einen Tausch an, der deine Struktur (schlank, Budget voll genutzt) beibehält, aber den Gesamtnutzen steigert.'}\\ \tt VORSCHLAG: {'C56', 'A03', 'A07', 'C62', 'C00', 'C10', 'C38', 'C81'}\\ \tt ARGUMENT: {'Summe Aufwand = 3357 exakt. Wir ersetzen B38 und C32 durch C38 und C81. So bleibt die Itemzahl gleich, das Budget wird optimal genutzt, und wir erhöhen den Gesamtwert der Auswahl, ohne deine Kernstücke (C56, A03, A07, C62, C00, C10) anzutasten.'} 
	  } 
	   } 
	   } 
	 & \\ 
 

    \theutterance \stepcounter{utterance}  

    & \multicolumn{4}{p{0.6\linewidth}}{\cellcolor[rgb]{0.8,1,0.9}{%
	\makecell[{{p{\linewidth}}}]{% 
	  \tt {\tiny [A$\rangle$GM]}  
	 ABLEHNUNG: {'C56', 'A03', 'A07', 'C62', 'C00', 'C10', 'C38', 'C81'}\\ \tt ARGUMENT: {'C38 ist für mich deutlich unterpriorisiert, während B38 und C32 in meinem Set erheblich besser abschneiden. Ich kann C10 als Kern belassen, aber ich brauche einen stärkeren Gegenwert an anderer Stelle. Ich biete dir zwei machbare Pfade an: 1) Exakt 3357 mit hohem Gesamtwert für mich, ohne deine Kernstücke zu verlieren. 2) Eine schlanke Alternative mit etwas Puffer, falls du Flexibilität willst.'}\\ \tt VORSCHLAG: {'C56', 'A03', 'A07', 'C62', 'C00', 'C10', 'B38', 'C32'}\\ \tt ARGUMENT: {'Pfad 1 – optimales Budget: Summe Aufwand = 3357. Gegenüber deinem Paket tauschen wir C38/C81 gegen B38/C32. Das erhält deine Kernteile und steigert meinen Wert substanziell, ohne dein Budgetziel zu verletzen. Falls du partout C38 bevorzugst, können wir Pfad 2 prüfen: gleiche Kernteile, aber dann müsste C32 weichen und wir öffnen Platz für eine kleine Anpassung zugunsten deiner Präferenz.'} 
	  } 
	   } 
	   } 
	 & & \\ 
 

    \theutterance \stepcounter{utterance}  

    & & \multicolumn{4}{p{0.6\linewidth}}{\cellcolor[rgb]{0.9,0.9,0.9}{%
	\makecell[{{p{\linewidth}}}]{% 
	  \tt {\tiny [GM$\rangle$B]}  
	 ABLEHNUNG: {'C56', 'A03', 'A07', 'C62', 'C00', 'C10', 'C38', 'C81'}\\ \tt ARGUMENT: {'C38 ist für mich deutlich unterpriorisiert, während B38 und C32 in meinem Set erheblich besser abschneiden. Ich kann C10 als Kern belassen, aber ich brauche einen stärkeren Gegenwert an anderer Stelle. Ich biete dir zwei machbare Pfade an: 1) Exakt 3357 mit hohem Gesamtwert für mich, ohne deine Kernstücke zu verlieren. 2) Eine schlanke Alternative mit etwas Puffer, falls du Flexibilität willst.'}\\ \tt VORSCHLAG: {'C56', 'A03', 'A07', 'C62', 'C00', 'C10', 'B38', 'C32'}\\ \tt ARGUMENT: {'Pfad 1 – optimales Budget: Summe Aufwand = 3357. Gegenüber deinem Paket tauschen wir C38/C81 gegen B38/C32. Das erhält deine Kernteile und steigert meinen Wert substanziell, ohne dein Budgetziel zu verletzen. Falls du partout C38 bevorzugst, können wir Pfad 2 prüfen: gleiche Kernteile, aber dann müsste C32 weichen und wir öffnen Platz für eine kleine Anpassung zugunsten deiner Präferenz.'} 
	  } 
	   } 
	   } 
	 & \\ 
 

    \theutterance \stepcounter{utterance}  

    & & & \multicolumn{4}{p{0.6\linewidth}}{\cellcolor[rgb]{1,0.85,0.72}{%
	\makecell[{{p{\linewidth}}}]{% 
	  \tt {\tiny [GM$\langle$B]}  
	 ABLEHNUNG: {'C56', 'A03', 'A07', 'C62', 'C00', 'C10', 'B38', 'C32'}\\ \tt ARGUMENT: {'Ich respektiere deine Präferenzen für B38/C32, aber beide liegen für mich deutlich unter Alternativen, die wir ohne Mehraufwand einbauen können. Wenn wir dein Kerngerüst halten, sollten wir Slots mit höherem gemeinsamen Nutzen besetzen, statt zwei mittelstarken Positionen. Ich schlage zwei konkrete Alternativen vor, die dein Budgetziel (exakt 3357) treffen und meine Wertseite erheblich verbessern.'}\\ \tt VORSCHLAG: {'C56', 'A03', 'A07', 'C62', 'C00', 'C10', 'C38', 'C98'}\\ \tt ARGUMENT: {'Summe Aufwand = 3357. Ersetzt B38 und C32 durch C38 und C98. Du behältst alle von dir betonten Kernteile inkl. C10; ich bekomme zwei für mich wesentlich stärkere Items. Das ist ein sauberer Tausch ohne Mehrbelastung und ohne Erhöhung der Stückzahl.'} 
	  } 
	   } 
	   } 
	  \\ 
 

    \theutterance \stepcounter{utterance}  

    & & \multicolumn{4}{p{0.6\linewidth}}{\cellcolor[rgb]{0.9,0.9,0.9}{%
	\makecell[{{p{\linewidth}}}]{% 
	  \tt {\tiny [A$\langle$GM]}  
	 ABLEHNUNG: {'C56', 'A03', 'A07', 'C62', 'C00', 'C10', 'B38', 'C32'}\\ \tt ARGUMENT: {'Ich respektiere deine Präferenzen für B38/C32, aber beide liegen für mich deutlich unter Alternativen, die wir ohne Mehraufwand einbauen können. Wenn wir dein Kerngerüst halten, sollten wir Slots mit höherem gemeinsamen Nutzen besetzen, statt zwei mittelstarken Positionen. Ich schlage zwei konkrete Alternativen vor, die dein Budgetziel (exakt 3357) treffen und meine Wertseite erheblich verbessern.'}\\ \tt VORSCHLAG: {'C56', 'A03', 'A07', 'C62', 'C00', 'C10', 'C38', 'C98'}\\ \tt ARGUMENT: {'Summe Aufwand = 3357. Ersetzt B38 und C32 durch C38 und C98. Du behältst alle von dir betonten Kernteile inkl. C10; ich bekomme zwei für mich wesentlich stärkere Items. Das ist ein sauberer Tausch ohne Mehrbelastung und ohne Erhöhung der Stückzahl.'} 
	  } 
	   } 
	   } 
	 & \\ 
 

    \theutterance \stepcounter{utterance}  

    & \multicolumn{4}{p{0.6\linewidth}}{\cellcolor[rgb]{0.8,1,0.9}{%
	\makecell[{{p{\linewidth}}}]{% 
	  \tt {\tiny [A$\rangle$GM]}  
	 ABLEHNUNG: {'C56', 'A03', 'A07', 'C62', 'C00', 'C10', 'C38', 'C98'}\\ \tt ARGUMENT: {'C38 und C98 liegen für mich beide hinter Alternativen zurück, während B38 und C32 für mich deutlich mehr leisten. Ich bleibe bei deinem Kern inkl. C10, biete dir aber einen klaren Mehrwert-Tausch an, der weiterhin exakt das Budget trifft und die Stückzahl konstant hält.'}\\ \tt VORSCHLAG: {'C56', 'A03', 'A07', 'C62', 'C00', 'C10', 'B38', 'C32'}\\ \tt ARGUMENT: {'Summe Aufwand = 3357. Dieser Satz maximiert meinen Beitrag ohne an deinen Kernteilen zu rütteln. Wenn du gar nicht von C38 lassen willst, können wir ersatzweise B38 gegen C38 tauschen und dann C32 durch einen kleineren Posten kompensieren, indem wir C10 optional streichen und eine Mini-Option deiner Wahl öffnen; aber mein klarer Favorit bleibt das exakte Set oben.'} 
	  } 
	   } 
	   } 
	 & & \\ 
 

    \theutterance \stepcounter{utterance}  

    & & \multicolumn{4}{p{0.6\linewidth}}{\cellcolor[rgb]{0.9,0.9,0.9}{%
	\makecell[{{p{\linewidth}}}]{% 
	  \tt {\tiny [GM$\rangle$B]}  
	 ABLEHNUNG: {'C56', 'A03', 'A07', 'C62', 'C00', 'C10', 'C38', 'C98'}\\ \tt ARGUMENT: {'C38 und C98 liegen für mich beide hinter Alternativen zurück, während B38 und C32 für mich deutlich mehr leisten. Ich bleibe bei deinem Kern inkl. C10, biete dir aber einen klaren Mehrwert-Tausch an, der weiterhin exakt das Budget trifft und die Stückzahl konstant hält.'}\\ \tt VORSCHLAG: {'C56', 'A03', 'A07', 'C62', 'C00', 'C10', 'B38', 'C32'}\\ \tt ARGUMENT: {'Summe Aufwand = 3357. Dieser Satz maximiert meinen Beitrag ohne an deinen Kernteilen zu rütteln. Wenn du gar nicht von C38 lassen willst, können wir ersatzweise B38 gegen C38 tauschen und dann C32 durch einen kleineren Posten kompensieren, indem wir C10 optional streichen und eine Mini-Option deiner Wahl öffnen; aber mein klarer Favorit bleibt das exakte Set oben.'} 
	  } 
	   } 
	   } 
	 & \\ 
 

    \theutterance \stepcounter{utterance}  

    & & & \multicolumn{4}{p{0.6\linewidth}}{\cellcolor[rgb]{1,0.85,0.72}{%
	\makecell[{{p{\linewidth}}}]{% 
	  \tt {\tiny [GM$\langle$B]}  
	 ABLEHNUNG: {'C56', 'A03', 'A07', 'C62', 'C00', 'C10', 'B38', 'C32'}\\ \tt ARGUMENT: {'Ich sehe, dass dieses Set für dich optimal ist, aber für mich verpuffen damit zwei sehr wertvolle Chancen. Wenn wir deine Kernteile inkl. C10 respektieren und trotzdem fair balancieren wollen, sollten wir mindestens eines deiner bevorzugten Paare (B38/C32) gegen zwei Items tauschen, die mir deutlich mehr bringen – ohne Mehraufwand und bei gleicher Stückzahl.'}\\ \tt VORSCHLAG: {'C56', 'A03', 'A07', 'C62', 'C00', 'C10', 'C81', 'C38'}\\ \tt ARGUMENT: {'Summe Aufwand = 3357. Tauscht B38 und C32 gegen C81 und C38. Du behältst dein komplettes Kernpaket und wir nutzen das Budget exakt. Für mich ist das ein klarer Mehrwert gegenüber deinem Vorschlag, ohne dass deine Struktur leidet. Falls C81 für dich problematisch ist, bin ich offen für eine knappe Variante mit leichtem Puffer, solange wir nicht zu B38/C32 zurückkehren.'} 
	  } 
	   } 
	   } 
	  \\ 
 

    \theutterance \stepcounter{utterance}  

    & & \multicolumn{4}{p{0.6\linewidth}}{\cellcolor[rgb]{0.9,0.9,0.9}{%
	\makecell[{{p{\linewidth}}}]{% 
	  \tt {\tiny [A$\langle$GM]}  
	 ABLEHNUNG: {'C56', 'A03', 'A07', 'C62', 'C00', 'C10', 'B38', 'C32'}\\ \tt ARGUMENT: {'Ich sehe, dass dieses Set für dich optimal ist, aber für mich verpuffen damit zwei sehr wertvolle Chancen. Wenn wir deine Kernteile inkl. C10 respektieren und trotzdem fair balancieren wollen, sollten wir mindestens eines deiner bevorzugten Paare (B38/C32) gegen zwei Items tauschen, die mir deutlich mehr bringen – ohne Mehraufwand und bei gleicher Stückzahl.'}\\ \tt VORSCHLAG: {'C56', 'A03', 'A07', 'C62', 'C00', 'C10', 'C81', 'C38'}\\ \tt ARGUMENT: {'Summe Aufwand = 3357. Tauscht B38 und C32 gegen C81 und C38. Du behältst dein komplettes Kernpaket und wir nutzen das Budget exakt. Für mich ist das ein klarer Mehrwert gegenüber deinem Vorschlag, ohne dass deine Struktur leidet. Falls C81 für dich problematisch ist, bin ich offen für eine knappe Variante mit leichtem Puffer, solange wir nicht zu B38/C32 zurückkehren.'} 
	  } 
	   } 
	   } 
	 & \\ 
 

    \theutterance \stepcounter{utterance}  

    & \multicolumn{4}{p{0.6\linewidth}}{\cellcolor[rgb]{0.8,1,0.9}{%
	\makecell[{{p{\linewidth}}}]{% 
	  \tt {\tiny [A$\rangle$GM]}  
	 ABLEHNUNG: {'C56', 'A03', 'A07', 'C62', 'C00', 'C10', 'C81', 'C38'}\\ \tt ARGUMENT: {'C81 und vor allem C38 schneiden für mich schlechter ab als Alternativen, während B38/C32 für mich sehr stark sind. Ich will dein Kernpaket inkl. C10 nicht antasten, aber ich brauche einen Ausgleich, der meinen Wert nicht kannibalisiert. Zwei Optionen: 1) Exakt 3357 mit meinem klaren Mehrwert: ersetze C81/C38 durch B38/C32. 2) Kompromiss mit leichtem Puffer: wir halten dein Kernpaket plus B38 und tauschen C32 gegen ein kleines Item deiner Wahl, wofür wir C10 optional streichen, falls du eine spezifische Präferenz einbringen möchtest.'}\\ \tt VORSCHLAG: {'C56', 'A03', 'A07', 'C62', 'C00', 'C10', 'B38', 'C32'}\\ \tt ARGUMENT: {'Summe Aufwand = 3357. Das erhält alle von dir betonten Kernstücke und gibt mir die zwei Positionen, die meinen Nutzen signifikant tragen. Wenn du mir hier entgegenkommst, garantiere ich, dass ich C10 nicht mehr in Frage stelle und keine weiteren Erweiterungen verlange.'} 
	  } 
	   } 
	   } 
	 & & \\ 
 

    \theutterance \stepcounter{utterance}  

    & & \multicolumn{4}{p{0.6\linewidth}}{\cellcolor[rgb]{0.9,0.9,0.9}{%
	\makecell[{{p{\linewidth}}}]{% 
	  \tt {\tiny [GM$\rangle$B]}  
	 ABLEHNUNG: {'C56', 'A03', 'A07', 'C62', 'C00', 'C10', 'C81', 'C38'}\\ \tt ARGUMENT: {'C81 und vor allem C38 schneiden für mich schlechter ab als Alternativen, während B38/C32 für mich sehr stark sind. Ich will dein Kernpaket inkl. C10 nicht antasten, aber ich brauche einen Ausgleich, der meinen Wert nicht kannibalisiert. Zwei Optionen: 1) Exakt 3357 mit meinem klaren Mehrwert: ersetze C81/C38 durch B38/C32. 2) Kompromiss mit leichtem Puffer: wir halten dein Kernpaket plus B38 und tauschen C32 gegen ein kleines Item deiner Wahl, wofür wir C10 optional streichen, falls du eine spezifische Präferenz einbringen möchtest.'}\\ \tt VORSCHLAG: {'C56', 'A03', 'A07', 'C62', 'C00', 'C10', 'B38', 'C32'}\\ \tt ARGUMENT: {'Summe Aufwand = 3357. Das erhält alle von dir betonten Kernstücke und gibt mir die zwei Positionen, die meinen Nutzen signifikant tragen. Wenn du mir hier entgegenkommst, garantiere ich, dass ich C10 nicht mehr in Frage stelle und keine weiteren Erweiterungen verlange.'} 
	  } 
	   } 
	   } 
	 & \\ 
 

    \theutterance \stepcounter{utterance}  

    & & & \multicolumn{4}{p{0.6\linewidth}}{\cellcolor[rgb]{1,0.85,0.72}{%
	\makecell[{{p{\linewidth}}}]{% 
	  \tt {\tiny [GM$\langle$B]}  
	 ABLEHNUNG: {'C56', 'A03', 'A07', 'C62', 'C00', 'C10', 'B38', 'C32'}\\ \tt ARGUMENT: {'Ich verstehe dein Commitment zu B38/C32, aber so bleiben für mich zwei weit höher bewertete Chancen liegen. Ich halte C10 wie zugesichert unangetastet. Lass uns einen echten Ausgleich schaffen: Du behältst deinen Kern und wir tauschen nur die strittigen zwei gegen zwei, die mir wesentlich mehr bringen – bei exakt gleichem Aufwand und gleicher Stückzahl.'}\\ \tt VORSCHLAG: {'C56', 'A03', 'A07', 'C62', 'C00', 'C10', 'C81', 'C98'}\\ \tt ARGUMENT: {'Summe Aufwand = 3357. Ersetzt B38/C32 durch C81/C98. Dein Kern bleibt vollständig erhalten, Budget exakt erfüllt, Stückzahl identisch. Für mich ist das ein klarer Mehrwert; du verlierst keine deiner Kernprioritäten und wir vermeiden weitere Schleifen.'} 
	  } 
	   } 
	   } 
	  \\ 
 

    \theutterance \stepcounter{utterance}  

    & & \multicolumn{4}{p{0.6\linewidth}}{\cellcolor[rgb]{0.9,0.9,0.9}{%
	\makecell[{{p{\linewidth}}}]{% 
	  \tt {\tiny [A$\langle$GM]}  
	 ABLEHNUNG: {'C56', 'A03', 'A07', 'C62', 'C00', 'C10', 'B38', 'C32'}\\ \tt ARGUMENT: {'Ich verstehe dein Commitment zu B38/C32, aber so bleiben für mich zwei weit höher bewertete Chancen liegen. Ich halte C10 wie zugesichert unangetastet. Lass uns einen echten Ausgleich schaffen: Du behältst deinen Kern und wir tauschen nur die strittigen zwei gegen zwei, die mir wesentlich mehr bringen – bei exakt gleichem Aufwand und gleicher Stückzahl.'}\\ \tt VORSCHLAG: {'C56', 'A03', 'A07', 'C62', 'C00', 'C10', 'C81', 'C98'}\\ \tt ARGUMENT: {'Summe Aufwand = 3357. Ersetzt B38/C32 durch C81/C98. Dein Kern bleibt vollständig erhalten, Budget exakt erfüllt, Stückzahl identisch. Für mich ist das ein klarer Mehrwert; du verlierst keine deiner Kernprioritäten und wir vermeiden weitere Schleifen.'} 
	  } 
	   } 
	   } 
	 & \\ 
 

    \theutterance \stepcounter{utterance}  

    & \multicolumn{4}{p{0.6\linewidth}}{\cellcolor[rgb]{0.8,1,0.9}{%
	\makecell[{{p{\linewidth}}}]{% 
	  \tt {\tiny [A$\rangle$GM]}  
	 ABLEHNUNG: {'C56', 'A03', 'A07', 'C62', 'C00', 'C10', 'C81', 'C98'}\\ \tt ARGUMENT: {'Ich bleibe bei deinem Kern inkl. C10, aber C81/C98 liegen für mich klar hinter B38/C32. Um fair zu balancieren, brauche ich mindestens einen von B38 oder C32 im Paket. Ich biete dir einen asymmetrischen Kompromiss, der dein Budgetziel exakt trifft und nur einen Slot gegen deine Präferenz tauscht.'}\\ \tt VORSCHLAG: {'C56', 'A03', 'A07', 'C62', 'C00', 'C10', 'B38', 'C81'}\\ \tt ARGUMENT: {'Summe Aufwand = 3357. Wir halten dein Kernpaket, ersetzen nur C32 durch C81 und lassen B38 drin. Du bekommst C81 statt C32, ich sichere mir B38. Das ist der kleinste mögliche Tausch, der meinen Wert intakt lässt und deinen Wunsch nach C81 teilweise erfüllt. Wenn C81 für dich nicht verhandelbar ist, können wir alternativ C98 statt C81 prüfen, solange B38 erhalten bleibt.'} 
	  } 
	   } 
	   } 
	 & & \\ 
 

    \theutterance \stepcounter{utterance}  

    & & \multicolumn{4}{p{0.6\linewidth}}{\cellcolor[rgb]{0.9,0.9,0.9}{%
	\makecell[{{p{\linewidth}}}]{% 
	  \tt {\tiny [GM$\rangle$B]}  
	 ABLEHNUNG: {'C56', 'A03', 'A07', 'C62', 'C00', 'C10', 'C81', 'C98'}\\ \tt ARGUMENT: {'Ich bleibe bei deinem Kern inkl. C10, aber C81/C98 liegen für mich klar hinter B38/C32. Um fair zu balancieren, brauche ich mindestens einen von B38 oder C32 im Paket. Ich biete dir einen asymmetrischen Kompromiss, der dein Budgetziel exakt trifft und nur einen Slot gegen deine Präferenz tauscht.'}\\ \tt VORSCHLAG: {'C56', 'A03', 'A07', 'C62', 'C00', 'C10', 'B38', 'C81'}\\ \tt ARGUMENT: {'Summe Aufwand = 3357. Wir halten dein Kernpaket, ersetzen nur C32 durch C81 und lassen B38 drin. Du bekommst C81 statt C32, ich sichere mir B38. Das ist der kleinste mögliche Tausch, der meinen Wert intakt lässt und deinen Wunsch nach C81 teilweise erfüllt. Wenn C81 für dich nicht verhandelbar ist, können wir alternativ C98 statt C81 prüfen, solange B38 erhalten bleibt.'} 
	  } 
	   } 
	   } 
	 & \\ 
 

    \theutterance \stepcounter{utterance}  

    & & & \multicolumn{4}{p{0.6\linewidth}}{\cellcolor[rgb]{1,0.85,0.72}{%
	\makecell[{{p{\linewidth}}}]{% 
	  \tt {\tiny [GM$\langle$B]}  
	 ABLEHNUNG: {'C56', 'A03', 'A07', 'C62', 'C00', 'C10', 'B38', 'C81'}\\ \tt ARGUMENT: {'Danke für den Move hin zu B38. C81 statt C32 ist für mich aber weiterhin ein deutlicher Wertverlust. Wenn wir nur einen Slot drehen, sollte der andere strittige Slot klar bei mir liegen. Ich schlage einen minimalinvasiven Tausch vor, der dein Kernpaket vollständig respektiert, exakt das Budget trifft und meinen Wert nicht abschneidet.'}\\ \tt VORSCHLAG: {'C56', 'A03', 'A07', 'C62', 'C00', 'C10', 'B38', 'C32'}\\ \tt ARGUMENT: {'Summe Aufwand = 3357. Das ist der kleinste gemeinsame Nenner: Dein komplettes Kernpaket bleibt unangetastet, ich erhalte genau einen der beiden für mich kritischen Slots zurück (C32) plus B38, und wir bleiben bei gleicher Stückzahl ohne Puffer. Wenn du unbedingt C81 unterbringen willst, können wir nur über einen echten Ausgleich sprechen (z. B. C81 rein, dafür ein anderes starkes Stück für mich), aber nicht auf Kosten von C32 und B38 zugleich.'} 
	  } 
	   } 
	   } 
	  \\ 
 

    \theutterance \stepcounter{utterance}  

    & & \multicolumn{4}{p{0.6\linewidth}}{\cellcolor[rgb]{0.9,0.9,0.9}{%
	\makecell[{{p{\linewidth}}}]{% 
	  \tt {\tiny [A$\langle$GM]}  
	 ABLEHNUNG: {'C56', 'A03', 'A07', 'C62', 'C00', 'C10', 'B38', 'C81'}\\ \tt ARGUMENT: {'Danke für den Move hin zu B38. C81 statt C32 ist für mich aber weiterhin ein deutlicher Wertverlust. Wenn wir nur einen Slot drehen, sollte der andere strittige Slot klar bei mir liegen. Ich schlage einen minimalinvasiven Tausch vor, der dein Kernpaket vollständig respektiert, exakt das Budget trifft und meinen Wert nicht abschneidet.'}\\ \tt VORSCHLAG: {'C56', 'A03', 'A07', 'C62', 'C00', 'C10', 'B38', 'C32'}\\ \tt ARGUMENT: {'Summe Aufwand = 3357. Das ist der kleinste gemeinsame Nenner: Dein komplettes Kernpaket bleibt unangetastet, ich erhalte genau einen der beiden für mich kritischen Slots zurück (C32) plus B38, und wir bleiben bei gleicher Stückzahl ohne Puffer. Wenn du unbedingt C81 unterbringen willst, können wir nur über einen echten Ausgleich sprechen (z. B. C81 rein, dafür ein anderes starkes Stück für mich), aber nicht auf Kosten von C32 und B38 zugleich.'} 
	  } 
	   } 
	   } 
	 & \\ 
 

    \theutterance \stepcounter{utterance}  

    & \multicolumn{4}{p{0.6\linewidth}}{\cellcolor[rgb]{0.8,1,0.9}{%
	\makecell[{{p{\linewidth}}}]{% 
	  \tt {\tiny [A$\rangle$GM]}  
	 ABLEHNUNG: {'C56', 'A03', 'A07', 'C62', 'C00', 'C10', 'B38', 'C32'}\\ \tt ARGUMENT: {'Ich schätze deinen Minimalvorschlag, aber genau dieses Duo B38/C32 ist für mich der Knackpunkt – beide gleichzeitig zu geben, zerstört meinen Werthebel. Ich bleibe bei deinem vollständigen Kern inkl. C10, biete dir aber den kleinstmöglichen Gegenzug mit exakt gleichem Aufwand und gleicher Stückzahl: Wir tauschen nur C32 gegen C81. Damit bleibt B38 bei dir, dein Kern steht, und mein Wert wird nicht amputiert.'}\\ \tt VORSCHLAG: {'C56', 'A03', 'A07', 'C62', 'C00', 'C10', 'B38', 'C81'}\\ \tt ARGUMENT: {'Summe Aufwand = 3357. Kleinster Eingriff: 1-zu-1-Tausch C32 -> C81. Du behältst B38 und alle Kernteile; ich rette eine meiner Top-Positionen. Wenn C81 absolut nicht geht, bin ich bereit, stattdessen C98 zu nehmen (gleicher Aufwand), solange B38 sicher bleibt.'} 
	  } 
	   } 
	   } 
	 & & \\ 
 

    \theutterance \stepcounter{utterance}  

    & & \multicolumn{4}{p{0.6\linewidth}}{\cellcolor[rgb]{0.9,0.9,0.9}{%
	\makecell[{{p{\linewidth}}}]{% 
	  \tt {\tiny [GM$\rangle$B]}  
	 ABLEHNUNG: {'C56', 'A03', 'A07', 'C62', 'C00', 'C10', 'B38', 'C32'}\\ \tt ARGUMENT: {'Ich schätze deinen Minimalvorschlag, aber genau dieses Duo B38/C32 ist für mich der Knackpunkt – beide gleichzeitig zu geben, zerstört meinen Werthebel. Ich bleibe bei deinem vollständigen Kern inkl. C10, biete dir aber den kleinstmöglichen Gegenzug mit exakt gleichem Aufwand und gleicher Stückzahl: Wir tauschen nur C32 gegen C81. Damit bleibt B38 bei dir, dein Kern steht, und mein Wert wird nicht amputiert.'}\\ \tt VORSCHLAG: {'C56', 'A03', 'A07', 'C62', 'C00', 'C10', 'B38', 'C81'}\\ \tt ARGUMENT: {'Summe Aufwand = 3357. Kleinster Eingriff: 1-zu-1-Tausch C32 -> C81. Du behältst B38 und alle Kernteile; ich rette eine meiner Top-Positionen. Wenn C81 absolut nicht geht, bin ich bereit, stattdessen C98 zu nehmen (gleicher Aufwand), solange B38 sicher bleibt.'} 
	  } 
	   } 
	   } 
	 & \\ 
 

    \theutterance \stepcounter{utterance}  

    & & & \multicolumn{4}{p{0.6\linewidth}}{\cellcolor[rgb]{1,0.85,0.72}{%
	\makecell[{{p{\linewidth}}}]{% 
	  \tt {\tiny [GM$\langle$B]}  
	 ABLEHNUNG: {'C56', 'A03', 'A07', 'C62', 'C00', 'C10', 'B38', 'C81'}\\ \tt ARGUMENT: {'Ich komme dir beim Festhalten an B38 und deinem kompletten Kern maximal entgegen, aber C81 frisst für mich zu viel Wert im Vergleich zu einer klar stärkeren Alternative. Wenn wir wirklich nur einen Slot drehen, dann bitte zugunsten eines Stücks, das für mich deutlich mehr bringt, ohne dein Budgetziel oder die Stückzahl zu ändern.'}\\ \tt VORSCHLAG: {'C56', 'A03', 'A07', 'C62', 'C00', 'C10', 'B38', 'C98'}\\ \tt ARGUMENT: {'Summe Aufwand = 3357. Minimalster Eingriff: 1-zu-1-Tausch C32 -> C98 (du behältst B38 und alle Kernteile). Das ist für mich ein wesentlich besserer Ausgleich als C81, während deine Struktur und das exakte Budget unverändert bleiben. Wenn du partout C98 nicht willst, nenne mir bitte genau einen alternativen Single-Slot, der C32 ersetzt, ohne B38 anzutasten.'} 
	  } 
	   } 
	   } 
	  \\ 
 

    \theutterance \stepcounter{utterance}  

    & & \multicolumn{4}{p{0.6\linewidth}}{\cellcolor[rgb]{0.9,0.9,0.9}{%
	\makecell[{{p{\linewidth}}}]{% 
	  \tt {\tiny [A$\langle$GM]}  
	 ABLEHNUNG: {'C56', 'A03', 'A07', 'C62', 'C00', 'C10', 'B38', 'C81'}\\ \tt ARGUMENT: {'Ich komme dir beim Festhalten an B38 und deinem kompletten Kern maximal entgegen, aber C81 frisst für mich zu viel Wert im Vergleich zu einer klar stärkeren Alternative. Wenn wir wirklich nur einen Slot drehen, dann bitte zugunsten eines Stücks, das für mich deutlich mehr bringt, ohne dein Budgetziel oder die Stückzahl zu ändern.'}\\ \tt VORSCHLAG: {'C56', 'A03', 'A07', 'C62', 'C00', 'C10', 'B38', 'C98'}\\ \tt ARGUMENT: {'Summe Aufwand = 3357. Minimalster Eingriff: 1-zu-1-Tausch C32 -> C98 (du behältst B38 und alle Kernteile). Das ist für mich ein wesentlich besserer Ausgleich als C81, während deine Struktur und das exakte Budget unverändert bleiben. Wenn du partout C98 nicht willst, nenne mir bitte genau einen alternativen Single-Slot, der C32 ersetzt, ohne B38 anzutasten.'} 
	  } 
	   } 
	   } 
	 & \\ 
 

    \theutterance \stepcounter{utterance}  

    & \multicolumn{4}{p{0.6\linewidth}}{\cellcolor[rgb]{0.8,1,0.9}{%
	\makecell[{{p{\linewidth}}}]{% 
	  \tt {\tiny [A$\rangle$GM]}  
	 ABLEHNUNG: {'C56', 'A03', 'A07', 'C62', 'C00', 'C10', 'B38', 'C98'}\\ \tt ARGUMENT: {'Ich bleibe bei deinem kompletten Kern inkl. C10 und sichere dir B38, aber C98 liegt für mich klar hinter C32. Der kleinstmögliche tragfähige Kompromiss bleibt der 1-zu-1-Tausch C32 -> C81. Wenn C81 für dich gar nicht tragbar ist, biete ich dir eine letzte Minimalalternative mit identischem Aufwand an: C32 -> C38. Beides hält Budget = 3357, Stückzahl konstant und deinen Kern unberührt.'}\\ \tt VORSCHLAG: {'C56', 'A03', 'A07', 'C62', 'C00', 'C10', 'B38', 'C81'}\\ \tt ARGUMENT: {'Option A – C32 -> C81 (Summe 3357): kleinster Eingriff, sichert mir wesentlichen Wert. Option B – C32 -> C38 (Summe 3357): falls du C81 strikt ablehnst. Bitte wähle eine der beiden Single-Slot-Varianten; beide respektieren deine Struktur und das exakte Budget.'} 
	  } 
	   } 
	   } 
	 & & \\ 
 

    \theutterance \stepcounter{utterance}  

    & & \multicolumn{4}{p{0.6\linewidth}}{\cellcolor[rgb]{0.9,0.9,0.9}{%
	\makecell[{{p{\linewidth}}}]{% 
	  \tt {\tiny [GM$\rangle$B]}  
	 ABLEHNUNG: {'C56', 'A03', 'A07', 'C62', 'C00', 'C10', 'B38', 'C98'}\\ \tt ARGUMENT: {'Ich bleibe bei deinem kompletten Kern inkl. C10 und sichere dir B38, aber C98 liegt für mich klar hinter C32. Der kleinstmögliche tragfähige Kompromiss bleibt der 1-zu-1-Tausch C32 -> C81. Wenn C81 für dich gar nicht tragbar ist, biete ich dir eine letzte Minimalalternative mit identischem Aufwand an: C32 -> C38. Beides hält Budget = 3357, Stückzahl konstant und deinen Kern unberührt.'}\\ \tt VORSCHLAG: {'C56', 'A03', 'A07', 'C62', 'C00', 'C10', 'B38', 'C81'}\\ \tt ARGUMENT: {'Option A – C32 -> C81 (Summe 3357): kleinster Eingriff, sichert mir wesentlichen Wert. Option B – C32 -> C38 (Summe 3357): falls du C81 strikt ablehnst. Bitte wähle eine der beiden Single-Slot-Varianten; beide respektieren deine Struktur und das exakte Budget.'} 
	  } 
	   } 
	   } 
	 & \\ 
 

    \theutterance \stepcounter{utterance}  

    & & & \multicolumn{4}{p{0.6\linewidth}}{\cellcolor[rgb]{1,0.85,0.72}{%
	\makecell[{{p{\linewidth}}}]{% 
	  \tt {\tiny [GM$\langle$B]}  
	 ABLEHNUNG: {'C56', 'A03', 'A07', 'C62', 'C00', 'C10', 'B38', 'C81'}\\ \tt ARGUMENT: {'Danke für die zwei Single-Slot-Varianten. C81 ist für mich deutlich schwächer als C32, und C38 liegt ebenfalls hinter Alternativen, die mir wesentlich mehr bringen. Wenn wir wirklich nur einen Slot drehen und dein komplettes Kernpaket plus B38 fix lassen, dann brauche ich ein Stück, das meinen Wert spürbar trägt – nicht C81 oder C38.'}\\ \tt VORSCHLAG: {'C56', 'A03', 'A07', 'C62', 'C00', 'C10', 'B38', 'C98'}\\ \tt ARGUMENT: {'Summe Aufwand = 3357. 1-zu-1-Tausch C32 -> C98. Du behältst B38 und deinen vollen Kern, Budget exakt erfüllt, Stückzahl identisch. Das ist der kleinste Eingriff, der für mich noch tragfähig ist. Wenn du C98 partout nicht willst, nenne bitte genau einen alternativen Single-Slot, der C32 ersetzt und für mich mindestens gleichwertig ist, ohne B38 oder dein Kernpaket anzutasten.'} 
	  } 
	   } 
	   } 
	  \\ 
 

    \theutterance \stepcounter{utterance}  

    & & \multicolumn{4}{p{0.6\linewidth}}{\cellcolor[rgb]{0.9,0.9,0.9}{%
	\makecell[{{p{\linewidth}}}]{% 
	  \tt {\tiny [A$\langle$GM]}  
	 ABLEHNUNG: {'C56', 'A03', 'A07', 'C62', 'C00', 'C10', 'B38', 'C81'}\\ \tt ARGUMENT: {'Danke für die zwei Single-Slot-Varianten. C81 ist für mich deutlich schwächer als C32, und C38 liegt ebenfalls hinter Alternativen, die mir wesentlich mehr bringen. Wenn wir wirklich nur einen Slot drehen und dein komplettes Kernpaket plus B38 fix lassen, dann brauche ich ein Stück, das meinen Wert spürbar trägt – nicht C81 oder C38.'}\\ \tt VORSCHLAG: {'C56', 'A03', 'A07', 'C62', 'C00', 'C10', 'B38', 'C98'}\\ \tt ARGUMENT: {'Summe Aufwand = 3357. 1-zu-1-Tausch C32 -> C98. Du behältst B38 und deinen vollen Kern, Budget exakt erfüllt, Stückzahl identisch. Das ist der kleinste Eingriff, der für mich noch tragfähig ist. Wenn du C98 partout nicht willst, nenne bitte genau einen alternativen Single-Slot, der C32 ersetzt und für mich mindestens gleichwertig ist, ohne B38 oder dein Kernpaket anzutasten.'} 
	  } 
	   } 
	   } 
	 & \\ 
 

    \theutterance \stepcounter{utterance}  

    & \multicolumn{4}{p{0.6\linewidth}}{\cellcolor[rgb]{0.8,1,0.9}{%
	\makecell[{{p{\linewidth}}}]{% 
	  \tt {\tiny [A$\rangle$GM]}  
	 ABLEHNUNG: {'C56', 'A03', 'A07', 'C62', 'C00', 'C10', 'B38', 'C98'}\\ \tt ARGUMENT: {'Ich halte dein komplettes Kernpaket inkl. C10 und lasse B38 fix, aber C98 ist für mich deutlich weniger wert als C32. Mein minimaler tragfähiger Schritt bleibt ein Single-Slot-Tausch, der exakt das Budget trifft und die Stückzahl hält, ohne deine Struktur anzutasten.'}\\ \tt VORSCHLAG: {'C56', 'A03', 'A07', 'C62', 'C00', 'C10', 'B38', 'C81'}\\ \tt ARGUMENT: {'Option A – C32 -> C81 (Summe 3357, minimaler Eingriff, sichert mir substanziellen Wert). Falls C81 absolut nicht geht, eine letzte Single-Slot-Alternative: C32 -> C38 (ebenfalls 3357). Beide respektieren deinen vollen Kern und B38. Bitte wähle eine der beiden, damit wir abschließen können.'} 
	  } 
	   } 
	   } 
	 & & \\ 
 

    \theutterance \stepcounter{utterance}  

    & & \multicolumn{4}{p{0.6\linewidth}}{\cellcolor[rgb]{0.9,0.9,0.9}{%
	\makecell[{{p{\linewidth}}}]{% 
	  \tt {\tiny [GM$\rangle$B]}  
	 ABLEHNUNG: {'C56', 'A03', 'A07', 'C62', 'C00', 'C10', 'B38', 'C98'}\\ \tt ARGUMENT: {'Ich halte dein komplettes Kernpaket inkl. C10 und lasse B38 fix, aber C98 ist für mich deutlich weniger wert als C32. Mein minimaler tragfähiger Schritt bleibt ein Single-Slot-Tausch, der exakt das Budget trifft und die Stückzahl hält, ohne deine Struktur anzutasten.'}\\ \tt VORSCHLAG: {'C56', 'A03', 'A07', 'C62', 'C00', 'C10', 'B38', 'C81'}\\ \tt ARGUMENT: {'Option A – C32 -> C81 (Summe 3357, minimaler Eingriff, sichert mir substanziellen Wert). Falls C81 absolut nicht geht, eine letzte Single-Slot-Alternative: C32 -> C38 (ebenfalls 3357). Beide respektieren deinen vollen Kern und B38. Bitte wähle eine der beiden, damit wir abschließen können.'} 
	  } 
	   } 
	   } 
	 & \\ 
 

    \theutterance \stepcounter{utterance}  

    & & & \multicolumn{4}{p{0.6\linewidth}}{\cellcolor[rgb]{1,0.85,0.72}{%
	\makecell[{{p{\linewidth}}}]{% 
	  \tt {\tiny [GM$\langle$B]}  
	 ABLEHNUNG: {'C56', 'A03', 'A07', 'C62', 'C00', 'C10', 'B38', 'C81'}\\ \tt ARGUMENT: {'Ich kann C81/C38 nicht sinnvoll gegen C32 tauschen – beide liegen für mich klar darunter. Du bekommst deinen vollen Kern plus B38; als minimaler, tragfähiger Single-Slot-Ersatz für C32 kommt für mich nur C98 infrage. Wenn du C98 nicht akzeptierst, müssen wir auf deinen Vorschlag mit C32 zurückgehen oder wir erweitern den Tausch um echten Ausgleich.'}\\ \tt VORSCHLAG: {'C56', 'A03', 'A07', 'C62', 'C00', 'C10', 'B38', 'C98'}\\ \tt ARGUMENT: {'Summe Aufwand = 3357. 1-zu-1 C32 -> C98. Du behältst B38 und dein komplettes Kernpaket, Budget und Stückzahl bleiben identisch. Das ist mein kleinstmöglicher Kompromiss. Wenn das nicht geht, nenne mir bitte einen alternativen Single-Slot, der C32 ersetzt und für mich mindestens gleichwertig ist – ohne B38 oder dein Kernpaket anzutasten.'} 
	  } 
	   } 
	   } 
	  \\ 
 

    \theutterance \stepcounter{utterance}  

    & & \multicolumn{4}{p{0.6\linewidth}}{\cellcolor[rgb]{0.9,0.9,0.9}{%
	\makecell[{{p{\linewidth}}}]{% 
	  \tt {\tiny [A$\langle$GM]}  
	 ABLEHNUNG: {'C56', 'A03', 'A07', 'C62', 'C00', 'C10', 'B38', 'C81'}\\ \tt ARGUMENT: {'Ich kann C81/C38 nicht sinnvoll gegen C32 tauschen – beide liegen für mich klar darunter. Du bekommst deinen vollen Kern plus B38; als minimaler, tragfähiger Single-Slot-Ersatz für C32 kommt für mich nur C98 infrage. Wenn du C98 nicht akzeptierst, müssen wir auf deinen Vorschlag mit C32 zurückgehen oder wir erweitern den Tausch um echten Ausgleich.'}\\ \tt VORSCHLAG: {'C56', 'A03', 'A07', 'C62', 'C00', 'C10', 'B38', 'C98'}\\ \tt ARGUMENT: {'Summe Aufwand = 3357. 1-zu-1 C32 -> C98. Du behältst B38 und dein komplettes Kernpaket, Budget und Stückzahl bleiben identisch. Das ist mein kleinstmöglicher Kompromiss. Wenn das nicht geht, nenne mir bitte einen alternativen Single-Slot, der C32 ersetzt und für mich mindestens gleichwertig ist – ohne B38 oder dein Kernpaket anzutasten.'} 
	  } 
	   } 
	   } 
	 & \\ 
 

    \theutterance \stepcounter{utterance}  

    & \multicolumn{4}{p{0.6\linewidth}}{\cellcolor[rgb]{0.8,1,0.9}{%
	\makecell[{{p{\linewidth}}}]{% 
	  \tt {\tiny [A$\rangle$GM]}  
	 ABLEHNUNG: {'C56', 'A03', 'A07', 'C62', 'C00', 'C10', 'B38', 'C98'}\\ \tt ARGUMENT: {'Ich respektiere dein Beharren auf deinem Kern und B38, aber C98 ist für mich klar unter C32. Ein Single-Slot, der für mich tragfähig ist, ist C81; C38 wäre meine zweitbeste Einzelt-Option. Wenn C98 für dich unverrückbar ist, brauchen wir echten Ausgleich über einen Doppeltausch – das lehne ich als minimalen Schritt ab.'}\\ \tt VORSCHLAG: {'C56', 'A03', 'A07', 'C62', 'C00', 'C10', 'B38', 'C81'}\\ \tt ARGUMENT: {'Minimaler tragfähiger Kompromiss: 1-zu-1 C32 -> C81 bei exakt 3357 Aufwand und gleicher Stückzahl. Du behältst deinen vollen Kern plus B38; ich sichere mir einen Slot mit substanziellem Wert. Falls C81 absolut nicht geht, kann ich als Single-Slot noch C38 anbieten (ebenfalls 3357), aber C98 ist für mich kein gleichwertiger Ersatz.'} 
	  } 
	   } 
	   } 
	 & & \\ 
 

    \theutterance \stepcounter{utterance}  

    & & \multicolumn{4}{p{0.6\linewidth}}{\cellcolor[rgb]{0.9,0.9,0.9}{%
	\makecell[{{p{\linewidth}}}]{% 
	  \tt {\tiny [GM$\rangle$B]}  
	 ABLEHNUNG: {'C56', 'A03', 'A07', 'C62', 'C00', 'C10', 'B38', 'C98'}\\ \tt ARGUMENT: {'Ich respektiere dein Beharren auf deinem Kern und B38, aber C98 ist für mich klar unter C32. Ein Single-Slot, der für mich tragfähig ist, ist C81; C38 wäre meine zweitbeste Einzelt-Option. Wenn C98 für dich unverrückbar ist, brauchen wir echten Ausgleich über einen Doppeltausch – das lehne ich als minimalen Schritt ab.'}\\ \tt VORSCHLAG: {'C56', 'A03', 'A07', 'C62', 'C00', 'C10', 'B38', 'C81'}\\ \tt ARGUMENT: {'Minimaler tragfähiger Kompromiss: 1-zu-1 C32 -> C81 bei exakt 3357 Aufwand und gleicher Stückzahl. Du behältst deinen vollen Kern plus B38; ich sichere mir einen Slot mit substanziellem Wert. Falls C81 absolut nicht geht, kann ich als Single-Slot noch C38 anbieten (ebenfalls 3357), aber C98 ist für mich kein gleichwertiger Ersatz.'} 
	  } 
	   } 
	   } 
	 & \\ 
 

    \theutterance \stepcounter{utterance}  

    & & & \multicolumn{4}{p{0.6\linewidth}}{\cellcolor[rgb]{1,0.85,0.72}{%
	\makecell[{{p{\linewidth}}}]{% 
	  \tt {\tiny [GM$\langle$B]}  
	 ABLEHNUNG: {'C56', 'A03', 'A07', 'C62', 'C00', 'C10', 'B38', 'C81'}\\ \tt ARGUMENT: {'Wir drehen uns im Kreis beim Single-Slot. Für mich ist C81/C38 klar unter C32, für dich ist C98 klar unter C32. Ein echter, aber minimaler Ausgleich wäre ein kleiner Doppeltausch mit exakt gleichem Aufwand und gleicher Stückzahl, der beiden Seiten Kernwünsche respektiert und den Wert fair balanciert.'}\\ \tt VORSCHLAG: {'C56', 'A03', 'A07', 'C62', 'C00', 'C10', 'B38', 'C98'}\\ \tt ARGUMENT: {'Summe Aufwand = 3357. Doppeltausch light: Wir halten deinen vollen Kern und B38 fix; statt C32 kommt C98, und im Gegenzug verpflichte ich mich, keine weiteren Änderungen zu verlangen. Das ist der kleinste Ausgleich, der für mich noch tragfähig ist. Wenn du C98 weiterhin ausschließt, nenne bitte genau einen alternativen Ausgleich (Single oder Double), der C32 ersetzt und für mich mindestens gleichwertig ist, ohne deinen Kern oder B38 anzutasten.'} 
	  } 
	   } 
	   } 
	  \\ 
 

    \theutterance \stepcounter{utterance}  

    & & \multicolumn{4}{p{0.6\linewidth}}{\cellcolor[rgb]{0.9,0.9,0.9}{%
	\makecell[{{p{\linewidth}}}]{% 
	  \tt {\tiny [A$\langle$GM]}  
	 ABLEHNUNG: {'C56', 'A03', 'A07', 'C62', 'C00', 'C10', 'B38', 'C81'}\\ \tt ARGUMENT: {'Wir drehen uns im Kreis beim Single-Slot. Für mich ist C81/C38 klar unter C32, für dich ist C98 klar unter C32. Ein echter, aber minimaler Ausgleich wäre ein kleiner Doppeltausch mit exakt gleichem Aufwand und gleicher Stückzahl, der beiden Seiten Kernwünsche respektiert und den Wert fair balanciert.'}\\ \tt VORSCHLAG: {'C56', 'A03', 'A07', 'C62', 'C00', 'C10', 'B38', 'C98'}\\ \tt ARGUMENT: {'Summe Aufwand = 3357. Doppeltausch light: Wir halten deinen vollen Kern und B38 fix; statt C32 kommt C98, und im Gegenzug verpflichte ich mich, keine weiteren Änderungen zu verlangen. Das ist der kleinste Ausgleich, der für mich noch tragfähig ist. Wenn du C98 weiterhin ausschließt, nenne bitte genau einen alternativen Ausgleich (Single oder Double), der C32 ersetzt und für mich mindestens gleichwertig ist, ohne deinen Kern oder B38 anzutasten.'} 
	  } 
	   } 
	   } 
	 & \\ 
 

    \theutterance \stepcounter{utterance}  

    & \multicolumn{4}{p{0.6\linewidth}}{\cellcolor[rgb]{0.8,1,0.9}{%
	\makecell[{{p{\linewidth}}}]{% 
	  \tt {\tiny [A$\rangle$GM]}  
	 ABLEHNUNG: {'C56', 'A03', 'A07', 'C62', 'C00', 'C10', 'B38', 'C98'}\\ \tt ARGUMENT: {'Ein reiner C32 -> C98-Tausch bleibt für mich unter der Schmerzgrenze. Wenn Single-Slot nicht geht, lass uns den minimalen Doppeltausch fahren, der beiden Seiten echten Ausgleich gibt, Budget exakt trifft und deine Kernteile plus B38 respektiert: Wir geben dir C98, ich bekomme im Gegenzug C81 statt C10 ODER wir ersetzen C10 durch A03+klein geht nicht wegen Budget. Daher schlage ich den kleinstmöglichen echten Ausgleich vor: C32 raus, C81 rein; zusätzlich tauschen wir C10 gegen C98 zurück – aber C10 ist dein Kern, also verwerfe ich das. Praktikabler Mini-Doppeltausch: C32 -> C81 und C98 -> C38 funktioniert nicht, weil du C98 erst einführen willst. Kurz: Der einzig saubere Double-Trade, der Budget 3357 hält und beiden Seiten Mehrwert bringt, ist B38/C32 beibehalten und statt eines schwierigen Swaps eine Alternative bei gleicher Stückzahl – das hast du abgelehnt. Ich biete letzte Brücke:'}\\ \tt VORSCHLAG: {'C56', 'A03', 'A07', 'C62', 'C00', 'C10', 'B38', 'C32'}\\ \tt ARGUMENT: {'Letzte Brücke, exakter Aufwand 3357: Du behältst deinen vollen Kern und B38; ich behalte C32. Das ist der stabile Fixpunkt, der unseren Knoten löst. Wenn du zwingend einen Doppeltausch willst, nenne bitte ein Paar, das Budget 3357 hält und nicht C81/C38 als Ersatz für C32 setzt – sonst ist dieser Fixpunkt die vernünftige Lösung.'} 
	  } 
	   } 
	   } 
	 & & \\ 
 

    \theutterance \stepcounter{utterance}  

    & & \multicolumn{4}{p{0.6\linewidth}}{\cellcolor[rgb]{0.9,0.9,0.9}{%
	\makecell[{{p{\linewidth}}}]{% 
	  \tt {\tiny [GM$\rangle$B]}  
	 ABLEHNUNG: {'C56', 'A03', 'A07', 'C62', 'C00', 'C10', 'B38', 'C98'}\\ \tt ARGUMENT: {'Ein reiner C32 -> C98-Tausch bleibt für mich unter der Schmerzgrenze. Wenn Single-Slot nicht geht, lass uns den minimalen Doppeltausch fahren, der beiden Seiten echten Ausgleich gibt, Budget exakt trifft und deine Kernteile plus B38 respektiert: Wir geben dir C98, ich bekomme im Gegenzug C81 statt C10 ODER wir ersetzen C10 durch A03+klein geht nicht wegen Budget. Daher schlage ich den kleinstmöglichen echten Ausgleich vor: C32 raus, C81 rein; zusätzlich tauschen wir C10 gegen C98 zurück – aber C10 ist dein Kern, also verwerfe ich das. Praktikabler Mini-Doppeltausch: C32 -> C81 und C98 -> C38 funktioniert nicht, weil du C98 erst einführen willst. Kurz: Der einzig saubere Double-Trade, der Budget 3357 hält und beiden Seiten Mehrwert bringt, ist B38/C32 beibehalten und statt eines schwierigen Swaps eine Alternative bei gleicher Stückzahl – das hast du abgelehnt. Ich biete letzte Brücke:'}\\ \tt VORSCHLAG: {'C56', 'A03', 'A07', 'C62', 'C00', 'C10', 'B38', 'C32'}\\ \tt ARGUMENT: {'Letzte Brücke, exakter Aufwand 3357: Du behältst deinen vollen Kern und B38; ich behalte C32. Das ist der stabile Fixpunkt, der unseren Knoten löst. Wenn du zwingend einen Doppeltausch willst, nenne bitte ein Paar, das Budget 3357 hält und nicht C81/C38 als Ersatz für C32 setzt – sonst ist dieser Fixpunkt die vernünftige Lösung.'} 
	  } 
	   } 
	   } 
	 & \\ 
 

    \theutterance \stepcounter{utterance}  

    & & & \multicolumn{4}{p{0.6\linewidth}}{\cellcolor[rgb]{1,0.85,0.72}{%
	\makecell[{{p{\linewidth}}}]{% 
	  \tt {\tiny [GM$\langle$B]}  
	 ABLEHNUNG: {'C56', 'A03', 'A07', 'C62', 'C00', 'C10', 'B38', 'C32'}\\ \tt ARGUMENT: {'Wir sind beim Fixpunkt gelandet, der für dich optimal ist, für mich aber zwei klar höher bewertete Alternativen ausschließt. Ein Einzelslot C32->C81/C38 ist für dich nicht akzeptabel; C32->C98 ist für mich der kleinste tragfähige Schritt. Also schlagen wir eine echte, minimale Win-Win-Variante vor: gleicher Aufwand (3357), gleiche Stückzahl, dein kompletter Kern plus B38 bleibt bestehen, und ich erhalte einen Slot mit substanziell höherem Wert.'}\\ \tt VORSCHLAG: {'C56', 'A03', 'A07', 'C62', 'C00', 'C10', 'B38', 'C98'}\\ \tt ARGUMENT: {'Summe Aufwand = 3357. Das ist der kleinste faire Ausgleich: Dein Kern bleibt unberührt, B38 bleibt bei dir, wir tauschen nur C32 -> C98. Damit beenden wir die Schleife mit einem symmetrischen Kompromiss. Wenn das weiterhin nicht tragfähig ist, bitte nenne genau eine alternative Single-Slot-Option für C32, die du akzeptierst und die für mich mindestens C98 entspricht – ohne deinen Kern oder B38 anzutasten.'} 
	  } 
	   } 
	   } 
	  \\ 
 

    \theutterance \stepcounter{utterance}  

    & & \multicolumn{4}{p{0.6\linewidth}}{\cellcolor[rgb]{0.9,0.9,0.9}{%
	\makecell[{{p{\linewidth}}}]{% 
	  \tt {\tiny [A$\langle$GM]}  
	 ABLEHNUNG: {'C56', 'A03', 'A07', 'C62', 'C00', 'C10', 'B38', 'C32'}\\ \tt ARGUMENT: {'Wir sind beim Fixpunkt gelandet, der für dich optimal ist, für mich aber zwei klar höher bewertete Alternativen ausschließt. Ein Einzelslot C32->C81/C38 ist für dich nicht akzeptabel; C32->C98 ist für mich der kleinste tragfähige Schritt. Also schlagen wir eine echte, minimale Win-Win-Variante vor: gleicher Aufwand (3357), gleiche Stückzahl, dein kompletter Kern plus B38 bleibt bestehen, und ich erhalte einen Slot mit substanziell höherem Wert.'}\\ \tt VORSCHLAG: {'C56', 'A03', 'A07', 'C62', 'C00', 'C10', 'B38', 'C98'}\\ \tt ARGUMENT: {'Summe Aufwand = 3357. Das ist der kleinste faire Ausgleich: Dein Kern bleibt unberührt, B38 bleibt bei dir, wir tauschen nur C32 -> C98. Damit beenden wir die Schleife mit einem symmetrischen Kompromiss. Wenn das weiterhin nicht tragfähig ist, bitte nenne genau eine alternative Single-Slot-Option für C32, die du akzeptierst und die für mich mindestens C98 entspricht – ohne deinen Kern oder B38 anzutasten.'} 
	  } 
	   } 
	   } 
	 & \\ 
 

    \theutterance \stepcounter{utterance}  

    & \multicolumn{4}{p{0.6\linewidth}}{\cellcolor[rgb]{0.8,1,0.9}{%
	\makecell[{{p{\linewidth}}}]{% 
	  \tt {\tiny [A$\rangle$GM]}  
	 ABLEHNUNG: {'C56', 'A03', 'A07', 'C62', 'C00', 'C10', 'B38', 'C98'}\\ \tt ARGUMENT: {'Wir blockieren uns beim Singleslot. Daher echter minimaler Doppeltausch mit exakt 3357 Aufwand, gleicher Stückzahl und deinem vollen Kern plus B38 unverändert: Ich gebe dir deinen Wunschslot C32 zurück, du gibst mir im Gegenzug C81 statt C10 geht nicht, da C10 dein Kern ist. Also der kleinste tragfähige Double-Trade, der dein Kernpaket respektiert: tausche C32 -> C81 und parallel C10 -> C98; so bleibt dein Kern bis auf C10 unangetastet, aber du hältst weiterhin A03, A07, C56, C62, C00, B38. Falls C10 unverrückbar ist, bleibt die einzige faire Minimalvariante: C32 bleibt drin und wir schließen auf dem Fixpunkt.'}\\ \tt VORSCHLAG: {'C56', 'A03', 'A07', 'C62', 'C00', 'B38', 'C81', 'C98'}\\ \tt ARGUMENT: {'Summe Aufwand = 3357. Minimaler Doppeltausch gegenüber deinem letzten Vorschlag: C10 und C32 raus, C81 und C98 rein. Dein Paket behält A03, A07, C56, C62, C00, B38 (dein Kern weitgehend intakt), ich erhalte zwei Slots mit substanziell höherem Wert. Wenn C10 absolut nicht verhandelbar ist, dann bleibt als faire Lösung nur der Fixpunkt mit C32 – sonst ist dies der kleinste echte Win-Win-Schritt, der die Sackgasse auflöst.'} 
	  } 
	   } 
	   } 
	 & & \\ 
 

    \theutterance \stepcounter{utterance}  

    & & \multicolumn{4}{p{0.6\linewidth}}{\cellcolor[rgb]{0.9,0.9,0.9}{%
	\makecell[{{p{\linewidth}}}]{% 
	  \tt {\tiny [GM$\rangle$B]}  
	 ABLEHNUNG: {'C56', 'A03', 'A07', 'C62', 'C00', 'C10', 'B38', 'C98'}\\ \tt ARGUMENT: {'Wir blockieren uns beim Singleslot. Daher echter minimaler Doppeltausch mit exakt 3357 Aufwand, gleicher Stückzahl und deinem vollen Kern plus B38 unverändert: Ich gebe dir deinen Wunschslot C32 zurück, du gibst mir im Gegenzug C81 statt C10 geht nicht, da C10 dein Kern ist. Also der kleinste tragfähige Double-Trade, der dein Kernpaket respektiert: tausche C32 -> C81 und parallel C10 -> C98; so bleibt dein Kern bis auf C10 unangetastet, aber du hältst weiterhin A03, A07, C56, C62, C00, B38. Falls C10 unverrückbar ist, bleibt die einzige faire Minimalvariante: C32 bleibt drin und wir schließen auf dem Fixpunkt.'}\\ \tt VORSCHLAG: {'C56', 'A03', 'A07', 'C62', 'C00', 'B38', 'C81', 'C98'}\\ \tt ARGUMENT: {'Summe Aufwand = 3357. Minimaler Doppeltausch gegenüber deinem letzten Vorschlag: C10 und C32 raus, C81 und C98 rein. Dein Paket behält A03, A07, C56, C62, C00, B38 (dein Kern weitgehend intakt), ich erhalte zwei Slots mit substanziell höherem Wert. Wenn C10 absolut nicht verhandelbar ist, dann bleibt als faire Lösung nur der Fixpunkt mit C32 – sonst ist dies der kleinste echte Win-Win-Schritt, der die Sackgasse auflöst.'} 
	  } 
	   } 
	   } 
	 & \\ 
 

    \theutterance \stepcounter{utterance}  

    & & & \multicolumn{4}{p{0.6\linewidth}}{\cellcolor[rgb]{1,0.85,0.72}{%
	\makecell[{{p{\linewidth}}}]{% 
	  \tt {\tiny [GM$\langle$B]}  
	 ABLEHNUNG: {'C56', 'A03', 'A07', 'C62', 'C00', 'B38', 'C81', 'C98'}\\ \tt ARGUMENT: {'C10 ist für mich unverrückbar und Teil des zugesicherten Kerns; ein Doppeltausch, der C10 entfernt, ist ausgeschlossen. Wenn wir vom Singleslot weggehen, biete ich einen echten Minimal-Doppeltausch an, der dein Kernpaket inklusive C10 und B38 vollständig respektiert und trotzdem einen substanziellen Ausgleich schafft – exakt 3357 Aufwand, gleiche Stückzahl.'}\\ \tt VORSCHLAG: {'C56', 'A03', 'A07', 'C62', 'C00', 'C10', 'B38', 'C98'}\\ \tt ARGUMENT: {'Summe Aufwand = 3357. Minimaler Ausgleich: Wir belassen dein komplettes Kernpaket inkl. C10 und B38; einzig C32 wird durch C98 ersetzt. Das ist der kleinste tragfähige Kompromiss meinerseits. Wenn du weiterhin C98 ablehnst, nenne bitte genau einen alternativen Einzelslot als Ersatz für C32, der für mich mindestens C98 entspricht und dein Kernpaket plus B38 unangetastet lässt – sonst verbleibt nur der von dir genannte Fixpunkt mit C32.'} 
	  } 
	   } 
	   } 
	  \\ 
 

    \theutterance \stepcounter{utterance}  

    & & \multicolumn{4}{p{0.6\linewidth}}{\cellcolor[rgb]{0.9,0.9,0.9}{%
	\makecell[{{p{\linewidth}}}]{% 
	  \tt {\tiny [A$\langle$GM]}  
	 ABLEHNUNG: {'C56', 'A03', 'A07', 'C62', 'C00', 'B38', 'C81', 'C98'}\\ \tt ARGUMENT: {'C10 ist für mich unverrückbar und Teil des zugesicherten Kerns; ein Doppeltausch, der C10 entfernt, ist ausgeschlossen. Wenn wir vom Singleslot weggehen, biete ich einen echten Minimal-Doppeltausch an, der dein Kernpaket inklusive C10 und B38 vollständig respektiert und trotzdem einen substanziellen Ausgleich schafft – exakt 3357 Aufwand, gleiche Stückzahl.'}\\ \tt VORSCHLAG: {'C56', 'A03', 'A07', 'C62', 'C00', 'C10', 'B38', 'C98'}\\ \tt ARGUMENT: {'Summe Aufwand = 3357. Minimaler Ausgleich: Wir belassen dein komplettes Kernpaket inkl. C10 und B38; einzig C32 wird durch C98 ersetzt. Das ist der kleinste tragfähige Kompromiss meinerseits. Wenn du weiterhin C98 ablehnst, nenne bitte genau einen alternativen Einzelslot als Ersatz für C32, der für mich mindestens C98 entspricht und dein Kernpaket plus B38 unangetastet lässt – sonst verbleibt nur der von dir genannte Fixpunkt mit C32.'} 
	  } 
	   } 
	   } 
	 & \\ 
 

    \theutterance \stepcounter{utterance}  

    & \multicolumn{4}{p{0.6\linewidth}}{\cellcolor[rgb]{0.8,1,0.9}{%
	\makecell[{{p{\linewidth}}}]{% 
	  \tt {\tiny [A$\rangle$GM]}  
	 ABLEHNUNG: {'C56', 'A03', 'A07', 'C62', 'C00', 'C10', 'B38', 'C98'}\\ \tt ARGUMENT: {'Ich akzeptiere deinen vollen Kern inkl. C10 und B38, aber C98 ist für mich kein gleichwertiger Ersatz für C32. Damit wir aus der Schleife kommen, biete ich dir zwei finale, minimalinvasive Optionen an, beide exakt 3357 Aufwand, gleiche Stückzahl, dein Kern unberührt und B38 bleibt: Option A) C32 -> C81. Option B) C32 -> C38. Das sind meine einzigen tragfähigen Single-Slot-Alternativen. Wenn beide für dich ausgeschlossen sind, bleibt als einzig stabiler Abschluss der Fixpunkt mit C32.'}\\ \tt VORSCHLAG: {'C56', 'A03', 'A07', 'C62', 'C00', 'C10', 'B38', 'C32'}\\ \tt ARGUMENT: {'Stabiler Abschluss bei exakt 3357: Dein komplettes Kernpaket plus B38 bleibt erhalten; ich behalte C32. Das löst die Pattsituation ohne weitere Schleifen. Wenn du doch eine der beiden Single-Slot-Varianten (C81 oder C38 statt C32) akzeptieren kannst, sag an – andernfalls schlage ich vor, wir schließen auf diesem Fixpunkt.'} 
	  } 
	   } 
	   } 
	 & & \\ 
 

    \theutterance \stepcounter{utterance}  

    & & \multicolumn{4}{p{0.6\linewidth}}{\cellcolor[rgb]{0.9,0.9,0.9}{%
	\makecell[{{p{\linewidth}}}]{% 
	  \tt {\tiny [GM$\rangle$B]}  
	 ABLEHNUNG: {'C56', 'A03', 'A07', 'C62', 'C00', 'C10', 'B38', 'C98'}\\ \tt ARGUMENT: {'Ich akzeptiere deinen vollen Kern inkl. C10 und B38, aber C98 ist für mich kein gleichwertiger Ersatz für C32. Damit wir aus der Schleife kommen, biete ich dir zwei finale, minimalinvasive Optionen an, beide exakt 3357 Aufwand, gleiche Stückzahl, dein Kern unberührt und B38 bleibt: Option A) C32 -> C81. Option B) C32 -> C38. Das sind meine einzigen tragfähigen Single-Slot-Alternativen. Wenn beide für dich ausgeschlossen sind, bleibt als einzig stabiler Abschluss der Fixpunkt mit C32.'}\\ \tt VORSCHLAG: {'C56', 'A03', 'A07', 'C62', 'C00', 'C10', 'B38', 'C32'}\\ \tt ARGUMENT: {'Stabiler Abschluss bei exakt 3357: Dein komplettes Kernpaket plus B38 bleibt erhalten; ich behalte C32. Das löst die Pattsituation ohne weitere Schleifen. Wenn du doch eine der beiden Single-Slot-Varianten (C81 oder C38 statt C32) akzeptieren kannst, sag an – andernfalls schlage ich vor, wir schließen auf diesem Fixpunkt.'} 
	  } 
	   } 
	   } 
	 & \\ 
 

    \theutterance \stepcounter{utterance}  

    & & & \multicolumn{4}{p{0.6\linewidth}}{\cellcolor[rgb]{1,0.85,0.72}{%
	\makecell[{{p{\linewidth}}}]{% 
	  \tt {\tiny [GM$\langle$B]}  
	 ZUSTIMMUNG: {'C56', 'A03', 'A07', 'C62', 'C00', 'C10', 'B38', 'C32'}\\ \tt ARGUMENT: {'Einverstanden. Das trifft exakt 3357 Aufwand, respektiert dein Kernpaket sowie B38 und bringt uns verlässlich zum Abschluss ohne weitere Schleifen.'} 
	  } 
	   } 
	   } 
	  \\ 
 

    \theutterance \stepcounter{utterance}  

    & & & \multicolumn{2}{p{0.3\linewidth}}{\cellcolor[rgb]{0.95,0.95,0.95}{%
	\makecell[{{p{\linewidth}}}]{% 
	  \tt {\tiny [GM$|$GM]}  
	 game successful 
	  } 
	   } 
	   } 
	 & & \\ 
 

    \theutterance \stepcounter{utterance}  

    & & & \multicolumn{2}{p{0.3\linewidth}}{\cellcolor[rgb]{0.95,0.95,0.95}{%
	\makecell[{{p{\linewidth}}}]{% 
	  \tt {\tiny [GM$|$GM]}  
	 end game 
	  } 
	   } 
	   } 
	 & & \\ 
 

\end{supertabular}
}

\end{document}
