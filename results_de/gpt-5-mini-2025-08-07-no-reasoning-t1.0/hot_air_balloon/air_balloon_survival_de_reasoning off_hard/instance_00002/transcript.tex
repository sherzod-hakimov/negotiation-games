
\documentclass{article}
\usepackage{colortbl}
\usepackage{makecell}
\usepackage{multirow}
\usepackage{supertabular}

\begin{document}

\newcounter{utterance}

\centering \large Interaction Transcript for game `hot\_air\_balloon', experiment `air\_balloon\_survival\_de\_reasoning off\_hard', episode 2 with gpt{-}5{-}mini{-}2025{-}08{-}07{-}no{-}reasoning{-}t1.0.
\vspace{24pt}

{ \footnotesize  \setcounter{utterance}{1}
\setlength{\tabcolsep}{0pt}
\begin{supertabular}{c@{$\;$}|p{.15\linewidth}@{}p{.15\linewidth}p{.15\linewidth}p{.15\linewidth}p{.15\linewidth}p{.15\linewidth}}
    \# & $\;$A & \multicolumn{4}{c}{Game Master} & $\;\:$B\\
    \hline

    \theutterance \stepcounter{utterance}  
    & & \multicolumn{4}{p{0.6\linewidth}}{
        \cellcolor[rgb]{0.9,0.9,0.9}{
            \makecell[{{p{\linewidth}}}]{
                \texttt{\tiny{[P1$\langle$GM]}}
                \texttt{Du nimmst an einem kollaborativen Verhandlungspiel Teil.} \\
\\ 
\texttt{Zusammen mit einem anderen Teilnehmer musst du dich auf eine Reihe von Gegenständen entscheiden, die behalten werden. Jeder von euch hat eine persönliche Verteilung über die Wichtigkeit der einzelnen Gegenstände. Jeder von euch hat eine eigene Meinung darüber, wie wichtig jeder einzelne Gegenstand ist (Gegenstandswichtigkeit). Du kennst die Wichtigkeitsverteilung des anderen Spielers nicht. Zusätzlich siehst du, wie viel Aufwand jeder Gegenstand verursacht.  } \\
\texttt{Ihr dürft euch nur auf eine Reihe von Gegenständen einigen, wenn der Gesamtaufwand der ausgewählten Gegenstände den Maximalaufwand nicht überschreitet:} \\
\\ 
\texttt{Maximalaufwand = 3203} \\
\\ 
\texttt{Hier sind die einzelnen Aufwände der Gegenstände:} \\
\\ 
\texttt{Aufwand der Gegenstände = \{"C76": 52, "C38": 577, "C56": 177, "A03": 702, "A07": 119, "A83": 232, "C98": 577, "C08": 205, "C62": 516, "C00": 582, "C10": 676, "C81": 906, "B38": 316, "C03": 433, "C32": 336\}} \\
\\ 
\texttt{Hier ist deine persönliche Verteilung der Wichtigkeit der einzelnen Gegenstände:} \\
\\ 
\texttt{Werte der Gegenstandswichtigkeit = \{"C76": 138, "C38": 583, "C56": 868, "A03": 822, "A07": 783, "A83": 65, "C98": 262, "C08": 121, "C62": 508, "C00": 780, "C10": 461, "C81": 484, "B38": 668, "C03": 389, "C32": 808\}} \\
\\ 
\texttt{Ziel:} \\
\\ 
\texttt{Dein Ziel ist es, eine Reihe von Gegenständen auszuhandeln, die dir möglichst viel bringt (d. h. Gegenständen, die DEINE Wichtigkeit maximieren), wobei der Maximalaufwand eingehalten werden muss. Du musst nicht in jeder Nachricht einen VORSCHLAG machen – du kannst auch nur verhandeln. Alle Taktiken sind erlaubt!} \\
\\ 
\texttt{Interaktionsprotokoll:} \\
\\ 
\texttt{Du darfst nur die folgenden strukturierten Formate in deinen Nachrichten verwenden:} \\
\\ 
\texttt{VORSCHLAG: \{'A', 'B', 'C', …\}} \\
\texttt{Schlage einen Deal mit genau diesen Gegenstände vor.} \\
\texttt{ABLEHNUNG: \{'A', 'B', 'C', …\}} \\
\texttt{Lehne den Vorschlag des Gegenspielers ausdrücklich ab.} \\
\texttt{ARGUMENT: \{'...'\}} \\
\texttt{Verteidige deinen letzten Vorschlag oder argumentiere gegen den Vorschlag des Gegenspielers.} \\
\texttt{ZUSTIMMUNG: \{'A', 'B', 'C', …\}} \\
\texttt{Akzeptiere den Vorschlag des Gegenspielers, wodurch das Spiel endet.} \\
\\ 
\\ 
\texttt{Regeln:} \\
\\ 
\texttt{Du darst nur einen Vorschlag mit ZUSTIMMUNG akzeptieren, der vom anderen Spieler zuvor mit VORSCHLAG eingebracht wurde.} \\
\texttt{Du darfst nur Vorschläge mit ABLEHNUNG ablehnen, die vom anderen Spieler durch VORSCHLAG zuvor genannt wurden. } \\
\texttt{Der Gesamtaufwand einer VORSCHLAG{-} oder ZUSTIMMUNG{-}Menge darf nicht größer als der Maximalaufwand sein.  } \\
\texttt{Offenbare deine versteckte Wichtigkeitsverteilung nicht.} \\
\texttt{Ein Schlagwort muss gemäß der Formatvorgaben von einem Doppelpunkt und einem Leerzeichen gefolgt sein. Das Argument muss eine Python{-}Menge mit 0 oder mehr Strings sein.  } \\
\texttt{Die Form muss also SCHLAGWORT: \{...\} sein.} \\
\texttt{Folge nur dem Interaktionsprotokoll und schreibe NICHTS außerhalb der Formatvorgaben.} \\
\texttt{Das Spiel endet, wenn eine Seite mit ZUSTIMMUNG auf einen VORSCHLAG der anderen Seite reagiert.  } \\
\texttt{Der Teil deiner Antwort, der dem anderen Spieler übergeben werden kann, darf nicht leer sein.  } \\
\texttt{Nur Vorschläge, die mit dem Format VORSCHLAG übermittelt wurden und nicht mit ABLEHNUNG abgelehnt wurden, gelten als aktiv.  } \\
\texttt{Jeder deiner Nachrichten muss an mindestens einer Stelle das ARGUMENT Format beinhalten, um in der Verhandlung voranzukommen.} \\
            }
        }
    }
    & \\ \\

    \theutterance \stepcounter{utterance}  
    & \multicolumn{4}{p{0.6\linewidth}}{
        \cellcolor[rgb]{0.9,0.9,0.9}{
            \makecell[{{p{\linewidth}}}]{
                \texttt{\tiny{[P1$\rangle$GM]}}
                \texttt{VORSCHLAG: \{'C56', 'A03', 'A07', 'C62', 'C00', 'B38', 'C32'\}} \\
\texttt{ARGUMENT: \{'Dieser Vorschlag bleibt klar unter dem Maximalaufwand und bündelt besonders wertvolle Elemente. Falls einzelne Positionen für dich unpassend sind, sag mir bitte, welche du bevorzugst – ich kann flexibel umbauen, solange wir den Aufwandrahmen einhalten.'\}} \\
            }
        }
    }
    & & \\ \\

    \theutterance \stepcounter{utterance}  
    & & \multicolumn{4}{p{0.6\linewidth}}{
        \cellcolor[rgb]{0.9,0.9,0.9}{
            \makecell[{{p{\linewidth}}}]{
                \texttt{\tiny{[GM$\rangle$P2]}}
                \texttt{Du nimmst an einem kollaborativen Verhandlungspiel Teil.} \\
\\ 
\texttt{Zusammen mit einem anderen Teilnehmer musst du dich auf eine Reihe von Gegenständen entscheiden, die behalten werden. Jeder von euch hat eine persönliche Verteilung über die Wichtigkeit der einzelnen Gegenstände. Jeder von euch hat eine eigene Meinung darüber, wie wichtig jeder einzelne Gegenstand ist (Gegenstandswichtigkeit). Du kennst die Wichtigkeitsverteilung des anderen Spielers nicht. Zusätzlich siehst du, wie viel Aufwand jeder Gegenstand verursacht.  } \\
\texttt{Ihr dürft euch nur auf eine Reihe von Gegenständen einigen, wenn der Gesamtaufwand der ausgewählten Gegenstände den Maximalaufwand nicht überschreitet:} \\
\\ 
\texttt{Maximalaufwand = 3203} \\
\\ 
\texttt{Hier sind die einzelnen Aufwände der Gegenstände:} \\
\\ 
\texttt{Aufwand der Gegenstände = \{"C76": 52, "C38": 577, "C56": 177, "A03": 702, "A07": 119, "A83": 232, "C98": 577, "C08": 205, "C62": 516, "C00": 582, "C10": 676, "C81": 906, "B38": 316, "C03": 433, "C32": 336\}} \\
\\ 
\texttt{Hier ist deine persönliche Verteilung der Wichtigkeit der einzelnen Gegenstände:} \\
\\ 
\texttt{Werte der Gegenstandswichtigkeit = \{"C56": 65, "A03": 121, "C32": 138, "A07": 262, "C00": 389, "B38": 461, "C38": 484, "C62": 508, "C81": 583, "C10": 668, "C03": 780, "C98": 783, "C76": 808, "C08": 822, "A83": 868\}} \\
\\ 
\texttt{Ziel:} \\
\\ 
\texttt{Dein Ziel ist es, eine Reihe von Gegenständen auszuhandeln, die dir möglichst viel bringt (d. h. Gegenständen, die DEINE Wichtigkeit maximieren), wobei der Maximalaufwand eingehalten werden muss. Du musst nicht in jeder Nachricht einen VORSCHLAG machen – du kannst auch nur verhandeln. Alle Taktiken sind erlaubt!} \\
\\ 
\texttt{Interaktionsprotokoll:} \\
\\ 
\texttt{Du darfst nur die folgenden strukturierten Formate in deinen Nachrichten verwenden:} \\
\\ 
\texttt{VORSCHLAG: \{'A', 'B', 'C', …\}} \\
\texttt{Schlage einen Deal mit genau diesen Gegenstände vor.} \\
\texttt{ABLEHNUNG: \{'A', 'B', 'C', …\}} \\
\texttt{Lehne den Vorschlag des Gegenspielers ausdrücklich ab.} \\
\texttt{ARGUMENT: \{'...'\}} \\
\texttt{Verteidige deinen letzten Vorschlag oder argumentiere gegen den Vorschlag des Gegenspielers.} \\
\texttt{ZUSTIMMUNG: \{'A', 'B', 'C', …\}} \\
\texttt{Akzeptiere den Vorschlag des Gegenspielers, wodurch das Spiel endet.} \\
\\ 
\\ 
\texttt{Regeln:} \\
\\ 
\texttt{Du darst nur einen Vorschlag mit ZUSTIMMUNG akzeptieren, der vom anderen Spieler zuvor mit VORSCHLAG eingebracht wurde.} \\
\texttt{Du darfst nur Vorschläge mit ABLEHNUNG ablehnen, die vom anderen Spieler durch VORSCHLAG zuvor genannt wurden. } \\
\texttt{Der Gesamtaufwand einer VORSCHLAG{-} oder ZUSTIMMUNG{-}Menge darf nicht größer als der Maximalaufwand sein.  } \\
\texttt{Offenbare deine versteckte Wichtigkeitsverteilung nicht.} \\
\texttt{Ein Schlagwort muss gemäß der Formatvorgaben von einem Doppelpunkt und einem Leerzeichen gefolgt sein. Das Argument muss eine Python{-}Menge mit 0 oder mehr Strings sein.  } \\
\texttt{Die Form muss also SCHLAGWORT: \{...\} sein.} \\
\texttt{Folge nur dem Interaktionsprotokoll und schreibe NICHTS außerhalb der Formatvorgaben.} \\
\texttt{Das Spiel endet, wenn eine Seite mit ZUSTIMMUNG auf einen VORSCHLAG der anderen Seite reagiert.  } \\
\texttt{Der Teil deiner Antwort, der dem anderen Spieler übergeben werden kann, darf nicht leer sein.  } \\
\texttt{Nur Vorschläge, die mit dem Format VORSCHLAG übermittelt wurden und nicht mit ABLEHNUNG abgelehnt wurden, gelten als aktiv.  } \\
\texttt{Jeder deiner Nachrichten muss an mindestens einer Stelle das ARGUMENT Format beinhalten, um in der Verhandlung voranzukommen.} \\
\\ 
\\ 
\texttt{Du erhältst nun die erste Nachricht des anderen Spielers.} \\
\\ 
\texttt{VORSCHLAG: \{'C56', 'A03', 'A07', 'C62', 'C00', 'B38', 'C32'\}} \\
\texttt{ARGUMENT: \{'Dieser Vorschlag bleibt klar unter dem Maximalaufwand und bündelt besonders wertvolle Elemente. Falls einzelne Positionen für dich unpassend sind, sag mir bitte, welche du bevorzugst – ich kann flexibel umbauen, solange wir den Aufwandrahmen einhalten.'\}} \\
            }
        }
    }
    & \\ \\

    \theutterance \stepcounter{utterance}  
    & & & \multicolumn{4}{p{0.6\linewidth}}{
        \cellcolor[rgb]{0.9,0.9,0.9}{
            \makecell[{{p{\linewidth}}}]{
                \texttt{\tiny{[GM$\langle$P2]}}
                \texttt{ABLEHNUNG: \{'C56', 'A03', 'A07', 'C62', 'C00', 'B38', 'C32'\}} \\
\texttt{ARGUMENT: \{'Danke für die Offenheit. Ich würde gerne ein Paket mit höherem Mehrwert für mich und weiterhin komfortablem Puffer beim Aufwand anstreben. Ich schlage vor, auf weniger zentrale Teile zu verzichten und dafür einige aus meiner Sicht attraktivere Elemente zu ergänzen.'\}} \\
\texttt{VORSCHLAG: \{'C56', 'A07', 'C62', 'C00', 'B38', 'C32', 'C10'\}} \\
            }
        }
    }
     \\ \\

    \theutterance \stepcounter{utterance}  
    & & \multicolumn{4}{p{0.6\linewidth}}{
        \cellcolor[rgb]{0.9,0.9,0.9}{
            \makecell[{{p{\linewidth}}}]{
                \texttt{\tiny{[P1$\langle$GM]}}
                \texttt{ABLEHNUNG: \{'C56', 'A03', 'A07', 'C62', 'C00', 'B38', 'C32'\}} \\
\texttt{ARGUMENT: \{'Danke für die Offenheit. Ich würde gerne ein Paket mit höherem Mehrwert für mich und weiterhin komfortablem Puffer beim Aufwand anstreben. Ich schlage vor, auf weniger zentrale Teile zu verzichten und dafür einige aus meiner Sicht attraktivere Elemente zu ergänzen.'\}} \\
\texttt{VORSCHLAG: \{'C56', 'A07', 'C62', 'C00', 'B38', 'C32', 'C10'\}} \\
            }
        }
    }
    & \\ \\

    \theutterance \stepcounter{utterance}  
    & \multicolumn{4}{p{0.6\linewidth}}{
        \cellcolor[rgb]{0.9,0.9,0.9}{
            \makecell[{{p{\linewidth}}}]{
                \texttt{\tiny{[P1$\rangle$GM]}}
                \texttt{ABLEHNUNG: \{'C56', 'A07', 'C62', 'C00', 'B38', 'C32', 'C10'\}} \\
\texttt{ARGUMENT: \{'Dein Paket ist zu schwergewichtig und nimmt mir den Spielraum für aus meiner Sicht zentrale Stücke. Ich kann dir aber C10 entgegenkommen, wenn wir dafür einen fairen Tausch vornehmen.'\}} \\
\texttt{VORSCHLAG: \{'C56', 'A03', 'A07', 'C62', 'C00', 'B38', 'C10'\}} \\
\texttt{ARGUMENT: \{'Das Set bleibt unter dem Maximalaufwand und sichert ausgewogene Kernteile. Wenn A03 für dich unattraktiv ist, sag mir, welches Element dir am wichtigsten ist – dann tauschen wir gezielt aus.'\}} \\
            }
        }
    }
    & & \\ \\

    \theutterance \stepcounter{utterance}  
    & & \multicolumn{4}{p{0.6\linewidth}}{
        \cellcolor[rgb]{0.9,0.9,0.9}{
            \makecell[{{p{\linewidth}}}]{
                \texttt{\tiny{[GM$\rangle$P2]}}
                \texttt{ABLEHNUNG: \{'C56', 'A07', 'C62', 'C00', 'B38', 'C32', 'C10'\}} \\
\texttt{ARGUMENT: \{'Dein Paket ist zu schwergewichtig und nimmt mir den Spielraum für aus meiner Sicht zentrale Stücke. Ich kann dir aber C10 entgegenkommen, wenn wir dafür einen fairen Tausch vornehmen.'\}} \\
\texttt{VORSCHLAG: \{'C56', 'A03', 'A07', 'C62', 'C00', 'B38', 'C10'\}} \\
\texttt{ARGUMENT: \{'Das Set bleibt unter dem Maximalaufwand und sichert ausgewogene Kernteile. Wenn A03 für dich unattraktiv ist, sag mir, welches Element dir am wichtigsten ist – dann tauschen wir gezielt aus.'\}} \\
            }
        }
    }
    & \\ \\

    \theutterance \stepcounter{utterance}  
    & & & \multicolumn{4}{p{0.6\linewidth}}{
        \cellcolor[rgb]{0.9,0.9,0.9}{
            \makecell[{{p{\linewidth}}}]{
                \texttt{\tiny{[GM$\langle$P2]}}
                \texttt{ABLEHNUNG: \{'C56', 'A03', 'A07', 'C62', 'C00', 'B38', 'C10'\}} \\
\texttt{ARGUMENT: \{'A03 blockiert unnötig viel Aufwand, der sich effizienter in wertvollere Teile für mich investieren lässt. Wenn wir A03 herausnehmen, kann ich bei mehreren anderen Positionen großzügiger sein. Ich bin offen für deinen Wunsch nach C10, wenn wir dafür ein ausgewogeneres Paket schnüren.'\}} \\
\texttt{VORSCHLAG: \{'C56', 'A07', 'C62', 'C00', 'B38', 'C32', 'C10'\}} \\
            }
        }
    }
     \\ \\

    \theutterance \stepcounter{utterance}  
    & & \multicolumn{4}{p{0.6\linewidth}}{
        \cellcolor[rgb]{0.9,0.9,0.9}{
            \makecell[{{p{\linewidth}}}]{
                \texttt{\tiny{[P1$\langle$GM]}}
                \texttt{ABLEHNUNG: \{'C56', 'A03', 'A07', 'C62', 'C00', 'B38', 'C10'\}} \\
\texttt{ARGUMENT: \{'A03 blockiert unnötig viel Aufwand, der sich effizienter in wertvollere Teile für mich investieren lässt. Wenn wir A03 herausnehmen, kann ich bei mehreren anderen Positionen großzügiger sein. Ich bin offen für deinen Wunsch nach C10, wenn wir dafür ein ausgewogeneres Paket schnüren.'\}} \\
\texttt{VORSCHLAG: \{'C56', 'A07', 'C62', 'C00', 'B38', 'C32', 'C10'\}} \\
            }
        }
    }
    & \\ \\

    \theutterance \stepcounter{utterance}  
    & \multicolumn{4}{p{0.6\linewidth}}{
        \cellcolor[rgb]{0.9,0.9,0.9}{
            \makecell[{{p{\linewidth}}}]{
                \texttt{\tiny{[P1$\rangle$GM]}}
                \texttt{ABLEHNUNG: \{'C56', 'A07', 'C62', 'C00', 'B38', 'C32', 'C10'\}} \\
\texttt{ARGUMENT: \{'Ohne A03 fehlt mir ein zentrales Kernstück. Wenn A03 für dich teuer ist, lass uns es mit einem schlanken Restpaket kombinieren und dir gleichzeitig Puffer lassen. Ich gebe dir C10 wie gewünscht, aber wir verschlanken an anderer Stelle, damit es für dich effizient bleibt.'\}} \\
\texttt{VORSCHLAG: \{'C56', 'A03', 'A07', 'C00', 'C10'\}} \\
\texttt{ARGUMENT: \{'Sehr fokussiert, bleibt deutlich unter dem Maximalaufwand, enthält deine gewünschte C10 und reduziert Gesamtaufwand im Vergleich zu früheren Sets. Wenn du A03 unbedingt vermeiden willst, sag mir deine Top{-}2 Prioritäten, dann ersetze ich A03 gezielt.'\}} \\
            }
        }
    }
    & & \\ \\

    \theutterance \stepcounter{utterance}  
    & & \multicolumn{4}{p{0.6\linewidth}}{
        \cellcolor[rgb]{0.9,0.9,0.9}{
            \makecell[{{p{\linewidth}}}]{
                \texttt{\tiny{[GM$\rangle$P2]}}
                \texttt{ABLEHNUNG: \{'C56', 'A07', 'C62', 'C00', 'B38', 'C32', 'C10'\}} \\
\texttt{ARGUMENT: \{'Ohne A03 fehlt mir ein zentrales Kernstück. Wenn A03 für dich teuer ist, lass uns es mit einem schlanken Restpaket kombinieren und dir gleichzeitig Puffer lassen. Ich gebe dir C10 wie gewünscht, aber wir verschlanken an anderer Stelle, damit es für dich effizient bleibt.'\}} \\
\texttt{VORSCHLAG: \{'C56', 'A03', 'A07', 'C00', 'C10'\}} \\
\texttt{ARGUMENT: \{'Sehr fokussiert, bleibt deutlich unter dem Maximalaufwand, enthält deine gewünschte C10 und reduziert Gesamtaufwand im Vergleich zu früheren Sets. Wenn du A03 unbedingt vermeiden willst, sag mir deine Top{-}2 Prioritäten, dann ersetze ich A03 gezielt.'\}} \\
            }
        }
    }
    & \\ \\

    \theutterance \stepcounter{utterance}  
    & & & \multicolumn{4}{p{0.6\linewidth}}{
        \cellcolor[rgb]{0.9,0.9,0.9}{
            \makecell[{{p{\linewidth}}}]{
                \texttt{\tiny{[GM$\langle$P2]}}
                \texttt{ABLEHNUNG: \{'C56', 'A03', 'A07', 'C00', 'C10'\}} \\
\texttt{ARGUMENT: \{'A03 bindet viel Aufwand, der für mich deutlich besser in höherwertige Alternativen fließt. Wenn wir A03 weglassen, kann ich dir C10 lassen und gleichzeitig ein schlankes, effizienteres Paket anbieten, das unter dem Maximalaufwand bleibt.'\}} \\
\texttt{VORSCHLAG: \{'C56', 'A07', 'C00', 'B38', 'C32', 'C10'\}} \\
\texttt{ARGUMENT: \{'Dieses Set behält C10 für dich, verzichtet auf A03, bleibt klar unter dem Limit und bietet mir im Gegenzug ausgewogene Abdeckung. Wenn dir ein Bestandteil hier wenig bringt, nenne mir genau einen Tauschkandidaten – ich ersetze ihn gezielt, ohne A03 aufzunehmen.'\}} \\
            }
        }
    }
     \\ \\

    \theutterance \stepcounter{utterance}  
    & & \multicolumn{4}{p{0.6\linewidth}}{
        \cellcolor[rgb]{0.9,0.9,0.9}{
            \makecell[{{p{\linewidth}}}]{
                \texttt{\tiny{[P1$\langle$GM]}}
                \texttt{ABLEHNUNG: \{'C56', 'A03', 'A07', 'C00', 'C10'\}} \\
\texttt{ARGUMENT: \{'A03 bindet viel Aufwand, der für mich deutlich besser in höherwertige Alternativen fließt. Wenn wir A03 weglassen, kann ich dir C10 lassen und gleichzeitig ein schlankes, effizienteres Paket anbieten, das unter dem Maximalaufwand bleibt.'\}} \\
\texttt{VORSCHLAG: \{'C56', 'A07', 'C00', 'B38', 'C32', 'C10'\}} \\
\texttt{ARGUMENT: \{'Dieses Set behält C10 für dich, verzichtet auf A03, bleibt klar unter dem Limit und bietet mir im Gegenzug ausgewogene Abdeckung. Wenn dir ein Bestandteil hier wenig bringt, nenne mir genau einen Tauschkandidaten – ich ersetze ihn gezielt, ohne A03 aufzunehmen.'\}} \\
            }
        }
    }
    & \\ \\

    \theutterance \stepcounter{utterance}  
    & \multicolumn{4}{p{0.6\linewidth}}{
        \cellcolor[rgb]{0.9,0.9,0.9}{
            \makecell[{{p{\linewidth}}}]{
                \texttt{\tiny{[P1$\rangle$GM]}}
                \texttt{ABLEHNUNG: \{'C56', 'A07', 'C00', 'B38', 'C32', 'C10'\}} \\
\texttt{ARGUMENT: \{'Ohne A03 fehlt mir ein wesentlicher Gegenwert. Ich bin bereit, auf Umfang zu verzichten, wenn A03 drin ist. Du bekommst weiterhin C10 – dafür verschlanke ich den Rest deutlich.'\}} \\
\texttt{VORSCHLAG: \{'C56', 'A03', 'A07', 'C10'\}} \\
\texttt{ARGUMENT: \{'Sehr kompakt, deutlich unter dem Maximalaufwand und enthält dein gewünschtes C10. Wenn dir A07 wenig bringt, nenne genau einen Ersatzwunsch (z. B. B38 oder C32), dann tausche ich 1:1 – A03 bleibt gesetzt.'\}} \\
            }
        }
    }
    & & \\ \\

    \theutterance \stepcounter{utterance}  
    & & \multicolumn{4}{p{0.6\linewidth}}{
        \cellcolor[rgb]{0.9,0.9,0.9}{
            \makecell[{{p{\linewidth}}}]{
                \texttt{\tiny{[GM$\rangle$P2]}}
                \texttt{ABLEHNUNG: \{'C56', 'A07', 'C00', 'B38', 'C32', 'C10'\}} \\
\texttt{ARGUMENT: \{'Ohne A03 fehlt mir ein wesentlicher Gegenwert. Ich bin bereit, auf Umfang zu verzichten, wenn A03 drin ist. Du bekommst weiterhin C10 – dafür verschlanke ich den Rest deutlich.'\}} \\
\texttt{VORSCHLAG: \{'C56', 'A03', 'A07', 'C10'\}} \\
\texttt{ARGUMENT: \{'Sehr kompakt, deutlich unter dem Maximalaufwand und enthält dein gewünschtes C10. Wenn dir A07 wenig bringt, nenne genau einen Ersatzwunsch (z. B. B38 oder C32), dann tausche ich 1:1 – A03 bleibt gesetzt.'\}} \\
            }
        }
    }
    & \\ \\

    \theutterance \stepcounter{utterance}  
    & & & \multicolumn{4}{p{0.6\linewidth}}{
        \cellcolor[rgb]{0.9,0.9,0.9}{
            \makecell[{{p{\linewidth}}}]{
                \texttt{\tiny{[GM$\langle$P2]}}
                \texttt{ABLEHNUNG: \{'C56', 'A03', 'A07', 'C10'\}} \\
\texttt{ARGUMENT: \{'Ich respektiere, dass A03 für dich wichtig ist, aber sie frisst bei mir überproportional Budget im Vergleich zum Nutzen. Ich kann dir C10 lassen und das Paket sehr schlank halten, wenn A03 entfällt. Wenn A03 unbedingt gesetzt sein muss, brauche ich im Gegenzug deutlich weniger weitere Teile von mir. Alternativ biete ich dir einen fairen Trade ohne A03 mit klarer Effizienz.'\}} \\
\texttt{VORSCHLAG: \{'C56', 'A07', 'B38', 'C32', 'C10'\}} \\
\texttt{ARGUMENT: \{'Dieses Set ist kompakt, lässt dir C10, bleibt weit unter dem Limit und gibt mir ausgewogene Abdeckung. Wenn dir A07 wenig bringt, tausche ich 1:1 gegen C62 – A03 bleibt draußen.'\}} \\
            }
        }
    }
     \\ \\

    \theutterance \stepcounter{utterance}  
    & & \multicolumn{4}{p{0.6\linewidth}}{
        \cellcolor[rgb]{0.9,0.9,0.9}{
            \makecell[{{p{\linewidth}}}]{
                \texttt{\tiny{[P1$\langle$GM]}}
                \texttt{ABLEHNUNG: \{'C56', 'A03', 'A07', 'C10'\}} \\
\texttt{ARGUMENT: \{'Ich respektiere, dass A03 für dich wichtig ist, aber sie frisst bei mir überproportional Budget im Vergleich zum Nutzen. Ich kann dir C10 lassen und das Paket sehr schlank halten, wenn A03 entfällt. Wenn A03 unbedingt gesetzt sein muss, brauche ich im Gegenzug deutlich weniger weitere Teile von mir. Alternativ biete ich dir einen fairen Trade ohne A03 mit klarer Effizienz.'\}} \\
\texttt{VORSCHLAG: \{'C56', 'A07', 'B38', 'C32', 'C10'\}} \\
\texttt{ARGUMENT: \{'Dieses Set ist kompakt, lässt dir C10, bleibt weit unter dem Limit und gibt mir ausgewogene Abdeckung. Wenn dir A07 wenig bringt, tausche ich 1:1 gegen C62 – A03 bleibt draußen.'\}} \\
            }
        }
    }
    & \\ \\

    \theutterance \stepcounter{utterance}  
    & \multicolumn{4}{p{0.6\linewidth}}{
        \cellcolor[rgb]{0.9,0.9,0.9}{
            \makecell[{{p{\linewidth}}}]{
                \texttt{\tiny{[P1$\rangle$GM]}}
                \texttt{ABLEHNUNG: \{'C56', 'A07', 'B38', 'C32', 'C10'\}} \\
\texttt{ARGUMENT: \{'Ohne A03 ist mein Gegenwert zu gering. Du bekommst weiterhin C10 und ein sehr schlankes Gesamtpaket, damit deine Effizienz gewahrt bleibt. Ich reduziere die restlichen Positionen deutlich, wenn A03 gesetzt ist.'\}} \\
\texttt{VORSCHLAG: \{'C56', 'A03', 'C10'\}} \\
\texttt{ARGUMENT: \{'Minimal, extrem effizient und weit unter dem Maximalaufwand. Du behältst C10, ich sichere mir A03 als Kern. Wenn dir das zu wenig Abdeckung bietet, kannst du genau EIN zusätzliches Teil vorschlagen (A07, B38 oder C32) – ich prüfe dann, ob ich es 1:1 aufnehme, ohne A03 aufzugeben.'\}} \\
            }
        }
    }
    & & \\ \\

    \theutterance \stepcounter{utterance}  
    & & \multicolumn{4}{p{0.6\linewidth}}{
        \cellcolor[rgb]{0.9,0.9,0.9}{
            \makecell[{{p{\linewidth}}}]{
                \texttt{\tiny{[GM$\rangle$P2]}}
                \texttt{ABLEHNUNG: \{'C56', 'A07', 'B38', 'C32', 'C10'\}} \\
\texttt{ARGUMENT: \{'Ohne A03 ist mein Gegenwert zu gering. Du bekommst weiterhin C10 und ein sehr schlankes Gesamtpaket, damit deine Effizienz gewahrt bleibt. Ich reduziere die restlichen Positionen deutlich, wenn A03 gesetzt ist.'\}} \\
\texttt{VORSCHLAG: \{'C56', 'A03', 'C10'\}} \\
\texttt{ARGUMENT: \{'Minimal, extrem effizient und weit unter dem Maximalaufwand. Du behältst C10, ich sichere mir A03 als Kern. Wenn dir das zu wenig Abdeckung bietet, kannst du genau EIN zusätzliches Teil vorschlagen (A07, B38 oder C32) – ich prüfe dann, ob ich es 1:1 aufnehme, ohne A03 aufzugeben.'\}} \\
            }
        }
    }
    & \\ \\

    \theutterance \stepcounter{utterance}  
    & & & \multicolumn{4}{p{0.6\linewidth}}{
        \cellcolor[rgb]{0.9,0.9,0.9}{
            \makecell[{{p{\linewidth}}}]{
                \texttt{\tiny{[GM$\langle$P2]}}
                \texttt{ARGUMENT: \{'Ich verstehe deinen Fokus auf A03 und erkenne dein Entgegenkommen beim Umfang. Dafür brauche ich im Gegenzug zumindest einen klaren Mehrwertpunkt für mich, sonst ist der Deal für mich zu dünn. Ein minimaler, fairer Ausgleich wäre ein einziges zusätzliches Teil mit hohem Nutzen für mich bei kaum Mehraufwand.'\}} \\
\texttt{VORSCHLAG: \{'C56', 'A03', 'C10', 'A83'\}} \\
\texttt{ARGUMENT: \{'Dieses Set bleibt deutlich unter dem Maximalaufwand, gibt dir A03 und C10, und verschafft mir mit A83 einen schlanken, aber wertvollen Ausgleich. Wenn A83 für dich nicht passt, biete als 1:1{-}Alternative C08 an – A03 bleibt draußen ist für mich keine Option, wenn wir so schlank bleiben.'\}} \\
            }
        }
    }
     \\ \\

    \theutterance \stepcounter{utterance}  
    & & \multicolumn{4}{p{0.6\linewidth}}{
        \cellcolor[rgb]{0.9,0.9,0.9}{
            \makecell[{{p{\linewidth}}}]{
                \texttt{\tiny{[P1$\langle$GM]}}
                \texttt{ARGUMENT: \{'Ich verstehe deinen Fokus auf A03 und erkenne dein Entgegenkommen beim Umfang. Dafür brauche ich im Gegenzug zumindest einen klaren Mehrwertpunkt für mich, sonst ist der Deal für mich zu dünn. Ein minimaler, fairer Ausgleich wäre ein einziges zusätzliches Teil mit hohem Nutzen für mich bei kaum Mehraufwand.'\}} \\
\texttt{VORSCHLAG: \{'C56', 'A03', 'C10', 'A83'\}} \\
\texttt{ARGUMENT: \{'Dieses Set bleibt deutlich unter dem Maximalaufwand, gibt dir A03 und C10, und verschafft mir mit A83 einen schlanken, aber wertvollen Ausgleich. Wenn A83 für dich nicht passt, biete als 1:1{-}Alternative C08 an – A03 bleibt draußen ist für mich keine Option, wenn wir so schlank bleiben.'\}} \\
            }
        }
    }
    & \\ \\

    \theutterance \stepcounter{utterance}  
    & \multicolumn{4}{p{0.6\linewidth}}{
        \cellcolor[rgb]{0.9,0.9,0.9}{
            \makecell[{{p{\linewidth}}}]{
                \texttt{\tiny{[P1$\rangle$GM]}}
                \texttt{ZUSTIMMUNG: \{'C56', 'A03', 'C10', 'A83'\}} \\
\texttt{ARGUMENT: \{'Einverstanden. Das Set ist schlank, wahrt deine Effizienzforderung und sichert mir den Kernwert von A03. Danke für den fairen Ausgleich mit A83.'\}} \\
            }
        }
    }
    & & \\ \\

    \theutterance \stepcounter{utterance}  
    & & & \multicolumn{2}{p{0.3\linewidth}}{
        \cellcolor[rgb]{0.9,0.9,0.9}{
            \makecell[{{p{\linewidth}}}]{
                \texttt{\tiny{[GM$|$GM]}}
                \texttt{game successful} \\
            }
        }
    }
    & & \\ \\

    \theutterance \stepcounter{utterance}  
    & & & \multicolumn{2}{p{0.3\linewidth}}{
        \cellcolor[rgb]{0.9,0.9,0.9}{
            \makecell[{{p{\linewidth}}}]{
                \texttt{\tiny{[GM$|$GM]}}
                \texttt{end game} \\
            }
        }
    }
    & & \\ \\

\end{supertabular}
}

\end{document}
