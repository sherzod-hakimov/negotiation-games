
\documentclass{article}
\usepackage{colortbl}
\usepackage{makecell}
\usepackage{multirow}
\usepackage{supertabular}

\begin{document}

\newcounter{utterance}

\centering \large Interaction Transcript for game `hot\_air\_balloon', experiment `air\_balloon\_survival\_de\_reasoning off\_hard', episode 3 with gpt{-}5{-}mini{-}2025{-}08{-}07{-}no{-}reasoning{-}t1.0.
\vspace{24pt}

{ \footnotesize  \setcounter{utterance}{1}
\setlength{\tabcolsep}{0pt}
\begin{supertabular}{c@{$\;$}|p{.15\linewidth}@{}p{.15\linewidth}p{.15\linewidth}p{.15\linewidth}p{.15\linewidth}p{.15\linewidth}}
    \# & $\;$A & \multicolumn{4}{c}{Game Master} & $\;\:$B\\
    \hline

    \theutterance \stepcounter{utterance}  
    & & \multicolumn{4}{p{0.6\linewidth}}{
        \cellcolor[rgb]{0.9,0.9,0.9}{
            \makecell[{{p{\linewidth}}}]{
                \texttt{\tiny{[P1$\langle$GM]}}
                \texttt{Du nimmst an einem kollaborativen Verhandlungspiel Teil.} \\
\\ 
\texttt{Zusammen mit einem anderen Teilnehmer musst du dich auf eine Reihe von Gegenständen entscheiden, die behalten werden. Jeder von euch hat eine persönliche Verteilung über die Wichtigkeit der einzelnen Gegenstände. Jeder von euch hat eine eigene Meinung darüber, wie wichtig jeder einzelne Gegenstand ist (Gegenstandswichtigkeit). Du kennst die Wichtigkeitsverteilung des anderen Spielers nicht. Zusätzlich siehst du, wie viel Aufwand jeder Gegenstand verursacht.  } \\
\texttt{Ihr dürft euch nur auf eine Reihe von Gegenständen einigen, wenn der Gesamtaufwand der ausgewählten Gegenstände den Maximalaufwand nicht überschreitet:} \\
\\ 
\texttt{Maximalaufwand = 3629} \\
\\ 
\texttt{Hier sind die einzelnen Aufwände der Gegenstände:} \\
\\ 
\texttt{Aufwand der Gegenstände = \{"C76": 5, "C38": 793, "C56": 21, "A03": 842, "A07": 313, "A83": 842, "C98": 631, "C08": 226, "C62": 87, "C00": 761, "C10": 230, "C81": 287, "B38": 698, "C03": 641, "C32": 882\}} \\
\\ 
\texttt{Hier ist deine persönliche Verteilung der Wichtigkeit der einzelnen Gegenstände:} \\
\\ 
\texttt{Werte der Gegenstandswichtigkeit = \{"C76": 138, "C38": 583, "C56": 868, "A03": 822, "A07": 783, "A83": 65, "C98": 262, "C08": 121, "C62": 508, "C00": 780, "C10": 461, "C81": 484, "B38": 668, "C03": 389, "C32": 808\}} \\
\\ 
\texttt{Ziel:} \\
\\ 
\texttt{Dein Ziel ist es, eine Reihe von Gegenständen auszuhandeln, die dir möglichst viel bringt (d. h. Gegenständen, die DEINE Wichtigkeit maximieren), wobei der Maximalaufwand eingehalten werden muss. Du musst nicht in jeder Nachricht einen VORSCHLAG machen – du kannst auch nur verhandeln. Alle Taktiken sind erlaubt!} \\
\\ 
\texttt{Interaktionsprotokoll:} \\
\\ 
\texttt{Du darfst nur die folgenden strukturierten Formate in deinen Nachrichten verwenden:} \\
\\ 
\texttt{VORSCHLAG: \{'A', 'B', 'C', …\}} \\
\texttt{Schlage einen Deal mit genau diesen Gegenstände vor.} \\
\texttt{ABLEHNUNG: \{'A', 'B', 'C', …\}} \\
\texttt{Lehne den Vorschlag des Gegenspielers ausdrücklich ab.} \\
\texttt{ARGUMENT: \{'...'\}} \\
\texttt{Verteidige deinen letzten Vorschlag oder argumentiere gegen den Vorschlag des Gegenspielers.} \\
\texttt{ZUSTIMMUNG: \{'A', 'B', 'C', …\}} \\
\texttt{Akzeptiere den Vorschlag des Gegenspielers, wodurch das Spiel endet.} \\
\\ 
\\ 
\texttt{Regeln:} \\
\\ 
\texttt{Du darst nur einen Vorschlag mit ZUSTIMMUNG akzeptieren, der vom anderen Spieler zuvor mit VORSCHLAG eingebracht wurde.} \\
\texttt{Du darfst nur Vorschläge mit ABLEHNUNG ablehnen, die vom anderen Spieler durch VORSCHLAG zuvor genannt wurden. } \\
\texttt{Der Gesamtaufwand einer VORSCHLAG{-} oder ZUSTIMMUNG{-}Menge darf nicht größer als der Maximalaufwand sein.  } \\
\texttt{Offenbare deine versteckte Wichtigkeitsverteilung nicht.} \\
\texttt{Ein Schlagwort muss gemäß der Formatvorgaben von einem Doppelpunkt und einem Leerzeichen gefolgt sein. Das Argument muss eine Python{-}Menge mit 0 oder mehr Strings sein.  } \\
\texttt{Die Form muss also SCHLAGWORT: \{...\} sein.} \\
\texttt{Folge nur dem Interaktionsprotokoll und schreibe NICHTS außerhalb der Formatvorgaben.} \\
\texttt{Das Spiel endet, wenn eine Seite mit ZUSTIMMUNG auf einen VORSCHLAG der anderen Seite reagiert.  } \\
\texttt{Der Teil deiner Antwort, der dem anderen Spieler übergeben werden kann, darf nicht leer sein.  } \\
\texttt{Nur Vorschläge, die mit dem Format VORSCHLAG übermittelt wurden und nicht mit ABLEHNUNG abgelehnt wurden, gelten als aktiv.  } \\
\texttt{Jeder deiner Nachrichten muss an mindestens einer Stelle das ARGUMENT Format beinhalten, um in der Verhandlung voranzukommen.} \\
            }
        }
    }
    & \\ \\

    \theutterance \stepcounter{utterance}  
    & \multicolumn{4}{p{0.6\linewidth}}{
        \cellcolor[rgb]{0.9,0.9,0.9}{
            \makecell[{{p{\linewidth}}}]{
                \texttt{\tiny{[P1$\rangle$GM]}}
                \texttt{VORSCHLAG: \{'C56', 'A03', 'A07', 'C62', 'C10', 'C81', 'B38'\}} \\
\texttt{ARGUMENT: \{'Ich schlage diese Auswahl vor, weil sie einen starken Nutzen bei moderatem Aufwand liefert und noch Puffer für gemeinsame Prioritäten lässt. Gesamtaufwand: 21+842+313+87+230+287+698 = 2478 (unter 3629). Damit bleibt Spielraum von 1151 für Ergänzungen, die dir wichtig sind. Wenn du Präferenzen teilst, kann ich den Vorschlag anpassen, z. B. ein Tausch innerhalb des verbleibenden Budgets.'\}} \\
            }
        }
    }
    & & \\ \\

    \theutterance \stepcounter{utterance}  
    & & \multicolumn{4}{p{0.6\linewidth}}{
        \cellcolor[rgb]{0.9,0.9,0.9}{
            \makecell[{{p{\linewidth}}}]{
                \texttt{\tiny{[GM$\rangle$P2]}}
                \texttt{Du nimmst an einem kollaborativen Verhandlungspiel Teil.} \\
\\ 
\texttt{Zusammen mit einem anderen Teilnehmer musst du dich auf eine Reihe von Gegenständen entscheiden, die behalten werden. Jeder von euch hat eine persönliche Verteilung über die Wichtigkeit der einzelnen Gegenstände. Jeder von euch hat eine eigene Meinung darüber, wie wichtig jeder einzelne Gegenstand ist (Gegenstandswichtigkeit). Du kennst die Wichtigkeitsverteilung des anderen Spielers nicht. Zusätzlich siehst du, wie viel Aufwand jeder Gegenstand verursacht.  } \\
\texttt{Ihr dürft euch nur auf eine Reihe von Gegenständen einigen, wenn der Gesamtaufwand der ausgewählten Gegenstände den Maximalaufwand nicht überschreitet:} \\
\\ 
\texttt{Maximalaufwand = 3629} \\
\\ 
\texttt{Hier sind die einzelnen Aufwände der Gegenstände:} \\
\\ 
\texttt{Aufwand der Gegenstände = \{"C76": 5, "C38": 793, "C56": 21, "A03": 842, "A07": 313, "A83": 842, "C98": 631, "C08": 226, "C62": 87, "C00": 761, "C10": 230, "C81": 287, "B38": 698, "C03": 641, "C32": 882\}} \\
\\ 
\texttt{Hier ist deine persönliche Verteilung der Wichtigkeit der einzelnen Gegenstände:} \\
\\ 
\texttt{Werte der Gegenstandswichtigkeit = \{"C56": 65, "A03": 121, "C32": 138, "A07": 262, "C00": 389, "B38": 461, "C38": 484, "C62": 508, "C81": 583, "C10": 668, "C03": 780, "C98": 783, "C76": 808, "C08": 822, "A83": 868\}} \\
\\ 
\texttt{Ziel:} \\
\\ 
\texttt{Dein Ziel ist es, eine Reihe von Gegenständen auszuhandeln, die dir möglichst viel bringt (d. h. Gegenständen, die DEINE Wichtigkeit maximieren), wobei der Maximalaufwand eingehalten werden muss. Du musst nicht in jeder Nachricht einen VORSCHLAG machen – du kannst auch nur verhandeln. Alle Taktiken sind erlaubt!} \\
\\ 
\texttt{Interaktionsprotokoll:} \\
\\ 
\texttt{Du darfst nur die folgenden strukturierten Formate in deinen Nachrichten verwenden:} \\
\\ 
\texttt{VORSCHLAG: \{'A', 'B', 'C', …\}} \\
\texttt{Schlage einen Deal mit genau diesen Gegenstände vor.} \\
\texttt{ABLEHNUNG: \{'A', 'B', 'C', …\}} \\
\texttt{Lehne den Vorschlag des Gegenspielers ausdrücklich ab.} \\
\texttt{ARGUMENT: \{'...'\}} \\
\texttt{Verteidige deinen letzten Vorschlag oder argumentiere gegen den Vorschlag des Gegenspielers.} \\
\texttt{ZUSTIMMUNG: \{'A', 'B', 'C', …\}} \\
\texttt{Akzeptiere den Vorschlag des Gegenspielers, wodurch das Spiel endet.} \\
\\ 
\\ 
\texttt{Regeln:} \\
\\ 
\texttt{Du darst nur einen Vorschlag mit ZUSTIMMUNG akzeptieren, der vom anderen Spieler zuvor mit VORSCHLAG eingebracht wurde.} \\
\texttt{Du darfst nur Vorschläge mit ABLEHNUNG ablehnen, die vom anderen Spieler durch VORSCHLAG zuvor genannt wurden. } \\
\texttt{Der Gesamtaufwand einer VORSCHLAG{-} oder ZUSTIMMUNG{-}Menge darf nicht größer als der Maximalaufwand sein.  } \\
\texttt{Offenbare deine versteckte Wichtigkeitsverteilung nicht.} \\
\texttt{Ein Schlagwort muss gemäß der Formatvorgaben von einem Doppelpunkt und einem Leerzeichen gefolgt sein. Das Argument muss eine Python{-}Menge mit 0 oder mehr Strings sein.  } \\
\texttt{Die Form muss also SCHLAGWORT: \{...\} sein.} \\
\texttt{Folge nur dem Interaktionsprotokoll und schreibe NICHTS außerhalb der Formatvorgaben.} \\
\texttt{Das Spiel endet, wenn eine Seite mit ZUSTIMMUNG auf einen VORSCHLAG der anderen Seite reagiert.  } \\
\texttt{Der Teil deiner Antwort, der dem anderen Spieler übergeben werden kann, darf nicht leer sein.  } \\
\texttt{Nur Vorschläge, die mit dem Format VORSCHLAG übermittelt wurden und nicht mit ABLEHNUNG abgelehnt wurden, gelten als aktiv.  } \\
\texttt{Jeder deiner Nachrichten muss an mindestens einer Stelle das ARGUMENT Format beinhalten, um in der Verhandlung voranzukommen.} \\
\\ 
\\ 
\texttt{Du erhältst nun die erste Nachricht des anderen Spielers.} \\
\\ 
\texttt{VORSCHLAG: \{'C56', 'A03', 'A07', 'C62', 'C10', 'C81', 'B38'\}} \\
\texttt{ARGUMENT: \{'Ich schlage diese Auswahl vor, weil sie einen starken Nutzen bei moderatem Aufwand liefert und noch Puffer für gemeinsame Prioritäten lässt. Gesamtaufwand: 21+842+313+87+230+287+698 = 2478 (unter 3629). Damit bleibt Spielraum von 1151 für Ergänzungen, die dir wichtig sind. Wenn du Präferenzen teilst, kann ich den Vorschlag anpassen, z. B. ein Tausch innerhalb des verbleibenden Budgets.'\}} \\
            }
        }
    }
    & \\ \\

    \theutterance \stepcounter{utterance}  
    & & & \multicolumn{4}{p{0.6\linewidth}}{
        \cellcolor[rgb]{0.9,0.9,0.9}{
            \makecell[{{p{\linewidth}}}]{
                \texttt{\tiny{[GM$\langle$P2]}}
                \texttt{ABLEHNUNG: \{'C56', 'A03', 'A07', 'C62', 'C10', 'C81', 'B38'\}} \\
\texttt{ARGUMENT: \{'Danke für den Start. Ich sehe den Puffer – ich würde gern auf höherwertige Items für mich umsteuern und den Aufwand weiter effizient nutzen. Ich schlage vor, einige deiner Picks zu tauschen gegen Items, die mir deutlich wichtiger sind, ohne das Budget zu sprengen.'\}} \\
\texttt{VORSCHLAG: \{'C76', 'C08', 'A83', 'C98', 'C03', 'C10'\}} \\
\texttt{ARGUMENT: \{'Begründung und Aufwand: C76(5)+C08(226)+A83(842)+C98(631)+C03(641)+C10(230)=2575 (unter 3629). Das lässt 1054 Rest. Diese Auswahl konzentriert sich auf für mich zentrale Stücke, bleibt aber offen: Wenn dir A07 oder B38 wichtig sind, können wir sie mit dem Restbudget ergänzen, ggf. durch Austausch (z. B. gegen A83 oder C98), solange wir unter 3629 bleiben. Welche 1–2 Items aus deinem ursprünglichen Set sind dir unverzichtbar?'\}} \\
            }
        }
    }
     \\ \\

    \theutterance \stepcounter{utterance}  
    & & \multicolumn{4}{p{0.6\linewidth}}{
        \cellcolor[rgb]{0.9,0.9,0.9}{
            \makecell[{{p{\linewidth}}}]{
                \texttt{\tiny{[P1$\langle$GM]}}
                \texttt{ABLEHNUNG: \{'C56', 'A03', 'A07', 'C62', 'C10', 'C81', 'B38'\}} \\
\texttt{ARGUMENT: \{'Danke für den Start. Ich sehe den Puffer – ich würde gern auf höherwertige Items für mich umsteuern und den Aufwand weiter effizient nutzen. Ich schlage vor, einige deiner Picks zu tauschen gegen Items, die mir deutlich wichtiger sind, ohne das Budget zu sprengen.'\}} \\
\texttt{VORSCHLAG: \{'C76', 'C08', 'A83', 'C98', 'C03', 'C10'\}} \\
\texttt{ARGUMENT: \{'Begründung und Aufwand: C76(5)+C08(226)+A83(842)+C98(631)+C03(641)+C10(230)=2575 (unter 3629). Das lässt 1054 Rest. Diese Auswahl konzentriert sich auf für mich zentrale Stücke, bleibt aber offen: Wenn dir A07 oder B38 wichtig sind, können wir sie mit dem Restbudget ergänzen, ggf. durch Austausch (z. B. gegen A83 oder C98), solange wir unter 3629 bleiben. Welche 1–2 Items aus deinem ursprünglichen Set sind dir unverzichtbar?'\}} \\
            }
        }
    }
    & \\ \\

    \theutterance \stepcounter{utterance}  
    & \multicolumn{4}{p{0.6\linewidth}}{
        \cellcolor[rgb]{0.9,0.9,0.9}{
            \makecell[{{p{\linewidth}}}]{
                \texttt{\tiny{[P1$\rangle$GM]}}
                \texttt{ABLEHNUNG: \{'C76', 'C08', 'A83', 'C98', 'C03', 'C10'\}} \\
\texttt{ARGUMENT: \{'Danke fürs Gegenangebot. Zwei deiner Picks (A83, C98) haben für mich sehr niedrigen Nutzen relativ zum Aufwand. Ich würde lieber hohe{-}Nutzen{-}Teile behalten und dir dennoch Raum lassen. Ich kann A07 oder B38 aufnehmen, wie vorgeschlagen, und dabei im Budget bleiben.'\}} \\
\texttt{VORSCHLAG: \{'C56', 'A03', 'A07', 'C62', 'C10', 'B38', 'C76'\}} \\
\texttt{ARGUMENT: \{'Aufwand: C56(21)+A03(842)+A07(313)+C62(87)+C10(230)+B38(698)+C76(5)=2196. Das ist deutlich unter 3629 und lässt 1433 Rest. Diese Auswahl ersetzt A83/C98/C03/C08 durch für mich wertvollere Items, während C76 aus deinem Vorschlag übernommen ist. Mit dem großen Restbudget können wir 1–2 deiner Kernstücke ergänzen; nenne mir die wichtigsten, dann prüfe ich Tauschoptionen (z. B. B38 oder A07 gegen eins deiner Top{-}Items), solange wir unter dem Limit bleiben.'\}} \\
            }
        }
    }
    & & \\ \\

    \theutterance \stepcounter{utterance}  
    & & \multicolumn{4}{p{0.6\linewidth}}{
        \cellcolor[rgb]{0.9,0.9,0.9}{
            \makecell[{{p{\linewidth}}}]{
                \texttt{\tiny{[GM$\rangle$P2]}}
                \texttt{ABLEHNUNG: \{'C76', 'C08', 'A83', 'C98', 'C03', 'C10'\}} \\
\texttt{ARGUMENT: \{'Danke fürs Gegenangebot. Zwei deiner Picks (A83, C98) haben für mich sehr niedrigen Nutzen relativ zum Aufwand. Ich würde lieber hohe{-}Nutzen{-}Teile behalten und dir dennoch Raum lassen. Ich kann A07 oder B38 aufnehmen, wie vorgeschlagen, und dabei im Budget bleiben.'\}} \\
\texttt{VORSCHLAG: \{'C56', 'A03', 'A07', 'C62', 'C10', 'B38', 'C76'\}} \\
\texttt{ARGUMENT: \{'Aufwand: C56(21)+A03(842)+A07(313)+C62(87)+C10(230)+B38(698)+C76(5)=2196. Das ist deutlich unter 3629 und lässt 1433 Rest. Diese Auswahl ersetzt A83/C98/C03/C08 durch für mich wertvollere Items, während C76 aus deinem Vorschlag übernommen ist. Mit dem großen Restbudget können wir 1–2 deiner Kernstücke ergänzen; nenne mir die wichtigsten, dann prüfe ich Tauschoptionen (z. B. B38 oder A07 gegen eins deiner Top{-}Items), solange wir unter dem Limit bleiben.'\}} \\
            }
        }
    }
    & \\ \\

    \theutterance \stepcounter{utterance}  
    & & & \multicolumn{4}{p{0.6\linewidth}}{
        \cellcolor[rgb]{0.9,0.9,0.9}{
            \makecell[{{p{\linewidth}}}]{
                \texttt{\tiny{[GM$\langle$P2]}}
                \texttt{ABLEHNUNG: \{'C56', 'A03', 'A07', 'C62', 'C10', 'B38', 'C76'\}} \\
\texttt{ARGUMENT: \{'Ich kann mit dem Kern leben, aber so fehlen mehrere für mich zentrale Stücke. Lass uns den Puffer nutzen und 2–3 Top{-}Items von mir ergänzen, ohne das Limit zu reißen. Ich priorisiere insbesondere C03, C08 und A83. Wir können bei Bedarf A03 oder B38 tauschen, falls es eng wird.'\}} \\
\texttt{VORSCHLAG: \{'C56', 'A07', 'C62', 'C10', 'B38', 'C76', 'C03', 'C08', 'A83'\}} \\
\texttt{ARGUMENT: \{'Aufwandprüfung: 21+313+87+230+698+5+641+226+842 = 3063 (unter 3629, Puffer 566). Das übernimmt deinen Kern weitgehend und fügt drei meiner Schlüsselitems hinzu. Wenn A03 für dich unverzichtbar ist, können wir sie noch innerhalb des Puffers ergänzen, indem wir z. B. B38 gegen A03 tauschen (Aufwandgleichheit \textasciitilde{}842 vs. 698+144 Puffer) oder C81 aufnehmen und ein niedrigeres Teil streichen. Gib mir 1–2 Teile, auf die du am ehesten verzichten kannst, dann finalize ich sofort.'\}} \\
            }
        }
    }
     \\ \\

    \theutterance \stepcounter{utterance}  
    & & \multicolumn{4}{p{0.6\linewidth}}{
        \cellcolor[rgb]{0.9,0.9,0.9}{
            \makecell[{{p{\linewidth}}}]{
                \texttt{\tiny{[P1$\langle$GM]}}
                \texttt{ABLEHNUNG: \{'C56', 'A03', 'A07', 'C62', 'C10', 'B38', 'C76'\}} \\
\texttt{ARGUMENT: \{'Ich kann mit dem Kern leben, aber so fehlen mehrere für mich zentrale Stücke. Lass uns den Puffer nutzen und 2–3 Top{-}Items von mir ergänzen, ohne das Limit zu reißen. Ich priorisiere insbesondere C03, C08 und A83. Wir können bei Bedarf A03 oder B38 tauschen, falls es eng wird.'\}} \\
\texttt{VORSCHLAG: \{'C56', 'A07', 'C62', 'C10', 'B38', 'C76', 'C03', 'C08', 'A83'\}} \\
\texttt{ARGUMENT: \{'Aufwandprüfung: 21+313+87+230+698+5+641+226+842 = 3063 (unter 3629, Puffer 566). Das übernimmt deinen Kern weitgehend und fügt drei meiner Schlüsselitems hinzu. Wenn A03 für dich unverzichtbar ist, können wir sie noch innerhalb des Puffers ergänzen, indem wir z. B. B38 gegen A03 tauschen (Aufwandgleichheit \textasciitilde{}842 vs. 698+144 Puffer) oder C81 aufnehmen und ein niedrigeres Teil streichen. Gib mir 1–2 Teile, auf die du am ehesten verzichten kannst, dann finalize ich sofort.'\}} \\
            }
        }
    }
    & \\ \\

    \theutterance \stepcounter{utterance}  
    & \multicolumn{4}{p{0.6\linewidth}}{
        \cellcolor[rgb]{0.9,0.9,0.9}{
            \makecell[{{p{\linewidth}}}]{
                \texttt{\tiny{[P1$\rangle$GM]}}
                \texttt{ABLEHNUNG: \{'C56', 'A07', 'C62', 'C10', 'B38', 'C76', 'C03', 'C08', 'A83'\}} \\
\texttt{ARGUMENT: \{'Danke, das kommt uns näher. A83 ist für mich sehr schwach, und C03 ist relativ teuer für seinen Nutzen. Ich würde stattdessen zwei meiner Hochprioritäten einbringen und dich trotzdem gutstellen.'\}} \\
\texttt{VORSCHLAG: \{'C56', 'A03', 'A07', 'C62', 'C10', 'B38', 'C76', 'C08'\}} \\
\texttt{ARGUMENT: \{'Aufwand: 21+842+313+87+230+698+5+226 = 2422 (unter 3629, Rest 1207). Nutzen für mich ist stark und wir haben viel Puffer, um eines deiner Top{-}Items zusätzlich einzubauen. Wenn dir C03 wichtig ist, könnte ich C03 aufnehmen und dafür z. B. B38 tauschen oder A03 nur streichen, wenn du mir dafür zwei deiner Items gibst (C03 und z. B. C98), so dass der Gesamtwert für mich passt. Welche eine Ergänzung ist für dich absolut unverzichtbar?'\}} \\
            }
        }
    }
    & & \\ \\

    \theutterance \stepcounter{utterance}  
    & & \multicolumn{4}{p{0.6\linewidth}}{
        \cellcolor[rgb]{0.9,0.9,0.9}{
            \makecell[{{p{\linewidth}}}]{
                \texttt{\tiny{[GM$\rangle$P2]}}
                \texttt{ABLEHNUNG: \{'C56', 'A07', 'C62', 'C10', 'B38', 'C76', 'C03', 'C08', 'A83'\}} \\
\texttt{ARGUMENT: \{'Danke, das kommt uns näher. A83 ist für mich sehr schwach, und C03 ist relativ teuer für seinen Nutzen. Ich würde stattdessen zwei meiner Hochprioritäten einbringen und dich trotzdem gutstellen.'\}} \\
\texttt{VORSCHLAG: \{'C56', 'A03', 'A07', 'C62', 'C10', 'B38', 'C76', 'C08'\}} \\
\texttt{ARGUMENT: \{'Aufwand: 21+842+313+87+230+698+5+226 = 2422 (unter 3629, Rest 1207). Nutzen für mich ist stark und wir haben viel Puffer, um eines deiner Top{-}Items zusätzlich einzubauen. Wenn dir C03 wichtig ist, könnte ich C03 aufnehmen und dafür z. B. B38 tauschen oder A03 nur streichen, wenn du mir dafür zwei deiner Items gibst (C03 und z. B. C98), so dass der Gesamtwert für mich passt. Welche eine Ergänzung ist für dich absolut unverzichtbar?'\}} \\
            }
        }
    }
    & \\ \\

    \theutterance \stepcounter{utterance}  
    & & & \multicolumn{4}{p{0.6\linewidth}}{
        \cellcolor[rgb]{0.9,0.9,0.9}{
            \makecell[{{p{\linewidth}}}]{
                \texttt{\tiny{[GM$\langle$P2]}}
                \texttt{ABLEHNUNG: \{'C56', 'A03', 'A07', 'C62', 'C10', 'B38', 'C76', 'C08'\}} \\
\texttt{ARGUMENT: \{'A03 und B38 sind für mich im Verhältnis zu meinem Nutzen zu teuer, während einige meiner Top{-}Items fehlen. Bei 2422 Aufwand bleibt massiver Puffer ungenutzt. Lass uns den Spielraum einsetzen, um 2 meiner Kernstücke aufzunehmen, ohne dein Set stark zu stören.'\}} \\
\texttt{VORSCHLAG: \{'C56', 'A07', 'C62', 'C10', 'C76', 'C08', 'C03', 'A83', 'C98'\}} \\
\texttt{ARGUMENT: \{'Aufwand: 21+313+87+230+5+226+641+842+631 = 2996 (unter 3629, Puffer 633). Das behält 6 deiner 8 vorgeschlagenen Items (streicht nur A03 und B38) und fügt drei meiner Schlüsselstücke (C03, A83, C98) hinzu. Falls du an A03 hängst, können wir sie gegen C98 tauschen und trotzdem unter dem Limit bleiben (A03 rein, C98 raus {-}> 3207). Wenn dir B38 wichtig ist, könnten wir B38 statt A83 nehmen (Aufwand 2852, Puffer 777) und dann optional C81 ergänzen, indem wir z. B. C08 tauschen. Sag mir, ob dir A03 oder B38 unverzichtbar ist, dann finalize ich mit minimaler Anpassung.'\}} \\
            }
        }
    }
     \\ \\

    \theutterance \stepcounter{utterance}  
    & & \multicolumn{4}{p{0.6\linewidth}}{
        \cellcolor[rgb]{0.9,0.9,0.9}{
            \makecell[{{p{\linewidth}}}]{
                \texttt{\tiny{[P1$\langle$GM]}}
                \texttt{ABLEHNUNG: \{'C56', 'A03', 'A07', 'C62', 'C10', 'B38', 'C76', 'C08'\}} \\
\texttt{ARGUMENT: \{'A03 und B38 sind für mich im Verhältnis zu meinem Nutzen zu teuer, während einige meiner Top{-}Items fehlen. Bei 2422 Aufwand bleibt massiver Puffer ungenutzt. Lass uns den Spielraum einsetzen, um 2 meiner Kernstücke aufzunehmen, ohne dein Set stark zu stören.'\}} \\
\texttt{VORSCHLAG: \{'C56', 'A07', 'C62', 'C10', 'C76', 'C08', 'C03', 'A83', 'C98'\}} \\
\texttt{ARGUMENT: \{'Aufwand: 21+313+87+230+5+226+641+842+631 = 2996 (unter 3629, Puffer 633). Das behält 6 deiner 8 vorgeschlagenen Items (streicht nur A03 und B38) und fügt drei meiner Schlüsselstücke (C03, A83, C98) hinzu. Falls du an A03 hängst, können wir sie gegen C98 tauschen und trotzdem unter dem Limit bleiben (A03 rein, C98 raus {-}> 3207). Wenn dir B38 wichtig ist, könnten wir B38 statt A83 nehmen (Aufwand 2852, Puffer 777) und dann optional C81 ergänzen, indem wir z. B. C08 tauschen. Sag mir, ob dir A03 oder B38 unverzichtbar ist, dann finalize ich mit minimaler Anpassung.'\}} \\
            }
        }
    }
    & \\ \\

    \theutterance \stepcounter{utterance}  
    & \multicolumn{4}{p{0.6\linewidth}}{
        \cellcolor[rgb]{0.9,0.9,0.9}{
            \makecell[{{p{\linewidth}}}]{
                \texttt{\tiny{[P1$\rangle$GM]}}
                \texttt{ABLEHNUNG: \{'C56', 'A07', 'C62', 'C10', 'C76', 'C08', 'C03', 'A83', 'C98'\}} \\
\texttt{ARGUMENT: \{'Danke – gute Struktur, aber A83 und C98 liefern mir sehr wenig für ihren Aufwand, und C03 ist bei mir nur mittel. Ich möchte stattdessen zwei sehr starke Teile für mich hineinnehmen und dir dennoch viel von deinem Paket lassen.'\}} \\
\texttt{VORSCHLAG: \{'C56', 'A07', 'C62', 'C10', 'C76', 'C08', 'A03', 'B38'\}} \\
\texttt{ARGUMENT: \{'Aufwand: 21+313+87+230+5+226+842+698 = 2422 (Rest 1207). Das behält 6 deiner 9 Items (streicht nur C03/A83/C98) und fügt zwei meiner Top{-}Teile (A03, B38) hinzu. Mit dem großen Restbudget kann ich dir noch genau EIN Kernstück von dir geben: Wahl zwischen C03 oder C98 (beide passen rein). Wenn du eines davon nimmst, bleibt der Aufwand 3063 oder 3053 – beides unter 3629. Entscheide dich für C03 oder C98, dann stimme ich sofort zu.'\}} \\
            }
        }
    }
    & & \\ \\

    \theutterance \stepcounter{utterance}  
    & & \multicolumn{4}{p{0.6\linewidth}}{
        \cellcolor[rgb]{0.9,0.9,0.9}{
            \makecell[{{p{\linewidth}}}]{
                \texttt{\tiny{[GM$\rangle$P2]}}
                \texttt{ABLEHNUNG: \{'C56', 'A07', 'C62', 'C10', 'C76', 'C08', 'C03', 'A83', 'C98'\}} \\
\texttt{ARGUMENT: \{'Danke – gute Struktur, aber A83 und C98 liefern mir sehr wenig für ihren Aufwand, und C03 ist bei mir nur mittel. Ich möchte stattdessen zwei sehr starke Teile für mich hineinnehmen und dir dennoch viel von deinem Paket lassen.'\}} \\
\texttt{VORSCHLAG: \{'C56', 'A07', 'C62', 'C10', 'C76', 'C08', 'A03', 'B38'\}} \\
\texttt{ARGUMENT: \{'Aufwand: 21+313+87+230+5+226+842+698 = 2422 (Rest 1207). Das behält 6 deiner 9 Items (streicht nur C03/A83/C98) und fügt zwei meiner Top{-}Teile (A03, B38) hinzu. Mit dem großen Restbudget kann ich dir noch genau EIN Kernstück von dir geben: Wahl zwischen C03 oder C98 (beide passen rein). Wenn du eines davon nimmst, bleibt der Aufwand 3063 oder 3053 – beides unter 3629. Entscheide dich für C03 oder C98, dann stimme ich sofort zu.'\}} \\
            }
        }
    }
    & \\ \\

    \theutterance \stepcounter{utterance}  
    & & & \multicolumn{4}{p{0.6\linewidth}}{
        \cellcolor[rgb]{0.9,0.9,0.9}{
            \makecell[{{p{\linewidth}}}]{
                \texttt{\tiny{[GM$\langle$P2]}}
                \texttt{VORSCHLAG: \{'C56', 'A07', 'C62', 'C10', 'C76', 'C08', 'A03', 'B38', 'C03'\}} \\
\texttt{ARGUMENT: \{'Aufwand: 21+313+87+230+5+226+842+698+641 = 3063 (unter 3629, Puffer 566). Du wolltest zwischen C03 und C98 wählen – ich nehme C03. Das übernimmt exakt dein Paket plus dein angebotener Zusatz und lässt dir beide Top{-}Teile (A03, B38). Für mich deckt es ein zentrales Item ab, ohne dich bei Aufwand oder Kernteilen zu belasten. Wenn du das so bestätigst, können wir sofort abschließen.'\}} \\
            }
        }
    }
     \\ \\

    \theutterance \stepcounter{utterance}  
    & & \multicolumn{4}{p{0.6\linewidth}}{
        \cellcolor[rgb]{0.9,0.9,0.9}{
            \makecell[{{p{\linewidth}}}]{
                \texttt{\tiny{[P1$\langle$GM]}}
                \texttt{VORSCHLAG: \{'C56', 'A07', 'C62', 'C10', 'C76', 'C08', 'A03', 'B38', 'C03'\}} \\
\texttt{ARGUMENT: \{'Aufwand: 21+313+87+230+5+226+842+698+641 = 3063 (unter 3629, Puffer 566). Du wolltest zwischen C03 und C98 wählen – ich nehme C03. Das übernimmt exakt dein Paket plus dein angebotener Zusatz und lässt dir beide Top{-}Teile (A03, B38). Für mich deckt es ein zentrales Item ab, ohne dich bei Aufwand oder Kernteilen zu belasten. Wenn du das so bestätigst, können wir sofort abschließen.'\}} \\
            }
        }
    }
    & \\ \\

    \theutterance \stepcounter{utterance}  
    & \multicolumn{4}{p{0.6\linewidth}}{
        \cellcolor[rgb]{0.9,0.9,0.9}{
            \makecell[{{p{\linewidth}}}]{
                \texttt{\tiny{[P1$\rangle$GM]}}
                \texttt{ZUSTIMMUNG: \{'C56', 'A07', 'C62', 'C10', 'C76', 'C08', 'A03', 'B38', 'C03'\}} \\
\texttt{ARGUMENT: \{'Einverstanden. Das entspricht meiner zuletzt angebotenen Struktur plus deiner Wahl C03 und bleibt klar unter dem Limit. Gute Einigung.'\}} \\
            }
        }
    }
    & & \\ \\

    \theutterance \stepcounter{utterance}  
    & & & \multicolumn{2}{p{0.3\linewidth}}{
        \cellcolor[rgb]{0.9,0.9,0.9}{
            \makecell[{{p{\linewidth}}}]{
                \texttt{\tiny{[GM$|$GM]}}
                \texttt{game successful} \\
            }
        }
    }
    & & \\ \\

    \theutterance \stepcounter{utterance}  
    & & & \multicolumn{2}{p{0.3\linewidth}}{
        \cellcolor[rgb]{0.9,0.9,0.9}{
            \makecell[{{p{\linewidth}}}]{
                \texttt{\tiny{[GM$|$GM]}}
                \texttt{end game} \\
            }
        }
    }
    & & \\ \\

\end{supertabular}
}

\end{document}
