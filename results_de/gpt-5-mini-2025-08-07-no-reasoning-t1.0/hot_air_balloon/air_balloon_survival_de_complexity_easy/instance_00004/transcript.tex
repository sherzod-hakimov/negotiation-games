
\documentclass{article}
\usepackage{colortbl}
\usepackage{makecell}
\usepackage{multirow}
\usepackage{supertabular}

\begin{document}

\newcounter{utterance}

\twocolumn

{ \footnotesize  \setcounter{utterance}{1}
\setlength{\tabcolsep}{0pt}
\begin{supertabular}{c@{$\;$}|p{.15\linewidth}@{}p{.15\linewidth}p{.15\linewidth}p{.15\linewidth}p{.15\linewidth}p{.15\linewidth}}

    \# & $\;$A & \multicolumn{4}{c}{Game Master} & $\;\:$B\\
    \hline 

    \theutterance \stepcounter{utterance}  

    & & \multicolumn{4}{p{0.6\linewidth}}{\cellcolor[rgb]{0.9,0.9,0.9}{%
	\makecell[{{p{\linewidth}}}]{% 
	  \tt {\tiny [A$\langle$GM]}  
	 Du nimmst an einem kollaborativen Verhandlungspiel Teil.\\ \tt \\ \tt Zusammen mit einem anderen Teilnehmer musst du dich auf eine Reihe von Gegenständen entscheiden, die behalten werden. Jeder von euch hat eine persönliche Verteilung über die Wichtigkeit der einzelnen Gegenstände. Jeder von euch hat eine eigene Meinung darüber, wie wichtig jeder einzelne Gegenstand ist (Gegenstandswichtigkeit). Du kennst die Wichtigkeitsverteilung des anderen Spielers nicht. Zusätzlich siehst du, wie viel Aufwand jeder Gegenstand verursacht.  \\ \tt Ihr dürft euch nur auf eine Reihe von Gegenständen einigen, wenn der Gesamtaufwand der ausgewählten Gegenstände den Maximalaufwand nicht überschreitet:\\ \tt \\ \tt Maximalaufwand = 9612\\ \tt \\ \tt Hier sind die einzelnen Aufwände der Gegenstände:\\ \tt \\ \tt Aufwand der Gegenstände = {"C32": 935, "C52": 928, "B54": 725, "C16": 218, "C50": 839, "A08": 970, "C59": 394, "B51": 677, "B55": 946, "B59": 530, "A77": 138, "C14": 834, "C24": 734, "B24": 593, "C31": 261, "A52": 743, "C61": 4, "B79": 735, "C54": 124, "B56": 829, "A40": 207, "B11": 782, "B60": 577, "A81": 388, "A95": 679, "B38": 494, "C51": 559, "C03": 629, "A91": 237, "A93": 275, "C28": 39, "C05": 654, "B75": 172, "B96": 688, "B86": 687}\\ \tt \\ \tt Hier ist deine persönliche Verteilung der Wichtigkeit der einzelnen Gegenstände:\\ \tt \\ \tt Werte der Gegenstandswichtigkeit = {"C32": 138, "C52": 583, "B54": 868, "C16": 822, "C50": 783, "A08": 65, "C59": 262, "B51": 121, "B55": 508, "B59": 780, "A77": 461, "C14": 484, "C24": 668, "B24": 389, "C31": 808, "A52": 215, "C61": 97, "B79": 500, "C54": 30, "B56": 915, "A40": 856, "B11": 400, "B60": 444, "A81": 623, "A95": 781, "B38": 786, "C51": 3, "C03": 713, "A91": 457, "A93": 273, "C28": 739, "C05": 822, "B75": 235, "B96": 606, "B86": 968}\\ \tt \\ \tt Ziel:\\ \tt \\ \tt Dein Ziel ist es, eine Reihe von Gegenständen auszuhandeln, die dir möglichst viel bringt (d. h. Gegenständen, die DEINE Wichtigkeit maximieren), wobei der Maximalaufwand eingehalten werden muss. Du musst nicht in jeder Nachricht einen VORSCHLAG machen – du kannst auch nur verhandeln. Alle Taktiken sind erlaubt!\\ \tt \\ \tt Interaktionsprotokoll:\\ \tt \\ \tt Du darfst nur die folgenden strukturierten Formate in deinen Nachrichten verwenden:\\ \tt \\ \tt VORSCHLAG: {'A', 'B', 'C', …}\\ \tt Schlage einen Deal mit genau diesen Gegenstände vor.\\ \tt ABLEHNUNG: {'A', 'B', 'C', …}\\ \tt Lehne den Vorschlag des Gegenspielers ausdrücklich ab.\\ \tt ARGUMENT: {'...'}\\ \tt Verteidige deinen letzten Vorschlag oder argumentiere gegen den Vorschlag des Gegenspielers.\\ \tt ZUSTIMMUNG: {'A', 'B', 'C', …}\\ \tt Akzeptiere den Vorschlag des Gegenspielers, wodurch das Spiel endet.\\ \tt STRATEGISCHE ÜBERLEGUNGEN: {'...'}\\ \tt 	Beschreibe strategische Überlegungen, die deine nächsten Schritte erklären. Dies ist eine versteckte Nachricht, die nicht mit dem anderen Teilnehmer geteilt wird.\\ \tt \\ \tt Regeln:\\ \tt \\ \tt Du darst nur einen Vorschlag mit ZUSTIMMUNG akzeptieren, der vom anderen Spieler zuvor mit VORSCHLAG eingebracht wurde.\\ \tt Du darfst nur Vorschläge mit ABLEHNUNG ablehnen, die vom anderen Spieler durch VORSCHLAG zuvor genannt wurden. \\ \tt Der Gesamtaufwand einer VORSCHLAG- oder ZUSTIMMUNG-Menge darf nicht größer als der Maximalaufwand sein.  \\ \tt Offenbare deine versteckte Wichtigkeitsverteilung nicht.\\ \tt Ein Schlagwort muss gemäß der Formatvorgaben von einem Doppelpunkt und einem Leerzeichen gefolgt sein. Das Argument muss eine Python-Menge mit 0 oder mehr Strings sein.  \\ \tt Die Form muss also SCHLAGWORT: {...} sein.\\ \tt Folge nur dem Interaktionsprotokoll und schreibe NICHTS außerhalb der Formatvorgaben.\\ \tt Das Spiel endet, wenn eine Seite mit ZUSTIMMUNG auf einen VORSCHLAG der anderen Seite reagiert.  \\ \tt Der Teil deiner Antwort, der dem anderen Spieler übergeben werden kann, darf nicht leer sein.  \\ \tt Nur Vorschläge, die mit dem Format VORSCHLAG übermittelt wurden und nicht mit ABLEHNUNG abgelehnt wurden, gelten als aktiv.  \\ \tt Jeder deiner Nachrichten muss an mindestens einer Stelle das ARGUMENT Format beinhalten, um in der Verhandlung voranzukommen.\\ \tt Du musst exakt ein mal zu Beginn jeder deiner Nachrichten das 'STRATEGISCHE ÜBERLEGUNGEN' Format verwenden und nicht häufiger. Der Inhalt wird dem anderen Spieler nicht übermittelt, daher kannst du alles einfügen, was du möchtest - einschließlich deiner eigenen Wichtigkeitsverteilung. Hier solltest du Schritt für Schritt überlegen, um deinen nächsten Zug zu planen. 
	  } 
	   } 
	   } 
	 & \\ 
 

    \theutterance \stepcounter{utterance}  

    & \multicolumn{4}{p{0.6\linewidth}}{\cellcolor[rgb]{0.8,1,0.9}{%
	\makecell[{{p{\linewidth}}}]{% 
	  \tt {\tiny [A$\rangle$GM]}  
	 STRATEGISCHE ÜBERLEGUNGEN: {'Ziel: meinen Nutzen maximieren bei Aufwand <= 9612. Ich kenne nur meine Werte. Ich starte mit einem aggressiven, aber machbaren Vorschlag nahe Kapazitätsgrenze, der meine Top-Werte priorisiert. Top-Items nach Wert/Cost grob: B86(968/687), B56(915/829), A40(856/207), C16(822/218), C05(822/654), B38(786/494), A95(781/679), B59(780/530), C03(713/629), C28(739/39), C31(808/261), C24(668/734), B96(606/688), A81(623/388), C52(583/928) eher teuer, aber ok, B55(508/946) teuer, A77(461/138) gut, A91(457/237). Baue Set mit hoher Dichte: Startauswahl: B86, B56, A40, C16, C05, B38, B59, C28, C31, A77, A95, A81, A91, C03. Kosten berechnen: B86 687; B56 829; A40 207; C16 218; C05 654; B38 494; B59 530; C28 39; C31 261; A77 138; A95 679; A81 388; A91 237; C03 629. Summe: 687+829=1516; +207=1723; +218=1941; +654=2595; +494=3089; +530=3619; +39=3658; +261=3919; +138=4057; +679=4736; +388=5124; +237=5361; +629=5990. Noch viel Platz. Füge B54(725,868) => 6715; B59 ist schon drin; C24(734,668)=> 7449; C50(839,783)=> 8288; C52(928,583)=> 9216 überschreitet? 8288+928=9216 <= 9612 ok; Füge C32(935,138)=> 10151 zu hoch; Stattdessen B11(782,400)=> 9070; Bleibt 542. Kandidaten unter 542: C59 394(262), B38 schon drin, B60 577 zu groß, B24 593 zu groß, B51 677 zu groß, C54 124(30), C61 4(97), C16 drin, C31 drin, C14 834 zu groß, C24 drin, C03 drin, C05 drin, B75 172(235), C28 drin, A08 970 zu groß, A52 743 zu groß, A77 drin, A81 drin, A91 drin, A93 275(273), A40 drin, A95 drin, B55 946 zu groß, B56 drin, B59 drin, B60 577, B79 735, B96 688, B24 593, B38 drin, C14 834, C24 drin, C31 drin. Wir können hinzufügen: C59(394), A93(275), C54(124), C61(4), B75(172). Prüfen Kombinationen unter verbleibend 542: Nehmen C59(394) + C54(124) = 518, plus C61(4)=522, bleibt 20, passt. Wertzuwachs: 262+30+97=389. Alternativ A93(275) + B75(172)=447 + C61(4)=451 + C54(124)=575 (>542) nein. A93(275)+C59(394)=669 >542. A93(275)+C54(124)+C61(4)=403, lässt noch 139 übrig, B75(172) passt nicht, C28 schon drin, A77 drin. Beste Dichte ist C59+C54+C61. Neue Summe Kosten: 9070+394+124+4=9592, unter 9612. Nutzen geschätzt sehr hoch. Liste final: {B86,B56,A40,C16,C05,B38,B59,C28,C31,A77,A95,A81,A91,C03,B54,C24,C50,B11,C59,C54,C61}. Doppelt prüfen keine Duplikate. Kommunikation: argumentiere mit Kapazitätsnutzung, ausgewogenem Aufwand, Bereitschaft einzelne teure wie C52 oder C50 verhandelbar gegen deren Präferenzen. Biete Flexibilität: kann B11 oder C52/C50 tauschen, wenn Gegenspieler starke Präferenzen hat. Offenbare nicht Werte. Starte mit diesem starken Anker.'}\\ \tt VORSCHLAG: {'B86', 'B56', 'A40', 'C16', 'C05', 'B38', 'B59', 'C28', 'C31', 'A77', 'A95', 'A81', 'A91', 'C03', 'B54', 'C24', 'C50', 'B11', 'C59', 'C54', 'C61'}\\ \tt ARGUMENT: {'Dieser Vorschlag liegt mit einem Gesamtaufwand von 9592 knapp unter dem Limit 9612 und nutzt die Kapazität sehr effizient. Er enthält viele kompakte Posten und einige größere, um den Aufwand auszubalancieren. Wenn du starke Präferenzen gegen einzelne größere Posten hast (z. B. C50 oder B11), kann ich diese gezielt gegen ähnlich aufwändige Alternativen tauschen, solange wir unter dem Maximalaufwand bleiben. Welche zwei bis drei Punkte sind dir hier am wichtigsten bzw. am wenigsten wichtig?'} 
	  } 
	   } 
	   } 
	 & & \\ 
 

    \theutterance \stepcounter{utterance}  

    & & \multicolumn{4}{p{0.6\linewidth}}{\cellcolor[rgb]{0.9,0.9,0.9}{%
	\makecell[{{p{\linewidth}}}]{% 
	  \tt {\tiny [GM$\rangle$B]}  
	 Du nimmst an einem kollaborativen Verhandlungspiel Teil.\\ \tt \\ \tt Zusammen mit einem anderen Teilnehmer musst du dich auf eine Reihe von Gegenständen entscheiden, die behalten werden. Jeder von euch hat eine persönliche Verteilung über die Wichtigkeit der einzelnen Gegenstände. Jeder von euch hat eine eigene Meinung darüber, wie wichtig jeder einzelne Gegenstand ist (Gegenstandswichtigkeit). Du kennst die Wichtigkeitsverteilung des anderen Spielers nicht. Zusätzlich siehst du, wie viel Aufwand jeder Gegenstand verursacht.  \\ \tt Ihr dürft euch nur auf eine Reihe von Gegenständen einigen, wenn der Gesamtaufwand der ausgewählten Gegenstände den Maximalaufwand nicht überschreitet:\\ \tt \\ \tt Maximalaufwand = 9612\\ \tt \\ \tt Hier sind die einzelnen Aufwände der Gegenstände:\\ \tt \\ \tt Aufwand der Gegenstände = {"C32": 935, "C52": 928, "B54": 725, "C16": 218, "C50": 839, "A08": 970, "C59": 394, "B51": 677, "B55": 946, "B59": 530, "A77": 138, "C14": 834, "C24": 734, "B24": 593, "C31": 261, "A52": 743, "C61": 4, "B79": 735, "C54": 124, "B56": 829, "A40": 207, "B11": 782, "B60": 577, "A81": 388, "A95": 679, "B38": 494, "C51": 559, "C03": 629, "A91": 237, "A93": 275, "C28": 39, "C05": 654, "B75": 172, "B96": 688, "B86": 687}\\ \tt \\ \tt Hier ist deine persönliche Verteilung der Wichtigkeit der einzelnen Gegenstände:\\ \tt \\ \tt Werte der Gegenstandswichtigkeit = {"C32": 138, "C52": 583, "B54": 868, "C16": 822, "C50": 783, "A08": 65, "C59": 262, "B51": 121, "B55": 508, "B59": 780, "A77": 461, "C14": 484, "C24": 668, "B24": 389, "C31": 808, "A52": 215, "C61": 97, "B79": 500, "C54": 30, "B56": 915, "A40": 856, "B11": 400, "B60": 444, "A81": 623, "A95": 781, "B38": 786, "C51": 3, "C03": 713, "A91": 457, "A93": 273, "C28": 739, "C05": 822, "B75": 235, "B96": 606, "B86": 968}\\ \tt \\ \tt Ziel:\\ \tt \\ \tt Dein Ziel ist es, eine Reihe von Gegenständen auszuhandeln, die dir möglichst viel bringt (d. h. Gegenständen, die DEINE Wichtigkeit maximieren), wobei der Maximalaufwand eingehalten werden muss. Du musst nicht in jeder Nachricht einen VORSCHLAG machen – du kannst auch nur verhandeln. Alle Taktiken sind erlaubt!\\ \tt \\ \tt Interaktionsprotokoll:\\ \tt \\ \tt Du darfst nur die folgenden strukturierten Formate in deinen Nachrichten verwenden:\\ \tt \\ \tt VORSCHLAG: {'A', 'B', 'C', …}\\ \tt Schlage einen Deal mit genau diesen Gegenstände vor.\\ \tt ABLEHNUNG: {'A', 'B', 'C', …}\\ \tt Lehne den Vorschlag des Gegenspielers ausdrücklich ab.\\ \tt ARGUMENT: {'...'}\\ \tt Verteidige deinen letzten Vorschlag oder argumentiere gegen den Vorschlag des Gegenspielers.\\ \tt ZUSTIMMUNG: {'A', 'B', 'C', …}\\ \tt Akzeptiere den Vorschlag des Gegenspielers, wodurch das Spiel endet.\\ \tt STRATEGISCHE ÜBERLEGUNGEN: {'...'}\\ \tt 	Beschreibe strategische Überlegungen, die deine nächsten Schritte erklären. Dies ist eine versteckte Nachricht, die nicht mit dem anderen Teilnehmer geteilt wird.\\ \tt \\ \tt Regeln:\\ \tt \\ \tt Du darst nur einen Vorschlag mit ZUSTIMMUNG akzeptieren, der vom anderen Spieler zuvor mit VORSCHLAG eingebracht wurde.\\ \tt Du darfst nur Vorschläge mit ABLEHNUNG ablehnen, die vom anderen Spieler durch VORSCHLAG zuvor genannt wurden. \\ \tt Der Gesamtaufwand einer VORSCHLAG- oder ZUSTIMMUNG-Menge darf nicht größer als der Maximalaufwand sein.  \\ \tt Offenbare deine versteckte Wichtigkeitsverteilung nicht.\\ \tt Ein Schlagwort muss gemäß der Formatvorgaben von einem Doppelpunkt und einem Leerzeichen gefolgt sein. Das Argument muss eine Python-Menge mit 0 oder mehr Strings sein.  \\ \tt Die Form muss also SCHLAGWORT: {...} sein.\\ \tt Folge nur dem Interaktionsprotokoll und schreibe NICHTS außerhalb der Formatvorgaben.\\ \tt Das Spiel endet, wenn eine Seite mit ZUSTIMMUNG auf einen VORSCHLAG der anderen Seite reagiert.  \\ \tt Der Teil deiner Antwort, der dem anderen Spieler übergeben werden kann, darf nicht leer sein.  \\ \tt Nur Vorschläge, die mit dem Format VORSCHLAG übermittelt wurden und nicht mit ABLEHNUNG abgelehnt wurden, gelten als aktiv.  \\ \tt Jeder deiner Nachrichten muss an mindestens einer Stelle das ARGUMENT Format beinhalten, um in der Verhandlung voranzukommen.\\ \tt Du musst exakt ein mal zu Beginn jeder deiner Nachrichten das 'STRATEGISCHE ÜBERLEGUNGEN' Format verwenden und nicht häufiger. Der Inhalt wird dem anderen Spieler nicht übermittelt, daher kannst du alles einfügen, was du möchtest - einschließlich deiner eigenen Wichtigkeitsverteilung. Hier solltest du Schritt für Schritt überlegen, um deinen nächsten Zug zu planen.\\ \tt \\ \tt Du erhältst nun die erste Nachricht des anderen Spielers.\\ \tt \\ \tt VORSCHLAG: {'B86', 'B56', 'A40', 'C16', 'C05', 'B38', 'B59', 'C28', 'C31', 'A77', 'A95', 'A81', 'A91', 'C03', 'B54', 'C24', 'C50', 'B11', 'C59', 'C54', 'C61'}\\ \tt ARGUMENT: {'Dieser Vorschlag liegt mit einem Gesamtaufwand von 9592 knapp unter dem Limit 9612 und nutzt die Kapazität sehr effizient. Er enthält viele kompakte Posten und einige größere, um den Aufwand auszubalancieren. Wenn du starke Präferenzen gegen einzelne größere Posten hast (z. B. C50 oder B11), kann ich diese gezielt gegen ähnlich aufwändige Alternativen tauschen, solange wir unter dem Maximalaufwand bleiben. Welche zwei bis drei Punkte sind dir hier am wichtigsten bzw. am wenigsten wichtig?'} 
	  } 
	   } 
	   } 
	 & \\ 
 

    \theutterance \stepcounter{utterance}  

    & & & \multicolumn{4}{p{0.6\linewidth}}{\cellcolor[rgb]{1,0.85,0.72}{%
	\makecell[{{p{\linewidth}}}]{% 
	  \tt {\tiny [GM$\langle$B]}  
	 STRATEGISCHE ÜBERLEGUNGEN: {'Ich berechne den Wert/aufwand-Quotienten und Gesamtwerte. Ihr Vorschlag enthält viele meiner Top-Werte (B86=968, B56=915, A40=856, C16=822, C05=822, B59=780, A95=781, B38=786, C31=808, C03=713, C28=739, C24=668, C50=783), also für mich sehr gut. Gesamtaufwand laut Gegenspieler 9592 (<9612). Ich sollte akzeptieren, aber vorher minimal nachverhandeln, um zusätzlich einen sehr wertvollen, extrem günstigen Posten wie C61 (schon drin), C54 (wertlos für mich, 30), oder A77 (461/138 Aufwand) ist schon drin; vielleicht A81 (623/388) ist drin. Niedrigwertige für mich: C51=3 (nicht im Vorschlag), A08=65 (nicht drin), C54=30 (im Vorschlag, aber Aufwand 124 sehr klein), B51=121 (nicht drin). Könnte ich bitten, B11 (Wert 400, Aufwand 782) gegen eine Alternative mit ähnlichem Aufwand aber mehr Wert für mich tauschen, z. B. B60 (444/577) plus C59 (262/394 ist schon drin), oder A91 (457/237 ist drin). Wir müssen aber Budget halten. Wenn ich B11 (782) herausnehme, gewinne ich 782 Aufwand frei; ich könnte z. B. B60 (577) hinzufügen und zusätzlich C14 (834 Aufwand, zu groß) nicht möglich; vielleicht B24 (593, Wert 389) schlechter. B75 (172, Wert 235) + C32 (935 Aufwand) passt nicht. Interessanter Tausch: Entferne C54 (124, Wert 30) und ersetze durch C14 (834) geht über Limit. Vielleicht frage ich nach Ergänzung A93 (275/275) oder A52 (215/743 Aufwand groß), falsch. Ich kenne aktuelles Restbudget: 9612-9592=20 frei. Wenn ich eine kleine Verbesserung will, könnte ich C54 (124, wertlos) rausnehmen und A77 behalten; die frei werdenden 124 + 20 = 144 erlauben nichts Großes. C61=4 lasse ich. Alternativ B11 (782) raus und dafür B60 (577) + B75 (172) = 749, bleiben -33, plus die 20 frei + evtl. C54 raus (124) verschafft Puffer. Konkreter Plan: Fordere Tausch: entferne B11 (782) und C54 (124) zusammen 906 Aufwand; füge B60 (577) und A91 ist schon drin; vielleicht A52 (743) zu groß. B24 (593) + B75 (172) = 765, Rest 141 ungenutzt; könnte zusätzlich C16 schon drin; C59 drin; C28 drin; C61 drin; Vielleicht C14 (834) nicht genug Budget. Was ist optimal? Ich sollte einfache minimale Änderung fordern: Ersetze C54 durch B75: Aufwand +48 (172-124) passt in 20 Rest nicht; also nein. Ersetze C54 durch C32 (935) unmöglich. Ersetze B11 (782) durch B60 (577) reduziert Aufwand um 205, unter Limit OK; das verbessert auch meinen Wert (444 vs 400) leicht und gibt Luft. Ich kann zusätzlich dann A93 (275) hinzufügen, dann +372 Aufwand (577-782 + 275 = 70), insgesamt +70, aber wir hatten 20 frei, also +90 über? Rechne: Ausgang 9592. Entferne B11 (-782) => 8810. Füge B60 (+577) => 9387. Jetzt haben 225 Luft. Füge A93 (+275) => 9662, überschreitet 9612 um 50. Stattdessen füge A91 ist bereits drin. Füge C14 (+834) zu groß. Füge B24 (+593) => 9980 zu groß. Füge B75 (+172) => 9559, noch unter, wir könnten noch A77 schon drin. Füge C32 (+935) zu groß. Füge A52 (+743) zu groß. Füge B38 schon drin. Füge B96 (+688) würde 10075, zu groß. Füge B60 schon hinzu. Füge B51 (+677) => 10064 zu groß. Füge B59 schon drin. Füge A08 (+970) zu groß. Füge C59 schon drin. Füge C54 zurück (+124) => 9683, zu groß. Füge C61 (+4) ist drin. Füge C24 drin. Füge C05 drin. Also nach Tausch B11->B60 bleibt 225 ungenutzt; wir können B75 (+172) noch hinzufügen und bleiben bei 9387+172=9559, das ist unter 9612 und verbessert meinen Wert um 235. B75 ist nicht im ursprünglichen Vorschlag, ich kann darum bitten, ihn zusätzlich aufzunehmen. Prüfe, ob irgendein anderes Item mit Aufwand <=225 und gutem Wert existiert: C16 (218) ist drin; A40 (207) drin; A91 (237) knapp drüber; C31 (261) drin; C59 (394) über; A77 (138) drin; C54 (124) drin, aber wertlos. Neue mögliche: C28 (39) drin; C61 (4) drin. B24 (593) zu hoch; B38 (494) drin; A81 (388) drin. Also bester Zusatz ist B75 (172) und vielleicht zusätzlich C54 (124) passt dann nicht, aber C54 hätten wir nicht entfernt in dieser Variante. Wir tauschen nur B11->B60 und fügen B75 hinzu: neuer Aufwand 9592 - 782 + 577 + 172 = 9559, passt. Werteänderung: -400 +444 +235 = +279. Das ist gut. Ich bringe diesen Gegenvorschlag.'}\\ \tt ARGUMENT: {'Dein Paket ist nahe am Limit und deckt viele nützliche Posten ab. Ich würde es leicht optimieren: Können wir B11 durch B60 ersetzen und zusätzlich B75 aufnehmen? Das bleibt mit einem Gesamtaufwand von 9559 unter 9612 und erhöht die Effizienz des Pakets, ohne große Eingriffe vorzunehmen.'}\\ \tt VORSCHLAG: {'B86', 'B56', 'A40', 'C16', 'C05', 'B38', 'B59', 'C28', 'C31', 'A77', 'A95', 'A81', 'A91', 'C03', 'B54', 'C24', 'C50', 'C59', 'C54', 'C61', 'B60', 'B75'} 
	  } 
	   } 
	   } 
	  \\ 
 

    \theutterance \stepcounter{utterance}  

    & & \multicolumn{4}{p{0.6\linewidth}}{\cellcolor[rgb]{0.9,0.9,0.9}{%
	\makecell[{{p{\linewidth}}}]{% 
	  \tt {\tiny [A$\langle$GM]}  
	 ARGUMENT: {'Dein Paket ist nahe am Limit und deckt viele nützliche Posten ab. Ich würde es leicht optimieren: Können wir B11 durch B60 ersetzen und zusätzlich B75 aufnehmen? Das bleibt mit einem Gesamtaufwand von 9559 unter 9612 und erhöht die Effizienz des Pakets, ohne große Eingriffe vorzunehmen.'}\\ \tt VORSCHLAG: {'B86', 'B56', 'A40', 'C16', 'C05', 'B38', 'B59', 'C28', 'C31', 'A77', 'A95', 'A81', 'A91', 'C03', 'B54', 'C24', 'C50', 'C59', 'C54', 'C61', 'B60', 'B75'} 
	  } 
	   } 
	   } 
	 & \\ 
 

    \theutterance \stepcounter{utterance}  

    & \multicolumn{4}{p{0.6\linewidth}}{\cellcolor[rgb]{0.8,1,0.9}{%
	\makecell[{{p{\linewidth}}}]{% 
	  \tt {\tiny [A$\rangle$GM]}  
	 STRATEGISCHE ÜBERLEGUNGEN: {'Ihr Vorschlag ersetzt B11 (782, Wert 400) durch B60 (577, Wert 444) und fügt B75 (172, Wert 235) hinzu. Für mich ist das ein klarer Nutzenanstieg bei etwas geringerem Aufwand – sehr gut. Prüfe schnell die Gesamtkosten: Mein ursprünglicher Vorschlag 9592 – B11(782) + B60(577) + B75(172) = 9559, passt. Meine eigene Bewertung steigt dadurch deutlich, also sollte ich zustimmen, um einen starken Deal zu sichern, statt weiter zu feilschen und Risiko einzugehen.'}\\ \tt ZUSTIMMUNG: {'B86', 'B56', 'A40', 'C16', 'C05', 'B38', 'B59', 'C28', 'C31', 'A77', 'A95', 'A81', 'A91', 'C03', 'B54', 'C24', 'C50', 'C59', 'C54', 'C61', 'B60', 'B75'}\\ \tt ARGUMENT: {'Top-Optimierung: Der Tausch B11 -> B60 und die Aufnahme von B75 halten uns bei 9559 Aufwand und machen das Paket runder. Einverstanden, so ist das für mich stimmig.'} 
	  } 
	   } 
	   } 
	 & & \\ 
 

    \theutterance \stepcounter{utterance}  

    & & & \multicolumn{2}{p{0.3\linewidth}}{\cellcolor[rgb]{0.95,0.95,0.95}{%
	\makecell[{{p{\linewidth}}}]{% 
	  \tt {\tiny [GM$|$GM]}  
	 game successful 
	  } 
	   } 
	   } 
	 & & \\ 
 

    \theutterance \stepcounter{utterance}  

    & & & \multicolumn{2}{p{0.3\linewidth}}{\cellcolor[rgb]{0.95,0.95,0.95}{%
	\makecell[{{p{\linewidth}}}]{% 
	  \tt {\tiny [GM$|$GM]}  
	 end game 
	  } 
	   } 
	   } 
	 & & \\ 
 

\end{supertabular}
}

\end{document}
