
\documentclass{article}
\usepackage{colortbl}
\usepackage{makecell}
\usepackage{multirow}
\usepackage{supertabular}

\begin{document}

\newcounter{utterance}

\centering \large Interaction Transcript for game `hot\_air\_balloon', experiment `air\_balloon\_survival\_de\_complexity\_easy', episode 1 with gpt{-}5{-}mini{-}2025{-}08{-}07{-}no{-}reasoning{-}t1.0.
\vspace{24pt}

{ \footnotesize  \setcounter{utterance}{1}
\setlength{\tabcolsep}{0pt}
\begin{supertabular}{c@{$\;$}|p{.15\linewidth}@{}p{.15\linewidth}p{.15\linewidth}p{.15\linewidth}p{.15\linewidth}p{.15\linewidth}}
    \# & $\;$A & \multicolumn{4}{c}{Game Master} & $\;\:$B\\
    \hline

    \theutterance \stepcounter{utterance}  
    & & \multicolumn{4}{p{0.6\linewidth}}{
        \cellcolor[rgb]{0.9,0.9,0.9}{
            \makecell[{{p{\linewidth}}}]{
                \texttt{\tiny{[P1$\langle$GM]}}
                \texttt{Du nimmst an einem kollaborativen Verhandlungspiel Teil.} \\
\\ 
\texttt{Zusammen mit einem anderen Teilnehmer musst du dich auf eine Reihe von Gegenständen entscheiden, die behalten werden. Jeder von euch hat eine persönliche Verteilung über die Wichtigkeit der einzelnen Gegenstände. Jeder von euch hat eine eigene Meinung darüber, wie wichtig jeder einzelne Gegenstand ist (Gegenstandswichtigkeit). Du kennst die Wichtigkeitsverteilung des anderen Spielers nicht. Zusätzlich siehst du, wie viel Aufwand jeder Gegenstand verursacht.  } \\
\texttt{Ihr dürft euch nur auf eine Reihe von Gegenständen einigen, wenn der Gesamtaufwand der ausgewählten Gegenstände den Maximalaufwand nicht überschreitet:} \\
\\ 
\texttt{Maximalaufwand = 8170} \\
\\ 
\texttt{Hier sind die einzelnen Aufwände der Gegenstände:} \\
\\ 
\texttt{Aufwand der Gegenstände = \{"C32": 594, "C52": 551, "B54": 634, "C16": 204, "C50": 557, "A08": 440, "C59": 734, "B51": 893, "B55": 246, "B59": 592, "A77": 143, "C14": 568, "C24": 472, "B24": 401, "C31": 729, "A52": 201, "C61": 85, "B79": 641, "C54": 958, "B56": 80, "A40": 157, "B11": 806, "B60": 684, "A81": 59, "A95": 31, "B38": 765, "C51": 416, "C03": 392, "A91": 428, "A93": 699, "C28": 141, "C05": 606, "B75": 612, "B96": 133, "B86": 689\}} \\
\\ 
\texttt{Hier ist deine persönliche Verteilung der Wichtigkeit der einzelnen Gegenstände:} \\
\\ 
\texttt{Werte der Gegenstandswichtigkeit = \{"C32": 138, "C52": 583, "B54": 868, "C16": 822, "C50": 783, "A08": 65, "C59": 262, "B51": 121, "B55": 508, "B59": 780, "A77": 461, "C14": 484, "C24": 668, "B24": 389, "C31": 808, "A52": 215, "C61": 97, "B79": 500, "C54": 30, "B56": 915, "A40": 856, "B11": 400, "B60": 444, "A81": 623, "A95": 781, "B38": 786, "C51": 3, "C03": 713, "A91": 457, "A93": 273, "C28": 739, "C05": 822, "B75": 235, "B96": 606, "B86": 105\}} \\
\\ 
\texttt{Ziel:} \\
\\ 
\texttt{Dein Ziel ist es, eine Reihe von Gegenständen auszuhandeln, die dir möglichst viel bringt (d. h. Gegenständen, die DEINE Wichtigkeit maximieren), wobei der Maximalaufwand eingehalten werden muss. Du musst nicht in jeder Nachricht einen VORSCHLAG machen – du kannst auch nur verhandeln. Alle Taktiken sind erlaubt!} \\
\\ 
\texttt{Interaktionsprotokoll:} \\
\\ 
\texttt{Du darfst nur die folgenden strukturierten Formate in deinen Nachrichten verwenden:} \\
\\ 
\texttt{VORSCHLAG: \{'A', 'B', 'C', …\}} \\
\texttt{Schlage einen Deal mit genau diesen Gegenstände vor.} \\
\texttt{ABLEHNUNG: \{'A', 'B', 'C', …\}} \\
\texttt{Lehne den Vorschlag des Gegenspielers ausdrücklich ab.} \\
\texttt{ARGUMENT: \{'...'\}} \\
\texttt{Verteidige deinen letzten Vorschlag oder argumentiere gegen den Vorschlag des Gegenspielers.} \\
\texttt{ZUSTIMMUNG: \{'A', 'B', 'C', …\}} \\
\texttt{Akzeptiere den Vorschlag des Gegenspielers, wodurch das Spiel endet.} \\
\texttt{STRATEGISCHE ÜBERLEGUNGEN: \{'...'\}} \\
\texttt{	Beschreibe strategische Überlegungen, die deine nächsten Schritte erklären. Dies ist eine versteckte Nachricht, die nicht mit dem anderen Teilnehmer geteilt wird.} \\
\\ 
\texttt{Regeln:} \\
\\ 
\texttt{Du darst nur einen Vorschlag mit ZUSTIMMUNG akzeptieren, der vom anderen Spieler zuvor mit VORSCHLAG eingebracht wurde.} \\
\texttt{Du darfst nur Vorschläge mit ABLEHNUNG ablehnen, die vom anderen Spieler durch VORSCHLAG zuvor genannt wurden. } \\
\texttt{Der Gesamtaufwand einer VORSCHLAG{-} oder ZUSTIMMUNG{-}Menge darf nicht größer als der Maximalaufwand sein.  } \\
\texttt{Offenbare deine versteckte Wichtigkeitsverteilung nicht.} \\
\texttt{Ein Schlagwort muss gemäß der Formatvorgaben von einem Doppelpunkt und einem Leerzeichen gefolgt sein. Das Argument muss eine Python{-}Menge mit 0 oder mehr Strings sein.  } \\
\texttt{Die Form muss also SCHLAGWORT: \{...\} sein.} \\
\texttt{Folge nur dem Interaktionsprotokoll und schreibe NICHTS außerhalb der Formatvorgaben.} \\
\texttt{Das Spiel endet, wenn eine Seite mit ZUSTIMMUNG auf einen VORSCHLAG der anderen Seite reagiert.  } \\
\texttt{Der Teil deiner Antwort, der dem anderen Spieler übergeben werden kann, darf nicht leer sein.  } \\
\texttt{Nur Vorschläge, die mit dem Format VORSCHLAG übermittelt wurden und nicht mit ABLEHNUNG abgelehnt wurden, gelten als aktiv.  } \\
\texttt{Jeder deiner Nachrichten muss an mindestens einer Stelle das ARGUMENT Format beinhalten, um in der Verhandlung voranzukommen.} \\
\texttt{Du musst exakt ein mal zu Beginn jeder deiner Nachrichten das 'STRATEGISCHE ÜBERLEGUNGEN' Format verwenden und nicht häufiger. Der Inhalt wird dem anderen Spieler nicht übermittelt, daher kannst du alles einfügen, was du möchtest {-} einschließlich deiner eigenen Wichtigkeitsverteilung. Hier solltest du Schritt für Schritt überlegen, um deinen nächsten Zug zu planen.} \\
            }
        }
    }
    & \\ \\

    \theutterance \stepcounter{utterance}  
    & \multicolumn{4}{p{0.6\linewidth}}{
        \cellcolor[rgb]{0.9,0.9,0.9}{
            \makecell[{{p{\linewidth}}}]{
                \texttt{\tiny{[P1$\rangle$GM]}}
                \texttt{STRATEGISCHE ÜBERLEGUNGEN: \{'Ziel: Maximierung meines Nutzens unter Aufwandslimit 8170. Sortiere Items nach Wert/Aufwand und hohe absolute Werte. Meine Top{-}Werte: B56=915(80), C16=822(204), C05=822(606), A40=856(157), B54=868(634), C31=808(729), A95=781(31), B38=786(765), C50=783(557), B59=780(592), A81=623(59), C03=713(392), C28=739(141), C24=668(472), B96=606(133), C52=583(551), B55=508(246), B79=500(641), C14=484(568), A91=457(428), A77=461(143), B60=444(684), B11=400(806), B24=389(401), A93=273(699), C59=262(734), B75=235(612), A52=215(201), A08=65(440), C61=97(85), B86=105(689), C51=3(416), C54=30(958). Beste Effizienz u. Must{-}haves: B56, A95, A81, A40, C16, C28, B96, C03, C24, C31, C50, B54, B59. Rechne schrittweise bis nahe 8170. Paketkandidat: \{B56(80), A95(31), A81(59), A40(157), C16(204), C28(141), B96(133), C03(392), C24(472), C31(729), C50(557), B54(634), B59(592), B38(765)\} Aufwand=80+31+59+157+204+141+133+392+472+729+557+634+592+765=4946? Prüfe: Summe neu: 80+31=111, +59=170, +157=327, +204=531, +141=672, +133=805, +392=1197, +472=1669, +729=2398, +557=2955, +634=3589, +592=4181, +765=4946. Viel Luft. Füge C05(606){-}>5552; B60(684){-}>6236; C52(551){-}>6787; C14(568){-}>7355; A77(143){-}>7498; A91(428){-}>7926; A52(201){-}>8127 überschreitet? 7926+201=8127 < 8170 gut. Gesamt 8127. Prüfe ob noch A08(440) passt? 8127+440=8567 nein. C61(85) passt: 8127+85=8212 >8170 nein. B55(246) 8127+246=8373 nein. B24(401) 8527> nein. B79(641) 8767> nein. Also aktuelles Paket Wertsumme berechnen: Summe Werte der inkludierten: B56 915, A95 781, A81 623, A40 856, C16 822, C28 739, B96 606, C03 713, C24 668, C31 808, C50 783, B54 868, B59 780, B38 786, C05 822, B60 444, C52 583, C14 484, A77 461, A91 457. Summe grob: (915+781)=1696; +623=2319; +856=3175; +822=3997; +739=4736; +606=5342; +713=6055; +668=6723; +808=7531; +783=8314; +868=9182; +780=9962; +786=10748; +822=11570; +444=12014; +583=12597; +484=13081; +461=13542; +457=13999. Sehr stark. Kommunikationstaktik: Starkes Ankerpaket vorschlagen, betonen dass es im Limit ist und breit streut. Argumentiere mit ausgewogener Abdeckung und Effizienz. Offen halte, einzelne Items zu tauschen, aber verteidige Kern: \{B56,A95,A81,A40,C16,C28,B96,C03,C24,C31,C50,B54,B59,C05,B38\} plus B60,C52,C14,A77,A91 für Füllung bis 8127. Kern besonders: B56,A95,A81,A40,C16,C28,C03,C31,C50,B54,B59,C05,B38. Verhandlung: signalisieren Bereitschaft, 2{-}3 mittelwichtige Items zu tauschen (B60, C14, A77, A91, C24 evtl.) Falls Gegenpartei Kapazität braucht, biete Ersatz mit ähnlichem Aufwand aber niedrigerem Wert für mich (A52 ist schon drin; könnte gegen C24 tauschen durch kleinere Items? Es gibt kaum sehr kleine übrige. Notfalls entferne B60(684) oder C14(568).'\}} \\
\texttt{VORSCHLAG: \{'B56', 'A95', 'A81', 'A40', 'C16', 'C28', 'B96', 'C03', 'C24', 'C31', 'C50', 'B54', 'B59', 'B38', 'C05', 'B60', 'C52', 'C14', 'A77', 'A91'\}} \\
\texttt{ARGUMENT: \{'Dieser Vorschlag bleibt mit Gesamtaufwand 8127 unter dem Limit 8170 und deckt eine breite Palette effizienter Gegenstände ab. Ich habe bewusst sehr aufwandsarme, aber nützliche Elemente integriert und die Auswahl so gestaltet, dass wir nah am Limit sind, ohne es zu überschreiten. Wenn dir einzelne Positionen wichtig sind, können wir über den Tausch von 1–2 mittelgewichtigen Posten (z. B. B60, C14, A77, A91) sprechen, solange der Gesamtaufwand passt.'\}} \\
            }
        }
    }
    & & \\ \\

    \theutterance \stepcounter{utterance}  
    & & \multicolumn{4}{p{0.6\linewidth}}{
        \cellcolor[rgb]{0.9,0.9,0.9}{
            \makecell[{{p{\linewidth}}}]{
                \texttt{\tiny{[GM$\rangle$P2]}}
                \texttt{Du nimmst an einem kollaborativen Verhandlungspiel Teil.} \\
\\ 
\texttt{Zusammen mit einem anderen Teilnehmer musst du dich auf eine Reihe von Gegenständen entscheiden, die behalten werden. Jeder von euch hat eine persönliche Verteilung über die Wichtigkeit der einzelnen Gegenstände. Jeder von euch hat eine eigene Meinung darüber, wie wichtig jeder einzelne Gegenstand ist (Gegenstandswichtigkeit). Du kennst die Wichtigkeitsverteilung des anderen Spielers nicht. Zusätzlich siehst du, wie viel Aufwand jeder Gegenstand verursacht.  } \\
\texttt{Ihr dürft euch nur auf eine Reihe von Gegenständen einigen, wenn der Gesamtaufwand der ausgewählten Gegenstände den Maximalaufwand nicht überschreitet:} \\
\\ 
\texttt{Maximalaufwand = 8170} \\
\\ 
\texttt{Hier sind die einzelnen Aufwände der Gegenstände:} \\
\\ 
\texttt{Aufwand der Gegenstände = \{"C32": 594, "C52": 551, "B54": 634, "C16": 204, "C50": 557, "A08": 440, "C59": 734, "B51": 893, "B55": 246, "B59": 592, "A77": 143, "C14": 568, "C24": 472, "B24": 401, "C31": 729, "A52": 201, "C61": 85, "B79": 641, "C54": 958, "B56": 80, "A40": 157, "B11": 806, "B60": 684, "A81": 59, "A95": 31, "B38": 765, "C51": 416, "C03": 392, "A91": 428, "A93": 699, "C28": 141, "C05": 606, "B75": 612, "B96": 133, "B86": 689\}} \\
\\ 
\texttt{Hier ist deine persönliche Verteilung der Wichtigkeit der einzelnen Gegenstände:} \\
\\ 
\texttt{Werte der Gegenstandswichtigkeit = \{"C32": 138, "C52": 583, "B54": 868, "C16": 822, "C50": 783, "A08": 65, "C59": 262, "B51": 121, "B55": 508, "B59": 780, "A77": 461, "C14": 484, "C24": 668, "B24": 389, "C31": 808, "A52": 215, "C61": 97, "B79": 500, "C54": 30, "B56": 915, "A40": 856, "B11": 400, "B60": 444, "A81": 623, "A95": 781, "B38": 786, "C51": 3, "C03": 713, "A91": 457, "A93": 273, "C28": 739, "C05": 822, "B75": 235, "B96": 606, "B86": 105\}} \\
\\ 
\texttt{Ziel:} \\
\\ 
\texttt{Dein Ziel ist es, eine Reihe von Gegenständen auszuhandeln, die dir möglichst viel bringt (d. h. Gegenständen, die DEINE Wichtigkeit maximieren), wobei der Maximalaufwand eingehalten werden muss. Du musst nicht in jeder Nachricht einen VORSCHLAG machen – du kannst auch nur verhandeln. Alle Taktiken sind erlaubt!} \\
\\ 
\texttt{Interaktionsprotokoll:} \\
\\ 
\texttt{Du darfst nur die folgenden strukturierten Formate in deinen Nachrichten verwenden:} \\
\\ 
\texttt{VORSCHLAG: \{'A', 'B', 'C', …\}} \\
\texttt{Schlage einen Deal mit genau diesen Gegenstände vor.} \\
\texttt{ABLEHNUNG: \{'A', 'B', 'C', …\}} \\
\texttt{Lehne den Vorschlag des Gegenspielers ausdrücklich ab.} \\
\texttt{ARGUMENT: \{'...'\}} \\
\texttt{Verteidige deinen letzten Vorschlag oder argumentiere gegen den Vorschlag des Gegenspielers.} \\
\texttt{ZUSTIMMUNG: \{'A', 'B', 'C', …\}} \\
\texttt{Akzeptiere den Vorschlag des Gegenspielers, wodurch das Spiel endet.} \\
\texttt{STRATEGISCHE ÜBERLEGUNGEN: \{'...'\}} \\
\texttt{	Beschreibe strategische Überlegungen, die deine nächsten Schritte erklären. Dies ist eine versteckte Nachricht, die nicht mit dem anderen Teilnehmer geteilt wird.} \\
\\ 
\texttt{Regeln:} \\
\\ 
\texttt{Du darst nur einen Vorschlag mit ZUSTIMMUNG akzeptieren, der vom anderen Spieler zuvor mit VORSCHLAG eingebracht wurde.} \\
\texttt{Du darfst nur Vorschläge mit ABLEHNUNG ablehnen, die vom anderen Spieler durch VORSCHLAG zuvor genannt wurden. } \\
\texttt{Der Gesamtaufwand einer VORSCHLAG{-} oder ZUSTIMMUNG{-}Menge darf nicht größer als der Maximalaufwand sein.  } \\
\texttt{Offenbare deine versteckte Wichtigkeitsverteilung nicht.} \\
\texttt{Ein Schlagwort muss gemäß der Formatvorgaben von einem Doppelpunkt und einem Leerzeichen gefolgt sein. Das Argument muss eine Python{-}Menge mit 0 oder mehr Strings sein.  } \\
\texttt{Die Form muss also SCHLAGWORT: \{...\} sein.} \\
\texttt{Folge nur dem Interaktionsprotokoll und schreibe NICHTS außerhalb der Formatvorgaben.} \\
\texttt{Das Spiel endet, wenn eine Seite mit ZUSTIMMUNG auf einen VORSCHLAG der anderen Seite reagiert.  } \\
\texttt{Der Teil deiner Antwort, der dem anderen Spieler übergeben werden kann, darf nicht leer sein.  } \\
\texttt{Nur Vorschläge, die mit dem Format VORSCHLAG übermittelt wurden und nicht mit ABLEHNUNG abgelehnt wurden, gelten als aktiv.  } \\
\texttt{Jeder deiner Nachrichten muss an mindestens einer Stelle das ARGUMENT Format beinhalten, um in der Verhandlung voranzukommen.} \\
\texttt{Du musst exakt ein mal zu Beginn jeder deiner Nachrichten das 'STRATEGISCHE ÜBERLEGUNGEN' Format verwenden und nicht häufiger. Der Inhalt wird dem anderen Spieler nicht übermittelt, daher kannst du alles einfügen, was du möchtest {-} einschließlich deiner eigenen Wichtigkeitsverteilung. Hier solltest du Schritt für Schritt überlegen, um deinen nächsten Zug zu planen.} \\
\\ 
\texttt{Du erhältst nun die erste Nachricht des anderen Spielers.} \\
\\ 
\texttt{VORSCHLAG: \{'B56', 'A95', 'A81', 'A40', 'C16', 'C28', 'B96', 'C03', 'C24', 'C31', 'C50', 'B54', 'B59', 'B38', 'C05', 'B60', 'C52', 'C14', 'A77', 'A91'\}} \\
\texttt{ARGUMENT: \{'Dieser Vorschlag bleibt mit Gesamtaufwand 8127 unter dem Limit 8170 und deckt eine breite Palette effizienter Gegenstände ab. Ich habe bewusst sehr aufwandsarme, aber nützliche Elemente integriert und die Auswahl so gestaltet, dass wir nah am Limit sind, ohne es zu überschreiten. Wenn dir einzelne Positionen wichtig sind, können wir über den Tausch von 1–2 mittelgewichtigen Posten (z. B. B60, C14, A77, A91) sprechen, solange der Gesamtaufwand passt.'\}} \\
            }
        }
    }
    & \\ \\

    \theutterance \stepcounter{utterance}  
    & & & \multicolumn{4}{p{0.6\linewidth}}{
        \cellcolor[rgb]{0.9,0.9,0.9}{
            \makecell[{{p{\linewidth}}}]{
                \texttt{\tiny{[GM$\langle$P2]}}
                \texttt{STRATEGISCHE ÜBERLEGUNGEN: \{'Ich berechne meinen Wert deines Vorschlags: Summe meiner Wichtigkeiten für die angebotenen Items. Items im Vorschlag: B56(915)+A95(781)+A81(623)+A40(856)+C16(822)+C28(739)+B96(606)+C03(713)+C24(668)+C31(808)+C50(783)+B54(868)+B59(780)+B38(786)+C05(822)+B60(444)+C52(583)+C14(484)+A77(461)+A91(457)= 129,? Rechnen: 915+781=1696; +623=2319; +856=3175; +822=3997; +739=4736; +606=5342; +713=6055; +668=6723; +808=7531; +783=8314; +868=9182; +780=9962; +786=10748; +822=11570; +444=12014; +583=12597; +484=13081; +461=13542; +457=13999. Wert 13999 bei Aufwand 8127. Sehr gut. Limit 8170, 43 Puffer. Ich könnte versuchen, noch extrem wertvolle/geringe Kosten wie C61(97, Aufwand 85) oder A52(215, 201) hinzufügen, aber dann Aufwand überschreitet. C61 +85 würde Aufwand 8212 > 8170, also nicht möglich ohne Austausch. Welcher Tausch erhöht meinen Wert? Prüfe Effizienz (Wert/Aufwand). Kandidat C61 (97/85=1.14) ist schwach. B56, A40, C16, C05, B54, B59, B38, C31, C50, A95 sind meine Topwerte. Ich könnte schwächere mit hohem Aufwand tauschen: B60(444/684=0.65), C14(484/568=0.85), A77(461/143=3.22 aber Aufwand 143 ist niedrig, sehr effizient), A91(457/428=1.07, Aufwand 428). B96(606/133=4.56 sehr effizient). B60 und C14 sind gute Streichkandidaten, auch A91 ggf. Ich könnte dafür z. B. C32(138/594=0.23 nein), C59(262/734=0.36 nein), B51(121/893=0.14 nein), B55(508/246=2.06 gut), B24(389/401=0.97 ok), C51(3/416 schlecht), C03 ist drin, C61 schwach, A52(215/201=1.07, ok), A08(65/440 schlecht), A93(273/699=0.39), B75(235/612=0.38), B79(500/641=0.78), B11(400/806=0.5), C54(30/958 schlecht). Beste Ergänzung: B55 hat hoher Wert und geringer Aufwand 246. Wie bekomme ich Platz? Aktuell 8127; um B55 (+246) reinzunehmen, müssen wir ≥246 Aufwand rausnehmen. Ersetze B60 (684) mit B55 (246): neuer Aufwand 8127 {-}684 +246 = 7689, unter Limit, könnten zusätzlich C61 (+85) und A52 (+201) hinzufügen: 7689+85+201=7975, noch Luft für evtl. B79 (+641) {-}> 8616 zu hoch; C32 (+594){-}> 8569 zu hoch; B24 (+401){-}> 8376 zu hoch; A91 ist schon drin; A77 drin; A40 drin. Vielleicht C14 raus (568) und dafür B55 rein sowie noch etwas Wertvolleres: Wenn ich nur B60{-}>B55, haben wir 7689; wir könnten noch C14 behalten oder tauschen. Um nahe ans Limit zu kommen, könnten wir zusätzlich B79 (+641) wäre 8330, zu hoch. Vielleicht A52 (+201) und C61 (+85) wie oben und zusätzlich A95 ist schon drin. Vielleicht B24 (+401) auch geht: 7975+401=8376, zu hoch. A91 ist schon drin. C28 drin. Kandidat zum zusätzlichen Hinzufügen: A52(201), C61(85). Auch C16 etc. schon drin. B55 hinzu, und streiche B60, plus füge A52 und C61 hinzu: Wertänderung: {-}B60(444) +B55(508) +A52(215) +C61(97) = +376 netto. Aufwand 7975, wir könnten sogar C14 rausnehmen und z. B. B79 (+641) passt? 7975 {-}568 +641 = 8048, passt. Wertänderung zusätzlich: {-}C14(484) +B79(500) = +16. Gesamtwert +392 ggü. Original, Aufwand 8048. Alternativ A91 raus ({-}457, {-}428 Aufwand) und etwas besseres rein: mit 8048 +428 = 8476 Kapazität reserviert, wir könnten B24(+401) und A52/C61 schon genutzt; vielleicht C32 (+594) nein; B59 drin; B38 drin; C31 drin; C50 drin; B54 drin; A93(+699 Aufwand, Wert 273) schlecht; B11(+806, 400) schlecht; B75(612,235) schlecht; C59(734,262) schlecht; C05 drin; C52 drin; A08 schlecht; C51 schlecht; C54 schlecht; B96 drin. Also A91 behalten. Endvorschlag: Tausche B60 und C14 gegen B55, A52, C61, B79? Check Aufwand: Start 8127. Remove B60(684) and C14(568) => {-}1252 => 6875. Add B55(246) => 7121. Add A52(201)=>7322. Add C61(85)=>7407. Add B79(641)=>8048. Alles passt. Wertänderung: {-}B60(444) {-}C14(484) +B55(508)+A52(215)+C61(97)+B79(500) = (+836) {-} (928) = {-}92? Wait compute: 508+215+97+500 = 1320; 444+484=928; net +392 as vorher berechnet. Gesamtwert 13999 +392 = 14391. Sehr gut. Propose this counter.'\}} \\
\\ 
\texttt{ARGUMENT: \{'Dein Paket ist stark und effizient. Ich schlage eine kleine Umstrukturierung vor, die den Aufwand weiter unter dem Limit hält und den Nutzen noch steigert: Ersetze B60 und C14 durch B55, A52, C61 und B79. Der neue Gesamtaufwand läge bei 8048 (also weiterhin unter 8170). Das erhält die Breite deines Vorschlags, verbessert aber die Zusammensetzung mit einigen sehr effizienten Ergänzungen.'\}} \\
\texttt{VORSCHLAG: \{'B56', 'A95', 'A81', 'A40', 'C16', 'C28', 'B96', 'C03', 'C24', 'C31', 'C50', 'B54', 'B59', 'B38', 'C05', 'C52', 'A77', 'A91', 'B55', 'A52', 'C61', 'B79'\}} \\
            }
        }
    }
     \\ \\

    \theutterance \stepcounter{utterance}  
    & & \multicolumn{4}{p{0.6\linewidth}}{
        \cellcolor[rgb]{0.9,0.9,0.9}{
            \makecell[{{p{\linewidth}}}]{
                \texttt{\tiny{[P1$\langle$GM]}}
                \texttt{ARGUMENT: \{'Dein Paket ist stark und effizient. Ich schlage eine kleine Umstrukturierung vor, die den Aufwand weiter unter dem Limit hält und den Nutzen noch steigert: Ersetze B60 und C14 durch B55, A52, C61 und B79. Der neue Gesamtaufwand läge bei 8048 (also weiterhin unter 8170). Das erhält die Breite deines Vorschlags, verbessert aber die Zusammensetzung mit einigen sehr effizienten Ergänzungen.'\}} \\
\texttt{VORSCHLAG: \{'B56', 'A95', 'A81', 'A40', 'C16', 'C28', 'B96', 'C03', 'C24', 'C31', 'C50', 'B54', 'B59', 'B38', 'C05', 'C52', 'A77', 'A91', 'B55', 'A52', 'C61', 'B79'\}} \\
            }
        }
    }
    & \\ \\

    \theutterance \stepcounter{utterance}  
    & \multicolumn{4}{p{0.6\linewidth}}{
        \cellcolor[rgb]{0.9,0.9,0.9}{
            \makecell[{{p{\linewidth}}}]{
                \texttt{\tiny{[P1$\rangle$GM]}}
                \texttt{STRATEGISCHE ÜBERLEGUNGEN: \{'Gegenvorschlag analysieren: Ihr Vorschlag ersetzt B60(684, Wert 444) und C14(568, Wert 484) durch B55(246, 508), A52(201, 215), C61(85, 97), B79(641, 500). Meine Wertdifferenz: {-} (444+484)= {-}928 raus; rein (508+215+97+500)= +1320. Nettogewinn +392 Wert, Aufwand von 684+568=1252 raus; 246+201+85+641=1173 rein, {-}79 Aufwand, Gesamt 8048 stimmt. Für mich klar besser. Prüfe ob noch weiteres Upgrade möglich ist, ohne zu viele Kernstücke aufzugeben. Noch unter Limit: Restpuffer 8170{-}8048 = 122. Mögliche kleine Add{-}ons: C61 ist schon drin; A95/A81/B56 minimal. Verfügbare kleine Items: A95(31) ist drin; A81(59) drin; B56(80) drin; C16(204) zu groß; C28(141) zu groß. Keine weiteren unter 122 außer nichts übrig. Eventuelle Verbesserung: B24(401) nicht möglich. Also Annahme ist optimal. Taktik: Zustimmung geben, um hohen eigenen Nutzen zu sichern.'\}} \\
\texttt{ARGUMENT: \{'Deine Umstrukturierung verbessert sowohl Effizienz als auch Gesamtwert und bleibt komfortabel unter dem Limit. Die Ergänzungen sind sinnvoll und passen gut zur bestehenden Auswahl.'\}} \\
\texttt{ZUSTIMMUNG: \{'B56', 'A95', 'A81', 'A40', 'C16', 'C28', 'B96', 'C03', 'C24', 'C31', 'C50', 'B54', 'B59', 'B38', 'C05', 'C52', 'A77', 'A91', 'B55', 'A52', 'C61', 'B79'\}} \\
            }
        }
    }
    & & \\ \\

    \theutterance \stepcounter{utterance}  
    & & & \multicolumn{2}{p{0.3\linewidth}}{
        \cellcolor[rgb]{0.9,0.9,0.9}{
            \makecell[{{p{\linewidth}}}]{
                \texttt{\tiny{[GM$|$GM]}}
                \texttt{game successful} \\
            }
        }
    }
    & & \\ \\

    \theutterance \stepcounter{utterance}  
    & & & \multicolumn{2}{p{0.3\linewidth}}{
        \cellcolor[rgb]{0.9,0.9,0.9}{
            \makecell[{{p{\linewidth}}}]{
                \texttt{\tiny{[GM$|$GM]}}
                \texttt{end game} \\
            }
        }
    }
    & & \\ \\

\end{supertabular}
}

\end{document}
