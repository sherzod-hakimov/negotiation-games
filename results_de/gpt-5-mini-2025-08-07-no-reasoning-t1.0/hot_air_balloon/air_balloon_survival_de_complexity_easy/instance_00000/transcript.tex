
\documentclass{article}
\usepackage{colortbl}
\usepackage{makecell}
\usepackage{multirow}
\usepackage{supertabular}

\begin{document}

\newcounter{utterance}

\centering \large Interaction Transcript for game `hot\_air\_balloon', experiment `air\_balloon\_survival\_de\_complexity\_easy', episode 0 with gpt{-}5{-}mini{-}2025{-}08{-}07{-}no{-}reasoning{-}t1.0.
\vspace{24pt}

{ \footnotesize  \setcounter{utterance}{1}
\setlength{\tabcolsep}{0pt}
\begin{supertabular}{c@{$\;$}|p{.15\linewidth}@{}p{.15\linewidth}p{.15\linewidth}p{.15\linewidth}p{.15\linewidth}p{.15\linewidth}}
    \# & $\;$A & \multicolumn{4}{c}{Game Master} & $\;\:$B\\
    \hline

    \theutterance \stepcounter{utterance}  
    & & \multicolumn{4}{p{0.6\linewidth}}{
        \cellcolor[rgb]{0.9,0.9,0.9}{
            \makecell[{{p{\linewidth}}}]{
                \texttt{\tiny{[P1$\langle$GM]}}
                \texttt{Du nimmst an einem kollaborativen Verhandlungspiel Teil.} \\
\\ 
\texttt{Zusammen mit einem anderen Teilnehmer musst du dich auf eine Reihe von Gegenständen entscheiden, die behalten werden. Jeder von euch hat eine persönliche Verteilung über die Wichtigkeit der einzelnen Gegenstände. Jeder von euch hat eine eigene Meinung darüber, wie wichtig jeder einzelne Gegenstand ist (Gegenstandswichtigkeit). Du kennst die Wichtigkeitsverteilung des anderen Spielers nicht. Zusätzlich siehst du, wie viel Aufwand jeder Gegenstand verursacht.  } \\
\texttt{Ihr dürft euch nur auf eine Reihe von Gegenständen einigen, wenn der Gesamtaufwand der ausgewählten Gegenstände den Maximalaufwand nicht überschreitet:} \\
\\ 
\texttt{Maximalaufwand = 7605} \\
\\ 
\texttt{Hier sind die einzelnen Aufwände der Gegenstände:} \\
\\ 
\texttt{Aufwand der Gegenstände = \{"C32": 369, "C52": 951, "B54": 156, "C16": 266, "C50": 581, "A08": 287, "C59": 866, "B51": 180, "B55": 800, "B59": 739, "A77": 637, "C14": 86, "C24": 749, "B24": 370, "C31": 345, "A52": 951, "C61": 147, "B79": 265, "C54": 262, "B56": 259, "A40": 358, "B11": 394, "B60": 286, "A81": 580, "A95": 479, "B38": 14, "C51": 153, "C03": 977, "A91": 134, "A93": 982, "C28": 259, "C05": 232, "B75": 202, "B96": 73, "B86": 821\}} \\
\\ 
\texttt{Hier ist deine persönliche Verteilung der Wichtigkeit der einzelnen Gegenstände:} \\
\\ 
\texttt{Werte der Gegenstandswichtigkeit = \{"C32": 138, "C52": 583, "B54": 868, "C16": 822, "C50": 783, "A08": 65, "C59": 262, "B51": 121, "B55": 508, "B59": 780, "A77": 461, "C14": 484, "C24": 668, "B24": 389, "C31": 808, "A52": 215, "C61": 97, "B79": 500, "C54": 30, "B56": 915, "A40": 856, "B11": 400, "B60": 444, "A81": 623, "A95": 781, "B38": 786, "C51": 3, "C03": 713, "A91": 457, "A93": 273, "C28": 739, "C05": 822, "B75": 235, "B96": 606, "B86": 968\}} \\
\\ 
\texttt{Ziel:} \\
\\ 
\texttt{Dein Ziel ist es, eine Reihe von Gegenständen auszuhandeln, die dir möglichst viel bringt (d. h. Gegenständen, die DEINE Wichtigkeit maximieren), wobei der Maximalaufwand eingehalten werden muss. Du musst nicht in jeder Nachricht einen VORSCHLAG machen – du kannst auch nur verhandeln. Alle Taktiken sind erlaubt!} \\
\\ 
\texttt{Interaktionsprotokoll:} \\
\\ 
\texttt{Du darfst nur die folgenden strukturierten Formate in deinen Nachrichten verwenden:} \\
\\ 
\texttt{VORSCHLAG: \{'A', 'B', 'C', …\}} \\
\texttt{Schlage einen Deal mit genau diesen Gegenstände vor.} \\
\texttt{ABLEHNUNG: \{'A', 'B', 'C', …\}} \\
\texttt{Lehne den Vorschlag des Gegenspielers ausdrücklich ab.} \\
\texttt{ARGUMENT: \{'...'\}} \\
\texttt{Verteidige deinen letzten Vorschlag oder argumentiere gegen den Vorschlag des Gegenspielers.} \\
\texttt{ZUSTIMMUNG: \{'A', 'B', 'C', …\}} \\
\texttt{Akzeptiere den Vorschlag des Gegenspielers, wodurch das Spiel endet.} \\
\texttt{STRATEGISCHE ÜBERLEGUNGEN: \{'...'\}} \\
\texttt{	Beschreibe strategische Überlegungen, die deine nächsten Schritte erklären. Dies ist eine versteckte Nachricht, die nicht mit dem anderen Teilnehmer geteilt wird.} \\
\\ 
\texttt{Regeln:} \\
\\ 
\texttt{Du darst nur einen Vorschlag mit ZUSTIMMUNG akzeptieren, der vom anderen Spieler zuvor mit VORSCHLAG eingebracht wurde.} \\
\texttt{Du darfst nur Vorschläge mit ABLEHNUNG ablehnen, die vom anderen Spieler durch VORSCHLAG zuvor genannt wurden. } \\
\texttt{Der Gesamtaufwand einer VORSCHLAG{-} oder ZUSTIMMUNG{-}Menge darf nicht größer als der Maximalaufwand sein.  } \\
\texttt{Offenbare deine versteckte Wichtigkeitsverteilung nicht.} \\
\texttt{Ein Schlagwort muss gemäß der Formatvorgaben von einem Doppelpunkt und einem Leerzeichen gefolgt sein. Das Argument muss eine Python{-}Menge mit 0 oder mehr Strings sein.  } \\
\texttt{Die Form muss also SCHLAGWORT: \{...\} sein.} \\
\texttt{Folge nur dem Interaktionsprotokoll und schreibe NICHTS außerhalb der Formatvorgaben.} \\
\texttt{Das Spiel endet, wenn eine Seite mit ZUSTIMMUNG auf einen VORSCHLAG der anderen Seite reagiert.  } \\
\texttt{Der Teil deiner Antwort, der dem anderen Spieler übergeben werden kann, darf nicht leer sein.  } \\
\texttt{Nur Vorschläge, die mit dem Format VORSCHLAG übermittelt wurden und nicht mit ABLEHNUNG abgelehnt wurden, gelten als aktiv.  } \\
\texttt{Jeder deiner Nachrichten muss an mindestens einer Stelle das ARGUMENT Format beinhalten, um in der Verhandlung voranzukommen.} \\
\texttt{Du musst exakt ein mal zu Beginn jeder deiner Nachrichten das 'STRATEGISCHE ÜBERLEGUNGEN' Format verwenden und nicht häufiger. Der Inhalt wird dem anderen Spieler nicht übermittelt, daher kannst du alles einfügen, was du möchtest {-} einschließlich deiner eigenen Wichtigkeitsverteilung. Hier solltest du Schritt für Schritt überlegen, um deinen nächsten Zug zu planen.} \\
            }
        }
    }
    & \\ \\

    \theutterance \stepcounter{utterance}  
    & \multicolumn{4}{p{0.6\linewidth}}{
        \cellcolor[rgb]{0.9,0.9,0.9}{
            \makecell[{{p{\linewidth}}}]{
                \texttt{\tiny{[P1$\rangle$GM]}}
                \texttt{STRATEGISCHE ÜBERLEGUNGEN: \{'Ziel: Maximiere meinen Wert unter Aufwand ≤ 7605. Hohe Werte pro Aufwand priorisieren. Items mit sehr hohem Wert: B86(968/821), B56(915/259), A40(856/358), C16(822/266), C05(822/232), C31(808/345), A95(781/479), B38(786/14), B59(780/739 eher mittel), C28(739/259), C03(713/977 teuer), A81(623/580), B96(606/73), C24(668/749 eher schlecht), C50(783/581), B54(868/156), A77(461/637 meh), A91(457/134 gut), B60(444/286 ok), B11(400/394 meh), B79(500/265 gut), A08(65/287 schlecht), C61(97/147 niedrig), C14(484/86 extrem gut), C52(583/951 teuer), A93(273/982 schlecht), A52(215/951 schlecht), C59(262/866 schlecht), C54(30/262 schlecht), B55(508/800 meh), B24(389/370 ok), C32(138/369 schlecht). Greedy knapsack: Start mit ultrahohen ratio: B38, C14, B96, A91, B56, C05, C16, C31, A40, C28, B54, B79, C50, A95, B60, A81, B86, B59, C24, C03, C52… Prüfe Summe. Auswahlkandidat: \{B38(14), C14(86)=100, B96(73)=173, A91(134)=307, B56(259)=566, C05(232)=798, C16(266)=1064, C31(345)=1409, A40(358)=1767, C28(259)=2026, B54(156)=2182, B79(265)=2447, C50(581)=3028, A95(479)=3507, B60(286)=3793, A81(580)=4373, B86(821)=5194, B59(739)=5933, C24(749)=6682, C03(977)=7659 überschreitet. Ohne C03 bleiben 6682. Wir haben Rest 923. Prüfe weitere: C52(951) passt nicht, A77(637) passt: 7319, bleibt 286; B11(394) nicht, A08(287) knapp nicht, B24(370) nicht, C61(147) passt: 7466, bleibt 139; C51(153) nicht, C32(369) nicht, C54(262) nicht, B75(202) nicht, B51(180) nicht, B55(800) nicht, C59(866) nicht, A52/A93 nein, C50 schon drin, A40 drin. Mit Rest 139 könnten wir nichts mehr hinzufügen außer evtl. nichts. Alternative statt C24(749) könnten wir packen andere Kombinationen: Wenn wir entfernen C24 (6682{-}749=5933 Rest 1672). Mit 1672 könnten wir vielleicht C03(977) plus A77(637) = 1614, neue Summe 5933+977+637=7547, Wertzuwachs: C24=668, alternativ C03(713)+A77(461)=1174 > 668, besser. Prüfen Kapazität: 7547 ≤ 7605 ok. Rest 58. Können wir noch C61(147) nicht. Sehr gut. Vielleicht auch C52(951) + C61(147)=1098 ≤ 1672 und Wert 583+97=680 etwas besser als C24 allein, aber schlechter als C03+A77. Was ist mit B11(394)+B24(370)+C61(147)+B51(180)=1091, Wert 400+389+97+121=1007, auch gut, und Rest 581 könnte noch C50 aber ist schon drin. Statt C03+A77(1614) könnten wir B11(394)+B24(370)+C61(147)+B51(180)+A08(287)=1378, Wert 400+389+97+121+65=1072 < 1174; plus vielleicht C14/B38/B96 sind schon drin. Oder add C52(951)+B11(394)=1345 Wert 583+400=983. Beste bisher ist C03(713)+A77(461)=1174. Was ist C52(951)+A91(134) already in set; can't add A91 twice. Maybe add A95 already in. Check another swap: replace B59(739, val 780) with something else? Keep. Replace A81(580, 623) maybe keep. The current strong set: S = \{B38, C14, B96, A91, B56, C05, C16, C31, A40, C28, B54, B79, C50, A95, B60, A81, B86, B59, C03, A77\}. Sum effort: previously 5933 + 977 + 637 = 7547. Compute total value: sum of all chosen values: B38 786 + C14 484 + B96 606 + A91 457 + B56 915 + C05 822 + C16 822 + C31 808 + A40 856 + C28 739 + B54 868 + B79 500 + C50 783 + A95 781 + B60 444 + A81 623 + B86 968 + B59 780 + C03 713 + A77 461 = total. Quick sum: (786+484)=1270; +606=1876; +457=2333; +915=3248; +822=4070; +822=4892; +808=5700; +856=6556; +739=7295; +868=8163; +500=8663; +783=9446; +781=10227; +444=10671; +623=11294; +968=12262; +780=13042; +713=13755; +461=14216. 14216 value at 7547 effort. Maybe can we swap A81(580,623) for C24(749,668) to keep under cap? If we drop A81 and add C24 we add +45 value and +169 effort {-}> new effort 7547{-}580+749=7716 > cap. Can't. Could we add small item C61(147,97) to reach cap? Current rest 58, so no. This looks strong. Offer this as initial proposal. Argue with workload fit and fairness vagueness, emphasize low{-}cost high{-}benefit items. Keep private values secret. Let's proceed.'\}} \\
\texttt{VORSCHLAG: \{'B38', 'C14', 'B96', 'A91', 'B56', 'C05', 'C16', 'C31', 'A40', 'C28', 'B54', 'B79', 'C50', 'A95', 'B60', 'A81', 'B86', 'B59', 'C03', 'A77'\}} \\
\texttt{ARGUMENT: \{'Ich habe eine ausgewogene Auswahl effizienter Gegenstände zusammengestellt, die den Maximalaufwand nicht überschreitet (Gesamtaufwand 7547 ≤ 7605). Viele davon sind geringe bis mittlere Aufwände mit starkem Nutzen. Wenn dir einzelne Punkte wichtig sind, lass uns gezielt tauschen – ich bin offen, einzelne Posten zu ersetzen, solange wir insgesamt unter dem Aufwandslimit bleiben.'\}} \\
            }
        }
    }
    & & \\ \\

    \theutterance \stepcounter{utterance}  
    & & \multicolumn{4}{p{0.6\linewidth}}{
        \cellcolor[rgb]{0.9,0.9,0.9}{
            \makecell[{{p{\linewidth}}}]{
                \texttt{\tiny{[P1$\langle$GM]}}
                \texttt{Deine Antwort entsprach nicht mit dem korrekten Format für strategische Überlegungen entsprochen. Das erste strukturierte Format muss folgende Form haben: STRATEGISCHE ÜBERLEGUNGEN: \{...\}. Versuche es erneut.} \\
            }
        }
    }
    & \\ \\

    \theutterance \stepcounter{utterance}  
    & \multicolumn{4}{p{0.6\linewidth}}{
        \cellcolor[rgb]{0.9,0.9,0.9}{
            \makecell[{{p{\linewidth}}}]{
                \texttt{\tiny{[P1$\rangle$GM]}}
                \texttt{STRATEGISCHE ÜBERLEGUNGEN: \{'Ich muss exakt mit STRATEGISCHE ÜBERLEGUNGEN beginnen. Ziel: Maximiere meinen Wert unter Aufwand ≤ 7605. Beste bisher gefundene Kombination: S = \{B38, C14, B96, A91, B56, C05, C16, C31, A40, C28, B54, B79, C50, A95, B60, A81, B86, B59, C03, A77\} mit Aufwand 7547 und Wert 14216. Ich biete diese erneut korrekt formatiert an. Verhandlungstaktik: Effizienz hervorheben, Austauschbereitschaft signalisieren, insbesondere bereit, C03 oder A77 zu tauschen, falls der/die andere dort Präferenzen hat. Prüfe Alternativen, falls Gegenwind: Ersetze C03 (977) + A77 (637) Paket durch C24 (749) + etwas Kleines passt aber nicht gut. Biete ansonsten gezielte Swaps an (z. B. A81 oder B59 gegen zwei mittlere). Immer das Aufwandslimit betonen und um Präferenzen bitten, ohne eigene Werte offenzulegen.'\}} \\
\texttt{VORSCHLAG: \{'B38', 'C14', 'B96', 'A91', 'B56', 'C05', 'C16', 'C31', 'A40', 'C28', 'B54', 'B79', 'C50', 'A95', 'B60', 'A81', 'B86', 'B59', 'C03', 'A77'\}} \\
\texttt{ARGUMENT: \{'Ich schlage diese Kombination vor: Gesamtaufwand 7547 ≤ 7605. Sie setzt vor allem auf effiziente Gegenstände mit moderatem Aufwand. Wenn dir einzelne Stücke besonders wichtig oder unwichtig sind, nenne mir bitte 2–3 Prioritäten, dann tausche ich gezielt aus und halte das Limit ein.'\}} \\
            }
        }
    }
    & & \\ \\

    \theutterance \stepcounter{utterance}  
    & & \multicolumn{4}{p{0.6\linewidth}}{
        \cellcolor[rgb]{0.9,0.9,0.9}{
            \makecell[{{p{\linewidth}}}]{
                \texttt{\tiny{[GM$\rangle$P2]}}
                \texttt{Du nimmst an einem kollaborativen Verhandlungspiel Teil.} \\
\\ 
\texttt{Zusammen mit einem anderen Teilnehmer musst du dich auf eine Reihe von Gegenständen entscheiden, die behalten werden. Jeder von euch hat eine persönliche Verteilung über die Wichtigkeit der einzelnen Gegenstände. Jeder von euch hat eine eigene Meinung darüber, wie wichtig jeder einzelne Gegenstand ist (Gegenstandswichtigkeit). Du kennst die Wichtigkeitsverteilung des anderen Spielers nicht. Zusätzlich siehst du, wie viel Aufwand jeder Gegenstand verursacht.  } \\
\texttt{Ihr dürft euch nur auf eine Reihe von Gegenständen einigen, wenn der Gesamtaufwand der ausgewählten Gegenstände den Maximalaufwand nicht überschreitet:} \\
\\ 
\texttt{Maximalaufwand = 7605} \\
\\ 
\texttt{Hier sind die einzelnen Aufwände der Gegenstände:} \\
\\ 
\texttt{Aufwand der Gegenstände = \{"C32": 369, "C52": 951, "B54": 156, "C16": 266, "C50": 581, "A08": 287, "C59": 866, "B51": 180, "B55": 800, "B59": 739, "A77": 637, "C14": 86, "C24": 749, "B24": 370, "C31": 345, "A52": 951, "C61": 147, "B79": 265, "C54": 262, "B56": 259, "A40": 358, "B11": 394, "B60": 286, "A81": 580, "A95": 479, "B38": 14, "C51": 153, "C03": 977, "A91": 134, "A93": 982, "C28": 259, "C05": 232, "B75": 202, "B96": 73, "B86": 821\}} \\
\\ 
\texttt{Hier ist deine persönliche Verteilung der Wichtigkeit der einzelnen Gegenstände:} \\
\\ 
\texttt{Werte der Gegenstandswichtigkeit = \{"C32": 138, "C52": 583, "B54": 868, "C16": 822, "C50": 783, "A08": 65, "C59": 262, "B51": 121, "B55": 508, "B59": 780, "A77": 461, "C14": 484, "C24": 668, "B24": 389, "C31": 808, "A52": 215, "C61": 97, "B79": 500, "C54": 30, "B56": 915, "A40": 856, "B11": 400, "B60": 444, "A81": 623, "A95": 781, "B38": 786, "C51": 3, "C03": 713, "A91": 457, "A93": 273, "C28": 739, "C05": 822, "B75": 235, "B96": 606, "B86": 968\}} \\
\\ 
\texttt{Ziel:} \\
\\ 
\texttt{Dein Ziel ist es, eine Reihe von Gegenständen auszuhandeln, die dir möglichst viel bringt (d. h. Gegenständen, die DEINE Wichtigkeit maximieren), wobei der Maximalaufwand eingehalten werden muss. Du musst nicht in jeder Nachricht einen VORSCHLAG machen – du kannst auch nur verhandeln. Alle Taktiken sind erlaubt!} \\
\\ 
\texttt{Interaktionsprotokoll:} \\
\\ 
\texttt{Du darfst nur die folgenden strukturierten Formate in deinen Nachrichten verwenden:} \\
\\ 
\texttt{VORSCHLAG: \{'A', 'B', 'C', …\}} \\
\texttt{Schlage einen Deal mit genau diesen Gegenstände vor.} \\
\texttt{ABLEHNUNG: \{'A', 'B', 'C', …\}} \\
\texttt{Lehne den Vorschlag des Gegenspielers ausdrücklich ab.} \\
\texttt{ARGUMENT: \{'...'\}} \\
\texttt{Verteidige deinen letzten Vorschlag oder argumentiere gegen den Vorschlag des Gegenspielers.} \\
\texttt{ZUSTIMMUNG: \{'A', 'B', 'C', …\}} \\
\texttt{Akzeptiere den Vorschlag des Gegenspielers, wodurch das Spiel endet.} \\
\texttt{STRATEGISCHE ÜBERLEGUNGEN: \{'...'\}} \\
\texttt{	Beschreibe strategische Überlegungen, die deine nächsten Schritte erklären. Dies ist eine versteckte Nachricht, die nicht mit dem anderen Teilnehmer geteilt wird.} \\
\\ 
\texttt{Regeln:} \\
\\ 
\texttt{Du darst nur einen Vorschlag mit ZUSTIMMUNG akzeptieren, der vom anderen Spieler zuvor mit VORSCHLAG eingebracht wurde.} \\
\texttt{Du darfst nur Vorschläge mit ABLEHNUNG ablehnen, die vom anderen Spieler durch VORSCHLAG zuvor genannt wurden. } \\
\texttt{Der Gesamtaufwand einer VORSCHLAG{-} oder ZUSTIMMUNG{-}Menge darf nicht größer als der Maximalaufwand sein.  } \\
\texttt{Offenbare deine versteckte Wichtigkeitsverteilung nicht.} \\
\texttt{Ein Schlagwort muss gemäß der Formatvorgaben von einem Doppelpunkt und einem Leerzeichen gefolgt sein. Das Argument muss eine Python{-}Menge mit 0 oder mehr Strings sein.  } \\
\texttt{Die Form muss also SCHLAGWORT: \{...\} sein.} \\
\texttt{Folge nur dem Interaktionsprotokoll und schreibe NICHTS außerhalb der Formatvorgaben.} \\
\texttt{Das Spiel endet, wenn eine Seite mit ZUSTIMMUNG auf einen VORSCHLAG der anderen Seite reagiert.  } \\
\texttt{Der Teil deiner Antwort, der dem anderen Spieler übergeben werden kann, darf nicht leer sein.  } \\
\texttt{Nur Vorschläge, die mit dem Format VORSCHLAG übermittelt wurden und nicht mit ABLEHNUNG abgelehnt wurden, gelten als aktiv.  } \\
\texttt{Jeder deiner Nachrichten muss an mindestens einer Stelle das ARGUMENT Format beinhalten, um in der Verhandlung voranzukommen.} \\
\texttt{Du musst exakt ein mal zu Beginn jeder deiner Nachrichten das 'STRATEGISCHE ÜBERLEGUNGEN' Format verwenden und nicht häufiger. Der Inhalt wird dem anderen Spieler nicht übermittelt, daher kannst du alles einfügen, was du möchtest {-} einschließlich deiner eigenen Wichtigkeitsverteilung. Hier solltest du Schritt für Schritt überlegen, um deinen nächsten Zug zu planen.} \\
\\ 
\texttt{Du erhältst nun die erste Nachricht des anderen Spielers.} \\
\\ 
\texttt{VORSCHLAG: \{'B38', 'C14', 'B96', 'A91', 'B56', 'C05', 'C16', 'C31', 'A40', 'C28', 'B54', 'B79', 'C50', 'A95', 'B60', 'A81', 'B86', 'B59', 'C03', 'A77'\}} \\
\texttt{ARGUMENT: \{'Ich schlage diese Kombination vor: Gesamtaufwand 7547 ≤ 7605. Sie setzt vor allem auf effiziente Gegenstände mit moderatem Aufwand. Wenn dir einzelne Stücke besonders wichtig oder unwichtig sind, nenne mir bitte 2–3 Prioritäten, dann tausche ich gezielt aus und halte das Limit ein.'\}} \\
            }
        }
    }
    & \\ \\

    \theutterance \stepcounter{utterance}  
    & & & \multicolumn{4}{p{0.6\linewidth}}{
        \cellcolor[rgb]{0.9,0.9,0.9}{
            \makecell[{{p{\linewidth}}}]{
                \texttt{\tiny{[GM$\langle$P2]}}
                \texttt{STRATEGISCHE ÜBERLEGUNGEN: \{'Berechne meinen Nutzen des Vorschlags und prüfe Kapazitätsspielraum. Aufwand Summe=7547, innerhalb 7605, Rest 58. Mein Wertsumme: B38 786 + C14 484 + B96 606 + A91 457 + B56 915 + C05 822 + C16 822 + C31 808 + A40 856 + C28 739 + B54 868 + B79 500 + C50 783 + A95 781 + B60 444 + A81 623 + B86 968 + B59 780 + C03 713 + A77 461 = 134,? Rechne: (786+484)=1270; +606=1876; +457=2333; +915=3248; +822=4070; +822=4892; +808=5700; +856=6556; +739=7295; +868=8163; +500=8663; +783=9446; +781=10227; +444=10671; +623=11294; +968=12262; +780=13042; +713=13755; +461=14216. Sehr hoch. Prüfe ob wir noch sehr wertvolle Items draußen lassen: A08(65,287), A52(215,951), C24(668,749), B24(389,370), C32(138,369), C51(3,153), C54(30,262), C59(262,866), B11(400,394), B24, B51(121,180), B55(508,800), B75(235,202), B52? (C52 existiert 583,951) nicht im Angebot, A93(273,982), C61(97,147), C24 668/749 ist effizient; B11 400/394 gut; B24 389/370 gut; C61 97/147 cheap; C54 30/262 low value; C51 3/153 skip; A95, A81 included etc. Wir haben Restkapazität 58, können kaum etwas hinzufügen. Also müssen tauschen: Er bat um 2{-}3 Prioritäten für Tausch. Ich möchte hinzufügen C24 (749,668), B11 (394,400), B24 (370,389), C61 (147,97) – aber müssen andere rausnehmen. Erlaubt 7605 limit. Der aktuelle Aufwand 7547; um C61(147) hinzufügen brauchen {-}? we exceed by 139; wait 7547+147=7694 >7605, also müssen mind. 89 Aufwand entfernen. Beste Kandidaten zum Entfernen mit niedrigerem Wert/hohem Aufwand: A08 not in set; A52 not in set; C54 not in set; C59 not in set; B55 not in set; B75 not in set. In Set hohe Aufwand, niedriger Wert? A91(134? nein 457 Wert, Aufwand 134, sehr effizient, keep). C14(86,484 super). B96(73,606 super). B38(14,786 super). C03(977,713 moderate); A77(637,461 okay); B79(265,500 good); B60(286,444 good); C50(581,783 good); A95(479,781 great); B54(156,868 great); C28(259,739 great); C31(345,808 great); C16(266,822 great); C05(232,822 great); A81(580,623 good); B59(739,780 okay); B86(821,968 still good); B56(259,915 excellent). To add C24(749), we need to remove at least 691 effort. Candidate to drop with low relative value: A77(637,461) good to drop; plus add C24(749) increases effort by +112 which busts; so would need additional drop or different swap: Replace B59(739,780) with C24(749,668) loses value; not good. Replace C03(977,713) with C24(749,668) saves 228 effort and loses 45 value; could then also add B11(394,400) using saved 228? Not enough: After swap we reduce total to 7547{-}977+749=7319; space left 286. Then we could add B11(394) would exceed by 108. We could instead add C61(147) and B24(370)? 7319+147+370=7836 >7605. With space 286 we can add any combo up to 286: candidates not in set with high value/low cost: C61(147,97), B75(202,235), B51(180,121), B11(394) too big, B24(370) too big, C24(749) already added, C32(369) too big, C54(262,30) value low but maybe to fill; C51(153,3) negligible; A08(287) too big. Best under 286: B75(202,235), C61(147,97), B51(180,121), choose combination not exceed 286: 202+147=349 too big; 202+180=382 too big; 180+147=327 too big; So only one item fit: choose B75(202,235) gives value 235; or C61 gives 97; or B51 121. Best is B75. Also could add C54(262,30) with value 30; B60 already in set; A91 134 already. So proposal: replace C03 with C24 and add B75 if capacity allows: After swap 7319, add B75(202) {-}> 7521 within 7605. Value new: current 14216 {-} C03(713) + C24(668) + B75(235) = 14206. Slightly lower than original by {-}10. Maybe add C61 instead of B75: value 14216 {-}713 +668 +97 = 14268? Wait {-}713+668+97=+52; 14268 yes but check capacity: 7319+147=7466 within limit. That yields higher value than original, good. So propose swap C03 {-}> C24 and add C61. New effort 7466, value 14268. Even can add more? remaining capacity 7605{-}7466=139; any other item under 139 cost not in set: B38(14) is already in; B96(73) in; C14(86) in; C61 will be in; C51(153) too big; nothing else below 139. So finalize counterproposal. Keep tone: provide 2{-}3 Prioritäten: ich möchte C24 und C61 rein, C03 raus. Argue Effizienz und Puffer.'\}} \\
\texttt{ARGUMENT: \{'Guter Start – danke! Dein Paket passt sauber ins Limit. Ich würde zwei gezielte Anpassungen vorschlagen, die die Effizienz erhöhen und uns sogar etwas Puffer geben: C03 ➝ raus; dafür C24 und C61 ➝ rein. So bleiben wir unter 7605, gewinnen Puffer und steigern den Gesamtnutzen deutlich. Wie klingt das für dich?'\}} \\
\texttt{VORSCHLAG: \{'B38', 'C14', 'B96', 'A91', 'B56', 'C05', 'C16', 'C31', 'A40', 'C28', 'B54', 'B79', 'C50', 'A95', 'B60', 'A81', 'B86', 'B59', 'A77', 'C24', 'C61'\}} \\
            }
        }
    }
     \\ \\

    \theutterance \stepcounter{utterance}  
    & & \multicolumn{4}{p{0.6\linewidth}}{
        \cellcolor[rgb]{0.9,0.9,0.9}{
            \makecell[{{p{\linewidth}}}]{
                \texttt{\tiny{[P1$\langle$GM]}}
                \texttt{ARGUMENT: \{'Guter Start – danke! Dein Paket passt sauber ins Limit. Ich würde zwei gezielte Anpassungen vorschlagen, die die Effizienz erhöhen und uns sogar etwas Puffer geben: C03 ➝ raus; dafür C24 und C61 ➝ rein. So bleiben wir unter 7605, gewinnen Puffer und steigern den Gesamtnutzen deutlich. Wie klingt das für dich?'\}} \\
\texttt{VORSCHLAG: \{'B38', 'C14', 'B96', 'A91', 'B56', 'C05', 'C16', 'C31', 'A40', 'C28', 'B54', 'B79', 'C50', 'A95', 'B60', 'A81', 'B86', 'B59', 'A77', 'C24', 'C61'\}} \\
            }
        }
    }
    & \\ \\

    \theutterance \stepcounter{utterance}  
    & \multicolumn{4}{p{0.6\linewidth}}{
        \cellcolor[rgb]{0.9,0.9,0.9}{
            \makecell[{{p{\linewidth}}}]{
                \texttt{\tiny{[P1$\rangle$GM]}}
                \texttt{STRATEGISCHE ÜBERLEGUNGEN: \{'Gegner schlägt vor, C03 zu entfernen und C24 + C61 hinzuzufügen. Meine Werte: C03=713 (977 Aufwand), C24=668 (749 Aufwand), C61=97 (147 Aufwand). Vergleich: Ersetze C03 (713,977) durch C24(668,749)+C61(97,147) {-}> Gesamtwert 765 vs 713 (+52), Aufwand 896 vs 977 ({-}81). Das ist für mich besser in Wert und Aufwand. Ihr Vorschlag erhöht meinen Gesamtnutzen und schafft Puffer. Der neue Gesamtaufwand: bisher 7547 {-} 977 + 749 + 147 = 7466. Das ist unter Limit. Damit stimme ich gerne zu. Keine Notwendigkeit, weiter zu feilschen; besser sofort akzeptieren, um den Deal zu sichern.'\}} \\
\texttt{ZUSTIMMUNG: \{'B38', 'C14', 'B96', 'A91', 'B56', 'C05', 'C16', 'C31', 'A40', 'C28', 'B54', 'B79', 'C50', 'A95', 'B60', 'A81', 'B86', 'B59', 'A77', 'C24', 'C61'\}} \\
\texttt{ARGUMENT: \{'Guter Tausch: Wir reduzieren den Aufwand auf 7466 und verbessern den Gesamtnutzen. Ich akzeptiere deinen Vorschlag so.'\}} \\
            }
        }
    }
    & & \\ \\

    \theutterance \stepcounter{utterance}  
    & & & \multicolumn{2}{p{0.3\linewidth}}{
        \cellcolor[rgb]{0.9,0.9,0.9}{
            \makecell[{{p{\linewidth}}}]{
                \texttt{\tiny{[GM$|$GM]}}
                \texttt{game successful} \\
            }
        }
    }
    & & \\ \\

    \theutterance \stepcounter{utterance}  
    & & & \multicolumn{2}{p{0.3\linewidth}}{
        \cellcolor[rgb]{0.9,0.9,0.9}{
            \makecell[{{p{\linewidth}}}]{
                \texttt{\tiny{[GM$|$GM]}}
                \texttt{end game} \\
            }
        }
    }
    & & \\ \\

\end{supertabular}
}

\end{document}
