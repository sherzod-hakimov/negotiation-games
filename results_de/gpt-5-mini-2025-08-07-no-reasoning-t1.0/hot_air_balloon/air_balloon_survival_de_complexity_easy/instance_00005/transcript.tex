
\documentclass{article}
\usepackage{colortbl}
\usepackage{makecell}
\usepackage{multirow}
\usepackage{supertabular}

\begin{document}

\newcounter{utterance}

\twocolumn

{ \footnotesize  \setcounter{utterance}{1}
\setlength{\tabcolsep}{0pt}
\begin{supertabular}{c@{$\;$}|p{.15\linewidth}@{}p{.15\linewidth}p{.15\linewidth}p{.15\linewidth}p{.15\linewidth}p{.15\linewidth}}

    \# & $\;$A & \multicolumn{4}{c}{Game Master} & $\;\:$B\\
    \hline 

    \theutterance \stepcounter{utterance}  

    & & \multicolumn{4}{p{0.6\linewidth}}{\cellcolor[rgb]{0.9,0.9,0.9}{%
	\makecell[{{p{\linewidth}}}]{% 
	  \tt {\tiny [A$\langle$GM]}  
	 Du nimmst an einem kollaborativen Verhandlungspiel Teil.\\ \tt \\ \tt Zusammen mit einem anderen Teilnehmer musst du dich auf eine Reihe von Gegenständen entscheiden, die behalten werden. Jeder von euch hat eine persönliche Verteilung über die Wichtigkeit der einzelnen Gegenstände. Jeder von euch hat eine eigene Meinung darüber, wie wichtig jeder einzelne Gegenstand ist (Gegenstandswichtigkeit). Du kennst die Wichtigkeitsverteilung des anderen Spielers nicht. Zusätzlich siehst du, wie viel Aufwand jeder Gegenstand verursacht.  \\ \tt Ihr dürft euch nur auf eine Reihe von Gegenständen einigen, wenn der Gesamtaufwand der ausgewählten Gegenstände den Maximalaufwand nicht überschreitet:\\ \tt \\ \tt Maximalaufwand = 9475\\ \tt \\ \tt Hier sind die einzelnen Aufwände der Gegenstände:\\ \tt \\ \tt Aufwand der Gegenstände = {"C32": 944, "C52": 568, "B54": 515, "C16": 239, "C50": 876, "A08": 421, "C59": 991, "B51": 281, "B55": 788, "B59": 678, "A77": 432, "C14": 409, "C24": 279, "B24": 506, "C31": 101, "A52": 686, "C61": 851, "B79": 854, "C54": 133, "B56": 192, "A40": 573, "B11": 17, "B60": 465, "A81": 772, "A95": 46, "B38": 501, "C51": 220, "C03": 404, "A91": 843, "A93": 748, "C28": 552, "C05": 843, "B75": 947, "B96": 345, "B86": 930}\\ \tt \\ \tt Hier ist deine persönliche Verteilung der Wichtigkeit der einzelnen Gegenstände:\\ \tt \\ \tt Werte der Gegenstandswichtigkeit = {"C32": 138, "C52": 583, "B54": 868, "C16": 822, "C50": 783, "A08": 65, "C59": 262, "B51": 121, "B55": 508, "B59": 780, "A77": 461, "C14": 484, "C24": 668, "B24": 389, "C31": 808, "A52": 215, "C61": 97, "B79": 500, "C54": 30, "B56": 915, "A40": 856, "B11": 400, "B60": 444, "A81": 623, "A95": 781, "B38": 786, "C51": 3, "C03": 713, "A91": 457, "A93": 273, "C28": 739, "C05": 822, "B75": 235, "B96": 606, "B86": 968}\\ \tt \\ \tt Ziel:\\ \tt \\ \tt Dein Ziel ist es, eine Reihe von Gegenständen auszuhandeln, die dir möglichst viel bringt (d. h. Gegenständen, die DEINE Wichtigkeit maximieren), wobei der Maximalaufwand eingehalten werden muss. Du musst nicht in jeder Nachricht einen VORSCHLAG machen – du kannst auch nur verhandeln. Alle Taktiken sind erlaubt!\\ \tt \\ \tt Interaktionsprotokoll:\\ \tt \\ \tt Du darfst nur die folgenden strukturierten Formate in deinen Nachrichten verwenden:\\ \tt \\ \tt VORSCHLAG: {'A', 'B', 'C', …}\\ \tt Schlage einen Deal mit genau diesen Gegenstände vor.\\ \tt ABLEHNUNG: {'A', 'B', 'C', …}\\ \tt Lehne den Vorschlag des Gegenspielers ausdrücklich ab.\\ \tt ARGUMENT: {'...'}\\ \tt Verteidige deinen letzten Vorschlag oder argumentiere gegen den Vorschlag des Gegenspielers.\\ \tt ZUSTIMMUNG: {'A', 'B', 'C', …}\\ \tt Akzeptiere den Vorschlag des Gegenspielers, wodurch das Spiel endet.\\ \tt STRATEGISCHE ÜBERLEGUNGEN: {'...'}\\ \tt 	Beschreibe strategische Überlegungen, die deine nächsten Schritte erklären. Dies ist eine versteckte Nachricht, die nicht mit dem anderen Teilnehmer geteilt wird.\\ \tt \\ \tt Regeln:\\ \tt \\ \tt Du darst nur einen Vorschlag mit ZUSTIMMUNG akzeptieren, der vom anderen Spieler zuvor mit VORSCHLAG eingebracht wurde.\\ \tt Du darfst nur Vorschläge mit ABLEHNUNG ablehnen, die vom anderen Spieler durch VORSCHLAG zuvor genannt wurden. \\ \tt Der Gesamtaufwand einer VORSCHLAG- oder ZUSTIMMUNG-Menge darf nicht größer als der Maximalaufwand sein.  \\ \tt Offenbare deine versteckte Wichtigkeitsverteilung nicht.\\ \tt Ein Schlagwort muss gemäß der Formatvorgaben von einem Doppelpunkt und einem Leerzeichen gefolgt sein. Das Argument muss eine Python-Menge mit 0 oder mehr Strings sein.  \\ \tt Die Form muss also SCHLAGWORT: {...} sein.\\ \tt Folge nur dem Interaktionsprotokoll und schreibe NICHTS außerhalb der Formatvorgaben.\\ \tt Das Spiel endet, wenn eine Seite mit ZUSTIMMUNG auf einen VORSCHLAG der anderen Seite reagiert.  \\ \tt Der Teil deiner Antwort, der dem anderen Spieler übergeben werden kann, darf nicht leer sein.  \\ \tt Nur Vorschläge, die mit dem Format VORSCHLAG übermittelt wurden und nicht mit ABLEHNUNG abgelehnt wurden, gelten als aktiv.  \\ \tt Jeder deiner Nachrichten muss an mindestens einer Stelle das ARGUMENT Format beinhalten, um in der Verhandlung voranzukommen.\\ \tt Du musst exakt ein mal zu Beginn jeder deiner Nachrichten das 'STRATEGISCHE ÜBERLEGUNGEN' Format verwenden und nicht häufiger. Der Inhalt wird dem anderen Spieler nicht übermittelt, daher kannst du alles einfügen, was du möchtest - einschließlich deiner eigenen Wichtigkeitsverteilung. Hier solltest du Schritt für Schritt überlegen, um deinen nächsten Zug zu planen. 
	  } 
	   } 
	   } 
	 & \\ 
 

    \theutterance \stepcounter{utterance}  

    & \multicolumn{4}{p{0.6\linewidth}}{\cellcolor[rgb]{0.8,1,0.9}{%
	\makecell[{{p{\linewidth}}}]{% 
	  \tt {\tiny [A$\rangle$GM]}  
	 STRATEGISCHE ÜBERLEGUNGEN: {'Ziel: möglichst hoher eigener Wert unter 9475 Aufwand, gleichzeitig Raum für Zugeständnisse lassen. Sortiere nach Wert/Aufwand und absolutem Wert. Top-Werte: B86(968/930), B56(915/192), B54(868/515), A40(856/573), C16(822/239), C05(822/843), C31(808/101), A95(781/46), B38(786/501), B59(780/678), C28(739/552), C03(713/404), A81(623/772), B96(606/345), C24(668/279), C52(583/568), B79(500/854), B60(444/465), A77(461/432), A91(457/843), C14(484/409). Baue initial einen starken, aber nicht maximalen Korb mit sehr gutem Wert/Aufwand, um Anker zu setzen, aber noch Platz für Tausch (z.B. verzichte zunächst auf teure mittelwertige: B79, A81, A91). Rechne Aufwand und halte deutlich unter 9475, um später Erweiterungen zu tauschen. Auswahl: {B86, B56, B54, A40, C16, C31, A95, B38, C28, C03, C24, B96, C52, C14, B60, A77}. Aufwandsumme prüfen: B86 930 + B56 192 =1122; + B54 515=1637; + A40 573=2210; + C16 239=2449; + C31 101=2550; + A95 46=2596; + B38 501=3097; + C28 552=3649; + C03 404=4053; + C24 279=4332; + B96 345=4677; + C52 568=5245; + C14 409=5654; + B60 465=6119; + A77 432=6551. Wertsumme hoch (~9300+). Lasse C05(843/843), B59(780/678), A81, B79 als mögliche spätere Add-ons/Trade-offs. Strategie: Präsentieren mit Argument Effizienz und Luft nach oben; signalisiere Bereitschaft, einzelne Items gegen gleichaufwändige zu tauschen.'}\\ \tt VORSCHLAG: {'B86', 'B56', 'B54', 'A40', 'C16', 'C31', 'A95', 'B38', 'C28', 'C03', 'C24', 'B96', 'C52', 'C14', 'B60', 'A77'}\\ \tt ARGUMENT: {'Ich schlage diese Auswahl vor, weil sie bei moderatem Gesamtaufwand eine sehr ausgewogene, effiziente Kombination ergibt und noch ausreichend Budget-Puffer lässt. Wenn dir einzelne Positionen besonders wichtig sind, können wir 1:1 gegen ähnlich aufwändige Items tauschen oder um ein bis zwei Posten erweitern, solange wir unter dem Limit bleiben.'} 
	  } 
	   } 
	   } 
	 & & \\ 
 

    \theutterance \stepcounter{utterance}  

    & & \multicolumn{4}{p{0.6\linewidth}}{\cellcolor[rgb]{0.9,0.9,0.9}{%
	\makecell[{{p{\linewidth}}}]{% 
	  \tt {\tiny [GM$\rangle$B]}  
	 Du nimmst an einem kollaborativen Verhandlungspiel Teil.\\ \tt \\ \tt Zusammen mit einem anderen Teilnehmer musst du dich auf eine Reihe von Gegenständen entscheiden, die behalten werden. Jeder von euch hat eine persönliche Verteilung über die Wichtigkeit der einzelnen Gegenstände. Jeder von euch hat eine eigene Meinung darüber, wie wichtig jeder einzelne Gegenstand ist (Gegenstandswichtigkeit). Du kennst die Wichtigkeitsverteilung des anderen Spielers nicht. Zusätzlich siehst du, wie viel Aufwand jeder Gegenstand verursacht.  \\ \tt Ihr dürft euch nur auf eine Reihe von Gegenständen einigen, wenn der Gesamtaufwand der ausgewählten Gegenstände den Maximalaufwand nicht überschreitet:\\ \tt \\ \tt Maximalaufwand = 9475\\ \tt \\ \tt Hier sind die einzelnen Aufwände der Gegenstände:\\ \tt \\ \tt Aufwand der Gegenstände = {"C32": 944, "C52": 568, "B54": 515, "C16": 239, "C50": 876, "A08": 421, "C59": 991, "B51": 281, "B55": 788, "B59": 678, "A77": 432, "C14": 409, "C24": 279, "B24": 506, "C31": 101, "A52": 686, "C61": 851, "B79": 854, "C54": 133, "B56": 192, "A40": 573, "B11": 17, "B60": 465, "A81": 772, "A95": 46, "B38": 501, "C51": 220, "C03": 404, "A91": 843, "A93": 748, "C28": 552, "C05": 843, "B75": 947, "B96": 345, "B86": 930}\\ \tt \\ \tt Hier ist deine persönliche Verteilung der Wichtigkeit der einzelnen Gegenstände:\\ \tt \\ \tt Werte der Gegenstandswichtigkeit = {"C32": 138, "C52": 583, "B54": 868, "C16": 822, "C50": 783, "A08": 65, "C59": 262, "B51": 121, "B55": 508, "B59": 780, "A77": 461, "C14": 484, "C24": 668, "B24": 389, "C31": 808, "A52": 215, "C61": 97, "B79": 500, "C54": 30, "B56": 915, "A40": 856, "B11": 400, "B60": 444, "A81": 623, "A95": 781, "B38": 786, "C51": 3, "C03": 713, "A91": 457, "A93": 273, "C28": 739, "C05": 822, "B75": 235, "B96": 606, "B86": 968}\\ \tt \\ \tt Ziel:\\ \tt \\ \tt Dein Ziel ist es, eine Reihe von Gegenständen auszuhandeln, die dir möglichst viel bringt (d. h. Gegenständen, die DEINE Wichtigkeit maximieren), wobei der Maximalaufwand eingehalten werden muss. Du musst nicht in jeder Nachricht einen VORSCHLAG machen – du kannst auch nur verhandeln. Alle Taktiken sind erlaubt!\\ \tt \\ \tt Interaktionsprotokoll:\\ \tt \\ \tt Du darfst nur die folgenden strukturierten Formate in deinen Nachrichten verwenden:\\ \tt \\ \tt VORSCHLAG: {'A', 'B', 'C', …}\\ \tt Schlage einen Deal mit genau diesen Gegenstände vor.\\ \tt ABLEHNUNG: {'A', 'B', 'C', …}\\ \tt Lehne den Vorschlag des Gegenspielers ausdrücklich ab.\\ \tt ARGUMENT: {'...'}\\ \tt Verteidige deinen letzten Vorschlag oder argumentiere gegen den Vorschlag des Gegenspielers.\\ \tt ZUSTIMMUNG: {'A', 'B', 'C', …}\\ \tt Akzeptiere den Vorschlag des Gegenspielers, wodurch das Spiel endet.\\ \tt STRATEGISCHE ÜBERLEGUNGEN: {'...'}\\ \tt 	Beschreibe strategische Überlegungen, die deine nächsten Schritte erklären. Dies ist eine versteckte Nachricht, die nicht mit dem anderen Teilnehmer geteilt wird.\\ \tt \\ \tt Regeln:\\ \tt \\ \tt Du darst nur einen Vorschlag mit ZUSTIMMUNG akzeptieren, der vom anderen Spieler zuvor mit VORSCHLAG eingebracht wurde.\\ \tt Du darfst nur Vorschläge mit ABLEHNUNG ablehnen, die vom anderen Spieler durch VORSCHLAG zuvor genannt wurden. \\ \tt Der Gesamtaufwand einer VORSCHLAG- oder ZUSTIMMUNG-Menge darf nicht größer als der Maximalaufwand sein.  \\ \tt Offenbare deine versteckte Wichtigkeitsverteilung nicht.\\ \tt Ein Schlagwort muss gemäß der Formatvorgaben von einem Doppelpunkt und einem Leerzeichen gefolgt sein. Das Argument muss eine Python-Menge mit 0 oder mehr Strings sein.  \\ \tt Die Form muss also SCHLAGWORT: {...} sein.\\ \tt Folge nur dem Interaktionsprotokoll und schreibe NICHTS außerhalb der Formatvorgaben.\\ \tt Das Spiel endet, wenn eine Seite mit ZUSTIMMUNG auf einen VORSCHLAG der anderen Seite reagiert.  \\ \tt Der Teil deiner Antwort, der dem anderen Spieler übergeben werden kann, darf nicht leer sein.  \\ \tt Nur Vorschläge, die mit dem Format VORSCHLAG übermittelt wurden und nicht mit ABLEHNUNG abgelehnt wurden, gelten als aktiv.  \\ \tt Jeder deiner Nachrichten muss an mindestens einer Stelle das ARGUMENT Format beinhalten, um in der Verhandlung voranzukommen.\\ \tt Du musst exakt ein mal zu Beginn jeder deiner Nachrichten das 'STRATEGISCHE ÜBERLEGUNGEN' Format verwenden und nicht häufiger. Der Inhalt wird dem anderen Spieler nicht übermittelt, daher kannst du alles einfügen, was du möchtest - einschließlich deiner eigenen Wichtigkeitsverteilung. Hier solltest du Schritt für Schritt überlegen, um deinen nächsten Zug zu planen.\\ \tt \\ \tt Du erhältst nun die erste Nachricht des anderen Spielers.\\ \tt \\ \tt VORSCHLAG: {'B86', 'B56', 'B54', 'A40', 'C16', 'C31', 'A95', 'B38', 'C28', 'C03', 'C24', 'B96', 'C52', 'C14', 'B60', 'A77'}\\ \tt ARGUMENT: {'Ich schlage diese Auswahl vor, weil sie bei moderatem Gesamtaufwand eine sehr ausgewogene, effiziente Kombination ergibt und noch ausreichend Budget-Puffer lässt. Wenn dir einzelne Positionen besonders wichtig sind, können wir 1:1 gegen ähnlich aufwändige Items tauschen oder um ein bis zwei Posten erweitern, solange wir unter dem Limit bleiben.'} 
	  } 
	   } 
	   } 
	 & \\ 
 

    \theutterance \stepcounter{utterance}  

    & & & \multicolumn{4}{p{0.6\linewidth}}{\cellcolor[rgb]{1,0.85,0.72}{%
	\makecell[{{p{\linewidth}}}]{% 
	  \tt {\tiny [GM$\langle$B]}  
	 STRATEGISCHE ÜBERLEGUNGEN: {'Zuerst prüfe ich Kosten und meinen Nutzen. Kosten der vorgeschlagenen Menge: B86 930 + B56 192 + B54 515 + A40 573 + C16 239 + C31 101 + A95 46 + B38 501 + C28 552 + C03 404 + C24 279 + B96 345 + C52 568 + C14 409 + B60 465 + A77 432 = 6551 (unter 9475). Mein Nutzen dieser Menge: B86 968 + B56 915 + B54 868 + A40 856 + C16 822 + C31 808 + A95 781 + B38 786 + C28 739 + C03 713 + C24 668 + B96 606 + C52 583 + C14 484 + B60 444 + A77 461 = 112,? Summiere genau: 968+915=1883; +868=2751; +856=3607; +822=4429; +808=5237; +781=6018; +786=6804; +739=7543; +713=8256; +668=8924; +606=9530; +583=10113; +484=10597; +444=11041; +461=11502. Sehr gut für mich. Es bleibt Budget 2924 übrig. Ich möchte weitere für mich sehr wertvolle Items hinzufügen: B59 780 (Kosten 678), B38 ist schon drin, A95 ist drin, A40 drin, B56 drin, C05 822 (Kosten 843), C16/C31 drin, B54 drin, B86 drin. Weitere hohe Werte: B60 schon drin, C28 drin, C03 drin, C24 drin. Hohe: C50 783 (Kosten 876), A81 623 (Kosten 772), A91 457 (843 Kosten), A93 273 (748), B79 500 (854), B55 508 (788), A52 215 (686), C59 262 (991, teuer), C61 97 (851, gering Wert), C51 3 (220, unnötig), B11 400 (17, extrem günstig, guter Zusatz), A08 65 (421, niedrig Wert), C32 138 (944, meh), B24 389 (506), B75 235 (947), B52? (n/a). Ich sollte vorschlagen, Budget mit hoch-nutzen/günstig zu füllen: B11 (17) sofort. Danach B59 (678/780), C05 (843/822), C50 (876/783). Kosten addieren: aktuelle 6551 +17=6568; +678=7246; +843=8089; +876=8965. Restbudget 510. Mit Rest könnte B24 (506/389) genau passen: 8965+506=9471 <= 9475. Gesamtwert-Zuwachs: B11 400, B59 780, C05 822, C50 783, B24 389 -> +3174. Neuer Gesamtwert ca 14676, Kosten 9471 unter Limit. Prüfe andere Kombis: vielleicht statt B24 etwas B60 ist drin; A77 drin. Könnten auch C54 (133/30) geringer Wert, aber billig; C31 schon drin. Vielleicht B51 (281/121) billig; aber Wert niedriger. Meine Auswahl oben maximiert meinen Wert stark. Risiko: Gegenspieler könnte einige davon hoch bewerten; aber wir haben Flexibilität. Ich argumentiere mit Effizienz und verbleibendem Puffer ausgeschöpft. Ich akzeptiere vieles von ihm, erweitere um 5 Items. Taktik: Kooperativ, aber pushen meiner Favoriten. Ich mache VORSCHLAG mit ursprünglicher Menge plus {B11, B59, C05, C50, B24}. Ich rahme es als Nutzung des Budgets und biete Bereitschaft zu 1:1 Tausch bei Kostenähnlichkeit an. Ich verrate nicht meine Werte.'}\\ \tt ARGUMENT: {'Dein Vorschlag ist solide und bleibt deutlich unter dem Limit. Ich würde das Budget gern effizient ausschöpfen und die Auswahl minimal erweitern, ohne das Kosten-Nutzen-Verhältnis zu verwässern. Ich schlage daher vor, sehr kosteneffiziente Ergänzungen aufzunehmen und das verbleibende Polster sinnvoll zu nutzen. Wenn dir einzelne dieser Ergänzungen nicht passen, können wir sie 1:1 gegen ähnlich aufwändige Posten tauschen.'}\\ \tt VORSCHLAG: {'B86', 'B56', 'B54', 'A40', 'C16', 'C31', 'A95', 'B38', 'C28', 'C03', 'C24', 'B96', 'C52', 'C14', 'B60', 'A77', 'B11', 'B59', 'C05', 'C50', 'B24'} 
	  } 
	   } 
	   } 
	  \\ 
 

    \theutterance \stepcounter{utterance}  

    & & \multicolumn{4}{p{0.6\linewidth}}{\cellcolor[rgb]{0.9,0.9,0.9}{%
	\makecell[{{p{\linewidth}}}]{% 
	  \tt {\tiny [A$\langle$GM]}  
	 ARGUMENT: {'Dein Vorschlag ist solide und bleibt deutlich unter dem Limit. Ich würde das Budget gern effizient ausschöpfen und die Auswahl minimal erweitern, ohne das Kosten-Nutzen-Verhältnis zu verwässern. Ich schlage daher vor, sehr kosteneffiziente Ergänzungen aufzunehmen und das verbleibende Polster sinnvoll zu nutzen. Wenn dir einzelne dieser Ergänzungen nicht passen, können wir sie 1:1 gegen ähnlich aufwändige Posten tauschen.'}\\ \tt VORSCHLAG: {'B86', 'B56', 'B54', 'A40', 'C16', 'C31', 'A95', 'B38', 'C28', 'C03', 'C24', 'B96', 'C52', 'C14', 'B60', 'A77', 'B11', 'B59', 'C05', 'C50', 'B24'} 
	  } 
	   } 
	   } 
	 & \\ 
 

    \theutterance \stepcounter{utterance}  

    & \multicolumn{4}{p{0.6\linewidth}}{\cellcolor[rgb]{0.8,1,0.9}{%
	\makecell[{{p{\linewidth}}}]{% 
	  \tt {\tiny [A$\rangle$GM]}  
	 STRATEGISCHE ÜBERLEGUNGEN: {'Gegner erweitert stark und nutzt Budget aus. Prüfe Aufwand: Mein Set 6551. Ergänzungen: B11 17 -> 6568; B59 678 -> 7246; C05 843 -> 8089; C50 876 -> 8965; B24 506 -> 9471. Gesamt 9471 <= 9475, passt. Werte für mich der Ergänzungen: B11(400) gut, B59(780) sehr gut, C05(822) top, C50(783) top, B24(389) mittel. Das Paket ist für mich extrem wertvoll und innerhalb des Limits. Annehmen sofort ist optimal, da kaum etwas zu verbessern ohne Risiko. Keine weiteren Forderungen; direkt Zustimmung geben.'}\\ \tt ARGUMENT: {'Dein erweitertes Paket nutzt das Budget optimal aus, bleibt unter dem Limit und ist aus meiner Sicht sehr effizient zusammengesetzt. Gute Balance aus Aufwand und Nutzen – so passt es.'}\\ \tt ZUSTIMMUNG: {'B86', 'B56', 'B54', 'A40', 'C16', 'C31', 'A95', 'B38', 'C28', 'C03', 'C24', 'B96', 'C52', 'C14', 'B60', 'A77', 'B11', 'B59', 'C05', 'C50', 'B24'} 
	  } 
	   } 
	   } 
	 & & \\ 
 

    \theutterance \stepcounter{utterance}  

    & & & \multicolumn{2}{p{0.3\linewidth}}{\cellcolor[rgb]{0.95,0.95,0.95}{%
	\makecell[{{p{\linewidth}}}]{% 
	  \tt {\tiny [GM$|$GM]}  
	 game successful 
	  } 
	   } 
	   } 
	 & & \\ 
 

    \theutterance \stepcounter{utterance}  

    & & & \multicolumn{2}{p{0.3\linewidth}}{\cellcolor[rgb]{0.95,0.95,0.95}{%
	\makecell[{{p{\linewidth}}}]{% 
	  \tt {\tiny [GM$|$GM]}  
	 end game 
	  } 
	   } 
	   } 
	 & & \\ 
 

\end{supertabular}
}

\end{document}
