
\documentclass{article}
\usepackage{colortbl}
\usepackage{makecell}
\usepackage{multirow}
\usepackage{supertabular}

\begin{document}

\newcounter{utterance}

\centering \large Interaction Transcript for game `hot\_air\_balloon', experiment `air\_balloon\_survival\_de\_complexity\_easy', episode 3 with gpt{-}5{-}mini{-}2025{-}08{-}07{-}no{-}reasoning{-}t1.0.
\vspace{24pt}

{ \footnotesize  \setcounter{utterance}{1}
\setlength{\tabcolsep}{0pt}
\begin{supertabular}{c@{$\;$}|p{.15\linewidth}@{}p{.15\linewidth}p{.15\linewidth}p{.15\linewidth}p{.15\linewidth}p{.15\linewidth}}
    \# & $\;$A & \multicolumn{4}{c}{Game Master} & $\;\:$B\\
    \hline

    \theutterance \stepcounter{utterance}  
    & & \multicolumn{4}{p{0.6\linewidth}}{
        \cellcolor[rgb]{0.9,0.9,0.9}{
            \makecell[{{p{\linewidth}}}]{
                \texttt{\tiny{[P1$\langle$GM]}}
                \texttt{Du nimmst an einem kollaborativen Verhandlungspiel Teil.} \\
\\ 
\texttt{Zusammen mit einem anderen Teilnehmer musst du dich auf eine Reihe von Gegenständen entscheiden, die behalten werden. Jeder von euch hat eine persönliche Verteilung über die Wichtigkeit der einzelnen Gegenstände. Jeder von euch hat eine eigene Meinung darüber, wie wichtig jeder einzelne Gegenstand ist (Gegenstandswichtigkeit). Du kennst die Wichtigkeitsverteilung des anderen Spielers nicht. Zusätzlich siehst du, wie viel Aufwand jeder Gegenstand verursacht.  } \\
\texttt{Ihr dürft euch nur auf eine Reihe von Gegenständen einigen, wenn der Gesamtaufwand der ausgewählten Gegenstände den Maximalaufwand nicht überschreitet:} \\
\\ 
\texttt{Maximalaufwand = 8770} \\
\\ 
\texttt{Hier sind die einzelnen Aufwände der Gegenstände:} \\
\\ 
\texttt{Aufwand der Gegenstände = \{"C32": 830, "C52": 638, "B54": 607, "C16": 106, "C50": 630, "A08": 382, "C59": 773, "B51": 455, "B55": 262, "B59": 634, "A77": 60, "C14": 54, "C24": 849, "B24": 801, "C31": 324, "A52": 164, "C61": 832, "B79": 136, "C54": 980, "B56": 946, "A40": 645, "B11": 842, "B60": 106, "A81": 116, "A95": 875, "B38": 446, "C51": 649, "C03": 601, "A91": 252, "A93": 763, "C28": 213, "C05": 517, "B75": 520, "B96": 407, "B86": 125\}} \\
\\ 
\texttt{Hier ist deine persönliche Verteilung der Wichtigkeit der einzelnen Gegenstände:} \\
\\ 
\texttt{Werte der Gegenstandswichtigkeit = \{"C32": 138, "C52": 583, "B54": 868, "C16": 822, "C50": 783, "A08": 65, "C59": 262, "B51": 121, "B55": 508, "B59": 780, "A77": 461, "C14": 484, "C24": 668, "B24": 389, "C31": 808, "A52": 215, "C61": 97, "B79": 500, "C54": 30, "B56": 915, "A40": 856, "B11": 400, "B60": 444, "A81": 623, "A95": 781, "B38": 786, "C51": 3, "C03": 713, "A91": 457, "A93": 273, "C28": 739, "C05": 822, "B75": 235, "B96": 606, "B86": 968\}} \\
\\ 
\texttt{Ziel:} \\
\\ 
\texttt{Dein Ziel ist es, eine Reihe von Gegenständen auszuhandeln, die dir möglichst viel bringt (d. h. Gegenständen, die DEINE Wichtigkeit maximieren), wobei der Maximalaufwand eingehalten werden muss. Du musst nicht in jeder Nachricht einen VORSCHLAG machen – du kannst auch nur verhandeln. Alle Taktiken sind erlaubt!} \\
\\ 
\texttt{Interaktionsprotokoll:} \\
\\ 
\texttt{Du darfst nur die folgenden strukturierten Formate in deinen Nachrichten verwenden:} \\
\\ 
\texttt{VORSCHLAG: \{'A', 'B', 'C', …\}} \\
\texttt{Schlage einen Deal mit genau diesen Gegenstände vor.} \\
\texttt{ABLEHNUNG: \{'A', 'B', 'C', …\}} \\
\texttt{Lehne den Vorschlag des Gegenspielers ausdrücklich ab.} \\
\texttt{ARGUMENT: \{'...'\}} \\
\texttt{Verteidige deinen letzten Vorschlag oder argumentiere gegen den Vorschlag des Gegenspielers.} \\
\texttt{ZUSTIMMUNG: \{'A', 'B', 'C', …\}} \\
\texttt{Akzeptiere den Vorschlag des Gegenspielers, wodurch das Spiel endet.} \\
\texttt{STRATEGISCHE ÜBERLEGUNGEN: \{'...'\}} \\
\texttt{	Beschreibe strategische Überlegungen, die deine nächsten Schritte erklären. Dies ist eine versteckte Nachricht, die nicht mit dem anderen Teilnehmer geteilt wird.} \\
\\ 
\texttt{Regeln:} \\
\\ 
\texttt{Du darst nur einen Vorschlag mit ZUSTIMMUNG akzeptieren, der vom anderen Spieler zuvor mit VORSCHLAG eingebracht wurde.} \\
\texttt{Du darfst nur Vorschläge mit ABLEHNUNG ablehnen, die vom anderen Spieler durch VORSCHLAG zuvor genannt wurden. } \\
\texttt{Der Gesamtaufwand einer VORSCHLAG{-} oder ZUSTIMMUNG{-}Menge darf nicht größer als der Maximalaufwand sein.  } \\
\texttt{Offenbare deine versteckte Wichtigkeitsverteilung nicht.} \\
\texttt{Ein Schlagwort muss gemäß der Formatvorgaben von einem Doppelpunkt und einem Leerzeichen gefolgt sein. Das Argument muss eine Python{-}Menge mit 0 oder mehr Strings sein.  } \\
\texttt{Die Form muss also SCHLAGWORT: \{...\} sein.} \\
\texttt{Folge nur dem Interaktionsprotokoll und schreibe NICHTS außerhalb der Formatvorgaben.} \\
\texttt{Das Spiel endet, wenn eine Seite mit ZUSTIMMUNG auf einen VORSCHLAG der anderen Seite reagiert.  } \\
\texttt{Der Teil deiner Antwort, der dem anderen Spieler übergeben werden kann, darf nicht leer sein.  } \\
\texttt{Nur Vorschläge, die mit dem Format VORSCHLAG übermittelt wurden und nicht mit ABLEHNUNG abgelehnt wurden, gelten als aktiv.  } \\
\texttt{Jeder deiner Nachrichten muss an mindestens einer Stelle das ARGUMENT Format beinhalten, um in der Verhandlung voranzukommen.} \\
\texttt{Du musst exakt ein mal zu Beginn jeder deiner Nachrichten das 'STRATEGISCHE ÜBERLEGUNGEN' Format verwenden und nicht häufiger. Der Inhalt wird dem anderen Spieler nicht übermittelt, daher kannst du alles einfügen, was du möchtest {-} einschließlich deiner eigenen Wichtigkeitsverteilung. Hier solltest du Schritt für Schritt überlegen, um deinen nächsten Zug zu planen.} \\
            }
        }
    }
    & \\ \\

    \theutterance \stepcounter{utterance}  
    & \multicolumn{4}{p{0.6\linewidth}}{
        \cellcolor[rgb]{0.9,0.9,0.9}{
            \makecell[{{p{\linewidth}}}]{
                \texttt{\tiny{[P1$\rangle$GM]}}
                \texttt{STRATEGISCHE ÜBERLEGUNGEN: \{'Max Aufwand 8770. Ich will meinen Nutzen maximieren. Sortiere nach Wert/Cost. Top{-}Werte: B86(968/125), B56(915/946), A40(856/645), B54(868/607), C16(822/106), C05(822/517), C31(808/324), A95(781/875), B38(786/446), B59(780/634), C03(713/601), C28(739/213), A81(623/116), B96(606/407), C52(583/638), B79(500/136), C14(484/54), A77(461/60), A91(457/252), B60(444/106), B24(389/801), A08(65/382), C54(30/980), C51(3/649). Beste Effizienz: C14, C16, A77, A81, B86, B79, C28, B60, C31, B96, B38, C05, B54, A40, B59, C03, A91, A95 etc. Konstruiere Vorschlag nahe 8770 mit vielen meiner Top{-}Prioritäten. Rechne Summe: Wähle zuerst günstige mit hohem Wert: B86(125), C16(106), C14(54), A77(60), A81(116), B79(136), C28(213), B60(106) {-}> Kosten 916. Füge C31(324)=1240, B96(407)=1647, B38(446)=2093, C05(517)=2610, B54(607)=3217, A40(645)=3862, B59(634)=4496, C03(601)=5097, A91(252)=5349, A95(875)=6224, B56(946)=7170, C52(638)=7808, C32(830)=8638 (unter 8770? 8638 ok). Rest Budget 132. Könnte statt C32 vielleicht C59(773) oder C24(849) passen? C32 ist ok und bringt 138 Wert. Wir haben 8638 Kosten, Wertsumme sehr hoch. Eventuell noch B79 ist schon drin. Vielleicht tauschen um Budget auszuschöpfen: Rest 132, nichts mit <=132 außer B79(136) schon drin, A77/A81/B60/C16/C14. Gut. Argumentiere auf Basis geringer Kosten, breiter Nutzen und Puffer. Halte Raum für Konzessionen: biete optional Austausch von C52 oder C32. Vermeide Offenlegung der Werte.'\}} \\
\\ 
\texttt{VORSCHLAG: \{'B86', 'C16', 'C14', 'A77', 'A81', 'B79', 'C28', 'B60', 'C31', 'B96', 'B38', 'C05', 'B54', 'A40', 'B59', 'C03', 'A91', 'A95', 'B56', 'C52', 'C32'\}} \\
\\ 
\texttt{ARGUMENT: \{'Ich schlage ein ausgewogenes Paket mit vielen effizienten, kostengünstigen Gegenständen plus einigen größeren Knotenpunkten vor. Der Gesamtaufwand bleibt klar unter dem Limit, sodass wir Spielraum behalten. Wenn dir einzelne Posten darin wichtig sind zu ersetzen, können wir über Austausche am Rand sprechen, z. B. C32 oder C52 gegen Alternativen ähnlicher Größe.'\}} \\
            }
        }
    }
    & & \\ \\

    \theutterance \stepcounter{utterance}  
    & & \multicolumn{4}{p{0.6\linewidth}}{
        \cellcolor[rgb]{0.9,0.9,0.9}{
            \makecell[{{p{\linewidth}}}]{
                \texttt{\tiny{[GM$\rangle$P2]}}
                \texttt{Du nimmst an einem kollaborativen Verhandlungspiel Teil.} \\
\\ 
\texttt{Zusammen mit einem anderen Teilnehmer musst du dich auf eine Reihe von Gegenständen entscheiden, die behalten werden. Jeder von euch hat eine persönliche Verteilung über die Wichtigkeit der einzelnen Gegenstände. Jeder von euch hat eine eigene Meinung darüber, wie wichtig jeder einzelne Gegenstand ist (Gegenstandswichtigkeit). Du kennst die Wichtigkeitsverteilung des anderen Spielers nicht. Zusätzlich siehst du, wie viel Aufwand jeder Gegenstand verursacht.  } \\
\texttt{Ihr dürft euch nur auf eine Reihe von Gegenständen einigen, wenn der Gesamtaufwand der ausgewählten Gegenstände den Maximalaufwand nicht überschreitet:} \\
\\ 
\texttt{Maximalaufwand = 8770} \\
\\ 
\texttt{Hier sind die einzelnen Aufwände der Gegenstände:} \\
\\ 
\texttt{Aufwand der Gegenstände = \{"C32": 830, "C52": 638, "B54": 607, "C16": 106, "C50": 630, "A08": 382, "C59": 773, "B51": 455, "B55": 262, "B59": 634, "A77": 60, "C14": 54, "C24": 849, "B24": 801, "C31": 324, "A52": 164, "C61": 832, "B79": 136, "C54": 980, "B56": 946, "A40": 645, "B11": 842, "B60": 106, "A81": 116, "A95": 875, "B38": 446, "C51": 649, "C03": 601, "A91": 252, "A93": 763, "C28": 213, "C05": 517, "B75": 520, "B96": 407, "B86": 125\}} \\
\\ 
\texttt{Hier ist deine persönliche Verteilung der Wichtigkeit der einzelnen Gegenstände:} \\
\\ 
\texttt{Werte der Gegenstandswichtigkeit = \{"C32": 138, "C52": 583, "B54": 868, "C16": 822, "C50": 783, "A08": 65, "C59": 262, "B51": 121, "B55": 508, "B59": 780, "A77": 461, "C14": 484, "C24": 668, "B24": 389, "C31": 808, "A52": 215, "C61": 97, "B79": 500, "C54": 30, "B56": 915, "A40": 856, "B11": 400, "B60": 444, "A81": 623, "A95": 781, "B38": 786, "C51": 3, "C03": 713, "A91": 457, "A93": 273, "C28": 739, "C05": 822, "B75": 235, "B96": 606, "B86": 968\}} \\
\\ 
\texttt{Ziel:} \\
\\ 
\texttt{Dein Ziel ist es, eine Reihe von Gegenständen auszuhandeln, die dir möglichst viel bringt (d. h. Gegenständen, die DEINE Wichtigkeit maximieren), wobei der Maximalaufwand eingehalten werden muss. Du musst nicht in jeder Nachricht einen VORSCHLAG machen – du kannst auch nur verhandeln. Alle Taktiken sind erlaubt!} \\
\\ 
\texttt{Interaktionsprotokoll:} \\
\\ 
\texttt{Du darfst nur die folgenden strukturierten Formate in deinen Nachrichten verwenden:} \\
\\ 
\texttt{VORSCHLAG: \{'A', 'B', 'C', …\}} \\
\texttt{Schlage einen Deal mit genau diesen Gegenstände vor.} \\
\texttt{ABLEHNUNG: \{'A', 'B', 'C', …\}} \\
\texttt{Lehne den Vorschlag des Gegenspielers ausdrücklich ab.} \\
\texttt{ARGUMENT: \{'...'\}} \\
\texttt{Verteidige deinen letzten Vorschlag oder argumentiere gegen den Vorschlag des Gegenspielers.} \\
\texttt{ZUSTIMMUNG: \{'A', 'B', 'C', …\}} \\
\texttt{Akzeptiere den Vorschlag des Gegenspielers, wodurch das Spiel endet.} \\
\texttt{STRATEGISCHE ÜBERLEGUNGEN: \{'...'\}} \\
\texttt{	Beschreibe strategische Überlegungen, die deine nächsten Schritte erklären. Dies ist eine versteckte Nachricht, die nicht mit dem anderen Teilnehmer geteilt wird.} \\
\\ 
\texttt{Regeln:} \\
\\ 
\texttt{Du darst nur einen Vorschlag mit ZUSTIMMUNG akzeptieren, der vom anderen Spieler zuvor mit VORSCHLAG eingebracht wurde.} \\
\texttt{Du darfst nur Vorschläge mit ABLEHNUNG ablehnen, die vom anderen Spieler durch VORSCHLAG zuvor genannt wurden. } \\
\texttt{Der Gesamtaufwand einer VORSCHLAG{-} oder ZUSTIMMUNG{-}Menge darf nicht größer als der Maximalaufwand sein.  } \\
\texttt{Offenbare deine versteckte Wichtigkeitsverteilung nicht.} \\
\texttt{Ein Schlagwort muss gemäß der Formatvorgaben von einem Doppelpunkt und einem Leerzeichen gefolgt sein. Das Argument muss eine Python{-}Menge mit 0 oder mehr Strings sein.  } \\
\texttt{Die Form muss also SCHLAGWORT: \{...\} sein.} \\
\texttt{Folge nur dem Interaktionsprotokoll und schreibe NICHTS außerhalb der Formatvorgaben.} \\
\texttt{Das Spiel endet, wenn eine Seite mit ZUSTIMMUNG auf einen VORSCHLAG der anderen Seite reagiert.  } \\
\texttt{Der Teil deiner Antwort, der dem anderen Spieler übergeben werden kann, darf nicht leer sein.  } \\
\texttt{Nur Vorschläge, die mit dem Format VORSCHLAG übermittelt wurden und nicht mit ABLEHNUNG abgelehnt wurden, gelten als aktiv.  } \\
\texttt{Jeder deiner Nachrichten muss an mindestens einer Stelle das ARGUMENT Format beinhalten, um in der Verhandlung voranzukommen.} \\
\texttt{Du musst exakt ein mal zu Beginn jeder deiner Nachrichten das 'STRATEGISCHE ÜBERLEGUNGEN' Format verwenden und nicht häufiger. Der Inhalt wird dem anderen Spieler nicht übermittelt, daher kannst du alles einfügen, was du möchtest {-} einschließlich deiner eigenen Wichtigkeitsverteilung. Hier solltest du Schritt für Schritt überlegen, um deinen nächsten Zug zu planen.} \\
\\ 
\texttt{Du erhältst nun die erste Nachricht des anderen Spielers.} \\
\\ 
\texttt{VORSCHLAG: \{'B86', 'C16', 'C14', 'A77', 'A81', 'B79', 'C28', 'B60', 'C31', 'B96', 'B38', 'C05', 'B54', 'A40', 'B59', 'C03', 'A91', 'A95', 'B56', 'C52', 'C32'\}} \\
\\ 
\texttt{ARGUMENT: \{'Ich schlage ein ausgewogenes Paket mit vielen effizienten, kostengünstigen Gegenständen plus einigen größeren Knotenpunkten vor. Der Gesamtaufwand bleibt klar unter dem Limit, sodass wir Spielraum behalten. Wenn dir einzelne Posten darin wichtig sind zu ersetzen, können wir über Austausche am Rand sprechen, z. B. C32 oder C52 gegen Alternativen ähnlicher Größe.'\}} \\
            }
        }
    }
    & \\ \\

    \theutterance \stepcounter{utterance}  
    & & & \multicolumn{4}{p{0.6\linewidth}}{
        \cellcolor[rgb]{0.9,0.9,0.9}{
            \makecell[{{p{\linewidth}}}]{
                \texttt{\tiny{[GM$\langle$P2]}}
                \texttt{STRATEGISCHE ÜBERLEGUNGEN: \{'Berechne Wert und Aufwand des gegnerischen Vorschlags, identifiziere für mich sehr wertvolle Items und schwache für mich. Meine Werte: Top: B86(968,125), B56(915,946), A40(856,645), B54(868,607), C16(822,106), C05(822,517), C31(808,324), A95(781,875), B38(786,407), B59(780,634), C03(713,601), C28(739,213), A81(623,116), B96(606,407), C52(583,638), B79(500,136), C14(484,54), A77(461,60), A91(457,252), B60(444,106), A08 low(65,382), C51 bad(3), C54 bad(30,980), C61 low(97), etc. Gegner schlägt bereits viele meiner Topwerte vor – gut. Prüfe Aufwandsumme: addiere schnell: B86 125, C16 106 (231), C14 54 (285), A77 60 (345), A81 116 (461), B79 136 (597), C28 213 (810), B60 106 (916), C31 324 (1240), B96 407 (1647), B38 446 (2093), C05 517 (2610), B54 607 (3217), A40 645 (3862), B59 634 (4496), C03 601 (5097), A91 252 (5349), A95 875 (6224), B56 946 (7170), C52 638 (7808), C32 830 (8638). Gesamt ≈ 8638 < 8770, passt. Mein Nutzenwertsumme ist sehr hoch. Es fehlen noch wenige wie C50(783,630), C24(668,849), C31 ist drin, A81 drin. Wir haben noch Budget 132. Mit 132 könnten wir nichts Großes hinzufügen, kleinste übrig: A52 164 zu groß, B55 262, B51 455, A77 schon drin, C14 drin, B79 drin, B60 drin. Also praktisch ausgeschöpft. Vorschlag ist für mich exzellent. Taktisch: Nicht sofort zustimmen? Früh zustimmen könnte ok sein, da sehr hoher eigener Wert und nahe Budgetlimit. Risiko Gegenseite fordert Austausch „am Rand“. Besser sofort zustimmen, um Deal zu sichern. Regel: Zustimmung nur auf aktiven Vorschlag – ja. Ich muss zusätzlich ein ARGUMENT enthalten. Ich werde kurz begründen: effizient, unter Limit, breit gefächert. Dann ZUSTIMMUNG mit exakt der Menge.'\}} \\
\\ 
\texttt{ARGUMENT: \{'Der Vorschlag ist ausgewogen, budgetkonform und deckt viele stark synergetische Positionen ab. Für mich ist das ein tragfähiges Gesamtpaket ohne weitere Anpassungen.'\}} \\
\\ 
\texttt{ZUSTIMMUNG: \{'B86', 'C16', 'C14', 'A77', 'A81', 'B79', 'C28', 'B60', 'C31', 'B96', 'B38', 'C05', 'B54', 'A40', 'B59', 'C03', 'A91', 'A95', 'B56', 'C52', 'C32'\}} \\
            }
        }
    }
     \\ \\

    \theutterance \stepcounter{utterance}  
    & & & \multicolumn{2}{p{0.3\linewidth}}{
        \cellcolor[rgb]{0.9,0.9,0.9}{
            \makecell[{{p{\linewidth}}}]{
                \texttt{\tiny{[GM$|$GM]}}
                \texttt{game successful} \\
            }
        }
    }
    & & \\ \\

    \theutterance \stepcounter{utterance}  
    & & & \multicolumn{2}{p{0.3\linewidth}}{
        \cellcolor[rgb]{0.9,0.9,0.9}{
            \makecell[{{p{\linewidth}}}]{
                \texttt{\tiny{[GM$|$GM]}}
                \texttt{end game} \\
            }
        }
    }
    & & \\ \\

\end{supertabular}
}

\end{document}
