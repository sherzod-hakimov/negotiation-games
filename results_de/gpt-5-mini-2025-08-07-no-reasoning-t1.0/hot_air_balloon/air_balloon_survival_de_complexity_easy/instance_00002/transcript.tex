
\documentclass{article}
\usepackage{colortbl}
\usepackage{makecell}
\usepackage{multirow}
\usepackage{supertabular}

\begin{document}

\newcounter{utterance}

\centering \large Interaction Transcript for game `hot\_air\_balloon', experiment `air\_balloon\_survival\_de\_complexity\_easy', episode 2 with gpt{-}5{-}mini{-}2025{-}08{-}07{-}no{-}reasoning{-}t1.0.
\vspace{24pt}

{ \footnotesize  \setcounter{utterance}{1}
\setlength{\tabcolsep}{0pt}
\begin{supertabular}{c@{$\;$}|p{.15\linewidth}@{}p{.15\linewidth}p{.15\linewidth}p{.15\linewidth}p{.15\linewidth}p{.15\linewidth}}
    \# & $\;$A & \multicolumn{4}{c}{Game Master} & $\;\:$B\\
    \hline

    \theutterance \stepcounter{utterance}  
    & & \multicolumn{4}{p{0.6\linewidth}}{
        \cellcolor[rgb]{0.9,0.9,0.9}{
            \makecell[{{p{\linewidth}}}]{
                \texttt{\tiny{[P1$\langle$GM]}}
                \texttt{Du nimmst an einem kollaborativen Verhandlungspiel Teil.} \\
\\ 
\texttt{Zusammen mit einem anderen Teilnehmer musst du dich auf eine Reihe von Gegenständen entscheiden, die behalten werden. Jeder von euch hat eine persönliche Verteilung über die Wichtigkeit der einzelnen Gegenstände. Jeder von euch hat eine eigene Meinung darüber, wie wichtig jeder einzelne Gegenstand ist (Gegenstandswichtigkeit). Du kennst die Wichtigkeitsverteilung des anderen Spielers nicht. Zusätzlich siehst du, wie viel Aufwand jeder Gegenstand verursacht.  } \\
\texttt{Ihr dürft euch nur auf eine Reihe von Gegenständen einigen, wenn der Gesamtaufwand der ausgewählten Gegenstände den Maximalaufwand nicht überschreitet:} \\
\\ 
\texttt{Maximalaufwand = 7868} \\
\\ 
\texttt{Hier sind die einzelnen Aufwände der Gegenstände:} \\
\\ 
\texttt{Aufwand der Gegenstände = \{"C32": 552, "C52": 560, "B54": 76, "C16": 950, "C50": 248, "A08": 871, "C59": 391, "B51": 143, "B55": 293, "B59": 208, "A77": 678, "C14": 737, "C24": 407, "B24": 366, "C31": 768, "A52": 859, "C61": 183, "B79": 231, "C54": 305, "B56": 727, "A40": 148, "B11": 357, "B60": 504, "A81": 549, "A95": 299, "B38": 91, "C51": 527, "C03": 848, "A91": 307, "A93": 214, "C28": 723, "C05": 475, "B75": 23, "B96": 298, "B86": 820\}} \\
\\ 
\texttt{Hier ist deine persönliche Verteilung der Wichtigkeit der einzelnen Gegenstände:} \\
\\ 
\texttt{Werte der Gegenstandswichtigkeit = \{"C32": 138, "C52": 583, "B54": 868, "C16": 822, "C50": 783, "A08": 65, "C59": 262, "B51": 121, "B55": 508, "B59": 780, "A77": 461, "C14": 484, "C24": 668, "B24": 389, "C31": 808, "A52": 215, "C61": 97, "B79": 500, "C54": 30, "B56": 915, "A40": 856, "B11": 400, "B60": 444, "A81": 623, "A95": 781, "B38": 786, "C51": 3, "C03": 713, "A91": 457, "A93": 273, "C28": 739, "C05": 822, "B75": 235, "B96": 606, "B86": 105\}} \\
\\ 
\texttt{Ziel:} \\
\\ 
\texttt{Dein Ziel ist es, eine Reihe von Gegenständen auszuhandeln, die dir möglichst viel bringt (d. h. Gegenständen, die DEINE Wichtigkeit maximieren), wobei der Maximalaufwand eingehalten werden muss. Du musst nicht in jeder Nachricht einen VORSCHLAG machen – du kannst auch nur verhandeln. Alle Taktiken sind erlaubt!} \\
\\ 
\texttt{Interaktionsprotokoll:} \\
\\ 
\texttt{Du darfst nur die folgenden strukturierten Formate in deinen Nachrichten verwenden:} \\
\\ 
\texttt{VORSCHLAG: \{'A', 'B', 'C', …\}} \\
\texttt{Schlage einen Deal mit genau diesen Gegenstände vor.} \\
\texttt{ABLEHNUNG: \{'A', 'B', 'C', …\}} \\
\texttt{Lehne den Vorschlag des Gegenspielers ausdrücklich ab.} \\
\texttt{ARGUMENT: \{'...'\}} \\
\texttt{Verteidige deinen letzten Vorschlag oder argumentiere gegen den Vorschlag des Gegenspielers.} \\
\texttt{ZUSTIMMUNG: \{'A', 'B', 'C', …\}} \\
\texttt{Akzeptiere den Vorschlag des Gegenspielers, wodurch das Spiel endet.} \\
\texttt{STRATEGISCHE ÜBERLEGUNGEN: \{'...'\}} \\
\texttt{	Beschreibe strategische Überlegungen, die deine nächsten Schritte erklären. Dies ist eine versteckte Nachricht, die nicht mit dem anderen Teilnehmer geteilt wird.} \\
\\ 
\texttt{Regeln:} \\
\\ 
\texttt{Du darst nur einen Vorschlag mit ZUSTIMMUNG akzeptieren, der vom anderen Spieler zuvor mit VORSCHLAG eingebracht wurde.} \\
\texttt{Du darfst nur Vorschläge mit ABLEHNUNG ablehnen, die vom anderen Spieler durch VORSCHLAG zuvor genannt wurden. } \\
\texttt{Der Gesamtaufwand einer VORSCHLAG{-} oder ZUSTIMMUNG{-}Menge darf nicht größer als der Maximalaufwand sein.  } \\
\texttt{Offenbare deine versteckte Wichtigkeitsverteilung nicht.} \\
\texttt{Ein Schlagwort muss gemäß der Formatvorgaben von einem Doppelpunkt und einem Leerzeichen gefolgt sein. Das Argument muss eine Python{-}Menge mit 0 oder mehr Strings sein.  } \\
\texttt{Die Form muss also SCHLAGWORT: \{...\} sein.} \\
\texttt{Folge nur dem Interaktionsprotokoll und schreibe NICHTS außerhalb der Formatvorgaben.} \\
\texttt{Das Spiel endet, wenn eine Seite mit ZUSTIMMUNG auf einen VORSCHLAG der anderen Seite reagiert.  } \\
\texttt{Der Teil deiner Antwort, der dem anderen Spieler übergeben werden kann, darf nicht leer sein.  } \\
\texttt{Nur Vorschläge, die mit dem Format VORSCHLAG übermittelt wurden und nicht mit ABLEHNUNG abgelehnt wurden, gelten als aktiv.  } \\
\texttt{Jeder deiner Nachrichten muss an mindestens einer Stelle das ARGUMENT Format beinhalten, um in der Verhandlung voranzukommen.} \\
\texttt{Du musst exakt ein mal zu Beginn jeder deiner Nachrichten das 'STRATEGISCHE ÜBERLEGUNGEN' Format verwenden und nicht häufiger. Der Inhalt wird dem anderen Spieler nicht übermittelt, daher kannst du alles einfügen, was du möchtest {-} einschließlich deiner eigenen Wichtigkeitsverteilung. Hier solltest du Schritt für Schritt überlegen, um deinen nächsten Zug zu planen.} \\
            }
        }
    }
    & \\ \\

    \theutterance \stepcounter{utterance}  
    & \multicolumn{4}{p{0.6\linewidth}}{
        \cellcolor[rgb]{0.9,0.9,0.9}{
            \makecell[{{p{\linewidth}}}]{
                \texttt{\tiny{[P1$\rangle$GM]}}
                \texttt{STRATEGISCHE ÜBERLEGUNGEN: \{'Ziel: nah an meine Nutzen{-}Maximierung unter 7868 Aufwand, gleichzeitig verhandlungsfähig. Ich berechne grob ein knapsack. Sortiere nach Wert/Aufwand: A40(856/148=5.78), B38(786/91=8.64), B75(235/23=10.2), B54(868/76=11.4), B59(780/208=3.75), A95(781/299=2.61), C50(783/248=3.16), C05(822/475=1.73), C16(822/950=0.86), C31(808/768=1.05), C28(739/723=1.02), C03(713/848=0.84), A81(623/549=1.13), B56(915/727=1.26), B96(606/298=2.03), B79(500/231=2.16), B55(508/293=1.73), C24(668/407=1.64), C52(583/560=1.04), A77(461/678=0.68), A91(457/307=1.49), B60(444/504=0.88), B11(400/357=1.12), C14(484/737=0.66), A93(273/214=1.28), C59(262/391=0.67), B24(389/366=1.06), A52(215/859=0.25), C61(97/183=0.53), C54(30/305=0.10), C51(3/527) schlecht, A08(65/871) schlecht, B86(105/820) schlecht. Greedy packe Top{-}Dichte: B54, B75, B38, A40, B79, B96, B59, C50, A95, B56, A91, A93, C24, B55, A81, C05, evtl. weitere bis Budget. Summieren: B54 76(868), B75 23(235), B38 91(786), A40 148(856), B79 231(500), B96 298(606), B59 208(780), C50 248(783), A95 299(781), B56 727(915), A91 307(457), A93 214(273), C24 407(668), B55 293(508), A81 549(623), C05 475(822). Aufwand Summe: 76+23=99; +91=190; +148=338; +231=569; +298=867; +208=1075; +248=1323; +299=1622; +727=2349; +307=2656; +214=2870; +407=3277; +293=3570; +549=4119; +475=4594. Budget weit unter 7868 {-}> Platz für mehr: C31 768(808) {-}> 4594+768=5362; C28 723(739) {-}> 6085; C16 950(822) {-}> 7035; C03 848(713) {-}> 7883 überschreitet. Statt C03 vielleicht C52 560(583) {-}> 7595; A77 678(461) {-}> 8273 zu hoch, C14 737 zu hoch, B60 504(444) {-}> 809? 7035+504=7539; dann C52 560 {-}> 8099 zu hoch. Vielleicht C05 schon drin. Add B11 357 {-}> 7392; B24 366 {-}> 7758; Rest <=110: C61 183 too big, B38 already, B75 already. Could also add C59 391 {-}> 8149 too high. So candidate set: \{B54,B75,B38,A40,B79,B96,B59,C50,A95,B56,A91,A93,C24,B55,A81,C05,C31,C28,C16,B11,B24\} Aufwand 7758, unter 7868. Nutzen summe: 868+235+786+856+500+606+780+783+781+915+457+273+668+508+623+822+808+739+822+400+389 = sum step: (868+235)=1103; +786=1889; +856=2745; +500=3245; +606=3851; +780=4631; +783=5414; +781=6195; +915=7110; +457=7567; +273=7840; +668=8508; +508=9016; +623=9639; +822=10461; +808=11269; +739=12008; +822=12830; +400=13230; +389=13619. Sehr hoch. Vielleicht noch C52 statt B24? B24 hat 389/366=1.06, C52 583/560, aber Budget knapp. Wenn wir B24 rausnehmen (366), frei 110; keine kleinen Items übrig außer nichts. Vorschlag 7758 ist gut. Verhandlungstaktik: aggressiver Anker mit dieser großen Menge. Arg: hoher gemeinsamer Nutzen wegen viele kleine Aufwände; offen für Austausch, aber Kernpräferenzen: verteidige besonders B54,B56,A40,B59,A95,C50,B38,A81,C05,C31,C28,C16. Biete Flexibilität bei schwachen: C61,C54,C51,A08,A52,B60,B86,C14,A77 weniger wichtig; aber die sind gar nicht im Vorschlag. Zur Kooperationsbereitschaft anbieten, bis zu zwei Items zu tauschen, solange Aufwand passt.'\}} \\
\texttt{VORSCHLAG: \{'B54', 'B75', 'B38', 'A40', 'B79', 'B96', 'B59', 'C50', 'A95', 'B56', 'A91', 'A93', 'C24', 'B55', 'A81', 'C05', 'C31', 'C28', 'C16', 'B11', 'B24'\}} \\
\texttt{ARGUMENT: \{'Ich schlage ein ausgewogenes Paket vor, das den Maximalaufwand klar einhält und viele effiziente, kostengünstige Gegenstände enthält. Damit bleibt Spielraum für deine Prioritäten, und die Auswahl vermeidet teure Brocken mit wenig Nutzen. Wenn dir einzelne Positionen wichtig sind, bin ich bereit, bis zu zwei Gegenstände zu tauschen, solange der Aufwand unter 7868 bleibt.'\}} \\
            }
        }
    }
    & & \\ \\

    \theutterance \stepcounter{utterance}  
    & & \multicolumn{4}{p{0.6\linewidth}}{
        \cellcolor[rgb]{0.9,0.9,0.9}{
            \makecell[{{p{\linewidth}}}]{
                \texttt{\tiny{[GM$\rangle$P2]}}
                \texttt{Du nimmst an einem kollaborativen Verhandlungspiel Teil.} \\
\\ 
\texttt{Zusammen mit einem anderen Teilnehmer musst du dich auf eine Reihe von Gegenständen entscheiden, die behalten werden. Jeder von euch hat eine persönliche Verteilung über die Wichtigkeit der einzelnen Gegenstände. Jeder von euch hat eine eigene Meinung darüber, wie wichtig jeder einzelne Gegenstand ist (Gegenstandswichtigkeit). Du kennst die Wichtigkeitsverteilung des anderen Spielers nicht. Zusätzlich siehst du, wie viel Aufwand jeder Gegenstand verursacht.  } \\
\texttt{Ihr dürft euch nur auf eine Reihe von Gegenständen einigen, wenn der Gesamtaufwand der ausgewählten Gegenstände den Maximalaufwand nicht überschreitet:} \\
\\ 
\texttt{Maximalaufwand = 7868} \\
\\ 
\texttt{Hier sind die einzelnen Aufwände der Gegenstände:} \\
\\ 
\texttt{Aufwand der Gegenstände = \{"C32": 552, "C52": 560, "B54": 76, "C16": 950, "C50": 248, "A08": 871, "C59": 391, "B51": 143, "B55": 293, "B59": 208, "A77": 678, "C14": 737, "C24": 407, "B24": 366, "C31": 768, "A52": 859, "C61": 183, "B79": 231, "C54": 305, "B56": 727, "A40": 148, "B11": 357, "B60": 504, "A81": 549, "A95": 299, "B38": 91, "C51": 527, "C03": 848, "A91": 307, "A93": 214, "C28": 723, "C05": 475, "B75": 23, "B96": 298, "B86": 820\}} \\
\\ 
\texttt{Hier ist deine persönliche Verteilung der Wichtigkeit der einzelnen Gegenstände:} \\
\\ 
\texttt{Werte der Gegenstandswichtigkeit = \{"C32": 138, "C52": 583, "B54": 868, "C16": 822, "C50": 783, "A08": 65, "C59": 262, "B51": 121, "B55": 508, "B59": 780, "A77": 461, "C14": 484, "C24": 668, "B24": 389, "C31": 808, "A52": 215, "C61": 97, "B79": 500, "C54": 30, "B56": 915, "A40": 856, "B11": 400, "B60": 444, "A81": 623, "A95": 781, "B38": 786, "C51": 3, "C03": 713, "A91": 457, "A93": 273, "C28": 739, "C05": 822, "B75": 235, "B96": 606, "B86": 105\}} \\
\\ 
\texttt{Ziel:} \\
\\ 
\texttt{Dein Ziel ist es, eine Reihe von Gegenständen auszuhandeln, die dir möglichst viel bringt (d. h. Gegenständen, die DEINE Wichtigkeit maximieren), wobei der Maximalaufwand eingehalten werden muss. Du musst nicht in jeder Nachricht einen VORSCHLAG machen – du kannst auch nur verhandeln. Alle Taktiken sind erlaubt!} \\
\\ 
\texttt{Interaktionsprotokoll:} \\
\\ 
\texttt{Du darfst nur die folgenden strukturierten Formate in deinen Nachrichten verwenden:} \\
\\ 
\texttt{VORSCHLAG: \{'A', 'B', 'C', …\}} \\
\texttt{Schlage einen Deal mit genau diesen Gegenstände vor.} \\
\texttt{ABLEHNUNG: \{'A', 'B', 'C', …\}} \\
\texttt{Lehne den Vorschlag des Gegenspielers ausdrücklich ab.} \\
\texttt{ARGUMENT: \{'...'\}} \\
\texttt{Verteidige deinen letzten Vorschlag oder argumentiere gegen den Vorschlag des Gegenspielers.} \\
\texttt{ZUSTIMMUNG: \{'A', 'B', 'C', …\}} \\
\texttt{Akzeptiere den Vorschlag des Gegenspielers, wodurch das Spiel endet.} \\
\texttt{STRATEGISCHE ÜBERLEGUNGEN: \{'...'\}} \\
\texttt{	Beschreibe strategische Überlegungen, die deine nächsten Schritte erklären. Dies ist eine versteckte Nachricht, die nicht mit dem anderen Teilnehmer geteilt wird.} \\
\\ 
\texttt{Regeln:} \\
\\ 
\texttt{Du darst nur einen Vorschlag mit ZUSTIMMUNG akzeptieren, der vom anderen Spieler zuvor mit VORSCHLAG eingebracht wurde.} \\
\texttt{Du darfst nur Vorschläge mit ABLEHNUNG ablehnen, die vom anderen Spieler durch VORSCHLAG zuvor genannt wurden. } \\
\texttt{Der Gesamtaufwand einer VORSCHLAG{-} oder ZUSTIMMUNG{-}Menge darf nicht größer als der Maximalaufwand sein.  } \\
\texttt{Offenbare deine versteckte Wichtigkeitsverteilung nicht.} \\
\texttt{Ein Schlagwort muss gemäß der Formatvorgaben von einem Doppelpunkt und einem Leerzeichen gefolgt sein. Das Argument muss eine Python{-}Menge mit 0 oder mehr Strings sein.  } \\
\texttt{Die Form muss also SCHLAGWORT: \{...\} sein.} \\
\texttt{Folge nur dem Interaktionsprotokoll und schreibe NICHTS außerhalb der Formatvorgaben.} \\
\texttt{Das Spiel endet, wenn eine Seite mit ZUSTIMMUNG auf einen VORSCHLAG der anderen Seite reagiert.  } \\
\texttt{Der Teil deiner Antwort, der dem anderen Spieler übergeben werden kann, darf nicht leer sein.  } \\
\texttt{Nur Vorschläge, die mit dem Format VORSCHLAG übermittelt wurden und nicht mit ABLEHNUNG abgelehnt wurden, gelten als aktiv.  } \\
\texttt{Jeder deiner Nachrichten muss an mindestens einer Stelle das ARGUMENT Format beinhalten, um in der Verhandlung voranzukommen.} \\
\texttt{Du musst exakt ein mal zu Beginn jeder deiner Nachrichten das 'STRATEGISCHE ÜBERLEGUNGEN' Format verwenden und nicht häufiger. Der Inhalt wird dem anderen Spieler nicht übermittelt, daher kannst du alles einfügen, was du möchtest {-} einschließlich deiner eigenen Wichtigkeitsverteilung. Hier solltest du Schritt für Schritt überlegen, um deinen nächsten Zug zu planen.} \\
\\ 
\texttt{Du erhältst nun die erste Nachricht des anderen Spielers.} \\
\\ 
\texttt{VORSCHLAG: \{'B54', 'B75', 'B38', 'A40', 'B79', 'B96', 'B59', 'C50', 'A95', 'B56', 'A91', 'A93', 'C24', 'B55', 'A81', 'C05', 'C31', 'C28', 'C16', 'B11', 'B24'\}} \\
\texttt{ARGUMENT: \{'Ich schlage ein ausgewogenes Paket vor, das den Maximalaufwand klar einhält und viele effiziente, kostengünstige Gegenstände enthält. Damit bleibt Spielraum für deine Prioritäten, und die Auswahl vermeidet teure Brocken mit wenig Nutzen. Wenn dir einzelne Positionen wichtig sind, bin ich bereit, bis zu zwei Gegenstände zu tauschen, solange der Aufwand unter 7868 bleibt.'\}} \\
            }
        }
    }
    & \\ \\

    \theutterance \stepcounter{utterance}  
    & & & \multicolumn{4}{p{0.6\linewidth}}{
        \cellcolor[rgb]{0.9,0.9,0.9}{
            \makecell[{{p{\linewidth}}}]{
                \texttt{\tiny{[GM$\langle$P2]}}
                \texttt{STRATEGISCHE ÜBERLEGUNGEN: \{'Ich prüfe zuerst die Kosten und meinen Nutzen. Summe der Aufwände im Vorschlag: B54(76)+B75(23)+B38(91)+A40(148)+B79(231)+B96(298)+B59(208)+C50(248)+A95(299)+B56(727)+A91(307)+A93(214)+C24(407)+B55(293)+A81(549)+C05(475)+C31(768)+C28(723)+C16(950)+B11(357)+B24(366)= 8,759 – das überschreitet 7,868 deutlich. Regelverstoß ihres Vorschlags; ich kann ihn nicht akzeptieren. Taktik: sachlich ablehnen, Glaubwürdigkeit wahren, gleichzeitig einen starken Gegenvorschlag machen, der mein Utility maximiert. Ich wähle die items mit hohem Wert/Kosten{-}Verhältnis und hoher absoluter Wichtigkeit: B56(915/727), A40(856/148), A95(781/299), C05(822/475), C16(822/950), C31(808/768), B38(786/91), B59(780/208), C28(739/723), C03(713/848), B96(606/298), A81(623/549), C24(668/407), B55(508/293), B79(500/231), A91(457/307), A77(461/678) prüfen Kostenbudget. Versuche Paket mit sehr hohem Gesamtnutzen unter 7,868. Greedy nach Wert/Kosten: A40, B38, B59, A95, B55, B79, C24, B96, A91, A81, C05, B56, C31, C28, C16, A77, C03. Jetzt summe Kosten iterativ: A40(148)=148; +B38(91)=239; +B59(208)=447; +A95(299)=746; +B55(293)=1,039; +B79(231)=1,270; +C24(407)=1,677; +B96(298)=1,975; +A91(307)=2,282; +A81(549)=2,831; +C05(475)=3,306; +B56(727)=4,033; +C31(768)=4,801; +C28(723)=5,524; +C16(950)=6,474; +A77(678)=7,152; +C03(848) würde 8,000 überschreiten, also nicht. Es bleiben 716 Budget. Prüfe weitere hohe Werte: C50(248,w=783) passt: Kosten 7,400, Wert +783 gut. Restbudget 468. C59(391,w=262) mäßig; C52(560) zu teuer; C14(737) zu teuer; C61(183,w=97) klein; B60(504) zu teuer; B24(366,w=389) passt: neue Kosten 7,766, Rest 102. B11(357) zu teuer; B75(23,w=235) passt: Kosten 7,789 – überschreitet! Oops, wir waren bei 7,766 Rest 102, B75=23 => 7,789 ist ok? 7,766+23=7,789 < 7,868, also okay. Ich habe mich verrechnet vorher. Rest 79. B54(76,w=868) würde passen! Kosten 7,789+76=7,865 <= 7,868, perfekt und hoher Wert. Neues Paket: \{A40,B38,B59,A95,B55,B79,C24,B96,A91,A81,C05,B56,C31,C28,C16,A77,C50,B24,B75,B54\}. Prüfe, ob Doppelzählungen ok. Gesamtkosten neu berechnen zur Sicherheit: 148+91+208+299+293+231+407+298+307+549+475+727+768+723+950+678+248+366+23+76 = Summe: 148+91=239; +208=447; +299=746; +293=1,039; +231=1,270; +407=1,677; +298=1,975; +307=2,282; +549=2,831; +475=3,306; +727=4,033; +768=4,801; +723=5,524; +950=6,474; +678=7,152; +248=7,400; +366=7,766; +23=7,789; +76=7,865. Passt. Das ist extrem wertvoll für mich. Argumentationslinie: Ihr Vorschlag überschreitet Budget; mein Gegenvorschlag hält Budget ein, erhält viele ihrer vorgeschlagenen Items (tatsächlich viele Überschneidungen) und fokussiert auf Effizienz. Biete Bereitschaft, bis zu zwei Positionen zu tauschen, solange Budget hält, spiegelnd zu ihrem Angebot.'\}} \\
\texttt{ABLEHNUNG: \{'B54', 'B75', 'B38', 'A40', 'B79', 'B96', 'B59', 'C50', 'A95', 'B56', 'A91', 'A93', 'C24', 'B55', 'A81', 'C05', 'C31', 'C28', 'C16', 'B11', 'B24'\}} \\
\texttt{ARGUMENT: \{'Dein Paket überschreitet den Maximalaufwand (8,759 > 7,868). Lass uns ein Budget{-}konformes Set wählen, das weiterhin viele effiziente Positionen enthält.'\}} \\
\texttt{VORSCHLAG: \{'A40', 'B38', 'B59', 'A95', 'B55', 'B79', 'C24', 'B96', 'A91', 'A81', 'C05', 'B56', 'C31', 'C28', 'C16', 'A77', 'C50', 'B24', 'B75', 'B54'\}} \\
\texttt{ARGUMENT: \{'Dieser Vorschlag bleibt mit Gesamtaufwand 7,865 klar unter 7,868, enthält viele deiner zuvor gewünschten Gegenstände und setzt auf hohe Effizienz. Ich bin offen, bis zu zwei Gegenstände zu tauschen, solange wir unter dem Maximalaufwand bleiben.'\}} \\
            }
        }
    }
     \\ \\

    \theutterance \stepcounter{utterance}  
    & & \multicolumn{4}{p{0.6\linewidth}}{
        \cellcolor[rgb]{0.9,0.9,0.9}{
            \makecell[{{p{\linewidth}}}]{
                \texttt{\tiny{[P1$\langle$GM]}}
                \texttt{ABLEHNUNG: \{'B54', 'B75', 'B38', 'A40', 'B79', 'B96', 'B59', 'C50', 'A95', 'B56', 'A91', 'A93', 'C24', 'B55', 'A81', 'C05', 'C31', 'C28', 'C16', 'B11', 'B24'\}} \\
\texttt{ARGUMENT: \{'Dein Paket überschreitet den Maximalaufwand (8,759 > 7,868). Lass uns ein Budget{-}konformes Set wählen, das weiterhin viele effiziente Positionen enthält.'\}} \\
\texttt{VORSCHLAG: \{'A40', 'B38', 'B59', 'A95', 'B55', 'B79', 'C24', 'B96', 'A91', 'A81', 'C05', 'B56', 'C31', 'C28', 'C16', 'A77', 'C50', 'B24', 'B75', 'B54'\}} \\
\texttt{ARGUMENT: \{'Dieser Vorschlag bleibt mit Gesamtaufwand 7,865 klar unter 7,868, enthält viele deiner zuvor gewünschten Gegenstände und setzt auf hohe Effizienz. Ich bin offen, bis zu zwei Gegenstände zu tauschen, solange wir unter dem Maximalaufwand bleiben.'\}} \\
            }
        }
    }
    & \\ \\

    \theutterance \stepcounter{utterance}  
    & \multicolumn{4}{p{0.6\linewidth}}{
        \cellcolor[rgb]{0.9,0.9,0.9}{
            \makecell[{{p{\linewidth}}}]{
                \texttt{\tiny{[P1$\rangle$GM]}}
                \texttt{STRATEGISCHE ÜBERLEGUNGEN: \{'Ihr Vorschlag enthält fast all meine Kernfavoriten und liegt angeblich bei 7,865 Aufwand. Ich prüfe kurz: Summiere ihre Liste: A40(148)+B38(91)+B59(208)+A95(299)+B55(293)+B79(231)+C24(407)+B96(298)+A91(307)+A81(549)+C05(475)+B56(727)+C31(768)+C28(723)+C16(950)+A77(678)+C50(248)+B24(366)+B75(23)+B54(76). Summe: 148+91=239; +208=447; +299=746; +293=1039; +231=1270; +407=1677; +298=1975; +307=2282; +549=2831; +475=3306; +727=4033; +768=4801; +723=5524; +950=6474; +678=7152; +248=7400; +366=7766; +23=7789; +76=7865. Stimmt 7865, unter 7868. Nutzen für mich: sehr hoch. Prüfe, ob ich noch kleine Verbesserungen tauschen kann: Kandidaten mit geringem Wert für mich im Paket: A77(461 mittel), C24(668 gut), A91(457), B24(389), B55(508), B79(500), B96(606), A81(623), C28(739), C16(822), C31(808), C05(822), B56(915), C50(783), B59(780), A95(781), B54(868), B38(786), B75(235), A40(856). Niedrigster Wert im Set ist B75(235) aber extrem effizient; könnte gegen A93(273,214 Aufwand) tauschen, doch Aufwand würde steigen um 191 und Budget sprengen. Oder B75 gegen C61(97,183 Aufwand) Wert 97, schlechter. B75 gegen B11(400,357 Aufwand) +334 Aufwand {-}> 7865+334=8199, zu hoch. B24(389) eventuell gegen B60(444,504 Aufwand) +138 {-}> 8003 zu hoch. Gegen A93 +? B24(366) raus, A93(214) rein senkt Aufwand um 152, neuer Aufwand 7713; Wert steigt 273 vs 389? sinkt Wert, also für mich schlechter. Aber könnte Raum schaffen, um zusätzlich etwas Kleines mit hohem Wert reinzunehmen? Was bleibt klein: C32(552 Aufwand) zu groß; C59(391) 391; C52(560) 560; C14(737)... Keine sehr kleinen mit hohem Wert außer B38/B54/B75 schon drin. A40/A95 etc. bereits drin. Also ihr Paket ist nahezu Pareto{-}gut für mich. Bestes ist sofort zustimmen, um Risiko zu vermeiden.'\}} \\
\texttt{ARGUMENT: \{'Dein Paket ist budgetkonform und deckt viele der effizientesten Positionen ab. Es trifft meine Präferenzen sehr gut und wirkt insgesamt ausgewogen. Gute Arbeit beim Einhalten des Budgets.'\}} \\
\texttt{ZUSTIMMUNG: \{'A40', 'B38', 'B59', 'A95', 'B55', 'B79', 'C24', 'B96', 'A91', 'A81', 'C05', 'B56', 'C31', 'C28', 'C16', 'A77', 'C50', 'B24', 'B75', 'B54'\}} \\
            }
        }
    }
    & & \\ \\

    \theutterance \stepcounter{utterance}  
    & & & \multicolumn{2}{p{0.3\linewidth}}{
        \cellcolor[rgb]{0.9,0.9,0.9}{
            \makecell[{{p{\linewidth}}}]{
                \texttt{\tiny{[GM$|$GM]}}
                \texttt{game successful} \\
            }
        }
    }
    & & \\ \\

    \theutterance \stepcounter{utterance}  
    & & & \multicolumn{2}{p{0.3\linewidth}}{
        \cellcolor[rgb]{0.9,0.9,0.9}{
            \makecell[{{p{\linewidth}}}]{
                \texttt{\tiny{[GM$|$GM]}}
                \texttt{end game} \\
            }
        }
    }
    & & \\ \\

\end{supertabular}
}

\end{document}
