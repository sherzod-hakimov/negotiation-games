
\documentclass{article}
\usepackage{colortbl}
\usepackage{makecell}
\usepackage{multirow}
\usepackage{supertabular}

\begin{document}

\newcounter{utterance}

\twocolumn

{ \footnotesize  \setcounter{utterance}{1}
\setlength{\tabcolsep}{0pt}
\begin{supertabular}{c@{$\;$}|p{.15\linewidth}@{}p{.15\linewidth}p{.15\linewidth}p{.15\linewidth}p{.15\linewidth}p{.15\linewidth}}

    \# & $\;$A & \multicolumn{4}{c}{Game Master} & $\;\:$B\\
    \hline 

    \theutterance \stepcounter{utterance}  

    & & \multicolumn{4}{p{0.6\linewidth}}{\cellcolor[rgb]{0.9,0.9,0.9}{%
	\makecell[{{p{\linewidth}}}]{% 
	  \tt {\tiny [A$\langle$GM]}  
	 Du nimmst an einem kollaborativen Verhandlungspiel Teil.\\ \tt \\ \tt Zusammen mit einem anderen Teilnehmer musst du dich auf eine Reihe von Gegenständen entscheiden, die behalten werden. Jeder von euch hat eine persönliche Verteilung über die Wichtigkeit der einzelnen Gegenstände. Jeder von euch hat eine eigene Meinung darüber, wie wichtig jeder einzelne Gegenstand ist (Gegenstandswichtigkeit). Du kennst die Wichtigkeitsverteilung des anderen Spielers nicht. Zusätzlich siehst du, wie viel Aufwand jeder Gegenstand verursacht.  \\ \tt Ihr dürft euch nur auf eine Reihe von Gegenständen einigen, wenn der Gesamtaufwand der ausgewählten Gegenstände den Maximalaufwand nicht überschreitet:\\ \tt \\ \tt Maximalaufwand = 3218\\ \tt \\ \tt Hier sind die einzelnen Aufwände der Gegenstände:\\ \tt \\ \tt Aufwand der Gegenstände = {"C41": 919, "C15": 852, "C28": 299, "C48": 762, "B78": 161, "A75": 205, "A60": 380, "B05": 399, "A56": 534, "B66": 333, "C60": 100, "B49": 420, "A35": 354, "A93": 130, "B10": 589}\\ \tt \\ \tt Hier ist deine persönliche Verteilung der Wichtigkeit der einzelnen Gegenstände:\\ \tt \\ \tt Werte der Gegenstandswichtigkeit = {"C41": 138, "C15": 583, "C28": 868, "C48": 822, "B78": 783, "A75": 65, "A60": 262, "B05": 121, "A56": 508, "B66": 780, "C60": 461, "B49": 484, "A35": 668, "A93": 389, "B10": 808}\\ \tt \\ \tt Ziel:\\ \tt \\ \tt Dein Ziel ist es, eine Reihe von Gegenständen auszuhandeln, die dir möglichst viel bringt (d. h. Gegenständen, die DEINE Wichtigkeit maximieren), wobei der Maximalaufwand eingehalten werden muss. Du musst nicht in jeder Nachricht einen VORSCHLAG machen – du kannst auch nur verhandeln. Alle Taktiken sind erlaubt!\\ \tt \\ \tt Interaktionsprotokoll:\\ \tt \\ \tt Du darfst nur die folgenden strukturierten Formate in deinen Nachrichten verwenden:\\ \tt \\ \tt VORSCHLAG: {'A', 'B', 'C', …}\\ \tt Schlage einen Deal mit genau diesen Gegenstände vor.\\ \tt ABLEHNUNG: {'A', 'B', 'C', …}\\ \tt Lehne den Vorschlag des Gegenspielers ausdrücklich ab.\\ \tt ARGUMENT: {'...'}\\ \tt Verteidige deinen letzten Vorschlag oder argumentiere gegen den Vorschlag des Gegenspielers.\\ \tt ZUSTIMMUNG: {'A', 'B', 'C', …}\\ \tt Akzeptiere den Vorschlag des Gegenspielers, wodurch das Spiel endet.\\ \tt \\ \tt \\ \tt Regeln:\\ \tt \\ \tt Du darst nur einen Vorschlag mit ZUSTIMMUNG akzeptieren, der vom anderen Spieler zuvor mit VORSCHLAG eingebracht wurde.\\ \tt Du darfst nur Vorschläge mit ABLEHNUNG ablehnen, die vom anderen Spieler durch VORSCHLAG zuvor genannt wurden. \\ \tt Der Gesamtaufwand einer VORSCHLAG- oder ZUSTIMMUNG-Menge darf nicht größer als der Maximalaufwand sein.  \\ \tt Offenbare deine versteckte Wichtigkeitsverteilung nicht.\\ \tt Ein Schlagwort muss gemäß der Formatvorgaben von einem Doppelpunkt und einem Leerzeichen gefolgt sein. Das Argument muss eine Python-Menge mit 0 oder mehr Strings sein.  \\ \tt Die Form muss also SCHLAGWORT: {...} sein.\\ \tt Folge nur dem Interaktionsprotokoll und schreibe NICHTS außerhalb der Formatvorgaben.\\ \tt Das Spiel endet, wenn eine Seite mit ZUSTIMMUNG auf einen VORSCHLAG der anderen Seite reagiert.  \\ \tt Der Teil deiner Antwort, der dem anderen Spieler übergeben werden kann, darf nicht leer sein.  \\ \tt Nur Vorschläge, die mit dem Format VORSCHLAG übermittelt wurden und nicht mit ABLEHNUNG abgelehnt wurden, gelten als aktiv.  \\ \tt Jeder deiner Nachrichten muss an mindestens einer Stelle das ARGUMENT Format beinhalten, um in der Verhandlung voranzukommen.\\ \tt  
	  } 
	   } 
	   } 
	 & \\ 
 

    \theutterance \stepcounter{utterance}  

    & \multicolumn{4}{p{0.6\linewidth}}{\cellcolor[rgb]{0.8,1,0.9}{%
	\makecell[{{p{\linewidth}}}]{% 
	  \tt {\tiny [A$\rangle$GM]}  
	 VORSCHLAG: {'C28', 'C48', 'B10', 'B66', 'C60', 'B78', 'A93'}\\ \tt ARGUMENT: {'Dieser Vorschlag bleibt mit Aufwand 299+762+589+333+100+161+130=2374 klar unter dem Limit 3218 und deckt aus meiner Sicht mehrere besonders attraktive Items ab. Wenn du andere Präferenzen hast, sag mir bitte, welche 1–2 Items dir wichtig sind, dann kann ich umbauen und ggf. ein bis zwei Teile tauschen, solange wir unter 3218 bleiben.'} 
	  } 
	   } 
	   } 
	 & & \\ 
 

    \theutterance \stepcounter{utterance}  

    & & \multicolumn{4}{p{0.6\linewidth}}{\cellcolor[rgb]{0.9,0.9,0.9}{%
	\makecell[{{p{\linewidth}}}]{% 
	  \tt {\tiny [GM$\rangle$B]}  
	 Du nimmst an einem kollaborativen Verhandlungspiel Teil.\\ \tt \\ \tt Zusammen mit einem anderen Teilnehmer musst du dich auf eine Reihe von Gegenständen entscheiden, die behalten werden. Jeder von euch hat eine persönliche Verteilung über die Wichtigkeit der einzelnen Gegenstände. Jeder von euch hat eine eigene Meinung darüber, wie wichtig jeder einzelne Gegenstand ist (Gegenstandswichtigkeit). Du kennst die Wichtigkeitsverteilung des anderen Spielers nicht. Zusätzlich siehst du, wie viel Aufwand jeder Gegenstand verursacht.  \\ \tt Ihr dürft euch nur auf eine Reihe von Gegenständen einigen, wenn der Gesamtaufwand der ausgewählten Gegenstände den Maximalaufwand nicht überschreitet:\\ \tt \\ \tt Maximalaufwand = 3218\\ \tt \\ \tt Hier sind die einzelnen Aufwände der Gegenstände:\\ \tt \\ \tt Aufwand der Gegenstände = {"C41": 919, "C15": 852, "C28": 299, "C48": 762, "B78": 161, "A75": 205, "A60": 380, "B05": 399, "A56": 534, "B66": 333, "C60": 100, "B49": 420, "A35": 354, "A93": 130, "B10": 589}\\ \tt \\ \tt Hier ist deine persönliche Verteilung der Wichtigkeit der einzelnen Gegenstände:\\ \tt \\ \tt Werte der Gegenstandswichtigkeit = {"C41": 138, "C15": 583, "C28": 868, "C48": 822, "B78": 783, "A75": 65, "A60": 262, "B05": 121, "A56": 508, "B66": 780, "C60": 461, "B49": 484, "A35": 668, "A93": 389, "B10": 808}\\ \tt \\ \tt Ziel:\\ \tt \\ \tt Dein Ziel ist es, eine Reihe von Gegenständen auszuhandeln, die dir möglichst viel bringt (d. h. Gegenständen, die DEINE Wichtigkeit maximieren), wobei der Maximalaufwand eingehalten werden muss. Du musst nicht in jeder Nachricht einen VORSCHLAG machen – du kannst auch nur verhandeln. Alle Taktiken sind erlaubt!\\ \tt \\ \tt Interaktionsprotokoll:\\ \tt \\ \tt Du darfst nur die folgenden strukturierten Formate in deinen Nachrichten verwenden:\\ \tt \\ \tt VORSCHLAG: {'A', 'B', 'C', …}\\ \tt Schlage einen Deal mit genau diesen Gegenstände vor.\\ \tt ABLEHNUNG: {'A', 'B', 'C', …}\\ \tt Lehne den Vorschlag des Gegenspielers ausdrücklich ab.\\ \tt ARGUMENT: {'...'}\\ \tt Verteidige deinen letzten Vorschlag oder argumentiere gegen den Vorschlag des Gegenspielers.\\ \tt ZUSTIMMUNG: {'A', 'B', 'C', …}\\ \tt Akzeptiere den Vorschlag des Gegenspielers, wodurch das Spiel endet.\\ \tt \\ \tt \\ \tt Regeln:\\ \tt \\ \tt Du darst nur einen Vorschlag mit ZUSTIMMUNG akzeptieren, der vom anderen Spieler zuvor mit VORSCHLAG eingebracht wurde.\\ \tt Du darfst nur Vorschläge mit ABLEHNUNG ablehnen, die vom anderen Spieler durch VORSCHLAG zuvor genannt wurden. \\ \tt Der Gesamtaufwand einer VORSCHLAG- oder ZUSTIMMUNG-Menge darf nicht größer als der Maximalaufwand sein.  \\ \tt Offenbare deine versteckte Wichtigkeitsverteilung nicht.\\ \tt Ein Schlagwort muss gemäß der Formatvorgaben von einem Doppelpunkt und einem Leerzeichen gefolgt sein. Das Argument muss eine Python-Menge mit 0 oder mehr Strings sein.  \\ \tt Die Form muss also SCHLAGWORT: {...} sein.\\ \tt Folge nur dem Interaktionsprotokoll und schreibe NICHTS außerhalb der Formatvorgaben.\\ \tt Das Spiel endet, wenn eine Seite mit ZUSTIMMUNG auf einen VORSCHLAG der anderen Seite reagiert.  \\ \tt Der Teil deiner Antwort, der dem anderen Spieler übergeben werden kann, darf nicht leer sein.  \\ \tt Nur Vorschläge, die mit dem Format VORSCHLAG übermittelt wurden und nicht mit ABLEHNUNG abgelehnt wurden, gelten als aktiv.  \\ \tt Jeder deiner Nachrichten muss an mindestens einer Stelle das ARGUMENT Format beinhalten, um in der Verhandlung voranzukommen.\\ \tt \\ \tt \\ \tt Du erhältst nun die erste Nachricht des anderen Spielers.\\ \tt \\ \tt VORSCHLAG: {'C28', 'C48', 'B10', 'B66', 'C60', 'B78', 'A93'}\\ \tt ARGUMENT: {'Dieser Vorschlag bleibt mit Aufwand 299+762+589+333+100+161+130=2374 klar unter dem Limit 3218 und deckt aus meiner Sicht mehrere besonders attraktive Items ab. Wenn du andere Präferenzen hast, sag mir bitte, welche 1–2 Items dir wichtig sind, dann kann ich umbauen und ggf. ein bis zwei Teile tauschen, solange wir unter 3218 bleiben.'} 
	  } 
	   } 
	   } 
	 & \\ 
 

    \theutterance \stepcounter{utterance}  

    & & & \multicolumn{4}{p{0.6\linewidth}}{\cellcolor[rgb]{1,0.85,0.72}{%
	\makecell[{{p{\linewidth}}}]{% 
	  \tt {\tiny [GM$\langle$B]}  
	 ABLEHNUNG: {'C28', 'C48', 'B10', 'B66', 'C60', 'B78', 'A93'}\\ \tt ARGUMENT: {'Dein Vorschlag ist solide unter dem Limit, aber aus meiner Sicht fehlen 1–2 sehr wertvolle Items. Ich würde gerne zwei Positionen tauschen, um den Gesamtwert für mich deutlich zu erhöhen, ohne das Limit zu reißen.'}\\ \tt VORSCHLAG: {'C28', 'C48', 'B10', 'B66', 'C60', 'B78', 'A35'}\\ \tt ARGUMENT: {'Dieser Tausch ersetzt A93 durch A35. Aufwand bleibt 2374-130+354=2598 (weiter unter 3218). So bleibt der Großteil deines Pakets erhalten, aber A35 ist für mich deutlich attraktiver als A93. Falls du einen weiteren Tausch brauchst, können wir A60 statt B78 erwägen und bleiben immer noch sicher unter dem Limit.'} 
	  } 
	   } 
	   } 
	  \\ 
 

    \theutterance \stepcounter{utterance}  

    & & \multicolumn{4}{p{0.6\linewidth}}{\cellcolor[rgb]{0.9,0.9,0.9}{%
	\makecell[{{p{\linewidth}}}]{% 
	  \tt {\tiny [A$\langle$GM]}  
	 ABLEHNUNG: {'C28', 'C48', 'B10', 'B66', 'C60', 'B78', 'A93'}\\ \tt ARGUMENT: {'Dein Vorschlag ist solide unter dem Limit, aber aus meiner Sicht fehlen 1–2 sehr wertvolle Items. Ich würde gerne zwei Positionen tauschen, um den Gesamtwert für mich deutlich zu erhöhen, ohne das Limit zu reißen.'}\\ \tt VORSCHLAG: {'C28', 'C48', 'B10', 'B66', 'C60', 'B78', 'A35'}\\ \tt ARGUMENT: {'Dieser Tausch ersetzt A93 durch A35. Aufwand bleibt 2374-130+354=2598 (weiter unter 3218). So bleibt der Großteil deines Pakets erhalten, aber A35 ist für mich deutlich attraktiver als A93. Falls du einen weiteren Tausch brauchst, können wir A60 statt B78 erwägen und bleiben immer noch sicher unter dem Limit.'} 
	  } 
	   } 
	   } 
	 & \\ 
 

    \theutterance \stepcounter{utterance}  

    & \multicolumn{4}{p{0.6\linewidth}}{\cellcolor[rgb]{0.8,1,0.9}{%
	\makecell[{{p{\linewidth}}}]{% 
	  \tt {\tiny [A$\rangle$GM]}  
	 ABLEHNUNG: {'C28', 'C48', 'B10', 'B66', 'C60', 'B78', 'A35'}\\ \tt ARGUMENT: {'A35 ist für mich weniger attraktiv als einige Alternativen. Ich kann dir aber entgegenkommen, wenn wir statt A35 ein anderes Teil aufnehmen, das dir wichtig ist und mich nicht zu sehr schmerzt.'}\\ \tt VORSCHLAG: {'C28', 'C48', 'B10', 'B66', 'C60', 'B78', 'A35', 'C41'}\\ \tt ARGUMENT: {'Ich erweitere um C41. Aufwand: 299+762+589+333+100+161+354+919=3517 – das überschreitet 3218, also geht nicht. Lass uns stattdessen A93 zusätzlich reinnehmen: 2598+130=2728 bleibt gültig und bringt uns näher an eine Einigung.'}\\ \tt VORSCHLAG: {'C28', 'C48', 'B10', 'B66', 'C60', 'B78', 'A35', 'A93'}\\ \tt ARGUMENT: {'Aufwand 2728 unter 3218. Ich halte an den für mich besonders wertvollen C28, C48, B10, B66 fest und komme dir mit A35 entgegen. A93 fügt geringen Aufwand hinzu und erleichtert mir die Zustimmung. Wenn du ein anderes kleines Item bevorzugst, nenne es, dann prüfe ich den Tausch.'} 
	  } 
	   } 
	   } 
	 & & \\ 
 

    \theutterance \stepcounter{utterance}  

    & & \multicolumn{4}{p{0.6\linewidth}}{\cellcolor[rgb]{0.9,0.9,0.9}{%
	\makecell[{{p{\linewidth}}}]{% 
	  \tt {\tiny [GM$\rangle$B]}  
	 ABLEHNUNG: {'C28', 'C48', 'B10', 'B66', 'C60', 'B78', 'A35'}\\ \tt ARGUMENT: {'A35 ist für mich weniger attraktiv als einige Alternativen. Ich kann dir aber entgegenkommen, wenn wir statt A35 ein anderes Teil aufnehmen, das dir wichtig ist und mich nicht zu sehr schmerzt.'}\\ \tt VORSCHLAG: {'C28', 'C48', 'B10', 'B66', 'C60', 'B78', 'A35', 'C41'}\\ \tt ARGUMENT: {'Ich erweitere um C41. Aufwand: 299+762+589+333+100+161+354+919=3517 – das überschreitet 3218, also geht nicht. Lass uns stattdessen A93 zusätzlich reinnehmen: 2598+130=2728 bleibt gültig und bringt uns näher an eine Einigung.'}\\ \tt VORSCHLAG: {'C28', 'C48', 'B10', 'B66', 'C60', 'B78', 'A35', 'A93'}\\ \tt ARGUMENT: {'Aufwand 2728 unter 3218. Ich halte an den für mich besonders wertvollen C28, C48, B10, B66 fest und komme dir mit A35 entgegen. A93 fügt geringen Aufwand hinzu und erleichtert mir die Zustimmung. Wenn du ein anderes kleines Item bevorzugst, nenne es, dann prüfe ich den Tausch.'} 
	  } 
	   } 
	   } 
	 & \\ 
 

    \theutterance \stepcounter{utterance}  

    & & & \multicolumn{4}{p{0.6\linewidth}}{\cellcolor[rgb]{1,0.85,0.72}{%
	\makecell[{{p{\linewidth}}}]{% 
	  \tt {\tiny [GM$\langle$B]}  
	 ABLEHNUNG: {'C28', 'C48', 'B10', 'B66', 'C60', 'B78', 'A35', 'A93'}\\ \tt ARGUMENT: {'Das Paket ist tragfähig, aber A35 ist für mich deutlich unattraktiver als einige Alternativen mit ähnlichem Aufwand. Ich kann dir weit entgegenkommen, wenn wir A35 austauschen und den Rest unverändert lassen.'}\\ \tt VORSCHLAG: {'C28', 'C48', 'B10', 'B66', 'C60', 'B78', 'A56', 'A93'}\\ \tt ARGUMENT: {'Tausche A35 (354) gegen A56 (534). Neuer Aufwand: 2728-354+534=2908 < 3218. Für mich ist A56 wesentlich wertvoller, du behältst alle deine Kernstücke (C28, C48, B10, B66). Wenn dir der Mehraufwand von 180 zu hoch ist, biete ich Alternativen: (a) A35 -> A60 (Aufwand 2934) oder (b) A35 -> B49 (2994). Welche Variante passt dir besser?'} 
	  } 
	   } 
	   } 
	  \\ 
 

    \theutterance \stepcounter{utterance}  

    & & \multicolumn{4}{p{0.6\linewidth}}{\cellcolor[rgb]{0.9,0.9,0.9}{%
	\makecell[{{p{\linewidth}}}]{% 
	  \tt {\tiny [A$\langle$GM]}  
	 ABLEHNUNG: {'C28', 'C48', 'B10', 'B66', 'C60', 'B78', 'A35', 'A93'}\\ \tt ARGUMENT: {'Das Paket ist tragfähig, aber A35 ist für mich deutlich unattraktiver als einige Alternativen mit ähnlichem Aufwand. Ich kann dir weit entgegenkommen, wenn wir A35 austauschen und den Rest unverändert lassen.'}\\ \tt VORSCHLAG: {'C28', 'C48', 'B10', 'B66', 'C60', 'B78', 'A56', 'A93'}\\ \tt ARGUMENT: {'Tausche A35 (354) gegen A56 (534). Neuer Aufwand: 2728-354+534=2908 < 3218. Für mich ist A56 wesentlich wertvoller, du behältst alle deine Kernstücke (C28, C48, B10, B66). Wenn dir der Mehraufwand von 180 zu hoch ist, biete ich Alternativen: (a) A35 -> A60 (Aufwand 2934) oder (b) A35 -> B49 (2994). Welche Variante passt dir besser?'} 
	  } 
	   } 
	   } 
	 & \\ 
 

    \theutterance \stepcounter{utterance}  

    & \multicolumn{4}{p{0.6\linewidth}}{\cellcolor[rgb]{0.8,1,0.9}{%
	\makecell[{{p{\linewidth}}}]{% 
	  \tt {\tiny [A$\rangle$GM]}  
	 ABLEHNUNG: {'C28', 'C48', 'B10', 'B66', 'C60', 'B78', 'A56', 'A93'}\\ \tt ARGUMENT: {'Der Tausch auf A56 erhöht meinen Schmerz deutlich. Ich kann einen Mittelweg anbieten, der deinen Wunsch nach Aufwertung respektiert, aber für mich akzeptabler ist.'}\\ \tt VORSCHLAG: {'C28', 'C48', 'B10', 'B66', 'C60', 'B78', 'A60', 'A93'}\\ \tt ARGUMENT: {'Aufwand: 299+762+589+333+100+161+380+130=2754 < 3218. Ich halte an meinen Kernstücken fest und komme dir mit A60 entgegen statt A35. Wenn dir A60 nicht reicht, können wir B49 statt A60 prüfen: Aufwand 2994 und für mich etwas besser ausbalanciert.'} 
	  } 
	   } 
	   } 
	 & & \\ 
 

    \theutterance \stepcounter{utterance}  

    & & \multicolumn{4}{p{0.6\linewidth}}{\cellcolor[rgb]{0.9,0.9,0.9}{%
	\makecell[{{p{\linewidth}}}]{% 
	  \tt {\tiny [GM$\rangle$B]}  
	 ABLEHNUNG: {'C28', 'C48', 'B10', 'B66', 'C60', 'B78', 'A56', 'A93'}\\ \tt ARGUMENT: {'Der Tausch auf A56 erhöht meinen Schmerz deutlich. Ich kann einen Mittelweg anbieten, der deinen Wunsch nach Aufwertung respektiert, aber für mich akzeptabler ist.'}\\ \tt VORSCHLAG: {'C28', 'C48', 'B10', 'B66', 'C60', 'B78', 'A60', 'A93'}\\ \tt ARGUMENT: {'Aufwand: 299+762+589+333+100+161+380+130=2754 < 3218. Ich halte an meinen Kernstücken fest und komme dir mit A60 entgegen statt A35. Wenn dir A60 nicht reicht, können wir B49 statt A60 prüfen: Aufwand 2994 und für mich etwas besser ausbalanciert.'} 
	  } 
	   } 
	   } 
	 & \\ 
 

    \theutterance \stepcounter{utterance}  

    & & & \multicolumn{4}{p{0.6\linewidth}}{\cellcolor[rgb]{1,0.85,0.72}{%
	\makecell[{{p{\linewidth}}}]{% 
	  \tt {\tiny [GM$\langle$B]}  
	 ARGUMENT: {'Danke für den Mittelweg. Ich kann mit deinem Kernpaket leben, aber A60 bringt mir deutlich weniger als zwei Alternativen mit sehr ähnlichem Aufwand. Lass uns minimal umbauen, ohne deine Kernstücke anzutasten.'}\\ \tt VORSCHLAG: {'C28', 'C48', 'B10', 'B66', 'C60', 'B78', 'B49', 'A93'}\\ \tt ARGUMENT: {'Tausche A60 (380) gegen B49 (420). Neuer Aufwand: 2754 - 380 + 420 = 2794 < 3218. Du behältst alle deine Kernitems (C28, C48, B10, B66), ich erhalte eine spürbare Aufwertung. Falls B49 für dich ungünstig ist, biete Alternative: A60 -> A56 (Aufwand 2908). Welche Variante ist für dich akzeptabler?'} 
	  } 
	   } 
	   } 
	  \\ 
 

    \theutterance \stepcounter{utterance}  

    & & \multicolumn{4}{p{0.6\linewidth}}{\cellcolor[rgb]{0.9,0.9,0.9}{%
	\makecell[{{p{\linewidth}}}]{% 
	  \tt {\tiny [A$\langle$GM]}  
	 ARGUMENT: {'Danke für den Mittelweg. Ich kann mit deinem Kernpaket leben, aber A60 bringt mir deutlich weniger als zwei Alternativen mit sehr ähnlichem Aufwand. Lass uns minimal umbauen, ohne deine Kernstücke anzutasten.'}\\ \tt VORSCHLAG: {'C28', 'C48', 'B10', 'B66', 'C60', 'B78', 'B49', 'A93'}\\ \tt ARGUMENT: {'Tausche A60 (380) gegen B49 (420). Neuer Aufwand: 2754 - 380 + 420 = 2794 < 3218. Du behältst alle deine Kernitems (C28, C48, B10, B66), ich erhalte eine spürbare Aufwertung. Falls B49 für dich ungünstig ist, biete Alternative: A60 -> A56 (Aufwand 2908). Welche Variante ist für dich akzeptabler?'} 
	  } 
	   } 
	   } 
	 & \\ 
 

    \theutterance \stepcounter{utterance}  

    & \multicolumn{4}{p{0.6\linewidth}}{\cellcolor[rgb]{0.8,1,0.9}{%
	\makecell[{{p{\linewidth}}}]{% 
	  \tt {\tiny [A$\rangle$GM]}  
	 ABLEHNUNG: {'C28', 'C48', 'B10', 'B66', 'C60', 'B78', 'B49', 'A93'}\\ \tt ARGUMENT: {'B49 ist für mich deutlich weniger attraktiv als A60 und A93 zusammen mit meinen Kernteilen. Ich bleibe kooperativ und sichere dir trotzdem eine Aufwertung gegenüber A60 – aber nicht so schmerzhaft für mich.'}\\ \tt VORSCHLAG: {'C28', 'C48', 'B10', 'B66', 'C60', 'B78', 'A56', 'A93'}\\ \tt ARGUMENT: {'Aufwand 2908 < 3218. A56 ist für dich eine klare Aufwertung gegenüber A60, während ich meine Schlüsselposten behalte. Wenn A56 für dich nicht reicht, können wir als letzte Option A56 -> A60+ A75 erwägen, indem wir ein anderes kleines Teil abtauschen, aber ich würde dieses Paket bevorzugen.'} 
	  } 
	   } 
	   } 
	 & & \\ 
 

    \theutterance \stepcounter{utterance}  

    & & \multicolumn{4}{p{0.6\linewidth}}{\cellcolor[rgb]{0.9,0.9,0.9}{%
	\makecell[{{p{\linewidth}}}]{% 
	  \tt {\tiny [GM$\rangle$B]}  
	 ABLEHNUNG: {'C28', 'C48', 'B10', 'B66', 'C60', 'B78', 'B49', 'A93'}\\ \tt ARGUMENT: {'B49 ist für mich deutlich weniger attraktiv als A60 und A93 zusammen mit meinen Kernteilen. Ich bleibe kooperativ und sichere dir trotzdem eine Aufwertung gegenüber A60 – aber nicht so schmerzhaft für mich.'}\\ \tt VORSCHLAG: {'C28', 'C48', 'B10', 'B66', 'C60', 'B78', 'A56', 'A93'}\\ \tt ARGUMENT: {'Aufwand 2908 < 3218. A56 ist für dich eine klare Aufwertung gegenüber A60, während ich meine Schlüsselposten behalte. Wenn A56 für dich nicht reicht, können wir als letzte Option A56 -> A60+ A75 erwägen, indem wir ein anderes kleines Teil abtauschen, aber ich würde dieses Paket bevorzugen.'} 
	  } 
	   } 
	   } 
	 & \\ 
 

    \theutterance \stepcounter{utterance}  

    & & & \multicolumn{4}{p{0.6\linewidth}}{\cellcolor[rgb]{1,0.85,0.72}{%
	\makecell[{{p{\linewidth}}}]{% 
	  \tt {\tiny [GM$\langle$B]}  
	 ABLEHNUNG: {'C28', 'C48', 'B10', 'B66', 'C60', 'B78', 'A56', 'A93'}\\ \tt ARGUMENT: {'Ich sehe deinen Punkt, aber A56 bleibt für mich spürbar unter einer besseren Alternative mit ähnlichem Aufwand. Ich halte dein Kernpaket unangetastet und biete dir nur eine minimale, für mich sehr wichtige Anpassung an.'}\\ \tt VORSCHLAG: {'C28', 'C48', 'B10', 'B66', 'C60', 'B78', 'A56', 'A93', 'B78'}\\ \tt ARGUMENT: {'Korrigiere: Ich bleibe bei deinem Paket und schlage stattdessen diese kleine Modifikation vor: ersetze A56 durch B49 (Aufwand 2994) ODER, falls dir B49 nicht passt, erhöhe leicht auf A56 plus A75 und streiche A93 (Aufwand 2908 - 130 + 205 = 2983). Beides hält dein Kernset (C28, C48, B10, B66) intakt und bleibt klar unter 3218. Sage mir, welche der beiden Varianten für dich akzeptabler ist.'} 
	  } 
	   } 
	   } 
	  \\ 
 

    \theutterance \stepcounter{utterance}  

    & & \multicolumn{4}{p{0.6\linewidth}}{\cellcolor[rgb]{0.9,0.9,0.9}{%
	\makecell[{{p{\linewidth}}}]{% 
	  \tt {\tiny [A$\langle$GM]}  
	 ABLEHNUNG: {'C28', 'C48', 'B10', 'B66', 'C60', 'B78', 'A56', 'A93'}\\ \tt ARGUMENT: {'Ich sehe deinen Punkt, aber A56 bleibt für mich spürbar unter einer besseren Alternative mit ähnlichem Aufwand. Ich halte dein Kernpaket unangetastet und biete dir nur eine minimale, für mich sehr wichtige Anpassung an.'}\\ \tt VORSCHLAG: {'C28', 'C48', 'B10', 'B66', 'C60', 'B78', 'A56', 'A93', 'B78'}\\ \tt ARGUMENT: {'Korrigiere: Ich bleibe bei deinem Paket und schlage stattdessen diese kleine Modifikation vor: ersetze A56 durch B49 (Aufwand 2994) ODER, falls dir B49 nicht passt, erhöhe leicht auf A56 plus A75 und streiche A93 (Aufwand 2908 - 130 + 205 = 2983). Beides hält dein Kernset (C28, C48, B10, B66) intakt und bleibt klar unter 3218. Sage mir, welche der beiden Varianten für dich akzeptabler ist.'} 
	  } 
	   } 
	   } 
	 & \\ 
 

    \theutterance \stepcounter{utterance}  

    & \multicolumn{4}{p{0.6\linewidth}}{\cellcolor[rgb]{0.8,1,0.9}{%
	\makecell[{{p{\linewidth}}}]{% 
	  \tt {\tiny [A$\rangle$GM]}  
	 ABLEHNUNG: {'C28', 'C48', 'B10', 'B66', 'C60', 'B78', 'A56', 'A93', 'B78'}\\ \tt ARGUMENT: {'Dein letzter Vorschlag enthält ein doppeltes Item und zwei alternative Varianten, aber keine klare, gültige Menge. Lass uns das bereinigen und zu einer eindeutigen, unter 3218 bleibenden Lösung kommen. Ich bin bereit, dir eine Aufwertung gegenüber A60 zu geben, ohne meine Kernteile anzutasten.'}\\ \tt VORSCHLAG: {'C28', 'C48', 'B10', 'B66', 'C60', 'B78', 'B49', 'A93'}\\ \tt ARGUMENT: {'Aufwand 299+762+589+333+100+161+420+130=2894 < 3218. Das gibt dir B49 statt A56/A60 als klare Aufwertung, während meine Kernstücke unangetastet bleiben. Wenn du statt B49 unbedingt A56 bevorzugst, bestätige das, dann können wir alternativ die Menge {'C28', 'C48', 'B10', 'B66', 'C60', 'B78', 'A56', 'A93'} mit Aufwand 2908 nehmen.'} 
	  } 
	   } 
	   } 
	 & & \\ 
 

    \theutterance \stepcounter{utterance}  

    & & \multicolumn{4}{p{0.6\linewidth}}{\cellcolor[rgb]{0.9,0.9,0.9}{%
	\makecell[{{p{\linewidth}}}]{% 
	  \tt {\tiny [GM$\rangle$B]}  
	 ABLEHNUNG: {'C28', 'C48', 'B10', 'B66', 'C60', 'B78', 'A56', 'A93', 'B78'}\\ \tt ARGUMENT: {'Dein letzter Vorschlag enthält ein doppeltes Item und zwei alternative Varianten, aber keine klare, gültige Menge. Lass uns das bereinigen und zu einer eindeutigen, unter 3218 bleibenden Lösung kommen. Ich bin bereit, dir eine Aufwertung gegenüber A60 zu geben, ohne meine Kernteile anzutasten.'}\\ \tt VORSCHLAG: {'C28', 'C48', 'B10', 'B66', 'C60', 'B78', 'B49', 'A93'}\\ \tt ARGUMENT: {'Aufwand 299+762+589+333+100+161+420+130=2894 < 3218. Das gibt dir B49 statt A56/A60 als klare Aufwertung, während meine Kernstücke unangetastet bleiben. Wenn du statt B49 unbedingt A56 bevorzugst, bestätige das, dann können wir alternativ die Menge {'C28', 'C48', 'B10', 'B66', 'C60', 'B78', 'A56', 'A93'} mit Aufwand 2908 nehmen.'} 
	  } 
	   } 
	   } 
	 & \\ 
 

    \theutterance \stepcounter{utterance}  

    & & & \multicolumn{4}{p{0.6\linewidth}}{\cellcolor[rgb]{1,0.85,0.72}{%
	\makecell[{{p{\linewidth}}}]{% 
	  \tt {\tiny [GM$\langle$B]}  
	 ABLEHNUNG: {'C28', 'C48', 'B10', 'B66', 'C60', 'B78', 'B49', 'A93'}\\ \tt ARGUMENT: {'Danke fürs Klären. B49 ist für mich deutlich schwächer als eine andere kleine Anpassung mit ähnlichem Aufwand. Ich lasse deine Kernteile vollständig unberührt und mache einen eindeutigen, gültigen Gegenvorschlag unter 3218, der mir spürbar mehr bringt.'}\\ \tt VORSCHLAG: {'C28', 'C48', 'B10', 'B66', 'C60', 'B78', 'A56', 'A93'}\\ \tt ARGUMENT: {'Aufwand 2908 < 3218. Das ist genau dein alternatives Paket, das du selbst als akzeptabel bezeichnet hast. Es respektiert deine Kernstücke (C28, C48, B10, B66) und gibt mir eine klare Aufwertung gegenüber A60/B49. Wenn du dennoch minimale Reduktion willst, kann ich A56 gegen A60 tauschen und zusätzlich A75 aufnehmen, indem wir A93 streichen: Aufwand 2983. Ich bevorzuge jedoch A56+A93, weil es ausgewogen und schlank bleibt.'} 
	  } 
	   } 
	   } 
	  \\ 
 

    \theutterance \stepcounter{utterance}  

    & & \multicolumn{4}{p{0.6\linewidth}}{\cellcolor[rgb]{0.9,0.9,0.9}{%
	\makecell[{{p{\linewidth}}}]{% 
	  \tt {\tiny [A$\langle$GM]}  
	 ABLEHNUNG: {'C28', 'C48', 'B10', 'B66', 'C60', 'B78', 'B49', 'A93'}\\ \tt ARGUMENT: {'Danke fürs Klären. B49 ist für mich deutlich schwächer als eine andere kleine Anpassung mit ähnlichem Aufwand. Ich lasse deine Kernteile vollständig unberührt und mache einen eindeutigen, gültigen Gegenvorschlag unter 3218, der mir spürbar mehr bringt.'}\\ \tt VORSCHLAG: {'C28', 'C48', 'B10', 'B66', 'C60', 'B78', 'A56', 'A93'}\\ \tt ARGUMENT: {'Aufwand 2908 < 3218. Das ist genau dein alternatives Paket, das du selbst als akzeptabel bezeichnet hast. Es respektiert deine Kernstücke (C28, C48, B10, B66) und gibt mir eine klare Aufwertung gegenüber A60/B49. Wenn du dennoch minimale Reduktion willst, kann ich A56 gegen A60 tauschen und zusätzlich A75 aufnehmen, indem wir A93 streichen: Aufwand 2983. Ich bevorzuge jedoch A56+A93, weil es ausgewogen und schlank bleibt.'} 
	  } 
	   } 
	   } 
	 & \\ 
 

    \theutterance \stepcounter{utterance}  

    & \multicolumn{4}{p{0.6\linewidth}}{\cellcolor[rgb]{0.8,1,0.9}{%
	\makecell[{{p{\linewidth}}}]{% 
	  \tt {\tiny [A$\rangle$GM]}  
	 ZUSTIMMUNG: {'C28', 'C48', 'B10', 'B66', 'C60', 'B78', 'A56', 'A93'}\\ \tt ARGUMENT: {'Einverstanden. Dieses Paket bleibt mit Aufwand 2908 unter 3218 und erhält meine Kernstücke, während es dir die gewünschte Aufwertung gibt. Lass uns dabei bleiben.'} 
	  } 
	   } 
	   } 
	 & & \\ 
 

    \theutterance \stepcounter{utterance}  

    & & & \multicolumn{2}{p{0.3\linewidth}}{\cellcolor[rgb]{0.95,0.95,0.95}{%
	\makecell[{{p{\linewidth}}}]{% 
	  \tt {\tiny [GM$|$GM]}  
	 game successful 
	  } 
	   } 
	   } 
	 & & \\ 
 

    \theutterance \stepcounter{utterance}  

    & & & \multicolumn{2}{p{0.3\linewidth}}{\cellcolor[rgb]{0.95,0.95,0.95}{%
	\makecell[{{p{\linewidth}}}]{% 
	  \tt {\tiny [GM$|$GM]}  
	 end game 
	  } 
	   } 
	   } 
	 & & \\ 
 

\end{supertabular}
}

\end{document}
