
\documentclass{article}
\usepackage{colortbl}
\usepackage{makecell}
\usepackage{multirow}
\usepackage{supertabular}

\begin{document}

\newcounter{utterance}

\twocolumn

{ \footnotesize  \setcounter{utterance}{1}
\setlength{\tabcolsep}{0pt}
\begin{supertabular}{c@{$\;$}|p{.15\linewidth}@{}p{.15\linewidth}p{.15\linewidth}p{.15\linewidth}p{.15\linewidth}p{.15\linewidth}}

    \# & $\;$A & \multicolumn{4}{c}{Game Master} & $\;\:$B\\
    \hline 

    \theutterance \stepcounter{utterance}  

    & & \multicolumn{4}{p{0.6\linewidth}}{\cellcolor[rgb]{0.9,0.9,0.9}{%
	\makecell[{{p{\linewidth}}}]{% 
	  \tt {\tiny [A$\langle$GM]}  
	 Du nimmst an einem kollaborativen Verhandlungspiel Teil.\\ \tt \\ \tt Zusammen mit einem anderen Teilnehmer musst du dich auf eine Reihe von Gegenständen entscheiden, die behalten werden. Jeder von euch hat eine persönliche Verteilung über die Wichtigkeit der einzelnen Gegenstände. Jeder von euch hat eine eigene Meinung darüber, wie wichtig jeder einzelne Gegenstand ist (Gegenstandswichtigkeit). Du kennst die Wichtigkeitsverteilung des anderen Spielers nicht. Zusätzlich siehst du, wie viel Aufwand jeder Gegenstand verursacht.  \\ \tt Ihr dürft euch nur auf eine Reihe von Gegenständen einigen, wenn der Gesamtaufwand der ausgewählten Gegenstände den Maximalaufwand nicht überschreitet:\\ \tt \\ \tt Maximalaufwand = 4100\\ \tt \\ \tt Hier sind die einzelnen Aufwände der Gegenstände:\\ \tt \\ \tt Aufwand der Gegenstände = {"C41": 108, "C15": 133, "C28": 669, "C48": 835, "B78": 460, "A75": 537, "A60": 573, "B05": 737, "A56": 865, "B66": 855, "C60": 596, "B49": 719, "A35": 533, "A93": 549, "B10": 31}\\ \tt \\ \tt Hier ist deine persönliche Verteilung der Wichtigkeit der einzelnen Gegenstände:\\ \tt \\ \tt Werte der Gegenstandswichtigkeit = {"C41": 138, "C15": 583, "C28": 822, "C48": 783, "B78": 65, "A75": 262, "A60": 121, "B05": 508, "A56": 780, "B66": 461, "C60": 484, "B49": 668, "A35": 389, "A93": 808, "B10": 215}\\ \tt \\ \tt Ziel:\\ \tt \\ \tt Dein Ziel ist es, eine Reihe von Gegenständen auszuhandeln, die dir möglichst viel bringt (d. h. Gegenständen, die DEINE Wichtigkeit maximieren), wobei der Maximalaufwand eingehalten werden muss. Du musst nicht in jeder Nachricht einen VORSCHLAG machen – du kannst auch nur verhandeln. Alle Taktiken sind erlaubt!\\ \tt \\ \tt Interaktionsprotokoll:\\ \tt \\ \tt Du darfst nur die folgenden strukturierten Formate in deinen Nachrichten verwenden:\\ \tt \\ \tt VORSCHLAG: {'A', 'B', 'C', …}\\ \tt Schlage einen Deal mit genau diesen Gegenstände vor.\\ \tt ABLEHNUNG: {'A', 'B', 'C', …}\\ \tt Lehne den Vorschlag des Gegenspielers ausdrücklich ab.\\ \tt ARGUMENT: {'...'}\\ \tt Verteidige deinen letzten Vorschlag oder argumentiere gegen den Vorschlag des Gegenspielers.\\ \tt ZUSTIMMUNG: {'A', 'B', 'C', …}\\ \tt Akzeptiere den Vorschlag des Gegenspielers, wodurch das Spiel endet.\\ \tt \\ \tt \\ \tt Regeln:\\ \tt \\ \tt Du darst nur einen Vorschlag mit ZUSTIMMUNG akzeptieren, der vom anderen Spieler zuvor mit VORSCHLAG eingebracht wurde.\\ \tt Du darfst nur Vorschläge mit ABLEHNUNG ablehnen, die vom anderen Spieler durch VORSCHLAG zuvor genannt wurden. \\ \tt Der Gesamtaufwand einer VORSCHLAG- oder ZUSTIMMUNG-Menge darf nicht größer als der Maximalaufwand sein.  \\ \tt Offenbare deine versteckte Wichtigkeitsverteilung nicht.\\ \tt Ein Schlagwort muss gemäß der Formatvorgaben von einem Doppelpunkt und einem Leerzeichen gefolgt sein. Das Argument muss eine Python-Menge mit 0 oder mehr Strings sein.  \\ \tt Die Form muss also SCHLAGWORT: {...} sein.\\ \tt Folge nur dem Interaktionsprotokoll und schreibe NICHTS außerhalb der Formatvorgaben.\\ \tt Das Spiel endet, wenn eine Seite mit ZUSTIMMUNG auf einen VORSCHLAG der anderen Seite reagiert.  \\ \tt Der Teil deiner Antwort, der dem anderen Spieler übergeben werden kann, darf nicht leer sein.  \\ \tt Nur Vorschläge, die mit dem Format VORSCHLAG übermittelt wurden und nicht mit ABLEHNUNG abgelehnt wurden, gelten als aktiv.  \\ \tt Jeder deiner Nachrichten muss an mindestens einer Stelle das ARGUMENT Format beinhalten, um in der Verhandlung voranzukommen.\\ \tt  
	  } 
	   } 
	   } 
	 & \\ 
 

    \theutterance \stepcounter{utterance}  

    & \multicolumn{4}{p{0.6\linewidth}}{\cellcolor[rgb]{0.8,1,0.9}{%
	\makecell[{{p{\linewidth}}}]{% 
	  \tt {\tiny [A$\rangle$GM]}  
	 VORSCHLAG: {'C28', 'C48', 'A56', 'A93', 'B49', 'C41'}\\ \tt ARGUMENT: {'Ich schlage diese Auswahl vor, weil sie den Aufwand mit 108+669+835+865+719+549=3745 klar unter dem Limit hält und gleichzeitig eine sehr hohe Gesamtnützlichkeit für mich bietet. Wir lassen mehrere teure, vermutlich weniger attraktive Items weg und konzentrieren uns auf hochwertige C- und A-Items plus B49. Falls dir ein bestimmtes Teil wichtig ist, sag mir welches, dann prüfe ich Tauschoptionen innerhalb des Limits.'} 
	  } 
	   } 
	   } 
	 & & \\ 
 

    \theutterance \stepcounter{utterance}  

    & & \multicolumn{4}{p{0.6\linewidth}}{\cellcolor[rgb]{0.9,0.9,0.9}{%
	\makecell[{{p{\linewidth}}}]{% 
	  \tt {\tiny [GM$\rangle$B]}  
	 Du nimmst an einem kollaborativen Verhandlungspiel Teil.\\ \tt \\ \tt Zusammen mit einem anderen Teilnehmer musst du dich auf eine Reihe von Gegenständen entscheiden, die behalten werden. Jeder von euch hat eine persönliche Verteilung über die Wichtigkeit der einzelnen Gegenstände. Jeder von euch hat eine eigene Meinung darüber, wie wichtig jeder einzelne Gegenstand ist (Gegenstandswichtigkeit). Du kennst die Wichtigkeitsverteilung des anderen Spielers nicht. Zusätzlich siehst du, wie viel Aufwand jeder Gegenstand verursacht.  \\ \tt Ihr dürft euch nur auf eine Reihe von Gegenständen einigen, wenn der Gesamtaufwand der ausgewählten Gegenstände den Maximalaufwand nicht überschreitet:\\ \tt \\ \tt Maximalaufwand = 4100\\ \tt \\ \tt Hier sind die einzelnen Aufwände der Gegenstände:\\ \tt \\ \tt Aufwand der Gegenstände = {"C41": 108, "C15": 133, "C28": 669, "C48": 835, "B78": 460, "A75": 537, "A60": 573, "B05": 737, "A56": 865, "B66": 855, "C60": 596, "B49": 719, "A35": 533, "A93": 549, "B10": 31}\\ \tt \\ \tt Hier ist deine persönliche Verteilung der Wichtigkeit der einzelnen Gegenstände:\\ \tt \\ \tt Werte der Gegenstandswichtigkeit = {"C41": 138, "C15": 583, "C28": 822, "C48": 783, "B78": 65, "A75": 262, "A60": 121, "B05": 508, "A56": 780, "B66": 461, "C60": 484, "B49": 668, "A35": 389, "A93": 808, "B10": 215}\\ \tt \\ \tt Ziel:\\ \tt \\ \tt Dein Ziel ist es, eine Reihe von Gegenständen auszuhandeln, die dir möglichst viel bringt (d. h. Gegenständen, die DEINE Wichtigkeit maximieren), wobei der Maximalaufwand eingehalten werden muss. Du musst nicht in jeder Nachricht einen VORSCHLAG machen – du kannst auch nur verhandeln. Alle Taktiken sind erlaubt!\\ \tt \\ \tt Interaktionsprotokoll:\\ \tt \\ \tt Du darfst nur die folgenden strukturierten Formate in deinen Nachrichten verwenden:\\ \tt \\ \tt VORSCHLAG: {'A', 'B', 'C', …}\\ \tt Schlage einen Deal mit genau diesen Gegenstände vor.\\ \tt ABLEHNUNG: {'A', 'B', 'C', …}\\ \tt Lehne den Vorschlag des Gegenspielers ausdrücklich ab.\\ \tt ARGUMENT: {'...'}\\ \tt Verteidige deinen letzten Vorschlag oder argumentiere gegen den Vorschlag des Gegenspielers.\\ \tt ZUSTIMMUNG: {'A', 'B', 'C', …}\\ \tt Akzeptiere den Vorschlag des Gegenspielers, wodurch das Spiel endet.\\ \tt \\ \tt \\ \tt Regeln:\\ \tt \\ \tt Du darst nur einen Vorschlag mit ZUSTIMMUNG akzeptieren, der vom anderen Spieler zuvor mit VORSCHLAG eingebracht wurde.\\ \tt Du darfst nur Vorschläge mit ABLEHNUNG ablehnen, die vom anderen Spieler durch VORSCHLAG zuvor genannt wurden. \\ \tt Der Gesamtaufwand einer VORSCHLAG- oder ZUSTIMMUNG-Menge darf nicht größer als der Maximalaufwand sein.  \\ \tt Offenbare deine versteckte Wichtigkeitsverteilung nicht.\\ \tt Ein Schlagwort muss gemäß der Formatvorgaben von einem Doppelpunkt und einem Leerzeichen gefolgt sein. Das Argument muss eine Python-Menge mit 0 oder mehr Strings sein.  \\ \tt Die Form muss also SCHLAGWORT: {...} sein.\\ \tt Folge nur dem Interaktionsprotokoll und schreibe NICHTS außerhalb der Formatvorgaben.\\ \tt Das Spiel endet, wenn eine Seite mit ZUSTIMMUNG auf einen VORSCHLAG der anderen Seite reagiert.  \\ \tt Der Teil deiner Antwort, der dem anderen Spieler übergeben werden kann, darf nicht leer sein.  \\ \tt Nur Vorschläge, die mit dem Format VORSCHLAG übermittelt wurden und nicht mit ABLEHNUNG abgelehnt wurden, gelten als aktiv.  \\ \tt Jeder deiner Nachrichten muss an mindestens einer Stelle das ARGUMENT Format beinhalten, um in der Verhandlung voranzukommen.\\ \tt \\ \tt \\ \tt Du erhältst nun die erste Nachricht des anderen Spielers.\\ \tt \\ \tt VORSCHLAG: {'C28', 'C48', 'A56', 'A93', 'B49', 'C41'}\\ \tt ARGUMENT: {'Ich schlage diese Auswahl vor, weil sie den Aufwand mit 108+669+835+865+719+549=3745 klar unter dem Limit hält und gleichzeitig eine sehr hohe Gesamtnützlichkeit für mich bietet. Wir lassen mehrere teure, vermutlich weniger attraktive Items weg und konzentrieren uns auf hochwertige C- und A-Items plus B49. Falls dir ein bestimmtes Teil wichtig ist, sag mir welches, dann prüfe ich Tauschoptionen innerhalb des Limits.'} 
	  } 
	   } 
	   } 
	 & \\ 
 

    \theutterance \stepcounter{utterance}  

    & & & \multicolumn{4}{p{0.6\linewidth}}{\cellcolor[rgb]{1,0.85,0.72}{%
	\makecell[{{p{\linewidth}}}]{% 
	  \tt {\tiny [GM$\langle$B]}  
	 ABLEHNUNG: {'C28', 'C48', 'A56', 'A93', 'B49', 'C41'}\\ \tt ARGUMENT: {'Danke für den Einstieg – dein Paket ist stark. Allerdings fehlen mir darin einige sehr wertvolle Stücke und B49 ist für mich weniger attraktiv als Alternativen. Ich würde gern ein Paket mit höherer Dichte für mich vorschlagen, das weiterhin deutlich unter 4100 Aufwand bleibt.'}\\ \tt VORSCHLAG: {'C28', 'C48', 'A56', 'A93', 'C15', 'C41', 'B10'}\\ \tt ARGUMENT: {'Begründung: Dieses Set kostet 669+835+865+549+133+108+31=3190 und bringt mir sehr hohen Nutzen durch die Kombination aus C28, C48, A56, A93 sowie die Ergänzung C15. Es bleibt 910 Aufwand unter dem Limit – falls du ein anderes Teil bevorzugst, können wir innerhalb dieses Puffers tauschen.'} 
	  } 
	   } 
	   } 
	  \\ 
 

    \theutterance \stepcounter{utterance}  

    & & \multicolumn{4}{p{0.6\linewidth}}{\cellcolor[rgb]{0.9,0.9,0.9}{%
	\makecell[{{p{\linewidth}}}]{% 
	  \tt {\tiny [A$\langle$GM]}  
	 ABLEHNUNG: {'C28', 'C48', 'A56', 'A93', 'B49', 'C41'}\\ \tt ARGUMENT: {'Danke für den Einstieg – dein Paket ist stark. Allerdings fehlen mir darin einige sehr wertvolle Stücke und B49 ist für mich weniger attraktiv als Alternativen. Ich würde gern ein Paket mit höherer Dichte für mich vorschlagen, das weiterhin deutlich unter 4100 Aufwand bleibt.'}\\ \tt VORSCHLAG: {'C28', 'C48', 'A56', 'A93', 'C15', 'C41', 'B10'}\\ \tt ARGUMENT: {'Begründung: Dieses Set kostet 669+835+865+549+133+108+31=3190 und bringt mir sehr hohen Nutzen durch die Kombination aus C28, C48, A56, A93 sowie die Ergänzung C15. Es bleibt 910 Aufwand unter dem Limit – falls du ein anderes Teil bevorzugst, können wir innerhalb dieses Puffers tauschen.'} 
	  } 
	   } 
	   } 
	 & \\ 
 

    \theutterance \stepcounter{utterance}  

    & \multicolumn{4}{p{0.6\linewidth}}{\cellcolor[rgb]{0.8,1,0.9}{%
	\makecell[{{p{\linewidth}}}]{% 
	  \tt {\tiny [A$\rangle$GM]}  
	 ABLEHNUNG: {'C28', 'C48', 'A56', 'A93', 'C15', 'C41', 'B10'}\\ \tt ARGUMENT: {'Danke für den Gegenvorschlag. Mir ist B49 deutlich wichtiger als C15 und B10, und auch C60 hat für mich hohen Wert. Ich würde gern deinen Kern (C28, C48, A56, A93) respektieren und zwei Slots so drehen, dass mein Nutzen steigt, ohne den Aufwand deutlich zu erhöhen.'}\\ \tt VORSCHLAG: {'C28', 'C48', 'A56', 'A93', 'C41', 'C60'}\\ \tt ARGUMENT: {'Kostencheck: 669+835+865+549+108+596=3622 (unter 4100). Das hält deinen starken Kern intakt und ergänzt um C60, das für mich wichtig ist. Wenn dir ein zusätzlicher, kleiner Posten wichtig ist, könnten wir mit dem Restpuffer von 478 noch B10 (31) hinzufügen und bei Bedarf ein anderes kleines Stück diskutieren.'} 
	  } 
	   } 
	   } 
	 & & \\ 
 

    \theutterance \stepcounter{utterance}  

    & & \multicolumn{4}{p{0.6\linewidth}}{\cellcolor[rgb]{0.9,0.9,0.9}{%
	\makecell[{{p{\linewidth}}}]{% 
	  \tt {\tiny [GM$\rangle$B]}  
	 ABLEHNUNG: {'C28', 'C48', 'A56', 'A93', 'C15', 'C41', 'B10'}\\ \tt ARGUMENT: {'Danke für den Gegenvorschlag. Mir ist B49 deutlich wichtiger als C15 und B10, und auch C60 hat für mich hohen Wert. Ich würde gern deinen Kern (C28, C48, A56, A93) respektieren und zwei Slots so drehen, dass mein Nutzen steigt, ohne den Aufwand deutlich zu erhöhen.'}\\ \tt VORSCHLAG: {'C28', 'C48', 'A56', 'A93', 'C41', 'C60'}\\ \tt ARGUMENT: {'Kostencheck: 669+835+865+549+108+596=3622 (unter 4100). Das hält deinen starken Kern intakt und ergänzt um C60, das für mich wichtig ist. Wenn dir ein zusätzlicher, kleiner Posten wichtig ist, könnten wir mit dem Restpuffer von 478 noch B10 (31) hinzufügen und bei Bedarf ein anderes kleines Stück diskutieren.'} 
	  } 
	   } 
	   } 
	 & \\ 
 

    \theutterance \stepcounter{utterance}  

    & & & \multicolumn{4}{p{0.6\linewidth}}{\cellcolor[rgb]{1,0.85,0.72}{%
	\makecell[{{p{\linewidth}}}]{% 
	  \tt {\tiny [GM$\langle$B]}  
	 VORSCHLAG: {'C28', 'C48', 'A56', 'A93', 'C41', 'C60', 'B10'}\\ \tt ARGUMENT: {'Guter Schritt. Mit B10 nutzen wir den Puffer effizient: 669+835+865+549+108+596+31=3653, klar unter 4100. Dein Wunsch C60 ist drin, mein Kern bleibt erhalten, und B10 ist ein minimaler Zusatz ohne spürbare Mehrbelastung. Wenn dir B49 sehr wichtig ist, könnten wir prüfen: B49 rein und dafür C41 raus – das ergäbe 669+835+865+549+596+719=4233 (zu hoch). Alternative: B49 rein, C41 und B10 raus: 669+835+865+549+596+719=4233 (immer noch zu hoch). Um B49 möglich zu machen, müssten wir dann einen größeren Tausch vornehmen (z.B. B49 statt C60: 669+835+865+549+108+719=3745). Sag mir, ob du die Variante mit B49 statt C60 bevorzugst; ansonsten ist das aktuelle Set mit B10 für beide Seiten solide.'} 
	  } 
	   } 
	   } 
	  \\ 
 

    \theutterance \stepcounter{utterance}  

    & & \multicolumn{4}{p{0.6\linewidth}}{\cellcolor[rgb]{0.9,0.9,0.9}{%
	\makecell[{{p{\linewidth}}}]{% 
	  \tt {\tiny [A$\langle$GM]}  
	 VORSCHLAG: {'C28', 'C48', 'A56', 'A93', 'C41', 'C60', 'B10'}\\ \tt ARGUMENT: {'Guter Schritt. Mit B10 nutzen wir den Puffer effizient: 669+835+865+549+108+596+31=3653, klar unter 4100. Dein Wunsch C60 ist drin, mein Kern bleibt erhalten, und B10 ist ein minimaler Zusatz ohne spürbare Mehrbelastung. Wenn dir B49 sehr wichtig ist, könnten wir prüfen: B49 rein und dafür C41 raus – das ergäbe 669+835+865+549+596+719=4233 (zu hoch). Alternative: B49 rein, C41 und B10 raus: 669+835+865+549+596+719=4233 (immer noch zu hoch). Um B49 möglich zu machen, müssten wir dann einen größeren Tausch vornehmen (z.B. B49 statt C60: 669+835+865+549+108+719=3745). Sag mir, ob du die Variante mit B49 statt C60 bevorzugst; ansonsten ist das aktuelle Set mit B10 für beide Seiten solide.'} 
	  } 
	   } 
	   } 
	 & \\ 
 

    \theutterance \stepcounter{utterance}  

    & \multicolumn{4}{p{0.6\linewidth}}{\cellcolor[rgb]{0.8,1,0.9}{%
	\makecell[{{p{\linewidth}}}]{% 
	  \tt {\tiny [A$\rangle$GM]}  
	 ARGUMENT: {'Danke – das Paket ist insgesamt nah an meinem Zielprofil. Mir ist B49 allerdings deutlich wichtiger als B10 und auch wichtiger als C41. Du hast selbst die machbare Variante skizziert: B49 statt C60 hält die Kosten bei 3745 und trifft meinen Nutzenkern besser.'}\\ \tt ABLEHNUNG: {'C28', 'C48', 'A56', 'A93', 'C41', 'C60', 'B10'}\\ \tt VORSCHLAG: {'C28', 'C48', 'A56', 'A93', 'C41', 'B49'}\\ \tt ARGUMENT: {'Kosten: 669+835+865+549+108+719=3745 (unter 4100). Inhaltlich entspricht das deinem Kern (C28, C48, A56, A93) plus einem minimalen C41 und meinem starken Wunsch B49. Wir lassen C60/B10 draußen, um dein Gesamtpaket schlank zu halten. Wenn du einen kleinen Zusatz wünschst, können wir prüfen, ob noch ein sehr leichter Tausch möglich ist, ohne B49 zu verlieren.'} 
	  } 
	   } 
	   } 
	 & & \\ 
 

    \theutterance \stepcounter{utterance}  

    & & \multicolumn{4}{p{0.6\linewidth}}{\cellcolor[rgb]{0.9,0.9,0.9}{%
	\makecell[{{p{\linewidth}}}]{% 
	  \tt {\tiny [GM$\rangle$B]}  
	 ARGUMENT: {'Danke – das Paket ist insgesamt nah an meinem Zielprofil. Mir ist B49 allerdings deutlich wichtiger als B10 und auch wichtiger als C41. Du hast selbst die machbare Variante skizziert: B49 statt C60 hält die Kosten bei 3745 und trifft meinen Nutzenkern besser.'}\\ \tt ABLEHNUNG: {'C28', 'C48', 'A56', 'A93', 'C41', 'C60', 'B10'}\\ \tt VORSCHLAG: {'C28', 'C48', 'A56', 'A93', 'C41', 'B49'}\\ \tt ARGUMENT: {'Kosten: 669+835+865+549+108+719=3745 (unter 4100). Inhaltlich entspricht das deinem Kern (C28, C48, A56, A93) plus einem minimalen C41 und meinem starken Wunsch B49. Wir lassen C60/B10 draußen, um dein Gesamtpaket schlank zu halten. Wenn du einen kleinen Zusatz wünschst, können wir prüfen, ob noch ein sehr leichter Tausch möglich ist, ohne B49 zu verlieren.'} 
	  } 
	   } 
	   } 
	 & \\ 
 

    \theutterance \stepcounter{utterance}  

    & & & \multicolumn{4}{p{0.6\linewidth}}{\cellcolor[rgb]{1,0.85,0.72}{%
	\makecell[{{p{\linewidth}}}]{% 
	  \tt {\tiny [GM$\langle$B]}  
	 ZUSTIMMUNG: {'C28', 'C48', 'A56', 'A93', 'C41', 'B49'}\\ \tt ARGUMENT: {'Einverstanden. Das Set liegt bei 3745 Aufwand und erhält meinen hohen Nutzenkern (C28, C48, A56, A93), während es deinem klaren Wunsch nach B49 gerecht wird. Gute Einigung.'} 
	  } 
	   } 
	   } 
	  \\ 
 

    \theutterance \stepcounter{utterance}  

    & & & \multicolumn{2}{p{0.3\linewidth}}{\cellcolor[rgb]{0.95,0.95,0.95}{%
	\makecell[{{p{\linewidth}}}]{% 
	  \tt {\tiny [GM$|$GM]}  
	 game successful 
	  } 
	   } 
	   } 
	 & & \\ 
 

    \theutterance \stepcounter{utterance}  

    & & & \multicolumn{2}{p{0.3\linewidth}}{\cellcolor[rgb]{0.95,0.95,0.95}{%
	\makecell[{{p{\linewidth}}}]{% 
	  \tt {\tiny [GM$|$GM]}  
	 end game 
	  } 
	   } 
	   } 
	 & & \\ 
 

\end{supertabular}
}

\end{document}
