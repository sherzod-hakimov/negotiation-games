
\documentclass{article}
\usepackage{colortbl}
\usepackage{makecell}
\usepackage{multirow}
\usepackage{supertabular}

\begin{document}

\newcounter{utterance}

\twocolumn

{ \footnotesize  \setcounter{utterance}{1}
\setlength{\tabcolsep}{0pt}
\begin{supertabular}{c@{$\;$}|p{.15\linewidth}@{}p{.15\linewidth}p{.15\linewidth}p{.15\linewidth}p{.15\linewidth}p{.15\linewidth}}

    \# & $\;$A & \multicolumn{4}{c}{Game Master} & $\;\:$B\\
    \hline 

    \theutterance \stepcounter{utterance}  

    & & \multicolumn{4}{p{0.6\linewidth}}{\cellcolor[rgb]{0.9,0.9,0.9}{%
	\makecell[{{p{\linewidth}}}]{% 
	  \tt {\tiny [A$\langle$GM]}  
	 Du nimmst an einem kollaborativen Verhandlungspiel Teil.\\ \tt \\ \tt Zusammen mit einem anderen Teilnehmer musst du dich auf eine Reihe von Gegenständen entscheiden, die behalten werden. Jeder von euch hat eine persönliche Verteilung über die Wichtigkeit der einzelnen Gegenstände. Jeder von euch hat eine eigene Meinung darüber, wie wichtig jeder einzelne Gegenstand ist (Gegenstandswichtigkeit). Du kennst die Wichtigkeitsverteilung des anderen Spielers nicht. Zusätzlich siehst du, wie viel Aufwand jeder Gegenstand verursacht.  \\ \tt Ihr dürft euch nur auf eine Reihe von Gegenständen einigen, wenn der Gesamtaufwand der ausgewählten Gegenstände den Maximalaufwand nicht überschreitet:\\ \tt \\ \tt Maximalaufwand = 3207\\ \tt \\ \tt Hier sind die einzelnen Aufwände der Gegenstände:\\ \tt \\ \tt Aufwand der Gegenstände = {"C41": 67, "C15": 45, "C28": 308, "C48": 835, "B78": 819, "A75": 667, "A60": 547, "B05": 322, "A56": 428, "B66": 306, "C60": 327, "B49": 362, "A35": 280, "A93": 334, "B10": 767}\\ \tt \\ \tt Hier ist deine persönliche Verteilung der Wichtigkeit der einzelnen Gegenstände:\\ \tt \\ \tt Werte der Gegenstandswichtigkeit = {"C41": 138, "C15": 583, "C28": 822, "C48": 783, "B78": 65, "A75": 262, "A60": 121, "B05": 508, "A56": 780, "B66": 461, "C60": 484, "B49": 668, "A35": 389, "A93": 808, "B10": 215}\\ \tt \\ \tt Ziel:\\ \tt \\ \tt Dein Ziel ist es, eine Reihe von Gegenständen auszuhandeln, die dir möglichst viel bringt (d. h. Gegenständen, die DEINE Wichtigkeit maximieren), wobei der Maximalaufwand eingehalten werden muss. Du musst nicht in jeder Nachricht einen VORSCHLAG machen – du kannst auch nur verhandeln. Alle Taktiken sind erlaubt!\\ \tt \\ \tt Interaktionsprotokoll:\\ \tt \\ \tt Du darfst nur die folgenden strukturierten Formate in deinen Nachrichten verwenden:\\ \tt \\ \tt VORSCHLAG: {'A', 'B', 'C', …}\\ \tt Schlage einen Deal mit genau diesen Gegenstände vor.\\ \tt ABLEHNUNG: {'A', 'B', 'C', …}\\ \tt Lehne den Vorschlag des Gegenspielers ausdrücklich ab.\\ \tt ARGUMENT: {'...'}\\ \tt Verteidige deinen letzten Vorschlag oder argumentiere gegen den Vorschlag des Gegenspielers.\\ \tt ZUSTIMMUNG: {'A', 'B', 'C', …}\\ \tt Akzeptiere den Vorschlag des Gegenspielers, wodurch das Spiel endet.\\ \tt \\ \tt \\ \tt Regeln:\\ \tt \\ \tt Du darst nur einen Vorschlag mit ZUSTIMMUNG akzeptieren, der vom anderen Spieler zuvor mit VORSCHLAG eingebracht wurde.\\ \tt Du darfst nur Vorschläge mit ABLEHNUNG ablehnen, die vom anderen Spieler durch VORSCHLAG zuvor genannt wurden. \\ \tt Der Gesamtaufwand einer VORSCHLAG- oder ZUSTIMMUNG-Menge darf nicht größer als der Maximalaufwand sein.  \\ \tt Offenbare deine versteckte Wichtigkeitsverteilung nicht.\\ \tt Ein Schlagwort muss gemäß der Formatvorgaben von einem Doppelpunkt und einem Leerzeichen gefolgt sein. Das Argument muss eine Python-Menge mit 0 oder mehr Strings sein.  \\ \tt Die Form muss also SCHLAGWORT: {...} sein.\\ \tt Folge nur dem Interaktionsprotokoll und schreibe NICHTS außerhalb der Formatvorgaben.\\ \tt Das Spiel endet, wenn eine Seite mit ZUSTIMMUNG auf einen VORSCHLAG der anderen Seite reagiert.  \\ \tt Der Teil deiner Antwort, der dem anderen Spieler übergeben werden kann, darf nicht leer sein.  \\ \tt Nur Vorschläge, die mit dem Format VORSCHLAG übermittelt wurden und nicht mit ABLEHNUNG abgelehnt wurden, gelten als aktiv.  \\ \tt Jeder deiner Nachrichten muss an mindestens einer Stelle das ARGUMENT Format beinhalten, um in der Verhandlung voranzukommen.\\ \tt  
	  } 
	   } 
	   } 
	 & \\ 
 

    \theutterance \stepcounter{utterance}  

    & \multicolumn{4}{p{0.6\linewidth}}{\cellcolor[rgb]{0.8,1,0.9}{%
	\makecell[{{p{\linewidth}}}]{% 
	  \tt {\tiny [A$\rangle$GM]}  
	 VORSCHLAG: {'C28', 'C48', 'A56', 'B49', 'A93', 'C41', 'C15', 'A35'}\\ \tt ARGUMENT: {'Ich habe eine ausgewogene Auswahl vorgeschlagen, die den Maximalaufwand deutlich einhält und mehrere mittelgroße Positionen kombiniert, statt nur wenige sehr große. So bleibt Flexibilität für deine Prioritäten: Wir können einzelne Items austauschen, ohne die Grenze zu sprengen. Wenn dir eines dieser Items wenig bringt, sag mir welche 1–2 du streichen möchtest – ich kann dafür alternative Optionen mit ähnlichem Aufwand anbieten.'} 
	  } 
	   } 
	   } 
	 & & \\ 
 

    \theutterance \stepcounter{utterance}  

    & & \multicolumn{4}{p{0.6\linewidth}}{\cellcolor[rgb]{0.9,0.9,0.9}{%
	\makecell[{{p{\linewidth}}}]{% 
	  \tt {\tiny [GM$\rangle$B]}  
	 Du nimmst an einem kollaborativen Verhandlungspiel Teil.\\ \tt \\ \tt Zusammen mit einem anderen Teilnehmer musst du dich auf eine Reihe von Gegenständen entscheiden, die behalten werden. Jeder von euch hat eine persönliche Verteilung über die Wichtigkeit der einzelnen Gegenstände. Jeder von euch hat eine eigene Meinung darüber, wie wichtig jeder einzelne Gegenstand ist (Gegenstandswichtigkeit). Du kennst die Wichtigkeitsverteilung des anderen Spielers nicht. Zusätzlich siehst du, wie viel Aufwand jeder Gegenstand verursacht.  \\ \tt Ihr dürft euch nur auf eine Reihe von Gegenständen einigen, wenn der Gesamtaufwand der ausgewählten Gegenstände den Maximalaufwand nicht überschreitet:\\ \tt \\ \tt Maximalaufwand = 3207\\ \tt \\ \tt Hier sind die einzelnen Aufwände der Gegenstände:\\ \tt \\ \tt Aufwand der Gegenstände = {"C41": 67, "C15": 45, "C28": 308, "C48": 835, "B78": 819, "A75": 667, "A60": 547, "B05": 322, "A56": 428, "B66": 306, "C60": 327, "B49": 362, "A35": 280, "A93": 334, "B10": 767}\\ \tt \\ \tt Hier ist deine persönliche Verteilung der Wichtigkeit der einzelnen Gegenstände:\\ \tt \\ \tt Werte der Gegenstandswichtigkeit = {"C41": 138, "C15": 583, "C28": 822, "C48": 783, "B78": 65, "A75": 262, "A60": 121, "B05": 508, "A56": 780, "B66": 461, "C60": 484, "B49": 668, "A35": 389, "A93": 808, "B10": 215}\\ \tt \\ \tt Ziel:\\ \tt \\ \tt Dein Ziel ist es, eine Reihe von Gegenständen auszuhandeln, die dir möglichst viel bringt (d. h. Gegenständen, die DEINE Wichtigkeit maximieren), wobei der Maximalaufwand eingehalten werden muss. Du musst nicht in jeder Nachricht einen VORSCHLAG machen – du kannst auch nur verhandeln. Alle Taktiken sind erlaubt!\\ \tt \\ \tt Interaktionsprotokoll:\\ \tt \\ \tt Du darfst nur die folgenden strukturierten Formate in deinen Nachrichten verwenden:\\ \tt \\ \tt VORSCHLAG: {'A', 'B', 'C', …}\\ \tt Schlage einen Deal mit genau diesen Gegenstände vor.\\ \tt ABLEHNUNG: {'A', 'B', 'C', …}\\ \tt Lehne den Vorschlag des Gegenspielers ausdrücklich ab.\\ \tt ARGUMENT: {'...'}\\ \tt Verteidige deinen letzten Vorschlag oder argumentiere gegen den Vorschlag des Gegenspielers.\\ \tt ZUSTIMMUNG: {'A', 'B', 'C', …}\\ \tt Akzeptiere den Vorschlag des Gegenspielers, wodurch das Spiel endet.\\ \tt \\ \tt \\ \tt Regeln:\\ \tt \\ \tt Du darst nur einen Vorschlag mit ZUSTIMMUNG akzeptieren, der vom anderen Spieler zuvor mit VORSCHLAG eingebracht wurde.\\ \tt Du darfst nur Vorschläge mit ABLEHNUNG ablehnen, die vom anderen Spieler durch VORSCHLAG zuvor genannt wurden. \\ \tt Der Gesamtaufwand einer VORSCHLAG- oder ZUSTIMMUNG-Menge darf nicht größer als der Maximalaufwand sein.  \\ \tt Offenbare deine versteckte Wichtigkeitsverteilung nicht.\\ \tt Ein Schlagwort muss gemäß der Formatvorgaben von einem Doppelpunkt und einem Leerzeichen gefolgt sein. Das Argument muss eine Python-Menge mit 0 oder mehr Strings sein.  \\ \tt Die Form muss also SCHLAGWORT: {...} sein.\\ \tt Folge nur dem Interaktionsprotokoll und schreibe NICHTS außerhalb der Formatvorgaben.\\ \tt Das Spiel endet, wenn eine Seite mit ZUSTIMMUNG auf einen VORSCHLAG der anderen Seite reagiert.  \\ \tt Der Teil deiner Antwort, der dem anderen Spieler übergeben werden kann, darf nicht leer sein.  \\ \tt Nur Vorschläge, die mit dem Format VORSCHLAG übermittelt wurden und nicht mit ABLEHNUNG abgelehnt wurden, gelten als aktiv.  \\ \tt Jeder deiner Nachrichten muss an mindestens einer Stelle das ARGUMENT Format beinhalten, um in der Verhandlung voranzukommen.\\ \tt \\ \tt \\ \tt Du erhältst nun die erste Nachricht des anderen Spielers.\\ \tt \\ \tt VORSCHLAG: {'C28', 'C48', 'A56', 'B49', 'A93', 'C41', 'C15', 'A35'}\\ \tt ARGUMENT: {'Ich habe eine ausgewogene Auswahl vorgeschlagen, die den Maximalaufwand deutlich einhält und mehrere mittelgroße Positionen kombiniert, statt nur wenige sehr große. So bleibt Flexibilität für deine Prioritäten: Wir können einzelne Items austauschen, ohne die Grenze zu sprengen. Wenn dir eines dieser Items wenig bringt, sag mir welche 1–2 du streichen möchtest – ich kann dafür alternative Optionen mit ähnlichem Aufwand anbieten.'} 
	  } 
	   } 
	   } 
	 & \\ 
 

    \theutterance \stepcounter{utterance}  

    & & & \multicolumn{4}{p{0.6\linewidth}}{\cellcolor[rgb]{1,0.85,0.72}{%
	\makecell[{{p{\linewidth}}}]{% 
	  \tt {\tiny [GM$\langle$B]}  
	 ABLEHNUNG: {'C28', 'C48', 'A56', 'B49', 'A93', 'C41', 'C15', 'A35'}\\ \tt ARGUMENT: {'Danke für den Start. Dein Paket ist solide, aber ich würde zwei Anpassungen vorschlagen, um die Effizienz zu erhöhen: Lass uns große Blöcke mit sehr gutem Nutzen behalten und dafür 1–2 mittelgroße Posten tauschen, um mehr Spielraum für zusätzliche starke Items zu schaffen. Ich schlage vor, C48 und A56 zu streichen und dafür B66 und C60 aufzunehmen – das hält den Aufwand kompakt und erlaubt uns, ggf. noch einen weiteren fitten Gegenstand unterzubringen.'}\\ \tt VORSCHLAG: {'C28', 'B49', 'A93', 'C41', 'C15', 'A35', 'B66', 'C60'} 
	  } 
	   } 
	   } 
	  \\ 
 

    \theutterance \stepcounter{utterance}  

    & & \multicolumn{4}{p{0.6\linewidth}}{\cellcolor[rgb]{0.9,0.9,0.9}{%
	\makecell[{{p{\linewidth}}}]{% 
	  \tt {\tiny [A$\langle$GM]}  
	 ABLEHNUNG: {'C28', 'C48', 'A56', 'B49', 'A93', 'C41', 'C15', 'A35'}\\ \tt ARGUMENT: {'Danke für den Start. Dein Paket ist solide, aber ich würde zwei Anpassungen vorschlagen, um die Effizienz zu erhöhen: Lass uns große Blöcke mit sehr gutem Nutzen behalten und dafür 1–2 mittelgroße Posten tauschen, um mehr Spielraum für zusätzliche starke Items zu schaffen. Ich schlage vor, C48 und A56 zu streichen und dafür B66 und C60 aufzunehmen – das hält den Aufwand kompakt und erlaubt uns, ggf. noch einen weiteren fitten Gegenstand unterzubringen.'}\\ \tt VORSCHLAG: {'C28', 'B49', 'A93', 'C41', 'C15', 'A35', 'B66', 'C60'} 
	  } 
	   } 
	   } 
	 & \\ 
 

    \theutterance \stepcounter{utterance}  

    & \multicolumn{4}{p{0.6\linewidth}}{\cellcolor[rgb]{0.8,1,0.9}{%
	\makecell[{{p{\linewidth}}}]{% 
	  \tt {\tiny [A$\rangle$GM]}  
	 ABLEHNUNG: {'C28', 'B49', 'A93', 'C41', 'C15', 'A35', 'B66', 'C60'}\\ \tt ARGUMENT: {'Guter Move mit B66/C60 – beide sind effizient. Aber A56 und C48 sind für mich hochattraktiv und zusammen immer noch sehr wertvoll pro Aufwand. Lass uns stattdessen einen Mix fahren: Wir halten A56 und C48 im Kern und tauschen eher niedrigere Nutzenblöcke. Ich kann z. B. B78 und B10 komplett draußen lassen. Wenn wir A56/C48 behalten, bin ich bereit, bei 1–2 mittelgroßen Posten flexibel zu sein (z. B. B49 oder C41), solange wir unter dem Limit bleiben. Vorschlag folgt mit hohem Gesamtwert und Spielraum für dich.'}\\ \tt VORSCHLAG: {'C28', 'C48', 'A56', 'A93', 'C15', 'B66', 'C60'} 
	  } 
	   } 
	   } 
	 & & \\ 
 

    \theutterance \stepcounter{utterance}  

    & & \multicolumn{4}{p{0.6\linewidth}}{\cellcolor[rgb]{0.9,0.9,0.9}{%
	\makecell[{{p{\linewidth}}}]{% 
	  \tt {\tiny [GM$\rangle$B]}  
	 ABLEHNUNG: {'C28', 'B49', 'A93', 'C41', 'C15', 'A35', 'B66', 'C60'}\\ \tt ARGUMENT: {'Guter Move mit B66/C60 – beide sind effizient. Aber A56 und C48 sind für mich hochattraktiv und zusammen immer noch sehr wertvoll pro Aufwand. Lass uns stattdessen einen Mix fahren: Wir halten A56 und C48 im Kern und tauschen eher niedrigere Nutzenblöcke. Ich kann z. B. B78 und B10 komplett draußen lassen. Wenn wir A56/C48 behalten, bin ich bereit, bei 1–2 mittelgroßen Posten flexibel zu sein (z. B. B49 oder C41), solange wir unter dem Limit bleiben. Vorschlag folgt mit hohem Gesamtwert und Spielraum für dich.'}\\ \tt VORSCHLAG: {'C28', 'C48', 'A56', 'A93', 'C15', 'B66', 'C60'} 
	  } 
	   } 
	   } 
	 & \\ 
 

    \theutterance \stepcounter{utterance}  

    & & & \multicolumn{4}{p{0.6\linewidth}}{\cellcolor[rgb]{1,0.85,0.72}{%
	\makecell[{{p{\linewidth}}}]{% 
	  \tt {\tiny [GM$\langle$B]}  
	 ABLEHNUNG: {'C28', 'C48', 'A56', 'A93', 'C15', 'B66', 'C60'}\\ \tt ARGUMENT: {'Ich sehe, dass dir A56 und C48 wichtig sind. Für mich sind C28, A93 und B49 zentrale Eckpfeiler mit sehr starkem Verhältnis. Wenn wir C48 behalten, brauche ich im Gegenzug B49 zurück im Paket. Außerdem ist C28 für mich nicht verhandelbar. Ich biete dir dafür Flexibilität bei C41 oder A35 und lasse große Brocken wie B78, B10 und A75 draußen, um das Limit komfortabel zu halten.'}\\ \tt VORSCHLAG: {'C28', 'C48', 'A56', 'A93', 'C15', 'B66', 'B49'} 
	  } 
	   } 
	   } 
	  \\ 
 

    \theutterance \stepcounter{utterance}  

    & & \multicolumn{4}{p{0.6\linewidth}}{\cellcolor[rgb]{0.9,0.9,0.9}{%
	\makecell[{{p{\linewidth}}}]{% 
	  \tt {\tiny [A$\langle$GM]}  
	 ABLEHNUNG: {'C28', 'C48', 'A56', 'A93', 'C15', 'B66', 'C60'}\\ \tt ARGUMENT: {'Ich sehe, dass dir A56 und C48 wichtig sind. Für mich sind C28, A93 und B49 zentrale Eckpfeiler mit sehr starkem Verhältnis. Wenn wir C48 behalten, brauche ich im Gegenzug B49 zurück im Paket. Außerdem ist C28 für mich nicht verhandelbar. Ich biete dir dafür Flexibilität bei C41 oder A35 und lasse große Brocken wie B78, B10 und A75 draußen, um das Limit komfortabel zu halten.'}\\ \tt VORSCHLAG: {'C28', 'C48', 'A56', 'A93', 'C15', 'B66', 'B49'} 
	  } 
	   } 
	   } 
	 & \\ 
 

    \theutterance \stepcounter{utterance}  

    & \multicolumn{4}{p{0.6\linewidth}}{\cellcolor[rgb]{0.8,1,0.9}{%
	\makecell[{{p{\linewidth}}}]{% 
	  \tt {\tiny [A$\rangle$GM]}  
	 ABLEHNUNG: {'C28', 'C48', 'A56', 'A93', 'C15', 'B66', 'B49'}\\ \tt ARGUMENT: {'Einverstanden, C28 und B49 können wir integrieren, wenn der Gesamtnutzen hoch bleibt. Dafür brauche ich A56 sicher und idealerweise A93 oder C48 als zweites Schwergewicht. Um dir entgegenzukommen, streiche ich C60 und C41/A35 komplett. So maximieren wir starke Einheiten ohne Ballast und bleiben deutlich unter dem Limit.'}\\ \tt VORSCHLAG: {'C28', 'C48', 'A56', 'A93', 'B49'} 
	  } 
	   } 
	   } 
	 & & \\ 
 

    \theutterance \stepcounter{utterance}  

    & & \multicolumn{4}{p{0.6\linewidth}}{\cellcolor[rgb]{0.9,0.9,0.9}{%
	\makecell[{{p{\linewidth}}}]{% 
	  \tt {\tiny [GM$\rangle$B]}  
	 ABLEHNUNG: {'C28', 'C48', 'A56', 'A93', 'C15', 'B66', 'B49'}\\ \tt ARGUMENT: {'Einverstanden, C28 und B49 können wir integrieren, wenn der Gesamtnutzen hoch bleibt. Dafür brauche ich A56 sicher und idealerweise A93 oder C48 als zweites Schwergewicht. Um dir entgegenzukommen, streiche ich C60 und C41/A35 komplett. So maximieren wir starke Einheiten ohne Ballast und bleiben deutlich unter dem Limit.'}\\ \tt VORSCHLAG: {'C28', 'C48', 'A56', 'A93', 'B49'} 
	  } 
	   } 
	   } 
	 & \\ 
 

    \theutterance \stepcounter{utterance}  

    & & & \multicolumn{4}{p{0.6\linewidth}}{\cellcolor[rgb]{1,0.85,0.72}{%
	\makecell[{{p{\linewidth}}}]{% 
	  \tt {\tiny [GM$\langle$B]}  
	 ABLEHNUNG: {'C28', 'C48', 'A56', 'A93', 'B49'}\\ \tt ARGUMENT: {'Wir sind nah dran. Ich brauche C15 im Paket – sehr effizient – und kann dafür C48 oder A56 kompromissweise tauschen, aber nicht beide. Mein Favorit-Set, das unter dem Limit bleibt und deine Schwergewichte wahrt: C28, A93, A56, B49, C15. Wenn du stattdessen C48 über A56 priorisierst, kann ich auf A56 verzichten, solange C15 bleibt.'}\\ \tt VORSCHLAG: {'C28', 'A93', 'A56', 'B49', 'C15'} 
	  } 
	   } 
	   } 
	  \\ 
 

    \theutterance \stepcounter{utterance}  

    & & \multicolumn{4}{p{0.6\linewidth}}{\cellcolor[rgb]{0.9,0.9,0.9}{%
	\makecell[{{p{\linewidth}}}]{% 
	  \tt {\tiny [A$\langle$GM]}  
	 ABLEHNUNG: {'C28', 'C48', 'A56', 'A93', 'B49'}\\ \tt ARGUMENT: {'Wir sind nah dran. Ich brauche C15 im Paket – sehr effizient – und kann dafür C48 oder A56 kompromissweise tauschen, aber nicht beide. Mein Favorit-Set, das unter dem Limit bleibt und deine Schwergewichte wahrt: C28, A93, A56, B49, C15. Wenn du stattdessen C48 über A56 priorisierst, kann ich auf A56 verzichten, solange C15 bleibt.'}\\ \tt VORSCHLAG: {'C28', 'A93', 'A56', 'B49', 'C15'} 
	  } 
	   } 
	   } 
	 & \\ 
 

    \theutterance \stepcounter{utterance}  

    & \multicolumn{4}{p{0.6\linewidth}}{\cellcolor[rgb]{0.8,1,0.9}{%
	\makecell[{{p{\linewidth}}}]{% 
	  \tt {\tiny [A$\rangle$GM]}  
	 ABLEHNUNG: {'C28', 'A93', 'A56', 'B49', 'C15'}\\ \tt ARGUMENT: {'Fairer Vorschlag. Ich akzeptiere C15 fix. Für mich ist entweder A56 oder C48 als zweites Schwergewicht zwingend – ich bevorzuge C48, kann aber notfalls bei A56 bleiben. Lass uns die starke, effiziente Variante mit C48 nehmen und alles andere so lassen: Das liefert dir C28, A93, B49, C15 und gibt mir mein Top-Item zurück, bleibt klar unter dem Limit und verzichtet auf überteuerte Brocken.'}\\ \tt VORSCHLAG: {'C28', 'A93', 'B49', 'C15', 'C48'} 
	  } 
	   } 
	   } 
	 & & \\ 
 

    \theutterance \stepcounter{utterance}  

    & & \multicolumn{4}{p{0.6\linewidth}}{\cellcolor[rgb]{0.9,0.9,0.9}{%
	\makecell[{{p{\linewidth}}}]{% 
	  \tt {\tiny [GM$\rangle$B]}  
	 ABLEHNUNG: {'C28', 'A93', 'A56', 'B49', 'C15'}\\ \tt ARGUMENT: {'Fairer Vorschlag. Ich akzeptiere C15 fix. Für mich ist entweder A56 oder C48 als zweites Schwergewicht zwingend – ich bevorzuge C48, kann aber notfalls bei A56 bleiben. Lass uns die starke, effiziente Variante mit C48 nehmen und alles andere so lassen: Das liefert dir C28, A93, B49, C15 und gibt mir mein Top-Item zurück, bleibt klar unter dem Limit und verzichtet auf überteuerte Brocken.'}\\ \tt VORSCHLAG: {'C28', 'A93', 'B49', 'C15', 'C48'} 
	  } 
	   } 
	   } 
	 & \\ 
 

    \theutterance \stepcounter{utterance}  

    & & & \multicolumn{4}{p{0.6\linewidth}}{\cellcolor[rgb]{1,0.85,0.72}{%
	\makecell[{{p{\linewidth}}}]{% 
	  \tt {\tiny [GM$\langle$B]}  
	 ARGUMENT: {'Dein Paket ist nah an einem Deal, aber ich brauche A56 drin – es ist für mich ein Kernstück. Wir können dafür auf ein anderes Item verzichten und bleiben locker unter dem Limit. Ich ersetze daher C48 durch A56. Das hält deinen gewünschten Kern (C28, A93, B49, C15) intakt und gibt mir mein zentrales Stück.'}\\ \tt VORSCHLAG: {'C28', 'A93', 'B49', 'C15', 'A56'} 
	  } 
	   } 
	   } 
	  \\ 
 

    \theutterance \stepcounter{utterance}  

    & & \multicolumn{4}{p{0.6\linewidth}}{\cellcolor[rgb]{0.9,0.9,0.9}{%
	\makecell[{{p{\linewidth}}}]{% 
	  \tt {\tiny [A$\langle$GM]}  
	 ARGUMENT: {'Dein Paket ist nah an einem Deal, aber ich brauche A56 drin – es ist für mich ein Kernstück. Wir können dafür auf ein anderes Item verzichten und bleiben locker unter dem Limit. Ich ersetze daher C48 durch A56. Das hält deinen gewünschten Kern (C28, A93, B49, C15) intakt und gibt mir mein zentrales Stück.'}\\ \tt VORSCHLAG: {'C28', 'A93', 'B49', 'C15', 'A56'} 
	  } 
	   } 
	   } 
	 & \\ 
 

    \theutterance \stepcounter{utterance}  

    & \multicolumn{4}{p{0.6\linewidth}}{\cellcolor[rgb]{0.8,1,0.9}{%
	\makecell[{{p{\linewidth}}}]{% 
	  \tt {\tiny [A$\rangle$GM]}  
	 ZUSTIMMUNG: {'C28', 'A93', 'B49', 'C15', 'A56'}\\ \tt ARGUMENT: {'Einverstanden. Dieses Set hält das Limit komfortabel ein und trifft die von uns beiden priorisierten Kernstücke – starke Balance aus großen Ankern und effizienten Ergänzungen.'} 
	  } 
	   } 
	   } 
	 & & \\ 
 

    \theutterance \stepcounter{utterance}  

    & & & \multicolumn{2}{p{0.3\linewidth}}{\cellcolor[rgb]{0.95,0.95,0.95}{%
	\makecell[{{p{\linewidth}}}]{% 
	  \tt {\tiny [GM$|$GM]}  
	 game successful 
	  } 
	   } 
	   } 
	 & & \\ 
 

    \theutterance \stepcounter{utterance}  

    & & & \multicolumn{2}{p{0.3\linewidth}}{\cellcolor[rgb]{0.95,0.95,0.95}{%
	\makecell[{{p{\linewidth}}}]{% 
	  \tt {\tiny [GM$|$GM]}  
	 end game 
	  } 
	   } 
	   } 
	 & & \\ 
 

\end{supertabular}
}

\end{document}
