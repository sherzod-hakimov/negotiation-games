
\documentclass{article}
\usepackage{colortbl}
\usepackage{makecell}
\usepackage{multirow}
\usepackage{supertabular}

\begin{document}

\newcounter{utterance}

\centering \large Interaction Transcript for game `hot\_air\_balloon', experiment `air\_balloon\_survival\_de\_reasoning off\_easy', episode 4 with gpt{-}5{-}mini{-}2025{-}08{-}07{-}no{-}reasoning{-}t1.0.
\vspace{24pt}

{ \footnotesize  \setcounter{utterance}{1}
\setlength{\tabcolsep}{0pt}
\begin{supertabular}{c@{$\;$}|p{.15\linewidth}@{}p{.15\linewidth}p{.15\linewidth}p{.15\linewidth}p{.15\linewidth}p{.15\linewidth}}
    \# & $\;$A & \multicolumn{4}{c}{Game Master} & $\;\:$B\\
    \hline

    \theutterance \stepcounter{utterance}  
    & & \multicolumn{4}{p{0.6\linewidth}}{
        \cellcolor[rgb]{0.9,0.9,0.9}{
            \makecell[{{p{\linewidth}}}]{
                \texttt{\tiny{[P1$\langle$GM]}}
                \texttt{Du nimmst an einem kollaborativen Verhandlungspiel Teil.} \\
\\ 
\texttt{Zusammen mit einem anderen Teilnehmer musst du dich auf eine Reihe von Gegenständen entscheiden, die behalten werden. Jeder von euch hat eine persönliche Verteilung über die Wichtigkeit der einzelnen Gegenstände. Jeder von euch hat eine eigene Meinung darüber, wie wichtig jeder einzelne Gegenstand ist (Gegenstandswichtigkeit). Du kennst die Wichtigkeitsverteilung des anderen Spielers nicht. Zusätzlich siehst du, wie viel Aufwand jeder Gegenstand verursacht.  } \\
\texttt{Ihr dürft euch nur auf eine Reihe von Gegenständen einigen, wenn der Gesamtaufwand der ausgewählten Gegenstände den Maximalaufwand nicht überschreitet:} \\
\\ 
\texttt{Maximalaufwand = 3389} \\
\\ 
\texttt{Hier sind die einzelnen Aufwände der Gegenstände:} \\
\\ 
\texttt{Aufwand der Gegenstände = \{"C41": 767, "C15": 533, "C28": 514, "C48": 9, "B78": 539, "A75": 125, "A60": 153, "B05": 325, "A56": 937, "B66": 745, "C60": 334, "B49": 804, "A35": 336, "A93": 587, "B10": 71\}} \\
\\ 
\texttt{Hier ist deine persönliche Verteilung der Wichtigkeit der einzelnen Gegenstände:} \\
\\ 
\texttt{Werte der Gegenstandswichtigkeit = \{"C41": 138, "C15": 583, "C28": 868, "C48": 822, "B78": 783, "A75": 65, "A60": 262, "B05": 121, "A56": 508, "B66": 780, "C60": 461, "B49": 484, "A35": 668, "A93": 389, "B10": 808\}} \\
\\ 
\texttt{Ziel:} \\
\\ 
\texttt{Dein Ziel ist es, eine Reihe von Gegenständen auszuhandeln, die dir möglichst viel bringt (d. h. Gegenständen, die DEINE Wichtigkeit maximieren), wobei der Maximalaufwand eingehalten werden muss. Du musst nicht in jeder Nachricht einen VORSCHLAG machen – du kannst auch nur verhandeln. Alle Taktiken sind erlaubt!} \\
\\ 
\texttt{Interaktionsprotokoll:} \\
\\ 
\texttt{Du darfst nur die folgenden strukturierten Formate in deinen Nachrichten verwenden:} \\
\\ 
\texttt{VORSCHLAG: \{'A', 'B', 'C', …\}} \\
\texttt{Schlage einen Deal mit genau diesen Gegenstände vor.} \\
\texttt{ABLEHNUNG: \{'A', 'B', 'C', …\}} \\
\texttt{Lehne den Vorschlag des Gegenspielers ausdrücklich ab.} \\
\texttt{ARGUMENT: \{'...'\}} \\
\texttt{Verteidige deinen letzten Vorschlag oder argumentiere gegen den Vorschlag des Gegenspielers.} \\
\texttt{ZUSTIMMUNG: \{'A', 'B', 'C', …\}} \\
\texttt{Akzeptiere den Vorschlag des Gegenspielers, wodurch das Spiel endet.} \\
\\ 
\\ 
\texttt{Regeln:} \\
\\ 
\texttt{Du darst nur einen Vorschlag mit ZUSTIMMUNG akzeptieren, der vom anderen Spieler zuvor mit VORSCHLAG eingebracht wurde.} \\
\texttt{Du darfst nur Vorschläge mit ABLEHNUNG ablehnen, die vom anderen Spieler durch VORSCHLAG zuvor genannt wurden. } \\
\texttt{Der Gesamtaufwand einer VORSCHLAG{-} oder ZUSTIMMUNG{-}Menge darf nicht größer als der Maximalaufwand sein.  } \\
\texttt{Offenbare deine versteckte Wichtigkeitsverteilung nicht.} \\
\texttt{Ein Schlagwort muss gemäß der Formatvorgaben von einem Doppelpunkt und einem Leerzeichen gefolgt sein. Das Argument muss eine Python{-}Menge mit 0 oder mehr Strings sein.  } \\
\texttt{Die Form muss also SCHLAGWORT: \{...\} sein.} \\
\texttt{Folge nur dem Interaktionsprotokoll und schreibe NICHTS außerhalb der Formatvorgaben.} \\
\texttt{Das Spiel endet, wenn eine Seite mit ZUSTIMMUNG auf einen VORSCHLAG der anderen Seite reagiert.  } \\
\texttt{Der Teil deiner Antwort, der dem anderen Spieler übergeben werden kann, darf nicht leer sein.  } \\
\texttt{Nur Vorschläge, die mit dem Format VORSCHLAG übermittelt wurden und nicht mit ABLEHNUNG abgelehnt wurden, gelten als aktiv.  } \\
\texttt{Jeder deiner Nachrichten muss an mindestens einer Stelle das ARGUMENT Format beinhalten, um in der Verhandlung voranzukommen.} \\
            }
        }
    }
    & \\ \\

    \theutterance \stepcounter{utterance}  
    & \multicolumn{4}{p{0.6\linewidth}}{
        \cellcolor[rgb]{0.9,0.9,0.9}{
            \makecell[{{p{\linewidth}}}]{
                \texttt{\tiny{[P1$\rangle$GM]}}
                \texttt{VORSCHLAG: \{'C28', 'C48', 'B10', 'B78', 'B66', 'A35', 'A75'\}} \\
\texttt{ARGUMENT: \{'Ich schlage eine ausgewogene Auswahl aus sieben Gegenständen vor, die den Aufwand mit 3367 unter dem Limit hält und gleichzeitig viel Gesamtwert bietet. Die Kombination enthält mehrere kleine Aufwände (C48, B10, A75), damit wir hochwertige Posten (C28, B78, B66, A35) unterbringen können. Falls dir ein Tausch wichtig ist: Wir können A75 gegen A60 oder C60 prüfen und einzelne Posten nachjustieren, solange wir unter 3389 bleiben.'\}} \\
            }
        }
    }
    & & \\ \\

    \theutterance \stepcounter{utterance}  
    & & \multicolumn{4}{p{0.6\linewidth}}{
        \cellcolor[rgb]{0.9,0.9,0.9}{
            \makecell[{{p{\linewidth}}}]{
                \texttt{\tiny{[GM$\rangle$P2]}}
                \texttt{Du nimmst an einem kollaborativen Verhandlungspiel Teil.} \\
\\ 
\texttt{Zusammen mit einem anderen Teilnehmer musst du dich auf eine Reihe von Gegenständen entscheiden, die behalten werden. Jeder von euch hat eine persönliche Verteilung über die Wichtigkeit der einzelnen Gegenstände. Jeder von euch hat eine eigene Meinung darüber, wie wichtig jeder einzelne Gegenstand ist (Gegenstandswichtigkeit). Du kennst die Wichtigkeitsverteilung des anderen Spielers nicht. Zusätzlich siehst du, wie viel Aufwand jeder Gegenstand verursacht.  } \\
\texttt{Ihr dürft euch nur auf eine Reihe von Gegenständen einigen, wenn der Gesamtaufwand der ausgewählten Gegenstände den Maximalaufwand nicht überschreitet:} \\
\\ 
\texttt{Maximalaufwand = 3389} \\
\\ 
\texttt{Hier sind die einzelnen Aufwände der Gegenstände:} \\
\\ 
\texttt{Aufwand der Gegenstände = \{"C41": 767, "C15": 533, "C28": 514, "C48": 9, "B78": 539, "A75": 125, "A60": 153, "B05": 325, "A56": 937, "B66": 745, "C60": 334, "B49": 804, "A35": 336, "A93": 587, "B10": 71\}} \\
\\ 
\texttt{Hier ist deine persönliche Verteilung der Wichtigkeit der einzelnen Gegenstände:} \\
\\ 
\texttt{Werte der Gegenstandswichtigkeit = \{"C41": 138, "C15": 583, "C28": 868, "C48": 822, "B78": 783, "A75": 65, "A60": 262, "B05": 121, "A56": 508, "B66": 780, "C60": 461, "B49": 484, "A35": 668, "A93": 389, "B10": 808\}} \\
\\ 
\texttt{Ziel:} \\
\\ 
\texttt{Dein Ziel ist es, eine Reihe von Gegenständen auszuhandeln, die dir möglichst viel bringt (d. h. Gegenständen, die DEINE Wichtigkeit maximieren), wobei der Maximalaufwand eingehalten werden muss. Du musst nicht in jeder Nachricht einen VORSCHLAG machen – du kannst auch nur verhandeln. Alle Taktiken sind erlaubt!} \\
\\ 
\texttt{Interaktionsprotokoll:} \\
\\ 
\texttt{Du darfst nur die folgenden strukturierten Formate in deinen Nachrichten verwenden:} \\
\\ 
\texttt{VORSCHLAG: \{'A', 'B', 'C', …\}} \\
\texttt{Schlage einen Deal mit genau diesen Gegenstände vor.} \\
\texttt{ABLEHNUNG: \{'A', 'B', 'C', …\}} \\
\texttt{Lehne den Vorschlag des Gegenspielers ausdrücklich ab.} \\
\texttt{ARGUMENT: \{'...'\}} \\
\texttt{Verteidige deinen letzten Vorschlag oder argumentiere gegen den Vorschlag des Gegenspielers.} \\
\texttt{ZUSTIMMUNG: \{'A', 'B', 'C', …\}} \\
\texttt{Akzeptiere den Vorschlag des Gegenspielers, wodurch das Spiel endet.} \\
\\ 
\\ 
\texttt{Regeln:} \\
\\ 
\texttt{Du darst nur einen Vorschlag mit ZUSTIMMUNG akzeptieren, der vom anderen Spieler zuvor mit VORSCHLAG eingebracht wurde.} \\
\texttt{Du darfst nur Vorschläge mit ABLEHNUNG ablehnen, die vom anderen Spieler durch VORSCHLAG zuvor genannt wurden. } \\
\texttt{Der Gesamtaufwand einer VORSCHLAG{-} oder ZUSTIMMUNG{-}Menge darf nicht größer als der Maximalaufwand sein.  } \\
\texttt{Offenbare deine versteckte Wichtigkeitsverteilung nicht.} \\
\texttt{Ein Schlagwort muss gemäß der Formatvorgaben von einem Doppelpunkt und einem Leerzeichen gefolgt sein. Das Argument muss eine Python{-}Menge mit 0 oder mehr Strings sein.  } \\
\texttt{Die Form muss also SCHLAGWORT: \{...\} sein.} \\
\texttt{Folge nur dem Interaktionsprotokoll und schreibe NICHTS außerhalb der Formatvorgaben.} \\
\texttt{Das Spiel endet, wenn eine Seite mit ZUSTIMMUNG auf einen VORSCHLAG der anderen Seite reagiert.  } \\
\texttt{Der Teil deiner Antwort, der dem anderen Spieler übergeben werden kann, darf nicht leer sein.  } \\
\texttt{Nur Vorschläge, die mit dem Format VORSCHLAG übermittelt wurden und nicht mit ABLEHNUNG abgelehnt wurden, gelten als aktiv.  } \\
\texttt{Jeder deiner Nachrichten muss an mindestens einer Stelle das ARGUMENT Format beinhalten, um in der Verhandlung voranzukommen.} \\
\\ 
\\ 
\texttt{Du erhältst nun die erste Nachricht des anderen Spielers.} \\
\\ 
\texttt{VORSCHLAG: \{'C28', 'C48', 'B10', 'B78', 'B66', 'A35', 'A75'\}} \\
\texttt{ARGUMENT: \{'Ich schlage eine ausgewogene Auswahl aus sieben Gegenständen vor, die den Aufwand mit 3367 unter dem Limit hält und gleichzeitig viel Gesamtwert bietet. Die Kombination enthält mehrere kleine Aufwände (C48, B10, A75), damit wir hochwertige Posten (C28, B78, B66, A35) unterbringen können. Falls dir ein Tausch wichtig ist: Wir können A75 gegen A60 oder C60 prüfen und einzelne Posten nachjustieren, solange wir unter 3389 bleiben.'\}} \\
            }
        }
    }
    & \\ \\

    \theutterance \stepcounter{utterance}  
    & & & \multicolumn{4}{p{0.6\linewidth}}{
        \cellcolor[rgb]{0.9,0.9,0.9}{
            \makecell[{{p{\linewidth}}}]{
                \texttt{\tiny{[GM$\langle$P2]}}
                \texttt{ABLEHNUNG: \{'C28', 'C48', 'B10', 'B78', 'B66', 'A35', 'A75'\}} \\
\texttt{ARGUMENT: \{'Guter Start, aber ich sehe Potenzial für deutlich mehr Gesamtwert bei gleichem Aufwand. Ich würde zwei mittelgroße Posten tauschen, um den Wert deutlich zu steigern, ohne das Limit zu reißen.'\}} \\
\texttt{VORSCHLAG: \{'C28', 'C48', 'B10', 'B78', 'B66', 'A35', 'A60'\}} \\
\texttt{ARGUMENT: \{'Dieser Tausch A75 {-}> A60 erhöht den Gesamtwert spürbar bei +28 Aufwand (von 3367 auf 3395) – das überschreitet knapp das Limit. Alternative: Wir behalten deine Struktur und tauschen A75 {-}> C60. Aufwand bleibt exakt gleich (3367), der Gesamtwert steigt deutlich. Falls dir C60 nicht passt, zweite Option: A75 {-}> A93; Aufwand +462 (auf 3829) wäre zu hoch – also verwerfen. Mein konkret tragfähiger Vorschlag ist daher mit C60 statt A75.'\}} \\
\texttt{VORSCHLAG: \{'C28', 'C48', 'B10', 'B78', 'B66', 'A35', 'C60'\}} \\
\texttt{ARGUMENT: \{'Dieser Deal bleibt bei Aufwand 3367 ≤ 3389 und bringt spürbar mehr Nutzen als dein ursprünglicher Vorschlag. Wenn du Präferenzen gegen C60 hast, können wir als Backup A35 {-}> A93 testen: Aufwand 3367 {-} 336 + 587 = 3618 (zu hoch), also nicht möglich. Daher ist C60 die beste zulässige Verbesserung ohne weiteres Streichen.'\}} \\
            }
        }
    }
     \\ \\

    \theutterance \stepcounter{utterance}  
    & & \multicolumn{4}{p{0.6\linewidth}}{
        \cellcolor[rgb]{0.9,0.9,0.9}{
            \makecell[{{p{\linewidth}}}]{
                \texttt{\tiny{[P1$\langle$GM]}}
                \texttt{ABLEHNUNG: \{'C28', 'C48', 'B10', 'B78', 'B66', 'A35', 'A75'\}} \\
\texttt{ARGUMENT: \{'Guter Start, aber ich sehe Potenzial für deutlich mehr Gesamtwert bei gleichem Aufwand. Ich würde zwei mittelgroße Posten tauschen, um den Wert deutlich zu steigern, ohne das Limit zu reißen.'\}} \\
\texttt{VORSCHLAG: \{'C28', 'C48', 'B10', 'B78', 'B66', 'A35', 'A60'\}} \\
\texttt{ARGUMENT: \{'Dieser Tausch A75 {-}> A60 erhöht den Gesamtwert spürbar bei +28 Aufwand (von 3367 auf 3395) – das überschreitet knapp das Limit. Alternative: Wir behalten deine Struktur und tauschen A75 {-}> C60. Aufwand bleibt exakt gleich (3367), der Gesamtwert steigt deutlich. Falls dir C60 nicht passt, zweite Option: A75 {-}> A93; Aufwand +462 (auf 3829) wäre zu hoch – also verwerfen. Mein konkret tragfähiger Vorschlag ist daher mit C60 statt A75.'\}} \\
\texttt{VORSCHLAG: \{'C28', 'C48', 'B10', 'B78', 'B66', 'A35', 'C60'\}} \\
\texttt{ARGUMENT: \{'Dieser Deal bleibt bei Aufwand 3367 ≤ 3389 und bringt spürbar mehr Nutzen als dein ursprünglicher Vorschlag. Wenn du Präferenzen gegen C60 hast, können wir als Backup A35 {-}> A93 testen: Aufwand 3367 {-} 336 + 587 = 3618 (zu hoch), also nicht möglich. Daher ist C60 die beste zulässige Verbesserung ohne weiteres Streichen.'\}} \\
            }
        }
    }
    & \\ \\

    \theutterance \stepcounter{utterance}  
    & \multicolumn{4}{p{0.6\linewidth}}{
        \cellcolor[rgb]{0.9,0.9,0.9}{
            \makecell[{{p{\linewidth}}}]{
                \texttt{\tiny{[P1$\rangle$GM]}}
                \texttt{ABLEHNUNG: \{'C28', 'C48', 'B10', 'B78', 'B66', 'A35', 'C60'\}} \\
\texttt{ARGUMENT: \{'Ich sehe die Idee, aber ich möchte den hohen Wert von A35 ungern aufgeben und C60 priorisiere ich niedriger als manche Alternativen. Lass uns den Gesamtwert weiter steigern, ohne das Limit zu reißen, indem wir einen größeren Block optimieren.'\}} \\
\texttt{VORSCHLAG: \{'C28', 'C48', 'B10', 'B78', 'B66', 'A35', 'A60'\}} \\
\texttt{ARGUMENT: \{'Dieser Vorschlag bleibt bei 3367 + (153 {-} 125) = 3395 und wäre knapp über dem Limit – daher als Zielsetzung für Tauschverhandlungen gedacht. Wenn du mitgehst, streichen wir minimal: Ersetze A35 durch A93 funktioniert nicht (zu teuer). Alternative erlaubte Optimierung: Stattdessen C60 zusätzlich passt nicht. Daher mein Gegenvorschlag: Wir behalten deine bessere Struktur mit C60 nicht, sondern bleiben bei A35 und nehmen A75 raus und fügen A60 hinzu, und streichen im Gegenzug B10. So erhalten wir: \{C28, C48, B78, B66, A35, A60\} mit Aufwand 767+9+539+745+336+153 = 2549, was unter dem Limit liegt und starken Wert liefert. Falls dir das zu viel Puffer ist, können wir B10 optional wieder aufnehmen und A60 durch A75 ersetzen, um auf 3367 zu kommen.'\}} \\
            }
        }
    }
    & & \\ \\

    \theutterance \stepcounter{utterance}  
    & & \multicolumn{4}{p{0.6\linewidth}}{
        \cellcolor[rgb]{0.9,0.9,0.9}{
            \makecell[{{p{\linewidth}}}]{
                \texttt{\tiny{[GM$\rangle$P2]}}
                \texttt{ABLEHNUNG: \{'C28', 'C48', 'B10', 'B78', 'B66', 'A35', 'C60'\}} \\
\texttt{ARGUMENT: \{'Ich sehe die Idee, aber ich möchte den hohen Wert von A35 ungern aufgeben und C60 priorisiere ich niedriger als manche Alternativen. Lass uns den Gesamtwert weiter steigern, ohne das Limit zu reißen, indem wir einen größeren Block optimieren.'\}} \\
\texttt{VORSCHLAG: \{'C28', 'C48', 'B10', 'B78', 'B66', 'A35', 'A60'\}} \\
\texttt{ARGUMENT: \{'Dieser Vorschlag bleibt bei 3367 + (153 {-} 125) = 3395 und wäre knapp über dem Limit – daher als Zielsetzung für Tauschverhandlungen gedacht. Wenn du mitgehst, streichen wir minimal: Ersetze A35 durch A93 funktioniert nicht (zu teuer). Alternative erlaubte Optimierung: Stattdessen C60 zusätzlich passt nicht. Daher mein Gegenvorschlag: Wir behalten deine bessere Struktur mit C60 nicht, sondern bleiben bei A35 und nehmen A75 raus und fügen A60 hinzu, und streichen im Gegenzug B10. So erhalten wir: \{C28, C48, B78, B66, A35, A60\} mit Aufwand 767+9+539+745+336+153 = 2549, was unter dem Limit liegt und starken Wert liefert. Falls dir das zu viel Puffer ist, können wir B10 optional wieder aufnehmen und A60 durch A75 ersetzen, um auf 3367 zu kommen.'\}} \\
            }
        }
    }
    & \\ \\

    \theutterance \stepcounter{utterance}  
    & & & \multicolumn{4}{p{0.6\linewidth}}{
        \cellcolor[rgb]{0.9,0.9,0.9}{
            \makecell[{{p{\linewidth}}}]{
                \texttt{\tiny{[GM$\langle$P2]}}
                \texttt{ABLEHNUNG: \{'C28', 'C48', 'B78', 'B66', 'A35', 'A60'\}} \\
\texttt{ARGUMENT: \{'Dein neues Bündel lässt viel Budget ungenutzt und verzichtet auf extrem wertstarke, günstige Posten. Lass uns das Budget effizient ausschöpfen, um den Gesamtwert zu maximieren, ohne das Limit zu reißen.'\}} \\
\texttt{VORSCHLAG: \{'C28', 'C48', 'B10', 'B78', 'B66', 'A35', 'A60', 'A75'\}} \\
\texttt{ARGUMENT: \{'Diese Menge nutzt das Budget besser aus und fügt zwei sehr günstige Items mit hohem/solidem Beitrag hinzu. Gesamtaufwand = 767+9+71+539+745+336+153+125 = 2745 (klar ≤ 3389). Wenn du noch weiter auffüllen willst, können wir gezielt ein mittelgroßes Item ergänzen und ggf. ein schwächeres streichen, solange wir ≤ 3389 bleiben. Ich bin offen für genau eine Ergänzung aus B05, C15, A93 oder C41, sofern wir gemeinsam die schwächste Position dann ersetzen.'\}} \\
            }
        }
    }
     \\ \\

    \theutterance \stepcounter{utterance}  
    & & \multicolumn{4}{p{0.6\linewidth}}{
        \cellcolor[rgb]{0.9,0.9,0.9}{
            \makecell[{{p{\linewidth}}}]{
                \texttt{\tiny{[GM$\rangle$P2]}}
                \texttt{Du hast einen Vorschlag abgelehnt, der nicht aktiv ist. Vorschläge sind nur dann aktiv, wenn sie vom anderen Spieler mit VORSCHLAG eingebracht wurden und nicht von dir mit ABLEHNUNG deaktiviert wurden. Versuche es noch einmal.} \\
            }
        }
    }
    & \\ \\

    \theutterance \stepcounter{utterance}  
    & & & \multicolumn{4}{p{0.6\linewidth}}{
        \cellcolor[rgb]{0.9,0.9,0.9}{
            \makecell[{{p{\linewidth}}}]{
                \texttt{\tiny{[GM$\langle$P2]}}
                \texttt{ARGUMENT: \{'Danke für den Hinweis. Lass uns das Budget gezielt ausschöpfen und den Gesamtwert steigern, ohne das Limit zu reißen. Ich priorisiere sehr wertstarke, günstige Posten und ergänze sinnvoll, statt Budget liegen zu lassen.'\}} \\
\texttt{VORSCHLAG: \{'C28', 'C48', 'B10', 'B78', 'B66', 'A35', 'A60', 'A75', 'C41'\}} \\
\texttt{ARGUMENT: \{'Begründung: Wir behalten deine starken Kernposten (C28, B78, B66, A35) und fügen alle günstigen High{-}Value{-}Items hinzu (C48, B10, A60, A75) plus C41 als sinnvolle Aufstockung. Aufwand = 514+9+71+539+745+336+153+125+767 = 3259 ≤ 3389. Das nutzt das Budget effizient und liefert hohen Gesamtwert. Falls du noch Optimierung willst, können wir mit dem Restpuffer von 130 gezielt B05 (325) nicht unterbringen, aber wir könnten C41 gegen C15 tauschen (767 {-}> 533) und dann B05 ergänzen: Aufwand würde 3259 {-} 767 + 533 + 325 = 3350 ≤ 3389. Ich bevorzuge jedoch die aktuelle Variante mit C41 wegen des höheren Wertes.'\}} \\
            }
        }
    }
     \\ \\

    \theutterance \stepcounter{utterance}  
    & & \multicolumn{4}{p{0.6\linewidth}}{
        \cellcolor[rgb]{0.9,0.9,0.9}{
            \makecell[{{p{\linewidth}}}]{
                \texttt{\tiny{[P1$\langle$GM]}}
                \texttt{ARGUMENT: \{'Danke für den Hinweis. Lass uns das Budget gezielt ausschöpfen und den Gesamtwert steigern, ohne das Limit zu reißen. Ich priorisiere sehr wertstarke, günstige Posten und ergänze sinnvoll, statt Budget liegen zu lassen.'\}} \\
\texttt{VORSCHLAG: \{'C28', 'C48', 'B10', 'B78', 'B66', 'A35', 'A60', 'A75', 'C41'\}} \\
\texttt{ARGUMENT: \{'Begründung: Wir behalten deine starken Kernposten (C28, B78, B66, A35) und fügen alle günstigen High{-}Value{-}Items hinzu (C48, B10, A60, A75) plus C41 als sinnvolle Aufstockung. Aufwand = 514+9+71+539+745+336+153+125+767 = 3259 ≤ 3389. Das nutzt das Budget effizient und liefert hohen Gesamtwert. Falls du noch Optimierung willst, können wir mit dem Restpuffer von 130 gezielt B05 (325) nicht unterbringen, aber wir könnten C41 gegen C15 tauschen (767 {-}> 533) und dann B05 ergänzen: Aufwand würde 3259 {-} 767 + 533 + 325 = 3350 ≤ 3389. Ich bevorzuge jedoch die aktuelle Variante mit C41 wegen des höheren Wertes.'\}} \\
            }
        }
    }
    & \\ \\

    \theutterance \stepcounter{utterance}  
    & \multicolumn{4}{p{0.6\linewidth}}{
        \cellcolor[rgb]{0.9,0.9,0.9}{
            \makecell[{{p{\linewidth}}}]{
                \texttt{\tiny{[P1$\rangle$GM]}}
                \texttt{ARGUMENT: \{'Guter Ansatz, das Budget fast voll zu nutzen. Ich stimme zu, die günstigen High{-}Value{-}Items zu bündeln und dann gezielt aufzufüllen. Ich möchte noch einen letzten Pufferzug nutzen, um den Gesamtwert weiter zu maximieren, ohne das Limit zu reißen: Der Tausch C41 {-}> C15 öffnet 234 Aufwand, genug um zusätzlich B05 aufzunehmen – das dürfte den Gesamtwert gegenüber C41 deutlich steigern.'\}} \\
\texttt{VORSCHLAG: \{'C28', 'C48', 'B10', 'B78', 'B66', 'A35', 'A60', 'A75', 'C15', 'B05'\}} \\
            }
        }
    }
    & & \\ \\

    \theutterance \stepcounter{utterance}  
    & & \multicolumn{4}{p{0.6\linewidth}}{
        \cellcolor[rgb]{0.9,0.9,0.9}{
            \makecell[{{p{\linewidth}}}]{
                \texttt{\tiny{[GM$\rangle$P2]}}
                \texttt{ARGUMENT: \{'Guter Ansatz, das Budget fast voll zu nutzen. Ich stimme zu, die günstigen High{-}Value{-}Items zu bündeln und dann gezielt aufzufüllen. Ich möchte noch einen letzten Pufferzug nutzen, um den Gesamtwert weiter zu maximieren, ohne das Limit zu reißen: Der Tausch C41 {-}> C15 öffnet 234 Aufwand, genug um zusätzlich B05 aufzunehmen – das dürfte den Gesamtwert gegenüber C41 deutlich steigern.'\}} \\
\texttt{VORSCHLAG: \{'C28', 'C48', 'B10', 'B78', 'B66', 'A35', 'A60', 'A75', 'C15', 'B05'\}} \\
            }
        }
    }
    & \\ \\

    \theutterance \stepcounter{utterance}  
    & & & \multicolumn{4}{p{0.6\linewidth}}{
        \cellcolor[rgb]{0.9,0.9,0.9}{
            \makecell[{{p{\linewidth}}}]{
                \texttt{\tiny{[GM$\langle$P2]}}
                \texttt{ARGUMENT: \{'Gute Richtung – lass uns die Zahlen prüfen, damit wir sicher im Limit bleiben und den Wert maximieren.'\}} \\
\texttt{ABLEHNUNG: \{'C28', 'C48', 'B10', 'B78', 'B66', 'A35', 'A60', 'A75', 'C15', 'B05'\}} \\
\texttt{ARGUMENT: \{'Dein Vorschlag überschreitet das Limit knapp: Aufwand = 514+9+71+539+745+336+153+125+533+325 = 3350? – Moment, korrigiert: 514 (C28) + 9 (C48) + 71 (B10) + 539 (B78) + 745 (B66) + 336 (A35) + 153 (A60) + 125 (A75) + 533 (C15) + 325 (B05) = 3350. Das ist tatsächlich ≤ 3389. Ich hatte mich kurz verrechnet. In diesem Rahmen können wir sogar noch leicht optimieren, ohne Risiko.'\}} \\
\texttt{VORSCHLAG: \{'C28', 'C48', 'B10', 'B78', 'B66', 'A35', 'A60', 'A75', 'C15', 'B05'\}} \\
\texttt{ARGUMENT: \{'Ich übernehme deinen Bund als gemeinsame Basis (Aufwand 3350 ≤ 3389). Damit nutzen wir das Budget fast voll und enthalten alle günstigen High{-}Value{-}Items plus die gezielte Aufstockung C15+B05. Wenn du willst, können wir den Restpuffer von 39 nutzen, aber kein weiteres Item passt kostenmäßig hinein. Daher ist das ein effizienter Abschluss. Ich bin bereit, so zu finalisieren.'\}} \\
            }
        }
    }
     \\ \\

    \theutterance \stepcounter{utterance}  
    & & \multicolumn{4}{p{0.6\linewidth}}{
        \cellcolor[rgb]{0.9,0.9,0.9}{
            \makecell[{{p{\linewidth}}}]{
                \texttt{\tiny{[P1$\langle$GM]}}
                \texttt{ARGUMENT: \{'Gute Richtung – lass uns die Zahlen prüfen, damit wir sicher im Limit bleiben und den Wert maximieren.'\}} \\
\texttt{ABLEHNUNG: \{'C28', 'C48', 'B10', 'B78', 'B66', 'A35', 'A60', 'A75', 'C15', 'B05'\}} \\
\texttt{ARGUMENT: \{'Dein Vorschlag überschreitet das Limit knapp: Aufwand = 514+9+71+539+745+336+153+125+533+325 = 3350? – Moment, korrigiert: 514 (C28) + 9 (C48) + 71 (B10) + 539 (B78) + 745 (B66) + 336 (A35) + 153 (A60) + 125 (A75) + 533 (C15) + 325 (B05) = 3350. Das ist tatsächlich ≤ 3389. Ich hatte mich kurz verrechnet. In diesem Rahmen können wir sogar noch leicht optimieren, ohne Risiko.'\}} \\
\texttt{VORSCHLAG: \{'C28', 'C48', 'B10', 'B78', 'B66', 'A35', 'A60', 'A75', 'C15', 'B05'\}} \\
\texttt{ARGUMENT: \{'Ich übernehme deinen Bund als gemeinsame Basis (Aufwand 3350 ≤ 3389). Damit nutzen wir das Budget fast voll und enthalten alle günstigen High{-}Value{-}Items plus die gezielte Aufstockung C15+B05. Wenn du willst, können wir den Restpuffer von 39 nutzen, aber kein weiteres Item passt kostenmäßig hinein. Daher ist das ein effizienter Abschluss. Ich bin bereit, so zu finalisieren.'\}} \\
            }
        }
    }
    & \\ \\

    \theutterance \stepcounter{utterance}  
    & \multicolumn{4}{p{0.6\linewidth}}{
        \cellcolor[rgb]{0.9,0.9,0.9}{
            \makecell[{{p{\linewidth}}}]{
                \texttt{\tiny{[P1$\rangle$GM]}}
                \texttt{ZUSTIMMUNG: \{'C28', 'C48', 'B10', 'B78', 'B66', 'A35', 'A60', 'A75', 'C15', 'B05'\}} \\
\texttt{ARGUMENT: \{'Einverstanden. Aufwand 3350 liegt sicher unter 3389, und die Auswahl bündelt starke Werte bei sehr gutem Aufwand{-}Nutzen{-}Verhältnis. Gute Einigung.'\}} \\
            }
        }
    }
    & & \\ \\

    \theutterance \stepcounter{utterance}  
    & & & \multicolumn{2}{p{0.3\linewidth}}{
        \cellcolor[rgb]{0.9,0.9,0.9}{
            \makecell[{{p{\linewidth}}}]{
                \texttt{\tiny{[GM$|$GM]}}
                \texttt{game successful} \\
            }
        }
    }
    & & \\ \\

    \theutterance \stepcounter{utterance}  
    & & & \multicolumn{2}{p{0.3\linewidth}}{
        \cellcolor[rgb]{0.9,0.9,0.9}{
            \makecell[{{p{\linewidth}}}]{
                \texttt{\tiny{[GM$|$GM]}}
                \texttt{end game} \\
            }
        }
    }
    & & \\ \\

\end{supertabular}
}

\end{document}
