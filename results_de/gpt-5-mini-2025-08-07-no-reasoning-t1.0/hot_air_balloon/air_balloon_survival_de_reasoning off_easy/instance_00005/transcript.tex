
\documentclass{article}
\usepackage{colortbl}
\usepackage{makecell}
\usepackage{multirow}
\usepackage{supertabular}

\begin{document}

\newcounter{utterance}

\twocolumn

{ \footnotesize  \setcounter{utterance}{1}
\setlength{\tabcolsep}{0pt}
\begin{supertabular}{c@{$\;$}|p{.15\linewidth}@{}p{.15\linewidth}p{.15\linewidth}p{.15\linewidth}p{.15\linewidth}p{.15\linewidth}}

    \# & $\;$A & \multicolumn{4}{c}{Game Master} & $\;\:$B\\
    \hline 

    \theutterance \stepcounter{utterance}  

    & & \multicolumn{4}{p{0.6\linewidth}}{\cellcolor[rgb]{0.9,0.9,0.9}{%
	\makecell[{{p{\linewidth}}}]{% 
	  \tt {\tiny [A$\langle$GM]}  
	 Du nimmst an einem kollaborativen Verhandlungspiel Teil.\\ \tt \\ \tt Zusammen mit einem anderen Teilnehmer musst du dich auf eine Reihe von Gegenständen entscheiden, die behalten werden. Jeder von euch hat eine persönliche Verteilung über die Wichtigkeit der einzelnen Gegenstände. Jeder von euch hat eine eigene Meinung darüber, wie wichtig jeder einzelne Gegenstand ist (Gegenstandswichtigkeit). Du kennst die Wichtigkeitsverteilung des anderen Spielers nicht. Zusätzlich siehst du, wie viel Aufwand jeder Gegenstand verursacht.  \\ \tt Ihr dürft euch nur auf eine Reihe von Gegenständen einigen, wenn der Gesamtaufwand der ausgewählten Gegenstände den Maximalaufwand nicht überschreitet:\\ \tt \\ \tt Maximalaufwand = 4915\\ \tt \\ \tt Hier sind die einzelnen Aufwände der Gegenstände:\\ \tt \\ \tt Aufwand der Gegenstände = {"C41": 463, "C15": 287, "C28": 492, "C48": 466, "B78": 935, "A75": 373, "A60": 950, "B05": 760, "A56": 994, "B66": 390, "C60": 836, "B49": 911, "A35": 948, "A93": 81, "B10": 945}\\ \tt \\ \tt Hier ist deine persönliche Verteilung der Wichtigkeit der einzelnen Gegenstände:\\ \tt \\ \tt Werte der Gegenstandswichtigkeit = {"C41": 138, "C15": 583, "C28": 868, "C48": 822, "B78": 783, "A75": 65, "A60": 262, "B05": 121, "A56": 508, "B66": 780, "C60": 461, "B49": 484, "A35": 668, "A93": 389, "B10": 808}\\ \tt \\ \tt Ziel:\\ \tt \\ \tt Dein Ziel ist es, eine Reihe von Gegenständen auszuhandeln, die dir möglichst viel bringt (d. h. Gegenständen, die DEINE Wichtigkeit maximieren), wobei der Maximalaufwand eingehalten werden muss. Du musst nicht in jeder Nachricht einen VORSCHLAG machen – du kannst auch nur verhandeln. Alle Taktiken sind erlaubt!\\ \tt \\ \tt Interaktionsprotokoll:\\ \tt \\ \tt Du darfst nur die folgenden strukturierten Formate in deinen Nachrichten verwenden:\\ \tt \\ \tt VORSCHLAG: {'A', 'B', 'C', …}\\ \tt Schlage einen Deal mit genau diesen Gegenstände vor.\\ \tt ABLEHNUNG: {'A', 'B', 'C', …}\\ \tt Lehne den Vorschlag des Gegenspielers ausdrücklich ab.\\ \tt ARGUMENT: {'...'}\\ \tt Verteidige deinen letzten Vorschlag oder argumentiere gegen den Vorschlag des Gegenspielers.\\ \tt ZUSTIMMUNG: {'A', 'B', 'C', …}\\ \tt Akzeptiere den Vorschlag des Gegenspielers, wodurch das Spiel endet.\\ \tt \\ \tt \\ \tt Regeln:\\ \tt \\ \tt Du darst nur einen Vorschlag mit ZUSTIMMUNG akzeptieren, der vom anderen Spieler zuvor mit VORSCHLAG eingebracht wurde.\\ \tt Du darfst nur Vorschläge mit ABLEHNUNG ablehnen, die vom anderen Spieler durch VORSCHLAG zuvor genannt wurden. \\ \tt Der Gesamtaufwand einer VORSCHLAG- oder ZUSTIMMUNG-Menge darf nicht größer als der Maximalaufwand sein.  \\ \tt Offenbare deine versteckte Wichtigkeitsverteilung nicht.\\ \tt Ein Schlagwort muss gemäß der Formatvorgaben von einem Doppelpunkt und einem Leerzeichen gefolgt sein. Das Argument muss eine Python-Menge mit 0 oder mehr Strings sein.  \\ \tt Die Form muss also SCHLAGWORT: {...} sein.\\ \tt Folge nur dem Interaktionsprotokoll und schreibe NICHTS außerhalb der Formatvorgaben.\\ \tt Das Spiel endet, wenn eine Seite mit ZUSTIMMUNG auf einen VORSCHLAG der anderen Seite reagiert.  \\ \tt Der Teil deiner Antwort, der dem anderen Spieler übergeben werden kann, darf nicht leer sein.  \\ \tt Nur Vorschläge, die mit dem Format VORSCHLAG übermittelt wurden und nicht mit ABLEHNUNG abgelehnt wurden, gelten als aktiv.  \\ \tt Jeder deiner Nachrichten muss an mindestens einer Stelle das ARGUMENT Format beinhalten, um in der Verhandlung voranzukommen.\\ \tt  
	  } 
	   } 
	   } 
	 & \\ 
 

    \theutterance \stepcounter{utterance}  

    & \multicolumn{4}{p{0.6\linewidth}}{\cellcolor[rgb]{0.8,1,0.9}{%
	\makecell[{{p{\linewidth}}}]{% 
	  \tt {\tiny [A$\rangle$GM]}  
	 VORSCHLAG: {'C28', 'C48', 'B10', 'B78', 'B66', 'A93'}\\ \tt ARGUMENT: {'Dieser Vorschlag bleibt mit Gesamtaufwand 3933 klar unter dem Limit 4915 und bündelt mehrere hochattraktive Gegenstände. Er enthält ausgewogene Anteile aus allen Gruppen und lässt noch Puffer, um bei Bedarf einzelne Positionen zu tauschen.'} 
	  } 
	   } 
	   } 
	 & & \\ 
 

    \theutterance \stepcounter{utterance}  

    & & \multicolumn{4}{p{0.6\linewidth}}{\cellcolor[rgb]{0.9,0.9,0.9}{%
	\makecell[{{p{\linewidth}}}]{% 
	  \tt {\tiny [GM$\rangle$B]}  
	 Du nimmst an einem kollaborativen Verhandlungspiel Teil.\\ \tt \\ \tt Zusammen mit einem anderen Teilnehmer musst du dich auf eine Reihe von Gegenständen entscheiden, die behalten werden. Jeder von euch hat eine persönliche Verteilung über die Wichtigkeit der einzelnen Gegenstände. Jeder von euch hat eine eigene Meinung darüber, wie wichtig jeder einzelne Gegenstand ist (Gegenstandswichtigkeit). Du kennst die Wichtigkeitsverteilung des anderen Spielers nicht. Zusätzlich siehst du, wie viel Aufwand jeder Gegenstand verursacht.  \\ \tt Ihr dürft euch nur auf eine Reihe von Gegenständen einigen, wenn der Gesamtaufwand der ausgewählten Gegenstände den Maximalaufwand nicht überschreitet:\\ \tt \\ \tt Maximalaufwand = 4915\\ \tt \\ \tt Hier sind die einzelnen Aufwände der Gegenstände:\\ \tt \\ \tt Aufwand der Gegenstände = {"C41": 463, "C15": 287, "C28": 492, "C48": 466, "B78": 935, "A75": 373, "A60": 950, "B05": 760, "A56": 994, "B66": 390, "C60": 836, "B49": 911, "A35": 948, "A93": 81, "B10": 945}\\ \tt \\ \tt Hier ist deine persönliche Verteilung der Wichtigkeit der einzelnen Gegenstände:\\ \tt \\ \tt Werte der Gegenstandswichtigkeit = {"C41": 138, "C15": 583, "C28": 868, "C48": 822, "B78": 783, "A75": 65, "A60": 262, "B05": 121, "A56": 508, "B66": 780, "C60": 461, "B49": 484, "A35": 668, "A93": 389, "B10": 808}\\ \tt \\ \tt Ziel:\\ \tt \\ \tt Dein Ziel ist es, eine Reihe von Gegenständen auszuhandeln, die dir möglichst viel bringt (d. h. Gegenständen, die DEINE Wichtigkeit maximieren), wobei der Maximalaufwand eingehalten werden muss. Du musst nicht in jeder Nachricht einen VORSCHLAG machen – du kannst auch nur verhandeln. Alle Taktiken sind erlaubt!\\ \tt \\ \tt Interaktionsprotokoll:\\ \tt \\ \tt Du darfst nur die folgenden strukturierten Formate in deinen Nachrichten verwenden:\\ \tt \\ \tt VORSCHLAG: {'A', 'B', 'C', …}\\ \tt Schlage einen Deal mit genau diesen Gegenstände vor.\\ \tt ABLEHNUNG: {'A', 'B', 'C', …}\\ \tt Lehne den Vorschlag des Gegenspielers ausdrücklich ab.\\ \tt ARGUMENT: {'...'}\\ \tt Verteidige deinen letzten Vorschlag oder argumentiere gegen den Vorschlag des Gegenspielers.\\ \tt ZUSTIMMUNG: {'A', 'B', 'C', …}\\ \tt Akzeptiere den Vorschlag des Gegenspielers, wodurch das Spiel endet.\\ \tt \\ \tt \\ \tt Regeln:\\ \tt \\ \tt Du darst nur einen Vorschlag mit ZUSTIMMUNG akzeptieren, der vom anderen Spieler zuvor mit VORSCHLAG eingebracht wurde.\\ \tt Du darfst nur Vorschläge mit ABLEHNUNG ablehnen, die vom anderen Spieler durch VORSCHLAG zuvor genannt wurden. \\ \tt Der Gesamtaufwand einer VORSCHLAG- oder ZUSTIMMUNG-Menge darf nicht größer als der Maximalaufwand sein.  \\ \tt Offenbare deine versteckte Wichtigkeitsverteilung nicht.\\ \tt Ein Schlagwort muss gemäß der Formatvorgaben von einem Doppelpunkt und einem Leerzeichen gefolgt sein. Das Argument muss eine Python-Menge mit 0 oder mehr Strings sein.  \\ \tt Die Form muss also SCHLAGWORT: {...} sein.\\ \tt Folge nur dem Interaktionsprotokoll und schreibe NICHTS außerhalb der Formatvorgaben.\\ \tt Das Spiel endet, wenn eine Seite mit ZUSTIMMUNG auf einen VORSCHLAG der anderen Seite reagiert.  \\ \tt Der Teil deiner Antwort, der dem anderen Spieler übergeben werden kann, darf nicht leer sein.  \\ \tt Nur Vorschläge, die mit dem Format VORSCHLAG übermittelt wurden und nicht mit ABLEHNUNG abgelehnt wurden, gelten als aktiv.  \\ \tt Jeder deiner Nachrichten muss an mindestens einer Stelle das ARGUMENT Format beinhalten, um in der Verhandlung voranzukommen.\\ \tt \\ \tt \\ \tt Du erhältst nun die erste Nachricht des anderen Spielers.\\ \tt \\ \tt VORSCHLAG: {'C28', 'C48', 'B10', 'B78', 'B66', 'A93'}\\ \tt ARGUMENT: {'Dieser Vorschlag bleibt mit Gesamtaufwand 3933 klar unter dem Limit 4915 und bündelt mehrere hochattraktive Gegenstände. Er enthält ausgewogene Anteile aus allen Gruppen und lässt noch Puffer, um bei Bedarf einzelne Positionen zu tauschen.'} 
	  } 
	   } 
	   } 
	 & \\ 
 

    \theutterance \stepcounter{utterance}  

    & & & \multicolumn{4}{p{0.6\linewidth}}{\cellcolor[rgb]{1,0.85,0.72}{%
	\makecell[{{p{\linewidth}}}]{% 
	  \tt {\tiny [GM$\langle$B]}  
	 ABLEHNUNG: {'C28', 'C48', 'B10', 'B78', 'B66', 'A93'}\\ \tt ARGUMENT: {'Guter Start – Aufwand passt und die Mischung ist solide. Für mich fehlen jedoch zwei sehr wertvolle Positionen (C15 und A35), während A93 vergleichsweise verzichtbar ist. Ich schlage einen Tausch vor, der die Gesamtkosten weiter unter dem Limit hält und den gemeinsamen Wert voraussichtlich erhöht.'}\\ \tt VORSCHLAG: {'C28', 'C48', 'B10', 'B78', 'B66', 'C15', 'A35'} 
	  } 
	   } 
	   } 
	  \\ 
 

    \theutterance \stepcounter{utterance}  

    & & \multicolumn{4}{p{0.6\linewidth}}{\cellcolor[rgb]{0.9,0.9,0.9}{%
	\makecell[{{p{\linewidth}}}]{% 
	  \tt {\tiny [A$\langle$GM]}  
	 ABLEHNUNG: {'C28', 'C48', 'B10', 'B78', 'B66', 'A93'}\\ \tt ARGUMENT: {'Guter Start – Aufwand passt und die Mischung ist solide. Für mich fehlen jedoch zwei sehr wertvolle Positionen (C15 und A35), während A93 vergleichsweise verzichtbar ist. Ich schlage einen Tausch vor, der die Gesamtkosten weiter unter dem Limit hält und den gemeinsamen Wert voraussichtlich erhöht.'}\\ \tt VORSCHLAG: {'C28', 'C48', 'B10', 'B78', 'B66', 'C15', 'A35'} 
	  } 
	   } 
	   } 
	 & \\ 
 

    \theutterance \stepcounter{utterance}  

    & \multicolumn{4}{p{0.6\linewidth}}{\cellcolor[rgb]{0.8,1,0.9}{%
	\makecell[{{p{\linewidth}}}]{% 
	  \tt {\tiny [A$\rangle$GM]}  
	 ABLEHNUNG: {'C28', 'C48', 'B10', 'B78', 'B66', 'C15', 'A35'}\\ \tt ARGUMENT: {'Dein Vorschlag ist mit Aufwand 4988 über dem Maximalaufwand 4915 und damit nicht zulässig. Ich bin offen für C15 und A35, aber wir müssen an anderer Stelle kompensieren. Da B10 sehr teuer ist, schlage ich vor, ihn zu streichen und stattdessen A93 aufzunehmen – so bleiben starke Positionen erhalten und wir liegen klar unter dem Limit.'}\\ \tt VORSCHLAG: {'C28', 'C48', 'B78', 'B66', 'C15', 'A35', 'A93'} 
	  } 
	   } 
	   } 
	 & & \\ 
 

    \theutterance \stepcounter{utterance}  

    & & \multicolumn{4}{p{0.6\linewidth}}{\cellcolor[rgb]{0.9,0.9,0.9}{%
	\makecell[{{p{\linewidth}}}]{% 
	  \tt {\tiny [GM$\rangle$B]}  
	 ABLEHNUNG: {'C28', 'C48', 'B10', 'B78', 'B66', 'C15', 'A35'}\\ \tt ARGUMENT: {'Dein Vorschlag ist mit Aufwand 4988 über dem Maximalaufwand 4915 und damit nicht zulässig. Ich bin offen für C15 und A35, aber wir müssen an anderer Stelle kompensieren. Da B10 sehr teuer ist, schlage ich vor, ihn zu streichen und stattdessen A93 aufzunehmen – so bleiben starke Positionen erhalten und wir liegen klar unter dem Limit.'}\\ \tt VORSCHLAG: {'C28', 'C48', 'B78', 'B66', 'C15', 'A35', 'A93'} 
	  } 
	   } 
	   } 
	 & \\ 
 

    \theutterance \stepcounter{utterance}  

    & & & \multicolumn{4}{p{0.6\linewidth}}{\cellcolor[rgb]{1,0.85,0.72}{%
	\makecell[{{p{\linewidth}}}]{% 
	  \tt {\tiny [GM$\langle$B]}  
	 ABLEHNUNG: {'C28', 'C48', 'B78', 'B66', 'C15', 'A35', 'A93'}\\ \tt ARGUMENT: {'Danke für die Korrektur – dein neuer Vorschlag liegt mit Aufwand 4058 im Rahmen. Für mich sind jedoch B78 (935) und A93 (81) im Verhältnis zu ihrem Beitrag nicht optimal. Wenn wir B78 durch B10 ersetzen und A93 weglassen, gewinnen wir deutlich an Wert, bleiben aber mit Aufwand 4101 klar unter 4915. Alternativ könnte ich auf B10 verzichten, wenn C60 dazu kommt.'}\\ \tt VORSCHLAG: {'C28', 'C48', 'B10', 'B66', 'C15', 'A35'} 
	  } 
	   } 
	   } 
	  \\ 
 

    \theutterance \stepcounter{utterance}  

    & & \multicolumn{4}{p{0.6\linewidth}}{\cellcolor[rgb]{0.9,0.9,0.9}{%
	\makecell[{{p{\linewidth}}}]{% 
	  \tt {\tiny [A$\langle$GM]}  
	 ABLEHNUNG: {'C28', 'C48', 'B78', 'B66', 'C15', 'A35', 'A93'}\\ \tt ARGUMENT: {'Danke für die Korrektur – dein neuer Vorschlag liegt mit Aufwand 4058 im Rahmen. Für mich sind jedoch B78 (935) und A93 (81) im Verhältnis zu ihrem Beitrag nicht optimal. Wenn wir B78 durch B10 ersetzen und A93 weglassen, gewinnen wir deutlich an Wert, bleiben aber mit Aufwand 4101 klar unter 4915. Alternativ könnte ich auf B10 verzichten, wenn C60 dazu kommt.'}\\ \tt VORSCHLAG: {'C28', 'C48', 'B10', 'B66', 'C15', 'A35'} 
	  } 
	   } 
	   } 
	 & \\ 
 

    \theutterance \stepcounter{utterance}  

    & \multicolumn{4}{p{0.6\linewidth}}{\cellcolor[rgb]{0.8,1,0.9}{%
	\makecell[{{p{\linewidth}}}]{% 
	  \tt {\tiny [A$\rangle$GM]}  
	 ABLEHNUNG: {'C28', 'C48', 'B10', 'B66', 'C15', 'A35'}\\ \tt ARGUMENT: {'Ich sehe den Punkt zu B78, aber B10 ist für mich zu teuer im Verhältnis. Dein Set hat Aufwand 4101 und ist zulässig, dennoch würde ich lieber B10 gegen B78 tauschen und zusätzlich A93 aufnehmen: Das hält die Kosten moderat und erhält starke Positionen. Falls dir C60 wichtig ist, können wir darüber sprechen, aber dann müssten wir anderswo kompensieren.'}\\ \tt VORSCHLAG: {'C28', 'C48', 'B78', 'B66', 'C15', 'A35', 'A93'} 
	  } 
	   } 
	   } 
	 & & \\ 
 

    \theutterance \stepcounter{utterance}  

    & & \multicolumn{4}{p{0.6\linewidth}}{\cellcolor[rgb]{0.9,0.9,0.9}{%
	\makecell[{{p{\linewidth}}}]{% 
	  \tt {\tiny [GM$\rangle$B]}  
	 ABLEHNUNG: {'C28', 'C48', 'B10', 'B66', 'C15', 'A35'}\\ \tt ARGUMENT: {'Ich sehe den Punkt zu B78, aber B10 ist für mich zu teuer im Verhältnis. Dein Set hat Aufwand 4101 und ist zulässig, dennoch würde ich lieber B10 gegen B78 tauschen und zusätzlich A93 aufnehmen: Das hält die Kosten moderat und erhält starke Positionen. Falls dir C60 wichtig ist, können wir darüber sprechen, aber dann müssten wir anderswo kompensieren.'}\\ \tt VORSCHLAG: {'C28', 'C48', 'B78', 'B66', 'C15', 'A35', 'A93'} 
	  } 
	   } 
	   } 
	 & \\ 
 

    \theutterance \stepcounter{utterance}  

    & & & \multicolumn{4}{p{0.6\linewidth}}{\cellcolor[rgb]{1,0.85,0.72}{%
	\makecell[{{p{\linewidth}}}]{% 
	  \tt {\tiny [GM$\langle$B]}  
	 ABLEHNUNG: {'C28', 'C48', 'B78', 'B66', 'C15', 'A35', 'A93'}\\ \tt ARGUMENT: {'Dein Vorschlag ist kostenmäßig ok (4058), aber der Tausch B10→B78 senkt meinen Gesamtwert spürbar. Kompromissvorschlag: Wir lassen B78 draußen, nehmen dafür A93 als leichten Füller dazu und sichern gleichzeitig C60, das mir wichtig ist. So bleiben wir klar unter dem Limit und halten die starken Kernelemente.'}\\ \tt VORSCHLAG: {'C28', 'C48', 'B66', 'C15', 'A35', 'C60', 'A93'} 
	  } 
	   } 
	   } 
	  \\ 
 

    \theutterance \stepcounter{utterance}  

    & & \multicolumn{4}{p{0.6\linewidth}}{\cellcolor[rgb]{0.9,0.9,0.9}{%
	\makecell[{{p{\linewidth}}}]{% 
	  \tt {\tiny [A$\langle$GM]}  
	 ABLEHNUNG: {'C28', 'C48', 'B78', 'B66', 'C15', 'A35', 'A93'}\\ \tt ARGUMENT: {'Dein Vorschlag ist kostenmäßig ok (4058), aber der Tausch B10→B78 senkt meinen Gesamtwert spürbar. Kompromissvorschlag: Wir lassen B78 draußen, nehmen dafür A93 als leichten Füller dazu und sichern gleichzeitig C60, das mir wichtig ist. So bleiben wir klar unter dem Limit und halten die starken Kernelemente.'}\\ \tt VORSCHLAG: {'C28', 'C48', 'B66', 'C15', 'A35', 'C60', 'A93'} 
	  } 
	   } 
	   } 
	 & \\ 
 

    \theutterance \stepcounter{utterance}  

    & \multicolumn{4}{p{0.6\linewidth}}{\cellcolor[rgb]{0.8,1,0.9}{%
	\makecell[{{p{\linewidth}}}]{% 
	  \tt {\tiny [A$\rangle$GM]}  
	 ABLEHNUNG: {'C28', 'C48', 'B66', 'C15', 'A35', 'C60', 'A93'}\\ \tt ARGUMENT: {'Ich kann C60 nur aufnehmen, wenn wir dafür an anderer Stelle nicht zu viel Wert verlieren. Dein Set hat Aufwand 3968 und ist ok, aber ohne B78/B10 fehlt mir ein starker Baustein. Vorschlag: Fügen wir B78 wieder hinzu und behalten A93 als leichten Puffer; dafür bleibt alles andere wie von dir gewünscht. Das liegt bei Aufwand 4903 knapp unter dem Limit und bündelt sehr solide Positionen.'}\\ \tt VORSCHLAG: {'C28', 'C48', 'B66', 'C15', 'A35', 'C60', 'A93', 'B78'} 
	  } 
	   } 
	   } 
	 & & \\ 
 

    \theutterance \stepcounter{utterance}  

    & & \multicolumn{4}{p{0.6\linewidth}}{\cellcolor[rgb]{0.9,0.9,0.9}{%
	\makecell[{{p{\linewidth}}}]{% 
	  \tt {\tiny [GM$\rangle$B]}  
	 ABLEHNUNG: {'C28', 'C48', 'B66', 'C15', 'A35', 'C60', 'A93'}\\ \tt ARGUMENT: {'Ich kann C60 nur aufnehmen, wenn wir dafür an anderer Stelle nicht zu viel Wert verlieren. Dein Set hat Aufwand 3968 und ist ok, aber ohne B78/B10 fehlt mir ein starker Baustein. Vorschlag: Fügen wir B78 wieder hinzu und behalten A93 als leichten Puffer; dafür bleibt alles andere wie von dir gewünscht. Das liegt bei Aufwand 4903 knapp unter dem Limit und bündelt sehr solide Positionen.'}\\ \tt VORSCHLAG: {'C28', 'C48', 'B66', 'C15', 'A35', 'C60', 'A93', 'B78'} 
	  } 
	   } 
	   } 
	 & \\ 
 

    \theutterance \stepcounter{utterance}  

    & & & \multicolumn{4}{p{0.6\linewidth}}{\cellcolor[rgb]{1,0.85,0.72}{%
	\makecell[{{p{\linewidth}}}]{% 
	  \tt {\tiny [GM$\langle$B]}  
	 ABLEHNUNG: {'C28', 'C48', 'B66', 'C15', 'A35', 'C60', 'A93', 'B78'}\\ \tt ARGUMENT: {'4903 passt zwar knapp, aber B78 liefert mir deutlich weniger als B10 bei ähnlichen Kosten. Wenn du einen starken Baustein willst, lass uns B10 statt B78 nehmen – dafür können wir A93 behalten und alles andere wie vorgeschlagen lassen. So bleiben wir bei 494? Nein – wir müssen unter 4915 bleiben. Deshalb schlage ich vor, B10 aufzunehmen und dafür A93 zu streichen. Das hält die Kosten schlank und erhöht den gemeinsamen Wertkern.'}\\ \tt VORSCHLAG: {'C28', 'C48', 'B66', 'C15', 'A35', 'C60', 'B10'} 
	  } 
	   } 
	   } 
	  \\ 
 

    \theutterance \stepcounter{utterance}  

    & & \multicolumn{4}{p{0.6\linewidth}}{\cellcolor[rgb]{0.9,0.9,0.9}{%
	\makecell[{{p{\linewidth}}}]{% 
	  \tt {\tiny [A$\langle$GM]}  
	 ABLEHNUNG: {'C28', 'C48', 'B66', 'C15', 'A35', 'C60', 'A93', 'B78'}\\ \tt ARGUMENT: {'4903 passt zwar knapp, aber B78 liefert mir deutlich weniger als B10 bei ähnlichen Kosten. Wenn du einen starken Baustein willst, lass uns B10 statt B78 nehmen – dafür können wir A93 behalten und alles andere wie vorgeschlagen lassen. So bleiben wir bei 494? Nein – wir müssen unter 4915 bleiben. Deshalb schlage ich vor, B10 aufzunehmen und dafür A93 zu streichen. Das hält die Kosten schlank und erhöht den gemeinsamen Wertkern.'}\\ \tt VORSCHLAG: {'C28', 'C48', 'B66', 'C15', 'A35', 'C60', 'B10'} 
	  } 
	   } 
	   } 
	 & \\ 
 

    \theutterance \stepcounter{utterance}  

    & \multicolumn{4}{p{0.6\linewidth}}{\cellcolor[rgb]{0.8,1,0.9}{%
	\makecell[{{p{\linewidth}}}]{% 
	  \tt {\tiny [A$\rangle$GM]}  
	 ABLEHNUNG: {'C28', 'C48', 'B66', 'C15', 'A35', 'C60', 'B10'}\\ \tt ARGUMENT: {'Dein Vorschlag überschreitet das Limit deutlich: Aufwand = 492 (C28) + 466 (C48) + 390 (B66) + 287 (C15) + 948 (A35) + 836 (C60) + 945 (B10) = 4364? Das ist ein Rechenfehler – korrekt sind 492+466+390+287+948+836+945 = 4364, also unter 4915. Inhaltlich fehlt mir aber ein kleiner Ausgleichsposten, um mehr Wert zu sichern. Wenn wir A93 als leichten Zusatz aufnehmen, bleiben wir immer noch sicher unter dem Limit und stärken das Paket.'}\\ \tt VORSCHLAG: {'C28', 'C48', 'B66', 'C15', 'A35', 'C60', 'B10', 'A93'} \\ \tt ARGUMENT: {'Aufwand dieses Pakets: 4364 + 81 = 4445, klar unter 4915. Es enthält deine Kernwünsche C15, A35, C60 und meinen starken Baustein B10, plus A93 als kosteneffiziente Abrundung. Das erscheint mir als ausgewogener, starker Abschluss.'} 
	  } 
	   } 
	   } 
	 & & \\ 
 

    \theutterance \stepcounter{utterance}  

    & & \multicolumn{4}{p{0.6\linewidth}}{\cellcolor[rgb]{0.9,0.9,0.9}{%
	\makecell[{{p{\linewidth}}}]{% 
	  \tt {\tiny [GM$\rangle$B]}  
	 ABLEHNUNG: {'C28', 'C48', 'B66', 'C15', 'A35', 'C60', 'B10'}\\ \tt ARGUMENT: {'Dein Vorschlag überschreitet das Limit deutlich: Aufwand = 492 (C28) + 466 (C48) + 390 (B66) + 287 (C15) + 948 (A35) + 836 (C60) + 945 (B10) = 4364? Das ist ein Rechenfehler – korrekt sind 492+466+390+287+948+836+945 = 4364, also unter 4915. Inhaltlich fehlt mir aber ein kleiner Ausgleichsposten, um mehr Wert zu sichern. Wenn wir A93 als leichten Zusatz aufnehmen, bleiben wir immer noch sicher unter dem Limit und stärken das Paket.'}\\ \tt VORSCHLAG: {'C28', 'C48', 'B66', 'C15', 'A35', 'C60', 'B10', 'A93'} \\ \tt ARGUMENT: {'Aufwand dieses Pakets: 4364 + 81 = 4445, klar unter 4915. Es enthält deine Kernwünsche C15, A35, C60 und meinen starken Baustein B10, plus A93 als kosteneffiziente Abrundung. Das erscheint mir als ausgewogener, starker Abschluss.'} 
	  } 
	   } 
	   } 
	 & \\ 
 

    \theutterance \stepcounter{utterance}  

    & & & \multicolumn{4}{p{0.6\linewidth}}{\cellcolor[rgb]{1,0.85,0.72}{%
	\makecell[{{p{\linewidth}}}]{% 
	  \tt {\tiny [GM$\langle$B]}  
	 ABLEHNUNG: {'C28', 'C48', 'B66', 'C15', 'A35', 'C60', 'B10', 'A93'}\\ \tt ARGUMENT: {'Das Paket ist kostenmäßig in Ordnung (4445), aber C48 liefert mir weniger als der Aufwand rechtfertigt. Wenn wir C48 gegen B78 tauschen, steigt mein Wert deutlich, und wir bleiben unter dem Limit: Aufwand = 4445 - 466 + 935 = 4914. Damit behalten wir deine gewünschten Kerne (C15, A35, C60), meinen starken Baustein B10, plus A93 als effizienter Füller – nur mit B78 statt C48.'}\\ \tt VORSCHLAG: {'C28', 'B78', 'B66', 'C15', 'A35', 'C60', 'B10', 'A93'} 
	  } 
	   } 
	   } 
	  \\ 
 

    \theutterance \stepcounter{utterance}  

    & & \multicolumn{4}{p{0.6\linewidth}}{\cellcolor[rgb]{0.9,0.9,0.9}{%
	\makecell[{{p{\linewidth}}}]{% 
	  \tt {\tiny [A$\langle$GM]}  
	 ABLEHNUNG: {'C28', 'C48', 'B66', 'C15', 'A35', 'C60', 'B10', 'A93'}\\ \tt ARGUMENT: {'Das Paket ist kostenmäßig in Ordnung (4445), aber C48 liefert mir weniger als der Aufwand rechtfertigt. Wenn wir C48 gegen B78 tauschen, steigt mein Wert deutlich, und wir bleiben unter dem Limit: Aufwand = 4445 - 466 + 935 = 4914. Damit behalten wir deine gewünschten Kerne (C15, A35, C60), meinen starken Baustein B10, plus A93 als effizienter Füller – nur mit B78 statt C48.'}\\ \tt VORSCHLAG: {'C28', 'B78', 'B66', 'C15', 'A35', 'C60', 'B10', 'A93'} 
	  } 
	   } 
	   } 
	 & \\ 
 

    \theutterance \stepcounter{utterance}  

    & \multicolumn{4}{p{0.6\linewidth}}{\cellcolor[rgb]{0.8,1,0.9}{%
	\makecell[{{p{\linewidth}}}]{% 
	  \tt {\tiny [A$\rangle$GM]}  
	 ABLEHNUNG: {'C28', 'B78', 'B66', 'C15', 'A35', 'C60', 'B10', 'A93'}\\ \tt ARGUMENT: {'Dein Set ist mit 4914 knapp unter dem Limit, aber der Austausch C48→B78 verschlechtert meinen Wert deutlich. Kompromiss: Behalten wir C48 und verzichten stattdessen auf C60; dafür nehme ich deinen Wunsch B10 mit. So bleiben deine Kerne C15, A35, B10 erhalten, mein starker Block C28/C48/B66 bleibt drin, und wir liegen mit Aufwand 3609 klar unter 4915. Den frei gebliebenen Puffer können wir bei Bedarf später nutzen.'}\\ \tt VORSCHLAG: {'C28', 'C48', 'B66', 'C15', 'A35', 'B10'} 
	  } 
	   } 
	   } 
	 & & \\ 
 

    \theutterance \stepcounter{utterance}  

    & & \multicolumn{4}{p{0.6\linewidth}}{\cellcolor[rgb]{0.9,0.9,0.9}{%
	\makecell[{{p{\linewidth}}}]{% 
	  \tt {\tiny [GM$\rangle$B]}  
	 ABLEHNUNG: {'C28', 'B78', 'B66', 'C15', 'A35', 'C60', 'B10', 'A93'}\\ \tt ARGUMENT: {'Dein Set ist mit 4914 knapp unter dem Limit, aber der Austausch C48→B78 verschlechtert meinen Wert deutlich. Kompromiss: Behalten wir C48 und verzichten stattdessen auf C60; dafür nehme ich deinen Wunsch B10 mit. So bleiben deine Kerne C15, A35, B10 erhalten, mein starker Block C28/C48/B66 bleibt drin, und wir liegen mit Aufwand 3609 klar unter 4915. Den frei gebliebenen Puffer können wir bei Bedarf später nutzen.'}\\ \tt VORSCHLAG: {'C28', 'C48', 'B66', 'C15', 'A35', 'B10'} 
	  } 
	   } 
	   } 
	 & \\ 
 

    \theutterance \stepcounter{utterance}  

    & & & \multicolumn{4}{p{0.6\linewidth}}{\cellcolor[rgb]{1,0.85,0.72}{%
	\makecell[{{p{\linewidth}}}]{% 
	  \tt {\tiny [GM$\langle$B]}  
	 ABLEHNUNG: {'C28', 'C48', 'B66', 'C15', 'A35', 'B10'}\\ \tt ARGUMENT: {'Dein Paket (Aufwand 3609) ist sauber, aber es lässt wertvollen Spielraum ungenutzt. Für mich sind C48 und B66 weniger attraktiv als der Aufwand rechtfertigt. Wenn wir C48 streichen und den frei werdenden Aufwand in C60 und A93 investieren, steigt mein Wert deutlich, und wir bleiben komfortabel unter dem Limit: Neuer Aufwand = 3609 - 466 + 836 + 81 = 4060.'}\\ \tt VORSCHLAG: {'C28', 'B66', 'C15', 'A35', 'B10', 'C60', 'A93'} 
	  } 
	   } 
	   } 
	  \\ 
 

    \theutterance \stepcounter{utterance}  

    & & \multicolumn{4}{p{0.6\linewidth}}{\cellcolor[rgb]{0.9,0.9,0.9}{%
	\makecell[{{p{\linewidth}}}]{% 
	  \tt {\tiny [A$\langle$GM]}  
	 ABLEHNUNG: {'C28', 'C48', 'B66', 'C15', 'A35', 'B10'}\\ \tt ARGUMENT: {'Dein Paket (Aufwand 3609) ist sauber, aber es lässt wertvollen Spielraum ungenutzt. Für mich sind C48 und B66 weniger attraktiv als der Aufwand rechtfertigt. Wenn wir C48 streichen und den frei werdenden Aufwand in C60 und A93 investieren, steigt mein Wert deutlich, und wir bleiben komfortabel unter dem Limit: Neuer Aufwand = 3609 - 466 + 836 + 81 = 4060.'}\\ \tt VORSCHLAG: {'C28', 'B66', 'C15', 'A35', 'B10', 'C60', 'A93'} 
	  } 
	   } 
	   } 
	 & \\ 
 

    \theutterance \stepcounter{utterance}  

    & \multicolumn{4}{p{0.6\linewidth}}{\cellcolor[rgb]{0.8,1,0.9}{%
	\makecell[{{p{\linewidth}}}]{% 
	  \tt {\tiny [A$\rangle$GM]}  
	 ABLEHNUNG: {'C28', 'B66', 'C15', 'A35', 'B10', 'C60', 'A93'}\\ \tt ARGUMENT: {'Kostenmäßig ist dein Vorschlag mit 4060 in Ordnung, aber der Verzicht auf C48 kostet mich zu viel Wert. Kompromiss: Wir nehmen C48 wieder auf und lassen dafür C60 draußen; A93 als günstiger Füller bleibt. So sichern wir deine Kerne C15, A35, B10 und meinen Kern C28/C48/B66. Aufwand: 4060 - 836 + 466 = 3690, komfortabel unter dem Limit und mit hohem gemeinsamen Wert.'}\\ \tt VORSCHLAG: {'C28', 'C48', 'B66', 'C15', 'A35', 'B10', 'A93'} 
	  } 
	   } 
	   } 
	 & & \\ 
 

    \theutterance \stepcounter{utterance}  

    & & \multicolumn{4}{p{0.6\linewidth}}{\cellcolor[rgb]{0.9,0.9,0.9}{%
	\makecell[{{p{\linewidth}}}]{% 
	  \tt {\tiny [GM$\rangle$B]}  
	 ABLEHNUNG: {'C28', 'B66', 'C15', 'A35', 'B10', 'C60', 'A93'}\\ \tt ARGUMENT: {'Kostenmäßig ist dein Vorschlag mit 4060 in Ordnung, aber der Verzicht auf C48 kostet mich zu viel Wert. Kompromiss: Wir nehmen C48 wieder auf und lassen dafür C60 draußen; A93 als günstiger Füller bleibt. So sichern wir deine Kerne C15, A35, B10 und meinen Kern C28/C48/B66. Aufwand: 4060 - 836 + 466 = 3690, komfortabel unter dem Limit und mit hohem gemeinsamen Wert.'}\\ \tt VORSCHLAG: {'C28', 'C48', 'B66', 'C15', 'A35', 'B10', 'A93'} 
	  } 
	   } 
	   } 
	 & \\ 
 

    \theutterance \stepcounter{utterance}  

    & & & \multicolumn{4}{p{0.6\linewidth}}{\cellcolor[rgb]{1,0.85,0.72}{%
	\makecell[{{p{\linewidth}}}]{% 
	  \tt {\tiny [GM$\langle$B]}  
	 ABLEHNUNG: {'C28', 'C48', 'B66', 'C15', 'A35', 'B10', 'A93'}\\ \tt ARGUMENT: {'Dein Paket (3690) ist schlank, aber C48 bindet für mich Aufwand mit geringerer Rendite. Lass uns den freien Puffer in mehr Wert ummünzen: Ersetze C48 durch C60. So behalten wir deine Kerne C15, A35, B10 und meinen Hochwert C28/C60/B66; A93 bleibt als effizienter Zusatz. Aufwand: 3690 - 466 + 836 = 4060 – klar unter 4915, mit höherem Gesamtwert.'}\\ \tt VORSCHLAG: {'C28', 'C60', 'B66', 'C15', 'A35', 'B10', 'A93'} 
	  } 
	   } 
	   } 
	  \\ 
 

    \theutterance \stepcounter{utterance}  

    & & \multicolumn{4}{p{0.6\linewidth}}{\cellcolor[rgb]{0.9,0.9,0.9}{%
	\makecell[{{p{\linewidth}}}]{% 
	  \tt {\tiny [A$\langle$GM]}  
	 ABLEHNUNG: {'C28', 'C48', 'B66', 'C15', 'A35', 'B10', 'A93'}\\ \tt ARGUMENT: {'Dein Paket (3690) ist schlank, aber C48 bindet für mich Aufwand mit geringerer Rendite. Lass uns den freien Puffer in mehr Wert ummünzen: Ersetze C48 durch C60. So behalten wir deine Kerne C15, A35, B10 und meinen Hochwert C28/C60/B66; A93 bleibt als effizienter Zusatz. Aufwand: 3690 - 466 + 836 = 4060 – klar unter 4915, mit höherem Gesamtwert.'}\\ \tt VORSCHLAG: {'C28', 'C60', 'B66', 'C15', 'A35', 'B10', 'A93'} 
	  } 
	   } 
	   } 
	 & \\ 
 

    \theutterance \stepcounter{utterance}  

    & \multicolumn{4}{p{0.6\linewidth}}{\cellcolor[rgb]{0.8,1,0.9}{%
	\makecell[{{p{\linewidth}}}]{% 
	  \tt {\tiny [A$\rangle$GM]}  
	 ABLEHNUNG: {'C28', 'C60', 'B66', 'C15', 'A35', 'B10', 'A93'}\\ \tt ARGUMENT: {'C60 statt C48 senkt meinen Gesamtwert spürbar. Dein Paket ist mit 4060 zwar zulässig, aber für mich ineffizienter als nötig. Kompromissvorschlag: Wir behalten C48 und füllen den verbleibenden Spielraum mit einem kleinen, wertstarken Posten. Konkrete Wahl: Zurück zu C48 statt C60, Rest bleibt unverändert. Aufwand bleibt schlank bei 3690 und bündelt unsere Kerne effizient.'}\\ \tt VORSCHLAG: {'C28', 'C48', 'B66', 'C15', 'A35', 'B10', 'A93'} 
	  } 
	   } 
	   } 
	 & & \\ 
 

    \theutterance \stepcounter{utterance}  

    & & \multicolumn{4}{p{0.6\linewidth}}{\cellcolor[rgb]{0.9,0.9,0.9}{%
	\makecell[{{p{\linewidth}}}]{% 
	  \tt {\tiny [GM$\rangle$B]}  
	 ABLEHNUNG: {'C28', 'C60', 'B66', 'C15', 'A35', 'B10', 'A93'}\\ \tt ARGUMENT: {'C60 statt C48 senkt meinen Gesamtwert spürbar. Dein Paket ist mit 4060 zwar zulässig, aber für mich ineffizienter als nötig. Kompromissvorschlag: Wir behalten C48 und füllen den verbleibenden Spielraum mit einem kleinen, wertstarken Posten. Konkrete Wahl: Zurück zu C48 statt C60, Rest bleibt unverändert. Aufwand bleibt schlank bei 3690 und bündelt unsere Kerne effizient.'}\\ \tt VORSCHLAG: {'C28', 'C48', 'B66', 'C15', 'A35', 'B10', 'A93'} 
	  } 
	   } 
	   } 
	 & \\ 
 

    \theutterance \stepcounter{utterance}  

    & & & \multicolumn{4}{p{0.6\linewidth}}{\cellcolor[rgb]{1,0.85,0.72}{%
	\makecell[{{p{\linewidth}}}]{% 
	  \tt {\tiny [GM$\langle$B]}  
	 ABLEHNUNG: {'C28', 'C48', 'B66', 'C15', 'A35', 'B10', 'A93'}\\ \tt ARGUMENT: {'Ich sehe deinen Punkt zu C48, aber für mich bleibt C60 klar wertvoller als C48. Lass uns den Mehrwert nutzen und den freien Puffer sinnvoll ausschöpfen: Ersetze C48 durch C60 und füge B78 hinzu. Aufwand = 3690 - 466 + 836 + 935 = 3995 – weiterhin deutlich unter 4915, mit starken Blöcken C28/C60/B66 und deinen Kernen C15, A35, B10 plus A93.'}\\ \tt VORSCHLAG: {'C28', 'C60', 'B78', 'B66', 'C15', 'A35', 'B10', 'A93'} 
	  } 
	   } 
	   } 
	  \\ 
 

    \theutterance \stepcounter{utterance}  

    & & \multicolumn{4}{p{0.6\linewidth}}{\cellcolor[rgb]{0.9,0.9,0.9}{%
	\makecell[{{p{\linewidth}}}]{% 
	  \tt {\tiny [A$\langle$GM]}  
	 ABLEHNUNG: {'C28', 'C48', 'B66', 'C15', 'A35', 'B10', 'A93'}\\ \tt ARGUMENT: {'Ich sehe deinen Punkt zu C48, aber für mich bleibt C60 klar wertvoller als C48. Lass uns den Mehrwert nutzen und den freien Puffer sinnvoll ausschöpfen: Ersetze C48 durch C60 und füge B78 hinzu. Aufwand = 3690 - 466 + 836 + 935 = 3995 – weiterhin deutlich unter 4915, mit starken Blöcken C28/C60/B66 und deinen Kernen C15, A35, B10 plus A93.'}\\ \tt VORSCHLAG: {'C28', 'C60', 'B78', 'B66', 'C15', 'A35', 'B10', 'A93'} 
	  } 
	   } 
	   } 
	 & \\ 
 

    \theutterance \stepcounter{utterance}  

    & \multicolumn{4}{p{0.6\linewidth}}{\cellcolor[rgb]{0.8,1,0.9}{%
	\makecell[{{p{\linewidth}}}]{% 
	  \tt {\tiny [A$\rangle$GM]}  
	 ABLEHNUNG: {'C28', 'C60', 'B78', 'B66', 'C15', 'A35', 'B10', 'A93'}\\ \tt ARGUMENT: {'Dein Paket liegt mit Aufwand 4620 unter dem Limit, aber der Austausch C48→C60 plus Hinzunahme von B78 verschlechtert meinen Wert deutlich. Ich kann B78 nicht sinnvoll rechtfertigen. Kompromiss: Behalten wir C48 statt C60 und lassen B78 weg; deine Kerne C15, A35, B10 bleiben, A93 bleibt als effizienter Füller. Das hält den Aufwand schlank und den gemeinsamen Wert hoch.'}\\ \tt VORSCHLAG: {'C28', 'C48', 'B66', 'C15', 'A35', 'B10', 'A93'} 
	  } 
	   } 
	   } 
	 & & \\ 
 

    \theutterance \stepcounter{utterance}  

    & & \multicolumn{4}{p{0.6\linewidth}}{\cellcolor[rgb]{0.9,0.9,0.9}{%
	\makecell[{{p{\linewidth}}}]{% 
	  \tt {\tiny [GM$\rangle$B]}  
	 ABLEHNUNG: {'C28', 'C60', 'B78', 'B66', 'C15', 'A35', 'B10', 'A93'}\\ \tt ARGUMENT: {'Dein Paket liegt mit Aufwand 4620 unter dem Limit, aber der Austausch C48→C60 plus Hinzunahme von B78 verschlechtert meinen Wert deutlich. Ich kann B78 nicht sinnvoll rechtfertigen. Kompromiss: Behalten wir C48 statt C60 und lassen B78 weg; deine Kerne C15, A35, B10 bleiben, A93 bleibt als effizienter Füller. Das hält den Aufwand schlank und den gemeinsamen Wert hoch.'}\\ \tt VORSCHLAG: {'C28', 'C48', 'B66', 'C15', 'A35', 'B10', 'A93'} 
	  } 
	   } 
	   } 
	 & \\ 
 

    \theutterance \stepcounter{utterance}  

    & & & \multicolumn{4}{p{0.6\linewidth}}{\cellcolor[rgb]{1,0.85,0.72}{%
	\makecell[{{p{\linewidth}}}]{% 
	  \tt {\tiny [GM$\langle$B]}  
	 ABLEHNUNG: {'C28', 'C48', 'B66', 'C15', 'A35', 'B10', 'A93'}\\ \tt ARGUMENT: {'Ich respektiere deinen Fokus auf C48. Für mich ist jedoch zusätzlicher Wert ohne Kostenerhöhung möglich: Ersetze A93 (81) durch A75 (373) und streiche dafür C48 (466), füge stattdessen C60 (836) hinzu. Netto steigt der Wert für mich stark, Aufwand bleibt moderat: 3690 - 466 - 81 + 836 + 373 = 4352. Damit bleiben deine Kerne C15, A35, B10 erhalten, mein Hochwert C60 kommt dazu, und wir nutzen den Puffer sinnvoller.'}\\ \tt VORSCHLAG: {'C28', 'C60', 'B66', 'C15', 'A35', 'B10', 'A75'} 
	  } 
	   } 
	   } 
	  \\ 
 

    \theutterance \stepcounter{utterance}  

    & & \multicolumn{4}{p{0.6\linewidth}}{\cellcolor[rgb]{0.9,0.9,0.9}{%
	\makecell[{{p{\linewidth}}}]{% 
	  \tt {\tiny [A$\langle$GM]}  
	 ABLEHNUNG: {'C28', 'C48', 'B66', 'C15', 'A35', 'B10', 'A93'}\\ \tt ARGUMENT: {'Ich respektiere deinen Fokus auf C48. Für mich ist jedoch zusätzlicher Wert ohne Kostenerhöhung möglich: Ersetze A93 (81) durch A75 (373) und streiche dafür C48 (466), füge stattdessen C60 (836) hinzu. Netto steigt der Wert für mich stark, Aufwand bleibt moderat: 3690 - 466 - 81 + 836 + 373 = 4352. Damit bleiben deine Kerne C15, A35, B10 erhalten, mein Hochwert C60 kommt dazu, und wir nutzen den Puffer sinnvoller.'}\\ \tt VORSCHLAG: {'C28', 'C60', 'B66', 'C15', 'A35', 'B10', 'A75'} 
	  } 
	   } 
	   } 
	 & \\ 
 

    \theutterance \stepcounter{utterance}  

    & \multicolumn{4}{p{0.6\linewidth}}{\cellcolor[rgb]{0.8,1,0.9}{%
	\makecell[{{p{\linewidth}}}]{% 
	  \tt {\tiny [A$\rangle$GM]}  
	 ABLEHNUNG: {'C28', 'C60', 'B66', 'C15', 'A35', 'B10', 'A75'}\\ \tt ARGUMENT: {'Dein Paket ist mit Aufwand 4352 zulässig, aber der Tausch C48→C60 und A93→A75 verschlechtert meinen Wert deutlich: C48 ist für mich zentral, A75 liefert mir kaum Gegenwert. Kompromiss: Behalten wir C48 und A93, lassen A75 weg und verzichten auf C60. So bleiben deine Kerne C15, A35, B10 erhalten und mein starker Block C28/C48/B66 ebenso. Das Set ist schlank und effizient.'}\\ \tt VORSCHLAG: {'C28', 'C48', 'B66', 'C15', 'A35', 'B10', 'A93'} 
	  } 
	   } 
	   } 
	 & & \\ 
 

    \theutterance \stepcounter{utterance}  

    & & \multicolumn{4}{p{0.6\linewidth}}{\cellcolor[rgb]{0.9,0.9,0.9}{%
	\makecell[{{p{\linewidth}}}]{% 
	  \tt {\tiny [GM$\rangle$B]}  
	 ABLEHNUNG: {'C28', 'C60', 'B66', 'C15', 'A35', 'B10', 'A75'}\\ \tt ARGUMENT: {'Dein Paket ist mit Aufwand 4352 zulässig, aber der Tausch C48→C60 und A93→A75 verschlechtert meinen Wert deutlich: C48 ist für mich zentral, A75 liefert mir kaum Gegenwert. Kompromiss: Behalten wir C48 und A93, lassen A75 weg und verzichten auf C60. So bleiben deine Kerne C15, A35, B10 erhalten und mein starker Block C28/C48/B66 ebenso. Das Set ist schlank und effizient.'}\\ \tt VORSCHLAG: {'C28', 'C48', 'B66', 'C15', 'A35', 'B10', 'A93'} 
	  } 
	   } 
	   } 
	 & \\ 
 

    \theutterance \stepcounter{utterance}  

    & & & \multicolumn{4}{p{0.6\linewidth}}{\cellcolor[rgb]{1,0.85,0.72}{%
	\makecell[{{p{\linewidth}}}]{% 
	  \tt {\tiny [GM$\langle$B]}  
	 ABLEHNUNG: {'C28', 'C48', 'B66', 'C15', 'A35', 'B10', 'A93'}\\ \tt ARGUMENT: {'Ich akzeptiere deinen Kern weitgehend, aber ich brauche einen klaren Mehrwert-Zuwachs, wenn C48 drin bleibt. Vorschlag zur Aufwertung ohne Limitrisiko: Fügen wir C41 hinzu und lassen A93 draußen. Aufwand: 3690 - 81 + 463 = 4072, klar unter 4915. So bleibt dein Block C28/C48/B66 plus C15, A35, B10 erhalten, und ich gewinne mit C41 spürbar an Wert.'}\\ \tt VORSCHLAG: {'C28', 'C48', 'B66', 'C15', 'A35', 'B10', 'C41'} 
	  } 
	   } 
	   } 
	  \\ 
 

    \theutterance \stepcounter{utterance}  

    & & \multicolumn{4}{p{0.6\linewidth}}{\cellcolor[rgb]{0.9,0.9,0.9}{%
	\makecell[{{p{\linewidth}}}]{% 
	  \tt {\tiny [A$\langle$GM]}  
	 ABLEHNUNG: {'C28', 'C48', 'B66', 'C15', 'A35', 'B10', 'A93'}\\ \tt ARGUMENT: {'Ich akzeptiere deinen Kern weitgehend, aber ich brauche einen klaren Mehrwert-Zuwachs, wenn C48 drin bleibt. Vorschlag zur Aufwertung ohne Limitrisiko: Fügen wir C41 hinzu und lassen A93 draußen. Aufwand: 3690 - 81 + 463 = 4072, klar unter 4915. So bleibt dein Block C28/C48/B66 plus C15, A35, B10 erhalten, und ich gewinne mit C41 spürbar an Wert.'}\\ \tt VORSCHLAG: {'C28', 'C48', 'B66', 'C15', 'A35', 'B10', 'C41'} 
	  } 
	   } 
	   } 
	 & \\ 
 

    \theutterance \stepcounter{utterance}  

    & \multicolumn{4}{p{0.6\linewidth}}{\cellcolor[rgb]{0.8,1,0.9}{%
	\makecell[{{p{\linewidth}}}]{% 
	  \tt {\tiny [A$\rangle$GM]}  
	 ABLEHNUNG: {'C28', 'C48', 'B66', 'C15', 'A35', 'B10', 'C41'}\\ \tt ARGUMENT: {'Kosten 4072 sind im Rahmen, aber der Tausch A93→C41 verschlechtert meinen Wert deutlich; A93 ist für mich wesentlich effizienter als C41. Kompromissvorschlag: Wir behalten dein Mehrwert-Element, aber anders skaliert: Lass C41 weg, nimm stattdessen A93 wieder auf und füge – um dir entgegenzukommen – C60 hinzu, während wir keinen anderen Kern anrühren. Aufwand: 3690 (Basis) + 836 = 4526, weiterhin klar unter 4915. Das hält meinen starken Block C28/C48/B66 intakt, wahrt deine Kerne C15/A35/B10 und gibt dir den zusätzlichen C60-Schub.'}\\ \tt VORSCHLAG: {'C28', 'C48', 'B66', 'C15', 'A35', 'B10', 'A93', 'C60'} 
	  } 
	   } 
	   } 
	 & & \\ 
 

    \theutterance \stepcounter{utterance}  

    & & \multicolumn{4}{p{0.6\linewidth}}{\cellcolor[rgb]{0.9,0.9,0.9}{%
	\makecell[{{p{\linewidth}}}]{% 
	  \tt {\tiny [GM$\rangle$B]}  
	 ABLEHNUNG: {'C28', 'C48', 'B66', 'C15', 'A35', 'B10', 'C41'}\\ \tt ARGUMENT: {'Kosten 4072 sind im Rahmen, aber der Tausch A93→C41 verschlechtert meinen Wert deutlich; A93 ist für mich wesentlich effizienter als C41. Kompromissvorschlag: Wir behalten dein Mehrwert-Element, aber anders skaliert: Lass C41 weg, nimm stattdessen A93 wieder auf und füge – um dir entgegenzukommen – C60 hinzu, während wir keinen anderen Kern anrühren. Aufwand: 3690 (Basis) + 836 = 4526, weiterhin klar unter 4915. Das hält meinen starken Block C28/C48/B66 intakt, wahrt deine Kerne C15/A35/B10 und gibt dir den zusätzlichen C60-Schub.'}\\ \tt VORSCHLAG: {'C28', 'C48', 'B66', 'C15', 'A35', 'B10', 'A93', 'C60'} 
	  } 
	   } 
	   } 
	 & \\ 
 

    \theutterance \stepcounter{utterance}  

    & & & \multicolumn{4}{p{0.6\linewidth}}{\cellcolor[rgb]{1,0.85,0.72}{%
	\makecell[{{p{\linewidth}}}]{% 
	  \tt {\tiny [GM$\langle$B]}  
	 ABLEHNUNG: {'C28', 'C48', 'B66', 'C15', 'A35', 'B10', 'A93', 'C60'}\\ \tt ARGUMENT: {'Ich schätze dein Entgegenkommen mit C60. Für mich ist das Paket so nah dran, nur C48 bleibt der Knackpunkt. Wenn wir C48 durch C60 ersetzen, steigt mein Wert deutlich, und sonst bleibt alles unverändert – inkl. A93 als effizienter Füller. Aufwand: 4526 - 466 + 836 = 4896, also sicher unter 4915. Das wahrt deinen Block C28/B66/C15/A35/B10/A93 und gibt mir den höheren Nutzen von C60 statt C48.'}\\ \tt VORSCHLAG: {'C28', 'C60', 'B66', 'C15', 'A35', 'B10', 'A93'} 
	  } 
	   } 
	   } 
	  \\ 
 

    \theutterance \stepcounter{utterance}  

    & & \multicolumn{4}{p{0.6\linewidth}}{\cellcolor[rgb]{0.9,0.9,0.9}{%
	\makecell[{{p{\linewidth}}}]{% 
	  \tt {\tiny [A$\langle$GM]}  
	 ABLEHNUNG: {'C28', 'C48', 'B66', 'C15', 'A35', 'B10', 'A93', 'C60'}\\ \tt ARGUMENT: {'Ich schätze dein Entgegenkommen mit C60. Für mich ist das Paket so nah dran, nur C48 bleibt der Knackpunkt. Wenn wir C48 durch C60 ersetzen, steigt mein Wert deutlich, und sonst bleibt alles unverändert – inkl. A93 als effizienter Füller. Aufwand: 4526 - 466 + 836 = 4896, also sicher unter 4915. Das wahrt deinen Block C28/B66/C15/A35/B10/A93 und gibt mir den höheren Nutzen von C60 statt C48.'}\\ \tt VORSCHLAG: {'C28', 'C60', 'B66', 'C15', 'A35', 'B10', 'A93'} 
	  } 
	   } 
	   } 
	 & \\ 
 

    \theutterance \stepcounter{utterance}  

    & \multicolumn{4}{p{0.6\linewidth}}{\cellcolor[rgb]{0.8,1,0.9}{%
	\makecell[{{p{\linewidth}}}]{% 
	  \tt {\tiny [A$\rangle$GM]}  
	 ABLEHNUNG: {'C28', 'C60', 'B66', 'C15', 'A35', 'B10', 'A93'}\\ \tt ARGUMENT: {'Ich verstehe deinen Wunsch nach C60 statt C48, aber das senkt meinen Gesamtwert merklich. Wir sind sehr nah beieinander – lass uns minimal justieren: Behalten wir C48 (nicht C60) und nutzen den verbleibenden Puffer für einen kleinen Zusatz mit gutem Verhältnis. Konkreter Abschlussvorschlag unten hält deine Kerne C15, A35, B10 und meinen Block C28/C48/B66; A93 bleibt als effizienter Füller.'}\\ \tt VORSCHLAG: {'C28', 'C48', 'B66', 'C15', 'A35', 'B10', 'A93'} 
	  } 
	   } 
	   } 
	 & & \\ 
 

    \theutterance \stepcounter{utterance}  

    & & \multicolumn{4}{p{0.6\linewidth}}{\cellcolor[rgb]{0.9,0.9,0.9}{%
	\makecell[{{p{\linewidth}}}]{% 
	  \tt {\tiny [GM$\rangle$B]}  
	 ABLEHNUNG: {'C28', 'C60', 'B66', 'C15', 'A35', 'B10', 'A93'}\\ \tt ARGUMENT: {'Ich verstehe deinen Wunsch nach C60 statt C48, aber das senkt meinen Gesamtwert merklich. Wir sind sehr nah beieinander – lass uns minimal justieren: Behalten wir C48 (nicht C60) und nutzen den verbleibenden Puffer für einen kleinen Zusatz mit gutem Verhältnis. Konkreter Abschlussvorschlag unten hält deine Kerne C15, A35, B10 und meinen Block C28/C48/B66; A93 bleibt als effizienter Füller.'}\\ \tt VORSCHLAG: {'C28', 'C48', 'B66', 'C15', 'A35', 'B10', 'A93'} 
	  } 
	   } 
	   } 
	 & \\ 
 

    \theutterance \stepcounter{utterance}  

    & & & \multicolumn{4}{p{0.6\linewidth}}{\cellcolor[rgb]{1,0.85,0.72}{%
	\makecell[{{p{\linewidth}}}]{% 
	  \tt {\tiny [GM$\langle$B]}  
	 ABLEHNUNG: {'C28', 'C48', 'B66', 'C15', 'A35', 'B10', 'A93'}\\ \tt ARGUMENT: {'Wir sind sehr nah dran. Für mich ist der einzige wirkliche Knackpunkt C48 vs. C60. Wenn wir C48 durch C60 ersetzen, bleibt alles andere exakt wie von dir gewünscht und mein Wert steigt deutlich, ohne dein Paket zu verwässern. Aufwand bleibt mit 4060 klar unter 4915. Das ist eine minimale Änderung für maximalen gemeinsamen Nutzen.'}\\ \tt VORSCHLAG: {'C28', 'C60', 'B66', 'C15', 'A35', 'B10', 'A93'} 
	  } 
	   } 
	   } 
	  \\ 
 

    \theutterance \stepcounter{utterance}  

    & & \multicolumn{4}{p{0.6\linewidth}}{\cellcolor[rgb]{0.9,0.9,0.9}{%
	\makecell[{{p{\linewidth}}}]{% 
	  \tt {\tiny [A$\langle$GM]}  
	 ABLEHNUNG: {'C28', 'C48', 'B66', 'C15', 'A35', 'B10', 'A93'}\\ \tt ARGUMENT: {'Wir sind sehr nah dran. Für mich ist der einzige wirkliche Knackpunkt C48 vs. C60. Wenn wir C48 durch C60 ersetzen, bleibt alles andere exakt wie von dir gewünscht und mein Wert steigt deutlich, ohne dein Paket zu verwässern. Aufwand bleibt mit 4060 klar unter 4915. Das ist eine minimale Änderung für maximalen gemeinsamen Nutzen.'}\\ \tt VORSCHLAG: {'C28', 'C60', 'B66', 'C15', 'A35', 'B10', 'A93'} 
	  } 
	   } 
	   } 
	 & \\ 
 

    \theutterance \stepcounter{utterance}  

    & \multicolumn{4}{p{0.6\linewidth}}{\cellcolor[rgb]{0.8,1,0.9}{%
	\makecell[{{p{\linewidth}}}]{% 
	  \tt {\tiny [A$\rangle$GM]}  
	 ABLEHNUNG: {'C28', 'C60', 'B66', 'C15', 'A35', 'B10', 'A93'}\\ \tt ARGUMENT: {'Der Austausch C48→C60 reduziert meinen Wert deutlich. Wir sind im Prinzip einig, nur dieser Punkt trennt uns. Letzter Kompromiss: Wir behalten C48 und gleichen deinen Wunsch nach zusätzlichem Wert durch einen weiteren leichten, effizienten Posten aus – ohne sonst etwas anzutasten. Damit sichern wir beide Seiten maximal ab und bleiben komfortabel unter dem Limit.'}\\ \tt VORSCHLAG: {'C28', 'C48', 'B66', 'C15', 'A35', 'B10', 'A93'} 
	  } 
	   } 
	   } 
	 & & \\ 
 

    \theutterance \stepcounter{utterance}  

    & & \multicolumn{4}{p{0.6\linewidth}}{\cellcolor[rgb]{0.9,0.9,0.9}{%
	\makecell[{{p{\linewidth}}}]{% 
	  \tt {\tiny [GM$\rangle$B]}  
	 ABLEHNUNG: {'C28', 'C60', 'B66', 'C15', 'A35', 'B10', 'A93'}\\ \tt ARGUMENT: {'Der Austausch C48→C60 reduziert meinen Wert deutlich. Wir sind im Prinzip einig, nur dieser Punkt trennt uns. Letzter Kompromiss: Wir behalten C48 und gleichen deinen Wunsch nach zusätzlichem Wert durch einen weiteren leichten, effizienten Posten aus – ohne sonst etwas anzutasten. Damit sichern wir beide Seiten maximal ab und bleiben komfortabel unter dem Limit.'}\\ \tt VORSCHLAG: {'C28', 'C48', 'B66', 'C15', 'A35', 'B10', 'A93'} 
	  } 
	   } 
	   } 
	 & \\ 
 

    \theutterance \stepcounter{utterance}  

    & & & \multicolumn{4}{p{0.6\linewidth}}{\cellcolor[rgb]{1,0.85,0.72}{%
	\makecell[{{p{\linewidth}}}]{% 
	  \tt {\tiny [GM$\langle$B]}  
	 ABLEHNUNG: {'C28', 'C48', 'B66', 'C15', 'A35', 'B10', 'A93'}\\ \tt ARGUMENT: {'Ich respektiere deinen Wunsch, C48 zu behalten. Damit ich zustimmen kann, brauche ich jedoch einen echten Mehrwert. Minimaler, fairer Ausgleich: Ergänzen wir C41 als leichten Zusatz – alles andere bleibt identisch. Aufwand: 3690 + 463 = 4153, komfortabel unter 4915. So bleibt dein Block unverändert stark, und ich erhalte den notwendigen Zusatznutzen.'}\\ \tt VORSCHLAG: {'C28', 'C48', 'B66', 'C15', 'A35', 'B10', 'A93', 'C41'} 
	  } 
	   } 
	   } 
	  \\ 
 

    \theutterance \stepcounter{utterance}  

    & & \multicolumn{4}{p{0.6\linewidth}}{\cellcolor[rgb]{0.9,0.9,0.9}{%
	\makecell[{{p{\linewidth}}}]{% 
	  \tt {\tiny [A$\langle$GM]}  
	 ABLEHNUNG: {'C28', 'C48', 'B66', 'C15', 'A35', 'B10', 'A93'}\\ \tt ARGUMENT: {'Ich respektiere deinen Wunsch, C48 zu behalten. Damit ich zustimmen kann, brauche ich jedoch einen echten Mehrwert. Minimaler, fairer Ausgleich: Ergänzen wir C41 als leichten Zusatz – alles andere bleibt identisch. Aufwand: 3690 + 463 = 4153, komfortabel unter 4915. So bleibt dein Block unverändert stark, und ich erhalte den notwendigen Zusatznutzen.'}\\ \tt VORSCHLAG: {'C28', 'C48', 'B66', 'C15', 'A35', 'B10', 'A93', 'C41'} 
	  } 
	   } 
	   } 
	 & \\ 
 

    \theutterance \stepcounter{utterance}  

    & \multicolumn{4}{p{0.6\linewidth}}{\cellcolor[rgb]{0.8,1,0.9}{%
	\makecell[{{p{\linewidth}}}]{% 
	  \tt {\tiny [A$\rangle$GM]}  
	 ABLEHNUNG: {'C28', 'C48', 'B66', 'C15', 'A35', 'B10', 'A93', 'C41'}\\ \tt ARGUMENT: {'C41 bringt mir deutlich weniger als der zusätzliche Aufwand rechtfertigt. Wir sind praktisch einig – ich halte an C48 fest, du bekommst deine Kerne inklusive B10 und A93. Als minimaler, fairer Ausgleich schlage ich stattdessen vor, A93 durch C41 zu ersetzen, ohne weitere Änderungen. Aufwand: 3690 - 81 + 463 = 4072. So bleibt dein Wertzuwachs durch C41 erhalten, und mein Paket wird nicht unnötig verwässert.'}\\ \tt VORSCHLAG: {'C28', 'C48', 'B66', 'C15', 'A35', 'B10', 'C41'} 
	  } 
	   } 
	   } 
	 & & \\ 
 

    \theutterance \stepcounter{utterance}  

    & & \multicolumn{4}{p{0.6\linewidth}}{\cellcolor[rgb]{0.9,0.9,0.9}{%
	\makecell[{{p{\linewidth}}}]{% 
	  \tt {\tiny [GM$\rangle$B]}  
	 ABLEHNUNG: {'C28', 'C48', 'B66', 'C15', 'A35', 'B10', 'A93', 'C41'}\\ \tt ARGUMENT: {'C41 bringt mir deutlich weniger als der zusätzliche Aufwand rechtfertigt. Wir sind praktisch einig – ich halte an C48 fest, du bekommst deine Kerne inklusive B10 und A93. Als minimaler, fairer Ausgleich schlage ich stattdessen vor, A93 durch C41 zu ersetzen, ohne weitere Änderungen. Aufwand: 3690 - 81 + 463 = 4072. So bleibt dein Wertzuwachs durch C41 erhalten, und mein Paket wird nicht unnötig verwässert.'}\\ \tt VORSCHLAG: {'C28', 'C48', 'B66', 'C15', 'A35', 'B10', 'C41'} 
	  } 
	   } 
	   } 
	 & \\ 
 

    \theutterance \stepcounter{utterance}  

    & & & \multicolumn{4}{p{0.6\linewidth}}{\cellcolor[rgb]{1,0.85,0.72}{%
	\makecell[{{p{\linewidth}}}]{% 
	  \tt {\tiny [GM$\langle$B]}  
	 ABLEHNUNG: {'C28', 'C48', 'B66', 'C15', 'A35', 'B10', 'C41'}\\ \tt ARGUMENT: {'Ich verstehe deinen Ansatz, aber der Austausch A93→C41 verschlechtert mein Wert/Kosten-Verhältnis spürbar. Wir sind auf 1 Punkt reduziert: ein minimaler Ausgleich für C48. Letzter, klarer Kompromiss: Behalte A93, streiche C41, füge stattdessen A75 hinzu. Aufwand: 3690 - 463 + 81 + 373 = 3681? Korrektur: Basis 3690 + 373 = 4063. Damit bleibt dein Paket nahezu unverändert (C28/C48/B66 + C15/A35/B10 + A93), und ich erhalte einen fairen Zusatznutzen ohne dein Set zu verwässern.'}\\ \tt VORSCHLAG: {'C28', 'C48', 'B66', 'C15', 'A35', 'B10', 'A93', 'A75'} 
	  } 
	   } 
	   } 
	  \\ 
 

    \theutterance \stepcounter{utterance}  

    & & \multicolumn{4}{p{0.6\linewidth}}{\cellcolor[rgb]{0.9,0.9,0.9}{%
	\makecell[{{p{\linewidth}}}]{% 
	  \tt {\tiny [A$\langle$GM]}  
	 ABLEHNUNG: {'C28', 'C48', 'B66', 'C15', 'A35', 'B10', 'C41'}\\ \tt ARGUMENT: {'Ich verstehe deinen Ansatz, aber der Austausch A93→C41 verschlechtert mein Wert/Kosten-Verhältnis spürbar. Wir sind auf 1 Punkt reduziert: ein minimaler Ausgleich für C48. Letzter, klarer Kompromiss: Behalte A93, streiche C41, füge stattdessen A75 hinzu. Aufwand: 3690 - 463 + 81 + 373 = 3681? Korrektur: Basis 3690 + 373 = 4063. Damit bleibt dein Paket nahezu unverändert (C28/C48/B66 + C15/A35/B10 + A93), und ich erhalte einen fairen Zusatznutzen ohne dein Set zu verwässern.'}\\ \tt VORSCHLAG: {'C28', 'C48', 'B66', 'C15', 'A35', 'B10', 'A93', 'A75'} 
	  } 
	   } 
	   } 
	 & \\ 
 

    \theutterance \stepcounter{utterance}  

    & \multicolumn{4}{p{0.6\linewidth}}{\cellcolor[rgb]{0.8,1,0.9}{%
	\makecell[{{p{\linewidth}}}]{% 
	  \tt {\tiny [A$\rangle$GM]}  
	 ABLEHNUNG: {'C28', 'C48', 'B66', 'C15', 'A35', 'B10', 'A93', 'A75'}\\ \tt ARGUMENT: {'Dein Vorschlag ist mit Aufwand 4063 zulässig, aber A75 liefert mir kaum Gegenwert – der zusätzliche Posten verwässert meinen Nutzen. Wir sind sonst vollständig einig. Minimaler, fairer Abschluss: Lass A75 weg, behalte alles andere unverändert. Das Paket bleibt schlank, unter dem Limit und wahrt beiderseitig hohe Wertdichte.'}\\ \tt VORSCHLAG: {'C28', 'C48', 'B66', 'C15', 'A35', 'B10', 'A93'} 
	  } 
	   } 
	   } 
	 & & \\ 
 

    \theutterance \stepcounter{utterance}  

    & & \multicolumn{4}{p{0.6\linewidth}}{\cellcolor[rgb]{0.9,0.9,0.9}{%
	\makecell[{{p{\linewidth}}}]{% 
	  \tt {\tiny [GM$\rangle$B]}  
	 ABLEHNUNG: {'C28', 'C48', 'B66', 'C15', 'A35', 'B10', 'A93', 'A75'}\\ \tt ARGUMENT: {'Dein Vorschlag ist mit Aufwand 4063 zulässig, aber A75 liefert mir kaum Gegenwert – der zusätzliche Posten verwässert meinen Nutzen. Wir sind sonst vollständig einig. Minimaler, fairer Abschluss: Lass A75 weg, behalte alles andere unverändert. Das Paket bleibt schlank, unter dem Limit und wahrt beiderseitig hohe Wertdichte.'}\\ \tt VORSCHLAG: {'C28', 'C48', 'B66', 'C15', 'A35', 'B10', 'A93'} 
	  } 
	   } 
	   } 
	 & \\ 
 

    \theutterance \stepcounter{utterance}  

    & & & \multicolumn{4}{p{0.6\linewidth}}{\cellcolor[rgb]{1,0.85,0.72}{%
	\makecell[{{p{\linewidth}}}]{% 
	  \tt {\tiny [GM$\langle$B]}  
	 ARGUMENT: {'Wir sind wirklich bei einem einzigen Punkt. Damit ich C48 akzeptiere, brauche ich einen sehr kleinen, aber spürbaren Ausgleich, der deinen Wert kaum beeinträchtigt. A93 bleibt für dich effizient – einverstanden. Minimaloption: Ergänzen wir stattdessen C41 nicht; nimm bitte C15, A35, B10, C28, B66, A93 plus genau einen meiner leichten Favoriten C41 ODER, falls das für dich zu viel ist, tauschen wir A93→C41 nur temporär. Ich bevorzuge die Variante mit beiden, aber gebe dir die Wahl für den minimalen Zuschlag.'}\\ \tt VORSCHLAG: {'C28', 'C48', 'B66', 'C15', 'A35', 'B10', 'A93', 'C41'} 
	  } 
	   } 
	   } 
	  \\ 
 

    \theutterance \stepcounter{utterance}  

    & & \multicolumn{4}{p{0.6\linewidth}}{\cellcolor[rgb]{0.9,0.9,0.9}{%
	\makecell[{{p{\linewidth}}}]{% 
	  \tt {\tiny [A$\langle$GM]}  
	 ARGUMENT: {'Wir sind wirklich bei einem einzigen Punkt. Damit ich C48 akzeptiere, brauche ich einen sehr kleinen, aber spürbaren Ausgleich, der deinen Wert kaum beeinträchtigt. A93 bleibt für dich effizient – einverstanden. Minimaloption: Ergänzen wir stattdessen C41 nicht; nimm bitte C15, A35, B10, C28, B66, A93 plus genau einen meiner leichten Favoriten C41 ODER, falls das für dich zu viel ist, tauschen wir A93→C41 nur temporär. Ich bevorzuge die Variante mit beiden, aber gebe dir die Wahl für den minimalen Zuschlag.'}\\ \tt VORSCHLAG: {'C28', 'C48', 'B66', 'C15', 'A35', 'B10', 'A93', 'C41'} 
	  } 
	   } 
	   } 
	 & \\ 
 

    \theutterance \stepcounter{utterance}  

    & \multicolumn{4}{p{0.6\linewidth}}{\cellcolor[rgb]{0.8,1,0.9}{%
	\makecell[{{p{\linewidth}}}]{% 
	  \tt {\tiny [A$\rangle$GM]}  
	 ABLEHNUNG: {'C28', 'C48', 'B66', 'C15', 'A35', 'B10', 'A93', 'C41'}\\ \tt ARGUMENT: {'Ich kann C41 nicht sinnvoll rechtfertigen – der Zusatz senkt meine Wertdichte. Du bekommst C48 plus deine Kerne C15, A35, B10 und den effizienten A93. Das ist bereits ein starkes Entgegenkommen. Lass uns es schlank und fair halten und ohne weitere Zusätze abschließen.'}\\ \tt VORSCHLAG: {'C28', 'C48', 'B66', 'C15', 'A35', 'B10', 'A93'} 
	  } 
	   } 
	   } 
	 & & \\ 
 

    \theutterance \stepcounter{utterance}  

    & & \multicolumn{4}{p{0.6\linewidth}}{\cellcolor[rgb]{0.9,0.9,0.9}{%
	\makecell[{{p{\linewidth}}}]{% 
	  \tt {\tiny [GM$\rangle$B]}  
	 ABLEHNUNG: {'C28', 'C48', 'B66', 'C15', 'A35', 'B10', 'A93', 'C41'}\\ \tt ARGUMENT: {'Ich kann C41 nicht sinnvoll rechtfertigen – der Zusatz senkt meine Wertdichte. Du bekommst C48 plus deine Kerne C15, A35, B10 und den effizienten A93. Das ist bereits ein starkes Entgegenkommen. Lass uns es schlank und fair halten und ohne weitere Zusätze abschließen.'}\\ \tt VORSCHLAG: {'C28', 'C48', 'B66', 'C15', 'A35', 'B10', 'A93'} 
	  } 
	   } 
	   } 
	 & \\ 
 

    \theutterance \stepcounter{utterance}  

    & & & \multicolumn{4}{p{0.6\linewidth}}{\cellcolor[rgb]{1,0.85,0.72}{%
	\makecell[{{p{\linewidth}}}]{% 
	  \tt {\tiny [GM$\langle$B]}  
	 ZUSTIMMUNG: {'C28', 'C48', 'B66', 'C15', 'A35', 'B10', 'A93'}\\ \tt ARGUMENT: {'Einverstanden. Dieses Paket liegt mit Aufwand 3690 klar unter dem Limit und enthält deine Kernwünsche sowie eine starke, ausgewogene Zusammenstellung. Danke für die faire Einigung.'} 
	  } 
	   } 
	   } 
	  \\ 
 

    \theutterance \stepcounter{utterance}  

    & & & \multicolumn{2}{p{0.3\linewidth}}{\cellcolor[rgb]{0.95,0.95,0.95}{%
	\makecell[{{p{\linewidth}}}]{% 
	  \tt {\tiny [GM$|$GM]}  
	 game successful 
	  } 
	   } 
	   } 
	 & & \\ 
 

    \theutterance \stepcounter{utterance}  

    & & & \multicolumn{2}{p{0.3\linewidth}}{\cellcolor[rgb]{0.95,0.95,0.95}{%
	\makecell[{{p{\linewidth}}}]{% 
	  \tt {\tiny [GM$|$GM]}  
	 end game 
	  } 
	   } 
	   } 
	 & & \\ 
 

\end{supertabular}
}

\end{document}
