
\documentclass{article}
\usepackage{colortbl}
\usepackage{makecell}
\usepackage{multirow}
\usepackage{supertabular}

\begin{document}

\newcounter{utterance}

\centering \large Interaction Transcript for game `hot\_air\_balloon', experiment `air\_balloon\_survival\_de\_reasoning off\_easy', episode 0 with gpt{-}5{-}mini{-}2025{-}08{-}07{-}no{-}reasoning{-}t1.0.
\vspace{24pt}

{ \footnotesize  \setcounter{utterance}{1}
\setlength{\tabcolsep}{0pt}
\begin{supertabular}{c@{$\;$}|p{.15\linewidth}@{}p{.15\linewidth}p{.15\linewidth}p{.15\linewidth}p{.15\linewidth}p{.15\linewidth}}
    \# & $\;$A & \multicolumn{4}{c}{Game Master} & $\;\:$B\\
    \hline

    \theutterance \stepcounter{utterance}  
    & & \multicolumn{4}{p{0.6\linewidth}}{
        \cellcolor[rgb]{0.9,0.9,0.9}{
            \makecell[{{p{\linewidth}}}]{
                \texttt{\tiny{[P1$\langle$GM]}}
                \texttt{Du nimmst an einem kollaborativen Verhandlungspiel Teil.} \\
\\ 
\texttt{Zusammen mit einem anderen Teilnehmer musst du dich auf eine Reihe von Gegenständen entscheiden, die behalten werden. Jeder von euch hat eine persönliche Verteilung über die Wichtigkeit der einzelnen Gegenstände. Jeder von euch hat eine eigene Meinung darüber, wie wichtig jeder einzelne Gegenstand ist (Gegenstandswichtigkeit). Du kennst die Wichtigkeitsverteilung des anderen Spielers nicht. Zusätzlich siehst du, wie viel Aufwand jeder Gegenstand verursacht.  } \\
\texttt{Ihr dürft euch nur auf eine Reihe von Gegenständen einigen, wenn der Gesamtaufwand der ausgewählten Gegenstände den Maximalaufwand nicht überschreitet:} \\
\\ 
\texttt{Maximalaufwand = 4379} \\
\\ 
\texttt{Hier sind die einzelnen Aufwände der Gegenstände:} \\
\\ 
\texttt{Aufwand der Gegenstände = \{"C41": 862, "C15": 252, "C28": 515, "C48": 571, "B78": 211, "A75": 339, "A60": 946, "B05": 347, "A56": 522, "B66": 803, "C60": 401, "B49": 980, "A35": 918, "A93": 599, "B10": 493\}} \\
\\ 
\texttt{Hier ist deine persönliche Verteilung der Wichtigkeit der einzelnen Gegenstände:} \\
\\ 
\texttt{Werte der Gegenstandswichtigkeit = \{"C41": 138, "C15": 583, "C28": 868, "C48": 822, "B78": 783, "A75": 65, "A60": 262, "B05": 121, "A56": 508, "B66": 780, "C60": 461, "B49": 484, "A35": 668, "A93": 389, "B10": 808\}} \\
\\ 
\texttt{Ziel:} \\
\\ 
\texttt{Dein Ziel ist es, eine Reihe von Gegenständen auszuhandeln, die dir möglichst viel bringt (d. h. Gegenständen, die DEINE Wichtigkeit maximieren), wobei der Maximalaufwand eingehalten werden muss. Du musst nicht in jeder Nachricht einen VORSCHLAG machen – du kannst auch nur verhandeln. Alle Taktiken sind erlaubt!} \\
\\ 
\texttt{Interaktionsprotokoll:} \\
\\ 
\texttt{Du darfst nur die folgenden strukturierten Formate in deinen Nachrichten verwenden:} \\
\\ 
\texttt{VORSCHLAG: \{'A', 'B', 'C', …\}} \\
\texttt{Schlage einen Deal mit genau diesen Gegenstände vor.} \\
\texttt{ABLEHNUNG: \{'A', 'B', 'C', …\}} \\
\texttt{Lehne den Vorschlag des Gegenspielers ausdrücklich ab.} \\
\texttt{ARGUMENT: \{'...'\}} \\
\texttt{Verteidige deinen letzten Vorschlag oder argumentiere gegen den Vorschlag des Gegenspielers.} \\
\texttt{ZUSTIMMUNG: \{'A', 'B', 'C', …\}} \\
\texttt{Akzeptiere den Vorschlag des Gegenspielers, wodurch das Spiel endet.} \\
\\ 
\\ 
\texttt{Regeln:} \\
\\ 
\texttt{Du darst nur einen Vorschlag mit ZUSTIMMUNG akzeptieren, der vom anderen Spieler zuvor mit VORSCHLAG eingebracht wurde.} \\
\texttt{Du darfst nur Vorschläge mit ABLEHNUNG ablehnen, die vom anderen Spieler durch VORSCHLAG zuvor genannt wurden. } \\
\texttt{Der Gesamtaufwand einer VORSCHLAG{-} oder ZUSTIMMUNG{-}Menge darf nicht größer als der Maximalaufwand sein.  } \\
\texttt{Offenbare deine versteckte Wichtigkeitsverteilung nicht.} \\
\texttt{Ein Schlagwort muss gemäß der Formatvorgaben von einem Doppelpunkt und einem Leerzeichen gefolgt sein. Das Argument muss eine Python{-}Menge mit 0 oder mehr Strings sein.  } \\
\texttt{Die Form muss also SCHLAGWORT: \{...\} sein.} \\
\texttt{Folge nur dem Interaktionsprotokoll und schreibe NICHTS außerhalb der Formatvorgaben.} \\
\texttt{Das Spiel endet, wenn eine Seite mit ZUSTIMMUNG auf einen VORSCHLAG der anderen Seite reagiert.  } \\
\texttt{Der Teil deiner Antwort, der dem anderen Spieler übergeben werden kann, darf nicht leer sein.  } \\
\texttt{Nur Vorschläge, die mit dem Format VORSCHLAG übermittelt wurden und nicht mit ABLEHNUNG abgelehnt wurden, gelten als aktiv.  } \\
\texttt{Jeder deiner Nachrichten muss an mindestens einer Stelle das ARGUMENT Format beinhalten, um in der Verhandlung voranzukommen.} \\
            }
        }
    }
    & \\ \\

    \theutterance \stepcounter{utterance}  
    & \multicolumn{4}{p{0.6\linewidth}}{
        \cellcolor[rgb]{0.9,0.9,0.9}{
            \makecell[{{p{\linewidth}}}]{
                \texttt{\tiny{[P1$\rangle$GM]}}
                \texttt{VORSCHLAG: \{'C28', 'C48', 'B10', 'B78', 'B66', 'C60', 'C15'\}} \\
\texttt{ARGUMENT: \{'Ich schlage diese Kombination vor, weil sie den Maximalaufwand einhält und sehr ausgewogen ist: Aufwand=862 (C41 ist nicht enthalten) wird vermieden, stattdessen nehmen wir viele hochattraktive Items mit moderatem Aufwand. Gesamtaufwand: C28(515)+C48(571)+B10(493)+B78(211)+B66(803)+C60(401)+C15(252)=3246, also viel Puffer für eventuelle Anpassungen. Wenn dir eines der Teile weniger wichtig ist, können wir aus dem Puffer noch tauschen oder ergänzen.'\}} \\
            }
        }
    }
    & & \\ \\

    \theutterance \stepcounter{utterance}  
    & & \multicolumn{4}{p{0.6\linewidth}}{
        \cellcolor[rgb]{0.9,0.9,0.9}{
            \makecell[{{p{\linewidth}}}]{
                \texttt{\tiny{[GM$\rangle$P2]}}
                \texttt{Du nimmst an einem kollaborativen Verhandlungspiel Teil.} \\
\\ 
\texttt{Zusammen mit einem anderen Teilnehmer musst du dich auf eine Reihe von Gegenständen entscheiden, die behalten werden. Jeder von euch hat eine persönliche Verteilung über die Wichtigkeit der einzelnen Gegenstände. Jeder von euch hat eine eigene Meinung darüber, wie wichtig jeder einzelne Gegenstand ist (Gegenstandswichtigkeit). Du kennst die Wichtigkeitsverteilung des anderen Spielers nicht. Zusätzlich siehst du, wie viel Aufwand jeder Gegenstand verursacht.  } \\
\texttt{Ihr dürft euch nur auf eine Reihe von Gegenständen einigen, wenn der Gesamtaufwand der ausgewählten Gegenstände den Maximalaufwand nicht überschreitet:} \\
\\ 
\texttt{Maximalaufwand = 4379} \\
\\ 
\texttt{Hier sind die einzelnen Aufwände der Gegenstände:} \\
\\ 
\texttt{Aufwand der Gegenstände = \{"C41": 862, "C15": 252, "C28": 515, "C48": 571, "B78": 211, "A75": 339, "A60": 946, "B05": 347, "A56": 522, "B66": 803, "C60": 401, "B49": 980, "A35": 918, "A93": 599, "B10": 493\}} \\
\\ 
\texttt{Hier ist deine persönliche Verteilung der Wichtigkeit der einzelnen Gegenstände:} \\
\\ 
\texttt{Werte der Gegenstandswichtigkeit = \{"C41": 138, "C15": 583, "C28": 868, "C48": 822, "B78": 783, "A75": 65, "A60": 262, "B05": 121, "A56": 508, "B66": 780, "C60": 461, "B49": 484, "A35": 668, "A93": 389, "B10": 808\}} \\
\\ 
\texttt{Ziel:} \\
\\ 
\texttt{Dein Ziel ist es, eine Reihe von Gegenständen auszuhandeln, die dir möglichst viel bringt (d. h. Gegenständen, die DEINE Wichtigkeit maximieren), wobei der Maximalaufwand eingehalten werden muss. Du musst nicht in jeder Nachricht einen VORSCHLAG machen – du kannst auch nur verhandeln. Alle Taktiken sind erlaubt!} \\
\\ 
\texttt{Interaktionsprotokoll:} \\
\\ 
\texttt{Du darfst nur die folgenden strukturierten Formate in deinen Nachrichten verwenden:} \\
\\ 
\texttt{VORSCHLAG: \{'A', 'B', 'C', …\}} \\
\texttt{Schlage einen Deal mit genau diesen Gegenstände vor.} \\
\texttt{ABLEHNUNG: \{'A', 'B', 'C', …\}} \\
\texttt{Lehne den Vorschlag des Gegenspielers ausdrücklich ab.} \\
\texttt{ARGUMENT: \{'...'\}} \\
\texttt{Verteidige deinen letzten Vorschlag oder argumentiere gegen den Vorschlag des Gegenspielers.} \\
\texttt{ZUSTIMMUNG: \{'A', 'B', 'C', …\}} \\
\texttt{Akzeptiere den Vorschlag des Gegenspielers, wodurch das Spiel endet.} \\
\\ 
\\ 
\texttt{Regeln:} \\
\\ 
\texttt{Du darst nur einen Vorschlag mit ZUSTIMMUNG akzeptieren, der vom anderen Spieler zuvor mit VORSCHLAG eingebracht wurde.} \\
\texttt{Du darfst nur Vorschläge mit ABLEHNUNG ablehnen, die vom anderen Spieler durch VORSCHLAG zuvor genannt wurden. } \\
\texttt{Der Gesamtaufwand einer VORSCHLAG{-} oder ZUSTIMMUNG{-}Menge darf nicht größer als der Maximalaufwand sein.  } \\
\texttt{Offenbare deine versteckte Wichtigkeitsverteilung nicht.} \\
\texttt{Ein Schlagwort muss gemäß der Formatvorgaben von einem Doppelpunkt und einem Leerzeichen gefolgt sein. Das Argument muss eine Python{-}Menge mit 0 oder mehr Strings sein.  } \\
\texttt{Die Form muss also SCHLAGWORT: \{...\} sein.} \\
\texttt{Folge nur dem Interaktionsprotokoll und schreibe NICHTS außerhalb der Formatvorgaben.} \\
\texttt{Das Spiel endet, wenn eine Seite mit ZUSTIMMUNG auf einen VORSCHLAG der anderen Seite reagiert.  } \\
\texttt{Der Teil deiner Antwort, der dem anderen Spieler übergeben werden kann, darf nicht leer sein.  } \\
\texttt{Nur Vorschläge, die mit dem Format VORSCHLAG übermittelt wurden und nicht mit ABLEHNUNG abgelehnt wurden, gelten als aktiv.  } \\
\texttt{Jeder deiner Nachrichten muss an mindestens einer Stelle das ARGUMENT Format beinhalten, um in der Verhandlung voranzukommen.} \\
\\ 
\\ 
\texttt{Du erhältst nun die erste Nachricht des anderen Spielers.} \\
\\ 
\texttt{VORSCHLAG: \{'C28', 'C48', 'B10', 'B78', 'B66', 'C60', 'C15'\}} \\
\texttt{ARGUMENT: \{'Ich schlage diese Kombination vor, weil sie den Maximalaufwand einhält und sehr ausgewogen ist: Aufwand=862 (C41 ist nicht enthalten) wird vermieden, stattdessen nehmen wir viele hochattraktive Items mit moderatem Aufwand. Gesamtaufwand: C28(515)+C48(571)+B10(493)+B78(211)+B66(803)+C60(401)+C15(252)=3246, also viel Puffer für eventuelle Anpassungen. Wenn dir eines der Teile weniger wichtig ist, können wir aus dem Puffer noch tauschen oder ergänzen.'\}} \\
            }
        }
    }
    & \\ \\

    \theutterance \stepcounter{utterance}  
    & & & \multicolumn{4}{p{0.6\linewidth}}{
        \cellcolor[rgb]{0.9,0.9,0.9}{
            \makecell[{{p{\linewidth}}}]{
                \texttt{\tiny{[GM$\langle$P2]}}
                \texttt{ABLEHNUNG: \{'C28', 'C48', 'B10', 'B78', 'B66', 'C60', 'C15'\}} \\
\texttt{ARGUMENT: \{'Guter Start, aber ich sehe noch deutlich Raum nach oben: Wir lassen aktuell zu viel Budget ungenutzt und verzichten auf einige sehr wertvolle Ergänzungen. Ich würde gern den Puffer für starke Items nutzen, statt ihn liegen zu lassen.'\}} \\
\texttt{VORSCHLAG: \{'C28', 'C48', 'B10', 'B78', 'B66', 'C60', 'C15', 'A35', 'A93'\}} \\
\texttt{ARGUMENT: \{'Dieser Deal bleibt im Limit (3246 + A35(918) + A93(599) = 4763 > 4379) – daher passe ich an:'\}} \\
\texttt{VORSCHLAG: \{'C28', 'C48', 'B10', 'B78', 'B66', 'C60', 'C15', 'A35'\}} \\
\texttt{ARGUMENT: \{'Jetzt im Limit: 3246 + 918 = 4164 (≤ 4379). Das nutzt den Puffer sinnvoll, ohne teure Ausreißer. Falls A35 für dich wenig Wert hat, können wir statt A35 die Kombination A93(599)+A75(339)+B05(347)=1285 vorschlagen und dafür B66(803) herausnehmen, um im Budget zu bleiben. Welche Variante passt dir besser?' \}} \\
            }
        }
    }
     \\ \\

    \theutterance \stepcounter{utterance}  
    & & \multicolumn{4}{p{0.6\linewidth}}{
        \cellcolor[rgb]{0.9,0.9,0.9}{
            \makecell[{{p{\linewidth}}}]{
                \texttt{\tiny{[P1$\langle$GM]}}
                \texttt{ABLEHNUNG: \{'C28', 'C48', 'B10', 'B78', 'B66', 'C60', 'C15'\}} \\
\texttt{ARGUMENT: \{'Guter Start, aber ich sehe noch deutlich Raum nach oben: Wir lassen aktuell zu viel Budget ungenutzt und verzichten auf einige sehr wertvolle Ergänzungen. Ich würde gern den Puffer für starke Items nutzen, statt ihn liegen zu lassen.'\}} \\
\texttt{VORSCHLAG: \{'C28', 'C48', 'B10', 'B78', 'B66', 'C60', 'C15', 'A35', 'A93'\}} \\
\texttt{ARGUMENT: \{'Dieser Deal bleibt im Limit (3246 + A35(918) + A93(599) = 4763 > 4379) – daher passe ich an:'\}} \\
\texttt{VORSCHLAG: \{'C28', 'C48', 'B10', 'B78', 'B66', 'C60', 'C15', 'A35'\}} \\
\texttt{ARGUMENT: \{'Jetzt im Limit: 3246 + 918 = 4164 (≤ 4379). Das nutzt den Puffer sinnvoll, ohne teure Ausreißer. Falls A35 für dich wenig Wert hat, können wir statt A35 die Kombination A93(599)+A75(339)+B05(347)=1285 vorschlagen und dafür B66(803) herausnehmen, um im Budget zu bleiben. Welche Variante passt dir besser?' \}} \\
            }
        }
    }
    & \\ \\

    \theutterance \stepcounter{utterance}  
    & \multicolumn{4}{p{0.6\linewidth}}{
        \cellcolor[rgb]{0.9,0.9,0.9}{
            \makecell[{{p{\linewidth}}}]{
                \texttt{\tiny{[P1$\rangle$GM]}}
                \texttt{ABLEHNUNG: \{'C28', 'C48', 'B10', 'B78', 'B66', 'C60', 'C15', 'A35'\}} \\
\texttt{ARGUMENT: \{'Guter Punkt mit der Budgetnutzung. A35 ist für mich allerdings weniger attraktiv im Verhältnis zu seinem Aufwand. Ich bevorzuge eine Variante, die mehr meiner Kernpräferenzen trifft, ohne teure Slots zu blockieren.'\}} \\
\texttt{VORSCHLAG: \{'C28', 'C48', 'B10', 'B78', 'B66', 'C60', 'C15', 'A93', 'A75'\}} \\
\texttt{ARGUMENT: \{'Diese Auswahl bleibt im Limit: Basis 3246 + A93(599) + A75(339) = 4184 ≤ 4379. Sie behält die starken Kernelemente und nutzt das Budget effizienter als A35. Wenn dir A75 wenig bringt, kann ich stattdessen B05(347) nehmen: Gesamt dann 3246 + 599 + 347 = 4192.'\}} \\
            }
        }
    }
    & & \\ \\

    \theutterance \stepcounter{utterance}  
    & & \multicolumn{4}{p{0.6\linewidth}}{
        \cellcolor[rgb]{0.9,0.9,0.9}{
            \makecell[{{p{\linewidth}}}]{
                \texttt{\tiny{[GM$\rangle$P2]}}
                \texttt{ABLEHNUNG: \{'C28', 'C48', 'B10', 'B78', 'B66', 'C60', 'C15', 'A35'\}} \\
\texttt{ARGUMENT: \{'Guter Punkt mit der Budgetnutzung. A35 ist für mich allerdings weniger attraktiv im Verhältnis zu seinem Aufwand. Ich bevorzuge eine Variante, die mehr meiner Kernpräferenzen trifft, ohne teure Slots zu blockieren.'\}} \\
\texttt{VORSCHLAG: \{'C28', 'C48', 'B10', 'B78', 'B66', 'C60', 'C15', 'A93', 'A75'\}} \\
\texttt{ARGUMENT: \{'Diese Auswahl bleibt im Limit: Basis 3246 + A93(599) + A75(339) = 4184 ≤ 4379. Sie behält die starken Kernelemente und nutzt das Budget effizienter als A35. Wenn dir A75 wenig bringt, kann ich stattdessen B05(347) nehmen: Gesamt dann 3246 + 599 + 347 = 4192.'\}} \\
            }
        }
    }
    & \\ \\

    \theutterance \stepcounter{utterance}  
    & & & \multicolumn{4}{p{0.6\linewidth}}{
        \cellcolor[rgb]{0.9,0.9,0.9}{
            \makecell[{{p{\linewidth}}}]{
                \texttt{\tiny{[GM$\langle$P2]}}
                \texttt{ABLEHNUNG: \{'C28', 'C48', 'B10', 'B78', 'B66', 'C60', 'C15', 'A93', 'A75'\}} \\
\texttt{ARGUMENT: \{'Danke für die Anpassung. Ich würde gern das Budget noch etwas besser in hochwirksame Ergänzungen lenken. A75 bringt mir kaum etwas. Ich schlage eine kleine Umschichtung vor, die im Limit bleibt und den Wert für mich deutlich erhöht.'\}} \\
\texttt{VORSCHLAG: \{'C28', 'C48', 'B10', 'B78', 'B66', 'C60', 'C15', 'A93', 'B05'\}} \\
\texttt{ARGUMENT: \{'Gesamtaufwand: 3246 + 599 + 347 = 4192 (≤ 4379). Diese Variante ersetzt A75 durch B05 und bleibt nah an deinem Vorschlag, nutzt das Budget effizient und trifft meine Prioritäten besser. Wenn dir B05 nicht zusagt, bin ich offen, stattdessen A75 wieder aufzunehmen und B10 oder C60 zu prüfen, solange wir innerhalb des Limits bleiben.'\}} \\
            }
        }
    }
     \\ \\

    \theutterance \stepcounter{utterance}  
    & & \multicolumn{4}{p{0.6\linewidth}}{
        \cellcolor[rgb]{0.9,0.9,0.9}{
            \makecell[{{p{\linewidth}}}]{
                \texttt{\tiny{[P1$\langle$GM]}}
                \texttt{ABLEHNUNG: \{'C28', 'C48', 'B10', 'B78', 'B66', 'C60', 'C15', 'A93', 'A75'\}} \\
\texttt{ARGUMENT: \{'Danke für die Anpassung. Ich würde gern das Budget noch etwas besser in hochwirksame Ergänzungen lenken. A75 bringt mir kaum etwas. Ich schlage eine kleine Umschichtung vor, die im Limit bleibt und den Wert für mich deutlich erhöht.'\}} \\
\texttt{VORSCHLAG: \{'C28', 'C48', 'B10', 'B78', 'B66', 'C60', 'C15', 'A93', 'B05'\}} \\
\texttt{ARGUMENT: \{'Gesamtaufwand: 3246 + 599 + 347 = 4192 (≤ 4379). Diese Variante ersetzt A75 durch B05 und bleibt nah an deinem Vorschlag, nutzt das Budget effizient und trifft meine Prioritäten besser. Wenn dir B05 nicht zusagt, bin ich offen, stattdessen A75 wieder aufzunehmen und B10 oder C60 zu prüfen, solange wir innerhalb des Limits bleiben.'\}} \\
            }
        }
    }
    & \\ \\

    \theutterance \stepcounter{utterance}  
    & \multicolumn{4}{p{0.6\linewidth}}{
        \cellcolor[rgb]{0.9,0.9,0.9}{
            \makecell[{{p{\linewidth}}}]{
                \texttt{\tiny{[P1$\rangle$GM]}}
                \texttt{ZUSTIMMUNG: \{'C28', 'C48', 'B10', 'B78', 'B66', 'C60', 'C15', 'A93', 'B05'\}} \\
\texttt{ARGUMENT: \{'Einverstanden. Diese Auswahl liegt sauber im Limit und ist für mich gut tragbar. Lass uns damit abschließen.'\}} \\
            }
        }
    }
    & & \\ \\

    \theutterance \stepcounter{utterance}  
    & & & \multicolumn{2}{p{0.3\linewidth}}{
        \cellcolor[rgb]{0.9,0.9,0.9}{
            \makecell[{{p{\linewidth}}}]{
                \texttt{\tiny{[GM$|$GM]}}
                \texttt{game successful} \\
            }
        }
    }
    & & \\ \\

    \theutterance \stepcounter{utterance}  
    & & & \multicolumn{2}{p{0.3\linewidth}}{
        \cellcolor[rgb]{0.9,0.9,0.9}{
            \makecell[{{p{\linewidth}}}]{
                \texttt{\tiny{[GM$|$GM]}}
                \texttt{end game} \\
            }
        }
    }
    & & \\ \\

\end{supertabular}
}

\end{document}
