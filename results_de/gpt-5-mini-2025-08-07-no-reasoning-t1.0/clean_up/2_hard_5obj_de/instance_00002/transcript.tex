
\documentclass{article}
\usepackage{colortbl}
\usepackage{makecell}
\usepackage{multirow}
\usepackage{supertabular}

\begin{document}

\newcounter{utterance}

\centering \large Interaction Transcript for game `clean\_up', experiment `2\_hard\_5obj\_de', episode 2 with gpt{-}5{-}mini{-}2025{-}08{-}07{-}no{-}reasoning{-}t1.0.
\vspace{24pt}

{ \footnotesize  \setcounter{utterance}{1}
\setlength{\tabcolsep}{0pt}
\begin{supertabular}{c@{$\;$}|p{.15\linewidth}@{}p{.15\linewidth}p{.15\linewidth}p{.15\linewidth}p{.15\linewidth}p{.15\linewidth}}
    \# & $\;$A & \multicolumn{4}{c}{Game Master} & $\;\:$B\\
    \hline

    \theutterance \stepcounter{utterance}  
    & & \multicolumn{4}{p{0.6\linewidth}}{
        \cellcolor[rgb]{0.9,0.9,0.9}{
            \makecell[{{p{\linewidth}}}]{
                \texttt{\tiny{[P1$\langle$GM]}}
                \texttt{Ich bin der Spielleiter, und du spielst ein kooperatives Spiel mit dem folgenden Raster als Spielbrett:} \\
\\ 
\texttt{```} \\
\texttt{ 1234567} \\
\texttt{╔═╤═╤═╤═╗} \\
\texttt{║T│◌│◌│◌║ 1} \\
\texttt{╟─┼─┤I│◌║ 2} \\
\texttt{║W│◌│◌│H║ 3} \\
\texttt{║◌│◌├─┴─╢ 4} \\
\texttt{╟─┴─┤◌◌◌║ 5} \\
\texttt{║◌C◌└─┬─╢ 6} \\
\texttt{║◌◌◌◌◌│◌║ 7} \\
\texttt{╚═════╧═╝} \\
\\ 
\texttt{```} \\
\\ 
\texttt{* Am oberen Rand sind die x{-}Koordinaten angegeben, die nach rechts hin zunehmen, und am rechten Rand die y{-}Koordinaten, die nach unten zunehmen.} \\
\texttt{* Die folgenden Objekte sind zufällig auf deinem Raster platziert: 'W', 'I', 'T', 'C', 'H'.} \\
\\ 
\\ 
\texttt{Dein Mitspieler sieht eine Variation des Spielbretts, auf dem dieselben Objekte an anderen zufälligen Positionen platziert sind. Du kannst das Spielbrett des anderen Spielers nicht sehen, und er kann deins nicht sehen.} \\
\\ 
\\ 
\texttt{**Ziel des Spiels:**} \\
\\ 
\texttt{Beide Spieler müssen die Objekte auf ihrem jeweiligen Spielbrett so bewegen, dass identische Objekte am Ende an denselben Koordinaten liegen. Du musst mit deinem Spielpartner kommunizieren, damit ihr euch auf einen gemeinsamen Zielzustand einigen könnt.} \\
\\ 
\\ 
\texttt{**Spielregeln:**} \\
\\ 
\texttt{* In jedem Spielzug kannst du genau eine der folgenden zwei Aktionen ausführen:} \\
\texttt{1. `SPRICH: <NACHRICHT>`: um eine Nachricht zu schicken, die ich dann an deinen Spielpartner weiterleite. Als Nachricht gilt alles bis zum nächsten Zeilenumbruch.} \\
\texttt{2. `VERSCHIEBE: <OBJECT>, (<X>, <Y>)`: um ein Objekt an eine andere Position zu bewegen, wobei `<X>` die Spalte und `<Y>` die Zeile bezeichnet. Ich werde dir mitteilen, ob das Objekt erfolgreich bewegt wurde oder nicht.} \\
\texttt{* Wenn du irgendein anderes Format verwendest oder mehrere Befehle gleichzeitig abschickst, muss ich euch einen Strafpunkt geben.} \\
\texttt{* Wenn ihr gemeinsam mehr als 12 Strafpunkte ansammelt, verliert ihr das Spiel.} \\
\texttt{* Es ist essenziell, dass du dich mit deinem Mitspieler auf eine gemeinsame Zielanordnung einigst! Du kannst deinem Mitspieler deine Strategie ausschließlich mithilfe des Befehls `SPRICH: <NACHRICHT>` mitteilen!} \\
\\ 
\\ 
\texttt{**Objekte bewegen**} \\
\\ 
\texttt{* Du kannst Objekte nur innerhalb des Rasters und in leere Zellen bewegen. Leere Zellen erkennst am Zeichen '◌'.} \\
\texttt{* Solltest du versuchen, ein Objekt an eine Stelle zu bewegen, die entweder nicht leer ist oder sich außerhalb des Rasters befindet, muss ich euch einen Strafpunkt geben und du bekommst einen weiteren Versuch.} \\
\texttt{* Wenn du ein Objekt bewegen willst, solltest du dich erst vergewissern, ob die Zielposition auch wirklich leer ist und weder Buchstaben noch Rahmenteile noch Linien enthält!} \\
\\ 
\\ 
\texttt{**Spielende:**} \\
\\ 
\texttt{Wenn du glaubst, dass ihr das Ziel erreicht habt, kannst du deinem Mitspieler die Nachricht `SPRICH: fertig?` schicken. Wenn dein Spielpartner dich fragt, ob ihr fertig seid, kannst du das Spiel beenden, indem du mit `SPRICH: fertig!` antwortest.} \\
\\ 
\texttt{Ihr gewinnt gemeinsam, wenn ihr das Spiel innerhalb von \#max\_rounds Runden beendet. Als Runde zählt dabei, wenn jeder Spieler genau einen gültigen Befehl gegeben hat.} \\
\\ 
\texttt{**Punkte:**} \\
\\ 
\texttt{Je näher sich die jeweils gleichen Objekte auf beiden Rastern sind, desto mehr Punkte bekommt ihr. Für Strafpunkte gibt es Punktabzug. Könnt ihr den Rekord brechen?} \\
\\ 
\texttt{Schicke eine Nachricht an deinen Mitspieler um das Spiel zu starten!} \\
            }
        }
    }
    & \\ \\

    \theutterance \stepcounter{utterance}  
    & \multicolumn{4}{p{0.6\linewidth}}{
        \cellcolor[rgb]{0.9,0.9,0.9}{
            \makecell[{{p{\linewidth}}}]{
                \texttt{\tiny{[P1$\rangle$GM]}}
                \texttt{SPRICH: Hallo! Lass uns zuerst feste Zielkoordinaten vereinbaren. Vorschlag als einfache Form: Ordne „WITCH“ in einer horizontalen Linie von links nach rechts an. Konkrete Zielpositionen: W(1,3), I(3,2), T(1,1), C(2,6), H(4,3). Bitte bestätige, ob du diese Ziele setzen kannst oder schlage alternative Koordinaten vor.} \\
            }
        }
    }
    & & \\ \\

    \theutterance \stepcounter{utterance}  
    & & & \multicolumn{2}{p{0.3\linewidth}}{
        \cellcolor[rgb]{0.9,0.9,0.9}{
            \makecell[{{p{\linewidth}}}]{
                \texttt{\tiny{[GM$|$GM]}}
                \texttt{SPRICH: Hallo! Lass uns zuerst feste Zielkoordinaten vereinbaren. Vorschlag als einfache Form: Ordne „WITCH“ in einer horizontalen Linie von links nach rechts an. Konkrete Zielpositionen: W(1,3), I(3,2), T(1,1), C(2,6), H(4,3). Bitte bestätige, ob du diese Ziele setzen kannst oder schlage alternative Koordinaten vor.} \\
            }
        }
    }
    & & \\ \\

    \theutterance \stepcounter{utterance}  
    & & \multicolumn{4}{p{0.6\linewidth}}{
        \cellcolor[rgb]{0.9,0.9,0.9}{
            \makecell[{{p{\linewidth}}}]{
                \texttt{\tiny{[GM$\rangle$P2]}}
                \texttt{Ich bin der Spielleiter, und du spielst ein kooperatives Spiel mit dem folgenden Raster als Spielbrett:} \\
\\ 
\texttt{```} \\
\texttt{ 1234567} \\
\texttt{╔═╤═╤═╤═╗} \\
\texttt{║◌│◌│◌│◌║ 1} \\
\texttt{╟─┼─┤◌│◌║ 2} \\
\texttt{║◌│W│◌│◌║ 3} \\
\texttt{║◌│C├─┴─╢ 4} \\
\texttt{╟─┴─┤I◌◌║ 5} \\
\texttt{║◌H◌└─┬─╢ 6} \\
\texttt{║◌◌T◌◌│◌║ 7} \\
\texttt{╚═════╧═╝} \\
\\ 
\texttt{```} \\
\\ 
\texttt{* Am oberen Rand sind die x{-}Koordinaten angegeben, die nach rechts hin zunehmen, und am rechten Rand die y{-}Koordinaten, die nach unten zunehmen.} \\
\texttt{* Die folgenden Objekte sind zufällig auf deinem Raster platziert: 'W', 'I', 'T', 'C', 'H'.} \\
\\ 
\\ 
\texttt{Dein Mitspieler sieht eine Variation des Spielbretts, auf dem dieselben Objekte an anderen zufälligen Positionen platziert sind. Du kannst das Spielbrett des anderen Spielers nicht sehen, und er kann deins nicht sehen.} \\
\\ 
\\ 
\texttt{**Ziel des Spiels:**} \\
\\ 
\texttt{Beide Spieler müssen die Objekte auf ihrem jeweiligen Spielbrett so bewegen, dass identische Objekte am Ende an denselben Koordinaten liegen. Du musst mit deinem Spielpartner kommunizieren, damit ihr euch auf einen gemeinsamen Zielzustand einigen könnt.} \\
\\ 
\\ 
\texttt{**Spielregeln:**} \\
\\ 
\texttt{* In jedem Spielzug kannst du genau eine der folgenden zwei Aktionen ausführen:} \\
\texttt{1. `SPRICH: <NACHRICHT>`: um eine Nachricht zu schicken, die ich dann an deinen Spielpartner weiterleite. Als Nachricht gilt alles bis zum nächsten Zeilenumbruch.} \\
\texttt{2. `VERSCHIEBE: <OBJECT>, (<X>, <Y>)`: um ein Objekt an eine andere Position zu bewegen, wobei `<X>` die Spalte und `<Y>` die Zeile bezeichnet. Ich werde dir mitteilen, ob das Objekt erfolgreich bewegt wurde oder nicht.} \\
\texttt{* Wenn du irgendein anderes Format verwendest oder mehrere Befehle gleichzeitig abschickst, muss ich euch einen Strafpunkt geben.} \\
\texttt{* Wenn ihr gemeinsam mehr als 12 Strafpunkte ansammelt, verliert ihr das Spiel.} \\
\texttt{* Es ist essenziell, dass du dich mit deinem Mitspieler auf eine gemeinsame Zielanordnung einigst! Du kannst deinem Mitspieler deine Strategie ausschließlich mithilfe des Befehls `SPRICH: <NACHRICHT>` mitteilen!} \\
\\ 
\\ 
\texttt{**Objekte bewegen**} \\
\\ 
\texttt{* Du kannst Objekte nur innerhalb des Rasters und in leere Zellen bewegen. Leere Zellen erkennst am Zeichen '◌'.} \\
\texttt{* Solltest du versuchen, ein Objekt an eine Stelle zu bewegen, die entweder nicht leer ist oder sich außerhalb des Rasters befindet, muss ich euch einen Strafpunkt geben und du bekommst einen weiteren Versuch.} \\
\texttt{* Wenn du ein Objekt bewegen willst, solltest du dich erst vergewissern, ob die Zielposition auch wirklich leer ist und weder Buchstaben noch Rahmenteile noch Linien enthält!} \\
\\ 
\\ 
\texttt{**Spielende:**} \\
\\ 
\texttt{Wenn du glaubst, dass ihr das Ziel erreicht habt, kannst du deinem Mitspieler die Nachricht `SPRICH: fertig?` schicken. Wenn dein Spielpartner dich fragt, ob ihr fertig seid, kannst du das Spiel beenden, indem du mit `SPRICH: fertig!` antwortest.} \\
\\ 
\texttt{Ihr gewinnt gemeinsam, wenn ihr das Spiel innerhalb von \#max\_rounds Runden beendet. Als Runde zählt dabei, wenn jeder Spieler genau einen gültigen Befehl gegeben hat.} \\
\\ 
\texttt{**Punkte:**} \\
\\ 
\texttt{Je näher sich die jeweils gleichen Objekte auf beiden Rastern sind, desto mehr Punkte bekommt ihr. Für Strafpunkte gibt es Punktabzug. Könnt ihr den Rekord brechen?} \\
\\ 
\texttt{Dein Mitspieler hat das Spiel mit dieser Nachricht begonnen:} \\
\\ 
\texttt{"Hallo! Lass uns zuerst feste Zielkoordinaten vereinbaren. Vorschlag als einfache Form: Ordne „WITCH“ in einer horizontalen Linie von links nach rechts an. Konkrete Zielpositionen: W(1,3), I(3,2), T(1,1), C(2,6), H(4,3). Bitte bestätige, ob du diese Ziele setzen kannst oder schlage alternative Koordinaten vor."} \\
\\ 
\texttt{Wie lautet dein erster Befehl?} \\
            }
        }
    }
    & \\ \\

    \theutterance \stepcounter{utterance}  
    & & & \multicolumn{4}{p{0.6\linewidth}}{
        \cellcolor[rgb]{0.9,0.9,0.9}{
            \makecell[{{p{\linewidth}}}]{
                \texttt{\tiny{[GM$\langle$P2]}}
                \texttt{SPRICH: Danke! Ich kann das umsetzen. Zur Sicherheit bestätige ich die Zielkoordinaten als Endzustand: W(1,3), I(3,2), T(1,1), C(2,6), H(4,3). Bitte bestätige „ok“, dann beginne ich mit meinen Zügen.} \\
            }
        }
    }
     \\ \\

    \theutterance \stepcounter{utterance}  
    & & & \multicolumn{2}{p{0.3\linewidth}}{
        \cellcolor[rgb]{0.9,0.9,0.9}{
            \makecell[{{p{\linewidth}}}]{
                \texttt{\tiny{[GM$|$GM]}}
                \texttt{SPRICH: Danke! Ich kann das umsetzen. Zur Sicherheit bestätige ich die Zielkoordinaten als Endzustand: W(1,3), I(3,2), T(1,1), C(2,6), H(4,3). Bitte bestätige „ok“, dann beginne ich mit meinen Zügen.} \\
            }
        }
    }
    & & \\ \\

    \theutterance \stepcounter{utterance}  
    & & \multicolumn{4}{p{0.6\linewidth}}{
        \cellcolor[rgb]{0.9,0.9,0.9}{
            \makecell[{{p{\linewidth}}}]{
                \texttt{\tiny{[P1$\langle$GM]}}
                \texttt{Deine Nachricht wurde an deinen Mitspieler übermittelt.} \\
\\ 
\\ 
\texttt{Ihr seid gerade bei Runde 1 von maximal 20.} \\
\texttt{Ihr habt gemeinsam 0 von maximal 12 Strafpunkten angesammelt.} \\
\texttt{Dein Mitspieler hat dir diese Nachricht geschickt:} \\
\\ 
\\ 
\texttt{"Danke! Ich kann das umsetzen. Zur Sicherheit bestätige ich die Zielkoordinaten als Endzustand: W(1,3), I(3,2), T(1,1), C(2,6), H(4,3). Bitte bestätige „ok“, dann beginne ich mit meinen Zügen."} \\
\\ 
\\ 
\texttt{Wie lautet dein nächster Befehl?} \\
            }
        }
    }
    & \\ \\

    \theutterance \stepcounter{utterance}  
    & \multicolumn{4}{p{0.6\linewidth}}{
        \cellcolor[rgb]{0.9,0.9,0.9}{
            \makecell[{{p{\linewidth}}}]{
                \texttt{\tiny{[P1$\rangle$GM]}}
                \texttt{SPRICH: ok} \\
            }
        }
    }
    & & \\ \\

    \theutterance \stepcounter{utterance}  
    & & & \multicolumn{2}{p{0.3\linewidth}}{
        \cellcolor[rgb]{0.9,0.9,0.9}{
            \makecell[{{p{\linewidth}}}]{
                \texttt{\tiny{[GM$|$GM]}}
                \texttt{SPRICH: ok} \\
            }
        }
    }
    & & \\ \\

    \theutterance \stepcounter{utterance}  
    & & \multicolumn{4}{p{0.6\linewidth}}{
        \cellcolor[rgb]{0.9,0.9,0.9}{
            \makecell[{{p{\linewidth}}}]{
                \texttt{\tiny{[GM$\rangle$P2]}}
                \texttt{Deine Nachricht wurde an deinen Mitspieler übermittelt.} \\
\\ 
\\ 
\texttt{Ihr seid gerade bei Runde 2 von maximal 20.} \\
\texttt{Ihr habt gemeinsam 0 von maximal 12 Strafpunkten angesammelt.} \\
\texttt{Dein Mitspieler hat dir diese Nachricht geschickt:} \\
\\ 
\\ 
\texttt{"ok"} \\
\\ 
\\ 
\texttt{Wie lautet dein nächster Befehl?} \\
            }
        }
    }
    & \\ \\

    \theutterance \stepcounter{utterance}  
    & & & \multicolumn{4}{p{0.6\linewidth}}{
        \cellcolor[rgb]{0.9,0.9,0.9}{
            \makecell[{{p{\linewidth}}}]{
                \texttt{\tiny{[GM$\langle$P2]}}
                \texttt{VERSCHIEBE: T, (1,1)} \\
            }
        }
    }
     \\ \\

    \theutterance \stepcounter{utterance}  
    & & & \multicolumn{2}{p{0.3\linewidth}}{
        \cellcolor[rgb]{0.9,0.9,0.9}{
            \makecell[{{p{\linewidth}}}]{
                \texttt{\tiny{[GM$|$GM]}}
                \texttt{VERSCHIEBE: T, (1,1)} \\
            }
        }
    }
    & & \\ \\

    \theutterance \stepcounter{utterance}  
    & & & \multicolumn{2}{p{0.3\linewidth}}{
        \cellcolor[rgb]{0.9,0.9,0.9}{
            \makecell[{{p{\linewidth}}}]{
                \texttt{\tiny{[GM$|$GM]}}
                \texttt{'T' erfolgreich nach (1,1) verschoben. Dein Raster sieht jetzt so aus:} \\
\texttt{```} \\
\texttt{ 1234567} \\
\texttt{╔═╤═╤═╤═╗} \\
\texttt{║T│◌│◌│◌║ 1} \\
\texttt{╟─┼─┤◌│◌║ 2} \\
\texttt{║◌│W│◌│◌║ 3} \\
\texttt{║◌│C├─┴─╢ 4} \\
\texttt{╟─┴─┤I◌◌║ 5} \\
\texttt{║◌H◌└─┬─╢ 6} \\
\texttt{║◌◌◌◌◌│◌║ 7} \\
\texttt{╚═════╧═╝} \\
\texttt{```} \\
            }
        }
    }
    & & \\ \\

    \theutterance \stepcounter{utterance}  
    & & \multicolumn{4}{p{0.6\linewidth}}{
        \cellcolor[rgb]{0.9,0.9,0.9}{
            \makecell[{{p{\linewidth}}}]{
                \texttt{\tiny{[P1$\langle$GM]}}
                \texttt{Deine Nachricht wurde an deinen Mitspieler übermittelt.} \\
\\ 
\\ 
\texttt{Ihr seid gerade bei Runde 2 von maximal 20.} \\
\texttt{Ihr habt gemeinsam 0 von maximal 12 Strafpunkten angesammelt.} \\
\texttt{Dein Mitspieler hat ein Objekt auf seinem Raster bewegt.} \\
\\ 
\\ 
\texttt{Wie lautet dein nächster Befehl?} \\
            }
        }
    }
    & \\ \\

    \theutterance \stepcounter{utterance}  
    & \multicolumn{4}{p{0.6\linewidth}}{
        \cellcolor[rgb]{0.9,0.9,0.9}{
            \makecell[{{p{\linewidth}}}]{
                \texttt{\tiny{[P1$\rangle$GM]}}
                \texttt{VERSCHIEBE: T, (1,1)} \\
            }
        }
    }
    & & \\ \\

    \theutterance \stepcounter{utterance}  
    & & & \multicolumn{2}{p{0.3\linewidth}}{
        \cellcolor[rgb]{0.9,0.9,0.9}{
            \makecell[{{p{\linewidth}}}]{
                \texttt{\tiny{[GM$|$GM]}}
                \texttt{VERSCHIEBE: T, (1,1)} \\
            }
        }
    }
    & & \\ \\

    \theutterance \stepcounter{utterance}  
    & & & \multicolumn{2}{p{0.3\linewidth}}{
        \cellcolor[rgb]{0.9,0.9,0.9}{
            \makecell[{{p{\linewidth}}}]{
                \texttt{\tiny{[GM$|$GM]}}
                \texttt{Strafpunkt: (1,1) ist nicht leer, sondern enthält 'T'.} \\
\\ 
\texttt{Ihr habt gemeinsam 1 von maximal 12 Strafpunkten angesammelt.} \\
\texttt{Bitte versuche es noch mal!} \\
            }
        }
    }
    & & \\ \\

    \theutterance \stepcounter{utterance}  
    & & \multicolumn{4}{p{0.6\linewidth}}{
        \cellcolor[rgb]{0.9,0.9,0.9}{
            \makecell[{{p{\linewidth}}}]{
                \texttt{\tiny{[P1$\langle$GM]}}
                \texttt{Strafpunkt: (1,1) ist nicht leer, sondern enthält 'T'.} \\
\\ 
\texttt{Ihr habt gemeinsam 1 von maximal 12 Strafpunkten angesammelt.} \\
\texttt{Bitte versuche es noch mal!} \\
            }
        }
    }
    & \\ \\

    \theutterance \stepcounter{utterance}  
    & \multicolumn{4}{p{0.6\linewidth}}{
        \cellcolor[rgb]{0.9,0.9,0.9}{
            \makecell[{{p{\linewidth}}}]{
                \texttt{\tiny{[P1$\rangle$GM]}}
                \texttt{SPRICH: Danke für den Hinweis. Ich lasse T stehen. Ich bewege jetzt nur noch, falls nötig. Bitte sag Bescheid, wenn du bereit bist für „fertig?“.} \\
            }
        }
    }
    & & \\ \\

    \theutterance \stepcounter{utterance}  
    & & & \multicolumn{2}{p{0.3\linewidth}}{
        \cellcolor[rgb]{0.9,0.9,0.9}{
            \makecell[{{p{\linewidth}}}]{
                \texttt{\tiny{[GM$|$GM]}}
                \texttt{SPRICH: Danke für den Hinweis. Ich lasse T stehen. Ich bewege jetzt nur noch, falls nötig. Bitte sag Bescheid, wenn du bereit bist für „fertig?“.} \\
            }
        }
    }
    & & \\ \\

    \theutterance \stepcounter{utterance}  
    & & \multicolumn{4}{p{0.6\linewidth}}{
        \cellcolor[rgb]{0.9,0.9,0.9}{
            \makecell[{{p{\linewidth}}}]{
                \texttt{\tiny{[GM$\rangle$P2]}}
                \texttt{'T' erfolgreich nach (1,1) verschoben. Dein Raster sieht jetzt so aus:} \\
\texttt{```} \\
\texttt{ 1234567} \\
\texttt{╔═╤═╤═╤═╗} \\
\texttt{║T│◌│◌│◌║ 1} \\
\texttt{╟─┼─┤◌│◌║ 2} \\
\texttt{║◌│W│◌│◌║ 3} \\
\texttt{║◌│C├─┴─╢ 4} \\
\texttt{╟─┴─┤I◌◌║ 5} \\
\texttt{║◌H◌└─┬─╢ 6} \\
\texttt{║◌◌◌◌◌│◌║ 7} \\
\texttt{╚═════╧═╝} \\
\texttt{```} \\
\texttt{Ihr seid gerade bei Runde 3 von maximal 20.} \\
\texttt{Ihr habt gemeinsam 1 von maximal 12 Strafpunkten angesammelt.} \\
\texttt{Dein Mitspieler hat dir diese Nachricht geschickt:} \\
\\ 
\\ 
\texttt{"Danke für den Hinweis. Ich lasse T stehen. Ich bewege jetzt nur noch, falls nötig. Bitte sag Bescheid, wenn du bereit bist für „fertig?“."} \\
\\ 
\\ 
\texttt{Wie lautet dein nächster Befehl?} \\
            }
        }
    }
    & \\ \\

    \theutterance \stepcounter{utterance}  
    & & & \multicolumn{4}{p{0.6\linewidth}}{
        \cellcolor[rgb]{0.9,0.9,0.9}{
            \makecell[{{p{\linewidth}}}]{
                \texttt{\tiny{[GM$\langle$P2]}}
                \texttt{VERSCHIEBE: W, (1,3)} \\
            }
        }
    }
     \\ \\

    \theutterance \stepcounter{utterance}  
    & & & \multicolumn{2}{p{0.3\linewidth}}{
        \cellcolor[rgb]{0.9,0.9,0.9}{
            \makecell[{{p{\linewidth}}}]{
                \texttt{\tiny{[GM$|$GM]}}
                \texttt{VERSCHIEBE: W, (1,3)} \\
            }
        }
    }
    & & \\ \\

    \theutterance \stepcounter{utterance}  
    & & & \multicolumn{2}{p{0.3\linewidth}}{
        \cellcolor[rgb]{0.9,0.9,0.9}{
            \makecell[{{p{\linewidth}}}]{
                \texttt{\tiny{[GM$|$GM]}}
                \texttt{'W' erfolgreich nach (1,3) verschoben. Dein Raster sieht jetzt so aus:} \\
\texttt{```} \\
\texttt{ 1234567} \\
\texttt{╔═╤═╤═╤═╗} \\
\texttt{║T│◌│◌│◌║ 1} \\
\texttt{╟─┼─┤◌│◌║ 2} \\
\texttt{║W│◌│◌│◌║ 3} \\
\texttt{║◌│C├─┴─╢ 4} \\
\texttt{╟─┴─┤I◌◌║ 5} \\
\texttt{║◌H◌└─┬─╢ 6} \\
\texttt{║◌◌◌◌◌│◌║ 7} \\
\texttt{╚═════╧═╝} \\
\texttt{```} \\
            }
        }
    }
    & & \\ \\

    \theutterance \stepcounter{utterance}  
    & & \multicolumn{4}{p{0.6\linewidth}}{
        \cellcolor[rgb]{0.9,0.9,0.9}{
            \makecell[{{p{\linewidth}}}]{
                \texttt{\tiny{[P1$\langle$GM]}}
                \texttt{Deine Nachricht wurde an deinen Mitspieler übermittelt.} \\
\\ 
\\ 
\texttt{Ihr seid gerade bei Runde 3 von maximal 20.} \\
\texttt{Ihr habt gemeinsam 1 von maximal 12 Strafpunkten angesammelt.} \\
\texttt{Dein Mitspieler hat ein Objekt auf seinem Raster bewegt.} \\
\\ 
\\ 
\texttt{Wie lautet dein nächster Befehl?} \\
            }
        }
    }
    & \\ \\

    \theutterance \stepcounter{utterance}  
    & \multicolumn{4}{p{0.6\linewidth}}{
        \cellcolor[rgb]{0.9,0.9,0.9}{
            \makecell[{{p{\linewidth}}}]{
                \texttt{\tiny{[P1$\rangle$GM]}}
                \texttt{VERSCHIEBE: W, (1,3)} \\
            }
        }
    }
    & & \\ \\

    \theutterance \stepcounter{utterance}  
    & & & \multicolumn{2}{p{0.3\linewidth}}{
        \cellcolor[rgb]{0.9,0.9,0.9}{
            \makecell[{{p{\linewidth}}}]{
                \texttt{\tiny{[GM$|$GM]}}
                \texttt{VERSCHIEBE: W, (1,3)} \\
            }
        }
    }
    & & \\ \\

    \theutterance \stepcounter{utterance}  
    & & & \multicolumn{2}{p{0.3\linewidth}}{
        \cellcolor[rgb]{0.9,0.9,0.9}{
            \makecell[{{p{\linewidth}}}]{
                \texttt{\tiny{[GM$|$GM]}}
                \texttt{Strafpunkt: (1,3) ist nicht leer, sondern enthält 'W'.} \\
\\ 
\texttt{Ihr habt gemeinsam 2 von maximal 12 Strafpunkten angesammelt.} \\
\texttt{Bitte versuche es noch mal!} \\
            }
        }
    }
    & & \\ \\

    \theutterance \stepcounter{utterance}  
    & & \multicolumn{4}{p{0.6\linewidth}}{
        \cellcolor[rgb]{0.9,0.9,0.9}{
            \makecell[{{p{\linewidth}}}]{
                \texttt{\tiny{[P1$\langle$GM]}}
                \texttt{Strafpunkt: (1,3) ist nicht leer, sondern enthält 'W'.} \\
\\ 
\texttt{Ihr habt gemeinsam 2 von maximal 12 Strafpunkten angesammelt.} \\
\texttt{Bitte versuche es noch mal!} \\
            }
        }
    }
    & \\ \\

    \theutterance \stepcounter{utterance}  
    & \multicolumn{4}{p{0.6\linewidth}}{
        \cellcolor[rgb]{0.9,0.9,0.9}{
            \makecell[{{p{\linewidth}}}]{
                \texttt{\tiny{[P1$\rangle$GM]}}
                \texttt{VERSCHIEBE: I, (3,2)} \\
            }
        }
    }
    & & \\ \\

    \theutterance \stepcounter{utterance}  
    & & & \multicolumn{2}{p{0.3\linewidth}}{
        \cellcolor[rgb]{0.9,0.9,0.9}{
            \makecell[{{p{\linewidth}}}]{
                \texttt{\tiny{[GM$|$GM]}}
                \texttt{VERSCHIEBE: I, (3,2)} \\
            }
        }
    }
    & & \\ \\

    \theutterance \stepcounter{utterance}  
    & & & \multicolumn{2}{p{0.3\linewidth}}{
        \cellcolor[rgb]{0.9,0.9,0.9}{
            \makecell[{{p{\linewidth}}}]{
                \texttt{\tiny{[GM$|$GM]}}
                \texttt{Strafpunkt: (3,2) ist nicht leer, sondern enthält '─'.} \\
\\ 
\texttt{Ihr habt gemeinsam 3 von maximal 12 Strafpunkten angesammelt.} \\
\texttt{Bitte versuche es noch mal!} \\
            }
        }
    }
    & & \\ \\

    \theutterance \stepcounter{utterance}  
    & & \multicolumn{4}{p{0.6\linewidth}}{
        \cellcolor[rgb]{0.9,0.9,0.9}{
            \makecell[{{p{\linewidth}}}]{
                \texttt{\tiny{[P1$\langle$GM]}}
                \texttt{Strafpunkt: (3,2) ist nicht leer, sondern enthält '─'.} \\
\\ 
\texttt{Ihr habt gemeinsam 3 von maximal 12 Strafpunkten angesammelt.} \\
\texttt{Bitte versuche es noch mal!} \\
            }
        }
    }
    & \\ \\

    \theutterance \stepcounter{utterance}  
    & \multicolumn{4}{p{0.6\linewidth}}{
        \cellcolor[rgb]{0.9,0.9,0.9}{
            \makecell[{{p{\linewidth}}}]{
                \texttt{\tiny{[P1$\rangle$GM]}}
                \texttt{SPRICH: Ich kann I nicht nach (3,2) legen, dort ist eine Linie. Lass uns die Ziele an meine belegten/linearen Felder anpassen: Behalte T(1,1), W(1,3), H(4,3). Schlage neue Ziele für I und C auf freien Feldern vor. Freie Kandidaten bei mir: (2,1), (3,1), (2,3), (2,4), (2,5), (1,6), (3,6), (1,7)–(5,7), (6,7). Vorschlag: I(2,1), C(3,6). Passt das bei dir?} \\
            }
        }
    }
    & & \\ \\

    \theutterance \stepcounter{utterance}  
    & & & \multicolumn{2}{p{0.3\linewidth}}{
        \cellcolor[rgb]{0.9,0.9,0.9}{
            \makecell[{{p{\linewidth}}}]{
                \texttt{\tiny{[GM$|$GM]}}
                \texttt{SPRICH: Ich kann I nicht nach (3,2) legen, dort ist eine Linie. Lass uns die Ziele an meine belegten/linearen Felder anpassen: Behalte T(1,1), W(1,3), H(4,3). Schlage neue Ziele für I und C auf freien Feldern vor. Freie Kandidaten bei mir: (2,1), (3,1), (2,3), (2,4), (2,5), (1,6), (3,6), (1,7)–(5,7), (6,7). Vorschlag: I(2,1), C(3,6). Passt das bei dir?} \\
            }
        }
    }
    & & \\ \\

    \theutterance \stepcounter{utterance}  
    & & \multicolumn{4}{p{0.6\linewidth}}{
        \cellcolor[rgb]{0.9,0.9,0.9}{
            \makecell[{{p{\linewidth}}}]{
                \texttt{\tiny{[GM$\rangle$P2]}}
                \texttt{'W' erfolgreich nach (1,3) verschoben. Dein Raster sieht jetzt so aus:} \\
\texttt{```} \\
\texttt{ 1234567} \\
\texttt{╔═╤═╤═╤═╗} \\
\texttt{║T│◌│◌│◌║ 1} \\
\texttt{╟─┼─┤◌│◌║ 2} \\
\texttt{║W│◌│◌│◌║ 3} \\
\texttt{║◌│C├─┴─╢ 4} \\
\texttt{╟─┴─┤I◌◌║ 5} \\
\texttt{║◌H◌└─┬─╢ 6} \\
\texttt{║◌◌◌◌◌│◌║ 7} \\
\texttt{╚═════╧═╝} \\
\texttt{```} \\
\texttt{Ihr seid gerade bei Runde 4 von maximal 20.} \\
\texttt{Ihr habt gemeinsam 3 von maximal 12 Strafpunkten angesammelt.} \\
\texttt{Dein Mitspieler hat dir diese Nachricht geschickt:} \\
\\ 
\\ 
\texttt{"Ich kann I nicht nach (3,2) legen, dort ist eine Linie. Lass uns die Ziele an meine belegten/linearen Felder anpassen: Behalte T(1,1), W(1,3), H(4,3). Schlage neue Ziele für I und C auf freien Feldern vor. Freie Kandidaten bei mir: (2,1), (3,1), (2,3), (2,4), (2,5), (1,6), (3,6), (1,7)–(5,7), (6,7). Vorschlag: I(2,1), C(3,6). Passt das bei dir?"} \\
\\ 
\\ 
\texttt{Wie lautet dein nächster Befehl?} \\
            }
        }
    }
    & \\ \\

    \theutterance \stepcounter{utterance}  
    & & & \multicolumn{4}{p{0.6\linewidth}}{
        \cellcolor[rgb]{0.9,0.9,0.9}{
            \makecell[{{p{\linewidth}}}]{
                \texttt{\tiny{[GM$\langle$P2]}}
                \texttt{SPRICH: Passt. Ich bestätige neue Ziele: I(2,1), C(3,6). Aktueller Zielzustand: W(1,3), I(2,1), T(1,1), C(3,6), H(4,3). Sag „ok“, dann mache ich weiter.} \\
            }
        }
    }
     \\ \\

    \theutterance \stepcounter{utterance}  
    & & & \multicolumn{2}{p{0.3\linewidth}}{
        \cellcolor[rgb]{0.9,0.9,0.9}{
            \makecell[{{p{\linewidth}}}]{
                \texttt{\tiny{[GM$|$GM]}}
                \texttt{SPRICH: Passt. Ich bestätige neue Ziele: I(2,1), C(3,6). Aktueller Zielzustand: W(1,3), I(2,1), T(1,1), C(3,6), H(4,3). Sag „ok“, dann mache ich weiter.} \\
            }
        }
    }
    & & \\ \\

    \theutterance \stepcounter{utterance}  
    & & \multicolumn{4}{p{0.6\linewidth}}{
        \cellcolor[rgb]{0.9,0.9,0.9}{
            \makecell[{{p{\linewidth}}}]{
                \texttt{\tiny{[P1$\langle$GM]}}
                \texttt{Deine Nachricht wurde an deinen Mitspieler übermittelt.} \\
\\ 
\\ 
\texttt{Ihr seid gerade bei Runde 4 von maximal 20.} \\
\texttt{Ihr habt gemeinsam 3 von maximal 12 Strafpunkten angesammelt.} \\
\texttt{Dein Mitspieler hat dir diese Nachricht geschickt:} \\
\\ 
\\ 
\texttt{"Passt. Ich bestätige neue Ziele: I(2,1), C(3,6). Aktueller Zielzustand: W(1,3), I(2,1), T(1,1), C(3,6), H(4,3). Sag „ok“, dann mache ich weiter."} \\
\\ 
\\ 
\texttt{Wie lautet dein nächster Befehl?} \\
            }
        }
    }
    & \\ \\

    \theutterance \stepcounter{utterance}  
    & \multicolumn{4}{p{0.6\linewidth}}{
        \cellcolor[rgb]{0.9,0.9,0.9}{
            \makecell[{{p{\linewidth}}}]{
                \texttt{\tiny{[P1$\rangle$GM]}}
                \texttt{SPRICH: ok} \\
            }
        }
    }
    & & \\ \\

    \theutterance \stepcounter{utterance}  
    & & & \multicolumn{2}{p{0.3\linewidth}}{
        \cellcolor[rgb]{0.9,0.9,0.9}{
            \makecell[{{p{\linewidth}}}]{
                \texttt{\tiny{[GM$|$GM]}}
                \texttt{SPRICH: ok} \\
            }
        }
    }
    & & \\ \\

    \theutterance \stepcounter{utterance}  
    & & \multicolumn{4}{p{0.6\linewidth}}{
        \cellcolor[rgb]{0.9,0.9,0.9}{
            \makecell[{{p{\linewidth}}}]{
                \texttt{\tiny{[GM$\rangle$P2]}}
                \texttt{Deine Nachricht wurde an deinen Mitspieler übermittelt.} \\
\\ 
\\ 
\texttt{Ihr seid gerade bei Runde 5 von maximal 20.} \\
\texttt{Ihr habt gemeinsam 3 von maximal 12 Strafpunkten angesammelt.} \\
\texttt{Dein Mitspieler hat dir diese Nachricht geschickt:} \\
\\ 
\\ 
\texttt{"ok"} \\
\\ 
\\ 
\texttt{Wie lautet dein nächster Befehl?} \\
            }
        }
    }
    & \\ \\

    \theutterance \stepcounter{utterance}  
    & & & \multicolumn{4}{p{0.6\linewidth}}{
        \cellcolor[rgb]{0.9,0.9,0.9}{
            \makecell[{{p{\linewidth}}}]{
                \texttt{\tiny{[GM$\langle$P2]}}
                \texttt{VERSCHIEBE: I, (2,1)} \\
            }
        }
    }
     \\ \\

    \theutterance \stepcounter{utterance}  
    & & & \multicolumn{2}{p{0.3\linewidth}}{
        \cellcolor[rgb]{0.9,0.9,0.9}{
            \makecell[{{p{\linewidth}}}]{
                \texttt{\tiny{[GM$|$GM]}}
                \texttt{VERSCHIEBE: I, (2,1)} \\
            }
        }
    }
    & & \\ \\

    \theutterance \stepcounter{utterance}  
    & & & \multicolumn{2}{p{0.3\linewidth}}{
        \cellcolor[rgb]{0.9,0.9,0.9}{
            \makecell[{{p{\linewidth}}}]{
                \texttt{\tiny{[GM$|$GM]}}
                \texttt{Strafpunkt: (2,1) ist nicht leer, sondern enthält '│'.} \\
\\ 
\texttt{Ihr habt gemeinsam 4 von maximal 12 Strafpunkten angesammelt.} \\
\texttt{Bitte versuche es noch mal!} \\
            }
        }
    }
    & & \\ \\

    \theutterance \stepcounter{utterance}  
    & & \multicolumn{4}{p{0.6\linewidth}}{
        \cellcolor[rgb]{0.9,0.9,0.9}{
            \makecell[{{p{\linewidth}}}]{
                \texttt{\tiny{[GM$\rangle$P2]}}
                \texttt{Strafpunkt: (2,1) ist nicht leer, sondern enthält '│'.} \\
\\ 
\texttt{Ihr habt gemeinsam 4 von maximal 12 Strafpunkten angesammelt.} \\
\texttt{Bitte versuche es noch mal!} \\
            }
        }
    }
    & \\ \\

    \theutterance \stepcounter{utterance}  
    & & & \multicolumn{4}{p{0.6\linewidth}}{
        \cellcolor[rgb]{0.9,0.9,0.9}{
            \makecell[{{p{\linewidth}}}]{
                \texttt{\tiny{[GM$\langle$P2]}}
                \texttt{SPRICH: Danke für den Hinweis. Ich passe an: Neue sichere Ziele, die bei mir frei sind und deinen Vorschlägen entsprechen: I(2,3) statt (2,1) und C(3,6) bleibt. Aktueller Zielzustand: W(1,3), I(2,3), T(1,1), C(3,6), H(4,3). Bitte bestätige „ok“.} \\
            }
        }
    }
     \\ \\

    \theutterance \stepcounter{utterance}  
    & & & \multicolumn{2}{p{0.3\linewidth}}{
        \cellcolor[rgb]{0.9,0.9,0.9}{
            \makecell[{{p{\linewidth}}}]{
                \texttt{\tiny{[GM$|$GM]}}
                \texttt{SPRICH: Danke für den Hinweis. Ich passe an: Neue sichere Ziele, die bei mir frei sind und deinen Vorschlägen entsprechen: I(2,3) statt (2,1) und C(3,6) bleibt. Aktueller Zielzustand: W(1,3), I(2,3), T(1,1), C(3,6), H(4,3). Bitte bestätige „ok“.} \\
            }
        }
    }
    & & \\ \\

    \theutterance \stepcounter{utterance}  
    & & \multicolumn{4}{p{0.6\linewidth}}{
        \cellcolor[rgb]{0.9,0.9,0.9}{
            \makecell[{{p{\linewidth}}}]{
                \texttt{\tiny{[P1$\langle$GM]}}
                \texttt{Deine Nachricht wurde an deinen Mitspieler übermittelt.} \\
\\ 
\\ 
\texttt{Ihr seid gerade bei Runde 5 von maximal 20.} \\
\texttt{Ihr habt gemeinsam 4 von maximal 12 Strafpunkten angesammelt.} \\
\texttt{Dein Mitspieler hat dir diese Nachricht geschickt:} \\
\\ 
\\ 
\texttt{"Danke für den Hinweis. Ich passe an: Neue sichere Ziele, die bei mir frei sind und deinen Vorschlägen entsprechen: I(2,3) statt (2,1) und C(3,6) bleibt. Aktueller Zielzustand: W(1,3), I(2,3), T(1,1), C(3,6), H(4,3). Bitte bestätige „ok“."} \\
\\ 
\\ 
\texttt{Wie lautet dein nächster Befehl?} \\
            }
        }
    }
    & \\ \\

    \theutterance \stepcounter{utterance}  
    & \multicolumn{4}{p{0.6\linewidth}}{
        \cellcolor[rgb]{0.9,0.9,0.9}{
            \makecell[{{p{\linewidth}}}]{
                \texttt{\tiny{[P1$\rangle$GM]}}
                \texttt{SPRICH: ok} \\
            }
        }
    }
    & & \\ \\

    \theutterance \stepcounter{utterance}  
    & & & \multicolumn{2}{p{0.3\linewidth}}{
        \cellcolor[rgb]{0.9,0.9,0.9}{
            \makecell[{{p{\linewidth}}}]{
                \texttt{\tiny{[GM$|$GM]}}
                \texttt{SPRICH: ok} \\
            }
        }
    }
    & & \\ \\

    \theutterance \stepcounter{utterance}  
    & & \multicolumn{4}{p{0.6\linewidth}}{
        \cellcolor[rgb]{0.9,0.9,0.9}{
            \makecell[{{p{\linewidth}}}]{
                \texttt{\tiny{[GM$\rangle$P2]}}
                \texttt{Deine Nachricht wurde an deinen Mitspieler übermittelt.} \\
\\ 
\\ 
\texttt{Ihr seid gerade bei Runde 6 von maximal 20.} \\
\texttt{Ihr habt gemeinsam 4 von maximal 12 Strafpunkten angesammelt.} \\
\texttt{Dein Mitspieler hat dir diese Nachricht geschickt:} \\
\\ 
\\ 
\texttt{"ok"} \\
\\ 
\\ 
\texttt{Wie lautet dein nächster Befehl?} \\
            }
        }
    }
    & \\ \\

    \theutterance \stepcounter{utterance}  
    & & & \multicolumn{4}{p{0.6\linewidth}}{
        \cellcolor[rgb]{0.9,0.9,0.9}{
            \makecell[{{p{\linewidth}}}]{
                \texttt{\tiny{[GM$\langle$P2]}}
                \texttt{VERSCHIEBE: I, (2,3)} \\
            }
        }
    }
     \\ \\

    \theutterance \stepcounter{utterance}  
    & & & \multicolumn{2}{p{0.3\linewidth}}{
        \cellcolor[rgb]{0.9,0.9,0.9}{
            \makecell[{{p{\linewidth}}}]{
                \texttt{\tiny{[GM$|$GM]}}
                \texttt{VERSCHIEBE: I, (2,3)} \\
            }
        }
    }
    & & \\ \\

    \theutterance \stepcounter{utterance}  
    & & & \multicolumn{2}{p{0.3\linewidth}}{
        \cellcolor[rgb]{0.9,0.9,0.9}{
            \makecell[{{p{\linewidth}}}]{
                \texttt{\tiny{[GM$|$GM]}}
                \texttt{Strafpunkt: (2,3) ist nicht leer, sondern enthält '│'.} \\
\\ 
\texttt{Ihr habt gemeinsam 5 von maximal 12 Strafpunkten angesammelt.} \\
\texttt{Bitte versuche es noch mal!} \\
            }
        }
    }
    & & \\ \\

    \theutterance \stepcounter{utterance}  
    & & \multicolumn{4}{p{0.6\linewidth}}{
        \cellcolor[rgb]{0.9,0.9,0.9}{
            \makecell[{{p{\linewidth}}}]{
                \texttt{\tiny{[GM$\rangle$P2]}}
                \texttt{Strafpunkt: (2,3) ist nicht leer, sondern enthält '│'.} \\
\\ 
\texttt{Ihr habt gemeinsam 5 von maximal 12 Strafpunkten angesammelt.} \\
\texttt{Bitte versuche es noch mal!} \\
            }
        }
    }
    & \\ \\

    \theutterance \stepcounter{utterance}  
    & & & \multicolumn{4}{p{0.6\linewidth}}{
        \cellcolor[rgb]{0.9,0.9,0.9}{
            \makecell[{{p{\linewidth}}}]{
                \texttt{\tiny{[GM$\langle$P2]}}
                \texttt{SPRICH: Bei mir sind (2,1) und (2,3) blockiert. Ich schlage neue freie Ziele vor: I(3,1) und C(3,6) bleibt. Gesamtziel: W(1,3), I(3,1), T(1,1), C(3,6), H(4,3). Bitte bestätige „ok“.} \\
            }
        }
    }
     \\ \\

    \theutterance \stepcounter{utterance}  
    & & & \multicolumn{2}{p{0.3\linewidth}}{
        \cellcolor[rgb]{0.9,0.9,0.9}{
            \makecell[{{p{\linewidth}}}]{
                \texttt{\tiny{[GM$|$GM]}}
                \texttt{SPRICH: Bei mir sind (2,1) und (2,3) blockiert. Ich schlage neue freie Ziele vor: I(3,1) und C(3,6) bleibt. Gesamtziel: W(1,3), I(3,1), T(1,1), C(3,6), H(4,3). Bitte bestätige „ok“.} \\
            }
        }
    }
    & & \\ \\

    \theutterance \stepcounter{utterance}  
    & & \multicolumn{4}{p{0.6\linewidth}}{
        \cellcolor[rgb]{0.9,0.9,0.9}{
            \makecell[{{p{\linewidth}}}]{
                \texttt{\tiny{[P1$\langle$GM]}}
                \texttt{Deine Nachricht wurde an deinen Mitspieler übermittelt.} \\
\\ 
\\ 
\texttt{Ihr seid gerade bei Runde 6 von maximal 20.} \\
\texttt{Ihr habt gemeinsam 5 von maximal 12 Strafpunkten angesammelt.} \\
\texttt{Dein Mitspieler hat dir diese Nachricht geschickt:} \\
\\ 
\\ 
\texttt{"Bei mir sind (2,1) und (2,3) blockiert. Ich schlage neue freie Ziele vor: I(3,1) und C(3,6) bleibt. Gesamtziel: W(1,3), I(3,1), T(1,1), C(3,6), H(4,3). Bitte bestätige „ok“."} \\
\\ 
\\ 
\texttt{Wie lautet dein nächster Befehl?} \\
            }
        }
    }
    & \\ \\

    \theutterance \stepcounter{utterance}  
    & \multicolumn{4}{p{0.6\linewidth}}{
        \cellcolor[rgb]{0.9,0.9,0.9}{
            \makecell[{{p{\linewidth}}}]{
                \texttt{\tiny{[P1$\rangle$GM]}}
                \texttt{SPRICH: ok} \\
            }
        }
    }
    & & \\ \\

    \theutterance \stepcounter{utterance}  
    & & & \multicolumn{2}{p{0.3\linewidth}}{
        \cellcolor[rgb]{0.9,0.9,0.9}{
            \makecell[{{p{\linewidth}}}]{
                \texttt{\tiny{[GM$|$GM]}}
                \texttt{SPRICH: ok} \\
            }
        }
    }
    & & \\ \\

    \theutterance \stepcounter{utterance}  
    & & \multicolumn{4}{p{0.6\linewidth}}{
        \cellcolor[rgb]{0.9,0.9,0.9}{
            \makecell[{{p{\linewidth}}}]{
                \texttt{\tiny{[GM$\rangle$P2]}}
                \texttt{Deine Nachricht wurde an deinen Mitspieler übermittelt.} \\
\\ 
\\ 
\texttt{Ihr seid gerade bei Runde 7 von maximal 20.} \\
\texttt{Ihr habt gemeinsam 5 von maximal 12 Strafpunkten angesammelt.} \\
\texttt{Dein Mitspieler hat dir diese Nachricht geschickt:} \\
\\ 
\\ 
\texttt{"ok"} \\
\\ 
\\ 
\texttt{Wie lautet dein nächster Befehl?} \\
            }
        }
    }
    & \\ \\

    \theutterance \stepcounter{utterance}  
    & & & \multicolumn{4}{p{0.6\linewidth}}{
        \cellcolor[rgb]{0.9,0.9,0.9}{
            \makecell[{{p{\linewidth}}}]{
                \texttt{\tiny{[GM$\langle$P2]}}
                \texttt{VERSCHIEBE: I, (3,1)} \\
            }
        }
    }
     \\ \\

    \theutterance \stepcounter{utterance}  
    & & & \multicolumn{2}{p{0.3\linewidth}}{
        \cellcolor[rgb]{0.9,0.9,0.9}{
            \makecell[{{p{\linewidth}}}]{
                \texttt{\tiny{[GM$|$GM]}}
                \texttt{VERSCHIEBE: I, (3,1)} \\
            }
        }
    }
    & & \\ \\

    \theutterance \stepcounter{utterance}  
    & & & \multicolumn{2}{p{0.3\linewidth}}{
        \cellcolor[rgb]{0.9,0.9,0.9}{
            \makecell[{{p{\linewidth}}}]{
                \texttt{\tiny{[GM$|$GM]}}
                \texttt{'I' erfolgreich nach (3,1) verschoben. Dein Raster sieht jetzt so aus:} \\
\texttt{```} \\
\texttt{ 1234567} \\
\texttt{╔═╤═╤═╤═╗} \\
\texttt{║T│I│◌│◌║ 1} \\
\texttt{╟─┼─┤◌│◌║ 2} \\
\texttt{║W│◌│◌│◌║ 3} \\
\texttt{║◌│C├─┴─╢ 4} \\
\texttt{╟─┴─┤◌◌◌║ 5} \\
\texttt{║◌H◌└─┬─╢ 6} \\
\texttt{║◌◌◌◌◌│◌║ 7} \\
\texttt{╚═════╧═╝} \\
\texttt{```} \\
            }
        }
    }
    & & \\ \\

    \theutterance \stepcounter{utterance}  
    & & \multicolumn{4}{p{0.6\linewidth}}{
        \cellcolor[rgb]{0.9,0.9,0.9}{
            \makecell[{{p{\linewidth}}}]{
                \texttt{\tiny{[P1$\langle$GM]}}
                \texttt{Deine Nachricht wurde an deinen Mitspieler übermittelt.} \\
\\ 
\\ 
\texttt{Ihr seid gerade bei Runde 7 von maximal 20.} \\
\texttt{Ihr habt gemeinsam 5 von maximal 12 Strafpunkten angesammelt.} \\
\texttt{Dein Mitspieler hat ein Objekt auf seinem Raster bewegt.} \\
\\ 
\\ 
\texttt{Wie lautet dein nächster Befehl?} \\
            }
        }
    }
    & \\ \\

    \theutterance \stepcounter{utterance}  
    & \multicolumn{4}{p{0.6\linewidth}}{
        \cellcolor[rgb]{0.9,0.9,0.9}{
            \makecell[{{p{\linewidth}}}]{
                \texttt{\tiny{[P1$\rangle$GM]}}
                \texttt{VERSCHIEBE: C, (3,6)} \\
            }
        }
    }
    & & \\ \\

    \theutterance \stepcounter{utterance}  
    & & & \multicolumn{2}{p{0.3\linewidth}}{
        \cellcolor[rgb]{0.9,0.9,0.9}{
            \makecell[{{p{\linewidth}}}]{
                \texttt{\tiny{[GM$|$GM]}}
                \texttt{VERSCHIEBE: C, (3,6)} \\
            }
        }
    }
    & & \\ \\

    \theutterance \stepcounter{utterance}  
    & & & \multicolumn{2}{p{0.3\linewidth}}{
        \cellcolor[rgb]{0.9,0.9,0.9}{
            \makecell[{{p{\linewidth}}}]{
                \texttt{\tiny{[GM$|$GM]}}
                \texttt{'C' erfolgreich nach (3,6) verschoben. Dein Raster sieht jetzt so aus:} \\
\texttt{```} \\
\texttt{ 1234567} \\
\texttt{╔═╤═╤═╤═╗} \\
\texttt{║T│◌│◌│◌║ 1} \\
\texttt{╟─┼─┤I│◌║ 2} \\
\texttt{║W│◌│◌│H║ 3} \\
\texttt{║◌│◌├─┴─╢ 4} \\
\texttt{╟─┴─┤◌◌◌║ 5} \\
\texttt{║◌◌C└─┬─╢ 6} \\
\texttt{║◌◌◌◌◌│◌║ 7} \\
\texttt{╚═════╧═╝} \\
\texttt{```} \\
            }
        }
    }
    & & \\ \\

    \theutterance \stepcounter{utterance}  
    & & \multicolumn{4}{p{0.6\linewidth}}{
        \cellcolor[rgb]{0.9,0.9,0.9}{
            \makecell[{{p{\linewidth}}}]{
                \texttt{\tiny{[GM$\rangle$P2]}}
                \texttt{'I' erfolgreich nach (3,1) verschoben. Dein Raster sieht jetzt so aus:} \\
\texttt{```} \\
\texttt{ 1234567} \\
\texttt{╔═╤═╤═╤═╗} \\
\texttt{║T│I│◌│◌║ 1} \\
\texttt{╟─┼─┤◌│◌║ 2} \\
\texttt{║W│◌│◌│◌║ 3} \\
\texttt{║◌│C├─┴─╢ 4} \\
\texttt{╟─┴─┤◌◌◌║ 5} \\
\texttt{║◌H◌└─┬─╢ 6} \\
\texttt{║◌◌◌◌◌│◌║ 7} \\
\texttt{╚═════╧═╝} \\
\texttt{```} \\
\texttt{Ihr seid gerade bei Runde 8 von maximal 20.} \\
\texttt{Ihr habt gemeinsam 5 von maximal 12 Strafpunkten angesammelt.} \\
\texttt{Dein Mitspieler hat ein Objekt auf seinem Raster bewegt.} \\
\\ 
\\ 
\texttt{Wie lautet dein nächster Befehl?} \\
            }
        }
    }
    & \\ \\

    \theutterance \stepcounter{utterance}  
    & & & \multicolumn{4}{p{0.6\linewidth}}{
        \cellcolor[rgb]{0.9,0.9,0.9}{
            \makecell[{{p{\linewidth}}}]{
                \texttt{\tiny{[GM$\langle$P2]}}
                \texttt{VERSCHIEBE: C, (3,6)} \\
            }
        }
    }
     \\ \\

    \theutterance \stepcounter{utterance}  
    & & & \multicolumn{2}{p{0.3\linewidth}}{
        \cellcolor[rgb]{0.9,0.9,0.9}{
            \makecell[{{p{\linewidth}}}]{
                \texttt{\tiny{[GM$|$GM]}}
                \texttt{VERSCHIEBE: C, (3,6)} \\
            }
        }
    }
    & & \\ \\

    \theutterance \stepcounter{utterance}  
    & & & \multicolumn{2}{p{0.3\linewidth}}{
        \cellcolor[rgb]{0.9,0.9,0.9}{
            \makecell[{{p{\linewidth}}}]{
                \texttt{\tiny{[GM$|$GM]}}
                \texttt{'C' erfolgreich nach (3,6) verschoben. Dein Raster sieht jetzt so aus:} \\
\texttt{```} \\
\texttt{ 1234567} \\
\texttt{╔═╤═╤═╤═╗} \\
\texttt{║T│I│◌│◌║ 1} \\
\texttt{╟─┼─┤◌│◌║ 2} \\
\texttt{║W│◌│◌│◌║ 3} \\
\texttt{║◌│◌├─┴─╢ 4} \\
\texttt{╟─┴─┤◌◌◌║ 5} \\
\texttt{║◌HC└─┬─╢ 6} \\
\texttt{║◌◌◌◌◌│◌║ 7} \\
\texttt{╚═════╧═╝} \\
\texttt{```} \\
            }
        }
    }
    & & \\ \\

    \theutterance \stepcounter{utterance}  
    & & \multicolumn{4}{p{0.6\linewidth}}{
        \cellcolor[rgb]{0.9,0.9,0.9}{
            \makecell[{{p{\linewidth}}}]{
                \texttt{\tiny{[P1$\langle$GM]}}
                \texttt{'C' erfolgreich nach (3,6) verschoben. Dein Raster sieht jetzt so aus:} \\
\texttt{```} \\
\texttt{ 1234567} \\
\texttt{╔═╤═╤═╤═╗} \\
\texttt{║T│◌│◌│◌║ 1} \\
\texttt{╟─┼─┤I│◌║ 2} \\
\texttt{║W│◌│◌│H║ 3} \\
\texttt{║◌│◌├─┴─╢ 4} \\
\texttt{╟─┴─┤◌◌◌║ 5} \\
\texttt{║◌◌C└─┬─╢ 6} \\
\texttt{║◌◌◌◌◌│◌║ 7} \\
\texttt{╚═════╧═╝} \\
\texttt{```} \\
\texttt{Ihr seid gerade bei Runde 8 von maximal 20.} \\
\texttt{Ihr habt gemeinsam 5 von maximal 12 Strafpunkten angesammelt.} \\
\texttt{Dein Mitspieler hat ein Objekt auf seinem Raster bewegt.} \\
\\ 
\\ 
\texttt{Wie lautet dein nächster Befehl?} \\
            }
        }
    }
    & \\ \\

    \theutterance \stepcounter{utterance}  
    & \multicolumn{4}{p{0.6\linewidth}}{
        \cellcolor[rgb]{0.9,0.9,0.9}{
            \makecell[{{p{\linewidth}}}]{
                \texttt{\tiny{[P1$\rangle$GM]}}
                \texttt{SPRICH: Status bei mir: T(1,1), W(1,3), H(4,3) bereits korrekt, C auf Ziel (3,6) gesetzt, I steht aktuell auf (4,2) wegen Linie bei (3,2). Euer Zielzustand wie vereinbart: W(1,3), I(3,1), T(1,1), C(3,6), H(4,3). Sobald du soweit bist, sag „fertig?“ oder gib mir Bescheid, falls I(3,1) noch geändert werden muss.} \\
            }
        }
    }
    & & \\ \\

    \theutterance \stepcounter{utterance}  
    & & & \multicolumn{2}{p{0.3\linewidth}}{
        \cellcolor[rgb]{0.9,0.9,0.9}{
            \makecell[{{p{\linewidth}}}]{
                \texttt{\tiny{[GM$|$GM]}}
                \texttt{SPRICH: Status bei mir: T(1,1), W(1,3), H(4,3) bereits korrekt, C auf Ziel (3,6) gesetzt, I steht aktuell auf (4,2) wegen Linie bei (3,2). Euer Zielzustand wie vereinbart: W(1,3), I(3,1), T(1,1), C(3,6), H(4,3). Sobald du soweit bist, sag „fertig?“ oder gib mir Bescheid, falls I(3,1) noch geändert werden muss.} \\
            }
        }
    }
    & & \\ \\

    \theutterance \stepcounter{utterance}  
    & & \multicolumn{4}{p{0.6\linewidth}}{
        \cellcolor[rgb]{0.9,0.9,0.9}{
            \makecell[{{p{\linewidth}}}]{
                \texttt{\tiny{[GM$\rangle$P2]}}
                \texttt{'C' erfolgreich nach (3,6) verschoben. Dein Raster sieht jetzt so aus:} \\
\texttt{```} \\
\texttt{ 1234567} \\
\texttt{╔═╤═╤═╤═╗} \\
\texttt{║T│I│◌│◌║ 1} \\
\texttt{╟─┼─┤◌│◌║ 2} \\
\texttt{║W│◌│◌│◌║ 3} \\
\texttt{║◌│◌├─┴─╢ 4} \\
\texttt{╟─┴─┤◌◌◌║ 5} \\
\texttt{║◌HC└─┬─╢ 6} \\
\texttt{║◌◌◌◌◌│◌║ 7} \\
\texttt{╚═════╧═╝} \\
\texttt{```} \\
\texttt{Ihr seid gerade bei Runde 9 von maximal 20.} \\
\texttt{Ihr habt gemeinsam 5 von maximal 12 Strafpunkten angesammelt.} \\
\texttt{Dein Mitspieler hat dir diese Nachricht geschickt:} \\
\\ 
\\ 
\texttt{"Status bei mir: T(1,1), W(1,3), H(4,3) bereits korrekt, C auf Ziel (3,6) gesetzt, I steht aktuell auf (4,2) wegen Linie bei (3,2). Euer Zielzustand wie vereinbart: W(1,3), I(3,1), T(1,1), C(3,6), H(4,3). Sobald du soweit bist, sag „fertig?“ oder gib mir Bescheid, falls I(3,1) noch geändert werden muss."} \\
\\ 
\\ 
\texttt{Wie lautet dein nächster Befehl?} \\
            }
        }
    }
    & \\ \\

    \theutterance \stepcounter{utterance}  
    & & & \multicolumn{4}{p{0.6\linewidth}}{
        \cellcolor[rgb]{0.9,0.9,0.9}{
            \makecell[{{p{\linewidth}}}]{
                \texttt{\tiny{[GM$\langle$P2]}}
                \texttt{SPRICH: fertig?} \\
            }
        }
    }
     \\ \\

    \theutterance \stepcounter{utterance}  
    & & & \multicolumn{2}{p{0.3\linewidth}}{
        \cellcolor[rgb]{0.9,0.9,0.9}{
            \makecell[{{p{\linewidth}}}]{
                \texttt{\tiny{[GM$|$GM]}}
                \texttt{SPRICH: fertig?} \\
            }
        }
    }
    & & \\ \\

    \theutterance \stepcounter{utterance}  
    & & \multicolumn{4}{p{0.6\linewidth}}{
        \cellcolor[rgb]{0.9,0.9,0.9}{
            \makecell[{{p{\linewidth}}}]{
                \texttt{\tiny{[P1$\langle$GM]}}
                \texttt{Deine Nachricht wurde an deinen Mitspieler übermittelt.} \\
\\ 
\\ 
\texttt{Ihr seid gerade bei Runde 9 von maximal 20.} \\
\texttt{Ihr habt gemeinsam 5 von maximal 12 Strafpunkten angesammelt.} \\
\texttt{Dein Mitspieler hat dir diese Nachricht geschickt:} \\
\\ 
\\ 
\texttt{"fertig?"} \\
\\ 
\\ 
\texttt{Wie lautet dein nächster Befehl?} \\
            }
        }
    }
    & \\ \\

    \theutterance \stepcounter{utterance}  
    & \multicolumn{4}{p{0.6\linewidth}}{
        \cellcolor[rgb]{0.9,0.9,0.9}{
            \makecell[{{p{\linewidth}}}]{
                \texttt{\tiny{[P1$\rangle$GM]}}
                \texttt{SPRICH: fertig!} \\
            }
        }
    }
    & & \\ \\

    \theutterance \stepcounter{utterance}  
    & & & \multicolumn{2}{p{0.3\linewidth}}{
        \cellcolor[rgb]{0.9,0.9,0.9}{
            \makecell[{{p{\linewidth}}}]{
                \texttt{\tiny{[GM$|$GM]}}
                \texttt{SPRICH: fertig!} \\
            }
        }
    }
    & & \\ \\

    \theutterance \stepcounter{utterance}  
    & & & \multicolumn{2}{p{0.3\linewidth}}{
        \cellcolor[rgb]{0.9,0.9,0.9}{
            \makecell[{{p{\linewidth}}}]{
                \texttt{\tiny{[GM$|$GM]}}
                \texttt{true} \\
            }
        }
    }
    & & \\ \\

    \theutterance \stepcounter{utterance}  
    & & & \multicolumn{2}{p{0.3\linewidth}}{
        \cellcolor[rgb]{0.9,0.9,0.9}{
            \makecell[{{p{\linewidth}}}]{
                \texttt{\tiny{[GM$|$GM]}}
                \texttt{* success: True} \\
\texttt{* lose: False} \\
\texttt{* aborted: False} \\
\texttt{{-}{-}{-}{-}{-}{-}{-}} \\
\texttt{* Shifts: 3.00} \\
\texttt{* Max Shifts: 8.00} \\
\texttt{* Min Shifts: 4.00} \\
\texttt{* End Distance Sum: 8.07} \\
\texttt{* Init Distance Sum: 19.39} \\
\texttt{* Expected Distance Sum: 20.95} \\
\texttt{* Penalties: 5.00} \\
\texttt{* Max Penalties: 12.00} \\
\texttt{* Rounds: 9.00} \\
\texttt{* Max Rounds: 20.00} \\
\texttt{* Object Count: 5.00} \\
            }
        }
    }
    & & \\ \\

    \theutterance \stepcounter{utterance}  
    & & & \multicolumn{2}{p{0.3\linewidth}}{
        \cellcolor[rgb]{0.9,0.9,0.9}{
            \makecell[{{p{\linewidth}}}]{
                \texttt{\tiny{[GM$|$GM]}}
                \texttt{Initial Grids:} \\
\texttt{```} \\
\texttt{Player 1:     Player 2:} \\
\texttt{ 1234567       1234567} \\
\texttt{╔═╤═╤═╤═╗     ╔═╤═╤═╤═╗} \\
\texttt{║T│◌│◌│◌║ 1   ║◌│◌│◌│◌║ 1} \\
\texttt{╟─┼─┤I│◌║ 2   ╟─┼─┤◌│◌║ 2} \\
\texttt{║W│◌│◌│H║ 3   ║◌│W│◌│◌║ 3} \\
\texttt{║◌│◌├─┴─╢ 4   ║◌│C├─┴─╢ 4} \\
\texttt{╟─┴─┤◌◌◌║ 5   ╟─┴─┤I◌◌║ 5} \\
\texttt{║◌C◌└─┬─╢ 6   ║◌H◌└─┬─╢ 6} \\
\texttt{║◌◌◌◌◌│◌║ 7   ║◌◌T◌◌│◌║ 7} \\
\texttt{╚═════╧═╝     ╚═════╧═╝} \\
\texttt{```} \\
\\ 
\texttt{Final Grids:} \\
\texttt{```} \\
\texttt{Player 1:     Player 2:} \\
\texttt{ 1234567       1234567} \\
\texttt{╔═╤═╤═╤═╗     ╔═╤═╤═╤═╗} \\
\texttt{║T│◌│◌│◌║ 1   ║T│I│◌│◌║ 1} \\
\texttt{╟─┼─┤I│◌║ 2   ╟─┼─┤◌│◌║ 2} \\
\texttt{║W│◌│◌│H║ 3   ║W│◌│◌│◌║ 3} \\
\texttt{║◌│◌├─┴─╢ 4   ║◌│◌├─┴─╢ 4} \\
\texttt{╟─┴─┤◌◌◌║ 5   ╟─┴─┤◌◌◌║ 5} \\
\texttt{║◌◌C└─┬─╢ 6   ║◌HC└─┬─╢ 6} \\
\texttt{║◌◌◌◌◌│◌║ 7   ║◌◌◌◌◌│◌║ 7} \\
\texttt{╚═════╧═╝     ╚═════╧═╝} \\
\texttt{```} \\
            }
        }
    }
    & & \\ \\

\end{supertabular}
}

\end{document}
