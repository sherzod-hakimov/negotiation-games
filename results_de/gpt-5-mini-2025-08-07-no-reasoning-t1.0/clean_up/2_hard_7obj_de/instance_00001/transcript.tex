
\documentclass{article}
\usepackage{colortbl}
\usepackage{makecell}
\usepackage{multirow}
\usepackage{supertabular}

\begin{document}

\newcounter{utterance}

\twocolumn

{ \footnotesize  \setcounter{utterance}{1}
\setlength{\tabcolsep}{0pt}
\begin{supertabular}{c@{$\;$}|p{.15\linewidth}@{}p{.15\linewidth}p{.15\linewidth}p{.15\linewidth}p{.15\linewidth}p{.15\linewidth}}

    \# & $\;$A & \multicolumn{4}{c}{Game Master} & $\;\:$B\\
    \hline 

    \theutterance \stepcounter{utterance}  

    & & \multicolumn{4}{p{0.6\linewidth}}{\cellcolor[rgb]{0.9,0.9,0.9}{%
	\makecell[{{p{\linewidth}}}]{% 
	  \tt {\tiny [A$\langle$GM]}  
	 Ich bin der Spielleiter, und du spielst ein kooperatives Spiel mit dem folgenden Raster als Spielbrett:\\ \tt \\ \tt ```\\ \tt  1234567\\ \tt ╔═══╤═╤═╗\\ \tt ║◌S◌│◌│A║ 1\\ \tt ╟─┬─┼─┴─╢ 2\\ \tt ║◌│◌│◌◌◌║ 3\\ \tt ║P├─┤D┌─╢ 4\\ \tt ╟─┘◌│◌│M║ 5\\ \tt ║◌◌◌└┬┴─╢ 6\\ \tt ║◌OT◌│◌◌║ 7\\ \tt ╚════╧══╝\\ \tt \\ \tt ```\\ \tt \\ \tt * Am oberen Rand sind die x-Koordinaten angegeben, die nach rechts hin zunehmen, und am rechten Rand die y-Koordinaten, die nach unten zunehmen.\\ \tt * Die folgenden Objekte sind zufällig auf deinem Raster platziert: 'P', 'O', 'T', 'S', 'D', 'A', 'M'.\\ \tt \\ \tt \\ \tt Dein Mitspieler sieht eine Variation des Spielbretts, auf dem dieselben Objekte an anderen zufälligen Positionen platziert sind. Du kannst das Spielbrett des anderen Spielers nicht sehen, und er kann deins nicht sehen.\\ \tt \\ \tt \\ \tt **Ziel des Spiels:**\\ \tt \\ \tt Beide Spieler müssen die Objekte auf ihrem jeweiligen Spielbrett so bewegen, dass identische Objekte am Ende an denselben Koordinaten liegen. Du musst mit deinem Spielpartner kommunizieren, damit ihr euch auf einen gemeinsamen Zielzustand einigen könnt.\\ \tt \\ \tt \\ \tt **Spielregeln:**\\ \tt \\ \tt * In jedem Spielzug kannst du genau eine der folgenden zwei Aktionen ausführen:\\ \tt 1. `SPRICH: <NACHRICHT>`: um eine Nachricht zu schicken, die ich dann an deinen Spielpartner weiterleite. Als Nachricht gilt alles bis zum nächsten Zeilenumbruch.\\ \tt 2. `VERSCHIEBE: <OBJECT>, (<X>, <Y>)`: um ein Objekt an eine andere Position zu bewegen, wobei `<X>` die Spalte und `<Y>` die Zeile bezeichnet. Ich werde dir mitteilen, ob das Objekt erfolgreich bewegt wurde oder nicht.\\ \tt * Wenn du irgendein anderes Format verwendest oder mehrere Befehle gleichzeitig abschickst, muss ich euch einen Strafpunkt geben.\\ \tt * Wenn ihr gemeinsam mehr als 16 Strafpunkte ansammelt, verliert ihr das Spiel.\\ \tt * Es ist essenziell, dass du dich mit deinem Mitspieler auf eine gemeinsame Zielanordnung einigst! Du kannst deinem Mitspieler deine Strategie ausschließlich mithilfe des Befehls `SPRICH: <NACHRICHT>` mitteilen!\\ \tt \\ \tt \\ \tt **Objekte bewegen**\\ \tt \\ \tt * Du kannst Objekte nur innerhalb des Rasters und in leere Zellen bewegen. Leere Zellen erkennst am Zeichen '◌'.\\ \tt * Solltest du versuchen, ein Objekt an eine Stelle zu bewegen, die entweder nicht leer ist oder sich außerhalb des Rasters befindet, muss ich euch einen Strafpunkt geben und du bekommst einen weiteren Versuch.\\ \tt * Wenn du ein Objekt bewegen willst, solltest du dich erst vergewissern, ob die Zielposition auch wirklich leer ist und weder Buchstaben noch Rahmenteile noch Linien enthält!\\ \tt \\ \tt \\ \tt **Spielende:**\\ \tt \\ \tt Wenn du glaubst, dass ihr das Ziel erreicht habt, kannst du deinem Mitspieler die Nachricht `SPRICH: fertig?` schicken. Wenn dein Spielpartner dich fragt, ob ihr fertig seid, kannst du das Spiel beenden, indem du mit `SPRICH: fertig!` antwortest.\\ \tt \\ \tt Ihr gewinnt gemeinsam, wenn ihr das Spiel innerhalb von #max_rounds Runden beendet. Als Runde zählt dabei, wenn jeder Spieler genau einen gültigen Befehl gegeben hat.\\ \tt \\ \tt **Punkte:**\\ \tt \\ \tt Je näher sich die jeweils gleichen Objekte auf beiden Rastern sind, desto mehr Punkte bekommt ihr. Für Strafpunkte gibt es Punktabzug. Könnt ihr den Rekord brechen?\\ \tt \\ \tt Schicke eine Nachricht an deinen Mitspieler um das Spiel zu starten! 
	  } 
	   } 
	   } 
	 & \\ 
 

    \theutterance \stepcounter{utterance}  

    & \multicolumn{4}{p{0.6\linewidth}}{\cellcolor[rgb]{0.8,1,0.9}{%
	\makecell[{{p{\linewidth}}}]{% 
	  \tt {\tiny [A$\rangle$GM]}  
	 SPRICH: Hallo! Lass uns eine gemeinsame Zielanordnung festlegen. Vorschlag: Ordnen wir die Objekte in alphabetischer Reihenfolge in Zeile 4 von links nach rechts an: A bei (1,4), D bei (2,4), M bei (3,4), O bei (4,4), P bei (5,4), S bei (6,4), T bei (7,4). Antworte mit OK oder mache einen anderen kompakten Vorschlag. 
	  } 
	   } 
	   } 
	 & & \\ 
 

    \theutterance \stepcounter{utterance}  

    & & & \multicolumn{2}{p{0.3\linewidth}}{\cellcolor[rgb]{0.95,0.95,0.95}{%
	\makecell[{{p{\linewidth}}}]{% 
	  \tt {\tiny [GM$|$GM]}  
	 SPRICH: Hallo! Lass uns eine gemeinsame Zielanordnung festlegen. Vorschlag: Ordnen wir die Objekte in alphabetischer Reihenfolge in Zeile 4 von links nach rechts an: A bei (1,4), D bei (2,4), M bei (3,4), O bei (4,4), P bei (5,4), S bei (6,4), T bei (7,4). Antworte mit OK oder mache einen anderen kompakten Vorschlag. 
	  } 
	   } 
	   } 
	 & & \\ 
 

    \theutterance \stepcounter{utterance}  

    & & \multicolumn{4}{p{0.6\linewidth}}{\cellcolor[rgb]{0.9,0.9,0.9}{%
	\makecell[{{p{\linewidth}}}]{% 
	  \tt {\tiny [GM$\rangle$B]}  
	 Ich bin der Spielleiter, und du spielst ein kooperatives Spiel mit dem folgenden Raster als Spielbrett:\\ \tt \\ \tt ```\\ \tt  1234567\\ \tt ╔═══╤═╤═╗\\ \tt ║P◌A│◌│◌║ 1\\ \tt ╟─┬─┼─┴─╢ 2\\ \tt ║◌│D│◌◌◌║ 3\\ \tt ║◌├─┤◌┌─╢ 4\\ \tt ╟─┘◌│◌│O║ 5\\ \tt ║◌◌◌└┬┴─╢ 6\\ \tt ║◌S◌T│◌M║ 7\\ \tt ╚════╧══╝\\ \tt \\ \tt ```\\ \tt \\ \tt * Am oberen Rand sind die x-Koordinaten angegeben, die nach rechts hin zunehmen, und am rechten Rand die y-Koordinaten, die nach unten zunehmen.\\ \tt * Die folgenden Objekte sind zufällig auf deinem Raster platziert: 'P', 'O', 'T', 'S', 'D', 'A', 'M'.\\ \tt \\ \tt \\ \tt Dein Mitspieler sieht eine Variation des Spielbretts, auf dem dieselben Objekte an anderen zufälligen Positionen platziert sind. Du kannst das Spielbrett des anderen Spielers nicht sehen, und er kann deins nicht sehen.\\ \tt \\ \tt \\ \tt **Ziel des Spiels:**\\ \tt \\ \tt Beide Spieler müssen die Objekte auf ihrem jeweiligen Spielbrett so bewegen, dass identische Objekte am Ende an denselben Koordinaten liegen. Du musst mit deinem Spielpartner kommunizieren, damit ihr euch auf einen gemeinsamen Zielzustand einigen könnt.\\ \tt \\ \tt \\ \tt **Spielregeln:**\\ \tt \\ \tt * In jedem Spielzug kannst du genau eine der folgenden zwei Aktionen ausführen:\\ \tt 1. `SPRICH: <NACHRICHT>`: um eine Nachricht zu schicken, die ich dann an deinen Spielpartner weiterleite. Als Nachricht gilt alles bis zum nächsten Zeilenumbruch.\\ \tt 2. `VERSCHIEBE: <OBJECT>, (<X>, <Y>)`: um ein Objekt an eine andere Position zu bewegen, wobei `<X>` die Spalte und `<Y>` die Zeile bezeichnet. Ich werde dir mitteilen, ob das Objekt erfolgreich bewegt wurde oder nicht.\\ \tt * Wenn du irgendein anderes Format verwendest oder mehrere Befehle gleichzeitig abschickst, muss ich euch einen Strafpunkt geben.\\ \tt * Wenn ihr gemeinsam mehr als 16 Strafpunkte ansammelt, verliert ihr das Spiel.\\ \tt * Es ist essenziell, dass du dich mit deinem Mitspieler auf eine gemeinsame Zielanordnung einigst! Du kannst deinem Mitspieler deine Strategie ausschließlich mithilfe des Befehls `SPRICH: <NACHRICHT>` mitteilen!\\ \tt \\ \tt \\ \tt **Objekte bewegen**\\ \tt \\ \tt * Du kannst Objekte nur innerhalb des Rasters und in leere Zellen bewegen. Leere Zellen erkennst am Zeichen '◌'.\\ \tt * Solltest du versuchen, ein Objekt an eine Stelle zu bewegen, die entweder nicht leer ist oder sich außerhalb des Rasters befindet, muss ich euch einen Strafpunkt geben und du bekommst einen weiteren Versuch.\\ \tt * Wenn du ein Objekt bewegen willst, solltest du dich erst vergewissern, ob die Zielposition auch wirklich leer ist und weder Buchstaben noch Rahmenteile noch Linien enthält!\\ \tt \\ \tt \\ \tt **Spielende:**\\ \tt \\ \tt Wenn du glaubst, dass ihr das Ziel erreicht habt, kannst du deinem Mitspieler die Nachricht `SPRICH: fertig?` schicken. Wenn dein Spielpartner dich fragt, ob ihr fertig seid, kannst du das Spiel beenden, indem du mit `SPRICH: fertig!` antwortest.\\ \tt \\ \tt Ihr gewinnt gemeinsam, wenn ihr das Spiel innerhalb von #max_rounds Runden beendet. Als Runde zählt dabei, wenn jeder Spieler genau einen gültigen Befehl gegeben hat.\\ \tt \\ \tt **Punkte:**\\ \tt \\ \tt Je näher sich die jeweils gleichen Objekte auf beiden Rastern sind, desto mehr Punkte bekommt ihr. Für Strafpunkte gibt es Punktabzug. Könnt ihr den Rekord brechen?\\ \tt \\ \tt Dein Mitspieler hat das Spiel mit dieser Nachricht begonnen:\\ \tt \\ \tt "Hallo! Lass uns eine gemeinsame Zielanordnung festlegen. Vorschlag: Ordnen wir die Objekte in alphabetischer Reihenfolge in Zeile 4 von links nach rechts an: A bei (1,4), D bei (2,4), M bei (3,4), O bei (4,4), P bei (5,4), S bei (6,4), T bei (7,4). Antworte mit OK oder mache einen anderen kompakten Vorschlag."\\ \tt \\ \tt Wie lautet dein erster Befehl? 
	  } 
	   } 
	   } 
	 & \\ 
 

    \theutterance \stepcounter{utterance}  

    & & & \multicolumn{4}{p{0.6\linewidth}}{\cellcolor[rgb]{1,0.85,0.72}{%
	\makecell[{{p{\linewidth}}}]{% 
	  \tt {\tiny [GM$\langle$B]}  
	 SPRICH: OK. Ziel: Zeile 4 von links nach rechts A(1,4) D(2,4) M(3,4) O(4,4) P(5,4) S(6,4) T(7,4). Ich beginne mit ersten Zügen, sag Bescheid wenn du abweichst. 
	  } 
	   } 
	   } 
	  \\ 
 

    \theutterance \stepcounter{utterance}  

    & & & \multicolumn{2}{p{0.3\linewidth}}{\cellcolor[rgb]{0.95,0.95,0.95}{%
	\makecell[{{p{\linewidth}}}]{% 
	  \tt {\tiny [GM$|$GM]}  
	 SPRICH: OK. Ziel: Zeile 4 von links nach rechts A(1,4) D(2,4) M(3,4) O(4,4) P(5,4) S(6,4) T(7,4). Ich beginne mit ersten Zügen, sag Bescheid wenn du abweichst. 
	  } 
	   } 
	   } 
	 & & \\ 
 

    \theutterance \stepcounter{utterance}  

    & & \multicolumn{4}{p{0.6\linewidth}}{\cellcolor[rgb]{0.9,0.9,0.9}{%
	\makecell[{{p{\linewidth}}}]{% 
	  \tt {\tiny [A$\langle$GM]}  
	 Deine Nachricht wurde an deinen Mitspieler übermittelt.\\ \tt \\ \tt \\ \tt Ihr seid gerade bei Runde 1 von maximal 28.\\ \tt Ihr habt gemeinsam 0 von maximal 16 Strafpunkten angesammelt.\\ \tt Dein Mitspieler hat dir diese Nachricht geschickt:\\ \tt \\ \tt \\ \tt "OK. Ziel: Zeile 4 von links nach rechts A(1,4) D(2,4) M(3,4) O(4,4) P(5,4) S(6,4) T(7,4). Ich beginne mit ersten Zügen, sag Bescheid wenn du abweichst."\\ \tt \\ \tt \\ \tt Wie lautet dein nächster Befehl? 
	  } 
	   } 
	   } 
	 & \\ 
 

    \theutterance \stepcounter{utterance}  

    & \multicolumn{4}{p{0.6\linewidth}}{\cellcolor[rgb]{0.8,1,0.9}{%
	\makecell[{{p{\linewidth}}}]{% 
	  \tt {\tiny [A$\rangle$GM]}  
	 VERSCHIEBE: A, (1,4) 
	  } 
	   } 
	   } 
	 & & \\ 
 

    \theutterance \stepcounter{utterance}  

    & & & \multicolumn{2}{p{0.3\linewidth}}{\cellcolor[rgb]{0.95,0.95,0.95}{%
	\makecell[{{p{\linewidth}}}]{% 
	  \tt {\tiny [GM$|$GM]}  
	 VERSCHIEBE: A, (1,4) 
	  } 
	   } 
	   } 
	 & & \\ 
 

    \theutterance \stepcounter{utterance}  

    & & & \multicolumn{2}{p{0.3\linewidth}}{\cellcolor[rgb]{0.95,0.95,0.95}{%
	\makecell[{{p{\linewidth}}}]{% 
	  \tt {\tiny [GM$|$GM]}  
	 Strafpunkt: (1,4) ist nicht leer, sondern enthält 'P'.\\ \tt \\ \tt Ihr habt gemeinsam 1 von maximal 16 Strafpunkten angesammelt.\\ \tt Bitte versuche es noch mal! 
	  } 
	   } 
	   } 
	 & & \\ 
 

    \theutterance \stepcounter{utterance}  

    & & \multicolumn{4}{p{0.6\linewidth}}{\cellcolor[rgb]{0.9,0.9,0.9}{%
	\makecell[{{p{\linewidth}}}]{% 
	  \tt {\tiny [A$\langle$GM]}  
	 Strafpunkt: (1,4) ist nicht leer, sondern enthält 'P'.\\ \tt \\ \tt Ihr habt gemeinsam 1 von maximal 16 Strafpunkten angesammelt.\\ \tt Bitte versuche es noch mal! 
	  } 
	   } 
	   } 
	 & \\ 
 

    \theutterance \stepcounter{utterance}  

    & \multicolumn{4}{p{0.6\linewidth}}{\cellcolor[rgb]{0.8,1,0.9}{%
	\makecell[{{p{\linewidth}}}]{% 
	  \tt {\tiny [A$\rangle$GM]}  
	 SPRICH: Danke, mein Fehler. Ich prüfe erst die Felder. Aktuelle Positionen bei mir: P(1,4), O(2,7), T(3,7), S(2,1), D(3,4), A(7,1), M(7,5). Ich beginne mit S nach (6,4), da Ziel dort S ist und es frei wirkt. Bitte sag, falls bei dir Konflikt entsteht. 
	  } 
	   } 
	   } 
	 & & \\ 
 

    \theutterance \stepcounter{utterance}  

    & & & \multicolumn{2}{p{0.3\linewidth}}{\cellcolor[rgb]{0.95,0.95,0.95}{%
	\makecell[{{p{\linewidth}}}]{% 
	  \tt {\tiny [GM$|$GM]}  
	 SPRICH: Danke, mein Fehler. Ich prüfe erst die Felder. Aktuelle Positionen bei mir: P(1,4), O(2,7), T(3,7), S(2,1), D(3,4), A(7,1), M(7,5). Ich beginne mit S nach (6,4), da Ziel dort S ist und es frei wirkt. Bitte sag, falls bei dir Konflikt entsteht. 
	  } 
	   } 
	   } 
	 & & \\ 
 

    \theutterance \stepcounter{utterance}  

    & & \multicolumn{4}{p{0.6\linewidth}}{\cellcolor[rgb]{0.9,0.9,0.9}{%
	\makecell[{{p{\linewidth}}}]{% 
	  \tt {\tiny [GM$\rangle$B]}  
	 Deine Nachricht wurde an deinen Mitspieler übermittelt.\\ \tt \\ \tt \\ \tt Ihr seid gerade bei Runde 2 von maximal 28.\\ \tt Ihr habt gemeinsam 1 von maximal 16 Strafpunkten angesammelt.\\ \tt Dein Mitspieler hat dir diese Nachricht geschickt:\\ \tt \\ \tt \\ \tt "Danke, mein Fehler. Ich prüfe erst die Felder. Aktuelle Positionen bei mir: P(1,4), O(2,7), T(3,7), S(2,1), D(3,4), A(7,1), M(7,5). Ich beginne mit S nach (6,4), da Ziel dort S ist und es frei wirkt. Bitte sag, falls bei dir Konflikt entsteht."\\ \tt \\ \tt \\ \tt Wie lautet dein nächster Befehl? 
	  } 
	   } 
	   } 
	 & \\ 
 

    \theutterance \stepcounter{utterance}  

    & & & \multicolumn{4}{p{0.6\linewidth}}{\cellcolor[rgb]{1,0.85,0.72}{%
	\makecell[{{p{\linewidth}}}]{% 
	  \tt {\tiny [GM$\langle$B]}  
	 SPRICH: Meine Startpositionen: P(1,1), O(7,5), T(4,7), S(2,7), D(2,3), A(3,1), M(7,7). Zielreihe (4) prüfen: (1,4) frei, (2,4) hat Wand, (3,4) hat Wand, (4,4) hat Wand/Block, (5,4) frei, (6,4) frei, (7,4) hat Rahmen rechts. Daher erreichbar und frei scheinen nur (1,4), (5,4), (6,4). Vorschlag: Ich platziere A→(1,4), P→(5,4), S→(6,4). Rest später, wenn Wege frei. Ich starte mit A nach (1,4). Bitte bestätige S→(6,4) bei dir. 
	  } 
	   } 
	   } 
	  \\ 
 

    \theutterance \stepcounter{utterance}  

    & & & \multicolumn{2}{p{0.3\linewidth}}{\cellcolor[rgb]{0.95,0.95,0.95}{%
	\makecell[{{p{\linewidth}}}]{% 
	  \tt {\tiny [GM$|$GM]}  
	 SPRICH: Meine Startpositionen: P(1,1), O(7,5), T(4,7), S(2,7), D(2,3), A(3,1), M(7,7). Zielreihe (4) prüfen: (1,4) frei, (2,4) hat Wand, (3,4) hat Wand, (4,4) hat Wand/Block, (5,4) frei, (6,4) frei, (7,4) hat Rahmen rechts. Daher erreichbar und frei scheinen nur (1,4), (5,4), (6,4). Vorschlag: Ich platziere A→(1,4), P→(5,4), S→(6,4). Rest später, wenn Wege frei. Ich starte mit A nach (1,4). Bitte bestätige S→(6,4) bei dir. 
	  } 
	   } 
	   } 
	 & & \\ 
 

    \theutterance \stepcounter{utterance}  

    & & \multicolumn{4}{p{0.6\linewidth}}{\cellcolor[rgb]{0.9,0.9,0.9}{%
	\makecell[{{p{\linewidth}}}]{% 
	  \tt {\tiny [A$\langle$GM]}  
	 Deine Nachricht wurde an deinen Mitspieler übermittelt.\\ \tt \\ \tt \\ \tt Ihr seid gerade bei Runde 2 von maximal 28.\\ \tt Ihr habt gemeinsam 1 von maximal 16 Strafpunkten angesammelt.\\ \tt Dein Mitspieler hat dir diese Nachricht geschickt:\\ \tt \\ \tt \\ \tt "Meine Startpositionen: P(1,1), O(7,5), T(4,7), S(2,7), D(2,3), A(3,1), M(7,7). Zielreihe (4) prüfen: (1,4) frei, (2,4) hat Wand, (3,4) hat Wand, (4,4) hat Wand/Block, (5,4) frei, (6,4) frei, (7,4) hat Rahmen rechts. Daher erreichbar und frei scheinen nur (1,4), (5,4), (6,4). Vorschlag: Ich platziere A→(1,4), P→(5,4), S→(6,4). Rest später, wenn Wege frei. Ich starte mit A nach (1,4). Bitte bestätige S→(6,4) bei dir."\\ \tt \\ \tt \\ \tt Wie lautet dein nächster Befehl? 
	  } 
	   } 
	   } 
	 & \\ 
 

    \theutterance \stepcounter{utterance}  

    & \multicolumn{4}{p{0.6\linewidth}}{\cellcolor[rgb]{0.8,1,0.9}{%
	\makecell[{{p{\linewidth}}}]{% 
	  \tt {\tiny [A$\rangle$GM]}  
	 VERSCHIEBE: S, (6,4) 
	  } 
	   } 
	   } 
	 & & \\ 
 

    \theutterance \stepcounter{utterance}  

    & & & \multicolumn{2}{p{0.3\linewidth}}{\cellcolor[rgb]{0.95,0.95,0.95}{%
	\makecell[{{p{\linewidth}}}]{% 
	  \tt {\tiny [GM$|$GM]}  
	 VERSCHIEBE: S, (6,4) 
	  } 
	   } 
	   } 
	 & & \\ 
 

    \theutterance \stepcounter{utterance}  

    & & & \multicolumn{2}{p{0.3\linewidth}}{\cellcolor[rgb]{0.95,0.95,0.95}{%
	\makecell[{{p{\linewidth}}}]{% 
	  \tt {\tiny [GM$|$GM]}  
	 Strafpunkt: (6,4) ist nicht leer, sondern enthält '┌'.\\ \tt \\ \tt Ihr habt gemeinsam 2 von maximal 16 Strafpunkten angesammelt.\\ \tt Bitte versuche es noch mal! 
	  } 
	   } 
	   } 
	 & & \\ 
 

    \theutterance \stepcounter{utterance}  

    & & \multicolumn{4}{p{0.6\linewidth}}{\cellcolor[rgb]{0.9,0.9,0.9}{%
	\makecell[{{p{\linewidth}}}]{% 
	  \tt {\tiny [A$\langle$GM]}  
	 Strafpunkt: (6,4) ist nicht leer, sondern enthält '┌'.\\ \tt \\ \tt Ihr habt gemeinsam 2 von maximal 16 Strafpunkten angesammelt.\\ \tt Bitte versuche es noch mal! 
	  } 
	   } 
	   } 
	 & \\ 
 

    \theutterance \stepcounter{utterance}  

    & \multicolumn{4}{p{0.6\linewidth}}{\cellcolor[rgb]{0.8,1,0.9}{%
	\makecell[{{p{\linewidth}}}]{% 
	  \tt {\tiny [A$\rangle$GM]}  
	 SPRICH: Danke für den Hinweis. Ich kartiere meine freie Zielreihe 4: (1,4)=P, (2,4)=─, (3,4)=┤, (4,4)=D, (5,4)=┌, (6,4)=─, (7,4)=╢. Frei in Reihe 4 ist aktuell nichts. Vorschlag: Wechsel auf Zielspalten statt Zielreihe, um Kollisionen zu vermeiden. Neuer gemeinsamer Zielplan: Ordnen wir die Buchstaben in Spalte 7 von oben nach unten A(7,1), D(7,2), M(7,3), O(7,4), P(7,5), S(7,6), T(7,7). Bei mir sind 7,1=A und 7,5=M belegt, Rest teils frei. Bist du einverstanden? Wenn ja, starte ich mit O→(7,4), sofern frei. 
	  } 
	   } 
	   } 
	 & & \\ 
 

    \theutterance \stepcounter{utterance}  

    & & & \multicolumn{2}{p{0.3\linewidth}}{\cellcolor[rgb]{0.95,0.95,0.95}{%
	\makecell[{{p{\linewidth}}}]{% 
	  \tt {\tiny [GM$|$GM]}  
	 SPRICH: Danke für den Hinweis. Ich kartiere meine freie Zielreihe 4: (1,4)=P, (2,4)=─, (3,4)=┤, (4,4)=D, (5,4)=┌, (6,4)=─, (7,4)=╢. Frei in Reihe 4 ist aktuell nichts. Vorschlag: Wechsel auf Zielspalten statt Zielreihe, um Kollisionen zu vermeiden. Neuer gemeinsamer Zielplan: Ordnen wir die Buchstaben in Spalte 7 von oben nach unten A(7,1), D(7,2), M(7,3), O(7,4), P(7,5), S(7,6), T(7,7). Bei mir sind 7,1=A und 7,5=M belegt, Rest teils frei. Bist du einverstanden? Wenn ja, starte ich mit O→(7,4), sofern frei. 
	  } 
	   } 
	   } 
	 & & \\ 
 

    \theutterance \stepcounter{utterance}  

    & & \multicolumn{4}{p{0.6\linewidth}}{\cellcolor[rgb]{0.9,0.9,0.9}{%
	\makecell[{{p{\linewidth}}}]{% 
	  \tt {\tiny [GM$\rangle$B]}  
	 Deine Nachricht wurde an deinen Mitspieler übermittelt.\\ \tt \\ \tt \\ \tt Ihr seid gerade bei Runde 3 von maximal 28.\\ \tt Ihr habt gemeinsam 2 von maximal 16 Strafpunkten angesammelt.\\ \tt Dein Mitspieler hat dir diese Nachricht geschickt:\\ \tt \\ \tt \\ \tt "Danke für den Hinweis. Ich kartiere meine freie Zielreihe 4: (1,4)=P, (2,4)=─, (3,4)=┤, (4,4)=D, (5,4)=┌, (6,4)=─, (7,4)=╢. Frei in Reihe 4 ist aktuell nichts. Vorschlag: Wechsel auf Zielspalten statt Zielreihe, um Kollisionen zu vermeiden. Neuer gemeinsamer Zielplan: Ordnen wir die Buchstaben in Spalte 7 von oben nach unten A(7,1), D(7,2), M(7,3), O(7,4), P(7,5), S(7,6), T(7,7). Bei mir sind 7,1=A und 7,5=M belegt, Rest teils frei. Bist du einverstanden? Wenn ja, starte ich mit O→(7,4), sofern frei."\\ \tt \\ \tt \\ \tt Wie lautet dein nächster Befehl? 
	  } 
	   } 
	   } 
	 & \\ 
 

    \theutterance \stepcounter{utterance}  

    & & & \multicolumn{4}{p{0.6\linewidth}}{\cellcolor[rgb]{1,0.85,0.72}{%
	\makecell[{{p{\linewidth}}}]{% 
	  \tt {\tiny [GM$\langle$B]}  
	 SPRICH: Einverstanden mit Spalte 7: A(7,1), D(7,2), M(7,3), O(7,4), P(7,5), S(7,6), T(7,7). Bei mir aktuell in Spalte 7: O(7,5), M(7,7); Zellen (7,1) Rahmen, (7,2) Rahmen, (7,3) Rahmen, (7,4) Rahmen/Block?, (7,5)=O, (7,6) frei?, (7,7)=M. Ich kann S→(7,6) bewegen; bestätige, ob (7,6) bei dir frei ist, und führe O→(7,4) aus, wenn frei. Ich starte danach mit S→(7,6). 
	  } 
	   } 
	   } 
	  \\ 
 

    \theutterance \stepcounter{utterance}  

    & & & \multicolumn{2}{p{0.3\linewidth}}{\cellcolor[rgb]{0.95,0.95,0.95}{%
	\makecell[{{p{\linewidth}}}]{% 
	  \tt {\tiny [GM$|$GM]}  
	 SPRICH: Einverstanden mit Spalte 7: A(7,1), D(7,2), M(7,3), O(7,4), P(7,5), S(7,6), T(7,7). Bei mir aktuell in Spalte 7: O(7,5), M(7,7); Zellen (7,1) Rahmen, (7,2) Rahmen, (7,3) Rahmen, (7,4) Rahmen/Block?, (7,5)=O, (7,6) frei?, (7,7)=M. Ich kann S→(7,6) bewegen; bestätige, ob (7,6) bei dir frei ist, und führe O→(7,4) aus, wenn frei. Ich starte danach mit S→(7,6). 
	  } 
	   } 
	   } 
	 & & \\ 
 

    \theutterance \stepcounter{utterance}  

    & & \multicolumn{4}{p{0.6\linewidth}}{\cellcolor[rgb]{0.9,0.9,0.9}{%
	\makecell[{{p{\linewidth}}}]{% 
	  \tt {\tiny [A$\langle$GM]}  
	 Deine Nachricht wurde an deinen Mitspieler übermittelt.\\ \tt \\ \tt \\ \tt Ihr seid gerade bei Runde 3 von maximal 28.\\ \tt Ihr habt gemeinsam 2 von maximal 16 Strafpunkten angesammelt.\\ \tt Dein Mitspieler hat dir diese Nachricht geschickt:\\ \tt \\ \tt \\ \tt "Einverstanden mit Spalte 7: A(7,1), D(7,2), M(7,3), O(7,4), P(7,5), S(7,6), T(7,7). Bei mir aktuell in Spalte 7: O(7,5), M(7,7); Zellen (7,1) Rahmen, (7,2) Rahmen, (7,3) Rahmen, (7,4) Rahmen/Block?, (7,5)=O, (7,6) frei?, (7,7)=M. Ich kann S→(7,6) bewegen; bestätige, ob (7,6) bei dir frei ist, und führe O→(7,4) aus, wenn frei. Ich starte danach mit S→(7,6)."\\ \tt \\ \tt \\ \tt Wie lautet dein nächster Befehl? 
	  } 
	   } 
	   } 
	 & \\ 
 

    \theutterance \stepcounter{utterance}  

    & \multicolumn{4}{p{0.6\linewidth}}{\cellcolor[rgb]{0.8,1,0.9}{%
	\makecell[{{p{\linewidth}}}]{% 
	  \tt {\tiny [A$\rangle$GM]}  
	 SPRICH: Bei mir Spalte 7: (7,1)=A, (7,2)=╢ Wand, (7,3)=◌ frei, (7,4)=╢ Wand, (7,5)=M, (7,6)=╢ Wand, (7,7)=◌ frei. Also (7,6) ist bei mir blockiert, nicht frei. Ich kann O nicht nach (7,4) setzen (Wand). Vorschlag: Alternativ Spalte 3 anpeilen: A(3,1), D(3,2), M(3,3), O(3,4), P(3,5), S(3,6), T(3,7). Bei mir in Spalte 3 sind (3,1)=◌, (3,2)=─ Block, (3,3)=◌, (3,4)=D, (3,5)=◌, (3,6)=◌, (3,7)=T. Ich könnte P→(3,5) sofort setzen. Passt Spalte 3 bei dir weitgehend? 
	  } 
	   } 
	   } 
	 & & \\ 
 

    \theutterance \stepcounter{utterance}  

    & & & \multicolumn{2}{p{0.3\linewidth}}{\cellcolor[rgb]{0.95,0.95,0.95}{%
	\makecell[{{p{\linewidth}}}]{% 
	  \tt {\tiny [GM$|$GM]}  
	 SPRICH: Bei mir Spalte 7: (7,1)=A, (7,2)=╢ Wand, (7,3)=◌ frei, (7,4)=╢ Wand, (7,5)=M, (7,6)=╢ Wand, (7,7)=◌ frei. Also (7,6) ist bei mir blockiert, nicht frei. Ich kann O nicht nach (7,4) setzen (Wand). Vorschlag: Alternativ Spalte 3 anpeilen: A(3,1), D(3,2), M(3,3), O(3,4), P(3,5), S(3,6), T(3,7). Bei mir in Spalte 3 sind (3,1)=◌, (3,2)=─ Block, (3,3)=◌, (3,4)=D, (3,5)=◌, (3,6)=◌, (3,7)=T. Ich könnte P→(3,5) sofort setzen. Passt Spalte 3 bei dir weitgehend? 
	  } 
	   } 
	   } 
	 & & \\ 
 

    \theutterance \stepcounter{utterance}  

    & & \multicolumn{4}{p{0.6\linewidth}}{\cellcolor[rgb]{0.9,0.9,0.9}{%
	\makecell[{{p{\linewidth}}}]{% 
	  \tt {\tiny [GM$\rangle$B]}  
	 Deine Nachricht wurde an deinen Mitspieler übermittelt.\\ \tt \\ \tt \\ \tt Ihr seid gerade bei Runde 4 von maximal 28.\\ \tt Ihr habt gemeinsam 2 von maximal 16 Strafpunkten angesammelt.\\ \tt Dein Mitspieler hat dir diese Nachricht geschickt:\\ \tt \\ \tt \\ \tt "Bei mir Spalte 7: (7,1)=A, (7,2)=╢ Wand, (7,3)=◌ frei, (7,4)=╢ Wand, (7,5)=M, (7,6)=╢ Wand, (7,7)=◌ frei. Also (7,6) ist bei mir blockiert, nicht frei. Ich kann O nicht nach (7,4) setzen (Wand). Vorschlag: Alternativ Spalte 3 anpeilen: A(3,1), D(3,2), M(3,3), O(3,4), P(3,5), S(3,6), T(3,7). Bei mir in Spalte 3 sind (3,1)=◌, (3,2)=─ Block, (3,3)=◌, (3,4)=D, (3,5)=◌, (3,6)=◌, (3,7)=T. Ich könnte P→(3,5) sofort setzen. Passt Spalte 3 bei dir weitgehend?"\\ \tt \\ \tt \\ \tt Wie lautet dein nächster Befehl? 
	  } 
	   } 
	   } 
	 & \\ 
 

    \theutterance \stepcounter{utterance}  

    & & & \multicolumn{4}{p{0.6\linewidth}}{\cellcolor[rgb]{1,0.85,0.72}{%
	\makecell[{{p{\linewidth}}}]{% 
	  \tt {\tiny [GM$\langle$B]}  
	 SPRICH: Spalte 3 passt gut. Bei mir Spalte 3: (3,1)=A, (3,2)=Wand, (3,3)=frei, (3,4)=Wand, (3,5)=frei, (3,6)=frei, (3,7)=frei. Ziel bestätigt: A(3,1), D(3,2), M(3,3), O(3,4), P(3,5), S(3,6), T(3,7). Ich setze jetzt S→(3,6). Bitte setze P→(3,5) wie vorgeschlagen. 
	  } 
	   } 
	   } 
	  \\ 
 

    \theutterance \stepcounter{utterance}  

    & & & \multicolumn{2}{p{0.3\linewidth}}{\cellcolor[rgb]{0.95,0.95,0.95}{%
	\makecell[{{p{\linewidth}}}]{% 
	  \tt {\tiny [GM$|$GM]}  
	 SPRICH: Spalte 3 passt gut. Bei mir Spalte 3: (3,1)=A, (3,2)=Wand, (3,3)=frei, (3,4)=Wand, (3,5)=frei, (3,6)=frei, (3,7)=frei. Ziel bestätigt: A(3,1), D(3,2), M(3,3), O(3,4), P(3,5), S(3,6), T(3,7). Ich setze jetzt S→(3,6). Bitte setze P→(3,5) wie vorgeschlagen. 
	  } 
	   } 
	   } 
	 & & \\ 
 

    \theutterance \stepcounter{utterance}  

    & & \multicolumn{4}{p{0.6\linewidth}}{\cellcolor[rgb]{0.9,0.9,0.9}{%
	\makecell[{{p{\linewidth}}}]{% 
	  \tt {\tiny [A$\langle$GM]}  
	 Deine Nachricht wurde an deinen Mitspieler übermittelt.\\ \tt \\ \tt \\ \tt Ihr seid gerade bei Runde 4 von maximal 28.\\ \tt Ihr habt gemeinsam 2 von maximal 16 Strafpunkten angesammelt.\\ \tt Dein Mitspieler hat dir diese Nachricht geschickt:\\ \tt \\ \tt \\ \tt "Spalte 3 passt gut. Bei mir Spalte 3: (3,1)=A, (3,2)=Wand, (3,3)=frei, (3,4)=Wand, (3,5)=frei, (3,6)=frei, (3,7)=frei. Ziel bestätigt: A(3,1), D(3,2), M(3,3), O(3,4), P(3,5), S(3,6), T(3,7). Ich setze jetzt S→(3,6). Bitte setze P→(3,5) wie vorgeschlagen."\\ \tt \\ \tt \\ \tt Wie lautet dein nächster Befehl? 
	  } 
	   } 
	   } 
	 & \\ 
 

    \theutterance \stepcounter{utterance}  

    & \multicolumn{4}{p{0.6\linewidth}}{\cellcolor[rgb]{0.8,1,0.9}{%
	\makecell[{{p{\linewidth}}}]{% 
	  \tt {\tiny [A$\rangle$GM]}  
	 VERSCHIEBE: P, (3,5) 
	  } 
	   } 
	   } 
	 & & \\ 
 

    \theutterance \stepcounter{utterance}  

    & & & \multicolumn{2}{p{0.3\linewidth}}{\cellcolor[rgb]{0.95,0.95,0.95}{%
	\makecell[{{p{\linewidth}}}]{% 
	  \tt {\tiny [GM$|$GM]}  
	 VERSCHIEBE: P, (3,5) 
	  } 
	   } 
	   } 
	 & & \\ 
 

    \theutterance \stepcounter{utterance}  

    & & & \multicolumn{2}{p{0.3\linewidth}}{\cellcolor[rgb]{0.95,0.95,0.95}{%
	\makecell[{{p{\linewidth}}}]{% 
	  \tt {\tiny [GM$|$GM]}  
	 'P' erfolgreich nach (3,5) verschoben. Dein Raster sieht jetzt so aus:\\ \tt ```\\ \tt  1234567\\ \tt ╔═══╤═╤═╗\\ \tt ║◌S◌│◌│A║ 1\\ \tt ╟─┬─┼─┴─╢ 2\\ \tt ║◌│◌│◌◌◌║ 3\\ \tt ║◌├─┤D┌─╢ 4\\ \tt ╟─┘P│◌│M║ 5\\ \tt ║◌◌◌└┬┴─╢ 6\\ \tt ║◌OT◌│◌◌║ 7\\ \tt ╚════╧══╝\\ \tt ```\\ \tt  
	  } 
	   } 
	   } 
	 & & \\ 
 

    \theutterance \stepcounter{utterance}  

    & & \multicolumn{4}{p{0.6\linewidth}}{\cellcolor[rgb]{0.9,0.9,0.9}{%
	\makecell[{{p{\linewidth}}}]{% 
	  \tt {\tiny [GM$\rangle$B]}  
	 Deine Nachricht wurde an deinen Mitspieler übermittelt.\\ \tt \\ \tt \\ \tt Ihr seid gerade bei Runde 5 von maximal 28.\\ \tt Ihr habt gemeinsam 2 von maximal 16 Strafpunkten angesammelt.\\ \tt Dein Mitspieler hat ein Objekt auf seinem Raster bewegt.\\ \tt \\ \tt \\ \tt Wie lautet dein nächster Befehl? 
	  } 
	   } 
	   } 
	 & \\ 
 

    \theutterance \stepcounter{utterance}  

    & & & \multicolumn{4}{p{0.6\linewidth}}{\cellcolor[rgb]{1,0.85,0.72}{%
	\makecell[{{p{\linewidth}}}]{% 
	  \tt {\tiny [GM$\langle$B]}  
	 VERSCHIEBE: S, (3,6) 
	  } 
	   } 
	   } 
	  \\ 
 

    \theutterance \stepcounter{utterance}  

    & & & \multicolumn{2}{p{0.3\linewidth}}{\cellcolor[rgb]{0.95,0.95,0.95}{%
	\makecell[{{p{\linewidth}}}]{% 
	  \tt {\tiny [GM$|$GM]}  
	 VERSCHIEBE: S, (3,6) 
	  } 
	   } 
	   } 
	 & & \\ 
 

    \theutterance \stepcounter{utterance}  

    & & & \multicolumn{2}{p{0.3\linewidth}}{\cellcolor[rgb]{0.95,0.95,0.95}{%
	\makecell[{{p{\linewidth}}}]{% 
	  \tt {\tiny [GM$|$GM]}  
	 'S' erfolgreich nach (3,6) verschoben. Dein Raster sieht jetzt so aus:\\ \tt ```\\ \tt  1234567\\ \tt ╔═══╤═╤═╗\\ \tt ║P◌A│◌│◌║ 1\\ \tt ╟─┬─┼─┴─╢ 2\\ \tt ║◌│D│◌◌◌║ 3\\ \tt ║◌├─┤◌┌─╢ 4\\ \tt ╟─┘◌│◌│O║ 5\\ \tt ║◌◌S└┬┴─╢ 6\\ \tt ║◌◌◌T│◌M║ 7\\ \tt ╚════╧══╝\\ \tt ```\\ \tt  
	  } 
	   } 
	   } 
	 & & \\ 
 

    \theutterance \stepcounter{utterance}  

    & & \multicolumn{4}{p{0.6\linewidth}}{\cellcolor[rgb]{0.9,0.9,0.9}{%
	\makecell[{{p{\linewidth}}}]{% 
	  \tt {\tiny [A$\langle$GM]}  
	 'P' erfolgreich nach (3,5) verschoben. Dein Raster sieht jetzt so aus:\\ \tt ```\\ \tt  1234567\\ \tt ╔═══╤═╤═╗\\ \tt ║◌S◌│◌│A║ 1\\ \tt ╟─┬─┼─┴─╢ 2\\ \tt ║◌│◌│◌◌◌║ 3\\ \tt ║◌├─┤D┌─╢ 4\\ \tt ╟─┘P│◌│M║ 5\\ \tt ║◌◌◌└┬┴─╢ 6\\ \tt ║◌OT◌│◌◌║ 7\\ \tt ╚════╧══╝\\ \tt ```\\ \tt Ihr seid gerade bei Runde 5 von maximal 28.\\ \tt Ihr habt gemeinsam 2 von maximal 16 Strafpunkten angesammelt.\\ \tt Dein Mitspieler hat ein Objekt auf seinem Raster bewegt.\\ \tt \\ \tt \\ \tt Wie lautet dein nächster Befehl? 
	  } 
	   } 
	   } 
	 & \\ 
 

    \theutterance \stepcounter{utterance}  

    & \multicolumn{4}{p{0.6\linewidth}}{\cellcolor[rgb]{0.8,1,0.9}{%
	\makecell[{{p{\linewidth}}}]{% 
	  \tt {\tiny [A$\rangle$GM]}  
	 SPRICH: P steht bei mir nun korrekt auf (3,5). Nächster Plan: Ich kann T→(3,7) setzen (Zielplatz). Danach O idealerweise →(3,4), aber dort steht bei mir D, also später Tausch. Ich ziehe jetzt T nach (3,7), sofern für dich ok. 
	  } 
	   } 
	   } 
	 & & \\ 
 

    \theutterance \stepcounter{utterance}  

    & & & \multicolumn{2}{p{0.3\linewidth}}{\cellcolor[rgb]{0.95,0.95,0.95}{%
	\makecell[{{p{\linewidth}}}]{% 
	  \tt {\tiny [GM$|$GM]}  
	 SPRICH: P steht bei mir nun korrekt auf (3,5). Nächster Plan: Ich kann T→(3,7) setzen (Zielplatz). Danach O idealerweise →(3,4), aber dort steht bei mir D, also später Tausch. Ich ziehe jetzt T nach (3,7), sofern für dich ok. 
	  } 
	   } 
	   } 
	 & & \\ 
 

    \theutterance \stepcounter{utterance}  

    & & \multicolumn{4}{p{0.6\linewidth}}{\cellcolor[rgb]{0.9,0.9,0.9}{%
	\makecell[{{p{\linewidth}}}]{% 
	  \tt {\tiny [GM$\rangle$B]}  
	 'S' erfolgreich nach (3,6) verschoben. Dein Raster sieht jetzt so aus:\\ \tt ```\\ \tt  1234567\\ \tt ╔═══╤═╤═╗\\ \tt ║P◌A│◌│◌║ 1\\ \tt ╟─┬─┼─┴─╢ 2\\ \tt ║◌│D│◌◌◌║ 3\\ \tt ║◌├─┤◌┌─╢ 4\\ \tt ╟─┘◌│◌│O║ 5\\ \tt ║◌◌S└┬┴─╢ 6\\ \tt ║◌◌◌T│◌M║ 7\\ \tt ╚════╧══╝\\ \tt ```\\ \tt Ihr seid gerade bei Runde 6 von maximal 28.\\ \tt Ihr habt gemeinsam 2 von maximal 16 Strafpunkten angesammelt.\\ \tt Dein Mitspieler hat dir diese Nachricht geschickt:\\ \tt \\ \tt \\ \tt "P steht bei mir nun korrekt auf (3,5). Nächster Plan: Ich kann T→(3,7) setzen (Zielplatz). Danach O idealerweise →(3,4), aber dort steht bei mir D, also später Tausch. Ich ziehe jetzt T nach (3,7), sofern für dich ok."\\ \tt \\ \tt \\ \tt Wie lautet dein nächster Befehl? 
	  } 
	   } 
	   } 
	 & \\ 
 

    \theutterance \stepcounter{utterance}  

    & & & \multicolumn{4}{p{0.6\linewidth}}{\cellcolor[rgb]{1,0.85,0.72}{%
	\makecell[{{p{\linewidth}}}]{% 
	  \tt {\tiny [GM$\langle$B]}  
	 SPRICH: Klingt gut. Bei mir stehen für Spalte 3 aktuell A(3,1) belegt, (3,3),(3,5) frei, (3,6)=S, (3,7)=frei. Dein Zug T→(3,7) passt. Ich setze als Nächstes M→(3,3), da frei. Bitte bestätige, wenn dein T steht. 
	  } 
	   } 
	   } 
	  \\ 
 

    \theutterance \stepcounter{utterance}  

    & & & \multicolumn{2}{p{0.3\linewidth}}{\cellcolor[rgb]{0.95,0.95,0.95}{%
	\makecell[{{p{\linewidth}}}]{% 
	  \tt {\tiny [GM$|$GM]}  
	 SPRICH: Klingt gut. Bei mir stehen für Spalte 3 aktuell A(3,1) belegt, (3,3),(3,5) frei, (3,6)=S, (3,7)=frei. Dein Zug T→(3,7) passt. Ich setze als Nächstes M→(3,3), da frei. Bitte bestätige, wenn dein T steht. 
	  } 
	   } 
	   } 
	 & & \\ 
 

    \theutterance \stepcounter{utterance}  

    & & \multicolumn{4}{p{0.6\linewidth}}{\cellcolor[rgb]{0.9,0.9,0.9}{%
	\makecell[{{p{\linewidth}}}]{% 
	  \tt {\tiny [A$\langle$GM]}  
	 Deine Nachricht wurde an deinen Mitspieler übermittelt.\\ \tt \\ \tt \\ \tt Ihr seid gerade bei Runde 6 von maximal 28.\\ \tt Ihr habt gemeinsam 2 von maximal 16 Strafpunkten angesammelt.\\ \tt Dein Mitspieler hat dir diese Nachricht geschickt:\\ \tt \\ \tt \\ \tt "Klingt gut. Bei mir stehen für Spalte 3 aktuell A(3,1) belegt, (3,3),(3,5) frei, (3,6)=S, (3,7)=frei. Dein Zug T→(3,7) passt. Ich setze als Nächstes M→(3,3), da frei. Bitte bestätige, wenn dein T steht."\\ \tt \\ \tt \\ \tt Wie lautet dein nächster Befehl? 
	  } 
	   } 
	   } 
	 & \\ 
 

    \theutterance \stepcounter{utterance}  

    & \multicolumn{4}{p{0.6\linewidth}}{\cellcolor[rgb]{0.8,1,0.9}{%
	\makecell[{{p{\linewidth}}}]{% 
	  \tt {\tiny [A$\rangle$GM]}  
	 VERSCHIEBE: T, (3,7) 
	  } 
	   } 
	   } 
	 & & \\ 
 

    \theutterance \stepcounter{utterance}  

    & & & \multicolumn{2}{p{0.3\linewidth}}{\cellcolor[rgb]{0.95,0.95,0.95}{%
	\makecell[{{p{\linewidth}}}]{% 
	  \tt {\tiny [GM$|$GM]}  
	 VERSCHIEBE: T, (3,7) 
	  } 
	   } 
	   } 
	 & & \\ 
 

    \theutterance \stepcounter{utterance}  

    & & & \multicolumn{2}{p{0.3\linewidth}}{\cellcolor[rgb]{0.95,0.95,0.95}{%
	\makecell[{{p{\linewidth}}}]{% 
	  \tt {\tiny [GM$|$GM]}  
	 Strafpunkt: (3,7) ist nicht leer, sondern enthält 'T'.\\ \tt \\ \tt Ihr habt gemeinsam 3 von maximal 16 Strafpunkten angesammelt.\\ \tt Bitte versuche es noch mal! 
	  } 
	   } 
	   } 
	 & & \\ 
 

    \theutterance \stepcounter{utterance}  

    & & \multicolumn{4}{p{0.6\linewidth}}{\cellcolor[rgb]{0.9,0.9,0.9}{%
	\makecell[{{p{\linewidth}}}]{% 
	  \tt {\tiny [A$\langle$GM]}  
	 Strafpunkt: (3,7) ist nicht leer, sondern enthält 'T'.\\ \tt \\ \tt Ihr habt gemeinsam 3 von maximal 16 Strafpunkten angesammelt.\\ \tt Bitte versuche es noch mal! 
	  } 
	   } 
	   } 
	 & \\ 
 

    \theutterance \stepcounter{utterance}  

    & \multicolumn{4}{p{0.6\linewidth}}{\cellcolor[rgb]{0.8,1,0.9}{%
	\makecell[{{p{\linewidth}}}]{% 
	  \tt {\tiny [A$\rangle$GM]}  
	 SPRICH: Stimmt, T ist bereits auf (3,7). Aktueller Stand bei mir in Spalte 3: (3,1)=◌, (3,2)=─(Wand), (3,3)=◌, (3,4)=D, (3,5)=P, (3,6)=◌, (3,7)=T. Nächster sinnvoller Zug: M→(3,3), falls frei – das kollidiert bei mir nicht. Ich lasse dir den Vortritt für M→(3,3). Danach versuche ich O näher an (3,4) zu bringen, sobald D umsetzbar ist. 
	  } 
	   } 
	   } 
	 & & \\ 
 

    \theutterance \stepcounter{utterance}  

    & & & \multicolumn{2}{p{0.3\linewidth}}{\cellcolor[rgb]{0.95,0.95,0.95}{%
	\makecell[{{p{\linewidth}}}]{% 
	  \tt {\tiny [GM$|$GM]}  
	 SPRICH: Stimmt, T ist bereits auf (3,7). Aktueller Stand bei mir in Spalte 3: (3,1)=◌, (3,2)=─(Wand), (3,3)=◌, (3,4)=D, (3,5)=P, (3,6)=◌, (3,7)=T. Nächster sinnvoller Zug: M→(3,3), falls frei – das kollidiert bei mir nicht. Ich lasse dir den Vortritt für M→(3,3). Danach versuche ich O näher an (3,4) zu bringen, sobald D umsetzbar ist. 
	  } 
	   } 
	   } 
	 & & \\ 
 

    \theutterance \stepcounter{utterance}  

    & & \multicolumn{4}{p{0.6\linewidth}}{\cellcolor[rgb]{0.9,0.9,0.9}{%
	\makecell[{{p{\linewidth}}}]{% 
	  \tt {\tiny [GM$\rangle$B]}  
	 Deine Nachricht wurde an deinen Mitspieler übermittelt.\\ \tt \\ \tt \\ \tt Ihr seid gerade bei Runde 7 von maximal 28.\\ \tt Ihr habt gemeinsam 3 von maximal 16 Strafpunkten angesammelt.\\ \tt Dein Mitspieler hat dir diese Nachricht geschickt:\\ \tt \\ \tt \\ \tt "Stimmt, T ist bereits auf (3,7). Aktueller Stand bei mir in Spalte 3: (3,1)=◌, (3,2)=─(Wand), (3,3)=◌, (3,4)=D, (3,5)=P, (3,6)=◌, (3,7)=T. Nächster sinnvoller Zug: M→(3,3), falls frei – das kollidiert bei mir nicht. Ich lasse dir den Vortritt für M→(3,3). Danach versuche ich O näher an (3,4) zu bringen, sobald D umsetzbar ist."\\ \tt \\ \tt \\ \tt Wie lautet dein nächster Befehl? 
	  } 
	   } 
	   } 
	 & \\ 
 

    \theutterance \stepcounter{utterance}  

    & & & \multicolumn{4}{p{0.6\linewidth}}{\cellcolor[rgb]{1,0.85,0.72}{%
	\makecell[{{p{\linewidth}}}]{% 
	  \tt {\tiny [GM$\langle$B]}  
	 VERSCHIEBE: M, (3,3) 
	  } 
	   } 
	   } 
	  \\ 
 

    \theutterance \stepcounter{utterance}  

    & & & \multicolumn{2}{p{0.3\linewidth}}{\cellcolor[rgb]{0.95,0.95,0.95}{%
	\makecell[{{p{\linewidth}}}]{% 
	  \tt {\tiny [GM$|$GM]}  
	 VERSCHIEBE: M, (3,3) 
	  } 
	   } 
	   } 
	 & & \\ 
 

    \theutterance \stepcounter{utterance}  

    & & & \multicolumn{2}{p{0.3\linewidth}}{\cellcolor[rgb]{0.95,0.95,0.95}{%
	\makecell[{{p{\linewidth}}}]{% 
	  \tt {\tiny [GM$|$GM]}  
	 Strafpunkt: (3,3) ist nicht leer, sondern enthält 'D'.\\ \tt \\ \tt Ihr habt gemeinsam 4 von maximal 16 Strafpunkten angesammelt.\\ \tt Bitte versuche es noch mal! 
	  } 
	   } 
	   } 
	 & & \\ 
 

    \theutterance \stepcounter{utterance}  

    & & \multicolumn{4}{p{0.6\linewidth}}{\cellcolor[rgb]{0.9,0.9,0.9}{%
	\makecell[{{p{\linewidth}}}]{% 
	  \tt {\tiny [GM$\rangle$B]}  
	 Strafpunkt: (3,3) ist nicht leer, sondern enthält 'D'.\\ \tt \\ \tt Ihr habt gemeinsam 4 von maximal 16 Strafpunkten angesammelt.\\ \tt Bitte versuche es noch mal! 
	  } 
	   } 
	   } 
	 & \\ 
 

    \theutterance \stepcounter{utterance}  

    & & & \multicolumn{4}{p{0.6\linewidth}}{\cellcolor[rgb]{1,0.85,0.72}{%
	\makecell[{{p{\linewidth}}}]{% 
	  \tt {\tiny [GM$\langle$B]}  
	 SPRICH: Danke, bei mir war (3,3) belegt mit D – mein Fehler. Aktualisiere Zielplan: Ich kann P→(3,5) setzen (frei) und O steht bei mir auf (7,5). Vorschlag: Ich ziehe jetzt P→(3,5). Bestätige, ob das für dich passt; du kannst parallel D/O bei dir vorbereiten. 
	  } 
	   } 
	   } 
	  \\ 
 

    \theutterance \stepcounter{utterance}  

    & & & \multicolumn{2}{p{0.3\linewidth}}{\cellcolor[rgb]{0.95,0.95,0.95}{%
	\makecell[{{p{\linewidth}}}]{% 
	  \tt {\tiny [GM$|$GM]}  
	 SPRICH: Danke, bei mir war (3,3) belegt mit D – mein Fehler. Aktualisiere Zielplan: Ich kann P→(3,5) setzen (frei) und O steht bei mir auf (7,5). Vorschlag: Ich ziehe jetzt P→(3,5). Bestätige, ob das für dich passt; du kannst parallel D/O bei dir vorbereiten. 
	  } 
	   } 
	   } 
	 & & \\ 
 

    \theutterance \stepcounter{utterance}  

    & & \multicolumn{4}{p{0.6\linewidth}}{\cellcolor[rgb]{0.9,0.9,0.9}{%
	\makecell[{{p{\linewidth}}}]{% 
	  \tt {\tiny [A$\langle$GM]}  
	 Deine Nachricht wurde an deinen Mitspieler übermittelt.\\ \tt \\ \tt \\ \tt Ihr seid gerade bei Runde 7 von maximal 28.\\ \tt Ihr habt gemeinsam 4 von maximal 16 Strafpunkten angesammelt.\\ \tt Dein Mitspieler hat dir diese Nachricht geschickt:\\ \tt \\ \tt \\ \tt "Danke, bei mir war (3,3) belegt mit D – mein Fehler. Aktualisiere Zielplan: Ich kann P→(3,5) setzen (frei) und O steht bei mir auf (7,5). Vorschlag: Ich ziehe jetzt P→(3,5). Bestätige, ob das für dich passt; du kannst parallel D/O bei dir vorbereiten."\\ \tt \\ \tt \\ \tt Wie lautet dein nächster Befehl? 
	  } 
	   } 
	   } 
	 & \\ 
 

    \theutterance \stepcounter{utterance}  

    & \multicolumn{4}{p{0.6\linewidth}}{\cellcolor[rgb]{0.8,1,0.9}{%
	\makecell[{{p{\linewidth}}}]{% 
	  \tt {\tiny [A$\rangle$GM]}  
	 SPRICH: Passt. Bei mir steht P bereits auf (3,5), T auf (3,7). Ich bereite D/O vor: Ich versuche, D von (4,4) wegzubewegen, um später O→(3,4) zu ermöglichen. Geplanter Zwischenschritt: D→(5,3), da dort frei ist; danach O näher in Spalte 3 ziehen. Sag Bescheid, falls du Einwände hast, sonst führe ich D→(5,3) im nächsten Zug aus. 
	  } 
	   } 
	   } 
	 & & \\ 
 

    \theutterance \stepcounter{utterance}  

    & & & \multicolumn{2}{p{0.3\linewidth}}{\cellcolor[rgb]{0.95,0.95,0.95}{%
	\makecell[{{p{\linewidth}}}]{% 
	  \tt {\tiny [GM$|$GM]}  
	 SPRICH: Passt. Bei mir steht P bereits auf (3,5), T auf (3,7). Ich bereite D/O vor: Ich versuche, D von (4,4) wegzubewegen, um später O→(3,4) zu ermöglichen. Geplanter Zwischenschritt: D→(5,3), da dort frei ist; danach O näher in Spalte 3 ziehen. Sag Bescheid, falls du Einwände hast, sonst führe ich D→(5,3) im nächsten Zug aus. 
	  } 
	   } 
	   } 
	 & & \\ 
 

    \theutterance \stepcounter{utterance}  

    & & \multicolumn{4}{p{0.6\linewidth}}{\cellcolor[rgb]{0.9,0.9,0.9}{%
	\makecell[{{p{\linewidth}}}]{% 
	  \tt {\tiny [GM$\rangle$B]}  
	 Deine Nachricht wurde an deinen Mitspieler übermittelt.\\ \tt \\ \tt \\ \tt Ihr seid gerade bei Runde 8 von maximal 28.\\ \tt Ihr habt gemeinsam 4 von maximal 16 Strafpunkten angesammelt.\\ \tt Dein Mitspieler hat dir diese Nachricht geschickt:\\ \tt \\ \tt \\ \tt "Passt. Bei mir steht P bereits auf (3,5), T auf (3,7). Ich bereite D/O vor: Ich versuche, D von (4,4) wegzubewegen, um später O→(3,4) zu ermöglichen. Geplanter Zwischenschritt: D→(5,3), da dort frei ist; danach O näher in Spalte 3 ziehen. Sag Bescheid, falls du Einwände hast, sonst führe ich D→(5,3) im nächsten Zug aus."\\ \tt \\ \tt \\ \tt Wie lautet dein nächster Befehl? 
	  } 
	   } 
	   } 
	 & \\ 
 

    \theutterance \stepcounter{utterance}  

    & & & \multicolumn{4}{p{0.6\linewidth}}{\cellcolor[rgb]{1,0.85,0.72}{%
	\makecell[{{p{\linewidth}}}]{% 
	  \tt {\tiny [GM$\langle$B]}  
	 SPRICH: Einverstanden. Bei mir setze ich jetzt P→(3,5) (Zielplatz ist frei). Führ du D→(5,3) durch, damit O später nach (3,4) kann. 
	  } 
	   } 
	   } 
	  \\ 
 

    \theutterance \stepcounter{utterance}  

    & & & \multicolumn{2}{p{0.3\linewidth}}{\cellcolor[rgb]{0.95,0.95,0.95}{%
	\makecell[{{p{\linewidth}}}]{% 
	  \tt {\tiny [GM$|$GM]}  
	 SPRICH: Einverstanden. Bei mir setze ich jetzt P→(3,5) (Zielplatz ist frei). Führ du D→(5,3) durch, damit O später nach (3,4) kann. 
	  } 
	   } 
	   } 
	 & & \\ 
 

    \theutterance \stepcounter{utterance}  

    & & \multicolumn{4}{p{0.6\linewidth}}{\cellcolor[rgb]{0.9,0.9,0.9}{%
	\makecell[{{p{\linewidth}}}]{% 
	  \tt {\tiny [A$\langle$GM]}  
	 Deine Nachricht wurde an deinen Mitspieler übermittelt.\\ \tt \\ \tt \\ \tt Ihr seid gerade bei Runde 8 von maximal 28.\\ \tt Ihr habt gemeinsam 4 von maximal 16 Strafpunkten angesammelt.\\ \tt Dein Mitspieler hat dir diese Nachricht geschickt:\\ \tt \\ \tt \\ \tt "Einverstanden. Bei mir setze ich jetzt P→(3,5) (Zielplatz ist frei). Führ du D→(5,3) durch, damit O später nach (3,4) kann."\\ \tt \\ \tt \\ \tt Wie lautet dein nächster Befehl? 
	  } 
	   } 
	   } 
	 & \\ 
 

    \theutterance \stepcounter{utterance}  

    & \multicolumn{4}{p{0.6\linewidth}}{\cellcolor[rgb]{0.8,1,0.9}{%
	\makecell[{{p{\linewidth}}}]{% 
	  \tt {\tiny [A$\rangle$GM]}  
	 VERSCHIEBE: D, (5,3) 
	  } 
	   } 
	   } 
	 & & \\ 
 

    \theutterance \stepcounter{utterance}  

    & & & \multicolumn{2}{p{0.3\linewidth}}{\cellcolor[rgb]{0.95,0.95,0.95}{%
	\makecell[{{p{\linewidth}}}]{% 
	  \tt {\tiny [GM$|$GM]}  
	 VERSCHIEBE: D, (5,3) 
	  } 
	   } 
	   } 
	 & & \\ 
 

    \theutterance \stepcounter{utterance}  

    & & & \multicolumn{2}{p{0.3\linewidth}}{\cellcolor[rgb]{0.95,0.95,0.95}{%
	\makecell[{{p{\linewidth}}}]{% 
	  \tt {\tiny [GM$|$GM]}  
	 'D' erfolgreich nach (5,3) verschoben. Dein Raster sieht jetzt so aus:\\ \tt ```\\ \tt  1234567\\ \tt ╔═══╤═╤═╗\\ \tt ║◌S◌│◌│A║ 1\\ \tt ╟─┬─┼─┴─╢ 2\\ \tt ║◌│◌│D◌◌║ 3\\ \tt ║◌├─┤◌┌─╢ 4\\ \tt ╟─┘P│◌│M║ 5\\ \tt ║◌◌◌└┬┴─╢ 6\\ \tt ║◌OT◌│◌◌║ 7\\ \tt ╚════╧══╝\\ \tt ```\\ \tt  
	  } 
	   } 
	   } 
	 & & \\ 
 

    \theutterance \stepcounter{utterance}  

    & & \multicolumn{4}{p{0.6\linewidth}}{\cellcolor[rgb]{0.9,0.9,0.9}{%
	\makecell[{{p{\linewidth}}}]{% 
	  \tt {\tiny [GM$\rangle$B]}  
	 Deine Nachricht wurde an deinen Mitspieler übermittelt.\\ \tt \\ \tt \\ \tt Ihr seid gerade bei Runde 9 von maximal 28.\\ \tt Ihr habt gemeinsam 4 von maximal 16 Strafpunkten angesammelt.\\ \tt Dein Mitspieler hat ein Objekt auf seinem Raster bewegt.\\ \tt \\ \tt \\ \tt Wie lautet dein nächster Befehl? 
	  } 
	   } 
	   } 
	 & \\ 
 

    \theutterance \stepcounter{utterance}  

    & & & \multicolumn{4}{p{0.6\linewidth}}{\cellcolor[rgb]{1,0.85,0.72}{%
	\makecell[{{p{\linewidth}}}]{% 
	  \tt {\tiny [GM$\langle$B]}  
	 VERSCHIEBE: P, (3,5) 
	  } 
	   } 
	   } 
	  \\ 
 

    \theutterance \stepcounter{utterance}  

    & & & \multicolumn{2}{p{0.3\linewidth}}{\cellcolor[rgb]{0.95,0.95,0.95}{%
	\makecell[{{p{\linewidth}}}]{% 
	  \tt {\tiny [GM$|$GM]}  
	 VERSCHIEBE: P, (3,5) 
	  } 
	   } 
	   } 
	 & & \\ 
 

    \theutterance \stepcounter{utterance}  

    & & & \multicolumn{2}{p{0.3\linewidth}}{\cellcolor[rgb]{0.95,0.95,0.95}{%
	\makecell[{{p{\linewidth}}}]{% 
	  \tt {\tiny [GM$|$GM]}  
	 'P' erfolgreich nach (3,5) verschoben. Dein Raster sieht jetzt so aus:\\ \tt ```\\ \tt  1234567\\ \tt ╔═══╤═╤═╗\\ \tt ║◌◌A│◌│◌║ 1\\ \tt ╟─┬─┼─┴─╢ 2\\ \tt ║◌│D│◌◌◌║ 3\\ \tt ║◌├─┤◌┌─╢ 4\\ \tt ╟─┘P│◌│O║ 5\\ \tt ║◌◌S└┬┴─╢ 6\\ \tt ║◌◌◌T│◌M║ 7\\ \tt ╚════╧══╝\\ \tt ```\\ \tt  
	  } 
	   } 
	   } 
	 & & \\ 
 

    \theutterance \stepcounter{utterance}  

    & & \multicolumn{4}{p{0.6\linewidth}}{\cellcolor[rgb]{0.9,0.9,0.9}{%
	\makecell[{{p{\linewidth}}}]{% 
	  \tt {\tiny [A$\langle$GM]}  
	 'D' erfolgreich nach (5,3) verschoben. Dein Raster sieht jetzt so aus:\\ \tt ```\\ \tt  1234567\\ \tt ╔═══╤═╤═╗\\ \tt ║◌S◌│◌│A║ 1\\ \tt ╟─┬─┼─┴─╢ 2\\ \tt ║◌│◌│D◌◌║ 3\\ \tt ║◌├─┤◌┌─╢ 4\\ \tt ╟─┘P│◌│M║ 5\\ \tt ║◌◌◌└┬┴─╢ 6\\ \tt ║◌OT◌│◌◌║ 7\\ \tt ╚════╧══╝\\ \tt ```\\ \tt Ihr seid gerade bei Runde 9 von maximal 28.\\ \tt Ihr habt gemeinsam 4 von maximal 16 Strafpunkten angesammelt.\\ \tt Dein Mitspieler hat ein Objekt auf seinem Raster bewegt.\\ \tt \\ \tt \\ \tt Wie lautet dein nächster Befehl? 
	  } 
	   } 
	   } 
	 & \\ 
 

    \theutterance \stepcounter{utterance}  

    & \multicolumn{4}{p{0.6\linewidth}}{\cellcolor[rgb]{0.8,1,0.9}{%
	\makecell[{{p{\linewidth}}}]{% 
	  \tt {\tiny [A$\rangle$GM]}  
	 SPRICH: D steht nun auf (5,3). Ich versuche jetzt O näher an (3,4) zu bringen. Geplanter Schritt: O→(3,1) ist frei und in Spalte 3; danach später O→(3,4), wenn der Weg möglich ist. Führe O→(3,1) aus, sofern du keine Einwände hast. 
	  } 
	   } 
	   } 
	 & & \\ 
 

    \theutterance \stepcounter{utterance}  

    & & & \multicolumn{2}{p{0.3\linewidth}}{\cellcolor[rgb]{0.95,0.95,0.95}{%
	\makecell[{{p{\linewidth}}}]{% 
	  \tt {\tiny [GM$|$GM]}  
	 SPRICH: D steht nun auf (5,3). Ich versuche jetzt O näher an (3,4) zu bringen. Geplanter Schritt: O→(3,1) ist frei und in Spalte 3; danach später O→(3,4), wenn der Weg möglich ist. Führe O→(3,1) aus, sofern du keine Einwände hast. 
	  } 
	   } 
	   } 
	 & & \\ 
 

    \theutterance \stepcounter{utterance}  

    & & \multicolumn{4}{p{0.6\linewidth}}{\cellcolor[rgb]{0.9,0.9,0.9}{%
	\makecell[{{p{\linewidth}}}]{% 
	  \tt {\tiny [GM$\rangle$B]}  
	 'P' erfolgreich nach (3,5) verschoben. Dein Raster sieht jetzt so aus:\\ \tt ```\\ \tt  1234567\\ \tt ╔═══╤═╤═╗\\ \tt ║◌◌A│◌│◌║ 1\\ \tt ╟─┬─┼─┴─╢ 2\\ \tt ║◌│D│◌◌◌║ 3\\ \tt ║◌├─┤◌┌─╢ 4\\ \tt ╟─┘P│◌│O║ 5\\ \tt ║◌◌S└┬┴─╢ 6\\ \tt ║◌◌◌T│◌M║ 7\\ \tt ╚════╧══╝\\ \tt ```\\ \tt Ihr seid gerade bei Runde 10 von maximal 28.\\ \tt Ihr habt gemeinsam 4 von maximal 16 Strafpunkten angesammelt.\\ \tt Dein Mitspieler hat dir diese Nachricht geschickt:\\ \tt \\ \tt \\ \tt "D steht nun auf (5,3). Ich versuche jetzt O näher an (3,4) zu bringen. Geplanter Schritt: O→(3,1) ist frei und in Spalte 3; danach später O→(3,4), wenn der Weg möglich ist. Führe O→(3,1) aus, sofern du keine Einwände hast."\\ \tt \\ \tt \\ \tt Wie lautet dein nächster Befehl? 
	  } 
	   } 
	   } 
	 & \\ 
 

    \theutterance \stepcounter{utterance}  

    & & & \multicolumn{4}{p{0.6\linewidth}}{\cellcolor[rgb]{1,0.85,0.72}{%
	\makecell[{{p{\linewidth}}}]{% 
	  \tt {\tiny [GM$\langle$B]}  
	 SPRICH: Einverstanden. Bei mir Status Spalte 3: A(3,1)=A belegt, (3,2)=Wand, (3,3)=D, (3,4)=Wand, (3,5)=P, (3,6)=S, (3,7)=T. O in Spalte 3 ist bei mir aktuell nicht möglich. Führe du O→(3,1) aus. Ich bereite M→(3,3) vor, sobald D bei mir versetzt ist. 
	  } 
	   } 
	   } 
	  \\ 
 

    \theutterance \stepcounter{utterance}  

    & & & \multicolumn{2}{p{0.3\linewidth}}{\cellcolor[rgb]{0.95,0.95,0.95}{%
	\makecell[{{p{\linewidth}}}]{% 
	  \tt {\tiny [GM$|$GM]}  
	 SPRICH: Einverstanden. Bei mir Status Spalte 3: A(3,1)=A belegt, (3,2)=Wand, (3,3)=D, (3,4)=Wand, (3,5)=P, (3,6)=S, (3,7)=T. O in Spalte 3 ist bei mir aktuell nicht möglich. Führe du O→(3,1) aus. Ich bereite M→(3,3) vor, sobald D bei mir versetzt ist. 
	  } 
	   } 
	   } 
	 & & \\ 
 

    \theutterance \stepcounter{utterance}  

    & & \multicolumn{4}{p{0.6\linewidth}}{\cellcolor[rgb]{0.9,0.9,0.9}{%
	\makecell[{{p{\linewidth}}}]{% 
	  \tt {\tiny [A$\langle$GM]}  
	 Deine Nachricht wurde an deinen Mitspieler übermittelt.\\ \tt \\ \tt \\ \tt Ihr seid gerade bei Runde 10 von maximal 28.\\ \tt Ihr habt gemeinsam 4 von maximal 16 Strafpunkten angesammelt.\\ \tt Dein Mitspieler hat dir diese Nachricht geschickt:\\ \tt \\ \tt \\ \tt "Einverstanden. Bei mir Status Spalte 3: A(3,1)=A belegt, (3,2)=Wand, (3,3)=D, (3,4)=Wand, (3,5)=P, (3,6)=S, (3,7)=T. O in Spalte 3 ist bei mir aktuell nicht möglich. Führe du O→(3,1) aus. Ich bereite M→(3,3) vor, sobald D bei mir versetzt ist."\\ \tt \\ \tt \\ \tt Wie lautet dein nächster Befehl? 
	  } 
	   } 
	   } 
	 & \\ 
 

    \theutterance \stepcounter{utterance}  

    & \multicolumn{4}{p{0.6\linewidth}}{\cellcolor[rgb]{0.8,1,0.9}{%
	\makecell[{{p{\linewidth}}}]{% 
	  \tt {\tiny [A$\rangle$GM]}  
	 VERSCHIEBE: O, (3,1) 
	  } 
	   } 
	   } 
	 & & \\ 
 

    \theutterance \stepcounter{utterance}  

    & & & \multicolumn{2}{p{0.3\linewidth}}{\cellcolor[rgb]{0.95,0.95,0.95}{%
	\makecell[{{p{\linewidth}}}]{% 
	  \tt {\tiny [GM$|$GM]}  
	 VERSCHIEBE: O, (3,1) 
	  } 
	   } 
	   } 
	 & & \\ 
 

    \theutterance \stepcounter{utterance}  

    & & & \multicolumn{2}{p{0.3\linewidth}}{\cellcolor[rgb]{0.95,0.95,0.95}{%
	\makecell[{{p{\linewidth}}}]{% 
	  \tt {\tiny [GM$|$GM]}  
	 'O' erfolgreich nach (3,1) verschoben. Dein Raster sieht jetzt so aus:\\ \tt ```\\ \tt  1234567\\ \tt ╔═══╤═╤═╗\\ \tt ║◌SO│◌│A║ 1\\ \tt ╟─┬─┼─┴─╢ 2\\ \tt ║◌│◌│D◌◌║ 3\\ \tt ║◌├─┤◌┌─╢ 4\\ \tt ╟─┘P│◌│M║ 5\\ \tt ║◌◌◌└┬┴─╢ 6\\ \tt ║◌◌T◌│◌◌║ 7\\ \tt ╚════╧══╝\\ \tt ```\\ \tt  
	  } 
	   } 
	   } 
	 & & \\ 
 

    \theutterance \stepcounter{utterance}  

    & & \multicolumn{4}{p{0.6\linewidth}}{\cellcolor[rgb]{0.9,0.9,0.9}{%
	\makecell[{{p{\linewidth}}}]{% 
	  \tt {\tiny [GM$\rangle$B]}  
	 Deine Nachricht wurde an deinen Mitspieler übermittelt.\\ \tt \\ \tt \\ \tt Ihr seid gerade bei Runde 11 von maximal 28.\\ \tt Ihr habt gemeinsam 4 von maximal 16 Strafpunkten angesammelt.\\ \tt Dein Mitspieler hat ein Objekt auf seinem Raster bewegt.\\ \tt \\ \tt \\ \tt Wie lautet dein nächster Befehl? 
	  } 
	   } 
	   } 
	 & \\ 
 

    \theutterance \stepcounter{utterance}  

    & & & \multicolumn{4}{p{0.6\linewidth}}{\cellcolor[rgb]{1,0.85,0.72}{%
	\makecell[{{p{\linewidth}}}]{% 
	  \tt {\tiny [GM$\langle$B]}  
	 SPRICH: O in Spalte 3 bei dir notiert. Bei mir als nächstes sinnvoll: A steht bereits korrekt auf (3,1), P(3,5), S(3,6), T(3,7) korrekt; D blockiert (3,3). Ich kann derzeit nur O von (7,5) wegziehen. Vorschlag Zwischenziel: O→(5,5) falls frei, um später Richtung (3,4) zu manövrieren. Ich versuche O→(5,5) im nächsten Zug, sofern du keine Einwände hast. 
	  } 
	   } 
	   } 
	  \\ 
 

    \theutterance \stepcounter{utterance}  

    & & & \multicolumn{2}{p{0.3\linewidth}}{\cellcolor[rgb]{0.95,0.95,0.95}{%
	\makecell[{{p{\linewidth}}}]{% 
	  \tt {\tiny [GM$|$GM]}  
	 SPRICH: O in Spalte 3 bei dir notiert. Bei mir als nächstes sinnvoll: A steht bereits korrekt auf (3,1), P(3,5), S(3,6), T(3,7) korrekt; D blockiert (3,3). Ich kann derzeit nur O von (7,5) wegziehen. Vorschlag Zwischenziel: O→(5,5) falls frei, um später Richtung (3,4) zu manövrieren. Ich versuche O→(5,5) im nächsten Zug, sofern du keine Einwände hast. 
	  } 
	   } 
	   } 
	 & & \\ 
 

    \theutterance \stepcounter{utterance}  

    & & \multicolumn{4}{p{0.6\linewidth}}{\cellcolor[rgb]{0.9,0.9,0.9}{%
	\makecell[{{p{\linewidth}}}]{% 
	  \tt {\tiny [A$\langle$GM]}  
	 'O' erfolgreich nach (3,1) verschoben. Dein Raster sieht jetzt so aus:\\ \tt ```\\ \tt  1234567\\ \tt ╔═══╤═╤═╗\\ \tt ║◌SO│◌│A║ 1\\ \tt ╟─┬─┼─┴─╢ 2\\ \tt ║◌│◌│D◌◌║ 3\\ \tt ║◌├─┤◌┌─╢ 4\\ \tt ╟─┘P│◌│M║ 5\\ \tt ║◌◌◌└┬┴─╢ 6\\ \tt ║◌◌T◌│◌◌║ 7\\ \tt ╚════╧══╝\\ \tt ```\\ \tt Ihr seid gerade bei Runde 11 von maximal 28.\\ \tt Ihr habt gemeinsam 4 von maximal 16 Strafpunkten angesammelt.\\ \tt Dein Mitspieler hat dir diese Nachricht geschickt:\\ \tt \\ \tt \\ \tt "O in Spalte 3 bei dir notiert. Bei mir als nächstes sinnvoll: A steht bereits korrekt auf (3,1), P(3,5), S(3,6), T(3,7) korrekt; D blockiert (3,3). Ich kann derzeit nur O von (7,5) wegziehen. Vorschlag Zwischenziel: O→(5,5) falls frei, um später Richtung (3,4) zu manövrieren. Ich versuche O→(5,5) im nächsten Zug, sofern du keine Einwände hast."\\ \tt \\ \tt \\ \tt Wie lautet dein nächster Befehl? 
	  } 
	   } 
	   } 
	 & \\ 
 

    \theutterance \stepcounter{utterance}  

    & \multicolumn{4}{p{0.6\linewidth}}{\cellcolor[rgb]{0.8,1,0.9}{%
	\makecell[{{p{\linewidth}}}]{% 
	  \tt {\tiny [A$\rangle$GM]}  
	 SPRICH: Einverstanden. Bei mir als nächstes setze ich M→(3,3), da frei und Zielplatz. Danach kann ich O von (3,1) später auf (3,4) schieben, sobald der Weg möglich ist. Führ du O→(5,5) aus; ich mache jetzt M→(3,3). 
	  } 
	   } 
	   } 
	 & & \\ 
 

    \theutterance \stepcounter{utterance}  

    & & & \multicolumn{2}{p{0.3\linewidth}}{\cellcolor[rgb]{0.95,0.95,0.95}{%
	\makecell[{{p{\linewidth}}}]{% 
	  \tt {\tiny [GM$|$GM]}  
	 SPRICH: Einverstanden. Bei mir als nächstes setze ich M→(3,3), da frei und Zielplatz. Danach kann ich O von (3,1) später auf (3,4) schieben, sobald der Weg möglich ist. Führ du O→(5,5) aus; ich mache jetzt M→(3,3). 
	  } 
	   } 
	   } 
	 & & \\ 
 

    \theutterance \stepcounter{utterance}  

    & & \multicolumn{4}{p{0.6\linewidth}}{\cellcolor[rgb]{0.9,0.9,0.9}{%
	\makecell[{{p{\linewidth}}}]{% 
	  \tt {\tiny [GM$\rangle$B]}  
	 Deine Nachricht wurde an deinen Mitspieler übermittelt.\\ \tt \\ \tt \\ \tt Ihr seid gerade bei Runde 12 von maximal 28.\\ \tt Ihr habt gemeinsam 4 von maximal 16 Strafpunkten angesammelt.\\ \tt Dein Mitspieler hat dir diese Nachricht geschickt:\\ \tt \\ \tt \\ \tt "Einverstanden. Bei mir als nächstes setze ich M→(3,3), da frei und Zielplatz. Danach kann ich O von (3,1) später auf (3,4) schieben, sobald der Weg möglich ist. Führ du O→(5,5) aus; ich mache jetzt M→(3,3)."\\ \tt \\ \tt \\ \tt Wie lautet dein nächster Befehl? 
	  } 
	   } 
	   } 
	 & \\ 
 

    \theutterance \stepcounter{utterance}  

    & & & \multicolumn{4}{p{0.6\linewidth}}{\cellcolor[rgb]{1,0.85,0.72}{%
	\makecell[{{p{\linewidth}}}]{% 
	  \tt {\tiny [GM$\langle$B]}  
	 VERSCHIEBE: O, (5,5) 
	  } 
	   } 
	   } 
	  \\ 
 

    \theutterance \stepcounter{utterance}  

    & & & \multicolumn{2}{p{0.3\linewidth}}{\cellcolor[rgb]{0.95,0.95,0.95}{%
	\makecell[{{p{\linewidth}}}]{% 
	  \tt {\tiny [GM$|$GM]}  
	 VERSCHIEBE: O, (5,5) 
	  } 
	   } 
	   } 
	 & & \\ 
 

    \theutterance \stepcounter{utterance}  

    & & & \multicolumn{2}{p{0.3\linewidth}}{\cellcolor[rgb]{0.95,0.95,0.95}{%
	\makecell[{{p{\linewidth}}}]{% 
	  \tt {\tiny [GM$|$GM]}  
	 'O' erfolgreich nach (5,5) verschoben. Dein Raster sieht jetzt so aus:\\ \tt ```\\ \tt  1234567\\ \tt ╔═══╤═╤═╗\\ \tt ║◌◌A│◌│◌║ 1\\ \tt ╟─┬─┼─┴─╢ 2\\ \tt ║◌│D│◌◌◌║ 3\\ \tt ║◌├─┤◌┌─╢ 4\\ \tt ╟─┘P│O│◌║ 5\\ \tt ║◌◌S└┬┴─╢ 6\\ \tt ║◌◌◌T│◌M║ 7\\ \tt ╚════╧══╝\\ \tt ```\\ \tt  
	  } 
	   } 
	   } 
	 & & \\ 
 

    \theutterance \stepcounter{utterance}  

    & & \multicolumn{4}{p{0.6\linewidth}}{\cellcolor[rgb]{0.9,0.9,0.9}{%
	\makecell[{{p{\linewidth}}}]{% 
	  \tt {\tiny [A$\langle$GM]}  
	 Deine Nachricht wurde an deinen Mitspieler übermittelt.\\ \tt \\ \tt \\ \tt Ihr seid gerade bei Runde 12 von maximal 28.\\ \tt Ihr habt gemeinsam 4 von maximal 16 Strafpunkten angesammelt.\\ \tt Dein Mitspieler hat ein Objekt auf seinem Raster bewegt.\\ \tt \\ \tt \\ \tt Wie lautet dein nächster Befehl? 
	  } 
	   } 
	   } 
	 & \\ 
 

    \theutterance \stepcounter{utterance}  

    & \multicolumn{4}{p{0.6\linewidth}}{\cellcolor[rgb]{0.8,1,0.9}{%
	\makecell[{{p{\linewidth}}}]{% 
	  \tt {\tiny [A$\rangle$GM]}  
	 VERSCHIEBE: M, (3,3) 
	  } 
	   } 
	   } 
	 & & \\ 
 

    \theutterance \stepcounter{utterance}  

    & & & \multicolumn{2}{p{0.3\linewidth}}{\cellcolor[rgb]{0.95,0.95,0.95}{%
	\makecell[{{p{\linewidth}}}]{% 
	  \tt {\tiny [GM$|$GM]}  
	 VERSCHIEBE: M, (3,3) 
	  } 
	   } 
	   } 
	 & & \\ 
 

    \theutterance \stepcounter{utterance}  

    & & & \multicolumn{2}{p{0.3\linewidth}}{\cellcolor[rgb]{0.95,0.95,0.95}{%
	\makecell[{{p{\linewidth}}}]{% 
	  \tt {\tiny [GM$|$GM]}  
	 'M' erfolgreich nach (3,3) verschoben. Dein Raster sieht jetzt so aus:\\ \tt ```\\ \tt  1234567\\ \tt ╔═══╤═╤═╗\\ \tt ║◌SO│◌│A║ 1\\ \tt ╟─┬─┼─┴─╢ 2\\ \tt ║◌│M│D◌◌║ 3\\ \tt ║◌├─┤◌┌─╢ 4\\ \tt ╟─┘P│◌│◌║ 5\\ \tt ║◌◌◌└┬┴─╢ 6\\ \tt ║◌◌T◌│◌◌║ 7\\ \tt ╚════╧══╝\\ \tt ```\\ \tt  
	  } 
	   } 
	   } 
	 & & \\ 
 

    \theutterance \stepcounter{utterance}  

    & & \multicolumn{4}{p{0.6\linewidth}}{\cellcolor[rgb]{0.9,0.9,0.9}{%
	\makecell[{{p{\linewidth}}}]{% 
	  \tt {\tiny [GM$\rangle$B]}  
	 'O' erfolgreich nach (5,5) verschoben. Dein Raster sieht jetzt so aus:\\ \tt ```\\ \tt  1234567\\ \tt ╔═══╤═╤═╗\\ \tt ║◌◌A│◌│◌║ 1\\ \tt ╟─┬─┼─┴─╢ 2\\ \tt ║◌│D│◌◌◌║ 3\\ \tt ║◌├─┤◌┌─╢ 4\\ \tt ╟─┘P│O│◌║ 5\\ \tt ║◌◌S└┬┴─╢ 6\\ \tt ║◌◌◌T│◌M║ 7\\ \tt ╚════╧══╝\\ \tt ```\\ \tt Ihr seid gerade bei Runde 13 von maximal 28.\\ \tt Ihr habt gemeinsam 4 von maximal 16 Strafpunkten angesammelt.\\ \tt Dein Mitspieler hat ein Objekt auf seinem Raster bewegt.\\ \tt \\ \tt \\ \tt Wie lautet dein nächster Befehl? 
	  } 
	   } 
	   } 
	 & \\ 
 

    \theutterance \stepcounter{utterance}  

    & & & \multicolumn{4}{p{0.6\linewidth}}{\cellcolor[rgb]{1,0.85,0.72}{%
	\makecell[{{p{\linewidth}}}]{% 
	  \tt {\tiny [GM$\langle$B]}  
	 SPRICH: Update bei mir: A(3,1), P(3,5), S(3,6), T(3,7) korrekt; O jetzt auf (5,5); D steht noch auf (2,3) und blockiert M→(3,3). Ich plane D→(1,3) (frei) als Zwischenzug, danach M→(3,3) wenn möglich. Passt das? Wenn ja, ziehe ich als Nächstes D→(1,3). 
	  } 
	   } 
	   } 
	  \\ 
 

    \theutterance \stepcounter{utterance}  

    & & & \multicolumn{2}{p{0.3\linewidth}}{\cellcolor[rgb]{0.95,0.95,0.95}{%
	\makecell[{{p{\linewidth}}}]{% 
	  \tt {\tiny [GM$|$GM]}  
	 SPRICH: Update bei mir: A(3,1), P(3,5), S(3,6), T(3,7) korrekt; O jetzt auf (5,5); D steht noch auf (2,3) und blockiert M→(3,3). Ich plane D→(1,3) (frei) als Zwischenzug, danach M→(3,3) wenn möglich. Passt das? Wenn ja, ziehe ich als Nächstes D→(1,3). 
	  } 
	   } 
	   } 
	 & & \\ 
 

    \theutterance \stepcounter{utterance}  

    & & \multicolumn{4}{p{0.6\linewidth}}{\cellcolor[rgb]{0.9,0.9,0.9}{%
	\makecell[{{p{\linewidth}}}]{% 
	  \tt {\tiny [A$\langle$GM]}  
	 'M' erfolgreich nach (3,3) verschoben. Dein Raster sieht jetzt so aus:\\ \tt ```\\ \tt  1234567\\ \tt ╔═══╤═╤═╗\\ \tt ║◌SO│◌│A║ 1\\ \tt ╟─┬─┼─┴─╢ 2\\ \tt ║◌│M│D◌◌║ 3\\ \tt ║◌├─┤◌┌─╢ 4\\ \tt ╟─┘P│◌│◌║ 5\\ \tt ║◌◌◌└┬┴─╢ 6\\ \tt ║◌◌T◌│◌◌║ 7\\ \tt ╚════╧══╝\\ \tt ```\\ \tt Ihr seid gerade bei Runde 13 von maximal 28.\\ \tt Ihr habt gemeinsam 4 von maximal 16 Strafpunkten angesammelt.\\ \tt Dein Mitspieler hat dir diese Nachricht geschickt:\\ \tt \\ \tt \\ \tt "Update bei mir: A(3,1), P(3,5), S(3,6), T(3,7) korrekt; O jetzt auf (5,5); D steht noch auf (2,3) und blockiert M→(3,3). Ich plane D→(1,3) (frei) als Zwischenzug, danach M→(3,3) wenn möglich. Passt das? Wenn ja, ziehe ich als Nächstes D→(1,3)."\\ \tt \\ \tt \\ \tt Wie lautet dein nächster Befehl? 
	  } 
	   } 
	   } 
	 & \\ 
 

    \theutterance \stepcounter{utterance}  

    & \multicolumn{4}{p{0.6\linewidth}}{\cellcolor[rgb]{0.8,1,0.9}{%
	\makecell[{{p{\linewidth}}}]{% 
	  \tt {\tiny [A$\rangle$GM]}  
	 SPRICH: Passt. Bei mir stehen nun O(3,1), M(3,3), P(3,5), T(3,7). Nächster Plan: Ich bringe O später auf (3,4), sobald der Weg frei ist. Bestätige deinen Zug D→(1,3); danach kannst du M→(3,3) setzen. Ich warte mit weiteren Zügen, bis du D versetzt hast. 
	  } 
	   } 
	   } 
	 & & \\ 
 

    \theutterance \stepcounter{utterance}  

    & & & \multicolumn{2}{p{0.3\linewidth}}{\cellcolor[rgb]{0.95,0.95,0.95}{%
	\makecell[{{p{\linewidth}}}]{% 
	  \tt {\tiny [GM$|$GM]}  
	 SPRICH: Passt. Bei mir stehen nun O(3,1), M(3,3), P(3,5), T(3,7). Nächster Plan: Ich bringe O später auf (3,4), sobald der Weg frei ist. Bestätige deinen Zug D→(1,3); danach kannst du M→(3,3) setzen. Ich warte mit weiteren Zügen, bis du D versetzt hast. 
	  } 
	   } 
	   } 
	 & & \\ 
 

    \theutterance \stepcounter{utterance}  

    & & \multicolumn{4}{p{0.6\linewidth}}{\cellcolor[rgb]{0.9,0.9,0.9}{%
	\makecell[{{p{\linewidth}}}]{% 
	  \tt {\tiny [GM$\rangle$B]}  
	 Deine Nachricht wurde an deinen Mitspieler übermittelt.\\ \tt \\ \tt \\ \tt Ihr seid gerade bei Runde 14 von maximal 28.\\ \tt Ihr habt gemeinsam 4 von maximal 16 Strafpunkten angesammelt.\\ \tt Dein Mitspieler hat dir diese Nachricht geschickt:\\ \tt \\ \tt \\ \tt "Passt. Bei mir stehen nun O(3,1), M(3,3), P(3,5), T(3,7). Nächster Plan: Ich bringe O später auf (3,4), sobald der Weg frei ist. Bestätige deinen Zug D→(1,3); danach kannst du M→(3,3) setzen. Ich warte mit weiteren Zügen, bis du D versetzt hast."\\ \tt \\ \tt \\ \tt Wie lautet dein nächster Befehl? 
	  } 
	   } 
	   } 
	 & \\ 
 

    \theutterance \stepcounter{utterance}  

    & & & \multicolumn{4}{p{0.6\linewidth}}{\cellcolor[rgb]{1,0.85,0.72}{%
	\makecell[{{p{\linewidth}}}]{% 
	  \tt {\tiny [GM$\langle$B]}  
	 VERSCHIEBE: D, (1,3) 
	  } 
	   } 
	   } 
	  \\ 
 

    \theutterance \stepcounter{utterance}  

    & & & \multicolumn{2}{p{0.3\linewidth}}{\cellcolor[rgb]{0.95,0.95,0.95}{%
	\makecell[{{p{\linewidth}}}]{% 
	  \tt {\tiny [GM$|$GM]}  
	 VERSCHIEBE: D, (1,3) 
	  } 
	   } 
	   } 
	 & & \\ 
 

    \theutterance \stepcounter{utterance}  

    & & & \multicolumn{2}{p{0.3\linewidth}}{\cellcolor[rgb]{0.95,0.95,0.95}{%
	\makecell[{{p{\linewidth}}}]{% 
	  \tt {\tiny [GM$|$GM]}  
	 'D' erfolgreich nach (1,3) verschoben. Dein Raster sieht jetzt so aus:\\ \tt ```\\ \tt  1234567\\ \tt ╔═══╤═╤═╗\\ \tt ║◌◌A│◌│◌║ 1\\ \tt ╟─┬─┼─┴─╢ 2\\ \tt ║D│◌│◌◌◌║ 3\\ \tt ║◌├─┤◌┌─╢ 4\\ \tt ╟─┘P│O│◌║ 5\\ \tt ║◌◌S└┬┴─╢ 6\\ \tt ║◌◌◌T│◌M║ 7\\ \tt ╚════╧══╝\\ \tt ```\\ \tt  
	  } 
	   } 
	   } 
	 & & \\ 
 

    \theutterance \stepcounter{utterance}  

    & & \multicolumn{4}{p{0.6\linewidth}}{\cellcolor[rgb]{0.9,0.9,0.9}{%
	\makecell[{{p{\linewidth}}}]{% 
	  \tt {\tiny [A$\langle$GM]}  
	 Deine Nachricht wurde an deinen Mitspieler übermittelt.\\ \tt \\ \tt \\ \tt Ihr seid gerade bei Runde 14 von maximal 28.\\ \tt Ihr habt gemeinsam 4 von maximal 16 Strafpunkten angesammelt.\\ \tt Dein Mitspieler hat ein Objekt auf seinem Raster bewegt.\\ \tt \\ \tt \\ \tt Wie lautet dein nächster Befehl? 
	  } 
	   } 
	   } 
	 & \\ 
 

    \theutterance \stepcounter{utterance}  

    & \multicolumn{4}{p{0.6\linewidth}}{\cellcolor[rgb]{0.8,1,0.9}{%
	\makecell[{{p{\linewidth}}}]{% 
	  \tt {\tiny [A$\rangle$GM]}  
	 SPRICH: Stand bei mir in Spalte 3: O(3,1), M(3,3), P(3,5), T(3,7); (3,2)=Wand, (3,4)=frei geworden, (3,6)=frei. Optimaler nächster Schritt: O→(3,4). Ich führe jetzt O→(3,4) aus, wenn du keinen Einwand hast. 
	  } 
	   } 
	   } 
	 & & \\ 
 

    \theutterance \stepcounter{utterance}  

    & & & \multicolumn{2}{p{0.3\linewidth}}{\cellcolor[rgb]{0.95,0.95,0.95}{%
	\makecell[{{p{\linewidth}}}]{% 
	  \tt {\tiny [GM$|$GM]}  
	 SPRICH: Stand bei mir in Spalte 3: O(3,1), M(3,3), P(3,5), T(3,7); (3,2)=Wand, (3,4)=frei geworden, (3,6)=frei. Optimaler nächster Schritt: O→(3,4). Ich führe jetzt O→(3,4) aus, wenn du keinen Einwand hast. 
	  } 
	   } 
	   } 
	 & & \\ 
 

    \theutterance \stepcounter{utterance}  

    & & \multicolumn{4}{p{0.6\linewidth}}{\cellcolor[rgb]{0.9,0.9,0.9}{%
	\makecell[{{p{\linewidth}}}]{% 
	  \tt {\tiny [GM$\rangle$B]}  
	 'D' erfolgreich nach (1,3) verschoben. Dein Raster sieht jetzt so aus:\\ \tt ```\\ \tt  1234567\\ \tt ╔═══╤═╤═╗\\ \tt ║◌◌A│◌│◌║ 1\\ \tt ╟─┬─┼─┴─╢ 2\\ \tt ║D│◌│◌◌◌║ 3\\ \tt ║◌├─┤◌┌─╢ 4\\ \tt ╟─┘P│O│◌║ 5\\ \tt ║◌◌S└┬┴─╢ 6\\ \tt ║◌◌◌T│◌M║ 7\\ \tt ╚════╧══╝\\ \tt ```\\ \tt Ihr seid gerade bei Runde 15 von maximal 28.\\ \tt Ihr habt gemeinsam 4 von maximal 16 Strafpunkten angesammelt.\\ \tt Dein Mitspieler hat dir diese Nachricht geschickt:\\ \tt \\ \tt \\ \tt "Stand bei mir in Spalte 3: O(3,1), M(3,3), P(3,5), T(3,7); (3,2)=Wand, (3,4)=frei geworden, (3,6)=frei. Optimaler nächster Schritt: O→(3,4). Ich führe jetzt O→(3,4) aus, wenn du keinen Einwand hast."\\ \tt \\ \tt \\ \tt Wie lautet dein nächster Befehl? 
	  } 
	   } 
	   } 
	 & \\ 
 

    \theutterance \stepcounter{utterance}  

    & & & \multicolumn{4}{p{0.6\linewidth}}{\cellcolor[rgb]{1,0.85,0.72}{%
	\makecell[{{p{\linewidth}}}]{% 
	  \tt {\tiny [GM$\langle$B]}  
	 VERSCHIEBE: M, (3,3) 
	  } 
	   } 
	   } 
	  \\ 
 

    \theutterance \stepcounter{utterance}  

    & & & \multicolumn{2}{p{0.3\linewidth}}{\cellcolor[rgb]{0.95,0.95,0.95}{%
	\makecell[{{p{\linewidth}}}]{% 
	  \tt {\tiny [GM$|$GM]}  
	 VERSCHIEBE: M, (3,3) 
	  } 
	   } 
	   } 
	 & & \\ 
 

    \theutterance \stepcounter{utterance}  

    & & & \multicolumn{2}{p{0.3\linewidth}}{\cellcolor[rgb]{0.95,0.95,0.95}{%
	\makecell[{{p{\linewidth}}}]{% 
	  \tt {\tiny [GM$|$GM]}  
	 'M' erfolgreich nach (3,3) verschoben. Dein Raster sieht jetzt so aus:\\ \tt ```\\ \tt  1234567\\ \tt ╔═══╤═╤═╗\\ \tt ║◌◌A│◌│◌║ 1\\ \tt ╟─┬─┼─┴─╢ 2\\ \tt ║D│M│◌◌◌║ 3\\ \tt ║◌├─┤◌┌─╢ 4\\ \tt ╟─┘P│O│◌║ 5\\ \tt ║◌◌S└┬┴─╢ 6\\ \tt ║◌◌◌T│◌◌║ 7\\ \tt ╚════╧══╝\\ \tt ```\\ \tt  
	  } 
	   } 
	   } 
	 & & \\ 
 

    \theutterance \stepcounter{utterance}  

    & & \multicolumn{4}{p{0.6\linewidth}}{\cellcolor[rgb]{0.9,0.9,0.9}{%
	\makecell[{{p{\linewidth}}}]{% 
	  \tt {\tiny [A$\langle$GM]}  
	 Deine Nachricht wurde an deinen Mitspieler übermittelt.\\ \tt \\ \tt \\ \tt Ihr seid gerade bei Runde 15 von maximal 28.\\ \tt Ihr habt gemeinsam 4 von maximal 16 Strafpunkten angesammelt.\\ \tt Dein Mitspieler hat ein Objekt auf seinem Raster bewegt.\\ \tt \\ \tt \\ \tt Wie lautet dein nächster Befehl? 
	  } 
	   } 
	   } 
	 & \\ 
 

    \theutterance \stepcounter{utterance}  

    & \multicolumn{4}{p{0.6\linewidth}}{\cellcolor[rgb]{0.8,1,0.9}{%
	\makecell[{{p{\linewidth}}}]{% 
	  \tt {\tiny [A$\rangle$GM]}  
	 VERSCHIEBE: O, (3,4) 
	  } 
	   } 
	   } 
	 & & \\ 
 

    \theutterance \stepcounter{utterance}  

    & & & \multicolumn{2}{p{0.3\linewidth}}{\cellcolor[rgb]{0.95,0.95,0.95}{%
	\makecell[{{p{\linewidth}}}]{% 
	  \tt {\tiny [GM$|$GM]}  
	 VERSCHIEBE: O, (3,4) 
	  } 
	   } 
	   } 
	 & & \\ 
 

    \theutterance \stepcounter{utterance}  

    & & & \multicolumn{2}{p{0.3\linewidth}}{\cellcolor[rgb]{0.95,0.95,0.95}{%
	\makecell[{{p{\linewidth}}}]{% 
	  \tt {\tiny [GM$|$GM]}  
	 Strafpunkt: (3,4) ist nicht leer, sondern enthält '─'.\\ \tt \\ \tt Ihr habt gemeinsam 5 von maximal 16 Strafpunkten angesammelt.\\ \tt Bitte versuche es noch mal! 
	  } 
	   } 
	   } 
	 & & \\ 
 

    \theutterance \stepcounter{utterance}  

    & & \multicolumn{4}{p{0.6\linewidth}}{\cellcolor[rgb]{0.9,0.9,0.9}{%
	\makecell[{{p{\linewidth}}}]{% 
	  \tt {\tiny [A$\langle$GM]}  
	 Strafpunkt: (3,4) ist nicht leer, sondern enthält '─'.\\ \tt \\ \tt Ihr habt gemeinsam 5 von maximal 16 Strafpunkten angesammelt.\\ \tt Bitte versuche es noch mal! 
	  } 
	   } 
	   } 
	 & \\ 
 

    \theutterance \stepcounter{utterance}  

    & \multicolumn{4}{p{0.6\linewidth}}{\cellcolor[rgb]{0.8,1,0.9}{%
	\makecell[{{p{\linewidth}}}]{% 
	  \tt {\tiny [A$\rangle$GM]}  
	 SPRICH: Danke, (3,4) ist bei mir eine Wand. Korrigierter Plan Spalte 3 bleibt: A(3,1)=O falsch, D(3,2)=Wand, M(3,3)=richtig, O(3,4)=blockiert, P(3,5)=richtig, S(3,6)=frei, T(3,7)=richtig. Vorschlag: Wir tauschen Ziele für A und O, um Block zu umgehen: Setze O dauerhaft auf (3,1) und A bleibt außerhalb der Spalte. Alternativ könnten wir S auf (3,6) setzen (bei mir frei), damit mehr Übereinstimmung entsteht. Ich würde jetzt S→(3,6) ziehen, wenn du einverstanden bist. 
	  } 
	   } 
	   } 
	 & & \\ 
 

    \theutterance \stepcounter{utterance}  

    & & & \multicolumn{2}{p{0.3\linewidth}}{\cellcolor[rgb]{0.95,0.95,0.95}{%
	\makecell[{{p{\linewidth}}}]{% 
	  \tt {\tiny [GM$|$GM]}  
	 SPRICH: Danke, (3,4) ist bei mir eine Wand. Korrigierter Plan Spalte 3 bleibt: A(3,1)=O falsch, D(3,2)=Wand, M(3,3)=richtig, O(3,4)=blockiert, P(3,5)=richtig, S(3,6)=frei, T(3,7)=richtig. Vorschlag: Wir tauschen Ziele für A und O, um Block zu umgehen: Setze O dauerhaft auf (3,1) und A bleibt außerhalb der Spalte. Alternativ könnten wir S auf (3,6) setzen (bei mir frei), damit mehr Übereinstimmung entsteht. Ich würde jetzt S→(3,6) ziehen, wenn du einverstanden bist. 
	  } 
	   } 
	   } 
	 & & \\ 
 

    \theutterance \stepcounter{utterance}  

    & & \multicolumn{4}{p{0.6\linewidth}}{\cellcolor[rgb]{0.9,0.9,0.9}{%
	\makecell[{{p{\linewidth}}}]{% 
	  \tt {\tiny [GM$\rangle$B]}  
	 'M' erfolgreich nach (3,3) verschoben. Dein Raster sieht jetzt so aus:\\ \tt ```\\ \tt  1234567\\ \tt ╔═══╤═╤═╗\\ \tt ║◌◌A│◌│◌║ 1\\ \tt ╟─┬─┼─┴─╢ 2\\ \tt ║D│M│◌◌◌║ 3\\ \tt ║◌├─┤◌┌─╢ 4\\ \tt ╟─┘P│O│◌║ 5\\ \tt ║◌◌S└┬┴─╢ 6\\ \tt ║◌◌◌T│◌◌║ 7\\ \tt ╚════╧══╝\\ \tt ```\\ \tt Ihr seid gerade bei Runde 16 von maximal 28.\\ \tt Ihr habt gemeinsam 5 von maximal 16 Strafpunkten angesammelt.\\ \tt Dein Mitspieler hat dir diese Nachricht geschickt:\\ \tt \\ \tt \\ \tt "Danke, (3,4) ist bei mir eine Wand. Korrigierter Plan Spalte 3 bleibt: A(3,1)=O falsch, D(3,2)=Wand, M(3,3)=richtig, O(3,4)=blockiert, P(3,5)=richtig, S(3,6)=frei, T(3,7)=richtig. Vorschlag: Wir tauschen Ziele für A und O, um Block zu umgehen: Setze O dauerhaft auf (3,1) und A bleibt außerhalb der Spalte. Alternativ könnten wir S auf (3,6) setzen (bei mir frei), damit mehr Übereinstimmung entsteht. Ich würde jetzt S→(3,6) ziehen, wenn du einverstanden bist."\\ \tt \\ \tt \\ \tt Wie lautet dein nächster Befehl? 
	  } 
	   } 
	   } 
	 & \\ 
 

    \theutterance \stepcounter{utterance}  

    & & & \multicolumn{4}{p{0.6\linewidth}}{\cellcolor[rgb]{1,0.85,0.72}{%
	\makecell[{{p{\linewidth}}}]{% 
	  \tt {\tiny [GM$\langle$B]}  
	 SPRICH: Einverstanden mit S→(3,6). Bei mir steht S bereits auf (3,6). Aktueller Abgleich bei mir in Spalte 3: A(3,1)=A, M(3,3)=M, P(3,5)=P, S(3,6)=S, T(3,7)=T; (3,2)=Wand, (3,4)=Wand. Wenn wir O bei dir auf (3,1) belassen, weicht das von meinem A(3,1) ab. Vorschlag für gemeinsamen Zielplan angepasst an Hindernisse: Fixieren wir M(3,3), P(3,5), S(3,6), T(3,7). Für A/O wählen wir stattdessen Spalte 5: O(5,5) und A außerhalb der Spalte. Bei mir ist O bereits (5,5). Kannst du O auf (5,5) bringen? Danach gleichen wir A-Position ab. 
	  } 
	   } 
	   } 
	  \\ 
 

    \theutterance \stepcounter{utterance}  

    & & & \multicolumn{2}{p{0.3\linewidth}}{\cellcolor[rgb]{0.95,0.95,0.95}{%
	\makecell[{{p{\linewidth}}}]{% 
	  \tt {\tiny [GM$|$GM]}  
	 SPRICH: Einverstanden mit S→(3,6). Bei mir steht S bereits auf (3,6). Aktueller Abgleich bei mir in Spalte 3: A(3,1)=A, M(3,3)=M, P(3,5)=P, S(3,6)=S, T(3,7)=T; (3,2)=Wand, (3,4)=Wand. Wenn wir O bei dir auf (3,1) belassen, weicht das von meinem A(3,1) ab. Vorschlag für gemeinsamen Zielplan angepasst an Hindernisse: Fixieren wir M(3,3), P(3,5), S(3,6), T(3,7). Für A/O wählen wir stattdessen Spalte 5: O(5,5) und A außerhalb der Spalte. Bei mir ist O bereits (5,5). Kannst du O auf (5,5) bringen? Danach gleichen wir A-Position ab. 
	  } 
	   } 
	   } 
	 & & \\ 
 

    \theutterance \stepcounter{utterance}  

    & & \multicolumn{4}{p{0.6\linewidth}}{\cellcolor[rgb]{0.9,0.9,0.9}{%
	\makecell[{{p{\linewidth}}}]{% 
	  \tt {\tiny [A$\langle$GM]}  
	 Deine Nachricht wurde an deinen Mitspieler übermittelt.\\ \tt \\ \tt \\ \tt Ihr seid gerade bei Runde 16 von maximal 28.\\ \tt Ihr habt gemeinsam 5 von maximal 16 Strafpunkten angesammelt.\\ \tt Dein Mitspieler hat dir diese Nachricht geschickt:\\ \tt \\ \tt \\ \tt "Einverstanden mit S→(3,6). Bei mir steht S bereits auf (3,6). Aktueller Abgleich bei mir in Spalte 3: A(3,1)=A, M(3,3)=M, P(3,5)=P, S(3,6)=S, T(3,7)=T; (3,2)=Wand, (3,4)=Wand. Wenn wir O bei dir auf (3,1) belassen, weicht das von meinem A(3,1) ab. Vorschlag für gemeinsamen Zielplan angepasst an Hindernisse: Fixieren wir M(3,3), P(3,5), S(3,6), T(3,7). Für A/O wählen wir stattdessen Spalte 5: O(5,5) und A außerhalb der Spalte. Bei mir ist O bereits (5,5). Kannst du O auf (5,5) bringen? Danach gleichen wir A-Position ab."\\ \tt \\ \tt \\ \tt Wie lautet dein nächster Befehl? 
	  } 
	   } 
	   } 
	 & \\ 
 

    \theutterance \stepcounter{utterance}  

    & \multicolumn{4}{p{0.6\linewidth}}{\cellcolor[rgb]{0.8,1,0.9}{%
	\makecell[{{p{\linewidth}}}]{% 
	  \tt {\tiny [A$\rangle$GM]}  
	 VERSCHIEBE: S, (3,6) 
	  } 
	   } 
	   } 
	 & & \\ 
 

    \theutterance \stepcounter{utterance}  

    & & & \multicolumn{2}{p{0.3\linewidth}}{\cellcolor[rgb]{0.95,0.95,0.95}{%
	\makecell[{{p{\linewidth}}}]{% 
	  \tt {\tiny [GM$|$GM]}  
	 VERSCHIEBE: S, (3,6) 
	  } 
	   } 
	   } 
	 & & \\ 
 

    \theutterance \stepcounter{utterance}  

    & & & \multicolumn{2}{p{0.3\linewidth}}{\cellcolor[rgb]{0.95,0.95,0.95}{%
	\makecell[{{p{\linewidth}}}]{% 
	  \tt {\tiny [GM$|$GM]}  
	 'S' erfolgreich nach (3,6) verschoben. Dein Raster sieht jetzt so aus:\\ \tt ```\\ \tt  1234567\\ \tt ╔═══╤═╤═╗\\ \tt ║◌◌O│◌│A║ 1\\ \tt ╟─┬─┼─┴─╢ 2\\ \tt ║◌│M│D◌◌║ 3\\ \tt ║◌├─┤◌┌─╢ 4\\ \tt ╟─┘P│◌│◌║ 5\\ \tt ║◌◌S└┬┴─╢ 6\\ \tt ║◌◌T◌│◌◌║ 7\\ \tt ╚════╧══╝\\ \tt ```\\ \tt  
	  } 
	   } 
	   } 
	 & & \\ 
 

    \theutterance \stepcounter{utterance}  

    & & \multicolumn{4}{p{0.6\linewidth}}{\cellcolor[rgb]{0.9,0.9,0.9}{%
	\makecell[{{p{\linewidth}}}]{% 
	  \tt {\tiny [GM$\rangle$B]}  
	 Deine Nachricht wurde an deinen Mitspieler übermittelt.\\ \tt \\ \tt \\ \tt Ihr seid gerade bei Runde 17 von maximal 28.\\ \tt Ihr habt gemeinsam 5 von maximal 16 Strafpunkten angesammelt.\\ \tt Dein Mitspieler hat ein Objekt auf seinem Raster bewegt.\\ \tt \\ \tt \\ \tt Wie lautet dein nächster Befehl? 
	  } 
	   } 
	   } 
	 & \\ 
 

    \theutterance \stepcounter{utterance}  

    & & & \multicolumn{4}{p{0.6\linewidth}}{\cellcolor[rgb]{1,0.85,0.72}{%
	\makecell[{{p{\linewidth}}}]{% 
	  \tt {\tiny [GM$\langle$B]}  
	 SPRICH: Notiert, S bei dir nun auf (3,6). Bei mir stehen M(3,3), P(3,5), S(3,6), T(3,7) korrekt; A(3,1), O(5,5), D(1,3). Vorschlag für gemeinsamen Zielplan: Fixieren wir M(3,3), P(3,5), S(3,6), T(3,7). Für die restlichen drei: D auf (1,3), O auf (5,5), A auf (3,1). Das entspricht genau meinem Stand; kannst du D→(1,3), O→(5,5), A→(3,1) herstellen? Wenn ja, sag Bescheid, dann prüfen wir Gleichstand und ich sende “fertig?”. 
	  } 
	   } 
	   } 
	  \\ 
 

    \theutterance \stepcounter{utterance}  

    & & & \multicolumn{2}{p{0.3\linewidth}}{\cellcolor[rgb]{0.95,0.95,0.95}{%
	\makecell[{{p{\linewidth}}}]{% 
	  \tt {\tiny [GM$|$GM]}  
	 SPRICH: Notiert, S bei dir nun auf (3,6). Bei mir stehen M(3,3), P(3,5), S(3,6), T(3,7) korrekt; A(3,1), O(5,5), D(1,3). Vorschlag für gemeinsamen Zielplan: Fixieren wir M(3,3), P(3,5), S(3,6), T(3,7). Für die restlichen drei: D auf (1,3), O auf (5,5), A auf (3,1). Das entspricht genau meinem Stand; kannst du D→(1,3), O→(5,5), A→(3,1) herstellen? Wenn ja, sag Bescheid, dann prüfen wir Gleichstand und ich sende “fertig?”. 
	  } 
	   } 
	   } 
	 & & \\ 
 

    \theutterance \stepcounter{utterance}  

    & & \multicolumn{4}{p{0.6\linewidth}}{\cellcolor[rgb]{0.9,0.9,0.9}{%
	\makecell[{{p{\linewidth}}}]{% 
	  \tt {\tiny [A$\langle$GM]}  
	 'S' erfolgreich nach (3,6) verschoben. Dein Raster sieht jetzt so aus:\\ \tt ```\\ \tt  1234567\\ \tt ╔═══╤═╤═╗\\ \tt ║◌◌O│◌│A║ 1\\ \tt ╟─┬─┼─┴─╢ 2\\ \tt ║◌│M│D◌◌║ 3\\ \tt ║◌├─┤◌┌─╢ 4\\ \tt ╟─┘P│◌│◌║ 5\\ \tt ║◌◌S└┬┴─╢ 6\\ \tt ║◌◌T◌│◌◌║ 7\\ \tt ╚════╧══╝\\ \tt ```\\ \tt Ihr seid gerade bei Runde 17 von maximal 28.\\ \tt Ihr habt gemeinsam 5 von maximal 16 Strafpunkten angesammelt.\\ \tt Dein Mitspieler hat dir diese Nachricht geschickt:\\ \tt \\ \tt \\ \tt "Notiert, S bei dir nun auf (3,6). Bei mir stehen M(3,3), P(3,5), S(3,6), T(3,7) korrekt; A(3,1), O(5,5), D(1,3). Vorschlag für gemeinsamen Zielplan: Fixieren wir M(3,3), P(3,5), S(3,6), T(3,7). Für die restlichen drei: D auf (1,3), O auf (5,5), A auf (3,1). Das entspricht genau meinem Stand; kannst du D→(1,3), O→(5,5), A→(3,1) herstellen? Wenn ja, sag Bescheid, dann prüfen wir Gleichstand und ich sende “fertig?”."\\ \tt \\ \tt \\ \tt Wie lautet dein nächster Befehl? 
	  } 
	   } 
	   } 
	 & \\ 
 

    \theutterance \stepcounter{utterance}  

    & \multicolumn{4}{p{0.6\linewidth}}{\cellcolor[rgb]{0.8,1,0.9}{%
	\makecell[{{p{\linewidth}}}]{% 
	  \tt {\tiny [A$\rangle$GM]}  
	 SPRICH: Einverstanden mit Zielplan: D(1,3), O(5,5), A(3,1), sowie M(3,3), P(3,5), S(3,6), T(3,7). Aktueller Stand bei mir: A(7,1), D(5,3), M(3,3), O(3,1), P(3,5), S(3,6), T(3,7). Ich werde nun schrittweise anpassen: zuerst D→(1,3), dann O→(5,5), A→(3,1) ist bereits durch O belegt, daher tauschen wir zuletzt O/A: Sobald O auf (5,5) ist, ziehe ich A→(3,1). Ich beginne im nächsten Zug mit D→(1,3). Bitte gib “fertig?” sobald du bereit bist, nachdem ich die Züge ausgeführt habe. 
	  } 
	   } 
	   } 
	 & & \\ 
 

    \theutterance \stepcounter{utterance}  

    & & & \multicolumn{2}{p{0.3\linewidth}}{\cellcolor[rgb]{0.95,0.95,0.95}{%
	\makecell[{{p{\linewidth}}}]{% 
	  \tt {\tiny [GM$|$GM]}  
	 SPRICH: Einverstanden mit Zielplan: D(1,3), O(5,5), A(3,1), sowie M(3,3), P(3,5), S(3,6), T(3,7). Aktueller Stand bei mir: A(7,1), D(5,3), M(3,3), O(3,1), P(3,5), S(3,6), T(3,7). Ich werde nun schrittweise anpassen: zuerst D→(1,3), dann O→(5,5), A→(3,1) ist bereits durch O belegt, daher tauschen wir zuletzt O/A: Sobald O auf (5,5) ist, ziehe ich A→(3,1). Ich beginne im nächsten Zug mit D→(1,3). Bitte gib “fertig?” sobald du bereit bist, nachdem ich die Züge ausgeführt habe. 
	  } 
	   } 
	   } 
	 & & \\ 
 

    \theutterance \stepcounter{utterance}  

    & & \multicolumn{4}{p{0.6\linewidth}}{\cellcolor[rgb]{0.9,0.9,0.9}{%
	\makecell[{{p{\linewidth}}}]{% 
	  \tt {\tiny [GM$\rangle$B]}  
	 Deine Nachricht wurde an deinen Mitspieler übermittelt.\\ \tt \\ \tt \\ \tt Ihr seid gerade bei Runde 18 von maximal 28.\\ \tt Ihr habt gemeinsam 5 von maximal 16 Strafpunkten angesammelt.\\ \tt Dein Mitspieler hat dir diese Nachricht geschickt:\\ \tt \\ \tt \\ \tt "Einverstanden mit Zielplan: D(1,3), O(5,5), A(3,1), sowie M(3,3), P(3,5), S(3,6), T(3,7). Aktueller Stand bei mir: A(7,1), D(5,3), M(3,3), O(3,1), P(3,5), S(3,6), T(3,7). Ich werde nun schrittweise anpassen: zuerst D→(1,3), dann O→(5,5), A→(3,1) ist bereits durch O belegt, daher tauschen wir zuletzt O/A: Sobald O auf (5,5) ist, ziehe ich A→(3,1). Ich beginne im nächsten Zug mit D→(1,3). Bitte gib “fertig?” sobald du bereit bist, nachdem ich die Züge ausgeführt habe."\\ \tt \\ \tt \\ \tt Wie lautet dein nächster Befehl? 
	  } 
	   } 
	   } 
	 & \\ 
 

    \theutterance \stepcounter{utterance}  

    & & & \multicolumn{4}{p{0.6\linewidth}}{\cellcolor[rgb]{1,0.85,0.72}{%
	\makecell[{{p{\linewidth}}}]{% 
	  \tt {\tiny [GM$\langle$B]}  
	 SPRICH: Klingt gut. Führe D→(1,3) und danach O→(5,5), dann A→(3,1) aus. Mein Brett entspricht bereits dem Zielplan: A(3,1), D(1,3), M(3,3), O(5,5), P(3,5), S(3,6), T(3,7). Sag Bescheid, sobald du fertig bist – dann sende ich „fertig?“. 
	  } 
	   } 
	   } 
	  \\ 
 

    \theutterance \stepcounter{utterance}  

    & & & \multicolumn{2}{p{0.3\linewidth}}{\cellcolor[rgb]{0.95,0.95,0.95}{%
	\makecell[{{p{\linewidth}}}]{% 
	  \tt {\tiny [GM$|$GM]}  
	 SPRICH: Klingt gut. Führe D→(1,3) und danach O→(5,5), dann A→(3,1) aus. Mein Brett entspricht bereits dem Zielplan: A(3,1), D(1,3), M(3,3), O(5,5), P(3,5), S(3,6), T(3,7). Sag Bescheid, sobald du fertig bist – dann sende ich „fertig?“. 
	  } 
	   } 
	   } 
	 & & \\ 
 

    \theutterance \stepcounter{utterance}  

    & & \multicolumn{4}{p{0.6\linewidth}}{\cellcolor[rgb]{0.9,0.9,0.9}{%
	\makecell[{{p{\linewidth}}}]{% 
	  \tt {\tiny [A$\langle$GM]}  
	 Deine Nachricht wurde an deinen Mitspieler übermittelt.\\ \tt \\ \tt \\ \tt Ihr seid gerade bei Runde 18 von maximal 28.\\ \tt Ihr habt gemeinsam 5 von maximal 16 Strafpunkten angesammelt.\\ \tt Dein Mitspieler hat dir diese Nachricht geschickt:\\ \tt \\ \tt \\ \tt "Klingt gut. Führe D→(1,3) und danach O→(5,5), dann A→(3,1) aus. Mein Brett entspricht bereits dem Zielplan: A(3,1), D(1,3), M(3,3), O(5,5), P(3,5), S(3,6), T(3,7). Sag Bescheid, sobald du fertig bist – dann sende ich „fertig?“."\\ \tt \\ \tt \\ \tt Wie lautet dein nächster Befehl? 
	  } 
	   } 
	   } 
	 & \\ 
 

    \theutterance \stepcounter{utterance}  

    & \multicolumn{4}{p{0.6\linewidth}}{\cellcolor[rgb]{0.8,1,0.9}{%
	\makecell[{{p{\linewidth}}}]{% 
	  \tt {\tiny [A$\rangle$GM]}  
	 VERSCHIEBE: D, (1,3) 
	  } 
	   } 
	   } 
	 & & \\ 
 

    \theutterance \stepcounter{utterance}  

    & & & \multicolumn{2}{p{0.3\linewidth}}{\cellcolor[rgb]{0.95,0.95,0.95}{%
	\makecell[{{p{\linewidth}}}]{% 
	  \tt {\tiny [GM$|$GM]}  
	 VERSCHIEBE: D, (1,3) 
	  } 
	   } 
	   } 
	 & & \\ 
 

    \theutterance \stepcounter{utterance}  

    & & & \multicolumn{2}{p{0.3\linewidth}}{\cellcolor[rgb]{0.95,0.95,0.95}{%
	\makecell[{{p{\linewidth}}}]{% 
	  \tt {\tiny [GM$|$GM]}  
	 'D' erfolgreich nach (1,3) verschoben. Dein Raster sieht jetzt so aus:\\ \tt ```\\ \tt  1234567\\ \tt ╔═══╤═╤═╗\\ \tt ║◌◌O│◌│A║ 1\\ \tt ╟─┬─┼─┴─╢ 2\\ \tt ║D│M│◌◌◌║ 3\\ \tt ║◌├─┤◌┌─╢ 4\\ \tt ╟─┘P│◌│◌║ 5\\ \tt ║◌◌S└┬┴─╢ 6\\ \tt ║◌◌T◌│◌◌║ 7\\ \tt ╚════╧══╝\\ \tt ```\\ \tt  
	  } 
	   } 
	   } 
	 & & \\ 
 

    \theutterance \stepcounter{utterance}  

    & & \multicolumn{4}{p{0.6\linewidth}}{\cellcolor[rgb]{0.9,0.9,0.9}{%
	\makecell[{{p{\linewidth}}}]{% 
	  \tt {\tiny [GM$\rangle$B]}  
	 Deine Nachricht wurde an deinen Mitspieler übermittelt.\\ \tt \\ \tt \\ \tt Ihr seid gerade bei Runde 19 von maximal 28.\\ \tt Ihr habt gemeinsam 5 von maximal 16 Strafpunkten angesammelt.\\ \tt Dein Mitspieler hat ein Objekt auf seinem Raster bewegt.\\ \tt \\ \tt \\ \tt Wie lautet dein nächster Befehl? 
	  } 
	   } 
	   } 
	 & \\ 
 

    \theutterance \stepcounter{utterance}  

    & & & \multicolumn{4}{p{0.6\linewidth}}{\cellcolor[rgb]{1,0.85,0.72}{%
	\makecell[{{p{\linewidth}}}]{% 
	  \tt {\tiny [GM$\langle$B]}  
	 SPRICH: fertig? 
	  } 
	   } 
	   } 
	  \\ 
 

    \theutterance \stepcounter{utterance}  

    & & & \multicolumn{2}{p{0.3\linewidth}}{\cellcolor[rgb]{0.95,0.95,0.95}{%
	\makecell[{{p{\linewidth}}}]{% 
	  \tt {\tiny [GM$|$GM]}  
	 SPRICH: fertig? 
	  } 
	   } 
	   } 
	 & & \\ 
 

    \theutterance \stepcounter{utterance}  

    & & \multicolumn{4}{p{0.6\linewidth}}{\cellcolor[rgb]{0.9,0.9,0.9}{%
	\makecell[{{p{\linewidth}}}]{% 
	  \tt {\tiny [A$\langle$GM]}  
	 'D' erfolgreich nach (1,3) verschoben. Dein Raster sieht jetzt so aus:\\ \tt ```\\ \tt  1234567\\ \tt ╔═══╤═╤═╗\\ \tt ║◌◌O│◌│A║ 1\\ \tt ╟─┬─┼─┴─╢ 2\\ \tt ║D│M│◌◌◌║ 3\\ \tt ║◌├─┤◌┌─╢ 4\\ \tt ╟─┘P│◌│◌║ 5\\ \tt ║◌◌S└┬┴─╢ 6\\ \tt ║◌◌T◌│◌◌║ 7\\ \tt ╚════╧══╝\\ \tt ```\\ \tt Ihr seid gerade bei Runde 19 von maximal 28.\\ \tt Ihr habt gemeinsam 5 von maximal 16 Strafpunkten angesammelt.\\ \tt Dein Mitspieler hat dir diese Nachricht geschickt:\\ \tt \\ \tt \\ \tt "fertig?"\\ \tt \\ \tt \\ \tt Wie lautet dein nächster Befehl? 
	  } 
	   } 
	   } 
	 & \\ 
 

    \theutterance \stepcounter{utterance}  

    & \multicolumn{4}{p{0.6\linewidth}}{\cellcolor[rgb]{0.8,1,0.9}{%
	\makecell[{{p{\linewidth}}}]{% 
	  \tt {\tiny [A$\rangle$GM]}  
	 SPRICH: fertig! 
	  } 
	   } 
	   } 
	 & & \\ 
 

    \theutterance \stepcounter{utterance}  

    & & & \multicolumn{2}{p{0.3\linewidth}}{\cellcolor[rgb]{0.95,0.95,0.95}{%
	\makecell[{{p{\linewidth}}}]{% 
	  \tt {\tiny [GM$|$GM]}  
	 SPRICH: fertig! 
	  } 
	   } 
	   } 
	 & & \\ 
 

    \theutterance \stepcounter{utterance}  

    & & & \multicolumn{2}{p{0.3\linewidth}}{\cellcolor[rgb]{0.95,0.95,0.95}{%
	\makecell[{{p{\linewidth}}}]{% 
	  \tt {\tiny [GM$|$GM]}  
	 true 
	  } 
	   } 
	   } 
	 & & \\ 
 

    \theutterance \stepcounter{utterance}  

    & & & \multicolumn{2}{p{0.3\linewidth}}{\cellcolor[rgb]{0.95,0.95,0.95}{%
	\makecell[{{p{\linewidth}}}]{% 
	  \tt {\tiny [GM$|$GM]}  
	 * success: True\\ \tt * lose: False\\ \tt * aborted: False\\ \tt -------\\ \tt * Shifts: 9.00\\ \tt * Max Shifts: 12.00\\ \tt * Min Shifts: 6.00\\ \tt * End Distance Sum: 9.47\\ \tt * Init Distance Sum: 23.62\\ \tt * Expected Distance Sum: 29.33\\ \tt * Penalties: 5.00\\ \tt * Max Penalties: 16.00\\ \tt * Rounds: 19.00\\ \tt * Max Rounds: 28.00\\ \tt * Object Count: 7.00\\ \tt  
	  } 
	   } 
	   } 
	 & & \\ 
 

    \theutterance \stepcounter{utterance}  

    & & & \multicolumn{2}{p{0.3\linewidth}}{\cellcolor[rgb]{0.95,0.95,0.95}{%
	\makecell[{{p{\linewidth}}}]{% 
	  \tt {\tiny [GM$|$GM]}  
	 * Main Score: 47.10\\ \tt \\ \tt -------\\ \tt * Distance Score: 0.64\\ \tt * Consistency Score: 0.57\\ \tt * Coverage Score: 0.56\\ \tt * Penalty Score: 0.80\\ \tt * Alternative Penalty Score: 0.89\\ \tt * Alternative Main Score: 56.71\\ \tt \\ \tt -------\\ \tt * Shifts: 9.00\\ \tt * Max Shifts: 12.00\\ \tt * Min Shifts: 6.00\\ \tt * End Distance Sum: 9.47\\ \tt * Init Distance Sum: 23.62\\ \tt * Expected Distance Sum: 29.33\\ \tt * Penalties: 5.00\\ \tt * Max Penalties: 16.00\\ \tt * Rounds: 19.00\\ \tt * Max Rounds: 28.00\\ \tt * Object Count: 7.00\\ \tt  
	  } 
	   } 
	   } 
	 & & \\ 
 

\end{supertabular}
}

\end{document}
