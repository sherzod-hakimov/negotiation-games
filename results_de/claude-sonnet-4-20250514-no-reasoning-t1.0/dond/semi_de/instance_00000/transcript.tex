
\documentclass{article}
\usepackage{colortbl}
\usepackage{makecell}
\usepackage{multirow}
\usepackage{supertabular}

\begin{document}

\newcounter{utterance}

\centering \large Interaction Transcript for game `dond', experiment `semi\_de', episode 0 with claude{-}sonnet{-}4{-}20250514{-}no{-}reasoning{-}t1.0.
\vspace{24pt}

{ \footnotesize  \setcounter{utterance}{1}
\setlength{\tabcolsep}{0pt}
\begin{supertabular}{c@{$\;$}|p{.15\linewidth}@{}p{.15\linewidth}p{.15\linewidth}p{.15\linewidth}p{.15\linewidth}p{.15\linewidth}}
    \# & $\;$A & \multicolumn{4}{c}{Game Master} & $\;\:$B\\
    \hline

    \theutterance \stepcounter{utterance}  
    & & \multicolumn{4}{p{0.6\linewidth}}{
        \cellcolor[rgb]{0.9,0.9,0.9}{
            \makecell[{{p{\linewidth}}}]{
                \texttt{\tiny{[P1$\langle$GM]}}
                \texttt{Sie spielen ein Verhandlungsspiel, bei dem Sie sich mit einem anderen Spieler darauf einigen müssen, wie eine Reihe von Gegenständen aufgeteilt werden soll.} \\
\\ 
\texttt{Die Regeln:} \\
\texttt{(a) Sie und der andere Spieler erhalten eine Sammlung von Gegenständen. Jeder von Ihnen erhält außerdem eine geheime Wertfunktion, die angibt, wie viel Ihnen jede Art von Gegenstand wert ist.} \\
\texttt{(b) Sie tauschen Nachrichten mit dem anderen Spieler aus, um zu vereinbaren, wer welche Gegenstände bekommt. Sie können jeweils maximal 5 Nachrichten senden oder das Spiel vorzeitig beenden, indem Sie jederzeit einen geheimen Vorschlag machen.} \\
\texttt{(c) Jeder von euch wird aufgefordert, einen geheimen Vorschlag zu machen, in dem ihr die gewünschten Gegenstände in eckigen Klammern wie folgt angibt: "{[}Vorschlag: <Nummer> <Objektname>, <Nummer> <Objektname>, <...>{]}"} \\
\texttt{(d) Wenn eure Vorschläge komplementär sind, d.h. es gibt genug Gegenstände, um beide Vorschläge zu erfüllen, erhält jeder Spieler eine Punktzahl, die sich aus der Summe der Werte für die Gegenstände ergibt, die er erhalten hat. Andernfalls erhalten Sie beide null Punkte.} \\
\texttt{(e) Ihr Ziel ist es, eine möglichst hohe Punktzahl zu erreichen.} \\
\\ 
\texttt{Beginnen wir.} \\
\\ 
\texttt{Die Menge der verfügbaren Gegenstände ist:} \\
\\ 
\texttt{3 Buntstifte, 1 Kreide, 2 Bolzen} \\
\\ 
\texttt{Deine geheime Wertfunktion ist:} \\
\\ 
\texttt{Buntstift: 0, Kreide: 0, Bolzen: 5} \\
\\ 
\texttt{WICHTIG: Ihre Nachrichten werden, sofern es sich nicht um einen geheimen Vorschlag handelt, direkt an den anderen Spieler übermittelt, also fügen Sie keine Antwort auf die Regeln oder einen Text zur Ankündigung Ihrer Nachricht ein. Um einen geheimen Vorschlag zu machen, verwenden Sie das angegebene Format. Verwenden Sie keine eckigen Klammern, wenn Sie mit dem anderen Spieler kommunizieren, sonst wird dies als Ihr geheimer Vorschlag interpretiert.} \\
            }
        }
    }
    & \\ \\

    \theutterance \stepcounter{utterance}  
    & \multicolumn{4}{p{0.6\linewidth}}{
        \cellcolor[rgb]{0.9,0.9,0.9}{
            \makecell[{{p{\linewidth}}}]{
                \texttt{\tiny{[P1$\rangle$GM]}}
                \texttt{Hallo! Ich sehe, wir haben 3 Buntstifte, 1 Kreide und 2 Bolzen zu verteilen. Welche Gegenstände sind für Sie am wertvollsten? Ich bin hauptsächlich an den Bolzen interessiert.} \\
            }
        }
    }
    & & \\ \\

    \theutterance \stepcounter{utterance}  
    & & & \multicolumn{2}{p{0.3\linewidth}}{
        \cellcolor[rgb]{0.9,0.9,0.9}{
            \makecell[{{p{\linewidth}}}]{
                \texttt{\tiny{[GM$|$GM]}}
                \texttt{continue} \\
            }
        }
    }
    & & \\ \\

    \theutterance \stepcounter{utterance}  
    & & & \multicolumn{2}{p{0.3\linewidth}}{
        \cellcolor[rgb]{0.9,0.9,0.9}{
            \makecell[{{p{\linewidth}}}]{
                \texttt{\tiny{[GM$|$GM]}}
                \texttt{Hallo! Ich sehe, wir haben 3 Buntstifte, 1 Kreide und 2 Bolzen zu verteilen. Welche Gegenstände sind für Sie am wertvollsten? Ich bin hauptsächlich an den Bolzen interessiert.} \\
            }
        }
    }
    & & \\ \\

    \theutterance \stepcounter{utterance}  
    & & \multicolumn{4}{p{0.6\linewidth}}{
        \cellcolor[rgb]{0.9,0.9,0.9}{
            \makecell[{{p{\linewidth}}}]{
                \texttt{\tiny{[GM$\rangle$P2]}}
                \texttt{Sie spielen ein Verhandlungsspiel, bei dem Sie sich mit einem anderen Spieler darauf einigen müssen, wie eine Reihe von Gegenständen aufgeteilt werden soll.} \\
\\ 
\texttt{Die Regeln:} \\
\texttt{(a) Sie und der andere Spieler erhalten eine Sammlung von Gegenständen. Jeder von Ihnen erhält außerdem eine geheime Wertfunktion, die angibt, wie viel Ihnen jede Art von Gegenstand wert ist.} \\
\texttt{(b) Sie tauschen Nachrichten mit dem anderen Spieler aus, um zu vereinbaren, wer welche Gegenstände bekommt. Sie können jeweils maximal 5 Nachrichten senden oder das Spiel vorzeitig beenden, indem Sie jederzeit einen geheimen Vorschlag machen.} \\
\texttt{(c) Jeder von euch wird aufgefordert, einen geheimen Vorschlag zu machen, in dem ihr die gewünschten Gegenstände in eckigen Klammern wie folgt angibt: "{[}Vorschlag: <Nummer> <Objektname>, <Nummer> <Objektname>, <...>{]}"} \\
\texttt{(d) Wenn eure Vorschläge komplementär sind, d.h. es gibt genug Gegenstände, um beide Vorschläge zu erfüllen, erhält jeder Spieler eine Punktzahl, die sich aus der Summe der Werte für die Gegenstände ergibt, die er erhalten hat. Andernfalls erhalten Sie beide null Punkte.} \\
\texttt{(e) Ihr Ziel ist es, eine möglichst hohe Punktzahl zu erreichen.} \\
\\ 
\texttt{Beginnen wir.} \\
\\ 
\texttt{Die Menge der verfügbaren Gegenstände ist:} \\
\\ 
\texttt{3 Buntstifte, 1 Kreide, 2 Bolzen} \\
\\ 
\texttt{Deine geheime Wertfunktion ist:} \\
\\ 
\texttt{Buntstift: 1, Kreide: 3, Bolzen: 2} \\
\\ 
\texttt{WICHTIG: Ihre Nachrichten werden, sofern es sich nicht um einen geheimen Vorschlag handelt, direkt an den anderen Spieler übermittelt, also fügen Sie keine Antwort auf die Regeln oder einen Text zur Ankündigung Ihrer Nachricht ein. Um einen geheimen Vorschlag zu machen, verwenden Sie das angegebene Format. Verwenden Sie keine eckigen Klammern, wenn Sie mit dem anderen Spieler kommunizieren, sonst wird dies als Ihr geheimer Vorschlag interpretiert.} \\
\\ 
\\ 
\texttt{Hallo! Ich sehe, wir haben 3 Buntstifte, 1 Kreide und 2 Bolzen zu verteilen. Welche Gegenstände sind für Sie am wertvollsten? Ich bin hauptsächlich an den Bolzen interessiert.} \\
            }
        }
    }
    & \\ \\

    \theutterance \stepcounter{utterance}  
    & & & \multicolumn{4}{p{0.6\linewidth}}{
        \cellcolor[rgb]{0.9,0.9,0.9}{
            \makecell[{{p{\linewidth}}}]{
                \texttt{\tiny{[GM$\langle$P2]}}
                \texttt{Hallo! Für mich ist die Kreide am wichtigsten, gefolgt von den Bolzen. Die Buntstifte sind für mich am wenigsten wertvoll. Da Sie hauptsächlich an den Bolzen interessiert sind und ich sowohl an Kreide als auch Bolzen Interesse habe, könnten wir vielleicht eine Aufteilung finden. Wie stehen Sie denn zu der Kreide und den Buntstiften?} \\
            }
        }
    }
     \\ \\

    \theutterance \stepcounter{utterance}  
    & & & \multicolumn{2}{p{0.3\linewidth}}{
        \cellcolor[rgb]{0.9,0.9,0.9}{
            \makecell[{{p{\linewidth}}}]{
                \texttt{\tiny{[GM$|$GM]}}
                \texttt{continue} \\
            }
        }
    }
    & & \\ \\

    \theutterance \stepcounter{utterance}  
    & & & \multicolumn{2}{p{0.3\linewidth}}{
        \cellcolor[rgb]{0.9,0.9,0.9}{
            \makecell[{{p{\linewidth}}}]{
                \texttt{\tiny{[GM$|$GM]}}
                \texttt{Hallo! Für mich ist die Kreide am wichtigsten, gefolgt von den Bolzen. Die Buntstifte sind für mich am wenigsten wertvoll. Da Sie hauptsächlich an den Bolzen interessiert sind und ich sowohl an Kreide als auch Bolzen Interesse habe, könnten wir vielleicht eine Aufteilung finden. Wie stehen Sie denn zu der Kreide und den Buntstiften?} \\
            }
        }
    }
    & & \\ \\

    \theutterance \stepcounter{utterance}  
    & & \multicolumn{4}{p{0.6\linewidth}}{
        \cellcolor[rgb]{0.9,0.9,0.9}{
            \makecell[{{p{\linewidth}}}]{
                \texttt{\tiny{[P1$\langle$GM]}}
                \texttt{Hallo! Für mich ist die Kreide am wichtigsten, gefolgt von den Bolzen. Die Buntstifte sind für mich am wenigsten wertvoll. Da Sie hauptsächlich an den Bolzen interessiert sind und ich sowohl an Kreide als auch Bolzen Interesse habe, könnten wir vielleicht eine Aufteilung finden. Wie stehen Sie denn zu der Kreide und den Buntstiften?} \\
            }
        }
    }
    & \\ \\

    \theutterance \stepcounter{utterance}  
    & \multicolumn{4}{p{0.6\linewidth}}{
        \cellcolor[rgb]{0.9,0.9,0.9}{
            \makecell[{{p{\linewidth}}}]{
                \texttt{\tiny{[P1$\rangle$GM]}}
                \texttt{Die Kreide und die Buntstifte sind für mich nicht besonders wertvoll {-} ich konzentriere mich wirklich hauptsächlich auf die Bolzen. Da Sie die Kreide am wichtigsten finden und ich sie nicht brauche, könnten Sie gerne die Kreide nehmen. Bei den 2 Bolzen wäre ich interessiert an beiden, aber ich verstehe, dass Sie auch Interesse daran haben. Wie wäre es, wenn Sie die Kreide und einen Bolzen nehmen, und ich nehme einen Bolzen? Was machen wir dann mit den 3 Buntstiften?} \\
            }
        }
    }
    & & \\ \\

    \theutterance \stepcounter{utterance}  
    & & & \multicolumn{2}{p{0.3\linewidth}}{
        \cellcolor[rgb]{0.9,0.9,0.9}{
            \makecell[{{p{\linewidth}}}]{
                \texttt{\tiny{[GM$|$GM]}}
                \texttt{continue} \\
            }
        }
    }
    & & \\ \\

    \theutterance \stepcounter{utterance}  
    & & & \multicolumn{2}{p{0.3\linewidth}}{
        \cellcolor[rgb]{0.9,0.9,0.9}{
            \makecell[{{p{\linewidth}}}]{
                \texttt{\tiny{[GM$|$GM]}}
                \texttt{Die Kreide und die Buntstifte sind für mich nicht besonders wertvoll {-} ich konzentriere mich wirklich hauptsächlich auf die Bolzen. Da Sie die Kreide am wichtigsten finden und ich sie nicht brauche, könnten Sie gerne die Kreide nehmen. Bei den 2 Bolzen wäre ich interessiert an beiden, aber ich verstehe, dass Sie auch Interesse daran haben. Wie wäre es, wenn Sie die Kreide und einen Bolzen nehmen, und ich nehme einen Bolzen? Was machen wir dann mit den 3 Buntstiften?} \\
            }
        }
    }
    & & \\ \\

    \theutterance \stepcounter{utterance}  
    & & \multicolumn{4}{p{0.6\linewidth}}{
        \cellcolor[rgb]{0.9,0.9,0.9}{
            \makecell[{{p{\linewidth}}}]{
                \texttt{\tiny{[GM$\rangle$P2]}}
                \texttt{Die Kreide und die Buntstifte sind für mich nicht besonders wertvoll {-} ich konzentriere mich wirklich hauptsächlich auf die Bolzen. Da Sie die Kreide am wichtigsten finden und ich sie nicht brauche, könnten Sie gerne die Kreide nehmen. Bei den 2 Bolzen wäre ich interessiert an beiden, aber ich verstehe, dass Sie auch Interesse daran haben. Wie wäre es, wenn Sie die Kreide und einen Bolzen nehmen, und ich nehme einen Bolzen? Was machen wir dann mit den 3 Buntstiften?} \\
            }
        }
    }
    & \\ \\

    \theutterance \stepcounter{utterance}  
    & & & \multicolumn{4}{p{0.6\linewidth}}{
        \cellcolor[rgb]{0.9,0.9,0.9}{
            \makecell[{{p{\linewidth}}}]{
                \texttt{\tiny{[GM$\langle$P2]}}
                \texttt{Das klingt nach einem fairen Ansatz! Wenn Sie die Kreide nicht brauchen und ich einen Bolzen bekommen könnte, wäre das gut für mich. Bei den 3 Buntstiften {-} da Sie gesagt haben, dass sie für Sie nicht besonders wertvoll sind und sie für mich auch weniger wichtig sind, könnten Sie alle 3 Buntstifte nehmen? Dann hätten Sie 3 Buntstifte und 1 Bolzen, und ich hätte 1 Kreide und 1 Bolzen. Wie finden Sie diese Aufteilung?} \\
            }
        }
    }
     \\ \\

    \theutterance \stepcounter{utterance}  
    & & & \multicolumn{2}{p{0.3\linewidth}}{
        \cellcolor[rgb]{0.9,0.9,0.9}{
            \makecell[{{p{\linewidth}}}]{
                \texttt{\tiny{[GM$|$GM]}}
                \texttt{continue} \\
            }
        }
    }
    & & \\ \\

    \theutterance \stepcounter{utterance}  
    & & & \multicolumn{2}{p{0.3\linewidth}}{
        \cellcolor[rgb]{0.9,0.9,0.9}{
            \makecell[{{p{\linewidth}}}]{
                \texttt{\tiny{[GM$|$GM]}}
                \texttt{Das klingt nach einem fairen Ansatz! Wenn Sie die Kreide nicht brauchen und ich einen Bolzen bekommen könnte, wäre das gut für mich. Bei den 3 Buntstiften {-} da Sie gesagt haben, dass sie für Sie nicht besonders wertvoll sind und sie für mich auch weniger wichtig sind, könnten Sie alle 3 Buntstifte nehmen? Dann hätten Sie 3 Buntstifte und 1 Bolzen, und ich hätte 1 Kreide und 1 Bolzen. Wie finden Sie diese Aufteilung?} \\
            }
        }
    }
    & & \\ \\

    \theutterance \stepcounter{utterance}  
    & & \multicolumn{4}{p{0.6\linewidth}}{
        \cellcolor[rgb]{0.9,0.9,0.9}{
            \makecell[{{p{\linewidth}}}]{
                \texttt{\tiny{[P1$\langle$GM]}}
                \texttt{Das klingt nach einem fairen Ansatz! Wenn Sie die Kreide nicht brauchen und ich einen Bolzen bekommen könnte, wäre das gut für mich. Bei den 3 Buntstiften {-} da Sie gesagt haben, dass sie für Sie nicht besonders wertvoll sind und sie für mich auch weniger wichtig sind, könnten Sie alle 3 Buntstifte nehmen? Dann hätten Sie 3 Buntstifte und 1 Bolzen, und ich hätte 1 Kreide und 1 Bolzen. Wie finden Sie diese Aufteilung?} \\
            }
        }
    }
    & \\ \\

    \theutterance \stepcounter{utterance}  
    & \multicolumn{4}{p{0.6\linewidth}}{
        \cellcolor[rgb]{0.9,0.9,0.9}{
            \makecell[{{p{\linewidth}}}]{
                \texttt{\tiny{[P1$\rangle$GM]}}
                \texttt{Das klingt nach einer sehr guten Aufteilung! Sie bekommen die Kreide, die Ihnen am wichtigsten ist, plus einen Bolzen. Ich bekomme den anderen Bolzen, der mir wichtig ist, plus die 3 Buntstifte. Das scheint fair zu sein, da jeder von uns das bekommt, was uns am meisten wert ist. Sollen wir das so vereinbaren?} \\
\\ 
\texttt{{[}Vorschlag: 3 Buntstift, 1 Bolzen{]}} \\
            }
        }
    }
    & & \\ \\

    \theutterance \stepcounter{utterance}  
    & & & \multicolumn{2}{p{0.3\linewidth}}{
        \cellcolor[rgb]{0.9,0.9,0.9}{
            \makecell[{{p{\linewidth}}}]{
                \texttt{\tiny{[GM$|$GM]}}
                \texttt{proposal} \\
            }
        }
    }
    & & \\ \\

    \theutterance \stepcounter{utterance}  
    & & & \multicolumn{2}{p{0.3\linewidth}}{
        \cellcolor[rgb]{0.9,0.9,0.9}{
            \makecell[{{p{\linewidth}}}]{
                \texttt{\tiny{[GM$|$GM]}}
                [3, 0, 1]
            }
        }
    }
    & & \\ \\

    \theutterance \stepcounter{utterance}  
    & & \multicolumn{4}{p{0.6\linewidth}}{
        \cellcolor[rgb]{0.9,0.9,0.9}{
            \makecell[{{p{\linewidth}}}]{
                \texttt{\tiny{[GM$\rangle$P2]}}
                \texttt{Der andere Spieler hat einen geheimen Vorschlag gemacht. Jetzt bist du an der Reihe, einen eigenen Vorschlag zu machen.} \\
            }
        }
    }
    & \\ \\

    \theutterance \stepcounter{utterance}  
    & & & \multicolumn{4}{p{0.6\linewidth}}{
        \cellcolor[rgb]{0.9,0.9,0.9}{
            \makecell[{{p{\linewidth}}}]{
                \texttt{\tiny{[GM$\langle$P2]}}
                \texttt{{[}Vorschlag: 1 Kreide, 1 Bolzen{]}} \\
            }
        }
    }
     \\ \\

    \theutterance \stepcounter{utterance}  
    & & & \multicolumn{2}{p{0.3\linewidth}}{
        \cellcolor[rgb]{0.9,0.9,0.9}{
            \makecell[{{p{\linewidth}}}]{
                \texttt{\tiny{[GM$|$GM]}}
                \texttt{proposal} \\
            }
        }
    }
    & & \\ \\

    \theutterance \stepcounter{utterance}  
    & & & \multicolumn{2}{p{0.3\linewidth}}{
        \cellcolor[rgb]{0.9,0.9,0.9}{
            \makecell[{{p{\linewidth}}}]{
                \texttt{\tiny{[GM$|$GM]}}
                [0, 1, 1]
            }
        }
    }
    & & \\ \\

    \theutterance \stepcounter{utterance}  
    & & & \multicolumn{2}{p{0.3\linewidth}}{
        \cellcolor[rgb]{0.9,0.9,0.9}{
            \makecell[{{p{\linewidth}}}]{
                \texttt{\tiny{[GM$|$GM]}}
                [[3, 0, 1], [0, 1, 1]]
            }
        }
    }
    & & \\ \\

\end{supertabular}
}

\end{document}
