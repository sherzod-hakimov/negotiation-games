
\documentclass{article}
\usepackage{colortbl}
\usepackage{makecell}
\usepackage{multirow}
\usepackage{supertabular}

\begin{document}

\newcounter{utterance}

\twocolumn

{ \footnotesize  \setcounter{utterance}{1}
\setlength{\tabcolsep}{0pt}
\begin{supertabular}{c@{$\;$}|p{.15\linewidth}@{}p{.15\linewidth}p{.15\linewidth}p{.15\linewidth}p{.15\linewidth}p{.15\linewidth}}

    \# & $\;$A & \multicolumn{4}{c}{Game Master} & $\;\:$B\\
    \hline 

    \theutterance \stepcounter{utterance}  

    & & \multicolumn{4}{p{0.6\linewidth}}{\cellcolor[rgb]{0.9,0.9,0.9}{%
	\makecell[{{p{\linewidth}}}]{% 
	  \tt {\tiny [A$\langle$GM]}  
	 Sie spielen ein Verhandlungsspiel, bei dem Sie sich mit einem anderen Spieler darauf einigen müssen, wie eine Reihe von Gegenständen aufgeteilt werden soll.\\ \tt \\ \tt Die Regeln:\\ \tt (a) Sie und der andere Spieler erhalten eine Sammlung von Gegenständen. Jeder von Ihnen erhält außerdem eine geheime Wertfunktion, die angibt, wie viel Ihnen jede Art von Gegenstand wert ist.\\ \tt (b) Sie tauschen Nachrichten mit dem anderen Spieler aus, um zu vereinbaren, wer welche Gegenstände bekommt. Sie können jeweils maximal 5 Nachrichten senden oder das Spiel vorzeitig beenden, indem Sie jederzeit einen geheimen Vorschlag machen.\\ \tt (c) Jeder von euch wird aufgefordert, einen geheimen Vorschlag zu machen, in dem ihr die gewünschten Gegenstände in eckigen Klammern wie folgt angibt: "[Vorschlag: <Nummer> <Objektname>, <Nummer> <Objektname>, <...>]"\\ \tt (d) Wenn eure Vorschläge komplementär sind, d.h. es gibt genug Gegenstände, um beide Vorschläge zu erfüllen, erhält jeder Spieler eine Punktzahl, die sich aus der Summe der Werte für die Gegenstände ergibt, die er erhalten hat. Andernfalls erhalten Sie beide null Punkte.\\ \tt (e) Ihr Ziel ist es, eine möglichst hohe Punktzahl zu erreichen.\\ \tt \\ \tt Beginnen wir.\\ \tt \\ \tt Die Menge der verfügbaren Gegenstände ist:\\ \tt \\ \tt 1 Eimer, 3 Messer, 1 Batterie\\ \tt \\ \tt Deine geheime Wertfunktion ist:\\ \tt \\ \tt Eimer: 2, Messer: 2, Batterie: 2\\ \tt \\ \tt WICHTIG: Ihre Nachrichten werden, sofern es sich nicht um einen geheimen Vorschlag handelt, direkt an den anderen Spieler übermittelt, also fügen Sie keine Antwort auf die Regeln oder einen Text zur Ankündigung Ihrer Nachricht ein. Um einen geheimen Vorschlag zu machen, verwenden Sie das angegebene Format. Verwenden Sie keine eckigen Klammern, wenn Sie mit dem anderen Spieler kommunizieren, sonst wird dies als Ihr geheimer Vorschlag interpretiert.\\ \tt  
	  } 
	   } 
	   } 
	 & \\ 
 

    \theutterance \stepcounter{utterance}  

    & \multicolumn{4}{p{0.6\linewidth}}{\cellcolor[rgb]{0.8,1,0.9}{%
	\makecell[{{p{\linewidth}}}]{% 
	  \tt {\tiny [A$\rangle$GM]}  
	 Hallo! Ich sehe, wir haben einen Eimer, drei Messer und eine Batterie zu verteilen. Was sind denn deine Präferenzen? Welche Gegenstände sind für dich besonders wertvoll? 
	  } 
	   } 
	   } 
	 & & \\ 
 

    \theutterance \stepcounter{utterance}  

    & & & \multicolumn{2}{p{0.3\linewidth}}{\cellcolor[rgb]{0.95,0.95,0.95}{%
	\makecell[{{p{\linewidth}}}]{% 
	  \tt {\tiny [GM$|$GM]}  
	 continue 
	  } 
	   } 
	   } 
	 & & \\ 
 

    \theutterance \stepcounter{utterance}  

    & & & \multicolumn{2}{p{0.3\linewidth}}{\cellcolor[rgb]{0.95,0.95,0.95}{%
	\makecell[{{p{\linewidth}}}]{% 
	  \tt {\tiny [GM$|$GM]}  
	 Hallo! Ich sehe, wir haben einen Eimer, drei Messer und eine Batterie zu verteilen. Was sind denn deine Präferenzen? Welche Gegenstände sind für dich besonders wertvoll? 
	  } 
	   } 
	   } 
	 & & \\ 
 

    \theutterance \stepcounter{utterance}  

    & & \multicolumn{4}{p{0.6\linewidth}}{\cellcolor[rgb]{0.9,0.9,0.9}{%
	\makecell[{{p{\linewidth}}}]{% 
	  \tt {\tiny [GM$\rangle$B]}  
	 Sie spielen ein Verhandlungsspiel, bei dem Sie sich mit einem anderen Spieler darauf einigen müssen, wie eine Reihe von Gegenständen aufgeteilt werden soll.\\ \tt \\ \tt Die Regeln:\\ \tt (a) Sie und der andere Spieler erhalten eine Sammlung von Gegenständen. Jeder von Ihnen erhält außerdem eine geheime Wertfunktion, die angibt, wie viel Ihnen jede Art von Gegenstand wert ist.\\ \tt (b) Sie tauschen Nachrichten mit dem anderen Spieler aus, um zu vereinbaren, wer welche Gegenstände bekommt. Sie können jeweils maximal 5 Nachrichten senden oder das Spiel vorzeitig beenden, indem Sie jederzeit einen geheimen Vorschlag machen.\\ \tt (c) Jeder von euch wird aufgefordert, einen geheimen Vorschlag zu machen, in dem ihr die gewünschten Gegenstände in eckigen Klammern wie folgt angibt: "[Vorschlag: <Nummer> <Objektname>, <Nummer> <Objektname>, <...>]"\\ \tt (d) Wenn eure Vorschläge komplementär sind, d.h. es gibt genug Gegenstände, um beide Vorschläge zu erfüllen, erhält jeder Spieler eine Punktzahl, die sich aus der Summe der Werte für die Gegenstände ergibt, die er erhalten hat. Andernfalls erhalten Sie beide null Punkte.\\ \tt (e) Ihr Ziel ist es, eine möglichst hohe Punktzahl zu erreichen.\\ \tt \\ \tt Beginnen wir.\\ \tt \\ \tt Die Menge der verfügbaren Gegenstände ist:\\ \tt \\ \tt 1 Eimer, 3 Messer, 1 Batterie\\ \tt \\ \tt Deine geheime Wertfunktion ist:\\ \tt \\ \tt Eimer: 3, Messer: 2, Batterie: 1\\ \tt \\ \tt WICHTIG: Ihre Nachrichten werden, sofern es sich nicht um einen geheimen Vorschlag handelt, direkt an den anderen Spieler übermittelt, also fügen Sie keine Antwort auf die Regeln oder einen Text zur Ankündigung Ihrer Nachricht ein. Um einen geheimen Vorschlag zu machen, verwenden Sie das angegebene Format. Verwenden Sie keine eckigen Klammern, wenn Sie mit dem anderen Spieler kommunizieren, sonst wird dies als Ihr geheimer Vorschlag interpretiert.\\ \tt \\ \tt \\ \tt Hallo! Ich sehe, wir haben einen Eimer, drei Messer und eine Batterie zu verteilen. Was sind denn deine Präferenzen? Welche Gegenstände sind für dich besonders wertvoll? 
	  } 
	   } 
	   } 
	 & \\ 
 

    \theutterance \stepcounter{utterance}  

    & & & \multicolumn{4}{p{0.6\linewidth}}{\cellcolor[rgb]{1,0.85,0.72}{%
	\makecell[{{p{\linewidth}}}]{% 
	  \tt {\tiny [GM$\langle$B]}  
	 Hallo! Für mich sind alle Gegenstände nützlich, aber ich habe unterschiedliche Präferenzen. Die Messer sind für mich ziemlich wichtig, da ich sie gut gebrauchen kann. Der Eimer wäre auch sehr nützlich. Die Batterie ist weniger kritisch für mich. Was denkst du denn über eine mögliche Aufteilung? Welche Gegenstände brauchst du am dringendsten? 
	  } 
	   } 
	   } 
	  \\ 
 

    \theutterance \stepcounter{utterance}  

    & & & \multicolumn{2}{p{0.3\linewidth}}{\cellcolor[rgb]{0.95,0.95,0.95}{%
	\makecell[{{p{\linewidth}}}]{% 
	  \tt {\tiny [GM$|$GM]}  
	 continue 
	  } 
	   } 
	   } 
	 & & \\ 
 

    \theutterance \stepcounter{utterance}  

    & & & \multicolumn{2}{p{0.3\linewidth}}{\cellcolor[rgb]{0.95,0.95,0.95}{%
	\makecell[{{p{\linewidth}}}]{% 
	  \tt {\tiny [GM$|$GM]}  
	 Hallo! Für mich sind alle Gegenstände nützlich, aber ich habe unterschiedliche Präferenzen. Die Messer sind für mich ziemlich wichtig, da ich sie gut gebrauchen kann. Der Eimer wäre auch sehr nützlich. Die Batterie ist weniger kritisch für mich. Was denkst du denn über eine mögliche Aufteilung? Welche Gegenstände brauchst du am dringendsten? 
	  } 
	   } 
	   } 
	 & & \\ 
 

    \theutterance \stepcounter{utterance}  

    & & \multicolumn{4}{p{0.6\linewidth}}{\cellcolor[rgb]{0.9,0.9,0.9}{%
	\makecell[{{p{\linewidth}}}]{% 
	  \tt {\tiny [A$\langle$GM]}  
	 Hallo! Für mich sind alle Gegenstände nützlich, aber ich habe unterschiedliche Präferenzen. Die Messer sind für mich ziemlich wichtig, da ich sie gut gebrauchen kann. Der Eimer wäre auch sehr nützlich. Die Batterie ist weniger kritisch für mich. Was denkst du denn über eine mögliche Aufteilung? Welche Gegenstände brauchst du am dringendsten? 
	  } 
	   } 
	   } 
	 & \\ 
 

    \theutterance \stepcounter{utterance}  

    & \multicolumn{4}{p{0.6\linewidth}}{\cellcolor[rgb]{0.8,1,0.9}{%
	\makecell[{{p{\linewidth}}}]{% 
	  \tt {\tiny [A$\rangle$GM]}  
	 Das ist interessant! Da du die Messer besonders wichtig findest, könnte ich mir vorstellen, dass du einen Großteil oder alle drei Messer bekommst. Für mich sind alle Gegenstände etwa gleich wertvoll, also bin ich flexibel. \\ \tt \\ \tt Eine Idee wäre: Du bekommst alle 3 Messer und den Eimer, ich nehme die Batterie. Oder falls dir das zu unausgewogen erscheint: Du bekommst 2 Messer und den Eimer, ich bekomme 1 Messer und die Batterie. Was denkst du darüber? 
	  } 
	   } 
	   } 
	 & & \\ 
 

    \theutterance \stepcounter{utterance}  

    & & & \multicolumn{2}{p{0.3\linewidth}}{\cellcolor[rgb]{0.95,0.95,0.95}{%
	\makecell[{{p{\linewidth}}}]{% 
	  \tt {\tiny [GM$|$GM]}  
	 continue 
	  } 
	   } 
	   } 
	 & & \\ 
 

    \theutterance \stepcounter{utterance}  

    & & & \multicolumn{2}{p{0.3\linewidth}}{\cellcolor[rgb]{0.95,0.95,0.95}{%
	\makecell[{{p{\linewidth}}}]{% 
	  \tt {\tiny [GM$|$GM]}  
	 Das ist interessant! Da du die Messer besonders wichtig findest, könnte ich mir vorstellen, dass du einen Großteil oder alle drei Messer bekommst. Für mich sind alle Gegenstände etwa gleich wertvoll, also bin ich flexibel. \\ \tt \\ \tt Eine Idee wäre: Du bekommst alle 3 Messer und den Eimer, ich nehme die Batterie. Oder falls dir das zu unausgewogen erscheint: Du bekommst 2 Messer und den Eimer, ich bekomme 1 Messer und die Batterie. Was denkst du darüber? 
	  } 
	   } 
	   } 
	 & & \\ 
 

    \theutterance \stepcounter{utterance}  

    & & \multicolumn{4}{p{0.6\linewidth}}{\cellcolor[rgb]{0.9,0.9,0.9}{%
	\makecell[{{p{\linewidth}}}]{% 
	  \tt {\tiny [GM$\rangle$B]}  
	 Das ist interessant! Da du die Messer besonders wichtig findest, könnte ich mir vorstellen, dass du einen Großteil oder alle drei Messer bekommst. Für mich sind alle Gegenstände etwa gleich wertvoll, also bin ich flexibel. \\ \tt \\ \tt Eine Idee wäre: Du bekommst alle 3 Messer und den Eimer, ich nehme die Batterie. Oder falls dir das zu unausgewogen erscheint: Du bekommst 2 Messer und den Eimer, ich bekomme 1 Messer und die Batterie. Was denkst du darüber? 
	  } 
	   } 
	   } 
	 & \\ 
 

    \theutterance \stepcounter{utterance}  

    & & & \multicolumn{4}{p{0.6\linewidth}}{\cellcolor[rgb]{1,0.85,0.72}{%
	\makecell[{{p{\linewidth}}}]{% 
	  \tt {\tiny [GM$\langle$B]}  
	 Das sind sehr großzügige Vorschläge! Ich schätze deine Flexibilität sehr. Der erste Vorschlag (ich bekomme alle 3 Messer und den Eimer, du die Batterie) klingt tatsächlich sehr attraktiv für mich, da die Messer wichtig für mich sind und der Eimer auch sehr nützlich wäre.\\ \tt \\ \tt Aber ich möchte sicherstellen, dass das auch für dich fair ist. Du sagst, alle Gegenstände sind für dich etwa gleich wertvoll - würdest du dich mit nur der Batterie wirklich zufrieden fühlen? Oder wäre die zweite Option mit 2 Messern und dem Eimer für mich und 1 Messer plus Batterie für dich ausgewogener für uns beide? 
	  } 
	   } 
	   } 
	  \\ 
 

    \theutterance \stepcounter{utterance}  

    & & & \multicolumn{2}{p{0.3\linewidth}}{\cellcolor[rgb]{0.95,0.95,0.95}{%
	\makecell[{{p{\linewidth}}}]{% 
	  \tt {\tiny [GM$|$GM]}  
	 continue 
	  } 
	   } 
	   } 
	 & & \\ 
 

    \theutterance \stepcounter{utterance}  

    & & & \multicolumn{2}{p{0.3\linewidth}}{\cellcolor[rgb]{0.95,0.95,0.95}{%
	\makecell[{{p{\linewidth}}}]{% 
	  \tt {\tiny [GM$|$GM]}  
	 Das sind sehr großzügige Vorschläge! Ich schätze deine Flexibilität sehr. Der erste Vorschlag (ich bekomme alle 3 Messer und den Eimer, du die Batterie) klingt tatsächlich sehr attraktiv für mich, da die Messer wichtig für mich sind und der Eimer auch sehr nützlich wäre.\\ \tt \\ \tt Aber ich möchte sicherstellen, dass das auch für dich fair ist. Du sagst, alle Gegenstände sind für dich etwa gleich wertvoll - würdest du dich mit nur der Batterie wirklich zufrieden fühlen? Oder wäre die zweite Option mit 2 Messern und dem Eimer für mich und 1 Messer plus Batterie für dich ausgewogener für uns beide? 
	  } 
	   } 
	   } 
	 & & \\ 
 

    \theutterance \stepcounter{utterance}  

    & & \multicolumn{4}{p{0.6\linewidth}}{\cellcolor[rgb]{0.9,0.9,0.9}{%
	\makecell[{{p{\linewidth}}}]{% 
	  \tt {\tiny [A$\langle$GM]}  
	 Das sind sehr großzügige Vorschläge! Ich schätze deine Flexibilität sehr. Der erste Vorschlag (ich bekomme alle 3 Messer und den Eimer, du die Batterie) klingt tatsächlich sehr attraktiv für mich, da die Messer wichtig für mich sind und der Eimer auch sehr nützlich wäre.\\ \tt \\ \tt Aber ich möchte sicherstellen, dass das auch für dich fair ist. Du sagst, alle Gegenstände sind für dich etwa gleich wertvoll - würdest du dich mit nur der Batterie wirklich zufrieden fühlen? Oder wäre die zweite Option mit 2 Messern und dem Eimer für mich und 1 Messer plus Batterie für dich ausgewogener für uns beide? 
	  } 
	   } 
	   } 
	 & \\ 
 

    \theutterance \stepcounter{utterance}  

    & \multicolumn{4}{p{0.6\linewidth}}{\cellcolor[rgb]{0.8,1,0.9}{%
	\makecell[{{p{\linewidth}}}]{% 
	  \tt {\tiny [A$\rangle$GM]}  
	 Deine Fairness schätze ich sehr! Da für mich alle Gegenstände tatsächlich gleich wertvoll sind, bin ich mit nur der Batterie völlig zufrieden - besonders wenn das für dich optimal ist. Es macht für mich keinen großen Unterschied, ob ich 1 oder 3 Gegenstände bekomme, solange wir uns einigen können.\\ \tt \\ \tt Wenn du mit allen 3 Messern und dem Eimer am besten bedient bist, dann lass uns dabei bleiben. Soll ich den Vorschlag offiziell machen oder möchtest du das tun? 
	  } 
	   } 
	   } 
	 & & \\ 
 

    \theutterance \stepcounter{utterance}  

    & & & \multicolumn{2}{p{0.3\linewidth}}{\cellcolor[rgb]{0.95,0.95,0.95}{%
	\makecell[{{p{\linewidth}}}]{% 
	  \tt {\tiny [GM$|$GM]}  
	 continue 
	  } 
	   } 
	   } 
	 & & \\ 
 

    \theutterance \stepcounter{utterance}  

    & & & \multicolumn{2}{p{0.3\linewidth}}{\cellcolor[rgb]{0.95,0.95,0.95}{%
	\makecell[{{p{\linewidth}}}]{% 
	  \tt {\tiny [GM$|$GM]}  
	 Deine Fairness schätze ich sehr! Da für mich alle Gegenstände tatsächlich gleich wertvoll sind, bin ich mit nur der Batterie völlig zufrieden - besonders wenn das für dich optimal ist. Es macht für mich keinen großen Unterschied, ob ich 1 oder 3 Gegenstände bekomme, solange wir uns einigen können.\\ \tt \\ \tt Wenn du mit allen 3 Messern und dem Eimer am besten bedient bist, dann lass uns dabei bleiben. Soll ich den Vorschlag offiziell machen oder möchtest du das tun? 
	  } 
	   } 
	   } 
	 & & \\ 
 

    \theutterance \stepcounter{utterance}  

    & & \multicolumn{4}{p{0.6\linewidth}}{\cellcolor[rgb]{0.9,0.9,0.9}{%
	\makecell[{{p{\linewidth}}}]{% 
	  \tt {\tiny [GM$\rangle$B]}  
	 Deine Fairness schätze ich sehr! Da für mich alle Gegenstände tatsächlich gleich wertvoll sind, bin ich mit nur der Batterie völlig zufrieden - besonders wenn das für dich optimal ist. Es macht für mich keinen großen Unterschied, ob ich 1 oder 3 Gegenstände bekomme, solange wir uns einigen können.\\ \tt \\ \tt Wenn du mit allen 3 Messern und dem Eimer am besten bedient bist, dann lass uns dabei bleiben. Soll ich den Vorschlag offiziell machen oder möchtest du das tun? 
	  } 
	   } 
	   } 
	 & \\ 
 

    \theutterance \stepcounter{utterance}  

    & & & \multicolumn{4}{p{0.6\linewidth}}{\cellcolor[rgb]{1,0.85,0.72}{%
	\makecell[{{p{\linewidth}}}]{% 
	  \tt {\tiny [GM$\langle$B]}  
	 Vielen Dank für deine Großzügigkeit und Fairness! Wenn du wirklich zufrieden bist mit nur der Batterie und es für dich keinen großen Unterschied macht, dann nehme ich das Angebot gerne an. Das wäre wirklich optimal für mich.\\ \tt \\ \tt Ich kann den offiziellen Vorschlag machen - dann sind wir beide auf derselben Seite und es gibt keine Missverständnisse.\\ \tt \\ \tt [Vorschlag: 1 Eimer, 3 Messer] 
	  } 
	   } 
	   } 
	  \\ 
 

    \theutterance \stepcounter{utterance}  

    & & & \multicolumn{2}{p{0.3\linewidth}}{\cellcolor[rgb]{0.95,0.95,0.95}{%
	\makecell[{{p{\linewidth}}}]{% 
	  \tt {\tiny [GM$|$GM]}  
	 proposal 
	  } 
	   } 
	   } 
	 & & \\ 
 

    \theutterance \stepcounter{utterance}  

    & & & \multicolumn{2}{p{0.3\linewidth}}{\cellcolor[rgb]{0.95,0.95,0.95}{%
	\makecell[{{p{\linewidth}}}]{% 
	  \tt {\tiny [GM$|$GM]}  
	 [1, 3, 0] 
	  } 
	   } 
	   } 
	 & & \\ 
 

    \theutterance \stepcounter{utterance}  

    & & \multicolumn{4}{p{0.6\linewidth}}{\cellcolor[rgb]{0.9,0.9,0.9}{%
	\makecell[{{p{\linewidth}}}]{% 
	  \tt {\tiny [A$\langle$GM]}  
	 Der andere Spieler hat einen geheimen Vorschlag gemacht. Jetzt bist du an der Reihe, einen eigenen Vorschlag zu machen.\\ \tt  
	  } 
	   } 
	   } 
	 & \\ 
 

    \theutterance \stepcounter{utterance}  

    & \multicolumn{4}{p{0.6\linewidth}}{\cellcolor[rgb]{0.8,1,0.9}{%
	\makecell[{{p{\linewidth}}}]{% 
	  \tt {\tiny [A$\rangle$GM]}  
	 [Vorschlag: 1 Batterie] 
	  } 
	   } 
	   } 
	 & & \\ 
 

    \theutterance \stepcounter{utterance}  

    & & & \multicolumn{2}{p{0.3\linewidth}}{\cellcolor[rgb]{0.95,0.95,0.95}{%
	\makecell[{{p{\linewidth}}}]{% 
	  \tt {\tiny [GM$|$GM]}  
	 proposal 
	  } 
	   } 
	   } 
	 & & \\ 
 

    \theutterance \stepcounter{utterance}  

    & & & \multicolumn{2}{p{0.3\linewidth}}{\cellcolor[rgb]{0.95,0.95,0.95}{%
	\makecell[{{p{\linewidth}}}]{% 
	  \tt {\tiny [GM$|$GM]}  
	 [0, 0, 1] 
	  } 
	   } 
	   } 
	 & & \\ 
 

    \theutterance \stepcounter{utterance}  

    & & & \multicolumn{2}{p{0.3\linewidth}}{\cellcolor[rgb]{0.95,0.95,0.95}{%
	\makecell[{{p{\linewidth}}}]{% 
	  \tt {\tiny [GM$|$GM]}  
	 [[0, 0, 1], [1, 3, 0]] 
	  } 
	   } 
	   } 
	 & & \\ 
 

\end{supertabular}
}

\end{document}
