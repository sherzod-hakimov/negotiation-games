
\documentclass{article}
\usepackage{colortbl}
\usepackage{makecell}
\usepackage{multirow}
\usepackage{supertabular}

\begin{document}

\newcounter{utterance}

\twocolumn

{ \footnotesize  \setcounter{utterance}{1}
\setlength{\tabcolsep}{0pt}
\begin{supertabular}{c@{$\;$}|p{.15\linewidth}@{}p{.15\linewidth}p{.15\linewidth}p{.15\linewidth}p{.15\linewidth}p{.15\linewidth}}

    \# & $\;$A & \multicolumn{4}{c}{Game Master} & $\;\:$B\\
    \hline 

    \theutterance \stepcounter{utterance}  

    & & \multicolumn{4}{p{0.6\linewidth}}{\cellcolor[rgb]{0.9,0.9,0.9}{%
	\makecell[{{p{\linewidth}}}]{% 
	  \tt {\tiny [A$\langle$GM]}  
	 Sie spielen ein Verhandlungsspiel, bei dem Sie sich mit einem anderen Spieler darauf einigen müssen, wie eine Reihe von Gegenständen aufgeteilt werden soll.\\ \tt \\ \tt Die Regeln:\\ \tt (a) Sie und der andere Spieler erhalten eine Sammlung von Gegenständen. Jeder von Ihnen erhält außerdem eine geheime Wertfunktion, die angibt, wie viel Ihnen jede Art von Gegenstand wert ist.\\ \tt (b) Sie tauschen Nachrichten mit dem anderen Spieler aus, um zu vereinbaren, wer welche Gegenstände bekommt. Sie können jeweils maximal 5 Nachrichten senden oder das Spiel vorzeitig beenden, indem Sie jederzeit einen geheimen Vorschlag machen.\\ \tt (c) Jeder von euch wird aufgefordert, einen geheimen Vorschlag zu machen, in dem ihr die gewünschten Gegenstände in eckigen Klammern wie folgt angibt: "[Vorschlag: <Nummer> <Objektname>, <Nummer> <Objektname>, <...>]"\\ \tt (d) Wenn eure Vorschläge komplementär sind, d.h. es gibt genug Gegenstände, um beide Vorschläge zu erfüllen, erhält jeder Spieler eine Punktzahl, die sich aus der Summe der Werte für die Gegenstände ergibt, die er erhalten hat. Andernfalls erhalten Sie beide null Punkte.\\ \tt (e) Ihr Ziel ist es, eine möglichst hohe Punktzahl zu erreichen.\\ \tt \\ \tt Beginnen wir.\\ \tt \\ \tt Die Menge der verfügbaren Gegenstände ist:\\ \tt \\ \tt 4 Papiere, 2 Uhren, 2 Kartoffeln\\ \tt \\ \tt Deine geheime Wertfunktion ist:\\ \tt \\ \tt Papier: 1, Uhr: 1, Kartoffel: 2\\ \tt \\ \tt WICHTIG: Ihre Nachrichten werden, sofern es sich nicht um einen geheimen Vorschlag handelt, direkt an den anderen Spieler übermittelt, also fügen Sie keine Antwort auf die Regeln oder einen Text zur Ankündigung Ihrer Nachricht ein. Um einen geheimen Vorschlag zu machen, verwenden Sie das angegebene Format. Verwenden Sie keine eckigen Klammern, wenn Sie mit dem anderen Spieler kommunizieren, sonst wird dies als Ihr geheimer Vorschlag interpretiert.\\ \tt  
	  } 
	   } 
	   } 
	 & \\ 
 

    \theutterance \stepcounter{utterance}  

    & \multicolumn{4}{p{0.6\linewidth}}{\cellcolor[rgb]{0.8,1,0.9}{%
	\makecell[{{p{\linewidth}}}]{% 
	  \tt {\tiny [A$\rangle$GM]}  
	 Hallo! Lass uns schaffen, dass wir beide möglichst viele Punkte bekommen. Welche Gegenstände sind für dich am wertvollsten? Ich bin interessiert zu erfahren, was dir wichtig ist, damit wir eine faire Aufteilung finden können. 
	  } 
	   } 
	   } 
	 & & \\ 
 

    \theutterance \stepcounter{utterance}  

    & & & \multicolumn{2}{p{0.3\linewidth}}{\cellcolor[rgb]{0.95,0.95,0.95}{%
	\makecell[{{p{\linewidth}}}]{% 
	  \tt {\tiny [GM$|$GM]}  
	 continue 
	  } 
	   } 
	   } 
	 & & \\ 
 

    \theutterance \stepcounter{utterance}  

    & & & \multicolumn{2}{p{0.3\linewidth}}{\cellcolor[rgb]{0.95,0.95,0.95}{%
	\makecell[{{p{\linewidth}}}]{% 
	  \tt {\tiny [GM$|$GM]}  
	 Hallo! Lass uns schaffen, dass wir beide möglichst viele Punkte bekommen. Welche Gegenstände sind für dich am wertvollsten? Ich bin interessiert zu erfahren, was dir wichtig ist, damit wir eine faire Aufteilung finden können. 
	  } 
	   } 
	   } 
	 & & \\ 
 

    \theutterance \stepcounter{utterance}  

    & & \multicolumn{4}{p{0.6\linewidth}}{\cellcolor[rgb]{0.9,0.9,0.9}{%
	\makecell[{{p{\linewidth}}}]{% 
	  \tt {\tiny [GM$\rangle$B]}  
	 Sie spielen ein Verhandlungsspiel, bei dem Sie sich mit einem anderen Spieler darauf einigen müssen, wie eine Reihe von Gegenständen aufgeteilt werden soll.\\ \tt \\ \tt Die Regeln:\\ \tt (a) Sie und der andere Spieler erhalten eine Sammlung von Gegenständen. Jeder von Ihnen erhält außerdem eine geheime Wertfunktion, die angibt, wie viel Ihnen jede Art von Gegenstand wert ist.\\ \tt (b) Sie tauschen Nachrichten mit dem anderen Spieler aus, um zu vereinbaren, wer welche Gegenstände bekommt. Sie können jeweils maximal 5 Nachrichten senden oder das Spiel vorzeitig beenden, indem Sie jederzeit einen geheimen Vorschlag machen.\\ \tt (c) Jeder von euch wird aufgefordert, einen geheimen Vorschlag zu machen, in dem ihr die gewünschten Gegenstände in eckigen Klammern wie folgt angibt: "[Vorschlag: <Nummer> <Objektname>, <Nummer> <Objektname>, <...>]"\\ \tt (d) Wenn eure Vorschläge komplementär sind, d.h. es gibt genug Gegenstände, um beide Vorschläge zu erfüllen, erhält jeder Spieler eine Punktzahl, die sich aus der Summe der Werte für die Gegenstände ergibt, die er erhalten hat. Andernfalls erhalten Sie beide null Punkte.\\ \tt (e) Ihr Ziel ist es, eine möglichst hohe Punktzahl zu erreichen.\\ \tt \\ \tt Beginnen wir.\\ \tt \\ \tt Die Menge der verfügbaren Gegenstände ist:\\ \tt \\ \tt 4 Papiere, 2 Uhren, 2 Kartoffeln\\ \tt \\ \tt Deine geheime Wertfunktion ist:\\ \tt \\ \tt Papier: 0, Uhr: 4, Kartoffel: 1\\ \tt \\ \tt WICHTIG: Ihre Nachrichten werden, sofern es sich nicht um einen geheimen Vorschlag handelt, direkt an den anderen Spieler übermittelt, also fügen Sie keine Antwort auf die Regeln oder einen Text zur Ankündigung Ihrer Nachricht ein. Um einen geheimen Vorschlag zu machen, verwenden Sie das angegebene Format. Verwenden Sie keine eckigen Klammern, wenn Sie mit dem anderen Spieler kommunizieren, sonst wird dies als Ihr geheimer Vorschlag interpretiert.\\ \tt \\ \tt \\ \tt Hallo! Lass uns schaffen, dass wir beide möglichst viele Punkte bekommen. Welche Gegenstände sind für dich am wertvollsten? Ich bin interessiert zu erfahren, was dir wichtig ist, damit wir eine faire Aufteilung finden können. 
	  } 
	   } 
	   } 
	 & \\ 
 

    \theutterance \stepcounter{utterance}  

    & & & \multicolumn{4}{p{0.6\linewidth}}{\cellcolor[rgb]{1,0.85,0.72}{%
	\makecell[{{p{\linewidth}}}]{% 
	  \tt {\tiny [GM$\langle$B]}  
	 Hallo! Das ist eine gute Herangehensweise. Für mich sind die Uhren am wertvollsten, gefolgt von den Kartoffeln. Die Papiere sind für mich nicht so wichtig. Und bei dir? Was schätzt du am meisten? 
	  } 
	   } 
	   } 
	  \\ 
 

    \theutterance \stepcounter{utterance}  

    & & & \multicolumn{2}{p{0.3\linewidth}}{\cellcolor[rgb]{0.95,0.95,0.95}{%
	\makecell[{{p{\linewidth}}}]{% 
	  \tt {\tiny [GM$|$GM]}  
	 continue 
	  } 
	   } 
	   } 
	 & & \\ 
 

    \theutterance \stepcounter{utterance}  

    & & & \multicolumn{2}{p{0.3\linewidth}}{\cellcolor[rgb]{0.95,0.95,0.95}{%
	\makecell[{{p{\linewidth}}}]{% 
	  \tt {\tiny [GM$|$GM]}  
	 Hallo! Das ist eine gute Herangehensweise. Für mich sind die Uhren am wertvollsten, gefolgt von den Kartoffeln. Die Papiere sind für mich nicht so wichtig. Und bei dir? Was schätzt du am meisten? 
	  } 
	   } 
	   } 
	 & & \\ 
 

    \theutterance \stepcounter{utterance}  

    & & \multicolumn{4}{p{0.6\linewidth}}{\cellcolor[rgb]{0.9,0.9,0.9}{%
	\makecell[{{p{\linewidth}}}]{% 
	  \tt {\tiny [A$\langle$GM]}  
	 Hallo! Das ist eine gute Herangehensweise. Für mich sind die Uhren am wertvollsten, gefolgt von den Kartoffeln. Die Papiere sind für mich nicht so wichtig. Und bei dir? Was schätzt du am meisten? 
	  } 
	   } 
	   } 
	 & \\ 
 

    \theutterance \stepcounter{utterance}  

    & \multicolumn{4}{p{0.6\linewidth}}{\cellcolor[rgb]{0.8,1,0.9}{%
	\makecell[{{p{\linewidth}}}]{% 
	  \tt {\tiny [A$\rangle$GM]}  
	 Das ist interessant! Für mich sind die Kartoffeln am wertvollsten, dann sind Papiere und Uhren etwa gleich wichtig für mich. \\ \tt \\ \tt Da du die Uhren am meisten schätzt und ich die Kartoffeln, könnten wir eine Win-Win-Situation schaffen. Wie wäre es, wenn du beide Uhren bekommst und ich beide Kartoffeln? Dann müssen wir nur noch die 4 Papiere aufteilen. Was denkst du darüber? 
	  } 
	   } 
	   } 
	 & & \\ 
 

    \theutterance \stepcounter{utterance}  

    & & & \multicolumn{2}{p{0.3\linewidth}}{\cellcolor[rgb]{0.95,0.95,0.95}{%
	\makecell[{{p{\linewidth}}}]{% 
	  \tt {\tiny [GM$|$GM]}  
	 continue 
	  } 
	   } 
	   } 
	 & & \\ 
 

    \theutterance \stepcounter{utterance}  

    & & & \multicolumn{2}{p{0.3\linewidth}}{\cellcolor[rgb]{0.95,0.95,0.95}{%
	\makecell[{{p{\linewidth}}}]{% 
	  \tt {\tiny [GM$|$GM]}  
	 Das ist interessant! Für mich sind die Kartoffeln am wertvollsten, dann sind Papiere und Uhren etwa gleich wichtig für mich. \\ \tt \\ \tt Da du die Uhren am meisten schätzt und ich die Kartoffeln, könnten wir eine Win-Win-Situation schaffen. Wie wäre es, wenn du beide Uhren bekommst und ich beide Kartoffeln? Dann müssen wir nur noch die 4 Papiere aufteilen. Was denkst du darüber? 
	  } 
	   } 
	   } 
	 & & \\ 
 

    \theutterance \stepcounter{utterance}  

    & & \multicolumn{4}{p{0.6\linewidth}}{\cellcolor[rgb]{0.9,0.9,0.9}{%
	\makecell[{{p{\linewidth}}}]{% 
	  \tt {\tiny [GM$\rangle$B]}  
	 Das ist interessant! Für mich sind die Kartoffeln am wertvollsten, dann sind Papiere und Uhren etwa gleich wichtig für mich. \\ \tt \\ \tt Da du die Uhren am meisten schätzt und ich die Kartoffeln, könnten wir eine Win-Win-Situation schaffen. Wie wäre es, wenn du beide Uhren bekommst und ich beide Kartoffeln? Dann müssen wir nur noch die 4 Papiere aufteilen. Was denkst du darüber? 
	  } 
	   } 
	   } 
	 & \\ 
 

    \theutterance \stepcounter{utterance}  

    & & & \multicolumn{4}{p{0.6\linewidth}}{\cellcolor[rgb]{1,0.85,0.72}{%
	\makecell[{{p{\linewidth}}}]{% 
	  \tt {\tiny [GM$\langle$B]}  
	 Das klingt wie ein sehr guter Ausgangspunkt! Wenn du die Kartoffeln am meisten schätzt und ich die Uhren, dann profitieren wir beide maximal von diesem Tausch.\\ \tt \\ \tt Für die 4 Papiere - da sie für mich weniger wichtig sind und du gesagt hast, dass Papiere und Uhren etwa gleich wichtig für dich sind, wäre ich bereit, dir den Großteil oder sogar alle Papiere zu überlassen. Du könntest also 2 Kartoffeln und 3-4 Papiere bekommen, während ich die 2 Uhren und vielleicht 0-1 Papier nehme.\\ \tt \\ \tt Was denkst du - sollen wir so verhandeln? 
	  } 
	   } 
	   } 
	  \\ 
 

    \theutterance \stepcounter{utterance}  

    & & & \multicolumn{2}{p{0.3\linewidth}}{\cellcolor[rgb]{0.95,0.95,0.95}{%
	\makecell[{{p{\linewidth}}}]{% 
	  \tt {\tiny [GM$|$GM]}  
	 continue 
	  } 
	   } 
	   } 
	 & & \\ 
 

    \theutterance \stepcounter{utterance}  

    & & & \multicolumn{2}{p{0.3\linewidth}}{\cellcolor[rgb]{0.95,0.95,0.95}{%
	\makecell[{{p{\linewidth}}}]{% 
	  \tt {\tiny [GM$|$GM]}  
	 Das klingt wie ein sehr guter Ausgangspunkt! Wenn du die Kartoffeln am meisten schätzt und ich die Uhren, dann profitieren wir beide maximal von diesem Tausch.\\ \tt \\ \tt Für die 4 Papiere - da sie für mich weniger wichtig sind und du gesagt hast, dass Papiere und Uhren etwa gleich wichtig für dich sind, wäre ich bereit, dir den Großteil oder sogar alle Papiere zu überlassen. Du könntest also 2 Kartoffeln und 3-4 Papiere bekommen, während ich die 2 Uhren und vielleicht 0-1 Papier nehme.\\ \tt \\ \tt Was denkst du - sollen wir so verhandeln? 
	  } 
	   } 
	   } 
	 & & \\ 
 

    \theutterance \stepcounter{utterance}  

    & & \multicolumn{4}{p{0.6\linewidth}}{\cellcolor[rgb]{0.9,0.9,0.9}{%
	\makecell[{{p{\linewidth}}}]{% 
	  \tt {\tiny [A$\langle$GM]}  
	 Das klingt wie ein sehr guter Ausgangspunkt! Wenn du die Kartoffeln am meisten schätzt und ich die Uhren, dann profitieren wir beide maximal von diesem Tausch.\\ \tt \\ \tt Für die 4 Papiere - da sie für mich weniger wichtig sind und du gesagt hast, dass Papiere und Uhren etwa gleich wichtig für dich sind, wäre ich bereit, dir den Großteil oder sogar alle Papiere zu überlassen. Du könntest also 2 Kartoffeln und 3-4 Papiere bekommen, während ich die 2 Uhren und vielleicht 0-1 Papier nehme.\\ \tt \\ \tt Was denkst du - sollen wir so verhandeln? 
	  } 
	   } 
	   } 
	 & \\ 
 

    \theutterance \stepcounter{utterance}  

    & \multicolumn{4}{p{0.6\linewidth}}{\cellcolor[rgb]{0.8,1,0.9}{%
	\makecell[{{p{\linewidth}}}]{% 
	  \tt {\tiny [A$\rangle$GM]}  
	 Das ist sehr großzügig von dir! Da du die Papiere nicht so wichtig findest, wäre ich gerne bereit, alle 4 Papiere zu nehmen. Das würde bedeuten:\\ \tt \\ \tt Du bekommst: 2 Uhren (deine wertvollsten Gegenstände)\\ \tt Ich bekomme: 2 Kartoffeln und 4 Papiere (meine wertvollsten Gegenstände plus die Papiere)\\ \tt \\ \tt Das scheint für uns beide optimal zu sein - du konzentrierst dich auf das, was dir am wichtigsten ist, und ich bekomme meine bevorzugten Kartoffeln plus die Papiere. Bist du mit dieser Aufteilung einverstanden? 
	  } 
	   } 
	   } 
	 & & \\ 
 

    \theutterance \stepcounter{utterance}  

    & & & \multicolumn{2}{p{0.3\linewidth}}{\cellcolor[rgb]{0.95,0.95,0.95}{%
	\makecell[{{p{\linewidth}}}]{% 
	  \tt {\tiny [GM$|$GM]}  
	 continue 
	  } 
	   } 
	   } 
	 & & \\ 
 

    \theutterance \stepcounter{utterance}  

    & & & \multicolumn{2}{p{0.3\linewidth}}{\cellcolor[rgb]{0.95,0.95,0.95}{%
	\makecell[{{p{\linewidth}}}]{% 
	  \tt {\tiny [GM$|$GM]}  
	 Das ist sehr großzügig von dir! Da du die Papiere nicht so wichtig findest, wäre ich gerne bereit, alle 4 Papiere zu nehmen. Das würde bedeuten:\\ \tt \\ \tt Du bekommst: 2 Uhren (deine wertvollsten Gegenstände)\\ \tt Ich bekomme: 2 Kartoffeln und 4 Papiere (meine wertvollsten Gegenstände plus die Papiere)\\ \tt \\ \tt Das scheint für uns beide optimal zu sein - du konzentrierst dich auf das, was dir am wichtigsten ist, und ich bekomme meine bevorzugten Kartoffeln plus die Papiere. Bist du mit dieser Aufteilung einverstanden? 
	  } 
	   } 
	   } 
	 & & \\ 
 

    \theutterance \stepcounter{utterance}  

    & & \multicolumn{4}{p{0.6\linewidth}}{\cellcolor[rgb]{0.9,0.9,0.9}{%
	\makecell[{{p{\linewidth}}}]{% 
	  \tt {\tiny [GM$\rangle$B]}  
	 Das ist sehr großzügig von dir! Da du die Papiere nicht so wichtig findest, wäre ich gerne bereit, alle 4 Papiere zu nehmen. Das würde bedeuten:\\ \tt \\ \tt Du bekommst: 2 Uhren (deine wertvollsten Gegenstände)\\ \tt Ich bekomme: 2 Kartoffeln und 4 Papiere (meine wertvollsten Gegenstände plus die Papiere)\\ \tt \\ \tt Das scheint für uns beide optimal zu sein - du konzentrierst dich auf das, was dir am wichtigsten ist, und ich bekomme meine bevorzugten Kartoffeln plus die Papiere. Bist du mit dieser Aufteilung einverstanden? 
	  } 
	   } 
	   } 
	 & \\ 
 

    \theutterance \stepcounter{utterance}  

    & & & \multicolumn{4}{p{0.6\linewidth}}{\cellcolor[rgb]{1,0.85,0.72}{%
	\makecell[{{p{\linewidth}}}]{% 
	  \tt {\tiny [GM$\langle$B]}  
	 Ja, das klingt perfekt! Diese Aufteilung macht für uns beide Sinn:\\ \tt \\ \tt Ich bekomme: 2 Uhren\\ \tt Du bekommst: 2 Kartoffeln, 4 Papiere\\ \tt \\ \tt Wir haben beide unsere wertvollsten Gegenstände bekommen und eine faire Lösung gefunden. Ich bin mit dieser Aufteilung einverstanden!\\ \tt \\ \tt [Vorschlag: 2 Uhr] 
	  } 
	   } 
	   } 
	  \\ 
 

    \theutterance \stepcounter{utterance}  

    & & & \multicolumn{2}{p{0.3\linewidth}}{\cellcolor[rgb]{0.95,0.95,0.95}{%
	\makecell[{{p{\linewidth}}}]{% 
	  \tt {\tiny [GM$|$GM]}  
	 proposal 
	  } 
	   } 
	   } 
	 & & \\ 
 

    \theutterance \stepcounter{utterance}  

    & & & \multicolumn{2}{p{0.3\linewidth}}{\cellcolor[rgb]{0.95,0.95,0.95}{%
	\makecell[{{p{\linewidth}}}]{% 
	  \tt {\tiny [GM$|$GM]}  
	 [0, 2, 0] 
	  } 
	   } 
	   } 
	 & & \\ 
 

    \theutterance \stepcounter{utterance}  

    & & \multicolumn{4}{p{0.6\linewidth}}{\cellcolor[rgb]{0.9,0.9,0.9}{%
	\makecell[{{p{\linewidth}}}]{% 
	  \tt {\tiny [A$\langle$GM]}  
	 Der andere Spieler hat einen geheimen Vorschlag gemacht. Jetzt bist du an der Reihe, einen eigenen Vorschlag zu machen.\\ \tt  
	  } 
	   } 
	   } 
	 & \\ 
 

    \theutterance \stepcounter{utterance}  

    & \multicolumn{4}{p{0.6\linewidth}}{\cellcolor[rgb]{0.8,1,0.9}{%
	\makecell[{{p{\linewidth}}}]{% 
	  \tt {\tiny [A$\rangle$GM]}  
	 [Vorschlag: 2 Kartoffel, 4 Papier] 
	  } 
	   } 
	   } 
	 & & \\ 
 

    \theutterance \stepcounter{utterance}  

    & & & \multicolumn{2}{p{0.3\linewidth}}{\cellcolor[rgb]{0.95,0.95,0.95}{%
	\makecell[{{p{\linewidth}}}]{% 
	  \tt {\tiny [GM$|$GM]}  
	 proposal 
	  } 
	   } 
	   } 
	 & & \\ 
 

    \theutterance \stepcounter{utterance}  

    & & & \multicolumn{2}{p{0.3\linewidth}}{\cellcolor[rgb]{0.95,0.95,0.95}{%
	\makecell[{{p{\linewidth}}}]{% 
	  \tt {\tiny [GM$|$GM]}  
	 [4, 0, 2] 
	  } 
	   } 
	   } 
	 & & \\ 
 

    \theutterance \stepcounter{utterance}  

    & & & \multicolumn{2}{p{0.3\linewidth}}{\cellcolor[rgb]{0.95,0.95,0.95}{%
	\makecell[{{p{\linewidth}}}]{% 
	  \tt {\tiny [GM$|$GM]}  
	 [[4, 0, 2], [0, 2, 0]] 
	  } 
	   } 
	   } 
	 & & \\ 
 

\end{supertabular}
}

\end{document}
