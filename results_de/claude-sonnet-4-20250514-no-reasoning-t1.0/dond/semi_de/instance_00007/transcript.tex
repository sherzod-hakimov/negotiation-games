
\documentclass{article}
\usepackage{colortbl}
\usepackage{makecell}
\usepackage{multirow}
\usepackage{supertabular}

\begin{document}

\newcounter{utterance}

\centering \large Interaction Transcript for game `dond', experiment `semi\_de', episode 7 with claude{-}sonnet{-}4{-}20250514{-}no{-}reasoning{-}t1.0.
\vspace{24pt}

{ \footnotesize  \setcounter{utterance}{1}
\setlength{\tabcolsep}{0pt}
\begin{supertabular}{c@{$\;$}|p{.15\linewidth}@{}p{.15\linewidth}p{.15\linewidth}p{.15\linewidth}p{.15\linewidth}p{.15\linewidth}}
    \# & $\;$A & \multicolumn{4}{c}{Game Master} & $\;\:$B\\
    \hline

    \theutterance \stepcounter{utterance}  
    & & \multicolumn{4}{p{0.6\linewidth}}{
        \cellcolor[rgb]{0.9,0.9,0.9}{
            \makecell[{{p{\linewidth}}}]{
                \texttt{\tiny{[P1$\langle$GM]}}
                \texttt{Sie spielen ein Verhandlungsspiel, bei dem Sie sich mit einem anderen Spieler darauf einigen müssen, wie eine Reihe von Gegenständen aufgeteilt werden soll.} \\
\\ 
\texttt{Die Regeln:} \\
\texttt{(a) Sie und der andere Spieler erhalten eine Sammlung von Gegenständen. Jeder von Ihnen erhält außerdem eine geheime Wertfunktion, die angibt, wie viel Ihnen jede Art von Gegenstand wert ist.} \\
\texttt{(b) Sie tauschen Nachrichten mit dem anderen Spieler aus, um zu vereinbaren, wer welche Gegenstände bekommt. Sie können jeweils maximal 5 Nachrichten senden oder das Spiel vorzeitig beenden, indem Sie jederzeit einen geheimen Vorschlag machen.} \\
\texttt{(c) Jeder von euch wird aufgefordert, einen geheimen Vorschlag zu machen, in dem ihr die gewünschten Gegenstände in eckigen Klammern wie folgt angibt: "{[}Vorschlag: <Nummer> <Objektname>, <Nummer> <Objektname>, <...>{]}"} \\
\texttt{(d) Wenn eure Vorschläge komplementär sind, d.h. es gibt genug Gegenstände, um beide Vorschläge zu erfüllen, erhält jeder Spieler eine Punktzahl, die sich aus der Summe der Werte für die Gegenstände ergibt, die er erhalten hat. Andernfalls erhalten Sie beide null Punkte.} \\
\texttt{(e) Ihr Ziel ist es, eine möglichst hohe Punktzahl zu erreichen.} \\
\\ 
\texttt{Beginnen wir.} \\
\\ 
\texttt{Die Menge der verfügbaren Gegenstände ist:} \\
\\ 
\texttt{2 Drähte, 2 Gläser, 2 Nadeln, 2 Tassen} \\
\\ 
\texttt{Deine geheime Wertfunktion ist:} \\
\\ 
\texttt{Draht: 1, Glas: 0, Nadel: 0, Tasse: 4} \\
\\ 
\texttt{WICHTIG: Ihre Nachrichten werden, sofern es sich nicht um einen geheimen Vorschlag handelt, direkt an den anderen Spieler übermittelt, also fügen Sie keine Antwort auf die Regeln oder einen Text zur Ankündigung Ihrer Nachricht ein. Um einen geheimen Vorschlag zu machen, verwenden Sie das angegebene Format. Verwenden Sie keine eckigen Klammern, wenn Sie mit dem anderen Spieler kommunizieren, sonst wird dies als Ihr geheimer Vorschlag interpretiert.} \\
            }
        }
    }
    & \\ \\

    \theutterance \stepcounter{utterance}  
    & \multicolumn{4}{p{0.6\linewidth}}{
        \cellcolor[rgb]{0.9,0.9,0.9}{
            \makecell[{{p{\linewidth}}}]{
                \texttt{\tiny{[P1$\rangle$GM]}}
                \texttt{Hallo! Ich schlage vor, dass wir unsere Präferenzen für die verschiedenen Gegenstände diskutieren. Welche Gegenstände sind für Sie am wertvollsten? Ich bin besonders interessiert an den Tassen, aber flexibel bei den anderen Gegenständen.} \\
            }
        }
    }
    & & \\ \\

    \theutterance \stepcounter{utterance}  
    & & & \multicolumn{2}{p{0.3\linewidth}}{
        \cellcolor[rgb]{0.9,0.9,0.9}{
            \makecell[{{p{\linewidth}}}]{
                \texttt{\tiny{[GM$|$GM]}}
                \texttt{continue} \\
            }
        }
    }
    & & \\ \\

    \theutterance \stepcounter{utterance}  
    & & & \multicolumn{2}{p{0.3\linewidth}}{
        \cellcolor[rgb]{0.9,0.9,0.9}{
            \makecell[{{p{\linewidth}}}]{
                \texttt{\tiny{[GM$|$GM]}}
                \texttt{Hallo! Ich schlage vor, dass wir unsere Präferenzen für die verschiedenen Gegenstände diskutieren. Welche Gegenstände sind für Sie am wertvollsten? Ich bin besonders interessiert an den Tassen, aber flexibel bei den anderen Gegenständen.} \\
            }
        }
    }
    & & \\ \\

    \theutterance \stepcounter{utterance}  
    & & \multicolumn{4}{p{0.6\linewidth}}{
        \cellcolor[rgb]{0.9,0.9,0.9}{
            \makecell[{{p{\linewidth}}}]{
                \texttt{\tiny{[GM$\rangle$P2]}}
                \texttt{Sie spielen ein Verhandlungsspiel, bei dem Sie sich mit einem anderen Spieler darauf einigen müssen, wie eine Reihe von Gegenständen aufgeteilt werden soll.} \\
\\ 
\texttt{Die Regeln:} \\
\texttt{(a) Sie und der andere Spieler erhalten eine Sammlung von Gegenständen. Jeder von Ihnen erhält außerdem eine geheime Wertfunktion, die angibt, wie viel Ihnen jede Art von Gegenstand wert ist.} \\
\texttt{(b) Sie tauschen Nachrichten mit dem anderen Spieler aus, um zu vereinbaren, wer welche Gegenstände bekommt. Sie können jeweils maximal 5 Nachrichten senden oder das Spiel vorzeitig beenden, indem Sie jederzeit einen geheimen Vorschlag machen.} \\
\texttt{(c) Jeder von euch wird aufgefordert, einen geheimen Vorschlag zu machen, in dem ihr die gewünschten Gegenstände in eckigen Klammern wie folgt angibt: "{[}Vorschlag: <Nummer> <Objektname>, <Nummer> <Objektname>, <...>{]}"} \\
\texttt{(d) Wenn eure Vorschläge komplementär sind, d.h. es gibt genug Gegenstände, um beide Vorschläge zu erfüllen, erhält jeder Spieler eine Punktzahl, die sich aus der Summe der Werte für die Gegenstände ergibt, die er erhalten hat. Andernfalls erhalten Sie beide null Punkte.} \\
\texttt{(e) Ihr Ziel ist es, eine möglichst hohe Punktzahl zu erreichen.} \\
\\ 
\texttt{Beginnen wir.} \\
\\ 
\texttt{Die Menge der verfügbaren Gegenstände ist:} \\
\\ 
\texttt{2 Drähte, 2 Gläser, 2 Nadeln, 2 Tassen} \\
\\ 
\texttt{Deine geheime Wertfunktion ist:} \\
\\ 
\texttt{Draht: 1, Glas: 2, Nadel: 2, Tasse: 0} \\
\\ 
\texttt{WICHTIG: Ihre Nachrichten werden, sofern es sich nicht um einen geheimen Vorschlag handelt, direkt an den anderen Spieler übermittelt, also fügen Sie keine Antwort auf die Regeln oder einen Text zur Ankündigung Ihrer Nachricht ein. Um einen geheimen Vorschlag zu machen, verwenden Sie das angegebene Format. Verwenden Sie keine eckigen Klammern, wenn Sie mit dem anderen Spieler kommunizieren, sonst wird dies als Ihr geheimer Vorschlag interpretiert.} \\
\\ 
\\ 
\texttt{Hallo! Ich schlage vor, dass wir unsere Präferenzen für die verschiedenen Gegenstände diskutieren. Welche Gegenstände sind für Sie am wertvollsten? Ich bin besonders interessiert an den Tassen, aber flexibel bei den anderen Gegenständen.} \\
            }
        }
    }
    & \\ \\

    \theutterance \stepcounter{utterance}  
    & & & \multicolumn{4}{p{0.6\linewidth}}{
        \cellcolor[rgb]{0.9,0.9,0.9}{
            \makecell[{{p{\linewidth}}}]{
                \texttt{\tiny{[GM$\langle$P2]}}
                \texttt{Hallo! Interessant, dass Sie sich für die Tassen interessieren. Für mich sind die Gläser und Nadeln am wertvollsten, während die Tassen für mich weniger wichtig sind. Die Drähte haben für mich einen geringen Wert. } \\
\\ 
\texttt{Da Sie die Tassen möchten und ich sie nicht so sehr brauche, könnte das eine gute Ausgangsbasis für eine Einigung sein. Wie stehen Sie denn zu den Gläsern und Nadeln?} \\
            }
        }
    }
     \\ \\

    \theutterance \stepcounter{utterance}  
    & & & \multicolumn{2}{p{0.3\linewidth}}{
        \cellcolor[rgb]{0.9,0.9,0.9}{
            \makecell[{{p{\linewidth}}}]{
                \texttt{\tiny{[GM$|$GM]}}
                \texttt{continue} \\
            }
        }
    }
    & & \\ \\

    \theutterance \stepcounter{utterance}  
    & & & \multicolumn{2}{p{0.3\linewidth}}{
        \cellcolor[rgb]{0.9,0.9,0.9}{
            \makecell[{{p{\linewidth}}}]{
                \texttt{\tiny{[GM$|$GM]}}
                \texttt{Hallo! Interessant, dass Sie sich für die Tassen interessieren. Für mich sind die Gläser und Nadeln am wertvollsten, während die Tassen für mich weniger wichtig sind. Die Drähte haben für mich einen geringen Wert. } \\
\\ 
\texttt{Da Sie die Tassen möchten und ich sie nicht so sehr brauche, könnte das eine gute Ausgangsbasis für eine Einigung sein. Wie stehen Sie denn zu den Gläsern und Nadeln?} \\
            }
        }
    }
    & & \\ \\

    \theutterance \stepcounter{utterance}  
    & & \multicolumn{4}{p{0.6\linewidth}}{
        \cellcolor[rgb]{0.9,0.9,0.9}{
            \makecell[{{p{\linewidth}}}]{
                \texttt{\tiny{[P1$\langle$GM]}}
                \texttt{Hallo! Interessant, dass Sie sich für die Tassen interessieren. Für mich sind die Gläser und Nadeln am wertvollsten, während die Tassen für mich weniger wichtig sind. Die Drähte haben für mich einen geringen Wert. } \\
\\ 
\texttt{Da Sie die Tassen möchten und ich sie nicht so sehr brauche, könnte das eine gute Ausgangsbasis für eine Einigung sein. Wie stehen Sie denn zu den Gläsern und Nadeln?} \\
            }
        }
    }
    & \\ \\

    \theutterance \stepcounter{utterance}  
    & \multicolumn{4}{p{0.6\linewidth}}{
        \cellcolor[rgb]{0.9,0.9,0.9}{
            \makecell[{{p{\linewidth}}}]{
                \texttt{\tiny{[P1$\rangle$GM]}}
                \texttt{Das klingt nach einer vielversprechenden Ausgangslage! Die Gläser und Nadeln sind für mich tatsächlich nicht so wichtig {-} ich konzentriere mich hauptsächlich auf die Tassen. Die Drähte haben für mich einen geringen, aber dennoch vorhandenen Wert.} \\
\\ 
\texttt{Wie wäre es mit folgendem Vorschlag: Sie bekommen beide Gläser und beide Nadeln, da diese für Sie wertvoll sind. Im Gegenzug würde ich gerne beide Tassen nehmen. Bei den Drähten könnten wir uns einigen {-} vielleicht können wir sie aufteilen oder Sie überlassen sie mir, da sie für Sie weniger wertvoll sind?} \\
            }
        }
    }
    & & \\ \\

    \theutterance \stepcounter{utterance}  
    & & & \multicolumn{2}{p{0.3\linewidth}}{
        \cellcolor[rgb]{0.9,0.9,0.9}{
            \makecell[{{p{\linewidth}}}]{
                \texttt{\tiny{[GM$|$GM]}}
                \texttt{continue} \\
            }
        }
    }
    & & \\ \\

    \theutterance \stepcounter{utterance}  
    & & & \multicolumn{2}{p{0.3\linewidth}}{
        \cellcolor[rgb]{0.9,0.9,0.9}{
            \makecell[{{p{\linewidth}}}]{
                \texttt{\tiny{[GM$|$GM]}}
                \texttt{Das klingt nach einer vielversprechenden Ausgangslage! Die Gläser und Nadeln sind für mich tatsächlich nicht so wichtig {-} ich konzentriere mich hauptsächlich auf die Tassen. Die Drähte haben für mich einen geringen, aber dennoch vorhandenen Wert.} \\
\\ 
\texttt{Wie wäre es mit folgendem Vorschlag: Sie bekommen beide Gläser und beide Nadeln, da diese für Sie wertvoll sind. Im Gegenzug würde ich gerne beide Tassen nehmen. Bei den Drähten könnten wir uns einigen {-} vielleicht können wir sie aufteilen oder Sie überlassen sie mir, da sie für Sie weniger wertvoll sind?} \\
            }
        }
    }
    & & \\ \\

    \theutterance \stepcounter{utterance}  
    & & \multicolumn{4}{p{0.6\linewidth}}{
        \cellcolor[rgb]{0.9,0.9,0.9}{
            \makecell[{{p{\linewidth}}}]{
                \texttt{\tiny{[GM$\rangle$P2]}}
                \texttt{Das klingt nach einer vielversprechenden Ausgangslage! Die Gläser und Nadeln sind für mich tatsächlich nicht so wichtig {-} ich konzentriere mich hauptsächlich auf die Tassen. Die Drähte haben für mich einen geringen, aber dennoch vorhandenen Wert.} \\
\\ 
\texttt{Wie wäre es mit folgendem Vorschlag: Sie bekommen beide Gläser und beide Nadeln, da diese für Sie wertvoll sind. Im Gegenzug würde ich gerne beide Tassen nehmen. Bei den Drähten könnten wir uns einigen {-} vielleicht können wir sie aufteilen oder Sie überlassen sie mir, da sie für Sie weniger wertvoll sind?} \\
            }
        }
    }
    & \\ \\

    \theutterance \stepcounter{utterance}  
    & & & \multicolumn{4}{p{0.6\linewidth}}{
        \cellcolor[rgb]{0.9,0.9,0.9}{
            \makecell[{{p{\linewidth}}}]{
                \texttt{\tiny{[GM$\langle$P2]}}
                \texttt{Das klingt nach einem sehr fairen Vorschlag! Sie haben recht {-} die Gläser und Nadeln sind für mich deutlich wertvoller, während Sie sich auf die Tassen konzentrieren. } \\
\\ 
\texttt{Was die Drähte angeht: Da sie für mich wirklich nur geringen Wert haben und Sie erwähnt haben, dass sie für Sie einen gewissen Wert darstellen, können Sie gerne beide Drähte nehmen.} \\
\\ 
\texttt{Zusammengefasst: Sie erhalten beide Tassen und beide Drähte, ich bekomme beide Gläser und beide Nadeln. Das scheint eine Win{-}Win{-}Situation für uns beide zu sein. Sind Sie mit dieser Aufteilung einverstanden?} \\
            }
        }
    }
     \\ \\

    \theutterance \stepcounter{utterance}  
    & & & \multicolumn{2}{p{0.3\linewidth}}{
        \cellcolor[rgb]{0.9,0.9,0.9}{
            \makecell[{{p{\linewidth}}}]{
                \texttt{\tiny{[GM$|$GM]}}
                \texttt{continue} \\
            }
        }
    }
    & & \\ \\

    \theutterance \stepcounter{utterance}  
    & & & \multicolumn{2}{p{0.3\linewidth}}{
        \cellcolor[rgb]{0.9,0.9,0.9}{
            \makecell[{{p{\linewidth}}}]{
                \texttt{\tiny{[GM$|$GM]}}
                \texttt{Das klingt nach einem sehr fairen Vorschlag! Sie haben recht {-} die Gläser und Nadeln sind für mich deutlich wertvoller, während Sie sich auf die Tassen konzentrieren. } \\
\\ 
\texttt{Was die Drähte angeht: Da sie für mich wirklich nur geringen Wert haben und Sie erwähnt haben, dass sie für Sie einen gewissen Wert darstellen, können Sie gerne beide Drähte nehmen.} \\
\\ 
\texttt{Zusammengefasst: Sie erhalten beide Tassen und beide Drähte, ich bekomme beide Gläser und beide Nadeln. Das scheint eine Win{-}Win{-}Situation für uns beide zu sein. Sind Sie mit dieser Aufteilung einverstanden?} \\
            }
        }
    }
    & & \\ \\

    \theutterance \stepcounter{utterance}  
    & & \multicolumn{4}{p{0.6\linewidth}}{
        \cellcolor[rgb]{0.9,0.9,0.9}{
            \makecell[{{p{\linewidth}}}]{
                \texttt{\tiny{[P1$\langle$GM]}}
                \texttt{Das klingt nach einem sehr fairen Vorschlag! Sie haben recht {-} die Gläser und Nadeln sind für mich deutlich wertvoller, während Sie sich auf die Tassen konzentrieren. } \\
\\ 
\texttt{Was die Drähte angeht: Da sie für mich wirklich nur geringen Wert haben und Sie erwähnt haben, dass sie für Sie einen gewissen Wert darstellen, können Sie gerne beide Drähte nehmen.} \\
\\ 
\texttt{Zusammengefasst: Sie erhalten beide Tassen und beide Drähte, ich bekomme beide Gläser und beide Nadeln. Das scheint eine Win{-}Win{-}Situation für uns beide zu sein. Sind Sie mit dieser Aufteilung einverstanden?} \\
            }
        }
    }
    & \\ \\

    \theutterance \stepcounter{utterance}  
    & \multicolumn{4}{p{0.6\linewidth}}{
        \cellcolor[rgb]{0.9,0.9,0.9}{
            \makecell[{{p{\linewidth}}}]{
                \texttt{\tiny{[P1$\rangle$GM]}}
                \texttt{Das ist perfekt! Diese Aufteilung scheint wirklich ideal für uns beide zu sein {-} Sie bekommen die für Sie wertvollen Gläser und Nadeln, ich bekomme die Tassen, die mir wichtig sind, plus die Drähte. Das ist eine klassische Win{-}Win{-}Situation.} \\
\\ 
\texttt{Ich bin vollkommen einverstanden mit dieser Aufteilung. Sollen wir das jetzt als unsere finalen Vorschläge einreichen?} \\
\\ 
\texttt{{[}Vorschlag: 2 Tassen, 2 Drähte{]}} \\
            }
        }
    }
    & & \\ \\

    \theutterance \stepcounter{utterance}  
    & & & \multicolumn{2}{p{0.3\linewidth}}{
        \cellcolor[rgb]{0.9,0.9,0.9}{
            \makecell[{{p{\linewidth}}}]{
                \texttt{\tiny{[GM$|$GM]}}
                \texttt{proposal} \\
            }
        }
    }
    & & \\ \\

    \theutterance \stepcounter{utterance}  
    & & & \multicolumn{2}{p{0.3\linewidth}}{
        \cellcolor[rgb]{0.9,0.9,0.9}{
            \makecell[{{p{\linewidth}}}]{
                \texttt{\tiny{[GM$|$GM]}}
                [2, 0, 0, 2]
            }
        }
    }
    & & \\ \\

    \theutterance \stepcounter{utterance}  
    & & \multicolumn{4}{p{0.6\linewidth}}{
        \cellcolor[rgb]{0.9,0.9,0.9}{
            \makecell[{{p{\linewidth}}}]{
                \texttt{\tiny{[GM$\rangle$P2]}}
                \texttt{Der andere Spieler hat einen geheimen Vorschlag gemacht. Jetzt bist du an der Reihe, einen eigenen Vorschlag zu machen.} \\
            }
        }
    }
    & \\ \\

    \theutterance \stepcounter{utterance}  
    & & & \multicolumn{4}{p{0.6\linewidth}}{
        \cellcolor[rgb]{0.9,0.9,0.9}{
            \makecell[{{p{\linewidth}}}]{
                \texttt{\tiny{[GM$\langle$P2]}}
                \texttt{Basierend auf unserer Diskussion mache ich folgenden Vorschlag:} \\
\\ 
\texttt{{[}Vorschlag: 2 Glas, 2 Nadel{]}} \\
            }
        }
    }
     \\ \\

    \theutterance \stepcounter{utterance}  
    & & & \multicolumn{2}{p{0.3\linewidth}}{
        \cellcolor[rgb]{0.9,0.9,0.9}{
            \makecell[{{p{\linewidth}}}]{
                \texttt{\tiny{[GM$|$GM]}}
                \texttt{proposal} \\
            }
        }
    }
    & & \\ \\

    \theutterance \stepcounter{utterance}  
    & & & \multicolumn{2}{p{0.3\linewidth}}{
        \cellcolor[rgb]{0.9,0.9,0.9}{
            \makecell[{{p{\linewidth}}}]{
                \texttt{\tiny{[GM$|$GM]}}
                [0, 2, 2, 0]
            }
        }
    }
    & & \\ \\

    \theutterance \stepcounter{utterance}  
    & & & \multicolumn{2}{p{0.3\linewidth}}{
        \cellcolor[rgb]{0.9,0.9,0.9}{
            \makecell[{{p{\linewidth}}}]{
                \texttt{\tiny{[GM$|$GM]}}
                [[2, 0, 0, 2], [0, 2, 2, 0]]
            }
        }
    }
    & & \\ \\

\end{supertabular}
}

\end{document}
