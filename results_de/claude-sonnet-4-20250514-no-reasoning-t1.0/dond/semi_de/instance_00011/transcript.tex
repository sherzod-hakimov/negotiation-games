
\documentclass{article}
\usepackage{colortbl}
\usepackage{makecell}
\usepackage{multirow}
\usepackage{supertabular}

\begin{document}

\newcounter{utterance}

\centering \large Interaction Transcript for game `dond', experiment `semi\_de', episode 11 with claude{-}sonnet{-}4{-}20250514{-}no{-}reasoning{-}t1.0.
\vspace{24pt}

{ \footnotesize  \setcounter{utterance}{1}
\setlength{\tabcolsep}{0pt}
\begin{supertabular}{c@{$\;$}|p{.15\linewidth}@{}p{.15\linewidth}p{.15\linewidth}p{.15\linewidth}p{.15\linewidth}p{.15\linewidth}}
    \# & $\;$A & \multicolumn{4}{c}{Game Master} & $\;\:$B\\
    \hline

    \theutterance \stepcounter{utterance}  
    & & \multicolumn{4}{p{0.6\linewidth}}{
        \cellcolor[rgb]{0.9,0.9,0.9}{
            \makecell[{{p{\linewidth}}}]{
                \texttt{\tiny{[P1$\langle$GM]}}
                \texttt{Sie spielen ein Verhandlungsspiel, bei dem Sie sich mit einem anderen Spieler darauf einigen müssen, wie eine Reihe von Gegenständen aufgeteilt werden soll.} \\
\\ 
\texttt{Die Regeln:} \\
\texttt{(a) Sie und der andere Spieler erhalten eine Sammlung von Gegenständen. Jeder von Ihnen erhält außerdem eine geheime Wertfunktion, die angibt, wie viel Ihnen jede Art von Gegenstand wert ist.} \\
\texttt{(b) Sie tauschen Nachrichten mit dem anderen Spieler aus, um zu vereinbaren, wer welche Gegenstände bekommt. Sie können jeweils maximal 5 Nachrichten senden oder das Spiel vorzeitig beenden, indem Sie jederzeit einen geheimen Vorschlag machen.} \\
\texttt{(c) Jeder von euch wird aufgefordert, einen geheimen Vorschlag zu machen, in dem ihr die gewünschten Gegenstände in eckigen Klammern wie folgt angibt: "{[}Vorschlag: <Nummer> <Objektname>, <Nummer> <Objektname>, <...>{]}"} \\
\texttt{(d) Wenn eure Vorschläge komplementär sind, d.h. es gibt genug Gegenstände, um beide Vorschläge zu erfüllen, erhält jeder Spieler eine Punktzahl, die sich aus der Summe der Werte für die Gegenstände ergibt, die er erhalten hat. Andernfalls erhalten Sie beide null Punkte.} \\
\texttt{(e) Ihr Ziel ist es, eine möglichst hohe Punktzahl zu erreichen.} \\
\\ 
\texttt{Beginnen wir.} \\
\\ 
\texttt{Die Menge der verfügbaren Gegenstände ist:} \\
\\ 
\texttt{1 Topf, 1 Süßigkeit, 1 Pfeife, 2 Würfel} \\
\\ 
\texttt{Deine geheime Wertfunktion ist:} \\
\\ 
\texttt{Topf: 2, Süßigkeit: 3, Pfeife: 3, Würfel: 1} \\
\\ 
\texttt{WICHTIG: Ihre Nachrichten werden, sofern es sich nicht um einen geheimen Vorschlag handelt, direkt an den anderen Spieler übermittelt, also fügen Sie keine Antwort auf die Regeln oder einen Text zur Ankündigung Ihrer Nachricht ein. Um einen geheimen Vorschlag zu machen, verwenden Sie das angegebene Format. Verwenden Sie keine eckigen Klammern, wenn Sie mit dem anderen Spieler kommunizieren, sonst wird dies als Ihr geheimer Vorschlag interpretiert.} \\
            }
        }
    }
    & \\ \\

    \theutterance \stepcounter{utterance}  
    & \multicolumn{4}{p{0.6\linewidth}}{
        \cellcolor[rgb]{0.9,0.9,0.9}{
            \makecell[{{p{\linewidth}}}]{
                \texttt{\tiny{[P1$\rangle$GM]}}
                \texttt{Hallo! Lass uns über die Aufteilung der Gegenstände sprechen. Wir haben einen Topf, eine Süßigkeit, eine Pfeife und zwei Würfel zur Verfügung. Welche Gegenstände sind für dich besonders wertvoll?} \\
            }
        }
    }
    & & \\ \\

    \theutterance \stepcounter{utterance}  
    & & & \multicolumn{2}{p{0.3\linewidth}}{
        \cellcolor[rgb]{0.9,0.9,0.9}{
            \makecell[{{p{\linewidth}}}]{
                \texttt{\tiny{[GM$|$GM]}}
                \texttt{continue} \\
            }
        }
    }
    & & \\ \\

    \theutterance \stepcounter{utterance}  
    & & & \multicolumn{2}{p{0.3\linewidth}}{
        \cellcolor[rgb]{0.9,0.9,0.9}{
            \makecell[{{p{\linewidth}}}]{
                \texttt{\tiny{[GM$|$GM]}}
                \texttt{Hallo! Lass uns über die Aufteilung der Gegenstände sprechen. Wir haben einen Topf, eine Süßigkeit, eine Pfeife und zwei Würfel zur Verfügung. Welche Gegenstände sind für dich besonders wertvoll?} \\
            }
        }
    }
    & & \\ \\

    \theutterance \stepcounter{utterance}  
    & & \multicolumn{4}{p{0.6\linewidth}}{
        \cellcolor[rgb]{0.9,0.9,0.9}{
            \makecell[{{p{\linewidth}}}]{
                \texttt{\tiny{[GM$\rangle$P2]}}
                \texttt{Sie spielen ein Verhandlungsspiel, bei dem Sie sich mit einem anderen Spieler darauf einigen müssen, wie eine Reihe von Gegenständen aufgeteilt werden soll.} \\
\\ 
\texttt{Die Regeln:} \\
\texttt{(a) Sie und der andere Spieler erhalten eine Sammlung von Gegenständen. Jeder von Ihnen erhält außerdem eine geheime Wertfunktion, die angibt, wie viel Ihnen jede Art von Gegenstand wert ist.} \\
\texttt{(b) Sie tauschen Nachrichten mit dem anderen Spieler aus, um zu vereinbaren, wer welche Gegenstände bekommt. Sie können jeweils maximal 5 Nachrichten senden oder das Spiel vorzeitig beenden, indem Sie jederzeit einen geheimen Vorschlag machen.} \\
\texttt{(c) Jeder von euch wird aufgefordert, einen geheimen Vorschlag zu machen, in dem ihr die gewünschten Gegenstände in eckigen Klammern wie folgt angibt: "{[}Vorschlag: <Nummer> <Objektname>, <Nummer> <Objektname>, <...>{]}"} \\
\texttt{(d) Wenn eure Vorschläge komplementär sind, d.h. es gibt genug Gegenstände, um beide Vorschläge zu erfüllen, erhält jeder Spieler eine Punktzahl, die sich aus der Summe der Werte für die Gegenstände ergibt, die er erhalten hat. Andernfalls erhalten Sie beide null Punkte.} \\
\texttt{(e) Ihr Ziel ist es, eine möglichst hohe Punktzahl zu erreichen.} \\
\\ 
\texttt{Beginnen wir.} \\
\\ 
\texttt{Die Menge der verfügbaren Gegenstände ist:} \\
\\ 
\texttt{1 Topf, 1 Süßigkeit, 1 Pfeife, 2 Würfel} \\
\\ 
\texttt{Deine geheime Wertfunktion ist:} \\
\\ 
\texttt{Topf: 8, Süßigkeit: 0, Pfeife: 0, Würfel: 1} \\
\\ 
\texttt{WICHTIG: Ihre Nachrichten werden, sofern es sich nicht um einen geheimen Vorschlag handelt, direkt an den anderen Spieler übermittelt, also fügen Sie keine Antwort auf die Regeln oder einen Text zur Ankündigung Ihrer Nachricht ein. Um einen geheimen Vorschlag zu machen, verwenden Sie das angegebene Format. Verwenden Sie keine eckigen Klammern, wenn Sie mit dem anderen Spieler kommunizieren, sonst wird dies als Ihr geheimer Vorschlag interpretiert.} \\
\\ 
\\ 
\texttt{Hallo! Lass uns über die Aufteilung der Gegenstände sprechen. Wir haben einen Topf, eine Süßigkeit, eine Pfeife und zwei Würfel zur Verfügung. Welche Gegenstände sind für dich besonders wertvoll?} \\
            }
        }
    }
    & \\ \\

    \theutterance \stepcounter{utterance}  
    & & & \multicolumn{4}{p{0.6\linewidth}}{
        \cellcolor[rgb]{0.9,0.9,0.9}{
            \makecell[{{p{\linewidth}}}]{
                \texttt{\tiny{[GM$\langle$P2]}}
                \texttt{Hallo! Ich interessiere mich hauptsächlich für die Süßigkeit und die Pfeife {-} die sind für mich am wertvollsten. Die Würfel finde ich auch ganz interessant. Der Topf ist für mich weniger wichtig. Wie sieht es bei dir aus? Was sind deine Prioritäten?} \\
            }
        }
    }
     \\ \\

    \theutterance \stepcounter{utterance}  
    & & & \multicolumn{2}{p{0.3\linewidth}}{
        \cellcolor[rgb]{0.9,0.9,0.9}{
            \makecell[{{p{\linewidth}}}]{
                \texttt{\tiny{[GM$|$GM]}}
                \texttt{continue} \\
            }
        }
    }
    & & \\ \\

    \theutterance \stepcounter{utterance}  
    & & & \multicolumn{2}{p{0.3\linewidth}}{
        \cellcolor[rgb]{0.9,0.9,0.9}{
            \makecell[{{p{\linewidth}}}]{
                \texttt{\tiny{[GM$|$GM]}}
                \texttt{Hallo! Ich interessiere mich hauptsächlich für die Süßigkeit und die Pfeife {-} die sind für mich am wertvollsten. Die Würfel finde ich auch ganz interessant. Der Topf ist für mich weniger wichtig. Wie sieht es bei dir aus? Was sind deine Prioritäten?} \\
            }
        }
    }
    & & \\ \\

    \theutterance \stepcounter{utterance}  
    & & \multicolumn{4}{p{0.6\linewidth}}{
        \cellcolor[rgb]{0.9,0.9,0.9}{
            \makecell[{{p{\linewidth}}}]{
                \texttt{\tiny{[P1$\langle$GM]}}
                \texttt{Hallo! Ich interessiere mich hauptsächlich für die Süßigkeit und die Pfeife {-} die sind für mich am wertvollsten. Die Würfel finde ich auch ganz interessant. Der Topf ist für mich weniger wichtig. Wie sieht es bei dir aus? Was sind deine Prioritäten?} \\
            }
        }
    }
    & \\ \\

    \theutterance \stepcounter{utterance}  
    & \multicolumn{4}{p{0.6\linewidth}}{
        \cellcolor[rgb]{0.9,0.9,0.9}{
            \makecell[{{p{\linewidth}}}]{
                \texttt{\tiny{[P1$\rangle$GM]}}
                \texttt{Das ist interessant! Die Süßigkeit und die Pfeife sind tatsächlich auch für mich sehr wertvoll. Der Topf hat für mich einen mittleren Wert, und die Würfel sind für mich am wenigsten wichtig. } \\
\\ 
\texttt{Da wir beide an Süßigkeit und Pfeife interessiert sind, müssen wir einen Kompromiss finden. Vielleicht könnte einer von uns die Süßigkeit nehmen und der andere die Pfeife? Wie würdest du das sehen, und was wäre mit dem Topf und den Würfeln?} \\
            }
        }
    }
    & & \\ \\

    \theutterance \stepcounter{utterance}  
    & & & \multicolumn{2}{p{0.3\linewidth}}{
        \cellcolor[rgb]{0.9,0.9,0.9}{
            \makecell[{{p{\linewidth}}}]{
                \texttt{\tiny{[GM$|$GM]}}
                \texttt{continue} \\
            }
        }
    }
    & & \\ \\

    \theutterance \stepcounter{utterance}  
    & & & \multicolumn{2}{p{0.3\linewidth}}{
        \cellcolor[rgb]{0.9,0.9,0.9}{
            \makecell[{{p{\linewidth}}}]{
                \texttt{\tiny{[GM$|$GM]}}
                \texttt{Das ist interessant! Die Süßigkeit und die Pfeife sind tatsächlich auch für mich sehr wertvoll. Der Topf hat für mich einen mittleren Wert, und die Würfel sind für mich am wenigsten wichtig. } \\
\\ 
\texttt{Da wir beide an Süßigkeit und Pfeife interessiert sind, müssen wir einen Kompromiss finden. Vielleicht könnte einer von uns die Süßigkeit nehmen und der andere die Pfeife? Wie würdest du das sehen, und was wäre mit dem Topf und den Würfeln?} \\
            }
        }
    }
    & & \\ \\

    \theutterance \stepcounter{utterance}  
    & & \multicolumn{4}{p{0.6\linewidth}}{
        \cellcolor[rgb]{0.9,0.9,0.9}{
            \makecell[{{p{\linewidth}}}]{
                \texttt{\tiny{[GM$\rangle$P2]}}
                \texttt{Das ist interessant! Die Süßigkeit und die Pfeife sind tatsächlich auch für mich sehr wertvoll. Der Topf hat für mich einen mittleren Wert, und die Würfel sind für mich am wenigsten wichtig. } \\
\\ 
\texttt{Da wir beide an Süßigkeit und Pfeife interessiert sind, müssen wir einen Kompromiss finden. Vielleicht könnte einer von uns die Süßigkeit nehmen und der andere die Pfeife? Wie würdest du das sehen, und was wäre mit dem Topf und den Würfeln?} \\
            }
        }
    }
    & \\ \\

    \theutterance \stepcounter{utterance}  
    & & & \multicolumn{4}{p{0.6\linewidth}}{
        \cellcolor[rgb]{0.9,0.9,0.9}{
            \makecell[{{p{\linewidth}}}]{
                \texttt{\tiny{[GM$\langle$P2]}}
                \texttt{Das klingt nach einem vernünftigen Ansatz! Da wir beide an Süßigkeit und Pfeife interessiert sind, macht eine Aufteilung Sinn.} \\
\\ 
\texttt{Ich wäre bereit, dir den Topf zu überlassen, da du sagst, dass er für dich einen mittleren Wert hat und für mich weniger wichtig ist. Bei Süßigkeit und Pfeife {-} hast du eine Präferenz, welche du lieber hättest? Ich könnte mit beiden leben.} \\
\\ 
\texttt{Die zwei Würfel könnten wir uns teilen {-} jeder bekommt einen. Was denkst du über diese Aufteilung: Du bekommst den Topf, einen der wertvollen Gegenstände (Süßigkeit oder Pfeife) und einen Würfel, ich bekomme den anderen wertvollen Gegenstand und den anderen Würfel?} \\
            }
        }
    }
     \\ \\

    \theutterance \stepcounter{utterance}  
    & & & \multicolumn{2}{p{0.3\linewidth}}{
        \cellcolor[rgb]{0.9,0.9,0.9}{
            \makecell[{{p{\linewidth}}}]{
                \texttt{\tiny{[GM$|$GM]}}
                \texttt{continue} \\
            }
        }
    }
    & & \\ \\

    \theutterance \stepcounter{utterance}  
    & & & \multicolumn{2}{p{0.3\linewidth}}{
        \cellcolor[rgb]{0.9,0.9,0.9}{
            \makecell[{{p{\linewidth}}}]{
                \texttt{\tiny{[GM$|$GM]}}
                \texttt{Das klingt nach einem vernünftigen Ansatz! Da wir beide an Süßigkeit und Pfeife interessiert sind, macht eine Aufteilung Sinn.} \\
\\ 
\texttt{Ich wäre bereit, dir den Topf zu überlassen, da du sagst, dass er für dich einen mittleren Wert hat und für mich weniger wichtig ist. Bei Süßigkeit und Pfeife {-} hast du eine Präferenz, welche du lieber hättest? Ich könnte mit beiden leben.} \\
\\ 
\texttt{Die zwei Würfel könnten wir uns teilen {-} jeder bekommt einen. Was denkst du über diese Aufteilung: Du bekommst den Topf, einen der wertvollen Gegenstände (Süßigkeit oder Pfeife) und einen Würfel, ich bekomme den anderen wertvollen Gegenstand und den anderen Würfel?} \\
            }
        }
    }
    & & \\ \\

    \theutterance \stepcounter{utterance}  
    & & \multicolumn{4}{p{0.6\linewidth}}{
        \cellcolor[rgb]{0.9,0.9,0.9}{
            \makecell[{{p{\linewidth}}}]{
                \texttt{\tiny{[P1$\langle$GM]}}
                \texttt{Das klingt nach einem vernünftigen Ansatz! Da wir beide an Süßigkeit und Pfeife interessiert sind, macht eine Aufteilung Sinn.} \\
\\ 
\texttt{Ich wäre bereit, dir den Topf zu überlassen, da du sagst, dass er für dich einen mittleren Wert hat und für mich weniger wichtig ist. Bei Süßigkeit und Pfeife {-} hast du eine Präferenz, welche du lieber hättest? Ich könnte mit beiden leben.} \\
\\ 
\texttt{Die zwei Würfel könnten wir uns teilen {-} jeder bekommt einen. Was denkst du über diese Aufteilung: Du bekommst den Topf, einen der wertvollen Gegenstände (Süßigkeit oder Pfeife) und einen Würfel, ich bekomme den anderen wertvollen Gegenstand und den anderen Würfel?} \\
            }
        }
    }
    & \\ \\

    \theutterance \stepcounter{utterance}  
    & \multicolumn{4}{p{0.6\linewidth}}{
        \cellcolor[rgb]{0.9,0.9,0.9}{
            \makecell[{{p{\linewidth}}}]{
                \texttt{\tiny{[P1$\rangle$GM]}}
                \texttt{Das klingt wie eine faire Aufteilung! Wenn ich zwischen Süßigkeit und Pfeife wählen kann, würde ich gerne die Süßigkeit nehmen {-} die ist für mich etwas attraktiver.} \\
\\ 
\texttt{Also zusammengefasst: Ich würde den Topf, die Süßigkeit und einen Würfel nehmen, und du würdest die Pfeife und den anderen Würfel bekommen. Das scheint für beide von uns ein guter Deal zu sein, da wir jeweils einen der wertvollen Gegenstände bekommen und ich zusätzlich den Topf, der für dich weniger wichtig ist.} \\
\\ 
\texttt{Bist du mit dieser Aufteilung einverstanden?} \\
            }
        }
    }
    & & \\ \\

    \theutterance \stepcounter{utterance}  
    & & & \multicolumn{2}{p{0.3\linewidth}}{
        \cellcolor[rgb]{0.9,0.9,0.9}{
            \makecell[{{p{\linewidth}}}]{
                \texttt{\tiny{[GM$|$GM]}}
                \texttt{continue} \\
            }
        }
    }
    & & \\ \\

    \theutterance \stepcounter{utterance}  
    & & & \multicolumn{2}{p{0.3\linewidth}}{
        \cellcolor[rgb]{0.9,0.9,0.9}{
            \makecell[{{p{\linewidth}}}]{
                \texttt{\tiny{[GM$|$GM]}}
                \texttt{Das klingt wie eine faire Aufteilung! Wenn ich zwischen Süßigkeit und Pfeife wählen kann, würde ich gerne die Süßigkeit nehmen {-} die ist für mich etwas attraktiver.} \\
\\ 
\texttt{Also zusammengefasst: Ich würde den Topf, die Süßigkeit und einen Würfel nehmen, und du würdest die Pfeife und den anderen Würfel bekommen. Das scheint für beide von uns ein guter Deal zu sein, da wir jeweils einen der wertvollen Gegenstände bekommen und ich zusätzlich den Topf, der für dich weniger wichtig ist.} \\
\\ 
\texttt{Bist du mit dieser Aufteilung einverstanden?} \\
            }
        }
    }
    & & \\ \\

    \theutterance \stepcounter{utterance}  
    & & \multicolumn{4}{p{0.6\linewidth}}{
        \cellcolor[rgb]{0.9,0.9,0.9}{
            \makecell[{{p{\linewidth}}}]{
                \texttt{\tiny{[GM$\rangle$P2]}}
                \texttt{Das klingt wie eine faire Aufteilung! Wenn ich zwischen Süßigkeit und Pfeife wählen kann, würde ich gerne die Süßigkeit nehmen {-} die ist für mich etwas attraktiver.} \\
\\ 
\texttt{Also zusammengefasst: Ich würde den Topf, die Süßigkeit und einen Würfel nehmen, und du würdest die Pfeife und den anderen Würfel bekommen. Das scheint für beide von uns ein guter Deal zu sein, da wir jeweils einen der wertvollen Gegenstände bekommen und ich zusätzlich den Topf, der für dich weniger wichtig ist.} \\
\\ 
\texttt{Bist du mit dieser Aufteilung einverstanden?} \\
            }
        }
    }
    & \\ \\

    \theutterance \stepcounter{utterance}  
    & & & \multicolumn{4}{p{0.6\linewidth}}{
        \cellcolor[rgb]{0.9,0.9,0.9}{
            \makecell[{{p{\linewidth}}}]{
                \texttt{\tiny{[GM$\langle$P2]}}
                \texttt{Ja, das klingt perfekt! Ich bin mit dieser Aufteilung sehr zufrieden. Du bekommst den Topf, die Süßigkeit und einen Würfel {-} das sind die Gegenstände, die für dich wertvoll sind. Ich bekomme die Pfeife und einen Würfel, was für mich auch gut passt.} \\
\\ 
\texttt{Sollen wir diese Vereinbarung jetzt finalisieren?} \\
\\ 
\texttt{{[}Vorschlag: 1 Pfeife, 1 Würfel{]}} \\
            }
        }
    }
     \\ \\

    \theutterance \stepcounter{utterance}  
    & & & \multicolumn{2}{p{0.3\linewidth}}{
        \cellcolor[rgb]{0.9,0.9,0.9}{
            \makecell[{{p{\linewidth}}}]{
                \texttt{\tiny{[GM$|$GM]}}
                \texttt{proposal} \\
            }
        }
    }
    & & \\ \\

    \theutterance \stepcounter{utterance}  
    & & & \multicolumn{2}{p{0.3\linewidth}}{
        \cellcolor[rgb]{0.9,0.9,0.9}{
            \makecell[{{p{\linewidth}}}]{
                \texttt{\tiny{[GM$|$GM]}}
                [0, 0, 1, 1]
            }
        }
    }
    & & \\ \\

    \theutterance \stepcounter{utterance}  
    & & \multicolumn{4}{p{0.6\linewidth}}{
        \cellcolor[rgb]{0.9,0.9,0.9}{
            \makecell[{{p{\linewidth}}}]{
                \texttt{\tiny{[P1$\langle$GM]}}
                \texttt{Der andere Spieler hat einen geheimen Vorschlag gemacht. Jetzt bist du an der Reihe, einen eigenen Vorschlag zu machen.} \\
            }
        }
    }
    & \\ \\

    \theutterance \stepcounter{utterance}  
    & \multicolumn{4}{p{0.6\linewidth}}{
        \cellcolor[rgb]{0.9,0.9,0.9}{
            \makecell[{{p{\linewidth}}}]{
                \texttt{\tiny{[P1$\rangle$GM]}}
                \texttt{{[}Vorschlag: 1 Topf, 1 Süßigkeit, 1 Würfel{]}} \\
            }
        }
    }
    & & \\ \\

    \theutterance \stepcounter{utterance}  
    & & & \multicolumn{2}{p{0.3\linewidth}}{
        \cellcolor[rgb]{0.9,0.9,0.9}{
            \makecell[{{p{\linewidth}}}]{
                \texttt{\tiny{[GM$|$GM]}}
                \texttt{proposal} \\
            }
        }
    }
    & & \\ \\

    \theutterance \stepcounter{utterance}  
    & & & \multicolumn{2}{p{0.3\linewidth}}{
        \cellcolor[rgb]{0.9,0.9,0.9}{
            \makecell[{{p{\linewidth}}}]{
                \texttt{\tiny{[GM$|$GM]}}
                [1, 1, 0, 1]
            }
        }
    }
    & & \\ \\

    \theutterance \stepcounter{utterance}  
    & & & \multicolumn{2}{p{0.3\linewidth}}{
        \cellcolor[rgb]{0.9,0.9,0.9}{
            \makecell[{{p{\linewidth}}}]{
                \texttt{\tiny{[GM$|$GM]}}
                [[1, 1, 0, 1], [0, 0, 1, 1]]
            }
        }
    }
    & & \\ \\

\end{supertabular}
}

\end{document}
