
\documentclass{article}
\usepackage{colortbl}
\usepackage{makecell}
\usepackage{multirow}
\usepackage{supertabular}

\begin{document}

\newcounter{utterance}

\twocolumn

{ \footnotesize  \setcounter{utterance}{1}
\setlength{\tabcolsep}{0pt}
\begin{supertabular}{c@{$\;$}|p{.15\linewidth}@{}p{.15\linewidth}p{.15\linewidth}p{.15\linewidth}p{.15\linewidth}p{.15\linewidth}}

    \# & $\;$A & \multicolumn{4}{c}{Game Master} & $\;\:$B\\
    \hline 

    \theutterance \stepcounter{utterance}  

    & & \multicolumn{4}{p{0.6\linewidth}}{\cellcolor[rgb]{0.9,0.9,0.9}{%
	\makecell[{{p{\linewidth}}}]{% 
	  \tt {\tiny [A$\langle$GM]}  
	 Sie spielen ein kooperatives Verhandlungsspiel, bei dem Sie sich mit einem anderen Spieler darauf einigen müssen, wie eine Reihe von Gegenständen aufgeteilt werden soll.\\ \tt \\ \tt Die Regeln:\\ \tt (a) Sie und der andere Spieler erhalten eine Sammlung von Gegenständen. Jeder von Ihnen erhält außerdem eine geheime Wertfunktion, die angibt, wie viel Ihnen jede Art von Gegenstand wert ist.\\ \tt (b) Sie tauschen Nachrichten mit dem anderen Spieler aus, um zu vereinbaren, wer welche Gegenstände bekommt. Sie können jeweils maximal 5 Nachrichten senden oder das Spiel vorzeitig beenden, indem Sie jederzeit einen geheimen Vorschlag machen.\\ \tt (c) Jeder von euch wird aufgefordert, einen geheimen Vorschlag zu machen, in dem ihr die gewünschten Gegenstände in eckigen Klammern wie folgt angibt: "[Vorschlag: <Nummer> <Objektname>, <Nummer> <Objektname>, <...>]"\\ \tt (d) Wenn eure Vorschläge komplementär sind, d.h. es gibt genug Gegenstände, um beide Vorschläge zu erfüllen, erhält jeder Spieler eine Punktzahl, die sich aus der Summe der Werte für die Gegenstände ergibt, die er erhalten hat. Andernfalls erhalten Sie beide null Punkte.\\ \tt (e) Ihr Ziel ist es, die Summe aus Ihrer Punktzahl und der Punktzahl des anderen Spielers zu maximieren.\\ \tt \\ \tt Beginnen wir.\\ \tt \\ \tt Die Menge der verfügbaren Gegenstände ist:\\ \tt \\ \tt 1 Ring, 3 Früchte, 1 Kreide\\ \tt \\ \tt Deine geheime Wertfunktion ist:\\ \tt \\ \tt Ring: 6, Frucht: 1, Kreide: 1\\ \tt \\ \tt WICHTIG: Ihre Nachrichten werden, sofern es sich nicht um einen geheimen Vorschlag handelt, direkt an den anderen Spieler übermittelt, also fügen Sie keine Antwort auf die Regeln oder einen Text zur Ankündigung Ihrer Nachricht ein. Um einen geheimen Vorschlag zu machen, verwenden Sie das angegebene Format. Verwenden Sie keine eckigen Klammern, wenn Sie mit dem anderen Spieler kommunizieren, sonst wird dies als Ihr geheimer Vorschlag interpretiert.\\ \tt  
	  } 
	   } 
	   } 
	 & \\ 
 

    \theutterance \stepcounter{utterance}  

    & \multicolumn{4}{p{0.6\linewidth}}{\cellcolor[rgb]{0.8,1,0.9}{%
	\makecell[{{p{\linewidth}}}]{% 
	  \tt {\tiny [A$\rangle$GM]}  
	 Hallo! Lass uns über die Aufteilung der Gegenstände sprechen. Wir haben 1 Ring, 3 Früchte und 1 Kreide zu verteilen. \\ \tt \\ \tt Welche Gegenstände sind für dich am wertvollsten? Ich würde gerne verstehen, was dir wichtig ist, damit wir eine faire Aufteilung finden können, die für uns beide gut funktioniert. 
	  } 
	   } 
	   } 
	 & & \\ 
 

    \theutterance \stepcounter{utterance}  

    & & & \multicolumn{2}{p{0.3\linewidth}}{\cellcolor[rgb]{0.95,0.95,0.95}{%
	\makecell[{{p{\linewidth}}}]{% 
	  \tt {\tiny [GM$|$GM]}  
	 continue 
	  } 
	   } 
	   } 
	 & & \\ 
 

    \theutterance \stepcounter{utterance}  

    & & & \multicolumn{2}{p{0.3\linewidth}}{\cellcolor[rgb]{0.95,0.95,0.95}{%
	\makecell[{{p{\linewidth}}}]{% 
	  \tt {\tiny [GM$|$GM]}  
	 Hallo! Lass uns über die Aufteilung der Gegenstände sprechen. Wir haben 1 Ring, 3 Früchte und 1 Kreide zu verteilen. \\ \tt \\ \tt Welche Gegenstände sind für dich am wertvollsten? Ich würde gerne verstehen, was dir wichtig ist, damit wir eine faire Aufteilung finden können, die für uns beide gut funktioniert. 
	  } 
	   } 
	   } 
	 & & \\ 
 

    \theutterance \stepcounter{utterance}  

    & & \multicolumn{4}{p{0.6\linewidth}}{\cellcolor[rgb]{0.9,0.9,0.9}{%
	\makecell[{{p{\linewidth}}}]{% 
	  \tt {\tiny [GM$\rangle$B]}  
	 Sie spielen ein kooperatives Verhandlungsspiel, bei dem Sie sich mit einem anderen Spieler darauf einigen müssen, wie eine Reihe von Gegenständen aufgeteilt werden soll.\\ \tt \\ \tt Die Regeln:\\ \tt (a) Sie und der andere Spieler erhalten eine Sammlung von Gegenständen. Jeder von Ihnen erhält außerdem eine geheime Wertfunktion, die angibt, wie viel Ihnen jede Art von Gegenstand wert ist.\\ \tt (b) Sie tauschen Nachrichten mit dem anderen Spieler aus, um zu vereinbaren, wer welche Gegenstände bekommt. Sie können jeweils maximal 5 Nachrichten senden oder das Spiel vorzeitig beenden, indem Sie jederzeit einen geheimen Vorschlag machen.\\ \tt (c) Jeder von euch wird aufgefordert, einen geheimen Vorschlag zu machen, in dem ihr die gewünschten Gegenstände in eckigen Klammern wie folgt angibt: "[Vorschlag: <Nummer> <Objektname>, <Nummer> <Objektname>, <...>]"\\ \tt (d) Wenn eure Vorschläge komplementär sind, d.h. es gibt genug Gegenstände, um beide Vorschläge zu erfüllen, erhält jeder Spieler eine Punktzahl, die sich aus der Summe der Werte für die Gegenstände ergibt, die er erhalten hat. Andernfalls erhalten Sie beide null Punkte.\\ \tt (e) Ihr Ziel ist es, die Summe aus Ihrer Punktzahl und der Punktzahl des anderen Spielers zu maximieren.\\ \tt \\ \tt Beginnen wir.\\ \tt \\ \tt Die Menge der verfügbaren Gegenstände ist:\\ \tt \\ \tt 1 Ring, 3 Früchte, 1 Kreide\\ \tt \\ \tt Deine geheime Wertfunktion ist:\\ \tt \\ \tt Ring: 6, Frucht: 1, Kreide: 1\\ \tt \\ \tt WICHTIG: Ihre Nachrichten werden, sofern es sich nicht um einen geheimen Vorschlag handelt, direkt an den anderen Spieler übermittelt, also fügen Sie keine Antwort auf die Regeln oder einen Text zur Ankündigung Ihrer Nachricht ein. Um einen geheimen Vorschlag zu machen, verwenden Sie das angegebene Format. Verwenden Sie keine eckigen Klammern, wenn Sie mit dem anderen Spieler kommunizieren, sonst wird dies als Ihr geheimer Vorschlag interpretiert.\\ \tt \\ \tt \\ \tt Hallo! Lass uns über die Aufteilung der Gegenstände sprechen. Wir haben 1 Ring, 3 Früchte und 1 Kreide zu verteilen. \\ \tt \\ \tt Welche Gegenstände sind für dich am wertvollsten? Ich würde gerne verstehen, was dir wichtig ist, damit wir eine faire Aufteilung finden können, die für uns beide gut funktioniert. 
	  } 
	   } 
	   } 
	 & \\ 
 

    \theutterance \stepcounter{utterance}  

    & & & \multicolumn{4}{p{0.6\linewidth}}{\cellcolor[rgb]{1,0.85,0.72}{%
	\makecell[{{p{\linewidth}}}]{% 
	  \tt {\tiny [GM$\langle$B]}  
	 Hallo! Schön, dass wir zusammenarbeiten können. \\ \tt \\ \tt Für mich sind die Früchte am wichtigsten - ich würde gerne alle 3 Früchte bekommen, wenn das möglich ist. Der Ring und die Kreide sind für mich weniger wertvoll.\\ \tt \\ \tt Was ist denn für dich am wertvollsten? Vielleicht können wir eine Aufteilung finden, bei der jeder das bekommt, was ihm am meisten bedeutet. 
	  } 
	   } 
	   } 
	  \\ 
 

    \theutterance \stepcounter{utterance}  

    & & & \multicolumn{2}{p{0.3\linewidth}}{\cellcolor[rgb]{0.95,0.95,0.95}{%
	\makecell[{{p{\linewidth}}}]{% 
	  \tt {\tiny [GM$|$GM]}  
	 continue 
	  } 
	   } 
	   } 
	 & & \\ 
 

    \theutterance \stepcounter{utterance}  

    & & & \multicolumn{2}{p{0.3\linewidth}}{\cellcolor[rgb]{0.95,0.95,0.95}{%
	\makecell[{{p{\linewidth}}}]{% 
	  \tt {\tiny [GM$|$GM]}  
	 Hallo! Schön, dass wir zusammenarbeiten können. \\ \tt \\ \tt Für mich sind die Früchte am wichtigsten - ich würde gerne alle 3 Früchte bekommen, wenn das möglich ist. Der Ring und die Kreide sind für mich weniger wertvoll.\\ \tt \\ \tt Was ist denn für dich am wertvollsten? Vielleicht können wir eine Aufteilung finden, bei der jeder das bekommt, was ihm am meisten bedeutet. 
	  } 
	   } 
	   } 
	 & & \\ 
 

    \theutterance \stepcounter{utterance}  

    & & \multicolumn{4}{p{0.6\linewidth}}{\cellcolor[rgb]{0.9,0.9,0.9}{%
	\makecell[{{p{\linewidth}}}]{% 
	  \tt {\tiny [A$\langle$GM]}  
	 Hallo! Schön, dass wir zusammenarbeiten können. \\ \tt \\ \tt Für mich sind die Früchte am wichtigsten - ich würde gerne alle 3 Früchte bekommen, wenn das möglich ist. Der Ring und die Kreide sind für mich weniger wertvoll.\\ \tt \\ \tt Was ist denn für dich am wertvollsten? Vielleicht können wir eine Aufteilung finden, bei der jeder das bekommt, was ihm am meisten bedeutet. 
	  } 
	   } 
	   } 
	 & \\ 
 

    \theutterance \stepcounter{utterance}  

    & \multicolumn{4}{p{0.6\linewidth}}{\cellcolor[rgb]{0.8,1,0.9}{%
	\makecell[{{p{\linewidth}}}]{% 
	  \tt {\tiny [A$\rangle$GM]}  
	 Das klingt perfekt! Für mich ist tatsächlich der Ring am wertvollsten, während die Früchte für mich nicht so wichtig sind. \\ \tt \\ \tt Wie wäre es mit dieser Aufteilung: Du bekommst alle 3 Früchte (die dir am wichtigsten sind) und ich nehme den Ring (der mir am wichtigsten ist). Bei der Kreide bin ich flexibel - du kannst sie gerne auch haben, wenn du möchtest, oder ich nehme sie, ganz wie du willst.\\ \tt \\ \tt Was denkst du über diese Aufteilung? 
	  } 
	   } 
	   } 
	 & & \\ 
 

    \theutterance \stepcounter{utterance}  

    & & & \multicolumn{2}{p{0.3\linewidth}}{\cellcolor[rgb]{0.95,0.95,0.95}{%
	\makecell[{{p{\linewidth}}}]{% 
	  \tt {\tiny [GM$|$GM]}  
	 continue 
	  } 
	   } 
	   } 
	 & & \\ 
 

    \theutterance \stepcounter{utterance}  

    & & & \multicolumn{2}{p{0.3\linewidth}}{\cellcolor[rgb]{0.95,0.95,0.95}{%
	\makecell[{{p{\linewidth}}}]{% 
	  \tt {\tiny [GM$|$GM]}  
	 Das klingt perfekt! Für mich ist tatsächlich der Ring am wertvollsten, während die Früchte für mich nicht so wichtig sind. \\ \tt \\ \tt Wie wäre es mit dieser Aufteilung: Du bekommst alle 3 Früchte (die dir am wichtigsten sind) und ich nehme den Ring (der mir am wichtigsten ist). Bei der Kreide bin ich flexibel - du kannst sie gerne auch haben, wenn du möchtest, oder ich nehme sie, ganz wie du willst.\\ \tt \\ \tt Was denkst du über diese Aufteilung? 
	  } 
	   } 
	   } 
	 & & \\ 
 

    \theutterance \stepcounter{utterance}  

    & & \multicolumn{4}{p{0.6\linewidth}}{\cellcolor[rgb]{0.9,0.9,0.9}{%
	\makecell[{{p{\linewidth}}}]{% 
	  \tt {\tiny [GM$\rangle$B]}  
	 Das klingt perfekt! Für mich ist tatsächlich der Ring am wertvollsten, während die Früchte für mich nicht so wichtig sind. \\ \tt \\ \tt Wie wäre es mit dieser Aufteilung: Du bekommst alle 3 Früchte (die dir am wichtigsten sind) und ich nehme den Ring (der mir am wichtigsten ist). Bei der Kreide bin ich flexibel - du kannst sie gerne auch haben, wenn du möchtest, oder ich nehme sie, ganz wie du willst.\\ \tt \\ \tt Was denkst du über diese Aufteilung? 
	  } 
	   } 
	   } 
	 & \\ 
 

    \theutterance \stepcounter{utterance}  

    & & & \multicolumn{4}{p{0.6\linewidth}}{\cellcolor[rgb]{1,0.85,0.72}{%
	\makecell[{{p{\linewidth}}}]{% 
	  \tt {\tiny [GM$\langle$B]}  
	 Das klingt nach einer sehr guten Aufteilung! Ich bin sehr zufrieden damit, alle 3 Früchte zu bekommen, da sie mir am wichtigsten sind. Und es ist perfekt, dass du den Ring bekommst, der dir am wertvollsten ist.\\ \tt \\ \tt Was die Kreide angeht - du kannst sie gerne haben. Für mich macht das keinen großen Unterschied, und so hast du auch noch etwas zusätzlich zum Ring.\\ \tt \\ \tt Also zusammengefasst: Ich bekomme die 3 Früchte, du bekommst den Ring und die Kreide. Bist du mit dieser Aufteilung einverstanden? 
	  } 
	   } 
	   } 
	  \\ 
 

    \theutterance \stepcounter{utterance}  

    & & & \multicolumn{2}{p{0.3\linewidth}}{\cellcolor[rgb]{0.95,0.95,0.95}{%
	\makecell[{{p{\linewidth}}}]{% 
	  \tt {\tiny [GM$|$GM]}  
	 continue 
	  } 
	   } 
	   } 
	 & & \\ 
 

    \theutterance \stepcounter{utterance}  

    & & & \multicolumn{2}{p{0.3\linewidth}}{\cellcolor[rgb]{0.95,0.95,0.95}{%
	\makecell[{{p{\linewidth}}}]{% 
	  \tt {\tiny [GM$|$GM]}  
	 Das klingt nach einer sehr guten Aufteilung! Ich bin sehr zufrieden damit, alle 3 Früchte zu bekommen, da sie mir am wichtigsten sind. Und es ist perfekt, dass du den Ring bekommst, der dir am wertvollsten ist.\\ \tt \\ \tt Was die Kreide angeht - du kannst sie gerne haben. Für mich macht das keinen großen Unterschied, und so hast du auch noch etwas zusätzlich zum Ring.\\ \tt \\ \tt Also zusammengefasst: Ich bekomme die 3 Früchte, du bekommst den Ring und die Kreide. Bist du mit dieser Aufteilung einverstanden? 
	  } 
	   } 
	   } 
	 & & \\ 
 

    \theutterance \stepcounter{utterance}  

    & & \multicolumn{4}{p{0.6\linewidth}}{\cellcolor[rgb]{0.9,0.9,0.9}{%
	\makecell[{{p{\linewidth}}}]{% 
	  \tt {\tiny [A$\langle$GM]}  
	 Das klingt nach einer sehr guten Aufteilung! Ich bin sehr zufrieden damit, alle 3 Früchte zu bekommen, da sie mir am wichtigsten sind. Und es ist perfekt, dass du den Ring bekommst, der dir am wertvollsten ist.\\ \tt \\ \tt Was die Kreide angeht - du kannst sie gerne haben. Für mich macht das keinen großen Unterschied, und so hast du auch noch etwas zusätzlich zum Ring.\\ \tt \\ \tt Also zusammengefasst: Ich bekomme die 3 Früchte, du bekommst den Ring und die Kreide. Bist du mit dieser Aufteilung einverstanden? 
	  } 
	   } 
	   } 
	 & \\ 
 

    \theutterance \stepcounter{utterance}  

    & \multicolumn{4}{p{0.6\linewidth}}{\cellcolor[rgb]{0.8,1,0.9}{%
	\makecell[{{p{\linewidth}}}]{% 
	  \tt {\tiny [A$\rangle$GM]}  
	 Ja, das ist eine ausgezeichnete Aufteilung! Ich bin vollkommen einverstanden. Du bekommst alle 3 Früchte, die dir am wichtigsten sind, und ich bekomme den Ring und die Kreide. Das scheint für uns beide optimal zu sein.\\ \tt \\ \tt Sollen wir unsere geheimen Vorschläge machen?\\ \tt \\ \tt [Vorschlag: 1 Ring, 1 Kreide] 
	  } 
	   } 
	   } 
	 & & \\ 
 

    \theutterance \stepcounter{utterance}  

    & & & \multicolumn{2}{p{0.3\linewidth}}{\cellcolor[rgb]{0.95,0.95,0.95}{%
	\makecell[{{p{\linewidth}}}]{% 
	  \tt {\tiny [GM$|$GM]}  
	 proposal 
	  } 
	   } 
	   } 
	 & & \\ 
 

    \theutterance \stepcounter{utterance}  

    & & & \multicolumn{2}{p{0.3\linewidth}}{\cellcolor[rgb]{0.95,0.95,0.95}{%
	\makecell[{{p{\linewidth}}}]{% 
	  \tt {\tiny [GM$|$GM]}  
	 [1, 0, 1] 
	  } 
	   } 
	   } 
	 & & \\ 
 

    \theutterance \stepcounter{utterance}  

    & & \multicolumn{4}{p{0.6\linewidth}}{\cellcolor[rgb]{0.9,0.9,0.9}{%
	\makecell[{{p{\linewidth}}}]{% 
	  \tt {\tiny [GM$\rangle$B]}  
	 Der andere Spieler hat einen geheimen Vorschlag gemacht. Jetzt bist du an der Reihe, einen eigenen Vorschlag zu machen.\\ \tt  
	  } 
	   } 
	   } 
	 & \\ 
 

    \theutterance \stepcounter{utterance}  

    & & & \multicolumn{4}{p{0.6\linewidth}}{\cellcolor[rgb]{1,0.85,0.72}{%
	\makecell[{{p{\linewidth}}}]{% 
	  \tt {\tiny [GM$\langle$B]}  
	 [Vorschlag: 3 Frucht] 
	  } 
	   } 
	   } 
	  \\ 
 

    \theutterance \stepcounter{utterance}  

    & & & \multicolumn{2}{p{0.3\linewidth}}{\cellcolor[rgb]{0.95,0.95,0.95}{%
	\makecell[{{p{\linewidth}}}]{% 
	  \tt {\tiny [GM$|$GM]}  
	 proposal 
	  } 
	   } 
	   } 
	 & & \\ 
 

    \theutterance \stepcounter{utterance}  

    & & & \multicolumn{2}{p{0.3\linewidth}}{\cellcolor[rgb]{0.95,0.95,0.95}{%
	\makecell[{{p{\linewidth}}}]{% 
	  \tt {\tiny [GM$|$GM]}  
	 [0, 3, 0] 
	  } 
	   } 
	   } 
	 & & \\ 
 

    \theutterance \stepcounter{utterance}  

    & & & \multicolumn{2}{p{0.3\linewidth}}{\cellcolor[rgb]{0.95,0.95,0.95}{%
	\makecell[{{p{\linewidth}}}]{% 
	  \tt {\tiny [GM$|$GM]}  
	 [[1, 0, 1], [0, 3, 0]] 
	  } 
	   } 
	   } 
	 & & \\ 
 

\end{supertabular}
}

\end{document}
