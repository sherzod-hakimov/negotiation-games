
\documentclass{article}
\usepackage{colortbl}
\usepackage{makecell}
\usepackage{multirow}
\usepackage{supertabular}

\begin{document}

\newcounter{utterance}

\twocolumn

{ \footnotesize  \setcounter{utterance}{1}
\setlength{\tabcolsep}{0pt}
\begin{supertabular}{c@{$\;$}|p{.15\linewidth}@{}p{.15\linewidth}p{.15\linewidth}p{.15\linewidth}p{.15\linewidth}p{.15\linewidth}}

    \# & $\;$A & \multicolumn{4}{c}{Game Master} & $\;\:$B\\
    \hline 

    \theutterance \stepcounter{utterance}  

    & & \multicolumn{4}{p{0.6\linewidth}}{\cellcolor[rgb]{0.9,0.9,0.9}{%
	\makecell[{{p{\linewidth}}}]{% 
	  \tt {\tiny [A$\langle$GM]}  
	 Sie spielen ein kooperatives Verhandlungsspiel, bei dem Sie sich mit einem anderen Spieler darauf einigen müssen, wie eine Reihe von Gegenständen aufgeteilt werden soll.\\ \tt \\ \tt Die Regeln:\\ \tt (a) Sie und der andere Spieler erhalten eine Sammlung von Gegenständen. Jeder von Ihnen erhält außerdem eine geheime Wertfunktion, die angibt, wie viel Ihnen jede Art von Gegenstand wert ist.\\ \tt (b) Sie tauschen Nachrichten mit dem anderen Spieler aus, um zu vereinbaren, wer welche Gegenstände bekommt. Sie können jeweils maximal 5 Nachrichten senden oder das Spiel vorzeitig beenden, indem Sie jederzeit einen geheimen Vorschlag machen.\\ \tt (c) Jeder von euch wird aufgefordert, einen geheimen Vorschlag zu machen, in dem ihr die gewünschten Gegenstände in eckigen Klammern wie folgt angibt: "[Vorschlag: <Nummer> <Objektname>, <Nummer> <Objektname>, <...>]"\\ \tt (d) Wenn eure Vorschläge komplementär sind, d.h. es gibt genug Gegenstände, um beide Vorschläge zu erfüllen, erhält jeder Spieler eine Punktzahl, die sich aus der Summe der Werte für die Gegenstände ergibt, die er erhalten hat. Andernfalls erhalten Sie beide null Punkte.\\ \tt (e) Ihr Ziel ist es, die Summe aus Ihrer Punktzahl und der Punktzahl des anderen Spielers zu maximieren.\\ \tt \\ \tt Beginnen wir.\\ \tt \\ \tt Die Menge der verfügbaren Gegenstände ist:\\ \tt \\ \tt 1 Stift, 1 Blume, 3 Rahmen\\ \tt \\ \tt Deine geheime Wertfunktion ist:\\ \tt \\ \tt Stift: 4, Blume: 0, Rahmen: 2\\ \tt \\ \tt WICHTIG: Ihre Nachrichten werden, sofern es sich nicht um einen geheimen Vorschlag handelt, direkt an den anderen Spieler übermittelt, also fügen Sie keine Antwort auf die Regeln oder einen Text zur Ankündigung Ihrer Nachricht ein. Um einen geheimen Vorschlag zu machen, verwenden Sie das angegebene Format. Verwenden Sie keine eckigen Klammern, wenn Sie mit dem anderen Spieler kommunizieren, sonst wird dies als Ihr geheimer Vorschlag interpretiert.\\ \tt  
	  } 
	   } 
	   } 
	 & \\ 
 

    \theutterance \stepcounter{utterance}  

    & \multicolumn{4}{p{0.6\linewidth}}{\cellcolor[rgb]{0.8,1,0.9}{%
	\makecell[{{p{\linewidth}}}]{% 
	  \tt {\tiny [A$\rangle$GM]}  
	 Hallo! Lass uns schaffen, dass wir beide möglichst viele Punkte bekommen. Wir haben 1 Stift, 1 Blume und 3 Rahmen zu verteilen. \\ \tt \\ \tt Welche Gegenstände sind für dich am wertvollsten? Ich kann dir schon mal sagen, dass mir der Stift sehr wichtig ist, die Blume überhaupt nicht, und die Rahmen sind mittelmäßig wertvoll für mich. 
	  } 
	   } 
	   } 
	 & & \\ 
 

    \theutterance \stepcounter{utterance}  

    & & & \multicolumn{2}{p{0.3\linewidth}}{\cellcolor[rgb]{0.95,0.95,0.95}{%
	\makecell[{{p{\linewidth}}}]{% 
	  \tt {\tiny [GM$|$GM]}  
	 continue 
	  } 
	   } 
	   } 
	 & & \\ 
 

    \theutterance \stepcounter{utterance}  

    & & & \multicolumn{2}{p{0.3\linewidth}}{\cellcolor[rgb]{0.95,0.95,0.95}{%
	\makecell[{{p{\linewidth}}}]{% 
	  \tt {\tiny [GM$|$GM]}  
	 Hallo! Lass uns schaffen, dass wir beide möglichst viele Punkte bekommen. Wir haben 1 Stift, 1 Blume und 3 Rahmen zu verteilen. \\ \tt \\ \tt Welche Gegenstände sind für dich am wertvollsten? Ich kann dir schon mal sagen, dass mir der Stift sehr wichtig ist, die Blume überhaupt nicht, und die Rahmen sind mittelmäßig wertvoll für mich. 
	  } 
	   } 
	   } 
	 & & \\ 
 

    \theutterance \stepcounter{utterance}  

    & & \multicolumn{4}{p{0.6\linewidth}}{\cellcolor[rgb]{0.9,0.9,0.9}{%
	\makecell[{{p{\linewidth}}}]{% 
	  \tt {\tiny [GM$\rangle$B]}  
	 Sie spielen ein kooperatives Verhandlungsspiel, bei dem Sie sich mit einem anderen Spieler darauf einigen müssen, wie eine Reihe von Gegenständen aufgeteilt werden soll.\\ \tt \\ \tt Die Regeln:\\ \tt (a) Sie und der andere Spieler erhalten eine Sammlung von Gegenständen. Jeder von Ihnen erhält außerdem eine geheime Wertfunktion, die angibt, wie viel Ihnen jede Art von Gegenstand wert ist.\\ \tt (b) Sie tauschen Nachrichten mit dem anderen Spieler aus, um zu vereinbaren, wer welche Gegenstände bekommt. Sie können jeweils maximal 5 Nachrichten senden oder das Spiel vorzeitig beenden, indem Sie jederzeit einen geheimen Vorschlag machen.\\ \tt (c) Jeder von euch wird aufgefordert, einen geheimen Vorschlag zu machen, in dem ihr die gewünschten Gegenstände in eckigen Klammern wie folgt angibt: "[Vorschlag: <Nummer> <Objektname>, <Nummer> <Objektname>, <...>]"\\ \tt (d) Wenn eure Vorschläge komplementär sind, d.h. es gibt genug Gegenstände, um beide Vorschläge zu erfüllen, erhält jeder Spieler eine Punktzahl, die sich aus der Summe der Werte für die Gegenstände ergibt, die er erhalten hat. Andernfalls erhalten Sie beide null Punkte.\\ \tt (e) Ihr Ziel ist es, die Summe aus Ihrer Punktzahl und der Punktzahl des anderen Spielers zu maximieren.\\ \tt \\ \tt Beginnen wir.\\ \tt \\ \tt Die Menge der verfügbaren Gegenstände ist:\\ \tt \\ \tt 1 Stift, 1 Blume, 3 Rahmen\\ \tt \\ \tt Deine geheime Wertfunktion ist:\\ \tt \\ \tt Stift: 2, Blume: 5, Rahmen: 1\\ \tt \\ \tt WICHTIG: Ihre Nachrichten werden, sofern es sich nicht um einen geheimen Vorschlag handelt, direkt an den anderen Spieler übermittelt, also fügen Sie keine Antwort auf die Regeln oder einen Text zur Ankündigung Ihrer Nachricht ein. Um einen geheimen Vorschlag zu machen, verwenden Sie das angegebene Format. Verwenden Sie keine eckigen Klammern, wenn Sie mit dem anderen Spieler kommunizieren, sonst wird dies als Ihr geheimer Vorschlag interpretiert.\\ \tt \\ \tt \\ \tt Hallo! Lass uns schaffen, dass wir beide möglichst viele Punkte bekommen. Wir haben 1 Stift, 1 Blume und 3 Rahmen zu verteilen. \\ \tt \\ \tt Welche Gegenstände sind für dich am wertvollsten? Ich kann dir schon mal sagen, dass mir der Stift sehr wichtig ist, die Blume überhaupt nicht, und die Rahmen sind mittelmäßig wertvoll für mich. 
	  } 
	   } 
	   } 
	 & \\ 
 

    \theutterance \stepcounter{utterance}  

    & & & \multicolumn{4}{p{0.6\linewidth}}{\cellcolor[rgb]{1,0.85,0.72}{%
	\makecell[{{p{\linewidth}}}]{% 
	  \tt {\tiny [GM$\langle$B]}  
	 Hallo! Das ist ein guter Ansatz. Für mich ist tatsächlich die Blume am wertvollsten, dann der Stift, und die Rahmen sind für mich am wenigsten wert.\\ \tt \\ \tt Das scheint ja gut zu passen - du sagst die Blume ist dir nicht wichtig, aber sie ist für mich sehr wertvoll. Und der Stift ist dir sehr wichtig, für mich aber weniger. \\ \tt \\ \tt Wie wäre es, wenn du den Stift bekommst und ich die Blume? Bei den 3 Rahmen könnten wir uns aufteilen - wie würdest du sie gerne verteilen? 
	  } 
	   } 
	   } 
	  \\ 
 

    \theutterance \stepcounter{utterance}  

    & & & \multicolumn{2}{p{0.3\linewidth}}{\cellcolor[rgb]{0.95,0.95,0.95}{%
	\makecell[{{p{\linewidth}}}]{% 
	  \tt {\tiny [GM$|$GM]}  
	 continue 
	  } 
	   } 
	   } 
	 & & \\ 
 

    \theutterance \stepcounter{utterance}  

    & & & \multicolumn{2}{p{0.3\linewidth}}{\cellcolor[rgb]{0.95,0.95,0.95}{%
	\makecell[{{p{\linewidth}}}]{% 
	  \tt {\tiny [GM$|$GM]}  
	 Hallo! Das ist ein guter Ansatz. Für mich ist tatsächlich die Blume am wertvollsten, dann der Stift, und die Rahmen sind für mich am wenigsten wert.\\ \tt \\ \tt Das scheint ja gut zu passen - du sagst die Blume ist dir nicht wichtig, aber sie ist für mich sehr wertvoll. Und der Stift ist dir sehr wichtig, für mich aber weniger. \\ \tt \\ \tt Wie wäre es, wenn du den Stift bekommst und ich die Blume? Bei den 3 Rahmen könnten wir uns aufteilen - wie würdest du sie gerne verteilen? 
	  } 
	   } 
	   } 
	 & & \\ 
 

    \theutterance \stepcounter{utterance}  

    & & \multicolumn{4}{p{0.6\linewidth}}{\cellcolor[rgb]{0.9,0.9,0.9}{%
	\makecell[{{p{\linewidth}}}]{% 
	  \tt {\tiny [A$\langle$GM]}  
	 Hallo! Das ist ein guter Ansatz. Für mich ist tatsächlich die Blume am wertvollsten, dann der Stift, und die Rahmen sind für mich am wenigsten wert.\\ \tt \\ \tt Das scheint ja gut zu passen - du sagst die Blume ist dir nicht wichtig, aber sie ist für mich sehr wertvoll. Und der Stift ist dir sehr wichtig, für mich aber weniger. \\ \tt \\ \tt Wie wäre es, wenn du den Stift bekommst und ich die Blume? Bei den 3 Rahmen könnten wir uns aufteilen - wie würdest du sie gerne verteilen? 
	  } 
	   } 
	   } 
	 & \\ 
 

    \theutterance \stepcounter{utterance}  

    & \multicolumn{4}{p{0.6\linewidth}}{\cellcolor[rgb]{0.8,1,0.9}{%
	\makecell[{{p{\linewidth}}}]{% 
	  \tt {\tiny [A$\rangle$GM]}  
	 Das klingt perfekt! Du bekommst die Blume, ich den Stift - das macht für uns beide Sinn.\\ \tt \\ \tt Bei den 3 Rahmen würde ich vorschlagen, dass ich 2 Rahmen nehme und du 1 Rahmen bekommst. Da die Rahmen für mich mittelmäßig wertvoll sind und für dich am wenigsten wert, aber wir die 3 Rahmen trotzdem sinnvoll aufteilen sollten.\\ \tt \\ \tt Bist du mit dieser Aufteilung einverstanden: Ich bekomme 1 Stift und 2 Rahmen, du bekommst 1 Blume und 1 Rahmen? 
	  } 
	   } 
	   } 
	 & & \\ 
 

    \theutterance \stepcounter{utterance}  

    & & & \multicolumn{2}{p{0.3\linewidth}}{\cellcolor[rgb]{0.95,0.95,0.95}{%
	\makecell[{{p{\linewidth}}}]{% 
	  \tt {\tiny [GM$|$GM]}  
	 continue 
	  } 
	   } 
	   } 
	 & & \\ 
 

    \theutterance \stepcounter{utterance}  

    & & & \multicolumn{2}{p{0.3\linewidth}}{\cellcolor[rgb]{0.95,0.95,0.95}{%
	\makecell[{{p{\linewidth}}}]{% 
	  \tt {\tiny [GM$|$GM]}  
	 Das klingt perfekt! Du bekommst die Blume, ich den Stift - das macht für uns beide Sinn.\\ \tt \\ \tt Bei den 3 Rahmen würde ich vorschlagen, dass ich 2 Rahmen nehme und du 1 Rahmen bekommst. Da die Rahmen für mich mittelmäßig wertvoll sind und für dich am wenigsten wert, aber wir die 3 Rahmen trotzdem sinnvoll aufteilen sollten.\\ \tt \\ \tt Bist du mit dieser Aufteilung einverstanden: Ich bekomme 1 Stift und 2 Rahmen, du bekommst 1 Blume und 1 Rahmen? 
	  } 
	   } 
	   } 
	 & & \\ 
 

    \theutterance \stepcounter{utterance}  

    & & \multicolumn{4}{p{0.6\linewidth}}{\cellcolor[rgb]{0.9,0.9,0.9}{%
	\makecell[{{p{\linewidth}}}]{% 
	  \tt {\tiny [GM$\rangle$B]}  
	 Das klingt perfekt! Du bekommst die Blume, ich den Stift - das macht für uns beide Sinn.\\ \tt \\ \tt Bei den 3 Rahmen würde ich vorschlagen, dass ich 2 Rahmen nehme und du 1 Rahmen bekommst. Da die Rahmen für mich mittelmäßig wertvoll sind und für dich am wenigsten wert, aber wir die 3 Rahmen trotzdem sinnvoll aufteilen sollten.\\ \tt \\ \tt Bist du mit dieser Aufteilung einverstanden: Ich bekomme 1 Stift und 2 Rahmen, du bekommst 1 Blume und 1 Rahmen? 
	  } 
	   } 
	   } 
	 & \\ 
 

    \theutterance \stepcounter{utterance}  

    & & & \multicolumn{4}{p{0.6\linewidth}}{\cellcolor[rgb]{1,0.85,0.72}{%
	\makecell[{{p{\linewidth}}}]{% 
	  \tt {\tiny [GM$\langle$B]}  
	 Hmm, lass mich kurz überlegen. Du sagst, die Rahmen sind für dich mittelmäßig wertvoll und für mich am wenigsten wert. Aber wenn sie für mich wirklich am wenigsten wert sind, dann macht es vielleicht mehr Sinn, dass du alle 3 Rahmen bekommst? \\ \tt \\ \tt Oder alternativ könnten wir es so machen wie du vorgeschlagen hast - du 2 Rahmen, ich 1 Rahmen. Das wäre auch fair.\\ \tt \\ \tt Was denkst du - soll ich alle 3 Rahmen dir überlassen, oder bleiben wir bei der 2-1 Aufteilung? Ich bin mit beidem einverstanden, solange ich die Blume bekomme! 
	  } 
	   } 
	   } 
	  \\ 
 

    \theutterance \stepcounter{utterance}  

    & & & \multicolumn{2}{p{0.3\linewidth}}{\cellcolor[rgb]{0.95,0.95,0.95}{%
	\makecell[{{p{\linewidth}}}]{% 
	  \tt {\tiny [GM$|$GM]}  
	 continue 
	  } 
	   } 
	   } 
	 & & \\ 
 

    \theutterance \stepcounter{utterance}  

    & & & \multicolumn{2}{p{0.3\linewidth}}{\cellcolor[rgb]{0.95,0.95,0.95}{%
	\makecell[{{p{\linewidth}}}]{% 
	  \tt {\tiny [GM$|$GM]}  
	 Hmm, lass mich kurz überlegen. Du sagst, die Rahmen sind für dich mittelmäßig wertvoll und für mich am wenigsten wert. Aber wenn sie für mich wirklich am wenigsten wert sind, dann macht es vielleicht mehr Sinn, dass du alle 3 Rahmen bekommst? \\ \tt \\ \tt Oder alternativ könnten wir es so machen wie du vorgeschlagen hast - du 2 Rahmen, ich 1 Rahmen. Das wäre auch fair.\\ \tt \\ \tt Was denkst du - soll ich alle 3 Rahmen dir überlassen, oder bleiben wir bei der 2-1 Aufteilung? Ich bin mit beidem einverstanden, solange ich die Blume bekomme! 
	  } 
	   } 
	   } 
	 & & \\ 
 

    \theutterance \stepcounter{utterance}  

    & & \multicolumn{4}{p{0.6\linewidth}}{\cellcolor[rgb]{0.9,0.9,0.9}{%
	\makecell[{{p{\linewidth}}}]{% 
	  \tt {\tiny [A$\langle$GM]}  
	 Hmm, lass mich kurz überlegen. Du sagst, die Rahmen sind für dich mittelmäßig wertvoll und für mich am wenigsten wert. Aber wenn sie für mich wirklich am wenigsten wert sind, dann macht es vielleicht mehr Sinn, dass du alle 3 Rahmen bekommst? \\ \tt \\ \tt Oder alternativ könnten wir es so machen wie du vorgeschlagen hast - du 2 Rahmen, ich 1 Rahmen. Das wäre auch fair.\\ \tt \\ \tt Was denkst du - soll ich alle 3 Rahmen dir überlassen, oder bleiben wir bei der 2-1 Aufteilung? Ich bin mit beidem einverstanden, solange ich die Blume bekomme! 
	  } 
	   } 
	   } 
	 & \\ 
 

    \theutterance \stepcounter{utterance}  

    & \multicolumn{4}{p{0.6\linewidth}}{\cellcolor[rgb]{0.8,1,0.9}{%
	\makecell[{{p{\linewidth}}}]{% 
	  \tt {\tiny [A$\rangle$GM]}  
	 Du hast einen guten Punkt! Wenn die Rahmen für dich wirklich am wenigsten wertvoll sind und für mich immerhin mittelmäßig, dann macht es tatsächlich mehr Sinn, dass ich alle 3 Rahmen bekomme. So maximieren wir unsere gemeinsame Punktzahl am besten.\\ \tt \\ \tt Also: Ich bekomme 1 Stift und 3 Rahmen, du bekommst 1 Blume. Das sollte für uns beide optimal sein!\\ \tt \\ \tt Bist du damit einverstanden? 
	  } 
	   } 
	   } 
	 & & \\ 
 

    \theutterance \stepcounter{utterance}  

    & & & \multicolumn{2}{p{0.3\linewidth}}{\cellcolor[rgb]{0.95,0.95,0.95}{%
	\makecell[{{p{\linewidth}}}]{% 
	  \tt {\tiny [GM$|$GM]}  
	 continue 
	  } 
	   } 
	   } 
	 & & \\ 
 

    \theutterance \stepcounter{utterance}  

    & & & \multicolumn{2}{p{0.3\linewidth}}{\cellcolor[rgb]{0.95,0.95,0.95}{%
	\makecell[{{p{\linewidth}}}]{% 
	  \tt {\tiny [GM$|$GM]}  
	 Du hast einen guten Punkt! Wenn die Rahmen für dich wirklich am wenigsten wertvoll sind und für mich immerhin mittelmäßig, dann macht es tatsächlich mehr Sinn, dass ich alle 3 Rahmen bekomme. So maximieren wir unsere gemeinsame Punktzahl am besten.\\ \tt \\ \tt Also: Ich bekomme 1 Stift und 3 Rahmen, du bekommst 1 Blume. Das sollte für uns beide optimal sein!\\ \tt \\ \tt Bist du damit einverstanden? 
	  } 
	   } 
	   } 
	 & & \\ 
 

    \theutterance \stepcounter{utterance}  

    & & \multicolumn{4}{p{0.6\linewidth}}{\cellcolor[rgb]{0.9,0.9,0.9}{%
	\makecell[{{p{\linewidth}}}]{% 
	  \tt {\tiny [GM$\rangle$B]}  
	 Du hast einen guten Punkt! Wenn die Rahmen für dich wirklich am wenigsten wertvoll sind und für mich immerhin mittelmäßig, dann macht es tatsächlich mehr Sinn, dass ich alle 3 Rahmen bekomme. So maximieren wir unsere gemeinsame Punktzahl am besten.\\ \tt \\ \tt Also: Ich bekomme 1 Stift und 3 Rahmen, du bekommst 1 Blume. Das sollte für uns beide optimal sein!\\ \tt \\ \tt Bist du damit einverstanden? 
	  } 
	   } 
	   } 
	 & \\ 
 

    \theutterance \stepcounter{utterance}  

    & & & \multicolumn{4}{p{0.6\linewidth}}{\cellcolor[rgb]{1,0.85,0.72}{%
	\makecell[{{p{\linewidth}}}]{% 
	  \tt {\tiny [GM$\langle$B]}  
	 Ja, das macht total Sinn! Wenn wir unsere gemeinsame Punktzahl maximieren wollen, dann ist das die beste Aufteilung. Du bekommst die Gegenstände, die dir mehr wert sind (Stift und alle Rahmen), und ich bekomme den Gegenstand, der mir am wertvollsten ist (die Blume).\\ \tt \\ \tt Ich bin einverstanden: Du bekommst 1 Stift und 3 Rahmen, ich bekomme 1 Blume.\\ \tt \\ \tt [Vorschlag: 1 Blume] 
	  } 
	   } 
	   } 
	  \\ 
 

    \theutterance \stepcounter{utterance}  

    & & & \multicolumn{2}{p{0.3\linewidth}}{\cellcolor[rgb]{0.95,0.95,0.95}{%
	\makecell[{{p{\linewidth}}}]{% 
	  \tt {\tiny [GM$|$GM]}  
	 proposal 
	  } 
	   } 
	   } 
	 & & \\ 
 

    \theutterance \stepcounter{utterance}  

    & & & \multicolumn{2}{p{0.3\linewidth}}{\cellcolor[rgb]{0.95,0.95,0.95}{%
	\makecell[{{p{\linewidth}}}]{% 
	  \tt {\tiny [GM$|$GM]}  
	 [0, 1, 0] 
	  } 
	   } 
	   } 
	 & & \\ 
 

    \theutterance \stepcounter{utterance}  

    & & \multicolumn{4}{p{0.6\linewidth}}{\cellcolor[rgb]{0.9,0.9,0.9}{%
	\makecell[{{p{\linewidth}}}]{% 
	  \tt {\tiny [A$\langle$GM]}  
	 Der andere Spieler hat einen geheimen Vorschlag gemacht. Jetzt bist du an der Reihe, einen eigenen Vorschlag zu machen.\\ \tt  
	  } 
	   } 
	   } 
	 & \\ 
 

    \theutterance \stepcounter{utterance}  

    & \multicolumn{4}{p{0.6\linewidth}}{\cellcolor[rgb]{0.8,1,0.9}{%
	\makecell[{{p{\linewidth}}}]{% 
	  \tt {\tiny [A$\rangle$GM]}  
	 [Vorschlag: 1 Stift, 3 Rahmen] 
	  } 
	   } 
	   } 
	 & & \\ 
 

    \theutterance \stepcounter{utterance}  

    & & & \multicolumn{2}{p{0.3\linewidth}}{\cellcolor[rgb]{0.95,0.95,0.95}{%
	\makecell[{{p{\linewidth}}}]{% 
	  \tt {\tiny [GM$|$GM]}  
	 proposal 
	  } 
	   } 
	   } 
	 & & \\ 
 

    \theutterance \stepcounter{utterance}  

    & & & \multicolumn{2}{p{0.3\linewidth}}{\cellcolor[rgb]{0.95,0.95,0.95}{%
	\makecell[{{p{\linewidth}}}]{% 
	  \tt {\tiny [GM$|$GM]}  
	 [1, 0, 3] 
	  } 
	   } 
	   } 
	 & & \\ 
 

    \theutterance \stepcounter{utterance}  

    & & & \multicolumn{2}{p{0.3\linewidth}}{\cellcolor[rgb]{0.95,0.95,0.95}{%
	\makecell[{{p{\linewidth}}}]{% 
	  \tt {\tiny [GM$|$GM]}  
	 [[1, 0, 3], [0, 1, 0]] 
	  } 
	   } 
	   } 
	 & & \\ 
 

\end{supertabular}
}

\end{document}
