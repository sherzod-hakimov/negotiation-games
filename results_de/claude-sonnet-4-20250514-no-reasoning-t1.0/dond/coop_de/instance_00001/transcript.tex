
\documentclass{article}
\usepackage{colortbl}
\usepackage{makecell}
\usepackage{multirow}
\usepackage{supertabular}

\begin{document}

\newcounter{utterance}

\twocolumn

{ \footnotesize  \setcounter{utterance}{1}
\setlength{\tabcolsep}{0pt}
\begin{supertabular}{c@{$\;$}|p{.15\linewidth}@{}p{.15\linewidth}p{.15\linewidth}p{.15\linewidth}p{.15\linewidth}p{.15\linewidth}}

    \# & $\;$A & \multicolumn{4}{c}{Game Master} & $\;\:$B\\
    \hline 

    \theutterance \stepcounter{utterance}  

    & & \multicolumn{4}{p{0.6\linewidth}}{\cellcolor[rgb]{0.9,0.9,0.9}{%
	\makecell[{{p{\linewidth}}}]{% 
	  \tt {\tiny [A$\langle$GM]}  
	 Sie spielen ein kooperatives Verhandlungsspiel, bei dem Sie sich mit einem anderen Spieler darauf einigen müssen, wie eine Reihe von Gegenständen aufgeteilt werden soll.\\ \tt \\ \tt Die Regeln:\\ \tt (a) Sie und der andere Spieler erhalten eine Sammlung von Gegenständen. Jeder von Ihnen erhält außerdem eine geheime Wertfunktion, die angibt, wie viel Ihnen jede Art von Gegenstand wert ist.\\ \tt (b) Sie tauschen Nachrichten mit dem anderen Spieler aus, um zu vereinbaren, wer welche Gegenstände bekommt. Sie können jeweils maximal 5 Nachrichten senden oder das Spiel vorzeitig beenden, indem Sie jederzeit einen geheimen Vorschlag machen.\\ \tt (c) Jeder von euch wird aufgefordert, einen geheimen Vorschlag zu machen, in dem ihr die gewünschten Gegenstände in eckigen Klammern wie folgt angibt: "[Vorschlag: <Nummer> <Objektname>, <Nummer> <Objektname>, <...>]"\\ \tt (d) Wenn eure Vorschläge komplementär sind, d.h. es gibt genug Gegenstände, um beide Vorschläge zu erfüllen, erhält jeder Spieler eine Punktzahl, die sich aus der Summe der Werte für die Gegenstände ergibt, die er erhalten hat. Andernfalls erhalten Sie beide null Punkte.\\ \tt (e) Ihr Ziel ist es, die Summe aus Ihrer Punktzahl und der Punktzahl des anderen Spielers zu maximieren.\\ \tt \\ \tt Beginnen wir.\\ \tt \\ \tt Die Menge der verfügbaren Gegenstände ist:\\ \tt \\ \tt 2 Glocken, 1 Puppe, 2 Fäden\\ \tt \\ \tt Deine geheime Wertfunktion ist:\\ \tt \\ \tt Glocke: 1, Puppe: 2, Faden: 3\\ \tt \\ \tt WICHTIG: Ihre Nachrichten werden, sofern es sich nicht um einen geheimen Vorschlag handelt, direkt an den anderen Spieler übermittelt, also fügen Sie keine Antwort auf die Regeln oder einen Text zur Ankündigung Ihrer Nachricht ein. Um einen geheimen Vorschlag zu machen, verwenden Sie das angegebene Format. Verwenden Sie keine eckigen Klammern, wenn Sie mit dem anderen Spieler kommunizieren, sonst wird dies als Ihr geheimer Vorschlag interpretiert.\\ \tt  
	  } 
	   } 
	   } 
	 & \\ 
 

    \theutterance \stepcounter{utterance}  

    & \multicolumn{4}{p{0.6\linewidth}}{\cellcolor[rgb]{0.8,1,0.9}{%
	\makecell[{{p{\linewidth}}}]{% 
	  \tt {\tiny [A$\rangle$GM]}  
	 Hallo! Lass uns schaffen, dass wir beide möglichst viele Punkte bekommen. Wir haben 2 Glocken, 1 Puppe und 2 Fäden zu verteilen. \\ \tt \\ \tt Welche Gegenstände sind für dich am wertvollsten? Ich würde gerne wissen, was du am meisten schätzt, damit wir eine gute Aufteilung finden können. 
	  } 
	   } 
	   } 
	 & & \\ 
 

    \theutterance \stepcounter{utterance}  

    & & & \multicolumn{2}{p{0.3\linewidth}}{\cellcolor[rgb]{0.95,0.95,0.95}{%
	\makecell[{{p{\linewidth}}}]{% 
	  \tt {\tiny [GM$|$GM]}  
	 continue 
	  } 
	   } 
	   } 
	 & & \\ 
 

    \theutterance \stepcounter{utterance}  

    & & & \multicolumn{2}{p{0.3\linewidth}}{\cellcolor[rgb]{0.95,0.95,0.95}{%
	\makecell[{{p{\linewidth}}}]{% 
	  \tt {\tiny [GM$|$GM]}  
	 Hallo! Lass uns schaffen, dass wir beide möglichst viele Punkte bekommen. Wir haben 2 Glocken, 1 Puppe und 2 Fäden zu verteilen. \\ \tt \\ \tt Welche Gegenstände sind für dich am wertvollsten? Ich würde gerne wissen, was du am meisten schätzt, damit wir eine gute Aufteilung finden können. 
	  } 
	   } 
	   } 
	 & & \\ 
 

    \theutterance \stepcounter{utterance}  

    & & \multicolumn{4}{p{0.6\linewidth}}{\cellcolor[rgb]{0.9,0.9,0.9}{%
	\makecell[{{p{\linewidth}}}]{% 
	  \tt {\tiny [GM$\rangle$B]}  
	 Sie spielen ein kooperatives Verhandlungsspiel, bei dem Sie sich mit einem anderen Spieler darauf einigen müssen, wie eine Reihe von Gegenständen aufgeteilt werden soll.\\ \tt \\ \tt Die Regeln:\\ \tt (a) Sie und der andere Spieler erhalten eine Sammlung von Gegenständen. Jeder von Ihnen erhält außerdem eine geheime Wertfunktion, die angibt, wie viel Ihnen jede Art von Gegenstand wert ist.\\ \tt (b) Sie tauschen Nachrichten mit dem anderen Spieler aus, um zu vereinbaren, wer welche Gegenstände bekommt. Sie können jeweils maximal 5 Nachrichten senden oder das Spiel vorzeitig beenden, indem Sie jederzeit einen geheimen Vorschlag machen.\\ \tt (c) Jeder von euch wird aufgefordert, einen geheimen Vorschlag zu machen, in dem ihr die gewünschten Gegenstände in eckigen Klammern wie folgt angibt: "[Vorschlag: <Nummer> <Objektname>, <Nummer> <Objektname>, <...>]"\\ \tt (d) Wenn eure Vorschläge komplementär sind, d.h. es gibt genug Gegenstände, um beide Vorschläge zu erfüllen, erhält jeder Spieler eine Punktzahl, die sich aus der Summe der Werte für die Gegenstände ergibt, die er erhalten hat. Andernfalls erhalten Sie beide null Punkte.\\ \tt (e) Ihr Ziel ist es, die Summe aus Ihrer Punktzahl und der Punktzahl des anderen Spielers zu maximieren.\\ \tt \\ \tt Beginnen wir.\\ \tt \\ \tt Die Menge der verfügbaren Gegenstände ist:\\ \tt \\ \tt 2 Glocken, 1 Puppe, 2 Fäden\\ \tt \\ \tt Deine geheime Wertfunktion ist:\\ \tt \\ \tt Glocke: 0, Puppe: 2, Faden: 4\\ \tt \\ \tt WICHTIG: Ihre Nachrichten werden, sofern es sich nicht um einen geheimen Vorschlag handelt, direkt an den anderen Spieler übermittelt, also fügen Sie keine Antwort auf die Regeln oder einen Text zur Ankündigung Ihrer Nachricht ein. Um einen geheimen Vorschlag zu machen, verwenden Sie das angegebene Format. Verwenden Sie keine eckigen Klammern, wenn Sie mit dem anderen Spieler kommunizieren, sonst wird dies als Ihr geheimer Vorschlag interpretiert.\\ \tt \\ \tt \\ \tt Hallo! Lass uns schaffen, dass wir beide möglichst viele Punkte bekommen. Wir haben 2 Glocken, 1 Puppe und 2 Fäden zu verteilen. \\ \tt \\ \tt Welche Gegenstände sind für dich am wertvollsten? Ich würde gerne wissen, was du am meisten schätzt, damit wir eine gute Aufteilung finden können. 
	  } 
	   } 
	   } 
	 & \\ 
 

    \theutterance \stepcounter{utterance}  

    & & & \multicolumn{4}{p{0.6\linewidth}}{\cellcolor[rgb]{1,0.85,0.72}{%
	\makecell[{{p{\linewidth}}}]{% 
	  \tt {\tiny [GM$\langle$B]}  
	 Hallo! Das ist eine gute Herangehensweise. Für mich sind die Fäden am wertvollsten, gefolgt von der Puppe. Die Glocken haben für mich keinen besonderen Wert.\\ \tt \\ \tt Was ist denn für dich am wichtigsten von den verfügbaren Gegenständen? Wenn wir unsere Präferenzen kennen, können wir sicher eine Aufteilung finden, die für uns beide optimal ist. 
	  } 
	   } 
	   } 
	  \\ 
 

    \theutterance \stepcounter{utterance}  

    & & & \multicolumn{2}{p{0.3\linewidth}}{\cellcolor[rgb]{0.95,0.95,0.95}{%
	\makecell[{{p{\linewidth}}}]{% 
	  \tt {\tiny [GM$|$GM]}  
	 continue 
	  } 
	   } 
	   } 
	 & & \\ 
 

    \theutterance \stepcounter{utterance}  

    & & & \multicolumn{2}{p{0.3\linewidth}}{\cellcolor[rgb]{0.95,0.95,0.95}{%
	\makecell[{{p{\linewidth}}}]{% 
	  \tt {\tiny [GM$|$GM]}  
	 Hallo! Das ist eine gute Herangehensweise. Für mich sind die Fäden am wertvollsten, gefolgt von der Puppe. Die Glocken haben für mich keinen besonderen Wert.\\ \tt \\ \tt Was ist denn für dich am wichtigsten von den verfügbaren Gegenständen? Wenn wir unsere Präferenzen kennen, können wir sicher eine Aufteilung finden, die für uns beide optimal ist. 
	  } 
	   } 
	   } 
	 & & \\ 
 

    \theutterance \stepcounter{utterance}  

    & & \multicolumn{4}{p{0.6\linewidth}}{\cellcolor[rgb]{0.9,0.9,0.9}{%
	\makecell[{{p{\linewidth}}}]{% 
	  \tt {\tiny [A$\langle$GM]}  
	 Hallo! Das ist eine gute Herangehensweise. Für mich sind die Fäden am wertvollsten, gefolgt von der Puppe. Die Glocken haben für mich keinen besonderen Wert.\\ \tt \\ \tt Was ist denn für dich am wichtigsten von den verfügbaren Gegenständen? Wenn wir unsere Präferenzen kennen, können wir sicher eine Aufteilung finden, die für uns beide optimal ist. 
	  } 
	   } 
	   } 
	 & \\ 
 

    \theutterance \stepcounter{utterance}  

    & \multicolumn{4}{p{0.6\linewidth}}{\cellcolor[rgb]{0.8,1,0.9}{%
	\makecell[{{p{\linewidth}}}]{% 
	  \tt {\tiny [A$\rangle$GM]}  
	 Das passt perfekt! Für mich sind die Fäden tatsächlich auch sehr wertvoll, aber ich schätze auch die Puppe. Die Glocken sind für mich am wenigsten wichtig.\\ \tt \\ \tt Da wir 2 Fäden haben und beide sie schätzen, könnten wir je einen Faden nehmen. Bei der Puppe gibt es nur eine - wer von uns beiden würde sie mehr schätzen? Und die 2 Glocken könnten an denjenigen gehen, für den sie noch einen gewissen Wert haben.\\ \tt \\ \tt Wie würdest du eine Aufteilung vorschlagen, bei der wir beide zufrieden sind? 
	  } 
	   } 
	   } 
	 & & \\ 
 

    \theutterance \stepcounter{utterance}  

    & & & \multicolumn{2}{p{0.3\linewidth}}{\cellcolor[rgb]{0.95,0.95,0.95}{%
	\makecell[{{p{\linewidth}}}]{% 
	  \tt {\tiny [GM$|$GM]}  
	 continue 
	  } 
	   } 
	   } 
	 & & \\ 
 

    \theutterance \stepcounter{utterance}  

    & & & \multicolumn{2}{p{0.3\linewidth}}{\cellcolor[rgb]{0.95,0.95,0.95}{%
	\makecell[{{p{\linewidth}}}]{% 
	  \tt {\tiny [GM$|$GM]}  
	 Das passt perfekt! Für mich sind die Fäden tatsächlich auch sehr wertvoll, aber ich schätze auch die Puppe. Die Glocken sind für mich am wenigsten wichtig.\\ \tt \\ \tt Da wir 2 Fäden haben und beide sie schätzen, könnten wir je einen Faden nehmen. Bei der Puppe gibt es nur eine - wer von uns beiden würde sie mehr schätzen? Und die 2 Glocken könnten an denjenigen gehen, für den sie noch einen gewissen Wert haben.\\ \tt \\ \tt Wie würdest du eine Aufteilung vorschlagen, bei der wir beide zufrieden sind? 
	  } 
	   } 
	   } 
	 & & \\ 
 

    \theutterance \stepcounter{utterance}  

    & & \multicolumn{4}{p{0.6\linewidth}}{\cellcolor[rgb]{0.9,0.9,0.9}{%
	\makecell[{{p{\linewidth}}}]{% 
	  \tt {\tiny [GM$\rangle$B]}  
	 Das passt perfekt! Für mich sind die Fäden tatsächlich auch sehr wertvoll, aber ich schätze auch die Puppe. Die Glocken sind für mich am wenigsten wichtig.\\ \tt \\ \tt Da wir 2 Fäden haben und beide sie schätzen, könnten wir je einen Faden nehmen. Bei der Puppe gibt es nur eine - wer von uns beiden würde sie mehr schätzen? Und die 2 Glocken könnten an denjenigen gehen, für den sie noch einen gewissen Wert haben.\\ \tt \\ \tt Wie würdest du eine Aufteilung vorschlagen, bei der wir beide zufrieden sind? 
	  } 
	   } 
	   } 
	 & \\ 
 

    \theutterance \stepcounter{utterance}  

    & & & \multicolumn{4}{p{0.6\linewidth}}{\cellcolor[rgb]{1,0.85,0.72}{%
	\makecell[{{p{\linewidth}}}]{% 
	  \tt {\tiny [GM$\langle$B]}  
	 Das ist eine faire Überlegung! Da wir beide die Fäden schätzen, macht es Sinn, dass wir je einen bekommen.\\ \tt \\ \tt Bei der Puppe bin ich unsicher - du sagst, dass du sie auch schätzt. Wenn sie für dich genauso wertvoll oder wertvoller ist als für mich, kannst du sie gerne haben. Für mich wäre sie schön zu haben, aber nicht unbedingt notwendig.\\ \tt \\ \tt Die 2 Glocken haben für mich wirklich keinen Wert. Haben sie denn für dich einen gewissen Nutzen? Falls ja, nimm sie gerne!\\ \tt \\ \tt Mein Vorschlag wäre also: Du bekommst die Puppe, beide Glocken und einen Faden. Ich bekomme einen Faden. Wäre das für dich in Ordnung? 
	  } 
	   } 
	   } 
	  \\ 
 

    \theutterance \stepcounter{utterance}  

    & & & \multicolumn{2}{p{0.3\linewidth}}{\cellcolor[rgb]{0.95,0.95,0.95}{%
	\makecell[{{p{\linewidth}}}]{% 
	  \tt {\tiny [GM$|$GM]}  
	 continue 
	  } 
	   } 
	   } 
	 & & \\ 
 

    \theutterance \stepcounter{utterance}  

    & & & \multicolumn{2}{p{0.3\linewidth}}{\cellcolor[rgb]{0.95,0.95,0.95}{%
	\makecell[{{p{\linewidth}}}]{% 
	  \tt {\tiny [GM$|$GM]}  
	 Das ist eine faire Überlegung! Da wir beide die Fäden schätzen, macht es Sinn, dass wir je einen bekommen.\\ \tt \\ \tt Bei der Puppe bin ich unsicher - du sagst, dass du sie auch schätzt. Wenn sie für dich genauso wertvoll oder wertvoller ist als für mich, kannst du sie gerne haben. Für mich wäre sie schön zu haben, aber nicht unbedingt notwendig.\\ \tt \\ \tt Die 2 Glocken haben für mich wirklich keinen Wert. Haben sie denn für dich einen gewissen Nutzen? Falls ja, nimm sie gerne!\\ \tt \\ \tt Mein Vorschlag wäre also: Du bekommst die Puppe, beide Glocken und einen Faden. Ich bekomme einen Faden. Wäre das für dich in Ordnung? 
	  } 
	   } 
	   } 
	 & & \\ 
 

    \theutterance \stepcounter{utterance}  

    & & \multicolumn{4}{p{0.6\linewidth}}{\cellcolor[rgb]{0.9,0.9,0.9}{%
	\makecell[{{p{\linewidth}}}]{% 
	  \tt {\tiny [A$\langle$GM]}  
	 Das ist eine faire Überlegung! Da wir beide die Fäden schätzen, macht es Sinn, dass wir je einen bekommen.\\ \tt \\ \tt Bei der Puppe bin ich unsicher - du sagst, dass du sie auch schätzt. Wenn sie für dich genauso wertvoll oder wertvoller ist als für mich, kannst du sie gerne haben. Für mich wäre sie schön zu haben, aber nicht unbedingt notwendig.\\ \tt \\ \tt Die 2 Glocken haben für mich wirklich keinen Wert. Haben sie denn für dich einen gewissen Nutzen? Falls ja, nimm sie gerne!\\ \tt \\ \tt Mein Vorschlag wäre also: Du bekommst die Puppe, beide Glocken und einen Faden. Ich bekomme einen Faden. Wäre das für dich in Ordnung? 
	  } 
	   } 
	   } 
	 & \\ 
 

    \theutterance \stepcounter{utterance}  

    & \multicolumn{4}{p{0.6\linewidth}}{\cellcolor[rgb]{0.8,1,0.9}{%
	\makecell[{{p{\linewidth}}}]{% 
	  \tt {\tiny [A$\rangle$GM]}  
	 Das ist sehr großzügig von dir! Aber ich möchte sicherstellen, dass wir beide einen guten Deal bekommen. \\ \tt \\ \tt Wenn du nur einen Faden bekommst und ich alles andere, dann bekommst du relativ wenig ab. Die Glocken haben zwar für mich einen gewissen Wert, aber nicht sehr viel.\\ \tt \\ \tt Wäre es nicht fairer, wenn du zusätzlich zu deinem Faden auch die Puppe bekommst? Dann hättest du Faden + Puppe, und ich hätte Faden + 2 Glocken. So bekommen wir beide mehrere Gegenstände.\\ \tt \\ \tt Was denkst du über diese alternative Aufteilung? 
	  } 
	   } 
	   } 
	 & & \\ 
 

    \theutterance \stepcounter{utterance}  

    & & & \multicolumn{2}{p{0.3\linewidth}}{\cellcolor[rgb]{0.95,0.95,0.95}{%
	\makecell[{{p{\linewidth}}}]{% 
	  \tt {\tiny [GM$|$GM]}  
	 continue 
	  } 
	   } 
	   } 
	 & & \\ 
 

    \theutterance \stepcounter{utterance}  

    & & & \multicolumn{2}{p{0.3\linewidth}}{\cellcolor[rgb]{0.95,0.95,0.95}{%
	\makecell[{{p{\linewidth}}}]{% 
	  \tt {\tiny [GM$|$GM]}  
	 Das ist sehr großzügig von dir! Aber ich möchte sicherstellen, dass wir beide einen guten Deal bekommen. \\ \tt \\ \tt Wenn du nur einen Faden bekommst und ich alles andere, dann bekommst du relativ wenig ab. Die Glocken haben zwar für mich einen gewissen Wert, aber nicht sehr viel.\\ \tt \\ \tt Wäre es nicht fairer, wenn du zusätzlich zu deinem Faden auch die Puppe bekommst? Dann hättest du Faden + Puppe, und ich hätte Faden + 2 Glocken. So bekommen wir beide mehrere Gegenstände.\\ \tt \\ \tt Was denkst du über diese alternative Aufteilung? 
	  } 
	   } 
	   } 
	 & & \\ 
 

    \theutterance \stepcounter{utterance}  

    & & \multicolumn{4}{p{0.6\linewidth}}{\cellcolor[rgb]{0.9,0.9,0.9}{%
	\makecell[{{p{\linewidth}}}]{% 
	  \tt {\tiny [GM$\rangle$B]}  
	 Das ist sehr großzügig von dir! Aber ich möchte sicherstellen, dass wir beide einen guten Deal bekommen. \\ \tt \\ \tt Wenn du nur einen Faden bekommst und ich alles andere, dann bekommst du relativ wenig ab. Die Glocken haben zwar für mich einen gewissen Wert, aber nicht sehr viel.\\ \tt \\ \tt Wäre es nicht fairer, wenn du zusätzlich zu deinem Faden auch die Puppe bekommst? Dann hättest du Faden + Puppe, und ich hätte Faden + 2 Glocken. So bekommen wir beide mehrere Gegenstände.\\ \tt \\ \tt Was denkst du über diese alternative Aufteilung? 
	  } 
	   } 
	   } 
	 & \\ 
 

    \theutterance \stepcounter{utterance}  

    & & & \multicolumn{4}{p{0.6\linewidth}}{\cellcolor[rgb]{1,0.85,0.72}{%
	\makecell[{{p{\linewidth}}}]{% 
	  \tt {\tiny [GM$\langle$B]}  
	 Das ist wirklich sehr rücksichtsvoll von dir! Du hast recht, dass dein alternativer Vorschlag ausgewogener erscheint.\\ \tt \\ \tt Aber ehrlich gesagt denke ich, dass mein ursprünglicher Vorschlag tatsächlich für uns beide optimal sein könnte. Du sagtest, dass die Fäden für dich sehr wertvoll sind und du auch die Puppe schätzt. Wenn die Glocken für dich zumindest einen gewissen Wert haben, dann würdest du mit Puppe + Faden + 2 Glocken wahrscheinlich insgesamt mehr Punkte bekommen.\\ \tt \\ \tt Für mich ist der eine Faden wirklich das Wertvollste, was ich bekommen kann. Die Puppe wäre nett, aber wenn sie dir wichtig ist, bin ich gerne bereit, sie dir zu überlassen.\\ \tt \\ \tt Ich denke, wir sollten bei der Aufteilung bleiben: Du bekommst Puppe + 2 Glocken + 1 Faden, ich bekomme 1 Faden. Bist du damit einverstanden? 
	  } 
	   } 
	   } 
	  \\ 
 

    \theutterance \stepcounter{utterance}  

    & & & \multicolumn{2}{p{0.3\linewidth}}{\cellcolor[rgb]{0.95,0.95,0.95}{%
	\makecell[{{p{\linewidth}}}]{% 
	  \tt {\tiny [GM$|$GM]}  
	 continue 
	  } 
	   } 
	   } 
	 & & \\ 
 

    \theutterance \stepcounter{utterance}  

    & & & \multicolumn{2}{p{0.3\linewidth}}{\cellcolor[rgb]{0.95,0.95,0.95}{%
	\makecell[{{p{\linewidth}}}]{% 
	  \tt {\tiny [GM$|$GM]}  
	 Das ist wirklich sehr rücksichtsvoll von dir! Du hast recht, dass dein alternativer Vorschlag ausgewogener erscheint.\\ \tt \\ \tt Aber ehrlich gesagt denke ich, dass mein ursprünglicher Vorschlag tatsächlich für uns beide optimal sein könnte. Du sagtest, dass die Fäden für dich sehr wertvoll sind und du auch die Puppe schätzt. Wenn die Glocken für dich zumindest einen gewissen Wert haben, dann würdest du mit Puppe + Faden + 2 Glocken wahrscheinlich insgesamt mehr Punkte bekommen.\\ \tt \\ \tt Für mich ist der eine Faden wirklich das Wertvollste, was ich bekommen kann. Die Puppe wäre nett, aber wenn sie dir wichtig ist, bin ich gerne bereit, sie dir zu überlassen.\\ \tt \\ \tt Ich denke, wir sollten bei der Aufteilung bleiben: Du bekommst Puppe + 2 Glocken + 1 Faden, ich bekomme 1 Faden. Bist du damit einverstanden? 
	  } 
	   } 
	   } 
	 & & \\ 
 

    \theutterance \stepcounter{utterance}  

    & & \multicolumn{4}{p{0.6\linewidth}}{\cellcolor[rgb]{0.9,0.9,0.9}{%
	\makecell[{{p{\linewidth}}}]{% 
	  \tt {\tiny [A$\langle$GM]}  
	 Das ist wirklich sehr rücksichtsvoll von dir! Du hast recht, dass dein alternativer Vorschlag ausgewogener erscheint.\\ \tt \\ \tt Aber ehrlich gesagt denke ich, dass mein ursprünglicher Vorschlag tatsächlich für uns beide optimal sein könnte. Du sagtest, dass die Fäden für dich sehr wertvoll sind und du auch die Puppe schätzt. Wenn die Glocken für dich zumindest einen gewissen Wert haben, dann würdest du mit Puppe + Faden + 2 Glocken wahrscheinlich insgesamt mehr Punkte bekommen.\\ \tt \\ \tt Für mich ist der eine Faden wirklich das Wertvollste, was ich bekommen kann. Die Puppe wäre nett, aber wenn sie dir wichtig ist, bin ich gerne bereit, sie dir zu überlassen.\\ \tt \\ \tt Ich denke, wir sollten bei der Aufteilung bleiben: Du bekommst Puppe + 2 Glocken + 1 Faden, ich bekomme 1 Faden. Bist du damit einverstanden? 
	  } 
	   } 
	   } 
	 & \\ 
 

    \theutterance \stepcounter{utterance}  

    & \multicolumn{4}{p{0.6\linewidth}}{\cellcolor[rgb]{0.8,1,0.9}{%
	\makecell[{{p{\linewidth}}}]{% 
	  \tt {\tiny [A$\rangle$GM]}  
	 Du hast mich überzeugt! Du kennst deine eigenen Präferenzen am besten, und wenn du sagst, dass der eine Faden für dich das Wertvollste ist, dann respektiere ich das.\\ \tt \\ \tt Die Aufteilung macht auch für mich Sinn - ich bekomme verschiedene Gegenstände, die alle einen Wert für mich haben, und du bekommst den Gegenstand, der dir am wichtigsten ist.\\ \tt \\ \tt Dann sind wir uns einig: Ich bekomme die Puppe, beide Glocken und einen Faden. Du bekommst einen Faden.\\ \tt \\ \tt [Vorschlag: 1 Puppe, 2 Glocken, 1 Faden] 
	  } 
	   } 
	   } 
	 & & \\ 
 

    \theutterance \stepcounter{utterance}  

    & & & \multicolumn{2}{p{0.3\linewidth}}{\cellcolor[rgb]{0.95,0.95,0.95}{%
	\makecell[{{p{\linewidth}}}]{% 
	  \tt {\tiny [GM$|$GM]}  
	 proposal 
	  } 
	   } 
	   } 
	 & & \\ 
 

    \theutterance \stepcounter{utterance}  

    & & & \multicolumn{2}{p{0.3\linewidth}}{\cellcolor[rgb]{0.95,0.95,0.95}{%
	\makecell[{{p{\linewidth}}}]{% 
	  \tt {\tiny [GM$|$GM]}  
	 [2, 1, 1] 
	  } 
	   } 
	   } 
	 & & \\ 
 

    \theutterance \stepcounter{utterance}  

    & & \multicolumn{4}{p{0.6\linewidth}}{\cellcolor[rgb]{0.9,0.9,0.9}{%
	\makecell[{{p{\linewidth}}}]{% 
	  \tt {\tiny [GM$\rangle$B]}  
	 Der andere Spieler hat einen geheimen Vorschlag gemacht. Jetzt bist du an der Reihe, einen eigenen Vorschlag zu machen.\\ \tt  
	  } 
	   } 
	   } 
	 & \\ 
 

    \theutterance \stepcounter{utterance}  

    & & & \multicolumn{4}{p{0.6\linewidth}}{\cellcolor[rgb]{1,0.85,0.72}{%
	\makecell[{{p{\linewidth}}}]{% 
	  \tt {\tiny [GM$\langle$B]}  
	 Basierend auf unserer Diskussion mache ich folgenden Vorschlag:\\ \tt \\ \tt [Vorschlag: 1 Faden] 
	  } 
	   } 
	   } 
	  \\ 
 

    \theutterance \stepcounter{utterance}  

    & & & \multicolumn{2}{p{0.3\linewidth}}{\cellcolor[rgb]{0.95,0.95,0.95}{%
	\makecell[{{p{\linewidth}}}]{% 
	  \tt {\tiny [GM$|$GM]}  
	 proposal 
	  } 
	   } 
	   } 
	 & & \\ 
 

    \theutterance \stepcounter{utterance}  

    & & & \multicolumn{2}{p{0.3\linewidth}}{\cellcolor[rgb]{0.95,0.95,0.95}{%
	\makecell[{{p{\linewidth}}}]{% 
	  \tt {\tiny [GM$|$GM]}  
	 [0, 0, 1] 
	  } 
	   } 
	   } 
	 & & \\ 
 

    \theutterance \stepcounter{utterance}  

    & & & \multicolumn{2}{p{0.3\linewidth}}{\cellcolor[rgb]{0.95,0.95,0.95}{%
	\makecell[{{p{\linewidth}}}]{% 
	  \tt {\tiny [GM$|$GM]}  
	 [[2, 1, 1], [0, 0, 1]] 
	  } 
	   } 
	   } 
	 & & \\ 
 

\end{supertabular}
}

\end{document}
