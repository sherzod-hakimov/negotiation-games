
\documentclass{article}
\usepackage{colortbl}
\usepackage{makecell}
\usepackage{multirow}
\usepackage{supertabular}

\begin{document}

\newcounter{utterance}

\twocolumn

{ \footnotesize  \setcounter{utterance}{1}
\setlength{\tabcolsep}{0pt}
\begin{supertabular}{c@{$\;$}|p{.15\linewidth}@{}p{.15\linewidth}p{.15\linewidth}p{.15\linewidth}p{.15\linewidth}p{.15\linewidth}}

    \# & $\;$A & \multicolumn{4}{c}{Game Master} & $\;\:$B\\
    \hline 

    \theutterance \stepcounter{utterance}  

    & & \multicolumn{4}{p{0.6\linewidth}}{\cellcolor[rgb]{0.9,0.9,0.9}{%
	\makecell[{{p{\linewidth}}}]{% 
	  \tt {\tiny [A$\langle$GM]}  
	 Sie spielen ein kooperatives Verhandlungsspiel, bei dem Sie sich mit einem anderen Spieler darauf einigen müssen, wie eine Reihe von Gegenständen aufgeteilt werden soll.\\ \tt \\ \tt Die Regeln:\\ \tt (a) Sie und der andere Spieler erhalten eine Sammlung von Gegenständen. Jeder von Ihnen erhält außerdem eine geheime Wertfunktion, die angibt, wie viel Ihnen jede Art von Gegenstand wert ist.\\ \tt (b) Sie tauschen Nachrichten mit dem anderen Spieler aus, um zu vereinbaren, wer welche Gegenstände bekommt. Sie können jeweils maximal 5 Nachrichten senden oder das Spiel vorzeitig beenden, indem Sie jederzeit einen geheimen Vorschlag machen.\\ \tt (c) Jeder von euch wird aufgefordert, einen geheimen Vorschlag zu machen, in dem ihr die gewünschten Gegenstände in eckigen Klammern wie folgt angibt: "[Vorschlag: <Nummer> <Objektname>, <Nummer> <Objektname>, <...>]"\\ \tt (d) Wenn eure Vorschläge komplementär sind, d.h. es gibt genug Gegenstände, um beide Vorschläge zu erfüllen, erhält jeder Spieler eine Punktzahl, die sich aus der Summe der Werte für die Gegenstände ergibt, die er erhalten hat. Andernfalls erhalten Sie beide null Punkte.\\ \tt (e) Ihr Ziel ist es, die Summe aus Ihrer Punktzahl und der Punktzahl des anderen Spielers zu maximieren.\\ \tt \\ \tt Beginnen wir.\\ \tt \\ \tt Die Menge der verfügbaren Gegenstände ist:\\ \tt \\ \tt 1 Eimer, 5 Bücher, 1 Spiegel\\ \tt \\ \tt Deine geheime Wertfunktion ist:\\ \tt \\ \tt Eimer: 4, Buch: 1, Spiegel: 1\\ \tt \\ \tt WICHTIG: Ihre Nachrichten werden, sofern es sich nicht um einen geheimen Vorschlag handelt, direkt an den anderen Spieler übermittelt, also fügen Sie keine Antwort auf die Regeln oder einen Text zur Ankündigung Ihrer Nachricht ein. Um einen geheimen Vorschlag zu machen, verwenden Sie das angegebene Format. Verwenden Sie keine eckigen Klammern, wenn Sie mit dem anderen Spieler kommunizieren, sonst wird dies als Ihr geheimer Vorschlag interpretiert.\\ \tt  
	  } 
	   } 
	   } 
	 & \\ 
 

    \theutterance \stepcounter{utterance}  

    & \multicolumn{4}{p{0.6\linewidth}}{\cellcolor[rgb]{0.8,1,0.9}{%
	\makecell[{{p{\linewidth}}}]{% 
	  \tt {\tiny [A$\rangle$GM]}  
	 Hallo! Lass uns überlegen, wie wir die Gegenstände am besten aufteilen können. Wir haben 1 Eimer, 5 Bücher und 1 Spiegel zu verteilen. \\ \tt \\ \tt Was sind denn deine Präferenzen? Welche Gegenstände sind für dich besonders wertvoll? 
	  } 
	   } 
	   } 
	 & & \\ 
 

    \theutterance \stepcounter{utterance}  

    & & & \multicolumn{2}{p{0.3\linewidth}}{\cellcolor[rgb]{0.95,0.95,0.95}{%
	\makecell[{{p{\linewidth}}}]{% 
	  \tt {\tiny [GM$|$GM]}  
	 continue 
	  } 
	   } 
	   } 
	 & & \\ 
 

    \theutterance \stepcounter{utterance}  

    & & & \multicolumn{2}{p{0.3\linewidth}}{\cellcolor[rgb]{0.95,0.95,0.95}{%
	\makecell[{{p{\linewidth}}}]{% 
	  \tt {\tiny [GM$|$GM]}  
	 Hallo! Lass uns überlegen, wie wir die Gegenstände am besten aufteilen können. Wir haben 1 Eimer, 5 Bücher und 1 Spiegel zu verteilen. \\ \tt \\ \tt Was sind denn deine Präferenzen? Welche Gegenstände sind für dich besonders wertvoll? 
	  } 
	   } 
	   } 
	 & & \\ 
 

    \theutterance \stepcounter{utterance}  

    & & \multicolumn{4}{p{0.6\linewidth}}{\cellcolor[rgb]{0.9,0.9,0.9}{%
	\makecell[{{p{\linewidth}}}]{% 
	  \tt {\tiny [GM$\rangle$B]}  
	 Sie spielen ein kooperatives Verhandlungsspiel, bei dem Sie sich mit einem anderen Spieler darauf einigen müssen, wie eine Reihe von Gegenständen aufgeteilt werden soll.\\ \tt \\ \tt Die Regeln:\\ \tt (a) Sie und der andere Spieler erhalten eine Sammlung von Gegenständen. Jeder von Ihnen erhält außerdem eine geheime Wertfunktion, die angibt, wie viel Ihnen jede Art von Gegenstand wert ist.\\ \tt (b) Sie tauschen Nachrichten mit dem anderen Spieler aus, um zu vereinbaren, wer welche Gegenstände bekommt. Sie können jeweils maximal 5 Nachrichten senden oder das Spiel vorzeitig beenden, indem Sie jederzeit einen geheimen Vorschlag machen.\\ \tt (c) Jeder von euch wird aufgefordert, einen geheimen Vorschlag zu machen, in dem ihr die gewünschten Gegenstände in eckigen Klammern wie folgt angibt: "[Vorschlag: <Nummer> <Objektname>, <Nummer> <Objektname>, <...>]"\\ \tt (d) Wenn eure Vorschläge komplementär sind, d.h. es gibt genug Gegenstände, um beide Vorschläge zu erfüllen, erhält jeder Spieler eine Punktzahl, die sich aus der Summe der Werte für die Gegenstände ergibt, die er erhalten hat. Andernfalls erhalten Sie beide null Punkte.\\ \tt (e) Ihr Ziel ist es, die Summe aus Ihrer Punktzahl und der Punktzahl des anderen Spielers zu maximieren.\\ \tt \\ \tt Beginnen wir.\\ \tt \\ \tt Die Menge der verfügbaren Gegenstände ist:\\ \tt \\ \tt 1 Eimer, 5 Bücher, 1 Spiegel\\ \tt \\ \tt Deine geheime Wertfunktion ist:\\ \tt \\ \tt Eimer: 2, Buch: 0, Spiegel: 8\\ \tt \\ \tt WICHTIG: Ihre Nachrichten werden, sofern es sich nicht um einen geheimen Vorschlag handelt, direkt an den anderen Spieler übermittelt, also fügen Sie keine Antwort auf die Regeln oder einen Text zur Ankündigung Ihrer Nachricht ein. Um einen geheimen Vorschlag zu machen, verwenden Sie das angegebene Format. Verwenden Sie keine eckigen Klammern, wenn Sie mit dem anderen Spieler kommunizieren, sonst wird dies als Ihr geheimer Vorschlag interpretiert.\\ \tt \\ \tt \\ \tt Hallo! Lass uns überlegen, wie wir die Gegenstände am besten aufteilen können. Wir haben 1 Eimer, 5 Bücher und 1 Spiegel zu verteilen. \\ \tt \\ \tt Was sind denn deine Präferenzen? Welche Gegenstände sind für dich besonders wertvoll? 
	  } 
	   } 
	   } 
	 & \\ 
 

    \theutterance \stepcounter{utterance}  

    & & & \multicolumn{4}{p{0.6\linewidth}}{\cellcolor[rgb]{1,0.85,0.72}{%
	\makecell[{{p{\linewidth}}}]{% 
	  \tt {\tiny [GM$\langle$B]}  
	 Hallo! Schön, dass wir zusammenarbeiten können. \\ \tt \\ \tt Für mich sind der Spiegel und der Eimer interessant - der Spiegel ist mir besonders wichtig. Bei den Büchern bin ich eher flexibel.\\ \tt \\ \tt Wie sieht es denn bei dir aus? Welche Gegenstände würdest du gerne haben? 
	  } 
	   } 
	   } 
	  \\ 
 

    \theutterance \stepcounter{utterance}  

    & & & \multicolumn{2}{p{0.3\linewidth}}{\cellcolor[rgb]{0.95,0.95,0.95}{%
	\makecell[{{p{\linewidth}}}]{% 
	  \tt {\tiny [GM$|$GM]}  
	 continue 
	  } 
	   } 
	   } 
	 & & \\ 
 

    \theutterance \stepcounter{utterance}  

    & & & \multicolumn{2}{p{0.3\linewidth}}{\cellcolor[rgb]{0.95,0.95,0.95}{%
	\makecell[{{p{\linewidth}}}]{% 
	  \tt {\tiny [GM$|$GM]}  
	 Hallo! Schön, dass wir zusammenarbeiten können. \\ \tt \\ \tt Für mich sind der Spiegel und der Eimer interessant - der Spiegel ist mir besonders wichtig. Bei den Büchern bin ich eher flexibel.\\ \tt \\ \tt Wie sieht es denn bei dir aus? Welche Gegenstände würdest du gerne haben? 
	  } 
	   } 
	   } 
	 & & \\ 
 

    \theutterance \stepcounter{utterance}  

    & & \multicolumn{4}{p{0.6\linewidth}}{\cellcolor[rgb]{0.9,0.9,0.9}{%
	\makecell[{{p{\linewidth}}}]{% 
	  \tt {\tiny [A$\langle$GM]}  
	 Hallo! Schön, dass wir zusammenarbeiten können. \\ \tt \\ \tt Für mich sind der Spiegel und der Eimer interessant - der Spiegel ist mir besonders wichtig. Bei den Büchern bin ich eher flexibel.\\ \tt \\ \tt Wie sieht es denn bei dir aus? Welche Gegenstände würdest du gerne haben? 
	  } 
	   } 
	   } 
	 & \\ 
 

    \theutterance \stepcounter{utterance}  

    & \multicolumn{4}{p{0.6\linewidth}}{\cellcolor[rgb]{0.8,1,0.9}{%
	\makecell[{{p{\linewidth}}}]{% 
	  \tt {\tiny [A$\rangle$GM]}  
	 Das passt ja gut! Der Eimer ist für mich tatsächlich sehr wertvoll, während mir der Spiegel nicht so wichtig ist. Bei den Büchern bin ich auch interessiert, aber nicht übermäßig.\\ \tt \\ \tt Wie wäre es, wenn du den Spiegel bekommst, der dir ja besonders wichtig ist, und ich nehme den Eimer? Bei den 5 Büchern könnten wir uns aufteilen - hast du eine Vorstellung, wie viele Bücher du gerne hättest? 
	  } 
	   } 
	   } 
	 & & \\ 
 

    \theutterance \stepcounter{utterance}  

    & & & \multicolumn{2}{p{0.3\linewidth}}{\cellcolor[rgb]{0.95,0.95,0.95}{%
	\makecell[{{p{\linewidth}}}]{% 
	  \tt {\tiny [GM$|$GM]}  
	 continue 
	  } 
	   } 
	   } 
	 & & \\ 
 

    \theutterance \stepcounter{utterance}  

    & & & \multicolumn{2}{p{0.3\linewidth}}{\cellcolor[rgb]{0.95,0.95,0.95}{%
	\makecell[{{p{\linewidth}}}]{% 
	  \tt {\tiny [GM$|$GM]}  
	 Das passt ja gut! Der Eimer ist für mich tatsächlich sehr wertvoll, während mir der Spiegel nicht so wichtig ist. Bei den Büchern bin ich auch interessiert, aber nicht übermäßig.\\ \tt \\ \tt Wie wäre es, wenn du den Spiegel bekommst, der dir ja besonders wichtig ist, und ich nehme den Eimer? Bei den 5 Büchern könnten wir uns aufteilen - hast du eine Vorstellung, wie viele Bücher du gerne hättest? 
	  } 
	   } 
	   } 
	 & & \\ 
 

    \theutterance \stepcounter{utterance}  

    & & \multicolumn{4}{p{0.6\linewidth}}{\cellcolor[rgb]{0.9,0.9,0.9}{%
	\makecell[{{p{\linewidth}}}]{% 
	  \tt {\tiny [GM$\rangle$B]}  
	 Das passt ja gut! Der Eimer ist für mich tatsächlich sehr wertvoll, während mir der Spiegel nicht so wichtig ist. Bei den Büchern bin ich auch interessiert, aber nicht übermäßig.\\ \tt \\ \tt Wie wäre es, wenn du den Spiegel bekommst, der dir ja besonders wichtig ist, und ich nehme den Eimer? Bei den 5 Büchern könnten wir uns aufteilen - hast du eine Vorstellung, wie viele Bücher du gerne hättest? 
	  } 
	   } 
	   } 
	 & \\ 
 

    \theutterance \stepcounter{utterance}  

    & & & \multicolumn{4}{p{0.6\linewidth}}{\cellcolor[rgb]{1,0.85,0.72}{%
	\makecell[{{p{\linewidth}}}]{% 
	  \tt {\tiny [GM$\langle$B]}  
	 Das klingt nach einem sehr guten Ausgangspunkt! Du bekommst den Eimer, ich den Spiegel - das passt perfekt zu unseren Präferenzen.\\ \tt \\ \tt Bei den Büchern bin ich wirklich flexibel, da sie für mich nicht so wertvoll sind. Du könntest gerne den Großteil oder sogar alle 5 Bücher nehmen, wenn sie dir wichtig sind. \\ \tt \\ \tt Was denkst du - würden dir alle 5 Bücher helfen, oder soll ich trotzdem ein paar nehmen, um eine ausgewogenere Aufteilung zu haben? 
	  } 
	   } 
	   } 
	  \\ 
 

    \theutterance \stepcounter{utterance}  

    & & & \multicolumn{2}{p{0.3\linewidth}}{\cellcolor[rgb]{0.95,0.95,0.95}{%
	\makecell[{{p{\linewidth}}}]{% 
	  \tt {\tiny [GM$|$GM]}  
	 continue 
	  } 
	   } 
	   } 
	 & & \\ 
 

    \theutterance \stepcounter{utterance}  

    & & & \multicolumn{2}{p{0.3\linewidth}}{\cellcolor[rgb]{0.95,0.95,0.95}{%
	\makecell[{{p{\linewidth}}}]{% 
	  \tt {\tiny [GM$|$GM]}  
	 Das klingt nach einem sehr guten Ausgangspunkt! Du bekommst den Eimer, ich den Spiegel - das passt perfekt zu unseren Präferenzen.\\ \tt \\ \tt Bei den Büchern bin ich wirklich flexibel, da sie für mich nicht so wertvoll sind. Du könntest gerne den Großteil oder sogar alle 5 Bücher nehmen, wenn sie dir wichtig sind. \\ \tt \\ \tt Was denkst du - würden dir alle 5 Bücher helfen, oder soll ich trotzdem ein paar nehmen, um eine ausgewogenere Aufteilung zu haben? 
	  } 
	   } 
	   } 
	 & & \\ 
 

    \theutterance \stepcounter{utterance}  

    & & \multicolumn{4}{p{0.6\linewidth}}{\cellcolor[rgb]{0.9,0.9,0.9}{%
	\makecell[{{p{\linewidth}}}]{% 
	  \tt {\tiny [A$\langle$GM]}  
	 Das klingt nach einem sehr guten Ausgangspunkt! Du bekommst den Eimer, ich den Spiegel - das passt perfekt zu unseren Präferenzen.\\ \tt \\ \tt Bei den Büchern bin ich wirklich flexibel, da sie für mich nicht so wertvoll sind. Du könntest gerne den Großteil oder sogar alle 5 Bücher nehmen, wenn sie dir wichtig sind. \\ \tt \\ \tt Was denkst du - würden dir alle 5 Bücher helfen, oder soll ich trotzdem ein paar nehmen, um eine ausgewogenere Aufteilung zu haben? 
	  } 
	   } 
	   } 
	 & \\ 
 

    \theutterance \stepcounter{utterance}  

    & \multicolumn{4}{p{0.6\linewidth}}{\cellcolor[rgb]{0.8,1,0.9}{%
	\makecell[{{p{\linewidth}}}]{% 
	  \tt {\tiny [A$\rangle$GM]}  
	 Das ist sehr großzügig von dir! Da du sagst, dass die Bücher für dich nicht so wertvoll sind und du flexibel bist, würde ich gerne alle 5 Bücher nehmen, wenn das für dich in Ordnung ist.\\ \tt \\ \tt So würde die Aufteilung aussehen:\\ \tt - Du: 1 Spiegel (dein wichtigster Gegenstand)\\ \tt - Ich: 1 Eimer und 5 Bücher\\ \tt \\ \tt Das scheint für uns beide eine gute Lösung zu sein, da jeder das bekommt, was ihm am wichtigsten ist. Bist du mit dieser Aufteilung einverstanden? 
	  } 
	   } 
	   } 
	 & & \\ 
 

    \theutterance \stepcounter{utterance}  

    & & & \multicolumn{2}{p{0.3\linewidth}}{\cellcolor[rgb]{0.95,0.95,0.95}{%
	\makecell[{{p{\linewidth}}}]{% 
	  \tt {\tiny [GM$|$GM]}  
	 continue 
	  } 
	   } 
	   } 
	 & & \\ 
 

    \theutterance \stepcounter{utterance}  

    & & & \multicolumn{2}{p{0.3\linewidth}}{\cellcolor[rgb]{0.95,0.95,0.95}{%
	\makecell[{{p{\linewidth}}}]{% 
	  \tt {\tiny [GM$|$GM]}  
	 Das ist sehr großzügig von dir! Da du sagst, dass die Bücher für dich nicht so wertvoll sind und du flexibel bist, würde ich gerne alle 5 Bücher nehmen, wenn das für dich in Ordnung ist.\\ \tt \\ \tt So würde die Aufteilung aussehen:\\ \tt - Du: 1 Spiegel (dein wichtigster Gegenstand)\\ \tt - Ich: 1 Eimer und 5 Bücher\\ \tt \\ \tt Das scheint für uns beide eine gute Lösung zu sein, da jeder das bekommt, was ihm am wichtigsten ist. Bist du mit dieser Aufteilung einverstanden? 
	  } 
	   } 
	   } 
	 & & \\ 
 

    \theutterance \stepcounter{utterance}  

    & & \multicolumn{4}{p{0.6\linewidth}}{\cellcolor[rgb]{0.9,0.9,0.9}{%
	\makecell[{{p{\linewidth}}}]{% 
	  \tt {\tiny [GM$\rangle$B]}  
	 Das ist sehr großzügig von dir! Da du sagst, dass die Bücher für dich nicht so wertvoll sind und du flexibel bist, würde ich gerne alle 5 Bücher nehmen, wenn das für dich in Ordnung ist.\\ \tt \\ \tt So würde die Aufteilung aussehen:\\ \tt - Du: 1 Spiegel (dein wichtigster Gegenstand)\\ \tt - Ich: 1 Eimer und 5 Bücher\\ \tt \\ \tt Das scheint für uns beide eine gute Lösung zu sein, da jeder das bekommt, was ihm am wichtigsten ist. Bist du mit dieser Aufteilung einverstanden? 
	  } 
	   } 
	   } 
	 & \\ 
 

    \theutterance \stepcounter{utterance}  

    & & & \multicolumn{4}{p{0.6\linewidth}}{\cellcolor[rgb]{1,0.85,0.72}{%
	\makecell[{{p{\linewidth}}}]{% 
	  \tt {\tiny [GM$\langle$B]}  
	 Das klingt perfekt! Ich bin vollkommen einverstanden mit dieser Aufteilung. Du bekommst den Eimer, der dir sehr wertvoll ist, plus alle 5 Bücher, und ich bekomme den Spiegel, der für mich besonders wichtig ist.\\ \tt \\ \tt Das ist eine Win-Win-Situation für uns beide!\\ \tt \\ \tt [Vorschlag: 1 Spiegel] 
	  } 
	   } 
	   } 
	  \\ 
 

    \theutterance \stepcounter{utterance}  

    & & & \multicolumn{2}{p{0.3\linewidth}}{\cellcolor[rgb]{0.95,0.95,0.95}{%
	\makecell[{{p{\linewidth}}}]{% 
	  \tt {\tiny [GM$|$GM]}  
	 proposal 
	  } 
	   } 
	   } 
	 & & \\ 
 

    \theutterance \stepcounter{utterance}  

    & & & \multicolumn{2}{p{0.3\linewidth}}{\cellcolor[rgb]{0.95,0.95,0.95}{%
	\makecell[{{p{\linewidth}}}]{% 
	  \tt {\tiny [GM$|$GM]}  
	 [0, 0, 1] 
	  } 
	   } 
	   } 
	 & & \\ 
 

    \theutterance \stepcounter{utterance}  

    & & \multicolumn{4}{p{0.6\linewidth}}{\cellcolor[rgb]{0.9,0.9,0.9}{%
	\makecell[{{p{\linewidth}}}]{% 
	  \tt {\tiny [A$\langle$GM]}  
	 Der andere Spieler hat einen geheimen Vorschlag gemacht. Jetzt bist du an der Reihe, einen eigenen Vorschlag zu machen.\\ \tt  
	  } 
	   } 
	   } 
	 & \\ 
 

    \theutterance \stepcounter{utterance}  

    & \multicolumn{4}{p{0.6\linewidth}}{\cellcolor[rgb]{0.8,1,0.9}{%
	\makecell[{{p{\linewidth}}}]{% 
	  \tt {\tiny [A$\rangle$GM]}  
	 [Vorschlag: 1 Eimer, 5 Bücher] 
	  } 
	   } 
	   } 
	 & & \\ 
 

    \theutterance \stepcounter{utterance}  

    & & & \multicolumn{2}{p{0.3\linewidth}}{\cellcolor[rgb]{0.95,0.95,0.95}{%
	\makecell[{{p{\linewidth}}}]{% 
	  \tt {\tiny [GM$|$GM]}  
	 proposal 
	  } 
	   } 
	   } 
	 & & \\ 
 

    \theutterance \stepcounter{utterance}  

    & & & \multicolumn{2}{p{0.3\linewidth}}{\cellcolor[rgb]{0.95,0.95,0.95}{%
	\makecell[{{p{\linewidth}}}]{% 
	  \tt {\tiny [GM$|$GM]}  
	 [1, 5, 0] 
	  } 
	   } 
	   } 
	 & & \\ 
 

    \theutterance \stepcounter{utterance}  

    & & & \multicolumn{2}{p{0.3\linewidth}}{\cellcolor[rgb]{0.95,0.95,0.95}{%
	\makecell[{{p{\linewidth}}}]{% 
	  \tt {\tiny [GM$|$GM]}  
	 [[1, 5, 0], [0, 0, 1]] 
	  } 
	   } 
	   } 
	 & & \\ 
 

\end{supertabular}
}

\end{document}
