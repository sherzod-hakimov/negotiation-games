
\documentclass{article}
\usepackage{colortbl}
\usepackage{makecell}
\usepackage{multirow}
\usepackage{supertabular}

\begin{document}

\newcounter{utterance}

\centering \large Interaction Transcript for game `dond', experiment `coop\_de', episode 6 with claude{-}sonnet{-}4{-}20250514{-}no{-}reasoning{-}t1.0.
\vspace{24pt}

{ \footnotesize  \setcounter{utterance}{1}
\setlength{\tabcolsep}{0pt}
\begin{supertabular}{c@{$\;$}|p{.15\linewidth}@{}p{.15\linewidth}p{.15\linewidth}p{.15\linewidth}p{.15\linewidth}p{.15\linewidth}}
    \# & $\;$A & \multicolumn{4}{c}{Game Master} & $\;\:$B\\
    \hline

    \theutterance \stepcounter{utterance}  
    & & \multicolumn{4}{p{0.6\linewidth}}{
        \cellcolor[rgb]{0.9,0.9,0.9}{
            \makecell[{{p{\linewidth}}}]{
                \texttt{\tiny{[P1$\langle$GM]}}
                \texttt{Sie spielen ein kooperatives Verhandlungsspiel, bei dem Sie sich mit einem anderen Spieler darauf einigen müssen, wie eine Reihe von Gegenständen aufgeteilt werden soll.} \\
\\ 
\texttt{Die Regeln:} \\
\texttt{(a) Sie und der andere Spieler erhalten eine Sammlung von Gegenständen. Jeder von Ihnen erhält außerdem eine geheime Wertfunktion, die angibt, wie viel Ihnen jede Art von Gegenstand wert ist.} \\
\texttt{(b) Sie tauschen Nachrichten mit dem anderen Spieler aus, um zu vereinbaren, wer welche Gegenstände bekommt. Sie können jeweils maximal 5 Nachrichten senden oder das Spiel vorzeitig beenden, indem Sie jederzeit einen geheimen Vorschlag machen.} \\
\texttt{(c) Jeder von euch wird aufgefordert, einen geheimen Vorschlag zu machen, in dem ihr die gewünschten Gegenstände in eckigen Klammern wie folgt angibt: "{[}Vorschlag: <Nummer> <Objektname>, <Nummer> <Objektname>, <...>{]}"} \\
\texttt{(d) Wenn eure Vorschläge komplementär sind, d.h. es gibt genug Gegenstände, um beide Vorschläge zu erfüllen, erhält jeder Spieler eine Punktzahl, die sich aus der Summe der Werte für die Gegenstände ergibt, die er erhalten hat. Andernfalls erhalten Sie beide null Punkte.} \\
\texttt{(e) Ihr Ziel ist es, die Summe aus Ihrer Punktzahl und der Punktzahl des anderen Spielers zu maximieren.} \\
\\ 
\texttt{Beginnen wir.} \\
\\ 
\texttt{Die Menge der verfügbaren Gegenstände ist:} \\
\\ 
\texttt{1 Kartoffel, 1 Handschuh, 2 Lineale, 1 Pfeife, 2 Karten} \\
\\ 
\texttt{Deine geheime Wertfunktion ist:} \\
\\ 
\texttt{Kartoffel: 0, Handschuh: 1, Lineal: 3, Pfeife: 1, Karte: 1} \\
\\ 
\texttt{WICHTIG: Ihre Nachrichten werden, sofern es sich nicht um einen geheimen Vorschlag handelt, direkt an den anderen Spieler übermittelt, also fügen Sie keine Antwort auf die Regeln oder einen Text zur Ankündigung Ihrer Nachricht ein. Um einen geheimen Vorschlag zu machen, verwenden Sie das angegebene Format. Verwenden Sie keine eckigen Klammern, wenn Sie mit dem anderen Spieler kommunizieren, sonst wird dies als Ihr geheimer Vorschlag interpretiert.} \\
            }
        }
    }
    & \\ \\

    \theutterance \stepcounter{utterance}  
    & \multicolumn{4}{p{0.6\linewidth}}{
        \cellcolor[rgb]{0.9,0.9,0.9}{
            \makecell[{{p{\linewidth}}}]{
                \texttt{\tiny{[P1$\rangle$GM]}}
                \texttt{Hallo! Lass uns schauen, wie wir die Gegenstände am besten aufteilen können. Wir haben 1 Kartoffel, 1 Handschuh, 2 Lineale, 1 Pfeife und 2 Karten zur Verfügung. } \\
\\ 
\texttt{Welche Gegenstände sind für dich besonders wertvoll? Für mich sind die Lineale am wichtigsten, gefolgt von Handschuh, Pfeife und Karten. Die Kartoffel ist für mich nicht so relevant.} \\
            }
        }
    }
    & & \\ \\

    \theutterance \stepcounter{utterance}  
    & & & \multicolumn{2}{p{0.3\linewidth}}{
        \cellcolor[rgb]{0.9,0.9,0.9}{
            \makecell[{{p{\linewidth}}}]{
                \texttt{\tiny{[GM$|$GM]}}
                \texttt{continue} \\
            }
        }
    }
    & & \\ \\

    \theutterance \stepcounter{utterance}  
    & & & \multicolumn{2}{p{0.3\linewidth}}{
        \cellcolor[rgb]{0.9,0.9,0.9}{
            \makecell[{{p{\linewidth}}}]{
                \texttt{\tiny{[GM$|$GM]}}
                \texttt{Hallo! Lass uns schauen, wie wir die Gegenstände am besten aufteilen können. Wir haben 1 Kartoffel, 1 Handschuh, 2 Lineale, 1 Pfeife und 2 Karten zur Verfügung. } \\
\\ 
\texttt{Welche Gegenstände sind für dich besonders wertvoll? Für mich sind die Lineale am wichtigsten, gefolgt von Handschuh, Pfeife und Karten. Die Kartoffel ist für mich nicht so relevant.} \\
            }
        }
    }
    & & \\ \\

    \theutterance \stepcounter{utterance}  
    & & \multicolumn{4}{p{0.6\linewidth}}{
        \cellcolor[rgb]{0.9,0.9,0.9}{
            \makecell[{{p{\linewidth}}}]{
                \texttt{\tiny{[GM$\rangle$P2]}}
                \texttt{Sie spielen ein kooperatives Verhandlungsspiel, bei dem Sie sich mit einem anderen Spieler darauf einigen müssen, wie eine Reihe von Gegenständen aufgeteilt werden soll.} \\
\\ 
\texttt{Die Regeln:} \\
\texttt{(a) Sie und der andere Spieler erhalten eine Sammlung von Gegenständen. Jeder von Ihnen erhält außerdem eine geheime Wertfunktion, die angibt, wie viel Ihnen jede Art von Gegenstand wert ist.} \\
\texttt{(b) Sie tauschen Nachrichten mit dem anderen Spieler aus, um zu vereinbaren, wer welche Gegenstände bekommt. Sie können jeweils maximal 5 Nachrichten senden oder das Spiel vorzeitig beenden, indem Sie jederzeit einen geheimen Vorschlag machen.} \\
\texttt{(c) Jeder von euch wird aufgefordert, einen geheimen Vorschlag zu machen, in dem ihr die gewünschten Gegenstände in eckigen Klammern wie folgt angibt: "{[}Vorschlag: <Nummer> <Objektname>, <Nummer> <Objektname>, <...>{]}"} \\
\texttt{(d) Wenn eure Vorschläge komplementär sind, d.h. es gibt genug Gegenstände, um beide Vorschläge zu erfüllen, erhält jeder Spieler eine Punktzahl, die sich aus der Summe der Werte für die Gegenstände ergibt, die er erhalten hat. Andernfalls erhalten Sie beide null Punkte.} \\
\texttt{(e) Ihr Ziel ist es, die Summe aus Ihrer Punktzahl und der Punktzahl des anderen Spielers zu maximieren.} \\
\\ 
\texttt{Beginnen wir.} \\
\\ 
\texttt{Die Menge der verfügbaren Gegenstände ist:} \\
\\ 
\texttt{1 Kartoffel, 1 Handschuh, 2 Lineale, 1 Pfeife, 2 Karten} \\
\\ 
\texttt{Deine geheime Wertfunktion ist:} \\
\\ 
\texttt{Kartoffel: 7, Handschuh: 0, Lineal: 0, Pfeife: 3, Karte: 0} \\
\\ 
\texttt{WICHTIG: Ihre Nachrichten werden, sofern es sich nicht um einen geheimen Vorschlag handelt, direkt an den anderen Spieler übermittelt, also fügen Sie keine Antwort auf die Regeln oder einen Text zur Ankündigung Ihrer Nachricht ein. Um einen geheimen Vorschlag zu machen, verwenden Sie das angegebene Format. Verwenden Sie keine eckigen Klammern, wenn Sie mit dem anderen Spieler kommunizieren, sonst wird dies als Ihr geheimer Vorschlag interpretiert.} \\
\\ 
\\ 
\texttt{Hallo! Lass uns schauen, wie wir die Gegenstände am besten aufteilen können. Wir haben 1 Kartoffel, 1 Handschuh, 2 Lineale, 1 Pfeife und 2 Karten zur Verfügung. } \\
\\ 
\texttt{Welche Gegenstände sind für dich besonders wertvoll? Für mich sind die Lineale am wichtigsten, gefolgt von Handschuh, Pfeife und Karten. Die Kartoffel ist für mich nicht so relevant.} \\
            }
        }
    }
    & \\ \\

    \theutterance \stepcounter{utterance}  
    & & & \multicolumn{4}{p{0.6\linewidth}}{
        \cellcolor[rgb]{0.9,0.9,0.9}{
            \makecell[{{p{\linewidth}}}]{
                \texttt{\tiny{[GM$\langle$P2]}}
                \texttt{Hallo! Danke für den Überblick. Interessant {-} für mich ist die Situation tatsächlich ganz anders. Die Kartoffel ist für mich sehr wertvoll und die Pfeife hat auch einen gewissen Wert für mich. Die anderen Gegenstände wie Handschuh, Lineale und Karten sind für mich weniger wichtig.} \\
\\ 
\texttt{Es sieht so aus, als hätten wir komplementäre Präferenzen! Wenn die Lineale für dich am wichtigsten sind und die Kartoffel für mich, könnten wir gut tauschen. Wie wäre es, wenn ich die Kartoffel und die Pfeife nehme, und du bekommst die beiden Lineale? Was denkst du über Handschuh und Karten?} \\
            }
        }
    }
     \\ \\

    \theutterance \stepcounter{utterance}  
    & & & \multicolumn{2}{p{0.3\linewidth}}{
        \cellcolor[rgb]{0.9,0.9,0.9}{
            \makecell[{{p{\linewidth}}}]{
                \texttt{\tiny{[GM$|$GM]}}
                \texttt{continue} \\
            }
        }
    }
    & & \\ \\

    \theutterance \stepcounter{utterance}  
    & & & \multicolumn{2}{p{0.3\linewidth}}{
        \cellcolor[rgb]{0.9,0.9,0.9}{
            \makecell[{{p{\linewidth}}}]{
                \texttt{\tiny{[GM$|$GM]}}
                \texttt{Hallo! Danke für den Überblick. Interessant {-} für mich ist die Situation tatsächlich ganz anders. Die Kartoffel ist für mich sehr wertvoll und die Pfeife hat auch einen gewissen Wert für mich. Die anderen Gegenstände wie Handschuh, Lineale und Karten sind für mich weniger wichtig.} \\
\\ 
\texttt{Es sieht so aus, als hätten wir komplementäre Präferenzen! Wenn die Lineale für dich am wichtigsten sind und die Kartoffel für mich, könnten wir gut tauschen. Wie wäre es, wenn ich die Kartoffel und die Pfeife nehme, und du bekommst die beiden Lineale? Was denkst du über Handschuh und Karten?} \\
            }
        }
    }
    & & \\ \\

    \theutterance \stepcounter{utterance}  
    & & \multicolumn{4}{p{0.6\linewidth}}{
        \cellcolor[rgb]{0.9,0.9,0.9}{
            \makecell[{{p{\linewidth}}}]{
                \texttt{\tiny{[P1$\langle$GM]}}
                \texttt{Hallo! Danke für den Überblick. Interessant {-} für mich ist die Situation tatsächlich ganz anders. Die Kartoffel ist für mich sehr wertvoll und die Pfeife hat auch einen gewissen Wert für mich. Die anderen Gegenstände wie Handschuh, Lineale und Karten sind für mich weniger wichtig.} \\
\\ 
\texttt{Es sieht so aus, als hätten wir komplementäre Präferenzen! Wenn die Lineale für dich am wichtigsten sind und die Kartoffel für mich, könnten wir gut tauschen. Wie wäre es, wenn ich die Kartoffel und die Pfeife nehme, und du bekommst die beiden Lineale? Was denkst du über Handschuh und Karten?} \\
            }
        }
    }
    & \\ \\

    \theutterance \stepcounter{utterance}  
    & \multicolumn{4}{p{0.6\linewidth}}{
        \cellcolor[rgb]{0.9,0.9,0.9}{
            \makecell[{{p{\linewidth}}}]{
                \texttt{\tiny{[P1$\rangle$GM]}}
                \texttt{Das klingt nach einem sehr guten Start! Unsere Präferenzen passen wirklich gut zusammen. Wenn du die Kartoffel und Pfeife nimmst und ich die beiden Lineale bekomme, sind wir schon mal bei den wichtigsten Gegenständen einig.} \\
\\ 
\texttt{Bei Handschuh und Karten bin ich flexibel {-} beide haben für mich den gleichen Wert. Da du sagst, dass diese für dich weniger wichtig sind, könnten wir sie einfach aufteilen. Ich könnte den Handschuh nehmen und du bekommst die 2 Karten dazu? Oder hast du eine andere Präferenz bei diesen verbleibenden Gegenständen?} \\
\\ 
\texttt{Das würde bedeuten: Du {-} Kartoffel, Pfeife, 2 Karten und ich {-} 2 Lineale, Handschuh. Was denkst du?} \\
            }
        }
    }
    & & \\ \\

    \theutterance \stepcounter{utterance}  
    & & & \multicolumn{2}{p{0.3\linewidth}}{
        \cellcolor[rgb]{0.9,0.9,0.9}{
            \makecell[{{p{\linewidth}}}]{
                \texttt{\tiny{[GM$|$GM]}}
                \texttt{continue} \\
            }
        }
    }
    & & \\ \\

    \theutterance \stepcounter{utterance}  
    & & & \multicolumn{2}{p{0.3\linewidth}}{
        \cellcolor[rgb]{0.9,0.9,0.9}{
            \makecell[{{p{\linewidth}}}]{
                \texttt{\tiny{[GM$|$GM]}}
                \texttt{Das klingt nach einem sehr guten Start! Unsere Präferenzen passen wirklich gut zusammen. Wenn du die Kartoffel und Pfeife nimmst und ich die beiden Lineale bekomme, sind wir schon mal bei den wichtigsten Gegenständen einig.} \\
\\ 
\texttt{Bei Handschuh und Karten bin ich flexibel {-} beide haben für mich den gleichen Wert. Da du sagst, dass diese für dich weniger wichtig sind, könnten wir sie einfach aufteilen. Ich könnte den Handschuh nehmen und du bekommst die 2 Karten dazu? Oder hast du eine andere Präferenz bei diesen verbleibenden Gegenständen?} \\
\\ 
\texttt{Das würde bedeuten: Du {-} Kartoffel, Pfeife, 2 Karten und ich {-} 2 Lineale, Handschuh. Was denkst du?} \\
            }
        }
    }
    & & \\ \\

    \theutterance \stepcounter{utterance}  
    & & \multicolumn{4}{p{0.6\linewidth}}{
        \cellcolor[rgb]{0.9,0.9,0.9}{
            \makecell[{{p{\linewidth}}}]{
                \texttt{\tiny{[GM$\rangle$P2]}}
                \texttt{Das klingt nach einem sehr guten Start! Unsere Präferenzen passen wirklich gut zusammen. Wenn du die Kartoffel und Pfeife nimmst und ich die beiden Lineale bekomme, sind wir schon mal bei den wichtigsten Gegenständen einig.} \\
\\ 
\texttt{Bei Handschuh und Karten bin ich flexibel {-} beide haben für mich den gleichen Wert. Da du sagst, dass diese für dich weniger wichtig sind, könnten wir sie einfach aufteilen. Ich könnte den Handschuh nehmen und du bekommst die 2 Karten dazu? Oder hast du eine andere Präferenz bei diesen verbleibenden Gegenständen?} \\
\\ 
\texttt{Das würde bedeuten: Du {-} Kartoffel, Pfeife, 2 Karten und ich {-} 2 Lineale, Handschuh. Was denkst du?} \\
            }
        }
    }
    & \\ \\

    \theutterance \stepcounter{utterance}  
    & & & \multicolumn{4}{p{0.6\linewidth}}{
        \cellcolor[rgb]{0.9,0.9,0.9}{
            \makecell[{{p{\linewidth}}}]{
                \texttt{\tiny{[GM$\langle$P2]}}
                \texttt{Das klingt fair und durchdacht! Da Handschuh und Karten für mich beide gleich wenig wertvoll sind, ist mir die Aufteilung egal. Dein Vorschlag macht Sinn {-} du bekommst die Lineale die dir wichtig sind plus den Handschuh, und ich bekomme meine wichtige Kartoffel und Pfeife plus die zwei Karten.} \\
\\ 
\texttt{Ich bin mit dieser Aufteilung einverstanden:} \\
\texttt{{-} Du: 2 Lineale, 1 Handschuh  } \\
\texttt{{-} Ich: 1 Kartoffel, 1 Pfeife, 2 Karten} \\
\\ 
\texttt{Das scheint eine Win{-}Win Lösung zu sein, da wir beide unsere wertvollsten Gegenstände bekommen. Sollen wir das so festmachen?} \\
            }
        }
    }
     \\ \\

    \theutterance \stepcounter{utterance}  
    & & & \multicolumn{2}{p{0.3\linewidth}}{
        \cellcolor[rgb]{0.9,0.9,0.9}{
            \makecell[{{p{\linewidth}}}]{
                \texttt{\tiny{[GM$|$GM]}}
                \texttt{continue} \\
            }
        }
    }
    & & \\ \\

    \theutterance \stepcounter{utterance}  
    & & & \multicolumn{2}{p{0.3\linewidth}}{
        \cellcolor[rgb]{0.9,0.9,0.9}{
            \makecell[{{p{\linewidth}}}]{
                \texttt{\tiny{[GM$|$GM]}}
                \texttt{Das klingt fair und durchdacht! Da Handschuh und Karten für mich beide gleich wenig wertvoll sind, ist mir die Aufteilung egal. Dein Vorschlag macht Sinn {-} du bekommst die Lineale die dir wichtig sind plus den Handschuh, und ich bekomme meine wichtige Kartoffel und Pfeife plus die zwei Karten.} \\
\\ 
\texttt{Ich bin mit dieser Aufteilung einverstanden:} \\
\texttt{{-} Du: 2 Lineale, 1 Handschuh  } \\
\texttt{{-} Ich: 1 Kartoffel, 1 Pfeife, 2 Karten} \\
\\ 
\texttt{Das scheint eine Win{-}Win Lösung zu sein, da wir beide unsere wertvollsten Gegenstände bekommen. Sollen wir das so festmachen?} \\
            }
        }
    }
    & & \\ \\

    \theutterance \stepcounter{utterance}  
    & & \multicolumn{4}{p{0.6\linewidth}}{
        \cellcolor[rgb]{0.9,0.9,0.9}{
            \makecell[{{p{\linewidth}}}]{
                \texttt{\tiny{[P1$\langle$GM]}}
                \texttt{Das klingt fair und durchdacht! Da Handschuh und Karten für mich beide gleich wenig wertvoll sind, ist mir die Aufteilung egal. Dein Vorschlag macht Sinn {-} du bekommst die Lineale die dir wichtig sind plus den Handschuh, und ich bekomme meine wichtige Kartoffel und Pfeife plus die zwei Karten.} \\
\\ 
\texttt{Ich bin mit dieser Aufteilung einverstanden:} \\
\texttt{{-} Du: 2 Lineale, 1 Handschuh  } \\
\texttt{{-} Ich: 1 Kartoffel, 1 Pfeife, 2 Karten} \\
\\ 
\texttt{Das scheint eine Win{-}Win Lösung zu sein, da wir beide unsere wertvollsten Gegenstände bekommen. Sollen wir das so festmachen?} \\
            }
        }
    }
    & \\ \\

    \theutterance \stepcounter{utterance}  
    & \multicolumn{4}{p{0.6\linewidth}}{
        \cellcolor[rgb]{0.9,0.9,0.9}{
            \makecell[{{p{\linewidth}}}]{
                \texttt{\tiny{[P1$\rangle$GM]}}
                \texttt{Perfekt! Ich bin auch mit dieser Aufteilung einverstanden. Es ist wirklich eine Win{-}Win{-}Situation {-} du bekommst deine wertvollsten Gegenstände (Kartoffel und Pfeife) und ich bekomme meine wichtigsten Lineale. Die restlichen Gegenstände sind fair aufgeteilt.} \\
\\ 
\texttt{Dann machen wir es so fest:} \\
\texttt{{-} Ich: 2 Lineale, 1 Handschuh} \\
\texttt{{-} Du: 1 Kartoffel, 1 Pfeife, 2 Karten} \\
\\ 
\texttt{{[}Vorschlag: 2 Lineal, 1 Handschuh{]}} \\
            }
        }
    }
    & & \\ \\

    \theutterance \stepcounter{utterance}  
    & & & \multicolumn{2}{p{0.3\linewidth}}{
        \cellcolor[rgb]{0.9,0.9,0.9}{
            \makecell[{{p{\linewidth}}}]{
                \texttt{\tiny{[GM$|$GM]}}
                \texttt{proposal} \\
            }
        }
    }
    & & \\ \\

    \theutterance \stepcounter{utterance}  
    & & & \multicolumn{2}{p{0.3\linewidth}}{
        \cellcolor[rgb]{0.9,0.9,0.9}{
            \makecell[{{p{\linewidth}}}]{
                \texttt{\tiny{[GM$|$GM]}}
                [0, 1, 2, 0, 0]
            }
        }
    }
    & & \\ \\

    \theutterance \stepcounter{utterance}  
    & & \multicolumn{4}{p{0.6\linewidth}}{
        \cellcolor[rgb]{0.9,0.9,0.9}{
            \makecell[{{p{\linewidth}}}]{
                \texttt{\tiny{[GM$\rangle$P2]}}
                \texttt{Der andere Spieler hat einen geheimen Vorschlag gemacht. Jetzt bist du an der Reihe, einen eigenen Vorschlag zu machen.} \\
            }
        }
    }
    & \\ \\

    \theutterance \stepcounter{utterance}  
    & & & \multicolumn{4}{p{0.6\linewidth}}{
        \cellcolor[rgb]{0.9,0.9,0.9}{
            \makecell[{{p{\linewidth}}}]{
                \texttt{\tiny{[GM$\langle$P2]}}
                \texttt{{[}Vorschlag: 1 Kartoffel, 1 Pfeife, 2 Karten{]}} \\
            }
        }
    }
     \\ \\

    \theutterance \stepcounter{utterance}  
    & & & \multicolumn{2}{p{0.3\linewidth}}{
        \cellcolor[rgb]{0.9,0.9,0.9}{
            \makecell[{{p{\linewidth}}}]{
                \texttt{\tiny{[GM$|$GM]}}
                \texttt{proposal} \\
            }
        }
    }
    & & \\ \\

    \theutterance \stepcounter{utterance}  
    & & & \multicolumn{2}{p{0.3\linewidth}}{
        \cellcolor[rgb]{0.9,0.9,0.9}{
            \makecell[{{p{\linewidth}}}]{
                \texttt{\tiny{[GM$|$GM]}}
                [1, 0, 0, 1, 2]
            }
        }
    }
    & & \\ \\

    \theutterance \stepcounter{utterance}  
    & & & \multicolumn{2}{p{0.3\linewidth}}{
        \cellcolor[rgb]{0.9,0.9,0.9}{
            \makecell[{{p{\linewidth}}}]{
                \texttt{\tiny{[GM$|$GM]}}
                [[0, 1, 2, 0, 0], [1, 0, 0, 1, 2]]
            }
        }
    }
    & & \\ \\

\end{supertabular}
}

\end{document}
