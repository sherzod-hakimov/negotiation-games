
\documentclass{article}
\usepackage{colortbl}
\usepackage{makecell}
\usepackage{multirow}
\usepackage{supertabular}

\begin{document}

\newcounter{utterance}

\centering \large Interaction Transcript for game `dond', experiment `coop\_de', episode 12 with claude{-}sonnet{-}4{-}20250514{-}no{-}reasoning{-}t1.0.
\vspace{24pt}

{ \footnotesize  \setcounter{utterance}{1}
\setlength{\tabcolsep}{0pt}
\begin{supertabular}{c@{$\;$}|p{.15\linewidth}@{}p{.15\linewidth}p{.15\linewidth}p{.15\linewidth}p{.15\linewidth}p{.15\linewidth}}
    \# & $\;$A & \multicolumn{4}{c}{Game Master} & $\;\:$B\\
    \hline

    \theutterance \stepcounter{utterance}  
    & & \multicolumn{4}{p{0.6\linewidth}}{
        \cellcolor[rgb]{0.9,0.9,0.9}{
            \makecell[{{p{\linewidth}}}]{
                \texttt{\tiny{[P1$\langle$GM]}}
                \texttt{Sie spielen ein kooperatives Verhandlungsspiel, bei dem Sie sich mit einem anderen Spieler darauf einigen müssen, wie eine Reihe von Gegenständen aufgeteilt werden soll.} \\
\\ 
\texttt{Die Regeln:} \\
\texttt{(a) Sie und der andere Spieler erhalten eine Sammlung von Gegenständen. Jeder von Ihnen erhält außerdem eine geheime Wertfunktion, die angibt, wie viel Ihnen jede Art von Gegenstand wert ist.} \\
\texttt{(b) Sie tauschen Nachrichten mit dem anderen Spieler aus, um zu vereinbaren, wer welche Gegenstände bekommt. Sie können jeweils maximal 5 Nachrichten senden oder das Spiel vorzeitig beenden, indem Sie jederzeit einen geheimen Vorschlag machen.} \\
\texttt{(c) Jeder von euch wird aufgefordert, einen geheimen Vorschlag zu machen, in dem ihr die gewünschten Gegenstände in eckigen Klammern wie folgt angibt: "{[}Vorschlag: <Nummer> <Objektname>, <Nummer> <Objektname>, <...>{]}"} \\
\texttt{(d) Wenn eure Vorschläge komplementär sind, d.h. es gibt genug Gegenstände, um beide Vorschläge zu erfüllen, erhält jeder Spieler eine Punktzahl, die sich aus der Summe der Werte für die Gegenstände ergibt, die er erhalten hat. Andernfalls erhalten Sie beide null Punkte.} \\
\texttt{(e) Ihr Ziel ist es, die Summe aus Ihrer Punktzahl und der Punktzahl des anderen Spielers zu maximieren.} \\
\\ 
\texttt{Beginnen wir.} \\
\\ 
\texttt{Die Menge der verfügbaren Gegenstände ist:} \\
\\ 
\texttt{3 Lineale, 2 Eier, 1 Batterie} \\
\\ 
\texttt{Deine geheime Wertfunktion ist:} \\
\\ 
\texttt{Lineal: 1, Ei: 2, Batterie: 3} \\
\\ 
\texttt{WICHTIG: Ihre Nachrichten werden, sofern es sich nicht um einen geheimen Vorschlag handelt, direkt an den anderen Spieler übermittelt, also fügen Sie keine Antwort auf die Regeln oder einen Text zur Ankündigung Ihrer Nachricht ein. Um einen geheimen Vorschlag zu machen, verwenden Sie das angegebene Format. Verwenden Sie keine eckigen Klammern, wenn Sie mit dem anderen Spieler kommunizieren, sonst wird dies als Ihr geheimer Vorschlag interpretiert.} \\
            }
        }
    }
    & \\ \\

    \theutterance \stepcounter{utterance}  
    & \multicolumn{4}{p{0.6\linewidth}}{
        \cellcolor[rgb]{0.9,0.9,0.9}{
            \makecell[{{p{\linewidth}}}]{
                \texttt{\tiny{[P1$\rangle$GM]}}
                \texttt{Hallo! Lass uns schaffen, eine gute Aufteilung für uns beide zu finden. Wir haben 3 Lineale, 2 Eier und 1 Batterie zu verteilen.} \\
\\ 
\texttt{Um eine optimale Lösung zu finden, wäre es hilfreich zu wissen, welche Gegenstände dir besonders wichtig sind. Könntest du mir sagen, welche Gegenstände du am meisten schätzt?} \\
\\ 
\texttt{Ich kann dir schon mal sagen, dass mir die Batterie sehr wichtig ist, die Eier sind mir mittelmäßig wichtig, und die Lineale sind für mich am wenigsten wertvoll.} \\
            }
        }
    }
    & & \\ \\

    \theutterance \stepcounter{utterance}  
    & & & \multicolumn{2}{p{0.3\linewidth}}{
        \cellcolor[rgb]{0.9,0.9,0.9}{
            \makecell[{{p{\linewidth}}}]{
                \texttt{\tiny{[GM$|$GM]}}
                \texttt{continue} \\
            }
        }
    }
    & & \\ \\

    \theutterance \stepcounter{utterance}  
    & & & \multicolumn{2}{p{0.3\linewidth}}{
        \cellcolor[rgb]{0.9,0.9,0.9}{
            \makecell[{{p{\linewidth}}}]{
                \texttt{\tiny{[GM$|$GM]}}
                \texttt{Hallo! Lass uns schaffen, eine gute Aufteilung für uns beide zu finden. Wir haben 3 Lineale, 2 Eier und 1 Batterie zu verteilen.} \\
\\ 
\texttt{Um eine optimale Lösung zu finden, wäre es hilfreich zu wissen, welche Gegenstände dir besonders wichtig sind. Könntest du mir sagen, welche Gegenstände du am meisten schätzt?} \\
\\ 
\texttt{Ich kann dir schon mal sagen, dass mir die Batterie sehr wichtig ist, die Eier sind mir mittelmäßig wichtig, und die Lineale sind für mich am wenigsten wertvoll.} \\
            }
        }
    }
    & & \\ \\

    \theutterance \stepcounter{utterance}  
    & & \multicolumn{4}{p{0.6\linewidth}}{
        \cellcolor[rgb]{0.9,0.9,0.9}{
            \makecell[{{p{\linewidth}}}]{
                \texttt{\tiny{[GM$\rangle$P2]}}
                \texttt{Sie spielen ein kooperatives Verhandlungsspiel, bei dem Sie sich mit einem anderen Spieler darauf einigen müssen, wie eine Reihe von Gegenständen aufgeteilt werden soll.} \\
\\ 
\texttt{Die Regeln:} \\
\texttt{(a) Sie und der andere Spieler erhalten eine Sammlung von Gegenständen. Jeder von Ihnen erhält außerdem eine geheime Wertfunktion, die angibt, wie viel Ihnen jede Art von Gegenstand wert ist.} \\
\texttt{(b) Sie tauschen Nachrichten mit dem anderen Spieler aus, um zu vereinbaren, wer welche Gegenstände bekommt. Sie können jeweils maximal 5 Nachrichten senden oder das Spiel vorzeitig beenden, indem Sie jederzeit einen geheimen Vorschlag machen.} \\
\texttt{(c) Jeder von euch wird aufgefordert, einen geheimen Vorschlag zu machen, in dem ihr die gewünschten Gegenstände in eckigen Klammern wie folgt angibt: "{[}Vorschlag: <Nummer> <Objektname>, <Nummer> <Objektname>, <...>{]}"} \\
\texttt{(d) Wenn eure Vorschläge komplementär sind, d.h. es gibt genug Gegenstände, um beide Vorschläge zu erfüllen, erhält jeder Spieler eine Punktzahl, die sich aus der Summe der Werte für die Gegenstände ergibt, die er erhalten hat. Andernfalls erhalten Sie beide null Punkte.} \\
\texttt{(e) Ihr Ziel ist es, die Summe aus Ihrer Punktzahl und der Punktzahl des anderen Spielers zu maximieren.} \\
\\ 
\texttt{Beginnen wir.} \\
\\ 
\texttt{Die Menge der verfügbaren Gegenstände ist:} \\
\\ 
\texttt{3 Lineale, 2 Eier, 1 Batterie} \\
\\ 
\texttt{Deine geheime Wertfunktion ist:} \\
\\ 
\texttt{Lineal: 3, Ei: 0, Batterie: 1} \\
\\ 
\texttt{WICHTIG: Ihre Nachrichten werden, sofern es sich nicht um einen geheimen Vorschlag handelt, direkt an den anderen Spieler übermittelt, also fügen Sie keine Antwort auf die Regeln oder einen Text zur Ankündigung Ihrer Nachricht ein. Um einen geheimen Vorschlag zu machen, verwenden Sie das angegebene Format. Verwenden Sie keine eckigen Klammern, wenn Sie mit dem anderen Spieler kommunizieren, sonst wird dies als Ihr geheimer Vorschlag interpretiert.} \\
\\ 
\\ 
\texttt{Hallo! Lass uns schaffen, eine gute Aufteilung für uns beide zu finden. Wir haben 3 Lineale, 2 Eier und 1 Batterie zu verteilen.} \\
\\ 
\texttt{Um eine optimale Lösung zu finden, wäre es hilfreich zu wissen, welche Gegenstände dir besonders wichtig sind. Könntest du mir sagen, welche Gegenstände du am meisten schätzt?} \\
\\ 
\texttt{Ich kann dir schon mal sagen, dass mir die Batterie sehr wichtig ist, die Eier sind mir mittelmäßig wichtig, und die Lineale sind für mich am wenigsten wertvoll.} \\
            }
        }
    }
    & \\ \\

    \theutterance \stepcounter{utterance}  
    & & & \multicolumn{4}{p{0.6\linewidth}}{
        \cellcolor[rgb]{0.9,0.9,0.9}{
            \makecell[{{p{\linewidth}}}]{
                \texttt{\tiny{[GM$\langle$P2]}}
                \texttt{Hallo! Danke für deine Offenheit. Interessant {-} unsere Präferenzen scheinen sich zu unterscheiden.} \\
\\ 
\texttt{Für mich sind tatsächlich die Lineale am wertvollsten, die Batterie ist mir weniger wichtig, und die Eier sind für mich praktisch wertlos.} \\
\\ 
\texttt{Das klingt nach einer guten Basis für eine Win{-}Win{-}Situation! Wenn ich das richtig verstehe, wärst du bereit, mir die Lineale zu überlassen, während du die Batterie nimmst? Was denkst du über die Eier {-} da du sie als mittelmäßig wichtig einschätzt, könntest du sie nehmen?} \\
            }
        }
    }
     \\ \\

    \theutterance \stepcounter{utterance}  
    & & & \multicolumn{2}{p{0.3\linewidth}}{
        \cellcolor[rgb]{0.9,0.9,0.9}{
            \makecell[{{p{\linewidth}}}]{
                \texttt{\tiny{[GM$|$GM]}}
                \texttt{continue} \\
            }
        }
    }
    & & \\ \\

    \theutterance \stepcounter{utterance}  
    & & & \multicolumn{2}{p{0.3\linewidth}}{
        \cellcolor[rgb]{0.9,0.9,0.9}{
            \makecell[{{p{\linewidth}}}]{
                \texttt{\tiny{[GM$|$GM]}}
                \texttt{Hallo! Danke für deine Offenheit. Interessant {-} unsere Präferenzen scheinen sich zu unterscheiden.} \\
\\ 
\texttt{Für mich sind tatsächlich die Lineale am wertvollsten, die Batterie ist mir weniger wichtig, und die Eier sind für mich praktisch wertlos.} \\
\\ 
\texttt{Das klingt nach einer guten Basis für eine Win{-}Win{-}Situation! Wenn ich das richtig verstehe, wärst du bereit, mir die Lineale zu überlassen, während du die Batterie nimmst? Was denkst du über die Eier {-} da du sie als mittelmäßig wichtig einschätzt, könntest du sie nehmen?} \\
            }
        }
    }
    & & \\ \\

    \theutterance \stepcounter{utterance}  
    & & \multicolumn{4}{p{0.6\linewidth}}{
        \cellcolor[rgb]{0.9,0.9,0.9}{
            \makecell[{{p{\linewidth}}}]{
                \texttt{\tiny{[P1$\langle$GM]}}
                \texttt{Hallo! Danke für deine Offenheit. Interessant {-} unsere Präferenzen scheinen sich zu unterscheiden.} \\
\\ 
\texttt{Für mich sind tatsächlich die Lineale am wertvollsten, die Batterie ist mir weniger wichtig, und die Eier sind für mich praktisch wertlos.} \\
\\ 
\texttt{Das klingt nach einer guten Basis für eine Win{-}Win{-}Situation! Wenn ich das richtig verstehe, wärst du bereit, mir die Lineale zu überlassen, während du die Batterie nimmst? Was denkst du über die Eier {-} da du sie als mittelmäßig wichtig einschätzt, könntest du sie nehmen?} \\
            }
        }
    }
    & \\ \\

    \theutterance \stepcounter{utterance}  
    & \multicolumn{4}{p{0.6\linewidth}}{
        \cellcolor[rgb]{0.9,0.9,0.9}{
            \makecell[{{p{\linewidth}}}]{
                \texttt{\tiny{[P1$\rangle$GM]}}
                \texttt{Das klingt nach einer ausgezeichneten Lösung! Du hast recht {-} unsere unterschiedlichen Präferenzen ermöglichen wirklich eine Win{-}Win{-}Situation.} \\
\\ 
\texttt{Wenn du alle 3 Lineale nimmst (die für dich am wertvollsten sind) und ich die Batterie bekomme (die für mich sehr wichtig ist), dann sind wir beide mit den wichtigsten Gegenständen zufrieden.} \\
\\ 
\texttt{Bei den Eiern bin ich gerne bereit, beide zu nehmen, da sie für dich praktisch wertlos sind, während sie für mich immerhin mittelmäßig wichtig sind.} \\
\\ 
\texttt{Zusammengefasst: Du bekommst die 3 Lineale, ich bekomme die Batterie und die 2 Eier. Ist das für dich in Ordnung?} \\
            }
        }
    }
    & & \\ \\

    \theutterance \stepcounter{utterance}  
    & & & \multicolumn{2}{p{0.3\linewidth}}{
        \cellcolor[rgb]{0.9,0.9,0.9}{
            \makecell[{{p{\linewidth}}}]{
                \texttt{\tiny{[GM$|$GM]}}
                \texttt{continue} \\
            }
        }
    }
    & & \\ \\

    \theutterance \stepcounter{utterance}  
    & & & \multicolumn{2}{p{0.3\linewidth}}{
        \cellcolor[rgb]{0.9,0.9,0.9}{
            \makecell[{{p{\linewidth}}}]{
                \texttt{\tiny{[GM$|$GM]}}
                \texttt{Das klingt nach einer ausgezeichneten Lösung! Du hast recht {-} unsere unterschiedlichen Präferenzen ermöglichen wirklich eine Win{-}Win{-}Situation.} \\
\\ 
\texttt{Wenn du alle 3 Lineale nimmst (die für dich am wertvollsten sind) und ich die Batterie bekomme (die für mich sehr wichtig ist), dann sind wir beide mit den wichtigsten Gegenständen zufrieden.} \\
\\ 
\texttt{Bei den Eiern bin ich gerne bereit, beide zu nehmen, da sie für dich praktisch wertlos sind, während sie für mich immerhin mittelmäßig wichtig sind.} \\
\\ 
\texttt{Zusammengefasst: Du bekommst die 3 Lineale, ich bekomme die Batterie und die 2 Eier. Ist das für dich in Ordnung?} \\
            }
        }
    }
    & & \\ \\

    \theutterance \stepcounter{utterance}  
    & & \multicolumn{4}{p{0.6\linewidth}}{
        \cellcolor[rgb]{0.9,0.9,0.9}{
            \makecell[{{p{\linewidth}}}]{
                \texttt{\tiny{[GM$\rangle$P2]}}
                \texttt{Das klingt nach einer ausgezeichneten Lösung! Du hast recht {-} unsere unterschiedlichen Präferenzen ermöglichen wirklich eine Win{-}Win{-}Situation.} \\
\\ 
\texttt{Wenn du alle 3 Lineale nimmst (die für dich am wertvollsten sind) und ich die Batterie bekomme (die für mich sehr wichtig ist), dann sind wir beide mit den wichtigsten Gegenständen zufrieden.} \\
\\ 
\texttt{Bei den Eiern bin ich gerne bereit, beide zu nehmen, da sie für dich praktisch wertlos sind, während sie für mich immerhin mittelmäßig wichtig sind.} \\
\\ 
\texttt{Zusammengefasst: Du bekommst die 3 Lineale, ich bekomme die Batterie und die 2 Eier. Ist das für dich in Ordnung?} \\
            }
        }
    }
    & \\ \\

    \theutterance \stepcounter{utterance}  
    & & & \multicolumn{4}{p{0.6\linewidth}}{
        \cellcolor[rgb]{0.9,0.9,0.9}{
            \makecell[{{p{\linewidth}}}]{
                \texttt{\tiny{[GM$\langle$P2]}}
                \texttt{Perfekt! Das ist eine ausgezeichnete Aufteilung, bei der wir beide das bekommen, was uns am wichtigsten ist. Du bekommst die Batterie, die dir sehr wertvoll ist, plus die Eier, die für mich keinen Wert haben, aber für dich mittelmäßig wichtig sind. Ich bekomme alle Lineale, die für mich den höchsten Wert haben.} \\
\\ 
\texttt{Ich bin mit dieser Aufteilung vollkommen einverstanden: Du nimmst 1 Batterie und 2 Eier, ich nehme 3 Lineale.} \\
\\ 
\texttt{{[}Vorschlag: 3 Lineal{]}} \\
            }
        }
    }
     \\ \\

    \theutterance \stepcounter{utterance}  
    & & & \multicolumn{2}{p{0.3\linewidth}}{
        \cellcolor[rgb]{0.9,0.9,0.9}{
            \makecell[{{p{\linewidth}}}]{
                \texttt{\tiny{[GM$|$GM]}}
                \texttt{proposal} \\
            }
        }
    }
    & & \\ \\

    \theutterance \stepcounter{utterance}  
    & & & \multicolumn{2}{p{0.3\linewidth}}{
        \cellcolor[rgb]{0.9,0.9,0.9}{
            \makecell[{{p{\linewidth}}}]{
                \texttt{\tiny{[GM$|$GM]}}
                [3, 0, 0]
            }
        }
    }
    & & \\ \\

    \theutterance \stepcounter{utterance}  
    & & \multicolumn{4}{p{0.6\linewidth}}{
        \cellcolor[rgb]{0.9,0.9,0.9}{
            \makecell[{{p{\linewidth}}}]{
                \texttt{\tiny{[P1$\langle$GM]}}
                \texttt{Der andere Spieler hat einen geheimen Vorschlag gemacht. Jetzt bist du an der Reihe, einen eigenen Vorschlag zu machen.} \\
            }
        }
    }
    & \\ \\

    \theutterance \stepcounter{utterance}  
    & \multicolumn{4}{p{0.6\linewidth}}{
        \cellcolor[rgb]{0.9,0.9,0.9}{
            \makecell[{{p{\linewidth}}}]{
                \texttt{\tiny{[P1$\rangle$GM]}}
                \texttt{{[}Vorschlag: 1 Batterie, 2 Eier{]}} \\
            }
        }
    }
    & & \\ \\

    \theutterance \stepcounter{utterance}  
    & & & \multicolumn{2}{p{0.3\linewidth}}{
        \cellcolor[rgb]{0.9,0.9,0.9}{
            \makecell[{{p{\linewidth}}}]{
                \texttt{\tiny{[GM$|$GM]}}
                \texttt{proposal} \\
            }
        }
    }
    & & \\ \\

    \theutterance \stepcounter{utterance}  
    & & & \multicolumn{2}{p{0.3\linewidth}}{
        \cellcolor[rgb]{0.9,0.9,0.9}{
            \makecell[{{p{\linewidth}}}]{
                \texttt{\tiny{[GM$|$GM]}}
                [0, 2, 1]
            }
        }
    }
    & & \\ \\

    \theutterance \stepcounter{utterance}  
    & & & \multicolumn{2}{p{0.3\linewidth}}{
        \cellcolor[rgb]{0.9,0.9,0.9}{
            \makecell[{{p{\linewidth}}}]{
                \texttt{\tiny{[GM$|$GM]}}
                [[0, 2, 1], [3, 0, 0]]
            }
        }
    }
    & & \\ \\

\end{supertabular}
}

\end{document}
