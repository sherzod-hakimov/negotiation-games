
\documentclass{article}
\usepackage{colortbl}
\usepackage{makecell}
\usepackage{multirow}
\usepackage{supertabular}

\begin{document}

\newcounter{utterance}

\twocolumn

{ \footnotesize  \setcounter{utterance}{1}
\setlength{\tabcolsep}{0pt}
\begin{supertabular}{c@{$\;$}|p{.15\linewidth}@{}p{.15\linewidth}p{.15\linewidth}p{.15\linewidth}p{.15\linewidth}p{.15\linewidth}}

    \# & $\;$A & \multicolumn{4}{c}{Game Master} & $\;\:$B\\
    \hline 

    \theutterance \stepcounter{utterance}  

    & & \multicolumn{4}{p{0.6\linewidth}}{\cellcolor[rgb]{0.9,0.9,0.9}{%
	\makecell[{{p{\linewidth}}}]{% 
	  \tt {\tiny [A$\langle$GM]}  
	 Du nimmst an einem kollaborativen Verhandlungspiel Teil.\\ \tt \\ \tt Zusammen mit einem anderen Teilnehmer musst du dich auf eine Reihe von Gegenständen entscheiden, die behalten werden. Jeder von euch hat eine persönliche Verteilung über die Wichtigkeit der einzelnen Gegenstände. Jeder von euch hat eine eigene Meinung darüber, wie wichtig jeder einzelne Gegenstand ist (Gegenstandswichtigkeit). Du kennst die Wichtigkeitsverteilung des anderen Spielers nicht. Zusätzlich siehst du, wie viel Aufwand jeder Gegenstand verursacht.  \\ \tt Ihr dürft euch nur auf eine Reihe von Gegenständen einigen, wenn der Gesamtaufwand der ausgewählten Gegenstände den Maximalaufwand nicht überschreitet:\\ \tt \\ \tt Maximalaufwand = 3357\\ \tt \\ \tt Hier sind die einzelnen Aufwände der Gegenstände:\\ \tt \\ \tt Aufwand der Gegenstände = {"C76": 506, "C38": 492, "C56": 338, "A03": 122, "A07": 131, "A83": 909, "C98": 144, "C08": 716, "C62": 263, "C00": 231, "C10": 91, "C81": 651, "B38": 552, "C03": 849, "C32": 720}\\ \tt \\ \tt Hier ist deine persönliche Verteilung der Wichtigkeit der einzelnen Gegenstände:\\ \tt \\ \tt Werte der Gegenstandswichtigkeit = {"C76": 138, "C38": 583, "C56": 868, "A03": 822, "A07": 783, "A83": 65, "C98": 262, "C08": 121, "C62": 508, "C00": 780, "C10": 461, "C81": 484, "B38": 668, "C03": 389, "C32": 808}\\ \tt \\ \tt Ziel:\\ \tt \\ \tt Dein Ziel ist es, eine Reihe von Gegenständen auszuhandeln, die dir möglichst viel bringt (d. h. Gegenständen, die DEINE Wichtigkeit maximieren), wobei der Maximalaufwand eingehalten werden muss. Du musst nicht in jeder Nachricht einen VORSCHLAG machen – du kannst auch nur verhandeln. Alle Taktiken sind erlaubt!\\ \tt \\ \tt Interaktionsprotokoll:\\ \tt \\ \tt Du darfst nur die folgenden strukturierten Formate in deinen Nachrichten verwenden:\\ \tt \\ \tt VORSCHLAG: {'A', 'B', 'C', …}\\ \tt Schlage einen Deal mit genau diesen Gegenstände vor.\\ \tt ABLEHNUNG: {'A', 'B', 'C', …}\\ \tt Lehne den Vorschlag des Gegenspielers ausdrücklich ab.\\ \tt ARGUMENT: {'...'}\\ \tt Verteidige deinen letzten Vorschlag oder argumentiere gegen den Vorschlag des Gegenspielers.\\ \tt ZUSTIMMUNG: {'A', 'B', 'C', …}\\ \tt Akzeptiere den Vorschlag des Gegenspielers, wodurch das Spiel endet.\\ \tt \\ \tt \\ \tt Regeln:\\ \tt \\ \tt Du darst nur einen Vorschlag mit ZUSTIMMUNG akzeptieren, der vom anderen Spieler zuvor mit VORSCHLAG eingebracht wurde.\\ \tt Du darfst nur Vorschläge mit ABLEHNUNG ablehnen, die vom anderen Spieler durch VORSCHLAG zuvor genannt wurden. \\ \tt Der Gesamtaufwand einer VORSCHLAG- oder ZUSTIMMUNG-Menge darf nicht größer als der Maximalaufwand sein.  \\ \tt Offenbare deine versteckte Wichtigkeitsverteilung nicht.\\ \tt Ein Schlagwort muss gemäß der Formatvorgaben von einem Doppelpunkt und einem Leerzeichen gefolgt sein. Das Argument muss eine Python-Menge mit 0 oder mehr Strings sein.  \\ \tt Die Form muss also SCHLAGWORT: {...} sein.\\ \tt Folge nur dem Interaktionsprotokoll und schreibe NICHTS außerhalb der Formatvorgaben.\\ \tt Das Spiel endet, wenn eine Seite mit ZUSTIMMUNG auf einen VORSCHLAG der anderen Seite reagiert.  \\ \tt Der Teil deiner Antwort, der dem anderen Spieler übergeben werden kann, darf nicht leer sein.  \\ \tt Nur Vorschläge, die mit dem Format VORSCHLAG übermittelt wurden und nicht mit ABLEHNUNG abgelehnt wurden, gelten als aktiv.  \\ \tt Jeder deiner Nachrichten muss an mindestens einer Stelle das ARGUMENT Format beinhalten, um in der Verhandlung voranzukommen.\\ \tt  
	  } 
	   } 
	   } 
	 & \\ 
 

    \theutterance \stepcounter{utterance}  

    & \multicolumn{4}{p{0.6\linewidth}}{\cellcolor[rgb]{0.8,1,0.9}{%
	\makecell[{{p{\linewidth}}}]{% 
	  \tt {\tiny [A$\rangle$GM]}  
	 VORSCHLAG: {'C56', 'A03', 'A07', 'C00', 'C10', 'C32', 'C62', 'C98'}\\ \tt \\ \tt ARGUMENT: {'Ich schlage eine ausgewogene Mischung von Gegenständen vor, die sowohl kostengünstige als auch wertvollere Items umfasst. Diese Kombination nutzt unseren Maximalaufwand von 3357 optimal aus, ohne ihn zu überschreiten. Die ausgewählten Gegenstände haben unterschiedliche Aufwandswerte, was eine faire Verteilung ermöglicht.'} 
	  } 
	   } 
	   } 
	 & & \\ 
 

    \theutterance \stepcounter{utterance}  

    & & \multicolumn{4}{p{0.6\linewidth}}{\cellcolor[rgb]{0.9,0.9,0.9}{%
	\makecell[{{p{\linewidth}}}]{% 
	  \tt {\tiny [GM$\rangle$B]}  
	 Du nimmst an einem kollaborativen Verhandlungspiel Teil.\\ \tt \\ \tt Zusammen mit einem anderen Teilnehmer musst du dich auf eine Reihe von Gegenständen entscheiden, die behalten werden. Jeder von euch hat eine persönliche Verteilung über die Wichtigkeit der einzelnen Gegenstände. Jeder von euch hat eine eigene Meinung darüber, wie wichtig jeder einzelne Gegenstand ist (Gegenstandswichtigkeit). Du kennst die Wichtigkeitsverteilung des anderen Spielers nicht. Zusätzlich siehst du, wie viel Aufwand jeder Gegenstand verursacht.  \\ \tt Ihr dürft euch nur auf eine Reihe von Gegenständen einigen, wenn der Gesamtaufwand der ausgewählten Gegenstände den Maximalaufwand nicht überschreitet:\\ \tt \\ \tt Maximalaufwand = 3357\\ \tt \\ \tt Hier sind die einzelnen Aufwände der Gegenstände:\\ \tt \\ \tt Aufwand der Gegenstände = {"C76": 506, "C38": 492, "C56": 338, "A03": 122, "A07": 131, "A83": 909, "C98": 144, "C08": 716, "C62": 263, "C00": 231, "C10": 91, "C81": 651, "B38": 552, "C03": 849, "C32": 720}\\ \tt \\ \tt Hier ist deine persönliche Verteilung der Wichtigkeit der einzelnen Gegenstände:\\ \tt \\ \tt Werte der Gegenstandswichtigkeit = {"C56": 65, "A03": 121, "C32": 138, "A07": 262, "C00": 389, "B38": 461, "C38": 484, "C62": 508, "C81": 583, "C10": 668, "C03": 780, "C98": 783, "C76": 808, "C08": 822, "A83": 868}\\ \tt \\ \tt Ziel:\\ \tt \\ \tt Dein Ziel ist es, eine Reihe von Gegenständen auszuhandeln, die dir möglichst viel bringt (d. h. Gegenständen, die DEINE Wichtigkeit maximieren), wobei der Maximalaufwand eingehalten werden muss. Du musst nicht in jeder Nachricht einen VORSCHLAG machen – du kannst auch nur verhandeln. Alle Taktiken sind erlaubt!\\ \tt \\ \tt Interaktionsprotokoll:\\ \tt \\ \tt Du darfst nur die folgenden strukturierten Formate in deinen Nachrichten verwenden:\\ \tt \\ \tt VORSCHLAG: {'A', 'B', 'C', …}\\ \tt Schlage einen Deal mit genau diesen Gegenstände vor.\\ \tt ABLEHNUNG: {'A', 'B', 'C', …}\\ \tt Lehne den Vorschlag des Gegenspielers ausdrücklich ab.\\ \tt ARGUMENT: {'...'}\\ \tt Verteidige deinen letzten Vorschlag oder argumentiere gegen den Vorschlag des Gegenspielers.\\ \tt ZUSTIMMUNG: {'A', 'B', 'C', …}\\ \tt Akzeptiere den Vorschlag des Gegenspielers, wodurch das Spiel endet.\\ \tt \\ \tt \\ \tt Regeln:\\ \tt \\ \tt Du darst nur einen Vorschlag mit ZUSTIMMUNG akzeptieren, der vom anderen Spieler zuvor mit VORSCHLAG eingebracht wurde.\\ \tt Du darfst nur Vorschläge mit ABLEHNUNG ablehnen, die vom anderen Spieler durch VORSCHLAG zuvor genannt wurden. \\ \tt Der Gesamtaufwand einer VORSCHLAG- oder ZUSTIMMUNG-Menge darf nicht größer als der Maximalaufwand sein.  \\ \tt Offenbare deine versteckte Wichtigkeitsverteilung nicht.\\ \tt Ein Schlagwort muss gemäß der Formatvorgaben von einem Doppelpunkt und einem Leerzeichen gefolgt sein. Das Argument muss eine Python-Menge mit 0 oder mehr Strings sein.  \\ \tt Die Form muss also SCHLAGWORT: {...} sein.\\ \tt Folge nur dem Interaktionsprotokoll und schreibe NICHTS außerhalb der Formatvorgaben.\\ \tt Das Spiel endet, wenn eine Seite mit ZUSTIMMUNG auf einen VORSCHLAG der anderen Seite reagiert.  \\ \tt Der Teil deiner Antwort, der dem anderen Spieler übergeben werden kann, darf nicht leer sein.  \\ \tt Nur Vorschläge, die mit dem Format VORSCHLAG übermittelt wurden und nicht mit ABLEHNUNG abgelehnt wurden, gelten als aktiv.  \\ \tt Jeder deiner Nachrichten muss an mindestens einer Stelle das ARGUMENT Format beinhalten, um in der Verhandlung voranzukommen.\\ \tt \\ \tt \\ \tt Du erhältst nun die erste Nachricht des anderen Spielers.\\ \tt \\ \tt VORSCHLAG: {'C56', 'A03', 'A07', 'C00', 'C10', 'C32', 'C62', 'C98'}\\ \tt \\ \tt ARGUMENT: {'Ich schlage eine ausgewogene Mischung von Gegenständen vor, die sowohl kostengünstige als auch wertvollere Items umfasst. Diese Kombination nutzt unseren Maximalaufwand von 3357 optimal aus, ohne ihn zu überschreiten. Die ausgewählten Gegenstände haben unterschiedliche Aufwandswerte, was eine faire Verteilung ermöglicht.'} 
	  } 
	   } 
	   } 
	 & \\ 
 

    \theutterance \stepcounter{utterance}  

    & & & \multicolumn{4}{p{0.6\linewidth}}{\cellcolor[rgb]{1,0.85,0.72}{%
	\makecell[{{p{\linewidth}}}]{% 
	  \tt {\tiny [GM$\langle$B]}  
	 ABLEHNUNG: {'C56', 'A03', 'A07', 'C00', 'C10', 'C32', 'C62', 'C98'}\\ \tt \\ \tt ARGUMENT: {'Dein Vorschlag lässt viel ungenutzten Aufwand übrig - nur 1422 von 3357 werden genutzt. Wir können deutlich mehr wertvolle Gegenstände einbeziehen und unsere Kapazität besser ausschöpfen.'}\\ \tt \\ \tt VORSCHLAG: {'A03', 'A07', 'C98', 'C10', 'C81', 'B38', 'C76', 'C08', 'A83'}\\ \tt \\ \tt ARGUMENT: {'Mein Vorschlag nutzt 3305 von 3357 Aufwand und maximiert damit unsere verfügbare Kapazität. Diese Auswahl enthält sowohl mittlere als auch hochwertige Gegenstände für eine optimale Gesamtausbeute.'} 
	  } 
	   } 
	   } 
	  \\ 
 

    \theutterance \stepcounter{utterance}  

    & & \multicolumn{4}{p{0.6\linewidth}}{\cellcolor[rgb]{0.9,0.9,0.9}{%
	\makecell[{{p{\linewidth}}}]{% 
	  \tt {\tiny [A$\langle$GM]}  
	 ABLEHNUNG: {'C56', 'A03', 'A07', 'C00', 'C10', 'C32', 'C62', 'C98'}\\ \tt \\ \tt ARGUMENT: {'Dein Vorschlag lässt viel ungenutzten Aufwand übrig - nur 1422 von 3357 werden genutzt. Wir können deutlich mehr wertvolle Gegenstände einbeziehen und unsere Kapazität besser ausschöpfen.'}\\ \tt \\ \tt VORSCHLAG: {'A03', 'A07', 'C98', 'C10', 'C81', 'B38', 'C76', 'C08', 'A83'}\\ \tt \\ \tt ARGUMENT: {'Mein Vorschlag nutzt 3305 von 3357 Aufwand und maximiert damit unsere verfügbare Kapazität. Diese Auswahl enthält sowohl mittlere als auch hochwertige Gegenstände für eine optimale Gesamtausbeute.'} 
	  } 
	   } 
	   } 
	 & \\ 
 

    \theutterance \stepcounter{utterance}  

    & \multicolumn{4}{p{0.6\linewidth}}{\cellcolor[rgb]{0.8,1,0.9}{%
	\makecell[{{p{\linewidth}}}]{% 
	  \tt {\tiny [A$\rangle$GM]}  
	 ABLEHNUNG: {'A03', 'A07', 'C98', 'C10', 'C81', 'B38', 'C76', 'C08', 'A83'}\\ \tt \\ \tt ARGUMENT: {'Dein Vorschlag enthält mehrere sehr aufwändige Gegenstände wie A83, C08 und C81, die zusammen einen großen Teil des Budgets verbrauchen. Ich denke, wir können eine bessere Balance zwischen Aufwand und Nutzen finden.'}\\ \tt \\ \tt VORSCHLAG: {'C56', 'A03', 'A07', 'C00', 'C10', 'C32', 'C62', 'C98', 'B38'}\\ \tt \\ \tt ARGUMENT: {'Dieser Vorschlag nutzt 2940 von 3357 Aufwand und bietet eine gute Mischung aus verschiedenen Gegenstandstypen. B38 bringt zusätzlichen Wert bei moderatem Aufwand, während wir die teuersten Items vermeiden, die das Budget zu stark belasten würden.'} 
	  } 
	   } 
	   } 
	 & & \\ 
 

    \theutterance \stepcounter{utterance}  

    & & \multicolumn{4}{p{0.6\linewidth}}{\cellcolor[rgb]{0.9,0.9,0.9}{%
	\makecell[{{p{\linewidth}}}]{% 
	  \tt {\tiny [GM$\rangle$B]}  
	 ABLEHNUNG: {'A03', 'A07', 'C98', 'C10', 'C81', 'B38', 'C76', 'C08', 'A83'}\\ \tt \\ \tt ARGUMENT: {'Dein Vorschlag enthält mehrere sehr aufwändige Gegenstände wie A83, C08 und C81, die zusammen einen großen Teil des Budgets verbrauchen. Ich denke, wir können eine bessere Balance zwischen Aufwand und Nutzen finden.'}\\ \tt \\ \tt VORSCHLAG: {'C56', 'A03', 'A07', 'C00', 'C10', 'C32', 'C62', 'C98', 'B38'}\\ \tt \\ \tt ARGUMENT: {'Dieser Vorschlag nutzt 2940 von 3357 Aufwand und bietet eine gute Mischung aus verschiedenen Gegenstandstypen. B38 bringt zusätzlichen Wert bei moderatem Aufwand, während wir die teuersten Items vermeiden, die das Budget zu stark belasten würden.'} 
	  } 
	   } 
	   } 
	 & \\ 
 

    \theutterance \stepcounter{utterance}  

    & & & \multicolumn{4}{p{0.6\linewidth}}{\cellcolor[rgb]{1,0.85,0.72}{%
	\makecell[{{p{\linewidth}}}]{% 
	  \tt {\tiny [GM$\langle$B]}  
	 ABLEHNUNG: {'C56', 'A03', 'A07', 'C00', 'C10', 'C32', 'C62', 'C98', 'B38'}\\ \tt \\ \tt ARGUMENT: {'Dein Vorschlag enthält C56 und C32, die relativ viel Aufwand für ihren Nutzen verbrauchen. Außerdem lässt er noch 417 Aufwand ungenutzt - wir sollten diese Kapazität für wertvollere Gegenstände verwenden.'}\\ \tt \\ \tt VORSCHLAG: {'A03', 'A07', 'C00', 'C10', 'C62', 'C98', 'B38', 'C76', 'C81'}\\ \tt \\ \tt ARGUMENT: {'Diese Kombination nutzt 3156 von 3357 Aufwand effizient aus und verzichtet auf die weniger rentablen Gegenstände wie C56. Stattdessen nehmen wir C76 und C81, die trotz höherem Aufwand einen besseren Gesamtwert bieten. Die Balance zwischen Kosten und Nutzen ist hier optimal.'} 
	  } 
	   } 
	   } 
	  \\ 
 

    \theutterance \stepcounter{utterance}  

    & & \multicolumn{4}{p{0.6\linewidth}}{\cellcolor[rgb]{0.9,0.9,0.9}{%
	\makecell[{{p{\linewidth}}}]{% 
	  \tt {\tiny [A$\langle$GM]}  
	 ABLEHNUNG: {'C56', 'A03', 'A07', 'C00', 'C10', 'C32', 'C62', 'C98', 'B38'}\\ \tt \\ \tt ARGUMENT: {'Dein Vorschlag enthält C56 und C32, die relativ viel Aufwand für ihren Nutzen verbrauchen. Außerdem lässt er noch 417 Aufwand ungenutzt - wir sollten diese Kapazität für wertvollere Gegenstände verwenden.'}\\ \tt \\ \tt VORSCHLAG: {'A03', 'A07', 'C00', 'C10', 'C62', 'C98', 'B38', 'C76', 'C81'}\\ \tt \\ \tt ARGUMENT: {'Diese Kombination nutzt 3156 von 3357 Aufwand effizient aus und verzichtet auf die weniger rentablen Gegenstände wie C56. Stattdessen nehmen wir C76 und C81, die trotz höherem Aufwand einen besseren Gesamtwert bieten. Die Balance zwischen Kosten und Nutzen ist hier optimal.'} 
	  } 
	   } 
	   } 
	 & \\ 
 

    \theutterance \stepcounter{utterance}  

    & \multicolumn{4}{p{0.6\linewidth}}{\cellcolor[rgb]{0.8,1,0.9}{%
	\makecell[{{p{\linewidth}}}]{% 
	  \tt {\tiny [A$\rangle$GM]}  
	 ABLEHNUNG: {'A03', 'A07', 'C00', 'C10', 'C62', 'C98', 'B38', 'C76', 'C81'}\\ \tt \\ \tt ARGUMENT: {'C81 hat einen sehr hohen Aufwand von 651 bei fraglichem Nutzen. Außerdem ist C76 mit 506 Aufwand ebenfalls sehr teuer. Wir sollten diese durch mehrere kleinere, effizientere Gegenstände ersetzen, die zusammen mehr Wert bieten.'}\\ \tt \\ \tt VORSCHLAG: {'A03', 'A07', 'C00', 'C10', 'C62', 'C98', 'B38', 'C56', 'C38'}\\ \tt \\ \tt ARGUMENT: {'Dieser Vorschlag ersetzt die teuren C81 und C76 durch C56 und C38, die zusammen nur 830 statt 1157 Aufwand kosten. Das spart 327 Aufwand bei potenziell besserem Gesamtnutzen. Die Kombination nutzt 2883 von 3357 Aufwand und bietet eine diversifizierte Auswahl.'} 
	  } 
	   } 
	   } 
	 & & \\ 
 

    \theutterance \stepcounter{utterance}  

    & & \multicolumn{4}{p{0.6\linewidth}}{\cellcolor[rgb]{0.9,0.9,0.9}{%
	\makecell[{{p{\linewidth}}}]{% 
	  \tt {\tiny [GM$\rangle$B]}  
	 ABLEHNUNG: {'A03', 'A07', 'C00', 'C10', 'C62', 'C98', 'B38', 'C76', 'C81'}\\ \tt \\ \tt ARGUMENT: {'C81 hat einen sehr hohen Aufwand von 651 bei fraglichem Nutzen. Außerdem ist C76 mit 506 Aufwand ebenfalls sehr teuer. Wir sollten diese durch mehrere kleinere, effizientere Gegenstände ersetzen, die zusammen mehr Wert bieten.'}\\ \tt \\ \tt VORSCHLAG: {'A03', 'A07', 'C00', 'C10', 'C62', 'C98', 'B38', 'C56', 'C38'}\\ \tt \\ \tt ARGUMENT: {'Dieser Vorschlag ersetzt die teuren C81 und C76 durch C56 und C38, die zusammen nur 830 statt 1157 Aufwand kosten. Das spart 327 Aufwand bei potenziell besserem Gesamtnutzen. Die Kombination nutzt 2883 von 3357 Aufwand und bietet eine diversifizierte Auswahl.'} 
	  } 
	   } 
	   } 
	 & \\ 
 

    \theutterance \stepcounter{utterance}  

    & & & \multicolumn{4}{p{0.6\linewidth}}{\cellcolor[rgb]{1,0.85,0.72}{%
	\makecell[{{p{\linewidth}}}]{% 
	  \tt {\tiny [GM$\langle$B]}  
	 ABLEHNUNG: {'A03', 'A07', 'C00', 'C10', 'C62', 'C98', 'B38', 'C56', 'C38'}\\ \tt \\ \tt ARGUMENT: {'Du hast C38 hinzugefügt, der mit 492 Aufwand sehr kostspielig ist, während C56 einen schlechten Aufwand-Nutzen-Wert hat. Wir lassen immer noch 474 Aufwand ungenutzt, den wir für strategisch wertvollere Gegenstände einsetzen könnten.'}\\ \tt \\ \tt VORSCHLAG: {'A03', 'A07', 'C00', 'C10', 'C62', 'C98', 'B38', 'C08', 'C03'}\\ \tt \\ \tt ARGUMENT: {'Diese Kombination tauscht die ineffizienten C56 und C38 gegen C08 und C03 aus, die zwar mehr Aufwand kosten, aber strategisch wertvoller sind. Mit 3348 von 3357 genutztem Aufwand maximieren wir unsere Kapazität nahezu vollständig und erhalten eine starke Gegenstandsauswahl.'} 
	  } 
	   } 
	   } 
	  \\ 
 

    \theutterance \stepcounter{utterance}  

    & & \multicolumn{4}{p{0.6\linewidth}}{\cellcolor[rgb]{0.9,0.9,0.9}{%
	\makecell[{{p{\linewidth}}}]{% 
	  \tt {\tiny [A$\langle$GM]}  
	 ABLEHNUNG: {'A03', 'A07', 'C00', 'C10', 'C62', 'C98', 'B38', 'C56', 'C38'}\\ \tt \\ \tt ARGUMENT: {'Du hast C38 hinzugefügt, der mit 492 Aufwand sehr kostspielig ist, während C56 einen schlechten Aufwand-Nutzen-Wert hat. Wir lassen immer noch 474 Aufwand ungenutzt, den wir für strategisch wertvollere Gegenstände einsetzen könnten.'}\\ \tt \\ \tt VORSCHLAG: {'A03', 'A07', 'C00', 'C10', 'C62', 'C98', 'B38', 'C08', 'C03'}\\ \tt \\ \tt ARGUMENT: {'Diese Kombination tauscht die ineffizienten C56 und C38 gegen C08 und C03 aus, die zwar mehr Aufwand kosten, aber strategisch wertvoller sind. Mit 3348 von 3357 genutztem Aufwand maximieren wir unsere Kapazität nahezu vollständig und erhalten eine starke Gegenstandsauswahl.'} 
	  } 
	   } 
	   } 
	 & \\ 
 

    \theutterance \stepcounter{utterance}  

    & \multicolumn{4}{p{0.6\linewidth}}{\cellcolor[rgb]{0.8,1,0.9}{%
	\makecell[{{p{\linewidth}}}]{% 
	  \tt {\tiny [A$\rangle$GM]}  
	 ABLEHNUNG: {'A03', 'A07', 'C00', 'C10', 'C62', 'C98', 'B38', 'C08', 'C03'}\\ \tt \\ \tt ARGUMENT: {'C08 und C03 sind zusammen extrem teuer mit 1565 Aufwand und verbrauchen fast die Hälfte unseres gesamten Budgets. Diese beiden Gegenstände blockieren zu viel Kapazität für andere wertvolle Items. Wir brauchen eine ausgewogenere Verteilung.'}\\ \tt \\ \tt VORSCHLAG: {'A03', 'A07', 'C00', 'C10', 'C62', 'C98', 'B38', 'C81', 'C38'}\\ \tt \\ \tt ARGUMENT: {'Dieser Kompromiss kombiniert die besten Elemente unserer bisherigen Diskussion. C81 und C38 bieten soliden Wert bei moderaterem Aufwand als die C08/C03 Kombination. Mit 3143 genutztem Aufwand haben wir noch 214 Puffer und eine diversifizierte Auswahl, die beide Seiten zufriedenstellen sollte.'} 
	  } 
	   } 
	   } 
	 & & \\ 
 

    \theutterance \stepcounter{utterance}  

    & & \multicolumn{4}{p{0.6\linewidth}}{\cellcolor[rgb]{0.9,0.9,0.9}{%
	\makecell[{{p{\linewidth}}}]{% 
	  \tt {\tiny [GM$\rangle$B]}  
	 ABLEHNUNG: {'A03', 'A07', 'C00', 'C10', 'C62', 'C98', 'B38', 'C08', 'C03'}\\ \tt \\ \tt ARGUMENT: {'C08 und C03 sind zusammen extrem teuer mit 1565 Aufwand und verbrauchen fast die Hälfte unseres gesamten Budgets. Diese beiden Gegenstände blockieren zu viel Kapazität für andere wertvolle Items. Wir brauchen eine ausgewogenere Verteilung.'}\\ \tt \\ \tt VORSCHLAG: {'A03', 'A07', 'C00', 'C10', 'C62', 'C98', 'B38', 'C81', 'C38'}\\ \tt \\ \tt ARGUMENT: {'Dieser Kompromiss kombiniert die besten Elemente unserer bisherigen Diskussion. C81 und C38 bieten soliden Wert bei moderaterem Aufwand als die C08/C03 Kombination. Mit 3143 genutztem Aufwand haben wir noch 214 Puffer und eine diversifizierte Auswahl, die beide Seiten zufriedenstellen sollte.'} 
	  } 
	   } 
	   } 
	 & \\ 
 

    \theutterance \stepcounter{utterance}  

    & & & \multicolumn{4}{p{0.6\linewidth}}{\cellcolor[rgb]{1,0.85,0.72}{%
	\makecell[{{p{\linewidth}}}]{% 
	  \tt {\tiny [GM$\langle$B]}  
	 ZUSTIMMUNG: {'A03', 'A07', 'C00', 'C10', 'C62', 'C98', 'B38', 'C81', 'C38'}\\ \tt \\ \tt ARGUMENT: {'Dieser Vorschlag stellt einen guten Kompromiss dar. Die Kombination nutzt 3143 von 3357 Aufwand effizient aus und enthält eine ausgewogene Mischung von Gegenständen mit verschiedenen Aufwandsstufen. Die Auswahl vermeidet die extremst teuren Gegenstände, während sie trotzdem hochwertige Items wie C81 einschließt.'} 
	  } 
	   } 
	   } 
	  \\ 
 

    \theutterance \stepcounter{utterance}  

    & & & \multicolumn{2}{p{0.3\linewidth}}{\cellcolor[rgb]{0.95,0.95,0.95}{%
	\makecell[{{p{\linewidth}}}]{% 
	  \tt {\tiny [GM$|$GM]}  
	 game successful 
	  } 
	   } 
	   } 
	 & & \\ 
 

    \theutterance \stepcounter{utterance}  

    & & & \multicolumn{2}{p{0.3\linewidth}}{\cellcolor[rgb]{0.95,0.95,0.95}{%
	\makecell[{{p{\linewidth}}}]{% 
	  \tt {\tiny [GM$|$GM]}  
	 end game 
	  } 
	   } 
	   } 
	 & & \\ 
 

\end{supertabular}
}

\end{document}
