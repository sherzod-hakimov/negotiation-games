
\documentclass{article}
\usepackage{colortbl}
\usepackage{makecell}
\usepackage{multirow}
\usepackage{supertabular}

\begin{document}

\newcounter{utterance}

\centering \large Interaction Transcript for game `hot\_air\_balloon', experiment `air\_balloon\_survival\_de\_reasoning off\_hard', episode 4 with claude{-}sonnet{-}4{-}20250514{-}no{-}reasoning{-}t1.0.
\vspace{24pt}

{ \footnotesize  \setcounter{utterance}{1}
\setlength{\tabcolsep}{0pt}
\begin{supertabular}{c@{$\;$}|p{.15\linewidth}@{}p{.15\linewidth}p{.15\linewidth}p{.15\linewidth}p{.15\linewidth}p{.15\linewidth}}
    \# & $\;$A & \multicolumn{4}{c}{Game Master} & $\;\:$B\\
    \hline

    \theutterance \stepcounter{utterance}  
    & & \multicolumn{4}{p{0.6\linewidth}}{
        \cellcolor[rgb]{0.9,0.9,0.9}{
            \makecell[{{p{\linewidth}}}]{
                \texttt{\tiny{[P1$\langle$GM]}}
                \texttt{Du nimmst an einem kollaborativen Verhandlungspiel Teil.} \\
\\ 
\texttt{Zusammen mit einem anderen Teilnehmer musst du dich auf eine Reihe von Gegenständen entscheiden, die behalten werden. Jeder von euch hat eine persönliche Verteilung über die Wichtigkeit der einzelnen Gegenstände. Jeder von euch hat eine eigene Meinung darüber, wie wichtig jeder einzelne Gegenstand ist (Gegenstandswichtigkeit). Du kennst die Wichtigkeitsverteilung des anderen Spielers nicht. Zusätzlich siehst du, wie viel Aufwand jeder Gegenstand verursacht.  } \\
\texttt{Ihr dürft euch nur auf eine Reihe von Gegenständen einigen, wenn der Gesamtaufwand der ausgewählten Gegenstände den Maximalaufwand nicht überschreitet:} \\
\\ 
\texttt{Maximalaufwand = 2984} \\
\\ 
\texttt{Hier sind die einzelnen Aufwände der Gegenstände:} \\
\\ 
\texttt{Aufwand der Gegenstände = \{"C76": 350, "C38": 276, "C56": 616, "A03": 737, "A07": 531, "A83": 389, "C98": 24, "C08": 125, "C62": 338, "C00": 356, "C10": 143, "C81": 117, "B38": 257, "C03": 921, "C32": 789\}} \\
\\ 
\texttt{Hier ist deine persönliche Verteilung der Wichtigkeit der einzelnen Gegenstände:} \\
\\ 
\texttt{Werte der Gegenstandswichtigkeit = \{"C76": 138, "C38": 583, "C56": 868, "A03": 822, "A07": 783, "A83": 65, "C98": 262, "C08": 121, "C62": 508, "C00": 780, "C10": 461, "C81": 484, "B38": 668, "C03": 389, "C32": 808\}} \\
\\ 
\texttt{Ziel:} \\
\\ 
\texttt{Dein Ziel ist es, eine Reihe von Gegenständen auszuhandeln, die dir möglichst viel bringt (d. h. Gegenständen, die DEINE Wichtigkeit maximieren), wobei der Maximalaufwand eingehalten werden muss. Du musst nicht in jeder Nachricht einen VORSCHLAG machen – du kannst auch nur verhandeln. Alle Taktiken sind erlaubt!} \\
\\ 
\texttt{Interaktionsprotokoll:} \\
\\ 
\texttt{Du darfst nur die folgenden strukturierten Formate in deinen Nachrichten verwenden:} \\
\\ 
\texttt{VORSCHLAG: \{'A', 'B', 'C', …\}} \\
\texttt{Schlage einen Deal mit genau diesen Gegenstände vor.} \\
\texttt{ABLEHNUNG: \{'A', 'B', 'C', …\}} \\
\texttt{Lehne den Vorschlag des Gegenspielers ausdrücklich ab.} \\
\texttt{ARGUMENT: \{'...'\}} \\
\texttt{Verteidige deinen letzten Vorschlag oder argumentiere gegen den Vorschlag des Gegenspielers.} \\
\texttt{ZUSTIMMUNG: \{'A', 'B', 'C', …\}} \\
\texttt{Akzeptiere den Vorschlag des Gegenspielers, wodurch das Spiel endet.} \\
\\ 
\\ 
\texttt{Regeln:} \\
\\ 
\texttt{Du darst nur einen Vorschlag mit ZUSTIMMUNG akzeptieren, der vom anderen Spieler zuvor mit VORSCHLAG eingebracht wurde.} \\
\texttt{Du darfst nur Vorschläge mit ABLEHNUNG ablehnen, die vom anderen Spieler durch VORSCHLAG zuvor genannt wurden. } \\
\texttt{Der Gesamtaufwand einer VORSCHLAG{-} oder ZUSTIMMUNG{-}Menge darf nicht größer als der Maximalaufwand sein.  } \\
\texttt{Offenbare deine versteckte Wichtigkeitsverteilung nicht.} \\
\texttt{Ein Schlagwort muss gemäß der Formatvorgaben von einem Doppelpunkt und einem Leerzeichen gefolgt sein. Das Argument muss eine Python{-}Menge mit 0 oder mehr Strings sein.  } \\
\texttt{Die Form muss also SCHLAGWORT: \{...\} sein.} \\
\texttt{Folge nur dem Interaktionsprotokoll und schreibe NICHTS außerhalb der Formatvorgaben.} \\
\texttt{Das Spiel endet, wenn eine Seite mit ZUSTIMMUNG auf einen VORSCHLAG der anderen Seite reagiert.  } \\
\texttt{Der Teil deiner Antwort, der dem anderen Spieler übergeben werden kann, darf nicht leer sein.  } \\
\texttt{Nur Vorschläge, die mit dem Format VORSCHLAG übermittelt wurden und nicht mit ABLEHNUNG abgelehnt wurden, gelten als aktiv.  } \\
\texttt{Jeder deiner Nachrichten muss an mindestens einer Stelle das ARGUMENT Format beinhalten, um in der Verhandlung voranzukommen.} \\
            }
        }
    }
    & \\ \\

    \theutterance \stepcounter{utterance}  
    & \multicolumn{4}{p{0.6\linewidth}}{
        \cellcolor[rgb]{0.9,0.9,0.9}{
            \makecell[{{p{\linewidth}}}]{
                \texttt{\tiny{[P1$\rangle$GM]}}
                \texttt{VORSCHLAG: \{'C56', 'A03', 'A07', 'C00', 'C10', 'C81', 'B38', 'C32'\}} \\
\\ 
\texttt{ARGUMENT: \{'Ich schlage eine ausgewogene Mischung von Gegenständen vor, die sowohl hochwertige als auch kosteneffiziente Items umfasst. Diese Kombination nutzt unseren verfügbaren Aufwand optimal aus und berücksichtigt verschiedene Kategorien, was für beide Seiten vorteilhaft sein sollte.'\}} \\
            }
        }
    }
    & & \\ \\

    \theutterance \stepcounter{utterance}  
    & & \multicolumn{4}{p{0.6\linewidth}}{
        \cellcolor[rgb]{0.9,0.9,0.9}{
            \makecell[{{p{\linewidth}}}]{
                \texttt{\tiny{[GM$\rangle$P2]}}
                \texttt{Du nimmst an einem kollaborativen Verhandlungspiel Teil.} \\
\\ 
\texttt{Zusammen mit einem anderen Teilnehmer musst du dich auf eine Reihe von Gegenständen entscheiden, die behalten werden. Jeder von euch hat eine persönliche Verteilung über die Wichtigkeit der einzelnen Gegenstände. Jeder von euch hat eine eigene Meinung darüber, wie wichtig jeder einzelne Gegenstand ist (Gegenstandswichtigkeit). Du kennst die Wichtigkeitsverteilung des anderen Spielers nicht. Zusätzlich siehst du, wie viel Aufwand jeder Gegenstand verursacht.  } \\
\texttt{Ihr dürft euch nur auf eine Reihe von Gegenständen einigen, wenn der Gesamtaufwand der ausgewählten Gegenstände den Maximalaufwand nicht überschreitet:} \\
\\ 
\texttt{Maximalaufwand = 2984} \\
\\ 
\texttt{Hier sind die einzelnen Aufwände der Gegenstände:} \\
\\ 
\texttt{Aufwand der Gegenstände = \{"C76": 350, "C38": 276, "C56": 616, "A03": 737, "A07": 531, "A83": 389, "C98": 24, "C08": 125, "C62": 338, "C00": 356, "C10": 143, "C81": 117, "B38": 257, "C03": 921, "C32": 789\}} \\
\\ 
\texttt{Hier ist deine persönliche Verteilung der Wichtigkeit der einzelnen Gegenstände:} \\
\\ 
\texttt{Werte der Gegenstandswichtigkeit = \{"C56": 65, "A03": 121, "C32": 138, "A07": 262, "C00": 389, "B38": 461, "C38": 484, "C62": 508, "C81": 583, "C10": 668, "C03": 780, "C98": 783, "C76": 808, "C08": 822, "A83": 868\}} \\
\\ 
\texttt{Ziel:} \\
\\ 
\texttt{Dein Ziel ist es, eine Reihe von Gegenständen auszuhandeln, die dir möglichst viel bringt (d. h. Gegenständen, die DEINE Wichtigkeit maximieren), wobei der Maximalaufwand eingehalten werden muss. Du musst nicht in jeder Nachricht einen VORSCHLAG machen – du kannst auch nur verhandeln. Alle Taktiken sind erlaubt!} \\
\\ 
\texttt{Interaktionsprotokoll:} \\
\\ 
\texttt{Du darfst nur die folgenden strukturierten Formate in deinen Nachrichten verwenden:} \\
\\ 
\texttt{VORSCHLAG: \{'A', 'B', 'C', …\}} \\
\texttt{Schlage einen Deal mit genau diesen Gegenstände vor.} \\
\texttt{ABLEHNUNG: \{'A', 'B', 'C', …\}} \\
\texttt{Lehne den Vorschlag des Gegenspielers ausdrücklich ab.} \\
\texttt{ARGUMENT: \{'...'\}} \\
\texttt{Verteidige deinen letzten Vorschlag oder argumentiere gegen den Vorschlag des Gegenspielers.} \\
\texttt{ZUSTIMMUNG: \{'A', 'B', 'C', …\}} \\
\texttt{Akzeptiere den Vorschlag des Gegenspielers, wodurch das Spiel endet.} \\
\\ 
\\ 
\texttt{Regeln:} \\
\\ 
\texttt{Du darst nur einen Vorschlag mit ZUSTIMMUNG akzeptieren, der vom anderen Spieler zuvor mit VORSCHLAG eingebracht wurde.} \\
\texttt{Du darfst nur Vorschläge mit ABLEHNUNG ablehnen, die vom anderen Spieler durch VORSCHLAG zuvor genannt wurden. } \\
\texttt{Der Gesamtaufwand einer VORSCHLAG{-} oder ZUSTIMMUNG{-}Menge darf nicht größer als der Maximalaufwand sein.  } \\
\texttt{Offenbare deine versteckte Wichtigkeitsverteilung nicht.} \\
\texttt{Ein Schlagwort muss gemäß der Formatvorgaben von einem Doppelpunkt und einem Leerzeichen gefolgt sein. Das Argument muss eine Python{-}Menge mit 0 oder mehr Strings sein.  } \\
\texttt{Die Form muss also SCHLAGWORT: \{...\} sein.} \\
\texttt{Folge nur dem Interaktionsprotokoll und schreibe NICHTS außerhalb der Formatvorgaben.} \\
\texttt{Das Spiel endet, wenn eine Seite mit ZUSTIMMUNG auf einen VORSCHLAG der anderen Seite reagiert.  } \\
\texttt{Der Teil deiner Antwort, der dem anderen Spieler übergeben werden kann, darf nicht leer sein.  } \\
\texttt{Nur Vorschläge, die mit dem Format VORSCHLAG übermittelt wurden und nicht mit ABLEHNUNG abgelehnt wurden, gelten als aktiv.  } \\
\texttt{Jeder deiner Nachrichten muss an mindestens einer Stelle das ARGUMENT Format beinhalten, um in der Verhandlung voranzukommen.} \\
\\ 
\\ 
\texttt{Du erhältst nun die erste Nachricht des anderen Spielers.} \\
\\ 
\texttt{VORSCHLAG: \{'C56', 'A03', 'A07', 'C00', 'C10', 'C81', 'B38', 'C32'\}} \\
\\ 
\texttt{ARGUMENT: \{'Ich schlage eine ausgewogene Mischung von Gegenständen vor, die sowohl hochwertige als auch kosteneffiziente Items umfasst. Diese Kombination nutzt unseren verfügbaren Aufwand optimal aus und berücksichtigt verschiedene Kategorien, was für beide Seiten vorteilhaft sein sollte.'\}} \\
            }
        }
    }
    & \\ \\

    \theutterance \stepcounter{utterance}  
    & & & \multicolumn{4}{p{0.6\linewidth}}{
        \cellcolor[rgb]{0.9,0.9,0.9}{
            \makecell[{{p{\linewidth}}}]{
                \texttt{\tiny{[GM$\langle$P2]}}
                \texttt{ABLEHNUNG: \{'C56', 'A03', 'A07', 'C00', 'C10', 'C81', 'B38', 'C32'\}} \\
\\ 
\texttt{ARGUMENT: \{'Dieser Vorschlag enthält mehrere Gegenstände mit sehr hohem Aufwand wie A03 und C32, aber relativ geringem Nutzen. Wir können eine effizientere Kombination finden, die mehr Wert bei geringerem Gesamtaufwand bietet.'\}} \\
\\ 
\texttt{VORSCHLAG: \{'C98', 'C08', 'C81', 'C10', 'A83', 'C76', 'C38', 'B38'\}} \\
            }
        }
    }
     \\ \\

    \theutterance \stepcounter{utterance}  
    & & \multicolumn{4}{p{0.6\linewidth}}{
        \cellcolor[rgb]{0.9,0.9,0.9}{
            \makecell[{{p{\linewidth}}}]{
                \texttt{\tiny{[P1$\langle$GM]}}
                \texttt{ABLEHNUNG: \{'C56', 'A03', 'A07', 'C00', 'C10', 'C81', 'B38', 'C32'\}} \\
\\ 
\texttt{ARGUMENT: \{'Dieser Vorschlag enthält mehrere Gegenstände mit sehr hohem Aufwand wie A03 und C32, aber relativ geringem Nutzen. Wir können eine effizientere Kombination finden, die mehr Wert bei geringerem Gesamtaufwand bietet.'\}} \\
\\ 
\texttt{VORSCHLAG: \{'C98', 'C08', 'C81', 'C10', 'A83', 'C76', 'C38', 'B38'\}} \\
            }
        }
    }
    & \\ \\

    \theutterance \stepcounter{utterance}  
    & \multicolumn{4}{p{0.6\linewidth}}{
        \cellcolor[rgb]{0.9,0.9,0.9}{
            \makecell[{{p{\linewidth}}}]{
                \texttt{\tiny{[P1$\rangle$GM]}}
                \texttt{ABLEHNUNG: \{'C98', 'C08', 'C81', 'C10', 'A83', 'C76', 'C38', 'B38'\}} \\
\\ 
\texttt{ARGUMENT: \{'Ihr Vorschlag konzentriert sich zu stark auf Gegenstände mit niedrigem Aufwand, aber vernachlässigt einige wertvolle Items mit besserem Nutzen{-}Aufwand{-}Verhältnis. Wir sollten strategisch wichtigere Gegenstände einbeziehen.'\}} \\
\\ 
\texttt{VORSCHLAG: \{'C56', 'A03', 'C38', 'C62', 'C10', 'C81', 'B38', 'C98'\}} \\
            }
        }
    }
    & & \\ \\

    \theutterance \stepcounter{utterance}  
    & & \multicolumn{4}{p{0.6\linewidth}}{
        \cellcolor[rgb]{0.9,0.9,0.9}{
            \makecell[{{p{\linewidth}}}]{
                \texttt{\tiny{[GM$\rangle$P2]}}
                \texttt{ABLEHNUNG: \{'C98', 'C08', 'C81', 'C10', 'A83', 'C76', 'C38', 'B38'\}} \\
\\ 
\texttt{ARGUMENT: \{'Ihr Vorschlag konzentriert sich zu stark auf Gegenstände mit niedrigem Aufwand, aber vernachlässigt einige wertvolle Items mit besserem Nutzen{-}Aufwand{-}Verhältnis. Wir sollten strategisch wichtigere Gegenstände einbeziehen.'\}} \\
\\ 
\texttt{VORSCHLAG: \{'C56', 'A03', 'C38', 'C62', 'C10', 'C81', 'B38', 'C98'\}} \\
            }
        }
    }
    & \\ \\

    \theutterance \stepcounter{utterance}  
    & & & \multicolumn{4}{p{0.6\linewidth}}{
        \cellcolor[rgb]{0.9,0.9,0.9}{
            \makecell[{{p{\linewidth}}}]{
                \texttt{\tiny{[GM$\langle$P2]}}
                \texttt{ABLEHNUNG: \{'C56', 'A03', 'C38', 'C62', 'C10', 'C81', 'B38', 'C98'\}} \\
\\ 
\texttt{ARGUMENT: \{'Ihr neuer Vorschlag enthält immer noch A03 mit extrem hohem Aufwand von 737, während C56 ebenfalls sehr aufwändig ist. Diese beiden Gegenstände verbrauchen fast die Hälfte unseres Budgets. Wir können bessere Kombinationen finden.'\}} \\
\\ 
\texttt{VORSCHLAG: \{'C08', 'C81', 'C10', 'A83', 'C76', 'C38', 'C62', 'C00'\}} \\
            }
        }
    }
     \\ \\

    \theutterance \stepcounter{utterance}  
    & & \multicolumn{4}{p{0.6\linewidth}}{
        \cellcolor[rgb]{0.9,0.9,0.9}{
            \makecell[{{p{\linewidth}}}]{
                \texttt{\tiny{[P1$\langle$GM]}}
                \texttt{ABLEHNUNG: \{'C56', 'A03', 'C38', 'C62', 'C10', 'C81', 'B38', 'C98'\}} \\
\\ 
\texttt{ARGUMENT: \{'Ihr neuer Vorschlag enthält immer noch A03 mit extrem hohem Aufwand von 737, während C56 ebenfalls sehr aufwändig ist. Diese beiden Gegenstände verbrauchen fast die Hälfte unseres Budgets. Wir können bessere Kombinationen finden.'\}} \\
\\ 
\texttt{VORSCHLAG: \{'C08', 'C81', 'C10', 'A83', 'C76', 'C38', 'C62', 'C00'\}} \\
            }
        }
    }
    & \\ \\

    \theutterance \stepcounter{utterance}  
    & \multicolumn{4}{p{0.6\linewidth}}{
        \cellcolor[rgb]{0.9,0.9,0.9}{
            \makecell[{{p{\linewidth}}}]{
                \texttt{\tiny{[P1$\rangle$GM]}}
                \texttt{ABLEHNUNG: \{'C08', 'C81', 'C10', 'A83', 'C76', 'C38', 'C62', 'C00'\}} \\
\\ 
\texttt{ARGUMENT: \{'Ihr Vorschlag schließt wichtige hochwertige Gegenstände aus und fokussiert sich zu sehr auf niedrigere Aufwandskategorien. Eine ausgewogenere Lösung würde bessere Gesamtergebnisse erzielen.'\}} \\
\\ 
\texttt{VORSCHLAG: \{'A07', 'C38', 'C62', 'C00', 'C10', 'C81', 'B38', 'C98'\}} \\
            }
        }
    }
    & & \\ \\

    \theutterance \stepcounter{utterance}  
    & & \multicolumn{4}{p{0.6\linewidth}}{
        \cellcolor[rgb]{0.9,0.9,0.9}{
            \makecell[{{p{\linewidth}}}]{
                \texttt{\tiny{[GM$\rangle$P2]}}
                \texttt{ABLEHNUNG: \{'C08', 'C81', 'C10', 'A83', 'C76', 'C38', 'C62', 'C00'\}} \\
\\ 
\texttt{ARGUMENT: \{'Ihr Vorschlag schließt wichtige hochwertige Gegenstände aus und fokussiert sich zu sehr auf niedrigere Aufwandskategorien. Eine ausgewogenere Lösung würde bessere Gesamtergebnisse erzielen.'\}} \\
\\ 
\texttt{VORSCHLAG: \{'A07', 'C38', 'C62', 'C00', 'C10', 'C81', 'B38', 'C98'\}} \\
            }
        }
    }
    & \\ \\

    \theutterance \stepcounter{utterance}  
    & & & \multicolumn{4}{p{0.6\linewidth}}{
        \cellcolor[rgb]{0.9,0.9,0.9}{
            \makecell[{{p{\linewidth}}}]{
                \texttt{\tiny{[GM$\langle$P2]}}
                \texttt{ABLEHNUNG: \{'A07', 'C38', 'C62', 'C00', 'C10', 'C81', 'B38', 'C98'\}} \\
\\ 
\texttt{ARGUMENT: \{'Obwohl A07 einen hohen Aufwand von 531 hat, scheinen Sie konsequent darauf zu bestehen. Ich sehe jedoch bessere Alternativen mit günstigerem Aufwand{-}Nutzen{-}Verhältnis.'\}} \\
\\ 
\texttt{VORSCHLAG: \{'C08', 'C98', 'C81', 'C10', 'C76', 'A83', 'C38', 'C62', 'B38'\}} \\
            }
        }
    }
     \\ \\

    \theutterance \stepcounter{utterance}  
    & & \multicolumn{4}{p{0.6\linewidth}}{
        \cellcolor[rgb]{0.9,0.9,0.9}{
            \makecell[{{p{\linewidth}}}]{
                \texttt{\tiny{[P1$\langle$GM]}}
                \texttt{ABLEHNUNG: \{'A07', 'C38', 'C62', 'C00', 'C10', 'C81', 'B38', 'C98'\}} \\
\\ 
\texttt{ARGUMENT: \{'Obwohl A07 einen hohen Aufwand von 531 hat, scheinen Sie konsequent darauf zu bestehen. Ich sehe jedoch bessere Alternativen mit günstigerem Aufwand{-}Nutzen{-}Verhältnis.'\}} \\
\\ 
\texttt{VORSCHLAG: \{'C08', 'C98', 'C81', 'C10', 'C76', 'A83', 'C38', 'C62', 'B38'\}} \\
            }
        }
    }
    & \\ \\

    \theutterance \stepcounter{utterance}  
    & \multicolumn{4}{p{0.6\linewidth}}{
        \cellcolor[rgb]{0.9,0.9,0.9}{
            \makecell[{{p{\linewidth}}}]{
                \texttt{\tiny{[P1$\rangle$GM]}}
                \texttt{ABLEHNUNG: \{'C08', 'C98', 'C81', 'C10', 'C76', 'A83', 'C38', 'C62', 'B38'\}} \\
\\ 
\texttt{ARGUMENT: \{'Ihr Vorschlag hat zwar einen geringen Gesamtaufwand, aber wir nutzen unser Budget nicht optimal aus. Wir haben noch über 1300 Aufwandspunkte übrig, die wir für wertvollere Gegenstände einsetzen könnten.'\}} \\
\\ 
\texttt{VORSCHLAG: \{'C56', 'C38', 'C00', 'C10', 'C81', 'B38', 'C98', 'A83'\}} \\
            }
        }
    }
    & & \\ \\

    \theutterance \stepcounter{utterance}  
    & & \multicolumn{4}{p{0.6\linewidth}}{
        \cellcolor[rgb]{0.9,0.9,0.9}{
            \makecell[{{p{\linewidth}}}]{
                \texttt{\tiny{[GM$\rangle$P2]}}
                \texttt{ABLEHNUNG: \{'C08', 'C98', 'C81', 'C10', 'C76', 'A83', 'C38', 'C62', 'B38'\}} \\
\\ 
\texttt{ARGUMENT: \{'Ihr Vorschlag hat zwar einen geringen Gesamtaufwand, aber wir nutzen unser Budget nicht optimal aus. Wir haben noch über 1300 Aufwandspunkte übrig, die wir für wertvollere Gegenstände einsetzen könnten.'\}} \\
\\ 
\texttt{VORSCHLAG: \{'C56', 'C38', 'C00', 'C10', 'C81', 'B38', 'C98', 'A83'\}} \\
            }
        }
    }
    & \\ \\

    \theutterance \stepcounter{utterance}  
    & & & \multicolumn{4}{p{0.6\linewidth}}{
        \cellcolor[rgb]{0.9,0.9,0.9}{
            \makecell[{{p{\linewidth}}}]{
                \texttt{\tiny{[GM$\langle$P2]}}
                \texttt{ABLEHNUNG: \{'C56', 'C38', 'C00', 'C10', 'C81', 'B38', 'C98', 'A83'\}} \\
\\ 
\texttt{ARGUMENT: \{'Sie haben C56 wieder hinzugefügt, obwohl dieser einen sehr hohen Aufwand von 616 hat. Ich sehe eine Möglichkeit für einen Kompromiss, bei dem wir beide profitieren können.'\}} \\
\\ 
\texttt{VORSCHLAG: \{'C03', 'C08', 'C98', 'C81', 'C10', 'C76', 'C38', 'B38'\}} \\
            }
        }
    }
     \\ \\

    \theutterance \stepcounter{utterance}  
    & & \multicolumn{4}{p{0.6\linewidth}}{
        \cellcolor[rgb]{0.9,0.9,0.9}{
            \makecell[{{p{\linewidth}}}]{
                \texttt{\tiny{[P1$\langle$GM]}}
                \texttt{ABLEHNUNG: \{'C56', 'C38', 'C00', 'C10', 'C81', 'B38', 'C98', 'A83'\}} \\
\\ 
\texttt{ARGUMENT: \{'Sie haben C56 wieder hinzugefügt, obwohl dieser einen sehr hohen Aufwand von 616 hat. Ich sehe eine Möglichkeit für einen Kompromiss, bei dem wir beide profitieren können.'\}} \\
\\ 
\texttt{VORSCHLAG: \{'C03', 'C08', 'C98', 'C81', 'C10', 'C76', 'C38', 'B38'\}} \\
            }
        }
    }
    & \\ \\

    \theutterance \stepcounter{utterance}  
    & \multicolumn{4}{p{0.6\linewidth}}{
        \cellcolor[rgb]{0.9,0.9,0.9}{
            \makecell[{{p{\linewidth}}}]{
                \texttt{\tiny{[P1$\rangle$GM]}}
                \texttt{ABLEHNUNG: \{'C03', 'C08', 'C98', 'C81', 'C10', 'C76', 'C38', 'B38'\}} \\
\\ 
\texttt{ARGUMENT: \{'C03 hat einen extrem hohen Aufwand von 921 und lässt wenig Raum für andere wertvolle Gegenstände. Lassen Sie uns einen ausgewogeneren Ansatz versuchen, der mehr Flexibilität bietet.'\}} \\
\\ 
\texttt{VORSCHLAG: \{'A07', 'C38', 'C00', 'C10', 'C81', 'C98', 'C62'\}} \\
            }
        }
    }
    & & \\ \\

    \theutterance \stepcounter{utterance}  
    & & \multicolumn{4}{p{0.6\linewidth}}{
        \cellcolor[rgb]{0.9,0.9,0.9}{
            \makecell[{{p{\linewidth}}}]{
                \texttt{\tiny{[GM$\rangle$P2]}}
                \texttt{ABLEHNUNG: \{'C03', 'C08', 'C98', 'C81', 'C10', 'C76', 'C38', 'B38'\}} \\
\\ 
\texttt{ARGUMENT: \{'C03 hat einen extrem hohen Aufwand von 921 und lässt wenig Raum für andere wertvolle Gegenstände. Lassen Sie uns einen ausgewogeneren Ansatz versuchen, der mehr Flexibilität bietet.'\}} \\
\\ 
\texttt{VORSCHLAG: \{'A07', 'C38', 'C00', 'C10', 'C81', 'C98', 'C62'\}} \\
            }
        }
    }
    & \\ \\

    \theutterance \stepcounter{utterance}  
    & & & \multicolumn{4}{p{0.6\linewidth}}{
        \cellcolor[rgb]{0.9,0.9,0.9}{
            \makecell[{{p{\linewidth}}}]{
                \texttt{\tiny{[GM$\langle$P2]}}
                \texttt{ABLEHNUNG: \{'A07', 'C38', 'C00', 'C10', 'C81', 'C98', 'C62'\}} \\
\\ 
\texttt{ARGUMENT: \{'Dieser Vorschlag nutzt nur etwa 2000 von 2984 verfügbaren Aufwandspunkten. Wir verschenken fast 1000 Punkte Potenzial. Lassen Sie uns das Budget besser ausschöpfen.'\}} \\
\\ 
\texttt{VORSCHLAG: \{'A07', 'C38', 'C00', 'C10', 'C81', 'C98', 'C62', 'A83', 'C76'\}} \\
            }
        }
    }
     \\ \\

    \theutterance \stepcounter{utterance}  
    & & \multicolumn{4}{p{0.6\linewidth}}{
        \cellcolor[rgb]{0.9,0.9,0.9}{
            \makecell[{{p{\linewidth}}}]{
                \texttt{\tiny{[P1$\langle$GM]}}
                \texttt{ABLEHNUNG: \{'A07', 'C38', 'C00', 'C10', 'C81', 'C98', 'C62'\}} \\
\\ 
\texttt{ARGUMENT: \{'Dieser Vorschlag nutzt nur etwa 2000 von 2984 verfügbaren Aufwandspunkten. Wir verschenken fast 1000 Punkte Potenzial. Lassen Sie uns das Budget besser ausschöpfen.'\}} \\
\\ 
\texttt{VORSCHLAG: \{'A07', 'C38', 'C00', 'C10', 'C81', 'C98', 'C62', 'A83', 'C76'\}} \\
            }
        }
    }
    & \\ \\

    \theutterance \stepcounter{utterance}  
    & \multicolumn{4}{p{0.6\linewidth}}{
        \cellcolor[rgb]{0.9,0.9,0.9}{
            \makecell[{{p{\linewidth}}}]{
                \texttt{\tiny{[P1$\rangle$GM]}}
                \texttt{ZUSTIMMUNG: \{'A07', 'C38', 'C00', 'C10', 'C81', 'C98', 'C62', 'A83', 'C76'\}} \\
\\ 
\texttt{ARGUMENT: \{'Ihr überarbeiteter Vorschlag nutzt unser Budget deutlich besser aus und bietet eine gute Balance verschiedener Gegenstände. Dies scheint ein fairer Kompromiss zu sein, mit dem beide Seiten zufrieden sein können.'\}} \\
            }
        }
    }
    & & \\ \\

    \theutterance \stepcounter{utterance}  
    & & & \multicolumn{2}{p{0.3\linewidth}}{
        \cellcolor[rgb]{0.9,0.9,0.9}{
            \makecell[{{p{\linewidth}}}]{
                \texttt{\tiny{[GM$|$GM]}}
                \texttt{game successful} \\
            }
        }
    }
    & & \\ \\

    \theutterance \stepcounter{utterance}  
    & & & \multicolumn{2}{p{0.3\linewidth}}{
        \cellcolor[rgb]{0.9,0.9,0.9}{
            \makecell[{{p{\linewidth}}}]{
                \texttt{\tiny{[GM$|$GM]}}
                \texttt{end game} \\
            }
        }
    }
    & & \\ \\

\end{supertabular}
}

\end{document}
