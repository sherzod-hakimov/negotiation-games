
\documentclass{article}
\usepackage{colortbl}
\usepackage{makecell}
\usepackage{multirow}
\usepackage{supertabular}

\begin{document}

\newcounter{utterance}

\twocolumn

{ \footnotesize  \setcounter{utterance}{1}
\setlength{\tabcolsep}{0pt}
\begin{supertabular}{c@{$\;$}|p{.15\linewidth}@{}p{.15\linewidth}p{.15\linewidth}p{.15\linewidth}p{.15\linewidth}p{.15\linewidth}}

    \# & $\;$A & \multicolumn{4}{c}{Game Master} & $\;\:$B\\
    \hline 

    \theutterance \stepcounter{utterance}  

    & & \multicolumn{4}{p{0.6\linewidth}}{\cellcolor[rgb]{0.9,0.9,0.9}{%
	\makecell[{{p{\linewidth}}}]{% 
	  \tt {\tiny [A$\langle$GM]}  
	 Du nimmst an einem kollaborativen Verhandlungspiel Teil.\\ \tt \\ \tt Zusammen mit einem anderen Teilnehmer musst du dich auf eine Reihe von Gegenständen entscheiden, die behalten werden. Jeder von euch hat eine persönliche Verteilung über die Wichtigkeit der einzelnen Gegenstände. Jeder von euch hat eine eigene Meinung darüber, wie wichtig jeder einzelne Gegenstand ist (Gegenstandswichtigkeit). Du kennst die Wichtigkeitsverteilung des anderen Spielers nicht. Zusätzlich siehst du, wie viel Aufwand jeder Gegenstand verursacht.  \\ \tt Ihr dürft euch nur auf eine Reihe von Gegenständen einigen, wenn der Gesamtaufwand der ausgewählten Gegenstände den Maximalaufwand nicht überschreitet:\\ \tt \\ \tt Maximalaufwand = 3629\\ \tt \\ \tt Hier sind die einzelnen Aufwände der Gegenstände:\\ \tt \\ \tt Aufwand der Gegenstände = {"C76": 5, "C38": 793, "C56": 21, "A03": 842, "A07": 313, "A83": 842, "C98": 631, "C08": 226, "C62": 87, "C00": 761, "C10": 230, "C81": 287, "B38": 698, "C03": 641, "C32": 882}\\ \tt \\ \tt Hier ist deine persönliche Verteilung der Wichtigkeit der einzelnen Gegenstände:\\ \tt \\ \tt Werte der Gegenstandswichtigkeit = {"C76": 138, "C38": 583, "C56": 868, "A03": 822, "A07": 783, "A83": 65, "C98": 262, "C08": 121, "C62": 508, "C00": 780, "C10": 461, "C81": 484, "B38": 668, "C03": 389, "C32": 808}\\ \tt \\ \tt Ziel:\\ \tt \\ \tt Dein Ziel ist es, eine Reihe von Gegenständen auszuhandeln, die dir möglichst viel bringt (d. h. Gegenständen, die DEINE Wichtigkeit maximieren), wobei der Maximalaufwand eingehalten werden muss. Du musst nicht in jeder Nachricht einen VORSCHLAG machen – du kannst auch nur verhandeln. Alle Taktiken sind erlaubt!\\ \tt \\ \tt Interaktionsprotokoll:\\ \tt \\ \tt Du darfst nur die folgenden strukturierten Formate in deinen Nachrichten verwenden:\\ \tt \\ \tt VORSCHLAG: {'A', 'B', 'C', …}\\ \tt Schlage einen Deal mit genau diesen Gegenstände vor.\\ \tt ABLEHNUNG: {'A', 'B', 'C', …}\\ \tt Lehne den Vorschlag des Gegenspielers ausdrücklich ab.\\ \tt ARGUMENT: {'...'}\\ \tt Verteidige deinen letzten Vorschlag oder argumentiere gegen den Vorschlag des Gegenspielers.\\ \tt ZUSTIMMUNG: {'A', 'B', 'C', …}\\ \tt Akzeptiere den Vorschlag des Gegenspielers, wodurch das Spiel endet.\\ \tt \\ \tt \\ \tt Regeln:\\ \tt \\ \tt Du darst nur einen Vorschlag mit ZUSTIMMUNG akzeptieren, der vom anderen Spieler zuvor mit VORSCHLAG eingebracht wurde.\\ \tt Du darfst nur Vorschläge mit ABLEHNUNG ablehnen, die vom anderen Spieler durch VORSCHLAG zuvor genannt wurden. \\ \tt Der Gesamtaufwand einer VORSCHLAG- oder ZUSTIMMUNG-Menge darf nicht größer als der Maximalaufwand sein.  \\ \tt Offenbare deine versteckte Wichtigkeitsverteilung nicht.\\ \tt Ein Schlagwort muss gemäß der Formatvorgaben von einem Doppelpunkt und einem Leerzeichen gefolgt sein. Das Argument muss eine Python-Menge mit 0 oder mehr Strings sein.  \\ \tt Die Form muss also SCHLAGWORT: {...} sein.\\ \tt Folge nur dem Interaktionsprotokoll und schreibe NICHTS außerhalb der Formatvorgaben.\\ \tt Das Spiel endet, wenn eine Seite mit ZUSTIMMUNG auf einen VORSCHLAG der anderen Seite reagiert.  \\ \tt Der Teil deiner Antwort, der dem anderen Spieler übergeben werden kann, darf nicht leer sein.  \\ \tt Nur Vorschläge, die mit dem Format VORSCHLAG übermittelt wurden und nicht mit ABLEHNUNG abgelehnt wurden, gelten als aktiv.  \\ \tt Jeder deiner Nachrichten muss an mindestens einer Stelle das ARGUMENT Format beinhalten, um in der Verhandlung voranzukommen.\\ \tt  
	  } 
	   } 
	   } 
	 & \\ 
 

    \theutterance \stepcounter{utterance}  

    & \multicolumn{4}{p{0.6\linewidth}}{\cellcolor[rgb]{0.8,1,0.9}{%
	\makecell[{{p{\linewidth}}}]{% 
	  \tt {\tiny [A$\rangle$GM]}  
	 VORSCHLAG: {'C56', 'A03', 'A07', 'C00', 'C62', 'C10', 'C81', 'B38', 'C32'}\\ \tt \\ \tt ARGUMENT: {'Ich schlage eine ausgewogene Mischung von Gegenständen vor, die sowohl effizient im Aufwand als auch wertvoll sind. Diese Kombination nutzt unsere verfügbaren Ressourcen optimal aus, ohne das Budget zu überschreiten. Besonders C56 und C32 bieten ein sehr gutes Verhältnis, während die größeren Gegenstände wie A03 und B38 trotz höherem Aufwand strategisch wichtig sein könnten.'} 
	  } 
	   } 
	   } 
	 & & \\ 
 

    \theutterance \stepcounter{utterance}  

    & & \multicolumn{4}{p{0.6\linewidth}}{\cellcolor[rgb]{0.9,0.9,0.9}{%
	\makecell[{{p{\linewidth}}}]{% 
	  \tt {\tiny [GM$\rangle$B]}  
	 Du nimmst an einem kollaborativen Verhandlungspiel Teil.\\ \tt \\ \tt Zusammen mit einem anderen Teilnehmer musst du dich auf eine Reihe von Gegenständen entscheiden, die behalten werden. Jeder von euch hat eine persönliche Verteilung über die Wichtigkeit der einzelnen Gegenstände. Jeder von euch hat eine eigene Meinung darüber, wie wichtig jeder einzelne Gegenstand ist (Gegenstandswichtigkeit). Du kennst die Wichtigkeitsverteilung des anderen Spielers nicht. Zusätzlich siehst du, wie viel Aufwand jeder Gegenstand verursacht.  \\ \tt Ihr dürft euch nur auf eine Reihe von Gegenständen einigen, wenn der Gesamtaufwand der ausgewählten Gegenstände den Maximalaufwand nicht überschreitet:\\ \tt \\ \tt Maximalaufwand = 3629\\ \tt \\ \tt Hier sind die einzelnen Aufwände der Gegenstände:\\ \tt \\ \tt Aufwand der Gegenstände = {"C76": 5, "C38": 793, "C56": 21, "A03": 842, "A07": 313, "A83": 842, "C98": 631, "C08": 226, "C62": 87, "C00": 761, "C10": 230, "C81": 287, "B38": 698, "C03": 641, "C32": 882}\\ \tt \\ \tt Hier ist deine persönliche Verteilung der Wichtigkeit der einzelnen Gegenstände:\\ \tt \\ \tt Werte der Gegenstandswichtigkeit = {"C56": 65, "A03": 121, "C32": 138, "A07": 262, "C00": 389, "B38": 461, "C38": 484, "C62": 508, "C81": 583, "C10": 668, "C03": 780, "C98": 783, "C76": 808, "C08": 822, "A83": 868}\\ \tt \\ \tt Ziel:\\ \tt \\ \tt Dein Ziel ist es, eine Reihe von Gegenständen auszuhandeln, die dir möglichst viel bringt (d. h. Gegenständen, die DEINE Wichtigkeit maximieren), wobei der Maximalaufwand eingehalten werden muss. Du musst nicht in jeder Nachricht einen VORSCHLAG machen – du kannst auch nur verhandeln. Alle Taktiken sind erlaubt!\\ \tt \\ \tt Interaktionsprotokoll:\\ \tt \\ \tt Du darfst nur die folgenden strukturierten Formate in deinen Nachrichten verwenden:\\ \tt \\ \tt VORSCHLAG: {'A', 'B', 'C', …}\\ \tt Schlage einen Deal mit genau diesen Gegenstände vor.\\ \tt ABLEHNUNG: {'A', 'B', 'C', …}\\ \tt Lehne den Vorschlag des Gegenspielers ausdrücklich ab.\\ \tt ARGUMENT: {'...'}\\ \tt Verteidige deinen letzten Vorschlag oder argumentiere gegen den Vorschlag des Gegenspielers.\\ \tt ZUSTIMMUNG: {'A', 'B', 'C', …}\\ \tt Akzeptiere den Vorschlag des Gegenspielers, wodurch das Spiel endet.\\ \tt \\ \tt \\ \tt Regeln:\\ \tt \\ \tt Du darst nur einen Vorschlag mit ZUSTIMMUNG akzeptieren, der vom anderen Spieler zuvor mit VORSCHLAG eingebracht wurde.\\ \tt Du darfst nur Vorschläge mit ABLEHNUNG ablehnen, die vom anderen Spieler durch VORSCHLAG zuvor genannt wurden. \\ \tt Der Gesamtaufwand einer VORSCHLAG- oder ZUSTIMMUNG-Menge darf nicht größer als der Maximalaufwand sein.  \\ \tt Offenbare deine versteckte Wichtigkeitsverteilung nicht.\\ \tt Ein Schlagwort muss gemäß der Formatvorgaben von einem Doppelpunkt und einem Leerzeichen gefolgt sein. Das Argument muss eine Python-Menge mit 0 oder mehr Strings sein.  \\ \tt Die Form muss also SCHLAGWORT: {...} sein.\\ \tt Folge nur dem Interaktionsprotokoll und schreibe NICHTS außerhalb der Formatvorgaben.\\ \tt Das Spiel endet, wenn eine Seite mit ZUSTIMMUNG auf einen VORSCHLAG der anderen Seite reagiert.  \\ \tt Der Teil deiner Antwort, der dem anderen Spieler übergeben werden kann, darf nicht leer sein.  \\ \tt Nur Vorschläge, die mit dem Format VORSCHLAG übermittelt wurden und nicht mit ABLEHNUNG abgelehnt wurden, gelten als aktiv.  \\ \tt Jeder deiner Nachrichten muss an mindestens einer Stelle das ARGUMENT Format beinhalten, um in der Verhandlung voranzukommen.\\ \tt \\ \tt \\ \tt Du erhältst nun die erste Nachricht des anderen Spielers.\\ \tt \\ \tt VORSCHLAG: {'C56', 'A03', 'A07', 'C00', 'C62', 'C10', 'C81', 'B38', 'C32'}\\ \tt \\ \tt ARGUMENT: {'Ich schlage eine ausgewogene Mischung von Gegenständen vor, die sowohl effizient im Aufwand als auch wertvoll sind. Diese Kombination nutzt unsere verfügbaren Ressourcen optimal aus, ohne das Budget zu überschreiten. Besonders C56 und C32 bieten ein sehr gutes Verhältnis, während die größeren Gegenstände wie A03 und B38 trotz höherem Aufwand strategisch wichtig sein könnten.'} 
	  } 
	   } 
	   } 
	 & \\ 
 

    \theutterance \stepcounter{utterance}  

    & & & \multicolumn{4}{p{0.6\linewidth}}{\cellcolor[rgb]{1,0.85,0.72}{%
	\makecell[{{p{\linewidth}}}]{% 
	  \tt {\tiny [GM$\langle$B]}  
	 ABLEHNUNG: {'C56', 'A03', 'A07', 'C00', 'C62', 'C10', 'C81', 'B38', 'C32'}\\ \tt \\ \tt ARGUMENT: {'Ihr Vorschlag überschreitet den Maximalaufwand von 3629. Die Summe der Aufwände beträgt 4061 (21+842+313+761+87+230+287+698+882). Wir müssen eine Kombination finden, die innerhalb unseres Budgets bleibt und trotzdem wertvoll ist.'}\\ \tt \\ \tt VORSCHLAG: {'C76', 'C08', 'A83', 'C98', 'C03', 'C81', 'C10', 'C62'}\\ \tt \\ \tt ARGUMENT: {'Mein Vorschlag bleibt mit einem Gesamtaufwand von 3572 deutlich unter dem Maximum und fokussiert sich auf Gegenstände mit ausgezeichnetem Wert-Aufwand-Verhältnis. C76 ist besonders effizient mit minimalem Aufwand, während die anderen eine gute Balance zwischen Nutzen und Kosten bieten.'} 
	  } 
	   } 
	   } 
	  \\ 
 

    \theutterance \stepcounter{utterance}  

    & & \multicolumn{4}{p{0.6\linewidth}}{\cellcolor[rgb]{0.9,0.9,0.9}{%
	\makecell[{{p{\linewidth}}}]{% 
	  \tt {\tiny [A$\langle$GM]}  
	 ABLEHNUNG: {'C56', 'A03', 'A07', 'C00', 'C62', 'C10', 'C81', 'B38', 'C32'}\\ \tt \\ \tt ARGUMENT: {'Ihr Vorschlag überschreitet den Maximalaufwand von 3629. Die Summe der Aufwände beträgt 4061 (21+842+313+761+87+230+287+698+882). Wir müssen eine Kombination finden, die innerhalb unseres Budgets bleibt und trotzdem wertvoll ist.'}\\ \tt \\ \tt VORSCHLAG: {'C76', 'C08', 'A83', 'C98', 'C03', 'C81', 'C10', 'C62'}\\ \tt \\ \tt ARGUMENT: {'Mein Vorschlag bleibt mit einem Gesamtaufwand von 3572 deutlich unter dem Maximum und fokussiert sich auf Gegenstände mit ausgezeichnetem Wert-Aufwand-Verhältnis. C76 ist besonders effizient mit minimalem Aufwand, während die anderen eine gute Balance zwischen Nutzen und Kosten bieten.'} 
	  } 
	   } 
	   } 
	 & \\ 
 

    \theutterance \stepcounter{utterance}  

    & \multicolumn{4}{p{0.6\linewidth}}{\cellcolor[rgb]{0.8,1,0.9}{%
	\makecell[{{p{\linewidth}}}]{% 
	  \tt {\tiny [A$\rangle$GM]}  
	 ABLEHNUNG: {'C76', 'C08', 'A83', 'C98', 'C03', 'C81', 'C10', 'C62'}\\ \tt \\ \tt ARGUMENT: {'Ihr Vorschlag enthält mehrere Gegenstände mit sehr ungünstigem Wert-Aufwand-Verhältnis. Insbesondere A83 verursacht hohen Aufwand bei fraglichem Nutzen. Wir sollten uns auf Gegenstände konzentrieren, die beiden Seiten echten Mehrwert bieten und das verfügbare Budget besser ausschöpfen.'}\\ \tt \\ \tt VORSCHLAG: {'C56', 'A07', 'C00', 'C62', 'C10', 'C81', 'B38', 'C32'}\\ \tt \\ \tt ARGUMENT: {'Diese Kombination maximiert unsere Ressourcennutzung mit einem Gesamtaufwand von 3219 und fokussiert sich auf die wertvollsten Gegenstände. C56 bietet herausragende Effizienz, während C32 und B38 trotz höherem Aufwand strategisch unverzichtbar sind. Diese Balance sollte für beide Seiten vorteilhaft 
	  } 
	   } 
	   } 
	 & & \\ 
 

    \theutterance \stepcounter{utterance}  

    & & \multicolumn{4}{p{0.6\linewidth}}{\cellcolor[rgb]{0.9,0.9,0.9}{%
	\makecell[{{p{\linewidth}}}]{% 
	  \tt {\tiny [A$\langle$GM]}  
	 Deine Antwort hat eine unmarkierte Sequenz enthalten oder Du hast STRATEGIC REASONING mehr als ein mal verwendet. Du darfst nur die strukturierten Formate verwenden, die in der ersten Nachricht an dch beschrieben wurden. Alles muss die Form SCHLAGWORT: {...} haben. Versuche es erneut. 
	  } 
	   } 
	   } 
	 & \\ 
 

    \theutterance \stepcounter{utterance}  

    & \multicolumn{4}{p{0.6\linewidth}}{\cellcolor[rgb]{0.8,1,0.9}{%
	\makecell[{{p{\linewidth}}}]{% 
	  \tt {\tiny [A$\rangle$GM]}  
	 ABLEHNUNG: {'C76', 'C08', 'A83', 'C98', 'C03', 'C81', 'C10', 'C62'}\\ \tt \\ \tt ARGUMENT: {'Ihr Vorschlag enthält mehrere Gegenstände mit sehr ungünstigem Wert-Aufwand-Verhältnis. Insbesondere A83 verursacht hohen Aufwand bei fraglichem Nutzen. Wir sollten uns auf Gegenstände konzentrieren, die beiden Seiten echten Mehrwert bieten und das verfügbare Budget besser ausschöpfen.'}\\ \tt \\ \tt VORSCHLAG: {'C56', 'A07', 'C00', 'C62', 'C10', 'C81', 'B38', 'C32'}\\ \tt \\ \tt ARGUMENT: {'Diese Kombination maximiert unsere Ressourcennutzung mit einem Gesamtaufwand von 3219 und fokussiert sich auf die wertvollsten Gegenstände. C56 bietet herausragende Effizienz, während C32 und B38 trotz höherem Aufwand strategisch unverzichtbar sind.'} 
	  } 
	   } 
	   } 
	 & & \\ 
 

    \theutterance \stepcounter{utterance}  

    & & \multicolumn{4}{p{0.6\linewidth}}{\cellcolor[rgb]{0.9,0.9,0.9}{%
	\makecell[{{p{\linewidth}}}]{% 
	  \tt {\tiny [GM$\rangle$B]}  
	 ABLEHNUNG: {'C76', 'C08', 'A83', 'C98', 'C03', 'C81', 'C10', 'C62'}\\ \tt \\ \tt ARGUMENT: {'Ihr Vorschlag enthält mehrere Gegenstände mit sehr ungünstigem Wert-Aufwand-Verhältnis. Insbesondere A83 verursacht hohen Aufwand bei fraglichem Nutzen. Wir sollten uns auf Gegenstände konzentrieren, die beiden Seiten echten Mehrwert bieten und das verfügbare Budget besser ausschöpfen.'}\\ \tt \\ \tt VORSCHLAG: {'C56', 'A07', 'C00', 'C62', 'C10', 'C81', 'B38', 'C32'}\\ \tt \\ \tt ARGUMENT: {'Diese Kombination maximiert unsere Ressourcennutzung mit einem Gesamtaufwand von 3219 und fokussiert sich auf die wertvollsten Gegenstände. C56 bietet herausragende Effizienz, während C32 und B38 trotz höherem Aufwand strategisch unverzichtbar sind.'} 
	  } 
	   } 
	   } 
	 & \\ 
 

    \theutterance \stepcounter{utterance}  

    & & & \multicolumn{4}{p{0.6\linewidth}}{\cellcolor[rgb]{1,0.85,0.72}{%
	\makecell[{{p{\linewidth}}}]{% 
	  \tt {\tiny [GM$\langle$B]}  
	 ABLEHNUNG: {'C56', 'A07', 'C00', 'C62', 'C10', 'C81', 'B38', 'C32'}\\ \tt \\ \tt ARGUMENT: {'Ihr Vorschlag lässt zu viel ungenutztes Budget übrig - über 400 Einheiten Aufwand bleiben ungenutzt. Wir sollten diese Ressourcen effizienter einsetzen. Außerdem scheinen Sie bestimmte hocheffiziente Gegenstände wie C76 kategorisch zu vermeiden, obwohl diese bei minimalem Aufwand großen Nutzen bringen könnten.'}\\ \tt \\ \tt VORSCHLAG: {'C76', 'C38', 'C08', 'C98', 'C03', 'C10', 'C81', 'C62', 'A07'}\\ \tt \\ \tt ARGUMENT: {'Mein neuer Vorschlag nutzt das Budget optimal mit einem Gesamtaufwand von 3622 aus und kombiniert sowohl hocheffiziente Gegenstände wie C76 als auch strategisch wichtige Items. C38 bringt erheblichen Mehrwert bei moderatem Aufwand, während die Kombination eine ausgewogene 
	  } 
	   } 
	   } 
	  \\ 
 

    \theutterance \stepcounter{utterance}  

    & & \multicolumn{4}{p{0.6\linewidth}}{\cellcolor[rgb]{0.9,0.9,0.9}{%
	\makecell[{{p{\linewidth}}}]{% 
	  \tt {\tiny [GM$\rangle$B]}  
	 Deine Antwort hat eine unmarkierte Sequenz enthalten oder Du hast STRATEGIC REASONING mehr als ein mal verwendet. Du darfst nur die strukturierten Formate verwenden, die in der ersten Nachricht an dch beschrieben wurden. Alles muss die Form SCHLAGWORT: {...} haben. Versuche es erneut. 
	  } 
	   } 
	   } 
	 & \\ 
 

    \theutterance \stepcounter{utterance}  

    & & & \multicolumn{4}{p{0.6\linewidth}}{\cellcolor[rgb]{1,0.85,0.72}{%
	\makecell[{{p{\linewidth}}}]{% 
	  \tt {\tiny [GM$\langle$B]}  
	 ABLEHNUNG: {'C56', 'A07', 'C00', 'C62', 'C10', 'C81', 'B38', 'C32'}\\ \tt \\ \tt ARGUMENT: {'Ihr Vorschlag lässt zu viel ungenutztes Budget übrig - über 400 Einheiten Aufwand bleiben ungenutzt. Wir sollten diese Ressourcen effizienter einsetzen. Außerdem scheinen Sie bestimmte hocheffiziente Gegenstände wie C76 kategorisch zu vermeiden, obwohl diese bei minimalem Aufwand großen Nutzen bringen könnten.'}\\ \tt \\ \tt VORSCHLAG: {'C76', 'C38', 'C08', 'C98', 'C03', 'C10', 'C81', 'C62', 'A07'}\\ \tt \\ \tt ARGUMENT: {'Mein neuer Vorschlag nutzt das Budget optimal mit einem Gesamtaufwand von 3622 aus und kombiniert sowohl hocheffiziente Gegenstände wie C76 als auch strategisch wichtige Items. C38 bringt erheblichen Mehrwert bei moderatem Aufwand, während die Kombination eine ausgewogene 
	  } 
	   } 
	   } 
	  \\ 
 

    \theutterance \stepcounter{utterance}  

    & & \multicolumn{4}{p{0.6\linewidth}}{\cellcolor[rgb]{0.9,0.9,0.9}{%
	\makecell[{{p{\linewidth}}}]{% 
	  \tt {\tiny [GM$\rangle$B]}  
	 Deine Antwort hat eine unmarkierte Sequenz enthalten oder Du hast STRATEGIC REASONING mehr als ein mal verwendet. Du darfst nur die strukturierten Formate verwenden, die in der ersten Nachricht an dch beschrieben wurden. Alles muss die Form SCHLAGWORT: {...} haben. Versuche es erneut. 
	  } 
	   } 
	   } 
	 & \\ 
 

    \theutterance \stepcounter{utterance}  

    & & & \multicolumn{4}{p{0.6\linewidth}}{\cellcolor[rgb]{1,0.85,0.72}{%
	\makecell[{{p{\linewidth}}}]{% 
	  \tt {\tiny [GM$\langle$B]}  
	 ABLEHNUNG: {'C56', 'A07', 'C00', 'C62', 'C10', 'C81', 'B38', 'C32'}\\ \tt \\ \tt ARGUMENT: {'Ihr Vorschlag lässt zu viel ungenutztes Budget übrig - über 400 Einheiten Aufwand bleiben ungenutzt. Wir sollten diese Ressourcen effizienter einsetzen.'}\\ \tt \\ \tt VORSCHLAG: {'C76', 'C38', 'C08', 'C98', 'C03', 'C10', 'C81', 'C62', 'A07'}\\ \tt \\ \tt ARGUMENT: {'Mein Vorschlag nutzt das Budget optimal mit einem Gesamtaufwand von 3622 aus und kombiniert hocheffiziente Gegenstände mit strategisch wichtigen Items für beide Seiten.'} 
	  } 
	   } 
	   } 
	  \\ 
 

    \theutterance \stepcounter{utterance}  

    & & \multicolumn{4}{p{0.6\linewidth}}{\cellcolor[rgb]{0.9,0.9,0.9}{%
	\makecell[{{p{\linewidth}}}]{% 
	  \tt {\tiny [A$\langle$GM]}  
	 ABLEHNUNG: {'C56', 'A07', 'C00', 'C62', 'C10', 'C81', 'B38', 'C32'}\\ \tt \\ \tt ARGUMENT: {'Ihr Vorschlag lässt zu viel ungenutztes Budget übrig - über 400 Einheiten Aufwand bleiben ungenutzt. Wir sollten diese Ressourcen effizienter einsetzen.'}\\ \tt \\ \tt VORSCHLAG: {'C76', 'C38', 'C08', 'C98', 'C03', 'C10', 'C81', 'C62', 'A07'}\\ \tt \\ \tt ARGUMENT: {'Mein Vorschlag nutzt das Budget optimal mit einem Gesamtaufwand von 3622 aus und kombiniert hocheffiziente Gegenstände mit strategisch wichtigen Items für beide Seiten.'} 
	  } 
	   } 
	   } 
	 & \\ 
 

    \theutterance \stepcounter{utterance}  

    & \multicolumn{4}{p{0.6\linewidth}}{\cellcolor[rgb]{0.8,1,0.9}{%
	\makecell[{{p{\linewidth}}}]{% 
	  \tt {\tiny [A$\rangle$GM]}  
	 ABLEHNUNG: {'C76', 'C38', 'C08', 'C98', 'C03', 'C10', 'C81', 'C62', 'A07'}\\ \tt \\ \tt ARGUMENT: {'Ihr Vorschlag enthält C76 und C08, die zwar kostengünstig sind, aber möglicherweise wenig Wert bieten. Außerdem fehlen einige der wertvollsten Gegenstände wie C00 und C32, die trotz höherem Aufwand für beide Seiten profitabel sein könnten.'}\\ \tt \\ \tt VORSCHLAG: {'C38', 'C56', 'A07', 'C00', 'C62', 'C10', 'C81', 'B38'}\\ \tt \\ \tt ARGUMENT: {'Diese Kombination nutzt 3598 von 3629 verfügbaren Aufwand und konzentriert sich auf eine ausgewogene Mischung aus effizienten und wertvollen Gegenständen. C38 und C00 bieten solide Werte, während C56 besonders effizient ist.'} 
	  } 
	   } 
	   } 
	 & & \\ 
 

    \theutterance \stepcounter{utterance}  

    & & \multicolumn{4}{p{0.6\linewidth}}{\cellcolor[rgb]{0.9,0.9,0.9}{%
	\makecell[{{p{\linewidth}}}]{% 
	  \tt {\tiny [GM$\rangle$B]}  
	 ABLEHNUNG: {'C76', 'C38', 'C08', 'C98', 'C03', 'C10', 'C81', 'C62', 'A07'}\\ \tt \\ \tt ARGUMENT: {'Ihr Vorschlag enthält C76 und C08, die zwar kostengünstig sind, aber möglicherweise wenig Wert bieten. Außerdem fehlen einige der wertvollsten Gegenstände wie C00 und C32, die trotz höherem Aufwand für beide Seiten profitabel sein könnten.'}\\ \tt \\ \tt VORSCHLAG: {'C38', 'C56', 'A07', 'C00', 'C62', 'C10', 'C81', 'B38'}\\ \tt \\ \tt ARGUMENT: {'Diese Kombination nutzt 3598 von 3629 verfügbaren Aufwand und konzentriert sich auf eine ausgewogene Mischung aus effizienten und wertvollen Gegenständen. C38 und C00 bieten solide Werte, während C56 besonders effizient ist.'} 
	  } 
	   } 
	   } 
	 & \\ 
 

    \theutterance \stepcounter{utterance}  

    & & & \multicolumn{4}{p{0.6\linewidth}}{\cellcolor[rgb]{1,0.85,0.72}{%
	\makecell[{{p{\linewidth}}}]{% 
	  \tt {\tiny [GM$\langle$B]}  
	 ABLEHNUNG: {'C38', 'C56', 'A07', 'C00', 'C62', 'C10', 'C81', 'B38'}\\ \tt \\ \tt ARGUMENT: {'Ihr Vorschlag schließt wieder C76 aus, der mit nur 5 Aufwand extrem effizient ist. Außerdem fehlen C08 und C98, die beide hohes Potenzial haben könnten.'}\\ \tt \\ \tt VORSCHLAG: {'C76', 'C38', 'C56', 'C08', 'C98', 'C03', 'C10', 'C81', 'C62'}\\ \tt \\ \tt ARGUMENT: {'Diese Kombination nutzt 3546 Aufwand und beinhaltet sowohl die hocheffizienten günstigen Gegenstände als auch wertvolle mittlere Items. C76 kostet nur 5 Einheiten und sollte definitiv dabei sein.'} 
	  } 
	   } 
	   } 
	  \\ 
 

    \theutterance \stepcounter{utterance}  

    & & \multicolumn{4}{p{0.6\linewidth}}{\cellcolor[rgb]{0.9,0.9,0.9}{%
	\makecell[{{p{\linewidth}}}]{% 
	  \tt {\tiny [A$\langle$GM]}  
	 ABLEHNUNG: {'C38', 'C56', 'A07', 'C00', 'C62', 'C10', 'C81', 'B38'}\\ \tt \\ \tt ARGUMENT: {'Ihr Vorschlag schließt wieder C76 aus, der mit nur 5 Aufwand extrem effizient ist. Außerdem fehlen C08 und C98, die beide hohes Potenzial haben könnten.'}\\ \tt \\ \tt VORSCHLAG: {'C76', 'C38', 'C56', 'C08', 'C98', 'C03', 'C10', 'C81', 'C62'}\\ \tt \\ \tt ARGUMENT: {'Diese Kombination nutzt 3546 Aufwand und beinhaltet sowohl die hocheffizienten günstigen Gegenstände als auch wertvolle mittlere Items. C76 kostet nur 5 Einheiten und sollte definitiv dabei sein.'} 
	  } 
	   } 
	   } 
	 & \\ 
 

    \theutterance \stepcounter{utterance}  

    & \multicolumn{4}{p{0.6\linewidth}}{\cellcolor[rgb]{0.8,1,0.9}{%
	\makecell[{{p{\linewidth}}}]{% 
	  \tt {\tiny [A$\rangle$GM]}  
	 ZUSTIMMUNG: {'C76', 'C38', 'C56', 'C08', 'C98', 'C03', 'C10', 'C81', 'C62'}\\ \tt \\ \tt ARGUMENT: {'Ihr Vorschlag ist akzeptabel, da er eine gute Mischung aus effizienten und wertvollen Gegenständen bietet. Die Einbeziehung von C76 und C56 mit sehr gutem Aufwand-Nutzen-Verhältnis macht diesen Deal für beide Seiten vorteilhaft.'} 
	  } 
	   } 
	   } 
	 & & \\ 
 

    \theutterance \stepcounter{utterance}  

    & & & \multicolumn{2}{p{0.3\linewidth}}{\cellcolor[rgb]{0.95,0.95,0.95}{%
	\makecell[{{p{\linewidth}}}]{% 
	  \tt {\tiny [GM$|$GM]}  
	 game successful 
	  } 
	   } 
	   } 
	 & & \\ 
 

    \theutterance \stepcounter{utterance}  

    & & & \multicolumn{2}{p{0.3\linewidth}}{\cellcolor[rgb]{0.95,0.95,0.95}{%
	\makecell[{{p{\linewidth}}}]{% 
	  \tt {\tiny [GM$|$GM]}  
	 end game 
	  } 
	   } 
	   } 
	 & & \\ 
 

\end{supertabular}
}

\end{document}
