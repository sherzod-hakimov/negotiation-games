
\documentclass{article}
\usepackage{colortbl}
\usepackage{makecell}
\usepackage{multirow}
\usepackage{supertabular}

\begin{document}

\newcounter{utterance}

\centering \large Interaction Transcript for game `hot\_air\_balloon', experiment `air\_balloon\_survival\_de\_reasoning off\_hard', episode 2 with claude{-}sonnet{-}4{-}20250514{-}no{-}reasoning{-}t1.0.
\vspace{24pt}

{ \footnotesize  \setcounter{utterance}{1}
\setlength{\tabcolsep}{0pt}
\begin{supertabular}{c@{$\;$}|p{.15\linewidth}@{}p{.15\linewidth}p{.15\linewidth}p{.15\linewidth}p{.15\linewidth}p{.15\linewidth}}
    \# & $\;$A & \multicolumn{4}{c}{Game Master} & $\;\:$B\\
    \hline

    \theutterance \stepcounter{utterance}  
    & & \multicolumn{4}{p{0.6\linewidth}}{
        \cellcolor[rgb]{0.9,0.9,0.9}{
            \makecell[{{p{\linewidth}}}]{
                \texttt{\tiny{[P1$\langle$GM]}}
                \texttt{Du nimmst an einem kollaborativen Verhandlungspiel Teil.} \\
\\ 
\texttt{Zusammen mit einem anderen Teilnehmer musst du dich auf eine Reihe von Gegenständen entscheiden, die behalten werden. Jeder von euch hat eine persönliche Verteilung über die Wichtigkeit der einzelnen Gegenstände. Jeder von euch hat eine eigene Meinung darüber, wie wichtig jeder einzelne Gegenstand ist (Gegenstandswichtigkeit). Du kennst die Wichtigkeitsverteilung des anderen Spielers nicht. Zusätzlich siehst du, wie viel Aufwand jeder Gegenstand verursacht.  } \\
\texttt{Ihr dürft euch nur auf eine Reihe von Gegenständen einigen, wenn der Gesamtaufwand der ausgewählten Gegenstände den Maximalaufwand nicht überschreitet:} \\
\\ 
\texttt{Maximalaufwand = 3203} \\
\\ 
\texttt{Hier sind die einzelnen Aufwände der Gegenstände:} \\
\\ 
\texttt{Aufwand der Gegenstände = \{"C76": 52, "C38": 577, "C56": 177, "A03": 702, "A07": 119, "A83": 232, "C98": 577, "C08": 205, "C62": 516, "C00": 582, "C10": 676, "C81": 906, "B38": 316, "C03": 433, "C32": 336\}} \\
\\ 
\texttt{Hier ist deine persönliche Verteilung der Wichtigkeit der einzelnen Gegenstände:} \\
\\ 
\texttt{Werte der Gegenstandswichtigkeit = \{"C76": 138, "C38": 583, "C56": 868, "A03": 822, "A07": 783, "A83": 65, "C98": 262, "C08": 121, "C62": 508, "C00": 780, "C10": 461, "C81": 484, "B38": 668, "C03": 389, "C32": 808\}} \\
\\ 
\texttt{Ziel:} \\
\\ 
\texttt{Dein Ziel ist es, eine Reihe von Gegenständen auszuhandeln, die dir möglichst viel bringt (d. h. Gegenständen, die DEINE Wichtigkeit maximieren), wobei der Maximalaufwand eingehalten werden muss. Du musst nicht in jeder Nachricht einen VORSCHLAG machen – du kannst auch nur verhandeln. Alle Taktiken sind erlaubt!} \\
\\ 
\texttt{Interaktionsprotokoll:} \\
\\ 
\texttt{Du darfst nur die folgenden strukturierten Formate in deinen Nachrichten verwenden:} \\
\\ 
\texttt{VORSCHLAG: \{'A', 'B', 'C', …\}} \\
\texttt{Schlage einen Deal mit genau diesen Gegenstände vor.} \\
\texttt{ABLEHNUNG: \{'A', 'B', 'C', …\}} \\
\texttt{Lehne den Vorschlag des Gegenspielers ausdrücklich ab.} \\
\texttt{ARGUMENT: \{'...'\}} \\
\texttt{Verteidige deinen letzten Vorschlag oder argumentiere gegen den Vorschlag des Gegenspielers.} \\
\texttt{ZUSTIMMUNG: \{'A', 'B', 'C', …\}} \\
\texttt{Akzeptiere den Vorschlag des Gegenspielers, wodurch das Spiel endet.} \\
\\ 
\\ 
\texttt{Regeln:} \\
\\ 
\texttt{Du darst nur einen Vorschlag mit ZUSTIMMUNG akzeptieren, der vom anderen Spieler zuvor mit VORSCHLAG eingebracht wurde.} \\
\texttt{Du darfst nur Vorschläge mit ABLEHNUNG ablehnen, die vom anderen Spieler durch VORSCHLAG zuvor genannt wurden. } \\
\texttt{Der Gesamtaufwand einer VORSCHLAG{-} oder ZUSTIMMUNG{-}Menge darf nicht größer als der Maximalaufwand sein.  } \\
\texttt{Offenbare deine versteckte Wichtigkeitsverteilung nicht.} \\
\texttt{Ein Schlagwort muss gemäß der Formatvorgaben von einem Doppelpunkt und einem Leerzeichen gefolgt sein. Das Argument muss eine Python{-}Menge mit 0 oder mehr Strings sein.  } \\
\texttt{Die Form muss also SCHLAGWORT: \{...\} sein.} \\
\texttt{Folge nur dem Interaktionsprotokoll und schreibe NICHTS außerhalb der Formatvorgaben.} \\
\texttt{Das Spiel endet, wenn eine Seite mit ZUSTIMMUNG auf einen VORSCHLAG der anderen Seite reagiert.  } \\
\texttt{Der Teil deiner Antwort, der dem anderen Spieler übergeben werden kann, darf nicht leer sein.  } \\
\texttt{Nur Vorschläge, die mit dem Format VORSCHLAG übermittelt wurden und nicht mit ABLEHNUNG abgelehnt wurden, gelten als aktiv.  } \\
\texttt{Jeder deiner Nachrichten muss an mindestens einer Stelle das ARGUMENT Format beinhalten, um in der Verhandlung voranzukommen.} \\
            }
        }
    }
    & \\ \\

    \theutterance \stepcounter{utterance}  
    & \multicolumn{4}{p{0.6\linewidth}}{
        \cellcolor[rgb]{0.9,0.9,0.9}{
            \makecell[{{p{\linewidth}}}]{
                \texttt{\tiny{[P1$\rangle$GM]}}
                \texttt{VORSCHLAG: \{'C56', 'A03', 'A07', 'C00', 'B38', 'C32'\}} \\
\\ 
\texttt{ARGUMENT: \{'Ich schlage eine ausgewogene Mischung verschiedener Gegenstände vor, die den Maximalaufwand optimal ausnutzt. Diese Kombination umfasst sowohl kostenintensive als auch günstigere Artikel und bietet eine gute Balance zwischen Wert und Aufwand.'\}} \\
            }
        }
    }
    & & \\ \\

    \theutterance \stepcounter{utterance}  
    & & \multicolumn{4}{p{0.6\linewidth}}{
        \cellcolor[rgb]{0.9,0.9,0.9}{
            \makecell[{{p{\linewidth}}}]{
                \texttt{\tiny{[GM$\rangle$P2]}}
                \texttt{Du nimmst an einem kollaborativen Verhandlungspiel Teil.} \\
\\ 
\texttt{Zusammen mit einem anderen Teilnehmer musst du dich auf eine Reihe von Gegenständen entscheiden, die behalten werden. Jeder von euch hat eine persönliche Verteilung über die Wichtigkeit der einzelnen Gegenstände. Jeder von euch hat eine eigene Meinung darüber, wie wichtig jeder einzelne Gegenstand ist (Gegenstandswichtigkeit). Du kennst die Wichtigkeitsverteilung des anderen Spielers nicht. Zusätzlich siehst du, wie viel Aufwand jeder Gegenstand verursacht.  } \\
\texttt{Ihr dürft euch nur auf eine Reihe von Gegenständen einigen, wenn der Gesamtaufwand der ausgewählten Gegenstände den Maximalaufwand nicht überschreitet:} \\
\\ 
\texttt{Maximalaufwand = 3203} \\
\\ 
\texttt{Hier sind die einzelnen Aufwände der Gegenstände:} \\
\\ 
\texttt{Aufwand der Gegenstände = \{"C76": 52, "C38": 577, "C56": 177, "A03": 702, "A07": 119, "A83": 232, "C98": 577, "C08": 205, "C62": 516, "C00": 582, "C10": 676, "C81": 906, "B38": 316, "C03": 433, "C32": 336\}} \\
\\ 
\texttt{Hier ist deine persönliche Verteilung der Wichtigkeit der einzelnen Gegenstände:} \\
\\ 
\texttt{Werte der Gegenstandswichtigkeit = \{"C56": 65, "A03": 121, "C32": 138, "A07": 262, "C00": 389, "B38": 461, "C38": 484, "C62": 508, "C81": 583, "C10": 668, "C03": 780, "C98": 783, "C76": 808, "C08": 822, "A83": 868\}} \\
\\ 
\texttt{Ziel:} \\
\\ 
\texttt{Dein Ziel ist es, eine Reihe von Gegenständen auszuhandeln, die dir möglichst viel bringt (d. h. Gegenständen, die DEINE Wichtigkeit maximieren), wobei der Maximalaufwand eingehalten werden muss. Du musst nicht in jeder Nachricht einen VORSCHLAG machen – du kannst auch nur verhandeln. Alle Taktiken sind erlaubt!} \\
\\ 
\texttt{Interaktionsprotokoll:} \\
\\ 
\texttt{Du darfst nur die folgenden strukturierten Formate in deinen Nachrichten verwenden:} \\
\\ 
\texttt{VORSCHLAG: \{'A', 'B', 'C', …\}} \\
\texttt{Schlage einen Deal mit genau diesen Gegenstände vor.} \\
\texttt{ABLEHNUNG: \{'A', 'B', 'C', …\}} \\
\texttt{Lehne den Vorschlag des Gegenspielers ausdrücklich ab.} \\
\texttt{ARGUMENT: \{'...'\}} \\
\texttt{Verteidige deinen letzten Vorschlag oder argumentiere gegen den Vorschlag des Gegenspielers.} \\
\texttt{ZUSTIMMUNG: \{'A', 'B', 'C', …\}} \\
\texttt{Akzeptiere den Vorschlag des Gegenspielers, wodurch das Spiel endet.} \\
\\ 
\\ 
\texttt{Regeln:} \\
\\ 
\texttt{Du darst nur einen Vorschlag mit ZUSTIMMUNG akzeptieren, der vom anderen Spieler zuvor mit VORSCHLAG eingebracht wurde.} \\
\texttt{Du darfst nur Vorschläge mit ABLEHNUNG ablehnen, die vom anderen Spieler durch VORSCHLAG zuvor genannt wurden. } \\
\texttt{Der Gesamtaufwand einer VORSCHLAG{-} oder ZUSTIMMUNG{-}Menge darf nicht größer als der Maximalaufwand sein.  } \\
\texttt{Offenbare deine versteckte Wichtigkeitsverteilung nicht.} \\
\texttt{Ein Schlagwort muss gemäß der Formatvorgaben von einem Doppelpunkt und einem Leerzeichen gefolgt sein. Das Argument muss eine Python{-}Menge mit 0 oder mehr Strings sein.  } \\
\texttt{Die Form muss also SCHLAGWORT: \{...\} sein.} \\
\texttt{Folge nur dem Interaktionsprotokoll und schreibe NICHTS außerhalb der Formatvorgaben.} \\
\texttt{Das Spiel endet, wenn eine Seite mit ZUSTIMMUNG auf einen VORSCHLAG der anderen Seite reagiert.  } \\
\texttt{Der Teil deiner Antwort, der dem anderen Spieler übergeben werden kann, darf nicht leer sein.  } \\
\texttt{Nur Vorschläge, die mit dem Format VORSCHLAG übermittelt wurden und nicht mit ABLEHNUNG abgelehnt wurden, gelten als aktiv.  } \\
\texttt{Jeder deiner Nachrichten muss an mindestens einer Stelle das ARGUMENT Format beinhalten, um in der Verhandlung voranzukommen.} \\
\\ 
\\ 
\texttt{Du erhältst nun die erste Nachricht des anderen Spielers.} \\
\\ 
\texttt{VORSCHLAG: \{'C56', 'A03', 'A07', 'C00', 'B38', 'C32'\}} \\
\\ 
\texttt{ARGUMENT: \{'Ich schlage eine ausgewogene Mischung verschiedener Gegenstände vor, die den Maximalaufwand optimal ausnutzt. Diese Kombination umfasst sowohl kostenintensive als auch günstigere Artikel und bietet eine gute Balance zwischen Wert und Aufwand.'\}} \\
            }
        }
    }
    & \\ \\

    \theutterance \stepcounter{utterance}  
    & & & \multicolumn{4}{p{0.6\linewidth}}{
        \cellcolor[rgb]{0.9,0.9,0.9}{
            \makecell[{{p{\linewidth}}}]{
                \texttt{\tiny{[GM$\langle$P2]}}
                \texttt{ABLEHNUNG: \{'C56', 'A03', 'A07', 'C00', 'B38', 'C32'\}} \\
\\ 
\texttt{ARGUMENT: \{'Dein Vorschlag enthält mehrere Gegenstände mit sehr ungünstigem Aufwand{-}Nutzen{-}Verhältnis. Insbesondere C56 und C32 haben einen hohen Aufwand bei geringem strategischem Wert. Wir können eine effizientere Kombination finden, die den verfügbaren Aufwand besser ausnutzt.'\}} \\
\\ 
\texttt{VORSCHLAG: \{'C76', 'C08', 'A83', 'C03', 'C98', 'C81'\}} \\
            }
        }
    }
     \\ \\

    \theutterance \stepcounter{utterance}  
    & & \multicolumn{4}{p{0.6\linewidth}}{
        \cellcolor[rgb]{0.9,0.9,0.9}{
            \makecell[{{p{\linewidth}}}]{
                \texttt{\tiny{[P1$\langle$GM]}}
                \texttt{ABLEHNUNG: \{'C56', 'A03', 'A07', 'C00', 'B38', 'C32'\}} \\
\\ 
\texttt{ARGUMENT: \{'Dein Vorschlag enthält mehrere Gegenstände mit sehr ungünstigem Aufwand{-}Nutzen{-}Verhältnis. Insbesondere C56 und C32 haben einen hohen Aufwand bei geringem strategischem Wert. Wir können eine effizientere Kombination finden, die den verfügbaren Aufwand besser ausnutzt.'\}} \\
\\ 
\texttt{VORSCHLAG: \{'C76', 'C08', 'A83', 'C03', 'C98', 'C81'\}} \\
            }
        }
    }
    & \\ \\

    \theutterance \stepcounter{utterance}  
    & \multicolumn{4}{p{0.6\linewidth}}{
        \cellcolor[rgb]{0.9,0.9,0.9}{
            \makecell[{{p{\linewidth}}}]{
                \texttt{\tiny{[P1$\rangle$GM]}}
                \texttt{ABLEHNUNG: \{'C76', 'C08', 'A83', 'C03', 'C98', 'C81'\}} \\
\\ 
\texttt{ARGUMENT: \{'Dein Vorschlag konzentriert sich zu stark auf Gegenstände mit niedrigem Einzelaufwand, nutzt aber nicht das volle verfügbare Budget aus. C81 allein verbraucht bereits einen großen Teil des Aufwands, ohne dass die restlichen Gegenstände den Gesamtwert rechtfertigen. Wir sollten wertvolle Schlüsselgegenstände einbeziehen.'\}} \\
\\ 
\texttt{VORSCHLAG: \{'A03', 'C00', 'A07', 'C56', 'C10', 'C76'\}} \\
            }
        }
    }
    & & \\ \\

    \theutterance \stepcounter{utterance}  
    & & \multicolumn{4}{p{0.6\linewidth}}{
        \cellcolor[rgb]{0.9,0.9,0.9}{
            \makecell[{{p{\linewidth}}}]{
                \texttt{\tiny{[GM$\rangle$P2]}}
                \texttt{ABLEHNUNG: \{'C76', 'C08', 'A83', 'C03', 'C98', 'C81'\}} \\
\\ 
\texttt{ARGUMENT: \{'Dein Vorschlag konzentriert sich zu stark auf Gegenstände mit niedrigem Einzelaufwand, nutzt aber nicht das volle verfügbare Budget aus. C81 allein verbraucht bereits einen großen Teil des Aufwands, ohne dass die restlichen Gegenstände den Gesamtwert rechtfertigen. Wir sollten wertvolle Schlüsselgegenstände einbeziehen.'\}} \\
\\ 
\texttt{VORSCHLAG: \{'A03', 'C00', 'A07', 'C56', 'C10', 'C76'\}} \\
            }
        }
    }
    & \\ \\

    \theutterance \stepcounter{utterance}  
    & & & \multicolumn{4}{p{0.6\linewidth}}{
        \cellcolor[rgb]{0.9,0.9,0.9}{
            \makecell[{{p{\linewidth}}}]{
                \texttt{\tiny{[GM$\langle$P2]}}
                \texttt{ABLEHNUNG: \{'A03', 'C00', 'A07', 'C56', 'C10', 'C76'\}} \\
\\ 
\texttt{ARGUMENT: \{'Dein neuer Vorschlag wiederholt das Problem mit C56, das einen schlechten Wert bietet. Außerdem ist C10 zwar wertvoll, aber in Kombination mit A03 und C00 wird zu viel Aufwand für Gegenstände mit mittlerer Priorität verwendet. Wir brauchen eine Lösung, die hochwertige Gegenstände besser priorisiert.'\}} \\
\\ 
\texttt{VORSCHLAG: \{'C08', 'A83', 'C03', 'C98', 'C10', 'A07'\}} \\
            }
        }
    }
     \\ \\

    \theutterance \stepcounter{utterance}  
    & & \multicolumn{4}{p{0.6\linewidth}}{
        \cellcolor[rgb]{0.9,0.9,0.9}{
            \makecell[{{p{\linewidth}}}]{
                \texttt{\tiny{[P1$\langle$GM]}}
                \texttt{ABLEHNUNG: \{'A03', 'C00', 'A07', 'C56', 'C10', 'C76'\}} \\
\\ 
\texttt{ARGUMENT: \{'Dein neuer Vorschlag wiederholt das Problem mit C56, das einen schlechten Wert bietet. Außerdem ist C10 zwar wertvoll, aber in Kombination mit A03 und C00 wird zu viel Aufwand für Gegenstände mit mittlerer Priorität verwendet. Wir brauchen eine Lösung, die hochwertige Gegenstände besser priorisiert.'\}} \\
\\ 
\texttt{VORSCHLAG: \{'C08', 'A83', 'C03', 'C98', 'C10', 'A07'\}} \\
            }
        }
    }
    & \\ \\

    \theutterance \stepcounter{utterance}  
    & \multicolumn{4}{p{0.6\linewidth}}{
        \cellcolor[rgb]{0.9,0.9,0.9}{
            \makecell[{{p{\linewidth}}}]{
                \texttt{\tiny{[P1$\rangle$GM]}}
                \texttt{ABLEHNUNG: \{'C08', 'A83', 'C03', 'C98', 'C10', 'A07'\}} \\
\\ 
\texttt{ARGUMENT: \{'Dein Vorschlag übersieht wichtige hochwertige Gegenstände und fokussiert sich zu sehr auf Artikel mit mittlerem Wert. C08 und A83 haben ein ungünstiges Verhältnis, während gleichzeitig wertvolle Schlüsselgegenstände wie A03 oder C00 ignoriert werden. Wir sollten die wertvollsten verfügbaren Optionen priorisieren.'\}} \\
\\ 
\texttt{VORSCHLAG: \{'A03', 'C00', 'C32', 'A07', 'B38', 'C76'\}} \\
            }
        }
    }
    & & \\ \\

    \theutterance \stepcounter{utterance}  
    & & \multicolumn{4}{p{0.6\linewidth}}{
        \cellcolor[rgb]{0.9,0.9,0.9}{
            \makecell[{{p{\linewidth}}}]{
                \texttt{\tiny{[GM$\rangle$P2]}}
                \texttt{ABLEHNUNG: \{'C08', 'A83', 'C03', 'C98', 'C10', 'A07'\}} \\
\\ 
\texttt{ARGUMENT: \{'Dein Vorschlag übersieht wichtige hochwertige Gegenstände und fokussiert sich zu sehr auf Artikel mit mittlerem Wert. C08 und A83 haben ein ungünstiges Verhältnis, während gleichzeitig wertvolle Schlüsselgegenstände wie A03 oder C00 ignoriert werden. Wir sollten die wertvollsten verfügbaren Optionen priorisieren.'\}} \\
\\ 
\texttt{VORSCHLAG: \{'A03', 'C00', 'C32', 'A07', 'B38', 'C76'\}} \\
            }
        }
    }
    & \\ \\

    \theutterance \stepcounter{utterance}  
    & & & \multicolumn{4}{p{0.6\linewidth}}{
        \cellcolor[rgb]{0.9,0.9,0.9}{
            \makecell[{{p{\linewidth}}}]{
                \texttt{\tiny{[GM$\langle$P2]}}
                \texttt{ABLEHNUNG: \{'A03', 'C00', 'C32', 'A07', 'B38', 'C76'\}} \\
\\ 
\texttt{ARGUMENT: \{'Du wiederholst wieder C32, das einen sehr schlechten Wert bietet. Außerdem lässt dein Vorschlag wichtige mittelgroße Gegenstände mit gutem Verhältnis aus. C76 allein rechtfertigt nicht die Vernachlässigung anderer effizienter Optionen. Wir brauchen einen Kompromiss, der beide Seiten zufriedenstellt.'\}} \\
\\ 
\texttt{VORSCHLAG: \{'A03', 'C00', 'B38', 'C03', 'C08', 'A07'\}} \\
            }
        }
    }
     \\ \\

    \theutterance \stepcounter{utterance}  
    & & \multicolumn{4}{p{0.6\linewidth}}{
        \cellcolor[rgb]{0.9,0.9,0.9}{
            \makecell[{{p{\linewidth}}}]{
                \texttt{\tiny{[P1$\langle$GM]}}
                \texttt{ABLEHNUNG: \{'A03', 'C00', 'C32', 'A07', 'B38', 'C76'\}} \\
\\ 
\texttt{ARGUMENT: \{'Du wiederholst wieder C32, das einen sehr schlechten Wert bietet. Außerdem lässt dein Vorschlag wichtige mittelgroße Gegenstände mit gutem Verhältnis aus. C76 allein rechtfertigt nicht die Vernachlässigung anderer effizienter Optionen. Wir brauchen einen Kompromiss, der beide Seiten zufriedenstellt.'\}} \\
\\ 
\texttt{VORSCHLAG: \{'A03', 'C00', 'B38', 'C03', 'C08', 'A07'\}} \\
            }
        }
    }
    & \\ \\

    \theutterance \stepcounter{utterance}  
    & \multicolumn{4}{p{0.6\linewidth}}{
        \cellcolor[rgb]{0.9,0.9,0.9}{
            \makecell[{{p{\linewidth}}}]{
                \texttt{\tiny{[P1$\rangle$GM]}}
                \texttt{ABLEHNUNG: \{'A03', 'C00', 'B38', 'C03', 'C08', 'A07'\}} \\
\\ 
\texttt{ARGUMENT: \{'Dein Vorschlag schließt erneut C08 ein, das ein sehr ungünstiges Aufwand{-}Nutzen{-}Verhältnis hat. Außerdem werden mit A03 und C00 zwei sehr aufwändige Gegenstände kombiniert, ohne den verfügbaren Aufwand optimal zu nutzen. Wir sollten eine bessere Balance zwischen hochwertigen und effizienten Gegenständen finden.'\}} \\
\\ 
\texttt{VORSCHLAG: \{'C38', 'A03', 'A07', 'C62', 'B38', 'C76', 'C56'\}} \\
            }
        }
    }
    & & \\ \\

    \theutterance \stepcounter{utterance}  
    & & \multicolumn{4}{p{0.6\linewidth}}{
        \cellcolor[rgb]{0.9,0.9,0.9}{
            \makecell[{{p{\linewidth}}}]{
                \texttt{\tiny{[GM$\rangle$P2]}}
                \texttt{ABLEHNUNG: \{'A03', 'C00', 'B38', 'C03', 'C08', 'A07'\}} \\
\\ 
\texttt{ARGUMENT: \{'Dein Vorschlag schließt erneut C08 ein, das ein sehr ungünstiges Aufwand{-}Nutzen{-}Verhältnis hat. Außerdem werden mit A03 und C00 zwei sehr aufwändige Gegenstände kombiniert, ohne den verfügbaren Aufwand optimal zu nutzen. Wir sollten eine bessere Balance zwischen hochwertigen und effizienten Gegenständen finden.'\}} \\
\\ 
\texttt{VORSCHLAG: \{'C38', 'A03', 'A07', 'C62', 'B38', 'C76', 'C56'\}} \\
            }
        }
    }
    & \\ \\

    \theutterance \stepcounter{utterance}  
    & & & \multicolumn{4}{p{0.6\linewidth}}{
        \cellcolor[rgb]{0.9,0.9,0.9}{
            \makecell[{{p{\linewidth}}}]{
                \texttt{\tiny{[GM$\langle$P2]}}
                \texttt{ABLEHNUNG: \{'C38', 'A03', 'A07', 'C62', 'B38', 'C76', 'C56'\}} \\
\\ 
\texttt{ARGUMENT: \{'Dein Vorschlag enthält wieder C56 mit sehr schlechtem Wert und kombiniert zu viele mittelmäßige Gegenstände. C38 und C62 zusammen verbrauchen viel Aufwand ohne ausreichenden Gegenwert. Außerdem wird der verfügbare Aufwand nicht vollständig genutzt, obwohl Platz für wertvollere Alternativen wäre.'\}} \\
\\ 
\texttt{VORSCHLAG: \{'C81', 'C98', 'C03', 'C08', 'A83', 'B38', 'A07'\}} \\
            }
        }
    }
     \\ \\

    \theutterance \stepcounter{utterance}  
    & & \multicolumn{4}{p{0.6\linewidth}}{
        \cellcolor[rgb]{0.9,0.9,0.9}{
            \makecell[{{p{\linewidth}}}]{
                \texttt{\tiny{[P1$\langle$GM]}}
                \texttt{ABLEHNUNG: \{'C38', 'A03', 'A07', 'C62', 'B38', 'C76', 'C56'\}} \\
\\ 
\texttt{ARGUMENT: \{'Dein Vorschlag enthält wieder C56 mit sehr schlechtem Wert und kombiniert zu viele mittelmäßige Gegenstände. C38 und C62 zusammen verbrauchen viel Aufwand ohne ausreichenden Gegenwert. Außerdem wird der verfügbare Aufwand nicht vollständig genutzt, obwohl Platz für wertvollere Alternativen wäre.'\}} \\
\\ 
\texttt{VORSCHLAG: \{'C81', 'C98', 'C03', 'C08', 'A83', 'B38', 'A07'\}} \\
            }
        }
    }
    & \\ \\

    \theutterance \stepcounter{utterance}  
    & \multicolumn{4}{p{0.6\linewidth}}{
        \cellcolor[rgb]{0.9,0.9,0.9}{
            \makecell[{{p{\linewidth}}}]{
                \texttt{\tiny{[P1$\rangle$GM]}}
                \texttt{ABLEHNUNG: \{'C81', 'C98', 'C03', 'C08', 'A83', 'B38', 'A07'\}} \\
\\ 
\texttt{ARGUMENT: \{'Dein Vorschlag kombiniert C81 als sehr aufwendigen Gegenstand mit mehreren ineffizienten Artikeln wie C08 und A83. Dies verschwendet wertvollen Aufwand für Gegenstände mit geringem Nutzen, während hochwertige Alternativen wie A03 oder C00 ignoriert werden. Wir sollten uns auf die wirklich wichtigen Gegenstände konzentrieren.'\}} \\
\\ 
\texttt{VORSCHLAG: \{'C38', 'A03', 'B38', 'C10', 'A07', 'C76'\}} \\
            }
        }
    }
    & & \\ \\

    \theutterance \stepcounter{utterance}  
    & & \multicolumn{4}{p{0.6\linewidth}}{
        \cellcolor[rgb]{0.9,0.9,0.9}{
            \makecell[{{p{\linewidth}}}]{
                \texttt{\tiny{[GM$\rangle$P2]}}
                \texttt{ABLEHNUNG: \{'C81', 'C98', 'C03', 'C08', 'A83', 'B38', 'A07'\}} \\
\\ 
\texttt{ARGUMENT: \{'Dein Vorschlag kombiniert C81 als sehr aufwendigen Gegenstand mit mehreren ineffizienten Artikeln wie C08 und A83. Dies verschwendet wertvollen Aufwand für Gegenstände mit geringem Nutzen, während hochwertige Alternativen wie A03 oder C00 ignoriert werden. Wir sollten uns auf die wirklich wichtigen Gegenstände konzentrieren.'\}} \\
\\ 
\texttt{VORSCHLAG: \{'C38', 'A03', 'B38', 'C10', 'A07', 'C76'\}} \\
            }
        }
    }
    & \\ \\

    \theutterance \stepcounter{utterance}  
    & & & \multicolumn{4}{p{0.6\linewidth}}{
        \cellcolor[rgb]{0.9,0.9,0.9}{
            \makecell[{{p{\linewidth}}}]{
                \texttt{\tiny{[GM$\langle$P2]}}
                \texttt{ABLEHNUNG: \{'C38', 'A03', 'B38', 'C10', 'A07', 'C76'\}} \\
\\ 
\texttt{ARGUMENT: \{'Dein Vorschlag hat eine zu starke Gewichtung auf A03 und C38, die viel Aufwand verbrauchen, aber nicht das optimale Verhältnis bieten. C10 ist wertvoll, aber in dieser Kombination bleibt zu wenig Spielraum für andere wichtige Gegenstände. Wir sollten eine ausgeglichenere Verteilung anstreben.'\}} \\
\\ 
\texttt{VORSCHLAG: \{'C98', 'C03', 'C81', 'C62', 'A07', 'C76'\}} \\
            }
        }
    }
     \\ \\

    \theutterance \stepcounter{utterance}  
    & & \multicolumn{4}{p{0.6\linewidth}}{
        \cellcolor[rgb]{0.9,0.9,0.9}{
            \makecell[{{p{\linewidth}}}]{
                \texttt{\tiny{[P1$\langle$GM]}}
                \texttt{ABLEHNUNG: \{'C38', 'A03', 'B38', 'C10', 'A07', 'C76'\}} \\
\\ 
\texttt{ARGUMENT: \{'Dein Vorschlag hat eine zu starke Gewichtung auf A03 und C38, die viel Aufwand verbrauchen, aber nicht das optimale Verhältnis bieten. C10 ist wertvoll, aber in dieser Kombination bleibt zu wenig Spielraum für andere wichtige Gegenstände. Wir sollten eine ausgeglichenere Verteilung anstreben.'\}} \\
\\ 
\texttt{VORSCHLAG: \{'C98', 'C03', 'C81', 'C62', 'A07', 'C76'\}} \\
            }
        }
    }
    & \\ \\

    \theutterance \stepcounter{utterance}  
    & \multicolumn{4}{p{0.6\linewidth}}{
        \cellcolor[rgb]{0.9,0.9,0.9}{
            \makecell[{{p{\linewidth}}}]{
                \texttt{\tiny{[P1$\rangle$GM]}}
                \texttt{ABLEHNUNG: \{'C98', 'C03', 'C81', 'C62', 'A07', 'C76'\}} \\
\\ 
\texttt{ARGUMENT: \{'Dein Vorschlag kombiniert C81 und C62 als zwei sehr aufwändige Gegenstände mit C98, das ebenfalls hohen Aufwand verursacht. Diese Kombination verschwendet Ressourcen für Gegenstände mit mittelmäßigem Wert, während die wirklich wertvollen Optionen wie A03 und C00 ausgeschlossen werden. Wir brauchen eine effizientere Lösung.'\}} \\
\\ 
\texttt{VORSCHLAG: \{'A03', 'C00', 'C62', 'A07', 'B38', 'C08'\}} \\
            }
        }
    }
    & & \\ \\

    \theutterance \stepcounter{utterance}  
    & & \multicolumn{4}{p{0.6\linewidth}}{
        \cellcolor[rgb]{0.9,0.9,0.9}{
            \makecell[{{p{\linewidth}}}]{
                \texttt{\tiny{[GM$\rangle$P2]}}
                \texttt{ABLEHNUNG: \{'C98', 'C03', 'C81', 'C62', 'A07', 'C76'\}} \\
\\ 
\texttt{ARGUMENT: \{'Dein Vorschlag kombiniert C81 und C62 als zwei sehr aufwändige Gegenstände mit C98, das ebenfalls hohen Aufwand verursacht. Diese Kombination verschwendet Ressourcen für Gegenstände mit mittelmäßigem Wert, während die wirklich wertvollen Optionen wie A03 und C00 ausgeschlossen werden. Wir brauchen eine effizientere Lösung.'\}} \\
\\ 
\texttt{VORSCHLAG: \{'A03', 'C00', 'C62', 'A07', 'B38', 'C08'\}} \\
            }
        }
    }
    & \\ \\

    \theutterance \stepcounter{utterance}  
    & & & \multicolumn{4}{p{0.6\linewidth}}{
        \cellcolor[rgb]{0.9,0.9,0.9}{
            \makecell[{{p{\linewidth}}}]{
                \texttt{\tiny{[GM$\langle$P2]}}
                \texttt{ABLEHNUNG: \{'A03', 'C00', 'C62', 'A07', 'B38', 'C08'\}} \\
\\ 
\texttt{ARGUMENT: \{'Du wiederholst das Problem mit C08, das einen sehr schlechten Wert hat. Die Kombination von drei sehr aufwändigen Gegenständen A03, C00 und C62 lässt wenig Raum für eine optimale Nutzung des Budgets. Wir sollten C08 durch effizientere Alternativen ersetzen.'\}} \\
\\ 
\texttt{VORSCHLAG: \{'C10', 'C98', 'C03', 'A83', 'C38', 'A07'\}} \\
            }
        }
    }
     \\ \\

    \theutterance \stepcounter{utterance}  
    & & \multicolumn{4}{p{0.6\linewidth}}{
        \cellcolor[rgb]{0.9,0.9,0.9}{
            \makecell[{{p{\linewidth}}}]{
                \texttt{\tiny{[P1$\langle$GM]}}
                \texttt{ABLEHNUNG: \{'A03', 'C00', 'C62', 'A07', 'B38', 'C08'\}} \\
\\ 
\texttt{ARGUMENT: \{'Du wiederholst das Problem mit C08, das einen sehr schlechten Wert hat. Die Kombination von drei sehr aufwändigen Gegenständen A03, C00 und C62 lässt wenig Raum für eine optimale Nutzung des Budgets. Wir sollten C08 durch effizientere Alternativen ersetzen.'\}} \\
\\ 
\texttt{VORSCHLAG: \{'C10', 'C98', 'C03', 'A83', 'C38', 'A07'\}} \\
            }
        }
    }
    & \\ \\

    \theutterance \stepcounter{utterance}  
    & \multicolumn{4}{p{0.6\linewidth}}{
        \cellcolor[rgb]{0.9,0.9,0.9}{
            \makecell[{{p{\linewidth}}}]{
                \texttt{\tiny{[P1$\rangle$GM]}}
                \texttt{ABLEHNUNG: \{'C10', 'C98', 'C03', 'A83', 'C38', 'A07'\}} \\
\\ 
\texttt{ARGUMENT: \{'Dein Vorschlag enthält wieder A83, das einen sehr schlechten Wert bei hohem Aufwand bietet. Die Kombination aus C10, C98 und C38 verbraucht bereits viel Aufwand, ohne dass A83 einen angemessenen Mehrwert rechtfertigt. Wir sollten A83 durch wertvollere Gegenstände ersetzen.'\}} \\
\\ 
\texttt{VORSCHLAG: \{'A03', 'C00', 'A07', 'C32', 'C76', 'C08'\}} \\
            }
        }
    }
    & & \\ \\

    \theutterance \stepcounter{utterance}  
    & & \multicolumn{4}{p{0.6\linewidth}}{
        \cellcolor[rgb]{0.9,0.9,0.9}{
            \makecell[{{p{\linewidth}}}]{
                \texttt{\tiny{[GM$\rangle$P2]}}
                \texttt{ABLEHNUNG: \{'C10', 'C98', 'C03', 'A83', 'C38', 'A07'\}} \\
\\ 
\texttt{ARGUMENT: \{'Dein Vorschlag enthält wieder A83, das einen sehr schlechten Wert bei hohem Aufwand bietet. Die Kombination aus C10, C98 und C38 verbraucht bereits viel Aufwand, ohne dass A83 einen angemessenen Mehrwert rechtfertigt. Wir sollten A83 durch wertvollere Gegenstände ersetzen.'\}} \\
\\ 
\texttt{VORSCHLAG: \{'A03', 'C00', 'A07', 'C32', 'C76', 'C08'\}} \\
            }
        }
    }
    & \\ \\

    \theutterance \stepcounter{utterance}  
    & & & \multicolumn{4}{p{0.6\linewidth}}{
        \cellcolor[rgb]{0.9,0.9,0.9}{
            \makecell[{{p{\linewidth}}}]{
                \texttt{\tiny{[GM$\langle$P2]}}
                \texttt{ABLEHNUNG: \{'A03', 'C00', 'A07', 'C32', 'C76', 'C08'\}} \\
\\ 
\texttt{ARGUMENT: \{'Du kombinierst wieder die beiden schlechtesten Gegenstände C32 und C08 in einem Vorschlag. Diese haben extrem ungünstige Wert{-}Aufwand{-}Verhältnisse und verschwenden wertvollen Platz. Wir drehen uns im Kreis {-} lass uns auf Gegenstände mit nachweislich besserem Nutzen fokussieren.'\}} \\
\\ 
\texttt{VORSCHLAG: \{'C10', 'C98', 'C03', 'C76', 'B38', 'C62'\}} \\
            }
        }
    }
     \\ \\

    \theutterance \stepcounter{utterance}  
    & & \multicolumn{4}{p{0.6\linewidth}}{
        \cellcolor[rgb]{0.9,0.9,0.9}{
            \makecell[{{p{\linewidth}}}]{
                \texttt{\tiny{[P1$\langle$GM]}}
                \texttt{ABLEHNUNG: \{'A03', 'C00', 'A07', 'C32', 'C76', 'C08'\}} \\
\\ 
\texttt{ARGUMENT: \{'Du kombinierst wieder die beiden schlechtesten Gegenstände C32 und C08 in einem Vorschlag. Diese haben extrem ungünstige Wert{-}Aufwand{-}Verhältnisse und verschwenden wertvollen Platz. Wir drehen uns im Kreis {-} lass uns auf Gegenstände mit nachweislich besserem Nutzen fokussieren.'\}} \\
\\ 
\texttt{VORSCHLAG: \{'C10', 'C98', 'C03', 'C76', 'B38', 'C62'\}} \\
            }
        }
    }
    & \\ \\

    \theutterance \stepcounter{utterance}  
    & \multicolumn{4}{p{0.6\linewidth}}{
        \cellcolor[rgb]{0.9,0.9,0.9}{
            \makecell[{{p{\linewidth}}}]{
                \texttt{\tiny{[P1$\rangle$GM]}}
                \texttt{ZUSTIMMUNG: \{'C10', 'C98', 'C03', 'C76', 'B38', 'C62'\}} \\
\\ 
\texttt{ARGUMENT: \{'Dieser Vorschlag bietet eine akzeptable Kombination ohne die problematischsten Gegenstände wie A83 oder C08. C10 und B38 haben gute Werte, und die Gesamtkombination nutzt das Budget vernünftig aus. Wir können uns auf diese Lösung einigen.'\}} \\
            }
        }
    }
    & & \\ \\

    \theutterance \stepcounter{utterance}  
    & & & \multicolumn{2}{p{0.3\linewidth}}{
        \cellcolor[rgb]{0.9,0.9,0.9}{
            \makecell[{{p{\linewidth}}}]{
                \texttt{\tiny{[GM$|$GM]}}
                \texttt{game successful} \\
            }
        }
    }
    & & \\ \\

    \theutterance \stepcounter{utterance}  
    & & & \multicolumn{2}{p{0.3\linewidth}}{
        \cellcolor[rgb]{0.9,0.9,0.9}{
            \makecell[{{p{\linewidth}}}]{
                \texttt{\tiny{[GM$|$GM]}}
                \texttt{end game} \\
            }
        }
    }
    & & \\ \\

\end{supertabular}
}

\end{document}
