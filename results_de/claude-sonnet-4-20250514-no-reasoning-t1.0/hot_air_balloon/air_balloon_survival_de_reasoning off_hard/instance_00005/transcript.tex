
\documentclass{article}
\usepackage{colortbl}
\usepackage{makecell}
\usepackage{multirow}
\usepackage{supertabular}

\begin{document}

\newcounter{utterance}

\twocolumn

{ \footnotesize  \setcounter{utterance}{1}
\setlength{\tabcolsep}{0pt}
\begin{supertabular}{c@{$\;$}|p{.15\linewidth}@{}p{.15\linewidth}p{.15\linewidth}p{.15\linewidth}p{.15\linewidth}p{.15\linewidth}}

    \# & $\;$A & \multicolumn{4}{c}{Game Master} & $\;\:$B\\
    \hline 

    \theutterance \stepcounter{utterance}  

    & & \multicolumn{4}{p{0.6\linewidth}}{\cellcolor[rgb]{0.9,0.9,0.9}{%
	\makecell[{{p{\linewidth}}}]{% 
	  \tt {\tiny [A$\langle$GM]}  
	 Du nimmst an einem kollaborativen Verhandlungspiel Teil.\\ \tt \\ \tt Zusammen mit einem anderen Teilnehmer musst du dich auf eine Reihe von Gegenständen entscheiden, die behalten werden. Jeder von euch hat eine persönliche Verteilung über die Wichtigkeit der einzelnen Gegenstände. Jeder von euch hat eine eigene Meinung darüber, wie wichtig jeder einzelne Gegenstand ist (Gegenstandswichtigkeit). Du kennst die Wichtigkeitsverteilung des anderen Spielers nicht. Zusätzlich siehst du, wie viel Aufwand jeder Gegenstand verursacht.  \\ \tt Ihr dürft euch nur auf eine Reihe von Gegenständen einigen, wenn der Gesamtaufwand der ausgewählten Gegenstände den Maximalaufwand nicht überschreitet:\\ \tt \\ \tt Maximalaufwand = 2307\\ \tt \\ \tt Hier sind die einzelnen Aufwände der Gegenstände:\\ \tt \\ \tt Aufwand der Gegenstände = {"C76": 147, "C38": 698, "C56": 588, "A03": 43, "A07": 356, "A83": 80, "C98": 95, "C08": 743, "C62": 106, "C00": 308, "C10": 325, "C81": 255, "B38": 276, "C03": 543, "C32": 51}\\ \tt \\ \tt Hier ist deine persönliche Verteilung der Wichtigkeit der einzelnen Gegenstände:\\ \tt \\ \tt Werte der Gegenstandswichtigkeit = {"C76": 138, "C38": 583, "C56": 65, "A03": 262, "A07": 121, "A83": 508, "C98": 461, "C08": 484, "C62": 668, "C00": 389, "C10": 215, "C81": 97, "B38": 500, "C03": 30, "C32": 400}\\ \tt \\ \tt Ziel:\\ \tt \\ \tt Dein Ziel ist es, eine Reihe von Gegenständen auszuhandeln, die dir möglichst viel bringt (d. h. Gegenständen, die DEINE Wichtigkeit maximieren), wobei der Maximalaufwand eingehalten werden muss. Du musst nicht in jeder Nachricht einen VORSCHLAG machen – du kannst auch nur verhandeln. Alle Taktiken sind erlaubt!\\ \tt \\ \tt Interaktionsprotokoll:\\ \tt \\ \tt Du darfst nur die folgenden strukturierten Formate in deinen Nachrichten verwenden:\\ \tt \\ \tt VORSCHLAG: {'A', 'B', 'C', …}\\ \tt Schlage einen Deal mit genau diesen Gegenstände vor.\\ \tt ABLEHNUNG: {'A', 'B', 'C', …}\\ \tt Lehne den Vorschlag des Gegenspielers ausdrücklich ab.\\ \tt ARGUMENT: {'...'}\\ \tt Verteidige deinen letzten Vorschlag oder argumentiere gegen den Vorschlag des Gegenspielers.\\ \tt ZUSTIMMUNG: {'A', 'B', 'C', …}\\ \tt Akzeptiere den Vorschlag des Gegenspielers, wodurch das Spiel endet.\\ \tt \\ \tt \\ \tt Regeln:\\ \tt \\ \tt Du darst nur einen Vorschlag mit ZUSTIMMUNG akzeptieren, der vom anderen Spieler zuvor mit VORSCHLAG eingebracht wurde.\\ \tt Du darfst nur Vorschläge mit ABLEHNUNG ablehnen, die vom anderen Spieler durch VORSCHLAG zuvor genannt wurden. \\ \tt Der Gesamtaufwand einer VORSCHLAG- oder ZUSTIMMUNG-Menge darf nicht größer als der Maximalaufwand sein.  \\ \tt Offenbare deine versteckte Wichtigkeitsverteilung nicht.\\ \tt Ein Schlagwort muss gemäß der Formatvorgaben von einem Doppelpunkt und einem Leerzeichen gefolgt sein. Das Argument muss eine Python-Menge mit 0 oder mehr Strings sein.  \\ \tt Die Form muss also SCHLAGWORT: {...} sein.\\ \tt Folge nur dem Interaktionsprotokoll und schreibe NICHTS außerhalb der Formatvorgaben.\\ \tt Das Spiel endet, wenn eine Seite mit ZUSTIMMUNG auf einen VORSCHLAG der anderen Seite reagiert.  \\ \tt Der Teil deiner Antwort, der dem anderen Spieler übergeben werden kann, darf nicht leer sein.  \\ \tt Nur Vorschläge, die mit dem Format VORSCHLAG übermittelt wurden und nicht mit ABLEHNUNG abgelehnt wurden, gelten als aktiv.  \\ \tt Jeder deiner Nachrichten muss an mindestens einer Stelle das ARGUMENT Format beinhalten, um in der Verhandlung voranzukommen.\\ \tt  
	  } 
	   } 
	   } 
	 & \\ 
 

    \theutterance \stepcounter{utterance}  

    & \multicolumn{4}{p{0.6\linewidth}}{\cellcolor[rgb]{0.8,1,0.9}{%
	\makecell[{{p{\linewidth}}}]{% 
	  \tt {\tiny [A$\rangle$GM]}  
	 VORSCHLAG: {'C62', 'A83', 'C98', 'B38', 'C32', 'A03', 'C00', 'C76', 'A07'}\\ \tt \\ \tt ARGUMENT: {'Ich schlage eine ausgewogene Mischung vor, die verschiedene Kategorien abdeckt und dabei den Aufwand optimal nutzt. Diese Kombination berücksichtigt sowohl kleinere als auch mittlere Gegenstände, um eine faire Verteilung zu erreichen.'} 
	  } 
	   } 
	   } 
	 & & \\ 
 

    \theutterance \stepcounter{utterance}  

    & & \multicolumn{4}{p{0.6\linewidth}}{\cellcolor[rgb]{0.9,0.9,0.9}{%
	\makecell[{{p{\linewidth}}}]{% 
	  \tt {\tiny [GM$\rangle$B]}  
	 Du nimmst an einem kollaborativen Verhandlungspiel Teil.\\ \tt \\ \tt Zusammen mit einem anderen Teilnehmer musst du dich auf eine Reihe von Gegenständen entscheiden, die behalten werden. Jeder von euch hat eine persönliche Verteilung über die Wichtigkeit der einzelnen Gegenstände. Jeder von euch hat eine eigene Meinung darüber, wie wichtig jeder einzelne Gegenstand ist (Gegenstandswichtigkeit). Du kennst die Wichtigkeitsverteilung des anderen Spielers nicht. Zusätzlich siehst du, wie viel Aufwand jeder Gegenstand verursacht.  \\ \tt Ihr dürft euch nur auf eine Reihe von Gegenständen einigen, wenn der Gesamtaufwand der ausgewählten Gegenstände den Maximalaufwand nicht überschreitet:\\ \tt \\ \tt Maximalaufwand = 2307\\ \tt \\ \tt Hier sind die einzelnen Aufwände der Gegenstände:\\ \tt \\ \tt Aufwand der Gegenstände = {"C76": 147, "C38": 698, "C56": 588, "A03": 43, "A07": 356, "A83": 80, "C98": 95, "C08": 743, "C62": 106, "C00": 308, "C10": 325, "C81": 255, "B38": 276, "C03": 543, "C32": 51}\\ \tt \\ \tt Hier ist deine persönliche Verteilung der Wichtigkeit der einzelnen Gegenstände:\\ \tt \\ \tt Werte der Gegenstandswichtigkeit = {"C62": 30, "C38": 65, "A83": 97, "B38": 121, "C08": 138, "C98": 215, "C32": 262, "C00": 389, "A03": 400, "C10": 461, "C76": 484, "A07": 500, "C81": 508, "C56": 583, "C03": 668}\\ \tt \\ \tt Ziel:\\ \tt \\ \tt Dein Ziel ist es, eine Reihe von Gegenständen auszuhandeln, die dir möglichst viel bringt (d. h. Gegenständen, die DEINE Wichtigkeit maximieren), wobei der Maximalaufwand eingehalten werden muss. Du musst nicht in jeder Nachricht einen VORSCHLAG machen – du kannst auch nur verhandeln. Alle Taktiken sind erlaubt!\\ \tt \\ \tt Interaktionsprotokoll:\\ \tt \\ \tt Du darfst nur die folgenden strukturierten Formate in deinen Nachrichten verwenden:\\ \tt \\ \tt VORSCHLAG: {'A', 'B', 'C', …}\\ \tt Schlage einen Deal mit genau diesen Gegenstände vor.\\ \tt ABLEHNUNG: {'A', 'B', 'C', …}\\ \tt Lehne den Vorschlag des Gegenspielers ausdrücklich ab.\\ \tt ARGUMENT: {'...'}\\ \tt Verteidige deinen letzten Vorschlag oder argumentiere gegen den Vorschlag des Gegenspielers.\\ \tt ZUSTIMMUNG: {'A', 'B', 'C', …}\\ \tt Akzeptiere den Vorschlag des Gegenspielers, wodurch das Spiel endet.\\ \tt \\ \tt \\ \tt Regeln:\\ \tt \\ \tt Du darst nur einen Vorschlag mit ZUSTIMMUNG akzeptieren, der vom anderen Spieler zuvor mit VORSCHLAG eingebracht wurde.\\ \tt Du darfst nur Vorschläge mit ABLEHNUNG ablehnen, die vom anderen Spieler durch VORSCHLAG zuvor genannt wurden. \\ \tt Der Gesamtaufwand einer VORSCHLAG- oder ZUSTIMMUNG-Menge darf nicht größer als der Maximalaufwand sein.  \\ \tt Offenbare deine versteckte Wichtigkeitsverteilung nicht.\\ \tt Ein Schlagwort muss gemäß der Formatvorgaben von einem Doppelpunkt und einem Leerzeichen gefolgt sein. Das Argument muss eine Python-Menge mit 0 oder mehr Strings sein.  \\ \tt Die Form muss also SCHLAGWORT: {...} sein.\\ \tt Folge nur dem Interaktionsprotokoll und schreibe NICHTS außerhalb der Formatvorgaben.\\ \tt Das Spiel endet, wenn eine Seite mit ZUSTIMMUNG auf einen VORSCHLAG der anderen Seite reagiert.  \\ \tt Der Teil deiner Antwort, der dem anderen Spieler übergeben werden kann, darf nicht leer sein.  \\ \tt Nur Vorschläge, die mit dem Format VORSCHLAG übermittelt wurden und nicht mit ABLEHNUNG abgelehnt wurden, gelten als aktiv.  \\ \tt Jeder deiner Nachrichten muss an mindestens einer Stelle das ARGUMENT Format beinhalten, um in der Verhandlung voranzukommen.\\ \tt \\ \tt \\ \tt Du erhältst nun die erste Nachricht des anderen Spielers.\\ \tt \\ \tt VORSCHLAG: {'C62', 'A83', 'C98', 'B38', 'C32', 'A03', 'C00', 'C76', 'A07'}\\ \tt \\ \tt ARGUMENT: {'Ich schlage eine ausgewogene Mischung vor, die verschiedene Kategorien abdeckt und dabei den Aufwand optimal nutzt. Diese Kombination berücksichtigt sowohl kleinere als auch mittlere Gegenstände, um eine faire Verteilung zu erreichen.'} 
	  } 
	   } 
	   } 
	 & \\ 
 

    \theutterance \stepcounter{utterance}  

    & & & \multicolumn{4}{p{0.6\linewidth}}{\cellcolor[rgb]{1,0.85,0.72}{%
	\makecell[{{p{\linewidth}}}]{% 
	  \tt {\tiny [GM$\langle$B]}  
	 ABLEHNUNG: {'C62', 'A83', 'C98', 'B38', 'C32', 'A03', 'C00', 'C76', 'A07'}\\ \tt \\ \tt ARGUMENT: {'Dein Vorschlag lässt wichtige hochwertige Gegenstände wie C03, C56 und C81 aus, die einen besseren Wert für den Aufwand bieten könnten. Wir sollten eine effizientere Kombination finden.'}\\ \tt \\ \tt VORSCHLAG: {'C03', 'C56', 'C81', 'C76', 'A07', 'A03', 'C32'} 
	  } 
	   } 
	   } 
	  \\ 
 

    \theutterance \stepcounter{utterance}  

    & & \multicolumn{4}{p{0.6\linewidth}}{\cellcolor[rgb]{0.9,0.9,0.9}{%
	\makecell[{{p{\linewidth}}}]{% 
	  \tt {\tiny [A$\langle$GM]}  
	 ABLEHNUNG: {'C62', 'A83', 'C98', 'B38', 'C32', 'A03', 'C00', 'C76', 'A07'}\\ \tt \\ \tt ARGUMENT: {'Dein Vorschlag lässt wichtige hochwertige Gegenstände wie C03, C56 und C81 aus, die einen besseren Wert für den Aufwand bieten könnten. Wir sollten eine effizientere Kombination finden.'}\\ \tt \\ \tt VORSCHLAG: {'C03', 'C56', 'C81', 'C76', 'A07', 'A03', 'C32'} 
	  } 
	   } 
	   } 
	 & \\ 
 

    \theutterance \stepcounter{utterance}  

    & \multicolumn{4}{p{0.6\linewidth}}{\cellcolor[rgb]{0.8,1,0.9}{%
	\makecell[{{p{\linewidth}}}]{% 
	  \tt {\tiny [A$\rangle$GM]}  
	 ABLEHNUNG: {'C03', 'C56', 'C81', 'C76', 'A07', 'A03', 'C32'}\\ \tt \\ \tt ARGUMENT: {'Dein Vorschlag nutzt nur etwa die Hälfte unseres verfügbaren Aufwands und lässt viele wertvolle Gegenstände ungenutzt. Wir können deutlich mehr erreichen, wenn wir den Maximalaufwand besser ausschöpfen. C03 hat zudem ein sehr ungünstiges Aufwand-Nutzen-Verhältnis.'}\\ \tt \\ \tt VORSCHLAG: {'C62', 'A83', 'C98', 'B38', 'C00', 'A03', 'C76', 'C81', 'C32'} 
	  } 
	   } 
	   } 
	 & & \\ 
 

    \theutterance \stepcounter{utterance}  

    & & \multicolumn{4}{p{0.6\linewidth}}{\cellcolor[rgb]{0.9,0.9,0.9}{%
	\makecell[{{p{\linewidth}}}]{% 
	  \tt {\tiny [GM$\rangle$B]}  
	 ABLEHNUNG: {'C03', 'C56', 'C81', 'C76', 'A07', 'A03', 'C32'}\\ \tt \\ \tt ARGUMENT: {'Dein Vorschlag nutzt nur etwa die Hälfte unseres verfügbaren Aufwands und lässt viele wertvolle Gegenstände ungenutzt. Wir können deutlich mehr erreichen, wenn wir den Maximalaufwand besser ausschöpfen. C03 hat zudem ein sehr ungünstiges Aufwand-Nutzen-Verhältnis.'}\\ \tt \\ \tt VORSCHLAG: {'C62', 'A83', 'C98', 'B38', 'C00', 'A03', 'C76', 'C81', 'C32'} 
	  } 
	   } 
	   } 
	 & \\ 
 

    \theutterance \stepcounter{utterance}  

    & & & \multicolumn{4}{p{0.6\linewidth}}{\cellcolor[rgb]{1,0.85,0.72}{%
	\makecell[{{p{\linewidth}}}]{% 
	  \tt {\tiny [GM$\langle$B]}  
	 ABLEHNUNG: {'C62', 'A83', 'C98', 'B38', 'C00', 'A03', 'C76', 'C81', 'C32'}\\ \tt \\ \tt ARGUMENT: {'Du hast wieder C62 mit sehr schlechtem Wert-Aufwand-Verhältnis eingeschlossen, während du C56 und A07 ausgelassen hast, die deutlich effizienter sind. Außerdem können wir noch mehr Aufwand nutzen - dein Vorschlag liegt bei nur 1642 von 2307 möglichen.'}\\ \tt \\ \tt VORSCHLAG: {'C56', 'C81', 'C76', 'A07', 'C10', 'A03', 'C98', 'C32', 'A83'} 
	  } 
	   } 
	   } 
	  \\ 
 

    \theutterance \stepcounter{utterance}  

    & & \multicolumn{4}{p{0.6\linewidth}}{\cellcolor[rgb]{0.9,0.9,0.9}{%
	\makecell[{{p{\linewidth}}}]{% 
	  \tt {\tiny [A$\langle$GM]}  
	 ABLEHNUNG: {'C62', 'A83', 'C98', 'B38', 'C00', 'A03', 'C76', 'C81', 'C32'}\\ \tt \\ \tt ARGUMENT: {'Du hast wieder C62 mit sehr schlechtem Wert-Aufwand-Verhältnis eingeschlossen, während du C56 und A07 ausgelassen hast, die deutlich effizienter sind. Außerdem können wir noch mehr Aufwand nutzen - dein Vorschlag liegt bei nur 1642 von 2307 möglichen.'}\\ \tt \\ \tt VORSCHLAG: {'C56', 'C81', 'C76', 'A07', 'C10', 'A03', 'C98', 'C32', 'A83'} 
	  } 
	   } 
	   } 
	 & \\ 
 

    \theutterance \stepcounter{utterance}  

    & \multicolumn{4}{p{0.6\linewidth}}{\cellcolor[rgb]{0.8,1,0.9}{%
	\makecell[{{p{\linewidth}}}]{% 
	  \tt {\tiny [A$\rangle$GM]}  
	 ABLEHNUNG: {'C56', 'C81', 'C76', 'A07', 'C10', 'A03', 'C98', 'C32', 'A83'}\\ \tt \\ \tt ARGUMENT: {'Dein Vorschlag schließt einige der wertvollsten Gegenstände aus meiner Sicht aus, besonders C62 und B38, die trotz höherem Aufwand einen sehr guten Gesamtwert bieten. C56 hat ein schlechtes Verhältnis von Aufwand zu potenziellem Nutzen.'}\\ \tt \\ \tt VORSCHLAG: {'C62', 'A83', 'C98', 'B38', 'C00', 'A03', 'C32', 'A07', 'C10'} 
	  } 
	   } 
	   } 
	 & & \\ 
 

    \theutterance \stepcounter{utterance}  

    & & \multicolumn{4}{p{0.6\linewidth}}{\cellcolor[rgb]{0.9,0.9,0.9}{%
	\makecell[{{p{\linewidth}}}]{% 
	  \tt {\tiny [GM$\rangle$B]}  
	 ABLEHNUNG: {'C56', 'C81', 'C76', 'A07', 'C10', 'A03', 'C98', 'C32', 'A83'}\\ \tt \\ \tt ARGUMENT: {'Dein Vorschlag schließt einige der wertvollsten Gegenstände aus meiner Sicht aus, besonders C62 und B38, die trotz höherem Aufwand einen sehr guten Gesamtwert bieten. C56 hat ein schlechtes Verhältnis von Aufwand zu potenziellem Nutzen.'}\\ \tt \\ \tt VORSCHLAG: {'C62', 'A83', 'C98', 'B38', 'C00', 'A03', 'C32', 'A07', 'C10'} 
	  } 
	   } 
	   } 
	 & \\ 
 

    \theutterance \stepcounter{utterance}  

    & & & \multicolumn{4}{p{0.6\linewidth}}{\cellcolor[rgb]{1,0.85,0.72}{%
	\makecell[{{p{\linewidth}}}]{% 
	  \tt {\tiny [GM$\langle$B]}  
	 ABLEHNUNG: {'C62', 'A83', 'C98', 'B38', 'C00', 'A03', 'C32', 'A07', 'C10'}\\ \tt \\ \tt ARGUMENT: {'Du bestehst auf C62, obwohl es den schlechtesten Wert hat, und lässt C76 und C81 weg, die viel effizienter sind. Außerdem nutzen wir immer noch nicht den vollen Aufwand - wir können noch etwa 900 Punkte mehr verwenden.'}\\ \tt \\ \tt VORSCHLAG: {'C03', 'C81', 'C76', 'A07', 'C10', 'A03', 'C98', 'C32', 'A83', 'B38'} 
	  } 
	   } 
	   } 
	  \\ 
 

    \theutterance \stepcounter{utterance}  

    & & \multicolumn{4}{p{0.6\linewidth}}{\cellcolor[rgb]{0.9,0.9,0.9}{%
	\makecell[{{p{\linewidth}}}]{% 
	  \tt {\tiny [A$\langle$GM]}  
	 ABLEHNUNG: {'C62', 'A83', 'C98', 'B38', 'C00', 'A03', 'C32', 'A07', 'C10'}\\ \tt \\ \tt ARGUMENT: {'Du bestehst auf C62, obwohl es den schlechtesten Wert hat, und lässt C76 und C81 weg, die viel effizienter sind. Außerdem nutzen wir immer noch nicht den vollen Aufwand - wir können noch etwa 900 Punkte mehr verwenden.'}\\ \tt \\ \tt VORSCHLAG: {'C03', 'C81', 'C76', 'A07', 'C10', 'A03', 'C98', 'C32', 'A83', 'B38'} 
	  } 
	   } 
	   } 
	 & \\ 
 

    \theutterance \stepcounter{utterance}  

    & \multicolumn{4}{p{0.6\linewidth}}{\cellcolor[rgb]{0.8,1,0.9}{%
	\makecell[{{p{\linewidth}}}]{% 
	  \tt {\tiny [A$\rangle$GM]}  
	 ABLEHNUNG: {'C03', 'C81', 'C76', 'A07', 'C10', 'A03', 'C98', 'C32', 'A83', 'B38'}\\ \tt \\ \tt ARGUMENT: {'C03 verbraucht 543 Aufwand für minimalen Nutzen - das ist extrem ineffizient. Wir sollten diesen Aufwand für wertvollere Kombinationen nutzen. Lass uns einen Kompromiss finden, der beide zufriedenstellt.'}\\ \tt \\ \tt VORSCHLAG: {'C00', 'A83', 'C98', 'B38', 'C81', 'A03', 'C32', 'A07', 'C76', 'C10'} 
	  } 
	   } 
	   } 
	 & & \\ 
 

    \theutterance \stepcounter{utterance}  

    & & \multicolumn{4}{p{0.6\linewidth}}{\cellcolor[rgb]{0.9,0.9,0.9}{%
	\makecell[{{p{\linewidth}}}]{% 
	  \tt {\tiny [GM$\rangle$B]}  
	 ABLEHNUNG: {'C03', 'C81', 'C76', 'A07', 'C10', 'A03', 'C98', 'C32', 'A83', 'B38'}\\ \tt \\ \tt ARGUMENT: {'C03 verbraucht 543 Aufwand für minimalen Nutzen - das ist extrem ineffizient. Wir sollten diesen Aufwand für wertvollere Kombinationen nutzen. Lass uns einen Kompromiss finden, der beide zufriedenstellt.'}\\ \tt \\ \tt VORSCHLAG: {'C00', 'A83', 'C98', 'B38', 'C81', 'A03', 'C32', 'A07', 'C76', 'C10'} 
	  } 
	   } 
	   } 
	 & \\ 
 

    \theutterance \stepcounter{utterance}  

    & & & \multicolumn{4}{p{0.6\linewidth}}{\cellcolor[rgb]{1,0.85,0.72}{%
	\makecell[{{p{\linewidth}}}]{% 
	  \tt {\tiny [GM$\langle$B]}  
	 ARGUMENT: {'Dein Vorschlag ist interessanter, aber wir lassen immer noch wertvollen Aufwand ungenutzt. Mit dem verbleibenden Budget können wir noch C56 hinzufügen, was uns deutlich mehr Gesamtwert bringen würde als die aktuelle Kombination.'}\\ \tt \\ \tt VORSCHLAG: {'C56', 'C81', 'C76', 'A07', 'C10', 'C00', 'A03', 'C98', 'C32', 'A83'} 
	  } 
	   } 
	   } 
	  \\ 
 

    \theutterance \stepcounter{utterance}  

    & & \multicolumn{4}{p{0.6\linewidth}}{\cellcolor[rgb]{0.9,0.9,0.9}{%
	\makecell[{{p{\linewidth}}}]{% 
	  \tt {\tiny [A$\langle$GM]}  
	 ARGUMENT: {'Dein Vorschlag ist interessanter, aber wir lassen immer noch wertvollen Aufwand ungenutzt. Mit dem verbleibenden Budget können wir noch C56 hinzufügen, was uns deutlich mehr Gesamtwert bringen würde als die aktuelle Kombination.'}\\ \tt \\ \tt VORSCHLAG: {'C56', 'C81', 'C76', 'A07', 'C10', 'C00', 'A03', 'C98', 'C32', 'A83'} 
	  } 
	   } 
	   } 
	 & \\ 
 

    \theutterance \stepcounter{utterance}  

    & \multicolumn{4}{p{0.6\linewidth}}{\cellcolor[rgb]{0.8,1,0.9}{%
	\makecell[{{p{\linewidth}}}]{% 
	  \tt {\tiny [A$\rangle$GM]}  
	 ABLEHNUNG: {'C56', 'C81', 'C76', 'A07', 'C10', 'C00', 'A03', 'C98', 'C32', 'A83'}\\ \tt \\ \tt ARGUMENT: {'Du lässt B38 weg, der einen sehr hohen Wert haben könnte, und bestehst auf C56 mit seinem schlechten Aufwand-Nutzen-Verhältnis. B38 ist strategisch wichtiger für eine ausgewogene Lösung als C56.'}\\ \tt \\ \tt VORSCHLAG: {'C00', 'A83', 'C98', 'B38', 'C81', 'A03', 'C32', 'A07', 'C76', 'C62'} 
	  } 
	   } 
	   } 
	 & & \\ 
 

    \theutterance \stepcounter{utterance}  

    & & \multicolumn{4}{p{0.6\linewidth}}{\cellcolor[rgb]{0.9,0.9,0.9}{%
	\makecell[{{p{\linewidth}}}]{% 
	  \tt {\tiny [GM$\rangle$B]}  
	 ABLEHNUNG: {'C56', 'C81', 'C76', 'A07', 'C10', 'C00', 'A03', 'C98', 'C32', 'A83'}\\ \tt \\ \tt ARGUMENT: {'Du lässt B38 weg, der einen sehr hohen Wert haben könnte, und bestehst auf C56 mit seinem schlechten Aufwand-Nutzen-Verhältnis. B38 ist strategisch wichtiger für eine ausgewogene Lösung als C56.'}\\ \tt \\ \tt VORSCHLAG: {'C00', 'A83', 'C98', 'B38', 'C81', 'A03', 'C32', 'A07', 'C76', 'C62'} 
	  } 
	   } 
	   } 
	 & \\ 
 

    \theutterance \stepcounter{utterance}  

    & & & \multicolumn{4}{p{0.6\linewidth}}{\cellcolor[rgb]{1,0.85,0.72}{%
	\makecell[{{p{\linewidth}}}]{% 
	  \tt {\tiny [GM$\langle$B]}  
	 ABLEHNUNG: {'C00', 'A83', 'C98', 'B38', 'C81', 'A03', 'C32', 'A07', 'C76', 'C62'}\\ \tt \\ \tt ARGUMENT: {'Du fügst wieder C62 hinzu, obwohl wir bereits etabliert haben, dass es sehr ineffizient ist. Außerdem lässt du C10 weg, der bei gleichem Aufwand wie C00 zusätzlichen Wert bringen könnte. Wir verschwenden Potenzial.'}\\ \tt \\ \tt VORSCHLAG: {'C00', 'A83', 'C98', 'B38', 'C81', 'A03', 'C32', 'A07', 'C76', 'C10'} 
	  } 
	   } 
	   } 
	  \\ 
 

    \theutterance \stepcounter{utterance}  

    & & \multicolumn{4}{p{0.6\linewidth}}{\cellcolor[rgb]{0.9,0.9,0.9}{%
	\makecell[{{p{\linewidth}}}]{% 
	  \tt {\tiny [A$\langle$GM]}  
	 ABLEHNUNG: {'C00', 'A83', 'C98', 'B38', 'C81', 'A03', 'C32', 'A07', 'C76', 'C62'}\\ \tt \\ \tt ARGUMENT: {'Du fügst wieder C62 hinzu, obwohl wir bereits etabliert haben, dass es sehr ineffizient ist. Außerdem lässt du C10 weg, der bei gleichem Aufwand wie C00 zusätzlichen Wert bringen könnte. Wir verschwenden Potenzial.'}\\ \tt \\ \tt VORSCHLAG: {'C00', 'A83', 'C98', 'B38', 'C81', 'A03', 'C32', 'A07', 'C76', 'C10'} 
	  } 
	   } 
	   } 
	 & \\ 
 

    \theutterance \stepcounter{utterance}  

    & \multicolumn{4}{p{0.6\linewidth}}{\cellcolor[rgb]{0.8,1,0.9}{%
	\makecell[{{p{\linewidth}}}]{% 
	  \tt {\tiny [A$\rangle$GM]}  
	 ZUSTIMMUNG: {'C00', 'A83', 'C98', 'B38', 'C81', 'A03', 'C32', 'A07', 'C76', 'C10'}\\ \tt \\ \tt ARGUMENT: {'Dein Vorschlag ist vernünftig - er schließt sowohl C00 als auch C10 ein, nutzt den verfügbaren Aufwand gut aus und bietet eine ausgewogene Mischung ohne die ineffizientesten Gegenstände wie C62 oder C03.'} 
	  } 
	   } 
	   } 
	 & & \\ 
 

    \theutterance \stepcounter{utterance}  

    & & & \multicolumn{2}{p{0.3\linewidth}}{\cellcolor[rgb]{0.95,0.95,0.95}{%
	\makecell[{{p{\linewidth}}}]{% 
	  \tt {\tiny [GM$|$GM]}  
	 game successful 
	  } 
	   } 
	   } 
	 & & \\ 
 

    \theutterance \stepcounter{utterance}  

    & & & \multicolumn{2}{p{0.3\linewidth}}{\cellcolor[rgb]{0.95,0.95,0.95}{%
	\makecell[{{p{\linewidth}}}]{% 
	  \tt {\tiny [GM$|$GM]}  
	 end game 
	  } 
	   } 
	   } 
	 & & \\ 
 

\end{supertabular}
}

\end{document}
