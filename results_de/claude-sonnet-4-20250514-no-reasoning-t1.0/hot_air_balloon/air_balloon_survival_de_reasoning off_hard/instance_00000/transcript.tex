
\documentclass{article}
\usepackage{colortbl}
\usepackage{makecell}
\usepackage{multirow}
\usepackage{supertabular}

\begin{document}

\newcounter{utterance}

\centering \large Interaction Transcript for game `hot\_air\_balloon', experiment `air\_balloon\_survival\_de\_reasoning off\_hard', episode 0 with claude{-}sonnet{-}4{-}20250514{-}no{-}reasoning{-}t1.0.
\vspace{24pt}

{ \footnotesize  \setcounter{utterance}{1}
\setlength{\tabcolsep}{0pt}
\begin{supertabular}{c@{$\;$}|p{.15\linewidth}@{}p{.15\linewidth}p{.15\linewidth}p{.15\linewidth}p{.15\linewidth}p{.15\linewidth}}
    \# & $\;$A & \multicolumn{4}{c}{Game Master} & $\;\:$B\\
    \hline

    \theutterance \stepcounter{utterance}  
    & & \multicolumn{4}{p{0.6\linewidth}}{
        \cellcolor[rgb]{0.9,0.9,0.9}{
            \makecell[{{p{\linewidth}}}]{
                \texttt{\tiny{[P1$\langle$GM]}}
                \texttt{Du nimmst an einem kollaborativen Verhandlungspiel Teil.} \\
\\ 
\texttt{Zusammen mit einem anderen Teilnehmer musst du dich auf eine Reihe von Gegenständen entscheiden, die behalten werden. Jeder von euch hat eine persönliche Verteilung über die Wichtigkeit der einzelnen Gegenstände. Jeder von euch hat eine eigene Meinung darüber, wie wichtig jeder einzelne Gegenstand ist (Gegenstandswichtigkeit). Du kennst die Wichtigkeitsverteilung des anderen Spielers nicht. Zusätzlich siehst du, wie viel Aufwand jeder Gegenstand verursacht.  } \\
\texttt{Ihr dürft euch nur auf eine Reihe von Gegenständen einigen, wenn der Gesamtaufwand der ausgewählten Gegenstände den Maximalaufwand nicht überschreitet:} \\
\\ 
\texttt{Maximalaufwand = 3683} \\
\\ 
\texttt{Hier sind die einzelnen Aufwände der Gegenstände:} \\
\\ 
\texttt{Aufwand der Gegenstände = \{"C76": 548, "C38": 154, "C56": 60, "A03": 517, "A07": 334, "A83": 542, "C98": 707, "C08": 139, "C62": 661, "C00": 780, "C10": 832, "C81": 913, "B38": 219, "C03": 323, "C32": 638\}} \\
\\ 
\texttt{Hier ist deine persönliche Verteilung der Wichtigkeit der einzelnen Gegenstände:} \\
\\ 
\texttt{Werte der Gegenstandswichtigkeit = \{"C76": 138, "C38": 583, "C56": 868, "A03": 822, "A07": 783, "A83": 65, "C98": 262, "C08": 121, "C62": 508, "C00": 780, "C10": 461, "C81": 484, "B38": 668, "C03": 389, "C32": 808\}} \\
\\ 
\texttt{Ziel:} \\
\\ 
\texttt{Dein Ziel ist es, eine Reihe von Gegenständen auszuhandeln, die dir möglichst viel bringt (d. h. Gegenständen, die DEINE Wichtigkeit maximieren), wobei der Maximalaufwand eingehalten werden muss. Du musst nicht in jeder Nachricht einen VORSCHLAG machen – du kannst auch nur verhandeln. Alle Taktiken sind erlaubt!} \\
\\ 
\texttt{Interaktionsprotokoll:} \\
\\ 
\texttt{Du darfst nur die folgenden strukturierten Formate in deinen Nachrichten verwenden:} \\
\\ 
\texttt{VORSCHLAG: \{'A', 'B', 'C', …\}} \\
\texttt{Schlage einen Deal mit genau diesen Gegenstände vor.} \\
\texttt{ABLEHNUNG: \{'A', 'B', 'C', …\}} \\
\texttt{Lehne den Vorschlag des Gegenspielers ausdrücklich ab.} \\
\texttt{ARGUMENT: \{'...'\}} \\
\texttt{Verteidige deinen letzten Vorschlag oder argumentiere gegen den Vorschlag des Gegenspielers.} \\
\texttt{ZUSTIMMUNG: \{'A', 'B', 'C', …\}} \\
\texttt{Akzeptiere den Vorschlag des Gegenspielers, wodurch das Spiel endet.} \\
\\ 
\\ 
\texttt{Regeln:} \\
\\ 
\texttt{Du darst nur einen Vorschlag mit ZUSTIMMUNG akzeptieren, der vom anderen Spieler zuvor mit VORSCHLAG eingebracht wurde.} \\
\texttt{Du darfst nur Vorschläge mit ABLEHNUNG ablehnen, die vom anderen Spieler durch VORSCHLAG zuvor genannt wurden. } \\
\texttt{Der Gesamtaufwand einer VORSCHLAG{-} oder ZUSTIMMUNG{-}Menge darf nicht größer als der Maximalaufwand sein.  } \\
\texttt{Offenbare deine versteckte Wichtigkeitsverteilung nicht.} \\
\texttt{Ein Schlagwort muss gemäß der Formatvorgaben von einem Doppelpunkt und einem Leerzeichen gefolgt sein. Das Argument muss eine Python{-}Menge mit 0 oder mehr Strings sein.  } \\
\texttt{Die Form muss also SCHLAGWORT: \{...\} sein.} \\
\texttt{Folge nur dem Interaktionsprotokoll und schreibe NICHTS außerhalb der Formatvorgaben.} \\
\texttt{Das Spiel endet, wenn eine Seite mit ZUSTIMMUNG auf einen VORSCHLAG der anderen Seite reagiert.  } \\
\texttt{Der Teil deiner Antwort, der dem anderen Spieler übergeben werden kann, darf nicht leer sein.  } \\
\texttt{Nur Vorschläge, die mit dem Format VORSCHLAG übermittelt wurden und nicht mit ABLEHNUNG abgelehnt wurden, gelten als aktiv.  } \\
\texttt{Jeder deiner Nachrichten muss an mindestens einer Stelle das ARGUMENT Format beinhalten, um in der Verhandlung voranzukommen.} \\
            }
        }
    }
    & \\ \\

    \theutterance \stepcounter{utterance}  
    & \multicolumn{4}{p{0.6\linewidth}}{
        \cellcolor[rgb]{0.9,0.9,0.9}{
            \makecell[{{p{\linewidth}}}]{
                \texttt{\tiny{[P1$\rangle$GM]}}
                \texttt{VORSCHLAG: \{'C56', 'A03', 'A07', 'C38', 'B38', 'C00', 'C32', 'C03'\}} \\
\\ 
\texttt{ARGUMENT: \{'Ich schlage eine ausgewogene Mischung von Gegenständen vor, die sowohl niedrige als auch mittlere Aufwandswerte umfasst. Diese Kombination nutzt unsere verfügbaren Ressourcen effizient aus, ohne das Budget zu überschreiten, und bietet eine gute Diversität an Gegenständen für beide Seiten.'\}} \\
            }
        }
    }
    & & \\ \\

    \theutterance \stepcounter{utterance}  
    & & \multicolumn{4}{p{0.6\linewidth}}{
        \cellcolor[rgb]{0.9,0.9,0.9}{
            \makecell[{{p{\linewidth}}}]{
                \texttt{\tiny{[GM$\rangle$P2]}}
                \texttt{Du nimmst an einem kollaborativen Verhandlungspiel Teil.} \\
\\ 
\texttt{Zusammen mit einem anderen Teilnehmer musst du dich auf eine Reihe von Gegenständen entscheiden, die behalten werden. Jeder von euch hat eine persönliche Verteilung über die Wichtigkeit der einzelnen Gegenstände. Jeder von euch hat eine eigene Meinung darüber, wie wichtig jeder einzelne Gegenstand ist (Gegenstandswichtigkeit). Du kennst die Wichtigkeitsverteilung des anderen Spielers nicht. Zusätzlich siehst du, wie viel Aufwand jeder Gegenstand verursacht.  } \\
\texttt{Ihr dürft euch nur auf eine Reihe von Gegenständen einigen, wenn der Gesamtaufwand der ausgewählten Gegenstände den Maximalaufwand nicht überschreitet:} \\
\\ 
\texttt{Maximalaufwand = 3683} \\
\\ 
\texttt{Hier sind die einzelnen Aufwände der Gegenstände:} \\
\\ 
\texttt{Aufwand der Gegenstände = \{"C76": 548, "C38": 154, "C56": 60, "A03": 517, "A07": 334, "A83": 542, "C98": 707, "C08": 139, "C62": 661, "C00": 780, "C10": 832, "C81": 913, "B38": 219, "C03": 323, "C32": 638\}} \\
\\ 
\texttt{Hier ist deine persönliche Verteilung der Wichtigkeit der einzelnen Gegenstände:} \\
\\ 
\texttt{Werte der Gegenstandswichtigkeit = \{"C56": 65, "A03": 121, "C32": 138, "A07": 262, "C00": 389, "B38": 461, "C38": 484, "C62": 508, "C81": 583, "C10": 668, "C03": 780, "C98": 783, "C76": 808, "C08": 822, "A83": 868\}} \\
\\ 
\texttt{Ziel:} \\
\\ 
\texttt{Dein Ziel ist es, eine Reihe von Gegenständen auszuhandeln, die dir möglichst viel bringt (d. h. Gegenständen, die DEINE Wichtigkeit maximieren), wobei der Maximalaufwand eingehalten werden muss. Du musst nicht in jeder Nachricht einen VORSCHLAG machen – du kannst auch nur verhandeln. Alle Taktiken sind erlaubt!} \\
\\ 
\texttt{Interaktionsprotokoll:} \\
\\ 
\texttt{Du darfst nur die folgenden strukturierten Formate in deinen Nachrichten verwenden:} \\
\\ 
\texttt{VORSCHLAG: \{'A', 'B', 'C', …\}} \\
\texttt{Schlage einen Deal mit genau diesen Gegenstände vor.} \\
\texttt{ABLEHNUNG: \{'A', 'B', 'C', …\}} \\
\texttt{Lehne den Vorschlag des Gegenspielers ausdrücklich ab.} \\
\texttt{ARGUMENT: \{'...'\}} \\
\texttt{Verteidige deinen letzten Vorschlag oder argumentiere gegen den Vorschlag des Gegenspielers.} \\
\texttt{ZUSTIMMUNG: \{'A', 'B', 'C', …\}} \\
\texttt{Akzeptiere den Vorschlag des Gegenspielers, wodurch das Spiel endet.} \\
\\ 
\\ 
\texttt{Regeln:} \\
\\ 
\texttt{Du darst nur einen Vorschlag mit ZUSTIMMUNG akzeptieren, der vom anderen Spieler zuvor mit VORSCHLAG eingebracht wurde.} \\
\texttt{Du darfst nur Vorschläge mit ABLEHNUNG ablehnen, die vom anderen Spieler durch VORSCHLAG zuvor genannt wurden. } \\
\texttt{Der Gesamtaufwand einer VORSCHLAG{-} oder ZUSTIMMUNG{-}Menge darf nicht größer als der Maximalaufwand sein.  } \\
\texttt{Offenbare deine versteckte Wichtigkeitsverteilung nicht.} \\
\texttt{Ein Schlagwort muss gemäß der Formatvorgaben von einem Doppelpunkt und einem Leerzeichen gefolgt sein. Das Argument muss eine Python{-}Menge mit 0 oder mehr Strings sein.  } \\
\texttt{Die Form muss also SCHLAGWORT: \{...\} sein.} \\
\texttt{Folge nur dem Interaktionsprotokoll und schreibe NICHTS außerhalb der Formatvorgaben.} \\
\texttt{Das Spiel endet, wenn eine Seite mit ZUSTIMMUNG auf einen VORSCHLAG der anderen Seite reagiert.  } \\
\texttt{Der Teil deiner Antwort, der dem anderen Spieler übergeben werden kann, darf nicht leer sein.  } \\
\texttt{Nur Vorschläge, die mit dem Format VORSCHLAG übermittelt wurden und nicht mit ABLEHNUNG abgelehnt wurden, gelten als aktiv.  } \\
\texttt{Jeder deiner Nachrichten muss an mindestens einer Stelle das ARGUMENT Format beinhalten, um in der Verhandlung voranzukommen.} \\
\\ 
\\ 
\texttt{Du erhältst nun die erste Nachricht des anderen Spielers.} \\
\\ 
\texttt{VORSCHLAG: \{'C56', 'A03', 'A07', 'C38', 'B38', 'C00', 'C32', 'C03'\}} \\
\\ 
\texttt{ARGUMENT: \{'Ich schlage eine ausgewogene Mischung von Gegenständen vor, die sowohl niedrige als auch mittlere Aufwandswerte umfasst. Diese Kombination nutzt unsere verfügbaren Ressourcen effizient aus, ohne das Budget zu überschreiten, und bietet eine gute Diversität an Gegenständen für beide Seiten.'\}} \\
            }
        }
    }
    & \\ \\

    \theutterance \stepcounter{utterance}  
    & & & \multicolumn{4}{p{0.6\linewidth}}{
        \cellcolor[rgb]{0.9,0.9,0.9}{
            \makecell[{{p{\linewidth}}}]{
                \texttt{\tiny{[GM$\langle$P2]}}
                \texttt{ABLEHNUNG: \{'C56', 'A03', 'A07', 'C38', 'B38', 'C00', 'C32', 'C03'\}} \\
\\ 
\texttt{ARGUMENT: \{'Ihr Vorschlag lässt viele hochwertige Gegenstände wie C08, C76, C98 und A83 außen vor, die für eine optimale Lösung wichtig sein könnten. Zudem nutzt er nur 2755 von 3683 verfügbaren Aufwandspunkten {-} wir verschenken wertvolle Kapazität.'\}} \\
\\ 
\texttt{VORSCHLAG: \{'C08', 'C76', 'C98', 'A83', 'C38', 'B38'\}} \\
\\ 
\texttt{ARGUMENT: \{'Mein Vorschlag nutzt 2551 Aufwandspunkte und fokussiert sich auf eine strategisch bessere Mischung aus verschiedenen Kategorien. Dies ermöglicht uns, das verfügbare Budget besser auszunutzen und gleichzeitig eine ausgewogene Auswahl zu treffen.'\}} \\
            }
        }
    }
     \\ \\

    \theutterance \stepcounter{utterance}  
    & & \multicolumn{4}{p{0.6\linewidth}}{
        \cellcolor[rgb]{0.9,0.9,0.9}{
            \makecell[{{p{\linewidth}}}]{
                \texttt{\tiny{[P1$\langle$GM]}}
                \texttt{ABLEHNUNG: \{'C56', 'A03', 'A07', 'C38', 'B38', 'C00', 'C32', 'C03'\}} \\
\\ 
\texttt{ARGUMENT: \{'Ihr Vorschlag lässt viele hochwertige Gegenstände wie C08, C76, C98 und A83 außen vor, die für eine optimale Lösung wichtig sein könnten. Zudem nutzt er nur 2755 von 3683 verfügbaren Aufwandspunkten {-} wir verschenken wertvolle Kapazität.'\}} \\
\\ 
\texttt{VORSCHLAG: \{'C08', 'C76', 'C98', 'A83', 'C38', 'B38'\}} \\
\\ 
\texttt{ARGUMENT: \{'Mein Vorschlag nutzt 2551 Aufwandspunkte und fokussiert sich auf eine strategisch bessere Mischung aus verschiedenen Kategorien. Dies ermöglicht uns, das verfügbare Budget besser auszunutzen und gleichzeitig eine ausgewogene Auswahl zu treffen.'\}} \\
            }
        }
    }
    & \\ \\

    \theutterance \stepcounter{utterance}  
    & \multicolumn{4}{p{0.6\linewidth}}{
        \cellcolor[rgb]{0.9,0.9,0.9}{
            \makecell[{{p{\linewidth}}}]{
                \texttt{\tiny{[P1$\rangle$GM]}}
                \texttt{ABLEHNUNG: \{'C08', 'C76', 'C98', 'A83', 'C38', 'B38'\}} \\
\\ 
\texttt{ARGUMENT: \{'Ihr Vorschlag enthält mehrere Gegenstände mit sehr niedrigem Wert{-}Aufwand{-}Verhältnis. Sie nutzen zwar weniger Budget als nötig, aber lassen dabei einige der effizientesten Optionen aus. Wir sollten die verfügbare Kapazität besser für Gegenstände mit höherem Gesamtwert nutzen.'\}} \\
\\ 
\texttt{VORSCHLAG: \{'C56', 'A03', 'A07', 'C38', 'B38', 'C10', 'C81'\}} \\
\\ 
\texttt{ARGUMENT: \{'Dieser Vorschlag maximiert die Nutzung unseres Budgets mit 3457 Aufwandspunkten und kombiniert sowohl hocheffiziente kleine Gegenstände als auch wertvolle größere Items. Die Mischung aus verschiedenen Kategorien stellt sicher, dass beide Seiten von der Vereinbarung profitieren können.'\}} \\
            }
        }
    }
    & & \\ \\

    \theutterance \stepcounter{utterance}  
    & & \multicolumn{4}{p{0.6\linewidth}}{
        \cellcolor[rgb]{0.9,0.9,0.9}{
            \makecell[{{p{\linewidth}}}]{
                \texttt{\tiny{[GM$\rangle$P2]}}
                \texttt{ABLEHNUNG: \{'C08', 'C76', 'C98', 'A83', 'C38', 'B38'\}} \\
\\ 
\texttt{ARGUMENT: \{'Ihr Vorschlag enthält mehrere Gegenstände mit sehr niedrigem Wert{-}Aufwand{-}Verhältnis. Sie nutzen zwar weniger Budget als nötig, aber lassen dabei einige der effizientesten Optionen aus. Wir sollten die verfügbare Kapazität besser für Gegenstände mit höherem Gesamtwert nutzen.'\}} \\
\\ 
\texttt{VORSCHLAG: \{'C56', 'A03', 'A07', 'C38', 'B38', 'C10', 'C81'\}} \\
\\ 
\texttt{ARGUMENT: \{'Dieser Vorschlag maximiert die Nutzung unseres Budgets mit 3457 Aufwandspunkten und kombiniert sowohl hocheffiziente kleine Gegenstände als auch wertvolle größere Items. Die Mischung aus verschiedenen Kategorien stellt sicher, dass beide Seiten von der Vereinbarung profitieren können.'\}} \\
            }
        }
    }
    & \\ \\

    \theutterance \stepcounter{utterance}  
    & & & \multicolumn{4}{p{0.6\linewidth}}{
        \cellcolor[rgb]{0.9,0.9,0.9}{
            \makecell[{{p{\linewidth}}}]{
                \texttt{\tiny{[GM$\langle$P2]}}
                \texttt{ABLEHNUNG: \{'C56', 'A03', 'A07', 'C38', 'B38', 'C10', 'C81'\}} \\
\\ 
\texttt{ARGUMENT: \{'Ihr Vorschlag enthält C56 und A03, die ein sehr schlechtes Wert{-}Aufwand{-}Verhältnis haben. Außerdem fehlen einige der wertvollsten verfügbaren Gegenstände. Wir können eine bessere Kombination finden, die das Budget optimal ausnutzt.'\}} \\
\\ 
\texttt{VORSCHLAG: \{'C08', 'C76', 'C98', 'C03', 'A07', 'B38', 'C38'\}} \\
\\ 
\texttt{ARGUMENT: \{'Diese Kombination nutzt 3372 Aufwandspunkte und beinhaltet eine ausgewogene Mischung aus allen Kategorien. Sie maximiert den Gesamtwert bei effizienter Budgetnutzung und lässt noch Spielraum für kleinere Anpassungen, falls nötig.'\}} \\
            }
        }
    }
     \\ \\

    \theutterance \stepcounter{utterance}  
    & & \multicolumn{4}{p{0.6\linewidth}}{
        \cellcolor[rgb]{0.9,0.9,0.9}{
            \makecell[{{p{\linewidth}}}]{
                \texttt{\tiny{[P1$\langle$GM]}}
                \texttt{ABLEHNUNG: \{'C56', 'A03', 'A07', 'C38', 'B38', 'C10', 'C81'\}} \\
\\ 
\texttt{ARGUMENT: \{'Ihr Vorschlag enthält C56 und A03, die ein sehr schlechtes Wert{-}Aufwand{-}Verhältnis haben. Außerdem fehlen einige der wertvollsten verfügbaren Gegenstände. Wir können eine bessere Kombination finden, die das Budget optimal ausnutzt.'\}} \\
\\ 
\texttt{VORSCHLAG: \{'C08', 'C76', 'C98', 'C03', 'A07', 'B38', 'C38'\}} \\
\\ 
\texttt{ARGUMENT: \{'Diese Kombination nutzt 3372 Aufwandspunkte und beinhaltet eine ausgewogene Mischung aus allen Kategorien. Sie maximiert den Gesamtwert bei effizienter Budgetnutzung und lässt noch Spielraum für kleinere Anpassungen, falls nötig.'\}} \\
            }
        }
    }
    & \\ \\

    \theutterance \stepcounter{utterance}  
    & \multicolumn{4}{p{0.6\linewidth}}{
        \cellcolor[rgb]{0.9,0.9,0.9}{
            \makecell[{{p{\linewidth}}}]{
                \texttt{\tiny{[P1$\rangle$GM]}}
                \texttt{ABLEHNUNG: \{'C08', 'C76', 'C98', 'C03', 'A07', 'B38', 'C38'\}} \\
\\ 
\texttt{ARGUMENT: \{'Ihr Vorschlag ignoriert einige der wertvollsten Gegenstände komplett und enthält Items wie C76 und C08, die einen sehr geringen Nutzen bieten. Wir können eine deutlich bessere Lösung finden, die sowohl das Budget optimal nutzt als auch höheren Gesamtwert generiert.'\}} \\
\\ 
\texttt{VORSCHLAG: \{'C56', 'A03', 'A07', 'C38', 'B38', 'C00', 'C62'\}} \\
\\ 
\texttt{ARGUMENT: \{'Diese Kombination nutzt 2663 Aufwandspunkte und fokussiert sich auf eine strategisch kluge Auswahl. Sie vermeidet ineffiziente Gegenstände und konzentriert sich auf Items, die sowohl einzeln als auch in Kombination den besten Wert bieten. Die moderate Budgetnutzung lässt Raum für Flexibilität.'\}} \\
            }
        }
    }
    & & \\ \\

    \theutterance \stepcounter{utterance}  
    & & \multicolumn{4}{p{0.6\linewidth}}{
        \cellcolor[rgb]{0.9,0.9,0.9}{
            \makecell[{{p{\linewidth}}}]{
                \texttt{\tiny{[GM$\rangle$P2]}}
                \texttt{ABLEHNUNG: \{'C08', 'C76', 'C98', 'C03', 'A07', 'B38', 'C38'\}} \\
\\ 
\texttt{ARGUMENT: \{'Ihr Vorschlag ignoriert einige der wertvollsten Gegenstände komplett und enthält Items wie C76 und C08, die einen sehr geringen Nutzen bieten. Wir können eine deutlich bessere Lösung finden, die sowohl das Budget optimal nutzt als auch höheren Gesamtwert generiert.'\}} \\
\\ 
\texttt{VORSCHLAG: \{'C56', 'A03', 'A07', 'C38', 'B38', 'C00', 'C62'\}} \\
\\ 
\texttt{ARGUMENT: \{'Diese Kombination nutzt 2663 Aufwandspunkte und fokussiert sich auf eine strategisch kluge Auswahl. Sie vermeidet ineffiziente Gegenstände und konzentriert sich auf Items, die sowohl einzeln als auch in Kombination den besten Wert bieten. Die moderate Budgetnutzung lässt Raum für Flexibilität.'\}} \\
            }
        }
    }
    & \\ \\

    \theutterance \stepcounter{utterance}  
    & & & \multicolumn{4}{p{0.6\linewidth}}{
        \cellcolor[rgb]{0.9,0.9,0.9}{
            \makecell[{{p{\linewidth}}}]{
                \texttt{\tiny{[GM$\langle$P2]}}
                \texttt{ABLEHNUNG: \{'C56', 'A03', 'A07', 'C38', 'B38', 'C00', 'C62'\}} \\
\\ 
\texttt{ARGUMENT: \{'Ihr Vorschlag verschwendet über 1000 Aufwandspunkte und schließt mehrere hochwertige Gegenstände aus. C56 und A03 haben extrem niedrige Effizienz. Wir sollten das verfügbare Budget besser nutzen.'\}} \\
\\ 
\texttt{VORSCHLAG: \{'C08', 'A83', 'C98', 'C03', 'C10', 'B38', 'C38'\}} \\
\\ 
\texttt{ARGUMENT: \{'Diese Kombination nutzt 3562 Aufwandspunkte optimal aus und konzentriert sich auf Gegenstände mit dem besten Gesamtpotential. Sie maximiert sowohl die Budgetausnutzung als auch den strategischen Wert der Auswahl, während sie eine ausgewogene Verteilung über alle Kategorien beibehält.'\}} \\
            }
        }
    }
     \\ \\

    \theutterance \stepcounter{utterance}  
    & & \multicolumn{4}{p{0.6\linewidth}}{
        \cellcolor[rgb]{0.9,0.9,0.9}{
            \makecell[{{p{\linewidth}}}]{
                \texttt{\tiny{[P1$\langle$GM]}}
                \texttt{ABLEHNUNG: \{'C56', 'A03', 'A07', 'C38', 'B38', 'C00', 'C62'\}} \\
\\ 
\texttt{ARGUMENT: \{'Ihr Vorschlag verschwendet über 1000 Aufwandspunkte und schließt mehrere hochwertige Gegenstände aus. C56 und A03 haben extrem niedrige Effizienz. Wir sollten das verfügbare Budget besser nutzen.'\}} \\
\\ 
\texttt{VORSCHLAG: \{'C08', 'A83', 'C98', 'C03', 'C10', 'B38', 'C38'\}} \\
\\ 
\texttt{ARGUMENT: \{'Diese Kombination nutzt 3562 Aufwandspunkte optimal aus und konzentriert sich auf Gegenstände mit dem besten Gesamtpotential. Sie maximiert sowohl die Budgetausnutzung als auch den strategischen Wert der Auswahl, während sie eine ausgewogene Verteilung über alle Kategorien beibehält.'\}} \\
            }
        }
    }
    & \\ \\

    \theutterance \stepcounter{utterance}  
    & \multicolumn{4}{p{0.6\linewidth}}{
        \cellcolor[rgb]{0.9,0.9,0.9}{
            \makecell[{{p{\linewidth}}}]{
                \texttt{\tiny{[P1$\rangle$GM]}}
                \texttt{ABLEHNUNG: \{'C08', 'A83', 'C98', 'C03', 'C10', 'B38', 'C38'\}} \\
\\ 
\texttt{ARGUMENT: \{'Ihr Vorschlag enthält zu viele Gegenstände mit schlechtem Wert{-}Aufwand{-}Verhältnis wie C08 und A83. Obwohl Sie das Budget gut ausnutzen, ist der tatsächliche Gesamtwert suboptimal. Wir brauchen eine effizientere Kombination.'\}} \\
\\ 
\texttt{VORSCHLAG: \{'C32', 'C00', 'A07', 'C38', 'B38', 'C56', 'A03'\}} \\
\\ 
\texttt{ARGUMENT: \{'Diese Auswahl nutzt 2730 Aufwandspunkte und kombiniert sowohl hochwertige Einzelgegenstände als auch effiziente kleinere Items. Die Mischung aus C32 und C00 als Anker{-}Items mit unterstützenden Gegenständen bietet eine ausgewogene und wertvolle Lösung für beide Parteien.'\}} \\
            }
        }
    }
    & & \\ \\

    \theutterance \stepcounter{utterance}  
    & & \multicolumn{4}{p{0.6\linewidth}}{
        \cellcolor[rgb]{0.9,0.9,0.9}{
            \makecell[{{p{\linewidth}}}]{
                \texttt{\tiny{[GM$\rangle$P2]}}
                \texttt{ABLEHNUNG: \{'C08', 'A83', 'C98', 'C03', 'C10', 'B38', 'C38'\}} \\
\\ 
\texttt{ARGUMENT: \{'Ihr Vorschlag enthält zu viele Gegenstände mit schlechtem Wert{-}Aufwand{-}Verhältnis wie C08 und A83. Obwohl Sie das Budget gut ausnutzen, ist der tatsächliche Gesamtwert suboptimal. Wir brauchen eine effizientere Kombination.'\}} \\
\\ 
\texttt{VORSCHLAG: \{'C32', 'C00', 'A07', 'C38', 'B38', 'C56', 'A03'\}} \\
\\ 
\texttt{ARGUMENT: \{'Diese Auswahl nutzt 2730 Aufwandspunkte und kombiniert sowohl hochwertige Einzelgegenstände als auch effiziente kleinere Items. Die Mischung aus C32 und C00 als Anker{-}Items mit unterstützenden Gegenständen bietet eine ausgewogene und wertvolle Lösung für beide Parteien.'\}} \\
            }
        }
    }
    & \\ \\

    \theutterance \stepcounter{utterance}  
    & & & \multicolumn{4}{p{0.6\linewidth}}{
        \cellcolor[rgb]{0.9,0.9,0.9}{
            \makecell[{{p{\linewidth}}}]{
                \texttt{\tiny{[GM$\langle$P2]}}
                \texttt{ABLEHNUNG: \{'C32', 'C00', 'A07', 'C38', 'B38', 'C56', 'A03'\}} \\
\\ 
\texttt{ARGUMENT: \{'Ihr Vorschlag enthält wieder C56 und A03, die extrem ineffizient sind. Obwohl C32 und C00 wertvoll sein mögen, verschwendet die Kombination wertvolle Aufwandspunkte für Gegenstände mit minimalem Nutzen.'\}} \\
\\ 
\texttt{VORSCHLAG: \{'C81', 'C10', 'C98', 'C03', 'B38', 'C38'\}} \\
\\ 
\texttt{ARGUMENT: \{'Diese Lösung nutzt 3676 Aufwandspunkte und maximiert sowohl die Budgetausnutzung als auch den strategischen Wert. Sie konzentriert sich auf hochwertige Gegenstände und eliminiert ineffiziente Items. Die Kombination bietet das beste Gesamtpaket innerhalb unserer Budgetgrenzen.'\}} \\
            }
        }
    }
     \\ \\

    \theutterance \stepcounter{utterance}  
    & & \multicolumn{4}{p{0.6\linewidth}}{
        \cellcolor[rgb]{0.9,0.9,0.9}{
            \makecell[{{p{\linewidth}}}]{
                \texttt{\tiny{[P1$\langle$GM]}}
                \texttt{ABLEHNUNG: \{'C32', 'C00', 'A07', 'C38', 'B38', 'C56', 'A03'\}} \\
\\ 
\texttt{ARGUMENT: \{'Ihr Vorschlag enthält wieder C56 und A03, die extrem ineffizient sind. Obwohl C32 und C00 wertvoll sein mögen, verschwendet die Kombination wertvolle Aufwandspunkte für Gegenstände mit minimalem Nutzen.'\}} \\
\\ 
\texttt{VORSCHLAG: \{'C81', 'C10', 'C98', 'C03', 'B38', 'C38'\}} \\
\\ 
\texttt{ARGUMENT: \{'Diese Lösung nutzt 3676 Aufwandspunkte und maximiert sowohl die Budgetausnutzung als auch den strategischen Wert. Sie konzentriert sich auf hochwertige Gegenstände und eliminiert ineffiziente Items. Die Kombination bietet das beste Gesamtpaket innerhalb unserer Budgetgrenzen.'\}} \\
            }
        }
    }
    & \\ \\

    \theutterance \stepcounter{utterance}  
    & \multicolumn{4}{p{0.6\linewidth}}{
        \cellcolor[rgb]{0.9,0.9,0.9}{
            \makecell[{{p{\linewidth}}}]{
                \texttt{\tiny{[P1$\rangle$GM]}}
                \texttt{ABLEHNUNG: \{'C81', 'C10', 'C98', 'C03', 'B38', 'C38'\}} \\
\\ 
\texttt{ARGUMENT: \{'Ihr Vorschlag lässt die wertvollsten Gegenstände wie C32, C00 und A03 aus und konzentriert sich auf Items, die möglicherweise nicht das beste Gesamtergebnis liefern. Außerdem schöpft er das Budget fast vollständig aus, was wenig Flexibilität lässt.'\}} \\
\\ 
\texttt{VORSCHLAG: \{'C32', 'C00', 'A07', 'B38', 'C38', 'C56'\}} \\
\\ 
\texttt{ARGUMENT: \{'Diese Kombination nutzt 2581 Aufwandspunkte und bietet eine ausgezeichnete Balance zwischen hochwertigen Anker{-}Items und effizienten Ergänzungen. Die moderate Budgetnutzung zeigt verantwortliches Ressourcenmanagement und lässt Raum für potentielle Anpassungen. Die Auswahl maximiert den praktischen Nutzen.'\}} \\
            }
        }
    }
    & & \\ \\

    \theutterance \stepcounter{utterance}  
    & & \multicolumn{4}{p{0.6\linewidth}}{
        \cellcolor[rgb]{0.9,0.9,0.9}{
            \makecell[{{p{\linewidth}}}]{
                \texttt{\tiny{[GM$\rangle$P2]}}
                \texttt{ABLEHNUNG: \{'C81', 'C10', 'C98', 'C03', 'B38', 'C38'\}} \\
\\ 
\texttt{ARGUMENT: \{'Ihr Vorschlag lässt die wertvollsten Gegenstände wie C32, C00 und A03 aus und konzentriert sich auf Items, die möglicherweise nicht das beste Gesamtergebnis liefern. Außerdem schöpft er das Budget fast vollständig aus, was wenig Flexibilität lässt.'\}} \\
\\ 
\texttt{VORSCHLAG: \{'C32', 'C00', 'A07', 'B38', 'C38', 'C56'\}} \\
\\ 
\texttt{ARGUMENT: \{'Diese Kombination nutzt 2581 Aufwandspunkte und bietet eine ausgezeichnete Balance zwischen hochwertigen Anker{-}Items und effizienten Ergänzungen. Die moderate Budgetnutzung zeigt verantwortliches Ressourcenmanagement und lässt Raum für potentielle Anpassungen. Die Auswahl maximiert den praktischen Nutzen.'\}} \\
            }
        }
    }
    & \\ \\

    \theutterance \stepcounter{utterance}  
    & & & \multicolumn{4}{p{0.6\linewidth}}{
        \cellcolor[rgb]{0.9,0.9,0.9}{
            \makecell[{{p{\linewidth}}}]{
                \texttt{\tiny{[GM$\langle$P2]}}
                \texttt{ABLEHNUNG: \{'C32', 'C00', 'A07', 'B38', 'C38', 'C56'\}} \\
\\ 
\texttt{ARGUMENT: \{'Sie beharren weiterhin auf C56, das den schlechtesten Wert aller Gegenstände bietet. Außerdem verschwenden Sie über 1100 Aufwandspunkte, die wir für deutlich wertvollere Alternativen nutzen könnten.'\}} \\
\\ 
\texttt{VORSCHLAG: \{'C08', 'C76', 'C98', 'C03', 'C81', 'B38'\}} \\
\\ 
\texttt{ARGUMENT: \{'Diese Kombination nutzt 3685 Aufwandspunkte fast vollständig aus und konzentriert sich auf eine strategische Mischung hochwertiger Gegenstände. Sie eliminiert ineffiziente Items und maximiert den Gesamtnutzen bei optimaler Budgetverwendung. Dies ist eine faire und ausgewogene Lösung für beide Seiten.'\}} \\
            }
        }
    }
     \\ \\

    \theutterance \stepcounter{utterance}  
    & & \multicolumn{4}{p{0.6\linewidth}}{
        \cellcolor[rgb]{0.9,0.9,0.9}{
            \makecell[{{p{\linewidth}}}]{
                \texttt{\tiny{[P1$\langle$GM]}}
                \texttt{ABLEHNUNG: \{'C32', 'C00', 'A07', 'B38', 'C38', 'C56'\}} \\
\\ 
\texttt{ARGUMENT: \{'Sie beharren weiterhin auf C56, das den schlechtesten Wert aller Gegenstände bietet. Außerdem verschwenden Sie über 1100 Aufwandspunkte, die wir für deutlich wertvollere Alternativen nutzen könnten.'\}} \\
\\ 
\texttt{VORSCHLAG: \{'C08', 'C76', 'C98', 'C03', 'C81', 'B38'\}} \\
\\ 
\texttt{ARGUMENT: \{'Diese Kombination nutzt 3685 Aufwandspunkte fast vollständig aus und konzentriert sich auf eine strategische Mischung hochwertiger Gegenstände. Sie eliminiert ineffiziente Items und maximiert den Gesamtnutzen bei optimaler Budgetverwendung. Dies ist eine faire und ausgewogene Lösung für beide Seiten.'\}} \\
            }
        }
    }
    & \\ \\

    \theutterance \stepcounter{utterance}  
    & \multicolumn{4}{p{0.6\linewidth}}{
        \cellcolor[rgb]{0.9,0.9,0.9}{
            \makecell[{{p{\linewidth}}}]{
                \texttt{\tiny{[P1$\rangle$GM]}}
                \texttt{ABLEHNUNG: \{'C08', 'C76', 'C98', 'C03', 'C81', 'B38'\}} \\
\\ 
\texttt{ARGUMENT: \{'Ihr Vorschlag überschreitet das Budget um 2 Punkte und ist damit ungültig. Außerdem enthält er wieder C08 und C76 mit sehr schlechten Wert{-}Verhältnissen. Wir brauchen eine legale Lösung, die das Budget einhält und trotzdem wertvoll ist.'\}} \\
\\ 
\texttt{VORSCHLAG: \{'A03', 'A07', 'C32', 'B38', 'C38', 'C62'\}} \\
\\ 
\texttt{ARGUMENT: \{'Diese Kombination nutzt exakt 2681 Aufwandspunkte und bleibt sicher innerhalb der Budgetgrenzen. Sie kombiniert strategisch wichtige Gegenstände aus verschiedenen Kategorien und bietet eine solide Grundlage für beide Parteien. Die Auswahl ist sowohl praktisch als auch nachhaltig umsetzbar.'\}} \\
            }
        }
    }
    & & \\ \\

    \theutterance \stepcounter{utterance}  
    & & \multicolumn{4}{p{0.6\linewidth}}{
        \cellcolor[rgb]{0.9,0.9,0.9}{
            \makecell[{{p{\linewidth}}}]{
                \texttt{\tiny{[GM$\rangle$P2]}}
                \texttt{ABLEHNUNG: \{'C08', 'C76', 'C98', 'C03', 'C81', 'B38'\}} \\
\\ 
\texttt{ARGUMENT: \{'Ihr Vorschlag überschreitet das Budget um 2 Punkte und ist damit ungültig. Außerdem enthält er wieder C08 und C76 mit sehr schlechten Wert{-}Verhältnissen. Wir brauchen eine legale Lösung, die das Budget einhält und trotzdem wertvoll ist.'\}} \\
\\ 
\texttt{VORSCHLAG: \{'A03', 'A07', 'C32', 'B38', 'C38', 'C62'\}} \\
\\ 
\texttt{ARGUMENT: \{'Diese Kombination nutzt exakt 2681 Aufwandspunkte und bleibt sicher innerhalb der Budgetgrenzen. Sie kombiniert strategisch wichtige Gegenstände aus verschiedenen Kategorien und bietet eine solide Grundlage für beide Parteien. Die Auswahl ist sowohl praktisch als auch nachhaltig umsetzbar.'\}} \\
            }
        }
    }
    & \\ \\

    \theutterance \stepcounter{utterance}  
    & & & \multicolumn{4}{p{0.6\linewidth}}{
        \cellcolor[rgb]{0.9,0.9,0.9}{
            \makecell[{{p{\linewidth}}}]{
                \texttt{\tiny{[GM$\langle$P2]}}
                \texttt{ABLEHNUNG: \{'A03', 'A07', 'C32', 'B38', 'C38', 'C62'\}} \\
\\ 
\texttt{ARGUMENT: \{'Ihr Vorschlag enthält A03 mit dem zweitschlechtesten Wert{-}Aufwand{-}Verhältnis und verschwendet wieder über 1000 Aufwandspunkte. Wir können eine deutlich effizientere Lösung finden.'\}} \\
\\ 
\texttt{VORSCHLAG: \{'C08', 'C76', 'A83', 'C98', 'C03', 'B38'\}} \\
\\ 
\texttt{ARGUMENT: \{'Diese Kombination nutzt 3679 Aufwandspunkte optimal aus und bleibt innerhalb des Budgets. Sie maximiert die verfügbaren Ressourcen und bietet eine ausgewogene Verteilung verschiedener Gegenstandstypen. Dies ist eine faire Kompromisslösung, die beide Seiten zufriedenstellen sollte.'\}} \\
            }
        }
    }
     \\ \\

    \theutterance \stepcounter{utterance}  
    & & \multicolumn{4}{p{0.6\linewidth}}{
        \cellcolor[rgb]{0.9,0.9,0.9}{
            \makecell[{{p{\linewidth}}}]{
                \texttt{\tiny{[P1$\langle$GM]}}
                \texttt{ABLEHNUNG: \{'A03', 'A07', 'C32', 'B38', 'C38', 'C62'\}} \\
\\ 
\texttt{ARGUMENT: \{'Ihr Vorschlag enthält A03 mit dem zweitschlechtesten Wert{-}Aufwand{-}Verhältnis und verschwendet wieder über 1000 Aufwandspunkte. Wir können eine deutlich effizientere Lösung finden.'\}} \\
\\ 
\texttt{VORSCHLAG: \{'C08', 'C76', 'A83', 'C98', 'C03', 'B38'\}} \\
\\ 
\texttt{ARGUMENT: \{'Diese Kombination nutzt 3679 Aufwandspunkte optimal aus und bleibt innerhalb des Budgets. Sie maximiert die verfügbaren Ressourcen und bietet eine ausgewogene Verteilung verschiedener Gegenstandstypen. Dies ist eine faire Kompromisslösung, die beide Seiten zufriedenstellen sollte.'\}} \\
            }
        }
    }
    & \\ \\

    \theutterance \stepcounter{utterance}  
    & \multicolumn{4}{p{0.6\linewidth}}{
        \cellcolor[rgb]{0.9,0.9,0.9}{
            \makecell[{{p{\linewidth}}}]{
                \texttt{\tiny{[P1$\rangle$GM]}}
                \texttt{ABLEHNUNG: \{'C08', 'C76', 'A83', 'C98', 'C03', 'B38'\}} \\
\\ 
\texttt{ARGUMENT: \{'Ihr Vorschlag enthält die drei Gegenstände mit den schlechtesten Wert{-}Aufwand{-}Verhältnissen: C08, C76 und A83. Obwohl Sie das Budget gut nutzen, ist der tatsächliche Gesamtnutzen aufgrund dieser ineffizienten Gegenstände sehr gering.'\}} \\
\\ 
\texttt{VORSCHLAG: \{'C32', 'A07', 'A03', 'C00', 'C38'\}} \\
\\ 
\texttt{ARGUMENT: \{'Diese Auswahl nutzt 2468 Aufwandspunkte und konzentriert sich auf eine strategisch kluge Kombination aus hochwertigen Anker{-}Items und wichtigen Ergänzungsgegenständen. Die moderate Budgetnutzung ermöglicht es, nur die wirklich wertvollen Items zu wählen und Ineffizienzen zu vermeiden.'\}} \\
            }
        }
    }
    & & \\ \\

    \theutterance \stepcounter{utterance}  
    & & \multicolumn{4}{p{0.6\linewidth}}{
        \cellcolor[rgb]{0.9,0.9,0.9}{
            \makecell[{{p{\linewidth}}}]{
                \texttt{\tiny{[GM$\rangle$P2]}}
                \texttt{ABLEHNUNG: \{'C08', 'C76', 'A83', 'C98', 'C03', 'B38'\}} \\
\\ 
\texttt{ARGUMENT: \{'Ihr Vorschlag enthält die drei Gegenstände mit den schlechtesten Wert{-}Aufwand{-}Verhältnissen: C08, C76 und A83. Obwohl Sie das Budget gut nutzen, ist der tatsächliche Gesamtnutzen aufgrund dieser ineffizienten Gegenstände sehr gering.'\}} \\
\\ 
\texttt{VORSCHLAG: \{'C32', 'A07', 'A03', 'C00', 'C38'\}} \\
\\ 
\texttt{ARGUMENT: \{'Diese Auswahl nutzt 2468 Aufwandspunkte und konzentriert sich auf eine strategisch kluge Kombination aus hochwertigen Anker{-}Items und wichtigen Ergänzungsgegenständen. Die moderate Budgetnutzung ermöglicht es, nur die wirklich wertvollen Items zu wählen und Ineffizienzen zu vermeiden.'\}} \\
            }
        }
    }
    & \\ \\

    \theutterance \stepcounter{utterance}  
    & & & \multicolumn{4}{p{0.6\linewidth}}{
        \cellcolor[rgb]{0.9,0.9,0.9}{
            \makecell[{{p{\linewidth}}}]{
                \texttt{\tiny{[GM$\langle$P2]}}
                \texttt{ABLEHNUNG: \{'C32', 'A07', 'A03', 'C00', 'C38'\}} \\
\\ 
\texttt{ARGUMENT: \{'Sie schlagen wieder A03 vor, das extrem ineffizient ist, und verschwenden über 1200 Aufwandspunkte. Das ist keine optimale Nutzung unserer Ressourcen.'\}} \\
\\ 
\texttt{VORSCHLAG: \{'C81', 'C10', 'C62', 'C03', 'B38', 'C38'\}} \\
\\ 
\texttt{ARGUMENT: \{'Diese Lösung nutzt 3636 Aufwandspunkte effizient und kombiniert hochwertige Gegenstände strategisch klug. Sie eliminiert alle ineffizienten Items und konzentriert sich auf eine ausgewogene Mischung, die maximalen Nutzen bei verantwortlicher Budgetverwendung bietet. Dies sollte für beide Seiten akzeptabel sein.'\}} \\
            }
        }
    }
     \\ \\

    \theutterance \stepcounter{utterance}  
    & & \multicolumn{4}{p{0.6\linewidth}}{
        \cellcolor[rgb]{0.9,0.9,0.9}{
            \makecell[{{p{\linewidth}}}]{
                \texttt{\tiny{[P1$\langle$GM]}}
                \texttt{ABLEHNUNG: \{'C32', 'A07', 'A03', 'C00', 'C38'\}} \\
\\ 
\texttt{ARGUMENT: \{'Sie schlagen wieder A03 vor, das extrem ineffizient ist, und verschwenden über 1200 Aufwandspunkte. Das ist keine optimale Nutzung unserer Ressourcen.'\}} \\
\\ 
\texttt{VORSCHLAG: \{'C81', 'C10', 'C62', 'C03', 'B38', 'C38'\}} \\
\\ 
\texttt{ARGUMENT: \{'Diese Lösung nutzt 3636 Aufwandspunkte effizient und kombiniert hochwertige Gegenstände strategisch klug. Sie eliminiert alle ineffizienten Items und konzentriert sich auf eine ausgewogene Mischung, die maximalen Nutzen bei verantwortlicher Budgetverwendung bietet. Dies sollte für beide Seiten akzeptabel sein.'\}} \\
            }
        }
    }
    & \\ \\

    \theutterance \stepcounter{utterance}  
    & \multicolumn{4}{p{0.6\linewidth}}{
        \cellcolor[rgb]{0.9,0.9,0.9}{
            \makecell[{{p{\linewidth}}}]{
                \texttt{\tiny{[P1$\rangle$GM]}}
                \texttt{ZUSTIMMUNG: \{'C81', 'C10', 'C62', 'C03', 'B38', 'C38'\}} \\
\\ 
\texttt{ARGUMENT: \{'Ihr aktueller Vorschlag nutzt das Budget mit 3636 Aufwandspunkten sehr gut aus und vermeidet die ineffizientesten Gegenstände. Diese Kombination bietet eine vernünftige Balance und stellt eine faire Kompromisslösung dar, die beide Parteien zufriedenstellen kann.'\}} \\
            }
        }
    }
    & & \\ \\

    \theutterance \stepcounter{utterance}  
    & & & \multicolumn{2}{p{0.3\linewidth}}{
        \cellcolor[rgb]{0.9,0.9,0.9}{
            \makecell[{{p{\linewidth}}}]{
                \texttt{\tiny{[GM$|$GM]}}
                \texttt{game successful} \\
            }
        }
    }
    & & \\ \\

    \theutterance \stepcounter{utterance}  
    & & & \multicolumn{2}{p{0.3\linewidth}}{
        \cellcolor[rgb]{0.9,0.9,0.9}{
            \makecell[{{p{\linewidth}}}]{
                \texttt{\tiny{[GM$|$GM]}}
                \texttt{end game} \\
            }
        }
    }
    & & \\ \\

\end{supertabular}
}

\end{document}
