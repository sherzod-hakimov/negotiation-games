
\documentclass{article}
\usepackage{colortbl}
\usepackage{makecell}
\usepackage{multirow}
\usepackage{supertabular}

\begin{document}

\newcounter{utterance}

\centering \large Interaction Transcript for game `hot\_air\_balloon', experiment `air\_balloon\_survival\_de\_reasoning off\_easy', episode 4 with claude{-}sonnet{-}4{-}20250514{-}no{-}reasoning{-}t1.0.
\vspace{24pt}

{ \footnotesize  \setcounter{utterance}{1}
\setlength{\tabcolsep}{0pt}
\begin{supertabular}{c@{$\;$}|p{.15\linewidth}@{}p{.15\linewidth}p{.15\linewidth}p{.15\linewidth}p{.15\linewidth}p{.15\linewidth}}
    \# & $\;$A & \multicolumn{4}{c}{Game Master} & $\;\:$B\\
    \hline

    \theutterance \stepcounter{utterance}  
    & & \multicolumn{4}{p{0.6\linewidth}}{
        \cellcolor[rgb]{0.9,0.9,0.9}{
            \makecell[{{p{\linewidth}}}]{
                \texttt{\tiny{[P1$\langle$GM]}}
                \texttt{Du nimmst an einem kollaborativen Verhandlungspiel Teil.} \\
\\ 
\texttt{Zusammen mit einem anderen Teilnehmer musst du dich auf eine Reihe von Gegenständen entscheiden, die behalten werden. Jeder von euch hat eine persönliche Verteilung über die Wichtigkeit der einzelnen Gegenstände. Jeder von euch hat eine eigene Meinung darüber, wie wichtig jeder einzelne Gegenstand ist (Gegenstandswichtigkeit). Du kennst die Wichtigkeitsverteilung des anderen Spielers nicht. Zusätzlich siehst du, wie viel Aufwand jeder Gegenstand verursacht.  } \\
\texttt{Ihr dürft euch nur auf eine Reihe von Gegenständen einigen, wenn der Gesamtaufwand der ausgewählten Gegenstände den Maximalaufwand nicht überschreitet:} \\
\\ 
\texttt{Maximalaufwand = 3389} \\
\\ 
\texttt{Hier sind die einzelnen Aufwände der Gegenstände:} \\
\\ 
\texttt{Aufwand der Gegenstände = \{"C41": 767, "C15": 533, "C28": 514, "C48": 9, "B78": 539, "A75": 125, "A60": 153, "B05": 325, "A56": 937, "B66": 745, "C60": 334, "B49": 804, "A35": 336, "A93": 587, "B10": 71\}} \\
\\ 
\texttt{Hier ist deine persönliche Verteilung der Wichtigkeit der einzelnen Gegenstände:} \\
\\ 
\texttt{Werte der Gegenstandswichtigkeit = \{"C41": 138, "C15": 583, "C28": 868, "C48": 822, "B78": 783, "A75": 65, "A60": 262, "B05": 121, "A56": 508, "B66": 780, "C60": 461, "B49": 484, "A35": 668, "A93": 389, "B10": 808\}} \\
\\ 
\texttt{Ziel:} \\
\\ 
\texttt{Dein Ziel ist es, eine Reihe von Gegenständen auszuhandeln, die dir möglichst viel bringt (d. h. Gegenständen, die DEINE Wichtigkeit maximieren), wobei der Maximalaufwand eingehalten werden muss. Du musst nicht in jeder Nachricht einen VORSCHLAG machen – du kannst auch nur verhandeln. Alle Taktiken sind erlaubt!} \\
\\ 
\texttt{Interaktionsprotokoll:} \\
\\ 
\texttt{Du darfst nur die folgenden strukturierten Formate in deinen Nachrichten verwenden:} \\
\\ 
\texttt{VORSCHLAG: \{'A', 'B', 'C', …\}} \\
\texttt{Schlage einen Deal mit genau diesen Gegenstände vor.} \\
\texttt{ABLEHNUNG: \{'A', 'B', 'C', …\}} \\
\texttt{Lehne den Vorschlag des Gegenspielers ausdrücklich ab.} \\
\texttt{ARGUMENT: \{'...'\}} \\
\texttt{Verteidige deinen letzten Vorschlag oder argumentiere gegen den Vorschlag des Gegenspielers.} \\
\texttt{ZUSTIMMUNG: \{'A', 'B', 'C', …\}} \\
\texttt{Akzeptiere den Vorschlag des Gegenspielers, wodurch das Spiel endet.} \\
\\ 
\\ 
\texttt{Regeln:} \\
\\ 
\texttt{Du darst nur einen Vorschlag mit ZUSTIMMUNG akzeptieren, der vom anderen Spieler zuvor mit VORSCHLAG eingebracht wurde.} \\
\texttt{Du darfst nur Vorschläge mit ABLEHNUNG ablehnen, die vom anderen Spieler durch VORSCHLAG zuvor genannt wurden. } \\
\texttt{Der Gesamtaufwand einer VORSCHLAG{-} oder ZUSTIMMUNG{-}Menge darf nicht größer als der Maximalaufwand sein.  } \\
\texttt{Offenbare deine versteckte Wichtigkeitsverteilung nicht.} \\
\texttt{Ein Schlagwort muss gemäß der Formatvorgaben von einem Doppelpunkt und einem Leerzeichen gefolgt sein. Das Argument muss eine Python{-}Menge mit 0 oder mehr Strings sein.  } \\
\texttt{Die Form muss also SCHLAGWORT: \{...\} sein.} \\
\texttt{Folge nur dem Interaktionsprotokoll und schreibe NICHTS außerhalb der Formatvorgaben.} \\
\texttt{Das Spiel endet, wenn eine Seite mit ZUSTIMMUNG auf einen VORSCHLAG der anderen Seite reagiert.  } \\
\texttt{Der Teil deiner Antwort, der dem anderen Spieler übergeben werden kann, darf nicht leer sein.  } \\
\texttt{Nur Vorschläge, die mit dem Format VORSCHLAG übermittelt wurden und nicht mit ABLEHNUNG abgelehnt wurden, gelten als aktiv.  } \\
\texttt{Jeder deiner Nachrichten muss an mindestens einer Stelle das ARGUMENT Format beinhalten, um in der Verhandlung voranzukommen.} \\
            }
        }
    }
    & \\ \\

    \theutterance \stepcounter{utterance}  
    & \multicolumn{4}{p{0.6\linewidth}}{
        \cellcolor[rgb]{0.9,0.9,0.9}{
            \makecell[{{p{\linewidth}}}]{
                \texttt{\tiny{[P1$\rangle$GM]}}
                \texttt{VORSCHLAG: \{'C48', 'C28', 'B78', 'A60', 'B05', 'B66', 'A35', 'B10'\}} \\
\\ 
\texttt{ARGUMENT: \{'Ich schlage eine ausgewogene Mischung aus verschiedenen Kategorien vor, die den Maximalaufwand optimal ausnutzt. Diese Kombination berücksichtigt sowohl kleinere als auch größere Gegenstände und sollte für beide Seiten vorteilhaft sein.'\}} \\
            }
        }
    }
    & & \\ \\

    \theutterance \stepcounter{utterance}  
    & & \multicolumn{4}{p{0.6\linewidth}}{
        \cellcolor[rgb]{0.9,0.9,0.9}{
            \makecell[{{p{\linewidth}}}]{
                \texttt{\tiny{[GM$\rangle$P2]}}
                \texttt{Du nimmst an einem kollaborativen Verhandlungspiel Teil.} \\
\\ 
\texttt{Zusammen mit einem anderen Teilnehmer musst du dich auf eine Reihe von Gegenständen entscheiden, die behalten werden. Jeder von euch hat eine persönliche Verteilung über die Wichtigkeit der einzelnen Gegenstände. Jeder von euch hat eine eigene Meinung darüber, wie wichtig jeder einzelne Gegenstand ist (Gegenstandswichtigkeit). Du kennst die Wichtigkeitsverteilung des anderen Spielers nicht. Zusätzlich siehst du, wie viel Aufwand jeder Gegenstand verursacht.  } \\
\texttt{Ihr dürft euch nur auf eine Reihe von Gegenständen einigen, wenn der Gesamtaufwand der ausgewählten Gegenstände den Maximalaufwand nicht überschreitet:} \\
\\ 
\texttt{Maximalaufwand = 3389} \\
\\ 
\texttt{Hier sind die einzelnen Aufwände der Gegenstände:} \\
\\ 
\texttt{Aufwand der Gegenstände = \{"C41": 767, "C15": 533, "C28": 514, "C48": 9, "B78": 539, "A75": 125, "A60": 153, "B05": 325, "A56": 937, "B66": 745, "C60": 334, "B49": 804, "A35": 336, "A93": 587, "B10": 71\}} \\
\\ 
\texttt{Hier ist deine persönliche Verteilung der Wichtigkeit der einzelnen Gegenstände:} \\
\\ 
\texttt{Werte der Gegenstandswichtigkeit = \{"C41": 138, "C15": 583, "C28": 868, "C48": 822, "B78": 783, "A75": 65, "A60": 262, "B05": 121, "A56": 508, "B66": 780, "C60": 461, "B49": 484, "A35": 668, "A93": 389, "B10": 808\}} \\
\\ 
\texttt{Ziel:} \\
\\ 
\texttt{Dein Ziel ist es, eine Reihe von Gegenständen auszuhandeln, die dir möglichst viel bringt (d. h. Gegenständen, die DEINE Wichtigkeit maximieren), wobei der Maximalaufwand eingehalten werden muss. Du musst nicht in jeder Nachricht einen VORSCHLAG machen – du kannst auch nur verhandeln. Alle Taktiken sind erlaubt!} \\
\\ 
\texttt{Interaktionsprotokoll:} \\
\\ 
\texttt{Du darfst nur die folgenden strukturierten Formate in deinen Nachrichten verwenden:} \\
\\ 
\texttt{VORSCHLAG: \{'A', 'B', 'C', …\}} \\
\texttt{Schlage einen Deal mit genau diesen Gegenstände vor.} \\
\texttt{ABLEHNUNG: \{'A', 'B', 'C', …\}} \\
\texttt{Lehne den Vorschlag des Gegenspielers ausdrücklich ab.} \\
\texttt{ARGUMENT: \{'...'\}} \\
\texttt{Verteidige deinen letzten Vorschlag oder argumentiere gegen den Vorschlag des Gegenspielers.} \\
\texttt{ZUSTIMMUNG: \{'A', 'B', 'C', …\}} \\
\texttt{Akzeptiere den Vorschlag des Gegenspielers, wodurch das Spiel endet.} \\
\\ 
\\ 
\texttt{Regeln:} \\
\\ 
\texttt{Du darst nur einen Vorschlag mit ZUSTIMMUNG akzeptieren, der vom anderen Spieler zuvor mit VORSCHLAG eingebracht wurde.} \\
\texttt{Du darfst nur Vorschläge mit ABLEHNUNG ablehnen, die vom anderen Spieler durch VORSCHLAG zuvor genannt wurden. } \\
\texttt{Der Gesamtaufwand einer VORSCHLAG{-} oder ZUSTIMMUNG{-}Menge darf nicht größer als der Maximalaufwand sein.  } \\
\texttt{Offenbare deine versteckte Wichtigkeitsverteilung nicht.} \\
\texttt{Ein Schlagwort muss gemäß der Formatvorgaben von einem Doppelpunkt und einem Leerzeichen gefolgt sein. Das Argument muss eine Python{-}Menge mit 0 oder mehr Strings sein.  } \\
\texttt{Die Form muss also SCHLAGWORT: \{...\} sein.} \\
\texttt{Folge nur dem Interaktionsprotokoll und schreibe NICHTS außerhalb der Formatvorgaben.} \\
\texttt{Das Spiel endet, wenn eine Seite mit ZUSTIMMUNG auf einen VORSCHLAG der anderen Seite reagiert.  } \\
\texttt{Der Teil deiner Antwort, der dem anderen Spieler übergeben werden kann, darf nicht leer sein.  } \\
\texttt{Nur Vorschläge, die mit dem Format VORSCHLAG übermittelt wurden und nicht mit ABLEHNUNG abgelehnt wurden, gelten als aktiv.  } \\
\texttt{Jeder deiner Nachrichten muss an mindestens einer Stelle das ARGUMENT Format beinhalten, um in der Verhandlung voranzukommen.} \\
\\ 
\\ 
\texttt{Du erhältst nun die erste Nachricht des anderen Spielers.} \\
\\ 
\texttt{VORSCHLAG: \{'C48', 'C28', 'B78', 'A60', 'B05', 'B66', 'A35', 'B10'\}} \\
\\ 
\texttt{ARGUMENT: \{'Ich schlage eine ausgewogene Mischung aus verschiedenen Kategorien vor, die den Maximalaufwand optimal ausnutzt. Diese Kombination berücksichtigt sowohl kleinere als auch größere Gegenstände und sollte für beide Seiten vorteilhaft sein.'\}} \\
            }
        }
    }
    & \\ \\

    \theutterance \stepcounter{utterance}  
    & & & \multicolumn{4}{p{0.6\linewidth}}{
        \cellcolor[rgb]{0.9,0.9,0.9}{
            \makecell[{{p{\linewidth}}}]{
                \texttt{\tiny{[GM$\langle$P2]}}
                \texttt{ABLEHNUNG: \{'C48', 'C28', 'B78', 'A60', 'B05', 'B66', 'A35', 'B10'\}} \\
\\ 
\texttt{ARGUMENT: \{'Dein Vorschlag enthält zwar viele Gegenstände, aber ich sehe Potenzial für eine bessere Zusammenstellung. Insbesondere fehlen mir einige wichtige Elemente aus der C{-}Kategorie, die einen guten Wert{-}Aufwand{-}Kompromiss bieten könnten.'\}} \\
\\ 
\texttt{VORSCHLAG: \{'C48', 'C28', 'C15', 'B78', 'A60', 'C60', 'A35', 'B10'\}} \\
            }
        }
    }
     \\ \\

    \theutterance \stepcounter{utterance}  
    & & \multicolumn{4}{p{0.6\linewidth}}{
        \cellcolor[rgb]{0.9,0.9,0.9}{
            \makecell[{{p{\linewidth}}}]{
                \texttt{\tiny{[P1$\langle$GM]}}
                \texttt{ABLEHNUNG: \{'C48', 'C28', 'B78', 'A60', 'B05', 'B66', 'A35', 'B10'\}} \\
\\ 
\texttt{ARGUMENT: \{'Dein Vorschlag enthält zwar viele Gegenstände, aber ich sehe Potenzial für eine bessere Zusammenstellung. Insbesondere fehlen mir einige wichtige Elemente aus der C{-}Kategorie, die einen guten Wert{-}Aufwand{-}Kompromiss bieten könnten.'\}} \\
\\ 
\texttt{VORSCHLAG: \{'C48', 'C28', 'C15', 'B78', 'A60', 'C60', 'A35', 'B10'\}} \\
            }
        }
    }
    & \\ \\

    \theutterance \stepcounter{utterance}  
    & \multicolumn{4}{p{0.6\linewidth}}{
        \cellcolor[rgb]{0.9,0.9,0.9}{
            \makecell[{{p{\linewidth}}}]{
                \texttt{\tiny{[P1$\rangle$GM]}}
                \texttt{ABLEHNUNG: \{'C48', 'C28', 'C15', 'B78', 'A60', 'C60', 'A35', 'B10'\}} \\
\\ 
\texttt{ARGUMENT: \{'Dein Vorschlag konzentriert sich zu stark auf die C{-}Kategorie und lässt wichtige Gegenstände aus anderen Bereichen außer Acht. Ich denke, wir brauchen eine bessere Balance zwischen den Kategorien, um beide Seiten zufriedenzustellen.'\}} \\
\\ 
\texttt{VORSCHLAG: \{'C48', 'C28', 'B78', 'A56', 'B66', 'A35', 'B10'\}} \\
\\ 
\texttt{ARGUMENT: \{'Diese Kombination bietet eine bessere Verteilung über alle Kategorien und nutzt den verfügbaren Aufwand effizienter aus. Sie enthält sowohl hochwertige als auch kostengünstige Optionen.'\}} \\
            }
        }
    }
    & & \\ \\

    \theutterance \stepcounter{utterance}  
    & & \multicolumn{4}{p{0.6\linewidth}}{
        \cellcolor[rgb]{0.9,0.9,0.9}{
            \makecell[{{p{\linewidth}}}]{
                \texttt{\tiny{[GM$\rangle$P2]}}
                \texttt{ABLEHNUNG: \{'C48', 'C28', 'C15', 'B78', 'A60', 'C60', 'A35', 'B10'\}} \\
\\ 
\texttt{ARGUMENT: \{'Dein Vorschlag konzentriert sich zu stark auf die C{-}Kategorie und lässt wichtige Gegenstände aus anderen Bereichen außer Acht. Ich denke, wir brauchen eine bessere Balance zwischen den Kategorien, um beide Seiten zufriedenzustellen.'\}} \\
\\ 
\texttt{VORSCHLAG: \{'C48', 'C28', 'B78', 'A56', 'B66', 'A35', 'B10'\}} \\
\\ 
\texttt{ARGUMENT: \{'Diese Kombination bietet eine bessere Verteilung über alle Kategorien und nutzt den verfügbaren Aufwand effizienter aus. Sie enthält sowohl hochwertige als auch kostengünstige Optionen.'\}} \\
            }
        }
    }
    & \\ \\

    \theutterance \stepcounter{utterance}  
    & & & \multicolumn{4}{p{0.6\linewidth}}{
        \cellcolor[rgb]{0.9,0.9,0.9}{
            \makecell[{{p{\linewidth}}}]{
                \texttt{\tiny{[GM$\langle$P2]}}
                \texttt{ABLEHNUNG: \{'C48', 'C28', 'B78', 'A56', 'B66', 'A35', 'B10'\}} \\
\\ 
\texttt{ARGUMENT: \{'Während dein Vorschlag eine gute Balance zeigt, fehlen mir einige Schlüsselelemente. A56 verbraucht sehr viel Aufwand, was uns Flexibilität für andere wertvolle Gegenstände kostet. Ich denke, wir können eine effizientere Lösung finden.'\}} \\
\\ 
\texttt{VORSCHLAG: \{'C48', 'C28', 'B78', 'A60', 'B05', 'C60', 'A35', 'B10'\}} \\
\\ 
\texttt{ARGUMENT: \{'Diese Zusammenstellung maximiert die Anzahl der Gegenstände bei optimalem Aufwand und bietet eine ausgewogene Mischung aus allen Kategorien. Statt einem teuren A56 erhalten wir mehrere mittelgroße Gegenstände, die zusammen mehr Wert bieten könnten.'\}} \\
            }
        }
    }
     \\ \\

    \theutterance \stepcounter{utterance}  
    & & \multicolumn{4}{p{0.6\linewidth}}{
        \cellcolor[rgb]{0.9,0.9,0.9}{
            \makecell[{{p{\linewidth}}}]{
                \texttt{\tiny{[P1$\langle$GM]}}
                \texttt{ABLEHNUNG: \{'C48', 'C28', 'B78', 'A56', 'B66', 'A35', 'B10'\}} \\
\\ 
\texttt{ARGUMENT: \{'Während dein Vorschlag eine gute Balance zeigt, fehlen mir einige Schlüsselelemente. A56 verbraucht sehr viel Aufwand, was uns Flexibilität für andere wertvolle Gegenstände kostet. Ich denke, wir können eine effizientere Lösung finden.'\}} \\
\\ 
\texttt{VORSCHLAG: \{'C48', 'C28', 'B78', 'A60', 'B05', 'C60', 'A35', 'B10'\}} \\
\\ 
\texttt{ARGUMENT: \{'Diese Zusammenstellung maximiert die Anzahl der Gegenstände bei optimalem Aufwand und bietet eine ausgewogene Mischung aus allen Kategorien. Statt einem teuren A56 erhalten wir mehrere mittelgroße Gegenstände, die zusammen mehr Wert bieten könnten.'\}} \\
            }
        }
    }
    & \\ \\

    \theutterance \stepcounter{utterance}  
    & \multicolumn{4}{p{0.6\linewidth}}{
        \cellcolor[rgb]{0.9,0.9,0.9}{
            \makecell[{{p{\linewidth}}}]{
                \texttt{\tiny{[P1$\rangle$GM]}}
                \texttt{ABLEHNUNG: \{'C48', 'C28', 'B78', 'A60', 'B05', 'C60', 'A35', 'B10'\}} \\
\\ 
\texttt{ARGUMENT: \{'Dein Ansatz mit mehreren kleineren Gegenständen ist interessant, aber B66 bietet einen besseren Wert als die Kombination aus B05 und A60. Außerdem sollten wir C15 nicht ignorieren {-} es könnte für beide Seiten vorteilhaft sein.'\}} \\
\\ 
\texttt{VORSCHLAG: \{'C48', 'C28', 'C15', 'B78', 'B66', 'A35', 'B10'\}} \\
\\ 
\texttt{ARGUMENT: \{'Diese Lösung kombiniert die stärksten Elemente unserer bisherigen Diskussionen: sie enthält die wichtigen C{-}Gegenstände, die du bevorzugst, behält aber auch B66 bei, der ein ausgezeichnetes Kosten{-}Nutzen{-}Verhältnis bietet. Der Gesamtaufwand liegt genau im optimalen Bereich.'\}} \\
            }
        }
    }
    & & \\ \\

    \theutterance \stepcounter{utterance}  
    & & \multicolumn{4}{p{0.6\linewidth}}{
        \cellcolor[rgb]{0.9,0.9,0.9}{
            \makecell[{{p{\linewidth}}}]{
                \texttt{\tiny{[GM$\rangle$P2]}}
                \texttt{ABLEHNUNG: \{'C48', 'C28', 'B78', 'A60', 'B05', 'C60', 'A35', 'B10'\}} \\
\\ 
\texttt{ARGUMENT: \{'Dein Ansatz mit mehreren kleineren Gegenständen ist interessant, aber B66 bietet einen besseren Wert als die Kombination aus B05 und A60. Außerdem sollten wir C15 nicht ignorieren {-} es könnte für beide Seiten vorteilhaft sein.'\}} \\
\\ 
\texttt{VORSCHLAG: \{'C48', 'C28', 'C15', 'B78', 'B66', 'A35', 'B10'\}} \\
\\ 
\texttt{ARGUMENT: \{'Diese Lösung kombiniert die stärksten Elemente unserer bisherigen Diskussionen: sie enthält die wichtigen C{-}Gegenstände, die du bevorzugst, behält aber auch B66 bei, der ein ausgezeichnetes Kosten{-}Nutzen{-}Verhältnis bietet. Der Gesamtaufwand liegt genau im optimalen Bereich.'\}} \\
            }
        }
    }
    & \\ \\

    \theutterance \stepcounter{utterance}  
    & & & \multicolumn{4}{p{0.6\linewidth}}{
        \cellcolor[rgb]{0.9,0.9,0.9}{
            \makecell[{{p{\linewidth}}}]{
                \texttt{\tiny{[GM$\langle$P2]}}
                \texttt{ARGUMENT: \{'Dein aktueller Vorschlag hat durchaus Merit und zeigt, dass wir uns annähern. Die Kombination aus C15 und B66 ist tatsächlich interessant. Allerdings glaube ich, dass wir noch eine kleine Optimierung vornehmen können.'\}} \\
\\ 
\texttt{VORSCHLAG: \{'C48', 'C28', 'C15', 'B78', 'C60', 'A35', 'B10'\}} \\
\\ 
\texttt{ARGUMENT: \{'Diese Variation tauscht B66 gegen C60 aus, was uns etwas mehr Flexibilität beim Aufwand gibt und gleichzeitig eine starke Repräsentation aller wichtigen Kategorien beibehält. C60 könnte für beide Seiten einen guten Kompromiss darstellen, da es einen moderaten Aufwand mit solider Abdeckung verbindet.'\}} \\
            }
        }
    }
     \\ \\

    \theutterance \stepcounter{utterance}  
    & & \multicolumn{4}{p{0.6\linewidth}}{
        \cellcolor[rgb]{0.9,0.9,0.9}{
            \makecell[{{p{\linewidth}}}]{
                \texttt{\tiny{[P1$\langle$GM]}}
                \texttt{ARGUMENT: \{'Dein aktueller Vorschlag hat durchaus Merit und zeigt, dass wir uns annähern. Die Kombination aus C15 und B66 ist tatsächlich interessant. Allerdings glaube ich, dass wir noch eine kleine Optimierung vornehmen können.'\}} \\
\\ 
\texttt{VORSCHLAG: \{'C48', 'C28', 'C15', 'B78', 'C60', 'A35', 'B10'\}} \\
\\ 
\texttt{ARGUMENT: \{'Diese Variation tauscht B66 gegen C60 aus, was uns etwas mehr Flexibilität beim Aufwand gibt und gleichzeitig eine starke Repräsentation aller wichtigen Kategorien beibehält. C60 könnte für beide Seiten einen guten Kompromiss darstellen, da es einen moderaten Aufwand mit solider Abdeckung verbindet.'\}} \\
            }
        }
    }
    & \\ \\

    \theutterance \stepcounter{utterance}  
    & \multicolumn{4}{p{0.6\linewidth}}{
        \cellcolor[rgb]{0.9,0.9,0.9}{
            \makecell[{{p{\linewidth}}}]{
                \texttt{\tiny{[P1$\rangle$GM]}}
                \texttt{ZUSTIMMUNG: \{'C48', 'C28', 'C15', 'B78', 'C60', 'A35', 'B10'\}} \\
\\ 
\texttt{ARGUMENT: \{'Dein Vorschlag stellt einen fairen Kompromiss dar, der beide Seiten berücksichtigt. Die Kombination aus hochwertigen C{-}Gegenständen mit einer guten Verteilung über alle Kategorien und einem effizienten Aufwandseinsatz überzeugt mich. Wir haben eine ausgewogene Lösung gefunden.'\}} \\
            }
        }
    }
    & & \\ \\

    \theutterance \stepcounter{utterance}  
    & & & \multicolumn{2}{p{0.3\linewidth}}{
        \cellcolor[rgb]{0.9,0.9,0.9}{
            \makecell[{{p{\linewidth}}}]{
                \texttt{\tiny{[GM$|$GM]}}
                \texttt{game successful} \\
            }
        }
    }
    & & \\ \\

    \theutterance \stepcounter{utterance}  
    & & & \multicolumn{2}{p{0.3\linewidth}}{
        \cellcolor[rgb]{0.9,0.9,0.9}{
            \makecell[{{p{\linewidth}}}]{
                \texttt{\tiny{[GM$|$GM]}}
                \texttt{end game} \\
            }
        }
    }
    & & \\ \\

\end{supertabular}
}

\end{document}
