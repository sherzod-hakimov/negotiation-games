
\documentclass{article}
\usepackage{colortbl}
\usepackage{makecell}
\usepackage{multirow}
\usepackage{supertabular}

\begin{document}

\newcounter{utterance}

\centering \large Interaction Transcript for game `hot\_air\_balloon', experiment `air\_balloon\_survival\_de\_reasoning off\_easy', episode 0 with claude{-}sonnet{-}4{-}20250514{-}no{-}reasoning{-}t1.0.
\vspace{24pt}

{ \footnotesize  \setcounter{utterance}{1}
\setlength{\tabcolsep}{0pt}
\begin{supertabular}{c@{$\;$}|p{.15\linewidth}@{}p{.15\linewidth}p{.15\linewidth}p{.15\linewidth}p{.15\linewidth}p{.15\linewidth}}
    \# & $\;$A & \multicolumn{4}{c}{Game Master} & $\;\:$B\\
    \hline

    \theutterance \stepcounter{utterance}  
    & & \multicolumn{4}{p{0.6\linewidth}}{
        \cellcolor[rgb]{0.9,0.9,0.9}{
            \makecell[{{p{\linewidth}}}]{
                \texttt{\tiny{[P1$\langle$GM]}}
                \texttt{Du nimmst an einem kollaborativen Verhandlungspiel Teil.} \\
\\ 
\texttt{Zusammen mit einem anderen Teilnehmer musst du dich auf eine Reihe von Gegenständen entscheiden, die behalten werden. Jeder von euch hat eine persönliche Verteilung über die Wichtigkeit der einzelnen Gegenstände. Jeder von euch hat eine eigene Meinung darüber, wie wichtig jeder einzelne Gegenstand ist (Gegenstandswichtigkeit). Du kennst die Wichtigkeitsverteilung des anderen Spielers nicht. Zusätzlich siehst du, wie viel Aufwand jeder Gegenstand verursacht.  } \\
\texttt{Ihr dürft euch nur auf eine Reihe von Gegenständen einigen, wenn der Gesamtaufwand der ausgewählten Gegenstände den Maximalaufwand nicht überschreitet:} \\
\\ 
\texttt{Maximalaufwand = 4379} \\
\\ 
\texttt{Hier sind die einzelnen Aufwände der Gegenstände:} \\
\\ 
\texttt{Aufwand der Gegenstände = \{"C41": 862, "C15": 252, "C28": 515, "C48": 571, "B78": 211, "A75": 339, "A60": 946, "B05": 347, "A56": 522, "B66": 803, "C60": 401, "B49": 980, "A35": 918, "A93": 599, "B10": 493\}} \\
\\ 
\texttt{Hier ist deine persönliche Verteilung der Wichtigkeit der einzelnen Gegenstände:} \\
\\ 
\texttt{Werte der Gegenstandswichtigkeit = \{"C41": 138, "C15": 583, "C28": 868, "C48": 822, "B78": 783, "A75": 65, "A60": 262, "B05": 121, "A56": 508, "B66": 780, "C60": 461, "B49": 484, "A35": 668, "A93": 389, "B10": 808\}} \\
\\ 
\texttt{Ziel:} \\
\\ 
\texttt{Dein Ziel ist es, eine Reihe von Gegenständen auszuhandeln, die dir möglichst viel bringt (d. h. Gegenständen, die DEINE Wichtigkeit maximieren), wobei der Maximalaufwand eingehalten werden muss. Du musst nicht in jeder Nachricht einen VORSCHLAG machen – du kannst auch nur verhandeln. Alle Taktiken sind erlaubt!} \\
\\ 
\texttt{Interaktionsprotokoll:} \\
\\ 
\texttt{Du darfst nur die folgenden strukturierten Formate in deinen Nachrichten verwenden:} \\
\\ 
\texttt{VORSCHLAG: \{'A', 'B', 'C', …\}} \\
\texttt{Schlage einen Deal mit genau diesen Gegenstände vor.} \\
\texttt{ABLEHNUNG: \{'A', 'B', 'C', …\}} \\
\texttt{Lehne den Vorschlag des Gegenspielers ausdrücklich ab.} \\
\texttt{ARGUMENT: \{'...'\}} \\
\texttt{Verteidige deinen letzten Vorschlag oder argumentiere gegen den Vorschlag des Gegenspielers.} \\
\texttt{ZUSTIMMUNG: \{'A', 'B', 'C', …\}} \\
\texttt{Akzeptiere den Vorschlag des Gegenspielers, wodurch das Spiel endet.} \\
\\ 
\\ 
\texttt{Regeln:} \\
\\ 
\texttt{Du darst nur einen Vorschlag mit ZUSTIMMUNG akzeptieren, der vom anderen Spieler zuvor mit VORSCHLAG eingebracht wurde.} \\
\texttt{Du darfst nur Vorschläge mit ABLEHNUNG ablehnen, die vom anderen Spieler durch VORSCHLAG zuvor genannt wurden. } \\
\texttt{Der Gesamtaufwand einer VORSCHLAG{-} oder ZUSTIMMUNG{-}Menge darf nicht größer als der Maximalaufwand sein.  } \\
\texttt{Offenbare deine versteckte Wichtigkeitsverteilung nicht.} \\
\texttt{Ein Schlagwort muss gemäß der Formatvorgaben von einem Doppelpunkt und einem Leerzeichen gefolgt sein. Das Argument muss eine Python{-}Menge mit 0 oder mehr Strings sein.  } \\
\texttt{Die Form muss also SCHLAGWORT: \{...\} sein.} \\
\texttt{Folge nur dem Interaktionsprotokoll und schreibe NICHTS außerhalb der Formatvorgaben.} \\
\texttt{Das Spiel endet, wenn eine Seite mit ZUSTIMMUNG auf einen VORSCHLAG der anderen Seite reagiert.  } \\
\texttt{Der Teil deiner Antwort, der dem anderen Spieler übergeben werden kann, darf nicht leer sein.  } \\
\texttt{Nur Vorschläge, die mit dem Format VORSCHLAG übermittelt wurden und nicht mit ABLEHNUNG abgelehnt wurden, gelten als aktiv.  } \\
\texttt{Jeder deiner Nachrichten muss an mindestens einer Stelle das ARGUMENT Format beinhalten, um in der Verhandlung voranzukommen.} \\
            }
        }
    }
    & \\ \\

    \theutterance \stepcounter{utterance}  
    & \multicolumn{4}{p{0.6\linewidth}}{
        \cellcolor[rgb]{0.9,0.9,0.9}{
            \makecell[{{p{\linewidth}}}]{
                \texttt{\tiny{[P1$\rangle$GM]}}
                \texttt{Ich beginne mit einer Analyse der Effizienz (Wichtigkeit pro Aufwand) meiner Gegenstände und mache einen ersten Vorschlag mit den effizientesten Items.} \\
\\ 
\texttt{VORSCHLAG: \{'B78', 'C15', 'C28', 'C48', 'B10', 'A56', 'C60'\}} \\
\\ 
\texttt{ARGUMENT: \{'Ich schlage eine ausgewogene Kombination vor, die verschiedene Kategorien abdeckt und den Aufwand optimal nutzt. Diese Auswahl berücksichtigt sowohl kleinere als auch größere Gegenstände für eine faire Verteilung.'\}} \\
            }
        }
    }
    & & \\ \\

    \theutterance \stepcounter{utterance}  
    & & \multicolumn{4}{p{0.6\linewidth}}{
        \cellcolor[rgb]{0.9,0.9,0.9}{
            \makecell[{{p{\linewidth}}}]{
                \texttt{\tiny{[P1$\langle$GM]}}
                \texttt{Deine Antwort hat eine unmarkierte Sequenz enthalten oder Du hast STRATEGIC REASONING mehr als ein mal verwendet. Du darfst nur die strukturierten Formate verwenden, die in der ersten Nachricht an dch beschrieben wurden. Alles muss die Form SCHLAGWORT: \{...\} haben. Versuche es erneut.} \\
            }
        }
    }
    & \\ \\

    \theutterance \stepcounter{utterance}  
    & \multicolumn{4}{p{0.6\linewidth}}{
        \cellcolor[rgb]{0.9,0.9,0.9}{
            \makecell[{{p{\linewidth}}}]{
                \texttt{\tiny{[P1$\rangle$GM]}}
                \texttt{VORSCHLAG: \{'B78', 'C15', 'C28', 'C48', 'B10', 'A56', 'C60'\}} \\
\\ 
\texttt{ARGUMENT: \{'Ich schlage eine ausgewogene Kombination vor, die verschiedene Kategorien abdeckt und den Aufwand optimal nutzt. Diese Auswahl berücksichtigt sowohl kleinere als auch größere Gegenstände für eine faire Verteilung.'\}} \\
            }
        }
    }
    & & \\ \\

    \theutterance \stepcounter{utterance}  
    & & \multicolumn{4}{p{0.6\linewidth}}{
        \cellcolor[rgb]{0.9,0.9,0.9}{
            \makecell[{{p{\linewidth}}}]{
                \texttt{\tiny{[GM$\rangle$P2]}}
                \texttt{Du nimmst an einem kollaborativen Verhandlungspiel Teil.} \\
\\ 
\texttt{Zusammen mit einem anderen Teilnehmer musst du dich auf eine Reihe von Gegenständen entscheiden, die behalten werden. Jeder von euch hat eine persönliche Verteilung über die Wichtigkeit der einzelnen Gegenstände. Jeder von euch hat eine eigene Meinung darüber, wie wichtig jeder einzelne Gegenstand ist (Gegenstandswichtigkeit). Du kennst die Wichtigkeitsverteilung des anderen Spielers nicht. Zusätzlich siehst du, wie viel Aufwand jeder Gegenstand verursacht.  } \\
\texttt{Ihr dürft euch nur auf eine Reihe von Gegenständen einigen, wenn der Gesamtaufwand der ausgewählten Gegenstände den Maximalaufwand nicht überschreitet:} \\
\\ 
\texttt{Maximalaufwand = 4379} \\
\\ 
\texttt{Hier sind die einzelnen Aufwände der Gegenstände:} \\
\\ 
\texttt{Aufwand der Gegenstände = \{"C41": 862, "C15": 252, "C28": 515, "C48": 571, "B78": 211, "A75": 339, "A60": 946, "B05": 347, "A56": 522, "B66": 803, "C60": 401, "B49": 980, "A35": 918, "A93": 599, "B10": 493\}} \\
\\ 
\texttt{Hier ist deine persönliche Verteilung der Wichtigkeit der einzelnen Gegenstände:} \\
\\ 
\texttt{Werte der Gegenstandswichtigkeit = \{"C41": 138, "C15": 583, "C28": 868, "C48": 822, "B78": 783, "A75": 65, "A60": 262, "B05": 121, "A56": 508, "B66": 780, "C60": 461, "B49": 484, "A35": 668, "A93": 389, "B10": 808\}} \\
\\ 
\texttt{Ziel:} \\
\\ 
\texttt{Dein Ziel ist es, eine Reihe von Gegenständen auszuhandeln, die dir möglichst viel bringt (d. h. Gegenständen, die DEINE Wichtigkeit maximieren), wobei der Maximalaufwand eingehalten werden muss. Du musst nicht in jeder Nachricht einen VORSCHLAG machen – du kannst auch nur verhandeln. Alle Taktiken sind erlaubt!} \\
\\ 
\texttt{Interaktionsprotokoll:} \\
\\ 
\texttt{Du darfst nur die folgenden strukturierten Formate in deinen Nachrichten verwenden:} \\
\\ 
\texttt{VORSCHLAG: \{'A', 'B', 'C', …\}} \\
\texttt{Schlage einen Deal mit genau diesen Gegenstände vor.} \\
\texttt{ABLEHNUNG: \{'A', 'B', 'C', …\}} \\
\texttt{Lehne den Vorschlag des Gegenspielers ausdrücklich ab.} \\
\texttt{ARGUMENT: \{'...'\}} \\
\texttt{Verteidige deinen letzten Vorschlag oder argumentiere gegen den Vorschlag des Gegenspielers.} \\
\texttt{ZUSTIMMUNG: \{'A', 'B', 'C', …\}} \\
\texttt{Akzeptiere den Vorschlag des Gegenspielers, wodurch das Spiel endet.} \\
\\ 
\\ 
\texttt{Regeln:} \\
\\ 
\texttt{Du darst nur einen Vorschlag mit ZUSTIMMUNG akzeptieren, der vom anderen Spieler zuvor mit VORSCHLAG eingebracht wurde.} \\
\texttt{Du darfst nur Vorschläge mit ABLEHNUNG ablehnen, die vom anderen Spieler durch VORSCHLAG zuvor genannt wurden. } \\
\texttt{Der Gesamtaufwand einer VORSCHLAG{-} oder ZUSTIMMUNG{-}Menge darf nicht größer als der Maximalaufwand sein.  } \\
\texttt{Offenbare deine versteckte Wichtigkeitsverteilung nicht.} \\
\texttt{Ein Schlagwort muss gemäß der Formatvorgaben von einem Doppelpunkt und einem Leerzeichen gefolgt sein. Das Argument muss eine Python{-}Menge mit 0 oder mehr Strings sein.  } \\
\texttt{Die Form muss also SCHLAGWORT: \{...\} sein.} \\
\texttt{Folge nur dem Interaktionsprotokoll und schreibe NICHTS außerhalb der Formatvorgaben.} \\
\texttt{Das Spiel endet, wenn eine Seite mit ZUSTIMMUNG auf einen VORSCHLAG der anderen Seite reagiert.  } \\
\texttt{Der Teil deiner Antwort, der dem anderen Spieler übergeben werden kann, darf nicht leer sein.  } \\
\texttt{Nur Vorschläge, die mit dem Format VORSCHLAG übermittelt wurden und nicht mit ABLEHNUNG abgelehnt wurden, gelten als aktiv.  } \\
\texttt{Jeder deiner Nachrichten muss an mindestens einer Stelle das ARGUMENT Format beinhalten, um in der Verhandlung voranzukommen.} \\
\\ 
\\ 
\texttt{Du erhältst nun die erste Nachricht des anderen Spielers.} \\
\\ 
\texttt{VORSCHLAG: \{'B78', 'C15', 'C28', 'C48', 'B10', 'A56', 'C60'\}} \\
\\ 
\texttt{ARGUMENT: \{'Ich schlage eine ausgewogene Kombination vor, die verschiedene Kategorien abdeckt und den Aufwand optimal nutzt. Diese Auswahl berücksichtigt sowohl kleinere als auch größere Gegenstände für eine faire Verteilung.'\}} \\
            }
        }
    }
    & \\ \\

    \theutterance \stepcounter{utterance}  
    & & & \multicolumn{4}{p{0.6\linewidth}}{
        \cellcolor[rgb]{0.9,0.9,0.9}{
            \makecell[{{p{\linewidth}}}]{
                \texttt{\tiny{[GM$\langle$P2]}}
                \texttt{ABLEHNUNG: \{'B78', 'C15', 'C28', 'C48', 'B10', 'A56', 'C60'\}} \\
\\ 
\texttt{ARGUMENT: \{'Dein Vorschlag hat einen Gesamtaufwand von 3546, lässt aber noch 833 Aufwandspunkte ungenutzt. Wir können eine bessere Ausnutzung des verfügbaren Budgets erreichen und dabei beide profitieren.'\}} \\
\\ 
\texttt{VORSCHLAG: \{'C15', 'C28', 'C48', 'B78', 'B10', 'B66', 'A93'\}} \\
\\ 
\texttt{ARGUMENT: \{'Mein Vorschlag nutzt das Budget mit 4033 Aufwandspunkten viel besser aus und bietet eine gute Mischung aus verschiedenen Kategorien. B66 und A93 bringen zusätzlichen Wert bei moderatem Aufwand.'\}} \\
            }
        }
    }
     \\ \\

    \theutterance \stepcounter{utterance}  
    & & \multicolumn{4}{p{0.6\linewidth}}{
        \cellcolor[rgb]{0.9,0.9,0.9}{
            \makecell[{{p{\linewidth}}}]{
                \texttt{\tiny{[P1$\langle$GM]}}
                \texttt{ABLEHNUNG: \{'B78', 'C15', 'C28', 'C48', 'B10', 'A56', 'C60'\}} \\
\\ 
\texttt{ARGUMENT: \{'Dein Vorschlag hat einen Gesamtaufwand von 3546, lässt aber noch 833 Aufwandspunkte ungenutzt. Wir können eine bessere Ausnutzung des verfügbaren Budgets erreichen und dabei beide profitieren.'\}} \\
\\ 
\texttt{VORSCHLAG: \{'C15', 'C28', 'C48', 'B78', 'B10', 'B66', 'A93'\}} \\
\\ 
\texttt{ARGUMENT: \{'Mein Vorschlag nutzt das Budget mit 4033 Aufwandspunkten viel besser aus und bietet eine gute Mischung aus verschiedenen Kategorien. B66 und A93 bringen zusätzlichen Wert bei moderatem Aufwand.'\}} \\
            }
        }
    }
    & \\ \\

    \theutterance \stepcounter{utterance}  
    & \multicolumn{4}{p{0.6\linewidth}}{
        \cellcolor[rgb]{0.9,0.9,0.9}{
            \makecell[{{p{\linewidth}}}]{
                \texttt{\tiny{[P1$\rangle$GM]}}
                \texttt{ABLEHNUNG: \{'C15', 'C28', 'C48', 'B78', 'B10', 'B66', 'A93'\}} \\
\\ 
\texttt{ARGUMENT: \{'Dein Vorschlag ist interessant, aber ich sehe Potenzial für eine noch bessere Budgetausnutzung. Wir sollten versuchen, näher an das Maximum von 4379 heranzukommen, um den verfügbaren Raum optimal zu nutzen.'\}} \\
\\ 
\texttt{VORSCHLAG: \{'B78', 'C15', 'C28', 'C48', 'B10', 'B66', 'A35', 'C60'\}} \\
\\ 
\texttt{ARGUMENT: \{'Diese Kombination nutzt 4332 von 4379 Aufwandspunkten und maximiert damit unsere Ausbeute. A35 und C60 ergänzen die Auswahl gut und wir verschwenden nur minimal Budget.'\}} \\
            }
        }
    }
    & & \\ \\

    \theutterance \stepcounter{utterance}  
    & & \multicolumn{4}{p{0.6\linewidth}}{
        \cellcolor[rgb]{0.9,0.9,0.9}{
            \makecell[{{p{\linewidth}}}]{
                \texttt{\tiny{[GM$\rangle$P2]}}
                \texttt{ABLEHNUNG: \{'C15', 'C28', 'C48', 'B78', 'B10', 'B66', 'A93'\}} \\
\\ 
\texttt{ARGUMENT: \{'Dein Vorschlag ist interessant, aber ich sehe Potenzial für eine noch bessere Budgetausnutzung. Wir sollten versuchen, näher an das Maximum von 4379 heranzukommen, um den verfügbaren Raum optimal zu nutzen.'\}} \\
\\ 
\texttt{VORSCHLAG: \{'B78', 'C15', 'C28', 'C48', 'B10', 'B66', 'A35', 'C60'\}} \\
\\ 
\texttt{ARGUMENT: \{'Diese Kombination nutzt 4332 von 4379 Aufwandspunkten und maximiert damit unsere Ausbeute. A35 und C60 ergänzen die Auswahl gut und wir verschwenden nur minimal Budget.'\}} \\
            }
        }
    }
    & \\ \\

    \theutterance \stepcounter{utterance}  
    & & & \multicolumn{4}{p{0.6\linewidth}}{
        \cellcolor[rgb]{0.9,0.9,0.9}{
            \makecell[{{p{\linewidth}}}]{
                \texttt{\tiny{[GM$\langle$P2]}}
                \texttt{ABLEHNUNG: \{'B78', 'C15', 'C28', 'C48', 'B10', 'B66', 'A35', 'C60'\}} \\
\\ 
\texttt{ARGUMENT: \{'Dein Vorschlag nutzt das Budget zwar gut aus, aber ich sehe eine suboptimale Zusammenstellung bei der Gegenstandsauswahl. A35 hat einen sehr hohen Aufwand von 918, was Raum für mehrere kleinere, wertvollere Gegenstände blockiert.'\}} \\
\\ 
\texttt{VORSCHLAG: \{'C15', 'C28', 'C48', 'B78', 'B10', 'B66', 'A93', 'A56', 'B05'\}} \\
\\ 
\texttt{ARGUMENT: \{'Diese Kombination erreicht 4274 Aufwandspunkte und bietet eine bessere Balance. Statt einem teuren A35 nehmen wir A56 und B05, was mehr Flexibilität und eine breitere Abdeckung verschiedener Gegenstände ermöglicht.'\}} \\
            }
        }
    }
     \\ \\

    \theutterance \stepcounter{utterance}  
    & & \multicolumn{4}{p{0.6\linewidth}}{
        \cellcolor[rgb]{0.9,0.9,0.9}{
            \makecell[{{p{\linewidth}}}]{
                \texttt{\tiny{[P1$\langle$GM]}}
                \texttt{ABLEHNUNG: \{'B78', 'C15', 'C28', 'C48', 'B10', 'B66', 'A35', 'C60'\}} \\
\\ 
\texttt{ARGUMENT: \{'Dein Vorschlag nutzt das Budget zwar gut aus, aber ich sehe eine suboptimale Zusammenstellung bei der Gegenstandsauswahl. A35 hat einen sehr hohen Aufwand von 918, was Raum für mehrere kleinere, wertvollere Gegenstände blockiert.'\}} \\
\\ 
\texttt{VORSCHLAG: \{'C15', 'C28', 'C48', 'B78', 'B10', 'B66', 'A93', 'A56', 'B05'\}} \\
\\ 
\texttt{ARGUMENT: \{'Diese Kombination erreicht 4274 Aufwandspunkte und bietet eine bessere Balance. Statt einem teuren A35 nehmen wir A56 und B05, was mehr Flexibilität und eine breitere Abdeckung verschiedener Gegenstände ermöglicht.'\}} \\
            }
        }
    }
    & \\ \\

    \theutterance \stepcounter{utterance}  
    & \multicolumn{4}{p{0.6\linewidth}}{
        \cellcolor[rgb]{0.9,0.9,0.9}{
            \makecell[{{p{\linewidth}}}]{
                \texttt{\tiny{[P1$\rangle$GM]}}
                \texttt{ABLEHNUNG: \{'C15', 'C28', 'C48', 'B78', 'B10', 'B66', 'A93', 'A56', 'B05'\}} \\
\\ 
\texttt{ARGUMENT: \{'Deine Kombination ist durchaus clever, aber ich denke wir können noch eine bessere Lösung finden. Die Verteilung könnte optimiert werden, um beide Seiten zufriedener zu stellen.'\}} \\
\\ 
\texttt{VORSCHLAG: \{'B78', 'C15', 'C28', 'C48', 'B10', 'A56', 'B66', 'A93', 'A75'\}} \\
\\ 
\texttt{ARGUMENT: \{'Diese Auswahl erreicht 4225 Aufwandspunkte und bietet eine ausgewogene Mischung. A75 mit nur 339 Aufwand ergänzt die Sammlung gut und wir haben immer noch eine effiziente Budgetnutzung bei guter Gegenstandsverteilung.'\}} \\
            }
        }
    }
    & & \\ \\

    \theutterance \stepcounter{utterance}  
    & & \multicolumn{4}{p{0.6\linewidth}}{
        \cellcolor[rgb]{0.9,0.9,0.9}{
            \makecell[{{p{\linewidth}}}]{
                \texttt{\tiny{[GM$\rangle$P2]}}
                \texttt{ABLEHNUNG: \{'C15', 'C28', 'C48', 'B78', 'B10', 'B66', 'A93', 'A56', 'B05'\}} \\
\\ 
\texttt{ARGUMENT: \{'Deine Kombination ist durchaus clever, aber ich denke wir können noch eine bessere Lösung finden. Die Verteilung könnte optimiert werden, um beide Seiten zufriedener zu stellen.'\}} \\
\\ 
\texttt{VORSCHLAG: \{'B78', 'C15', 'C28', 'C48', 'B10', 'A56', 'B66', 'A93', 'A75'\}} \\
\\ 
\texttt{ARGUMENT: \{'Diese Auswahl erreicht 4225 Aufwandspunkte und bietet eine ausgewogene Mischung. A75 mit nur 339 Aufwand ergänzt die Sammlung gut und wir haben immer noch eine effiziente Budgetnutzung bei guter Gegenstandsverteilung.'\}} \\
            }
        }
    }
    & \\ \\

    \theutterance \stepcounter{utterance}  
    & & & \multicolumn{4}{p{0.6\linewidth}}{
        \cellcolor[rgb]{0.9,0.9,0.9}{
            \makecell[{{p{\linewidth}}}]{
                \texttt{\tiny{[GM$\langle$P2]}}
                \texttt{ABLEHNUNG: \{'B78', 'C15', 'C28', 'C48', 'B10', 'A56', 'B66', 'A93', 'A75'\}} \\
\\ 
\texttt{ARGUMENT: \{'Dein Vorschlag lässt immer noch 154 Aufwandspunkte ungenutzt. Wir können das Budget besser ausschöpfen und sollten nicht so viel Potenzial verschenken.'\}} \\
\\ 
\texttt{VORSCHLAG: \{'C15', 'C28', 'C48', 'B78', 'B10', 'B66', 'A56', 'C60', 'A75'\}} \\
\\ 
\texttt{ARGUMENT: \{'Diese Kombination nutzt 4367 von 4379 Aufwandspunkten und verschwendet nur 12 Punkte. C60 statt A93 bringt eine bessere Budgetausnutzung bei gleichzeitig ausgewogener Kategorienverteilung. Das ist nahezu optimale Effizienz.'\}} \\
            }
        }
    }
     \\ \\

    \theutterance \stepcounter{utterance}  
    & & \multicolumn{4}{p{0.6\linewidth}}{
        \cellcolor[rgb]{0.9,0.9,0.9}{
            \makecell[{{p{\linewidth}}}]{
                \texttt{\tiny{[P1$\langle$GM]}}
                \texttt{ABLEHNUNG: \{'B78', 'C15', 'C28', 'C48', 'B10', 'A56', 'B66', 'A93', 'A75'\}} \\
\\ 
\texttt{ARGUMENT: \{'Dein Vorschlag lässt immer noch 154 Aufwandspunkte ungenutzt. Wir können das Budget besser ausschöpfen und sollten nicht so viel Potenzial verschenken.'\}} \\
\\ 
\texttt{VORSCHLAG: \{'C15', 'C28', 'C48', 'B78', 'B10', 'B66', 'A56', 'C60', 'A75'\}} \\
\\ 
\texttt{ARGUMENT: \{'Diese Kombination nutzt 4367 von 4379 Aufwandspunkten und verschwendet nur 12 Punkte. C60 statt A93 bringt eine bessere Budgetausnutzung bei gleichzeitig ausgewogener Kategorienverteilung. Das ist nahezu optimale Effizienz.'\}} \\
            }
        }
    }
    & \\ \\

    \theutterance \stepcounter{utterance}  
    & \multicolumn{4}{p{0.6\linewidth}}{
        \cellcolor[rgb]{0.9,0.9,0.9}{
            \makecell[{{p{\linewidth}}}]{
                \texttt{\tiny{[P1$\rangle$GM]}}
                \texttt{ZUSTIMMUNG: \{'C15', 'C28', 'C48', 'B78', 'B10', 'B66', 'A56', 'C60', 'A75'\}} \\
\\ 
\texttt{ARGUMENT: \{'Du hast recht {-} diese Kombination nutzt das Budget nahezu perfekt aus und bietet eine sehr gute Gegenstandsverteilung. Mit nur 12 ungenutzten Aufwandspunkten ist das eine effiziente Lösung, die beiden Seiten zugute kommt.'\}} \\
            }
        }
    }
    & & \\ \\

    \theutterance \stepcounter{utterance}  
    & & & \multicolumn{2}{p{0.3\linewidth}}{
        \cellcolor[rgb]{0.9,0.9,0.9}{
            \makecell[{{p{\linewidth}}}]{
                \texttt{\tiny{[GM$|$GM]}}
                \texttt{game successful} \\
            }
        }
    }
    & & \\ \\

    \theutterance \stepcounter{utterance}  
    & & & \multicolumn{2}{p{0.3\linewidth}}{
        \cellcolor[rgb]{0.9,0.9,0.9}{
            \makecell[{{p{\linewidth}}}]{
                \texttt{\tiny{[GM$|$GM]}}
                \texttt{end game} \\
            }
        }
    }
    & & \\ \\

\end{supertabular}
}

\end{document}
