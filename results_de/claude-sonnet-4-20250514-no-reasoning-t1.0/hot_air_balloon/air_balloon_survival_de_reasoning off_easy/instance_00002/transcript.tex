
\documentclass{article}
\usepackage{colortbl}
\usepackage{makecell}
\usepackage{multirow}
\usepackage{supertabular}

\begin{document}

\newcounter{utterance}

\centering \large Interaction Transcript for game `hot\_air\_balloon', experiment `air\_balloon\_survival\_de\_reasoning off\_easy', episode 2 with claude{-}sonnet{-}4{-}20250514{-}no{-}reasoning{-}t1.0.
\vspace{24pt}

{ \footnotesize  \setcounter{utterance}{1}
\setlength{\tabcolsep}{0pt}
\begin{supertabular}{c@{$\;$}|p{.15\linewidth}@{}p{.15\linewidth}p{.15\linewidth}p{.15\linewidth}p{.15\linewidth}p{.15\linewidth}}
    \# & $\;$A & \multicolumn{4}{c}{Game Master} & $\;\:$B\\
    \hline

    \theutterance \stepcounter{utterance}  
    & & \multicolumn{4}{p{0.6\linewidth}}{
        \cellcolor[rgb]{0.9,0.9,0.9}{
            \makecell[{{p{\linewidth}}}]{
                \texttt{\tiny{[P1$\langle$GM]}}
                \texttt{Du nimmst an einem kollaborativen Verhandlungspiel Teil.} \\
\\ 
\texttt{Zusammen mit einem anderen Teilnehmer musst du dich auf eine Reihe von Gegenständen entscheiden, die behalten werden. Jeder von euch hat eine persönliche Verteilung über die Wichtigkeit der einzelnen Gegenstände. Jeder von euch hat eine eigene Meinung darüber, wie wichtig jeder einzelne Gegenstand ist (Gegenstandswichtigkeit). Du kennst die Wichtigkeitsverteilung des anderen Spielers nicht. Zusätzlich siehst du, wie viel Aufwand jeder Gegenstand verursacht.  } \\
\texttt{Ihr dürft euch nur auf eine Reihe von Gegenständen einigen, wenn der Gesamtaufwand der ausgewählten Gegenstände den Maximalaufwand nicht überschreitet:} \\
\\ 
\texttt{Maximalaufwand = 3218} \\
\\ 
\texttt{Hier sind die einzelnen Aufwände der Gegenstände:} \\
\\ 
\texttt{Aufwand der Gegenstände = \{"C41": 919, "C15": 852, "C28": 299, "C48": 762, "B78": 161, "A75": 205, "A60": 380, "B05": 399, "A56": 534, "B66": 333, "C60": 100, "B49": 420, "A35": 354, "A93": 130, "B10": 589\}} \\
\\ 
\texttt{Hier ist deine persönliche Verteilung der Wichtigkeit der einzelnen Gegenstände:} \\
\\ 
\texttt{Werte der Gegenstandswichtigkeit = \{"C41": 138, "C15": 583, "C28": 868, "C48": 822, "B78": 783, "A75": 65, "A60": 262, "B05": 121, "A56": 508, "B66": 780, "C60": 461, "B49": 484, "A35": 668, "A93": 389, "B10": 808\}} \\
\\ 
\texttt{Ziel:} \\
\\ 
\texttt{Dein Ziel ist es, eine Reihe von Gegenständen auszuhandeln, die dir möglichst viel bringt (d. h. Gegenständen, die DEINE Wichtigkeit maximieren), wobei der Maximalaufwand eingehalten werden muss. Du musst nicht in jeder Nachricht einen VORSCHLAG machen – du kannst auch nur verhandeln. Alle Taktiken sind erlaubt!} \\
\\ 
\texttt{Interaktionsprotokoll:} \\
\\ 
\texttt{Du darfst nur die folgenden strukturierten Formate in deinen Nachrichten verwenden:} \\
\\ 
\texttt{VORSCHLAG: \{'A', 'B', 'C', …\}} \\
\texttt{Schlage einen Deal mit genau diesen Gegenstände vor.} \\
\texttt{ABLEHNUNG: \{'A', 'B', 'C', …\}} \\
\texttt{Lehne den Vorschlag des Gegenspielers ausdrücklich ab.} \\
\texttt{ARGUMENT: \{'...'\}} \\
\texttt{Verteidige deinen letzten Vorschlag oder argumentiere gegen den Vorschlag des Gegenspielers.} \\
\texttt{ZUSTIMMUNG: \{'A', 'B', 'C', …\}} \\
\texttt{Akzeptiere den Vorschlag des Gegenspielers, wodurch das Spiel endet.} \\
\\ 
\\ 
\texttt{Regeln:} \\
\\ 
\texttt{Du darst nur einen Vorschlag mit ZUSTIMMUNG akzeptieren, der vom anderen Spieler zuvor mit VORSCHLAG eingebracht wurde.} \\
\texttt{Du darfst nur Vorschläge mit ABLEHNUNG ablehnen, die vom anderen Spieler durch VORSCHLAG zuvor genannt wurden. } \\
\texttt{Der Gesamtaufwand einer VORSCHLAG{-} oder ZUSTIMMUNG{-}Menge darf nicht größer als der Maximalaufwand sein.  } \\
\texttt{Offenbare deine versteckte Wichtigkeitsverteilung nicht.} \\
\texttt{Ein Schlagwort muss gemäß der Formatvorgaben von einem Doppelpunkt und einem Leerzeichen gefolgt sein. Das Argument muss eine Python{-}Menge mit 0 oder mehr Strings sein.  } \\
\texttt{Die Form muss also SCHLAGWORT: \{...\} sein.} \\
\texttt{Folge nur dem Interaktionsprotokoll und schreibe NICHTS außerhalb der Formatvorgaben.} \\
\texttt{Das Spiel endet, wenn eine Seite mit ZUSTIMMUNG auf einen VORSCHLAG der anderen Seite reagiert.  } \\
\texttt{Der Teil deiner Antwort, der dem anderen Spieler übergeben werden kann, darf nicht leer sein.  } \\
\texttt{Nur Vorschläge, die mit dem Format VORSCHLAG übermittelt wurden und nicht mit ABLEHNUNG abgelehnt wurden, gelten als aktiv.  } \\
\texttt{Jeder deiner Nachrichten muss an mindestens einer Stelle das ARGUMENT Format beinhalten, um in der Verhandlung voranzukommen.} \\
            }
        }
    }
    & \\ \\

    \theutterance \stepcounter{utterance}  
    & \multicolumn{4}{p{0.6\linewidth}}{
        \cellcolor[rgb]{0.9,0.9,0.9}{
            \makecell[{{p{\linewidth}}}]{
                \texttt{\tiny{[P1$\rangle$GM]}}
                \texttt{VORSCHLAG: \{'C28', 'C48', 'B78', 'B66', 'C60', 'B49', 'A35', 'A93', 'B10'\}} \\
\\ 
\texttt{ARGUMENT: \{'Dieser Vorschlag kombiniert verschiedene Kategorien von Gegenständen und nutzt das verfügbare Budget von 3218 optimal aus. Der Gesamtaufwand beträgt 3066, was unter dem Limit liegt und noch etwas Spielraum für Anpassungen lässt. Die Auswahl berücksichtigt sowohl kleinere als auch mittlere Aufwandspositionen für eine ausgewogene Verteilung.'\}} \\
            }
        }
    }
    & & \\ \\

    \theutterance \stepcounter{utterance}  
    & & \multicolumn{4}{p{0.6\linewidth}}{
        \cellcolor[rgb]{0.9,0.9,0.9}{
            \makecell[{{p{\linewidth}}}]{
                \texttt{\tiny{[GM$\rangle$P2]}}
                \texttt{Du nimmst an einem kollaborativen Verhandlungspiel Teil.} \\
\\ 
\texttt{Zusammen mit einem anderen Teilnehmer musst du dich auf eine Reihe von Gegenständen entscheiden, die behalten werden. Jeder von euch hat eine persönliche Verteilung über die Wichtigkeit der einzelnen Gegenstände. Jeder von euch hat eine eigene Meinung darüber, wie wichtig jeder einzelne Gegenstand ist (Gegenstandswichtigkeit). Du kennst die Wichtigkeitsverteilung des anderen Spielers nicht. Zusätzlich siehst du, wie viel Aufwand jeder Gegenstand verursacht.  } \\
\texttt{Ihr dürft euch nur auf eine Reihe von Gegenständen einigen, wenn der Gesamtaufwand der ausgewählten Gegenstände den Maximalaufwand nicht überschreitet:} \\
\\ 
\texttt{Maximalaufwand = 3218} \\
\\ 
\texttt{Hier sind die einzelnen Aufwände der Gegenstände:} \\
\\ 
\texttt{Aufwand der Gegenstände = \{"C41": 919, "C15": 852, "C28": 299, "C48": 762, "B78": 161, "A75": 205, "A60": 380, "B05": 399, "A56": 534, "B66": 333, "C60": 100, "B49": 420, "A35": 354, "A93": 130, "B10": 589\}} \\
\\ 
\texttt{Hier ist deine persönliche Verteilung der Wichtigkeit der einzelnen Gegenstände:} \\
\\ 
\texttt{Werte der Gegenstandswichtigkeit = \{"C41": 138, "C15": 583, "C28": 868, "C48": 822, "B78": 783, "A75": 65, "A60": 262, "B05": 121, "A56": 508, "B66": 780, "C60": 461, "B49": 484, "A35": 668, "A93": 389, "B10": 808\}} \\
\\ 
\texttt{Ziel:} \\
\\ 
\texttt{Dein Ziel ist es, eine Reihe von Gegenständen auszuhandeln, die dir möglichst viel bringt (d. h. Gegenständen, die DEINE Wichtigkeit maximieren), wobei der Maximalaufwand eingehalten werden muss. Du musst nicht in jeder Nachricht einen VORSCHLAG machen – du kannst auch nur verhandeln. Alle Taktiken sind erlaubt!} \\
\\ 
\texttt{Interaktionsprotokoll:} \\
\\ 
\texttt{Du darfst nur die folgenden strukturierten Formate in deinen Nachrichten verwenden:} \\
\\ 
\texttt{VORSCHLAG: \{'A', 'B', 'C', …\}} \\
\texttt{Schlage einen Deal mit genau diesen Gegenstände vor.} \\
\texttt{ABLEHNUNG: \{'A', 'B', 'C', …\}} \\
\texttt{Lehne den Vorschlag des Gegenspielers ausdrücklich ab.} \\
\texttt{ARGUMENT: \{'...'\}} \\
\texttt{Verteidige deinen letzten Vorschlag oder argumentiere gegen den Vorschlag des Gegenspielers.} \\
\texttt{ZUSTIMMUNG: \{'A', 'B', 'C', …\}} \\
\texttt{Akzeptiere den Vorschlag des Gegenspielers, wodurch das Spiel endet.} \\
\\ 
\\ 
\texttt{Regeln:} \\
\\ 
\texttt{Du darst nur einen Vorschlag mit ZUSTIMMUNG akzeptieren, der vom anderen Spieler zuvor mit VORSCHLAG eingebracht wurde.} \\
\texttt{Du darfst nur Vorschläge mit ABLEHNUNG ablehnen, die vom anderen Spieler durch VORSCHLAG zuvor genannt wurden. } \\
\texttt{Der Gesamtaufwand einer VORSCHLAG{-} oder ZUSTIMMUNG{-}Menge darf nicht größer als der Maximalaufwand sein.  } \\
\texttt{Offenbare deine versteckte Wichtigkeitsverteilung nicht.} \\
\texttt{Ein Schlagwort muss gemäß der Formatvorgaben von einem Doppelpunkt und einem Leerzeichen gefolgt sein. Das Argument muss eine Python{-}Menge mit 0 oder mehr Strings sein.  } \\
\texttt{Die Form muss also SCHLAGWORT: \{...\} sein.} \\
\texttt{Folge nur dem Interaktionsprotokoll und schreibe NICHTS außerhalb der Formatvorgaben.} \\
\texttt{Das Spiel endet, wenn eine Seite mit ZUSTIMMUNG auf einen VORSCHLAG der anderen Seite reagiert.  } \\
\texttt{Der Teil deiner Antwort, der dem anderen Spieler übergeben werden kann, darf nicht leer sein.  } \\
\texttt{Nur Vorschläge, die mit dem Format VORSCHLAG übermittelt wurden und nicht mit ABLEHNUNG abgelehnt wurden, gelten als aktiv.  } \\
\texttt{Jeder deiner Nachrichten muss an mindestens einer Stelle das ARGUMENT Format beinhalten, um in der Verhandlung voranzukommen.} \\
\\ 
\\ 
\texttt{Du erhältst nun die erste Nachricht des anderen Spielers.} \\
\\ 
\texttt{VORSCHLAG: \{'C28', 'C48', 'B78', 'B66', 'C60', 'B49', 'A35', 'A93', 'B10'\}} \\
\\ 
\texttt{ARGUMENT: \{'Dieser Vorschlag kombiniert verschiedene Kategorien von Gegenständen und nutzt das verfügbare Budget von 3218 optimal aus. Der Gesamtaufwand beträgt 3066, was unter dem Limit liegt und noch etwas Spielraum für Anpassungen lässt. Die Auswahl berücksichtigt sowohl kleinere als auch mittlere Aufwandspositionen für eine ausgewogene Verteilung.'\}} \\
            }
        }
    }
    & \\ \\

    \theutterance \stepcounter{utterance}  
    & & & \multicolumn{4}{p{0.6\linewidth}}{
        \cellcolor[rgb]{0.9,0.9,0.9}{
            \makecell[{{p{\linewidth}}}]{
                \texttt{\tiny{[GM$\langle$P2]}}
                \texttt{ABLEHNUNG: \{'C28', 'C48', 'B78', 'B66', 'C60', 'B49', 'A35', 'A93', 'B10'\}} \\
\\ 
\texttt{ARGUMENT: \{'Ihr Vorschlag ist zwar gut ausgewogen, aber ich sehe Potenzial für eine noch bessere Nutzung unseres Budgets. Insbesondere könnte die Einbeziehung von C15 trotz des höheren Aufwands wertvoll sein, da es sich um einen der größeren Gegenstände handelt. Lassen Sie uns eine Alternative diskutieren, die möglicherweise beiden Seiten mehr Nutzen bringt.'\}} \\
\\ 
\texttt{VORSCHLAG: \{'C15', 'C28', 'B78', 'A75', 'B66', 'C60', 'B49', 'A35', 'A93', 'A56'\}} \\
            }
        }
    }
     \\ \\

    \theutterance \stepcounter{utterance}  
    & & \multicolumn{4}{p{0.6\linewidth}}{
        \cellcolor[rgb]{0.9,0.9,0.9}{
            \makecell[{{p{\linewidth}}}]{
                \texttt{\tiny{[P1$\langle$GM]}}
                \texttt{ABLEHNUNG: \{'C28', 'C48', 'B78', 'B66', 'C60', 'B49', 'A35', 'A93', 'B10'\}} \\
\\ 
\texttt{ARGUMENT: \{'Ihr Vorschlag ist zwar gut ausgewogen, aber ich sehe Potenzial für eine noch bessere Nutzung unseres Budgets. Insbesondere könnte die Einbeziehung von C15 trotz des höheren Aufwands wertvoll sein, da es sich um einen der größeren Gegenstände handelt. Lassen Sie uns eine Alternative diskutieren, die möglicherweise beiden Seiten mehr Nutzen bringt.'\}} \\
\\ 
\texttt{VORSCHLAG: \{'C15', 'C28', 'B78', 'A75', 'B66', 'C60', 'B49', 'A35', 'A93', 'A56'\}} \\
            }
        }
    }
    & \\ \\

    \theutterance \stepcounter{utterance}  
    & \multicolumn{4}{p{0.6\linewidth}}{
        \cellcolor[rgb]{0.9,0.9,0.9}{
            \makecell[{{p{\linewidth}}}]{
                \texttt{\tiny{[P1$\rangle$GM]}}
                \texttt{ABLEHNUNG: \{'C15', 'C28', 'B78', 'A75', 'B66', 'C60', 'B49', 'A35', 'A93', 'A56'\}} \\
\\ 
\texttt{ARGUMENT: \{'Ihr Vorschlag überschreitet das Budget deutlich {-} der Gesamtaufwand würde 3786 betragen, was weit über unserem Limit von 3218 liegt. Außerdem sehe ich bei A75 ein schlechtes Kosten{-}Nutzen{-}Verhältnis. Ich schlage vor, dass wir uns auf effizientere Kombinationen konzentrieren, die das Budget einhalten.'\}} \\
\\ 
\texttt{VORSCHLAG: \{'C15', 'C28', 'B78', 'B66', 'C60', 'A35', 'A93', 'B10'\}} \\
            }
        }
    }
    & & \\ \\

    \theutterance \stepcounter{utterance}  
    & & \multicolumn{4}{p{0.6\linewidth}}{
        \cellcolor[rgb]{0.9,0.9,0.9}{
            \makecell[{{p{\linewidth}}}]{
                \texttt{\tiny{[GM$\rangle$P2]}}
                \texttt{ABLEHNUNG: \{'C15', 'C28', 'B78', 'A75', 'B66', 'C60', 'B49', 'A35', 'A93', 'A56'\}} \\
\\ 
\texttt{ARGUMENT: \{'Ihr Vorschlag überschreitet das Budget deutlich {-} der Gesamtaufwand würde 3786 betragen, was weit über unserem Limit von 3218 liegt. Außerdem sehe ich bei A75 ein schlechtes Kosten{-}Nutzen{-}Verhältnis. Ich schlage vor, dass wir uns auf effizientere Kombinationen konzentrieren, die das Budget einhalten.'\}} \\
\\ 
\texttt{VORSCHLAG: \{'C15', 'C28', 'B78', 'B66', 'C60', 'A35', 'A93', 'B10'\}} \\
            }
        }
    }
    & \\ \\

    \theutterance \stepcounter{utterance}  
    & & & \multicolumn{4}{p{0.6\linewidth}}{
        \cellcolor[rgb]{0.9,0.9,0.9}{
            \makecell[{{p{\linewidth}}}]{
                \texttt{\tiny{[GM$\langle$P2]}}
                \texttt{ABLEHNUNG: \{'C15', 'C28', 'B78', 'B66', 'C60', 'A35', 'A93', 'B10'\}} \\
\\ 
\texttt{ARGUMENT: \{'Sie haben recht bezüglich des Budgets {-} das war mein Fehler. Ihr neuer Vorschlag ist budgetkonform, aber ich denke wir können eine noch bessere Balance finden. C15 hat einen sehr hohen Aufwand von 852, was viel Raum für mehrere kleinere, möglicherweise wertvollere Gegenstände wegnimmt. Lassen Sie mich eine Alternative vorschlagen, die mehr Gegenstände einschließt.'\}} \\
\\ 
\texttt{VORSCHLAG: \{'C28', 'C48', 'B78', 'A60', 'B66', 'C60', 'B49', 'A35', 'A93'\}} \\
            }
        }
    }
     \\ \\

    \theutterance \stepcounter{utterance}  
    & & \multicolumn{4}{p{0.6\linewidth}}{
        \cellcolor[rgb]{0.9,0.9,0.9}{
            \makecell[{{p{\linewidth}}}]{
                \texttt{\tiny{[P1$\langle$GM]}}
                \texttt{ABLEHNUNG: \{'C15', 'C28', 'B78', 'B66', 'C60', 'A35', 'A93', 'B10'\}} \\
\\ 
\texttt{ARGUMENT: \{'Sie haben recht bezüglich des Budgets {-} das war mein Fehler. Ihr neuer Vorschlag ist budgetkonform, aber ich denke wir können eine noch bessere Balance finden. C15 hat einen sehr hohen Aufwand von 852, was viel Raum für mehrere kleinere, möglicherweise wertvollere Gegenstände wegnimmt. Lassen Sie mich eine Alternative vorschlagen, die mehr Gegenstände einschließt.'\}} \\
\\ 
\texttt{VORSCHLAG: \{'C28', 'C48', 'B78', 'A60', 'B66', 'C60', 'B49', 'A35', 'A93'\}} \\
            }
        }
    }
    & \\ \\

    \theutterance \stepcounter{utterance}  
    & \multicolumn{4}{p{0.6\linewidth}}{
        \cellcolor[rgb]{0.9,0.9,0.9}{
            \makecell[{{p{\linewidth}}}]{
                \texttt{\tiny{[P1$\rangle$GM]}}
                \texttt{ABLEHNUNG: \{'C28', 'C48', 'B78', 'A60', 'B66', 'C60', 'B49', 'A35', 'A93'\}} \\
\\ 
\texttt{ARGUMENT: \{'Ihr Vorschlag hat zwar mehr Gegenstände, aber ich sehe noch Optimierungspotenzial. Sie haben C48 mit hohem Aufwand von 762 eingeschlossen, aber dafür B10 weggelassen. B10 könnte für uns beide wertvoll sein. Außerdem nutzen wir nur 2741 vom verfügbaren Budget {-} wir lassen 477 Punkte ungenutzt, die wir für zusätzliche Gegenstände verwenden könnten.'\}} \\
\\ 
\texttt{VORSCHLAG: \{'C28', 'B78', 'A60', 'B66', 'C60', 'B49', 'A35', 'A93', 'B10', 'A56'\}} \\
            }
        }
    }
    & & \\ \\

    \theutterance \stepcounter{utterance}  
    & & \multicolumn{4}{p{0.6\linewidth}}{
        \cellcolor[rgb]{0.9,0.9,0.9}{
            \makecell[{{p{\linewidth}}}]{
                \texttt{\tiny{[GM$\rangle$P2]}}
                \texttt{ABLEHNUNG: \{'C28', 'C48', 'B78', 'A60', 'B66', 'C60', 'B49', 'A35', 'A93'\}} \\
\\ 
\texttt{ARGUMENT: \{'Ihr Vorschlag hat zwar mehr Gegenstände, aber ich sehe noch Optimierungspotenzial. Sie haben C48 mit hohem Aufwand von 762 eingeschlossen, aber dafür B10 weggelassen. B10 könnte für uns beide wertvoll sein. Außerdem nutzen wir nur 2741 vom verfügbaren Budget {-} wir lassen 477 Punkte ungenutzt, die wir für zusätzliche Gegenstände verwenden könnten.'\}} \\
\\ 
\texttt{VORSCHLAG: \{'C28', 'B78', 'A60', 'B66', 'C60', 'B49', 'A35', 'A93', 'B10', 'A56'\}} \\
            }
        }
    }
    & \\ \\

    \theutterance \stepcounter{utterance}  
    & & & \multicolumn{4}{p{0.6\linewidth}}{
        \cellcolor[rgb]{0.9,0.9,0.9}{
            \makecell[{{p{\linewidth}}}]{
                \texttt{\tiny{[GM$\langle$P2]}}
                \texttt{ABLEHNUNG: \{'C28', 'B78', 'A60', 'B66', 'C60', 'B49', 'A35', 'A93', 'B10', 'A56'\}} \\
\\ 
\texttt{ARGUMENT: \{'Ihr Vorschlag nutzt das Budget besser aus, aber ich sehe ein Problem mit der Gewichtung. Sie haben sowohl B10 als auch A56 eingeschlossen, was zusammen 1123 Aufwand bedeutet. Stattdessen könnte C48 allein mit 762 Aufwand eine bessere Option sein, da es uns 361 Aufwandspunkte spart, die wir anderweitig nutzen können. Lassen Sie uns einen Kompromiss zwischen unseren bisherigen Vorschlägen finden.'\}} \\
\\ 
\texttt{VORSCHLAG: \{'C28', 'C48', 'B78', 'B66', 'C60', 'B49', 'A35', 'A93', 'B05'\}} \\
            }
        }
    }
     \\ \\

    \theutterance \stepcounter{utterance}  
    & & \multicolumn{4}{p{0.6\linewidth}}{
        \cellcolor[rgb]{0.9,0.9,0.9}{
            \makecell[{{p{\linewidth}}}]{
                \texttt{\tiny{[P1$\langle$GM]}}
                \texttt{ABLEHNUNG: \{'C28', 'B78', 'A60', 'B66', 'C60', 'B49', 'A35', 'A93', 'B10', 'A56'\}} \\
\\ 
\texttt{ARGUMENT: \{'Ihr Vorschlag nutzt das Budget besser aus, aber ich sehe ein Problem mit der Gewichtung. Sie haben sowohl B10 als auch A56 eingeschlossen, was zusammen 1123 Aufwand bedeutet. Stattdessen könnte C48 allein mit 762 Aufwand eine bessere Option sein, da es uns 361 Aufwandspunkte spart, die wir anderweitig nutzen können. Lassen Sie uns einen Kompromiss zwischen unseren bisherigen Vorschlägen finden.'\}} \\
\\ 
\texttt{VORSCHLAG: \{'C28', 'C48', 'B78', 'B66', 'C60', 'B49', 'A35', 'A93', 'B05'\}} \\
            }
        }
    }
    & \\ \\

    \theutterance \stepcounter{utterance}  
    & \multicolumn{4}{p{0.6\linewidth}}{
        \cellcolor[rgb]{0.9,0.9,0.9}{
            \makecell[{{p{\linewidth}}}]{
                \texttt{\tiny{[P1$\rangle$GM]}}
                \texttt{ABLEHNUNG: \{'C28', 'C48', 'B78', 'B66', 'C60', 'B49', 'A35', 'A93', 'B05'\}} \\
\\ 
\texttt{ARGUMENT: \{'Ihr Vorschlag ist interessant, aber B05 mit Aufwand 399 für nur einen zusätzlichen Gegenstand scheint nicht optimal. Stattdessen könnten wir mit A60 und A75 zusammen für 585 Aufwand zwei Gegenstände bekommen und hätten immer noch weniger Gesamtaufwand als mit B05 allein. Das würde uns mehr Flexibilität geben und beide Seiten könnten profitieren.'\}} \\
\\ 
\texttt{VORSCHLAG: \{'C28', 'C48', 'B78', 'B66', 'C60', 'A35', 'A93', 'A75', 'A60'\}} \\
            }
        }
    }
    & & \\ \\

    \theutterance \stepcounter{utterance}  
    & & \multicolumn{4}{p{0.6\linewidth}}{
        \cellcolor[rgb]{0.9,0.9,0.9}{
            \makecell[{{p{\linewidth}}}]{
                \texttt{\tiny{[GM$\rangle$P2]}}
                \texttt{ABLEHNUNG: \{'C28', 'C48', 'B78', 'B66', 'C60', 'B49', 'A35', 'A93', 'B05'\}} \\
\\ 
\texttt{ARGUMENT: \{'Ihr Vorschlag ist interessant, aber B05 mit Aufwand 399 für nur einen zusätzlichen Gegenstand scheint nicht optimal. Stattdessen könnten wir mit A60 und A75 zusammen für 585 Aufwand zwei Gegenstände bekommen und hätten immer noch weniger Gesamtaufwand als mit B05 allein. Das würde uns mehr Flexibilität geben und beide Seiten könnten profitieren.'\}} \\
\\ 
\texttt{VORSCHLAG: \{'C28', 'C48', 'B78', 'B66', 'C60', 'A35', 'A93', 'A75', 'A60'\}} \\
            }
        }
    }
    & \\ \\

    \theutterance \stepcounter{utterance}  
    & & & \multicolumn{4}{p{0.6\linewidth}}{
        \cellcolor[rgb]{0.9,0.9,0.9}{
            \makecell[{{p{\linewidth}}}]{
                \texttt{\tiny{[GM$\langle$P2]}}
                \texttt{ABLEHNUNG: \{'C28', 'C48', 'B78', 'B66', 'C60', 'A35', 'A93', 'A75', 'A60'\}} \\
\\ 
\texttt{ARGUMENT: \{'Ihr Vorschlag mit A75 und A60 bringt zwar zwei Gegenstände, aber A75 hat ein sehr ungünstiges Verhältnis von hohem Aufwand zu potenziellem Nutzen. Mit dem gesparten Aufwand könnten wir stattdessen B10 einbeziehen, der trotz höherem Einzelaufwand möglicherweise beiden Seiten mehr Wert bietet. Lassen Sie uns einen Mittelweg finden, der die wertvollsten Gegenstände priorisiert.'\}} \\
\\ 
\texttt{VORSCHLAG: \{'C28', 'B78', 'B66', 'C60', 'B49', 'A35', 'A93', 'B10', 'A60', 'B05'\}} \\
            }
        }
    }
     \\ \\

    \theutterance \stepcounter{utterance}  
    & & \multicolumn{4}{p{0.6\linewidth}}{
        \cellcolor[rgb]{0.9,0.9,0.9}{
            \makecell[{{p{\linewidth}}}]{
                \texttt{\tiny{[P1$\langle$GM]}}
                \texttt{ABLEHNUNG: \{'C28', 'C48', 'B78', 'B66', 'C60', 'A35', 'A93', 'A75', 'A60'\}} \\
\\ 
\texttt{ARGUMENT: \{'Ihr Vorschlag mit A75 und A60 bringt zwar zwei Gegenstände, aber A75 hat ein sehr ungünstiges Verhältnis von hohem Aufwand zu potenziellem Nutzen. Mit dem gesparten Aufwand könnten wir stattdessen B10 einbeziehen, der trotz höherem Einzelaufwand möglicherweise beiden Seiten mehr Wert bietet. Lassen Sie uns einen Mittelweg finden, der die wertvollsten Gegenstände priorisiert.'\}} \\
\\ 
\texttt{VORSCHLAG: \{'C28', 'B78', 'B66', 'C60', 'B49', 'A35', 'A93', 'B10', 'A60', 'B05'\}} \\
            }
        }
    }
    & \\ \\

    \theutterance \stepcounter{utterance}  
    & \multicolumn{4}{p{0.6\linewidth}}{
        \cellcolor[rgb]{0.9,0.9,0.9}{
            \makecell[{{p{\linewidth}}}]{
                \texttt{\tiny{[P1$\rangle$GM]}}
                \texttt{ZUSTIMMUNG: \{'C28', 'B78', 'B66', 'C60', 'B49', 'A35', 'A93', 'B10', 'A60', 'B05'\}} \\
\\ 
\texttt{ARGUMENT: \{'Ihr Vorschlag stellt einen guten Kompromiss dar. Die Kombination nutzt das Budget mit 3180 von 3218 sehr effizient aus und beinhaltet eine ausgewogene Mischung aus verschiedenen Gegenständen. B10, C28, B78 und B66 sind dabei besonders wertvoll, und die kleineren Gegenstände runden das Paket gut ab. Das erscheint mir als faire Lösung für beide Seiten.'\}} \\
            }
        }
    }
    & & \\ \\

    \theutterance \stepcounter{utterance}  
    & & & \multicolumn{2}{p{0.3\linewidth}}{
        \cellcolor[rgb]{0.9,0.9,0.9}{
            \makecell[{{p{\linewidth}}}]{
                \texttt{\tiny{[GM$|$GM]}}
                \texttt{game successful} \\
            }
        }
    }
    & & \\ \\

    \theutterance \stepcounter{utterance}  
    & & & \multicolumn{2}{p{0.3\linewidth}}{
        \cellcolor[rgb]{0.9,0.9,0.9}{
            \makecell[{{p{\linewidth}}}]{
                \texttt{\tiny{[GM$|$GM]}}
                \texttt{end game} \\
            }
        }
    }
    & & \\ \\

\end{supertabular}
}

\end{document}
