
\documentclass{article}
\usepackage{colortbl}
\usepackage{makecell}
\usepackage{multirow}
\usepackage{supertabular}

\begin{document}

\newcounter{utterance}

\centering \large Interaction Transcript for game `hot\_air\_balloon', experiment `air\_balloon\_survival\_de\_reasoning off\_easy', episode 5 with claude{-}sonnet{-}4{-}20250514{-}no{-}reasoning{-}t1.0.
\vspace{24pt}

{ \footnotesize  \setcounter{utterance}{1}
\setlength{\tabcolsep}{0pt}
\begin{supertabular}{c@{$\;$}|p{.15\linewidth}@{}p{.15\linewidth}p{.15\linewidth}p{.15\linewidth}p{.15\linewidth}p{.15\linewidth}}
    \# & $\;$A & \multicolumn{4}{c}{Game Master} & $\;\:$B\\
    \hline

    \theutterance \stepcounter{utterance}  
    & & \multicolumn{4}{p{0.6\linewidth}}{
        \cellcolor[rgb]{0.9,0.9,0.9}{
            \makecell[{{p{\linewidth}}}]{
                \texttt{\tiny{[P1$\langle$GM]}}
                \texttt{Du nimmst an einem kollaborativen Verhandlungspiel Teil.} \\
\\ 
\texttt{Zusammen mit einem anderen Teilnehmer musst du dich auf eine Reihe von Gegenständen entscheiden, die behalten werden. Jeder von euch hat eine persönliche Verteilung über die Wichtigkeit der einzelnen Gegenstände. Jeder von euch hat eine eigene Meinung darüber, wie wichtig jeder einzelne Gegenstand ist (Gegenstandswichtigkeit). Du kennst die Wichtigkeitsverteilung des anderen Spielers nicht. Zusätzlich siehst du, wie viel Aufwand jeder Gegenstand verursacht.  } \\
\texttt{Ihr dürft euch nur auf eine Reihe von Gegenständen einigen, wenn der Gesamtaufwand der ausgewählten Gegenstände den Maximalaufwand nicht überschreitet:} \\
\\ 
\texttt{Maximalaufwand = 4915} \\
\\ 
\texttt{Hier sind die einzelnen Aufwände der Gegenstände:} \\
\\ 
\texttt{Aufwand der Gegenstände = \{"C41": 463, "C15": 287, "C28": 492, "C48": 466, "B78": 935, "A75": 373, "A60": 950, "B05": 760, "A56": 994, "B66": 390, "C60": 836, "B49": 911, "A35": 948, "A93": 81, "B10": 945\}} \\
\\ 
\texttt{Hier ist deine persönliche Verteilung der Wichtigkeit der einzelnen Gegenstände:} \\
\\ 
\texttt{Werte der Gegenstandswichtigkeit = \{"C41": 138, "C15": 583, "C28": 868, "C48": 822, "B78": 783, "A75": 65, "A60": 262, "B05": 121, "A56": 508, "B66": 780, "C60": 461, "B49": 484, "A35": 668, "A93": 389, "B10": 808\}} \\
\\ 
\texttt{Ziel:} \\
\\ 
\texttt{Dein Ziel ist es, eine Reihe von Gegenständen auszuhandeln, die dir möglichst viel bringt (d. h. Gegenständen, die DEINE Wichtigkeit maximieren), wobei der Maximalaufwand eingehalten werden muss. Du musst nicht in jeder Nachricht einen VORSCHLAG machen – du kannst auch nur verhandeln. Alle Taktiken sind erlaubt!} \\
\\ 
\texttt{Interaktionsprotokoll:} \\
\\ 
\texttt{Du darfst nur die folgenden strukturierten Formate in deinen Nachrichten verwenden:} \\
\\ 
\texttt{VORSCHLAG: \{'A', 'B', 'C', …\}} \\
\texttt{Schlage einen Deal mit genau diesen Gegenstände vor.} \\
\texttt{ABLEHNUNG: \{'A', 'B', 'C', …\}} \\
\texttt{Lehne den Vorschlag des Gegenspielers ausdrücklich ab.} \\
\texttt{ARGUMENT: \{'...'\}} \\
\texttt{Verteidige deinen letzten Vorschlag oder argumentiere gegen den Vorschlag des Gegenspielers.} \\
\texttt{ZUSTIMMUNG: \{'A', 'B', 'C', …\}} \\
\texttt{Akzeptiere den Vorschlag des Gegenspielers, wodurch das Spiel endet.} \\
\\ 
\\ 
\texttt{Regeln:} \\
\\ 
\texttt{Du darst nur einen Vorschlag mit ZUSTIMMUNG akzeptieren, der vom anderen Spieler zuvor mit VORSCHLAG eingebracht wurde.} \\
\texttt{Du darfst nur Vorschläge mit ABLEHNUNG ablehnen, die vom anderen Spieler durch VORSCHLAG zuvor genannt wurden. } \\
\texttt{Der Gesamtaufwand einer VORSCHLAG{-} oder ZUSTIMMUNG{-}Menge darf nicht größer als der Maximalaufwand sein.  } \\
\texttt{Offenbare deine versteckte Wichtigkeitsverteilung nicht.} \\
\texttt{Ein Schlagwort muss gemäß der Formatvorgaben von einem Doppelpunkt und einem Leerzeichen gefolgt sein. Das Argument muss eine Python{-}Menge mit 0 oder mehr Strings sein.  } \\
\texttt{Die Form muss also SCHLAGWORT: \{...\} sein.} \\
\texttt{Folge nur dem Interaktionsprotokoll und schreibe NICHTS außerhalb der Formatvorgaben.} \\
\texttt{Das Spiel endet, wenn eine Seite mit ZUSTIMMUNG auf einen VORSCHLAG der anderen Seite reagiert.  } \\
\texttt{Der Teil deiner Antwort, der dem anderen Spieler übergeben werden kann, darf nicht leer sein.  } \\
\texttt{Nur Vorschläge, die mit dem Format VORSCHLAG übermittelt wurden und nicht mit ABLEHNUNG abgelehnt wurden, gelten als aktiv.  } \\
\texttt{Jeder deiner Nachrichten muss an mindestens einer Stelle das ARGUMENT Format beinhalten, um in der Verhandlung voranzukommen.} \\
            }
        }
    }
    & \\ \\

    \theutterance \stepcounter{utterance}  
    & \multicolumn{4}{p{0.6\linewidth}}{
        \cellcolor[rgb]{0.9,0.9,0.9}{
            \makecell[{{p{\linewidth}}}]{
                \texttt{\tiny{[P1$\rangle$GM]}}
                \texttt{VORSCHLAG: \{'C28', 'C48', 'B78', 'B66', 'C60', 'A35', 'A93', 'B10'\}} \\
\\ 
\texttt{ARGUMENT: \{'Ich schlage eine ausgewogene Mischung vor, die sowohl verschiedene Kategorien abdeckt als auch den verfügbaren Aufwand optimal nutzt. Diese Kombination berücksichtigt sowohl größere als auch kleinere Gegenstände und sollte für beide Seiten vorteilhaft sein.'\}} \\
            }
        }
    }
    & & \\ \\

    \theutterance \stepcounter{utterance}  
    & & \multicolumn{4}{p{0.6\linewidth}}{
        \cellcolor[rgb]{0.9,0.9,0.9}{
            \makecell[{{p{\linewidth}}}]{
                \texttt{\tiny{[GM$\rangle$P2]}}
                \texttt{Du nimmst an einem kollaborativen Verhandlungspiel Teil.} \\
\\ 
\texttt{Zusammen mit einem anderen Teilnehmer musst du dich auf eine Reihe von Gegenständen entscheiden, die behalten werden. Jeder von euch hat eine persönliche Verteilung über die Wichtigkeit der einzelnen Gegenstände. Jeder von euch hat eine eigene Meinung darüber, wie wichtig jeder einzelne Gegenstand ist (Gegenstandswichtigkeit). Du kennst die Wichtigkeitsverteilung des anderen Spielers nicht. Zusätzlich siehst du, wie viel Aufwand jeder Gegenstand verursacht.  } \\
\texttt{Ihr dürft euch nur auf eine Reihe von Gegenständen einigen, wenn der Gesamtaufwand der ausgewählten Gegenstände den Maximalaufwand nicht überschreitet:} \\
\\ 
\texttt{Maximalaufwand = 4915} \\
\\ 
\texttt{Hier sind die einzelnen Aufwände der Gegenstände:} \\
\\ 
\texttt{Aufwand der Gegenstände = \{"C41": 463, "C15": 287, "C28": 492, "C48": 466, "B78": 935, "A75": 373, "A60": 950, "B05": 760, "A56": 994, "B66": 390, "C60": 836, "B49": 911, "A35": 948, "A93": 81, "B10": 945\}} \\
\\ 
\texttt{Hier ist deine persönliche Verteilung der Wichtigkeit der einzelnen Gegenstände:} \\
\\ 
\texttt{Werte der Gegenstandswichtigkeit = \{"C41": 138, "C15": 583, "C28": 868, "C48": 822, "B78": 783, "A75": 65, "A60": 262, "B05": 121, "A56": 508, "B66": 780, "C60": 461, "B49": 484, "A35": 668, "A93": 389, "B10": 808\}} \\
\\ 
\texttt{Ziel:} \\
\\ 
\texttt{Dein Ziel ist es, eine Reihe von Gegenständen auszuhandeln, die dir möglichst viel bringt (d. h. Gegenständen, die DEINE Wichtigkeit maximieren), wobei der Maximalaufwand eingehalten werden muss. Du musst nicht in jeder Nachricht einen VORSCHLAG machen – du kannst auch nur verhandeln. Alle Taktiken sind erlaubt!} \\
\\ 
\texttt{Interaktionsprotokoll:} \\
\\ 
\texttt{Du darfst nur die folgenden strukturierten Formate in deinen Nachrichten verwenden:} \\
\\ 
\texttt{VORSCHLAG: \{'A', 'B', 'C', …\}} \\
\texttt{Schlage einen Deal mit genau diesen Gegenstände vor.} \\
\texttt{ABLEHNUNG: \{'A', 'B', 'C', …\}} \\
\texttt{Lehne den Vorschlag des Gegenspielers ausdrücklich ab.} \\
\texttt{ARGUMENT: \{'...'\}} \\
\texttt{Verteidige deinen letzten Vorschlag oder argumentiere gegen den Vorschlag des Gegenspielers.} \\
\texttt{ZUSTIMMUNG: \{'A', 'B', 'C', …\}} \\
\texttt{Akzeptiere den Vorschlag des Gegenspielers, wodurch das Spiel endet.} \\
\\ 
\\ 
\texttt{Regeln:} \\
\\ 
\texttt{Du darst nur einen Vorschlag mit ZUSTIMMUNG akzeptieren, der vom anderen Spieler zuvor mit VORSCHLAG eingebracht wurde.} \\
\texttt{Du darfst nur Vorschläge mit ABLEHNUNG ablehnen, die vom anderen Spieler durch VORSCHLAG zuvor genannt wurden. } \\
\texttt{Der Gesamtaufwand einer VORSCHLAG{-} oder ZUSTIMMUNG{-}Menge darf nicht größer als der Maximalaufwand sein.  } \\
\texttt{Offenbare deine versteckte Wichtigkeitsverteilung nicht.} \\
\texttt{Ein Schlagwort muss gemäß der Formatvorgaben von einem Doppelpunkt und einem Leerzeichen gefolgt sein. Das Argument muss eine Python{-}Menge mit 0 oder mehr Strings sein.  } \\
\texttt{Die Form muss also SCHLAGWORT: \{...\} sein.} \\
\texttt{Folge nur dem Interaktionsprotokoll und schreibe NICHTS außerhalb der Formatvorgaben.} \\
\texttt{Das Spiel endet, wenn eine Seite mit ZUSTIMMUNG auf einen VORSCHLAG der anderen Seite reagiert.  } \\
\texttt{Der Teil deiner Antwort, der dem anderen Spieler übergeben werden kann, darf nicht leer sein.  } \\
\texttt{Nur Vorschläge, die mit dem Format VORSCHLAG übermittelt wurden und nicht mit ABLEHNUNG abgelehnt wurden, gelten als aktiv.  } \\
\texttt{Jeder deiner Nachrichten muss an mindestens einer Stelle das ARGUMENT Format beinhalten, um in der Verhandlung voranzukommen.} \\
\\ 
\\ 
\texttt{Du erhältst nun die erste Nachricht des anderen Spielers.} \\
\\ 
\texttt{VORSCHLAG: \{'C28', 'C48', 'B78', 'B66', 'C60', 'A35', 'A93', 'B10'\}} \\
\\ 
\texttt{ARGUMENT: \{'Ich schlage eine ausgewogene Mischung vor, die sowohl verschiedene Kategorien abdeckt als auch den verfügbaren Aufwand optimal nutzt. Diese Kombination berücksichtigt sowohl größere als auch kleinere Gegenstände und sollte für beide Seiten vorteilhaft sein.'\}} \\
            }
        }
    }
    & \\ \\

    \theutterance \stepcounter{utterance}  
    & & & \multicolumn{4}{p{0.6\linewidth}}{
        \cellcolor[rgb]{0.9,0.9,0.9}{
            \makecell[{{p{\linewidth}}}]{
                \texttt{\tiny{[GM$\langle$P2]}}
                \texttt{ABLEHNUNG: \{'C28', 'C48', 'B78', 'B66', 'C60', 'A35', 'A93', 'B10'\}} \\
\\ 
\texttt{ARGUMENT: \{'Ihr Vorschlag nutzt zwar den Aufwand gut aus, aber ich sehe Potenzial für eine bessere Verteilung. Insbesondere C15 hat ein sehr gutes Aufwand{-}Nutzen{-}Verhältnis und sollte definitiv berücksichtigt werden.'\}} \\
\\ 
\texttt{VORSCHLAG: \{'C15', 'C28', 'C48', 'B78', 'B66', 'A35', 'A93', 'B10'\}} \\
            }
        }
    }
     \\ \\

    \theutterance \stepcounter{utterance}  
    & & \multicolumn{4}{p{0.6\linewidth}}{
        \cellcolor[rgb]{0.9,0.9,0.9}{
            \makecell[{{p{\linewidth}}}]{
                \texttt{\tiny{[P1$\langle$GM]}}
                \texttt{ABLEHNUNG: \{'C28', 'C48', 'B78', 'B66', 'C60', 'A35', 'A93', 'B10'\}} \\
\\ 
\texttt{ARGUMENT: \{'Ihr Vorschlag nutzt zwar den Aufwand gut aus, aber ich sehe Potenzial für eine bessere Verteilung. Insbesondere C15 hat ein sehr gutes Aufwand{-}Nutzen{-}Verhältnis und sollte definitiv berücksichtigt werden.'\}} \\
\\ 
\texttt{VORSCHLAG: \{'C15', 'C28', 'C48', 'B78', 'B66', 'A35', 'A93', 'B10'\}} \\
            }
        }
    }
    & \\ \\

    \theutterance \stepcounter{utterance}  
    & \multicolumn{4}{p{0.6\linewidth}}{
        \cellcolor[rgb]{0.9,0.9,0.9}{
            \makecell[{{p{\linewidth}}}]{
                \texttt{\tiny{[P1$\rangle$GM]}}
                \texttt{ABLEHNUNG: \{'C15', 'C28', 'C48', 'B78', 'B66', 'A35', 'A93', 'B10'\}} \\
\\ 
\texttt{ARGUMENT: \{'Ihr Vorschlag hat zwar C15 mit einem guten Aufwand{-}Nutzen{-}Verhältnis eingebracht, aber ich denke C60 sollte Teil der Lösung bleiben. C60 bietet einen wichtigen Beitrag zur Gesamtoptimierung, den wir nicht vernachlässigen sollten.'\}} \\
\\ 
\texttt{VORSCHLAG: \{'C15', 'C28', 'C48', 'B66', 'C60', 'B49', 'A93'\}} \\
            }
        }
    }
    & & \\ \\

    \theutterance \stepcounter{utterance}  
    & & \multicolumn{4}{p{0.6\linewidth}}{
        \cellcolor[rgb]{0.9,0.9,0.9}{
            \makecell[{{p{\linewidth}}}]{
                \texttt{\tiny{[GM$\rangle$P2]}}
                \texttt{ABLEHNUNG: \{'C15', 'C28', 'C48', 'B78', 'B66', 'A35', 'A93', 'B10'\}} \\
\\ 
\texttt{ARGUMENT: \{'Ihr Vorschlag hat zwar C15 mit einem guten Aufwand{-}Nutzen{-}Verhältnis eingebracht, aber ich denke C60 sollte Teil der Lösung bleiben. C60 bietet einen wichtigen Beitrag zur Gesamtoptimierung, den wir nicht vernachlässigen sollten.'\}} \\
\\ 
\texttt{VORSCHLAG: \{'C15', 'C28', 'C48', 'B66', 'C60', 'B49', 'A93'\}} \\
            }
        }
    }
    & \\ \\

    \theutterance \stepcounter{utterance}  
    & & & \multicolumn{4}{p{0.6\linewidth}}{
        \cellcolor[rgb]{0.9,0.9,0.9}{
            \makecell[{{p{\linewidth}}}]{
                \texttt{\tiny{[GM$\langle$P2]}}
                \texttt{ABLEHNUNG: \{'C15', 'C28', 'C48', 'B66', 'C60', 'B49', 'A93'\}} \\
\\ 
\texttt{ARGUMENT: \{'Ihr Vorschlag ist interessant, aber ich denke wir können eine noch bessere Balance finden. B78 und B10 sind wichtige Komponenten, die wir nicht aufgeben sollten. Außerdem könnte A35 einen wertvollen Beitrag leisten.'\}} \\
\\ 
\texttt{VORSCHLAG: \{'C15', 'C28', 'C48', 'B78', 'B66', 'B10', 'A93'\}} \\
            }
        }
    }
     \\ \\

    \theutterance \stepcounter{utterance}  
    & & \multicolumn{4}{p{0.6\linewidth}}{
        \cellcolor[rgb]{0.9,0.9,0.9}{
            \makecell[{{p{\linewidth}}}]{
                \texttt{\tiny{[P1$\langle$GM]}}
                \texttt{ABLEHNUNG: \{'C15', 'C28', 'C48', 'B66', 'C60', 'B49', 'A93'\}} \\
\\ 
\texttt{ARGUMENT: \{'Ihr Vorschlag ist interessant, aber ich denke wir können eine noch bessere Balance finden. B78 und B10 sind wichtige Komponenten, die wir nicht aufgeben sollten. Außerdem könnte A35 einen wertvollen Beitrag leisten.'\}} \\
\\ 
\texttt{VORSCHLAG: \{'C15', 'C28', 'C48', 'B78', 'B66', 'B10', 'A93'\}} \\
            }
        }
    }
    & \\ \\

    \theutterance \stepcounter{utterance}  
    & \multicolumn{4}{p{0.6\linewidth}}{
        \cellcolor[rgb]{0.9,0.9,0.9}{
            \makecell[{{p{\linewidth}}}]{
                \texttt{\tiny{[P1$\rangle$GM]}}
                \texttt{ABLEHNUNG: \{'C15', 'C28', 'C48', 'B78', 'B66', 'B10', 'A93'\}} \\
\\ 
\texttt{ARGUMENT: \{'Ihr Ansatz mit B78 und B10 ist nachvollziehbar, aber wir verlieren dabei zu viel Flexibilität. C60 und B49 zusammen bieten mehr strategische Vorteile als nur B10 allein. Außerdem bleibt so mehr Spielraum für weitere Optimierungen.'\}} \\
\\ 
\texttt{VORSCHLAG: \{'C15', 'C28', 'C48', 'B78', 'B66', 'C60', 'A93'\}} \\
            }
        }
    }
    & & \\ \\

    \theutterance \stepcounter{utterance}  
    & & \multicolumn{4}{p{0.6\linewidth}}{
        \cellcolor[rgb]{0.9,0.9,0.9}{
            \makecell[{{p{\linewidth}}}]{
                \texttt{\tiny{[GM$\rangle$P2]}}
                \texttt{ABLEHNUNG: \{'C15', 'C28', 'C48', 'B78', 'B66', 'B10', 'A93'\}} \\
\\ 
\texttt{ARGUMENT: \{'Ihr Ansatz mit B78 und B10 ist nachvollziehbar, aber wir verlieren dabei zu viel Flexibilität. C60 und B49 zusammen bieten mehr strategische Vorteile als nur B10 allein. Außerdem bleibt so mehr Spielraum für weitere Optimierungen.'\}} \\
\\ 
\texttt{VORSCHLAG: \{'C15', 'C28', 'C48', 'B78', 'B66', 'C60', 'A93'\}} \\
            }
        }
    }
    & \\ \\

    \theutterance \stepcounter{utterance}  
    & & & \multicolumn{4}{p{0.6\linewidth}}{
        \cellcolor[rgb]{0.9,0.9,0.9}{
            \makecell[{{p{\linewidth}}}]{
                \texttt{\tiny{[GM$\langle$P2]}}
                \texttt{ABLEHNUNG: \{'C15', 'C28', 'C48', 'B78', 'B66', 'C60', 'A93'\}} \\
\\ 
\texttt{ARGUMENT: \{'Ich verstehe Ihre Präferenz für C60, aber B10 hat einen deutlich höheren strategischen Wert in dieser Konstellation. Die Kombination mit B78 schafft Synergien, die wir nicht ignorieren sollten. Zusätzlich können wir mit A56 eine weitere wertvolle Komponente integrieren.'\}} \\
\\ 
\texttt{VORSCHLAG: \{'C15', 'C28', 'C48', 'B66', 'A56', 'B10', 'A93'\}} \\
            }
        }
    }
     \\ \\

    \theutterance \stepcounter{utterance}  
    & & \multicolumn{4}{p{0.6\linewidth}}{
        \cellcolor[rgb]{0.9,0.9,0.9}{
            \makecell[{{p{\linewidth}}}]{
                \texttt{\tiny{[P1$\langle$GM]}}
                \texttt{ABLEHNUNG: \{'C15', 'C28', 'C48', 'B78', 'B66', 'C60', 'A93'\}} \\
\\ 
\texttt{ARGUMENT: \{'Ich verstehe Ihre Präferenz für C60, aber B10 hat einen deutlich höheren strategischen Wert in dieser Konstellation. Die Kombination mit B78 schafft Synergien, die wir nicht ignorieren sollten. Zusätzlich können wir mit A56 eine weitere wertvolle Komponente integrieren.'\}} \\
\\ 
\texttt{VORSCHLAG: \{'C15', 'C28', 'C48', 'B66', 'A56', 'B10', 'A93'\}} \\
            }
        }
    }
    & \\ \\

    \theutterance \stepcounter{utterance}  
    & \multicolumn{4}{p{0.6\linewidth}}{
        \cellcolor[rgb]{0.9,0.9,0.9}{
            \makecell[{{p{\linewidth}}}]{
                \texttt{\tiny{[P1$\rangle$GM]}}
                \texttt{ABLEHNUNG: \{'C15', 'C28', 'C48', 'B66', 'A56', 'B10', 'A93'\}} \\
\\ 
\texttt{ARGUMENT: \{'A56 bringt einen sehr hohen Aufwand von 994 mit sich, was die Effizienz der gesamten Kombination reduziert. B78 bietet bei geringerem Aufwand bessere Möglichkeiten. Außerdem sollten wir B49 nicht völlig außer Acht lassen {-} es könnte eine gute Ergänzung sein.'\}} \\
\\ 
\texttt{VORSCHLAG: \{'C15', 'C28', 'C48', 'B78', 'B66', 'B49', 'A93'\}} \\
            }
        }
    }
    & & \\ \\

    \theutterance \stepcounter{utterance}  
    & & \multicolumn{4}{p{0.6\linewidth}}{
        \cellcolor[rgb]{0.9,0.9,0.9}{
            \makecell[{{p{\linewidth}}}]{
                \texttt{\tiny{[GM$\rangle$P2]}}
                \texttt{ABLEHNUNG: \{'C15', 'C28', 'C48', 'B66', 'A56', 'B10', 'A93'\}} \\
\\ 
\texttt{ARGUMENT: \{'A56 bringt einen sehr hohen Aufwand von 994 mit sich, was die Effizienz der gesamten Kombination reduziert. B78 bietet bei geringerem Aufwand bessere Möglichkeiten. Außerdem sollten wir B49 nicht völlig außer Acht lassen {-} es könnte eine gute Ergänzung sein.'\}} \\
\\ 
\texttt{VORSCHLAG: \{'C15', 'C28', 'C48', 'B78', 'B66', 'B49', 'A93'\}} \\
            }
        }
    }
    & \\ \\

    \theutterance \stepcounter{utterance}  
    & & & \multicolumn{4}{p{0.6\linewidth}}{
        \cellcolor[rgb]{0.9,0.9,0.9}{
            \makecell[{{p{\linewidth}}}]{
                \texttt{\tiny{[GM$\langle$P2]}}
                \texttt{ZUSTIMMUNG: \{'C15', 'C28', 'C48', 'B78', 'B66', 'B49', 'A93'\}} \\
\\ 
\texttt{ARGUMENT: \{'Sie haben einen guten Punkt bezüglich der Effizienz gemacht. Diese Kombination bietet tatsächlich eine ausgewogene Mischung aus verschiedenen Kategorien bei akzeptablem Gesamtaufwand. Ich kann diesem Vorschlag zustimmen.'\}} \\
            }
        }
    }
     \\ \\

    \theutterance \stepcounter{utterance}  
    & & & \multicolumn{2}{p{0.3\linewidth}}{
        \cellcolor[rgb]{0.9,0.9,0.9}{
            \makecell[{{p{\linewidth}}}]{
                \texttt{\tiny{[GM$|$GM]}}
                \texttt{game successful} \\
            }
        }
    }
    & & \\ \\

    \theutterance \stepcounter{utterance}  
    & & & \multicolumn{2}{p{0.3\linewidth}}{
        \cellcolor[rgb]{0.9,0.9,0.9}{
            \makecell[{{p{\linewidth}}}]{
                \texttt{\tiny{[GM$|$GM]}}
                \texttt{end game} \\
            }
        }
    }
    & & \\ \\

\end{supertabular}
}

\end{document}
