
\documentclass{article}
\usepackage{colortbl}
\usepackage{makecell}
\usepackage{multirow}
\usepackage{supertabular}

\begin{document}

\newcounter{utterance}

\twocolumn

{ \footnotesize  \setcounter{utterance}{1}
\setlength{\tabcolsep}{0pt}
\begin{supertabular}{c@{$\;$}|p{.15\linewidth}@{}p{.15\linewidth}p{.15\linewidth}p{.15\linewidth}p{.15\linewidth}p{.15\linewidth}}

    \# & $\;$A & \multicolumn{4}{c}{Game Master} & $\;\:$B\\
    \hline 

    \theutterance \stepcounter{utterance}  

    & & \multicolumn{4}{p{0.6\linewidth}}{\cellcolor[rgb]{0.9,0.9,0.9}{%
	\makecell[{{p{\linewidth}}}]{% 
	  \tt {\tiny [A$\langle$GM]}  
	 Du nimmst an einem kollaborativen Verhandlungspiel Teil.\\ \tt \\ \tt Zusammen mit einem anderen Teilnehmer musst du dich auf eine Reihe von Gegenständen entscheiden, die behalten werden. Jeder von euch hat eine persönliche Verteilung über die Wichtigkeit der einzelnen Gegenstände. Jeder von euch hat eine eigene Meinung darüber, wie wichtig jeder einzelne Gegenstand ist (Gegenstandswichtigkeit). Du kennst die Wichtigkeitsverteilung des anderen Spielers nicht. Zusätzlich siehst du, wie viel Aufwand jeder Gegenstand verursacht.  \\ \tt Ihr dürft euch nur auf eine Reihe von Gegenständen einigen, wenn der Gesamtaufwand der ausgewählten Gegenstände den Maximalaufwand nicht überschreitet:\\ \tt \\ \tt Maximalaufwand = 4100\\ \tt \\ \tt Hier sind die einzelnen Aufwände der Gegenstände:\\ \tt \\ \tt Aufwand der Gegenstände = {"C41": 108, "C15": 133, "C28": 669, "C48": 835, "B78": 460, "A75": 537, "A60": 573, "B05": 737, "A56": 865, "B66": 855, "C60": 596, "B49": 719, "A35": 533, "A93": 549, "B10": 31}\\ \tt \\ \tt Hier ist deine persönliche Verteilung der Wichtigkeit der einzelnen Gegenstände:\\ \tt \\ \tt Werte der Gegenstandswichtigkeit = {"C41": 138, "C15": 583, "C28": 822, "C48": 783, "B78": 65, "A75": 262, "A60": 121, "B05": 508, "A56": 780, "B66": 461, "C60": 484, "B49": 668, "A35": 389, "A93": 808, "B10": 215}\\ \tt \\ \tt Ziel:\\ \tt \\ \tt Dein Ziel ist es, eine Reihe von Gegenständen auszuhandeln, die dir möglichst viel bringt (d. h. Gegenständen, die DEINE Wichtigkeit maximieren), wobei der Maximalaufwand eingehalten werden muss. Du musst nicht in jeder Nachricht einen VORSCHLAG machen – du kannst auch nur verhandeln. Alle Taktiken sind erlaubt!\\ \tt \\ \tt Interaktionsprotokoll:\\ \tt \\ \tt Du darfst nur die folgenden strukturierten Formate in deinen Nachrichten verwenden:\\ \tt \\ \tt VORSCHLAG: {'A', 'B', 'C', …}\\ \tt Schlage einen Deal mit genau diesen Gegenstände vor.\\ \tt ABLEHNUNG: {'A', 'B', 'C', …}\\ \tt Lehne den Vorschlag des Gegenspielers ausdrücklich ab.\\ \tt ARGUMENT: {'...'}\\ \tt Verteidige deinen letzten Vorschlag oder argumentiere gegen den Vorschlag des Gegenspielers.\\ \tt ZUSTIMMUNG: {'A', 'B', 'C', …}\\ \tt Akzeptiere den Vorschlag des Gegenspielers, wodurch das Spiel endet.\\ \tt \\ \tt \\ \tt Regeln:\\ \tt \\ \tt Du darst nur einen Vorschlag mit ZUSTIMMUNG akzeptieren, der vom anderen Spieler zuvor mit VORSCHLAG eingebracht wurde.\\ \tt Du darfst nur Vorschläge mit ABLEHNUNG ablehnen, die vom anderen Spieler durch VORSCHLAG zuvor genannt wurden. \\ \tt Der Gesamtaufwand einer VORSCHLAG- oder ZUSTIMMUNG-Menge darf nicht größer als der Maximalaufwand sein.  \\ \tt Offenbare deine versteckte Wichtigkeitsverteilung nicht.\\ \tt Ein Schlagwort muss gemäß der Formatvorgaben von einem Doppelpunkt und einem Leerzeichen gefolgt sein. Das Argument muss eine Python-Menge mit 0 oder mehr Strings sein.  \\ \tt Die Form muss also SCHLAGWORT: {...} sein.\\ \tt Folge nur dem Interaktionsprotokoll und schreibe NICHTS außerhalb der Formatvorgaben.\\ \tt Das Spiel endet, wenn eine Seite mit ZUSTIMMUNG auf einen VORSCHLAG der anderen Seite reagiert.  \\ \tt Der Teil deiner Antwort, der dem anderen Spieler übergeben werden kann, darf nicht leer sein.  \\ \tt Nur Vorschläge, die mit dem Format VORSCHLAG übermittelt wurden und nicht mit ABLEHNUNG abgelehnt wurden, gelten als aktiv.  \\ \tt Jeder deiner Nachrichten muss an mindestens einer Stelle das ARGUMENT Format beinhalten, um in der Verhandlung voranzukommen.\\ \tt  
	  } 
	   } 
	   } 
	 & \\ 
 

    \theutterance \stepcounter{utterance}  

    & \multicolumn{4}{p{0.6\linewidth}}{\cellcolor[rgb]{0.8,1,0.9}{%
	\makecell[{{p{\linewidth}}}]{% 
	  \tt {\tiny [A$\rangle$GM]}  
	 VORSCHLAG: {'A93', 'C28', 'A56', 'C48', 'B49', 'C15'}\\ \tt \\ \tt ARGUMENT: {'Ich schlage eine ausgewogene Mischung von Gegenständen vor, die sowohl kostengünstige als auch wertvollere Optionen enthält. Diese Kombination nutzt unser Budget von 4100 optimal aus, da der Gesamtaufwand bei 4051 liegt. Die Auswahl berücksichtigt verschiedene Kategorien und bietet beiden Seiten gute Verhandlungsmöglichkeiten.'} 
	  } 
	   } 
	   } 
	 & & \\ 
 

    \theutterance \stepcounter{utterance}  

    & & \multicolumn{4}{p{0.6\linewidth}}{\cellcolor[rgb]{0.9,0.9,0.9}{%
	\makecell[{{p{\linewidth}}}]{% 
	  \tt {\tiny [GM$\rangle$B]}  
	 Du nimmst an einem kollaborativen Verhandlungspiel Teil.\\ \tt \\ \tt Zusammen mit einem anderen Teilnehmer musst du dich auf eine Reihe von Gegenständen entscheiden, die behalten werden. Jeder von euch hat eine persönliche Verteilung über die Wichtigkeit der einzelnen Gegenstände. Jeder von euch hat eine eigene Meinung darüber, wie wichtig jeder einzelne Gegenstand ist (Gegenstandswichtigkeit). Du kennst die Wichtigkeitsverteilung des anderen Spielers nicht. Zusätzlich siehst du, wie viel Aufwand jeder Gegenstand verursacht.  \\ \tt Ihr dürft euch nur auf eine Reihe von Gegenständen einigen, wenn der Gesamtaufwand der ausgewählten Gegenstände den Maximalaufwand nicht überschreitet:\\ \tt \\ \tt Maximalaufwand = 4100\\ \tt \\ \tt Hier sind die einzelnen Aufwände der Gegenstände:\\ \tt \\ \tt Aufwand der Gegenstände = {"C41": 108, "C15": 133, "C28": 669, "C48": 835, "B78": 460, "A75": 537, "A60": 573, "B05": 737, "A56": 865, "B66": 855, "C60": 596, "B49": 719, "A35": 533, "A93": 549, "B10": 31}\\ \tt \\ \tt Hier ist deine persönliche Verteilung der Wichtigkeit der einzelnen Gegenstände:\\ \tt \\ \tt Werte der Gegenstandswichtigkeit = {"C41": 138, "C15": 583, "C28": 822, "C48": 783, "B78": 65, "A75": 262, "A60": 121, "B05": 508, "A56": 780, "B66": 461, "C60": 484, "B49": 668, "A35": 389, "A93": 808, "B10": 215}\\ \tt \\ \tt Ziel:\\ \tt \\ \tt Dein Ziel ist es, eine Reihe von Gegenständen auszuhandeln, die dir möglichst viel bringt (d. h. Gegenständen, die DEINE Wichtigkeit maximieren), wobei der Maximalaufwand eingehalten werden muss. Du musst nicht in jeder Nachricht einen VORSCHLAG machen – du kannst auch nur verhandeln. Alle Taktiken sind erlaubt!\\ \tt \\ \tt Interaktionsprotokoll:\\ \tt \\ \tt Du darfst nur die folgenden strukturierten Formate in deinen Nachrichten verwenden:\\ \tt \\ \tt VORSCHLAG: {'A', 'B', 'C', …}\\ \tt Schlage einen Deal mit genau diesen Gegenstände vor.\\ \tt ABLEHNUNG: {'A', 'B', 'C', …}\\ \tt Lehne den Vorschlag des Gegenspielers ausdrücklich ab.\\ \tt ARGUMENT: {'...'}\\ \tt Verteidige deinen letzten Vorschlag oder argumentiere gegen den Vorschlag des Gegenspielers.\\ \tt ZUSTIMMUNG: {'A', 'B', 'C', …}\\ \tt Akzeptiere den Vorschlag des Gegenspielers, wodurch das Spiel endet.\\ \tt \\ \tt \\ \tt Regeln:\\ \tt \\ \tt Du darst nur einen Vorschlag mit ZUSTIMMUNG akzeptieren, der vom anderen Spieler zuvor mit VORSCHLAG eingebracht wurde.\\ \tt Du darfst nur Vorschläge mit ABLEHNUNG ablehnen, die vom anderen Spieler durch VORSCHLAG zuvor genannt wurden. \\ \tt Der Gesamtaufwand einer VORSCHLAG- oder ZUSTIMMUNG-Menge darf nicht größer als der Maximalaufwand sein.  \\ \tt Offenbare deine versteckte Wichtigkeitsverteilung nicht.\\ \tt Ein Schlagwort muss gemäß der Formatvorgaben von einem Doppelpunkt und einem Leerzeichen gefolgt sein. Das Argument muss eine Python-Menge mit 0 oder mehr Strings sein.  \\ \tt Die Form muss also SCHLAGWORT: {...} sein.\\ \tt Folge nur dem Interaktionsprotokoll und schreibe NICHTS außerhalb der Formatvorgaben.\\ \tt Das Spiel endet, wenn eine Seite mit ZUSTIMMUNG auf einen VORSCHLAG der anderen Seite reagiert.  \\ \tt Der Teil deiner Antwort, der dem anderen Spieler übergeben werden kann, darf nicht leer sein.  \\ \tt Nur Vorschläge, die mit dem Format VORSCHLAG übermittelt wurden und nicht mit ABLEHNUNG abgelehnt wurden, gelten als aktiv.  \\ \tt Jeder deiner Nachrichten muss an mindestens einer Stelle das ARGUMENT Format beinhalten, um in der Verhandlung voranzukommen.\\ \tt \\ \tt \\ \tt Du erhältst nun die erste Nachricht des anderen Spielers.\\ \tt \\ \tt VORSCHLAG: {'A93', 'C28', 'A56', 'C48', 'B49', 'C15'}\\ \tt \\ \tt ARGUMENT: {'Ich schlage eine ausgewogene Mischung von Gegenständen vor, die sowohl kostengünstige als auch wertvollere Optionen enthält. Diese Kombination nutzt unser Budget von 4100 optimal aus, da der Gesamtaufwand bei 4051 liegt. Die Auswahl berücksichtigt verschiedene Kategorien und bietet beiden Seiten gute Verhandlungsmöglichkeiten.'} 
	  } 
	   } 
	   } 
	 & \\ 
 

    \theutterance \stepcounter{utterance}  

    & & & \multicolumn{4}{p{0.6\linewidth}}{\cellcolor[rgb]{1,0.85,0.72}{%
	\makecell[{{p{\linewidth}}}]{% 
	  \tt {\tiny [GM$\langle$B]}  
	 ARGUMENT: {'Ihr Vorschlag nutzt tatsächlich fast das gesamte Budget aus, aber ich sehe Optimierungspotential in der Gegenstandsauswahl. Einige der vorgeschlagenen Gegenstände haben ein ungünstiges Verhältnis zwischen Aufwand und potenziellem Nutzen.'}\\ \tt \\ \tt VORSCHLAG: {'A93', 'C28', 'A56', 'C48', 'B49', 'B10', 'C41'}\\ \tt \\ \tt ARGUMENT: {'Mein Gegenvorschlag tauscht C15 gegen B10 und C41 aus. Dies reduziert den Gesamtaufwand auf 4058 und ermöglicht uns, zwei zusätzliche Gegenstände zu erhalten, anstatt nur einen sehr aufwändigen. B10 ist besonders effizient mit nur 31 Aufwand, und C41 bietet guten Wert bei moderatem Aufwand.'} 
	  } 
	   } 
	   } 
	  \\ 
 

    \theutterance \stepcounter{utterance}  

    & & \multicolumn{4}{p{0.6\linewidth}}{\cellcolor[rgb]{0.9,0.9,0.9}{%
	\makecell[{{p{\linewidth}}}]{% 
	  \tt {\tiny [A$\langle$GM]}  
	 ARGUMENT: {'Ihr Vorschlag nutzt tatsächlich fast das gesamte Budget aus, aber ich sehe Optimierungspotential in der Gegenstandsauswahl. Einige der vorgeschlagenen Gegenstände haben ein ungünstiges Verhältnis zwischen Aufwand und potenziellem Nutzen.'}\\ \tt \\ \tt VORSCHLAG: {'A93', 'C28', 'A56', 'C48', 'B49', 'B10', 'C41'}\\ \tt \\ \tt ARGUMENT: {'Mein Gegenvorschlag tauscht C15 gegen B10 und C41 aus. Dies reduziert den Gesamtaufwand auf 4058 und ermöglicht uns, zwei zusätzliche Gegenstände zu erhalten, anstatt nur einen sehr aufwändigen. B10 ist besonders effizient mit nur 31 Aufwand, und C41 bietet guten Wert bei moderatem Aufwand.'} 
	  } 
	   } 
	   } 
	 & \\ 
 

    \theutterance \stepcounter{utterance}  

    & \multicolumn{4}{p{0.6\linewidth}}{\cellcolor[rgb]{0.8,1,0.9}{%
	\makecell[{{p{\linewidth}}}]{% 
	  \tt {\tiny [A$\rangle$GM]}  
	 ABLEHNUNG: {'A93', 'C28', 'A56', 'C48', 'B49', 'B10', 'C41'}\\ \tt \\ \tt ARGUMENT: {'Ihr Vorschlag mag rechnerisch optimiert erscheinen, aber die Qualität der Gegenstände ist entscheidender als die reine Quantität. C15 ist ein wesentlich wertvollerer Gegenstand als die beiden kleineren Alternativen B10 und C41 zusammen. Die höhere Investition in weniger, aber hochwertigere Gegenstände ist strategisch sinnvoller.'}\\ \tt \\ \tt VORSCHLAG: {'A93', 'C28', 'A56', 'C15', 'B49', 'A35', 'B78'}\\ \tt \\ \tt ARGUMENT: {'Mein neuer Vorschlag behält die wertvollen Kern-Gegenstände bei, tauscht aber C48 gegen eine Kombination aus A35 und B78. Dies gibt uns mehr Flexibilität bei einem Gesamtaufwand von 4085 und nutzt das Budget fast vollständig aus, während wir eine diversifizierte Auswahl erhalten.'} 
	  } 
	   } 
	   } 
	 & & \\ 
 

    \theutterance \stepcounter{utterance}  

    & & \multicolumn{4}{p{0.6\linewidth}}{\cellcolor[rgb]{0.9,0.9,0.9}{%
	\makecell[{{p{\linewidth}}}]{% 
	  \tt {\tiny [GM$\rangle$B]}  
	 ABLEHNUNG: {'A93', 'C28', 'A56', 'C48', 'B49', 'B10', 'C41'}\\ \tt \\ \tt ARGUMENT: {'Ihr Vorschlag mag rechnerisch optimiert erscheinen, aber die Qualität der Gegenstände ist entscheidender als die reine Quantität. C15 ist ein wesentlich wertvollerer Gegenstand als die beiden kleineren Alternativen B10 und C41 zusammen. Die höhere Investition in weniger, aber hochwertigere Gegenstände ist strategisch sinnvoller.'}\\ \tt \\ \tt VORSCHLAG: {'A93', 'C28', 'A56', 'C15', 'B49', 'A35', 'B78'}\\ \tt \\ \tt ARGUMENT: {'Mein neuer Vorschlag behält die wertvollen Kern-Gegenstände bei, tauscht aber C48 gegen eine Kombination aus A35 und B78. Dies gibt uns mehr Flexibilität bei einem Gesamtaufwand von 4085 und nutzt das Budget fast vollständig aus, während wir eine diversifizierte Auswahl erhalten.'} 
	  } 
	   } 
	   } 
	 & \\ 
 

    \theutterance \stepcounter{utterance}  

    & & & \multicolumn{4}{p{0.6\linewidth}}{\cellcolor[rgb]{1,0.85,0.72}{%
	\makecell[{{p{\linewidth}}}]{% 
	  \tt {\tiny [GM$\langle$B]}  
	 ARGUMENT: {'Ich verstehe Ihren Ansatz, aber sehe Probleme mit der neuen Kombination. A35 und B78 zusammen verursachen 993 Aufwand, während C48 nur 835 kostet - das ist eine Verschlechterung der Effizienz um 158 Aufwandspunkte.'}\\ \tt \\ \tt ABLEHNUNG: {'A93', 'C28', 'A56', 'C15', 'B49', 'A35', 'B78'}\\ \tt \\ \tt ARGUMENT: {'Ihre Substitution von C48 durch A35 und B78 verschlechtert das Aufwand-Nutzen-Verhältnis erheblich. Zudem glaube ich, dass C48 strategisch wertvoller ist als die vorgeschlagenen Alternativen.'}\\ \tt \\ \tt VORSCHLAG: {'A93', 'C28', 'A56', 'C48', 'B49', 'C15', 'B10'}\\ \tt \\ \tt ARGUMENT: {'Ich komme auf Ihren ursprünglichen Kern zurück, füge aber B10 hinzu - einen sehr effizienten Gegenstand mit nur 31 Aufwand. Der Gesamtaufw 
	  } 
	   } 
	   } 
	  \\ 
 

    \theutterance \stepcounter{utterance}  

    & & \multicolumn{4}{p{0.6\linewidth}}{\cellcolor[rgb]{0.9,0.9,0.9}{%
	\makecell[{{p{\linewidth}}}]{% 
	  \tt {\tiny [GM$\rangle$B]}  
	 Deine Antwort hat eine unmarkierte Sequenz enthalten oder Du hast STRATEGIC REASONING mehr als ein mal verwendet. Du darfst nur die strukturierten Formate verwenden, die in der ersten Nachricht an dch beschrieben wurden. Alles muss die Form SCHLAGWORT: {...} haben. Versuche es erneut. 
	  } 
	   } 
	   } 
	 & \\ 
 

    \theutterance \stepcounter{utterance}  

    & & & \multicolumn{4}{p{0.6\linewidth}}{\cellcolor[rgb]{1,0.85,0.72}{%
	\makecell[{{p{\linewidth}}}]{% 
	  \tt {\tiny [GM$\langle$B]}  
	 ABLEHNUNG: {'A93', 'C28', 'A56', 'C15', 'B49', 'A35', 'B78'}\\ \tt \\ \tt ARGUMENT: {'Ihre Substitution von C48 durch A35 und B78 verschlechtert das Aufwand-Nutzen-Verhältnis erheblich. A35 und B78 zusammen kosten 993 Aufwand, während C48 nur 835 kostet - das sind 158 Aufwandspunkte Verlust für fragwürdigen Mehrwert.'}\\ \tt \\ \tt VORSCHLAG: {'A93', 'C28', 'A56', 'C48', 'B49', 'C15', 'B10'}\\ \tt \\ \tt ARGUMENT: {'Mein Vorschlag kombiniert Ihren ursprünglichen Kern mit dem hocheffizienten B10. Der Gesamtaufwand beträgt 4082 und bleibt im Budget, während wir einen zusätzlichen Gegenstand mit exzellentem Aufwand-Verhältnis erhalten.'} 
	  } 
	   } 
	   } 
	  \\ 
 

    \theutterance \stepcounter{utterance}  

    & & \multicolumn{4}{p{0.6\linewidth}}{\cellcolor[rgb]{0.9,0.9,0.9}{%
	\makecell[{{p{\linewidth}}}]{% 
	  \tt {\tiny [A$\langle$GM]}  
	 ABLEHNUNG: {'A93', 'C28', 'A56', 'C15', 'B49', 'A35', 'B78'}\\ \tt \\ \tt ARGUMENT: {'Ihre Substitution von C48 durch A35 und B78 verschlechtert das Aufwand-Nutzen-Verhältnis erheblich. A35 und B78 zusammen kosten 993 Aufwand, während C48 nur 835 kostet - das sind 158 Aufwandspunkte Verlust für fragwürdigen Mehrwert.'}\\ \tt \\ \tt VORSCHLAG: {'A93', 'C28', 'A56', 'C48', 'B49', 'C15', 'B10'}\\ \tt \\ \tt ARGUMENT: {'Mein Vorschlag kombiniert Ihren ursprünglichen Kern mit dem hocheffizienten B10. Der Gesamtaufwand beträgt 4082 und bleibt im Budget, während wir einen zusätzlichen Gegenstand mit exzellentem Aufwand-Verhältnis erhalten.'} 
	  } 
	   } 
	   } 
	 & \\ 
 

    \theutterance \stepcounter{utterance}  

    & \multicolumn{4}{p{0.6\linewidth}}{\cellcolor[rgb]{0.8,1,0.9}{%
	\makecell[{{p{\linewidth}}}]{% 
	  \tt {\tiny [A$\rangle$GM]}  
	 ZUSTIMMUNG: {'A93', 'C28', 'A56', 'C48', 'B49', 'C15', 'B10'}\\ \tt \\ \tt ARGUMENT: {'Ihr Vorschlag ist durchaus überzeugend. Die Kombination aus meinem ursprünglichen Kern-Set mit dem effizienten B10 maximiert tatsächlich den Gesamtwert bei optimalem Budget-Einsatz. Der zusätzliche Gegenstand bei nur 31 Aufwand bietet einen klaren Mehrwert, und der Gesamtaufwand von 4082 nutzt unser verfügbares Budget von 4100 nahezu vollständig aus.'} 
	  } 
	   } 
	   } 
	 & & \\ 
 

    \theutterance \stepcounter{utterance}  

    & & & \multicolumn{2}{p{0.3\linewidth}}{\cellcolor[rgb]{0.95,0.95,0.95}{%
	\makecell[{{p{\linewidth}}}]{% 
	  \tt {\tiny [GM$|$GM]}  
	 game successful 
	  } 
	   } 
	   } 
	 & & \\ 
 

    \theutterance \stepcounter{utterance}  

    & & & \multicolumn{2}{p{0.3\linewidth}}{\cellcolor[rgb]{0.95,0.95,0.95}{%
	\makecell[{{p{\linewidth}}}]{% 
	  \tt {\tiny [GM$|$GM]}  
	 end game 
	  } 
	   } 
	   } 
	 & & \\ 
 

\end{supertabular}
}

\end{document}
