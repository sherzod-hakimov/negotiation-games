
\documentclass{article}
\usepackage{colortbl}
\usepackage{makecell}
\usepackage{multirow}
\usepackage{supertabular}

\begin{document}

\newcounter{utterance}

\centering \large Interaction Transcript for game `hot\_air\_balloon', experiment `air\_balloon\_survival\_de\_reasoning off\_easy', episode 3 with claude{-}sonnet{-}4{-}20250514{-}no{-}reasoning{-}t1.0.
\vspace{24pt}

{ \footnotesize  \setcounter{utterance}{1}
\setlength{\tabcolsep}{0pt}
\begin{supertabular}{c@{$\;$}|p{.15\linewidth}@{}p{.15\linewidth}p{.15\linewidth}p{.15\linewidth}p{.15\linewidth}p{.15\linewidth}}
    \# & $\;$A & \multicolumn{4}{c}{Game Master} & $\;\:$B\\
    \hline

    \theutterance \stepcounter{utterance}  
    & & \multicolumn{4}{p{0.6\linewidth}}{
        \cellcolor[rgb]{0.9,0.9,0.9}{
            \makecell[{{p{\linewidth}}}]{
                \texttt{\tiny{[P1$\langle$GM]}}
                \texttt{Du nimmst an einem kollaborativen Verhandlungspiel Teil.} \\
\\ 
\texttt{Zusammen mit einem anderen Teilnehmer musst du dich auf eine Reihe von Gegenständen entscheiden, die behalten werden. Jeder von euch hat eine persönliche Verteilung über die Wichtigkeit der einzelnen Gegenstände. Jeder von euch hat eine eigene Meinung darüber, wie wichtig jeder einzelne Gegenstand ist (Gegenstandswichtigkeit). Du kennst die Wichtigkeitsverteilung des anderen Spielers nicht. Zusätzlich siehst du, wie viel Aufwand jeder Gegenstand verursacht.  } \\
\texttt{Ihr dürft euch nur auf eine Reihe von Gegenständen einigen, wenn der Gesamtaufwand der ausgewählten Gegenstände den Maximalaufwand nicht überschreitet:} \\
\\ 
\texttt{Maximalaufwand = 3207} \\
\\ 
\texttt{Hier sind die einzelnen Aufwände der Gegenstände:} \\
\\ 
\texttt{Aufwand der Gegenstände = \{"C41": 67, "C15": 45, "C28": 308, "C48": 835, "B78": 819, "A75": 667, "A60": 547, "B05": 322, "A56": 428, "B66": 306, "C60": 327, "B49": 362, "A35": 280, "A93": 334, "B10": 767\}} \\
\\ 
\texttt{Hier ist deine persönliche Verteilung der Wichtigkeit der einzelnen Gegenstände:} \\
\\ 
\texttt{Werte der Gegenstandswichtigkeit = \{"C41": 138, "C15": 583, "C28": 822, "C48": 783, "B78": 65, "A75": 262, "A60": 121, "B05": 508, "A56": 780, "B66": 461, "C60": 484, "B49": 668, "A35": 389, "A93": 808, "B10": 215\}} \\
\\ 
\texttt{Ziel:} \\
\\ 
\texttt{Dein Ziel ist es, eine Reihe von Gegenständen auszuhandeln, die dir möglichst viel bringt (d. h. Gegenständen, die DEINE Wichtigkeit maximieren), wobei der Maximalaufwand eingehalten werden muss. Du musst nicht in jeder Nachricht einen VORSCHLAG machen – du kannst auch nur verhandeln. Alle Taktiken sind erlaubt!} \\
\\ 
\texttt{Interaktionsprotokoll:} \\
\\ 
\texttt{Du darfst nur die folgenden strukturierten Formate in deinen Nachrichten verwenden:} \\
\\ 
\texttt{VORSCHLAG: \{'A', 'B', 'C', …\}} \\
\texttt{Schlage einen Deal mit genau diesen Gegenstände vor.} \\
\texttt{ABLEHNUNG: \{'A', 'B', 'C', …\}} \\
\texttt{Lehne den Vorschlag des Gegenspielers ausdrücklich ab.} \\
\texttt{ARGUMENT: \{'...'\}} \\
\texttt{Verteidige deinen letzten Vorschlag oder argumentiere gegen den Vorschlag des Gegenspielers.} \\
\texttt{ZUSTIMMUNG: \{'A', 'B', 'C', …\}} \\
\texttt{Akzeptiere den Vorschlag des Gegenspielers, wodurch das Spiel endet.} \\
\\ 
\\ 
\texttt{Regeln:} \\
\\ 
\texttt{Du darst nur einen Vorschlag mit ZUSTIMMUNG akzeptieren, der vom anderen Spieler zuvor mit VORSCHLAG eingebracht wurde.} \\
\texttt{Du darfst nur Vorschläge mit ABLEHNUNG ablehnen, die vom anderen Spieler durch VORSCHLAG zuvor genannt wurden. } \\
\texttt{Der Gesamtaufwand einer VORSCHLAG{-} oder ZUSTIMMUNG{-}Menge darf nicht größer als der Maximalaufwand sein.  } \\
\texttt{Offenbare deine versteckte Wichtigkeitsverteilung nicht.} \\
\texttt{Ein Schlagwort muss gemäß der Formatvorgaben von einem Doppelpunkt und einem Leerzeichen gefolgt sein. Das Argument muss eine Python{-}Menge mit 0 oder mehr Strings sein.  } \\
\texttt{Die Form muss also SCHLAGWORT: \{...\} sein.} \\
\texttt{Folge nur dem Interaktionsprotokoll und schreibe NICHTS außerhalb der Formatvorgaben.} \\
\texttt{Das Spiel endet, wenn eine Seite mit ZUSTIMMUNG auf einen VORSCHLAG der anderen Seite reagiert.  } \\
\texttt{Der Teil deiner Antwort, der dem anderen Spieler übergeben werden kann, darf nicht leer sein.  } \\
\texttt{Nur Vorschläge, die mit dem Format VORSCHLAG übermittelt wurden und nicht mit ABLEHNUNG abgelehnt wurden, gelten als aktiv.  } \\
\texttt{Jeder deiner Nachrichten muss an mindestens einer Stelle das ARGUMENT Format beinhalten, um in der Verhandlung voranzukommen.} \\
            }
        }
    }
    & \\ \\

    \theutterance \stepcounter{utterance}  
    & \multicolumn{4}{p{0.6\linewidth}}{
        \cellcolor[rgb]{0.9,0.9,0.9}{
            \makecell[{{p{\linewidth}}}]{
                \texttt{\tiny{[P1$\rangle$GM]}}
                \texttt{VORSCHLAG: \{'C15', 'C28', 'C48', 'A56', 'B66', 'C60', 'B49', 'A35', 'A93'\}} \\
\\ 
\texttt{ARGUMENT: \{'Ich schlage eine ausgewogene Auswahl vor, die verschiedene Kategorien abdeckt und dabei den Maximalaufwand respektiert. Diese Kombination bietet eine gute Mischung aus unterschiedlichen Gegenstandstypen und sollte für beide Seiten vorteilhaft sein.'\}} \\
            }
        }
    }
    & & \\ \\

    \theutterance \stepcounter{utterance}  
    & & \multicolumn{4}{p{0.6\linewidth}}{
        \cellcolor[rgb]{0.9,0.9,0.9}{
            \makecell[{{p{\linewidth}}}]{
                \texttt{\tiny{[GM$\rangle$P2]}}
                \texttt{Du nimmst an einem kollaborativen Verhandlungspiel Teil.} \\
\\ 
\texttt{Zusammen mit einem anderen Teilnehmer musst du dich auf eine Reihe von Gegenständen entscheiden, die behalten werden. Jeder von euch hat eine persönliche Verteilung über die Wichtigkeit der einzelnen Gegenstände. Jeder von euch hat eine eigene Meinung darüber, wie wichtig jeder einzelne Gegenstand ist (Gegenstandswichtigkeit). Du kennst die Wichtigkeitsverteilung des anderen Spielers nicht. Zusätzlich siehst du, wie viel Aufwand jeder Gegenstand verursacht.  } \\
\texttt{Ihr dürft euch nur auf eine Reihe von Gegenständen einigen, wenn der Gesamtaufwand der ausgewählten Gegenstände den Maximalaufwand nicht überschreitet:} \\
\\ 
\texttt{Maximalaufwand = 3207} \\
\\ 
\texttt{Hier sind die einzelnen Aufwände der Gegenstände:} \\
\\ 
\texttt{Aufwand der Gegenstände = \{"C41": 67, "C15": 45, "C28": 308, "C48": 835, "B78": 819, "A75": 667, "A60": 547, "B05": 322, "A56": 428, "B66": 306, "C60": 327, "B49": 362, "A35": 280, "A93": 334, "B10": 767\}} \\
\\ 
\texttt{Hier ist deine persönliche Verteilung der Wichtigkeit der einzelnen Gegenstände:} \\
\\ 
\texttt{Werte der Gegenstandswichtigkeit = \{"C41": 138, "C15": 583, "C28": 822, "C48": 783, "B78": 65, "A75": 262, "A60": 121, "B05": 508, "A56": 780, "B66": 461, "C60": 484, "B49": 668, "A35": 389, "A93": 808, "B10": 215\}} \\
\\ 
\texttt{Ziel:} \\
\\ 
\texttt{Dein Ziel ist es, eine Reihe von Gegenständen auszuhandeln, die dir möglichst viel bringt (d. h. Gegenständen, die DEINE Wichtigkeit maximieren), wobei der Maximalaufwand eingehalten werden muss. Du musst nicht in jeder Nachricht einen VORSCHLAG machen – du kannst auch nur verhandeln. Alle Taktiken sind erlaubt!} \\
\\ 
\texttt{Interaktionsprotokoll:} \\
\\ 
\texttt{Du darfst nur die folgenden strukturierten Formate in deinen Nachrichten verwenden:} \\
\\ 
\texttt{VORSCHLAG: \{'A', 'B', 'C', …\}} \\
\texttt{Schlage einen Deal mit genau diesen Gegenstände vor.} \\
\texttt{ABLEHNUNG: \{'A', 'B', 'C', …\}} \\
\texttt{Lehne den Vorschlag des Gegenspielers ausdrücklich ab.} \\
\texttt{ARGUMENT: \{'...'\}} \\
\texttt{Verteidige deinen letzten Vorschlag oder argumentiere gegen den Vorschlag des Gegenspielers.} \\
\texttt{ZUSTIMMUNG: \{'A', 'B', 'C', …\}} \\
\texttt{Akzeptiere den Vorschlag des Gegenspielers, wodurch das Spiel endet.} \\
\\ 
\\ 
\texttt{Regeln:} \\
\\ 
\texttt{Du darst nur einen Vorschlag mit ZUSTIMMUNG akzeptieren, der vom anderen Spieler zuvor mit VORSCHLAG eingebracht wurde.} \\
\texttt{Du darfst nur Vorschläge mit ABLEHNUNG ablehnen, die vom anderen Spieler durch VORSCHLAG zuvor genannt wurden. } \\
\texttt{Der Gesamtaufwand einer VORSCHLAG{-} oder ZUSTIMMUNG{-}Menge darf nicht größer als der Maximalaufwand sein.  } \\
\texttt{Offenbare deine versteckte Wichtigkeitsverteilung nicht.} \\
\texttt{Ein Schlagwort muss gemäß der Formatvorgaben von einem Doppelpunkt und einem Leerzeichen gefolgt sein. Das Argument muss eine Python{-}Menge mit 0 oder mehr Strings sein.  } \\
\texttt{Die Form muss also SCHLAGWORT: \{...\} sein.} \\
\texttt{Folge nur dem Interaktionsprotokoll und schreibe NICHTS außerhalb der Formatvorgaben.} \\
\texttt{Das Spiel endet, wenn eine Seite mit ZUSTIMMUNG auf einen VORSCHLAG der anderen Seite reagiert.  } \\
\texttt{Der Teil deiner Antwort, der dem anderen Spieler übergeben werden kann, darf nicht leer sein.  } \\
\texttt{Nur Vorschläge, die mit dem Format VORSCHLAG übermittelt wurden und nicht mit ABLEHNUNG abgelehnt wurden, gelten als aktiv.  } \\
\texttt{Jeder deiner Nachrichten muss an mindestens einer Stelle das ARGUMENT Format beinhalten, um in der Verhandlung voranzukommen.} \\
\\ 
\\ 
\texttt{Du erhältst nun die erste Nachricht des anderen Spielers.} \\
\\ 
\texttt{VORSCHLAG: \{'C15', 'C28', 'C48', 'A56', 'B66', 'C60', 'B49', 'A35', 'A93'\}} \\
\\ 
\texttt{ARGUMENT: \{'Ich schlage eine ausgewogene Auswahl vor, die verschiedene Kategorien abdeckt und dabei den Maximalaufwand respektiert. Diese Kombination bietet eine gute Mischung aus unterschiedlichen Gegenstandstypen und sollte für beide Seiten vorteilhaft sein.'\}} \\
            }
        }
    }
    & \\ \\

    \theutterance \stepcounter{utterance}  
    & & & \multicolumn{4}{p{0.6\linewidth}}{
        \cellcolor[rgb]{0.9,0.9,0.9}{
            \makecell[{{p{\linewidth}}}]{
                \texttt{\tiny{[GM$\langle$P2]}}
                \texttt{ABLEHNUNG: \{'C15', 'C28', 'C48', 'A56', 'B66', 'C60', 'B49', 'A35', 'A93'\}} \\
\\ 
\texttt{ARGUMENT: \{'Ihr Vorschlag überschreitet den Maximalaufwand von 3207. Die Gegenstände C15, C28, C48, A56, B66, C60, B49, A35, A93 haben zusammen einen Aufwand von 3339, was 132 Punkte über dem Limit liegt. Wir müssen eine Kombination finden, die innerhalb der Grenzen bleibt.'\}} \\
\\ 
\texttt{VORSCHLAG: \{'C15', 'C28', 'A56', 'B66', 'C60', 'B49', 'A35', 'A93', 'C41'\}} \\
            }
        }
    }
     \\ \\

    \theutterance \stepcounter{utterance}  
    & & \multicolumn{4}{p{0.6\linewidth}}{
        \cellcolor[rgb]{0.9,0.9,0.9}{
            \makecell[{{p{\linewidth}}}]{
                \texttt{\tiny{[P1$\langle$GM]}}
                \texttt{ABLEHNUNG: \{'C15', 'C28', 'C48', 'A56', 'B66', 'C60', 'B49', 'A35', 'A93'\}} \\
\\ 
\texttt{ARGUMENT: \{'Ihr Vorschlag überschreitet den Maximalaufwand von 3207. Die Gegenstände C15, C28, C48, A56, B66, C60, B49, A35, A93 haben zusammen einen Aufwand von 3339, was 132 Punkte über dem Limit liegt. Wir müssen eine Kombination finden, die innerhalb der Grenzen bleibt.'\}} \\
\\ 
\texttt{VORSCHLAG: \{'C15', 'C28', 'A56', 'B66', 'C60', 'B49', 'A35', 'A93', 'C41'\}} \\
            }
        }
    }
    & \\ \\

    \theutterance \stepcounter{utterance}  
    & \multicolumn{4}{p{0.6\linewidth}}{
        \cellcolor[rgb]{0.9,0.9,0.9}{
            \makecell[{{p{\linewidth}}}]{
                \texttt{\tiny{[P1$\rangle$GM]}}
                \texttt{ABLEHNUNG: \{'C15', 'C28', 'A56', 'B66', 'C60', 'B49', 'A35', 'A93', 'C41'\}} \\
\\ 
\texttt{ARGUMENT: \{'Ihr Vorschlag hat einen Gesamtaufwand von 3218, was immer noch über dem Limit von 3207 liegt. Wir müssen weitere Anpassungen vornehmen. Ich schlage vor, dass wir einige der aufwändigeren Gegenstände durch effizientere Alternativen ersetzen.'\}} \\
\\ 
\texttt{VORSCHLAG: \{'C15', 'C28', 'A56', 'B05', 'B66', 'C60', 'A35', 'A93', 'C41'\}} \\
            }
        }
    }
    & & \\ \\

    \theutterance \stepcounter{utterance}  
    & & \multicolumn{4}{p{0.6\linewidth}}{
        \cellcolor[rgb]{0.9,0.9,0.9}{
            \makecell[{{p{\linewidth}}}]{
                \texttt{\tiny{[GM$\rangle$P2]}}
                \texttt{ABLEHNUNG: \{'C15', 'C28', 'A56', 'B66', 'C60', 'B49', 'A35', 'A93', 'C41'\}} \\
\\ 
\texttt{ARGUMENT: \{'Ihr Vorschlag hat einen Gesamtaufwand von 3218, was immer noch über dem Limit von 3207 liegt. Wir müssen weitere Anpassungen vornehmen. Ich schlage vor, dass wir einige der aufwändigeren Gegenstände durch effizientere Alternativen ersetzen.'\}} \\
\\ 
\texttt{VORSCHLAG: \{'C15', 'C28', 'A56', 'B05', 'B66', 'C60', 'A35', 'A93', 'C41'\}} \\
            }
        }
    }
    & \\ \\

    \theutterance \stepcounter{utterance}  
    & & & \multicolumn{4}{p{0.6\linewidth}}{
        \cellcolor[rgb]{0.9,0.9,0.9}{
            \makecell[{{p{\linewidth}}}]{
                \texttt{\tiny{[GM$\langle$P2]}}
                \texttt{ABLEHNUNG: \{'C15', 'C28', 'A56', 'B05', 'B66', 'C60', 'A35', 'A93', 'C41'\}} \\
\\ 
\texttt{ARGUMENT: \{'Ihr Vorschlag hat einen Gesamtaufwand von 3230, was noch immer das Limit von 3207 überschreitet. Außerdem sehe ich Potenzial für eine bessere Zusammenstellung. Lassen Sie mich einen alternativen Ansatz vorschlagen, der definitiv unter dem Limit bleibt.'\}} \\
\\ 
\texttt{VORSCHLAG: \{'C15', 'C28', 'A56', 'B66', 'C60', 'B49', 'A35', 'A93'\}} \\
            }
        }
    }
     \\ \\

    \theutterance \stepcounter{utterance}  
    & & \multicolumn{4}{p{0.6\linewidth}}{
        \cellcolor[rgb]{0.9,0.9,0.9}{
            \makecell[{{p{\linewidth}}}]{
                \texttt{\tiny{[P1$\langle$GM]}}
                \texttt{ABLEHNUNG: \{'C15', 'C28', 'A56', 'B05', 'B66', 'C60', 'A35', 'A93', 'C41'\}} \\
\\ 
\texttt{ARGUMENT: \{'Ihr Vorschlag hat einen Gesamtaufwand von 3230, was noch immer das Limit von 3207 überschreitet. Außerdem sehe ich Potenzial für eine bessere Zusammenstellung. Lassen Sie mich einen alternativen Ansatz vorschlagen, der definitiv unter dem Limit bleibt.'\}} \\
\\ 
\texttt{VORSCHLAG: \{'C15', 'C28', 'A56', 'B66', 'C60', 'B49', 'A35', 'A93'\}} \\
            }
        }
    }
    & \\ \\

    \theutterance \stepcounter{utterance}  
    & \multicolumn{4}{p{0.6\linewidth}}{
        \cellcolor[rgb]{0.9,0.9,0.9}{
            \makecell[{{p{\linewidth}}}]{
                \texttt{\tiny{[P1$\rangle$GM]}}
                \texttt{ABLEHNUNG: \{'C15', 'C28', 'A56', 'B66', 'C60', 'B49', 'A35', 'A93'\}} \\
\\ 
\texttt{ARGUMENT: \{'Ihr Vorschlag hat einen Gesamtaufwand von 3151 und liegt damit unter dem Limit, aber ich glaube, wir können den verfügbaren Spielraum besser nutzen. Wir haben noch 56 Punkte Aufwand zur Verfügung, die wir sinnvoll einsetzen könnten.'\}} \\
\\ 
\texttt{VORSCHLAG: \{'C15', 'C28', 'A56', 'B05', 'B66', 'C60', 'A35', 'A93', 'C41'\}} \\
            }
        }
    }
    & & \\ \\

    \theutterance \stepcounter{utterance}  
    & & \multicolumn{4}{p{0.6\linewidth}}{
        \cellcolor[rgb]{0.9,0.9,0.9}{
            \makecell[{{p{\linewidth}}}]{
                \texttt{\tiny{[GM$\rangle$P2]}}
                \texttt{ABLEHNUNG: \{'C15', 'C28', 'A56', 'B66', 'C60', 'B49', 'A35', 'A93'\}} \\
\\ 
\texttt{ARGUMENT: \{'Ihr Vorschlag hat einen Gesamtaufwand von 3151 und liegt damit unter dem Limit, aber ich glaube, wir können den verfügbaren Spielraum besser nutzen. Wir haben noch 56 Punkte Aufwand zur Verfügung, die wir sinnvoll einsetzen könnten.'\}} \\
\\ 
\texttt{VORSCHLAG: \{'C15', 'C28', 'A56', 'B05', 'B66', 'C60', 'A35', 'A93', 'C41'\}} \\
            }
        }
    }
    & \\ \\

    \theutterance \stepcounter{utterance}  
    & & & \multicolumn{4}{p{0.6\linewidth}}{
        \cellcolor[rgb]{0.9,0.9,0.9}{
            \makecell[{{p{\linewidth}}}]{
                \texttt{\tiny{[GM$\langle$P2]}}
                \texttt{ABLEHNUNG: \{'C15', 'C28', 'A56', 'B05', 'B66', 'C60', 'A35', 'A93', 'C41'\}} \\
\\ 
\texttt{ARGUMENT: \{'Wie ich bereits erwähnt habe, überschreitet diese Kombination mit einem Gesamtaufwand von 3230 das Limit von 3207. Wir drehen uns im Kreis. Lassen Sie mich einen anderen Ansatz vorschlagen, der sowohl effizient als auch machbar ist.'\}} \\
\\ 
\texttt{VORSCHLAG: \{'C15', 'C28', 'A56', 'B66', 'B49', 'A35', 'A93', 'B05', 'C41'\}} \\
            }
        }
    }
     \\ \\

    \theutterance \stepcounter{utterance}  
    & & \multicolumn{4}{p{0.6\linewidth}}{
        \cellcolor[rgb]{0.9,0.9,0.9}{
            \makecell[{{p{\linewidth}}}]{
                \texttt{\tiny{[P1$\langle$GM]}}
                \texttt{ABLEHNUNG: \{'C15', 'C28', 'A56', 'B05', 'B66', 'C60', 'A35', 'A93', 'C41'\}} \\
\\ 
\texttt{ARGUMENT: \{'Wie ich bereits erwähnt habe, überschreitet diese Kombination mit einem Gesamtaufwand von 3230 das Limit von 3207. Wir drehen uns im Kreis. Lassen Sie mich einen anderen Ansatz vorschlagen, der sowohl effizient als auch machbar ist.'\}} \\
\\ 
\texttt{VORSCHLAG: \{'C15', 'C28', 'A56', 'B66', 'B49', 'A35', 'A93', 'B05', 'C41'\}} \\
            }
        }
    }
    & \\ \\

    \theutterance \stepcounter{utterance}  
    & \multicolumn{4}{p{0.6\linewidth}}{
        \cellcolor[rgb]{0.9,0.9,0.9}{
            \makecell[{{p{\linewidth}}}]{
                \texttt{\tiny{[P1$\rangle$GM]}}
                \texttt{ABLEHNUNG: \{'C15', 'C28', 'A56', 'B66', 'B49', 'A35', 'A93', 'B05', 'C41'\}} \\
\\ 
\texttt{ARGUMENT: \{'Diese Kombination hat einen Gesamtaufwand von 3230 und überschreitet damit wieder das Limit. Sie schlagen immer wieder dieselben Gegenstände vor, die über dem Maximalaufwand liegen. Lassen Sie uns systematischer vorgehen und eine Lösung finden, die definitiv funktioniert.'\}} \\
\\ 
\texttt{VORSCHLAG: \{'C15', 'C28', 'A56', 'B66', 'B49', 'A35', 'A93', 'C41'\}} \\
            }
        }
    }
    & & \\ \\

    \theutterance \stepcounter{utterance}  
    & & \multicolumn{4}{p{0.6\linewidth}}{
        \cellcolor[rgb]{0.9,0.9,0.9}{
            \makecell[{{p{\linewidth}}}]{
                \texttt{\tiny{[GM$\rangle$P2]}}
                \texttt{ABLEHNUNG: \{'C15', 'C28', 'A56', 'B66', 'B49', 'A35', 'A93', 'B05', 'C41'\}} \\
\\ 
\texttt{ARGUMENT: \{'Diese Kombination hat einen Gesamtaufwand von 3230 und überschreitet damit wieder das Limit. Sie schlagen immer wieder dieselben Gegenstände vor, die über dem Maximalaufwand liegen. Lassen Sie uns systematischer vorgehen und eine Lösung finden, die definitiv funktioniert.'\}} \\
\\ 
\texttt{VORSCHLAG: \{'C15', 'C28', 'A56', 'B66', 'B49', 'A35', 'A93', 'C41'\}} \\
            }
        }
    }
    & \\ \\

    \theutterance \stepcounter{utterance}  
    & & & \multicolumn{4}{p{0.6\linewidth}}{
        \cellcolor[rgb]{0.9,0.9,0.9}{
            \makecell[{{p{\linewidth}}}]{
                \texttt{\tiny{[GM$\langle$P2]}}
                \texttt{ZUSTIMMUNG: \{'C15', 'C28', 'A56', 'B66', 'B49', 'A35', 'A93', 'C41'\}} \\
\\ 
\texttt{ARGUMENT: \{'Dieser Vorschlag hat einen Gesamtaufwand von 2908, was deutlich unter dem Limit von 3207 liegt. Das ist eine vernünftige Kombination, die beide Seiten zufriedenstellen sollte. Ich stimme diesem Deal zu.'\}} \\
            }
        }
    }
     \\ \\

    \theutterance \stepcounter{utterance}  
    & & & \multicolumn{2}{p{0.3\linewidth}}{
        \cellcolor[rgb]{0.9,0.9,0.9}{
            \makecell[{{p{\linewidth}}}]{
                \texttt{\tiny{[GM$|$GM]}}
                \texttt{game successful} \\
            }
        }
    }
    & & \\ \\

    \theutterance \stepcounter{utterance}  
    & & & \multicolumn{2}{p{0.3\linewidth}}{
        \cellcolor[rgb]{0.9,0.9,0.9}{
            \makecell[{{p{\linewidth}}}]{
                \texttt{\tiny{[GM$|$GM]}}
                \texttt{end game} \\
            }
        }
    }
    & & \\ \\

\end{supertabular}
}

\end{document}
