
\documentclass{article}
\usepackage{colortbl}
\usepackage{makecell}
\usepackage{multirow}
\usepackage{supertabular}

\begin{document}

\newcounter{utterance}

\centering \large Interaction Transcript for game `clean\_up', experiment `2\_hard\_3obj\_de', episode 1 with claude{-}sonnet{-}4{-}20250514{-}no{-}reasoning{-}t1.0.
\vspace{24pt}

{ \footnotesize  \setcounter{utterance}{1}
\setlength{\tabcolsep}{0pt}
\begin{supertabular}{c@{$\;$}|p{.15\linewidth}@{}p{.15\linewidth}p{.15\linewidth}p{.15\linewidth}p{.15\linewidth}p{.15\linewidth}}
    \# & $\;$A & \multicolumn{4}{c}{Game Master} & $\;\:$B\\
    \hline

    \theutterance \stepcounter{utterance}  
    & & \multicolumn{4}{p{0.6\linewidth}}{
        \cellcolor[rgb]{0.9,0.9,0.9}{
            \makecell[{{p{\linewidth}}}]{
                \texttt{\tiny{[P1$\langle$GM]}}
                \texttt{Ich bin der Spielleiter, und du spielst ein kooperatives Spiel mit dem folgenden Raster als Spielbrett:} \\
\\ 
\texttt{```} \\
\texttt{ 1234567} \\
\texttt{╔═══╤═╤═╗} \\
\texttt{║◌◌◌│◌│◌║ 1} \\
\texttt{╟─┬─┼─┴─╢ 2} \\
\texttt{║◌│◌│C◌◌║ 3} \\
\texttt{║◌├─┤◌┌─╢ 4} \\
\texttt{╟─┘◌│◌│L║ 5} \\
\texttt{║◌◌◌└┬┴─╢ 6} \\
\texttt{║P◌◌◌│◌◌║ 7} \\
\texttt{╚════╧══╝} \\
\\ 
\texttt{```} \\
\\ 
\texttt{* Am oberen Rand sind die x{-}Koordinaten angegeben, die nach rechts hin zunehmen, und am rechten Rand die y{-}Koordinaten, die nach unten zunehmen.} \\
\texttt{* Die folgenden Objekte sind zufällig auf deinem Raster platziert: 'C', 'L', 'P'.} \\
\\ 
\\ 
\texttt{Dein Mitspieler sieht eine Variation des Spielbretts, auf dem dieselben Objekte an anderen zufälligen Positionen platziert sind. Du kannst das Spielbrett des anderen Spielers nicht sehen, und er kann deins nicht sehen.} \\
\\ 
\\ 
\texttt{**Ziel des Spiels:**} \\
\\ 
\texttt{Beide Spieler müssen die Objekte auf ihrem jeweiligen Spielbrett so bewegen, dass identische Objekte am Ende an denselben Koordinaten liegen. Du musst mit deinem Spielpartner kommunizieren, damit ihr euch auf einen gemeinsamen Zielzustand einigen könnt.} \\
\\ 
\\ 
\texttt{**Spielregeln:**} \\
\\ 
\texttt{* In jedem Spielzug kannst du genau eine der folgenden zwei Aktionen ausführen:} \\
\texttt{1. `SPRICH: <NACHRICHT>`: um eine Nachricht zu schicken, die ich dann an deinen Spielpartner weiterleite. Als Nachricht gilt alles bis zum nächsten Zeilenumbruch.} \\
\texttt{2. `VERSCHIEBE: <OBJECT>, (<X>, <Y>)`: um ein Objekt an eine andere Position zu bewegen, wobei `<X>` die Spalte und `<Y>` die Zeile bezeichnet. Ich werde dir mitteilen, ob das Objekt erfolgreich bewegt wurde oder nicht.} \\
\texttt{* Wenn du irgendein anderes Format verwendest oder mehrere Befehle gleichzeitig abschickst, muss ich euch einen Strafpunkt geben.} \\
\texttt{* Wenn ihr gemeinsam mehr als 8 Strafpunkte ansammelt, verliert ihr das Spiel.} \\
\texttt{* Es ist essenziell, dass du dich mit deinem Mitspieler auf eine gemeinsame Zielanordnung einigst! Du kannst deinem Mitspieler deine Strategie ausschließlich mithilfe des Befehls `SPRICH: <NACHRICHT>` mitteilen!} \\
\\ 
\\ 
\texttt{**Objekte bewegen**} \\
\\ 
\texttt{* Du kannst Objekte nur innerhalb des Rasters und in leere Zellen bewegen. Leere Zellen erkennst am Zeichen '◌'.} \\
\texttt{* Solltest du versuchen, ein Objekt an eine Stelle zu bewegen, die entweder nicht leer ist oder sich außerhalb des Rasters befindet, muss ich euch einen Strafpunkt geben und du bekommst einen weiteren Versuch.} \\
\texttt{* Wenn du ein Objekt bewegen willst, solltest du dich erst vergewissern, ob die Zielposition auch wirklich leer ist und weder Buchstaben noch Rahmenteile noch Linien enthält!} \\
\\ 
\\ 
\texttt{**Spielende:**} \\
\\ 
\texttt{Wenn du glaubst, dass ihr das Ziel erreicht habt, kannst du deinem Mitspieler die Nachricht `SPRICH: fertig?` schicken. Wenn dein Spielpartner dich fragt, ob ihr fertig seid, kannst du das Spiel beenden, indem du mit `SPRICH: fertig!` antwortest.} \\
\\ 
\texttt{Ihr gewinnt gemeinsam, wenn ihr das Spiel innerhalb von \#max\_rounds Runden beendet. Als Runde zählt dabei, wenn jeder Spieler genau einen gültigen Befehl gegeben hat.} \\
\\ 
\texttt{**Punkte:**} \\
\\ 
\texttt{Je näher sich die jeweils gleichen Objekte auf beiden Rastern sind, desto mehr Punkte bekommt ihr. Für Strafpunkte gibt es Punktabzug. Könnt ihr den Rekord brechen?} \\
\\ 
\texttt{Schicke eine Nachricht an deinen Mitspieler um das Spiel zu starten!} \\
            }
        }
    }
    & \\ \\

    \theutterance \stepcounter{utterance}  
    & \multicolumn{4}{p{0.6\linewidth}}{
        \cellcolor[rgb]{0.9,0.9,0.9}{
            \makecell[{{p{\linewidth}}}]{
                \texttt{\tiny{[P1$\rangle$GM]}}
                \texttt{SPRICH: Hallo! Lass uns koordinieren. Meine Objekte sind: C bei (5,3), L bei (7,5), P bei (2,7). Wo sind deine Objekte? Wir sollten gemeinsame Zielkoordinaten festlegen.} \\
            }
        }
    }
    & & \\ \\

    \theutterance \stepcounter{utterance}  
    & & & \multicolumn{2}{p{0.3\linewidth}}{
        \cellcolor[rgb]{0.9,0.9,0.9}{
            \makecell[{{p{\linewidth}}}]{
                \texttt{\tiny{[GM$|$GM]}}
                \texttt{SPRICH: Hallo! Lass uns koordinieren. Meine Objekte sind: C bei (5,3), L bei (7,5), P bei (2,7). Wo sind deine Objekte? Wir sollten gemeinsame Zielkoordinaten festlegen.} \\
            }
        }
    }
    & & \\ \\

    \theutterance \stepcounter{utterance}  
    & & \multicolumn{4}{p{0.6\linewidth}}{
        \cellcolor[rgb]{0.9,0.9,0.9}{
            \makecell[{{p{\linewidth}}}]{
                \texttt{\tiny{[GM$\rangle$P2]}}
                \texttt{Ich bin der Spielleiter, und du spielst ein kooperatives Spiel mit dem folgenden Raster als Spielbrett:} \\
\\ 
\texttt{```} \\
\texttt{ 1234567} \\
\texttt{╔═══╤═╤═╗} \\
\texttt{║◌◌◌│◌│◌║ 1} \\
\texttt{╟─┬─┼─┴─╢ 2} \\
\texttt{║◌│L│◌P◌║ 3} \\
\texttt{║◌├─┤◌┌─╢ 4} \\
\texttt{╟─┘◌│◌│◌║ 5} \\
\texttt{║◌◌◌└┬┴─╢ 6} \\
\texttt{║◌C◌◌│◌◌║ 7} \\
\texttt{╚════╧══╝} \\
\\ 
\texttt{```} \\
\\ 
\texttt{* Am oberen Rand sind die x{-}Koordinaten angegeben, die nach rechts hin zunehmen, und am rechten Rand die y{-}Koordinaten, die nach unten zunehmen.} \\
\texttt{* Die folgenden Objekte sind zufällig auf deinem Raster platziert: 'C', 'L', 'P'.} \\
\\ 
\\ 
\texttt{Dein Mitspieler sieht eine Variation des Spielbretts, auf dem dieselben Objekte an anderen zufälligen Positionen platziert sind. Du kannst das Spielbrett des anderen Spielers nicht sehen, und er kann deins nicht sehen.} \\
\\ 
\\ 
\texttt{**Ziel des Spiels:**} \\
\\ 
\texttt{Beide Spieler müssen die Objekte auf ihrem jeweiligen Spielbrett so bewegen, dass identische Objekte am Ende an denselben Koordinaten liegen. Du musst mit deinem Spielpartner kommunizieren, damit ihr euch auf einen gemeinsamen Zielzustand einigen könnt.} \\
\\ 
\\ 
\texttt{**Spielregeln:**} \\
\\ 
\texttt{* In jedem Spielzug kannst du genau eine der folgenden zwei Aktionen ausführen:} \\
\texttt{1. `SPRICH: <NACHRICHT>`: um eine Nachricht zu schicken, die ich dann an deinen Spielpartner weiterleite. Als Nachricht gilt alles bis zum nächsten Zeilenumbruch.} \\
\texttt{2. `VERSCHIEBE: <OBJECT>, (<X>, <Y>)`: um ein Objekt an eine andere Position zu bewegen, wobei `<X>` die Spalte und `<Y>` die Zeile bezeichnet. Ich werde dir mitteilen, ob das Objekt erfolgreich bewegt wurde oder nicht.} \\
\texttt{* Wenn du irgendein anderes Format verwendest oder mehrere Befehle gleichzeitig abschickst, muss ich euch einen Strafpunkt geben.} \\
\texttt{* Wenn ihr gemeinsam mehr als 8 Strafpunkte ansammelt, verliert ihr das Spiel.} \\
\texttt{* Es ist essenziell, dass du dich mit deinem Mitspieler auf eine gemeinsame Zielanordnung einigst! Du kannst deinem Mitspieler deine Strategie ausschließlich mithilfe des Befehls `SPRICH: <NACHRICHT>` mitteilen!} \\
\\ 
\\ 
\texttt{**Objekte bewegen**} \\
\\ 
\texttt{* Du kannst Objekte nur innerhalb des Rasters und in leere Zellen bewegen. Leere Zellen erkennst am Zeichen '◌'.} \\
\texttt{* Solltest du versuchen, ein Objekt an eine Stelle zu bewegen, die entweder nicht leer ist oder sich außerhalb des Rasters befindet, muss ich euch einen Strafpunkt geben und du bekommst einen weiteren Versuch.} \\
\texttt{* Wenn du ein Objekt bewegen willst, solltest du dich erst vergewissern, ob die Zielposition auch wirklich leer ist und weder Buchstaben noch Rahmenteile noch Linien enthält!} \\
\\ 
\\ 
\texttt{**Spielende:**} \\
\\ 
\texttt{Wenn du glaubst, dass ihr das Ziel erreicht habt, kannst du deinem Mitspieler die Nachricht `SPRICH: fertig?` schicken. Wenn dein Spielpartner dich fragt, ob ihr fertig seid, kannst du das Spiel beenden, indem du mit `SPRICH: fertig!` antwortest.} \\
\\ 
\texttt{Ihr gewinnt gemeinsam, wenn ihr das Spiel innerhalb von \#max\_rounds Runden beendet. Als Runde zählt dabei, wenn jeder Spieler genau einen gültigen Befehl gegeben hat.} \\
\\ 
\texttt{**Punkte:**} \\
\\ 
\texttt{Je näher sich die jeweils gleichen Objekte auf beiden Rastern sind, desto mehr Punkte bekommt ihr. Für Strafpunkte gibt es Punktabzug. Könnt ihr den Rekord brechen?} \\
\\ 
\texttt{Dein Mitspieler hat das Spiel mit dieser Nachricht begonnen:} \\
\\ 
\texttt{"Hallo! Lass uns koordinieren. Meine Objekte sind: C bei (5,3), L bei (7,5), P bei (2,7). Wo sind deine Objekte? Wir sollten gemeinsame Zielkoordinaten festlegen."} \\
\\ 
\texttt{Wie lautet dein erster Befehl?} \\
            }
        }
    }
    & \\ \\

    \theutterance \stepcounter{utterance}  
    & & & \multicolumn{4}{p{0.6\linewidth}}{
        \cellcolor[rgb]{0.9,0.9,0.9}{
            \makecell[{{p{\linewidth}}}]{
                \texttt{\tiny{[GM$\langle$P2]}}
                \texttt{Ich schaue mir zuerst mein Spielbrett an und identifiziere die Positionen meiner Objekte:} \\
\\ 
\texttt{{-} L ist bei (3, 3)} \\
\texttt{{-} P ist bei (5, 3) } \\
\texttt{{-} C ist bei (3, 7)} \\
\\ 
\texttt{Jetzt antworte ich meinem Mitspieler:} \\
\\ 
\texttt{SPRICH: Hallo! Meine Objekte sind: L bei (3,3), P bei (5,3), C bei (3,7). Lass uns Zielkoordinaten festlegen. Ich schlage vor: C nach (4,4), L nach (4,5), P nach (4,6) {-} das sind zentrale Positionen die für beide gut erreichbar sein sollten. Was denkst du?} \\
            }
        }
    }
     \\ \\

    \theutterance \stepcounter{utterance}  
    & & & \multicolumn{2}{p{0.3\linewidth}}{
        \cellcolor[rgb]{0.9,0.9,0.9}{
            \makecell[{{p{\linewidth}}}]{
                \texttt{\tiny{[GM$|$GM]}}
                \texttt{Ich schaue mir zuerst mein Spielbrett an und identifiziere die Positionen meiner Objekte:} \\
\\ 
\texttt{{-} L ist bei (3, 3)} \\
\texttt{{-} P ist bei (5, 3) } \\
\texttt{{-} C ist bei (3, 7)} \\
\\ 
\texttt{Jetzt antworte ich meinem Mitspieler:} \\
\\ 
\texttt{SPRICH: Hallo! Meine Objekte sind: L bei (3,3), P bei (5,3), C bei (3,7). Lass uns Zielkoordinaten festlegen. Ich schlage vor: C nach (4,4), L nach (4,5), P nach (4,6) {-} das sind zentrale Positionen die für beide gut erreichbar sein sollten. Was denkst du?} \\
            }
        }
    }
    & & \\ \\

    \theutterance \stepcounter{utterance}  
    & & & \multicolumn{2}{p{0.3\linewidth}}{
        \cellcolor[rgb]{0.9,0.9,0.9}{
            \makecell[{{p{\linewidth}}}]{
                \texttt{\tiny{[GM$|$GM]}}
                \texttt{Invalid format: head is not empty: 'Ich schaue mir zuerst mein Spielbrett an und identifiziere die Positionen meiner Objekte:} \\
\\ 
\texttt{{-} L ist bei (3, 3)} \\
\texttt{{-} P ist bei (5, 3) } \\
\texttt{{-} C ist bei (3, 7)} \\
\\ 
\texttt{Jetzt antworte ich meinem Mitspieler:} \\
\\ 
\texttt{'} \\
            }
        }
    }
    & & \\ \\

    \theutterance \stepcounter{utterance}  
    & & \multicolumn{4}{p{0.6\linewidth}}{
        \cellcolor[rgb]{0.9,0.9,0.9}{
            \makecell[{{p{\linewidth}}}]{
                \texttt{\tiny{[GM$\rangle$P2]}}
                \texttt{Strafpunkt: Deine Nachricht darf nichts vor dem Befehl enthalten! Stelle sicher, dass deine Antwort ausschließlich einen der beiden Befehle `SPRICH: <NACHRICHT>` und `VERSCHIEBE: <OBJECT>, (<X>, <Y>)` und sonst nichts enthält!} \\
\\ 
\texttt{Ihr habt gemeinsam 1 von maximal 8 Strafpunkten angesammelt.} \\
\texttt{Bitte versuche es noch mal!} \\
            }
        }
    }
    & \\ \\

    \theutterance \stepcounter{utterance}  
    & & & \multicolumn{4}{p{0.6\linewidth}}{
        \cellcolor[rgb]{0.9,0.9,0.9}{
            \makecell[{{p{\linewidth}}}]{
                \texttt{\tiny{[GM$\langle$P2]}}
                \texttt{SPRICH: Hallo! Meine Objekte sind: L bei (3,3), P bei (5,3), C bei (3,7). Lass uns Zielkoordinaten festlegen. Ich schlage vor: C nach (4,4), L nach (4,5), P nach (4,6) {-} das sind zentrale Positionen die für beide gut erreichbar sein sollten. Was denkst du?} \\
            }
        }
    }
     \\ \\

    \theutterance \stepcounter{utterance}  
    & & & \multicolumn{2}{p{0.3\linewidth}}{
        \cellcolor[rgb]{0.9,0.9,0.9}{
            \makecell[{{p{\linewidth}}}]{
                \texttt{\tiny{[GM$|$GM]}}
                \texttt{SPRICH: Hallo! Meine Objekte sind: L bei (3,3), P bei (5,3), C bei (3,7). Lass uns Zielkoordinaten festlegen. Ich schlage vor: C nach (4,4), L nach (4,5), P nach (4,6) {-} das sind zentrale Positionen die für beide gut erreichbar sein sollten. Was denkst du?} \\
            }
        }
    }
    & & \\ \\

    \theutterance \stepcounter{utterance}  
    & & \multicolumn{4}{p{0.6\linewidth}}{
        \cellcolor[rgb]{0.9,0.9,0.9}{
            \makecell[{{p{\linewidth}}}]{
                \texttt{\tiny{[P1$\langle$GM]}}
                \texttt{Deine Nachricht wurde an deinen Mitspieler übermittelt.} \\
\\ 
\\ 
\texttt{Ihr seid gerade bei Runde 1 von maximal 12.} \\
\texttt{Ihr habt gemeinsam 1 von maximal 8 Strafpunkten angesammelt.} \\
\texttt{Dein Mitspieler hat dir diese Nachricht geschickt:} \\
\\ 
\\ 
\texttt{"Hallo! Meine Objekte sind: L bei (3,3), P bei (5,3), C bei (3,7). Lass uns Zielkoordinaten festlegen. Ich schlage vor: C nach (4,4), L nach (4,5), P nach (4,6) {-} das sind zentrale Positionen die für beide gut erreichbar sein sollten. Was denkst du?"} \\
\\ 
\\ 
\texttt{Wie lautet dein nächster Befehl?} \\
            }
        }
    }
    & \\ \\

    \theutterance \stepcounter{utterance}  
    & \multicolumn{4}{p{0.6\linewidth}}{
        \cellcolor[rgb]{0.9,0.9,0.9}{
            \makecell[{{p{\linewidth}}}]{
                \texttt{\tiny{[P1$\rangle$GM]}}
                \texttt{SPRICH: Gute Idee mit den zentralen Positionen! Ich überprüfe die vorgeschlagenen Koordinaten: (4,4), (4,5), (4,6). Bei mir sehe ich bei (4,4) das Zeichen '◌', bei (4,5) auch '◌', aber bei (4,6) sehe ich '└'. Können wir stattdessen (4,4) für C, (4,5) für L und (3,6) für P nehmen? Die Position (3,6) ist bei mir leer.} \\
            }
        }
    }
    & & \\ \\

    \theutterance \stepcounter{utterance}  
    & & & \multicolumn{2}{p{0.3\linewidth}}{
        \cellcolor[rgb]{0.9,0.9,0.9}{
            \makecell[{{p{\linewidth}}}]{
                \texttt{\tiny{[GM$|$GM]}}
                \texttt{SPRICH: Gute Idee mit den zentralen Positionen! Ich überprüfe die vorgeschlagenen Koordinaten: (4,4), (4,5), (4,6). Bei mir sehe ich bei (4,4) das Zeichen '◌', bei (4,5) auch '◌', aber bei (4,6) sehe ich '└'. Können wir stattdessen (4,4) für C, (4,5) für L und (3,6) für P nehmen? Die Position (3,6) ist bei mir leer.} \\
            }
        }
    }
    & & \\ \\

    \theutterance \stepcounter{utterance}  
    & & \multicolumn{4}{p{0.6\linewidth}}{
        \cellcolor[rgb]{0.9,0.9,0.9}{
            \makecell[{{p{\linewidth}}}]{
                \texttt{\tiny{[GM$\rangle$P2]}}
                \texttt{Deine Nachricht wurde an deinen Mitspieler übermittelt.} \\
\\ 
\\ 
\texttt{Ihr seid gerade bei Runde 2 von maximal 12.} \\
\texttt{Ihr habt gemeinsam 1 von maximal 8 Strafpunkten angesammelt.} \\
\texttt{Dein Mitspieler hat dir diese Nachricht geschickt:} \\
\\ 
\\ 
\texttt{"Gute Idee mit den zentralen Positionen! Ich überprüfe die vorgeschlagenen Koordinaten: (4,4), (4,5), (4,6). Bei mir sehe ich bei (4,4) das Zeichen '◌', bei (4,5) auch '◌', aber bei (4,6) sehe ich '└'. Können wir stattdessen (4,4) für C, (4,5) für L und (3,6) für P nehmen? Die Position (3,6) ist bei mir leer."} \\
\\ 
\\ 
\texttt{Wie lautet dein nächster Befehl?} \\
            }
        }
    }
    & \\ \\

    \theutterance \stepcounter{utterance}  
    & & & \multicolumn{4}{p{0.6\linewidth}}{
        \cellcolor[rgb]{0.9,0.9,0.9}{
            \makecell[{{p{\linewidth}}}]{
                \texttt{\tiny{[GM$\langle$P2]}}
                \texttt{SPRICH: Ja, das klingt gut! Ich überprüfe (3,6) {-} bei mir ist dort auch '◌', also leer. Dann nehmen wir: C nach (4,4), L nach (4,5), P nach (3,6). Ich fange an und bewege zuerst mein C von (3,7) nach (4,4).} \\
            }
        }
    }
     \\ \\

    \theutterance \stepcounter{utterance}  
    & & & \multicolumn{2}{p{0.3\linewidth}}{
        \cellcolor[rgb]{0.9,0.9,0.9}{
            \makecell[{{p{\linewidth}}}]{
                \texttt{\tiny{[GM$|$GM]}}
                \texttt{SPRICH: Ja, das klingt gut! Ich überprüfe (3,6) {-} bei mir ist dort auch '◌', also leer. Dann nehmen wir: C nach (4,4), L nach (4,5), P nach (3,6). Ich fange an und bewege zuerst mein C von (3,7) nach (4,4).} \\
            }
        }
    }
    & & \\ \\

    \theutterance \stepcounter{utterance}  
    & & \multicolumn{4}{p{0.6\linewidth}}{
        \cellcolor[rgb]{0.9,0.9,0.9}{
            \makecell[{{p{\linewidth}}}]{
                \texttt{\tiny{[P1$\langle$GM]}}
                \texttt{Deine Nachricht wurde an deinen Mitspieler übermittelt.} \\
\\ 
\\ 
\texttt{Ihr seid gerade bei Runde 2 von maximal 12.} \\
\texttt{Ihr habt gemeinsam 1 von maximal 8 Strafpunkten angesammelt.} \\
\texttt{Dein Mitspieler hat dir diese Nachricht geschickt:} \\
\\ 
\\ 
\texttt{"Ja, das klingt gut! Ich überprüfe (3,6) {-} bei mir ist dort auch '◌', also leer. Dann nehmen wir: C nach (4,4), L nach (4,5), P nach (3,6). Ich fange an und bewege zuerst mein C von (3,7) nach (4,4)."} \\
\\ 
\\ 
\texttt{Wie lautet dein nächster Befehl?} \\
            }
        }
    }
    & \\ \\

    \theutterance \stepcounter{utterance}  
    & \multicolumn{4}{p{0.6\linewidth}}{
        \cellcolor[rgb]{0.9,0.9,0.9}{
            \makecell[{{p{\linewidth}}}]{
                \texttt{\tiny{[P1$\rangle$GM]}}
                \texttt{SPRICH: Perfect! Ich bewege jetzt mein C von (5,3) nach (4,4).} \\
            }
        }
    }
    & & \\ \\

    \theutterance \stepcounter{utterance}  
    & & & \multicolumn{2}{p{0.3\linewidth}}{
        \cellcolor[rgb]{0.9,0.9,0.9}{
            \makecell[{{p{\linewidth}}}]{
                \texttt{\tiny{[GM$|$GM]}}
                \texttt{SPRICH: Perfect! Ich bewege jetzt mein C von (5,3) nach (4,4).} \\
            }
        }
    }
    & & \\ \\

    \theutterance \stepcounter{utterance}  
    & & \multicolumn{4}{p{0.6\linewidth}}{
        \cellcolor[rgb]{0.9,0.9,0.9}{
            \makecell[{{p{\linewidth}}}]{
                \texttt{\tiny{[GM$\rangle$P2]}}
                \texttt{Deine Nachricht wurde an deinen Mitspieler übermittelt.} \\
\\ 
\\ 
\texttt{Ihr seid gerade bei Runde 3 von maximal 12.} \\
\texttt{Ihr habt gemeinsam 1 von maximal 8 Strafpunkten angesammelt.} \\
\texttt{Dein Mitspieler hat dir diese Nachricht geschickt:} \\
\\ 
\\ 
\texttt{"Perfect! Ich bewege jetzt mein C von (5,3) nach (4,4)."} \\
\\ 
\\ 
\texttt{Wie lautet dein nächster Befehl?} \\
            }
        }
    }
    & \\ \\

    \theutterance \stepcounter{utterance}  
    & & & \multicolumn{4}{p{0.6\linewidth}}{
        \cellcolor[rgb]{0.9,0.9,0.9}{
            \makecell[{{p{\linewidth}}}]{
                \texttt{\tiny{[GM$\langle$P2]}}
                \texttt{VERSCHIEBE: C, (4, 4)} \\
            }
        }
    }
     \\ \\

    \theutterance \stepcounter{utterance}  
    & & & \multicolumn{2}{p{0.3\linewidth}}{
        \cellcolor[rgb]{0.9,0.9,0.9}{
            \makecell[{{p{\linewidth}}}]{
                \texttt{\tiny{[GM$|$GM]}}
                \texttt{VERSCHIEBE: C, (4, 4)} \\
            }
        }
    }
    & & \\ \\

    \theutterance \stepcounter{utterance}  
    & & & \multicolumn{2}{p{0.3\linewidth}}{
        \cellcolor[rgb]{0.9,0.9,0.9}{
            \makecell[{{p{\linewidth}}}]{
                \texttt{\tiny{[GM$|$GM]}}
                \texttt{Strafpunkt: (4,4) ist nicht leer, sondern enthält '┤'.} \\
\\ 
\texttt{Ihr habt gemeinsam 2 von maximal 8 Strafpunkten angesammelt.} \\
\texttt{Bitte versuche es noch mal!} \\
            }
        }
    }
    & & \\ \\

    \theutterance \stepcounter{utterance}  
    & & \multicolumn{4}{p{0.6\linewidth}}{
        \cellcolor[rgb]{0.9,0.9,0.9}{
            \makecell[{{p{\linewidth}}}]{
                \texttt{\tiny{[GM$\rangle$P2]}}
                \texttt{Strafpunkt: (4,4) ist nicht leer, sondern enthält '┤'.} \\
\\ 
\texttt{Ihr habt gemeinsam 2 von maximal 8 Strafpunkten angesammelt.} \\
\texttt{Bitte versuche es noch mal!} \\
            }
        }
    }
    & \\ \\

    \theutterance \stepcounter{utterance}  
    & & & \multicolumn{4}{p{0.6\linewidth}}{
        \cellcolor[rgb]{0.9,0.9,0.9}{
            \makecell[{{p{\linewidth}}}]{
                \texttt{\tiny{[GM$\langle$P2]}}
                \texttt{SPRICH: Entschuldigung, (4,4) ist bei mir nicht leer {-} dort ist '┤'. Lass uns andere Zielkoordinaten wählen. Welche Positionen sind bei dir definitiv leer? Ich kann folgende leere Positionen sehen: (1,1), (2,1), (3,1), (5,1), (7,1), (1,3), (3,5), (1,7), (3,7), (4,7), (6,7), (7,7).} \\
            }
        }
    }
     \\ \\

    \theutterance \stepcounter{utterance}  
    & & & \multicolumn{2}{p{0.3\linewidth}}{
        \cellcolor[rgb]{0.9,0.9,0.9}{
            \makecell[{{p{\linewidth}}}]{
                \texttt{\tiny{[GM$|$GM]}}
                \texttt{SPRICH: Entschuldigung, (4,4) ist bei mir nicht leer {-} dort ist '┤'. Lass uns andere Zielkoordinaten wählen. Welche Positionen sind bei dir definitiv leer? Ich kann folgende leere Positionen sehen: (1,1), (2,1), (3,1), (5,1), (7,1), (1,3), (3,5), (1,7), (3,7), (4,7), (6,7), (7,7).} \\
            }
        }
    }
    & & \\ \\

    \theutterance \stepcounter{utterance}  
    & & \multicolumn{4}{p{0.6\linewidth}}{
        \cellcolor[rgb]{0.9,0.9,0.9}{
            \makecell[{{p{\linewidth}}}]{
                \texttt{\tiny{[P1$\langle$GM]}}
                \texttt{Deine Nachricht wurde an deinen Mitspieler übermittelt.} \\
\\ 
\\ 
\texttt{Ihr seid gerade bei Runde 3 von maximal 12.} \\
\texttt{Ihr habt gemeinsam 2 von maximal 8 Strafpunkten angesammelt.} \\
\texttt{Dein Mitspieler hat dir diese Nachricht geschickt:} \\
\\ 
\\ 
\texttt{"Entschuldigung, (4,4) ist bei mir nicht leer {-} dort ist '┤'. Lass uns andere Zielkoordinaten wählen. Welche Positionen sind bei dir definitiv leer? Ich kann folgende leere Positionen sehen: (1,1), (2,1), (3,1), (5,1), (7,1), (1,3), (3,5), (1,7), (3,7), (4,7), (6,7), (7,7)."} \\
\\ 
\\ 
\texttt{Wie lautet dein nächster Befehl?} \\
            }
        }
    }
    & \\ \\

    \theutterance \stepcounter{utterance}  
    & \multicolumn{4}{p{0.6\linewidth}}{
        \cellcolor[rgb]{0.9,0.9,0.9}{
            \makecell[{{p{\linewidth}}}]{
                \texttt{\tiny{[P1$\rangle$GM]}}
                \texttt{SPRICH: Bei mir sind diese Positionen leer: (1,1), (2,1), (3,1), (5,1), (7,1), (1,3), (4,3), (6,3), (7,3), (1,4), (4,4), (4,5), (1,5), (3,5), (2,6), (3,6), (4,6), (2,7), (3,7), (4,7), (6,7), (7,7). Aus deiner Liste und meiner sind gemeinsam leer: (1,1), (2,1), (3,1), (5,1), (7,1), (1,3), (3,5), (1,7), (3,7), (4,7), (6,7), (7,7). Wie wäre: C nach (3,7), L nach (4,7), P nach (6,7)?} \\
            }
        }
    }
    & & \\ \\

    \theutterance \stepcounter{utterance}  
    & & & \multicolumn{2}{p{0.3\linewidth}}{
        \cellcolor[rgb]{0.9,0.9,0.9}{
            \makecell[{{p{\linewidth}}}]{
                \texttt{\tiny{[GM$|$GM]}}
                \texttt{SPRICH: Bei mir sind diese Positionen leer: (1,1), (2,1), (3,1), (5,1), (7,1), (1,3), (4,3), (6,3), (7,3), (1,4), (4,4), (4,5), (1,5), (3,5), (2,6), (3,6), (4,6), (2,7), (3,7), (4,7), (6,7), (7,7). Aus deiner Liste und meiner sind gemeinsam leer: (1,1), (2,1), (3,1), (5,1), (7,1), (1,3), (3,5), (1,7), (3,7), (4,7), (6,7), (7,7). Wie wäre: C nach (3,7), L nach (4,7), P nach (6,7)?} \\
            }
        }
    }
    & & \\ \\

    \theutterance \stepcounter{utterance}  
    & & \multicolumn{4}{p{0.6\linewidth}}{
        \cellcolor[rgb]{0.9,0.9,0.9}{
            \makecell[{{p{\linewidth}}}]{
                \texttt{\tiny{[GM$\rangle$P2]}}
                \texttt{Deine Nachricht wurde an deinen Mitspieler übermittelt.} \\
\\ 
\\ 
\texttt{Ihr seid gerade bei Runde 4 von maximal 12.} \\
\texttt{Ihr habt gemeinsam 2 von maximal 8 Strafpunkten angesammelt.} \\
\texttt{Dein Mitspieler hat dir diese Nachricht geschickt:} \\
\\ 
\\ 
\texttt{"Bei mir sind diese Positionen leer: (1,1), (2,1), (3,1), (5,1), (7,1), (1,3), (4,3), (6,3), (7,3), (1,4), (4,4), (4,5), (1,5), (3,5), (2,6), (3,6), (4,6), (2,7), (3,7), (4,7), (6,7), (7,7). Aus deiner Liste und meiner sind gemeinsam leer: (1,1), (2,1), (3,1), (5,1), (7,1), (1,3), (3,5), (1,7), (3,7), (4,7), (6,7), (7,7). Wie wäre: C nach (3,7), L nach (4,7), P nach (6,7)?"} \\
\\ 
\\ 
\texttt{Wie lautet dein nächster Befehl?} \\
            }
        }
    }
    & \\ \\

    \theutterance \stepcounter{utterance}  
    & & & \multicolumn{4}{p{0.6\linewidth}}{
        \cellcolor[rgb]{0.9,0.9,0.9}{
            \makecell[{{p{\linewidth}}}]{
                \texttt{\tiny{[GM$\langle$P2]}}
                \texttt{SPRICH: Perfekt! Die Positionen (3,7), (4,7), (6,7) sind bei mir alle leer. Also: C nach (3,7), L nach (4,7), P nach (6,7). Ich bewege zuerst mein C von (3,7) nach... Moment, mein C ist bereits bei (3,7)! Dann bewege ich als erstes mein L von (3,3) nach (4,7).} \\
            }
        }
    }
     \\ \\

    \theutterance \stepcounter{utterance}  
    & & & \multicolumn{2}{p{0.3\linewidth}}{
        \cellcolor[rgb]{0.9,0.9,0.9}{
            \makecell[{{p{\linewidth}}}]{
                \texttt{\tiny{[GM$|$GM]}}
                \texttt{SPRICH: Perfekt! Die Positionen (3,7), (4,7), (6,7) sind bei mir alle leer. Also: C nach (3,7), L nach (4,7), P nach (6,7). Ich bewege zuerst mein C von (3,7) nach... Moment, mein C ist bereits bei (3,7)! Dann bewege ich als erstes mein L von (3,3) nach (4,7).} \\
            }
        }
    }
    & & \\ \\

    \theutterance \stepcounter{utterance}  
    & & \multicolumn{4}{p{0.6\linewidth}}{
        \cellcolor[rgb]{0.9,0.9,0.9}{
            \makecell[{{p{\linewidth}}}]{
                \texttt{\tiny{[P1$\langle$GM]}}
                \texttt{Deine Nachricht wurde an deinen Mitspieler übermittelt.} \\
\\ 
\\ 
\texttt{Ihr seid gerade bei Runde 4 von maximal 12.} \\
\texttt{Ihr habt gemeinsam 2 von maximal 8 Strafpunkten angesammelt.} \\
\texttt{Dein Mitspieler hat dir diese Nachricht geschickt:} \\
\\ 
\\ 
\texttt{"Perfekt! Die Positionen (3,7), (4,7), (6,7) sind bei mir alle leer. Also: C nach (3,7), L nach (4,7), P nach (6,7). Ich bewege zuerst mein C von (3,7) nach... Moment, mein C ist bereits bei (3,7)! Dann bewege ich als erstes mein L von (3,3) nach (4,7)."} \\
\\ 
\\ 
\texttt{Wie lautet dein nächster Befehl?} \\
            }
        }
    }
    & \\ \\

    \theutterance \stepcounter{utterance}  
    & \multicolumn{4}{p{0.6\linewidth}}{
        \cellcolor[rgb]{0.9,0.9,0.9}{
            \makecell[{{p{\linewidth}}}]{
                \texttt{\tiny{[P1$\rangle$GM]}}
                \texttt{VERSCHIEBE: C, (3, 7)} \\
            }
        }
    }
    & & \\ \\

    \theutterance \stepcounter{utterance}  
    & & & \multicolumn{2}{p{0.3\linewidth}}{
        \cellcolor[rgb]{0.9,0.9,0.9}{
            \makecell[{{p{\linewidth}}}]{
                \texttt{\tiny{[GM$|$GM]}}
                \texttt{VERSCHIEBE: C, (3, 7)} \\
            }
        }
    }
    & & \\ \\

    \theutterance \stepcounter{utterance}  
    & & & \multicolumn{2}{p{0.3\linewidth}}{
        \cellcolor[rgb]{0.9,0.9,0.9}{
            \makecell[{{p{\linewidth}}}]{
                \texttt{\tiny{[GM$|$GM]}}
                \texttt{'C' erfolgreich nach (3,7) verschoben. Dein Raster sieht jetzt so aus:} \\
\texttt{```} \\
\texttt{ 1234567} \\
\texttt{╔═══╤═╤═╗} \\
\texttt{║◌◌◌│◌│◌║ 1} \\
\texttt{╟─┬─┼─┴─╢ 2} \\
\texttt{║◌│◌│◌◌◌║ 3} \\
\texttt{║◌├─┤◌┌─╢ 4} \\
\texttt{╟─┘◌│◌│L║ 5} \\
\texttt{║◌◌◌└┬┴─╢ 6} \\
\texttt{║P◌C◌│◌◌║ 7} \\
\texttt{╚════╧══╝} \\
\texttt{```} \\
            }
        }
    }
    & & \\ \\

    \theutterance \stepcounter{utterance}  
    & & \multicolumn{4}{p{0.6\linewidth}}{
        \cellcolor[rgb]{0.9,0.9,0.9}{
            \makecell[{{p{\linewidth}}}]{
                \texttt{\tiny{[GM$\rangle$P2]}}
                \texttt{Deine Nachricht wurde an deinen Mitspieler übermittelt.} \\
\\ 
\\ 
\texttt{Ihr seid gerade bei Runde 5 von maximal 12.} \\
\texttt{Ihr habt gemeinsam 2 von maximal 8 Strafpunkten angesammelt.} \\
\texttt{Dein Mitspieler hat ein Objekt auf seinem Raster bewegt.} \\
\\ 
\\ 
\texttt{Wie lautet dein nächster Befehl?} \\
            }
        }
    }
    & \\ \\

    \theutterance \stepcounter{utterance}  
    & & & \multicolumn{4}{p{0.6\linewidth}}{
        \cellcolor[rgb]{0.9,0.9,0.9}{
            \makecell[{{p{\linewidth}}}]{
                \texttt{\tiny{[GM$\langle$P2]}}
                \texttt{VERSCHIEBE: L, (4, 7)} \\
            }
        }
    }
     \\ \\

    \theutterance \stepcounter{utterance}  
    & & & \multicolumn{2}{p{0.3\linewidth}}{
        \cellcolor[rgb]{0.9,0.9,0.9}{
            \makecell[{{p{\linewidth}}}]{
                \texttt{\tiny{[GM$|$GM]}}
                \texttt{VERSCHIEBE: L, (4, 7)} \\
            }
        }
    }
    & & \\ \\

    \theutterance \stepcounter{utterance}  
    & & & \multicolumn{2}{p{0.3\linewidth}}{
        \cellcolor[rgb]{0.9,0.9,0.9}{
            \makecell[{{p{\linewidth}}}]{
                \texttt{\tiny{[GM$|$GM]}}
                \texttt{'L' erfolgreich nach (4,7) verschoben. Dein Raster sieht jetzt so aus:} \\
\texttt{```} \\
\texttt{ 1234567} \\
\texttt{╔═══╤═╤═╗} \\
\texttt{║◌◌◌│◌│◌║ 1} \\
\texttt{╟─┬─┼─┴─╢ 2} \\
\texttt{║◌│◌│◌P◌║ 3} \\
\texttt{║◌├─┤◌┌─╢ 4} \\
\texttt{╟─┘◌│◌│◌║ 5} \\
\texttt{║◌◌◌└┬┴─╢ 6} \\
\texttt{║◌C◌L│◌◌║ 7} \\
\texttt{╚════╧══╝} \\
\texttt{```} \\
            }
        }
    }
    & & \\ \\

    \theutterance \stepcounter{utterance}  
    & & \multicolumn{4}{p{0.6\linewidth}}{
        \cellcolor[rgb]{0.9,0.9,0.9}{
            \makecell[{{p{\linewidth}}}]{
                \texttt{\tiny{[P1$\langle$GM]}}
                \texttt{'C' erfolgreich nach (3,7) verschoben. Dein Raster sieht jetzt so aus:} \\
\texttt{```} \\
\texttt{ 1234567} \\
\texttt{╔═══╤═╤═╗} \\
\texttt{║◌◌◌│◌│◌║ 1} \\
\texttt{╟─┬─┼─┴─╢ 2} \\
\texttt{║◌│◌│◌◌◌║ 3} \\
\texttt{║◌├─┤◌┌─╢ 4} \\
\texttt{╟─┘◌│◌│L║ 5} \\
\texttt{║◌◌◌└┬┴─╢ 6} \\
\texttt{║P◌C◌│◌◌║ 7} \\
\texttt{╚════╧══╝} \\
\texttt{```} \\
\texttt{Ihr seid gerade bei Runde 5 von maximal 12.} \\
\texttt{Ihr habt gemeinsam 2 von maximal 8 Strafpunkten angesammelt.} \\
\texttt{Dein Mitspieler hat ein Objekt auf seinem Raster bewegt.} \\
\\ 
\\ 
\texttt{Wie lautet dein nächster Befehl?} \\
            }
        }
    }
    & \\ \\

    \theutterance \stepcounter{utterance}  
    & \multicolumn{4}{p{0.6\linewidth}}{
        \cellcolor[rgb]{0.9,0.9,0.9}{
            \makecell[{{p{\linewidth}}}]{
                \texttt{\tiny{[P1$\rangle$GM]}}
                \texttt{VERSCHIEBE: L, (4, 7)} \\
            }
        }
    }
    & & \\ \\

    \theutterance \stepcounter{utterance}  
    & & & \multicolumn{2}{p{0.3\linewidth}}{
        \cellcolor[rgb]{0.9,0.9,0.9}{
            \makecell[{{p{\linewidth}}}]{
                \texttt{\tiny{[GM$|$GM]}}
                \texttt{VERSCHIEBE: L, (4, 7)} \\
            }
        }
    }
    & & \\ \\

    \theutterance \stepcounter{utterance}  
    & & & \multicolumn{2}{p{0.3\linewidth}}{
        \cellcolor[rgb]{0.9,0.9,0.9}{
            \makecell[{{p{\linewidth}}}]{
                \texttt{\tiny{[GM$|$GM]}}
                \texttt{'L' erfolgreich nach (4,7) verschoben. Dein Raster sieht jetzt so aus:} \\
\texttt{```} \\
\texttt{ 1234567} \\
\texttt{╔═══╤═╤═╗} \\
\texttt{║◌◌◌│◌│◌║ 1} \\
\texttt{╟─┬─┼─┴─╢ 2} \\
\texttt{║◌│◌│◌◌◌║ 3} \\
\texttt{║◌├─┤◌┌─╢ 4} \\
\texttt{╟─┘◌│◌│◌║ 5} \\
\texttt{║◌◌◌└┬┴─╢ 6} \\
\texttt{║P◌CL│◌◌║ 7} \\
\texttt{╚════╧══╝} \\
\texttt{```} \\
            }
        }
    }
    & & \\ \\

    \theutterance \stepcounter{utterance}  
    & & \multicolumn{4}{p{0.6\linewidth}}{
        \cellcolor[rgb]{0.9,0.9,0.9}{
            \makecell[{{p{\linewidth}}}]{
                \texttt{\tiny{[GM$\rangle$P2]}}
                \texttt{'L' erfolgreich nach (4,7) verschoben. Dein Raster sieht jetzt so aus:} \\
\texttt{```} \\
\texttt{ 1234567} \\
\texttt{╔═══╤═╤═╗} \\
\texttt{║◌◌◌│◌│◌║ 1} \\
\texttt{╟─┬─┼─┴─╢ 2} \\
\texttt{║◌│◌│◌P◌║ 3} \\
\texttt{║◌├─┤◌┌─╢ 4} \\
\texttt{╟─┘◌│◌│◌║ 5} \\
\texttt{║◌◌◌└┬┴─╢ 6} \\
\texttt{║◌C◌L│◌◌║ 7} \\
\texttt{╚════╧══╝} \\
\texttt{```} \\
\texttt{Ihr seid gerade bei Runde 6 von maximal 12.} \\
\texttt{Ihr habt gemeinsam 2 von maximal 8 Strafpunkten angesammelt.} \\
\texttt{Dein Mitspieler hat ein Objekt auf seinem Raster bewegt.} \\
\\ 
\\ 
\texttt{Wie lautet dein nächster Befehl?} \\
            }
        }
    }
    & \\ \\

    \theutterance \stepcounter{utterance}  
    & & & \multicolumn{4}{p{0.6\linewidth}}{
        \cellcolor[rgb]{0.9,0.9,0.9}{
            \makecell[{{p{\linewidth}}}]{
                \texttt{\tiny{[GM$\langle$P2]}}
                \texttt{VERSCHIEBE: P, (6, 7)} \\
            }
        }
    }
     \\ \\

    \theutterance \stepcounter{utterance}  
    & & & \multicolumn{2}{p{0.3\linewidth}}{
        \cellcolor[rgb]{0.9,0.9,0.9}{
            \makecell[{{p{\linewidth}}}]{
                \texttt{\tiny{[GM$|$GM]}}
                \texttt{VERSCHIEBE: P, (6, 7)} \\
            }
        }
    }
    & & \\ \\

    \theutterance \stepcounter{utterance}  
    & & & \multicolumn{2}{p{0.3\linewidth}}{
        \cellcolor[rgb]{0.9,0.9,0.9}{
            \makecell[{{p{\linewidth}}}]{
                \texttt{\tiny{[GM$|$GM]}}
                \texttt{'P' erfolgreich nach (6,7) verschoben. Dein Raster sieht jetzt so aus:} \\
\texttt{```} \\
\texttt{ 1234567} \\
\texttt{╔═══╤═╤═╗} \\
\texttt{║◌◌◌│◌│◌║ 1} \\
\texttt{╟─┬─┼─┴─╢ 2} \\
\texttt{║◌│◌│◌◌◌║ 3} \\
\texttt{║◌├─┤◌┌─╢ 4} \\
\texttt{╟─┘◌│◌│◌║ 5} \\
\texttt{║◌◌◌└┬┴─╢ 6} \\
\texttt{║◌C◌L│P◌║ 7} \\
\texttt{╚════╧══╝} \\
\texttt{```} \\
            }
        }
    }
    & & \\ \\

    \theutterance \stepcounter{utterance}  
    & & \multicolumn{4}{p{0.6\linewidth}}{
        \cellcolor[rgb]{0.9,0.9,0.9}{
            \makecell[{{p{\linewidth}}}]{
                \texttt{\tiny{[P1$\langle$GM]}}
                \texttt{'L' erfolgreich nach (4,7) verschoben. Dein Raster sieht jetzt so aus:} \\
\texttt{```} \\
\texttt{ 1234567} \\
\texttt{╔═══╤═╤═╗} \\
\texttt{║◌◌◌│◌│◌║ 1} \\
\texttt{╟─┬─┼─┴─╢ 2} \\
\texttt{║◌│◌│◌◌◌║ 3} \\
\texttt{║◌├─┤◌┌─╢ 4} \\
\texttt{╟─┘◌│◌│◌║ 5} \\
\texttt{║◌◌◌└┬┴─╢ 6} \\
\texttt{║P◌CL│◌◌║ 7} \\
\texttt{╚════╧══╝} \\
\texttt{```} \\
\texttt{Ihr seid gerade bei Runde 6 von maximal 12.} \\
\texttt{Ihr habt gemeinsam 2 von maximal 8 Strafpunkten angesammelt.} \\
\texttt{Dein Mitspieler hat ein Objekt auf seinem Raster bewegt.} \\
\\ 
\\ 
\texttt{Wie lautet dein nächster Befehl?} \\
            }
        }
    }
    & \\ \\

    \theutterance \stepcounter{utterance}  
    & \multicolumn{4}{p{0.6\linewidth}}{
        \cellcolor[rgb]{0.9,0.9,0.9}{
            \makecell[{{p{\linewidth}}}]{
                \texttt{\tiny{[P1$\rangle$GM]}}
                \texttt{VERSCHIEBE: P, (6, 7)} \\
            }
        }
    }
    & & \\ \\

    \theutterance \stepcounter{utterance}  
    & & & \multicolumn{2}{p{0.3\linewidth}}{
        \cellcolor[rgb]{0.9,0.9,0.9}{
            \makecell[{{p{\linewidth}}}]{
                \texttt{\tiny{[GM$|$GM]}}
                \texttt{VERSCHIEBE: P, (6, 7)} \\
            }
        }
    }
    & & \\ \\

    \theutterance \stepcounter{utterance}  
    & & & \multicolumn{2}{p{0.3\linewidth}}{
        \cellcolor[rgb]{0.9,0.9,0.9}{
            \makecell[{{p{\linewidth}}}]{
                \texttt{\tiny{[GM$|$GM]}}
                \texttt{'P' erfolgreich nach (6,7) verschoben. Dein Raster sieht jetzt so aus:} \\
\texttt{```} \\
\texttt{ 1234567} \\
\texttt{╔═══╤═╤═╗} \\
\texttt{║◌◌◌│◌│◌║ 1} \\
\texttt{╟─┬─┼─┴─╢ 2} \\
\texttt{║◌│◌│◌◌◌║ 3} \\
\texttt{║◌├─┤◌┌─╢ 4} \\
\texttt{╟─┘◌│◌│◌║ 5} \\
\texttt{║◌◌◌└┬┴─╢ 6} \\
\texttt{║◌◌CL│P◌║ 7} \\
\texttt{╚════╧══╝} \\
\texttt{```} \\
            }
        }
    }
    & & \\ \\

    \theutterance \stepcounter{utterance}  
    & & \multicolumn{4}{p{0.6\linewidth}}{
        \cellcolor[rgb]{0.9,0.9,0.9}{
            \makecell[{{p{\linewidth}}}]{
                \texttt{\tiny{[GM$\rangle$P2]}}
                \texttt{'P' erfolgreich nach (6,7) verschoben. Dein Raster sieht jetzt so aus:} \\
\texttt{```} \\
\texttt{ 1234567} \\
\texttt{╔═══╤═╤═╗} \\
\texttt{║◌◌◌│◌│◌║ 1} \\
\texttt{╟─┬─┼─┴─╢ 2} \\
\texttt{║◌│◌│◌◌◌║ 3} \\
\texttt{║◌├─┤◌┌─╢ 4} \\
\texttt{╟─┘◌│◌│◌║ 5} \\
\texttt{║◌◌◌└┬┴─╢ 6} \\
\texttt{║◌C◌L│P◌║ 7} \\
\texttt{╚════╧══╝} \\
\texttt{```} \\
\texttt{Ihr seid gerade bei Runde 7 von maximal 12.} \\
\texttt{Ihr habt gemeinsam 2 von maximal 8 Strafpunkten angesammelt.} \\
\texttt{Dein Mitspieler hat ein Objekt auf seinem Raster bewegt.} \\
\\ 
\\ 
\texttt{Wie lautet dein nächster Befehl?} \\
            }
        }
    }
    & \\ \\

    \theutterance \stepcounter{utterance}  
    & & & \multicolumn{4}{p{0.6\linewidth}}{
        \cellcolor[rgb]{0.9,0.9,0.9}{
            \makecell[{{p{\linewidth}}}]{
                \texttt{\tiny{[GM$\langle$P2]}}
                \texttt{SPRICH: Perfekt! Ich habe alle meine Objekte an die Zielkoordinaten bewegt: C bei (3,7), L bei (4,7), P bei (6,7). Hast du auch alle deine Objekte an diese Positionen bewegt? Wenn ja, dann sind wir fertig!} \\
            }
        }
    }
     \\ \\

    \theutterance \stepcounter{utterance}  
    & & & \multicolumn{2}{p{0.3\linewidth}}{
        \cellcolor[rgb]{0.9,0.9,0.9}{
            \makecell[{{p{\linewidth}}}]{
                \texttt{\tiny{[GM$|$GM]}}
                \texttt{SPRICH: Perfekt! Ich habe alle meine Objekte an die Zielkoordinaten bewegt: C bei (3,7), L bei (4,7), P bei (6,7). Hast du auch alle deine Objekte an diese Positionen bewegt? Wenn ja, dann sind wir fertig!} \\
            }
        }
    }
    & & \\ \\

    \theutterance \stepcounter{utterance}  
    & & \multicolumn{4}{p{0.6\linewidth}}{
        \cellcolor[rgb]{0.9,0.9,0.9}{
            \makecell[{{p{\linewidth}}}]{
                \texttt{\tiny{[P1$\langle$GM]}}
                \texttt{'P' erfolgreich nach (6,7) verschoben. Dein Raster sieht jetzt so aus:} \\
\texttt{```} \\
\texttt{ 1234567} \\
\texttt{╔═══╤═╤═╗} \\
\texttt{║◌◌◌│◌│◌║ 1} \\
\texttt{╟─┬─┼─┴─╢ 2} \\
\texttt{║◌│◌│◌◌◌║ 3} \\
\texttt{║◌├─┤◌┌─╢ 4} \\
\texttt{╟─┘◌│◌│◌║ 5} \\
\texttt{║◌◌◌└┬┴─╢ 6} \\
\texttt{║◌◌CL│P◌║ 7} \\
\texttt{╚════╧══╝} \\
\texttt{```} \\
\texttt{Ihr seid gerade bei Runde 7 von maximal 12.} \\
\texttt{Ihr habt gemeinsam 2 von maximal 8 Strafpunkten angesammelt.} \\
\texttt{Dein Mitspieler hat dir diese Nachricht geschickt:} \\
\\ 
\\ 
\texttt{"Perfekt! Ich habe alle meine Objekte an die Zielkoordinaten bewegt: C bei (3,7), L bei (4,7), P bei (6,7). Hast du auch alle deine Objekte an diese Positionen bewegt? Wenn ja, dann sind wir fertig!"} \\
\\ 
\\ 
\texttt{Wie lautet dein nächster Befehl?} \\
            }
        }
    }
    & \\ \\

    \theutterance \stepcounter{utterance}  
    & \multicolumn{4}{p{0.6\linewidth}}{
        \cellcolor[rgb]{0.9,0.9,0.9}{
            \makecell[{{p{\linewidth}}}]{
                \texttt{\tiny{[P1$\rangle$GM]}}
                \texttt{SPRICH: fertig?} \\
            }
        }
    }
    & & \\ \\

    \theutterance \stepcounter{utterance}  
    & & & \multicolumn{2}{p{0.3\linewidth}}{
        \cellcolor[rgb]{0.9,0.9,0.9}{
            \makecell[{{p{\linewidth}}}]{
                \texttt{\tiny{[GM$|$GM]}}
                \texttt{SPRICH: fertig?} \\
            }
        }
    }
    & & \\ \\

    \theutterance \stepcounter{utterance}  
    & & \multicolumn{4}{p{0.6\linewidth}}{
        \cellcolor[rgb]{0.9,0.9,0.9}{
            \makecell[{{p{\linewidth}}}]{
                \texttt{\tiny{[GM$\rangle$P2]}}
                \texttt{Deine Nachricht wurde an deinen Mitspieler übermittelt.} \\
\\ 
\\ 
\texttt{Ihr seid gerade bei Runde 8 von maximal 12.} \\
\texttt{Ihr habt gemeinsam 2 von maximal 8 Strafpunkten angesammelt.} \\
\texttt{Dein Mitspieler hat dir diese Nachricht geschickt:} \\
\\ 
\\ 
\texttt{"fertig?"} \\
\\ 
\\ 
\texttt{Wie lautet dein nächster Befehl?} \\
            }
        }
    }
    & \\ \\

    \theutterance \stepcounter{utterance}  
    & & & \multicolumn{4}{p{0.6\linewidth}}{
        \cellcolor[rgb]{0.9,0.9,0.9}{
            \makecell[{{p{\linewidth}}}]{
                \texttt{\tiny{[GM$\langle$P2]}}
                \texttt{SPRICH: fertig!} \\
            }
        }
    }
     \\ \\

    \theutterance \stepcounter{utterance}  
    & & & \multicolumn{2}{p{0.3\linewidth}}{
        \cellcolor[rgb]{0.9,0.9,0.9}{
            \makecell[{{p{\linewidth}}}]{
                \texttt{\tiny{[GM$|$GM]}}
                \texttt{SPRICH: fertig!} \\
            }
        }
    }
    & & \\ \\

    \theutterance \stepcounter{utterance}  
    & & & \multicolumn{2}{p{0.3\linewidth}}{
        \cellcolor[rgb]{0.9,0.9,0.9}{
            \makecell[{{p{\linewidth}}}]{
                \texttt{\tiny{[GM$|$GM]}}
                \texttt{true} \\
            }
        }
    }
    & & \\ \\

    \theutterance \stepcounter{utterance}  
    & & & \multicolumn{2}{p{0.3\linewidth}}{
        \cellcolor[rgb]{0.9,0.9,0.9}{
            \makecell[{{p{\linewidth}}}]{
                \texttt{\tiny{[GM$|$GM]}}
                \texttt{* success: True} \\
\texttt{* lose: False} \\
\texttt{* aborted: False} \\
\texttt{{-}{-}{-}{-}{-}{-}{-}} \\
\texttt{* Shifts: 2.00} \\
\texttt{* Max Shifts: 4.00} \\
\texttt{* Min Shifts: 2.00} \\
\texttt{* End Distance Sum: 1.00} \\
\texttt{* Init Distance Sum: 15.88} \\
\texttt{* Expected Distance Sum: 12.57} \\
\texttt{* Penalties: 2.00} \\
\texttt{* Max Penalties: 8.00} \\
\texttt{* Rounds: 8.00} \\
\texttt{* Max Rounds: 12.00} \\
\texttt{* Object Count: 3.00} \\
            }
        }
    }
    & & \\ \\

    \theutterance \stepcounter{utterance}  
    & & & \multicolumn{2}{p{0.3\linewidth}}{
        \cellcolor[rgb]{0.9,0.9,0.9}{
            \makecell[{{p{\linewidth}}}]{
                \texttt{\tiny{[GM$|$GM]}}
                \texttt{Initial Grids:} \\
\texttt{```} \\
\texttt{Player 1:     Player 2:} \\
\texttt{ 1234567       1234567} \\
\texttt{╔═══╤═╤═╗     ╔═══╤═╤═╗} \\
\texttt{║◌◌◌│◌│◌║ 1   ║◌◌◌│◌│◌║ 1} \\
\texttt{╟─┬─┼─┴─╢ 2   ╟─┬─┼─┴─╢ 2} \\
\texttt{║◌│◌│C◌◌║ 3   ║◌│L│◌P◌║ 3} \\
\texttt{║◌├─┤◌┌─╢ 4   ║◌├─┤◌┌─╢ 4} \\
\texttt{╟─┘◌│◌│L║ 5   ╟─┘◌│◌│◌║ 5} \\
\texttt{║◌◌◌└┬┴─╢ 6   ║◌◌◌└┬┴─╢ 6} \\
\texttt{║P◌◌◌│◌◌║ 7   ║◌C◌◌│◌◌║ 7} \\
\texttt{╚════╧══╝     ╚════╧══╝} \\
\texttt{```} \\
\\ 
\texttt{Final Grids:} \\
\texttt{```} \\
\texttt{Player 1:     Player 2:} \\
\texttt{ 1234567       1234567} \\
\texttt{╔═══╤═╤═╗     ╔═══╤═╤═╗} \\
\texttt{║◌◌◌│◌│◌║ 1   ║◌◌◌│◌│◌║ 1} \\
\texttt{╟─┬─┼─┴─╢ 2   ╟─┬─┼─┴─╢ 2} \\
\texttt{║◌│◌│◌◌◌║ 3   ║◌│◌│◌◌◌║ 3} \\
\texttt{║◌├─┤◌┌─╢ 4   ║◌├─┤◌┌─╢ 4} \\
\texttt{╟─┘◌│◌│◌║ 5   ╟─┘◌│◌│◌║ 5} \\
\texttt{║◌◌◌└┬┴─╢ 6   ║◌◌◌└┬┴─╢ 6} \\
\texttt{║◌◌CL│P◌║ 7   ║◌C◌L│P◌║ 7} \\
\texttt{╚════╧══╝     ╚════╧══╝} \\
\texttt{```} \\
            }
        }
    }
    & & \\ \\

\end{supertabular}
}

\end{document}
