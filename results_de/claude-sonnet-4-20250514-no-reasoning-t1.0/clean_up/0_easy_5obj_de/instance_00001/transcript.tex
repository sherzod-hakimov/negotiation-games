
\documentclass{article}
\usepackage{colortbl}
\usepackage{makecell}
\usepackage{multirow}
\usepackage{supertabular}

\begin{document}

\newcounter{utterance}

\twocolumn

{ \footnotesize  \setcounter{utterance}{1}
\setlength{\tabcolsep}{0pt}
\begin{supertabular}{c@{$\;$}|p{.15\linewidth}@{}p{.15\linewidth}p{.15\linewidth}p{.15\linewidth}p{.15\linewidth}p{.15\linewidth}}

    \# & $\;$A & \multicolumn{4}{c}{Game Master} & $\;\:$B\\
    \hline 

    \theutterance \stepcounter{utterance}  

    & & \multicolumn{4}{p{0.6\linewidth}}{\cellcolor[rgb]{0.9,0.9,0.9}{%
	\makecell[{{p{\linewidth}}}]{% 
	  \tt {\tiny [A$\langle$GM]}  
	 Ich bin der Spielleiter, und du spielst ein kooperatives Spiel mit dem folgenden Raster als Spielbrett:\\ \tt \\ \tt ```\\ \tt  1234567\\ \tt ╔═══════╗\\ \tt ║I◌◌◌◌◌◌║ 1\\ \tt ║◌◌◌◌◌◌◌║ 2\\ \tt ║◌◌◌T◌┌─╢ 3\\ \tt ║◌◌◌┌─┤◌║ 4\\ \tt ║◌◌W│C├─╢ 5\\ \tt ╟───┼─┘H║ 6\\ \tt ║◌◌◌│◌◌◌║ 7\\ \tt ╚═══╧═══╝\\ \tt \\ \tt ```\\ \tt \\ \tt * Am oberen Rand sind die x-Koordinaten angegeben, die nach rechts hin zunehmen, und am rechten Rand die y-Koordinaten, die nach unten zunehmen.\\ \tt * Die folgenden Objekte sind zufällig auf deinem Raster platziert: 'W', 'I', 'T', 'C', 'H'.\\ \tt \\ \tt \\ \tt Dein Mitspieler sieht eine Variation des Spielbretts, auf dem dieselben Objekte an anderen zufälligen Positionen platziert sind. Du kannst das Spielbrett des anderen Spielers nicht sehen, und er kann deins nicht sehen.\\ \tt \\ \tt \\ \tt **Ziel des Spiels:**\\ \tt \\ \tt Beide Spieler müssen die Objekte auf ihrem jeweiligen Spielbrett so bewegen, dass identische Objekte am Ende an denselben Koordinaten liegen. Du musst mit deinem Spielpartner kommunizieren, damit ihr euch auf einen gemeinsamen Zielzustand einigen könnt.\\ \tt \\ \tt \\ \tt **Spielregeln:**\\ \tt \\ \tt * In jedem Spielzug kannst du genau eine der folgenden zwei Aktionen ausführen:\\ \tt 1. `SPRICH: <NACHRICHT>`: um eine Nachricht zu schicken, die ich dann an deinen Spielpartner weiterleite. Als Nachricht gilt alles bis zum nächsten Zeilenumbruch.\\ \tt 2. `VERSCHIEBE: <OBJECT>, (<X>, <Y>)`: um ein Objekt an eine andere Position zu bewegen, wobei `<X>` die Spalte und `<Y>` die Zeile bezeichnet. Ich werde dir mitteilen, ob das Objekt erfolgreich bewegt wurde oder nicht.\\ \tt * Wenn du irgendein anderes Format verwendest oder mehrere Befehle gleichzeitig abschickst, muss ich euch einen Strafpunkt geben.\\ \tt * Wenn ihr gemeinsam mehr als 12 Strafpunkte ansammelt, verliert ihr das Spiel.\\ \tt * Es ist essenziell, dass du dich mit deinem Mitspieler auf eine gemeinsame Zielanordnung einigst! Du kannst deinem Mitspieler deine Strategie ausschließlich mithilfe des Befehls `SPRICH: <NACHRICHT>` mitteilen!\\ \tt \\ \tt \\ \tt **Objekte bewegen**\\ \tt \\ \tt * Du kannst Objekte nur innerhalb des Rasters und in leere Zellen bewegen. Leere Zellen erkennst am Zeichen '◌'.\\ \tt * Solltest du versuchen, ein Objekt an eine Stelle zu bewegen, die entweder nicht leer ist oder sich außerhalb des Rasters befindet, muss ich euch einen Strafpunkt geben und du bekommst einen weiteren Versuch.\\ \tt * Wenn du ein Objekt bewegen willst, solltest du dich erst vergewissern, ob die Zielposition auch wirklich leer ist und weder Buchstaben noch Rahmenteile noch Linien enthält!\\ \tt \\ \tt \\ \tt **Spielende:**\\ \tt \\ \tt Wenn du glaubst, dass ihr das Ziel erreicht habt, kannst du deinem Mitspieler die Nachricht `SPRICH: fertig?` schicken. Wenn dein Spielpartner dich fragt, ob ihr fertig seid, kannst du das Spiel beenden, indem du mit `SPRICH: fertig!` antwortest.\\ \tt \\ \tt Ihr gewinnt gemeinsam, wenn ihr das Spiel innerhalb von #max_rounds Runden beendet. Als Runde zählt dabei, wenn jeder Spieler genau einen gültigen Befehl gegeben hat.\\ \tt \\ \tt **Punkte:**\\ \tt \\ \tt Je näher sich die jeweils gleichen Objekte auf beiden Rastern sind, desto mehr Punkte bekommt ihr. Für Strafpunkte gibt es Punktabzug. Könnt ihr den Rekord brechen?\\ \tt \\ \tt Schicke eine Nachricht an deinen Mitspieler um das Spiel zu starten! 
	  } 
	   } 
	   } 
	 & \\ 
 

    \theutterance \stepcounter{utterance}  

    & \multicolumn{4}{p{0.6\linewidth}}{\cellcolor[rgb]{0.8,1,0.9}{%
	\makecell[{{p{\linewidth}}}]{% 
	  \tt {\tiny [A$\rangle$GM]}  
	 Ich analysiere zunächst mein Spielbrett:\\ \tt \\ \tt ```\\ \tt  1234567\\ \tt ╔═══════╗\\ \tt ║I◌◌◌◌◌◌║ 1\\ \tt ║◌◌◌◌◌◌◌║ 2\\ \tt ║◌◌◌T◌┌─╢ 3\\ \tt ║◌◌◌┌─┤◌║ 4\\ \tt ║◌◌W│C├─╢ 5\\ \tt ║╟───┼─┘H║ 6\\ \tt ║◌◌◌│◌◌◌║ 7\\ \tt ╚═══╧═══╝\\ \tt ```\\ \tt \\ \tt Die Objekte befinden sich bei mir an folgenden Positionen:\\ \tt - I: (1,1)\\ \tt - T: (4,3) \\ \tt - W: (3,5)\\ \tt - C: (5,5)\\ \tt - H: (7,6)\\ \tt \\ \tt SPRICH: Hallo! Lass uns gemeinsam eine Zielanordnung festlegen. Meine Objekte stehen aktuell bei: I(1,1), T(4,3), W(3,5), C(5,5), H(7 
	  } 
	   } 
	   } 
	 & & \\ 
 

    \theutterance \stepcounter{utterance}  

    & & & \multicolumn{2}{p{0.3\linewidth}}{\cellcolor[rgb]{0.95,0.95,0.95}{%
	\makecell[{{p{\linewidth}}}]{% 
	  \tt {\tiny [GM$|$GM]}  
	 Ich analysiere zunächst mein Spielbrett:\\ \tt \\ \tt ```\\ \tt  1234567\\ \tt ╔═══════╗\\ \tt ║I◌◌◌◌◌◌║ 1\\ \tt ║◌◌◌◌◌◌◌║ 2\\ \tt ║◌◌◌T◌┌─╢ 3\\ \tt ║◌◌◌┌─┤◌║ 4\\ \tt ║◌◌W│C├─╢ 5\\ \tt ║╟───┼─┘H║ 6\\ \tt ║◌◌◌│◌◌◌║ 7\\ \tt ╚═══╧═══╝\\ \tt ```\\ \tt \\ \tt Die Objekte befinden sich bei mir an folgenden Positionen:\\ \tt - I: (1,1)\\ \tt - T: (4,3) \\ \tt - W: (3,5)\\ \tt - C: (5,5)\\ \tt - H: (7,6)\\ \tt \\ \tt SPRICH: Hallo! Lass uns gemeinsam eine Zielanordnung festlegen. Meine Objekte stehen aktuell bei: I(1,1), T(4,3), W(3,5), C(5,5), H(7 
	  } 
	   } 
	   } 
	 & & \\ 
 

    \theutterance \stepcounter{utterance}  

    & & & \multicolumn{2}{p{0.3\linewidth}}{\cellcolor[rgb]{0.95,0.95,0.95}{%
	\makecell[{{p{\linewidth}}}]{% 
	  \tt {\tiny [GM$|$GM]}  
	 Invalid format: head is not empty: 'Ich analysiere zunächst mein Spielbrett:\\ \tt \\ \tt \\ \tt  1234567\\ \tt ╔═══════╗\\ \tt ║I◌◌◌◌◌◌║ 1\\ \tt ║◌◌◌◌◌◌◌║ 2\\ \tt ║◌◌◌T◌┌─╢ 3\\ \tt ║◌◌◌┌─┤◌║ 4\\ \tt ║◌◌W│C├─╢ 5\\ \tt ║╟───┼─┘H║ 6\\ \tt ║◌◌◌│◌◌◌║ 7\\ \tt ╚═══╧═══╝\\ \tt \\ \tt \\ \tt Die Objekte befinden sich bei mir an folgenden Positionen:\\ \tt - I: (1,1)\\ \tt - T: (4,3) \\ \tt - W: (3,5)\\ \tt - C: (5,5)\\ \tt - H: (7,6)\\ \tt \\ \tt ' 
	  } 
	   } 
	   } 
	 & & \\ 
 

    \theutterance \stepcounter{utterance}  

    & & \multicolumn{4}{p{0.6\linewidth}}{\cellcolor[rgb]{0.9,0.9,0.9}{%
	\makecell[{{p{\linewidth}}}]{% 
	  \tt {\tiny [A$\langle$GM]}  
	 Strafpunkt: Deine Nachricht darf nichts vor dem Befehl enthalten! Stelle sicher, dass deine Antwort ausschließlich einen der beiden Befehle `SPRICH: <NACHRICHT>` und `VERSCHIEBE: <OBJECT>, (<X>, <Y>)` und sonst nichts enthält!\\ \tt \\ \tt Ihr habt gemeinsam 1 von maximal 12 Strafpunkten angesammelt.\\ \tt Bitte versuche es noch mal! 
	  } 
	   } 
	   } 
	 & \\ 
 

    \theutterance \stepcounter{utterance}  

    & \multicolumn{4}{p{0.6\linewidth}}{\cellcolor[rgb]{0.8,1,0.9}{%
	\makecell[{{p{\linewidth}}}]{% 
	  \tt {\tiny [A$\rangle$GM]}  
	 SPRICH: Hallo! Lass uns gemeinsam eine Zielanordnung festlegen. Meine Objekte stehen aktuell bei: I(1,1), T(4,3), W(3,5), C(5,5), H(7,6). Wo stehen deine Objekte? Ich schlage vor, wir einigen uns auf eine einfache Zielanordnung in der obersten Reihe von links nach rechts: W(1,1), I(2,1), T(3,1), C(4,1), H(5,1). Was denkst du? 
	  } 
	   } 
	   } 
	 & & \\ 
 

    \theutterance \stepcounter{utterance}  

    & & & \multicolumn{2}{p{0.3\linewidth}}{\cellcolor[rgb]{0.95,0.95,0.95}{%
	\makecell[{{p{\linewidth}}}]{% 
	  \tt {\tiny [GM$|$GM]}  
	 SPRICH: Hallo! Lass uns gemeinsam eine Zielanordnung festlegen. Meine Objekte stehen aktuell bei: I(1,1), T(4,3), W(3,5), C(5,5), H(7,6). Wo stehen deine Objekte? Ich schlage vor, wir einigen uns auf eine einfache Zielanordnung in der obersten Reihe von links nach rechts: W(1,1), I(2,1), T(3,1), C(4,1), H(5,1). Was denkst du? 
	  } 
	   } 
	   } 
	 & & \\ 
 

    \theutterance \stepcounter{utterance}  

    & & \multicolumn{4}{p{0.6\linewidth}}{\cellcolor[rgb]{0.9,0.9,0.9}{%
	\makecell[{{p{\linewidth}}}]{% 
	  \tt {\tiny [GM$\rangle$B]}  
	 Ich bin der Spielleiter, und du spielst ein kooperatives Spiel mit dem folgenden Raster als Spielbrett:\\ \tt \\ \tt ```\\ \tt  1234567\\ \tt ╔═══════╗\\ \tt ║◌◌H◌◌◌◌║ 1\\ \tt ║◌◌◌C◌◌◌║ 2\\ \tt ║◌◌◌◌◌┌─╢ 3\\ \tt ║◌◌◌┌─┤◌║ 4\\ \tt ║◌IW│◌├─╢ 5\\ \tt ╟───┼─┘◌║ 6\\ \tt ║T◌◌│◌◌◌║ 7\\ \tt ╚═══╧═══╝\\ \tt \\ \tt ```\\ \tt \\ \tt * Am oberen Rand sind die x-Koordinaten angegeben, die nach rechts hin zunehmen, und am rechten Rand die y-Koordinaten, die nach unten zunehmen.\\ \tt * Die folgenden Objekte sind zufällig auf deinem Raster platziert: 'W', 'I', 'T', 'C', 'H'.\\ \tt \\ \tt \\ \tt Dein Mitspieler sieht eine Variation des Spielbretts, auf dem dieselben Objekte an anderen zufälligen Positionen platziert sind. Du kannst das Spielbrett des anderen Spielers nicht sehen, und er kann deins nicht sehen.\\ \tt \\ \tt \\ \tt **Ziel des Spiels:**\\ \tt \\ \tt Beide Spieler müssen die Objekte auf ihrem jeweiligen Spielbrett so bewegen, dass identische Objekte am Ende an denselben Koordinaten liegen. Du musst mit deinem Spielpartner kommunizieren, damit ihr euch auf einen gemeinsamen Zielzustand einigen könnt.\\ \tt \\ \tt \\ \tt **Spielregeln:**\\ \tt \\ \tt * In jedem Spielzug kannst du genau eine der folgenden zwei Aktionen ausführen:\\ \tt 1. `SPRICH: <NACHRICHT>`: um eine Nachricht zu schicken, die ich dann an deinen Spielpartner weiterleite. Als Nachricht gilt alles bis zum nächsten Zeilenumbruch.\\ \tt 2. `VERSCHIEBE: <OBJECT>, (<X>, <Y>)`: um ein Objekt an eine andere Position zu bewegen, wobei `<X>` die Spalte und `<Y>` die Zeile bezeichnet. Ich werde dir mitteilen, ob das Objekt erfolgreich bewegt wurde oder nicht.\\ \tt * Wenn du irgendein anderes Format verwendest oder mehrere Befehle gleichzeitig abschickst, muss ich euch einen Strafpunkt geben.\\ \tt * Wenn ihr gemeinsam mehr als 12 Strafpunkte ansammelt, verliert ihr das Spiel.\\ \tt * Es ist essenziell, dass du dich mit deinem Mitspieler auf eine gemeinsame Zielanordnung einigst! Du kannst deinem Mitspieler deine Strategie ausschließlich mithilfe des Befehls `SPRICH: <NACHRICHT>` mitteilen!\\ \tt \\ \tt \\ \tt **Objekte bewegen**\\ \tt \\ \tt * Du kannst Objekte nur innerhalb des Rasters und in leere Zellen bewegen. Leere Zellen erkennst am Zeichen '◌'.\\ \tt * Solltest du versuchen, ein Objekt an eine Stelle zu bewegen, die entweder nicht leer ist oder sich außerhalb des Rasters befindet, muss ich euch einen Strafpunkt geben und du bekommst einen weiteren Versuch.\\ \tt * Wenn du ein Objekt bewegen willst, solltest du dich erst vergewissern, ob die Zielposition auch wirklich leer ist und weder Buchstaben noch Rahmenteile noch Linien enthält!\\ \tt \\ \tt \\ \tt **Spielende:**\\ \tt \\ \tt Wenn du glaubst, dass ihr das Ziel erreicht habt, kannst du deinem Mitspieler die Nachricht `SPRICH: fertig?` schicken. Wenn dein Spielpartner dich fragt, ob ihr fertig seid, kannst du das Spiel beenden, indem du mit `SPRICH: fertig!` antwortest.\\ \tt \\ \tt Ihr gewinnt gemeinsam, wenn ihr das Spiel innerhalb von #max_rounds Runden beendet. Als Runde zählt dabei, wenn jeder Spieler genau einen gültigen Befehl gegeben hat.\\ \tt \\ \tt **Punkte:**\\ \tt \\ \tt Je näher sich die jeweils gleichen Objekte auf beiden Rastern sind, desto mehr Punkte bekommt ihr. Für Strafpunkte gibt es Punktabzug. Könnt ihr den Rekord brechen?\\ \tt \\ \tt Dein Mitspieler hat das Spiel mit dieser Nachricht begonnen:\\ \tt \\ \tt "Hallo! Lass uns gemeinsam eine Zielanordnung festlegen. Meine Objekte stehen aktuell bei: I(1,1), T(4,3), W(3,5), C(5,5), H(7,6). Wo stehen deine Objekte? Ich schlage vor, wir einigen uns auf eine einfache Zielanordnung in der obersten Reihe von links nach rechts: W(1,1), I(2,1), T(3,1), C(4,1), H(5,1). Was denkst du?"\\ \tt \\ \tt Wie lautet dein erster Befehl? 
	  } 
	   } 
	   } 
	 & \\ 
 

    \theutterance \stepcounter{utterance}  

    & & & \multicolumn{4}{p{0.6\linewidth}}{\cellcolor[rgb]{1,0.85,0.72}{%
	\makecell[{{p{\linewidth}}}]{% 
	  \tt {\tiny [GM$\langle$B]}  
	 Ich schaue mir zuerst mein Spielbrett an und identifiziere die Positionen meiner Objekte:\\ \tt \\ \tt - H steht bei (3,1)\\ \tt - C steht bei (4,2)  \\ \tt - W steht bei (3,5)\\ \tt - I steht bei (2,5)\\ \tt - T steht bei (1,7)\\ \tt \\ \tt Mein Mitspieler schlägt eine Zielanordnung in der obersten Reihe vor: W(1,1), I(2,1), T(3,1), C(4,1), H(5,1). Das klingt nach einem guten Plan, da es einfach und systematisch ist.\\ \tt \\ \tt SPRICH: Hallo! Meine Objekte stehen bei: H(3,1), C(4,2), W(3,5), I(2,5), T(1,7). Dein Vorschlag für die Zielanordnung W(1,1), I(2,1), T(3,1), C(4,1), H(5,1) in der obersten Reihe gefällt mir sehr gut! Das ist klar und einfach umzusetzen. Lass uns das so machen. Ich begin 
	  } 
	   } 
	   } 
	  \\ 
 

    \theutterance \stepcounter{utterance}  

    & & & \multicolumn{2}{p{0.3\linewidth}}{\cellcolor[rgb]{0.95,0.95,0.95}{%
	\makecell[{{p{\linewidth}}}]{% 
	  \tt {\tiny [GM$|$GM]}  
	 Ich schaue mir zuerst mein Spielbrett an und identifiziere die Positionen meiner Objekte:\\ \tt \\ \tt - H steht bei (3,1)\\ \tt - C steht bei (4,2)  \\ \tt - W steht bei (3,5)\\ \tt - I steht bei (2,5)\\ \tt - T steht bei (1,7)\\ \tt \\ \tt Mein Mitspieler schlägt eine Zielanordnung in der obersten Reihe vor: W(1,1), I(2,1), T(3,1), C(4,1), H(5,1). Das klingt nach einem guten Plan, da es einfach und systematisch ist.\\ \tt \\ \tt SPRICH: Hallo! Meine Objekte stehen bei: H(3,1), C(4,2), W(3,5), I(2,5), T(1,7). Dein Vorschlag für die Zielanordnung W(1,1), I(2,1), T(3,1), C(4,1), H(5,1) in der obersten Reihe gefällt mir sehr gut! Das ist klar und einfach umzusetzen. Lass uns das so machen. Ich begin 
	  } 
	   } 
	   } 
	 & & \\ 
 

    \theutterance \stepcounter{utterance}  

    & & & \multicolumn{2}{p{0.3\linewidth}}{\cellcolor[rgb]{0.95,0.95,0.95}{%
	\makecell[{{p{\linewidth}}}]{% 
	  \tt {\tiny [GM$|$GM]}  
	 Invalid format: head is not empty: 'Ich schaue mir zuerst mein Spielbrett an und identifiziere die Positionen meiner Objekte:\\ \tt \\ \tt - H steht bei (3,1)\\ \tt - C steht bei (4,2)  \\ \tt - W steht bei (3,5)\\ \tt - I steht bei (2,5)\\ \tt - T steht bei (1,7)\\ \tt \\ \tt Mein Mitspieler schlägt eine Zielanordnung in der obersten Reihe vor: W(1,1), I(2,1), T(3,1), C(4,1), H(5,1). Das klingt nach einem guten Plan, da es einfach und systematisch ist.\\ \tt \\ \tt ' 
	  } 
	   } 
	   } 
	 & & \\ 
 

    \theutterance \stepcounter{utterance}  

    & & \multicolumn{4}{p{0.6\linewidth}}{\cellcolor[rgb]{0.9,0.9,0.9}{%
	\makecell[{{p{\linewidth}}}]{% 
	  \tt {\tiny [GM$\rangle$B]}  
	 Strafpunkt: Deine Nachricht darf nichts vor dem Befehl enthalten! Stelle sicher, dass deine Antwort ausschließlich einen der beiden Befehle `SPRICH: <NACHRICHT>` und `VERSCHIEBE: <OBJECT>, (<X>, <Y>)` und sonst nichts enthält!\\ \tt \\ \tt Ihr habt gemeinsam 2 von maximal 12 Strafpunkten angesammelt.\\ \tt Bitte versuche es noch mal! 
	  } 
	   } 
	   } 
	 & \\ 
 

    \theutterance \stepcounter{utterance}  

    & & & \multicolumn{4}{p{0.6\linewidth}}{\cellcolor[rgb]{1,0.85,0.72}{%
	\makecell[{{p{\linewidth}}}]{% 
	  \tt {\tiny [GM$\langle$B]}  
	 SPRICH: Hallo! Meine Objekte stehen bei: H(3,1), C(4,2), W(3,5), I(2,5), T(1,7). Dein Vorschlag für die Zielanordnung W(1,1), I(2,1), T(3,1), C(4,1), H(5,1) in der obersten Reihe gefällt mir sehr gut! Das ist klar und einfach umzusetzen. Lass uns das so machen. Ich beginne mit den Bewegungen. 
	  } 
	   } 
	   } 
	  \\ 
 

    \theutterance \stepcounter{utterance}  

    & & & \multicolumn{2}{p{0.3\linewidth}}{\cellcolor[rgb]{0.95,0.95,0.95}{%
	\makecell[{{p{\linewidth}}}]{% 
	  \tt {\tiny [GM$|$GM]}  
	 SPRICH: Hallo! Meine Objekte stehen bei: H(3,1), C(4,2), W(3,5), I(2,5), T(1,7). Dein Vorschlag für die Zielanordnung W(1,1), I(2,1), T(3,1), C(4,1), H(5,1) in der obersten Reihe gefällt mir sehr gut! Das ist klar und einfach umzusetzen. Lass uns das so machen. Ich beginne mit den Bewegungen. 
	  } 
	   } 
	   } 
	 & & \\ 
 

    \theutterance \stepcounter{utterance}  

    & & \multicolumn{4}{p{0.6\linewidth}}{\cellcolor[rgb]{0.9,0.9,0.9}{%
	\makecell[{{p{\linewidth}}}]{% 
	  \tt {\tiny [A$\langle$GM]}  
	 Deine Nachricht wurde an deinen Mitspieler übermittelt.\\ \tt \\ \tt \\ \tt Ihr seid gerade bei Runde 1 von maximal 20.\\ \tt Ihr habt gemeinsam 2 von maximal 12 Strafpunkten angesammelt.\\ \tt Dein Mitspieler hat dir diese Nachricht geschickt:\\ \tt \\ \tt \\ \tt "Hallo! Meine Objekte stehen bei: H(3,1), C(4,2), W(3,5), I(2,5), T(1,7). Dein Vorschlag für die Zielanordnung W(1,1), I(2,1), T(3,1), C(4,1), H(5,1) in der obersten Reihe gefällt mir sehr gut! Das ist klar und einfach umzusetzen. Lass uns das so machen. Ich beginne mit den Bewegungen."\\ \tt \\ \tt \\ \tt Wie lautet dein nächster Befehl? 
	  } 
	   } 
	   } 
	 & \\ 
 

    \theutterance \stepcounter{utterance}  

    & \multicolumn{4}{p{0.6\linewidth}}{\cellcolor[rgb]{0.8,1,0.9}{%
	\makecell[{{p{\linewidth}}}]{% 
	  \tt {\tiny [A$\rangle$GM]}  
	 SPRICH: Perfekt! Dann machen wir das so: W(1,1), I(2,1), T(3,1), C(4,1), H(5,1). Ich beginne auch mit meinen Bewegungen. Zuerst verschiebe ich W von (3,5) nach (1,1). 
	  } 
	   } 
	   } 
	 & & \\ 
 

    \theutterance \stepcounter{utterance}  

    & & & \multicolumn{2}{p{0.3\linewidth}}{\cellcolor[rgb]{0.95,0.95,0.95}{%
	\makecell[{{p{\linewidth}}}]{% 
	  \tt {\tiny [GM$|$GM]}  
	 SPRICH: Perfekt! Dann machen wir das so: W(1,1), I(2,1), T(3,1), C(4,1), H(5,1). Ich beginne auch mit meinen Bewegungen. Zuerst verschiebe ich W von (3,5) nach (1,1). 
	  } 
	   } 
	   } 
	 & & \\ 
 

    \theutterance \stepcounter{utterance}  

    & & \multicolumn{4}{p{0.6\linewidth}}{\cellcolor[rgb]{0.9,0.9,0.9}{%
	\makecell[{{p{\linewidth}}}]{% 
	  \tt {\tiny [GM$\rangle$B]}  
	 Deine Nachricht wurde an deinen Mitspieler übermittelt.\\ \tt \\ \tt \\ \tt Ihr seid gerade bei Runde 2 von maximal 20.\\ \tt Ihr habt gemeinsam 2 von maximal 12 Strafpunkten angesammelt.\\ \tt Dein Mitspieler hat dir diese Nachricht geschickt:\\ \tt \\ \tt \\ \tt "Perfekt! Dann machen wir das so: W(1,1), I(2,1), T(3,1), C(4,1), H(5,1). Ich beginne auch mit meinen Bewegungen. Zuerst verschiebe ich W von (3,5) nach (1,1)."\\ \tt \\ \tt \\ \tt Wie lautet dein nächster Befehl? 
	  } 
	   } 
	   } 
	 & \\ 
 

    \theutterance \stepcounter{utterance}  

    & & & \multicolumn{4}{p{0.6\linewidth}}{\cellcolor[rgb]{1,0.85,0.72}{%
	\makecell[{{p{\linewidth}}}]{% 
	  \tt {\tiny [GM$\langle$B]}  
	 VERSCHIEBE: W, (1,1) 
	  } 
	   } 
	   } 
	  \\ 
 

    \theutterance \stepcounter{utterance}  

    & & & \multicolumn{2}{p{0.3\linewidth}}{\cellcolor[rgb]{0.95,0.95,0.95}{%
	\makecell[{{p{\linewidth}}}]{% 
	  \tt {\tiny [GM$|$GM]}  
	 VERSCHIEBE: W, (1,1) 
	  } 
	   } 
	   } 
	 & & \\ 
 

    \theutterance \stepcounter{utterance}  

    & & & \multicolumn{2}{p{0.3\linewidth}}{\cellcolor[rgb]{0.95,0.95,0.95}{%
	\makecell[{{p{\linewidth}}}]{% 
	  \tt {\tiny [GM$|$GM]}  
	 'W' erfolgreich nach (1,1) verschoben. Dein Raster sieht jetzt so aus:\\ \tt ```\\ \tt  1234567\\ \tt ╔═══════╗\\ \tt ║W◌H◌◌◌◌║ 1\\ \tt ║◌◌◌C◌◌◌║ 2\\ \tt ║◌◌◌◌◌┌─╢ 3\\ \tt ║◌◌◌┌─┤◌║ 4\\ \tt ║◌I◌│◌├─╢ 5\\ \tt ╟───┼─┘◌║ 6\\ \tt ║T◌◌│◌◌◌║ 7\\ \tt ╚═══╧═══╝\\ \tt ```\\ \tt  
	  } 
	   } 
	   } 
	 & & \\ 
 

    \theutterance \stepcounter{utterance}  

    & & \multicolumn{4}{p{0.6\linewidth}}{\cellcolor[rgb]{0.9,0.9,0.9}{%
	\makecell[{{p{\linewidth}}}]{% 
	  \tt {\tiny [A$\langle$GM]}  
	 Deine Nachricht wurde an deinen Mitspieler übermittelt.\\ \tt \\ \tt \\ \tt Ihr seid gerade bei Runde 2 von maximal 20.\\ \tt Ihr habt gemeinsam 2 von maximal 12 Strafpunkten angesammelt.\\ \tt Dein Mitspieler hat ein Objekt auf seinem Raster bewegt.\\ \tt \\ \tt \\ \tt Wie lautet dein nächster Befehl? 
	  } 
	   } 
	   } 
	 & \\ 
 

    \theutterance \stepcounter{utterance}  

    & \multicolumn{4}{p{0.6\linewidth}}{\cellcolor[rgb]{0.8,1,0.9}{%
	\makecell[{{p{\linewidth}}}]{% 
	  \tt {\tiny [A$\rangle$GM]}  
	 VERSCHIEBE: W, (1, 1) 
	  } 
	   } 
	   } 
	 & & \\ 
 

    \theutterance \stepcounter{utterance}  

    & & & \multicolumn{2}{p{0.3\linewidth}}{\cellcolor[rgb]{0.95,0.95,0.95}{%
	\makecell[{{p{\linewidth}}}]{% 
	  \tt {\tiny [GM$|$GM]}  
	 VERSCHIEBE: W, (1, 1) 
	  } 
	   } 
	   } 
	 & & \\ 
 

    \theutterance \stepcounter{utterance}  

    & & & \multicolumn{2}{p{0.3\linewidth}}{\cellcolor[rgb]{0.95,0.95,0.95}{%
	\makecell[{{p{\linewidth}}}]{% 
	  \tt {\tiny [GM$|$GM]}  
	 Strafpunkt: (1,1) ist nicht leer, sondern enthält 'I'.\\ \tt \\ \tt Ihr habt gemeinsam 3 von maximal 12 Strafpunkten angesammelt.\\ \tt Bitte versuche es noch mal! 
	  } 
	   } 
	   } 
	 & & \\ 
 

    \theutterance \stepcounter{utterance}  

    & & \multicolumn{4}{p{0.6\linewidth}}{\cellcolor[rgb]{0.9,0.9,0.9}{%
	\makecell[{{p{\linewidth}}}]{% 
	  \tt {\tiny [A$\langle$GM]}  
	 Strafpunkt: (1,1) ist nicht leer, sondern enthält 'I'.\\ \tt \\ \tt Ihr habt gemeinsam 3 von maximal 12 Strafpunkten angesammelt.\\ \tt Bitte versuche es noch mal! 
	  } 
	   } 
	   } 
	 & \\ 
 

    \theutterance \stepcounter{utterance}  

    & \multicolumn{4}{p{0.6\linewidth}}{\cellcolor[rgb]{0.8,1,0.9}{%
	\makecell[{{p{\linewidth}}}]{% 
	  \tt {\tiny [A$\rangle$GM]}  
	 VERSCHIEBE: W, (2, 2) 
	  } 
	   } 
	   } 
	 & & \\ 
 

    \theutterance \stepcounter{utterance}  

    & & & \multicolumn{2}{p{0.3\linewidth}}{\cellcolor[rgb]{0.95,0.95,0.95}{%
	\makecell[{{p{\linewidth}}}]{% 
	  \tt {\tiny [GM$|$GM]}  
	 VERSCHIEBE: W, (2, 2) 
	  } 
	   } 
	   } 
	 & & \\ 
 

    \theutterance \stepcounter{utterance}  

    & & & \multicolumn{2}{p{0.3\linewidth}}{\cellcolor[rgb]{0.95,0.95,0.95}{%
	\makecell[{{p{\linewidth}}}]{% 
	  \tt {\tiny [GM$|$GM]}  
	 'W' erfolgreich nach (2,2) verschoben. Dein Raster sieht jetzt so aus:\\ \tt ```\\ \tt  1234567\\ \tt ╔═══════╗\\ \tt ║I◌◌◌◌◌◌║ 1\\ \tt ║◌W◌◌◌◌◌║ 2\\ \tt ║◌◌◌T◌┌─╢ 3\\ \tt ║◌◌◌┌─┤◌║ 4\\ \tt ║◌◌◌│C├─╢ 5\\ \tt ╟───┼─┘H║ 6\\ \tt ║◌◌◌│◌◌◌║ 7\\ \tt ╚═══╧═══╝\\ \tt ```\\ \tt  
	  } 
	   } 
	   } 
	 & & \\ 
 

    \theutterance \stepcounter{utterance}  

    & & \multicolumn{4}{p{0.6\linewidth}}{\cellcolor[rgb]{0.9,0.9,0.9}{%
	\makecell[{{p{\linewidth}}}]{% 
	  \tt {\tiny [GM$\rangle$B]}  
	 'W' erfolgreich nach (1,1) verschoben. Dein Raster sieht jetzt so aus:\\ \tt ```\\ \tt  1234567\\ \tt ╔═══════╗\\ \tt ║W◌H◌◌◌◌║ 1\\ \tt ║◌◌◌C◌◌◌║ 2\\ \tt ║◌◌◌◌◌┌─╢ 3\\ \tt ║◌◌◌┌─┤◌║ 4\\ \tt ║◌I◌│◌├─╢ 5\\ \tt ╟───┼─┘◌║ 6\\ \tt ║T◌◌│◌◌◌║ 7\\ \tt ╚═══╧═══╝\\ \tt ```\\ \tt Ihr seid gerade bei Runde 3 von maximal 20.\\ \tt Ihr habt gemeinsam 3 von maximal 12 Strafpunkten angesammelt.\\ \tt Dein Mitspieler hat ein Objekt auf seinem Raster bewegt.\\ \tt \\ \tt \\ \tt Wie lautet dein nächster Befehl? 
	  } 
	   } 
	   } 
	 & \\ 
 

    \theutterance \stepcounter{utterance}  

    & & & \multicolumn{4}{p{0.6\linewidth}}{\cellcolor[rgb]{1,0.85,0.72}{%
	\makecell[{{p{\linewidth}}}]{% 
	  \tt {\tiny [GM$\langle$B]}  
	 VERSCHIEBE: I, (2,1) 
	  } 
	   } 
	   } 
	  \\ 
 

    \theutterance \stepcounter{utterance}  

    & & & \multicolumn{2}{p{0.3\linewidth}}{\cellcolor[rgb]{0.95,0.95,0.95}{%
	\makecell[{{p{\linewidth}}}]{% 
	  \tt {\tiny [GM$|$GM]}  
	 VERSCHIEBE: I, (2,1) 
	  } 
	   } 
	   } 
	 & & \\ 
 

    \theutterance \stepcounter{utterance}  

    & & & \multicolumn{2}{p{0.3\linewidth}}{\cellcolor[rgb]{0.95,0.95,0.95}{%
	\makecell[{{p{\linewidth}}}]{% 
	  \tt {\tiny [GM$|$GM]}  
	 'I' erfolgreich nach (2,1) verschoben. Dein Raster sieht jetzt so aus:\\ \tt ```\\ \tt  1234567\\ \tt ╔═══════╗\\ \tt ║WIH◌◌◌◌║ 1\\ \tt ║◌◌◌C◌◌◌║ 2\\ \tt ║◌◌◌◌◌┌─╢ 3\\ \tt ║◌◌◌┌─┤◌║ 4\\ \tt ║◌◌◌│◌├─╢ 5\\ \tt ╟───┼─┘◌║ 6\\ \tt ║T◌◌│◌◌◌║ 7\\ \tt ╚═══╧═══╝\\ \tt ```\\ \tt  
	  } 
	   } 
	   } 
	 & & \\ 
 

    \theutterance \stepcounter{utterance}  

    & & \multicolumn{4}{p{0.6\linewidth}}{\cellcolor[rgb]{0.9,0.9,0.9}{%
	\makecell[{{p{\linewidth}}}]{% 
	  \tt {\tiny [A$\langle$GM]}  
	 'W' erfolgreich nach (2,2) verschoben. Dein Raster sieht jetzt so aus:\\ \tt ```\\ \tt  1234567\\ \tt ╔═══════╗\\ \tt ║I◌◌◌◌◌◌║ 1\\ \tt ║◌W◌◌◌◌◌║ 2\\ \tt ║◌◌◌T◌┌─╢ 3\\ \tt ║◌◌◌┌─┤◌║ 4\\ \tt ║◌◌◌│C├─╢ 5\\ \tt ╟───┼─┘H║ 6\\ \tt ║◌◌◌│◌◌◌║ 7\\ \tt ╚═══╧═══╝\\ \tt ```\\ \tt Ihr seid gerade bei Runde 3 von maximal 20.\\ \tt Ihr habt gemeinsam 3 von maximal 12 Strafpunkten angesammelt.\\ \tt Dein Mitspieler hat ein Objekt auf seinem Raster bewegt.\\ \tt \\ \tt \\ \tt Wie lautet dein nächster Befehl? 
	  } 
	   } 
	   } 
	 & \\ 
 

    \theutterance \stepcounter{utterance}  

    & \multicolumn{4}{p{0.6\linewidth}}{\cellcolor[rgb]{0.8,1,0.9}{%
	\makecell[{{p{\linewidth}}}]{% 
	  \tt {\tiny [A$\rangle$GM]}  
	 VERSCHIEBE: I, (2, 1) 
	  } 
	   } 
	   } 
	 & & \\ 
 

    \theutterance \stepcounter{utterance}  

    & & & \multicolumn{2}{p{0.3\linewidth}}{\cellcolor[rgb]{0.95,0.95,0.95}{%
	\makecell[{{p{\linewidth}}}]{% 
	  \tt {\tiny [GM$|$GM]}  
	 VERSCHIEBE: I, (2, 1) 
	  } 
	   } 
	   } 
	 & & \\ 
 

    \theutterance \stepcounter{utterance}  

    & & & \multicolumn{2}{p{0.3\linewidth}}{\cellcolor[rgb]{0.95,0.95,0.95}{%
	\makecell[{{p{\linewidth}}}]{% 
	  \tt {\tiny [GM$|$GM]}  
	 'I' erfolgreich nach (2,1) verschoben. Dein Raster sieht jetzt so aus:\\ \tt ```\\ \tt  1234567\\ \tt ╔═══════╗\\ \tt ║◌I◌◌◌◌◌║ 1\\ \tt ║◌W◌◌◌◌◌║ 2\\ \tt ║◌◌◌T◌┌─╢ 3\\ \tt ║◌◌◌┌─┤◌║ 4\\ \tt ║◌◌◌│C├─╢ 5\\ \tt ╟───┼─┘H║ 6\\ \tt ║◌◌◌│◌◌◌║ 7\\ \tt ╚═══╧═══╝\\ \tt ```\\ \tt  
	  } 
	   } 
	   } 
	 & & \\ 
 

    \theutterance \stepcounter{utterance}  

    & & \multicolumn{4}{p{0.6\linewidth}}{\cellcolor[rgb]{0.9,0.9,0.9}{%
	\makecell[{{p{\linewidth}}}]{% 
	  \tt {\tiny [GM$\rangle$B]}  
	 'I' erfolgreich nach (2,1) verschoben. Dein Raster sieht jetzt so aus:\\ \tt ```\\ \tt  1234567\\ \tt ╔═══════╗\\ \tt ║WIH◌◌◌◌║ 1\\ \tt ║◌◌◌C◌◌◌║ 2\\ \tt ║◌◌◌◌◌┌─╢ 3\\ \tt ║◌◌◌┌─┤◌║ 4\\ \tt ║◌◌◌│◌├─╢ 5\\ \tt ╟───┼─┘◌║ 6\\ \tt ║T◌◌│◌◌◌║ 7\\ \tt ╚═══╧═══╝\\ \tt ```\\ \tt Ihr seid gerade bei Runde 4 von maximal 20.\\ \tt Ihr habt gemeinsam 3 von maximal 12 Strafpunkten angesammelt.\\ \tt Dein Mitspieler hat ein Objekt auf seinem Raster bewegt.\\ \tt \\ \tt \\ \tt Wie lautet dein nächster Befehl? 
	  } 
	   } 
	   } 
	 & \\ 
 

    \theutterance \stepcounter{utterance}  

    & & & \multicolumn{4}{p{0.6\linewidth}}{\cellcolor[rgb]{1,0.85,0.72}{%
	\makecell[{{p{\linewidth}}}]{% 
	  \tt {\tiny [GM$\langle$B]}  
	 VERSCHIEBE: T, (3,1) 
	  } 
	   } 
	   } 
	  \\ 
 

    \theutterance \stepcounter{utterance}  

    & & & \multicolumn{2}{p{0.3\linewidth}}{\cellcolor[rgb]{0.95,0.95,0.95}{%
	\makecell[{{p{\linewidth}}}]{% 
	  \tt {\tiny [GM$|$GM]}  
	 VERSCHIEBE: T, (3,1) 
	  } 
	   } 
	   } 
	 & & \\ 
 

    \theutterance \stepcounter{utterance}  

    & & & \multicolumn{2}{p{0.3\linewidth}}{\cellcolor[rgb]{0.95,0.95,0.95}{%
	\makecell[{{p{\linewidth}}}]{% 
	  \tt {\tiny [GM$|$GM]}  
	 Strafpunkt: (3,1) ist nicht leer, sondern enthält 'H'.\\ \tt \\ \tt Ihr habt gemeinsam 4 von maximal 12 Strafpunkten angesammelt.\\ \tt Bitte versuche es noch mal! 
	  } 
	   } 
	   } 
	 & & \\ 
 

    \theutterance \stepcounter{utterance}  

    & & \multicolumn{4}{p{0.6\linewidth}}{\cellcolor[rgb]{0.9,0.9,0.9}{%
	\makecell[{{p{\linewidth}}}]{% 
	  \tt {\tiny [GM$\rangle$B]}  
	 Strafpunkt: (3,1) ist nicht leer, sondern enthält 'H'.\\ \tt \\ \tt Ihr habt gemeinsam 4 von maximal 12 Strafpunkten angesammelt.\\ \tt Bitte versuche es noch mal! 
	  } 
	   } 
	   } 
	 & \\ 
 

    \theutterance \stepcounter{utterance}  

    & & & \multicolumn{4}{p{0.6\linewidth}}{\cellcolor[rgb]{1,0.85,0.72}{%
	\makecell[{{p{\linewidth}}}]{% 
	  \tt {\tiny [GM$\langle$B]}  
	 VERSCHIEBE: H, (5,1) 
	  } 
	   } 
	   } 
	  \\ 
 

    \theutterance \stepcounter{utterance}  

    & & & \multicolumn{2}{p{0.3\linewidth}}{\cellcolor[rgb]{0.95,0.95,0.95}{%
	\makecell[{{p{\linewidth}}}]{% 
	  \tt {\tiny [GM$|$GM]}  
	 VERSCHIEBE: H, (5,1) 
	  } 
	   } 
	   } 
	 & & \\ 
 

    \theutterance \stepcounter{utterance}  

    & & & \multicolumn{2}{p{0.3\linewidth}}{\cellcolor[rgb]{0.95,0.95,0.95}{%
	\makecell[{{p{\linewidth}}}]{% 
	  \tt {\tiny [GM$|$GM]}  
	 'H' erfolgreich nach (5,1) verschoben. Dein Raster sieht jetzt so aus:\\ \tt ```\\ \tt  1234567\\ \tt ╔═══════╗\\ \tt ║WI◌◌H◌◌║ 1\\ \tt ║◌◌◌C◌◌◌║ 2\\ \tt ║◌◌◌◌◌┌─╢ 3\\ \tt ║◌◌◌┌─┤◌║ 4\\ \tt ║◌◌◌│◌├─╢ 5\\ \tt ╟───┼─┘◌║ 6\\ \tt ║T◌◌│◌◌◌║ 7\\ \tt ╚═══╧═══╝\\ \tt ```\\ \tt  
	  } 
	   } 
	   } 
	 & & \\ 
 

    \theutterance \stepcounter{utterance}  

    & & \multicolumn{4}{p{0.6\linewidth}}{\cellcolor[rgb]{0.9,0.9,0.9}{%
	\makecell[{{p{\linewidth}}}]{% 
	  \tt {\tiny [A$\langle$GM]}  
	 'I' erfolgreich nach (2,1) verschoben. Dein Raster sieht jetzt so aus:\\ \tt ```\\ \tt  1234567\\ \tt ╔═══════╗\\ \tt ║◌I◌◌◌◌◌║ 1\\ \tt ║◌W◌◌◌◌◌║ 2\\ \tt ║◌◌◌T◌┌─╢ 3\\ \tt ║◌◌◌┌─┤◌║ 4\\ \tt ║◌◌◌│C├─╢ 5\\ \tt ╟───┼─┘H║ 6\\ \tt ║◌◌◌│◌◌◌║ 7\\ \tt ╚═══╧═══╝\\ \tt ```\\ \tt Ihr seid gerade bei Runde 4 von maximal 20.\\ \tt Ihr habt gemeinsam 4 von maximal 12 Strafpunkten angesammelt.\\ \tt Dein Mitspieler hat ein Objekt auf seinem Raster bewegt.\\ \tt \\ \tt \\ \tt Wie lautet dein nächster Befehl? 
	  } 
	   } 
	   } 
	 & \\ 
 

    \theutterance \stepcounter{utterance}  

    & \multicolumn{4}{p{0.6\linewidth}}{\cellcolor[rgb]{0.8,1,0.9}{%
	\makecell[{{p{\linewidth}}}]{% 
	  \tt {\tiny [A$\rangle$GM]}  
	 VERSCHIEBE: W, (1, 1) 
	  } 
	   } 
	   } 
	 & & \\ 
 

    \theutterance \stepcounter{utterance}  

    & & & \multicolumn{2}{p{0.3\linewidth}}{\cellcolor[rgb]{0.95,0.95,0.95}{%
	\makecell[{{p{\linewidth}}}]{% 
	  \tt {\tiny [GM$|$GM]}  
	 VERSCHIEBE: W, (1, 1) 
	  } 
	   } 
	   } 
	 & & \\ 
 

    \theutterance \stepcounter{utterance}  

    & & & \multicolumn{2}{p{0.3\linewidth}}{\cellcolor[rgb]{0.95,0.95,0.95}{%
	\makecell[{{p{\linewidth}}}]{% 
	  \tt {\tiny [GM$|$GM]}  
	 'W' erfolgreich nach (1,1) verschoben. Dein Raster sieht jetzt so aus:\\ \tt ```\\ \tt  1234567\\ \tt ╔═══════╗\\ \tt ║WI◌◌◌◌◌║ 1\\ \tt ║◌◌◌◌◌◌◌║ 2\\ \tt ║◌◌◌T◌┌─╢ 3\\ \tt ║◌◌◌┌─┤◌║ 4\\ \tt ║◌◌◌│C├─╢ 5\\ \tt ╟───┼─┘H║ 6\\ \tt ║◌◌◌│◌◌◌║ 7\\ \tt ╚═══╧═══╝\\ \tt ```\\ \tt  
	  } 
	   } 
	   } 
	 & & \\ 
 

    \theutterance \stepcounter{utterance}  

    & & \multicolumn{4}{p{0.6\linewidth}}{\cellcolor[rgb]{0.9,0.9,0.9}{%
	\makecell[{{p{\linewidth}}}]{% 
	  \tt {\tiny [GM$\rangle$B]}  
	 'H' erfolgreich nach (5,1) verschoben. Dein Raster sieht jetzt so aus:\\ \tt ```\\ \tt  1234567\\ \tt ╔═══════╗\\ \tt ║WI◌◌H◌◌║ 1\\ \tt ║◌◌◌C◌◌◌║ 2\\ \tt ║◌◌◌◌◌┌─╢ 3\\ \tt ║◌◌◌┌─┤◌║ 4\\ \tt ║◌◌◌│◌├─╢ 5\\ \tt ╟───┼─┘◌║ 6\\ \tt ║T◌◌│◌◌◌║ 7\\ \tt ╚═══╧═══╝\\ \tt ```\\ \tt Ihr seid gerade bei Runde 5 von maximal 20.\\ \tt Ihr habt gemeinsam 4 von maximal 12 Strafpunkten angesammelt.\\ \tt Dein Mitspieler hat ein Objekt auf seinem Raster bewegt.\\ \tt \\ \tt \\ \tt Wie lautet dein nächster Befehl? 
	  } 
	   } 
	   } 
	 & \\ 
 

    \theutterance \stepcounter{utterance}  

    & & & \multicolumn{4}{p{0.6\linewidth}}{\cellcolor[rgb]{1,0.85,0.72}{%
	\makecell[{{p{\linewidth}}}]{% 
	  \tt {\tiny [GM$\langle$B]}  
	 VERSCHIEBE: T, (3,1) 
	  } 
	   } 
	   } 
	  \\ 
 

    \theutterance \stepcounter{utterance}  

    & & & \multicolumn{2}{p{0.3\linewidth}}{\cellcolor[rgb]{0.95,0.95,0.95}{%
	\makecell[{{p{\linewidth}}}]{% 
	  \tt {\tiny [GM$|$GM]}  
	 VERSCHIEBE: T, (3,1) 
	  } 
	   } 
	   } 
	 & & \\ 
 

    \theutterance \stepcounter{utterance}  

    & & & \multicolumn{2}{p{0.3\linewidth}}{\cellcolor[rgb]{0.95,0.95,0.95}{%
	\makecell[{{p{\linewidth}}}]{% 
	  \tt {\tiny [GM$|$GM]}  
	 'T' erfolgreich nach (3,1) verschoben. Dein Raster sieht jetzt so aus:\\ \tt ```\\ \tt  1234567\\ \tt ╔═══════╗\\ \tt ║WIT◌H◌◌║ 1\\ \tt ║◌◌◌C◌◌◌║ 2\\ \tt ║◌◌◌◌◌┌─╢ 3\\ \tt ║◌◌◌┌─┤◌║ 4\\ \tt ║◌◌◌│◌├─╢ 5\\ \tt ╟───┼─┘◌║ 6\\ \tt ║◌◌◌│◌◌◌║ 7\\ \tt ╚═══╧═══╝\\ \tt ```\\ \tt  
	  } 
	   } 
	   } 
	 & & \\ 
 

    \theutterance \stepcounter{utterance}  

    & & \multicolumn{4}{p{0.6\linewidth}}{\cellcolor[rgb]{0.9,0.9,0.9}{%
	\makecell[{{p{\linewidth}}}]{% 
	  \tt {\tiny [A$\langle$GM]}  
	 'W' erfolgreich nach (1,1) verschoben. Dein Raster sieht jetzt so aus:\\ \tt ```\\ \tt  1234567\\ \tt ╔═══════╗\\ \tt ║WI◌◌◌◌◌║ 1\\ \tt ║◌◌◌◌◌◌◌║ 2\\ \tt ║◌◌◌T◌┌─╢ 3\\ \tt ║◌◌◌┌─┤◌║ 4\\ \tt ║◌◌◌│C├─╢ 5\\ \tt ╟───┼─┘H║ 6\\ \tt ║◌◌◌│◌◌◌║ 7\\ \tt ╚═══╧═══╝\\ \tt ```\\ \tt Ihr seid gerade bei Runde 5 von maximal 20.\\ \tt Ihr habt gemeinsam 4 von maximal 12 Strafpunkten angesammelt.\\ \tt Dein Mitspieler hat ein Objekt auf seinem Raster bewegt.\\ \tt \\ \tt \\ \tt Wie lautet dein nächster Befehl? 
	  } 
	   } 
	   } 
	 & \\ 
 

    \theutterance \stepcounter{utterance}  

    & \multicolumn{4}{p{0.6\linewidth}}{\cellcolor[rgb]{0.8,1,0.9}{%
	\makecell[{{p{\linewidth}}}]{% 
	  \tt {\tiny [A$\rangle$GM]}  
	 VERSCHIEBE: T, (3, 1) 
	  } 
	   } 
	   } 
	 & & \\ 
 

    \theutterance \stepcounter{utterance}  

    & & & \multicolumn{2}{p{0.3\linewidth}}{\cellcolor[rgb]{0.95,0.95,0.95}{%
	\makecell[{{p{\linewidth}}}]{% 
	  \tt {\tiny [GM$|$GM]}  
	 VERSCHIEBE: T, (3, 1) 
	  } 
	   } 
	   } 
	 & & \\ 
 

    \theutterance \stepcounter{utterance}  

    & & & \multicolumn{2}{p{0.3\linewidth}}{\cellcolor[rgb]{0.95,0.95,0.95}{%
	\makecell[{{p{\linewidth}}}]{% 
	  \tt {\tiny [GM$|$GM]}  
	 'T' erfolgreich nach (3,1) verschoben. Dein Raster sieht jetzt so aus:\\ \tt ```\\ \tt  1234567\\ \tt ╔═══════╗\\ \tt ║WIT◌◌◌◌║ 1\\ \tt ║◌◌◌◌◌◌◌║ 2\\ \tt ║◌◌◌◌◌┌─╢ 3\\ \tt ║◌◌◌┌─┤◌║ 4\\ \tt ║◌◌◌│C├─╢ 5\\ \tt ╟───┼─┘H║ 6\\ \tt ║◌◌◌│◌◌◌║ 7\\ \tt ╚═══╧═══╝\\ \tt ```\\ \tt  
	  } 
	   } 
	   } 
	 & & \\ 
 

    \theutterance \stepcounter{utterance}  

    & & \multicolumn{4}{p{0.6\linewidth}}{\cellcolor[rgb]{0.9,0.9,0.9}{%
	\makecell[{{p{\linewidth}}}]{% 
	  \tt {\tiny [GM$\rangle$B]}  
	 'T' erfolgreich nach (3,1) verschoben. Dein Raster sieht jetzt so aus:\\ \tt ```\\ \tt  1234567\\ \tt ╔═══════╗\\ \tt ║WIT◌H◌◌║ 1\\ \tt ║◌◌◌C◌◌◌║ 2\\ \tt ║◌◌◌◌◌┌─╢ 3\\ \tt ║◌◌◌┌─┤◌║ 4\\ \tt ║◌◌◌│◌├─╢ 5\\ \tt ╟───┼─┘◌║ 6\\ \tt ║◌◌◌│◌◌◌║ 7\\ \tt ╚═══╧═══╝\\ \tt ```\\ \tt Ihr seid gerade bei Runde 6 von maximal 20.\\ \tt Ihr habt gemeinsam 4 von maximal 12 Strafpunkten angesammelt.\\ \tt Dein Mitspieler hat ein Objekt auf seinem Raster bewegt.\\ \tt \\ \tt \\ \tt Wie lautet dein nächster Befehl? 
	  } 
	   } 
	   } 
	 & \\ 
 

    \theutterance \stepcounter{utterance}  

    & & & \multicolumn{4}{p{0.6\linewidth}}{\cellcolor[rgb]{1,0.85,0.72}{%
	\makecell[{{p{\linewidth}}}]{% 
	  \tt {\tiny [GM$\langle$B]}  
	 VERSCHIEBE: C, (4,1) 
	  } 
	   } 
	   } 
	  \\ 
 

    \theutterance \stepcounter{utterance}  

    & & & \multicolumn{2}{p{0.3\linewidth}}{\cellcolor[rgb]{0.95,0.95,0.95}{%
	\makecell[{{p{\linewidth}}}]{% 
	  \tt {\tiny [GM$|$GM]}  
	 VERSCHIEBE: C, (4,1) 
	  } 
	   } 
	   } 
	 & & \\ 
 

    \theutterance \stepcounter{utterance}  

    & & & \multicolumn{2}{p{0.3\linewidth}}{\cellcolor[rgb]{0.95,0.95,0.95}{%
	\makecell[{{p{\linewidth}}}]{% 
	  \tt {\tiny [GM$|$GM]}  
	 'C' erfolgreich nach (4,1) verschoben. Dein Raster sieht jetzt so aus:\\ \tt ```\\ \tt  1234567\\ \tt ╔═══════╗\\ \tt ║WITCH◌◌║ 1\\ \tt ║◌◌◌◌◌◌◌║ 2\\ \tt ║◌◌◌◌◌┌─╢ 3\\ \tt ║◌◌◌┌─┤◌║ 4\\ \tt ║◌◌◌│◌├─╢ 5\\ \tt ╟───┼─┘◌║ 6\\ \tt ║◌◌◌│◌◌◌║ 7\\ \tt ╚═══╧═══╝\\ \tt ```\\ \tt  
	  } 
	   } 
	   } 
	 & & \\ 
 

    \theutterance \stepcounter{utterance}  

    & & \multicolumn{4}{p{0.6\linewidth}}{\cellcolor[rgb]{0.9,0.9,0.9}{%
	\makecell[{{p{\linewidth}}}]{% 
	  \tt {\tiny [A$\langle$GM]}  
	 'T' erfolgreich nach (3,1) verschoben. Dein Raster sieht jetzt so aus:\\ \tt ```\\ \tt  1234567\\ \tt ╔═══════╗\\ \tt ║WIT◌◌◌◌║ 1\\ \tt ║◌◌◌◌◌◌◌║ 2\\ \tt ║◌◌◌◌◌┌─╢ 3\\ \tt ║◌◌◌┌─┤◌║ 4\\ \tt ║◌◌◌│C├─╢ 5\\ \tt ╟───┼─┘H║ 6\\ \tt ║◌◌◌│◌◌◌║ 7\\ \tt ╚═══╧═══╝\\ \tt ```\\ \tt Ihr seid gerade bei Runde 6 von maximal 20.\\ \tt Ihr habt gemeinsam 4 von maximal 12 Strafpunkten angesammelt.\\ \tt Dein Mitspieler hat ein Objekt auf seinem Raster bewegt.\\ \tt \\ \tt \\ \tt Wie lautet dein nächster Befehl? 
	  } 
	   } 
	   } 
	 & \\ 
 

    \theutterance \stepcounter{utterance}  

    & \multicolumn{4}{p{0.6\linewidth}}{\cellcolor[rgb]{0.8,1,0.9}{%
	\makecell[{{p{\linewidth}}}]{% 
	  \tt {\tiny [A$\rangle$GM]}  
	 VERSCHIEBE: C, (4, 1) 
	  } 
	   } 
	   } 
	 & & \\ 
 

    \theutterance \stepcounter{utterance}  

    & & & \multicolumn{2}{p{0.3\linewidth}}{\cellcolor[rgb]{0.95,0.95,0.95}{%
	\makecell[{{p{\linewidth}}}]{% 
	  \tt {\tiny [GM$|$GM]}  
	 VERSCHIEBE: C, (4, 1) 
	  } 
	   } 
	   } 
	 & & \\ 
 

    \theutterance \stepcounter{utterance}  

    & & & \multicolumn{2}{p{0.3\linewidth}}{\cellcolor[rgb]{0.95,0.95,0.95}{%
	\makecell[{{p{\linewidth}}}]{% 
	  \tt {\tiny [GM$|$GM]}  
	 'C' erfolgreich nach (4,1) verschoben. Dein Raster sieht jetzt so aus:\\ \tt ```\\ \tt  1234567\\ \tt ╔═══════╗\\ \tt ║WITC◌◌◌║ 1\\ \tt ║◌◌◌◌◌◌◌║ 2\\ \tt ║◌◌◌◌◌┌─╢ 3\\ \tt ║◌◌◌┌─┤◌║ 4\\ \tt ║◌◌◌│◌├─╢ 5\\ \tt ╟───┼─┘H║ 6\\ \tt ║◌◌◌│◌◌◌║ 7\\ \tt ╚═══╧═══╝\\ \tt ```\\ \tt  
	  } 
	   } 
	   } 
	 & & \\ 
 

    \theutterance \stepcounter{utterance}  

    & & \multicolumn{4}{p{0.6\linewidth}}{\cellcolor[rgb]{0.9,0.9,0.9}{%
	\makecell[{{p{\linewidth}}}]{% 
	  \tt {\tiny [GM$\rangle$B]}  
	 'C' erfolgreich nach (4,1) verschoben. Dein Raster sieht jetzt so aus:\\ \tt ```\\ \tt  1234567\\ \tt ╔═══════╗\\ \tt ║WITCH◌◌║ 1\\ \tt ║◌◌◌◌◌◌◌║ 2\\ \tt ║◌◌◌◌◌┌─╢ 3\\ \tt ║◌◌◌┌─┤◌║ 4\\ \tt ║◌◌◌│◌├─╢ 5\\ \tt ╟───┼─┘◌║ 6\\ \tt ║◌◌◌│◌◌◌║ 7\\ \tt ╚═══╧═══╝\\ \tt ```\\ \tt Ihr seid gerade bei Runde 7 von maximal 20.\\ \tt Ihr habt gemeinsam 4 von maximal 12 Strafpunkten angesammelt.\\ \tt Dein Mitspieler hat ein Objekt auf seinem Raster bewegt.\\ \tt \\ \tt \\ \tt Wie lautet dein nächster Befehl? 
	  } 
	   } 
	   } 
	 & \\ 
 

    \theutterance \stepcounter{utterance}  

    & & & \multicolumn{4}{p{0.6\linewidth}}{\cellcolor[rgb]{1,0.85,0.72}{%
	\makecell[{{p{\linewidth}}}]{% 
	  \tt {\tiny [GM$\langle$B]}  
	 SPRICH: fertig? 
	  } 
	   } 
	   } 
	  \\ 
 

    \theutterance \stepcounter{utterance}  

    & & & \multicolumn{2}{p{0.3\linewidth}}{\cellcolor[rgb]{0.95,0.95,0.95}{%
	\makecell[{{p{\linewidth}}}]{% 
	  \tt {\tiny [GM$|$GM]}  
	 SPRICH: fertig? 
	  } 
	   } 
	   } 
	 & & \\ 
 

    \theutterance \stepcounter{utterance}  

    & & \multicolumn{4}{p{0.6\linewidth}}{\cellcolor[rgb]{0.9,0.9,0.9}{%
	\makecell[{{p{\linewidth}}}]{% 
	  \tt {\tiny [A$\langle$GM]}  
	 'C' erfolgreich nach (4,1) verschoben. Dein Raster sieht jetzt so aus:\\ \tt ```\\ \tt  1234567\\ \tt ╔═══════╗\\ \tt ║WITC◌◌◌║ 1\\ \tt ║◌◌◌◌◌◌◌║ 2\\ \tt ║◌◌◌◌◌┌─╢ 3\\ \tt ║◌◌◌┌─┤◌║ 4\\ \tt ║◌◌◌│◌├─╢ 5\\ \tt ╟───┼─┘H║ 6\\ \tt ║◌◌◌│◌◌◌║ 7\\ \tt ╚═══╧═══╝\\ \tt ```\\ \tt Ihr seid gerade bei Runde 7 von maximal 20.\\ \tt Ihr habt gemeinsam 4 von maximal 12 Strafpunkten angesammelt.\\ \tt Dein Mitspieler hat dir diese Nachricht geschickt:\\ \tt \\ \tt \\ \tt "fertig?"\\ \tt \\ \tt \\ \tt Wie lautet dein nächster Befehl? 
	  } 
	   } 
	   } 
	 & \\ 
 

    \theutterance \stepcounter{utterance}  

    & \multicolumn{4}{p{0.6\linewidth}}{\cellcolor[rgb]{0.8,1,0.9}{%
	\makecell[{{p{\linewidth}}}]{% 
	  \tt {\tiny [A$\rangle$GM]}  
	 VERSCHIEBE: H, (5, 1) 
	  } 
	   } 
	   } 
	 & & \\ 
 

    \theutterance \stepcounter{utterance}  

    & & & \multicolumn{2}{p{0.3\linewidth}}{\cellcolor[rgb]{0.95,0.95,0.95}{%
	\makecell[{{p{\linewidth}}}]{% 
	  \tt {\tiny [GM$|$GM]}  
	 VERSCHIEBE: H, (5, 1) 
	  } 
	   } 
	   } 
	 & & \\ 
 

    \theutterance \stepcounter{utterance}  

    & & & \multicolumn{2}{p{0.3\linewidth}}{\cellcolor[rgb]{0.95,0.95,0.95}{%
	\makecell[{{p{\linewidth}}}]{% 
	  \tt {\tiny [GM$|$GM]}  
	 'H' erfolgreich nach (5,1) verschoben. Dein Raster sieht jetzt so aus:\\ \tt ```\\ \tt  1234567\\ \tt ╔═══════╗\\ \tt ║WITCH◌◌║ 1\\ \tt ║◌◌◌◌◌◌◌║ 2\\ \tt ║◌◌◌◌◌┌─╢ 3\\ \tt ║◌◌◌┌─┤◌║ 4\\ \tt ║◌◌◌│◌├─╢ 5\\ \tt ╟───┼─┘◌║ 6\\ \tt ║◌◌◌│◌◌◌║ 7\\ \tt ╚═══╧═══╝\\ \tt ```\\ \tt  
	  } 
	   } 
	   } 
	 & & \\ 
 

    \theutterance \stepcounter{utterance}  

    & & \multicolumn{4}{p{0.6\linewidth}}{\cellcolor[rgb]{0.9,0.9,0.9}{%
	\makecell[{{p{\linewidth}}}]{% 
	  \tt {\tiny [GM$\rangle$B]}  
	 Deine Nachricht wurde an deinen Mitspieler übermittelt.\\ \tt \\ \tt \\ \tt Ihr seid gerade bei Runde 8 von maximal 20.\\ \tt Ihr habt gemeinsam 4 von maximal 12 Strafpunkten angesammelt.\\ \tt Dein Mitspieler hat ein Objekt auf seinem Raster bewegt.\\ \tt \\ \tt \\ \tt Wie lautet dein nächster Befehl? 
	  } 
	   } 
	   } 
	 & \\ 
 

    \theutterance \stepcounter{utterance}  

    & & & \multicolumn{4}{p{0.6\linewidth}}{\cellcolor[rgb]{1,0.85,0.72}{%
	\makecell[{{p{\linewidth}}}]{% 
	  \tt {\tiny [GM$\langle$B]}  
	 SPRICH: Ich warte auf deine Antwort, ob wir fertig sind. Alle meine Objekte stehen jetzt an den vereinbarten Positionen: W(1,1), I(2,1), T(3,1), C(4,1), H(5,1). 
	  } 
	   } 
	   } 
	  \\ 
 

    \theutterance \stepcounter{utterance}  

    & & & \multicolumn{2}{p{0.3\linewidth}}{\cellcolor[rgb]{0.95,0.95,0.95}{%
	\makecell[{{p{\linewidth}}}]{% 
	  \tt {\tiny [GM$|$GM]}  
	 SPRICH: Ich warte auf deine Antwort, ob wir fertig sind. Alle meine Objekte stehen jetzt an den vereinbarten Positionen: W(1,1), I(2,1), T(3,1), C(4,1), H(5,1). 
	  } 
	   } 
	   } 
	 & & \\ 
 

    \theutterance \stepcounter{utterance}  

    & & \multicolumn{4}{p{0.6\linewidth}}{\cellcolor[rgb]{0.9,0.9,0.9}{%
	\makecell[{{p{\linewidth}}}]{% 
	  \tt {\tiny [A$\langle$GM]}  
	 'H' erfolgreich nach (5,1) verschoben. Dein Raster sieht jetzt so aus:\\ \tt ```\\ \tt  1234567\\ \tt ╔═══════╗\\ \tt ║WITCH◌◌║ 1\\ \tt ║◌◌◌◌◌◌◌║ 2\\ \tt ║◌◌◌◌◌┌─╢ 3\\ \tt ║◌◌◌┌─┤◌║ 4\\ \tt ║◌◌◌│◌├─╢ 5\\ \tt ╟───┼─┘◌║ 6\\ \tt ║◌◌◌│◌◌◌║ 7\\ \tt ╚═══╧═══╝\\ \tt ```\\ \tt Ihr seid gerade bei Runde 8 von maximal 20.\\ \tt Ihr habt gemeinsam 4 von maximal 12 Strafpunkten angesammelt.\\ \tt Dein Mitspieler hat dir diese Nachricht geschickt:\\ \tt \\ \tt \\ \tt "Ich warte auf deine Antwort, ob wir fertig sind. Alle meine Objekte stehen jetzt an den vereinbarten Positionen: W(1,1), I(2,1), T(3,1), C(4,1), H(5,1)."\\ \tt \\ \tt \\ \tt Wie lautet dein nächster Befehl? 
	  } 
	   } 
	   } 
	 & \\ 
 

    \theutterance \stepcounter{utterance}  

    & \multicolumn{4}{p{0.6\linewidth}}{\cellcolor[rgb]{0.8,1,0.9}{%
	\makecell[{{p{\linewidth}}}]{% 
	  \tt {\tiny [A$\rangle$GM]}  
	 SPRICH: fertig! 
	  } 
	   } 
	   } 
	 & & \\ 
 

    \theutterance \stepcounter{utterance}  

    & & & \multicolumn{2}{p{0.3\linewidth}}{\cellcolor[rgb]{0.95,0.95,0.95}{%
	\makecell[{{p{\linewidth}}}]{% 
	  \tt {\tiny [GM$|$GM]}  
	 SPRICH: fertig! 
	  } 
	   } 
	   } 
	 & & \\ 
 

    \theutterance \stepcounter{utterance}  

    & & & \multicolumn{2}{p{0.3\linewidth}}{\cellcolor[rgb]{0.95,0.95,0.95}{%
	\makecell[{{p{\linewidth}}}]{% 
	  \tt {\tiny [GM$|$GM]}  
	 true 
	  } 
	   } 
	   } 
	 & & \\ 
 

    \theutterance \stepcounter{utterance}  

    & & & \multicolumn{2}{p{0.3\linewidth}}{\cellcolor[rgb]{0.95,0.95,0.95}{%
	\makecell[{{p{\linewidth}}}]{% 
	  \tt {\tiny [GM$|$GM]}  
	 * success: True\\ \tt * lose: False\\ \tt * aborted: False\\ \tt -------\\ \tt * Shifts: 6.00\\ \tt * Max Shifts: 8.00\\ \tt * Min Shifts: 4.00\\ \tt * End Distance Sum: 0.00\\ \tt * Init Distance Sum: 18.69\\ \tt * Expected Distance Sum: 20.95\\ \tt * Penalties: 4.00\\ \tt * Max Penalties: 12.00\\ \tt * Rounds: 8.00\\ \tt * Max Rounds: 20.00\\ \tt * Object Count: 5.00\\ \tt  
	  } 
	   } 
	   } 
	 & & \\ 
 

    \theutterance \stepcounter{utterance}  

    & & & \multicolumn{2}{p{0.3\linewidth}}{\cellcolor[rgb]{0.95,0.95,0.95}{%
	\makecell[{{p{\linewidth}}}]{% 
	  \tt {\tiny [GM$|$GM]}  
	 * Main Score: 66.94\\ \tt \\ \tt -------\\ \tt * Distance Score: 1.00\\ \tt * Consistency Score: 0.60\\ \tt * Coverage Score: 1.00\\ \tt * Penalty Score: 0.82\\ \tt * Alternative Penalty Score: 0.90\\ \tt * Alternative Main Score: 90.00\\ \tt \\ \tt -------\\ \tt * Shifts: 6.00\\ \tt * Max Shifts: 8.00\\ \tt * Min Shifts: 4.00\\ \tt * End Distance Sum: 0.00\\ \tt * Init Distance Sum: 18.69\\ \tt * Expected Distance Sum: 20.95\\ \tt * Penalties: 4.00\\ \tt * Max Penalties: 12.00\\ \tt * Rounds: 8.00\\ \tt * Max Rounds: 20.00\\ \tt * Object Count: 5.00\\ \tt  
	  } 
	   } 
	   } 
	 & & \\ 
 

\end{supertabular}
}

\end{document}
