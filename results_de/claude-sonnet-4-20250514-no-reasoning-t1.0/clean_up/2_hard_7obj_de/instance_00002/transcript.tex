
\documentclass{article}
\usepackage{colortbl}
\usepackage{makecell}
\usepackage{multirow}
\usepackage{supertabular}

\begin{document}

\newcounter{utterance}

\twocolumn

{ \footnotesize  \setcounter{utterance}{1}
\setlength{\tabcolsep}{0pt}
\begin{supertabular}{c@{$\;$}|p{.15\linewidth}@{}p{.15\linewidth}p{.15\linewidth}p{.15\linewidth}p{.15\linewidth}p{.15\linewidth}}

    \# & $\;$A & \multicolumn{4}{c}{Game Master} & $\;\:$B\\
    \hline 

    \theutterance \stepcounter{utterance}  

    & & \multicolumn{4}{p{0.6\linewidth}}{\cellcolor[rgb]{0.9,0.9,0.9}{%
	\makecell[{{p{\linewidth}}}]{% 
	  \tt {\tiny [A$\langle$GM]}  
	 Ich bin der Spielleiter, und du spielst ein kooperatives Spiel mit dem folgenden Raster als Spielbrett:\\ \tt \\ \tt ```\\ \tt  1234567\\ \tt ╔═╤═╤═╤═╗\\ \tt ║A│◌│O│◌║ 1\\ \tt ╟─┼─┤◌│◌║ 2\\ \tt ║◌│T│◌│P║ 3\\ \tt ║◌│◌├─┴─╢ 4\\ \tt ╟─┴─┤◌◌◌║ 5\\ \tt ║D◌◌└─┬─╢ 6\\ \tt ║◌◌◌S◌│M║ 7\\ \tt ╚═════╧═╝\\ \tt \\ \tt ```\\ \tt \\ \tt * Am oberen Rand sind die x-Koordinaten angegeben, die nach rechts hin zunehmen, und am rechten Rand die y-Koordinaten, die nach unten zunehmen.\\ \tt * Die folgenden Objekte sind zufällig auf deinem Raster platziert: 'P', 'O', 'T', 'S', 'D', 'A', 'M'.\\ \tt \\ \tt \\ \tt Dein Mitspieler sieht eine Variation des Spielbretts, auf dem dieselben Objekte an anderen zufälligen Positionen platziert sind. Du kannst das Spielbrett des anderen Spielers nicht sehen, und er kann deins nicht sehen.\\ \tt \\ \tt \\ \tt **Ziel des Spiels:**\\ \tt \\ \tt Beide Spieler müssen die Objekte auf ihrem jeweiligen Spielbrett so bewegen, dass identische Objekte am Ende an denselben Koordinaten liegen. Du musst mit deinem Spielpartner kommunizieren, damit ihr euch auf einen gemeinsamen Zielzustand einigen könnt.\\ \tt \\ \tt \\ \tt **Spielregeln:**\\ \tt \\ \tt * In jedem Spielzug kannst du genau eine der folgenden zwei Aktionen ausführen:\\ \tt 1. `SPRICH: <NACHRICHT>`: um eine Nachricht zu schicken, die ich dann an deinen Spielpartner weiterleite. Als Nachricht gilt alles bis zum nächsten Zeilenumbruch.\\ \tt 2. `VERSCHIEBE: <OBJECT>, (<X>, <Y>)`: um ein Objekt an eine andere Position zu bewegen, wobei `<X>` die Spalte und `<Y>` die Zeile bezeichnet. Ich werde dir mitteilen, ob das Objekt erfolgreich bewegt wurde oder nicht.\\ \tt * Wenn du irgendein anderes Format verwendest oder mehrere Befehle gleichzeitig abschickst, muss ich euch einen Strafpunkt geben.\\ \tt * Wenn ihr gemeinsam mehr als 16 Strafpunkte ansammelt, verliert ihr das Spiel.\\ \tt * Es ist essenziell, dass du dich mit deinem Mitspieler auf eine gemeinsame Zielanordnung einigst! Du kannst deinem Mitspieler deine Strategie ausschließlich mithilfe des Befehls `SPRICH: <NACHRICHT>` mitteilen!\\ \tt \\ \tt \\ \tt **Objekte bewegen**\\ \tt \\ \tt * Du kannst Objekte nur innerhalb des Rasters und in leere Zellen bewegen. Leere Zellen erkennst am Zeichen '◌'.\\ \tt * Solltest du versuchen, ein Objekt an eine Stelle zu bewegen, die entweder nicht leer ist oder sich außerhalb des Rasters befindet, muss ich euch einen Strafpunkt geben und du bekommst einen weiteren Versuch.\\ \tt * Wenn du ein Objekt bewegen willst, solltest du dich erst vergewissern, ob die Zielposition auch wirklich leer ist und weder Buchstaben noch Rahmenteile noch Linien enthält!\\ \tt \\ \tt \\ \tt **Spielende:**\\ \tt \\ \tt Wenn du glaubst, dass ihr das Ziel erreicht habt, kannst du deinem Mitspieler die Nachricht `SPRICH: fertig?` schicken. Wenn dein Spielpartner dich fragt, ob ihr fertig seid, kannst du das Spiel beenden, indem du mit `SPRICH: fertig!` antwortest.\\ \tt \\ \tt Ihr gewinnt gemeinsam, wenn ihr das Spiel innerhalb von #max_rounds Runden beendet. Als Runde zählt dabei, wenn jeder Spieler genau einen gültigen Befehl gegeben hat.\\ \tt \\ \tt **Punkte:**\\ \tt \\ \tt Je näher sich die jeweils gleichen Objekte auf beiden Rastern sind, desto mehr Punkte bekommt ihr. Für Strafpunkte gibt es Punktabzug. Könnt ihr den Rekord brechen?\\ \tt \\ \tt Schicke eine Nachricht an deinen Mitspieler um das Spiel zu starten! 
	  } 
	   } 
	   } 
	 & \\ 
 

    \theutterance \stepcounter{utterance}  

    & \multicolumn{4}{p{0.6\linewidth}}{\cellcolor[rgb]{0.8,1,0.9}{%
	\makecell[{{p{\linewidth}}}]{% 
	  \tt {\tiny [A$\rangle$GM]}  
	 Ich analysiere zunächst mein Spielbrett und die Positionen der Objekte:\\ \tt \\ \tt - A: (1,1)\\ \tt - O: (3,1) \\ \tt - T: (2,3)\\ \tt - P: (4,3)\\ \tt - D: (1,6)\\ \tt - S: (4,7)\\ \tt - M: (6,7)\\ \tt \\ \tt SPRICH: Hallo! Lass uns koordinieren. Meine Objekte stehen so: A(1,1), O(3,1), T(2,3), P(4,3), D(1,6), S(4,7), M(6,7). Wo stehen deine Objekte? Dann können wir eine gemeinsame Zielanordnung festlegen. 
	  } 
	   } 
	   } 
	 & & \\ 
 

    \theutterance \stepcounter{utterance}  

    & & & \multicolumn{2}{p{0.3\linewidth}}{\cellcolor[rgb]{0.95,0.95,0.95}{%
	\makecell[{{p{\linewidth}}}]{% 
	  \tt {\tiny [GM$|$GM]}  
	 Ich analysiere zunächst mein Spielbrett und die Positionen der Objekte:\\ \tt \\ \tt - A: (1,1)\\ \tt - O: (3,1) \\ \tt - T: (2,3)\\ \tt - P: (4,3)\\ \tt - D: (1,6)\\ \tt - S: (4,7)\\ \tt - M: (6,7)\\ \tt \\ \tt SPRICH: Hallo! Lass uns koordinieren. Meine Objekte stehen so: A(1,1), O(3,1), T(2,3), P(4,3), D(1,6), S(4,7), M(6,7). Wo stehen deine Objekte? Dann können wir eine gemeinsame Zielanordnung festlegen. 
	  } 
	   } 
	   } 
	 & & \\ 
 

    \theutterance \stepcounter{utterance}  

    & & & \multicolumn{2}{p{0.3\linewidth}}{\cellcolor[rgb]{0.95,0.95,0.95}{%
	\makecell[{{p{\linewidth}}}]{% 
	  \tt {\tiny [GM$|$GM]}  
	 Invalid format: head is not empty: 'Ich analysiere zunächst mein Spielbrett und die Positionen der Objekte:\\ \tt \\ \tt - A: (1,1)\\ \tt - O: (3,1) \\ \tt - T: (2,3)\\ \tt - P: (4,3)\\ \tt - D: (1,6)\\ \tt - S: (4,7)\\ \tt - M: (6,7)\\ \tt \\ \tt ' 
	  } 
	   } 
	   } 
	 & & \\ 
 

    \theutterance \stepcounter{utterance}  

    & & \multicolumn{4}{p{0.6\linewidth}}{\cellcolor[rgb]{0.9,0.9,0.9}{%
	\makecell[{{p{\linewidth}}}]{% 
	  \tt {\tiny [A$\langle$GM]}  
	 Strafpunkt: Deine Nachricht darf nichts vor dem Befehl enthalten! Stelle sicher, dass deine Antwort ausschließlich einen der beiden Befehle `SPRICH: <NACHRICHT>` und `VERSCHIEBE: <OBJECT>, (<X>, <Y>)` und sonst nichts enthält!\\ \tt \\ \tt Ihr habt gemeinsam 1 von maximal 16 Strafpunkten angesammelt.\\ \tt Bitte versuche es noch mal! 
	  } 
	   } 
	   } 
	 & \\ 
 

    \theutterance \stepcounter{utterance}  

    & \multicolumn{4}{p{0.6\linewidth}}{\cellcolor[rgb]{0.8,1,0.9}{%
	\makecell[{{p{\linewidth}}}]{% 
	  \tt {\tiny [A$\rangle$GM]}  
	 SPRICH: Hallo! Lass uns koordinieren. Meine Objekte stehen so: A(1,1), O(3,1), T(2,3), P(4,3), D(1,6), S(4,7), M(6,7). Wo stehen deine Objekte? Dann können wir eine gemeinsame Zielanordnung festlegen. 
	  } 
	   } 
	   } 
	 & & \\ 
 

    \theutterance \stepcounter{utterance}  

    & & & \multicolumn{2}{p{0.3\linewidth}}{\cellcolor[rgb]{0.95,0.95,0.95}{%
	\makecell[{{p{\linewidth}}}]{% 
	  \tt {\tiny [GM$|$GM]}  
	 SPRICH: Hallo! Lass uns koordinieren. Meine Objekte stehen so: A(1,1), O(3,1), T(2,3), P(4,3), D(1,6), S(4,7), M(6,7). Wo stehen deine Objekte? Dann können wir eine gemeinsame Zielanordnung festlegen. 
	  } 
	   } 
	   } 
	 & & \\ 
 

    \theutterance \stepcounter{utterance}  

    & & \multicolumn{4}{p{0.6\linewidth}}{\cellcolor[rgb]{0.9,0.9,0.9}{%
	\makecell[{{p{\linewidth}}}]{% 
	  \tt {\tiny [GM$\rangle$B]}  
	 Ich bin der Spielleiter, und du spielst ein kooperatives Spiel mit dem folgenden Raster als Spielbrett:\\ \tt \\ \tt ```\\ \tt  1234567\\ \tt ╔═╤═╤═╤═╗\\ \tt ║A│◌│◌│M║ 1\\ \tt ╟─┼─┤O│T║ 2\\ \tt ║◌│D│◌│◌║ 3\\ \tt ║◌│◌├─┴─╢ 4\\ \tt ╟─┴─┤◌◌◌║ 5\\ \tt ║PS◌└─┬─╢ 6\\ \tt ║◌◌◌◌◌│◌║ 7\\ \tt ╚═════╧═╝\\ \tt \\ \tt ```\\ \tt \\ \tt * Am oberen Rand sind die x-Koordinaten angegeben, die nach rechts hin zunehmen, und am rechten Rand die y-Koordinaten, die nach unten zunehmen.\\ \tt * Die folgenden Objekte sind zufällig auf deinem Raster platziert: 'P', 'O', 'T', 'S', 'D', 'A', 'M'.\\ \tt \\ \tt \\ \tt Dein Mitspieler sieht eine Variation des Spielbretts, auf dem dieselben Objekte an anderen zufälligen Positionen platziert sind. Du kannst das Spielbrett des anderen Spielers nicht sehen, und er kann deins nicht sehen.\\ \tt \\ \tt \\ \tt **Ziel des Spiels:**\\ \tt \\ \tt Beide Spieler müssen die Objekte auf ihrem jeweiligen Spielbrett so bewegen, dass identische Objekte am Ende an denselben Koordinaten liegen. Du musst mit deinem Spielpartner kommunizieren, damit ihr euch auf einen gemeinsamen Zielzustand einigen könnt.\\ \tt \\ \tt \\ \tt **Spielregeln:**\\ \tt \\ \tt * In jedem Spielzug kannst du genau eine der folgenden zwei Aktionen ausführen:\\ \tt 1. `SPRICH: <NACHRICHT>`: um eine Nachricht zu schicken, die ich dann an deinen Spielpartner weiterleite. Als Nachricht gilt alles bis zum nächsten Zeilenumbruch.\\ \tt 2. `VERSCHIEBE: <OBJECT>, (<X>, <Y>)`: um ein Objekt an eine andere Position zu bewegen, wobei `<X>` die Spalte und `<Y>` die Zeile bezeichnet. Ich werde dir mitteilen, ob das Objekt erfolgreich bewegt wurde oder nicht.\\ \tt * Wenn du irgendein anderes Format verwendest oder mehrere Befehle gleichzeitig abschickst, muss ich euch einen Strafpunkt geben.\\ \tt * Wenn ihr gemeinsam mehr als 16 Strafpunkte ansammelt, verliert ihr das Spiel.\\ \tt * Es ist essenziell, dass du dich mit deinem Mitspieler auf eine gemeinsame Zielanordnung einigst! Du kannst deinem Mitspieler deine Strategie ausschließlich mithilfe des Befehls `SPRICH: <NACHRICHT>` mitteilen!\\ \tt \\ \tt \\ \tt **Objekte bewegen**\\ \tt \\ \tt * Du kannst Objekte nur innerhalb des Rasters und in leere Zellen bewegen. Leere Zellen erkennst am Zeichen '◌'.\\ \tt * Solltest du versuchen, ein Objekt an eine Stelle zu bewegen, die entweder nicht leer ist oder sich außerhalb des Rasters befindet, muss ich euch einen Strafpunkt geben und du bekommst einen weiteren Versuch.\\ \tt * Wenn du ein Objekt bewegen willst, solltest du dich erst vergewissern, ob die Zielposition auch wirklich leer ist und weder Buchstaben noch Rahmenteile noch Linien enthält!\\ \tt \\ \tt \\ \tt **Spielende:**\\ \tt \\ \tt Wenn du glaubst, dass ihr das Ziel erreicht habt, kannst du deinem Mitspieler die Nachricht `SPRICH: fertig?` schicken. Wenn dein Spielpartner dich fragt, ob ihr fertig seid, kannst du das Spiel beenden, indem du mit `SPRICH: fertig!` antwortest.\\ \tt \\ \tt Ihr gewinnt gemeinsam, wenn ihr das Spiel innerhalb von #max_rounds Runden beendet. Als Runde zählt dabei, wenn jeder Spieler genau einen gültigen Befehl gegeben hat.\\ \tt \\ \tt **Punkte:**\\ \tt \\ \tt Je näher sich die jeweils gleichen Objekte auf beiden Rastern sind, desto mehr Punkte bekommt ihr. Für Strafpunkte gibt es Punktabzug. Könnt ihr den Rekord brechen?\\ \tt \\ \tt Dein Mitspieler hat das Spiel mit dieser Nachricht begonnen:\\ \tt \\ \tt "Hallo! Lass uns koordinieren. Meine Objekte stehen so: A(1,1), O(3,1), T(2,3), P(4,3), D(1,6), S(4,7), M(6,7). Wo stehen deine Objekte? Dann können wir eine gemeinsame Zielanordnung festlegen."\\ \tt \\ \tt Wie lautet dein erster Befehl? 
	  } 
	   } 
	   } 
	 & \\ 
 

    \theutterance \stepcounter{utterance}  

    & & & \multicolumn{4}{p{0.6\linewidth}}{\cellcolor[rgb]{1,0.85,0.72}{%
	\makecell[{{p{\linewidth}}}]{% 
	  \tt {\tiny [GM$\langle$B]}  
	 Ich schaue mir zuerst mein Spielbrett an und identifiziere die Positionen meiner Objekte:\\ \tt \\ \tt - A: (1,1)\\ \tt - M: (7,1) \\ \tt - O: (4,2)\\ \tt - T: (7,2)\\ \tt - D: (3,3)\\ \tt - P: (1,6)\\ \tt - S: (2,6)\\ \tt \\ \tt Jetzt antworte ich meinem Mitspieler:\\ \tt \\ \tt SPRICH: Hallo! Meine Objekte stehen so: A(1,1), M(7,1), O(4,2), T(7,2), D(3,3), P(1,6), S(2,6). Sollen wir eine gemeinsame Zielanordnung festlegen? Ich schlage vor, wir einigen uns auf eine Anordnung, die für beide gut erreichbar ist. 
	  } 
	   } 
	   } 
	  \\ 
 

    \theutterance \stepcounter{utterance}  

    & & & \multicolumn{2}{p{0.3\linewidth}}{\cellcolor[rgb]{0.95,0.95,0.95}{%
	\makecell[{{p{\linewidth}}}]{% 
	  \tt {\tiny [GM$|$GM]}  
	 Ich schaue mir zuerst mein Spielbrett an und identifiziere die Positionen meiner Objekte:\\ \tt \\ \tt - A: (1,1)\\ \tt - M: (7,1) \\ \tt - O: (4,2)\\ \tt - T: (7,2)\\ \tt - D: (3,3)\\ \tt - P: (1,6)\\ \tt - S: (2,6)\\ \tt \\ \tt Jetzt antworte ich meinem Mitspieler:\\ \tt \\ \tt SPRICH: Hallo! Meine Objekte stehen so: A(1,1), M(7,1), O(4,2), T(7,2), D(3,3), P(1,6), S(2,6). Sollen wir eine gemeinsame Zielanordnung festlegen? Ich schlage vor, wir einigen uns auf eine Anordnung, die für beide gut erreichbar ist. 
	  } 
	   } 
	   } 
	 & & \\ 
 

    \theutterance \stepcounter{utterance}  

    & & & \multicolumn{2}{p{0.3\linewidth}}{\cellcolor[rgb]{0.95,0.95,0.95}{%
	\makecell[{{p{\linewidth}}}]{% 
	  \tt {\tiny [GM$|$GM]}  
	 Invalid format: head is not empty: 'Ich schaue mir zuerst mein Spielbrett an und identifiziere die Positionen meiner Objekte:\\ \tt \\ \tt - A: (1,1)\\ \tt - M: (7,1) \\ \tt - O: (4,2)\\ \tt - T: (7,2)\\ \tt - D: (3,3)\\ \tt - P: (1,6)\\ \tt - S: (2,6)\\ \tt \\ \tt Jetzt antworte ich meinem Mitspieler:\\ \tt \\ \tt ' 
	  } 
	   } 
	   } 
	 & & \\ 
 

    \theutterance \stepcounter{utterance}  

    & & \multicolumn{4}{p{0.6\linewidth}}{\cellcolor[rgb]{0.9,0.9,0.9}{%
	\makecell[{{p{\linewidth}}}]{% 
	  \tt {\tiny [GM$\rangle$B]}  
	 Strafpunkt: Deine Nachricht darf nichts vor dem Befehl enthalten! Stelle sicher, dass deine Antwort ausschließlich einen der beiden Befehle `SPRICH: <NACHRICHT>` und `VERSCHIEBE: <OBJECT>, (<X>, <Y>)` und sonst nichts enthält!\\ \tt \\ \tt Ihr habt gemeinsam 2 von maximal 16 Strafpunkten angesammelt.\\ \tt Bitte versuche es noch mal! 
	  } 
	   } 
	   } 
	 & \\ 
 

    \theutterance \stepcounter{utterance}  

    & & & \multicolumn{4}{p{0.6\linewidth}}{\cellcolor[rgb]{1,0.85,0.72}{%
	\makecell[{{p{\linewidth}}}]{% 
	  \tt {\tiny [GM$\langle$B]}  
	 SPRICH: Hallo! Meine Objekte stehen so: A(1,1), M(7,1), O(4,2), T(7,2), D(3,3), P(1,6), S(2,6). Sollen wir eine gemeinsame Zielanordnung festlegen? Ich schlage vor, wir einigen uns auf eine Anordnung, die für beide gut erreichbar ist. 
	  } 
	   } 
	   } 
	  \\ 
 

    \theutterance \stepcounter{utterance}  

    & & & \multicolumn{2}{p{0.3\linewidth}}{\cellcolor[rgb]{0.95,0.95,0.95}{%
	\makecell[{{p{\linewidth}}}]{% 
	  \tt {\tiny [GM$|$GM]}  
	 SPRICH: Hallo! Meine Objekte stehen so: A(1,1), M(7,1), O(4,2), T(7,2), D(3,3), P(1,6), S(2,6). Sollen wir eine gemeinsame Zielanordnung festlegen? Ich schlage vor, wir einigen uns auf eine Anordnung, die für beide gut erreichbar ist. 
	  } 
	   } 
	   } 
	 & & \\ 
 

    \theutterance \stepcounter{utterance}  

    & & \multicolumn{4}{p{0.6\linewidth}}{\cellcolor[rgb]{0.9,0.9,0.9}{%
	\makecell[{{p{\linewidth}}}]{% 
	  \tt {\tiny [A$\langle$GM]}  
	 Deine Nachricht wurde an deinen Mitspieler übermittelt.\\ \tt \\ \tt \\ \tt Ihr seid gerade bei Runde 1 von maximal 28.\\ \tt Ihr habt gemeinsam 2 von maximal 16 Strafpunkten angesammelt.\\ \tt Dein Mitspieler hat dir diese Nachricht geschickt:\\ \tt \\ \tt \\ \tt "Hallo! Meine Objekte stehen so: A(1,1), M(7,1), O(4,2), T(7,2), D(3,3), P(1,6), S(2,6). Sollen wir eine gemeinsame Zielanordnung festlegen? Ich schlage vor, wir einigen uns auf eine Anordnung, die für beide gut erreichbar ist."\\ \tt \\ \tt \\ \tt Wie lautet dein nächster Befehl? 
	  } 
	   } 
	   } 
	 & \\ 
 

    \theutterance \stepcounter{utterance}  

    & \multicolumn{4}{p{0.6\linewidth}}{\cellcolor[rgb]{0.8,1,0.9}{%
	\makecell[{{p{\linewidth}}}]{% 
	  \tt {\tiny [A$\rangle$GM]}  
	 SPRICH: Gute Idee! Lass uns eine mittlere Position für jedes Objekt wählen. Ich schlage vor: A(1,1), O(3,2), T(2,2), P(2,5), D(2,4), S(3,6), M(6,2). Das sollte für beide gut erreichbar sein. Bist du einverstanden? 
	  } 
	   } 
	   } 
	 & & \\ 
 

    \theutterance \stepcounter{utterance}  

    & & & \multicolumn{2}{p{0.3\linewidth}}{\cellcolor[rgb]{0.95,0.95,0.95}{%
	\makecell[{{p{\linewidth}}}]{% 
	  \tt {\tiny [GM$|$GM]}  
	 SPRICH: Gute Idee! Lass uns eine mittlere Position für jedes Objekt wählen. Ich schlage vor: A(1,1), O(3,2), T(2,2), P(2,5), D(2,4), S(3,6), M(6,2). Das sollte für beide gut erreichbar sein. Bist du einverstanden? 
	  } 
	   } 
	   } 
	 & & \\ 
 

    \theutterance \stepcounter{utterance}  

    & & \multicolumn{4}{p{0.6\linewidth}}{\cellcolor[rgb]{0.9,0.9,0.9}{%
	\makecell[{{p{\linewidth}}}]{% 
	  \tt {\tiny [GM$\rangle$B]}  
	 Deine Nachricht wurde an deinen Mitspieler übermittelt.\\ \tt \\ \tt \\ \tt Ihr seid gerade bei Runde 2 von maximal 28.\\ \tt Ihr habt gemeinsam 2 von maximal 16 Strafpunkten angesammelt.\\ \tt Dein Mitspieler hat dir diese Nachricht geschickt:\\ \tt \\ \tt \\ \tt "Gute Idee! Lass uns eine mittlere Position für jedes Objekt wählen. Ich schlage vor: A(1,1), O(3,2), T(2,2), P(2,5), D(2,4), S(3,6), M(6,2). Das sollte für beide gut erreichbar sein. Bist du einverstanden?"\\ \tt \\ \tt \\ \tt Wie lautet dein nächster Befehl? 
	  } 
	   } 
	   } 
	 & \\ 
 

    \theutterance \stepcounter{utterance}  

    & & & \multicolumn{4}{p{0.6\linewidth}}{\cellcolor[rgb]{1,0.85,0.72}{%
	\makecell[{{p{\linewidth}}}]{% 
	  \tt {\tiny [GM$\langle$B]}  
	 SPRICH: Ja, das klingt gut! Ich bin mit der Zielanordnung einverstanden: A(1,1), O(3,2), T(2,2), P(2,5), D(2,4), S(3,6), M(6,2). Lass uns anfangen! Ich beginne mit dem Verschieben von M von (7,1) nach (6,2). 
	  } 
	   } 
	   } 
	  \\ 
 

    \theutterance \stepcounter{utterance}  

    & & & \multicolumn{2}{p{0.3\linewidth}}{\cellcolor[rgb]{0.95,0.95,0.95}{%
	\makecell[{{p{\linewidth}}}]{% 
	  \tt {\tiny [GM$|$GM]}  
	 SPRICH: Ja, das klingt gut! Ich bin mit der Zielanordnung einverstanden: A(1,1), O(3,2), T(2,2), P(2,5), D(2,4), S(3,6), M(6,2). Lass uns anfangen! Ich beginne mit dem Verschieben von M von (7,1) nach (6,2). 
	  } 
	   } 
	   } 
	 & & \\ 
 

    \theutterance \stepcounter{utterance}  

    & & \multicolumn{4}{p{0.6\linewidth}}{\cellcolor[rgb]{0.9,0.9,0.9}{%
	\makecell[{{p{\linewidth}}}]{% 
	  \tt {\tiny [A$\langle$GM]}  
	 Deine Nachricht wurde an deinen Mitspieler übermittelt.\\ \tt \\ \tt \\ \tt Ihr seid gerade bei Runde 2 von maximal 28.\\ \tt Ihr habt gemeinsam 2 von maximal 16 Strafpunkten angesammelt.\\ \tt Dein Mitspieler hat dir diese Nachricht geschickt:\\ \tt \\ \tt \\ \tt "Ja, das klingt gut! Ich bin mit der Zielanordnung einverstanden: A(1,1), O(3,2), T(2,2), P(2,5), D(2,4), S(3,6), M(6,2). Lass uns anfangen! Ich beginne mit dem Verschieben von M von (7,1) nach (6,2)."\\ \tt \\ \tt \\ \tt Wie lautet dein nächster Befehl? 
	  } 
	   } 
	   } 
	 & \\ 
 

    \theutterance \stepcounter{utterance}  

    & \multicolumn{4}{p{0.6\linewidth}}{\cellcolor[rgb]{0.8,1,0.9}{%
	\makecell[{{p{\linewidth}}}]{% 
	  \tt {\tiny [A$\rangle$GM]}  
	 VERSCHIEBE: O, (3, 2) 
	  } 
	   } 
	   } 
	 & & \\ 
 

    \theutterance \stepcounter{utterance}  

    & & & \multicolumn{2}{p{0.3\linewidth}}{\cellcolor[rgb]{0.95,0.95,0.95}{%
	\makecell[{{p{\linewidth}}}]{% 
	  \tt {\tiny [GM$|$GM]}  
	 VERSCHIEBE: O, (3, 2) 
	  } 
	   } 
	   } 
	 & & \\ 
 

    \theutterance \stepcounter{utterance}  

    & & & \multicolumn{2}{p{0.3\linewidth}}{\cellcolor[rgb]{0.95,0.95,0.95}{%
	\makecell[{{p{\linewidth}}}]{% 
	  \tt {\tiny [GM$|$GM]}  
	 Strafpunkt: (3,2) ist nicht leer, sondern enthält '─'.\\ \tt \\ \tt Ihr habt gemeinsam 3 von maximal 16 Strafpunkten angesammelt.\\ \tt Bitte versuche es noch mal! 
	  } 
	   } 
	   } 
	 & & \\ 
 

    \theutterance \stepcounter{utterance}  

    & & \multicolumn{4}{p{0.6\linewidth}}{\cellcolor[rgb]{0.9,0.9,0.9}{%
	\makecell[{{p{\linewidth}}}]{% 
	  \tt {\tiny [A$\langle$GM]}  
	 Strafpunkt: (3,2) ist nicht leer, sondern enthält '─'.\\ \tt \\ \tt Ihr habt gemeinsam 3 von maximal 16 Strafpunkten angesammelt.\\ \tt Bitte versuche es noch mal! 
	  } 
	   } 
	   } 
	 & \\ 
 

    \theutterance \stepcounter{utterance}  

    & \multicolumn{4}{p{0.6\linewidth}}{\cellcolor[rgb]{0.8,1,0.9}{%
	\makecell[{{p{\linewidth}}}]{% 
	  \tt {\tiny [A$\rangle$GM]}  
	 VERSCHIEBE: T, (2, 2) 
	  } 
	   } 
	   } 
	 & & \\ 
 

    \theutterance \stepcounter{utterance}  

    & & & \multicolumn{2}{p{0.3\linewidth}}{\cellcolor[rgb]{0.95,0.95,0.95}{%
	\makecell[{{p{\linewidth}}}]{% 
	  \tt {\tiny [GM$|$GM]}  
	 VERSCHIEBE: T, (2, 2) 
	  } 
	   } 
	   } 
	 & & \\ 
 

    \theutterance \stepcounter{utterance}  

    & & & \multicolumn{2}{p{0.3\linewidth}}{\cellcolor[rgb]{0.95,0.95,0.95}{%
	\makecell[{{p{\linewidth}}}]{% 
	  \tt {\tiny [GM$|$GM]}  
	 Strafpunkt: (2,2) ist nicht leer, sondern enthält '┼'.\\ \tt \\ \tt Ihr habt gemeinsam 4 von maximal 16 Strafpunkten angesammelt.\\ \tt Bitte versuche es noch mal! 
	  } 
	   } 
	   } 
	 & & \\ 
 

    \theutterance \stepcounter{utterance}  

    & & \multicolumn{4}{p{0.6\linewidth}}{\cellcolor[rgb]{0.9,0.9,0.9}{%
	\makecell[{{p{\linewidth}}}]{% 
	  \tt {\tiny [A$\langle$GM]}  
	 Strafpunkt: (2,2) ist nicht leer, sondern enthält '┼'.\\ \tt \\ \tt Ihr habt gemeinsam 4 von maximal 16 Strafpunkten angesammelt.\\ \tt Bitte versuche es noch mal! 
	  } 
	   } 
	   } 
	 & \\ 
 

    \theutterance \stepcounter{utterance}  

    & \multicolumn{4}{p{0.6\linewidth}}{\cellcolor[rgb]{0.8,1,0.9}{%
	\makecell[{{p{\linewidth}}}]{% 
	  \tt {\tiny [A$\rangle$GM]}  
	 SPRICH: Entschuldigung, ich muss die Zielkoordinaten anpassen - einige Positionen haben Rahmenlinien. Neue Vorschläge für freie Positionen: A(1,1), O(1,2), T(3,2), P(2,5), D(2,4), S(3,6), M(5,5). Einverstanden? 
	  } 
	   } 
	   } 
	 & & \\ 
 

    \theutterance \stepcounter{utterance}  

    & & & \multicolumn{2}{p{0.3\linewidth}}{\cellcolor[rgb]{0.95,0.95,0.95}{%
	\makecell[{{p{\linewidth}}}]{% 
	  \tt {\tiny [GM$|$GM]}  
	 SPRICH: Entschuldigung, ich muss die Zielkoordinaten anpassen - einige Positionen haben Rahmenlinien. Neue Vorschläge für freie Positionen: A(1,1), O(1,2), T(3,2), P(2,5), D(2,4), S(3,6), M(5,5). Einverstanden? 
	  } 
	   } 
	   } 
	 & & \\ 
 

    \theutterance \stepcounter{utterance}  

    & & \multicolumn{4}{p{0.6\linewidth}}{\cellcolor[rgb]{0.9,0.9,0.9}{%
	\makecell[{{p{\linewidth}}}]{% 
	  \tt {\tiny [GM$\rangle$B]}  
	 Deine Nachricht wurde an deinen Mitspieler übermittelt.\\ \tt \\ \tt \\ \tt Ihr seid gerade bei Runde 3 von maximal 28.\\ \tt Ihr habt gemeinsam 4 von maximal 16 Strafpunkten angesammelt.\\ \tt Dein Mitspieler hat dir diese Nachricht geschickt:\\ \tt \\ \tt \\ \tt "Entschuldigung, ich muss die Zielkoordinaten anpassen - einige Positionen haben Rahmenlinien. Neue Vorschläge für freie Positionen: A(1,1), O(1,2), T(3,2), P(2,5), D(2,4), S(3,6), M(5,5). Einverstanden?"\\ \tt \\ \tt \\ \tt Wie lautet dein nächster Befehl? 
	  } 
	   } 
	   } 
	 & \\ 
 

    \theutterance \stepcounter{utterance}  

    & & & \multicolumn{4}{p{0.6\linewidth}}{\cellcolor[rgb]{1,0.85,0.72}{%
	\makecell[{{p{\linewidth}}}]{% 
	  \tt {\tiny [GM$\langle$B]}  
	 SPRICH: Ja, einverstanden! Die neuen Zielkoordinaten sind gut: A(1,1), O(1,2), T(3,2), P(2,5), D(2,4), S(3,6), M(5,5). Ich fange jetzt an mit den Verschiebungen. 
	  } 
	   } 
	   } 
	  \\ 
 

    \theutterance \stepcounter{utterance}  

    & & & \multicolumn{2}{p{0.3\linewidth}}{\cellcolor[rgb]{0.95,0.95,0.95}{%
	\makecell[{{p{\linewidth}}}]{% 
	  \tt {\tiny [GM$|$GM]}  
	 SPRICH: Ja, einverstanden! Die neuen Zielkoordinaten sind gut: A(1,1), O(1,2), T(3,2), P(2,5), D(2,4), S(3,6), M(5,5). Ich fange jetzt an mit den Verschiebungen. 
	  } 
	   } 
	   } 
	 & & \\ 
 

    \theutterance \stepcounter{utterance}  

    & & \multicolumn{4}{p{0.6\linewidth}}{\cellcolor[rgb]{0.9,0.9,0.9}{%
	\makecell[{{p{\linewidth}}}]{% 
	  \tt {\tiny [A$\langle$GM]}  
	 Deine Nachricht wurde an deinen Mitspieler übermittelt.\\ \tt \\ \tt \\ \tt Ihr seid gerade bei Runde 3 von maximal 28.\\ \tt Ihr habt gemeinsam 4 von maximal 16 Strafpunkten angesammelt.\\ \tt Dein Mitspieler hat dir diese Nachricht geschickt:\\ \tt \\ \tt \\ \tt "Ja, einverstanden! Die neuen Zielkoordinaten sind gut: A(1,1), O(1,2), T(3,2), P(2,5), D(2,4), S(3,6), M(5,5). Ich fange jetzt an mit den Verschiebungen."\\ \tt \\ \tt \\ \tt Wie lautet dein nächster Befehl? 
	  } 
	   } 
	   } 
	 & \\ 
 

    \theutterance \stepcounter{utterance}  

    & \multicolumn{4}{p{0.6\linewidth}}{\cellcolor[rgb]{0.8,1,0.9}{%
	\makecell[{{p{\linewidth}}}]{% 
	  \tt {\tiny [A$\rangle$GM]}  
	 VERSCHIEBE: O, (1, 2) 
	  } 
	   } 
	   } 
	 & & \\ 
 

    \theutterance \stepcounter{utterance}  

    & & & \multicolumn{2}{p{0.3\linewidth}}{\cellcolor[rgb]{0.95,0.95,0.95}{%
	\makecell[{{p{\linewidth}}}]{% 
	  \tt {\tiny [GM$|$GM]}  
	 VERSCHIEBE: O, (1, 2) 
	  } 
	   } 
	   } 
	 & & \\ 
 

    \theutterance \stepcounter{utterance}  

    & & & \multicolumn{2}{p{0.3\linewidth}}{\cellcolor[rgb]{0.95,0.95,0.95}{%
	\makecell[{{p{\linewidth}}}]{% 
	  \tt {\tiny [GM$|$GM]}  
	 Strafpunkt: (1,2) ist nicht leer, sondern enthält '─'.\\ \tt \\ \tt Ihr habt gemeinsam 5 von maximal 16 Strafpunkten angesammelt.\\ \tt Bitte versuche es noch mal! 
	  } 
	   } 
	   } 
	 & & \\ 
 

    \theutterance \stepcounter{utterance}  

    & & \multicolumn{4}{p{0.6\linewidth}}{\cellcolor[rgb]{0.9,0.9,0.9}{%
	\makecell[{{p{\linewidth}}}]{% 
	  \tt {\tiny [A$\langle$GM]}  
	 Strafpunkt: (1,2) ist nicht leer, sondern enthält '─'.\\ \tt \\ \tt Ihr habt gemeinsam 5 von maximal 16 Strafpunkten angesammelt.\\ \tt Bitte versuche es noch mal! 
	  } 
	   } 
	   } 
	 & \\ 
 

    \theutterance \stepcounter{utterance}  

    & \multicolumn{4}{p{0.6\linewidth}}{\cellcolor[rgb]{0.8,1,0.9}{%
	\makecell[{{p{\linewidth}}}]{% 
	  \tt {\tiny [A$\rangle$GM]}  
	 SPRICH: Ich muss das Spielbrett genauer analysieren. Lass mich nur freie Positionen mit ◌ vorschlagen: A bleibt bei (1,1), O nach (4,1), T nach (1,3), P nach (4,4), D nach (2,5), S nach (5,6), M nach (7,6). Einverstanden? 
	  } 
	   } 
	   } 
	 & & \\ 
 

    \theutterance \stepcounter{utterance}  

    & & & \multicolumn{2}{p{0.3\linewidth}}{\cellcolor[rgb]{0.95,0.95,0.95}{%
	\makecell[{{p{\linewidth}}}]{% 
	  \tt {\tiny [GM$|$GM]}  
	 SPRICH: Ich muss das Spielbrett genauer analysieren. Lass mich nur freie Positionen mit ◌ vorschlagen: A bleibt bei (1,1), O nach (4,1), T nach (1,3), P nach (4,4), D nach (2,5), S nach (5,6), M nach (7,6). Einverstanden? 
	  } 
	   } 
	   } 
	 & & \\ 
 

    \theutterance \stepcounter{utterance}  

    & & \multicolumn{4}{p{0.6\linewidth}}{\cellcolor[rgb]{0.9,0.9,0.9}{%
	\makecell[{{p{\linewidth}}}]{% 
	  \tt {\tiny [GM$\rangle$B]}  
	 Deine Nachricht wurde an deinen Mitspieler übermittelt.\\ \tt \\ \tt \\ \tt Ihr seid gerade bei Runde 4 von maximal 28.\\ \tt Ihr habt gemeinsam 5 von maximal 16 Strafpunkten angesammelt.\\ \tt Dein Mitspieler hat dir diese Nachricht geschickt:\\ \tt \\ \tt \\ \tt "Ich muss das Spielbrett genauer analysieren. Lass mich nur freie Positionen mit ◌ vorschlagen: A bleibt bei (1,1), O nach (4,1), T nach (1,3), P nach (4,4), D nach (2,5), S nach (5,6), M nach (7,6). Einverstanden?"\\ \tt \\ \tt \\ \tt Wie lautet dein nächster Befehl? 
	  } 
	   } 
	   } 
	 & \\ 
 

    \theutterance \stepcounter{utterance}  

    & & & \multicolumn{4}{p{0.6\linewidth}}{\cellcolor[rgb]{1,0.85,0.72}{%
	\makecell[{{p{\linewidth}}}]{% 
	  \tt {\tiny [GM$\langle$B]}  
	 SPRICH: Einverstanden! Die Zielkoordinaten sind: A(1,1), O(4,1), T(1,3), P(4,4), D(2,5), S(5,6), M(7,6). Ich beginne jetzt mit den Verschiebungen. 
	  } 
	   } 
	   } 
	  \\ 
 

    \theutterance \stepcounter{utterance}  

    & & & \multicolumn{2}{p{0.3\linewidth}}{\cellcolor[rgb]{0.95,0.95,0.95}{%
	\makecell[{{p{\linewidth}}}]{% 
	  \tt {\tiny [GM$|$GM]}  
	 SPRICH: Einverstanden! Die Zielkoordinaten sind: A(1,1), O(4,1), T(1,3), P(4,4), D(2,5), S(5,6), M(7,6). Ich beginne jetzt mit den Verschiebungen. 
	  } 
	   } 
	   } 
	 & & \\ 
 

    \theutterance \stepcounter{utterance}  

    & & \multicolumn{4}{p{0.6\linewidth}}{\cellcolor[rgb]{0.9,0.9,0.9}{%
	\makecell[{{p{\linewidth}}}]{% 
	  \tt {\tiny [A$\langle$GM]}  
	 Deine Nachricht wurde an deinen Mitspieler übermittelt.\\ \tt \\ \tt \\ \tt Ihr seid gerade bei Runde 4 von maximal 28.\\ \tt Ihr habt gemeinsam 5 von maximal 16 Strafpunkten angesammelt.\\ \tt Dein Mitspieler hat dir diese Nachricht geschickt:\\ \tt \\ \tt \\ \tt "Einverstanden! Die Zielkoordinaten sind: A(1,1), O(4,1), T(1,3), P(4,4), D(2,5), S(5,6), M(7,6). Ich beginne jetzt mit den Verschiebungen."\\ \tt \\ \tt \\ \tt Wie lautet dein nächster Befehl? 
	  } 
	   } 
	   } 
	 & \\ 
 

    \theutterance \stepcounter{utterance}  

    & \multicolumn{4}{p{0.6\linewidth}}{\cellcolor[rgb]{0.8,1,0.9}{%
	\makecell[{{p{\linewidth}}}]{% 
	  \tt {\tiny [A$\rangle$GM]}  
	 VERSCHIEBE: O, (4, 1) 
	  } 
	   } 
	   } 
	 & & \\ 
 

    \theutterance \stepcounter{utterance}  

    & & & \multicolumn{2}{p{0.3\linewidth}}{\cellcolor[rgb]{0.95,0.95,0.95}{%
	\makecell[{{p{\linewidth}}}]{% 
	  \tt {\tiny [GM$|$GM]}  
	 VERSCHIEBE: O, (4, 1) 
	  } 
	   } 
	   } 
	 & & \\ 
 

    \theutterance \stepcounter{utterance}  

    & & & \multicolumn{2}{p{0.3\linewidth}}{\cellcolor[rgb]{0.95,0.95,0.95}{%
	\makecell[{{p{\linewidth}}}]{% 
	  \tt {\tiny [GM$|$GM]}  
	 Strafpunkt: (4,1) ist nicht leer, sondern enthält '│'.\\ \tt \\ \tt Ihr habt gemeinsam 6 von maximal 16 Strafpunkten angesammelt.\\ \tt Bitte versuche es noch mal! 
	  } 
	   } 
	   } 
	 & & \\ 
 

    \theutterance \stepcounter{utterance}  

    & & \multicolumn{4}{p{0.6\linewidth}}{\cellcolor[rgb]{0.9,0.9,0.9}{%
	\makecell[{{p{\linewidth}}}]{% 
	  \tt {\tiny [A$\langle$GM]}  
	 Strafpunkt: (4,1) ist nicht leer, sondern enthält '│'.\\ \tt \\ \tt Ihr habt gemeinsam 6 von maximal 16 Strafpunkten angesammelt.\\ \tt Bitte versuche es noch mal! 
	  } 
	   } 
	   } 
	 & \\ 
 

    \theutterance \stepcounter{utterance}  

    & \multicolumn{4}{p{0.6\linewidth}}{\cellcolor[rgb]{0.8,1,0.9}{%
	\makecell[{{p{\linewidth}}}]{% 
	  \tt {\tiny [A$\rangle$GM]}  
	 VERSCHIEBE: T, (1, 3) 
	  } 
	   } 
	   } 
	 & & \\ 
 

    \theutterance \stepcounter{utterance}  

    & & & \multicolumn{2}{p{0.3\linewidth}}{\cellcolor[rgb]{0.95,0.95,0.95}{%
	\makecell[{{p{\linewidth}}}]{% 
	  \tt {\tiny [GM$|$GM]}  
	 VERSCHIEBE: T, (1, 3) 
	  } 
	   } 
	   } 
	 & & \\ 
 

    \theutterance \stepcounter{utterance}  

    & & & \multicolumn{2}{p{0.3\linewidth}}{\cellcolor[rgb]{0.95,0.95,0.95}{%
	\makecell[{{p{\linewidth}}}]{% 
	  \tt {\tiny [GM$|$GM]}  
	 'T' erfolgreich nach (1,3) verschoben. Dein Raster sieht jetzt so aus:\\ \tt ```\\ \tt  1234567\\ \tt ╔═╤═╤═╤═╗\\ \tt ║A│◌│O│◌║ 1\\ \tt ╟─┼─┤◌│◌║ 2\\ \tt ║T│◌│◌│P║ 3\\ \tt ║◌│◌├─┴─╢ 4\\ \tt ╟─┴─┤◌◌◌║ 5\\ \tt ║D◌◌└─┬─╢ 6\\ \tt ║◌◌◌S◌│M║ 7\\ \tt ╚═════╧═╝\\ \tt ```\\ \tt  
	  } 
	   } 
	   } 
	 & & \\ 
 

    \theutterance \stepcounter{utterance}  

    & & \multicolumn{4}{p{0.6\linewidth}}{\cellcolor[rgb]{0.9,0.9,0.9}{%
	\makecell[{{p{\linewidth}}}]{% 
	  \tt {\tiny [GM$\rangle$B]}  
	 Deine Nachricht wurde an deinen Mitspieler übermittelt.\\ \tt \\ \tt \\ \tt Ihr seid gerade bei Runde 5 von maximal 28.\\ \tt Ihr habt gemeinsam 6 von maximal 16 Strafpunkten angesammelt.\\ \tt Dein Mitspieler hat ein Objekt auf seinem Raster bewegt.\\ \tt \\ \tt \\ \tt Wie lautet dein nächster Befehl? 
	  } 
	   } 
	   } 
	 & \\ 
 

    \theutterance \stepcounter{utterance}  

    & & & \multicolumn{4}{p{0.6\linewidth}}{\cellcolor[rgb]{1,0.85,0.72}{%
	\makecell[{{p{\linewidth}}}]{% 
	  \tt {\tiny [GM$\langle$B]}  
	 VERSCHIEBE: O, (4, 1) 
	  } 
	   } 
	   } 
	  \\ 
 

    \theutterance \stepcounter{utterance}  

    & & & \multicolumn{2}{p{0.3\linewidth}}{\cellcolor[rgb]{0.95,0.95,0.95}{%
	\makecell[{{p{\linewidth}}}]{% 
	  \tt {\tiny [GM$|$GM]}  
	 VERSCHIEBE: O, (4, 1) 
	  } 
	   } 
	   } 
	 & & \\ 
 

    \theutterance \stepcounter{utterance}  

    & & & \multicolumn{2}{p{0.3\linewidth}}{\cellcolor[rgb]{0.95,0.95,0.95}{%
	\makecell[{{p{\linewidth}}}]{% 
	  \tt {\tiny [GM$|$GM]}  
	 Strafpunkt: (4,1) ist nicht leer, sondern enthält '│'.\\ \tt \\ \tt Ihr habt gemeinsam 7 von maximal 16 Strafpunkten angesammelt.\\ \tt Bitte versuche es noch mal! 
	  } 
	   } 
	   } 
	 & & \\ 
 

    \theutterance \stepcounter{utterance}  

    & & \multicolumn{4}{p{0.6\linewidth}}{\cellcolor[rgb]{0.9,0.9,0.9}{%
	\makecell[{{p{\linewidth}}}]{% 
	  \tt {\tiny [GM$\rangle$B]}  
	 Strafpunkt: (4,1) ist nicht leer, sondern enthält '│'.\\ \tt \\ \tt Ihr habt gemeinsam 7 von maximal 16 Strafpunkten angesammelt.\\ \tt Bitte versuche es noch mal! 
	  } 
	   } 
	   } 
	 & \\ 
 

    \theutterance \stepcounter{utterance}  

    & & & \multicolumn{4}{p{0.6\linewidth}}{\cellcolor[rgb]{1,0.85,0.72}{%
	\makecell[{{p{\linewidth}}}]{% 
	  \tt {\tiny [GM$\langle$B]}  
	 SPRICH: Position (4,1) ist bei mir nicht frei - dort ist eine Linie. Können wir die Zielkoordinaten nochmal überprüfen? Welche Positionen sind auf deinem Brett wirklich frei? 
	  } 
	   } 
	   } 
	  \\ 
 

    \theutterance \stepcounter{utterance}  

    & & & \multicolumn{2}{p{0.3\linewidth}}{\cellcolor[rgb]{0.95,0.95,0.95}{%
	\makecell[{{p{\linewidth}}}]{% 
	  \tt {\tiny [GM$|$GM]}  
	 SPRICH: Position (4,1) ist bei mir nicht frei - dort ist eine Linie. Können wir die Zielkoordinaten nochmal überprüfen? Welche Positionen sind auf deinem Brett wirklich frei? 
	  } 
	   } 
	   } 
	 & & \\ 
 

    \theutterance \stepcounter{utterance}  

    & & \multicolumn{4}{p{0.6\linewidth}}{\cellcolor[rgb]{0.9,0.9,0.9}{%
	\makecell[{{p{\linewidth}}}]{% 
	  \tt {\tiny [A$\langle$GM]}  
	 'T' erfolgreich nach (1,3) verschoben. Dein Raster sieht jetzt so aus:\\ \tt ```\\ \tt  1234567\\ \tt ╔═╤═╤═╤═╗\\ \tt ║A│◌│O│◌║ 1\\ \tt ╟─┼─┤◌│◌║ 2\\ \tt ║T│◌│◌│P║ 3\\ \tt ║◌│◌├─┴─╢ 4\\ \tt ╟─┴─┤◌◌◌║ 5\\ \tt ║D◌◌└─┬─╢ 6\\ \tt ║◌◌◌S◌│M║ 7\\ \tt ╚═════╧═╝\\ \tt ```\\ \tt Ihr seid gerade bei Runde 5 von maximal 28.\\ \tt Ihr habt gemeinsam 7 von maximal 16 Strafpunkten angesammelt.\\ \tt Dein Mitspieler hat dir diese Nachricht geschickt:\\ \tt \\ \tt \\ \tt "Position (4,1) ist bei mir nicht frei - dort ist eine Linie. Können wir die Zielkoordinaten nochmal überprüfen? Welche Positionen sind auf deinem Brett wirklich frei?"\\ \tt \\ \tt \\ \tt Wie lautet dein nächster Befehl? 
	  } 
	   } 
	   } 
	 & \\ 
 

    \theutterance \stepcounter{utterance}  

    & \multicolumn{4}{p{0.6\linewidth}}{\cellcolor[rgb]{0.8,1,0.9}{%
	\makecell[{{p{\linewidth}}}]{% 
	  \tt {\tiny [A$\rangle$GM]}  
	 SPRICH: Du hast recht! Lass mich nur wirklich freie Positionen nennen. Auf meinem Brett sind frei: (2,1), (4,1), (3,2), (4,2), (1,4), (2,4), (4,5), (5,5), (6,5), (2,6), (3,6), (1,7), (2,7), (3,7), (5,7). Vorschlag: A(1,1), O(2,1), T(1,3), P(4,2), D(1,4), S(2,6), M(5,7). 
	  } 
	   } 
	   } 
	 & & \\ 
 

    \theutterance \stepcounter{utterance}  

    & & & \multicolumn{2}{p{0.3\linewidth}}{\cellcolor[rgb]{0.95,0.95,0.95}{%
	\makecell[{{p{\linewidth}}}]{% 
	  \tt {\tiny [GM$|$GM]}  
	 SPRICH: Du hast recht! Lass mich nur wirklich freie Positionen nennen. Auf meinem Brett sind frei: (2,1), (4,1), (3,2), (4,2), (1,4), (2,4), (4,5), (5,5), (6,5), (2,6), (3,6), (1,7), (2,7), (3,7), (5,7). Vorschlag: A(1,1), O(2,1), T(1,3), P(4,2), D(1,4), S(2,6), M(5,7). 
	  } 
	   } 
	   } 
	 & & \\ 
 

    \theutterance \stepcounter{utterance}  

    & & \multicolumn{4}{p{0.6\linewidth}}{\cellcolor[rgb]{0.9,0.9,0.9}{%
	\makecell[{{p{\linewidth}}}]{% 
	  \tt {\tiny [GM$\rangle$B]}  
	 Deine Nachricht wurde an deinen Mitspieler übermittelt.\\ \tt \\ \tt \\ \tt Ihr seid gerade bei Runde 6 von maximal 28.\\ \tt Ihr habt gemeinsam 7 von maximal 16 Strafpunkten angesammelt.\\ \tt Dein Mitspieler hat dir diese Nachricht geschickt:\\ \tt \\ \tt \\ \tt "Du hast recht! Lass mich nur wirklich freie Positionen nennen. Auf meinem Brett sind frei: (2,1), (4,1), (3,2), (4,2), (1,4), (2,4), (4,5), (5,5), (6,5), (2,6), (3,6), (1,7), (2,7), (3,7), (5,7). Vorschlag: A(1,1), O(2,1), T(1,3), P(4,2), D(1,4), S(2,6), M(5,7)."\\ \tt \\ \tt \\ \tt Wie lautet dein nächster Befehl? 
	  } 
	   } 
	   } 
	 & \\ 
 

    \theutterance \stepcounter{utterance}  

    & & & \multicolumn{4}{p{0.6\linewidth}}{\cellcolor[rgb]{1,0.85,0.72}{%
	\makecell[{{p{\linewidth}}}]{% 
	  \tt {\tiny [GM$\langle$B]}  
	 SPRICH: Gut! Ich prüfe diese Positionen auf meinem Brett: (2,1) ist frei, (1,3) ist frei, (4,2) ist nicht frei (dort ist T), (1,4) ist frei, (2,6) ist nicht frei (dort ist S), (5,7) ist frei. Können wir anpassen? Vorschlag: A(1,1), O(2,1), T(1,3), P(1,4), D(2,4), S(3,6), M(5,7)? 
	  } 
	   } 
	   } 
	  \\ 
 

    \theutterance \stepcounter{utterance}  

    & & & \multicolumn{2}{p{0.3\linewidth}}{\cellcolor[rgb]{0.95,0.95,0.95}{%
	\makecell[{{p{\linewidth}}}]{% 
	  \tt {\tiny [GM$|$GM]}  
	 SPRICH: Gut! Ich prüfe diese Positionen auf meinem Brett: (2,1) ist frei, (1,3) ist frei, (4,2) ist nicht frei (dort ist T), (1,4) ist frei, (2,6) ist nicht frei (dort ist S), (5,7) ist frei. Können wir anpassen? Vorschlag: A(1,1), O(2,1), T(1,3), P(1,4), D(2,4), S(3,6), M(5,7)? 
	  } 
	   } 
	   } 
	 & & \\ 
 

    \theutterance \stepcounter{utterance}  

    & & \multicolumn{4}{p{0.6\linewidth}}{\cellcolor[rgb]{0.9,0.9,0.9}{%
	\makecell[{{p{\linewidth}}}]{% 
	  \tt {\tiny [A$\langle$GM]}  
	 Deine Nachricht wurde an deinen Mitspieler übermittelt.\\ \tt \\ \tt \\ \tt Ihr seid gerade bei Runde 6 von maximal 28.\\ \tt Ihr habt gemeinsam 7 von maximal 16 Strafpunkten angesammelt.\\ \tt Dein Mitspieler hat dir diese Nachricht geschickt:\\ \tt \\ \tt \\ \tt "Gut! Ich prüfe diese Positionen auf meinem Brett: (2,1) ist frei, (1,3) ist frei, (4,2) ist nicht frei (dort ist T), (1,4) ist frei, (2,6) ist nicht frei (dort ist S), (5,7) ist frei. Können wir anpassen? Vorschlag: A(1,1), O(2,1), T(1,3), P(1,4), D(2,4), S(3,6), M(5,7)?"\\ \tt \\ \tt \\ \tt Wie lautet dein nächster Befehl? 
	  } 
	   } 
	   } 
	 & \\ 
 

    \theutterance \stepcounter{utterance}  

    & \multicolumn{4}{p{0.6\linewidth}}{\cellcolor[rgb]{0.8,1,0.9}{%
	\makecell[{{p{\linewidth}}}]{% 
	  \tt {\tiny [A$\rangle$GM]}  
	 SPRICH: Perfekt! Ich bin einverstanden: A(1,1), O(2,1), T(1,3), P(1,4), D(2,4), S(3,6), M(5,7). A und T sind schon richtig positioniert. Ich verschiebe als nächstes O von (3,1) nach (2,1). 
	  } 
	   } 
	   } 
	 & & \\ 
 

    \theutterance \stepcounter{utterance}  

    & & & \multicolumn{2}{p{0.3\linewidth}}{\cellcolor[rgb]{0.95,0.95,0.95}{%
	\makecell[{{p{\linewidth}}}]{% 
	  \tt {\tiny [GM$|$GM]}  
	 SPRICH: Perfekt! Ich bin einverstanden: A(1,1), O(2,1), T(1,3), P(1,4), D(2,4), S(3,6), M(5,7). A und T sind schon richtig positioniert. Ich verschiebe als nächstes O von (3,1) nach (2,1). 
	  } 
	   } 
	   } 
	 & & \\ 
 

    \theutterance \stepcounter{utterance}  

    & & \multicolumn{4}{p{0.6\linewidth}}{\cellcolor[rgb]{0.9,0.9,0.9}{%
	\makecell[{{p{\linewidth}}}]{% 
	  \tt {\tiny [GM$\rangle$B]}  
	 Deine Nachricht wurde an deinen Mitspieler übermittelt.\\ \tt \\ \tt \\ \tt Ihr seid gerade bei Runde 7 von maximal 28.\\ \tt Ihr habt gemeinsam 7 von maximal 16 Strafpunkten angesammelt.\\ \tt Dein Mitspieler hat dir diese Nachricht geschickt:\\ \tt \\ \tt \\ \tt "Perfekt! Ich bin einverstanden: A(1,1), O(2,1), T(1,3), P(1,4), D(2,4), S(3,6), M(5,7). A und T sind schon richtig positioniert. Ich verschiebe als nächstes O von (3,1) nach (2,1)."\\ \tt \\ \tt \\ \tt Wie lautet dein nächster Befehl? 
	  } 
	   } 
	   } 
	 & \\ 
 

    \theutterance \stepcounter{utterance}  

    & & & \multicolumn{4}{p{0.6\linewidth}}{\cellcolor[rgb]{1,0.85,0.72}{%
	\makecell[{{p{\linewidth}}}]{% 
	  \tt {\tiny [GM$\langle$B]}  
	 VERSCHIEBE: O, (2, 1) 
	  } 
	   } 
	   } 
	  \\ 
 

    \theutterance \stepcounter{utterance}  

    & & & \multicolumn{2}{p{0.3\linewidth}}{\cellcolor[rgb]{0.95,0.95,0.95}{%
	\makecell[{{p{\linewidth}}}]{% 
	  \tt {\tiny [GM$|$GM]}  
	 VERSCHIEBE: O, (2, 1) 
	  } 
	   } 
	   } 
	 & & \\ 
 

    \theutterance \stepcounter{utterance}  

    & & & \multicolumn{2}{p{0.3\linewidth}}{\cellcolor[rgb]{0.95,0.95,0.95}{%
	\makecell[{{p{\linewidth}}}]{% 
	  \tt {\tiny [GM$|$GM]}  
	 Strafpunkt: (2,1) ist nicht leer, sondern enthält '│'.\\ \tt \\ \tt Ihr habt gemeinsam 8 von maximal 16 Strafpunkten angesammelt.\\ \tt Bitte versuche es noch mal! 
	  } 
	   } 
	   } 
	 & & \\ 
 

    \theutterance \stepcounter{utterance}  

    & & \multicolumn{4}{p{0.6\linewidth}}{\cellcolor[rgb]{0.9,0.9,0.9}{%
	\makecell[{{p{\linewidth}}}]{% 
	  \tt {\tiny [GM$\rangle$B]}  
	 Strafpunkt: (2,1) ist nicht leer, sondern enthält '│'.\\ \tt \\ \tt Ihr habt gemeinsam 8 von maximal 16 Strafpunkten angesammelt.\\ \tt Bitte versuche es noch mal! 
	  } 
	   } 
	   } 
	 & \\ 
 

    \theutterance \stepcounter{utterance}  

    & & & \multicolumn{4}{p{0.6\linewidth}}{\cellcolor[rgb]{1,0.85,0.72}{%
	\makecell[{{p{\linewidth}}}]{% 
	  \tt {\tiny [GM$\langle$B]}  
	 SPRICH: (2,1) ist bei mir nicht frei - dort ist eine Linie. Wir haben unterschiedliche Spielbretter! Lass mich meine freien Positionen nennen: (2,1) nicht frei, (3,1) frei, (1,3) frei, (4,3) frei, (1,4) frei, (2,4) frei, (1,5) frei, (3,5) frei, (4,5) frei, (5,5) frei, (6,5) frei, (3,6) frei, (4,6) frei, (5,6) frei, (6,6) frei, (1,7) frei, (2,7) frei, (3,7) frei, (4,7) frei, (5,7) frei, (6,7) frei. 
	  } 
	   } 
	   } 
	  \\ 
 

    \theutterance \stepcounter{utterance}  

    & & & \multicolumn{2}{p{0.3\linewidth}}{\cellcolor[rgb]{0.95,0.95,0.95}{%
	\makecell[{{p{\linewidth}}}]{% 
	  \tt {\tiny [GM$|$GM]}  
	 SPRICH: (2,1) ist bei mir nicht frei - dort ist eine Linie. Wir haben unterschiedliche Spielbretter! Lass mich meine freien Positionen nennen: (2,1) nicht frei, (3,1) frei, (1,3) frei, (4,3) frei, (1,4) frei, (2,4) frei, (1,5) frei, (3,5) frei, (4,5) frei, (5,5) frei, (6,5) frei, (3,6) frei, (4,6) frei, (5,6) frei, (6,6) frei, (1,7) frei, (2,7) frei, (3,7) frei, (4,7) frei, (5,7) frei, (6,7) frei. 
	  } 
	   } 
	   } 
	 & & \\ 
 

    \theutterance \stepcounter{utterance}  

    & & \multicolumn{4}{p{0.6\linewidth}}{\cellcolor[rgb]{0.9,0.9,0.9}{%
	\makecell[{{p{\linewidth}}}]{% 
	  \tt {\tiny [A$\langle$GM]}  
	 Deine Nachricht wurde an deinen Mitspieler übermittelt.\\ \tt \\ \tt \\ \tt Ihr seid gerade bei Runde 7 von maximal 28.\\ \tt Ihr habt gemeinsam 8 von maximal 16 Strafpunkten angesammelt.\\ \tt Dein Mitspieler hat dir diese Nachricht geschickt:\\ \tt \\ \tt \\ \tt "(2,1) ist bei mir nicht frei - dort ist eine Linie. Wir haben unterschiedliche Spielbretter! Lass mich meine freien Positionen nennen: (2,1) nicht frei, (3,1) frei, (1,3) frei, (4,3) frei, (1,4) frei, (2,4) frei, (1,5) frei, (3,5) frei, (4,5) frei, (5,5) frei, (6,5) frei, (3,6) frei, (4,6) frei, (5,6) frei, (6,6) frei, (1,7) frei, (2,7) frei, (3,7) frei, (4,7) frei, (5,7) frei, (6,7) frei."\\ \tt \\ \tt \\ \tt Wie lautet dein nächster Befehl? 
	  } 
	   } 
	   } 
	 & \\ 
 

    \theutterance \stepcounter{utterance}  

    & \multicolumn{4}{p{0.6\linewidth}}{\cellcolor[rgb]{0.8,1,0.9}{%
	\makecell[{{p{\linewidth}}}]{% 
	  \tt {\tiny [A$\rangle$GM]}  
	 SPRICH: Du hast recht, unsere Bretter sind unterschiedlich! Lass uns Positionen finden, die bei beiden frei sind. Aus deiner Liste sind bei mir auch frei: (3,1), (1,4), (4,5), (5,5), (6,5), (3,6), (1,7), (2,7), (3,7), (5,7). Neuer Vorschlag: A(1,1), O(3,1), T(1,3), P(1,4), D(2,4), S(3,6), M(5,7)? 
	  } 
	   } 
	   } 
	 & & \\ 
 

    \theutterance \stepcounter{utterance}  

    & & & \multicolumn{2}{p{0.3\linewidth}}{\cellcolor[rgb]{0.95,0.95,0.95}{%
	\makecell[{{p{\linewidth}}}]{% 
	  \tt {\tiny [GM$|$GM]}  
	 SPRICH: Du hast recht, unsere Bretter sind unterschiedlich! Lass uns Positionen finden, die bei beiden frei sind. Aus deiner Liste sind bei mir auch frei: (3,1), (1,4), (4,5), (5,5), (6,5), (3,6), (1,7), (2,7), (3,7), (5,7). Neuer Vorschlag: A(1,1), O(3,1), T(1,3), P(1,4), D(2,4), S(3,6), M(5,7)? 
	  } 
	   } 
	   } 
	 & & \\ 
 

    \theutterance \stepcounter{utterance}  

    & & \multicolumn{4}{p{0.6\linewidth}}{\cellcolor[rgb]{0.9,0.9,0.9}{%
	\makecell[{{p{\linewidth}}}]{% 
	  \tt {\tiny [GM$\rangle$B]}  
	 Deine Nachricht wurde an deinen Mitspieler übermittelt.\\ \tt \\ \tt \\ \tt Ihr seid gerade bei Runde 8 von maximal 28.\\ \tt Ihr habt gemeinsam 8 von maximal 16 Strafpunkten angesammelt.\\ \tt Dein Mitspieler hat dir diese Nachricht geschickt:\\ \tt \\ \tt \\ \tt "Du hast recht, unsere Bretter sind unterschiedlich! Lass uns Positionen finden, die bei beiden frei sind. Aus deiner Liste sind bei mir auch frei: (3,1), (1,4), (4,5), (5,5), (6,5), (3,6), (1,7), (2,7), (3,7), (5,7). Neuer Vorschlag: A(1,1), O(3,1), T(1,3), P(1,4), D(2,4), S(3,6), M(5,7)?"\\ \tt \\ \tt \\ \tt Wie lautet dein nächster Befehl? 
	  } 
	   } 
	   } 
	 & \\ 
 

    \theutterance \stepcounter{utterance}  

    & & & \multicolumn{4}{p{0.6\linewidth}}{\cellcolor[rgb]{1,0.85,0.72}{%
	\makecell[{{p{\linewidth}}}]{% 
	  \tt {\tiny [GM$\langle$B]}  
	 SPRICH: Perfekt! Diese Positionen sind bei mir alle frei. Einverstanden mit: A(1,1), O(3,1), T(1,3), P(1,4), D(2,4), S(3,6), M(5,7). A ist schon richtig. Ich verschiebe jetzt O von (4,2) nach (3,1). 
	  } 
	   } 
	   } 
	  \\ 
 

    \theutterance \stepcounter{utterance}  

    & & & \multicolumn{2}{p{0.3\linewidth}}{\cellcolor[rgb]{0.95,0.95,0.95}{%
	\makecell[{{p{\linewidth}}}]{% 
	  \tt {\tiny [GM$|$GM]}  
	 SPRICH: Perfekt! Diese Positionen sind bei mir alle frei. Einverstanden mit: A(1,1), O(3,1), T(1,3), P(1,4), D(2,4), S(3,6), M(5,7). A ist schon richtig. Ich verschiebe jetzt O von (4,2) nach (3,1). 
	  } 
	   } 
	   } 
	 & & \\ 
 

    \theutterance \stepcounter{utterance}  

    & & \multicolumn{4}{p{0.6\linewidth}}{\cellcolor[rgb]{0.9,0.9,0.9}{%
	\makecell[{{p{\linewidth}}}]{% 
	  \tt {\tiny [A$\langle$GM]}  
	 Deine Nachricht wurde an deinen Mitspieler übermittelt.\\ \tt \\ \tt \\ \tt Ihr seid gerade bei Runde 8 von maximal 28.\\ \tt Ihr habt gemeinsam 8 von maximal 16 Strafpunkten angesammelt.\\ \tt Dein Mitspieler hat dir diese Nachricht geschickt:\\ \tt \\ \tt \\ \tt "Perfekt! Diese Positionen sind bei mir alle frei. Einverstanden mit: A(1,1), O(3,1), T(1,3), P(1,4), D(2,4), S(3,6), M(5,7). A ist schon richtig. Ich verschiebe jetzt O von (4,2) nach (3,1)."\\ \tt \\ \tt \\ \tt Wie lautet dein nächster Befehl? 
	  } 
	   } 
	   } 
	 & \\ 
 

    \theutterance \stepcounter{utterance}  

    & \multicolumn{4}{p{0.6\linewidth}}{\cellcolor[rgb]{0.8,1,0.9}{%
	\makecell[{{p{\linewidth}}}]{% 
	  \tt {\tiny [A$\rangle$GM]}  
	 VERSCHIEBE: O, (3, 1) 
	  } 
	   } 
	   } 
	 & & \\ 
 

    \theutterance \stepcounter{utterance}  

    & & & \multicolumn{2}{p{0.3\linewidth}}{\cellcolor[rgb]{0.95,0.95,0.95}{%
	\makecell[{{p{\linewidth}}}]{% 
	  \tt {\tiny [GM$|$GM]}  
	 VERSCHIEBE: O, (3, 1) 
	  } 
	   } 
	   } 
	 & & \\ 
 

    \theutterance \stepcounter{utterance}  

    & & & \multicolumn{2}{p{0.3\linewidth}}{\cellcolor[rgb]{0.95,0.95,0.95}{%
	\makecell[{{p{\linewidth}}}]{% 
	  \tt {\tiny [GM$|$GM]}  
	 'O' erfolgreich nach (3,1) verschoben. Dein Raster sieht jetzt so aus:\\ \tt ```\\ \tt  1234567\\ \tt ╔═╤═╤═╤═╗\\ \tt ║A│O│◌│◌║ 1\\ \tt ╟─┼─┤◌│◌║ 2\\ \tt ║T│◌│◌│P║ 3\\ \tt ║◌│◌├─┴─╢ 4\\ \tt ╟─┴─┤◌◌◌║ 5\\ \tt ║D◌◌└─┬─╢ 6\\ \tt ║◌◌◌S◌│M║ 7\\ \tt ╚═════╧═╝\\ \tt ```\\ \tt  
	  } 
	   } 
	   } 
	 & & \\ 
 

    \theutterance \stepcounter{utterance}  

    & & \multicolumn{4}{p{0.6\linewidth}}{\cellcolor[rgb]{0.9,0.9,0.9}{%
	\makecell[{{p{\linewidth}}}]{% 
	  \tt {\tiny [GM$\rangle$B]}  
	 Deine Nachricht wurde an deinen Mitspieler übermittelt.\\ \tt \\ \tt \\ \tt Ihr seid gerade bei Runde 9 von maximal 28.\\ \tt Ihr habt gemeinsam 8 von maximal 16 Strafpunkten angesammelt.\\ \tt Dein Mitspieler hat ein Objekt auf seinem Raster bewegt.\\ \tt \\ \tt \\ \tt Wie lautet dein nächster Befehl? 
	  } 
	   } 
	   } 
	 & \\ 
 

    \theutterance \stepcounter{utterance}  

    & & & \multicolumn{4}{p{0.6\linewidth}}{\cellcolor[rgb]{1,0.85,0.72}{%
	\makecell[{{p{\linewidth}}}]{% 
	  \tt {\tiny [GM$\langle$B]}  
	 VERSCHIEBE: O, (3, 1) 
	  } 
	   } 
	   } 
	  \\ 
 

    \theutterance \stepcounter{utterance}  

    & & & \multicolumn{2}{p{0.3\linewidth}}{\cellcolor[rgb]{0.95,0.95,0.95}{%
	\makecell[{{p{\linewidth}}}]{% 
	  \tt {\tiny [GM$|$GM]}  
	 VERSCHIEBE: O, (3, 1) 
	  } 
	   } 
	   } 
	 & & \\ 
 

    \theutterance \stepcounter{utterance}  

    & & & \multicolumn{2}{p{0.3\linewidth}}{\cellcolor[rgb]{0.95,0.95,0.95}{%
	\makecell[{{p{\linewidth}}}]{% 
	  \tt {\tiny [GM$|$GM]}  
	 'O' erfolgreich nach (3,1) verschoben. Dein Raster sieht jetzt so aus:\\ \tt ```\\ \tt  1234567\\ \tt ╔═╤═╤═╤═╗\\ \tt ║A│O│◌│M║ 1\\ \tt ╟─┼─┤◌│T║ 2\\ \tt ║◌│D│◌│◌║ 3\\ \tt ║◌│◌├─┴─╢ 4\\ \tt ╟─┴─┤◌◌◌║ 5\\ \tt ║PS◌└─┬─╢ 6\\ \tt ║◌◌◌◌◌│◌║ 7\\ \tt ╚═════╧═╝\\ \tt ```\\ \tt  
	  } 
	   } 
	   } 
	 & & \\ 
 

    \theutterance \stepcounter{utterance}  

    & & \multicolumn{4}{p{0.6\linewidth}}{\cellcolor[rgb]{0.9,0.9,0.9}{%
	\makecell[{{p{\linewidth}}}]{% 
	  \tt {\tiny [A$\langle$GM]}  
	 'O' erfolgreich nach (3,1) verschoben. Dein Raster sieht jetzt so aus:\\ \tt ```\\ \tt  1234567\\ \tt ╔═╤═╤═╤═╗\\ \tt ║A│O│◌│◌║ 1\\ \tt ╟─┼─┤◌│◌║ 2\\ \tt ║T│◌│◌│P║ 3\\ \tt ║◌│◌├─┴─╢ 4\\ \tt ╟─┴─┤◌◌◌║ 5\\ \tt ║D◌◌└─┬─╢ 6\\ \tt ║◌◌◌S◌│M║ 7\\ \tt ╚═════╧═╝\\ \tt ```\\ \tt Ihr seid gerade bei Runde 9 von maximal 28.\\ \tt Ihr habt gemeinsam 8 von maximal 16 Strafpunkten angesammelt.\\ \tt Dein Mitspieler hat ein Objekt auf seinem Raster bewegt.\\ \tt \\ \tt \\ \tt Wie lautet dein nächster Befehl? 
	  } 
	   } 
	   } 
	 & \\ 
 

    \theutterance \stepcounter{utterance}  

    & \multicolumn{4}{p{0.6\linewidth}}{\cellcolor[rgb]{0.8,1,0.9}{%
	\makecell[{{p{\linewidth}}}]{% 
	  \tt {\tiny [A$\rangle$GM]}  
	 VERSCHIEBE: P, (1, 4) 
	  } 
	   } 
	   } 
	 & & \\ 
 

    \theutterance \stepcounter{utterance}  

    & & & \multicolumn{2}{p{0.3\linewidth}}{\cellcolor[rgb]{0.95,0.95,0.95}{%
	\makecell[{{p{\linewidth}}}]{% 
	  \tt {\tiny [GM$|$GM]}  
	 VERSCHIEBE: P, (1, 4) 
	  } 
	   } 
	   } 
	 & & \\ 
 

    \theutterance \stepcounter{utterance}  

    & & & \multicolumn{2}{p{0.3\linewidth}}{\cellcolor[rgb]{0.95,0.95,0.95}{%
	\makecell[{{p{\linewidth}}}]{% 
	  \tt {\tiny [GM$|$GM]}  
	 'P' erfolgreich nach (1,4) verschoben. Dein Raster sieht jetzt so aus:\\ \tt ```\\ \tt  1234567\\ \tt ╔═╤═╤═╤═╗\\ \tt ║A│O│◌│◌║ 1\\ \tt ╟─┼─┤◌│◌║ 2\\ \tt ║T│◌│◌│◌║ 3\\ \tt ║P│◌├─┴─╢ 4\\ \tt ╟─┴─┤◌◌◌║ 5\\ \tt ║D◌◌└─┬─╢ 6\\ \tt ║◌◌◌S◌│M║ 7\\ \tt ╚═════╧═╝\\ \tt ```\\ \tt  
	  } 
	   } 
	   } 
	 & & \\ 
 

    \theutterance \stepcounter{utterance}  

    & & \multicolumn{4}{p{0.6\linewidth}}{\cellcolor[rgb]{0.9,0.9,0.9}{%
	\makecell[{{p{\linewidth}}}]{% 
	  \tt {\tiny [GM$\rangle$B]}  
	 'O' erfolgreich nach (3,1) verschoben. Dein Raster sieht jetzt so aus:\\ \tt ```\\ \tt  1234567\\ \tt ╔═╤═╤═╤═╗\\ \tt ║A│O│◌│M║ 1\\ \tt ╟─┼─┤◌│T║ 2\\ \tt ║◌│D│◌│◌║ 3\\ \tt ║◌│◌├─┴─╢ 4\\ \tt ╟─┴─┤◌◌◌║ 5\\ \tt ║PS◌└─┬─╢ 6\\ \tt ║◌◌◌◌◌│◌║ 7\\ \tt ╚═════╧═╝\\ \tt ```\\ \tt Ihr seid gerade bei Runde 10 von maximal 28.\\ \tt Ihr habt gemeinsam 8 von maximal 16 Strafpunkten angesammelt.\\ \tt Dein Mitspieler hat ein Objekt auf seinem Raster bewegt.\\ \tt \\ \tt \\ \tt Wie lautet dein nächster Befehl? 
	  } 
	   } 
	   } 
	 & \\ 
 

    \theutterance \stepcounter{utterance}  

    & & & \multicolumn{4}{p{0.6\linewidth}}{\cellcolor[rgb]{1,0.85,0.72}{%
	\makecell[{{p{\linewidth}}}]{% 
	  \tt {\tiny [GM$\langle$B]}  
	 VERSCHIEBE: T, (1, 3) 
	  } 
	   } 
	   } 
	  \\ 
 

    \theutterance \stepcounter{utterance}  

    & & & \multicolumn{2}{p{0.3\linewidth}}{\cellcolor[rgb]{0.95,0.95,0.95}{%
	\makecell[{{p{\linewidth}}}]{% 
	  \tt {\tiny [GM$|$GM]}  
	 VERSCHIEBE: T, (1, 3) 
	  } 
	   } 
	   } 
	 & & \\ 
 

    \theutterance \stepcounter{utterance}  

    & & & \multicolumn{2}{p{0.3\linewidth}}{\cellcolor[rgb]{0.95,0.95,0.95}{%
	\makecell[{{p{\linewidth}}}]{% 
	  \tt {\tiny [GM$|$GM]}  
	 'T' erfolgreich nach (1,3) verschoben. Dein Raster sieht jetzt so aus:\\ \tt ```\\ \tt  1234567\\ \tt ╔═╤═╤═╤═╗\\ \tt ║A│O│◌│M║ 1\\ \tt ╟─┼─┤◌│◌║ 2\\ \tt ║T│D│◌│◌║ 3\\ \tt ║◌│◌├─┴─╢ 4\\ \tt ╟─┴─┤◌◌◌║ 5\\ \tt ║PS◌└─┬─╢ 6\\ \tt ║◌◌◌◌◌│◌║ 7\\ \tt ╚═════╧═╝\\ \tt ```\\ \tt  
	  } 
	   } 
	   } 
	 & & \\ 
 

    \theutterance \stepcounter{utterance}  

    & & \multicolumn{4}{p{0.6\linewidth}}{\cellcolor[rgb]{0.9,0.9,0.9}{%
	\makecell[{{p{\linewidth}}}]{% 
	  \tt {\tiny [A$\langle$GM]}  
	 'P' erfolgreich nach (1,4) verschoben. Dein Raster sieht jetzt so aus:\\ \tt ```\\ \tt  1234567\\ \tt ╔═╤═╤═╤═╗\\ \tt ║A│O│◌│◌║ 1\\ \tt ╟─┼─┤◌│◌║ 2\\ \tt ║T│◌│◌│◌║ 3\\ \tt ║P│◌├─┴─╢ 4\\ \tt ╟─┴─┤◌◌◌║ 5\\ \tt ║D◌◌└─┬─╢ 6\\ \tt ║◌◌◌S◌│M║ 7\\ \tt ╚═════╧═╝\\ \tt ```\\ \tt Ihr seid gerade bei Runde 10 von maximal 28.\\ \tt Ihr habt gemeinsam 8 von maximal 16 Strafpunkten angesammelt.\\ \tt Dein Mitspieler hat ein Objekt auf seinem Raster bewegt.\\ \tt \\ \tt \\ \tt Wie lautet dein nächster Befehl? 
	  } 
	   } 
	   } 
	 & \\ 
 

    \theutterance \stepcounter{utterance}  

    & \multicolumn{4}{p{0.6\linewidth}}{\cellcolor[rgb]{0.8,1,0.9}{%
	\makecell[{{p{\linewidth}}}]{% 
	  \tt {\tiny [A$\rangle$GM]}  
	 VERSCHIEBE: D, (2, 4) 
	  } 
	   } 
	   } 
	 & & \\ 
 

    \theutterance \stepcounter{utterance}  

    & & & \multicolumn{2}{p{0.3\linewidth}}{\cellcolor[rgb]{0.95,0.95,0.95}{%
	\makecell[{{p{\linewidth}}}]{% 
	  \tt {\tiny [GM$|$GM]}  
	 VERSCHIEBE: D, (2, 4) 
	  } 
	   } 
	   } 
	 & & \\ 
 

    \theutterance \stepcounter{utterance}  

    & & & \multicolumn{2}{p{0.3\linewidth}}{\cellcolor[rgb]{0.95,0.95,0.95}{%
	\makecell[{{p{\linewidth}}}]{% 
	  \tt {\tiny [GM$|$GM]}  
	 Strafpunkt: (2,4) ist nicht leer, sondern enthält '│'.\\ \tt \\ \tt Ihr habt gemeinsam 9 von maximal 16 Strafpunkten angesammelt.\\ \tt Bitte versuche es noch mal! 
	  } 
	   } 
	   } 
	 & & \\ 
 

    \theutterance \stepcounter{utterance}  

    & & \multicolumn{4}{p{0.6\linewidth}}{\cellcolor[rgb]{0.9,0.9,0.9}{%
	\makecell[{{p{\linewidth}}}]{% 
	  \tt {\tiny [A$\langle$GM]}  
	 Strafpunkt: (2,4) ist nicht leer, sondern enthält '│'.\\ \tt \\ \tt Ihr habt gemeinsam 9 von maximal 16 Strafpunkten angesammelt.\\ \tt Bitte versuche es noch mal! 
	  } 
	   } 
	   } 
	 & \\ 
 

    \theutterance \stepcounter{utterance}  

    & \multicolumn{4}{p{0.6\linewidth}}{\cellcolor[rgb]{0.8,1,0.9}{%
	\makecell[{{p{\linewidth}}}]{% 
	  \tt {\tiny [A$\rangle$GM]}  
	 VERSCHIEBE: S, (3, 6) 
	  } 
	   } 
	   } 
	 & & \\ 
 

    \theutterance \stepcounter{utterance}  

    & & & \multicolumn{2}{p{0.3\linewidth}}{\cellcolor[rgb]{0.95,0.95,0.95}{%
	\makecell[{{p{\linewidth}}}]{% 
	  \tt {\tiny [GM$|$GM]}  
	 VERSCHIEBE: S, (3, 6) 
	  } 
	   } 
	   } 
	 & & \\ 
 

    \theutterance \stepcounter{utterance}  

    & & & \multicolumn{2}{p{0.3\linewidth}}{\cellcolor[rgb]{0.95,0.95,0.95}{%
	\makecell[{{p{\linewidth}}}]{% 
	  \tt {\tiny [GM$|$GM]}  
	 'S' erfolgreich nach (3,6) verschoben. Dein Raster sieht jetzt so aus:\\ \tt ```\\ \tt  1234567\\ \tt ╔═╤═╤═╤═╗\\ \tt ║A│O│◌│◌║ 1\\ \tt ╟─┼─┤◌│◌║ 2\\ \tt ║T│◌│◌│◌║ 3\\ \tt ║P│◌├─┴─╢ 4\\ \tt ╟─┴─┤◌◌◌║ 5\\ \tt ║D◌S└─┬─╢ 6\\ \tt ║◌◌◌◌◌│M║ 7\\ \tt ╚═════╧═╝\\ \tt ```\\ \tt  
	  } 
	   } 
	   } 
	 & & \\ 
 

    \theutterance \stepcounter{utterance}  

    & & \multicolumn{4}{p{0.6\linewidth}}{\cellcolor[rgb]{0.9,0.9,0.9}{%
	\makecell[{{p{\linewidth}}}]{% 
	  \tt {\tiny [GM$\rangle$B]}  
	 'T' erfolgreich nach (1,3) verschoben. Dein Raster sieht jetzt so aus:\\ \tt ```\\ \tt  1234567\\ \tt ╔═╤═╤═╤═╗\\ \tt ║A│O│◌│M║ 1\\ \tt ╟─┼─┤◌│◌║ 2\\ \tt ║T│D│◌│◌║ 3\\ \tt ║◌│◌├─┴─╢ 4\\ \tt ╟─┴─┤◌◌◌║ 5\\ \tt ║PS◌└─┬─╢ 6\\ \tt ║◌◌◌◌◌│◌║ 7\\ \tt ╚═════╧═╝\\ \tt ```\\ \tt Ihr seid gerade bei Runde 11 von maximal 28.\\ \tt Ihr habt gemeinsam 9 von maximal 16 Strafpunkten angesammelt.\\ \tt Dein Mitspieler hat ein Objekt auf seinem Raster bewegt.\\ \tt \\ \tt \\ \tt Wie lautet dein nächster Befehl? 
	  } 
	   } 
	   } 
	 & \\ 
 

    \theutterance \stepcounter{utterance}  

    & & & \multicolumn{4}{p{0.6\linewidth}}{\cellcolor[rgb]{1,0.85,0.72}{%
	\makecell[{{p{\linewidth}}}]{% 
	  \tt {\tiny [GM$\langle$B]}  
	 VERSCHIEBE: P, (1, 4) 
	  } 
	   } 
	   } 
	  \\ 
 

    \theutterance \stepcounter{utterance}  

    & & & \multicolumn{2}{p{0.3\linewidth}}{\cellcolor[rgb]{0.95,0.95,0.95}{%
	\makecell[{{p{\linewidth}}}]{% 
	  \tt {\tiny [GM$|$GM]}  
	 VERSCHIEBE: P, (1, 4) 
	  } 
	   } 
	   } 
	 & & \\ 
 

    \theutterance \stepcounter{utterance}  

    & & & \multicolumn{2}{p{0.3\linewidth}}{\cellcolor[rgb]{0.95,0.95,0.95}{%
	\makecell[{{p{\linewidth}}}]{% 
	  \tt {\tiny [GM$|$GM]}  
	 'P' erfolgreich nach (1,4) verschoben. Dein Raster sieht jetzt so aus:\\ \tt ```\\ \tt  1234567\\ \tt ╔═╤═╤═╤═╗\\ \tt ║A│O│◌│M║ 1\\ \tt ╟─┼─┤◌│◌║ 2\\ \tt ║T│D│◌│◌║ 3\\ \tt ║P│◌├─┴─╢ 4\\ \tt ╟─┴─┤◌◌◌║ 5\\ \tt ║◌S◌└─┬─╢ 6\\ \tt ║◌◌◌◌◌│◌║ 7\\ \tt ╚═════╧═╝\\ \tt ```\\ \tt  
	  } 
	   } 
	   } 
	 & & \\ 
 

    \theutterance \stepcounter{utterance}  

    & & \multicolumn{4}{p{0.6\linewidth}}{\cellcolor[rgb]{0.9,0.9,0.9}{%
	\makecell[{{p{\linewidth}}}]{% 
	  \tt {\tiny [A$\langle$GM]}  
	 'S' erfolgreich nach (3,6) verschoben. Dein Raster sieht jetzt so aus:\\ \tt ```\\ \tt  1234567\\ \tt ╔═╤═╤═╤═╗\\ \tt ║A│O│◌│◌║ 1\\ \tt ╟─┼─┤◌│◌║ 2\\ \tt ║T│◌│◌│◌║ 3\\ \tt ║P│◌├─┴─╢ 4\\ \tt ╟─┴─┤◌◌◌║ 5\\ \tt ║D◌S└─┬─╢ 6\\ \tt ║◌◌◌◌◌│M║ 7\\ \tt ╚═════╧═╝\\ \tt ```\\ \tt Ihr seid gerade bei Runde 11 von maximal 28.\\ \tt Ihr habt gemeinsam 9 von maximal 16 Strafpunkten angesammelt.\\ \tt Dein Mitspieler hat ein Objekt auf seinem Raster bewegt.\\ \tt \\ \tt \\ \tt Wie lautet dein nächster Befehl? 
	  } 
	   } 
	   } 
	 & \\ 
 

    \theutterance \stepcounter{utterance}  

    & \multicolumn{4}{p{0.6\linewidth}}{\cellcolor[rgb]{0.8,1,0.9}{%
	\makecell[{{p{\linewidth}}}]{% 
	  \tt {\tiny [A$\rangle$GM]}  
	 VERSCHIEBE: M, (5, 7) 
	  } 
	   } 
	   } 
	 & & \\ 
 

    \theutterance \stepcounter{utterance}  

    & & & \multicolumn{2}{p{0.3\linewidth}}{\cellcolor[rgb]{0.95,0.95,0.95}{%
	\makecell[{{p{\linewidth}}}]{% 
	  \tt {\tiny [GM$|$GM]}  
	 VERSCHIEBE: M, (5, 7) 
	  } 
	   } 
	   } 
	 & & \\ 
 

    \theutterance \stepcounter{utterance}  

    & & & \multicolumn{2}{p{0.3\linewidth}}{\cellcolor[rgb]{0.95,0.95,0.95}{%
	\makecell[{{p{\linewidth}}}]{% 
	  \tt {\tiny [GM$|$GM]}  
	 'M' erfolgreich nach (5,7) verschoben. Dein Raster sieht jetzt so aus:\\ \tt ```\\ \tt  1234567\\ \tt ╔═╤═╤═╤═╗\\ \tt ║A│O│◌│◌║ 1\\ \tt ╟─┼─┤◌│◌║ 2\\ \tt ║T│◌│◌│◌║ 3\\ \tt ║P│◌├─┴─╢ 4\\ \tt ╟─┴─┤◌◌◌║ 5\\ \tt ║D◌S└─┬─╢ 6\\ \tt ║◌◌◌◌M│◌║ 7\\ \tt ╚═════╧═╝\\ \tt ```\\ \tt  
	  } 
	   } 
	   } 
	 & & \\ 
 

    \theutterance \stepcounter{utterance}  

    & & \multicolumn{4}{p{0.6\linewidth}}{\cellcolor[rgb]{0.9,0.9,0.9}{%
	\makecell[{{p{\linewidth}}}]{% 
	  \tt {\tiny [GM$\rangle$B]}  
	 'P' erfolgreich nach (1,4) verschoben. Dein Raster sieht jetzt so aus:\\ \tt ```\\ \tt  1234567\\ \tt ╔═╤═╤═╤═╗\\ \tt ║A│O│◌│M║ 1\\ \tt ╟─┼─┤◌│◌║ 2\\ \tt ║T│D│◌│◌║ 3\\ \tt ║P│◌├─┴─╢ 4\\ \tt ╟─┴─┤◌◌◌║ 5\\ \tt ║◌S◌└─┬─╢ 6\\ \tt ║◌◌◌◌◌│◌║ 7\\ \tt ╚═════╧═╝\\ \tt ```\\ \tt Ihr seid gerade bei Runde 12 von maximal 28.\\ \tt Ihr habt gemeinsam 9 von maximal 16 Strafpunkten angesammelt.\\ \tt Dein Mitspieler hat ein Objekt auf seinem Raster bewegt.\\ \tt \\ \tt \\ \tt Wie lautet dein nächster Befehl? 
	  } 
	   } 
	   } 
	 & \\ 
 

    \theutterance \stepcounter{utterance}  

    & & & \multicolumn{4}{p{0.6\linewidth}}{\cellcolor[rgb]{1,0.85,0.72}{%
	\makecell[{{p{\linewidth}}}]{% 
	  \tt {\tiny [GM$\langle$B]}  
	 VERSCHIEBE: D, (2, 4) 
	  } 
	   } 
	   } 
	  \\ 
 

    \theutterance \stepcounter{utterance}  

    & & & \multicolumn{2}{p{0.3\linewidth}}{\cellcolor[rgb]{0.95,0.95,0.95}{%
	\makecell[{{p{\linewidth}}}]{% 
	  \tt {\tiny [GM$|$GM]}  
	 VERSCHIEBE: D, (2, 4) 
	  } 
	   } 
	   } 
	 & & \\ 
 

    \theutterance \stepcounter{utterance}  

    & & & \multicolumn{2}{p{0.3\linewidth}}{\cellcolor[rgb]{0.95,0.95,0.95}{%
	\makecell[{{p{\linewidth}}}]{% 
	  \tt {\tiny [GM$|$GM]}  
	 Strafpunkt: (2,4) ist nicht leer, sondern enthält '│'.\\ \tt \\ \tt Ihr habt gemeinsam 10 von maximal 16 Strafpunkten angesammelt.\\ \tt Bitte versuche es noch mal! 
	  } 
	   } 
	   } 
	 & & \\ 
 

    \theutterance \stepcounter{utterance}  

    & & \multicolumn{4}{p{0.6\linewidth}}{\cellcolor[rgb]{0.9,0.9,0.9}{%
	\makecell[{{p{\linewidth}}}]{% 
	  \tt {\tiny [GM$\rangle$B]}  
	 Strafpunkt: (2,4) ist nicht leer, sondern enthält '│'.\\ \tt \\ \tt Ihr habt gemeinsam 10 von maximal 16 Strafpunkten angesammelt.\\ \tt Bitte versuche es noch mal! 
	  } 
	   } 
	   } 
	 & \\ 
 

    \theutterance \stepcounter{utterance}  

    & & & \multicolumn{4}{p{0.6\linewidth}}{\cellcolor[rgb]{1,0.85,0.72}{%
	\makecell[{{p{\linewidth}}}]{% 
	  \tt {\tiny [GM$\langle$B]}  
	 SPRICH: (2,4) ist bei mir nicht frei - dort ist eine Linie. Wir brauchen eine andere Position für D. Welche freien Positionen haben wir beide noch? Ich kann D nach (2,3), (4,5), (5,5), (6,5), (1,7), (2,7), (3,7) oder (5,7) bewegen. 
	  } 
	   } 
	   } 
	  \\ 
 

    \theutterance \stepcounter{utterance}  

    & & & \multicolumn{2}{p{0.3\linewidth}}{\cellcolor[rgb]{0.95,0.95,0.95}{%
	\makecell[{{p{\linewidth}}}]{% 
	  \tt {\tiny [GM$|$GM]}  
	 SPRICH: (2,4) ist bei mir nicht frei - dort ist eine Linie. Wir brauchen eine andere Position für D. Welche freien Positionen haben wir beide noch? Ich kann D nach (2,3), (4,5), (5,5), (6,5), (1,7), (2,7), (3,7) oder (5,7) bewegen. 
	  } 
	   } 
	   } 
	 & & \\ 
 

    \theutterance \stepcounter{utterance}  

    & & \multicolumn{4}{p{0.6\linewidth}}{\cellcolor[rgb]{0.9,0.9,0.9}{%
	\makecell[{{p{\linewidth}}}]{% 
	  \tt {\tiny [A$\langle$GM]}  
	 'M' erfolgreich nach (5,7) verschoben. Dein Raster sieht jetzt so aus:\\ \tt ```\\ \tt  1234567\\ \tt ╔═╤═╤═╤═╗\\ \tt ║A│O│◌│◌║ 1\\ \tt ╟─┼─┤◌│◌║ 2\\ \tt ║T│◌│◌│◌║ 3\\ \tt ║P│◌├─┴─╢ 4\\ \tt ╟─┴─┤◌◌◌║ 5\\ \tt ║D◌S└─┬─╢ 6\\ \tt ║◌◌◌◌M│◌║ 7\\ \tt ╚═════╧═╝\\ \tt ```\\ \tt Ihr seid gerade bei Runde 12 von maximal 28.\\ \tt Ihr habt gemeinsam 10 von maximal 16 Strafpunkten angesammelt.\\ \tt Dein Mitspieler hat dir diese Nachricht geschickt:\\ \tt \\ \tt \\ \tt "(2,4) ist bei mir nicht frei - dort ist eine Linie. Wir brauchen eine andere Position für D. Welche freien Positionen haben wir beide noch? Ich kann D nach (2,3), (4,5), (5,5), (6,5), (1,7), (2,7), (3,7) oder (5,7) bewegen."\\ \tt \\ \tt \\ \tt Wie lautet dein nächster Befehl? 
	  } 
	   } 
	   } 
	 & \\ 
 

    \theutterance \stepcounter{utterance}  

    & \multicolumn{4}{p{0.6\linewidth}}{\cellcolor[rgb]{0.8,1,0.9}{%
	\makecell[{{p{\linewidth}}}]{% 
	  \tt {\tiny [A$\rangle$GM]}  
	 SPRICH: Von deinen Vorschlägen sind bei mir frei: (4,5), (5,5), (6,5), (1,7), (2,7), (3,7). Position (5,7) ist jetzt von M belegt. Lass uns D nach (4,5) setzen. Einverstanden? 
	  } 
	   } 
	   } 
	 & & \\ 
 

    \theutterance \stepcounter{utterance}  

    & & & \multicolumn{2}{p{0.3\linewidth}}{\cellcolor[rgb]{0.95,0.95,0.95}{%
	\makecell[{{p{\linewidth}}}]{% 
	  \tt {\tiny [GM$|$GM]}  
	 SPRICH: Von deinen Vorschlägen sind bei mir frei: (4,5), (5,5), (6,5), (1,7), (2,7), (3,7). Position (5,7) ist jetzt von M belegt. Lass uns D nach (4,5) setzen. Einverstanden? 
	  } 
	   } 
	   } 
	 & & \\ 
 

    \theutterance \stepcounter{utterance}  

    & & \multicolumn{4}{p{0.6\linewidth}}{\cellcolor[rgb]{0.9,0.9,0.9}{%
	\makecell[{{p{\linewidth}}}]{% 
	  \tt {\tiny [GM$\rangle$B]}  
	 Deine Nachricht wurde an deinen Mitspieler übermittelt.\\ \tt \\ \tt \\ \tt Ihr seid gerade bei Runde 13 von maximal 28.\\ \tt Ihr habt gemeinsam 10 von maximal 16 Strafpunkten angesammelt.\\ \tt Dein Mitspieler hat dir diese Nachricht geschickt:\\ \tt \\ \tt \\ \tt "Von deinen Vorschlägen sind bei mir frei: (4,5), (5,5), (6,5), (1,7), (2,7), (3,7). Position (5,7) ist jetzt von M belegt. Lass uns D nach (4,5) setzen. Einverstanden?"\\ \tt \\ \tt \\ \tt Wie lautet dein nächster Befehl? 
	  } 
	   } 
	   } 
	 & \\ 
 

    \theutterance \stepcounter{utterance}  

    & & & \multicolumn{4}{p{0.6\linewidth}}{\cellcolor[rgb]{1,0.85,0.72}{%
	\makecell[{{p{\linewidth}}}]{% 
	  \tt {\tiny [GM$\langle$B]}  
	 SPRICH: Einverstanden! D nach (4,5). Dann haben wir: A(1,1), O(3,1), T(1,3), P(1,4), D(4,5), S(3,6), M(5,7). 
	  } 
	   } 
	   } 
	  \\ 
 

    \theutterance \stepcounter{utterance}  

    & & & \multicolumn{2}{p{0.3\linewidth}}{\cellcolor[rgb]{0.95,0.95,0.95}{%
	\makecell[{{p{\linewidth}}}]{% 
	  \tt {\tiny [GM$|$GM]}  
	 SPRICH: Einverstanden! D nach (4,5). Dann haben wir: A(1,1), O(3,1), T(1,3), P(1,4), D(4,5), S(3,6), M(5,7). 
	  } 
	   } 
	   } 
	 & & \\ 
 

    \theutterance \stepcounter{utterance}  

    & & \multicolumn{4}{p{0.6\linewidth}}{\cellcolor[rgb]{0.9,0.9,0.9}{%
	\makecell[{{p{\linewidth}}}]{% 
	  \tt {\tiny [A$\langle$GM]}  
	 Deine Nachricht wurde an deinen Mitspieler übermittelt.\\ \tt \\ \tt \\ \tt Ihr seid gerade bei Runde 13 von maximal 28.\\ \tt Ihr habt gemeinsam 10 von maximal 16 Strafpunkten angesammelt.\\ \tt Dein Mitspieler hat dir diese Nachricht geschickt:\\ \tt \\ \tt \\ \tt "Einverstanden! D nach (4,5). Dann haben wir: A(1,1), O(3,1), T(1,3), P(1,4), D(4,5), S(3,6), M(5,7)."\\ \tt \\ \tt \\ \tt Wie lautet dein nächster Befehl? 
	  } 
	   } 
	   } 
	 & \\ 
 

    \theutterance \stepcounter{utterance}  

    & \multicolumn{4}{p{0.6\linewidth}}{\cellcolor[rgb]{0.8,1,0.9}{%
	\makecell[{{p{\linewidth}}}]{% 
	  \tt {\tiny [A$\rangle$GM]}  
	 VERSCHIEBE: D, (4, 5) 
	  } 
	   } 
	   } 
	 & & \\ 
 

    \theutterance \stepcounter{utterance}  

    & & & \multicolumn{2}{p{0.3\linewidth}}{\cellcolor[rgb]{0.95,0.95,0.95}{%
	\makecell[{{p{\linewidth}}}]{% 
	  \tt {\tiny [GM$|$GM]}  
	 VERSCHIEBE: D, (4, 5) 
	  } 
	   } 
	   } 
	 & & \\ 
 

    \theutterance \stepcounter{utterance}  

    & & & \multicolumn{2}{p{0.3\linewidth}}{\cellcolor[rgb]{0.95,0.95,0.95}{%
	\makecell[{{p{\linewidth}}}]{% 
	  \tt {\tiny [GM$|$GM]}  
	 Strafpunkt: (4,5) ist nicht leer, sondern enthält '┤'.\\ \tt \\ \tt Ihr habt gemeinsam 11 von maximal 16 Strafpunkten angesammelt.\\ \tt Bitte versuche es noch mal! 
	  } 
	   } 
	   } 
	 & & \\ 
 

    \theutterance \stepcounter{utterance}  

    & & \multicolumn{4}{p{0.6\linewidth}}{\cellcolor[rgb]{0.9,0.9,0.9}{%
	\makecell[{{p{\linewidth}}}]{% 
	  \tt {\tiny [A$\langle$GM]}  
	 Strafpunkt: (4,5) ist nicht leer, sondern enthält '┤'.\\ \tt \\ \tt Ihr habt gemeinsam 11 von maximal 16 Strafpunkten angesammelt.\\ \tt Bitte versuche es noch mal! 
	  } 
	   } 
	   } 
	 & \\ 
 

    \theutterance \stepcounter{utterance}  

    & \multicolumn{4}{p{0.6\linewidth}}{\cellcolor[rgb]{0.8,1,0.9}{%
	\makecell[{{p{\linewidth}}}]{% 
	  \tt {\tiny [A$\rangle$GM]}  
	 VERSCHIEBE: D, (5, 5) 
	  } 
	   } 
	   } 
	 & & \\ 
 

    \theutterance \stepcounter{utterance}  

    & & & \multicolumn{2}{p{0.3\linewidth}}{\cellcolor[rgb]{0.95,0.95,0.95}{%
	\makecell[{{p{\linewidth}}}]{% 
	  \tt {\tiny [GM$|$GM]}  
	 VERSCHIEBE: D, (5, 5) 
	  } 
	   } 
	   } 
	 & & \\ 
 

    \theutterance \stepcounter{utterance}  

    & & & \multicolumn{2}{p{0.3\linewidth}}{\cellcolor[rgb]{0.95,0.95,0.95}{%
	\makecell[{{p{\linewidth}}}]{% 
	  \tt {\tiny [GM$|$GM]}  
	 'D' erfolgreich nach (5,5) verschoben. Dein Raster sieht jetzt so aus:\\ \tt ```\\ \tt  1234567\\ \tt ╔═╤═╤═╤═╗\\ \tt ║A│O│◌│◌║ 1\\ \tt ╟─┼─┤◌│◌║ 2\\ \tt ║T│◌│◌│◌║ 3\\ \tt ║P│◌├─┴─╢ 4\\ \tt ╟─┴─┤D◌◌║ 5\\ \tt ║◌◌S└─┬─╢ 6\\ \tt ║◌◌◌◌M│◌║ 7\\ \tt ╚═════╧═╝\\ \tt ```\\ \tt  
	  } 
	   } 
	   } 
	 & & \\ 
 

    \theutterance \stepcounter{utterance}  

    & & \multicolumn{4}{p{0.6\linewidth}}{\cellcolor[rgb]{0.9,0.9,0.9}{%
	\makecell[{{p{\linewidth}}}]{% 
	  \tt {\tiny [GM$\rangle$B]}  
	 Deine Nachricht wurde an deinen Mitspieler übermittelt.\\ \tt \\ \tt \\ \tt Ihr seid gerade bei Runde 14 von maximal 28.\\ \tt Ihr habt gemeinsam 11 von maximal 16 Strafpunkten angesammelt.\\ \tt Dein Mitspieler hat ein Objekt auf seinem Raster bewegt.\\ \tt \\ \tt \\ \tt Wie lautet dein nächster Befehl? 
	  } 
	   } 
	   } 
	 & \\ 
 

    \theutterance \stepcounter{utterance}  

    & & & \multicolumn{4}{p{0.6\linewidth}}{\cellcolor[rgb]{1,0.85,0.72}{%
	\makecell[{{p{\linewidth}}}]{% 
	  \tt {\tiny [GM$\langle$B]}  
	 VERSCHIEBE: D, (4, 5) 
	  } 
	   } 
	   } 
	  \\ 
 

    \theutterance \stepcounter{utterance}  

    & & & \multicolumn{2}{p{0.3\linewidth}}{\cellcolor[rgb]{0.95,0.95,0.95}{%
	\makecell[{{p{\linewidth}}}]{% 
	  \tt {\tiny [GM$|$GM]}  
	 VERSCHIEBE: D, (4, 5) 
	  } 
	   } 
	   } 
	 & & \\ 
 

    \theutterance \stepcounter{utterance}  

    & & & \multicolumn{2}{p{0.3\linewidth}}{\cellcolor[rgb]{0.95,0.95,0.95}{%
	\makecell[{{p{\linewidth}}}]{% 
	  \tt {\tiny [GM$|$GM]}  
	 Strafpunkt: (4,5) ist nicht leer, sondern enthält '┤'.\\ \tt \\ \tt Ihr habt gemeinsam 12 von maximal 16 Strafpunkten angesammelt.\\ \tt Bitte versuche es noch mal! 
	  } 
	   } 
	   } 
	 & & \\ 
 

    \theutterance \stepcounter{utterance}  

    & & \multicolumn{4}{p{0.6\linewidth}}{\cellcolor[rgb]{0.9,0.9,0.9}{%
	\makecell[{{p{\linewidth}}}]{% 
	  \tt {\tiny [GM$\rangle$B]}  
	 Strafpunkt: (4,5) ist nicht leer, sondern enthält '┤'.\\ \tt \\ \tt Ihr habt gemeinsam 12 von maximal 16 Strafpunkten angesammelt.\\ \tt Bitte versuche es noch mal! 
	  } 
	   } 
	   } 
	 & \\ 
 

    \theutterance \stepcounter{utterance}  

    & & & \multicolumn{4}{p{0.6\linewidth}}{\cellcolor[rgb]{1,0.85,0.72}{%
	\makecell[{{p{\linewidth}}}]{% 
	  \tt {\tiny [GM$\langle$B]}  
	 SPRICH: (4,5) ist bei mir auch nicht frei - dort ist eine Linie! Von meinen ursprünglich genannten freien Positionen sind noch verfügbar: (5,5), (6,5), (1,7), (2,7), (3,7). Welche davon ist bei dir auch frei? 
	  } 
	   } 
	   } 
	  \\ 
 

    \theutterance \stepcounter{utterance}  

    & & & \multicolumn{2}{p{0.3\linewidth}}{\cellcolor[rgb]{0.95,0.95,0.95}{%
	\makecell[{{p{\linewidth}}}]{% 
	  \tt {\tiny [GM$|$GM]}  
	 SPRICH: (4,5) ist bei mir auch nicht frei - dort ist eine Linie! Von meinen ursprünglich genannten freien Positionen sind noch verfügbar: (5,5), (6,5), (1,7), (2,7), (3,7). Welche davon ist bei dir auch frei? 
	  } 
	   } 
	   } 
	 & & \\ 
 

    \theutterance \stepcounter{utterance}  

    & & \multicolumn{4}{p{0.6\linewidth}}{\cellcolor[rgb]{0.9,0.9,0.9}{%
	\makecell[{{p{\linewidth}}}]{% 
	  \tt {\tiny [A$\langle$GM]}  
	 'D' erfolgreich nach (5,5) verschoben. Dein Raster sieht jetzt so aus:\\ \tt ```\\ \tt  1234567\\ \tt ╔═╤═╤═╤═╗\\ \tt ║A│O│◌│◌║ 1\\ \tt ╟─┼─┤◌│◌║ 2\\ \tt ║T│◌│◌│◌║ 3\\ \tt ║P│◌├─┴─╢ 4\\ \tt ╟─┴─┤D◌◌║ 5\\ \tt ║◌◌S└─┬─╢ 6\\ \tt ║◌◌◌◌M│◌║ 7\\ \tt ╚═════╧═╝\\ \tt ```\\ \tt Ihr seid gerade bei Runde 14 von maximal 28.\\ \tt Ihr habt gemeinsam 12 von maximal 16 Strafpunkten angesammelt.\\ \tt Dein Mitspieler hat dir diese Nachricht geschickt:\\ \tt \\ \tt \\ \tt "(4,5) ist bei mir auch nicht frei - dort ist eine Linie! Von meinen ursprünglich genannten freien Positionen sind noch verfügbar: (5,5), (6,5), (1,7), (2,7), (3,7). Welche davon ist bei dir auch frei?"\\ \tt \\ \tt \\ \tt Wie lautet dein nächster Befehl? 
	  } 
	   } 
	   } 
	 & \\ 
 

    \theutterance \stepcounter{utterance}  

    & \multicolumn{4}{p{0.6\linewidth}}{\cellcolor[rgb]{0.8,1,0.9}{%
	\makecell[{{p{\linewidth}}}]{% 
	  \tt {\tiny [A$\rangle$GM]}  
	 SPRICH: Gut, D ist jetzt bei (5,5)! Von deinen anderen Optionen sind bei mir auch frei: (6,5), (1,7), (2,7), (3,7). Wir haben jetzt: A(1,1), O(3,1), T(1,3), P(1,4), D(5,5), S(3,6), M(5,7). Sind alle deine Objekte auch an den richtigen Positionen? 
	  } 
	   } 
	   } 
	 & & \\ 
 

    \theutterance \stepcounter{utterance}  

    & & & \multicolumn{2}{p{0.3\linewidth}}{\cellcolor[rgb]{0.95,0.95,0.95}{%
	\makecell[{{p{\linewidth}}}]{% 
	  \tt {\tiny [GM$|$GM]}  
	 SPRICH: Gut, D ist jetzt bei (5,5)! Von deinen anderen Optionen sind bei mir auch frei: (6,5), (1,7), (2,7), (3,7). Wir haben jetzt: A(1,1), O(3,1), T(1,3), P(1,4), D(5,5), S(3,6), M(5,7). Sind alle deine Objekte auch an den richtigen Positionen? 
	  } 
	   } 
	   } 
	 & & \\ 
 

    \theutterance \stepcounter{utterance}  

    & & \multicolumn{4}{p{0.6\linewidth}}{\cellcolor[rgb]{0.9,0.9,0.9}{%
	\makecell[{{p{\linewidth}}}]{% 
	  \tt {\tiny [GM$\rangle$B]}  
	 Deine Nachricht wurde an deinen Mitspieler übermittelt.\\ \tt \\ \tt \\ \tt Ihr seid gerade bei Runde 15 von maximal 28.\\ \tt Ihr habt gemeinsam 12 von maximal 16 Strafpunkten angesammelt.\\ \tt Dein Mitspieler hat dir diese Nachricht geschickt:\\ \tt \\ \tt \\ \tt "Gut, D ist jetzt bei (5,5)! Von deinen anderen Optionen sind bei mir auch frei: (6,5), (1,7), (2,7), (3,7). Wir haben jetzt: A(1,1), O(3,1), T(1,3), P(1,4), D(5,5), S(3,6), M(5,7). Sind alle deine Objekte auch an den richtigen Positionen?"\\ \tt \\ \tt \\ \tt Wie lautet dein nächster Befehl? 
	  } 
	   } 
	   } 
	 & \\ 
 

    \theutterance \stepcounter{utterance}  

    & & & \multicolumn{4}{p{0.6\linewidth}}{\cellcolor[rgb]{1,0.85,0.72}{%
	\makecell[{{p{\linewidth}}}]{% 
	  \tt {\tiny [GM$\langle$B]}  
	 VERSCHIEBE: D, (5, 5) 
	  } 
	   } 
	   } 
	  \\ 
 

    \theutterance \stepcounter{utterance}  

    & & & \multicolumn{2}{p{0.3\linewidth}}{\cellcolor[rgb]{0.95,0.95,0.95}{%
	\makecell[{{p{\linewidth}}}]{% 
	  \tt {\tiny [GM$|$GM]}  
	 VERSCHIEBE: D, (5, 5) 
	  } 
	   } 
	   } 
	 & & \\ 
 

    \theutterance \stepcounter{utterance}  

    & & & \multicolumn{2}{p{0.3\linewidth}}{\cellcolor[rgb]{0.95,0.95,0.95}{%
	\makecell[{{p{\linewidth}}}]{% 
	  \tt {\tiny [GM$|$GM]}  
	 'D' erfolgreich nach (5,5) verschoben. Dein Raster sieht jetzt so aus:\\ \tt ```\\ \tt  1234567\\ \tt ╔═╤═╤═╤═╗\\ \tt ║A│O│◌│M║ 1\\ \tt ╟─┼─┤◌│◌║ 2\\ \tt ║T│◌│◌│◌║ 3\\ \tt ║P│◌├─┴─╢ 4\\ \tt ╟─┴─┤D◌◌║ 5\\ \tt ║◌S◌└─┬─╢ 6\\ \tt ║◌◌◌◌◌│◌║ 7\\ \tt ╚═════╧═╝\\ \tt ```\\ \tt  
	  } 
	   } 
	   } 
	 & & \\ 
 

    \theutterance \stepcounter{utterance}  

    & & \multicolumn{4}{p{0.6\linewidth}}{\cellcolor[rgb]{0.9,0.9,0.9}{%
	\makecell[{{p{\linewidth}}}]{% 
	  \tt {\tiny [A$\langle$GM]}  
	 Deine Nachricht wurde an deinen Mitspieler übermittelt.\\ \tt \\ \tt \\ \tt Ihr seid gerade bei Runde 15 von maximal 28.\\ \tt Ihr habt gemeinsam 12 von maximal 16 Strafpunkten angesammelt.\\ \tt Dein Mitspieler hat ein Objekt auf seinem Raster bewegt.\\ \tt \\ \tt \\ \tt Wie lautet dein nächster Befehl? 
	  } 
	   } 
	   } 
	 & \\ 
 

    \theutterance \stepcounter{utterance}  

    & \multicolumn{4}{p{0.6\linewidth}}{\cellcolor[rgb]{0.8,1,0.9}{%
	\makecell[{{p{\linewidth}}}]{% 
	  \tt {\tiny [A$\rangle$GM]}  
	 SPRICH: fertig? 
	  } 
	   } 
	   } 
	 & & \\ 
 

    \theutterance \stepcounter{utterance}  

    & & & \multicolumn{2}{p{0.3\linewidth}}{\cellcolor[rgb]{0.95,0.95,0.95}{%
	\makecell[{{p{\linewidth}}}]{% 
	  \tt {\tiny [GM$|$GM]}  
	 SPRICH: fertig? 
	  } 
	   } 
	   } 
	 & & \\ 
 

    \theutterance \stepcounter{utterance}  

    & & \multicolumn{4}{p{0.6\linewidth}}{\cellcolor[rgb]{0.9,0.9,0.9}{%
	\makecell[{{p{\linewidth}}}]{% 
	  \tt {\tiny [GM$\rangle$B]}  
	 'D' erfolgreich nach (5,5) verschoben. Dein Raster sieht jetzt so aus:\\ \tt ```\\ \tt  1234567\\ \tt ╔═╤═╤═╤═╗\\ \tt ║A│O│◌│M║ 1\\ \tt ╟─┼─┤◌│◌║ 2\\ \tt ║T│◌│◌│◌║ 3\\ \tt ║P│◌├─┴─╢ 4\\ \tt ╟─┴─┤D◌◌║ 5\\ \tt ║◌S◌└─┬─╢ 6\\ \tt ║◌◌◌◌◌│◌║ 7\\ \tt ╚═════╧═╝\\ \tt ```\\ \tt Ihr seid gerade bei Runde 16 von maximal 28.\\ \tt Ihr habt gemeinsam 12 von maximal 16 Strafpunkten angesammelt.\\ \tt Dein Mitspieler hat dir diese Nachricht geschickt:\\ \tt \\ \tt \\ \tt "fertig?"\\ \tt \\ \tt \\ \tt Wie lautet dein nächster Befehl? 
	  } 
	   } 
	   } 
	 & \\ 
 

    \theutterance \stepcounter{utterance}  

    & & & \multicolumn{4}{p{0.6\linewidth}}{\cellcolor[rgb]{1,0.85,0.72}{%
	\makecell[{{p{\linewidth}}}]{% 
	  \tt {\tiny [GM$\langle$B]}  
	 VERSCHIEBE: S, (3, 6) 
	  } 
	   } 
	   } 
	  \\ 
 

    \theutterance \stepcounter{utterance}  

    & & & \multicolumn{2}{p{0.3\linewidth}}{\cellcolor[rgb]{0.95,0.95,0.95}{%
	\makecell[{{p{\linewidth}}}]{% 
	  \tt {\tiny [GM$|$GM]}  
	 VERSCHIEBE: S, (3, 6) 
	  } 
	   } 
	   } 
	 & & \\ 
 

    \theutterance \stepcounter{utterance}  

    & & & \multicolumn{2}{p{0.3\linewidth}}{\cellcolor[rgb]{0.95,0.95,0.95}{%
	\makecell[{{p{\linewidth}}}]{% 
	  \tt {\tiny [GM$|$GM]}  
	 'S' erfolgreich nach (3,6) verschoben. Dein Raster sieht jetzt so aus:\\ \tt ```\\ \tt  1234567\\ \tt ╔═╤═╤═╤═╗\\ \tt ║A│O│◌│M║ 1\\ \tt ╟─┼─┤◌│◌║ 2\\ \tt ║T│◌│◌│◌║ 3\\ \tt ║P│◌├─┴─╢ 4\\ \tt ╟─┴─┤D◌◌║ 5\\ \tt ║◌◌S└─┬─╢ 6\\ \tt ║◌◌◌◌◌│◌║ 7\\ \tt ╚═════╧═╝\\ \tt ```\\ \tt  
	  } 
	   } 
	   } 
	 & & \\ 
 

    \theutterance \stepcounter{utterance}  

    & & \multicolumn{4}{p{0.6\linewidth}}{\cellcolor[rgb]{0.9,0.9,0.9}{%
	\makecell[{{p{\linewidth}}}]{% 
	  \tt {\tiny [A$\langle$GM]}  
	 Deine Nachricht wurde an deinen Mitspieler übermittelt.\\ \tt \\ \tt \\ \tt Ihr seid gerade bei Runde 16 von maximal 28.\\ \tt Ihr habt gemeinsam 12 von maximal 16 Strafpunkten angesammelt.\\ \tt Dein Mitspieler hat ein Objekt auf seinem Raster bewegt.\\ \tt \\ \tt \\ \tt Wie lautet dein nächster Befehl? 
	  } 
	   } 
	   } 
	 & \\ 
 

    \theutterance \stepcounter{utterance}  

    & \multicolumn{4}{p{0.6\linewidth}}{\cellcolor[rgb]{0.8,1,0.9}{%
	\makecell[{{p{\linewidth}}}]{% 
	  \tt {\tiny [A$\rangle$GM]}  
	 SPRICH: Warten wir kurz - du bewegst noch Objekte. Lass mich wissen wenn du fertig bist, dann können wir prüfen ob alle Objekte an den vereinbarten Positionen stehen. 
	  } 
	   } 
	   } 
	 & & \\ 
 

    \theutterance \stepcounter{utterance}  

    & & & \multicolumn{2}{p{0.3\linewidth}}{\cellcolor[rgb]{0.95,0.95,0.95}{%
	\makecell[{{p{\linewidth}}}]{% 
	  \tt {\tiny [GM$|$GM]}  
	 SPRICH: Warten wir kurz - du bewegst noch Objekte. Lass mich wissen wenn du fertig bist, dann können wir prüfen ob alle Objekte an den vereinbarten Positionen stehen. 
	  } 
	   } 
	   } 
	 & & \\ 
 

    \theutterance \stepcounter{utterance}  

    & & \multicolumn{4}{p{0.6\linewidth}}{\cellcolor[rgb]{0.9,0.9,0.9}{%
	\makecell[{{p{\linewidth}}}]{% 
	  \tt {\tiny [GM$\rangle$B]}  
	 'S' erfolgreich nach (3,6) verschoben. Dein Raster sieht jetzt so aus:\\ \tt ```\\ \tt  1234567\\ \tt ╔═╤═╤═╤═╗\\ \tt ║A│O│◌│M║ 1\\ \tt ╟─┼─┤◌│◌║ 2\\ \tt ║T│◌│◌│◌║ 3\\ \tt ║P│◌├─┴─╢ 4\\ \tt ╟─┴─┤D◌◌║ 5\\ \tt ║◌◌S└─┬─╢ 6\\ \tt ║◌◌◌◌◌│◌║ 7\\ \tt ╚═════╧═╝\\ \tt ```\\ \tt Ihr seid gerade bei Runde 17 von maximal 28.\\ \tt Ihr habt gemeinsam 12 von maximal 16 Strafpunkten angesammelt.\\ \tt Dein Mitspieler hat dir diese Nachricht geschickt:\\ \tt \\ \tt \\ \tt "Warten wir kurz - du bewegst noch Objekte. Lass mich wissen wenn du fertig bist, dann können wir prüfen ob alle Objekte an den vereinbarten Positionen stehen."\\ \tt \\ \tt \\ \tt Wie lautet dein nächster Befehl? 
	  } 
	   } 
	   } 
	 & \\ 
 

    \theutterance \stepcounter{utterance}  

    & & & \multicolumn{4}{p{0.6\linewidth}}{\cellcolor[rgb]{1,0.85,0.72}{%
	\makecell[{{p{\linewidth}}}]{% 
	  \tt {\tiny [GM$\langle$B]}  
	 VERSCHIEBE: M, (5, 7) 
	  } 
	   } 
	   } 
	  \\ 
 

    \theutterance \stepcounter{utterance}  

    & & & \multicolumn{2}{p{0.3\linewidth}}{\cellcolor[rgb]{0.95,0.95,0.95}{%
	\makecell[{{p{\linewidth}}}]{% 
	  \tt {\tiny [GM$|$GM]}  
	 VERSCHIEBE: M, (5, 7) 
	  } 
	   } 
	   } 
	 & & \\ 
 

    \theutterance \stepcounter{utterance}  

    & & & \multicolumn{2}{p{0.3\linewidth}}{\cellcolor[rgb]{0.95,0.95,0.95}{%
	\makecell[{{p{\linewidth}}}]{% 
	  \tt {\tiny [GM$|$GM]}  
	 'M' erfolgreich nach (5,7) verschoben. Dein Raster sieht jetzt so aus:\\ \tt ```\\ \tt  1234567\\ \tt ╔═╤═╤═╤═╗\\ \tt ║A│O│◌│◌║ 1\\ \tt ╟─┼─┤◌│◌║ 2\\ \tt ║T│◌│◌│◌║ 3\\ \tt ║P│◌├─┴─╢ 4\\ \tt ╟─┴─┤D◌◌║ 5\\ \tt ║◌◌S└─┬─╢ 6\\ \tt ║◌◌◌◌M│◌║ 7\\ \tt ╚═════╧═╝\\ \tt ```\\ \tt  
	  } 
	   } 
	   } 
	 & & \\ 
 

    \theutterance \stepcounter{utterance}  

    & & \multicolumn{4}{p{0.6\linewidth}}{\cellcolor[rgb]{0.9,0.9,0.9}{%
	\makecell[{{p{\linewidth}}}]{% 
	  \tt {\tiny [A$\langle$GM]}  
	 Deine Nachricht wurde an deinen Mitspieler übermittelt.\\ \tt \\ \tt \\ \tt Ihr seid gerade bei Runde 17 von maximal 28.\\ \tt Ihr habt gemeinsam 12 von maximal 16 Strafpunkten angesammelt.\\ \tt Dein Mitspieler hat ein Objekt auf seinem Raster bewegt.\\ \tt \\ \tt \\ \tt Wie lautet dein nächster Befehl? 
	  } 
	   } 
	   } 
	 & \\ 
 

    \theutterance \stepcounter{utterance}  

    & \multicolumn{4}{p{0.6\linewidth}}{\cellcolor[rgb]{0.8,1,0.9}{%
	\makecell[{{p{\linewidth}}}]{% 
	  \tt {\tiny [A$\rangle$GM]}  
	 SPRICH: Ich warte auf deine Bestätigung. Meine Objekte stehen alle an den vereinbarten Positionen: A(1,1), O(3,1), T(1,3), P(1,4), D(5,5), S(3,6), M(5,7). Bist du auch fertig? 
	  } 
	   } 
	   } 
	 & & \\ 
 

    \theutterance \stepcounter{utterance}  

    & & & \multicolumn{2}{p{0.3\linewidth}}{\cellcolor[rgb]{0.95,0.95,0.95}{%
	\makecell[{{p{\linewidth}}}]{% 
	  \tt {\tiny [GM$|$GM]}  
	 SPRICH: Ich warte auf deine Bestätigung. Meine Objekte stehen alle an den vereinbarten Positionen: A(1,1), O(3,1), T(1,3), P(1,4), D(5,5), S(3,6), M(5,7). Bist du auch fertig? 
	  } 
	   } 
	   } 
	 & & \\ 
 

    \theutterance \stepcounter{utterance}  

    & & \multicolumn{4}{p{0.6\linewidth}}{\cellcolor[rgb]{0.9,0.9,0.9}{%
	\makecell[{{p{\linewidth}}}]{% 
	  \tt {\tiny [GM$\rangle$B]}  
	 'M' erfolgreich nach (5,7) verschoben. Dein Raster sieht jetzt so aus:\\ \tt ```\\ \tt  1234567\\ \tt ╔═╤═╤═╤═╗\\ \tt ║A│O│◌│◌║ 1\\ \tt ╟─┼─┤◌│◌║ 2\\ \tt ║T│◌│◌│◌║ 3\\ \tt ║P│◌├─┴─╢ 4\\ \tt ╟─┴─┤D◌◌║ 5\\ \tt ║◌◌S└─┬─╢ 6\\ \tt ║◌◌◌◌M│◌║ 7\\ \tt ╚═════╧═╝\\ \tt ```\\ \tt Ihr seid gerade bei Runde 18 von maximal 28.\\ \tt Ihr habt gemeinsam 12 von maximal 16 Strafpunkten angesammelt.\\ \tt Dein Mitspieler hat dir diese Nachricht geschickt:\\ \tt \\ \tt \\ \tt "Ich warte auf deine Bestätigung. Meine Objekte stehen alle an den vereinbarten Positionen: A(1,1), O(3,1), T(1,3), P(1,4), D(5,5), S(3,6), M(5,7). Bist du auch fertig?"\\ \tt \\ \tt \\ \tt Wie lautet dein nächster Befehl? 
	  } 
	   } 
	   } 
	 & \\ 
 

    \theutterance \stepcounter{utterance}  

    & & & \multicolumn{4}{p{0.6\linewidth}}{\cellcolor[rgb]{1,0.85,0.72}{%
	\makecell[{{p{\linewidth}}}]{% 
	  \tt {\tiny [GM$\langle$B]}  
	 SPRICH: fertig! 
	  } 
	   } 
	   } 
	  \\ 
 

    \theutterance \stepcounter{utterance}  

    & & & \multicolumn{2}{p{0.3\linewidth}}{\cellcolor[rgb]{0.95,0.95,0.95}{%
	\makecell[{{p{\linewidth}}}]{% 
	  \tt {\tiny [GM$|$GM]}  
	 SPRICH: fertig! 
	  } 
	   } 
	   } 
	 & & \\ 
 

    \theutterance \stepcounter{utterance}  

    & & & \multicolumn{2}{p{0.3\linewidth}}{\cellcolor[rgb]{0.95,0.95,0.95}{%
	\makecell[{{p{\linewidth}}}]{% 
	  \tt {\tiny [GM$|$GM]}  
	 true 
	  } 
	   } 
	   } 
	 & & \\ 
 

    \theutterance \stepcounter{utterance}  

    & & & \multicolumn{2}{p{0.3\linewidth}}{\cellcolor[rgb]{0.95,0.95,0.95}{%
	\makecell[{{p{\linewidth}}}]{% 
	  \tt {\tiny [GM$|$GM]}  
	 * success: True\\ \tt * lose: False\\ \tt * aborted: False\\ \tt -------\\ \tt * Shifts: 9.00\\ \tt * Max Shifts: 12.00\\ \tt * Min Shifts: 6.00\\ \tt * End Distance Sum: 0.00\\ \tt * Init Distance Sum: 23.67\\ \tt * Expected Distance Sum: 29.33\\ \tt * Penalties: 12.00\\ \tt * Max Penalties: 16.00\\ \tt * Rounds: 18.00\\ \tt * Max Rounds: 28.00\\ \tt * Object Count: 7.00\\ \tt  
	  } 
	   } 
	   } 
	 & & \\ 
 

    \theutterance \stepcounter{utterance}  

    & & & \multicolumn{2}{p{0.3\linewidth}}{\cellcolor[rgb]{0.95,0.95,0.95}{%
	\makecell[{{p{\linewidth}}}]{% 
	  \tt {\tiny [GM$|$GM]}  
	 * Main Score: 24.65\\ \tt \\ \tt -------\\ \tt * Distance Score: 1.00\\ \tt * Consistency Score: 0.57\\ \tt * Coverage Score: 0.77\\ \tt * Penalty Score: 0.33\\ \tt * Alternative Penalty Score: 0.50\\ \tt * Alternative Main Score: 50.00\\ \tt \\ \tt -------\\ \tt * Shifts: 9.00\\ \tt * Max Shifts: 12.00\\ \tt * Min Shifts: 6.00\\ \tt * End Distance Sum: 0.00\\ \tt * Init Distance Sum: 23.67\\ \tt * Expected Distance Sum: 29.33\\ \tt * Penalties: 12.00\\ \tt * Max Penalties: 16.00\\ \tt * Rounds: 18.00\\ \tt * Max Rounds: 28.00\\ \tt * Object Count: 7.00\\ \tt  
	  } 
	   } 
	   } 
	 & & \\ 
 

\end{supertabular}
}

\end{document}
