
\documentclass{article}
\usepackage{colortbl}
\usepackage{makecell}
\usepackage{multirow}
\usepackage{supertabular}

\begin{document}

\newcounter{utterance}

\centering \large Interaction Transcript for game `clean\_up', experiment `2\_hard\_7obj\_de', episode 1 with claude{-}sonnet{-}4{-}20250514{-}no{-}reasoning{-}t1.0.
\vspace{24pt}

{ \footnotesize  \setcounter{utterance}{1}
\setlength{\tabcolsep}{0pt}
\begin{supertabular}{c@{$\;$}|p{.15\linewidth}@{}p{.15\linewidth}p{.15\linewidth}p{.15\linewidth}p{.15\linewidth}p{.15\linewidth}}
    \# & $\;$A & \multicolumn{4}{c}{Game Master} & $\;\:$B\\
    \hline

    \theutterance \stepcounter{utterance}  
    & & \multicolumn{4}{p{0.6\linewidth}}{
        \cellcolor[rgb]{0.9,0.9,0.9}{
            \makecell[{{p{\linewidth}}}]{
                \texttt{\tiny{[P1$\langle$GM]}}
                \texttt{Ich bin der Spielleiter, und du spielst ein kooperatives Spiel mit dem folgenden Raster als Spielbrett:} \\
\\ 
\texttt{```} \\
\texttt{ 1234567} \\
\texttt{╔═══╤═╤═╗} \\
\texttt{║◌S◌│◌│A║ 1} \\
\texttt{╟─┬─┼─┴─╢ 2} \\
\texttt{║◌│◌│◌◌◌║ 3} \\
\texttt{║P├─┤D┌─╢ 4} \\
\texttt{╟─┘◌│◌│M║ 5} \\
\texttt{║◌◌◌└┬┴─╢ 6} \\
\texttt{║◌OT◌│◌◌║ 7} \\
\texttt{╚════╧══╝} \\
\\ 
\texttt{```} \\
\\ 
\texttt{* Am oberen Rand sind die x{-}Koordinaten angegeben, die nach rechts hin zunehmen, und am rechten Rand die y{-}Koordinaten, die nach unten zunehmen.} \\
\texttt{* Die folgenden Objekte sind zufällig auf deinem Raster platziert: 'P', 'O', 'T', 'S', 'D', 'A', 'M'.} \\
\\ 
\\ 
\texttt{Dein Mitspieler sieht eine Variation des Spielbretts, auf dem dieselben Objekte an anderen zufälligen Positionen platziert sind. Du kannst das Spielbrett des anderen Spielers nicht sehen, und er kann deins nicht sehen.} \\
\\ 
\\ 
\texttt{**Ziel des Spiels:**} \\
\\ 
\texttt{Beide Spieler müssen die Objekte auf ihrem jeweiligen Spielbrett so bewegen, dass identische Objekte am Ende an denselben Koordinaten liegen. Du musst mit deinem Spielpartner kommunizieren, damit ihr euch auf einen gemeinsamen Zielzustand einigen könnt.} \\
\\ 
\\ 
\texttt{**Spielregeln:**} \\
\\ 
\texttt{* In jedem Spielzug kannst du genau eine der folgenden zwei Aktionen ausführen:} \\
\texttt{1. `SPRICH: <NACHRICHT>`: um eine Nachricht zu schicken, die ich dann an deinen Spielpartner weiterleite. Als Nachricht gilt alles bis zum nächsten Zeilenumbruch.} \\
\texttt{2. `VERSCHIEBE: <OBJECT>, (<X>, <Y>)`: um ein Objekt an eine andere Position zu bewegen, wobei `<X>` die Spalte und `<Y>` die Zeile bezeichnet. Ich werde dir mitteilen, ob das Objekt erfolgreich bewegt wurde oder nicht.} \\
\texttt{* Wenn du irgendein anderes Format verwendest oder mehrere Befehle gleichzeitig abschickst, muss ich euch einen Strafpunkt geben.} \\
\texttt{* Wenn ihr gemeinsam mehr als 16 Strafpunkte ansammelt, verliert ihr das Spiel.} \\
\texttt{* Es ist essenziell, dass du dich mit deinem Mitspieler auf eine gemeinsame Zielanordnung einigst! Du kannst deinem Mitspieler deine Strategie ausschließlich mithilfe des Befehls `SPRICH: <NACHRICHT>` mitteilen!} \\
\\ 
\\ 
\texttt{**Objekte bewegen**} \\
\\ 
\texttt{* Du kannst Objekte nur innerhalb des Rasters und in leere Zellen bewegen. Leere Zellen erkennst am Zeichen '◌'.} \\
\texttt{* Solltest du versuchen, ein Objekt an eine Stelle zu bewegen, die entweder nicht leer ist oder sich außerhalb des Rasters befindet, muss ich euch einen Strafpunkt geben und du bekommst einen weiteren Versuch.} \\
\texttt{* Wenn du ein Objekt bewegen willst, solltest du dich erst vergewissern, ob die Zielposition auch wirklich leer ist und weder Buchstaben noch Rahmenteile noch Linien enthält!} \\
\\ 
\\ 
\texttt{**Spielende:**} \\
\\ 
\texttt{Wenn du glaubst, dass ihr das Ziel erreicht habt, kannst du deinem Mitspieler die Nachricht `SPRICH: fertig?` schicken. Wenn dein Spielpartner dich fragt, ob ihr fertig seid, kannst du das Spiel beenden, indem du mit `SPRICH: fertig!` antwortest.} \\
\\ 
\texttt{Ihr gewinnt gemeinsam, wenn ihr das Spiel innerhalb von \#max\_rounds Runden beendet. Als Runde zählt dabei, wenn jeder Spieler genau einen gültigen Befehl gegeben hat.} \\
\\ 
\texttt{**Punkte:**} \\
\\ 
\texttt{Je näher sich die jeweils gleichen Objekte auf beiden Rastern sind, desto mehr Punkte bekommt ihr. Für Strafpunkte gibt es Punktabzug. Könnt ihr den Rekord brechen?} \\
\\ 
\texttt{Schicke eine Nachricht an deinen Mitspieler um das Spiel zu starten!} \\
            }
        }
    }
    & \\ \\

    \theutterance \stepcounter{utterance}  
    & \multicolumn{4}{p{0.6\linewidth}}{
        \cellcolor[rgb]{0.9,0.9,0.9}{
            \makecell[{{p{\linewidth}}}]{
                \texttt{\tiny{[P1$\rangle$GM]}}
                \texttt{Ich analysiere mein Spielbrett und sehe die Positionen der Objekte:} \\
\\ 
\texttt{```} \\
\texttt{ 1234567} \\
\texttt{╔═══╤═╤═╗} \\
\texttt{║◌S◌│◌│A║ 1} \\
\texttt{╟─┬─┼─┴─╢ 2} \\
\texttt{║◌│◌│◌◌◌║ 3} \\
\texttt{║P├─┤D┌─╢ 4} \\
\texttt{╟─┘◌│◌│M║ 5} \\
\texttt{║◌◌◌└┬┴─╢ 6} \\
\texttt{║◌OT◌│◌◌║ 7} \\
\texttt{╚════╧══╝} \\
\texttt{```} \\
\\ 
\texttt{Meine aktuellen Objektpositionen:} \\
\texttt{{-} S bei (2,1)} \\
\texttt{{-} A bei (6,1) } \\
\texttt{{-} P bei (1,4)} \\
\texttt{{-} D bei (4,4)} \\
\texttt{{-} M bei (6,5)} \\
\texttt{{-} O bei (2,7)} \\
\texttt{{-} T bei (3,7)} \\
\\ 
\texttt{SPRICH: Hallo! Lass uns koordinieren. Meine Objektpositionen sind: S(2,1), A(6,1), P(1,4), D(4,4} \\
            }
        }
    }
    & & \\ \\

    \theutterance \stepcounter{utterance}  
    & & & \multicolumn{2}{p{0.3\linewidth}}{
        \cellcolor[rgb]{0.9,0.9,0.9}{
            \makecell[{{p{\linewidth}}}]{
                \texttt{\tiny{[GM$|$GM]}}
                \texttt{Ich analysiere mein Spielbrett und sehe die Positionen der Objekte:} \\
\\ 
\texttt{```} \\
\texttt{ 1234567} \\
\texttt{╔═══╤═╤═╗} \\
\texttt{║◌S◌│◌│A║ 1} \\
\texttt{╟─┬─┼─┴─╢ 2} \\
\texttt{║◌│◌│◌◌◌║ 3} \\
\texttt{║P├─┤D┌─╢ 4} \\
\texttt{╟─┘◌│◌│M║ 5} \\
\texttt{║◌◌◌└┬┴─╢ 6} \\
\texttt{║◌OT◌│◌◌║ 7} \\
\texttt{╚════╧══╝} \\
\texttt{```} \\
\\ 
\texttt{Meine aktuellen Objektpositionen:} \\
\texttt{{-} S bei (2,1)} \\
\texttt{{-} A bei (6,1) } \\
\texttt{{-} P bei (1,4)} \\
\texttt{{-} D bei (4,4)} \\
\texttt{{-} M bei (6,5)} \\
\texttt{{-} O bei (2,7)} \\
\texttt{{-} T bei (3,7)} \\
\\ 
\texttt{SPRICH: Hallo! Lass uns koordinieren. Meine Objektpositionen sind: S(2,1), A(6,1), P(1,4), D(4,4} \\
            }
        }
    }
    & & \\ \\

    \theutterance \stepcounter{utterance}  
    & & & \multicolumn{2}{p{0.3\linewidth}}{
        \cellcolor[rgb]{0.9,0.9,0.9}{
            \makecell[{{p{\linewidth}}}]{
                \texttt{\tiny{[GM$|$GM]}}
                \texttt{Invalid format: head is not empty: 'Ich analysiere mein Spielbrett und sehe die Positionen der Objekte:} \\
\\ 
\\ 
\texttt{ 1234567} \\
\texttt{╔═══╤═╤═╗} \\
\texttt{║◌S◌│◌│A║ 1} \\
\texttt{╟─┬─┼─┴─╢ 2} \\
\texttt{║◌│◌│◌◌◌║ 3} \\
\texttt{║P├─┤D┌─╢ 4} \\
\texttt{╟─┘◌│◌│M║ 5} \\
\texttt{║◌◌◌└┬┴─╢ 6} \\
\texttt{║◌OT◌│◌◌║ 7} \\
\texttt{╚════╧══╝} \\
\\ 
\\ 
\texttt{Meine aktuellen Objektpositionen:} \\
\texttt{{-} S bei (2,1)} \\
\texttt{{-} A bei (6,1) } \\
\texttt{{-} P bei (1,4)} \\
\texttt{{-} D bei (4,4)} \\
\texttt{{-} M bei (6,5)} \\
\texttt{{-} O bei (2,7)} \\
\texttt{{-} T bei (3,7)} \\
\\ 
\texttt{'} \\
            }
        }
    }
    & & \\ \\

    \theutterance \stepcounter{utterance}  
    & & \multicolumn{4}{p{0.6\linewidth}}{
        \cellcolor[rgb]{0.9,0.9,0.9}{
            \makecell[{{p{\linewidth}}}]{
                \texttt{\tiny{[P1$\langle$GM]}}
                \texttt{Strafpunkt: Deine Nachricht darf nichts vor dem Befehl enthalten! Stelle sicher, dass deine Antwort ausschließlich einen der beiden Befehle `SPRICH: <NACHRICHT>` und `VERSCHIEBE: <OBJECT>, (<X>, <Y>)` und sonst nichts enthält!} \\
\\ 
\texttt{Ihr habt gemeinsam 1 von maximal 16 Strafpunkten angesammelt.} \\
\texttt{Bitte versuche es noch mal!} \\
            }
        }
    }
    & \\ \\

    \theutterance \stepcounter{utterance}  
    & \multicolumn{4}{p{0.6\linewidth}}{
        \cellcolor[rgb]{0.9,0.9,0.9}{
            \makecell[{{p{\linewidth}}}]{
                \texttt{\tiny{[P1$\rangle$GM]}}
                \texttt{SPRICH: Hallo! Lass uns koordinieren. Meine Objektpositionen sind: S(2,1), A(6,1), P(1,4), D(4,4), M(6,5), O(2,7), T(3,7). Wo sind deine Objekte? Dann können wir eine gemeinsame Zielanordnung festlegen.} \\
            }
        }
    }
    & & \\ \\

    \theutterance \stepcounter{utterance}  
    & & & \multicolumn{2}{p{0.3\linewidth}}{
        \cellcolor[rgb]{0.9,0.9,0.9}{
            \makecell[{{p{\linewidth}}}]{
                \texttt{\tiny{[GM$|$GM]}}
                \texttt{SPRICH: Hallo! Lass uns koordinieren. Meine Objektpositionen sind: S(2,1), A(6,1), P(1,4), D(4,4), M(6,5), O(2,7), T(3,7). Wo sind deine Objekte? Dann können wir eine gemeinsame Zielanordnung festlegen.} \\
            }
        }
    }
    & & \\ \\

    \theutterance \stepcounter{utterance}  
    & & \multicolumn{4}{p{0.6\linewidth}}{
        \cellcolor[rgb]{0.9,0.9,0.9}{
            \makecell[{{p{\linewidth}}}]{
                \texttt{\tiny{[GM$\rangle$P2]}}
                \texttt{Ich bin der Spielleiter, und du spielst ein kooperatives Spiel mit dem folgenden Raster als Spielbrett:} \\
\\ 
\texttt{```} \\
\texttt{ 1234567} \\
\texttt{╔═══╤═╤═╗} \\
\texttt{║P◌A│◌│◌║ 1} \\
\texttt{╟─┬─┼─┴─╢ 2} \\
\texttt{║◌│D│◌◌◌║ 3} \\
\texttt{║◌├─┤◌┌─╢ 4} \\
\texttt{╟─┘◌│◌│O║ 5} \\
\texttt{║◌◌◌└┬┴─╢ 6} \\
\texttt{║◌S◌T│◌M║ 7} \\
\texttt{╚════╧══╝} \\
\\ 
\texttt{```} \\
\\ 
\texttt{* Am oberen Rand sind die x{-}Koordinaten angegeben, die nach rechts hin zunehmen, und am rechten Rand die y{-}Koordinaten, die nach unten zunehmen.} \\
\texttt{* Die folgenden Objekte sind zufällig auf deinem Raster platziert: 'P', 'O', 'T', 'S', 'D', 'A', 'M'.} \\
\\ 
\\ 
\texttt{Dein Mitspieler sieht eine Variation des Spielbretts, auf dem dieselben Objekte an anderen zufälligen Positionen platziert sind. Du kannst das Spielbrett des anderen Spielers nicht sehen, und er kann deins nicht sehen.} \\
\\ 
\\ 
\texttt{**Ziel des Spiels:**} \\
\\ 
\texttt{Beide Spieler müssen die Objekte auf ihrem jeweiligen Spielbrett so bewegen, dass identische Objekte am Ende an denselben Koordinaten liegen. Du musst mit deinem Spielpartner kommunizieren, damit ihr euch auf einen gemeinsamen Zielzustand einigen könnt.} \\
\\ 
\\ 
\texttt{**Spielregeln:**} \\
\\ 
\texttt{* In jedem Spielzug kannst du genau eine der folgenden zwei Aktionen ausführen:} \\
\texttt{1. `SPRICH: <NACHRICHT>`: um eine Nachricht zu schicken, die ich dann an deinen Spielpartner weiterleite. Als Nachricht gilt alles bis zum nächsten Zeilenumbruch.} \\
\texttt{2. `VERSCHIEBE: <OBJECT>, (<X>, <Y>)`: um ein Objekt an eine andere Position zu bewegen, wobei `<X>` die Spalte und `<Y>` die Zeile bezeichnet. Ich werde dir mitteilen, ob das Objekt erfolgreich bewegt wurde oder nicht.} \\
\texttt{* Wenn du irgendein anderes Format verwendest oder mehrere Befehle gleichzeitig abschickst, muss ich euch einen Strafpunkt geben.} \\
\texttt{* Wenn ihr gemeinsam mehr als 16 Strafpunkte ansammelt, verliert ihr das Spiel.} \\
\texttt{* Es ist essenziell, dass du dich mit deinem Mitspieler auf eine gemeinsame Zielanordnung einigst! Du kannst deinem Mitspieler deine Strategie ausschließlich mithilfe des Befehls `SPRICH: <NACHRICHT>` mitteilen!} \\
\\ 
\\ 
\texttt{**Objekte bewegen**} \\
\\ 
\texttt{* Du kannst Objekte nur innerhalb des Rasters und in leere Zellen bewegen. Leere Zellen erkennst am Zeichen '◌'.} \\
\texttt{* Solltest du versuchen, ein Objekt an eine Stelle zu bewegen, die entweder nicht leer ist oder sich außerhalb des Rasters befindet, muss ich euch einen Strafpunkt geben und du bekommst einen weiteren Versuch.} \\
\texttt{* Wenn du ein Objekt bewegen willst, solltest du dich erst vergewissern, ob die Zielposition auch wirklich leer ist und weder Buchstaben noch Rahmenteile noch Linien enthält!} \\
\\ 
\\ 
\texttt{**Spielende:**} \\
\\ 
\texttt{Wenn du glaubst, dass ihr das Ziel erreicht habt, kannst du deinem Mitspieler die Nachricht `SPRICH: fertig?` schicken. Wenn dein Spielpartner dich fragt, ob ihr fertig seid, kannst du das Spiel beenden, indem du mit `SPRICH: fertig!` antwortest.} \\
\\ 
\texttt{Ihr gewinnt gemeinsam, wenn ihr das Spiel innerhalb von \#max\_rounds Runden beendet. Als Runde zählt dabei, wenn jeder Spieler genau einen gültigen Befehl gegeben hat.} \\
\\ 
\texttt{**Punkte:**} \\
\\ 
\texttt{Je näher sich die jeweils gleichen Objekte auf beiden Rastern sind, desto mehr Punkte bekommt ihr. Für Strafpunkte gibt es Punktabzug. Könnt ihr den Rekord brechen?} \\
\\ 
\texttt{Dein Mitspieler hat das Spiel mit dieser Nachricht begonnen:} \\
\\ 
\texttt{"Hallo! Lass uns koordinieren. Meine Objektpositionen sind: S(2,1), A(6,1), P(1,4), D(4,4), M(6,5), O(2,7), T(3,7). Wo sind deine Objekte? Dann können wir eine gemeinsame Zielanordnung festlegen."} \\
\\ 
\texttt{Wie lautet dein erster Befehl?} \\
            }
        }
    }
    & \\ \\

    \theutterance \stepcounter{utterance}  
    & & & \multicolumn{4}{p{0.6\linewidth}}{
        \cellcolor[rgb]{0.9,0.9,0.9}{
            \makecell[{{p{\linewidth}}}]{
                \texttt{\tiny{[GM$\langle$P2]}}
                \texttt{Ich schaue mir zunächst mein Spielbrett an und identifiziere die Positionen meiner Objekte:} \\
\\ 
\texttt{```} \\
\texttt{ 1234567} \\
\texttt{╔═══╤═╤═╗} \\
\texttt{║P◌A│◌│◌║ 1} \\
\texttt{╟─┬─┼─┴─╢ 2} \\
\texttt{║◌│D│◌◌◌║ 3} \\
\texttt{║◌├─┤◌┌─╢ 4} \\
\texttt{╟─┘◌│◌│O║ 5} \\
\texttt{║◌◌◌└┬┴─╢ 6} \\
\texttt{║◌S◌T│◌M║ 7} \\
\texttt{╚════╧══╝} \\
\texttt{```} \\
\\ 
\texttt{Meine Objektpositionen sind:} \\
\texttt{{-} P: (1,1)} \\
\texttt{{-} A: (3,1)} \\
\texttt{{-} D: (3,3)} \\
\texttt{{-} O: (7,5)} \\
\texttt{{-} S: (2,7)} \\
\texttt{{-} T: (4,7)} \\
\texttt{{-} M: (7,7)} \\
\\ 
\texttt{SPRICH: Hallo! Meine Objektpositionen sind: P(1,1), A(3,1), D(3,3), O(7,5), S} \\
            }
        }
    }
     \\ \\

    \theutterance \stepcounter{utterance}  
    & & & \multicolumn{2}{p{0.3\linewidth}}{
        \cellcolor[rgb]{0.9,0.9,0.9}{
            \makecell[{{p{\linewidth}}}]{
                \texttt{\tiny{[GM$|$GM]}}
                \texttt{Ich schaue mir zunächst mein Spielbrett an und identifiziere die Positionen meiner Objekte:} \\
\\ 
\texttt{```} \\
\texttt{ 1234567} \\
\texttt{╔═══╤═╤═╗} \\
\texttt{║P◌A│◌│◌║ 1} \\
\texttt{╟─┬─┼─┴─╢ 2} \\
\texttt{║◌│D│◌◌◌║ 3} \\
\texttt{║◌├─┤◌┌─╢ 4} \\
\texttt{╟─┘◌│◌│O║ 5} \\
\texttt{║◌◌◌└┬┴─╢ 6} \\
\texttt{║◌S◌T│◌M║ 7} \\
\texttt{╚════╧══╝} \\
\texttt{```} \\
\\ 
\texttt{Meine Objektpositionen sind:} \\
\texttt{{-} P: (1,1)} \\
\texttt{{-} A: (3,1)} \\
\texttt{{-} D: (3,3)} \\
\texttt{{-} O: (7,5)} \\
\texttt{{-} S: (2,7)} \\
\texttt{{-} T: (4,7)} \\
\texttt{{-} M: (7,7)} \\
\\ 
\texttt{SPRICH: Hallo! Meine Objektpositionen sind: P(1,1), A(3,1), D(3,3), O(7,5), S} \\
            }
        }
    }
    & & \\ \\

    \theutterance \stepcounter{utterance}  
    & & & \multicolumn{2}{p{0.3\linewidth}}{
        \cellcolor[rgb]{0.9,0.9,0.9}{
            \makecell[{{p{\linewidth}}}]{
                \texttt{\tiny{[GM$|$GM]}}
                \texttt{Invalid format: head is not empty: 'Ich schaue mir zunächst mein Spielbrett an und identifiziere die Positionen meiner Objekte:} \\
\\ 
\\ 
\texttt{ 1234567} \\
\texttt{╔═══╤═╤═╗} \\
\texttt{║P◌A│◌│◌║ 1} \\
\texttt{╟─┬─┼─┴─╢ 2} \\
\texttt{║◌│D│◌◌◌║ 3} \\
\texttt{║◌├─┤◌┌─╢ 4} \\
\texttt{╟─┘◌│◌│O║ 5} \\
\texttt{║◌◌◌└┬┴─╢ 6} \\
\texttt{║◌S◌T│◌M║ 7} \\
\texttt{╚════╧══╝} \\
\\ 
\\ 
\texttt{Meine Objektpositionen sind:} \\
\texttt{{-} P: (1,1)} \\
\texttt{{-} A: (3,1)} \\
\texttt{{-} D: (3,3)} \\
\texttt{{-} O: (7,5)} \\
\texttt{{-} S: (2,7)} \\
\texttt{{-} T: (4,7)} \\
\texttt{{-} M: (7,7)} \\
\\ 
\texttt{'} \\
            }
        }
    }
    & & \\ \\

    \theutterance \stepcounter{utterance}  
    & & \multicolumn{4}{p{0.6\linewidth}}{
        \cellcolor[rgb]{0.9,0.9,0.9}{
            \makecell[{{p{\linewidth}}}]{
                \texttt{\tiny{[GM$\rangle$P2]}}
                \texttt{Strafpunkt: Deine Nachricht darf nichts vor dem Befehl enthalten! Stelle sicher, dass deine Antwort ausschließlich einen der beiden Befehle `SPRICH: <NACHRICHT>` und `VERSCHIEBE: <OBJECT>, (<X>, <Y>)` und sonst nichts enthält!} \\
\\ 
\texttt{Ihr habt gemeinsam 2 von maximal 16 Strafpunkten angesammelt.} \\
\texttt{Bitte versuche es noch mal!} \\
            }
        }
    }
    & \\ \\

    \theutterance \stepcounter{utterance}  
    & & & \multicolumn{4}{p{0.6\linewidth}}{
        \cellcolor[rgb]{0.9,0.9,0.9}{
            \makecell[{{p{\linewidth}}}]{
                \texttt{\tiny{[GM$\langle$P2]}}
                \texttt{SPRICH: Hallo! Meine Objektpositionen sind: P(1,1), A(3,1), D(3,3), O(7,5), S(2,7), T(4,7), M(7,7). Lass uns eine gemeinsame Zielanordnung festlegen!} \\
            }
        }
    }
     \\ \\

    \theutterance \stepcounter{utterance}  
    & & & \multicolumn{2}{p{0.3\linewidth}}{
        \cellcolor[rgb]{0.9,0.9,0.9}{
            \makecell[{{p{\linewidth}}}]{
                \texttt{\tiny{[GM$|$GM]}}
                \texttt{SPRICH: Hallo! Meine Objektpositionen sind: P(1,1), A(3,1), D(3,3), O(7,5), S(2,7), T(4,7), M(7,7). Lass uns eine gemeinsame Zielanordnung festlegen!} \\
            }
        }
    }
    & & \\ \\

    \theutterance \stepcounter{utterance}  
    & & \multicolumn{4}{p{0.6\linewidth}}{
        \cellcolor[rgb]{0.9,0.9,0.9}{
            \makecell[{{p{\linewidth}}}]{
                \texttt{\tiny{[P1$\langle$GM]}}
                \texttt{Deine Nachricht wurde an deinen Mitspieler übermittelt.} \\
\\ 
\\ 
\texttt{Ihr seid gerade bei Runde 1 von maximal 28.} \\
\texttt{Ihr habt gemeinsam 2 von maximal 16 Strafpunkten angesammelt.} \\
\texttt{Dein Mitspieler hat dir diese Nachricht geschickt:} \\
\\ 
\\ 
\texttt{"Hallo! Meine Objektpositionen sind: P(1,1), A(3,1), D(3,3), O(7,5), S(2,7), T(4,7), M(7,7). Lass uns eine gemeinsame Zielanordnung festlegen!"} \\
\\ 
\\ 
\texttt{Wie lautet dein nächster Befehl?} \\
            }
        }
    }
    & \\ \\

    \theutterance \stepcounter{utterance}  
    & \multicolumn{4}{p{0.6\linewidth}}{
        \cellcolor[rgb]{0.9,0.9,0.9}{
            \makecell[{{p{\linewidth}}}]{
                \texttt{\tiny{[P1$\rangle$GM]}}
                \texttt{SPRICH: Gut! Lass uns eine einfache Zielanordnung wählen. Ich schlage vor: P(1,1), A(3,1), S(2,1), D(3,3), O(7,5), T(4,7), M(7,7). Das entspricht größtenteils deinen aktuellen Positionen und minimiert unsere Bewegungen. Bist du einverstanden?} \\
            }
        }
    }
    & & \\ \\

    \theutterance \stepcounter{utterance}  
    & & & \multicolumn{2}{p{0.3\linewidth}}{
        \cellcolor[rgb]{0.9,0.9,0.9}{
            \makecell[{{p{\linewidth}}}]{
                \texttt{\tiny{[GM$|$GM]}}
                \texttt{SPRICH: Gut! Lass uns eine einfache Zielanordnung wählen. Ich schlage vor: P(1,1), A(3,1), S(2,1), D(3,3), O(7,5), T(4,7), M(7,7). Das entspricht größtenteils deinen aktuellen Positionen und minimiert unsere Bewegungen. Bist du einverstanden?} \\
            }
        }
    }
    & & \\ \\

    \theutterance \stepcounter{utterance}  
    & & \multicolumn{4}{p{0.6\linewidth}}{
        \cellcolor[rgb]{0.9,0.9,0.9}{
            \makecell[{{p{\linewidth}}}]{
                \texttt{\tiny{[GM$\rangle$P2]}}
                \texttt{Deine Nachricht wurde an deinen Mitspieler übermittelt.} \\
\\ 
\\ 
\texttt{Ihr seid gerade bei Runde 2 von maximal 28.} \\
\texttt{Ihr habt gemeinsam 2 von maximal 16 Strafpunkten angesammelt.} \\
\texttt{Dein Mitspieler hat dir diese Nachricht geschickt:} \\
\\ 
\\ 
\texttt{"Gut! Lass uns eine einfache Zielanordnung wählen. Ich schlage vor: P(1,1), A(3,1), S(2,1), D(3,3), O(7,5), T(4,7), M(7,7). Das entspricht größtenteils deinen aktuellen Positionen und minimiert unsere Bewegungen. Bist du einverstanden?"} \\
\\ 
\\ 
\texttt{Wie lautet dein nächster Befehl?} \\
            }
        }
    }
    & \\ \\

    \theutterance \stepcounter{utterance}  
    & & & \multicolumn{4}{p{0.6\linewidth}}{
        \cellcolor[rgb]{0.9,0.9,0.9}{
            \makecell[{{p{\linewidth}}}]{
                \texttt{\tiny{[GM$\langle$P2]}}
                \texttt{SPRICH: Ja, einverstanden! Das ist eine gute Strategie. Ich muss nur S von (2,7) nach (2,1) bewegen. Du musst mehr Objekte bewegen. Soll ich anfangen und mein S bewegen?} \\
            }
        }
    }
     \\ \\

    \theutterance \stepcounter{utterance}  
    & & & \multicolumn{2}{p{0.3\linewidth}}{
        \cellcolor[rgb]{0.9,0.9,0.9}{
            \makecell[{{p{\linewidth}}}]{
                \texttt{\tiny{[GM$|$GM]}}
                \texttt{SPRICH: Ja, einverstanden! Das ist eine gute Strategie. Ich muss nur S von (2,7) nach (2,1) bewegen. Du musst mehr Objekte bewegen. Soll ich anfangen und mein S bewegen?} \\
            }
        }
    }
    & & \\ \\

    \theutterance \stepcounter{utterance}  
    & & \multicolumn{4}{p{0.6\linewidth}}{
        \cellcolor[rgb]{0.9,0.9,0.9}{
            \makecell[{{p{\linewidth}}}]{
                \texttt{\tiny{[P1$\langle$GM]}}
                \texttt{Deine Nachricht wurde an deinen Mitspieler übermittelt.} \\
\\ 
\\ 
\texttt{Ihr seid gerade bei Runde 2 von maximal 28.} \\
\texttt{Ihr habt gemeinsam 2 von maximal 16 Strafpunkten angesammelt.} \\
\texttt{Dein Mitspieler hat dir diese Nachricht geschickt:} \\
\\ 
\\ 
\texttt{"Ja, einverstanden! Das ist eine gute Strategie. Ich muss nur S von (2,7) nach (2,1) bewegen. Du musst mehr Objekte bewegen. Soll ich anfangen und mein S bewegen?"} \\
\\ 
\\ 
\texttt{Wie lautet dein nächster Befehl?} \\
            }
        }
    }
    & \\ \\

    \theutterance \stepcounter{utterance}  
    & \multicolumn{4}{p{0.6\linewidth}}{
        \cellcolor[rgb]{0.9,0.9,0.9}{
            \makecell[{{p{\linewidth}}}]{
                \texttt{\tiny{[P1$\rangle$GM]}}
                \texttt{SPRICH: Ja, fang gerne an! Ich muss folgende Bewegungen machen: P von (1,4) nach (1,1), A von (6,1) nach (3,1), D von (4,4) nach (3,3), O von (2,7) nach (7,5), T von (3,7) nach (4,7), M von (6,5) nach (7,7). Bewege dein S, dann starte ich mit meinen Zügen.} \\
            }
        }
    }
    & & \\ \\

    \theutterance \stepcounter{utterance}  
    & & & \multicolumn{2}{p{0.3\linewidth}}{
        \cellcolor[rgb]{0.9,0.9,0.9}{
            \makecell[{{p{\linewidth}}}]{
                \texttt{\tiny{[GM$|$GM]}}
                \texttt{SPRICH: Ja, fang gerne an! Ich muss folgende Bewegungen machen: P von (1,4) nach (1,1), A von (6,1) nach (3,1), D von (4,4) nach (3,3), O von (2,7) nach (7,5), T von (3,7) nach (4,7), M von (6,5) nach (7,7). Bewege dein S, dann starte ich mit meinen Zügen.} \\
            }
        }
    }
    & & \\ \\

    \theutterance \stepcounter{utterance}  
    & & \multicolumn{4}{p{0.6\linewidth}}{
        \cellcolor[rgb]{0.9,0.9,0.9}{
            \makecell[{{p{\linewidth}}}]{
                \texttt{\tiny{[GM$\rangle$P2]}}
                \texttt{Deine Nachricht wurde an deinen Mitspieler übermittelt.} \\
\\ 
\\ 
\texttt{Ihr seid gerade bei Runde 3 von maximal 28.} \\
\texttt{Ihr habt gemeinsam 2 von maximal 16 Strafpunkten angesammelt.} \\
\texttt{Dein Mitspieler hat dir diese Nachricht geschickt:} \\
\\ 
\\ 
\texttt{"Ja, fang gerne an! Ich muss folgende Bewegungen machen: P von (1,4) nach (1,1), A von (6,1) nach (3,1), D von (4,4) nach (3,3), O von (2,7) nach (7,5), T von (3,7) nach (4,7), M von (6,5) nach (7,7). Bewege dein S, dann starte ich mit meinen Zügen."} \\
\\ 
\\ 
\texttt{Wie lautet dein nächster Befehl?} \\
            }
        }
    }
    & \\ \\

    \theutterance \stepcounter{utterance}  
    & & & \multicolumn{4}{p{0.6\linewidth}}{
        \cellcolor[rgb]{0.9,0.9,0.9}{
            \makecell[{{p{\linewidth}}}]{
                \texttt{\tiny{[GM$\langle$P2]}}
                \texttt{VERSCHIEBE: S, (2, 1)} \\
            }
        }
    }
     \\ \\

    \theutterance \stepcounter{utterance}  
    & & & \multicolumn{2}{p{0.3\linewidth}}{
        \cellcolor[rgb]{0.9,0.9,0.9}{
            \makecell[{{p{\linewidth}}}]{
                \texttt{\tiny{[GM$|$GM]}}
                \texttt{VERSCHIEBE: S, (2, 1)} \\
            }
        }
    }
    & & \\ \\

    \theutterance \stepcounter{utterance}  
    & & & \multicolumn{2}{p{0.3\linewidth}}{
        \cellcolor[rgb]{0.9,0.9,0.9}{
            \makecell[{{p{\linewidth}}}]{
                \texttt{\tiny{[GM$|$GM]}}
                \texttt{'S' erfolgreich nach (2,1) verschoben. Dein Raster sieht jetzt so aus:} \\
\texttt{```} \\
\texttt{ 1234567} \\
\texttt{╔═══╤═╤═╗} \\
\texttt{║PSA│◌│◌║ 1} \\
\texttt{╟─┬─┼─┴─╢ 2} \\
\texttt{║◌│D│◌◌◌║ 3} \\
\texttt{║◌├─┤◌┌─╢ 4} \\
\texttt{╟─┘◌│◌│O║ 5} \\
\texttt{║◌◌◌└┬┴─╢ 6} \\
\texttt{║◌◌◌T│◌M║ 7} \\
\texttt{╚════╧══╝} \\
\texttt{```} \\
            }
        }
    }
    & & \\ \\

    \theutterance \stepcounter{utterance}  
    & & \multicolumn{4}{p{0.6\linewidth}}{
        \cellcolor[rgb]{0.9,0.9,0.9}{
            \makecell[{{p{\linewidth}}}]{
                \texttt{\tiny{[P1$\langle$GM]}}
                \texttt{Deine Nachricht wurde an deinen Mitspieler übermittelt.} \\
\\ 
\\ 
\texttt{Ihr seid gerade bei Runde 3 von maximal 28.} \\
\texttt{Ihr habt gemeinsam 2 von maximal 16 Strafpunkten angesammelt.} \\
\texttt{Dein Mitspieler hat ein Objekt auf seinem Raster bewegt.} \\
\\ 
\\ 
\texttt{Wie lautet dein nächster Befehl?} \\
            }
        }
    }
    & \\ \\

    \theutterance \stepcounter{utterance}  
    & \multicolumn{4}{p{0.6\linewidth}}{
        \cellcolor[rgb]{0.9,0.9,0.9}{
            \makecell[{{p{\linewidth}}}]{
                \texttt{\tiny{[P1$\rangle$GM]}}
                \texttt{VERSCHIEBE: P, (1, 1)} \\
            }
        }
    }
    & & \\ \\

    \theutterance \stepcounter{utterance}  
    & & & \multicolumn{2}{p{0.3\linewidth}}{
        \cellcolor[rgb]{0.9,0.9,0.9}{
            \makecell[{{p{\linewidth}}}]{
                \texttt{\tiny{[GM$|$GM]}}
                \texttt{VERSCHIEBE: P, (1, 1)} \\
            }
        }
    }
    & & \\ \\

    \theutterance \stepcounter{utterance}  
    & & & \multicolumn{2}{p{0.3\linewidth}}{
        \cellcolor[rgb]{0.9,0.9,0.9}{
            \makecell[{{p{\linewidth}}}]{
                \texttt{\tiny{[GM$|$GM]}}
                \texttt{'P' erfolgreich nach (1,1) verschoben. Dein Raster sieht jetzt so aus:} \\
\texttt{```} \\
\texttt{ 1234567} \\
\texttt{╔═══╤═╤═╗} \\
\texttt{║PS◌│◌│A║ 1} \\
\texttt{╟─┬─┼─┴─╢ 2} \\
\texttt{║◌│◌│◌◌◌║ 3} \\
\texttt{║◌├─┤D┌─╢ 4} \\
\texttt{╟─┘◌│◌│M║ 5} \\
\texttt{║◌◌◌└┬┴─╢ 6} \\
\texttt{║◌OT◌│◌◌║ 7} \\
\texttt{╚════╧══╝} \\
\texttt{```} \\
            }
        }
    }
    & & \\ \\

    \theutterance \stepcounter{utterance}  
    & & \multicolumn{4}{p{0.6\linewidth}}{
        \cellcolor[rgb]{0.9,0.9,0.9}{
            \makecell[{{p{\linewidth}}}]{
                \texttt{\tiny{[GM$\rangle$P2]}}
                \texttt{'S' erfolgreich nach (2,1) verschoben. Dein Raster sieht jetzt so aus:} \\
\texttt{```} \\
\texttt{ 1234567} \\
\texttt{╔═══╤═╤═╗} \\
\texttt{║PSA│◌│◌║ 1} \\
\texttt{╟─┬─┼─┴─╢ 2} \\
\texttt{║◌│D│◌◌◌║ 3} \\
\texttt{║◌├─┤◌┌─╢ 4} \\
\texttt{╟─┘◌│◌│O║ 5} \\
\texttt{║◌◌◌└┬┴─╢ 6} \\
\texttt{║◌◌◌T│◌M║ 7} \\
\texttt{╚════╧══╝} \\
\texttt{```} \\
\texttt{Ihr seid gerade bei Runde 4 von maximal 28.} \\
\texttt{Ihr habt gemeinsam 2 von maximal 16 Strafpunkten angesammelt.} \\
\texttt{Dein Mitspieler hat ein Objekt auf seinem Raster bewegt.} \\
\\ 
\\ 
\texttt{Wie lautet dein nächster Befehl?} \\
            }
        }
    }
    & \\ \\

    \theutterance \stepcounter{utterance}  
    & & & \multicolumn{4}{p{0.6\linewidth}}{
        \cellcolor[rgb]{0.9,0.9,0.9}{
            \makecell[{{p{\linewidth}}}]{
                \texttt{\tiny{[GM$\langle$P2]}}
                \texttt{SPRICH: Perfekt! Ich bin bereits fertig {-} alle meine Objekte sind an den Zielpositionen: P(1,1), S(2,1), A(3,1), D(3,3), O(7,5), T(4,7), M(7,7). Bewege deine Objekte zu den vereinbarten Positionen!} \\
            }
        }
    }
     \\ \\

    \theutterance \stepcounter{utterance}  
    & & & \multicolumn{2}{p{0.3\linewidth}}{
        \cellcolor[rgb]{0.9,0.9,0.9}{
            \makecell[{{p{\linewidth}}}]{
                \texttt{\tiny{[GM$|$GM]}}
                \texttt{SPRICH: Perfekt! Ich bin bereits fertig {-} alle meine Objekte sind an den Zielpositionen: P(1,1), S(2,1), A(3,1), D(3,3), O(7,5), T(4,7), M(7,7). Bewege deine Objekte zu den vereinbarten Positionen!} \\
            }
        }
    }
    & & \\ \\

    \theutterance \stepcounter{utterance}  
    & & \multicolumn{4}{p{0.6\linewidth}}{
        \cellcolor[rgb]{0.9,0.9,0.9}{
            \makecell[{{p{\linewidth}}}]{
                \texttt{\tiny{[P1$\langle$GM]}}
                \texttt{'P' erfolgreich nach (1,1) verschoben. Dein Raster sieht jetzt so aus:} \\
\texttt{```} \\
\texttt{ 1234567} \\
\texttt{╔═══╤═╤═╗} \\
\texttt{║PS◌│◌│A║ 1} \\
\texttt{╟─┬─┼─┴─╢ 2} \\
\texttt{║◌│◌│◌◌◌║ 3} \\
\texttt{║◌├─┤D┌─╢ 4} \\
\texttt{╟─┘◌│◌│M║ 5} \\
\texttt{║◌◌◌└┬┴─╢ 6} \\
\texttt{║◌OT◌│◌◌║ 7} \\
\texttt{╚════╧══╝} \\
\texttt{```} \\
\texttt{Ihr seid gerade bei Runde 4 von maximal 28.} \\
\texttt{Ihr habt gemeinsam 2 von maximal 16 Strafpunkten angesammelt.} \\
\texttt{Dein Mitspieler hat dir diese Nachricht geschickt:} \\
\\ 
\\ 
\texttt{"Perfekt! Ich bin bereits fertig {-} alle meine Objekte sind an den Zielpositionen: P(1,1), S(2,1), A(3,1), D(3,3), O(7,5), T(4,7), M(7,7). Bewege deine Objekte zu den vereinbarten Positionen!"} \\
\\ 
\\ 
\texttt{Wie lautet dein nächster Befehl?} \\
            }
        }
    }
    & \\ \\

    \theutterance \stepcounter{utterance}  
    & \multicolumn{4}{p{0.6\linewidth}}{
        \cellcolor[rgb]{0.9,0.9,0.9}{
            \makecell[{{p{\linewidth}}}]{
                \texttt{\tiny{[P1$\rangle$GM]}}
                \texttt{VERSCHIEBE: A, (3, 1)} \\
            }
        }
    }
    & & \\ \\

    \theutterance \stepcounter{utterance}  
    & & & \multicolumn{2}{p{0.3\linewidth}}{
        \cellcolor[rgb]{0.9,0.9,0.9}{
            \makecell[{{p{\linewidth}}}]{
                \texttt{\tiny{[GM$|$GM]}}
                \texttt{VERSCHIEBE: A, (3, 1)} \\
            }
        }
    }
    & & \\ \\

    \theutterance \stepcounter{utterance}  
    & & & \multicolumn{2}{p{0.3\linewidth}}{
        \cellcolor[rgb]{0.9,0.9,0.9}{
            \makecell[{{p{\linewidth}}}]{
                \texttt{\tiny{[GM$|$GM]}}
                \texttt{'A' erfolgreich nach (3,1) verschoben. Dein Raster sieht jetzt so aus:} \\
\texttt{```} \\
\texttt{ 1234567} \\
\texttt{╔═══╤═╤═╗} \\
\texttt{║PSA│◌│◌║ 1} \\
\texttt{╟─┬─┼─┴─╢ 2} \\
\texttt{║◌│◌│◌◌◌║ 3} \\
\texttt{║◌├─┤D┌─╢ 4} \\
\texttt{╟─┘◌│◌│M║ 5} \\
\texttt{║◌◌◌└┬┴─╢ 6} \\
\texttt{║◌OT◌│◌◌║ 7} \\
\texttt{╚════╧══╝} \\
\texttt{```} \\
            }
        }
    }
    & & \\ \\

    \theutterance \stepcounter{utterance}  
    & & \multicolumn{4}{p{0.6\linewidth}}{
        \cellcolor[rgb]{0.9,0.9,0.9}{
            \makecell[{{p{\linewidth}}}]{
                \texttt{\tiny{[GM$\rangle$P2]}}
                \texttt{Deine Nachricht wurde an deinen Mitspieler übermittelt.} \\
\\ 
\\ 
\texttt{Ihr seid gerade bei Runde 5 von maximal 28.} \\
\texttt{Ihr habt gemeinsam 2 von maximal 16 Strafpunkten angesammelt.} \\
\texttt{Dein Mitspieler hat ein Objekt auf seinem Raster bewegt.} \\
\\ 
\\ 
\texttt{Wie lautet dein nächster Befehl?} \\
            }
        }
    }
    & \\ \\

    \theutterance \stepcounter{utterance}  
    & & & \multicolumn{4}{p{0.6\linewidth}}{
        \cellcolor[rgb]{0.9,0.9,0.9}{
            \makecell[{{p{\linewidth}}}]{
                \texttt{\tiny{[GM$\langle$P2]}}
                \texttt{SPRICH: Ich warte auf dich. Lass mich wissen, wenn du fertig bist oder ob du Hilfe brauchst!} \\
            }
        }
    }
     \\ \\

    \theutterance \stepcounter{utterance}  
    & & & \multicolumn{2}{p{0.3\linewidth}}{
        \cellcolor[rgb]{0.9,0.9,0.9}{
            \makecell[{{p{\linewidth}}}]{
                \texttt{\tiny{[GM$|$GM]}}
                \texttt{SPRICH: Ich warte auf dich. Lass mich wissen, wenn du fertig bist oder ob du Hilfe brauchst!} \\
            }
        }
    }
    & & \\ \\

    \theutterance \stepcounter{utterance}  
    & & \multicolumn{4}{p{0.6\linewidth}}{
        \cellcolor[rgb]{0.9,0.9,0.9}{
            \makecell[{{p{\linewidth}}}]{
                \texttt{\tiny{[P1$\langle$GM]}}
                \texttt{'A' erfolgreich nach (3,1) verschoben. Dein Raster sieht jetzt so aus:} \\
\texttt{```} \\
\texttt{ 1234567} \\
\texttt{╔═══╤═╤═╗} \\
\texttt{║PSA│◌│◌║ 1} \\
\texttt{╟─┬─┼─┴─╢ 2} \\
\texttt{║◌│◌│◌◌◌║ 3} \\
\texttt{║◌├─┤D┌─╢ 4} \\
\texttt{╟─┘◌│◌│M║ 5} \\
\texttt{║◌◌◌└┬┴─╢ 6} \\
\texttt{║◌OT◌│◌◌║ 7} \\
\texttt{╚════╧══╝} \\
\texttt{```} \\
\texttt{Ihr seid gerade bei Runde 5 von maximal 28.} \\
\texttt{Ihr habt gemeinsam 2 von maximal 16 Strafpunkten angesammelt.} \\
\texttt{Dein Mitspieler hat dir diese Nachricht geschickt:} \\
\\ 
\\ 
\texttt{"Ich warte auf dich. Lass mich wissen, wenn du fertig bist oder ob du Hilfe brauchst!"} \\
\\ 
\\ 
\texttt{Wie lautet dein nächster Befehl?} \\
            }
        }
    }
    & \\ \\

    \theutterance \stepcounter{utterance}  
    & \multicolumn{4}{p{0.6\linewidth}}{
        \cellcolor[rgb]{0.9,0.9,0.9}{
            \makecell[{{p{\linewidth}}}]{
                \texttt{\tiny{[P1$\rangle$GM]}}
                \texttt{VERSCHIEBE: D, (3, 3)} \\
            }
        }
    }
    & & \\ \\

    \theutterance \stepcounter{utterance}  
    & & & \multicolumn{2}{p{0.3\linewidth}}{
        \cellcolor[rgb]{0.9,0.9,0.9}{
            \makecell[{{p{\linewidth}}}]{
                \texttt{\tiny{[GM$|$GM]}}
                \texttt{VERSCHIEBE: D, (3, 3)} \\
            }
        }
    }
    & & \\ \\

    \theutterance \stepcounter{utterance}  
    & & & \multicolumn{2}{p{0.3\linewidth}}{
        \cellcolor[rgb]{0.9,0.9,0.9}{
            \makecell[{{p{\linewidth}}}]{
                \texttt{\tiny{[GM$|$GM]}}
                \texttt{'D' erfolgreich nach (3,3) verschoben. Dein Raster sieht jetzt so aus:} \\
\texttt{```} \\
\texttt{ 1234567} \\
\texttt{╔═══╤═╤═╗} \\
\texttt{║PSA│◌│◌║ 1} \\
\texttt{╟─┬─┼─┴─╢ 2} \\
\texttt{║◌│D│◌◌◌║ 3} \\
\texttt{║◌├─┤◌┌─╢ 4} \\
\texttt{╟─┘◌│◌│M║ 5} \\
\texttt{║◌◌◌└┬┴─╢ 6} \\
\texttt{║◌OT◌│◌◌║ 7} \\
\texttt{╚════╧══╝} \\
\texttt{```} \\
            }
        }
    }
    & & \\ \\

    \theutterance \stepcounter{utterance}  
    & & \multicolumn{4}{p{0.6\linewidth}}{
        \cellcolor[rgb]{0.9,0.9,0.9}{
            \makecell[{{p{\linewidth}}}]{
                \texttt{\tiny{[GM$\rangle$P2]}}
                \texttt{Deine Nachricht wurde an deinen Mitspieler übermittelt.} \\
\\ 
\\ 
\texttt{Ihr seid gerade bei Runde 6 von maximal 28.} \\
\texttt{Ihr habt gemeinsam 2 von maximal 16 Strafpunkten angesammelt.} \\
\texttt{Dein Mitspieler hat ein Objekt auf seinem Raster bewegt.} \\
\\ 
\\ 
\texttt{Wie lautet dein nächster Befehl?} \\
            }
        }
    }
    & \\ \\

    \theutterance \stepcounter{utterance}  
    & & & \multicolumn{4}{p{0.6\linewidth}}{
        \cellcolor[rgb]{0.9,0.9,0.9}{
            \makecell[{{p{\linewidth}}}]{
                \texttt{\tiny{[GM$\langle$P2]}}
                \texttt{SPRICH: Wie läuft es? Brauchst du noch Zeit für deine Bewegungen?} \\
            }
        }
    }
     \\ \\

    \theutterance \stepcounter{utterance}  
    & & & \multicolumn{2}{p{0.3\linewidth}}{
        \cellcolor[rgb]{0.9,0.9,0.9}{
            \makecell[{{p{\linewidth}}}]{
                \texttt{\tiny{[GM$|$GM]}}
                \texttt{SPRICH: Wie läuft es? Brauchst du noch Zeit für deine Bewegungen?} \\
            }
        }
    }
    & & \\ \\

    \theutterance \stepcounter{utterance}  
    & & \multicolumn{4}{p{0.6\linewidth}}{
        \cellcolor[rgb]{0.9,0.9,0.9}{
            \makecell[{{p{\linewidth}}}]{
                \texttt{\tiny{[P1$\langle$GM]}}
                \texttt{'D' erfolgreich nach (3,3) verschoben. Dein Raster sieht jetzt so aus:} \\
\texttt{```} \\
\texttt{ 1234567} \\
\texttt{╔═══╤═╤═╗} \\
\texttt{║PSA│◌│◌║ 1} \\
\texttt{╟─┬─┼─┴─╢ 2} \\
\texttt{║◌│D│◌◌◌║ 3} \\
\texttt{║◌├─┤◌┌─╢ 4} \\
\texttt{╟─┘◌│◌│M║ 5} \\
\texttt{║◌◌◌└┬┴─╢ 6} \\
\texttt{║◌OT◌│◌◌║ 7} \\
\texttt{╚════╧══╝} \\
\texttt{```} \\
\texttt{Ihr seid gerade bei Runde 6 von maximal 28.} \\
\texttt{Ihr habt gemeinsam 2 von maximal 16 Strafpunkten angesammelt.} \\
\texttt{Dein Mitspieler hat dir diese Nachricht geschickt:} \\
\\ 
\\ 
\texttt{"Wie läuft es? Brauchst du noch Zeit für deine Bewegungen?"} \\
\\ 
\\ 
\texttt{Wie lautet dein nächster Befehl?} \\
            }
        }
    }
    & \\ \\

    \theutterance \stepcounter{utterance}  
    & \multicolumn{4}{p{0.6\linewidth}}{
        \cellcolor[rgb]{0.9,0.9,0.9}{
            \makecell[{{p{\linewidth}}}]{
                \texttt{\tiny{[P1$\rangle$GM]}}
                \texttt{VERSCHIEBE: O, (7, 5)} \\
            }
        }
    }
    & & \\ \\

    \theutterance \stepcounter{utterance}  
    & & & \multicolumn{2}{p{0.3\linewidth}}{
        \cellcolor[rgb]{0.9,0.9,0.9}{
            \makecell[{{p{\linewidth}}}]{
                \texttt{\tiny{[GM$|$GM]}}
                \texttt{VERSCHIEBE: O, (7, 5)} \\
            }
        }
    }
    & & \\ \\

    \theutterance \stepcounter{utterance}  
    & & & \multicolumn{2}{p{0.3\linewidth}}{
        \cellcolor[rgb]{0.9,0.9,0.9}{
            \makecell[{{p{\linewidth}}}]{
                \texttt{\tiny{[GM$|$GM]}}
                \texttt{Strafpunkt: (7,5) ist nicht leer, sondern enthält 'M'.} \\
\\ 
\texttt{Ihr habt gemeinsam 3 von maximal 16 Strafpunkten angesammelt.} \\
\texttt{Bitte versuche es noch mal!} \\
            }
        }
    }
    & & \\ \\

    \theutterance \stepcounter{utterance}  
    & & \multicolumn{4}{p{0.6\linewidth}}{
        \cellcolor[rgb]{0.9,0.9,0.9}{
            \makecell[{{p{\linewidth}}}]{
                \texttt{\tiny{[P1$\langle$GM]}}
                \texttt{Strafpunkt: (7,5) ist nicht leer, sondern enthält 'M'.} \\
\\ 
\texttt{Ihr habt gemeinsam 3 von maximal 16 Strafpunkten angesammelt.} \\
\texttt{Bitte versuche es noch mal!} \\
            }
        }
    }
    & \\ \\

    \theutterance \stepcounter{utterance}  
    & \multicolumn{4}{p{0.6\linewidth}}{
        \cellcolor[rgb]{0.9,0.9,0.9}{
            \makecell[{{p{\linewidth}}}]{
                \texttt{\tiny{[P1$\rangle$GM]}}
                \texttt{VERSCHIEBE: M, (7, 7)} \\
            }
        }
    }
    & & \\ \\

    \theutterance \stepcounter{utterance}  
    & & & \multicolumn{2}{p{0.3\linewidth}}{
        \cellcolor[rgb]{0.9,0.9,0.9}{
            \makecell[{{p{\linewidth}}}]{
                \texttt{\tiny{[GM$|$GM]}}
                \texttt{VERSCHIEBE: M, (7, 7)} \\
            }
        }
    }
    & & \\ \\

    \theutterance \stepcounter{utterance}  
    & & & \multicolumn{2}{p{0.3\linewidth}}{
        \cellcolor[rgb]{0.9,0.9,0.9}{
            \makecell[{{p{\linewidth}}}]{
                \texttt{\tiny{[GM$|$GM]}}
                \texttt{'M' erfolgreich nach (7,7) verschoben. Dein Raster sieht jetzt so aus:} \\
\texttt{```} \\
\texttt{ 1234567} \\
\texttt{╔═══╤═╤═╗} \\
\texttt{║PSA│◌│◌║ 1} \\
\texttt{╟─┬─┼─┴─╢ 2} \\
\texttt{║◌│D│◌◌◌║ 3} \\
\texttt{║◌├─┤◌┌─╢ 4} \\
\texttt{╟─┘◌│◌│◌║ 5} \\
\texttt{║◌◌◌└┬┴─╢ 6} \\
\texttt{║◌OT◌│◌M║ 7} \\
\texttt{╚════╧══╝} \\
\texttt{```} \\
            }
        }
    }
    & & \\ \\

    \theutterance \stepcounter{utterance}  
    & & \multicolumn{4}{p{0.6\linewidth}}{
        \cellcolor[rgb]{0.9,0.9,0.9}{
            \makecell[{{p{\linewidth}}}]{
                \texttt{\tiny{[GM$\rangle$P2]}}
                \texttt{Deine Nachricht wurde an deinen Mitspieler übermittelt.} \\
\\ 
\\ 
\texttt{Ihr seid gerade bei Runde 7 von maximal 28.} \\
\texttt{Ihr habt gemeinsam 3 von maximal 16 Strafpunkten angesammelt.} \\
\texttt{Dein Mitspieler hat ein Objekt auf seinem Raster bewegt.} \\
\\ 
\\ 
\texttt{Wie lautet dein nächster Befehl?} \\
            }
        }
    }
    & \\ \\

    \theutterance \stepcounter{utterance}  
    & & & \multicolumn{4}{p{0.6\linewidth}}{
        \cellcolor[rgb]{0.9,0.9,0.9}{
            \makecell[{{p{\linewidth}}}]{
                \texttt{\tiny{[GM$\langle$P2]}}
                \texttt{SPRICH: Alles klar, ich warte geduldig. Nimm dir die Zeit, die du brauchst!} \\
            }
        }
    }
     \\ \\

    \theutterance \stepcounter{utterance}  
    & & & \multicolumn{2}{p{0.3\linewidth}}{
        \cellcolor[rgb]{0.9,0.9,0.9}{
            \makecell[{{p{\linewidth}}}]{
                \texttt{\tiny{[GM$|$GM]}}
                \texttt{SPRICH: Alles klar, ich warte geduldig. Nimm dir die Zeit, die du brauchst!} \\
            }
        }
    }
    & & \\ \\

    \theutterance \stepcounter{utterance}  
    & & \multicolumn{4}{p{0.6\linewidth}}{
        \cellcolor[rgb]{0.9,0.9,0.9}{
            \makecell[{{p{\linewidth}}}]{
                \texttt{\tiny{[P1$\langle$GM]}}
                \texttt{'M' erfolgreich nach (7,7) verschoben. Dein Raster sieht jetzt so aus:} \\
\texttt{```} \\
\texttt{ 1234567} \\
\texttt{╔═══╤═╤═╗} \\
\texttt{║PSA│◌│◌║ 1} \\
\texttt{╟─┬─┼─┴─╢ 2} \\
\texttt{║◌│D│◌◌◌║ 3} \\
\texttt{║◌├─┤◌┌─╢ 4} \\
\texttt{╟─┘◌│◌│◌║ 5} \\
\texttt{║◌◌◌└┬┴─╢ 6} \\
\texttt{║◌OT◌│◌M║ 7} \\
\texttt{╚════╧══╝} \\
\texttt{```} \\
\texttt{Ihr seid gerade bei Runde 7 von maximal 28.} \\
\texttt{Ihr habt gemeinsam 3 von maximal 16 Strafpunkten angesammelt.} \\
\texttt{Dein Mitspieler hat dir diese Nachricht geschickt:} \\
\\ 
\\ 
\texttt{"Alles klar, ich warte geduldig. Nimm dir die Zeit, die du brauchst!"} \\
\\ 
\\ 
\texttt{Wie lautet dein nächster Befehl?} \\
            }
        }
    }
    & \\ \\

    \theutterance \stepcounter{utterance}  
    & \multicolumn{4}{p{0.6\linewidth}}{
        \cellcolor[rgb]{0.9,0.9,0.9}{
            \makecell[{{p{\linewidth}}}]{
                \texttt{\tiny{[P1$\rangle$GM]}}
                \texttt{VERSCHIEBE: O, (7, 5)} \\
            }
        }
    }
    & & \\ \\

    \theutterance \stepcounter{utterance}  
    & & & \multicolumn{2}{p{0.3\linewidth}}{
        \cellcolor[rgb]{0.9,0.9,0.9}{
            \makecell[{{p{\linewidth}}}]{
                \texttt{\tiny{[GM$|$GM]}}
                \texttt{VERSCHIEBE: O, (7, 5)} \\
            }
        }
    }
    & & \\ \\

    \theutterance \stepcounter{utterance}  
    & & & \multicolumn{2}{p{0.3\linewidth}}{
        \cellcolor[rgb]{0.9,0.9,0.9}{
            \makecell[{{p{\linewidth}}}]{
                \texttt{\tiny{[GM$|$GM]}}
                \texttt{'O' erfolgreich nach (7,5) verschoben. Dein Raster sieht jetzt so aus:} \\
\texttt{```} \\
\texttt{ 1234567} \\
\texttt{╔═══╤═╤═╗} \\
\texttt{║PSA│◌│◌║ 1} \\
\texttt{╟─┬─┼─┴─╢ 2} \\
\texttt{║◌│D│◌◌◌║ 3} \\
\texttt{║◌├─┤◌┌─╢ 4} \\
\texttt{╟─┘◌│◌│O║ 5} \\
\texttt{║◌◌◌└┬┴─╢ 6} \\
\texttt{║◌◌T◌│◌M║ 7} \\
\texttt{╚════╧══╝} \\
\texttt{```} \\
            }
        }
    }
    & & \\ \\

    \theutterance \stepcounter{utterance}  
    & & \multicolumn{4}{p{0.6\linewidth}}{
        \cellcolor[rgb]{0.9,0.9,0.9}{
            \makecell[{{p{\linewidth}}}]{
                \texttt{\tiny{[GM$\rangle$P2]}}
                \texttt{Deine Nachricht wurde an deinen Mitspieler übermittelt.} \\
\\ 
\\ 
\texttt{Ihr seid gerade bei Runde 8 von maximal 28.} \\
\texttt{Ihr habt gemeinsam 3 von maximal 16 Strafpunkten angesammelt.} \\
\texttt{Dein Mitspieler hat ein Objekt auf seinem Raster bewegt.} \\
\\ 
\\ 
\texttt{Wie lautet dein nächster Befehl?} \\
            }
        }
    }
    & \\ \\

    \theutterance \stepcounter{utterance}  
    & & & \multicolumn{4}{p{0.6\linewidth}}{
        \cellcolor[rgb]{0.9,0.9,0.9}{
            \makecell[{{p{\linewidth}}}]{
                \texttt{\tiny{[GM$\langle$P2]}}
                \texttt{SPRICH: Ich bin bereit, wenn du soweit bist. Melde dich einfach!} \\
            }
        }
    }
     \\ \\

    \theutterance \stepcounter{utterance}  
    & & & \multicolumn{2}{p{0.3\linewidth}}{
        \cellcolor[rgb]{0.9,0.9,0.9}{
            \makecell[{{p{\linewidth}}}]{
                \texttt{\tiny{[GM$|$GM]}}
                \texttt{SPRICH: Ich bin bereit, wenn du soweit bist. Melde dich einfach!} \\
            }
        }
    }
    & & \\ \\

    \theutterance \stepcounter{utterance}  
    & & \multicolumn{4}{p{0.6\linewidth}}{
        \cellcolor[rgb]{0.9,0.9,0.9}{
            \makecell[{{p{\linewidth}}}]{
                \texttt{\tiny{[P1$\langle$GM]}}
                \texttt{'O' erfolgreich nach (7,5) verschoben. Dein Raster sieht jetzt so aus:} \\
\texttt{```} \\
\texttt{ 1234567} \\
\texttt{╔═══╤═╤═╗} \\
\texttt{║PSA│◌│◌║ 1} \\
\texttt{╟─┬─┼─┴─╢ 2} \\
\texttt{║◌│D│◌◌◌║ 3} \\
\texttt{║◌├─┤◌┌─╢ 4} \\
\texttt{╟─┘◌│◌│O║ 5} \\
\texttt{║◌◌◌└┬┴─╢ 6} \\
\texttt{║◌◌T◌│◌M║ 7} \\
\texttt{╚════╧══╝} \\
\texttt{```} \\
\texttt{Ihr seid gerade bei Runde 8 von maximal 28.} \\
\texttt{Ihr habt gemeinsam 3 von maximal 16 Strafpunkten angesammelt.} \\
\texttt{Dein Mitspieler hat dir diese Nachricht geschickt:} \\
\\ 
\\ 
\texttt{"Ich bin bereit, wenn du soweit bist. Melde dich einfach!"} \\
\\ 
\\ 
\texttt{Wie lautet dein nächster Befehl?} \\
            }
        }
    }
    & \\ \\

    \theutterance \stepcounter{utterance}  
    & \multicolumn{4}{p{0.6\linewidth}}{
        \cellcolor[rgb]{0.9,0.9,0.9}{
            \makecell[{{p{\linewidth}}}]{
                \texttt{\tiny{[P1$\rangle$GM]}}
                \texttt{VERSCHIEBE: T, (4, 7)} \\
            }
        }
    }
    & & \\ \\

    \theutterance \stepcounter{utterance}  
    & & & \multicolumn{2}{p{0.3\linewidth}}{
        \cellcolor[rgb]{0.9,0.9,0.9}{
            \makecell[{{p{\linewidth}}}]{
                \texttt{\tiny{[GM$|$GM]}}
                \texttt{VERSCHIEBE: T, (4, 7)} \\
            }
        }
    }
    & & \\ \\

    \theutterance \stepcounter{utterance}  
    & & & \multicolumn{2}{p{0.3\linewidth}}{
        \cellcolor[rgb]{0.9,0.9,0.9}{
            \makecell[{{p{\linewidth}}}]{
                \texttt{\tiny{[GM$|$GM]}}
                \texttt{'T' erfolgreich nach (4,7) verschoben. Dein Raster sieht jetzt so aus:} \\
\texttt{```} \\
\texttt{ 1234567} \\
\texttt{╔═══╤═╤═╗} \\
\texttt{║PSA│◌│◌║ 1} \\
\texttt{╟─┬─┼─┴─╢ 2} \\
\texttt{║◌│D│◌◌◌║ 3} \\
\texttt{║◌├─┤◌┌─╢ 4} \\
\texttt{╟─┘◌│◌│O║ 5} \\
\texttt{║◌◌◌└┬┴─╢ 6} \\
\texttt{║◌◌◌T│◌M║ 7} \\
\texttt{╚════╧══╝} \\
\texttt{```} \\
            }
        }
    }
    & & \\ \\

    \theutterance \stepcounter{utterance}  
    & & \multicolumn{4}{p{0.6\linewidth}}{
        \cellcolor[rgb]{0.9,0.9,0.9}{
            \makecell[{{p{\linewidth}}}]{
                \texttt{\tiny{[GM$\rangle$P2]}}
                \texttt{Deine Nachricht wurde an deinen Mitspieler übermittelt.} \\
\\ 
\\ 
\texttt{Ihr seid gerade bei Runde 9 von maximal 28.} \\
\texttt{Ihr habt gemeinsam 3 von maximal 16 Strafpunkten angesammelt.} \\
\texttt{Dein Mitspieler hat ein Objekt auf seinem Raster bewegt.} \\
\\ 
\\ 
\texttt{Wie lautet dein nächster Befehl?} \\
            }
        }
    }
    & \\ \\

    \theutterance \stepcounter{utterance}  
    & & & \multicolumn{4}{p{0.6\linewidth}}{
        \cellcolor[rgb]{0.9,0.9,0.9}{
            \makecell[{{p{\linewidth}}}]{
                \texttt{\tiny{[GM$\langle$P2]}}
                \texttt{SPRICH: fertig?} \\
            }
        }
    }
     \\ \\

    \theutterance \stepcounter{utterance}  
    & & & \multicolumn{2}{p{0.3\linewidth}}{
        \cellcolor[rgb]{0.9,0.9,0.9}{
            \makecell[{{p{\linewidth}}}]{
                \texttt{\tiny{[GM$|$GM]}}
                \texttt{SPRICH: fertig?} \\
            }
        }
    }
    & & \\ \\

    \theutterance \stepcounter{utterance}  
    & & \multicolumn{4}{p{0.6\linewidth}}{
        \cellcolor[rgb]{0.9,0.9,0.9}{
            \makecell[{{p{\linewidth}}}]{
                \texttt{\tiny{[P1$\langle$GM]}}
                \texttt{'T' erfolgreich nach (4,7) verschoben. Dein Raster sieht jetzt so aus:} \\
\texttt{```} \\
\texttt{ 1234567} \\
\texttt{╔═══╤═╤═╗} \\
\texttt{║PSA│◌│◌║ 1} \\
\texttt{╟─┬─┼─┴─╢ 2} \\
\texttt{║◌│D│◌◌◌║ 3} \\
\texttt{║◌├─┤◌┌─╢ 4} \\
\texttt{╟─┘◌│◌│O║ 5} \\
\texttt{║◌◌◌└┬┴─╢ 6} \\
\texttt{║◌◌◌T│◌M║ 7} \\
\texttt{╚════╧══╝} \\
\texttt{```} \\
\texttt{Ihr seid gerade bei Runde 9 von maximal 28.} \\
\texttt{Ihr habt gemeinsam 3 von maximal 16 Strafpunkten angesammelt.} \\
\texttt{Dein Mitspieler hat dir diese Nachricht geschickt:} \\
\\ 
\\ 
\texttt{"fertig?"} \\
\\ 
\\ 
\texttt{Wie lautet dein nächster Befehl?} \\
            }
        }
    }
    & \\ \\

    \theutterance \stepcounter{utterance}  
    & \multicolumn{4}{p{0.6\linewidth}}{
        \cellcolor[rgb]{0.9,0.9,0.9}{
            \makecell[{{p{\linewidth}}}]{
                \texttt{\tiny{[P1$\rangle$GM]}}
                \texttt{SPRICH: fertig!} \\
            }
        }
    }
    & & \\ \\

    \theutterance \stepcounter{utterance}  
    & & & \multicolumn{2}{p{0.3\linewidth}}{
        \cellcolor[rgb]{0.9,0.9,0.9}{
            \makecell[{{p{\linewidth}}}]{
                \texttt{\tiny{[GM$|$GM]}}
                \texttt{SPRICH: fertig!} \\
            }
        }
    }
    & & \\ \\

    \theutterance \stepcounter{utterance}  
    & & & \multicolumn{2}{p{0.3\linewidth}}{
        \cellcolor[rgb]{0.9,0.9,0.9}{
            \makecell[{{p{\linewidth}}}]{
                \texttt{\tiny{[GM$|$GM]}}
                \texttt{true} \\
            }
        }
    }
    & & \\ \\

    \theutterance \stepcounter{utterance}  
    & & & \multicolumn{2}{p{0.3\linewidth}}{
        \cellcolor[rgb]{0.9,0.9,0.9}{
            \makecell[{{p{\linewidth}}}]{
                \texttt{\tiny{[GM$|$GM]}}
                \texttt{* success: True} \\
\texttt{* lose: False} \\
\texttt{* aborted: False} \\
\texttt{{-}{-}{-}{-}{-}{-}{-}} \\
\texttt{* Shifts: 6.00} \\
\texttt{* Max Shifts: 12.00} \\
\texttt{* Min Shifts: 6.00} \\
\texttt{* End Distance Sum: 0.00} \\
\texttt{* Init Distance Sum: 23.62} \\
\texttt{* Expected Distance Sum: 29.33} \\
\texttt{* Penalties: 3.00} \\
\texttt{* Max Penalties: 16.00} \\
\texttt{* Rounds: 9.00} \\
\texttt{* Max Rounds: 28.00} \\
\texttt{* Object Count: 7.00} \\
            }
        }
    }
    & & \\ \\

    \theutterance \stepcounter{utterance}  
    & & & \multicolumn{2}{p{0.3\linewidth}}{
        \cellcolor[rgb]{0.9,0.9,0.9}{
            \makecell[{{p{\linewidth}}}]{
                \texttt{\tiny{[GM$|$GM]}}
                \texttt{Initial Grids:} \\
\texttt{```} \\
\texttt{Player 1:     Player 2:} \\
\texttt{ 1234567       1234567} \\
\texttt{╔═══╤═╤═╗     ╔═══╤═╤═╗} \\
\texttt{║◌S◌│◌│A║ 1   ║P◌A│◌│◌║ 1} \\
\texttt{╟─┬─┼─┴─╢ 2   ╟─┬─┼─┴─╢ 2} \\
\texttt{║◌│◌│◌◌◌║ 3   ║◌│D│◌◌◌║ 3} \\
\texttt{║P├─┤D┌─╢ 4   ║◌├─┤◌┌─╢ 4} \\
\texttt{╟─┘◌│◌│M║ 5   ╟─┘◌│◌│O║ 5} \\
\texttt{║◌◌◌└┬┴─╢ 6   ║◌◌◌└┬┴─╢ 6} \\
\texttt{║◌OT◌│◌◌║ 7   ║◌S◌T│◌M║ 7} \\
\texttt{╚════╧══╝     ╚════╧══╝} \\
\texttt{```} \\
\\ 
\texttt{Final Grids:} \\
\texttt{```} \\
\texttt{Player 1:     Player 2:} \\
\texttt{ 1234567       1234567} \\
\texttt{╔═══╤═╤═╗     ╔═══╤═╤═╗} \\
\texttt{║PSA│◌│◌║ 1   ║PSA│◌│◌║ 1} \\
\texttt{╟─┬─┼─┴─╢ 2   ╟─┬─┼─┴─╢ 2} \\
\texttt{║◌│D│◌◌◌║ 3   ║◌│D│◌◌◌║ 3} \\
\texttt{║◌├─┤◌┌─╢ 4   ║◌├─┤◌┌─╢ 4} \\
\texttt{╟─┘◌│◌│O║ 5   ╟─┘◌│◌│O║ 5} \\
\texttt{║◌◌◌└┬┴─╢ 6   ║◌◌◌└┬┴─╢ 6} \\
\texttt{║◌◌◌T│◌M║ 7   ║◌◌◌T│◌M║ 7} \\
\texttt{╚════╧══╝     ╚════╧══╝} \\
\texttt{```} \\
            }
        }
    }
    & & \\ \\

\end{supertabular}
}

\end{document}
