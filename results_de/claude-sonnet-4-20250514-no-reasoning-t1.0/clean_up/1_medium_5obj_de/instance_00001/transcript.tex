
\documentclass{article}
\usepackage{colortbl}
\usepackage{makecell}
\usepackage{multirow}
\usepackage{supertabular}

\begin{document}

\newcounter{utterance}

\centering \large Interaction Transcript for game `clean\_up', experiment `1\_medium\_5obj\_de', episode 1 with claude{-}sonnet{-}4{-}20250514{-}no{-}reasoning{-}t1.0.
\vspace{24pt}

{ \footnotesize  \setcounter{utterance}{1}
\setlength{\tabcolsep}{0pt}
\begin{supertabular}{c@{$\;$}|p{.15\linewidth}@{}p{.15\linewidth}p{.15\linewidth}p{.15\linewidth}p{.15\linewidth}p{.15\linewidth}}
    \# & $\;$A & \multicolumn{4}{c}{Game Master} & $\;\:$B\\
    \hline

    \theutterance \stepcounter{utterance}  
    & & \multicolumn{4}{p{0.6\linewidth}}{
        \cellcolor[rgb]{0.9,0.9,0.9}{
            \makecell[{{p{\linewidth}}}]{
                \texttt{\tiny{[P1$\langle$GM]}}
                \texttt{Ich bin der Spielleiter, und du spielst ein kooperatives Spiel mit dem folgenden Raster als Spielbrett:} \\
\\ 
\texttt{```} \\
\texttt{ 1234567} \\
\texttt{╔═══╤═╤═╗} \\
\texttt{║C◌◌│◌│◌║ 1} \\
\texttt{║◌I◌│◌├─╢ 2} \\
\texttt{║◌◌◌├─┘◌║ 3} \\
\texttt{║◌◌◌│◌◌◌║ 4} \\
\texttt{╟─┬─┼───╢ 5} \\
\texttt{║◌│◌│◌WT║ 6} \\
\texttt{║◌│◌│◌◌H║ 7} \\
\texttt{╚═╧═╧═══╝} \\
\\ 
\texttt{```} \\
\\ 
\texttt{* Am oberen Rand sind die x{-}Koordinaten angegeben, die nach rechts hin zunehmen, und am rechten Rand die y{-}Koordinaten, die nach unten zunehmen.} \\
\texttt{* Die folgenden Objekte sind zufällig auf deinem Raster platziert: 'W', 'I', 'T', 'C', 'H'.} \\
\\ 
\\ 
\texttt{Dein Mitspieler sieht eine Variation des Spielbretts, auf dem dieselben Objekte an anderen zufälligen Positionen platziert sind. Du kannst das Spielbrett des anderen Spielers nicht sehen, und er kann deins nicht sehen.} \\
\\ 
\\ 
\texttt{**Ziel des Spiels:**} \\
\\ 
\texttt{Beide Spieler müssen die Objekte auf ihrem jeweiligen Spielbrett so bewegen, dass identische Objekte am Ende an denselben Koordinaten liegen. Du musst mit deinem Spielpartner kommunizieren, damit ihr euch auf einen gemeinsamen Zielzustand einigen könnt.} \\
\\ 
\\ 
\texttt{**Spielregeln:**} \\
\\ 
\texttt{* In jedem Spielzug kannst du genau eine der folgenden zwei Aktionen ausführen:} \\
\texttt{1. `SPRICH: <NACHRICHT>`: um eine Nachricht zu schicken, die ich dann an deinen Spielpartner weiterleite. Als Nachricht gilt alles bis zum nächsten Zeilenumbruch.} \\
\texttt{2. `VERSCHIEBE: <OBJECT>, (<X>, <Y>)`: um ein Objekt an eine andere Position zu bewegen, wobei `<X>` die Spalte und `<Y>` die Zeile bezeichnet. Ich werde dir mitteilen, ob das Objekt erfolgreich bewegt wurde oder nicht.} \\
\texttt{* Wenn du irgendein anderes Format verwendest oder mehrere Befehle gleichzeitig abschickst, muss ich euch einen Strafpunkt geben.} \\
\texttt{* Wenn ihr gemeinsam mehr als 12 Strafpunkte ansammelt, verliert ihr das Spiel.} \\
\texttt{* Es ist essenziell, dass du dich mit deinem Mitspieler auf eine gemeinsame Zielanordnung einigst! Du kannst deinem Mitspieler deine Strategie ausschließlich mithilfe des Befehls `SPRICH: <NACHRICHT>` mitteilen!} \\
\\ 
\\ 
\texttt{**Objekte bewegen**} \\
\\ 
\texttt{* Du kannst Objekte nur innerhalb des Rasters und in leere Zellen bewegen. Leere Zellen erkennst am Zeichen '◌'.} \\
\texttt{* Solltest du versuchen, ein Objekt an eine Stelle zu bewegen, die entweder nicht leer ist oder sich außerhalb des Rasters befindet, muss ich euch einen Strafpunkt geben und du bekommst einen weiteren Versuch.} \\
\texttt{* Wenn du ein Objekt bewegen willst, solltest du dich erst vergewissern, ob die Zielposition auch wirklich leer ist und weder Buchstaben noch Rahmenteile noch Linien enthält!} \\
\\ 
\\ 
\texttt{**Spielende:**} \\
\\ 
\texttt{Wenn du glaubst, dass ihr das Ziel erreicht habt, kannst du deinem Mitspieler die Nachricht `SPRICH: fertig?` schicken. Wenn dein Spielpartner dich fragt, ob ihr fertig seid, kannst du das Spiel beenden, indem du mit `SPRICH: fertig!` antwortest.} \\
\\ 
\texttt{Ihr gewinnt gemeinsam, wenn ihr das Spiel innerhalb von \#max\_rounds Runden beendet. Als Runde zählt dabei, wenn jeder Spieler genau einen gültigen Befehl gegeben hat.} \\
\\ 
\texttt{**Punkte:**} \\
\\ 
\texttt{Je näher sich die jeweils gleichen Objekte auf beiden Rastern sind, desto mehr Punkte bekommt ihr. Für Strafpunkte gibt es Punktabzug. Könnt ihr den Rekord brechen?} \\
\\ 
\texttt{Schicke eine Nachricht an deinen Mitspieler um das Spiel zu starten!} \\
            }
        }
    }
    & \\ \\

    \theutterance \stepcounter{utterance}  
    & \multicolumn{4}{p{0.6\linewidth}}{
        \cellcolor[rgb]{0.9,0.9,0.9}{
            \makecell[{{p{\linewidth}}}]{
                \texttt{\tiny{[P1$\rangle$GM]}}
                \texttt{Ich schaue mir zunächst mein Spielbrett genau an:} \\
\\ 
\texttt{```} \\
\texttt{ 1234567} \\
\texttt{╔═══╤═╤═╗} \\
\texttt{║C◌◌│◌│◌║ 1} \\
\texttt{║◌I◌│◌├─╢ 2} \\
\texttt{║◌◌◌├─┘◌║ 3} \\
\texttt{║◌◌◌│◌◌◌║ 4} \\
\texttt{╟─┬─┼───╢ 5} \\
\texttt{║◌│◌│◌WT║ 6} \\
\texttt{║◌│◌│◌◌H║ 7} \\
\texttt{╚═╧═╧═══╝} \\
\texttt{```} \\
\\ 
\texttt{Meine Objekte befinden sich an folgenden Positionen:} \\
\texttt{{-} C bei (1,1)} \\
\texttt{{-} I bei (2,2)  } \\
\texttt{{-} W bei (6,6)} \\
\texttt{{-} T bei (7,6)} \\
\texttt{{-} H bei (7,7)} \\
\\ 
\texttt{Jetzt starte ich die Kommunikation mit meinem Mitspieler:} \\
\\ 
\texttt{SPRICH: Hallo! Lass uns koordinieren. Meine Objekte sind: C bei (1,1), I bei (2,2), W bei (6,} \\
            }
        }
    }
    & & \\ \\

    \theutterance \stepcounter{utterance}  
    & & & \multicolumn{2}{p{0.3\linewidth}}{
        \cellcolor[rgb]{0.9,0.9,0.9}{
            \makecell[{{p{\linewidth}}}]{
                \texttt{\tiny{[GM$|$GM]}}
                \texttt{Ich schaue mir zunächst mein Spielbrett genau an:} \\
\\ 
\texttt{```} \\
\texttt{ 1234567} \\
\texttt{╔═══╤═╤═╗} \\
\texttt{║C◌◌│◌│◌║ 1} \\
\texttt{║◌I◌│◌├─╢ 2} \\
\texttt{║◌◌◌├─┘◌║ 3} \\
\texttt{║◌◌◌│◌◌◌║ 4} \\
\texttt{╟─┬─┼───╢ 5} \\
\texttt{║◌│◌│◌WT║ 6} \\
\texttt{║◌│◌│◌◌H║ 7} \\
\texttt{╚═╧═╧═══╝} \\
\texttt{```} \\
\\ 
\texttt{Meine Objekte befinden sich an folgenden Positionen:} \\
\texttt{{-} C bei (1,1)} \\
\texttt{{-} I bei (2,2)  } \\
\texttt{{-} W bei (6,6)} \\
\texttt{{-} T bei (7,6)} \\
\texttt{{-} H bei (7,7)} \\
\\ 
\texttt{Jetzt starte ich die Kommunikation mit meinem Mitspieler:} \\
\\ 
\texttt{SPRICH: Hallo! Lass uns koordinieren. Meine Objekte sind: C bei (1,1), I bei (2,2), W bei (6,} \\
            }
        }
    }
    & & \\ \\

    \theutterance \stepcounter{utterance}  
    & & & \multicolumn{2}{p{0.3\linewidth}}{
        \cellcolor[rgb]{0.9,0.9,0.9}{
            \makecell[{{p{\linewidth}}}]{
                \texttt{\tiny{[GM$|$GM]}}
                \texttt{Invalid format: head is not empty: 'Ich schaue mir zunächst mein Spielbrett genau an:} \\
\\ 
\\ 
\texttt{ 1234567} \\
\texttt{╔═══╤═╤═╗} \\
\texttt{║C◌◌│◌│◌║ 1} \\
\texttt{║◌I◌│◌├─╢ 2} \\
\texttt{║◌◌◌├─┘◌║ 3} \\
\texttt{║◌◌◌│◌◌◌║ 4} \\
\texttt{╟─┬─┼───╢ 5} \\
\texttt{║◌│◌│◌WT║ 6} \\
\texttt{║◌│◌│◌◌H║ 7} \\
\texttt{╚═╧═╧═══╝} \\
\\ 
\\ 
\texttt{Meine Objekte befinden sich an folgenden Positionen:} \\
\texttt{{-} C bei (1,1)} \\
\texttt{{-} I bei (2,2)  } \\
\texttt{{-} W bei (6,6)} \\
\texttt{{-} T bei (7,6)} \\
\texttt{{-} H bei (7,7)} \\
\\ 
\texttt{Jetzt starte ich die Kommunikation mit meinem Mitspieler:} \\
\\ 
\texttt{'} \\
            }
        }
    }
    & & \\ \\

    \theutterance \stepcounter{utterance}  
    & & \multicolumn{4}{p{0.6\linewidth}}{
        \cellcolor[rgb]{0.9,0.9,0.9}{
            \makecell[{{p{\linewidth}}}]{
                \texttt{\tiny{[P1$\langle$GM]}}
                \texttt{Strafpunkt: Deine Nachricht darf nichts vor dem Befehl enthalten! Stelle sicher, dass deine Antwort ausschließlich einen der beiden Befehle `SPRICH: <NACHRICHT>` und `VERSCHIEBE: <OBJECT>, (<X>, <Y>)` und sonst nichts enthält!} \\
\\ 
\texttt{Ihr habt gemeinsam 1 von maximal 12 Strafpunkten angesammelt.} \\
\texttt{Bitte versuche es noch mal!} \\
            }
        }
    }
    & \\ \\

    \theutterance \stepcounter{utterance}  
    & \multicolumn{4}{p{0.6\linewidth}}{
        \cellcolor[rgb]{0.9,0.9,0.9}{
            \makecell[{{p{\linewidth}}}]{
                \texttt{\tiny{[P1$\rangle$GM]}}
                \texttt{SPRICH: Hallo! Lass uns koordinieren. Meine Objekte sind: C bei (1,1), I bei (2,2), W bei (6,6), T bei (7,6), H bei (7,7). Wo sind deine Objekte? Wir sollten uns auf Zielpositionen einigen!} \\
            }
        }
    }
    & & \\ \\

    \theutterance \stepcounter{utterance}  
    & & & \multicolumn{2}{p{0.3\linewidth}}{
        \cellcolor[rgb]{0.9,0.9,0.9}{
            \makecell[{{p{\linewidth}}}]{
                \texttt{\tiny{[GM$|$GM]}}
                \texttt{SPRICH: Hallo! Lass uns koordinieren. Meine Objekte sind: C bei (1,1), I bei (2,2), W bei (6,6), T bei (7,6), H bei (7,7). Wo sind deine Objekte? Wir sollten uns auf Zielpositionen einigen!} \\
            }
        }
    }
    & & \\ \\

    \theutterance \stepcounter{utterance}  
    & & \multicolumn{4}{p{0.6\linewidth}}{
        \cellcolor[rgb]{0.9,0.9,0.9}{
            \makecell[{{p{\linewidth}}}]{
                \texttt{\tiny{[GM$\rangle$P2]}}
                \texttt{Ich bin der Spielleiter, und du spielst ein kooperatives Spiel mit dem folgenden Raster als Spielbrett:} \\
\\ 
\texttt{```} \\
\texttt{ 1234567} \\
\texttt{╔═══╤═╤═╗} \\
\texttt{║◌◌◌│C│◌║ 1} \\
\texttt{║◌W◌│◌├─╢ 2} \\
\texttt{║◌◌◌├─┘◌║ 3} \\
\texttt{║◌◌◌│◌◌H║ 4} \\
\texttt{╟─┬─┼───╢ 5} \\
\texttt{║T│◌│I◌◌║ 6} \\
\texttt{║◌│◌│◌◌◌║ 7} \\
\texttt{╚═╧═╧═══╝} \\
\\ 
\texttt{```} \\
\\ 
\texttt{* Am oberen Rand sind die x{-}Koordinaten angegeben, die nach rechts hin zunehmen, und am rechten Rand die y{-}Koordinaten, die nach unten zunehmen.} \\
\texttt{* Die folgenden Objekte sind zufällig auf deinem Raster platziert: 'W', 'I', 'T', 'C', 'H'.} \\
\\ 
\\ 
\texttt{Dein Mitspieler sieht eine Variation des Spielbretts, auf dem dieselben Objekte an anderen zufälligen Positionen platziert sind. Du kannst das Spielbrett des anderen Spielers nicht sehen, und er kann deins nicht sehen.} \\
\\ 
\\ 
\texttt{**Ziel des Spiels:**} \\
\\ 
\texttt{Beide Spieler müssen die Objekte auf ihrem jeweiligen Spielbrett so bewegen, dass identische Objekte am Ende an denselben Koordinaten liegen. Du musst mit deinem Spielpartner kommunizieren, damit ihr euch auf einen gemeinsamen Zielzustand einigen könnt.} \\
\\ 
\\ 
\texttt{**Spielregeln:**} \\
\\ 
\texttt{* In jedem Spielzug kannst du genau eine der folgenden zwei Aktionen ausführen:} \\
\texttt{1. `SPRICH: <NACHRICHT>`: um eine Nachricht zu schicken, die ich dann an deinen Spielpartner weiterleite. Als Nachricht gilt alles bis zum nächsten Zeilenumbruch.} \\
\texttt{2. `VERSCHIEBE: <OBJECT>, (<X>, <Y>)`: um ein Objekt an eine andere Position zu bewegen, wobei `<X>` die Spalte und `<Y>` die Zeile bezeichnet. Ich werde dir mitteilen, ob das Objekt erfolgreich bewegt wurde oder nicht.} \\
\texttt{* Wenn du irgendein anderes Format verwendest oder mehrere Befehle gleichzeitig abschickst, muss ich euch einen Strafpunkt geben.} \\
\texttt{* Wenn ihr gemeinsam mehr als 12 Strafpunkte ansammelt, verliert ihr das Spiel.} \\
\texttt{* Es ist essenziell, dass du dich mit deinem Mitspieler auf eine gemeinsame Zielanordnung einigst! Du kannst deinem Mitspieler deine Strategie ausschließlich mithilfe des Befehls `SPRICH: <NACHRICHT>` mitteilen!} \\
\\ 
\\ 
\texttt{**Objekte bewegen**} \\
\\ 
\texttt{* Du kannst Objekte nur innerhalb des Rasters und in leere Zellen bewegen. Leere Zellen erkennst am Zeichen '◌'.} \\
\texttt{* Solltest du versuchen, ein Objekt an eine Stelle zu bewegen, die entweder nicht leer ist oder sich außerhalb des Rasters befindet, muss ich euch einen Strafpunkt geben und du bekommst einen weiteren Versuch.} \\
\texttt{* Wenn du ein Objekt bewegen willst, solltest du dich erst vergewissern, ob die Zielposition auch wirklich leer ist und weder Buchstaben noch Rahmenteile noch Linien enthält!} \\
\\ 
\\ 
\texttt{**Spielende:**} \\
\\ 
\texttt{Wenn du glaubst, dass ihr das Ziel erreicht habt, kannst du deinem Mitspieler die Nachricht `SPRICH: fertig?` schicken. Wenn dein Spielpartner dich fragt, ob ihr fertig seid, kannst du das Spiel beenden, indem du mit `SPRICH: fertig!` antwortest.} \\
\\ 
\texttt{Ihr gewinnt gemeinsam, wenn ihr das Spiel innerhalb von \#max\_rounds Runden beendet. Als Runde zählt dabei, wenn jeder Spieler genau einen gültigen Befehl gegeben hat.} \\
\\ 
\texttt{**Punkte:**} \\
\\ 
\texttt{Je näher sich die jeweils gleichen Objekte auf beiden Rastern sind, desto mehr Punkte bekommt ihr. Für Strafpunkte gibt es Punktabzug. Könnt ihr den Rekord brechen?} \\
\\ 
\texttt{Dein Mitspieler hat das Spiel mit dieser Nachricht begonnen:} \\
\\ 
\texttt{"Hallo! Lass uns koordinieren. Meine Objekte sind: C bei (1,1), I bei (2,2), W bei (6,6), T bei (7,6), H bei (7,7). Wo sind deine Objekte? Wir sollten uns auf Zielpositionen einigen!"} \\
\\ 
\texttt{Wie lautet dein erster Befehl?} \\
            }
        }
    }
    & \\ \\

    \theutterance \stepcounter{utterance}  
    & & & \multicolumn{4}{p{0.6\linewidth}}{
        \cellcolor[rgb]{0.9,0.9,0.9}{
            \makecell[{{p{\linewidth}}}]{
                \texttt{\tiny{[GM$\langle$P2]}}
                \texttt{SPRICH: Hallo! Meine Objekte sind: W bei (2,2), C bei (4,1), H bei (7,4), T bei (1,6), I bei (4,6). Lass uns die Zielpositionen festlegen. Ich schlage vor: C bei (1,1), W bei (2,2), H bei (3,3), I bei (4,4), T bei (5,5) {-} das wäre eine schöne Diagonale und alle Positionen sind auf beiden Brettern erreichbar. Was denkst du?} \\
            }
        }
    }
     \\ \\

    \theutterance \stepcounter{utterance}  
    & & & \multicolumn{2}{p{0.3\linewidth}}{
        \cellcolor[rgb]{0.9,0.9,0.9}{
            \makecell[{{p{\linewidth}}}]{
                \texttt{\tiny{[GM$|$GM]}}
                \texttt{SPRICH: Hallo! Meine Objekte sind: W bei (2,2), C bei (4,1), H bei (7,4), T bei (1,6), I bei (4,6). Lass uns die Zielpositionen festlegen. Ich schlage vor: C bei (1,1), W bei (2,2), H bei (3,3), I bei (4,4), T bei (5,5) {-} das wäre eine schöne Diagonale und alle Positionen sind auf beiden Brettern erreichbar. Was denkst du?} \\
            }
        }
    }
    & & \\ \\

    \theutterance \stepcounter{utterance}  
    & & \multicolumn{4}{p{0.6\linewidth}}{
        \cellcolor[rgb]{0.9,0.9,0.9}{
            \makecell[{{p{\linewidth}}}]{
                \texttt{\tiny{[P1$\langle$GM]}}
                \texttt{Deine Nachricht wurde an deinen Mitspieler übermittelt.} \\
\\ 
\\ 
\texttt{Ihr seid gerade bei Runde 1 von maximal 20.} \\
\texttt{Ihr habt gemeinsam 1 von maximal 12 Strafpunkten angesammelt.} \\
\texttt{Dein Mitspieler hat dir diese Nachricht geschickt:} \\
\\ 
\\ 
\texttt{"Hallo! Meine Objekte sind: W bei (2,2), C bei (4,1), H bei (7,4), T bei (1,6), I bei (4,6). Lass uns die Zielpositionen festlegen. Ich schlage vor: C bei (1,1), W bei (2,2), H bei (3,3), I bei (4,4), T bei (5,5) {-} das wäre eine schöne Diagonale und alle Positionen sind auf beiden Brettern erreichbar. Was denkst du?"} \\
\\ 
\\ 
\texttt{Wie lautet dein nächster Befehl?} \\
            }
        }
    }
    & \\ \\

    \theutterance \stepcounter{utterance}  
    & \multicolumn{4}{p{0.6\linewidth}}{
        \cellcolor[rgb]{0.9,0.9,0.9}{
            \makecell[{{p{\linewidth}}}]{
                \texttt{\tiny{[P1$\rangle$GM]}}
                \texttt{SPRICH: Gute Idee mit der Diagonale! Ich checke die Positionen: (1,1), (2,2), (3,3), (4,4), (5,5) sind alle frei auf meinem Brett. Das funktioniert perfekt! Ich stimme zu: C bei (1,1), W bei (2,2), H bei (3,3), I bei (4,4), T bei (5,5). Ich fange an und bewege W von (6,6) zu (2,2).} \\
            }
        }
    }
    & & \\ \\

    \theutterance \stepcounter{utterance}  
    & & & \multicolumn{2}{p{0.3\linewidth}}{
        \cellcolor[rgb]{0.9,0.9,0.9}{
            \makecell[{{p{\linewidth}}}]{
                \texttt{\tiny{[GM$|$GM]}}
                \texttt{SPRICH: Gute Idee mit der Diagonale! Ich checke die Positionen: (1,1), (2,2), (3,3), (4,4), (5,5) sind alle frei auf meinem Brett. Das funktioniert perfekt! Ich stimme zu: C bei (1,1), W bei (2,2), H bei (3,3), I bei (4,4), T bei (5,5). Ich fange an und bewege W von (6,6) zu (2,2).} \\
            }
        }
    }
    & & \\ \\

    \theutterance \stepcounter{utterance}  
    & & \multicolumn{4}{p{0.6\linewidth}}{
        \cellcolor[rgb]{0.9,0.9,0.9}{
            \makecell[{{p{\linewidth}}}]{
                \texttt{\tiny{[GM$\rangle$P2]}}
                \texttt{Deine Nachricht wurde an deinen Mitspieler übermittelt.} \\
\\ 
\\ 
\texttt{Ihr seid gerade bei Runde 2 von maximal 20.} \\
\texttt{Ihr habt gemeinsam 1 von maximal 12 Strafpunkten angesammelt.} \\
\texttt{Dein Mitspieler hat dir diese Nachricht geschickt:} \\
\\ 
\\ 
\texttt{"Gute Idee mit der Diagonale! Ich checke die Positionen: (1,1), (2,2), (3,3), (4,4), (5,5) sind alle frei auf meinem Brett. Das funktioniert perfekt! Ich stimme zu: C bei (1,1), W bei (2,2), H bei (3,3), I bei (4,4), T bei (5,5). Ich fange an und bewege W von (6,6) zu (2,2)."} \\
\\ 
\\ 
\texttt{Wie lautet dein nächster Befehl?} \\
            }
        }
    }
    & \\ \\

    \theutterance \stepcounter{utterance}  
    & & & \multicolumn{4}{p{0.6\linewidth}}{
        \cellcolor[rgb]{0.9,0.9,0.9}{
            \makecell[{{p{\linewidth}}}]{
                \texttt{\tiny{[GM$\langle$P2]}}
                \texttt{VERSCHIEBE: W, (2, 2)} \\
            }
        }
    }
     \\ \\

    \theutterance \stepcounter{utterance}  
    & & & \multicolumn{2}{p{0.3\linewidth}}{
        \cellcolor[rgb]{0.9,0.9,0.9}{
            \makecell[{{p{\linewidth}}}]{
                \texttt{\tiny{[GM$|$GM]}}
                \texttt{VERSCHIEBE: W, (2, 2)} \\
            }
        }
    }
    & & \\ \\

    \theutterance \stepcounter{utterance}  
    & & & \multicolumn{2}{p{0.3\linewidth}}{
        \cellcolor[rgb]{0.9,0.9,0.9}{
            \makecell[{{p{\linewidth}}}]{
                \texttt{\tiny{[GM$|$GM]}}
                \texttt{Strafpunkt: (2,2) ist nicht leer, sondern enthält 'W'.} \\
\\ 
\texttt{Ihr habt gemeinsam 2 von maximal 12 Strafpunkten angesammelt.} \\
\texttt{Bitte versuche es noch mal!} \\
            }
        }
    }
    & & \\ \\

    \theutterance \stepcounter{utterance}  
    & & \multicolumn{4}{p{0.6\linewidth}}{
        \cellcolor[rgb]{0.9,0.9,0.9}{
            \makecell[{{p{\linewidth}}}]{
                \texttt{\tiny{[GM$\rangle$P2]}}
                \texttt{Strafpunkt: (2,2) ist nicht leer, sondern enthält 'W'.} \\
\\ 
\texttt{Ihr habt gemeinsam 2 von maximal 12 Strafpunkten angesammelt.} \\
\texttt{Bitte versuche es noch mal!} \\
            }
        }
    }
    & \\ \\

    \theutterance \stepcounter{utterance}  
    & & & \multicolumn{4}{p{0.6\linewidth}}{
        \cellcolor[rgb]{0.9,0.9,0.9}{
            \makecell[{{p{\linewidth}}}]{
                \texttt{\tiny{[GM$\langle$P2]}}
                \texttt{SPRICH: Ups, mein W ist bereits bei (2,2)! Ich bewege stattdessen C von (4,1) zu (1,1).} \\
            }
        }
    }
     \\ \\

    \theutterance \stepcounter{utterance}  
    & & & \multicolumn{2}{p{0.3\linewidth}}{
        \cellcolor[rgb]{0.9,0.9,0.9}{
            \makecell[{{p{\linewidth}}}]{
                \texttt{\tiny{[GM$|$GM]}}
                \texttt{SPRICH: Ups, mein W ist bereits bei (2,2)! Ich bewege stattdessen C von (4,1) zu (1,1).} \\
            }
        }
    }
    & & \\ \\

    \theutterance \stepcounter{utterance}  
    & & \multicolumn{4}{p{0.6\linewidth}}{
        \cellcolor[rgb]{0.9,0.9,0.9}{
            \makecell[{{p{\linewidth}}}]{
                \texttt{\tiny{[P1$\langle$GM]}}
                \texttt{Deine Nachricht wurde an deinen Mitspieler übermittelt.} \\
\\ 
\\ 
\texttt{Ihr seid gerade bei Runde 2 von maximal 20.} \\
\texttt{Ihr habt gemeinsam 2 von maximal 12 Strafpunkten angesammelt.} \\
\texttt{Dein Mitspieler hat dir diese Nachricht geschickt:} \\
\\ 
\\ 
\texttt{"Ups, mein W ist bereits bei (2,2)! Ich bewege stattdessen C von (4,1) zu (1,1)."} \\
\\ 
\\ 
\texttt{Wie lautet dein nächster Befehl?} \\
            }
        }
    }
    & \\ \\

    \theutterance \stepcounter{utterance}  
    & \multicolumn{4}{p{0.6\linewidth}}{
        \cellcolor[rgb]{0.9,0.9,0.9}{
            \makecell[{{p{\linewidth}}}]{
                \texttt{\tiny{[P1$\rangle$GM]}}
                \texttt{VERSCHIEBE: H, (3, 3)} \\
            }
        }
    }
    & & \\ \\

    \theutterance \stepcounter{utterance}  
    & & & \multicolumn{2}{p{0.3\linewidth}}{
        \cellcolor[rgb]{0.9,0.9,0.9}{
            \makecell[{{p{\linewidth}}}]{
                \texttt{\tiny{[GM$|$GM]}}
                \texttt{VERSCHIEBE: H, (3, 3)} \\
            }
        }
    }
    & & \\ \\

    \theutterance \stepcounter{utterance}  
    & & & \multicolumn{2}{p{0.3\linewidth}}{
        \cellcolor[rgb]{0.9,0.9,0.9}{
            \makecell[{{p{\linewidth}}}]{
                \texttt{\tiny{[GM$|$GM]}}
                \texttt{'H' erfolgreich nach (3,3) verschoben. Dein Raster sieht jetzt so aus:} \\
\texttt{```} \\
\texttt{ 1234567} \\
\texttt{╔═══╤═╤═╗} \\
\texttt{║C◌◌│◌│◌║ 1} \\
\texttt{║◌I◌│◌├─╢ 2} \\
\texttt{║◌◌H├─┘◌║ 3} \\
\texttt{║◌◌◌│◌◌◌║ 4} \\
\texttt{╟─┬─┼───╢ 5} \\
\texttt{║◌│◌│◌WT║ 6} \\
\texttt{║◌│◌│◌◌◌║ 7} \\
\texttt{╚═╧═╧═══╝} \\
\texttt{```} \\
            }
        }
    }
    & & \\ \\

    \theutterance \stepcounter{utterance}  
    & & \multicolumn{4}{p{0.6\linewidth}}{
        \cellcolor[rgb]{0.9,0.9,0.9}{
            \makecell[{{p{\linewidth}}}]{
                \texttt{\tiny{[GM$\rangle$P2]}}
                \texttt{Deine Nachricht wurde an deinen Mitspieler übermittelt.} \\
\\ 
\\ 
\texttt{Ihr seid gerade bei Runde 3 von maximal 20.} \\
\texttt{Ihr habt gemeinsam 2 von maximal 12 Strafpunkten angesammelt.} \\
\texttt{Dein Mitspieler hat ein Objekt auf seinem Raster bewegt.} \\
\\ 
\\ 
\texttt{Wie lautet dein nächster Befehl?} \\
            }
        }
    }
    & \\ \\

    \theutterance \stepcounter{utterance}  
    & & & \multicolumn{4}{p{0.6\linewidth}}{
        \cellcolor[rgb]{0.9,0.9,0.9}{
            \makecell[{{p{\linewidth}}}]{
                \texttt{\tiny{[GM$\langle$P2]}}
                \texttt{VERSCHIEBE: C, (1, 1)} \\
            }
        }
    }
     \\ \\

    \theutterance \stepcounter{utterance}  
    & & & \multicolumn{2}{p{0.3\linewidth}}{
        \cellcolor[rgb]{0.9,0.9,0.9}{
            \makecell[{{p{\linewidth}}}]{
                \texttt{\tiny{[GM$|$GM]}}
                \texttt{VERSCHIEBE: C, (1, 1)} \\
            }
        }
    }
    & & \\ \\

    \theutterance \stepcounter{utterance}  
    & & & \multicolumn{2}{p{0.3\linewidth}}{
        \cellcolor[rgb]{0.9,0.9,0.9}{
            \makecell[{{p{\linewidth}}}]{
                \texttt{\tiny{[GM$|$GM]}}
                \texttt{'C' erfolgreich nach (1,1) verschoben. Dein Raster sieht jetzt so aus:} \\
\texttt{```} \\
\texttt{ 1234567} \\
\texttt{╔═══╤═╤═╗} \\
\texttt{║C◌◌│◌│◌║ 1} \\
\texttt{║◌W◌│◌├─╢ 2} \\
\texttt{║◌◌◌├─┘◌║ 3} \\
\texttt{║◌◌◌│◌◌H║ 4} \\
\texttt{╟─┬─┼───╢ 5} \\
\texttt{║T│◌│I◌◌║ 6} \\
\texttt{║◌│◌│◌◌◌║ 7} \\
\texttt{╚═╧═╧═══╝} \\
\texttt{```} \\
            }
        }
    }
    & & \\ \\

    \theutterance \stepcounter{utterance}  
    & & \multicolumn{4}{p{0.6\linewidth}}{
        \cellcolor[rgb]{0.9,0.9,0.9}{
            \makecell[{{p{\linewidth}}}]{
                \texttt{\tiny{[P1$\langle$GM]}}
                \texttt{'H' erfolgreich nach (3,3) verschoben. Dein Raster sieht jetzt so aus:} \\
\texttt{```} \\
\texttt{ 1234567} \\
\texttt{╔═══╤═╤═╗} \\
\texttt{║C◌◌│◌│◌║ 1} \\
\texttt{║◌I◌│◌├─╢ 2} \\
\texttt{║◌◌H├─┘◌║ 3} \\
\texttt{║◌◌◌│◌◌◌║ 4} \\
\texttt{╟─┬─┼───╢ 5} \\
\texttt{║◌│◌│◌WT║ 6} \\
\texttt{║◌│◌│◌◌◌║ 7} \\
\texttt{╚═╧═╧═══╝} \\
\texttt{```} \\
\texttt{Ihr seid gerade bei Runde 3 von maximal 20.} \\
\texttt{Ihr habt gemeinsam 2 von maximal 12 Strafpunkten angesammelt.} \\
\texttt{Dein Mitspieler hat ein Objekt auf seinem Raster bewegt.} \\
\\ 
\\ 
\texttt{Wie lautet dein nächster Befehl?} \\
            }
        }
    }
    & \\ \\

    \theutterance \stepcounter{utterance}  
    & \multicolumn{4}{p{0.6\linewidth}}{
        \cellcolor[rgb]{0.9,0.9,0.9}{
            \makecell[{{p{\linewidth}}}]{
                \texttt{\tiny{[P1$\rangle$GM]}}
                \texttt{VERSCHIEBE: I, (4, 4)} \\
            }
        }
    }
    & & \\ \\

    \theutterance \stepcounter{utterance}  
    & & & \multicolumn{2}{p{0.3\linewidth}}{
        \cellcolor[rgb]{0.9,0.9,0.9}{
            \makecell[{{p{\linewidth}}}]{
                \texttt{\tiny{[GM$|$GM]}}
                \texttt{VERSCHIEBE: I, (4, 4)} \\
            }
        }
    }
    & & \\ \\

    \theutterance \stepcounter{utterance}  
    & & & \multicolumn{2}{p{0.3\linewidth}}{
        \cellcolor[rgb]{0.9,0.9,0.9}{
            \makecell[{{p{\linewidth}}}]{
                \texttt{\tiny{[GM$|$GM]}}
                \texttt{Strafpunkt: (4,4) ist nicht leer, sondern enthält '│'.} \\
\\ 
\texttt{Ihr habt gemeinsam 3 von maximal 12 Strafpunkten angesammelt.} \\
\texttt{Bitte versuche es noch mal!} \\
            }
        }
    }
    & & \\ \\

    \theutterance \stepcounter{utterance}  
    & & \multicolumn{4}{p{0.6\linewidth}}{
        \cellcolor[rgb]{0.9,0.9,0.9}{
            \makecell[{{p{\linewidth}}}]{
                \texttt{\tiny{[P1$\langle$GM]}}
                \texttt{Strafpunkt: (4,4) ist nicht leer, sondern enthält '│'.} \\
\\ 
\texttt{Ihr habt gemeinsam 3 von maximal 12 Strafpunkten angesammelt.} \\
\texttt{Bitte versuche es noch mal!} \\
            }
        }
    }
    & \\ \\

    \theutterance \stepcounter{utterance}  
    & \multicolumn{4}{p{0.6\linewidth}}{
        \cellcolor[rgb]{0.9,0.9,0.9}{
            \makecell[{{p{\linewidth}}}]{
                \texttt{\tiny{[P1$\rangle$GM]}}
                \texttt{SPRICH: Fehler! Position (4,4) ist blockiert durch eine Linie. Wir brauchen neue Zielpositionen. Lass mich freie Positionen vorschlagen: C(1,1), W(1,2), H(3,3), I(2,3), T(1,4). Alle sind auf meinem Brett frei. Passt das bei dir?} \\
            }
        }
    }
    & & \\ \\

    \theutterance \stepcounter{utterance}  
    & & & \multicolumn{2}{p{0.3\linewidth}}{
        \cellcolor[rgb]{0.9,0.9,0.9}{
            \makecell[{{p{\linewidth}}}]{
                \texttt{\tiny{[GM$|$GM]}}
                \texttt{SPRICH: Fehler! Position (4,4) ist blockiert durch eine Linie. Wir brauchen neue Zielpositionen. Lass mich freie Positionen vorschlagen: C(1,1), W(1,2), H(3,3), I(2,3), T(1,4). Alle sind auf meinem Brett frei. Passt das bei dir?} \\
            }
        }
    }
    & & \\ \\

    \theutterance \stepcounter{utterance}  
    & & \multicolumn{4}{p{0.6\linewidth}}{
        \cellcolor[rgb]{0.9,0.9,0.9}{
            \makecell[{{p{\linewidth}}}]{
                \texttt{\tiny{[GM$\rangle$P2]}}
                \texttt{'C' erfolgreich nach (1,1) verschoben. Dein Raster sieht jetzt so aus:} \\
\texttt{```} \\
\texttt{ 1234567} \\
\texttt{╔═══╤═╤═╗} \\
\texttt{║C◌◌│◌│◌║ 1} \\
\texttt{║◌W◌│◌├─╢ 2} \\
\texttt{║◌◌◌├─┘◌║ 3} \\
\texttt{║◌◌◌│◌◌H║ 4} \\
\texttt{╟─┬─┼───╢ 5} \\
\texttt{║T│◌│I◌◌║ 6} \\
\texttt{║◌│◌│◌◌◌║ 7} \\
\texttt{╚═╧═╧═══╝} \\
\texttt{```} \\
\texttt{Ihr seid gerade bei Runde 4 von maximal 20.} \\
\texttt{Ihr habt gemeinsam 3 von maximal 12 Strafpunkten angesammelt.} \\
\texttt{Dein Mitspieler hat dir diese Nachricht geschickt:} \\
\\ 
\\ 
\texttt{"Fehler! Position (4,4) ist blockiert durch eine Linie. Wir brauchen neue Zielpositionen. Lass mich freie Positionen vorschlagen: C(1,1), W(1,2), H(3,3), I(2,3), T(1,4). Alle sind auf meinem Brett frei. Passt das bei dir?"} \\
\\ 
\\ 
\texttt{Wie lautet dein nächster Befehl?} \\
            }
        }
    }
    & \\ \\

    \theutterance \stepcounter{utterance}  
    & & & \multicolumn{4}{p{0.6\linewidth}}{
        \cellcolor[rgb]{0.9,0.9,0.9}{
            \makecell[{{p{\linewidth}}}]{
                \texttt{\tiny{[GM$\langle$P2]}}
                \texttt{SPRICH: Ja, die neuen Positionen funktionieren bei mir auch! C(1,1) ist bereits fertig, W muss von (2,2) zu (1,2), H von (7,4) zu (3,3), I von (4,6) zu (2,3), T von (1,6) zu (1,4). Ich bewege als nächstes W zu (1,2).} \\
            }
        }
    }
     \\ \\

    \theutterance \stepcounter{utterance}  
    & & & \multicolumn{2}{p{0.3\linewidth}}{
        \cellcolor[rgb]{0.9,0.9,0.9}{
            \makecell[{{p{\linewidth}}}]{
                \texttt{\tiny{[GM$|$GM]}}
                \texttt{SPRICH: Ja, die neuen Positionen funktionieren bei mir auch! C(1,1) ist bereits fertig, W muss von (2,2) zu (1,2), H von (7,4) zu (3,3), I von (4,6) zu (2,3), T von (1,6) zu (1,4). Ich bewege als nächstes W zu (1,2).} \\
            }
        }
    }
    & & \\ \\

    \theutterance \stepcounter{utterance}  
    & & \multicolumn{4}{p{0.6\linewidth}}{
        \cellcolor[rgb]{0.9,0.9,0.9}{
            \makecell[{{p{\linewidth}}}]{
                \texttt{\tiny{[P1$\langle$GM]}}
                \texttt{Deine Nachricht wurde an deinen Mitspieler übermittelt.} \\
\\ 
\\ 
\texttt{Ihr seid gerade bei Runde 4 von maximal 20.} \\
\texttt{Ihr habt gemeinsam 3 von maximal 12 Strafpunkten angesammelt.} \\
\texttt{Dein Mitspieler hat dir diese Nachricht geschickt:} \\
\\ 
\\ 
\texttt{"Ja, die neuen Positionen funktionieren bei mir auch! C(1,1) ist bereits fertig, W muss von (2,2) zu (1,2), H von (7,4) zu (3,3), I von (4,6) zu (2,3), T von (1,6) zu (1,4). Ich bewege als nächstes W zu (1,2)."} \\
\\ 
\\ 
\texttt{Wie lautet dein nächster Befehl?} \\
            }
        }
    }
    & \\ \\

    \theutterance \stepcounter{utterance}  
    & \multicolumn{4}{p{0.6\linewidth}}{
        \cellcolor[rgb]{0.9,0.9,0.9}{
            \makecell[{{p{\linewidth}}}]{
                \texttt{\tiny{[P1$\rangle$GM]}}
                \texttt{VERSCHIEBE: I, (2, 3)} \\
            }
        }
    }
    & & \\ \\

    \theutterance \stepcounter{utterance}  
    & & & \multicolumn{2}{p{0.3\linewidth}}{
        \cellcolor[rgb]{0.9,0.9,0.9}{
            \makecell[{{p{\linewidth}}}]{
                \texttt{\tiny{[GM$|$GM]}}
                \texttt{VERSCHIEBE: I, (2, 3)} \\
            }
        }
    }
    & & \\ \\

    \theutterance \stepcounter{utterance}  
    & & & \multicolumn{2}{p{0.3\linewidth}}{
        \cellcolor[rgb]{0.9,0.9,0.9}{
            \makecell[{{p{\linewidth}}}]{
                \texttt{\tiny{[GM$|$GM]}}
                \texttt{'I' erfolgreich nach (2,3) verschoben. Dein Raster sieht jetzt so aus:} \\
\texttt{```} \\
\texttt{ 1234567} \\
\texttt{╔═══╤═╤═╗} \\
\texttt{║C◌◌│◌│◌║ 1} \\
\texttt{║◌◌◌│◌├─╢ 2} \\
\texttt{║◌IH├─┘◌║ 3} \\
\texttt{║◌◌◌│◌◌◌║ 4} \\
\texttt{╟─┬─┼───╢ 5} \\
\texttt{║◌│◌│◌WT║ 6} \\
\texttt{║◌│◌│◌◌◌║ 7} \\
\texttt{╚═╧═╧═══╝} \\
\texttt{```} \\
            }
        }
    }
    & & \\ \\

    \theutterance \stepcounter{utterance}  
    & & \multicolumn{4}{p{0.6\linewidth}}{
        \cellcolor[rgb]{0.9,0.9,0.9}{
            \makecell[{{p{\linewidth}}}]{
                \texttt{\tiny{[GM$\rangle$P2]}}
                \texttt{Deine Nachricht wurde an deinen Mitspieler übermittelt.} \\
\\ 
\\ 
\texttt{Ihr seid gerade bei Runde 5 von maximal 20.} \\
\texttt{Ihr habt gemeinsam 3 von maximal 12 Strafpunkten angesammelt.} \\
\texttt{Dein Mitspieler hat ein Objekt auf seinem Raster bewegt.} \\
\\ 
\\ 
\texttt{Wie lautet dein nächster Befehl?} \\
            }
        }
    }
    & \\ \\

    \theutterance \stepcounter{utterance}  
    & & & \multicolumn{4}{p{0.6\linewidth}}{
        \cellcolor[rgb]{0.9,0.9,0.9}{
            \makecell[{{p{\linewidth}}}]{
                \texttt{\tiny{[GM$\langle$P2]}}
                \texttt{VERSCHIEBE: W, (1, 2)} \\
            }
        }
    }
     \\ \\

    \theutterance \stepcounter{utterance}  
    & & & \multicolumn{2}{p{0.3\linewidth}}{
        \cellcolor[rgb]{0.9,0.9,0.9}{
            \makecell[{{p{\linewidth}}}]{
                \texttt{\tiny{[GM$|$GM]}}
                \texttt{VERSCHIEBE: W, (1, 2)} \\
            }
        }
    }
    & & \\ \\

    \theutterance \stepcounter{utterance}  
    & & & \multicolumn{2}{p{0.3\linewidth}}{
        \cellcolor[rgb]{0.9,0.9,0.9}{
            \makecell[{{p{\linewidth}}}]{
                \texttt{\tiny{[GM$|$GM]}}
                \texttt{'W' erfolgreich nach (1,2) verschoben. Dein Raster sieht jetzt so aus:} \\
\texttt{```} \\
\texttt{ 1234567} \\
\texttt{╔═══╤═╤═╗} \\
\texttt{║C◌◌│◌│◌║ 1} \\
\texttt{║W◌◌│◌├─╢ 2} \\
\texttt{║◌◌◌├─┘◌║ 3} \\
\texttt{║◌◌◌│◌◌H║ 4} \\
\texttt{╟─┬─┼───╢ 5} \\
\texttt{║T│◌│I◌◌║ 6} \\
\texttt{║◌│◌│◌◌◌║ 7} \\
\texttt{╚═╧═╧═══╝} \\
\texttt{```} \\
            }
        }
    }
    & & \\ \\

    \theutterance \stepcounter{utterance}  
    & & \multicolumn{4}{p{0.6\linewidth}}{
        \cellcolor[rgb]{0.9,0.9,0.9}{
            \makecell[{{p{\linewidth}}}]{
                \texttt{\tiny{[P1$\langle$GM]}}
                \texttt{'I' erfolgreich nach (2,3) verschoben. Dein Raster sieht jetzt so aus:} \\
\texttt{```} \\
\texttt{ 1234567} \\
\texttt{╔═══╤═╤═╗} \\
\texttt{║C◌◌│◌│◌║ 1} \\
\texttt{║◌◌◌│◌├─╢ 2} \\
\texttt{║◌IH├─┘◌║ 3} \\
\texttt{║◌◌◌│◌◌◌║ 4} \\
\texttt{╟─┬─┼───╢ 5} \\
\texttt{║◌│◌│◌WT║ 6} \\
\texttt{║◌│◌│◌◌◌║ 7} \\
\texttt{╚═╧═╧═══╝} \\
\texttt{```} \\
\texttt{Ihr seid gerade bei Runde 5 von maximal 20.} \\
\texttt{Ihr habt gemeinsam 3 von maximal 12 Strafpunkten angesammelt.} \\
\texttt{Dein Mitspieler hat ein Objekt auf seinem Raster bewegt.} \\
\\ 
\\ 
\texttt{Wie lautet dein nächster Befehl?} \\
            }
        }
    }
    & \\ \\

    \theutterance \stepcounter{utterance}  
    & \multicolumn{4}{p{0.6\linewidth}}{
        \cellcolor[rgb]{0.9,0.9,0.9}{
            \makecell[{{p{\linewidth}}}]{
                \texttt{\tiny{[P1$\rangle$GM]}}
                \texttt{VERSCHIEBE: T, (1, 4)} \\
            }
        }
    }
    & & \\ \\

    \theutterance \stepcounter{utterance}  
    & & & \multicolumn{2}{p{0.3\linewidth}}{
        \cellcolor[rgb]{0.9,0.9,0.9}{
            \makecell[{{p{\linewidth}}}]{
                \texttt{\tiny{[GM$|$GM]}}
                \texttt{VERSCHIEBE: T, (1, 4)} \\
            }
        }
    }
    & & \\ \\

    \theutterance \stepcounter{utterance}  
    & & & \multicolumn{2}{p{0.3\linewidth}}{
        \cellcolor[rgb]{0.9,0.9,0.9}{
            \makecell[{{p{\linewidth}}}]{
                \texttt{\tiny{[GM$|$GM]}}
                \texttt{'T' erfolgreich nach (1,4) verschoben. Dein Raster sieht jetzt so aus:} \\
\texttt{```} \\
\texttt{ 1234567} \\
\texttt{╔═══╤═╤═╗} \\
\texttt{║C◌◌│◌│◌║ 1} \\
\texttt{║◌◌◌│◌├─╢ 2} \\
\texttt{║◌IH├─┘◌║ 3} \\
\texttt{║T◌◌│◌◌◌║ 4} \\
\texttt{╟─┬─┼───╢ 5} \\
\texttt{║◌│◌│◌W◌║ 6} \\
\texttt{║◌│◌│◌◌◌║ 7} \\
\texttt{╚═╧═╧═══╝} \\
\texttt{```} \\
            }
        }
    }
    & & \\ \\

    \theutterance \stepcounter{utterance}  
    & & \multicolumn{4}{p{0.6\linewidth}}{
        \cellcolor[rgb]{0.9,0.9,0.9}{
            \makecell[{{p{\linewidth}}}]{
                \texttt{\tiny{[GM$\rangle$P2]}}
                \texttt{'W' erfolgreich nach (1,2) verschoben. Dein Raster sieht jetzt so aus:} \\
\texttt{```} \\
\texttt{ 1234567} \\
\texttt{╔═══╤═╤═╗} \\
\texttt{║C◌◌│◌│◌║ 1} \\
\texttt{║W◌◌│◌├─╢ 2} \\
\texttt{║◌◌◌├─┘◌║ 3} \\
\texttt{║◌◌◌│◌◌H║ 4} \\
\texttt{╟─┬─┼───╢ 5} \\
\texttt{║T│◌│I◌◌║ 6} \\
\texttt{║◌│◌│◌◌◌║ 7} \\
\texttt{╚═╧═╧═══╝} \\
\texttt{```} \\
\texttt{Ihr seid gerade bei Runde 6 von maximal 20.} \\
\texttt{Ihr habt gemeinsam 3 von maximal 12 Strafpunkten angesammelt.} \\
\texttt{Dein Mitspieler hat ein Objekt auf seinem Raster bewegt.} \\
\\ 
\\ 
\texttt{Wie lautet dein nächster Befehl?} \\
            }
        }
    }
    & \\ \\

    \theutterance \stepcounter{utterance}  
    & & & \multicolumn{4}{p{0.6\linewidth}}{
        \cellcolor[rgb]{0.9,0.9,0.9}{
            \makecell[{{p{\linewidth}}}]{
                \texttt{\tiny{[GM$\langle$P2]}}
                \texttt{VERSCHIEBE: H, (3, 3)} \\
            }
        }
    }
     \\ \\

    \theutterance \stepcounter{utterance}  
    & & & \multicolumn{2}{p{0.3\linewidth}}{
        \cellcolor[rgb]{0.9,0.9,0.9}{
            \makecell[{{p{\linewidth}}}]{
                \texttt{\tiny{[GM$|$GM]}}
                \texttt{VERSCHIEBE: H, (3, 3)} \\
            }
        }
    }
    & & \\ \\

    \theutterance \stepcounter{utterance}  
    & & & \multicolumn{2}{p{0.3\linewidth}}{
        \cellcolor[rgb]{0.9,0.9,0.9}{
            \makecell[{{p{\linewidth}}}]{
                \texttt{\tiny{[GM$|$GM]}}
                \texttt{'H' erfolgreich nach (3,3) verschoben. Dein Raster sieht jetzt so aus:} \\
\texttt{```} \\
\texttt{ 1234567} \\
\texttt{╔═══╤═╤═╗} \\
\texttt{║C◌◌│◌│◌║ 1} \\
\texttt{║W◌◌│◌├─╢ 2} \\
\texttt{║◌◌H├─┘◌║ 3} \\
\texttt{║◌◌◌│◌◌◌║ 4} \\
\texttt{╟─┬─┼───╢ 5} \\
\texttt{║T│◌│I◌◌║ 6} \\
\texttt{║◌│◌│◌◌◌║ 7} \\
\texttt{╚═╧═╧═══╝} \\
\texttt{```} \\
            }
        }
    }
    & & \\ \\

    \theutterance \stepcounter{utterance}  
    & & \multicolumn{4}{p{0.6\linewidth}}{
        \cellcolor[rgb]{0.9,0.9,0.9}{
            \makecell[{{p{\linewidth}}}]{
                \texttt{\tiny{[P1$\langle$GM]}}
                \texttt{'T' erfolgreich nach (1,4) verschoben. Dein Raster sieht jetzt so aus:} \\
\texttt{```} \\
\texttt{ 1234567} \\
\texttt{╔═══╤═╤═╗} \\
\texttt{║C◌◌│◌│◌║ 1} \\
\texttt{║◌◌◌│◌├─╢ 2} \\
\texttt{║◌IH├─┘◌║ 3} \\
\texttt{║T◌◌│◌◌◌║ 4} \\
\texttt{╟─┬─┼───╢ 5} \\
\texttt{║◌│◌│◌W◌║ 6} \\
\texttt{║◌│◌│◌◌◌║ 7} \\
\texttt{╚═╧═╧═══╝} \\
\texttt{```} \\
\texttt{Ihr seid gerade bei Runde 6 von maximal 20.} \\
\texttt{Ihr habt gemeinsam 3 von maximal 12 Strafpunkten angesammelt.} \\
\texttt{Dein Mitspieler hat ein Objekt auf seinem Raster bewegt.} \\
\\ 
\\ 
\texttt{Wie lautet dein nächster Befehl?} \\
            }
        }
    }
    & \\ \\

    \theutterance \stepcounter{utterance}  
    & \multicolumn{4}{p{0.6\linewidth}}{
        \cellcolor[rgb]{0.9,0.9,0.9}{
            \makecell[{{p{\linewidth}}}]{
                \texttt{\tiny{[P1$\rangle$GM]}}
                \texttt{VERSCHIEBE: W, (1, 2)} \\
            }
        }
    }
    & & \\ \\

    \theutterance \stepcounter{utterance}  
    & & & \multicolumn{2}{p{0.3\linewidth}}{
        \cellcolor[rgb]{0.9,0.9,0.9}{
            \makecell[{{p{\linewidth}}}]{
                \texttt{\tiny{[GM$|$GM]}}
                \texttt{VERSCHIEBE: W, (1, 2)} \\
            }
        }
    }
    & & \\ \\

    \theutterance \stepcounter{utterance}  
    & & & \multicolumn{2}{p{0.3\linewidth}}{
        \cellcolor[rgb]{0.9,0.9,0.9}{
            \makecell[{{p{\linewidth}}}]{
                \texttt{\tiny{[GM$|$GM]}}
                \texttt{'W' erfolgreich nach (1,2) verschoben. Dein Raster sieht jetzt so aus:} \\
\texttt{```} \\
\texttt{ 1234567} \\
\texttt{╔═══╤═╤═╗} \\
\texttt{║C◌◌│◌│◌║ 1} \\
\texttt{║W◌◌│◌├─╢ 2} \\
\texttt{║◌IH├─┘◌║ 3} \\
\texttt{║T◌◌│◌◌◌║ 4} \\
\texttt{╟─┬─┼───╢ 5} \\
\texttt{║◌│◌│◌◌◌║ 6} \\
\texttt{║◌│◌│◌◌◌║ 7} \\
\texttt{╚═╧═╧═══╝} \\
\texttt{```} \\
            }
        }
    }
    & & \\ \\

    \theutterance \stepcounter{utterance}  
    & & \multicolumn{4}{p{0.6\linewidth}}{
        \cellcolor[rgb]{0.9,0.9,0.9}{
            \makecell[{{p{\linewidth}}}]{
                \texttt{\tiny{[GM$\rangle$P2]}}
                \texttt{'H' erfolgreich nach (3,3) verschoben. Dein Raster sieht jetzt so aus:} \\
\texttt{```} \\
\texttt{ 1234567} \\
\texttt{╔═══╤═╤═╗} \\
\texttt{║C◌◌│◌│◌║ 1} \\
\texttt{║W◌◌│◌├─╢ 2} \\
\texttt{║◌◌H├─┘◌║ 3} \\
\texttt{║◌◌◌│◌◌◌║ 4} \\
\texttt{╟─┬─┼───╢ 5} \\
\texttt{║T│◌│I◌◌║ 6} \\
\texttt{║◌│◌│◌◌◌║ 7} \\
\texttt{╚═╧═╧═══╝} \\
\texttt{```} \\
\texttt{Ihr seid gerade bei Runde 7 von maximal 20.} \\
\texttt{Ihr habt gemeinsam 3 von maximal 12 Strafpunkten angesammelt.} \\
\texttt{Dein Mitspieler hat ein Objekt auf seinem Raster bewegt.} \\
\\ 
\\ 
\texttt{Wie lautet dein nächster Befehl?} \\
            }
        }
    }
    & \\ \\

    \theutterance \stepcounter{utterance}  
    & & & \multicolumn{4}{p{0.6\linewidth}}{
        \cellcolor[rgb]{0.9,0.9,0.9}{
            \makecell[{{p{\linewidth}}}]{
                \texttt{\tiny{[GM$\langle$P2]}}
                \texttt{VERSCHIEBE: I, (2, 3)} \\
            }
        }
    }
     \\ \\

    \theutterance \stepcounter{utterance}  
    & & & \multicolumn{2}{p{0.3\linewidth}}{
        \cellcolor[rgb]{0.9,0.9,0.9}{
            \makecell[{{p{\linewidth}}}]{
                \texttt{\tiny{[GM$|$GM]}}
                \texttt{VERSCHIEBE: I, (2, 3)} \\
            }
        }
    }
    & & \\ \\

    \theutterance \stepcounter{utterance}  
    & & & \multicolumn{2}{p{0.3\linewidth}}{
        \cellcolor[rgb]{0.9,0.9,0.9}{
            \makecell[{{p{\linewidth}}}]{
                \texttt{\tiny{[GM$|$GM]}}
                \texttt{'I' erfolgreich nach (2,3) verschoben. Dein Raster sieht jetzt so aus:} \\
\texttt{```} \\
\texttt{ 1234567} \\
\texttt{╔═══╤═╤═╗} \\
\texttt{║C◌◌│◌│◌║ 1} \\
\texttt{║W◌◌│◌├─╢ 2} \\
\texttt{║◌IH├─┘◌║ 3} \\
\texttt{║◌◌◌│◌◌◌║ 4} \\
\texttt{╟─┬─┼───╢ 5} \\
\texttt{║T│◌│◌◌◌║ 6} \\
\texttt{║◌│◌│◌◌◌║ 7} \\
\texttt{╚═╧═╧═══╝} \\
\texttt{```} \\
            }
        }
    }
    & & \\ \\

    \theutterance \stepcounter{utterance}  
    & & \multicolumn{4}{p{0.6\linewidth}}{
        \cellcolor[rgb]{0.9,0.9,0.9}{
            \makecell[{{p{\linewidth}}}]{
                \texttt{\tiny{[P1$\langle$GM]}}
                \texttt{'W' erfolgreich nach (1,2) verschoben. Dein Raster sieht jetzt so aus:} \\
\texttt{```} \\
\texttt{ 1234567} \\
\texttt{╔═══╤═╤═╗} \\
\texttt{║C◌◌│◌│◌║ 1} \\
\texttt{║W◌◌│◌├─╢ 2} \\
\texttt{║◌IH├─┘◌║ 3} \\
\texttt{║T◌◌│◌◌◌║ 4} \\
\texttt{╟─┬─┼───╢ 5} \\
\texttt{║◌│◌│◌◌◌║ 6} \\
\texttt{║◌│◌│◌◌◌║ 7} \\
\texttt{╚═╧═╧═══╝} \\
\texttt{```} \\
\texttt{Ihr seid gerade bei Runde 7 von maximal 20.} \\
\texttt{Ihr habt gemeinsam 3 von maximal 12 Strafpunkten angesammelt.} \\
\texttt{Dein Mitspieler hat ein Objekt auf seinem Raster bewegt.} \\
\\ 
\\ 
\texttt{Wie lautet dein nächster Befehl?} \\
            }
        }
    }
    & \\ \\

    \theutterance \stepcounter{utterance}  
    & \multicolumn{4}{p{0.6\linewidth}}{
        \cellcolor[rgb]{0.9,0.9,0.9}{
            \makecell[{{p{\linewidth}}}]{
                \texttt{\tiny{[P1$\rangle$GM]}}
                \texttt{SPRICH: Perfekt! Alle meine Objekte sind jetzt an den Zielpositionen: C(1,1), W(1,2), H(3,3), I(2,3), T(1,4). Wie sieht es bei dir aus? Sind alle deine Objekte auch richtig positioniert?} \\
            }
        }
    }
    & & \\ \\

    \theutterance \stepcounter{utterance}  
    & & & \multicolumn{2}{p{0.3\linewidth}}{
        \cellcolor[rgb]{0.9,0.9,0.9}{
            \makecell[{{p{\linewidth}}}]{
                \texttt{\tiny{[GM$|$GM]}}
                \texttt{SPRICH: Perfekt! Alle meine Objekte sind jetzt an den Zielpositionen: C(1,1), W(1,2), H(3,3), I(2,3), T(1,4). Wie sieht es bei dir aus? Sind alle deine Objekte auch richtig positioniert?} \\
            }
        }
    }
    & & \\ \\

    \theutterance \stepcounter{utterance}  
    & & \multicolumn{4}{p{0.6\linewidth}}{
        \cellcolor[rgb]{0.9,0.9,0.9}{
            \makecell[{{p{\linewidth}}}]{
                \texttt{\tiny{[GM$\rangle$P2]}}
                \texttt{'I' erfolgreich nach (2,3) verschoben. Dein Raster sieht jetzt so aus:} \\
\texttt{```} \\
\texttt{ 1234567} \\
\texttt{╔═══╤═╤═╗} \\
\texttt{║C◌◌│◌│◌║ 1} \\
\texttt{║W◌◌│◌├─╢ 2} \\
\texttt{║◌IH├─┘◌║ 3} \\
\texttt{║◌◌◌│◌◌◌║ 4} \\
\texttt{╟─┬─┼───╢ 5} \\
\texttt{║T│◌│◌◌◌║ 6} \\
\texttt{║◌│◌│◌◌◌║ 7} \\
\texttt{╚═╧═╧═══╝} \\
\texttt{```} \\
\texttt{Ihr seid gerade bei Runde 8 von maximal 20.} \\
\texttt{Ihr habt gemeinsam 3 von maximal 12 Strafpunkten angesammelt.} \\
\texttt{Dein Mitspieler hat dir diese Nachricht geschickt:} \\
\\ 
\\ 
\texttt{"Perfekt! Alle meine Objekte sind jetzt an den Zielpositionen: C(1,1), W(1,2), H(3,3), I(2,3), T(1,4). Wie sieht es bei dir aus? Sind alle deine Objekte auch richtig positioniert?"} \\
\\ 
\\ 
\texttt{Wie lautet dein nächster Befehl?} \\
            }
        }
    }
    & \\ \\

    \theutterance \stepcounter{utterance}  
    & & & \multicolumn{4}{p{0.6\linewidth}}{
        \cellcolor[rgb]{0.9,0.9,0.9}{
            \makecell[{{p{\linewidth}}}]{
                \texttt{\tiny{[GM$\langle$P2]}}
                \texttt{VERSCHIEBE: T, (1, 4)} \\
            }
        }
    }
     \\ \\

    \theutterance \stepcounter{utterance}  
    & & & \multicolumn{2}{p{0.3\linewidth}}{
        \cellcolor[rgb]{0.9,0.9,0.9}{
            \makecell[{{p{\linewidth}}}]{
                \texttt{\tiny{[GM$|$GM]}}
                \texttt{VERSCHIEBE: T, (1, 4)} \\
            }
        }
    }
    & & \\ \\

    \theutterance \stepcounter{utterance}  
    & & & \multicolumn{2}{p{0.3\linewidth}}{
        \cellcolor[rgb]{0.9,0.9,0.9}{
            \makecell[{{p{\linewidth}}}]{
                \texttt{\tiny{[GM$|$GM]}}
                \texttt{'T' erfolgreich nach (1,4) verschoben. Dein Raster sieht jetzt so aus:} \\
\texttt{```} \\
\texttt{ 1234567} \\
\texttt{╔═══╤═╤═╗} \\
\texttt{║C◌◌│◌│◌║ 1} \\
\texttt{║W◌◌│◌├─╢ 2} \\
\texttt{║◌IH├─┘◌║ 3} \\
\texttt{║T◌◌│◌◌◌║ 4} \\
\texttt{╟─┬─┼───╢ 5} \\
\texttt{║◌│◌│◌◌◌║ 6} \\
\texttt{║◌│◌│◌◌◌║ 7} \\
\texttt{╚═╧═╧═══╝} \\
\texttt{```} \\
            }
        }
    }
    & & \\ \\

    \theutterance \stepcounter{utterance}  
    & & \multicolumn{4}{p{0.6\linewidth}}{
        \cellcolor[rgb]{0.9,0.9,0.9}{
            \makecell[{{p{\linewidth}}}]{
                \texttt{\tiny{[P1$\langle$GM]}}
                \texttt{Deine Nachricht wurde an deinen Mitspieler übermittelt.} \\
\\ 
\\ 
\texttt{Ihr seid gerade bei Runde 8 von maximal 20.} \\
\texttt{Ihr habt gemeinsam 3 von maximal 12 Strafpunkten angesammelt.} \\
\texttt{Dein Mitspieler hat ein Objekt auf seinem Raster bewegt.} \\
\\ 
\\ 
\texttt{Wie lautet dein nächster Befehl?} \\
            }
        }
    }
    & \\ \\

    \theutterance \stepcounter{utterance}  
    & \multicolumn{4}{p{0.6\linewidth}}{
        \cellcolor[rgb]{0.9,0.9,0.9}{
            \makecell[{{p{\linewidth}}}]{
                \texttt{\tiny{[P1$\rangle$GM]}}
                \texttt{SPRICH: fertig?} \\
            }
        }
    }
    & & \\ \\

    \theutterance \stepcounter{utterance}  
    & & & \multicolumn{2}{p{0.3\linewidth}}{
        \cellcolor[rgb]{0.9,0.9,0.9}{
            \makecell[{{p{\linewidth}}}]{
                \texttt{\tiny{[GM$|$GM]}}
                \texttt{SPRICH: fertig?} \\
            }
        }
    }
    & & \\ \\

    \theutterance \stepcounter{utterance}  
    & & \multicolumn{4}{p{0.6\linewidth}}{
        \cellcolor[rgb]{0.9,0.9,0.9}{
            \makecell[{{p{\linewidth}}}]{
                \texttt{\tiny{[GM$\rangle$P2]}}
                \texttt{'T' erfolgreich nach (1,4) verschoben. Dein Raster sieht jetzt so aus:} \\
\texttt{```} \\
\texttt{ 1234567} \\
\texttt{╔═══╤═╤═╗} \\
\texttt{║C◌◌│◌│◌║ 1} \\
\texttt{║W◌◌│◌├─╢ 2} \\
\texttt{║◌IH├─┘◌║ 3} \\
\texttt{║T◌◌│◌◌◌║ 4} \\
\texttt{╟─┬─┼───╢ 5} \\
\texttt{║◌│◌│◌◌◌║ 6} \\
\texttt{║◌│◌│◌◌◌║ 7} \\
\texttt{╚═╧═╧═══╝} \\
\texttt{```} \\
\texttt{Ihr seid gerade bei Runde 9 von maximal 20.} \\
\texttt{Ihr habt gemeinsam 3 von maximal 12 Strafpunkten angesammelt.} \\
\texttt{Dein Mitspieler hat dir diese Nachricht geschickt:} \\
\\ 
\\ 
\texttt{"fertig?"} \\
\\ 
\\ 
\texttt{Wie lautet dein nächster Befehl?} \\
            }
        }
    }
    & \\ \\

    \theutterance \stepcounter{utterance}  
    & & & \multicolumn{4}{p{0.6\linewidth}}{
        \cellcolor[rgb]{0.9,0.9,0.9}{
            \makecell[{{p{\linewidth}}}]{
                \texttt{\tiny{[GM$\langle$P2]}}
                \texttt{SPRICH: fertig!} \\
            }
        }
    }
     \\ \\

    \theutterance \stepcounter{utterance}  
    & & & \multicolumn{2}{p{0.3\linewidth}}{
        \cellcolor[rgb]{0.9,0.9,0.9}{
            \makecell[{{p{\linewidth}}}]{
                \texttt{\tiny{[GM$|$GM]}}
                \texttt{SPRICH: fertig!} \\
            }
        }
    }
    & & \\ \\

    \theutterance \stepcounter{utterance}  
    & & & \multicolumn{2}{p{0.3\linewidth}}{
        \cellcolor[rgb]{0.9,0.9,0.9}{
            \makecell[{{p{\linewidth}}}]{
                \texttt{\tiny{[GM$|$GM]}}
                \texttt{true} \\
            }
        }
    }
    & & \\ \\

    \theutterance \stepcounter{utterance}  
    & & & \multicolumn{2}{p{0.3\linewidth}}{
        \cellcolor[rgb]{0.9,0.9,0.9}{
            \makecell[{{p{\linewidth}}}]{
                \texttt{\tiny{[GM$|$GM]}}
                \texttt{* success: True} \\
\texttt{* lose: False} \\
\texttt{* aborted: False} \\
\texttt{{-}{-}{-}{-}{-}{-}{-}} \\
\texttt{* Shifts: 8.00} \\
\texttt{* Max Shifts: 8.00} \\
\texttt{* Min Shifts: 4.00} \\
\texttt{* End Distance Sum: 0.00} \\
\texttt{* Init Distance Sum: 23.66} \\
\texttt{* Expected Distance Sum: 20.95} \\
\texttt{* Penalties: 3.00} \\
\texttt{* Max Penalties: 12.00} \\
\texttt{* Rounds: 9.00} \\
\texttt{* Max Rounds: 20.00} \\
\texttt{* Object Count: 5.00} \\
            }
        }
    }
    & & \\ \\

    \theutterance \stepcounter{utterance}  
    & & & \multicolumn{2}{p{0.3\linewidth}}{
        \cellcolor[rgb]{0.9,0.9,0.9}{
            \makecell[{{p{\linewidth}}}]{
                \texttt{\tiny{[GM$|$GM]}}
                \texttt{Initial Grids:} \\
\texttt{```} \\
\texttt{Player 1:     Player 2:} \\
\texttt{ 1234567       1234567} \\
\texttt{╔═══╤═╤═╗     ╔═══╤═╤═╗} \\
\texttt{║C◌◌│◌│◌║ 1   ║◌◌◌│C│◌║ 1} \\
\texttt{║◌I◌│◌├─╢ 2   ║◌W◌│◌├─╢ 2} \\
\texttt{║◌◌◌├─┘◌║ 3   ║◌◌◌├─┘◌║ 3} \\
\texttt{║◌◌◌│◌◌◌║ 4   ║◌◌◌│◌◌H║ 4} \\
\texttt{╟─┬─┼───╢ 5   ╟─┬─┼───╢ 5} \\
\texttt{║◌│◌│◌WT║ 6   ║T│◌│I◌◌║ 6} \\
\texttt{║◌│◌│◌◌H║ 7   ║◌│◌│◌◌◌║ 7} \\
\texttt{╚═╧═╧═══╝     ╚═╧═╧═══╝} \\
\texttt{```} \\
\\ 
\texttt{Final Grids:} \\
\texttt{```} \\
\texttt{Player 1:     Player 2:} \\
\texttt{ 1234567       1234567} \\
\texttt{╔═══╤═╤═╗     ╔═══╤═╤═╗} \\
\texttt{║C◌◌│◌│◌║ 1   ║C◌◌│◌│◌║ 1} \\
\texttt{║W◌◌│◌├─╢ 2   ║W◌◌│◌├─╢ 2} \\
\texttt{║◌IH├─┘◌║ 3   ║◌IH├─┘◌║ 3} \\
\texttt{║T◌◌│◌◌◌║ 4   ║T◌◌│◌◌◌║ 4} \\
\texttt{╟─┬─┼───╢ 5   ╟─┬─┼───╢ 5} \\
\texttt{║◌│◌│◌◌◌║ 6   ║◌│◌│◌◌◌║ 6} \\
\texttt{║◌│◌│◌◌◌║ 7   ║◌│◌│◌◌◌║ 7} \\
\texttt{╚═╧═╧═══╝     ╚═╧═╧═══╝} \\
\texttt{```} \\
            }
        }
    }
    & & \\ \\

\end{supertabular}
}

\end{document}
