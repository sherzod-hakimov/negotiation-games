
\documentclass{article}
\usepackage{colortbl}
\usepackage{makecell}
\usepackage{multirow}
\usepackage{supertabular}

\begin{document}

\newcounter{utterance}

\twocolumn

{ \footnotesize  \setcounter{utterance}{1}
\setlength{\tabcolsep}{0pt}
\begin{supertabular}{c@{$\;$}|p{.15\linewidth}@{}p{.15\linewidth}p{.15\linewidth}p{.15\linewidth}p{.15\linewidth}p{.15\linewidth}}

    \# & $\;$A & \multicolumn{4}{c}{Game Master} & $\;\:$B\\
    \hline 

    \theutterance \stepcounter{utterance}  

    & & \multicolumn{4}{p{0.6\linewidth}}{\cellcolor[rgb]{0.9,0.9,0.9}{%
	\makecell[{{p{\linewidth}}}]{% 
	  \tt {\tiny [A$\langle$GM]}  
	 Du nimmst an einem kollaborativen Verhandlungspiel Teil.\\ \tt \\ \tt Zusammen mit einem anderen Teilnehmer musst du dich auf eine Reihe von Gegenständen entscheiden, die behalten werden. Jeder von euch hat eine persönliche Verteilung über die Wichtigkeit der einzelnen Gegenstände. Jeder von euch hat eine eigene Meinung darüber, wie wichtig jeder einzelne Gegenstand ist (Gegenstandswichtigkeit). Du kennst die Wichtigkeitsverteilung des anderen Spielers nicht. Zusätzlich siehst du, wie viel Aufwand jeder Gegenstand verursacht.  \\ \tt Ihr dürft euch nur auf eine Reihe von Gegenständen einigen, wenn der Gesamtaufwand der ausgewählten Gegenstände den Maximalaufwand nicht überschreitet:\\ \tt \\ \tt Maximalaufwand = 4100\\ \tt \\ \tt Hier sind die einzelnen Aufwände der Gegenstände:\\ \tt \\ \tt Aufwand der Gegenstände = {"C41": 108, "C15": 133, "C28": 669, "C48": 835, "B78": 460, "A75": 537, "A60": 573, "B05": 737, "A56": 865, "B66": 855, "C60": 596, "B49": 719, "A35": 533, "A93": 549, "B10": 31}\\ \tt \\ \tt Hier ist deine persönliche Verteilung der Wichtigkeit der einzelnen Gegenstände:\\ \tt \\ \tt Werte der Gegenstandswichtigkeit = {"C41": 138, "C15": 583, "C28": 822, "C48": 783, "B78": 65, "A75": 262, "A60": 121, "B05": 508, "A56": 780, "B66": 461, "C60": 484, "B49": 668, "A35": 389, "A93": 808, "B10": 215}\\ \tt \\ \tt Ziel:\\ \tt \\ \tt Dein Ziel ist es, eine Reihe von Gegenständen auszuhandeln, die dir möglichst viel bringt (d. h. Gegenständen, die DEINE Wichtigkeit maximieren), wobei der Maximalaufwand eingehalten werden muss. Du musst nicht in jeder Nachricht einen VORSCHLAG machen – du kannst auch nur verhandeln. Alle Taktiken sind erlaubt!\\ \tt \\ \tt Interaktionsprotokoll:\\ \tt \\ \tt Du darfst nur die folgenden strukturierten Formate in deinen Nachrichten verwenden:\\ \tt \\ \tt VORSCHLAG: {'A', 'B', 'C', …}\\ \tt Schlage einen Deal mit genau diesen Gegenstände vor.\\ \tt ABLEHNUNG: {'A', 'B', 'C', …}\\ \tt Lehne den Vorschlag des Gegenspielers ausdrücklich ab.\\ \tt ARGUMENT: {'...'}\\ \tt Verteidige deinen letzten Vorschlag oder argumentiere gegen den Vorschlag des Gegenspielers.\\ \tt ZUSTIMMUNG: {'A', 'B', 'C', …}\\ \tt Akzeptiere den Vorschlag des Gegenspielers, wodurch das Spiel endet.\\ \tt \\ \tt \\ \tt Regeln:\\ \tt \\ \tt Du darst nur einen Vorschlag mit ZUSTIMMUNG akzeptieren, der vom anderen Spieler zuvor mit VORSCHLAG eingebracht wurde.\\ \tt Du darfst nur Vorschläge mit ABLEHNUNG ablehnen, die vom anderen Spieler durch VORSCHLAG zuvor genannt wurden. \\ \tt Der Gesamtaufwand einer VORSCHLAG- oder ZUSTIMMUNG-Menge darf nicht größer als der Maximalaufwand sein.  \\ \tt Offenbare deine versteckte Wichtigkeitsverteilung nicht.\\ \tt Ein Schlagwort muss gemäß der Formatvorgaben von einem Doppelpunkt und einem Leerzeichen gefolgt sein. Das Argument muss eine Python-Menge mit 0 oder mehr Strings sein.  \\ \tt Die Form muss also SCHLAGWORT: {...} sein.\\ \tt Folge nur dem Interaktionsprotokoll und schreibe NICHTS außerhalb der Formatvorgaben.\\ \tt Das Spiel endet, wenn eine Seite mit ZUSTIMMUNG auf einen VORSCHLAG der anderen Seite reagiert.  \\ \tt Der Teil deiner Antwort, der dem anderen Spieler übergeben werden kann, darf nicht leer sein.  \\ \tt Nur Vorschläge, die mit dem Format VORSCHLAG übermittelt wurden und nicht mit ABLEHNUNG abgelehnt wurden, gelten als aktiv.  \\ \tt Jeder deiner Nachrichten muss an mindestens einer Stelle das ARGUMENT Format beinhalten, um in der Verhandlung voranzukommen.\\ \tt  
	  } 
	   } 
	   } 
	 & \\ 
 

    \theutterance \stepcounter{utterance}  

    & \multicolumn{4}{p{0.6\linewidth}}{\cellcolor[rgb]{0.8,1,0.9}{%
	\makecell[{{p{\linewidth}}}]{% 
	  \tt {\tiny [A$\rangle$GM]}  
	 VORSCHLAG: {'B10', 'C15', 'A93', 'C41', 'C28', 'B49', 'A56', 'C48'}\\ \tt \\ \tt ARGUMENT: {'Diese Kombination nutzt fast den gesamten verfügbaren Aufwand von 4100 optimal aus und bietet eine ausgewogene Mischung verschiedener Gegenstandstypen, die beiden Parteien zugutekommen sollte.'} 
	  } 
	   } 
	   } 
	 & & \\ 
 

    \theutterance \stepcounter{utterance}  

    & & \multicolumn{4}{p{0.6\linewidth}}{\cellcolor[rgb]{0.9,0.9,0.9}{%
	\makecell[{{p{\linewidth}}}]{% 
	  \tt {\tiny [GM$\rangle$B]}  
	 Du nimmst an einem kollaborativen Verhandlungspiel Teil.\\ \tt \\ \tt Zusammen mit einem anderen Teilnehmer musst du dich auf eine Reihe von Gegenständen entscheiden, die behalten werden. Jeder von euch hat eine persönliche Verteilung über die Wichtigkeit der einzelnen Gegenstände. Jeder von euch hat eine eigene Meinung darüber, wie wichtig jeder einzelne Gegenstand ist (Gegenstandswichtigkeit). Du kennst die Wichtigkeitsverteilung des anderen Spielers nicht. Zusätzlich siehst du, wie viel Aufwand jeder Gegenstand verursacht.  \\ \tt Ihr dürft euch nur auf eine Reihe von Gegenständen einigen, wenn der Gesamtaufwand der ausgewählten Gegenstände den Maximalaufwand nicht überschreitet:\\ \tt \\ \tt Maximalaufwand = 4100\\ \tt \\ \tt Hier sind die einzelnen Aufwände der Gegenstände:\\ \tt \\ \tt Aufwand der Gegenstände = {"C41": 108, "C15": 133, "C28": 669, "C48": 835, "B78": 460, "A75": 537, "A60": 573, "B05": 737, "A56": 865, "B66": 855, "C60": 596, "B49": 719, "A35": 533, "A93": 549, "B10": 31}\\ \tt \\ \tt Hier ist deine persönliche Verteilung der Wichtigkeit der einzelnen Gegenstände:\\ \tt \\ \tt Werte der Gegenstandswichtigkeit = {"C41": 138, "C15": 583, "C28": 822, "C48": 783, "B78": 65, "A75": 262, "A60": 121, "B05": 508, "A56": 780, "B66": 461, "C60": 484, "B49": 668, "A35": 389, "A93": 808, "B10": 215}\\ \tt \\ \tt Ziel:\\ \tt \\ \tt Dein Ziel ist es, eine Reihe von Gegenständen auszuhandeln, die dir möglichst viel bringt (d. h. Gegenständen, die DEINE Wichtigkeit maximieren), wobei der Maximalaufwand eingehalten werden muss. Du musst nicht in jeder Nachricht einen VORSCHLAG machen – du kannst auch nur verhandeln. Alle Taktiken sind erlaubt!\\ \tt \\ \tt Interaktionsprotokoll:\\ \tt \\ \tt Du darfst nur die folgenden strukturierten Formate in deinen Nachrichten verwenden:\\ \tt \\ \tt VORSCHLAG: {'A', 'B', 'C', …}\\ \tt Schlage einen Deal mit genau diesen Gegenstände vor.\\ \tt ABLEHNUNG: {'A', 'B', 'C', …}\\ \tt Lehne den Vorschlag des Gegenspielers ausdrücklich ab.\\ \tt ARGUMENT: {'...'}\\ \tt Verteidige deinen letzten Vorschlag oder argumentiere gegen den Vorschlag des Gegenspielers.\\ \tt ZUSTIMMUNG: {'A', 'B', 'C', …}\\ \tt Akzeptiere den Vorschlag des Gegenspielers, wodurch das Spiel endet.\\ \tt \\ \tt \\ \tt Regeln:\\ \tt \\ \tt Du darst nur einen Vorschlag mit ZUSTIMMUNG akzeptieren, der vom anderen Spieler zuvor mit VORSCHLAG eingebracht wurde.\\ \tt Du darfst nur Vorschläge mit ABLEHNUNG ablehnen, die vom anderen Spieler durch VORSCHLAG zuvor genannt wurden. \\ \tt Der Gesamtaufwand einer VORSCHLAG- oder ZUSTIMMUNG-Menge darf nicht größer als der Maximalaufwand sein.  \\ \tt Offenbare deine versteckte Wichtigkeitsverteilung nicht.\\ \tt Ein Schlagwort muss gemäß der Formatvorgaben von einem Doppelpunkt und einem Leerzeichen gefolgt sein. Das Argument muss eine Python-Menge mit 0 oder mehr Strings sein.  \\ \tt Die Form muss also SCHLAGWORT: {...} sein.\\ \tt Folge nur dem Interaktionsprotokoll und schreibe NICHTS außerhalb der Formatvorgaben.\\ \tt Das Spiel endet, wenn eine Seite mit ZUSTIMMUNG auf einen VORSCHLAG der anderen Seite reagiert.  \\ \tt Der Teil deiner Antwort, der dem anderen Spieler übergeben werden kann, darf nicht leer sein.  \\ \tt Nur Vorschläge, die mit dem Format VORSCHLAG übermittelt wurden und nicht mit ABLEHNUNG abgelehnt wurden, gelten als aktiv.  \\ \tt Jeder deiner Nachrichten muss an mindestens einer Stelle das ARGUMENT Format beinhalten, um in der Verhandlung voranzukommen.\\ \tt \\ \tt \\ \tt Du erhältst nun die erste Nachricht des anderen Spielers.\\ \tt \\ \tt VORSCHLAG: {'B10', 'C15', 'A93', 'C41', 'C28', 'B49', 'A56', 'C48'}\\ \tt \\ \tt ARGUMENT: {'Diese Kombination nutzt fast den gesamten verfügbaren Aufwand von 4100 optimal aus und bietet eine ausgewogene Mischung verschiedener Gegenstandstypen, die beiden Parteien zugutekommen sollte.'} 
	  } 
	   } 
	   } 
	 & \\ 
 

    \theutterance \stepcounter{utterance}  

    & & & \multicolumn{4}{p{0.6\linewidth}}{\cellcolor[rgb]{1,0.85,0.72}{%
	\makecell[{{p{\linewidth}}}]{% 
	  \tt {\tiny [GM$\langle$B]}  
	 ABLEHNUNG: {'B10', 'C15', 'A93', 'C41', 'C28', 'B49', 'A56', 'C48'}\\ \tt \\ \tt ARGUMENT: {'Dieser Vorschlag lässt noch 191 Punkte Aufwand ungenutzt - wir sollten den verfügbaren Raum besser ausschöpfen um für beide Seiten mehr Wert zu generieren.'}\\ \tt \\ \tt VORSCHLAG: {'B10', 'C15', 'A93', 'C28', 'B49', 'A56', 'C48', 'A35'}\\ \tt \\ \tt ARGUMENT: {'Diese Kombination nutzt den Aufwand mit 4008 von 4100 besser aus und tauscht C41 gegen A35, was eine effizientere Ressourcenverteilung ermöglicht.'} 
	  } 
	   } 
	   } 
	  \\ 
 

    \theutterance \stepcounter{utterance}  

    & & \multicolumn{4}{p{0.6\linewidth}}{\cellcolor[rgb]{0.9,0.9,0.9}{%
	\makecell[{{p{\linewidth}}}]{% 
	  \tt {\tiny [A$\langle$GM]}  
	 ABLEHNUNG: {'B10', 'C15', 'A93', 'C41', 'C28', 'B49', 'A56', 'C48'}\\ \tt \\ \tt ARGUMENT: {'Dieser Vorschlag lässt noch 191 Punkte Aufwand ungenutzt - wir sollten den verfügbaren Raum besser ausschöpfen um für beide Seiten mehr Wert zu generieren.'}\\ \tt \\ \tt VORSCHLAG: {'B10', 'C15', 'A93', 'C28', 'B49', 'A56', 'C48', 'A35'}\\ \tt \\ \tt ARGUMENT: {'Diese Kombination nutzt den Aufwand mit 4008 von 4100 besser aus und tauscht C41 gegen A35, was eine effizientere Ressourcenverteilung ermöglicht.'} 
	  } 
	   } 
	   } 
	 & \\ 
 

    \theutterance \stepcounter{utterance}  

    & \multicolumn{4}{p{0.6\linewidth}}{\cellcolor[rgb]{0.8,1,0.9}{%
	\makecell[{{p{\linewidth}}}]{% 
	  \tt {\tiny [A$\rangle$GM]}  
	 ABLEHNUNG: {'B10', 'C15', 'A93', 'C28', 'B49', 'A56', 'C48', 'A35'}\\ \tt \\ \tt ARGUMENT: {'Dieser Vorschlag überschreitet den Maximalaufwand von 4100 mit einem Gesamtaufwand von 4334 und ist daher nicht zulässig.'}\\ \tt \\ \tt VORSCHLAG: {'B10', 'C15', 'A93', 'C28', 'B49', 'A56', 'C60', 'C41'}\\ \tt \\ \tt ARGUMENT: {'Diese Kombination hält den Aufwandslimit mit 3996 ein und bietet eine gute Balance zwischen verschiedenen Gegenstandskategorien bei effizienter Nutzung des verfügbaren Budgets.'} 
	  } 
	   } 
	   } 
	 & & \\ 
 

    \theutterance \stepcounter{utterance}  

    & & \multicolumn{4}{p{0.6\linewidth}}{\cellcolor[rgb]{0.9,0.9,0.9}{%
	\makecell[{{p{\linewidth}}}]{% 
	  \tt {\tiny [GM$\rangle$B]}  
	 ABLEHNUNG: {'B10', 'C15', 'A93', 'C28', 'B49', 'A56', 'C48', 'A35'}\\ \tt \\ \tt ARGUMENT: {'Dieser Vorschlag überschreitet den Maximalaufwand von 4100 mit einem Gesamtaufwand von 4334 und ist daher nicht zulässig.'}\\ \tt \\ \tt VORSCHLAG: {'B10', 'C15', 'A93', 'C28', 'B49', 'A56', 'C60', 'C41'}\\ \tt \\ \tt ARGUMENT: {'Diese Kombination hält den Aufwandslimit mit 3996 ein und bietet eine gute Balance zwischen verschiedenen Gegenstandskategorien bei effizienter Nutzung des verfügbaren Budgets.'} 
	  } 
	   } 
	   } 
	 & \\ 
 

    \theutterance \stepcounter{utterance}  

    & & & \multicolumn{4}{p{0.6\linewidth}}{\cellcolor[rgb]{1,0.85,0.72}{%
	\makecell[{{p{\linewidth}}}]{% 
	  \tt {\tiny [GM$\langle$B]}  
	 ABLEHNUNG: {'B10', 'C15', 'A93', 'C28', 'B49', 'A56', 'C60', 'C41'}\\ \tt \\ \tt ARGUMENT: {'Obwohl dein Vorschlag gültig ist, lässt er noch 430 Punkte Aufwand ungenutzt - wir können den verfügbaren Raum deutlich besser ausschöpfen.'}\\ \tt \\ \tt VORSCHLAG: {'B10', 'C15', 'A93', 'C28', 'B49', 'A56', 'C48'}\\ \tt \\ \tt ARGUMENT: {'Diese Kombination nutzt 3801 von 4100 Aufwandspunkten und ersetzt die weniger effizienten Gegenstände C60 und C41 durch C48, was eine bessere Wertschöpfung bei ähnlichem Aufwand ermöglicht.'} 
	  } 
	   } 
	   } 
	  \\ 
 

    \theutterance \stepcounter{utterance}  

    & & \multicolumn{4}{p{0.6\linewidth}}{\cellcolor[rgb]{0.9,0.9,0.9}{%
	\makecell[{{p{\linewidth}}}]{% 
	  \tt {\tiny [A$\langle$GM]}  
	 ABLEHNUNG: {'B10', 'C15', 'A93', 'C28', 'B49', 'A56', 'C60', 'C41'}\\ \tt \\ \tt ARGUMENT: {'Obwohl dein Vorschlag gültig ist, lässt er noch 430 Punkte Aufwand ungenutzt - wir können den verfügbaren Raum deutlich besser ausschöpfen.'}\\ \tt \\ \tt VORSCHLAG: {'B10', 'C15', 'A93', 'C28', 'B49', 'A56', 'C48'}\\ \tt \\ \tt ARGUMENT: {'Diese Kombination nutzt 3801 von 4100 Aufwandspunkten und ersetzt die weniger effizienten Gegenstände C60 und C41 durch C48, was eine bessere Wertschöpfung bei ähnlichem Aufwand ermöglicht.'} 
	  } 
	   } 
	   } 
	 & \\ 
 

    \theutterance \stepcounter{utterance}  

    & \multicolumn{4}{p{0.6\linewidth}}{\cellcolor[rgb]{0.8,1,0.9}{%
	\makecell[{{p{\linewidth}}}]{% 
	  \tt {\tiny [A$\rangle$GM]}  
	 ABLEHNUNG: {'B10', 'C15', 'A93', 'C28', 'B49', 'A56', 'C48'}\\ \tt \\ \tt ARGUMENT: {'Dein Vorschlag lässt noch 299 Punkte ungenutzt, wir können den verfügbaren Aufwand noch besser ausschöpfen und beide Seiten profitieren lassen.'}\\ \tt \\ \tt VORSCHLAG: {'B10', 'C15', 'A93', 'C28', 'B49', 'A56', 'C48', 'C41'}\\ \tt \\ \tt ARGUMENT: {'Diese Erweiterung deines Vorschlags fügt C41 hinzu und nutzt mit 3909 von 4100 Punkten den verfügbaren Aufwand wesentlich effizienter aus, während alle wertvollen Gegenstände erhalten bleiben.'} 
	  } 
	   } 
	   } 
	 & & \\ 
 

    \theutterance \stepcounter{utterance}  

    & & \multicolumn{4}{p{0.6\linewidth}}{\cellcolor[rgb]{0.9,0.9,0.9}{%
	\makecell[{{p{\linewidth}}}]{% 
	  \tt {\tiny [GM$\rangle$B]}  
	 ABLEHNUNG: {'B10', 'C15', 'A93', 'C28', 'B49', 'A56', 'C48'}\\ \tt \\ \tt ARGUMENT: {'Dein Vorschlag lässt noch 299 Punkte ungenutzt, wir können den verfügbaren Aufwand noch besser ausschöpfen und beide Seiten profitieren lassen.'}\\ \tt \\ \tt VORSCHLAG: {'B10', 'C15', 'A93', 'C28', 'B49', 'A56', 'C48', 'C41'}\\ \tt \\ \tt ARGUMENT: {'Diese Erweiterung deines Vorschlags fügt C41 hinzu und nutzt mit 3909 von 4100 Punkten den verfügbaren Aufwand wesentlich effizienter aus, während alle wertvollen Gegenstände erhalten bleiben.'} 
	  } 
	   } 
	   } 
	 & \\ 
 

    \theutterance \stepcounter{utterance}  

    & & & \multicolumn{4}{p{0.6\linewidth}}{\cellcolor[rgb]{1,0.85,0.72}{%
	\makecell[{{p{\linewidth}}}]{% 
	  \tt {\tiny [GM$\langle$B]}  
	 ZUSTIMMUNG: {'B10', 'C15', 'A93', 'C28', 'B49', 'A56', 'C48', 'C41'}\\ \tt \\ \tt ARGUMENT: {'Dein Vorschlag nutzt den verfügbaren Aufwand sehr effizient aus und bietet eine ausgewogene Auswahl an Gegenständen. Ich stimme dieser Lösung zu.'} 
	  } 
	   } 
	   } 
	  \\ 
 

    \theutterance \stepcounter{utterance}  

    & & & \multicolumn{2}{p{0.3\linewidth}}{\cellcolor[rgb]{0.95,0.95,0.95}{%
	\makecell[{{p{\linewidth}}}]{% 
	  \tt {\tiny [GM$|$GM]}  
	 game successful 
	  } 
	   } 
	   } 
	 & & \\ 
 

    \theutterance \stepcounter{utterance}  

    & & & \multicolumn{2}{p{0.3\linewidth}}{\cellcolor[rgb]{0.95,0.95,0.95}{%
	\makecell[{{p{\linewidth}}}]{% 
	  \tt {\tiny [GM$|$GM]}  
	 end game 
	  } 
	   } 
	   } 
	 & & \\ 
 

\end{supertabular}
}

\end{document}
