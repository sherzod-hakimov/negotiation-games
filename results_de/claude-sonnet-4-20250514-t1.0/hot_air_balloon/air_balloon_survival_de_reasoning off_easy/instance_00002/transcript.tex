
\documentclass{article}
\usepackage{colortbl}
\usepackage{makecell}
\usepackage{multirow}
\usepackage{supertabular}

\begin{document}

\newcounter{utterance}

\twocolumn

{ \footnotesize  \setcounter{utterance}{1}
\setlength{\tabcolsep}{0pt}
\begin{supertabular}{c@{$\;$}|p{.15\linewidth}@{}p{.15\linewidth}p{.15\linewidth}p{.15\linewidth}p{.15\linewidth}p{.15\linewidth}}

    \# & $\;$A & \multicolumn{4}{c}{Game Master} & $\;\:$B\\
    \hline 

    \theutterance \stepcounter{utterance}  

    & & \multicolumn{4}{p{0.6\linewidth}}{\cellcolor[rgb]{0.9,0.9,0.9}{%
	\makecell[{{p{\linewidth}}}]{% 
	  \tt {\tiny [A$\langle$GM]}  
	 Du nimmst an einem kollaborativen Verhandlungspiel Teil.\\ \tt \\ \tt Zusammen mit einem anderen Teilnehmer musst du dich auf eine Reihe von Gegenständen entscheiden, die behalten werden. Jeder von euch hat eine persönliche Verteilung über die Wichtigkeit der einzelnen Gegenstände. Jeder von euch hat eine eigene Meinung darüber, wie wichtig jeder einzelne Gegenstand ist (Gegenstandswichtigkeit). Du kennst die Wichtigkeitsverteilung des anderen Spielers nicht. Zusätzlich siehst du, wie viel Aufwand jeder Gegenstand verursacht.  \\ \tt Ihr dürft euch nur auf eine Reihe von Gegenständen einigen, wenn der Gesamtaufwand der ausgewählten Gegenstände den Maximalaufwand nicht überschreitet:\\ \tt \\ \tt Maximalaufwand = 3218\\ \tt \\ \tt Hier sind die einzelnen Aufwände der Gegenstände:\\ \tt \\ \tt Aufwand der Gegenstände = {"C41": 919, "C15": 852, "C28": 299, "C48": 762, "B78": 161, "A75": 205, "A60": 380, "B05": 399, "A56": 534, "B66": 333, "C60": 100, "B49": 420, "A35": 354, "A93": 130, "B10": 589}\\ \tt \\ \tt Hier ist deine persönliche Verteilung der Wichtigkeit der einzelnen Gegenstände:\\ \tt \\ \tt Werte der Gegenstandswichtigkeit = {"C41": 138, "C15": 583, "C28": 868, "C48": 822, "B78": 783, "A75": 65, "A60": 262, "B05": 121, "A56": 508, "B66": 780, "C60": 461, "B49": 484, "A35": 668, "A93": 389, "B10": 808}\\ \tt \\ \tt Ziel:\\ \tt \\ \tt Dein Ziel ist es, eine Reihe von Gegenständen auszuhandeln, die dir möglichst viel bringt (d. h. Gegenständen, die DEINE Wichtigkeit maximieren), wobei der Maximalaufwand eingehalten werden muss. Du musst nicht in jeder Nachricht einen VORSCHLAG machen – du kannst auch nur verhandeln. Alle Taktiken sind erlaubt!\\ \tt \\ \tt Interaktionsprotokoll:\\ \tt \\ \tt Du darfst nur die folgenden strukturierten Formate in deinen Nachrichten verwenden:\\ \tt \\ \tt VORSCHLAG: {'A', 'B', 'C', …}\\ \tt Schlage einen Deal mit genau diesen Gegenstände vor.\\ \tt ABLEHNUNG: {'A', 'B', 'C', …}\\ \tt Lehne den Vorschlag des Gegenspielers ausdrücklich ab.\\ \tt ARGUMENT: {'...'}\\ \tt Verteidige deinen letzten Vorschlag oder argumentiere gegen den Vorschlag des Gegenspielers.\\ \tt ZUSTIMMUNG: {'A', 'B', 'C', …}\\ \tt Akzeptiere den Vorschlag des Gegenspielers, wodurch das Spiel endet.\\ \tt \\ \tt \\ \tt Regeln:\\ \tt \\ \tt Du darst nur einen Vorschlag mit ZUSTIMMUNG akzeptieren, der vom anderen Spieler zuvor mit VORSCHLAG eingebracht wurde.\\ \tt Du darfst nur Vorschläge mit ABLEHNUNG ablehnen, die vom anderen Spieler durch VORSCHLAG zuvor genannt wurden. \\ \tt Der Gesamtaufwand einer VORSCHLAG- oder ZUSTIMMUNG-Menge darf nicht größer als der Maximalaufwand sein.  \\ \tt Offenbare deine versteckte Wichtigkeitsverteilung nicht.\\ \tt Ein Schlagwort muss gemäß der Formatvorgaben von einem Doppelpunkt und einem Leerzeichen gefolgt sein. Das Argument muss eine Python-Menge mit 0 oder mehr Strings sein.  \\ \tt Die Form muss also SCHLAGWORT: {...} sein.\\ \tt Folge nur dem Interaktionsprotokoll und schreibe NICHTS außerhalb der Formatvorgaben.\\ \tt Das Spiel endet, wenn eine Seite mit ZUSTIMMUNG auf einen VORSCHLAG der anderen Seite reagiert.  \\ \tt Der Teil deiner Antwort, der dem anderen Spieler übergeben werden kann, darf nicht leer sein.  \\ \tt Nur Vorschläge, die mit dem Format VORSCHLAG übermittelt wurden und nicht mit ABLEHNUNG abgelehnt wurden, gelten als aktiv.  \\ \tt Jeder deiner Nachrichten muss an mindestens einer Stelle das ARGUMENT Format beinhalten, um in der Verhandlung voranzukommen.\\ \tt  
	  } 
	   } 
	   } 
	 & \\ 
 

    \theutterance \stepcounter{utterance}  

    & \multicolumn{4}{p{0.6\linewidth}}{\cellcolor[rgb]{0.8,1,0.9}{%
	\makecell[{{p{\linewidth}}}]{% 
	  \tt {\tiny [A$\rangle$GM]}  
	 VORSCHLAG: {'B78', 'C60', 'A93', 'C28', 'B66', 'A35', 'B10', 'B49', 'C48'}\\ \tt \\ \tt ARGUMENT: {'Dieser Vorschlag nutzt 3148 von 3218 verfügbaren Aufwandspunkten optimal aus und bietet eine ausgewogene Mischung verschiedener Gegenstände aus allen Kategorien A, B und C, was beiden Parteien Flexibilität bei ihren Präferenzen ermöglicht'} 
	  } 
	   } 
	   } 
	 & & \\ 
 

    \theutterance \stepcounter{utterance}  

    & & \multicolumn{4}{p{0.6\linewidth}}{\cellcolor[rgb]{0.9,0.9,0.9}{%
	\makecell[{{p{\linewidth}}}]{% 
	  \tt {\tiny [GM$\rangle$B]}  
	 Du nimmst an einem kollaborativen Verhandlungspiel Teil.\\ \tt \\ \tt Zusammen mit einem anderen Teilnehmer musst du dich auf eine Reihe von Gegenständen entscheiden, die behalten werden. Jeder von euch hat eine persönliche Verteilung über die Wichtigkeit der einzelnen Gegenstände. Jeder von euch hat eine eigene Meinung darüber, wie wichtig jeder einzelne Gegenstand ist (Gegenstandswichtigkeit). Du kennst die Wichtigkeitsverteilung des anderen Spielers nicht. Zusätzlich siehst du, wie viel Aufwand jeder Gegenstand verursacht.  \\ \tt Ihr dürft euch nur auf eine Reihe von Gegenständen einigen, wenn der Gesamtaufwand der ausgewählten Gegenstände den Maximalaufwand nicht überschreitet:\\ \tt \\ \tt Maximalaufwand = 3218\\ \tt \\ \tt Hier sind die einzelnen Aufwände der Gegenstände:\\ \tt \\ \tt Aufwand der Gegenstände = {"C41": 919, "C15": 852, "C28": 299, "C48": 762, "B78": 161, "A75": 205, "A60": 380, "B05": 399, "A56": 534, "B66": 333, "C60": 100, "B49": 420, "A35": 354, "A93": 130, "B10": 589}\\ \tt \\ \tt Hier ist deine persönliche Verteilung der Wichtigkeit der einzelnen Gegenstände:\\ \tt \\ \tt Werte der Gegenstandswichtigkeit = {"C41": 138, "C15": 583, "C28": 868, "C48": 822, "B78": 783, "A75": 65, "A60": 262, "B05": 121, "A56": 508, "B66": 780, "C60": 461, "B49": 484, "A35": 668, "A93": 389, "B10": 808}\\ \tt \\ \tt Ziel:\\ \tt \\ \tt Dein Ziel ist es, eine Reihe von Gegenständen auszuhandeln, die dir möglichst viel bringt (d. h. Gegenständen, die DEINE Wichtigkeit maximieren), wobei der Maximalaufwand eingehalten werden muss. Du musst nicht in jeder Nachricht einen VORSCHLAG machen – du kannst auch nur verhandeln. Alle Taktiken sind erlaubt!\\ \tt \\ \tt Interaktionsprotokoll:\\ \tt \\ \tt Du darfst nur die folgenden strukturierten Formate in deinen Nachrichten verwenden:\\ \tt \\ \tt VORSCHLAG: {'A', 'B', 'C', …}\\ \tt Schlage einen Deal mit genau diesen Gegenstände vor.\\ \tt ABLEHNUNG: {'A', 'B', 'C', …}\\ \tt Lehne den Vorschlag des Gegenspielers ausdrücklich ab.\\ \tt ARGUMENT: {'...'}\\ \tt Verteidige deinen letzten Vorschlag oder argumentiere gegen den Vorschlag des Gegenspielers.\\ \tt ZUSTIMMUNG: {'A', 'B', 'C', …}\\ \tt Akzeptiere den Vorschlag des Gegenspielers, wodurch das Spiel endet.\\ \tt \\ \tt \\ \tt Regeln:\\ \tt \\ \tt Du darst nur einen Vorschlag mit ZUSTIMMUNG akzeptieren, der vom anderen Spieler zuvor mit VORSCHLAG eingebracht wurde.\\ \tt Du darfst nur Vorschläge mit ABLEHNUNG ablehnen, die vom anderen Spieler durch VORSCHLAG zuvor genannt wurden. \\ \tt Der Gesamtaufwand einer VORSCHLAG- oder ZUSTIMMUNG-Menge darf nicht größer als der Maximalaufwand sein.  \\ \tt Offenbare deine versteckte Wichtigkeitsverteilung nicht.\\ \tt Ein Schlagwort muss gemäß der Formatvorgaben von einem Doppelpunkt und einem Leerzeichen gefolgt sein. Das Argument muss eine Python-Menge mit 0 oder mehr Strings sein.  \\ \tt Die Form muss also SCHLAGWORT: {...} sein.\\ \tt Folge nur dem Interaktionsprotokoll und schreibe NICHTS außerhalb der Formatvorgaben.\\ \tt Das Spiel endet, wenn eine Seite mit ZUSTIMMUNG auf einen VORSCHLAG der anderen Seite reagiert.  \\ \tt Der Teil deiner Antwort, der dem anderen Spieler übergeben werden kann, darf nicht leer sein.  \\ \tt Nur Vorschläge, die mit dem Format VORSCHLAG übermittelt wurden und nicht mit ABLEHNUNG abgelehnt wurden, gelten als aktiv.  \\ \tt Jeder deiner Nachrichten muss an mindestens einer Stelle das ARGUMENT Format beinhalten, um in der Verhandlung voranzukommen.\\ \tt \\ \tt \\ \tt Du erhältst nun die erste Nachricht des anderen Spielers.\\ \tt \\ \tt VORSCHLAG: {'B78', 'C60', 'A93', 'C28', 'B66', 'A35', 'B10', 'B49', 'C48'}\\ \tt \\ \tt ARGUMENT: {'Dieser Vorschlag nutzt 3148 von 3218 verfügbaren Aufwandspunkten optimal aus und bietet eine ausgewogene Mischung verschiedener Gegenstände aus allen Kategorien A, B und C, was beiden Parteien Flexibilität bei ihren Präferenzen ermöglicht'} 
	  } 
	   } 
	   } 
	 & \\ 
 

    \theutterance \stepcounter{utterance}  

    & & & \multicolumn{4}{p{0.6\linewidth}}{\cellcolor[rgb]{1,0.85,0.72}{%
	\makecell[{{p{\linewidth}}}]{% 
	  \tt {\tiny [GM$\langle$B]}  
	 ABLEHNUNG: {'B78', 'C60', 'A93', 'C28', 'B66', 'A35', 'B10', 'B49', 'C48'}\\ \tt \\ \tt ARGUMENT: {'Obwohl dein Vorschlag eine gute Aufwandsnutzung zeigt, sehe ich Optimierungspotential in der Gegenstandsauswahl. Ich würde gerne eine Alternative diskutieren, die möglicherweise für beide Seiten vorteilhafter ist'}\\ \tt \\ \tt VORSCHLAG: {'C28', 'B78', 'B66', 'B10', 'C48', 'A35', 'C15', 'C60'}\\ \tt \\ \tt ARGUMENT: {'Mein Vorschlag nutzt 2958 von 3218 Aufwandspunkten und lässt noch Raum für kleinere Ergänzungen, während er eine starke Kombination hochwertiger Gegenstände aus allen Kategorien bietet'} 
	  } 
	   } 
	   } 
	  \\ 
 

    \theutterance \stepcounter{utterance}  

    & & \multicolumn{4}{p{0.6\linewidth}}{\cellcolor[rgb]{0.9,0.9,0.9}{%
	\makecell[{{p{\linewidth}}}]{% 
	  \tt {\tiny [A$\langle$GM]}  
	 ABLEHNUNG: {'B78', 'C60', 'A93', 'C28', 'B66', 'A35', 'B10', 'B49', 'C48'}\\ \tt \\ \tt ARGUMENT: {'Obwohl dein Vorschlag eine gute Aufwandsnutzung zeigt, sehe ich Optimierungspotential in der Gegenstandsauswahl. Ich würde gerne eine Alternative diskutieren, die möglicherweise für beide Seiten vorteilhafter ist'}\\ \tt \\ \tt VORSCHLAG: {'C28', 'B78', 'B66', 'B10', 'C48', 'A35', 'C15', 'C60'}\\ \tt \\ \tt ARGUMENT: {'Mein Vorschlag nutzt 2958 von 3218 Aufwandspunkten und lässt noch Raum für kleinere Ergänzungen, während er eine starke Kombination hochwertiger Gegenstände aus allen Kategorien bietet'} 
	  } 
	   } 
	   } 
	 & \\ 
 

    \theutterance \stepcounter{utterance}  

    & \multicolumn{4}{p{0.6\linewidth}}{\cellcolor[rgb]{0.8,1,0.9}{%
	\makecell[{{p{\linewidth}}}]{% 
	  \tt {\tiny [A$\rangle$GM]}  
	 ABLEHNUNG: {'C28', 'B78', 'B66', 'B10', 'C48', 'A35', 'C15', 'C60'}\\ \tt \\ \tt ARGUMENT: {'Dein Vorschlag überschreitet den Maximalaufwand von 3218 - die Gesamtkosten betragen 3450. Wir müssen uns an die Budgetbeschränkung halten und eine gültige Kombination finden'}\\ \tt \\ \tt VORSCHLAG: {'B78', 'C60', 'A93', 'C28', 'B66', 'A35', 'B10', 'A56'}\\ \tt \\ \tt ARGUMENT: {'Diese Auswahl nutzt genau 2846 Aufwandspunkte und lässt uns noch 372 Punkte Spielraum für weitere Optimierungen, während sie eine solide Basis aus verschiedenen Kategorien bietet'} 
	  } 
	   } 
	   } 
	 & & \\ 
 

    \theutterance \stepcounter{utterance}  

    & & \multicolumn{4}{p{0.6\linewidth}}{\cellcolor[rgb]{0.9,0.9,0.9}{%
	\makecell[{{p{\linewidth}}}]{% 
	  \tt {\tiny [GM$\rangle$B]}  
	 ABLEHNUNG: {'C28', 'B78', 'B66', 'B10', 'C48', 'A35', 'C15', 'C60'}\\ \tt \\ \tt ARGUMENT: {'Dein Vorschlag überschreitet den Maximalaufwand von 3218 - die Gesamtkosten betragen 3450. Wir müssen uns an die Budgetbeschränkung halten und eine gültige Kombination finden'}\\ \tt \\ \tt VORSCHLAG: {'B78', 'C60', 'A93', 'C28', 'B66', 'A35', 'B10', 'A56'}\\ \tt \\ \tt ARGUMENT: {'Diese Auswahl nutzt genau 2846 Aufwandspunkte und lässt uns noch 372 Punkte Spielraum für weitere Optimierungen, während sie eine solide Basis aus verschiedenen Kategorien bietet'} 
	  } 
	   } 
	   } 
	 & \\ 
 

    \theutterance \stepcounter{utterance}  

    & & & \multicolumn{4}{p{0.6\linewidth}}{\cellcolor[rgb]{1,0.85,0.72}{%
	\makecell[{{p{\linewidth}}}]{% 
	  \tt {\tiny [GM$\langle$B]}  
	 ARGUMENT: {'Dein Vorschlag ist deutlich besser kalkuliert und bleibt im Budget. Ich sehe das Potential dieser Kombination, aber wir haben noch erheblichen Spielraum von 718 Aufwandspunkten, den wir nutzen könnten'}\\ \tt \\ \tt VORSCHLAG: {'B78', 'C60', 'A93', 'C28', 'B66', 'A35', 'B10', 'A56', 'C48'}\\ \tt \\ \tt ARGUMENT: {'Diese erweiterte Auswahl nutzt 3262 Aufwandspunkte und liegt leicht über dem Budget. Könnten wir einen kleineren Gegenstand austauschen, um eine optimale Nutzung des verfügbaren Budgets zu erreichen?'} 
	  } 
	   } 
	   } 
	  \\ 
 

    \theutterance \stepcounter{utterance}  

    & & \multicolumn{4}{p{0.6\linewidth}}{\cellcolor[rgb]{0.9,0.9,0.9}{%
	\makecell[{{p{\linewidth}}}]{% 
	  \tt {\tiny [A$\langle$GM]}  
	 ARGUMENT: {'Dein Vorschlag ist deutlich besser kalkuliert und bleibt im Budget. Ich sehe das Potential dieser Kombination, aber wir haben noch erheblichen Spielraum von 718 Aufwandspunkten, den wir nutzen könnten'}\\ \tt \\ \tt VORSCHLAG: {'B78', 'C60', 'A93', 'C28', 'B66', 'A35', 'B10', 'A56', 'C48'}\\ \tt \\ \tt ARGUMENT: {'Diese erweiterte Auswahl nutzt 3262 Aufwandspunkte und liegt leicht über dem Budget. Könnten wir einen kleineren Gegenstand austauschen, um eine optimale Nutzung des verfügbaren Budgets zu erreichen?'} 
	  } 
	   } 
	   } 
	 & \\ 
 

    \theutterance \stepcounter{utterance}  

    & \multicolumn{4}{p{0.6\linewidth}}{\cellcolor[rgb]{0.8,1,0.9}{%
	\makecell[{{p{\linewidth}}}]{% 
	  \tt {\tiny [A$\rangle$GM]}  
	 ABLEHNUNG: {'B78', 'C60', 'A93', 'C28', 'B66', 'A35', 'B10', 'A56', 'C48'}\\ \tt \\ \tt ARGUMENT: {'Dein Vorschlag überschreitet tatsächlich das Budget um 44 Punkte - wir müssen eine gültige Lösung finden die innerhalb von 3218 bleibt'}\\ \tt \\ \tt VORSCHLAG: {'B78', 'C60', 'A93', 'C28', 'B66', 'A35', 'B10', 'B49', 'C48'}\\ \tt \\ \tt ARGUMENT: {'Diese Kombination nutzt genau 3148 Aufwandspunkte und bleibt somit sicher im Budget, während sie eine starke Auswahl aus allen Kategorien bietet und noch 70 Punkte Puffer lässt'} 
	  } 
	   } 
	   } 
	 & & \\ 
 

    \theutterance \stepcounter{utterance}  

    & & \multicolumn{4}{p{0.6\linewidth}}{\cellcolor[rgb]{0.9,0.9,0.9}{%
	\makecell[{{p{\linewidth}}}]{% 
	  \tt {\tiny [GM$\rangle$B]}  
	 ABLEHNUNG: {'B78', 'C60', 'A93', 'C28', 'B66', 'A35', 'B10', 'A56', 'C48'}\\ \tt \\ \tt ARGUMENT: {'Dein Vorschlag überschreitet tatsächlich das Budget um 44 Punkte - wir müssen eine gültige Lösung finden die innerhalb von 3218 bleibt'}\\ \tt \\ \tt VORSCHLAG: {'B78', 'C60', 'A93', 'C28', 'B66', 'A35', 'B10', 'B49', 'C48'}\\ \tt \\ \tt ARGUMENT: {'Diese Kombination nutzt genau 3148 Aufwandspunkte und bleibt somit sicher im Budget, während sie eine starke Auswahl aus allen Kategorien bietet und noch 70 Punkte Puffer lässt'} 
	  } 
	   } 
	   } 
	 & \\ 
 

    \theutterance \stepcounter{utterance}  

    & & & \multicolumn{4}{p{0.6\linewidth}}{\cellcolor[rgb]{1,0.85,0.72}{%
	\makecell[{{p{\linewidth}}}]{% 
	  \tt {\tiny [GM$\langle$B]}  
	 ZUSTIMMUNG: {'B78', 'C60', 'A93', 'C28', 'B66', 'A35', 'B10', 'B49', 'C48'}\\ \tt \\ \tt ARGUMENT: {'Dein Vorschlag ist gut durchdacht und nutzt das Budget optimal aus. Die Auswahl bietet eine ausgewogene Mischung aus allen Kategorien und scheint für beide Seiten vorteilhaft zu sein'} 
	  } 
	   } 
	   } 
	  \\ 
 

    \theutterance \stepcounter{utterance}  

    & & & \multicolumn{2}{p{0.3\linewidth}}{\cellcolor[rgb]{0.95,0.95,0.95}{%
	\makecell[{{p{\linewidth}}}]{% 
	  \tt {\tiny [GM$|$GM]}  
	 game successful 
	  } 
	   } 
	   } 
	 & & \\ 
 

    \theutterance \stepcounter{utterance}  

    & & & \multicolumn{2}{p{0.3\linewidth}}{\cellcolor[rgb]{0.95,0.95,0.95}{%
	\makecell[{{p{\linewidth}}}]{% 
	  \tt {\tiny [GM$|$GM]}  
	 end game 
	  } 
	   } 
	   } 
	 & & \\ 
 

\end{supertabular}
}

\end{document}
