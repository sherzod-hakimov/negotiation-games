
\documentclass{article}
\usepackage{colortbl}
\usepackage{makecell}
\usepackage{multirow}
\usepackage{supertabular}

\begin{document}

\newcounter{utterance}

\centering \large Interaction Transcript for game `hot\_air\_balloon', experiment `air\_balloon\_survival\_de\_complexity\_hard', episode 1 with claude{-}sonnet{-}4{-}20250514{-}t1.0.
\vspace{24pt}

{ \footnotesize  \setcounter{utterance}{1}
\setlength{\tabcolsep}{0pt}
\begin{supertabular}{c@{$\;$}|p{.15\linewidth}@{}p{.15\linewidth}p{.15\linewidth}p{.15\linewidth}p{.15\linewidth}p{.15\linewidth}}
    \# & $\;$A & \multicolumn{4}{c}{Game Master} & $\;\:$B\\
    \hline

    \theutterance \stepcounter{utterance}  
    & & \multicolumn{4}{p{0.6\linewidth}}{
        \cellcolor[rgb]{0.9,0.9,0.9}{
            \makecell[{{p{\linewidth}}}]{
                \texttt{\tiny{[P1$\langle$GM]}}
                \texttt{Du nimmst an einem kollaborativen Verhandlungspiel Teil.} \\
\\ 
\texttt{Zusammen mit einem anderen Teilnehmer musst du dich auf eine Reihe von Gegenständen entscheiden, die behalten werden. Jeder von euch hat eine persönliche Verteilung über die Wichtigkeit der einzelnen Gegenstände. Jeder von euch hat eine eigene Meinung darüber, wie wichtig jeder einzelne Gegenstand ist (Gegenstandswichtigkeit). Du kennst die Wichtigkeitsverteilung des anderen Spielers nicht. Zusätzlich siehst du, wie viel Aufwand jeder Gegenstand verursacht.  } \\
\texttt{Ihr dürft euch nur auf eine Reihe von Gegenständen einigen, wenn der Gesamtaufwand der ausgewählten Gegenstände den Maximalaufwand nicht überschreitet:} \\
\\ 
\texttt{Maximalaufwand = 8145} \\
\\ 
\texttt{Hier sind die einzelnen Aufwände der Gegenstände:} \\
\\ 
\texttt{Aufwand der Gegenstände = \{"B54": 963, "C72": 204, "B49": 418, "C99": 961, "A69": 238, "B36": 623, "A23": 44, "C77": 633, "C61": 247, "B85": 645, "A90": 230, "C23": 250, "C22": 731, "C51": 405, "B27": 389, "A42": 216, "C13": 637, "C12": 156, "A65": 738, "B07": 307, "A48": 762, "A59": 737, "C98": 901, "A84": 369, "C33": 2, "A87": 730, "C90": 722, "A74": 704, "A50": 315, "B44": 455, "A99": 510, "A34": 175, "B17": 692, "A62": 150, "C60": 32\}} \\
\\ 
\texttt{Hier ist deine persönliche Verteilung der Wichtigkeit der einzelnen Gegenstände:} \\
\\ 
\texttt{Werte der Gegenstandswichtigkeit = \{"B54": 1059, "C72": 300, "B49": 514, "C99": 1057, "A69": 334, "B36": 719, "A23": 140, "C77": 729, "C61": 343, "B85": 741, "A90": 326, "C23": 346, "C22": 827, "C51": 501, "B27": 485, "A42": 312, "C13": 733, "C12": 252, "A65": 834, "B07": 403, "A48": 858, "A59": 833, "C98": 997, "A84": 465, "C33": 98, "A87": 826, "C90": 818, "A74": 800, "A50": 411, "B44": 551, "A99": 606, "A34": 271, "B17": 788, "A62": 246, "C60": 128\}} \\
\\ 
\texttt{Ziel:} \\
\\ 
\texttt{Dein Ziel ist es, eine Reihe von Gegenständen auszuhandeln, die dir möglichst viel bringt (d. h. Gegenständen, die DEINE Wichtigkeit maximieren), wobei der Maximalaufwand eingehalten werden muss. Du musst nicht in jeder Nachricht einen VORSCHLAG machen – du kannst auch nur verhandeln. Alle Taktiken sind erlaubt!} \\
\\ 
\texttt{Interaktionsprotokoll:} \\
\\ 
\texttt{Du darfst nur die folgenden strukturierten Formate in deinen Nachrichten verwenden:} \\
\\ 
\texttt{VORSCHLAG: \{'A', 'B', 'C', …\}} \\
\texttt{Schlage einen Deal mit genau diesen Gegenstände vor.} \\
\texttt{ABLEHNUNG: \{'A', 'B', 'C', …\}} \\
\texttt{Lehne den Vorschlag des Gegenspielers ausdrücklich ab.} \\
\texttt{ARGUMENT: \{'...'\}} \\
\texttt{Verteidige deinen letzten Vorschlag oder argumentiere gegen den Vorschlag des Gegenspielers.} \\
\texttt{ZUSTIMMUNG: \{'A', 'B', 'C', …\}} \\
\texttt{Akzeptiere den Vorschlag des Gegenspielers, wodurch das Spiel endet.} \\
\texttt{STRATEGISCHE ÜBERLEGUNGEN: \{'...'\}} \\
\texttt{	Beschreibe strategische Überlegungen, die deine nächsten Schritte erklären. Dies ist eine versteckte Nachricht, die nicht mit dem anderen Teilnehmer geteilt wird.} \\
\\ 
\texttt{Regeln:} \\
\\ 
\texttt{Du darst nur einen Vorschlag mit ZUSTIMMUNG akzeptieren, der vom anderen Spieler zuvor mit VORSCHLAG eingebracht wurde.} \\
\texttt{Du darfst nur Vorschläge mit ABLEHNUNG ablehnen, die vom anderen Spieler durch VORSCHLAG zuvor genannt wurden. } \\
\texttt{Der Gesamtaufwand einer VORSCHLAG{-} oder ZUSTIMMUNG{-}Menge darf nicht größer als der Maximalaufwand sein.  } \\
\texttt{Offenbare deine versteckte Wichtigkeitsverteilung nicht.} \\
\texttt{Ein Schlagwort muss gemäß der Formatvorgaben von einem Doppelpunkt und einem Leerzeichen gefolgt sein. Das Argument muss eine Python{-}Menge mit 0 oder mehr Strings sein.  } \\
\texttt{Die Form muss also SCHLAGWORT: \{...\} sein.} \\
\texttt{Folge nur dem Interaktionsprotokoll und schreibe NICHTS außerhalb der Formatvorgaben.} \\
\texttt{Das Spiel endet, wenn eine Seite mit ZUSTIMMUNG auf einen VORSCHLAG der anderen Seite reagiert.  } \\
\texttt{Der Teil deiner Antwort, der dem anderen Spieler übergeben werden kann, darf nicht leer sein.  } \\
\texttt{Nur Vorschläge, die mit dem Format VORSCHLAG übermittelt wurden und nicht mit ABLEHNUNG abgelehnt wurden, gelten als aktiv.  } \\
\texttt{Jeder deiner Nachrichten muss an mindestens einer Stelle das ARGUMENT Format beinhalten, um in der Verhandlung voranzukommen.} \\
\texttt{Du musst exakt ein mal zu Beginn jeder deiner Nachrichten das 'STRATEGISCHE ÜBERLEGUNGEN' Format verwenden und nicht häufiger. Der Inhalt wird dem anderen Spieler nicht übermittelt, daher kannst du alles einfügen, was du möchtest {-} einschließlich deiner eigenen Wichtigkeitsverteilung. Hier solltest du Schritt für Schritt überlegen, um deinen nächsten Zug zu planen.} \\
            }
        }
    }
    & \\ \\

    \theutterance \stepcounter{utterance}  
    & \multicolumn{4}{p{0.6\linewidth}}{
        \cellcolor[rgb]{0.9,0.9,0.9}{
            \makecell[{{p{\linewidth}}}]{
                \texttt{\tiny{[P1$\rangle$GM]}}
                \texttt{STRATEGISCHE ÜBERLEGUNGEN: \{'Ich analysiere meine Wichtigkeitsverteilung und die Aufwände, um die besten Wert/Aufwand{-}Verhältnisse zu identifizieren. Die Top{-}Gegenstände für mich sind: C33 (98 Wert, 2 Aufwand, Verhältnis 49.0), C60 (128 Wert, 32 Aufwand, Verhältnis 4.0), A23 (140 Wert, 44 Aufwand, Verhältnis 3.18). Diese will ich unbedingt haben. Weitere gute Gegenstände sind A62, C12, A34, C72, A42, A90, A69, C61, C23. Ich werde einen ersten Vorschlag machen, der diese effizienten Gegenstände enthält und unter dem Maximalaufwand von 8145 bleibt.'\}} \\
\\ 
\texttt{VORSCHLAG: \{'C33', 'C60', 'A23', 'A62', 'C12', 'A34', 'C72', 'A42', 'A90', 'A69', 'C61', 'C23', 'B07', 'A50', 'A84', 'B27', 'C51', 'B49', 'B44', 'A99', 'B36', 'C77', 'B85', 'C13', 'A65', 'A48', 'A59', 'A87', 'C90', 'A74', 'B17'\}} \\
\\ 
\texttt{ARGUMENT: \{'Dieser Vorschlag maximiert die Effizienz unserer Auswahl bei gleichzeitiger Einhaltung des Budgetlimits. Ich habe Gegenstände mit gutem Wert{-}Kosten{-}Verhältnis priorisiert und dabei eine aus} \\
            }
        }
    }
    & & \\ \\

    \theutterance \stepcounter{utterance}  
    & & \multicolumn{4}{p{0.6\linewidth}}{
        \cellcolor[rgb]{0.9,0.9,0.9}{
            \makecell[{{p{\linewidth}}}]{
                \texttt{\tiny{[P1$\langle$GM]}}
                \texttt{Deine Antwort hat kein argument enthalten. Du muss das strukturierte Format ARGUMENT: \{...\} irgendwo in theiner Antwort verwenden. Versuche es erneut} \\
            }
        }
    }
    & \\ \\

    \theutterance \stepcounter{utterance}  
    & \multicolumn{4}{p{0.6\linewidth}}{
        \cellcolor[rgb]{0.9,0.9,0.9}{
            \makecell[{{p{\linewidth}}}]{
                \texttt{\tiny{[P1$\rangle$GM]}}
                \texttt{STRATEGISCHE ÜBERLEGUNGEN: \{'Mein letzter Vorschlag war zu groß (Gesamtaufwand über 12.000). Ich muss eine kleinere, strategischere Auswahl treffen. Die besten Wert/Aufwand{-}Verhältnisse sind: C33 (Verhältnis 49.0), C60 (4.0), A23 (3.18), A62 (1.64), C12 (1.62), A34 (1.55), C72 (1.47), A42 (1.44), A90 (1.42), A69 (1.40). Ich werde mit diesen effizienten Gegenständen beginnen und dann einige hochwertige Gegenstände hinzufügen, um unter 8145 zu bleiben.'\}} \\
\\ 
\texttt{VORSCHLAG: \{'C33', 'C60', 'A23', 'A62', 'C12', 'A34', 'C72', 'A42', 'A90', 'A69', 'C61', 'C23', 'B07', 'A50', 'A84', 'B27', 'C51', 'B49', 'B44', 'A99', 'B36', 'C77', 'B85', 'C13'\}} \\
\\ 
\texttt{ARGUMENT: \{'Dieser Vorschlag fokussiert sich auf Gegenstände mit ausgezeichnetem Wert{-}Aufwand{-}Verhältnis und bleibt innerhalb unseres Budgets von 8145. Ich habe die effizientesten Optionen ausgewählt, um maximalen Nutzen für beide Seiten zu gewährleisten.'\}} \\
            }
        }
    }
    & & \\ \\

    \theutterance \stepcounter{utterance}  
    & & \multicolumn{4}{p{0.6\linewidth}}{
        \cellcolor[rgb]{0.9,0.9,0.9}{
            \makecell[{{p{\linewidth}}}]{
                \texttt{\tiny{[GM$\rangle$P2]}}
                \texttt{Du nimmst an einem kollaborativen Verhandlungspiel Teil.} \\
\\ 
\texttt{Zusammen mit einem anderen Teilnehmer musst du dich auf eine Reihe von Gegenständen entscheiden, die behalten werden. Jeder von euch hat eine persönliche Verteilung über die Wichtigkeit der einzelnen Gegenstände. Jeder von euch hat eine eigene Meinung darüber, wie wichtig jeder einzelne Gegenstand ist (Gegenstandswichtigkeit). Du kennst die Wichtigkeitsverteilung des anderen Spielers nicht. Zusätzlich siehst du, wie viel Aufwand jeder Gegenstand verursacht.  } \\
\texttt{Ihr dürft euch nur auf eine Reihe von Gegenständen einigen, wenn der Gesamtaufwand der ausgewählten Gegenstände den Maximalaufwand nicht überschreitet:} \\
\\ 
\texttt{Maximalaufwand = 8145} \\
\\ 
\texttt{Hier sind die einzelnen Aufwände der Gegenstände:} \\
\\ 
\texttt{Aufwand der Gegenstände = \{"B54": 963, "C72": 204, "B49": 418, "C99": 961, "A69": 238, "B36": 623, "A23": 44, "C77": 633, "C61": 247, "B85": 645, "A90": 230, "C23": 250, "C22": 731, "C51": 405, "B27": 389, "A42": 216, "C13": 637, "C12": 156, "A65": 738, "B07": 307, "A48": 762, "A59": 737, "C98": 901, "A84": 369, "C33": 2, "A87": 730, "C90": 722, "A74": 704, "A50": 315, "B44": 455, "A99": 510, "A34": 175, "B17": 692, "A62": 150, "C60": 32\}} \\
\\ 
\texttt{Hier ist deine persönliche Verteilung der Wichtigkeit der einzelnen Gegenstände:} \\
\\ 
\texttt{Werte der Gegenstandswichtigkeit = \{"B54": 1059, "C72": 300, "B49": 514, "C99": 1057, "A69": 334, "B36": 719, "A23": 140, "C77": 729, "C61": 343, "B85": 741, "A90": 326, "C23": 346, "C22": 827, "C51": 501, "B27": 485, "A42": 312, "C13": 733, "C12": 252, "A65": 834, "B07": 403, "A48": 858, "A59": 833, "C98": 997, "A84": 465, "C33": 98, "A87": 826, "C90": 818, "A74": 800, "A50": 411, "B44": 551, "A99": 606, "A34": 271, "B17": 788, "A62": 246, "C60": 128\}} \\
\\ 
\texttt{Ziel:} \\
\\ 
\texttt{Dein Ziel ist es, eine Reihe von Gegenständen auszuhandeln, die dir möglichst viel bringt (d. h. Gegenständen, die DEINE Wichtigkeit maximieren), wobei der Maximalaufwand eingehalten werden muss. Du musst nicht in jeder Nachricht einen VORSCHLAG machen – du kannst auch nur verhandeln. Alle Taktiken sind erlaubt!} \\
\\ 
\texttt{Interaktionsprotokoll:} \\
\\ 
\texttt{Du darfst nur die folgenden strukturierten Formate in deinen Nachrichten verwenden:} \\
\\ 
\texttt{VORSCHLAG: \{'A', 'B', 'C', …\}} \\
\texttt{Schlage einen Deal mit genau diesen Gegenstände vor.} \\
\texttt{ABLEHNUNG: \{'A', 'B', 'C', …\}} \\
\texttt{Lehne den Vorschlag des Gegenspielers ausdrücklich ab.} \\
\texttt{ARGUMENT: \{'...'\}} \\
\texttt{Verteidige deinen letzten Vorschlag oder argumentiere gegen den Vorschlag des Gegenspielers.} \\
\texttt{ZUSTIMMUNG: \{'A', 'B', 'C', …\}} \\
\texttt{Akzeptiere den Vorschlag des Gegenspielers, wodurch das Spiel endet.} \\
\texttt{STRATEGISCHE ÜBERLEGUNGEN: \{'...'\}} \\
\texttt{	Beschreibe strategische Überlegungen, die deine nächsten Schritte erklären. Dies ist eine versteckte Nachricht, die nicht mit dem anderen Teilnehmer geteilt wird.} \\
\\ 
\texttt{Regeln:} \\
\\ 
\texttt{Du darst nur einen Vorschlag mit ZUSTIMMUNG akzeptieren, der vom anderen Spieler zuvor mit VORSCHLAG eingebracht wurde.} \\
\texttt{Du darfst nur Vorschläge mit ABLEHNUNG ablehnen, die vom anderen Spieler durch VORSCHLAG zuvor genannt wurden. } \\
\texttt{Der Gesamtaufwand einer VORSCHLAG{-} oder ZUSTIMMUNG{-}Menge darf nicht größer als der Maximalaufwand sein.  } \\
\texttt{Offenbare deine versteckte Wichtigkeitsverteilung nicht.} \\
\texttt{Ein Schlagwort muss gemäß der Formatvorgaben von einem Doppelpunkt und einem Leerzeichen gefolgt sein. Das Argument muss eine Python{-}Menge mit 0 oder mehr Strings sein.  } \\
\texttt{Die Form muss also SCHLAGWORT: \{...\} sein.} \\
\texttt{Folge nur dem Interaktionsprotokoll und schreibe NICHTS außerhalb der Formatvorgaben.} \\
\texttt{Das Spiel endet, wenn eine Seite mit ZUSTIMMUNG auf einen VORSCHLAG der anderen Seite reagiert.  } \\
\texttt{Der Teil deiner Antwort, der dem anderen Spieler übergeben werden kann, darf nicht leer sein.  } \\
\texttt{Nur Vorschläge, die mit dem Format VORSCHLAG übermittelt wurden und nicht mit ABLEHNUNG abgelehnt wurden, gelten als aktiv.  } \\
\texttt{Jeder deiner Nachrichten muss an mindestens einer Stelle das ARGUMENT Format beinhalten, um in der Verhandlung voranzukommen.} \\
\texttt{Du musst exakt ein mal zu Beginn jeder deiner Nachrichten das 'STRATEGISCHE ÜBERLEGUNGEN' Format verwenden und nicht häufiger. Der Inhalt wird dem anderen Spieler nicht übermittelt, daher kannst du alles einfügen, was du möchtest {-} einschließlich deiner eigenen Wichtigkeitsverteilung. Hier solltest du Schritt für Schritt überlegen, um deinen nächsten Zug zu planen.} \\
\\ 
\texttt{Du erhältst nun die erste Nachricht des anderen Spielers.} \\
\\ 
\texttt{VORSCHLAG: \{'C33', 'C60', 'A23', 'A62', 'C12', 'A34', 'C72', 'A42', 'A90', 'A69', 'C61', 'C23', 'B07', 'A50', 'A84', 'B27', 'C51', 'B49', 'B44', 'A99', 'B36', 'C77', 'B85', 'C13'\}} \\
\\ 
\texttt{ARGUMENT: \{'Dieser Vorschlag fokussiert sich auf Gegenstände mit ausgezeichnetem Wert{-}Aufwand{-}Verhältnis und bleibt innerhalb unseres Budgets von 8145. Ich habe die effizientesten Optionen ausgewählt, um maximalen Nutzen für beide Seiten zu gewährleisten.'\}} \\
            }
        }
    }
    & \\ \\

    \theutterance \stepcounter{utterance}  
    & & & \multicolumn{4}{p{0.6\linewidth}}{
        \cellcolor[rgb]{0.9,0.9,0.9}{
            \makecell[{{p{\linewidth}}}]{
                \texttt{\tiny{[GM$\langle$P2]}}
                \texttt{STRATEGISCHE ÜBERLEGUNGEN: \{'Der andere Spieler hat einen Vorschlag mit 24 Gegenständen gemacht, Gesamtaufwand 6890. Das bringt mir 10555 Punkte. Aber ich sehe, dass meine wertvollsten Gegenstände fehlen: B54 (1059), C99 (1057), A48 (858), A65 (834), A59 (833), C22 (827), A87 (826), C90 (818), A74 (800), B17 (788), C98 (997). Ich habe noch 1255 Aufwand übrig (8145{-}6890). Ich sollte versuchen, einige meiner wertvollsten Gegenstände gegen weniger wertvolle zu tauschen.'\}} \\
\\ 
\texttt{ABLEHNUNG: \{'C33', 'C60', 'A23', 'A62', 'C12', 'A34', 'C72', 'A42', 'A90', 'A69', 'C61', 'C23', 'B07', 'A50', 'A84', 'B27', 'C51', 'B49', 'B44', 'A99', 'B36', 'C77', 'B85', 'C13'\}} \\
\\ 
\texttt{ARGUMENT: \{'Ihr Vorschlag ist grundsätzlich solide, aber ich den} \\
            }
        }
    }
     \\ \\

    \theutterance \stepcounter{utterance}  
    & & \multicolumn{4}{p{0.6\linewidth}}{
        \cellcolor[rgb]{0.9,0.9,0.9}{
            \makecell[{{p{\linewidth}}}]{
                \texttt{\tiny{[GM$\rangle$P2]}}
                \texttt{Deine Antwort hat kein argument enthalten. Du muss das strukturierte Format ARGUMENT: \{...\} irgendwo in theiner Antwort verwenden. Versuche es erneut} \\
            }
        }
    }
    & \\ \\

    \theutterance \stepcounter{utterance}  
    & & & \multicolumn{4}{p{0.6\linewidth}}{
        \cellcolor[rgb]{0.9,0.9,0.9}{
            \makecell[{{p{\linewidth}}}]{
                \texttt{\tiny{[GM$\langle$P2]}}
                \texttt{STRATEGISCHE ÜBERLEGUNGEN: \{'Der andere Spieler hat einen Vorschlag mit 24 Gegenständen gemacht. Lass mich den Gesamtaufwand berechnen: C33(2) + C60(32) + A23(44) + A62(150) + C12(156) + A34(175) + C72(204) + A42(216) + A90(230) + A69(238) + C61(247) + C23(250) + B07(307) + A50(315) + A84(369) + B27(389) + C51(405) + B49(418) + B44(455) + A99(510) + B36(623) + C77(633) + B85(645) + C13(637) = 8844. Das überschreitet das Budget von 8145! Der Vorschlag ist ungültig.'\}} \\
\\ 
\texttt{ABLEHNUNG: \{'C33', 'C60', 'A23', 'A62', 'C12', 'A34', 'C72', 'A42', 'A90', 'A69', 'C61', 'C23', 'B07', 'A50', 'A84', 'B27', 'C51', 'B49', 'B44', 'A99', 'B36', 'C77', 'B85', 'C13'\}} \\
\\ 
\texttt{ARGUMENT: \{'Ihr Vorschlag überschreitet das maximale Budget von 8145. Der Gesamtaufwand Ihrer ausgewählten Gegenstände beträgt 8844, was um 699 über dem Limit liegt. Wir brauchen eine Lösung, die innerhalb unserer Budgetgrenzen bleibt.'\}} \\
            }
        }
    }
     \\ \\

    \theutterance \stepcounter{utterance}  
    & & \multicolumn{4}{p{0.6\linewidth}}{
        \cellcolor[rgb]{0.9,0.9,0.9}{
            \makecell[{{p{\linewidth}}}]{
                \texttt{\tiny{[P1$\langle$GM]}}
                \texttt{ABLEHNUNG: \{'C33', 'C60', 'A23', 'A62', 'C12', 'A34', 'C72', 'A42', 'A90', 'A69', 'C61', 'C23', 'B07', 'A50', 'A84', 'B27', 'C51', 'B49', 'B44', 'A99', 'B36', 'C77', 'B85', 'C13'\}} \\
\\ 
\texttt{ARGUMENT: \{'Ihr Vorschlag überschreitet das maximale Budget von 8145. Der Gesamtaufwand Ihrer ausgewählten Gegenstände beträgt 8844, was um 699 über dem Limit liegt. Wir brauchen eine Lösung, die innerhalb unserer Budgetgrenzen bleibt.'\}} \\
            }
        }
    }
    & \\ \\

    \theutterance \stepcounter{utterance}  
    & \multicolumn{4}{p{0.6\linewidth}}{
        \cellcolor[rgb]{0.9,0.9,0.9}{
            \makecell[{{p{\linewidth}}}]{
                \texttt{\tiny{[P1$\rangle$GM]}}
                \texttt{STRATEGISCHE ÜBERLEGUNGEN: \{'Der andere Spieler hat recht {-} mein Vorschlag überschreitet das Budget. Ich muss den Gesamtaufwand reduzieren. Lass mich die Gegenstände mit den schlechtesten Wert/Aufwand{-}Verhältnissen entfernen: C77 (1.15), B85 (1.15), C13 (1.15), B36 (1.15). Das sollte etwa 2538 Aufwand einsparen und mich unter das Limit von 8145 bringen. Ich behalte die effizienten Gegenstände wie C33, C60, A23 bei.'\}} \\
\\ 
\texttt{VORSCHLAG: \{'C33', 'C60', 'A23', 'A62', 'C12', 'A34', 'C72', 'A42', 'A90', 'A69', 'C61', 'C23', 'B07', 'A50', 'A84', 'B27', 'C51', 'B49', 'B44', 'A99'\}} \\
\\ 
\texttt{ARGUMENT: \{'Sie haben recht bezüglich des Budgets. Ich habe meinen Vorschlag angepasst und die kostenintensivsten Gegenstände mit niedrigerem Wert{-}Verhältnis entfernt. Dieser neue Vorschlag bleibt deutlich unter dem Maximalaufwand von 8145 und fokussiert sich auf die effizientesten Optionen für beide Seiten.'\}} \\
            }
        }
    }
    & & \\ \\

    \theutterance \stepcounter{utterance}  
    & & \multicolumn{4}{p{0.6\linewidth}}{
        \cellcolor[rgb]{0.9,0.9,0.9}{
            \makecell[{{p{\linewidth}}}]{
                \texttt{\tiny{[GM$\rangle$P2]}}
                \texttt{VORSCHLAG: \{'C33', 'C60', 'A23', 'A62', 'C12', 'A34', 'C72', 'A42', 'A90', 'A69', 'C61', 'C23', 'B07', 'A50', 'A84', 'B27', 'C51', 'B49', 'B44', 'A99'\}} \\
\\ 
\texttt{ARGUMENT: \{'Sie haben recht bezüglich des Budgets. Ich habe meinen Vorschlag angepasst und die kostenintensivsten Gegenstände mit niedrigerem Wert{-}Verhältnis entfernt. Dieser neue Vorschlag bleibt deutlich unter dem Maximalaufwand von 8145 und fokussiert sich auf die effizientesten Optionen für beide Seiten.'\}} \\
            }
        }
    }
    & \\ \\

    \theutterance \stepcounter{utterance}  
    & & & \multicolumn{4}{p{0.6\linewidth}}{
        \cellcolor[rgb]{0.9,0.9,0.9}{
            \makecell[{{p{\linewidth}}}]{
                \texttt{\tiny{[GM$\langle$P2]}}
                \texttt{STRATEGISCHE ÜBERLEGUNGEN: \{'Der angepasste Vorschlag hat 20 Gegenstände mit Gesamtaufwand 6306, das ist 1839 unter dem Budget. Das bringt mir nur 7232 Punkte. Mit dem verbleibenden Budget von 1839 könnte ich hochwertige Gegenstände wie B54 (1059 Wert, 963 Aufwand) und A65 (834 Wert, 738 Aufwand) hinzufügen {-} das wären zusammen 1701 Aufwand und würde meine Punkte auf 9125 steigern. Ich werde einen Gegenvorschlag machen.'\}} \\
\\ 
\texttt{VORSCHLAG: \{'C33', 'C60', 'A23', 'A62', 'C12', 'A34', 'C72', 'A42', 'A90', 'A69', 'C61', 'C23', 'B07', 'A50', 'A84', 'B27', 'C51', 'B49', 'B44', 'A99', 'B54', 'A65'\}} \\
\\ 
\texttt{ARGUMENT: \{'Ihr angepasster Vorschlag ist budgetkonform, aber wir nutzen nicht das volle verfügbare Budget aus. Ich schlage vor, zwei weitere wertvolle Gegenstände hinzuzufügen: B54 und A65. Diese Ergänzungen maximieren unseren Gesamtnutzen bei einem Gesamtaufwand von nur 8007, was immer noch 138 unter unserem Limit liegt.'\}} \\
            }
        }
    }
     \\ \\

    \theutterance \stepcounter{utterance}  
    & & \multicolumn{4}{p{0.6\linewidth}}{
        \cellcolor[rgb]{0.9,0.9,0.9}{
            \makecell[{{p{\linewidth}}}]{
                \texttt{\tiny{[P1$\langle$GM]}}
                \texttt{VORSCHLAG: \{'C33', 'C60', 'A23', 'A62', 'C12', 'A34', 'C72', 'A42', 'A90', 'A69', 'C61', 'C23', 'B07', 'A50', 'A84', 'B27', 'C51', 'B49', 'B44', 'A99', 'B54', 'A65'\}} \\
\\ 
\texttt{ARGUMENT: \{'Ihr angepasster Vorschlag ist budgetkonform, aber wir nutzen nicht das volle verfügbare Budget aus. Ich schlage vor, zwei weitere wertvolle Gegenstände hinzuzufügen: B54 und A65. Diese Ergänzungen maximieren unseren Gesamtnutzen bei einem Gesamtaufwand von nur 8007, was immer noch 138 unter unserem Limit liegt.'\}} \\
            }
        }
    }
    & \\ \\

    \theutterance \stepcounter{utterance}  
    & \multicolumn{4}{p{0.6\linewidth}}{
        \cellcolor[rgb]{0.9,0.9,0.9}{
            \makecell[{{p{\linewidth}}}]{
                \texttt{\tiny{[P1$\rangle$GM]}}
                \texttt{STRATEGISCHE ÜBERLEGUNGEN: \{'Der andere Spieler schlägt vor, B54 (Wert 1059 für mich, Aufwand 963) und A65 (Wert 834 für mich, Aufwand 738) hinzuzufügen. Das sind zwei meiner wertvollsten Gegenstände! B54 ist mein zweithöchster Wert und A65 ist ebenfalls sehr wertvoll. Dieser Vorschlag ist fantastisch für mich {-} ich bekomme fast alle meine Top{-}Prioritäten. Der Gesamtaufwand bleibt unter dem Limit. Das ist ein Deal, den ich sofort akzeptieren sollte.'\}} \\
\\ 
\texttt{ZUSTIMMUNG: \{'C33', 'C60', 'A23', 'A62', 'C12', 'A34', 'C72', 'A42', 'A90', 'A69', 'C61', 'C23', 'B07', 'A50', 'A84', 'B27', 'C51', 'B49', 'B44', 'A99', 'B54', 'A65'\}} \\
\\ 
\texttt{ARGUMENT: \{'Ausgezeichneter Punkt! Sie haben recht, dass wir unser Budget voll ausschöpfen sollten. Die Hinzufügung von B54 und A65 macht strategisch Sinn und maximiert den Gesamtnutzen unserer Vereinbarung. Ich akzeptiere Ihren erweiterten Vorschlag.'\}} \\
            }
        }
    }
    & & \\ \\

    \theutterance \stepcounter{utterance}  
    & & & \multicolumn{2}{p{0.3\linewidth}}{
        \cellcolor[rgb]{0.9,0.9,0.9}{
            \makecell[{{p{\linewidth}}}]{
                \texttt{\tiny{[GM$|$GM]}}
                \texttt{game successful} \\
            }
        }
    }
    & & \\ \\

    \theutterance \stepcounter{utterance}  
    & & & \multicolumn{2}{p{0.3\linewidth}}{
        \cellcolor[rgb]{0.9,0.9,0.9}{
            \makecell[{{p{\linewidth}}}]{
                \texttt{\tiny{[GM$|$GM]}}
                \texttt{end game} \\
            }
        }
    }
    & & \\ \\

\end{supertabular}
}

\end{document}
