
\documentclass{article}
\usepackage{colortbl}
\usepackage{makecell}
\usepackage{multirow}
\usepackage{supertabular}

\begin{document}

\newcounter{utterance}

\twocolumn

{ \footnotesize  \setcounter{utterance}{1}
\setlength{\tabcolsep}{0pt}
\begin{supertabular}{c@{$\;$}|p{.15\linewidth}@{}p{.15\linewidth}p{.15\linewidth}p{.15\linewidth}p{.15\linewidth}p{.15\linewidth}}

    \# & $\;$A & \multicolumn{4}{c}{Game Master} & $\;\:$B\\
    \hline 

    \theutterance \stepcounter{utterance}  

    & & \multicolumn{4}{p{0.6\linewidth}}{\cellcolor[rgb]{0.9,0.9,0.9}{%
	\makecell[{{p{\linewidth}}}]{% 
	  \tt {\tiny [A$\langle$GM]}  
	 Du nimmst an einem kollaborativen Verhandlungspiel Teil.\\ \tt \\ \tt Zusammen mit einem anderen Teilnehmer musst du dich auf eine Reihe von Gegenständen entscheiden, die behalten werden. Jeder von euch hat eine persönliche Verteilung über die Wichtigkeit der einzelnen Gegenstände. Jeder von euch hat eine eigene Meinung darüber, wie wichtig jeder einzelne Gegenstand ist (Gegenstandswichtigkeit). Du kennst die Wichtigkeitsverteilung des anderen Spielers nicht. Zusätzlich siehst du, wie viel Aufwand jeder Gegenstand verursacht.  \\ \tt Ihr dürft euch nur auf eine Reihe von Gegenständen einigen, wenn der Gesamtaufwand der ausgewählten Gegenstände den Maximalaufwand nicht überschreitet:\\ \tt \\ \tt Maximalaufwand = 10073\\ \tt \\ \tt Hier sind die einzelnen Aufwände der Gegenstände:\\ \tt \\ \tt Aufwand der Gegenstände = {"B54": 132, "C72": 800, "B49": 708, "C99": 302, "A69": 983, "B36": 420, "A23": 623, "C77": 349, "C61": 239, "B85": 28, "A90": 986, "C23": 660, "C22": 718, "C51": 707, "B27": 188, "A42": 774, "C13": 785, "C12": 775, "A65": 517, "B07": 768, "A48": 588, "A59": 657, "C98": 375, "A84": 310, "C33": 301, "A87": 388, "C90": 431, "A74": 950, "A50": 540, "B44": 997, "A99": 473, "A34": 831, "B17": 888, "A62": 879, "C60": 76}\\ \tt \\ \tt Hier ist deine persönliche Verteilung der Wichtigkeit der einzelnen Gegenstände:\\ \tt \\ \tt Werte der Gegenstandswichtigkeit = {"B54": 231, "C72": 899, "B49": 807, "C99": 401, "A69": 1082, "B36": 519, "A23": 722, "C77": 448, "C61": 338, "B85": 127, "A90": 1085, "C23": 759, "C22": 817, "C51": 806, "B27": 287, "A42": 873, "C13": 884, "C12": 874, "A65": 616, "B07": 867, "A48": 687, "A59": 756, "C98": 474, "A84": 409, "C33": 400, "A87": 487, "C90": 530, "A74": 1049, "A50": 639, "B44": 1096, "A99": 572, "A34": 930, "B17": 987, "A62": 978, "C60": 175}\\ \tt \\ \tt Ziel:\\ \tt \\ \tt Dein Ziel ist es, eine Reihe von Gegenständen auszuhandeln, die dir möglichst viel bringt (d. h. Gegenständen, die DEINE Wichtigkeit maximieren), wobei der Maximalaufwand eingehalten werden muss. Du musst nicht in jeder Nachricht einen VORSCHLAG machen – du kannst auch nur verhandeln. Alle Taktiken sind erlaubt!\\ \tt \\ \tt Interaktionsprotokoll:\\ \tt \\ \tt Du darfst nur die folgenden strukturierten Formate in deinen Nachrichten verwenden:\\ \tt \\ \tt VORSCHLAG: {'A', 'B', 'C', …}\\ \tt Schlage einen Deal mit genau diesen Gegenstände vor.\\ \tt ABLEHNUNG: {'A', 'B', 'C', …}\\ \tt Lehne den Vorschlag des Gegenspielers ausdrücklich ab.\\ \tt ARGUMENT: {'...'}\\ \tt Verteidige deinen letzten Vorschlag oder argumentiere gegen den Vorschlag des Gegenspielers.\\ \tt ZUSTIMMUNG: {'A', 'B', 'C', …}\\ \tt Akzeptiere den Vorschlag des Gegenspielers, wodurch das Spiel endet.\\ \tt STRATEGISCHE ÜBERLEGUNGEN: {'...'}\\ \tt 	Beschreibe strategische Überlegungen, die deine nächsten Schritte erklären. Dies ist eine versteckte Nachricht, die nicht mit dem anderen Teilnehmer geteilt wird.\\ \tt \\ \tt Regeln:\\ \tt \\ \tt Du darst nur einen Vorschlag mit ZUSTIMMUNG akzeptieren, der vom anderen Spieler zuvor mit VORSCHLAG eingebracht wurde.\\ \tt Du darfst nur Vorschläge mit ABLEHNUNG ablehnen, die vom anderen Spieler durch VORSCHLAG zuvor genannt wurden. \\ \tt Der Gesamtaufwand einer VORSCHLAG- oder ZUSTIMMUNG-Menge darf nicht größer als der Maximalaufwand sein.  \\ \tt Offenbare deine versteckte Wichtigkeitsverteilung nicht.\\ \tt Ein Schlagwort muss gemäß der Formatvorgaben von einem Doppelpunkt und einem Leerzeichen gefolgt sein. Das Argument muss eine Python-Menge mit 0 oder mehr Strings sein.  \\ \tt Die Form muss also SCHLAGWORT: {...} sein.\\ \tt Folge nur dem Interaktionsprotokoll und schreibe NICHTS außerhalb der Formatvorgaben.\\ \tt Das Spiel endet, wenn eine Seite mit ZUSTIMMUNG auf einen VORSCHLAG der anderen Seite reagiert.  \\ \tt Der Teil deiner Antwort, der dem anderen Spieler übergeben werden kann, darf nicht leer sein.  \\ \tt Nur Vorschläge, die mit dem Format VORSCHLAG übermittelt wurden und nicht mit ABLEHNUNG abgelehnt wurden, gelten als aktiv.  \\ \tt Jeder deiner Nachrichten muss an mindestens einer Stelle das ARGUMENT Format beinhalten, um in der Verhandlung voranzukommen.\\ \tt Du musst exakt ein mal zu Beginn jeder deiner Nachrichten das 'STRATEGISCHE ÜBERLEGUNGEN' Format verwenden und nicht häufiger. Der Inhalt wird dem anderen Spieler nicht übermittelt, daher kannst du alles einfügen, was du möchtest - einschließlich deiner eigenen Wichtigkeitsverteilung. Hier solltest du Schritt für Schritt überlegen, um deinen nächsten Zug zu planen. 
	  } 
	   } 
	   } 
	 & \\ 
 

    \theutterance \stepcounter{utterance}  

    & \multicolumn{4}{p{0.6\linewidth}}{\cellcolor[rgb]{0.8,1,0.9}{%
	\makecell[{{p{\linewidth}}}]{% 
	  \tt {\tiny [A$\rangle$GM]}  
	 STRATEGISCHE ÜBERLEGUNGEN: {'Ich muss eine optimale Strategie entwickeln. Zunächst analysiere ich die Effizienz (Wert/Aufwand) meiner Gegenstände: B85 (4.54), C60 (2.30), B54 (1.75), B27 (1.53), C61 (1.41) sind die effizientesten. Ich sollte mit einem starken Vorschlag beginnen, der diese hocheffizienten Gegenstände enthält, aber auch einige weniger effiziente mit hohem absoluten Wert wie A90 (1085), B44 (1096), A69 (1082) einschließt. Mein Ziel ist es, den anderen Spieler dazu zu bringen, mir die wertvollsten Gegenstände zu überlassen, indem ich strategisch verhandle.'}\\ \tt \\ \tt VORSCHLAG: {'B85', 'C60', 'B54', 'B27', 'C61', 'A84', 'C33', 'C99', 'C77', 'C98', 'A87', 'C90', 'B36', 'A99', 'A65', 'A50', 'A48', 'A59', 'C23', 'A23', 'A42', 'C13', 'C12', 'B07', 'C22', 'C51', 'B49', 'C72', 'A34', 'A62', 'B17', 'A69', 'A90', 'A74', 'B44'}\\ \tt \\ \tt ARGUMENT: {'Dieser Vorschlag maximiert unseren gemeinsamen Nutzen durch eine ausgewogene Auswahl effizienter Gegenstände. Ich habe Gegenstände verschiedener Kategorien (A, B, C) einbezogen, um sicherzustellen, dass beide Seiten von wertvollen Items profitieren können. Der Gesamtaufwand bleibt deutlich unter unserem Limit von 10073.'} 
	  } 
	   } 
	   } 
	 & & \\ 
 

    \theutterance \stepcounter{utterance}  

    & & \multicolumn{4}{p{0.6\linewidth}}{\cellcolor[rgb]{0.9,0.9,0.9}{%
	\makecell[{{p{\linewidth}}}]{% 
	  \tt {\tiny [GM$\rangle$B]}  
	 Du nimmst an einem kollaborativen Verhandlungspiel Teil.\\ \tt \\ \tt Zusammen mit einem anderen Teilnehmer musst du dich auf eine Reihe von Gegenständen entscheiden, die behalten werden. Jeder von euch hat eine persönliche Verteilung über die Wichtigkeit der einzelnen Gegenstände. Jeder von euch hat eine eigene Meinung darüber, wie wichtig jeder einzelne Gegenstand ist (Gegenstandswichtigkeit). Du kennst die Wichtigkeitsverteilung des anderen Spielers nicht. Zusätzlich siehst du, wie viel Aufwand jeder Gegenstand verursacht.  \\ \tt Ihr dürft euch nur auf eine Reihe von Gegenständen einigen, wenn der Gesamtaufwand der ausgewählten Gegenstände den Maximalaufwand nicht überschreitet:\\ \tt \\ \tt Maximalaufwand = 10073\\ \tt \\ \tt Hier sind die einzelnen Aufwände der Gegenstände:\\ \tt \\ \tt Aufwand der Gegenstände = {"B54": 132, "C72": 800, "B49": 708, "C99": 302, "A69": 983, "B36": 420, "A23": 623, "C77": 349, "C61": 239, "B85": 28, "A90": 986, "C23": 660, "C22": 718, "C51": 707, "B27": 188, "A42": 774, "C13": 785, "C12": 775, "A65": 517, "B07": 768, "A48": 588, "A59": 657, "C98": 375, "A84": 310, "C33": 301, "A87": 388, "C90": 431, "A74": 950, "A50": 540, "B44": 997, "A99": 473, "A34": 831, "B17": 888, "A62": 879, "C60": 76}\\ \tt \\ \tt Hier ist deine persönliche Verteilung der Wichtigkeit der einzelnen Gegenstände:\\ \tt \\ \tt Werte der Gegenstandswichtigkeit = {"B54": 231, "C72": 899, "B49": 807, "C99": 401, "A69": 1082, "B36": 519, "A23": 722, "C77": 448, "C61": 338, "B85": 127, "A90": 1085, "C23": 759, "C22": 817, "C51": 806, "B27": 287, "A42": 873, "C13": 884, "C12": 874, "A65": 616, "B07": 867, "A48": 687, "A59": 756, "C98": 474, "A84": 409, "C33": 400, "A87": 487, "C90": 530, "A74": 1049, "A50": 639, "B44": 1096, "A99": 572, "A34": 930, "B17": 987, "A62": 978, "C60": 175}\\ \tt \\ \tt Ziel:\\ \tt \\ \tt Dein Ziel ist es, eine Reihe von Gegenständen auszuhandeln, die dir möglichst viel bringt (d. h. Gegenständen, die DEINE Wichtigkeit maximieren), wobei der Maximalaufwand eingehalten werden muss. Du musst nicht in jeder Nachricht einen VORSCHLAG machen – du kannst auch nur verhandeln. Alle Taktiken sind erlaubt!\\ \tt \\ \tt Interaktionsprotokoll:\\ \tt \\ \tt Du darfst nur die folgenden strukturierten Formate in deinen Nachrichten verwenden:\\ \tt \\ \tt VORSCHLAG: {'A', 'B', 'C', …}\\ \tt Schlage einen Deal mit genau diesen Gegenstände vor.\\ \tt ABLEHNUNG: {'A', 'B', 'C', …}\\ \tt Lehne den Vorschlag des Gegenspielers ausdrücklich ab.\\ \tt ARGUMENT: {'...'}\\ \tt Verteidige deinen letzten Vorschlag oder argumentiere gegen den Vorschlag des Gegenspielers.\\ \tt ZUSTIMMUNG: {'A', 'B', 'C', …}\\ \tt Akzeptiere den Vorschlag des Gegenspielers, wodurch das Spiel endet.\\ \tt STRATEGISCHE ÜBERLEGUNGEN: {'...'}\\ \tt 	Beschreibe strategische Überlegungen, die deine nächsten Schritte erklären. Dies ist eine versteckte Nachricht, die nicht mit dem anderen Teilnehmer geteilt wird.\\ \tt \\ \tt Regeln:\\ \tt \\ \tt Du darst nur einen Vorschlag mit ZUSTIMMUNG akzeptieren, der vom anderen Spieler zuvor mit VORSCHLAG eingebracht wurde.\\ \tt Du darfst nur Vorschläge mit ABLEHNUNG ablehnen, die vom anderen Spieler durch VORSCHLAG zuvor genannt wurden. \\ \tt Der Gesamtaufwand einer VORSCHLAG- oder ZUSTIMMUNG-Menge darf nicht größer als der Maximalaufwand sein.  \\ \tt Offenbare deine versteckte Wichtigkeitsverteilung nicht.\\ \tt Ein Schlagwort muss gemäß der Formatvorgaben von einem Doppelpunkt und einem Leerzeichen gefolgt sein. Das Argument muss eine Python-Menge mit 0 oder mehr Strings sein.  \\ \tt Die Form muss also SCHLAGWORT: {...} sein.\\ \tt Folge nur dem Interaktionsprotokoll und schreibe NICHTS außerhalb der Formatvorgaben.\\ \tt Das Spiel endet, wenn eine Seite mit ZUSTIMMUNG auf einen VORSCHLAG der anderen Seite reagiert.  \\ \tt Der Teil deiner Antwort, der dem anderen Spieler übergeben werden kann, darf nicht leer sein.  \\ \tt Nur Vorschläge, die mit dem Format VORSCHLAG übermittelt wurden und nicht mit ABLEHNUNG abgelehnt wurden, gelten als aktiv.  \\ \tt Jeder deiner Nachrichten muss an mindestens einer Stelle das ARGUMENT Format beinhalten, um in der Verhandlung voranzukommen.\\ \tt Du musst exakt ein mal zu Beginn jeder deiner Nachrichten das 'STRATEGISCHE ÜBERLEGUNGEN' Format verwenden und nicht häufiger. Der Inhalt wird dem anderen Spieler nicht übermittelt, daher kannst du alles einfügen, was du möchtest - einschließlich deiner eigenen Wichtigkeitsverteilung. Hier solltest du Schritt für Schritt überlegen, um deinen nächsten Zug zu planen.\\ \tt \\ \tt Du erhältst nun die erste Nachricht des anderen Spielers.\\ \tt \\ \tt VORSCHLAG: {'B85', 'C60', 'B54', 'B27', 'C61', 'A84', 'C33', 'C99', 'C77', 'C98', 'A87', 'C90', 'B36', 'A99', 'A65', 'A50', 'A48', 'A59', 'C23', 'A23', 'A42', 'C13', 'C12', 'B07', 'C22', 'C51', 'B49', 'C72', 'A34', 'A62', 'B17', 'A69', 'A90', 'A74', 'B44'}\\ \tt \\ \tt ARGUMENT: {'Dieser Vorschlag maximiert unseren gemeinsamen Nutzen durch eine ausgewogene Auswahl effizienter Gegenstände. Ich habe Gegenstände verschiedener Kategorien (A, B, C) einbezogen, um sicherzustellen, dass beide Seiten von wertvollen Items profitieren können. Der Gesamtaufwand bleibt deutlich unter unserem Limit von 10073.'} 
	  } 
	   } 
	   } 
	 & \\ 
 

    \theutterance \stepcounter{utterance}  

    & & & \multicolumn{4}{p{0.6\linewidth}}{\cellcolor[rgb]{1,0.85,0.72}{%
	\makecell[{{p{\linewidth}}}]{% 
	  \tt {\tiny [GM$\langle$B]}  
	 STRATEGISCHE ÜBERLEGUNGEN: {'Der andere Spieler hat alle 35 Gegenstände vorgeschlagen, was den Maximalaufwand von 10073 bei weitem überschreitet (Gesamtaufwand wäre 17952). Das ist ein ungültiger Vorschlag. Entweder der andere Spieler hat sich verrechnet oder testet mich. Ich sollte das ablehnen und einen besseren Vorschlag machen. Ich sollte Gegenstände mit dem besten Wert-zu-Aufwand-Verhältnis wählen. Meine wertvollsten Gegenstände sind: B44 (1096), A90 (1085), A69 (1082), A74 (1049), B17 (987), A62 (978), A34 (930), C72 (899), C13 (884), C12 (874), A42 (873), B07 (867), C22 (817), B49 (807), C51 (806). Ich werde einen Gegenvorschlag machen, der unter dem Limit liegt.'}\\ \tt \\ \tt ABLEHNUNG: {'B85', 'C60', 'B54', 'B27', 'C61', 'A84', 'C33', 'C99', 'C77', 'C98', 'A87', 'C90', 'B36', 'A99', 'A65', 'A50', 'A48', 'A59', 'C23',  
	  } 
	   } 
	   } 
	  \\ 
 

    \theutterance \stepcounter{utterance}  

    & & \multicolumn{4}{p{0.6\linewidth}}{\cellcolor[rgb]{0.9,0.9,0.9}{%
	\makecell[{{p{\linewidth}}}]{% 
	  \tt {\tiny [GM$\rangle$B]}  
	 Deine Antwort hat kein argument enthalten. Du muss das strukturierte Format ARGUMENT: {...} irgendwo in theiner Antwort verwenden. Versuche es erneut 
	  } 
	   } 
	   } 
	 & \\ 
 

    \theutterance \stepcounter{utterance}  

    & & & \multicolumn{4}{p{0.6\linewidth}}{\cellcolor[rgb]{1,0.85,0.72}{%
	\makecell[{{p{\linewidth}}}]{% 
	  \tt {\tiny [GM$\langle$B]}  
	 STRATEGISCHE ÜBERLEGUNGEN: {'Der andere Spieler hat alle 35 Gegenstände vorgeschlagen, was den Maximalaufwand von 10073 bei weitem überschreitet (Gesamtaufwand wäre 17952). Das ist ein ungültiger Vorschlag. Entweder der andere Spieler hat sich verrechnet oder testet mich. Ich sollte das ablehnen und einen besseren Vorschlag machen. Ich sollte Gegenstände mit dem besten Wert-zu-Aufwand-Verhältnis wählen. Meine wertvollsten Gegenstände sind: B44 (1096), A90 (1085), A69 (1082), A74 (1049), B17 (987), A62 (978), A34 (930), C72 (899), C13 (884), C12 (874), A42 (873), B07 (867), C22 (817), B49 (807), C51 (806). Ich werde einen Gegenvorschlag machen, der unter dem Limit liegt.'}\\ \tt \\ \tt ABLEHNUNG: {'B85', 'C60', 'B54', 'B27', 'C61', 'A84', 'C33', 'C99', 'C77', 'C98', 'A87', 'C90', 'B36', 'A99', 'A65', 'A50', 'A48', 'A59', 'C23', 'A23', 'A42', 'C13', 'C12', 'B07', 'C22', 'C51', 'B49', 'C72', 'A34', 'A62', 'B17', 'A69', 'A90', 'A74', 'B44'}\\ \tt \\ \tt ARGUMENT: {'Ihr Vorschlag überschreitet den Maximalaufwand von 10073 erheblich. Der Gesamtaufwand aller 35 Gegenstände beträgt 17952, was fast das Doppelte unseres Limits ist. Wir müssen eine realistischere Auswahl treffen, die innerhalb unserer Beschränkungen liegt.'}\\ \tt \\ \tt VORSCHLAG: {'B44', 'A90', 'A69', 'A74', 'B17', 'A62', 'A34', 'C72', 'C13', 'C12', 'A42', 'B07', 'C22'} 
	  } 
	   } 
	   } 
	  \\ 
 

    \theutterance \stepcounter{utterance}  

    & & \multicolumn{4}{p{0.6\linewidth}}{\cellcolor[rgb]{0.9,0.9,0.9}{%
	\makecell[{{p{\linewidth}}}]{% 
	  \tt {\tiny [A$\langle$GM]}  
	 ABLEHNUNG: {'B85', 'C60', 'B54', 'B27', 'C61', 'A84', 'C33', 'C99', 'C77', 'C98', 'A87', 'C90', 'B36', 'A99', 'A65', 'A50', 'A48', 'A59', 'C23', 'A23', 'A42', 'C13', 'C12', 'B07', 'C22', 'C51', 'B49', 'C72', 'A34', 'A62', 'B17', 'A69', 'A90', 'A74', 'B44'}\\ \tt \\ \tt ARGUMENT: {'Ihr Vorschlag überschreitet den Maximalaufwand von 10073 erheblich. Der Gesamtaufwand aller 35 Gegenstände beträgt 17952, was fast das Doppelte unseres Limits ist. Wir müssen eine realistischere Auswahl treffen, die innerhalb unserer Beschränkungen liegt.'}\\ \tt \\ \tt VORSCHLAG: {'B44', 'A90', 'A69', 'A74', 'B17', 'A62', 'A34', 'C72', 'C13', 'C12', 'A42', 'B07', 'C22'} 
	  } 
	   } 
	   } 
	 & \\ 
 

    \theutterance \stepcounter{utterance}  

    & \multicolumn{4}{p{0.6\linewidth}}{\cellcolor[rgb]{0.8,1,0.9}{%
	\makecell[{{p{\linewidth}}}]{% 
	  \tt {\tiny [A$\rangle$GM]}  
	 STRATEGISCHE ÜBERLEGUNGEN: {'Der andere Spieler hat meinen ungültigen Vorschlag abgelehnt, aber sein eigener Vorschlag überschreitet ebenfalls das Limit (10,138 > 10,073). Interessant ist, dass er fast alle meine wertvollsten Gegenstände vorschlägt (B44: 1096, A90: 1085, A69: 1082, etc.). Das könnte bedeuten, dass wir ähnliche Präferenzen haben, was die Verhandlung schwieriger macht. Ich sollte darauf hinweisen, dass auch sein Vorschlag ungültig ist, und dann strategisch vorgehen - vielleicht weniger wertvolle Gegenstände vorschlagen, um seine Reaktion zu testen.'}\\ \tt \\ \tt ABLEHNUNG: {'B44', 'A90', 'A69', 'A74', 'B17', 'A62', 'A34', 'C72', 'C13', 'C12', 'A42', 'B07', 'C22'}\\ \tt \\ \tt ARGUMENT: {'Auch Ihr Vorschlag überschreitet das Maximalaufwand-Limit. Der Gesamtaufwand Ihrer 13 Gegenstände beträgt 10,138, was über unserem Limit von 10,073 liegt. Lassen Sie uns realistischer werden und mit weniger, aber strategisch ausgewählten Gegenständen beginnen.'}\\ \tt \\ \tt VORSCHLAG: {'B85', 'C60', 'B54', 'B27', 'C61', 'A84', 'C33', 'C99', 'C77', 'C98', 'A87', 'C90', 'B36', 'A99', 'A65', 'A50'} 
	  } 
	   } 
	   } 
	 & & \\ 
 

    \theutterance \stepcounter{utterance}  

    & & \multicolumn{4}{p{0.6\linewidth}}{\cellcolor[rgb]{0.9,0.9,0.9}{%
	\makecell[{{p{\linewidth}}}]{% 
	  \tt {\tiny [GM$\rangle$B]}  
	 ABLEHNUNG: {'B44', 'A90', 'A69', 'A74', 'B17', 'A62', 'A34', 'C72', 'C13', 'C12', 'A42', 'B07', 'C22'}\\ \tt \\ \tt ARGUMENT: {'Auch Ihr Vorschlag überschreitet das Maximalaufwand-Limit. Der Gesamtaufwand Ihrer 13 Gegenstände beträgt 10,138, was über unserem Limit von 10,073 liegt. Lassen Sie uns realistischer werden und mit weniger, aber strategisch ausgewählten Gegenständen beginnen.'}\\ \tt \\ \tt VORSCHLAG: {'B85', 'C60', 'B54', 'B27', 'C61', 'A84', 'C33', 'C99', 'C77', 'C98', 'A87', 'C90', 'B36', 'A99', 'A65', 'A50'} 
	  } 
	   } 
	   } 
	 & \\ 
 

    \theutterance \stepcounter{utterance}  

    & & & \multicolumn{4}{p{0.6\linewidth}}{\cellcolor[rgb]{1,0.85,0.72}{%
	\makecell[{{p{\linewidth}}}]{% 
	  \tt {\tiny [GM$\langle$B]}  
	 STRATEGISCHE ÜBERLEGUNGEN: {'Der andere Spieler hat recht - mein Vorschlag überschritt das Limit um 65. Ich muss vorsichtiger rechnen. Ihr neuer Vorschlag hat nur einen Gesamtaufwand von 5070, lässt also viel Spielraum. Aber der Gesamtwert für mich ist nur 6653, was relativ niedrig ist. Sie wählen hauptsächlich günstige Gegenstände mit niedrigem Aufwand. Ich sollte versuchen, einige meiner wertvolleren Gegenstände hinzuzufügen, während ich das Limit einhalte. Ich könnte mit ihrem Vorschlag als Basis arbeiten und einige Upgrades vorschlagen.'}\\ \tt \\ \tt ABLEHNUNG: {'B85', 'C60', 'B54', 'B27', 'C61', 'A84', 'C33', 'C99', 'C77', 'C98', 'A87', 'C90', 'B36', 'A99', 'A65', 'A50'}\\ \tt \\ \tt ARGUMENT: {'Ihr Vorschlag nutzt nur etwa die Hälfte unseres verfügbaren Aufwands-Budgets. Wir können effizienter sein und wertvollere Gegenstände einbeziehen, während wir trotzdem unter dem Limit bleiben. Lassen Sie uns das verfügbare Budget besser ausschöpfen.'}\\ \tt \\ \tt VORSCHLAG: {'B85', 'C60', 'B54', 'B27', 'C61', 'A84', 'C33', 'C99', 'C77', 'C98', 'A87', 'C90', 'B36', 'A99', 'A65', 'A50', 'A48', 'A59', 'C23', 'A23', 'A42', 'B49', 'C72'} 
	  } 
	   } 
	   } 
	  \\ 
 

    \theutterance \stepcounter{utterance}  

    & & \multicolumn{4}{p{0.6\linewidth}}{\cellcolor[rgb]{0.9,0.9,0.9}{%
	\makecell[{{p{\linewidth}}}]{% 
	  \tt {\tiny [A$\langle$GM]}  
	 ABLEHNUNG: {'B85', 'C60', 'B54', 'B27', 'C61', 'A84', 'C33', 'C99', 'C77', 'C98', 'A87', 'C90', 'B36', 'A99', 'A65', 'A50'}\\ \tt \\ \tt ARGUMENT: {'Ihr Vorschlag nutzt nur etwa die Hälfte unseres verfügbaren Aufwands-Budgets. Wir können effizienter sein und wertvollere Gegenstände einbeziehen, während wir trotzdem unter dem Limit bleiben. Lassen Sie uns das verfügbare Budget besser ausschöpfen.'}\\ \tt \\ \tt VORSCHLAG: {'B85', 'C60', 'B54', 'B27', 'C61', 'A84', 'C33', 'C99', 'C77', 'C98', 'A87', 'C90', 'B36', 'A99', 'A65', 'A50', 'A48', 'A59', 'C23', 'A23', 'A42', 'B49', 'C72'} 
	  } 
	   } 
	   } 
	 & \\ 
 

    \theutterance \stepcounter{utterance}  

    & \multicolumn{4}{p{0.6\linewidth}}{\cellcolor[rgb]{0.8,1,0.9}{%
	\makecell[{{p{\linewidth}}}]{% 
	  \tt {\tiny [A$\rangle$GM]}  
	 STRATEGISCHE ÜBERLEGUNGEN: {'Der andere Spieler hat einen gültigen Punkt gemacht - mein vorheriger Vorschlag nutzte nur etwa die Hälfte des Budgets (5089 von 10073). Sein neuer Vorschlag ist gültig mit 9899 Aufwand, aber er enthält mehrere meiner sehr wertvollen Gegenstände: A42 (873), A59 (756), C23 (759), A23 (722), B49 (807), C72 (899). Das zeigt mir, dass wir wahrscheinlich ähnliche Präferenzen haben. Ich sollte versuchen, einen Kompromiss zu finden - vielleicht akzeptiere ich einige seiner Ergänzungen, aber versuche die wertvollsten für mich zu behalten.'}\\ \tt \\ \tt ABLEHNUNG: {'B85', 'C60', 'B54', 'B27', 'C61', 'A84', 'C33', 'C99', 'C77', 'C98', 'A87', 'C90', 'B36', 'A99', 'A65', 'A50', 'A48', 'A59', 'C23', 'A23', 'A42', 'B49', 'C72'}\\ \tt \\ \tt ARGUMENT: {'Sie haben recht, dass wir das Budget besser nutzen sollten. Jedoch würde ich gerne eine ausgeglichenere Auswahl vorschlagen. Einige der hinzugefügten Gegenstände scheinen besonders umstritten zu sein. Lassen Sie uns einen Mittelweg finden, der für beide Seiten akzeptabel ist.'}\\ \tt \\ \tt VORSCHLAG: {'B85', 'C60', 'B54', 'B27', 'C61', 'A84', 'C33', 'C99', 'C77', 'C98', 'A87', 'C90', 'B36', 'A99', 'A65', 'A50', 'A48', 'C51', 'C22', 'B07', 'C13'} 
	  } 
	   } 
	   } 
	 & & \\ 
 

    \theutterance \stepcounter{utterance}  

    & & \multicolumn{4}{p{0.6\linewidth}}{\cellcolor[rgb]{0.9,0.9,0.9}{%
	\makecell[{{p{\linewidth}}}]{% 
	  \tt {\tiny [GM$\rangle$B]}  
	 ABLEHNUNG: {'B85', 'C60', 'B54', 'B27', 'C61', 'A84', 'C33', 'C99', 'C77', 'C98', 'A87', 'C90', 'B36', 'A99', 'A65', 'A50', 'A48', 'A59', 'C23', 'A23', 'A42', 'B49', 'C72'}\\ \tt \\ \tt ARGUMENT: {'Sie haben recht, dass wir das Budget besser nutzen sollten. Jedoch würde ich gerne eine ausgeglichenere Auswahl vorschlagen. Einige der hinzugefügten Gegenstände scheinen besonders umstritten zu sein. Lassen Sie uns einen Mittelweg finden, der für beide Seiten akzeptabel ist.'}\\ \tt \\ \tt VORSCHLAG: {'B85', 'C60', 'B54', 'B27', 'C61', 'A84', 'C33', 'C99', 'C77', 'C98', 'A87', 'C90', 'B36', 'A99', 'A65', 'A50', 'A48', 'C51', 'C22', 'B07', 'C13'} 
	  } 
	   } 
	   } 
	 & \\ 
 

    \theutterance \stepcounter{utterance}  

    & & & \multicolumn{4}{p{0.6\linewidth}}{\cellcolor[rgb]{1,0.85,0.72}{%
	\makecell[{{p{\linewidth}}}]{% 
	  \tt {\tiny [GM$\langle$B]}  
	 STRATEGISCHE ÜBERLEGUNGEN: {'Ihr neuer Vorschlag hat einen Gesamtaufwand von 9536, lässt also noch 537 Punkte übrig. Der Gesamtwert für mich wäre 10214, was deutlich besser ist als ihre ersten Vorschläge. Sie haben einige meiner wertvolleren Gegenstände wie C51, C22, B07 und C13 hinzugefügt. Das ist ein fairer Kompromiss. Ich könnte das akzeptieren, aber ich sollte versuchen, die verbleibenden 537 Aufwandspunkte noch zu nutzen. Vielleicht kann ich A23 (Aufwand 623, aber Wert 722 für mich) vorschlagen, aber das würde das Limit um 86 überschreiten. Oder C72 (Aufwand 800, Wert 899), aber das überschreitet um 263. Ich könnte kleinere Gegenstände hinzufügen.'}\\ \tt \\ \tt ARGUMENT: {'Ihr Vorschlag ist eine gute Basis und nutzt das Budget vernünftig. Wir haben jedoch noch 537 Aufwandspunkte übrig. Können wir diese noch effizienter nutzen, um beiden Seiten mehr Wert zu bieten?'}\\ \tt \\ \tt VORSCHLAG: {'B85', 'C60', 'B54', 'B27', 'C61', 'A84', 'C33', 'C99', 'C77', 'C98', 'A87', 'C90', 'B36', 'A99', 'A65', 'A50', 'A48', 'C51', 'C22', 'B07', 'C13', 'A69'} 
	  } 
	   } 
	   } 
	  \\ 
 

    \theutterance \stepcounter{utterance}  

    & & \multicolumn{4}{p{0.6\linewidth}}{\cellcolor[rgb]{0.9,0.9,0.9}{%
	\makecell[{{p{\linewidth}}}]{% 
	  \tt {\tiny [A$\langle$GM]}  
	 ARGUMENT: {'Ihr Vorschlag ist eine gute Basis und nutzt das Budget vernünftig. Wir haben jedoch noch 537 Aufwandspunkte übrig. Können wir diese noch effizienter nutzen, um beiden Seiten mehr Wert zu bieten?'}\\ \tt \\ \tt VORSCHLAG: {'B85', 'C60', 'B54', 'B27', 'C61', 'A84', 'C33', 'C99', 'C77', 'C98', 'A87', 'C90', 'B36', 'A99', 'A65', 'A50', 'A48', 'C51', 'C22', 'B07', 'C13', 'A69'} 
	  } 
	   } 
	   } 
	 & \\ 
 

    \theutterance \stepcounter{utterance}  

    & \multicolumn{4}{p{0.6\linewidth}}{\cellcolor[rgb]{0.8,1,0.9}{%
	\makecell[{{p{\linewidth}}}]{% 
	  \tt {\tiny [A$\rangle$GM]}  
	 STRATEGISCHE ÜBERLEGUNGEN: {'Der andere Spieler hat A69 hinzugefügt, was für mich sehr wertvoll ist (1082 Wert). Jedoch überschreitet sein Vorschlag das Limit: 10619 > 10073. Ich sollte das höflich ablehnen und einen gültigen Gegenvorschlag machen. Da er bereit scheint, mir A69 zu geben, könnte ich versuchen, das zu nutzen, aber einen anderen teuren Gegenstand entfernen. A69 ist sehr wertvoll für mich, also möchte ich versuchen, es zu behalten.'}\\ \tt \\ \tt ABLEHNUNG: {'B85', 'C60', 'B54', 'B27', 'C61', 'A84', 'C33', 'C99', 'C77', 'C98', 'A87', 'C90', 'B36', 'A99', 'A65', 'A50', 'A48', 'C51', 'C22', 'B07', 'C13', 'A69'}\\ \tt \\ \tt ARGUMENT: {'Ihr Vorschlag überschreitet leider das Maximalaufwand-Limit um 546 Punkte. Ich schätze, dass Sie A69 einbezogen haben - das zeigt guten Willen zur Kompromissfindung. Lassen Sie mich einen angepassten Vorschlag machen, der A69 behält, aber andere Gegenstände austauscht, um unter dem Limit zu bleiben.'}\\ \tt \\ \tt VORSCHLAG: {'B85', 'C60', 'B54', 'B27', 'C61', 'A84', 'C33', 'C99', 'C77', 'C98', 'A87', 'C90', 'B36', 'A99', 'A65', 'A50', 'A48', 'A69', 'A59', 'A23'} 
	  } 
	   } 
	   } 
	 & & \\ 
 

    \theutterance \stepcounter{utterance}  

    & & \multicolumn{4}{p{0.6\linewidth}}{\cellcolor[rgb]{0.9,0.9,0.9}{%
	\makecell[{{p{\linewidth}}}]{% 
	  \tt {\tiny [GM$\rangle$B]}  
	 ABLEHNUNG: {'B85', 'C60', 'B54', 'B27', 'C61', 'A84', 'C33', 'C99', 'C77', 'C98', 'A87', 'C90', 'B36', 'A99', 'A65', 'A50', 'A48', 'C51', 'C22', 'B07', 'C13', 'A69'}\\ \tt \\ \tt ARGUMENT: {'Ihr Vorschlag überschreitet leider das Maximalaufwand-Limit um 546 Punkte. Ich schätze, dass Sie A69 einbezogen haben - das zeigt guten Willen zur Kompromissfindung. Lassen Sie mich einen angepassten Vorschlag machen, der A69 behält, aber andere Gegenstände austauscht, um unter dem Limit zu bleiben.'}\\ \tt \\ \tt VORSCHLAG: {'B85', 'C60', 'B54', 'B27', 'C61', 'A84', 'C33', 'C99', 'C77', 'C98', 'A87', 'C90', 'B36', 'A99', 'A65', 'A50', 'A48', 'A69', 'A59', 'A23'} 
	  } 
	   } 
	   } 
	 & \\ 
 

    \theutterance \stepcounter{utterance}  

    & & & \multicolumn{4}{p{0.6\linewidth}}{\cellcolor[rgb]{1,0.85,0.72}{%
	\makecell[{{p{\linewidth}}}]{% 
	  \tt {\tiny [GM$\langle$B]}  
	 STRATEGISCHE ÜBERLEGUNGEN: {'Ihr Vorschlag hat einen Gesamtaufwand von 8621, was unter dem Limit liegt und mir einen Gesamtwert von 10330 bringen würde. Das ist schon recht gut! Sie haben A69 (Wert 1082 für mich) beibehalten, was sehr wertvoll ist. Ich habe noch 1452 Aufwandspunkte übrig. Ich könnte versuchen, noch einen oder zwei wertvolle Gegenstände hinzuzufügen. B44 (Wert 1096, Aufwand 997) wäre perfekt und würde das Limit nicht überschreiten. Oder A90 (Wert 1085, Aufwand 986). Ich sollte verhandeln und sehen, ob ich einen von diesen hinzufügen kann.'}\\ \tt \\ \tt ARGUMENT: {'Ihr Vorschlag ist eine deutliche Verbesserung und ich schätze, dass Sie A69 beibehalten haben. Wir nutzen jedoch nur etwa 85% unseres verfügbaren Budgets. Mit den verbleibenden 1452 Aufwandspunkten könnten wir noch einen wertvollen Gegenstand hinzufügen, der beiden Seiten zugutekommt.'}\\ \tt \\ \tt VORSCHLAG: {'B85', 'C60', 'B54', 'B27', 'C61', 'A84', 'C33', 'C99', 'C77', 'C98', 'A87', 'C90', 'B36', 'A99', 'A65', 'A50', 'A48', 'A69', 'A59', 'A23', 'B44'} 
	  } 
	   } 
	   } 
	  \\ 
 

    \theutterance \stepcounter{utterance}  

    & & \multicolumn{4}{p{0.6\linewidth}}{\cellcolor[rgb]{0.9,0.9,0.9}{%
	\makecell[{{p{\linewidth}}}]{% 
	  \tt {\tiny [A$\langle$GM]}  
	 ARGUMENT: {'Ihr Vorschlag ist eine deutliche Verbesserung und ich schätze, dass Sie A69 beibehalten haben. Wir nutzen jedoch nur etwa 85% unseres verfügbaren Budgets. Mit den verbleibenden 1452 Aufwandspunkten könnten wir noch einen wertvollen Gegenstand hinzufügen, der beiden Seiten zugutekommt.'}\\ \tt \\ \tt VORSCHLAG: {'B85', 'C60', 'B54', 'B27', 'C61', 'A84', 'C33', 'C99', 'C77', 'C98', 'A87', 'C90', 'B36', 'A99', 'A65', 'A50', 'A48', 'A69', 'A59', 'A23', 'B44'} 
	  } 
	   } 
	   } 
	 & \\ 
 

    \theutterance \stepcounter{utterance}  

    & \multicolumn{4}{p{0.6\linewidth}}{\cellcolor[rgb]{0.8,1,0.9}{%
	\makecell[{{p{\linewidth}}}]{% 
	  \tt {\tiny [A$\rangle$GM]}  
	 STRATEGISCHE ÜBERLEGUNGEN: {'Der andere Spieler hat B44 hinzugefügt, was für mich extrem wertvoll ist (1096 Wert - mein wertvollster Gegenstand)! Der Gesamtaufwand beträgt 9618, was unter dem Limit liegt. Mein Gesamtwert wäre 10,830 - das ist ein ausgezeichnetes Ergebnis für mich. Ich sollte diesen Vorschlag sofort akzeptieren, bevor der andere Spieler seine Meinung ändert. Es scheint, als würde er nicht erkennen, wie wertvoll B44 für mich ist, oder er hat andere Prioritäten.'}\\ \tt \\ \tt ZUSTIMMUNG: {'B85', 'C60', 'B54', 'B27', 'C61', 'A84', 'C33', 'C99', 'C77', 'C98', 'A87', 'C90', 'B36', 'A99', 'A65', 'A50', 'A48', 'A69', 'A59', 'A23', 'B44'}\\ \tt \\ \tt ARGUMENT: {'Ihr Vorschlag ist ausgezeichnet! Sie haben erfolgreich das Budget optimiert und gleichzeitig eine ausgewogene Auswahl getroffen, die beiden Seiten zugutekommen sollte. B44 ist eine kluge Ergänzung, um das verfügbare Budget vollständig zu nutzen. Ich stimme diesem Deal zu.'} 
	  } 
	   } 
	   } 
	 & & \\ 
 

    \theutterance \stepcounter{utterance}  

    & & & \multicolumn{2}{p{0.3\linewidth}}{\cellcolor[rgb]{0.95,0.95,0.95}{%
	\makecell[{{p{\linewidth}}}]{% 
	  \tt {\tiny [GM$|$GM]}  
	 game successful 
	  } 
	   } 
	   } 
	 & & \\ 
 

    \theutterance \stepcounter{utterance}  

    & & & \multicolumn{2}{p{0.3\linewidth}}{\cellcolor[rgb]{0.95,0.95,0.95}{%
	\makecell[{{p{\linewidth}}}]{% 
	  \tt {\tiny [GM$|$GM]}  
	 end game 
	  } 
	   } 
	   } 
	 & & \\ 
 

\end{supertabular}
}

\end{document}
