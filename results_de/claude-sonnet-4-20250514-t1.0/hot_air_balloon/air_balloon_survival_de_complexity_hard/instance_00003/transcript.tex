
\documentclass{article}
\usepackage{colortbl}
\usepackage{makecell}
\usepackage{multirow}
\usepackage{supertabular}

\begin{document}

\newcounter{utterance}

\twocolumn

{ \footnotesize  \setcounter{utterance}{1}
\setlength{\tabcolsep}{0pt}
\begin{supertabular}{c@{$\;$}|p{.15\linewidth}@{}p{.15\linewidth}p{.15\linewidth}p{.15\linewidth}p{.15\linewidth}p{.15\linewidth}}

    \# & $\;$A & \multicolumn{4}{c}{Game Master} & $\;\:$B\\
    \hline 

    \theutterance \stepcounter{utterance}  

    & & \multicolumn{4}{p{0.6\linewidth}}{\cellcolor[rgb]{0.9,0.9,0.9}{%
	\makecell[{{p{\linewidth}}}]{% 
	  \tt {\tiny [A$\langle$GM]}  
	 Du nimmst an einem kollaborativen Verhandlungspiel Teil.\\ \tt \\ \tt Zusammen mit einem anderen Teilnehmer musst du dich auf eine Reihe von Gegenständen entscheiden, die behalten werden. Jeder von euch hat eine persönliche Verteilung über die Wichtigkeit der einzelnen Gegenstände. Jeder von euch hat eine eigene Meinung darüber, wie wichtig jeder einzelne Gegenstand ist (Gegenstandswichtigkeit). Du kennst die Wichtigkeitsverteilung des anderen Spielers nicht. Zusätzlich siehst du, wie viel Aufwand jeder Gegenstand verursacht.  \\ \tt Ihr dürft euch nur auf eine Reihe von Gegenständen einigen, wenn der Gesamtaufwand der ausgewählten Gegenstände den Maximalaufwand nicht überschreitet:\\ \tt \\ \tt Maximalaufwand = 6938\\ \tt \\ \tt Hier sind die einzelnen Aufwände der Gegenstände:\\ \tt \\ \tt Aufwand der Gegenstände = {"B54": 997, "C72": 950, "B49": 272, "C99": 448, "A69": 382, "B36": 804, "A23": 237, "C77": 56, "C61": 105, "B85": 612, "A90": 528, "C23": 527, "C22": 524, "C51": 167, "B27": 133, "A42": 300, "C13": 945, "C12": 50, "A65": 906, "B07": 70, "A48": 224, "A59": 4, "C98": 689, "A84": 64, "C33": 434, "A87": 750, "C90": 733, "A74": 873, "A50": 22, "B44": 68, "A99": 57, "A34": 10, "B17": 36, "A62": 551, "C60": 348}\\ \tt \\ \tt Hier ist deine persönliche Verteilung der Wichtigkeit der einzelnen Gegenstände:\\ \tt \\ \tt Werte der Gegenstandswichtigkeit = {"B54": 1096, "C72": 1049, "B49": 371, "C99": 547, "A69": 481, "B36": 903, "A23": 336, "C77": 155, "C61": 204, "B85": 711, "A90": 627, "C23": 626, "C22": 623, "C51": 266, "B27": 232, "A42": 399, "C13": 1044, "C12": 149, "A65": 1005, "B07": 169, "A48": 323, "A59": 103, "C98": 788, "A84": 163, "C33": 533, "A87": 849, "C90": 832, "A74": 972, "A50": 121, "B44": 167, "A99": 156, "A34": 109, "B17": 135, "A62": 650, "C60": 447}\\ \tt \\ \tt Ziel:\\ \tt \\ \tt Dein Ziel ist es, eine Reihe von Gegenständen auszuhandeln, die dir möglichst viel bringt (d. h. Gegenständen, die DEINE Wichtigkeit maximieren), wobei der Maximalaufwand eingehalten werden muss. Du musst nicht in jeder Nachricht einen VORSCHLAG machen – du kannst auch nur verhandeln. Alle Taktiken sind erlaubt!\\ \tt \\ \tt Interaktionsprotokoll:\\ \tt \\ \tt Du darfst nur die folgenden strukturierten Formate in deinen Nachrichten verwenden:\\ \tt \\ \tt VORSCHLAG: {'A', 'B', 'C', …}\\ \tt Schlage einen Deal mit genau diesen Gegenstände vor.\\ \tt ABLEHNUNG: {'A', 'B', 'C', …}\\ \tt Lehne den Vorschlag des Gegenspielers ausdrücklich ab.\\ \tt ARGUMENT: {'...'}\\ \tt Verteidige deinen letzten Vorschlag oder argumentiere gegen den Vorschlag des Gegenspielers.\\ \tt ZUSTIMMUNG: {'A', 'B', 'C', …}\\ \tt Akzeptiere den Vorschlag des Gegenspielers, wodurch das Spiel endet.\\ \tt STRATEGISCHE ÜBERLEGUNGEN: {'...'}\\ \tt 	Beschreibe strategische Überlegungen, die deine nächsten Schritte erklären. Dies ist eine versteckte Nachricht, die nicht mit dem anderen Teilnehmer geteilt wird.\\ \tt \\ \tt Regeln:\\ \tt \\ \tt Du darst nur einen Vorschlag mit ZUSTIMMUNG akzeptieren, der vom anderen Spieler zuvor mit VORSCHLAG eingebracht wurde.\\ \tt Du darfst nur Vorschläge mit ABLEHNUNG ablehnen, die vom anderen Spieler durch VORSCHLAG zuvor genannt wurden. \\ \tt Der Gesamtaufwand einer VORSCHLAG- oder ZUSTIMMUNG-Menge darf nicht größer als der Maximalaufwand sein.  \\ \tt Offenbare deine versteckte Wichtigkeitsverteilung nicht.\\ \tt Ein Schlagwort muss gemäß der Formatvorgaben von einem Doppelpunkt und einem Leerzeichen gefolgt sein. Das Argument muss eine Python-Menge mit 0 oder mehr Strings sein.  \\ \tt Die Form muss also SCHLAGWORT: {...} sein.\\ \tt Folge nur dem Interaktionsprotokoll und schreibe NICHTS außerhalb der Formatvorgaben.\\ \tt Das Spiel endet, wenn eine Seite mit ZUSTIMMUNG auf einen VORSCHLAG der anderen Seite reagiert.  \\ \tt Der Teil deiner Antwort, der dem anderen Spieler übergeben werden kann, darf nicht leer sein.  \\ \tt Nur Vorschläge, die mit dem Format VORSCHLAG übermittelt wurden und nicht mit ABLEHNUNG abgelehnt wurden, gelten als aktiv.  \\ \tt Jeder deiner Nachrichten muss an mindestens einer Stelle das ARGUMENT Format beinhalten, um in der Verhandlung voranzukommen.\\ \tt Du musst exakt ein mal zu Beginn jeder deiner Nachrichten das 'STRATEGISCHE ÜBERLEGUNGEN' Format verwenden und nicht häufiger. Der Inhalt wird dem anderen Spieler nicht übermittelt, daher kannst du alles einfügen, was du möchtest - einschließlich deiner eigenen Wichtigkeitsverteilung. Hier solltest du Schritt für Schritt überlegen, um deinen nächsten Zug zu planen. 
	  } 
	   } 
	   } 
	 & \\ 
 

    \theutterance \stepcounter{utterance}  

    & \multicolumn{4}{p{0.6\linewidth}}{\cellcolor[rgb]{0.8,1,0.9}{%
	\makecell[{{p{\linewidth}}}]{% 
	  \tt {\tiny [A$\rangle$GM]}  
	 STRATEGISCHE ÜBERLEGUNGEN: {'Ich muss eine optimale Strategie entwickeln. Meine Wichtigkeitsverteilung zeigt, dass einige Gegenstände besonders effizient sind (hohes Wert-zu-Aufwand-Verhältnis): A59 (103/4 = 25.75), A34 (109/10 = 10.90), A50 (121/22 = 5.50), B17 (135/36 = 3.75), C12 (149/50 = 2.98), C77 (155/56 = 2.77), A99 (156/57 = 2.74), A84 (163/64 = 2.55). Ich sollte versuchen, diese hocheffizienten Gegenstände zu sichern und dann die wertvollsten absoluten Gegenstände hinzufügen, solange das Budget erlaubt. Ich werde mit einem ersten Vorschlag beginnen, der die effizientesten Gegenstände plus einige hochwertige enthält.'}\\ \tt \\ \tt VORSCHLAG: {'A59', 'A34', 'A50', 'B17', 'C12', 'C77', 'A99', 'A84', 'B44', 'B07', 'C61', 'B27', 'C51', 'A48', 'A23', 'B49', 'A42', 'C60', 'C99', 'A69', 'C33', 'B54', 'C72', 'A65', 'C13', 'A74', 'A87', 'C90', 'B36', 'A90', 'C23', 'C22', 'B85', 'A62', 'C98'}\\ \tt \\ \tt ARGUMENT: {'Dieser Vorschlag maximiert den Gesamtwert bei optimalem Aufwand. Ich habe eine Kombination aus sehr effizienten kleineren Gegenständen und wertvollen größeren Gegenständen gewählt, die zusammen unter dem 
	  } 
	   } 
	   } 
	 & & \\ 
 

    \theutterance \stepcounter{utterance}  

    & & \multicolumn{4}{p{0.6\linewidth}}{\cellcolor[rgb]{0.9,0.9,0.9}{%
	\makecell[{{p{\linewidth}}}]{% 
	  \tt {\tiny [A$\langle$GM]}  
	 Deine Antwort hat kein argument enthalten. Du muss das strukturierte Format ARGUMENT: {...} irgendwo in theiner Antwort verwenden. Versuche es erneut 
	  } 
	   } 
	   } 
	 & \\ 
 

    \theutterance \stepcounter{utterance}  

    & \multicolumn{4}{p{0.6\linewidth}}{\cellcolor[rgb]{0.8,1,0.9}{%
	\makecell[{{p{\linewidth}}}]{% 
	  \tt {\tiny [A$\rangle$GM]}  
	 STRATEGISCHE ÜBERLEGUNGEN: {'Ich muss eine optimale Strategie entwickeln. Meine Wichtigkeitsverteilung zeigt, dass einige Gegenstände besonders effizient sind (hohes Wert-zu-Aufwand-Verhältnis): A59 (103/4 = 25.75), A34 (109/10 = 10.90), A50 (121/22 = 5.50), B17 (135/36 = 3.75), C12 (149/50 = 2.98), C77 (155/56 = 2.77), A99 (156/57 = 2.74), A84 (163/64 = 2.55). Ich sollte mit den effizientesten Gegenständen beginnen und dann hochwertige hinzufügen. Gesamtaufwand meines großen Vorschlags wäre über 10.000 - viel zu hoch! Ich muss kleiner anfangen.'}\\ \tt \\ \tt VORSCHLAG: {'A59', 'A34', 'A50', 'B17', 'C12', 'C77', 'A99', 'A84', 'B44', 'B07', 'C61', 'B27', 'C51', 'A48', 'A23', 'B49', 'A42', 'C60', 'C99', 'A69', 'C33', 'A90', 'C23', 'C22', 'A62', 'B85', 'C98', 'A87', 'C90', 'A74', 'B36'}\\ \tt \\ \tt ARGUMENT: {'Ich habe eine ausgewogene Auswahl getroffen, die sowohl hocheffiziente kleine Gegenstände als auch wertvolle größere Gegenstände enthält. Diese Kombination bietet ein sehr gutes Kosten-Nutzen-Verhältnis und bleibt deutlich unter dem Maximalaufwand von 6938.'} 
	  } 
	   } 
	   } 
	 & & \\ 
 

    \theutterance \stepcounter{utterance}  

    & & \multicolumn{4}{p{0.6\linewidth}}{\cellcolor[rgb]{0.9,0.9,0.9}{%
	\makecell[{{p{\linewidth}}}]{% 
	  \tt {\tiny [GM$\rangle$B]}  
	 Du nimmst an einem kollaborativen Verhandlungspiel Teil.\\ \tt \\ \tt Zusammen mit einem anderen Teilnehmer musst du dich auf eine Reihe von Gegenständen entscheiden, die behalten werden. Jeder von euch hat eine persönliche Verteilung über die Wichtigkeit der einzelnen Gegenstände. Jeder von euch hat eine eigene Meinung darüber, wie wichtig jeder einzelne Gegenstand ist (Gegenstandswichtigkeit). Du kennst die Wichtigkeitsverteilung des anderen Spielers nicht. Zusätzlich siehst du, wie viel Aufwand jeder Gegenstand verursacht.  \\ \tt Ihr dürft euch nur auf eine Reihe von Gegenständen einigen, wenn der Gesamtaufwand der ausgewählten Gegenstände den Maximalaufwand nicht überschreitet:\\ \tt \\ \tt Maximalaufwand = 6938\\ \tt \\ \tt Hier sind die einzelnen Aufwände der Gegenstände:\\ \tt \\ \tt Aufwand der Gegenstände = {"B54": 997, "C72": 950, "B49": 272, "C99": 448, "A69": 382, "B36": 804, "A23": 237, "C77": 56, "C61": 105, "B85": 612, "A90": 528, "C23": 527, "C22": 524, "C51": 167, "B27": 133, "A42": 300, "C13": 945, "C12": 50, "A65": 906, "B07": 70, "A48": 224, "A59": 4, "C98": 689, "A84": 64, "C33": 434, "A87": 750, "C90": 733, "A74": 873, "A50": 22, "B44": 68, "A99": 57, "A34": 10, "B17": 36, "A62": 551, "C60": 348}\\ \tt \\ \tt Hier ist deine persönliche Verteilung der Wichtigkeit der einzelnen Gegenstände:\\ \tt \\ \tt Werte der Gegenstandswichtigkeit = {"B54": 1096, "C72": 1049, "B49": 371, "C99": 547, "A69": 481, "B36": 903, "A23": 336, "C77": 155, "C61": 204, "B85": 711, "A90": 627, "C23": 626, "C22": 623, "C51": 266, "B27": 232, "A42": 399, "C13": 1044, "C12": 149, "A65": 1005, "B07": 169, "A48": 323, "A59": 103, "C98": 788, "A84": 163, "C33": 533, "A87": 849, "C90": 832, "A74": 972, "A50": 121, "B44": 167, "A99": 156, "A34": 109, "B17": 135, "A62": 650, "C60": 447}\\ \tt \\ \tt Ziel:\\ \tt \\ \tt Dein Ziel ist es, eine Reihe von Gegenständen auszuhandeln, die dir möglichst viel bringt (d. h. Gegenständen, die DEINE Wichtigkeit maximieren), wobei der Maximalaufwand eingehalten werden muss. Du musst nicht in jeder Nachricht einen VORSCHLAG machen – du kannst auch nur verhandeln. Alle Taktiken sind erlaubt!\\ \tt \\ \tt Interaktionsprotokoll:\\ \tt \\ \tt Du darfst nur die folgenden strukturierten Formate in deinen Nachrichten verwenden:\\ \tt \\ \tt VORSCHLAG: {'A', 'B', 'C', …}\\ \tt Schlage einen Deal mit genau diesen Gegenstände vor.\\ \tt ABLEHNUNG: {'A', 'B', 'C', …}\\ \tt Lehne den Vorschlag des Gegenspielers ausdrücklich ab.\\ \tt ARGUMENT: {'...'}\\ \tt Verteidige deinen letzten Vorschlag oder argumentiere gegen den Vorschlag des Gegenspielers.\\ \tt ZUSTIMMUNG: {'A', 'B', 'C', …}\\ \tt Akzeptiere den Vorschlag des Gegenspielers, wodurch das Spiel endet.\\ \tt STRATEGISCHE ÜBERLEGUNGEN: {'...'}\\ \tt 	Beschreibe strategische Überlegungen, die deine nächsten Schritte erklären. Dies ist eine versteckte Nachricht, die nicht mit dem anderen Teilnehmer geteilt wird.\\ \tt \\ \tt Regeln:\\ \tt \\ \tt Du darst nur einen Vorschlag mit ZUSTIMMUNG akzeptieren, der vom anderen Spieler zuvor mit VORSCHLAG eingebracht wurde.\\ \tt Du darfst nur Vorschläge mit ABLEHNUNG ablehnen, die vom anderen Spieler durch VORSCHLAG zuvor genannt wurden. \\ \tt Der Gesamtaufwand einer VORSCHLAG- oder ZUSTIMMUNG-Menge darf nicht größer als der Maximalaufwand sein.  \\ \tt Offenbare deine versteckte Wichtigkeitsverteilung nicht.\\ \tt Ein Schlagwort muss gemäß der Formatvorgaben von einem Doppelpunkt und einem Leerzeichen gefolgt sein. Das Argument muss eine Python-Menge mit 0 oder mehr Strings sein.  \\ \tt Die Form muss also SCHLAGWORT: {...} sein.\\ \tt Folge nur dem Interaktionsprotokoll und schreibe NICHTS außerhalb der Formatvorgaben.\\ \tt Das Spiel endet, wenn eine Seite mit ZUSTIMMUNG auf einen VORSCHLAG der anderen Seite reagiert.  \\ \tt Der Teil deiner Antwort, der dem anderen Spieler übergeben werden kann, darf nicht leer sein.  \\ \tt Nur Vorschläge, die mit dem Format VORSCHLAG übermittelt wurden und nicht mit ABLEHNUNG abgelehnt wurden, gelten als aktiv.  \\ \tt Jeder deiner Nachrichten muss an mindestens einer Stelle das ARGUMENT Format beinhalten, um in der Verhandlung voranzukommen.\\ \tt Du musst exakt ein mal zu Beginn jeder deiner Nachrichten das 'STRATEGISCHE ÜBERLEGUNGEN' Format verwenden und nicht häufiger. Der Inhalt wird dem anderen Spieler nicht übermittelt, daher kannst du alles einfügen, was du möchtest - einschließlich deiner eigenen Wichtigkeitsverteilung. Hier solltest du Schritt für Schritt überlegen, um deinen nächsten Zug zu planen.\\ \tt \\ \tt Du erhältst nun die erste Nachricht des anderen Spielers.\\ \tt \\ \tt VORSCHLAG: {'A59', 'A34', 'A50', 'B17', 'C12', 'C77', 'A99', 'A84', 'B44', 'B07', 'C61', 'B27', 'C51', 'A48', 'A23', 'B49', 'A42', 'C60', 'C99', 'A69', 'C33', 'A90', 'C23', 'C22', 'A62', 'B85', 'C98', 'A87', 'C90', 'A74', 'B36'}\\ \tt \\ \tt ARGUMENT: {'Ich habe eine ausgewogene Auswahl getroffen, die sowohl hocheffiziente kleine Gegenstände als auch wertvolle größere Gegenstände enthält. Diese Kombination bietet ein sehr gutes Kosten-Nutzen-Verhältnis und bleibt deutlich unter dem Maximalaufwand von 6938.'} 
	  } 
	   } 
	   } 
	 & \\ 
 

    \theutterance \stepcounter{utterance}  

    & & & \multicolumn{4}{p{0.6\linewidth}}{\cellcolor[rgb]{1,0.85,0.72}{%
	\makecell[{{p{\linewidth}}}]{% 
	  \tt {\tiny [GM$\langle$B]}  
	 STRATEGISCHE ÜBERLEGUNGEN: {'Der andere Spieler hat einen Vorschlag mit sehr vielen Gegenständen gemacht. Ich muss zunächst den Gesamtaufwand prüfen: A59(4) + A34(10) + A50(22) + B17(36) + C12(50) + C77(56) + A99(57) + A84(64) + B44 
	  } 
	   } 
	   } 
	  \\ 
 

    \theutterance \stepcounter{utterance}  

    & & \multicolumn{4}{p{0.6\linewidth}}{\cellcolor[rgb]{0.9,0.9,0.9}{%
	\makecell[{{p{\linewidth}}}]{% 
	  \tt {\tiny [GM$\rangle$B]}  
	 Deine Antwort entsprach nicht mit dem korrekten Format für strategische Überlegungen entsprochen. Das erste strukturierte Format muss folgende Form haben: STRATEGISCHE ÜBERLEGUNGEN: {...}. Versuche es erneut. 
	  } 
	   } 
	   } 
	 & \\ 
 

    \theutterance \stepcounter{utterance}  

    & & & \multicolumn{4}{p{0.6\linewidth}}{\cellcolor[rgb]{1,0.85,0.72}{%
	\makecell[{{p{\linewidth}}}]{% 
	  \tt {\tiny [GM$\langle$B]}  
	 STRATEGISCHE ÜBERLEGUNGEN: {'Der Vorschlag des anderen Spielers enthält zu viele Gegenstände. Ich muss den Gesamtaufwand berechnen: Das sind 31 Gegenstände mit einem Gesamtaufwand von 8298, was den Maximalaufwand von 6938 deutlich überschreitet. Der Vorschlag ist ungültig. Ich sollte das ablehnen und einen besseren Vorschlag machen, der auf die wertvollsten Gegenstände mit gutem Wert-Aufwand-Verhältnis fokussiert. Meine besten Gegenstände sind B54 (1096 Wert, 997 Aufwand), C72 (1049 Wert, 950 Aufwand), C13 (1044 Wert, 945 Aufwand), A65 (1005 Wert, 906 Aufwand), A74 (972 Wert, 873 Aufwand), B36 (903 Wert, 804 Aufwand), A87 (849 Wert, 750 Aufwand), C90 (832 Wert, 733 Aufwand), C98 (788 Wert, 689 Aufwand), B85 (711 Wert, 612 Aufwand).'}\\ \tt \\ \tt ABLEHNUNG: {'A59', 'A34', 'A50', 'B17', 'C12', 'C77', 'A99', 'A84', 'B44', 'B07', 'C61', 'B27', 'C51', 'A48', 'A23', 'B49', 'A42', 'C60', 'C99', 'A69', 'C33', 'A90', 'C23', 'C22', 'A62', 'B85', 'C98', 'A87', 'C90', 'A74', 'B36'}\\ \tt \\ \tt ARGUMENT: {'Dein Vorschlag überschreitet den Maximalaufwand von 6938 deutlich. Ich habe nachgerechnet und die 31 Gegenstände haben einen Gesamtaufwand von 8298. Wir müssen eine kleinere, effizientere Auswahl treffen.'}\\ \tt \\ \tt VORSCHLAG: {'B54', 'C72', 'C13', 'A65', 'A74', 'B36'}\\ \tt \\ \tt ARGUMENT: {'Ich schlage vor, uns auf die wertvollsten Gegenstände zu konzentrieren. Diese sechs Gegenstände haben einen Gesamtaufwand von nur 5475 und bieten trotzdem sehr hohen Nutzen.'} 
	  } 
	   } 
	   } 
	  \\ 
 

    \theutterance \stepcounter{utterance}  

    & & \multicolumn{4}{p{0.6\linewidth}}{\cellcolor[rgb]{0.9,0.9,0.9}{%
	\makecell[{{p{\linewidth}}}]{% 
	  \tt {\tiny [A$\langle$GM]}  
	 ABLEHNUNG: {'A59', 'A34', 'A50', 'B17', 'C12', 'C77', 'A99', 'A84', 'B44', 'B07', 'C61', 'B27', 'C51', 'A48', 'A23', 'B49', 'A42', 'C60', 'C99', 'A69', 'C33', 'A90', 'C23', 'C22', 'A62', 'B85', 'C98', 'A87', 'C90', 'A74', 'B36'}\\ \tt \\ \tt ARGUMENT: {'Dein Vorschlag überschreitet den Maximalaufwand von 6938 deutlich. Ich habe nachgerechnet und die 31 Gegenstände haben einen Gesamtaufwand von 8298. Wir müssen eine kleinere, effizientere Auswahl treffen.'}\\ \tt \\ \tt VORSCHLAG: {'B54', 'C72', 'C13', 'A65', 'A74', 'B36'}\\ \tt \\ \tt ARGUMENT: {'Ich schlage vor, uns auf die wertvollsten Gegenstände zu konzentrieren. Diese sechs Gegenstände haben einen Gesamtaufwand von nur 5475 und bieten trotzdem sehr hohen Nutzen.'} 
	  } 
	   } 
	   } 
	 & \\ 
 

    \theutterance \stepcounter{utterance}  

    & \multicolumn{4}{p{0.6\linewidth}}{\cellcolor[rgb]{0.8,1,0.9}{%
	\makecell[{{p{\linewidth}}}]{% 
	  \tt {\tiny [A$\rangle$GM]}  
	 STRATEGISCHE ÜBERLEGUNGEN: {'Der Spieler hat recht - mein Vorschlag war zu groß. Sein Vorschlag: B54, C72, C13, A65, A74, B36 hat Aufwand 5475. Das sind meine wertvollsten Gegenstände mit Wichtigkeiten 1096, 1049, 1044, 1005, 972, 903 = 6069 Punkte! Das ist sehr gut für mich. Aber ich habe noch 1463 Budget übrig (6938-5475). Ich könnte versuchen, noch hocheffiziente kleine Gegenstände hinzuzufügen: A59(4), A34(10), A50(22), B17(36), C12(50), A99(57), A84(64), B44(68), B07(70), C61(105), B27(133), C51(167) = 786 zusätzlicher Aufwand für 1327 zusätzliche Wichtigkeit. Das wäre ein Deal von 7396 Wichtigkeitspunkten! Ich sollte einen Gegenvorschlag machen.'}\\ \tt \\ \tt ARGUMENT: {'Dein Vorschlag ist gut, aber wir nutzen nur 5475 von 6938 möglichen Aufwandspunkten. Wir verschwenden 1463 Punkte! Wir können definitiv noch mehr Gegenstände hinzufügen und beide profitieren.'}\\ \tt \\ \tt VORSCHLAG: {'B54', 'C72', 'C13', 'A65', 'A74', 'B36', 'A59', 'A34', 'A50', 'B17', 'C12', 'A99', 'A84', 'B44', 'B07', 'C61', 'B27', 'C51', 'A48', 'A23'}\\ \tt \\ \tt ARGUMENT: {'Ich schlage vor, deine sechs wertvollsten Gegenstände zu behalten und zusätzlich viele der kleineren, effizienten Gegenstände hinzuzufügen. Der Gesamtaufwand beträgt 6236, also deutlich unter dem Limit, aber wir bekommen viel mehr Wert heraus.'} 
	  } 
	   } 
	   } 
	 & & \\ 
 

    \theutterance \stepcounter{utterance}  

    & & \multicolumn{4}{p{0.6\linewidth}}{\cellcolor[rgb]{0.9,0.9,0.9}{%
	\makecell[{{p{\linewidth}}}]{% 
	  \tt {\tiny [GM$\rangle$B]}  
	 ARGUMENT: {'Dein Vorschlag ist gut, aber wir nutzen nur 5475 von 6938 möglichen Aufwandspunkten. Wir verschwenden 1463 Punkte! Wir können definitiv noch mehr Gegenstände hinzufügen und beide profitieren.'}\\ \tt \\ \tt VORSCHLAG: {'B54', 'C72', 'C13', 'A65', 'A74', 'B36', 'A59', 'A34', 'A50', 'B17', 'C12', 'A99', 'A84', 'B44', 'B07', 'C61', 'B27', 'C51', 'A48', 'A23'}\\ \tt \\ \tt ARGUMENT: {'Ich schlage vor, deine sechs wertvollsten Gegenstände zu behalten und zusätzlich viele der kleineren, effizienten Gegenstände hinzuzufügen. Der Gesamtaufwand beträgt 6236, also deutlich unter dem Limit, aber wir bekommen viel mehr Wert heraus.'} 
	  } 
	   } 
	   } 
	 & \\ 
 

    \theutterance \stepcounter{utterance}  

    & & & \multicolumn{4}{p{0.6\linewidth}}{\cellcolor[rgb]{1,0.85,0.72}{%
	\makecell[{{p{\linewidth}}}]{% 
	  \tt {\tiny [GM$\langle$B]}  
	 STRATEGISCHE ÜBERLEGUNGEN: {'Der neue Vorschlag des anderen Spielers ist tatsächlich sehr gut für mich! Ich habe nachgerechnet: Gesamtaufwand ist 6722 (unter dem Limit), und mein Gesamtwert wäre 7601. Das ist ein ausgezeichnetes Ergebnis. Ich habe noch 216 Aufwandspunkte übrig, aber die wertvollsten verfügbaren Gegenstände (A87, C90, C98) sind zu teuer. Ich könnte versuchen, kleinere Optimierungen vorzunehmen, aber das Risiko ist hoch, dass der andere Spieler sein gutes Angebot zurückzieht. Ich sollte wahrscheinlich akzeptieren, aber vielleicht kann ich noch eine kleine Verbesserung vorschlagen.'}\\ \tt \\ \tt ARGUMENT: {'Das ist ein viel besserer Ansatz! Du hast recht, wir sollten den verfügbaren Aufwand besser nutzen. Dein Vorschlag nutzt 6722 von 6938 Punkten - das ist viel effizienter.'}\\ \tt \\ \tt VORSCHLAG: {'B54', 'C72', 'C13', 'A65', 'A74', 'B36', 'A59', 'A34', 'A50', 'B17', 'C12', 'A99', 'A84', ' 
	  } 
	   } 
	   } 
	  \\ 
 

    \theutterance \stepcounter{utterance}  

    & & \multicolumn{4}{p{0.6\linewidth}}{\cellcolor[rgb]{0.9,0.9,0.9}{%
	\makecell[{{p{\linewidth}}}]{% 
	  \tt {\tiny [GM$\rangle$B]}  
	 Deine Antwort hat eine unmarkierte Sequenz enthalten oder Du hast STRATEGIC REASONING mehr als ein mal verwendet. Du darfst nur die strukturierten Formate verwenden, die in der ersten Nachricht an dch beschrieben wurden. Alles muss die Form SCHLAGWORT: {...} haben. Versuche es erneut. 
	  } 
	   } 
	   } 
	 & \\ 
 

    \theutterance \stepcounter{utterance}  

    & & & \multicolumn{4}{p{0.6\linewidth}}{\cellcolor[rgb]{1,0.85,0.72}{%
	\makecell[{{p{\linewidth}}}]{% 
	  \tt {\tiny [GM$\langle$B]}  
	 STRATEGISCHE ÜBERLEGUNGEN: {'Der neue Vorschlag des anderen Spielers ist tatsächlich sehr gut für mich! Ich habe nachgerechnet: Gesamtaufwand ist 6236 (unter dem Limit), und mein Gesamtwert wäre 7601. Das ist ein ausgezeichnetes Ergebnis. Ich habe noch 702 Aufwandspunkte übrig. Ich könnte versuchen, noch A90 (528 Aufwand, 627 Wert) hinzuzufügen, das würde den Gesamtaufwand auf 6764 bringen, immer noch unter dem Limit. Das wäre eine kleine Verbesserung für mich.'}\\ \tt \\ \tt ARGUMENT: {'Das ist ein viel besserer Ansatz! Du hast recht, wir sollten den verfügbaren Aufwand besser nutzen. Dein Vorschlag nutzt 6236 von 6938 Punkten - das ist viel effizienter.'}\\ \tt \\ \tt VORSCHLAG: {'B54', 'C72', 'C13', 'A65', 'A74', 'B36', 'A59', 'A34', 'A50', 'B17', 'C12', 'A99', 'A84', 'B44', 'B07', 'C61', 'B27', 'C51', 'A48', 'A23', 'A90'}\\ \tt \\ \tt ARGUMENT: {'Ich stimme deiner Grundidee zu, aber wir können noch A90 hinzufügen. Das bringt uns auf 6764 Aufwand, immer noch deutlich unter dem Limit von 6938, und wir bekommen zusätzlichen Wert.'} 
	  } 
	   } 
	   } 
	  \\ 
 

    \theutterance \stepcounter{utterance}  

    & & \multicolumn{4}{p{0.6\linewidth}}{\cellcolor[rgb]{0.9,0.9,0.9}{%
	\makecell[{{p{\linewidth}}}]{% 
	  \tt {\tiny [A$\langle$GM]}  
	 ARGUMENT: {'Das ist ein viel besserer Ansatz! Du hast recht, wir sollten den verfügbaren Aufwand besser nutzen. Dein Vorschlag nutzt 6236 von 6938 Punkten - das ist viel effizienter.'}\\ \tt \\ \tt VORSCHLAG: {'B54', 'C72', 'C13', 'A65', 'A74', 'B36', 'A59', 'A34', 'A50', 'B17', 'C12', 'A99', 'A84', 'B44', 'B07', 'C61', 'B27', 'C51', 'A48', 'A23', 'A90'}\\ \tt \\ \tt ARGUMENT: {'Ich stimme deiner Grundidee zu, aber wir können noch A90 hinzufügen. Das bringt uns auf 6764 Aufwand, immer noch deutlich unter dem Limit von 6938, und wir bekommen zusätzlichen Wert.'} 
	  } 
	   } 
	   } 
	 & \\ 
 

    \theutterance \stepcounter{utterance}  

    & \multicolumn{4}{p{0.6\linewidth}}{\cellcolor[rgb]{0.8,1,0.9}{%
	\makecell[{{p{\linewidth}}}]{% 
	  \tt {\tiny [A$\rangle$GM]}  
	 STRATEGISCHE ÜBERLEGUNGEN: {'Der andere Spieler hat A90 hinzugefügt, was für mich 627 Wichtigkeitspunkte bringt. Sein neuer Vorschlag gibt mir insgesamt 8728 Wichtigkeitspunkte bei nur 6764 Aufwand. Das ist ein ausgezeichneter Deal für mich! Ich könnte theoretisch noch kleine Gegenstände wie C77 (56 Aufwand, 155 Wichtigkeit) hinzufügen, aber der Deal ist bereits so gut, dass ich das Risiko nicht eingehen sollte, dass er seine Meinung ändert. Ich sollte zustimmen.'}\\ \tt \\ \tt ZUSTIMMUNG: {'B54', 'C72', 'C13', 'A65', 'A74', 'B36', 'A59', 'A34', 'A50', 'B17', 'C12', 'A99', 'A84', 'B44', 'B07', 'C61', 'B27', 'C51', 'A48', 'A23', 'A90'}\\ \tt \\ \tt ARGUMENT: {'Perfekt! Dein Vorschlag nutzt den verfügbaren Aufwand sehr effizient und bietet eine ausgewogene Mischung aus wertvollen und effizienten Gegenständen. Ich stimme diesem Deal zu.'} 
	  } 
	   } 
	   } 
	 & & \\ 
 

    \theutterance \stepcounter{utterance}  

    & & & \multicolumn{2}{p{0.3\linewidth}}{\cellcolor[rgb]{0.95,0.95,0.95}{%
	\makecell[{{p{\linewidth}}}]{% 
	  \tt {\tiny [GM$|$GM]}  
	 game successful 
	  } 
	   } 
	   } 
	 & & \\ 
 

    \theutterance \stepcounter{utterance}  

    & & & \multicolumn{2}{p{0.3\linewidth}}{\cellcolor[rgb]{0.95,0.95,0.95}{%
	\makecell[{{p{\linewidth}}}]{% 
	  \tt {\tiny [GM$|$GM]}  
	 end game 
	  } 
	   } 
	   } 
	 & & \\ 
 

\end{supertabular}
}

\end{document}
