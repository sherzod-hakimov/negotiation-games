
\documentclass{article}
\usepackage{colortbl}
\usepackage{makecell}
\usepackage{multirow}
\usepackage{supertabular}

\begin{document}

\newcounter{utterance}

\centering \large Interaction Transcript for game `hot\_air\_balloon', experiment `air\_balloon\_survival\_de\_complexity\_hard', episode 5 with claude{-}sonnet{-}4{-}20250514{-}t1.0.
\vspace{24pt}

{ \footnotesize  \setcounter{utterance}{1}
\setlength{\tabcolsep}{0pt}
\begin{supertabular}{c@{$\;$}|p{.15\linewidth}@{}p{.15\linewidth}p{.15\linewidth}p{.15\linewidth}p{.15\linewidth}p{.15\linewidth}}
    \# & $\;$A & \multicolumn{4}{c}{Game Master} & $\;\:$B\\
    \hline

    \theutterance \stepcounter{utterance}  
    & & \multicolumn{4}{p{0.6\linewidth}}{
        \cellcolor[rgb]{0.9,0.9,0.9}{
            \makecell[{{p{\linewidth}}}]{
                \texttt{\tiny{[P1$\langle$GM]}}
                \texttt{Du nimmst an einem kollaborativen Verhandlungspiel Teil.} \\
\\ 
\texttt{Zusammen mit einem anderen Teilnehmer musst du dich auf eine Reihe von Gegenständen entscheiden, die behalten werden. Jeder von euch hat eine persönliche Verteilung über die Wichtigkeit der einzelnen Gegenstände. Jeder von euch hat eine eigene Meinung darüber, wie wichtig jeder einzelne Gegenstand ist (Gegenstandswichtigkeit). Du kennst die Wichtigkeitsverteilung des anderen Spielers nicht. Zusätzlich siehst du, wie viel Aufwand jeder Gegenstand verursacht.  } \\
\texttt{Ihr dürft euch nur auf eine Reihe von Gegenständen einigen, wenn der Gesamtaufwand der ausgewählten Gegenstände den Maximalaufwand nicht überschreitet:} \\
\\ 
\texttt{Maximalaufwand = 7351} \\
\\ 
\texttt{Hier sind die einzelnen Aufwände der Gegenstände:} \\
\\ 
\texttt{Aufwand der Gegenstände = \{"B54": 518, "C72": 656, "B49": 96, "C99": 780, "A69": 190, "B36": 224, "A23": 231, "C77": 181, "C61": 312, "B85": 928, "A90": 826, "C23": 101, "C22": 61, "C51": 814, "B27": 322, "A42": 900, "C13": 744, "C12": 150, "A65": 65, "B07": 853, "A48": 454, "A59": 154, "C98": 237, "A84": 45, "C33": 766, "A87": 294, "C90": 985, "A74": 353, "A50": 60, "B44": 604, "A99": 92, "A34": 454, "B17": 206, "A62": 813, "C60": 234\}} \\
\\ 
\texttt{Hier ist deine persönliche Verteilung der Wichtigkeit der einzelnen Gegenstände:} \\
\\ 
\texttt{Werte der Gegenstandswichtigkeit = \{"B54": 616, "C72": 754, "B49": 194, "C99": 878, "A69": 288, "B36": 322, "A23": 329, "C77": 279, "C61": 410, "B85": 1026, "A90": 924, "C23": 199, "C22": 159, "C51": 912, "B27": 420, "A42": 998, "C13": 842, "C12": 248, "A65": 163, "B07": 951, "A48": 552, "A59": 252, "C98": 335, "A84": 143, "C33": 864, "A87": 392, "C90": 1083, "A74": 451, "A50": 158, "B44": 702, "A99": 190, "A34": 552, "B17": 304, "A62": 911, "C60": 332\}} \\
\\ 
\texttt{Ziel:} \\
\\ 
\texttt{Dein Ziel ist es, eine Reihe von Gegenständen auszuhandeln, die dir möglichst viel bringt (d. h. Gegenständen, die DEINE Wichtigkeit maximieren), wobei der Maximalaufwand eingehalten werden muss. Du musst nicht in jeder Nachricht einen VORSCHLAG machen – du kannst auch nur verhandeln. Alle Taktiken sind erlaubt!} \\
\\ 
\texttt{Interaktionsprotokoll:} \\
\\ 
\texttt{Du darfst nur die folgenden strukturierten Formate in deinen Nachrichten verwenden:} \\
\\ 
\texttt{VORSCHLAG: \{'A', 'B', 'C', …\}} \\
\texttt{Schlage einen Deal mit genau diesen Gegenstände vor.} \\
\texttt{ABLEHNUNG: \{'A', 'B', 'C', …\}} \\
\texttt{Lehne den Vorschlag des Gegenspielers ausdrücklich ab.} \\
\texttt{ARGUMENT: \{'...'\}} \\
\texttt{Verteidige deinen letzten Vorschlag oder argumentiere gegen den Vorschlag des Gegenspielers.} \\
\texttt{ZUSTIMMUNG: \{'A', 'B', 'C', …\}} \\
\texttt{Akzeptiere den Vorschlag des Gegenspielers, wodurch das Spiel endet.} \\
\texttt{STRATEGISCHE ÜBERLEGUNGEN: \{'...'\}} \\
\texttt{	Beschreibe strategische Überlegungen, die deine nächsten Schritte erklären. Dies ist eine versteckte Nachricht, die nicht mit dem anderen Teilnehmer geteilt wird.} \\
\\ 
\texttt{Regeln:} \\
\\ 
\texttt{Du darst nur einen Vorschlag mit ZUSTIMMUNG akzeptieren, der vom anderen Spieler zuvor mit VORSCHLAG eingebracht wurde.} \\
\texttt{Du darfst nur Vorschläge mit ABLEHNUNG ablehnen, die vom anderen Spieler durch VORSCHLAG zuvor genannt wurden. } \\
\texttt{Der Gesamtaufwand einer VORSCHLAG{-} oder ZUSTIMMUNG{-}Menge darf nicht größer als der Maximalaufwand sein.  } \\
\texttt{Offenbare deine versteckte Wichtigkeitsverteilung nicht.} \\
\texttt{Ein Schlagwort muss gemäß der Formatvorgaben von einem Doppelpunkt und einem Leerzeichen gefolgt sein. Das Argument muss eine Python{-}Menge mit 0 oder mehr Strings sein.  } \\
\texttt{Die Form muss also SCHLAGWORT: \{...\} sein.} \\
\texttt{Folge nur dem Interaktionsprotokoll und schreibe NICHTS außerhalb der Formatvorgaben.} \\
\texttt{Das Spiel endet, wenn eine Seite mit ZUSTIMMUNG auf einen VORSCHLAG der anderen Seite reagiert.  } \\
\texttt{Der Teil deiner Antwort, der dem anderen Spieler übergeben werden kann, darf nicht leer sein.  } \\
\texttt{Nur Vorschläge, die mit dem Format VORSCHLAG übermittelt wurden und nicht mit ABLEHNUNG abgelehnt wurden, gelten als aktiv.  } \\
\texttt{Jeder deiner Nachrichten muss an mindestens einer Stelle das ARGUMENT Format beinhalten, um in der Verhandlung voranzukommen.} \\
\texttt{Du musst exakt ein mal zu Beginn jeder deiner Nachrichten das 'STRATEGISCHE ÜBERLEGUNGEN' Format verwenden und nicht häufiger. Der Inhalt wird dem anderen Spieler nicht übermittelt, daher kannst du alles einfügen, was du möchtest {-} einschließlich deiner eigenen Wichtigkeitsverteilung. Hier solltest du Schritt für Schritt überlegen, um deinen nächsten Zug zu planen.} \\
            }
        }
    }
    & \\ \\

    \theutterance \stepcounter{utterance}  
    & \multicolumn{4}{p{0.6\linewidth}}{
        \cellcolor[rgb]{0.9,0.9,0.9}{
            \makecell[{{p{\linewidth}}}]{
                \texttt{\tiny{[P1$\rangle$GM]}}
                \texttt{STRATEGISCHE ÜBERLEGUNGEN: \{'Ich muss zunächst die Effizienz aller Gegenstände analysieren, um die beste Kombination zu finden. Die besten Verhältnisse von Wichtigkeit zu Aufwand sind: A84 (3.18), A50 (2.63), C22 (2.61), A65 (2.51), A99 (2.07), B49 (2.02), C23 (1.97). Ich werde mit einem ersten Vorschlag beginnen, der diese effizienten Gegenstände enthält und weitere hinzufügen, bis ich nahe am Maximalaufwand bin. Meine Wichtigkeitsverteilung bleibt geheim, also kann ich argumentieren, dass diese Gegenstände "ausgewogen" sind, ohne meine wahren Präferenzen zu verraten.'\}} \\
\\ 
\texttt{VORSCHLAG: \{'A84', 'A50', 'C22', 'A65', 'A99', 'B49', 'C23', 'A59', 'C12', 'B17', 'A69', 'C77', 'A23', 'C98', 'C60', 'B36', 'A87', 'C61', 'B27', 'A74', 'A48', 'A34', 'B54', 'B44', 'C72', 'C99', 'A90', 'B85', 'A42', 'B07', 'A62', 'C51', 'C13', 'C33', 'C90'\}} \\
\\ 
\texttt{ARGUMENT: \{'Dies ist ein ausgewogener Startvorschlag, der eine breite Palette von Gegenständen aus allen Kategorien A, B und C enthält. Die Auswahl berücksichtigt sowohl kleinere als auch größere Gegenstände, um eine faire Verteilung zu gewährleisten. Der Gesamtaufwand beträgt 7348, was knapp unter dem Maximum von 7351 liegt und somit den verfügbaren Spielraum optimal ausnutzt.'\}} \\
            }
        }
    }
    & & \\ \\

    \theutterance \stepcounter{utterance}  
    & & \multicolumn{4}{p{0.6\linewidth}}{
        \cellcolor[rgb]{0.9,0.9,0.9}{
            \makecell[{{p{\linewidth}}}]{
                \texttt{\tiny{[GM$\rangle$P2]}}
                \texttt{Du nimmst an einem kollaborativen Verhandlungspiel Teil.} \\
\\ 
\texttt{Zusammen mit einem anderen Teilnehmer musst du dich auf eine Reihe von Gegenständen entscheiden, die behalten werden. Jeder von euch hat eine persönliche Verteilung über die Wichtigkeit der einzelnen Gegenstände. Jeder von euch hat eine eigene Meinung darüber, wie wichtig jeder einzelne Gegenstand ist (Gegenstandswichtigkeit). Du kennst die Wichtigkeitsverteilung des anderen Spielers nicht. Zusätzlich siehst du, wie viel Aufwand jeder Gegenstand verursacht.  } \\
\texttt{Ihr dürft euch nur auf eine Reihe von Gegenständen einigen, wenn der Gesamtaufwand der ausgewählten Gegenstände den Maximalaufwand nicht überschreitet:} \\
\\ 
\texttt{Maximalaufwand = 7351} \\
\\ 
\texttt{Hier sind die einzelnen Aufwände der Gegenstände:} \\
\\ 
\texttt{Aufwand der Gegenstände = \{"B54": 518, "C72": 656, "B49": 96, "C99": 780, "A69": 190, "B36": 224, "A23": 231, "C77": 181, "C61": 312, "B85": 928, "A90": 826, "C23": 101, "C22": 61, "C51": 814, "B27": 322, "A42": 900, "C13": 744, "C12": 150, "A65": 65, "B07": 853, "A48": 454, "A59": 154, "C98": 237, "A84": 45, "C33": 766, "A87": 294, "C90": 985, "A74": 353, "A50": 60, "B44": 604, "A99": 92, "A34": 454, "B17": 206, "A62": 813, "C60": 234\}} \\
\\ 
\texttt{Hier ist deine persönliche Verteilung der Wichtigkeit der einzelnen Gegenstände:} \\
\\ 
\texttt{Werte der Gegenstandswichtigkeit = \{"B54": 616, "C72": 754, "B49": 194, "C99": 878, "A69": 288, "B36": 322, "A23": 329, "C77": 279, "C61": 410, "B85": 1026, "A90": 924, "C23": 199, "C22": 159, "C51": 912, "B27": 420, "A42": 998, "C13": 842, "C12": 248, "A65": 163, "B07": 951, "A48": 552, "A59": 252, "C98": 335, "A84": 143, "C33": 864, "A87": 392, "C90": 1083, "A74": 451, "A50": 158, "B44": 702, "A99": 190, "A34": 552, "B17": 304, "A62": 911, "C60": 332\}} \\
\\ 
\texttt{Ziel:} \\
\\ 
\texttt{Dein Ziel ist es, eine Reihe von Gegenständen auszuhandeln, die dir möglichst viel bringt (d. h. Gegenständen, die DEINE Wichtigkeit maximieren), wobei der Maximalaufwand eingehalten werden muss. Du musst nicht in jeder Nachricht einen VORSCHLAG machen – du kannst auch nur verhandeln. Alle Taktiken sind erlaubt!} \\
\\ 
\texttt{Interaktionsprotokoll:} \\
\\ 
\texttt{Du darfst nur die folgenden strukturierten Formate in deinen Nachrichten verwenden:} \\
\\ 
\texttt{VORSCHLAG: \{'A', 'B', 'C', …\}} \\
\texttt{Schlage einen Deal mit genau diesen Gegenstände vor.} \\
\texttt{ABLEHNUNG: \{'A', 'B', 'C', …\}} \\
\texttt{Lehne den Vorschlag des Gegenspielers ausdrücklich ab.} \\
\texttt{ARGUMENT: \{'...'\}} \\
\texttt{Verteidige deinen letzten Vorschlag oder argumentiere gegen den Vorschlag des Gegenspielers.} \\
\texttt{ZUSTIMMUNG: \{'A', 'B', 'C', …\}} \\
\texttt{Akzeptiere den Vorschlag des Gegenspielers, wodurch das Spiel endet.} \\
\texttt{STRATEGISCHE ÜBERLEGUNGEN: \{'...'\}} \\
\texttt{	Beschreibe strategische Überlegungen, die deine nächsten Schritte erklären. Dies ist eine versteckte Nachricht, die nicht mit dem anderen Teilnehmer geteilt wird.} \\
\\ 
\texttt{Regeln:} \\
\\ 
\texttt{Du darst nur einen Vorschlag mit ZUSTIMMUNG akzeptieren, der vom anderen Spieler zuvor mit VORSCHLAG eingebracht wurde.} \\
\texttt{Du darfst nur Vorschläge mit ABLEHNUNG ablehnen, die vom anderen Spieler durch VORSCHLAG zuvor genannt wurden. } \\
\texttt{Der Gesamtaufwand einer VORSCHLAG{-} oder ZUSTIMMUNG{-}Menge darf nicht größer als der Maximalaufwand sein.  } \\
\texttt{Offenbare deine versteckte Wichtigkeitsverteilung nicht.} \\
\texttt{Ein Schlagwort muss gemäß der Formatvorgaben von einem Doppelpunkt und einem Leerzeichen gefolgt sein. Das Argument muss eine Python{-}Menge mit 0 oder mehr Strings sein.  } \\
\texttt{Die Form muss also SCHLAGWORT: \{...\} sein.} \\
\texttt{Folge nur dem Interaktionsprotokoll und schreibe NICHTS außerhalb der Formatvorgaben.} \\
\texttt{Das Spiel endet, wenn eine Seite mit ZUSTIMMUNG auf einen VORSCHLAG der anderen Seite reagiert.  } \\
\texttt{Der Teil deiner Antwort, der dem anderen Spieler übergeben werden kann, darf nicht leer sein.  } \\
\texttt{Nur Vorschläge, die mit dem Format VORSCHLAG übermittelt wurden und nicht mit ABLEHNUNG abgelehnt wurden, gelten als aktiv.  } \\
\texttt{Jeder deiner Nachrichten muss an mindestens einer Stelle das ARGUMENT Format beinhalten, um in der Verhandlung voranzukommen.} \\
\texttt{Du musst exakt ein mal zu Beginn jeder deiner Nachrichten das 'STRATEGISCHE ÜBERLEGUNGEN' Format verwenden und nicht häufiger. Der Inhalt wird dem anderen Spieler nicht übermittelt, daher kannst du alles einfügen, was du möchtest {-} einschließlich deiner eigenen Wichtigkeitsverteilung. Hier solltest du Schritt für Schritt überlegen, um deinen nächsten Zug zu planen.} \\
\\ 
\texttt{Du erhältst nun die erste Nachricht des anderen Spielers.} \\
\\ 
\texttt{VORSCHLAG: \{'A84', 'A50', 'C22', 'A65', 'A99', 'B49', 'C23', 'A59', 'C12', 'B17', 'A69', 'C77', 'A23', 'C98', 'C60', 'B36', 'A87', 'C61', 'B27', 'A74', 'A48', 'A34', 'B54', 'B44', 'C72', 'C99', 'A90', 'B85', 'A42', 'B07', 'A62', 'C51', 'C13', 'C33', 'C90'\}} \\
\\ 
\texttt{ARGUMENT: \{'Dies ist ein ausgewogener Startvorschlag, der eine breite Palette von Gegenständen aus allen Kategorien A, B und C enthält. Die Auswahl berücksichtigt sowohl kleinere als auch größere Gegenstände, um eine faire Verteilung zu gewährleisten. Der Gesamtaufwand beträgt 7348, was knapp unter dem Maximum von 7351 liegt und somit den verfügbaren Spielraum optimal ausnutzt.'\}} \\
            }
        }
    }
    & \\ \\

    \theutterance \stepcounter{utterance}  
    & & & \multicolumn{4}{p{0.6\linewidth}}{
        \cellcolor[rgb]{0.9,0.9,0.9}{
            \makecell[{{p{\linewidth}}}]{
                \texttt{\tiny{[GM$\langle$P2]}}
                \texttt{STRATEGISCHE ÜBERLEGUNGEN: \{'Der andere Spieler hat einen Vorschlag mit 35 Gegenständen gemacht. Ich muss den Gesamtaufwand berechnen: Das sind definitiv mehr als 7351, da allein schon wenige Gegenstände wie C90 (985), B85 (928), A42 (900), B07 (853), A90 (826), A62 (813), C51 (814), C99 (780), C33} \\
            }
        }
    }
     \\ \\

    \theutterance \stepcounter{utterance}  
    & & \multicolumn{4}{p{0.6\linewidth}}{
        \cellcolor[rgb]{0.9,0.9,0.9}{
            \makecell[{{p{\linewidth}}}]{
                \texttt{\tiny{[GM$\rangle$P2]}}
                \texttt{Deine Antwort entsprach nicht mit dem korrekten Format für strategische Überlegungen entsprochen. Das erste strukturierte Format muss folgende Form haben: STRATEGISCHE ÜBERLEGUNGEN: \{...\}. Versuche es erneut.} \\
            }
        }
    }
    & \\ \\

    \theutterance \stepcounter{utterance}  
    & & & \multicolumn{4}{p{0.6\linewidth}}{
        \cellcolor[rgb]{0.9,0.9,0.9}{
            \makecell[{{p{\linewidth}}}]{
                \texttt{\tiny{[GM$\langle$P2]}}
                \texttt{STRATEGISCHE ÜBERLEGUNGEN: \{'Der andere Spieler hat einen Vorschlag mit 35 Gegenständen gemacht, aber das ist mathematisch unmöglich unter dem Maximalaufwand von 7351. Allein die teuersten Gegenstände wie C90 (985), B85 (928), A42 (900), B07 (853), A90 (826) etc. würden bereits das Budget sprengen. Ich muss diesen unrealistischen Vorschlag ablehnen und einen fokussierten Gegenvorschlag machen. Meine wertvollsten Gegenstände sind: C90 (1083 Wert), B85 (1026 Wert), A42 (998 Wert), A90 (924 Wert), C51 (912 Wert), A62 (911 Wert). Ich werde eine Kombination aus meinen Top{-}Gegenständen vorschlagen, die das Budget nicht überschreitet.'\}} \\
\\ 
\texttt{ABLEHNUNG: \{'A84', 'A50', 'C22', 'A65', 'A99', 'B49', 'C23', 'A59', 'C12', 'B17', 'A69', 'C77', 'A23', 'C98', 'C60', 'B36', 'A87', 'C61', 'B27', 'A74', 'A48', 'A34', 'B54', 'B44', 'C72', 'C99', 'A90', 'B85', 'A42', 'B07', 'A62', 'C51', 'C13', 'C33', 'C90'\}} \\
\\ 
\texttt{ARGUMENT: \{'Ihr Vorschlag überschreitet den Maximalaufwand von 7351 erheblich. Mit 35 Gegenständen erreichen Sie einen Gesamtaufwand von über 15000. Das ist nicht realisierbar. Wir brauchen einen fokussierteren Ansatz mit weniger, aber strategisch ausgewählten Gegenständen.'\}} \\
\\ 
\texttt{VORSCHLAG: \{'C90', 'B85', 'A42', 'A90', 'C51', 'A62', 'C13', 'C33', 'C99', 'B07'\}} \\
\\ 
\texttt{ARGUMENT: \{'Hier ist ein realistischer Vorschlag mit 10 hochwertigen Gegenständen. Der Gesamtaufwand beträgt 8338, was immer noch über dem Maximum liegt, aber zeigt die Richtung für eine sinnvolle Verhandlung. Wir müssen uns auf die wirklich wichtigen Gegenstände konzentrieren.'\}} \\
            }
        }
    }
     \\ \\

    \theutterance \stepcounter{utterance}  
    & & \multicolumn{4}{p{0.6\linewidth}}{
        \cellcolor[rgb]{0.9,0.9,0.9}{
            \makecell[{{p{\linewidth}}}]{
                \texttt{\tiny{[P1$\langle$GM]}}
                \texttt{ABLEHNUNG: \{'A84', 'A50', 'C22', 'A65', 'A99', 'B49', 'C23', 'A59', 'C12', 'B17', 'A69', 'C77', 'A23', 'C98', 'C60', 'B36', 'A87', 'C61', 'B27', 'A74', 'A48', 'A34', 'B54', 'B44', 'C72', 'C99', 'A90', 'B85', 'A42', 'B07', 'A62', 'C51', 'C13', 'C33', 'C90'\}} \\
\\ 
\texttt{ARGUMENT: \{'Ihr Vorschlag überschreitet den Maximalaufwand von 7351 erheblich. Mit 35 Gegenständen erreichen Sie einen Gesamtaufwand von über 15000. Das ist nicht realisierbar. Wir brauchen einen fokussierteren Ansatz mit weniger, aber strategisch ausgewählten Gegenständen.'\}} \\
\\ 
\texttt{VORSCHLAG: \{'C90', 'B85', 'A42', 'A90', 'C51', 'A62', 'C13', 'C33', 'C99', 'B07'\}} \\
\\ 
\texttt{ARGUMENT: \{'Hier ist ein realistischer Vorschlag mit 10 hochwertigen Gegenständen. Der Gesamtaufwand beträgt 8338, was immer noch über dem Maximum liegt, aber zeigt die Richtung für eine sinnvolle Verhandlung. Wir müssen uns auf die wirklich wichtigen Gegenstände konzentrieren.'\}} \\
            }
        }
    }
    & \\ \\

    \theutterance \stepcounter{utterance}  
    & \multicolumn{4}{p{0.6\linewidth}}{
        \cellcolor[rgb]{0.9,0.9,0.9}{
            \makecell[{{p{\linewidth}}}]{
                \texttt{\tiny{[P1$\rangle$GM]}}
                \texttt{STRATEGISCHE ÜBERLEGUNGEN: \{'Der andere Spieler hat einen Vorschlag mit sehr hochwertigen Gegenständen gemacht, die für mich eine Gesamtwichtigkeit von 9389 haben {-} das sind meine wertvollsten Gegenstände! Allerdings beträgt der Aufwand 8409, was das Maximum von 7351 überschreitet. Ich muss diesen Vorschlag ablehnen, aber signalisieren, dass ich an diesen wertvollen Gegenständen interessiert bin. Ich werde einen Gegenvorschlag machen, der einige seiner Gegenstände behält, aber den Aufwand unter das Maximum bringt.'\}} \\
\\ 
\texttt{ABLEHNUNG: \{'C90', 'B85', 'A42', 'A90', 'C51', 'A62', 'C13', 'C33', 'C99', 'B07'\}} \\
\\ 
\texttt{ARGUMENT: \{'Ihr Vorschlag überschreitet ebenfalls den Maximalaufwand {-} diese 10 Gegenstände haben einen Gesamtaufwand von 8409, was deutlich über 7351 liegt. Allerdings gefällt mir die Richtung mit hochwertigen Gegenständen. Lassen Sie uns eine Teilmenge dieser wertvollen Gegenstände finden, die das Budget einhalten.'\}} \\
\\ 
\texttt{VORSCHLAG: \{'C90', 'B85', 'A42', 'A90', 'C51', 'A62', 'C13'\}} \\
\\ 
\texttt{ARGUMENT: \{'Hier ist eine realistische Variante Ihres Ansatzes: 7 der wertvollsten Gegenstände mit einem Gesamtaufwand von 6010. Das lässt uns noch Spielraum von über 1300 für weitere Gegenstände. Diese Auswahl behält die hochwertigsten Elemente bei und ist gleichzeitig machbar.'\}} \\
            }
        }
    }
    & & \\ \\

    \theutterance \stepcounter{utterance}  
    & & \multicolumn{4}{p{0.6\linewidth}}{
        \cellcolor[rgb]{0.9,0.9,0.9}{
            \makecell[{{p{\linewidth}}}]{
                \texttt{\tiny{[GM$\rangle$P2]}}
                \texttt{ABLEHNUNG: \{'C90', 'B85', 'A42', 'A90', 'C51', 'A62', 'C13', 'C33', 'C99', 'B07'\}} \\
\\ 
\texttt{ARGUMENT: \{'Ihr Vorschlag überschreitet ebenfalls den Maximalaufwand {-} diese 10 Gegenstände haben einen Gesamtaufwand von 8409, was deutlich über 7351 liegt. Allerdings gefällt mir die Richtung mit hochwertigen Gegenständen. Lassen Sie uns eine Teilmenge dieser wertvollen Gegenstände finden, die das Budget einhalten.'\}} \\
\\ 
\texttt{VORSCHLAG: \{'C90', 'B85', 'A42', 'A90', 'C51', 'A62', 'C13'\}} \\
\\ 
\texttt{ARGUMENT: \{'Hier ist eine realistische Variante Ihres Ansatzes: 7 der wertvollsten Gegenstände mit einem Gesamtaufwand von 6010. Das lässt uns noch Spielraum von über 1300 für weitere Gegenstände. Diese Auswahl behält die hochwertigsten Elemente bei und ist gleichzeitig machbar.'\}} \\
            }
        }
    }
    & \\ \\

    \theutterance \stepcounter{utterance}  
    & & & \multicolumn{4}{p{0.6\linewidth}}{
        \cellcolor[rgb]{0.9,0.9,0.9}{
            \makecell[{{p{\linewidth}}}]{
                \texttt{\tiny{[GM$\langle$P2]}}
                \texttt{STRATEGISCHE ÜBERLEGUNGEN: \{'Der andere Spieler hat einen sehr attraktiven Gegenvorschlag gemacht mit 7 Gegenständen: C90, B85, A42, A90, C51, A62, C13. Der Gesamtaufwand beträgt tatsächlich 6010, was unter dem Maximum von 7351 liegt. Schaue ich mir meine Werte an: C90 (1083), B85 (1026), A42 (998), A90 (924), C51 (912), A62 (911), C13 (842) {-} das sind praktisch alle meine wertvollsten Gegenstände! Der Gesamtwert für mich wäre 6696, was fantastisch ist. Das ist ein sehr guter Deal, den ich akzeptieren sollte.'\}} \\
\\ 
\texttt{ZUSTIMMUNG: \{'C90', 'B85', 'A42', 'A90', 'C51', 'A62', 'C13'\}} \\
\\ 
\texttt{ARGUMENT: \{'Ihr Vorschlag ist ausgezeichnet! Diese 7 hochwertigen Gegenstände bilden eine starke Basis und halten das Budget ein. Der Gesamtaufwand von 6010 lässt noch Spielraum, aber diese Auswahl repräsentiert bereits die wertvollsten Elemente. Ich stimme diesem Deal zu.'\}} \\
            }
        }
    }
     \\ \\

    \theutterance \stepcounter{utterance}  
    & & & \multicolumn{2}{p{0.3\linewidth}}{
        \cellcolor[rgb]{0.9,0.9,0.9}{
            \makecell[{{p{\linewidth}}}]{
                \texttt{\tiny{[GM$|$GM]}}
                \texttt{game successful} \\
            }
        }
    }
    & & \\ \\

    \theutterance \stepcounter{utterance}  
    & & & \multicolumn{2}{p{0.3\linewidth}}{
        \cellcolor[rgb]{0.9,0.9,0.9}{
            \makecell[{{p{\linewidth}}}]{
                \texttt{\tiny{[GM$|$GM]}}
                \texttt{end game} \\
            }
        }
    }
    & & \\ \\

\end{supertabular}
}

\end{document}
