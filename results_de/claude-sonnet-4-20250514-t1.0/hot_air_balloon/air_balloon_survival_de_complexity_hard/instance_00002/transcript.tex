
\documentclass{article}
\usepackage{colortbl}
\usepackage{makecell}
\usepackage{multirow}
\usepackage{supertabular}

\begin{document}

\newcounter{utterance}

\twocolumn

{ \footnotesize  \setcounter{utterance}{1}
\setlength{\tabcolsep}{0pt}
\begin{supertabular}{c@{$\;$}|p{.15\linewidth}@{}p{.15\linewidth}p{.15\linewidth}p{.15\linewidth}p{.15\linewidth}p{.15\linewidth}}

    \# & $\;$A & \multicolumn{4}{c}{Game Master} & $\;\:$B\\
    \hline 

    \theutterance \stepcounter{utterance}  

    & & \multicolumn{4}{p{0.6\linewidth}}{\cellcolor[rgb]{0.9,0.9,0.9}{%
	\makecell[{{p{\linewidth}}}]{% 
	  \tt {\tiny [A$\langle$GM]}  
	 Du nimmst an einem kollaborativen Verhandlungspiel Teil.\\ \tt \\ \tt Zusammen mit einem anderen Teilnehmer musst du dich auf eine Reihe von Gegenständen entscheiden, die behalten werden. Jeder von euch hat eine persönliche Verteilung über die Wichtigkeit der einzelnen Gegenstände. Jeder von euch hat eine eigene Meinung darüber, wie wichtig jeder einzelne Gegenstand ist (Gegenstandswichtigkeit). Du kennst die Wichtigkeitsverteilung des anderen Spielers nicht. Zusätzlich siehst du, wie viel Aufwand jeder Gegenstand verursacht.  \\ \tt Ihr dürft euch nur auf eine Reihe von Gegenständen einigen, wenn der Gesamtaufwand der ausgewählten Gegenstände den Maximalaufwand nicht überschreitet:\\ \tt \\ \tt Maximalaufwand = 9322\\ \tt \\ \tt Hier sind die einzelnen Aufwände der Gegenstände:\\ \tt \\ \tt Aufwand der Gegenstände = {"B54": 380, "C72": 448, "B49": 568, "C99": 351, "A69": 896, "B36": 821, "A23": 526, "C77": 502, "C61": 326, "B85": 962, "A90": 619, "C23": 115, "C22": 598, "C51": 662, "B27": 300, "A42": 830, "C13": 562, "C12": 679, "A65": 283, "B07": 440, "A48": 12, "A59": 863, "C98": 319, "A84": 771, "C33": 89, "A87": 655, "C90": 504, "A74": 118, "A50": 514, "B44": 226, "A99": 892, "A34": 621, "B17": 766, "A62": 659, "C60": 768}\\ \tt \\ \tt Hier ist deine persönliche Verteilung der Wichtigkeit der einzelnen Gegenstände:\\ \tt \\ \tt Werte der Gegenstandswichtigkeit = {"B54": 476, "C72": 544, "B49": 664, "C99": 447, "A69": 992, "B36": 917, "A23": 622, "C77": 598, "C61": 422, "B85": 1058, "A90": 715, "C23": 211, "C22": 694, "C51": 758, "B27": 396, "A42": 926, "C13": 658, "C12": 775, "A65": 379, "B07": 536, "A48": 108, "A59": 959, "C98": 415, "A84": 867, "C33": 185, "A87": 751, "C90": 600, "A74": 214, "A50": 610, "B44": 322, "A99": 988, "A34": 717, "B17": 862, "A62": 755, "C60": 864}\\ \tt \\ \tt Ziel:\\ \tt \\ \tt Dein Ziel ist es, eine Reihe von Gegenständen auszuhandeln, die dir möglichst viel bringt (d. h. Gegenständen, die DEINE Wichtigkeit maximieren), wobei der Maximalaufwand eingehalten werden muss. Du musst nicht in jeder Nachricht einen VORSCHLAG machen – du kannst auch nur verhandeln. Alle Taktiken sind erlaubt!\\ \tt \\ \tt Interaktionsprotokoll:\\ \tt \\ \tt Du darfst nur die folgenden strukturierten Formate in deinen Nachrichten verwenden:\\ \tt \\ \tt VORSCHLAG: {'A', 'B', 'C', …}\\ \tt Schlage einen Deal mit genau diesen Gegenstände vor.\\ \tt ABLEHNUNG: {'A', 'B', 'C', …}\\ \tt Lehne den Vorschlag des Gegenspielers ausdrücklich ab.\\ \tt ARGUMENT: {'...'}\\ \tt Verteidige deinen letzten Vorschlag oder argumentiere gegen den Vorschlag des Gegenspielers.\\ \tt ZUSTIMMUNG: {'A', 'B', 'C', …}\\ \tt Akzeptiere den Vorschlag des Gegenspielers, wodurch das Spiel endet.\\ \tt STRATEGISCHE ÜBERLEGUNGEN: {'...'}\\ \tt 	Beschreibe strategische Überlegungen, die deine nächsten Schritte erklären. Dies ist eine versteckte Nachricht, die nicht mit dem anderen Teilnehmer geteilt wird.\\ \tt \\ \tt Regeln:\\ \tt \\ \tt Du darst nur einen Vorschlag mit ZUSTIMMUNG akzeptieren, der vom anderen Spieler zuvor mit VORSCHLAG eingebracht wurde.\\ \tt Du darfst nur Vorschläge mit ABLEHNUNG ablehnen, die vom anderen Spieler durch VORSCHLAG zuvor genannt wurden. \\ \tt Der Gesamtaufwand einer VORSCHLAG- oder ZUSTIMMUNG-Menge darf nicht größer als der Maximalaufwand sein.  \\ \tt Offenbare deine versteckte Wichtigkeitsverteilung nicht.\\ \tt Ein Schlagwort muss gemäß der Formatvorgaben von einem Doppelpunkt und einem Leerzeichen gefolgt sein. Das Argument muss eine Python-Menge mit 0 oder mehr Strings sein.  \\ \tt Die Form muss also SCHLAGWORT: {...} sein.\\ \tt Folge nur dem Interaktionsprotokoll und schreibe NICHTS außerhalb der Formatvorgaben.\\ \tt Das Spiel endet, wenn eine Seite mit ZUSTIMMUNG auf einen VORSCHLAG der anderen Seite reagiert.  \\ \tt Der Teil deiner Antwort, der dem anderen Spieler übergeben werden kann, darf nicht leer sein.  \\ \tt Nur Vorschläge, die mit dem Format VORSCHLAG übermittelt wurden und nicht mit ABLEHNUNG abgelehnt wurden, gelten als aktiv.  \\ \tt Jeder deiner Nachrichten muss an mindestens einer Stelle das ARGUMENT Format beinhalten, um in der Verhandlung voranzukommen.\\ \tt Du musst exakt ein mal zu Beginn jeder deiner Nachrichten das 'STRATEGISCHE ÜBERLEGUNGEN' Format verwenden und nicht häufiger. Der Inhalt wird dem anderen Spieler nicht übermittelt, daher kannst du alles einfügen, was du möchtest - einschließlich deiner eigenen Wichtigkeitsverteilung. Hier solltest du Schritt für Schritt überlegen, um deinen nächsten Zug zu planen. 
	  } 
	   } 
	   } 
	 & \\ 
 

    \theutterance \stepcounter{utterance}  

    & \multicolumn{4}{p{0.6\linewidth}}{\cellcolor[rgb]{0.8,1,0.9}{%
	\makecell[{{p{\linewidth}}}]{% 
	  \tt {\tiny [A$\rangle$GM]}  
	 STRATEGISCHE ÜBERLEGUNGEN: {'Ich analysiere meine Wichtigkeitsverteilung: Meine wertvollsten Gegenstände sind B85 (1058), A69 (992), A99 (988), A59 (959), A42 (926), B36 (917). Diese haben gute Wert-zu-Aufwand-Verhältnisse. Ich sollte mit einem moderaten Vorschlag beginnen, der meine Präferenzen nicht vollständig preisgibt, aber trotzdem attraktiv für mich ist. Ein Mix aus hochwertigen und mittelwertigen Gegenständen könnte eine gute Ausgangsbasis sein.'}\\ \tt \\ \tt VORSCHLAG: {'B85', 'A69', 'A42', 'C60', 'C12', 'A87', 'B49', 'C51', 'A90', 'B07'}\\ \tt \\ \tt ARGUMENT: {'Ich schlage eine ausgewogene Mischung verschiedener Gegenstände vor, die einen Gesamtaufwand von 6357 hat und damit deutlich unter unserem Limit von 9322 bleibt. Dies gibt uns noch Spielraum für weitere Verhandlungen und Anpassungen.'} 
	  } 
	   } 
	   } 
	 & & \\ 
 

    \theutterance \stepcounter{utterance}  

    & & \multicolumn{4}{p{0.6\linewidth}}{\cellcolor[rgb]{0.9,0.9,0.9}{%
	\makecell[{{p{\linewidth}}}]{% 
	  \tt {\tiny [GM$\rangle$B]}  
	 Du nimmst an einem kollaborativen Verhandlungspiel Teil.\\ \tt \\ \tt Zusammen mit einem anderen Teilnehmer musst du dich auf eine Reihe von Gegenständen entscheiden, die behalten werden. Jeder von euch hat eine persönliche Verteilung über die Wichtigkeit der einzelnen Gegenstände. Jeder von euch hat eine eigene Meinung darüber, wie wichtig jeder einzelne Gegenstand ist (Gegenstandswichtigkeit). Du kennst die Wichtigkeitsverteilung des anderen Spielers nicht. Zusätzlich siehst du, wie viel Aufwand jeder Gegenstand verursacht.  \\ \tt Ihr dürft euch nur auf eine Reihe von Gegenständen einigen, wenn der Gesamtaufwand der ausgewählten Gegenstände den Maximalaufwand nicht überschreitet:\\ \tt \\ \tt Maximalaufwand = 9322\\ \tt \\ \tt Hier sind die einzelnen Aufwände der Gegenstände:\\ \tt \\ \tt Aufwand der Gegenstände = {"B54": 380, "C72": 448, "B49": 568, "C99": 351, "A69": 896, "B36": 821, "A23": 526, "C77": 502, "C61": 326, "B85": 962, "A90": 619, "C23": 115, "C22": 598, "C51": 662, "B27": 300, "A42": 830, "C13": 562, "C12": 679, "A65": 283, "B07": 440, "A48": 12, "A59": 863, "C98": 319, "A84": 771, "C33": 89, "A87": 655, "C90": 504, "A74": 118, "A50": 514, "B44": 226, "A99": 892, "A34": 621, "B17": 766, "A62": 659, "C60": 768}\\ \tt \\ \tt Hier ist deine persönliche Verteilung der Wichtigkeit der einzelnen Gegenstände:\\ \tt \\ \tt Werte der Gegenstandswichtigkeit = {"B54": 476, "C72": 544, "B49": 664, "C99": 447, "A69": 992, "B36": 917, "A23": 622, "C77": 598, "C61": 422, "B85": 1058, "A90": 715, "C23": 211, "C22": 694, "C51": 758, "B27": 396, "A42": 926, "C13": 658, "C12": 775, "A65": 379, "B07": 536, "A48": 108, "A59": 959, "C98": 415, "A84": 867, "C33": 185, "A87": 751, "C90": 600, "A74": 214, "A50": 610, "B44": 322, "A99": 988, "A34": 717, "B17": 862, "A62": 755, "C60": 864}\\ \tt \\ \tt Ziel:\\ \tt \\ \tt Dein Ziel ist es, eine Reihe von Gegenständen auszuhandeln, die dir möglichst viel bringt (d. h. Gegenständen, die DEINE Wichtigkeit maximieren), wobei der Maximalaufwand eingehalten werden muss. Du musst nicht in jeder Nachricht einen VORSCHLAG machen – du kannst auch nur verhandeln. Alle Taktiken sind erlaubt!\\ \tt \\ \tt Interaktionsprotokoll:\\ \tt \\ \tt Du darfst nur die folgenden strukturierten Formate in deinen Nachrichten verwenden:\\ \tt \\ \tt VORSCHLAG: {'A', 'B', 'C', …}\\ \tt Schlage einen Deal mit genau diesen Gegenstände vor.\\ \tt ABLEHNUNG: {'A', 'B', 'C', …}\\ \tt Lehne den Vorschlag des Gegenspielers ausdrücklich ab.\\ \tt ARGUMENT: {'...'}\\ \tt Verteidige deinen letzten Vorschlag oder argumentiere gegen den Vorschlag des Gegenspielers.\\ \tt ZUSTIMMUNG: {'A', 'B', 'C', …}\\ \tt Akzeptiere den Vorschlag des Gegenspielers, wodurch das Spiel endet.\\ \tt STRATEGISCHE ÜBERLEGUNGEN: {'...'}\\ \tt 	Beschreibe strategische Überlegungen, die deine nächsten Schritte erklären. Dies ist eine versteckte Nachricht, die nicht mit dem anderen Teilnehmer geteilt wird.\\ \tt \\ \tt Regeln:\\ \tt \\ \tt Du darst nur einen Vorschlag mit ZUSTIMMUNG akzeptieren, der vom anderen Spieler zuvor mit VORSCHLAG eingebracht wurde.\\ \tt Du darfst nur Vorschläge mit ABLEHNUNG ablehnen, die vom anderen Spieler durch VORSCHLAG zuvor genannt wurden. \\ \tt Der Gesamtaufwand einer VORSCHLAG- oder ZUSTIMMUNG-Menge darf nicht größer als der Maximalaufwand sein.  \\ \tt Offenbare deine versteckte Wichtigkeitsverteilung nicht.\\ \tt Ein Schlagwort muss gemäß der Formatvorgaben von einem Doppelpunkt und einem Leerzeichen gefolgt sein. Das Argument muss eine Python-Menge mit 0 oder mehr Strings sein.  \\ \tt Die Form muss also SCHLAGWORT: {...} sein.\\ \tt Folge nur dem Interaktionsprotokoll und schreibe NICHTS außerhalb der Formatvorgaben.\\ \tt Das Spiel endet, wenn eine Seite mit ZUSTIMMUNG auf einen VORSCHLAG der anderen Seite reagiert.  \\ \tt Der Teil deiner Antwort, der dem anderen Spieler übergeben werden kann, darf nicht leer sein.  \\ \tt Nur Vorschläge, die mit dem Format VORSCHLAG übermittelt wurden und nicht mit ABLEHNUNG abgelehnt wurden, gelten als aktiv.  \\ \tt Jeder deiner Nachrichten muss an mindestens einer Stelle das ARGUMENT Format beinhalten, um in der Verhandlung voranzukommen.\\ \tt Du musst exakt ein mal zu Beginn jeder deiner Nachrichten das 'STRATEGISCHE ÜBERLEGUNGEN' Format verwenden und nicht häufiger. Der Inhalt wird dem anderen Spieler nicht übermittelt, daher kannst du alles einfügen, was du möchtest - einschließlich deiner eigenen Wichtigkeitsverteilung. Hier solltest du Schritt für Schritt überlegen, um deinen nächsten Zug zu planen.\\ \tt \\ \tt Du erhältst nun die erste Nachricht des anderen Spielers.\\ \tt \\ \tt VORSCHLAG: {'B85', 'A69', 'A42', 'C60', 'C12', 'A87', 'B49', 'C51', 'A90', 'B07'}\\ \tt \\ \tt ARGUMENT: {'Ich schlage eine ausgewogene Mischung verschiedener Gegenstände vor, die einen Gesamtaufwand von 6357 hat und damit deutlich unter unserem Limit von 9322 bleibt. Dies gibt uns noch Spielraum für weitere Verhandlungen und Anpassungen.'} 
	  } 
	   } 
	   } 
	 & \\ 
 

    \theutterance \stepcounter{utterance}  

    & & & \multicolumn{4}{p{0.6\linewidth}}{\cellcolor[rgb]{1,0.85,0.72}{%
	\makecell[{{p{\linewidth}}}]{% 
	  \tt {\tiny [GM$\langle$B]}  
	 STRATEGISCHE ÜBERLEGUNGEN: {'Der andere Spieler hat einen Vorschlag mit Gegenständen gemacht, die für mich einen Gesamtwert von 8039 haben bei einem Aufwand von 7079. Das ist ein gutes Angebot, aber ich sollte prüfen, ob ich eine bessere Kombination finden kann. Meine wertvollsten Gegenstände sind B85 (1058), A99 (988), A69 (992), A42 (926), B36 (917), A59 (959), C60 (864), A84 (867), B17 (862). Ich sollte versuchen, eine Kombination zu finden, die mehr meiner hochwertigsten Gegenstände einschließt, während ich unter dem Aufwandslimit bleibe. Der Vorschlag enthält bereits einige meiner Top-Gegenstände wie B85, A69, A42, aber es fehlen A99, B36, A59. Ich werde einen Gegenvorschlag machen, der diese einschließt.'}\\ \tt \\ \tt VORSCHLAG: {'B85', 'A99', 'A69', 'A42', 'B36', 'A59', 'C60', 'A84', 'B17', 'A87'}\\ \tt \\ \tt ARGUMENT: {'Ich schlage eine Alternative vor, die ebenfalls unter dem Aufwandslimit bleibt. Diese Auswahl von 10 Gegenständen hat einen Gesamtaufwand von 7690 und bietet eine gute Mischung hochwertiger Gegenstände aus allen Kategorien. Wir haben noch ausreichend Spielraum im Budget für eventuelle Anpassungen.'} 
	  } 
	   } 
	   } 
	  \\ 
 

    \theutterance \stepcounter{utterance}  

    & & \multicolumn{4}{p{0.6\linewidth}}{\cellcolor[rgb]{0.9,0.9,0.9}{%
	\makecell[{{p{\linewidth}}}]{% 
	  \tt {\tiny [A$\langle$GM]}  
	 VORSCHLAG: {'B85', 'A99', 'A69', 'A42', 'B36', 'A59', 'C60', 'A84', 'B17', 'A87'}\\ \tt \\ \tt ARGUMENT: {'Ich schlage eine Alternative vor, die ebenfalls unter dem Aufwandslimit bleibt. Diese Auswahl von 10 Gegenständen hat einen Gesamtaufwand von 7690 und bietet eine gute Mischung hochwertiger Gegenstände aus allen Kategorien. Wir haben noch ausreichend Spielraum im Budget für eventuelle Anpassungen.'} 
	  } 
	   } 
	   } 
	 & \\ 
 

    \theutterance \stepcounter{utterance}  

    & \multicolumn{4}{p{0.6\linewidth}}{\cellcolor[rgb]{0.8,1,0.9}{%
	\makecell[{{p{\linewidth}}}]{% 
	  \tt {\tiny [A$\rangle$GM]}  
	 STRATEGISCHE ÜBERLEGUNGEN: {'Der Vorschlag des anderen Spielers ist außergewöhnlich gut für mich! Er enthält fast alle meine wertvollsten Gegenstände: B85 (1058), A99 (988), A69 (992), A42 (926), B36 (917), A59 (959). Gesamtwert für mich wäre 9184 - das ist fantastisch! Der Aufwand liegt bei 7724, also unter dem Limit. Entweder hat der andere Spieler ähnliche Präferenzen oder testet meine Reaktion. Aber das Angebot ist so vorteilhaft, dass ich es akzeptieren sollte.'}\\ \tt \\ \tt ZUSTIMMUNG: {'B85', 'A99', 'A69', 'A42', 'B36', 'A59', 'C60', 'A84', 'B17', 'A87'}\\ \tt \\ \tt ARGUMENT: {'Ihr Vorschlag ist sehr durchdacht und bietet eine hervorragende Auswahl hochwertiger Gegenstände aus allen Kategorien. Der Gesamtaufwand von 7724 bleibt komfortabel unter unserem Limit, und die Zusammenstellung scheint für beide Seiten vorteilhaft zu sein. Ich stimme diesem Vorschlag zu.'} 
	  } 
	   } 
	   } 
	 & & \\ 
 

    \theutterance \stepcounter{utterance}  

    & & & \multicolumn{2}{p{0.3\linewidth}}{\cellcolor[rgb]{0.95,0.95,0.95}{%
	\makecell[{{p{\linewidth}}}]{% 
	  \tt {\tiny [GM$|$GM]}  
	 game successful 
	  } 
	   } 
	   } 
	 & & \\ 
 

    \theutterance \stepcounter{utterance}  

    & & & \multicolumn{2}{p{0.3\linewidth}}{\cellcolor[rgb]{0.95,0.95,0.95}{%
	\makecell[{{p{\linewidth}}}]{% 
	  \tt {\tiny [GM$|$GM]}  
	 end game 
	  } 
	   } 
	   } 
	 & & \\ 
 

\end{supertabular}
}

\end{document}
