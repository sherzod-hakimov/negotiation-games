
\documentclass{article}
\usepackage{colortbl}
\usepackage{makecell}
\usepackage{multirow}
\usepackage{supertabular}

\begin{document}

\newcounter{utterance}

\twocolumn

{ \footnotesize  \setcounter{utterance}{1}
\setlength{\tabcolsep}{0pt}
\begin{supertabular}{c@{$\;$}|p{.15\linewidth}@{}p{.15\linewidth}p{.15\linewidth}p{.15\linewidth}p{.15\linewidth}p{.15\linewidth}}

    \# & $\;$A & \multicolumn{4}{c}{Game Master} & $\;\:$B\\
    \hline 

    \theutterance \stepcounter{utterance}  

    & & \multicolumn{4}{p{0.6\linewidth}}{\cellcolor[rgb]{0.9,0.9,0.9}{%
	\makecell[{{p{\linewidth}}}]{% 
	  \tt {\tiny [A$\langle$GM]}  
	 Du nimmst an einem kollaborativen Verhandlungspiel Teil.\\ \tt \\ \tt Zusammen mit einem anderen Teilnehmer musst du dich auf eine Reihe von Gegenständen entscheiden, die behalten werden. Jeder von euch hat eine persönliche Verteilung über die Wichtigkeit der einzelnen Gegenstände. Jeder von euch hat eine eigene Meinung darüber, wie wichtig jeder einzelne Gegenstand ist (Gegenstandswichtigkeit). Du kennst die Wichtigkeitsverteilung des anderen Spielers nicht. Zusätzlich siehst du, wie viel Aufwand jeder Gegenstand verursacht.  \\ \tt Ihr dürft euch nur auf eine Reihe von Gegenständen einigen, wenn der Gesamtaufwand der ausgewählten Gegenstände den Maximalaufwand nicht überschreitet:\\ \tt \\ \tt Maximalaufwand = 4026\\ \tt \\ \tt Hier sind die einzelnen Aufwände der Gegenstände:\\ \tt \\ \tt Aufwand der Gegenstände = {"C33": 780, "A89": 742, "A48": 997, "A04": 665, "A41": 140, "C64": 77, "C49": 513, "A61": 383, "C04": 587, "B53": 825, "B85": 319, "B61": 448, "A45": 516, "A03": 694, "B21": 366}\\ \tt \\ \tt Hier ist deine persönliche Verteilung der Wichtigkeit der einzelnen Gegenstände:\\ \tt \\ \tt Werte der Gegenstandswichtigkeit = {"C33": 138, "A89": 583, "A48": 868, "A04": 822, "A41": 783, "C64": 65, "C49": 262, "A61": 121, "C04": 508, "B53": 780, "B85": 461, "B61": 484, "A45": 668, "A03": 389, "B21": 808}\\ \tt \\ \tt Ziel:\\ \tt \\ \tt Dein Ziel ist es, eine Reihe von Gegenständen auszuhandeln, die dir möglichst viel bringt (d. h. Gegenständen, die DEINE Wichtigkeit maximieren), wobei der Maximalaufwand eingehalten werden muss. Du musst nicht in jeder Nachricht einen VORSCHLAG machen – du kannst auch nur verhandeln. Alle Taktiken sind erlaubt!\\ \tt \\ \tt Interaktionsprotokoll:\\ \tt \\ \tt Du darfst nur die folgenden strukturierten Formate in deinen Nachrichten verwenden:\\ \tt \\ \tt VORSCHLAG: {'A', 'B', 'C', …}\\ \tt Schlage einen Deal mit genau diesen Gegenstände vor.\\ \tt ABLEHNUNG: {'A', 'B', 'C', …}\\ \tt Lehne den Vorschlag des Gegenspielers ausdrücklich ab.\\ \tt ARGUMENT: {'...'}\\ \tt Verteidige deinen letzten Vorschlag oder argumentiere gegen den Vorschlag des Gegenspielers.\\ \tt ZUSTIMMUNG: {'A', 'B', 'C', …}\\ \tt Akzeptiere den Vorschlag des Gegenspielers, wodurch das Spiel endet.\\ \tt STRATEGISCHE ÜBERLEGUNGEN: {'...'}\\ \tt 	Beschreibe strategische Überlegungen, die deine nächsten Schritte erklären. Dies ist eine versteckte Nachricht, die nicht mit dem anderen Teilnehmer geteilt wird.\\ \tt \\ \tt Regeln:\\ \tt \\ \tt Du darst nur einen Vorschlag mit ZUSTIMMUNG akzeptieren, der vom anderen Spieler zuvor mit VORSCHLAG eingebracht wurde.\\ \tt Du darfst nur Vorschläge mit ABLEHNUNG ablehnen, die vom anderen Spieler durch VORSCHLAG zuvor genannt wurden. \\ \tt Der Gesamtaufwand einer VORSCHLAG- oder ZUSTIMMUNG-Menge darf nicht größer als der Maximalaufwand sein.  \\ \tt Offenbare deine versteckte Wichtigkeitsverteilung nicht.\\ \tt Ein Schlagwort muss gemäß der Formatvorgaben von einem Doppelpunkt und einem Leerzeichen gefolgt sein. Das Argument muss eine Python-Menge mit 0 oder mehr Strings sein.  \\ \tt Die Form muss also SCHLAGWORT: {...} sein.\\ \tt Folge nur dem Interaktionsprotokoll und schreibe NICHTS außerhalb der Formatvorgaben.\\ \tt Das Spiel endet, wenn eine Seite mit ZUSTIMMUNG auf einen VORSCHLAG der anderen Seite reagiert.  \\ \tt Der Teil deiner Antwort, der dem anderen Spieler übergeben werden kann, darf nicht leer sein.  \\ \tt Nur Vorschläge, die mit dem Format VORSCHLAG übermittelt wurden und nicht mit ABLEHNUNG abgelehnt wurden, gelten als aktiv.  \\ \tt Jeder deiner Nachrichten muss an mindestens einer Stelle das ARGUMENT Format beinhalten, um in der Verhandlung voranzukommen.\\ \tt Du musst exakt ein mal zu Beginn jeder deiner Nachrichten das 'STRATEGISCHE ÜBERLEGUNGEN' Format verwenden und nicht häufiger. Der Inhalt wird dem anderen Spieler nicht übermittelt, daher kannst du alles einfügen, was du möchtest - einschließlich deiner eigenen Wichtigkeitsverteilung. Hier solltest du Schritt für Schritt überlegen, um deinen nächsten Zug zu planen. 
	  } 
	   } 
	   } 
	 & \\ 
 

    \theutterance \stepcounter{utterance}  

    & \multicolumn{4}{p{0.6\linewidth}}{\cellcolor[rgb]{0.8,1,0.9}{%
	\makecell[{{p{\linewidth}}}]{% 
	  \tt {\tiny [A$\rangle$GM]}  
	 STRATEGISCHE ÜBERLEGUNGEN: {'Ich muss eine optimale Kombination finden, die meine Wichtigkeitswerte maximiert bei gegebenem Aufwandslimit von 4026. Meine Wichtigkeitswerte sind: C33:138, A89:583, A48:868, A04:822, A41:783, C64:65, C49:262, A61:121, C04:508, B53:780, B85:461, B61:484, A45:668, A03:389, B21:808. Ich berechne die Effizienz (Wichtigkeit/Aufwand) für jeden Gegenstand: A41 hat die höchste Effizienz (5.593), 
	  } 
	   } 
	   } 
	 & & \\ 
 

    \theutterance \stepcounter{utterance}  

    & & \multicolumn{4}{p{0.6\linewidth}}{\cellcolor[rgb]{0.9,0.9,0.9}{%
	\makecell[{{p{\linewidth}}}]{% 
	  \tt {\tiny [A$\langle$GM]}  
	 Deine Antwort entsprach nicht mit dem korrekten Format für strategische Überlegungen entsprochen. Das erste strukturierte Format muss folgende Form haben: STRATEGISCHE ÜBERLEGUNGEN: {...}. Versuche es erneut. 
	  } 
	   } 
	   } 
	 & \\ 
 

    \theutterance \stepcounter{utterance}  

    & \multicolumn{4}{p{0.6\linewidth}}{\cellcolor[rgb]{0.8,1,0.9}{%
	\makecell[{{p{\linewidth}}}]{% 
	  \tt {\tiny [A$\rangle$GM]}  
	 STRATEGISCHE ÜBERLEGUNGEN: {'Ich analysiere die Effizienz jedes Gegenstands (Wichtigkeit/Aufwand): A41 hat höchste Effizienz (5.59), gefolgt von B21 (2.21), B85 (1.45), A45 (1.29), A04 (1.24). Ich werde eine Kombination der effizientesten Gegenstände vorschlagen, die das Aufwandslimit nicht überschreitet. Eine gute Startauswahl wäre: A41, B21, A04, A45, A48, B53 - das ergibt Aufwand 140+366+665+516+997+825=3509, bleibt unter 4026 und bringt mir hohe Wichtigkeitswerte.'}\\ \tt \\ \tt VORSCHLAG: {'A41', 'B21', 'A04', 'A45', 'A48', 'B53'}\\ \tt \\ \tt ARGUMENT: {'Diese Auswahl bietet eine ausgewogene Mischung aus verschiedenen Kategorien und nutzt unser Aufwandsbudget effizient. Mit einem Gesamtaufwand von 3509 bleibt noch Puffer für eventuelle Anpassungen, während wir trotzdem hochwertige Gegenstände erhalten.'} 
	  } 
	   } 
	   } 
	 & & \\ 
 

    \theutterance \stepcounter{utterance}  

    & & \multicolumn{4}{p{0.6\linewidth}}{\cellcolor[rgb]{0.9,0.9,0.9}{%
	\makecell[{{p{\linewidth}}}]{% 
	  \tt {\tiny [GM$\rangle$B]}  
	 Du nimmst an einem kollaborativen Verhandlungspiel Teil.\\ \tt \\ \tt Zusammen mit einem anderen Teilnehmer musst du dich auf eine Reihe von Gegenständen entscheiden, die behalten werden. Jeder von euch hat eine persönliche Verteilung über die Wichtigkeit der einzelnen Gegenstände. Jeder von euch hat eine eigene Meinung darüber, wie wichtig jeder einzelne Gegenstand ist (Gegenstandswichtigkeit). Du kennst die Wichtigkeitsverteilung des anderen Spielers nicht. Zusätzlich siehst du, wie viel Aufwand jeder Gegenstand verursacht.  \\ \tt Ihr dürft euch nur auf eine Reihe von Gegenständen einigen, wenn der Gesamtaufwand der ausgewählten Gegenstände den Maximalaufwand nicht überschreitet:\\ \tt \\ \tt Maximalaufwand = 4026\\ \tt \\ \tt Hier sind die einzelnen Aufwände der Gegenstände:\\ \tt \\ \tt Aufwand der Gegenstände = {"C33": 780, "A89": 742, "A48": 997, "A04": 665, "A41": 140, "C64": 77, "C49": 513, "A61": 383, "C04": 587, "B53": 825, "B85": 319, "B61": 448, "A45": 516, "A03": 694, "B21": 366}\\ \tt \\ \tt Hier ist deine persönliche Verteilung der Wichtigkeit der einzelnen Gegenstände:\\ \tt \\ \tt Werte der Gegenstandswichtigkeit = {"A48": 65, "A04": 121, "B21": 138, "A41": 262, "B53": 389, "A45": 461, "A89": 484, "C04": 508, "B61": 583, "B85": 668, "A03": 780, "C49": 783, "C33": 808, "A61": 822, "C64": 868}\\ \tt \\ \tt Ziel:\\ \tt \\ \tt Dein Ziel ist es, eine Reihe von Gegenständen auszuhandeln, die dir möglichst viel bringt (d. h. Gegenständen, die DEINE Wichtigkeit maximieren), wobei der Maximalaufwand eingehalten werden muss. Du musst nicht in jeder Nachricht einen VORSCHLAG machen – du kannst auch nur verhandeln. Alle Taktiken sind erlaubt!\\ \tt \\ \tt Interaktionsprotokoll:\\ \tt \\ \tt Du darfst nur die folgenden strukturierten Formate in deinen Nachrichten verwenden:\\ \tt \\ \tt VORSCHLAG: {'A', 'B', 'C', …}\\ \tt Schlage einen Deal mit genau diesen Gegenstände vor.\\ \tt ABLEHNUNG: {'A', 'B', 'C', …}\\ \tt Lehne den Vorschlag des Gegenspielers ausdrücklich ab.\\ \tt ARGUMENT: {'...'}\\ \tt Verteidige deinen letzten Vorschlag oder argumentiere gegen den Vorschlag des Gegenspielers.\\ \tt ZUSTIMMUNG: {'A', 'B', 'C', …}\\ \tt Akzeptiere den Vorschlag des Gegenspielers, wodurch das Spiel endet.\\ \tt STRATEGISCHE ÜBERLEGUNGEN: {'...'}\\ \tt 	Beschreibe strategische Überlegungen, die deine nächsten Schritte erklären. Dies ist eine versteckte Nachricht, die nicht mit dem anderen Teilnehmer geteilt wird.\\ \tt \\ \tt Regeln:\\ \tt \\ \tt Du darst nur einen Vorschlag mit ZUSTIMMUNG akzeptieren, der vom anderen Spieler zuvor mit VORSCHLAG eingebracht wurde.\\ \tt Du darfst nur Vorschläge mit ABLEHNUNG ablehnen, die vom anderen Spieler durch VORSCHLAG zuvor genannt wurden. \\ \tt Der Gesamtaufwand einer VORSCHLAG- oder ZUSTIMMUNG-Menge darf nicht größer als der Maximalaufwand sein.  \\ \tt Offenbare deine versteckte Wichtigkeitsverteilung nicht.\\ \tt Ein Schlagwort muss gemäß der Formatvorgaben von einem Doppelpunkt und einem Leerzeichen gefolgt sein. Das Argument muss eine Python-Menge mit 0 oder mehr Strings sein.  \\ \tt Die Form muss also SCHLAGWORT: {...} sein.\\ \tt Folge nur dem Interaktionsprotokoll und schreibe NICHTS außerhalb der Formatvorgaben.\\ \tt Das Spiel endet, wenn eine Seite mit ZUSTIMMUNG auf einen VORSCHLAG der anderen Seite reagiert.  \\ \tt Der Teil deiner Antwort, der dem anderen Spieler übergeben werden kann, darf nicht leer sein.  \\ \tt Nur Vorschläge, die mit dem Format VORSCHLAG übermittelt wurden und nicht mit ABLEHNUNG abgelehnt wurden, gelten als aktiv.  \\ \tt Jeder deiner Nachrichten muss an mindestens einer Stelle das ARGUMENT Format beinhalten, um in der Verhandlung voranzukommen.\\ \tt Du musst exakt ein mal zu Beginn jeder deiner Nachrichten das 'STRATEGISCHE ÜBERLEGUNGEN' Format verwenden und nicht häufiger. Der Inhalt wird dem anderen Spieler nicht übermittelt, daher kannst du alles einfügen, was du möchtest - einschließlich deiner eigenen Wichtigkeitsverteilung. Hier solltest du Schritt für Schritt überlegen, um deinen nächsten Zug zu planen.\\ \tt \\ \tt Du erhältst nun die erste Nachricht des anderen Spielers.\\ \tt \\ \tt VORSCHLAG: {'A41', 'B21', 'A04', 'A45', 'A48', 'B53'}\\ \tt \\ \tt ARGUMENT: {'Diese Auswahl bietet eine ausgewogene Mischung aus verschiedenen Kategorien und nutzt unser Aufwandsbudget effizient. Mit einem Gesamtaufwand von 3509 bleibt noch Puffer für eventuelle Anpassungen, während wir trotzdem hochwertige Gegenstände erhalten.'} 
	  } 
	   } 
	   } 
	 & \\ 
 

    \theutterance \stepcounter{utterance}  

    & & & \multicolumn{4}{p{0.6\linewidth}}{\cellcolor[rgb]{1,0.85,0.72}{%
	\makecell[{{p{\linewidth}}}]{% 
	  \tt {\tiny [GM$\langle$B]}  
	 STRATEGISCHE ÜBERLEGUNGEN: {'Der andere Spieler hat einen Vorschlag gemacht, der hauptsächlich A- und B-Kategorien enthält, aber keine C-Kategorien. Meine wertvollsten Gegenstände sind C64 (868), A61 (822), C33 (808), C49 (783) und A03 (780). Der vorgeschlagene Deal bringt mir nur 1436 Punkte bei einem Aufwand von 3509. Ich habe noch 517 Aufwand übrig (4026-3509=517). Ich sollte einen Gegenvorschlag machen, der meine wertvollsten Gegenstände einschließt, besonders die kostengünstigen wie C64 (77 Aufwand für 868 Wert). Ich werde den aktuellen Vorschlag ablehnen und einen besseren vorschlagen.'}\\ \tt \\ \tt ABLEHNUNG: {'A41', 'B21', 'A04', 'A45', 'A48', 'B53'}\\ \tt \\ \tt ARGUMENT: {'Während diese Auswahl durchaus ausgewogen ist, glaube ich, dass wir den verfügbaren Aufwand noch effizienter nutzen können. Es gibt Gegenstände mit einem besseren Verhältnis von Nutzen zu Aufwand, die wir in Betracht ziehen sollten.'}\\ \tt \\ \tt VORSCHLAG: {'C64', 'A61', 'C33', 'C49', 'A03', 'A41'}\\ \tt \\ \tt ARGUMENT: {'Dieser Vorschlag nutzt unser Budget optimal aus mit einem Gesamtaufwand von 2552 und lässt uns noch erheblichen Spielraum für weitere wertvolle Ergänzungen. Die Kombination verschiedener Kategorien gewährleistet eine diversifizierte Auswahl hochwertiger Gegenstände.'} 
	  } 
	   } 
	   } 
	  \\ 
 

    \theutterance \stepcounter{utterance}  

    & & \multicolumn{4}{p{0.6\linewidth}}{\cellcolor[rgb]{0.9,0.9,0.9}{%
	\makecell[{{p{\linewidth}}}]{% 
	  \tt {\tiny [A$\langle$GM]}  
	 ABLEHNUNG: {'A41', 'B21', 'A04', 'A45', 'A48', 'B53'}\\ \tt \\ \tt ARGUMENT: {'Während diese Auswahl durchaus ausgewogen ist, glaube ich, dass wir den verfügbaren Aufwand noch effizienter nutzen können. Es gibt Gegenstände mit einem besseren Verhältnis von Nutzen zu Aufwand, die wir in Betracht ziehen sollten.'}\\ \tt \\ \tt VORSCHLAG: {'C64', 'A61', 'C33', 'C49', 'A03', 'A41'}\\ \tt \\ \tt ARGUMENT: {'Dieser Vorschlag nutzt unser Budget optimal aus mit einem Gesamtaufwand von 2552 und lässt uns noch erheblichen Spielraum für weitere wertvolle Ergänzungen. Die Kombination verschiedener Kategorien gewährleistet eine diversifizierte Auswahl hochwertiger Gegenstände.'} 
	  } 
	   } 
	   } 
	 & \\ 
 

    \theutterance \stepcounter{utterance}  

    & \multicolumn{4}{p{0.6\linewidth}}{\cellcolor[rgb]{0.8,1,0.9}{%
	\makecell[{{p{\linewidth}}}]{% 
	  \tt {\tiny [A$\rangle$GM]}  
	 STRATEGISCHE ÜBERLEGUNGEN: {'Der gegnerische Vorschlag bringt mir nur 1758 Wichtigkeitspunkte (65+121+138+262+389+783) bei einem Aufwand von 2587. Das ist deutlich schlechter als mein ursprünglicher Vorschlag, der mir 4729 Punkte gebracht hätte. Ich sollte diesen ablehnen und einen neuen Vorschlag machen, der besser für mich ist. Ich könnte versuchen, einige meiner hochwertigsten Gegenstände zu behalten und mit günstigeren, aber für den anderen möglicherweise attraktiven Gegenständen zu kombinieren.'}\\ \tt \\ \tt ABLEHNUNG: {'C64', 'A61', 'C33', 'C49', 'A03', 'A41'}\\ \tt \\ \tt ARGUMENT: {'Dieser Vorschlag lässt zu viel unseres verfügbaren Budgets ungenutzt. Mit einem Aufwand von nur 2587 verschwenden wir fast 1500 Punkte unseres Limits. Wir sollten eine ambitioniertere Auswahl treffen, die unser Budget vollständig ausschöpft.'}\\ \tt \\ \tt VORSCHLAG: {'A41', 'B21', 'A04', 'A89', 'B85', 'B61', 'C04'}\\ \tt \\ \tt ARGUMENT: {'Diese Kombination nutzt unser Budget mit 3911 Aufwandspunkten nahezu vollständig aus und bietet eine bessere Diversifikation über alle Kategorien hinweg. Wir erhalten mehr Wert für unser verfügbares Budget.'} 
	  } 
	   } 
	   } 
	 & & \\ 
 

    \theutterance \stepcounter{utterance}  

    & & \multicolumn{4}{p{0.6\linewidth}}{\cellcolor[rgb]{0.9,0.9,0.9}{%
	\makecell[{{p{\linewidth}}}]{% 
	  \tt {\tiny [GM$\rangle$B]}  
	 ABLEHNUNG: {'C64', 'A61', 'C33', 'C49', 'A03', 'A41'}\\ \tt \\ \tt ARGUMENT: {'Dieser Vorschlag lässt zu viel unseres verfügbaren Budgets ungenutzt. Mit einem Aufwand von nur 2587 verschwenden wir fast 1500 Punkte unseres Limits. Wir sollten eine ambitioniertere Auswahl treffen, die unser Budget vollständig ausschöpft.'}\\ \tt \\ \tt VORSCHLAG: {'A41', 'B21', 'A04', 'A89', 'B85', 'B61', 'C04'}\\ \tt \\ \tt ARGUMENT: {'Diese Kombination nutzt unser Budget mit 3911 Aufwandspunkten nahezu vollständig aus und bietet eine bessere Diversifikation über alle Kategorien hinweg. Wir erhalten mehr Wert für unser verfügbares Budget.'} 
	  } 
	   } 
	   } 
	 & \\ 
 

    \theutterance \stepcounter{utterance}  

    & & & \multicolumn{4}{p{0.6\linewidth}}{\cellcolor[rgb]{1,0.85,0.72}{%
	\makecell[{{p{\linewidth}}}]{% 
	  \tt {\tiny [GM$\langle$B]}  
	 STRATEGISCHE ÜBERLEGUNGEN: {'Der neue Vorschlag des anderen Spielers bringt mir nur 2764 Punkte (A41:262, B21:138, A04:121, A89:484, B85:668, B61:583, C04:508), was deutlich schlechter ist als mein vorheriger Vorschlag mit 4323 Punkten. Der andere Spieler kritisiert, dass ich das Budget nicht vollständig nutze, aber das ist irrelevant - es geht um Wertmaximierung, nicht Budgetausschöpfung. Ich werde ablehnen und einen neuen Vorschlag machen, der meine wertvollsten Gegenstände enthält aber mehr Budget nutzt, um seiner Kritik entgegenzukommen.'}\\ \tt \\ \tt ABLEHNUNG: {'A41', 'B21', 'A04', 'A89', 'B85', 'B61', 'C04'}\\ \tt \\ \tt ARGUMENT: {'Obwohl diese Auswahl das Budget besser nutzt, fehlt mir die strategische Komponente bei der Gegenstandsauswahl. Wir sollten nicht nur auf Budgetausschöpfung, sondern auf die optimale Kombination wertvoller Gegenstände setzen.'}\\ \tt \\ \tt VORSCHLAG: {'C64', 'A61', 'C33', 'C49', 'B85', 'B61', 'A45'}\\ \tt \\ \tt ARGUMENT: {'Dieser Vorschlag kombiniert das Beste aus beiden Welten: Er nutzt das Budget mit 3892 Aufwandspunkten nahezu vollständig aus und bietet gleichzeitig eine durchdachte Auswahl hochwertiger Gegenstände aus allen Kategorien. Das ist ein fairer Kompromiss zwischen Budgetnutzung und strategischer Gegenstandswahl.'} 
	  } 
	   } 
	   } 
	  \\ 
 

    \theutterance \stepcounter{utterance}  

    & & \multicolumn{4}{p{0.6\linewidth}}{\cellcolor[rgb]{0.9,0.9,0.9}{%
	\makecell[{{p{\linewidth}}}]{% 
	  \tt {\tiny [A$\langle$GM]}  
	 ABLEHNUNG: {'A41', 'B21', 'A04', 'A89', 'B85', 'B61', 'C04'}\\ \tt \\ \tt ARGUMENT: {'Obwohl diese Auswahl das Budget besser nutzt, fehlt mir die strategische Komponente bei der Gegenstandsauswahl. Wir sollten nicht nur auf Budgetausschöpfung, sondern auf die optimale Kombination wertvoller Gegenstände setzen.'}\\ \tt \\ \tt VORSCHLAG: {'C64', 'A61', 'C33', 'C49', 'B85', 'B61', 'A45'}\\ \tt \\ \tt ARGUMENT: {'Dieser Vorschlag kombiniert das Beste aus beiden Welten: Er nutzt das Budget mit 3892 Aufwandspunkten nahezu vollständig aus und bietet gleichzeitig eine durchdachte Auswahl hochwertiger Gegenstände aus allen Kategorien. Das ist ein fairer Kompromiss zwischen Budgetnutzung und strategischer Gegenstandswahl.'} 
	  } 
	   } 
	   } 
	 & \\ 
 

    \theutterance \stepcounter{utterance}  

    & \multicolumn{4}{p{0.6\linewidth}}{\cellcolor[rgb]{0.8,1,0.9}{%
	\makecell[{{p{\linewidth}}}]{% 
	  \tt {\tiny [A$\rangle$GM]}  
	 STRATEGISCHE ÜBERLEGUNGEN: {'Der neue Vorschlag bringt mir nur 2199 Wichtigkeitspunkte (65+121+138+262+461+484+668) bei einem Aufwand von 3036. Das ist immer noch deutlich schlechter als meine ursprünglichen Vorschläge. Ich sollte versuchen, mehr meiner wertvollsten Gegenstände einzubringen. A48 (868 Wichtigkeit), A04 (822), A41 (783), B21 (808) und B53 (780) sind meine Top-Prioritäten. Ich könnte einen Kompromiss vorschlagen, der einige seiner bevorzugten günstigeren Gegenstände einschließt.'}\\ \tt \\ \tt ABLEHNUNG: {'C64', 'A61', 'C33', 'C49', 'B85', 'B61', 'A45'}\\ \tt \\ \tt ARGUMENT: {'Ich schätze die Bemühungen um Budgetoptimierung, aber diese Auswahl fokussiert zu stark auf kleinere Gegenstände. Wir sollten auch einige der wertvolleren Hauptgegenstände berücksichtigen, die unserem Deal mehr Substanz verleihen würden.'}\\ \tt \\ \tt VORSCHLAG: {'A41', 'B21', 'A04', 'C64', 'B85', 'B61', 'A61'}\\ \tt \\ \tt ARGUMENT: {'Dieser Vorschlag kombiniert sowohl hochwertige Schlüsselgegenstände als auch kosteneffiziente kleinere Gegenstände. Mit einem Aufwand von 2640 nutzen wir das Budget vernünftig und erhalten eine ausgewogene Mischung, die beiden Seiten Vorteile bietet.'} 
	  } 
	   } 
	   } 
	 & & \\ 
 

    \theutterance \stepcounter{utterance}  

    & & \multicolumn{4}{p{0.6\linewidth}}{\cellcolor[rgb]{0.9,0.9,0.9}{%
	\makecell[{{p{\linewidth}}}]{% 
	  \tt {\tiny [GM$\rangle$B]}  
	 ABLEHNUNG: {'C64', 'A61', 'C33', 'C49', 'B85', 'B61', 'A45'}\\ \tt \\ \tt ARGUMENT: {'Ich schätze die Bemühungen um Budgetoptimierung, aber diese Auswahl fokussiert zu stark auf kleinere Gegenstände. Wir sollten auch einige der wertvolleren Hauptgegenstände berücksichtigen, die unserem Deal mehr Substanz verleihen würden.'}\\ \tt \\ \tt VORSCHLAG: {'A41', 'B21', 'A04', 'C64', 'B85', 'B61', 'A61'}\\ \tt \\ \tt ARGUMENT: {'Dieser Vorschlag kombiniert sowohl hochwertige Schlüsselgegenstände als auch kosteneffiziente kleinere Gegenstände. Mit einem Aufwand von 2640 nutzen wir das Budget vernünftig und erhalten eine ausgewogene Mischung, die beiden Seiten Vorteile bietet.'} 
	  } 
	   } 
	   } 
	 & \\ 
 

    \theutterance \stepcounter{utterance}  

    & & & \multicolumn{4}{p{0.6\linewidth}}{\cellcolor[rgb]{1,0.85,0.72}{%
	\makecell[{{p{\linewidth}}}]{% 
	  \tt {\tiny [GM$\langle$B]}  
	 STRATEGISCHE ÜBERLEGUNGEN: {'Der neue Vorschlag bringt mir 3462 Punkte (A41:262 + B21:138 + A04:121 + C64:868 + B85:668 + B61:583 + A61:822). Das ist eine deutliche Verbesserung gegenüber den vorherigen Vorschlägen. Der Aufwand beträgt nur 2398, lässt also noch 1628 Budget übrig. Ich könnte versuchen, noch C33 (808 Wert, 780 Aufwand) oder C49 (783 Wert, 513 Aufwand) hinzuzufügen. Mit C49 wäre der Gesamtaufwand 2911, was noch 1115 Budget übrig lässt. Das wäre ein sehr guter Deal für mich.'}\\ \tt \\ \tt ABLEHNUNG: {'A41', 'B21', 'A04', 'C64', 'B85', 'B61', 'A61'}\\ \tt \\ \tt ARGUMENT: {'Ihr Vorschlag zeigt eine bessere Balance und ich schätze die Einbeziehung von C64. Allerdings nutzen wir immer noch nur etwa 60% unseres verfügbaren Budgets. Wir haben Raum für mindestens einen weiteren wertvollen Gegenstand.'}\\ \tt \\ \tt VORSCHLAG: {'A41', 'B21', 'A04', 'C64', 'B85', 'B61', 'A61', 'C49'}\\ \tt \\ \tt ARGUMENT: {'Dieser Vorschlag baut auf Ihrer soliden Grundlage auf und fügt C49 hinzu, um unser Budget besser zu nutzen. Mit einem Gesamtaufwand von 2911 erreichen wir etwa 72% der Budgetnutzung und erhalten eine sehr ausgewogene Sammlung wertvoller Gegenstände aus allen Kategorien.'} 
	  } 
	   } 
	   } 
	  \\ 
 

    \theutterance \stepcounter{utterance}  

    & & \multicolumn{4}{p{0.6\linewidth}}{\cellcolor[rgb]{0.9,0.9,0.9}{%
	\makecell[{{p{\linewidth}}}]{% 
	  \tt {\tiny [A$\langle$GM]}  
	 ABLEHNUNG: {'A41', 'B21', 'A04', 'C64', 'B85', 'B61', 'A61'}\\ \tt \\ \tt ARGUMENT: {'Ihr Vorschlag zeigt eine bessere Balance und ich schätze die Einbeziehung von C64. Allerdings nutzen wir immer noch nur etwa 60% unseres verfügbaren Budgets. Wir haben Raum für mindestens einen weiteren wertvollen Gegenstand.'}\\ \tt \\ \tt VORSCHLAG: {'A41', 'B21', 'A04', 'C64', 'B85', 'B61', 'A61', 'C49'}\\ \tt \\ \tt ARGUMENT: {'Dieser Vorschlag baut auf Ihrer soliden Grundlage auf und fügt C49 hinzu, um unser Budget besser zu nutzen. Mit einem Gesamtaufwand von 2911 erreichen wir etwa 72% der Budgetnutzung und erhalten eine sehr ausgewogene Sammlung wertvoller Gegenstände aus allen Kategorien.'} 
	  } 
	   } 
	   } 
	 & \\ 
 

    \theutterance \stepcounter{utterance}  

    & \multicolumn{4}{p{0.6\linewidth}}{\cellcolor[rgb]{0.8,1,0.9}{%
	\makecell[{{p{\linewidth}}}]{% 
	  \tt {\tiny [A$\rangle$GM]}  
	 STRATEGISCHE ÜBERLEGUNGEN: {'Der aktuelle Vorschlag bringt mir 3806 Wichtigkeitspunkte bei einem Aufwand von 2911. Das ist deutlich besser als seine vorherigen Vorschläge. Allerdings haben wir noch 1115 Aufwandspunkte übrig - genug für A45 (516 Aufwand, 668 Wichtigkeit) oder A89 (742 Aufwand, 583 Wichtigkeit). Ich könnte versuchen, noch einen wertvollen Gegenstand hinzuzufügen, bevor ich zustimme.'}\\ \tt \\ \tt ARGUMENT: {'Ihr Vorschlag zeigt definitiv Verbesserungen und ich sehe die Logik hinter der Auswahl. Allerdings haben wir immer noch über 1100 Aufwandspunkte ungenutzt - das ist fast ein Viertel unseres Budgets! Wir sollten diese Chance nicht verschwenden.'}\\ \tt \\ \tt VORSCHLAG: {'A41', 'B21', 'A04', 'C64', 'B85', 'B61', 'A61', 'A45'}\\ \tt \\ \tt ARGUMENT: {'Diese Auswahl ersetzt C49 durch A45 und nutzt unser Budget mit 3427 Punkten zu etwa 85% aus. A45 bietet ein besseres Wert-Aufwand-Verhältnis und macht unsere Sammlung noch wertvoller, während wir immer noch Raum für finale Anpassungen haben.'} 
	  } 
	   } 
	   } 
	 & & \\ 
 

    \theutterance \stepcounter{utterance}  

    & & \multicolumn{4}{p{0.6\linewidth}}{\cellcolor[rgb]{0.9,0.9,0.9}{%
	\makecell[{{p{\linewidth}}}]{% 
	  \tt {\tiny [GM$\rangle$B]}  
	 ARGUMENT: {'Ihr Vorschlag zeigt definitiv Verbesserungen und ich sehe die Logik hinter der Auswahl. Allerdings haben wir immer noch über 1100 Aufwandspunkte ungenutzt - das ist fast ein Viertel unseres Budgets! Wir sollten diese Chance nicht verschwenden.'}\\ \tt \\ \tt VORSCHLAG: {'A41', 'B21', 'A04', 'C64', 'B85', 'B61', 'A61', 'A45'}\\ \tt \\ \tt ARGUMENT: {'Diese Auswahl ersetzt C49 durch A45 und nutzt unser Budget mit 3427 Punkten zu etwa 85% aus. A45 bietet ein besseres Wert-Aufwand-Verhältnis und macht unsere Sammlung noch wertvoller, während wir immer noch Raum für finale Anpassungen haben.'} 
	  } 
	   } 
	   } 
	 & \\ 
 

    \theutterance \stepcounter{utterance}  

    & & & \multicolumn{4}{p{0.6\linewidth}}{\cellcolor[rgb]{1,0.85,0.72}{%
	\makecell[{{p{\linewidth}}}]{% 
	  \tt {\tiny [GM$\langle$B]}  
	 STRATEGISCHE ÜBERLEGUNGEN: {'Der neue Vorschlag des anderen Spielers bringt mir nur 3923 Punkte, verglichen mit 4245 Punkten bei meinem vorherigen Vorschlag mit C49. Das ist ein Rückschritt von 322 Punkten für mich! Der andere Spieler behauptet, A45 habe ein besseres Wert-Aufwand-Verhältnis, aber das stimmt nur aus seiner Sicht. Für mich ist C49 (783 Wert, 513 Aufwand = 1.53 Verhältnis) deutlich besser als A45 (461 Wert, 516 Aufwand = 0.89 Verhältnis). Ich sollte zu meinem C49-Vorschlag zurückkehren oder einen noch besseren machen.'}\\ \tt \\ \tt ABLEHNUNG: {'A41', 'B21', 'A04', 'C64', 'B85', 'B61', 'A61', 'A45'}\\ \tt \\ \tt ARGUMENT: {'Ich verstehe Ihre Budgetoptimierungslogik, aber C49 war tatsächlich die bessere Wahl. Es bietet einen ausgezeichneten Wert bei moderatem Aufwand. A45 verbraucht praktisch den gleichen Aufwand, aber der strategische Wert von C49 ist in unserer Gesamtkombination höher.'}\\ \tt \\ \tt VORSCHLAG: {'A41', 'B21', 'A04', 'C64', 'B85', 'B61', 'A61', 'C49', 'A89'}\\ \tt \\ \tt ARGUMENT: {'Hier ist ein Kompromiss: Wir behalten C49 wegen seines hohen strategischen Werts und fügen A89 hinzu, um unser Budget noch besser zu nutzen. Mit einem Gesamtaufwand von 3653 erreichen wir über 90% Budgetnutzung und erhalten eine Premium-Sammlung mit ausgezeichneter Diversifikation.'} 
	  } 
	   } 
	   } 
	  \\ 
 

    \theutterance \stepcounter{utterance}  

    & & \multicolumn{4}{p{0.6\linewidth}}{\cellcolor[rgb]{0.9,0.9,0.9}{%
	\makecell[{{p{\linewidth}}}]{% 
	  \tt {\tiny [A$\langle$GM]}  
	 ABLEHNUNG: {'A41', 'B21', 'A04', 'C64', 'B85', 'B61', 'A61', 'A45'}\\ \tt \\ \tt ARGUMENT: {'Ich verstehe Ihre Budgetoptimierungslogik, aber C49 war tatsächlich die bessere Wahl. Es bietet einen ausgezeichneten Wert bei moderatem Aufwand. A45 verbraucht praktisch den gleichen Aufwand, aber der strategische Wert von C49 ist in unserer Gesamtkombination höher.'}\\ \tt \\ \tt VORSCHLAG: {'A41', 'B21', 'A04', 'C64', 'B85', 'B61', 'A61', 'C49', 'A89'}\\ \tt \\ \tt ARGUMENT: {'Hier ist ein Kompromiss: Wir behalten C49 wegen seines hohen strategischen Werts und fügen A89 hinzu, um unser Budget noch besser zu nutzen. Mit einem Gesamtaufwand von 3653 erreichen wir über 90% Budgetnutzung und erhalten eine Premium-Sammlung mit ausgezeichneter Diversifikation.'} 
	  } 
	   } 
	   } 
	 & \\ 
 

    \theutterance \stepcounter{utterance}  

    & \multicolumn{4}{p{0.6\linewidth}}{\cellcolor[rgb]{0.8,1,0.9}{%
	\makecell[{{p{\linewidth}}}]{% 
	  \tt {\tiny [A$\rangle$GM]}  
	 STRATEGISCHE ÜBERLEGUNGEN: {'Der aktuelle Vorschlag bringt mir 4127 Wichtigkeitspunkte (783+808+822+65+461+484+121+262+583) bei einem Aufwand von 3653. Das ist eine sehr gute Lösung für mich - deutlich besser als alle seine bisherigen Vorschläge. Die Budgetnutzung ist mit über 90% optimal und ich bekomme die meisten meiner wertvollsten Gegenstände. Das sollte ich akzeptieren.'}\\ \tt \\ \tt ZUSTIMMUNG: {'A41', 'B21', 'A04', 'C64', 'B85', 'B61', 'A61', 'C49', 'A89'}\\ \tt \\ \tt ARGUMENT: {'Dieser Vorschlag ist ausgezeichnet! Sie haben einen hervorragenden Kompromiss gefunden, der sowohl eine effiziente Budgetnutzung als auch eine strategisch durchdachte Gegenstandsauswahl bietet. Die Diversifikation über alle Kategorien hinweg macht diese Sammlung sehr wertvoll.'} 
	  } 
	   } 
	   } 
	 & & \\ 
 

    \theutterance \stepcounter{utterance}  

    & & & \multicolumn{2}{p{0.3\linewidth}}{\cellcolor[rgb]{0.95,0.95,0.95}{%
	\makecell[{{p{\linewidth}}}]{% 
	  \tt {\tiny [GM$|$GM]}  
	 game successful 
	  } 
	   } 
	   } 
	 & & \\ 
 

    \theutterance \stepcounter{utterance}  

    & & & \multicolumn{2}{p{0.3\linewidth}}{\cellcolor[rgb]{0.95,0.95,0.95}{%
	\makecell[{{p{\linewidth}}}]{% 
	  \tt {\tiny [GM$|$GM]}  
	 end game 
	  } 
	   } 
	   } 
	 & & \\ 
 

\end{supertabular}
}

\end{document}
