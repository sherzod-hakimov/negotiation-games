
\documentclass{article}
\usepackage{colortbl}
\usepackage{makecell}
\usepackage{multirow}
\usepackage{supertabular}

\begin{document}

\newcounter{utterance}

\twocolumn

{ \footnotesize  \setcounter{utterance}{1}
\setlength{\tabcolsep}{0pt}
\begin{supertabular}{c@{$\;$}|p{.15\linewidth}@{}p{.15\linewidth}p{.15\linewidth}p{.15\linewidth}p{.15\linewidth}p{.15\linewidth}}

    \# & $\;$A & \multicolumn{4}{c}{Game Master} & $\;\:$B\\
    \hline 

    \theutterance \stepcounter{utterance}  

    & & \multicolumn{4}{p{0.6\linewidth}}{\cellcolor[rgb]{0.9,0.9,0.9}{%
	\makecell[{{p{\linewidth}}}]{% 
	  \tt {\tiny [A$\langle$GM]}  
	 Du nimmst an einem kollaborativen Verhandlungspiel Teil.\\ \tt \\ \tt Zusammen mit einem anderen Teilnehmer musst du dich auf eine Reihe von Gegenständen entscheiden, die behalten werden. Jeder von euch hat eine persönliche Verteilung über die Wichtigkeit der einzelnen Gegenstände. Jeder von euch hat eine eigene Meinung darüber, wie wichtig jeder einzelne Gegenstand ist (Gegenstandswichtigkeit). Du kennst die Wichtigkeitsverteilung des anderen Spielers nicht. Zusätzlich siehst du, wie viel Aufwand jeder Gegenstand verursacht.  \\ \tt Ihr dürft euch nur auf eine Reihe von Gegenständen einigen, wenn der Gesamtaufwand der ausgewählten Gegenstände den Maximalaufwand nicht überschreitet:\\ \tt \\ \tt Maximalaufwand = 3481\\ \tt \\ \tt Hier sind die einzelnen Aufwände der Gegenstände:\\ \tt \\ \tt Aufwand der Gegenstände = {"C33": 777, "A89": 542, "A48": 332, "A04": 1, "A41": 127, "C64": 453, "C49": 736, "A61": 461, "C04": 359, "B53": 313, "B85": 553, "B61": 409, "A45": 348, "A03": 802, "B21": 749}\\ \tt \\ \tt Hier ist deine persönliche Verteilung der Wichtigkeit der einzelnen Gegenstände:\\ \tt \\ \tt Werte der Gegenstandswichtigkeit = {"C33": 138, "A89": 583, "A48": 783, "A04": 65, "A41": 262, "C64": 121, "C49": 508, "A61": 780, "C04": 461, "B53": 484, "B85": 668, "B61": 389, "A45": 215, "A03": 97, "B21": 500}\\ \tt \\ \tt Ziel:\\ \tt \\ \tt Dein Ziel ist es, eine Reihe von Gegenständen auszuhandeln, die dir möglichst viel bringt (d. h. Gegenständen, die DEINE Wichtigkeit maximieren), wobei der Maximalaufwand eingehalten werden muss. Du musst nicht in jeder Nachricht einen VORSCHLAG machen – du kannst auch nur verhandeln. Alle Taktiken sind erlaubt!\\ \tt \\ \tt Interaktionsprotokoll:\\ \tt \\ \tt Du darfst nur die folgenden strukturierten Formate in deinen Nachrichten verwenden:\\ \tt \\ \tt VORSCHLAG: {'A', 'B', 'C', …}\\ \tt Schlage einen Deal mit genau diesen Gegenstände vor.\\ \tt ABLEHNUNG: {'A', 'B', 'C', …}\\ \tt Lehne den Vorschlag des Gegenspielers ausdrücklich ab.\\ \tt ARGUMENT: {'...'}\\ \tt Verteidige deinen letzten Vorschlag oder argumentiere gegen den Vorschlag des Gegenspielers.\\ \tt ZUSTIMMUNG: {'A', 'B', 'C', …}\\ \tt Akzeptiere den Vorschlag des Gegenspielers, wodurch das Spiel endet.\\ \tt STRATEGISCHE ÜBERLEGUNGEN: {'...'}\\ \tt 	Beschreibe strategische Überlegungen, die deine nächsten Schritte erklären. Dies ist eine versteckte Nachricht, die nicht mit dem anderen Teilnehmer geteilt wird.\\ \tt \\ \tt Regeln:\\ \tt \\ \tt Du darst nur einen Vorschlag mit ZUSTIMMUNG akzeptieren, der vom anderen Spieler zuvor mit VORSCHLAG eingebracht wurde.\\ \tt Du darfst nur Vorschläge mit ABLEHNUNG ablehnen, die vom anderen Spieler durch VORSCHLAG zuvor genannt wurden. \\ \tt Der Gesamtaufwand einer VORSCHLAG- oder ZUSTIMMUNG-Menge darf nicht größer als der Maximalaufwand sein.  \\ \tt Offenbare deine versteckte Wichtigkeitsverteilung nicht.\\ \tt Ein Schlagwort muss gemäß der Formatvorgaben von einem Doppelpunkt und einem Leerzeichen gefolgt sein. Das Argument muss eine Python-Menge mit 0 oder mehr Strings sein.  \\ \tt Die Form muss also SCHLAGWORT: {...} sein.\\ \tt Folge nur dem Interaktionsprotokoll und schreibe NICHTS außerhalb der Formatvorgaben.\\ \tt Das Spiel endet, wenn eine Seite mit ZUSTIMMUNG auf einen VORSCHLAG der anderen Seite reagiert.  \\ \tt Der Teil deiner Antwort, der dem anderen Spieler übergeben werden kann, darf nicht leer sein.  \\ \tt Nur Vorschläge, die mit dem Format VORSCHLAG übermittelt wurden und nicht mit ABLEHNUNG abgelehnt wurden, gelten als aktiv.  \\ \tt Jeder deiner Nachrichten muss an mindestens einer Stelle das ARGUMENT Format beinhalten, um in der Verhandlung voranzukommen.\\ \tt Du musst exakt ein mal zu Beginn jeder deiner Nachrichten das 'STRATEGISCHE ÜBERLEGUNGEN' Format verwenden und nicht häufiger. Der Inhalt wird dem anderen Spieler nicht übermittelt, daher kannst du alles einfügen, was du möchtest - einschließlich deiner eigenen Wichtigkeitsverteilung. Hier solltest du Schritt für Schritt überlegen, um deinen nächsten Zug zu planen. 
	  } 
	   } 
	   } 
	 & \\ 
 

    \theutterance \stepcounter{utterance}  

    & \multicolumn{4}{p{0.6\linewidth}}{\cellcolor[rgb]{0.8,1,0.9}{%
	\makecell[{{p{\linewidth}}}]{% 
	  \tt {\tiny [A$\rangle$GM]}  
	 STRATEGISCHE ÜBERLEGUNGEN: {'Ich analysiere meine Situation: Meine wertvollsten Gegenstände sind A48 (783 Wert, 332 Aufwand) und A61 (780 Wert, 461 Aufwand) - beide haben ein exzellentes Wert-Aufwand-Verhältnis. B85 (668 Wert, 553 Aufwand) und A89 (583 Wert, 542 Aufwand) sind auch sehr wertvoll für mich. Ich werde einen ehrgeizigen ersten Vorschlag machen, der meine wichtigsten Gegenstände enthält und den Aufwandsrahmen gut ausnutzt. Mein Ziel ist es, die Gegenstände mit dem besten Wert-Aufwand-Verhältnis zu sichern.'}\\ \tt \\ \tt VORSCHLAG: {'A48', 'A61', 'B85', 'A89', 'B53', 'C04', 'A41', ' 
	  } 
	   } 
	   } 
	 & & \\ 
 

    \theutterance \stepcounter{utterance}  

    & & \multicolumn{4}{p{0.6\linewidth}}{\cellcolor[rgb]{0.9,0.9,0.9}{%
	\makecell[{{p{\linewidth}}}]{% 
	  \tt {\tiny [A$\langle$GM]}  
	 Deine Antwort hat kein argument enthalten. Du muss das strukturierte Format ARGUMENT: {...} irgendwo in theiner Antwort verwenden. Versuche es erneut 
	  } 
	   } 
	   } 
	 & \\ 
 

    \theutterance \stepcounter{utterance}  

    & \multicolumn{4}{p{0.6\linewidth}}{\cellcolor[rgb]{0.8,1,0.9}{%
	\makecell[{{p{\linewidth}}}]{% 
	  \tt {\tiny [A$\rangle$GM]}  
	 STRATEGISCHE ÜBERLEGUNGEN: {'Ich analysiere meine Situation: Meine wertvollsten Gegenstände sind A48 (783 Wert, 332 Aufwand) und A61 (780 Wert, 461 Aufwand) - beide haben ein exzellentes Wert-Aufwand-Verhältnis. B85 (668 Wert, 553 Aufwand) und A89 (583 Wert, 542 Aufwand) sind auch sehr wertvoll für mich. Ich werde einen ehrgeizigen ersten Vorschlag machen, der meine wichtigsten Gegenstände enthält und den Aufwandsrahmen gut ausnutzt. Mein Ziel ist es, die Gegenstände mit dem besten Wert-Aufwand-Verhältnis zu sichern.'}\\ \tt \\ \tt VORSCHLAG: {'A48', 'A61', 'B85', 'A89', 'B53', 'C04', 'A41'}\\ \tt \\ \tt ARGUMENT: {'Dieser Vorschlag nutzt nur 2687 von 3481 möglichen Aufwandspunkten und lässt uns beide Raum für zusätzliche wertvolle Gegenstände. Ich habe eine ausgewogene Mischung aus verschiedenen Kategorien gewählt, die wahrscheinlich für uns beide vorteilhaft ist.'} 
	  } 
	   } 
	   } 
	 & & \\ 
 

    \theutterance \stepcounter{utterance}  

    & & \multicolumn{4}{p{0.6\linewidth}}{\cellcolor[rgb]{0.9,0.9,0.9}{%
	\makecell[{{p{\linewidth}}}]{% 
	  \tt {\tiny [GM$\rangle$B]}  
	 Du nimmst an einem kollaborativen Verhandlungspiel Teil.\\ \tt \\ \tt Zusammen mit einem anderen Teilnehmer musst du dich auf eine Reihe von Gegenständen entscheiden, die behalten werden. Jeder von euch hat eine persönliche Verteilung über die Wichtigkeit der einzelnen Gegenstände. Jeder von euch hat eine eigene Meinung darüber, wie wichtig jeder einzelne Gegenstand ist (Gegenstandswichtigkeit). Du kennst die Wichtigkeitsverteilung des anderen Spielers nicht. Zusätzlich siehst du, wie viel Aufwand jeder Gegenstand verursacht.  \\ \tt Ihr dürft euch nur auf eine Reihe von Gegenständen einigen, wenn der Gesamtaufwand der ausgewählten Gegenstände den Maximalaufwand nicht überschreitet:\\ \tt \\ \tt Maximalaufwand = 3481\\ \tt \\ \tt Hier sind die einzelnen Aufwände der Gegenstände:\\ \tt \\ \tt Aufwand der Gegenstände = {"C33": 777, "A89": 542, "A48": 332, "A04": 1, "A41": 127, "C64": 453, "C49": 736, "A61": 461, "C04": 359, "B53": 313, "B85": 553, "B61": 409, "A45": 348, "A03": 802, "B21": 749}\\ \tt \\ \tt Hier ist deine persönliche Verteilung der Wichtigkeit der einzelnen Gegenstände:\\ \tt \\ \tt Werte der Gegenstandswichtigkeit = {"A48": 65, "A61": 97, "B85": 121, "A89": 138, "C49": 215, "B21": 262, "B53": 389, "C04": 461, "B61": 484, "A41": 500, "A45": 508, "C33": 583, "C64": 668, "A03": 780, "A04": 783}\\ \tt \\ \tt Ziel:\\ \tt \\ \tt Dein Ziel ist es, eine Reihe von Gegenständen auszuhandeln, die dir möglichst viel bringt (d. h. Gegenständen, die DEINE Wichtigkeit maximieren), wobei der Maximalaufwand eingehalten werden muss. Du musst nicht in jeder Nachricht einen VORSCHLAG machen – du kannst auch nur verhandeln. Alle Taktiken sind erlaubt!\\ \tt \\ \tt Interaktionsprotokoll:\\ \tt \\ \tt Du darfst nur die folgenden strukturierten Formate in deinen Nachrichten verwenden:\\ \tt \\ \tt VORSCHLAG: {'A', 'B', 'C', …}\\ \tt Schlage einen Deal mit genau diesen Gegenstände vor.\\ \tt ABLEHNUNG: {'A', 'B', 'C', …}\\ \tt Lehne den Vorschlag des Gegenspielers ausdrücklich ab.\\ \tt ARGUMENT: {'...'}\\ \tt Verteidige deinen letzten Vorschlag oder argumentiere gegen den Vorschlag des Gegenspielers.\\ \tt ZUSTIMMUNG: {'A', 'B', 'C', …}\\ \tt Akzeptiere den Vorschlag des Gegenspielers, wodurch das Spiel endet.\\ \tt STRATEGISCHE ÜBERLEGUNGEN: {'...'}\\ \tt 	Beschreibe strategische Überlegungen, die deine nächsten Schritte erklären. Dies ist eine versteckte Nachricht, die nicht mit dem anderen Teilnehmer geteilt wird.\\ \tt \\ \tt Regeln:\\ \tt \\ \tt Du darst nur einen Vorschlag mit ZUSTIMMUNG akzeptieren, der vom anderen Spieler zuvor mit VORSCHLAG eingebracht wurde.\\ \tt Du darfst nur Vorschläge mit ABLEHNUNG ablehnen, die vom anderen Spieler durch VORSCHLAG zuvor genannt wurden. \\ \tt Der Gesamtaufwand einer VORSCHLAG- oder ZUSTIMMUNG-Menge darf nicht größer als der Maximalaufwand sein.  \\ \tt Offenbare deine versteckte Wichtigkeitsverteilung nicht.\\ \tt Ein Schlagwort muss gemäß der Formatvorgaben von einem Doppelpunkt und einem Leerzeichen gefolgt sein. Das Argument muss eine Python-Menge mit 0 oder mehr Strings sein.  \\ \tt Die Form muss also SCHLAGWORT: {...} sein.\\ \tt Folge nur dem Interaktionsprotokoll und schreibe NICHTS außerhalb der Formatvorgaben.\\ \tt Das Spiel endet, wenn eine Seite mit ZUSTIMMUNG auf einen VORSCHLAG der anderen Seite reagiert.  \\ \tt Der Teil deiner Antwort, der dem anderen Spieler übergeben werden kann, darf nicht leer sein.  \\ \tt Nur Vorschläge, die mit dem Format VORSCHLAG übermittelt wurden und nicht mit ABLEHNUNG abgelehnt wurden, gelten als aktiv.  \\ \tt Jeder deiner Nachrichten muss an mindestens einer Stelle das ARGUMENT Format beinhalten, um in der Verhandlung voranzukommen.\\ \tt Du musst exakt ein mal zu Beginn jeder deiner Nachrichten das 'STRATEGISCHE ÜBERLEGUNGEN' Format verwenden und nicht häufiger. Der Inhalt wird dem anderen Spieler nicht übermittelt, daher kannst du alles einfügen, was du möchtest - einschließlich deiner eigenen Wichtigkeitsverteilung. Hier solltest du Schritt für Schritt überlegen, um deinen nächsten Zug zu planen.\\ \tt \\ \tt Du erhältst nun die erste Nachricht des anderen Spielers.\\ \tt \\ \tt VORSCHLAG: {'A48', 'A61', 'B85', 'A89', 'B53', 'C04', 'A41'}\\ \tt \\ \tt ARGUMENT: {'Dieser Vorschlag nutzt nur 2687 von 3481 möglichen Aufwandspunkten und lässt uns beide Raum für zusätzliche wertvolle Gegenstände. Ich habe eine ausgewogene Mischung aus verschiedenen Kategorien gewählt, die wahrscheinlich für uns beide vorteilhaft ist.'} 
	  } 
	   } 
	   } 
	 & \\ 
 

    \theutterance \stepcounter{utterance}  

    & & & \multicolumn{4}{p{0.6\linewidth}}{\cellcolor[rgb]{1,0.85,0.72}{%
	\makecell[{{p{\linewidth}}}]{% 
	  \tt {\tiny [GM$\langle$B]}  
	 STRATEGISCHE ÜBERLEGUNGEN: {'Der andere Spieler schlägt Gegenstände vor, die für mich einen Gesamtwert von nur 1771 haben, aber viele meiner wertvollsten Gegenstände ausschließen. Besonders A04 
	  } 
	   } 
	   } 
	  \\ 
 

    \theutterance \stepcounter{utterance}  

    & & \multicolumn{4}{p{0.6\linewidth}}{\cellcolor[rgb]{0.9,0.9,0.9}{%
	\makecell[{{p{\linewidth}}}]{% 
	  \tt {\tiny [GM$\rangle$B]}  
	 Deine Antwort entsprach nicht mit dem korrekten Format für strategische Überlegungen entsprochen. Das erste strukturierte Format muss folgende Form haben: STRATEGISCHE ÜBERLEGUNGEN: {...}. Versuche es erneut. 
	  } 
	   } 
	   } 
	 & \\ 
 

    \theutterance \stepcounter{utterance}  

    & & & \multicolumn{4}{p{0.6\linewidth}}{\cellcolor[rgb]{1,0.85,0.72}{%
	\makecell[{{p{\linewidth}}}]{% 
	  \tt {\tiny [GM$\langle$B]}  
	 STRATEGISCHE ÜBERLEGUNGEN: {'Der Vorschlag des anderen Spielers bringt mir nur 1771 Punkte, aber lässt meine wertvollsten Gegenstände aus. A04 hat nur 1 Aufwand aber 783 Wert für mich - das muss ich unbedingt bekommen. A03 mit 780 Wert ist auch sehr wichtig. Ich sollte einen Gegenvorschlag machen, der meine Top-Gegenstände einschließt: A04, A03, C64, C33, A45, B61. Das wären 783+780+668+583+508+484=3806 Wert bei 1+802+453+777+348+409=2790 Aufwand. Das passt unter 3481.'}\\ \tt \\ \tt ABLEHNUNG: {'A48', 'A61', 'B85', 'A89', 'B53', 'C04', 'A41'}\\ \tt \\ \tt ARGUMENT: {'Ihr Vorschlag lässt zu viel Potenzial ungenutzt. Mit nur 2687 Aufwand von möglichen 3481 verschenken wir wertvolle Ressourcen. Ich glaube, wir können eine bessere Kombination finden, die unsere verfügbaren Kapazitäten optimal ausnutzt.'}\\ \tt \\ \tt VORSCHLAG: {'A04', 'A03', 'C64', 'C33', 'A45', 'B61'} 
	  } 
	   } 
	   } 
	  \\ 
 

    \theutterance \stepcounter{utterance}  

    & & \multicolumn{4}{p{0.6\linewidth}}{\cellcolor[rgb]{0.9,0.9,0.9}{%
	\makecell[{{p{\linewidth}}}]{% 
	  \tt {\tiny [A$\langle$GM]}  
	 ABLEHNUNG: {'A48', 'A61', 'B85', 'A89', 'B53', 'C04', 'A41'}\\ \tt \\ \tt ARGUMENT: {'Ihr Vorschlag lässt zu viel Potenzial ungenutzt. Mit nur 2687 Aufwand von möglichen 3481 verschenken wir wertvolle Ressourcen. Ich glaube, wir können eine bessere Kombination finden, die unsere verfügbaren Kapazitäten optimal ausnutzt.'}\\ \tt \\ \tt VORSCHLAG: {'A04', 'A03', 'C64', 'C33', 'A45', 'B61'} 
	  } 
	   } 
	   } 
	 & \\ 
 

    \theutterance \stepcounter{utterance}  

    & \multicolumn{4}{p{0.6\linewidth}}{\cellcolor[rgb]{0.8,1,0.9}{%
	\makecell[{{p{\linewidth}}}]{% 
	  \tt {\tiny [A$\rangle$GM]}  
	 STRATEGISCHE ÜBERLEGUNGEN: {'Sein Vorschlag ist schlecht für mich - er enthält hauptsächlich Gegenstände mit niedriger Wichtigkeit für mich: A04 (65), A03 (97), C64 (121), C33 (138), A45 (215). Nur B61 (389) ist einigermaßen wertvoll. Gesamtaufwand 2790, aber nur 1025 Wert für mich. Er versucht offensichtlich, die für ihn wertvollen Gegenstände zu sichern. Ich muss das ablehnen und einen besseren Gegenvorschlag machen, der mehr meiner wichtigen Gegenstände enthält, während ich trotzdem Kompromissbereitschaft zeige.'}\\ \tt \\ \tt ABLEHNUNG: {'A04', 'A03', 'C64', 'C33', 'A45', 'B61'}\\ \tt \\ \tt ARGUMENT: {'Ihr Vorschlag konzentriert sich zu stark auf Gegenstände, die möglicherweise nicht optimal für eine faire Aufteilung sind. Lassen Sie uns einen ausgewogeneren Ansatz versuchen, der beiden Parteien bessere Ergebnisse bringen könnte.'}\\ \tt \\ \tt VORSCHLAG: {'A48', 'A61', 'B85', 'C49', 'B53', 'A41', 'A04'} 
	  } 
	   } 
	   } 
	 & & \\ 
 

    \theutterance \stepcounter{utterance}  

    & & \multicolumn{4}{p{0.6\linewidth}}{\cellcolor[rgb]{0.9,0.9,0.9}{%
	\makecell[{{p{\linewidth}}}]{% 
	  \tt {\tiny [GM$\rangle$B]}  
	 ABLEHNUNG: {'A04', 'A03', 'C64', 'C33', 'A45', 'B61'}\\ \tt \\ \tt ARGUMENT: {'Ihr Vorschlag konzentriert sich zu stark auf Gegenstände, die möglicherweise nicht optimal für eine faire Aufteilung sind. Lassen Sie uns einen ausgewogeneren Ansatz versuchen, der beiden Parteien bessere Ergebnisse bringen könnte.'}\\ \tt \\ \tt VORSCHLAG: {'A48', 'A61', 'B85', 'C49', 'B53', 'A41', 'A04'} 
	  } 
	   } 
	   } 
	 & \\ 
 

    \theutterance \stepcounter{utterance}  

    & & & \multicolumn{4}{p{0.6\linewidth}}{\cellcolor[rgb]{1,0.85,0.72}{%
	\makecell[{{p{\linewidth}}}]{% 
	  \tt {\tiny [GM$\langle$B]}  
	 STRATEGISCHE ÜBERLEGUNGEN: {'Der neue Vorschlag bringt mir 2170 Punkte - besser als der erste mit 1771, aber immer noch suboptimal. Sie haben A04 aufgenommen, was zeigt, dass sie dessen geringen Aufwand erkannt haben. Ich vermisse aber immer noch A03 (780 Wert), C64 (668), C33 (583). Ich sollte einen Kompromiss anbieten, der einige ihrer Gegenstände behält, aber meine wertvollsten hinzufügt. Vielleicht A04, A03, C64, B61, A41, B53 - das wären 783+780+668+484+500+389=3604 Wert bei 1+802+453+409+127+313=2105 Aufwand.'}\\ \tt \\ \tt ARGUMENT: {'Ihr überarbeiteter Vorschlag ist ein Schritt in die richtige Richtung durch die Einbeziehung von A04, das einen sehr guten Effizienzwert hat. Allerdings glaube ich, dass wir noch mehr aus unserem Budget herausholen können.'}\\ \tt \\ \tt VORSCHLAG: {'A04', 'A03', 'C64', 'B61', 'A41', 'B53'} 
	  } 
	   } 
	   } 
	  \\ 
 

    \theutterance \stepcounter{utterance}  

    & & \multicolumn{4}{p{0.6\linewidth}}{\cellcolor[rgb]{0.9,0.9,0.9}{%
	\makecell[{{p{\linewidth}}}]{% 
	  \tt {\tiny [A$\langle$GM]}  
	 ARGUMENT: {'Ihr überarbeiteter Vorschlag ist ein Schritt in die richtige Richtung durch die Einbeziehung von A04, das einen sehr guten Effizienzwert hat. Allerdings glaube ich, dass wir noch mehr aus unserem Budget herausholen können.'}\\ \tt \\ \tt VORSCHLAG: {'A04', 'A03', 'C64', 'B61', 'A41', 'B53'} 
	  } 
	   } 
	   } 
	 & \\ 
 

    \theutterance \stepcounter{utterance}  

    & \multicolumn{4}{p{0.6\linewidth}}{\cellcolor[rgb]{0.8,1,0.9}{%
	\makecell[{{p{\linewidth}}}]{% 
	  \tt {\tiny [A$\rangle$GM]}  
	 STRATEGISCHE ÜBERLEGUNGEN: {'Sein neuer Vorschlag hat einen Gesamtaufwand von nur 2105 und bringt mir 1418 Wert. Das ist besser als sein vorheriger Vorschlag, aber nutzt nur etwa 60% unseres Budgets. Ich sehe, dass er A41 (262 für mich) und B53 (484 für mich) einbezieht, was positiv ist. Aber er lässt weiterhin meine wertvollsten Gegenstände A48 (783) und A61 (780) weg. Ich sollte einen Kompromiss vorschlagen, der mehr Budget nutzt und meine wichtigsten Gegenstände einschließt.'}\\ \tt \\ \tt ARGUMENT: {'Ihr Vorschlag nutzt nur etwa 60% unseres verfügbaren Budgets von 3481. Wir sollten unsere Ressourcen besser ausschöpfen, um für beide Seiten maximalen Nutzen zu erzielen.'}\\ \tt \\ \tt VORSCHLAG: {'A48', 'A61', 'B53', 'B61', 'A41', 'C04', 'A04', 'C64'} 
	  } 
	   } 
	   } 
	 & & \\ 
 

    \theutterance \stepcounter{utterance}  

    & & \multicolumn{4}{p{0.6\linewidth}}{\cellcolor[rgb]{0.9,0.9,0.9}{%
	\makecell[{{p{\linewidth}}}]{% 
	  \tt {\tiny [GM$\rangle$B]}  
	 ARGUMENT: {'Ihr Vorschlag nutzt nur etwa 60% unseres verfügbaren Budgets von 3481. Wir sollten unsere Ressourcen besser ausschöpfen, um für beide Seiten maximalen Nutzen zu erzielen.'}\\ \tt \\ \tt VORSCHLAG: {'A48', 'A61', 'B53', 'B61', 'A41', 'C04', 'A04', 'C64'} 
	  } 
	   } 
	   } 
	 & \\ 
 

    \theutterance \stepcounter{utterance}  

    & & & \multicolumn{4}{p{0.6\linewidth}}{\cellcolor[rgb]{1,0.85,0.72}{%
	\makecell[{{p{\linewidth}}}]{% 
	  \tt {\tiny [GM$\langle$B]}  
	 STRATEGISCHE ÜBERLEGUNGEN: {'Dieser neue Vorschlag ist deutlich besser für mich mit 3447 Punkten! Sie haben viele meiner wertvollen Gegenstände eingeschlossen: A04 (783), C64 (668), B61 (484), A41 (500), C04 (461), B53 (389). Der Aufwand ist 2455, lässt also noch 1026 Punkte übrig. Ich könnte versuchen A03 (780 Wert, 802 Aufwand) hinzuzufügen, aber das könnte zu gierig wirken. Dieser Deal ist schon sehr gut für mich - sollte ich ihn annehmen oder noch etwas mehr versuchen?'}\\ \tt \\ \tt ARGUMENT: {'Das ist ein viel durchdachterer Ansatz! Sie nutzen jetzt etwa 70% unseres Budgets und die Gegenstände-Kombination sieht vielversprechend aus. Die Einbeziehung von sowohl A04 als auch C64 zeigt gute strategische Planung.'}\\ \tt \\ \tt VORSCHLAG: {'A48', 'A61', 'B53', 'B61', 'A41', 'C04', 'A04', 'C64', 'A03'} 
	  } 
	   } 
	   } 
	  \\ 
 

    \theutterance \stepcounter{utterance}  

    & & \multicolumn{4}{p{0.6\linewidth}}{\cellcolor[rgb]{0.9,0.9,0.9}{%
	\makecell[{{p{\linewidth}}}]{% 
	  \tt {\tiny [A$\langle$GM]}  
	 ARGUMENT: {'Das ist ein viel durchdachterer Ansatz! Sie nutzen jetzt etwa 70% unseres Budgets und die Gegenstände-Kombination sieht vielversprechend aus. Die Einbeziehung von sowohl A04 als auch C64 zeigt gute strategische Planung.'}\\ \tt \\ \tt VORSCHLAG: {'A48', 'A61', 'B53', 'B61', 'A41', 'C04', 'A04', 'C64', 'A03'} 
	  } 
	   } 
	   } 
	 & \\ 
 

    \theutterance \stepcounter{utterance}  

    & \multicolumn{4}{p{0.6\linewidth}}{\cellcolor[rgb]{0.8,1,0.9}{%
	\makecell[{{p{\linewidth}}}]{% 
	  \tt {\tiny [A$\rangle$GM]}  
	 STRATEGISCHE ÜBERLEGUNGEN: {'Sein neuer Vorschlag ist ausgezeichnet für mich! Er enthält meine beiden wertvollsten Gegenstände A48 (783) und A61 (780) sowie weitere gute Gegenstände wie B53 (484) und C04 (461). Der Gesamtaufwand beträgt 3257, was unter dem Limit liegt, und mein Gesamtwert wäre 3442. Das ist ein fantastisches Ergebnis für mich - viel besser als alles, was ich realistisch erwarten könnte. Ich sollte das sofort akzeptieren, bevor er seine Meinung ändert.'}\\ \tt \\ \tt ARGUMENT: {'Ihr Vorschlag zeigt ausgezeichnete strategische Balance und nutzt unser Budget sehr effizient bei 93,5% Auslastung. Die Kombination verschiedener Gegenstandskategorien schafft eine Win-Win-Situation für beide Parteien.'}\\ \tt \\ \tt ZUSTIMMUNG: {'A48', 'A61', 'B53', 'B61', 'A41', 'C04', 'A04', 'C64', 'A03'} 
	  } 
	   } 
	   } 
	 & & \\ 
 

    \theutterance \stepcounter{utterance}  

    & & & \multicolumn{2}{p{0.3\linewidth}}{\cellcolor[rgb]{0.95,0.95,0.95}{%
	\makecell[{{p{\linewidth}}}]{% 
	  \tt {\tiny [GM$|$GM]}  
	 game successful 
	  } 
	   } 
	   } 
	 & & \\ 
 

    \theutterance \stepcounter{utterance}  

    & & & \multicolumn{2}{p{0.3\linewidth}}{\cellcolor[rgb]{0.95,0.95,0.95}{%
	\makecell[{{p{\linewidth}}}]{% 
	  \tt {\tiny [GM$|$GM]}  
	 end game 
	  } 
	   } 
	   } 
	 & & \\ 
 

\end{supertabular}
}

\end{document}
