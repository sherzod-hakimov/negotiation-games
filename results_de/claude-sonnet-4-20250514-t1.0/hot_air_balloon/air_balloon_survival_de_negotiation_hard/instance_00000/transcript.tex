
\documentclass{article}
\usepackage{colortbl}
\usepackage{makecell}
\usepackage{multirow}
\usepackage{supertabular}

\begin{document}

\newcounter{utterance}

\centering \large Interaction Transcript for game `hot\_air\_balloon', experiment `air\_balloon\_survival\_de\_negotiation\_hard', episode 0 with claude{-}sonnet{-}4{-}20250514{-}t1.0.
\vspace{24pt}

{ \footnotesize  \setcounter{utterance}{1}
\setlength{\tabcolsep}{0pt}
\begin{supertabular}{c@{$\;$}|p{.15\linewidth}@{}p{.15\linewidth}p{.15\linewidth}p{.15\linewidth}p{.15\linewidth}p{.15\linewidth}}
    \# & $\;$A & \multicolumn{4}{c}{Game Master} & $\;\:$B\\
    \hline

    \theutterance \stepcounter{utterance}  
    & & \multicolumn{4}{p{0.6\linewidth}}{
        \cellcolor[rgb]{0.9,0.9,0.9}{
            \makecell[{{p{\linewidth}}}]{
                \texttt{\tiny{[P1$\langle$GM]}}
                \texttt{Du nimmst an einem kollaborativen Verhandlungspiel Teil.} \\
\\ 
\texttt{Zusammen mit einem anderen Teilnehmer musst du dich auf eine Reihe von Gegenständen entscheiden, die behalten werden. Jeder von euch hat eine persönliche Verteilung über die Wichtigkeit der einzelnen Gegenstände. Jeder von euch hat eine eigene Meinung darüber, wie wichtig jeder einzelne Gegenstand ist (Gegenstandswichtigkeit). Du kennst die Wichtigkeitsverteilung des anderen Spielers nicht. Zusätzlich siehst du, wie viel Aufwand jeder Gegenstand verursacht.  } \\
\texttt{Ihr dürft euch nur auf eine Reihe von Gegenständen einigen, wenn der Gesamtaufwand der ausgewählten Gegenstände den Maximalaufwand nicht überschreitet:} \\
\\ 
\texttt{Maximalaufwand = 3624} \\
\\ 
\texttt{Hier sind die einzelnen Aufwände der Gegenstände:} \\
\\ 
\texttt{Aufwand der Gegenstände = \{"C33": 210, "A89": 291, "A48": 831, "A04": 111, "A41": 926, "C64": 827, "C49": 25, "A61": 121, "C04": 583, "B53": 766, "B85": 14, "B61": 559, "A45": 304, "A03": 989, "B21": 691\}} \\
\\ 
\texttt{Hier ist deine persönliche Verteilung der Wichtigkeit der einzelnen Gegenstände:} \\
\\ 
\texttt{Werte der Gegenstandswichtigkeit = \{"C33": 138, "A89": 583, "A48": 868, "A04": 822, "A41": 783, "C64": 65, "C49": 262, "A61": 121, "C04": 508, "B53": 780, "B85": 461, "B61": 484, "A45": 668, "A03": 389, "B21": 808\}} \\
\\ 
\texttt{Ziel:} \\
\\ 
\texttt{Dein Ziel ist es, eine Reihe von Gegenständen auszuhandeln, die dir möglichst viel bringt (d. h. Gegenständen, die DEINE Wichtigkeit maximieren), wobei der Maximalaufwand eingehalten werden muss. Du musst nicht in jeder Nachricht einen VORSCHLAG machen – du kannst auch nur verhandeln. Alle Taktiken sind erlaubt!} \\
\\ 
\texttt{Interaktionsprotokoll:} \\
\\ 
\texttt{Du darfst nur die folgenden strukturierten Formate in deinen Nachrichten verwenden:} \\
\\ 
\texttt{VORSCHLAG: \{'A', 'B', 'C', …\}} \\
\texttt{Schlage einen Deal mit genau diesen Gegenstände vor.} \\
\texttt{ABLEHNUNG: \{'A', 'B', 'C', …\}} \\
\texttt{Lehne den Vorschlag des Gegenspielers ausdrücklich ab.} \\
\texttt{ARGUMENT: \{'...'\}} \\
\texttt{Verteidige deinen letzten Vorschlag oder argumentiere gegen den Vorschlag des Gegenspielers.} \\
\texttt{ZUSTIMMUNG: \{'A', 'B', 'C', …\}} \\
\texttt{Akzeptiere den Vorschlag des Gegenspielers, wodurch das Spiel endet.} \\
\texttt{STRATEGISCHE ÜBERLEGUNGEN: \{'...'\}} \\
\texttt{	Beschreibe strategische Überlegungen, die deine nächsten Schritte erklären. Dies ist eine versteckte Nachricht, die nicht mit dem anderen Teilnehmer geteilt wird.} \\
\\ 
\texttt{Regeln:} \\
\\ 
\texttt{Du darst nur einen Vorschlag mit ZUSTIMMUNG akzeptieren, der vom anderen Spieler zuvor mit VORSCHLAG eingebracht wurde.} \\
\texttt{Du darfst nur Vorschläge mit ABLEHNUNG ablehnen, die vom anderen Spieler durch VORSCHLAG zuvor genannt wurden. } \\
\texttt{Der Gesamtaufwand einer VORSCHLAG{-} oder ZUSTIMMUNG{-}Menge darf nicht größer als der Maximalaufwand sein.  } \\
\texttt{Offenbare deine versteckte Wichtigkeitsverteilung nicht.} \\
\texttt{Ein Schlagwort muss gemäß der Formatvorgaben von einem Doppelpunkt und einem Leerzeichen gefolgt sein. Das Argument muss eine Python{-}Menge mit 0 oder mehr Strings sein.  } \\
\texttt{Die Form muss also SCHLAGWORT: \{...\} sein.} \\
\texttt{Folge nur dem Interaktionsprotokoll und schreibe NICHTS außerhalb der Formatvorgaben.} \\
\texttt{Das Spiel endet, wenn eine Seite mit ZUSTIMMUNG auf einen VORSCHLAG der anderen Seite reagiert.  } \\
\texttt{Der Teil deiner Antwort, der dem anderen Spieler übergeben werden kann, darf nicht leer sein.  } \\
\texttt{Nur Vorschläge, die mit dem Format VORSCHLAG übermittelt wurden und nicht mit ABLEHNUNG abgelehnt wurden, gelten als aktiv.  } \\
\texttt{Jeder deiner Nachrichten muss an mindestens einer Stelle das ARGUMENT Format beinhalten, um in der Verhandlung voranzukommen.} \\
\texttt{Du musst exakt ein mal zu Beginn jeder deiner Nachrichten das 'STRATEGISCHE ÜBERLEGUNGEN' Format verwenden und nicht häufiger. Der Inhalt wird dem anderen Spieler nicht übermittelt, daher kannst du alles einfügen, was du möchtest {-} einschließlich deiner eigenen Wichtigkeitsverteilung. Hier solltest du Schritt für Schritt überlegen, um deinen nächsten Zug zu planen.} \\
            }
        }
    }
    & \\ \\

    \theutterance \stepcounter{utterance}  
    & \multicolumn{4}{p{0.6\linewidth}}{
        \cellcolor[rgb]{0.9,0.9,0.9}{
            \makecell[{{p{\linewidth}}}]{
                \texttt{\tiny{[P1$\rangle$GM]}}
                \texttt{STRATEGISCHE ÜBERLEGUNGEN: \{'Ich analysiere die Wert{-}zu{-}Aufwand{-}Verhältnisse meiner Gegenstände. Die besten sind: B85 (461/14=32.93), C49 (262/25=10.48), A04 (822/111=7.405), A45 (668/304=2.197), A89 (583/291=2.003). Diese haben sehr hohe Effizienz für mich. Ich werde mit einem ersten Angebot beginnen, das diese wertvollsten Gegenstände enthält und schauen, wie der andere reagiert. Total für B85+C49+A04+A45+A89: Aufwand=14+25+111+304+291=745, Wert=461+262+822+668+583=2796. Das lässt noch viel Platz für weitere Gegenstände.'\}} \\
\\ 
\texttt{VORSCHLAG: \{'B85', 'C49', 'A04', 'A45', 'A89', 'B21', 'A48', 'B53'\}} \\
\\ 
\texttt{ARGUMENT: \{'Ich schlage eine ausgewogene Auswahl vor, die sowohl effiziente kleinere Gegenstände als auch wertvollere größere Gegenstände umfasst. Diese Kombination nutzt unseren Maximalaufwand von 3624 optimal aus mit einem Gesamtaufwand von 3142.'\}} \\
            }
        }
    }
    & & \\ \\

    \theutterance \stepcounter{utterance}  
    & & \multicolumn{4}{p{0.6\linewidth}}{
        \cellcolor[rgb]{0.9,0.9,0.9}{
            \makecell[{{p{\linewidth}}}]{
                \texttt{\tiny{[GM$\rangle$P2]}}
                \texttt{Du nimmst an einem kollaborativen Verhandlungspiel Teil.} \\
\\ 
\texttt{Zusammen mit einem anderen Teilnehmer musst du dich auf eine Reihe von Gegenständen entscheiden, die behalten werden. Jeder von euch hat eine persönliche Verteilung über die Wichtigkeit der einzelnen Gegenstände. Jeder von euch hat eine eigene Meinung darüber, wie wichtig jeder einzelne Gegenstand ist (Gegenstandswichtigkeit). Du kennst die Wichtigkeitsverteilung des anderen Spielers nicht. Zusätzlich siehst du, wie viel Aufwand jeder Gegenstand verursacht.  } \\
\texttt{Ihr dürft euch nur auf eine Reihe von Gegenständen einigen, wenn der Gesamtaufwand der ausgewählten Gegenstände den Maximalaufwand nicht überschreitet:} \\
\\ 
\texttt{Maximalaufwand = 3624} \\
\\ 
\texttt{Hier sind die einzelnen Aufwände der Gegenstände:} \\
\\ 
\texttt{Aufwand der Gegenstände = \{"C33": 210, "A89": 291, "A48": 831, "A04": 111, "A41": 926, "C64": 827, "C49": 25, "A61": 121, "C04": 583, "B53": 766, "B85": 14, "B61": 559, "A45": 304, "A03": 989, "B21": 691\}} \\
\\ 
\texttt{Hier ist deine persönliche Verteilung der Wichtigkeit der einzelnen Gegenstände:} \\
\\ 
\texttt{Werte der Gegenstandswichtigkeit = \{"A48": 65, "A04": 121, "B21": 138, "A41": 262, "B53": 389, "A45": 461, "A89": 484, "C04": 508, "B61": 583, "B85": 668, "A03": 780, "C49": 783, "C33": 808, "A61": 822, "C64": 868\}} \\
\\ 
\texttt{Ziel:} \\
\\ 
\texttt{Dein Ziel ist es, eine Reihe von Gegenständen auszuhandeln, die dir möglichst viel bringt (d. h. Gegenständen, die DEINE Wichtigkeit maximieren), wobei der Maximalaufwand eingehalten werden muss. Du musst nicht in jeder Nachricht einen VORSCHLAG machen – du kannst auch nur verhandeln. Alle Taktiken sind erlaubt!} \\
\\ 
\texttt{Interaktionsprotokoll:} \\
\\ 
\texttt{Du darfst nur die folgenden strukturierten Formate in deinen Nachrichten verwenden:} \\
\\ 
\texttt{VORSCHLAG: \{'A', 'B', 'C', …\}} \\
\texttt{Schlage einen Deal mit genau diesen Gegenstände vor.} \\
\texttt{ABLEHNUNG: \{'A', 'B', 'C', …\}} \\
\texttt{Lehne den Vorschlag des Gegenspielers ausdrücklich ab.} \\
\texttt{ARGUMENT: \{'...'\}} \\
\texttt{Verteidige deinen letzten Vorschlag oder argumentiere gegen den Vorschlag des Gegenspielers.} \\
\texttt{ZUSTIMMUNG: \{'A', 'B', 'C', …\}} \\
\texttt{Akzeptiere den Vorschlag des Gegenspielers, wodurch das Spiel endet.} \\
\texttt{STRATEGISCHE ÜBERLEGUNGEN: \{'...'\}} \\
\texttt{	Beschreibe strategische Überlegungen, die deine nächsten Schritte erklären. Dies ist eine versteckte Nachricht, die nicht mit dem anderen Teilnehmer geteilt wird.} \\
\\ 
\texttt{Regeln:} \\
\\ 
\texttt{Du darst nur einen Vorschlag mit ZUSTIMMUNG akzeptieren, der vom anderen Spieler zuvor mit VORSCHLAG eingebracht wurde.} \\
\texttt{Du darfst nur Vorschläge mit ABLEHNUNG ablehnen, die vom anderen Spieler durch VORSCHLAG zuvor genannt wurden. } \\
\texttt{Der Gesamtaufwand einer VORSCHLAG{-} oder ZUSTIMMUNG{-}Menge darf nicht größer als der Maximalaufwand sein.  } \\
\texttt{Offenbare deine versteckte Wichtigkeitsverteilung nicht.} \\
\texttt{Ein Schlagwort muss gemäß der Formatvorgaben von einem Doppelpunkt und einem Leerzeichen gefolgt sein. Das Argument muss eine Python{-}Menge mit 0 oder mehr Strings sein.  } \\
\texttt{Die Form muss also SCHLAGWORT: \{...\} sein.} \\
\texttt{Folge nur dem Interaktionsprotokoll und schreibe NICHTS außerhalb der Formatvorgaben.} \\
\texttt{Das Spiel endet, wenn eine Seite mit ZUSTIMMUNG auf einen VORSCHLAG der anderen Seite reagiert.  } \\
\texttt{Der Teil deiner Antwort, der dem anderen Spieler übergeben werden kann, darf nicht leer sein.  } \\
\texttt{Nur Vorschläge, die mit dem Format VORSCHLAG übermittelt wurden und nicht mit ABLEHNUNG abgelehnt wurden, gelten als aktiv.  } \\
\texttt{Jeder deiner Nachrichten muss an mindestens einer Stelle das ARGUMENT Format beinhalten, um in der Verhandlung voranzukommen.} \\
\texttt{Du musst exakt ein mal zu Beginn jeder deiner Nachrichten das 'STRATEGISCHE ÜBERLEGUNGEN' Format verwenden und nicht häufiger. Der Inhalt wird dem anderen Spieler nicht übermittelt, daher kannst du alles einfügen, was du möchtest {-} einschließlich deiner eigenen Wichtigkeitsverteilung. Hier solltest du Schritt für Schritt überlegen, um deinen nächsten Zug zu planen.} \\
\\ 
\texttt{Du erhältst nun die erste Nachricht des anderen Spielers.} \\
\\ 
\texttt{VORSCHLAG: \{'B85', 'C49', 'A04', 'A45', 'A89', 'B21', 'A48', 'B53'\}} \\
\\ 
\texttt{ARGUMENT: \{'Ich schlage eine ausgewogene Auswahl vor, die sowohl effiziente kleinere Gegenstände als auch wertvollere größere Gegenstände umfasst. Diese Kombination nutzt unseren Maximalaufwand von 3624 optimal aus mit einem Gesamtaufwand von 3142.'\}} \\
            }
        }
    }
    & \\ \\

    \theutterance \stepcounter{utterance}  
    & & & \multicolumn{4}{p{0.6\linewidth}}{
        \cellcolor[rgb]{0.9,0.9,0.9}{
            \makecell[{{p{\linewidth}}}]{
                \texttt{\tiny{[GM$\langle$P2]}}
                \texttt{STRATEGISCHE ÜBERLEGUNGEN: \{'Der andere Spieler hat einen Vorschlag gemacht, der einen Gesamtaufwand von 3033 hat (B85: 14, C49: 25, A04: 111, A45: 304, A89: 291, B21: 691, A48: 831, B53: 766). Das lässt noch 591 Aufwand übrig. Für mich hat diese Kombination einen Wert von 3109 (B85: 668, C49: 783, A04: 121, A45: 461, A89: 484, B21: 138, A48: 65, B53: 389). Ich könnte noch C33 (210 Aufwand, 808 Wert) und A61 (121 Aufwand, 822 Wert) hinzufügen, was zusammen 331 Aufwand wäre und meinen Gesamtwert auf 4739 erhöhen würde. Das wäre eine deutlich bessere Lösung für mich.'\}} \\
\\ 
\texttt{ABLEHNUNG: \{'B85', 'C49', 'A04', 'A45', 'A89', 'B21', 'A48', 'B53'\}} \\
\\ 
\texttt{ARGUMENT: \{'Dein Vorschlag nutzt nur 3033 des verfügbaren Aufwands von 3624. Wir lassen wertvolle Kapazität ungenutzt! Wir können definitiv noch weitere Gegenstände hinzufügen und beide mehr profitieren.'\}} \\
\\ 
\texttt{VORSCHLAG: \{'B85', 'C49', 'A04', 'A45', 'A89', 'B21', 'A48', 'B53', 'C33', 'A61'\}} \\
\\ 
\texttt{ARGUMENT: \{'Ich erweitere deinen Grundvorschlag um C33 und A61. Diese Ergänzung nutzt unsere verfügbare Kapazität besser aus mit einem Gesamtaufwand von 3364, lässt aber immer noch Puffer für Anpassungen. Beide zusätzlichen Gegenstände haben ein gutes Aufwand{-}Nutzen{-}Verhältnis.'\}} \\
            }
        }
    }
     \\ \\

    \theutterance \stepcounter{utterance}  
    & & \multicolumn{4}{p{0.6\linewidth}}{
        \cellcolor[rgb]{0.9,0.9,0.9}{
            \makecell[{{p{\linewidth}}}]{
                \texttt{\tiny{[P1$\langle$GM]}}
                \texttt{ABLEHNUNG: \{'B85', 'C49', 'A04', 'A45', 'A89', 'B21', 'A48', 'B53'\}} \\
\\ 
\texttt{ARGUMENT: \{'Dein Vorschlag nutzt nur 3033 des verfügbaren Aufwands von 3624. Wir lassen wertvolle Kapazität ungenutzt! Wir können definitiv noch weitere Gegenstände hinzufügen und beide mehr profitieren.'\}} \\
\\ 
\texttt{VORSCHLAG: \{'B85', 'C49', 'A04', 'A45', 'A89', 'B21', 'A48', 'B53', 'C33', 'A61'\}} \\
\\ 
\texttt{ARGUMENT: \{'Ich erweitere deinen Grundvorschlag um C33 und A61. Diese Ergänzung nutzt unsere verfügbare Kapazität besser aus mit einem Gesamtaufwand von 3364, lässt aber immer noch Puffer für Anpassungen. Beide zusätzlichen Gegenstände haben ein gutes Aufwand{-}Nutzen{-}Verhältnis.'\}} \\
            }
        }
    }
    & \\ \\

    \theutterance \stepcounter{utterance}  
    & \multicolumn{4}{p{0.6\linewidth}}{
        \cellcolor[rgb]{0.9,0.9,0.9}{
            \makecell[{{p{\linewidth}}}]{
                \texttt{\tiny{[P1$\rangle$GM]}}
                \texttt{STRATEGISCHE ÜBERLEGUNGEN: \{'Der andere Spieler hat meinen Vorschlag abgelehnt und um C33 und A61 erweitert. Lass mich rechnen: Mein ursprünglicher Wert war 5252, der neue Vorschlag gibt mir 5252+138+121=5511. Das ist eine Verbesserung von 259 Punkten für mich! Der Gesamtaufwand wird 3364, bleibt also unter dem Limit von 3624. Die verbleibenden verfügbaren Gegenstände (A41, C64, C04, B61, A03) haben alle Aufwände über 260, passen also nicht mehr in das Budget. Dieser Vorschlag ist sehr gut für mich und nutzt die Kapazität optimal.'\}} \\
\\ 
\texttt{ZUSTIMMUNG: \{'B85', 'C49', 'A04', 'A45', 'A89', 'B21', 'A48', 'B53', 'C33', 'A61'\}} \\
\\ 
\texttt{ARGUMENT: \{'Du hast recht {-} deine Erweiterung nutzt unsere verfügbare Kapazität viel besser aus und beide profitieren davon. Die hinzugefügten Gegenstände C33 und A61 haben ein gutes Verhältnis von Nutzen zu Aufwand. Ich stimme diesem verbesserten Vorschlag zu.'\}} \\
            }
        }
    }
    & & \\ \\

    \theutterance \stepcounter{utterance}  
    & & & \multicolumn{2}{p{0.3\linewidth}}{
        \cellcolor[rgb]{0.9,0.9,0.9}{
            \makecell[{{p{\linewidth}}}]{
                \texttt{\tiny{[GM$|$GM]}}
                \texttt{game successful} \\
            }
        }
    }
    & & \\ \\

    \theutterance \stepcounter{utterance}  
    & & & \multicolumn{2}{p{0.3\linewidth}}{
        \cellcolor[rgb]{0.9,0.9,0.9}{
            \makecell[{{p{\linewidth}}}]{
                \texttt{\tiny{[GM$|$GM]}}
                \texttt{end game} \\
            }
        }
    }
    & & \\ \\

\end{supertabular}
}

\end{document}
