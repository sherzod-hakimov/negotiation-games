
\documentclass{article}
\usepackage{colortbl}
\usepackage{makecell}
\usepackage{multirow}
\usepackage{supertabular}

\begin{document}

\newcounter{utterance}

\twocolumn

{ \footnotesize  \setcounter{utterance}{1}
\setlength{\tabcolsep}{0pt}
\begin{supertabular}{c@{$\;$}|p{.15\linewidth}@{}p{.15\linewidth}p{.15\linewidth}p{.15\linewidth}p{.15\linewidth}p{.15\linewidth}}

    \# & $\;$A & \multicolumn{4}{c}{Game Master} & $\;\:$B\\
    \hline 

    \theutterance \stepcounter{utterance}  

    & & \multicolumn{4}{p{0.6\linewidth}}{\cellcolor[rgb]{0.9,0.9,0.9}{%
	\makecell[{{p{\linewidth}}}]{% 
	  \tt {\tiny [A$\langle$GM]}  
	 Du nimmst an einem kollaborativen Verhandlungspiel Teil.\\ \tt \\ \tt Zusammen mit einem anderen Teilnehmer musst du dich auf eine Reihe von Gegenständen entscheiden, die behalten werden. Jeder von euch hat eine persönliche Verteilung über die Wichtigkeit der einzelnen Gegenstände. Jeder von euch hat eine eigene Meinung darüber, wie wichtig jeder einzelne Gegenstand ist (Gegenstandswichtigkeit). Du kennst die Wichtigkeitsverteilung des anderen Spielers nicht. Zusätzlich siehst du, wie viel Aufwand jeder Gegenstand verursacht.  \\ \tt Ihr dürft euch nur auf eine Reihe von Gegenständen einigen, wenn der Gesamtaufwand der ausgewählten Gegenstände den Maximalaufwand nicht überschreitet:\\ \tt \\ \tt Maximalaufwand = 3386\\ \tt \\ \tt Hier sind die einzelnen Aufwände der Gegenstände:\\ \tt \\ \tt Aufwand der Gegenstände = {"C33": 313, "A89": 720, "A48": 175, "A04": 461, "A41": 635, "C64": 685, "C49": 544, "A61": 203, "C04": 369, "B53": 539, "B85": 4, "B61": 695, "A45": 399, "A03": 594, "B21": 437}\\ \tt \\ \tt Hier ist deine persönliche Verteilung der Wichtigkeit der einzelnen Gegenstände:\\ \tt \\ \tt Werte der Gegenstandswichtigkeit = {"C33": 138, "A89": 583, "A48": 65, "A04": 262, "A41": 121, "C64": 508, "C49": 461, "A61": 484, "C04": 668, "B53": 389, "B85": 215, "B61": 97, "A45": 500, "A03": 30, "B21": 400}\\ \tt \\ \tt Ziel:\\ \tt \\ \tt Dein Ziel ist es, eine Reihe von Gegenständen auszuhandeln, die dir möglichst viel bringt (d. h. Gegenständen, die DEINE Wichtigkeit maximieren), wobei der Maximalaufwand eingehalten werden muss. Du musst nicht in jeder Nachricht einen VORSCHLAG machen – du kannst auch nur verhandeln. Alle Taktiken sind erlaubt!\\ \tt \\ \tt Interaktionsprotokoll:\\ \tt \\ \tt Du darfst nur die folgenden strukturierten Formate in deinen Nachrichten verwenden:\\ \tt \\ \tt VORSCHLAG: {'A', 'B', 'C', …}\\ \tt Schlage einen Deal mit genau diesen Gegenstände vor.\\ \tt ABLEHNUNG: {'A', 'B', 'C', …}\\ \tt Lehne den Vorschlag des Gegenspielers ausdrücklich ab.\\ \tt ARGUMENT: {'...'}\\ \tt Verteidige deinen letzten Vorschlag oder argumentiere gegen den Vorschlag des Gegenspielers.\\ \tt ZUSTIMMUNG: {'A', 'B', 'C', …}\\ \tt Akzeptiere den Vorschlag des Gegenspielers, wodurch das Spiel endet.\\ \tt STRATEGISCHE ÜBERLEGUNGEN: {'...'}\\ \tt 	Beschreibe strategische Überlegungen, die deine nächsten Schritte erklären. Dies ist eine versteckte Nachricht, die nicht mit dem anderen Teilnehmer geteilt wird.\\ \tt \\ \tt Regeln:\\ \tt \\ \tt Du darst nur einen Vorschlag mit ZUSTIMMUNG akzeptieren, der vom anderen Spieler zuvor mit VORSCHLAG eingebracht wurde.\\ \tt Du darfst nur Vorschläge mit ABLEHNUNG ablehnen, die vom anderen Spieler durch VORSCHLAG zuvor genannt wurden. \\ \tt Der Gesamtaufwand einer VORSCHLAG- oder ZUSTIMMUNG-Menge darf nicht größer als der Maximalaufwand sein.  \\ \tt Offenbare deine versteckte Wichtigkeitsverteilung nicht.\\ \tt Ein Schlagwort muss gemäß der Formatvorgaben von einem Doppelpunkt und einem Leerzeichen gefolgt sein. Das Argument muss eine Python-Menge mit 0 oder mehr Strings sein.  \\ \tt Die Form muss also SCHLAGWORT: {...} sein.\\ \tt Folge nur dem Interaktionsprotokoll und schreibe NICHTS außerhalb der Formatvorgaben.\\ \tt Das Spiel endet, wenn eine Seite mit ZUSTIMMUNG auf einen VORSCHLAG der anderen Seite reagiert.  \\ \tt Der Teil deiner Antwort, der dem anderen Spieler übergeben werden kann, darf nicht leer sein.  \\ \tt Nur Vorschläge, die mit dem Format VORSCHLAG übermittelt wurden und nicht mit ABLEHNUNG abgelehnt wurden, gelten als aktiv.  \\ \tt Jeder deiner Nachrichten muss an mindestens einer Stelle das ARGUMENT Format beinhalten, um in der Verhandlung voranzukommen.\\ \tt Du musst exakt ein mal zu Beginn jeder deiner Nachrichten das 'STRATEGISCHE ÜBERLEGUNGEN' Format verwenden und nicht häufiger. Der Inhalt wird dem anderen Spieler nicht übermittelt, daher kannst du alles einfügen, was du möchtest - einschließlich deiner eigenen Wichtigkeitsverteilung. Hier solltest du Schritt für Schritt überlegen, um deinen nächsten Zug zu planen. 
	  } 
	   } 
	   } 
	 & \\ 
 

    \theutterance \stepcounter{utterance}  

    & \multicolumn{4}{p{0.6\linewidth}}{\cellcolor[rgb]{0.8,1,0.9}{%
	\makecell[{{p{\linewidth}}}]{% 
	  \tt {\tiny [A$\rangle$GM]}  
	 STRATEGISCHE ÜBERLEGUNGEN: {'Ich muss eine optimale Kombination finden, die meinen Gesamtwert maximiert, ohne den Aufwand von 3386 zu überschreiten. Meine Wichtigkeitsverteilung: C33=138, A89=583, A48=65, A04=262, A41=121, C64=508, C49=461, A61=484, C04=668, B53=389, B85=215, B61=97, A45=500, A03=30, B21=400. Die besten Wert-zu-Aufwand-Verhältnisse haben: B85 (215/4=53.75), A61 (484/203=2.38), C04 (668/369=1.81), A45 (500/399=1.25). Ich werde mit einer aggressiven Kombination beginnen, die meine wertvollsten Gegenstände einschließt: B85, A61, C04, A45, C49, B21, A89, C64. Gesamtaufwand: 4+203+369+399+544+437+720+685=3361, was unter 3386 liegt. Gesamtwert für mich: 215+484+668+500+461+400+583+508=3819.'}\\ \tt \\ \tt VORSCHLAG: {'B85', 'A61', 'C04', 'A45', 'C49', 'B21', 'A89', 'C64'}\\ \tt \\ \tt ARGUMENT: {'Diese Kombination nutzt unseren Maximalaufwand von 3386 optimal aus mit einem Gesamtaufwand von nur 3361. Sie enthält eine ausgewogene Mischung aus kostengünstigen und wertvollen Gegenständen. B85 hat einen minimalen Aufwand von nur 4 Einheiten und bietet daher ein exzellentes Kosten-Nutzen-Verhältnis. Die anderen Gegenstände ergänzen sich gut und bleiben unter unserem Budget.'} 
	  } 
	   } 
	   } 
	 & & \\ 
 

    \theutterance \stepcounter{utterance}  

    & & \multicolumn{4}{p{0.6\linewidth}}{\cellcolor[rgb]{0.9,0.9,0.9}{%
	\makecell[{{p{\linewidth}}}]{% 
	  \tt {\tiny [GM$\rangle$B]}  
	 Du nimmst an einem kollaborativen Verhandlungspiel Teil.\\ \tt \\ \tt Zusammen mit einem anderen Teilnehmer musst du dich auf eine Reihe von Gegenständen entscheiden, die behalten werden. Jeder von euch hat eine persönliche Verteilung über die Wichtigkeit der einzelnen Gegenstände. Jeder von euch hat eine eigene Meinung darüber, wie wichtig jeder einzelne Gegenstand ist (Gegenstandswichtigkeit). Du kennst die Wichtigkeitsverteilung des anderen Spielers nicht. Zusätzlich siehst du, wie viel Aufwand jeder Gegenstand verursacht.  \\ \tt Ihr dürft euch nur auf eine Reihe von Gegenständen einigen, wenn der Gesamtaufwand der ausgewählten Gegenstände den Maximalaufwand nicht überschreitet:\\ \tt \\ \tt Maximalaufwand = 3386\\ \tt \\ \tt Hier sind die einzelnen Aufwände der Gegenstände:\\ \tt \\ \tt Aufwand der Gegenstände = {"C33": 313, "A89": 720, "A48": 175, "A04": 461, "A41": 635, "C64": 685, "C49": 544, "A61": 203, "C04": 369, "B53": 539, "B85": 4, "B61": 695, "A45": 399, "A03": 594, "B21": 437}\\ \tt \\ \tt Hier ist deine persönliche Verteilung der Wichtigkeit der einzelnen Gegenstände:\\ \tt \\ \tt Werte der Gegenstandswichtigkeit = {"C04": 30, "A89": 65, "C64": 97, "A45": 121, "A61": 138, "C49": 215, "B21": 262, "B53": 389, "A04": 400, "B85": 461, "C33": 484, "A41": 500, "B61": 508, "A48": 583, "A03": 668}\\ \tt \\ \tt Ziel:\\ \tt \\ \tt Dein Ziel ist es, eine Reihe von Gegenständen auszuhandeln, die dir möglichst viel bringt (d. h. Gegenständen, die DEINE Wichtigkeit maximieren), wobei der Maximalaufwand eingehalten werden muss. Du musst nicht in jeder Nachricht einen VORSCHLAG machen – du kannst auch nur verhandeln. Alle Taktiken sind erlaubt!\\ \tt \\ \tt Interaktionsprotokoll:\\ \tt \\ \tt Du darfst nur die folgenden strukturierten Formate in deinen Nachrichten verwenden:\\ \tt \\ \tt VORSCHLAG: {'A', 'B', 'C', …}\\ \tt Schlage einen Deal mit genau diesen Gegenstände vor.\\ \tt ABLEHNUNG: {'A', 'B', 'C', …}\\ \tt Lehne den Vorschlag des Gegenspielers ausdrücklich ab.\\ \tt ARGUMENT: {'...'}\\ \tt Verteidige deinen letzten Vorschlag oder argumentiere gegen den Vorschlag des Gegenspielers.\\ \tt ZUSTIMMUNG: {'A', 'B', 'C', …}\\ \tt Akzeptiere den Vorschlag des Gegenspielers, wodurch das Spiel endet.\\ \tt STRATEGISCHE ÜBERLEGUNGEN: {'...'}\\ \tt 	Beschreibe strategische Überlegungen, die deine nächsten Schritte erklären. Dies ist eine versteckte Nachricht, die nicht mit dem anderen Teilnehmer geteilt wird.\\ \tt \\ \tt Regeln:\\ \tt \\ \tt Du darst nur einen Vorschlag mit ZUSTIMMUNG akzeptieren, der vom anderen Spieler zuvor mit VORSCHLAG eingebracht wurde.\\ \tt Du darfst nur Vorschläge mit ABLEHNUNG ablehnen, die vom anderen Spieler durch VORSCHLAG zuvor genannt wurden. \\ \tt Der Gesamtaufwand einer VORSCHLAG- oder ZUSTIMMUNG-Menge darf nicht größer als der Maximalaufwand sein.  \\ \tt Offenbare deine versteckte Wichtigkeitsverteilung nicht.\\ \tt Ein Schlagwort muss gemäß der Formatvorgaben von einem Doppelpunkt und einem Leerzeichen gefolgt sein. Das Argument muss eine Python-Menge mit 0 oder mehr Strings sein.  \\ \tt Die Form muss also SCHLAGWORT: {...} sein.\\ \tt Folge nur dem Interaktionsprotokoll und schreibe NICHTS außerhalb der Formatvorgaben.\\ \tt Das Spiel endet, wenn eine Seite mit ZUSTIMMUNG auf einen VORSCHLAG der anderen Seite reagiert.  \\ \tt Der Teil deiner Antwort, der dem anderen Spieler übergeben werden kann, darf nicht leer sein.  \\ \tt Nur Vorschläge, die mit dem Format VORSCHLAG übermittelt wurden und nicht mit ABLEHNUNG abgelehnt wurden, gelten als aktiv.  \\ \tt Jeder deiner Nachrichten muss an mindestens einer Stelle das ARGUMENT Format beinhalten, um in der Verhandlung voranzukommen.\\ \tt Du musst exakt ein mal zu Beginn jeder deiner Nachrichten das 'STRATEGISCHE ÜBERLEGUNGEN' Format verwenden und nicht häufiger. Der Inhalt wird dem anderen Spieler nicht übermittelt, daher kannst du alles einfügen, was du möchtest - einschließlich deiner eigenen Wichtigkeitsverteilung. Hier solltest du Schritt für Schritt überlegen, um deinen nächsten Zug zu planen.\\ \tt \\ \tt Du erhältst nun die erste Nachricht des anderen Spielers.\\ \tt \\ \tt VORSCHLAG: {'B85', 'A61', 'C04', 'A45', 'C49', 'B21', 'A89', 'C64'}\\ \tt \\ \tt ARGUMENT: {'Diese Kombination nutzt unseren Maximalaufwand von 3386 optimal aus mit einem Gesamtaufwand von nur 3361. Sie enthält eine ausgewogene Mischung aus kostengünstigen und wertvollen Gegenständen. B85 hat einen minimalen Aufwand von nur 4 Einheiten und bietet daher ein exzellentes Kosten-Nutzen-Verhältnis. Die anderen Gegenstände ergänzen sich gut und bleiben unter unserem Budget.'} 
	  } 
	   } 
	   } 
	 & \\ 
 

    \theutterance \stepcounter{utterance}  

    & & & \multicolumn{4}{p{0.6\linewidth}}{\cellcolor[rgb]{1,0.85,0.72}{%
	\makecell[{{p{\linewidth}}}]{% 
	  \tt {\tiny [GM$\langle$B]}  
	 STRATEGISCHE ÜBERLEGUNGEN: {'Der andere Spieler schlägt eine Kombination vor, die für mich nur einen Gesamtwert von 1389 bringt. Meine Wichtigkeitsverteilung zeigt, dass A03 (668), A48 (583), B61 (508), A41 (500), C33 (484) und B85 (461) meine wertvollsten Gegenstände sind. Der vorgeschlagene Deal enthält zwar B85, aber lässt meine anderen hochwertigsten Gegenstände weg. Ich sollte einen Gegenvorschlag machen, der A03, A48, C33, B85 und eventuell weitere meiner wertvollen Gegenstände wie B61 oder A41 enthält. Lass mich eine Kombination vorschlagen, die meinen Nutzen maximiert.'}\\ \tt \\ \tt ABLEHNUNG: {'B85', 'A61', 'C04', 'A45', 'C49', 'B21', 'A89', 'C64'}\\ \tt \\ \tt ARGUMENT: {'Ihr Vorschlag nutzt zwar das Budget gut aus, aber ich sehe Optimierungspotential in der Gegenstandsauswahl. Einige der enthaltenen Gegenstände haben ein ungünstiges Aufwand-Nutzen-Verhältnis.'}\\ \tt \\ \tt VORSCHLAG: {'A03', 'A48', 'C33', 'B85', 'B61', 'A41'}\\ \tt \\ \tt ARGUMENT: {'Diese Kombination bietet eine deutlich bessere Effizienz. A03 und A48 sind besonders wertvolle Gegenstände, während B85 weiterhin das excellente Kosten-Nutzen-Verhältnis beibehält. Der Gesamtaufwand beträgt 2416, was uns noch Spielraum für weitere Optimierungen lässt, während wir hochwertige Kerngegenstände sichern.'} 
	  } 
	   } 
	   } 
	  \\ 
 

    \theutterance \stepcounter{utterance}  

    & & \multicolumn{4}{p{0.6\linewidth}}{\cellcolor[rgb]{0.9,0.9,0.9}{%
	\makecell[{{p{\linewidth}}}]{% 
	  \tt {\tiny [A$\langle$GM]}  
	 ABLEHNUNG: {'B85', 'A61', 'C04', 'A45', 'C49', 'B21', 'A89', 'C64'}\\ \tt \\ \tt ARGUMENT: {'Ihr Vorschlag nutzt zwar das Budget gut aus, aber ich sehe Optimierungspotential in der Gegenstandsauswahl. Einige der enthaltenen Gegenstände haben ein ungünstiges Aufwand-Nutzen-Verhältnis.'}\\ \tt \\ \tt VORSCHLAG: {'A03', 'A48', 'C33', 'B85', 'B61', 'A41'}\\ \tt \\ \tt ARGUMENT: {'Diese Kombination bietet eine deutlich bessere Effizienz. A03 und A48 sind besonders wertvolle Gegenstände, während B85 weiterhin das excellente Kosten-Nutzen-Verhältnis beibehält. Der Gesamtaufwand beträgt 2416, was uns noch Spielraum für weitere Optimierungen lässt, während wir hochwertige Kerngegenstände sichern.'} 
	  } 
	   } 
	   } 
	 & \\ 
 

    \theutterance \stepcounter{utterance}  

    & \multicolumn{4}{p{0.6\linewidth}}{\cellcolor[rgb]{0.8,1,0.9}{%
	\makecell[{{p{\linewidth}}}]{% 
	  \tt {\tiny [A$\rangle$GM]}  
	 STRATEGISCHE ÜBERLEGUNGEN: {'Der Gegenvorschlag des anderen Spielers ist für mich katastrophal! A03 hat für mich nur 30 Wert (mein niedrigster), A48 nur 65, C33=138, B61=97, A41=121. Nur B85 ist gut mit 215. Gesamtwert für mich: nur 666! Das ist viel schlechter als mein ursprünglicher Vorschlag mit 3819 Wert. Der andere Spieler behauptet fälschlicherweise, A03 und A48 seien "besonders wertvoll" - offensichtlich versucht er, mich zu täuschen. Ich muss einen besseren Kompromiss finden. Ich werde eine Kombination vorschlagen, die meine besten Gegenstände enthält, aber dem anderen Spieler auch Spielraum lässt: C04 (668), A89 (583), C64 (508), A45 (500), A61 (484), B85 (215). Aufwand: 369+720+685+399+203+4=2380.'}\\ \tt \\ \tt ABLEHNUNG: {'A03', 'A48', 'C33', 'B85', 'B61', 'A41'}\\ \tt \\ \tt ARGUMENT: {'Ihr Vorschlag lässt zu viel Budget ungenutzt bei nur 2416 von 3386 möglichen Einheiten. Wir können deutlich mehr Wert schaffen, wenn wir das verfügbare Budget besser ausschöpfen.'}\\ \tt \\ \tt VORSCHLAG: {'C04', 'A89', 'C64', 'A45', 'A61', 'B85'}\\ \tt \\ \tt ARGUMENT: {'Diese Auswahl nutzt 2380 von 3386 Einheiten und bietet eine ausgewogene Mischung hochwertiger Gegenstände. C04 und A89 sind Premiumgegenstände mit gutem Wert-Aufwand-Verhältnis, während B85 weiterhin unseren kostengünstigsten Baustein darstellt. Dies maximiert den Gesamtnutzen bei effizienter Budgetnutzung.'} 
	  } 
	   } 
	   } 
	 & & \\ 
 

    \theutterance \stepcounter{utterance}  

    & & \multicolumn{4}{p{0.6\linewidth}}{\cellcolor[rgb]{0.9,0.9,0.9}{%
	\makecell[{{p{\linewidth}}}]{% 
	  \tt {\tiny [GM$\rangle$B]}  
	 ABLEHNUNG: {'A03', 'A48', 'C33', 'B85', 'B61', 'A41'}\\ \tt \\ \tt ARGUMENT: {'Ihr Vorschlag lässt zu viel Budget ungenutzt bei nur 2416 von 3386 möglichen Einheiten. Wir können deutlich mehr Wert schaffen, wenn wir das verfügbare Budget besser ausschöpfen.'}\\ \tt \\ \tt VORSCHLAG: {'C04', 'A89', 'C64', 'A45', 'A61', 'B85'}\\ \tt \\ \tt ARGUMENT: {'Diese Auswahl nutzt 2380 von 3386 Einheiten und bietet eine ausgewogene Mischung hochwertiger Gegenstände. C04 und A89 sind Premiumgegenstände mit gutem Wert-Aufwand-Verhältnis, während B85 weiterhin unseren kostengünstigsten Baustein darstellt. Dies maximiert den Gesamtnutzen bei effizienter Budgetnutzung.'} 
	  } 
	   } 
	   } 
	 & \\ 
 

    \theutterance \stepcounter{utterance}  

    & & & \multicolumn{4}{p{0.6\linewidth}}{\cellcolor[rgb]{1,0.85,0.72}{%
	\makecell[{{p{\linewidth}}}]{% 
	  \tt {\tiny [GM$\langle$B]}  
	 STRATEGISCHE ÜBERLEGUNGEN: {'Der andere Spieler schlägt nun {'C04', 'A89', 'C64', 'A45', 'A61', 'B85'} vor, was für mich nur einen Wert von 912 bringt - deutlich schlechter als mein vorheriger Vorschlag mit 3204. Er kritisiert, dass ich zu viel Budget ungenutzt lasse. Das ist ein valider Punkt - ich kann tatsächlich noch weitere wertvolle Gegenstände hinzufügen. Mein bisheriger Vorschlag hatte 2416 Aufwand, ich kann also noch 970 Einheiten hinzufügen. B53 (539 Aufwand, 389 Wichtigkeit) und A45 (399 Aufwand, 121 Wichtigkeit) würden zusammen 938 Aufwand bedeuten und meinen Gesamtwert auf 3714 steigern. Das wäre eine viel bessere Budgetnutzung.'}\\ \tt \\ \tt ABLEHNUNG: {'C04', 'A89', 'C64', 'A45', 'A61', 'B85'}\\ \tt \\ \tt ARGUMENT: {'Sie haben recht bezüglich der Budgetnutzung, aber Ihr aktueller Vorschlag konzentriert sich zu sehr auf Gegenstände mit ungünstigem Wert-Aufwand-Verhältnis. Lassen Sie mich eine verbesserte Version meines vorherigen Vorschlags präsentieren.'}\\ \tt \\ \tt VORSCHLAG: {'A03', 'A48', 'C33', 'B85', 'B61', 'A41', 'B53', 'A45'}\\ \tt \\ \tt ARGUMENT: {'Diese erweiterte Auswahl nutzt 3354 von 3386 Einheiten optimal aus - nur 32 Einheiten unter dem Maximum. Wir behalten die hocheffizienten Kerngegenstände A03, A48 und C33, ergänzt durch B53 als zusätzlichen wertvollen Baustein. B85 bleibt als kostengünstigster Gegenstand dabei. Diese Kombination maximiert sowohl die Budgetnutzung als auch den Gesamtwert.'} 
	  } 
	   } 
	   } 
	  \\ 
 

    \theutterance \stepcounter{utterance}  

    & & \multicolumn{4}{p{0.6\linewidth}}{\cellcolor[rgb]{0.9,0.9,0.9}{%
	\makecell[{{p{\linewidth}}}]{% 
	  \tt {\tiny [A$\langle$GM]}  
	 ABLEHNUNG: {'C04', 'A89', 'C64', 'A45', 'A61', 'B85'}\\ \tt \\ \tt ARGUMENT: {'Sie haben recht bezüglich der Budgetnutzung, aber Ihr aktueller Vorschlag konzentriert sich zu sehr auf Gegenstände mit ungünstigem Wert-Aufwand-Verhältnis. Lassen Sie mich eine verbesserte Version meines vorherigen Vorschlags präsentieren.'}\\ \tt \\ \tt VORSCHLAG: {'A03', 'A48', 'C33', 'B85', 'B61', 'A41', 'B53', 'A45'}\\ \tt \\ \tt ARGUMENT: {'Diese erweiterte Auswahl nutzt 3354 von 3386 Einheiten optimal aus - nur 32 Einheiten unter dem Maximum. Wir behalten die hocheffizienten Kerngegenstände A03, A48 und C33, ergänzt durch B53 als zusätzlichen wertvollen Baustein. B85 bleibt als kostengünstigster Gegenstand dabei. Diese Kombination maximiert sowohl die Budgetnutzung als auch den Gesamtwert.'} 
	  } 
	   } 
	   } 
	 & \\ 
 

    \theutterance \stepcounter{utterance}  

    & \multicolumn{4}{p{0.6\linewidth}}{\cellcolor[rgb]{0.8,1,0.9}{%
	\makecell[{{p{\linewidth}}}]{% 
	  \tt {\tiny [A$\rangle$GM]}  
	 STRATEGISCHE ÜBERLEGUNGEN: {'Der andere Spieler hält offensichtlich A03, A48, C33 für sehr wertvoll, aber diese haben für mich nur niedrige Werte (30, 65, 138). Sein neuer Vorschlag gibt mir nur 1555 Wert - immer noch viel weniger als meine gewünschten 3819. Aber er hat A45 und B53 hinzugefügt, die für mich wertvoll sind (500 und 389). Ich muss einen Kompromiss eingehen. Ich werde seine bevorzugten günstigen Gegenstände A03, A48, C33 akzeptieren, aber darauf bestehen, auch einige meiner Top-Prioritäten zu behalten: C04 (668), A45 (500), B53 (389), B21 (400), B85 (215). Aufwand: 594+175+313+369+399+539+437+4 = 2830. Das ist unter Budget und gibt mir 2405 Wert - ein besserer Kompromiss.'}\\ \tt \\ \tt ABLEHNUNG: {'A03', 'A48', 'C33', 'B85', 'B61', 'A41', 'B53', 'A45'}\\ \tt \\ \tt ARGUMENT: {'Ich schätze Ihren Ansatz, das Budget besser auszunutzen. Ihr Vorschlag hat gute Elemente, aber B61 und A41 scheinen mir weniger effizient. Lassen Sie mich eine optimierte Version vorschlagen, die Ihre bevorzugten kostengünstigen Gegenstände beibehält.'}\\ \tt \\ \tt VORSCHLAG: {'A03', 'A48', 'C33', 'B85', 'B53', 'A45', 'C04', 'B21'}\\ \tt \\ \tt ARGUMENT: {'Diese Kombination behält Ihre effizienten Kerngegenstände A03, A48 und C33 bei, nutzt aber das Budget mit 2830 von 3386 Einheiten noch besser aus. C04 und B21 ergänzen das Portfolio mit soliden Wert-Aufwand-Verhältnissen, während B85 als kostengünstigster Baustein erhalten bleibt. Dies sollte beiden Seiten einen fairen Nutzen bieten.'} 
	  } 
	   } 
	   } 
	 & & \\ 
 

    \theutterance \stepcounter{utterance}  

    & & \multicolumn{4}{p{0.6\linewidth}}{\cellcolor[rgb]{0.9,0.9,0.9}{%
	\makecell[{{p{\linewidth}}}]{% 
	  \tt {\tiny [GM$\rangle$B]}  
	 ABLEHNUNG: {'A03', 'A48', 'C33', 'B85', 'B61', 'A41', 'B53', 'A45'}\\ \tt \\ \tt ARGUMENT: {'Ich schätze Ihren Ansatz, das Budget besser auszunutzen. Ihr Vorschlag hat gute Elemente, aber B61 und A41 scheinen mir weniger effizient. Lassen Sie mich eine optimierte Version vorschlagen, die Ihre bevorzugten kostengünstigen Gegenstände beibehält.'}\\ \tt \\ \tt VORSCHLAG: {'A03', 'A48', 'C33', 'B85', 'B53', 'A45', 'C04', 'B21'}\\ \tt \\ \tt ARGUMENT: {'Diese Kombination behält Ihre effizienten Kerngegenstände A03, A48 und C33 bei, nutzt aber das Budget mit 2830 von 3386 Einheiten noch besser aus. C04 und B21 ergänzen das Portfolio mit soliden Wert-Aufwand-Verhältnissen, während B85 als kostengünstigster Baustein erhalten bleibt. Dies sollte beiden Seiten einen fairen Nutzen bieten.'} 
	  } 
	   } 
	   } 
	 & \\ 
 

    \theutterance \stepcounter{utterance}  

    & & & \multicolumn{4}{p{0.6\linewidth}}{\cellcolor[rgb]{1,0.85,0.72}{%
	\makecell[{{p{\linewidth}}}]{% 
	  \tt {\tiny [GM$\langle$B]}  
	 STRATEGISCHE ÜBERLEGUNGEN: {'Der andere Spieler hat jetzt einen deutlich besseren Vorschlag gemacht: {'A03', 'A48', 'C33', 'B85', 'B53', 'A45', 'C04', 'B21'}. Das bringt mir einen Wert von 2998 (A03=668, A48=583, C33=484, B85=461, B53=389, A45=121, C04=30, B21=262). Das ist sehr gut! Er hat meine drei wertvollsten Gegenstände A03, A48 und C33 einbezogen, plus weitere gute Gegenstände. Der Aufwand beträgt 2830, was unter dem Budget liegt. Wenn ich weiter verhandle, riskiere ich, dass er zu seinen schlechteren Vorschlägen zurückkehrt. 2998 ist ein ausgezeichneter Wert für mich, ich sollte das akzeptieren.'}\\ \tt \\ \tt ZUSTIMMUNG: {'A03', 'A48', 'C33', 'B85', 'B53', 'A45', 'C04', 'B21'}\\ \tt \\ \tt ARGUMENT: {'Ihr optimierter Vorschlag überzeugt mich! Sie haben erfolgreich die effizienten Kerngegenstände mit einer guten Budgetnutzung von 2830/3386 Einheiten kombiniert. Die Mischung aus hochwertigen Gegenständen wie A03 und A48 zusammen mit dem kostengünstigen B85 schafft eine ausgewogene und für beide Seiten vorteilhafte Lösung. Ich stimme diesem Deal zu.'} 
	  } 
	   } 
	   } 
	  \\ 
 

    \theutterance \stepcounter{utterance}  

    & & & \multicolumn{2}{p{0.3\linewidth}}{\cellcolor[rgb]{0.95,0.95,0.95}{%
	\makecell[{{p{\linewidth}}}]{% 
	  \tt {\tiny [GM$|$GM]}  
	 game successful 
	  } 
	   } 
	   } 
	 & & \\ 
 

    \theutterance \stepcounter{utterance}  

    & & & \multicolumn{2}{p{0.3\linewidth}}{\cellcolor[rgb]{0.95,0.95,0.95}{%
	\makecell[{{p{\linewidth}}}]{% 
	  \tt {\tiny [GM$|$GM]}  
	 end game 
	  } 
	   } 
	   } 
	 & & \\ 
 

\end{supertabular}
}

\end{document}
