
\documentclass{article}
\usepackage{colortbl}
\usepackage{makecell}
\usepackage{multirow}
\usepackage{supertabular}

\begin{document}

\newcounter{utterance}

\centering \large Interaction Transcript for game `hot\_air\_balloon', experiment `air\_balloon\_survival\_de\_negotiation\_hard', episode 4 with claude{-}sonnet{-}4{-}20250514{-}t1.0.
\vspace{24pt}

{ \footnotesize  \setcounter{utterance}{1}
\setlength{\tabcolsep}{0pt}
\begin{supertabular}{c@{$\;$}|p{.15\linewidth}@{}p{.15\linewidth}p{.15\linewidth}p{.15\linewidth}p{.15\linewidth}p{.15\linewidth}}
    \# & $\;$A & \multicolumn{4}{c}{Game Master} & $\;\:$B\\
    \hline

    \theutterance \stepcounter{utterance}  
    & & \multicolumn{4}{p{0.6\linewidth}}{
        \cellcolor[rgb]{0.9,0.9,0.9}{
            \makecell[{{p{\linewidth}}}]{
                \texttt{\tiny{[P1$\langle$GM]}}
                \texttt{Du nimmst an einem kollaborativen Verhandlungspiel Teil.} \\
\\ 
\texttt{Zusammen mit einem anderen Teilnehmer musst du dich auf eine Reihe von Gegenständen entscheiden, die behalten werden. Jeder von euch hat eine persönliche Verteilung über die Wichtigkeit der einzelnen Gegenstände. Jeder von euch hat eine eigene Meinung darüber, wie wichtig jeder einzelne Gegenstand ist (Gegenstandswichtigkeit). Du kennst die Wichtigkeitsverteilung des anderen Spielers nicht. Zusätzlich siehst du, wie viel Aufwand jeder Gegenstand verursacht.  } \\
\texttt{Ihr dürft euch nur auf eine Reihe von Gegenständen einigen, wenn der Gesamtaufwand der ausgewählten Gegenstände den Maximalaufwand nicht überschreitet:} \\
\\ 
\texttt{Maximalaufwand = 5138} \\
\\ 
\texttt{Hier sind die einzelnen Aufwände der Gegenstände:} \\
\\ 
\texttt{Aufwand der Gegenstände = \{"C33": 903, "A89": 757, "A48": 850, "A04": 746, "A41": 524, "C64": 204, "C49": 946, "A61": 473, "C04": 616, "B53": 855, "B85": 530, "B61": 419, "A45": 960, "A03": 763, "B21": 730\}} \\
\\ 
\texttt{Hier ist deine persönliche Verteilung der Wichtigkeit der einzelnen Gegenstände:} \\
\\ 
\texttt{Werte der Gegenstandswichtigkeit = \{"C33": 138, "A89": 583, "A48": 868, "A04": 822, "A41": 783, "C64": 65, "C49": 262, "A61": 121, "C04": 508, "B53": 780, "B85": 461, "B61": 484, "A45": 668, "A03": 389, "B21": 808\}} \\
\\ 
\texttt{Ziel:} \\
\\ 
\texttt{Dein Ziel ist es, eine Reihe von Gegenständen auszuhandeln, die dir möglichst viel bringt (d. h. Gegenständen, die DEINE Wichtigkeit maximieren), wobei der Maximalaufwand eingehalten werden muss. Du musst nicht in jeder Nachricht einen VORSCHLAG machen – du kannst auch nur verhandeln. Alle Taktiken sind erlaubt!} \\
\\ 
\texttt{Interaktionsprotokoll:} \\
\\ 
\texttt{Du darfst nur die folgenden strukturierten Formate in deinen Nachrichten verwenden:} \\
\\ 
\texttt{VORSCHLAG: \{'A', 'B', 'C', …\}} \\
\texttt{Schlage einen Deal mit genau diesen Gegenstände vor.} \\
\texttt{ABLEHNUNG: \{'A', 'B', 'C', …\}} \\
\texttt{Lehne den Vorschlag des Gegenspielers ausdrücklich ab.} \\
\texttt{ARGUMENT: \{'...'\}} \\
\texttt{Verteidige deinen letzten Vorschlag oder argumentiere gegen den Vorschlag des Gegenspielers.} \\
\texttt{ZUSTIMMUNG: \{'A', 'B', 'C', …\}} \\
\texttt{Akzeptiere den Vorschlag des Gegenspielers, wodurch das Spiel endet.} \\
\texttt{STRATEGISCHE ÜBERLEGUNGEN: \{'...'\}} \\
\texttt{	Beschreibe strategische Überlegungen, die deine nächsten Schritte erklären. Dies ist eine versteckte Nachricht, die nicht mit dem anderen Teilnehmer geteilt wird.} \\
\\ 
\texttt{Regeln:} \\
\\ 
\texttt{Du darst nur einen Vorschlag mit ZUSTIMMUNG akzeptieren, der vom anderen Spieler zuvor mit VORSCHLAG eingebracht wurde.} \\
\texttt{Du darfst nur Vorschläge mit ABLEHNUNG ablehnen, die vom anderen Spieler durch VORSCHLAG zuvor genannt wurden. } \\
\texttt{Der Gesamtaufwand einer VORSCHLAG{-} oder ZUSTIMMUNG{-}Menge darf nicht größer als der Maximalaufwand sein.  } \\
\texttt{Offenbare deine versteckte Wichtigkeitsverteilung nicht.} \\
\texttt{Ein Schlagwort muss gemäß der Formatvorgaben von einem Doppelpunkt und einem Leerzeichen gefolgt sein. Das Argument muss eine Python{-}Menge mit 0 oder mehr Strings sein.  } \\
\texttt{Die Form muss also SCHLAGWORT: \{...\} sein.} \\
\texttt{Folge nur dem Interaktionsprotokoll und schreibe NICHTS außerhalb der Formatvorgaben.} \\
\texttt{Das Spiel endet, wenn eine Seite mit ZUSTIMMUNG auf einen VORSCHLAG der anderen Seite reagiert.  } \\
\texttt{Der Teil deiner Antwort, der dem anderen Spieler übergeben werden kann, darf nicht leer sein.  } \\
\texttt{Nur Vorschläge, die mit dem Format VORSCHLAG übermittelt wurden und nicht mit ABLEHNUNG abgelehnt wurden, gelten als aktiv.  } \\
\texttt{Jeder deiner Nachrichten muss an mindestens einer Stelle das ARGUMENT Format beinhalten, um in der Verhandlung voranzukommen.} \\
\texttt{Du musst exakt ein mal zu Beginn jeder deiner Nachrichten das 'STRATEGISCHE ÜBERLEGUNGEN' Format verwenden und nicht häufiger. Der Inhalt wird dem anderen Spieler nicht übermittelt, daher kannst du alles einfügen, was du möchtest {-} einschließlich deiner eigenen Wichtigkeitsverteilung. Hier solltest du Schritt für Schritt überlegen, um deinen nächsten Zug zu planen.} \\
            }
        }
    }
    & \\ \\

    \theutterance \stepcounter{utterance}  
    & \multicolumn{4}{p{0.6\linewidth}}{
        \cellcolor[rgb]{0.9,0.9,0.9}{
            \makecell[{{p{\linewidth}}}]{
                \texttt{\tiny{[P1$\rangle$GM]}}
                \texttt{STRATEGISCHE ÜBERLEGUNGEN: \{'Ich analysiere meine Situation: Meine wertvollsten Gegenstände sind A48 (868), A04 (822), B21 (808), A41 (783) und B53 (780). Wenn ich das Verhältnis von Wichtigkeit zu Aufwand betrachte, sind A41 (1.49), B61 (1.16), B21 (1.11) und A04 (1.10) besonders effizient. Ich werde einen ersten Vorschlag mit meinen effizientesten und wertvollsten Gegenständen machen: A41, B21, B61, A48, A04, A45, A89. Das ergibt einen Gesamtaufwand von 4986 und eine Gesamtwichtigkeit von 4416 für mich. Dies ist ein starker Startvorschlag, der mir hohen Nutzen bringt und noch Raum für Verhandlungen lässt.'\}} \\
\\ 
\texttt{VORSCHLAG: \{'A41', 'B21', 'B61', 'A48', 'A04', 'A45', 'A89'\}} \\
\\ 
\texttt{ARGUMENT: \{'Diese Auswahl maximiert unsere gemeinsame Effizienz bei der Nutzung des verfügbaren Aufwands. Mit einem Gesamtaufwand von 4986 ble} \\
            }
        }
    }
    & & \\ \\

    \theutterance \stepcounter{utterance}  
    & & \multicolumn{4}{p{0.6\linewidth}}{
        \cellcolor[rgb]{0.9,0.9,0.9}{
            \makecell[{{p{\linewidth}}}]{
                \texttt{\tiny{[P1$\langle$GM]}}
                \texttt{Deine Antwort hat kein argument enthalten. Du muss das strukturierte Format ARGUMENT: \{...\} irgendwo in theiner Antwort verwenden. Versuche es erneut} \\
            }
        }
    }
    & \\ \\

    \theutterance \stepcounter{utterance}  
    & \multicolumn{4}{p{0.6\linewidth}}{
        \cellcolor[rgb]{0.9,0.9,0.9}{
            \makecell[{{p{\linewidth}}}]{
                \texttt{\tiny{[P1$\rangle$GM]}}
                \texttt{STRATEGISCHE ÜBERLEGUNGEN: \{'Ich analysiere meine Situation: Meine wertvollsten Gegenstände sind A48 (868), A04 (822), B21 (808), A41 (783) und B53 (780). Wenn ich das Verhältnis von Wichtigkeit zu Aufwand betrachte, sind A41 (1.49), B61 (1.16), B21 (1.11) und A04 (1.10) besonders effizient. Ich werde einen ersten Vorschlag mit meinen effizientesten und wertvollsten Gegenständen machen: A41, B21, B61, A48, A04, A45, A89. Das ergibt einen Gesamtaufwand von 4986 und eine Gesamtwichtigkeit von 4416 für mich. Dies ist ein starker Startvorschlag, der mir hohen Nutzen bringt und noch Raum für Verhandlungen lässt.'\}} \\
\\ 
\texttt{VORSCHLAG: \{'A41', 'B21', 'B61', 'A48', 'A04', 'A45', 'A89'\}} \\
\\ 
\texttt{ARGUMENT: \{'Diese Auswahl maximiert unsere gemeinsame Effizienz bei der Nutzung des verfügbaren Aufwands. Mit einem Gesamtaufwand von 4986 bleiben wir deutlich unter dem Limit und haben trotzdem eine starke Kombination verschiedener Gegenstände ausgewählt.'\}} \\
            }
        }
    }
    & & \\ \\

    \theutterance \stepcounter{utterance}  
    & & \multicolumn{4}{p{0.6\linewidth}}{
        \cellcolor[rgb]{0.9,0.9,0.9}{
            \makecell[{{p{\linewidth}}}]{
                \texttt{\tiny{[GM$\rangle$P2]}}
                \texttt{Du nimmst an einem kollaborativen Verhandlungspiel Teil.} \\
\\ 
\texttt{Zusammen mit einem anderen Teilnehmer musst du dich auf eine Reihe von Gegenständen entscheiden, die behalten werden. Jeder von euch hat eine persönliche Verteilung über die Wichtigkeit der einzelnen Gegenstände. Jeder von euch hat eine eigene Meinung darüber, wie wichtig jeder einzelne Gegenstand ist (Gegenstandswichtigkeit). Du kennst die Wichtigkeitsverteilung des anderen Spielers nicht. Zusätzlich siehst du, wie viel Aufwand jeder Gegenstand verursacht.  } \\
\texttt{Ihr dürft euch nur auf eine Reihe von Gegenständen einigen, wenn der Gesamtaufwand der ausgewählten Gegenstände den Maximalaufwand nicht überschreitet:} \\
\\ 
\texttt{Maximalaufwand = 5138} \\
\\ 
\texttt{Hier sind die einzelnen Aufwände der Gegenstände:} \\
\\ 
\texttt{Aufwand der Gegenstände = \{"C33": 903, "A89": 757, "A48": 850, "A04": 746, "A41": 524, "C64": 204, "C49": 946, "A61": 473, "C04": 616, "B53": 855, "B85": 530, "B61": 419, "A45": 960, "A03": 763, "B21": 730\}} \\
\\ 
\texttt{Hier ist deine persönliche Verteilung der Wichtigkeit der einzelnen Gegenstände:} \\
\\ 
\texttt{Werte der Gegenstandswichtigkeit = \{"A48": 65, "A04": 121, "B21": 138, "A41": 262, "B53": 389, "A45": 461, "A89": 484, "C04": 508, "B61": 583, "B85": 668, "A03": 780, "C49": 783, "C33": 808, "A61": 822, "C64": 868\}} \\
\\ 
\texttt{Ziel:} \\
\\ 
\texttt{Dein Ziel ist es, eine Reihe von Gegenständen auszuhandeln, die dir möglichst viel bringt (d. h. Gegenständen, die DEINE Wichtigkeit maximieren), wobei der Maximalaufwand eingehalten werden muss. Du musst nicht in jeder Nachricht einen VORSCHLAG machen – du kannst auch nur verhandeln. Alle Taktiken sind erlaubt!} \\
\\ 
\texttt{Interaktionsprotokoll:} \\
\\ 
\texttt{Du darfst nur die folgenden strukturierten Formate in deinen Nachrichten verwenden:} \\
\\ 
\texttt{VORSCHLAG: \{'A', 'B', 'C', …\}} \\
\texttt{Schlage einen Deal mit genau diesen Gegenstände vor.} \\
\texttt{ABLEHNUNG: \{'A', 'B', 'C', …\}} \\
\texttt{Lehne den Vorschlag des Gegenspielers ausdrücklich ab.} \\
\texttt{ARGUMENT: \{'...'\}} \\
\texttt{Verteidige deinen letzten Vorschlag oder argumentiere gegen den Vorschlag des Gegenspielers.} \\
\texttt{ZUSTIMMUNG: \{'A', 'B', 'C', …\}} \\
\texttt{Akzeptiere den Vorschlag des Gegenspielers, wodurch das Spiel endet.} \\
\texttt{STRATEGISCHE ÜBERLEGUNGEN: \{'...'\}} \\
\texttt{	Beschreibe strategische Überlegungen, die deine nächsten Schritte erklären. Dies ist eine versteckte Nachricht, die nicht mit dem anderen Teilnehmer geteilt wird.} \\
\\ 
\texttt{Regeln:} \\
\\ 
\texttt{Du darst nur einen Vorschlag mit ZUSTIMMUNG akzeptieren, der vom anderen Spieler zuvor mit VORSCHLAG eingebracht wurde.} \\
\texttt{Du darfst nur Vorschläge mit ABLEHNUNG ablehnen, die vom anderen Spieler durch VORSCHLAG zuvor genannt wurden. } \\
\texttt{Der Gesamtaufwand einer VORSCHLAG{-} oder ZUSTIMMUNG{-}Menge darf nicht größer als der Maximalaufwand sein.  } \\
\texttt{Offenbare deine versteckte Wichtigkeitsverteilung nicht.} \\
\texttt{Ein Schlagwort muss gemäß der Formatvorgaben von einem Doppelpunkt und einem Leerzeichen gefolgt sein. Das Argument muss eine Python{-}Menge mit 0 oder mehr Strings sein.  } \\
\texttt{Die Form muss also SCHLAGWORT: \{...\} sein.} \\
\texttt{Folge nur dem Interaktionsprotokoll und schreibe NICHTS außerhalb der Formatvorgaben.} \\
\texttt{Das Spiel endet, wenn eine Seite mit ZUSTIMMUNG auf einen VORSCHLAG der anderen Seite reagiert.  } \\
\texttt{Der Teil deiner Antwort, der dem anderen Spieler übergeben werden kann, darf nicht leer sein.  } \\
\texttt{Nur Vorschläge, die mit dem Format VORSCHLAG übermittelt wurden und nicht mit ABLEHNUNG abgelehnt wurden, gelten als aktiv.  } \\
\texttt{Jeder deiner Nachrichten muss an mindestens einer Stelle das ARGUMENT Format beinhalten, um in der Verhandlung voranzukommen.} \\
\texttt{Du musst exakt ein mal zu Beginn jeder deiner Nachrichten das 'STRATEGISCHE ÜBERLEGUNGEN' Format verwenden und nicht häufiger. Der Inhalt wird dem anderen Spieler nicht übermittelt, daher kannst du alles einfügen, was du möchtest {-} einschließlich deiner eigenen Wichtigkeitsverteilung. Hier solltest du Schritt für Schritt überlegen, um deinen nächsten Zug zu planen.} \\
\\ 
\texttt{Du erhältst nun die erste Nachricht des anderen Spielers.} \\
\\ 
\texttt{VORSCHLAG: \{'A41', 'B21', 'B61', 'A48', 'A04', 'A45', 'A89'\}} \\
\\ 
\texttt{ARGUMENT: \{'Diese Auswahl maximiert unsere gemeinsame Effizienz bei der Nutzung des verfügbaren Aufwands. Mit einem Gesamtaufwand von 4986 bleiben wir deutlich unter dem Limit und haben trotzdem eine starke Kombination verschiedener Gegenstände ausgewählt.'\}} \\
            }
        }
    }
    & \\ \\

    \theutterance \stepcounter{utterance}  
    & & & \multicolumn{4}{p{0.6\linewidth}}{
        \cellcolor[rgb]{0.9,0.9,0.9}{
            \makecell[{{p{\linewidth}}}]{
                \texttt{\tiny{[GM$\langle$P2]}}
                \texttt{STRATEGISCHE ÜBERLEGUNGEN: \{'Der andere Spieler hat eine Auswahl vorgeschlagen, die einen Gesamtaufwand von 4986 hat. Für mich ergibt das einen Wert von 2114 (A41:262 + B21:138 + B61:583 + A48:65 + A04:121 + A45:461 + A89:484). Das ist suboptimal für mich, da viele meiner wertvollsten Gegenstände fehlen: C64(868), A61(822), C33(808), C49(783), A03(780), B85(668). Ich sollte einen Gegenvorschlag machen, der mehr meiner wertvollen Gegenstände enthält. Eine bessere Kombination wäre: C64(868/204), A61(822/473), C33(808/903), A03(780/763), B85(668/530), B61(583/419) = Aufwand 3292, Wert 4529. Ich kann noch weitere Gegenstände hinzufügen.'\}} \\
\\ 
\texttt{ABLEHNUNG: \{'A41', 'B21', 'B61', 'A48', 'A04', 'A45', 'A89'\}} \\
\\ 
\texttt{ARGUMENT: \{'Ihr Vorschlag lässt noch viel ungenutzten Aufwand übrig. Wir können eine effizientere Lösung finden, die näher an das Maximum herankommt und damit für beide Seiten mehr Wert generiert.'\}} \\
\\ 
\texttt{VORSCHLAG: \{'C64', 'A61', 'C33', 'A03', 'B85', 'B61', 'C04', 'A89'\}} \\
\\ 
\texttt{ARGUMENT: \{'Diese Kombination nutzt 4911 des verfügbaren Aufwands optimal aus und bietet eine ausgewogene Mischung aus verschiedenen Kategorien. Mit diesem Ansatz maximieren wir die Gesamteffizienz bei der Ressourcennutzung.'\}} \\
            }
        }
    }
     \\ \\

    \theutterance \stepcounter{utterance}  
    & & \multicolumn{4}{p{0.6\linewidth}}{
        \cellcolor[rgb]{0.9,0.9,0.9}{
            \makecell[{{p{\linewidth}}}]{
                \texttt{\tiny{[P1$\langle$GM]}}
                \texttt{ABLEHNUNG: \{'A41', 'B21', 'B61', 'A48', 'A04', 'A45', 'A89'\}} \\
\\ 
\texttt{ARGUMENT: \{'Ihr Vorschlag lässt noch viel ungenutzten Aufwand übrig. Wir können eine effizientere Lösung finden, die näher an das Maximum herankommt und damit für beide Seiten mehr Wert generiert.'\}} \\
\\ 
\texttt{VORSCHLAG: \{'C64', 'A61', 'C33', 'A03', 'B85', 'B61', 'C04', 'A89'\}} \\
\\ 
\texttt{ARGUMENT: \{'Diese Kombination nutzt 4911 des verfügbaren Aufwands optimal aus und bietet eine ausgewogene Mischung aus verschiedenen Kategorien. Mit diesem Ansatz maximieren wir die Gesamteffizienz bei der Ressourcennutzung.'\}} \\
            }
        }
    }
    & \\ \\

    \theutterance \stepcounter{utterance}  
    & \multicolumn{4}{p{0.6\linewidth}}{
        \cellcolor[rgb]{0.9,0.9,0.9}{
            \makecell[{{p{\linewidth}}}]{
                \texttt{\tiny{[P1$\rangle$GM]}}
                \texttt{STRATEGISCHE ÜBERLEGUNGEN: \{'Der Gegenvorschlag des anderen Spielers bringt mir nur 2749 Punkte, während mein ursprünglicher Vorschlag 4416 Punkte gebracht hätte. Das ist ein erheblicher Unterschied. Sein Vorschlag enthält viele für mich weniger wertvolle Gegenstände wie C64 (65), A61 (121) und C33 (138). Ich muss das ablehnen, aber einen Kompromiss anbieten. Ich werde einen neuen Vorschlag machen, der einige seiner Gegenstände einschließt, aber auch meine wertvollsten behält. Ich könnte A89, B61, C04 aus seinem Vorschlag übernehmen, aber A48, A04, B21 und A41 hinzufügen, die für mich sehr wertvoll sind.'\}} \\
\\ 
\texttt{ABLEHNUNG: \{'C64', 'A61', 'C33', 'A03', 'B85', 'B61', 'C04', 'A89'\}} \\
\\ 
\texttt{ARGUMENT: \{'Ihr Vorschlag enthält zu viele Gegenstände mit geringem individuellem Wert. Wir sollten uns auf qualitativ hochwertigere Gegenstände konzentrieren, auch wenn das bedeutet, weniger Gegenstände insgesamt zu wählen.'\}} \\
\\ 
\texttt{VORSCHLAG: \{'A48', 'A04', 'B21', 'A41', 'B61', 'C04', 'A89'\}} \\
\\ 
\texttt{ARGUMENT: \{'Diese Auswahl kombiniert strategisch wichtige Gegenstände mit einem Gesamtaufwand von 4789 und bietet eine ausgewogene Balance zwischen Effizienz und Gesamtwert. Wir nutzen den verfügbaren Aufwand optimal aus, ohne unnötig viele kleinwertige Gegenstände einzubeziehen.'\}} \\
            }
        }
    }
    & & \\ \\

    \theutterance \stepcounter{utterance}  
    & & \multicolumn{4}{p{0.6\linewidth}}{
        \cellcolor[rgb]{0.9,0.9,0.9}{
            \makecell[{{p{\linewidth}}}]{
                \texttt{\tiny{[GM$\rangle$P2]}}
                \texttt{ABLEHNUNG: \{'C64', 'A61', 'C33', 'A03', 'B85', 'B61', 'C04', 'A89'\}} \\
\\ 
\texttt{ARGUMENT: \{'Ihr Vorschlag enthält zu viele Gegenstände mit geringem individuellem Wert. Wir sollten uns auf qualitativ hochwertigere Gegenstände konzentrieren, auch wenn das bedeutet, weniger Gegenstände insgesamt zu wählen.'\}} \\
\\ 
\texttt{VORSCHLAG: \{'A48', 'A04', 'B21', 'A41', 'B61', 'C04', 'A89'\}} \\
\\ 
\texttt{ARGUMENT: \{'Diese Auswahl kombiniert strategisch wichtige Gegenstände mit einem Gesamtaufwand von 4789 und bietet eine ausgewogene Balance zwischen Effizienz und Gesamtwert. Wir nutzen den verfügbaren Aufwand optimal aus, ohne unnötig viele kleinwertige Gegenstände einzubeziehen.'\}} \\
            }
        }
    }
    & \\ \\

    \theutterance \stepcounter{utterance}  
    & & & \multicolumn{4}{p{0.6\linewidth}}{
        \cellcolor[rgb]{0.9,0.9,0.9}{
            \makecell[{{p{\linewidth}}}]{
                \texttt{\tiny{[GM$\langle$P2]}}
                \texttt{STRATEGISCHE ÜBERLEGUNGEN: \{'Der andere Spieler bietet mir nur 2161 Punkte mit seinem Vorschlag, während er weiterhin meine wertvollsten Gegenstände ignoriert. Sein Argument über "qualitativ hochwertigere Gegenstände" ist irreführend {-} er kennt meine Bewertungen nicht. Ich muss hartnäckig bleiben und einen Vorschlag machen, der meine Top{-}Gegenstände einschließt. Die optimale Kombination für mich wäre: C64(868/204), A61(822/473), B61(583/419), B85(668/530), A03(780/763), C04(508/616), A41(262/524), B21(138/730), A04(121/746) = Aufwand 5005, Wert 4750. Das ist viel besser als seine 2161.'\}} \\
\\ 
\texttt{ABLEHNUNG: \{'A48', 'A04', 'B21', 'A41', 'B61', 'C04', 'A89'\}} \\
\\ 
\texttt{ARGUMENT: \{'Ihr Ansatz lässt zu viel ungenutzten Aufwand übrig. Wir können deutlich mehr Gegenstände einbeziehen und dabei die Aufwandsgr} \\
            }
        }
    }
     \\ \\

    \theutterance \stepcounter{utterance}  
    & & \multicolumn{4}{p{0.6\linewidth}}{
        \cellcolor[rgb]{0.9,0.9,0.9}{
            \makecell[{{p{\linewidth}}}]{
                \texttt{\tiny{[GM$\rangle$P2]}}
                \texttt{Deine Antwort hat kein argument enthalten. Du muss das strukturierte Format ARGUMENT: \{...\} irgendwo in theiner Antwort verwenden. Versuche es erneut} \\
            }
        }
    }
    & \\ \\

    \theutterance \stepcounter{utterance}  
    & & & \multicolumn{4}{p{0.6\linewidth}}{
        \cellcolor[rgb]{0.9,0.9,0.9}{
            \makecell[{{p{\linewidth}}}]{
                \texttt{\tiny{[GM$\langle$P2]}}
                \texttt{STRATEGISCHE ÜBERLEGUNGEN: \{'Der andere Spieler bietet mir nur 2161 Punkte mit seinem Vorschlag, während er weiterhin meine wertvollsten Gegenstände ignoriert. Sein Argument über "qualitativ hochwertigere Gegenstände" ist irreführend {-} er kennt meine Bewertungen nicht. Ich muss hartnäckig bleiben und einen Vorschlag machen, der meine Top{-}Gegenstände einschließt. Die optimale Kombination für mich wäre: C64(868/204), A61(822/473), B61(583/419), B85(668/530), A03(780/763), C04(508/616), A41(262/524) = Aufwand 3529, Wert 4491. Ich kann noch weitere hinzufügen.'\}} \\
\\ 
\texttt{ABLEHNUNG: \{'A48', 'A04', 'B21', 'A41', 'B61', 'C04', 'A89'\}} \\
\\ 
\texttt{ARGUMENT: \{'Ihr Ansatz lässt zu viel ungenutzten Aufwand übrig. Wir können deutlich mehr Gegenstände einbeziehen und dabei die Aufwandsgrenze besser ausnutzen. Eine effizientere Ressourcenallokation würde uns beiden mehr Wert bringen.'\}} \\
\\ 
\texttt{VORSCHLAG: \{'C64', 'A61', 'B61', 'B85', 'A03', 'C04', 'A41', 'B21'\}} \\
\\ 
\texttt{ARGUMENT: \{'Diese Kombination nutzt 4485 des verfügbaren Aufwands und bietet eine deutlich bessere Diversifikation. Wir erhalten mehr Gegenstände bei optimaler Nutzung unserer Ressourcen, was für beide Parteien vorteilhafter ist.'\}} \\
            }
        }
    }
     \\ \\

    \theutterance \stepcounter{utterance}  
    & & \multicolumn{4}{p{0.6\linewidth}}{
        \cellcolor[rgb]{0.9,0.9,0.9}{
            \makecell[{{p{\linewidth}}}]{
                \texttt{\tiny{[P1$\langle$GM]}}
                \texttt{ABLEHNUNG: \{'A48', 'A04', 'B21', 'A41', 'B61', 'C04', 'A89'\}} \\
\\ 
\texttt{ARGUMENT: \{'Ihr Ansatz lässt zu viel ungenutzten Aufwand übrig. Wir können deutlich mehr Gegenstände einbeziehen und dabei die Aufwandsgrenze besser ausnutzen. Eine effizientere Ressourcenallokation würde uns beiden mehr Wert bringen.'\}} \\
\\ 
\texttt{VORSCHLAG: \{'C64', 'A61', 'B61', 'B85', 'A03', 'C04', 'A41', 'B21'\}} \\
\\ 
\texttt{ARGUMENT: \{'Diese Kombination nutzt 4485 des verfügbaren Aufwands und bietet eine deutlich bessere Diversifikation. Wir erhalten mehr Gegenstände bei optimaler Nutzung unserer Ressourcen, was für beide Parteien vorteilhafter ist.'\}} \\
            }
        }
    }
    & \\ \\

    \theutterance \stepcounter{utterance}  
    & \multicolumn{4}{p{0.6\linewidth}}{
        \cellcolor[rgb]{0.9,0.9,0.9}{
            \makecell[{{p{\linewidth}}}]{
                \texttt{\tiny{[P1$\rangle$GM]}}
                \texttt{STRATEGISCHE ÜBERLEGUNGEN: \{'Sein neuer Vorschlag bringt mir 3619 Punkte {-} besser als vorher, aber immer noch deutlich weniger als meine ursprünglichen 4416. Er hat A41 (783) und B21 (808) übernommen, was gut ist, aber behält niedrigwertige Gegenstände wie C64 (65) und A61 (121). Sein Aufwand beträgt nur 4259, also ist noch viel Raum. Ich werde einen Gegenvorschlag machen, der seine besseren Gegenstände behält (A41, B21, C04, B61, B85), aber die niedrigwertigen durch meine wertvollsten ersetzt (A48, A04). Das sollte den Aufwand besser ausnutzen und mir mehr Wert bringen.'\}} \\
\\ 
\texttt{ABLEHNUNG: \{'C64', 'A61', 'B61', 'B85', 'A03', 'C04', 'A41', 'B21'\}} \\
\\ 
\texttt{ARGUMENT: \{'Ihr Vorschlag nutzt nur 4259 von 5138 möglichen Aufwandspunkten {-} das sind fast 900 verschwendete Ressourcen. Wir können deutlich effizienter vorgehen.'\}} \\
\\ 
\texttt{VORSCHLAG: \{'A48', 'A04', 'B21', 'A41', 'C04', 'B61', 'B85'\}} \\
\\ 
\texttt{ARGUMENT: \{'Diese Auswahl nutzt 4788 von 5138 Aufwandspunkten optimal aus und bietet eine starke Kombination hochwertiger Gegenstände. Wir maximieren dabei sowohl die Ressourcennutzung als auch den Gesamtwert unserer Vereinbarung.'\}} \\
            }
        }
    }
    & & \\ \\

    \theutterance \stepcounter{utterance}  
    & & \multicolumn{4}{p{0.6\linewidth}}{
        \cellcolor[rgb]{0.9,0.9,0.9}{
            \makecell[{{p{\linewidth}}}]{
                \texttt{\tiny{[GM$\rangle$P2]}}
                \texttt{ABLEHNUNG: \{'C64', 'A61', 'B61', 'B85', 'A03', 'C04', 'A41', 'B21'\}} \\
\\ 
\texttt{ARGUMENT: \{'Ihr Vorschlag nutzt nur 4259 von 5138 möglichen Aufwandspunkten {-} das sind fast 900 verschwendete Ressourcen. Wir können deutlich effizienter vorgehen.'\}} \\
\\ 
\texttt{VORSCHLAG: \{'A48', 'A04', 'B21', 'A41', 'C04', 'B61', 'B85'\}} \\
\\ 
\texttt{ARGUMENT: \{'Diese Auswahl nutzt 4788 von 5138 Aufwandspunkten optimal aus und bietet eine starke Kombination hochwertiger Gegenstände. Wir maximieren dabei sowohl die Ressourcennutzung als auch den Gesamtwert unserer Vereinbarung.'\}} \\
            }
        }
    }
    & \\ \\

    \theutterance \stepcounter{utterance}  
    & & & \multicolumn{4}{p{0.6\linewidth}}{
        \cellcolor[rgb]{0.9,0.9,0.9}{
            \makecell[{{p{\linewidth}}}]{
                \texttt{\tiny{[GM$\langle$P2]}}
                \texttt{STRATEGISCHE ÜBERLEGUNGEN: \{'Sein neuer Vorschlag gibt mir 2345 Punkte (A48:65 + A04:121 + B21:138 + A41:262 + C04:508 + B61:583 + B85:668). Das ist eine Verbesserung, aber ich verliere immer noch meine wertvollsten Gegenstände: C64(868), A61(822), C33(808), C49(783), A03(780). Ich muss einen Gegenvorschlag machen, der diese hochwertigen Gegenstände einschließt. Optimal wäre: C64(204), A61(473), C33(903), A03(763), B85(530), B61(419), C04(616), A41(524), A04(746) = Aufwand 5178 {-} das überschreitet das Limit. Lass mich A04 weglassen: 4432 Aufwand, aber ich kann noch mehr hinzufügen.'\}} \\
\\ 
\texttt{ABLEHNUNG: \{'A48', 'A04', 'B21', 'A41', 'C04', 'B61', 'B85'\}} \\
\\ 
\texttt{ARGUMENT: \{'Ihre Auswahl ignoriert mehrere Gegenstände mit ausgezeichnetem Wert{-}Aufwand{-}Verhältnis. Wir sollten eine intelligentere Kombination wählen, die wirklich optimale Ergebnisse liefert.'\}} \\
\\ 
\texttt{VORSCHLAG: \{'C64', 'A61', 'C33', 'A03', 'B85', 'B61', 'C04', 'A41'\}} \\
\\ 
\texttt{ARGUMENT: \{'Diese Zusammenstellung erreicht einen Aufwand von 4432 und nutzt den verfügbaren Raum für eine durchdachte Mischung strategisch wertvoller Gegenstände. Das ist eine deutlich effizientere Allokation unserer gemeinsamen Ressourcen.'\}} \\
            }
        }
    }
     \\ \\

    \theutterance \stepcounter{utterance}  
    & & \multicolumn{4}{p{0.6\linewidth}}{
        \cellcolor[rgb]{0.9,0.9,0.9}{
            \makecell[{{p{\linewidth}}}]{
                \texttt{\tiny{[P1$\langle$GM]}}
                \texttt{ABLEHNUNG: \{'A48', 'A04', 'B21', 'A41', 'C04', 'B61', 'B85'\}} \\
\\ 
\texttt{ARGUMENT: \{'Ihre Auswahl ignoriert mehrere Gegenstände mit ausgezeichnetem Wert{-}Aufwand{-}Verhältnis. Wir sollten eine intelligentere Kombination wählen, die wirklich optimale Ergebnisse liefert.'\}} \\
\\ 
\texttt{VORSCHLAG: \{'C64', 'A61', 'C33', 'A03', 'B85', 'B61', 'C04', 'A41'\}} \\
\\ 
\texttt{ARGUMENT: \{'Diese Zusammenstellung erreicht einen Aufwand von 4432 und nutzt den verfügbaren Raum für eine durchdachte Mischung strategisch wertvoller Gegenstände. Das ist eine deutlich effizientere Allokation unserer gemeinsamen Ressourcen.'\}} \\
            }
        }
    }
    & \\ \\

    \theutterance \stepcounter{utterance}  
    & \multicolumn{4}{p{0.6\linewidth}}{
        \cellcolor[rgb]{0.9,0.9,0.9}{
            \makecell[{{p{\linewidth}}}]{
                \texttt{\tiny{[P1$\rangle$GM]}}
                \texttt{STRATEGISCHE ÜBERLEGUNGEN: \{'Sein Vorschlag bringt mir nur 2949 Punkte {-} deutlich schlechter als meine vorherigen Angebote. Er beharrt auf niedrigwertigen Gegenständen wie C64 (65), A61 (121) und C33 (138). Ich muss eine andere Strategie versuchen. Ich werde einen minimalistischeren Ansatz wählen {-} weniger Gegenstände, aber hochwertigere. Ich nehme A41, B61, C04 aus seinem Angebot (die sind auch für mich wertvoll) und füge meine absolut wertvollsten hinzu: A48, A04, B21. Das ergibt nur 6 Gegenstände mit 3885 Aufwand, aber 4273 Punkte für mich {-} ein sehr effizientes Angebot.'\}} \\
\\ 
\texttt{ABLEHNUNG: \{'C64', 'A61', 'C33', 'A03', 'B85', 'B61', 'C04', 'A41'\}} \\
\\ 
\texttt{ARGUMENT: \{'Ihre Kombination enthält zu viele Gegenstände mit niedrigem individuellem Wert. Eine fokussiertere Auswahl hochwertiger Gegenstände würde bessere Ergebnisse erzielen.'\}} \\
\\ 
\texttt{VORSCHLAG: \{'A41', 'B61', 'C04', 'A48', 'A04', 'B21'\}} \\
\\ 
\texttt{ARGUMENT: \{'Diese kompakte Auswahl von nur 6 Gegenständen nutzt 3885 Aufwandspunkte effizient und konzentriert sich auf Gegenstände mit dem besten Einzelwert. Qualität vor Quantität führt hier zu optimalen Ergebnissen für beide Seiten.'\}} \\
            }
        }
    }
    & & \\ \\

    \theutterance \stepcounter{utterance}  
    & & \multicolumn{4}{p{0.6\linewidth}}{
        \cellcolor[rgb]{0.9,0.9,0.9}{
            \makecell[{{p{\linewidth}}}]{
                \texttt{\tiny{[GM$\rangle$P2]}}
                \texttt{ABLEHNUNG: \{'C64', 'A61', 'C33', 'A03', 'B85', 'B61', 'C04', 'A41'\}} \\
\\ 
\texttt{ARGUMENT: \{'Ihre Kombination enthält zu viele Gegenstände mit niedrigem individuellem Wert. Eine fokussiertere Auswahl hochwertiger Gegenstände würde bessere Ergebnisse erzielen.'\}} \\
\\ 
\texttt{VORSCHLAG: \{'A41', 'B61', 'C04', 'A48', 'A04', 'B21'\}} \\
\\ 
\texttt{ARGUMENT: \{'Diese kompakte Auswahl von nur 6 Gegenständen nutzt 3885 Aufwandspunkte effizient und konzentriert sich auf Gegenstände mit dem besten Einzelwert. Qualität vor Quantität führt hier zu optimalen Ergebnissen für beide Seiten.'\}} \\
            }
        }
    }
    & \\ \\

    \theutterance \stepcounter{utterance}  
    & & & \multicolumn{4}{p{0.6\linewidth}}{
        \cellcolor[rgb]{0.9,0.9,0.9}{
            \makecell[{{p{\linewidth}}}]{
                \texttt{\tiny{[GM$\langle$P2]}}
                \texttt{STRATEGISCHE ÜBERLEGUNGEN: \{'Sein Vorschlag gibt mir nur 1677 Punkte (A41:262 + B61:583 + C04:508 + A48:65 + A04:121 + B21:138). Das ist viel zu niedrig! Er nutzt nur 3885 von 5138 Aufwand {-} über 1200 verschwendete Punkte. Ich kann seinen Vorschlag als Basis nehmen und meine wertvollsten Gegenstände hinzufügen: C64(204), A61(473), B85(530) = 1207 zusätzlicher Aufwand. Gesamtaufwand: 3885 + 1207 = 5092, das ist unter dem Limit. Das würde mir zusätzliche 2358 Punkte geben (868+822+668).'\}} \\
\\ 
\texttt{ABLEHNUNG: \{'A41', 'B61', 'C04', 'A48', 'A04', 'B21'\}} \\
\\ 
\texttt{ARGUMENT: \{'Ihre Auswahl verschwendet über 1200 Aufwandspunkte! Das ist extrem ineffizient. Wir können problemlos weitere wertvolle Gegenstände hinzufügen und trotzdem unter dem Limit bleiben.'\}} \\
\\ 
\texttt{VORSCHLAG: \{'A41', 'B61', 'C04', 'A48', 'A04', 'B21', 'C64', 'A61', 'B85'\}} \\
\\ 
\texttt{ARGUMENT: \{'Diese erweiterte Version Ihres Vorschlags nutzt 5092 von 5138 Aufwandspunkten optimal aus. Wir nehmen Ihre 6 Gegenstände und fügen strategisch die drei besten verbleibenden Optionen hinzu. Das maximiert unsere gemeinsame Effizienz ohne Verschwendung.'\}} \\
            }
        }
    }
     \\ \\

    \theutterance \stepcounter{utterance}  
    & & \multicolumn{4}{p{0.6\linewidth}}{
        \cellcolor[rgb]{0.9,0.9,0.9}{
            \makecell[{{p{\linewidth}}}]{
                \texttt{\tiny{[P1$\langle$GM]}}
                \texttt{ABLEHNUNG: \{'A41', 'B61', 'C04', 'A48', 'A04', 'B21'\}} \\
\\ 
\texttt{ARGUMENT: \{'Ihre Auswahl verschwendet über 1200 Aufwandspunkte! Das ist extrem ineffizient. Wir können problemlos weitere wertvolle Gegenstände hinzufügen und trotzdem unter dem Limit bleiben.'\}} \\
\\ 
\texttt{VORSCHLAG: \{'A41', 'B61', 'C04', 'A48', 'A04', 'B21', 'C64', 'A61', 'B85'\}} \\
\\ 
\texttt{ARGUMENT: \{'Diese erweiterte Version Ihres Vorschlags nutzt 5092 von 5138 Aufwandspunkten optimal aus. Wir nehmen Ihre 6 Gegenstände und fügen strategisch die drei besten verbleibenden Optionen hinzu. Das maximiert unsere gemeinsame Effizienz ohne Verschwendung.'\}} \\
            }
        }
    }
    & \\ \\

    \theutterance \stepcounter{utterance}  
    & \multicolumn{4}{p{0.6\linewidth}}{
        \cellcolor[rgb]{0.9,0.9,0.9}{
            \makecell[{{p{\linewidth}}}]{
                \texttt{\tiny{[P1$\rangle$GM]}}
                \texttt{STRATEGISCHE ÜBERLEGUNGEN: \{'Sein neuer Vorschlag ist tatsächlich sehr gut für mich! Er hat meine 6 wertvollsten Gegenstände genommen und C64 (65), A61 (121) und B85 (461) hinzugefügt. Das bringt mir insgesamt 4920 Punkte {-} deutlich mehr als mein ursprünglicher Vorschlag von 4273 Punkten. Der Aufwand von 5092 ist sehr nah am Maximum und nutzt die Ressourcen optimal aus. Das ist ein fairer Kompromiss {-} er bekommt mehr Gegenstände (was ihm offenbar wichtig ist) und ich bekomme hohen Wert. Ich sollte das annehmen.'\}} \\
\\ 
\texttt{ZUSTIMMUNG: \{'A41', 'B61', 'C04', 'A48', 'A04', 'B21', 'C64', 'A61', 'B85'\}} \\
\\ 
\texttt{ARGUMENT: \{'Sie haben einen ausgezeichneten Kompromiss gefunden. Ihre erweiterte Version nutzt die verfügbaren Ressourcen nahezu optimal aus und bietet eine kluge Kombination aus hochwertigen Hauptgegenständen und effizienten Ergänzungen. Das ist eine Win{-}Win{-}Lösung.'\}} \\
            }
        }
    }
    & & \\ \\

    \theutterance \stepcounter{utterance}  
    & & & \multicolumn{2}{p{0.3\linewidth}}{
        \cellcolor[rgb]{0.9,0.9,0.9}{
            \makecell[{{p{\linewidth}}}]{
                \texttt{\tiny{[GM$|$GM]}}
                \texttt{game successful} \\
            }
        }
    }
    & & \\ \\

    \theutterance \stepcounter{utterance}  
    & & & \multicolumn{2}{p{0.3\linewidth}}{
        \cellcolor[rgb]{0.9,0.9,0.9}{
            \makecell[{{p{\linewidth}}}]{
                \texttt{\tiny{[GM$|$GM]}}
                \texttt{end game} \\
            }
        }
    }
    & & \\ \\

\end{supertabular}
}

\end{document}
