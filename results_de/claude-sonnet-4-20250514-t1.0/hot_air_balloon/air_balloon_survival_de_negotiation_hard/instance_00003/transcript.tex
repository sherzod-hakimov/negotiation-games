
\documentclass{article}
\usepackage{colortbl}
\usepackage{makecell}
\usepackage{multirow}
\usepackage{supertabular}

\begin{document}

\newcounter{utterance}

\twocolumn

{ \footnotesize  \setcounter{utterance}{1}
\setlength{\tabcolsep}{0pt}
\begin{supertabular}{c@{$\;$}|p{.15\linewidth}@{}p{.15\linewidth}p{.15\linewidth}p{.15\linewidth}p{.15\linewidth}p{.15\linewidth}}

    \# & $\;$A & \multicolumn{4}{c}{Game Master} & $\;\:$B\\
    \hline 

    \theutterance \stepcounter{utterance}  

    & & \multicolumn{4}{p{0.6\linewidth}}{\cellcolor[rgb]{0.9,0.9,0.9}{%
	\makecell[{{p{\linewidth}}}]{% 
	  \tt {\tiny [A$\langle$GM]}  
	 Du nimmst an einem kollaborativen Verhandlungspiel Teil.\\ \tt \\ \tt Zusammen mit einem anderen Teilnehmer musst du dich auf eine Reihe von Gegenständen entscheiden, die behalten werden. Jeder von euch hat eine persönliche Verteilung über die Wichtigkeit der einzelnen Gegenstände. Jeder von euch hat eine eigene Meinung darüber, wie wichtig jeder einzelne Gegenstand ist (Gegenstandswichtigkeit). Du kennst die Wichtigkeitsverteilung des anderen Spielers nicht. Zusätzlich siehst du, wie viel Aufwand jeder Gegenstand verursacht.  \\ \tt Ihr dürft euch nur auf eine Reihe von Gegenständen einigen, wenn der Gesamtaufwand der ausgewählten Gegenstände den Maximalaufwand nicht überschreitet:\\ \tt \\ \tt Maximalaufwand = 3681\\ \tt \\ \tt Hier sind die einzelnen Aufwände der Gegenstände:\\ \tt \\ \tt Aufwand der Gegenstände = {"C33": 700, "A89": 586, "A48": 505, "A04": 116, "A41": 664, "C64": 940, "C49": 387, "A61": 392, "C04": 209, "B53": 571, "B85": 4, "B61": 285, "A45": 651, "A03": 613, "B21": 740}\\ \tt \\ \tt Hier ist deine persönliche Verteilung der Wichtigkeit der einzelnen Gegenstände:\\ \tt \\ \tt Werte der Gegenstandswichtigkeit = {"C33": 138, "A89": 583, "A48": 868, "A04": 822, "A41": 783, "C64": 65, "C49": 262, "A61": 121, "C04": 508, "B53": 780, "B85": 461, "B61": 484, "A45": 668, "A03": 389, "B21": 808}\\ \tt \\ \tt Ziel:\\ \tt \\ \tt Dein Ziel ist es, eine Reihe von Gegenständen auszuhandeln, die dir möglichst viel bringt (d. h. Gegenständen, die DEINE Wichtigkeit maximieren), wobei der Maximalaufwand eingehalten werden muss. Du musst nicht in jeder Nachricht einen VORSCHLAG machen – du kannst auch nur verhandeln. Alle Taktiken sind erlaubt!\\ \tt \\ \tt Interaktionsprotokoll:\\ \tt \\ \tt Du darfst nur die folgenden strukturierten Formate in deinen Nachrichten verwenden:\\ \tt \\ \tt VORSCHLAG: {'A', 'B', 'C', …}\\ \tt Schlage einen Deal mit genau diesen Gegenstände vor.\\ \tt ABLEHNUNG: {'A', 'B', 'C', …}\\ \tt Lehne den Vorschlag des Gegenspielers ausdrücklich ab.\\ \tt ARGUMENT: {'...'}\\ \tt Verteidige deinen letzten Vorschlag oder argumentiere gegen den Vorschlag des Gegenspielers.\\ \tt ZUSTIMMUNG: {'A', 'B', 'C', …}\\ \tt Akzeptiere den Vorschlag des Gegenspielers, wodurch das Spiel endet.\\ \tt STRATEGISCHE ÜBERLEGUNGEN: {'...'}\\ \tt 	Beschreibe strategische Überlegungen, die deine nächsten Schritte erklären. Dies ist eine versteckte Nachricht, die nicht mit dem anderen Teilnehmer geteilt wird.\\ \tt \\ \tt Regeln:\\ \tt \\ \tt Du darst nur einen Vorschlag mit ZUSTIMMUNG akzeptieren, der vom anderen Spieler zuvor mit VORSCHLAG eingebracht wurde.\\ \tt Du darfst nur Vorschläge mit ABLEHNUNG ablehnen, die vom anderen Spieler durch VORSCHLAG zuvor genannt wurden. \\ \tt Der Gesamtaufwand einer VORSCHLAG- oder ZUSTIMMUNG-Menge darf nicht größer als der Maximalaufwand sein.  \\ \tt Offenbare deine versteckte Wichtigkeitsverteilung nicht.\\ \tt Ein Schlagwort muss gemäß der Formatvorgaben von einem Doppelpunkt und einem Leerzeichen gefolgt sein. Das Argument muss eine Python-Menge mit 0 oder mehr Strings sein.  \\ \tt Die Form muss also SCHLAGWORT: {...} sein.\\ \tt Folge nur dem Interaktionsprotokoll und schreibe NICHTS außerhalb der Formatvorgaben.\\ \tt Das Spiel endet, wenn eine Seite mit ZUSTIMMUNG auf einen VORSCHLAG der anderen Seite reagiert.  \\ \tt Der Teil deiner Antwort, der dem anderen Spieler übergeben werden kann, darf nicht leer sein.  \\ \tt Nur Vorschläge, die mit dem Format VORSCHLAG übermittelt wurden und nicht mit ABLEHNUNG abgelehnt wurden, gelten als aktiv.  \\ \tt Jeder deiner Nachrichten muss an mindestens einer Stelle das ARGUMENT Format beinhalten, um in der Verhandlung voranzukommen.\\ \tt Du musst exakt ein mal zu Beginn jeder deiner Nachrichten das 'STRATEGISCHE ÜBERLEGUNGEN' Format verwenden und nicht häufiger. Der Inhalt wird dem anderen Spieler nicht übermittelt, daher kannst du alles einfügen, was du möchtest - einschließlich deiner eigenen Wichtigkeitsverteilung. Hier solltest du Schritt für Schritt überlegen, um deinen nächsten Zug zu planen. 
	  } 
	   } 
	   } 
	 & \\ 
 

    \theutterance \stepcounter{utterance}  

    & \multicolumn{4}{p{0.6\linewidth}}{\cellcolor[rgb]{0.8,1,0.9}{%
	\makecell[{{p{\linewidth}}}]{% 
	  \tt {\tiny [A$\rangle$GM]}  
	 STRATEGISCHE ÜBERLEGUNGEN: {'Ich analysiere zuerst meine Wichtigkeitsverteilung und berechne das Wert-Kosten-Verhältnis für jeden Gegenstand. Die besten Verhältnisse haben: B85 (461/4=115.25), A04 (822/116=7.09), C04 (508/209=2.43), A48 (868/505=1.72) und B61 (484/285=1.70). Ich werde mit einem ersten Vorschlag beginnen, der diese hocheffizienten Gegenstände enthält, aber nicht alle meine Top-Prioritäten preisgeben möchte. Ich werde einen moderaten ersten Vorschlag machen und schauen, wie der andere reagiert.'}\\ \tt \\ \tt VORSCHLAG: {'B85', 'A04', 'C04', 'A48', 'B61', 'A89', 'B53'}\\ \tt \\ \tt ARGUMENT: {'Ich schlage diese Kombination vor, da sie eine gute Balance zwischen verschiedenen Kategorien bietet und den Aufwand optimal nutzt. Diese Auswahl hat einen Gesamtaufwand von 2167, was uns noch Spielraum für weitere Verhandlungen lässt.'} 
	  } 
	   } 
	   } 
	 & & \\ 
 

    \theutterance \stepcounter{utterance}  

    & & \multicolumn{4}{p{0.6\linewidth}}{\cellcolor[rgb]{0.9,0.9,0.9}{%
	\makecell[{{p{\linewidth}}}]{% 
	  \tt {\tiny [GM$\rangle$B]}  
	 Du nimmst an einem kollaborativen Verhandlungspiel Teil.\\ \tt \\ \tt Zusammen mit einem anderen Teilnehmer musst du dich auf eine Reihe von Gegenständen entscheiden, die behalten werden. Jeder von euch hat eine persönliche Verteilung über die Wichtigkeit der einzelnen Gegenstände. Jeder von euch hat eine eigene Meinung darüber, wie wichtig jeder einzelne Gegenstand ist (Gegenstandswichtigkeit). Du kennst die Wichtigkeitsverteilung des anderen Spielers nicht. Zusätzlich siehst du, wie viel Aufwand jeder Gegenstand verursacht.  \\ \tt Ihr dürft euch nur auf eine Reihe von Gegenständen einigen, wenn der Gesamtaufwand der ausgewählten Gegenstände den Maximalaufwand nicht überschreitet:\\ \tt \\ \tt Maximalaufwand = 3681\\ \tt \\ \tt Hier sind die einzelnen Aufwände der Gegenstände:\\ \tt \\ \tt Aufwand der Gegenstände = {"C33": 700, "A89": 586, "A48": 505, "A04": 116, "A41": 664, "C64": 940, "C49": 387, "A61": 392, "C04": 209, "B53": 571, "B85": 4, "B61": 285, "A45": 651, "A03": 613, "B21": 740}\\ \tt \\ \tt Hier ist deine persönliche Verteilung der Wichtigkeit der einzelnen Gegenstände:\\ \tt \\ \tt Werte der Gegenstandswichtigkeit = {"A48": 65, "A04": 121, "B21": 138, "A41": 262, "B53": 389, "A45": 461, "A89": 484, "C04": 508, "B61": 583, "B85": 668, "A03": 780, "C49": 783, "C33": 808, "A61": 822, "C64": 868}\\ \tt \\ \tt Ziel:\\ \tt \\ \tt Dein Ziel ist es, eine Reihe von Gegenständen auszuhandeln, die dir möglichst viel bringt (d. h. Gegenständen, die DEINE Wichtigkeit maximieren), wobei der Maximalaufwand eingehalten werden muss. Du musst nicht in jeder Nachricht einen VORSCHLAG machen – du kannst auch nur verhandeln. Alle Taktiken sind erlaubt!\\ \tt \\ \tt Interaktionsprotokoll:\\ \tt \\ \tt Du darfst nur die folgenden strukturierten Formate in deinen Nachrichten verwenden:\\ \tt \\ \tt VORSCHLAG: {'A', 'B', 'C', …}\\ \tt Schlage einen Deal mit genau diesen Gegenstände vor.\\ \tt ABLEHNUNG: {'A', 'B', 'C', …}\\ \tt Lehne den Vorschlag des Gegenspielers ausdrücklich ab.\\ \tt ARGUMENT: {'...'}\\ \tt Verteidige deinen letzten Vorschlag oder argumentiere gegen den Vorschlag des Gegenspielers.\\ \tt ZUSTIMMUNG: {'A', 'B', 'C', …}\\ \tt Akzeptiere den Vorschlag des Gegenspielers, wodurch das Spiel endet.\\ \tt STRATEGISCHE ÜBERLEGUNGEN: {'...'}\\ \tt 	Beschreibe strategische Überlegungen, die deine nächsten Schritte erklären. Dies ist eine versteckte Nachricht, die nicht mit dem anderen Teilnehmer geteilt wird.\\ \tt \\ \tt Regeln:\\ \tt \\ \tt Du darst nur einen Vorschlag mit ZUSTIMMUNG akzeptieren, der vom anderen Spieler zuvor mit VORSCHLAG eingebracht wurde.\\ \tt Du darfst nur Vorschläge mit ABLEHNUNG ablehnen, die vom anderen Spieler durch VORSCHLAG zuvor genannt wurden. \\ \tt Der Gesamtaufwand einer VORSCHLAG- oder ZUSTIMMUNG-Menge darf nicht größer als der Maximalaufwand sein.  \\ \tt Offenbare deine versteckte Wichtigkeitsverteilung nicht.\\ \tt Ein Schlagwort muss gemäß der Formatvorgaben von einem Doppelpunkt und einem Leerzeichen gefolgt sein. Das Argument muss eine Python-Menge mit 0 oder mehr Strings sein.  \\ \tt Die Form muss also SCHLAGWORT: {...} sein.\\ \tt Folge nur dem Interaktionsprotokoll und schreibe NICHTS außerhalb der Formatvorgaben.\\ \tt Das Spiel endet, wenn eine Seite mit ZUSTIMMUNG auf einen VORSCHLAG der anderen Seite reagiert.  \\ \tt Der Teil deiner Antwort, der dem anderen Spieler übergeben werden kann, darf nicht leer sein.  \\ \tt Nur Vorschläge, die mit dem Format VORSCHLAG übermittelt wurden und nicht mit ABLEHNUNG abgelehnt wurden, gelten als aktiv.  \\ \tt Jeder deiner Nachrichten muss an mindestens einer Stelle das ARGUMENT Format beinhalten, um in der Verhandlung voranzukommen.\\ \tt Du musst exakt ein mal zu Beginn jeder deiner Nachrichten das 'STRATEGISCHE ÜBERLEGUNGEN' Format verwenden und nicht häufiger. Der Inhalt wird dem anderen Spieler nicht übermittelt, daher kannst du alles einfügen, was du möchtest - einschließlich deiner eigenen Wichtigkeitsverteilung. Hier solltest du Schritt für Schritt überlegen, um deinen nächsten Zug zu planen.\\ \tt \\ \tt Du erhältst nun die erste Nachricht des anderen Spielers.\\ \tt \\ \tt VORSCHLAG: {'B85', 'A04', 'C04', 'A48', 'B61', 'A89', 'B53'}\\ \tt \\ \tt ARGUMENT: {'Ich schlage diese Kombination vor, da sie eine gute Balance zwischen verschiedenen Kategorien bietet und den Aufwand optimal nutzt. Diese Auswahl hat einen Gesamtaufwand von 2167, was uns noch Spielraum für weitere Verhandlungen lässt.'} 
	  } 
	   } 
	   } 
	 & \\ 
 

    \theutterance \stepcounter{utterance}  

    & & & \multicolumn{4}{p{0.6\linewidth}}{\cellcolor[rgb]{1,0.85,0.72}{%
	\makecell[{{p{\linewidth}}}]{% 
	  \tt {\tiny [GM$\langle$B]}  
	 STRATEGISCHE ÜBERLEGUNGEN: {'Der Vorschlag enthält Gegenstände mit einem Gesamtaufwand von 2276 und bringt mir 2818 Punkte. Das ist nicht schlecht, aber ich sehe Verbesserungsmöglichkeiten. Meine wertvollsten Gegenstände (C64: 868, A61: 822, C33: 808, C49: 783, A03: 780) sind nicht enthalten. Ich werde versuchen, einige davon in die Verhandlung einzubringen, indem ich weniger wertvolle Gegenstände aus dem aktuellen Vorschlag ersetze. A48 hat nur 65 Punkte für mich bei 505 Aufwand - das ist sehr ineffizient. Auch A04 (121 Punkte, 116 Aufwand) und B53 (389 Punkte, 571 Aufwand) könnten ersetzt werden.'}\\ \tt \\ \tt ABLEHNUNG: {'B85', 'A04', 'C04', 'A48', 'B61', 'A89', 'B53'}\\ \tt \\ \tt ARGUMENT: {'Während dein Vorschlag eine gute Grundlage darstellt, sehe ich Optimierungsmöglichkeiten. Besonders A48 scheint mir bei dem hohen Aufwand nicht optimal zu sein. Lass uns schauen, ob wir durch geschicktere Auswahl mehr Wert bei gleichem oder sogar geringerem Aufwand erreichen können.'}\\ \tt \\ \tt VORSCHLAG: {'B85', 'C04', 'B61', 'A89', 'C64', 'A61', 'C49'}\\ \tt \\ \tt ARGUMENT: {'Ich schlage diese alternative Kombination vor. Sie hat einen ähnlichen Gesamtaufwand, nutzt aber die verfügbaren Ressourcen möglicherweise effizienter. Was denkst du über diese Zusammenstellung?'} 
	  } 
	   } 
	   } 
	  \\ 
 

    \theutterance \stepcounter{utterance}  

    & & \multicolumn{4}{p{0.6\linewidth}}{\cellcolor[rgb]{0.9,0.9,0.9}{%
	\makecell[{{p{\linewidth}}}]{% 
	  \tt {\tiny [A$\langle$GM]}  
	 ABLEHNUNG: {'B85', 'A04', 'C04', 'A48', 'B61', 'A89', 'B53'}\\ \tt \\ \tt ARGUMENT: {'Während dein Vorschlag eine gute Grundlage darstellt, sehe ich Optimierungsmöglichkeiten. Besonders A48 scheint mir bei dem hohen Aufwand nicht optimal zu sein. Lass uns schauen, ob wir durch geschicktere Auswahl mehr Wert bei gleichem oder sogar geringerem Aufwand erreichen können.'}\\ \tt \\ \tt VORSCHLAG: {'B85', 'C04', 'B61', 'A89', 'C64', 'A61', 'C49'}\\ \tt \\ \tt ARGUMENT: {'Ich schlage diese alternative Kombination vor. Sie hat einen ähnlichen Gesamtaufwand, nutzt aber die verfügbaren Ressourcen möglicherweise effizienter. Was denkst du über diese Zusammenstellung?'} 
	  } 
	   } 
	   } 
	 & \\ 
 

    \theutterance \stepcounter{utterance}  

    & \multicolumn{4}{p{0.6\linewidth}}{\cellcolor[rgb]{0.8,1,0.9}{%
	\makecell[{{p{\linewidth}}}]{% 
	  \tt {\tiny [A$\rangle$GM]}  
	 STRATEGISCHE ÜBERLEGUNGEN: {'Sein Vorschlag entfernt A48 (868 Wert für mich) und A04 (822 Wert) und ersetzt sie mit C64 (65 Wert) und A61 (121 Wert) - das ist deutlich schlechter für mich. Gesamtwert seines Vorschlags für mich: 2484. Mein ursprünglicher Vorschlag hatte einen Wert von etwa 4708 für mich. Ich muss das ablehnen und einen besseren Kompromiss finden. Ich werde A04 und eventuell B21 oder A45 vorschlagen, die hohen Wert für mich haben.'}\\ \tt \\ \tt ABLEHNUNG: {'B85', 'C04', 'B61', 'A89', 'C64', 'A61', 'C49'}\\ \tt \\ \tt ARGUMENT: {'Dein Vorschlag hat interessante Aspekte, aber ich sehe einige Probleme bei der Effizienz. C64 hat einen sehr hohen Aufwand von 940 für einen einzelnen Gegenstand. Ich denke, wir können das Budget besser nutzen.'}\\ \tt \\ \tt VORSCHLAG: {'B85', 'A04', 'C04', 'B61', 'A89', 'B21', 'C49'}\\ \tt \\ \tt ARGUMENT: {'Hier ist ein Kompromiss: Ich habe A04 wieder hinzugefügt, da dieser einen sehr niedrigen Aufwand hat, und B21 statt der teuren C64. Das gibt uns mehr Wert bei ähnlichem Gesamtaufwand von 2822.'} 
	  } 
	   } 
	   } 
	 & & \\ 
 

    \theutterance \stepcounter{utterance}  

    & & \multicolumn{4}{p{0.6\linewidth}}{\cellcolor[rgb]{0.9,0.9,0.9}{%
	\makecell[{{p{\linewidth}}}]{% 
	  \tt {\tiny [GM$\rangle$B]}  
	 ABLEHNUNG: {'B85', 'C04', 'B61', 'A89', 'C64', 'A61', 'C49'}\\ \tt \\ \tt ARGUMENT: {'Dein Vorschlag hat interessante Aspekte, aber ich sehe einige Probleme bei der Effizienz. C64 hat einen sehr hohen Aufwand von 940 für einen einzelnen Gegenstand. Ich denke, wir können das Budget besser nutzen.'}\\ \tt \\ \tt VORSCHLAG: {'B85', 'A04', 'C04', 'B61', 'A89', 'B21', 'C49'}\\ \tt \\ \tt ARGUMENT: {'Hier ist ein Kompromiss: Ich habe A04 wieder hinzugefügt, da dieser einen sehr niedrigen Aufwand hat, und B21 statt der teuren C64. Das gibt uns mehr Wert bei ähnlichem Gesamtaufwand von 2822.'} 
	  } 
	   } 
	   } 
	 & \\ 
 

    \theutterance \stepcounter{utterance}  

    & & & \multicolumn{4}{p{0.6\linewidth}}{\cellcolor[rgb]{1,0.85,0.72}{%
	\makecell[{{p{\linewidth}}}]{% 
	  \tt {\tiny [GM$\langle$B]}  
	 STRATEGISCHE ÜBERLEGUNGEN: {'Sein neuer Vorschlag bringt mir nur 3285 Punkte bei 2327 Aufwand - das ist deutlich schlechter als mein vorheriger Vorschlag mit 3669 Punkten. Er hat meine wertvollsten Gegenstände C64 (868 Punkte) und A61 (822 Punkte) entfernt und stattdessen B21 (nur 138 Punkte für mich) hinzugefügt. Das ist eine schlechte Entwicklung für mich. Ich habe noch 1354 Aufwandspunkte übrig, was bedeutet, dass wir definitiv noch C64 oder andere wertvolle Gegenstände hinzufügen können. Ich werde argumentieren, dass wir das Budget nicht voll ausnutzen und versuchen, meine wertvollsten Gegenstände zurück zu bekommen.'}\\ \tt \\ \tt ABLEHNUNG: {'B85', 'A04', 'C04', 'B61', 'A89', 'B21', 'C49'}\\ \tt \\ \tt ARGUMENT: {'Dein Vorschlag nutzt nur 2327 von 3681 verfügbaren Aufwandspunkten - das sind über 1350 ungenutzte Punkte! Wir verschenken wertvolles Potenzial. B21 scheint mir auch nicht optimal zu sein bei diesem hohen Aufwand von 740.'}\\ \tt \\ \tt VORSCHLAG: {'B85', 'C04', 'B61', 'A89', 'C64', 'A61', 'A03'}\\ \tt \\ \tt ARGUMENT: {'Hier ist eine bessere Nutzung unseres Budgets mit 3597 Aufwand von 3681 verfügbaren. Wir ersetzen die weniger effizienten Gegenstände und nutzen fast das komplette Budget für maximalen Wert. Was hältst du davon?'} 
	  } 
	   } 
	   } 
	  \\ 
 

    \theutterance \stepcounter{utterance}  

    & & \multicolumn{4}{p{0.6\linewidth}}{\cellcolor[rgb]{0.9,0.9,0.9}{%
	\makecell[{{p{\linewidth}}}]{% 
	  \tt {\tiny [A$\langle$GM]}  
	 ABLEHNUNG: {'B85', 'A04', 'C04', 'B61', 'A89', 'B21', 'C49'}\\ \tt \\ \tt ARGUMENT: {'Dein Vorschlag nutzt nur 2327 von 3681 verfügbaren Aufwandspunkten - das sind über 1350 ungenutzte Punkte! Wir verschenken wertvolles Potenzial. B21 scheint mir auch nicht optimal zu sein bei diesem hohen Aufwand von 740.'}\\ \tt \\ \tt VORSCHLAG: {'B85', 'C04', 'B61', 'A89', 'C64', 'A61', 'A03'}\\ \tt \\ \tt ARGUMENT: {'Hier ist eine bessere Nutzung unseres Budgets mit 3597 Aufwand von 3681 verfügbaren. Wir ersetzen die weniger effizienten Gegenstände und nutzen fast das komplette Budget für maximalen Wert. Was hältst du davon?'} 
	  } 
	   } 
	   } 
	 & \\ 
 

    \theutterance \stepcounter{utterance}  

    & \multicolumn{4}{p{0.6\linewidth}}{\cellcolor[rgb]{0.8,1,0.9}{%
	\makecell[{{p{\linewidth}}}]{% 
	  \tt {\tiny [A$\rangle$GM]}  
	 STRATEGISCHE ÜBERLEGUNGEN: {'Sein Vorschlag nutzt das Budget besser, aber der Wert für mich ist nur 2611 - immer noch schlechter als meine bisherigen Vorschläge. Er entfernt hartnäckig A04 (822 Wert, nur 116 Aufwand) und behält C64 (nur 65 Wert, 940 Aufwand). Ich muss einen Vorschlag machen, der das Budget voll ausnutzt UND meine wertvollsten Gegenstände enthält. Ich werde A04, A48, B53, A45 einschließen - alle haben hohen Wert für mich.'}\\ \tt \\ \tt ABLEHNUNG: {'B85', 'C04', 'B61', 'A89', 'C64', 'A61', 'A03'}\\ \tt \\ \tt ARGUMENT: {'Du hast recht bezüglich der Budgetnutzung, aber C64 mit 940 Aufwand scheint mir immer noch sehr ineffizient. Lass uns das Budget voll nutzen, aber mit strategisch besseren Gegenständen.'}\\ \tt \\ \tt VORSCHLAG: {'B85', 'A04', 'A48', 'B53', 'A45', 'C04', 'B61'}\\ \tt \\ \tt ARGUMENT: {'Dieser Vorschlag nutzt 3559 von 3681 Aufwandspunkten optimal aus und kombiniert sowohl niedrig- als auch mittelaufwändige Gegenstände für maximale Effizienz. Wir erhalten eine ausgewogene Mischung ohne die sehr teuren Einzelposten.'} 
	  } 
	   } 
	   } 
	 & & \\ 
 

    \theutterance \stepcounter{utterance}  

    & & \multicolumn{4}{p{0.6\linewidth}}{\cellcolor[rgb]{0.9,0.9,0.9}{%
	\makecell[{{p{\linewidth}}}]{% 
	  \tt {\tiny [GM$\rangle$B]}  
	 ABLEHNUNG: {'B85', 'C04', 'B61', 'A89', 'C64', 'A61', 'A03'}\\ \tt \\ \tt ARGUMENT: {'Du hast recht bezüglich der Budgetnutzung, aber C64 mit 940 Aufwand scheint mir immer noch sehr ineffizient. Lass uns das Budget voll nutzen, aber mit strategisch besseren Gegenständen.'}\\ \tt \\ \tt VORSCHLAG: {'B85', 'A04', 'A48', 'B53', 'A45', 'C04', 'B61'}\\ \tt \\ \tt ARGUMENT: {'Dieser Vorschlag nutzt 3559 von 3681 Aufwandspunkten optimal aus und kombiniert sowohl niedrig- als auch mittelaufwändige Gegenstände für maximale Effizienz. Wir erhalten eine ausgewogene Mischung ohne die sehr teuren Einzelposten.'} 
	  } 
	   } 
	   } 
	 & \\ 
 

    \theutterance \stepcounter{utterance}  

    & & & \multicolumn{4}{p{0.6\linewidth}}{\cellcolor[rgb]{1,0.85,0.72}{%
	\makecell[{{p{\linewidth}}}]{% 
	  \tt {\tiny [GM$\langle$B]}  
	 STRATEGISCHE ÜBERLEGUNGEN: {'Sein Vorschlag bringt mir nur 2795 Punkte bei 2341 Aufwand - das ist viel schlechter als mein letzter Vorschlag mit 4713 Punkten! Er hat sich auch verrechnet und behauptet 3559 Aufwand, aber es sind nur 2341. Er entfernt systematisch meine wertvollsten Gegenstände (C64, A61, C33, A03) und fügt stattdessen A48 hinzu, das mir nur 65 Punkte bringt. Das ist eine sehr schlechte Entwicklung. Ich muss hartnäckiger sein und einen Kompromiss finden, der zumindest einige meiner wertvollsten Gegenstände enthält. Ich werde einen mittleren Weg vorschlagen.'}\\ \tt \\ \tt ABLEHNUNG: {'B85', 'A04', 'A48', 'B53', 'A45', 'C04', 'B61'}\\ \tt \\ \tt ARGUMENT: {'Ich muss dir widersprechen - deine Rechnung stimmt nicht. Die Gegenstände haben nur einen Gesamtaufwand von 2341, nicht 3559. Außerdem scheint mir A48 sehr ineffizient zu sein. Lass uns einen echten Kompromiss finden.'}\\ \tt \\ \tt VORSCHLAG: {'B85', 'A04', 'C04', 'B61', 'A89', 'A61', 'C33', 'C49'}\\ \tt \\ \tt ARGUMENT: {'Hier ist ein ausgewogener Kompromiss mit 3581 Aufwand von 3681 verfügbaren Punkten. Wir nutzen fast das gesamte Budget und kombinieren sowohl deine bevorzugten niedrigaufwändigen Gegenstände als auch einige mittelaufwändige für optimale Gesamteffizienz.'} 
	  } 
	   } 
	   } 
	  \\ 
 

    \theutterance \stepcounter{utterance}  

    & & \multicolumn{4}{p{0.6\linewidth}}{\cellcolor[rgb]{0.9,0.9,0.9}{%
	\makecell[{{p{\linewidth}}}]{% 
	  \tt {\tiny [A$\langle$GM]}  
	 ABLEHNUNG: {'B85', 'A04', 'A48', 'B53', 'A45', 'C04', 'B61'}\\ \tt \\ \tt ARGUMENT: {'Ich muss dir widersprechen - deine Rechnung stimmt nicht. Die Gegenstände haben nur einen Gesamtaufwand von 2341, nicht 3559. Außerdem scheint mir A48 sehr ineffizient zu sein. Lass uns einen echten Kompromiss finden.'}\\ \tt \\ \tt VORSCHLAG: {'B85', 'A04', 'C04', 'B61', 'A89', 'A61', 'C33', 'C49'}\\ \tt \\ \tt ARGUMENT: {'Hier ist ein ausgewogener Kompromiss mit 3581 Aufwand von 3681 verfügbaren Punkten. Wir nutzen fast das gesamte Budget und kombinieren sowohl deine bevorzugten niedrigaufwändigen Gegenstände als auch einige mittelaufwändige für optimale Gesamteffizienz.'} 
	  } 
	   } 
	   } 
	 & \\ 
 

    \theutterance \stepcounter{utterance}  

    & \multicolumn{4}{p{0.6\linewidth}}{\cellcolor[rgb]{0.8,1,0.9}{%
	\makecell[{{p{\linewidth}}}]{% 
	  \tt {\tiny [A$\rangle$GM]}  
	 STRATEGISCHE ÜBERLEGUNGEN: {'Seine Rechnung ist falsch - sein Vorschlag hat nur 2679 Aufwand, nicht 3581. Aber der Wert für mich ist 3379, was akzeptabel ist. Wir haben noch 1002 Budget übrig. Ich könnte B53 (780 Wert, 571 Aufwand), A41 (783 Wert, 664 Aufwand) oder B21 (808 Wert, 740 Aufwand) hinzufügen. Ich werde seinen Vorschlag als Basis nehmen und einen hochwertigen Gegenstand hinzufügen.'}\\ \tt \\ \tt ARGUMENT: {'Deine Rechnung ist leider falsch - dein Vorschlag hat nur 2679 Aufwand, nicht 3581. Das lässt uns mit über 1000 ungenutzten Punkten! Wir verschenken immer noch wertvolles Potenzial.'}\\ \tt \\ \tt VORSCHLAG: {'B85', 'A04', 'C04', 'B61', 'A89', 'A61', 'C33', 'C49', 'B53'}\\ \tt \\ \tt ARGUMENT: {'Ich nehme deinen Vorschlag als Basis und füge B53 hinzu. Das bringt uns auf 3250 Aufwand von 3681 verfügbaren Punkten und nutzt das Budget viel besser aus. Diese Kombination bietet eine ausgewogene Mischung aus allen Kategorien.'} 
	  } 
	   } 
	   } 
	 & & \\ 
 

    \theutterance \stepcounter{utterance}  

    & & \multicolumn{4}{p{0.6\linewidth}}{\cellcolor[rgb]{0.9,0.9,0.9}{%
	\makecell[{{p{\linewidth}}}]{% 
	  \tt {\tiny [GM$\rangle$B]}  
	 ARGUMENT: {'Deine Rechnung ist leider falsch - dein Vorschlag hat nur 2679 Aufwand, nicht 3581. Das lässt uns mit über 1000 ungenutzten Punkten! Wir verschenken immer noch wertvolles Potenzial.'}\\ \tt \\ \tt VORSCHLAG: {'B85', 'A04', 'C04', 'B61', 'A89', 'A61', 'C33', 'C49', 'B53'}\\ \tt \\ \tt ARGUMENT: {'Ich nehme deinen Vorschlag als Basis und füge B53 hinzu. Das bringt uns auf 3250 Aufwand von 3681 verfügbaren Punkten und nutzt das Budget viel besser aus. Diese Kombination bietet eine ausgewogene Mischung aus allen Kategorien.'} 
	  } 
	   } 
	   } 
	 & \\ 
 

    \theutterance \stepcounter{utterance}  

    & & & \multicolumn{4}{p{0.6\linewidth}}{\cellcolor[rgb]{1,0.85,0.72}{%
	\makecell[{{p{\linewidth}}}]{% 
	  \tt {\tiny [GM$\langle$B]}  
	 STRATEGISCHE ÜBERLEGUNGEN: {'Sein neuer Vorschlag bringt mir 5166 Punkte bei 3250 Aufwand - das ist der beste Deal bisher! Er enthält viele meiner wertvollsten Gegenstände: A61 (822), C33 (808), C49 (783), B85 (668), B61 (583), C04 (508), A89 (484), B53 (389) und A04 (121). Das ist eine ausgezeichnete Kombination für mich. Wir haben noch 431 Aufwandspunkte übrig, aber die verbleibenden Gegenstände (A45: 461 Punkte/651 Aufwand, A03: 780 Punkte/613 Aufwand, B21: 138 Punkte/740 Aufwand, A41: 262 Punkte/664 Aufwand, C64: 868 Punkte/940 Aufwand) passen alle nicht mehr rein oder würden den Aufwand überschreiten. Ich sollte diesen Vorschlag akzeptieren.'}\\ \tt \\ \tt ZUSTIMMUNG: {'B85', 'A04', 'C04', 'B61', 'A89', 'A61', 'C33', 'C49', 'B53'}\\ \tt \\ \tt ARGUMENT: {'Perfekt! Du hast recht - diese Kombination nutzt das Budget viel besser aus und bietet eine ausgewogene Mischung. Mit 3250 von 3681 Aufwandspunkten ist das eine solide Nutzung unserer Ressourcen. Ich stimme diesem Vorschlag zu!'} 
	  } 
	   } 
	   } 
	  \\ 
 

    \theutterance \stepcounter{utterance}  

    & & & \multicolumn{2}{p{0.3\linewidth}}{\cellcolor[rgb]{0.95,0.95,0.95}{%
	\makecell[{{p{\linewidth}}}]{% 
	  \tt {\tiny [GM$|$GM]}  
	 game successful 
	  } 
	   } 
	   } 
	 & & \\ 
 

    \theutterance \stepcounter{utterance}  

    & & & \multicolumn{2}{p{0.3\linewidth}}{\cellcolor[rgb]{0.95,0.95,0.95}{%
	\makecell[{{p{\linewidth}}}]{% 
	  \tt {\tiny [GM$|$GM]}  
	 end game 
	  } 
	   } 
	   } 
	 & & \\ 
 

\end{supertabular}
}

\end{document}
