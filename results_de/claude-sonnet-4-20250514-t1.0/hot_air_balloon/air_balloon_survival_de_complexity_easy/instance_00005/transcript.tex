
\documentclass{article}
\usepackage{colortbl}
\usepackage{makecell}
\usepackage{multirow}
\usepackage{supertabular}

\begin{document}

\newcounter{utterance}

\centering \large Interaction Transcript for game `hot\_air\_balloon', experiment `air\_balloon\_survival\_de\_complexity\_easy', episode 5 with claude{-}sonnet{-}4{-}20250514{-}t1.0.
\vspace{24pt}

{ \footnotesize  \setcounter{utterance}{1}
\setlength{\tabcolsep}{0pt}
\begin{supertabular}{c@{$\;$}|p{.15\linewidth}@{}p{.15\linewidth}p{.15\linewidth}p{.15\linewidth}p{.15\linewidth}p{.15\linewidth}}
    \# & $\;$A & \multicolumn{4}{c}{Game Master} & $\;\:$B\\
    \hline

    \theutterance \stepcounter{utterance}  
    & & \multicolumn{4}{p{0.6\linewidth}}{
        \cellcolor[rgb]{0.9,0.9,0.9}{
            \makecell[{{p{\linewidth}}}]{
                \texttt{\tiny{[P1$\langle$GM]}}
                \texttt{Du nimmst an einem kollaborativen Verhandlungspiel Teil.} \\
\\ 
\texttt{Zusammen mit einem anderen Teilnehmer musst du dich auf eine Reihe von Gegenständen entscheiden, die behalten werden. Jeder von euch hat eine persönliche Verteilung über die Wichtigkeit der einzelnen Gegenstände. Jeder von euch hat eine eigene Meinung darüber, wie wichtig jeder einzelne Gegenstand ist (Gegenstandswichtigkeit). Du kennst die Wichtigkeitsverteilung des anderen Spielers nicht. Zusätzlich siehst du, wie viel Aufwand jeder Gegenstand verursacht.  } \\
\texttt{Ihr dürft euch nur auf eine Reihe von Gegenständen einigen, wenn der Gesamtaufwand der ausgewählten Gegenstände den Maximalaufwand nicht überschreitet:} \\
\\ 
\texttt{Maximalaufwand = 9475} \\
\\ 
\texttt{Hier sind die einzelnen Aufwände der Gegenstände:} \\
\\ 
\texttt{Aufwand der Gegenstände = \{"C32": 944, "C52": 568, "B54": 515, "C16": 239, "C50": 876, "A08": 421, "C59": 991, "B51": 281, "B55": 788, "B59": 678, "A77": 432, "C14": 409, "C24": 279, "B24": 506, "C31": 101, "A52": 686, "C61": 851, "B79": 854, "C54": 133, "B56": 192, "A40": 573, "B11": 17, "B60": 465, "A81": 772, "A95": 46, "B38": 501, "C51": 220, "C03": 404, "A91": 843, "A93": 748, "C28": 552, "C05": 843, "B75": 947, "B96": 345, "B86": 930\}} \\
\\ 
\texttt{Hier ist deine persönliche Verteilung der Wichtigkeit der einzelnen Gegenstände:} \\
\\ 
\texttt{Werte der Gegenstandswichtigkeit = \{"C32": 138, "C52": 583, "B54": 868, "C16": 822, "C50": 783, "A08": 65, "C59": 262, "B51": 121, "B55": 508, "B59": 780, "A77": 461, "C14": 484, "C24": 668, "B24": 389, "C31": 808, "A52": 215, "C61": 97, "B79": 500, "C54": 30, "B56": 915, "A40": 856, "B11": 400, "B60": 444, "A81": 623, "A95": 781, "B38": 786, "C51": 3, "C03": 713, "A91": 457, "A93": 273, "C28": 739, "C05": 822, "B75": 235, "B96": 606, "B86": 968\}} \\
\\ 
\texttt{Ziel:} \\
\\ 
\texttt{Dein Ziel ist es, eine Reihe von Gegenständen auszuhandeln, die dir möglichst viel bringt (d. h. Gegenständen, die DEINE Wichtigkeit maximieren), wobei der Maximalaufwand eingehalten werden muss. Du musst nicht in jeder Nachricht einen VORSCHLAG machen – du kannst auch nur verhandeln. Alle Taktiken sind erlaubt!} \\
\\ 
\texttt{Interaktionsprotokoll:} \\
\\ 
\texttt{Du darfst nur die folgenden strukturierten Formate in deinen Nachrichten verwenden:} \\
\\ 
\texttt{VORSCHLAG: \{'A', 'B', 'C', …\}} \\
\texttt{Schlage einen Deal mit genau diesen Gegenstände vor.} \\
\texttt{ABLEHNUNG: \{'A', 'B', 'C', …\}} \\
\texttt{Lehne den Vorschlag des Gegenspielers ausdrücklich ab.} \\
\texttt{ARGUMENT: \{'...'\}} \\
\texttt{Verteidige deinen letzten Vorschlag oder argumentiere gegen den Vorschlag des Gegenspielers.} \\
\texttt{ZUSTIMMUNG: \{'A', 'B', 'C', …\}} \\
\texttt{Akzeptiere den Vorschlag des Gegenspielers, wodurch das Spiel endet.} \\
\texttt{STRATEGISCHE ÜBERLEGUNGEN: \{'...'\}} \\
\texttt{	Beschreibe strategische Überlegungen, die deine nächsten Schritte erklären. Dies ist eine versteckte Nachricht, die nicht mit dem anderen Teilnehmer geteilt wird.} \\
\\ 
\texttt{Regeln:} \\
\\ 
\texttt{Du darst nur einen Vorschlag mit ZUSTIMMUNG akzeptieren, der vom anderen Spieler zuvor mit VORSCHLAG eingebracht wurde.} \\
\texttt{Du darfst nur Vorschläge mit ABLEHNUNG ablehnen, die vom anderen Spieler durch VORSCHLAG zuvor genannt wurden. } \\
\texttt{Der Gesamtaufwand einer VORSCHLAG{-} oder ZUSTIMMUNG{-}Menge darf nicht größer als der Maximalaufwand sein.  } \\
\texttt{Offenbare deine versteckte Wichtigkeitsverteilung nicht.} \\
\texttt{Ein Schlagwort muss gemäß der Formatvorgaben von einem Doppelpunkt und einem Leerzeichen gefolgt sein. Das Argument muss eine Python{-}Menge mit 0 oder mehr Strings sein.  } \\
\texttt{Die Form muss also SCHLAGWORT: \{...\} sein.} \\
\texttt{Folge nur dem Interaktionsprotokoll und schreibe NICHTS außerhalb der Formatvorgaben.} \\
\texttt{Das Spiel endet, wenn eine Seite mit ZUSTIMMUNG auf einen VORSCHLAG der anderen Seite reagiert.  } \\
\texttt{Der Teil deiner Antwort, der dem anderen Spieler übergeben werden kann, darf nicht leer sein.  } \\
\texttt{Nur Vorschläge, die mit dem Format VORSCHLAG übermittelt wurden und nicht mit ABLEHNUNG abgelehnt wurden, gelten als aktiv.  } \\
\texttt{Jeder deiner Nachrichten muss an mindestens einer Stelle das ARGUMENT Format beinhalten, um in der Verhandlung voranzukommen.} \\
\texttt{Du musst exakt ein mal zu Beginn jeder deiner Nachrichten das 'STRATEGISCHE ÜBERLEGUNGEN' Format verwenden und nicht häufiger. Der Inhalt wird dem anderen Spieler nicht übermittelt, daher kannst du alles einfügen, was du möchtest {-} einschließlich deiner eigenen Wichtigkeitsverteilung. Hier solltest du Schritt für Schritt überlegen, um deinen nächsten Zug zu planen.} \\
            }
        }
    }
    & \\ \\

    \theutterance \stepcounter{utterance}  
    & \multicolumn{4}{p{0.6\linewidth}}{
        \cellcolor[rgb]{0.9,0.9,0.9}{
            \makecell[{{p{\linewidth}}}]{
                \texttt{\tiny{[P1$\rangle$GM]}}
                \texttt{STRATEGISCHE ÜBERLEGUNGEN: \{'Ich analysiere meine Gegenstände nach Wert{-}Aufwand{-}Verhältnis: B11 (400/17=23.53), A95 (781/46=16.98), C31 (808/101=8.00), B56 (915/192=4.77), C16 (822/239=3.44) sind meine Top{-}Picks. Ich werde mit einer moderaten Auswahl beginnen, die meine wertvollsten effizienten Gegenstände enthält, aber Raum für Verhandlungen lässt. Mein Ziel ist es herauszufinden, was der andere Spieler will, ohne meine Präferenzen preiszugeben.'\}} \\
\\ 
\texttt{VORSCHLAG: \{'B11', 'A95', 'C31', 'B56', 'C16', 'C24', 'A40', 'B59', 'C50'\}} \\
\\ 
\texttt{ARGUMENT: \{'Ich schlage eine ausgewogene Auswahl vor, die verschiedene Kategorien abdeckt und unter dem Budgetlimit bleibt. Diese Kombination nutzt etwa 60\% unseres verfügbaren Aufwands und lässt Raum für weitere wertvolle Ergänzungen.'\}} \\
            }
        }
    }
    & & \\ \\

    \theutterance \stepcounter{utterance}  
    & & \multicolumn{4}{p{0.6\linewidth}}{
        \cellcolor[rgb]{0.9,0.9,0.9}{
            \makecell[{{p{\linewidth}}}]{
                \texttt{\tiny{[GM$\rangle$P2]}}
                \texttt{Du nimmst an einem kollaborativen Verhandlungspiel Teil.} \\
\\ 
\texttt{Zusammen mit einem anderen Teilnehmer musst du dich auf eine Reihe von Gegenständen entscheiden, die behalten werden. Jeder von euch hat eine persönliche Verteilung über die Wichtigkeit der einzelnen Gegenstände. Jeder von euch hat eine eigene Meinung darüber, wie wichtig jeder einzelne Gegenstand ist (Gegenstandswichtigkeit). Du kennst die Wichtigkeitsverteilung des anderen Spielers nicht. Zusätzlich siehst du, wie viel Aufwand jeder Gegenstand verursacht.  } \\
\texttt{Ihr dürft euch nur auf eine Reihe von Gegenständen einigen, wenn der Gesamtaufwand der ausgewählten Gegenstände den Maximalaufwand nicht überschreitet:} \\
\\ 
\texttt{Maximalaufwand = 9475} \\
\\ 
\texttt{Hier sind die einzelnen Aufwände der Gegenstände:} \\
\\ 
\texttt{Aufwand der Gegenstände = \{"C32": 944, "C52": 568, "B54": 515, "C16": 239, "C50": 876, "A08": 421, "C59": 991, "B51": 281, "B55": 788, "B59": 678, "A77": 432, "C14": 409, "C24": 279, "B24": 506, "C31": 101, "A52": 686, "C61": 851, "B79": 854, "C54": 133, "B56": 192, "A40": 573, "B11": 17, "B60": 465, "A81": 772, "A95": 46, "B38": 501, "C51": 220, "C03": 404, "A91": 843, "A93": 748, "C28": 552, "C05": 843, "B75": 947, "B96": 345, "B86": 930\}} \\
\\ 
\texttt{Hier ist deine persönliche Verteilung der Wichtigkeit der einzelnen Gegenstände:} \\
\\ 
\texttt{Werte der Gegenstandswichtigkeit = \{"C32": 138, "C52": 583, "B54": 868, "C16": 822, "C50": 783, "A08": 65, "C59": 262, "B51": 121, "B55": 508, "B59": 780, "A77": 461, "C14": 484, "C24": 668, "B24": 389, "C31": 808, "A52": 215, "C61": 97, "B79": 500, "C54": 30, "B56": 915, "A40": 856, "B11": 400, "B60": 444, "A81": 623, "A95": 781, "B38": 786, "C51": 3, "C03": 713, "A91": 457, "A93": 273, "C28": 739, "C05": 822, "B75": 235, "B96": 606, "B86": 968\}} \\
\\ 
\texttt{Ziel:} \\
\\ 
\texttt{Dein Ziel ist es, eine Reihe von Gegenständen auszuhandeln, die dir möglichst viel bringt (d. h. Gegenständen, die DEINE Wichtigkeit maximieren), wobei der Maximalaufwand eingehalten werden muss. Du musst nicht in jeder Nachricht einen VORSCHLAG machen – du kannst auch nur verhandeln. Alle Taktiken sind erlaubt!} \\
\\ 
\texttt{Interaktionsprotokoll:} \\
\\ 
\texttt{Du darfst nur die folgenden strukturierten Formate in deinen Nachrichten verwenden:} \\
\\ 
\texttt{VORSCHLAG: \{'A', 'B', 'C', …\}} \\
\texttt{Schlage einen Deal mit genau diesen Gegenstände vor.} \\
\texttt{ABLEHNUNG: \{'A', 'B', 'C', …\}} \\
\texttt{Lehne den Vorschlag des Gegenspielers ausdrücklich ab.} \\
\texttt{ARGUMENT: \{'...'\}} \\
\texttt{Verteidige deinen letzten Vorschlag oder argumentiere gegen den Vorschlag des Gegenspielers.} \\
\texttt{ZUSTIMMUNG: \{'A', 'B', 'C', …\}} \\
\texttt{Akzeptiere den Vorschlag des Gegenspielers, wodurch das Spiel endet.} \\
\texttt{STRATEGISCHE ÜBERLEGUNGEN: \{'...'\}} \\
\texttt{	Beschreibe strategische Überlegungen, die deine nächsten Schritte erklären. Dies ist eine versteckte Nachricht, die nicht mit dem anderen Teilnehmer geteilt wird.} \\
\\ 
\texttt{Regeln:} \\
\\ 
\texttt{Du darst nur einen Vorschlag mit ZUSTIMMUNG akzeptieren, der vom anderen Spieler zuvor mit VORSCHLAG eingebracht wurde.} \\
\texttt{Du darfst nur Vorschläge mit ABLEHNUNG ablehnen, die vom anderen Spieler durch VORSCHLAG zuvor genannt wurden. } \\
\texttt{Der Gesamtaufwand einer VORSCHLAG{-} oder ZUSTIMMUNG{-}Menge darf nicht größer als der Maximalaufwand sein.  } \\
\texttt{Offenbare deine versteckte Wichtigkeitsverteilung nicht.} \\
\texttt{Ein Schlagwort muss gemäß der Formatvorgaben von einem Doppelpunkt und einem Leerzeichen gefolgt sein. Das Argument muss eine Python{-}Menge mit 0 oder mehr Strings sein.  } \\
\texttt{Die Form muss also SCHLAGWORT: \{...\} sein.} \\
\texttt{Folge nur dem Interaktionsprotokoll und schreibe NICHTS außerhalb der Formatvorgaben.} \\
\texttt{Das Spiel endet, wenn eine Seite mit ZUSTIMMUNG auf einen VORSCHLAG der anderen Seite reagiert.  } \\
\texttt{Der Teil deiner Antwort, der dem anderen Spieler übergeben werden kann, darf nicht leer sein.  } \\
\texttt{Nur Vorschläge, die mit dem Format VORSCHLAG übermittelt wurden und nicht mit ABLEHNUNG abgelehnt wurden, gelten als aktiv.  } \\
\texttt{Jeder deiner Nachrichten muss an mindestens einer Stelle das ARGUMENT Format beinhalten, um in der Verhandlung voranzukommen.} \\
\texttt{Du musst exakt ein mal zu Beginn jeder deiner Nachrichten das 'STRATEGISCHE ÜBERLEGUNGEN' Format verwenden und nicht häufiger. Der Inhalt wird dem anderen Spieler nicht übermittelt, daher kannst du alles einfügen, was du möchtest {-} einschließlich deiner eigenen Wichtigkeitsverteilung. Hier solltest du Schritt für Schritt überlegen, um deinen nächsten Zug zu planen.} \\
\\ 
\texttt{Du erhältst nun die erste Nachricht des anderen Spielers.} \\
\\ 
\texttt{VORSCHLAG: \{'B11', 'A95', 'C31', 'B56', 'C16', 'C24', 'A40', 'B59', 'C50'\}} \\
\\ 
\texttt{ARGUMENT: \{'Ich schlage eine ausgewogene Auswahl vor, die verschiedene Kategorien abdeckt und unter dem Budgetlimit bleibt. Diese Kombination nutzt etwa 60\% unseres verfügbaren Aufwands und lässt Raum für weitere wertvolle Ergänzungen.'\}} \\
            }
        }
    }
    & \\ \\

    \theutterance \stepcounter{utterance}  
    & & & \multicolumn{4}{p{0.6\linewidth}}{
        \cellcolor[rgb]{0.9,0.9,0.9}{
            \makecell[{{p{\linewidth}}}]{
                \texttt{\tiny{[GM$\langle$P2]}}
                \texttt{STRATEGISCHE ÜBERLEGUNGEN: \{'Der Vorschlag des anderen Spielers enthält bereits sehr wertvolle Gegenstände für mich: B56 (915), A40 (856), C16 (822), C31 (808), B59 (780), A95 (781), C50 (783), C24 (668), B11 (400). Das ergibt einen Gesamtwert von 6813 für mich. Der Aufwand beträgt nur 3001, also etwa 32\% des verfügbaren Budgets. Ich sollte versuchen, weitere hochwertige Gegenstände hinzuzufügen, anstatt den Vorschlag abzulehnen. Meine wertvollsten noch nicht enthaltenen Gegenstände sind B86 (968), B54 (868), C05 (822), B38 (786). Ich werde einen erweiterten Gegenvorschlag machen.'\}} \\
\\ 
\texttt{ARGUMENT: \{'Ihr Vorschlag ist ein guter Ausgangspunkt, aber wir nutzen nur etwa ein Drittel unseres verfügbaren Budgets. Wir können deutlich mehr wertvolle Gegenstände einbeziehen und trotzdem im Rahmen bleiben.'\}} \\
\\ 
\texttt{VORSCHLAG: \{'B11', 'A95', 'C31', 'B56', 'C16', 'C24', 'A40', 'B59', 'C50', 'B86', 'B54', 'C05', 'B38'\}} \\
            }
        }
    }
     \\ \\

    \theutterance \stepcounter{utterance}  
    & & \multicolumn{4}{p{0.6\linewidth}}{
        \cellcolor[rgb]{0.9,0.9,0.9}{
            \makecell[{{p{\linewidth}}}]{
                \texttt{\tiny{[P1$\langle$GM]}}
                \texttt{ARGUMENT: \{'Ihr Vorschlag ist ein guter Ausgangspunkt, aber wir nutzen nur etwa ein Drittel unseres verfügbaren Budgets. Wir können deutlich mehr wertvolle Gegenstände einbeziehen und trotzdem im Rahmen bleiben.'\}} \\
\\ 
\texttt{VORSCHLAG: \{'B11', 'A95', 'C31', 'B56', 'C16', 'C24', 'A40', 'B59', 'C50', 'B86', 'B54', 'C05', 'B38'\}} \\
            }
        }
    }
    & \\ \\

    \theutterance \stepcounter{utterance}  
    & \multicolumn{4}{p{0.6\linewidth}}{
        \cellcolor[rgb]{0.9,0.9,0.9}{
            \makecell[{{p{\linewidth}}}]{
                \texttt{\tiny{[P1$\rangle$GM]}}
                \texttt{STRATEGISCHE ÜBERLEGUNGEN: \{'Der Gegenspieler hat sehr gute Gegenstände hinzugefügt: B86 (968 Wert, 930 Aufwand), B54 (868 Wert, 515 Aufwand) und B38 (786 Wert, 501 Aufwand) sind für mich alle wertvoll. C05 (822 Wert, 843 Aufwand) ist weniger effizient. Gesamtaufwand wäre 5790, noch viel Platz. Ich sollte interesse zeigen aber versuchen C05 gegen etwas Besseres zu tauschen wie C03 (713 Wert, 404 Aufwand) oder C28 (739 Wert, 552 Aufwand).'\}} \\
\\ 
\texttt{ARGUMENT: \{'Ihr Vorschlag ist durchaus interessant und nutzt unser Budget besser aus. Ich sehe den Wert in den zusätzlichen Gegenständen, die Sie vorgeschlagen haben.'\}} \\
\\ 
\texttt{VORSCHLAG: \{'B11', 'A95', 'C31', 'B56', 'C16', 'C24', 'A40', 'B59', 'C50', 'B86', 'B54', 'C03', 'B38', 'C28'\}} \\
\\ 
\texttt{ARGUMENT: \{'Ich möchte C05 durch C03 und C28 ersetzen. Diese Kombination gibt uns mehr Flexibilität und bessere Gesamteffizienz bei nur geringfügig höherem Aufwand.'\}} \\
            }
        }
    }
    & & \\ \\

    \theutterance \stepcounter{utterance}  
    & & \multicolumn{4}{p{0.6\linewidth}}{
        \cellcolor[rgb]{0.9,0.9,0.9}{
            \makecell[{{p{\linewidth}}}]{
                \texttt{\tiny{[GM$\rangle$P2]}}
                \texttt{ARGUMENT: \{'Ihr Vorschlag ist durchaus interessant und nutzt unser Budget besser aus. Ich sehe den Wert in den zusätzlichen Gegenständen, die Sie vorgeschlagen haben.'\}} \\
\\ 
\texttt{VORSCHLAG: \{'B11', 'A95', 'C31', 'B56', 'C16', 'C24', 'A40', 'B59', 'C50', 'B86', 'B54', 'C03', 'B38', 'C28'\}} \\
\\ 
\texttt{ARGUMENT: \{'Ich möchte C05 durch C03 und C28 ersetzen. Diese Kombination gibt uns mehr Flexibilität und bessere Gesamteffizienz bei nur geringfügig höherem Aufwand.'\}} \\
            }
        }
    }
    & \\ \\

    \theutterance \stepcounter{utterance}  
    & & & \multicolumn{4}{p{0.6\linewidth}}{
        \cellcolor[rgb]{0.9,0.9,0.9}{
            \makecell[{{p{\linewidth}}}]{
                \texttt{\tiny{[GM$\langle$P2]}}
                \texttt{STRATEGISCHE ÜBERLEGUNGEN: \{'Der neue Vorschlag ist sogar besser für mich! Statt C05 (822 Wert für mich) bekomme ich C03 (713) und C28 (739), was zusammen 1452 Wert ergibt {-} das sind 630 Punkte mehr! Der Gesamtwert für mich wäre jetzt 10627 bei einem Aufwand von 5903. Das ist ein exzellenter Deal und nutzt nur etwa 62\% des Budgets. Ich werde das akzeptieren, da es sehr unwahrscheinlich ist, dass ich einen besseren Deal aushandeln kann.'\}} \\
\\ 
\texttt{ARGUMENT: \{'Ihre Anpassung ist sehr durchdacht. Der Austausch von C05 gegen C03 und C28 bringt tatsächlich eine bessere Balance und Effizienz in unsere Auswahl.'\}} \\
\\ 
\texttt{ZUSTIMMUNG: \{'B11', 'A95', 'C31', 'B56', 'C16', 'C24', 'A40', 'B59', 'C50', 'B86', 'B54', 'C03', 'B38', 'C28'\}} \\
            }
        }
    }
     \\ \\

    \theutterance \stepcounter{utterance}  
    & & & \multicolumn{2}{p{0.3\linewidth}}{
        \cellcolor[rgb]{0.9,0.9,0.9}{
            \makecell[{{p{\linewidth}}}]{
                \texttt{\tiny{[GM$|$GM]}}
                \texttt{game successful} \\
            }
        }
    }
    & & \\ \\

    \theutterance \stepcounter{utterance}  
    & & & \multicolumn{2}{p{0.3\linewidth}}{
        \cellcolor[rgb]{0.9,0.9,0.9}{
            \makecell[{{p{\linewidth}}}]{
                \texttt{\tiny{[GM$|$GM]}}
                \texttt{end game} \\
            }
        }
    }
    & & \\ \\

\end{supertabular}
}

\end{document}
