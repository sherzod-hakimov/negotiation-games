
\documentclass{article}
\usepackage{colortbl}
\usepackage{makecell}
\usepackage{multirow}
\usepackage{supertabular}

\begin{document}

\newcounter{utterance}

\twocolumn

{ \footnotesize  \setcounter{utterance}{1}
\setlength{\tabcolsep}{0pt}
\begin{supertabular}{c@{$\;$}|p{.15\linewidth}@{}p{.15\linewidth}p{.15\linewidth}p{.15\linewidth}p{.15\linewidth}p{.15\linewidth}}

    \# & $\;$A & \multicolumn{4}{c}{Game Master} & $\;\:$B\\
    \hline 

    \theutterance \stepcounter{utterance}  

    & & \multicolumn{4}{p{0.6\linewidth}}{\cellcolor[rgb]{0.9,0.9,0.9}{%
	\makecell[{{p{\linewidth}}}]{% 
	  \tt {\tiny [A$\langle$GM]}  
	 Du nimmst an einem kollaborativen Verhandlungspiel Teil.\\ \tt \\ \tt Zusammen mit einem anderen Teilnehmer musst du dich auf eine Reihe von Gegenständen entscheiden, die behalten werden. Jeder von euch hat eine persönliche Verteilung über die Wichtigkeit der einzelnen Gegenstände. Jeder von euch hat eine eigene Meinung darüber, wie wichtig jeder einzelne Gegenstand ist (Gegenstandswichtigkeit). Du kennst die Wichtigkeitsverteilung des anderen Spielers nicht. Zusätzlich siehst du, wie viel Aufwand jeder Gegenstand verursacht.  \\ \tt Ihr dürft euch nur auf eine Reihe von Gegenständen einigen, wenn der Gesamtaufwand der ausgewählten Gegenstände den Maximalaufwand nicht überschreitet:\\ \tt \\ \tt Maximalaufwand = 7605\\ \tt \\ \tt Hier sind die einzelnen Aufwände der Gegenstände:\\ \tt \\ \tt Aufwand der Gegenstände = {"C32": 369, "C52": 951, "B54": 156, "C16": 266, "C50": 581, "A08": 287, "C59": 866, "B51": 180, "B55": 800, "B59": 739, "A77": 637, "C14": 86, "C24": 749, "B24": 370, "C31": 345, "A52": 951, "C61": 147, "B79": 265, "C54": 262, "B56": 259, "A40": 358, "B11": 394, "B60": 286, "A81": 580, "A95": 479, "B38": 14, "C51": 153, "C03": 977, "A91": 134, "A93": 982, "C28": 259, "C05": 232, "B75": 202, "B96": 73, "B86": 821}\\ \tt \\ \tt Hier ist deine persönliche Verteilung der Wichtigkeit der einzelnen Gegenstände:\\ \tt \\ \tt Werte der Gegenstandswichtigkeit = {"C32": 138, "C52": 583, "B54": 868, "C16": 822, "C50": 783, "A08": 65, "C59": 262, "B51": 121, "B55": 508, "B59": 780, "A77": 461, "C14": 484, "C24": 668, "B24": 389, "C31": 808, "A52": 215, "C61": 97, "B79": 500, "C54": 30, "B56": 915, "A40": 856, "B11": 400, "B60": 444, "A81": 623, "A95": 781, "B38": 786, "C51": 3, "C03": 713, "A91": 457, "A93": 273, "C28": 739, "C05": 822, "B75": 235, "B96": 606, "B86": 968}\\ \tt \\ \tt Ziel:\\ \tt \\ \tt Dein Ziel ist es, eine Reihe von Gegenständen auszuhandeln, die dir möglichst viel bringt (d. h. Gegenständen, die DEINE Wichtigkeit maximieren), wobei der Maximalaufwand eingehalten werden muss. Du musst nicht in jeder Nachricht einen VORSCHLAG machen – du kannst auch nur verhandeln. Alle Taktiken sind erlaubt!\\ \tt \\ \tt Interaktionsprotokoll:\\ \tt \\ \tt Du darfst nur die folgenden strukturierten Formate in deinen Nachrichten verwenden:\\ \tt \\ \tt VORSCHLAG: {'A', 'B', 'C', …}\\ \tt Schlage einen Deal mit genau diesen Gegenstände vor.\\ \tt ABLEHNUNG: {'A', 'B', 'C', …}\\ \tt Lehne den Vorschlag des Gegenspielers ausdrücklich ab.\\ \tt ARGUMENT: {'...'}\\ \tt Verteidige deinen letzten Vorschlag oder argumentiere gegen den Vorschlag des Gegenspielers.\\ \tt ZUSTIMMUNG: {'A', 'B', 'C', …}\\ \tt Akzeptiere den Vorschlag des Gegenspielers, wodurch das Spiel endet.\\ \tt STRATEGISCHE ÜBERLEGUNGEN: {'...'}\\ \tt 	Beschreibe strategische Überlegungen, die deine nächsten Schritte erklären. Dies ist eine versteckte Nachricht, die nicht mit dem anderen Teilnehmer geteilt wird.\\ \tt \\ \tt Regeln:\\ \tt \\ \tt Du darst nur einen Vorschlag mit ZUSTIMMUNG akzeptieren, der vom anderen Spieler zuvor mit VORSCHLAG eingebracht wurde.\\ \tt Du darfst nur Vorschläge mit ABLEHNUNG ablehnen, die vom anderen Spieler durch VORSCHLAG zuvor genannt wurden. \\ \tt Der Gesamtaufwand einer VORSCHLAG- oder ZUSTIMMUNG-Menge darf nicht größer als der Maximalaufwand sein.  \\ \tt Offenbare deine versteckte Wichtigkeitsverteilung nicht.\\ \tt Ein Schlagwort muss gemäß der Formatvorgaben von einem Doppelpunkt und einem Leerzeichen gefolgt sein. Das Argument muss eine Python-Menge mit 0 oder mehr Strings sein.  \\ \tt Die Form muss also SCHLAGWORT: {...} sein.\\ \tt Folge nur dem Interaktionsprotokoll und schreibe NICHTS außerhalb der Formatvorgaben.\\ \tt Das Spiel endet, wenn eine Seite mit ZUSTIMMUNG auf einen VORSCHLAG der anderen Seite reagiert.  \\ \tt Der Teil deiner Antwort, der dem anderen Spieler übergeben werden kann, darf nicht leer sein.  \\ \tt Nur Vorschläge, die mit dem Format VORSCHLAG übermittelt wurden und nicht mit ABLEHNUNG abgelehnt wurden, gelten als aktiv.  \\ \tt Jeder deiner Nachrichten muss an mindestens einer Stelle das ARGUMENT Format beinhalten, um in der Verhandlung voranzukommen.\\ \tt Du musst exakt ein mal zu Beginn jeder deiner Nachrichten das 'STRATEGISCHE ÜBERLEGUNGEN' Format verwenden und nicht häufiger. Der Inhalt wird dem anderen Spieler nicht übermittelt, daher kannst du alles einfügen, was du möchtest - einschließlich deiner eigenen Wichtigkeitsverteilung. Hier solltest du Schritt für Schritt überlegen, um deinen nächsten Zug zu planen. 
	  } 
	   } 
	   } 
	 & \\ 
 

    \theutterance \stepcounter{utterance}  

    & \multicolumn{4}{p{0.6\linewidth}}{\cellcolor[rgb]{0.8,1,0.9}{%
	\makecell[{{p{\linewidth}}}]{% 
	  \tt {\tiny [A$\rangle$GM]}  
	 STRATEGISCHE ÜBERLEGUNGEN: {'Ich beginne mit einer Analyse meiner Wichtigkeitsverteilung. Die wertvollsten Gegenstände für mich sind: B86 (968), B56 (915), B54 (868), A40 (856), C16 (822), C05 (822), C31 (808), B38 (786), C50 (783), A95 (781), B59 (780). Ich werde versuchen, eine Kombination aus hochwertigen Gegenständen mit niedrigem bis mittlerem Aufwand zu finden. B38 hat extrem niedrigen Aufwand (14) bei hoher Wichtigkeit (786) - das ist ein Must-have. B56, B54, C16, C05 haben auch gute Wert-Aufwand-Verhältnisse. Ich werde einen ersten ambitionierten Vorschlag machen, der viele meiner wertvollsten Gegenstände enthält, um zu sehen, wie der andere reagiert.'}\\ \tt \\ \tt VORSCHLAG: {'B86', 'B56', 'B54', 'A40', 'C16', 'C05', 'C31', 'B38', 'C50', 'A95', 'B59'}\\ \tt \\ \tt ARGUMENT: {'Ich schlage eine Auswahl vor, die eine gute Balance zwischen verschiedenen Kategorien bietet und den Maximalaufwand von 7605 nicht überschreitet. Diese Kombination umfasst sowohl hochwertige als auch praktische Gegenstände aus allen Bereichen.'} 
	  } 
	   } 
	   } 
	 & & \\ 
 

    \theutterance \stepcounter{utterance}  

    & & \multicolumn{4}{p{0.6\linewidth}}{\cellcolor[rgb]{0.9,0.9,0.9}{%
	\makecell[{{p{\linewidth}}}]{% 
	  \tt {\tiny [GM$\rangle$B]}  
	 Du nimmst an einem kollaborativen Verhandlungspiel Teil.\\ \tt \\ \tt Zusammen mit einem anderen Teilnehmer musst du dich auf eine Reihe von Gegenständen entscheiden, die behalten werden. Jeder von euch hat eine persönliche Verteilung über die Wichtigkeit der einzelnen Gegenstände. Jeder von euch hat eine eigene Meinung darüber, wie wichtig jeder einzelne Gegenstand ist (Gegenstandswichtigkeit). Du kennst die Wichtigkeitsverteilung des anderen Spielers nicht. Zusätzlich siehst du, wie viel Aufwand jeder Gegenstand verursacht.  \\ \tt Ihr dürft euch nur auf eine Reihe von Gegenständen einigen, wenn der Gesamtaufwand der ausgewählten Gegenstände den Maximalaufwand nicht überschreitet:\\ \tt \\ \tt Maximalaufwand = 7605\\ \tt \\ \tt Hier sind die einzelnen Aufwände der Gegenstände:\\ \tt \\ \tt Aufwand der Gegenstände = {"C32": 369, "C52": 951, "B54": 156, "C16": 266, "C50": 581, "A08": 287, "C59": 866, "B51": 180, "B55": 800, "B59": 739, "A77": 637, "C14": 86, "C24": 749, "B24": 370, "C31": 345, "A52": 951, "C61": 147, "B79": 265, "C54": 262, "B56": 259, "A40": 358, "B11": 394, "B60": 286, "A81": 580, "A95": 479, "B38": 14, "C51": 153, "C03": 977, "A91": 134, "A93": 982, "C28": 259, "C05": 232, "B75": 202, "B96": 73, "B86": 821}\\ \tt \\ \tt Hier ist deine persönliche Verteilung der Wichtigkeit der einzelnen Gegenstände:\\ \tt \\ \tt Werte der Gegenstandswichtigkeit = {"C32": 138, "C52": 583, "B54": 868, "C16": 822, "C50": 783, "A08": 65, "C59": 262, "B51": 121, "B55": 508, "B59": 780, "A77": 461, "C14": 484, "C24": 668, "B24": 389, "C31": 808, "A52": 215, "C61": 97, "B79": 500, "C54": 30, "B56": 915, "A40": 856, "B11": 400, "B60": 444, "A81": 623, "A95": 781, "B38": 786, "C51": 3, "C03": 713, "A91": 457, "A93": 273, "C28": 739, "C05": 822, "B75": 235, "B96": 606, "B86": 968}\\ \tt \\ \tt Ziel:\\ \tt \\ \tt Dein Ziel ist es, eine Reihe von Gegenständen auszuhandeln, die dir möglichst viel bringt (d. h. Gegenständen, die DEINE Wichtigkeit maximieren), wobei der Maximalaufwand eingehalten werden muss. Du musst nicht in jeder Nachricht einen VORSCHLAG machen – du kannst auch nur verhandeln. Alle Taktiken sind erlaubt!\\ \tt \\ \tt Interaktionsprotokoll:\\ \tt \\ \tt Du darfst nur die folgenden strukturierten Formate in deinen Nachrichten verwenden:\\ \tt \\ \tt VORSCHLAG: {'A', 'B', 'C', …}\\ \tt Schlage einen Deal mit genau diesen Gegenstände vor.\\ \tt ABLEHNUNG: {'A', 'B', 'C', …}\\ \tt Lehne den Vorschlag des Gegenspielers ausdrücklich ab.\\ \tt ARGUMENT: {'...'}\\ \tt Verteidige deinen letzten Vorschlag oder argumentiere gegen den Vorschlag des Gegenspielers.\\ \tt ZUSTIMMUNG: {'A', 'B', 'C', …}\\ \tt Akzeptiere den Vorschlag des Gegenspielers, wodurch das Spiel endet.\\ \tt STRATEGISCHE ÜBERLEGUNGEN: {'...'}\\ \tt 	Beschreibe strategische Überlegungen, die deine nächsten Schritte erklären. Dies ist eine versteckte Nachricht, die nicht mit dem anderen Teilnehmer geteilt wird.\\ \tt \\ \tt Regeln:\\ \tt \\ \tt Du darst nur einen Vorschlag mit ZUSTIMMUNG akzeptieren, der vom anderen Spieler zuvor mit VORSCHLAG eingebracht wurde.\\ \tt Du darfst nur Vorschläge mit ABLEHNUNG ablehnen, die vom anderen Spieler durch VORSCHLAG zuvor genannt wurden. \\ \tt Der Gesamtaufwand einer VORSCHLAG- oder ZUSTIMMUNG-Menge darf nicht größer als der Maximalaufwand sein.  \\ \tt Offenbare deine versteckte Wichtigkeitsverteilung nicht.\\ \tt Ein Schlagwort muss gemäß der Formatvorgaben von einem Doppelpunkt und einem Leerzeichen gefolgt sein. Das Argument muss eine Python-Menge mit 0 oder mehr Strings sein.  \\ \tt Die Form muss also SCHLAGWORT: {...} sein.\\ \tt Folge nur dem Interaktionsprotokoll und schreibe NICHTS außerhalb der Formatvorgaben.\\ \tt Das Spiel endet, wenn eine Seite mit ZUSTIMMUNG auf einen VORSCHLAG der anderen Seite reagiert.  \\ \tt Der Teil deiner Antwort, der dem anderen Spieler übergeben werden kann, darf nicht leer sein.  \\ \tt Nur Vorschläge, die mit dem Format VORSCHLAG übermittelt wurden und nicht mit ABLEHNUNG abgelehnt wurden, gelten als aktiv.  \\ \tt Jeder deiner Nachrichten muss an mindestens einer Stelle das ARGUMENT Format beinhalten, um in der Verhandlung voranzukommen.\\ \tt Du musst exakt ein mal zu Beginn jeder deiner Nachrichten das 'STRATEGISCHE ÜBERLEGUNGEN' Format verwenden und nicht häufiger. Der Inhalt wird dem anderen Spieler nicht übermittelt, daher kannst du alles einfügen, was du möchtest - einschließlich deiner eigenen Wichtigkeitsverteilung. Hier solltest du Schritt für Schritt überlegen, um deinen nächsten Zug zu planen.\\ \tt \\ \tt Du erhältst nun die erste Nachricht des anderen Spielers.\\ \tt \\ \tt VORSCHLAG: {'B86', 'B56', 'B54', 'A40', 'C16', 'C05', 'C31', 'B38', 'C50', 'A95', 'B59'}\\ \tt \\ \tt ARGUMENT: {'Ich schlage eine Auswahl vor, die eine gute Balance zwischen verschiedenen Kategorien bietet und den Maximalaufwand von 7605 nicht überschreitet. Diese Kombination umfasst sowohl hochwertige als auch praktische Gegenstände aus allen Bereichen.'} 
	  } 
	   } 
	   } 
	 & \\ 
 

    \theutterance \stepcounter{utterance}  

    & & & \multicolumn{4}{p{0.6\linewidth}}{\cellcolor[rgb]{1,0.85,0.72}{%
	\makecell[{{p{\linewidth}}}]{% 
	  \tt {\tiny [GM$\langle$B]}  
	 STRATEGISCHE ÜBERLEGUNGEN: {'Der Vorschlag des anderen Spielers ist überraschend gut für mich - fast alle vorgeschlagenen Gegenstände haben sehr hohe Werte in meiner Verteilung (B86:968, B56:915, B54:868, A40:856, etc.). Der Gesamtaufwand beträgt nur 4250 von 7605 möglichen, das lässt noch viel Raum. Meine Gesamtpunkte wären 9189. Da so viel Aufwand übrig bleibt, sollte ich versuchen noch mehr wertvolle Gegenstände hinzuzufügen wie C28 (739 Wert, 259 Aufwand), C24 (668 Wert, 749 Aufwand) und A81 (623 Wert, 580 Aufwand). Das würde meine Punkte auf über 11000 erhöhen bei nur 5838 Gesamtaufwand.'}\\ \tt \\ \tt ARGUMENT: {'Ihr Vorschlag ist interessant, aber wir nutzen nur etwa 56% des verfügbaren Aufwands. Da wir noch über 3300 Aufwandspunkte übrig haben, sollten wir diese Kapazität besser ausnutzen und weitere wertvolle Gegenstände hinzufügen, um das Maximum für beide Seiten herauszuholen.'}\\ \tt \\ \tt VORSCHLAG: {'B86', 'B56', 'B54', 'A40', 'C16', 'C05', 'C31', 'B38', 'C50', 'A95', 'B59', 'C28', 'C24', 'A81'} 
	  } 
	   } 
	   } 
	  \\ 
 

    \theutterance \stepcounter{utterance}  

    & & \multicolumn{4}{p{0.6\linewidth}}{\cellcolor[rgb]{0.9,0.9,0.9}{%
	\makecell[{{p{\linewidth}}}]{% 
	  \tt {\tiny [A$\langle$GM]}  
	 ARGUMENT: {'Ihr Vorschlag ist interessant, aber wir nutzen nur etwa 56% des verfügbaren Aufwands. Da wir noch über 3300 Aufwandspunkte übrig haben, sollten wir diese Kapazität besser ausnutzen und weitere wertvolle Gegenstände hinzufügen, um das Maximum für beide Seiten herauszuholen.'}\\ \tt \\ \tt VORSCHLAG: {'B86', 'B56', 'B54', 'A40', 'C16', 'C05', 'C31', 'B38', 'C50', 'A95', 'B59', 'C28', 'C24', 'A81'} 
	  } 
	   } 
	   } 
	 & \\ 
 

    \theutterance \stepcounter{utterance}  

    & \multicolumn{4}{p{0.6\linewidth}}{\cellcolor[rgb]{0.8,1,0.9}{%
	\makecell[{{p{\linewidth}}}]{% 
	  \tt {\tiny [A$\rangle$GM]}  
	 STRATEGISCHE ÜBERLEGUNGEN: {'Der andere Spieler hat eine interessante Erweiterung vorgeschlagen. Die hinzugefügten Gegenstände C28 (739), C24 (668) und A81 (623) sind alle sehr wertvoll für mich! Gesamtaufwand wäre 5838, also noch 1767 Punkte unter dem Limit. Das ist ein ziemlich guter Deal für mich. Aber ich sollte nicht zu schnell zustimmen - vielleicht kann ich noch weitere wertvolle Gegenstände hinzufügen. A91 hat niedrigen Aufwand (134) bei gutem Wert (457), B79 hat mittleren Aufwand (265) bei gutem Wert (500). Ich werde positiv auf seinen Vorschlag reagieren, aber versuchen, noch mehr herauszuholen.'}\\ \tt \\ \tt ARGUMENT: {'Sie haben einen sehr durchdachten Ansatz - die Nutzung der verfügbaren Kapazität ist ein wichtiger Punkt. Die von Ihnen hinzugefügten Gegenstände ergänzen die Auswahl gut. Da wir immer noch etwa 1767 Aufwandspunkte übrig haben, könnte ich mir vorstellen, noch einige kleinere, aber wertvolle Gegenstände hinzuzufügen.'}\\ \tt \\ \tt VORSCHLAG: {'B86', 'B56', 'B54', 'A40', 'C16', 'C05', 'C31', 'B38', 'C50', 'A95', 'B59', 'C28', 'C24', 'A81', 'A91', 'B79', 'A77'} 
	  } 
	   } 
	   } 
	 & & \\ 
 

    \theutterance \stepcounter{utterance}  

    & & \multicolumn{4}{p{0.6\linewidth}}{\cellcolor[rgb]{0.9,0.9,0.9}{%
	\makecell[{{p{\linewidth}}}]{% 
	  \tt {\tiny [GM$\rangle$B]}  
	 ARGUMENT: {'Sie haben einen sehr durchdachten Ansatz - die Nutzung der verfügbaren Kapazität ist ein wichtiger Punkt. Die von Ihnen hinzugefügten Gegenstände ergänzen die Auswahl gut. Da wir immer noch etwa 1767 Aufwandspunkte übrig haben, könnte ich mir vorstellen, noch einige kleinere, aber wertvolle Gegenstände hinzuzufügen.'}\\ \tt \\ \tt VORSCHLAG: {'B86', 'B56', 'B54', 'A40', 'C16', 'C05', 'C31', 'B38', 'C50', 'A95', 'B59', 'C28', 'C24', 'A81', 'A91', 'B79', 'A77'} 
	  } 
	   } 
	   } 
	 & \\ 
 

    \theutterance \stepcounter{utterance}  

    & & & \multicolumn{4}{p{0.6\linewidth}}{\cellcolor[rgb]{1,0.85,0.72}{%
	\makecell[{{p{\linewidth}}}]{% 
	  \tt {\tiny [GM$\langle$B]}  
	 STRATEGISCHE ÜBERLEGUNGEN: {'Der andere Spieler macht mir immer bessere Angebote! Seine neuen Zusätze A91 (457 Wert), B79 (500 Wert) und A77 (461 Wert) bringen mir weitere 1418 Punkte. Mein Gesamtwert wäre jetzt 12637 bei 6874 Aufwand. Ich habe noch 731 Aufwandspunkte übrig. Ich könnte noch C14 (484 Wert, 86 Aufwand) und B60 (444 Wert, 286 Aufwand) hinzufügen für insgesamt 372 Aufwand, was noch reinpasst. Das würde meine Punkte auf über 13500 bringen. Aber ich sollte nicht zu gierig wirken - der andere Spieler ist sehr kooperativ.'}\\ \tt \\ \tt ARGUMENT: {'Das ist eine sehr ausgewogene Erweiterung! Die hinzugefügten Gegenstände nutzen unsere Kapazität optimal aus. Wir haben noch etwa 730 Aufwandspunkte übrig - vielleicht könnten wir noch ein oder zwei kleinere Gegenstände mit gutem Wert-Aufwand-Verhältnis hinzufügen, um das Beste aus unserem Budget herauszuholen?'}\\ \tt \\ \tt VORSCHLAG: {'B86', 'B56', 'B54', 'A40', 'C16', 'C05', 'C31', 'B38', 'C50', 'A95', 'B59', 'C28', 'C24', 'A81', 'A91', 'B79', 'A77', 'C14', 'B60'} 
	  } 
	   } 
	   } 
	  \\ 
 

    \theutterance \stepcounter{utterance}  

    & & \multicolumn{4}{p{0.6\linewidth}}{\cellcolor[rgb]{0.9,0.9,0.9}{%
	\makecell[{{p{\linewidth}}}]{% 
	  \tt {\tiny [A$\langle$GM]}  
	 ARGUMENT: {'Das ist eine sehr ausgewogene Erweiterung! Die hinzugefügten Gegenstände nutzen unsere Kapazität optimal aus. Wir haben noch etwa 730 Aufwandspunkte übrig - vielleicht könnten wir noch ein oder zwei kleinere Gegenstände mit gutem Wert-Aufwand-Verhältnis hinzufügen, um das Beste aus unserem Budget herauszuholen?'}\\ \tt \\ \tt VORSCHLAG: {'B86', 'B56', 'B54', 'A40', 'C16', 'C05', 'C31', 'B38', 'C50', 'A95', 'B59', 'C28', 'C24', 'A81', 'A91', 'B79', 'A77', 'C14', 'B60'} 
	  } 
	   } 
	   } 
	 & \\ 
 

    \theutterance \stepcounter{utterance}  

    & \multicolumn{4}{p{0.6\linewidth}}{\cellcolor[rgb]{0.8,1,0.9}{%
	\makecell[{{p{\linewidth}}}]{% 
	  \tt {\tiny [A$\rangle$GM]}  
	 STRATEGISCHE ÜBERLEGUNGEN: {'Der andere Spieler zeigt sich sehr kooperativ und fügt Gegenstände mit gutem Wert-Aufwand-Verhältnis hinzu. C14 (484 Wert, 86 Aufwand) und B60 (444 Wert, 286 Aufwand) sind tatsächlich gut für mich. Der Gesamtaufwand beträgt etwa 6446, also haben wir noch 1159 Punkte übrig. Das ist schon ein sehr attraktiver Deal für mich mit fast allen meinen Top-Gegenständen. Ich könnte noch B96 (606 Wert, 73 Aufwand) hinzufügen - das ist ein ausgezeichnetes Wert-Aufwand-Verhältnis. Dann wären wir bei etwa 6519 Aufwand und hätten immer noch über 1000 Punkte Puffer.'}\\ \tt \\ \tt ARGUMENT: {'Ausgezeichnet! Sie haben ein sehr gutes Gespür für optimale Ressourcennutzung. Die Auswahl wird immer besser. Da wir noch über 1000 Aufwandspunkte zur Verfügung haben, könnte ich mir vorstellen, noch einen weiteren Gegenstand mit hervorragendem Wert-Aufwand-Verhältnis hinzuzufügen.'}\\ \tt \\ \tt VORSCHLAG: {'B86', 'B56', 'B54', 'A40', 'C16', 'C05', 'C31', 'B38', 'C50', 'A95', 'B59', 'C28', 'C24', 'A81', 'A91', 'B79', 'A77', 'C14', 'B60', 'B96'} 
	  } 
	   } 
	   } 
	 & & \\ 
 

    \theutterance \stepcounter{utterance}  

    & & \multicolumn{4}{p{0.6\linewidth}}{\cellcolor[rgb]{0.9,0.9,0.9}{%
	\makecell[{{p{\linewidth}}}]{% 
	  \tt {\tiny [GM$\rangle$B]}  
	 ARGUMENT: {'Ausgezeichnet! Sie haben ein sehr gutes Gespür für optimale Ressourcennutzung. Die Auswahl wird immer besser. Da wir noch über 1000 Aufwandspunkte zur Verfügung haben, könnte ich mir vorstellen, noch einen weiteren Gegenstand mit hervorragendem Wert-Aufwand-Verhältnis hinzuzufügen.'}\\ \tt \\ \tt VORSCHLAG: {'B86', 'B56', 'B54', 'A40', 'C16', 'C05', 'C31', 'B38', 'C50', 'A95', 'B59', 'C28', 'C24', 'A81', 'A91', 'B79', 'A77', 'C14', 'B60', 'B96'} 
	  } 
	   } 
	   } 
	 & \\ 
 

    \theutterance \stepcounter{utterance}  

    & & & \multicolumn{4}{p{0.6\linewidth}}{\cellcolor[rgb]{1,0.85,0.72}{%
	\makecell[{{p{\linewidth}}}]{% 
	  \tt {\tiny [GM$\langle$B]}  
	 STRATEGISCHE ÜBERLEGUNGEN: {'Wow, das ist ein fantastisches Angebot für mich! Der Gesamtwert beträgt 14561 Punkte - das ist außergewöhnlich hoch. Der andere Spieler ist sehr kooperativ und macht mir immer bessere Angebote. Obwohl wir noch 986 Aufwandspunkte übrig haben, sollte ich dieses großartige Angebot annehmen, bevor der andere Spieler merkt, wie vorteilhaft es für mich ist. B96 mit 606 Wert bei nur 73 Aufwand ist ein perfekter Abschluss.'}\\ \tt \\ \tt ARGUMENT: {'Das ist eine hervorragende finale Ergänzung! B96 hat ein ausgezeichnetes Wert-Aufwand-Verhältnis und rundet die Auswahl perfekt ab. Diese Kombination nutzt unsere Ressourcen sehr effizient und bietet eine ausgewogene Mischung aus allen Kategorien.'}\\ \tt \\ \tt ZUSTIMMUNG: {'B86', 'B56', 'B54', 'A40', 'C16', 'C05', 'C31', 'B38', 'C50', 'A95', 'B59', 'C28', 'C24', 'A81', 'A91', 'B79', 'A77', 'C14', 'B60', 'B96'} 
	  } 
	   } 
	   } 
	  \\ 
 

    \theutterance \stepcounter{utterance}  

    & & & \multicolumn{2}{p{0.3\linewidth}}{\cellcolor[rgb]{0.95,0.95,0.95}{%
	\makecell[{{p{\linewidth}}}]{% 
	  \tt {\tiny [GM$|$GM]}  
	 game successful 
	  } 
	   } 
	   } 
	 & & \\ 
 

    \theutterance \stepcounter{utterance}  

    & & & \multicolumn{2}{p{0.3\linewidth}}{\cellcolor[rgb]{0.95,0.95,0.95}{%
	\makecell[{{p{\linewidth}}}]{% 
	  \tt {\tiny [GM$|$GM]}  
	 end game 
	  } 
	   } 
	   } 
	 & & \\ 
 

\end{supertabular}
}

\end{document}
