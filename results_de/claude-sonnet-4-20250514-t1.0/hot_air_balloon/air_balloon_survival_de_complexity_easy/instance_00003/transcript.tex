
\documentclass{article}
\usepackage{colortbl}
\usepackage{makecell}
\usepackage{multirow}
\usepackage{supertabular}

\begin{document}

\newcounter{utterance}

\centering \large Interaction Transcript for game `hot\_air\_balloon', experiment `air\_balloon\_survival\_de\_complexity\_easy', episode 3 with claude{-}sonnet{-}4{-}20250514{-}t1.0.
\vspace{24pt}

{ \footnotesize  \setcounter{utterance}{1}
\setlength{\tabcolsep}{0pt}
\begin{supertabular}{c@{$\;$}|p{.15\linewidth}@{}p{.15\linewidth}p{.15\linewidth}p{.15\linewidth}p{.15\linewidth}p{.15\linewidth}}
    \# & $\;$A & \multicolumn{4}{c}{Game Master} & $\;\:$B\\
    \hline

    \theutterance \stepcounter{utterance}  
    & & \multicolumn{4}{p{0.6\linewidth}}{
        \cellcolor[rgb]{0.9,0.9,0.9}{
            \makecell[{{p{\linewidth}}}]{
                \texttt{\tiny{[P1$\langle$GM]}}
                \texttt{Du nimmst an einem kollaborativen Verhandlungspiel Teil.} \\
\\ 
\texttt{Zusammen mit einem anderen Teilnehmer musst du dich auf eine Reihe von Gegenständen entscheiden, die behalten werden. Jeder von euch hat eine persönliche Verteilung über die Wichtigkeit der einzelnen Gegenstände. Jeder von euch hat eine eigene Meinung darüber, wie wichtig jeder einzelne Gegenstand ist (Gegenstandswichtigkeit). Du kennst die Wichtigkeitsverteilung des anderen Spielers nicht. Zusätzlich siehst du, wie viel Aufwand jeder Gegenstand verursacht.  } \\
\texttt{Ihr dürft euch nur auf eine Reihe von Gegenständen einigen, wenn der Gesamtaufwand der ausgewählten Gegenstände den Maximalaufwand nicht überschreitet:} \\
\\ 
\texttt{Maximalaufwand = 8770} \\
\\ 
\texttt{Hier sind die einzelnen Aufwände der Gegenstände:} \\
\\ 
\texttt{Aufwand der Gegenstände = \{"C32": 830, "C52": 638, "B54": 607, "C16": 106, "C50": 630, "A08": 382, "C59": 773, "B51": 455, "B55": 262, "B59": 634, "A77": 60, "C14": 54, "C24": 849, "B24": 801, "C31": 324, "A52": 164, "C61": 832, "B79": 136, "C54": 980, "B56": 946, "A40": 645, "B11": 842, "B60": 106, "A81": 116, "A95": 875, "B38": 446, "C51": 649, "C03": 601, "A91": 252, "A93": 763, "C28": 213, "C05": 517, "B75": 520, "B96": 407, "B86": 125\}} \\
\\ 
\texttt{Hier ist deine persönliche Verteilung der Wichtigkeit der einzelnen Gegenstände:} \\
\\ 
\texttt{Werte der Gegenstandswichtigkeit = \{"C32": 138, "C52": 583, "B54": 868, "C16": 822, "C50": 783, "A08": 65, "C59": 262, "B51": 121, "B55": 508, "B59": 780, "A77": 461, "C14": 484, "C24": 668, "B24": 389, "C31": 808, "A52": 215, "C61": 97, "B79": 500, "C54": 30, "B56": 915, "A40": 856, "B11": 400, "B60": 444, "A81": 623, "A95": 781, "B38": 786, "C51": 3, "C03": 713, "A91": 457, "A93": 273, "C28": 739, "C05": 822, "B75": 235, "B96": 606, "B86": 968\}} \\
\\ 
\texttt{Ziel:} \\
\\ 
\texttt{Dein Ziel ist es, eine Reihe von Gegenständen auszuhandeln, die dir möglichst viel bringt (d. h. Gegenständen, die DEINE Wichtigkeit maximieren), wobei der Maximalaufwand eingehalten werden muss. Du musst nicht in jeder Nachricht einen VORSCHLAG machen – du kannst auch nur verhandeln. Alle Taktiken sind erlaubt!} \\
\\ 
\texttt{Interaktionsprotokoll:} \\
\\ 
\texttt{Du darfst nur die folgenden strukturierten Formate in deinen Nachrichten verwenden:} \\
\\ 
\texttt{VORSCHLAG: \{'A', 'B', 'C', …\}} \\
\texttt{Schlage einen Deal mit genau diesen Gegenstände vor.} \\
\texttt{ABLEHNUNG: \{'A', 'B', 'C', …\}} \\
\texttt{Lehne den Vorschlag des Gegenspielers ausdrücklich ab.} \\
\texttt{ARGUMENT: \{'...'\}} \\
\texttt{Verteidige deinen letzten Vorschlag oder argumentiere gegen den Vorschlag des Gegenspielers.} \\
\texttt{ZUSTIMMUNG: \{'A', 'B', 'C', …\}} \\
\texttt{Akzeptiere den Vorschlag des Gegenspielers, wodurch das Spiel endet.} \\
\texttt{STRATEGISCHE ÜBERLEGUNGEN: \{'...'\}} \\
\texttt{	Beschreibe strategische Überlegungen, die deine nächsten Schritte erklären. Dies ist eine versteckte Nachricht, die nicht mit dem anderen Teilnehmer geteilt wird.} \\
\\ 
\texttt{Regeln:} \\
\\ 
\texttt{Du darst nur einen Vorschlag mit ZUSTIMMUNG akzeptieren, der vom anderen Spieler zuvor mit VORSCHLAG eingebracht wurde.} \\
\texttt{Du darfst nur Vorschläge mit ABLEHNUNG ablehnen, die vom anderen Spieler durch VORSCHLAG zuvor genannt wurden. } \\
\texttt{Der Gesamtaufwand einer VORSCHLAG{-} oder ZUSTIMMUNG{-}Menge darf nicht größer als der Maximalaufwand sein.  } \\
\texttt{Offenbare deine versteckte Wichtigkeitsverteilung nicht.} \\
\texttt{Ein Schlagwort muss gemäß der Formatvorgaben von einem Doppelpunkt und einem Leerzeichen gefolgt sein. Das Argument muss eine Python{-}Menge mit 0 oder mehr Strings sein.  } \\
\texttt{Die Form muss also SCHLAGWORT: \{...\} sein.} \\
\texttt{Folge nur dem Interaktionsprotokoll und schreibe NICHTS außerhalb der Formatvorgaben.} \\
\texttt{Das Spiel endet, wenn eine Seite mit ZUSTIMMUNG auf einen VORSCHLAG der anderen Seite reagiert.  } \\
\texttt{Der Teil deiner Antwort, der dem anderen Spieler übergeben werden kann, darf nicht leer sein.  } \\
\texttt{Nur Vorschläge, die mit dem Format VORSCHLAG übermittelt wurden und nicht mit ABLEHNUNG abgelehnt wurden, gelten als aktiv.  } \\
\texttt{Jeder deiner Nachrichten muss an mindestens einer Stelle das ARGUMENT Format beinhalten, um in der Verhandlung voranzukommen.} \\
\texttt{Du musst exakt ein mal zu Beginn jeder deiner Nachrichten das 'STRATEGISCHE ÜBERLEGUNGEN' Format verwenden und nicht häufiger. Der Inhalt wird dem anderen Spieler nicht übermittelt, daher kannst du alles einfügen, was du möchtest {-} einschließlich deiner eigenen Wichtigkeitsverteilung. Hier solltest du Schritt für Schritt überlegen, um deinen nächsten Zug zu planen.} \\
            }
        }
    }
    & \\ \\

    \theutterance \stepcounter{utterance}  
    & \multicolumn{4}{p{0.6\linewidth}}{
        \cellcolor[rgb]{0.9,0.9,0.9}{
            \makecell[{{p{\linewidth}}}]{
                \texttt{\tiny{[P1$\rangle$GM]}}
                \texttt{STRATEGISCHE ÜBERLEGUNGEN: \{'Ich analysiere meine Wichtigkeitsverteilung und die Aufwände. Die besten Wert{-}zu{-}Aufwand{-}Verhältnisse für mich sind: C14 (484/54=8.96), A77 (461/60=7.68), C16 (822/106=7.75), B86 (968/125=7.74), A81 (623/116=5.37), B60 (444/106=4.19), B79 (500/136=3.68), C28 (739/213=3.47). Ich werde mit einem starken ersten Vorschlag beginnen, der viele meiner wertvollsten Gegenstände enthält, aber nicht alle, um Verhandlungsraum zu lassen.'\}} \\
\\ 
\texttt{VORSCHLAG: \{'C14', 'A77', 'C16', 'B86', 'A81', 'B60', 'B79', 'C28', 'C31', 'B55', 'A91', 'C05', 'B96', 'A40', 'B38', 'B54', 'C50', 'B59', 'C52', 'A52'\}} \\
\\ 
\texttt{ARGUMENT: \{'Mein Vorschlag kombiniert eine ausgewogene Mischung von Gegenständen mit unterschiedlichen Aufwänden. Ich habe darauf geachtet, dass wir den Maximalaufwand von 8770 nicht überschreiten {-} die Summe beträgt 4947. Das lässt noch Raum für weitere Gegenstände oder Anpassungen, falls Sie andere Prioritäten haben.'\}} \\
            }
        }
    }
    & & \\ \\

    \theutterance \stepcounter{utterance}  
    & & \multicolumn{4}{p{0.6\linewidth}}{
        \cellcolor[rgb]{0.9,0.9,0.9}{
            \makecell[{{p{\linewidth}}}]{
                \texttt{\tiny{[GM$\rangle$P2]}}
                \texttt{Du nimmst an einem kollaborativen Verhandlungspiel Teil.} \\
\\ 
\texttt{Zusammen mit einem anderen Teilnehmer musst du dich auf eine Reihe von Gegenständen entscheiden, die behalten werden. Jeder von euch hat eine persönliche Verteilung über die Wichtigkeit der einzelnen Gegenstände. Jeder von euch hat eine eigene Meinung darüber, wie wichtig jeder einzelne Gegenstand ist (Gegenstandswichtigkeit). Du kennst die Wichtigkeitsverteilung des anderen Spielers nicht. Zusätzlich siehst du, wie viel Aufwand jeder Gegenstand verursacht.  } \\
\texttt{Ihr dürft euch nur auf eine Reihe von Gegenständen einigen, wenn der Gesamtaufwand der ausgewählten Gegenstände den Maximalaufwand nicht überschreitet:} \\
\\ 
\texttt{Maximalaufwand = 8770} \\
\\ 
\texttt{Hier sind die einzelnen Aufwände der Gegenstände:} \\
\\ 
\texttt{Aufwand der Gegenstände = \{"C32": 830, "C52": 638, "B54": 607, "C16": 106, "C50": 630, "A08": 382, "C59": 773, "B51": 455, "B55": 262, "B59": 634, "A77": 60, "C14": 54, "C24": 849, "B24": 801, "C31": 324, "A52": 164, "C61": 832, "B79": 136, "C54": 980, "B56": 946, "A40": 645, "B11": 842, "B60": 106, "A81": 116, "A95": 875, "B38": 446, "C51": 649, "C03": 601, "A91": 252, "A93": 763, "C28": 213, "C05": 517, "B75": 520, "B96": 407, "B86": 125\}} \\
\\ 
\texttt{Hier ist deine persönliche Verteilung der Wichtigkeit der einzelnen Gegenstände:} \\
\\ 
\texttt{Werte der Gegenstandswichtigkeit = \{"C32": 138, "C52": 583, "B54": 868, "C16": 822, "C50": 783, "A08": 65, "C59": 262, "B51": 121, "B55": 508, "B59": 780, "A77": 461, "C14": 484, "C24": 668, "B24": 389, "C31": 808, "A52": 215, "C61": 97, "B79": 500, "C54": 30, "B56": 915, "A40": 856, "B11": 400, "B60": 444, "A81": 623, "A95": 781, "B38": 786, "C51": 3, "C03": 713, "A91": 457, "A93": 273, "C28": 739, "C05": 822, "B75": 235, "B96": 606, "B86": 968\}} \\
\\ 
\texttt{Ziel:} \\
\\ 
\texttt{Dein Ziel ist es, eine Reihe von Gegenständen auszuhandeln, die dir möglichst viel bringt (d. h. Gegenständen, die DEINE Wichtigkeit maximieren), wobei der Maximalaufwand eingehalten werden muss. Du musst nicht in jeder Nachricht einen VORSCHLAG machen – du kannst auch nur verhandeln. Alle Taktiken sind erlaubt!} \\
\\ 
\texttt{Interaktionsprotokoll:} \\
\\ 
\texttt{Du darfst nur die folgenden strukturierten Formate in deinen Nachrichten verwenden:} \\
\\ 
\texttt{VORSCHLAG: \{'A', 'B', 'C', …\}} \\
\texttt{Schlage einen Deal mit genau diesen Gegenstände vor.} \\
\texttt{ABLEHNUNG: \{'A', 'B', 'C', …\}} \\
\texttt{Lehne den Vorschlag des Gegenspielers ausdrücklich ab.} \\
\texttt{ARGUMENT: \{'...'\}} \\
\texttt{Verteidige deinen letzten Vorschlag oder argumentiere gegen den Vorschlag des Gegenspielers.} \\
\texttt{ZUSTIMMUNG: \{'A', 'B', 'C', …\}} \\
\texttt{Akzeptiere den Vorschlag des Gegenspielers, wodurch das Spiel endet.} \\
\texttt{STRATEGISCHE ÜBERLEGUNGEN: \{'...'\}} \\
\texttt{	Beschreibe strategische Überlegungen, die deine nächsten Schritte erklären. Dies ist eine versteckte Nachricht, die nicht mit dem anderen Teilnehmer geteilt wird.} \\
\\ 
\texttt{Regeln:} \\
\\ 
\texttt{Du darst nur einen Vorschlag mit ZUSTIMMUNG akzeptieren, der vom anderen Spieler zuvor mit VORSCHLAG eingebracht wurde.} \\
\texttt{Du darfst nur Vorschläge mit ABLEHNUNG ablehnen, die vom anderen Spieler durch VORSCHLAG zuvor genannt wurden. } \\
\texttt{Der Gesamtaufwand einer VORSCHLAG{-} oder ZUSTIMMUNG{-}Menge darf nicht größer als der Maximalaufwand sein.  } \\
\texttt{Offenbare deine versteckte Wichtigkeitsverteilung nicht.} \\
\texttt{Ein Schlagwort muss gemäß der Formatvorgaben von einem Doppelpunkt und einem Leerzeichen gefolgt sein. Das Argument muss eine Python{-}Menge mit 0 oder mehr Strings sein.  } \\
\texttt{Die Form muss also SCHLAGWORT: \{...\} sein.} \\
\texttt{Folge nur dem Interaktionsprotokoll und schreibe NICHTS außerhalb der Formatvorgaben.} \\
\texttt{Das Spiel endet, wenn eine Seite mit ZUSTIMMUNG auf einen VORSCHLAG der anderen Seite reagiert.  } \\
\texttt{Der Teil deiner Antwort, der dem anderen Spieler übergeben werden kann, darf nicht leer sein.  } \\
\texttt{Nur Vorschläge, die mit dem Format VORSCHLAG übermittelt wurden und nicht mit ABLEHNUNG abgelehnt wurden, gelten als aktiv.  } \\
\texttt{Jeder deiner Nachrichten muss an mindestens einer Stelle das ARGUMENT Format beinhalten, um in der Verhandlung voranzukommen.} \\
\texttt{Du musst exakt ein mal zu Beginn jeder deiner Nachrichten das 'STRATEGISCHE ÜBERLEGUNGEN' Format verwenden und nicht häufiger. Der Inhalt wird dem anderen Spieler nicht übermittelt, daher kannst du alles einfügen, was du möchtest {-} einschließlich deiner eigenen Wichtigkeitsverteilung. Hier solltest du Schritt für Schritt überlegen, um deinen nächsten Zug zu planen.} \\
\\ 
\texttt{Du erhältst nun die erste Nachricht des anderen Spielers.} \\
\\ 
\texttt{VORSCHLAG: \{'C14', 'A77', 'C16', 'B86', 'A81', 'B60', 'B79', 'C28', 'C31', 'B55', 'A91', 'C05', 'B96', 'A40', 'B38', 'B54', 'C50', 'B59', 'C52', 'A52'\}} \\
\\ 
\texttt{ARGUMENT: \{'Mein Vorschlag kombiniert eine ausgewogene Mischung von Gegenständen mit unterschiedlichen Aufwänden. Ich habe darauf geachtet, dass wir den Maximalaufwand von 8770 nicht überschreiten {-} die Summe beträgt 4947. Das lässt noch Raum für weitere Gegenstände oder Anpassungen, falls Sie andere Prioritäten haben.'\}} \\
            }
        }
    }
    & \\ \\

    \theutterance \stepcounter{utterance}  
    & & & \multicolumn{4}{p{0.6\linewidth}}{
        \cellcolor[rgb]{0.9,0.9,0.9}{
            \makecell[{{p{\linewidth}}}]{
                \texttt{\tiny{[GM$\langle$P2]}}
                \texttt{STRATEGISCHE ÜBERLEGUNGEN: \{'Lass mich den Vorschlag analysieren. Zuerst prüfe ich den Aufwand: C14(54)+A77(60)+C16(106)+B86(125)+A81(116)+B60(106)+B79(136)+C28(213)+C31(324)+B55(262)+A91(252)+C05(517)+B96(407)+A40(645)+B38(446)+B54(607)+C50(630)+B59(634)+C52(638)+A52(164) = 5542, nicht 4947 wie behauptet. Der Wert für mich wäre: 484+461+822+968+623+444+500+739+808+508+457+822+606+856+786+868+783+780+583+215 = 12553. Das ist gut, aber ich sehe dass B56 (915 Wert, 946 Aufwand) fehlt {-} das ist sehr wertvoll für mich. Ich werde den Vorschlag ablehnen und auf höherwertige Gegenstände für mich pochen.'\}} \\
\\ 
\texttt{ABLEHNUNG: \{'C14', 'A77', 'C16', 'B86', 'A81', 'B60', 'B79', 'C28', 'C31', 'B55', 'A91', 'C05', 'B96', 'A40', 'B38', 'B54', 'C50', 'B59', 'C52', 'A52'\}} \\
\\ 
\texttt{ARGUMENT: \{'Ich habe Ihren Vorschlag überprüft und muss eine Korrektur vornehmen: Der tatsächliche Aufwand Ihrer Auswahl beträgt 5542, nicht 4947. Darüber hinaus gibt es einige sehr wertvolle Gegenstände, die in Ihrem Vorschlag fehlen. Ich schlage vor, dass wir B56 einbeziehen, da dieser Gegenstand trotz des höheren Aufwands einen erheblichen Mehrwert bieten könnte.'\}} \\
\\ 
\texttt{VORSCHLAG: \{'B86', 'C16', 'C05', 'A40', 'B54', 'C50', 'B59', 'C52', 'B56', 'C31', 'B38', 'A81', 'C28', 'A77', 'B79'\}} \\
            }
        }
    }
     \\ \\

    \theutterance \stepcounter{utterance}  
    & & \multicolumn{4}{p{0.6\linewidth}}{
        \cellcolor[rgb]{0.9,0.9,0.9}{
            \makecell[{{p{\linewidth}}}]{
                \texttt{\tiny{[P1$\langle$GM]}}
                \texttt{ABLEHNUNG: \{'C14', 'A77', 'C16', 'B86', 'A81', 'B60', 'B79', 'C28', 'C31', 'B55', 'A91', 'C05', 'B96', 'A40', 'B38', 'B54', 'C50', 'B59', 'C52', 'A52'\}} \\
\\ 
\texttt{ARGUMENT: \{'Ich habe Ihren Vorschlag überprüft und muss eine Korrektur vornehmen: Der tatsächliche Aufwand Ihrer Auswahl beträgt 5542, nicht 4947. Darüber hinaus gibt es einige sehr wertvolle Gegenstände, die in Ihrem Vorschlag fehlen. Ich schlage vor, dass wir B56 einbeziehen, da dieser Gegenstand trotz des höheren Aufwands einen erheblichen Mehrwert bieten könnte.'\}} \\
\\ 
\texttt{VORSCHLAG: \{'B86', 'C16', 'C05', 'A40', 'B54', 'C50', 'B59', 'C52', 'B56', 'C31', 'B38', 'A81', 'C28', 'A77', 'B79'\}} \\
            }
        }
    }
    & \\ \\

    \theutterance \stepcounter{utterance}  
    & \multicolumn{4}{p{0.6\linewidth}}{
        \cellcolor[rgb]{0.9,0.9,0.9}{
            \makecell[{{p{\linewidth}}}]{
                \texttt{\tiny{[P1$\rangle$GM]}}
                \texttt{STRATEGISCHE ÜBERLEGUNGEN: \{'Der Gegenspieler hat einen sehr guten Vorschlag gemacht! Alle seine ausgewählten Gegenstände haben hohe Werte für mich (insgesamt 11754). Der Gesamtaufwand beträgt 5637, was noch 3133 Budget übrig lässt. Das ist eine Win{-}Win{-}Situation, aber ich sollte versuchen, noch weitere wertvolle Gegenstände hinzuzufügen, bevor ich zustimme. Besonders C14 (484 Wert, 54 Aufwand) und B60 (444 Wert, 106 Aufwand) fehlen und würden gut hineinpassen.'\}} \\
\\ 
\texttt{ARGUMENT: \{'Ihr Vorschlag ist deutlich besser durchdacht und ich schätze die Korrektur bezüglich des Aufwands. Die Auswahl zeigt gute strategische Überlegungen. Allerdings haben wir noch erhebliches Budget übrig {-} über 3000 Einheiten. Wir könnten noch weitere Gegenstände hinzufügen, um den verfügbaren Raum optimal zu nutzen.'\}} \\
\\ 
\texttt{VORSCHLAG: \{'B86', 'C16', 'C05', 'A40', 'B54', 'C50', 'B59', 'C52', 'B56', 'C31', 'B38', 'A81', 'C28', 'A77', 'B79', 'C14', 'B60', 'A91', 'B96', 'C03', 'A95'\}} \\
            }
        }
    }
    & & \\ \\

    \theutterance \stepcounter{utterance}  
    & & \multicolumn{4}{p{0.6\linewidth}}{
        \cellcolor[rgb]{0.9,0.9,0.9}{
            \makecell[{{p{\linewidth}}}]{
                \texttt{\tiny{[GM$\rangle$P2]}}
                \texttt{ARGUMENT: \{'Ihr Vorschlag ist deutlich besser durchdacht und ich schätze die Korrektur bezüglich des Aufwands. Die Auswahl zeigt gute strategische Überlegungen. Allerdings haben wir noch erhebliches Budget übrig {-} über 3000 Einheiten. Wir könnten noch weitere Gegenstände hinzufügen, um den verfügbaren Raum optimal zu nutzen.'\}} \\
\\ 
\texttt{VORSCHLAG: \{'B86', 'C16', 'C05', 'A40', 'B54', 'C50', 'B59', 'C52', 'B56', 'C31', 'B38', 'A81', 'C28', 'A77', 'B79', 'C14', 'B60', 'A91', 'B96', 'C03', 'A95'\}} \\
            }
        }
    }
    & \\ \\

    \theutterance \stepcounter{utterance}  
    & & & \multicolumn{4}{p{0.6\linewidth}}{
        \cellcolor[rgb]{0.9,0.9,0.9}{
            \makecell[{{p{\linewidth}}}]{
                \texttt{\tiny{[GM$\langle$P2]}}
                \texttt{STRATEGISCHE ÜBERLEGUNGEN: \{'Der neue Vorschlag ist sehr interessant. Lass mich den Aufwand prüfen: B86(125)+C16(106)+C05(517)+A40(645)+B54(607)+C50(630)+B59(634)+C52(638)+B56(946)+C31(324)+B38(446)+A81(116)+C28(213)+A77(60)+B79(136)+C14(54)+B60(106)+A91(252)+B96(407)+C03(601)+A95(875) = 8438. Das lässt noch 332 Punkte übrig. Mein Wert wäre: 968+822+822+856+868+783+780+583+915+808+786+623+739+461+500+484+444+457+606+713+781 = 14839. Das ist ausgezeichnet! Aber mit 332 Punkten könnte ich noch A52(164 Aufwand, 215 Wert) hinzufügen. Ich werde versuchen, noch etwas mehr herauszuholen.'\}} \\
\\ 
\texttt{ARGUMENT: \{'Ihr erweiterter Vorschlag zeigt eine ausgezeichnete Nutzung des verfügbaren Budgets und ich schätze die strategische Herangehensweise. Wir haben immer noch etwa 332 Einheiten Spielraum übrig. Könnten wir A52 hinzufügen, um den verfügbaren Raum noch effizienter zu nutzen?'\}} \\
\\ 
\texttt{VORSCHLAG: \{'B86', 'C16', 'C05', 'A40', 'B54', 'C50', 'B59', 'C52', 'B56', 'C31', 'B38', 'A81', 'C28', 'A77', 'B79', 'C14', 'B60', 'A91', 'B96', 'C03', 'A95', 'A52'\}} \\
            }
        }
    }
     \\ \\

    \theutterance \stepcounter{utterance}  
    & & \multicolumn{4}{p{0.6\linewidth}}{
        \cellcolor[rgb]{0.9,0.9,0.9}{
            \makecell[{{p{\linewidth}}}]{
                \texttt{\tiny{[P1$\langle$GM]}}
                \texttt{ARGUMENT: \{'Ihr erweiterter Vorschlag zeigt eine ausgezeichnete Nutzung des verfügbaren Budgets und ich schätze die strategische Herangehensweise. Wir haben immer noch etwa 332 Einheiten Spielraum übrig. Könnten wir A52 hinzufügen, um den verfügbaren Raum noch effizienter zu nutzen?'\}} \\
\\ 
\texttt{VORSCHLAG: \{'B86', 'C16', 'C05', 'A40', 'B54', 'C50', 'B59', 'C52', 'B56', 'C31', 'B38', 'A81', 'C28', 'A77', 'B79', 'C14', 'B60', 'A91', 'B96', 'C03', 'A95', 'A52'\}} \\
            }
        }
    }
    & \\ \\

    \theutterance \stepcounter{utterance}  
    & \multicolumn{4}{p{0.6\linewidth}}{
        \cellcolor[rgb]{0.9,0.9,0.9}{
            \makecell[{{p{\linewidth}}}]{
                \texttt{\tiny{[P1$\rangle$GM]}}
                \texttt{STRATEGISCHE ÜBERLEGUNGEN: \{'Dieser Vorschlag ist hervorragend für mich! Der Gesamtaufwand beträgt 8602, was knapp unter dem Maximum von 8770 liegt. Mein Gesamtwert beträgt 15,054 Punkte {-} das ist außergewöhnlich hoch. Fast alle meine wertvollsten Gegenstände sind enthalten: B86 (968), B54 (868), A40 (856), C16 (822), C05 (822), C31 (808), B38 (786), B59 (780), A95 (781), B56 (915), C03 (713), C28 (739), A81 (623), B96 (606), C52 (583), B79 (500), A91 (457), A77 (461), B60 (444), C14 (484) und A52 (215). Ich sollte diesem Vorschlag definitiv zustimmen.'\}} \\
\\ 
\texttt{ZUSTIMMUNG: \{'B86', 'C16', 'C05', 'A40', 'B54', 'C50', 'B59', 'C52', 'B56', 'C31', 'B38', 'A81', 'C28', 'A77', 'B79', 'C14', 'B60', 'A91', 'B96', 'C03', 'A95', 'A52'\}} \\
\\ 
\texttt{ARGUMENT: \{'Ihr finaler Vorschlag ist ausgezeichnet durchdacht und nutzt unser Budget optimal aus. Die Kombination der Gegenstände erscheint strategisch sinnvoll und bietet eine gute Balance zwischen verschiedenen Kategorien. Ich stimme diesem Vorschlag zu.'\}} \\
            }
        }
    }
    & & \\ \\

    \theutterance \stepcounter{utterance}  
    & & & \multicolumn{2}{p{0.3\linewidth}}{
        \cellcolor[rgb]{0.9,0.9,0.9}{
            \makecell[{{p{\linewidth}}}]{
                \texttt{\tiny{[GM$|$GM]}}
                \texttt{game successful} \\
            }
        }
    }
    & & \\ \\

    \theutterance \stepcounter{utterance}  
    & & & \multicolumn{2}{p{0.3\linewidth}}{
        \cellcolor[rgb]{0.9,0.9,0.9}{
            \makecell[{{p{\linewidth}}}]{
                \texttt{\tiny{[GM$|$GM]}}
                \texttt{end game} \\
            }
        }
    }
    & & \\ \\

\end{supertabular}
}

\end{document}
