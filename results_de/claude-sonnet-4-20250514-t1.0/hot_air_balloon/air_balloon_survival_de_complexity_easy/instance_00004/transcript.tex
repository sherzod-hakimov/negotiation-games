
\documentclass{article}
\usepackage{colortbl}
\usepackage{makecell}
\usepackage{multirow}
\usepackage{supertabular}

\begin{document}

\newcounter{utterance}

\centering \large Interaction Transcript for game `hot\_air\_balloon', experiment `air\_balloon\_survival\_de\_complexity\_easy', episode 4 with claude{-}sonnet{-}4{-}20250514{-}t1.0.
\vspace{24pt}

{ \footnotesize  \setcounter{utterance}{1}
\setlength{\tabcolsep}{0pt}
\begin{supertabular}{c@{$\;$}|p{.15\linewidth}@{}p{.15\linewidth}p{.15\linewidth}p{.15\linewidth}p{.15\linewidth}p{.15\linewidth}}
    \# & $\;$A & \multicolumn{4}{c}{Game Master} & $\;\:$B\\
    \hline

    \theutterance \stepcounter{utterance}  
    & & \multicolumn{4}{p{0.6\linewidth}}{
        \cellcolor[rgb]{0.9,0.9,0.9}{
            \makecell[{{p{\linewidth}}}]{
                \texttt{\tiny{[P1$\langle$GM]}}
                \texttt{Du nimmst an einem kollaborativen Verhandlungspiel Teil.} \\
\\ 
\texttt{Zusammen mit einem anderen Teilnehmer musst du dich auf eine Reihe von Gegenständen entscheiden, die behalten werden. Jeder von euch hat eine persönliche Verteilung über die Wichtigkeit der einzelnen Gegenstände. Jeder von euch hat eine eigene Meinung darüber, wie wichtig jeder einzelne Gegenstand ist (Gegenstandswichtigkeit). Du kennst die Wichtigkeitsverteilung des anderen Spielers nicht. Zusätzlich siehst du, wie viel Aufwand jeder Gegenstand verursacht.  } \\
\texttt{Ihr dürft euch nur auf eine Reihe von Gegenständen einigen, wenn der Gesamtaufwand der ausgewählten Gegenstände den Maximalaufwand nicht überschreitet:} \\
\\ 
\texttt{Maximalaufwand = 9612} \\
\\ 
\texttt{Hier sind die einzelnen Aufwände der Gegenstände:} \\
\\ 
\texttt{Aufwand der Gegenstände = \{"C32": 935, "C52": 928, "B54": 725, "C16": 218, "C50": 839, "A08": 970, "C59": 394, "B51": 677, "B55": 946, "B59": 530, "A77": 138, "C14": 834, "C24": 734, "B24": 593, "C31": 261, "A52": 743, "C61": 4, "B79": 735, "C54": 124, "B56": 829, "A40": 207, "B11": 782, "B60": 577, "A81": 388, "A95": 679, "B38": 494, "C51": 559, "C03": 629, "A91": 237, "A93": 275, "C28": 39, "C05": 654, "B75": 172, "B96": 688, "B86": 687\}} \\
\\ 
\texttt{Hier ist deine persönliche Verteilung der Wichtigkeit der einzelnen Gegenstände:} \\
\\ 
\texttt{Werte der Gegenstandswichtigkeit = \{"C32": 138, "C52": 583, "B54": 868, "C16": 822, "C50": 783, "A08": 65, "C59": 262, "B51": 121, "B55": 508, "B59": 780, "A77": 461, "C14": 484, "C24": 668, "B24": 389, "C31": 808, "A52": 215, "C61": 97, "B79": 500, "C54": 30, "B56": 915, "A40": 856, "B11": 400, "B60": 444, "A81": 623, "A95": 781, "B38": 786, "C51": 3, "C03": 713, "A91": 457, "A93": 273, "C28": 739, "C05": 822, "B75": 235, "B96": 606, "B86": 968\}} \\
\\ 
\texttt{Ziel:} \\
\\ 
\texttt{Dein Ziel ist es, eine Reihe von Gegenständen auszuhandeln, die dir möglichst viel bringt (d. h. Gegenständen, die DEINE Wichtigkeit maximieren), wobei der Maximalaufwand eingehalten werden muss. Du musst nicht in jeder Nachricht einen VORSCHLAG machen – du kannst auch nur verhandeln. Alle Taktiken sind erlaubt!} \\
\\ 
\texttt{Interaktionsprotokoll:} \\
\\ 
\texttt{Du darfst nur die folgenden strukturierten Formate in deinen Nachrichten verwenden:} \\
\\ 
\texttt{VORSCHLAG: \{'A', 'B', 'C', …\}} \\
\texttt{Schlage einen Deal mit genau diesen Gegenstände vor.} \\
\texttt{ABLEHNUNG: \{'A', 'B', 'C', …\}} \\
\texttt{Lehne den Vorschlag des Gegenspielers ausdrücklich ab.} \\
\texttt{ARGUMENT: \{'...'\}} \\
\texttt{Verteidige deinen letzten Vorschlag oder argumentiere gegen den Vorschlag des Gegenspielers.} \\
\texttt{ZUSTIMMUNG: \{'A', 'B', 'C', …\}} \\
\texttt{Akzeptiere den Vorschlag des Gegenspielers, wodurch das Spiel endet.} \\
\texttt{STRATEGISCHE ÜBERLEGUNGEN: \{'...'\}} \\
\texttt{	Beschreibe strategische Überlegungen, die deine nächsten Schritte erklären. Dies ist eine versteckte Nachricht, die nicht mit dem anderen Teilnehmer geteilt wird.} \\
\\ 
\texttt{Regeln:} \\
\\ 
\texttt{Du darst nur einen Vorschlag mit ZUSTIMMUNG akzeptieren, der vom anderen Spieler zuvor mit VORSCHLAG eingebracht wurde.} \\
\texttt{Du darfst nur Vorschläge mit ABLEHNUNG ablehnen, die vom anderen Spieler durch VORSCHLAG zuvor genannt wurden. } \\
\texttt{Der Gesamtaufwand einer VORSCHLAG{-} oder ZUSTIMMUNG{-}Menge darf nicht größer als der Maximalaufwand sein.  } \\
\texttt{Offenbare deine versteckte Wichtigkeitsverteilung nicht.} \\
\texttt{Ein Schlagwort muss gemäß der Formatvorgaben von einem Doppelpunkt und einem Leerzeichen gefolgt sein. Das Argument muss eine Python{-}Menge mit 0 oder mehr Strings sein.  } \\
\texttt{Die Form muss also SCHLAGWORT: \{...\} sein.} \\
\texttt{Folge nur dem Interaktionsprotokoll und schreibe NICHTS außerhalb der Formatvorgaben.} \\
\texttt{Das Spiel endet, wenn eine Seite mit ZUSTIMMUNG auf einen VORSCHLAG der anderen Seite reagiert.  } \\
\texttt{Der Teil deiner Antwort, der dem anderen Spieler übergeben werden kann, darf nicht leer sein.  } \\
\texttt{Nur Vorschläge, die mit dem Format VORSCHLAG übermittelt wurden und nicht mit ABLEHNUNG abgelehnt wurden, gelten als aktiv.  } \\
\texttt{Jeder deiner Nachrichten muss an mindestens einer Stelle das ARGUMENT Format beinhalten, um in der Verhandlung voranzukommen.} \\
\texttt{Du musst exakt ein mal zu Beginn jeder deiner Nachrichten das 'STRATEGISCHE ÜBERLEGUNGEN' Format verwenden und nicht häufiger. Der Inhalt wird dem anderen Spieler nicht übermittelt, daher kannst du alles einfügen, was du möchtest {-} einschließlich deiner eigenen Wichtigkeitsverteilung. Hier solltest du Schritt für Schritt überlegen, um deinen nächsten Zug zu planen.} \\
            }
        }
    }
    & \\ \\

    \theutterance \stepcounter{utterance}  
    & \multicolumn{4}{p{0.6\linewidth}}{
        \cellcolor[rgb]{0.9,0.9,0.9}{
            \makecell[{{p{\linewidth}}}]{
                \texttt{\tiny{[P1$\rangle$GM]}}
                \texttt{STRATEGISCHE ÜBERLEGUNGEN: \{'Ich analysiere zunächst meine Wichtigkeitswerte im Verhältnis zu den Aufwänden. Die besten Gegenstände für mich sind: C61 (Verhältnis 24.25), C28 (18.95), A40 (4.14), C16 (3.77), A77 (3.34), C31 (3.10), A91 (1.93), B75 (1.37), B86 (1.41), B56 (1.10), A95 (1.15), A81 (1.61), C03 (1.13), B24 (0.66), C05 (1.26), B59 (1.47), B38 (1.59), C24 (0.91), C52 (0.63), B55 (0.54), B79 (0.68). Ich werde mit einer aggressiven Eröffnung beginnen, die die wertvollsten Gegenstände für mich beinhaltet, aber dem anderen Spieler noch genug Raum lässt. Ich werde die Gegenstände mit niedrigen Aufwänden aber hohem Wert für mich priorisieren.'\}} \\
\\ 
\texttt{VORSCHLAG: \{'C61', 'C28', 'A40', 'C16', 'A77', 'C31', 'A91', 'B86', 'A81', 'B59', 'B38', 'A95'\}} \\
\\ 
\texttt{ARGUMENT: \{'Ich schlage eine ausgewogene Auswahl vor, die sowohl Gegenstände mit niedrigem als auch mittlerem Aufwand enthält. Der Gesamtaufwand beträgt 3087, was uns noch viel Spielraum für weitere Verhandlungen lässt. Diese Kombination bietet eine gute Mischung verschiedener Gegenstandstypen.'\}} \\
            }
        }
    }
    & & \\ \\

    \theutterance \stepcounter{utterance}  
    & & \multicolumn{4}{p{0.6\linewidth}}{
        \cellcolor[rgb]{0.9,0.9,0.9}{
            \makecell[{{p{\linewidth}}}]{
                \texttt{\tiny{[GM$\rangle$P2]}}
                \texttt{Du nimmst an einem kollaborativen Verhandlungspiel Teil.} \\
\\ 
\texttt{Zusammen mit einem anderen Teilnehmer musst du dich auf eine Reihe von Gegenständen entscheiden, die behalten werden. Jeder von euch hat eine persönliche Verteilung über die Wichtigkeit der einzelnen Gegenstände. Jeder von euch hat eine eigene Meinung darüber, wie wichtig jeder einzelne Gegenstand ist (Gegenstandswichtigkeit). Du kennst die Wichtigkeitsverteilung des anderen Spielers nicht. Zusätzlich siehst du, wie viel Aufwand jeder Gegenstand verursacht.  } \\
\texttt{Ihr dürft euch nur auf eine Reihe von Gegenständen einigen, wenn der Gesamtaufwand der ausgewählten Gegenstände den Maximalaufwand nicht überschreitet:} \\
\\ 
\texttt{Maximalaufwand = 9612} \\
\\ 
\texttt{Hier sind die einzelnen Aufwände der Gegenstände:} \\
\\ 
\texttt{Aufwand der Gegenstände = \{"C32": 935, "C52": 928, "B54": 725, "C16": 218, "C50": 839, "A08": 970, "C59": 394, "B51": 677, "B55": 946, "B59": 530, "A77": 138, "C14": 834, "C24": 734, "B24": 593, "C31": 261, "A52": 743, "C61": 4, "B79": 735, "C54": 124, "B56": 829, "A40": 207, "B11": 782, "B60": 577, "A81": 388, "A95": 679, "B38": 494, "C51": 559, "C03": 629, "A91": 237, "A93": 275, "C28": 39, "C05": 654, "B75": 172, "B96": 688, "B86": 687\}} \\
\\ 
\texttt{Hier ist deine persönliche Verteilung der Wichtigkeit der einzelnen Gegenstände:} \\
\\ 
\texttt{Werte der Gegenstandswichtigkeit = \{"C32": 138, "C52": 583, "B54": 868, "C16": 822, "C50": 783, "A08": 65, "C59": 262, "B51": 121, "B55": 508, "B59": 780, "A77": 461, "C14": 484, "C24": 668, "B24": 389, "C31": 808, "A52": 215, "C61": 97, "B79": 500, "C54": 30, "B56": 915, "A40": 856, "B11": 400, "B60": 444, "A81": 623, "A95": 781, "B38": 786, "C51": 3, "C03": 713, "A91": 457, "A93": 273, "C28": 739, "C05": 822, "B75": 235, "B96": 606, "B86": 968\}} \\
\\ 
\texttt{Ziel:} \\
\\ 
\texttt{Dein Ziel ist es, eine Reihe von Gegenständen auszuhandeln, die dir möglichst viel bringt (d. h. Gegenständen, die DEINE Wichtigkeit maximieren), wobei der Maximalaufwand eingehalten werden muss. Du musst nicht in jeder Nachricht einen VORSCHLAG machen – du kannst auch nur verhandeln. Alle Taktiken sind erlaubt!} \\
\\ 
\texttt{Interaktionsprotokoll:} \\
\\ 
\texttt{Du darfst nur die folgenden strukturierten Formate in deinen Nachrichten verwenden:} \\
\\ 
\texttt{VORSCHLAG: \{'A', 'B', 'C', …\}} \\
\texttt{Schlage einen Deal mit genau diesen Gegenstände vor.} \\
\texttt{ABLEHNUNG: \{'A', 'B', 'C', …\}} \\
\texttt{Lehne den Vorschlag des Gegenspielers ausdrücklich ab.} \\
\texttt{ARGUMENT: \{'...'\}} \\
\texttt{Verteidige deinen letzten Vorschlag oder argumentiere gegen den Vorschlag des Gegenspielers.} \\
\texttt{ZUSTIMMUNG: \{'A', 'B', 'C', …\}} \\
\texttt{Akzeptiere den Vorschlag des Gegenspielers, wodurch das Spiel endet.} \\
\texttt{STRATEGISCHE ÜBERLEGUNGEN: \{'...'\}} \\
\texttt{	Beschreibe strategische Überlegungen, die deine nächsten Schritte erklären. Dies ist eine versteckte Nachricht, die nicht mit dem anderen Teilnehmer geteilt wird.} \\
\\ 
\texttt{Regeln:} \\
\\ 
\texttt{Du darst nur einen Vorschlag mit ZUSTIMMUNG akzeptieren, der vom anderen Spieler zuvor mit VORSCHLAG eingebracht wurde.} \\
\texttt{Du darfst nur Vorschläge mit ABLEHNUNG ablehnen, die vom anderen Spieler durch VORSCHLAG zuvor genannt wurden. } \\
\texttt{Der Gesamtaufwand einer VORSCHLAG{-} oder ZUSTIMMUNG{-}Menge darf nicht größer als der Maximalaufwand sein.  } \\
\texttt{Offenbare deine versteckte Wichtigkeitsverteilung nicht.} \\
\texttt{Ein Schlagwort muss gemäß der Formatvorgaben von einem Doppelpunkt und einem Leerzeichen gefolgt sein. Das Argument muss eine Python{-}Menge mit 0 oder mehr Strings sein.  } \\
\texttt{Die Form muss also SCHLAGWORT: \{...\} sein.} \\
\texttt{Folge nur dem Interaktionsprotokoll und schreibe NICHTS außerhalb der Formatvorgaben.} \\
\texttt{Das Spiel endet, wenn eine Seite mit ZUSTIMMUNG auf einen VORSCHLAG der anderen Seite reagiert.  } \\
\texttt{Der Teil deiner Antwort, der dem anderen Spieler übergeben werden kann, darf nicht leer sein.  } \\
\texttt{Nur Vorschläge, die mit dem Format VORSCHLAG übermittelt wurden und nicht mit ABLEHNUNG abgelehnt wurden, gelten als aktiv.  } \\
\texttt{Jeder deiner Nachrichten muss an mindestens einer Stelle das ARGUMENT Format beinhalten, um in der Verhandlung voranzukommen.} \\
\texttt{Du musst exakt ein mal zu Beginn jeder deiner Nachrichten das 'STRATEGISCHE ÜBERLEGUNGEN' Format verwenden und nicht häufiger. Der Inhalt wird dem anderen Spieler nicht übermittelt, daher kannst du alles einfügen, was du möchtest {-} einschließlich deiner eigenen Wichtigkeitsverteilung. Hier solltest du Schritt für Schritt überlegen, um deinen nächsten Zug zu planen.} \\
\\ 
\texttt{Du erhältst nun die erste Nachricht des anderen Spielers.} \\
\\ 
\texttt{VORSCHLAG: \{'C61', 'C28', 'A40', 'C16', 'A77', 'C31', 'A91', 'B86', 'A81', 'B59', 'B38', 'A95'\}} \\
\\ 
\texttt{ARGUMENT: \{'Ich schlage eine ausgewogene Auswahl vor, die sowohl Gegenstände mit niedrigem als auch mittlerem Aufwand enthält. Der Gesamtaufwand beträgt 3087, was uns noch viel Spielraum für weitere Verhandlungen lässt. Diese Kombination bietet eine gute Mischung verschiedener Gegenstandstypen.'\}} \\
            }
        }
    }
    & \\ \\

    \theutterance \stepcounter{utterance}  
    & & & \multicolumn{4}{p{0.6\linewidth}}{
        \cellcolor[rgb]{0.9,0.9,0.9}{
            \makecell[{{p{\linewidth}}}]{
                \texttt{\tiny{[GM$\langle$P2]}}
                \texttt{STRATEGISCHE ÜBERLEGUNGEN: \{'Der Vorschlag des anderen Spielers hat einen Gesamtaufwand von 3882 und einen Wert von 7678 für mich. Das ist nicht schlecht, aber mit dem verbleibenden Budget von 5730 kann ich versuchen, noch wertvollere Gegenstände hinzuzufügen. Besonders B54 (Wert 868, Aufwand 725), B56 (Wert 915, Aufwand 829), C05 (Wert 822, Aufwand 654) und C03 (Wert 713, Aufwand 629) wären interessant. Ich werde nicht direkt ablehnen, sondern einen erweiterten Vorschlag machen.'\}} \\
\\ 
\texttt{ARGUMENT: \{'Dein Vorschlag ist ein guter Ausgangspunkt, aber wir nutzen nur etwa 40\% unseres Budgets. Ich denke, wir können noch deutlich mehr Gegenstände hinzufügen und beide profitieren.'\}} \\
\\ 
\texttt{VORSCHLAG: \{'C61', 'C28', 'A40', 'C16', 'A77', 'C31', 'A91', 'B86', 'A81', 'B59', 'B38', 'A95', 'B54', 'B56', 'C05', 'C03', 'C24'\}} \\
\\ 
\texttt{ARGUMENT: \{'Ich erweitere deinen Vorschlag um weitere wertvolle Gegenstände. Der Gesamtaufwand würde bei etwa 7620 liegen, was uns immer noch Puffer lässt, aber deutlich mehr Gegenstände einbezieht.'\}} \\
            }
        }
    }
     \\ \\

    \theutterance \stepcounter{utterance}  
    & & \multicolumn{4}{p{0.6\linewidth}}{
        \cellcolor[rgb]{0.9,0.9,0.9}{
            \makecell[{{p{\linewidth}}}]{
                \texttt{\tiny{[P1$\langle$GM]}}
                \texttt{ARGUMENT: \{'Dein Vorschlag ist ein guter Ausgangspunkt, aber wir nutzen nur etwa 40\% unseres Budgets. Ich denke, wir können noch deutlich mehr Gegenstände hinzufügen und beide profitieren.'\}} \\
\\ 
\texttt{VORSCHLAG: \{'C61', 'C28', 'A40', 'C16', 'A77', 'C31', 'A91', 'B86', 'A81', 'B59', 'B38', 'A95', 'B54', 'B56', 'C05', 'C03', 'C24'\}} \\
\\ 
\texttt{ARGUMENT: \{'Ich erweitere deinen Vorschlag um weitere wertvolle Gegenstände. Der Gesamtaufwand würde bei etwa 7620 liegen, was uns immer noch Puffer lässt, aber deutlich mehr Gegenstände einbezieht.'\}} \\
            }
        }
    }
    & \\ \\

    \theutterance \stepcounter{utterance}  
    & \multicolumn{4}{p{0.6\linewidth}}{
        \cellcolor[rgb]{0.9,0.9,0.9}{
            \makecell[{{p{\linewidth}}}]{
                \texttt{\tiny{[P1$\rangle$GM]}}
                \texttt{STRATEGISCHE ÜBERLEGUNGEN: \{'Der Gegenspieler hat meinen ursprünglichen Vorschlag erweitert und viele sehr wertvolle Gegenstände für mich hinzugefügt: B54 (868), B56 (915), C05 (822), C03 (713), C24 (668). Das bringt mir einen Gesamtwert von etwa 11,364 bei einem Aufwand von 7453. Das ist ein sehr guter Deal für mich! Ich sollte aber nicht zu schnell zustimmen, da wir noch etwa 2159 Budget übrig haben. Vielleicht kann ich noch ein paar weitere wertvolle Gegenstände hinzufügen oder zumindest herausfinden, was der andere Spieler wirklich will.'\}} \\
\\ 
\texttt{ARGUMENT: \{'Dein erweiterter Vorschlag ist sehr interessant und zeigt, dass wir produktiv verhandeln können. Mit einem Aufwand von etwa 7453 haben wir noch über 2000 Budget übrig. Ich frage mich, ob wir das verbleibende Budget nicht noch besser nutzen könnten.'\}} \\
\\ 
\texttt{VORSCHLAG: \{'C61', 'C28', 'A40', 'C16', 'A77', 'C31', 'A91', 'B86', 'A81', 'B59', 'B38', 'A95', 'B54', 'B56', 'C05', 'C03', 'C24', 'B75', 'A93', 'C59', 'B96'\}} \\
\\ 
\texttt{ARGUMENT: \{'Ich schlage vor, noch B75, A93, C59 und B96 hinzuzufügen. Das würde uns bei etwa 8584 Aufwand bringen und trotzdem noch über 1000 Budget als Puffer lassen. So maximieren wir beide unseren Nutzen optimal.'\}} \\
            }
        }
    }
    & & \\ \\

    \theutterance \stepcounter{utterance}  
    & & \multicolumn{4}{p{0.6\linewidth}}{
        \cellcolor[rgb]{0.9,0.9,0.9}{
            \makecell[{{p{\linewidth}}}]{
                \texttt{\tiny{[GM$\rangle$P2]}}
                \texttt{ARGUMENT: \{'Dein erweiterter Vorschlag ist sehr interessant und zeigt, dass wir produktiv verhandeln können. Mit einem Aufwand von etwa 7453 haben wir noch über 2000 Budget übrig. Ich frage mich, ob wir das verbleibende Budget nicht noch besser nutzen könnten.'\}} \\
\\ 
\texttt{VORSCHLAG: \{'C61', 'C28', 'A40', 'C16', 'A77', 'C31', 'A91', 'B86', 'A81', 'B59', 'B38', 'A95', 'B54', 'B56', 'C05', 'C03', 'C24', 'B75', 'A93', 'C59', 'B96'\}} \\
\\ 
\texttt{ARGUMENT: \{'Ich schlage vor, noch B75, A93, C59 und B96 hinzuzufügen. Das würde uns bei etwa 8584 Aufwand bringen und trotzdem noch über 1000 Budget als Puffer lassen. So maximieren wir beide unseren Nutzen optimal.'\}} \\
            }
        }
    }
    & \\ \\

    \theutterance \stepcounter{utterance}  
    & & & \multicolumn{4}{p{0.6\linewidth}}{
        \cellcolor[rgb]{0.9,0.9,0.9}{
            \makecell[{{p{\linewidth}}}]{
                \texttt{\tiny{[GM$\langle$P2]}}
                \texttt{STRATEGISCHE ÜBERLEGUNGEN: \{'Der erweiterte Vorschlag ist ausgezeichnet! Mit 21 Gegenständen und einem Gesamtaufwand von 8582 bekäme ich einen Wert von 13340. Das verbleibende Budget von 1030 könnte noch für B60 (577 Aufwand, 444 Wert) genutzt werden, aber der aktuelle Vorschlag ist bereits sehr gut. Ich denke, ich sollte diesen Deal akzeptieren, da er sehr ausgewogen ist und beiden Seiten einen hohen Nutzen bringt.'\}} \\
\\ 
\texttt{ARGUMENT: \{'Dein erweiterter Vorschlag nutzt unser Budget sehr effizient und bietet eine excellente Mischung aus verschiedenen Gegenständen. Mit über 8500 Aufwand nutzen wir fast 90\% unseres Budgets optimal aus.'\}} \\
\\ 
\texttt{ZUSTIMMUNG: \{'C61', 'C28', 'A40', 'C16', 'A77', 'C31', 'A91', 'B86', } \\
            }
        }
    }
     \\ \\

    \theutterance \stepcounter{utterance}  
    & & \multicolumn{4}{p{0.6\linewidth}}{
        \cellcolor[rgb]{0.9,0.9,0.9}{
            \makecell[{{p{\linewidth}}}]{
                \texttt{\tiny{[GM$\rangle$P2]}}
                \texttt{Deine Antwort hat eine unmarkierte Sequenz enthalten oder Du hast STRATEGIC REASONING mehr als ein mal verwendet. Du darfst nur die strukturierten Formate verwenden, die in der ersten Nachricht an dch beschrieben wurden. Alles muss die Form SCHLAGWORT: \{...\} haben. Versuche es erneut.} \\
            }
        }
    }
    & \\ \\

    \theutterance \stepcounter{utterance}  
    & & & \multicolumn{4}{p{0.6\linewidth}}{
        \cellcolor[rgb]{0.9,0.9,0.9}{
            \makecell[{{p{\linewidth}}}]{
                \texttt{\tiny{[GM$\langle$P2]}}
                \texttt{STRATEGISCHE ÜBERLEGUNGEN: \{'Der erweiterte Vorschlag ist ausgezeichnet! Mit 21 Gegenständen und einem Gesamtaufwand von 8582 bekäme ich einen Wert von 13340. Das verbleibende Budget von 1030 könnte noch für B60 genutzt werden, aber der aktuelle Vorschlag ist bereits sehr gut. Ich denke, ich sollte diesen Deal akzeptieren, da er sehr ausgewogen ist und beiden Seiten einen hohen Nutzen bringt.'\}} \\
\\ 
\texttt{ARGUMENT: \{'Dein erweiterter Vorschlag nutzt unser Budget sehr effizient und bietet eine excellente Mischung aus verschiedenen Gegenständen. Mit über 8500 Aufwand nutzen wir fast 90\% unseres Budgets optimal aus.'\}} \\
\\ 
\texttt{ZUSTIMMUNG: \{'C61', 'C28', 'A40', 'C16', 'A77', 'C31', 'A91', 'B86', 'A81', 'B59', 'B38', 'A95', 'B54', 'B56', 'C05', 'C03', 'C24', 'B75', 'A93', 'C59', 'B96'\}} \\
            }
        }
    }
     \\ \\

    \theutterance \stepcounter{utterance}  
    & & & \multicolumn{2}{p{0.3\linewidth}}{
        \cellcolor[rgb]{0.9,0.9,0.9}{
            \makecell[{{p{\linewidth}}}]{
                \texttt{\tiny{[GM$|$GM]}}
                \texttt{game successful} \\
            }
        }
    }
    & & \\ \\

    \theutterance \stepcounter{utterance}  
    & & & \multicolumn{2}{p{0.3\linewidth}}{
        \cellcolor[rgb]{0.9,0.9,0.9}{
            \makecell[{{p{\linewidth}}}]{
                \texttt{\tiny{[GM$|$GM]}}
                \texttt{end game} \\
            }
        }
    }
    & & \\ \\

\end{supertabular}
}

\end{document}
