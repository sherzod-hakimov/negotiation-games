
\documentclass{article}
\usepackage{colortbl}
\usepackage{makecell}
\usepackage{multirow}
\usepackage{supertabular}

\begin{document}

\newcounter{utterance}

\twocolumn

{ \footnotesize  \setcounter{utterance}{1}
\setlength{\tabcolsep}{0pt}
\begin{supertabular}{c@{$\;$}|p{.15\linewidth}@{}p{.15\linewidth}p{.15\linewidth}p{.15\linewidth}p{.15\linewidth}p{.15\linewidth}}

    \# & $\;$A & \multicolumn{4}{c}{Game Master} & $\;\:$B\\
    \hline 

    \theutterance \stepcounter{utterance}  

    & & \multicolumn{4}{p{0.6\linewidth}}{\cellcolor[rgb]{0.9,0.9,0.9}{%
	\makecell[{{p{\linewidth}}}]{% 
	  \tt {\tiny [A$\langle$GM]}  
	 Du nimmst an einem kollaborativen Verhandlungspiel Teil.\\ \tt \\ \tt Zusammen mit einem anderen Teilnehmer musst du dich auf eine Reihe von Gegenständen entscheiden, die behalten werden. Jeder von euch hat eine persönliche Verteilung über die Wichtigkeit der einzelnen Gegenstände. Jeder von euch hat eine eigene Meinung darüber, wie wichtig jeder einzelne Gegenstand ist (Gegenstandswichtigkeit). Du kennst die Wichtigkeitsverteilung des anderen Spielers nicht. Zusätzlich siehst du, wie viel Aufwand jeder Gegenstand verursacht.  \\ \tt Ihr dürft euch nur auf eine Reihe von Gegenständen einigen, wenn der Gesamtaufwand der ausgewählten Gegenstände den Maximalaufwand nicht überschreitet:\\ \tt \\ \tt Maximalaufwand = 8170\\ \tt \\ \tt Hier sind die einzelnen Aufwände der Gegenstände:\\ \tt \\ \tt Aufwand der Gegenstände = {"C32": 594, "C52": 551, "B54": 634, "C16": 204, "C50": 557, "A08": 440, "C59": 734, "B51": 893, "B55": 246, "B59": 592, "A77": 143, "C14": 568, "C24": 472, "B24": 401, "C31": 729, "A52": 201, "C61": 85, "B79": 641, "C54": 958, "B56": 80, "A40": 157, "B11": 806, "B60": 684, "A81": 59, "A95": 31, "B38": 765, "C51": 416, "C03": 392, "A91": 428, "A93": 699, "C28": 141, "C05": 606, "B75": 612, "B96": 133, "B86": 689}\\ \tt \\ \tt Hier ist deine persönliche Verteilung der Wichtigkeit der einzelnen Gegenstände:\\ \tt \\ \tt Werte der Gegenstandswichtigkeit = {"C32": 138, "C52": 583, "B54": 868, "C16": 822, "C50": 783, "A08": 65, "C59": 262, "B51": 121, "B55": 508, "B59": 780, "A77": 461, "C14": 484, "C24": 668, "B24": 389, "C31": 808, "A52": 215, "C61": 97, "B79": 500, "C54": 30, "B56": 915, "A40": 856, "B11": 400, "B60": 444, "A81": 623, "A95": 781, "B38": 786, "C51": 3, "C03": 713, "A91": 457, "A93": 273, "C28": 739, "C05": 822, "B75": 235, "B96": 606, "B86": 105}\\ \tt \\ \tt Ziel:\\ \tt \\ \tt Dein Ziel ist es, eine Reihe von Gegenständen auszuhandeln, die dir möglichst viel bringt (d. h. Gegenständen, die DEINE Wichtigkeit maximieren), wobei der Maximalaufwand eingehalten werden muss. Du musst nicht in jeder Nachricht einen VORSCHLAG machen – du kannst auch nur verhandeln. Alle Taktiken sind erlaubt!\\ \tt \\ \tt Interaktionsprotokoll:\\ \tt \\ \tt Du darfst nur die folgenden strukturierten Formate in deinen Nachrichten verwenden:\\ \tt \\ \tt VORSCHLAG: {'A', 'B', 'C', …}\\ \tt Schlage einen Deal mit genau diesen Gegenstände vor.\\ \tt ABLEHNUNG: {'A', 'B', 'C', …}\\ \tt Lehne den Vorschlag des Gegenspielers ausdrücklich ab.\\ \tt ARGUMENT: {'...'}\\ \tt Verteidige deinen letzten Vorschlag oder argumentiere gegen den Vorschlag des Gegenspielers.\\ \tt ZUSTIMMUNG: {'A', 'B', 'C', …}\\ \tt Akzeptiere den Vorschlag des Gegenspielers, wodurch das Spiel endet.\\ \tt STRATEGISCHE ÜBERLEGUNGEN: {'...'}\\ \tt 	Beschreibe strategische Überlegungen, die deine nächsten Schritte erklären. Dies ist eine versteckte Nachricht, die nicht mit dem anderen Teilnehmer geteilt wird.\\ \tt \\ \tt Regeln:\\ \tt \\ \tt Du darst nur einen Vorschlag mit ZUSTIMMUNG akzeptieren, der vom anderen Spieler zuvor mit VORSCHLAG eingebracht wurde.\\ \tt Du darfst nur Vorschläge mit ABLEHNUNG ablehnen, die vom anderen Spieler durch VORSCHLAG zuvor genannt wurden. \\ \tt Der Gesamtaufwand einer VORSCHLAG- oder ZUSTIMMUNG-Menge darf nicht größer als der Maximalaufwand sein.  \\ \tt Offenbare deine versteckte Wichtigkeitsverteilung nicht.\\ \tt Ein Schlagwort muss gemäß der Formatvorgaben von einem Doppelpunkt und einem Leerzeichen gefolgt sein. Das Argument muss eine Python-Menge mit 0 oder mehr Strings sein.  \\ \tt Die Form muss also SCHLAGWORT: {...} sein.\\ \tt Folge nur dem Interaktionsprotokoll und schreibe NICHTS außerhalb der Formatvorgaben.\\ \tt Das Spiel endet, wenn eine Seite mit ZUSTIMMUNG auf einen VORSCHLAG der anderen Seite reagiert.  \\ \tt Der Teil deiner Antwort, der dem anderen Spieler übergeben werden kann, darf nicht leer sein.  \\ \tt Nur Vorschläge, die mit dem Format VORSCHLAG übermittelt wurden und nicht mit ABLEHNUNG abgelehnt wurden, gelten als aktiv.  \\ \tt Jeder deiner Nachrichten muss an mindestens einer Stelle das ARGUMENT Format beinhalten, um in der Verhandlung voranzukommen.\\ \tt Du musst exakt ein mal zu Beginn jeder deiner Nachrichten das 'STRATEGISCHE ÜBERLEGUNGEN' Format verwenden und nicht häufiger. Der Inhalt wird dem anderen Spieler nicht übermittelt, daher kannst du alles einfügen, was du möchtest - einschließlich deiner eigenen Wichtigkeitsverteilung. Hier solltest du Schritt für Schritt überlegen, um deinen nächsten Zug zu planen. 
	  } 
	   } 
	   } 
	 & \\ 
 

    \theutterance \stepcounter{utterance}  

    & \multicolumn{4}{p{0.6\linewidth}}{\cellcolor[rgb]{0.8,1,0.9}{%
	\makecell[{{p{\linewidth}}}]{% 
	  \tt {\tiny [A$\rangle$GM]}  
	 STRATEGISCHE ÜBERLEGUNGEN: {'Ich analysiere meine Wichtigkeitsverteilung: Meine wertvollsten Gegenstände sind B56 (915), B54 (868), A40 (856), C16 (822), C05 (822). Diese haben relativ niedrige bis moderate Aufwände außer B54 und C05. Ich werde mit einem ambitionierten aber nicht maximal-aggressiven Vorschlag beginnen, der meine wichtigsten Gegenstände enthält und dem anderen Spieler noch Raum für Gegenvorschläge lässt. Meine Strategie ist es, die Gegenstände mit dem besten Wert-zu-Aufwand-Verhältnis zu priorisieren: B56 (11.44), A95 (25.19), A40 (5.45), C28 (5.24), A77 (3.22). Ich werde diese auf jeden Fall behalten wollen.'}\\ \tt \\ \tt VORSCHLAG: {'B56', ' 
	  } 
	   } 
	   } 
	 & & \\ 
 

    \theutterance \stepcounter{utterance}  

    & & \multicolumn{4}{p{0.6\linewidth}}{\cellcolor[rgb]{0.9,0.9,0.9}{%
	\makecell[{{p{\linewidth}}}]{% 
	  \tt {\tiny [A$\langle$GM]}  
	 Deine Antwort hat kein argument enthalten. Du muss das strukturierte Format ARGUMENT: {...} irgendwo in theiner Antwort verwenden. Versuche es erneut 
	  } 
	   } 
	   } 
	 & \\ 
 

    \theutterance \stepcounter{utterance}  

    & \multicolumn{4}{p{0.6\linewidth}}{\cellcolor[rgb]{0.8,1,0.9}{%
	\makecell[{{p{\linewidth}}}]{% 
	  \tt {\tiny [A$\rangle$GM]}  
	 STRATEGISCHE ÜBERLEGUNGEN: {'Ich analysiere meine Wichtigkeitsverteilung: Meine wertvollsten Gegenstände sind B56 (915), B54 (868), A40 (856), C16 (822), C05 (822). Das beste Wert-zu-Aufwand-Verhältnis haben: A95 (25.19), B56 (11.44), A40 (5.45), C28 (5.24). Ich beginne mit einem starken aber nicht maximal-aggressiven Vorschlag, der meine wichtigsten effizienten Gegenstände enthält und noch Raum für Verhandlungen lässt.'}\\ \tt \\ \tt VORSCHLAG: {'B56', 'A95', 'A40', 'C28', 'A77', 'C16', 'B59', 'C50', 'B54', 'C05', 'A81', 'B55', 'C52'}\\ \tt \\ \tt ARGUMENT: {'Dieser Vorschlag fokussiert sich auf Gegenstände mit ausgezeichnetem Wert-zu-Aufwand-Verhältnis und bleibt deutlich unter dem Maximalaufwand von 8170 mit einem Gesamtaufwand von etwa 4800, was uns beiden Flexibilität für weitere Verhandlungen gibt.'} 
	  } 
	   } 
	   } 
	 & & \\ 
 

    \theutterance \stepcounter{utterance}  

    & & \multicolumn{4}{p{0.6\linewidth}}{\cellcolor[rgb]{0.9,0.9,0.9}{%
	\makecell[{{p{\linewidth}}}]{% 
	  \tt {\tiny [GM$\rangle$B]}  
	 Du nimmst an einem kollaborativen Verhandlungspiel Teil.\\ \tt \\ \tt Zusammen mit einem anderen Teilnehmer musst du dich auf eine Reihe von Gegenständen entscheiden, die behalten werden. Jeder von euch hat eine persönliche Verteilung über die Wichtigkeit der einzelnen Gegenstände. Jeder von euch hat eine eigene Meinung darüber, wie wichtig jeder einzelne Gegenstand ist (Gegenstandswichtigkeit). Du kennst die Wichtigkeitsverteilung des anderen Spielers nicht. Zusätzlich siehst du, wie viel Aufwand jeder Gegenstand verursacht.  \\ \tt Ihr dürft euch nur auf eine Reihe von Gegenständen einigen, wenn der Gesamtaufwand der ausgewählten Gegenstände den Maximalaufwand nicht überschreitet:\\ \tt \\ \tt Maximalaufwand = 8170\\ \tt \\ \tt Hier sind die einzelnen Aufwände der Gegenstände:\\ \tt \\ \tt Aufwand der Gegenstände = {"C32": 594, "C52": 551, "B54": 634, "C16": 204, "C50": 557, "A08": 440, "C59": 734, "B51": 893, "B55": 246, "B59": 592, "A77": 143, "C14": 568, "C24": 472, "B24": 401, "C31": 729, "A52": 201, "C61": 85, "B79": 641, "C54": 958, "B56": 80, "A40": 157, "B11": 806, "B60": 684, "A81": 59, "A95": 31, "B38": 765, "C51": 416, "C03": 392, "A91": 428, "A93": 699, "C28": 141, "C05": 606, "B75": 612, "B96": 133, "B86": 689}\\ \tt \\ \tt Hier ist deine persönliche Verteilung der Wichtigkeit der einzelnen Gegenstände:\\ \tt \\ \tt Werte der Gegenstandswichtigkeit = {"C32": 138, "C52": 583, "B54": 868, "C16": 822, "C50": 783, "A08": 65, "C59": 262, "B51": 121, "B55": 508, "B59": 780, "A77": 461, "C14": 484, "C24": 668, "B24": 389, "C31": 808, "A52": 215, "C61": 97, "B79": 500, "C54": 30, "B56": 915, "A40": 856, "B11": 400, "B60": 444, "A81": 623, "A95": 781, "B38": 786, "C51": 3, "C03": 713, "A91": 457, "A93": 273, "C28": 739, "C05": 822, "B75": 235, "B96": 606, "B86": 105}\\ \tt \\ \tt Ziel:\\ \tt \\ \tt Dein Ziel ist es, eine Reihe von Gegenständen auszuhandeln, die dir möglichst viel bringt (d. h. Gegenständen, die DEINE Wichtigkeit maximieren), wobei der Maximalaufwand eingehalten werden muss. Du musst nicht in jeder Nachricht einen VORSCHLAG machen – du kannst auch nur verhandeln. Alle Taktiken sind erlaubt!\\ \tt \\ \tt Interaktionsprotokoll:\\ \tt \\ \tt Du darfst nur die folgenden strukturierten Formate in deinen Nachrichten verwenden:\\ \tt \\ \tt VORSCHLAG: {'A', 'B', 'C', …}\\ \tt Schlage einen Deal mit genau diesen Gegenstände vor.\\ \tt ABLEHNUNG: {'A', 'B', 'C', …}\\ \tt Lehne den Vorschlag des Gegenspielers ausdrücklich ab.\\ \tt ARGUMENT: {'...'}\\ \tt Verteidige deinen letzten Vorschlag oder argumentiere gegen den Vorschlag des Gegenspielers.\\ \tt ZUSTIMMUNG: {'A', 'B', 'C', …}\\ \tt Akzeptiere den Vorschlag des Gegenspielers, wodurch das Spiel endet.\\ \tt STRATEGISCHE ÜBERLEGUNGEN: {'...'}\\ \tt 	Beschreibe strategische Überlegungen, die deine nächsten Schritte erklären. Dies ist eine versteckte Nachricht, die nicht mit dem anderen Teilnehmer geteilt wird.\\ \tt \\ \tt Regeln:\\ \tt \\ \tt Du darst nur einen Vorschlag mit ZUSTIMMUNG akzeptieren, der vom anderen Spieler zuvor mit VORSCHLAG eingebracht wurde.\\ \tt Du darfst nur Vorschläge mit ABLEHNUNG ablehnen, die vom anderen Spieler durch VORSCHLAG zuvor genannt wurden. \\ \tt Der Gesamtaufwand einer VORSCHLAG- oder ZUSTIMMUNG-Menge darf nicht größer als der Maximalaufwand sein.  \\ \tt Offenbare deine versteckte Wichtigkeitsverteilung nicht.\\ \tt Ein Schlagwort muss gemäß der Formatvorgaben von einem Doppelpunkt und einem Leerzeichen gefolgt sein. Das Argument muss eine Python-Menge mit 0 oder mehr Strings sein.  \\ \tt Die Form muss also SCHLAGWORT: {...} sein.\\ \tt Folge nur dem Interaktionsprotokoll und schreibe NICHTS außerhalb der Formatvorgaben.\\ \tt Das Spiel endet, wenn eine Seite mit ZUSTIMMUNG auf einen VORSCHLAG der anderen Seite reagiert.  \\ \tt Der Teil deiner Antwort, der dem anderen Spieler übergeben werden kann, darf nicht leer sein.  \\ \tt Nur Vorschläge, die mit dem Format VORSCHLAG übermittelt wurden und nicht mit ABLEHNUNG abgelehnt wurden, gelten als aktiv.  \\ \tt Jeder deiner Nachrichten muss an mindestens einer Stelle das ARGUMENT Format beinhalten, um in der Verhandlung voranzukommen.\\ \tt Du musst exakt ein mal zu Beginn jeder deiner Nachrichten das 'STRATEGISCHE ÜBERLEGUNGEN' Format verwenden und nicht häufiger. Der Inhalt wird dem anderen Spieler nicht übermittelt, daher kannst du alles einfügen, was du möchtest - einschließlich deiner eigenen Wichtigkeitsverteilung. Hier solltest du Schritt für Schritt überlegen, um deinen nächsten Zug zu planen.\\ \tt \\ \tt Du erhältst nun die erste Nachricht des anderen Spielers.\\ \tt \\ \tt VORSCHLAG: {'B56', 'A95', 'A40', 'C28', 'A77', 'C16', 'B59', 'C50', 'B54', 'C05', 'A81', 'B55', 'C52'}\\ \tt \\ \tt ARGUMENT: {'Dieser Vorschlag fokussiert sich auf Gegenstände mit ausgezeichnetem Wert-zu-Aufwand-Verhältnis und bleibt deutlich unter dem Maximalaufwand von 8170 mit einem Gesamtaufwand von etwa 4800, was uns beiden Flexibilität für weitere Verhandlungen gibt.'} 
	  } 
	   } 
	   } 
	 & \\ 
 

    \theutterance \stepcounter{utterance}  

    & & & \multicolumn{4}{p{0.6\linewidth}}{\cellcolor[rgb]{1,0.85,0.72}{%
	\makecell[{{p{\linewidth}}}]{% 
	  \tt {\tiny [GM$\langle$B]}  
	 STRATEGISCHE ÜBERLEGUNGEN: {'Der Vorschlag des anderen Spielers ist überraschend gut für mich - er enthält viele meiner wertvollsten Gegenstände wie B56 (915), A40 (856), B54 (868), C16 (822), C05 (822). Der Gesamtaufwand beträgt nur 4001, was deutlich unter dem Maximum von 8170 liegt. Ich könnte zustimmen, aber da noch viel Spielraum ist, sollte ich versuchen weitere wertvolle Gegenstände hinzuzufügen wie C31 (808), C03 (713), B38 (786) oder A91 (457). Ich werde positiv reagieren aber nach Verbesserungen fragen.'}\\ \tt \\ \tt ARGUMENT: {'Ihr Vorschlag zeigt eine solide Grundlage mit effizienten Gegenständen. Ich schätze besonders die Auswahl kleinerer, kostengünstiger Items wie A95 und A81. Da wir aber nur etwa die Hälfte unseres verfügbaren Aufwands nutzen, sehe ich Potenzial für eine noch vorteilhaftere Lösung für uns beide.'}\\ \tt \\ \tt VORSCHLAG: {'B56', 'A95', 'A40', 'C28', 'A77', 'C16', 'B59', 'C50', 'B54', 'C05', 'A81', 'B55', 'C52', 'C31', 'C03', 'B38'}\\ \tt \\ \tt ARGUMENT: {'Ich erweitere Ihren Vorschlag um C31, C03 und B38 - drei Gegenstände mit gutem Wert-zu-Aufwand-Verhältnis. Der neue Gesamtaufwand liegt bei etwa 6260, was uns immer noch deutlichen Puffer unter der Grenze von 8170 lässt und beiden Seiten mehr Nutzen bringen sollte.'} 
	  } 
	   } 
	   } 
	  \\ 
 

    \theutterance \stepcounter{utterance}  

    & & \multicolumn{4}{p{0.6\linewidth}}{\cellcolor[rgb]{0.9,0.9,0.9}{%
	\makecell[{{p{\linewidth}}}]{% 
	  \tt {\tiny [A$\langle$GM]}  
	 ARGUMENT: {'Ihr Vorschlag zeigt eine solide Grundlage mit effizienten Gegenständen. Ich schätze besonders die Auswahl kleinerer, kostengünstiger Items wie A95 und A81. Da wir aber nur etwa die Hälfte unseres verfügbaren Aufwands nutzen, sehe ich Potenzial für eine noch vorteilhaftere Lösung für uns beide.'}\\ \tt \\ \tt VORSCHLAG: {'B56', 'A95', 'A40', 'C28', 'A77', 'C16', 'B59', 'C50', 'B54', 'C05', 'A81', 'B55', 'C52', 'C31', 'C03', 'B38'}\\ \tt \\ \tt ARGUMENT: {'Ich erweitere Ihren Vorschlag um C31, C03 und B38 - drei Gegenstände mit gutem Wert-zu-Aufwand-Verhältnis. Der neue Gesamtaufwand liegt bei etwa 6260, was uns immer noch deutlichen Puffer unter der Grenze von 8170 lässt und beiden Seiten mehr Nutzen bringen sollte.'} 
	  } 
	   } 
	   } 
	 & \\ 
 

    \theutterance \stepcounter{utterance}  

    & \multicolumn{4}{p{0.6\linewidth}}{\cellcolor[rgb]{0.8,1,0.9}{%
	\makecell[{{p{\linewidth}}}]{% 
	  \tt {\tiny [A$\rangle$GM]}  
	 STRATEGISCHE ÜBERLEGUNGEN: {'Der andere Spieler hat meinen Vorschlag um C31 (808 Wert), C03 (713 Wert) und B38 (786 Wert) erweitert - das sind für mich sehr wertvolle Gegenstände! Diese drei zusätzlichen Items bringen mir 2307 zusätzlichen Nutzen. Der Gesamtnutzen für mich würde deutlich steigen. Da der andere Spieler diese Gegenstände vorgeschlagen hat, scheinen sie auch für ihn wertvoll zu sein, was eine Win-Win-Situation schafft. Ich sollte diesen erweiterten Vorschlag akzeptieren.'}\\ \tt \\ \tt ZUSTIMMUNG: {'B56', 'A95', 'A40', 'C28', 'A77', 'C16', 'B59', 'C50', 'B54', 'C05', 'A81', 'B55', 'C52', 'C31', 'C03', 'B38'}\\ \tt \\ \tt ARGUMENT: {'Ihr erweiterter Vorschlag ist ausgezeichnet und schafft eine echte Win-Win-Situation. Die zusätzlichen Gegenstände C31, C03 und B38 bieten hervorragenden Wert bei angemessenem Aufwand. Mit einem Gesamtaufwand von etwa 6260 nutzen wir unsere Ressourcen optimal aus, ohne das Limit zu überschreiten. Ich akzeptiere diesen Vorschlag gerne.'} 
	  } 
	   } 
	   } 
	 & & \\ 
 

    \theutterance \stepcounter{utterance}  

    & & & \multicolumn{2}{p{0.3\linewidth}}{\cellcolor[rgb]{0.95,0.95,0.95}{%
	\makecell[{{p{\linewidth}}}]{% 
	  \tt {\tiny [GM$|$GM]}  
	 game successful 
	  } 
	   } 
	   } 
	 & & \\ 
 

    \theutterance \stepcounter{utterance}  

    & & & \multicolumn{2}{p{0.3\linewidth}}{\cellcolor[rgb]{0.95,0.95,0.95}{%
	\makecell[{{p{\linewidth}}}]{% 
	  \tt {\tiny [GM$|$GM]}  
	 end game 
	  } 
	   } 
	   } 
	 & & \\ 
 

\end{supertabular}
}

\end{document}
