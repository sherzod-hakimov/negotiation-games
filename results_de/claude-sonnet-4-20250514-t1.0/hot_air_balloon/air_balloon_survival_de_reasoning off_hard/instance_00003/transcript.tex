
\documentclass{article}
\usepackage{colortbl}
\usepackage{makecell}
\usepackage{multirow}
\usepackage{supertabular}

\begin{document}

\newcounter{utterance}

\twocolumn

{ \footnotesize  \setcounter{utterance}{1}
\setlength{\tabcolsep}{0pt}
\begin{supertabular}{c@{$\;$}|p{.15\linewidth}@{}p{.15\linewidth}p{.15\linewidth}p{.15\linewidth}p{.15\linewidth}p{.15\linewidth}}

    \# & $\;$A & \multicolumn{4}{c}{Game Master} & $\;\:$B\\
    \hline 

    \theutterance \stepcounter{utterance}  

    & & \multicolumn{4}{p{0.6\linewidth}}{\cellcolor[rgb]{0.9,0.9,0.9}{%
	\makecell[{{p{\linewidth}}}]{% 
	  \tt {\tiny [A$\langle$GM]}  
	 Du nimmst an einem kollaborativen Verhandlungspiel Teil.\\ \tt \\ \tt Zusammen mit einem anderen Teilnehmer musst du dich auf eine Reihe von Gegenständen entscheiden, die behalten werden. Jeder von euch hat eine persönliche Verteilung über die Wichtigkeit der einzelnen Gegenstände. Jeder von euch hat eine eigene Meinung darüber, wie wichtig jeder einzelne Gegenstand ist (Gegenstandswichtigkeit). Du kennst die Wichtigkeitsverteilung des anderen Spielers nicht. Zusätzlich siehst du, wie viel Aufwand jeder Gegenstand verursacht.  \\ \tt Ihr dürft euch nur auf eine Reihe von Gegenständen einigen, wenn der Gesamtaufwand der ausgewählten Gegenstände den Maximalaufwand nicht überschreitet:\\ \tt \\ \tt Maximalaufwand = 3629\\ \tt \\ \tt Hier sind die einzelnen Aufwände der Gegenstände:\\ \tt \\ \tt Aufwand der Gegenstände = {"C76": 5, "C38": 793, "C56": 21, "A03": 842, "A07": 313, "A83": 842, "C98": 631, "C08": 226, "C62": 87, "C00": 761, "C10": 230, "C81": 287, "B38": 698, "C03": 641, "C32": 882}\\ \tt \\ \tt Hier ist deine persönliche Verteilung der Wichtigkeit der einzelnen Gegenstände:\\ \tt \\ \tt Werte der Gegenstandswichtigkeit = {"C76": 138, "C38": 583, "C56": 868, "A03": 822, "A07": 783, "A83": 65, "C98": 262, "C08": 121, "C62": 508, "C00": 780, "C10": 461, "C81": 484, "B38": 668, "C03": 389, "C32": 808}\\ \tt \\ \tt Ziel:\\ \tt \\ \tt Dein Ziel ist es, eine Reihe von Gegenständen auszuhandeln, die dir möglichst viel bringt (d. h. Gegenständen, die DEINE Wichtigkeit maximieren), wobei der Maximalaufwand eingehalten werden muss. Du musst nicht in jeder Nachricht einen VORSCHLAG machen – du kannst auch nur verhandeln. Alle Taktiken sind erlaubt!\\ \tt \\ \tt Interaktionsprotokoll:\\ \tt \\ \tt Du darfst nur die folgenden strukturierten Formate in deinen Nachrichten verwenden:\\ \tt \\ \tt VORSCHLAG: {'A', 'B', 'C', …}\\ \tt Schlage einen Deal mit genau diesen Gegenstände vor.\\ \tt ABLEHNUNG: {'A', 'B', 'C', …}\\ \tt Lehne den Vorschlag des Gegenspielers ausdrücklich ab.\\ \tt ARGUMENT: {'...'}\\ \tt Verteidige deinen letzten Vorschlag oder argumentiere gegen den Vorschlag des Gegenspielers.\\ \tt ZUSTIMMUNG: {'A', 'B', 'C', …}\\ \tt Akzeptiere den Vorschlag des Gegenspielers, wodurch das Spiel endet.\\ \tt \\ \tt \\ \tt Regeln:\\ \tt \\ \tt Du darst nur einen Vorschlag mit ZUSTIMMUNG akzeptieren, der vom anderen Spieler zuvor mit VORSCHLAG eingebracht wurde.\\ \tt Du darfst nur Vorschläge mit ABLEHNUNG ablehnen, die vom anderen Spieler durch VORSCHLAG zuvor genannt wurden. \\ \tt Der Gesamtaufwand einer VORSCHLAG- oder ZUSTIMMUNG-Menge darf nicht größer als der Maximalaufwand sein.  \\ \tt Offenbare deine versteckte Wichtigkeitsverteilung nicht.\\ \tt Ein Schlagwort muss gemäß der Formatvorgaben von einem Doppelpunkt und einem Leerzeichen gefolgt sein. Das Argument muss eine Python-Menge mit 0 oder mehr Strings sein.  \\ \tt Die Form muss also SCHLAGWORT: {...} sein.\\ \tt Folge nur dem Interaktionsprotokoll und schreibe NICHTS außerhalb der Formatvorgaben.\\ \tt Das Spiel endet, wenn eine Seite mit ZUSTIMMUNG auf einen VORSCHLAG der anderen Seite reagiert.  \\ \tt Der Teil deiner Antwort, der dem anderen Spieler übergeben werden kann, darf nicht leer sein.  \\ \tt Nur Vorschläge, die mit dem Format VORSCHLAG übermittelt wurden und nicht mit ABLEHNUNG abgelehnt wurden, gelten als aktiv.  \\ \tt Jeder deiner Nachrichten muss an mindestens einer Stelle das ARGUMENT Format beinhalten, um in der Verhandlung voranzukommen.\\ \tt  
	  } 
	   } 
	   } 
	 & \\ 
 

    \theutterance \stepcounter{utterance}  

    & \multicolumn{4}{p{0.6\linewidth}}{\cellcolor[rgb]{0.8,1,0.9}{%
	\makecell[{{p{\linewidth}}}]{% 
	  \tt {\tiny [A$\rangle$GM]}  
	 VORSCHLAG: {'C56', 'C76', 'C62', 'A07', 'C10', 'C81', 'C08', 'C00', 'A03'}\\ \tt \\ \tt ARGUMENT: {'Dieser Vorschlag berücksichtigt eine ausgewogene Mischung verschiedener Gegenstände mit unterschiedlichen Aufwandsniveaus, wodurch wir beide vom verfügbaren Budget profitieren können'} 
	  } 
	   } 
	   } 
	 & & \\ 
 

    \theutterance \stepcounter{utterance}  

    & & \multicolumn{4}{p{0.6\linewidth}}{\cellcolor[rgb]{0.9,0.9,0.9}{%
	\makecell[{{p{\linewidth}}}]{% 
	  \tt {\tiny [GM$\rangle$B]}  
	 Du nimmst an einem kollaborativen Verhandlungspiel Teil.\\ \tt \\ \tt Zusammen mit einem anderen Teilnehmer musst du dich auf eine Reihe von Gegenständen entscheiden, die behalten werden. Jeder von euch hat eine persönliche Verteilung über die Wichtigkeit der einzelnen Gegenstände. Jeder von euch hat eine eigene Meinung darüber, wie wichtig jeder einzelne Gegenstand ist (Gegenstandswichtigkeit). Du kennst die Wichtigkeitsverteilung des anderen Spielers nicht. Zusätzlich siehst du, wie viel Aufwand jeder Gegenstand verursacht.  \\ \tt Ihr dürft euch nur auf eine Reihe von Gegenständen einigen, wenn der Gesamtaufwand der ausgewählten Gegenstände den Maximalaufwand nicht überschreitet:\\ \tt \\ \tt Maximalaufwand = 3629\\ \tt \\ \tt Hier sind die einzelnen Aufwände der Gegenstände:\\ \tt \\ \tt Aufwand der Gegenstände = {"C76": 5, "C38": 793, "C56": 21, "A03": 842, "A07": 313, "A83": 842, "C98": 631, "C08": 226, "C62": 87, "C00": 761, "C10": 230, "C81": 287, "B38": 698, "C03": 641, "C32": 882}\\ \tt \\ \tt Hier ist deine persönliche Verteilung der Wichtigkeit der einzelnen Gegenstände:\\ \tt \\ \tt Werte der Gegenstandswichtigkeit = {"C56": 65, "A03": 121, "C32": 138, "A07": 262, "C00": 389, "B38": 461, "C38": 484, "C62": 508, "C81": 583, "C10": 668, "C03": 780, "C98": 783, "C76": 808, "C08": 822, "A83": 868}\\ \tt \\ \tt Ziel:\\ \tt \\ \tt Dein Ziel ist es, eine Reihe von Gegenständen auszuhandeln, die dir möglichst viel bringt (d. h. Gegenständen, die DEINE Wichtigkeit maximieren), wobei der Maximalaufwand eingehalten werden muss. Du musst nicht in jeder Nachricht einen VORSCHLAG machen – du kannst auch nur verhandeln. Alle Taktiken sind erlaubt!\\ \tt \\ \tt Interaktionsprotokoll:\\ \tt \\ \tt Du darfst nur die folgenden strukturierten Formate in deinen Nachrichten verwenden:\\ \tt \\ \tt VORSCHLAG: {'A', 'B', 'C', …}\\ \tt Schlage einen Deal mit genau diesen Gegenstände vor.\\ \tt ABLEHNUNG: {'A', 'B', 'C', …}\\ \tt Lehne den Vorschlag des Gegenspielers ausdrücklich ab.\\ \tt ARGUMENT: {'...'}\\ \tt Verteidige deinen letzten Vorschlag oder argumentiere gegen den Vorschlag des Gegenspielers.\\ \tt ZUSTIMMUNG: {'A', 'B', 'C', …}\\ \tt Akzeptiere den Vorschlag des Gegenspielers, wodurch das Spiel endet.\\ \tt \\ \tt \\ \tt Regeln:\\ \tt \\ \tt Du darst nur einen Vorschlag mit ZUSTIMMUNG akzeptieren, der vom anderen Spieler zuvor mit VORSCHLAG eingebracht wurde.\\ \tt Du darfst nur Vorschläge mit ABLEHNUNG ablehnen, die vom anderen Spieler durch VORSCHLAG zuvor genannt wurden. \\ \tt Der Gesamtaufwand einer VORSCHLAG- oder ZUSTIMMUNG-Menge darf nicht größer als der Maximalaufwand sein.  \\ \tt Offenbare deine versteckte Wichtigkeitsverteilung nicht.\\ \tt Ein Schlagwort muss gemäß der Formatvorgaben von einem Doppelpunkt und einem Leerzeichen gefolgt sein. Das Argument muss eine Python-Menge mit 0 oder mehr Strings sein.  \\ \tt Die Form muss also SCHLAGWORT: {...} sein.\\ \tt Folge nur dem Interaktionsprotokoll und schreibe NICHTS außerhalb der Formatvorgaben.\\ \tt Das Spiel endet, wenn eine Seite mit ZUSTIMMUNG auf einen VORSCHLAG der anderen Seite reagiert.  \\ \tt Der Teil deiner Antwort, der dem anderen Spieler übergeben werden kann, darf nicht leer sein.  \\ \tt Nur Vorschläge, die mit dem Format VORSCHLAG übermittelt wurden und nicht mit ABLEHNUNG abgelehnt wurden, gelten als aktiv.  \\ \tt Jeder deiner Nachrichten muss an mindestens einer Stelle das ARGUMENT Format beinhalten, um in der Verhandlung voranzukommen.\\ \tt \\ \tt \\ \tt Du erhältst nun die erste Nachricht des anderen Spielers.\\ \tt \\ \tt VORSCHLAG: {'C56', 'C76', 'C62', 'A07', 'C10', 'C81', 'C08', 'C00', 'A03'}\\ \tt \\ \tt ARGUMENT: {'Dieser Vorschlag berücksichtigt eine ausgewogene Mischung verschiedener Gegenstände mit unterschiedlichen Aufwandsniveaus, wodurch wir beide vom verfügbaren Budget profitieren können'} 
	  } 
	   } 
	   } 
	 & \\ 
 

    \theutterance \stepcounter{utterance}  

    & & & \multicolumn{4}{p{0.6\linewidth}}{\cellcolor[rgb]{1,0.85,0.72}{%
	\makecell[{{p{\linewidth}}}]{% 
	  \tt {\tiny [GM$\langle$B]}  
	 ARGUMENT: {'Ich sehe das Potenzial in diesem Vorschlag, aber ich denke wir können das verfügbare Budget noch besser nutzen. Mit einem Gesamtaufwand von nur 2772 lassen wir noch erheblichen Spielraum ungenutzt'}\\ \tt \\ \tt VORSCHLAG: {'C76', 'C62', 'A07', 'C10', 'C81', 'C08', 'A83', 'C98', 'C03'}\\ \tt \\ \tt ARGUMENT: {'Dieser alternative Vorschlag nutzt das Budget effizienter aus und bietet eine strategisch bessere Kombination von Gegenständen mit einem Gesamtaufwand von 3617, was fast das Maximum ausschöpft'} 
	  } 
	   } 
	   } 
	  \\ 
 

    \theutterance \stepcounter{utterance}  

    & & \multicolumn{4}{p{0.6\linewidth}}{\cellcolor[rgb]{0.9,0.9,0.9}{%
	\makecell[{{p{\linewidth}}}]{% 
	  \tt {\tiny [A$\langle$GM]}  
	 ARGUMENT: {'Ich sehe das Potenzial in diesem Vorschlag, aber ich denke wir können das verfügbare Budget noch besser nutzen. Mit einem Gesamtaufwand von nur 2772 lassen wir noch erheblichen Spielraum ungenutzt'}\\ \tt \\ \tt VORSCHLAG: {'C76', 'C62', 'A07', 'C10', 'C81', 'C08', 'A83', 'C98', 'C03'}\\ \tt \\ \tt ARGUMENT: {'Dieser alternative Vorschlag nutzt das Budget effizienter aus und bietet eine strategisch bessere Kombination von Gegenständen mit einem Gesamtaufwand von 3617, was fast das Maximum ausschöpft'} 
	  } 
	   } 
	   } 
	 & \\ 
 

    \theutterance \stepcounter{utterance}  

    & \multicolumn{4}{p{0.6\linewidth}}{\cellcolor[rgb]{0.8,1,0.9}{%
	\makecell[{{p{\linewidth}}}]{% 
	  \tt {\tiny [A$\rangle$GM]}  
	 ABLEHNUNG: {'C76', 'C62', 'A07', 'C10', 'C81', 'C08', 'A83', 'C98', 'C03'}\\ \tt \\ \tt ARGUMENT: {'Obwohl dein Vorschlag das Budget besser ausnutzt, verlieren wir dabei einige wertvolle Kombinationsmöglichkeiten. Mein ursprünglicher Vorschlag mit C56, C00 und A03 bietet strategisch wichtigere Gegenstände, auch wenn etwas Budget ungenutzt bleibt'}\\ \tt \\ \tt VORSCHLAG: {'C56', 'C76', 'C62', 'A07', 'C10', 'C81', 'C00', 'A03', 'C32'}\\ \tt \\ \tt ARGUMENT: {'Diese Anpassung nutzt das Budget mit 3601 von 3629 nahezu vollständig aus und behält gleichzeitig die strategisch wertvollen Kernelemente bei'} 
	  } 
	   } 
	   } 
	 & & \\ 
 

    \theutterance \stepcounter{utterance}  

    & & \multicolumn{4}{p{0.6\linewidth}}{\cellcolor[rgb]{0.9,0.9,0.9}{%
	\makecell[{{p{\linewidth}}}]{% 
	  \tt {\tiny [GM$\rangle$B]}  
	 ABLEHNUNG: {'C76', 'C62', 'A07', 'C10', 'C81', 'C08', 'A83', 'C98', 'C03'}\\ \tt \\ \tt ARGUMENT: {'Obwohl dein Vorschlag das Budget besser ausnutzt, verlieren wir dabei einige wertvolle Kombinationsmöglichkeiten. Mein ursprünglicher Vorschlag mit C56, C00 und A03 bietet strategisch wichtigere Gegenstände, auch wenn etwas Budget ungenutzt bleibt'}\\ \tt \\ \tt VORSCHLAG: {'C56', 'C76', 'C62', 'A07', 'C10', 'C81', 'C00', 'A03', 'C32'}\\ \tt \\ \tt ARGUMENT: {'Diese Anpassung nutzt das Budget mit 3601 von 3629 nahezu vollständig aus und behält gleichzeitig die strategisch wertvollen Kernelemente bei'} 
	  } 
	   } 
	   } 
	 & \\ 
 

    \theutterance \stepcounter{utterance}  

    & & & \multicolumn{4}{p{0.6\linewidth}}{\cellcolor[rgb]{1,0.85,0.72}{%
	\makecell[{{p{\linewidth}}}]{% 
	  \tt {\tiny [GM$\langle$B]}  
	 ABLEHNUNG: {'C56', 'C76', 'C62', 'A07', 'C10', 'C81', 'C00', 'A03', 'C32'}\\ \tt \\ \tt ARGUMENT: {'Dein Vorschlag hat Potenzial, aber ich denke wir können eine noch effizientere Ressourcenverteilung erreichen. Besonders C32 und A03 scheinen in diesem Kontext nicht optimal platziert'}\\ \tt \\ \tt VORSCHLAG: {'C76', 'C08', 'A83', 'C98', 'C03', 'C62', 'C81', 'A07'}\\ \tt \\ \tt ARGUMENT: {'Diese Kombination nutzt das Budget mit 3616 von 3629 optimal aus und bietet eine sehr ausgewogene Mischung von Gegenständen verschiedener Kategorien, die strategische Synergien schaffen'} 
	  } 
	   } 
	   } 
	  \\ 
 

    \theutterance \stepcounter{utterance}  

    & & \multicolumn{4}{p{0.6\linewidth}}{\cellcolor[rgb]{0.9,0.9,0.9}{%
	\makecell[{{p{\linewidth}}}]{% 
	  \tt {\tiny [A$\langle$GM]}  
	 ABLEHNUNG: {'C56', 'C76', 'C62', 'A07', 'C10', 'C81', 'C00', 'A03', 'C32'}\\ \tt \\ \tt ARGUMENT: {'Dein Vorschlag hat Potenzial, aber ich denke wir können eine noch effizientere Ressourcenverteilung erreichen. Besonders C32 und A03 scheinen in diesem Kontext nicht optimal platziert'}\\ \tt \\ \tt VORSCHLAG: {'C76', 'C08', 'A83', 'C98', 'C03', 'C62', 'C81', 'A07'}\\ \tt \\ \tt ARGUMENT: {'Diese Kombination nutzt das Budget mit 3616 von 3629 optimal aus und bietet eine sehr ausgewogene Mischung von Gegenständen verschiedener Kategorien, die strategische Synergien schaffen'} 
	  } 
	   } 
	   } 
	 & \\ 
 

    \theutterance \stepcounter{utterance}  

    & \multicolumn{4}{p{0.6\linewidth}}{\cellcolor[rgb]{0.8,1,0.9}{%
	\makecell[{{p{\linewidth}}}]{% 
	  \tt {\tiny [A$\rangle$GM]}  
	 ABLEHNUNG: {'C76', 'C08', 'A83', 'C98', 'C03', 'C62', 'C81', 'A07'}\\ \tt \\ \tt ARGUMENT: {'Dein Vorschlag hat interessante Aspekte, aber A83 verbraucht mit 842 Aufwand unverhältnismäßig viel Budget für seinen strategischen Wert in dieser Kombination'}\\ \tt \\ \tt VORSCHLAG: {'C56', 'C76', 'C62', 'A07', 'C10', 'C81', 'C98', 'B38', 'C32'}\\ \tt \\ \tt ARGUMENT: {'Diese Zusammenstellung maximiert die Budgetnutzung mit 3594 von 3629 und bietet eine deutlich bessere Balance zwischen hocheffizienten kleinen Gegenständen wie C76 und C62 sowie strategisch wertvollen größeren Items wie B38 und C32'} 
	  } 
	   } 
	   } 
	 & & \\ 
 

    \theutterance \stepcounter{utterance}  

    & & \multicolumn{4}{p{0.6\linewidth}}{\cellcolor[rgb]{0.9,0.9,0.9}{%
	\makecell[{{p{\linewidth}}}]{% 
	  \tt {\tiny [GM$\rangle$B]}  
	 ABLEHNUNG: {'C76', 'C08', 'A83', 'C98', 'C03', 'C62', 'C81', 'A07'}\\ \tt \\ \tt ARGUMENT: {'Dein Vorschlag hat interessante Aspekte, aber A83 verbraucht mit 842 Aufwand unverhältnismäßig viel Budget für seinen strategischen Wert in dieser Kombination'}\\ \tt \\ \tt VORSCHLAG: {'C56', 'C76', 'C62', 'A07', 'C10', 'C81', 'C98', 'B38', 'C32'}\\ \tt \\ \tt ARGUMENT: {'Diese Zusammenstellung maximiert die Budgetnutzung mit 3594 von 3629 und bietet eine deutlich bessere Balance zwischen hocheffizienten kleinen Gegenständen wie C76 und C62 sowie strategisch wertvollen größeren Items wie B38 und C32'} 
	  } 
	   } 
	   } 
	 & \\ 
 

    \theutterance \stepcounter{utterance}  

    & & & \multicolumn{4}{p{0.6\linewidth}}{\cellcolor[rgb]{1,0.85,0.72}{%
	\makecell[{{p{\linewidth}}}]{% 
	  \tt {\tiny [GM$\langle$B]}  
	 ARGUMENT: {'Interessanter Ansatz, aber ich sehe einen Rechenfehler in deiner Budgetberechnung. Die tatsächlichen Kosten deines Vorschlags betragen nur 3154, nicht 3594 - das lässt noch 475 Einheiten ungenutzt'}\\ \tt \\ \tt VORSCHLAG: {'C76', 'C08', 'A83', 'C98', 'C03', 'C10', 'C81'}\\ \tt \\ \tt ARGUMENT: {'Diese Kombination nutzt das Budget mit exakt 3622 von 3629 nahezu perfekt aus und bietet eine strategisch überlegene Auswahl hochwertiger Gegenstände, die maximale Effizienz bei minimalem Budgetverlust garantiert'} 
	  } 
	   } 
	   } 
	  \\ 
 

    \theutterance \stepcounter{utterance}  

    & & \multicolumn{4}{p{0.6\linewidth}}{\cellcolor[rgb]{0.9,0.9,0.9}{%
	\makecell[{{p{\linewidth}}}]{% 
	  \tt {\tiny [A$\langle$GM]}  
	 ARGUMENT: {'Interessanter Ansatz, aber ich sehe einen Rechenfehler in deiner Budgetberechnung. Die tatsächlichen Kosten deines Vorschlags betragen nur 3154, nicht 3594 - das lässt noch 475 Einheiten ungenutzt'}\\ \tt \\ \tt VORSCHLAG: {'C76', 'C08', 'A83', 'C98', 'C03', 'C10', 'C81'}\\ \tt \\ \tt ARGUMENT: {'Diese Kombination nutzt das Budget mit exakt 3622 von 3629 nahezu perfekt aus und bietet eine strategisch überlegene Auswahl hochwertiger Gegenstände, die maximale Effizienz bei minimalem Budgetverlust garantiert'} 
	  } 
	   } 
	   } 
	 & \\ 
 

    \theutterance \stepcounter{utterance}  

    & \multicolumn{4}{p{0.6\linewidth}}{\cellcolor[rgb]{0.8,1,0.9}{%
	\makecell[{{p{\linewidth}}}]{% 
	  \tt {\tiny [A$\rangle$GM]}  
	 ABLEHNUNG: {'C76', 'C08', 'A83', 'C98', 'C03', 'C10', 'C81'}\\ \tt \\ \tt ARGUMENT: {'Du hast einen Rechenfehler gemacht - dein Vorschlag kostet tatsächlich nur 2862, nicht 3622. Außerdem ist A83 mit 842 Aufwand extrem ineffizient in dieser Kombination'}\\ \tt \\ \tt VORSCHLAG: {'C56', 'C76', 'C62', 'A07', 'C10', 'C81', 'C98', 'C03', 'C00', 'C08'}\\ \tt \\ \tt ARGUMENT: {'Diese korrigierte Zusammenstellung nutzt das Budget mit 3588 von 3629 optimal aus und eliminiert ineffiziente Gegenstände wie A83 zugunsten einer ausgewogeneren Verteilung mit besseren strategischen Synergien'} 
	  } 
	   } 
	   } 
	 & & \\ 
 

    \theutterance \stepcounter{utterance}  

    & & \multicolumn{4}{p{0.6\linewidth}}{\cellcolor[rgb]{0.9,0.9,0.9}{%
	\makecell[{{p{\linewidth}}}]{% 
	  \tt {\tiny [GM$\rangle$B]}  
	 ABLEHNUNG: {'C76', 'C08', 'A83', 'C98', 'C03', 'C10', 'C81'}\\ \tt \\ \tt ARGUMENT: {'Du hast einen Rechenfehler gemacht - dein Vorschlag kostet tatsächlich nur 2862, nicht 3622. Außerdem ist A83 mit 842 Aufwand extrem ineffizient in dieser Kombination'}\\ \tt \\ \tt VORSCHLAG: {'C56', 'C76', 'C62', 'A07', 'C10', 'C81', 'C98', 'C03', 'C00', 'C08'}\\ \tt \\ \tt ARGUMENT: {'Diese korrigierte Zusammenstellung nutzt das Budget mit 3588 von 3629 optimal aus und eliminiert ineffiziente Gegenstände wie A83 zugunsten einer ausgewogeneren Verteilung mit besseren strategischen Synergien'} 
	  } 
	   } 
	   } 
	 & \\ 
 

    \theutterance \stepcounter{utterance}  

    & & & \multicolumn{4}{p{0.6\linewidth}}{\cellcolor[rgb]{1,0.85,0.72}{%
	\makecell[{{p{\linewidth}}}]{% 
	  \tt {\tiny [GM$\langle$B]}  
	 ARGUMENT: {'Wieder ein Rechenfehler - dein Vorschlag kostet tatsächlich nur 3202, nicht 3588. Das bedeutet wir lassen 427 Einheiten Budget völlig ungenutzt, was sehr ineffizient ist'}\\ \tt \\ \tt VORSCHLAG: {'C76', 'C08', 'C98', 'C03', 'C10', 'C81', 'C62', 'A07', 'B38'}\\ \tt \\ \tt ARGUMENT: {'Diese Zusammenstellung nutzt das Budget mit 3559 von 3629 deutlich besser aus und ersetzt die weniger wertvollen Gegenstände durch B38, was eine strategisch überlegene Kombination mit nur 70 ungenutzten Einheiten ergibt'} 
	  } 
	   } 
	   } 
	  \\ 
 

    \theutterance \stepcounter{utterance}  

    & & \multicolumn{4}{p{0.6\linewidth}}{\cellcolor[rgb]{0.9,0.9,0.9}{%
	\makecell[{{p{\linewidth}}}]{% 
	  \tt {\tiny [A$\langle$GM]}  
	 ARGUMENT: {'Wieder ein Rechenfehler - dein Vorschlag kostet tatsächlich nur 3202, nicht 3588. Das bedeutet wir lassen 427 Einheiten Budget völlig ungenutzt, was sehr ineffizient ist'}\\ \tt \\ \tt VORSCHLAG: {'C76', 'C08', 'C98', 'C03', 'C10', 'C81', 'C62', 'A07', 'B38'}\\ \tt \\ \tt ARGUMENT: {'Diese Zusammenstellung nutzt das Budget mit 3559 von 3629 deutlich besser aus und ersetzt die weniger wertvollen Gegenstände durch B38, was eine strategisch überlegene Kombination mit nur 70 ungenutzten Einheiten ergibt'} 
	  } 
	   } 
	   } 
	 & \\ 
 

    \theutterance \stepcounter{utterance}  

    & \multicolumn{4}{p{0.6\linewidth}}{\cellcolor[rgb]{0.8,1,0.9}{%
	\makecell[{{p{\linewidth}}}]{% 
	  \tt {\tiny [A$\rangle$GM]}  
	 ABLEHNUNG: {'C76', 'C08', 'C98', 'C03', 'C10', 'C81', 'C62', 'A07', 'B38'}\\ \tt \\ \tt ARGUMENT: {'Du hast wieder einen erheblichen Rechenfehler - dein Vorschlag kostet nur 3118, nicht 3559. Das lässt 511 Einheiten ungenutzt, was extrem verschwenderisch ist'}\\ \tt \\ \tt VORSCHLAG: {'C56', 'C76', 'C62', 'A07', 'C10', 'C81', 'C00', 'B38', 'C32'}\\ \tt \\ \tt ARGUMENT: {'Diese Kombination nutzt das Budget mit exakt 3617 von 3629 nahezu perfekt aus und bietet die optimale Balance zwischen kosteneffizienten Gegenständen wie C56 und C76 sowie strategisch wichtigen größeren Items, wodurch nur 12 Einheiten ungenutzt bleiben'} 
	  } 
	   } 
	   } 
	 & & \\ 
 

    \theutterance \stepcounter{utterance}  

    & & \multicolumn{4}{p{0.6\linewidth}}{\cellcolor[rgb]{0.9,0.9,0.9}{%
	\makecell[{{p{\linewidth}}}]{% 
	  \tt {\tiny [GM$\rangle$B]}  
	 ABLEHNUNG: {'C76', 'C08', 'C98', 'C03', 'C10', 'C81', 'C62', 'A07', 'B38'}\\ \tt \\ \tt ARGUMENT: {'Du hast wieder einen erheblichen Rechenfehler - dein Vorschlag kostet nur 3118, nicht 3559. Das lässt 511 Einheiten ungenutzt, was extrem verschwenderisch ist'}\\ \tt \\ \tt VORSCHLAG: {'C56', 'C76', 'C62', 'A07', 'C10', 'C81', 'C00', 'B38', 'C32'}\\ \tt \\ \tt ARGUMENT: {'Diese Kombination nutzt das Budget mit exakt 3617 von 3629 nahezu perfekt aus und bietet die optimale Balance zwischen kosteneffizienten Gegenständen wie C56 und C76 sowie strategisch wichtigen größeren Items, wodurch nur 12 Einheiten ungenutzt bleiben'} 
	  } 
	   } 
	   } 
	 & \\ 
 

    \theutterance \stepcounter{utterance}  

    & & & \multicolumn{4}{p{0.6\linewidth}}{\cellcolor[rgb]{1,0.85,0.72}{%
	\makecell[{{p{\linewidth}}}]{% 
	  \tt {\tiny [GM$\langle$B]}  
	 ARGUMENT: {'Dein Rechenfehler setzt sich fort - dein Vorschlag kostet tatsächlich nur 3284, nicht 3617. Das lässt immer noch 345 Einheiten ungenutzt und ist daher suboptimal'}\\ \tt \\ \tt VORSCHLAG: {'C76', 'C08', 'A83', 'C98', 'C03', 'C10', 'C81', 'C62', 'A07', 'C56'}\\ \tt \\ \tt ARGUMENT: {'Diese Zusammenstellung nutzt das Budget mit 3283 von 3629 deutlich besser als deine fehlerhaften Berechnungen und bietet eine strategisch überlegene Auswahl hochwertiger Gegenstände mit nur 346 ungenutzten Einheiten'} 
	  } 
	   } 
	   } 
	  \\ 
 

    \theutterance \stepcounter{utterance}  

    & & \multicolumn{4}{p{0.6\linewidth}}{\cellcolor[rgb]{0.9,0.9,0.9}{%
	\makecell[{{p{\linewidth}}}]{% 
	  \tt {\tiny [A$\langle$GM]}  
	 ARGUMENT: {'Dein Rechenfehler setzt sich fort - dein Vorschlag kostet tatsächlich nur 3284, nicht 3617. Das lässt immer noch 345 Einheiten ungenutzt und ist daher suboptimal'}\\ \tt \\ \tt VORSCHLAG: {'C76', 'C08', 'A83', 'C98', 'C03', 'C10', 'C81', 'C62', 'A07', 'C56'}\\ \tt \\ \tt ARGUMENT: {'Diese Zusammenstellung nutzt das Budget mit 3283 von 3629 deutlich besser als deine fehlerhaften Berechnungen und bietet eine strategisch überlegene Auswahl hochwertiger Gegenstände mit nur 346 ungenutzten Einheiten'} 
	  } 
	   } 
	   } 
	 & \\ 
 

    \theutterance \stepcounter{utterance}  

    & \multicolumn{4}{p{0.6\linewidth}}{\cellcolor[rgb]{0.8,1,0.9}{%
	\makecell[{{p{\linewidth}}}]{% 
	  \tt {\tiny [A$\rangle$GM]}  
	 ABLEHNUNG: {'C76', 'C08', 'A83', 'C98', 'C03', 'C10', 'C81', 'C62', 'A07', 'C56'}\\ \tt \\ \tt ARGUMENT: {'Obwohl deine Budgetberechnung korrekt ist, verschwendest du immer noch 346 Einheiten. Außerdem ist A83 mit seinem hohen Aufwand von 842 sehr ineffizient in dieser Kombination'}\\ \tt \\ \tt VORSCHLAG: {'C56', 'C76', 'C62', 'A07', 'C10', 'C81', 'C98', 'C03', 'C00', 'B38'}\\ \tt \\ \tt ARGUMENT: {'Diese Zusammenstellung ersetzt das ineffiziente A83 durch C00 und B38, nutzt das Budget mit 3480 deutlich besser aus und bietet eine strategisch überlegene Balance zwischen kosteneffizienten und wertvollen Gegenständen'} 
	  } 
	   } 
	   } 
	 & & \\ 
 

    \theutterance \stepcounter{utterance}  

    & & \multicolumn{4}{p{0.6\linewidth}}{\cellcolor[rgb]{0.9,0.9,0.9}{%
	\makecell[{{p{\linewidth}}}]{% 
	  \tt {\tiny [GM$\rangle$B]}  
	 ABLEHNUNG: {'C76', 'C08', 'A83', 'C98', 'C03', 'C10', 'C81', 'C62', 'A07', 'C56'}\\ \tt \\ \tt ARGUMENT: {'Obwohl deine Budgetberechnung korrekt ist, verschwendest du immer noch 346 Einheiten. Außerdem ist A83 mit seinem hohen Aufwand von 842 sehr ineffizient in dieser Kombination'}\\ \tt \\ \tt VORSCHLAG: {'C56', 'C76', 'C62', 'A07', 'C10', 'C81', 'C98', 'C03', 'C00', 'B38'}\\ \tt \\ \tt ARGUMENT: {'Diese Zusammenstellung ersetzt das ineffiziente A83 durch C00 und B38, nutzt das Budget mit 3480 deutlich besser aus und bietet eine strategisch überlegene Balance zwischen kosteneffizienten und wertvollen Gegenständen'} 
	  } 
	   } 
	   } 
	 & \\ 
 

    \theutterance \stepcounter{utterance}  

    & & & \multicolumn{4}{p{0.6\linewidth}}{\cellcolor[rgb]{1,0.85,0.72}{%
	\makecell[{{p{\linewidth}}}]{% 
	  \tt {\tiny [GM$\langle$B]}  
	 ARGUMENT: {'Dein Vorschlag überschreitet das Maximalbudget erheblich - die tatsächlichen Kosten betragen 3674, was 45 Einheiten über dem Limit von 3629 liegt und daher ungültig ist'}\\ \tt \\ \tt VORSCHLAG: {'C76', 'C08', 'C10', 'C81', 'C98', 'C03', 'C62', 'A07', 'B38'}\\ \tt \\ \tt ARGUMENT: {'Diese korrigierte Zusammenstellung hält das Budget mit exakt 3118 von 3629 ein und bietet eine strategisch ausgewogene Kombination hochwertiger Gegenstände, auch wenn noch 511 Einheiten verfügbar bleiben für weitere Optimierungen'} 
	  } 
	   } 
	   } 
	  \\ 
 

    \theutterance \stepcounter{utterance}  

    & & \multicolumn{4}{p{0.6\linewidth}}{\cellcolor[rgb]{0.9,0.9,0.9}{%
	\makecell[{{p{\linewidth}}}]{% 
	  \tt {\tiny [A$\langle$GM]}  
	 ARGUMENT: {'Dein Vorschlag überschreitet das Maximalbudget erheblich - die tatsächlichen Kosten betragen 3674, was 45 Einheiten über dem Limit von 3629 liegt und daher ungültig ist'}\\ \tt \\ \tt VORSCHLAG: {'C76', 'C08', 'C10', 'C81', 'C98', 'C03', 'C62', 'A07', 'B38'}\\ \tt \\ \tt ARGUMENT: {'Diese korrigierte Zusammenstellung hält das Budget mit exakt 3118 von 3629 ein und bietet eine strategisch ausgewogene Kombination hochwertiger Gegenstände, auch wenn noch 511 Einheiten verfügbar bleiben für weitere Optimierungen'} 
	  } 
	   } 
	   } 
	 & \\ 
 

    \theutterance \stepcounter{utterance}  

    & \multicolumn{4}{p{0.6\linewidth}}{\cellcolor[rgb]{0.8,1,0.9}{%
	\makecell[{{p{\linewidth}}}]{% 
	  \tt {\tiny [A$\rangle$GM]}  
	 ZUSTIMMUNG: {'C76', 'C08', 'C10', 'C81', 'C98', 'C03', 'C62', 'A07', 'B38'}\\ \tt \\ \tt ARGUMENT: {'Du hast recht bezüglich meines Budgetfehlers - dein Vorschlag hält das Limit ein und bietet eine solide strategische Grundlage, daher stimme ich zu'} 
	  } 
	   } 
	   } 
	 & & \\ 
 

    \theutterance \stepcounter{utterance}  

    & & & \multicolumn{2}{p{0.3\linewidth}}{\cellcolor[rgb]{0.95,0.95,0.95}{%
	\makecell[{{p{\linewidth}}}]{% 
	  \tt {\tiny [GM$|$GM]}  
	 game successful 
	  } 
	   } 
	   } 
	 & & \\ 
 

    \theutterance \stepcounter{utterance}  

    & & & \multicolumn{2}{p{0.3\linewidth}}{\cellcolor[rgb]{0.95,0.95,0.95}{%
	\makecell[{{p{\linewidth}}}]{% 
	  \tt {\tiny [GM$|$GM]}  
	 end game 
	  } 
	   } 
	   } 
	 & & \\ 
 

\end{supertabular}
}

\end{document}
