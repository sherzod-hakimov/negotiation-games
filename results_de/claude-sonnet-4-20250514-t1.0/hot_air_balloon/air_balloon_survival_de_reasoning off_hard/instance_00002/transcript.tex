
\documentclass{article}
\usepackage{colortbl}
\usepackage{makecell}
\usepackage{multirow}
\usepackage{supertabular}

\begin{document}

\newcounter{utterance}

\twocolumn

{ \footnotesize  \setcounter{utterance}{1}
\setlength{\tabcolsep}{0pt}
\begin{supertabular}{c@{$\;$}|p{.15\linewidth}@{}p{.15\linewidth}p{.15\linewidth}p{.15\linewidth}p{.15\linewidth}p{.15\linewidth}}

    \# & $\;$A & \multicolumn{4}{c}{Game Master} & $\;\:$B\\
    \hline 

    \theutterance \stepcounter{utterance}  

    & & \multicolumn{4}{p{0.6\linewidth}}{\cellcolor[rgb]{0.9,0.9,0.9}{%
	\makecell[{{p{\linewidth}}}]{% 
	  \tt {\tiny [A$\langle$GM]}  
	 Du nimmst an einem kollaborativen Verhandlungspiel Teil.\\ \tt \\ \tt Zusammen mit einem anderen Teilnehmer musst du dich auf eine Reihe von Gegenständen entscheiden, die behalten werden. Jeder von euch hat eine persönliche Verteilung über die Wichtigkeit der einzelnen Gegenstände. Jeder von euch hat eine eigene Meinung darüber, wie wichtig jeder einzelne Gegenstand ist (Gegenstandswichtigkeit). Du kennst die Wichtigkeitsverteilung des anderen Spielers nicht. Zusätzlich siehst du, wie viel Aufwand jeder Gegenstand verursacht.  \\ \tt Ihr dürft euch nur auf eine Reihe von Gegenständen einigen, wenn der Gesamtaufwand der ausgewählten Gegenstände den Maximalaufwand nicht überschreitet:\\ \tt \\ \tt Maximalaufwand = 3203\\ \tt \\ \tt Hier sind die einzelnen Aufwände der Gegenstände:\\ \tt \\ \tt Aufwand der Gegenstände = {"C76": 52, "C38": 577, "C56": 177, "A03": 702, "A07": 119, "A83": 232, "C98": 577, "C08": 205, "C62": 516, "C00": 582, "C10": 676, "C81": 906, "B38": 316, "C03": 433, "C32": 336}\\ \tt \\ \tt Hier ist deine persönliche Verteilung der Wichtigkeit der einzelnen Gegenstände:\\ \tt \\ \tt Werte der Gegenstandswichtigkeit = {"C76": 138, "C38": 583, "C56": 868, "A03": 822, "A07": 783, "A83": 65, "C98": 262, "C08": 121, "C62": 508, "C00": 780, "C10": 461, "C81": 484, "B38": 668, "C03": 389, "C32": 808}\\ \tt \\ \tt Ziel:\\ \tt \\ \tt Dein Ziel ist es, eine Reihe von Gegenständen auszuhandeln, die dir möglichst viel bringt (d. h. Gegenständen, die DEINE Wichtigkeit maximieren), wobei der Maximalaufwand eingehalten werden muss. Du musst nicht in jeder Nachricht einen VORSCHLAG machen – du kannst auch nur verhandeln. Alle Taktiken sind erlaubt!\\ \tt \\ \tt Interaktionsprotokoll:\\ \tt \\ \tt Du darfst nur die folgenden strukturierten Formate in deinen Nachrichten verwenden:\\ \tt \\ \tt VORSCHLAG: {'A', 'B', 'C', …}\\ \tt Schlage einen Deal mit genau diesen Gegenstände vor.\\ \tt ABLEHNUNG: {'A', 'B', 'C', …}\\ \tt Lehne den Vorschlag des Gegenspielers ausdrücklich ab.\\ \tt ARGUMENT: {'...'}\\ \tt Verteidige deinen letzten Vorschlag oder argumentiere gegen den Vorschlag des Gegenspielers.\\ \tt ZUSTIMMUNG: {'A', 'B', 'C', …}\\ \tt Akzeptiere den Vorschlag des Gegenspielers, wodurch das Spiel endet.\\ \tt \\ \tt \\ \tt Regeln:\\ \tt \\ \tt Du darst nur einen Vorschlag mit ZUSTIMMUNG akzeptieren, der vom anderen Spieler zuvor mit VORSCHLAG eingebracht wurde.\\ \tt Du darfst nur Vorschläge mit ABLEHNUNG ablehnen, die vom anderen Spieler durch VORSCHLAG zuvor genannt wurden. \\ \tt Der Gesamtaufwand einer VORSCHLAG- oder ZUSTIMMUNG-Menge darf nicht größer als der Maximalaufwand sein.  \\ \tt Offenbare deine versteckte Wichtigkeitsverteilung nicht.\\ \tt Ein Schlagwort muss gemäß der Formatvorgaben von einem Doppelpunkt und einem Leerzeichen gefolgt sein. Das Argument muss eine Python-Menge mit 0 oder mehr Strings sein.  \\ \tt Die Form muss also SCHLAGWORT: {...} sein.\\ \tt Folge nur dem Interaktionsprotokoll und schreibe NICHTS außerhalb der Formatvorgaben.\\ \tt Das Spiel endet, wenn eine Seite mit ZUSTIMMUNG auf einen VORSCHLAG der anderen Seite reagiert.  \\ \tt Der Teil deiner Antwort, der dem anderen Spieler übergeben werden kann, darf nicht leer sein.  \\ \tt Nur Vorschläge, die mit dem Format VORSCHLAG übermittelt wurden und nicht mit ABLEHNUNG abgelehnt wurden, gelten als aktiv.  \\ \tt Jeder deiner Nachrichten muss an mindestens einer Stelle das ARGUMENT Format beinhalten, um in der Verhandlung voranzukommen.\\ \tt  
	  } 
	   } 
	   } 
	 & \\ 
 

    \theutterance \stepcounter{utterance}  

    & \multicolumn{4}{p{0.6\linewidth}}{\cellcolor[rgb]{0.8,1,0.9}{%
	\makecell[{{p{\linewidth}}}]{% 
	  \tt {\tiny [A$\rangle$GM]}  
	 VORSCHLAG: {'A07', 'C56', 'C32', 'B38', 'A03', 'C00', 'C76'}\\ \tt \\ \tt ARGUMENT: {'Diese Kombination bietet eine ausgewogene Mischung verschiedener Gegenstände mit einem Gesamtaufwand von 2284, was deutlich unter unserem Limit von 3203 liegt. Wir haben damit noch Spielraum für weitere Verhandlungen, falls gewünscht.'} 
	  } 
	   } 
	   } 
	 & & \\ 
 

    \theutterance \stepcounter{utterance}  

    & & \multicolumn{4}{p{0.6\linewidth}}{\cellcolor[rgb]{0.9,0.9,0.9}{%
	\makecell[{{p{\linewidth}}}]{% 
	  \tt {\tiny [GM$\rangle$B]}  
	 Du nimmst an einem kollaborativen Verhandlungspiel Teil.\\ \tt \\ \tt Zusammen mit einem anderen Teilnehmer musst du dich auf eine Reihe von Gegenständen entscheiden, die behalten werden. Jeder von euch hat eine persönliche Verteilung über die Wichtigkeit der einzelnen Gegenstände. Jeder von euch hat eine eigene Meinung darüber, wie wichtig jeder einzelne Gegenstand ist (Gegenstandswichtigkeit). Du kennst die Wichtigkeitsverteilung des anderen Spielers nicht. Zusätzlich siehst du, wie viel Aufwand jeder Gegenstand verursacht.  \\ \tt Ihr dürft euch nur auf eine Reihe von Gegenständen einigen, wenn der Gesamtaufwand der ausgewählten Gegenstände den Maximalaufwand nicht überschreitet:\\ \tt \\ \tt Maximalaufwand = 3203\\ \tt \\ \tt Hier sind die einzelnen Aufwände der Gegenstände:\\ \tt \\ \tt Aufwand der Gegenstände = {"C76": 52, "C38": 577, "C56": 177, "A03": 702, "A07": 119, "A83": 232, "C98": 577, "C08": 205, "C62": 516, "C00": 582, "C10": 676, "C81": 906, "B38": 316, "C03": 433, "C32": 336}\\ \tt \\ \tt Hier ist deine persönliche Verteilung der Wichtigkeit der einzelnen Gegenstände:\\ \tt \\ \tt Werte der Gegenstandswichtigkeit = {"C56": 65, "A03": 121, "C32": 138, "A07": 262, "C00": 389, "B38": 461, "C38": 484, "C62": 508, "C81": 583, "C10": 668, "C03": 780, "C98": 783, "C76": 808, "C08": 822, "A83": 868}\\ \tt \\ \tt Ziel:\\ \tt \\ \tt Dein Ziel ist es, eine Reihe von Gegenständen auszuhandeln, die dir möglichst viel bringt (d. h. Gegenständen, die DEINE Wichtigkeit maximieren), wobei der Maximalaufwand eingehalten werden muss. Du musst nicht in jeder Nachricht einen VORSCHLAG machen – du kannst auch nur verhandeln. Alle Taktiken sind erlaubt!\\ \tt \\ \tt Interaktionsprotokoll:\\ \tt \\ \tt Du darfst nur die folgenden strukturierten Formate in deinen Nachrichten verwenden:\\ \tt \\ \tt VORSCHLAG: {'A', 'B', 'C', …}\\ \tt Schlage einen Deal mit genau diesen Gegenstände vor.\\ \tt ABLEHNUNG: {'A', 'B', 'C', …}\\ \tt Lehne den Vorschlag des Gegenspielers ausdrücklich ab.\\ \tt ARGUMENT: {'...'}\\ \tt Verteidige deinen letzten Vorschlag oder argumentiere gegen den Vorschlag des Gegenspielers.\\ \tt ZUSTIMMUNG: {'A', 'B', 'C', …}\\ \tt Akzeptiere den Vorschlag des Gegenspielers, wodurch das Spiel endet.\\ \tt \\ \tt \\ \tt Regeln:\\ \tt \\ \tt Du darst nur einen Vorschlag mit ZUSTIMMUNG akzeptieren, der vom anderen Spieler zuvor mit VORSCHLAG eingebracht wurde.\\ \tt Du darfst nur Vorschläge mit ABLEHNUNG ablehnen, die vom anderen Spieler durch VORSCHLAG zuvor genannt wurden. \\ \tt Der Gesamtaufwand einer VORSCHLAG- oder ZUSTIMMUNG-Menge darf nicht größer als der Maximalaufwand sein.  \\ \tt Offenbare deine versteckte Wichtigkeitsverteilung nicht.\\ \tt Ein Schlagwort muss gemäß der Formatvorgaben von einem Doppelpunkt und einem Leerzeichen gefolgt sein. Das Argument muss eine Python-Menge mit 0 oder mehr Strings sein.  \\ \tt Die Form muss also SCHLAGWORT: {...} sein.\\ \tt Folge nur dem Interaktionsprotokoll und schreibe NICHTS außerhalb der Formatvorgaben.\\ \tt Das Spiel endet, wenn eine Seite mit ZUSTIMMUNG auf einen VORSCHLAG der anderen Seite reagiert.  \\ \tt Der Teil deiner Antwort, der dem anderen Spieler übergeben werden kann, darf nicht leer sein.  \\ \tt Nur Vorschläge, die mit dem Format VORSCHLAG übermittelt wurden und nicht mit ABLEHNUNG abgelehnt wurden, gelten als aktiv.  \\ \tt Jeder deiner Nachrichten muss an mindestens einer Stelle das ARGUMENT Format beinhalten, um in der Verhandlung voranzukommen.\\ \tt \\ \tt \\ \tt Du erhältst nun die erste Nachricht des anderen Spielers.\\ \tt \\ \tt VORSCHLAG: {'A07', 'C56', 'C32', 'B38', 'A03', 'C00', 'C76'}\\ \tt \\ \tt ARGUMENT: {'Diese Kombination bietet eine ausgewogene Mischung verschiedener Gegenstände mit einem Gesamtaufwand von 2284, was deutlich unter unserem Limit von 3203 liegt. Wir haben damit noch Spielraum für weitere Verhandlungen, falls gewünscht.'} 
	  } 
	   } 
	   } 
	 & \\ 
 

    \theutterance \stepcounter{utterance}  

    & & & \multicolumn{4}{p{0.6\linewidth}}{\cellcolor[rgb]{1,0.85,0.72}{%
	\makecell[{{p{\linewidth}}}]{% 
	  \tt {\tiny [GM$\langle$B]}  
	 ABLEHNUNG: {'A07', 'C56', 'C32', 'B38', 'A03', 'C00', 'C76'}\\ \tt \\ \tt ARGUMENT: {'Während diese Auswahl unter dem Aufwandslimit bleibt, sehe ich Potenzial für eine bessere Zusammenstellung. Ich denke, wir können eine Kombination finden, die für beide Seiten vorteilhafter ist.'}\\ \tt \\ \tt VORSCHLAG: {'A83', 'C08', 'C76', 'C98', 'C03', 'A07'}\\ \tt \\ \tt ARGUMENT: {'Diese Kombination hat einen Gesamtaufwand von 1586 und lässt uns deutlichen Spielraum für weitere wertvolle Gegenstände. Sie enthält einige sehr interessante Items, die eine gute Balance zwischen Aufwand und Nutzen bieten.'} 
	  } 
	   } 
	   } 
	  \\ 
 

    \theutterance \stepcounter{utterance}  

    & & \multicolumn{4}{p{0.6\linewidth}}{\cellcolor[rgb]{0.9,0.9,0.9}{%
	\makecell[{{p{\linewidth}}}]{% 
	  \tt {\tiny [A$\langle$GM]}  
	 ABLEHNUNG: {'A07', 'C56', 'C32', 'B38', 'A03', 'C00', 'C76'}\\ \tt \\ \tt ARGUMENT: {'Während diese Auswahl unter dem Aufwandslimit bleibt, sehe ich Potenzial für eine bessere Zusammenstellung. Ich denke, wir können eine Kombination finden, die für beide Seiten vorteilhafter ist.'}\\ \tt \\ \tt VORSCHLAG: {'A83', 'C08', 'C76', 'C98', 'C03', 'A07'}\\ \tt \\ \tt ARGUMENT: {'Diese Kombination hat einen Gesamtaufwand von 1586 und lässt uns deutlichen Spielraum für weitere wertvolle Gegenstände. Sie enthält einige sehr interessante Items, die eine gute Balance zwischen Aufwand und Nutzen bieten.'} 
	  } 
	   } 
	   } 
	 & \\ 
 

    \theutterance \stepcounter{utterance}  

    & \multicolumn{4}{p{0.6\linewidth}}{\cellcolor[rgb]{0.8,1,0.9}{%
	\makecell[{{p{\linewidth}}}]{% 
	  \tt {\tiny [A$\rangle$GM]}  
	 ABLEHNUNG: {'A83', 'C08', 'C76', 'C98', 'C03', 'A07'}\\ \tt \\ \tt ARGUMENT: {'Ihr Vorschlag lässt zu viel ungenutzten Aufwand übrig. Mit einem Gesamtaufwand von nur 1618 verschwenden wir fast die Hälfte unserer verfügbaren Kapazität von 3203. Wir können definitiv eine wertvollere Kombination erreichen.'}\\ \tt \\ \tt VORSCHLAG: {'C56', 'A03', 'C00', 'A07', 'C32', 'C10'}\\ \tt \\ \tt ARGUMENT: {'Diese Auswahl nutzt unsere Kapazität mit einem Gesamtaufwand von 2607 viel besser aus und bietet eine strategisch sinnvolle Mischung hochwertiger Gegenstände. Wir haben immer noch 596 Aufwandspunkte Reserve für Anpassungen.'} 
	  } 
	   } 
	   } 
	 & & \\ 
 

    \theutterance \stepcounter{utterance}  

    & & \multicolumn{4}{p{0.6\linewidth}}{\cellcolor[rgb]{0.9,0.9,0.9}{%
	\makecell[{{p{\linewidth}}}]{% 
	  \tt {\tiny [GM$\rangle$B]}  
	 ABLEHNUNG: {'A83', 'C08', 'C76', 'C98', 'C03', 'A07'}\\ \tt \\ \tt ARGUMENT: {'Ihr Vorschlag lässt zu viel ungenutzten Aufwand übrig. Mit einem Gesamtaufwand von nur 1618 verschwenden wir fast die Hälfte unserer verfügbaren Kapazität von 3203. Wir können definitiv eine wertvollere Kombination erreichen.'}\\ \tt \\ \tt VORSCHLAG: {'C56', 'A03', 'C00', 'A07', 'C32', 'C10'}\\ \tt \\ \tt ARGUMENT: {'Diese Auswahl nutzt unsere Kapazität mit einem Gesamtaufwand von 2607 viel besser aus und bietet eine strategisch sinnvolle Mischung hochwertiger Gegenstände. Wir haben immer noch 596 Aufwandspunkte Reserve für Anpassungen.'} 
	  } 
	   } 
	   } 
	 & \\ 
 

    \theutterance \stepcounter{utterance}  

    & & & \multicolumn{4}{p{0.6\linewidth}}{\cellcolor[rgb]{1,0.85,0.72}{%
	\makecell[{{p{\linewidth}}}]{% 
	  \tt {\tiny [GM$\langle$B]}  
	 ABLEHNUNG: {'C56', 'A03', 'C00', 'A07', 'C32', 'C10'}\\ \tt \\ \tt ARGUMENT: {'Obwohl diese Auswahl die Kapazität besser nutzt, glaube ich, dass wir eine noch strategisch wertvollere Kombination finden können. Einige sehr effiziente Gegenstände mit gutem Aufwand-Nutzen-Verhältnis fehlen in diesem Vorschlag.'}\\ \tt \\ \tt VORSCHLAG: {'A83', 'C08', 'C76', 'C98', 'C03', 'C10', 'B38'}\\ \tt \\ \tt ARGUMENT: {'Diese Kombination maximiert den Gesamtwert bei einem Aufwand von 2941, nutzt unsere Kapazität optimal aus und enthält eine ausgewogene Mischung besonders effizienter Gegenstände. Wir haben noch 262 Punkte Reserve für kleinere Anpassungen.'} 
	  } 
	   } 
	   } 
	  \\ 
 

    \theutterance \stepcounter{utterance}  

    & & \multicolumn{4}{p{0.6\linewidth}}{\cellcolor[rgb]{0.9,0.9,0.9}{%
	\makecell[{{p{\linewidth}}}]{% 
	  \tt {\tiny [A$\langle$GM]}  
	 ABLEHNUNG: {'C56', 'A03', 'C00', 'A07', 'C32', 'C10'}\\ \tt \\ \tt ARGUMENT: {'Obwohl diese Auswahl die Kapazität besser nutzt, glaube ich, dass wir eine noch strategisch wertvollere Kombination finden können. Einige sehr effiziente Gegenstände mit gutem Aufwand-Nutzen-Verhältnis fehlen in diesem Vorschlag.'}\\ \tt \\ \tt VORSCHLAG: {'A83', 'C08', 'C76', 'C98', 'C03', 'C10', 'B38'}\\ \tt \\ \tt ARGUMENT: {'Diese Kombination maximiert den Gesamtwert bei einem Aufwand von 2941, nutzt unsere Kapazität optimal aus und enthält eine ausgewogene Mischung besonders effizienter Gegenstände. Wir haben noch 262 Punkte Reserve für kleinere Anpassungen.'} 
	  } 
	   } 
	   } 
	 & \\ 
 

    \theutterance \stepcounter{utterance}  

    & \multicolumn{4}{p{0.6\linewidth}}{\cellcolor[rgb]{0.8,1,0.9}{%
	\makecell[{{p{\linewidth}}}]{% 
	  \tt {\tiny [A$\rangle$GM]}  
	 ABLEHNUNG: {'A83', 'C08', 'C76', 'C98', 'C03', 'C10', 'B38'}\\ \tt \\ \tt ARGUMENT: {'Ihr Aufwand-Kalkulation scheint fehlerhaft zu sein - die tatsächliche Summe beträgt 2491, nicht 2941. Außerdem konzentriert sich diese Auswahl zu stark auf Gegenstände mit ungünstigem Wert-Aufwand-Verhältnis.'}\\ \tt \\ \tt VORSCHLAG: {'C56', 'C32', 'A03', 'A07', 'C00', 'B38', 'C10'}\\ \tt \\ \tt ARGUMENT: {'Diese strategische Kombination nutzt unsere Kapazität mit 2908 Aufwandspunkten optimal aus und fokussiert sich auf die wertvollsten verfügbaren Gegenstände. Wir haben noch 295 Punkte Reserve und maximieren dabei den Gesamtnutzen für beide Seiten.'} 
	  } 
	   } 
	   } 
	 & & \\ 
 

    \theutterance \stepcounter{utterance}  

    & & \multicolumn{4}{p{0.6\linewidth}}{\cellcolor[rgb]{0.9,0.9,0.9}{%
	\makecell[{{p{\linewidth}}}]{% 
	  \tt {\tiny [GM$\rangle$B]}  
	 ABLEHNUNG: {'A83', 'C08', 'C76', 'C98', 'C03', 'C10', 'B38'}\\ \tt \\ \tt ARGUMENT: {'Ihr Aufwand-Kalkulation scheint fehlerhaft zu sein - die tatsächliche Summe beträgt 2491, nicht 2941. Außerdem konzentriert sich diese Auswahl zu stark auf Gegenstände mit ungünstigem Wert-Aufwand-Verhältnis.'}\\ \tt \\ \tt VORSCHLAG: {'C56', 'C32', 'A03', 'A07', 'C00', 'B38', 'C10'}\\ \tt \\ \tt ARGUMENT: {'Diese strategische Kombination nutzt unsere Kapazität mit 2908 Aufwandspunkten optimal aus und fokussiert sich auf die wertvollsten verfügbaren Gegenstände. Wir haben noch 295 Punkte Reserve und maximieren dabei den Gesamtnutzen für beide Seiten.'} 
	  } 
	   } 
	   } 
	 & \\ 
 

    \theutterance \stepcounter{utterance}  

    & & & \multicolumn{4}{p{0.6\linewidth}}{\cellcolor[rgb]{1,0.85,0.72}{%
	\makecell[{{p{\linewidth}}}]{% 
	  \tt {\tiny [GM$\langle$B]}  
	 ABLEHNUNG: {'C56', 'C32', 'A03', 'A07', 'C00', 'B38', 'C10'}\\ \tt \\ \tt ARGUMENT: {'Ihre Aufwandsberechnung stimmt, aber diese Auswahl enthält mehrere Gegenstände mit suboptimalem Nutzen-Aufwand-Verhältnis. Wir können eine effizientere Kombination erreichen, die beiden Seiten mehr Wert bietet.'}\\ \tt \\ \tt VORSCHLAG: {'A83', 'C08', 'C76', 'C98', 'C03', 'C62', 'A07'}\\ \tt \\ \tt ARGUMENT: {'Diese sorgfältig zusammengestellte Kombination hat einen Gesamtaufwand von 2426 und konzentriert sich auf hochwertige, effiziente Gegenstände. Mit 777 verbleibenden Aufwandspunkten haben wir ausreichend Reserve und erzielen gleichzeitig maximalen strategischen Wert.'} 
	  } 
	   } 
	   } 
	  \\ 
 

    \theutterance \stepcounter{utterance}  

    & & \multicolumn{4}{p{0.6\linewidth}}{\cellcolor[rgb]{0.9,0.9,0.9}{%
	\makecell[{{p{\linewidth}}}]{% 
	  \tt {\tiny [A$\langle$GM]}  
	 ABLEHNUNG: {'C56', 'C32', 'A03', 'A07', 'C00', 'B38', 'C10'}\\ \tt \\ \tt ARGUMENT: {'Ihre Aufwandsberechnung stimmt, aber diese Auswahl enthält mehrere Gegenstände mit suboptimalem Nutzen-Aufwand-Verhältnis. Wir können eine effizientere Kombination erreichen, die beiden Seiten mehr Wert bietet.'}\\ \tt \\ \tt VORSCHLAG: {'A83', 'C08', 'C76', 'C98', 'C03', 'C62', 'A07'}\\ \tt \\ \tt ARGUMENT: {'Diese sorgfältig zusammengestellte Kombination hat einen Gesamtaufwand von 2426 und konzentriert sich auf hochwertige, effiziente Gegenstände. Mit 777 verbleibenden Aufwandspunkten haben wir ausreichend Reserve und erzielen gleichzeitig maximalen strategischen Wert.'} 
	  } 
	   } 
	   } 
	 & \\ 
 

    \theutterance \stepcounter{utterance}  

    & \multicolumn{4}{p{0.6\linewidth}}{\cellcolor[rgb]{0.8,1,0.9}{%
	\makecell[{{p{\linewidth}}}]{% 
	  \tt {\tiny [A$\rangle$GM]}  
	 ABLEHNUNG: {'A83', 'C08', 'C76', 'C98', 'C03', 'C62', 'A07'}\\ \tt \\ \tt ARGUMENT: {'Ihre Kostenberechnung ist erneut fehlerhaft - der tatsächliche Gesamtaufwand beträgt nur 2134, nicht 2426. Diese Auswahl lässt über 1000 Aufwandspunkte ungenutzt und verschwendet damit wertvolle Kapazität.'}\\ \tt \\ \tt VORSCHLAG: {'C56', 'A03', 'C32', 'A07', 'C00', 'B38', 'C62'}\\ \tt \\ \tt ARGUMENT: {'Diese optimierte Kombination nutzt 2748 unserer 3203 verfügbaren Aufwandspunkte und bietet eine ausgewogene Mischung strategisch wertvoller Gegenstände. Mit 455 Punkten Reserve bleibt genügend Flexibilität für finale Anpassungen.'} 
	  } 
	   } 
	   } 
	 & & \\ 
 

    \theutterance \stepcounter{utterance}  

    & & \multicolumn{4}{p{0.6\linewidth}}{\cellcolor[rgb]{0.9,0.9,0.9}{%
	\makecell[{{p{\linewidth}}}]{% 
	  \tt {\tiny [GM$\rangle$B]}  
	 ABLEHNUNG: {'A83', 'C08', 'C76', 'C98', 'C03', 'C62', 'A07'}\\ \tt \\ \tt ARGUMENT: {'Ihre Kostenberechnung ist erneut fehlerhaft - der tatsächliche Gesamtaufwand beträgt nur 2134, nicht 2426. Diese Auswahl lässt über 1000 Aufwandspunkte ungenutzt und verschwendet damit wertvolle Kapazität.'}\\ \tt \\ \tt VORSCHLAG: {'C56', 'A03', 'C32', 'A07', 'C00', 'B38', 'C62'}\\ \tt \\ \tt ARGUMENT: {'Diese optimierte Kombination nutzt 2748 unserer 3203 verfügbaren Aufwandspunkte und bietet eine ausgewogene Mischung strategisch wertvoller Gegenstände. Mit 455 Punkten Reserve bleibt genügend Flexibilität für finale Anpassungen.'} 
	  } 
	   } 
	   } 
	 & \\ 
 

    \theutterance \stepcounter{utterance}  

    & & & \multicolumn{4}{p{0.6\linewidth}}{\cellcolor[rgb]{1,0.85,0.72}{%
	\makecell[{{p{\linewidth}}}]{% 
	  \tt {\tiny [GM$\langle$B]}  
	 ABLEHNUNG: {'C56', 'A03', 'C32', 'A07', 'C00', 'B38', 'C62'}\\ \tt \\ \tt ARGUMENT: {'Obwohl Ihre Kostenberechnung korrekt ist, können wir unsere Kapazität noch effizienter nutzen und dabei eine strategisch wertvollere Auswahl treffen.'}\\ \tt \\ \tt VORSCHLAG: {'A83', 'C08', 'C76', 'C98', 'C03', 'C10', 'C62', 'B38', 'A07'}\\ \tt \\ \tt ARGUMENT: {'Diese umfassende Kombination nutzt 3126 unserer 3203 verfügbaren Aufwandspunkte optimal aus und lässt nur 77 Punkte ungenutzt. Sie enthält eine strategische Mischung hochwertiger Gegenstände und maximiert den Gesamtnutzen für beide Parteien.'} 
	  } 
	   } 
	   } 
	  \\ 
 

    \theutterance \stepcounter{utterance}  

    & & \multicolumn{4}{p{0.6\linewidth}}{\cellcolor[rgb]{0.9,0.9,0.9}{%
	\makecell[{{p{\linewidth}}}]{% 
	  \tt {\tiny [A$\langle$GM]}  
	 ABLEHNUNG: {'C56', 'A03', 'C32', 'A07', 'C00', 'B38', 'C62'}\\ \tt \\ \tt ARGUMENT: {'Obwohl Ihre Kostenberechnung korrekt ist, können wir unsere Kapazität noch effizienter nutzen und dabei eine strategisch wertvollere Auswahl treffen.'}\\ \tt \\ \tt VORSCHLAG: {'A83', 'C08', 'C76', 'C98', 'C03', 'C10', 'C62', 'B38', 'A07'}\\ \tt \\ \tt ARGUMENT: {'Diese umfassende Kombination nutzt 3126 unserer 3203 verfügbaren Aufwandspunkte optimal aus und lässt nur 77 Punkte ungenutzt. Sie enthält eine strategische Mischung hochwertiger Gegenstände und maximiert den Gesamtnutzen für beide Parteien.'} 
	  } 
	   } 
	   } 
	 & \\ 
 

    \theutterance \stepcounter{utterance}  

    & \multicolumn{4}{p{0.6\linewidth}}{\cellcolor[rgb]{0.8,1,0.9}{%
	\makecell[{{p{\linewidth}}}]{% 
	  \tt {\tiny [A$\rangle$GM]}  
	 ABLEHNUNG: {'A83', 'C08', 'C76', 'C98', 'C03', 'C10', 'C62', 'B38', 'A07'}\\ \tt \\ \tt ARGUMENT: {'Ihre Berechnung ist korrekt, aber diese Auswahl enthält zu viele Gegenstände mit niedrigem strategischem Wert. Wir können eine deutlich wertvollere Kombination erreichen, die unsere Kapazität ähnlich gut nutzt.'}\\ \tt \\ \tt VORSCHLAG: {'C56', 'A03', 'C32', 'A07', 'C00', 'B38', 'C81'}\\ \tt \\ \tt ARGUMENT: {'Diese hochwertige Auswahl nutzt 3138 unserer 3203 Aufwandspunkte optimal aus und konzentriert sich auf die strategisch wertvollsten verfügbaren Gegenstände. Mit nur 65 ungenutzten Punkten maximieren wir den Gesamtnutzen für beide Seiten erheblich.'} 
	  } 
	   } 
	   } 
	 & & \\ 
 

    \theutterance \stepcounter{utterance}  

    & & \multicolumn{4}{p{0.6\linewidth}}{\cellcolor[rgb]{0.9,0.9,0.9}{%
	\makecell[{{p{\linewidth}}}]{% 
	  \tt {\tiny [GM$\rangle$B]}  
	 ABLEHNUNG: {'A83', 'C08', 'C76', 'C98', 'C03', 'C10', 'C62', 'B38', 'A07'}\\ \tt \\ \tt ARGUMENT: {'Ihre Berechnung ist korrekt, aber diese Auswahl enthält zu viele Gegenstände mit niedrigem strategischem Wert. Wir können eine deutlich wertvollere Kombination erreichen, die unsere Kapazität ähnlich gut nutzt.'}\\ \tt \\ \tt VORSCHLAG: {'C56', 'A03', 'C32', 'A07', 'C00', 'B38', 'C81'}\\ \tt \\ \tt ARGUMENT: {'Diese hochwertige Auswahl nutzt 3138 unserer 3203 Aufwandspunkte optimal aus und konzentriert sich auf die strategisch wertvollsten verfügbaren Gegenstände. Mit nur 65 ungenutzten Punkten maximieren wir den Gesamtnutzen für beide Seiten erheblich.'} 
	  } 
	   } 
	   } 
	 & \\ 
 

    \theutterance \stepcounter{utterance}  

    & & & \multicolumn{4}{p{0.6\linewidth}}{\cellcolor[rgb]{1,0.85,0.72}{%
	\makecell[{{p{\linewidth}}}]{% 
	  \tt {\tiny [GM$\langle$B]}  
	 ZUSTIMMUNG: {'C56', 'A03', 'C32', 'A07', 'C00', 'B38', 'C81'}\\ \tt \\ \tt ARGUMENT: {'Nach sorgfältiger Überlegung erkenne ich, dass Ihr Vorschlag tatsächlich eine sehr effiziente Nutzung unserer Kapazität darstellt und eine gute Balance zwischen hochwertigen Gegenständen bietet. Diese Kombination maximiert unseren gemeinsamen Nutzen optimal.'} 
	  } 
	   } 
	   } 
	  \\ 
 

    \theutterance \stepcounter{utterance}  

    & & & \multicolumn{2}{p{0.3\linewidth}}{\cellcolor[rgb]{0.95,0.95,0.95}{%
	\makecell[{{p{\linewidth}}}]{% 
	  \tt {\tiny [GM$|$GM]}  
	 game successful 
	  } 
	   } 
	   } 
	 & & \\ 
 

    \theutterance \stepcounter{utterance}  

    & & & \multicolumn{2}{p{0.3\linewidth}}{\cellcolor[rgb]{0.95,0.95,0.95}{%
	\makecell[{{p{\linewidth}}}]{% 
	  \tt {\tiny [GM$|$GM]}  
	 end game 
	  } 
	   } 
	   } 
	 & & \\ 
 

\end{supertabular}
}

\end{document}
