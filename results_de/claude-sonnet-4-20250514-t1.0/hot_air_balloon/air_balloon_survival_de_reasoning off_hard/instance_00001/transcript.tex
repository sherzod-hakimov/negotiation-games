
\documentclass{article}
\usepackage{colortbl}
\usepackage{makecell}
\usepackage{multirow}
\usepackage{supertabular}

\begin{document}

\newcounter{utterance}

\centering \large Interaction Transcript for game `hot\_air\_balloon', experiment `air\_balloon\_survival\_de\_reasoning off\_hard', episode 1 with claude{-}sonnet{-}4{-}20250514{-}t1.0.
\vspace{24pt}

{ \footnotesize  \setcounter{utterance}{1}
\setlength{\tabcolsep}{0pt}
\begin{supertabular}{c@{$\;$}|p{.15\linewidth}@{}p{.15\linewidth}p{.15\linewidth}p{.15\linewidth}p{.15\linewidth}p{.15\linewidth}}
    \# & $\;$A & \multicolumn{4}{c}{Game Master} & $\;\:$B\\
    \hline

    \theutterance \stepcounter{utterance}  
    & & \multicolumn{4}{p{0.6\linewidth}}{
        \cellcolor[rgb]{0.9,0.9,0.9}{
            \makecell[{{p{\linewidth}}}]{
                \texttt{\tiny{[P1$\langle$GM]}}
                \texttt{Du nimmst an einem kollaborativen Verhandlungspiel Teil.} \\
\\ 
\texttt{Zusammen mit einem anderen Teilnehmer musst du dich auf eine Reihe von Gegenständen entscheiden, die behalten werden. Jeder von euch hat eine persönliche Verteilung über die Wichtigkeit der einzelnen Gegenstände. Jeder von euch hat eine eigene Meinung darüber, wie wichtig jeder einzelne Gegenstand ist (Gegenstandswichtigkeit). Du kennst die Wichtigkeitsverteilung des anderen Spielers nicht. Zusätzlich siehst du, wie viel Aufwand jeder Gegenstand verursacht.  } \\
\texttt{Ihr dürft euch nur auf eine Reihe von Gegenständen einigen, wenn der Gesamtaufwand der ausgewählten Gegenstände den Maximalaufwand nicht überschreitet:} \\
\\ 
\texttt{Maximalaufwand = 3357} \\
\\ 
\texttt{Hier sind die einzelnen Aufwände der Gegenstände:} \\
\\ 
\texttt{Aufwand der Gegenstände = \{"C76": 506, "C38": 492, "C56": 338, "A03": 122, "A07": 131, "A83": 909, "C98": 144, "C08": 716, "C62": 263, "C00": 231, "C10": 91, "C81": 651, "B38": 552, "C03": 849, "C32": 720\}} \\
\\ 
\texttt{Hier ist deine persönliche Verteilung der Wichtigkeit der einzelnen Gegenstände:} \\
\\ 
\texttt{Werte der Gegenstandswichtigkeit = \{"C76": 138, "C38": 583, "C56": 868, "A03": 822, "A07": 783, "A83": 65, "C98": 262, "C08": 121, "C62": 508, "C00": 780, "C10": 461, "C81": 484, "B38": 668, "C03": 389, "C32": 808\}} \\
\\ 
\texttt{Ziel:} \\
\\ 
\texttt{Dein Ziel ist es, eine Reihe von Gegenständen auszuhandeln, die dir möglichst viel bringt (d. h. Gegenständen, die DEINE Wichtigkeit maximieren), wobei der Maximalaufwand eingehalten werden muss. Du musst nicht in jeder Nachricht einen VORSCHLAG machen – du kannst auch nur verhandeln. Alle Taktiken sind erlaubt!} \\
\\ 
\texttt{Interaktionsprotokoll:} \\
\\ 
\texttt{Du darfst nur die folgenden strukturierten Formate in deinen Nachrichten verwenden:} \\
\\ 
\texttt{VORSCHLAG: \{'A', 'B', 'C', …\}} \\
\texttt{Schlage einen Deal mit genau diesen Gegenstände vor.} \\
\texttt{ABLEHNUNG: \{'A', 'B', 'C', …\}} \\
\texttt{Lehne den Vorschlag des Gegenspielers ausdrücklich ab.} \\
\texttt{ARGUMENT: \{'...'\}} \\
\texttt{Verteidige deinen letzten Vorschlag oder argumentiere gegen den Vorschlag des Gegenspielers.} \\
\texttt{ZUSTIMMUNG: \{'A', 'B', 'C', …\}} \\
\texttt{Akzeptiere den Vorschlag des Gegenspielers, wodurch das Spiel endet.} \\
\\ 
\\ 
\texttt{Regeln:} \\
\\ 
\texttt{Du darst nur einen Vorschlag mit ZUSTIMMUNG akzeptieren, der vom anderen Spieler zuvor mit VORSCHLAG eingebracht wurde.} \\
\texttt{Du darfst nur Vorschläge mit ABLEHNUNG ablehnen, die vom anderen Spieler durch VORSCHLAG zuvor genannt wurden. } \\
\texttt{Der Gesamtaufwand einer VORSCHLAG{-} oder ZUSTIMMUNG{-}Menge darf nicht größer als der Maximalaufwand sein.  } \\
\texttt{Offenbare deine versteckte Wichtigkeitsverteilung nicht.} \\
\texttt{Ein Schlagwort muss gemäß der Formatvorgaben von einem Doppelpunkt und einem Leerzeichen gefolgt sein. Das Argument muss eine Python{-}Menge mit 0 oder mehr Strings sein.  } \\
\texttt{Die Form muss also SCHLAGWORT: \{...\} sein.} \\
\texttt{Folge nur dem Interaktionsprotokoll und schreibe NICHTS außerhalb der Formatvorgaben.} \\
\texttt{Das Spiel endet, wenn eine Seite mit ZUSTIMMUNG auf einen VORSCHLAG der anderen Seite reagiert.  } \\
\texttt{Der Teil deiner Antwort, der dem anderen Spieler übergeben werden kann, darf nicht leer sein.  } \\
\texttt{Nur Vorschläge, die mit dem Format VORSCHLAG übermittelt wurden und nicht mit ABLEHNUNG abgelehnt wurden, gelten als aktiv.  } \\
\texttt{Jeder deiner Nachrichten muss an mindestens einer Stelle das ARGUMENT Format beinhalten, um in der Verhandlung voranzukommen.} \\
            }
        }
    }
    & \\ \\

    \theutterance \stepcounter{utterance}  
    & \multicolumn{4}{p{0.6\linewidth}}{
        \cellcolor[rgb]{0.9,0.9,0.9}{
            \makecell[{{p{\linewidth}}}]{
                \texttt{\tiny{[P1$\rangle$GM]}}
                \texttt{VORSCHLAG: \{'A03', 'A07', 'C10', 'C00', 'C56', 'C62', 'C98', 'C38', 'B38', 'C32'\}} \\
\\ 
\texttt{ARGUMENT: \{'Dieser Vorschlag nutzt unseren verfügbaren Aufwand sehr effizient aus und enthält eine ausgewogene Mischung verschiedener Gegenstände. Mit einem Gesamtaufwand von 3084 bleiben wir deutlich unter dem Limit von 3357 und haben trotzdem eine substanzielle Auswahl getroffen.'\}} \\
            }
        }
    }
    & & \\ \\

    \theutterance \stepcounter{utterance}  
    & & \multicolumn{4}{p{0.6\linewidth}}{
        \cellcolor[rgb]{0.9,0.9,0.9}{
            \makecell[{{p{\linewidth}}}]{
                \texttt{\tiny{[GM$\rangle$P2]}}
                \texttt{Du nimmst an einem kollaborativen Verhandlungspiel Teil.} \\
\\ 
\texttt{Zusammen mit einem anderen Teilnehmer musst du dich auf eine Reihe von Gegenständen entscheiden, die behalten werden. Jeder von euch hat eine persönliche Verteilung über die Wichtigkeit der einzelnen Gegenstände. Jeder von euch hat eine eigene Meinung darüber, wie wichtig jeder einzelne Gegenstand ist (Gegenstandswichtigkeit). Du kennst die Wichtigkeitsverteilung des anderen Spielers nicht. Zusätzlich siehst du, wie viel Aufwand jeder Gegenstand verursacht.  } \\
\texttt{Ihr dürft euch nur auf eine Reihe von Gegenständen einigen, wenn der Gesamtaufwand der ausgewählten Gegenstände den Maximalaufwand nicht überschreitet:} \\
\\ 
\texttt{Maximalaufwand = 3357} \\
\\ 
\texttt{Hier sind die einzelnen Aufwände der Gegenstände:} \\
\\ 
\texttt{Aufwand der Gegenstände = \{"C76": 506, "C38": 492, "C56": 338, "A03": 122, "A07": 131, "A83": 909, "C98": 144, "C08": 716, "C62": 263, "C00": 231, "C10": 91, "C81": 651, "B38": 552, "C03": 849, "C32": 720\}} \\
\\ 
\texttt{Hier ist deine persönliche Verteilung der Wichtigkeit der einzelnen Gegenstände:} \\
\\ 
\texttt{Werte der Gegenstandswichtigkeit = \{"C56": 65, "A03": 121, "C32": 138, "A07": 262, "C00": 389, "B38": 461, "C38": 484, "C62": 508, "C81": 583, "C10": 668, "C03": 780, "C98": 783, "C76": 808, "C08": 822, "A83": 868\}} \\
\\ 
\texttt{Ziel:} \\
\\ 
\texttt{Dein Ziel ist es, eine Reihe von Gegenständen auszuhandeln, die dir möglichst viel bringt (d. h. Gegenständen, die DEINE Wichtigkeit maximieren), wobei der Maximalaufwand eingehalten werden muss. Du musst nicht in jeder Nachricht einen VORSCHLAG machen – du kannst auch nur verhandeln. Alle Taktiken sind erlaubt!} \\
\\ 
\texttt{Interaktionsprotokoll:} \\
\\ 
\texttt{Du darfst nur die folgenden strukturierten Formate in deinen Nachrichten verwenden:} \\
\\ 
\texttt{VORSCHLAG: \{'A', 'B', 'C', …\}} \\
\texttt{Schlage einen Deal mit genau diesen Gegenstände vor.} \\
\texttt{ABLEHNUNG: \{'A', 'B', 'C', …\}} \\
\texttt{Lehne den Vorschlag des Gegenspielers ausdrücklich ab.} \\
\texttt{ARGUMENT: \{'...'\}} \\
\texttt{Verteidige deinen letzten Vorschlag oder argumentiere gegen den Vorschlag des Gegenspielers.} \\
\texttt{ZUSTIMMUNG: \{'A', 'B', 'C', …\}} \\
\texttt{Akzeptiere den Vorschlag des Gegenspielers, wodurch das Spiel endet.} \\
\\ 
\\ 
\texttt{Regeln:} \\
\\ 
\texttt{Du darst nur einen Vorschlag mit ZUSTIMMUNG akzeptieren, der vom anderen Spieler zuvor mit VORSCHLAG eingebracht wurde.} \\
\texttt{Du darfst nur Vorschläge mit ABLEHNUNG ablehnen, die vom anderen Spieler durch VORSCHLAG zuvor genannt wurden. } \\
\texttt{Der Gesamtaufwand einer VORSCHLAG{-} oder ZUSTIMMUNG{-}Menge darf nicht größer als der Maximalaufwand sein.  } \\
\texttt{Offenbare deine versteckte Wichtigkeitsverteilung nicht.} \\
\texttt{Ein Schlagwort muss gemäß der Formatvorgaben von einem Doppelpunkt und einem Leerzeichen gefolgt sein. Das Argument muss eine Python{-}Menge mit 0 oder mehr Strings sein.  } \\
\texttt{Die Form muss also SCHLAGWORT: \{...\} sein.} \\
\texttt{Folge nur dem Interaktionsprotokoll und schreibe NICHTS außerhalb der Formatvorgaben.} \\
\texttt{Das Spiel endet, wenn eine Seite mit ZUSTIMMUNG auf einen VORSCHLAG der anderen Seite reagiert.  } \\
\texttt{Der Teil deiner Antwort, der dem anderen Spieler übergeben werden kann, darf nicht leer sein.  } \\
\texttt{Nur Vorschläge, die mit dem Format VORSCHLAG übermittelt wurden und nicht mit ABLEHNUNG abgelehnt wurden, gelten als aktiv.  } \\
\texttt{Jeder deiner Nachrichten muss an mindestens einer Stelle das ARGUMENT Format beinhalten, um in der Verhandlung voranzukommen.} \\
\\ 
\\ 
\texttt{Du erhältst nun die erste Nachricht des anderen Spielers.} \\
\\ 
\texttt{VORSCHLAG: \{'A03', 'A07', 'C10', 'C00', 'C56', 'C62', 'C98', 'C38', 'B38', 'C32'\}} \\
\\ 
\texttt{ARGUMENT: \{'Dieser Vorschlag nutzt unseren verfügbaren Aufwand sehr effizient aus und enthält eine ausgewogene Mischung verschiedener Gegenstände. Mit einem Gesamtaufwand von 3084 bleiben wir deutlich unter dem Limit von 3357 und haben trotzdem eine substanzielle Auswahl getroffen.'\}} \\
            }
        }
    }
    & \\ \\

    \theutterance \stepcounter{utterance}  
    & & & \multicolumn{4}{p{0.6\linewidth}}{
        \cellcolor[rgb]{0.9,0.9,0.9}{
            \makecell[{{p{\linewidth}}}]{
                \texttt{\tiny{[GM$\langle$P2]}}
                \texttt{ABLEHNUNG: \{'A03', 'A07', 'C10', 'C00', 'C56', 'C62', 'C98', 'C38', 'B38', 'C32'\}} \\
\\ 
\texttt{VORSCHLAG: \{'A83', 'C08', 'C76', 'C98', 'C03', 'C10', 'A03'\}} \\
\\ 
\texttt{ARGUMENT: \{'Mein Gegenvorschlag konzentriert sich auf die wertvollsten Gegenstände bei optimaler Ressourcennutzung. Mit einem Gesamtaufwand von 3337 nutzen wir fast unseren gesamten verfügbaren Aufwand von 3357 aus und erhalten dafür eine Auswahl der hochwertigsten verfügbaren Gegenstände.'\}} \\
            }
        }
    }
     \\ \\

    \theutterance \stepcounter{utterance}  
    & & \multicolumn{4}{p{0.6\linewidth}}{
        \cellcolor[rgb]{0.9,0.9,0.9}{
            \makecell[{{p{\linewidth}}}]{
                \texttt{\tiny{[P1$\langle$GM]}}
                \texttt{ABLEHNUNG: \{'A03', 'A07', 'C10', 'C00', 'C56', 'C62', 'C98', 'C38', 'B38', 'C32'\}} \\
\\ 
\texttt{VORSCHLAG: \{'A83', 'C08', 'C76', 'C98', 'C03', 'C10', 'A03'\}} \\
\\ 
\texttt{ARGUMENT: \{'Mein Gegenvorschlag konzentriert sich auf die wertvollsten Gegenstände bei optimaler Ressourcennutzung. Mit einem Gesamtaufwand von 3337 nutzen wir fast unseren gesamten verfügbaren Aufwand von 3357 aus und erhalten dafür eine Auswahl der hochwertigsten verfügbaren Gegenstände.'\}} \\
            }
        }
    }
    & \\ \\

    \theutterance \stepcounter{utterance}  
    & \multicolumn{4}{p{0.6\linewidth}}{
        \cellcolor[rgb]{0.9,0.9,0.9}{
            \makecell[{{p{\linewidth}}}]{
                \texttt{\tiny{[P1$\rangle$GM]}}
                \texttt{ABLEHNUNG: \{'A83', 'C08', 'C76', 'C98', 'C03', 'C10', 'A03'\}} \\
\\ 
\texttt{VORSCHLAG: \{'A03', 'A07', 'C56', 'C00', 'C32', 'C38', 'B38', 'C62', 'C10'\}} \\
\\ 
\texttt{ARGUMENT: \{'Ihr Vorschlag enthält zu viele aufwandsintensive Gegenstände mit fragwürdigem Nutzen. Mein neuer Vorschlag bietet eine bessere Balance zwischen Aufwand und Nutzen. Mit einem Gesamtaufwand von 2940 nutzen wir unsere Ressourcen effizienter und erhalten dennoch eine starke Gegenstandsauswahl, die für beide Seiten vorteilhaft sein sollte.'\}} \\
            }
        }
    }
    & & \\ \\

    \theutterance \stepcounter{utterance}  
    & & \multicolumn{4}{p{0.6\linewidth}}{
        \cellcolor[rgb]{0.9,0.9,0.9}{
            \makecell[{{p{\linewidth}}}]{
                \texttt{\tiny{[GM$\rangle$P2]}}
                \texttt{ABLEHNUNG: \{'A83', 'C08', 'C76', 'C98', 'C03', 'C10', 'A03'\}} \\
\\ 
\texttt{VORSCHLAG: \{'A03', 'A07', 'C56', 'C00', 'C32', 'C38', 'B38', 'C62', 'C10'\}} \\
\\ 
\texttt{ARGUMENT: \{'Ihr Vorschlag enthält zu viele aufwandsintensive Gegenstände mit fragwürdigem Nutzen. Mein neuer Vorschlag bietet eine bessere Balance zwischen Aufwand und Nutzen. Mit einem Gesamtaufwand von 2940 nutzen wir unsere Ressourcen effizienter und erhalten dennoch eine starke Gegenstandsauswahl, die für beide Seiten vorteilhaft sein sollte.'\}} \\
            }
        }
    }
    & \\ \\

    \theutterance \stepcounter{utterance}  
    & & & \multicolumn{4}{p{0.6\linewidth}}{
        \cellcolor[rgb]{0.9,0.9,0.9}{
            \makecell[{{p{\linewidth}}}]{
                \texttt{\tiny{[GM$\langle$P2]}}
                \texttt{ABLEHNUNG: \{'A03', 'A07', 'C56', 'C00', 'C32', 'C38', 'B38', 'C62', 'C10'\}} \\
\\ 
\texttt{VORSCHLAG: \{'A83', 'C08', 'C76', 'C98', 'C10', 'A07', 'A03'\}} \\
\\ 
\texttt{ARGUMENT: \{'Ihr Vorschlag vernachlässigt einige der wertvollsten verfügbaren Optionen. Meine überarbeitete Auswahl kombiniert die hochwertigsten Gegenstände mit kleineren, effizienten Ergänzungen. Mit einem Gesamtaufwand von 3316 maximieren wir den Nutzen bei fast vollständiger Ausschöpfung unseres Budgets von 3357. Diese Kombination bietet deutlich mehr strategischen Wert.'\}} \\
            }
        }
    }
     \\ \\

    \theutterance \stepcounter{utterance}  
    & & \multicolumn{4}{p{0.6\linewidth}}{
        \cellcolor[rgb]{0.9,0.9,0.9}{
            \makecell[{{p{\linewidth}}}]{
                \texttt{\tiny{[P1$\langle$GM]}}
                \texttt{ABLEHNUNG: \{'A03', 'A07', 'C56', 'C00', 'C32', 'C38', 'B38', 'C62', 'C10'\}} \\
\\ 
\texttt{VORSCHLAG: \{'A83', 'C08', 'C76', 'C98', 'C10', 'A07', 'A03'\}} \\
\\ 
\texttt{ARGUMENT: \{'Ihr Vorschlag vernachlässigt einige der wertvollsten verfügbaren Optionen. Meine überarbeitete Auswahl kombiniert die hochwertigsten Gegenstände mit kleineren, effizienten Ergänzungen. Mit einem Gesamtaufwand von 3316 maximieren wir den Nutzen bei fast vollständiger Ausschöpfung unseres Budgets von 3357. Diese Kombination bietet deutlich mehr strategischen Wert.'\}} \\
            }
        }
    }
    & \\ \\

    \theutterance \stepcounter{utterance}  
    & \multicolumn{4}{p{0.6\linewidth}}{
        \cellcolor[rgb]{0.9,0.9,0.9}{
            \makecell[{{p{\linewidth}}}]{
                \texttt{\tiny{[P1$\rangle$GM]}}
                \texttt{ABLEHNUNG: \{'A83', 'C08', 'C76', 'C98', 'C10', 'A07', 'A03'\}} \\
\\ 
\texttt{VORSCHLAG: \{'A03', 'A07', 'C56', 'C00', 'C32', 'C10', 'C98', 'C62', 'B38'\}} \\
\\ 
\texttt{ARGUMENT: \{'Ihr Vorschlag enthält Gegenstände mit sehr ungünstigem Aufwand{-}Nutzen{-}Verhältnis. Mein Gegenvorschlag wählt strategisch Gegenstände aus, die sowohl einzeln als auch in Kombination deutlich effizienter sind. Mit einem Gesamtaufwand von 2828 haben wir noch ausreichend Spielraum für weitere Optimierungen, während wir eine ausgewogene und wertvolle Gegenstandsauswahl erhalten.'\}} \\
            }
        }
    }
    & & \\ \\

    \theutterance \stepcounter{utterance}  
    & & \multicolumn{4}{p{0.6\linewidth}}{
        \cellcolor[rgb]{0.9,0.9,0.9}{
            \makecell[{{p{\linewidth}}}]{
                \texttt{\tiny{[GM$\rangle$P2]}}
                \texttt{ABLEHNUNG: \{'A83', 'C08', 'C76', 'C98', 'C10', 'A07', 'A03'\}} \\
\\ 
\texttt{VORSCHLAG: \{'A03', 'A07', 'C56', 'C00', 'C32', 'C10', 'C98', 'C62', 'B38'\}} \\
\\ 
\texttt{ARGUMENT: \{'Ihr Vorschlag enthält Gegenstände mit sehr ungünstigem Aufwand{-}Nutzen{-}Verhältnis. Mein Gegenvorschlag wählt strategisch Gegenstände aus, die sowohl einzeln als auch in Kombination deutlich effizienter sind. Mit einem Gesamtaufwand von 2828 haben wir noch ausreichend Spielraum für weitere Optimierungen, während wir eine ausgewogene und wertvolle Gegenstandsauswahl erhalten.'\}} \\
            }
        }
    }
    & \\ \\

    \theutterance \stepcounter{utterance}  
    & & & \multicolumn{4}{p{0.6\linewidth}}{
        \cellcolor[rgb]{0.9,0.9,0.9}{
            \makecell[{{p{\linewidth}}}]{
                \texttt{\tiny{[GM$\langle$P2]}}
                \texttt{ABLEHNUNG: \{'A03', 'A07', 'C56', 'C00', 'C32', 'C10', 'C98', 'C62', 'B38'\}} \\
\\ 
\texttt{VORSCHLAG: \{'C08', 'C81', 'C76', 'C98', 'C10', 'C62', 'A03', 'A07'\}} \\
\\ 
\texttt{ARGUMENT: \{'Ihr Ansatz ist zu konservativ und verschwendet verfügbare Ressourcen. Mein Kompromissvorschlag kombiniert einige hochwertige Kernstücke mit mehreren Ihrer bevorzugten kleineren Gegenstände. Mit einem Gesamtaufwand von 3255 nutzen wir unsere Ressourcen optimal aus, ohne das Budget zu überschreiten. Diese Mischung sollte für beide Seiten akzeptable Ergebnisse liefern.'\}} \\
            }
        }
    }
     \\ \\

    \theutterance \stepcounter{utterance}  
    & & \multicolumn{4}{p{0.6\linewidth}}{
        \cellcolor[rgb]{0.9,0.9,0.9}{
            \makecell[{{p{\linewidth}}}]{
                \texttt{\tiny{[P1$\langle$GM]}}
                \texttt{ABLEHNUNG: \{'A03', 'A07', 'C56', 'C00', 'C32', 'C10', 'C98', 'C62', 'B38'\}} \\
\\ 
\texttt{VORSCHLAG: \{'C08', 'C81', 'C76', 'C98', 'C10', 'C62', 'A03', 'A07'\}} \\
\\ 
\texttt{ARGUMENT: \{'Ihr Ansatz ist zu konservativ und verschwendet verfügbare Ressourcen. Mein Kompromissvorschlag kombiniert einige hochwertige Kernstücke mit mehreren Ihrer bevorzugten kleineren Gegenstände. Mit einem Gesamtaufwand von 3255 nutzen wir unsere Ressourcen optimal aus, ohne das Budget zu überschreiten. Diese Mischung sollte für beide Seiten akzeptable Ergebnisse liefern.'\}} \\
            }
        }
    }
    & \\ \\

    \theutterance \stepcounter{utterance}  
    & \multicolumn{4}{p{0.6\linewidth}}{
        \cellcolor[rgb]{0.9,0.9,0.9}{
            \makecell[{{p{\linewidth}}}]{
                \texttt{\tiny{[P1$\rangle$GM]}}
                \texttt{ABLEHNUNG: \{'C08', 'C81', 'C76', 'C98', 'C10', 'C62', 'A03', 'A07'\}} \\
\\ 
\texttt{VORSCHLAG: \{'A03', 'A07', 'C00', 'C56', 'C32', 'C10', 'C62', 'C98', 'B38'\}} \\
\\ 
\texttt{ARGUMENT: \{'Ihr Vorschlag lässt zu viel ungenutztes Aufwandspotential verschwendet {-} mit nur 2624 von 3357 verfügbaren Punkten. Mein Vorschlag nutzt mit 2828 Aufwandspunkten unsere Ressourcen besser aus und bietet eine strategisch klügere Kombination aus wertvollen Kernelementen und effizienten Ergänzungen. Diese Auswahl maximiert den Gesamtnutzen bei besserer Ressourcenausnutzung.'\}} \\
            }
        }
    }
    & & \\ \\

    \theutterance \stepcounter{utterance}  
    & & \multicolumn{4}{p{0.6\linewidth}}{
        \cellcolor[rgb]{0.9,0.9,0.9}{
            \makecell[{{p{\linewidth}}}]{
                \texttt{\tiny{[GM$\rangle$P2]}}
                \texttt{ABLEHNUNG: \{'C08', 'C81', 'C76', 'C98', 'C10', 'C62', 'A03', 'A07'\}} \\
\\ 
\texttt{VORSCHLAG: \{'A03', 'A07', 'C00', 'C56', 'C32', 'C10', 'C62', 'C98', 'B38'\}} \\
\\ 
\texttt{ARGUMENT: \{'Ihr Vorschlag lässt zu viel ungenutztes Aufwandspotential verschwendet {-} mit nur 2624 von 3357 verfügbaren Punkten. Mein Vorschlag nutzt mit 2828 Aufwandspunkten unsere Ressourcen besser aus und bietet eine strategisch klügere Kombination aus wertvollen Kernelementen und effizienten Ergänzungen. Diese Auswahl maximiert den Gesamtnutzen bei besserer Ressourcenausnutzung.'\}} \\
            }
        }
    }
    & \\ \\

    \theutterance \stepcounter{utterance}  
    & & & \multicolumn{4}{p{0.6\linewidth}}{
        \cellcolor[rgb]{0.9,0.9,0.9}{
            \makecell[{{p{\linewidth}}}]{
                \texttt{\tiny{[GM$\langle$P2]}}
                \texttt{ABLEHNUNG: \{'A03', 'A07', 'C00', 'C56', 'C32', 'C10', 'C62', 'C98', 'B38'\}} \\
\\ 
\texttt{VORSCHLAG: \{'C08', 'C76', 'C98', 'C10', 'C62', 'C00', 'A03', 'A07', 'C56'\}} \\
\\ 
\texttt{ARGUMENT: \{'Ihr Vorschlag hat durchaus Verdienste, aber wir können noch mehr herausholen. Meine Auswahl kombiniert Ihre effizienten kleineren Gegenstände mit einigen strategisch wichtigen größeren Komponenten. Mit einem Gesamtaufwand von 3351 schöpfen wir unser verfügbares Budget von 3357 fast vollständig aus und erhalten eine ausgewogenere Mischung aus hochwertigen und kosteneffizienten Elementen.'\}} \\
            }
        }
    }
     \\ \\

    \theutterance \stepcounter{utterance}  
    & & \multicolumn{4}{p{0.6\linewidth}}{
        \cellcolor[rgb]{0.9,0.9,0.9}{
            \makecell[{{p{\linewidth}}}]{
                \texttt{\tiny{[P1$\langle$GM]}}
                \texttt{ABLEHNUNG: \{'A03', 'A07', 'C00', 'C56', 'C32', 'C10', 'C62', 'C98', 'B38'\}} \\
\\ 
\texttt{VORSCHLAG: \{'C08', 'C76', 'C98', 'C10', 'C62', 'C00', 'A03', 'A07', 'C56'\}} \\
\\ 
\texttt{ARGUMENT: \{'Ihr Vorschlag hat durchaus Verdienste, aber wir können noch mehr herausholen. Meine Auswahl kombiniert Ihre effizienten kleineren Gegenstände mit einigen strategisch wichtigen größeren Komponenten. Mit einem Gesamtaufwand von 3351 schöpfen wir unser verfügbares Budget von 3357 fast vollständig aus und erhalten eine ausgewogenere Mischung aus hochwertigen und kosteneffizienten Elementen.'\}} \\
            }
        }
    }
    & \\ \\

    \theutterance \stepcounter{utterance}  
    & \multicolumn{4}{p{0.6\linewidth}}{
        \cellcolor[rgb]{0.9,0.9,0.9}{
            \makecell[{{p{\linewidth}}}]{
                \texttt{\tiny{[P1$\rangle$GM]}}
                \texttt{ZUSTIMMUNG: \{'C08', 'C76', 'C98', 'C10', 'C62', 'C00', 'A03', 'A07', 'C56'\}} \\
\\ 
\texttt{ARGUMENT: \{'Nach sorgfältiger Überlegung erkenne ich, dass Ihr Vorschlag eine vernünftige Balance zwischen verschiedenen strategischen Prioritäten schafft. Die Kombination aus kosteneffizienten und wertvollen Gegenständen scheint für beide Seiten akzeptabel zu sein und nutzt unsere verfügbaren Ressourcen angemessen aus.'\}} \\
            }
        }
    }
    & & \\ \\

    \theutterance \stepcounter{utterance}  
    & & & \multicolumn{2}{p{0.3\linewidth}}{
        \cellcolor[rgb]{0.9,0.9,0.9}{
            \makecell[{{p{\linewidth}}}]{
                \texttt{\tiny{[GM$|$GM]}}
                \texttt{game successful} \\
            }
        }
    }
    & & \\ \\

    \theutterance \stepcounter{utterance}  
    & & & \multicolumn{2}{p{0.3\linewidth}}{
        \cellcolor[rgb]{0.9,0.9,0.9}{
            \makecell[{{p{\linewidth}}}]{
                \texttt{\tiny{[GM$|$GM]}}
                \texttt{end game} \\
            }
        }
    }
    & & \\ \\

\end{supertabular}
}

\end{document}
