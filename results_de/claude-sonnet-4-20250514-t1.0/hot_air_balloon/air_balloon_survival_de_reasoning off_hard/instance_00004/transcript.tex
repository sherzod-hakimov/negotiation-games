
\documentclass{article}
\usepackage{colortbl}
\usepackage{makecell}
\usepackage{multirow}
\usepackage{supertabular}

\begin{document}

\newcounter{utterance}

\twocolumn

{ \footnotesize  \setcounter{utterance}{1}
\setlength{\tabcolsep}{0pt}
\begin{supertabular}{c@{$\;$}|p{.15\linewidth}@{}p{.15\linewidth}p{.15\linewidth}p{.15\linewidth}p{.15\linewidth}p{.15\linewidth}}

    \# & $\;$A & \multicolumn{4}{c}{Game Master} & $\;\:$B\\
    \hline 

    \theutterance \stepcounter{utterance}  

    & & \multicolumn{4}{p{0.6\linewidth}}{\cellcolor[rgb]{0.9,0.9,0.9}{%
	\makecell[{{p{\linewidth}}}]{% 
	  \tt {\tiny [A$\langle$GM]}  
	 Du nimmst an einem kollaborativen Verhandlungspiel Teil.\\ \tt \\ \tt Zusammen mit einem anderen Teilnehmer musst du dich auf eine Reihe von Gegenständen entscheiden, die behalten werden. Jeder von euch hat eine persönliche Verteilung über die Wichtigkeit der einzelnen Gegenstände. Jeder von euch hat eine eigene Meinung darüber, wie wichtig jeder einzelne Gegenstand ist (Gegenstandswichtigkeit). Du kennst die Wichtigkeitsverteilung des anderen Spielers nicht. Zusätzlich siehst du, wie viel Aufwand jeder Gegenstand verursacht.  \\ \tt Ihr dürft euch nur auf eine Reihe von Gegenständen einigen, wenn der Gesamtaufwand der ausgewählten Gegenstände den Maximalaufwand nicht überschreitet:\\ \tt \\ \tt Maximalaufwand = 2984\\ \tt \\ \tt Hier sind die einzelnen Aufwände der Gegenstände:\\ \tt \\ \tt Aufwand der Gegenstände = {"C76": 350, "C38": 276, "C56": 616, "A03": 737, "A07": 531, "A83": 389, "C98": 24, "C08": 125, "C62": 338, "C00": 356, "C10": 143, "C81": 117, "B38": 257, "C03": 921, "C32": 789}\\ \tt \\ \tt Hier ist deine persönliche Verteilung der Wichtigkeit der einzelnen Gegenstände:\\ \tt \\ \tt Werte der Gegenstandswichtigkeit = {"C76": 138, "C38": 583, "C56": 868, "A03": 822, "A07": 783, "A83": 65, "C98": 262, "C08": 121, "C62": 508, "C00": 780, "C10": 461, "C81": 484, "B38": 668, "C03": 389, "C32": 808}\\ \tt \\ \tt Ziel:\\ \tt \\ \tt Dein Ziel ist es, eine Reihe von Gegenständen auszuhandeln, die dir möglichst viel bringt (d. h. Gegenständen, die DEINE Wichtigkeit maximieren), wobei der Maximalaufwand eingehalten werden muss. Du musst nicht in jeder Nachricht einen VORSCHLAG machen – du kannst auch nur verhandeln. Alle Taktiken sind erlaubt!\\ \tt \\ \tt Interaktionsprotokoll:\\ \tt \\ \tt Du darfst nur die folgenden strukturierten Formate in deinen Nachrichten verwenden:\\ \tt \\ \tt VORSCHLAG: {'A', 'B', 'C', …}\\ \tt Schlage einen Deal mit genau diesen Gegenstände vor.\\ \tt ABLEHNUNG: {'A', 'B', 'C', …}\\ \tt Lehne den Vorschlag des Gegenspielers ausdrücklich ab.\\ \tt ARGUMENT: {'...'}\\ \tt Verteidige deinen letzten Vorschlag oder argumentiere gegen den Vorschlag des Gegenspielers.\\ \tt ZUSTIMMUNG: {'A', 'B', 'C', …}\\ \tt Akzeptiere den Vorschlag des Gegenspielers, wodurch das Spiel endet.\\ \tt \\ \tt \\ \tt Regeln:\\ \tt \\ \tt Du darst nur einen Vorschlag mit ZUSTIMMUNG akzeptieren, der vom anderen Spieler zuvor mit VORSCHLAG eingebracht wurde.\\ \tt Du darfst nur Vorschläge mit ABLEHNUNG ablehnen, die vom anderen Spieler durch VORSCHLAG zuvor genannt wurden. \\ \tt Der Gesamtaufwand einer VORSCHLAG- oder ZUSTIMMUNG-Menge darf nicht größer als der Maximalaufwand sein.  \\ \tt Offenbare deine versteckte Wichtigkeitsverteilung nicht.\\ \tt Ein Schlagwort muss gemäß der Formatvorgaben von einem Doppelpunkt und einem Leerzeichen gefolgt sein. Das Argument muss eine Python-Menge mit 0 oder mehr Strings sein.  \\ \tt Die Form muss also SCHLAGWORT: {...} sein.\\ \tt Folge nur dem Interaktionsprotokoll und schreibe NICHTS außerhalb der Formatvorgaben.\\ \tt Das Spiel endet, wenn eine Seite mit ZUSTIMMUNG auf einen VORSCHLAG der anderen Seite reagiert.  \\ \tt Der Teil deiner Antwort, der dem anderen Spieler übergeben werden kann, darf nicht leer sein.  \\ \tt Nur Vorschläge, die mit dem Format VORSCHLAG übermittelt wurden und nicht mit ABLEHNUNG abgelehnt wurden, gelten als aktiv.  \\ \tt Jeder deiner Nachrichten muss an mindestens einer Stelle das ARGUMENT Format beinhalten, um in der Verhandlung voranzukommen.\\ \tt  
	  } 
	   } 
	   } 
	 & \\ 
 

    \theutterance \stepcounter{utterance}  

    & \multicolumn{4}{p{0.6\linewidth}}{\cellcolor[rgb]{0.8,1,0.9}{%
	\makecell[{{p{\linewidth}}}]{% 
	  \tt {\tiny [A$\rangle$GM]}  
	 VORSCHLAG: {'C98', 'C81', 'C10', 'B38', 'C00', 'C38', 'C62', 'A07', 'C56', 'C08'}\\ \tt \\ \tt ARGUMENT: {'Dieser Vorschlag nutzt den verfügbaren Aufwand sehr effizient aus und wählt eine ausgewogene Mischung verschiedener Gegenstände. Mit einem Gesamtaufwand von 2783 bleiben wir deutlich unter dem Limit und haben eine solide Grundlage für unsere Zusammenarbeit.'} 
	  } 
	   } 
	   } 
	 & & \\ 
 

    \theutterance \stepcounter{utterance}  

    & & \multicolumn{4}{p{0.6\linewidth}}{\cellcolor[rgb]{0.9,0.9,0.9}{%
	\makecell[{{p{\linewidth}}}]{% 
	  \tt {\tiny [GM$\rangle$B]}  
	 Du nimmst an einem kollaborativen Verhandlungspiel Teil.\\ \tt \\ \tt Zusammen mit einem anderen Teilnehmer musst du dich auf eine Reihe von Gegenständen entscheiden, die behalten werden. Jeder von euch hat eine persönliche Verteilung über die Wichtigkeit der einzelnen Gegenstände. Jeder von euch hat eine eigene Meinung darüber, wie wichtig jeder einzelne Gegenstand ist (Gegenstandswichtigkeit). Du kennst die Wichtigkeitsverteilung des anderen Spielers nicht. Zusätzlich siehst du, wie viel Aufwand jeder Gegenstand verursacht.  \\ \tt Ihr dürft euch nur auf eine Reihe von Gegenständen einigen, wenn der Gesamtaufwand der ausgewählten Gegenstände den Maximalaufwand nicht überschreitet:\\ \tt \\ \tt Maximalaufwand = 2984\\ \tt \\ \tt Hier sind die einzelnen Aufwände der Gegenstände:\\ \tt \\ \tt Aufwand der Gegenstände = {"C76": 350, "C38": 276, "C56": 616, "A03": 737, "A07": 531, "A83": 389, "C98": 24, "C08": 125, "C62": 338, "C00": 356, "C10": 143, "C81": 117, "B38": 257, "C03": 921, "C32": 789}\\ \tt \\ \tt Hier ist deine persönliche Verteilung der Wichtigkeit der einzelnen Gegenstände:\\ \tt \\ \tt Werte der Gegenstandswichtigkeit = {"C56": 65, "A03": 121, "C32": 138, "A07": 262, "C00": 389, "B38": 461, "C38": 484, "C62": 508, "C81": 583, "C10": 668, "C03": 780, "C98": 783, "C76": 808, "C08": 822, "A83": 868}\\ \tt \\ \tt Ziel:\\ \tt \\ \tt Dein Ziel ist es, eine Reihe von Gegenständen auszuhandeln, die dir möglichst viel bringt (d. h. Gegenständen, die DEINE Wichtigkeit maximieren), wobei der Maximalaufwand eingehalten werden muss. Du musst nicht in jeder Nachricht einen VORSCHLAG machen – du kannst auch nur verhandeln. Alle Taktiken sind erlaubt!\\ \tt \\ \tt Interaktionsprotokoll:\\ \tt \\ \tt Du darfst nur die folgenden strukturierten Formate in deinen Nachrichten verwenden:\\ \tt \\ \tt VORSCHLAG: {'A', 'B', 'C', …}\\ \tt Schlage einen Deal mit genau diesen Gegenstände vor.\\ \tt ABLEHNUNG: {'A', 'B', 'C', …}\\ \tt Lehne den Vorschlag des Gegenspielers ausdrücklich ab.\\ \tt ARGUMENT: {'...'}\\ \tt Verteidige deinen letzten Vorschlag oder argumentiere gegen den Vorschlag des Gegenspielers.\\ \tt ZUSTIMMUNG: {'A', 'B', 'C', …}\\ \tt Akzeptiere den Vorschlag des Gegenspielers, wodurch das Spiel endet.\\ \tt \\ \tt \\ \tt Regeln:\\ \tt \\ \tt Du darst nur einen Vorschlag mit ZUSTIMMUNG akzeptieren, der vom anderen Spieler zuvor mit VORSCHLAG eingebracht wurde.\\ \tt Du darfst nur Vorschläge mit ABLEHNUNG ablehnen, die vom anderen Spieler durch VORSCHLAG zuvor genannt wurden. \\ \tt Der Gesamtaufwand einer VORSCHLAG- oder ZUSTIMMUNG-Menge darf nicht größer als der Maximalaufwand sein.  \\ \tt Offenbare deine versteckte Wichtigkeitsverteilung nicht.\\ \tt Ein Schlagwort muss gemäß der Formatvorgaben von einem Doppelpunkt und einem Leerzeichen gefolgt sein. Das Argument muss eine Python-Menge mit 0 oder mehr Strings sein.  \\ \tt Die Form muss also SCHLAGWORT: {...} sein.\\ \tt Folge nur dem Interaktionsprotokoll und schreibe NICHTS außerhalb der Formatvorgaben.\\ \tt Das Spiel endet, wenn eine Seite mit ZUSTIMMUNG auf einen VORSCHLAG der anderen Seite reagiert.  \\ \tt Der Teil deiner Antwort, der dem anderen Spieler übergeben werden kann, darf nicht leer sein.  \\ \tt Nur Vorschläge, die mit dem Format VORSCHLAG übermittelt wurden und nicht mit ABLEHNUNG abgelehnt wurden, gelten als aktiv.  \\ \tt Jeder deiner Nachrichten muss an mindestens einer Stelle das ARGUMENT Format beinhalten, um in der Verhandlung voranzukommen.\\ \tt \\ \tt \\ \tt Du erhältst nun die erste Nachricht des anderen Spielers.\\ \tt \\ \tt VORSCHLAG: {'C98', 'C81', 'C10', 'B38', 'C00', 'C38', 'C62', 'A07', 'C56', 'C08'}\\ \tt \\ \tt ARGUMENT: {'Dieser Vorschlag nutzt den verfügbaren Aufwand sehr effizient aus und wählt eine ausgewogene Mischung verschiedener Gegenstände. Mit einem Gesamtaufwand von 2783 bleiben wir deutlich unter dem Limit und haben eine solide Grundlage für unsere Zusammenarbeit.'} 
	  } 
	   } 
	   } 
	 & \\ 
 

    \theutterance \stepcounter{utterance}  

    & & & \multicolumn{4}{p{0.6\linewidth}}{\cellcolor[rgb]{1,0.85,0.72}{%
	\makecell[{{p{\linewidth}}}]{% 
	  \tt {\tiny [GM$\langle$B]}  
	 ABLEHNUNG: {'C98', 'C81', 'C10', 'B38', 'C00', 'C38', 'C62', 'A07', 'C56', 'C08'}\\ \tt \\ \tt VORSCHLAG: {'C98', 'C81', 'C10', 'B38', 'C00', 'C38', 'C62', 'A83', 'C76', 'C08'}\\ \tt \\ \tt ARGUMENT: {'Ich sehe Optimierungspotential in der Auswahl. Durch den Austausch von C56 und A07 gegen A83 und C76 können wir den Gesamtaufwand auf 2375 reduzieren und haben damit noch mehr Puffer. Diese Kombination nutzt unsere Ressourcen effizienter und bietet eine bessere Balance zwischen verschiedenen Kategorien.'} 
	  } 
	   } 
	   } 
	  \\ 
 

    \theutterance \stepcounter{utterance}  

    & & \multicolumn{4}{p{0.6\linewidth}}{\cellcolor[rgb]{0.9,0.9,0.9}{%
	\makecell[{{p{\linewidth}}}]{% 
	  \tt {\tiny [A$\langle$GM]}  
	 ABLEHNUNG: {'C98', 'C81', 'C10', 'B38', 'C00', 'C38', 'C62', 'A07', 'C56', 'C08'}\\ \tt \\ \tt VORSCHLAG: {'C98', 'C81', 'C10', 'B38', 'C00', 'C38', 'C62', 'A83', 'C76', 'C08'}\\ \tt \\ \tt ARGUMENT: {'Ich sehe Optimierungspotential in der Auswahl. Durch den Austausch von C56 und A07 gegen A83 und C76 können wir den Gesamtaufwand auf 2375 reduzieren und haben damit noch mehr Puffer. Diese Kombination nutzt unsere Ressourcen effizienter und bietet eine bessere Balance zwischen verschiedenen Kategorien.'} 
	  } 
	   } 
	   } 
	 & \\ 
 

    \theutterance \stepcounter{utterance}  

    & \multicolumn{4}{p{0.6\linewidth}}{\cellcolor[rgb]{0.8,1,0.9}{%
	\makecell[{{p{\linewidth}}}]{% 
	  \tt {\tiny [A$\rangle$GM]}  
	 ABLEHNUNG: {'C98', 'C81', 'C10', 'B38', 'C00', 'C38', 'C62', 'A83', 'C76', 'C08'}\\ \tt \\ \tt VORSCHLAG: {'C98', 'C81', 'C10', 'B38', 'C00', 'C38', 'C62', 'A07', 'A03'}\\ \tt \\ \tt ARGUMENT: {'Ihr Vorschlag entfernt einige strategisch wichtige Gegenstände. Mein Gegenvorschlag konzentriert sich auf hochwertige Kernelemente und erreicht einen Gesamtaufwand von 2646. Diese Auswahl bietet eine bessere Kosten-Nutzen-Relation und stellt sicher, dass wir die wirklich wesentlichen Gegenstände nicht verlieren.'} 
	  } 
	   } 
	   } 
	 & & \\ 
 

    \theutterance \stepcounter{utterance}  

    & & \multicolumn{4}{p{0.6\linewidth}}{\cellcolor[rgb]{0.9,0.9,0.9}{%
	\makecell[{{p{\linewidth}}}]{% 
	  \tt {\tiny [GM$\rangle$B]}  
	 ABLEHNUNG: {'C98', 'C81', 'C10', 'B38', 'C00', 'C38', 'C62', 'A83', 'C76', 'C08'}\\ \tt \\ \tt VORSCHLAG: {'C98', 'C81', 'C10', 'B38', 'C00', 'C38', 'C62', 'A07', 'A03'}\\ \tt \\ \tt ARGUMENT: {'Ihr Vorschlag entfernt einige strategisch wichtige Gegenstände. Mein Gegenvorschlag konzentriert sich auf hochwertige Kernelemente und erreicht einen Gesamtaufwand von 2646. Diese Auswahl bietet eine bessere Kosten-Nutzen-Relation und stellt sicher, dass wir die wirklich wesentlichen Gegenstände nicht verlieren.'} 
	  } 
	   } 
	   } 
	 & \\ 
 

    \theutterance \stepcounter{utterance}  

    & & & \multicolumn{4}{p{0.6\linewidth}}{\cellcolor[rgb]{1,0.85,0.72}{%
	\makecell[{{p{\linewidth}}}]{% 
	  \tt {\tiny [GM$\langle$B]}  
	 ABLEHNUNG: {'C98', 'C81', 'C10', 'B38', 'C00', 'C38', 'C62', 'A07', 'A03'}\\ \tt \\ \tt VORSCHLAG: {'C98', 'C81', 'C10', 'B38', 'C00', 'C38', 'C62', 'A07', 'A03', 'C08'}\\ \tt \\ \tt ARGUMENT: {'Ihr Vorschlag ist durchaus interessant, aber wir haben noch ungenutzten Aufwand von über 200 Einheiten. Durch die Hinzunahme von C08 mit nur 125 Aufwand können wir unsere Ressourcen vollständig ausschöpfen und erreichen einen Gesamtaufwand von 2904. Das bringt uns näher an das Maximum und maximiert den Nutzen unserer Vereinbarung.'} 
	  } 
	   } 
	   } 
	  \\ 
 

    \theutterance \stepcounter{utterance}  

    & & \multicolumn{4}{p{0.6\linewidth}}{\cellcolor[rgb]{0.9,0.9,0.9}{%
	\makecell[{{p{\linewidth}}}]{% 
	  \tt {\tiny [A$\langle$GM]}  
	 ABLEHNUNG: {'C98', 'C81', 'C10', 'B38', 'C00', 'C38', 'C62', 'A07', 'A03'}\\ \tt \\ \tt VORSCHLAG: {'C98', 'C81', 'C10', 'B38', 'C00', 'C38', 'C62', 'A07', 'A03', 'C08'}\\ \tt \\ \tt ARGUMENT: {'Ihr Vorschlag ist durchaus interessant, aber wir haben noch ungenutzten Aufwand von über 200 Einheiten. Durch die Hinzunahme von C08 mit nur 125 Aufwand können wir unsere Ressourcen vollständig ausschöpfen und erreichen einen Gesamtaufwand von 2904. Das bringt uns näher an das Maximum und maximiert den Nutzen unserer Vereinbarung.'} 
	  } 
	   } 
	   } 
	 & \\ 
 

    \theutterance \stepcounter{utterance}  

    & \multicolumn{4}{p{0.6\linewidth}}{\cellcolor[rgb]{0.8,1,0.9}{%
	\makecell[{{p{\linewidth}}}]{% 
	  \tt {\tiny [A$\rangle$GM]}  
	 ZUSTIMMUNG: {'C98', 'C81', 'C10', 'B38', 'C00', 'C38', 'C62', 'A07', 'A03', 'C08'}\\ \tt \\ \tt ARGUMENT: {'Sie haben recht - die vollständige Ausschöpfung unserer verfügbaren Ressourcen ist sinnvoll. Ihr Vorschlag nutzt 2904 von 2984 Aufwandspunkten optimal aus und die Hinzunahme von C08 verbessert das Gesamtpaket. Diese Kombination stellt eine faire und effiziente Lösung dar.'} 
	  } 
	   } 
	   } 
	 & & \\ 
 

    \theutterance \stepcounter{utterance}  

    & & & \multicolumn{2}{p{0.3\linewidth}}{\cellcolor[rgb]{0.95,0.95,0.95}{%
	\makecell[{{p{\linewidth}}}]{% 
	  \tt {\tiny [GM$|$GM]}  
	 game successful 
	  } 
	   } 
	   } 
	 & & \\ 
 

    \theutterance \stepcounter{utterance}  

    & & & \multicolumn{2}{p{0.3\linewidth}}{\cellcolor[rgb]{0.95,0.95,0.95}{%
	\makecell[{{p{\linewidth}}}]{% 
	  \tt {\tiny [GM$|$GM]}  
	 end game 
	  } 
	   } 
	   } 
	 & & \\ 
 

\end{supertabular}
}

\end{document}
