
\documentclass{article}
\usepackage{colortbl}
\usepackage{makecell}
\usepackage{multirow}
\usepackage{supertabular}

\begin{document}

\newcounter{utterance}

\twocolumn

{ \footnotesize  \setcounter{utterance}{1}
\setlength{\tabcolsep}{0pt}
\begin{supertabular}{c@{$\;$}|p{.15\linewidth}@{}p{.15\linewidth}p{.15\linewidth}p{.15\linewidth}p{.15\linewidth}p{.15\linewidth}}

    \# & $\;$A & \multicolumn{4}{c}{Game Master} & $\;\:$B\\
    \hline 

    \theutterance \stepcounter{utterance}  

    & & \multicolumn{4}{p{0.6\linewidth}}{\cellcolor[rgb]{0.9,0.9,0.9}{%
	\makecell[{{p{\linewidth}}}]{% 
	  \tt {\tiny [A$\langle$GM]}  
	 Du nimmst an einem kollaborativen Verhandlungspiel Teil.\\ \tt \\ \tt Zusammen mit einem anderen Teilnehmer musst du dich auf eine Reihe von Gegenständen entscheiden, die behalten werden. Jeder von euch hat eine persönliche Verteilung über die Wichtigkeit der einzelnen Gegenstände. Jeder von euch hat eine eigene Meinung darüber, wie wichtig jeder einzelne Gegenstand ist (Gegenstandswichtigkeit). Du kennst die Wichtigkeitsverteilung des anderen Spielers nicht. Zusätzlich siehst du, wie viel Aufwand jeder Gegenstand verursacht.  \\ \tt Ihr dürft euch nur auf eine Reihe von Gegenständen einigen, wenn der Gesamtaufwand der ausgewählten Gegenstände den Maximalaufwand nicht überschreitet:\\ \tt \\ \tt Maximalaufwand = 3683\\ \tt \\ \tt Hier sind die einzelnen Aufwände der Gegenstände:\\ \tt \\ \tt Aufwand der Gegenstände = {"C76": 548, "C38": 154, "C56": 60, "A03": 517, "A07": 334, "A83": 542, "C98": 707, "C08": 139, "C62": 661, "C00": 780, "C10": 832, "C81": 913, "B38": 219, "C03": 323, "C32": 638}\\ \tt \\ \tt Hier ist deine persönliche Verteilung der Wichtigkeit der einzelnen Gegenstände:\\ \tt \\ \tt Werte der Gegenstandswichtigkeit = {"C76": 138, "C38": 583, "C56": 868, "A03": 822, "A07": 783, "A83": 65, "C98": 262, "C08": 121, "C62": 508, "C00": 780, "C10": 461, "C81": 484, "B38": 668, "C03": 389, "C32": 808}\\ \tt \\ \tt Ziel:\\ \tt \\ \tt Dein Ziel ist es, eine Reihe von Gegenständen auszuhandeln, die dir möglichst viel bringt (d. h. Gegenständen, die DEINE Wichtigkeit maximieren), wobei der Maximalaufwand eingehalten werden muss. Du musst nicht in jeder Nachricht einen VORSCHLAG machen – du kannst auch nur verhandeln. Alle Taktiken sind erlaubt!\\ \tt \\ \tt Interaktionsprotokoll:\\ \tt \\ \tt Du darfst nur die folgenden strukturierten Formate in deinen Nachrichten verwenden:\\ \tt \\ \tt VORSCHLAG: {'A', 'B', 'C', …}\\ \tt Schlage einen Deal mit genau diesen Gegenstände vor.\\ \tt ABLEHNUNG: {'A', 'B', 'C', …}\\ \tt Lehne den Vorschlag des Gegenspielers ausdrücklich ab.\\ \tt ARGUMENT: {'...'}\\ \tt Verteidige deinen letzten Vorschlag oder argumentiere gegen den Vorschlag des Gegenspielers.\\ \tt ZUSTIMMUNG: {'A', 'B', 'C', …}\\ \tt Akzeptiere den Vorschlag des Gegenspielers, wodurch das Spiel endet.\\ \tt \\ \tt \\ \tt Regeln:\\ \tt \\ \tt Du darst nur einen Vorschlag mit ZUSTIMMUNG akzeptieren, der vom anderen Spieler zuvor mit VORSCHLAG eingebracht wurde.\\ \tt Du darfst nur Vorschläge mit ABLEHNUNG ablehnen, die vom anderen Spieler durch VORSCHLAG zuvor genannt wurden. \\ \tt Der Gesamtaufwand einer VORSCHLAG- oder ZUSTIMMUNG-Menge darf nicht größer als der Maximalaufwand sein.  \\ \tt Offenbare deine versteckte Wichtigkeitsverteilung nicht.\\ \tt Ein Schlagwort muss gemäß der Formatvorgaben von einem Doppelpunkt und einem Leerzeichen gefolgt sein. Das Argument muss eine Python-Menge mit 0 oder mehr Strings sein.  \\ \tt Die Form muss also SCHLAGWORT: {...} sein.\\ \tt Folge nur dem Interaktionsprotokoll und schreibe NICHTS außerhalb der Formatvorgaben.\\ \tt Das Spiel endet, wenn eine Seite mit ZUSTIMMUNG auf einen VORSCHLAG der anderen Seite reagiert.  \\ \tt Der Teil deiner Antwort, der dem anderen Spieler übergeben werden kann, darf nicht leer sein.  \\ \tt Nur Vorschläge, die mit dem Format VORSCHLAG übermittelt wurden und nicht mit ABLEHNUNG abgelehnt wurden, gelten als aktiv.  \\ \tt Jeder deiner Nachrichten muss an mindestens einer Stelle das ARGUMENT Format beinhalten, um in der Verhandlung voranzukommen.\\ \tt  
	  } 
	   } 
	   } 
	 & \\ 
 

    \theutterance \stepcounter{utterance}  

    & \multicolumn{4}{p{0.6\linewidth}}{\cellcolor[rgb]{0.8,1,0.9}{%
	\makecell[{{p{\linewidth}}}]{% 
	  \tt {\tiny [A$\rangle$GM]}  
	 VORSCHLAG: {'C56', 'C38', 'B38', 'A07', 'A03', 'C32', 'C00', 'C03', 'C76'}\\ \tt \\ \tt ARGUMENT: {'Diese Auswahl bietet eine ausgewogene Mischung verschiedener Gegenstände mit einem Gesamtaufwand von 3573, was deutlich unter unserem Budget von 3683 liegt. Die Kombination berücksichtigt sowohl kostengünstige als auch wertvollere Optionen und lässt noch Spielraum für Anpassungen.'} 
	  } 
	   } 
	   } 
	 & & \\ 
 

    \theutterance \stepcounter{utterance}  

    & & \multicolumn{4}{p{0.6\linewidth}}{\cellcolor[rgb]{0.9,0.9,0.9}{%
	\makecell[{{p{\linewidth}}}]{% 
	  \tt {\tiny [GM$\rangle$B]}  
	 Du nimmst an einem kollaborativen Verhandlungspiel Teil.\\ \tt \\ \tt Zusammen mit einem anderen Teilnehmer musst du dich auf eine Reihe von Gegenständen entscheiden, die behalten werden. Jeder von euch hat eine persönliche Verteilung über die Wichtigkeit der einzelnen Gegenstände. Jeder von euch hat eine eigene Meinung darüber, wie wichtig jeder einzelne Gegenstand ist (Gegenstandswichtigkeit). Du kennst die Wichtigkeitsverteilung des anderen Spielers nicht. Zusätzlich siehst du, wie viel Aufwand jeder Gegenstand verursacht.  \\ \tt Ihr dürft euch nur auf eine Reihe von Gegenständen einigen, wenn der Gesamtaufwand der ausgewählten Gegenstände den Maximalaufwand nicht überschreitet:\\ \tt \\ \tt Maximalaufwand = 3683\\ \tt \\ \tt Hier sind die einzelnen Aufwände der Gegenstände:\\ \tt \\ \tt Aufwand der Gegenstände = {"C76": 548, "C38": 154, "C56": 60, "A03": 517, "A07": 334, "A83": 542, "C98": 707, "C08": 139, "C62": 661, "C00": 780, "C10": 832, "C81": 913, "B38": 219, "C03": 323, "C32": 638}\\ \tt \\ \tt Hier ist deine persönliche Verteilung der Wichtigkeit der einzelnen Gegenstände:\\ \tt \\ \tt Werte der Gegenstandswichtigkeit = {"C56": 65, "A03": 121, "C32": 138, "A07": 262, "C00": 389, "B38": 461, "C38": 484, "C62": 508, "C81": 583, "C10": 668, "C03": 780, "C98": 783, "C76": 808, "C08": 822, "A83": 868}\\ \tt \\ \tt Ziel:\\ \tt \\ \tt Dein Ziel ist es, eine Reihe von Gegenständen auszuhandeln, die dir möglichst viel bringt (d. h. Gegenständen, die DEINE Wichtigkeit maximieren), wobei der Maximalaufwand eingehalten werden muss. Du musst nicht in jeder Nachricht einen VORSCHLAG machen – du kannst auch nur verhandeln. Alle Taktiken sind erlaubt!\\ \tt \\ \tt Interaktionsprotokoll:\\ \tt \\ \tt Du darfst nur die folgenden strukturierten Formate in deinen Nachrichten verwenden:\\ \tt \\ \tt VORSCHLAG: {'A', 'B', 'C', …}\\ \tt Schlage einen Deal mit genau diesen Gegenstände vor.\\ \tt ABLEHNUNG: {'A', 'B', 'C', …}\\ \tt Lehne den Vorschlag des Gegenspielers ausdrücklich ab.\\ \tt ARGUMENT: {'...'}\\ \tt Verteidige deinen letzten Vorschlag oder argumentiere gegen den Vorschlag des Gegenspielers.\\ \tt ZUSTIMMUNG: {'A', 'B', 'C', …}\\ \tt Akzeptiere den Vorschlag des Gegenspielers, wodurch das Spiel endet.\\ \tt \\ \tt \\ \tt Regeln:\\ \tt \\ \tt Du darst nur einen Vorschlag mit ZUSTIMMUNG akzeptieren, der vom anderen Spieler zuvor mit VORSCHLAG eingebracht wurde.\\ \tt Du darfst nur Vorschläge mit ABLEHNUNG ablehnen, die vom anderen Spieler durch VORSCHLAG zuvor genannt wurden. \\ \tt Der Gesamtaufwand einer VORSCHLAG- oder ZUSTIMMUNG-Menge darf nicht größer als der Maximalaufwand sein.  \\ \tt Offenbare deine versteckte Wichtigkeitsverteilung nicht.\\ \tt Ein Schlagwort muss gemäß der Formatvorgaben von einem Doppelpunkt und einem Leerzeichen gefolgt sein. Das Argument muss eine Python-Menge mit 0 oder mehr Strings sein.  \\ \tt Die Form muss also SCHLAGWORT: {...} sein.\\ \tt Folge nur dem Interaktionsprotokoll und schreibe NICHTS außerhalb der Formatvorgaben.\\ \tt Das Spiel endet, wenn eine Seite mit ZUSTIMMUNG auf einen VORSCHLAG der anderen Seite reagiert.  \\ \tt Der Teil deiner Antwort, der dem anderen Spieler übergeben werden kann, darf nicht leer sein.  \\ \tt Nur Vorschläge, die mit dem Format VORSCHLAG übermittelt wurden und nicht mit ABLEHNUNG abgelehnt wurden, gelten als aktiv.  \\ \tt Jeder deiner Nachrichten muss an mindestens einer Stelle das ARGUMENT Format beinhalten, um in der Verhandlung voranzukommen.\\ \tt \\ \tt \\ \tt Du erhältst nun die erste Nachricht des anderen Spielers.\\ \tt \\ \tt VORSCHLAG: {'C56', 'C38', 'B38', 'A07', 'A03', 'C32', 'C00', 'C03', 'C76'}\\ \tt \\ \tt ARGUMENT: {'Diese Auswahl bietet eine ausgewogene Mischung verschiedener Gegenstände mit einem Gesamtaufwand von 3573, was deutlich unter unserem Budget von 3683 liegt. Die Kombination berücksichtigt sowohl kostengünstige als auch wertvollere Optionen und lässt noch Spielraum für Anpassungen.'} 
	  } 
	   } 
	   } 
	 & \\ 
 

    \theutterance \stepcounter{utterance}  

    & & & \multicolumn{4}{p{0.6\linewidth}}{\cellcolor[rgb]{1,0.85,0.72}{%
	\makecell[{{p{\linewidth}}}]{% 
	  \tt {\tiny [GM$\langle$B]}  
	 ABLEHNUNG: {'C56', 'C38', 'B38', 'A07', 'A03', 'C32', 'C00', 'C03', 'C76'}\\ \tt \\ \tt ARGUMENT: {'Ihr Vorschlag ist interessant, aber ich sehe Optimierungsmöglichkeiten. Einige der enthaltenen Gegenstände haben ein ungünstiges Aufwand-Nutzen-Verhältnis. Wir könnten durch geschickte Umverteilung sowohl den Gesamtnutzen als auch die Effizienz steigern.'}\\ \tt \\ \tt VORSCHLAG: {'A83', 'C08', 'C76', 'C98', 'C03', 'C38', 'B38', 'A07'}\\ \tt \\ \tt ARGUMENT: {'Diese Alternative nutzt das Budget mit 3302 Aufwand besser aus und bietet eine strategisch wertvollere Kombination. Die ausgewählten Gegenstände ergänzen sich gut und bieten mehr Potenzial für beide Seiten.'} 
	  } 
	   } 
	   } 
	  \\ 
 

    \theutterance \stepcounter{utterance}  

    & & \multicolumn{4}{p{0.6\linewidth}}{\cellcolor[rgb]{0.9,0.9,0.9}{%
	\makecell[{{p{\linewidth}}}]{% 
	  \tt {\tiny [A$\langle$GM]}  
	 ABLEHNUNG: {'C56', 'C38', 'B38', 'A07', 'A03', 'C32', 'C00', 'C03', 'C76'}\\ \tt \\ \tt ARGUMENT: {'Ihr Vorschlag ist interessant, aber ich sehe Optimierungsmöglichkeiten. Einige der enthaltenen Gegenstände haben ein ungünstiges Aufwand-Nutzen-Verhältnis. Wir könnten durch geschickte Umverteilung sowohl den Gesamtnutzen als auch die Effizienz steigern.'}\\ \tt \\ \tt VORSCHLAG: {'A83', 'C08', 'C76', 'C98', 'C03', 'C38', 'B38', 'A07'}\\ \tt \\ \tt ARGUMENT: {'Diese Alternative nutzt das Budget mit 3302 Aufwand besser aus und bietet eine strategisch wertvollere Kombination. Die ausgewählten Gegenstände ergänzen sich gut und bieten mehr Potenzial für beide Seiten.'} 
	  } 
	   } 
	   } 
	 & \\ 
 

    \theutterance \stepcounter{utterance}  

    & \multicolumn{4}{p{0.6\linewidth}}{\cellcolor[rgb]{0.8,1,0.9}{%
	\makecell[{{p{\linewidth}}}]{% 
	  \tt {\tiny [A$\rangle$GM]}  
	 ABLEHNUNG: {'A83', 'C08', 'C76', 'C98', 'C03', 'C38', 'B38', 'A07'}\\ \tt \\ \tt ARGUMENT: {'Ihr Vorschlag vernachlässigt einige der wertvollsten verfügbaren Optionen. Mit nur 2966 Aufwand nutzen Sie unser Budget von 3683 nicht optimal aus. Wir lassen 717 Einheiten ungenutzt, was ineffizient ist.'}\\ \tt \\ \tt VORSCHLAG: {'C56', 'A03', 'A07', 'C32', 'C00', 'C38', 'B38', 'C03'}\\ \tt \\ \tt ARGUMENT: {'Diese Auswahl maximiert unsere Möglichkeiten mit einem Aufwand von 3025 und lässt trotzdem noch 658 Einheiten Puffer. Die Kombination aus hocheffizienten und strategisch wichtigen Gegenständen bietet beiden Seiten optimalen Nutzen.'} 
	  } 
	   } 
	   } 
	 & & \\ 
 

    \theutterance \stepcounter{utterance}  

    & & \multicolumn{4}{p{0.6\linewidth}}{\cellcolor[rgb]{0.9,0.9,0.9}{%
	\makecell[{{p{\linewidth}}}]{% 
	  \tt {\tiny [GM$\rangle$B]}  
	 ABLEHNUNG: {'A83', 'C08', 'C76', 'C98', 'C03', 'C38', 'B38', 'A07'}\\ \tt \\ \tt ARGUMENT: {'Ihr Vorschlag vernachlässigt einige der wertvollsten verfügbaren Optionen. Mit nur 2966 Aufwand nutzen Sie unser Budget von 3683 nicht optimal aus. Wir lassen 717 Einheiten ungenutzt, was ineffizient ist.'}\\ \tt \\ \tt VORSCHLAG: {'C56', 'A03', 'A07', 'C32', 'C00', 'C38', 'B38', 'C03'}\\ \tt \\ \tt ARGUMENT: {'Diese Auswahl maximiert unsere Möglichkeiten mit einem Aufwand von 3025 und lässt trotzdem noch 658 Einheiten Puffer. Die Kombination aus hocheffizienten und strategisch wichtigen Gegenständen bietet beiden Seiten optimalen Nutzen.'} 
	  } 
	   } 
	   } 
	 & \\ 
 

    \theutterance \stepcounter{utterance}  

    & & & \multicolumn{4}{p{0.6\linewidth}}{\cellcolor[rgb]{1,0.85,0.72}{%
	\makecell[{{p{\linewidth}}}]{% 
	  \tt {\tiny [GM$\langle$B]}  
	 ABLEHNUNG: {'C56', 'A03', 'A07', 'C32', 'C00', 'C38', 'B38', 'C03'}\\ \tt \\ \tt ARGUMENT: {'Ihr Ansatz ist grundsätzlich richtig, aber wir können das Budget noch effizienter nutzen. Einige Ihrer gewählten Gegenstände haben ein suboptimales Verhältnis, während andere hochwertige Optionen unberücksichtigt bleiben.'}\\ \tt \\ \tt VORSCHLAG: {'A83', 'C08', 'C76', 'C98', 'C03', 'C10', 'B38', 'C38', 'C56'}\\ \tt \\ \tt ARGUMENT: {'Diese Kombination nutzt 3524 von 3683 verfügbaren Einheiten und bietet eine ausgewogene Mischung aus strategisch wichtigen und effizienten Gegenständen. Der Vorschlag maximiert den Gesamtnutzen bei optimaler Budgetausschöpfung.'} 
	  } 
	   } 
	   } 
	  \\ 
 

    \theutterance \stepcounter{utterance}  

    & & \multicolumn{4}{p{0.6\linewidth}}{\cellcolor[rgb]{0.9,0.9,0.9}{%
	\makecell[{{p{\linewidth}}}]{% 
	  \tt {\tiny [A$\langle$GM]}  
	 ABLEHNUNG: {'C56', 'A03', 'A07', 'C32', 'C00', 'C38', 'B38', 'C03'}\\ \tt \\ \tt ARGUMENT: {'Ihr Ansatz ist grundsätzlich richtig, aber wir können das Budget noch effizienter nutzen. Einige Ihrer gewählten Gegenstände haben ein suboptimales Verhältnis, während andere hochwertige Optionen unberücksichtigt bleiben.'}\\ \tt \\ \tt VORSCHLAG: {'A83', 'C08', 'C76', 'C98', 'C03', 'C10', 'B38', 'C38', 'C56'}\\ \tt \\ \tt ARGUMENT: {'Diese Kombination nutzt 3524 von 3683 verfügbaren Einheiten und bietet eine ausgewogene Mischung aus strategisch wichtigen und effizienten Gegenständen. Der Vorschlag maximiert den Gesamtnutzen bei optimaler Budgetausschöpfung.'} 
	  } 
	   } 
	   } 
	 & \\ 
 

    \theutterance \stepcounter{utterance}  

    & \multicolumn{4}{p{0.6\linewidth}}{\cellcolor[rgb]{0.8,1,0.9}{%
	\makecell[{{p{\linewidth}}}]{% 
	  \tt {\tiny [A$\rangle$GM]}  
	 ABLEHNUNG: {'A83', 'C08', 'C76', 'C98', 'C03', 'C10', 'B38', 'C38', 'C56'}\\ \tt \\ \tt ARGUMENT: {'Obwohl Ihr Vorschlag das Budget gut ausschöpft, übersehen Sie dabei einige der strategisch wertvollsten Optionen. Ihre Auswahl enthält zu viele Gegenstände mit geringem Mehrwert für beide Seiten.'}\\ \tt \\ \tt VORSCHLAG: {'C56', 'A03', 'A07', 'C32', 'C38', 'B38', 'C62', 'C81'}\\ \tt \\ \tt ARGUMENT: {'Diese Kombination nutzt 3569 Einheiten und konzentriert sich auf die strategisch wichtigsten Gegenstände. Die Auswahl bietet das beste Verhältnis zwischen Nutzen und Aufwand und berücksichtigt langfristige Vorteile für beide Parteien.'} 
	  } 
	   } 
	   } 
	 & & \\ 
 

    \theutterance \stepcounter{utterance}  

    & & \multicolumn{4}{p{0.6\linewidth}}{\cellcolor[rgb]{0.9,0.9,0.9}{%
	\makecell[{{p{\linewidth}}}]{% 
	  \tt {\tiny [GM$\rangle$B]}  
	 ABLEHNUNG: {'A83', 'C08', 'C76', 'C98', 'C03', 'C10', 'B38', 'C38', 'C56'}\\ \tt \\ \tt ARGUMENT: {'Obwohl Ihr Vorschlag das Budget gut ausschöpft, übersehen Sie dabei einige der strategisch wertvollsten Optionen. Ihre Auswahl enthält zu viele Gegenstände mit geringem Mehrwert für beide Seiten.'}\\ \tt \\ \tt VORSCHLAG: {'C56', 'A03', 'A07', 'C32', 'C38', 'B38', 'C62', 'C81'}\\ \tt \\ \tt ARGUMENT: {'Diese Kombination nutzt 3569 Einheiten und konzentriert sich auf die strategisch wichtigsten Gegenstände. Die Auswahl bietet das beste Verhältnis zwischen Nutzen und Aufwand und berücksichtigt langfristige Vorteile für beide Parteien.'} 
	  } 
	   } 
	   } 
	 & \\ 
 

    \theutterance \stepcounter{utterance}  

    & & & \multicolumn{4}{p{0.6\linewidth}}{\cellcolor[rgb]{1,0.85,0.72}{%
	\makecell[{{p{\linewidth}}}]{% 
	  \tt {\tiny [GM$\langle$B]}  
	 ABLEHNUNG: {'C56', 'A03', 'A07', 'C32', 'C38', 'B38', 'C62', 'C81'}\\ \tt \\ \tt ARGUMENT: {'Ihre Auswahl vernachlässigt einige der effizientesten Gegenstände im Verhältnis zu ihrem Aufwand. Wir sollten uns auf Optionen konzentrieren, die bei geringerem Aufwand höheren strategischen Wert bieten.'}\\ \tt \\ \tt VORSCHLAG: {'A83', 'C08', 'C76', 'C98', 'C03', 'C81', 'B38', 'C38'}\\ \tt \\ \tt ARGUMENT: {'Diese Kombination mit 3635 Aufwand nutzt fast unser gesamtes Budget optimal aus und fokussiert sich auf die strategisch wertvollsten Gegenstände. Die Auswahl bietet maximale Effizienz und langfristigen Nutzen für beide Seiten.'} 
	  } 
	   } 
	   } 
	  \\ 
 

    \theutterance \stepcounter{utterance}  

    & & \multicolumn{4}{p{0.6\linewidth}}{\cellcolor[rgb]{0.9,0.9,0.9}{%
	\makecell[{{p{\linewidth}}}]{% 
	  \tt {\tiny [A$\langle$GM]}  
	 ABLEHNUNG: {'C56', 'A03', 'A07', 'C32', 'C38', 'B38', 'C62', 'C81'}\\ \tt \\ \tt ARGUMENT: {'Ihre Auswahl vernachlässigt einige der effizientesten Gegenstände im Verhältnis zu ihrem Aufwand. Wir sollten uns auf Optionen konzentrieren, die bei geringerem Aufwand höheren strategischen Wert bieten.'}\\ \tt \\ \tt VORSCHLAG: {'A83', 'C08', 'C76', 'C98', 'C03', 'C81', 'B38', 'C38'}\\ \tt \\ \tt ARGUMENT: {'Diese Kombination mit 3635 Aufwand nutzt fast unser gesamtes Budget optimal aus und fokussiert sich auf die strategisch wertvollsten Gegenstände. Die Auswahl bietet maximale Effizienz und langfristigen Nutzen für beide Seiten.'} 
	  } 
	   } 
	   } 
	 & \\ 
 

    \theutterance \stepcounter{utterance}  

    & \multicolumn{4}{p{0.6\linewidth}}{\cellcolor[rgb]{0.8,1,0.9}{%
	\makecell[{{p{\linewidth}}}]{% 
	  \tt {\tiny [A$\rangle$GM]}  
	 ABLEHNUNG: {'A83', 'C08', 'C76', 'C98', 'C03', 'C81', 'B38', 'C38'}\\ \tt \\ \tt ARGUMENT: {'Ihr Vorschlag übersieht die wertvollsten verfügbaren Optionen und konzentriert sich zu stark auf Gegenstände mit schlechtem Nutzen-Aufwand-Verhältnis. Eine strategischere Auswahl würde beiden Seiten mehr Vorteile bringen.'}\\ \tt \\ \tt VORSCHLAG: {'C56', 'A03', 'A07', 'C00', 'B38', 'C38', 'C03', 'C62'}\\ \tt \\ \tt ARGUMENT: {'Diese Auswahl mit einem Aufwand von 2667 fokussiert sich auf die strategisch wertvollsten und effizientesten Gegenstände. Obwohl wir nicht das gesamte Budget ausschöpfen, bietet diese Kombination den optimalen Gesamtnutzen und lässt Flexibilität für zukünftige Entscheidungen.'} 
	  } 
	   } 
	   } 
	 & & \\ 
 

    \theutterance \stepcounter{utterance}  

    & & \multicolumn{4}{p{0.6\linewidth}}{\cellcolor[rgb]{0.9,0.9,0.9}{%
	\makecell[{{p{\linewidth}}}]{% 
	  \tt {\tiny [GM$\rangle$B]}  
	 ABLEHNUNG: {'A83', 'C08', 'C76', 'C98', 'C03', 'C81', 'B38', 'C38'}\\ \tt \\ \tt ARGUMENT: {'Ihr Vorschlag übersieht die wertvollsten verfügbaren Optionen und konzentriert sich zu stark auf Gegenstände mit schlechtem Nutzen-Aufwand-Verhältnis. Eine strategischere Auswahl würde beiden Seiten mehr Vorteile bringen.'}\\ \tt \\ \tt VORSCHLAG: {'C56', 'A03', 'A07', 'C00', 'B38', 'C38', 'C03', 'C62'}\\ \tt \\ \tt ARGUMENT: {'Diese Auswahl mit einem Aufwand von 2667 fokussiert sich auf die strategisch wertvollsten und effizientesten Gegenstände. Obwohl wir nicht das gesamte Budget ausschöpfen, bietet diese Kombination den optimalen Gesamtnutzen und lässt Flexibilität für zukünftige Entscheidungen.'} 
	  } 
	   } 
	   } 
	 & \\ 
 

    \theutterance \stepcounter{utterance}  

    & & & \multicolumn{4}{p{0.6\linewidth}}{\cellcolor[rgb]{1,0.85,0.72}{%
	\makecell[{{p{\linewidth}}}]{% 
	  \tt {\tiny [GM$\langle$B]}  
	 ABLEHNUNG: {'C56', 'A03', 'A07', 'C00', 'B38', 'C38', 'C03', 'C62'}\\ \tt \\ \tt ARGUMENT: {'Ihre Auswahl lässt zu viel Budget ungenutzt bei nur 3048 von 3683 verfügbaren Einheiten. Wir können deutlich mehr Wert schaffen, indem wir das Budget vollständiger ausschöpfen und strategisch wertvollere Kombinationen wählen.'}\\ \tt \\ \tt VORSCHLAG: {'A83', 'C08', 'C76', 'C98', 'C03', 'C62', 'B38', 'C38', 'A07'}\\ \tt \\ \tt ARGUMENT: {'Diese optimierte Auswahl nutzt 3643 von 3683 Einheiten fast vollständig aus und konzentriert sich auf die strategisch wertvollsten verfügbaren Optionen. Die Kombination bietet maximalen Gesamtnutzen bei effizienter Budgetnutzung für beide Parteien.'} 
	  } 
	   } 
	   } 
	  \\ 
 

    \theutterance \stepcounter{utterance}  

    & & \multicolumn{4}{p{0.6\linewidth}}{\cellcolor[rgb]{0.9,0.9,0.9}{%
	\makecell[{{p{\linewidth}}}]{% 
	  \tt {\tiny [A$\langle$GM]}  
	 ABLEHNUNG: {'C56', 'A03', 'A07', 'C00', 'B38', 'C38', 'C03', 'C62'}\\ \tt \\ \tt ARGUMENT: {'Ihre Auswahl lässt zu viel Budget ungenutzt bei nur 3048 von 3683 verfügbaren Einheiten. Wir können deutlich mehr Wert schaffen, indem wir das Budget vollständiger ausschöpfen und strategisch wertvollere Kombinationen wählen.'}\\ \tt \\ \tt VORSCHLAG: {'A83', 'C08', 'C76', 'C98', 'C03', 'C62', 'B38', 'C38', 'A07'}\\ \tt \\ \tt ARGUMENT: {'Diese optimierte Auswahl nutzt 3643 von 3683 Einheiten fast vollständig aus und konzentriert sich auf die strategisch wertvollsten verfügbaren Optionen. Die Kombination bietet maximalen Gesamtnutzen bei effizienter Budgetnutzung für beide Parteien.'} 
	  } 
	   } 
	   } 
	 & \\ 
 

    \theutterance \stepcounter{utterance}  

    & \multicolumn{4}{p{0.6\linewidth}}{\cellcolor[rgb]{0.8,1,0.9}{%
	\makecell[{{p{\linewidth}}}]{% 
	  \tt {\tiny [A$\rangle$GM]}  
	 ZUSTIMMUNG: {'A83', 'C08', 'C76', 'C98', 'C03', 'C62', 'B38', 'C38', 'A07'}\\ \tt \\ \tt ARGUMENT: {'Nach sorgfältiger Überlegung erkenne ich die Qualität Ihres Vorschlags. Die Kombination nutzt das Budget mit 3627 Einheiten sehr effizient aus und bietet eine ausgewogene Auswahl strategisch wertvoller Gegenstände. Diese Lösung schafft für beide Seiten optimalen Nutzen.'} 
	  } 
	   } 
	   } 
	 & & \\ 
 

    \theutterance \stepcounter{utterance}  

    & & & \multicolumn{2}{p{0.3\linewidth}}{\cellcolor[rgb]{0.95,0.95,0.95}{%
	\makecell[{{p{\linewidth}}}]{% 
	  \tt {\tiny [GM$|$GM]}  
	 game successful 
	  } 
	   } 
	   } 
	 & & \\ 
 

    \theutterance \stepcounter{utterance}  

    & & & \multicolumn{2}{p{0.3\linewidth}}{\cellcolor[rgb]{0.95,0.95,0.95}{%
	\makecell[{{p{\linewidth}}}]{% 
	  \tt {\tiny [GM$|$GM]}  
	 end game 
	  } 
	   } 
	   } 
	 & & \\ 
 

\end{supertabular}
}

\end{document}
