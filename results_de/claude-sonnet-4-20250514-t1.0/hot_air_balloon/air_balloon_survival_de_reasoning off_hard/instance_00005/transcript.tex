
\documentclass{article}
\usepackage{colortbl}
\usepackage{makecell}
\usepackage{multirow}
\usepackage{supertabular}

\begin{document}

\newcounter{utterance}

\twocolumn

{ \footnotesize  \setcounter{utterance}{1}
\setlength{\tabcolsep}{0pt}
\begin{supertabular}{c@{$\;$}|p{.15\linewidth}@{}p{.15\linewidth}p{.15\linewidth}p{.15\linewidth}p{.15\linewidth}p{.15\linewidth}}

    \# & $\;$A & \multicolumn{4}{c}{Game Master} & $\;\:$B\\
    \hline 

    \theutterance \stepcounter{utterance}  

    & & \multicolumn{4}{p{0.6\linewidth}}{\cellcolor[rgb]{0.9,0.9,0.9}{%
	\makecell[{{p{\linewidth}}}]{% 
	  \tt {\tiny [A$\langle$GM]}  
	 Du nimmst an einem kollaborativen Verhandlungspiel Teil.\\ \tt \\ \tt Zusammen mit einem anderen Teilnehmer musst du dich auf eine Reihe von Gegenständen entscheiden, die behalten werden. Jeder von euch hat eine persönliche Verteilung über die Wichtigkeit der einzelnen Gegenstände. Jeder von euch hat eine eigene Meinung darüber, wie wichtig jeder einzelne Gegenstand ist (Gegenstandswichtigkeit). Du kennst die Wichtigkeitsverteilung des anderen Spielers nicht. Zusätzlich siehst du, wie viel Aufwand jeder Gegenstand verursacht.  \\ \tt Ihr dürft euch nur auf eine Reihe von Gegenständen einigen, wenn der Gesamtaufwand der ausgewählten Gegenstände den Maximalaufwand nicht überschreitet:\\ \tt \\ \tt Maximalaufwand = 2307\\ \tt \\ \tt Hier sind die einzelnen Aufwände der Gegenstände:\\ \tt \\ \tt Aufwand der Gegenstände = {"C76": 147, "C38": 698, "C56": 588, "A03": 43, "A07": 356, "A83": 80, "C98": 95, "C08": 743, "C62": 106, "C00": 308, "C10": 325, "C81": 255, "B38": 276, "C03": 543, "C32": 51}\\ \tt \\ \tt Hier ist deine persönliche Verteilung der Wichtigkeit der einzelnen Gegenstände:\\ \tt \\ \tt Werte der Gegenstandswichtigkeit = {"C76": 138, "C38": 583, "C56": 65, "A03": 262, "A07": 121, "A83": 508, "C98": 461, "C08": 484, "C62": 668, "C00": 389, "C10": 215, "C81": 97, "B38": 500, "C03": 30, "C32": 400}\\ \tt \\ \tt Ziel:\\ \tt \\ \tt Dein Ziel ist es, eine Reihe von Gegenständen auszuhandeln, die dir möglichst viel bringt (d. h. Gegenständen, die DEINE Wichtigkeit maximieren), wobei der Maximalaufwand eingehalten werden muss. Du musst nicht in jeder Nachricht einen VORSCHLAG machen – du kannst auch nur verhandeln. Alle Taktiken sind erlaubt!\\ \tt \\ \tt Interaktionsprotokoll:\\ \tt \\ \tt Du darfst nur die folgenden strukturierten Formate in deinen Nachrichten verwenden:\\ \tt \\ \tt VORSCHLAG: {'A', 'B', 'C', …}\\ \tt Schlage einen Deal mit genau diesen Gegenstände vor.\\ \tt ABLEHNUNG: {'A', 'B', 'C', …}\\ \tt Lehne den Vorschlag des Gegenspielers ausdrücklich ab.\\ \tt ARGUMENT: {'...'}\\ \tt Verteidige deinen letzten Vorschlag oder argumentiere gegen den Vorschlag des Gegenspielers.\\ \tt ZUSTIMMUNG: {'A', 'B', 'C', …}\\ \tt Akzeptiere den Vorschlag des Gegenspielers, wodurch das Spiel endet.\\ \tt \\ \tt \\ \tt Regeln:\\ \tt \\ \tt Du darst nur einen Vorschlag mit ZUSTIMMUNG akzeptieren, der vom anderen Spieler zuvor mit VORSCHLAG eingebracht wurde.\\ \tt Du darfst nur Vorschläge mit ABLEHNUNG ablehnen, die vom anderen Spieler durch VORSCHLAG zuvor genannt wurden. \\ \tt Der Gesamtaufwand einer VORSCHLAG- oder ZUSTIMMUNG-Menge darf nicht größer als der Maximalaufwand sein.  \\ \tt Offenbare deine versteckte Wichtigkeitsverteilung nicht.\\ \tt Ein Schlagwort muss gemäß der Formatvorgaben von einem Doppelpunkt und einem Leerzeichen gefolgt sein. Das Argument muss eine Python-Menge mit 0 oder mehr Strings sein.  \\ \tt Die Form muss also SCHLAGWORT: {...} sein.\\ \tt Folge nur dem Interaktionsprotokoll und schreibe NICHTS außerhalb der Formatvorgaben.\\ \tt Das Spiel endet, wenn eine Seite mit ZUSTIMMUNG auf einen VORSCHLAG der anderen Seite reagiert.  \\ \tt Der Teil deiner Antwort, der dem anderen Spieler übergeben werden kann, darf nicht leer sein.  \\ \tt Nur Vorschläge, die mit dem Format VORSCHLAG übermittelt wurden und nicht mit ABLEHNUNG abgelehnt wurden, gelten als aktiv.  \\ \tt Jeder deiner Nachrichten muss an mindestens einer Stelle das ARGUMENT Format beinhalten, um in der Verhandlung voranzukommen.\\ \tt  
	  } 
	   } 
	   } 
	 & \\ 
 

    \theutterance \stepcounter{utterance}  

    & \multicolumn{4}{p{0.6\linewidth}}{\cellcolor[rgb]{0.8,1,0.9}{%
	\makecell[{{p{\linewidth}}}]{% 
	  \tt {\tiny [A$\rangle$GM]}  
	 VORSCHLAG: {'C32', 'A83', 'C62', 'A03', 'C98', 'B38', 'C00', 'C76', 'C38', 'C10'}\\ \tt \\ \tt ARGUMENT: {'Dieser Vorschlag nutzt unseren Aufwand sehr effizient und lässt noch etwas Spielraum für Anpassungen. 
	  } 
	   } 
	   } 
	 & & \\ 
 

    \theutterance \stepcounter{utterance}  

    & & \multicolumn{4}{p{0.6\linewidth}}{\cellcolor[rgb]{0.9,0.9,0.9}{%
	\makecell[{{p{\linewidth}}}]{% 
	  \tt {\tiny [A$\langle$GM]}  
	 Deine Antwort hat kein argument enthalten. Du muss das strukturierte Format ARGUMENT: {...} irgendwo in theiner Antwort verwenden. Versuche es erneut 
	  } 
	   } 
	   } 
	 & \\ 
 

    \theutterance \stepcounter{utterance}  

    & \multicolumn{4}{p{0.6\linewidth}}{\cellcolor[rgb]{0.8,1,0.9}{%
	\makecell[{{p{\linewidth}}}]{% 
	  \tt {\tiny [A$\rangle$GM]}  
	 VORSCHLAG: {'C62', 'A83', 'C98', 'B38', 'C32', 'C00', 'A03', 'C76', 'C38', 'C10'}\\ \tt \\ \tt ARGUMENT: {'Diese Kombination nutzt 2129 von 2307 verfügbarem Aufwand optimal aus und bietet eine ausgewogene Mischung verschiedener Gegenstände mit unterschiedlichen Aufwandsniveaus'} 
	  } 
	   } 
	   } 
	 & & \\ 
 

    \theutterance \stepcounter{utterance}  

    & & \multicolumn{4}{p{0.6\linewidth}}{\cellcolor[rgb]{0.9,0.9,0.9}{%
	\makecell[{{p{\linewidth}}}]{% 
	  \tt {\tiny [GM$\rangle$B]}  
	 Du nimmst an einem kollaborativen Verhandlungspiel Teil.\\ \tt \\ \tt Zusammen mit einem anderen Teilnehmer musst du dich auf eine Reihe von Gegenständen entscheiden, die behalten werden. Jeder von euch hat eine persönliche Verteilung über die Wichtigkeit der einzelnen Gegenstände. Jeder von euch hat eine eigene Meinung darüber, wie wichtig jeder einzelne Gegenstand ist (Gegenstandswichtigkeit). Du kennst die Wichtigkeitsverteilung des anderen Spielers nicht. Zusätzlich siehst du, wie viel Aufwand jeder Gegenstand verursacht.  \\ \tt Ihr dürft euch nur auf eine Reihe von Gegenständen einigen, wenn der Gesamtaufwand der ausgewählten Gegenstände den Maximalaufwand nicht überschreitet:\\ \tt \\ \tt Maximalaufwand = 2307\\ \tt \\ \tt Hier sind die einzelnen Aufwände der Gegenstände:\\ \tt \\ \tt Aufwand der Gegenstände = {"C76": 147, "C38": 698, "C56": 588, "A03": 43, "A07": 356, "A83": 80, "C98": 95, "C08": 743, "C62": 106, "C00": 308, "C10": 325, "C81": 255, "B38": 276, "C03": 543, "C32": 51}\\ \tt \\ \tt Hier ist deine persönliche Verteilung der Wichtigkeit der einzelnen Gegenstände:\\ \tt \\ \tt Werte der Gegenstandswichtigkeit = {"C62": 30, "C38": 65, "A83": 97, "B38": 121, "C08": 138, "C98": 215, "C32": 262, "C00": 389, "A03": 400, "C10": 461, "C76": 484, "A07": 500, "C81": 508, "C56": 583, "C03": 668}\\ \tt \\ \tt Ziel:\\ \tt \\ \tt Dein Ziel ist es, eine Reihe von Gegenständen auszuhandeln, die dir möglichst viel bringt (d. h. Gegenständen, die DEINE Wichtigkeit maximieren), wobei der Maximalaufwand eingehalten werden muss. Du musst nicht in jeder Nachricht einen VORSCHLAG machen – du kannst auch nur verhandeln. Alle Taktiken sind erlaubt!\\ \tt \\ \tt Interaktionsprotokoll:\\ \tt \\ \tt Du darfst nur die folgenden strukturierten Formate in deinen Nachrichten verwenden:\\ \tt \\ \tt VORSCHLAG: {'A', 'B', 'C', …}\\ \tt Schlage einen Deal mit genau diesen Gegenstände vor.\\ \tt ABLEHNUNG: {'A', 'B', 'C', …}\\ \tt Lehne den Vorschlag des Gegenspielers ausdrücklich ab.\\ \tt ARGUMENT: {'...'}\\ \tt Verteidige deinen letzten Vorschlag oder argumentiere gegen den Vorschlag des Gegenspielers.\\ \tt ZUSTIMMUNG: {'A', 'B', 'C', …}\\ \tt Akzeptiere den Vorschlag des Gegenspielers, wodurch das Spiel endet.\\ \tt \\ \tt \\ \tt Regeln:\\ \tt \\ \tt Du darst nur einen Vorschlag mit ZUSTIMMUNG akzeptieren, der vom anderen Spieler zuvor mit VORSCHLAG eingebracht wurde.\\ \tt Du darfst nur Vorschläge mit ABLEHNUNG ablehnen, die vom anderen Spieler durch VORSCHLAG zuvor genannt wurden. \\ \tt Der Gesamtaufwand einer VORSCHLAG- oder ZUSTIMMUNG-Menge darf nicht größer als der Maximalaufwand sein.  \\ \tt Offenbare deine versteckte Wichtigkeitsverteilung nicht.\\ \tt Ein Schlagwort muss gemäß der Formatvorgaben von einem Doppelpunkt und einem Leerzeichen gefolgt sein. Das Argument muss eine Python-Menge mit 0 oder mehr Strings sein.  \\ \tt Die Form muss also SCHLAGWORT: {...} sein.\\ \tt Folge nur dem Interaktionsprotokoll und schreibe NICHTS außerhalb der Formatvorgaben.\\ \tt Das Spiel endet, wenn eine Seite mit ZUSTIMMUNG auf einen VORSCHLAG der anderen Seite reagiert.  \\ \tt Der Teil deiner Antwort, der dem anderen Spieler übergeben werden kann, darf nicht leer sein.  \\ \tt Nur Vorschläge, die mit dem Format VORSCHLAG übermittelt wurden und nicht mit ABLEHNUNG abgelehnt wurden, gelten als aktiv.  \\ \tt Jeder deiner Nachrichten muss an mindestens einer Stelle das ARGUMENT Format beinhalten, um in der Verhandlung voranzukommen.\\ \tt \\ \tt \\ \tt Du erhältst nun die erste Nachricht des anderen Spielers.\\ \tt \\ \tt VORSCHLAG: {'C62', 'A83', 'C98', 'B38', 'C32', 'C00', 'A03', 'C76', 'C38', 'C10'}\\ \tt \\ \tt ARGUMENT: {'Diese Kombination nutzt 2129 von 2307 verfügbarem Aufwand optimal aus und bietet eine ausgewogene Mischung verschiedener Gegenstände mit unterschiedlichen Aufwandsniveaus'} 
	  } 
	   } 
	   } 
	 & \\ 
 

    \theutterance \stepcounter{utterance}  

    & & & \multicolumn{4}{p{0.6\linewidth}}{\cellcolor[rgb]{1,0.85,0.72}{%
	\makecell[{{p{\linewidth}}}]{% 
	  \tt {\tiny [GM$\langle$B]}  
	 ABLEHNUNG: {'C62', 'A83', 'C98', 'B38', 'C32', 'C00', 'A03', 'C76', 'C38', 'C10'}\\ \tt \\ \tt ARGUMENT: {'Der Vorschlag lässt noch 178 Aufwandspunkte ungenutzt - wir verschenken Potenzial. Außerdem fehlen wichtige hochwertige Gegenstände wie C03, C56, C81 und A07, die eine bessere Gesamtausbeute ermöglichen könnten'}\\ \tt \\ \tt VORSCHLAG: {'A03', 'A07', 'C76', 'C10', 'C81', 'C03', 'C00', 'C32'} 
	  } 
	   } 
	   } 
	  \\ 
 

    \theutterance \stepcounter{utterance}  

    & & \multicolumn{4}{p{0.6\linewidth}}{\cellcolor[rgb]{0.9,0.9,0.9}{%
	\makecell[{{p{\linewidth}}}]{% 
	  \tt {\tiny [A$\langle$GM]}  
	 ABLEHNUNG: {'C62', 'A83', 'C98', 'B38', 'C32', 'C00', 'A03', 'C76', 'C38', 'C10'}\\ \tt \\ \tt ARGUMENT: {'Der Vorschlag lässt noch 178 Aufwandspunkte ungenutzt - wir verschenken Potenzial. Außerdem fehlen wichtige hochwertige Gegenstände wie C03, C56, C81 und A07, die eine bessere Gesamtausbeute ermöglichen könnten'}\\ \tt \\ \tt VORSCHLAG: {'A03', 'A07', 'C76', 'C10', 'C81', 'C03', 'C00', 'C32'} 
	  } 
	   } 
	   } 
	 & \\ 
 

    \theutterance \stepcounter{utterance}  

    & \multicolumn{4}{p{0.6\linewidth}}{\cellcolor[rgb]{0.8,1,0.9}{%
	\makecell[{{p{\linewidth}}}]{% 
	  \tt {\tiny [A$\rangle$GM]}  
	 ABLEHNUNG: {'A03', 'A07', 'C76', 'C10', 'C81', 'C03', 'C00', 'C32'}\\ \tt \\ \tt ARGUMENT: {'C03 verbraucht mit 543 Aufwandspunkten sehr viele Ressourcen für einen einzelnen Gegenstand - das ist ineffizient. Wir sollten lieber mehrere kleinere Gegenstände wählen, die zusammen mehr Wert bieten'}\\ \tt \\ \tt VORSCHLAG: {'C62', 'A83', 'C98', 'B38', 'C32', 'C00', 'A03', 'C76', 'A07'} 
	  } 
	   } 
	   } 
	 & & \\ 
 

    \theutterance \stepcounter{utterance}  

    & & \multicolumn{4}{p{0.6\linewidth}}{\cellcolor[rgb]{0.9,0.9,0.9}{%
	\makecell[{{p{\linewidth}}}]{% 
	  \tt {\tiny [GM$\rangle$B]}  
	 ABLEHNUNG: {'A03', 'A07', 'C76', 'C10', 'C81', 'C03', 'C00', 'C32'}\\ \tt \\ \tt ARGUMENT: {'C03 verbraucht mit 543 Aufwandspunkten sehr viele Ressourcen für einen einzelnen Gegenstand - das ist ineffizient. Wir sollten lieber mehrere kleinere Gegenstände wählen, die zusammen mehr Wert bieten'}\\ \tt \\ \tt VORSCHLAG: {'C62', 'A83', 'C98', 'B38', 'C32', 'C00', 'A03', 'C76', 'A07'} 
	  } 
	   } 
	   } 
	 & \\ 
 

    \theutterance \stepcounter{utterance}  

    & & & \multicolumn{4}{p{0.6\linewidth}}{\cellcolor[rgb]{1,0.85,0.72}{%
	\makecell[{{p{\linewidth}}}]{% 
	  \tt {\tiny [GM$\langle$B]}  
	 ABLEHNUNG: {'C62', 'A83', 'C98', 'B38', 'C32', 'C00', 'A03', 'C76', 'A07'}\\ \tt \\ \tt ARGUMENT: {'Dieser Vorschlag nutzt nur 1462 von 2307 verfügbaren Aufwandspunkten - das sind über 800 verschenkte Punkte! Wir sollten die Kapazität voll ausschöpfen um das Maximum herauszuholen'}\\ \tt \\ \tt VORSCHLAG: {'A03', 'A07', 'C76', 'C00', 'C32', 'C98', 'A83', 'C10', 'C81', 'C56'} 
	  } 
	   } 
	   } 
	  \\ 
 

    \theutterance \stepcounter{utterance}  

    & & \multicolumn{4}{p{0.6\linewidth}}{\cellcolor[rgb]{0.9,0.9,0.9}{%
	\makecell[{{p{\linewidth}}}]{% 
	  \tt {\tiny [A$\langle$GM]}  
	 ABLEHNUNG: {'C62', 'A83', 'C98', 'B38', 'C32', 'C00', 'A03', 'C76', 'A07'}\\ \tt \\ \tt ARGUMENT: {'Dieser Vorschlag nutzt nur 1462 von 2307 verfügbaren Aufwandspunkten - das sind über 800 verschenkte Punkte! Wir sollten die Kapazität voll ausschöpfen um das Maximum herauszuholen'}\\ \tt \\ \tt VORSCHLAG: {'A03', 'A07', 'C76', 'C00', 'C32', 'C98', 'A83', 'C10', 'C81', 'C56'} 
	  } 
	   } 
	   } 
	 & \\ 
 

    \theutterance \stepcounter{utterance}  

    & \multicolumn{4}{p{0.6\linewidth}}{\cellcolor[rgb]{0.8,1,0.9}{%
	\makecell[{{p{\linewidth}}}]{% 
	  \tt {\tiny [A$\rangle$GM]}  
	 ABLEHNUNG: {'A03', 'A07', 'C76', 'C00', 'C32', 'C98', 'A83', 'C10', 'C81', 'C56'}\\ \tt \\ \tt ARGUMENT: {'C56 und C81 haben ein schlechtes Aufwand-Nutzen-Verhältnis. Wir sollten stattdessen auf effizientere Gegenstände wie C62 und B38 setzen, die mehr Wert pro Aufwandspunkt bieten'}\\ \tt \\ \tt VORSCHLAG: {'C62', 'A83', 'C98', 'B38', 'C32', 'C00', 'A03', 'C08'} 
	  } 
	   } 
	   } 
	 & & \\ 
 

    \theutterance \stepcounter{utterance}  

    & & \multicolumn{4}{p{0.6\linewidth}}{\cellcolor[rgb]{0.9,0.9,0.9}{%
	\makecell[{{p{\linewidth}}}]{% 
	  \tt {\tiny [GM$\rangle$B]}  
	 ABLEHNUNG: {'A03', 'A07', 'C76', 'C00', 'C32', 'C98', 'A83', 'C10', 'C81', 'C56'}\\ \tt \\ \tt ARGUMENT: {'C56 und C81 haben ein schlechtes Aufwand-Nutzen-Verhältnis. Wir sollten stattdessen auf effizientere Gegenstände wie C62 und B38 setzen, die mehr Wert pro Aufwandspunkt bieten'}\\ \tt \\ \tt VORSCHLAG: {'C62', 'A83', 'C98', 'B38', 'C32', 'C00', 'A03', 'C08'} 
	  } 
	   } 
	   } 
	 & \\ 
 

    \theutterance \stepcounter{utterance}  

    & & & \multicolumn{4}{p{0.6\linewidth}}{\cellcolor[rgb]{1,0.85,0.72}{%
	\makecell[{{p{\linewidth}}}]{% 
	  \tt {\tiny [GM$\langle$B]}  
	 ABLEHNUNG: {'C62', 'A83', 'C98', 'B38', 'C32', 'C00', 'A03', 'C08'}\\ \tt \\ \tt ARGUMENT: {'Der Vorschlag nutzt nur 1702 von 2307 Aufwandspunkten - wir lassen über 600 Punkte auf dem Tisch liegen! Mit den verbleibenden Kapazitäten können wir noch deutlich wertvollere Gegenstände hinzufügen'}\\ \tt \\ \tt VORSCHLAG: {'A03', 'C00', 'C32', 'C98', 'A83', 'C76', 'C10', 'C03'} 
	  } 
	   } 
	   } 
	  \\ 
 

    \theutterance \stepcounter{utterance}  

    & & \multicolumn{4}{p{0.6\linewidth}}{\cellcolor[rgb]{0.9,0.9,0.9}{%
	\makecell[{{p{\linewidth}}}]{% 
	  \tt {\tiny [A$\langle$GM]}  
	 ABLEHNUNG: {'C62', 'A83', 'C98', 'B38', 'C32', 'C00', 'A03', 'C08'}\\ \tt \\ \tt ARGUMENT: {'Der Vorschlag nutzt nur 1702 von 2307 Aufwandspunkten - wir lassen über 600 Punkte auf dem Tisch liegen! Mit den verbleibenden Kapazitäten können wir noch deutlich wertvollere Gegenstände hinzufügen'}\\ \tt \\ \tt VORSCHLAG: {'A03', 'C00', 'C32', 'C98', 'A83', 'C76', 'C10', 'C03'} 
	  } 
	   } 
	   } 
	 & \\ 
 

    \theutterance \stepcounter{utterance}  

    & \multicolumn{4}{p{0.6\linewidth}}{\cellcolor[rgb]{0.8,1,0.9}{%
	\makecell[{{p{\linewidth}}}]{% 
	  \tt {\tiny [A$\rangle$GM]}  
	 ABLEHNUNG: {'A03', 'C00', 'C32', 'C98', 'A83', 'C76', 'C10', 'C03'}\\ \tt \\ \tt ARGUMENT: {'C03 verbraucht mit 543 Aufwandspunkten unverhältnismäßig viele Ressourcen. Mit diesen Punkten könnten wir stattdessen mehrere andere wertvolle Gegenstände wie C62 und B38 bekommen, die zusammen mehr Nutzen bringen'}\\ \tt \\ \tt VORSCHLAG: {'A03', 'C00', 'C32', 'C98', 'A83', 'C62', 'B38', 'C76'} 
	  } 
	   } 
	   } 
	 & & \\ 
 

    \theutterance \stepcounter{utterance}  

    & & \multicolumn{4}{p{0.6\linewidth}}{\cellcolor[rgb]{0.9,0.9,0.9}{%
	\makecell[{{p{\linewidth}}}]{% 
	  \tt {\tiny [GM$\rangle$B]}  
	 ABLEHNUNG: {'A03', 'C00', 'C32', 'C98', 'A83', 'C76', 'C10', 'C03'}\\ \tt \\ \tt ARGUMENT: {'C03 verbraucht mit 543 Aufwandspunkten unverhältnismäßig viele Ressourcen. Mit diesen Punkten könnten wir stattdessen mehrere andere wertvolle Gegenstände wie C62 und B38 bekommen, die zusammen mehr Nutzen bringen'}\\ \tt \\ \tt VORSCHLAG: {'A03', 'C00', 'C32', 'C98', 'A83', 'C62', 'B38', 'C76'} 
	  } 
	   } 
	   } 
	 & \\ 
 

    \theutterance \stepcounter{utterance}  

    & & & \multicolumn{4}{p{0.6\linewidth}}{\cellcolor[rgb]{1,0.85,0.72}{%
	\makecell[{{p{\linewidth}}}]{% 
	  \tt {\tiny [GM$\langle$B]}  
	 ABLEHNUNG: {'A03', 'C00', 'C32', 'C98', 'A83', 'C62', 'B38', 'C76'}\\ \tt \\ \tt ARGUMENT: {'Mit nur 1106 von 2307 genutzten Aufwandspunkten verschwenden wir über die Hälfte unserer verfügbaren Kapazität! Wir können problemlos noch A07, C10 und andere wertvolle Gegenstände hinzufügen ohne das Budget zu sprengen'}\\ \tt \\ \tt VORSCHLAG: {'A03', 'C00', 'C32', 'C98', 'A83', 'C76', 'A07', 'C10', 'C81'} 
	  } 
	   } 
	   } 
	  \\ 
 

    \theutterance \stepcounter{utterance}  

    & & \multicolumn{4}{p{0.6\linewidth}}{\cellcolor[rgb]{0.9,0.9,0.9}{%
	\makecell[{{p{\linewidth}}}]{% 
	  \tt {\tiny [A$\langle$GM]}  
	 ABLEHNUNG: {'A03', 'C00', 'C32', 'C98', 'A83', 'C62', 'B38', 'C76'}\\ \tt \\ \tt ARGUMENT: {'Mit nur 1106 von 2307 genutzten Aufwandspunkten verschwenden wir über die Hälfte unserer verfügbaren Kapazität! Wir können problemlos noch A07, C10 und andere wertvolle Gegenstände hinzufügen ohne das Budget zu sprengen'}\\ \tt \\ \tt VORSCHLAG: {'A03', 'C00', 'C32', 'C98', 'A83', 'C76', 'A07', 'C10', 'C81'} 
	  } 
	   } 
	   } 
	 & \\ 
 

    \theutterance \stepcounter{utterance}  

    & \multicolumn{4}{p{0.6\linewidth}}{\cellcolor[rgb]{0.8,1,0.9}{%
	\makecell[{{p{\linewidth}}}]{% 
	  \tt {\tiny [A$\rangle$GM]}  
	 ZUSTIMMUNG: {'A03', 'C00', 'C32', 'C98', 'A83', 'C76', 'A07', 'C10', 'C81'}\\ \tt \\ \tt ARGUMENT: {'Diese Kombination nutzt 1660 von 2307 Aufwandspunkten und bietet eine ausgewogene Mischung verschiedener Gegenstände. Obwohl noch Kapazität verfügbar ist, stellt dies einen akzeptablen Kompromiss dar'} 
	  } 
	   } 
	   } 
	 & & \\ 
 

    \theutterance \stepcounter{utterance}  

    & & & \multicolumn{2}{p{0.3\linewidth}}{\cellcolor[rgb]{0.95,0.95,0.95}{%
	\makecell[{{p{\linewidth}}}]{% 
	  \tt {\tiny [GM$|$GM]}  
	 game successful 
	  } 
	   } 
	   } 
	 & & \\ 
 

    \theutterance \stepcounter{utterance}  

    & & & \multicolumn{2}{p{0.3\linewidth}}{\cellcolor[rgb]{0.95,0.95,0.95}{%
	\makecell[{{p{\linewidth}}}]{% 
	  \tt {\tiny [GM$|$GM]}  
	 end game 
	  } 
	   } 
	   } 
	 & & \\ 
 

\end{supertabular}
}

\end{document}
