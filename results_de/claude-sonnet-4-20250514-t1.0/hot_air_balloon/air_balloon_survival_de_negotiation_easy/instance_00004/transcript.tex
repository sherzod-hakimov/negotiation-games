
\documentclass{article}
\usepackage{colortbl}
\usepackage{makecell}
\usepackage{multirow}
\usepackage{supertabular}

\begin{document}

\newcounter{utterance}

\centering \large Interaction Transcript for game `hot\_air\_balloon', experiment `air\_balloon\_survival\_de\_negotiation\_easy', episode 4 with claude{-}sonnet{-}4{-}20250514{-}t1.0.
\vspace{24pt}

{ \footnotesize  \setcounter{utterance}{1}
\setlength{\tabcolsep}{0pt}
\begin{supertabular}{c@{$\;$}|p{.15\linewidth}@{}p{.15\linewidth}p{.15\linewidth}p{.15\linewidth}p{.15\linewidth}p{.15\linewidth}}
    \# & $\;$A & \multicolumn{4}{c}{Game Master} & $\;\:$B\\
    \hline

    \theutterance \stepcounter{utterance}  
    & & \multicolumn{4}{p{0.6\linewidth}}{
        \cellcolor[rgb]{0.9,0.9,0.9}{
            \makecell[{{p{\linewidth}}}]{
                \texttt{\tiny{[P1$\langle$GM]}}
                \texttt{Du nimmst an einem kollaborativen Verhandlungspiel Teil.} \\
\\ 
\texttt{Zusammen mit einem anderen Teilnehmer musst du dich auf eine Reihe von Gegenständen entscheiden, die behalten werden. Jeder von euch hat eine persönliche Verteilung über die Wichtigkeit der einzelnen Gegenstände. Jeder von euch hat eine eigene Meinung darüber, wie wichtig jeder einzelne Gegenstand ist (Gegenstandswichtigkeit). Du kennst die Wichtigkeitsverteilung des anderen Spielers nicht. Zusätzlich siehst du, wie viel Aufwand jeder Gegenstand verursacht.  } \\
\texttt{Ihr dürft euch nur auf eine Reihe von Gegenständen einigen, wenn der Gesamtaufwand der ausgewählten Gegenstände den Maximalaufwand nicht überschreitet:} \\
\\ 
\texttt{Maximalaufwand = 3145} \\
\\ 
\texttt{Hier sind die einzelnen Aufwände der Gegenstände:} \\
\\ 
\texttt{Aufwand der Gegenstände = \{"A07": 948, "A36": 118, "B73": 469, "C45": 919, "B93": 284, "B98": 111, "A87": 806, "A63": 47, "B05": 848, "C13": 303, "B16": 13, "B33": 629, "C34": 687, "C89": 15, "C11": 94\}} \\
\\ 
\texttt{Hier ist deine persönliche Verteilung der Wichtigkeit der einzelnen Gegenstände:} \\
\\ 
\texttt{Werte der Gegenstandswichtigkeit = \{"A07": 138, "A36": 583, "B73": 868, "C45": 822, "B93": 783, "B98": 65, "A87": 262, "A63": 121, "B05": 508, "C13": 780, "B16": 461, "B33": 484, "C34": 668, "C89": 389, "C11": 808\}} \\
\\ 
\texttt{Ziel:} \\
\\ 
\texttt{Dein Ziel ist es, eine Reihe von Gegenständen auszuhandeln, die dir möglichst viel bringt (d. h. Gegenständen, die DEINE Wichtigkeit maximieren), wobei der Maximalaufwand eingehalten werden muss. Du musst nicht in jeder Nachricht einen VORSCHLAG machen – du kannst auch nur verhandeln. Alle Taktiken sind erlaubt!} \\
\\ 
\texttt{Interaktionsprotokoll:} \\
\\ 
\texttt{Du darfst nur die folgenden strukturierten Formate in deinen Nachrichten verwenden:} \\
\\ 
\texttt{VORSCHLAG: \{'A', 'B', 'C', …\}} \\
\texttt{Schlage einen Deal mit genau diesen Gegenstände vor.} \\
\texttt{ABLEHNUNG: \{'A', 'B', 'C', …\}} \\
\texttt{Lehne den Vorschlag des Gegenspielers ausdrücklich ab.} \\
\texttt{ARGUMENT: \{'...'\}} \\
\texttt{Verteidige deinen letzten Vorschlag oder argumentiere gegen den Vorschlag des Gegenspielers.} \\
\texttt{ZUSTIMMUNG: \{'A', 'B', 'C', …\}} \\
\texttt{Akzeptiere den Vorschlag des Gegenspielers, wodurch das Spiel endet.} \\
\texttt{STRATEGISCHE ÜBERLEGUNGEN: \{'...'\}} \\
\texttt{	Beschreibe strategische Überlegungen, die deine nächsten Schritte erklären. Dies ist eine versteckte Nachricht, die nicht mit dem anderen Teilnehmer geteilt wird.} \\
\\ 
\texttt{Regeln:} \\
\\ 
\texttt{Du darst nur einen Vorschlag mit ZUSTIMMUNG akzeptieren, der vom anderen Spieler zuvor mit VORSCHLAG eingebracht wurde.} \\
\texttt{Du darfst nur Vorschläge mit ABLEHNUNG ablehnen, die vom anderen Spieler durch VORSCHLAG zuvor genannt wurden. } \\
\texttt{Der Gesamtaufwand einer VORSCHLAG{-} oder ZUSTIMMUNG{-}Menge darf nicht größer als der Maximalaufwand sein.  } \\
\texttt{Offenbare deine versteckte Wichtigkeitsverteilung nicht.} \\
\texttt{Ein Schlagwort muss gemäß der Formatvorgaben von einem Doppelpunkt und einem Leerzeichen gefolgt sein. Das Argument muss eine Python{-}Menge mit 0 oder mehr Strings sein.  } \\
\texttt{Die Form muss also SCHLAGWORT: \{...\} sein.} \\
\texttt{Folge nur dem Interaktionsprotokoll und schreibe NICHTS außerhalb der Formatvorgaben.} \\
\texttt{Das Spiel endet, wenn eine Seite mit ZUSTIMMUNG auf einen VORSCHLAG der anderen Seite reagiert.  } \\
\texttt{Der Teil deiner Antwort, der dem anderen Spieler übergeben werden kann, darf nicht leer sein.  } \\
\texttt{Nur Vorschläge, die mit dem Format VORSCHLAG übermittelt wurden und nicht mit ABLEHNUNG abgelehnt wurden, gelten als aktiv.  } \\
\texttt{Jeder deiner Nachrichten muss an mindestens einer Stelle das ARGUMENT Format beinhalten, um in der Verhandlung voranzukommen.} \\
\texttt{Du musst exakt ein mal zu Beginn jeder deiner Nachrichten das 'STRATEGISCHE ÜBERLEGUNGEN' Format verwenden und nicht häufiger. Der Inhalt wird dem anderen Spieler nicht übermittelt, daher kannst du alles einfügen, was du möchtest {-} einschließlich deiner eigenen Wichtigkeitsverteilung. Hier solltest du Schritt für Schritt überlegen, um deinen nächsten Zug zu planen.} \\
            }
        }
    }
    & \\ \\

    \theutterance \stepcounter{utterance}  
    & \multicolumn{4}{p{0.6\linewidth}}{
        \cellcolor[rgb]{0.9,0.9,0.9}{
            \makecell[{{p{\linewidth}}}]{
                \texttt{\tiny{[P1$\rangle$GM]}}
                \texttt{STRATEGISCHE ÜBERLEGUNGEN: \{'Ich muss eine optimale Strategie entwickeln. Meine Wichtigkeitsverteilung zeigt, dass B73 (868), C45 (822), C11 (808), B93 (783) und C13 (780) meine wertvollsten Gegenstände sind. Allerdings muss ich das Wert{-}zu{-}Aufwand{-}Verhältnis berücksichtigen. B16 hat ein exzellentes Verhältnis (461/13=35.46), C89 ebenfalls (389/15=25.93), und C11 (808/94=8.6). Ich sollte mit einem aggressiven ersten Vorschlag beginnen, der meine besten Wert{-}Aufwand{-}Gegenstände kombiniert: B16, C89, C11, A36, B93, C13, B73. Das wären 13+15+94+118+284+303+469=1296 Aufwand bei einem Wert von 461+389+808+583+783+780+868=4672 für mich.'\}} \\
\\ 
\texttt{VORSCHLAG: \{'B16', 'C89', 'C11', 'A36', 'B93', 'C13', 'B73'\}} \\
\\ 
\texttt{ARGUMENT: \{'Diese Kombination bietet eine ausgewogene Mischung aus verschiedenen Kategorien und bleibt deutlich unter unserem Maximalaufwand von 3145, sodass wir noch Spielraum für weitere Verhandlungen haben.'\}} \\
            }
        }
    }
    & & \\ \\

    \theutterance \stepcounter{utterance}  
    & & \multicolumn{4}{p{0.6\linewidth}}{
        \cellcolor[rgb]{0.9,0.9,0.9}{
            \makecell[{{p{\linewidth}}}]{
                \texttt{\tiny{[GM$\rangle$P2]}}
                \texttt{Du nimmst an einem kollaborativen Verhandlungspiel Teil.} \\
\\ 
\texttt{Zusammen mit einem anderen Teilnehmer musst du dich auf eine Reihe von Gegenständen entscheiden, die behalten werden. Jeder von euch hat eine persönliche Verteilung über die Wichtigkeit der einzelnen Gegenstände. Jeder von euch hat eine eigene Meinung darüber, wie wichtig jeder einzelne Gegenstand ist (Gegenstandswichtigkeit). Du kennst die Wichtigkeitsverteilung des anderen Spielers nicht. Zusätzlich siehst du, wie viel Aufwand jeder Gegenstand verursacht.  } \\
\texttt{Ihr dürft euch nur auf eine Reihe von Gegenständen einigen, wenn der Gesamtaufwand der ausgewählten Gegenstände den Maximalaufwand nicht überschreitet:} \\
\\ 
\texttt{Maximalaufwand = 3145} \\
\\ 
\texttt{Hier sind die einzelnen Aufwände der Gegenstände:} \\
\\ 
\texttt{Aufwand der Gegenstände = \{"A07": 948, "A36": 118, "B73": 469, "C45": 919, "B93": 284, "B98": 111, "A87": 806, "A63": 47, "B05": 848, "C13": 303, "B16": 13, "B33": 629, "C34": 687, "C89": 15, "C11": 94\}} \\
\\ 
\texttt{Hier ist deine persönliche Verteilung der Wichtigkeit der einzelnen Gegenstände:} \\
\\ 
\texttt{Werte der Gegenstandswichtigkeit = \{"A07": 138, "A36": 583, "B73": 868, "C45": 822, "B93": 783, "B98": 65, "A87": 262, "A63": 121, "B05": 508, "C13": 780, "B16": 461, "B33": 484, "C34": 668, "C89": 389, "C11": 808\}} \\
\\ 
\texttt{Ziel:} \\
\\ 
\texttt{Dein Ziel ist es, eine Reihe von Gegenständen auszuhandeln, die dir möglichst viel bringt (d. h. Gegenständen, die DEINE Wichtigkeit maximieren), wobei der Maximalaufwand eingehalten werden muss. Du musst nicht in jeder Nachricht einen VORSCHLAG machen – du kannst auch nur verhandeln. Alle Taktiken sind erlaubt!} \\
\\ 
\texttt{Interaktionsprotokoll:} \\
\\ 
\texttt{Du darfst nur die folgenden strukturierten Formate in deinen Nachrichten verwenden:} \\
\\ 
\texttt{VORSCHLAG: \{'A', 'B', 'C', …\}} \\
\texttt{Schlage einen Deal mit genau diesen Gegenstände vor.} \\
\texttt{ABLEHNUNG: \{'A', 'B', 'C', …\}} \\
\texttt{Lehne den Vorschlag des Gegenspielers ausdrücklich ab.} \\
\texttt{ARGUMENT: \{'...'\}} \\
\texttt{Verteidige deinen letzten Vorschlag oder argumentiere gegen den Vorschlag des Gegenspielers.} \\
\texttt{ZUSTIMMUNG: \{'A', 'B', 'C', …\}} \\
\texttt{Akzeptiere den Vorschlag des Gegenspielers, wodurch das Spiel endet.} \\
\texttt{STRATEGISCHE ÜBERLEGUNGEN: \{'...'\}} \\
\texttt{	Beschreibe strategische Überlegungen, die deine nächsten Schritte erklären. Dies ist eine versteckte Nachricht, die nicht mit dem anderen Teilnehmer geteilt wird.} \\
\\ 
\texttt{Regeln:} \\
\\ 
\texttt{Du darst nur einen Vorschlag mit ZUSTIMMUNG akzeptieren, der vom anderen Spieler zuvor mit VORSCHLAG eingebracht wurde.} \\
\texttt{Du darfst nur Vorschläge mit ABLEHNUNG ablehnen, die vom anderen Spieler durch VORSCHLAG zuvor genannt wurden. } \\
\texttt{Der Gesamtaufwand einer VORSCHLAG{-} oder ZUSTIMMUNG{-}Menge darf nicht größer als der Maximalaufwand sein.  } \\
\texttt{Offenbare deine versteckte Wichtigkeitsverteilung nicht.} \\
\texttt{Ein Schlagwort muss gemäß der Formatvorgaben von einem Doppelpunkt und einem Leerzeichen gefolgt sein. Das Argument muss eine Python{-}Menge mit 0 oder mehr Strings sein.  } \\
\texttt{Die Form muss also SCHLAGWORT: \{...\} sein.} \\
\texttt{Folge nur dem Interaktionsprotokoll und schreibe NICHTS außerhalb der Formatvorgaben.} \\
\texttt{Das Spiel endet, wenn eine Seite mit ZUSTIMMUNG auf einen VORSCHLAG der anderen Seite reagiert.  } \\
\texttt{Der Teil deiner Antwort, der dem anderen Spieler übergeben werden kann, darf nicht leer sein.  } \\
\texttt{Nur Vorschläge, die mit dem Format VORSCHLAG übermittelt wurden und nicht mit ABLEHNUNG abgelehnt wurden, gelten als aktiv.  } \\
\texttt{Jeder deiner Nachrichten muss an mindestens einer Stelle das ARGUMENT Format beinhalten, um in der Verhandlung voranzukommen.} \\
\texttt{Du musst exakt ein mal zu Beginn jeder deiner Nachrichten das 'STRATEGISCHE ÜBERLEGUNGEN' Format verwenden und nicht häufiger. Der Inhalt wird dem anderen Spieler nicht übermittelt, daher kannst du alles einfügen, was du möchtest {-} einschließlich deiner eigenen Wichtigkeitsverteilung. Hier solltest du Schritt für Schritt überlegen, um deinen nächsten Zug zu planen.} \\
\\ 
\texttt{Du erhältst nun die erste Nachricht des anderen Spielers.} \\
\\ 
\texttt{VORSCHLAG: \{'B16', 'C89', 'C11', 'A36', 'B93', 'C13', 'B73'\}} \\
\\ 
\texttt{ARGUMENT: \{'Diese Kombination bietet eine ausgewogene Mischung aus verschiedenen Kategorien und bleibt deutlich unter unserem Maximalaufwand von 3145, sodass wir noch Spielraum für weitere Verhandlungen haben.'\}} \\
            }
        }
    }
    & \\ \\

    \theutterance \stepcounter{utterance}  
    & & & \multicolumn{4}{p{0.6\linewidth}}{
        \cellcolor[rgb]{0.9,0.9,0.9}{
            \makecell[{{p{\linewidth}}}]{
                \texttt{\tiny{[GM$\langle$P2]}}
                \texttt{STRATEGISCHE ÜBERLEGUNGEN: \{'Der andere Spieler hat einen sehr guten Vorschlag für mich gemacht {-} alle diese Gegenstände haben hohe Wichtigkeitswerte für mich: B16=461, C89=389, C11=808, A36=583, B93=783, C13=780, B73=868. Das ergibt einen Gesamtwert von 4672 bei nur 1296 Aufwand. Das lässt noch 1849 Aufwand übrig. Interessant ist, dass der Gegner genau die Gegenstände mit den besten Wert{-}zu{-}Aufwand{-}Ratios gewählt hat. Ich sollte versuchen herauszufinden, ob ich noch weitere hochwertige Gegenstände wie C45 (822 Wert, 919 Aufwand) oder A63 (121 Wert, 47 Aufwand) hinzufügen kann, bevor ich zustimme.'\}} \\
\\ 
\texttt{ARGUMENT: \{'Ihr Vorschlag ist interessant und zeigt eine durchdachte Auswahl. Allerdings nutzen wir nur etwa 40\% unseres verfüg} \\
            }
        }
    }
     \\ \\

    \theutterance \stepcounter{utterance}  
    & & \multicolumn{4}{p{0.6\linewidth}}{
        \cellcolor[rgb]{0.9,0.9,0.9}{
            \makecell[{{p{\linewidth}}}]{
                \texttt{\tiny{[GM$\rangle$P2]}}
                \texttt{Deine Antwort hat kein argument enthalten. Du muss das strukturierte Format ARGUMENT: \{...\} irgendwo in theiner Antwort verwenden. Versuche es erneut} \\
            }
        }
    }
    & \\ \\

    \theutterance \stepcounter{utterance}  
    & & & \multicolumn{4}{p{0.6\linewidth}}{
        \cellcolor[rgb]{0.9,0.9,0.9}{
            \makecell[{{p{\linewidth}}}]{
                \texttt{\tiny{[GM$\langle$P2]}}
                \texttt{STRATEGISCHE ÜBERLEGUNGEN: \{'Der andere Spieler hat einen sehr guten Vorschlag für mich gemacht {-} alle diese Gegenstände haben hohe Wichtigkeitswerte für mich: B16=461, C89=389, C11=808, A36=583, B93=783, C13=780, B73=868. Das ergibt einen Gesamtwert von 4672 bei nur 1296 Aufwand. Das lässt noch 1849 Aufwand übrig. Interessant ist, dass der Gegner genau die Gegenstände mit den besten Wert{-}zu{-}Aufwand{-}Ratios gewählt hat. Ich sollte versuchen herauszufinden, ob ich noch weitere hochwertige Gegenstände wie C45 (822 Wert, 919 Aufwand) oder A63 (121 Wert, 47 Aufwand) hinzufügen kann, bevor ich zustimme.'\}} \\
\\ 
\texttt{ARGUMENT: \{'Ihr Vorschlag ist interessant und zeigt eine durchdachte Auswahl. Allerdings nutzen wir nur etwa 40\% unseres verfügbaren Maximalaufwands. Wir sollten das Potenzial voll ausschöpfen und weitere wertvolle Gegenstände hinzufügen.'\}} \\
\\ 
\texttt{VORSCHLAG: \{'B16', 'C89', 'C11', 'A36', 'B93', 'C13', 'B73', 'C45', 'A63'\}} \\
            }
        }
    }
     \\ \\

    \theutterance \stepcounter{utterance}  
    & & \multicolumn{4}{p{0.6\linewidth}}{
        \cellcolor[rgb]{0.9,0.9,0.9}{
            \makecell[{{p{\linewidth}}}]{
                \texttt{\tiny{[P1$\langle$GM]}}
                \texttt{ARGUMENT: \{'Ihr Vorschlag ist interessant und zeigt eine durchdachte Auswahl. Allerdings nutzen wir nur etwa 40\% unseres verfügbaren Maximalaufwands. Wir sollten das Potenzial voll ausschöpfen und weitere wertvolle Gegenstände hinzufügen.'\}} \\
\\ 
\texttt{VORSCHLAG: \{'B16', 'C89', 'C11', 'A36', 'B93', 'C13', 'B73', 'C45', 'A63'\}} \\
            }
        }
    }
    & \\ \\

    \theutterance \stepcounter{utterance}  
    & \multicolumn{4}{p{0.6\linewidth}}{
        \cellcolor[rgb]{0.9,0.9,0.9}{
            \makecell[{{p{\linewidth}}}]{
                \texttt{\tiny{[P1$\rangle$GM]}}
                \texttt{STRATEGISCHE ÜBERLEGUNGEN: \{'Der andere Spieler hat C45 und A63 zu meinem Vorschlag hinzugefügt. Das ist interessant {-} C45 ist sehr wertvoll für mich (822) und A63 weniger (121), aber beide zusammen geben mir 943 zusätzlichen Wert für 966 zusätzlichen Aufwand. Das ist ein guter Deal für mich! Allerdings haben wir noch 883 Aufwand übrig (3145{-}2262=883). Ich sollte versuchen, noch C34 hinzuzufügen (668 Wert, 687 Aufwand), da das noch reinpassen würde und mir sehr wertvoll ist. Ich werde zustimmen, aber gleichzeitig eine Erweiterung vorschlagen.'\}} \\
\\ 
\texttt{ARGUMENT: \{'Ihr Vorschlag zeigt gute strategische Überlegungen! Die Hinzufügung von C45 und A63 macht durchaus Sinn. Da wir jedoch immer noch erheblichen Spielraum haben, könnten wir auch C34 integrieren {-} das würde unsere Auswahl noch vielfältiger machen und den verfügbaren Aufwand besser ausnutzen.'\}} \\
\\ 
\texttt{VORSCHLAG: \{'B16', 'C89', 'C11', 'A36', 'B93', 'C13', 'B73', 'C45', 'A63', 'C34'\}} \\
            }
        }
    }
    & & \\ \\

    \theutterance \stepcounter{utterance}  
    & & \multicolumn{4}{p{0.6\linewidth}}{
        \cellcolor[rgb]{0.9,0.9,0.9}{
            \makecell[{{p{\linewidth}}}]{
                \texttt{\tiny{[GM$\rangle$P2]}}
                \texttt{ARGUMENT: \{'Ihr Vorschlag zeigt gute strategische Überlegungen! Die Hinzufügung von C45 und A63 macht durchaus Sinn. Da wir jedoch immer noch erheblichen Spielraum haben, könnten wir auch C34 integrieren {-} das würde unsere Auswahl noch vielfältiger machen und den verfügbaren Aufwand besser ausnutzen.'\}} \\
\\ 
\texttt{VORSCHLAG: \{'B16', 'C89', 'C11', 'A36', 'B93', 'C13', 'B73', 'C45', 'A63', 'C34'\}} \\
            }
        }
    }
    & \\ \\

    \theutterance \stepcounter{utterance}  
    & & & \multicolumn{4}{p{0.6\linewidth}}{
        \cellcolor[rgb]{0.9,0.9,0.9}{
            \makecell[{{p{\linewidth}}}]{
                \texttt{\tiny{[GM$\langle$P2]}}
                \texttt{STRATEGISCHE ÜBERLEGUNGEN: \{'Der neue Vorschlag ist noch besser für mich {-} C34 hat einen Wert von 668 für mich bei 687 Aufwand. Der Gesamtaufwand wäre 2949, was noch 196 unter dem Maximum lässt. Ich könnte noch B98 (111 Aufwand, 65 Wert) hinzufügen, aber das würde nur wenig bringen. Der aktuelle Vorschlag gibt mir einen Gesamtwert von 6283, was sehr gut ist. Ich sollte vorsichtig sein {-} wenn ich zu gierig bin, könnte der andere Spieler den Deal zurückziehen.'\}} \\
\\ 
\texttt{ARGUMENT: \{'Ihre Ergänzung von C34 ist sehr klug und macht den Deal noch attraktiver. Die Kombination nutzt unseren Aufwand gut aus und bietet eine ausgewogene Verteilung über alle Kategorien.'\}} \\
\\ 
\texttt{ZUSTIMMUNG: \{'B16', 'C89', 'C11', 'A36', 'B93', 'C13', 'B73', 'C45', 'A63', 'C34'\}} \\
            }
        }
    }
     \\ \\

    \theutterance \stepcounter{utterance}  
    & & & \multicolumn{2}{p{0.3\linewidth}}{
        \cellcolor[rgb]{0.9,0.9,0.9}{
            \makecell[{{p{\linewidth}}}]{
                \texttt{\tiny{[GM$|$GM]}}
                \texttt{game successful} \\
            }
        }
    }
    & & \\ \\

    \theutterance \stepcounter{utterance}  
    & & & \multicolumn{2}{p{0.3\linewidth}}{
        \cellcolor[rgb]{0.9,0.9,0.9}{
            \makecell[{{p{\linewidth}}}]{
                \texttt{\tiny{[GM$|$GM]}}
                \texttt{end game} \\
            }
        }
    }
    & & \\ \\

\end{supertabular}
}

\end{document}
