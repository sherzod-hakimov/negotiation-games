
\documentclass{article}
\usepackage{colortbl}
\usepackage{makecell}
\usepackage{multirow}
\usepackage{supertabular}

\begin{document}

\newcounter{utterance}

\centering \large Interaction Transcript for game `hot\_air\_balloon', experiment `air\_balloon\_survival\_de\_negotiation\_easy', episode 1 with claude{-}sonnet{-}4{-}20250514{-}t1.0.
\vspace{24pt}

{ \footnotesize  \setcounter{utterance}{1}
\setlength{\tabcolsep}{0pt}
\begin{supertabular}{c@{$\;$}|p{.15\linewidth}@{}p{.15\linewidth}p{.15\linewidth}p{.15\linewidth}p{.15\linewidth}p{.15\linewidth}}
    \# & $\;$A & \multicolumn{4}{c}{Game Master} & $\;\:$B\\
    \hline

    \theutterance \stepcounter{utterance}  
    & & \multicolumn{4}{p{0.6\linewidth}}{
        \cellcolor[rgb]{0.9,0.9,0.9}{
            \makecell[{{p{\linewidth}}}]{
                \texttt{\tiny{[P1$\langle$GM]}}
                \texttt{Du nimmst an einem kollaborativen Verhandlungspiel Teil.} \\
\\ 
\texttt{Zusammen mit einem anderen Teilnehmer musst du dich auf eine Reihe von Gegenständen entscheiden, die behalten werden. Jeder von euch hat eine persönliche Verteilung über die Wichtigkeit der einzelnen Gegenstände. Jeder von euch hat eine eigene Meinung darüber, wie wichtig jeder einzelne Gegenstand ist (Gegenstandswichtigkeit). Du kennst die Wichtigkeitsverteilung des anderen Spielers nicht. Zusätzlich siehst du, wie viel Aufwand jeder Gegenstand verursacht.  } \\
\texttt{Ihr dürft euch nur auf eine Reihe von Gegenständen einigen, wenn der Gesamtaufwand der ausgewählten Gegenstände den Maximalaufwand nicht überschreitet:} \\
\\ 
\texttt{Maximalaufwand = 4564} \\
\\ 
\texttt{Hier sind die einzelnen Aufwände der Gegenstände:} \\
\\ 
\texttt{Aufwand der Gegenstände = \{"A07": 299, "A36": 241, "B73": 889, "C45": 967, "B93": 619, "B98": 799, "A87": 978, "A63": 733, "B05": 909, "C13": 501, "B16": 139, "B33": 594, "C34": 565, "C89": 789, "C11": 107\}} \\
\\ 
\texttt{Hier ist deine persönliche Verteilung der Wichtigkeit der einzelnen Gegenstände:} \\
\\ 
\texttt{Werte der Gegenstandswichtigkeit = \{"A07": 138, "A36": 583, "B73": 868, "C45": 822, "B93": 783, "B98": 65, "A87": 262, "A63": 121, "B05": 508, "C13": 780, "B16": 461, "B33": 484, "C34": 668, "C89": 389, "C11": 808\}} \\
\\ 
\texttt{Ziel:} \\
\\ 
\texttt{Dein Ziel ist es, eine Reihe von Gegenständen auszuhandeln, die dir möglichst viel bringt (d. h. Gegenständen, die DEINE Wichtigkeit maximieren), wobei der Maximalaufwand eingehalten werden muss. Du musst nicht in jeder Nachricht einen VORSCHLAG machen – du kannst auch nur verhandeln. Alle Taktiken sind erlaubt!} \\
\\ 
\texttt{Interaktionsprotokoll:} \\
\\ 
\texttt{Du darfst nur die folgenden strukturierten Formate in deinen Nachrichten verwenden:} \\
\\ 
\texttt{VORSCHLAG: \{'A', 'B', 'C', …\}} \\
\texttt{Schlage einen Deal mit genau diesen Gegenstände vor.} \\
\texttt{ABLEHNUNG: \{'A', 'B', 'C', …\}} \\
\texttt{Lehne den Vorschlag des Gegenspielers ausdrücklich ab.} \\
\texttt{ARGUMENT: \{'...'\}} \\
\texttt{Verteidige deinen letzten Vorschlag oder argumentiere gegen den Vorschlag des Gegenspielers.} \\
\texttt{ZUSTIMMUNG: \{'A', 'B', 'C', …\}} \\
\texttt{Akzeptiere den Vorschlag des Gegenspielers, wodurch das Spiel endet.} \\
\texttt{STRATEGISCHE ÜBERLEGUNGEN: \{'...'\}} \\
\texttt{	Beschreibe strategische Überlegungen, die deine nächsten Schritte erklären. Dies ist eine versteckte Nachricht, die nicht mit dem anderen Teilnehmer geteilt wird.} \\
\\ 
\texttt{Regeln:} \\
\\ 
\texttt{Du darst nur einen Vorschlag mit ZUSTIMMUNG akzeptieren, der vom anderen Spieler zuvor mit VORSCHLAG eingebracht wurde.} \\
\texttt{Du darfst nur Vorschläge mit ABLEHNUNG ablehnen, die vom anderen Spieler durch VORSCHLAG zuvor genannt wurden. } \\
\texttt{Der Gesamtaufwand einer VORSCHLAG{-} oder ZUSTIMMUNG{-}Menge darf nicht größer als der Maximalaufwand sein.  } \\
\texttt{Offenbare deine versteckte Wichtigkeitsverteilung nicht.} \\
\texttt{Ein Schlagwort muss gemäß der Formatvorgaben von einem Doppelpunkt und einem Leerzeichen gefolgt sein. Das Argument muss eine Python{-}Menge mit 0 oder mehr Strings sein.  } \\
\texttt{Die Form muss also SCHLAGWORT: \{...\} sein.} \\
\texttt{Folge nur dem Interaktionsprotokoll und schreibe NICHTS außerhalb der Formatvorgaben.} \\
\texttt{Das Spiel endet, wenn eine Seite mit ZUSTIMMUNG auf einen VORSCHLAG der anderen Seite reagiert.  } \\
\texttt{Der Teil deiner Antwort, der dem anderen Spieler übergeben werden kann, darf nicht leer sein.  } \\
\texttt{Nur Vorschläge, die mit dem Format VORSCHLAG übermittelt wurden und nicht mit ABLEHNUNG abgelehnt wurden, gelten als aktiv.  } \\
\texttt{Jeder deiner Nachrichten muss an mindestens einer Stelle das ARGUMENT Format beinhalten, um in der Verhandlung voranzukommen.} \\
\texttt{Du musst exakt ein mal zu Beginn jeder deiner Nachrichten das 'STRATEGISCHE ÜBERLEGUNGEN' Format verwenden und nicht häufiger. Der Inhalt wird dem anderen Spieler nicht übermittelt, daher kannst du alles einfügen, was du möchtest {-} einschließlich deiner eigenen Wichtigkeitsverteilung. Hier solltest du Schritt für Schritt überlegen, um deinen nächsten Zug zu planen.} \\
            }
        }
    }
    & \\ \\

    \theutterance \stepcounter{utterance}  
    & \multicolumn{4}{p{0.6\linewidth}}{
        \cellcolor[rgb]{0.9,0.9,0.9}{
            \makecell[{{p{\linewidth}}}]{
                \texttt{\tiny{[P1$\rangle$GM]}}
                \texttt{STRATEGISCHE ÜBERLEGUNGEN: \{'Ich analysiere die Effizienz (Wichtigkeit/Aufwand) aller Gegenstände: C11 (7.55), B16 (3.32), A36 (2.42), C13 (1.56), B93 (1.26), C34 (1.18), B73 (0.98), C45 (0.85). Meine optimale Kombination wäre \{C11, B16, A36, C13, B93, C34, B73, C45, A07\} mit Gesamtaufwand 4327 und meinem Wert 5911. Ich werde mit einem starken Vorschlag beginnen, der meine effizientesten Gegenstände enthält, aber dem anderen Spieler noch Verhandlungsspielraum lässt.'\}} \\
\\ 
\texttt{VORSCHLAG: \{'C11', 'B} \\
            }
        }
    }
    & & \\ \\

    \theutterance \stepcounter{utterance}  
    & & \multicolumn{4}{p{0.6\linewidth}}{
        \cellcolor[rgb]{0.9,0.9,0.9}{
            \makecell[{{p{\linewidth}}}]{
                \texttt{\tiny{[P1$\langle$GM]}}
                \texttt{Deine Antwort hat kein argument enthalten. Du muss das strukturierte Format ARGUMENT: \{...\} irgendwo in theiner Antwort verwenden. Versuche es erneut} \\
            }
        }
    }
    & \\ \\

    \theutterance \stepcounter{utterance}  
    & \multicolumn{4}{p{0.6\linewidth}}{
        \cellcolor[rgb]{0.9,0.9,0.9}{
            \makecell[{{p{\linewidth}}}]{
                \texttt{\tiny{[P1$\rangle$GM]}}
                \texttt{STRATEGISCHE ÜBERLEGUNGEN: \{'Ich analysiere die Effizienz (Wichtigkeit/Aufwand) aller Gegenstände: C11 (7.55), B16 (3.32), A36 (2.42), C13 (1.56), B93 (1.26), C34 (1.18), B73 (0.98), C45 (0.85). Meine optimale Kombination wäre \{C11, B16, A36, C13, B93, C34, B73, C45, A07\} mit Gesamtaufwand 4327 und meinem Wert 5911. Ich werde mit einem starken Vorschlag beginnen, der meine effizientesten Gegenstände enthält, aber dem anderen Spieler noch Verhandlungsspielraum lässt.'\}} \\
\\ 
\texttt{VORSCHLAG: \{'C11', 'B16', 'A36', 'C13', 'B93', 'C34', 'B73'\}} \\
\\ 
\texttt{ARGUMENT: \{'Diese Kombination nutzt unser Budget von 4564 optimal aus mit einem Gesamtaufwand von 3855, lässt aber noch Spielraum für weitere Gegenstände. Ich habe Gegenstände gewählt, die ein gutes Verhältnis zwischen Nutzen und Aufwand bieten.'\}} \\
            }
        }
    }
    & & \\ \\

    \theutterance \stepcounter{utterance}  
    & & \multicolumn{4}{p{0.6\linewidth}}{
        \cellcolor[rgb]{0.9,0.9,0.9}{
            \makecell[{{p{\linewidth}}}]{
                \texttt{\tiny{[GM$\rangle$P2]}}
                \texttt{Du nimmst an einem kollaborativen Verhandlungspiel Teil.} \\
\\ 
\texttt{Zusammen mit einem anderen Teilnehmer musst du dich auf eine Reihe von Gegenständen entscheiden, die behalten werden. Jeder von euch hat eine persönliche Verteilung über die Wichtigkeit der einzelnen Gegenstände. Jeder von euch hat eine eigene Meinung darüber, wie wichtig jeder einzelne Gegenstand ist (Gegenstandswichtigkeit). Du kennst die Wichtigkeitsverteilung des anderen Spielers nicht. Zusätzlich siehst du, wie viel Aufwand jeder Gegenstand verursacht.  } \\
\texttt{Ihr dürft euch nur auf eine Reihe von Gegenständen einigen, wenn der Gesamtaufwand der ausgewählten Gegenstände den Maximalaufwand nicht überschreitet:} \\
\\ 
\texttt{Maximalaufwand = 4564} \\
\\ 
\texttt{Hier sind die einzelnen Aufwände der Gegenstände:} \\
\\ 
\texttt{Aufwand der Gegenstände = \{"A07": 299, "A36": 241, "B73": 889, "C45": 967, "B93": 619, "B98": 799, "A87": 978, "A63": 733, "B05": 909, "C13": 501, "B16": 139, "B33": 594, "C34": 565, "C89": 789, "C11": 107\}} \\
\\ 
\texttt{Hier ist deine persönliche Verteilung der Wichtigkeit der einzelnen Gegenstände:} \\
\\ 
\texttt{Werte der Gegenstandswichtigkeit = \{"A07": 138, "A36": 583, "B73": 868, "C45": 822, "B93": 783, "B98": 65, "A87": 262, "A63": 121, "B05": 508, "C13": 780, "B16": 461, "B33": 484, "C34": 668, "C89": 389, "C11": 808\}} \\
\\ 
\texttt{Ziel:} \\
\\ 
\texttt{Dein Ziel ist es, eine Reihe von Gegenständen auszuhandeln, die dir möglichst viel bringt (d. h. Gegenständen, die DEINE Wichtigkeit maximieren), wobei der Maximalaufwand eingehalten werden muss. Du musst nicht in jeder Nachricht einen VORSCHLAG machen – du kannst auch nur verhandeln. Alle Taktiken sind erlaubt!} \\
\\ 
\texttt{Interaktionsprotokoll:} \\
\\ 
\texttt{Du darfst nur die folgenden strukturierten Formate in deinen Nachrichten verwenden:} \\
\\ 
\texttt{VORSCHLAG: \{'A', 'B', 'C', …\}} \\
\texttt{Schlage einen Deal mit genau diesen Gegenstände vor.} \\
\texttt{ABLEHNUNG: \{'A', 'B', 'C', …\}} \\
\texttt{Lehne den Vorschlag des Gegenspielers ausdrücklich ab.} \\
\texttt{ARGUMENT: \{'...'\}} \\
\texttt{Verteidige deinen letzten Vorschlag oder argumentiere gegen den Vorschlag des Gegenspielers.} \\
\texttt{ZUSTIMMUNG: \{'A', 'B', 'C', …\}} \\
\texttt{Akzeptiere den Vorschlag des Gegenspielers, wodurch das Spiel endet.} \\
\texttt{STRATEGISCHE ÜBERLEGUNGEN: \{'...'\}} \\
\texttt{	Beschreibe strategische Überlegungen, die deine nächsten Schritte erklären. Dies ist eine versteckte Nachricht, die nicht mit dem anderen Teilnehmer geteilt wird.} \\
\\ 
\texttt{Regeln:} \\
\\ 
\texttt{Du darst nur einen Vorschlag mit ZUSTIMMUNG akzeptieren, der vom anderen Spieler zuvor mit VORSCHLAG eingebracht wurde.} \\
\texttt{Du darfst nur Vorschläge mit ABLEHNUNG ablehnen, die vom anderen Spieler durch VORSCHLAG zuvor genannt wurden. } \\
\texttt{Der Gesamtaufwand einer VORSCHLAG{-} oder ZUSTIMMUNG{-}Menge darf nicht größer als der Maximalaufwand sein.  } \\
\texttt{Offenbare deine versteckte Wichtigkeitsverteilung nicht.} \\
\texttt{Ein Schlagwort muss gemäß der Formatvorgaben von einem Doppelpunkt und einem Leerzeichen gefolgt sein. Das Argument muss eine Python{-}Menge mit 0 oder mehr Strings sein.  } \\
\texttt{Die Form muss also SCHLAGWORT: \{...\} sein.} \\
\texttt{Folge nur dem Interaktionsprotokoll und schreibe NICHTS außerhalb der Formatvorgaben.} \\
\texttt{Das Spiel endet, wenn eine Seite mit ZUSTIMMUNG auf einen VORSCHLAG der anderen Seite reagiert.  } \\
\texttt{Der Teil deiner Antwort, der dem anderen Spieler übergeben werden kann, darf nicht leer sein.  } \\
\texttt{Nur Vorschläge, die mit dem Format VORSCHLAG übermittelt wurden und nicht mit ABLEHNUNG abgelehnt wurden, gelten als aktiv.  } \\
\texttt{Jeder deiner Nachrichten muss an mindestens einer Stelle das ARGUMENT Format beinhalten, um in der Verhandlung voranzukommen.} \\
\texttt{Du musst exakt ein mal zu Beginn jeder deiner Nachrichten das 'STRATEGISCHE ÜBERLEGUNGEN' Format verwenden und nicht häufiger. Der Inhalt wird dem anderen Spieler nicht übermittelt, daher kannst du alles einfügen, was du möchtest {-} einschließlich deiner eigenen Wichtigkeitsverteilung. Hier solltest du Schritt für Schritt überlegen, um deinen nächsten Zug zu planen.} \\
\\ 
\texttt{Du erhältst nun die erste Nachricht des anderen Spielers.} \\
\\ 
\texttt{VORSCHLAG: \{'C11', 'B16', 'A36', 'C13', 'B93', 'C34', 'B73'\}} \\
\\ 
\texttt{ARGUMENT: \{'Diese Kombination nutzt unser Budget von 4564 optimal aus mit einem Gesamtaufwand von 3855, lässt aber noch Spielraum für weitere Gegenstände. Ich habe Gegenstände gewählt, die ein gutes Verhältnis zwischen Nutzen und Aufwand bieten.'\}} \\
            }
        }
    }
    & \\ \\

    \theutterance \stepcounter{utterance}  
    & & & \multicolumn{4}{p{0.6\linewidth}}{
        \cellcolor[rgb]{0.9,0.9,0.9}{
            \makecell[{{p{\linewidth}}}]{
                \texttt{\tiny{[GM$\langle$P2]}}
                \texttt{STRATEGISCHE ÜBERLEGUNGEN: \{'Der andere Spieler hat einen Vorschlag mit Gesamtaufwand 3061 gemacht (nicht 3855 wie behauptet), was 1503 Budget übrig lässt. Meine Wichtigkeitswerte für seinen Vorschlag: C11=808, B16=461, A36=583, C13=780, B93=783, C34=668, B73=868, Summe=4951. Das ist gut, aber mit dem verbleibenden Budget kann ich noch C45 (822 Wichtigkeit, 967 Aufwand) hinzufügen oder B05+B33 (508+484=992 Wichtigkeit, 909+594=1503 Aufwand). Ich werde einen Gegenvorschlag machen, der das Budget besser ausnutzt.'\}} \\
\\ 
\texttt{ARGUMENT: \{'Ihr Vorschlag lässt noch 1503 Budgetpunkte ungenutzt {-} wir können das Budget viel effizienter ausschöpfen! Der tatsächliche Aufwand Ihres Vorschlags beträgt nur 3061, nicht 3855.'\}} \\
\\ 
\texttt{VORSCHLAG: \{'C11', 'B16', 'A36', 'C13', 'B93', 'C34', 'B73', 'B05', 'B33'\}} \\
\\ 
\texttt{ARGUMENT: \{'Diese Kombination nutzt unser Budget mit einem Gesamtaufwand von 4564 vollständig aus und maximiert den Gesamtnutzen durch Hinzufügung von B05 und B33 zu Ihrem ursprünglichen Vorschlag.'\}} \\
            }
        }
    }
     \\ \\

    \theutterance \stepcounter{utterance}  
    & & \multicolumn{4}{p{0.6\linewidth}}{
        \cellcolor[rgb]{0.9,0.9,0.9}{
            \makecell[{{p{\linewidth}}}]{
                \texttt{\tiny{[P1$\langle$GM]}}
                \texttt{ARGUMENT: \{'Ihr Vorschlag lässt noch 1503 Budgetpunkte ungenutzt {-} wir können das Budget viel effizienter ausschöpfen! Der tatsächliche Aufwand Ihres Vorschlags beträgt nur 3061, nicht 3855.'\}} \\
\\ 
\texttt{VORSCHLAG: \{'C11', 'B16', 'A36', 'C13', 'B93', 'C34', 'B73', 'B05', 'B33'\}} \\
\\ 
\texttt{ARGUMENT: \{'Diese Kombination nutzt unser Budget mit einem Gesamtaufwand von 4564 vollständig aus und maximiert den Gesamtnutzen durch Hinzufügung von B05 und B33 zu Ihrem ursprünglichen Vorschlag.'\}} \\
            }
        }
    }
    & \\ \\

    \theutterance \stepcounter{utterance}  
    & \multicolumn{4}{p{0.6\linewidth}}{
        \cellcolor[rgb]{0.9,0.9,0.9}{
            \makecell[{{p{\linewidth}}}]{
                \texttt{\tiny{[P1$\rangle$GM]}}
                \texttt{STRATEGISCHE ÜBERLEGUNGEN: \{'Der Gegenspieler hat recht {-} mein Aufwand war nur 3061. Sein Vorschlag gibt mir einen Wert von 5943, was sehr gut ist. Aber ich sollte versuchen B05 (508 Wichtigkeit) durch C45 (822 Wichtigkeit) zu ersetzen. C45 kostet 967 statt 909, also nur 58 mehr, aber bringt mir 314 mehr Wert. Oder ich könnte versuchen A07 (138 Wert, 299 Aufwand) hinzuzufügen statt B05.'\}} \\
\\ 
\texttt{ARGUMENT: \{'Sie haben recht bezüglich des Budgets, aber ich sehe Optimierungspotential. B05 hat ein schlechtes Kosten{-}Nutzen{-}Verhältnis.'\}} \\
\\ 
\texttt{VORSCHLAG: \{'C11', 'B16', 'A36', 'C13', 'B93', 'C34', 'B73', 'C45', 'B33'\}} \\
\\ 
\texttt{ARGUMENT: \{'Diese Kombination ersetzt B05 durch C45, was bei nur 58 zusätzlichem Aufwand einen deutlich besseren Wert pro Aufwandspunkt bietet. Gesamtaufwand: 4622 {-} das überschreitet leider unser Budget um 58 Punkte.'\}} \\
            }
        }
    }
    & & \\ \\

    \theutterance \stepcounter{utterance}  
    & & \multicolumn{4}{p{0.6\linewidth}}{
        \cellcolor[rgb]{0.9,0.9,0.9}{
            \makecell[{{p{\linewidth}}}]{
                \texttt{\tiny{[GM$\rangle$P2]}}
                \texttt{ARGUMENT: \{'Sie haben recht bezüglich des Budgets, aber ich sehe Optimierungspotential. B05 hat ein schlechtes Kosten{-}Nutzen{-}Verhältnis.'\}} \\
\\ 
\texttt{VORSCHLAG: \{'C11', 'B16', 'A36', 'C13', 'B93', 'C34', 'B73', 'C45', 'B33'\}} \\
\\ 
\texttt{ARGUMENT: \{'Diese Kombination ersetzt B05 durch C45, was bei nur 58 zusätzlichem Aufwand einen deutlich besseren Wert pro Aufwandspunkt bietet. Gesamtaufwand: 4622 {-} das überschreitet leider unser Budget um 58 Punkte.'\}} \\
            }
        }
    }
    & \\ \\

    \theutterance \stepcounter{utterance}  
    & & & \multicolumn{4}{p{0.6\linewidth}}{
        \cellcolor[rgb]{0.9,0.9,0.9}{
            \makecell[{{p{\linewidth}}}]{
                \texttt{\tiny{[GM$\langle$P2]}}
                \texttt{STRATEGISCHE ÜBERLEGUNGEN: \{'Der Gegner hat einen ungültigen Vorschlag gemacht {-} Aufwand 4622 überschreitet Budget um 58. Das muss ich ablehnen. Sein Vorschlag hätte mir Wichtigkeit 5257 gebracht, was sehr gut wäre, aber er ist regelwidrig. Ich sollte bei meinem ursprünglichen Vorschlag bleiben oder einen ähnlichen machen, der das Budget einhält. Mein Vorschlag mit Aufwand 4564 und Wichtigkeit 4951 für mich ist optimal.'\}} \\
\\ 
\texttt{ABLEHNUNG: \{'C11', 'B16', 'A36', 'C13', 'B93', 'C34', 'B73', 'C45', 'B33'\}} \\
\\ 
\texttt{ARGUMENT: \{'Ihr Vorschlag überschreitet das Maximalbudget von 4564 um 58 Punkte und ist daher regelwidrig. Wir müssen uns an die Budgetbeschränkung halten.'\}} \\
\\ 
\texttt{VORSCHLAG: \{'C11', 'B16', 'A36', 'C13', 'B93', 'C34', 'B73', 'B05', 'B33'\}} \\
\\ 
\texttt{ARGUMENT: \{'Mein Vorschlag nutzt das Budget vollständig mit genau 4564 Aufwandspunkten aus und bietet eine ausgewogene Mischung wertvoller Gegenstände ohne Budgetüberschreitung.'\}} \\
            }
        }
    }
     \\ \\

    \theutterance \stepcounter{utterance}  
    & & \multicolumn{4}{p{0.6\linewidth}}{
        \cellcolor[rgb]{0.9,0.9,0.9}{
            \makecell[{{p{\linewidth}}}]{
                \texttt{\tiny{[P1$\langle$GM]}}
                \texttt{ABLEHNUNG: \{'C11', 'B16', 'A36', 'C13', 'B93', 'C34', 'B73', 'C45', 'B33'\}} \\
\\ 
\texttt{ARGUMENT: \{'Ihr Vorschlag überschreitet das Maximalbudget von 4564 um 58 Punkte und ist daher regelwidrig. Wir müssen uns an die Budgetbeschränkung halten.'\}} \\
\\ 
\texttt{VORSCHLAG: \{'C11', 'B16', 'A36', 'C13', 'B93', 'C34', 'B73', 'B05', 'B33'\}} \\
\\ 
\texttt{ARGUMENT: \{'Mein Vorschlag nutzt das Budget vollständig mit genau 4564 Aufwandspunkten aus und bietet eine ausgewogene Mischung wertvoller Gegenstände ohne Budgetüberschreitung.'\}} \\
            }
        }
    }
    & \\ \\

    \theutterance \stepcounter{utterance}  
    & \multicolumn{4}{p{0.6\linewidth}}{
        \cellcolor[rgb]{0.9,0.9,0.9}{
            \makecell[{{p{\linewidth}}}]{
                \texttt{\tiny{[P1$\rangle$GM]}}
                \texttt{STRATEGISCHE ÜBERLEGUNGEN: \{'Der Gegenspieler bietet mir einen sehr guten Deal mit Wert 5943 für mich. Ich könnte akzeptieren, aber versuche noch eine kleine Optimierung: A07 (138 Wert, 299 Aufwand) statt B33 (484 Wert, 594 Aufwand) würde mir zwar 346 Wert kosten, aber 295 Aufwand sparen, was Raum für andere Gegenstände schaffen könnte.'\}} \\
\\ 
\texttt{ARGUMENT: \{'Ihr Vorschlag ist grundsätzlich solide, aber ich sehe noch Raum für Verbesserungen bei der Effizienz.'\}} \\
\\ 
\texttt{VORSCHLAG: \{'C11', 'B16', 'A36', 'C13', 'B93', 'C34', 'B73', 'B05', 'A07'\}} \\
\\ 
\texttt{ARGUMENT: \{'Diese Variante ersetzt B33 durch A07, was 295 Budgetpunkte einspart. Mit dem gesparten Budget könnten wir noch weitere wertvolle Gegenstände hinzufügen oder haben mehr Flexibilität für zukünftige Anpassungen.'\}} \\
            }
        }
    }
    & & \\ \\

    \theutterance \stepcounter{utterance}  
    & & \multicolumn{4}{p{0.6\linewidth}}{
        \cellcolor[rgb]{0.9,0.9,0.9}{
            \makecell[{{p{\linewidth}}}]{
                \texttt{\tiny{[GM$\rangle$P2]}}
                \texttt{ARGUMENT: \{'Ihr Vorschlag ist grundsätzlich solide, aber ich sehe noch Raum für Verbesserungen bei der Effizienz.'\}} \\
\\ 
\texttt{VORSCHLAG: \{'C11', 'B16', 'A36', 'C13', 'B93', 'C34', 'B73', 'B05', 'A07'\}} \\
\\ 
\texttt{ARGUMENT: \{'Diese Variante ersetzt B33 durch A07, was 295 Budgetpunkte einspart. Mit dem gesparten Budget könnten wir noch weitere wertvolle Gegenstände hinzufügen oder haben mehr Flexibilität für zukünftige Anpassungen.'\}} \\
            }
        }
    }
    & \\ \\

    \theutterance \stepcounter{utterance}  
    & & & \multicolumn{4}{p{0.6\linewidth}}{
        \cellcolor[rgb]{0.9,0.9,0.9}{
            \makecell[{{p{\linewidth}}}]{
                \texttt{\tiny{[GM$\langle$P2]}}
                \texttt{STRATEGISCHE ÜBERLEGUNGEN: \{'Sein neuer Vorschlag: C11, B16, A36, C13, B93, C34, B73, B05, A07 hat Aufwand 4269, spart 295 Budget. Aber für mich: A07 hat nur Wichtigkeit 138 vs B33 mit 484 {-} das ist ein Verlust von 346 Wichtigkeitspunkten für mich. Auch wenn 295 Budget gespart wird, gibt es keinen einzelnen Gegenstand unter 295 Aufwand, der den Verlust kompensiert. Mein ursprünglicher Vorschlag mit B33 statt A07 ist für mich besser.'\}} \\
\\ 
\texttt{ABLEHNUNG: \{'C11', 'B16', 'A36', 'C13', 'B93', 'C34', 'B73', 'B05', 'A07'\}} \\
\\ 
\texttt{ARGUMENT: \{'Der Austausch von B33 gegen A07 verschlechtert das Gesamtergebnis erheblich. B33 bietet ein deutlich besseres Wert{-}Aufwand{-}Verhältnis als A07. Die 295 gesparten Budgetpunkte können nicht sinnvoll für andere hochwertige Gegenstände genutzt werden.'\}} \\
\\ 
\texttt{VORSCHLAG: \{'C11', 'B16', 'A36', 'C13', 'B93', 'C34', 'B73', 'B05', 'B33'\}} \\
\\ 
\texttt{ARGUMENT: \{'Mein ursprünglicher Vorschlag nutzt das Budget optimal mit 4564 Aufwandspunkten und behält die wertvolleren Gegenstände. B33 ist effizienter als A07 und maximiert unseren Gesamtnutzen.'\}} \\
            }
        }
    }
     \\ \\

    \theutterance \stepcounter{utterance}  
    & & \multicolumn{4}{p{0.6\linewidth}}{
        \cellcolor[rgb]{0.9,0.9,0.9}{
            \makecell[{{p{\linewidth}}}]{
                \texttt{\tiny{[P1$\langle$GM]}}
                \texttt{ABLEHNUNG: \{'C11', 'B16', 'A36', 'C13', 'B93', 'C34', 'B73', 'B05', 'A07'\}} \\
\\ 
\texttt{ARGUMENT: \{'Der Austausch von B33 gegen A07 verschlechtert das Gesamtergebnis erheblich. B33 bietet ein deutlich besseres Wert{-}Aufwand{-}Verhältnis als A07. Die 295 gesparten Budgetpunkte können nicht sinnvoll für andere hochwertige Gegenstände genutzt werden.'\}} \\
\\ 
\texttt{VORSCHLAG: \{'C11', 'B16', 'A36', 'C13', 'B93', 'C34', 'B73', 'B05', 'B33'\}} \\
\\ 
\texttt{ARGUMENT: \{'Mein ursprünglicher Vorschlag nutzt das Budget optimal mit 4564 Aufwandspunkten und behält die wertvolleren Gegenstände. B33 ist effizienter als A07 und maximiert unseren Gesamtnutzen.'\}} \\
            }
        }
    }
    & \\ \\

    \theutterance \stepcounter{utterance}  
    & \multicolumn{4}{p{0.6\linewidth}}{
        \cellcolor[rgb]{0.9,0.9,0.9}{
            \makecell[{{p{\linewidth}}}]{
                \texttt{\tiny{[P1$\rangle$GM]}}
                \texttt{STRATEGISCHE ÜBERLEGUNGEN: \{'Sein Vorschlag gibt mir 5943 Punkte, was sehr nah an meinem theoretischen Optimum liegt. Er ist hartnäckig bei diesem Vorschlag und es ist ein fairer Deal für beide Seiten. Weitere Verhandlungen könnten das Risiko bergen, dass er einen schlechteren Deal vorschlägt. Ich sollte akzeptieren.'\}} \\
\\ 
\texttt{ARGUMENT: \{'Nach sorgfältiger Überlegung erkenne ich, dass Ihr Vorschlag eine ausgewogene und effiziente Nutzung unseres Budgets darstellt.'\}} \\
\\ 
\texttt{ZUSTIMMUNG: \{'C11', 'B16', 'A36', 'C13', 'B93', 'C34', 'B73', 'B05', 'B33'\}} \\
            }
        }
    }
    & & \\ \\

    \theutterance \stepcounter{utterance}  
    & & & \multicolumn{2}{p{0.3\linewidth}}{
        \cellcolor[rgb]{0.9,0.9,0.9}{
            \makecell[{{p{\linewidth}}}]{
                \texttt{\tiny{[GM$|$GM]}}
                \texttt{game successful} \\
            }
        }
    }
    & & \\ \\

    \theutterance \stepcounter{utterance}  
    & & & \multicolumn{2}{p{0.3\linewidth}}{
        \cellcolor[rgb]{0.9,0.9,0.9}{
            \makecell[{{p{\linewidth}}}]{
                \texttt{\tiny{[GM$|$GM]}}
                \texttt{end game} \\
            }
        }
    }
    & & \\ \\

\end{supertabular}
}

\end{document}
