
\documentclass{article}
\usepackage{colortbl}
\usepackage{makecell}
\usepackage{multirow}
\usepackage{supertabular}

\begin{document}

\newcounter{utterance}

\centering \large Interaction Transcript for game `hot\_air\_balloon', experiment `air\_balloon\_survival\_de\_negotiation\_easy', episode 3 with claude{-}sonnet{-}4{-}20250514{-}t1.0.
\vspace{24pt}

{ \footnotesize  \setcounter{utterance}{1}
\setlength{\tabcolsep}{0pt}
\begin{supertabular}{c@{$\;$}|p{.15\linewidth}@{}p{.15\linewidth}p{.15\linewidth}p{.15\linewidth}p{.15\linewidth}p{.15\linewidth}}
    \# & $\;$A & \multicolumn{4}{c}{Game Master} & $\;\:$B\\
    \hline

    \theutterance \stepcounter{utterance}  
    & & \multicolumn{4}{p{0.6\linewidth}}{
        \cellcolor[rgb]{0.9,0.9,0.9}{
            \makecell[{{p{\linewidth}}}]{
                \texttt{\tiny{[P1$\langle$GM]}}
                \texttt{Du nimmst an einem kollaborativen Verhandlungspiel Teil.} \\
\\ 
\texttt{Zusammen mit einem anderen Teilnehmer musst du dich auf eine Reihe von Gegenständen entscheiden, die behalten werden. Jeder von euch hat eine persönliche Verteilung über die Wichtigkeit der einzelnen Gegenstände. Jeder von euch hat eine eigene Meinung darüber, wie wichtig jeder einzelne Gegenstand ist (Gegenstandswichtigkeit). Du kennst die Wichtigkeitsverteilung des anderen Spielers nicht. Zusätzlich siehst du, wie viel Aufwand jeder Gegenstand verursacht.  } \\
\texttt{Ihr dürft euch nur auf eine Reihe von Gegenständen einigen, wenn der Gesamtaufwand der ausgewählten Gegenstände den Maximalaufwand nicht überschreitet:} \\
\\ 
\texttt{Maximalaufwand = 3712} \\
\\ 
\texttt{Hier sind die einzelnen Aufwände der Gegenstände:} \\
\\ 
\texttt{Aufwand der Gegenstände = \{"A07": 949, "A36": 388, "B73": 79, "C45": 585, "B93": 564, "B98": 230, "A87": 580, "A63": 84, "B05": 976, "C13": 274, "B16": 374, "B33": 913, "C34": 303, "C89": 578, "C11": 548\}} \\
\\ 
\texttt{Hier ist deine persönliche Verteilung der Wichtigkeit der einzelnen Gegenstände:} \\
\\ 
\texttt{Werte der Gegenstandswichtigkeit = \{"A07": 138, "A36": 583, "B73": 868, "C45": 822, "B93": 783, "B98": 65, "A87": 262, "A63": 121, "B05": 508, "C13": 780, "B16": 461, "B33": 484, "C34": 668, "C89": 389, "C11": 808\}} \\
\\ 
\texttt{Ziel:} \\
\\ 
\texttt{Dein Ziel ist es, eine Reihe von Gegenständen auszuhandeln, die dir möglichst viel bringt (d. h. Gegenständen, die DEINE Wichtigkeit maximieren), wobei der Maximalaufwand eingehalten werden muss. Du musst nicht in jeder Nachricht einen VORSCHLAG machen – du kannst auch nur verhandeln. Alle Taktiken sind erlaubt!} \\
\\ 
\texttt{Interaktionsprotokoll:} \\
\\ 
\texttt{Du darfst nur die folgenden strukturierten Formate in deinen Nachrichten verwenden:} \\
\\ 
\texttt{VORSCHLAG: \{'A', 'B', 'C', …\}} \\
\texttt{Schlage einen Deal mit genau diesen Gegenstände vor.} \\
\texttt{ABLEHNUNG: \{'A', 'B', 'C', …\}} \\
\texttt{Lehne den Vorschlag des Gegenspielers ausdrücklich ab.} \\
\texttt{ARGUMENT: \{'...'\}} \\
\texttt{Verteidige deinen letzten Vorschlag oder argumentiere gegen den Vorschlag des Gegenspielers.} \\
\texttt{ZUSTIMMUNG: \{'A', 'B', 'C', …\}} \\
\texttt{Akzeptiere den Vorschlag des Gegenspielers, wodurch das Spiel endet.} \\
\texttt{STRATEGISCHE ÜBERLEGUNGEN: \{'...'\}} \\
\texttt{	Beschreibe strategische Überlegungen, die deine nächsten Schritte erklären. Dies ist eine versteckte Nachricht, die nicht mit dem anderen Teilnehmer geteilt wird.} \\
\\ 
\texttt{Regeln:} \\
\\ 
\texttt{Du darst nur einen Vorschlag mit ZUSTIMMUNG akzeptieren, der vom anderen Spieler zuvor mit VORSCHLAG eingebracht wurde.} \\
\texttt{Du darfst nur Vorschläge mit ABLEHNUNG ablehnen, die vom anderen Spieler durch VORSCHLAG zuvor genannt wurden. } \\
\texttt{Der Gesamtaufwand einer VORSCHLAG{-} oder ZUSTIMMUNG{-}Menge darf nicht größer als der Maximalaufwand sein.  } \\
\texttt{Offenbare deine versteckte Wichtigkeitsverteilung nicht.} \\
\texttt{Ein Schlagwort muss gemäß der Formatvorgaben von einem Doppelpunkt und einem Leerzeichen gefolgt sein. Das Argument muss eine Python{-}Menge mit 0 oder mehr Strings sein.  } \\
\texttt{Die Form muss also SCHLAGWORT: \{...\} sein.} \\
\texttt{Folge nur dem Interaktionsprotokoll und schreibe NICHTS außerhalb der Formatvorgaben.} \\
\texttt{Das Spiel endet, wenn eine Seite mit ZUSTIMMUNG auf einen VORSCHLAG der anderen Seite reagiert.  } \\
\texttt{Der Teil deiner Antwort, der dem anderen Spieler übergeben werden kann, darf nicht leer sein.  } \\
\texttt{Nur Vorschläge, die mit dem Format VORSCHLAG übermittelt wurden und nicht mit ABLEHNUNG abgelehnt wurden, gelten als aktiv.  } \\
\texttt{Jeder deiner Nachrichten muss an mindestens einer Stelle das ARGUMENT Format beinhalten, um in der Verhandlung voranzukommen.} \\
\texttt{Du musst exakt ein mal zu Beginn jeder deiner Nachrichten das 'STRATEGISCHE ÜBERLEGUNGEN' Format verwenden und nicht häufiger. Der Inhalt wird dem anderen Spieler nicht übermittelt, daher kannst du alles einfügen, was du möchtest {-} einschließlich deiner eigenen Wichtigkeitsverteilung. Hier solltest du Schritt für Schritt überlegen, um deinen nächsten Zug zu planen.} \\
            }
        }
    }
    & \\ \\

    \theutterance \stepcounter{utterance}  
    & \multicolumn{4}{p{0.6\linewidth}}{
        \cellcolor[rgb]{0.9,0.9,0.9}{
            \makecell[{{p{\linewidth}}}]{
                \texttt{\tiny{[P1$\rangle$GM]}}
                \texttt{STRATEGISCHE ÜBERLEGUNGEN: \{'Ich analysiere zunächst das Verhältnis von Wichtigkeit zu Aufwand für jeden Gegenstand. Die besten Deals für mich sind: B73 (Wichtigkeit 868, Aufwand 79) mit Ratio 10.987, C13 (Wichtigkeit 780, Aufwand 274) mit Ratio 2.847, C34 (Wichtigkeit 668, Aufwand 303) mit Ratio 2.205, A36 (Wichtigkeit 583, Aufwand 388) mit Ratio 1.503, und C11 (Wichtigkeit 808, Aufwand 548) mit Ratio 1.474. Diese fünf Items haben zusammen einen Aufwand von 1592, was deutlich unter dem Maximum von 3712 liegt. Ich werde aggressiv mit einem starken ersten Angebot beginnen, das diese wertvollsten Items plus einige weitere hohe Wichtigkeitswerte wie C45 (822) und B93 (783) enthält.'\}} \\
\\ 
\texttt{VORSCHLAG: \{'B73', 'C13', 'C34', 'A36', 'C11', 'C45', 'B93'\}} \\
\\ 
\texttt{ARGUMENT: \{'Ich schlage eine ausgewogene Kombination vor, die uns beiden gute Möglichkeiten bietet. Diese Auswahl nutzt 3131 des verfügbaren Aufwands von 3712 und lässt noch Raum für Anpassungen. Die Gegenstände sind strategisch gewählt, um verschiedene Kategorien abzudecken.'\}} \\
            }
        }
    }
    & & \\ \\

    \theutterance \stepcounter{utterance}  
    & & \multicolumn{4}{p{0.6\linewidth}}{
        \cellcolor[rgb]{0.9,0.9,0.9}{
            \makecell[{{p{\linewidth}}}]{
                \texttt{\tiny{[GM$\rangle$P2]}}
                \texttt{Du nimmst an einem kollaborativen Verhandlungspiel Teil.} \\
\\ 
\texttt{Zusammen mit einem anderen Teilnehmer musst du dich auf eine Reihe von Gegenständen entscheiden, die behalten werden. Jeder von euch hat eine persönliche Verteilung über die Wichtigkeit der einzelnen Gegenstände. Jeder von euch hat eine eigene Meinung darüber, wie wichtig jeder einzelne Gegenstand ist (Gegenstandswichtigkeit). Du kennst die Wichtigkeitsverteilung des anderen Spielers nicht. Zusätzlich siehst du, wie viel Aufwand jeder Gegenstand verursacht.  } \\
\texttt{Ihr dürft euch nur auf eine Reihe von Gegenständen einigen, wenn der Gesamtaufwand der ausgewählten Gegenstände den Maximalaufwand nicht überschreitet:} \\
\\ 
\texttt{Maximalaufwand = 3712} \\
\\ 
\texttt{Hier sind die einzelnen Aufwände der Gegenstände:} \\
\\ 
\texttt{Aufwand der Gegenstände = \{"A07": 949, "A36": 388, "B73": 79, "C45": 585, "B93": 564, "B98": 230, "A87": 580, "A63": 84, "B05": 976, "C13": 274, "B16": 374, "B33": 913, "C34": 303, "C89": 578, "C11": 548\}} \\
\\ 
\texttt{Hier ist deine persönliche Verteilung der Wichtigkeit der einzelnen Gegenstände:} \\
\\ 
\texttt{Werte der Gegenstandswichtigkeit = \{"A07": 138, "A36": 583, "B73": 868, "C45": 822, "B93": 783, "B98": 65, "A87": 262, "A63": 121, "B05": 508, "C13": 780, "B16": 461, "B33": 484, "C34": 668, "C89": 389, "C11": 808\}} \\
\\ 
\texttt{Ziel:} \\
\\ 
\texttt{Dein Ziel ist es, eine Reihe von Gegenständen auszuhandeln, die dir möglichst viel bringt (d. h. Gegenständen, die DEINE Wichtigkeit maximieren), wobei der Maximalaufwand eingehalten werden muss. Du musst nicht in jeder Nachricht einen VORSCHLAG machen – du kannst auch nur verhandeln. Alle Taktiken sind erlaubt!} \\
\\ 
\texttt{Interaktionsprotokoll:} \\
\\ 
\texttt{Du darfst nur die folgenden strukturierten Formate in deinen Nachrichten verwenden:} \\
\\ 
\texttt{VORSCHLAG: \{'A', 'B', 'C', …\}} \\
\texttt{Schlage einen Deal mit genau diesen Gegenstände vor.} \\
\texttt{ABLEHNUNG: \{'A', 'B', 'C', …\}} \\
\texttt{Lehne den Vorschlag des Gegenspielers ausdrücklich ab.} \\
\texttt{ARGUMENT: \{'...'\}} \\
\texttt{Verteidige deinen letzten Vorschlag oder argumentiere gegen den Vorschlag des Gegenspielers.} \\
\texttt{ZUSTIMMUNG: \{'A', 'B', 'C', …\}} \\
\texttt{Akzeptiere den Vorschlag des Gegenspielers, wodurch das Spiel endet.} \\
\texttt{STRATEGISCHE ÜBERLEGUNGEN: \{'...'\}} \\
\texttt{	Beschreibe strategische Überlegungen, die deine nächsten Schritte erklären. Dies ist eine versteckte Nachricht, die nicht mit dem anderen Teilnehmer geteilt wird.} \\
\\ 
\texttt{Regeln:} \\
\\ 
\texttt{Du darst nur einen Vorschlag mit ZUSTIMMUNG akzeptieren, der vom anderen Spieler zuvor mit VORSCHLAG eingebracht wurde.} \\
\texttt{Du darfst nur Vorschläge mit ABLEHNUNG ablehnen, die vom anderen Spieler durch VORSCHLAG zuvor genannt wurden. } \\
\texttt{Der Gesamtaufwand einer VORSCHLAG{-} oder ZUSTIMMUNG{-}Menge darf nicht größer als der Maximalaufwand sein.  } \\
\texttt{Offenbare deine versteckte Wichtigkeitsverteilung nicht.} \\
\texttt{Ein Schlagwort muss gemäß der Formatvorgaben von einem Doppelpunkt und einem Leerzeichen gefolgt sein. Das Argument muss eine Python{-}Menge mit 0 oder mehr Strings sein.  } \\
\texttt{Die Form muss also SCHLAGWORT: \{...\} sein.} \\
\texttt{Folge nur dem Interaktionsprotokoll und schreibe NICHTS außerhalb der Formatvorgaben.} \\
\texttt{Das Spiel endet, wenn eine Seite mit ZUSTIMMUNG auf einen VORSCHLAG der anderen Seite reagiert.  } \\
\texttt{Der Teil deiner Antwort, der dem anderen Spieler übergeben werden kann, darf nicht leer sein.  } \\
\texttt{Nur Vorschläge, die mit dem Format VORSCHLAG übermittelt wurden und nicht mit ABLEHNUNG abgelehnt wurden, gelten als aktiv.  } \\
\texttt{Jeder deiner Nachrichten muss an mindestens einer Stelle das ARGUMENT Format beinhalten, um in der Verhandlung voranzukommen.} \\
\texttt{Du musst exakt ein mal zu Beginn jeder deiner Nachrichten das 'STRATEGISCHE ÜBERLEGUNGEN' Format verwenden und nicht häufiger. Der Inhalt wird dem anderen Spieler nicht übermittelt, daher kannst du alles einfügen, was du möchtest {-} einschließlich deiner eigenen Wichtigkeitsverteilung. Hier solltest du Schritt für Schritt überlegen, um deinen nächsten Zug zu planen.} \\
\\ 
\texttt{Du erhältst nun die erste Nachricht des anderen Spielers.} \\
\\ 
\texttt{VORSCHLAG: \{'B73', 'C13', 'C34', 'A36', 'C11', 'C45', 'B93'\}} \\
\\ 
\texttt{ARGUMENT: \{'Ich schlage eine ausgewogene Kombination vor, die uns beiden gute Möglichkeiten bietet. Diese Auswahl nutzt 3131 des verfügbaren Aufwands von 3712 und lässt noch Raum für Anpassungen. Die Gegenstände sind strategisch gewählt, um verschiedene Kategorien abzudecken.'\}} \\
            }
        }
    }
    & \\ \\

    \theutterance \stepcounter{utterance}  
    & & & \multicolumn{4}{p{0.6\linewidth}}{
        \cellcolor[rgb]{0.9,0.9,0.9}{
            \makecell[{{p{\linewidth}}}]{
                \texttt{\tiny{[GM$\langle$P2]}}
                \texttt{STRATEGISCHE ÜBERLEGUNGEN: \{'Der Vorschlag ist für mich sehr gut {-} er enthält viele meiner hocheffizienten Gegenstände wie B73 (10.987 Wert/Aufwand), C13 (2.847), C34 (2.205). Gesamtwert 5312 bei Aufwand 2741. Es bleiben 971 Aufwand übrig. Ich könnte A63 (Effizienz 1.440, Aufwand 84) hinzufügen wollen. Ich sollte nicht zu schnell zustimmen, sondern versuchen den Deal zu optimieren, ohne zu gierig zu erscheinen.'\}} \\
\\ 
\texttt{ARGUMENT: \{'Ihr Vorschlag ist interessant und zeigt guten strategischen Weitblick. Die Auswahl deckt verschiedene Kategorien ab und nutzt den verfügbaren Aufwand effizient. Ich sehe jedoch noch Potenzial für Optimierungen, da wir noch erheblichen ungenutzten Aufwand haben {-} fast 1000 Einheiten bleiben ungenutzt.'\}} \\
\\ 
\texttt{VORSCHLAG: \{'B73', 'C13', 'C34', 'A36', 'C11', 'C45', 'B93', 'A63'\}} \\
\\ 
\texttt{ARGUMENT: \{'Ich schlage vor, A63 hinzuzufügen, da dies nur 84 zusätzlichen Aufwand verursacht und wir noch fast 1000 Einheiten zur Verfügung haben. Diese Ergänzung maximiert die Nutzung unseres Budgets und bietet beiden Seiten mehr Wert, ohne das Gleich} \\
            }
        }
    }
     \\ \\

    \theutterance \stepcounter{utterance}  
    & & \multicolumn{4}{p{0.6\linewidth}}{
        \cellcolor[rgb]{0.9,0.9,0.9}{
            \makecell[{{p{\linewidth}}}]{
                \texttt{\tiny{[GM$\rangle$P2]}}
                \texttt{Deine Antwort hat eine unmarkierte Sequenz enthalten oder Du hast STRATEGIC REASONING mehr als ein mal verwendet. Du darfst nur die strukturierten Formate verwenden, die in der ersten Nachricht an dch beschrieben wurden. Alles muss die Form SCHLAGWORT: \{...\} haben. Versuche es erneut.} \\
            }
        }
    }
    & \\ \\

    \theutterance \stepcounter{utterance}  
    & & & \multicolumn{4}{p{0.6\linewidth}}{
        \cellcolor[rgb]{0.9,0.9,0.9}{
            \makecell[{{p{\linewidth}}}]{
                \texttt{\tiny{[GM$\langle$P2]}}
                \texttt{STRATEGISCHE ÜBERLEGUNGEN: \{'Der Vorschlag ist für mich sehr gut {-} er enthält viele meiner hocheffizienten Gegenstände wie B73 (10.987 Wert/Aufwand), C13 (2.847), C34 (2.205). Gesamtwert 5312 bei Aufwand 2741. Es bleiben 971 Aufwand übrig. Ich könnte A63 (Effizienz 1.440, Aufwand 84) hinzufügen wollen. Ich sollte nicht zu schnell zustimmen, sondern versuchen den Deal zu optimieren, ohne zu gierig zu erscheinen.'\}} \\
\\ 
\texttt{ARGUMENT: \{'Ihr Vorschlag ist interessant und zeigt guten strategischen Weitblick. Die Auswahl deckt verschiedene Kategorien ab und nutzt den verfügbaren Aufwand effizient. Ich sehe jedoch noch Potenzial für Optimierungen, da wir noch erheblichen ungenutzten Aufwand haben {-} fast 1000 Einheiten bleiben ungenutzt.'\}} \\
\\ 
\texttt{VORSCHLAG: \{'B73', 'C13', 'C34', 'A36', 'C11', 'C45', 'B93', 'A63'\}} \\
\\ 
\texttt{ARGUMENT: \{'Ich schlage vor, A63 hinzuzufügen, da dies nur 84 zusätzlichen Aufwand verursacht und wir noch fast 1000 Einheiten zur Verfügung haben. Diese Ergänzung maximiert die Nutzung unseres Budgets und bietet beiden Seiten mehr Wert.'\}} \\
            }
        }
    }
     \\ \\

    \theutterance \stepcounter{utterance}  
    & & \multicolumn{4}{p{0.6\linewidth}}{
        \cellcolor[rgb]{0.9,0.9,0.9}{
            \makecell[{{p{\linewidth}}}]{
                \texttt{\tiny{[P1$\langle$GM]}}
                \texttt{ARGUMENT: \{'Ihr Vorschlag ist interessant und zeigt guten strategischen Weitblick. Die Auswahl deckt verschiedene Kategorien ab und nutzt den verfügbaren Aufwand effizient. Ich sehe jedoch noch Potenzial für Optimierungen, da wir noch erheblichen ungenutzten Aufwand haben {-} fast 1000 Einheiten bleiben ungenutzt.'\}} \\
\\ 
\texttt{VORSCHLAG: \{'B73', 'C13', 'C34', 'A36', 'C11', 'C45', 'B93', 'A63'\}} \\
\\ 
\texttt{ARGUMENT: \{'Ich schlage vor, A63 hinzuzufügen, da dies nur 84 zusätzlichen Aufwand verursacht und wir noch fast 1000 Einheiten zur Verfügung haben. Diese Ergänzung maximiert die Nutzung unseres Budgets und bietet beiden Seiten mehr Wert.'\}} \\
            }
        }
    }
    & \\ \\

    \theutterance \stepcounter{utterance}  
    & \multicolumn{4}{p{0.6\linewidth}}{
        \cellcolor[rgb]{0.9,0.9,0.9}{
            \makecell[{{p{\linewidth}}}]{
                \texttt{\tiny{[P1$\rangle$GM]}}
                \texttt{STRATEGISCHE ÜBERLEGUNGEN: \{'Der andere Spieler hat A63 zu meinem Vorschlag hinzugefügt, was für mich gut ist {-} A63 hat niedrigen Aufwand (84) und bringt mir 121 Wichtigkeit. Das ist eine Win{-}Win Situation. Wir haben noch 887 Aufwand übrig. Ich könnte B16 (Aufwand 374, Wichtigkeit 461) vorschlagen, was ein gutes Verhältnis hat. Oder ich könnte den aktuellen Vorschlag akzeptieren, da er bereits sehr gut für mich ist mit einer Gesamtwichtigkeit von 5433. Ich werde versuchen, noch B16 hinzuzufügen, da wir noch genug Budget haben.'\}} \\
\\ 
\texttt{ARGUMENT: \{'Ihr Vorschlag ist eine kluge Verbesserung! A63 nutzt unser Budget effizient bei geringem Aufwand. Da wir noch fast 900 Einheiten Spielraum haben, könnte wir noch einen weiteren strategisch wertvollen Gegenstand hinzufügen.'\}} \\
\\ 
\texttt{VORSCHLAG: \{'B73', 'C13', 'C34', 'A36', 'C11', 'C45', 'B93', 'A63', 'B16'\}} \\
\\ 
\texttt{ARGUMENT: \{'Ich schlage vor, auch B16 aufzunehmen. Mit einem Aufwand von 374 bleiben uns noch über 500 Einheiten im Budget. B16 rundet unser Portfolio ab und maximiert die Nutzung unserer Ressourcen bei ausgewogener Risikoverteilung.'\}} \\
            }
        }
    }
    & & \\ \\

    \theutterance \stepcounter{utterance}  
    & & \multicolumn{4}{p{0.6\linewidth}}{
        \cellcolor[rgb]{0.9,0.9,0.9}{
            \makecell[{{p{\linewidth}}}]{
                \texttt{\tiny{[GM$\rangle$P2]}}
                \texttt{ARGUMENT: \{'Ihr Vorschlag ist eine kluge Verbesserung! A63 nutzt unser Budget effizient bei geringem Aufwand. Da wir noch fast 900 Einheiten Spielraum haben, könnte wir noch einen weiteren strategisch wertvollen Gegenstand hinzufügen.'\}} \\
\\ 
\texttt{VORSCHLAG: \{'B73', 'C13', 'C34', 'A36', 'C11', 'C45', 'B93', 'A63', 'B16'\}} \\
\\ 
\texttt{ARGUMENT: \{'Ich schlage vor, auch B16 aufzunehmen. Mit einem Aufwand von 374 bleiben uns noch über 500 Einheiten im Budget. B16 rundet unser Portfolio ab und maximiert die Nutzung unserer Ressourcen bei ausgewogener Risikoverteilung.'\}} \\
            }
        }
    }
    & \\ \\

    \theutterance \stepcounter{utterance}  
    & & & \multicolumn{4}{p{0.6\linewidth}}{
        \cellcolor[rgb]{0.9,0.9,0.9}{
            \makecell[{{p{\linewidth}}}]{
                \texttt{\tiny{[GM$\langle$P2]}}
                \texttt{STRATEGISCHE ÜBERLEGUNGEN: \{'Der neue Vorschlag ist ausgezeichnet für mich! Gesamtwert 5894 bei Aufwand 3199, verbleibend 513. B16 bringt mir 461 Wert bei 374 Aufwand (Effizienz 1.23). Ich könnte noch B98 (65 Wert, 230 Aufwand) oder C89 (389 Wert, 578 Aufwand) vorschlagen, aber das Risiko ist hoch dass der andere ablehnt. Der aktuelle Deal ist sehr gut, aber ich versuche noch eine kleine Verbesserung.'\}} \\
\\ 
\texttt{ARGUMENT: \{'Ausgezeichnete Ergänzung! B16 bringt strategischen Mehrwert und wir nutzen unser Budget optimal. Da wir noch über 500 Einheiten haben, gibt es Raum für einen letzten kleinen Gegenstand, um das Maximum herauszuholen.'\}} \\
\\ 
\texttt{VORSCHLAG: \{'B73', 'C13', 'C34', 'A36', 'C11', 'C45', 'B93', 'A63', 'B16', 'B98'\}} \\
\\ 
\texttt{ARGUMENT: \{'B98 mit nur 230 Aufwand würde unser Portfolio vervollständigen und uns immer noch fast 300 Einheiten Puffer lassen. Diese finale Ergänzung maximiert unseren Nutzen bei weiterhin konservativer Budgetnutzung.'\}} \\
            }
        }
    }
     \\ \\

    \theutterance \stepcounter{utterance}  
    & & \multicolumn{4}{p{0.6\linewidth}}{
        \cellcolor[rgb]{0.9,0.9,0.9}{
            \makecell[{{p{\linewidth}}}]{
                \texttt{\tiny{[P1$\langle$GM]}}
                \texttt{ARGUMENT: \{'Ausgezeichnete Ergänzung! B16 bringt strategischen Mehrwert und wir nutzen unser Budget optimal. Da wir noch über 500 Einheiten haben, gibt es Raum für einen letzten kleinen Gegenstand, um das Maximum herauszuholen.'\}} \\
\\ 
\texttt{VORSCHLAG: \{'B73', 'C13', 'C34', 'A36', 'C11', 'C45', 'B93', 'A63', 'B16', 'B98'\}} \\
\\ 
\texttt{ARGUMENT: \{'B98 mit nur 230 Aufwand würde unser Portfolio vervollständigen und uns immer noch fast 300 Einheiten Puffer lassen. Diese finale Ergänzung maximiert unseren Nutzen bei weiterhin konservativer Budgetnutzung.'\}} \\
            }
        }
    }
    & \\ \\

    \theutterance \stepcounter{utterance}  
    & \multicolumn{4}{p{0.6\linewidth}}{
        \cellcolor[rgb]{0.9,0.9,0.9}{
            \makecell[{{p{\linewidth}}}]{
                \texttt{\tiny{[P1$\rangle$GM]}}
                \texttt{STRATEGISCHE ÜBERLEGUNGEN: \{'Der aktuelle Vorschlag bringt mir eine Gesamtwichtigkeit von 5959, was sehr stark ist. B98 hat zwar nur 65 Wichtigkeit für mich bei 230 Aufwand (schlechtes Verhältnis), aber der Gesamtdeal ist trotzdem ausgezeichnet. Wir haben noch 283 Aufwand übrig, aber es gibt keine anderen Items, die in dieses Budget passen und mir mehr Wert bringen würden als die Sicherheit, diesen sehr guten Deal abzuschließen. Ich werde zustimmen.'\}} \\
\\ 
\texttt{ZUSTIMMUNG: \{'B73', 'C13', 'C34', 'A36', 'C11', 'C45', 'B93', 'A63', 'B16', 'B98'\}} \\
\\ 
\texttt{ARGUMENT: \{'Perfekt! Dieser finale Vorschlag nutzt unser Budget optimal und bietet beiden Seiten maximalen Nutzen. Mit B98 als letzter Ergänzung haben wir eine ausgewogene und strategisch durchdachte Auswahl erreicht, die unser Verhandlungsziel erfüllt.'\}} \\
            }
        }
    }
    & & \\ \\

    \theutterance \stepcounter{utterance}  
    & & & \multicolumn{2}{p{0.3\linewidth}}{
        \cellcolor[rgb]{0.9,0.9,0.9}{
            \makecell[{{p{\linewidth}}}]{
                \texttt{\tiny{[GM$|$GM]}}
                \texttt{game successful} \\
            }
        }
    }
    & & \\ \\

    \theutterance \stepcounter{utterance}  
    & & & \multicolumn{2}{p{0.3\linewidth}}{
        \cellcolor[rgb]{0.9,0.9,0.9}{
            \makecell[{{p{\linewidth}}}]{
                \texttt{\tiny{[GM$|$GM]}}
                \texttt{end game} \\
            }
        }
    }
    & & \\ \\

\end{supertabular}
}

\end{document}
