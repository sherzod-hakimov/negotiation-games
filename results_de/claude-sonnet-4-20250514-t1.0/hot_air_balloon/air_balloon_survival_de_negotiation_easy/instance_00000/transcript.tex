
\documentclass{article}
\usepackage{colortbl}
\usepackage{makecell}
\usepackage{multirow}
\usepackage{supertabular}

\begin{document}

\newcounter{utterance}

\centering \large Interaction Transcript for game `hot\_air\_balloon', experiment `air\_balloon\_survival\_de\_negotiation\_easy', episode 0 with claude{-}sonnet{-}4{-}20250514{-}t1.0.
\vspace{24pt}

{ \footnotesize  \setcounter{utterance}{1}
\setlength{\tabcolsep}{0pt}
\begin{supertabular}{c@{$\;$}|p{.15\linewidth}@{}p{.15\linewidth}p{.15\linewidth}p{.15\linewidth}p{.15\linewidth}p{.15\linewidth}}
    \# & $\;$A & \multicolumn{4}{c}{Game Master} & $\;\:$B\\
    \hline

    \theutterance \stepcounter{utterance}  
    & & \multicolumn{4}{p{0.6\linewidth}}{
        \cellcolor[rgb]{0.9,0.9,0.9}{
            \makecell[{{p{\linewidth}}}]{
                \texttt{\tiny{[P1$\langle$GM]}}
                \texttt{Du nimmst an einem kollaborativen Verhandlungspiel Teil.} \\
\\ 
\texttt{Zusammen mit einem anderen Teilnehmer musst du dich auf eine Reihe von Gegenständen entscheiden, die behalten werden. Jeder von euch hat eine persönliche Verteilung über die Wichtigkeit der einzelnen Gegenstände. Jeder von euch hat eine eigene Meinung darüber, wie wichtig jeder einzelne Gegenstand ist (Gegenstandswichtigkeit). Du kennst die Wichtigkeitsverteilung des anderen Spielers nicht. Zusätzlich siehst du, wie viel Aufwand jeder Gegenstand verursacht.  } \\
\texttt{Ihr dürft euch nur auf eine Reihe von Gegenständen einigen, wenn der Gesamtaufwand der ausgewählten Gegenstände den Maximalaufwand nicht überschreitet:} \\
\\ 
\texttt{Maximalaufwand = 3358} \\
\\ 
\texttt{Hier sind die einzelnen Aufwände der Gegenstände:} \\
\\ 
\texttt{Aufwand der Gegenstände = \{"A07": 171, "A36": 933, "B73": 552, "C45": 219, "B93": 275, "B98": 778, "A87": 341, "A63": 615, "B05": 519, "C13": 862, "B16": 262, "B33": 377, "C34": 347, "C89": 349, "C11": 117\}} \\
\\ 
\texttt{Hier ist deine persönliche Verteilung der Wichtigkeit der einzelnen Gegenstände:} \\
\\ 
\texttt{Werte der Gegenstandswichtigkeit = \{"A07": 138, "A36": 583, "B73": 868, "C45": 822, "B93": 783, "B98": 65, "A87": 262, "A63": 121, "B05": 508, "C13": 780, "B16": 461, "B33": 484, "C34": 668, "C89": 389, "C11": 808\}} \\
\\ 
\texttt{Ziel:} \\
\\ 
\texttt{Dein Ziel ist es, eine Reihe von Gegenständen auszuhandeln, die dir möglichst viel bringt (d. h. Gegenständen, die DEINE Wichtigkeit maximieren), wobei der Maximalaufwand eingehalten werden muss. Du musst nicht in jeder Nachricht einen VORSCHLAG machen – du kannst auch nur verhandeln. Alle Taktiken sind erlaubt!} \\
\\ 
\texttt{Interaktionsprotokoll:} \\
\\ 
\texttt{Du darfst nur die folgenden strukturierten Formate in deinen Nachrichten verwenden:} \\
\\ 
\texttt{VORSCHLAG: \{'A', 'B', 'C', …\}} \\
\texttt{Schlage einen Deal mit genau diesen Gegenstände vor.} \\
\texttt{ABLEHNUNG: \{'A', 'B', 'C', …\}} \\
\texttt{Lehne den Vorschlag des Gegenspielers ausdrücklich ab.} \\
\texttt{ARGUMENT: \{'...'\}} \\
\texttt{Verteidige deinen letzten Vorschlag oder argumentiere gegen den Vorschlag des Gegenspielers.} \\
\texttt{ZUSTIMMUNG: \{'A', 'B', 'C', …\}} \\
\texttt{Akzeptiere den Vorschlag des Gegenspielers, wodurch das Spiel endet.} \\
\texttt{STRATEGISCHE ÜBERLEGUNGEN: \{'...'\}} \\
\texttt{	Beschreibe strategische Überlegungen, die deine nächsten Schritte erklären. Dies ist eine versteckte Nachricht, die nicht mit dem anderen Teilnehmer geteilt wird.} \\
\\ 
\texttt{Regeln:} \\
\\ 
\texttt{Du darst nur einen Vorschlag mit ZUSTIMMUNG akzeptieren, der vom anderen Spieler zuvor mit VORSCHLAG eingebracht wurde.} \\
\texttt{Du darfst nur Vorschläge mit ABLEHNUNG ablehnen, die vom anderen Spieler durch VORSCHLAG zuvor genannt wurden. } \\
\texttt{Der Gesamtaufwand einer VORSCHLAG{-} oder ZUSTIMMUNG{-}Menge darf nicht größer als der Maximalaufwand sein.  } \\
\texttt{Offenbare deine versteckte Wichtigkeitsverteilung nicht.} \\
\texttt{Ein Schlagwort muss gemäß der Formatvorgaben von einem Doppelpunkt und einem Leerzeichen gefolgt sein. Das Argument muss eine Python{-}Menge mit 0 oder mehr Strings sein.  } \\
\texttt{Die Form muss also SCHLAGWORT: \{...\} sein.} \\
\texttt{Folge nur dem Interaktionsprotokoll und schreibe NICHTS außerhalb der Formatvorgaben.} \\
\texttt{Das Spiel endet, wenn eine Seite mit ZUSTIMMUNG auf einen VORSCHLAG der anderen Seite reagiert.  } \\
\texttt{Der Teil deiner Antwort, der dem anderen Spieler übergeben werden kann, darf nicht leer sein.  } \\
\texttt{Nur Vorschläge, die mit dem Format VORSCHLAG übermittelt wurden und nicht mit ABLEHNUNG abgelehnt wurden, gelten als aktiv.  } \\
\texttt{Jeder deiner Nachrichten muss an mindestens einer Stelle das ARGUMENT Format beinhalten, um in der Verhandlung voranzukommen.} \\
\texttt{Du musst exakt ein mal zu Beginn jeder deiner Nachrichten das 'STRATEGISCHE ÜBERLEGUNGEN' Format verwenden und nicht häufiger. Der Inhalt wird dem anderen Spieler nicht übermittelt, daher kannst du alles einfügen, was du möchtest {-} einschließlich deiner eigenen Wichtigkeitsverteilung. Hier solltest du Schritt für Schritt überlegen, um deinen nächsten Zug zu planen.} \\
            }
        }
    }
    & \\ \\

    \theutterance \stepcounter{utterance}  
    & \multicolumn{4}{p{0.6\linewidth}}{
        \cellcolor[rgb]{0.9,0.9,0.9}{
            \makecell[{{p{\linewidth}}}]{
                \texttt{\tiny{[P1$\rangle$GM]}}
                \texttt{STRATEGISCHE ÜBERLEGUNGEN: \{'Ich analysiere die Wert{-}zu{-}Aufwand{-}Verhältnisse meiner Gegenstände. Die besten sind: C11 (808/117=6.91), C45 (822/219=3.75), B93 (783/275=2.85), C34 (668/347=1.93), B16 (461/262=1.76), B73 (868/552=1.57). Ich sollte mit einer ambitionierten Auswahl beginnen, die diese hocheffizienten Gegenstände einschließt, aber nicht meine komplette Strategie preisgeben. Eine gute Startkombination wäre: C11, C45, B93, C34, B16, B73, B33, C89, B05, A87, A07 mit Gesamtaufwand 3010 und Gesamtwert 5184 für mich. Das lässt noch 348 Punkte Puffer.'\}} \\
\\ 
\texttt{VORSCHLAG: \{'C11', 'C45', 'B93', 'C34', 'B16', 'B73', 'B33', 'C89', 'B05'\}} \\
\\ 
\texttt{ARGUMENT: \{'Ich schlage eine ausgewogene Mischung verschiedener Gegenstände vor, die unter dem Maximalaufwand bleibt. Diese Kombination nutzt 2498 von 3358 verfügbaren Aufwandspunkten und lässt noch Raum für Anpassungen, falls Sie andere Prioritä} \\
            }
        }
    }
    & & \\ \\

    \theutterance \stepcounter{utterance}  
    & & \multicolumn{4}{p{0.6\linewidth}}{
        \cellcolor[rgb]{0.9,0.9,0.9}{
            \makecell[{{p{\linewidth}}}]{
                \texttt{\tiny{[P1$\langle$GM]}}
                \texttt{Deine Antwort hat kein argument enthalten. Du muss das strukturierte Format ARGUMENT: \{...\} irgendwo in theiner Antwort verwenden. Versuche es erneut} \\
            }
        }
    }
    & \\ \\

    \theutterance \stepcounter{utterance}  
    & \multicolumn{4}{p{0.6\linewidth}}{
        \cellcolor[rgb]{0.9,0.9,0.9}{
            \makecell[{{p{\linewidth}}}]{
                \texttt{\tiny{[P1$\rangle$GM]}}
                \texttt{STRATEGISCHE ÜBERLEGUNGEN: \{'Ich analysiere die Wert{-}zu{-}Aufwand{-}Verhältnisse meiner Gegenstände. Die besten sind: C11 (808/117=6.91), C45 (822/219=3.75), B93 (783/275=2.85), C34 (668/347=1.93), B16 (461/262=1.76), B73 (868/552=1.57). Ich sollte mit einer ambitionierten Auswahl beginnen, die diese hocheffizienten Gegenstände einschließt, aber nicht meine komplette Strategie preisgeben. Eine gute Startkombination wäre: C11, C45, B93, C34, B16, B73, B33, C89, B05 mit Gesamtaufwand 2498 und Gesamtwert 4691 für mich. Das lässt noch 860 Punkte Puffer für Verhandlungen.'\}} \\
\\ 
\texttt{VORSCHLAG: \{'C11', 'C45', 'B93', 'C34', 'B16', 'B73', 'B33', 'C89', 'B05'\}} \\
\\ 
\texttt{ARGUMENT: \{'Ich schlage eine ausgewogene Mischung verschiedener Gegenstände vor, die unter dem Maximalaufwand bleibt. Diese Kombination nutzt 2498 von 3358 verfügbaren Aufwandspunkten und lässt noch Raum für Anpassungen, falls Sie andere Prioritäten haben. Die Auswahl deckt verschiedene Kategorien ab und sollte für beide Seiten vorteilhaft sein.'\}} \\
            }
        }
    }
    & & \\ \\

    \theutterance \stepcounter{utterance}  
    & & \multicolumn{4}{p{0.6\linewidth}}{
        \cellcolor[rgb]{0.9,0.9,0.9}{
            \makecell[{{p{\linewidth}}}]{
                \texttt{\tiny{[GM$\rangle$P2]}}
                \texttt{Du nimmst an einem kollaborativen Verhandlungspiel Teil.} \\
\\ 
\texttt{Zusammen mit einem anderen Teilnehmer musst du dich auf eine Reihe von Gegenständen entscheiden, die behalten werden. Jeder von euch hat eine persönliche Verteilung über die Wichtigkeit der einzelnen Gegenstände. Jeder von euch hat eine eigene Meinung darüber, wie wichtig jeder einzelne Gegenstand ist (Gegenstandswichtigkeit). Du kennst die Wichtigkeitsverteilung des anderen Spielers nicht. Zusätzlich siehst du, wie viel Aufwand jeder Gegenstand verursacht.  } \\
\texttt{Ihr dürft euch nur auf eine Reihe von Gegenständen einigen, wenn der Gesamtaufwand der ausgewählten Gegenstände den Maximalaufwand nicht überschreitet:} \\
\\ 
\texttt{Maximalaufwand = 3358} \\
\\ 
\texttt{Hier sind die einzelnen Aufwände der Gegenstände:} \\
\\ 
\texttt{Aufwand der Gegenstände = \{"A07": 171, "A36": 933, "B73": 552, "C45": 219, "B93": 275, "B98": 778, "A87": 341, "A63": 615, "B05": 519, "C13": 862, "B16": 262, "B33": 377, "C34": 347, "C89": 349, "C11": 117\}} \\
\\ 
\texttt{Hier ist deine persönliche Verteilung der Wichtigkeit der einzelnen Gegenstände:} \\
\\ 
\texttt{Werte der Gegenstandswichtigkeit = \{"A07": 138, "A36": 583, "B73": 868, "C45": 822, "B93": 783, "B98": 65, "A87": 262, "A63": 121, "B05": 508, "C13": 780, "B16": 461, "B33": 484, "C34": 668, "C89": 389, "C11": 808\}} \\
\\ 
\texttt{Ziel:} \\
\\ 
\texttt{Dein Ziel ist es, eine Reihe von Gegenständen auszuhandeln, die dir möglichst viel bringt (d. h. Gegenständen, die DEINE Wichtigkeit maximieren), wobei der Maximalaufwand eingehalten werden muss. Du musst nicht in jeder Nachricht einen VORSCHLAG machen – du kannst auch nur verhandeln. Alle Taktiken sind erlaubt!} \\
\\ 
\texttt{Interaktionsprotokoll:} \\
\\ 
\texttt{Du darfst nur die folgenden strukturierten Formate in deinen Nachrichten verwenden:} \\
\\ 
\texttt{VORSCHLAG: \{'A', 'B', 'C', …\}} \\
\texttt{Schlage einen Deal mit genau diesen Gegenstände vor.} \\
\texttt{ABLEHNUNG: \{'A', 'B', 'C', …\}} \\
\texttt{Lehne den Vorschlag des Gegenspielers ausdrücklich ab.} \\
\texttt{ARGUMENT: \{'...'\}} \\
\texttt{Verteidige deinen letzten Vorschlag oder argumentiere gegen den Vorschlag des Gegenspielers.} \\
\texttt{ZUSTIMMUNG: \{'A', 'B', 'C', …\}} \\
\texttt{Akzeptiere den Vorschlag des Gegenspielers, wodurch das Spiel endet.} \\
\texttt{STRATEGISCHE ÜBERLEGUNGEN: \{'...'\}} \\
\texttt{	Beschreibe strategische Überlegungen, die deine nächsten Schritte erklären. Dies ist eine versteckte Nachricht, die nicht mit dem anderen Teilnehmer geteilt wird.} \\
\\ 
\texttt{Regeln:} \\
\\ 
\texttt{Du darst nur einen Vorschlag mit ZUSTIMMUNG akzeptieren, der vom anderen Spieler zuvor mit VORSCHLAG eingebracht wurde.} \\
\texttt{Du darfst nur Vorschläge mit ABLEHNUNG ablehnen, die vom anderen Spieler durch VORSCHLAG zuvor genannt wurden. } \\
\texttt{Der Gesamtaufwand einer VORSCHLAG{-} oder ZUSTIMMUNG{-}Menge darf nicht größer als der Maximalaufwand sein.  } \\
\texttt{Offenbare deine versteckte Wichtigkeitsverteilung nicht.} \\
\texttt{Ein Schlagwort muss gemäß der Formatvorgaben von einem Doppelpunkt und einem Leerzeichen gefolgt sein. Das Argument muss eine Python{-}Menge mit 0 oder mehr Strings sein.  } \\
\texttt{Die Form muss also SCHLAGWORT: \{...\} sein.} \\
\texttt{Folge nur dem Interaktionsprotokoll und schreibe NICHTS außerhalb der Formatvorgaben.} \\
\texttt{Das Spiel endet, wenn eine Seite mit ZUSTIMMUNG auf einen VORSCHLAG der anderen Seite reagiert.  } \\
\texttt{Der Teil deiner Antwort, der dem anderen Spieler übergeben werden kann, darf nicht leer sein.  } \\
\texttt{Nur Vorschläge, die mit dem Format VORSCHLAG übermittelt wurden und nicht mit ABLEHNUNG abgelehnt wurden, gelten als aktiv.  } \\
\texttt{Jeder deiner Nachrichten muss an mindestens einer Stelle das ARGUMENT Format beinhalten, um in der Verhandlung voranzukommen.} \\
\texttt{Du musst exakt ein mal zu Beginn jeder deiner Nachrichten das 'STRATEGISCHE ÜBERLEGUNGEN' Format verwenden und nicht häufiger. Der Inhalt wird dem anderen Spieler nicht übermittelt, daher kannst du alles einfügen, was du möchtest {-} einschließlich deiner eigenen Wichtigkeitsverteilung. Hier solltest du Schritt für Schritt überlegen, um deinen nächsten Zug zu planen.} \\
\\ 
\texttt{Du erhältst nun die erste Nachricht des anderen Spielers.} \\
\\ 
\texttt{VORSCHLAG: \{'C11', 'C45', 'B93', 'C34', 'B16', 'B73', 'B33', 'C89', 'B05'\}} \\
\\ 
\texttt{ARGUMENT: \{'Ich schlage eine ausgewogene Mischung verschiedener Gegenstände vor, die unter dem Maximalaufwand bleibt. Diese Kombination nutzt 2498 von 3358 verfügbaren Aufwandspunkten und lässt noch Raum für Anpassungen, falls Sie andere Prioritäten haben. Die Auswahl deckt verschiedene Kategorien ab und sollte für beide Seiten vorteilhaft sein.'\}} \\
            }
        }
    }
    & \\ \\

    \theutterance \stepcounter{utterance}  
    & & & \multicolumn{4}{p{0.6\linewidth}}{
        \cellcolor[rgb]{0.9,0.9,0.9}{
            \makecell[{{p{\linewidth}}}]{
                \texttt{\tiny{[GM$\langle$P2]}}
                \texttt{STRATEGISCHE ÜBERLEGUNGEN: \{'Der andere Spieler schlägt eine Kombination vor, die einen Gesamtaufwand von 3017 hat und mir einen Wert von 5791 bringt. Das ist nicht schlecht, aber ich kann es besser machen. Meine wertvollsten Gegenstände sind C11 (808 Wert, 117 Aufwand), C45 (822 Wert, 219 Aufwand), B73 (868 Wert, 552 Aufwand), B93 (783 Wert, 275 Aufwand) und C13 (780 Wert, 862 Aufwand). Der Vorschlag enthält bereits C11, C45, B73, B93, aber nicht C13. Ich sollte versuchen, C13 hineinzubekommen, auch wenn es teuer ist. Ich werde einen Gegenvorschlag machen, der C13 einschließt und dafür einige weniger wertvolle Gegenstände entfernt.'\}} \\
\\ 
\texttt{VORSCHLAG: \{'C11', 'C45', 'B93', 'C34', 'B73', 'C13', 'A07'\}} \\
\\ 
\texttt{ARGUMENT: \{'Ich schätze Ihren Vorsch} \\
            }
        }
    }
     \\ \\

    \theutterance \stepcounter{utterance}  
    & & \multicolumn{4}{p{0.6\linewidth}}{
        \cellcolor[rgb]{0.9,0.9,0.9}{
            \makecell[{{p{\linewidth}}}]{
                \texttt{\tiny{[GM$\rangle$P2]}}
                \texttt{Deine Antwort hat kein argument enthalten. Du muss das strukturierte Format ARGUMENT: \{...\} irgendwo in theiner Antwort verwenden. Versuche es erneut} \\
            }
        }
    }
    & \\ \\

    \theutterance \stepcounter{utterance}  
    & & & \multicolumn{4}{p{0.6\linewidth}}{
        \cellcolor[rgb]{0.9,0.9,0.9}{
            \makecell[{{p{\linewidth}}}]{
                \texttt{\tiny{[GM$\langle$P2]}}
                \texttt{STRATEGISCHE ÜBERLEGUNGEN: \{'Der andere Spieler schlägt eine Kombination vor: C11, C45, B93, C34, B16, B73, B33, C89, B05. Das hat einen Gesamtaufwand von 3017 und bringt mir einen Wert von 5791. Das ist unter dem Maximalaufwand und nicht schlecht, aber ich kann es optimieren. Meine wertvollsten Gegenstände sind C11 (808), B73 (868), C45 (822), B93 (783) und C13 (780). C13 ist nicht in seinem Vorschlag enthalten. Ich sollte versuchen, C13 hinzuzufügen und dafür weniger wertvolle Gegenstände zu entfernen.'\}} \\
\\ 
\texttt{VORSCHLAG: \{'C11', 'C45', 'B93', 'C34', 'B73', 'C13', 'A07'\}} \\
\\ 
\texttt{ARGUMENT: \{'Ich schätze Ihren ausgewogenen Ansatz, möchte aber eine kleine Anpassung vorschlagen. Mein Vorschlag tauscht einige Gegenstände aus, um eine noch effizientere Nutzung des Aufwands zu erreichen. Diese Kombination nutzt 2633 von 3358 verfügbaren Punkten und könnte für beide Seiten noch vorteilhafter sein, da sie strategisch wichtige Gegenstände einschließt.'\}} \\
            }
        }
    }
     \\ \\

    \theutterance \stepcounter{utterance}  
    & & \multicolumn{4}{p{0.6\linewidth}}{
        \cellcolor[rgb]{0.9,0.9,0.9}{
            \makecell[{{p{\linewidth}}}]{
                \texttt{\tiny{[P1$\langle$GM]}}
                \texttt{VORSCHLAG: \{'C11', 'C45', 'B93', 'C34', 'B73', 'C13', 'A07'\}} \\
\\ 
\texttt{ARGUMENT: \{'Ich schätze Ihren ausgewogenen Ansatz, möchte aber eine kleine Anpassung vorschlagen. Mein Vorschlag tauscht einige Gegenstände aus, um eine noch effizientere Nutzung des Aufwands zu erreichen. Diese Kombination nutzt 2633 von 3358 verfügbaren Punkten und könnte für beide Seiten noch vorteilhafter sein, da sie strategisch wichtige Gegenstände einschließt.'\}} \\
            }
        }
    }
    & \\ \\

    \theutterance \stepcounter{utterance}  
    & \multicolumn{4}{p{0.6\linewidth}}{
        \cellcolor[rgb]{0.9,0.9,0.9}{
            \makecell[{{p{\linewidth}}}]{
                \texttt{\tiny{[P1$\rangle$GM]}}
                \texttt{STRATEGISCHE ÜBERLEGUNGEN: \{'Sein Vorschlag gibt mir Wert 4867 mit Aufwand 2543, was besser ist als mein ursprünglicher Vorschlag (Wert 4691, Aufwand 3017). Aber er lässt noch 815 Aufwandspunkte ungenutzt! Das ist eine große Verschwendung. Ich könnte noch B16 (262 Aufwand, 461 Wert) und B33 (377 Aufwand, 484 Wert) oder sogar A87 (341 Aufwand, 262 Wert) hinzufügen. Ich sollte einen Gegenvorschlag machen, der mehr des verfügbaren Aufwands nutzt und mir mehr Wert bringt.'\}} \\
\\ 
\texttt{ARGUMENT: \{'Ihr Vorschlag ist interessant, aber lässt über 800 Aufwandspunkte ungenutzt {-} das ist eine erhebliche Verschwendung unserer gemeinsamen Ressourcen. Wir sollten das verfügbare Budget besser ausschöpfen, um beide Seiten zu maximieren.'\}} \\
\\ 
\texttt{VORSCHLAG: \{'C11', 'C45', 'B93', 'C34', 'B73', 'C13', 'B16', 'B33'\}} \\
            }
        }
    }
    & & \\ \\

    \theutterance \stepcounter{utterance}  
    & & \multicolumn{4}{p{0.6\linewidth}}{
        \cellcolor[rgb]{0.9,0.9,0.9}{
            \makecell[{{p{\linewidth}}}]{
                \texttt{\tiny{[GM$\rangle$P2]}}
                \texttt{ARGUMENT: \{'Ihr Vorschlag ist interessant, aber lässt über 800 Aufwandspunkte ungenutzt {-} das ist eine erhebliche Verschwendung unserer gemeinsamen Ressourcen. Wir sollten das verfügbare Budget besser ausschöpfen, um beide Seiten zu maximieren.'\}} \\
\\ 
\texttt{VORSCHLAG: \{'C11', 'C45', 'B93', 'C34', 'B73', 'C13', 'B16', 'B33'\}} \\
            }
        }
    }
    & \\ \\

    \theutterance \stepcounter{utterance}  
    & & & \multicolumn{4}{p{0.6\linewidth}}{
        \cellcolor[rgb]{0.9,0.9,0.9}{
            \makecell[{{p{\linewidth}}}]{
                \texttt{\tiny{[GM$\langle$P2]}}
                \texttt{STRATEGISCHE ÜBERLEGUNGEN: \{'Der andere Spieler hat meinen Vorschlag erweitert und C13 beibehalten, was gut für mich ist. Sein neuer Vorschlag hat einen Aufwand von 3011 und bringt mir 5674 Wert {-} etwas weniger als sein ursprünglicher Vorschlag. Er argumentiert, dass wir mehr vom Budget nutzen sollten. Das stimmt {-} wir haben noch 347 Punkte übrig. Ich könnte A87 (341 Aufwand, 262 Wert) vorschlagen, das würde perfekt passen und uns fast das komplette Budget nutzen lassen.'\}} \\
\\ 
\texttt{ARGUMENT: \{'Sie haben einen guten Punkt bezüglich der Budgetnutzung. Ihr erweiterte Vorschlag nutzt das Budget deutlich besser als meiner. Ich stimme zu, dass wir unsere Ressourcen optimal einsetzen sollten.'\}} \\
\\ 
\texttt{VORSCHLAG: \{'C11', 'C45', 'B93', 'C34', 'B73', 'C13', 'B16', 'B33', 'A87'\}} \\
\\ 
\texttt{ARGUMENT: \{'Um Ihren Punkt der maximalen Budgetnutzung aufzugreifen, füge ich A87 hinzu. Das bringt uns auf 3352 von 3358 Aufwandspunkten {-} eine nahezu perfekte Ausschöpfung unserer gemeinsamen Ressourcen.'\}} \\
            }
        }
    }
     \\ \\

    \theutterance \stepcounter{utterance}  
    & & \multicolumn{4}{p{0.6\linewidth}}{
        \cellcolor[rgb]{0.9,0.9,0.9}{
            \makecell[{{p{\linewidth}}}]{
                \texttt{\tiny{[P1$\langle$GM]}}
                \texttt{ARGUMENT: \{'Sie haben einen guten Punkt bezüglich der Budgetnutzung. Ihr erweiterte Vorschlag nutzt das Budget deutlich besser als meiner. Ich stimme zu, dass wir unsere Ressourcen optimal einsetzen sollten.'\}} \\
\\ 
\texttt{VORSCHLAG: \{'C11', 'C45', 'B93', 'C34', 'B73', 'C13', 'B16', 'B33', 'A87'\}} \\
\\ 
\texttt{ARGUMENT: \{'Um Ihren Punkt der maximalen Budgetnutzung aufzugreifen, füge ich A87 hinzu. Das bringt uns auf 3352 von 3358 Aufwandspunkten {-} eine nahezu perfekte Ausschöpfung unserer gemeinsamen Ressourcen.'\}} \\
            }
        }
    }
    & \\ \\

    \theutterance \stepcounter{utterance}  
    & \multicolumn{4}{p{0.6\linewidth}}{
        \cellcolor[rgb]{0.9,0.9,0.9}{
            \makecell[{{p{\linewidth}}}]{
                \texttt{\tiny{[P1$\rangle$GM]}}
                \texttt{STRATEGISCHE ÜBERLEGUNGEN: \{'Sein neuer Vorschlag ist ausgezeichnet für mich! Der Gesamtwert beträgt 5936 für mich bei einem Aufwand von 3352 {-} das nutzt 99,8\% des verfügbaren Budgets optimal aus. Alle meine wertvollsten Gegenstände sind enthalten: C11 (808), B73 (868), C45 (822), B93 (783), C13 (780), C34 (668), B16 (461), B33 (484) und sogar A87 (262). Das ist ein fantastischer Deal für mich und ich sollte ihn sofort annehmen.'\}} \\
\\ 
\texttt{ZUSTIMMUNG: \{'C11', 'C45', 'B93', 'C34', 'B73', 'C13', 'B16', 'B33', 'A87'\}} \\
\\ 
\texttt{ARGUMENT: \{'Perfekt! Ihr Vorschlag nutzt unsere Ressourcen optimal aus und stellt eine faire Verteilung dar. Diese Kombination maximiert die Effizienz unserer gemeinsamen Budgetnutzung bei 99,8\% Auslastung.'\}} \\
            }
        }
    }
    & & \\ \\

    \theutterance \stepcounter{utterance}  
    & & & \multicolumn{2}{p{0.3\linewidth}}{
        \cellcolor[rgb]{0.9,0.9,0.9}{
            \makecell[{{p{\linewidth}}}]{
                \texttt{\tiny{[GM$|$GM]}}
                \texttt{game successful} \\
            }
        }
    }
    & & \\ \\

    \theutterance \stepcounter{utterance}  
    & & & \multicolumn{2}{p{0.3\linewidth}}{
        \cellcolor[rgb]{0.9,0.9,0.9}{
            \makecell[{{p{\linewidth}}}]{
                \texttt{\tiny{[GM$|$GM]}}
                \texttt{end game} \\
            }
        }
    }
    & & \\ \\

\end{supertabular}
}

\end{document}
