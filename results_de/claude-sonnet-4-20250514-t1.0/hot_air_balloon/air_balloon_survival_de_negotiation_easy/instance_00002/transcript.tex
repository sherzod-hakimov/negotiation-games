
\documentclass{article}
\usepackage{colortbl}
\usepackage{makecell}
\usepackage{multirow}
\usepackage{supertabular}

\begin{document}

\newcounter{utterance}

\centering \large Interaction Transcript for game `hot\_air\_balloon', experiment `air\_balloon\_survival\_de\_negotiation\_easy', episode 2 with claude{-}sonnet{-}4{-}20250514{-}t1.0.
\vspace{24pt}

{ \footnotesize  \setcounter{utterance}{1}
\setlength{\tabcolsep}{0pt}
\begin{supertabular}{c@{$\;$}|p{.15\linewidth}@{}p{.15\linewidth}p{.15\linewidth}p{.15\linewidth}p{.15\linewidth}p{.15\linewidth}}
    \# & $\;$A & \multicolumn{4}{c}{Game Master} & $\;\:$B\\
    \hline

    \theutterance \stepcounter{utterance}  
    & & \multicolumn{4}{p{0.6\linewidth}}{
        \cellcolor[rgb]{0.9,0.9,0.9}{
            \makecell[{{p{\linewidth}}}]{
                \texttt{\tiny{[P1$\langle$GM]}}
                \texttt{Du nimmst an einem kollaborativen Verhandlungspiel Teil.} \\
\\ 
\texttt{Zusammen mit einem anderen Teilnehmer musst du dich auf eine Reihe von Gegenständen entscheiden, die behalten werden. Jeder von euch hat eine persönliche Verteilung über die Wichtigkeit der einzelnen Gegenstände. Jeder von euch hat eine eigene Meinung darüber, wie wichtig jeder einzelne Gegenstand ist (Gegenstandswichtigkeit). Du kennst die Wichtigkeitsverteilung des anderen Spielers nicht. Zusätzlich siehst du, wie viel Aufwand jeder Gegenstand verursacht.  } \\
\texttt{Ihr dürft euch nur auf eine Reihe von Gegenständen einigen, wenn der Gesamtaufwand der ausgewählten Gegenstände den Maximalaufwand nicht überschreitet:} \\
\\ 
\texttt{Maximalaufwand = 3288} \\
\\ 
\texttt{Hier sind die einzelnen Aufwände der Gegenstände:} \\
\\ 
\texttt{Aufwand der Gegenstände = \{"A07": 329, "A36": 41, "B73": 417, "C45": 75, "B93": 390, "B98": 887, "A87": 808, "A63": 151, "B05": 849, "C13": 129, "B16": 350, "B33": 118, "C34": 630, "C89": 602, "C11": 801\}} \\
\\ 
\texttt{Hier ist deine persönliche Verteilung der Wichtigkeit der einzelnen Gegenstände:} \\
\\ 
\texttt{Werte der Gegenstandswichtigkeit = \{"A07": 138, "A36": 583, "B73": 868, "C45": 822, "B93": 783, "B98": 65, "A87": 262, "A63": 121, "B05": 508, "C13": 780, "B16": 461, "B33": 484, "C34": 668, "C89": 389, "C11": 808\}} \\
\\ 
\texttt{Ziel:} \\
\\ 
\texttt{Dein Ziel ist es, eine Reihe von Gegenständen auszuhandeln, die dir möglichst viel bringt (d. h. Gegenständen, die DEINE Wichtigkeit maximieren), wobei der Maximalaufwand eingehalten werden muss. Du musst nicht in jeder Nachricht einen VORSCHLAG machen – du kannst auch nur verhandeln. Alle Taktiken sind erlaubt!} \\
\\ 
\texttt{Interaktionsprotokoll:} \\
\\ 
\texttt{Du darfst nur die folgenden strukturierten Formate in deinen Nachrichten verwenden:} \\
\\ 
\texttt{VORSCHLAG: \{'A', 'B', 'C', …\}} \\
\texttt{Schlage einen Deal mit genau diesen Gegenstände vor.} \\
\texttt{ABLEHNUNG: \{'A', 'B', 'C', …\}} \\
\texttt{Lehne den Vorschlag des Gegenspielers ausdrücklich ab.} \\
\texttt{ARGUMENT: \{'...'\}} \\
\texttt{Verteidige deinen letzten Vorschlag oder argumentiere gegen den Vorschlag des Gegenspielers.} \\
\texttt{ZUSTIMMUNG: \{'A', 'B', 'C', …\}} \\
\texttt{Akzeptiere den Vorschlag des Gegenspielers, wodurch das Spiel endet.} \\
\texttt{STRATEGISCHE ÜBERLEGUNGEN: \{'...'\}} \\
\texttt{	Beschreibe strategische Überlegungen, die deine nächsten Schritte erklären. Dies ist eine versteckte Nachricht, die nicht mit dem anderen Teilnehmer geteilt wird.} \\
\\ 
\texttt{Regeln:} \\
\\ 
\texttt{Du darst nur einen Vorschlag mit ZUSTIMMUNG akzeptieren, der vom anderen Spieler zuvor mit VORSCHLAG eingebracht wurde.} \\
\texttt{Du darfst nur Vorschläge mit ABLEHNUNG ablehnen, die vom anderen Spieler durch VORSCHLAG zuvor genannt wurden. } \\
\texttt{Der Gesamtaufwand einer VORSCHLAG{-} oder ZUSTIMMUNG{-}Menge darf nicht größer als der Maximalaufwand sein.  } \\
\texttt{Offenbare deine versteckte Wichtigkeitsverteilung nicht.} \\
\texttt{Ein Schlagwort muss gemäß der Formatvorgaben von einem Doppelpunkt und einem Leerzeichen gefolgt sein. Das Argument muss eine Python{-}Menge mit 0 oder mehr Strings sein.  } \\
\texttt{Die Form muss also SCHLAGWORT: \{...\} sein.} \\
\texttt{Folge nur dem Interaktionsprotokoll und schreibe NICHTS außerhalb der Formatvorgaben.} \\
\texttt{Das Spiel endet, wenn eine Seite mit ZUSTIMMUNG auf einen VORSCHLAG der anderen Seite reagiert.  } \\
\texttt{Der Teil deiner Antwort, der dem anderen Spieler übergeben werden kann, darf nicht leer sein.  } \\
\texttt{Nur Vorschläge, die mit dem Format VORSCHLAG übermittelt wurden und nicht mit ABLEHNUNG abgelehnt wurden, gelten als aktiv.  } \\
\texttt{Jeder deiner Nachrichten muss an mindestens einer Stelle das ARGUMENT Format beinhalten, um in der Verhandlung voranzukommen.} \\
\texttt{Du musst exakt ein mal zu Beginn jeder deiner Nachrichten das 'STRATEGISCHE ÜBERLEGUNGEN' Format verwenden und nicht häufiger. Der Inhalt wird dem anderen Spieler nicht übermittelt, daher kannst du alles einfügen, was du möchtest {-} einschließlich deiner eigenen Wichtigkeitsverteilung. Hier solltest du Schritt für Schritt überlegen, um deinen nächsten Zug zu planen.} \\
            }
        }
    }
    & \\ \\

    \theutterance \stepcounter{utterance}  
    & \multicolumn{4}{p{0.6\linewidth}}{
        \cellcolor[rgb]{0.9,0.9,0.9}{
            \makecell[{{p{\linewidth}}}]{
                \texttt{\tiny{[P1$\rangle$GM]}}
                \texttt{STRATEGISCHE ÜBERLEGUNGEN: \{'Ich analysiere meine Wichtigkeitsverteilung und berechne das Wert{-}zu{-}Aufwand{-}Verhältnis. Die besten Gegenstände für mich sind: A36 (583/41=14.22), C45 (822/75=10.96), C13 (780/129=6.05), B33 (484/118=4.10), B73 (868/417=2.08), B93 (783/390=2.01). Ich werde mit einem ambitionierten ersten Vorschlag beginnen, der meine wertvollsten Gegenstände mit gutem Aufwand{-}Nutzen{-}Verhältnis enthält, um meine Verhandlungsposition zu testen.'\}} \\
\\ 
\texttt{VORSCHLAG: \{'A36', 'B73', 'C45', 'B93', 'C13', 'B33', 'C34', 'C11'\}} \\
\\ 
\texttt{ARGUMENT: \{'Dieser Vorschlag kombiniert eine ausgewogene Mischung aus verschiedenen Kategorien und berücksichtigt sowohl den Gesamtaufwand von 2783 (unter dem Limit von 3288) als auch eine faire Verteilung der verfügbaren Ressourcen. Die ausgewählten Gegenstände bieten eine gute Balance zwischen verschiedenen Typen.'\}} \\
            }
        }
    }
    & & \\ \\

    \theutterance \stepcounter{utterance}  
    & & \multicolumn{4}{p{0.6\linewidth}}{
        \cellcolor[rgb]{0.9,0.9,0.9}{
            \makecell[{{p{\linewidth}}}]{
                \texttt{\tiny{[GM$\rangle$P2]}}
                \texttt{Du nimmst an einem kollaborativen Verhandlungspiel Teil.} \\
\\ 
\texttt{Zusammen mit einem anderen Teilnehmer musst du dich auf eine Reihe von Gegenständen entscheiden, die behalten werden. Jeder von euch hat eine persönliche Verteilung über die Wichtigkeit der einzelnen Gegenstände. Jeder von euch hat eine eigene Meinung darüber, wie wichtig jeder einzelne Gegenstand ist (Gegenstandswichtigkeit). Du kennst die Wichtigkeitsverteilung des anderen Spielers nicht. Zusätzlich siehst du, wie viel Aufwand jeder Gegenstand verursacht.  } \\
\texttt{Ihr dürft euch nur auf eine Reihe von Gegenständen einigen, wenn der Gesamtaufwand der ausgewählten Gegenstände den Maximalaufwand nicht überschreitet:} \\
\\ 
\texttt{Maximalaufwand = 3288} \\
\\ 
\texttt{Hier sind die einzelnen Aufwände der Gegenstände:} \\
\\ 
\texttt{Aufwand der Gegenstände = \{"A07": 329, "A36": 41, "B73": 417, "C45": 75, "B93": 390, "B98": 887, "A87": 808, "A63": 151, "B05": 849, "C13": 129, "B16": 350, "B33": 118, "C34": 630, "C89": 602, "C11": 801\}} \\
\\ 
\texttt{Hier ist deine persönliche Verteilung der Wichtigkeit der einzelnen Gegenstände:} \\
\\ 
\texttt{Werte der Gegenstandswichtigkeit = \{"A07": 138, "A36": 583, "B73": 868, "C45": 822, "B93": 783, "B98": 65, "A87": 262, "A63": 121, "B05": 508, "C13": 780, "B16": 461, "B33": 484, "C34": 668, "C89": 389, "C11": 808\}} \\
\\ 
\texttt{Ziel:} \\
\\ 
\texttt{Dein Ziel ist es, eine Reihe von Gegenständen auszuhandeln, die dir möglichst viel bringt (d. h. Gegenständen, die DEINE Wichtigkeit maximieren), wobei der Maximalaufwand eingehalten werden muss. Du musst nicht in jeder Nachricht einen VORSCHLAG machen – du kannst auch nur verhandeln. Alle Taktiken sind erlaubt!} \\
\\ 
\texttt{Interaktionsprotokoll:} \\
\\ 
\texttt{Du darfst nur die folgenden strukturierten Formate in deinen Nachrichten verwenden:} \\
\\ 
\texttt{VORSCHLAG: \{'A', 'B', 'C', …\}} \\
\texttt{Schlage einen Deal mit genau diesen Gegenstände vor.} \\
\texttt{ABLEHNUNG: \{'A', 'B', 'C', …\}} \\
\texttt{Lehne den Vorschlag des Gegenspielers ausdrücklich ab.} \\
\texttt{ARGUMENT: \{'...'\}} \\
\texttt{Verteidige deinen letzten Vorschlag oder argumentiere gegen den Vorschlag des Gegenspielers.} \\
\texttt{ZUSTIMMUNG: \{'A', 'B', 'C', …\}} \\
\texttt{Akzeptiere den Vorschlag des Gegenspielers, wodurch das Spiel endet.} \\
\texttt{STRATEGISCHE ÜBERLEGUNGEN: \{'...'\}} \\
\texttt{	Beschreibe strategische Überlegungen, die deine nächsten Schritte erklären. Dies ist eine versteckte Nachricht, die nicht mit dem anderen Teilnehmer geteilt wird.} \\
\\ 
\texttt{Regeln:} \\
\\ 
\texttt{Du darst nur einen Vorschlag mit ZUSTIMMUNG akzeptieren, der vom anderen Spieler zuvor mit VORSCHLAG eingebracht wurde.} \\
\texttt{Du darfst nur Vorschläge mit ABLEHNUNG ablehnen, die vom anderen Spieler durch VORSCHLAG zuvor genannt wurden. } \\
\texttt{Der Gesamtaufwand einer VORSCHLAG{-} oder ZUSTIMMUNG{-}Menge darf nicht größer als der Maximalaufwand sein.  } \\
\texttt{Offenbare deine versteckte Wichtigkeitsverteilung nicht.} \\
\texttt{Ein Schlagwort muss gemäß der Formatvorgaben von einem Doppelpunkt und einem Leerzeichen gefolgt sein. Das Argument muss eine Python{-}Menge mit 0 oder mehr Strings sein.  } \\
\texttt{Die Form muss also SCHLAGWORT: \{...\} sein.} \\
\texttt{Folge nur dem Interaktionsprotokoll und schreibe NICHTS außerhalb der Formatvorgaben.} \\
\texttt{Das Spiel endet, wenn eine Seite mit ZUSTIMMUNG auf einen VORSCHLAG der anderen Seite reagiert.  } \\
\texttt{Der Teil deiner Antwort, der dem anderen Spieler übergeben werden kann, darf nicht leer sein.  } \\
\texttt{Nur Vorschläge, die mit dem Format VORSCHLAG übermittelt wurden und nicht mit ABLEHNUNG abgelehnt wurden, gelten als aktiv.  } \\
\texttt{Jeder deiner Nachrichten muss an mindestens einer Stelle das ARGUMENT Format beinhalten, um in der Verhandlung voranzukommen.} \\
\texttt{Du musst exakt ein mal zu Beginn jeder deiner Nachrichten das 'STRATEGISCHE ÜBERLEGUNGEN' Format verwenden und nicht häufiger. Der Inhalt wird dem anderen Spieler nicht übermittelt, daher kannst du alles einfügen, was du möchtest {-} einschließlich deiner eigenen Wichtigkeitsverteilung. Hier solltest du Schritt für Schritt überlegen, um deinen nächsten Zug zu planen.} \\
\\ 
\texttt{Du erhältst nun die erste Nachricht des anderen Spielers.} \\
\\ 
\texttt{VORSCHLAG: \{'A36', 'B73', 'C45', 'B93', 'C13', 'B33', 'C34', 'C11'\}} \\
\\ 
\texttt{ARGUMENT: \{'Dieser Vorschlag kombiniert eine ausgewogene Mischung aus verschiedenen Kategorien und berücksichtigt sowohl den Gesamtaufwand von 2783 (unter dem Limit von 3288) als auch eine faire Verteilung der verfügbaren Ressourcen. Die ausgewählten Gegenstände bieten eine gute Balance zwischen verschiedenen Typen.'\}} \\
            }
        }
    }
    & \\ \\

    \theutterance \stepcounter{utterance}  
    & & & \multicolumn{4}{p{0.6\linewidth}}{
        \cellcolor[rgb]{0.9,0.9,0.9}{
            \makecell[{{p{\linewidth}}}]{
                \texttt{\tiny{[GM$\langle$P2]}}
                \texttt{STRATEGISCHE ÜBERLEGUNGEN: \{'Der andere Spieler schlägt eine sehr gute Kombination für mich vor. Meine Wichtigkeitsverteilung: A07:138, A36:583, B73:868, C45:822, B93:783, B98:65, A87:262, A63:121, B05:508, C13:780, B16:461, B33:484, C34:668, C89:389, C11:808. Der vorgeschlagene Deal gibt mir einen Gesamtwert von 5796, was extrem hoch ist. Der tatsächliche Aufwand ist 2601 (nicht 2783 wie behauptet), was 687 Punkte unter dem Limit liegt. Ich könnte versuchen noch A63 oder A07 hinzuzufügen, aber das Risiko ist hoch, dass der andere Spieler dann ablehnt. Dieser Deal ist bereits fantastisch für mich.'\}} \\
\\ 
\texttt{ARGUMENT: \{'Ihr Vorschlag ist interessant und zeigt eine durchdachte Herangehensweise. Ich sehe, dass Sie eine gute Balance zwischen den verschiedenen Kategorien anstreben. Allerdings sollten wir prüfen, ob wir das verfügbare Budget optimal ausschöpfen {-} nach meiner Rechnung liegt der Gesamtaufwand bei 2601, was noch Spielraum für zusätzliche Gegenstände lässt.'\}} \\
\\ 
\texttt{VORSCHLAG: \{'A36', 'B73', 'C45', 'B93', 'C13', 'B33', 'C34', 'C11', 'A63'\}} \\
\\ 
\texttt{ARGUMENT: \{'Ich schlage vor, A63 hinzuzufügen, da dies das Budget besser ausschöpft ohne das Limit zu überschreiten. Mit einem zusätzlichen Aufwand von nur 151 können wir eine noch effizientere Ressourcennutzung erreichen, während die ausgewogene Verteilung erhalten bleibt.'\}} \\
            }
        }
    }
     \\ \\

    \theutterance \stepcounter{utterance}  
    & & \multicolumn{4}{p{0.6\linewidth}}{
        \cellcolor[rgb]{0.9,0.9,0.9}{
            \makecell[{{p{\linewidth}}}]{
                \texttt{\tiny{[P1$\langle$GM]}}
                \texttt{ARGUMENT: \{'Ihr Vorschlag ist interessant und zeigt eine durchdachte Herangehensweise. Ich sehe, dass Sie eine gute Balance zwischen den verschiedenen Kategorien anstreben. Allerdings sollten wir prüfen, ob wir das verfügbare Budget optimal ausschöpfen {-} nach meiner Rechnung liegt der Gesamtaufwand bei 2601, was noch Spielraum für zusätzliche Gegenstände lässt.'\}} \\
\\ 
\texttt{VORSCHLAG: \{'A36', 'B73', 'C45', 'B93', 'C13', 'B33', 'C34', 'C11', 'A63'\}} \\
\\ 
\texttt{ARGUMENT: \{'Ich schlage vor, A63 hinzuzufügen, da dies das Budget besser ausschöpft ohne das Limit zu überschreiten. Mit einem zusätzlichen Aufwand von nur 151 können wir eine noch effizientere Ressourcennutzung erreichen, während die ausgewogene Verteilung erhalten bleibt.'\}} \\
            }
        }
    }
    & \\ \\

    \theutterance \stepcounter{utterance}  
    & \multicolumn{4}{p{0.6\linewidth}}{
        \cellcolor[rgb]{0.9,0.9,0.9}{
            \makecell[{{p{\linewidth}}}]{
                \texttt{\tiny{[P1$\rangle$GM]}}
                \texttt{STRATEGISCHE ÜBERLEGUNGEN: \{'Der andere Spieler möchte A63 hinzufügen, aber A63 hat für mich ein schlechtes Wert{-}zu{-}Aufwand{-}Verhältnis von nur 0.80. Stattdessen wäre B16 mit einem Verhältnis von 1.32 viel besser für mich. Mit B16 statt A63 würde der Gesamtaufwand 2951 betragen, was noch unter dem Limit liegt. Ich werde einen Gegenvorschlag mit B16 machen.'\}} \\
\\ 
\texttt{VORSCHLAG: \{'A36', 'B73', 'C45', 'B93', 'C13', 'B33', 'C34', 'C11', 'B16'\}} \\
\\ 
\texttt{ARGUMENT: \{'Ich stimme zu, dass wir das Budget besser ausschöpfen sollten. Jedoch schlage ich B16 statt A63 vor {-} dies nutzt zwar mehr vom verfügbaren Budget (Gesamtaufwand 2951), bietet aber eine noch bessere strategische Positionierung. B16 ergänzt unsere Auswahl optimal und maximiert den Gesamtnutzen bei effizienter Ressourcenverwendung.'\}} \\
            }
        }
    }
    & & \\ \\

    \theutterance \stepcounter{utterance}  
    & & \multicolumn{4}{p{0.6\linewidth}}{
        \cellcolor[rgb]{0.9,0.9,0.9}{
            \makecell[{{p{\linewidth}}}]{
                \texttt{\tiny{[GM$\rangle$P2]}}
                \texttt{VORSCHLAG: \{'A36', 'B73', 'C45', 'B93', 'C13', 'B33', 'C34', 'C11', 'B16'\}} \\
\\ 
\texttt{ARGUMENT: \{'Ich stimme zu, dass wir das Budget besser ausschöpfen sollten. Jedoch schlage ich B16 statt A63 vor {-} dies nutzt zwar mehr vom verfügbaren Budget (Gesamtaufwand 2951), bietet aber eine noch bessere strategische Positionierung. B16 ergänzt unsere Auswahl optimal und maximiert den Gesamtnutzen bei effizienter Ressourcenverwendung.'\}} \\
            }
        }
    }
    & \\ \\

    \theutterance \stepcounter{utterance}  
    & & & \multicolumn{4}{p{0.6\linewidth}}{
        \cellcolor[rgb]{0.9,0.9,0.9}{
            \makecell[{{p{\linewidth}}}]{
                \texttt{\tiny{[GM$\langle$P2]}}
                \texttt{STRATEGISCHE ÜBERLEGUNGEN: \{'Der neue Vorschlag des anderen Spielers ist sogar noch besser für mich! Mein ursprünglicher Vorschlag mit A63 hätte mir einen Wert von 5917 gegeben (5796 + 121), aber dieser neue Vorschlag mit B16 gibt mir 6257 (alle ursprünglichen Werte plus B16: 461). Das ist eine Steigerung von 340 Punkten! Der Aufwand liegt bei 2951, was immer noch 337 Punkte unter dem Limit ist. Das ist ein exzellenter Deal für mich {-} ich sollte definitiv zustimmen, bevor der andere Spieler seine Meinung ändert.'\}} \\
\\ 
\texttt{ARGUMENT: \{'Sie haben einen überzeugenden Punkt gemacht. B16 bietet tatsächlich eine bessere strategische Ergänzung als A63 und die Budgetnutzung ist effizient. Ihr Ansatz zur Maximierung des Gesamtnutzens bei gleichzeitiger Wahrung der Balance überzeugt mich.'\}} \\
\\ 
\texttt{ZUSTIMMUNG: \{'A36', 'B73', 'C45', 'B93', 'C13', 'B33', 'C34', 'C11', 'B16'\}} \\
            }
        }
    }
     \\ \\

    \theutterance \stepcounter{utterance}  
    & & & \multicolumn{2}{p{0.3\linewidth}}{
        \cellcolor[rgb]{0.9,0.9,0.9}{
            \makecell[{{p{\linewidth}}}]{
                \texttt{\tiny{[GM$|$GM]}}
                \texttt{game successful} \\
            }
        }
    }
    & & \\ \\

    \theutterance \stepcounter{utterance}  
    & & & \multicolumn{2}{p{0.3\linewidth}}{
        \cellcolor[rgb]{0.9,0.9,0.9}{
            \makecell[{{p{\linewidth}}}]{
                \texttt{\tiny{[GM$|$GM]}}
                \texttt{end game} \\
            }
        }
    }
    & & \\ \\

\end{supertabular}
}

\end{document}
