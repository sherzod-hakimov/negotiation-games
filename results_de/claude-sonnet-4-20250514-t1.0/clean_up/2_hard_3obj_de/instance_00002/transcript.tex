
\documentclass{article}
\usepackage{colortbl}
\usepackage{makecell}
\usepackage{multirow}
\usepackage{supertabular}

\begin{document}

\newcounter{utterance}

\twocolumn

{ \footnotesize  \setcounter{utterance}{1}
\setlength{\tabcolsep}{0pt}
\begin{supertabular}{c@{$\;$}|p{.15\linewidth}@{}p{.15\linewidth}p{.15\linewidth}p{.15\linewidth}p{.15\linewidth}p{.15\linewidth}}

    \# & $\;$A & \multicolumn{4}{c}{Game Master} & $\;\:$B\\
    \hline 

    \theutterance \stepcounter{utterance}  

    & & \multicolumn{4}{p{0.6\linewidth}}{\cellcolor[rgb]{0.9,0.9,0.9}{%
	\makecell[{{p{\linewidth}}}]{% 
	  \tt {\tiny [A$\langle$GM]}  
	 Ich bin der Spielleiter, und du spielst ein kooperatives Spiel mit dem folgenden Raster als Spielbrett:\\ \tt \\ \tt ```\\ \tt  1234567\\ \tt ╔═╤═╤═╤═╗\\ \tt ║◌│◌│◌│◌║ 1\\ \tt ╟─┼─┤◌│◌║ 2\\ \tt ║◌│◌│◌│◌║ 3\\ \tt ║L│◌├─┴─╢ 4\\ \tt ╟─┴─┤◌◌◌║ 5\\ \tt ║◌◌C└─┬─╢ 6\\ \tt ║◌◌◌◌P│◌║ 7\\ \tt ╚═════╧═╝\\ \tt \\ \tt ```\\ \tt \\ \tt * Am oberen Rand sind die x-Koordinaten angegeben, die nach rechts hin zunehmen, und am rechten Rand die y-Koordinaten, die nach unten zunehmen.\\ \tt * Die folgenden Objekte sind zufällig auf deinem Raster platziert: 'C', 'L', 'P'.\\ \tt \\ \tt \\ \tt Dein Mitspieler sieht eine Variation des Spielbretts, auf dem dieselben Objekte an anderen zufälligen Positionen platziert sind. Du kannst das Spielbrett des anderen Spielers nicht sehen, und er kann deins nicht sehen.\\ \tt \\ \tt \\ \tt **Ziel des Spiels:**\\ \tt \\ \tt Beide Spieler müssen die Objekte auf ihrem jeweiligen Spielbrett so bewegen, dass identische Objekte am Ende an denselben Koordinaten liegen. Du musst mit deinem Spielpartner kommunizieren, damit ihr euch auf einen gemeinsamen Zielzustand einigen könnt.\\ \tt \\ \tt \\ \tt **Spielregeln:**\\ \tt \\ \tt * In jedem Spielzug kannst du genau eine der folgenden zwei Aktionen ausführen:\\ \tt 1. `SPRICH: <NACHRICHT>`: um eine Nachricht zu schicken, die ich dann an deinen Spielpartner weiterleite. Als Nachricht gilt alles bis zum nächsten Zeilenumbruch.\\ \tt 2. `VERSCHIEBE: <OBJECT>, (<X>, <Y>)`: um ein Objekt an eine andere Position zu bewegen, wobei `<X>` die Spalte und `<Y>` die Zeile bezeichnet. Ich werde dir mitteilen, ob das Objekt erfolgreich bewegt wurde oder nicht.\\ \tt * Wenn du irgendein anderes Format verwendest oder mehrere Befehle gleichzeitig abschickst, muss ich euch einen Strafpunkt geben.\\ \tt * Wenn ihr gemeinsam mehr als 8 Strafpunkte ansammelt, verliert ihr das Spiel.\\ \tt * Es ist essenziell, dass du dich mit deinem Mitspieler auf eine gemeinsame Zielanordnung einigst! Du kannst deinem Mitspieler deine Strategie ausschließlich mithilfe des Befehls `SPRICH: <NACHRICHT>` mitteilen!\\ \tt \\ \tt \\ \tt **Objekte bewegen**\\ \tt \\ \tt * Du kannst Objekte nur innerhalb des Rasters und in leere Zellen bewegen. Leere Zellen erkennst am Zeichen '◌'.\\ \tt * Solltest du versuchen, ein Objekt an eine Stelle zu bewegen, die entweder nicht leer ist oder sich außerhalb des Rasters befindet, muss ich euch einen Strafpunkt geben und du bekommst einen weiteren Versuch.\\ \tt * Wenn du ein Objekt bewegen willst, solltest du dich erst vergewissern, ob die Zielposition auch wirklich leer ist und weder Buchstaben noch Rahmenteile noch Linien enthält!\\ \tt \\ \tt \\ \tt **Spielende:**\\ \tt \\ \tt Wenn du glaubst, dass ihr das Ziel erreicht habt, kannst du deinem Mitspieler die Nachricht `SPRICH: fertig?` schicken. Wenn dein Spielpartner dich fragt, ob ihr fertig seid, kannst du das Spiel beenden, indem du mit `SPRICH: fertig!` antwortest.\\ \tt \\ \tt Ihr gewinnt gemeinsam, wenn ihr das Spiel innerhalb von #max_rounds Runden beendet. Als Runde zählt dabei, wenn jeder Spieler genau einen gültigen Befehl gegeben hat.\\ \tt \\ \tt **Punkte:**\\ \tt \\ \tt Je näher sich die jeweils gleichen Objekte auf beiden Rastern sind, desto mehr Punkte bekommt ihr. Für Strafpunkte gibt es Punktabzug. Könnt ihr den Rekord brechen?\\ \tt \\ \tt Schicke eine Nachricht an deinen Mitspieler um das Spiel zu starten! 
	  } 
	   } 
	   } 
	 & \\ 
 

    \theutterance \stepcounter{utterance}  

    & \multicolumn{4}{p{0.6\linewidth}}{\cellcolor[rgb]{0.8,1,0.9}{%
	\makecell[{{p{\linewidth}}}]{% 
	  \tt {\tiny [A$\rangle$GM]}  
	 SPRICH: Hallo! Lass uns das Spiel starten. Meine aktuelle Objektverteilung ist: L bei (1,4), C bei (3,6), P bei (5,7). Wo stehen deine Objekte? Dann können wir uns auf gemeinsame Zielpositionen einigen. 
	  } 
	   } 
	   } 
	 & & \\ 
 

    \theutterance \stepcounter{utterance}  

    & & & \multicolumn{2}{p{0.3\linewidth}}{\cellcolor[rgb]{0.95,0.95,0.95}{%
	\makecell[{{p{\linewidth}}}]{% 
	  \tt {\tiny [GM$|$GM]}  
	 SPRICH: Hallo! Lass uns das Spiel starten. Meine aktuelle Objektverteilung ist: L bei (1,4), C bei (3,6), P bei (5,7). Wo stehen deine Objekte? Dann können wir uns auf gemeinsame Zielpositionen einigen. 
	  } 
	   } 
	   } 
	 & & \\ 
 

    \theutterance \stepcounter{utterance}  

    & & \multicolumn{4}{p{0.6\linewidth}}{\cellcolor[rgb]{0.9,0.9,0.9}{%
	\makecell[{{p{\linewidth}}}]{% 
	  \tt {\tiny [GM$\rangle$B]}  
	 Ich bin der Spielleiter, und du spielst ein kooperatives Spiel mit dem folgenden Raster als Spielbrett:\\ \tt \\ \tt ```\\ \tt  1234567\\ \tt ╔═╤═╤═╤═╗\\ \tt ║L│◌│◌│◌║ 1\\ \tt ╟─┼─┤◌│◌║ 2\\ \tt ║◌│◌│◌│◌║ 3\\ \tt ║◌│◌├─┴─╢ 4\\ \tt ╟─┴─┤C◌◌║ 5\\ \tt ║◌◌◌└─┬─╢ 6\\ \tt ║◌P◌◌◌│◌║ 7\\ \tt ╚═════╧═╝\\ \tt \\ \tt ```\\ \tt \\ \tt * Am oberen Rand sind die x-Koordinaten angegeben, die nach rechts hin zunehmen, und am rechten Rand die y-Koordinaten, die nach unten zunehmen.\\ \tt * Die folgenden Objekte sind zufällig auf deinem Raster platziert: 'C', 'L', 'P'.\\ \tt \\ \tt \\ \tt Dein Mitspieler sieht eine Variation des Spielbretts, auf dem dieselben Objekte an anderen zufälligen Positionen platziert sind. Du kannst das Spielbrett des anderen Spielers nicht sehen, und er kann deins nicht sehen.\\ \tt \\ \tt \\ \tt **Ziel des Spiels:**\\ \tt \\ \tt Beide Spieler müssen die Objekte auf ihrem jeweiligen Spielbrett so bewegen, dass identische Objekte am Ende an denselben Koordinaten liegen. Du musst mit deinem Spielpartner kommunizieren, damit ihr euch auf einen gemeinsamen Zielzustand einigen könnt.\\ \tt \\ \tt \\ \tt **Spielregeln:**\\ \tt \\ \tt * In jedem Spielzug kannst du genau eine der folgenden zwei Aktionen ausführen:\\ \tt 1. `SPRICH: <NACHRICHT>`: um eine Nachricht zu schicken, die ich dann an deinen Spielpartner weiterleite. Als Nachricht gilt alles bis zum nächsten Zeilenumbruch.\\ \tt 2. `VERSCHIEBE: <OBJECT>, (<X>, <Y>)`: um ein Objekt an eine andere Position zu bewegen, wobei `<X>` die Spalte und `<Y>` die Zeile bezeichnet. Ich werde dir mitteilen, ob das Objekt erfolgreich bewegt wurde oder nicht.\\ \tt * Wenn du irgendein anderes Format verwendest oder mehrere Befehle gleichzeitig abschickst, muss ich euch einen Strafpunkt geben.\\ \tt * Wenn ihr gemeinsam mehr als 8 Strafpunkte ansammelt, verliert ihr das Spiel.\\ \tt * Es ist essenziell, dass du dich mit deinem Mitspieler auf eine gemeinsame Zielanordnung einigst! Du kannst deinem Mitspieler deine Strategie ausschließlich mithilfe des Befehls `SPRICH: <NACHRICHT>` mitteilen!\\ \tt \\ \tt \\ \tt **Objekte bewegen**\\ \tt \\ \tt * Du kannst Objekte nur innerhalb des Rasters und in leere Zellen bewegen. Leere Zellen erkennst am Zeichen '◌'.\\ \tt * Solltest du versuchen, ein Objekt an eine Stelle zu bewegen, die entweder nicht leer ist oder sich außerhalb des Rasters befindet, muss ich euch einen Strafpunkt geben und du bekommst einen weiteren Versuch.\\ \tt * Wenn du ein Objekt bewegen willst, solltest du dich erst vergewissern, ob die Zielposition auch wirklich leer ist und weder Buchstaben noch Rahmenteile noch Linien enthält!\\ \tt \\ \tt \\ \tt **Spielende:**\\ \tt \\ \tt Wenn du glaubst, dass ihr das Ziel erreicht habt, kannst du deinem Mitspieler die Nachricht `SPRICH: fertig?` schicken. Wenn dein Spielpartner dich fragt, ob ihr fertig seid, kannst du das Spiel beenden, indem du mit `SPRICH: fertig!` antwortest.\\ \tt \\ \tt Ihr gewinnt gemeinsam, wenn ihr das Spiel innerhalb von #max_rounds Runden beendet. Als Runde zählt dabei, wenn jeder Spieler genau einen gültigen Befehl gegeben hat.\\ \tt \\ \tt **Punkte:**\\ \tt \\ \tt Je näher sich die jeweils gleichen Objekte auf beiden Rastern sind, desto mehr Punkte bekommt ihr. Für Strafpunkte gibt es Punktabzug. Könnt ihr den Rekord brechen?\\ \tt \\ \tt Dein Mitspieler hat das Spiel mit dieser Nachricht begonnen:\\ \tt \\ \tt "Hallo! Lass uns das Spiel starten. Meine aktuelle Objektverteilung ist: L bei (1,4), C bei (3,6), P bei (5,7). Wo stehen deine Objekte? Dann können wir uns auf gemeinsame Zielpositionen einigen."\\ \tt \\ \tt Wie lautet dein erster Befehl? 
	  } 
	   } 
	   } 
	 & \\ 
 

    \theutterance \stepcounter{utterance}  

    & & & \multicolumn{4}{p{0.6\linewidth}}{\cellcolor[rgb]{1,0.85,0.72}{%
	\makecell[{{p{\linewidth}}}]{% 
	  \tt {\tiny [GM$\langle$B]}  
	 SPRICH: Hallo! Meine aktuelle Objektverteilung ist: L bei (1,1), C bei (4,5), P bei (2,7). Lass uns gemeinsame Zielpositionen wählen. Ich schlage vor: L nach (2,2), C nach (3,3), P nach (4,4) - das wären zentrale Positionen, die für uns beide gut erreichbar sein sollten. Was denkst du? 
	  } 
	   } 
	   } 
	  \\ 
 

    \theutterance \stepcounter{utterance}  

    & & & \multicolumn{2}{p{0.3\linewidth}}{\cellcolor[rgb]{0.95,0.95,0.95}{%
	\makecell[{{p{\linewidth}}}]{% 
	  \tt {\tiny [GM$|$GM]}  
	 SPRICH: Hallo! Meine aktuelle Objektverteilung ist: L bei (1,1), C bei (4,5), P bei (2,7). Lass uns gemeinsame Zielpositionen wählen. Ich schlage vor: L nach (2,2), C nach (3,3), P nach (4,4) - das wären zentrale Positionen, die für uns beide gut erreichbar sein sollten. Was denkst du? 
	  } 
	   } 
	   } 
	 & & \\ 
 

    \theutterance \stepcounter{utterance}  

    & & \multicolumn{4}{p{0.6\linewidth}}{\cellcolor[rgb]{0.9,0.9,0.9}{%
	\makecell[{{p{\linewidth}}}]{% 
	  \tt {\tiny [A$\langle$GM]}  
	 Deine Nachricht wurde an deinen Mitspieler übermittelt.\\ \tt \\ \tt \\ \tt Ihr seid gerade bei Runde 1 von maximal 12.\\ \tt Ihr habt gemeinsam 0 von maximal 8 Strafpunkten angesammelt.\\ \tt Dein Mitspieler hat dir diese Nachricht geschickt:\\ \tt \\ \tt \\ \tt "Hallo! Meine aktuelle Objektverteilung ist: L bei (1,1), C bei (4,5), P bei (2,7). Lass uns gemeinsame Zielpositionen wählen. Ich schlage vor: L nach (2,2), C nach (3,3), P nach (4,4) - das wären zentrale Positionen, die für uns beide gut erreichbar sein sollten. Was denkst du?"\\ \tt \\ \tt \\ \tt Wie lautet dein nächster Befehl? 
	  } 
	   } 
	   } 
	 & \\ 
 

    \theutterance \stepcounter{utterance}  

    & \multicolumn{4}{p{0.6\linewidth}}{\cellcolor[rgb]{0.8,1,0.9}{%
	\makecell[{{p{\linewidth}}}]{% 
	  \tt {\tiny [A$\rangle$GM]}  
	 SPRICH: Gute Idee mit den zentralen Positionen! Aber ich sehe ein Problem: Position (2,2) und (4,4) sind auf meinem Brett nicht frei - dort sind Linien. Lass uns stattdessen diese Alternativen nehmen: L nach (2,3), C nach (3,3), P nach (4,3) - alle in Zeile 3, das sollte auf beiden Brettern machbar sein. Was sagst du? 
	  } 
	   } 
	   } 
	 & & \\ 
 

    \theutterance \stepcounter{utterance}  

    & & & \multicolumn{2}{p{0.3\linewidth}}{\cellcolor[rgb]{0.95,0.95,0.95}{%
	\makecell[{{p{\linewidth}}}]{% 
	  \tt {\tiny [GM$|$GM]}  
	 SPRICH: Gute Idee mit den zentralen Positionen! Aber ich sehe ein Problem: Position (2,2) und (4,4) sind auf meinem Brett nicht frei - dort sind Linien. Lass uns stattdessen diese Alternativen nehmen: L nach (2,3), C nach (3,3), P nach (4,3) - alle in Zeile 3, das sollte auf beiden Brettern machbar sein. Was sagst du? 
	  } 
	   } 
	   } 
	 & & \\ 
 

    \theutterance \stepcounter{utterance}  

    & & \multicolumn{4}{p{0.6\linewidth}}{\cellcolor[rgb]{0.9,0.9,0.9}{%
	\makecell[{{p{\linewidth}}}]{% 
	  \tt {\tiny [GM$\rangle$B]}  
	 Deine Nachricht wurde an deinen Mitspieler übermittelt.\\ \tt \\ \tt \\ \tt Ihr seid gerade bei Runde 2 von maximal 12.\\ \tt Ihr habt gemeinsam 0 von maximal 8 Strafpunkten angesammelt.\\ \tt Dein Mitspieler hat dir diese Nachricht geschickt:\\ \tt \\ \tt \\ \tt "Gute Idee mit den zentralen Positionen! Aber ich sehe ein Problem: Position (2,2) und (4,4) sind auf meinem Brett nicht frei - dort sind Linien. Lass uns stattdessen diese Alternativen nehmen: L nach (2,3), C nach (3,3), P nach (4,3) - alle in Zeile 3, das sollte auf beiden Brettern machbar sein. Was sagst du?"\\ \tt \\ \tt \\ \tt Wie lautet dein nächster Befehl? 
	  } 
	   } 
	   } 
	 & \\ 
 

    \theutterance \stepcounter{utterance}  

    & & & \multicolumn{4}{p{0.6\linewidth}}{\cellcolor[rgb]{1,0.85,0.72}{%
	\makecell[{{p{\linewidth}}}]{% 
	  \tt {\tiny [GM$\langle$B]}  
	 SPRICH: Perfekt! Die Positionen L nach (2,3), C nach (3,3), P nach (4,3) sind auf meinem Brett alle frei. Das ist ein guter Plan. Ich fange an, meine Objekte dorthin zu bewegen. Sollen wir beide gleichzeitig anfangen? 
	  } 
	   } 
	   } 
	  \\ 
 

    \theutterance \stepcounter{utterance}  

    & & & \multicolumn{2}{p{0.3\linewidth}}{\cellcolor[rgb]{0.95,0.95,0.95}{%
	\makecell[{{p{\linewidth}}}]{% 
	  \tt {\tiny [GM$|$GM]}  
	 SPRICH: Perfekt! Die Positionen L nach (2,3), C nach (3,3), P nach (4,3) sind auf meinem Brett alle frei. Das ist ein guter Plan. Ich fange an, meine Objekte dorthin zu bewegen. Sollen wir beide gleichzeitig anfangen? 
	  } 
	   } 
	   } 
	 & & \\ 
 

    \theutterance \stepcounter{utterance}  

    & & \multicolumn{4}{p{0.6\linewidth}}{\cellcolor[rgb]{0.9,0.9,0.9}{%
	\makecell[{{p{\linewidth}}}]{% 
	  \tt {\tiny [A$\langle$GM]}  
	 Deine Nachricht wurde an deinen Mitspieler übermittelt.\\ \tt \\ \tt \\ \tt Ihr seid gerade bei Runde 2 von maximal 12.\\ \tt Ihr habt gemeinsam 0 von maximal 8 Strafpunkten angesammelt.\\ \tt Dein Mitspieler hat dir diese Nachricht geschickt:\\ \tt \\ \tt \\ \tt "Perfekt! Die Positionen L nach (2,3), C nach (3,3), P nach (4,3) sind auf meinem Brett alle frei. Das ist ein guter Plan. Ich fange an, meine Objekte dorthin zu bewegen. Sollen wir beide gleichzeitig anfangen?"\\ \tt \\ \tt \\ \tt Wie lautet dein nächster Befehl? 
	  } 
	   } 
	   } 
	 & \\ 
 

    \theutterance \stepcounter{utterance}  

    & \multicolumn{4}{p{0.6\linewidth}}{\cellcolor[rgb]{0.8,1,0.9}{%
	\makecell[{{p{\linewidth}}}]{% 
	  \tt {\tiny [A$\rangle$GM]}  
	 VERSCHIEBE: L, (2, 3) 
	  } 
	   } 
	   } 
	 & & \\ 
 

    \theutterance \stepcounter{utterance}  

    & & & \multicolumn{2}{p{0.3\linewidth}}{\cellcolor[rgb]{0.95,0.95,0.95}{%
	\makecell[{{p{\linewidth}}}]{% 
	  \tt {\tiny [GM$|$GM]}  
	 VERSCHIEBE: L, (2, 3) 
	  } 
	   } 
	   } 
	 & & \\ 
 

    \theutterance \stepcounter{utterance}  

    & & & \multicolumn{2}{p{0.3\linewidth}}{\cellcolor[rgb]{0.95,0.95,0.95}{%
	\makecell[{{p{\linewidth}}}]{% 
	  \tt {\tiny [GM$|$GM]}  
	 Strafpunkt: (2,3) ist nicht leer, sondern enthält '│'.\\ \tt \\ \tt Ihr habt gemeinsam 1 von maximal 8 Strafpunkten angesammelt.\\ \tt Bitte versuche es noch mal! 
	  } 
	   } 
	   } 
	 & & \\ 
 

    \theutterance \stepcounter{utterance}  

    & & \multicolumn{4}{p{0.6\linewidth}}{\cellcolor[rgb]{0.9,0.9,0.9}{%
	\makecell[{{p{\linewidth}}}]{% 
	  \tt {\tiny [A$\langle$GM]}  
	 Strafpunkt: (2,3) ist nicht leer, sondern enthält '│'.\\ \tt \\ \tt Ihr habt gemeinsam 1 von maximal 8 Strafpunkten angesammelt.\\ \tt Bitte versuche es noch mal! 
	  } 
	   } 
	   } 
	 & \\ 
 

    \theutterance \stepcounter{utterance}  

    & \multicolumn{4}{p{0.6\linewidth}}{\cellcolor[rgb]{0.8,1,0.9}{%
	\makecell[{{p{\linewidth}}}]{% 
	  \tt {\tiny [A$\rangle$GM]}  
	 SPRICH: Fehler! Position (2,3) ist auf meinem Brett nicht frei - dort ist eine vertikale Linie. Lass uns anpassen: L nach (1,3), C nach (3,3), P nach (4,3) - sind diese drei Positionen auf deinem Brett frei? 
	  } 
	   } 
	   } 
	 & & \\ 
 

    \theutterance \stepcounter{utterance}  

    & & & \multicolumn{2}{p{0.3\linewidth}}{\cellcolor[rgb]{0.95,0.95,0.95}{%
	\makecell[{{p{\linewidth}}}]{% 
	  \tt {\tiny [GM$|$GM]}  
	 SPRICH: Fehler! Position (2,3) ist auf meinem Brett nicht frei - dort ist eine vertikale Linie. Lass uns anpassen: L nach (1,3), C nach (3,3), P nach (4,3) - sind diese drei Positionen auf deinem Brett frei? 
	  } 
	   } 
	   } 
	 & & \\ 
 

    \theutterance \stepcounter{utterance}  

    & & \multicolumn{4}{p{0.6\linewidth}}{\cellcolor[rgb]{0.9,0.9,0.9}{%
	\makecell[{{p{\linewidth}}}]{% 
	  \tt {\tiny [GM$\rangle$B]}  
	 Deine Nachricht wurde an deinen Mitspieler übermittelt.\\ \tt \\ \tt \\ \tt Ihr seid gerade bei Runde 3 von maximal 12.\\ \tt Ihr habt gemeinsam 1 von maximal 8 Strafpunkten angesammelt.\\ \tt Dein Mitspieler hat dir diese Nachricht geschickt:\\ \tt \\ \tt \\ \tt "Fehler! Position (2,3) ist auf meinem Brett nicht frei - dort ist eine vertikale Linie. Lass uns anpassen: L nach (1,3), C nach (3,3), P nach (4,3) - sind diese drei Positionen auf deinem Brett frei?"\\ \tt \\ \tt \\ \tt Wie lautet dein nächster Befehl? 
	  } 
	   } 
	   } 
	 & \\ 
 

    \theutterance \stepcounter{utterance}  

    & & & \multicolumn{4}{p{0.6\linewidth}}{\cellcolor[rgb]{1,0.85,0.72}{%
	\makecell[{{p{\linewidth}}}]{% 
	  \tt {\tiny [GM$\langle$B]}  
	 SPRICH: Ja, alle drei Positionen sind auf meinem Brett frei: (1,3), (3,3) und (4,3) sind alle leer. Das funktioniert perfekt! Sollen wir anfangen, unsere Objekte dorthin zu bewegen? 
	  } 
	   } 
	   } 
	  \\ 
 

    \theutterance \stepcounter{utterance}  

    & & & \multicolumn{2}{p{0.3\linewidth}}{\cellcolor[rgb]{0.95,0.95,0.95}{%
	\makecell[{{p{\linewidth}}}]{% 
	  \tt {\tiny [GM$|$GM]}  
	 SPRICH: Ja, alle drei Positionen sind auf meinem Brett frei: (1,3), (3,3) und (4,3) sind alle leer. Das funktioniert perfekt! Sollen wir anfangen, unsere Objekte dorthin zu bewegen? 
	  } 
	   } 
	   } 
	 & & \\ 
 

    \theutterance \stepcounter{utterance}  

    & & \multicolumn{4}{p{0.6\linewidth}}{\cellcolor[rgb]{0.9,0.9,0.9}{%
	\makecell[{{p{\linewidth}}}]{% 
	  \tt {\tiny [A$\langle$GM]}  
	 Deine Nachricht wurde an deinen Mitspieler übermittelt.\\ \tt \\ \tt \\ \tt Ihr seid gerade bei Runde 3 von maximal 12.\\ \tt Ihr habt gemeinsam 1 von maximal 8 Strafpunkten angesammelt.\\ \tt Dein Mitspieler hat dir diese Nachricht geschickt:\\ \tt \\ \tt \\ \tt "Ja, alle drei Positionen sind auf meinem Brett frei: (1,3), (3,3) und (4,3) sind alle leer. Das funktioniert perfekt! Sollen wir anfangen, unsere Objekte dorthin zu bewegen?"\\ \tt \\ \tt \\ \tt Wie lautet dein nächster Befehl? 
	  } 
	   } 
	   } 
	 & \\ 
 

    \theutterance \stepcounter{utterance}  

    & \multicolumn{4}{p{0.6\linewidth}}{\cellcolor[rgb]{0.8,1,0.9}{%
	\makecell[{{p{\linewidth}}}]{% 
	  \tt {\tiny [A$\rangle$GM]}  
	 VERSCHIEBE: L, (1, 3) 
	  } 
	   } 
	   } 
	 & & \\ 
 

    \theutterance \stepcounter{utterance}  

    & & & \multicolumn{2}{p{0.3\linewidth}}{\cellcolor[rgb]{0.95,0.95,0.95}{%
	\makecell[{{p{\linewidth}}}]{% 
	  \tt {\tiny [GM$|$GM]}  
	 VERSCHIEBE: L, (1, 3) 
	  } 
	   } 
	   } 
	 & & \\ 
 

    \theutterance \stepcounter{utterance}  

    & & & \multicolumn{2}{p{0.3\linewidth}}{\cellcolor[rgb]{0.95,0.95,0.95}{%
	\makecell[{{p{\linewidth}}}]{% 
	  \tt {\tiny [GM$|$GM]}  
	 'L' erfolgreich nach (1,3) verschoben. Dein Raster sieht jetzt so aus:\\ \tt ```\\ \tt  1234567\\ \tt ╔═╤═╤═╤═╗\\ \tt ║◌│◌│◌│◌║ 1\\ \tt ╟─┼─┤◌│◌║ 2\\ \tt ║L│◌│◌│◌║ 3\\ \tt ║◌│◌├─┴─╢ 4\\ \tt ╟─┴─┤◌◌◌║ 5\\ \tt ║◌◌C└─┬─╢ 6\\ \tt ║◌◌◌◌P│◌║ 7\\ \tt ╚═════╧═╝\\ \tt ```\\ \tt  
	  } 
	   } 
	   } 
	 & & \\ 
 

    \theutterance \stepcounter{utterance}  

    & & \multicolumn{4}{p{0.6\linewidth}}{\cellcolor[rgb]{0.9,0.9,0.9}{%
	\makecell[{{p{\linewidth}}}]{% 
	  \tt {\tiny [GM$\rangle$B]}  
	 Deine Nachricht wurde an deinen Mitspieler übermittelt.\\ \tt \\ \tt \\ \tt Ihr seid gerade bei Runde 4 von maximal 12.\\ \tt Ihr habt gemeinsam 1 von maximal 8 Strafpunkten angesammelt.\\ \tt Dein Mitspieler hat ein Objekt auf seinem Raster bewegt.\\ \tt \\ \tt \\ \tt Wie lautet dein nächster Befehl? 
	  } 
	   } 
	   } 
	 & \\ 
 

    \theutterance \stepcounter{utterance}  

    & & & \multicolumn{4}{p{0.6\linewidth}}{\cellcolor[rgb]{1,0.85,0.72}{%
	\makecell[{{p{\linewidth}}}]{% 
	  \tt {\tiny [GM$\langle$B]}  
	 VERSCHIEBE: L, (1, 3) 
	  } 
	   } 
	   } 
	  \\ 
 

    \theutterance \stepcounter{utterance}  

    & & & \multicolumn{2}{p{0.3\linewidth}}{\cellcolor[rgb]{0.95,0.95,0.95}{%
	\makecell[{{p{\linewidth}}}]{% 
	  \tt {\tiny [GM$|$GM]}  
	 VERSCHIEBE: L, (1, 3) 
	  } 
	   } 
	   } 
	 & & \\ 
 

    \theutterance \stepcounter{utterance}  

    & & & \multicolumn{2}{p{0.3\linewidth}}{\cellcolor[rgb]{0.95,0.95,0.95}{%
	\makecell[{{p{\linewidth}}}]{% 
	  \tt {\tiny [GM$|$GM]}  
	 'L' erfolgreich nach (1,3) verschoben. Dein Raster sieht jetzt so aus:\\ \tt ```\\ \tt  1234567\\ \tt ╔═╤═╤═╤═╗\\ \tt ║◌│◌│◌│◌║ 1\\ \tt ╟─┼─┤◌│◌║ 2\\ \tt ║L│◌│◌│◌║ 3\\ \tt ║◌│◌├─┴─╢ 4\\ \tt ╟─┴─┤C◌◌║ 5\\ \tt ║◌◌◌└─┬─╢ 6\\ \tt ║◌P◌◌◌│◌║ 7\\ \tt ╚═════╧═╝\\ \tt ```\\ \tt  
	  } 
	   } 
	   } 
	 & & \\ 
 

    \theutterance \stepcounter{utterance}  

    & & \multicolumn{4}{p{0.6\linewidth}}{\cellcolor[rgb]{0.9,0.9,0.9}{%
	\makecell[{{p{\linewidth}}}]{% 
	  \tt {\tiny [A$\langle$GM]}  
	 'L' erfolgreich nach (1,3) verschoben. Dein Raster sieht jetzt so aus:\\ \tt ```\\ \tt  1234567\\ \tt ╔═╤═╤═╤═╗\\ \tt ║◌│◌│◌│◌║ 1\\ \tt ╟─┼─┤◌│◌║ 2\\ \tt ║L│◌│◌│◌║ 3\\ \tt ║◌│◌├─┴─╢ 4\\ \tt ╟─┴─┤◌◌◌║ 5\\ \tt ║◌◌C└─┬─╢ 6\\ \tt ║◌◌◌◌P│◌║ 7\\ \tt ╚═════╧═╝\\ \tt ```\\ \tt Ihr seid gerade bei Runde 4 von maximal 12.\\ \tt Ihr habt gemeinsam 1 von maximal 8 Strafpunkten angesammelt.\\ \tt Dein Mitspieler hat ein Objekt auf seinem Raster bewegt.\\ \tt \\ \tt \\ \tt Wie lautet dein nächster Befehl? 
	  } 
	   } 
	   } 
	 & \\ 
 

    \theutterance \stepcounter{utterance}  

    & \multicolumn{4}{p{0.6\linewidth}}{\cellcolor[rgb]{0.8,1,0.9}{%
	\makecell[{{p{\linewidth}}}]{% 
	  \tt {\tiny [A$\rangle$GM]}  
	 VERSCHIEBE: C, (3, 3) 
	  } 
	   } 
	   } 
	 & & \\ 
 

    \theutterance \stepcounter{utterance}  

    & & & \multicolumn{2}{p{0.3\linewidth}}{\cellcolor[rgb]{0.95,0.95,0.95}{%
	\makecell[{{p{\linewidth}}}]{% 
	  \tt {\tiny [GM$|$GM]}  
	 VERSCHIEBE: C, (3, 3) 
	  } 
	   } 
	   } 
	 & & \\ 
 

    \theutterance \stepcounter{utterance}  

    & & & \multicolumn{2}{p{0.3\linewidth}}{\cellcolor[rgb]{0.95,0.95,0.95}{%
	\makecell[{{p{\linewidth}}}]{% 
	  \tt {\tiny [GM$|$GM]}  
	 'C' erfolgreich nach (3,3) verschoben. Dein Raster sieht jetzt so aus:\\ \tt ```\\ \tt  1234567\\ \tt ╔═╤═╤═╤═╗\\ \tt ║◌│◌│◌│◌║ 1\\ \tt ╟─┼─┤◌│◌║ 2\\ \tt ║L│C│◌│◌║ 3\\ \tt ║◌│◌├─┴─╢ 4\\ \tt ╟─┴─┤◌◌◌║ 5\\ \tt ║◌◌◌└─┬─╢ 6\\ \tt ║◌◌◌◌P│◌║ 7\\ \tt ╚═════╧═╝\\ \tt ```\\ \tt  
	  } 
	   } 
	   } 
	 & & \\ 
 

    \theutterance \stepcounter{utterance}  

    & & \multicolumn{4}{p{0.6\linewidth}}{\cellcolor[rgb]{0.9,0.9,0.9}{%
	\makecell[{{p{\linewidth}}}]{% 
	  \tt {\tiny [GM$\rangle$B]}  
	 'L' erfolgreich nach (1,3) verschoben. Dein Raster sieht jetzt so aus:\\ \tt ```\\ \tt  1234567\\ \tt ╔═╤═╤═╤═╗\\ \tt ║◌│◌│◌│◌║ 1\\ \tt ╟─┼─┤◌│◌║ 2\\ \tt ║L│◌│◌│◌║ 3\\ \tt ║◌│◌├─┴─╢ 4\\ \tt ╟─┴─┤C◌◌║ 5\\ \tt ║◌◌◌└─┬─╢ 6\\ \tt ║◌P◌◌◌│◌║ 7\\ \tt ╚═════╧═╝\\ \tt ```\\ \tt Ihr seid gerade bei Runde 5 von maximal 12.\\ \tt Ihr habt gemeinsam 1 von maximal 8 Strafpunkten angesammelt.\\ \tt Dein Mitspieler hat ein Objekt auf seinem Raster bewegt.\\ \tt \\ \tt \\ \tt Wie lautet dein nächster Befehl? 
	  } 
	   } 
	   } 
	 & \\ 
 

    \theutterance \stepcounter{utterance}  

    & & & \multicolumn{4}{p{0.6\linewidth}}{\cellcolor[rgb]{1,0.85,0.72}{%
	\makecell[{{p{\linewidth}}}]{% 
	  \tt {\tiny [GM$\langle$B]}  
	 VERSCHIEBE: C, (3, 3) 
	  } 
	   } 
	   } 
	  \\ 
 

    \theutterance \stepcounter{utterance}  

    & & & \multicolumn{2}{p{0.3\linewidth}}{\cellcolor[rgb]{0.95,0.95,0.95}{%
	\makecell[{{p{\linewidth}}}]{% 
	  \tt {\tiny [GM$|$GM]}  
	 VERSCHIEBE: C, (3, 3) 
	  } 
	   } 
	   } 
	 & & \\ 
 

    \theutterance \stepcounter{utterance}  

    & & & \multicolumn{2}{p{0.3\linewidth}}{\cellcolor[rgb]{0.95,0.95,0.95}{%
	\makecell[{{p{\linewidth}}}]{% 
	  \tt {\tiny [GM$|$GM]}  
	 'C' erfolgreich nach (3,3) verschoben. Dein Raster sieht jetzt so aus:\\ \tt ```\\ \tt  1234567\\ \tt ╔═╤═╤═╤═╗\\ \tt ║◌│◌│◌│◌║ 1\\ \tt ╟─┼─┤◌│◌║ 2\\ \tt ║L│C│◌│◌║ 3\\ \tt ║◌│◌├─┴─╢ 4\\ \tt ╟─┴─┤◌◌◌║ 5\\ \tt ║◌◌◌└─┬─╢ 6\\ \tt ║◌P◌◌◌│◌║ 7\\ \tt ╚═════╧═╝\\ \tt ```\\ \tt  
	  } 
	   } 
	   } 
	 & & \\ 
 

    \theutterance \stepcounter{utterance}  

    & & \multicolumn{4}{p{0.6\linewidth}}{\cellcolor[rgb]{0.9,0.9,0.9}{%
	\makecell[{{p{\linewidth}}}]{% 
	  \tt {\tiny [A$\langle$GM]}  
	 'C' erfolgreich nach (3,3) verschoben. Dein Raster sieht jetzt so aus:\\ \tt ```\\ \tt  1234567\\ \tt ╔═╤═╤═╤═╗\\ \tt ║◌│◌│◌│◌║ 1\\ \tt ╟─┼─┤◌│◌║ 2\\ \tt ║L│C│◌│◌║ 3\\ \tt ║◌│◌├─┴─╢ 4\\ \tt ╟─┴─┤◌◌◌║ 5\\ \tt ║◌◌◌└─┬─╢ 6\\ \tt ║◌◌◌◌P│◌║ 7\\ \tt ╚═════╧═╝\\ \tt ```\\ \tt Ihr seid gerade bei Runde 5 von maximal 12.\\ \tt Ihr habt gemeinsam 1 von maximal 8 Strafpunkten angesammelt.\\ \tt Dein Mitspieler hat ein Objekt auf seinem Raster bewegt.\\ \tt \\ \tt \\ \tt Wie lautet dein nächster Befehl? 
	  } 
	   } 
	   } 
	 & \\ 
 

    \theutterance \stepcounter{utterance}  

    & \multicolumn{4}{p{0.6\linewidth}}{\cellcolor[rgb]{0.8,1,0.9}{%
	\makecell[{{p{\linewidth}}}]{% 
	  \tt {\tiny [A$\rangle$GM]}  
	 VERSCHIEBE: P, (4, 3) 
	  } 
	   } 
	   } 
	 & & \\ 
 

    \theutterance \stepcounter{utterance}  

    & & & \multicolumn{2}{p{0.3\linewidth}}{\cellcolor[rgb]{0.95,0.95,0.95}{%
	\makecell[{{p{\linewidth}}}]{% 
	  \tt {\tiny [GM$|$GM]}  
	 VERSCHIEBE: P, (4, 3) 
	  } 
	   } 
	   } 
	 & & \\ 
 

    \theutterance \stepcounter{utterance}  

    & & & \multicolumn{2}{p{0.3\linewidth}}{\cellcolor[rgb]{0.95,0.95,0.95}{%
	\makecell[{{p{\linewidth}}}]{% 
	  \tt {\tiny [GM$|$GM]}  
	 Strafpunkt: (4,3) ist nicht leer, sondern enthält '│'.\\ \tt \\ \tt Ihr habt gemeinsam 2 von maximal 8 Strafpunkten angesammelt.\\ \tt Bitte versuche es noch mal! 
	  } 
	   } 
	   } 
	 & & \\ 
 

    \theutterance \stepcounter{utterance}  

    & & \multicolumn{4}{p{0.6\linewidth}}{\cellcolor[rgb]{0.9,0.9,0.9}{%
	\makecell[{{p{\linewidth}}}]{% 
	  \tt {\tiny [A$\langle$GM]}  
	 Strafpunkt: (4,3) ist nicht leer, sondern enthält '│'.\\ \tt \\ \tt Ihr habt gemeinsam 2 von maximal 8 Strafpunkten angesammelt.\\ \tt Bitte versuche es noch mal! 
	  } 
	   } 
	   } 
	 & \\ 
 

    \theutterance \stepcounter{utterance}  

    & \multicolumn{4}{p{0.6\linewidth}}{\cellcolor[rgb]{0.8,1,0.9}{%
	\makecell[{{p{\linewidth}}}]{% 
	  \tt {\tiny [A$\rangle$GM]}  
	 SPRICH: Schon wieder ein Fehler! Position (4,3) ist auch nicht frei - da ist ebenfalls eine Linie. Wir müssen unseren Plan anpassen. Sind auf deinem Brett die Positionen (5,3), (6,3) oder (7,3) frei? Wir brauchen eine neue Zielposition für P. 
	  } 
	   } 
	   } 
	 & & \\ 
 

    \theutterance \stepcounter{utterance}  

    & & & \multicolumn{2}{p{0.3\linewidth}}{\cellcolor[rgb]{0.95,0.95,0.95}{%
	\makecell[{{p{\linewidth}}}]{% 
	  \tt {\tiny [GM$|$GM]}  
	 SPRICH: Schon wieder ein Fehler! Position (4,3) ist auch nicht frei - da ist ebenfalls eine Linie. Wir müssen unseren Plan anpassen. Sind auf deinem Brett die Positionen (5,3), (6,3) oder (7,3) frei? Wir brauchen eine neue Zielposition für P. 
	  } 
	   } 
	   } 
	 & & \\ 
 

    \theutterance \stepcounter{utterance}  

    & & \multicolumn{4}{p{0.6\linewidth}}{\cellcolor[rgb]{0.9,0.9,0.9}{%
	\makecell[{{p{\linewidth}}}]{% 
	  \tt {\tiny [GM$\rangle$B]}  
	 'C' erfolgreich nach (3,3) verschoben. Dein Raster sieht jetzt so aus:\\ \tt ```\\ \tt  1234567\\ \tt ╔═╤═╤═╤═╗\\ \tt ║◌│◌│◌│◌║ 1\\ \tt ╟─┼─┤◌│◌║ 2\\ \tt ║L│C│◌│◌║ 3\\ \tt ║◌│◌├─┴─╢ 4\\ \tt ╟─┴─┤◌◌◌║ 5\\ \tt ║◌◌◌└─┬─╢ 6\\ \tt ║◌P◌◌◌│◌║ 7\\ \tt ╚═════╧═╝\\ \tt ```\\ \tt Ihr seid gerade bei Runde 6 von maximal 12.\\ \tt Ihr habt gemeinsam 2 von maximal 8 Strafpunkten angesammelt.\\ \tt Dein Mitspieler hat dir diese Nachricht geschickt:\\ \tt \\ \tt \\ \tt "Schon wieder ein Fehler! Position (4,3) ist auch nicht frei - da ist ebenfalls eine Linie. Wir müssen unseren Plan anpassen. Sind auf deinem Brett die Positionen (5,3), (6,3) oder (7,3) frei? Wir brauchen eine neue Zielposition für P."\\ \tt \\ \tt \\ \tt Wie lautet dein nächster Befehl? 
	  } 
	   } 
	   } 
	 & \\ 
 

    \theutterance \stepcounter{utterance}  

    & & & \multicolumn{4}{p{0.6\linewidth}}{\cellcolor[rgb]{1,0.85,0.72}{%
	\makecell[{{p{\linewidth}}}]{% 
	  \tt {\tiny [GM$\langle$B]}  
	 SPRICH: Lass mich schauen... auf meinem Brett in Zeile 3 sehe ich: (3,3) und (4,3) sind noch frei. Was die Positionen (5,3), (6,3), (7,3) angeht - das ist schwer zu sehen in der Darstellung, aber ich glaube (4,3) sollte für P funktionieren, da es auf meinem Brett definitiv frei ist. Sollen wir es trotzdem mit P nach (4,3) versuchen? 
	  } 
	   } 
	   } 
	  \\ 
 

    \theutterance \stepcounter{utterance}  

    & & & \multicolumn{2}{p{0.3\linewidth}}{\cellcolor[rgb]{0.95,0.95,0.95}{%
	\makecell[{{p{\linewidth}}}]{% 
	  \tt {\tiny [GM$|$GM]}  
	 SPRICH: Lass mich schauen... auf meinem Brett in Zeile 3 sehe ich: (3,3) und (4,3) sind noch frei. Was die Positionen (5,3), (6,3), (7,3) angeht - das ist schwer zu sehen in der Darstellung, aber ich glaube (4,3) sollte für P funktionieren, da es auf meinem Brett definitiv frei ist. Sollen wir es trotzdem mit P nach (4,3) versuchen? 
	  } 
	   } 
	   } 
	 & & \\ 
 

    \theutterance \stepcounter{utterance}  

    & & \multicolumn{4}{p{0.6\linewidth}}{\cellcolor[rgb]{0.9,0.9,0.9}{%
	\makecell[{{p{\linewidth}}}]{% 
	  \tt {\tiny [A$\langle$GM]}  
	 Deine Nachricht wurde an deinen Mitspieler übermittelt.\\ \tt \\ \tt \\ \tt Ihr seid gerade bei Runde 6 von maximal 12.\\ \tt Ihr habt gemeinsam 2 von maximal 8 Strafpunkten angesammelt.\\ \tt Dein Mitspieler hat dir diese Nachricht geschickt:\\ \tt \\ \tt \\ \tt "Lass mich schauen... auf meinem Brett in Zeile 3 sehe ich: (3,3) und (4,3) sind noch frei. Was die Positionen (5,3), (6,3), (7,3) angeht - das ist schwer zu sehen in der Darstellung, aber ich glaube (4,3) sollte für P funktionieren, da es auf meinem Brett definitiv frei ist. Sollen wir es trotzdem mit P nach (4,3) versuchen?"\\ \tt \\ \tt \\ \tt Wie lautet dein nächster Befehl? 
	  } 
	   } 
	   } 
	 & \\ 
 

    \theutterance \stepcounter{utterance}  

    & \multicolumn{4}{p{0.6\linewidth}}{\cellcolor[rgb]{0.8,1,0.9}{%
	\makecell[{{p{\linewidth}}}]{% 
	  \tt {\tiny [A$\rangle$GM]}  
	 SPRICH: Das Problem ist, dass (4,3) auf meinem Brett eine Linie hat, auch wenn es auf deinem frei ist. Ich sehe aber, dass (5,3) und (6,3) auf meinem Brett frei sind (◌). Sind diese Positionen auch auf deinem Brett frei? Dann könnten wir P nach (5,3) bewegen. 
	  } 
	   } 
	   } 
	 & & \\ 
 

    \theutterance \stepcounter{utterance}  

    & & & \multicolumn{2}{p{0.3\linewidth}}{\cellcolor[rgb]{0.95,0.95,0.95}{%
	\makecell[{{p{\linewidth}}}]{% 
	  \tt {\tiny [GM$|$GM]}  
	 SPRICH: Das Problem ist, dass (4,3) auf meinem Brett eine Linie hat, auch wenn es auf deinem frei ist. Ich sehe aber, dass (5,3) und (6,3) auf meinem Brett frei sind (◌). Sind diese Positionen auch auf deinem Brett frei? Dann könnten wir P nach (5,3) bewegen. 
	  } 
	   } 
	   } 
	 & & \\ 
 

    \theutterance \stepcounter{utterance}  

    & & \multicolumn{4}{p{0.6\linewidth}}{\cellcolor[rgb]{0.9,0.9,0.9}{%
	\makecell[{{p{\linewidth}}}]{% 
	  \tt {\tiny [GM$\rangle$B]}  
	 Deine Nachricht wurde an deinen Mitspieler übermittelt.\\ \tt \\ \tt \\ \tt Ihr seid gerade bei Runde 7 von maximal 12.\\ \tt Ihr habt gemeinsam 2 von maximal 8 Strafpunkten angesammelt.\\ \tt Dein Mitspieler hat dir diese Nachricht geschickt:\\ \tt \\ \tt \\ \tt "Das Problem ist, dass (4,3) auf meinem Brett eine Linie hat, auch wenn es auf deinem frei ist. Ich sehe aber, dass (5,3) und (6,3) auf meinem Brett frei sind (◌). Sind diese Positionen auch auf deinem Brett frei? Dann könnten wir P nach (5,3) bewegen."\\ \tt \\ \tt \\ \tt Wie lautet dein nächster Befehl? 
	  } 
	   } 
	   } 
	 & \\ 
 

    \theutterance \stepcounter{utterance}  

    & & & \multicolumn{4}{p{0.6\linewidth}}{\cellcolor[rgb]{1,0.85,0.72}{%
	\makecell[{{p{\linewidth}}}]{% 
	  \tt {\tiny [GM$\langle$B]}  
	 SPRICH: Ja, sowohl (5,3) als auch (6,3) sind auf meinem Brett frei! Lass uns P nach (5,3) bewegen. Das sollte dann unsere finale Anordnung sein: L bei (1,3), C bei (3,3), P bei (5,3). 
	  } 
	   } 
	   } 
	  \\ 
 

    \theutterance \stepcounter{utterance}  

    & & & \multicolumn{2}{p{0.3\linewidth}}{\cellcolor[rgb]{0.95,0.95,0.95}{%
	\makecell[{{p{\linewidth}}}]{% 
	  \tt {\tiny [GM$|$GM]}  
	 SPRICH: Ja, sowohl (5,3) als auch (6,3) sind auf meinem Brett frei! Lass uns P nach (5,3) bewegen. Das sollte dann unsere finale Anordnung sein: L bei (1,3), C bei (3,3), P bei (5,3). 
	  } 
	   } 
	   } 
	 & & \\ 
 

    \theutterance \stepcounter{utterance}  

    & & \multicolumn{4}{p{0.6\linewidth}}{\cellcolor[rgb]{0.9,0.9,0.9}{%
	\makecell[{{p{\linewidth}}}]{% 
	  \tt {\tiny [A$\langle$GM]}  
	 Deine Nachricht wurde an deinen Mitspieler übermittelt.\\ \tt \\ \tt \\ \tt Ihr seid gerade bei Runde 7 von maximal 12.\\ \tt Ihr habt gemeinsam 2 von maximal 8 Strafpunkten angesammelt.\\ \tt Dein Mitspieler hat dir diese Nachricht geschickt:\\ \tt \\ \tt \\ \tt "Ja, sowohl (5,3) als auch (6,3) sind auf meinem Brett frei! Lass uns P nach (5,3) bewegen. Das sollte dann unsere finale Anordnung sein: L bei (1,3), C bei (3,3), P bei (5,3)."\\ \tt \\ \tt \\ \tt Wie lautet dein nächster Befehl? 
	  } 
	   } 
	   } 
	 & \\ 
 

    \theutterance \stepcounter{utterance}  

    & \multicolumn{4}{p{0.6\linewidth}}{\cellcolor[rgb]{0.8,1,0.9}{%
	\makecell[{{p{\linewidth}}}]{% 
	  \tt {\tiny [A$\rangle$GM]}  
	 VERSCHIEBE: P, (5, 3) 
	  } 
	   } 
	   } 
	 & & \\ 
 

    \theutterance \stepcounter{utterance}  

    & & & \multicolumn{2}{p{0.3\linewidth}}{\cellcolor[rgb]{0.95,0.95,0.95}{%
	\makecell[{{p{\linewidth}}}]{% 
	  \tt {\tiny [GM$|$GM]}  
	 VERSCHIEBE: P, (5, 3) 
	  } 
	   } 
	   } 
	 & & \\ 
 

    \theutterance \stepcounter{utterance}  

    & & & \multicolumn{2}{p{0.3\linewidth}}{\cellcolor[rgb]{0.95,0.95,0.95}{%
	\makecell[{{p{\linewidth}}}]{% 
	  \tt {\tiny [GM$|$GM]}  
	 'P' erfolgreich nach (5,3) verschoben. Dein Raster sieht jetzt so aus:\\ \tt ```\\ \tt  1234567\\ \tt ╔═╤═╤═╤═╗\\ \tt ║◌│◌│◌│◌║ 1\\ \tt ╟─┼─┤◌│◌║ 2\\ \tt ║L│C│P│◌║ 3\\ \tt ║◌│◌├─┴─╢ 4\\ \tt ╟─┴─┤◌◌◌║ 5\\ \tt ║◌◌◌└─┬─╢ 6\\ \tt ║◌◌◌◌◌│◌║ 7\\ \tt ╚═════╧═╝\\ \tt ```\\ \tt  
	  } 
	   } 
	   } 
	 & & \\ 
 

    \theutterance \stepcounter{utterance}  

    & & \multicolumn{4}{p{0.6\linewidth}}{\cellcolor[rgb]{0.9,0.9,0.9}{%
	\makecell[{{p{\linewidth}}}]{% 
	  \tt {\tiny [GM$\rangle$B]}  
	 Deine Nachricht wurde an deinen Mitspieler übermittelt.\\ \tt \\ \tt \\ \tt Ihr seid gerade bei Runde 8 von maximal 12.\\ \tt Ihr habt gemeinsam 2 von maximal 8 Strafpunkten angesammelt.\\ \tt Dein Mitspieler hat ein Objekt auf seinem Raster bewegt.\\ \tt \\ \tt \\ \tt Wie lautet dein nächster Befehl? 
	  } 
	   } 
	   } 
	 & \\ 
 

    \theutterance \stepcounter{utterance}  

    & & & \multicolumn{4}{p{0.6\linewidth}}{\cellcolor[rgb]{1,0.85,0.72}{%
	\makecell[{{p{\linewidth}}}]{% 
	  \tt {\tiny [GM$\langle$B]}  
	 VERSCHIEBE: P, (5, 3) 
	  } 
	   } 
	   } 
	  \\ 
 

    \theutterance \stepcounter{utterance}  

    & & & \multicolumn{2}{p{0.3\linewidth}}{\cellcolor[rgb]{0.95,0.95,0.95}{%
	\makecell[{{p{\linewidth}}}]{% 
	  \tt {\tiny [GM$|$GM]}  
	 VERSCHIEBE: P, (5, 3) 
	  } 
	   } 
	   } 
	 & & \\ 
 

    \theutterance \stepcounter{utterance}  

    & & & \multicolumn{2}{p{0.3\linewidth}}{\cellcolor[rgb]{0.95,0.95,0.95}{%
	\makecell[{{p{\linewidth}}}]{% 
	  \tt {\tiny [GM$|$GM]}  
	 'P' erfolgreich nach (5,3) verschoben. Dein Raster sieht jetzt so aus:\\ \tt ```\\ \tt  1234567\\ \tt ╔═╤═╤═╤═╗\\ \tt ║◌│◌│◌│◌║ 1\\ \tt ╟─┼─┤◌│◌║ 2\\ \tt ║L│C│P│◌║ 3\\ \tt ║◌│◌├─┴─╢ 4\\ \tt ╟─┴─┤◌◌◌║ 5\\ \tt ║◌◌◌└─┬─╢ 6\\ \tt ║◌◌◌◌◌│◌║ 7\\ \tt ╚═════╧═╝\\ \tt ```\\ \tt  
	  } 
	   } 
	   } 
	 & & \\ 
 

    \theutterance \stepcounter{utterance}  

    & & \multicolumn{4}{p{0.6\linewidth}}{\cellcolor[rgb]{0.9,0.9,0.9}{%
	\makecell[{{p{\linewidth}}}]{% 
	  \tt {\tiny [A$\langle$GM]}  
	 'P' erfolgreich nach (5,3) verschoben. Dein Raster sieht jetzt so aus:\\ \tt ```\\ \tt  1234567\\ \tt ╔═╤═╤═╤═╗\\ \tt ║◌│◌│◌│◌║ 1\\ \tt ╟─┼─┤◌│◌║ 2\\ \tt ║L│C│P│◌║ 3\\ \tt ║◌│◌├─┴─╢ 4\\ \tt ╟─┴─┤◌◌◌║ 5\\ \tt ║◌◌◌└─┬─╢ 6\\ \tt ║◌◌◌◌◌│◌║ 7\\ \tt ╚═════╧═╝\\ \tt ```\\ \tt Ihr seid gerade bei Runde 8 von maximal 12.\\ \tt Ihr habt gemeinsam 2 von maximal 8 Strafpunkten angesammelt.\\ \tt Dein Mitspieler hat ein Objekt auf seinem Raster bewegt.\\ \tt \\ \tt \\ \tt Wie lautet dein nächster Befehl? 
	  } 
	   } 
	   } 
	 & \\ 
 

    \theutterance \stepcounter{utterance}  

    & \multicolumn{4}{p{0.6\linewidth}}{\cellcolor[rgb]{0.8,1,0.9}{%
	\makecell[{{p{\linewidth}}}]{% 
	  \tt {\tiny [A$\rangle$GM]}  
	 SPRICH: fertig? 
	  } 
	   } 
	   } 
	 & & \\ 
 

    \theutterance \stepcounter{utterance}  

    & & & \multicolumn{2}{p{0.3\linewidth}}{\cellcolor[rgb]{0.95,0.95,0.95}{%
	\makecell[{{p{\linewidth}}}]{% 
	  \tt {\tiny [GM$|$GM]}  
	 SPRICH: fertig? 
	  } 
	   } 
	   } 
	 & & \\ 
 

    \theutterance \stepcounter{utterance}  

    & & \multicolumn{4}{p{0.6\linewidth}}{\cellcolor[rgb]{0.9,0.9,0.9}{%
	\makecell[{{p{\linewidth}}}]{% 
	  \tt {\tiny [GM$\rangle$B]}  
	 'P' erfolgreich nach (5,3) verschoben. Dein Raster sieht jetzt so aus:\\ \tt ```\\ \tt  1234567\\ \tt ╔═╤═╤═╤═╗\\ \tt ║◌│◌│◌│◌║ 1\\ \tt ╟─┼─┤◌│◌║ 2\\ \tt ║L│C│P│◌║ 3\\ \tt ║◌│◌├─┴─╢ 4\\ \tt ╟─┴─┤◌◌◌║ 5\\ \tt ║◌◌◌└─┬─╢ 6\\ \tt ║◌◌◌◌◌│◌║ 7\\ \tt ╚═════╧═╝\\ \tt ```\\ \tt Ihr seid gerade bei Runde 9 von maximal 12.\\ \tt Ihr habt gemeinsam 2 von maximal 8 Strafpunkten angesammelt.\\ \tt Dein Mitspieler hat dir diese Nachricht geschickt:\\ \tt \\ \tt \\ \tt "fertig?"\\ \tt \\ \tt \\ \tt Wie lautet dein nächster Befehl? 
	  } 
	   } 
	   } 
	 & \\ 
 

    \theutterance \stepcounter{utterance}  

    & & & \multicolumn{4}{p{0.6\linewidth}}{\cellcolor[rgb]{1,0.85,0.72}{%
	\makecell[{{p{\linewidth}}}]{% 
	  \tt {\tiny [GM$\langle$B]}  
	 SPRICH: fertig! 
	  } 
	   } 
	   } 
	  \\ 
 

    \theutterance \stepcounter{utterance}  

    & & & \multicolumn{2}{p{0.3\linewidth}}{\cellcolor[rgb]{0.95,0.95,0.95}{%
	\makecell[{{p{\linewidth}}}]{% 
	  \tt {\tiny [GM$|$GM]}  
	 SPRICH: fertig! 
	  } 
	   } 
	   } 
	 & & \\ 
 

    \theutterance \stepcounter{utterance}  

    & & & \multicolumn{2}{p{0.3\linewidth}}{\cellcolor[rgb]{0.95,0.95,0.95}{%
	\makecell[{{p{\linewidth}}}]{% 
	  \tt {\tiny [GM$|$GM]}  
	 true 
	  } 
	   } 
	   } 
	 & & \\ 
 

    \theutterance \stepcounter{utterance}  

    & & & \multicolumn{2}{p{0.3\linewidth}}{\cellcolor[rgb]{0.95,0.95,0.95}{%
	\makecell[{{p{\linewidth}}}]{% 
	  \tt {\tiny [GM$|$GM]}  
	 * success: True\\ \tt * lose: False\\ \tt * aborted: False\\ \tt -------\\ \tt * Shifts: 2.00\\ \tt * Max Shifts: 4.00\\ \tt * Min Shifts: 2.00\\ \tt * End Distance Sum: 0.00\\ \tt * Init Distance Sum: 8.24\\ \tt * Expected Distance Sum: 12.57\\ \tt * Penalties: 2.00\\ \tt * Max Penalties: 8.00\\ \tt * Rounds: 9.00\\ \tt * Max Rounds: 12.00\\ \tt * Object Count: 3.00\\ \tt  
	  } 
	   } 
	   } 
	 & & \\ 
 

    \theutterance \stepcounter{utterance}  

    & & & \multicolumn{2}{p{0.3\linewidth}}{\cellcolor[rgb]{0.95,0.95,0.95}{%
	\makecell[{{p{\linewidth}}}]{% 
	  \tt {\tiny [GM$|$GM]}  
	 * Main Score: 100.00\\ \tt \\ \tt -------\\ \tt * Distance Score: 1.00\\ \tt * Consistency Score: 1.00\\ \tt * Coverage Score: 1.00\\ \tt * Penalty Score: 1.00\\ \tt * Alternative Penalty Score: 1.00\\ \tt * Alternative Main Score: 100.00\\ \tt \\ \tt -------\\ \tt * Shifts: 2.00\\ \tt * Max Shifts: 4.00\\ \tt * Min Shifts: 2.00\\ \tt * End Distance Sum: 0.00\\ \tt * Init Distance Sum: 8.24\\ \tt * Expected Distance Sum: 12.57\\ \tt * Penalties: 2.00\\ \tt * Max Penalties: 8.00\\ \tt * Rounds: 9.00\\ \tt * Max Rounds: 12.00\\ \tt * Object Count: 3.00\\ \tt  
	  } 
	   } 
	   } 
	 & & \\ 
 

\end{supertabular}
}

\end{document}
