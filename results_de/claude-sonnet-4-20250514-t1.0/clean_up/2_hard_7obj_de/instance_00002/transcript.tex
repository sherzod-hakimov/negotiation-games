
\documentclass{article}
\usepackage{colortbl}
\usepackage{makecell}
\usepackage{multirow}
\usepackage{supertabular}

\begin{document}

\newcounter{utterance}

\twocolumn

{ \footnotesize  \setcounter{utterance}{1}
\setlength{\tabcolsep}{0pt}
\begin{supertabular}{c@{$\;$}|p{.15\linewidth}@{}p{.15\linewidth}p{.15\linewidth}p{.15\linewidth}p{.15\linewidth}p{.15\linewidth}}

    \# & $\;$A & \multicolumn{4}{c}{Game Master} & $\;\:$B\\
    \hline 

    \theutterance \stepcounter{utterance}  

    & & \multicolumn{4}{p{0.6\linewidth}}{\cellcolor[rgb]{0.9,0.9,0.9}{%
	\makecell[{{p{\linewidth}}}]{% 
	  \tt {\tiny [A$\langle$GM]}  
	 Ich bin der Spielleiter, und du spielst ein kooperatives Spiel mit dem folgenden Raster als Spielbrett:\\ \tt \\ \tt ```\\ \tt  1234567\\ \tt ╔═╤═╤═╤═╗\\ \tt ║A│◌│O│◌║ 1\\ \tt ╟─┼─┤◌│◌║ 2\\ \tt ║◌│T│◌│P║ 3\\ \tt ║◌│◌├─┴─╢ 4\\ \tt ╟─┴─┤◌◌◌║ 5\\ \tt ║D◌◌└─┬─╢ 6\\ \tt ║◌◌◌S◌│M║ 7\\ \tt ╚═════╧═╝\\ \tt \\ \tt ```\\ \tt \\ \tt * Am oberen Rand sind die x-Koordinaten angegeben, die nach rechts hin zunehmen, und am rechten Rand die y-Koordinaten, die nach unten zunehmen.\\ \tt * Die folgenden Objekte sind zufällig auf deinem Raster platziert: 'P', 'O', 'T', 'S', 'D', 'A', 'M'.\\ \tt \\ \tt \\ \tt Dein Mitspieler sieht eine Variation des Spielbretts, auf dem dieselben Objekte an anderen zufälligen Positionen platziert sind. Du kannst das Spielbrett des anderen Spielers nicht sehen, und er kann deins nicht sehen.\\ \tt \\ \tt \\ \tt **Ziel des Spiels:**\\ \tt \\ \tt Beide Spieler müssen die Objekte auf ihrem jeweiligen Spielbrett so bewegen, dass identische Objekte am Ende an denselben Koordinaten liegen. Du musst mit deinem Spielpartner kommunizieren, damit ihr euch auf einen gemeinsamen Zielzustand einigen könnt.\\ \tt \\ \tt \\ \tt **Spielregeln:**\\ \tt \\ \tt * In jedem Spielzug kannst du genau eine der folgenden zwei Aktionen ausführen:\\ \tt 1. `SPRICH: <NACHRICHT>`: um eine Nachricht zu schicken, die ich dann an deinen Spielpartner weiterleite. Als Nachricht gilt alles bis zum nächsten Zeilenumbruch.\\ \tt 2. `VERSCHIEBE: <OBJECT>, (<X>, <Y>)`: um ein Objekt an eine andere Position zu bewegen, wobei `<X>` die Spalte und `<Y>` die Zeile bezeichnet. Ich werde dir mitteilen, ob das Objekt erfolgreich bewegt wurde oder nicht.\\ \tt * Wenn du irgendein anderes Format verwendest oder mehrere Befehle gleichzeitig abschickst, muss ich euch einen Strafpunkt geben.\\ \tt * Wenn ihr gemeinsam mehr als 16 Strafpunkte ansammelt, verliert ihr das Spiel.\\ \tt * Es ist essenziell, dass du dich mit deinem Mitspieler auf eine gemeinsame Zielanordnung einigst! Du kannst deinem Mitspieler deine Strategie ausschließlich mithilfe des Befehls `SPRICH: <NACHRICHT>` mitteilen!\\ \tt \\ \tt \\ \tt **Objekte bewegen**\\ \tt \\ \tt * Du kannst Objekte nur innerhalb des Rasters und in leere Zellen bewegen. Leere Zellen erkennst am Zeichen '◌'.\\ \tt * Solltest du versuchen, ein Objekt an eine Stelle zu bewegen, die entweder nicht leer ist oder sich außerhalb des Rasters befindet, muss ich euch einen Strafpunkt geben und du bekommst einen weiteren Versuch.\\ \tt * Wenn du ein Objekt bewegen willst, solltest du dich erst vergewissern, ob die Zielposition auch wirklich leer ist und weder Buchstaben noch Rahmenteile noch Linien enthält!\\ \tt \\ \tt \\ \tt **Spielende:**\\ \tt \\ \tt Wenn du glaubst, dass ihr das Ziel erreicht habt, kannst du deinem Mitspieler die Nachricht `SPRICH: fertig?` schicken. Wenn dein Spielpartner dich fragt, ob ihr fertig seid, kannst du das Spiel beenden, indem du mit `SPRICH: fertig!` antwortest.\\ \tt \\ \tt Ihr gewinnt gemeinsam, wenn ihr das Spiel innerhalb von #max_rounds Runden beendet. Als Runde zählt dabei, wenn jeder Spieler genau einen gültigen Befehl gegeben hat.\\ \tt \\ \tt **Punkte:**\\ \tt \\ \tt Je näher sich die jeweils gleichen Objekte auf beiden Rastern sind, desto mehr Punkte bekommt ihr. Für Strafpunkte gibt es Punktabzug. Könnt ihr den Rekord brechen?\\ \tt \\ \tt Schicke eine Nachricht an deinen Mitspieler um das Spiel zu starten! 
	  } 
	   } 
	   } 
	 & \\ 
 

    \theutterance \stepcounter{utterance}  

    & \multicolumn{4}{p{0.6\linewidth}}{\cellcolor[rgb]{0.8,1,0.9}{%
	\makecell[{{p{\linewidth}}}]{% 
	  \tt {\tiny [A$\rangle$GM]}  
	 SPRICH: Hallo! Lass uns das Spiel beginnen. Hier ist meine aktuelle Anordnung: A(1,1), O(3,1), T(2,3), P(4,3), D(1,6), S(4,7), M(6,7). Kannst du mir deine Positionen mitteilen? Dann können wir uns auf eine gemeinsame Zielanordnung einigen! 
	  } 
	   } 
	   } 
	 & & \\ 
 

    \theutterance \stepcounter{utterance}  

    & & & \multicolumn{2}{p{0.3\linewidth}}{\cellcolor[rgb]{0.95,0.95,0.95}{%
	\makecell[{{p{\linewidth}}}]{% 
	  \tt {\tiny [GM$|$GM]}  
	 SPRICH: Hallo! Lass uns das Spiel beginnen. Hier ist meine aktuelle Anordnung: A(1,1), O(3,1), T(2,3), P(4,3), D(1,6), S(4,7), M(6,7). Kannst du mir deine Positionen mitteilen? Dann können wir uns auf eine gemeinsame Zielanordnung einigen! 
	  } 
	   } 
	   } 
	 & & \\ 
 

    \theutterance \stepcounter{utterance}  

    & & \multicolumn{4}{p{0.6\linewidth}}{\cellcolor[rgb]{0.9,0.9,0.9}{%
	\makecell[{{p{\linewidth}}}]{% 
	  \tt {\tiny [GM$\rangle$B]}  
	 Ich bin der Spielleiter, und du spielst ein kooperatives Spiel mit dem folgenden Raster als Spielbrett:\\ \tt \\ \tt ```\\ \tt  1234567\\ \tt ╔═╤═╤═╤═╗\\ \tt ║A│◌│◌│M║ 1\\ \tt ╟─┼─┤O│T║ 2\\ \tt ║◌│D│◌│◌║ 3\\ \tt ║◌│◌├─┴─╢ 4\\ \tt ╟─┴─┤◌◌◌║ 5\\ \tt ║PS◌└─┬─╢ 6\\ \tt ║◌◌◌◌◌│◌║ 7\\ \tt ╚═════╧═╝\\ \tt \\ \tt ```\\ \tt \\ \tt * Am oberen Rand sind die x-Koordinaten angegeben, die nach rechts hin zunehmen, und am rechten Rand die y-Koordinaten, die nach unten zunehmen.\\ \tt * Die folgenden Objekte sind zufällig auf deinem Raster platziert: 'P', 'O', 'T', 'S', 'D', 'A', 'M'.\\ \tt \\ \tt \\ \tt Dein Mitspieler sieht eine Variation des Spielbretts, auf dem dieselben Objekte an anderen zufälligen Positionen platziert sind. Du kannst das Spielbrett des anderen Spielers nicht sehen, und er kann deins nicht sehen.\\ \tt \\ \tt \\ \tt **Ziel des Spiels:**\\ \tt \\ \tt Beide Spieler müssen die Objekte auf ihrem jeweiligen Spielbrett so bewegen, dass identische Objekte am Ende an denselben Koordinaten liegen. Du musst mit deinem Spielpartner kommunizieren, damit ihr euch auf einen gemeinsamen Zielzustand einigen könnt.\\ \tt \\ \tt \\ \tt **Spielregeln:**\\ \tt \\ \tt * In jedem Spielzug kannst du genau eine der folgenden zwei Aktionen ausführen:\\ \tt 1. `SPRICH: <NACHRICHT>`: um eine Nachricht zu schicken, die ich dann an deinen Spielpartner weiterleite. Als Nachricht gilt alles bis zum nächsten Zeilenumbruch.\\ \tt 2. `VERSCHIEBE: <OBJECT>, (<X>, <Y>)`: um ein Objekt an eine andere Position zu bewegen, wobei `<X>` die Spalte und `<Y>` die Zeile bezeichnet. Ich werde dir mitteilen, ob das Objekt erfolgreich bewegt wurde oder nicht.\\ \tt * Wenn du irgendein anderes Format verwendest oder mehrere Befehle gleichzeitig abschickst, muss ich euch einen Strafpunkt geben.\\ \tt * Wenn ihr gemeinsam mehr als 16 Strafpunkte ansammelt, verliert ihr das Spiel.\\ \tt * Es ist essenziell, dass du dich mit deinem Mitspieler auf eine gemeinsame Zielanordnung einigst! Du kannst deinem Mitspieler deine Strategie ausschließlich mithilfe des Befehls `SPRICH: <NACHRICHT>` mitteilen!\\ \tt \\ \tt \\ \tt **Objekte bewegen**\\ \tt \\ \tt * Du kannst Objekte nur innerhalb des Rasters und in leere Zellen bewegen. Leere Zellen erkennst am Zeichen '◌'.\\ \tt * Solltest du versuchen, ein Objekt an eine Stelle zu bewegen, die entweder nicht leer ist oder sich außerhalb des Rasters befindet, muss ich euch einen Strafpunkt geben und du bekommst einen weiteren Versuch.\\ \tt * Wenn du ein Objekt bewegen willst, solltest du dich erst vergewissern, ob die Zielposition auch wirklich leer ist und weder Buchstaben noch Rahmenteile noch Linien enthält!\\ \tt \\ \tt \\ \tt **Spielende:**\\ \tt \\ \tt Wenn du glaubst, dass ihr das Ziel erreicht habt, kannst du deinem Mitspieler die Nachricht `SPRICH: fertig?` schicken. Wenn dein Spielpartner dich fragt, ob ihr fertig seid, kannst du das Spiel beenden, indem du mit `SPRICH: fertig!` antwortest.\\ \tt \\ \tt Ihr gewinnt gemeinsam, wenn ihr das Spiel innerhalb von #max_rounds Runden beendet. Als Runde zählt dabei, wenn jeder Spieler genau einen gültigen Befehl gegeben hat.\\ \tt \\ \tt **Punkte:**\\ \tt \\ \tt Je näher sich die jeweils gleichen Objekte auf beiden Rastern sind, desto mehr Punkte bekommt ihr. Für Strafpunkte gibt es Punktabzug. Könnt ihr den Rekord brechen?\\ \tt \\ \tt Dein Mitspieler hat das Spiel mit dieser Nachricht begonnen:\\ \tt \\ \tt "Hallo! Lass uns das Spiel beginnen. Hier ist meine aktuelle Anordnung: A(1,1), O(3,1), T(2,3), P(4,3), D(1,6), S(4,7), M(6,7). Kannst du mir deine Positionen mitteilen? Dann können wir uns auf eine gemeinsame Zielanordnung einigen!"\\ \tt \\ \tt Wie lautet dein erster Befehl? 
	  } 
	   } 
	   } 
	 & \\ 
 

    \theutterance \stepcounter{utterance}  

    & & & \multicolumn{4}{p{0.6\linewidth}}{\cellcolor[rgb]{1,0.85,0.72}{%
	\makecell[{{p{\linewidth}}}]{% 
	  \tt {\tiny [GM$\langle$B]}  
	 SPRICH: Hallo! Hier sind meine aktuellen Positionen: A(1,1), M(4,1), O(3,2), T(4,2), D(2,3), P(1,6), S(2,6). Interessant, wir haben bereits A beide an (1,1) und D beide in Zeile 6. Lass uns eine gemeinsame Zielanordnung festlegen - welche Positionen schlägst du vor? 
	  } 
	   } 
	   } 
	  \\ 
 

    \theutterance \stepcounter{utterance}  

    & & & \multicolumn{2}{p{0.3\linewidth}}{\cellcolor[rgb]{0.95,0.95,0.95}{%
	\makecell[{{p{\linewidth}}}]{% 
	  \tt {\tiny [GM$|$GM]}  
	 SPRICH: Hallo! Hier sind meine aktuellen Positionen: A(1,1), M(4,1), O(3,2), T(4,2), D(2,3), P(1,6), S(2,6). Interessant, wir haben bereits A beide an (1,1) und D beide in Zeile 6. Lass uns eine gemeinsame Zielanordnung festlegen - welche Positionen schlägst du vor? 
	  } 
	   } 
	   } 
	 & & \\ 
 

    \theutterance \stepcounter{utterance}  

    & & \multicolumn{4}{p{0.6\linewidth}}{\cellcolor[rgb]{0.9,0.9,0.9}{%
	\makecell[{{p{\linewidth}}}]{% 
	  \tt {\tiny [A$\langle$GM]}  
	 Deine Nachricht wurde an deinen Mitspieler übermittelt.\\ \tt \\ \tt \\ \tt Ihr seid gerade bei Runde 1 von maximal 28.\\ \tt Ihr habt gemeinsam 0 von maximal 16 Strafpunkten angesammelt.\\ \tt Dein Mitspieler hat dir diese Nachricht geschickt:\\ \tt \\ \tt \\ \tt "Hallo! Hier sind meine aktuellen Positionen: A(1,1), M(4,1), O(3,2), T(4,2), D(2,3), P(1,6), S(2,6). Interessant, wir haben bereits A beide an (1,1) und D beide in Zeile 6. Lass uns eine gemeinsame Zielanordnung festlegen - welche Positionen schlägst du vor?"\\ \tt \\ \tt \\ \tt Wie lautet dein nächster Befehl? 
	  } 
	   } 
	   } 
	 & \\ 
 

    \theutterance \stepcounter{utterance}  

    & \multicolumn{4}{p{0.6\linewidth}}{\cellcolor[rgb]{0.8,1,0.9}{%
	\makecell[{{p{\linewidth}}}]{% 
	  \tt {\tiny [A$\rangle$GM]}  
	 SPRICH: Perfekt, A ist bereits richtig positioniert! Lass mich eine Zielanordnung vorschlagen basierend auf den freien Plätzen: A(1,1) bleibt, O(2,1), T(2,2), P(3,2), D(4,2), S(1,4), M(2,4). Damit hätten wir alle Objekte in der oberen linken Ecke konzentriert. Was denkst du dazu? Oder hast du einen besseren Vorschlag? 
	  } 
	   } 
	   } 
	 & & \\ 
 

    \theutterance \stepcounter{utterance}  

    & & & \multicolumn{2}{p{0.3\linewidth}}{\cellcolor[rgb]{0.95,0.95,0.95}{%
	\makecell[{{p{\linewidth}}}]{% 
	  \tt {\tiny [GM$|$GM]}  
	 SPRICH: Perfekt, A ist bereits richtig positioniert! Lass mich eine Zielanordnung vorschlagen basierend auf den freien Plätzen: A(1,1) bleibt, O(2,1), T(2,2), P(3,2), D(4,2), S(1,4), M(2,4). Damit hätten wir alle Objekte in der oberen linken Ecke konzentriert. Was denkst du dazu? Oder hast du einen besseren Vorschlag? 
	  } 
	   } 
	   } 
	 & & \\ 
 

    \theutterance \stepcounter{utterance}  

    & & \multicolumn{4}{p{0.6\linewidth}}{\cellcolor[rgb]{0.9,0.9,0.9}{%
	\makecell[{{p{\linewidth}}}]{% 
	  \tt {\tiny [GM$\rangle$B]}  
	 Deine Nachricht wurde an deinen Mitspieler übermittelt.\\ \tt \\ \tt \\ \tt Ihr seid gerade bei Runde 2 von maximal 28.\\ \tt Ihr habt gemeinsam 0 von maximal 16 Strafpunkten angesammelt.\\ \tt Dein Mitspieler hat dir diese Nachricht geschickt:\\ \tt \\ \tt \\ \tt "Perfekt, A ist bereits richtig positioniert! Lass mich eine Zielanordnung vorschlagen basierend auf den freien Plätzen: A(1,1) bleibt, O(2,1), T(2,2), P(3,2), D(4,2), S(1,4), M(2,4). Damit hätten wir alle Objekte in der oberen linken Ecke konzentriert. Was denkst du dazu? Oder hast du einen besseren Vorschlag?"\\ \tt \\ \tt \\ \tt Wie lautet dein nächster Befehl? 
	  } 
	   } 
	   } 
	 & \\ 
 

    \theutterance \stepcounter{utterance}  

    & & & \multicolumn{4}{p{0.6\linewidth}}{\cellcolor[rgb]{1,0.85,0.72}{%
	\makecell[{{p{\linewidth}}}]{% 
	  \tt {\tiny [GM$\langle$B]}  
	 SPRICH: Ich sehe ein Problem mit deinem Vorschlag: Position (2,2) ist blockiert durch eine Linie, und (3,2) sowie (4,2) sind aktuell von meinem O und T belegt. Lass mich einen alternativen Vorschlag machen: A(1,1), O(2,1), M(3,1), T(1,3), D(2,3), P(3,3), S(1,4). Diese Positionen sind alle frei und ergeben eine kompakte Anordnung. Was denkst du? 
	  } 
	   } 
	   } 
	  \\ 
 

    \theutterance \stepcounter{utterance}  

    & & & \multicolumn{2}{p{0.3\linewidth}}{\cellcolor[rgb]{0.95,0.95,0.95}{%
	\makecell[{{p{\linewidth}}}]{% 
	  \tt {\tiny [GM$|$GM]}  
	 SPRICH: Ich sehe ein Problem mit deinem Vorschlag: Position (2,2) ist blockiert durch eine Linie, und (3,2) sowie (4,2) sind aktuell von meinem O und T belegt. Lass mich einen alternativen Vorschlag machen: A(1,1), O(2,1), M(3,1), T(1,3), D(2,3), P(3,3), S(1,4). Diese Positionen sind alle frei und ergeben eine kompakte Anordnung. Was denkst du? 
	  } 
	   } 
	   } 
	 & & \\ 
 

    \theutterance \stepcounter{utterance}  

    & & \multicolumn{4}{p{0.6\linewidth}}{\cellcolor[rgb]{0.9,0.9,0.9}{%
	\makecell[{{p{\linewidth}}}]{% 
	  \tt {\tiny [A$\langle$GM]}  
	 Deine Nachricht wurde an deinen Mitspieler übermittelt.\\ \tt \\ \tt \\ \tt Ihr seid gerade bei Runde 2 von maximal 28.\\ \tt Ihr habt gemeinsam 0 von maximal 16 Strafpunkten angesammelt.\\ \tt Dein Mitspieler hat dir diese Nachricht geschickt:\\ \tt \\ \tt \\ \tt "Ich sehe ein Problem mit deinem Vorschlag: Position (2,2) ist blockiert durch eine Linie, und (3,2) sowie (4,2) sind aktuell von meinem O und T belegt. Lass mich einen alternativen Vorschlag machen: A(1,1), O(2,1), M(3,1), T(1,3), D(2,3), P(3,3), S(1,4). Diese Positionen sind alle frei und ergeben eine kompakte Anordnung. Was denkst du?"\\ \tt \\ \tt \\ \tt Wie lautet dein nächster Befehl? 
	  } 
	   } 
	   } 
	 & \\ 
 

    \theutterance \stepcounter{utterance}  

    & \multicolumn{4}{p{0.6\linewidth}}{\cellcolor[rgb]{0.8,1,0.9}{%
	\makecell[{{p{\linewidth}}}]{% 
	  \tt {\tiny [A$\rangle$GM]}  
	 SPRICH: Dein Vorschlag ist viel besser! A(1,1), O(2,1), M(3,1), T(1,3), D(2,3), P(3,3), S(1,4) - das funktioniert perfekt und alle Positionen sind auf meinem Brett verfügbar. Ich stimme zu! Soll ich anfangen zu bewegen oder möchtest du beginnen? 
	  } 
	   } 
	   } 
	 & & \\ 
 

    \theutterance \stepcounter{utterance}  

    & & & \multicolumn{2}{p{0.3\linewidth}}{\cellcolor[rgb]{0.95,0.95,0.95}{%
	\makecell[{{p{\linewidth}}}]{% 
	  \tt {\tiny [GM$|$GM]}  
	 SPRICH: Dein Vorschlag ist viel besser! A(1,1), O(2,1), M(3,1), T(1,3), D(2,3), P(3,3), S(1,4) - das funktioniert perfekt und alle Positionen sind auf meinem Brett verfügbar. Ich stimme zu! Soll ich anfangen zu bewegen oder möchtest du beginnen? 
	  } 
	   } 
	   } 
	 & & \\ 
 

    \theutterance \stepcounter{utterance}  

    & & \multicolumn{4}{p{0.6\linewidth}}{\cellcolor[rgb]{0.9,0.9,0.9}{%
	\makecell[{{p{\linewidth}}}]{% 
	  \tt {\tiny [GM$\rangle$B]}  
	 Deine Nachricht wurde an deinen Mitspieler übermittelt.\\ \tt \\ \tt \\ \tt Ihr seid gerade bei Runde 3 von maximal 28.\\ \tt Ihr habt gemeinsam 0 von maximal 16 Strafpunkten angesammelt.\\ \tt Dein Mitspieler hat dir diese Nachricht geschickt:\\ \tt \\ \tt \\ \tt "Dein Vorschlag ist viel besser! A(1,1), O(2,1), M(3,1), T(1,3), D(2,3), P(3,3), S(1,4) - das funktioniert perfekt und alle Positionen sind auf meinem Brett verfügbar. Ich stimme zu! Soll ich anfangen zu bewegen oder möchtest du beginnen?"\\ \tt \\ \tt \\ \tt Wie lautet dein nächster Befehl? 
	  } 
	   } 
	   } 
	 & \\ 
 

    \theutterance \stepcounter{utterance}  

    & & & \multicolumn{4}{p{0.6\linewidth}}{\cellcolor[rgb]{1,0.85,0.72}{%
	\makecell[{{p{\linewidth}}}]{% 
	  \tt {\tiny [GM$\langle$B]}  
	 VERSCHIEBE: M, (3, 1) 
	  } 
	   } 
	   } 
	  \\ 
 

    \theutterance \stepcounter{utterance}  

    & & & \multicolumn{2}{p{0.3\linewidth}}{\cellcolor[rgb]{0.95,0.95,0.95}{%
	\makecell[{{p{\linewidth}}}]{% 
	  \tt {\tiny [GM$|$GM]}  
	 VERSCHIEBE: M, (3, 1) 
	  } 
	   } 
	   } 
	 & & \\ 
 

    \theutterance \stepcounter{utterance}  

    & & & \multicolumn{2}{p{0.3\linewidth}}{\cellcolor[rgb]{0.95,0.95,0.95}{%
	\makecell[{{p{\linewidth}}}]{% 
	  \tt {\tiny [GM$|$GM]}  
	 'M' erfolgreich nach (3,1) verschoben. Dein Raster sieht jetzt so aus:\\ \tt ```\\ \tt  1234567\\ \tt ╔═╤═╤═╤═╗\\ \tt ║A│M│◌│◌║ 1\\ \tt ╟─┼─┤O│T║ 2\\ \tt ║◌│D│◌│◌║ 3\\ \tt ║◌│◌├─┴─╢ 4\\ \tt ╟─┴─┤◌◌◌║ 5\\ \tt ║PS◌└─┬─╢ 6\\ \tt ║◌◌◌◌◌│◌║ 7\\ \tt ╚═════╧═╝\\ \tt ```\\ \tt  
	  } 
	   } 
	   } 
	 & & \\ 
 

    \theutterance \stepcounter{utterance}  

    & & \multicolumn{4}{p{0.6\linewidth}}{\cellcolor[rgb]{0.9,0.9,0.9}{%
	\makecell[{{p{\linewidth}}}]{% 
	  \tt {\tiny [A$\langle$GM]}  
	 Deine Nachricht wurde an deinen Mitspieler übermittelt.\\ \tt \\ \tt \\ \tt Ihr seid gerade bei Runde 3 von maximal 28.\\ \tt Ihr habt gemeinsam 0 von maximal 16 Strafpunkten angesammelt.\\ \tt Dein Mitspieler hat ein Objekt auf seinem Raster bewegt.\\ \tt \\ \tt \\ \tt Wie lautet dein nächster Befehl? 
	  } 
	   } 
	   } 
	 & \\ 
 

    \theutterance \stepcounter{utterance}  

    & \multicolumn{4}{p{0.6\linewidth}}{\cellcolor[rgb]{0.8,1,0.9}{%
	\makecell[{{p{\linewidth}}}]{% 
	  \tt {\tiny [A$\rangle$GM]}  
	 VERSCHIEBE: O, (2, 1) 
	  } 
	   } 
	   } 
	 & & \\ 
 

    \theutterance \stepcounter{utterance}  

    & & & \multicolumn{2}{p{0.3\linewidth}}{\cellcolor[rgb]{0.95,0.95,0.95}{%
	\makecell[{{p{\linewidth}}}]{% 
	  \tt {\tiny [GM$|$GM]}  
	 VERSCHIEBE: O, (2, 1) 
	  } 
	   } 
	   } 
	 & & \\ 
 

    \theutterance \stepcounter{utterance}  

    & & & \multicolumn{2}{p{0.3\linewidth}}{\cellcolor[rgb]{0.95,0.95,0.95}{%
	\makecell[{{p{\linewidth}}}]{% 
	  \tt {\tiny [GM$|$GM]}  
	 Strafpunkt: (2,1) ist nicht leer, sondern enthält '│'.\\ \tt \\ \tt Ihr habt gemeinsam 1 von maximal 16 Strafpunkten angesammelt.\\ \tt Bitte versuche es noch mal! 
	  } 
	   } 
	   } 
	 & & \\ 
 

    \theutterance \stepcounter{utterance}  

    & & \multicolumn{4}{p{0.6\linewidth}}{\cellcolor[rgb]{0.9,0.9,0.9}{%
	\makecell[{{p{\linewidth}}}]{% 
	  \tt {\tiny [A$\langle$GM]}  
	 Strafpunkt: (2,1) ist nicht leer, sondern enthält '│'.\\ \tt \\ \tt Ihr habt gemeinsam 1 von maximal 16 Strafpunkten angesammelt.\\ \tt Bitte versuche es noch mal! 
	  } 
	   } 
	   } 
	 & \\ 
 

    \theutterance \stepcounter{utterance}  

    & \multicolumn{4}{p{0.6\linewidth}}{\cellcolor[rgb]{0.8,1,0.9}{%
	\makecell[{{p{\linewidth}}}]{% 
	  \tt {\tiny [A$\rangle$GM]}  
	 SPRICH: Entschuldigung, (2,1) ist blockiert durch eine Linie. Wir müssen unseren Plan anpassen. Wie wäre es mit: A(1,1) bleibt, O(4,1), M(5,1), T(1,3), D(2,3), P(3,3), S(5,5)? Kannst du prüfen ob diese Positionen bei dir frei sind? 
	  } 
	   } 
	   } 
	 & & \\ 
 

    \theutterance \stepcounter{utterance}  

    & & & \multicolumn{2}{p{0.3\linewidth}}{\cellcolor[rgb]{0.95,0.95,0.95}{%
	\makecell[{{p{\linewidth}}}]{% 
	  \tt {\tiny [GM$|$GM]}  
	 SPRICH: Entschuldigung, (2,1) ist blockiert durch eine Linie. Wir müssen unseren Plan anpassen. Wie wäre es mit: A(1,1) bleibt, O(4,1), M(5,1), T(1,3), D(2,3), P(3,3), S(5,5)? Kannst du prüfen ob diese Positionen bei dir frei sind? 
	  } 
	   } 
	   } 
	 & & \\ 
 

    \theutterance \stepcounter{utterance}  

    & & \multicolumn{4}{p{0.6\linewidth}}{\cellcolor[rgb]{0.9,0.9,0.9}{%
	\makecell[{{p{\linewidth}}}]{% 
	  \tt {\tiny [GM$\rangle$B]}  
	 'M' erfolgreich nach (3,1) verschoben. Dein Raster sieht jetzt so aus:\\ \tt ```\\ \tt  1234567\\ \tt ╔═╤═╤═╤═╗\\ \tt ║A│M│◌│◌║ 1\\ \tt ╟─┼─┤O│T║ 2\\ \tt ║◌│D│◌│◌║ 3\\ \tt ║◌│◌├─┴─╢ 4\\ \tt ╟─┴─┤◌◌◌║ 5\\ \tt ║PS◌└─┬─╢ 6\\ \tt ║◌◌◌◌◌│◌║ 7\\ \tt ╚═════╧═╝\\ \tt ```\\ \tt Ihr seid gerade bei Runde 4 von maximal 28.\\ \tt Ihr habt gemeinsam 1 von maximal 16 Strafpunkten angesammelt.\\ \tt Dein Mitspieler hat dir diese Nachricht geschickt:\\ \tt \\ \tt \\ \tt "Entschuldigung, (2,1) ist blockiert durch eine Linie. Wir müssen unseren Plan anpassen. Wie wäre es mit: A(1,1) bleibt, O(4,1), M(5,1), T(1,3), D(2,3), P(3,3), S(5,5)? Kannst du prüfen ob diese Positionen bei dir frei sind?"\\ \tt \\ \tt \\ \tt Wie lautet dein nächster Befehl? 
	  } 
	   } 
	   } 
	 & \\ 
 

    \theutterance \stepcounter{utterance}  

    & & & \multicolumn{4}{p{0.6\linewidth}}{\cellcolor[rgb]{1,0.85,0.72}{%
	\makecell[{{p{\linewidth}}}]{% 
	  \tt {\tiny [GM$\langle$B]}  
	 SPRICH: Problem: Positionen (5,1) und (5,5) liegen außerhalb meines Rasters - mein Brett geht nur bis Spalte 4! Lass mich einen neuen Vorschlag machen mit verfügbaren Positionen: A(1,1), O(4,1), M(3,1) bleibt, T(1,3), D(2,3) bleibt, P(3,3), S(1,7). Alle diese Positionen sind auf meinem Brett frei. Passt das auch bei dir? 
	  } 
	   } 
	   } 
	  \\ 
 

    \theutterance \stepcounter{utterance}  

    & & & \multicolumn{2}{p{0.3\linewidth}}{\cellcolor[rgb]{0.95,0.95,0.95}{%
	\makecell[{{p{\linewidth}}}]{% 
	  \tt {\tiny [GM$|$GM]}  
	 SPRICH: Problem: Positionen (5,1) und (5,5) liegen außerhalb meines Rasters - mein Brett geht nur bis Spalte 4! Lass mich einen neuen Vorschlag machen mit verfügbaren Positionen: A(1,1), O(4,1), M(3,1) bleibt, T(1,3), D(2,3) bleibt, P(3,3), S(1,7). Alle diese Positionen sind auf meinem Brett frei. Passt das auch bei dir? 
	  } 
	   } 
	   } 
	 & & \\ 
 

    \theutterance \stepcounter{utterance}  

    & & \multicolumn{4}{p{0.6\linewidth}}{\cellcolor[rgb]{0.9,0.9,0.9}{%
	\makecell[{{p{\linewidth}}}]{% 
	  \tt {\tiny [A$\langle$GM]}  
	 Deine Nachricht wurde an deinen Mitspieler übermittelt.\\ \tt \\ \tt \\ \tt Ihr seid gerade bei Runde 4 von maximal 28.\\ \tt Ihr habt gemeinsam 1 von maximal 16 Strafpunkten angesammelt.\\ \tt Dein Mitspieler hat dir diese Nachricht geschickt:\\ \tt \\ \tt \\ \tt "Problem: Positionen (5,1) und (5,5) liegen außerhalb meines Rasters - mein Brett geht nur bis Spalte 4! Lass mich einen neuen Vorschlag machen mit verfügbaren Positionen: A(1,1), O(4,1), M(3,1) bleibt, T(1,3), D(2,3) bleibt, P(3,3), S(1,7). Alle diese Positionen sind auf meinem Brett frei. Passt das auch bei dir?"\\ \tt \\ \tt \\ \tt Wie lautet dein nächster Befehl? 
	  } 
	   } 
	   } 
	 & \\ 
 

    \theutterance \stepcounter{utterance}  

    & \multicolumn{4}{p{0.6\linewidth}}{\cellcolor[rgb]{0.8,1,0.9}{%
	\makecell[{{p{\linewidth}}}]{% 
	  \tt {\tiny [A$\rangle$GM]}  
	 SPRICH: Bei mir gibt es Probleme: (3,1) ist von meinem O besetzt, und (2,3) ist von meinem T besetzt. Lass mich schauen... Freie Positionen bei mir: (1,2), (4,1), (1,4), (2,4), (3,4), (1,5), (2,5), (1,7), (2,7), (3,7), (5,7). Können wir eine neue Anordnung aus diesen wählen? 
	  } 
	   } 
	   } 
	 & & \\ 
 

    \theutterance \stepcounter{utterance}  

    & & & \multicolumn{2}{p{0.3\linewidth}}{\cellcolor[rgb]{0.95,0.95,0.95}{%
	\makecell[{{p{\linewidth}}}]{% 
	  \tt {\tiny [GM$|$GM]}  
	 SPRICH: Bei mir gibt es Probleme: (3,1) ist von meinem O besetzt, und (2,3) ist von meinem T besetzt. Lass mich schauen... Freie Positionen bei mir: (1,2), (4,1), (1,4), (2,4), (3,4), (1,5), (2,5), (1,7), (2,7), (3,7), (5,7). Können wir eine neue Anordnung aus diesen wählen? 
	  } 
	   } 
	   } 
	 & & \\ 
 

    \theutterance \stepcounter{utterance}  

    & & \multicolumn{4}{p{0.6\linewidth}}{\cellcolor[rgb]{0.9,0.9,0.9}{%
	\makecell[{{p{\linewidth}}}]{% 
	  \tt {\tiny [GM$\rangle$B]}  
	 Deine Nachricht wurde an deinen Mitspieler übermittelt.\\ \tt \\ \tt \\ \tt Ihr seid gerade bei Runde 5 von maximal 28.\\ \tt Ihr habt gemeinsam 1 von maximal 16 Strafpunkten angesammelt.\\ \tt Dein Mitspieler hat dir diese Nachricht geschickt:\\ \tt \\ \tt \\ \tt "Bei mir gibt es Probleme: (3,1) ist von meinem O besetzt, und (2,3) ist von meinem T besetzt. Lass mich schauen... Freie Positionen bei mir: (1,2), (4,1), (1,4), (2,4), (3,4), (1,5), (2,5), (1,7), (2,7), (3,7), (5,7). Können wir eine neue Anordnung aus diesen wählen?"\\ \tt \\ \tt \\ \tt Wie lautet dein nächster Befehl? 
	  } 
	   } 
	   } 
	 & \\ 
 

    \theutterance \stepcounter{utterance}  

    & & & \multicolumn{4}{p{0.6\linewidth}}{\cellcolor[rgb]{1,0.85,0.72}{%
	\makecell[{{p{\linewidth}}}]{% 
	  \tt {\tiny [GM$\langle$B]}  
	 SPRICH: Lass mich die gemeinsamen freien Positionen identifizieren: (4,1), (1,4), (2,4), (1,7), (2,7), (3,7). Das reicht für unsere verbleibenden Objekte! Mein Vorschlag: A(1,1) bleibt, M(3,1) bleibt bei mir, O(4,1), T(1,4), D(2,4), P(1,7), S(2,7). Sind alle diese Positionen bei dir frei? 
	  } 
	   } 
	   } 
	  \\ 
 

    \theutterance \stepcounter{utterance}  

    & & & \multicolumn{2}{p{0.3\linewidth}}{\cellcolor[rgb]{0.95,0.95,0.95}{%
	\makecell[{{p{\linewidth}}}]{% 
	  \tt {\tiny [GM$|$GM]}  
	 SPRICH: Lass mich die gemeinsamen freien Positionen identifizieren: (4,1), (1,4), (2,4), (1,7), (2,7), (3,7). Das reicht für unsere verbleibenden Objekte! Mein Vorschlag: A(1,1) bleibt, M(3,1) bleibt bei mir, O(4,1), T(1,4), D(2,4), P(1,7), S(2,7). Sind alle diese Positionen bei dir frei? 
	  } 
	   } 
	   } 
	 & & \\ 
 

    \theutterance \stepcounter{utterance}  

    & & \multicolumn{4}{p{0.6\linewidth}}{\cellcolor[rgb]{0.9,0.9,0.9}{%
	\makecell[{{p{\linewidth}}}]{% 
	  \tt {\tiny [A$\langle$GM]}  
	 Deine Nachricht wurde an deinen Mitspieler übermittelt.\\ \tt \\ \tt \\ \tt Ihr seid gerade bei Runde 5 von maximal 28.\\ \tt Ihr habt gemeinsam 1 von maximal 16 Strafpunkten angesammelt.\\ \tt Dein Mitspieler hat dir diese Nachricht geschickt:\\ \tt \\ \tt \\ \tt "Lass mich die gemeinsamen freien Positionen identifizieren: (4,1), (1,4), (2,4), (1,7), (2,7), (3,7). Das reicht für unsere verbleibenden Objekte! Mein Vorschlag: A(1,1) bleibt, M(3,1) bleibt bei mir, O(4,1), T(1,4), D(2,4), P(1,7), S(2,7). Sind alle diese Positionen bei dir frei?"\\ \tt \\ \tt \\ \tt Wie lautet dein nächster Befehl? 
	  } 
	   } 
	   } 
	 & \\ 
 

    \theutterance \stepcounter{utterance}  

    & \multicolumn{4}{p{0.6\linewidth}}{\cellcolor[rgb]{0.8,1,0.9}{%
	\makecell[{{p{\linewidth}}}]{% 
	  \tt {\tiny [A$\rangle$GM]}  
	 SPRICH: Moment - bei mir ist (3,1) von meinem O besetzt, nicht frei! Und wo soll mein M hin? Du sagst M(3,1) bleibt bei dir, aber ich habe mein M bei (6,7). Wir brauchen eine Position für beide M. Wie wäre: A(1,1), O(4,1), T(1,4), D(2,4), P(1,7), S(2,7), M(3,7)? 
	  } 
	   } 
	   } 
	 & & \\ 
 

    \theutterance \stepcounter{utterance}  

    & & & \multicolumn{2}{p{0.3\linewidth}}{\cellcolor[rgb]{0.95,0.95,0.95}{%
	\makecell[{{p{\linewidth}}}]{% 
	  \tt {\tiny [GM$|$GM]}  
	 SPRICH: Moment - bei mir ist (3,1) von meinem O besetzt, nicht frei! Und wo soll mein M hin? Du sagst M(3,1) bleibt bei dir, aber ich habe mein M bei (6,7). Wir brauchen eine Position für beide M. Wie wäre: A(1,1), O(4,1), T(1,4), D(2,4), P(1,7), S(2,7), M(3,7)? 
	  } 
	   } 
	   } 
	 & & \\ 
 

    \theutterance \stepcounter{utterance}  

    & & \multicolumn{4}{p{0.6\linewidth}}{\cellcolor[rgb]{0.9,0.9,0.9}{%
	\makecell[{{p{\linewidth}}}]{% 
	  \tt {\tiny [GM$\rangle$B]}  
	 Deine Nachricht wurde an deinen Mitspieler übermittelt.\\ \tt \\ \tt \\ \tt Ihr seid gerade bei Runde 6 von maximal 28.\\ \tt Ihr habt gemeinsam 1 von maximal 16 Strafpunkten angesammelt.\\ \tt Dein Mitspieler hat dir diese Nachricht geschickt:\\ \tt \\ \tt \\ \tt "Moment - bei mir ist (3,1) von meinem O besetzt, nicht frei! Und wo soll mein M hin? Du sagst M(3,1) bleibt bei dir, aber ich habe mein M bei (6,7). Wir brauchen eine Position für beide M. Wie wäre: A(1,1), O(4,1), T(1,4), D(2,4), P(1,7), S(2,7), M(3,7)?"\\ \tt \\ \tt \\ \tt Wie lautet dein nächster Befehl? 
	  } 
	   } 
	   } 
	 & \\ 
 

    \theutterance \stepcounter{utterance}  

    & & & \multicolumn{4}{p{0.6\linewidth}}{\cellcolor[rgb]{1,0.85,0.72}{%
	\makecell[{{p{\linewidth}}}]{% 
	  \tt {\tiny [GM$\langle$B]}  
	 SPRICH: Du hast recht, entschuldige die Verwirrung! Dein Vorschlag ist perfekt: A(1,1), O(4,1), T(1,4), D(2,4), P(1,7), S(2,7), M(3,7). Alle diese Positionen sind auf meinem Brett frei. Ich stimme zu! Ich werde jetzt mein M von (3,1) nach (3,7) verschieben. 
	  } 
	   } 
	   } 
	  \\ 
 

    \theutterance \stepcounter{utterance}  

    & & & \multicolumn{2}{p{0.3\linewidth}}{\cellcolor[rgb]{0.95,0.95,0.95}{%
	\makecell[{{p{\linewidth}}}]{% 
	  \tt {\tiny [GM$|$GM]}  
	 SPRICH: Du hast recht, entschuldige die Verwirrung! Dein Vorschlag ist perfekt: A(1,1), O(4,1), T(1,4), D(2,4), P(1,7), S(2,7), M(3,7). Alle diese Positionen sind auf meinem Brett frei. Ich stimme zu! Ich werde jetzt mein M von (3,1) nach (3,7) verschieben. 
	  } 
	   } 
	   } 
	 & & \\ 
 

    \theutterance \stepcounter{utterance}  

    & & \multicolumn{4}{p{0.6\linewidth}}{\cellcolor[rgb]{0.9,0.9,0.9}{%
	\makecell[{{p{\linewidth}}}]{% 
	  \tt {\tiny [A$\langle$GM]}  
	 Deine Nachricht wurde an deinen Mitspieler übermittelt.\\ \tt \\ \tt \\ \tt Ihr seid gerade bei Runde 6 von maximal 28.\\ \tt Ihr habt gemeinsam 1 von maximal 16 Strafpunkten angesammelt.\\ \tt Dein Mitspieler hat dir diese Nachricht geschickt:\\ \tt \\ \tt \\ \tt "Du hast recht, entschuldige die Verwirrung! Dein Vorschlag ist perfekt: A(1,1), O(4,1), T(1,4), D(2,4), P(1,7), S(2,7), M(3,7). Alle diese Positionen sind auf meinem Brett frei. Ich stimme zu! Ich werde jetzt mein M von (3,1) nach (3,7) verschieben."\\ \tt \\ \tt \\ \tt Wie lautet dein nächster Befehl? 
	  } 
	   } 
	   } 
	 & \\ 
 

    \theutterance \stepcounter{utterance}  

    & \multicolumn{4}{p{0.6\linewidth}}{\cellcolor[rgb]{0.8,1,0.9}{%
	\makecell[{{p{\linewidth}}}]{% 
	  \tt {\tiny [A$\rangle$GM]}  
	 VERSCHIEBE: O, (4, 1) 
	  } 
	   } 
	   } 
	 & & \\ 
 

    \theutterance \stepcounter{utterance}  

    & & & \multicolumn{2}{p{0.3\linewidth}}{\cellcolor[rgb]{0.95,0.95,0.95}{%
	\makecell[{{p{\linewidth}}}]{% 
	  \tt {\tiny [GM$|$GM]}  
	 VERSCHIEBE: O, (4, 1) 
	  } 
	   } 
	   } 
	 & & \\ 
 

    \theutterance \stepcounter{utterance}  

    & & & \multicolumn{2}{p{0.3\linewidth}}{\cellcolor[rgb]{0.95,0.95,0.95}{%
	\makecell[{{p{\linewidth}}}]{% 
	  \tt {\tiny [GM$|$GM]}  
	 Strafpunkt: (4,1) ist nicht leer, sondern enthält '│'.\\ \tt \\ \tt Ihr habt gemeinsam 2 von maximal 16 Strafpunkten angesammelt.\\ \tt Bitte versuche es noch mal! 
	  } 
	   } 
	   } 
	 & & \\ 
 

    \theutterance \stepcounter{utterance}  

    & & \multicolumn{4}{p{0.6\linewidth}}{\cellcolor[rgb]{0.9,0.9,0.9}{%
	\makecell[{{p{\linewidth}}}]{% 
	  \tt {\tiny [A$\langle$GM]}  
	 Strafpunkt: (4,1) ist nicht leer, sondern enthält '│'.\\ \tt \\ \tt Ihr habt gemeinsam 2 von maximal 16 Strafpunkten angesammelt.\\ \tt Bitte versuche es noch mal! 
	  } 
	   } 
	   } 
	 & \\ 
 

    \theutterance \stepcounter{utterance}  

    & \multicolumn{4}{p{0.6\linewidth}}{\cellcolor[rgb]{0.8,1,0.9}{%
	\makecell[{{p{\linewidth}}}]{% 
	  \tt {\tiny [A$\rangle$GM]}  
	 SPRICH: Entschuldigung, (4,1) ist auch blockiert! Ich muss das Brett genauer anschauen. Freie Positionen bei mir sind: (1,2), (3,2), (4,2), (1,4), (2,4), (3,5), (4,5), (5,5), (2,6), (3,6), (1,7), (2,7), (3,7), (5,7). Welche davon sind auch bei dir frei? 
	  } 
	   } 
	   } 
	 & & \\ 
 

    \theutterance \stepcounter{utterance}  

    & & & \multicolumn{2}{p{0.3\linewidth}}{\cellcolor[rgb]{0.95,0.95,0.95}{%
	\makecell[{{p{\linewidth}}}]{% 
	  \tt {\tiny [GM$|$GM]}  
	 SPRICH: Entschuldigung, (4,1) ist auch blockiert! Ich muss das Brett genauer anschauen. Freie Positionen bei mir sind: (1,2), (3,2), (4,2), (1,4), (2,4), (3,5), (4,5), (5,5), (2,6), (3,6), (1,7), (2,7), (3,7), (5,7). Welche davon sind auch bei dir frei? 
	  } 
	   } 
	   } 
	 & & \\ 
 

    \theutterance \stepcounter{utterance}  

    & & \multicolumn{4}{p{0.6\linewidth}}{\cellcolor[rgb]{0.9,0.9,0.9}{%
	\makecell[{{p{\linewidth}}}]{% 
	  \tt {\tiny [GM$\rangle$B]}  
	 Deine Nachricht wurde an deinen Mitspieler übermittelt.\\ \tt \\ \tt \\ \tt Ihr seid gerade bei Runde 7 von maximal 28.\\ \tt Ihr habt gemeinsam 2 von maximal 16 Strafpunkten angesammelt.\\ \tt Dein Mitspieler hat dir diese Nachricht geschickt:\\ \tt \\ \tt \\ \tt "Entschuldigung, (4,1) ist auch blockiert! Ich muss das Brett genauer anschauen. Freie Positionen bei mir sind: (1,2), (3,2), (4,2), (1,4), (2,4), (3,5), (4,5), (5,5), (2,6), (3,6), (1,7), (2,7), (3,7), (5,7). Welche davon sind auch bei dir frei?"\\ \tt \\ \tt \\ \tt Wie lautet dein nächster Befehl? 
	  } 
	   } 
	   } 
	 & \\ 
 

    \theutterance \stepcounter{utterance}  

    & & & \multicolumn{4}{p{0.6\linewidth}}{\cellcolor[rgb]{1,0.85,0.72}{%
	\makecell[{{p{\linewidth}}}]{% 
	  \tt {\tiny [GM$\langle$B]}  
	 SPRICH: Aus deiner Liste sind bei mir diese Positionen frei: (1,4), (3,5), (4,5), (5,5), (3,6), (1,7), (2,7), (3,7), (5,7). Das sind 9 freie Positionen für unsere 7 Objekte - mehr als genug! Welche Anordnung schlägst du aus diesen vor? A(1,1) bleibt ja schon richtig. 
	  } 
	   } 
	   } 
	  \\ 
 

    \theutterance \stepcounter{utterance}  

    & & & \multicolumn{2}{p{0.3\linewidth}}{\cellcolor[rgb]{0.95,0.95,0.95}{%
	\makecell[{{p{\linewidth}}}]{% 
	  \tt {\tiny [GM$|$GM]}  
	 SPRICH: Aus deiner Liste sind bei mir diese Positionen frei: (1,4), (3,5), (4,5), (5,5), (3,6), (1,7), (2,7), (3,7), (5,7). Das sind 9 freie Positionen für unsere 7 Objekte - mehr als genug! Welche Anordnung schlägst du aus diesen vor? A(1,1) bleibt ja schon richtig. 
	  } 
	   } 
	   } 
	 & & \\ 
 

    \theutterance \stepcounter{utterance}  

    & & \multicolumn{4}{p{0.6\linewidth}}{\cellcolor[rgb]{0.9,0.9,0.9}{%
	\makecell[{{p{\linewidth}}}]{% 
	  \tt {\tiny [A$\langle$GM]}  
	 Deine Nachricht wurde an deinen Mitspieler übermittelt.\\ \tt \\ \tt \\ \tt Ihr seid gerade bei Runde 7 von maximal 28.\\ \tt Ihr habt gemeinsam 2 von maximal 16 Strafpunkten angesammelt.\\ \tt Dein Mitspieler hat dir diese Nachricht geschickt:\\ \tt \\ \tt \\ \tt "Aus deiner Liste sind bei mir diese Positionen frei: (1,4), (3,5), (4,5), (5,5), (3,6), (1,7), (2,7), (3,7), (5,7). Das sind 9 freie Positionen für unsere 7 Objekte - mehr als genug! Welche Anordnung schlägst du aus diesen vor? A(1,1) bleibt ja schon richtig."\\ \tt \\ \tt \\ \tt Wie lautet dein nächster Befehl? 
	  } 
	   } 
	   } 
	 & \\ 
 

    \theutterance \stepcounter{utterance}  

    & \multicolumn{4}{p{0.6\linewidth}}{\cellcolor[rgb]{0.8,1,0.9}{%
	\makecell[{{p{\linewidth}}}]{% 
	  \tt {\tiny [A$\rangle$GM]}  
	 SPRICH: Perfekt! Aus den gemeinsamen freien Positionen schlage ich vor: A(1,1) bleibt, O(1,4), T(3,5), D(4,5), P(1,7), S(2,7), M(3,7). Das ergibt eine schöne kompakte Anordnung. Stimmst du zu? 
	  } 
	   } 
	   } 
	 & & \\ 
 

    \theutterance \stepcounter{utterance}  

    & & & \multicolumn{2}{p{0.3\linewidth}}{\cellcolor[rgb]{0.95,0.95,0.95}{%
	\makecell[{{p{\linewidth}}}]{% 
	  \tt {\tiny [GM$|$GM]}  
	 SPRICH: Perfekt! Aus den gemeinsamen freien Positionen schlage ich vor: A(1,1) bleibt, O(1,4), T(3,5), D(4,5), P(1,7), S(2,7), M(3,7). Das ergibt eine schöne kompakte Anordnung. Stimmst du zu? 
	  } 
	   } 
	   } 
	 & & \\ 
 

    \theutterance \stepcounter{utterance}  

    & & \multicolumn{4}{p{0.6\linewidth}}{\cellcolor[rgb]{0.9,0.9,0.9}{%
	\makecell[{{p{\linewidth}}}]{% 
	  \tt {\tiny [GM$\rangle$B]}  
	 Deine Nachricht wurde an deinen Mitspieler übermittelt.\\ \tt \\ \tt \\ \tt Ihr seid gerade bei Runde 8 von maximal 28.\\ \tt Ihr habt gemeinsam 2 von maximal 16 Strafpunkten angesammelt.\\ \tt Dein Mitspieler hat dir diese Nachricht geschickt:\\ \tt \\ \tt \\ \tt "Perfekt! Aus den gemeinsamen freien Positionen schlage ich vor: A(1,1) bleibt, O(1,4), T(3,5), D(4,5), P(1,7), S(2,7), M(3,7). Das ergibt eine schöne kompakte Anordnung. Stimmst du zu?"\\ \tt \\ \tt \\ \tt Wie lautet dein nächster Befehl? 
	  } 
	   } 
	   } 
	 & \\ 
 

    \theutterance \stepcounter{utterance}  

    & & & \multicolumn{4}{p{0.6\linewidth}}{\cellcolor[rgb]{1,0.85,0.72}{%
	\makecell[{{p{\linewidth}}}]{% 
	  \tt {\tiny [GM$\langle$B]}  
	 SPRICH: Perfekt! Ich stimme vollständig zu: A(1,1), O(1,4), T(3,5), D(4,5), P(1,7), S(2,7), M(3,7). Lass uns anfangen! Ich verschiebe zuerst mein M von (3,1) nach (3,7). 
	  } 
	   } 
	   } 
	  \\ 
 

    \theutterance \stepcounter{utterance}  

    & & & \multicolumn{2}{p{0.3\linewidth}}{\cellcolor[rgb]{0.95,0.95,0.95}{%
	\makecell[{{p{\linewidth}}}]{% 
	  \tt {\tiny [GM$|$GM]}  
	 SPRICH: Perfekt! Ich stimme vollständig zu: A(1,1), O(1,4), T(3,5), D(4,5), P(1,7), S(2,7), M(3,7). Lass uns anfangen! Ich verschiebe zuerst mein M von (3,1) nach (3,7). 
	  } 
	   } 
	   } 
	 & & \\ 
 

    \theutterance \stepcounter{utterance}  

    & & \multicolumn{4}{p{0.6\linewidth}}{\cellcolor[rgb]{0.9,0.9,0.9}{%
	\makecell[{{p{\linewidth}}}]{% 
	  \tt {\tiny [A$\langle$GM]}  
	 Deine Nachricht wurde an deinen Mitspieler übermittelt.\\ \tt \\ \tt \\ \tt Ihr seid gerade bei Runde 8 von maximal 28.\\ \tt Ihr habt gemeinsam 2 von maximal 16 Strafpunkten angesammelt.\\ \tt Dein Mitspieler hat dir diese Nachricht geschickt:\\ \tt \\ \tt \\ \tt "Perfekt! Ich stimme vollständig zu: A(1,1), O(1,4), T(3,5), D(4,5), P(1,7), S(2,7), M(3,7). Lass uns anfangen! Ich verschiebe zuerst mein M von (3,1) nach (3,7)."\\ \tt \\ \tt \\ \tt Wie lautet dein nächster Befehl? 
	  } 
	   } 
	   } 
	 & \\ 
 

    \theutterance \stepcounter{utterance}  

    & \multicolumn{4}{p{0.6\linewidth}}{\cellcolor[rgb]{0.8,1,0.9}{%
	\makecell[{{p{\linewidth}}}]{% 
	  \tt {\tiny [A$\rangle$GM]}  
	 VERSCHIEBE: O, (1, 4) 
	  } 
	   } 
	   } 
	 & & \\ 
 

    \theutterance \stepcounter{utterance}  

    & & & \multicolumn{2}{p{0.3\linewidth}}{\cellcolor[rgb]{0.95,0.95,0.95}{%
	\makecell[{{p{\linewidth}}}]{% 
	  \tt {\tiny [GM$|$GM]}  
	 VERSCHIEBE: O, (1, 4) 
	  } 
	   } 
	   } 
	 & & \\ 
 

    \theutterance \stepcounter{utterance}  

    & & & \multicolumn{2}{p{0.3\linewidth}}{\cellcolor[rgb]{0.95,0.95,0.95}{%
	\makecell[{{p{\linewidth}}}]{% 
	  \tt {\tiny [GM$|$GM]}  
	 'O' erfolgreich nach (1,4) verschoben. Dein Raster sieht jetzt so aus:\\ \tt ```\\ \tt  1234567\\ \tt ╔═╤═╤═╤═╗\\ \tt ║A│◌│◌│◌║ 1\\ \tt ╟─┼─┤◌│◌║ 2\\ \tt ║◌│T│◌│P║ 3\\ \tt ║O│◌├─┴─╢ 4\\ \tt ╟─┴─┤◌◌◌║ 5\\ \tt ║D◌◌└─┬─╢ 6\\ \tt ║◌◌◌S◌│M║ 7\\ \tt ╚═════╧═╝\\ \tt ```\\ \tt  
	  } 
	   } 
	   } 
	 & & \\ 
 

    \theutterance \stepcounter{utterance}  

    & & \multicolumn{4}{p{0.6\linewidth}}{\cellcolor[rgb]{0.9,0.9,0.9}{%
	\makecell[{{p{\linewidth}}}]{% 
	  \tt {\tiny [GM$\rangle$B]}  
	 Deine Nachricht wurde an deinen Mitspieler übermittelt.\\ \tt \\ \tt \\ \tt Ihr seid gerade bei Runde 9 von maximal 28.\\ \tt Ihr habt gemeinsam 2 von maximal 16 Strafpunkten angesammelt.\\ \tt Dein Mitspieler hat ein Objekt auf seinem Raster bewegt.\\ \tt \\ \tt \\ \tt Wie lautet dein nächster Befehl? 
	  } 
	   } 
	   } 
	 & \\ 
 

    \theutterance \stepcounter{utterance}  

    & & & \multicolumn{4}{p{0.6\linewidth}}{\cellcolor[rgb]{1,0.85,0.72}{%
	\makecell[{{p{\linewidth}}}]{% 
	  \tt {\tiny [GM$\langle$B]}  
	 VERSCHIEBE: M, (3, 7) 
	  } 
	   } 
	   } 
	  \\ 
 

    \theutterance \stepcounter{utterance}  

    & & & \multicolumn{2}{p{0.3\linewidth}}{\cellcolor[rgb]{0.95,0.95,0.95}{%
	\makecell[{{p{\linewidth}}}]{% 
	  \tt {\tiny [GM$|$GM]}  
	 VERSCHIEBE: M, (3, 7) 
	  } 
	   } 
	   } 
	 & & \\ 
 

    \theutterance \stepcounter{utterance}  

    & & & \multicolumn{2}{p{0.3\linewidth}}{\cellcolor[rgb]{0.95,0.95,0.95}{%
	\makecell[{{p{\linewidth}}}]{% 
	  \tt {\tiny [GM$|$GM]}  
	 'M' erfolgreich nach (3,7) verschoben. Dein Raster sieht jetzt so aus:\\ \tt ```\\ \tt  1234567\\ \tt ╔═╤═╤═╤═╗\\ \tt ║A│◌│◌│◌║ 1\\ \tt ╟─┼─┤O│T║ 2\\ \tt ║◌│D│◌│◌║ 3\\ \tt ║◌│◌├─┴─╢ 4\\ \tt ╟─┴─┤◌◌◌║ 5\\ \tt ║PS◌└─┬─╢ 6\\ \tt ║◌◌M◌◌│◌║ 7\\ \tt ╚═════╧═╝\\ \tt ```\\ \tt  
	  } 
	   } 
	   } 
	 & & \\ 
 

    \theutterance \stepcounter{utterance}  

    & & \multicolumn{4}{p{0.6\linewidth}}{\cellcolor[rgb]{0.9,0.9,0.9}{%
	\makecell[{{p{\linewidth}}}]{% 
	  \tt {\tiny [A$\langle$GM]}  
	 'O' erfolgreich nach (1,4) verschoben. Dein Raster sieht jetzt so aus:\\ \tt ```\\ \tt  1234567\\ \tt ╔═╤═╤═╤═╗\\ \tt ║A│◌│◌│◌║ 1\\ \tt ╟─┼─┤◌│◌║ 2\\ \tt ║◌│T│◌│P║ 3\\ \tt ║O│◌├─┴─╢ 4\\ \tt ╟─┴─┤◌◌◌║ 5\\ \tt ║D◌◌└─┬─╢ 6\\ \tt ║◌◌◌S◌│M║ 7\\ \tt ╚═════╧═╝\\ \tt ```\\ \tt Ihr seid gerade bei Runde 9 von maximal 28.\\ \tt Ihr habt gemeinsam 2 von maximal 16 Strafpunkten angesammelt.\\ \tt Dein Mitspieler hat ein Objekt auf seinem Raster bewegt.\\ \tt \\ \tt \\ \tt Wie lautet dein nächster Befehl? 
	  } 
	   } 
	   } 
	 & \\ 
 

    \theutterance \stepcounter{utterance}  

    & \multicolumn{4}{p{0.6\linewidth}}{\cellcolor[rgb]{0.8,1,0.9}{%
	\makecell[{{p{\linewidth}}}]{% 
	  \tt {\tiny [A$\rangle$GM]}  
	 VERSCHIEBE: T, (3, 5) 
	  } 
	   } 
	   } 
	 & & \\ 
 

    \theutterance \stepcounter{utterance}  

    & & & \multicolumn{2}{p{0.3\linewidth}}{\cellcolor[rgb]{0.95,0.95,0.95}{%
	\makecell[{{p{\linewidth}}}]{% 
	  \tt {\tiny [GM$|$GM]}  
	 VERSCHIEBE: T, (3, 5) 
	  } 
	   } 
	   } 
	 & & \\ 
 

    \theutterance \stepcounter{utterance}  

    & & & \multicolumn{2}{p{0.3\linewidth}}{\cellcolor[rgb]{0.95,0.95,0.95}{%
	\makecell[{{p{\linewidth}}}]{% 
	  \tt {\tiny [GM$|$GM]}  
	 Strafpunkt: (3,5) ist nicht leer, sondern enthält '─'.\\ \tt \\ \tt Ihr habt gemeinsam 3 von maximal 16 Strafpunkten angesammelt.\\ \tt Bitte versuche es noch mal! 
	  } 
	   } 
	   } 
	 & & \\ 
 

    \theutterance \stepcounter{utterance}  

    & & \multicolumn{4}{p{0.6\linewidth}}{\cellcolor[rgb]{0.9,0.9,0.9}{%
	\makecell[{{p{\linewidth}}}]{% 
	  \tt {\tiny [A$\langle$GM]}  
	 Strafpunkt: (3,5) ist nicht leer, sondern enthält '─'.\\ \tt \\ \tt Ihr habt gemeinsam 3 von maximal 16 Strafpunkten angesammelt.\\ \tt Bitte versuche es noch mal! 
	  } 
	   } 
	   } 
	 & \\ 
 

    \theutterance \stepcounter{utterance}  

    & \multicolumn{4}{p{0.6\linewidth}}{\cellcolor[rgb]{0.8,1,0.9}{%
	\makecell[{{p{\linewidth}}}]{% 
	  \tt {\tiny [A$\rangle$GM]}  
	 SPRICH: Problem! (3,5) ist blockiert durch eine Linie. Wir müssen eine neue Position für T finden. Aus unseren gemeinsamen freien Positionen: (5,5), (3,6), (5,7) sind noch verfügbar. Welche soll T nehmen? Ich schlage T(5,5) vor. 
	  } 
	   } 
	   } 
	 & & \\ 
 

    \theutterance \stepcounter{utterance}  

    & & & \multicolumn{2}{p{0.3\linewidth}}{\cellcolor[rgb]{0.95,0.95,0.95}{%
	\makecell[{{p{\linewidth}}}]{% 
	  \tt {\tiny [GM$|$GM]}  
	 SPRICH: Problem! (3,5) ist blockiert durch eine Linie. Wir müssen eine neue Position für T finden. Aus unseren gemeinsamen freien Positionen: (5,5), (3,6), (5,7) sind noch verfügbar. Welche soll T nehmen? Ich schlage T(5,5) vor. 
	  } 
	   } 
	   } 
	 & & \\ 
 

    \theutterance \stepcounter{utterance}  

    & & \multicolumn{4}{p{0.6\linewidth}}{\cellcolor[rgb]{0.9,0.9,0.9}{%
	\makecell[{{p{\linewidth}}}]{% 
	  \tt {\tiny [GM$\rangle$B]}  
	 'M' erfolgreich nach (3,7) verschoben. Dein Raster sieht jetzt so aus:\\ \tt ```\\ \tt  1234567\\ \tt ╔═╤═╤═╤═╗\\ \tt ║A│◌│◌│◌║ 1\\ \tt ╟─┼─┤O│T║ 2\\ \tt ║◌│D│◌│◌║ 3\\ \tt ║◌│◌├─┴─╢ 4\\ \tt ╟─┴─┤◌◌◌║ 5\\ \tt ║PS◌└─┬─╢ 6\\ \tt ║◌◌M◌◌│◌║ 7\\ \tt ╚═════╧═╝\\ \tt ```\\ \tt Ihr seid gerade bei Runde 10 von maximal 28.\\ \tt Ihr habt gemeinsam 3 von maximal 16 Strafpunkten angesammelt.\\ \tt Dein Mitspieler hat dir diese Nachricht geschickt:\\ \tt \\ \tt \\ \tt "Problem! (3,5) ist blockiert durch eine Linie. Wir müssen eine neue Position für T finden. Aus unseren gemeinsamen freien Positionen: (5,5), (3,6), (5,7) sind noch verfügbar. Welche soll T nehmen? Ich schlage T(5,5) vor."\\ \tt \\ \tt \\ \tt Wie lautet dein nächster Befehl? 
	  } 
	   } 
	   } 
	 & \\ 
 

    \theutterance \stepcounter{utterance}  

    & & & \multicolumn{4}{p{0.6\linewidth}}{\cellcolor[rgb]{1,0.85,0.72}{%
	\makecell[{{p{\linewidth}}}]{% 
	  \tt {\tiny [GM$\langle$B]}  
	 SPRICH: T(5,5) ist bei mir frei, das funktioniert! Unsere finale Anordnung ist dann: A(1,1), O(1,4), T(5,5), D(4,5), P(1,7), S(2,7), M(3,7). Ich verschiebe als nächstes mein O von (3,2) nach (1,4). 
	  } 
	   } 
	   } 
	  \\ 
 

    \theutterance \stepcounter{utterance}  

    & & & \multicolumn{2}{p{0.3\linewidth}}{\cellcolor[rgb]{0.95,0.95,0.95}{%
	\makecell[{{p{\linewidth}}}]{% 
	  \tt {\tiny [GM$|$GM]}  
	 SPRICH: T(5,5) ist bei mir frei, das funktioniert! Unsere finale Anordnung ist dann: A(1,1), O(1,4), T(5,5), D(4,5), P(1,7), S(2,7), M(3,7). Ich verschiebe als nächstes mein O von (3,2) nach (1,4). 
	  } 
	   } 
	   } 
	 & & \\ 
 

    \theutterance \stepcounter{utterance}  

    & & \multicolumn{4}{p{0.6\linewidth}}{\cellcolor[rgb]{0.9,0.9,0.9}{%
	\makecell[{{p{\linewidth}}}]{% 
	  \tt {\tiny [A$\langle$GM]}  
	 Deine Nachricht wurde an deinen Mitspieler übermittelt.\\ \tt \\ \tt \\ \tt Ihr seid gerade bei Runde 10 von maximal 28.\\ \tt Ihr habt gemeinsam 3 von maximal 16 Strafpunkten angesammelt.\\ \tt Dein Mitspieler hat dir diese Nachricht geschickt:\\ \tt \\ \tt \\ \tt "T(5,5) ist bei mir frei, das funktioniert! Unsere finale Anordnung ist dann: A(1,1), O(1,4), T(5,5), D(4,5), P(1,7), S(2,7), M(3,7). Ich verschiebe als nächstes mein O von (3,2) nach (1,4)."\\ \tt \\ \tt \\ \tt Wie lautet dein nächster Befehl? 
	  } 
	   } 
	   } 
	 & \\ 
 

    \theutterance \stepcounter{utterance}  

    & \multicolumn{4}{p{0.6\linewidth}}{\cellcolor[rgb]{0.8,1,0.9}{%
	\makecell[{{p{\linewidth}}}]{% 
	  \tt {\tiny [A$\rangle$GM]}  
	 VERSCHIEBE: T, (5, 5) 
	  } 
	   } 
	   } 
	 & & \\ 
 

    \theutterance \stepcounter{utterance}  

    & & & \multicolumn{2}{p{0.3\linewidth}}{\cellcolor[rgb]{0.95,0.95,0.95}{%
	\makecell[{{p{\linewidth}}}]{% 
	  \tt {\tiny [GM$|$GM]}  
	 VERSCHIEBE: T, (5, 5) 
	  } 
	   } 
	   } 
	 & & \\ 
 

    \theutterance \stepcounter{utterance}  

    & & & \multicolumn{2}{p{0.3\linewidth}}{\cellcolor[rgb]{0.95,0.95,0.95}{%
	\makecell[{{p{\linewidth}}}]{% 
	  \tt {\tiny [GM$|$GM]}  
	 'T' erfolgreich nach (5,5) verschoben. Dein Raster sieht jetzt so aus:\\ \tt ```\\ \tt  1234567\\ \tt ╔═╤═╤═╤═╗\\ \tt ║A│◌│◌│◌║ 1\\ \tt ╟─┼─┤◌│◌║ 2\\ \tt ║◌│◌│◌│P║ 3\\ \tt ║O│◌├─┴─╢ 4\\ \tt ╟─┴─┤T◌◌║ 5\\ \tt ║D◌◌└─┬─╢ 6\\ \tt ║◌◌◌S◌│M║ 7\\ \tt ╚═════╧═╝\\ \tt ```\\ \tt  
	  } 
	   } 
	   } 
	 & & \\ 
 

    \theutterance \stepcounter{utterance}  

    & & \multicolumn{4}{p{0.6\linewidth}}{\cellcolor[rgb]{0.9,0.9,0.9}{%
	\makecell[{{p{\linewidth}}}]{% 
	  \tt {\tiny [GM$\rangle$B]}  
	 Deine Nachricht wurde an deinen Mitspieler übermittelt.\\ \tt \\ \tt \\ \tt Ihr seid gerade bei Runde 11 von maximal 28.\\ \tt Ihr habt gemeinsam 3 von maximal 16 Strafpunkten angesammelt.\\ \tt Dein Mitspieler hat ein Objekt auf seinem Raster bewegt.\\ \tt \\ \tt \\ \tt Wie lautet dein nächster Befehl? 
	  } 
	   } 
	   } 
	 & \\ 
 

    \theutterance \stepcounter{utterance}  

    & & & \multicolumn{4}{p{0.6\linewidth}}{\cellcolor[rgb]{1,0.85,0.72}{%
	\makecell[{{p{\linewidth}}}]{% 
	  \tt {\tiny [GM$\langle$B]}  
	 VERSCHIEBE: O, (1, 4) 
	  } 
	   } 
	   } 
	  \\ 
 

    \theutterance \stepcounter{utterance}  

    & & & \multicolumn{2}{p{0.3\linewidth}}{\cellcolor[rgb]{0.95,0.95,0.95}{%
	\makecell[{{p{\linewidth}}}]{% 
	  \tt {\tiny [GM$|$GM]}  
	 VERSCHIEBE: O, (1, 4) 
	  } 
	   } 
	   } 
	 & & \\ 
 

    \theutterance \stepcounter{utterance}  

    & & & \multicolumn{2}{p{0.3\linewidth}}{\cellcolor[rgb]{0.95,0.95,0.95}{%
	\makecell[{{p{\linewidth}}}]{% 
	  \tt {\tiny [GM$|$GM]}  
	 'O' erfolgreich nach (1,4) verschoben. Dein Raster sieht jetzt so aus:\\ \tt ```\\ \tt  1234567\\ \tt ╔═╤═╤═╤═╗\\ \tt ║A│◌│◌│◌║ 1\\ \tt ╟─┼─┤◌│T║ 2\\ \tt ║◌│D│◌│◌║ 3\\ \tt ║O│◌├─┴─╢ 4\\ \tt ╟─┴─┤◌◌◌║ 5\\ \tt ║PS◌└─┬─╢ 6\\ \tt ║◌◌M◌◌│◌║ 7\\ \tt ╚═════╧═╝\\ \tt ```\\ \tt  
	  } 
	   } 
	   } 
	 & & \\ 
 

    \theutterance \stepcounter{utterance}  

    & & \multicolumn{4}{p{0.6\linewidth}}{\cellcolor[rgb]{0.9,0.9,0.9}{%
	\makecell[{{p{\linewidth}}}]{% 
	  \tt {\tiny [A$\langle$GM]}  
	 'T' erfolgreich nach (5,5) verschoben. Dein Raster sieht jetzt so aus:\\ \tt ```\\ \tt  1234567\\ \tt ╔═╤═╤═╤═╗\\ \tt ║A│◌│◌│◌║ 1\\ \tt ╟─┼─┤◌│◌║ 2\\ \tt ║◌│◌│◌│P║ 3\\ \tt ║O│◌├─┴─╢ 4\\ \tt ╟─┴─┤T◌◌║ 5\\ \tt ║D◌◌└─┬─╢ 6\\ \tt ║◌◌◌S◌│M║ 7\\ \tt ╚═════╧═╝\\ \tt ```\\ \tt Ihr seid gerade bei Runde 11 von maximal 28.\\ \tt Ihr habt gemeinsam 3 von maximal 16 Strafpunkten angesammelt.\\ \tt Dein Mitspieler hat ein Objekt auf seinem Raster bewegt.\\ \tt \\ \tt \\ \tt Wie lautet dein nächster Befehl? 
	  } 
	   } 
	   } 
	 & \\ 
 

    \theutterance \stepcounter{utterance}  

    & \multicolumn{4}{p{0.6\linewidth}}{\cellcolor[rgb]{0.8,1,0.9}{%
	\makecell[{{p{\linewidth}}}]{% 
	  \tt {\tiny [A$\rangle$GM]}  
	 VERSCHIEBE: D, (4, 5) 
	  } 
	   } 
	   } 
	 & & \\ 
 

    \theutterance \stepcounter{utterance}  

    & & & \multicolumn{2}{p{0.3\linewidth}}{\cellcolor[rgb]{0.95,0.95,0.95}{%
	\makecell[{{p{\linewidth}}}]{% 
	  \tt {\tiny [GM$|$GM]}  
	 VERSCHIEBE: D, (4, 5) 
	  } 
	   } 
	   } 
	 & & \\ 
 

    \theutterance \stepcounter{utterance}  

    & & & \multicolumn{2}{p{0.3\linewidth}}{\cellcolor[rgb]{0.95,0.95,0.95}{%
	\makecell[{{p{\linewidth}}}]{% 
	  \tt {\tiny [GM$|$GM]}  
	 Strafpunkt: (4,5) ist nicht leer, sondern enthält '┤'.\\ \tt \\ \tt Ihr habt gemeinsam 4 von maximal 16 Strafpunkten angesammelt.\\ \tt Bitte versuche es noch mal! 
	  } 
	   } 
	   } 
	 & & \\ 
 

    \theutterance \stepcounter{utterance}  

    & & \multicolumn{4}{p{0.6\linewidth}}{\cellcolor[rgb]{0.9,0.9,0.9}{%
	\makecell[{{p{\linewidth}}}]{% 
	  \tt {\tiny [A$\langle$GM]}  
	 Strafpunkt: (4,5) ist nicht leer, sondern enthält '┤'.\\ \tt \\ \tt Ihr habt gemeinsam 4 von maximal 16 Strafpunkten angesammelt.\\ \tt Bitte versuche es noch mal! 
	  } 
	   } 
	   } 
	 & \\ 
 

    \theutterance \stepcounter{utterance}  

    & \multicolumn{4}{p{0.6\linewidth}}{\cellcolor[rgb]{0.8,1,0.9}{%
	\makecell[{{p{\linewidth}}}]{% 
	  \tt {\tiny [A$\rangle$GM]}  
	 SPRICH: (4,5) ist auch blockiert! D braucht eine neue Position. Von unseren vereinbarten freien Positionen sind noch verfügbar: (3,6), (5,7). Soll D nach (3,6)? 
	  } 
	   } 
	   } 
	 & & \\ 
 

    \theutterance \stepcounter{utterance}  

    & & & \multicolumn{2}{p{0.3\linewidth}}{\cellcolor[rgb]{0.95,0.95,0.95}{%
	\makecell[{{p{\linewidth}}}]{% 
	  \tt {\tiny [GM$|$GM]}  
	 SPRICH: (4,5) ist auch blockiert! D braucht eine neue Position. Von unseren vereinbarten freien Positionen sind noch verfügbar: (3,6), (5,7). Soll D nach (3,6)? 
	  } 
	   } 
	   } 
	 & & \\ 
 

    \theutterance \stepcounter{utterance}  

    & & \multicolumn{4}{p{0.6\linewidth}}{\cellcolor[rgb]{0.9,0.9,0.9}{%
	\makecell[{{p{\linewidth}}}]{% 
	  \tt {\tiny [GM$\rangle$B]}  
	 'O' erfolgreich nach (1,4) verschoben. Dein Raster sieht jetzt so aus:\\ \tt ```\\ \tt  1234567\\ \tt ╔═╤═╤═╤═╗\\ \tt ║A│◌│◌│◌║ 1\\ \tt ╟─┼─┤◌│T║ 2\\ \tt ║◌│D│◌│◌║ 3\\ \tt ║O│◌├─┴─╢ 4\\ \tt ╟─┴─┤◌◌◌║ 5\\ \tt ║PS◌└─┬─╢ 6\\ \tt ║◌◌M◌◌│◌║ 7\\ \tt ╚═════╧═╝\\ \tt ```\\ \tt Ihr seid gerade bei Runde 12 von maximal 28.\\ \tt Ihr habt gemeinsam 4 von maximal 16 Strafpunkten angesammelt.\\ \tt Dein Mitspieler hat dir diese Nachricht geschickt:\\ \tt \\ \tt \\ \tt "(4,5) ist auch blockiert! D braucht eine neue Position. Von unseren vereinbarten freien Positionen sind noch verfügbar: (3,6), (5,7). Soll D nach (3,6)?"\\ \tt \\ \tt \\ \tt Wie lautet dein nächster Befehl? 
	  } 
	   } 
	   } 
	 & \\ 
 

    \theutterance \stepcounter{utterance}  

    & & & \multicolumn{4}{p{0.6\linewidth}}{\cellcolor[rgb]{1,0.85,0.72}{%
	\makecell[{{p{\linewidth}}}]{% 
	  \tt {\tiny [GM$\langle$B]}  
	 SPRICH: Ja, D(3,6) ist bei mir frei! Unsere finale Anordnung ist dann: A(1,1), O(1,4), T(5,5), D(3,6), P(1,7), S(2,7), M(3,7). Ich verschiebe als nächstes mein T von (4,2) nach (5,5). 
	  } 
	   } 
	   } 
	  \\ 
 

    \theutterance \stepcounter{utterance}  

    & & & \multicolumn{2}{p{0.3\linewidth}}{\cellcolor[rgb]{0.95,0.95,0.95}{%
	\makecell[{{p{\linewidth}}}]{% 
	  \tt {\tiny [GM$|$GM]}  
	 SPRICH: Ja, D(3,6) ist bei mir frei! Unsere finale Anordnung ist dann: A(1,1), O(1,4), T(5,5), D(3,6), P(1,7), S(2,7), M(3,7). Ich verschiebe als nächstes mein T von (4,2) nach (5,5). 
	  } 
	   } 
	   } 
	 & & \\ 
 

    \theutterance \stepcounter{utterance}  

    & & \multicolumn{4}{p{0.6\linewidth}}{\cellcolor[rgb]{0.9,0.9,0.9}{%
	\makecell[{{p{\linewidth}}}]{% 
	  \tt {\tiny [A$\langle$GM]}  
	 Deine Nachricht wurde an deinen Mitspieler übermittelt.\\ \tt \\ \tt \\ \tt Ihr seid gerade bei Runde 12 von maximal 28.\\ \tt Ihr habt gemeinsam 4 von maximal 16 Strafpunkten angesammelt.\\ \tt Dein Mitspieler hat dir diese Nachricht geschickt:\\ \tt \\ \tt \\ \tt "Ja, D(3,6) ist bei mir frei! Unsere finale Anordnung ist dann: A(1,1), O(1,4), T(5,5), D(3,6), P(1,7), S(2,7), M(3,7). Ich verschiebe als nächstes mein T von (4,2) nach (5,5)."\\ \tt \\ \tt \\ \tt Wie lautet dein nächster Befehl? 
	  } 
	   } 
	   } 
	 & \\ 
 

    \theutterance \stepcounter{utterance}  

    & \multicolumn{4}{p{0.6\linewidth}}{\cellcolor[rgb]{0.8,1,0.9}{%
	\makecell[{{p{\linewidth}}}]{% 
	  \tt {\tiny [A$\rangle$GM]}  
	 VERSCHIEBE: D, (3, 6) 
	  } 
	   } 
	   } 
	 & & \\ 
 

    \theutterance \stepcounter{utterance}  

    & & & \multicolumn{2}{p{0.3\linewidth}}{\cellcolor[rgb]{0.95,0.95,0.95}{%
	\makecell[{{p{\linewidth}}}]{% 
	  \tt {\tiny [GM$|$GM]}  
	 VERSCHIEBE: D, (3, 6) 
	  } 
	   } 
	   } 
	 & & \\ 
 

    \theutterance \stepcounter{utterance}  

    & & & \multicolumn{2}{p{0.3\linewidth}}{\cellcolor[rgb]{0.95,0.95,0.95}{%
	\makecell[{{p{\linewidth}}}]{% 
	  \tt {\tiny [GM$|$GM]}  
	 'D' erfolgreich nach (3,6) verschoben. Dein Raster sieht jetzt so aus:\\ \tt ```\\ \tt  1234567\\ \tt ╔═╤═╤═╤═╗\\ \tt ║A│◌│◌│◌║ 1\\ \tt ╟─┼─┤◌│◌║ 2\\ \tt ║◌│◌│◌│P║ 3\\ \tt ║O│◌├─┴─╢ 4\\ \tt ╟─┴─┤T◌◌║ 5\\ \tt ║◌◌D└─┬─╢ 6\\ \tt ║◌◌◌S◌│M║ 7\\ \tt ╚═════╧═╝\\ \tt ```\\ \tt  
	  } 
	   } 
	   } 
	 & & \\ 
 

    \theutterance \stepcounter{utterance}  

    & & \multicolumn{4}{p{0.6\linewidth}}{\cellcolor[rgb]{0.9,0.9,0.9}{%
	\makecell[{{p{\linewidth}}}]{% 
	  \tt {\tiny [GM$\rangle$B]}  
	 Deine Nachricht wurde an deinen Mitspieler übermittelt.\\ \tt \\ \tt \\ \tt Ihr seid gerade bei Runde 13 von maximal 28.\\ \tt Ihr habt gemeinsam 4 von maximal 16 Strafpunkten angesammelt.\\ \tt Dein Mitspieler hat ein Objekt auf seinem Raster bewegt.\\ \tt \\ \tt \\ \tt Wie lautet dein nächster Befehl? 
	  } 
	   } 
	   } 
	 & \\ 
 

    \theutterance \stepcounter{utterance}  

    & & & \multicolumn{4}{p{0.6\linewidth}}{\cellcolor[rgb]{1,0.85,0.72}{%
	\makecell[{{p{\linewidth}}}]{% 
	  \tt {\tiny [GM$\langle$B]}  
	 VERSCHIEBE: T, (5, 5) 
	  } 
	   } 
	   } 
	  \\ 
 

    \theutterance \stepcounter{utterance}  

    & & & \multicolumn{2}{p{0.3\linewidth}}{\cellcolor[rgb]{0.95,0.95,0.95}{%
	\makecell[{{p{\linewidth}}}]{% 
	  \tt {\tiny [GM$|$GM]}  
	 VERSCHIEBE: T, (5, 5) 
	  } 
	   } 
	   } 
	 & & \\ 
 

    \theutterance \stepcounter{utterance}  

    & & & \multicolumn{2}{p{0.3\linewidth}}{\cellcolor[rgb]{0.95,0.95,0.95}{%
	\makecell[{{p{\linewidth}}}]{% 
	  \tt {\tiny [GM$|$GM]}  
	 'T' erfolgreich nach (5,5) verschoben. Dein Raster sieht jetzt so aus:\\ \tt ```\\ \tt  1234567\\ \tt ╔═╤═╤═╤═╗\\ \tt ║A│◌│◌│◌║ 1\\ \tt ╟─┼─┤◌│◌║ 2\\ \tt ║◌│D│◌│◌║ 3\\ \tt ║O│◌├─┴─╢ 4\\ \tt ╟─┴─┤T◌◌║ 5\\ \tt ║PS◌└─┬─╢ 6\\ \tt ║◌◌M◌◌│◌║ 7\\ \tt ╚═════╧═╝\\ \tt ```\\ \tt  
	  } 
	   } 
	   } 
	 & & \\ 
 

    \theutterance \stepcounter{utterance}  

    & & \multicolumn{4}{p{0.6\linewidth}}{\cellcolor[rgb]{0.9,0.9,0.9}{%
	\makecell[{{p{\linewidth}}}]{% 
	  \tt {\tiny [A$\langle$GM]}  
	 'D' erfolgreich nach (3,6) verschoben. Dein Raster sieht jetzt so aus:\\ \tt ```\\ \tt  1234567\\ \tt ╔═╤═╤═╤═╗\\ \tt ║A│◌│◌│◌║ 1\\ \tt ╟─┼─┤◌│◌║ 2\\ \tt ║◌│◌│◌│P║ 3\\ \tt ║O│◌├─┴─╢ 4\\ \tt ╟─┴─┤T◌◌║ 5\\ \tt ║◌◌D└─┬─╢ 6\\ \tt ║◌◌◌S◌│M║ 7\\ \tt ╚═════╧═╝\\ \tt ```\\ \tt Ihr seid gerade bei Runde 13 von maximal 28.\\ \tt Ihr habt gemeinsam 4 von maximal 16 Strafpunkten angesammelt.\\ \tt Dein Mitspieler hat ein Objekt auf seinem Raster bewegt.\\ \tt \\ \tt \\ \tt Wie lautet dein nächster Befehl? 
	  } 
	   } 
	   } 
	 & \\ 
 

    \theutterance \stepcounter{utterance}  

    & \multicolumn{4}{p{0.6\linewidth}}{\cellcolor[rgb]{0.8,1,0.9}{%
	\makecell[{{p{\linewidth}}}]{% 
	  \tt {\tiny [A$\rangle$GM]}  
	 VERSCHIEBE: P, (1, 7) 
	  } 
	   } 
	   } 
	 & & \\ 
 

    \theutterance \stepcounter{utterance}  

    & & & \multicolumn{2}{p{0.3\linewidth}}{\cellcolor[rgb]{0.95,0.95,0.95}{%
	\makecell[{{p{\linewidth}}}]{% 
	  \tt {\tiny [GM$|$GM]}  
	 VERSCHIEBE: P, (1, 7) 
	  } 
	   } 
	   } 
	 & & \\ 
 

    \theutterance \stepcounter{utterance}  

    & & & \multicolumn{2}{p{0.3\linewidth}}{\cellcolor[rgb]{0.95,0.95,0.95}{%
	\makecell[{{p{\linewidth}}}]{% 
	  \tt {\tiny [GM$|$GM]}  
	 'P' erfolgreich nach (1,7) verschoben. Dein Raster sieht jetzt so aus:\\ \tt ```\\ \tt  1234567\\ \tt ╔═╤═╤═╤═╗\\ \tt ║A│◌│◌│◌║ 1\\ \tt ╟─┼─┤◌│◌║ 2\\ \tt ║◌│◌│◌│◌║ 3\\ \tt ║O│◌├─┴─╢ 4\\ \tt ╟─┴─┤T◌◌║ 5\\ \tt ║◌◌D└─┬─╢ 6\\ \tt ║P◌◌S◌│M║ 7\\ \tt ╚═════╧═╝\\ \tt ```\\ \tt  
	  } 
	   } 
	   } 
	 & & \\ 
 

    \theutterance \stepcounter{utterance}  

    & & \multicolumn{4}{p{0.6\linewidth}}{\cellcolor[rgb]{0.9,0.9,0.9}{%
	\makecell[{{p{\linewidth}}}]{% 
	  \tt {\tiny [GM$\rangle$B]}  
	 'T' erfolgreich nach (5,5) verschoben. Dein Raster sieht jetzt so aus:\\ \tt ```\\ \tt  1234567\\ \tt ╔═╤═╤═╤═╗\\ \tt ║A│◌│◌│◌║ 1\\ \tt ╟─┼─┤◌│◌║ 2\\ \tt ║◌│D│◌│◌║ 3\\ \tt ║O│◌├─┴─╢ 4\\ \tt ╟─┴─┤T◌◌║ 5\\ \tt ║PS◌└─┬─╢ 6\\ \tt ║◌◌M◌◌│◌║ 7\\ \tt ╚═════╧═╝\\ \tt ```\\ \tt Ihr seid gerade bei Runde 14 von maximal 28.\\ \tt Ihr habt gemeinsam 4 von maximal 16 Strafpunkten angesammelt.\\ \tt Dein Mitspieler hat ein Objekt auf seinem Raster bewegt.\\ \tt \\ \tt \\ \tt Wie lautet dein nächster Befehl? 
	  } 
	   } 
	   } 
	 & \\ 
 

    \theutterance \stepcounter{utterance}  

    & & & \multicolumn{4}{p{0.6\linewidth}}{\cellcolor[rgb]{1,0.85,0.72}{%
	\makecell[{{p{\linewidth}}}]{% 
	  \tt {\tiny [GM$\langle$B]}  
	 VERSCHIEBE: D, (3, 6) 
	  } 
	   } 
	   } 
	  \\ 
 

    \theutterance \stepcounter{utterance}  

    & & & \multicolumn{2}{p{0.3\linewidth}}{\cellcolor[rgb]{0.95,0.95,0.95}{%
	\makecell[{{p{\linewidth}}}]{% 
	  \tt {\tiny [GM$|$GM]}  
	 VERSCHIEBE: D, (3, 6) 
	  } 
	   } 
	   } 
	 & & \\ 
 

    \theutterance \stepcounter{utterance}  

    & & & \multicolumn{2}{p{0.3\linewidth}}{\cellcolor[rgb]{0.95,0.95,0.95}{%
	\makecell[{{p{\linewidth}}}]{% 
	  \tt {\tiny [GM$|$GM]}  
	 'D' erfolgreich nach (3,6) verschoben. Dein Raster sieht jetzt so aus:\\ \tt ```\\ \tt  1234567\\ \tt ╔═╤═╤═╤═╗\\ \tt ║A│◌│◌│◌║ 1\\ \tt ╟─┼─┤◌│◌║ 2\\ \tt ║◌│◌│◌│◌║ 3\\ \tt ║O│◌├─┴─╢ 4\\ \tt ╟─┴─┤T◌◌║ 5\\ \tt ║PSD└─┬─╢ 6\\ \tt ║◌◌M◌◌│◌║ 7\\ \tt ╚═════╧═╝\\ \tt ```\\ \tt  
	  } 
	   } 
	   } 
	 & & \\ 
 

    \theutterance \stepcounter{utterance}  

    & & \multicolumn{4}{p{0.6\linewidth}}{\cellcolor[rgb]{0.9,0.9,0.9}{%
	\makecell[{{p{\linewidth}}}]{% 
	  \tt {\tiny [A$\langle$GM]}  
	 'P' erfolgreich nach (1,7) verschoben. Dein Raster sieht jetzt so aus:\\ \tt ```\\ \tt  1234567\\ \tt ╔═╤═╤═╤═╗\\ \tt ║A│◌│◌│◌║ 1\\ \tt ╟─┼─┤◌│◌║ 2\\ \tt ║◌│◌│◌│◌║ 3\\ \tt ║O│◌├─┴─╢ 4\\ \tt ╟─┴─┤T◌◌║ 5\\ \tt ║◌◌D└─┬─╢ 6\\ \tt ║P◌◌S◌│M║ 7\\ \tt ╚═════╧═╝\\ \tt ```\\ \tt Ihr seid gerade bei Runde 14 von maximal 28.\\ \tt Ihr habt gemeinsam 4 von maximal 16 Strafpunkten angesammelt.\\ \tt Dein Mitspieler hat ein Objekt auf seinem Raster bewegt.\\ \tt \\ \tt \\ \tt Wie lautet dein nächster Befehl? 
	  } 
	   } 
	   } 
	 & \\ 
 

    \theutterance \stepcounter{utterance}  

    & \multicolumn{4}{p{0.6\linewidth}}{\cellcolor[rgb]{0.8,1,0.9}{%
	\makecell[{{p{\linewidth}}}]{% 
	  \tt {\tiny [A$\rangle$GM]}  
	 VERSCHIEBE: S, (2, 7) 
	  } 
	   } 
	   } 
	 & & \\ 
 

    \theutterance \stepcounter{utterance}  

    & & & \multicolumn{2}{p{0.3\linewidth}}{\cellcolor[rgb]{0.95,0.95,0.95}{%
	\makecell[{{p{\linewidth}}}]{% 
	  \tt {\tiny [GM$|$GM]}  
	 VERSCHIEBE: S, (2, 7) 
	  } 
	   } 
	   } 
	 & & \\ 
 

    \theutterance \stepcounter{utterance}  

    & & & \multicolumn{2}{p{0.3\linewidth}}{\cellcolor[rgb]{0.95,0.95,0.95}{%
	\makecell[{{p{\linewidth}}}]{% 
	  \tt {\tiny [GM$|$GM]}  
	 'S' erfolgreich nach (2,7) verschoben. Dein Raster sieht jetzt so aus:\\ \tt ```\\ \tt  1234567\\ \tt ╔═╤═╤═╤═╗\\ \tt ║A│◌│◌│◌║ 1\\ \tt ╟─┼─┤◌│◌║ 2\\ \tt ║◌│◌│◌│◌║ 3\\ \tt ║O│◌├─┴─╢ 4\\ \tt ╟─┴─┤T◌◌║ 5\\ \tt ║◌◌D└─┬─╢ 6\\ \tt ║PS◌◌◌│M║ 7\\ \tt ╚═════╧═╝\\ \tt ```\\ \tt  
	  } 
	   } 
	   } 
	 & & \\ 
 

    \theutterance \stepcounter{utterance}  

    & & \multicolumn{4}{p{0.6\linewidth}}{\cellcolor[rgb]{0.9,0.9,0.9}{%
	\makecell[{{p{\linewidth}}}]{% 
	  \tt {\tiny [GM$\rangle$B]}  
	 'D' erfolgreich nach (3,6) verschoben. Dein Raster sieht jetzt so aus:\\ \tt ```\\ \tt  1234567\\ \tt ╔═╤═╤═╤═╗\\ \tt ║A│◌│◌│◌║ 1\\ \tt ╟─┼─┤◌│◌║ 2\\ \tt ║◌│◌│◌│◌║ 3\\ \tt ║O│◌├─┴─╢ 4\\ \tt ╟─┴─┤T◌◌║ 5\\ \tt ║PSD└─┬─╢ 6\\ \tt ║◌◌M◌◌│◌║ 7\\ \tt ╚═════╧═╝\\ \tt ```\\ \tt Ihr seid gerade bei Runde 15 von maximal 28.\\ \tt Ihr habt gemeinsam 4 von maximal 16 Strafpunkten angesammelt.\\ \tt Dein Mitspieler hat ein Objekt auf seinem Raster bewegt.\\ \tt \\ \tt \\ \tt Wie lautet dein nächster Befehl? 
	  } 
	   } 
	   } 
	 & \\ 
 

    \theutterance \stepcounter{utterance}  

    & & & \multicolumn{4}{p{0.6\linewidth}}{\cellcolor[rgb]{1,0.85,0.72}{%
	\makecell[{{p{\linewidth}}}]{% 
	  \tt {\tiny [GM$\langle$B]}  
	 VERSCHIEBE: P, (1, 7) 
	  } 
	   } 
	   } 
	  \\ 
 

    \theutterance \stepcounter{utterance}  

    & & & \multicolumn{2}{p{0.3\linewidth}}{\cellcolor[rgb]{0.95,0.95,0.95}{%
	\makecell[{{p{\linewidth}}}]{% 
	  \tt {\tiny [GM$|$GM]}  
	 VERSCHIEBE: P, (1, 7) 
	  } 
	   } 
	   } 
	 & & \\ 
 

    \theutterance \stepcounter{utterance}  

    & & & \multicolumn{2}{p{0.3\linewidth}}{\cellcolor[rgb]{0.95,0.95,0.95}{%
	\makecell[{{p{\linewidth}}}]{% 
	  \tt {\tiny [GM$|$GM]}  
	 'P' erfolgreich nach (1,7) verschoben. Dein Raster sieht jetzt so aus:\\ \tt ```\\ \tt  1234567\\ \tt ╔═╤═╤═╤═╗\\ \tt ║A│◌│◌│◌║ 1\\ \tt ╟─┼─┤◌│◌║ 2\\ \tt ║◌│◌│◌│◌║ 3\\ \tt ║O│◌├─┴─╢ 4\\ \tt ╟─┴─┤T◌◌║ 5\\ \tt ║◌SD└─┬─╢ 6\\ \tt ║P◌M◌◌│◌║ 7\\ \tt ╚═════╧═╝\\ \tt ```\\ \tt  
	  } 
	   } 
	   } 
	 & & \\ 
 

    \theutterance \stepcounter{utterance}  

    & & \multicolumn{4}{p{0.6\linewidth}}{\cellcolor[rgb]{0.9,0.9,0.9}{%
	\makecell[{{p{\linewidth}}}]{% 
	  \tt {\tiny [A$\langle$GM]}  
	 'S' erfolgreich nach (2,7) verschoben. Dein Raster sieht jetzt so aus:\\ \tt ```\\ \tt  1234567\\ \tt ╔═╤═╤═╤═╗\\ \tt ║A│◌│◌│◌║ 1\\ \tt ╟─┼─┤◌│◌║ 2\\ \tt ║◌│◌│◌│◌║ 3\\ \tt ║O│◌├─┴─╢ 4\\ \tt ╟─┴─┤T◌◌║ 5\\ \tt ║◌◌D└─┬─╢ 6\\ \tt ║PS◌◌◌│M║ 7\\ \tt ╚═════╧═╝\\ \tt ```\\ \tt Ihr seid gerade bei Runde 15 von maximal 28.\\ \tt Ihr habt gemeinsam 4 von maximal 16 Strafpunkten angesammelt.\\ \tt Dein Mitspieler hat ein Objekt auf seinem Raster bewegt.\\ \tt \\ \tt \\ \tt Wie lautet dein nächster Befehl? 
	  } 
	   } 
	   } 
	 & \\ 
 

    \theutterance \stepcounter{utterance}  

    & \multicolumn{4}{p{0.6\linewidth}}{\cellcolor[rgb]{0.8,1,0.9}{%
	\makecell[{{p{\linewidth}}}]{% 
	  \tt {\tiny [A$\rangle$GM]}  
	 VERSCHIEBE: M, (3, 7) 
	  } 
	   } 
	   } 
	 & & \\ 
 

    \theutterance \stepcounter{utterance}  

    & & & \multicolumn{2}{p{0.3\linewidth}}{\cellcolor[rgb]{0.95,0.95,0.95}{%
	\makecell[{{p{\linewidth}}}]{% 
	  \tt {\tiny [GM$|$GM]}  
	 VERSCHIEBE: M, (3, 7) 
	  } 
	   } 
	   } 
	 & & \\ 
 

    \theutterance \stepcounter{utterance}  

    & & & \multicolumn{2}{p{0.3\linewidth}}{\cellcolor[rgb]{0.95,0.95,0.95}{%
	\makecell[{{p{\linewidth}}}]{% 
	  \tt {\tiny [GM$|$GM]}  
	 'M' erfolgreich nach (3,7) verschoben. Dein Raster sieht jetzt so aus:\\ \tt ```\\ \tt  1234567\\ \tt ╔═╤═╤═╤═╗\\ \tt ║A│◌│◌│◌║ 1\\ \tt ╟─┼─┤◌│◌║ 2\\ \tt ║◌│◌│◌│◌║ 3\\ \tt ║O│◌├─┴─╢ 4\\ \tt ╟─┴─┤T◌◌║ 5\\ \tt ║◌◌D└─┬─╢ 6\\ \tt ║PSM◌◌│◌║ 7\\ \tt ╚═════╧═╝\\ \tt ```\\ \tt  
	  } 
	   } 
	   } 
	 & & \\ 
 

    \theutterance \stepcounter{utterance}  

    & & \multicolumn{4}{p{0.6\linewidth}}{\cellcolor[rgb]{0.9,0.9,0.9}{%
	\makecell[{{p{\linewidth}}}]{% 
	  \tt {\tiny [GM$\rangle$B]}  
	 'P' erfolgreich nach (1,7) verschoben. Dein Raster sieht jetzt so aus:\\ \tt ```\\ \tt  1234567\\ \tt ╔═╤═╤═╤═╗\\ \tt ║A│◌│◌│◌║ 1\\ \tt ╟─┼─┤◌│◌║ 2\\ \tt ║◌│◌│◌│◌║ 3\\ \tt ║O│◌├─┴─╢ 4\\ \tt ╟─┴─┤T◌◌║ 5\\ \tt ║◌SD└─┬─╢ 6\\ \tt ║P◌M◌◌│◌║ 7\\ \tt ╚═════╧═╝\\ \tt ```\\ \tt Ihr seid gerade bei Runde 16 von maximal 28.\\ \tt Ihr habt gemeinsam 4 von maximal 16 Strafpunkten angesammelt.\\ \tt Dein Mitspieler hat ein Objekt auf seinem Raster bewegt.\\ \tt \\ \tt \\ \tt Wie lautet dein nächster Befehl? 
	  } 
	   } 
	   } 
	 & \\ 
 

    \theutterance \stepcounter{utterance}  

    & & & \multicolumn{4}{p{0.6\linewidth}}{\cellcolor[rgb]{1,0.85,0.72}{%
	\makecell[{{p{\linewidth}}}]{% 
	  \tt {\tiny [GM$\langle$B]}  
	 VERSCHIEBE: S, (2, 7) 
	  } 
	   } 
	   } 
	  \\ 
 

    \theutterance \stepcounter{utterance}  

    & & & \multicolumn{2}{p{0.3\linewidth}}{\cellcolor[rgb]{0.95,0.95,0.95}{%
	\makecell[{{p{\linewidth}}}]{% 
	  \tt {\tiny [GM$|$GM]}  
	 VERSCHIEBE: S, (2, 7) 
	  } 
	   } 
	   } 
	 & & \\ 
 

    \theutterance \stepcounter{utterance}  

    & & & \multicolumn{2}{p{0.3\linewidth}}{\cellcolor[rgb]{0.95,0.95,0.95}{%
	\makecell[{{p{\linewidth}}}]{% 
	  \tt {\tiny [GM$|$GM]}  
	 'S' erfolgreich nach (2,7) verschoben. Dein Raster sieht jetzt so aus:\\ \tt ```\\ \tt  1234567\\ \tt ╔═╤═╤═╤═╗\\ \tt ║A│◌│◌│◌║ 1\\ \tt ╟─┼─┤◌│◌║ 2\\ \tt ║◌│◌│◌│◌║ 3\\ \tt ║O│◌├─┴─╢ 4\\ \tt ╟─┴─┤T◌◌║ 5\\ \tt ║◌◌D└─┬─╢ 6\\ \tt ║PSM◌◌│◌║ 7\\ \tt ╚═════╧═╝\\ \tt ```\\ \tt  
	  } 
	   } 
	   } 
	 & & \\ 
 

    \theutterance \stepcounter{utterance}  

    & & \multicolumn{4}{p{0.6\linewidth}}{\cellcolor[rgb]{0.9,0.9,0.9}{%
	\makecell[{{p{\linewidth}}}]{% 
	  \tt {\tiny [A$\langle$GM]}  
	 'M' erfolgreich nach (3,7) verschoben. Dein Raster sieht jetzt so aus:\\ \tt ```\\ \tt  1234567\\ \tt ╔═╤═╤═╤═╗\\ \tt ║A│◌│◌│◌║ 1\\ \tt ╟─┼─┤◌│◌║ 2\\ \tt ║◌│◌│◌│◌║ 3\\ \tt ║O│◌├─┴─╢ 4\\ \tt ╟─┴─┤T◌◌║ 5\\ \tt ║◌◌D└─┬─╢ 6\\ \tt ║PSM◌◌│◌║ 7\\ \tt ╚═════╧═╝\\ \tt ```\\ \tt Ihr seid gerade bei Runde 16 von maximal 28.\\ \tt Ihr habt gemeinsam 4 von maximal 16 Strafpunkten angesammelt.\\ \tt Dein Mitspieler hat ein Objekt auf seinem Raster bewegt.\\ \tt \\ \tt \\ \tt Wie lautet dein nächster Befehl? 
	  } 
	   } 
	   } 
	 & \\ 
 

    \theutterance \stepcounter{utterance}  

    & \multicolumn{4}{p{0.6\linewidth}}{\cellcolor[rgb]{0.8,1,0.9}{%
	\makecell[{{p{\linewidth}}}]{% 
	  \tt {\tiny [A$\rangle$GM]}  
	 SPRICH: Perfekt! Alle meine Objekte sind jetzt an den vereinbarten Positionen: A(1,1), O(1,4), T(5,5), D(3,6), P(1,7), S(2,7), M(3,7). Bist du auch fertig mit deinen Bewegungen? fertig? 
	  } 
	   } 
	   } 
	 & & \\ 
 

    \theutterance \stepcounter{utterance}  

    & & & \multicolumn{2}{p{0.3\linewidth}}{\cellcolor[rgb]{0.95,0.95,0.95}{%
	\makecell[{{p{\linewidth}}}]{% 
	  \tt {\tiny [GM$|$GM]}  
	 SPRICH: Perfekt! Alle meine Objekte sind jetzt an den vereinbarten Positionen: A(1,1), O(1,4), T(5,5), D(3,6), P(1,7), S(2,7), M(3,7). Bist du auch fertig mit deinen Bewegungen? fertig? 
	  } 
	   } 
	   } 
	 & & \\ 
 

    \theutterance \stepcounter{utterance}  

    & & \multicolumn{4}{p{0.6\linewidth}}{\cellcolor[rgb]{0.9,0.9,0.9}{%
	\makecell[{{p{\linewidth}}}]{% 
	  \tt {\tiny [GM$\rangle$B]}  
	 'S' erfolgreich nach (2,7) verschoben. Dein Raster sieht jetzt so aus:\\ \tt ```\\ \tt  1234567\\ \tt ╔═╤═╤═╤═╗\\ \tt ║A│◌│◌│◌║ 1\\ \tt ╟─┼─┤◌│◌║ 2\\ \tt ║◌│◌│◌│◌║ 3\\ \tt ║O│◌├─┴─╢ 4\\ \tt ╟─┴─┤T◌◌║ 5\\ \tt ║◌◌D└─┬─╢ 6\\ \tt ║PSM◌◌│◌║ 7\\ \tt ╚═════╧═╝\\ \tt ```\\ \tt Ihr seid gerade bei Runde 17 von maximal 28.\\ \tt Ihr habt gemeinsam 4 von maximal 16 Strafpunkten angesammelt.\\ \tt Dein Mitspieler hat dir diese Nachricht geschickt:\\ \tt \\ \tt \\ \tt "Perfekt! Alle meine Objekte sind jetzt an den vereinbarten Positionen: A(1,1), O(1,4), T(5,5), D(3,6), P(1,7), S(2,7), M(3,7). Bist du auch fertig mit deinen Bewegungen? fertig?"\\ \tt \\ \tt \\ \tt Wie lautet dein nächster Befehl? 
	  } 
	   } 
	   } 
	 & \\ 
 

    \theutterance \stepcounter{utterance}  

    & & & \multicolumn{4}{p{0.6\linewidth}}{\cellcolor[rgb]{1,0.85,0.72}{%
	\makecell[{{p{\linewidth}}}]{% 
	  \tt {\tiny [GM$\langle$B]}  
	 SPRICH: fertig! 
	  } 
	   } 
	   } 
	  \\ 
 

    \theutterance \stepcounter{utterance}  

    & & & \multicolumn{2}{p{0.3\linewidth}}{\cellcolor[rgb]{0.95,0.95,0.95}{%
	\makecell[{{p{\linewidth}}}]{% 
	  \tt {\tiny [GM$|$GM]}  
	 SPRICH: fertig! 
	  } 
	   } 
	   } 
	 & & \\ 
 

    \theutterance \stepcounter{utterance}  

    & & & \multicolumn{2}{p{0.3\linewidth}}{\cellcolor[rgb]{0.95,0.95,0.95}{%
	\makecell[{{p{\linewidth}}}]{% 
	  \tt {\tiny [GM$|$GM]}  
	 true 
	  } 
	   } 
	   } 
	 & & \\ 
 

    \theutterance \stepcounter{utterance}  

    & & & \multicolumn{2}{p{0.3\linewidth}}{\cellcolor[rgb]{0.95,0.95,0.95}{%
	\makecell[{{p{\linewidth}}}]{% 
	  \tt {\tiny [GM$|$GM]}  
	 * success: True\\ \tt * lose: False\\ \tt * aborted: False\\ \tt -------\\ \tt * Shifts: 12.00\\ \tt * Max Shifts: 12.00\\ \tt * Min Shifts: 6.00\\ \tt * End Distance Sum: 0.00\\ \tt * Init Distance Sum: 23.67\\ \tt * Expected Distance Sum: 29.33\\ \tt * Penalties: 4.00\\ \tt * Max Penalties: 16.00\\ \tt * Rounds: 17.00\\ \tt * Max Rounds: 28.00\\ \tt * Object Count: 7.00\\ \tt  
	  } 
	   } 
	   } 
	 & & \\ 
 

    \theutterance \stepcounter{utterance}  

    & & & \multicolumn{2}{p{0.3\linewidth}}{\cellcolor[rgb]{0.95,0.95,0.95}{%
	\makecell[{{p{\linewidth}}}]{% 
	  \tt {\tiny [GM$|$GM]}  
	 * Main Score: 27.94\\ \tt \\ \tt -------\\ \tt * Distance Score: 1.00\\ \tt * Consistency Score: 0.14\\ \tt * Coverage Score: 0.77\\ \tt * Penalty Score: 0.87\\ \tt * Alternative Penalty Score: 0.93\\ \tt * Alternative Main Score: 92.86\\ \tt \\ \tt -------\\ \tt * Shifts: 12.00\\ \tt * Max Shifts: 12.00\\ \tt * Min Shifts: 6.00\\ \tt * End Distance Sum: 0.00\\ \tt * Init Distance Sum: 23.67\\ \tt * Expected Distance Sum: 29.33\\ \tt * Penalties: 4.00\\ \tt * Max Penalties: 16.00\\ \tt * Rounds: 17.00\\ \tt * Max Rounds: 28.00\\ \tt * Object Count: 7.00\\ \tt  
	  } 
	   } 
	   } 
	 & & \\ 
 

\end{supertabular}
}

\end{document}
