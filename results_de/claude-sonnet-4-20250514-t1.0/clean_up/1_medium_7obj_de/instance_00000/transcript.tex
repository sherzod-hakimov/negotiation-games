
\documentclass{article}
\usepackage{colortbl}
\usepackage{makecell}
\usepackage{multirow}
\usepackage{supertabular}

\begin{document}

\newcounter{utterance}

\centering \large Interaction Transcript for game `clean\_up', experiment `1\_medium\_7obj\_de', episode 0 with claude{-}sonnet{-}4{-}20250514{-}t1.0.
\vspace{24pt}

{ \footnotesize  \setcounter{utterance}{1}
\setlength{\tabcolsep}{0pt}
\begin{supertabular}{c@{$\;$}|p{.15\linewidth}@{}p{.15\linewidth}p{.15\linewidth}p{.15\linewidth}p{.15\linewidth}p{.15\linewidth}}
    \# & $\;$A & \multicolumn{4}{c}{Game Master} & $\;\:$B\\
    \hline

    \theutterance \stepcounter{utterance}  
    & & \multicolumn{4}{p{0.6\linewidth}}{
        \cellcolor[rgb]{0.9,0.9,0.9}{
            \makecell[{{p{\linewidth}}}]{
                \texttt{\tiny{[P1$\langle$GM]}}
                \texttt{Ich bin der Spielleiter, und du spielst ein kooperatives Spiel mit dem folgenden Raster als Spielbrett:} \\
\\ 
\texttt{```} \\
\texttt{ 1234567} \\
\texttt{╔═╤═╤═╤═╗} \\
\texttt{║◌│◌│◌│◌║ 1} \\
\texttt{║P└─┼─┴─╢ 2} \\
\texttt{║◌SA│◌◌◌║ 3} \\
\texttt{╟──┬┴───╢ 4} \\
\texttt{║◌◌│◌◌O◌║ 5} \\
\texttt{║◌◌│◌◌MD║ 6} \\
\texttt{║◌◌│◌◌◌T║ 7} \\
\texttt{╚══╧════╝} \\
\\ 
\texttt{```} \\
\\ 
\texttt{* Am oberen Rand sind die x{-}Koordinaten angegeben, die nach rechts hin zunehmen, und am rechten Rand die y{-}Koordinaten, die nach unten zunehmen.} \\
\texttt{* Die folgenden Objekte sind zufällig auf deinem Raster platziert: 'P', 'O', 'T', 'S', 'D', 'A', 'M'.} \\
\\ 
\\ 
\texttt{Dein Mitspieler sieht eine Variation des Spielbretts, auf dem dieselben Objekte an anderen zufälligen Positionen platziert sind. Du kannst das Spielbrett des anderen Spielers nicht sehen, und er kann deins nicht sehen.} \\
\\ 
\\ 
\texttt{**Ziel des Spiels:**} \\
\\ 
\texttt{Beide Spieler müssen die Objekte auf ihrem jeweiligen Spielbrett so bewegen, dass identische Objekte am Ende an denselben Koordinaten liegen. Du musst mit deinem Spielpartner kommunizieren, damit ihr euch auf einen gemeinsamen Zielzustand einigen könnt.} \\
\\ 
\\ 
\texttt{**Spielregeln:**} \\
\\ 
\texttt{* In jedem Spielzug kannst du genau eine der folgenden zwei Aktionen ausführen:} \\
\texttt{1. `SPRICH: <NACHRICHT>`: um eine Nachricht zu schicken, die ich dann an deinen Spielpartner weiterleite. Als Nachricht gilt alles bis zum nächsten Zeilenumbruch.} \\
\texttt{2. `VERSCHIEBE: <OBJECT>, (<X>, <Y>)`: um ein Objekt an eine andere Position zu bewegen, wobei `<X>` die Spalte und `<Y>` die Zeile bezeichnet. Ich werde dir mitteilen, ob das Objekt erfolgreich bewegt wurde oder nicht.} \\
\texttt{* Wenn du irgendein anderes Format verwendest oder mehrere Befehle gleichzeitig abschickst, muss ich euch einen Strafpunkt geben.} \\
\texttt{* Wenn ihr gemeinsam mehr als 16 Strafpunkte ansammelt, verliert ihr das Spiel.} \\
\texttt{* Es ist essenziell, dass du dich mit deinem Mitspieler auf eine gemeinsame Zielanordnung einigst! Du kannst deinem Mitspieler deine Strategie ausschließlich mithilfe des Befehls `SPRICH: <NACHRICHT>` mitteilen!} \\
\\ 
\\ 
\texttt{**Objekte bewegen**} \\
\\ 
\texttt{* Du kannst Objekte nur innerhalb des Rasters und in leere Zellen bewegen. Leere Zellen erkennst am Zeichen '◌'.} \\
\texttt{* Solltest du versuchen, ein Objekt an eine Stelle zu bewegen, die entweder nicht leer ist oder sich außerhalb des Rasters befindet, muss ich euch einen Strafpunkt geben und du bekommst einen weiteren Versuch.} \\
\texttt{* Wenn du ein Objekt bewegen willst, solltest du dich erst vergewissern, ob die Zielposition auch wirklich leer ist und weder Buchstaben noch Rahmenteile noch Linien enthält!} \\
\\ 
\\ 
\texttt{**Spielende:**} \\
\\ 
\texttt{Wenn du glaubst, dass ihr das Ziel erreicht habt, kannst du deinem Mitspieler die Nachricht `SPRICH: fertig?` schicken. Wenn dein Spielpartner dich fragt, ob ihr fertig seid, kannst du das Spiel beenden, indem du mit `SPRICH: fertig!` antwortest.} \\
\\ 
\texttt{Ihr gewinnt gemeinsam, wenn ihr das Spiel innerhalb von \#max\_rounds Runden beendet. Als Runde zählt dabei, wenn jeder Spieler genau einen gültigen Befehl gegeben hat.} \\
\\ 
\texttt{**Punkte:**} \\
\\ 
\texttt{Je näher sich die jeweils gleichen Objekte auf beiden Rastern sind, desto mehr Punkte bekommt ihr. Für Strafpunkte gibt es Punktabzug. Könnt ihr den Rekord brechen?} \\
\\ 
\texttt{Schicke eine Nachricht an deinen Mitspieler um das Spiel zu starten!} \\
            }
        }
    }
    & \\ \\

    \theutterance \stepcounter{utterance}  
    & \multicolumn{4}{p{0.6\linewidth}}{
        \cellcolor[rgb]{0.9,0.9,0.9}{
            \makecell[{{p{\linewidth}}}]{
                \texttt{\tiny{[P1$\rangle$GM]}}
                \texttt{SPRICH: Hallo! Lass uns das Spiel starten. Hier sind die Positionen meiner Objekte: P(1,2), S(2,3), A(3,3), O(6,5), M(6,6), D(7,6), T(7,7). Wo stehen deine Objekte? Dann können wir uns auf eine gemeinsame Zielanordnung einigen.} \\
            }
        }
    }
    & & \\ \\

    \theutterance \stepcounter{utterance}  
    & & & \multicolumn{2}{p{0.3\linewidth}}{
        \cellcolor[rgb]{0.9,0.9,0.9}{
            \makecell[{{p{\linewidth}}}]{
                \texttt{\tiny{[GM$|$GM]}}
                \texttt{SPRICH: Hallo! Lass uns das Spiel starten. Hier sind die Positionen meiner Objekte: P(1,2), S(2,3), A(3,3), O(6,5), M(6,6), D(7,6), T(7,7). Wo stehen deine Objekte? Dann können wir uns auf eine gemeinsame Zielanordnung einigen.} \\
            }
        }
    }
    & & \\ \\

    \theutterance \stepcounter{utterance}  
    & & \multicolumn{4}{p{0.6\linewidth}}{
        \cellcolor[rgb]{0.9,0.9,0.9}{
            \makecell[{{p{\linewidth}}}]{
                \texttt{\tiny{[GM$\rangle$P2]}}
                \texttt{Ich bin der Spielleiter, und du spielst ein kooperatives Spiel mit dem folgenden Raster als Spielbrett:} \\
\\ 
\texttt{```} \\
\texttt{ 1234567} \\
\texttt{╔═╤═╤═╤═╗} \\
\texttt{║◌│◌│◌│P║ 1} \\
\texttt{║T└─┼─┴─╢ 2} \\
\texttt{║◌◌◌│M◌◌║ 3} \\
\texttt{╟──┬┴───╢ 4} \\
\texttt{║◌D│◌◌O◌║ 5} \\
\texttt{║◌◌│◌S◌◌║ 6} \\
\texttt{║A◌│◌◌◌◌║ 7} \\
\texttt{╚══╧════╝} \\
\\ 
\texttt{```} \\
\\ 
\texttt{* Am oberen Rand sind die x{-}Koordinaten angegeben, die nach rechts hin zunehmen, und am rechten Rand die y{-}Koordinaten, die nach unten zunehmen.} \\
\texttt{* Die folgenden Objekte sind zufällig auf deinem Raster platziert: 'P', 'O', 'T', 'S', 'D', 'A', 'M'.} \\
\\ 
\\ 
\texttt{Dein Mitspieler sieht eine Variation des Spielbretts, auf dem dieselben Objekte an anderen zufälligen Positionen platziert sind. Du kannst das Spielbrett des anderen Spielers nicht sehen, und er kann deins nicht sehen.} \\
\\ 
\\ 
\texttt{**Ziel des Spiels:**} \\
\\ 
\texttt{Beide Spieler müssen die Objekte auf ihrem jeweiligen Spielbrett so bewegen, dass identische Objekte am Ende an denselben Koordinaten liegen. Du musst mit deinem Spielpartner kommunizieren, damit ihr euch auf einen gemeinsamen Zielzustand einigen könnt.} \\
\\ 
\\ 
\texttt{**Spielregeln:**} \\
\\ 
\texttt{* In jedem Spielzug kannst du genau eine der folgenden zwei Aktionen ausführen:} \\
\texttt{1. `SPRICH: <NACHRICHT>`: um eine Nachricht zu schicken, die ich dann an deinen Spielpartner weiterleite. Als Nachricht gilt alles bis zum nächsten Zeilenumbruch.} \\
\texttt{2. `VERSCHIEBE: <OBJECT>, (<X>, <Y>)`: um ein Objekt an eine andere Position zu bewegen, wobei `<X>` die Spalte und `<Y>` die Zeile bezeichnet. Ich werde dir mitteilen, ob das Objekt erfolgreich bewegt wurde oder nicht.} \\
\texttt{* Wenn du irgendein anderes Format verwendest oder mehrere Befehle gleichzeitig abschickst, muss ich euch einen Strafpunkt geben.} \\
\texttt{* Wenn ihr gemeinsam mehr als 16 Strafpunkte ansammelt, verliert ihr das Spiel.} \\
\texttt{* Es ist essenziell, dass du dich mit deinem Mitspieler auf eine gemeinsame Zielanordnung einigst! Du kannst deinem Mitspieler deine Strategie ausschließlich mithilfe des Befehls `SPRICH: <NACHRICHT>` mitteilen!} \\
\\ 
\\ 
\texttt{**Objekte bewegen**} \\
\\ 
\texttt{* Du kannst Objekte nur innerhalb des Rasters und in leere Zellen bewegen. Leere Zellen erkennst am Zeichen '◌'.} \\
\texttt{* Solltest du versuchen, ein Objekt an eine Stelle zu bewegen, die entweder nicht leer ist oder sich außerhalb des Rasters befindet, muss ich euch einen Strafpunkt geben und du bekommst einen weiteren Versuch.} \\
\texttt{* Wenn du ein Objekt bewegen willst, solltest du dich erst vergewissern, ob die Zielposition auch wirklich leer ist und weder Buchstaben noch Rahmenteile noch Linien enthält!} \\
\\ 
\\ 
\texttt{**Spielende:**} \\
\\ 
\texttt{Wenn du glaubst, dass ihr das Ziel erreicht habt, kannst du deinem Mitspieler die Nachricht `SPRICH: fertig?` schicken. Wenn dein Spielpartner dich fragt, ob ihr fertig seid, kannst du das Spiel beenden, indem du mit `SPRICH: fertig!` antwortest.} \\
\\ 
\texttt{Ihr gewinnt gemeinsam, wenn ihr das Spiel innerhalb von \#max\_rounds Runden beendet. Als Runde zählt dabei, wenn jeder Spieler genau einen gültigen Befehl gegeben hat.} \\
\\ 
\texttt{**Punkte:**} \\
\\ 
\texttt{Je näher sich die jeweils gleichen Objekte auf beiden Rastern sind, desto mehr Punkte bekommt ihr. Für Strafpunkte gibt es Punktabzug. Könnt ihr den Rekord brechen?} \\
\\ 
\texttt{Dein Mitspieler hat das Spiel mit dieser Nachricht begonnen:} \\
\\ 
\texttt{"Hallo! Lass uns das Spiel starten. Hier sind die Positionen meiner Objekte: P(1,2), S(2,3), A(3,3), O(6,5), M(6,6), D(7,6), T(7,7). Wo stehen deine Objekte? Dann können wir uns auf eine gemeinsame Zielanordnung einigen."} \\
\\ 
\texttt{Wie lautet dein erster Befehl?} \\
            }
        }
    }
    & \\ \\

    \theutterance \stepcounter{utterance}  
    & & & \multicolumn{4}{p{0.6\linewidth}}{
        \cellcolor[rgb]{0.9,0.9,0.9}{
            \makecell[{{p{\linewidth}}}]{
                \texttt{\tiny{[GM$\langle$P2]}}
                \texttt{SPRICH: Hallo! Hier sind die Positionen meiner Objekte: P(7,1), T(1,2), M(4,3), D(2,5), O(5,5), S(4,6), A(1,7). Lass uns eine gemeinsame Zielanordnung finden. Welche Koordinaten schlägst du vor?} \\
            }
        }
    }
     \\ \\

    \theutterance \stepcounter{utterance}  
    & & & \multicolumn{2}{p{0.3\linewidth}}{
        \cellcolor[rgb]{0.9,0.9,0.9}{
            \makecell[{{p{\linewidth}}}]{
                \texttt{\tiny{[GM$|$GM]}}
                \texttt{SPRICH: Hallo! Hier sind die Positionen meiner Objekte: P(7,1), T(1,2), M(4,3), D(2,5), O(5,5), S(4,6), A(1,7). Lass uns eine gemeinsame Zielanordnung finden. Welche Koordinaten schlägst du vor?} \\
            }
        }
    }
    & & \\ \\

    \theutterance \stepcounter{utterance}  
    & & \multicolumn{4}{p{0.6\linewidth}}{
        \cellcolor[rgb]{0.9,0.9,0.9}{
            \makecell[{{p{\linewidth}}}]{
                \texttt{\tiny{[P1$\langle$GM]}}
                \texttt{Deine Nachricht wurde an deinen Mitspieler übermittelt.} \\
\\ 
\\ 
\texttt{Ihr seid gerade bei Runde 1 von maximal 28.} \\
\texttt{Ihr habt gemeinsam 0 von maximal 16 Strafpunkten angesammelt.} \\
\texttt{Dein Mitspieler hat dir diese Nachricht geschickt:} \\
\\ 
\\ 
\texttt{"Hallo! Hier sind die Positionen meiner Objekte: P(7,1), T(1,2), M(4,3), D(2,5), O(5,5), S(4,6), A(1,7). Lass uns eine gemeinsame Zielanordnung finden. Welche Koordinaten schlägst du vor?"} \\
\\ 
\\ 
\texttt{Wie lautet dein nächster Befehl?} \\
            }
        }
    }
    & \\ \\

    \theutterance \stepcounter{utterance}  
    & \multicolumn{4}{p{0.6\linewidth}}{
        \cellcolor[rgb]{0.9,0.9,0.9}{
            \makecell[{{p{\linewidth}}}]{
                \texttt{\tiny{[P1$\rangle$GM]}}
                \texttt{SPRICH: Danke für die Positionen! Ich schlage folgende zentrale Zielanordnung vor: P(3,2), T(4,2), M(3,3), D(4,3), O(3,4), S(4,4), A(3,5). Das wäre eine kompakte 2x4 Anordnung in der Mitte des Spielfelds. Was denkst du darüber?} \\
            }
        }
    }
    & & \\ \\

    \theutterance \stepcounter{utterance}  
    & & & \multicolumn{2}{p{0.3\linewidth}}{
        \cellcolor[rgb]{0.9,0.9,0.9}{
            \makecell[{{p{\linewidth}}}]{
                \texttt{\tiny{[GM$|$GM]}}
                \texttt{SPRICH: Danke für die Positionen! Ich schlage folgende zentrale Zielanordnung vor: P(3,2), T(4,2), M(3,3), D(4,3), O(3,4), S(4,4), A(3,5). Das wäre eine kompakte 2x4 Anordnung in der Mitte des Spielfelds. Was denkst du darüber?} \\
            }
        }
    }
    & & \\ \\

    \theutterance \stepcounter{utterance}  
    & & \multicolumn{4}{p{0.6\linewidth}}{
        \cellcolor[rgb]{0.9,0.9,0.9}{
            \makecell[{{p{\linewidth}}}]{
                \texttt{\tiny{[GM$\rangle$P2]}}
                \texttt{Deine Nachricht wurde an deinen Mitspieler übermittelt.} \\
\\ 
\\ 
\texttt{Ihr seid gerade bei Runde 2 von maximal 28.} \\
\texttt{Ihr habt gemeinsam 0 von maximal 16 Strafpunkten angesammelt.} \\
\texttt{Dein Mitspieler hat dir diese Nachricht geschickt:} \\
\\ 
\\ 
\texttt{"Danke für die Positionen! Ich schlage folgende zentrale Zielanordnung vor: P(3,2), T(4,2), M(3,3), D(4,3), O(3,4), S(4,4), A(3,5). Das wäre eine kompakte 2x4 Anordnung in der Mitte des Spielfelds. Was denkst du darüber?"} \\
\\ 
\\ 
\texttt{Wie lautet dein nächster Befehl?} \\
            }
        }
    }
    & \\ \\

    \theutterance \stepcounter{utterance}  
    & & & \multicolumn{4}{p{0.6\linewidth}}{
        \cellcolor[rgb]{0.9,0.9,0.9}{
            \makecell[{{p{\linewidth}}}]{
                \texttt{\tiny{[GM$\langle$P2]}}
                \texttt{SPRICH: Die Idee einer kompakten Anordnung ist gut, aber ich sehe ein Problem: Position (4,2) scheint bei mir eine Linie/Rahmenteil zu enthalten. Lass uns eine sichere Alternative wählen: P(2,3), T(3,3), M(4,3), D(5,3), O(2,4), S(3,4), A(4,4). Das ist eine 4x2 Anordnung, die sicher auf freien Feldern liegt. Einverstanden?} \\
            }
        }
    }
     \\ \\

    \theutterance \stepcounter{utterance}  
    & & & \multicolumn{2}{p{0.3\linewidth}}{
        \cellcolor[rgb]{0.9,0.9,0.9}{
            \makecell[{{p{\linewidth}}}]{
                \texttt{\tiny{[GM$|$GM]}}
                \texttt{SPRICH: Die Idee einer kompakten Anordnung ist gut, aber ich sehe ein Problem: Position (4,2) scheint bei mir eine Linie/Rahmenteil zu enthalten. Lass uns eine sichere Alternative wählen: P(2,3), T(3,3), M(4,3), D(5,3), O(2,4), S(3,4), A(4,4). Das ist eine 4x2 Anordnung, die sicher auf freien Feldern liegt. Einverstanden?} \\
            }
        }
    }
    & & \\ \\

    \theutterance \stepcounter{utterance}  
    & & \multicolumn{4}{p{0.6\linewidth}}{
        \cellcolor[rgb]{0.9,0.9,0.9}{
            \makecell[{{p{\linewidth}}}]{
                \texttt{\tiny{[P1$\langle$GM]}}
                \texttt{Deine Nachricht wurde an deinen Mitspieler übermittelt.} \\
\\ 
\\ 
\texttt{Ihr seid gerade bei Runde 2 von maximal 28.} \\
\texttt{Ihr habt gemeinsam 0 von maximal 16 Strafpunkten angesammelt.} \\
\texttt{Dein Mitspieler hat dir diese Nachricht geschickt:} \\
\\ 
\\ 
\texttt{"Die Idee einer kompakten Anordnung ist gut, aber ich sehe ein Problem: Position (4,2) scheint bei mir eine Linie/Rahmenteil zu enthalten. Lass uns eine sichere Alternative wählen: P(2,3), T(3,3), M(4,3), D(5,3), O(2,4), S(3,4), A(4,4). Das ist eine 4x2 Anordnung, die sicher auf freien Feldern liegt. Einverstanden?"} \\
\\ 
\\ 
\texttt{Wie lautet dein nächster Befehl?} \\
            }
        }
    }
    & \\ \\

    \theutterance \stepcounter{utterance}  
    & \multicolumn{4}{p{0.6\linewidth}}{
        \cellcolor[rgb]{0.9,0.9,0.9}{
            \makecell[{{p{\linewidth}}}]{
                \texttt{\tiny{[P1$\rangle$GM]}}
                \texttt{SPRICH: Du hast recht mit (4,2)! Aber ich sehe auch bei deinem Vorschlag Probleme {-} (2,4), (3,4) und (4,4) enthalten bei mir Linien. Wie wäre es mit einer einfachen vertikalen Anordnung in Spalte 1: P(1,1), T(1,3), M(1,4), D(1,5), O(1,6), S(1,7), A(1,2)? Diese Positionen sind bei mir alle frei (◌). Prüf bitte, ob sie bei dir auch frei sind!} \\
            }
        }
    }
    & & \\ \\

    \theutterance \stepcounter{utterance}  
    & & & \multicolumn{2}{p{0.3\linewidth}}{
        \cellcolor[rgb]{0.9,0.9,0.9}{
            \makecell[{{p{\linewidth}}}]{
                \texttt{\tiny{[GM$|$GM]}}
                \texttt{SPRICH: Du hast recht mit (4,2)! Aber ich sehe auch bei deinem Vorschlag Probleme {-} (2,4), (3,4) und (4,4) enthalten bei mir Linien. Wie wäre es mit einer einfachen vertikalen Anordnung in Spalte 1: P(1,1), T(1,3), M(1,4), D(1,5), O(1,6), S(1,7), A(1,2)? Diese Positionen sind bei mir alle frei (◌). Prüf bitte, ob sie bei dir auch frei sind!} \\
            }
        }
    }
    & & \\ \\

    \theutterance \stepcounter{utterance}  
    & & \multicolumn{4}{p{0.6\linewidth}}{
        \cellcolor[rgb]{0.9,0.9,0.9}{
            \makecell[{{p{\linewidth}}}]{
                \texttt{\tiny{[GM$\rangle$P2]}}
                \texttt{Deine Nachricht wurde an deinen Mitspieler übermittelt.} \\
\\ 
\\ 
\texttt{Ihr seid gerade bei Runde 3 von maximal 28.} \\
\texttt{Ihr habt gemeinsam 0 von maximal 16 Strafpunkten angesammelt.} \\
\texttt{Dein Mitspieler hat dir diese Nachricht geschickt:} \\
\\ 
\\ 
\texttt{"Du hast recht mit (4,2)! Aber ich sehe auch bei deinem Vorschlag Probleme {-} (2,4), (3,4) und (4,4) enthalten bei mir Linien. Wie wäre es mit einer einfachen vertikalen Anordnung in Spalte 1: P(1,1), T(1,3), M(1,4), D(1,5), O(1,6), S(1,7), A(1,2)? Diese Positionen sind bei mir alle frei (◌). Prüf bitte, ob sie bei dir auch frei sind!"} \\
\\ 
\\ 
\texttt{Wie lautet dein nächster Befehl?} \\
            }
        }
    }
    & \\ \\

    \theutterance \stepcounter{utterance}  
    & & & \multicolumn{4}{p{0.6\linewidth}}{
        \cellcolor[rgb]{0.9,0.9,0.9}{
            \makecell[{{p{\linewidth}}}]{
                \texttt{\tiny{[GM$\langle$P2]}}
                \texttt{SPRICH: Ich prüfe die Spalte 1: P(1,1) ist frei ◌, aber A(1,2) ist bei mir belegt mit T, M(1,4) enthält eine Linie ╟, und S(1,7) ist belegt mit A. Wie wäre es mit Spalte 2? Kannst du prüfen: P(2,1), T(2,2), M(2,3), D(2,4), O(2,5), S(2,6), A(2,7)? Bei mir sind alle diese Positionen frei (◌).} \\
            }
        }
    }
     \\ \\

    \theutterance \stepcounter{utterance}  
    & & & \multicolumn{2}{p{0.3\linewidth}}{
        \cellcolor[rgb]{0.9,0.9,0.9}{
            \makecell[{{p{\linewidth}}}]{
                \texttt{\tiny{[GM$|$GM]}}
                \texttt{SPRICH: Ich prüfe die Spalte 1: P(1,1) ist frei ◌, aber A(1,2) ist bei mir belegt mit T, M(1,4) enthält eine Linie ╟, und S(1,7) ist belegt mit A. Wie wäre es mit Spalte 2? Kannst du prüfen: P(2,1), T(2,2), M(2,3), D(2,4), O(2,5), S(2,6), A(2,7)? Bei mir sind alle diese Positionen frei (◌).} \\
            }
        }
    }
    & & \\ \\

    \theutterance \stepcounter{utterance}  
    & & \multicolumn{4}{p{0.6\linewidth}}{
        \cellcolor[rgb]{0.9,0.9,0.9}{
            \makecell[{{p{\linewidth}}}]{
                \texttt{\tiny{[P1$\langle$GM]}}
                \texttt{Deine Nachricht wurde an deinen Mitspieler übermittelt.} \\
\\ 
\\ 
\texttt{Ihr seid gerade bei Runde 3 von maximal 28.} \\
\texttt{Ihr habt gemeinsam 0 von maximal 16 Strafpunkten angesammelt.} \\
\texttt{Dein Mitspieler hat dir diese Nachricht geschickt:} \\
\\ 
\\ 
\texttt{"Ich prüfe die Spalte 1: P(1,1) ist frei ◌, aber A(1,2) ist bei mir belegt mit T, M(1,4) enthält eine Linie ╟, und S(1,7) ist belegt mit A. Wie wäre es mit Spalte 2? Kannst du prüfen: P(2,1), T(2,2), M(2,3), D(2,4), O(2,5), S(2,6), A(2,7)? Bei mir sind alle diese Positionen frei (◌)."} \\
\\ 
\\ 
\texttt{Wie lautet dein nächster Befehl?} \\
            }
        }
    }
    & \\ \\

    \theutterance \stepcounter{utterance}  
    & \multicolumn{4}{p{0.6\linewidth}}{
        \cellcolor[rgb]{0.9,0.9,0.9}{
            \makecell[{{p{\linewidth}}}]{
                \texttt{\tiny{[P1$\rangle$GM]}}
                \texttt{SPRICH: Ich prüfe Spalte 2: P(2,1) ist frei ◌, aber T(2,2) enthält eine Linie └, M(2,3) ist belegt mit S, und D(2,4) enthält eine Linie –. Die anderen sind frei: O(2,5), S(2,6), A(2,7). Spalte 2 funktioniert also auch nicht vollständig. Lass uns Spalte 7 versuchen: P(7,1), T(7,2), M(7,3), D(7,4), O(7,5), S(7,6), A(7,7). Kannst du diese Positionen bei dir prüfen?} \\
            }
        }
    }
    & & \\ \\

    \theutterance \stepcounter{utterance}  
    & & & \multicolumn{2}{p{0.3\linewidth}}{
        \cellcolor[rgb]{0.9,0.9,0.9}{
            \makecell[{{p{\linewidth}}}]{
                \texttt{\tiny{[GM$|$GM]}}
                \texttt{SPRICH: Ich prüfe Spalte 2: P(2,1) ist frei ◌, aber T(2,2) enthält eine Linie └, M(2,3) ist belegt mit S, und D(2,4) enthält eine Linie –. Die anderen sind frei: O(2,5), S(2,6), A(2,7). Spalte 2 funktioniert also auch nicht vollständig. Lass uns Spalte 7 versuchen: P(7,1), T(7,2), M(7,3), D(7,4), O(7,5), S(7,6), A(7,7). Kannst du diese Positionen bei dir prüfen?} \\
            }
        }
    }
    & & \\ \\

    \theutterance \stepcounter{utterance}  
    & & \multicolumn{4}{p{0.6\linewidth}}{
        \cellcolor[rgb]{0.9,0.9,0.9}{
            \makecell[{{p{\linewidth}}}]{
                \texttt{\tiny{[GM$\rangle$P2]}}
                \texttt{Deine Nachricht wurde an deinen Mitspieler übermittelt.} \\
\\ 
\\ 
\texttt{Ihr seid gerade bei Runde 4 von maximal 28.} \\
\texttt{Ihr habt gemeinsam 0 von maximal 16 Strafpunkten angesammelt.} \\
\texttt{Dein Mitspieler hat dir diese Nachricht geschickt:} \\
\\ 
\\ 
\texttt{"Ich prüfe Spalte 2: P(2,1) ist frei ◌, aber T(2,2) enthält eine Linie └, M(2,3) ist belegt mit S, und D(2,4) enthält eine Linie –. Die anderen sind frei: O(2,5), S(2,6), A(2,7). Spalte 2 funktioniert also auch nicht vollständig. Lass uns Spalte 7 versuchen: P(7,1), T(7,2), M(7,3), D(7,4), O(7,5), S(7,6), A(7,7). Kannst du diese Positionen bei dir prüfen?"} \\
\\ 
\\ 
\texttt{Wie lautet dein nächster Befehl?} \\
            }
        }
    }
    & \\ \\

    \theutterance \stepcounter{utterance}  
    & & & \multicolumn{4}{p{0.6\linewidth}}{
        \cellcolor[rgb]{0.9,0.9,0.9}{
            \makecell[{{p{\linewidth}}}]{
                \texttt{\tiny{[GM$\langle$P2]}}
                \texttt{SPRICH: Ich prüfe Spalte 7: P(7,1) ist bereits mit P belegt, T(7,2) und D(7,4) enthalten Linien (╢). Die anderen Positionen M(7,3), O(7,5), S(7,6), A(7,7) sind frei. Spalte 7 funktioniert also auch nicht vollständig. Lass uns eine andere Strategie versuchen {-} können wir einzelne freie Positionen sammeln? Welche Positionen sind bei dir definitiv frei?} \\
            }
        }
    }
     \\ \\

    \theutterance \stepcounter{utterance}  
    & & & \multicolumn{2}{p{0.3\linewidth}}{
        \cellcolor[rgb]{0.9,0.9,0.9}{
            \makecell[{{p{\linewidth}}}]{
                \texttt{\tiny{[GM$|$GM]}}
                \texttt{SPRICH: Ich prüfe Spalte 7: P(7,1) ist bereits mit P belegt, T(7,2) und D(7,4) enthalten Linien (╢). Die anderen Positionen M(7,3), O(7,5), S(7,6), A(7,7) sind frei. Spalte 7 funktioniert also auch nicht vollständig. Lass uns eine andere Strategie versuchen {-} können wir einzelne freie Positionen sammeln? Welche Positionen sind bei dir definitiv frei?} \\
            }
        }
    }
    & & \\ \\

    \theutterance \stepcounter{utterance}  
    & & \multicolumn{4}{p{0.6\linewidth}}{
        \cellcolor[rgb]{0.9,0.9,0.9}{
            \makecell[{{p{\linewidth}}}]{
                \texttt{\tiny{[P1$\langle$GM]}}
                \texttt{Deine Nachricht wurde an deinen Mitspieler übermittelt.} \\
\\ 
\\ 
\texttt{Ihr seid gerade bei Runde 4 von maximal 28.} \\
\texttt{Ihr habt gemeinsam 0 von maximal 16 Strafpunkten angesammelt.} \\
\texttt{Dein Mitspieler hat dir diese Nachricht geschickt:} \\
\\ 
\\ 
\texttt{"Ich prüfe Spalte 7: P(7,1) ist bereits mit P belegt, T(7,2) und D(7,4) enthalten Linien (╢). Die anderen Positionen M(7,3), O(7,5), S(7,6), A(7,7) sind frei. Spalte 7 funktioniert also auch nicht vollständig. Lass uns eine andere Strategie versuchen {-} können wir einzelne freie Positionen sammeln? Welche Positionen sind bei dir definitiv frei?"} \\
\\ 
\\ 
\texttt{Wie lautet dein nächster Befehl?} \\
            }
        }
    }
    & \\ \\

    \theutterance \stepcounter{utterance}  
    & \multicolumn{4}{p{0.6\linewidth}}{
        \cellcolor[rgb]{0.9,0.9,0.9}{
            \makecell[{{p{\linewidth}}}]{
                \texttt{\tiny{[P1$\rangle$GM]}}
                \texttt{SPRICH: Gute Idee! Hier sind alle definitiv freien Positionen (◌) bei mir: (1,1), (3,1), (5,1), (7,1), (1,3), (5,3), (6,3), (7,3), (1,5), (2,5), (4,5), (5,5), (7,5), (1,6), (2,6), (4,6), (5,6), (1,7), (2,7), (4,7), (5,7), (6,7). Das sind 22 freie Felder. Welche von diesen sind auch bei dir frei? Dann können wir 7 gemeinsame auswählen.} \\
            }
        }
    }
    & & \\ \\

    \theutterance \stepcounter{utterance}  
    & & & \multicolumn{2}{p{0.3\linewidth}}{
        \cellcolor[rgb]{0.9,0.9,0.9}{
            \makecell[{{p{\linewidth}}}]{
                \texttt{\tiny{[GM$|$GM]}}
                \texttt{SPRICH: Gute Idee! Hier sind alle definitiv freien Positionen (◌) bei mir: (1,1), (3,1), (5,1), (7,1), (1,3), (5,3), (6,3), (7,3), (1,5), (2,5), (4,5), (5,5), (7,5), (1,6), (2,6), (4,6), (5,6), (1,7), (2,7), (4,7), (5,7), (6,7). Das sind 22 freie Felder. Welche von diesen sind auch bei dir frei? Dann können wir 7 gemeinsame auswählen.} \\
            }
        }
    }
    & & \\ \\

    \theutterance \stepcounter{utterance}  
    & & \multicolumn{4}{p{0.6\linewidth}}{
        \cellcolor[rgb]{0.9,0.9,0.9}{
            \makecell[{{p{\linewidth}}}]{
                \texttt{\tiny{[GM$\rangle$P2]}}
                \texttt{Deine Nachricht wurde an deinen Mitspieler übermittelt.} \\
\\ 
\\ 
\texttt{Ihr seid gerade bei Runde 5 von maximal 28.} \\
\texttt{Ihr habt gemeinsam 0 von maximal 16 Strafpunkten angesammelt.} \\
\texttt{Dein Mitspieler hat dir diese Nachricht geschickt:} \\
\\ 
\\ 
\texttt{"Gute Idee! Hier sind alle definitiv freien Positionen (◌) bei mir: (1,1), (3,1), (5,1), (7,1), (1,3), (5,3), (6,3), (7,3), (1,5), (2,5), (4,5), (5,5), (7,5), (1,6), (2,6), (4,6), (5,6), (1,7), (2,7), (4,7), (5,7), (6,7). Das sind 22 freie Felder. Welche von diesen sind auch bei dir frei? Dann können wir 7 gemeinsame auswählen."} \\
\\ 
\\ 
\texttt{Wie lautet dein nächster Befehl?} \\
            }
        }
    }
    & \\ \\

    \theutterance \stepcounter{utterance}  
    & & & \multicolumn{4}{p{0.6\linewidth}}{
        \cellcolor[rgb]{0.9,0.9,0.9}{
            \makecell[{{p{\linewidth}}}]{
                \texttt{\tiny{[GM$\langle$P2]}}
                \texttt{SPRICH: Ich prüfe deine Liste. Bei mir sind diese Positionen AUCH frei: (3,1), (5,1), (1,3), (5,3), (6,3), (7,3), (1,5), (4,5), (7,5), (1,6), (2,6), (5,6), (2,7), (4,7), (5,7), (6,7). Das sind 16 gemeinsame freie Felder! Wir brauchen nur 7 davon. Schlage vor: P(3,1), T(5,1), M(1,3), D(5,3), O(6,3), S(7,3), A(1,5). Einverstanden?} \\
            }
        }
    }
     \\ \\

    \theutterance \stepcounter{utterance}  
    & & & \multicolumn{2}{p{0.3\linewidth}}{
        \cellcolor[rgb]{0.9,0.9,0.9}{
            \makecell[{{p{\linewidth}}}]{
                \texttt{\tiny{[GM$|$GM]}}
                \texttt{SPRICH: Ich prüfe deine Liste. Bei mir sind diese Positionen AUCH frei: (3,1), (5,1), (1,3), (5,3), (6,3), (7,3), (1,5), (4,5), (7,5), (1,6), (2,6), (5,6), (2,7), (4,7), (5,7), (6,7). Das sind 16 gemeinsame freie Felder! Wir brauchen nur 7 davon. Schlage vor: P(3,1), T(5,1), M(1,3), D(5,3), O(6,3), S(7,3), A(1,5). Einverstanden?} \\
            }
        }
    }
    & & \\ \\

    \theutterance \stepcounter{utterance}  
    & & \multicolumn{4}{p{0.6\linewidth}}{
        \cellcolor[rgb]{0.9,0.9,0.9}{
            \makecell[{{p{\linewidth}}}]{
                \texttt{\tiny{[P1$\langle$GM]}}
                \texttt{Deine Nachricht wurde an deinen Mitspieler übermittelt.} \\
\\ 
\\ 
\texttt{Ihr seid gerade bei Runde 5 von maximal 28.} \\
\texttt{Ihr habt gemeinsam 0 von maximal 16 Strafpunkten angesammelt.} \\
\texttt{Dein Mitspieler hat dir diese Nachricht geschickt:} \\
\\ 
\\ 
\texttt{"Ich prüfe deine Liste. Bei mir sind diese Positionen AUCH frei: (3,1), (5,1), (1,3), (5,3), (6,3), (7,3), (1,5), (4,5), (7,5), (1,6), (2,6), (5,6), (2,7), (4,7), (5,7), (6,7). Das sind 16 gemeinsame freie Felder! Wir brauchen nur 7 davon. Schlage vor: P(3,1), T(5,1), M(1,3), D(5,3), O(6,3), S(7,3), A(1,5). Einverstanden?"} \\
\\ 
\\ 
\texttt{Wie lautet dein nächster Befehl?} \\
            }
        }
    }
    & \\ \\

    \theutterance \stepcounter{utterance}  
    & \multicolumn{4}{p{0.6\linewidth}}{
        \cellcolor[rgb]{0.9,0.9,0.9}{
            \makecell[{{p{\linewidth}}}]{
                \texttt{\tiny{[P1$\rangle$GM]}}
                \texttt{SPRICH: Perfekt! Alle diese Positionen sind auch bei mir frei. Einverstanden mit: P(3,1), T(5,1), M(1,3), D(5,3), O(6,3), S(7,3), A(1,5). Lass uns anfangen zu bewegen. Ich beginne mit P von (1,2) nach (3,1).} \\
            }
        }
    }
    & & \\ \\

    \theutterance \stepcounter{utterance}  
    & & & \multicolumn{2}{p{0.3\linewidth}}{
        \cellcolor[rgb]{0.9,0.9,0.9}{
            \makecell[{{p{\linewidth}}}]{
                \texttt{\tiny{[GM$|$GM]}}
                \texttt{SPRICH: Perfekt! Alle diese Positionen sind auch bei mir frei. Einverstanden mit: P(3,1), T(5,1), M(1,3), D(5,3), O(6,3), S(7,3), A(1,5). Lass uns anfangen zu bewegen. Ich beginne mit P von (1,2) nach (3,1).} \\
            }
        }
    }
    & & \\ \\

    \theutterance \stepcounter{utterance}  
    & & \multicolumn{4}{p{0.6\linewidth}}{
        \cellcolor[rgb]{0.9,0.9,0.9}{
            \makecell[{{p{\linewidth}}}]{
                \texttt{\tiny{[GM$\rangle$P2]}}
                \texttt{Deine Nachricht wurde an deinen Mitspieler übermittelt.} \\
\\ 
\\ 
\texttt{Ihr seid gerade bei Runde 6 von maximal 28.} \\
\texttt{Ihr habt gemeinsam 0 von maximal 16 Strafpunkten angesammelt.} \\
\texttt{Dein Mitspieler hat dir diese Nachricht geschickt:} \\
\\ 
\\ 
\texttt{"Perfekt! Alle diese Positionen sind auch bei mir frei. Einverstanden mit: P(3,1), T(5,1), M(1,3), D(5,3), O(6,3), S(7,3), A(1,5). Lass uns anfangen zu bewegen. Ich beginne mit P von (1,2) nach (3,1)."} \\
\\ 
\\ 
\texttt{Wie lautet dein nächster Befehl?} \\
            }
        }
    }
    & \\ \\

    \theutterance \stepcounter{utterance}  
    & & & \multicolumn{4}{p{0.6\linewidth}}{
        \cellcolor[rgb]{0.9,0.9,0.9}{
            \makecell[{{p{\linewidth}}}]{
                \texttt{\tiny{[GM$\langle$P2]}}
                \texttt{VERSCHIEBE: T, (5, 1)} \\
            }
        }
    }
     \\ \\

    \theutterance \stepcounter{utterance}  
    & & & \multicolumn{2}{p{0.3\linewidth}}{
        \cellcolor[rgb]{0.9,0.9,0.9}{
            \makecell[{{p{\linewidth}}}]{
                \texttt{\tiny{[GM$|$GM]}}
                \texttt{VERSCHIEBE: T, (5, 1)} \\
            }
        }
    }
    & & \\ \\

    \theutterance \stepcounter{utterance}  
    & & & \multicolumn{2}{p{0.3\linewidth}}{
        \cellcolor[rgb]{0.9,0.9,0.9}{
            \makecell[{{p{\linewidth}}}]{
                \texttt{\tiny{[GM$|$GM]}}
                \texttt{'T' erfolgreich nach (5,1) verschoben. Dein Raster sieht jetzt so aus:} \\
\texttt{```} \\
\texttt{ 1234567} \\
\texttt{╔═╤═╤═╤═╗} \\
\texttt{║◌│◌│T│P║ 1} \\
\texttt{║◌└─┼─┴─╢ 2} \\
\texttt{║◌◌◌│M◌◌║ 3} \\
\texttt{╟──┬┴───╢ 4} \\
\texttt{║◌D│◌◌O◌║ 5} \\
\texttt{║◌◌│◌S◌◌║ 6} \\
\texttt{║A◌│◌◌◌◌║ 7} \\
\texttt{╚══╧════╝} \\
\texttt{```} \\
            }
        }
    }
    & & \\ \\

    \theutterance \stepcounter{utterance}  
    & & \multicolumn{4}{p{0.6\linewidth}}{
        \cellcolor[rgb]{0.9,0.9,0.9}{
            \makecell[{{p{\linewidth}}}]{
                \texttt{\tiny{[P1$\langle$GM]}}
                \texttt{Deine Nachricht wurde an deinen Mitspieler übermittelt.} \\
\\ 
\\ 
\texttt{Ihr seid gerade bei Runde 6 von maximal 28.} \\
\texttt{Ihr habt gemeinsam 0 von maximal 16 Strafpunkten angesammelt.} \\
\texttt{Dein Mitspieler hat ein Objekt auf seinem Raster bewegt.} \\
\\ 
\\ 
\texttt{Wie lautet dein nächster Befehl?} \\
            }
        }
    }
    & \\ \\

    \theutterance \stepcounter{utterance}  
    & \multicolumn{4}{p{0.6\linewidth}}{
        \cellcolor[rgb]{0.9,0.9,0.9}{
            \makecell[{{p{\linewidth}}}]{
                \texttt{\tiny{[P1$\rangle$GM]}}
                \texttt{VERSCHIEBE: P, (3, 1)} \\
            }
        }
    }
    & & \\ \\

    \theutterance \stepcounter{utterance}  
    & & & \multicolumn{2}{p{0.3\linewidth}}{
        \cellcolor[rgb]{0.9,0.9,0.9}{
            \makecell[{{p{\linewidth}}}]{
                \texttt{\tiny{[GM$|$GM]}}
                \texttt{VERSCHIEBE: P, (3, 1)} \\
            }
        }
    }
    & & \\ \\

    \theutterance \stepcounter{utterance}  
    & & & \multicolumn{2}{p{0.3\linewidth}}{
        \cellcolor[rgb]{0.9,0.9,0.9}{
            \makecell[{{p{\linewidth}}}]{
                \texttt{\tiny{[GM$|$GM]}}
                \texttt{'P' erfolgreich nach (3,1) verschoben. Dein Raster sieht jetzt so aus:} \\
\texttt{```} \\
\texttt{ 1234567} \\
\texttt{╔═╤═╤═╤═╗} \\
\texttt{║◌│P│◌│◌║ 1} \\
\texttt{║◌└─┼─┴─╢ 2} \\
\texttt{║◌SA│◌◌◌║ 3} \\
\texttt{╟──┬┴───╢ 4} \\
\texttt{║◌◌│◌◌O◌║ 5} \\
\texttt{║◌◌│◌◌MD║ 6} \\
\texttt{║◌◌│◌◌◌T║ 7} \\
\texttt{╚══╧════╝} \\
\texttt{```} \\
            }
        }
    }
    & & \\ \\

    \theutterance \stepcounter{utterance}  
    & & \multicolumn{4}{p{0.6\linewidth}}{
        \cellcolor[rgb]{0.9,0.9,0.9}{
            \makecell[{{p{\linewidth}}}]{
                \texttt{\tiny{[GM$\rangle$P2]}}
                \texttt{'T' erfolgreich nach (5,1) verschoben. Dein Raster sieht jetzt so aus:} \\
\texttt{```} \\
\texttt{ 1234567} \\
\texttt{╔═╤═╤═╤═╗} \\
\texttt{║◌│◌│T│P║ 1} \\
\texttt{║◌└─┼─┴─╢ 2} \\
\texttt{║◌◌◌│M◌◌║ 3} \\
\texttt{╟──┬┴───╢ 4} \\
\texttt{║◌D│◌◌O◌║ 5} \\
\texttt{║◌◌│◌S◌◌║ 6} \\
\texttt{║A◌│◌◌◌◌║ 7} \\
\texttt{╚══╧════╝} \\
\texttt{```} \\
\texttt{Ihr seid gerade bei Runde 7 von maximal 28.} \\
\texttt{Ihr habt gemeinsam 0 von maximal 16 Strafpunkten angesammelt.} \\
\texttt{Dein Mitspieler hat ein Objekt auf seinem Raster bewegt.} \\
\\ 
\\ 
\texttt{Wie lautet dein nächster Befehl?} \\
            }
        }
    }
    & \\ \\

    \theutterance \stepcounter{utterance}  
    & & & \multicolumn{4}{p{0.6\linewidth}}{
        \cellcolor[rgb]{0.9,0.9,0.9}{
            \makecell[{{p{\linewidth}}}]{
                \texttt{\tiny{[GM$\langle$P2]}}
                \texttt{VERSCHIEBE: P, (3, 1)} \\
            }
        }
    }
     \\ \\

    \theutterance \stepcounter{utterance}  
    & & & \multicolumn{2}{p{0.3\linewidth}}{
        \cellcolor[rgb]{0.9,0.9,0.9}{
            \makecell[{{p{\linewidth}}}]{
                \texttt{\tiny{[GM$|$GM]}}
                \texttt{VERSCHIEBE: P, (3, 1)} \\
            }
        }
    }
    & & \\ \\

    \theutterance \stepcounter{utterance}  
    & & & \multicolumn{2}{p{0.3\linewidth}}{
        \cellcolor[rgb]{0.9,0.9,0.9}{
            \makecell[{{p{\linewidth}}}]{
                \texttt{\tiny{[GM$|$GM]}}
                \texttt{'P' erfolgreich nach (3,1) verschoben. Dein Raster sieht jetzt so aus:} \\
\texttt{```} \\
\texttt{ 1234567} \\
\texttt{╔═╤═╤═╤═╗} \\
\texttt{║◌│P│T│◌║ 1} \\
\texttt{║◌└─┼─┴─╢ 2} \\
\texttt{║◌◌◌│M◌◌║ 3} \\
\texttt{╟──┬┴───╢ 4} \\
\texttt{║◌D│◌◌O◌║ 5} \\
\texttt{║◌◌│◌S◌◌║ 6} \\
\texttt{║A◌│◌◌◌◌║ 7} \\
\texttt{╚══╧════╝} \\
\texttt{```} \\
            }
        }
    }
    & & \\ \\

    \theutterance \stepcounter{utterance}  
    & & \multicolumn{4}{p{0.6\linewidth}}{
        \cellcolor[rgb]{0.9,0.9,0.9}{
            \makecell[{{p{\linewidth}}}]{
                \texttt{\tiny{[P1$\langle$GM]}}
                \texttt{'P' erfolgreich nach (3,1) verschoben. Dein Raster sieht jetzt so aus:} \\
\texttt{```} \\
\texttt{ 1234567} \\
\texttt{╔═╤═╤═╤═╗} \\
\texttt{║◌│P│◌│◌║ 1} \\
\texttt{║◌└─┼─┴─╢ 2} \\
\texttt{║◌SA│◌◌◌║ 3} \\
\texttt{╟──┬┴───╢ 4} \\
\texttt{║◌◌│◌◌O◌║ 5} \\
\texttt{║◌◌│◌◌MD║ 6} \\
\texttt{║◌◌│◌◌◌T║ 7} \\
\texttt{╚══╧════╝} \\
\texttt{```} \\
\texttt{Ihr seid gerade bei Runde 7 von maximal 28.} \\
\texttt{Ihr habt gemeinsam 0 von maximal 16 Strafpunkten angesammelt.} \\
\texttt{Dein Mitspieler hat ein Objekt auf seinem Raster bewegt.} \\
\\ 
\\ 
\texttt{Wie lautet dein nächster Befehl?} \\
            }
        }
    }
    & \\ \\

    \theutterance \stepcounter{utterance}  
    & \multicolumn{4}{p{0.6\linewidth}}{
        \cellcolor[rgb]{0.9,0.9,0.9}{
            \makecell[{{p{\linewidth}}}]{
                \texttt{\tiny{[P1$\rangle$GM]}}
                \texttt{VERSCHIEBE: S, (7, 3)} \\
            }
        }
    }
    & & \\ \\

    \theutterance \stepcounter{utterance}  
    & & & \multicolumn{2}{p{0.3\linewidth}}{
        \cellcolor[rgb]{0.9,0.9,0.9}{
            \makecell[{{p{\linewidth}}}]{
                \texttt{\tiny{[GM$|$GM]}}
                \texttt{VERSCHIEBE: S, (7, 3)} \\
            }
        }
    }
    & & \\ \\

    \theutterance \stepcounter{utterance}  
    & & & \multicolumn{2}{p{0.3\linewidth}}{
        \cellcolor[rgb]{0.9,0.9,0.9}{
            \makecell[{{p{\linewidth}}}]{
                \texttt{\tiny{[GM$|$GM]}}
                \texttt{'S' erfolgreich nach (7,3) verschoben. Dein Raster sieht jetzt so aus:} \\
\texttt{```} \\
\texttt{ 1234567} \\
\texttt{╔═╤═╤═╤═╗} \\
\texttt{║◌│P│◌│◌║ 1} \\
\texttt{║◌└─┼─┴─╢ 2} \\
\texttt{║◌◌A│◌◌S║ 3} \\
\texttt{╟──┬┴───╢ 4} \\
\texttt{║◌◌│◌◌O◌║ 5} \\
\texttt{║◌◌│◌◌MD║ 6} \\
\texttt{║◌◌│◌◌◌T║ 7} \\
\texttt{╚══╧════╝} \\
\texttt{```} \\
            }
        }
    }
    & & \\ \\

    \theutterance \stepcounter{utterance}  
    & & \multicolumn{4}{p{0.6\linewidth}}{
        \cellcolor[rgb]{0.9,0.9,0.9}{
            \makecell[{{p{\linewidth}}}]{
                \texttt{\tiny{[GM$\rangle$P2]}}
                \texttt{'P' erfolgreich nach (3,1) verschoben. Dein Raster sieht jetzt so aus:} \\
\texttt{```} \\
\texttt{ 1234567} \\
\texttt{╔═╤═╤═╤═╗} \\
\texttt{║◌│P│T│◌║ 1} \\
\texttt{║◌└─┼─┴─╢ 2} \\
\texttt{║◌◌◌│M◌◌║ 3} \\
\texttt{╟──┬┴───╢ 4} \\
\texttt{║◌D│◌◌O◌║ 5} \\
\texttt{║◌◌│◌S◌◌║ 6} \\
\texttt{║A◌│◌◌◌◌║ 7} \\
\texttt{╚══╧════╝} \\
\texttt{```} \\
\texttt{Ihr seid gerade bei Runde 8 von maximal 28.} \\
\texttt{Ihr habt gemeinsam 0 von maximal 16 Strafpunkten angesammelt.} \\
\texttt{Dein Mitspieler hat ein Objekt auf seinem Raster bewegt.} \\
\\ 
\\ 
\texttt{Wie lautet dein nächster Befehl?} \\
            }
        }
    }
    & \\ \\

    \theutterance \stepcounter{utterance}  
    & & & \multicolumn{4}{p{0.6\linewidth}}{
        \cellcolor[rgb]{0.9,0.9,0.9}{
            \makecell[{{p{\linewidth}}}]{
                \texttt{\tiny{[GM$\langle$P2]}}
                \texttt{VERSCHIEBE: M, (1, 3)} \\
            }
        }
    }
     \\ \\

    \theutterance \stepcounter{utterance}  
    & & & \multicolumn{2}{p{0.3\linewidth}}{
        \cellcolor[rgb]{0.9,0.9,0.9}{
            \makecell[{{p{\linewidth}}}]{
                \texttt{\tiny{[GM$|$GM]}}
                \texttt{VERSCHIEBE: M, (1, 3)} \\
            }
        }
    }
    & & \\ \\

    \theutterance \stepcounter{utterance}  
    & & & \multicolumn{2}{p{0.3\linewidth}}{
        \cellcolor[rgb]{0.9,0.9,0.9}{
            \makecell[{{p{\linewidth}}}]{
                \texttt{\tiny{[GM$|$GM]}}
                \texttt{'M' erfolgreich nach (1,3) verschoben. Dein Raster sieht jetzt so aus:} \\
\texttt{```} \\
\texttt{ 1234567} \\
\texttt{╔═╤═╤═╤═╗} \\
\texttt{║◌│P│T│◌║ 1} \\
\texttt{║◌└─┼─┴─╢ 2} \\
\texttt{║M◌◌│◌◌◌║ 3} \\
\texttt{╟──┬┴───╢ 4} \\
\texttt{║◌D│◌◌O◌║ 5} \\
\texttt{║◌◌│◌S◌◌║ 6} \\
\texttt{║A◌│◌◌◌◌║ 7} \\
\texttt{╚══╧════╝} \\
\texttt{```} \\
            }
        }
    }
    & & \\ \\

    \theutterance \stepcounter{utterance}  
    & & \multicolumn{4}{p{0.6\linewidth}}{
        \cellcolor[rgb]{0.9,0.9,0.9}{
            \makecell[{{p{\linewidth}}}]{
                \texttt{\tiny{[P1$\langle$GM]}}
                \texttt{'S' erfolgreich nach (7,3) verschoben. Dein Raster sieht jetzt so aus:} \\
\texttt{```} \\
\texttt{ 1234567} \\
\texttt{╔═╤═╤═╤═╗} \\
\texttt{║◌│P│◌│◌║ 1} \\
\texttt{║◌└─┼─┴─╢ 2} \\
\texttt{║◌◌A│◌◌S║ 3} \\
\texttt{╟──┬┴───╢ 4} \\
\texttt{║◌◌│◌◌O◌║ 5} \\
\texttt{║◌◌│◌◌MD║ 6} \\
\texttt{║◌◌│◌◌◌T║ 7} \\
\texttt{╚══╧════╝} \\
\texttt{```} \\
\texttt{Ihr seid gerade bei Runde 8 von maximal 28.} \\
\texttt{Ihr habt gemeinsam 0 von maximal 16 Strafpunkten angesammelt.} \\
\texttt{Dein Mitspieler hat ein Objekt auf seinem Raster bewegt.} \\
\\ 
\\ 
\texttt{Wie lautet dein nächster Befehl?} \\
            }
        }
    }
    & \\ \\

    \theutterance \stepcounter{utterance}  
    & \multicolumn{4}{p{0.6\linewidth}}{
        \cellcolor[rgb]{0.9,0.9,0.9}{
            \makecell[{{p{\linewidth}}}]{
                \texttt{\tiny{[P1$\rangle$GM]}}
                \texttt{VERSCHIEBE: A, (1, 5)} \\
            }
        }
    }
    & & \\ \\

    \theutterance \stepcounter{utterance}  
    & & & \multicolumn{2}{p{0.3\linewidth}}{
        \cellcolor[rgb]{0.9,0.9,0.9}{
            \makecell[{{p{\linewidth}}}]{
                \texttt{\tiny{[GM$|$GM]}}
                \texttt{VERSCHIEBE: A, (1, 5)} \\
            }
        }
    }
    & & \\ \\

    \theutterance \stepcounter{utterance}  
    & & & \multicolumn{2}{p{0.3\linewidth}}{
        \cellcolor[rgb]{0.9,0.9,0.9}{
            \makecell[{{p{\linewidth}}}]{
                \texttt{\tiny{[GM$|$GM]}}
                \texttt{'A' erfolgreich nach (1,5) verschoben. Dein Raster sieht jetzt so aus:} \\
\texttt{```} \\
\texttt{ 1234567} \\
\texttt{╔═╤═╤═╤═╗} \\
\texttt{║◌│P│◌│◌║ 1} \\
\texttt{║◌└─┼─┴─╢ 2} \\
\texttt{║◌◌◌│◌◌S║ 3} \\
\texttt{╟──┬┴───╢ 4} \\
\texttt{║A◌│◌◌O◌║ 5} \\
\texttt{║◌◌│◌◌MD║ 6} \\
\texttt{║◌◌│◌◌◌T║ 7} \\
\texttt{╚══╧════╝} \\
\texttt{```} \\
            }
        }
    }
    & & \\ \\

    \theutterance \stepcounter{utterance}  
    & & \multicolumn{4}{p{0.6\linewidth}}{
        \cellcolor[rgb]{0.9,0.9,0.9}{
            \makecell[{{p{\linewidth}}}]{
                \texttt{\tiny{[GM$\rangle$P2]}}
                \texttt{'M' erfolgreich nach (1,3) verschoben. Dein Raster sieht jetzt so aus:} \\
\texttt{```} \\
\texttt{ 1234567} \\
\texttt{╔═╤═╤═╤═╗} \\
\texttt{║◌│P│T│◌║ 1} \\
\texttt{║◌└─┼─┴─╢ 2} \\
\texttt{║M◌◌│◌◌◌║ 3} \\
\texttt{╟──┬┴───╢ 4} \\
\texttt{║◌D│◌◌O◌║ 5} \\
\texttt{║◌◌│◌S◌◌║ 6} \\
\texttt{║A◌│◌◌◌◌║ 7} \\
\texttt{╚══╧════╝} \\
\texttt{```} \\
\texttt{Ihr seid gerade bei Runde 9 von maximal 28.} \\
\texttt{Ihr habt gemeinsam 0 von maximal 16 Strafpunkten angesammelt.} \\
\texttt{Dein Mitspieler hat ein Objekt auf seinem Raster bewegt.} \\
\\ 
\\ 
\texttt{Wie lautet dein nächster Befehl?} \\
            }
        }
    }
    & \\ \\

    \theutterance \stepcounter{utterance}  
    & & & \multicolumn{4}{p{0.6\linewidth}}{
        \cellcolor[rgb]{0.9,0.9,0.9}{
            \makecell[{{p{\linewidth}}}]{
                \texttt{\tiny{[GM$\langle$P2]}}
                \texttt{VERSCHIEBE: D, (5, 3)} \\
            }
        }
    }
     \\ \\

    \theutterance \stepcounter{utterance}  
    & & & \multicolumn{2}{p{0.3\linewidth}}{
        \cellcolor[rgb]{0.9,0.9,0.9}{
            \makecell[{{p{\linewidth}}}]{
                \texttt{\tiny{[GM$|$GM]}}
                \texttt{VERSCHIEBE: D, (5, 3)} \\
            }
        }
    }
    & & \\ \\

    \theutterance \stepcounter{utterance}  
    & & & \multicolumn{2}{p{0.3\linewidth}}{
        \cellcolor[rgb]{0.9,0.9,0.9}{
            \makecell[{{p{\linewidth}}}]{
                \texttt{\tiny{[GM$|$GM]}}
                \texttt{'D' erfolgreich nach (5,3) verschoben. Dein Raster sieht jetzt so aus:} \\
\texttt{```} \\
\texttt{ 1234567} \\
\texttt{╔═╤═╤═╤═╗} \\
\texttt{║◌│P│T│◌║ 1} \\
\texttt{║◌└─┼─┴─╢ 2} \\
\texttt{║M◌◌│D◌◌║ 3} \\
\texttt{╟──┬┴───╢ 4} \\
\texttt{║◌◌│◌◌O◌║ 5} \\
\texttt{║◌◌│◌S◌◌║ 6} \\
\texttt{║A◌│◌◌◌◌║ 7} \\
\texttt{╚══╧════╝} \\
\texttt{```} \\
            }
        }
    }
    & & \\ \\

    \theutterance \stepcounter{utterance}  
    & & \multicolumn{4}{p{0.6\linewidth}}{
        \cellcolor[rgb]{0.9,0.9,0.9}{
            \makecell[{{p{\linewidth}}}]{
                \texttt{\tiny{[P1$\langle$GM]}}
                \texttt{'A' erfolgreich nach (1,5) verschoben. Dein Raster sieht jetzt so aus:} \\
\texttt{```} \\
\texttt{ 1234567} \\
\texttt{╔═╤═╤═╤═╗} \\
\texttt{║◌│P│◌│◌║ 1} \\
\texttt{║◌└─┼─┴─╢ 2} \\
\texttt{║◌◌◌│◌◌S║ 3} \\
\texttt{╟──┬┴───╢ 4} \\
\texttt{║A◌│◌◌O◌║ 5} \\
\texttt{║◌◌│◌◌MD║ 6} \\
\texttt{║◌◌│◌◌◌T║ 7} \\
\texttt{╚══╧════╝} \\
\texttt{```} \\
\texttt{Ihr seid gerade bei Runde 9 von maximal 28.} \\
\texttt{Ihr habt gemeinsam 0 von maximal 16 Strafpunkten angesammelt.} \\
\texttt{Dein Mitspieler hat ein Objekt auf seinem Raster bewegt.} \\
\\ 
\\ 
\texttt{Wie lautet dein nächster Befehl?} \\
            }
        }
    }
    & \\ \\

    \theutterance \stepcounter{utterance}  
    & \multicolumn{4}{p{0.6\linewidth}}{
        \cellcolor[rgb]{0.9,0.9,0.9}{
            \makecell[{{p{\linewidth}}}]{
                \texttt{\tiny{[P1$\rangle$GM]}}
                \texttt{VERSCHIEBE: T, (5, 1)} \\
            }
        }
    }
    & & \\ \\

    \theutterance \stepcounter{utterance}  
    & & & \multicolumn{2}{p{0.3\linewidth}}{
        \cellcolor[rgb]{0.9,0.9,0.9}{
            \makecell[{{p{\linewidth}}}]{
                \texttt{\tiny{[GM$|$GM]}}
                \texttt{VERSCHIEBE: T, (5, 1)} \\
            }
        }
    }
    & & \\ \\

    \theutterance \stepcounter{utterance}  
    & & & \multicolumn{2}{p{0.3\linewidth}}{
        \cellcolor[rgb]{0.9,0.9,0.9}{
            \makecell[{{p{\linewidth}}}]{
                \texttt{\tiny{[GM$|$GM]}}
                \texttt{'T' erfolgreich nach (5,1) verschoben. Dein Raster sieht jetzt so aus:} \\
\texttt{```} \\
\texttt{ 1234567} \\
\texttt{╔═╤═╤═╤═╗} \\
\texttt{║◌│P│T│◌║ 1} \\
\texttt{║◌└─┼─┴─╢ 2} \\
\texttt{║◌◌◌│◌◌S║ 3} \\
\texttt{╟──┬┴───╢ 4} \\
\texttt{║A◌│◌◌O◌║ 5} \\
\texttt{║◌◌│◌◌MD║ 6} \\
\texttt{║◌◌│◌◌◌◌║ 7} \\
\texttt{╚══╧════╝} \\
\texttt{```} \\
            }
        }
    }
    & & \\ \\

    \theutterance \stepcounter{utterance}  
    & & \multicolumn{4}{p{0.6\linewidth}}{
        \cellcolor[rgb]{0.9,0.9,0.9}{
            \makecell[{{p{\linewidth}}}]{
                \texttt{\tiny{[GM$\rangle$P2]}}
                \texttt{'D' erfolgreich nach (5,3) verschoben. Dein Raster sieht jetzt so aus:} \\
\texttt{```} \\
\texttt{ 1234567} \\
\texttt{╔═╤═╤═╤═╗} \\
\texttt{║◌│P│T│◌║ 1} \\
\texttt{║◌└─┼─┴─╢ 2} \\
\texttt{║M◌◌│D◌◌║ 3} \\
\texttt{╟──┬┴───╢ 4} \\
\texttt{║◌◌│◌◌O◌║ 5} \\
\texttt{║◌◌│◌S◌◌║ 6} \\
\texttt{║A◌│◌◌◌◌║ 7} \\
\texttt{╚══╧════╝} \\
\texttt{```} \\
\texttt{Ihr seid gerade bei Runde 10 von maximal 28.} \\
\texttt{Ihr habt gemeinsam 0 von maximal 16 Strafpunkten angesammelt.} \\
\texttt{Dein Mitspieler hat ein Objekt auf seinem Raster bewegt.} \\
\\ 
\\ 
\texttt{Wie lautet dein nächster Befehl?} \\
            }
        }
    }
    & \\ \\

    \theutterance \stepcounter{utterance}  
    & & & \multicolumn{4}{p{0.6\linewidth}}{
        \cellcolor[rgb]{0.9,0.9,0.9}{
            \makecell[{{p{\linewidth}}}]{
                \texttt{\tiny{[GM$\langle$P2]}}
                \texttt{VERSCHIEBE: O, (6, 3)} \\
            }
        }
    }
     \\ \\

    \theutterance \stepcounter{utterance}  
    & & & \multicolumn{2}{p{0.3\linewidth}}{
        \cellcolor[rgb]{0.9,0.9,0.9}{
            \makecell[{{p{\linewidth}}}]{
                \texttt{\tiny{[GM$|$GM]}}
                \texttt{VERSCHIEBE: O, (6, 3)} \\
            }
        }
    }
    & & \\ \\

    \theutterance \stepcounter{utterance}  
    & & & \multicolumn{2}{p{0.3\linewidth}}{
        \cellcolor[rgb]{0.9,0.9,0.9}{
            \makecell[{{p{\linewidth}}}]{
                \texttt{\tiny{[GM$|$GM]}}
                \texttt{'O' erfolgreich nach (6,3) verschoben. Dein Raster sieht jetzt so aus:} \\
\texttt{```} \\
\texttt{ 1234567} \\
\texttt{╔═╤═╤═╤═╗} \\
\texttt{║◌│P│T│◌║ 1} \\
\texttt{║◌└─┼─┴─╢ 2} \\
\texttt{║M◌◌│DO◌║ 3} \\
\texttt{╟──┬┴───╢ 4} \\
\texttt{║◌◌│◌◌◌◌║ 5} \\
\texttt{║◌◌│◌S◌◌║ 6} \\
\texttt{║A◌│◌◌◌◌║ 7} \\
\texttt{╚══╧════╝} \\
\texttt{```} \\
            }
        }
    }
    & & \\ \\

    \theutterance \stepcounter{utterance}  
    & & \multicolumn{4}{p{0.6\linewidth}}{
        \cellcolor[rgb]{0.9,0.9,0.9}{
            \makecell[{{p{\linewidth}}}]{
                \texttt{\tiny{[P1$\langle$GM]}}
                \texttt{'T' erfolgreich nach (5,1) verschoben. Dein Raster sieht jetzt so aus:} \\
\texttt{```} \\
\texttt{ 1234567} \\
\texttt{╔═╤═╤═╤═╗} \\
\texttt{║◌│P│T│◌║ 1} \\
\texttt{║◌└─┼─┴─╢ 2} \\
\texttt{║◌◌◌│◌◌S║ 3} \\
\texttt{╟──┬┴───╢ 4} \\
\texttt{║A◌│◌◌O◌║ 5} \\
\texttt{║◌◌│◌◌MD║ 6} \\
\texttt{║◌◌│◌◌◌◌║ 7} \\
\texttt{╚══╧════╝} \\
\texttt{```} \\
\texttt{Ihr seid gerade bei Runde 10 von maximal 28.} \\
\texttt{Ihr habt gemeinsam 0 von maximal 16 Strafpunkten angesammelt.} \\
\texttt{Dein Mitspieler hat ein Objekt auf seinem Raster bewegt.} \\
\\ 
\\ 
\texttt{Wie lautet dein nächster Befehl?} \\
            }
        }
    }
    & \\ \\

    \theutterance \stepcounter{utterance}  
    & \multicolumn{4}{p{0.6\linewidth}}{
        \cellcolor[rgb]{0.9,0.9,0.9}{
            \makecell[{{p{\linewidth}}}]{
                \texttt{\tiny{[P1$\rangle$GM]}}
                \texttt{VERSCHIEBE: O, (6, 3)} \\
            }
        }
    }
    & & \\ \\

    \theutterance \stepcounter{utterance}  
    & & & \multicolumn{2}{p{0.3\linewidth}}{
        \cellcolor[rgb]{0.9,0.9,0.9}{
            \makecell[{{p{\linewidth}}}]{
                \texttt{\tiny{[GM$|$GM]}}
                \texttt{VERSCHIEBE: O, (6, 3)} \\
            }
        }
    }
    & & \\ \\

    \theutterance \stepcounter{utterance}  
    & & & \multicolumn{2}{p{0.3\linewidth}}{
        \cellcolor[rgb]{0.9,0.9,0.9}{
            \makecell[{{p{\linewidth}}}]{
                \texttt{\tiny{[GM$|$GM]}}
                \texttt{'O' erfolgreich nach (6,3) verschoben. Dein Raster sieht jetzt so aus:} \\
\texttt{```} \\
\texttt{ 1234567} \\
\texttt{╔═╤═╤═╤═╗} \\
\texttt{║◌│P│T│◌║ 1} \\
\texttt{║◌└─┼─┴─╢ 2} \\
\texttt{║◌◌◌│◌OS║ 3} \\
\texttt{╟──┬┴───╢ 4} \\
\texttt{║A◌│◌◌◌◌║ 5} \\
\texttt{║◌◌│◌◌MD║ 6} \\
\texttt{║◌◌│◌◌◌◌║ 7} \\
\texttt{╚══╧════╝} \\
\texttt{```} \\
            }
        }
    }
    & & \\ \\

    \theutterance \stepcounter{utterance}  
    & & \multicolumn{4}{p{0.6\linewidth}}{
        \cellcolor[rgb]{0.9,0.9,0.9}{
            \makecell[{{p{\linewidth}}}]{
                \texttt{\tiny{[GM$\rangle$P2]}}
                \texttt{'O' erfolgreich nach (6,3) verschoben. Dein Raster sieht jetzt so aus:} \\
\texttt{```} \\
\texttt{ 1234567} \\
\texttt{╔═╤═╤═╤═╗} \\
\texttt{║◌│P│T│◌║ 1} \\
\texttt{║◌└─┼─┴─╢ 2} \\
\texttt{║M◌◌│DO◌║ 3} \\
\texttt{╟──┬┴───╢ 4} \\
\texttt{║◌◌│◌◌◌◌║ 5} \\
\texttt{║◌◌│◌S◌◌║ 6} \\
\texttt{║A◌│◌◌◌◌║ 7} \\
\texttt{╚══╧════╝} \\
\texttt{```} \\
\texttt{Ihr seid gerade bei Runde 11 von maximal 28.} \\
\texttt{Ihr habt gemeinsam 0 von maximal 16 Strafpunkten angesammelt.} \\
\texttt{Dein Mitspieler hat ein Objekt auf seinem Raster bewegt.} \\
\\ 
\\ 
\texttt{Wie lautet dein nächster Befehl?} \\
            }
        }
    }
    & \\ \\

    \theutterance \stepcounter{utterance}  
    & & & \multicolumn{4}{p{0.6\linewidth}}{
        \cellcolor[rgb]{0.9,0.9,0.9}{
            \makecell[{{p{\linewidth}}}]{
                \texttt{\tiny{[GM$\langle$P2]}}
                \texttt{VERSCHIEBE: S, (7, 3)} \\
            }
        }
    }
     \\ \\

    \theutterance \stepcounter{utterance}  
    & & & \multicolumn{2}{p{0.3\linewidth}}{
        \cellcolor[rgb]{0.9,0.9,0.9}{
            \makecell[{{p{\linewidth}}}]{
                \texttt{\tiny{[GM$|$GM]}}
                \texttt{VERSCHIEBE: S, (7, 3)} \\
            }
        }
    }
    & & \\ \\

    \theutterance \stepcounter{utterance}  
    & & & \multicolumn{2}{p{0.3\linewidth}}{
        \cellcolor[rgb]{0.9,0.9,0.9}{
            \makecell[{{p{\linewidth}}}]{
                \texttt{\tiny{[GM$|$GM]}}
                \texttt{'S' erfolgreich nach (7,3) verschoben. Dein Raster sieht jetzt so aus:} \\
\texttt{```} \\
\texttt{ 1234567} \\
\texttt{╔═╤═╤═╤═╗} \\
\texttt{║◌│P│T│◌║ 1} \\
\texttt{║◌└─┼─┴─╢ 2} \\
\texttt{║M◌◌│DOS║ 3} \\
\texttt{╟──┬┴───╢ 4} \\
\texttt{║◌◌│◌◌◌◌║ 5} \\
\texttt{║◌◌│◌◌◌◌║ 6} \\
\texttt{║A◌│◌◌◌◌║ 7} \\
\texttt{╚══╧════╝} \\
\texttt{```} \\
            }
        }
    }
    & & \\ \\

    \theutterance \stepcounter{utterance}  
    & & \multicolumn{4}{p{0.6\linewidth}}{
        \cellcolor[rgb]{0.9,0.9,0.9}{
            \makecell[{{p{\linewidth}}}]{
                \texttt{\tiny{[P1$\langle$GM]}}
                \texttt{'O' erfolgreich nach (6,3) verschoben. Dein Raster sieht jetzt so aus:} \\
\texttt{```} \\
\texttt{ 1234567} \\
\texttt{╔═╤═╤═╤═╗} \\
\texttt{║◌│P│T│◌║ 1} \\
\texttt{║◌└─┼─┴─╢ 2} \\
\texttt{║◌◌◌│◌OS║ 3} \\
\texttt{╟──┬┴───╢ 4} \\
\texttt{║A◌│◌◌◌◌║ 5} \\
\texttt{║◌◌│◌◌MD║ 6} \\
\texttt{║◌◌│◌◌◌◌║ 7} \\
\texttt{╚══╧════╝} \\
\texttt{```} \\
\texttt{Ihr seid gerade bei Runde 11 von maximal 28.} \\
\texttt{Ihr habt gemeinsam 0 von maximal 16 Strafpunkten angesammelt.} \\
\texttt{Dein Mitspieler hat ein Objekt auf seinem Raster bewegt.} \\
\\ 
\\ 
\texttt{Wie lautet dein nächster Befehl?} \\
            }
        }
    }
    & \\ \\

    \theutterance \stepcounter{utterance}  
    & \multicolumn{4}{p{0.6\linewidth}}{
        \cellcolor[rgb]{0.9,0.9,0.9}{
            \makecell[{{p{\linewidth}}}]{
                \texttt{\tiny{[P1$\rangle$GM]}}
                \texttt{VERSCHIEBE: D, (5, 3)} \\
            }
        }
    }
    & & \\ \\

    \theutterance \stepcounter{utterance}  
    & & & \multicolumn{2}{p{0.3\linewidth}}{
        \cellcolor[rgb]{0.9,0.9,0.9}{
            \makecell[{{p{\linewidth}}}]{
                \texttt{\tiny{[GM$|$GM]}}
                \texttt{VERSCHIEBE: D, (5, 3)} \\
            }
        }
    }
    & & \\ \\

    \theutterance \stepcounter{utterance}  
    & & & \multicolumn{2}{p{0.3\linewidth}}{
        \cellcolor[rgb]{0.9,0.9,0.9}{
            \makecell[{{p{\linewidth}}}]{
                \texttt{\tiny{[GM$|$GM]}}
                \texttt{'D' erfolgreich nach (5,3) verschoben. Dein Raster sieht jetzt so aus:} \\
\texttt{```} \\
\texttt{ 1234567} \\
\texttt{╔═╤═╤═╤═╗} \\
\texttt{║◌│P│T│◌║ 1} \\
\texttt{║◌└─┼─┴─╢ 2} \\
\texttt{║◌◌◌│DOS║ 3} \\
\texttt{╟──┬┴───╢ 4} \\
\texttt{║A◌│◌◌◌◌║ 5} \\
\texttt{║◌◌│◌◌M◌║ 6} \\
\texttt{║◌◌│◌◌◌◌║ 7} \\
\texttt{╚══╧════╝} \\
\texttt{```} \\
            }
        }
    }
    & & \\ \\

    \theutterance \stepcounter{utterance}  
    & & \multicolumn{4}{p{0.6\linewidth}}{
        \cellcolor[rgb]{0.9,0.9,0.9}{
            \makecell[{{p{\linewidth}}}]{
                \texttt{\tiny{[GM$\rangle$P2]}}
                \texttt{'S' erfolgreich nach (7,3) verschoben. Dein Raster sieht jetzt so aus:} \\
\texttt{```} \\
\texttt{ 1234567} \\
\texttt{╔═╤═╤═╤═╗} \\
\texttt{║◌│P│T│◌║ 1} \\
\texttt{║◌└─┼─┴─╢ 2} \\
\texttt{║M◌◌│DOS║ 3} \\
\texttt{╟──┬┴───╢ 4} \\
\texttt{║◌◌│◌◌◌◌║ 5} \\
\texttt{║◌◌│◌◌◌◌║ 6} \\
\texttt{║A◌│◌◌◌◌║ 7} \\
\texttt{╚══╧════╝} \\
\texttt{```} \\
\texttt{Ihr seid gerade bei Runde 12 von maximal 28.} \\
\texttt{Ihr habt gemeinsam 0 von maximal 16 Strafpunkten angesammelt.} \\
\texttt{Dein Mitspieler hat ein Objekt auf seinem Raster bewegt.} \\
\\ 
\\ 
\texttt{Wie lautet dein nächster Befehl?} \\
            }
        }
    }
    & \\ \\

    \theutterance \stepcounter{utterance}  
    & & & \multicolumn{4}{p{0.6\linewidth}}{
        \cellcolor[rgb]{0.9,0.9,0.9}{
            \makecell[{{p{\linewidth}}}]{
                \texttt{\tiny{[GM$\langle$P2]}}
                \texttt{VERSCHIEBE: A, (1, 5)} \\
            }
        }
    }
     \\ \\

    \theutterance \stepcounter{utterance}  
    & & & \multicolumn{2}{p{0.3\linewidth}}{
        \cellcolor[rgb]{0.9,0.9,0.9}{
            \makecell[{{p{\linewidth}}}]{
                \texttt{\tiny{[GM$|$GM]}}
                \texttt{VERSCHIEBE: A, (1, 5)} \\
            }
        }
    }
    & & \\ \\

    \theutterance \stepcounter{utterance}  
    & & & \multicolumn{2}{p{0.3\linewidth}}{
        \cellcolor[rgb]{0.9,0.9,0.9}{
            \makecell[{{p{\linewidth}}}]{
                \texttt{\tiny{[GM$|$GM]}}
                \texttt{'A' erfolgreich nach (1,5) verschoben. Dein Raster sieht jetzt so aus:} \\
\texttt{```} \\
\texttt{ 1234567} \\
\texttt{╔═╤═╤═╤═╗} \\
\texttt{║◌│P│T│◌║ 1} \\
\texttt{║◌└─┼─┴─╢ 2} \\
\texttt{║M◌◌│DOS║ 3} \\
\texttt{╟──┬┴───╢ 4} \\
\texttt{║A◌│◌◌◌◌║ 5} \\
\texttt{║◌◌│◌◌◌◌║ 6} \\
\texttt{║◌◌│◌◌◌◌║ 7} \\
\texttt{╚══╧════╝} \\
\texttt{```} \\
            }
        }
    }
    & & \\ \\

    \theutterance \stepcounter{utterance}  
    & & \multicolumn{4}{p{0.6\linewidth}}{
        \cellcolor[rgb]{0.9,0.9,0.9}{
            \makecell[{{p{\linewidth}}}]{
                \texttt{\tiny{[P1$\langle$GM]}}
                \texttt{'D' erfolgreich nach (5,3) verschoben. Dein Raster sieht jetzt so aus:} \\
\texttt{```} \\
\texttt{ 1234567} \\
\texttt{╔═╤═╤═╤═╗} \\
\texttt{║◌│P│T│◌║ 1} \\
\texttt{║◌└─┼─┴─╢ 2} \\
\texttt{║◌◌◌│DOS║ 3} \\
\texttt{╟──┬┴───╢ 4} \\
\texttt{║A◌│◌◌◌◌║ 5} \\
\texttt{║◌◌│◌◌M◌║ 6} \\
\texttt{║◌◌│◌◌◌◌║ 7} \\
\texttt{╚══╧════╝} \\
\texttt{```} \\
\texttt{Ihr seid gerade bei Runde 12 von maximal 28.} \\
\texttt{Ihr habt gemeinsam 0 von maximal 16 Strafpunkten angesammelt.} \\
\texttt{Dein Mitspieler hat ein Objekt auf seinem Raster bewegt.} \\
\\ 
\\ 
\texttt{Wie lautet dein nächster Befehl?} \\
            }
        }
    }
    & \\ \\

    \theutterance \stepcounter{utterance}  
    & \multicolumn{4}{p{0.6\linewidth}}{
        \cellcolor[rgb]{0.9,0.9,0.9}{
            \makecell[{{p{\linewidth}}}]{
                \texttt{\tiny{[P1$\rangle$GM]}}
                \texttt{VERSCHIEBE: M, (1, 3)} \\
            }
        }
    }
    & & \\ \\

    \theutterance \stepcounter{utterance}  
    & & & \multicolumn{2}{p{0.3\linewidth}}{
        \cellcolor[rgb]{0.9,0.9,0.9}{
            \makecell[{{p{\linewidth}}}]{
                \texttt{\tiny{[GM$|$GM]}}
                \texttt{VERSCHIEBE: M, (1, 3)} \\
            }
        }
    }
    & & \\ \\

    \theutterance \stepcounter{utterance}  
    & & & \multicolumn{2}{p{0.3\linewidth}}{
        \cellcolor[rgb]{0.9,0.9,0.9}{
            \makecell[{{p{\linewidth}}}]{
                \texttt{\tiny{[GM$|$GM]}}
                \texttt{'M' erfolgreich nach (1,3) verschoben. Dein Raster sieht jetzt so aus:} \\
\texttt{```} \\
\texttt{ 1234567} \\
\texttt{╔═╤═╤═╤═╗} \\
\texttt{║◌│P│T│◌║ 1} \\
\texttt{║◌└─┼─┴─╢ 2} \\
\texttt{║M◌◌│DOS║ 3} \\
\texttt{╟──┬┴───╢ 4} \\
\texttt{║A◌│◌◌◌◌║ 5} \\
\texttt{║◌◌│◌◌◌◌║ 6} \\
\texttt{║◌◌│◌◌◌◌║ 7} \\
\texttt{╚══╧════╝} \\
\texttt{```} \\
            }
        }
    }
    & & \\ \\

    \theutterance \stepcounter{utterance}  
    & & \multicolumn{4}{p{0.6\linewidth}}{
        \cellcolor[rgb]{0.9,0.9,0.9}{
            \makecell[{{p{\linewidth}}}]{
                \texttt{\tiny{[GM$\rangle$P2]}}
                \texttt{'A' erfolgreich nach (1,5) verschoben. Dein Raster sieht jetzt so aus:} \\
\texttt{```} \\
\texttt{ 1234567} \\
\texttt{╔═╤═╤═╤═╗} \\
\texttt{║◌│P│T│◌║ 1} \\
\texttt{║◌└─┼─┴─╢ 2} \\
\texttt{║M◌◌│DOS║ 3} \\
\texttt{╟──┬┴───╢ 4} \\
\texttt{║A◌│◌◌◌◌║ 5} \\
\texttt{║◌◌│◌◌◌◌║ 6} \\
\texttt{║◌◌│◌◌◌◌║ 7} \\
\texttt{╚══╧════╝} \\
\texttt{```} \\
\texttt{Ihr seid gerade bei Runde 13 von maximal 28.} \\
\texttt{Ihr habt gemeinsam 0 von maximal 16 Strafpunkten angesammelt.} \\
\texttt{Dein Mitspieler hat ein Objekt auf seinem Raster bewegt.} \\
\\ 
\\ 
\texttt{Wie lautet dein nächster Befehl?} \\
            }
        }
    }
    & \\ \\

    \theutterance \stepcounter{utterance}  
    & & & \multicolumn{4}{p{0.6\linewidth}}{
        \cellcolor[rgb]{0.9,0.9,0.9}{
            \makecell[{{p{\linewidth}}}]{
                \texttt{\tiny{[GM$\langle$P2]}}
                \texttt{SPRICH: fertig?} \\
            }
        }
    }
     \\ \\

    \theutterance \stepcounter{utterance}  
    & & & \multicolumn{2}{p{0.3\linewidth}}{
        \cellcolor[rgb]{0.9,0.9,0.9}{
            \makecell[{{p{\linewidth}}}]{
                \texttt{\tiny{[GM$|$GM]}}
                \texttt{SPRICH: fertig?} \\
            }
        }
    }
    & & \\ \\

    \theutterance \stepcounter{utterance}  
    & & \multicolumn{4}{p{0.6\linewidth}}{
        \cellcolor[rgb]{0.9,0.9,0.9}{
            \makecell[{{p{\linewidth}}}]{
                \texttt{\tiny{[P1$\langle$GM]}}
                \texttt{'M' erfolgreich nach (1,3) verschoben. Dein Raster sieht jetzt so aus:} \\
\texttt{```} \\
\texttt{ 1234567} \\
\texttt{╔═╤═╤═╤═╗} \\
\texttt{║◌│P│T│◌║ 1} \\
\texttt{║◌└─┼─┴─╢ 2} \\
\texttt{║M◌◌│DOS║ 3} \\
\texttt{╟──┬┴───╢ 4} \\
\texttt{║A◌│◌◌◌◌║ 5} \\
\texttt{║◌◌│◌◌◌◌║ 6} \\
\texttt{║◌◌│◌◌◌◌║ 7} \\
\texttt{╚══╧════╝} \\
\texttt{```} \\
\texttt{Ihr seid gerade bei Runde 13 von maximal 28.} \\
\texttt{Ihr habt gemeinsam 0 von maximal 16 Strafpunkten angesammelt.} \\
\texttt{Dein Mitspieler hat dir diese Nachricht geschickt:} \\
\\ 
\\ 
\texttt{"fertig?"} \\
\\ 
\\ 
\texttt{Wie lautet dein nächster Befehl?} \\
            }
        }
    }
    & \\ \\

    \theutterance \stepcounter{utterance}  
    & \multicolumn{4}{p{0.6\linewidth}}{
        \cellcolor[rgb]{0.9,0.9,0.9}{
            \makecell[{{p{\linewidth}}}]{
                \texttt{\tiny{[P1$\rangle$GM]}}
                \texttt{SPRICH: fertig!} \\
            }
        }
    }
    & & \\ \\

    \theutterance \stepcounter{utterance}  
    & & & \multicolumn{2}{p{0.3\linewidth}}{
        \cellcolor[rgb]{0.9,0.9,0.9}{
            \makecell[{{p{\linewidth}}}]{
                \texttt{\tiny{[GM$|$GM]}}
                \texttt{SPRICH: fertig!} \\
            }
        }
    }
    & & \\ \\

    \theutterance \stepcounter{utterance}  
    & & & \multicolumn{2}{p{0.3\linewidth}}{
        \cellcolor[rgb]{0.9,0.9,0.9}{
            \makecell[{{p{\linewidth}}}]{
                \texttt{\tiny{[GM$|$GM]}}
                \texttt{true} \\
            }
        }
    }
    & & \\ \\

    \theutterance \stepcounter{utterance}  
    & & & \multicolumn{2}{p{0.3\linewidth}}{
        \cellcolor[rgb]{0.9,0.9,0.9}{
            \makecell[{{p{\linewidth}}}]{
                \texttt{\tiny{[GM$|$GM]}}
                \texttt{* success: True} \\
\texttt{* lose: False} \\
\texttt{* aborted: False} \\
\texttt{{-}{-}{-}{-}{-}{-}{-}} \\
\texttt{* Shifts: 11.00} \\
\texttt{* Max Shifts: 12.00} \\
\texttt{* Min Shifts: 6.00} \\
\texttt{* End Distance Sum: 0.00} \\
\texttt{* Init Distance Sum: 30.87} \\
\texttt{* Expected Distance Sum: 29.33} \\
\texttt{* Penalties: 0.00} \\
\texttt{* Max Penalties: 16.00} \\
\texttt{* Rounds: 13.00} \\
\texttt{* Max Rounds: 28.00} \\
\texttt{* Object Count: 7.00} \\
            }
        }
    }
    & & \\ \\

    \theutterance \stepcounter{utterance}  
    & & & \multicolumn{2}{p{0.3\linewidth}}{
        \cellcolor[rgb]{0.9,0.9,0.9}{
            \makecell[{{p{\linewidth}}}]{
                \texttt{\tiny{[GM$|$GM]}}
                \texttt{Initial Grids:} \\
\texttt{```} \\
\texttt{Player 1:     Player 2:} \\
\texttt{ 1234567       1234567} \\
\texttt{╔═╤═╤═╤═╗     ╔═╤═╤═╤═╗} \\
\texttt{║◌│◌│◌│◌║ 1   ║◌│◌│◌│P║ 1} \\
\texttt{║P└─┼─┴─╢ 2   ║T└─┼─┴─╢ 2} \\
\texttt{║◌SA│◌◌◌║ 3   ║◌◌◌│M◌◌║ 3} \\
\texttt{╟──┬┴───╢ 4   ╟──┬┴───╢ 4} \\
\texttt{║◌◌│◌◌O◌║ 5   ║◌D│◌◌O◌║ 5} \\
\texttt{║◌◌│◌◌MD║ 6   ║◌◌│◌S◌◌║ 6} \\
\texttt{║◌◌│◌◌◌T║ 7   ║A◌│◌◌◌◌║ 7} \\
\texttt{╚══╧════╝     ╚══╧════╝} \\
\texttt{```} \\
\\ 
\texttt{Final Grids:} \\
\texttt{```} \\
\texttt{Player 1:     Player 2:} \\
\texttt{ 1234567       1234567} \\
\texttt{╔═╤═╤═╤═╗     ╔═╤═╤═╤═╗} \\
\texttt{║◌│P│T│◌║ 1   ║◌│P│T│◌║ 1} \\
\texttt{║◌└─┼─┴─╢ 2   ║◌└─┼─┴─╢ 2} \\
\texttt{║M◌◌│DOS║ 3   ║M◌◌│DOS║ 3} \\
\texttt{╟──┬┴───╢ 4   ╟──┬┴───╢ 4} \\
\texttt{║A◌│◌◌◌◌║ 5   ║A◌│◌◌◌◌║ 5} \\
\texttt{║◌◌│◌◌◌◌║ 6   ║◌◌│◌◌◌◌║ 6} \\
\texttt{║◌◌│◌◌◌◌║ 7   ║◌◌│◌◌◌◌║ 7} \\
\texttt{╚══╧════╝     ╚══╧════╝} \\
\texttt{```} \\
            }
        }
    }
    & & \\ \\

\end{supertabular}
}

\end{document}
