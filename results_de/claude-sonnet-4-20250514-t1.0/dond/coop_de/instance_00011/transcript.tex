
\documentclass{article}
\usepackage{colortbl}
\usepackage{makecell}
\usepackage{multirow}
\usepackage{supertabular}

\begin{document}

\newcounter{utterance}

\twocolumn

{ \footnotesize  \setcounter{utterance}{1}
\setlength{\tabcolsep}{0pt}
\begin{supertabular}{c@{$\;$}|p{.15\linewidth}@{}p{.15\linewidth}p{.15\linewidth}p{.15\linewidth}p{.15\linewidth}p{.15\linewidth}}

    \# & $\;$A & \multicolumn{4}{c}{Game Master} & $\;\:$B\\
    \hline 

    \theutterance \stepcounter{utterance}  

    & & \multicolumn{4}{p{0.6\linewidth}}{\cellcolor[rgb]{0.9,0.9,0.9}{%
	\makecell[{{p{\linewidth}}}]{% 
	  \tt {\tiny [A$\langle$GM]}  
	 Sie spielen ein kooperatives Verhandlungsspiel, bei dem Sie sich mit einem anderen Spieler darauf einigen müssen, wie eine Reihe von Gegenständen aufgeteilt werden soll.\\ \tt \\ \tt Die Regeln:\\ \tt (a) Sie und der andere Spieler erhalten eine Sammlung von Gegenständen. Jeder von Ihnen erhält außerdem eine geheime Wertfunktion, die angibt, wie viel Ihnen jede Art von Gegenstand wert ist.\\ \tt (b) Sie tauschen Nachrichten mit dem anderen Spieler aus, um zu vereinbaren, wer welche Gegenstände bekommt. Sie können jeweils maximal 5 Nachrichten senden oder das Spiel vorzeitig beenden, indem Sie jederzeit einen geheimen Vorschlag machen.\\ \tt (c) Jeder von euch wird aufgefordert, einen geheimen Vorschlag zu machen, in dem ihr die gewünschten Gegenstände in eckigen Klammern wie folgt angibt: "[Vorschlag: <Nummer> <Objektname>, <Nummer> <Objektname>, <...>]"\\ \tt (d) Wenn eure Vorschläge komplementär sind, d.h. es gibt genug Gegenstände, um beide Vorschläge zu erfüllen, erhält jeder Spieler eine Punktzahl, die sich aus der Summe der Werte für die Gegenstände ergibt, die er erhalten hat. Andernfalls erhalten Sie beide null Punkte.\\ \tt (e) Ihr Ziel ist es, die Summe aus Ihrer Punktzahl und der Punktzahl des anderen Spielers zu maximieren.\\ \tt \\ \tt Beginnen wir.\\ \tt \\ \tt Die Menge der verfügbaren Gegenstände ist:\\ \tt \\ \tt 1 Ring, 3 Früchte, 1 Kreide\\ \tt \\ \tt Deine geheime Wertfunktion ist:\\ \tt \\ \tt Ring: 6, Frucht: 1, Kreide: 1\\ \tt \\ \tt WICHTIG: Ihre Nachrichten werden, sofern es sich nicht um einen geheimen Vorschlag handelt, direkt an den anderen Spieler übermittelt, also fügen Sie keine Antwort auf die Regeln oder einen Text zur Ankündigung Ihrer Nachricht ein. Um einen geheimen Vorschlag zu machen, verwenden Sie das angegebene Format. Verwenden Sie keine eckigen Klammern, wenn Sie mit dem anderen Spieler kommunizieren, sonst wird dies als Ihr geheimer Vorschlag interpretiert.\\ \tt  
	  } 
	   } 
	   } 
	 & \\ 
 

    \theutterance \stepcounter{utterance}  

    & \multicolumn{4}{p{0.6\linewidth}}{\cellcolor[rgb]{0.8,1,0.9}{%
	\makecell[{{p{\linewidth}}}]{% 
	  \tt {\tiny [A$\rangle$GM]}  
	 Hallo! Ich sehe, wir haben einen Ring, drei Früchte und eine Kreide zu verteilen. Um die beste Lösung für uns beide zu finden, würde ich gerne wissen, welche Gegenstände für dich besonders wichtig sind. Welche würdest du am liebsten haben? 
	  } 
	   } 
	   } 
	 & & \\ 
 

    \theutterance \stepcounter{utterance}  

    & & & \multicolumn{2}{p{0.3\linewidth}}{\cellcolor[rgb]{0.95,0.95,0.95}{%
	\makecell[{{p{\linewidth}}}]{% 
	  \tt {\tiny [GM$|$GM]}  
	 continue 
	  } 
	   } 
	   } 
	 & & \\ 
 

    \theutterance \stepcounter{utterance}  

    & & & \multicolumn{2}{p{0.3\linewidth}}{\cellcolor[rgb]{0.95,0.95,0.95}{%
	\makecell[{{p{\linewidth}}}]{% 
	  \tt {\tiny [GM$|$GM]}  
	 Hallo! Ich sehe, wir haben einen Ring, drei Früchte und eine Kreide zu verteilen. Um die beste Lösung für uns beide zu finden, würde ich gerne wissen, welche Gegenstände für dich besonders wichtig sind. Welche würdest du am liebsten haben? 
	  } 
	   } 
	   } 
	 & & \\ 
 

    \theutterance \stepcounter{utterance}  

    & & \multicolumn{4}{p{0.6\linewidth}}{\cellcolor[rgb]{0.9,0.9,0.9}{%
	\makecell[{{p{\linewidth}}}]{% 
	  \tt {\tiny [GM$\rangle$B]}  
	 Sie spielen ein kooperatives Verhandlungsspiel, bei dem Sie sich mit einem anderen Spieler darauf einigen müssen, wie eine Reihe von Gegenständen aufgeteilt werden soll.\\ \tt \\ \tt Die Regeln:\\ \tt (a) Sie und der andere Spieler erhalten eine Sammlung von Gegenständen. Jeder von Ihnen erhält außerdem eine geheime Wertfunktion, die angibt, wie viel Ihnen jede Art von Gegenstand wert ist.\\ \tt (b) Sie tauschen Nachrichten mit dem anderen Spieler aus, um zu vereinbaren, wer welche Gegenstände bekommt. Sie können jeweils maximal 5 Nachrichten senden oder das Spiel vorzeitig beenden, indem Sie jederzeit einen geheimen Vorschlag machen.\\ \tt (c) Jeder von euch wird aufgefordert, einen geheimen Vorschlag zu machen, in dem ihr die gewünschten Gegenstände in eckigen Klammern wie folgt angibt: "[Vorschlag: <Nummer> <Objektname>, <Nummer> <Objektname>, <...>]"\\ \tt (d) Wenn eure Vorschläge komplementär sind, d.h. es gibt genug Gegenstände, um beide Vorschläge zu erfüllen, erhält jeder Spieler eine Punktzahl, die sich aus der Summe der Werte für die Gegenstände ergibt, die er erhalten hat. Andernfalls erhalten Sie beide null Punkte.\\ \tt (e) Ihr Ziel ist es, die Summe aus Ihrer Punktzahl und der Punktzahl des anderen Spielers zu maximieren.\\ \tt \\ \tt Beginnen wir.\\ \tt \\ \tt Die Menge der verfügbaren Gegenstände ist:\\ \tt \\ \tt 1 Ring, 3 Früchte, 1 Kreide\\ \tt \\ \tt Deine geheime Wertfunktion ist:\\ \tt \\ \tt Ring: 6, Frucht: 1, Kreide: 1\\ \tt \\ \tt WICHTIG: Ihre Nachrichten werden, sofern es sich nicht um einen geheimen Vorschlag handelt, direkt an den anderen Spieler übermittelt, also fügen Sie keine Antwort auf die Regeln oder einen Text zur Ankündigung Ihrer Nachricht ein. Um einen geheimen Vorschlag zu machen, verwenden Sie das angegebene Format. Verwenden Sie keine eckigen Klammern, wenn Sie mit dem anderen Spieler kommunizieren, sonst wird dies als Ihr geheimer Vorschlag interpretiert.\\ \tt \\ \tt \\ \tt Hallo! Ich sehe, wir haben einen Ring, drei Früchte und eine Kreide zu verteilen. Um die beste Lösung für uns beide zu finden, würde ich gerne wissen, welche Gegenstände für dich besonders wichtig sind. Welche würdest du am liebsten haben? 
	  } 
	   } 
	   } 
	 & \\ 
 

    \theutterance \stepcounter{utterance}  

    & & & \multicolumn{4}{p{0.6\linewidth}}{\cellcolor[rgb]{1,0.85,0.72}{%
	\makecell[{{p{\linewidth}}}]{% 
	  \tt {\tiny [GM$\langle$B]}  
	 Hallo! Der Ring wäre für mich sehr interessant. Bei den Früchten und der Kreide bin ich flexibler. Was ist denn für dich am wichtigsten? Vielleicht können wir eine Aufteilung finden, die für uns beide gut funktioniert. 
	  } 
	   } 
	   } 
	  \\ 
 

    \theutterance \stepcounter{utterance}  

    & & & \multicolumn{2}{p{0.3\linewidth}}{\cellcolor[rgb]{0.95,0.95,0.95}{%
	\makecell[{{p{\linewidth}}}]{% 
	  \tt {\tiny [GM$|$GM]}  
	 continue 
	  } 
	   } 
	   } 
	 & & \\ 
 

    \theutterance \stepcounter{utterance}  

    & & & \multicolumn{2}{p{0.3\linewidth}}{\cellcolor[rgb]{0.95,0.95,0.95}{%
	\makecell[{{p{\linewidth}}}]{% 
	  \tt {\tiny [GM$|$GM]}  
	 Hallo! Der Ring wäre für mich sehr interessant. Bei den Früchten und der Kreide bin ich flexibler. Was ist denn für dich am wichtigsten? Vielleicht können wir eine Aufteilung finden, die für uns beide gut funktioniert. 
	  } 
	   } 
	   } 
	 & & \\ 
 

    \theutterance \stepcounter{utterance}  

    & & \multicolumn{4}{p{0.6\linewidth}}{\cellcolor[rgb]{0.9,0.9,0.9}{%
	\makecell[{{p{\linewidth}}}]{% 
	  \tt {\tiny [A$\langle$GM]}  
	 Hallo! Der Ring wäre für mich sehr interessant. Bei den Früchten und der Kreide bin ich flexibler. Was ist denn für dich am wichtigsten? Vielleicht können wir eine Aufteilung finden, die für uns beide gut funktioniert. 
	  } 
	   } 
	   } 
	 & \\ 
 

    \theutterance \stepcounter{utterance}  

    & \multicolumn{4}{p{0.6\linewidth}}{\cellcolor[rgb]{0.8,1,0.9}{%
	\makecell[{{p{\linewidth}}}]{% 
	  \tt {\tiny [A$\rangle$GM]}  
	 Der Ring ist tatsächlich auch für mich sehr wertvoll! Das ist ein Dilemma, da wir beide ihn gerne hätten. Wie wäre es, wenn wir das so lösen: Einer von uns bekommt den Ring, und der andere bekommt alle drei Früchte plus die Kreide als Ausgleich? Wäre das für dich fair, wenn du alle Früchte und die Kreide bekommen würdest und ich den Ring? 
	  } 
	   } 
	   } 
	 & & \\ 
 

    \theutterance \stepcounter{utterance}  

    & & & \multicolumn{2}{p{0.3\linewidth}}{\cellcolor[rgb]{0.95,0.95,0.95}{%
	\makecell[{{p{\linewidth}}}]{% 
	  \tt {\tiny [GM$|$GM]}  
	 continue 
	  } 
	   } 
	   } 
	 & & \\ 
 

    \theutterance \stepcounter{utterance}  

    & & & \multicolumn{2}{p{0.3\linewidth}}{\cellcolor[rgb]{0.95,0.95,0.95}{%
	\makecell[{{p{\linewidth}}}]{% 
	  \tt {\tiny [GM$|$GM]}  
	 Der Ring ist tatsächlich auch für mich sehr wertvoll! Das ist ein Dilemma, da wir beide ihn gerne hätten. Wie wäre es, wenn wir das so lösen: Einer von uns bekommt den Ring, und der andere bekommt alle drei Früchte plus die Kreide als Ausgleich? Wäre das für dich fair, wenn du alle Früchte und die Kreide bekommen würdest und ich den Ring? 
	  } 
	   } 
	   } 
	 & & \\ 
 

    \theutterance \stepcounter{utterance}  

    & & \multicolumn{4}{p{0.6\linewidth}}{\cellcolor[rgb]{0.9,0.9,0.9}{%
	\makecell[{{p{\linewidth}}}]{% 
	  \tt {\tiny [GM$\rangle$B]}  
	 Der Ring ist tatsächlich auch für mich sehr wertvoll! Das ist ein Dilemma, da wir beide ihn gerne hätten. Wie wäre es, wenn wir das so lösen: Einer von uns bekommt den Ring, und der andere bekommt alle drei Früchte plus die Kreide als Ausgleich? Wäre das für dich fair, wenn du alle Früchte und die Kreide bekommen würdest und ich den Ring? 
	  } 
	   } 
	   } 
	 & \\ 
 

    \theutterance \stepcounter{utterance}  

    & & & \multicolumn{4}{p{0.6\linewidth}}{\cellcolor[rgb]{1,0.85,0.72}{%
	\makecell[{{p{\linewidth}}}]{% 
	  \tt {\tiny [GM$\langle$B]}  
	 Hmm, das ist ein interessanter Vorschlag, aber der Ring bedeutet mir wirklich sehr viel. Wie wichtig sind denn die Früchte für dich? Falls die Früchte für dich sehr wertvoll sind, könnte ich mir vorstellen, dass du alle drei Früchte bekommst und ich den Ring plus die Kreide nehme. Was denkst du darüber? 
	  } 
	   } 
	   } 
	  \\ 
 

    \theutterance \stepcounter{utterance}  

    & & & \multicolumn{2}{p{0.3\linewidth}}{\cellcolor[rgb]{0.95,0.95,0.95}{%
	\makecell[{{p{\linewidth}}}]{% 
	  \tt {\tiny [GM$|$GM]}  
	 continue 
	  } 
	   } 
	   } 
	 & & \\ 
 

    \theutterance \stepcounter{utterance}  

    & & & \multicolumn{2}{p{0.3\linewidth}}{\cellcolor[rgb]{0.95,0.95,0.95}{%
	\makecell[{{p{\linewidth}}}]{% 
	  \tt {\tiny [GM$|$GM]}  
	 Hmm, das ist ein interessanter Vorschlag, aber der Ring bedeutet mir wirklich sehr viel. Wie wichtig sind denn die Früchte für dich? Falls die Früchte für dich sehr wertvoll sind, könnte ich mir vorstellen, dass du alle drei Früchte bekommst und ich den Ring plus die Kreide nehme. Was denkst du darüber? 
	  } 
	   } 
	   } 
	 & & \\ 
 

    \theutterance \stepcounter{utterance}  

    & & \multicolumn{4}{p{0.6\linewidth}}{\cellcolor[rgb]{0.9,0.9,0.9}{%
	\makecell[{{p{\linewidth}}}]{% 
	  \tt {\tiny [A$\langle$GM]}  
	 Hmm, das ist ein interessanter Vorschlag, aber der Ring bedeutet mir wirklich sehr viel. Wie wichtig sind denn die Früchte für dich? Falls die Früchte für dich sehr wertvoll sind, könnte ich mir vorstellen, dass du alle drei Früchte bekommst und ich den Ring plus die Kreide nehme. Was denkst du darüber? 
	  } 
	   } 
	   } 
	 & \\ 
 

    \theutterance \stepcounter{utterance}  

    & \multicolumn{4}{p{0.6\linewidth}}{\cellcolor[rgb]{0.8,1,0.9}{%
	\makecell[{{p{\linewidth}}}]{% 
	  \tt {\tiny [A$\rangle$GM]}  
	 Ich muss ehrlich sagen, dass die Früchte für mich nicht besonders wertvoll sind. Der Ring bedeutet mir wirklich sehr viel - genauso wie dir offenbar auch! Das macht es schwierig. Vielleicht sollten wir nochmal überlegen: Wie wertvoll ist denn die Kreide für dich im Vergleich zu den Früchten? Falls die Kreide für dich wichtiger ist als die Früchte, könnte eine andere Aufteilung sinnvoll sein. 
	  } 
	   } 
	   } 
	 & & \\ 
 

    \theutterance \stepcounter{utterance}  

    & & & \multicolumn{2}{p{0.3\linewidth}}{\cellcolor[rgb]{0.95,0.95,0.95}{%
	\makecell[{{p{\linewidth}}}]{% 
	  \tt {\tiny [GM$|$GM]}  
	 continue 
	  } 
	   } 
	   } 
	 & & \\ 
 

    \theutterance \stepcounter{utterance}  

    & & & \multicolumn{2}{p{0.3\linewidth}}{\cellcolor[rgb]{0.95,0.95,0.95}{%
	\makecell[{{p{\linewidth}}}]{% 
	  \tt {\tiny [GM$|$GM]}  
	 Ich muss ehrlich sagen, dass die Früchte für mich nicht besonders wertvoll sind. Der Ring bedeutet mir wirklich sehr viel - genauso wie dir offenbar auch! Das macht es schwierig. Vielleicht sollten wir nochmal überlegen: Wie wertvoll ist denn die Kreide für dich im Vergleich zu den Früchten? Falls die Kreide für dich wichtiger ist als die Früchte, könnte eine andere Aufteilung sinnvoll sein. 
	  } 
	   } 
	   } 
	 & & \\ 
 

    \theutterance \stepcounter{utterance}  

    & & \multicolumn{4}{p{0.6\linewidth}}{\cellcolor[rgb]{0.9,0.9,0.9}{%
	\makecell[{{p{\linewidth}}}]{% 
	  \tt {\tiny [GM$\rangle$B]}  
	 Ich muss ehrlich sagen, dass die Früchte für mich nicht besonders wertvoll sind. Der Ring bedeutet mir wirklich sehr viel - genauso wie dir offenbar auch! Das macht es schwierig. Vielleicht sollten wir nochmal überlegen: Wie wertvoll ist denn die Kreide für dich im Vergleich zu den Früchten? Falls die Kreide für dich wichtiger ist als die Früchte, könnte eine andere Aufteilung sinnvoll sein. 
	  } 
	   } 
	   } 
	 & \\ 
 

    \theutterance \stepcounter{utterance}  

    & & & \multicolumn{4}{p{0.6\linewidth}}{\cellcolor[rgb]{1,0.85,0.72}{%
	\makecell[{{p{\linewidth}}}]{% 
	  \tt {\tiny [GM$\langle$B]}  
	 Ehrlich gesagt sind mir Kreide und Früchte ziemlich ähnlich wichtig - nicht besonders wertvoll im Vergleich zum Ring. Da wir beide den Ring so sehr wollen, wird das schwierig. Vielleicht sollten wir beide ehrlich über unsere Präferenzen sein: Für mich ist der Ring deutlich wertvoller als alle anderen Gegenstände zusammen. Wie ist das bei dir? Wenn der Ring auch für dich viel mehr wert ist als Früchte und Kreide zusammen, müssen wir uns irgendwie einigen wer ihn bekommt. 
	  } 
	   } 
	   } 
	  \\ 
 

    \theutterance \stepcounter{utterance}  

    & & & \multicolumn{2}{p{0.3\linewidth}}{\cellcolor[rgb]{0.95,0.95,0.95}{%
	\makecell[{{p{\linewidth}}}]{% 
	  \tt {\tiny [GM$|$GM]}  
	 continue 
	  } 
	   } 
	   } 
	 & & \\ 
 

    \theutterance \stepcounter{utterance}  

    & & & \multicolumn{2}{p{0.3\linewidth}}{\cellcolor[rgb]{0.95,0.95,0.95}{%
	\makecell[{{p{\linewidth}}}]{% 
	  \tt {\tiny [GM$|$GM]}  
	 Ehrlich gesagt sind mir Kreide und Früchte ziemlich ähnlich wichtig - nicht besonders wertvoll im Vergleich zum Ring. Da wir beide den Ring so sehr wollen, wird das schwierig. Vielleicht sollten wir beide ehrlich über unsere Präferenzen sein: Für mich ist der Ring deutlich wertvoller als alle anderen Gegenstände zusammen. Wie ist das bei dir? Wenn der Ring auch für dich viel mehr wert ist als Früchte und Kreide zusammen, müssen wir uns irgendwie einigen wer ihn bekommt. 
	  } 
	   } 
	   } 
	 & & \\ 
 

    \theutterance \stepcounter{utterance}  

    & & \multicolumn{4}{p{0.6\linewidth}}{\cellcolor[rgb]{0.9,0.9,0.9}{%
	\makecell[{{p{\linewidth}}}]{% 
	  \tt {\tiny [A$\langle$GM]}  
	 Ehrlich gesagt sind mir Kreide und Früchte ziemlich ähnlich wichtig - nicht besonders wertvoll im Vergleich zum Ring. Da wir beide den Ring so sehr wollen, wird das schwierig. Vielleicht sollten wir beide ehrlich über unsere Präferenzen sein: Für mich ist der Ring deutlich wertvoller als alle anderen Gegenstände zusammen. Wie ist das bei dir? Wenn der Ring auch für dich viel mehr wert ist als Früchte und Kreide zusammen, müssen wir uns irgendwie einigen wer ihn bekommt. 
	  } 
	   } 
	   } 
	 & \\ 
 

    \theutterance \stepcounter{utterance}  

    & \multicolumn{4}{p{0.6\linewidth}}{\cellcolor[rgb]{0.8,1,0.9}{%
	\makecell[{{p{\linewidth}}}]{% 
	  \tt {\tiny [A$\rangle$GM]}  
	 Du hast recht - ich bin in derselben Situation. Der Ring ist auch für mich deutlich wertvoller als alle anderen Gegenstände zusammen. Das ist wirklich ein Dilemma!\\ \tt \\ \tt Da wir beide so ehrlich sind: Ich denke, wir sollten einen konkreten Vorschlag machen. Ich würde vorschlagen, dass ich den Ring nehme und du alle drei Früchte plus die Kreide bekommst. Das wäre zumindest eine klare Aufteilung. Falls du aber einen anderen Vorschlag hast oder denkst, dass du den Ring bekommen solltest, lass es mich wissen!\\ \tt \\ \tt [Vorschlag: 1 Ring] 
	  } 
	   } 
	   } 
	 & & \\ 
 

    \theutterance \stepcounter{utterance}  

    & & & \multicolumn{2}{p{0.3\linewidth}}{\cellcolor[rgb]{0.95,0.95,0.95}{%
	\makecell[{{p{\linewidth}}}]{% 
	  \tt {\tiny [GM$|$GM]}  
	 proposal 
	  } 
	   } 
	   } 
	 & & \\ 
 

    \theutterance \stepcounter{utterance}  

    & & & \multicolumn{2}{p{0.3\linewidth}}{\cellcolor[rgb]{0.95,0.95,0.95}{%
	\makecell[{{p{\linewidth}}}]{% 
	  \tt {\tiny [GM$|$GM]}  
	 [1, 0, 0] 
	  } 
	   } 
	   } 
	 & & \\ 
 

    \theutterance \stepcounter{utterance}  

    & & \multicolumn{4}{p{0.6\linewidth}}{\cellcolor[rgb]{0.9,0.9,0.9}{%
	\makecell[{{p{\linewidth}}}]{% 
	  \tt {\tiny [GM$\rangle$B]}  
	 Der andere Spieler hat einen geheimen Vorschlag gemacht. Jetzt bist du an der Reihe, einen eigenen Vorschlag zu machen.\\ \tt  
	  } 
	   } 
	   } 
	 & \\ 
 

    \theutterance \stepcounter{utterance}  

    & & & \multicolumn{4}{p{0.6\linewidth}}{\cellcolor[rgb]{1,0.85,0.72}{%
	\makecell[{{p{\linewidth}}}]{% 
	  \tt {\tiny [GM$\langle$B]}  
	 [Vorschlag: 1 Ring] 
	  } 
	   } 
	   } 
	  \\ 
 

    \theutterance \stepcounter{utterance}  

    & & & \multicolumn{2}{p{0.3\linewidth}}{\cellcolor[rgb]{0.95,0.95,0.95}{%
	\makecell[{{p{\linewidth}}}]{% 
	  \tt {\tiny [GM$|$GM]}  
	 proposal 
	  } 
	   } 
	   } 
	 & & \\ 
 

    \theutterance \stepcounter{utterance}  

    & & & \multicolumn{2}{p{0.3\linewidth}}{\cellcolor[rgb]{0.95,0.95,0.95}{%
	\makecell[{{p{\linewidth}}}]{% 
	  \tt {\tiny [GM$|$GM]}  
	 [1, 0, 0] 
	  } 
	   } 
	   } 
	 & & \\ 
 

    \theutterance \stepcounter{utterance}  

    & & & \multicolumn{2}{p{0.3\linewidth}}{\cellcolor[rgb]{0.95,0.95,0.95}{%
	\makecell[{{p{\linewidth}}}]{% 
	  \tt {\tiny [GM$|$GM]}  
	 [[1, 0, 0], [1, 0, 0]] 
	  } 
	   } 
	   } 
	 & & \\ 
 

\end{supertabular}
}

\end{document}
