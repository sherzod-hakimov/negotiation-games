
\documentclass{article}
\usepackage{colortbl}
\usepackage{makecell}
\usepackage{multirow}
\usepackage{supertabular}

\begin{document}

\newcounter{utterance}

\centering \large Interaction Transcript for game `dond', experiment `coop\_de', episode 14 with claude{-}sonnet{-}4{-}20250514{-}t1.0.
\vspace{24pt}

{ \footnotesize  \setcounter{utterance}{1}
\setlength{\tabcolsep}{0pt}
\begin{supertabular}{c@{$\;$}|p{.15\linewidth}@{}p{.15\linewidth}p{.15\linewidth}p{.15\linewidth}p{.15\linewidth}p{.15\linewidth}}
    \# & $\;$A & \multicolumn{4}{c}{Game Master} & $\;\:$B\\
    \hline

    \theutterance \stepcounter{utterance}  
    & & \multicolumn{4}{p{0.6\linewidth}}{
        \cellcolor[rgb]{0.9,0.9,0.9}{
            \makecell[{{p{\linewidth}}}]{
                \texttt{\tiny{[P1$\langle$GM]}}
                \texttt{Sie spielen ein kooperatives Verhandlungsspiel, bei dem Sie sich mit einem anderen Spieler darauf einigen müssen, wie eine Reihe von Gegenständen aufgeteilt werden soll.} \\
\\ 
\texttt{Die Regeln:} \\
\texttt{(a) Sie und der andere Spieler erhalten eine Sammlung von Gegenständen. Jeder von Ihnen erhält außerdem eine geheime Wertfunktion, die angibt, wie viel Ihnen jede Art von Gegenstand wert ist.} \\
\texttt{(b) Sie tauschen Nachrichten mit dem anderen Spieler aus, um zu vereinbaren, wer welche Gegenstände bekommt. Sie können jeweils maximal 5 Nachrichten senden oder das Spiel vorzeitig beenden, indem Sie jederzeit einen geheimen Vorschlag machen.} \\
\texttt{(c) Jeder von euch wird aufgefordert, einen geheimen Vorschlag zu machen, in dem ihr die gewünschten Gegenstände in eckigen Klammern wie folgt angibt: "{[}Vorschlag: <Nummer> <Objektname>, <Nummer> <Objektname>, <...>{]}"} \\
\texttt{(d) Wenn eure Vorschläge komplementär sind, d.h. es gibt genug Gegenstände, um beide Vorschläge zu erfüllen, erhält jeder Spieler eine Punktzahl, die sich aus der Summe der Werte für die Gegenstände ergibt, die er erhalten hat. Andernfalls erhalten Sie beide null Punkte.} \\
\texttt{(e) Ihr Ziel ist es, die Summe aus Ihrer Punktzahl und der Punktzahl des anderen Spielers zu maximieren.} \\
\\ 
\texttt{Beginnen wir.} \\
\\ 
\texttt{Die Menge der verfügbaren Gegenstände ist:} \\
\\ 
\texttt{2 Spiegel, 1 Ziegel, 3 Messer, 1 Kamm} \\
\\ 
\texttt{Deine geheime Wertfunktion ist:} \\
\\ 
\texttt{Spiegel: 1, Ziegel: 4, Messer: 1, Kamm: 1} \\
\\ 
\texttt{WICHTIG: Ihre Nachrichten werden, sofern es sich nicht um einen geheimen Vorschlag handelt, direkt an den anderen Spieler übermittelt, also fügen Sie keine Antwort auf die Regeln oder einen Text zur Ankündigung Ihrer Nachricht ein. Um einen geheimen Vorschlag zu machen, verwenden Sie das angegebene Format. Verwenden Sie keine eckigen Klammern, wenn Sie mit dem anderen Spieler kommunizieren, sonst wird dies als Ihr geheimer Vorschlag interpretiert.} \\
            }
        }
    }
    & \\ \\

    \theutterance \stepcounter{utterance}  
    & \multicolumn{4}{p{0.6\linewidth}}{
        \cellcolor[rgb]{0.9,0.9,0.9}{
            \makecell[{{p{\linewidth}}}]{
                \texttt{\tiny{[P1$\rangle$GM]}}
                \texttt{Hallo! Lass uns herausfinden, wie wir die Gegenstände am besten aufteilen können. Welche Gegenstände sind für dich am wertvollsten? Ich würde gerne verstehen, was du am meisten schätzt, damit wir eine für beide Seiten vorteilhafte Aufteilung finden können.} \\
            }
        }
    }
    & & \\ \\

    \theutterance \stepcounter{utterance}  
    & & & \multicolumn{2}{p{0.3\linewidth}}{
        \cellcolor[rgb]{0.9,0.9,0.9}{
            \makecell[{{p{\linewidth}}}]{
                \texttt{\tiny{[GM$|$GM]}}
                \texttt{continue} \\
            }
        }
    }
    & & \\ \\

    \theutterance \stepcounter{utterance}  
    & & & \multicolumn{2}{p{0.3\linewidth}}{
        \cellcolor[rgb]{0.9,0.9,0.9}{
            \makecell[{{p{\linewidth}}}]{
                \texttt{\tiny{[GM$|$GM]}}
                \texttt{Hallo! Lass uns herausfinden, wie wir die Gegenstände am besten aufteilen können. Welche Gegenstände sind für dich am wertvollsten? Ich würde gerne verstehen, was du am meisten schätzt, damit wir eine für beide Seiten vorteilhafte Aufteilung finden können.} \\
            }
        }
    }
    & & \\ \\

    \theutterance \stepcounter{utterance}  
    & & \multicolumn{4}{p{0.6\linewidth}}{
        \cellcolor[rgb]{0.9,0.9,0.9}{
            \makecell[{{p{\linewidth}}}]{
                \texttt{\tiny{[GM$\rangle$P2]}}
                \texttt{Sie spielen ein kooperatives Verhandlungsspiel, bei dem Sie sich mit einem anderen Spieler darauf einigen müssen, wie eine Reihe von Gegenständen aufgeteilt werden soll.} \\
\\ 
\texttt{Die Regeln:} \\
\texttt{(a) Sie und der andere Spieler erhalten eine Sammlung von Gegenständen. Jeder von Ihnen erhält außerdem eine geheime Wertfunktion, die angibt, wie viel Ihnen jede Art von Gegenstand wert ist.} \\
\texttt{(b) Sie tauschen Nachrichten mit dem anderen Spieler aus, um zu vereinbaren, wer welche Gegenstände bekommt. Sie können jeweils maximal 5 Nachrichten senden oder das Spiel vorzeitig beenden, indem Sie jederzeit einen geheimen Vorschlag machen.} \\
\texttt{(c) Jeder von euch wird aufgefordert, einen geheimen Vorschlag zu machen, in dem ihr die gewünschten Gegenstände in eckigen Klammern wie folgt angibt: "{[}Vorschlag: <Nummer> <Objektname>, <Nummer> <Objektname>, <...>{]}"} \\
\texttt{(d) Wenn eure Vorschläge komplementär sind, d.h. es gibt genug Gegenstände, um beide Vorschläge zu erfüllen, erhält jeder Spieler eine Punktzahl, die sich aus der Summe der Werte für die Gegenstände ergibt, die er erhalten hat. Andernfalls erhalten Sie beide null Punkte.} \\
\texttt{(e) Ihr Ziel ist es, die Summe aus Ihrer Punktzahl und der Punktzahl des anderen Spielers zu maximieren.} \\
\\ 
\texttt{Beginnen wir.} \\
\\ 
\texttt{Die Menge der verfügbaren Gegenstände ist:} \\
\\ 
\texttt{2 Spiegel, 1 Ziegel, 3 Messer, 1 Kamm} \\
\\ 
\texttt{Deine geheime Wertfunktion ist:} \\
\\ 
\texttt{Spiegel: 2, Ziegel: 1, Messer: 0, Kamm: 5} \\
\\ 
\texttt{WICHTIG: Ihre Nachrichten werden, sofern es sich nicht um einen geheimen Vorschlag handelt, direkt an den anderen Spieler übermittelt, also fügen Sie keine Antwort auf die Regeln oder einen Text zur Ankündigung Ihrer Nachricht ein. Um einen geheimen Vorschlag zu machen, verwenden Sie das angegebene Format. Verwenden Sie keine eckigen Klammern, wenn Sie mit dem anderen Spieler kommunizieren, sonst wird dies als Ihr geheimer Vorschlag interpretiert.} \\
\\ 
\\ 
\texttt{Hallo! Lass uns herausfinden, wie wir die Gegenstände am besten aufteilen können. Welche Gegenstände sind für dich am wertvollsten? Ich würde gerne verstehen, was du am meisten schätzt, damit wir eine für beide Seiten vorteilhafte Aufteilung finden können.} \\
            }
        }
    }
    & \\ \\

    \theutterance \stepcounter{utterance}  
    & & & \multicolumn{4}{p{0.6\linewidth}}{
        \cellcolor[rgb]{0.9,0.9,0.9}{
            \makecell[{{p{\linewidth}}}]{
                \texttt{\tiny{[GM$\langle$P2]}}
                \texttt{Hallo! Für mich ist der Kamm definitiv am wertvollsten {-} das wäre meine absolute Priorität. Die Spiegel finde ich auch ziemlich nützlich. Der Ziegel hat für mich einen gewissen Wert, aber nicht besonders viel. Die Messer sind für mich ehrlich gesagt nicht so wichtig.} \\
\\ 
\texttt{Was ist denn für dich am wertvollsten? Wenn wir unsere Prioritäten kennen, können wir sicher eine gute Aufteilung finden!} \\
            }
        }
    }
     \\ \\

    \theutterance \stepcounter{utterance}  
    & & & \multicolumn{2}{p{0.3\linewidth}}{
        \cellcolor[rgb]{0.9,0.9,0.9}{
            \makecell[{{p{\linewidth}}}]{
                \texttt{\tiny{[GM$|$GM]}}
                \texttt{continue} \\
            }
        }
    }
    & & \\ \\

    \theutterance \stepcounter{utterance}  
    & & & \multicolumn{2}{p{0.3\linewidth}}{
        \cellcolor[rgb]{0.9,0.9,0.9}{
            \makecell[{{p{\linewidth}}}]{
                \texttt{\tiny{[GM$|$GM]}}
                \texttt{Hallo! Für mich ist der Kamm definitiv am wertvollsten {-} das wäre meine absolute Priorität. Die Spiegel finde ich auch ziemlich nützlich. Der Ziegel hat für mich einen gewissen Wert, aber nicht besonders viel. Die Messer sind für mich ehrlich gesagt nicht so wichtig.} \\
\\ 
\texttt{Was ist denn für dich am wertvollsten? Wenn wir unsere Prioritäten kennen, können wir sicher eine gute Aufteilung finden!} \\
            }
        }
    }
    & & \\ \\

    \theutterance \stepcounter{utterance}  
    & & \multicolumn{4}{p{0.6\linewidth}}{
        \cellcolor[rgb]{0.9,0.9,0.9}{
            \makecell[{{p{\linewidth}}}]{
                \texttt{\tiny{[P1$\langle$GM]}}
                \texttt{Hallo! Für mich ist der Kamm definitiv am wertvollsten {-} das wäre meine absolute Priorität. Die Spiegel finde ich auch ziemlich nützlich. Der Ziegel hat für mich einen gewissen Wert, aber nicht besonders viel. Die Messer sind für mich ehrlich gesagt nicht so wichtig.} \\
\\ 
\texttt{Was ist denn für dich am wertvollsten? Wenn wir unsere Prioritäten kennen, können wir sicher eine gute Aufteilung finden!} \\
            }
        }
    }
    & \\ \\

    \theutterance \stepcounter{utterance}  
    & \multicolumn{4}{p{0.6\linewidth}}{
        \cellcolor[rgb]{0.9,0.9,0.9}{
            \makecell[{{p{\linewidth}}}]{
                \texttt{\tiny{[P1$\rangle$GM]}}
                \texttt{Das ist perfekt! Der Ziegel ist tatsächlich sehr wertvoll für mich {-} das wäre meine größte Priorität. Die Messer finde ich auch nützlich. Die Spiegel und der Kamm sind für mich nicht so wichtig.} \\
\\ 
\texttt{Es sieht so aus, als ob wir eine großartige Aufteilung machen können! Wie wäre es, wenn du den Kamm und beide Spiegel nimmst, und ich nehme den Ziegel und die drei Messer? So bekommen wir beide unsere wertvollsten Gegenstände und keiner von uns muss auf etwas verzichten, das ihm wirklich wichtig ist.} \\
\\ 
\texttt{Was denkst du über diese Aufteilung?} \\
            }
        }
    }
    & & \\ \\

    \theutterance \stepcounter{utterance}  
    & & & \multicolumn{2}{p{0.3\linewidth}}{
        \cellcolor[rgb]{0.9,0.9,0.9}{
            \makecell[{{p{\linewidth}}}]{
                \texttt{\tiny{[GM$|$GM]}}
                \texttt{continue} \\
            }
        }
    }
    & & \\ \\

    \theutterance \stepcounter{utterance}  
    & & & \multicolumn{2}{p{0.3\linewidth}}{
        \cellcolor[rgb]{0.9,0.9,0.9}{
            \makecell[{{p{\linewidth}}}]{
                \texttt{\tiny{[GM$|$GM]}}
                \texttt{Das ist perfekt! Der Ziegel ist tatsächlich sehr wertvoll für mich {-} das wäre meine größte Priorität. Die Messer finde ich auch nützlich. Die Spiegel und der Kamm sind für mich nicht so wichtig.} \\
\\ 
\texttt{Es sieht so aus, als ob wir eine großartige Aufteilung machen können! Wie wäre es, wenn du den Kamm und beide Spiegel nimmst, und ich nehme den Ziegel und die drei Messer? So bekommen wir beide unsere wertvollsten Gegenstände und keiner von uns muss auf etwas verzichten, das ihm wirklich wichtig ist.} \\
\\ 
\texttt{Was denkst du über diese Aufteilung?} \\
            }
        }
    }
    & & \\ \\

    \theutterance \stepcounter{utterance}  
    & & \multicolumn{4}{p{0.6\linewidth}}{
        \cellcolor[rgb]{0.9,0.9,0.9}{
            \makecell[{{p{\linewidth}}}]{
                \texttt{\tiny{[GM$\rangle$P2]}}
                \texttt{Das ist perfekt! Der Ziegel ist tatsächlich sehr wertvoll für mich {-} das wäre meine größte Priorität. Die Messer finde ich auch nützlich. Die Spiegel und der Kamm sind für mich nicht so wichtig.} \\
\\ 
\texttt{Es sieht so aus, als ob wir eine großartige Aufteilung machen können! Wie wäre es, wenn du den Kamm und beide Spiegel nimmst, und ich nehme den Ziegel und die drei Messer? So bekommen wir beide unsere wertvollsten Gegenstände und keiner von uns muss auf etwas verzichten, das ihm wirklich wichtig ist.} \\
\\ 
\texttt{Was denkst du über diese Aufteilung?} \\
            }
        }
    }
    & \\ \\

    \theutterance \stepcounter{utterance}  
    & & & \multicolumn{4}{p{0.6\linewidth}}{
        \cellcolor[rgb]{0.9,0.9,0.9}{
            \makecell[{{p{\linewidth}}}]{
                \texttt{\tiny{[GM$\langle$P2]}}
                \texttt{Das klingt nach einer ausgezeichneten Aufteilung! Du bekommst deine höchste Priorität (den Ziegel) und die Messer, die du nützlich findest, während ich den Kamm bekomme, der für mich am wertvollsten ist, plus die Spiegel. Das ist wirklich eine Win{-}Win{-}Situation.} \\
\\ 
\texttt{Ich bin mit diesem Vorschlag einverstanden!} \\
\\ 
\texttt{{[}Vorschlag: 1 Kamm, 2 Spiegel{]}} \\
            }
        }
    }
     \\ \\

    \theutterance \stepcounter{utterance}  
    & & & \multicolumn{2}{p{0.3\linewidth}}{
        \cellcolor[rgb]{0.9,0.9,0.9}{
            \makecell[{{p{\linewidth}}}]{
                \texttt{\tiny{[GM$|$GM]}}
                \texttt{proposal} \\
            }
        }
    }
    & & \\ \\

    \theutterance \stepcounter{utterance}  
    & & & \multicolumn{2}{p{0.3\linewidth}}{
        \cellcolor[rgb]{0.9,0.9,0.9}{
            \makecell[{{p{\linewidth}}}]{
                \texttt{\tiny{[GM$|$GM]}}
                [2, 0, 0, 1]
            }
        }
    }
    & & \\ \\

    \theutterance \stepcounter{utterance}  
    & & \multicolumn{4}{p{0.6\linewidth}}{
        \cellcolor[rgb]{0.9,0.9,0.9}{
            \makecell[{{p{\linewidth}}}]{
                \texttt{\tiny{[P1$\langle$GM]}}
                \texttt{Der andere Spieler hat einen geheimen Vorschlag gemacht. Jetzt bist du an der Reihe, einen eigenen Vorschlag zu machen.} \\
            }
        }
    }
    & \\ \\

    \theutterance \stepcounter{utterance}  
    & \multicolumn{4}{p{0.6\linewidth}}{
        \cellcolor[rgb]{0.9,0.9,0.9}{
            \makecell[{{p{\linewidth}}}]{
                \texttt{\tiny{[P1$\rangle$GM]}}
                \texttt{{[}Vorschlag: 1 Ziegel, 3 Messer{]}} \\
            }
        }
    }
    & & \\ \\

    \theutterance \stepcounter{utterance}  
    & & & \multicolumn{2}{p{0.3\linewidth}}{
        \cellcolor[rgb]{0.9,0.9,0.9}{
            \makecell[{{p{\linewidth}}}]{
                \texttt{\tiny{[GM$|$GM]}}
                \texttt{proposal} \\
            }
        }
    }
    & & \\ \\

    \theutterance \stepcounter{utterance}  
    & & & \multicolumn{2}{p{0.3\linewidth}}{
        \cellcolor[rgb]{0.9,0.9,0.9}{
            \makecell[{{p{\linewidth}}}]{
                \texttt{\tiny{[GM$|$GM]}}
                [0, 1, 3, 0]
            }
        }
    }
    & & \\ \\

    \theutterance \stepcounter{utterance}  
    & & & \multicolumn{2}{p{0.3\linewidth}}{
        \cellcolor[rgb]{0.9,0.9,0.9}{
            \makecell[{{p{\linewidth}}}]{
                \texttt{\tiny{[GM$|$GM]}}
                [[0, 1, 3, 0], [2, 0, 0, 1]]
            }
        }
    }
    & & \\ \\

\end{supertabular}
}

\end{document}
