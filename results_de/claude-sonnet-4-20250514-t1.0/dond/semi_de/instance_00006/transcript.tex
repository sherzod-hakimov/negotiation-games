
\documentclass{article}
\usepackage{colortbl}
\usepackage{makecell}
\usepackage{multirow}
\usepackage{supertabular}

\begin{document}

\newcounter{utterance}

\centering \large Interaction Transcript for game `dond', experiment `semi\_de', episode 6 with claude{-}sonnet{-}4{-}20250514{-}t1.0.
\vspace{24pt}

{ \footnotesize  \setcounter{utterance}{1}
\setlength{\tabcolsep}{0pt}
\begin{supertabular}{c@{$\;$}|p{.15\linewidth}@{}p{.15\linewidth}p{.15\linewidth}p{.15\linewidth}p{.15\linewidth}p{.15\linewidth}}
    \# & $\;$A & \multicolumn{4}{c}{Game Master} & $\;\:$B\\
    \hline

    \theutterance \stepcounter{utterance}  
    & & \multicolumn{4}{p{0.6\linewidth}}{
        \cellcolor[rgb]{0.9,0.9,0.9}{
            \makecell[{{p{\linewidth}}}]{
                \texttt{\tiny{[P1$\langle$GM]}}
                \texttt{Sie spielen ein Verhandlungsspiel, bei dem Sie sich mit einem anderen Spieler darauf einigen müssen, wie eine Reihe von Gegenständen aufgeteilt werden soll.} \\
\\ 
\texttt{Die Regeln:} \\
\texttt{(a) Sie und der andere Spieler erhalten eine Sammlung von Gegenständen. Jeder von Ihnen erhält außerdem eine geheime Wertfunktion, die angibt, wie viel Ihnen jede Art von Gegenstand wert ist.} \\
\texttt{(b) Sie tauschen Nachrichten mit dem anderen Spieler aus, um zu vereinbaren, wer welche Gegenstände bekommt. Sie können jeweils maximal 5 Nachrichten senden oder das Spiel vorzeitig beenden, indem Sie jederzeit einen geheimen Vorschlag machen.} \\
\texttt{(c) Jeder von euch wird aufgefordert, einen geheimen Vorschlag zu machen, in dem ihr die gewünschten Gegenstände in eckigen Klammern wie folgt angibt: "{[}Vorschlag: <Nummer> <Objektname>, <Nummer> <Objektname>, <...>{]}"} \\
\texttt{(d) Wenn eure Vorschläge komplementär sind, d.h. es gibt genug Gegenstände, um beide Vorschläge zu erfüllen, erhält jeder Spieler eine Punktzahl, die sich aus der Summe der Werte für die Gegenstände ergibt, die er erhalten hat. Andernfalls erhalten Sie beide null Punkte.} \\
\texttt{(e) Ihr Ziel ist es, eine möglichst hohe Punktzahl zu erreichen.} \\
\\ 
\texttt{Beginnen wir.} \\
\\ 
\texttt{Die Menge der verfügbaren Gegenstände ist:} \\
\\ 
\texttt{3 Äpfel, 1 Seife, 1 Ei} \\
\\ 
\texttt{Deine geheime Wertfunktion ist:} \\
\\ 
\texttt{Apfel: 1, Seife: 3, Ei: 4} \\
\\ 
\texttt{WICHTIG: Ihre Nachrichten werden, sofern es sich nicht um einen geheimen Vorschlag handelt, direkt an den anderen Spieler übermittelt, also fügen Sie keine Antwort auf die Regeln oder einen Text zur Ankündigung Ihrer Nachricht ein. Um einen geheimen Vorschlag zu machen, verwenden Sie das angegebene Format. Verwenden Sie keine eckigen Klammern, wenn Sie mit dem anderen Spieler kommunizieren, sonst wird dies als Ihr geheimer Vorschlag interpretiert.} \\
            }
        }
    }
    & \\ \\

    \theutterance \stepcounter{utterance}  
    & \multicolumn{4}{p{0.6\linewidth}}{
        \cellcolor[rgb]{0.9,0.9,0.9}{
            \makecell[{{p{\linewidth}}}]{
                \texttt{\tiny{[P1$\rangle$GM]}}
                \texttt{Hallo! Ich sehe, wir haben 3 Äpfel, 1 Seife und 1 Ei zu verteilen. Was ist dir denn besonders wichtig von diesen Gegenständen? Vielleicht können wir eine Aufteilung finden, die für uns beide gut funktioniert.} \\
            }
        }
    }
    & & \\ \\

    \theutterance \stepcounter{utterance}  
    & & & \multicolumn{2}{p{0.3\linewidth}}{
        \cellcolor[rgb]{0.9,0.9,0.9}{
            \makecell[{{p{\linewidth}}}]{
                \texttt{\tiny{[GM$|$GM]}}
                \texttt{continue} \\
            }
        }
    }
    & & \\ \\

    \theutterance \stepcounter{utterance}  
    & & & \multicolumn{2}{p{0.3\linewidth}}{
        \cellcolor[rgb]{0.9,0.9,0.9}{
            \makecell[{{p{\linewidth}}}]{
                \texttt{\tiny{[GM$|$GM]}}
                \texttt{Hallo! Ich sehe, wir haben 3 Äpfel, 1 Seife und 1 Ei zu verteilen. Was ist dir denn besonders wichtig von diesen Gegenständen? Vielleicht können wir eine Aufteilung finden, die für uns beide gut funktioniert.} \\
            }
        }
    }
    & & \\ \\

    \theutterance \stepcounter{utterance}  
    & & \multicolumn{4}{p{0.6\linewidth}}{
        \cellcolor[rgb]{0.9,0.9,0.9}{
            \makecell[{{p{\linewidth}}}]{
                \texttt{\tiny{[GM$\rangle$P2]}}
                \texttt{Sie spielen ein Verhandlungsspiel, bei dem Sie sich mit einem anderen Spieler darauf einigen müssen, wie eine Reihe von Gegenständen aufgeteilt werden soll.} \\
\\ 
\texttt{Die Regeln:} \\
\texttt{(a) Sie und der andere Spieler erhalten eine Sammlung von Gegenständen. Jeder von Ihnen erhält außerdem eine geheime Wertfunktion, die angibt, wie viel Ihnen jede Art von Gegenstand wert ist.} \\
\texttt{(b) Sie tauschen Nachrichten mit dem anderen Spieler aus, um zu vereinbaren, wer welche Gegenstände bekommt. Sie können jeweils maximal 5 Nachrichten senden oder das Spiel vorzeitig beenden, indem Sie jederzeit einen geheimen Vorschlag machen.} \\
\texttt{(c) Jeder von euch wird aufgefordert, einen geheimen Vorschlag zu machen, in dem ihr die gewünschten Gegenstände in eckigen Klammern wie folgt angibt: "{[}Vorschlag: <Nummer> <Objektname>, <Nummer> <Objektname>, <...>{]}"} \\
\texttt{(d) Wenn eure Vorschläge komplementär sind, d.h. es gibt genug Gegenstände, um beide Vorschläge zu erfüllen, erhält jeder Spieler eine Punktzahl, die sich aus der Summe der Werte für die Gegenstände ergibt, die er erhalten hat. Andernfalls erhalten Sie beide null Punkte.} \\
\texttt{(e) Ihr Ziel ist es, eine möglichst hohe Punktzahl zu erreichen.} \\
\\ 
\texttt{Beginnen wir.} \\
\\ 
\texttt{Die Menge der verfügbaren Gegenstände ist:} \\
\\ 
\texttt{3 Äpfel, 1 Seife, 1 Ei} \\
\\ 
\texttt{Deine geheime Wertfunktion ist:} \\
\\ 
\texttt{Apfel: 0, Seife: 7, Ei: 3} \\
\\ 
\texttt{WICHTIG: Ihre Nachrichten werden, sofern es sich nicht um einen geheimen Vorschlag handelt, direkt an den anderen Spieler übermittelt, also fügen Sie keine Antwort auf die Regeln oder einen Text zur Ankündigung Ihrer Nachricht ein. Um einen geheimen Vorschlag zu machen, verwenden Sie das angegebene Format. Verwenden Sie keine eckigen Klammern, wenn Sie mit dem anderen Spieler kommunizieren, sonst wird dies als Ihr geheimer Vorschlag interpretiert.} \\
\\ 
\\ 
\texttt{Hallo! Ich sehe, wir haben 3 Äpfel, 1 Seife und 1 Ei zu verteilen. Was ist dir denn besonders wichtig von diesen Gegenständen? Vielleicht können wir eine Aufteilung finden, die für uns beide gut funktioniert.} \\
            }
        }
    }
    & \\ \\

    \theutterance \stepcounter{utterance}  
    & & & \multicolumn{4}{p{0.6\linewidth}}{
        \cellcolor[rgb]{0.9,0.9,0.9}{
            \makecell[{{p{\linewidth}}}]{
                \texttt{\tiny{[GM$\langle$P2]}}
                \texttt{Hallo! Ja, lass uns eine faire Lösung finden. Mir ist ehrlich gesagt die Seife ziemlich wichtig {-} ich könnte die gut gebrauchen. Bei den Äpfeln bin ich flexibel, die sind für mich nicht so entscheidend. Das Ei wäre auch nett zu haben, aber nicht unbedingt nötig. } \\
\\ 
\texttt{Was denkst du denn {-} welche Gegenstände sind für dich am wertvollsten?} \\
            }
        }
    }
     \\ \\

    \theutterance \stepcounter{utterance}  
    & & & \multicolumn{2}{p{0.3\linewidth}}{
        \cellcolor[rgb]{0.9,0.9,0.9}{
            \makecell[{{p{\linewidth}}}]{
                \texttt{\tiny{[GM$|$GM]}}
                \texttt{continue} \\
            }
        }
    }
    & & \\ \\

    \theutterance \stepcounter{utterance}  
    & & & \multicolumn{2}{p{0.3\linewidth}}{
        \cellcolor[rgb]{0.9,0.9,0.9}{
            \makecell[{{p{\linewidth}}}]{
                \texttt{\tiny{[GM$|$GM]}}
                \texttt{Hallo! Ja, lass uns eine faire Lösung finden. Mir ist ehrlich gesagt die Seife ziemlich wichtig {-} ich könnte die gut gebrauchen. Bei den Äpfeln bin ich flexibel, die sind für mich nicht so entscheidend. Das Ei wäre auch nett zu haben, aber nicht unbedingt nötig. } \\
\\ 
\texttt{Was denkst du denn {-} welche Gegenstände sind für dich am wertvollsten?} \\
            }
        }
    }
    & & \\ \\

    \theutterance \stepcounter{utterance}  
    & & \multicolumn{4}{p{0.6\linewidth}}{
        \cellcolor[rgb]{0.9,0.9,0.9}{
            \makecell[{{p{\linewidth}}}]{
                \texttt{\tiny{[P1$\langle$GM]}}
                \texttt{Hallo! Ja, lass uns eine faire Lösung finden. Mir ist ehrlich gesagt die Seife ziemlich wichtig {-} ich könnte die gut gebrauchen. Bei den Äpfeln bin ich flexibel, die sind für mich nicht so entscheidend. Das Ei wäre auch nett zu haben, aber nicht unbedingt nötig. } \\
\\ 
\texttt{Was denkst du denn {-} welche Gegenstände sind für dich am wertvollsten?} \\
            }
        }
    }
    & \\ \\

    \theutterance \stepcounter{utterance}  
    & \multicolumn{4}{p{0.6\linewidth}}{
        \cellcolor[rgb]{0.9,0.9,0.9}{
            \makecell[{{p{\linewidth}}}]{
                \texttt{\tiny{[P1$\rangle$GM]}}
                \texttt{Das ist interessant! Für mich ist tatsächlich das Ei am wichtigsten {-} das hätte ich sehr gerne. Die Seife ist auch wertvoll für mich, aber wenn sie dir so wichtig ist, können wir da sicher eine Lösung finden. } \\
\\ 
\texttt{Die Äpfel sind für mich auch nicht das Entscheidende, also da bin ich auch flexibel.} \\
\\ 
\texttt{Wie wäre es, wenn du die Seife nimmst, ich das Ei, und wir die 3 Äpfel irgendwie aufteilen? Würdest du mit 1 oder 2 Äpfeln zufrieden sein?} \\
            }
        }
    }
    & & \\ \\

    \theutterance \stepcounter{utterance}  
    & & & \multicolumn{2}{p{0.3\linewidth}}{
        \cellcolor[rgb]{0.9,0.9,0.9}{
            \makecell[{{p{\linewidth}}}]{
                \texttt{\tiny{[GM$|$GM]}}
                \texttt{continue} \\
            }
        }
    }
    & & \\ \\

    \theutterance \stepcounter{utterance}  
    & & & \multicolumn{2}{p{0.3\linewidth}}{
        \cellcolor[rgb]{0.9,0.9,0.9}{
            \makecell[{{p{\linewidth}}}]{
                \texttt{\tiny{[GM$|$GM]}}
                \texttt{Das ist interessant! Für mich ist tatsächlich das Ei am wichtigsten {-} das hätte ich sehr gerne. Die Seife ist auch wertvoll für mich, aber wenn sie dir so wichtig ist, können wir da sicher eine Lösung finden. } \\
\\ 
\texttt{Die Äpfel sind für mich auch nicht das Entscheidende, also da bin ich auch flexibel.} \\
\\ 
\texttt{Wie wäre es, wenn du die Seife nimmst, ich das Ei, und wir die 3 Äpfel irgendwie aufteilen? Würdest du mit 1 oder 2 Äpfeln zufrieden sein?} \\
            }
        }
    }
    & & \\ \\

    \theutterance \stepcounter{utterance}  
    & & \multicolumn{4}{p{0.6\linewidth}}{
        \cellcolor[rgb]{0.9,0.9,0.9}{
            \makecell[{{p{\linewidth}}}]{
                \texttt{\tiny{[GM$\rangle$P2]}}
                \texttt{Das ist interessant! Für mich ist tatsächlich das Ei am wichtigsten {-} das hätte ich sehr gerne. Die Seife ist auch wertvoll für mich, aber wenn sie dir so wichtig ist, können wir da sicher eine Lösung finden. } \\
\\ 
\texttt{Die Äpfel sind für mich auch nicht das Entscheidende, also da bin ich auch flexibel.} \\
\\ 
\texttt{Wie wäre es, wenn du die Seife nimmst, ich das Ei, und wir die 3 Äpfel irgendwie aufteilen? Würdest du mit 1 oder 2 Äpfeln zufrieden sein?} \\
            }
        }
    }
    & \\ \\

    \theutterance \stepcounter{utterance}  
    & & & \multicolumn{4}{p{0.6\linewidth}}{
        \cellcolor[rgb]{0.9,0.9,0.9}{
            \makecell[{{p{\linewidth}}}]{
                \texttt{\tiny{[GM$\langle$P2]}}
                \texttt{Das klingt nach einer sehr guten Lösung! Wenn du das Ei bekommst und ich die Seife, sind wir beide mit den wichtigsten Sachen versorgt. } \\
\\ 
\texttt{Bei den Äpfeln bin ich wirklich flexibel {-} du kannst gerne 2 Äpfel nehmen und ich nehme 1, oder wir machen es andersrum. Was würdest du denn bevorzugen?} \\
\\ 
\texttt{Ich denke, so eine Aufteilung wäre fair für uns beide!} \\
            }
        }
    }
     \\ \\

    \theutterance \stepcounter{utterance}  
    & & & \multicolumn{2}{p{0.3\linewidth}}{
        \cellcolor[rgb]{0.9,0.9,0.9}{
            \makecell[{{p{\linewidth}}}]{
                \texttt{\tiny{[GM$|$GM]}}
                \texttt{continue} \\
            }
        }
    }
    & & \\ \\

    \theutterance \stepcounter{utterance}  
    & & & \multicolumn{2}{p{0.3\linewidth}}{
        \cellcolor[rgb]{0.9,0.9,0.9}{
            \makecell[{{p{\linewidth}}}]{
                \texttt{\tiny{[GM$|$GM]}}
                \texttt{Das klingt nach einer sehr guten Lösung! Wenn du das Ei bekommst und ich die Seife, sind wir beide mit den wichtigsten Sachen versorgt. } \\
\\ 
\texttt{Bei den Äpfeln bin ich wirklich flexibel {-} du kannst gerne 2 Äpfel nehmen und ich nehme 1, oder wir machen es andersrum. Was würdest du denn bevorzugen?} \\
\\ 
\texttt{Ich denke, so eine Aufteilung wäre fair für uns beide!} \\
            }
        }
    }
    & & \\ \\

    \theutterance \stepcounter{utterance}  
    & & \multicolumn{4}{p{0.6\linewidth}}{
        \cellcolor[rgb]{0.9,0.9,0.9}{
            \makecell[{{p{\linewidth}}}]{
                \texttt{\tiny{[P1$\langle$GM]}}
                \texttt{Das klingt nach einer sehr guten Lösung! Wenn du das Ei bekommst und ich die Seife, sind wir beide mit den wichtigsten Sachen versorgt. } \\
\\ 
\texttt{Bei den Äpfeln bin ich wirklich flexibel {-} du kannst gerne 2 Äpfel nehmen und ich nehme 1, oder wir machen es andersrum. Was würdest du denn bevorzugen?} \\
\\ 
\texttt{Ich denke, so eine Aufteilung wäre fair für uns beide!} \\
            }
        }
    }
    & \\ \\

    \theutterance \stepcounter{utterance}  
    & \multicolumn{4}{p{0.6\linewidth}}{
        \cellcolor[rgb]{0.9,0.9,0.9}{
            \makecell[{{p{\linewidth}}}]{
                \texttt{\tiny{[P1$\rangle$GM]}}
                \texttt{Das freut mich, dass wir so schnell eine gute Lösung gefunden haben! } \\
\\ 
\texttt{Da du bei den Äpfeln so flexibel bist, wäre es für mich tatsächlich schön, wenn ich 2 Äpfel nehmen könnte und du 1 {-} aber nur wenn das für dich wirklich in Ordnung ist.} \\
\\ 
\texttt{Also die Aufteilung wäre dann:} \\
\texttt{{-} Du: 1 Seife, 1 Apfel} \\
\texttt{{-} Ich: 1 Ei, 2 Äpfel} \\
\\ 
\texttt{Klingt das gut für dich?} \\
            }
        }
    }
    & & \\ \\

    \theutterance \stepcounter{utterance}  
    & & & \multicolumn{2}{p{0.3\linewidth}}{
        \cellcolor[rgb]{0.9,0.9,0.9}{
            \makecell[{{p{\linewidth}}}]{
                \texttt{\tiny{[GM$|$GM]}}
                \texttt{continue} \\
            }
        }
    }
    & & \\ \\

    \theutterance \stepcounter{utterance}  
    & & & \multicolumn{2}{p{0.3\linewidth}}{
        \cellcolor[rgb]{0.9,0.9,0.9}{
            \makecell[{{p{\linewidth}}}]{
                \texttt{\tiny{[GM$|$GM]}}
                \texttt{Das freut mich, dass wir so schnell eine gute Lösung gefunden haben! } \\
\\ 
\texttt{Da du bei den Äpfeln so flexibel bist, wäre es für mich tatsächlich schön, wenn ich 2 Äpfel nehmen könnte und du 1 {-} aber nur wenn das für dich wirklich in Ordnung ist.} \\
\\ 
\texttt{Also die Aufteilung wäre dann:} \\
\texttt{{-} Du: 1 Seife, 1 Apfel} \\
\texttt{{-} Ich: 1 Ei, 2 Äpfel} \\
\\ 
\texttt{Klingt das gut für dich?} \\
            }
        }
    }
    & & \\ \\

    \theutterance \stepcounter{utterance}  
    & & \multicolumn{4}{p{0.6\linewidth}}{
        \cellcolor[rgb]{0.9,0.9,0.9}{
            \makecell[{{p{\linewidth}}}]{
                \texttt{\tiny{[GM$\rangle$P2]}}
                \texttt{Das freut mich, dass wir so schnell eine gute Lösung gefunden haben! } \\
\\ 
\texttt{Da du bei den Äpfeln so flexibel bist, wäre es für mich tatsächlich schön, wenn ich 2 Äpfel nehmen könnte und du 1 {-} aber nur wenn das für dich wirklich in Ordnung ist.} \\
\\ 
\texttt{Also die Aufteilung wäre dann:} \\
\texttt{{-} Du: 1 Seife, 1 Apfel} \\
\texttt{{-} Ich: 1 Ei, 2 Äpfel} \\
\\ 
\texttt{Klingt das gut für dich?} \\
            }
        }
    }
    & \\ \\

    \theutterance \stepcounter{utterance}  
    & & & \multicolumn{4}{p{0.6\linewidth}}{
        \cellcolor[rgb]{0.9,0.9,0.9}{
            \makecell[{{p{\linewidth}}}]{
                \texttt{\tiny{[GM$\langle$P2]}}
                \texttt{Das klingt perfekt! Du bekommst das Ei, das dir am wichtigsten ist, plus 2 Äpfel, und ich bekomme die Seife, die mir wichtig ist, plus 1 Apfel. Das ist eine wirklich faire Aufteilung für uns beide.} \\
\\ 
\texttt{{[}Vorschlag: 1 Seife, 1 Apfel{]}} \\
            }
        }
    }
     \\ \\

    \theutterance \stepcounter{utterance}  
    & & & \multicolumn{2}{p{0.3\linewidth}}{
        \cellcolor[rgb]{0.9,0.9,0.9}{
            \makecell[{{p{\linewidth}}}]{
                \texttt{\tiny{[GM$|$GM]}}
                \texttt{proposal} \\
            }
        }
    }
    & & \\ \\

    \theutterance \stepcounter{utterance}  
    & & & \multicolumn{2}{p{0.3\linewidth}}{
        \cellcolor[rgb]{0.9,0.9,0.9}{
            \makecell[{{p{\linewidth}}}]{
                \texttt{\tiny{[GM$|$GM]}}
                [1, 1, 0]
            }
        }
    }
    & & \\ \\

    \theutterance \stepcounter{utterance}  
    & & \multicolumn{4}{p{0.6\linewidth}}{
        \cellcolor[rgb]{0.9,0.9,0.9}{
            \makecell[{{p{\linewidth}}}]{
                \texttt{\tiny{[P1$\langle$GM]}}
                \texttt{Der andere Spieler hat einen geheimen Vorschlag gemacht. Jetzt bist du an der Reihe, einen eigenen Vorschlag zu machen.} \\
            }
        }
    }
    & \\ \\

    \theutterance \stepcounter{utterance}  
    & \multicolumn{4}{p{0.6\linewidth}}{
        \cellcolor[rgb]{0.9,0.9,0.9}{
            \makecell[{{p{\linewidth}}}]{
                \texttt{\tiny{[P1$\rangle$GM]}}
                \texttt{{[}Vorschlag: 1 Ei, 2 Äpfel{]}} \\
            }
        }
    }
    & & \\ \\

    \theutterance \stepcounter{utterance}  
    & & & \multicolumn{2}{p{0.3\linewidth}}{
        \cellcolor[rgb]{0.9,0.9,0.9}{
            \makecell[{{p{\linewidth}}}]{
                \texttt{\tiny{[GM$|$GM]}}
                \texttt{proposal} \\
            }
        }
    }
    & & \\ \\

    \theutterance \stepcounter{utterance}  
    & & & \multicolumn{2}{p{0.3\linewidth}}{
        \cellcolor[rgb]{0.9,0.9,0.9}{
            \makecell[{{p{\linewidth}}}]{
                \texttt{\tiny{[GM$|$GM]}}
                [2, 0, 1]
            }
        }
    }
    & & \\ \\

    \theutterance \stepcounter{utterance}  
    & & & \multicolumn{2}{p{0.3\linewidth}}{
        \cellcolor[rgb]{0.9,0.9,0.9}{
            \makecell[{{p{\linewidth}}}]{
                \texttt{\tiny{[GM$|$GM]}}
                [[2, 0, 1], [1, 1, 0]]
            }
        }
    }
    & & \\ \\

\end{supertabular}
}

\end{document}
