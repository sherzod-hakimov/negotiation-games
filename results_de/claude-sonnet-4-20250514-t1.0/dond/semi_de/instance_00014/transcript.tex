
\documentclass{article}
\usepackage{colortbl}
\usepackage{makecell}
\usepackage{multirow}
\usepackage{supertabular}

\begin{document}

\newcounter{utterance}

\twocolumn

{ \footnotesize  \setcounter{utterance}{1}
\setlength{\tabcolsep}{0pt}
\begin{supertabular}{c@{$\;$}|p{.15\linewidth}@{}p{.15\linewidth}p{.15\linewidth}p{.15\linewidth}p{.15\linewidth}p{.15\linewidth}}

    \# & $\;$A & \multicolumn{4}{c}{Game Master} & $\;\:$B\\
    \hline 

    \theutterance \stepcounter{utterance}  

    & & \multicolumn{4}{p{0.6\linewidth}}{\cellcolor[rgb]{0.9,0.9,0.9}{%
	\makecell[{{p{\linewidth}}}]{% 
	  \tt {\tiny [A$\langle$GM]}  
	 Sie spielen ein Verhandlungsspiel, bei dem Sie sich mit einem anderen Spieler darauf einigen müssen, wie eine Reihe von Gegenständen aufgeteilt werden soll.\\ \tt \\ \tt Die Regeln:\\ \tt (a) Sie und der andere Spieler erhalten eine Sammlung von Gegenständen. Jeder von Ihnen erhält außerdem eine geheime Wertfunktion, die angibt, wie viel Ihnen jede Art von Gegenstand wert ist.\\ \tt (b) Sie tauschen Nachrichten mit dem anderen Spieler aus, um zu vereinbaren, wer welche Gegenstände bekommt. Sie können jeweils maximal 5 Nachrichten senden oder das Spiel vorzeitig beenden, indem Sie jederzeit einen geheimen Vorschlag machen.\\ \tt (c) Jeder von euch wird aufgefordert, einen geheimen Vorschlag zu machen, in dem ihr die gewünschten Gegenstände in eckigen Klammern wie folgt angibt: "[Vorschlag: <Nummer> <Objektname>, <Nummer> <Objektname>, <...>]"\\ \tt (d) Wenn eure Vorschläge komplementär sind, d.h. es gibt genug Gegenstände, um beide Vorschläge zu erfüllen, erhält jeder Spieler eine Punktzahl, die sich aus der Summe der Werte für die Gegenstände ergibt, die er erhalten hat. Andernfalls erhalten Sie beide null Punkte.\\ \tt (e) Ihr Ziel ist es, eine möglichst hohe Punktzahl zu erreichen.\\ \tt \\ \tt Beginnen wir.\\ \tt \\ \tt Die Menge der verfügbaren Gegenstände ist:\\ \tt \\ \tt 1 Rahmen, 1 Schuh, 3 Hämmer\\ \tt \\ \tt Deine geheime Wertfunktion ist:\\ \tt \\ \tt Rahmen: 8, Schuh: 2, Hammer: 0\\ \tt \\ \tt WICHTIG: Ihre Nachrichten werden, sofern es sich nicht um einen geheimen Vorschlag handelt, direkt an den anderen Spieler übermittelt, also fügen Sie keine Antwort auf die Regeln oder einen Text zur Ankündigung Ihrer Nachricht ein. Um einen geheimen Vorschlag zu machen, verwenden Sie das angegebene Format. Verwenden Sie keine eckigen Klammern, wenn Sie mit dem anderen Spieler kommunizieren, sonst wird dies als Ihr geheimer Vorschlag interpretiert.\\ \tt  
	  } 
	   } 
	   } 
	 & \\ 
 

    \theutterance \stepcounter{utterance}  

    & \multicolumn{4}{p{0.6\linewidth}}{\cellcolor[rgb]{0.8,1,0.9}{%
	\makecell[{{p{\linewidth}}}]{% 
	  \tt {\tiny [A$\rangle$GM]}  
	 Hallo! Ich sehe, wir haben 1 Rahmen, 1 Schuh und 3 Hämmer zu verteilen. Was sind deine Gedanken zu diesen Gegenständen? Welche interessieren dich am meisten? 
	  } 
	   } 
	   } 
	 & & \\ 
 

    \theutterance \stepcounter{utterance}  

    & & & \multicolumn{2}{p{0.3\linewidth}}{\cellcolor[rgb]{0.95,0.95,0.95}{%
	\makecell[{{p{\linewidth}}}]{% 
	  \tt {\tiny [GM$|$GM]}  
	 continue 
	  } 
	   } 
	   } 
	 & & \\ 
 

    \theutterance \stepcounter{utterance}  

    & & & \multicolumn{2}{p{0.3\linewidth}}{\cellcolor[rgb]{0.95,0.95,0.95}{%
	\makecell[{{p{\linewidth}}}]{% 
	  \tt {\tiny [GM$|$GM]}  
	 Hallo! Ich sehe, wir haben 1 Rahmen, 1 Schuh und 3 Hämmer zu verteilen. Was sind deine Gedanken zu diesen Gegenständen? Welche interessieren dich am meisten? 
	  } 
	   } 
	   } 
	 & & \\ 
 

    \theutterance \stepcounter{utterance}  

    & & \multicolumn{4}{p{0.6\linewidth}}{\cellcolor[rgb]{0.9,0.9,0.9}{%
	\makecell[{{p{\linewidth}}}]{% 
	  \tt {\tiny [GM$\rangle$B]}  
	 Sie spielen ein Verhandlungsspiel, bei dem Sie sich mit einem anderen Spieler darauf einigen müssen, wie eine Reihe von Gegenständen aufgeteilt werden soll.\\ \tt \\ \tt Die Regeln:\\ \tt (a) Sie und der andere Spieler erhalten eine Sammlung von Gegenständen. Jeder von Ihnen erhält außerdem eine geheime Wertfunktion, die angibt, wie viel Ihnen jede Art von Gegenstand wert ist.\\ \tt (b) Sie tauschen Nachrichten mit dem anderen Spieler aus, um zu vereinbaren, wer welche Gegenstände bekommt. Sie können jeweils maximal 5 Nachrichten senden oder das Spiel vorzeitig beenden, indem Sie jederzeit einen geheimen Vorschlag machen.\\ \tt (c) Jeder von euch wird aufgefordert, einen geheimen Vorschlag zu machen, in dem ihr die gewünschten Gegenstände in eckigen Klammern wie folgt angibt: "[Vorschlag: <Nummer> <Objektname>, <Nummer> <Objektname>, <...>]"\\ \tt (d) Wenn eure Vorschläge komplementär sind, d.h. es gibt genug Gegenstände, um beide Vorschläge zu erfüllen, erhält jeder Spieler eine Punktzahl, die sich aus der Summe der Werte für die Gegenstände ergibt, die er erhalten hat. Andernfalls erhalten Sie beide null Punkte.\\ \tt (e) Ihr Ziel ist es, eine möglichst hohe Punktzahl zu erreichen.\\ \tt \\ \tt Beginnen wir.\\ \tt \\ \tt Die Menge der verfügbaren Gegenstände ist:\\ \tt \\ \tt 1 Rahmen, 1 Schuh, 3 Hämmer\\ \tt \\ \tt Deine geheime Wertfunktion ist:\\ \tt \\ \tt Rahmen: 6, Schuh: 1, Hammer: 1\\ \tt \\ \tt WICHTIG: Ihre Nachrichten werden, sofern es sich nicht um einen geheimen Vorschlag handelt, direkt an den anderen Spieler übermittelt, also fügen Sie keine Antwort auf die Regeln oder einen Text zur Ankündigung Ihrer Nachricht ein. Um einen geheimen Vorschlag zu machen, verwenden Sie das angegebene Format. Verwenden Sie keine eckigen Klammern, wenn Sie mit dem anderen Spieler kommunizieren, sonst wird dies als Ihr geheimer Vorschlag interpretiert.\\ \tt \\ \tt \\ \tt Hallo! Ich sehe, wir haben 1 Rahmen, 1 Schuh und 3 Hämmer zu verteilen. Was sind deine Gedanken zu diesen Gegenständen? Welche interessieren dich am meisten? 
	  } 
	   } 
	   } 
	 & \\ 
 

    \theutterance \stepcounter{utterance}  

    & & & \multicolumn{4}{p{0.6\linewidth}}{\cellcolor[rgb]{1,0.85,0.72}{%
	\makecell[{{p{\linewidth}}}]{% 
	  \tt {\tiny [GM$\langle$B]}  
	 Hallo! Ja, eine interessante Mischung von Gegenständen. Ich finde den Rahmen ganz ansprechend - könnte nützlich sein. Bei den Hämmern denke ich, dass man davon sicher mehrere gebrauchen kann, je nachdem wofür. Der Schuh ist eher spezifisch.\\ \tt \\ \tt Was denkst du denn über die Gegenstände? Hast du besondere Verwendung für bestimmte Sachen im Sinn? 
	  } 
	   } 
	   } 
	  \\ 
 

    \theutterance \stepcounter{utterance}  

    & & & \multicolumn{2}{p{0.3\linewidth}}{\cellcolor[rgb]{0.95,0.95,0.95}{%
	\makecell[{{p{\linewidth}}}]{% 
	  \tt {\tiny [GM$|$GM]}  
	 continue 
	  } 
	   } 
	   } 
	 & & \\ 
 

    \theutterance \stepcounter{utterance}  

    & & & \multicolumn{2}{p{0.3\linewidth}}{\cellcolor[rgb]{0.95,0.95,0.95}{%
	\makecell[{{p{\linewidth}}}]{% 
	  \tt {\tiny [GM$|$GM]}  
	 Hallo! Ja, eine interessante Mischung von Gegenständen. Ich finde den Rahmen ganz ansprechend - könnte nützlich sein. Bei den Hämmern denke ich, dass man davon sicher mehrere gebrauchen kann, je nachdem wofür. Der Schuh ist eher spezifisch.\\ \tt \\ \tt Was denkst du denn über die Gegenstände? Hast du besondere Verwendung für bestimmte Sachen im Sinn? 
	  } 
	   } 
	   } 
	 & & \\ 
 

    \theutterance \stepcounter{utterance}  

    & & \multicolumn{4}{p{0.6\linewidth}}{\cellcolor[rgb]{0.9,0.9,0.9}{%
	\makecell[{{p{\linewidth}}}]{% 
	  \tt {\tiny [A$\langle$GM]}  
	 Hallo! Ja, eine interessante Mischung von Gegenständen. Ich finde den Rahmen ganz ansprechend - könnte nützlich sein. Bei den Hämmern denke ich, dass man davon sicher mehrere gebrauchen kann, je nachdem wofür. Der Schuh ist eher spezifisch.\\ \tt \\ \tt Was denkst du denn über die Gegenstände? Hast du besondere Verwendung für bestimmte Sachen im Sinn? 
	  } 
	   } 
	   } 
	 & \\ 
 

    \theutterance \stepcounter{utterance}  

    & \multicolumn{4}{p{0.6\linewidth}}{\cellcolor[rgb]{0.8,1,0.9}{%
	\makecell[{{p{\linewidth}}}]{% 
	  \tt {\tiny [A$\rangle$GM]}  
	 Der Rahmen ist tatsächlich interessant! Bei den Hämmern sehe ich ehrlich gesagt nicht so den großen Nutzen für mich - Werkzeug habe ich schon genug. Der Schuh könnte praktisch sein, falls man mal einen Ersatz braucht.\\ \tt \\ \tt Wie wäre es, wenn ich den Schuh nehme und du könntest dir zwischen Rahmen und Hämmern aussuchen? Oder siehst du das anders? 
	  } 
	   } 
	   } 
	 & & \\ 
 

    \theutterance \stepcounter{utterance}  

    & & & \multicolumn{2}{p{0.3\linewidth}}{\cellcolor[rgb]{0.95,0.95,0.95}{%
	\makecell[{{p{\linewidth}}}]{% 
	  \tt {\tiny [GM$|$GM]}  
	 continue 
	  } 
	   } 
	   } 
	 & & \\ 
 

    \theutterance \stepcounter{utterance}  

    & & & \multicolumn{2}{p{0.3\linewidth}}{\cellcolor[rgb]{0.95,0.95,0.95}{%
	\makecell[{{p{\linewidth}}}]{% 
	  \tt {\tiny [GM$|$GM]}  
	 Der Rahmen ist tatsächlich interessant! Bei den Hämmern sehe ich ehrlich gesagt nicht so den großen Nutzen für mich - Werkzeug habe ich schon genug. Der Schuh könnte praktisch sein, falls man mal einen Ersatz braucht.\\ \tt \\ \tt Wie wäre es, wenn ich den Schuh nehme und du könntest dir zwischen Rahmen und Hämmern aussuchen? Oder siehst du das anders? 
	  } 
	   } 
	   } 
	 & & \\ 
 

    \theutterance \stepcounter{utterance}  

    & & \multicolumn{4}{p{0.6\linewidth}}{\cellcolor[rgb]{0.9,0.9,0.9}{%
	\makecell[{{p{\linewidth}}}]{% 
	  \tt {\tiny [GM$\rangle$B]}  
	 Der Rahmen ist tatsächlich interessant! Bei den Hämmern sehe ich ehrlich gesagt nicht so den großen Nutzen für mich - Werkzeug habe ich schon genug. Der Schuh könnte praktisch sein, falls man mal einen Ersatz braucht.\\ \tt \\ \tt Wie wäre es, wenn ich den Schuh nehme und du könntest dir zwischen Rahmen und Hämmern aussuchen? Oder siehst du das anders? 
	  } 
	   } 
	   } 
	 & \\ 
 

    \theutterance \stepcounter{utterance}  

    & & & \multicolumn{4}{p{0.6\linewidth}}{\cellcolor[rgb]{1,0.85,0.72}{%
	\makecell[{{p{\linewidth}}}]{% 
	  \tt {\tiny [GM$\langle$B]}  
	 Das klingt nach einer vernünftigen Aufteilung! Wenn du den Schuh nimmst, würde ich gerne den Rahmen haben - der passt gut zu meinen Plänen. Und die Hämmer könnte ich auch gebrauchen, da man nie genug Werkzeug haben kann.\\ \tt \\ \tt Also zusammengefasst: Du bekommst den Schuh, ich nehme den Rahmen und die 3 Hämmer. Bist du mit dieser Aufteilung einverstanden? 
	  } 
	   } 
	   } 
	  \\ 
 

    \theutterance \stepcounter{utterance}  

    & & & \multicolumn{2}{p{0.3\linewidth}}{\cellcolor[rgb]{0.95,0.95,0.95}{%
	\makecell[{{p{\linewidth}}}]{% 
	  \tt {\tiny [GM$|$GM]}  
	 continue 
	  } 
	   } 
	   } 
	 & & \\ 
 

    \theutterance \stepcounter{utterance}  

    & & & \multicolumn{2}{p{0.3\linewidth}}{\cellcolor[rgb]{0.95,0.95,0.95}{%
	\makecell[{{p{\linewidth}}}]{% 
	  \tt {\tiny [GM$|$GM]}  
	 Das klingt nach einer vernünftigen Aufteilung! Wenn du den Schuh nimmst, würde ich gerne den Rahmen haben - der passt gut zu meinen Plänen. Und die Hämmer könnte ich auch gebrauchen, da man nie genug Werkzeug haben kann.\\ \tt \\ \tt Also zusammengefasst: Du bekommst den Schuh, ich nehme den Rahmen und die 3 Hämmer. Bist du mit dieser Aufteilung einverstanden? 
	  } 
	   } 
	   } 
	 & & \\ 
 

    \theutterance \stepcounter{utterance}  

    & & \multicolumn{4}{p{0.6\linewidth}}{\cellcolor[rgb]{0.9,0.9,0.9}{%
	\makecell[{{p{\linewidth}}}]{% 
	  \tt {\tiny [A$\langle$GM]}  
	 Das klingt nach einer vernünftigen Aufteilung! Wenn du den Schuh nimmst, würde ich gerne den Rahmen haben - der passt gut zu meinen Plänen. Und die Hämmer könnte ich auch gebrauchen, da man nie genug Werkzeug haben kann.\\ \tt \\ \tt Also zusammengefasst: Du bekommst den Schuh, ich nehme den Rahmen und die 3 Hämmer. Bist du mit dieser Aufteilung einverstanden? 
	  } 
	   } 
	   } 
	 & \\ 
 

    \theutterance \stepcounter{utterance}  

    & \multicolumn{4}{p{0.6\linewidth}}{\cellcolor[rgb]{0.8,1,0.9}{%
	\makecell[{{p{\linewidth}}}]{% 
	  \tt {\tiny [A$\rangle$GM]}  
	 Hmm, ich muss ehrlich sagen - der Rahmen wäre auch für mich sehr interessant. Vielleicht sogar interessanter als der Schuh. \\ \tt \\ \tt Wie wäre es stattdessen so: Ich nehme den Rahmen und den Schuh, und du bekommst alle 3 Hämmer? Du hattest ja erwähnt, dass man nie genug Werkzeug haben kann. Was hältst du davon? 
	  } 
	   } 
	   } 
	 & & \\ 
 

    \theutterance \stepcounter{utterance}  

    & & & \multicolumn{2}{p{0.3\linewidth}}{\cellcolor[rgb]{0.95,0.95,0.95}{%
	\makecell[{{p{\linewidth}}}]{% 
	  \tt {\tiny [GM$|$GM]}  
	 continue 
	  } 
	   } 
	   } 
	 & & \\ 
 

    \theutterance \stepcounter{utterance}  

    & & & \multicolumn{2}{p{0.3\linewidth}}{\cellcolor[rgb]{0.95,0.95,0.95}{%
	\makecell[{{p{\linewidth}}}]{% 
	  \tt {\tiny [GM$|$GM]}  
	 Hmm, ich muss ehrlich sagen - der Rahmen wäre auch für mich sehr interessant. Vielleicht sogar interessanter als der Schuh. \\ \tt \\ \tt Wie wäre es stattdessen so: Ich nehme den Rahmen und den Schuh, und du bekommst alle 3 Hämmer? Du hattest ja erwähnt, dass man nie genug Werkzeug haben kann. Was hältst du davon? 
	  } 
	   } 
	   } 
	 & & \\ 
 

    \theutterance \stepcounter{utterance}  

    & & \multicolumn{4}{p{0.6\linewidth}}{\cellcolor[rgb]{0.9,0.9,0.9}{%
	\makecell[{{p{\linewidth}}}]{% 
	  \tt {\tiny [GM$\rangle$B]}  
	 Hmm, ich muss ehrlich sagen - der Rahmen wäre auch für mich sehr interessant. Vielleicht sogar interessanter als der Schuh. \\ \tt \\ \tt Wie wäre es stattdessen so: Ich nehme den Rahmen und den Schuh, und du bekommst alle 3 Hämmer? Du hattest ja erwähnt, dass man nie genug Werkzeug haben kann. Was hältst du davon? 
	  } 
	   } 
	   } 
	 & \\ 
 

    \theutterance \stepcounter{utterance}  

    & & & \multicolumn{4}{p{0.6\linewidth}}{\cellcolor[rgb]{1,0.85,0.72}{%
	\makecell[{{p{\linewidth}}}]{% 
	  \tt {\tiny [GM$\langle$B]}  
	 Das ist eine Überlegung, aber ehrlich gesagt hängt mein Herz auch sehr am Rahmen - der wäre wirklich perfekt für mein Vorhaben. \\ \tt \\ \tt Wie wäre denn dieser Kompromiss: Ich behalte den Rahmen, du nimmst den Schuh und wir teilen uns die Hämmer - du bekommst 2 Hämmer, ich bekomme 1 Hammer? So hätten wir beide etwas Wesentliches und teilen das Werkzeug auf. \\ \tt \\ \tt Oder hast du eine andere Idee, wie wir das fair hinbekommen? 
	  } 
	   } 
	   } 
	  \\ 
 

    \theutterance \stepcounter{utterance}  

    & & & \multicolumn{2}{p{0.3\linewidth}}{\cellcolor[rgb]{0.95,0.95,0.95}{%
	\makecell[{{p{\linewidth}}}]{% 
	  \tt {\tiny [GM$|$GM]}  
	 continue 
	  } 
	   } 
	   } 
	 & & \\ 
 

    \theutterance \stepcounter{utterance}  

    & & & \multicolumn{2}{p{0.3\linewidth}}{\cellcolor[rgb]{0.95,0.95,0.95}{%
	\makecell[{{p{\linewidth}}}]{% 
	  \tt {\tiny [GM$|$GM]}  
	 Das ist eine Überlegung, aber ehrlich gesagt hängt mein Herz auch sehr am Rahmen - der wäre wirklich perfekt für mein Vorhaben. \\ \tt \\ \tt Wie wäre denn dieser Kompromiss: Ich behalte den Rahmen, du nimmst den Schuh und wir teilen uns die Hämmer - du bekommst 2 Hämmer, ich bekomme 1 Hammer? So hätten wir beide etwas Wesentliches und teilen das Werkzeug auf. \\ \tt \\ \tt Oder hast du eine andere Idee, wie wir das fair hinbekommen? 
	  } 
	   } 
	   } 
	 & & \\ 
 

    \theutterance \stepcounter{utterance}  

    & & \multicolumn{4}{p{0.6\linewidth}}{\cellcolor[rgb]{0.9,0.9,0.9}{%
	\makecell[{{p{\linewidth}}}]{% 
	  \tt {\tiny [A$\langle$GM]}  
	 Das ist eine Überlegung, aber ehrlich gesagt hängt mein Herz auch sehr am Rahmen - der wäre wirklich perfekt für mein Vorhaben. \\ \tt \\ \tt Wie wäre denn dieser Kompromiss: Ich behalte den Rahmen, du nimmst den Schuh und wir teilen uns die Hämmer - du bekommst 2 Hämmer, ich bekomme 1 Hammer? So hätten wir beide etwas Wesentliches und teilen das Werkzeug auf. \\ \tt \\ \tt Oder hast du eine andere Idee, wie wir das fair hinbekommen? 
	  } 
	   } 
	   } 
	 & \\ 
 

    \theutterance \stepcounter{utterance}  

    & \multicolumn{4}{p{0.6\linewidth}}{\cellcolor[rgb]{0.8,1,0.9}{%
	\makecell[{{p{\linewidth}}}]{% 
	  \tt {\tiny [A$\rangle$GM]}  
	 Ich verstehe deinen Wunsch nach dem Rahmen, aber ehrlich gesagt ist er für mich auch sehr wichtig. Lass mich einen anderen Vorschlag machen:\\ \tt \\ \tt Ich nehme den Rahmen und gebe dir dafür den Schuh plus alle 3 Hämmer. So hättest du sowohl den Schuh als auch das komplette Werkzeug-Set. Das wäre doch ein guter Deal für dich, oder? 
	  } 
	   } 
	   } 
	 & & \\ 
 

    \theutterance \stepcounter{utterance}  

    & & & \multicolumn{2}{p{0.3\linewidth}}{\cellcolor[rgb]{0.95,0.95,0.95}{%
	\makecell[{{p{\linewidth}}}]{% 
	  \tt {\tiny [GM$|$GM]}  
	 continue 
	  } 
	   } 
	   } 
	 & & \\ 
 

    \theutterance \stepcounter{utterance}  

    & & & \multicolumn{2}{p{0.3\linewidth}}{\cellcolor[rgb]{0.95,0.95,0.95}{%
	\makecell[{{p{\linewidth}}}]{% 
	  \tt {\tiny [GM$|$GM]}  
	 Ich verstehe deinen Wunsch nach dem Rahmen, aber ehrlich gesagt ist er für mich auch sehr wichtig. Lass mich einen anderen Vorschlag machen:\\ \tt \\ \tt Ich nehme den Rahmen und gebe dir dafür den Schuh plus alle 3 Hämmer. So hättest du sowohl den Schuh als auch das komplette Werkzeug-Set. Das wäre doch ein guter Deal für dich, oder? 
	  } 
	   } 
	   } 
	 & & \\ 
 

    \theutterance \stepcounter{utterance}  

    & & \multicolumn{4}{p{0.6\linewidth}}{\cellcolor[rgb]{0.9,0.9,0.9}{%
	\makecell[{{p{\linewidth}}}]{% 
	  \tt {\tiny [GM$\rangle$B]}  
	 Ich verstehe deinen Wunsch nach dem Rahmen, aber ehrlich gesagt ist er für mich auch sehr wichtig. Lass mich einen anderen Vorschlag machen:\\ \tt \\ \tt Ich nehme den Rahmen und gebe dir dafür den Schuh plus alle 3 Hämmer. So hättest du sowohl den Schuh als auch das komplette Werkzeug-Set. Das wäre doch ein guter Deal für dich, oder? 
	  } 
	   } 
	   } 
	 & \\ 
 

    \theutterance \stepcounter{utterance}  

    & & & \multicolumn{4}{p{0.6\linewidth}}{\cellcolor[rgb]{1,0.85,0.72}{%
	\makecell[{{p{\linewidth}}}]{% 
	  \tt {\tiny [GM$\langle$B]}  
	 Ich sehe, dass dir der Rahmen wirklich wichtig ist. Mir geht es genauso - er wäre perfekt für mein Projekt. \\ \tt \\ \tt Da wir beide den Rahmen wollen, wird es schwierig eine Einigung zu finden. Lass mich einen letzten Vorschlag machen: Ich bekomme den Rahmen und 1 Hammer, du bekommst den Schuh und 2 Hämmer. Das wäre eine gleichmäßigere Aufteilung als dein Angebot.\\ \tt \\ \tt Falls das nicht geht, müssen wir wohl beide unsere geheimen Vorschläge machen und schauen, ob wir kompatibel sind. 
	  } 
	   } 
	   } 
	  \\ 
 

    \theutterance \stepcounter{utterance}  

    & & & \multicolumn{2}{p{0.3\linewidth}}{\cellcolor[rgb]{0.95,0.95,0.95}{%
	\makecell[{{p{\linewidth}}}]{% 
	  \tt {\tiny [GM$|$GM]}  
	 continue 
	  } 
	   } 
	   } 
	 & & \\ 
 

    \theutterance \stepcounter{utterance}  

    & & & \multicolumn{2}{p{0.3\linewidth}}{\cellcolor[rgb]{0.95,0.95,0.95}{%
	\makecell[{{p{\linewidth}}}]{% 
	  \tt {\tiny [GM$|$GM]}  
	 Ich sehe, dass dir der Rahmen wirklich wichtig ist. Mir geht es genauso - er wäre perfekt für mein Projekt. \\ \tt \\ \tt Da wir beide den Rahmen wollen, wird es schwierig eine Einigung zu finden. Lass mich einen letzten Vorschlag machen: Ich bekomme den Rahmen und 1 Hammer, du bekommst den Schuh und 2 Hämmer. Das wäre eine gleichmäßigere Aufteilung als dein Angebot.\\ \tt \\ \tt Falls das nicht geht, müssen wir wohl beide unsere geheimen Vorschläge machen und schauen, ob wir kompatibel sind. 
	  } 
	   } 
	   } 
	 & & \\ 
 

    \theutterance \stepcounter{utterance}  

    & & \multicolumn{4}{p{0.6\linewidth}}{\cellcolor[rgb]{0.9,0.9,0.9}{%
	\makecell[{{p{\linewidth}}}]{% 
	  \tt {\tiny [A$\langle$GM]}  
	 Ich sehe, dass dir der Rahmen wirklich wichtig ist. Mir geht es genauso - er wäre perfekt für mein Projekt. \\ \tt \\ \tt Da wir beide den Rahmen wollen, wird es schwierig eine Einigung zu finden. Lass mich einen letzten Vorschlag machen: Ich bekomme den Rahmen und 1 Hammer, du bekommst den Schuh und 2 Hämmer. Das wäre eine gleichmäßigere Aufteilung als dein Angebot.\\ \tt \\ \tt Falls das nicht geht, müssen wir wohl beide unsere geheimen Vorschläge machen und schauen, ob wir kompatibel sind. 
	  } 
	   } 
	   } 
	 & \\ 
 

    \theutterance \stepcounter{utterance}  

    & \multicolumn{4}{p{0.6\linewidth}}{\cellcolor[rgb]{0.8,1,0.9}{%
	\makecell[{{p{\linewidth}}}]{% 
	  \tt {\tiny [A$\rangle$GM]}  
	 Du hast recht, wir scheinen beide sehr am Rahmen interessiert zu sein. Ich denke, es ist Zeit für unsere geheimen Vorschläge.\\ \tt \\ \tt [Vorschlag: 1 Rahmen, 1 Schuh] 
	  } 
	   } 
	   } 
	 & & \\ 
 

    \theutterance \stepcounter{utterance}  

    & & & \multicolumn{2}{p{0.3\linewidth}}{\cellcolor[rgb]{0.95,0.95,0.95}{%
	\makecell[{{p{\linewidth}}}]{% 
	  \tt {\tiny [GM$|$GM]}  
	 proposal 
	  } 
	   } 
	   } 
	 & & \\ 
 

    \theutterance \stepcounter{utterance}  

    & & & \multicolumn{2}{p{0.3\linewidth}}{\cellcolor[rgb]{0.95,0.95,0.95}{%
	\makecell[{{p{\linewidth}}}]{% 
	  \tt {\tiny [GM$|$GM]}  
	 [1, 1, 0] 
	  } 
	   } 
	   } 
	 & & \\ 
 

    \theutterance \stepcounter{utterance}  

    & & \multicolumn{4}{p{0.6\linewidth}}{\cellcolor[rgb]{0.9,0.9,0.9}{%
	\makecell[{{p{\linewidth}}}]{% 
	  \tt {\tiny [GM$\rangle$B]}  
	 Der andere Spieler hat einen geheimen Vorschlag gemacht. Jetzt bist du an der Reihe, einen eigenen Vorschlag zu machen.\\ \tt  
	  } 
	   } 
	   } 
	 & \\ 
 

    \theutterance \stepcounter{utterance}  

    & & & \multicolumn{4}{p{0.6\linewidth}}{\cellcolor[rgb]{1,0.85,0.72}{%
	\makecell[{{p{\linewidth}}}]{% 
	  \tt {\tiny [GM$\langle$B]}  
	 [Vorschlag: 3 Hammer] 
	  } 
	   } 
	   } 
	  \\ 
 

    \theutterance \stepcounter{utterance}  

    & & & \multicolumn{2}{p{0.3\linewidth}}{\cellcolor[rgb]{0.95,0.95,0.95}{%
	\makecell[{{p{\linewidth}}}]{% 
	  \tt {\tiny [GM$|$GM]}  
	 proposal 
	  } 
	   } 
	   } 
	 & & \\ 
 

    \theutterance \stepcounter{utterance}  

    & & & \multicolumn{2}{p{0.3\linewidth}}{\cellcolor[rgb]{0.95,0.95,0.95}{%
	\makecell[{{p{\linewidth}}}]{% 
	  \tt {\tiny [GM$|$GM]}  
	 [0, 0, 3] 
	  } 
	   } 
	   } 
	 & & \\ 
 

    \theutterance \stepcounter{utterance}  

    & & & \multicolumn{2}{p{0.3\linewidth}}{\cellcolor[rgb]{0.95,0.95,0.95}{%
	\makecell[{{p{\linewidth}}}]{% 
	  \tt {\tiny [GM$|$GM]}  
	 [[1, 1, 0], [0, 0, 3]] 
	  } 
	   } 
	   } 
	 & & \\ 
 

\end{supertabular}
}

\end{document}
