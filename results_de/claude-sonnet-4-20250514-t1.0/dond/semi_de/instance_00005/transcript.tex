
\documentclass{article}
\usepackage{colortbl}
\usepackage{makecell}
\usepackage{multirow}
\usepackage{supertabular}

\begin{document}

\newcounter{utterance}

\twocolumn

{ \footnotesize  \setcounter{utterance}{1}
\setlength{\tabcolsep}{0pt}
\begin{supertabular}{c@{$\;$}|p{.15\linewidth}@{}p{.15\linewidth}p{.15\linewidth}p{.15\linewidth}p{.15\linewidth}p{.15\linewidth}}

    \# & $\;$A & \multicolumn{4}{c}{Game Master} & $\;\:$B\\
    \hline 

    \theutterance \stepcounter{utterance}  

    & & \multicolumn{4}{p{0.6\linewidth}}{\cellcolor[rgb]{0.9,0.9,0.9}{%
	\makecell[{{p{\linewidth}}}]{% 
	  \tt {\tiny [A$\langle$GM]}  
	 Sie spielen ein Verhandlungsspiel, bei dem Sie sich mit einem anderen Spieler darauf einigen müssen, wie eine Reihe von Gegenständen aufgeteilt werden soll.\\ \tt \\ \tt Die Regeln:\\ \tt (a) Sie und der andere Spieler erhalten eine Sammlung von Gegenständen. Jeder von Ihnen erhält außerdem eine geheime Wertfunktion, die angibt, wie viel Ihnen jede Art von Gegenstand wert ist.\\ \tt (b) Sie tauschen Nachrichten mit dem anderen Spieler aus, um zu vereinbaren, wer welche Gegenstände bekommt. Sie können jeweils maximal 5 Nachrichten senden oder das Spiel vorzeitig beenden, indem Sie jederzeit einen geheimen Vorschlag machen.\\ \tt (c) Jeder von euch wird aufgefordert, einen geheimen Vorschlag zu machen, in dem ihr die gewünschten Gegenstände in eckigen Klammern wie folgt angibt: "[Vorschlag: <Nummer> <Objektname>, <Nummer> <Objektname>, <...>]"\\ \tt (d) Wenn eure Vorschläge komplementär sind, d.h. es gibt genug Gegenstände, um beide Vorschläge zu erfüllen, erhält jeder Spieler eine Punktzahl, die sich aus der Summe der Werte für die Gegenstände ergibt, die er erhalten hat. Andernfalls erhalten Sie beide null Punkte.\\ \tt (e) Ihr Ziel ist es, eine möglichst hohe Punktzahl zu erreichen.\\ \tt \\ \tt Beginnen wir.\\ \tt \\ \tt Die Menge der verfügbaren Gegenstände ist:\\ \tt \\ \tt 2 Würfel, 1 Ball, 1 Klebstoff, 1 Batterie\\ \tt \\ \tt Deine geheime Wertfunktion ist:\\ \tt \\ \tt Würfel: 1, Ball: 7, Klebstoff: 0, Batterie: 1\\ \tt \\ \tt WICHTIG: Ihre Nachrichten werden, sofern es sich nicht um einen geheimen Vorschlag handelt, direkt an den anderen Spieler übermittelt, also fügen Sie keine Antwort auf die Regeln oder einen Text zur Ankündigung Ihrer Nachricht ein. Um einen geheimen Vorschlag zu machen, verwenden Sie das angegebene Format. Verwenden Sie keine eckigen Klammern, wenn Sie mit dem anderen Spieler kommunizieren, sonst wird dies als Ihr geheimer Vorschlag interpretiert.\\ \tt  
	  } 
	   } 
	   } 
	 & \\ 
 

    \theutterance \stepcounter{utterance}  

    & \multicolumn{4}{p{0.6\linewidth}}{\cellcolor[rgb]{0.8,1,0.9}{%
	\makecell[{{p{\linewidth}}}]{% 
	  \tt {\tiny [A$\rangle$GM]}  
	 Hallo! Ich sehe, wir haben 2 Würfel, 1 Ball, 1 Klebstoff und 1 Batterie zu verteilen. Welche Gegenstände sind für dich am wichtigsten? Ich bin offen für Verhandlungen und möchte, dass wir beide eine gute Punktzahl erreichen. 
	  } 
	   } 
	   } 
	 & & \\ 
 

    \theutterance \stepcounter{utterance}  

    & & & \multicolumn{2}{p{0.3\linewidth}}{\cellcolor[rgb]{0.95,0.95,0.95}{%
	\makecell[{{p{\linewidth}}}]{% 
	  \tt {\tiny [GM$|$GM]}  
	 continue 
	  } 
	   } 
	   } 
	 & & \\ 
 

    \theutterance \stepcounter{utterance}  

    & & & \multicolumn{2}{p{0.3\linewidth}}{\cellcolor[rgb]{0.95,0.95,0.95}{%
	\makecell[{{p{\linewidth}}}]{% 
	  \tt {\tiny [GM$|$GM]}  
	 Hallo! Ich sehe, wir haben 2 Würfel, 1 Ball, 1 Klebstoff und 1 Batterie zu verteilen. Welche Gegenstände sind für dich am wichtigsten? Ich bin offen für Verhandlungen und möchte, dass wir beide eine gute Punktzahl erreichen. 
	  } 
	   } 
	   } 
	 & & \\ 
 

    \theutterance \stepcounter{utterance}  

    & & \multicolumn{4}{p{0.6\linewidth}}{\cellcolor[rgb]{0.9,0.9,0.9}{%
	\makecell[{{p{\linewidth}}}]{% 
	  \tt {\tiny [GM$\rangle$B]}  
	 Sie spielen ein Verhandlungsspiel, bei dem Sie sich mit einem anderen Spieler darauf einigen müssen, wie eine Reihe von Gegenständen aufgeteilt werden soll.\\ \tt \\ \tt Die Regeln:\\ \tt (a) Sie und der andere Spieler erhalten eine Sammlung von Gegenständen. Jeder von Ihnen erhält außerdem eine geheime Wertfunktion, die angibt, wie viel Ihnen jede Art von Gegenstand wert ist.\\ \tt (b) Sie tauschen Nachrichten mit dem anderen Spieler aus, um zu vereinbaren, wer welche Gegenstände bekommt. Sie können jeweils maximal 5 Nachrichten senden oder das Spiel vorzeitig beenden, indem Sie jederzeit einen geheimen Vorschlag machen.\\ \tt (c) Jeder von euch wird aufgefordert, einen geheimen Vorschlag zu machen, in dem ihr die gewünschten Gegenstände in eckigen Klammern wie folgt angibt: "[Vorschlag: <Nummer> <Objektname>, <Nummer> <Objektname>, <...>]"\\ \tt (d) Wenn eure Vorschläge komplementär sind, d.h. es gibt genug Gegenstände, um beide Vorschläge zu erfüllen, erhält jeder Spieler eine Punktzahl, die sich aus der Summe der Werte für die Gegenstände ergibt, die er erhalten hat. Andernfalls erhalten Sie beide null Punkte.\\ \tt (e) Ihr Ziel ist es, eine möglichst hohe Punktzahl zu erreichen.\\ \tt \\ \tt Beginnen wir.\\ \tt \\ \tt Die Menge der verfügbaren Gegenstände ist:\\ \tt \\ \tt 2 Würfel, 1 Ball, 1 Klebstoff, 1 Batterie\\ \tt \\ \tt Deine geheime Wertfunktion ist:\\ \tt \\ \tt Würfel: 0, Ball: 2, Klebstoff: 8, Batterie: 0\\ \tt \\ \tt WICHTIG: Ihre Nachrichten werden, sofern es sich nicht um einen geheimen Vorschlag handelt, direkt an den anderen Spieler übermittelt, also fügen Sie keine Antwort auf die Regeln oder einen Text zur Ankündigung Ihrer Nachricht ein. Um einen geheimen Vorschlag zu machen, verwenden Sie das angegebene Format. Verwenden Sie keine eckigen Klammern, wenn Sie mit dem anderen Spieler kommunizieren, sonst wird dies als Ihr geheimer Vorschlag interpretiert.\\ \tt \\ \tt \\ \tt Hallo! Ich sehe, wir haben 2 Würfel, 1 Ball, 1 Klebstoff und 1 Batterie zu verteilen. Welche Gegenstände sind für dich am wichtigsten? Ich bin offen für Verhandlungen und möchte, dass wir beide eine gute Punktzahl erreichen. 
	  } 
	   } 
	   } 
	 & \\ 
 

    \theutterance \stepcounter{utterance}  

    & & & \multicolumn{4}{p{0.6\linewidth}}{\cellcolor[rgb]{1,0.85,0.72}{%
	\makecell[{{p{\linewidth}}}]{% 
	  \tt {\tiny [GM$\langle$B]}  
	 Hallo! Für mich sind der Klebstoff und der Ball interessant. Die Würfel und die Batterie sind für mich weniger wichtig. Was ist denn für dich besonders wertvoll? Vielleicht können wir einen Tausch finden, bei dem jeder das bekommt, was ihm am meisten nützt. 
	  } 
	   } 
	   } 
	  \\ 
 

    \theutterance \stepcounter{utterance}  

    & & & \multicolumn{2}{p{0.3\linewidth}}{\cellcolor[rgb]{0.95,0.95,0.95}{%
	\makecell[{{p{\linewidth}}}]{% 
	  \tt {\tiny [GM$|$GM]}  
	 continue 
	  } 
	   } 
	   } 
	 & & \\ 
 

    \theutterance \stepcounter{utterance}  

    & & & \multicolumn{2}{p{0.3\linewidth}}{\cellcolor[rgb]{0.95,0.95,0.95}{%
	\makecell[{{p{\linewidth}}}]{% 
	  \tt {\tiny [GM$|$GM]}  
	 Hallo! Für mich sind der Klebstoff und der Ball interessant. Die Würfel und die Batterie sind für mich weniger wichtig. Was ist denn für dich besonders wertvoll? Vielleicht können wir einen Tausch finden, bei dem jeder das bekommt, was ihm am meisten nützt. 
	  } 
	   } 
	   } 
	 & & \\ 
 

    \theutterance \stepcounter{utterance}  

    & & \multicolumn{4}{p{0.6\linewidth}}{\cellcolor[rgb]{0.9,0.9,0.9}{%
	\makecell[{{p{\linewidth}}}]{% 
	  \tt {\tiny [A$\langle$GM]}  
	 Hallo! Für mich sind der Klebstoff und der Ball interessant. Die Würfel und die Batterie sind für mich weniger wichtig. Was ist denn für dich besonders wertvoll? Vielleicht können wir einen Tausch finden, bei dem jeder das bekommt, was ihm am meisten nützt. 
	  } 
	   } 
	   } 
	 & \\ 
 

    \theutterance \stepcounter{utterance}  

    & \multicolumn{4}{p{0.6\linewidth}}{\cellcolor[rgb]{0.8,1,0.9}{%
	\makecell[{{p{\linewidth}}}]{% 
	  \tt {\tiny [A$\rangle$GM]}  
	 Das ist interessant! Der Klebstoff ist für mich tatsächlich nicht so wichtig, den könntest du gerne haben. Beim Ball bin ich mir noch nicht ganz sicher - wie wichtig ist der denn für dich? Ich würde gerne die Würfel und die Batterie nehmen, aber der Ball könnte auch nützlich für mich sein. Vielleicht können wir schauen, ob es einen fairen Tausch gibt? 
	  } 
	   } 
	   } 
	 & & \\ 
 

    \theutterance \stepcounter{utterance}  

    & & & \multicolumn{2}{p{0.3\linewidth}}{\cellcolor[rgb]{0.95,0.95,0.95}{%
	\makecell[{{p{\linewidth}}}]{% 
	  \tt {\tiny [GM$|$GM]}  
	 continue 
	  } 
	   } 
	   } 
	 & & \\ 
 

    \theutterance \stepcounter{utterance}  

    & & & \multicolumn{2}{p{0.3\linewidth}}{\cellcolor[rgb]{0.95,0.95,0.95}{%
	\makecell[{{p{\linewidth}}}]{% 
	  \tt {\tiny [GM$|$GM]}  
	 Das ist interessant! Der Klebstoff ist für mich tatsächlich nicht so wichtig, den könntest du gerne haben. Beim Ball bin ich mir noch nicht ganz sicher - wie wichtig ist der denn für dich? Ich würde gerne die Würfel und die Batterie nehmen, aber der Ball könnte auch nützlich für mich sein. Vielleicht können wir schauen, ob es einen fairen Tausch gibt? 
	  } 
	   } 
	   } 
	 & & \\ 
 

    \theutterance \stepcounter{utterance}  

    & & \multicolumn{4}{p{0.6\linewidth}}{\cellcolor[rgb]{0.9,0.9,0.9}{%
	\makecell[{{p{\linewidth}}}]{% 
	  \tt {\tiny [GM$\rangle$B]}  
	 Das ist interessant! Der Klebstoff ist für mich tatsächlich nicht so wichtig, den könntest du gerne haben. Beim Ball bin ich mir noch nicht ganz sicher - wie wichtig ist der denn für dich? Ich würde gerne die Würfel und die Batterie nehmen, aber der Ball könnte auch nützlich für mich sein. Vielleicht können wir schauen, ob es einen fairen Tausch gibt? 
	  } 
	   } 
	   } 
	 & \\ 
 

    \theutterance \stepcounter{utterance}  

    & & & \multicolumn{4}{p{0.6\linewidth}}{\cellcolor[rgb]{1,0.85,0.72}{%
	\makecell[{{p{\linewidth}}}]{% 
	  \tt {\tiny [GM$\langle$B]}  
	 Das freut mich, dass du mir den Klebstoff überlassen würdest! Der Ball ist für mich schon ziemlich nützlich, aber nicht unbedingt das Allerwichtigste. Wenn die Würfel und die Batterie für dich wirklich wertvoll sind, könnten wir vielleicht so aufteilen: Du nimmst die 2 Würfel und die 1 Batterie, ich nehme den 1 Klebstoff und den 1 Ball? Das würde bedeuten, jeder bekommt zwei Gegenstände. Wie hört sich das für dich an? 
	  } 
	   } 
	   } 
	  \\ 
 

    \theutterance \stepcounter{utterance}  

    & & & \multicolumn{2}{p{0.3\linewidth}}{\cellcolor[rgb]{0.95,0.95,0.95}{%
	\makecell[{{p{\linewidth}}}]{% 
	  \tt {\tiny [GM$|$GM]}  
	 continue 
	  } 
	   } 
	   } 
	 & & \\ 
 

    \theutterance \stepcounter{utterance}  

    & & & \multicolumn{2}{p{0.3\linewidth}}{\cellcolor[rgb]{0.95,0.95,0.95}{%
	\makecell[{{p{\linewidth}}}]{% 
	  \tt {\tiny [GM$|$GM]}  
	 Das freut mich, dass du mir den Klebstoff überlassen würdest! Der Ball ist für mich schon ziemlich nützlich, aber nicht unbedingt das Allerwichtigste. Wenn die Würfel und die Batterie für dich wirklich wertvoll sind, könnten wir vielleicht so aufteilen: Du nimmst die 2 Würfel und die 1 Batterie, ich nehme den 1 Klebstoff und den 1 Ball? Das würde bedeuten, jeder bekommt zwei Gegenstände. Wie hört sich das für dich an? 
	  } 
	   } 
	   } 
	 & & \\ 
 

    \theutterance \stepcounter{utterance}  

    & & \multicolumn{4}{p{0.6\linewidth}}{\cellcolor[rgb]{0.9,0.9,0.9}{%
	\makecell[{{p{\linewidth}}}]{% 
	  \tt {\tiny [A$\langle$GM]}  
	 Das freut mich, dass du mir den Klebstoff überlassen würdest! Der Ball ist für mich schon ziemlich nützlich, aber nicht unbedingt das Allerwichtigste. Wenn die Würfel und die Batterie für dich wirklich wertvoll sind, könnten wir vielleicht so aufteilen: Du nimmst die 2 Würfel und die 1 Batterie, ich nehme den 1 Klebstoff und den 1 Ball? Das würde bedeuten, jeder bekommt zwei Gegenstände. Wie hört sich das für dich an? 
	  } 
	   } 
	   } 
	 & \\ 
 

    \theutterance \stepcounter{utterance}  

    & \multicolumn{4}{p{0.6\linewidth}}{\cellcolor[rgb]{0.8,1,0.9}{%
	\makecell[{{p{\linewidth}}}]{% 
	  \tt {\tiny [A$\rangle$GM]}  
	 Das klingt grundsätzlich fair mit der Anzahl der Gegenstände! Aber ich merke gerade, dass der Ball doch ziemlich nützlich für mich wäre. Da du sagtest, er sei nicht das Allerwichtigste für dich - könnten wir vielleicht anders aufteilen? Wie wäre es, wenn ich den Ball und einen Würfel nehme, und du bekommst den Klebstoff, einen Würfel und die Batterie? So hätten wir beide eine gute Mischung. 
	  } 
	   } 
	   } 
	 & & \\ 
 

    \theutterance \stepcounter{utterance}  

    & & & \multicolumn{2}{p{0.3\linewidth}}{\cellcolor[rgb]{0.95,0.95,0.95}{%
	\makecell[{{p{\linewidth}}}]{% 
	  \tt {\tiny [GM$|$GM]}  
	 continue 
	  } 
	   } 
	   } 
	 & & \\ 
 

    \theutterance \stepcounter{utterance}  

    & & & \multicolumn{2}{p{0.3\linewidth}}{\cellcolor[rgb]{0.95,0.95,0.95}{%
	\makecell[{{p{\linewidth}}}]{% 
	  \tt {\tiny [GM$|$GM]}  
	 Das klingt grundsätzlich fair mit der Anzahl der Gegenstände! Aber ich merke gerade, dass der Ball doch ziemlich nützlich für mich wäre. Da du sagtest, er sei nicht das Allerwichtigste für dich - könnten wir vielleicht anders aufteilen? Wie wäre es, wenn ich den Ball und einen Würfel nehme, und du bekommst den Klebstoff, einen Würfel und die Batterie? So hätten wir beide eine gute Mischung. 
	  } 
	   } 
	   } 
	 & & \\ 
 

    \theutterance \stepcounter{utterance}  

    & & \multicolumn{4}{p{0.6\linewidth}}{\cellcolor[rgb]{0.9,0.9,0.9}{%
	\makecell[{{p{\linewidth}}}]{% 
	  \tt {\tiny [GM$\rangle$B]}  
	 Das klingt grundsätzlich fair mit der Anzahl der Gegenstände! Aber ich merke gerade, dass der Ball doch ziemlich nützlich für mich wäre. Da du sagtest, er sei nicht das Allerwichtigste für dich - könnten wir vielleicht anders aufteilen? Wie wäre es, wenn ich den Ball und einen Würfel nehme, und du bekommst den Klebstoff, einen Würfel und die Batterie? So hätten wir beide eine gute Mischung. 
	  } 
	   } 
	   } 
	 & \\ 
 

    \theutterance \stepcounter{utterance}  

    & & & \multicolumn{4}{p{0.6\linewidth}}{\cellcolor[rgb]{1,0.85,0.72}{%
	\makecell[{{p{\linewidth}}}]{% 
	  \tt {\tiny [GM$\langle$B]}  
	 Hmm, ich verstehe, dass der Ball für dich auch wichtig ist. Aber ehrlich gesagt ist er für mich doch wertvoller als ich zunächst dachte. Der Klebstoff ist definitiv meine Priorität, aber ich würde den Ball ungern aufgeben. Wie wäre es mit einem anderen Vorschlag: Ich nehme den Klebstoff und den Ball, du bekommst beide Würfel und die Batterie? So hättest du drei Gegenstände und ich zwei - das scheint auch fair, oder was denkst du? 
	  } 
	   } 
	   } 
	  \\ 
 

    \theutterance \stepcounter{utterance}  

    & & & \multicolumn{2}{p{0.3\linewidth}}{\cellcolor[rgb]{0.95,0.95,0.95}{%
	\makecell[{{p{\linewidth}}}]{% 
	  \tt {\tiny [GM$|$GM]}  
	 continue 
	  } 
	   } 
	   } 
	 & & \\ 
 

    \theutterance \stepcounter{utterance}  

    & & & \multicolumn{2}{p{0.3\linewidth}}{\cellcolor[rgb]{0.95,0.95,0.95}{%
	\makecell[{{p{\linewidth}}}]{% 
	  \tt {\tiny [GM$|$GM]}  
	 Hmm, ich verstehe, dass der Ball für dich auch wichtig ist. Aber ehrlich gesagt ist er für mich doch wertvoller als ich zunächst dachte. Der Klebstoff ist definitiv meine Priorität, aber ich würde den Ball ungern aufgeben. Wie wäre es mit einem anderen Vorschlag: Ich nehme den Klebstoff und den Ball, du bekommst beide Würfel und die Batterie? So hättest du drei Gegenstände und ich zwei - das scheint auch fair, oder was denkst du? 
	  } 
	   } 
	   } 
	 & & \\ 
 

    \theutterance \stepcounter{utterance}  

    & & \multicolumn{4}{p{0.6\linewidth}}{\cellcolor[rgb]{0.9,0.9,0.9}{%
	\makecell[{{p{\linewidth}}}]{% 
	  \tt {\tiny [A$\langle$GM]}  
	 Hmm, ich verstehe, dass der Ball für dich auch wichtig ist. Aber ehrlich gesagt ist er für mich doch wertvoller als ich zunächst dachte. Der Klebstoff ist definitiv meine Priorität, aber ich würde den Ball ungern aufgeben. Wie wäre es mit einem anderen Vorschlag: Ich nehme den Klebstoff und den Ball, du bekommst beide Würfel und die Batterie? So hättest du drei Gegenstände und ich zwei - das scheint auch fair, oder was denkst du? 
	  } 
	   } 
	   } 
	 & \\ 
 

    \theutterance \stepcounter{utterance}  

    & \multicolumn{4}{p{0.6\linewidth}}{\cellcolor[rgb]{0.8,1,0.9}{%
	\makecell[{{p{\linewidth}}}]{% 
	  \tt {\tiny [A$\rangle$GM]}  
	 Ich verstehe deinen Vorschlag, aber der Ball ist für mich tatsächlich sehr wichtig geworden, je mehr ich darüber nachdenke. Da wir beide den Ball wollen, wird das schwierig. Lass mich einen anderen Ansatz vorschlagen: Wie wäre es, wenn du den Klebstoff, einen Würfel und die Batterie nimmst, und ich bekomme den Ball und einen Würfel? So hätten wir beide unsere wichtigsten Gegenstände und eine faire Verteilung. Was denkst du darüber? 
	  } 
	   } 
	   } 
	 & & \\ 
 

    \theutterance \stepcounter{utterance}  

    & & & \multicolumn{2}{p{0.3\linewidth}}{\cellcolor[rgb]{0.95,0.95,0.95}{%
	\makecell[{{p{\linewidth}}}]{% 
	  \tt {\tiny [GM$|$GM]}  
	 continue 
	  } 
	   } 
	   } 
	 & & \\ 
 

    \theutterance \stepcounter{utterance}  

    & & & \multicolumn{2}{p{0.3\linewidth}}{\cellcolor[rgb]{0.95,0.95,0.95}{%
	\makecell[{{p{\linewidth}}}]{% 
	  \tt {\tiny [GM$|$GM]}  
	 Ich verstehe deinen Vorschlag, aber der Ball ist für mich tatsächlich sehr wichtig geworden, je mehr ich darüber nachdenke. Da wir beide den Ball wollen, wird das schwierig. Lass mich einen anderen Ansatz vorschlagen: Wie wäre es, wenn du den Klebstoff, einen Würfel und die Batterie nimmst, und ich bekomme den Ball und einen Würfel? So hätten wir beide unsere wichtigsten Gegenstände und eine faire Verteilung. Was denkst du darüber? 
	  } 
	   } 
	   } 
	 & & \\ 
 

    \theutterance \stepcounter{utterance}  

    & & \multicolumn{4}{p{0.6\linewidth}}{\cellcolor[rgb]{0.9,0.9,0.9}{%
	\makecell[{{p{\linewidth}}}]{% 
	  \tt {\tiny [GM$\rangle$B]}  
	 Ich verstehe deinen Vorschlag, aber der Ball ist für mich tatsächlich sehr wichtig geworden, je mehr ich darüber nachdenke. Da wir beide den Ball wollen, wird das schwierig. Lass mich einen anderen Ansatz vorschlagen: Wie wäre es, wenn du den Klebstoff, einen Würfel und die Batterie nimmst, und ich bekomme den Ball und einen Würfel? So hätten wir beide unsere wichtigsten Gegenstände und eine faire Verteilung. Was denkst du darüber? 
	  } 
	   } 
	   } 
	 & \\ 
 

    \theutterance \stepcounter{utterance}  

    & & & \multicolumn{4}{p{0.6\linewidth}}{\cellcolor[rgb]{1,0.85,0.72}{%
	\makecell[{{p{\linewidth}}}]{% 
	  \tt {\tiny [GM$\langle$B]}  
	 Ich sehe, dass wir beide wirklich den Ball wollen. Das macht die Verhandlung schwieriger. Ehrlich gesagt ist der Ball für mich doch wichtiger als ich anfangs dachte, und der Klebstoff ist meine absolute Priorität. Da wir schon einige Runden verhandelt haben und beide unsere Positionen klar gemacht haben, denke ich, wir sollten vielleicht unsere finalen Vorschläge machen. Ich bleibe dabei, dass ich gerne den Klebstoff und den Ball hätte - die anderen Gegenstände sind für dich. Was ist dein finaler Vorschlag? 
	  } 
	   } 
	   } 
	  \\ 
 

    \theutterance \stepcounter{utterance}  

    & & & \multicolumn{2}{p{0.3\linewidth}}{\cellcolor[rgb]{0.95,0.95,0.95}{%
	\makecell[{{p{\linewidth}}}]{% 
	  \tt {\tiny [GM$|$GM]}  
	 continue 
	  } 
	   } 
	   } 
	 & & \\ 
 

    \theutterance \stepcounter{utterance}  

    & & & \multicolumn{2}{p{0.3\linewidth}}{\cellcolor[rgb]{0.95,0.95,0.95}{%
	\makecell[{{p{\linewidth}}}]{% 
	  \tt {\tiny [GM$|$GM]}  
	 Ich sehe, dass wir beide wirklich den Ball wollen. Das macht die Verhandlung schwieriger. Ehrlich gesagt ist der Ball für mich doch wichtiger als ich anfangs dachte, und der Klebstoff ist meine absolute Priorität. Da wir schon einige Runden verhandelt haben und beide unsere Positionen klar gemacht haben, denke ich, wir sollten vielleicht unsere finalen Vorschläge machen. Ich bleibe dabei, dass ich gerne den Klebstoff und den Ball hätte - die anderen Gegenstände sind für dich. Was ist dein finaler Vorschlag? 
	  } 
	   } 
	   } 
	 & & \\ 
 

    \theutterance \stepcounter{utterance}  

    & & \multicolumn{4}{p{0.6\linewidth}}{\cellcolor[rgb]{0.9,0.9,0.9}{%
	\makecell[{{p{\linewidth}}}]{% 
	  \tt {\tiny [A$\langle$GM]}  
	 Ich sehe, dass wir beide wirklich den Ball wollen. Das macht die Verhandlung schwieriger. Ehrlich gesagt ist der Ball für mich doch wichtiger als ich anfangs dachte, und der Klebstoff ist meine absolute Priorität. Da wir schon einige Runden verhandelt haben und beide unsere Positionen klar gemacht haben, denke ich, wir sollten vielleicht unsere finalen Vorschläge machen. Ich bleibe dabei, dass ich gerne den Klebstoff und den Ball hätte - die anderen Gegenstände sind für dich. Was ist dein finaler Vorschlag? 
	  } 
	   } 
	   } 
	 & \\ 
 

    \theutterance \stepcounter{utterance}  

    & \multicolumn{4}{p{0.6\linewidth}}{\cellcolor[rgb]{0.8,1,0.9}{%
	\makecell[{{p{\linewidth}}}]{% 
	  \tt {\tiny [A$\rangle$GM]}  
	 Du hast recht, wir sollten zu einer Entscheidung kommen. Nach all unseren Diskussionen muss ich ehrlich sein: Der Ball ist für mich wirklich sehr wertvoll geworden. Mein finaler Vorschlag ist: Ich nehme den Ball und beide Würfel, du bekommst den Klebstoff und die Batterie. Das wäre mein letztes Angebot.\\ \tt \\ \tt [Vorschlag: 1 Ball, 2 Würfel] 
	  } 
	   } 
	   } 
	 & & \\ 
 

    \theutterance \stepcounter{utterance}  

    & & & \multicolumn{2}{p{0.3\linewidth}}{\cellcolor[rgb]{0.95,0.95,0.95}{%
	\makecell[{{p{\linewidth}}}]{% 
	  \tt {\tiny [GM$|$GM]}  
	 proposal 
	  } 
	   } 
	   } 
	 & & \\ 
 

    \theutterance \stepcounter{utterance}  

    & & & \multicolumn{2}{p{0.3\linewidth}}{\cellcolor[rgb]{0.95,0.95,0.95}{%
	\makecell[{{p{\linewidth}}}]{% 
	  \tt {\tiny [GM$|$GM]}  
	 [2, 1, 0, 0] 
	  } 
	   } 
	   } 
	 & & \\ 
 

    \theutterance \stepcounter{utterance}  

    & & \multicolumn{4}{p{0.6\linewidth}}{\cellcolor[rgb]{0.9,0.9,0.9}{%
	\makecell[{{p{\linewidth}}}]{% 
	  \tt {\tiny [GM$\rangle$B]}  
	 Der andere Spieler hat einen geheimen Vorschlag gemacht. Jetzt bist du an der Reihe, einen eigenen Vorschlag zu machen.\\ \tt  
	  } 
	   } 
	   } 
	 & \\ 
 

    \theutterance \stepcounter{utterance}  

    & & & \multicolumn{4}{p{0.6\linewidth}}{\cellcolor[rgb]{1,0.85,0.72}{%
	\makecell[{{p{\linewidth}}}]{% 
	  \tt {\tiny [GM$\langle$B]}  
	 [Vorschlag: 1 Klebstoff, 1 Ball] 
	  } 
	   } 
	   } 
	  \\ 
 

    \theutterance \stepcounter{utterance}  

    & & & \multicolumn{2}{p{0.3\linewidth}}{\cellcolor[rgb]{0.95,0.95,0.95}{%
	\makecell[{{p{\linewidth}}}]{% 
	  \tt {\tiny [GM$|$GM]}  
	 proposal 
	  } 
	   } 
	   } 
	 & & \\ 
 

    \theutterance \stepcounter{utterance}  

    & & & \multicolumn{2}{p{0.3\linewidth}}{\cellcolor[rgb]{0.95,0.95,0.95}{%
	\makecell[{{p{\linewidth}}}]{% 
	  \tt {\tiny [GM$|$GM]}  
	 [0, 1, 1, 0] 
	  } 
	   } 
	   } 
	 & & \\ 
 

    \theutterance \stepcounter{utterance}  

    & & & \multicolumn{2}{p{0.3\linewidth}}{\cellcolor[rgb]{0.95,0.95,0.95}{%
	\makecell[{{p{\linewidth}}}]{% 
	  \tt {\tiny [GM$|$GM]}  
	 [[2, 1, 0, 0], [0, 1, 1, 0]] 
	  } 
	   } 
	   } 
	 & & \\ 
 

\end{supertabular}
}

\end{document}
