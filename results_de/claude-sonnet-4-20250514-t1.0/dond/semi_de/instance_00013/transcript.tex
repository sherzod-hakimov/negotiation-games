
\documentclass{article}
\usepackage{colortbl}
\usepackage{makecell}
\usepackage{multirow}
\usepackage{supertabular}

\begin{document}

\newcounter{utterance}

\centering \large Interaction Transcript for game `dond', experiment `semi\_de', episode 13 with claude{-}sonnet{-}4{-}20250514{-}t1.0.
\vspace{24pt}

{ \footnotesize  \setcounter{utterance}{1}
\setlength{\tabcolsep}{0pt}
\begin{supertabular}{c@{$\;$}|p{.15\linewidth}@{}p{.15\linewidth}p{.15\linewidth}p{.15\linewidth}p{.15\linewidth}p{.15\linewidth}}
    \# & $\;$A & \multicolumn{4}{c}{Game Master} & $\;\:$B\\
    \hline

    \theutterance \stepcounter{utterance}  
    & & \multicolumn{4}{p{0.6\linewidth}}{
        \cellcolor[rgb]{0.9,0.9,0.9}{
            \makecell[{{p{\linewidth}}}]{
                \texttt{\tiny{[P1$\langle$GM]}}
                \texttt{Sie spielen ein Verhandlungsspiel, bei dem Sie sich mit einem anderen Spieler darauf einigen müssen, wie eine Reihe von Gegenständen aufgeteilt werden soll.} \\
\\ 
\texttt{Die Regeln:} \\
\texttt{(a) Sie und der andere Spieler erhalten eine Sammlung von Gegenständen. Jeder von Ihnen erhält außerdem eine geheime Wertfunktion, die angibt, wie viel Ihnen jede Art von Gegenstand wert ist.} \\
\texttt{(b) Sie tauschen Nachrichten mit dem anderen Spieler aus, um zu vereinbaren, wer welche Gegenstände bekommt. Sie können jeweils maximal 5 Nachrichten senden oder das Spiel vorzeitig beenden, indem Sie jederzeit einen geheimen Vorschlag machen.} \\
\texttt{(c) Jeder von euch wird aufgefordert, einen geheimen Vorschlag zu machen, in dem ihr die gewünschten Gegenstände in eckigen Klammern wie folgt angibt: "{[}Vorschlag: <Nummer> <Objektname>, <Nummer> <Objektname>, <...>{]}"} \\
\texttt{(d) Wenn eure Vorschläge komplementär sind, d.h. es gibt genug Gegenstände, um beide Vorschläge zu erfüllen, erhält jeder Spieler eine Punktzahl, die sich aus der Summe der Werte für die Gegenstände ergibt, die er erhalten hat. Andernfalls erhalten Sie beide null Punkte.} \\
\texttt{(e) Ihr Ziel ist es, eine möglichst hohe Punktzahl zu erreichen.} \\
\\ 
\texttt{Beginnen wir.} \\
\\ 
\texttt{Die Menge der verfügbaren Gegenstände ist:} \\
\\ 
\texttt{4 Papiere, 2 Uhren, 2 Kartoffeln} \\
\\ 
\texttt{Deine geheime Wertfunktion ist:} \\
\\ 
\texttt{Papier: 1, Uhr: 1, Kartoffel: 2} \\
\\ 
\texttt{WICHTIG: Ihre Nachrichten werden, sofern es sich nicht um einen geheimen Vorschlag handelt, direkt an den anderen Spieler übermittelt, also fügen Sie keine Antwort auf die Regeln oder einen Text zur Ankündigung Ihrer Nachricht ein. Um einen geheimen Vorschlag zu machen, verwenden Sie das angegebene Format. Verwenden Sie keine eckigen Klammern, wenn Sie mit dem anderen Spieler kommunizieren, sonst wird dies als Ihr geheimer Vorschlag interpretiert.} \\
            }
        }
    }
    & \\ \\

    \theutterance \stepcounter{utterance}  
    & \multicolumn{4}{p{0.6\linewidth}}{
        \cellcolor[rgb]{0.9,0.9,0.9}{
            \makecell[{{p{\linewidth}}}]{
                \texttt{\tiny{[P1$\rangle$GM]}}
                \texttt{Hallo! Ich sehe, wir haben 4 Papiere, 2 Uhren und 2 Kartoffeln zu verteilen. Was sind denn deine Prioritäten? Welche Gegenstände sind dir am wichtigsten?} \\
            }
        }
    }
    & & \\ \\

    \theutterance \stepcounter{utterance}  
    & & & \multicolumn{2}{p{0.3\linewidth}}{
        \cellcolor[rgb]{0.9,0.9,0.9}{
            \makecell[{{p{\linewidth}}}]{
                \texttt{\tiny{[GM$|$GM]}}
                \texttt{continue} \\
            }
        }
    }
    & & \\ \\

    \theutterance \stepcounter{utterance}  
    & & & \multicolumn{2}{p{0.3\linewidth}}{
        \cellcolor[rgb]{0.9,0.9,0.9}{
            \makecell[{{p{\linewidth}}}]{
                \texttt{\tiny{[GM$|$GM]}}
                \texttt{Hallo! Ich sehe, wir haben 4 Papiere, 2 Uhren und 2 Kartoffeln zu verteilen. Was sind denn deine Prioritäten? Welche Gegenstände sind dir am wichtigsten?} \\
            }
        }
    }
    & & \\ \\

    \theutterance \stepcounter{utterance}  
    & & \multicolumn{4}{p{0.6\linewidth}}{
        \cellcolor[rgb]{0.9,0.9,0.9}{
            \makecell[{{p{\linewidth}}}]{
                \texttt{\tiny{[GM$\rangle$P2]}}
                \texttt{Sie spielen ein Verhandlungsspiel, bei dem Sie sich mit einem anderen Spieler darauf einigen müssen, wie eine Reihe von Gegenständen aufgeteilt werden soll.} \\
\\ 
\texttt{Die Regeln:} \\
\texttt{(a) Sie und der andere Spieler erhalten eine Sammlung von Gegenständen. Jeder von Ihnen erhält außerdem eine geheime Wertfunktion, die angibt, wie viel Ihnen jede Art von Gegenstand wert ist.} \\
\texttt{(b) Sie tauschen Nachrichten mit dem anderen Spieler aus, um zu vereinbaren, wer welche Gegenstände bekommt. Sie können jeweils maximal 5 Nachrichten senden oder das Spiel vorzeitig beenden, indem Sie jederzeit einen geheimen Vorschlag machen.} \\
\texttt{(c) Jeder von euch wird aufgefordert, einen geheimen Vorschlag zu machen, in dem ihr die gewünschten Gegenstände in eckigen Klammern wie folgt angibt: "{[}Vorschlag: <Nummer> <Objektname>, <Nummer> <Objektname>, <...>{]}"} \\
\texttt{(d) Wenn eure Vorschläge komplementär sind, d.h. es gibt genug Gegenstände, um beide Vorschläge zu erfüllen, erhält jeder Spieler eine Punktzahl, die sich aus der Summe der Werte für die Gegenstände ergibt, die er erhalten hat. Andernfalls erhalten Sie beide null Punkte.} \\
\texttt{(e) Ihr Ziel ist es, eine möglichst hohe Punktzahl zu erreichen.} \\
\\ 
\texttt{Beginnen wir.} \\
\\ 
\texttt{Die Menge der verfügbaren Gegenstände ist:} \\
\\ 
\texttt{4 Papiere, 2 Uhren, 2 Kartoffeln} \\
\\ 
\texttt{Deine geheime Wertfunktion ist:} \\
\\ 
\texttt{Papier: 0, Uhr: 4, Kartoffel: 1} \\
\\ 
\texttt{WICHTIG: Ihre Nachrichten werden, sofern es sich nicht um einen geheimen Vorschlag handelt, direkt an den anderen Spieler übermittelt, also fügen Sie keine Antwort auf die Regeln oder einen Text zur Ankündigung Ihrer Nachricht ein. Um einen geheimen Vorschlag zu machen, verwenden Sie das angegebene Format. Verwenden Sie keine eckigen Klammern, wenn Sie mit dem anderen Spieler kommunizieren, sonst wird dies als Ihr geheimer Vorschlag interpretiert.} \\
\\ 
\\ 
\texttt{Hallo! Ich sehe, wir haben 4 Papiere, 2 Uhren und 2 Kartoffeln zu verteilen. Was sind denn deine Prioritäten? Welche Gegenstände sind dir am wichtigsten?} \\
            }
        }
    }
    & \\ \\

    \theutterance \stepcounter{utterance}  
    & & & \multicolumn{4}{p{0.6\linewidth}}{
        \cellcolor[rgb]{0.9,0.9,0.9}{
            \makecell[{{p{\linewidth}}}]{
                \texttt{\tiny{[GM$\langle$P2]}}
                \texttt{Hallo! Ja, das ist eine interessante Aufteilung. Ich finde die Uhren ziemlich interessant {-} sie scheinen wertvoll zu sein. Bei den Kartoffeln bin ich etwas unentschlossen, und die Papiere... nun ja, davon gibt es ja reichlich.} \\
\\ 
\texttt{Was denkst du denn über die verschiedenen Gegenstände? Hast du bestimmte Präferenzen?} \\
            }
        }
    }
     \\ \\

    \theutterance \stepcounter{utterance}  
    & & & \multicolumn{2}{p{0.3\linewidth}}{
        \cellcolor[rgb]{0.9,0.9,0.9}{
            \makecell[{{p{\linewidth}}}]{
                \texttt{\tiny{[GM$|$GM]}}
                \texttt{continue} \\
            }
        }
    }
    & & \\ \\

    \theutterance \stepcounter{utterance}  
    & & & \multicolumn{2}{p{0.3\linewidth}}{
        \cellcolor[rgb]{0.9,0.9,0.9}{
            \makecell[{{p{\linewidth}}}]{
                \texttt{\tiny{[GM$|$GM]}}
                \texttt{Hallo! Ja, das ist eine interessante Aufteilung. Ich finde die Uhren ziemlich interessant {-} sie scheinen wertvoll zu sein. Bei den Kartoffeln bin ich etwas unentschlossen, und die Papiere... nun ja, davon gibt es ja reichlich.} \\
\\ 
\texttt{Was denkst du denn über die verschiedenen Gegenstände? Hast du bestimmte Präferenzen?} \\
            }
        }
    }
    & & \\ \\

    \theutterance \stepcounter{utterance}  
    & & \multicolumn{4}{p{0.6\linewidth}}{
        \cellcolor[rgb]{0.9,0.9,0.9}{
            \makecell[{{p{\linewidth}}}]{
                \texttt{\tiny{[P1$\langle$GM]}}
                \texttt{Hallo! Ja, das ist eine interessante Aufteilung. Ich finde die Uhren ziemlich interessant {-} sie scheinen wertvoll zu sein. Bei den Kartoffeln bin ich etwas unentschlossen, und die Papiere... nun ja, davon gibt es ja reichlich.} \\
\\ 
\texttt{Was denkst du denn über die verschiedenen Gegenstände? Hast du bestimmte Präferenzen?} \\
            }
        }
    }
    & \\ \\

    \theutterance \stepcounter{utterance}  
    & \multicolumn{4}{p{0.6\linewidth}}{
        \cellcolor[rgb]{0.9,0.9,0.9}{
            \makecell[{{p{\linewidth}}}]{
                \texttt{\tiny{[P1$\rangle$GM]}}
                \texttt{Interessant! Ja, die Uhren sehen definitiv wertvoll aus. Ich kann verstehen, warum die dein Interesse wecken.} \\
\\ 
\texttt{Was die Kartoffeln angeht {-} ich finde sie eigentlich ganz praktisch, falls du nicht so sehr daran interessiert bist? Die Papiere sind tatsächlich in größerer Menge da, da könnten wir uns vielleicht aufteilen.} \\
\\ 
\texttt{Wie stellst du dir denn eine faire Aufteilung vor? Wenn du die Uhren nimmst, wäre ich bereit, über die Kartoffeln zu sprechen, und bei den Papieren könnten wir jeweils die Hälfte nehmen?} \\
            }
        }
    }
    & & \\ \\

    \theutterance \stepcounter{utterance}  
    & & & \multicolumn{2}{p{0.3\linewidth}}{
        \cellcolor[rgb]{0.9,0.9,0.9}{
            \makecell[{{p{\linewidth}}}]{
                \texttt{\tiny{[GM$|$GM]}}
                \texttt{continue} \\
            }
        }
    }
    & & \\ \\

    \theutterance \stepcounter{utterance}  
    & & & \multicolumn{2}{p{0.3\linewidth}}{
        \cellcolor[rgb]{0.9,0.9,0.9}{
            \makecell[{{p{\linewidth}}}]{
                \texttt{\tiny{[GM$|$GM]}}
                \texttt{Interessant! Ja, die Uhren sehen definitiv wertvoll aus. Ich kann verstehen, warum die dein Interesse wecken.} \\
\\ 
\texttt{Was die Kartoffeln angeht {-} ich finde sie eigentlich ganz praktisch, falls du nicht so sehr daran interessiert bist? Die Papiere sind tatsächlich in größerer Menge da, da könnten wir uns vielleicht aufteilen.} \\
\\ 
\texttt{Wie stellst du dir denn eine faire Aufteilung vor? Wenn du die Uhren nimmst, wäre ich bereit, über die Kartoffeln zu sprechen, und bei den Papieren könnten wir jeweils die Hälfte nehmen?} \\
            }
        }
    }
    & & \\ \\

    \theutterance \stepcounter{utterance}  
    & & \multicolumn{4}{p{0.6\linewidth}}{
        \cellcolor[rgb]{0.9,0.9,0.9}{
            \makecell[{{p{\linewidth}}}]{
                \texttt{\tiny{[GM$\rangle$P2]}}
                \texttt{Interessant! Ja, die Uhren sehen definitiv wertvoll aus. Ich kann verstehen, warum die dein Interesse wecken.} \\
\\ 
\texttt{Was die Kartoffeln angeht {-} ich finde sie eigentlich ganz praktisch, falls du nicht so sehr daran interessiert bist? Die Papiere sind tatsächlich in größerer Menge da, da könnten wir uns vielleicht aufteilen.} \\
\\ 
\texttt{Wie stellst du dir denn eine faire Aufteilung vor? Wenn du die Uhren nimmst, wäre ich bereit, über die Kartoffeln zu sprechen, und bei den Papieren könnten wir jeweils die Hälfte nehmen?} \\
            }
        }
    }
    & \\ \\

    \theutterance \stepcounter{utterance}  
    & & & \multicolumn{4}{p{0.6\linewidth}}{
        \cellcolor[rgb]{0.9,0.9,0.9}{
            \makecell[{{p{\linewidth}}}]{
                \texttt{\tiny{[GM$\langle$P2]}}
                \texttt{Das klingt nach einem sehr vernünftigen Vorschlag! Ich finde es gut, dass du praktisch denkst {-} die Kartoffeln können sicherlich nützlich sein.} \\
\\ 
\texttt{Wenn ich die Uhren nehme und du die Kartoffeln, dann macht die Aufteilung der Papiere definitiv Sinn. Bei 4 Papieren könnten wir tatsächlich jeweils 2 nehmen.} \\
\\ 
\texttt{Aber ich frage mich {-} wäre es für dich auch in Ordnung, wenn ich nur die Uhren nehme und du den Rest bekommst? Also du würdest alle 4 Papiere und beide Kartoffeln erhalten, und ich nur die 2 Uhren? Mir geht es wirklich hauptsächlich um die Uhren.} \\
            }
        }
    }
     \\ \\

    \theutterance \stepcounter{utterance}  
    & & & \multicolumn{2}{p{0.3\linewidth}}{
        \cellcolor[rgb]{0.9,0.9,0.9}{
            \makecell[{{p{\linewidth}}}]{
                \texttt{\tiny{[GM$|$GM]}}
                \texttt{continue} \\
            }
        }
    }
    & & \\ \\

    \theutterance \stepcounter{utterance}  
    & & & \multicolumn{2}{p{0.3\linewidth}}{
        \cellcolor[rgb]{0.9,0.9,0.9}{
            \makecell[{{p{\linewidth}}}]{
                \texttt{\tiny{[GM$|$GM]}}
                \texttt{Das klingt nach einem sehr vernünftigen Vorschlag! Ich finde es gut, dass du praktisch denkst {-} die Kartoffeln können sicherlich nützlich sein.} \\
\\ 
\texttt{Wenn ich die Uhren nehme und du die Kartoffeln, dann macht die Aufteilung der Papiere definitiv Sinn. Bei 4 Papieren könnten wir tatsächlich jeweils 2 nehmen.} \\
\\ 
\texttt{Aber ich frage mich {-} wäre es für dich auch in Ordnung, wenn ich nur die Uhren nehme und du den Rest bekommst? Also du würdest alle 4 Papiere und beide Kartoffeln erhalten, und ich nur die 2 Uhren? Mir geht es wirklich hauptsächlich um die Uhren.} \\
            }
        }
    }
    & & \\ \\

    \theutterance \stepcounter{utterance}  
    & & \multicolumn{4}{p{0.6\linewidth}}{
        \cellcolor[rgb]{0.9,0.9,0.9}{
            \makecell[{{p{\linewidth}}}]{
                \texttt{\tiny{[P1$\langle$GM]}}
                \texttt{Das klingt nach einem sehr vernünftigen Vorschlag! Ich finde es gut, dass du praktisch denkst {-} die Kartoffeln können sicherlich nützlich sein.} \\
\\ 
\texttt{Wenn ich die Uhren nehme und du die Kartoffeln, dann macht die Aufteilung der Papiere definitiv Sinn. Bei 4 Papieren könnten wir tatsächlich jeweils 2 nehmen.} \\
\\ 
\texttt{Aber ich frage mich {-} wäre es für dich auch in Ordnung, wenn ich nur die Uhren nehme und du den Rest bekommst? Also du würdest alle 4 Papiere und beide Kartoffeln erhalten, und ich nur die 2 Uhren? Mir geht es wirklich hauptsächlich um die Uhren.} \\
            }
        }
    }
    & \\ \\

    \theutterance \stepcounter{utterance}  
    & \multicolumn{4}{p{0.6\linewidth}}{
        \cellcolor[rgb]{0.9,0.9,0.9}{
            \makecell[{{p{\linewidth}}}]{
                \texttt{\tiny{[P1$\rangle$GM]}}
                \texttt{Das ist ein sehr großzügiges Angebot von dir! Wenn dir die Uhren wirklich so wichtig sind, dann bin ich durchaus bereit, diese Aufteilung zu akzeptieren.} \\
\\ 
\texttt{Also du nimmst die 2 Uhren und ich bekomme alle 4 Papiere und beide Kartoffeln {-} das scheint mir fair zu sein, besonders da du so klar deine Prioritäten kommuniziert hast.} \\
\\ 
\texttt{Soll ich das als unsere finale Vereinbarung verstehen?} \\
            }
        }
    }
    & & \\ \\

    \theutterance \stepcounter{utterance}  
    & & & \multicolumn{2}{p{0.3\linewidth}}{
        \cellcolor[rgb]{0.9,0.9,0.9}{
            \makecell[{{p{\linewidth}}}]{
                \texttt{\tiny{[GM$|$GM]}}
                \texttt{continue} \\
            }
        }
    }
    & & \\ \\

    \theutterance \stepcounter{utterance}  
    & & & \multicolumn{2}{p{0.3\linewidth}}{
        \cellcolor[rgb]{0.9,0.9,0.9}{
            \makecell[{{p{\linewidth}}}]{
                \texttt{\tiny{[GM$|$GM]}}
                \texttt{Das ist ein sehr großzügiges Angebot von dir! Wenn dir die Uhren wirklich so wichtig sind, dann bin ich durchaus bereit, diese Aufteilung zu akzeptieren.} \\
\\ 
\texttt{Also du nimmst die 2 Uhren und ich bekomme alle 4 Papiere und beide Kartoffeln {-} das scheint mir fair zu sein, besonders da du so klar deine Prioritäten kommuniziert hast.} \\
\\ 
\texttt{Soll ich das als unsere finale Vereinbarung verstehen?} \\
            }
        }
    }
    & & \\ \\

    \theutterance \stepcounter{utterance}  
    & & \multicolumn{4}{p{0.6\linewidth}}{
        \cellcolor[rgb]{0.9,0.9,0.9}{
            \makecell[{{p{\linewidth}}}]{
                \texttt{\tiny{[GM$\rangle$P2]}}
                \texttt{Das ist ein sehr großzügiges Angebot von dir! Wenn dir die Uhren wirklich so wichtig sind, dann bin ich durchaus bereit, diese Aufteilung zu akzeptieren.} \\
\\ 
\texttt{Also du nimmst die 2 Uhren und ich bekomme alle 4 Papiere und beide Kartoffeln {-} das scheint mir fair zu sein, besonders da du so klar deine Prioritäten kommuniziert hast.} \\
\\ 
\texttt{Soll ich das als unsere finale Vereinbarung verstehen?} \\
            }
        }
    }
    & \\ \\

    \theutterance \stepcounter{utterance}  
    & & & \multicolumn{4}{p{0.6\linewidth}}{
        \cellcolor[rgb]{0.9,0.9,0.9}{
            \makecell[{{p{\linewidth}}}]{
                \texttt{\tiny{[GM$\langle$P2]}}
                \texttt{Ja, das sehe ich auch als unsere finale Vereinbarung! Du bekommst alle Papiere und Kartoffeln, ich bekomme die Uhren {-} das funktioniert perfekt für uns beide.} \\
\\ 
\texttt{Danke für die konstruktive Verhandlung!} \\
\\ 
\texttt{{[}Vorschlag: 2 Uhr{]}} \\
            }
        }
    }
     \\ \\

    \theutterance \stepcounter{utterance}  
    & & & \multicolumn{2}{p{0.3\linewidth}}{
        \cellcolor[rgb]{0.9,0.9,0.9}{
            \makecell[{{p{\linewidth}}}]{
                \texttt{\tiny{[GM$|$GM]}}
                \texttt{proposal} \\
            }
        }
    }
    & & \\ \\

    \theutterance \stepcounter{utterance}  
    & & & \multicolumn{2}{p{0.3\linewidth}}{
        \cellcolor[rgb]{0.9,0.9,0.9}{
            \makecell[{{p{\linewidth}}}]{
                \texttt{\tiny{[GM$|$GM]}}
                [0, 2, 0]
            }
        }
    }
    & & \\ \\

    \theutterance \stepcounter{utterance}  
    & & \multicolumn{4}{p{0.6\linewidth}}{
        \cellcolor[rgb]{0.9,0.9,0.9}{
            \makecell[{{p{\linewidth}}}]{
                \texttt{\tiny{[P1$\langle$GM]}}
                \texttt{Der andere Spieler hat einen geheimen Vorschlag gemacht. Jetzt bist du an der Reihe, einen eigenen Vorschlag zu machen.} \\
            }
        }
    }
    & \\ \\

    \theutterance \stepcounter{utterance}  
    & \multicolumn{4}{p{0.6\linewidth}}{
        \cellcolor[rgb]{0.9,0.9,0.9}{
            \makecell[{{p{\linewidth}}}]{
                \texttt{\tiny{[P1$\rangle$GM]}}
                \texttt{{[}Vorschlag: 4 Papiere, 2 Kartoffeln{]}} \\
            }
        }
    }
    & & \\ \\

    \theutterance \stepcounter{utterance}  
    & & & \multicolumn{2}{p{0.3\linewidth}}{
        \cellcolor[rgb]{0.9,0.9,0.9}{
            \makecell[{{p{\linewidth}}}]{
                \texttt{\tiny{[GM$|$GM]}}
                \texttt{proposal} \\
            }
        }
    }
    & & \\ \\

    \theutterance \stepcounter{utterance}  
    & & & \multicolumn{2}{p{0.3\linewidth}}{
        \cellcolor[rgb]{0.9,0.9,0.9}{
            \makecell[{{p{\linewidth}}}]{
                \texttt{\tiny{[GM$|$GM]}}
                [4, 0, 2]
            }
        }
    }
    & & \\ \\

    \theutterance \stepcounter{utterance}  
    & & & \multicolumn{2}{p{0.3\linewidth}}{
        \cellcolor[rgb]{0.9,0.9,0.9}{
            \makecell[{{p{\linewidth}}}]{
                \texttt{\tiny{[GM$|$GM]}}
                [[4, 0, 2], [0, 2, 0]]
            }
        }
    }
    & & \\ \\

\end{supertabular}
}

\end{document}
