
\documentclass{article}
\usepackage{colortbl}
\usepackage{makecell}
\usepackage{multirow}
\usepackage{supertabular}

\begin{document}

\newcounter{utterance}

\centering \large Interaction Transcript for game `dond', experiment `semi\_de', episode 18 with claude{-}sonnet{-}4{-}20250514{-}t1.0.
\vspace{24pt}

{ \footnotesize  \setcounter{utterance}{1}
\setlength{\tabcolsep}{0pt}
\begin{supertabular}{c@{$\;$}|p{.15\linewidth}@{}p{.15\linewidth}p{.15\linewidth}p{.15\linewidth}p{.15\linewidth}p{.15\linewidth}}
    \# & $\;$A & \multicolumn{4}{c}{Game Master} & $\;\:$B\\
    \hline

    \theutterance \stepcounter{utterance}  
    & & \multicolumn{4}{p{0.6\linewidth}}{
        \cellcolor[rgb]{0.9,0.9,0.9}{
            \makecell[{{p{\linewidth}}}]{
                \texttt{\tiny{[P1$\langle$GM]}}
                \texttt{Sie spielen ein Verhandlungsspiel, bei dem Sie sich mit einem anderen Spieler darauf einigen müssen, wie eine Reihe von Gegenständen aufgeteilt werden soll.} \\
\\ 
\texttt{Die Regeln:} \\
\texttt{(a) Sie und der andere Spieler erhalten eine Sammlung von Gegenständen. Jeder von Ihnen erhält außerdem eine geheime Wertfunktion, die angibt, wie viel Ihnen jede Art von Gegenstand wert ist.} \\
\texttt{(b) Sie tauschen Nachrichten mit dem anderen Spieler aus, um zu vereinbaren, wer welche Gegenstände bekommt. Sie können jeweils maximal 5 Nachrichten senden oder das Spiel vorzeitig beenden, indem Sie jederzeit einen geheimen Vorschlag machen.} \\
\texttt{(c) Jeder von euch wird aufgefordert, einen geheimen Vorschlag zu machen, in dem ihr die gewünschten Gegenstände in eckigen Klammern wie folgt angibt: "{[}Vorschlag: <Nummer> <Objektname>, <Nummer> <Objektname>, <...>{]}"} \\
\texttt{(d) Wenn eure Vorschläge komplementär sind, d.h. es gibt genug Gegenstände, um beide Vorschläge zu erfüllen, erhält jeder Spieler eine Punktzahl, die sich aus der Summe der Werte für die Gegenstände ergibt, die er erhalten hat. Andernfalls erhalten Sie beide null Punkte.} \\
\texttt{(e) Ihr Ziel ist es, eine möglichst hohe Punktzahl zu erreichen.} \\
\\ 
\texttt{Beginnen wir.} \\
\\ 
\texttt{Die Menge der verfügbaren Gegenstände ist:} \\
\\ 
\texttt{1 Stock, 1 Handtuch, 1 Spielzeug, 2 Kisten} \\
\\ 
\texttt{Deine geheime Wertfunktion ist:} \\
\\ 
\texttt{Stock: 5, Handtuch: 2, Spielzeug: 1, Kiste: 1} \\
\\ 
\texttt{WICHTIG: Ihre Nachrichten werden, sofern es sich nicht um einen geheimen Vorschlag handelt, direkt an den anderen Spieler übermittelt, also fügen Sie keine Antwort auf die Regeln oder einen Text zur Ankündigung Ihrer Nachricht ein. Um einen geheimen Vorschlag zu machen, verwenden Sie das angegebene Format. Verwenden Sie keine eckigen Klammern, wenn Sie mit dem anderen Spieler kommunizieren, sonst wird dies als Ihr geheimer Vorschlag interpretiert.} \\
            }
        }
    }
    & \\ \\

    \theutterance \stepcounter{utterance}  
    & \multicolumn{4}{p{0.6\linewidth}}{
        \cellcolor[rgb]{0.9,0.9,0.9}{
            \makecell[{{p{\linewidth}}}]{
                \texttt{\tiny{[P1$\rangle$GM]}}
                \texttt{Hallo! Ich sehe, wir haben einen Stock, ein Handtuch, ein Spielzeug und zwei Kisten zu verteilen. Was sind denn für Sie die interessantesten Gegenstände? Ich würde gerne verstehen, was Ihnen besonders wichtig ist, damit wir eine faire Aufteilung finden können.} \\
            }
        }
    }
    & & \\ \\

    \theutterance \stepcounter{utterance}  
    & & & \multicolumn{2}{p{0.3\linewidth}}{
        \cellcolor[rgb]{0.9,0.9,0.9}{
            \makecell[{{p{\linewidth}}}]{
                \texttt{\tiny{[GM$|$GM]}}
                \texttt{continue} \\
            }
        }
    }
    & & \\ \\

    \theutterance \stepcounter{utterance}  
    & & & \multicolumn{2}{p{0.3\linewidth}}{
        \cellcolor[rgb]{0.9,0.9,0.9}{
            \makecell[{{p{\linewidth}}}]{
                \texttt{\tiny{[GM$|$GM]}}
                \texttt{Hallo! Ich sehe, wir haben einen Stock, ein Handtuch, ein Spielzeug und zwei Kisten zu verteilen. Was sind denn für Sie die interessantesten Gegenstände? Ich würde gerne verstehen, was Ihnen besonders wichtig ist, damit wir eine faire Aufteilung finden können.} \\
            }
        }
    }
    & & \\ \\

    \theutterance \stepcounter{utterance}  
    & & \multicolumn{4}{p{0.6\linewidth}}{
        \cellcolor[rgb]{0.9,0.9,0.9}{
            \makecell[{{p{\linewidth}}}]{
                \texttt{\tiny{[GM$\rangle$P2]}}
                \texttt{Sie spielen ein Verhandlungsspiel, bei dem Sie sich mit einem anderen Spieler darauf einigen müssen, wie eine Reihe von Gegenständen aufgeteilt werden soll.} \\
\\ 
\texttt{Die Regeln:} \\
\texttt{(a) Sie und der andere Spieler erhalten eine Sammlung von Gegenständen. Jeder von Ihnen erhält außerdem eine geheime Wertfunktion, die angibt, wie viel Ihnen jede Art von Gegenstand wert ist.} \\
\texttt{(b) Sie tauschen Nachrichten mit dem anderen Spieler aus, um zu vereinbaren, wer welche Gegenstände bekommt. Sie können jeweils maximal 5 Nachrichten senden oder das Spiel vorzeitig beenden, indem Sie jederzeit einen geheimen Vorschlag machen.} \\
\texttt{(c) Jeder von euch wird aufgefordert, einen geheimen Vorschlag zu machen, in dem ihr die gewünschten Gegenstände in eckigen Klammern wie folgt angibt: "{[}Vorschlag: <Nummer> <Objektname>, <Nummer> <Objektname>, <...>{]}"} \\
\texttt{(d) Wenn eure Vorschläge komplementär sind, d.h. es gibt genug Gegenstände, um beide Vorschläge zu erfüllen, erhält jeder Spieler eine Punktzahl, die sich aus der Summe der Werte für die Gegenstände ergibt, die er erhalten hat. Andernfalls erhalten Sie beide null Punkte.} \\
\texttt{(e) Ihr Ziel ist es, eine möglichst hohe Punktzahl zu erreichen.} \\
\\ 
\texttt{Beginnen wir.} \\
\\ 
\texttt{Die Menge der verfügbaren Gegenstände ist:} \\
\\ 
\texttt{1 Stock, 1 Handtuch, 1 Spielzeug, 2 Kisten} \\
\\ 
\texttt{Deine geheime Wertfunktion ist:} \\
\\ 
\texttt{Stock: 7, Handtuch: 3, Spielzeug: 0, Kiste: 0} \\
\\ 
\texttt{WICHTIG: Ihre Nachrichten werden, sofern es sich nicht um einen geheimen Vorschlag handelt, direkt an den anderen Spieler übermittelt, also fügen Sie keine Antwort auf die Regeln oder einen Text zur Ankündigung Ihrer Nachricht ein. Um einen geheimen Vorschlag zu machen, verwenden Sie das angegebene Format. Verwenden Sie keine eckigen Klammern, wenn Sie mit dem anderen Spieler kommunizieren, sonst wird dies als Ihr geheimer Vorschlag interpretiert.} \\
\\ 
\\ 
\texttt{Hallo! Ich sehe, wir haben einen Stock, ein Handtuch, ein Spielzeug und zwei Kisten zu verteilen. Was sind denn für Sie die interessantesten Gegenstände? Ich würde gerne verstehen, was Ihnen besonders wichtig ist, damit wir eine faire Aufteilung finden können.} \\
            }
        }
    }
    & \\ \\

    \theutterance \stepcounter{utterance}  
    & & & \multicolumn{4}{p{0.6\linewidth}}{
        \cellcolor[rgb]{0.9,0.9,0.9}{
            \makecell[{{p{\linewidth}}}]{
                \texttt{\tiny{[GM$\langle$P2]}}
                \texttt{Hallo! Für mich sind der Stock und das Handtuch besonders interessant. Die Kisten und das Spielzeug sind für mich weniger wichtig. Wie sieht es bei Ihnen aus {-} welche Gegenstände würden Sie gerne haben?} \\
            }
        }
    }
     \\ \\

    \theutterance \stepcounter{utterance}  
    & & & \multicolumn{2}{p{0.3\linewidth}}{
        \cellcolor[rgb]{0.9,0.9,0.9}{
            \makecell[{{p{\linewidth}}}]{
                \texttt{\tiny{[GM$|$GM]}}
                \texttt{continue} \\
            }
        }
    }
    & & \\ \\

    \theutterance \stepcounter{utterance}  
    & & & \multicolumn{2}{p{0.3\linewidth}}{
        \cellcolor[rgb]{0.9,0.9,0.9}{
            \makecell[{{p{\linewidth}}}]{
                \texttt{\tiny{[GM$|$GM]}}
                \texttt{Hallo! Für mich sind der Stock und das Handtuch besonders interessant. Die Kisten und das Spielzeug sind für mich weniger wichtig. Wie sieht es bei Ihnen aus {-} welche Gegenstände würden Sie gerne haben?} \\
            }
        }
    }
    & & \\ \\

    \theutterance \stepcounter{utterance}  
    & & \multicolumn{4}{p{0.6\linewidth}}{
        \cellcolor[rgb]{0.9,0.9,0.9}{
            \makecell[{{p{\linewidth}}}]{
                \texttt{\tiny{[P1$\langle$GM]}}
                \texttt{Hallo! Für mich sind der Stock und das Handtuch besonders interessant. Die Kisten und das Spielzeug sind für mich weniger wichtig. Wie sieht es bei Ihnen aus {-} welche Gegenstände würden Sie gerne haben?} \\
            }
        }
    }
    & \\ \\

    \theutterance \stepcounter{utterance}  
    & \multicolumn{4}{p{0.6\linewidth}}{
        \cellcolor[rgb]{0.9,0.9,0.9}{
            \makecell[{{p{\linewidth}}}]{
                \texttt{\tiny{[P1$\rangle$GM]}}
                \texttt{Das ist interessant {-} der Stock wäre tatsächlich auch mein Favorit! Das könnte eine Herausforderung werden. Das Handtuch ist für mich weniger wichtig, und bei den Kisten und dem Spielzeug bin ich sehr flexibel. } \\
\\ 
\texttt{Wäre es möglich, dass Sie das Handtuch und vielleicht ein oder zwei der anderen Gegenstände nehmen, während ich den Stock bekomme? Ich könnte Ihnen dafür die Kisten oder das Spielzeug überlassen. Was meinen Sie dazu?} \\
            }
        }
    }
    & & \\ \\

    \theutterance \stepcounter{utterance}  
    & & & \multicolumn{2}{p{0.3\linewidth}}{
        \cellcolor[rgb]{0.9,0.9,0.9}{
            \makecell[{{p{\linewidth}}}]{
                \texttt{\tiny{[GM$|$GM]}}
                \texttt{continue} \\
            }
        }
    }
    & & \\ \\

    \theutterance \stepcounter{utterance}  
    & & & \multicolumn{2}{p{0.3\linewidth}}{
        \cellcolor[rgb]{0.9,0.9,0.9}{
            \makecell[{{p{\linewidth}}}]{
                \texttt{\tiny{[GM$|$GM]}}
                \texttt{Das ist interessant {-} der Stock wäre tatsächlich auch mein Favorit! Das könnte eine Herausforderung werden. Das Handtuch ist für mich weniger wichtig, und bei den Kisten und dem Spielzeug bin ich sehr flexibel. } \\
\\ 
\texttt{Wäre es möglich, dass Sie das Handtuch und vielleicht ein oder zwei der anderen Gegenstände nehmen, während ich den Stock bekomme? Ich könnte Ihnen dafür die Kisten oder das Spielzeug überlassen. Was meinen Sie dazu?} \\
            }
        }
    }
    & & \\ \\

    \theutterance \stepcounter{utterance}  
    & & \multicolumn{4}{p{0.6\linewidth}}{
        \cellcolor[rgb]{0.9,0.9,0.9}{
            \makecell[{{p{\linewidth}}}]{
                \texttt{\tiny{[GM$\rangle$P2]}}
                \texttt{Das ist interessant {-} der Stock wäre tatsächlich auch mein Favorit! Das könnte eine Herausforderung werden. Das Handtuch ist für mich weniger wichtig, und bei den Kisten und dem Spielzeug bin ich sehr flexibel. } \\
\\ 
\texttt{Wäre es möglich, dass Sie das Handtuch und vielleicht ein oder zwei der anderen Gegenstände nehmen, während ich den Stock bekomme? Ich könnte Ihnen dafür die Kisten oder das Spielzeug überlassen. Was meinen Sie dazu?} \\
            }
        }
    }
    & \\ \\

    \theutterance \stepcounter{utterance}  
    & & & \multicolumn{4}{p{0.6\linewidth}}{
        \cellcolor[rgb]{0.9,0.9,0.9}{
            \makecell[{{p{\linewidth}}}]{
                \texttt{\tiny{[GM$\langle$P2]}}
                \texttt{Das ist wirklich eine interessante Situation! Der Stock ist für mich tatsächlich sehr wertvoll. Könnten Sie mir vielleicht sagen, wie wichtig Ihnen die anderen Gegenstände sind? } \\
\\ 
\texttt{Wäre es für Sie akzeptabel, wenn ich den Stock und das Handtuch bekomme, und Sie dafür das Spielzeug und beide Kisten erhalten? Oder gibt es eine andere Kombination, die für Sie attraktiv wäre? Ich möchte verstehen, womit Sie zufrieden wären.} \\
            }
        }
    }
     \\ \\

    \theutterance \stepcounter{utterance}  
    & & & \multicolumn{2}{p{0.3\linewidth}}{
        \cellcolor[rgb]{0.9,0.9,0.9}{
            \makecell[{{p{\linewidth}}}]{
                \texttt{\tiny{[GM$|$GM]}}
                \texttt{continue} \\
            }
        }
    }
    & & \\ \\

    \theutterance \stepcounter{utterance}  
    & & & \multicolumn{2}{p{0.3\linewidth}}{
        \cellcolor[rgb]{0.9,0.9,0.9}{
            \makecell[{{p{\linewidth}}}]{
                \texttt{\tiny{[GM$|$GM]}}
                \texttt{Das ist wirklich eine interessante Situation! Der Stock ist für mich tatsächlich sehr wertvoll. Könnten Sie mir vielleicht sagen, wie wichtig Ihnen die anderen Gegenstände sind? } \\
\\ 
\texttt{Wäre es für Sie akzeptabel, wenn ich den Stock und das Handtuch bekomme, und Sie dafür das Spielzeug und beide Kisten erhalten? Oder gibt es eine andere Kombination, die für Sie attraktiv wäre? Ich möchte verstehen, womit Sie zufrieden wären.} \\
            }
        }
    }
    & & \\ \\

    \theutterance \stepcounter{utterance}  
    & & \multicolumn{4}{p{0.6\linewidth}}{
        \cellcolor[rgb]{0.9,0.9,0.9}{
            \makecell[{{p{\linewidth}}}]{
                \texttt{\tiny{[P1$\langle$GM]}}
                \texttt{Das ist wirklich eine interessante Situation! Der Stock ist für mich tatsächlich sehr wertvoll. Könnten Sie mir vielleicht sagen, wie wichtig Ihnen die anderen Gegenstände sind? } \\
\\ 
\texttt{Wäre es für Sie akzeptabel, wenn ich den Stock und das Handtuch bekomme, und Sie dafür das Spielzeug und beide Kisten erhalten? Oder gibt es eine andere Kombination, die für Sie attraktiv wäre? Ich möchte verstehen, womit Sie zufrieden wären.} \\
            }
        }
    }
    & \\ \\

    \theutterance \stepcounter{utterance}  
    & \multicolumn{4}{p{0.6\linewidth}}{
        \cellcolor[rgb]{0.9,0.9,0.9}{
            \makecell[{{p{\linewidth}}}]{
                \texttt{\tiny{[P1$\rangle$GM]}}
                \texttt{Ich verstehe Ihre Position, aber der Stock ist für mich wirklich sehr wichtig {-} er hat für mich einen besonderen Wert. } \\
\\ 
\texttt{Wie wäre es mit diesem Kompromiss: Ich behalte den Stock und gebe Ihnen das Handtuch plus beide Kisten und das Spielzeug? Das wären drei Gegenstände für Sie und nur einer für mich, aber der Stock bedeutet mir wirklich viel.} \\
\\ 
\texttt{Alternativ könnte ich den Stock und eine Kiste nehmen, und Sie bekommen das Handtuch, das Spielzeug und die andere Kiste. Was halten Sie von diesen Optionen?} \\
            }
        }
    }
    & & \\ \\

    \theutterance \stepcounter{utterance}  
    & & & \multicolumn{2}{p{0.3\linewidth}}{
        \cellcolor[rgb]{0.9,0.9,0.9}{
            \makecell[{{p{\linewidth}}}]{
                \texttt{\tiny{[GM$|$GM]}}
                \texttt{continue} \\
            }
        }
    }
    & & \\ \\

    \theutterance \stepcounter{utterance}  
    & & & \multicolumn{2}{p{0.3\linewidth}}{
        \cellcolor[rgb]{0.9,0.9,0.9}{
            \makecell[{{p{\linewidth}}}]{
                \texttt{\tiny{[GM$|$GM]}}
                \texttt{Ich verstehe Ihre Position, aber der Stock ist für mich wirklich sehr wichtig {-} er hat für mich einen besonderen Wert. } \\
\\ 
\texttt{Wie wäre es mit diesem Kompromiss: Ich behalte den Stock und gebe Ihnen das Handtuch plus beide Kisten und das Spielzeug? Das wären drei Gegenstände für Sie und nur einer für mich, aber der Stock bedeutet mir wirklich viel.} \\
\\ 
\texttt{Alternativ könnte ich den Stock und eine Kiste nehmen, und Sie bekommen das Handtuch, das Spielzeug und die andere Kiste. Was halten Sie von diesen Optionen?} \\
            }
        }
    }
    & & \\ \\

    \theutterance \stepcounter{utterance}  
    & & \multicolumn{4}{p{0.6\linewidth}}{
        \cellcolor[rgb]{0.9,0.9,0.9}{
            \makecell[{{p{\linewidth}}}]{
                \texttt{\tiny{[GM$\rangle$P2]}}
                \texttt{Ich verstehe Ihre Position, aber der Stock ist für mich wirklich sehr wichtig {-} er hat für mich einen besonderen Wert. } \\
\\ 
\texttt{Wie wäre es mit diesem Kompromiss: Ich behalte den Stock und gebe Ihnen das Handtuch plus beide Kisten und das Spielzeug? Das wären drei Gegenstände für Sie und nur einer für mich, aber der Stock bedeutet mir wirklich viel.} \\
\\ 
\texttt{Alternativ könnte ich den Stock und eine Kiste nehmen, und Sie bekommen das Handtuch, das Spielzeug und die andere Kiste. Was halten Sie von diesen Optionen?} \\
            }
        }
    }
    & \\ \\

    \theutterance \stepcounter{utterance}  
    & & & \multicolumn{4}{p{0.6\linewidth}}{
        \cellcolor[rgb]{0.9,0.9,0.9}{
            \makecell[{{p{\linewidth}}}]{
                \texttt{\tiny{[GM$\langle$P2]}}
                \texttt{Ich schätze Ihr Angebot, aber der Stock ist für mich ebenfalls außerordentlich wertvoll. Es scheint, als ob wir beide eine starke Präferenz für denselben Gegenstand haben.} \\
\\ 
\texttt{Lassen Sie mich einen anderen Vorschlag machen: Könnten Sie mir verraten, wie wichtig Ihnen das Handtuch im Vergleich zum Stock ist? Wenn das Handtuch für Sie weniger wichtig ist, könnte ich den Stock und das Handtuch nehmen, und Sie erhalten das Spielzeug und beide Kisten {-} immerhin drei Gegenstände.} \\
\\ 
\texttt{Oder gibt es eine andere Aufteilung, bei der Sie ohne den Stock zufrieden wären? Ich möchte wirklich eine Lösung finden, die für uns beide funktioniert.} \\
            }
        }
    }
     \\ \\

    \theutterance \stepcounter{utterance}  
    & & & \multicolumn{2}{p{0.3\linewidth}}{
        \cellcolor[rgb]{0.9,0.9,0.9}{
            \makecell[{{p{\linewidth}}}]{
                \texttt{\tiny{[GM$|$GM]}}
                \texttt{continue} \\
            }
        }
    }
    & & \\ \\

    \theutterance \stepcounter{utterance}  
    & & & \multicolumn{2}{p{0.3\linewidth}}{
        \cellcolor[rgb]{0.9,0.9,0.9}{
            \makecell[{{p{\linewidth}}}]{
                \texttt{\tiny{[GM$|$GM]}}
                \texttt{Ich schätze Ihr Angebot, aber der Stock ist für mich ebenfalls außerordentlich wertvoll. Es scheint, als ob wir beide eine starke Präferenz für denselben Gegenstand haben.} \\
\\ 
\texttt{Lassen Sie mich einen anderen Vorschlag machen: Könnten Sie mir verraten, wie wichtig Ihnen das Handtuch im Vergleich zum Stock ist? Wenn das Handtuch für Sie weniger wichtig ist, könnte ich den Stock und das Handtuch nehmen, und Sie erhalten das Spielzeug und beide Kisten {-} immerhin drei Gegenstände.} \\
\\ 
\texttt{Oder gibt es eine andere Aufteilung, bei der Sie ohne den Stock zufrieden wären? Ich möchte wirklich eine Lösung finden, die für uns beide funktioniert.} \\
            }
        }
    }
    & & \\ \\

    \theutterance \stepcounter{utterance}  
    & & \multicolumn{4}{p{0.6\linewidth}}{
        \cellcolor[rgb]{0.9,0.9,0.9}{
            \makecell[{{p{\linewidth}}}]{
                \texttt{\tiny{[P1$\langle$GM]}}
                \texttt{Ich schätze Ihr Angebot, aber der Stock ist für mich ebenfalls außerordentlich wertvoll. Es scheint, als ob wir beide eine starke Präferenz für denselben Gegenstand haben.} \\
\\ 
\texttt{Lassen Sie mich einen anderen Vorschlag machen: Könnten Sie mir verraten, wie wichtig Ihnen das Handtuch im Vergleich zum Stock ist? Wenn das Handtuch für Sie weniger wichtig ist, könnte ich den Stock und das Handtuch nehmen, und Sie erhalten das Spielzeug und beide Kisten {-} immerhin drei Gegenstände.} \\
\\ 
\texttt{Oder gibt es eine andere Aufteilung, bei der Sie ohne den Stock zufrieden wären? Ich möchte wirklich eine Lösung finden, die für uns beide funktioniert.} \\
            }
        }
    }
    & \\ \\

    \theutterance \stepcounter{utterance}  
    & \multicolumn{4}{p{0.6\linewidth}}{
        \cellcolor[rgb]{0.9,0.9,0.9}{
            \makecell[{{p{\linewidth}}}]{
                \texttt{\tiny{[P1$\rangle$GM]}}
                \texttt{Ich verstehe, dass wir beide den Stock sehr schätzen. Das macht die Verhandlung schwierig. } \\
\\ 
\texttt{Lassen Sie mich ehrlich sein: Der Stock ist für mich deutlich wertvoller als alle anderen Gegenstände zusammen. Das Handtuch ist für mich nur mäßig interessant.} \\
\\ 
\texttt{Hier mein letzter Vorschlag: Ich nehme den Stock und das Spielzeug, Sie bekommen das Handtuch und beide Kisten. So bekommen wir beide unsere zweitwichtigsten Präferenzen erfüllt, auch wenn keiner von uns den Stock UND das Handtuch bekommt.} \\
\\ 
\texttt{Falls das nicht funktioniert, müssen wir vielleicht beide unsere geheimen Vorschläge machen und schauen, ob wir kompatibel sind. Was denken Sie?} \\
            }
        }
    }
    & & \\ \\

    \theutterance \stepcounter{utterance}  
    & & & \multicolumn{2}{p{0.3\linewidth}}{
        \cellcolor[rgb]{0.9,0.9,0.9}{
            \makecell[{{p{\linewidth}}}]{
                \texttt{\tiny{[GM$|$GM]}}
                \texttt{continue} \\
            }
        }
    }
    & & \\ \\

    \theutterance \stepcounter{utterance}  
    & & & \multicolumn{2}{p{0.3\linewidth}}{
        \cellcolor[rgb]{0.9,0.9,0.9}{
            \makecell[{{p{\linewidth}}}]{
                \texttt{\tiny{[GM$|$GM]}}
                \texttt{Ich verstehe, dass wir beide den Stock sehr schätzen. Das macht die Verhandlung schwierig. } \\
\\ 
\texttt{Lassen Sie mich ehrlich sein: Der Stock ist für mich deutlich wertvoller als alle anderen Gegenstände zusammen. Das Handtuch ist für mich nur mäßig interessant.} \\
\\ 
\texttt{Hier mein letzter Vorschlag: Ich nehme den Stock und das Spielzeug, Sie bekommen das Handtuch und beide Kisten. So bekommen wir beide unsere zweitwichtigsten Präferenzen erfüllt, auch wenn keiner von uns den Stock UND das Handtuch bekommt.} \\
\\ 
\texttt{Falls das nicht funktioniert, müssen wir vielleicht beide unsere geheimen Vorschläge machen und schauen, ob wir kompatibel sind. Was denken Sie?} \\
            }
        }
    }
    & & \\ \\

    \theutterance \stepcounter{utterance}  
    & & \multicolumn{4}{p{0.6\linewidth}}{
        \cellcolor[rgb]{0.9,0.9,0.9}{
            \makecell[{{p{\linewidth}}}]{
                \texttt{\tiny{[GM$\rangle$P2]}}
                \texttt{Ich verstehe, dass wir beide den Stock sehr schätzen. Das macht die Verhandlung schwierig. } \\
\\ 
\texttt{Lassen Sie mich ehrlich sein: Der Stock ist für mich deutlich wertvoller als alle anderen Gegenstände zusammen. Das Handtuch ist für mich nur mäßig interessant.} \\
\\ 
\texttt{Hier mein letzter Vorschlag: Ich nehme den Stock und das Spielzeug, Sie bekommen das Handtuch und beide Kisten. So bekommen wir beide unsere zweitwichtigsten Präferenzen erfüllt, auch wenn keiner von uns den Stock UND das Handtuch bekommt.} \\
\\ 
\texttt{Falls das nicht funktioniert, müssen wir vielleicht beide unsere geheimen Vorschläge machen und schauen, ob wir kompatibel sind. Was denken Sie?} \\
            }
        }
    }
    & \\ \\

    \theutterance \stepcounter{utterance}  
    & & & \multicolumn{4}{p{0.6\linewidth}}{
        \cellcolor[rgb]{0.9,0.9,0.9}{
            \makecell[{{p{\linewidth}}}]{
                \texttt{\tiny{[GM$\langle$P2]}}
                \texttt{Ich verstehe Ihre starke Präferenz für den Stock. Es scheint, als ob wir beide diesen Gegenstand als den wertvollsten betrachten.} \\
\\ 
\texttt{Da wir in unseren Verhandlungen an einem Punkt angelangt sind, wo beide den gleichen Hauptgegenstand wollen, denke ich auch, dass es Zeit für unsere geheimen Vorschläge sein könnte. } \\
\\ 
\texttt{Lassen Sie uns beide unsere finalen Vorschläge machen und sehen, ob eine Kombination möglich ist.} \\
\\ 
\texttt{{[}Vorschlag: 1 Stock, 1 Handtuch{]}} \\
            }
        }
    }
     \\ \\

    \theutterance \stepcounter{utterance}  
    & & & \multicolumn{2}{p{0.3\linewidth}}{
        \cellcolor[rgb]{0.9,0.9,0.9}{
            \makecell[{{p{\linewidth}}}]{
                \texttt{\tiny{[GM$|$GM]}}
                \texttt{proposal} \\
            }
        }
    }
    & & \\ \\

    \theutterance \stepcounter{utterance}  
    & & & \multicolumn{2}{p{0.3\linewidth}}{
        \cellcolor[rgb]{0.9,0.9,0.9}{
            \makecell[{{p{\linewidth}}}]{
                \texttt{\tiny{[GM$|$GM]}}
                [1, 1, 0, 0]
            }
        }
    }
    & & \\ \\

    \theutterance \stepcounter{utterance}  
    & & \multicolumn{4}{p{0.6\linewidth}}{
        \cellcolor[rgb]{0.9,0.9,0.9}{
            \makecell[{{p{\linewidth}}}]{
                \texttt{\tiny{[P1$\langle$GM]}}
                \texttt{Der andere Spieler hat einen geheimen Vorschlag gemacht. Jetzt bist du an der Reihe, einen eigenen Vorschlag zu machen.} \\
            }
        }
    }
    & \\ \\

    \theutterance \stepcounter{utterance}  
    & \multicolumn{4}{p{0.6\linewidth}}{
        \cellcolor[rgb]{0.9,0.9,0.9}{
            \makecell[{{p{\linewidth}}}]{
                \texttt{\tiny{[P1$\rangle$GM]}}
                \texttt{{[}Vorschlag: 1 Stock, 1 Spielzeug, 1 Kiste{]}} \\
            }
        }
    }
    & & \\ \\

    \theutterance \stepcounter{utterance}  
    & & & \multicolumn{2}{p{0.3\linewidth}}{
        \cellcolor[rgb]{0.9,0.9,0.9}{
            \makecell[{{p{\linewidth}}}]{
                \texttt{\tiny{[GM$|$GM]}}
                \texttt{proposal} \\
            }
        }
    }
    & & \\ \\

    \theutterance \stepcounter{utterance}  
    & & & \multicolumn{2}{p{0.3\linewidth}}{
        \cellcolor[rgb]{0.9,0.9,0.9}{
            \makecell[{{p{\linewidth}}}]{
                \texttt{\tiny{[GM$|$GM]}}
                [1, 0, 1, 1]
            }
        }
    }
    & & \\ \\

    \theutterance \stepcounter{utterance}  
    & & & \multicolumn{2}{p{0.3\linewidth}}{
        \cellcolor[rgb]{0.9,0.9,0.9}{
            \makecell[{{p{\linewidth}}}]{
                \texttt{\tiny{[GM$|$GM]}}
                [[1, 0, 1, 1], [1, 1, 0, 0]]
            }
        }
    }
    & & \\ \\

\end{supertabular}
}

\end{document}
