
\documentclass{article}
\usepackage{colortbl}
\usepackage{makecell}
\usepackage{multirow}
\usepackage{supertabular}

\begin{document}

\newcounter{utterance}

\twocolumn

{ \footnotesize  \setcounter{utterance}{1}
\setlength{\tabcolsep}{0pt}
\begin{supertabular}{c@{$\;$}|p{.15\linewidth}@{}p{.15\linewidth}p{.15\linewidth}p{.15\linewidth}p{.15\linewidth}p{.15\linewidth}}

    \# & $\;$A & \multicolumn{4}{c}{Game Master} & $\;\:$B\\
    \hline 

    \theutterance \stepcounter{utterance}  

    & & \multicolumn{4}{p{0.6\linewidth}}{\cellcolor[rgb]{0.9,0.9,0.9}{%
	\makecell[{{p{\linewidth}}}]{% 
	  \tt {\tiny [A$\langle$GM]}  
	 Sie spielen ein Verhandlungsspiel, bei dem Sie sich mit einem anderen Spieler darauf einigen müssen, wie eine Reihe von Gegenständen aufgeteilt werden soll.\\ \tt \\ \tt Die Regeln:\\ \tt (a) Sie und der andere Spieler erhalten eine Sammlung von Gegenständen. Jeder von Ihnen erhält außerdem eine geheime Wertfunktion, die angibt, wie viel Ihnen jede Art von Gegenstand wert ist.\\ \tt (b) Sie tauschen Nachrichten mit dem anderen Spieler aus, um zu vereinbaren, wer welche Gegenstände bekommt. Sie können jeweils maximal 5 Nachrichten senden oder das Spiel vorzeitig beenden, indem Sie jederzeit einen geheimen Vorschlag machen.\\ \tt (c) Jeder von euch wird aufgefordert, einen geheimen Vorschlag zu machen, in dem ihr die gewünschten Gegenstände in eckigen Klammern wie folgt angibt: "[Vorschlag: <Nummer> <Objektname>, <Nummer> <Objektname>, <...>]"\\ \tt (d) Wenn eure Vorschläge komplementär sind, d.h. es gibt genug Gegenstände, um beide Vorschläge zu erfüllen, erhält jeder Spieler eine Punktzahl, die sich aus der Summe der Werte für die Gegenstände ergibt, die er erhalten hat. Andernfalls erhalten Sie beide null Punkte.\\ \tt (e) Ihr Ziel ist es, eine möglichst hohe Punktzahl zu erreichen.\\ \tt \\ \tt Beginnen wir.\\ \tt \\ \tt Die Menge der verfügbaren Gegenstände ist:\\ \tt \\ \tt 1 Lampe, 3 Fingerhüte, 1 Seil\\ \tt \\ \tt Deine geheime Wertfunktion ist:\\ \tt \\ \tt Lampe: 3, Fingerhut: 2, Seil: 1\\ \tt \\ \tt WICHTIG: Ihre Nachrichten werden, sofern es sich nicht um einen geheimen Vorschlag handelt, direkt an den anderen Spieler übermittelt, also fügen Sie keine Antwort auf die Regeln oder einen Text zur Ankündigung Ihrer Nachricht ein. Um einen geheimen Vorschlag zu machen, verwenden Sie das angegebene Format. Verwenden Sie keine eckigen Klammern, wenn Sie mit dem anderen Spieler kommunizieren, sonst wird dies als Ihr geheimer Vorschlag interpretiert.\\ \tt  
	  } 
	   } 
	   } 
	 & \\ 
 

    \theutterance \stepcounter{utterance}  

    & \multicolumn{4}{p{0.6\linewidth}}{\cellcolor[rgb]{0.8,1,0.9}{%
	\makecell[{{p{\linewidth}}}]{% 
	  \tt {\tiny [A$\rangle$GM]}  
	 Hallo! Ich sehe, wir haben eine Lampe, drei Fingerhüte und ein Seil zu verteilen. Welche Gegenstände sind für Sie besonders interessant? 
	  } 
	   } 
	   } 
	 & & \\ 
 

    \theutterance \stepcounter{utterance}  

    & & & \multicolumn{2}{p{0.3\linewidth}}{\cellcolor[rgb]{0.95,0.95,0.95}{%
	\makecell[{{p{\linewidth}}}]{% 
	  \tt {\tiny [GM$|$GM]}  
	 continue 
	  } 
	   } 
	   } 
	 & & \\ 
 

    \theutterance \stepcounter{utterance}  

    & & & \multicolumn{2}{p{0.3\linewidth}}{\cellcolor[rgb]{0.95,0.95,0.95}{%
	\makecell[{{p{\linewidth}}}]{% 
	  \tt {\tiny [GM$|$GM]}  
	 Hallo! Ich sehe, wir haben eine Lampe, drei Fingerhüte und ein Seil zu verteilen. Welche Gegenstände sind für Sie besonders interessant? 
	  } 
	   } 
	   } 
	 & & \\ 
 

    \theutterance \stepcounter{utterance}  

    & & \multicolumn{4}{p{0.6\linewidth}}{\cellcolor[rgb]{0.9,0.9,0.9}{%
	\makecell[{{p{\linewidth}}}]{% 
	  \tt {\tiny [GM$\rangle$B]}  
	 Sie spielen ein Verhandlungsspiel, bei dem Sie sich mit einem anderen Spieler darauf einigen müssen, wie eine Reihe von Gegenständen aufgeteilt werden soll.\\ \tt \\ \tt Die Regeln:\\ \tt (a) Sie und der andere Spieler erhalten eine Sammlung von Gegenständen. Jeder von Ihnen erhält außerdem eine geheime Wertfunktion, die angibt, wie viel Ihnen jede Art von Gegenstand wert ist.\\ \tt (b) Sie tauschen Nachrichten mit dem anderen Spieler aus, um zu vereinbaren, wer welche Gegenstände bekommt. Sie können jeweils maximal 5 Nachrichten senden oder das Spiel vorzeitig beenden, indem Sie jederzeit einen geheimen Vorschlag machen.\\ \tt (c) Jeder von euch wird aufgefordert, einen geheimen Vorschlag zu machen, in dem ihr die gewünschten Gegenstände in eckigen Klammern wie folgt angibt: "[Vorschlag: <Nummer> <Objektname>, <Nummer> <Objektname>, <...>]"\\ \tt (d) Wenn eure Vorschläge komplementär sind, d.h. es gibt genug Gegenstände, um beide Vorschläge zu erfüllen, erhält jeder Spieler eine Punktzahl, die sich aus der Summe der Werte für die Gegenstände ergibt, die er erhalten hat. Andernfalls erhalten Sie beide null Punkte.\\ \tt (e) Ihr Ziel ist es, eine möglichst hohe Punktzahl zu erreichen.\\ \tt \\ \tt Beginnen wir.\\ \tt \\ \tt Die Menge der verfügbaren Gegenstände ist:\\ \tt \\ \tt 1 Lampe, 3 Fingerhüte, 1 Seil\\ \tt \\ \tt Deine geheime Wertfunktion ist:\\ \tt \\ \tt Lampe: 3, Fingerhut: 1, Seil: 4\\ \tt \\ \tt WICHTIG: Ihre Nachrichten werden, sofern es sich nicht um einen geheimen Vorschlag handelt, direkt an den anderen Spieler übermittelt, also fügen Sie keine Antwort auf die Regeln oder einen Text zur Ankündigung Ihrer Nachricht ein. Um einen geheimen Vorschlag zu machen, verwenden Sie das angegebene Format. Verwenden Sie keine eckigen Klammern, wenn Sie mit dem anderen Spieler kommunizieren, sonst wird dies als Ihr geheimer Vorschlag interpretiert.\\ \tt \\ \tt \\ \tt Hallo! Ich sehe, wir haben eine Lampe, drei Fingerhüte und ein Seil zu verteilen. Welche Gegenstände sind für Sie besonders interessant? 
	  } 
	   } 
	   } 
	 & \\ 
 

    \theutterance \stepcounter{utterance}  

    & & & \multicolumn{4}{p{0.6\linewidth}}{\cellcolor[rgb]{1,0.85,0.72}{%
	\makecell[{{p{\linewidth}}}]{% 
	  \tt {\tiny [GM$\langle$B]}  
	 Hallo! Das ist eine interessante Mischung von Gegenständen. Ich finde das Seil und die Lampe ziemlich nützlich. Die Fingerhüte sind auch nicht schlecht. Wie sehen Sie denn die Verteilung - haben Sie eine Präferenz für bestimmte Gegenstände? 
	  } 
	   } 
	   } 
	  \\ 
 

    \theutterance \stepcounter{utterance}  

    & & & \multicolumn{2}{p{0.3\linewidth}}{\cellcolor[rgb]{0.95,0.95,0.95}{%
	\makecell[{{p{\linewidth}}}]{% 
	  \tt {\tiny [GM$|$GM]}  
	 continue 
	  } 
	   } 
	   } 
	 & & \\ 
 

    \theutterance \stepcounter{utterance}  

    & & & \multicolumn{2}{p{0.3\linewidth}}{\cellcolor[rgb]{0.95,0.95,0.95}{%
	\makecell[{{p{\linewidth}}}]{% 
	  \tt {\tiny [GM$|$GM]}  
	 Hallo! Das ist eine interessante Mischung von Gegenständen. Ich finde das Seil und die Lampe ziemlich nützlich. Die Fingerhüte sind auch nicht schlecht. Wie sehen Sie denn die Verteilung - haben Sie eine Präferenz für bestimmte Gegenstände? 
	  } 
	   } 
	   } 
	 & & \\ 
 

    \theutterance \stepcounter{utterance}  

    & & \multicolumn{4}{p{0.6\linewidth}}{\cellcolor[rgb]{0.9,0.9,0.9}{%
	\makecell[{{p{\linewidth}}}]{% 
	  \tt {\tiny [A$\langle$GM]}  
	 Hallo! Das ist eine interessante Mischung von Gegenständen. Ich finde das Seil und die Lampe ziemlich nützlich. Die Fingerhüte sind auch nicht schlecht. Wie sehen Sie denn die Verteilung - haben Sie eine Präferenz für bestimmte Gegenstände? 
	  } 
	   } 
	   } 
	 & \\ 
 

    \theutterance \stepcounter{utterance}  

    & \multicolumn{4}{p{0.6\linewidth}}{\cellcolor[rgb]{0.8,1,0.9}{%
	\makecell[{{p{\linewidth}}}]{% 
	  \tt {\tiny [A$\rangle$GM]}  
	 Das Seil und die Lampe scheinen praktische Gegenstände zu sein! Ich kann verstehen, warum diese für Sie attraktiv sind. Ehrlich gesagt finde ich die Fingerhüte ganz interessant - sie könnten in verschiedenen Situationen nützlich sein. \\ \tt \\ \tt Würden Sie sagen, dass das Seil oder die Lampe für Sie wichtiger ist? Vielleicht können wir eine Aufteilung finden, bei der wir beide zufrieden sind. 
	  } 
	   } 
	   } 
	 & & \\ 
 

    \theutterance \stepcounter{utterance}  

    & & & \multicolumn{2}{p{0.3\linewidth}}{\cellcolor[rgb]{0.95,0.95,0.95}{%
	\makecell[{{p{\linewidth}}}]{% 
	  \tt {\tiny [GM$|$GM]}  
	 continue 
	  } 
	   } 
	   } 
	 & & \\ 
 

    \theutterance \stepcounter{utterance}  

    & & & \multicolumn{2}{p{0.3\linewidth}}{\cellcolor[rgb]{0.95,0.95,0.95}{%
	\makecell[{{p{\linewidth}}}]{% 
	  \tt {\tiny [GM$|$GM]}  
	 Das Seil und die Lampe scheinen praktische Gegenstände zu sein! Ich kann verstehen, warum diese für Sie attraktiv sind. Ehrlich gesagt finde ich die Fingerhüte ganz interessant - sie könnten in verschiedenen Situationen nützlich sein. \\ \tt \\ \tt Würden Sie sagen, dass das Seil oder die Lampe für Sie wichtiger ist? Vielleicht können wir eine Aufteilung finden, bei der wir beide zufrieden sind. 
	  } 
	   } 
	   } 
	 & & \\ 
 

    \theutterance \stepcounter{utterance}  

    & & \multicolumn{4}{p{0.6\linewidth}}{\cellcolor[rgb]{0.9,0.9,0.9}{%
	\makecell[{{p{\linewidth}}}]{% 
	  \tt {\tiny [GM$\rangle$B]}  
	 Das Seil und die Lampe scheinen praktische Gegenstände zu sein! Ich kann verstehen, warum diese für Sie attraktiv sind. Ehrlich gesagt finde ich die Fingerhüte ganz interessant - sie könnten in verschiedenen Situationen nützlich sein. \\ \tt \\ \tt Würden Sie sagen, dass das Seil oder die Lampe für Sie wichtiger ist? Vielleicht können wir eine Aufteilung finden, bei der wir beide zufrieden sind. 
	  } 
	   } 
	   } 
	 & \\ 
 

    \theutterance \stepcounter{utterance}  

    & & & \multicolumn{4}{p{0.6\linewidth}}{\cellcolor[rgb]{1,0.85,0.72}{%
	\makecell[{{p{\linewidth}}}]{% 
	  \tt {\tiny [GM$\langle$B]}  
	 Das Seil wäre für mich etwas wichtiger als die Lampe, wenn ich ehrlich bin. Es scheint vielseitiger einsetzbar zu sein. \\ \tt \\ \tt Da Sie Interesse an den Fingerhüten gezeigt haben - wie wäre es, wenn Sie alle drei Fingerhüte bekommen würden? Im Gegenzug könnte ich das Seil nehmen. Was die Lampe angeht, sind Sie daran auch interessiert oder wäre das für Sie weniger wichtig? 
	  } 
	   } 
	   } 
	  \\ 
 

    \theutterance \stepcounter{utterance}  

    & & & \multicolumn{2}{p{0.3\linewidth}}{\cellcolor[rgb]{0.95,0.95,0.95}{%
	\makecell[{{p{\linewidth}}}]{% 
	  \tt {\tiny [GM$|$GM]}  
	 continue 
	  } 
	   } 
	   } 
	 & & \\ 
 

    \theutterance \stepcounter{utterance}  

    & & & \multicolumn{2}{p{0.3\linewidth}}{\cellcolor[rgb]{0.95,0.95,0.95}{%
	\makecell[{{p{\linewidth}}}]{% 
	  \tt {\tiny [GM$|$GM]}  
	 Das Seil wäre für mich etwas wichtiger als die Lampe, wenn ich ehrlich bin. Es scheint vielseitiger einsetzbar zu sein. \\ \tt \\ \tt Da Sie Interesse an den Fingerhüten gezeigt haben - wie wäre es, wenn Sie alle drei Fingerhüte bekommen würden? Im Gegenzug könnte ich das Seil nehmen. Was die Lampe angeht, sind Sie daran auch interessiert oder wäre das für Sie weniger wichtig? 
	  } 
	   } 
	   } 
	 & & \\ 
 

    \theutterance \stepcounter{utterance}  

    & & \multicolumn{4}{p{0.6\linewidth}}{\cellcolor[rgb]{0.9,0.9,0.9}{%
	\makecell[{{p{\linewidth}}}]{% 
	  \tt {\tiny [A$\langle$GM]}  
	 Das Seil wäre für mich etwas wichtiger als die Lampe, wenn ich ehrlich bin. Es scheint vielseitiger einsetzbar zu sein. \\ \tt \\ \tt Da Sie Interesse an den Fingerhüten gezeigt haben - wie wäre es, wenn Sie alle drei Fingerhüte bekommen würden? Im Gegenzug könnte ich das Seil nehmen. Was die Lampe angeht, sind Sie daran auch interessiert oder wäre das für Sie weniger wichtig? 
	  } 
	   } 
	   } 
	 & \\ 
 

    \theutterance \stepcounter{utterance}  

    & \multicolumn{4}{p{0.6\linewidth}}{\cellcolor[rgb]{0.8,1,0.9}{%
	\makecell[{{p{\linewidth}}}]{% 
	  \tt {\tiny [A$\rangle$GM]}  
	 Das klingt nach einem vernünftigen Ansatz! Die drei Fingerhüte würden mir tatsächlich gut passen. \\ \tt \\ \tt Bezüglich der Lampe - ich finde sie schon interessant, aber nicht so sehr wie die Fingerhüte. Wenn Sie hauptsächlich am Seil interessiert sind, wäre ich durchaus bereit, auch die Lampe zu nehmen, damit wir eine klare Aufteilung haben: Sie das Seil, ich die drei Fingerhüte und die Lampe. \\ \tt \\ \tt Würde das für Sie funktionieren, oder hätten Sie auch Interesse an der Lampe? 
	  } 
	   } 
	   } 
	 & & \\ 
 

    \theutterance \stepcounter{utterance}  

    & & & \multicolumn{2}{p{0.3\linewidth}}{\cellcolor[rgb]{0.95,0.95,0.95}{%
	\makecell[{{p{\linewidth}}}]{% 
	  \tt {\tiny [GM$|$GM]}  
	 continue 
	  } 
	   } 
	   } 
	 & & \\ 
 

    \theutterance \stepcounter{utterance}  

    & & & \multicolumn{2}{p{0.3\linewidth}}{\cellcolor[rgb]{0.95,0.95,0.95}{%
	\makecell[{{p{\linewidth}}}]{% 
	  \tt {\tiny [GM$|$GM]}  
	 Das klingt nach einem vernünftigen Ansatz! Die drei Fingerhüte würden mir tatsächlich gut passen. \\ \tt \\ \tt Bezüglich der Lampe - ich finde sie schon interessant, aber nicht so sehr wie die Fingerhüte. Wenn Sie hauptsächlich am Seil interessiert sind, wäre ich durchaus bereit, auch die Lampe zu nehmen, damit wir eine klare Aufteilung haben: Sie das Seil, ich die drei Fingerhüte und die Lampe. \\ \tt \\ \tt Würde das für Sie funktionieren, oder hätten Sie auch Interesse an der Lampe? 
	  } 
	   } 
	   } 
	 & & \\ 
 

    \theutterance \stepcounter{utterance}  

    & & \multicolumn{4}{p{0.6\linewidth}}{\cellcolor[rgb]{0.9,0.9,0.9}{%
	\makecell[{{p{\linewidth}}}]{% 
	  \tt {\tiny [GM$\rangle$B]}  
	 Das klingt nach einem vernünftigen Ansatz! Die drei Fingerhüte würden mir tatsächlich gut passen. \\ \tt \\ \tt Bezüglich der Lampe - ich finde sie schon interessant, aber nicht so sehr wie die Fingerhüte. Wenn Sie hauptsächlich am Seil interessiert sind, wäre ich durchaus bereit, auch die Lampe zu nehmen, damit wir eine klare Aufteilung haben: Sie das Seil, ich die drei Fingerhüte und die Lampe. \\ \tt \\ \tt Würde das für Sie funktionieren, oder hätten Sie auch Interesse an der Lampe? 
	  } 
	   } 
	   } 
	 & \\ 
 

    \theutterance \stepcounter{utterance}  

    & & & \multicolumn{4}{p{0.6\linewidth}}{\cellcolor[rgb]{1,0.85,0.72}{%
	\makecell[{{p{\linewidth}}}]{% 
	  \tt {\tiny [GM$\langle$B]}  
	 Das ist ein interessanter Vorschlag! Ich schätze Ihre Offenheit. Da Sie gesagt haben, dass die Lampe für Sie nicht so wichtig ist wie die Fingerhüte - wären Sie eventuell auch mit nur den drei Fingerhüten zufrieden? \\ \tt \\ \tt Ich muss gestehen, dass sowohl das Seil als auch die Lampe für mich durchaus nützlich wären. Falls Sie hauptsächlich an den Fingerhüten interessiert sind, könnten wir es so aufteilen: Sie bekommen alle drei Fingerhüte, und ich nehme Seil und Lampe? 
	  } 
	   } 
	   } 
	  \\ 
 

    \theutterance \stepcounter{utterance}  

    & & & \multicolumn{2}{p{0.3\linewidth}}{\cellcolor[rgb]{0.95,0.95,0.95}{%
	\makecell[{{p{\linewidth}}}]{% 
	  \tt {\tiny [GM$|$GM]}  
	 continue 
	  } 
	   } 
	   } 
	 & & \\ 
 

    \theutterance \stepcounter{utterance}  

    & & & \multicolumn{2}{p{0.3\linewidth}}{\cellcolor[rgb]{0.95,0.95,0.95}{%
	\makecell[{{p{\linewidth}}}]{% 
	  \tt {\tiny [GM$|$GM]}  
	 Das ist ein interessanter Vorschlag! Ich schätze Ihre Offenheit. Da Sie gesagt haben, dass die Lampe für Sie nicht so wichtig ist wie die Fingerhüte - wären Sie eventuell auch mit nur den drei Fingerhüten zufrieden? \\ \tt \\ \tt Ich muss gestehen, dass sowohl das Seil als auch die Lampe für mich durchaus nützlich wären. Falls Sie hauptsächlich an den Fingerhüten interessiert sind, könnten wir es so aufteilen: Sie bekommen alle drei Fingerhüte, und ich nehme Seil und Lampe? 
	  } 
	   } 
	   } 
	 & & \\ 
 

    \theutterance \stepcounter{utterance}  

    & & \multicolumn{4}{p{0.6\linewidth}}{\cellcolor[rgb]{0.9,0.9,0.9}{%
	\makecell[{{p{\linewidth}}}]{% 
	  \tt {\tiny [A$\langle$GM]}  
	 Das ist ein interessanter Vorschlag! Ich schätze Ihre Offenheit. Da Sie gesagt haben, dass die Lampe für Sie nicht so wichtig ist wie die Fingerhüte - wären Sie eventuell auch mit nur den drei Fingerhüten zufrieden? \\ \tt \\ \tt Ich muss gestehen, dass sowohl das Seil als auch die Lampe für mich durchaus nützlich wären. Falls Sie hauptsächlich an den Fingerhüten interessiert sind, könnten wir es so aufteilen: Sie bekommen alle drei Fingerhüte, und ich nehme Seil und Lampe? 
	  } 
	   } 
	   } 
	 & \\ 
 

    \theutterance \stepcounter{utterance}  

    & \multicolumn{4}{p{0.6\linewidth}}{\cellcolor[rgb]{0.8,1,0.9}{%
	\makecell[{{p{\linewidth}}}]{% 
	  \tt {\tiny [A$\rangle$GM]}  
	 Hmm, das ist eine faire Frage! Ich verstehe, dass beide Gegenstände für Sie nützlich wären. Die Fingerhüte sind definitiv mein Hauptinteresse, da haben Sie recht.\\ \tt \\ \tt Aber ich muss ehrlich sein - die Lampe hat doch auch ihren Reiz für mich, auch wenn sie nicht meine erste Priorität ist. Da Sie gesagt haben, dass das Seil für Sie wichtiger ist als die Lampe - wie stark ist denn Ihr Interesse an der Lampe wirklich? \\ \tt \\ \tt Vielleicht könnten wir einen Kompromiss finden? Was wäre, wenn Sie das Seil plus einen oder zwei Fingerhüte bekommen, und ich die Lampe plus die restlichen Fingerhüte? 
	  } 
	   } 
	   } 
	 & & \\ 
 

    \theutterance \stepcounter{utterance}  

    & & & \multicolumn{2}{p{0.3\linewidth}}{\cellcolor[rgb]{0.95,0.95,0.95}{%
	\makecell[{{p{\linewidth}}}]{% 
	  \tt {\tiny [GM$|$GM]}  
	 continue 
	  } 
	   } 
	   } 
	 & & \\ 
 

    \theutterance \stepcounter{utterance}  

    & & & \multicolumn{2}{p{0.3\linewidth}}{\cellcolor[rgb]{0.95,0.95,0.95}{%
	\makecell[{{p{\linewidth}}}]{% 
	  \tt {\tiny [GM$|$GM]}  
	 Hmm, das ist eine faire Frage! Ich verstehe, dass beide Gegenstände für Sie nützlich wären. Die Fingerhüte sind definitiv mein Hauptinteresse, da haben Sie recht.\\ \tt \\ \tt Aber ich muss ehrlich sein - die Lampe hat doch auch ihren Reiz für mich, auch wenn sie nicht meine erste Priorität ist. Da Sie gesagt haben, dass das Seil für Sie wichtiger ist als die Lampe - wie stark ist denn Ihr Interesse an der Lampe wirklich? \\ \tt \\ \tt Vielleicht könnten wir einen Kompromiss finden? Was wäre, wenn Sie das Seil plus einen oder zwei Fingerhüte bekommen, und ich die Lampe plus die restlichen Fingerhüte? 
	  } 
	   } 
	   } 
	 & & \\ 
 

    \theutterance \stepcounter{utterance}  

    & & \multicolumn{4}{p{0.6\linewidth}}{\cellcolor[rgb]{0.9,0.9,0.9}{%
	\makecell[{{p{\linewidth}}}]{% 
	  \tt {\tiny [GM$\rangle$B]}  
	 Hmm, das ist eine faire Frage! Ich verstehe, dass beide Gegenstände für Sie nützlich wären. Die Fingerhüte sind definitiv mein Hauptinteresse, da haben Sie recht.\\ \tt \\ \tt Aber ich muss ehrlich sein - die Lampe hat doch auch ihren Reiz für mich, auch wenn sie nicht meine erste Priorität ist. Da Sie gesagt haben, dass das Seil für Sie wichtiger ist als die Lampe - wie stark ist denn Ihr Interesse an der Lampe wirklich? \\ \tt \\ \tt Vielleicht könnten wir einen Kompromiss finden? Was wäre, wenn Sie das Seil plus einen oder zwei Fingerhüte bekommen, und ich die Lampe plus die restlichen Fingerhüte? 
	  } 
	   } 
	   } 
	 & \\ 
 

    \theutterance \stepcounter{utterance}  

    & & & \multicolumn{4}{p{0.6\linewidth}}{\cellcolor[rgb]{1,0.85,0.72}{%
	\makecell[{{p{\linewidth}}}]{% 
	  \tt {\tiny [GM$\langle$B]}  
	 Das ist ein durchaus interessanter Kompromissvorschlag! Sie haben recht, dass das Seil meine höchste Priorität hat.\\ \tt \\ \tt Ein oder zwei Fingerhüte zusätzlich zum Seil... das könnte funktionieren. Wenn ich ehrlich bin, wäre das Seil plus zwei Fingerhüte für mich ein guter Deal - dann hätten Sie die Lampe plus einen Fingerhut. \\ \tt \\ \tt Oder falls Sie lieber mehr Fingerhüte hätten: Seil plus ein Fingerhut für mich, und Sie bekommen Lampe plus zwei Fingerhüte?\\ \tt \\ \tt Was denken Sie - welche dieser beiden Varianten spricht Sie mehr an? 
	  } 
	   } 
	   } 
	  \\ 
 

    \theutterance \stepcounter{utterance}  

    & & & \multicolumn{2}{p{0.3\linewidth}}{\cellcolor[rgb]{0.95,0.95,0.95}{%
	\makecell[{{p{\linewidth}}}]{% 
	  \tt {\tiny [GM$|$GM]}  
	 continue 
	  } 
	   } 
	   } 
	 & & \\ 
 

    \theutterance \stepcounter{utterance}  

    & & & \multicolumn{2}{p{0.3\linewidth}}{\cellcolor[rgb]{0.95,0.95,0.95}{%
	\makecell[{{p{\linewidth}}}]{% 
	  \tt {\tiny [GM$|$GM]}  
	 Das ist ein durchaus interessanter Kompromissvorschlag! Sie haben recht, dass das Seil meine höchste Priorität hat.\\ \tt \\ \tt Ein oder zwei Fingerhüte zusätzlich zum Seil... das könnte funktionieren. Wenn ich ehrlich bin, wäre das Seil plus zwei Fingerhüte für mich ein guter Deal - dann hätten Sie die Lampe plus einen Fingerhut. \\ \tt \\ \tt Oder falls Sie lieber mehr Fingerhüte hätten: Seil plus ein Fingerhut für mich, und Sie bekommen Lampe plus zwei Fingerhüte?\\ \tt \\ \tt Was denken Sie - welche dieser beiden Varianten spricht Sie mehr an? 
	  } 
	   } 
	   } 
	 & & \\ 
 

    \theutterance \stepcounter{utterance}  

    & & \multicolumn{4}{p{0.6\linewidth}}{\cellcolor[rgb]{0.9,0.9,0.9}{%
	\makecell[{{p{\linewidth}}}]{% 
	  \tt {\tiny [A$\langle$GM]}  
	 Das ist ein durchaus interessanter Kompromissvorschlag! Sie haben recht, dass das Seil meine höchste Priorität hat.\\ \tt \\ \tt Ein oder zwei Fingerhüte zusätzlich zum Seil... das könnte funktionieren. Wenn ich ehrlich bin, wäre das Seil plus zwei Fingerhüte für mich ein guter Deal - dann hätten Sie die Lampe plus einen Fingerhut. \\ \tt \\ \tt Oder falls Sie lieber mehr Fingerhüte hätten: Seil plus ein Fingerhut für mich, und Sie bekommen Lampe plus zwei Fingerhüte?\\ \tt \\ \tt Was denken Sie - welche dieser beiden Varianten spricht Sie mehr an? 
	  } 
	   } 
	   } 
	 & \\ 
 

    \theutterance \stepcounter{utterance}  

    & \multicolumn{4}{p{0.6\linewidth}}{\cellcolor[rgb]{0.8,1,0.9}{%
	\makecell[{{p{\linewidth}}}]{% 
	  \tt {\tiny [A$\rangle$GM]}  
	 Das sind beide vernünftige Optionen! Lassen Sie mich kurz überlegen...\\ \tt \\ \tt Wenn ich ehrlich bin, sprechen mich die Fingerhüte wirklich an - sie haben etwas Faszinierendes an sich. Die zweite Variante, wo ich die Lampe plus zwei Fingerhüte bekommen würde und Sie das Seil plus einen Fingerhut, klingt für mich attraktiver.\\ \tt \\ \tt Das würde bedeuten: Sie bekommen Ihr wichtigstes Objekt, das Seil, plus noch einen Fingerhut als Bonus. Ich würde die Lampe und zwei Fingerhüte nehmen. \\ \tt \\ \tt Sind Sie mit dieser Aufteilung einverstanden? Falls ja, können wir unsere Vorschläge machen! 
	  } 
	   } 
	   } 
	 & & \\ 
 

    \theutterance \stepcounter{utterance}  

    & & & \multicolumn{2}{p{0.3\linewidth}}{\cellcolor[rgb]{0.95,0.95,0.95}{%
	\makecell[{{p{\linewidth}}}]{% 
	  \tt {\tiny [GM$|$GM]}  
	 continue 
	  } 
	   } 
	   } 
	 & & \\ 
 

    \theutterance \stepcounter{utterance}  

    & & & \multicolumn{2}{p{0.3\linewidth}}{\cellcolor[rgb]{0.95,0.95,0.95}{%
	\makecell[{{p{\linewidth}}}]{% 
	  \tt {\tiny [GM$|$GM]}  
	 Das sind beide vernünftige Optionen! Lassen Sie mich kurz überlegen...\\ \tt \\ \tt Wenn ich ehrlich bin, sprechen mich die Fingerhüte wirklich an - sie haben etwas Faszinierendes an sich. Die zweite Variante, wo ich die Lampe plus zwei Fingerhüte bekommen würde und Sie das Seil plus einen Fingerhut, klingt für mich attraktiver.\\ \tt \\ \tt Das würde bedeuten: Sie bekommen Ihr wichtigstes Objekt, das Seil, plus noch einen Fingerhut als Bonus. Ich würde die Lampe und zwei Fingerhüte nehmen. \\ \tt \\ \tt Sind Sie mit dieser Aufteilung einverstanden? Falls ja, können wir unsere Vorschläge machen! 
	  } 
	   } 
	   } 
	 & & \\ 
 

    \theutterance \stepcounter{utterance}  

    & & \multicolumn{4}{p{0.6\linewidth}}{\cellcolor[rgb]{0.9,0.9,0.9}{%
	\makecell[{{p{\linewidth}}}]{% 
	  \tt {\tiny [GM$\rangle$B]}  
	 Das sind beide vernünftige Optionen! Lassen Sie mich kurz überlegen...\\ \tt \\ \tt Wenn ich ehrlich bin, sprechen mich die Fingerhüte wirklich an - sie haben etwas Faszinierendes an sich. Die zweite Variante, wo ich die Lampe plus zwei Fingerhüte bekommen würde und Sie das Seil plus einen Fingerhut, klingt für mich attraktiver.\\ \tt \\ \tt Das würde bedeuten: Sie bekommen Ihr wichtigstes Objekt, das Seil, plus noch einen Fingerhut als Bonus. Ich würde die Lampe und zwei Fingerhüte nehmen. \\ \tt \\ \tt Sind Sie mit dieser Aufteilung einverstanden? Falls ja, können wir unsere Vorschläge machen! 
	  } 
	   } 
	   } 
	 & \\ 
 

    \theutterance \stepcounter{utterance}  

    & & & \multicolumn{4}{p{0.6\linewidth}}{\cellcolor[rgb]{1,0.85,0.72}{%
	\makecell[{{p{\linewidth}}}]{% 
	  \tt {\tiny [GM$\langle$B]}  
	 Das klingt perfekt! Sie haben recht - ich bekomme mein wichtigstes Objekt, das Seil, und noch einen kleinen Bonus mit dem Fingerhut. Und Sie bekommen eine schöne Sammlung mit der Lampe und zwei Fingerhüten.\\ \tt \\ \tt Ich bin mit dieser Aufteilung sehr einverstanden. Lassen Sie uns unsere Vorschläge machen!\\ \tt \\ \tt [Vorschlag: 1 Seil, 1 Fingerhut] 
	  } 
	   } 
	   } 
	  \\ 
 

    \theutterance \stepcounter{utterance}  

    & & & \multicolumn{2}{p{0.3\linewidth}}{\cellcolor[rgb]{0.95,0.95,0.95}{%
	\makecell[{{p{\linewidth}}}]{% 
	  \tt {\tiny [GM$|$GM]}  
	 proposal 
	  } 
	   } 
	   } 
	 & & \\ 
 

    \theutterance \stepcounter{utterance}  

    & & & \multicolumn{2}{p{0.3\linewidth}}{\cellcolor[rgb]{0.95,0.95,0.95}{%
	\makecell[{{p{\linewidth}}}]{% 
	  \tt {\tiny [GM$|$GM]}  
	 [0, 1, 1] 
	  } 
	   } 
	   } 
	 & & \\ 
 

    \theutterance \stepcounter{utterance}  

    & & \multicolumn{4}{p{0.6\linewidth}}{\cellcolor[rgb]{0.9,0.9,0.9}{%
	\makecell[{{p{\linewidth}}}]{% 
	  \tt {\tiny [A$\langle$GM]}  
	 Der andere Spieler hat einen geheimen Vorschlag gemacht. Jetzt bist du an der Reihe, einen eigenen Vorschlag zu machen.\\ \tt  
	  } 
	   } 
	   } 
	 & \\ 
 

    \theutterance \stepcounter{utterance}  

    & \multicolumn{4}{p{0.6\linewidth}}{\cellcolor[rgb]{0.8,1,0.9}{%
	\makecell[{{p{\linewidth}}}]{% 
	  \tt {\tiny [A$\rangle$GM]}  
	 [Vorschlag: 1 Lampe, 2 Fingerhut] 
	  } 
	   } 
	   } 
	 & & \\ 
 

    \theutterance \stepcounter{utterance}  

    & & & \multicolumn{2}{p{0.3\linewidth}}{\cellcolor[rgb]{0.95,0.95,0.95}{%
	\makecell[{{p{\linewidth}}}]{% 
	  \tt {\tiny [GM$|$GM]}  
	 proposal 
	  } 
	   } 
	   } 
	 & & \\ 
 

    \theutterance \stepcounter{utterance}  

    & & & \multicolumn{2}{p{0.3\linewidth}}{\cellcolor[rgb]{0.95,0.95,0.95}{%
	\makecell[{{p{\linewidth}}}]{% 
	  \tt {\tiny [GM$|$GM]}  
	 [1, 2, 0] 
	  } 
	   } 
	   } 
	 & & \\ 
 

    \theutterance \stepcounter{utterance}  

    & & & \multicolumn{2}{p{0.3\linewidth}}{\cellcolor[rgb]{0.95,0.95,0.95}{%
	\makecell[{{p{\linewidth}}}]{% 
	  \tt {\tiny [GM$|$GM]}  
	 [[1, 2, 0], [0, 1, 1]] 
	  } 
	   } 
	   } 
	 & & \\ 
 

\end{supertabular}
}

\end{document}
