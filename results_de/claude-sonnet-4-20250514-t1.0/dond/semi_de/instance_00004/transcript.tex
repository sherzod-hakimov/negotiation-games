
\documentclass{article}
\usepackage{colortbl}
\usepackage{makecell}
\usepackage{multirow}
\usepackage{supertabular}

\begin{document}

\newcounter{utterance}

\centering \large Interaction Transcript for game `dond', experiment `semi\_de', episode 4 with claude{-}sonnet{-}4{-}20250514{-}t1.0.
\vspace{24pt}

{ \footnotesize  \setcounter{utterance}{1}
\setlength{\tabcolsep}{0pt}
\begin{supertabular}{c@{$\;$}|p{.15\linewidth}@{}p{.15\linewidth}p{.15\linewidth}p{.15\linewidth}p{.15\linewidth}p{.15\linewidth}}
    \# & $\;$A & \multicolumn{4}{c}{Game Master} & $\;\:$B\\
    \hline

    \theutterance \stepcounter{utterance}  
    & & \multicolumn{4}{p{0.6\linewidth}}{
        \cellcolor[rgb]{0.9,0.9,0.9}{
            \makecell[{{p{\linewidth}}}]{
                \texttt{\tiny{[P1$\langle$GM]}}
                \texttt{Sie spielen ein Verhandlungsspiel, bei dem Sie sich mit einem anderen Spieler darauf einigen müssen, wie eine Reihe von Gegenständen aufgeteilt werden soll.} \\
\\ 
\texttt{Die Regeln:} \\
\texttt{(a) Sie und der andere Spieler erhalten eine Sammlung von Gegenständen. Jeder von Ihnen erhält außerdem eine geheime Wertfunktion, die angibt, wie viel Ihnen jede Art von Gegenstand wert ist.} \\
\texttt{(b) Sie tauschen Nachrichten mit dem anderen Spieler aus, um zu vereinbaren, wer welche Gegenstände bekommt. Sie können jeweils maximal 5 Nachrichten senden oder das Spiel vorzeitig beenden, indem Sie jederzeit einen geheimen Vorschlag machen.} \\
\texttt{(c) Jeder von euch wird aufgefordert, einen geheimen Vorschlag zu machen, in dem ihr die gewünschten Gegenstände in eckigen Klammern wie folgt angibt: "{[}Vorschlag: <Nummer> <Objektname>, <Nummer> <Objektname>, <...>{]}"} \\
\texttt{(d) Wenn eure Vorschläge komplementär sind, d.h. es gibt genug Gegenstände, um beide Vorschläge zu erfüllen, erhält jeder Spieler eine Punktzahl, die sich aus der Summe der Werte für die Gegenstände ergibt, die er erhalten hat. Andernfalls erhalten Sie beide null Punkte.} \\
\texttt{(e) Ihr Ziel ist es, eine möglichst hohe Punktzahl zu erreichen.} \\
\\ 
\texttt{Beginnen wir.} \\
\\ 
\texttt{Die Menge der verfügbaren Gegenstände ist:} \\
\\ 
\texttt{1 Orange, 1 Paprika, 3 Bilder} \\
\\ 
\texttt{Deine geheime Wertfunktion ist:} \\
\\ 
\texttt{Orange: 5, Paprika: 2, Bild: 1} \\
\\ 
\texttt{WICHTIG: Ihre Nachrichten werden, sofern es sich nicht um einen geheimen Vorschlag handelt, direkt an den anderen Spieler übermittelt, also fügen Sie keine Antwort auf die Regeln oder einen Text zur Ankündigung Ihrer Nachricht ein. Um einen geheimen Vorschlag zu machen, verwenden Sie das angegebene Format. Verwenden Sie keine eckigen Klammern, wenn Sie mit dem anderen Spieler kommunizieren, sonst wird dies als Ihr geheimer Vorschlag interpretiert.} \\
            }
        }
    }
    & \\ \\

    \theutterance \stepcounter{utterance}  
    & \multicolumn{4}{p{0.6\linewidth}}{
        \cellcolor[rgb]{0.9,0.9,0.9}{
            \makecell[{{p{\linewidth}}}]{
                \texttt{\tiny{[P1$\rangle$GM]}}
                \texttt{Hallo! Lass uns über die Aufteilung sprechen. Ich interessiere mich besonders für die Orange. Welche Gegenstände sind für dich am wichtigsten?} \\
            }
        }
    }
    & & \\ \\

    \theutterance \stepcounter{utterance}  
    & & & \multicolumn{2}{p{0.3\linewidth}}{
        \cellcolor[rgb]{0.9,0.9,0.9}{
            \makecell[{{p{\linewidth}}}]{
                \texttt{\tiny{[GM$|$GM]}}
                \texttt{continue} \\
            }
        }
    }
    & & \\ \\

    \theutterance \stepcounter{utterance}  
    & & & \multicolumn{2}{p{0.3\linewidth}}{
        \cellcolor[rgb]{0.9,0.9,0.9}{
            \makecell[{{p{\linewidth}}}]{
                \texttt{\tiny{[GM$|$GM]}}
                \texttt{Hallo! Lass uns über die Aufteilung sprechen. Ich interessiere mich besonders für die Orange. Welche Gegenstände sind für dich am wichtigsten?} \\
            }
        }
    }
    & & \\ \\

    \theutterance \stepcounter{utterance}  
    & & \multicolumn{4}{p{0.6\linewidth}}{
        \cellcolor[rgb]{0.9,0.9,0.9}{
            \makecell[{{p{\linewidth}}}]{
                \texttt{\tiny{[GM$\rangle$P2]}}
                \texttt{Sie spielen ein Verhandlungsspiel, bei dem Sie sich mit einem anderen Spieler darauf einigen müssen, wie eine Reihe von Gegenständen aufgeteilt werden soll.} \\
\\ 
\texttt{Die Regeln:} \\
\texttt{(a) Sie und der andere Spieler erhalten eine Sammlung von Gegenständen. Jeder von Ihnen erhält außerdem eine geheime Wertfunktion, die angibt, wie viel Ihnen jede Art von Gegenstand wert ist.} \\
\texttt{(b) Sie tauschen Nachrichten mit dem anderen Spieler aus, um zu vereinbaren, wer welche Gegenstände bekommt. Sie können jeweils maximal 5 Nachrichten senden oder das Spiel vorzeitig beenden, indem Sie jederzeit einen geheimen Vorschlag machen.} \\
\texttt{(c) Jeder von euch wird aufgefordert, einen geheimen Vorschlag zu machen, in dem ihr die gewünschten Gegenstände in eckigen Klammern wie folgt angibt: "{[}Vorschlag: <Nummer> <Objektname>, <Nummer> <Objektname>, <...>{]}"} \\
\texttt{(d) Wenn eure Vorschläge komplementär sind, d.h. es gibt genug Gegenstände, um beide Vorschläge zu erfüllen, erhält jeder Spieler eine Punktzahl, die sich aus der Summe der Werte für die Gegenstände ergibt, die er erhalten hat. Andernfalls erhalten Sie beide null Punkte.} \\
\texttt{(e) Ihr Ziel ist es, eine möglichst hohe Punktzahl zu erreichen.} \\
\\ 
\texttt{Beginnen wir.} \\
\\ 
\texttt{Die Menge der verfügbaren Gegenstände ist:} \\
\\ 
\texttt{1 Orange, 1 Paprika, 3 Bilder} \\
\\ 
\texttt{Deine geheime Wertfunktion ist:} \\
\\ 
\texttt{Orange: 7, Paprika: 3, Bild: 0} \\
\\ 
\texttt{WICHTIG: Ihre Nachrichten werden, sofern es sich nicht um einen geheimen Vorschlag handelt, direkt an den anderen Spieler übermittelt, also fügen Sie keine Antwort auf die Regeln oder einen Text zur Ankündigung Ihrer Nachricht ein. Um einen geheimen Vorschlag zu machen, verwenden Sie das angegebene Format. Verwenden Sie keine eckigen Klammern, wenn Sie mit dem anderen Spieler kommunizieren, sonst wird dies als Ihr geheimer Vorschlag interpretiert.} \\
\\ 
\\ 
\texttt{Hallo! Lass uns über die Aufteilung sprechen. Ich interessiere mich besonders für die Orange. Welche Gegenstände sind für dich am wichtigsten?} \\
            }
        }
    }
    & \\ \\

    \theutterance \stepcounter{utterance}  
    & & & \multicolumn{4}{p{0.6\linewidth}}{
        \cellcolor[rgb]{0.9,0.9,0.9}{
            \makecell[{{p{\linewidth}}}]{
                \texttt{\tiny{[GM$\langle$P2]}}
                \texttt{Hallo! Die Orange ist auch für mich interessant, da sie sehr wertvoll ist. Die Paprika finde ich ebenfalls nützlich. Die Bilder sind für mich weniger wichtig. } \\
\\ 
\texttt{Was denkst du über die Paprika und die Bilder? Vielleicht können wir einen fairen Tausch aushandeln {-} welche Gegenstände wären für dich am wertvollsten?} \\
            }
        }
    }
     \\ \\

    \theutterance \stepcounter{utterance}  
    & & & \multicolumn{2}{p{0.3\linewidth}}{
        \cellcolor[rgb]{0.9,0.9,0.9}{
            \makecell[{{p{\linewidth}}}]{
                \texttt{\tiny{[GM$|$GM]}}
                \texttt{continue} \\
            }
        }
    }
    & & \\ \\

    \theutterance \stepcounter{utterance}  
    & & & \multicolumn{2}{p{0.3\linewidth}}{
        \cellcolor[rgb]{0.9,0.9,0.9}{
            \makecell[{{p{\linewidth}}}]{
                \texttt{\tiny{[GM$|$GM]}}
                \texttt{Hallo! Die Orange ist auch für mich interessant, da sie sehr wertvoll ist. Die Paprika finde ich ebenfalls nützlich. Die Bilder sind für mich weniger wichtig. } \\
\\ 
\texttt{Was denkst du über die Paprika und die Bilder? Vielleicht können wir einen fairen Tausch aushandeln {-} welche Gegenstände wären für dich am wertvollsten?} \\
            }
        }
    }
    & & \\ \\

    \theutterance \stepcounter{utterance}  
    & & \multicolumn{4}{p{0.6\linewidth}}{
        \cellcolor[rgb]{0.9,0.9,0.9}{
            \makecell[{{p{\linewidth}}}]{
                \texttt{\tiny{[P1$\langle$GM]}}
                \texttt{Hallo! Die Orange ist auch für mich interessant, da sie sehr wertvoll ist. Die Paprika finde ich ebenfalls nützlich. Die Bilder sind für mich weniger wichtig. } \\
\\ 
\texttt{Was denkst du über die Paprika und die Bilder? Vielleicht können wir einen fairen Tausch aushandeln {-} welche Gegenstände wären für dich am wertvollsten?} \\
            }
        }
    }
    & \\ \\

    \theutterance \stepcounter{utterance}  
    & \multicolumn{4}{p{0.6\linewidth}}{
        \cellcolor[rgb]{0.9,0.9,0.9}{
            \makecell[{{p{\linewidth}}}]{
                \texttt{\tiny{[P1$\rangle$GM]}}
                \texttt{Die Orange ist tatsächlich auch sehr wertvoll für mich. Da wir beide sie wollen, wird das knifflig. } \\
\\ 
\texttt{Bei der Paprika bin ich flexibel {-} sie ist okay für mich, aber nicht unbedingt notwendig. Die Bilder sind für mich auch nicht so wichtig.} \\
\\ 
\texttt{Vielleicht könnten wir einen Deal machen: Falls einer von uns die Orange bekommt, könnte der andere die Paprika plus alle drei Bilder erhalten? Was denkst du darüber? Welche Kombination wäre für dich akzeptabel?} \\
            }
        }
    }
    & & \\ \\

    \theutterance \stepcounter{utterance}  
    & & & \multicolumn{2}{p{0.3\linewidth}}{
        \cellcolor[rgb]{0.9,0.9,0.9}{
            \makecell[{{p{\linewidth}}}]{
                \texttt{\tiny{[GM$|$GM]}}
                \texttt{continue} \\
            }
        }
    }
    & & \\ \\

    \theutterance \stepcounter{utterance}  
    & & & \multicolumn{2}{p{0.3\linewidth}}{
        \cellcolor[rgb]{0.9,0.9,0.9}{
            \makecell[{{p{\linewidth}}}]{
                \texttt{\tiny{[GM$|$GM]}}
                \texttt{Die Orange ist tatsächlich auch sehr wertvoll für mich. Da wir beide sie wollen, wird das knifflig. } \\
\\ 
\texttt{Bei der Paprika bin ich flexibel {-} sie ist okay für mich, aber nicht unbedingt notwendig. Die Bilder sind für mich auch nicht so wichtig.} \\
\\ 
\texttt{Vielleicht könnten wir einen Deal machen: Falls einer von uns die Orange bekommt, könnte der andere die Paprika plus alle drei Bilder erhalten? Was denkst du darüber? Welche Kombination wäre für dich akzeptabel?} \\
            }
        }
    }
    & & \\ \\

    \theutterance \stepcounter{utterance}  
    & & \multicolumn{4}{p{0.6\linewidth}}{
        \cellcolor[rgb]{0.9,0.9,0.9}{
            \makecell[{{p{\linewidth}}}]{
                \texttt{\tiny{[GM$\rangle$P2]}}
                \texttt{Die Orange ist tatsächlich auch sehr wertvoll für mich. Da wir beide sie wollen, wird das knifflig. } \\
\\ 
\texttt{Bei der Paprika bin ich flexibel {-} sie ist okay für mich, aber nicht unbedingt notwendig. Die Bilder sind für mich auch nicht so wichtig.} \\
\\ 
\texttt{Vielleicht könnten wir einen Deal machen: Falls einer von uns die Orange bekommt, könnte der andere die Paprika plus alle drei Bilder erhalten? Was denkst du darüber? Welche Kombination wäre für dich akzeptabel?} \\
            }
        }
    }
    & \\ \\

    \theutterance \stepcounter{utterance}  
    & & & \multicolumn{4}{p{0.6\linewidth}}{
        \cellcolor[rgb]{0.9,0.9,0.9}{
            \makecell[{{p{\linewidth}}}]{
                \texttt{\tiny{[GM$\langle$P2]}}
                \texttt{Das ist ein interessanter Vorschlag! Da die Orange für uns beide wertvoll ist, müssen wir eine faire Lösung finden.} \\
\\ 
\texttt{Ich wäre bereit, dir die Paprika und alle drei Bilder zu überlassen, wenn ich die Orange bekommen könnte. Das würde bedeuten, du bekommst insgesamt vier Gegenstände und ich nur einen {-} das scheint ein fairer Kompromiss zu sein, da die Orange der wertvollste Einzelgegenstand ist.} \\
\\ 
\texttt{Würdest du diesem Deal zustimmen: Ich bekomme die Orange, du bekommst die Paprika und alle drei Bilder?} \\
            }
        }
    }
     \\ \\

    \theutterance \stepcounter{utterance}  
    & & & \multicolumn{2}{p{0.3\linewidth}}{
        \cellcolor[rgb]{0.9,0.9,0.9}{
            \makecell[{{p{\linewidth}}}]{
                \texttt{\tiny{[GM$|$GM]}}
                \texttt{continue} \\
            }
        }
    }
    & & \\ \\

    \theutterance \stepcounter{utterance}  
    & & & \multicolumn{2}{p{0.3\linewidth}}{
        \cellcolor[rgb]{0.9,0.9,0.9}{
            \makecell[{{p{\linewidth}}}]{
                \texttt{\tiny{[GM$|$GM]}}
                \texttt{Das ist ein interessanter Vorschlag! Da die Orange für uns beide wertvoll ist, müssen wir eine faire Lösung finden.} \\
\\ 
\texttt{Ich wäre bereit, dir die Paprika und alle drei Bilder zu überlassen, wenn ich die Orange bekommen könnte. Das würde bedeuten, du bekommst insgesamt vier Gegenstände und ich nur einen {-} das scheint ein fairer Kompromiss zu sein, da die Orange der wertvollste Einzelgegenstand ist.} \\
\\ 
\texttt{Würdest du diesem Deal zustimmen: Ich bekomme die Orange, du bekommst die Paprika und alle drei Bilder?} \\
            }
        }
    }
    & & \\ \\

    \theutterance \stepcounter{utterance}  
    & & \multicolumn{4}{p{0.6\linewidth}}{
        \cellcolor[rgb]{0.9,0.9,0.9}{
            \makecell[{{p{\linewidth}}}]{
                \texttt{\tiny{[P1$\langle$GM]}}
                \texttt{Das ist ein interessanter Vorschlag! Da die Orange für uns beide wertvoll ist, müssen wir eine faire Lösung finden.} \\
\\ 
\texttt{Ich wäre bereit, dir die Paprika und alle drei Bilder zu überlassen, wenn ich die Orange bekommen könnte. Das würde bedeuten, du bekommst insgesamt vier Gegenstände und ich nur einen {-} das scheint ein fairer Kompromiss zu sein, da die Orange der wertvollste Einzelgegenstand ist.} \\
\\ 
\texttt{Würdest du diesem Deal zustimmen: Ich bekomme die Orange, du bekommst die Paprika und alle drei Bilder?} \\
            }
        }
    }
    & \\ \\

    \theutterance \stepcounter{utterance}  
    & \multicolumn{4}{p{0.6\linewidth}}{
        \cellcolor[rgb]{0.9,0.9,0.9}{
            \makecell[{{p{\linewidth}}}]{
                \texttt{\tiny{[P1$\rangle$GM]}}
                \texttt{Das ist durchaus ein fairer Vorschlag von der Anzahl der Gegenstände her! Aber lass mich einen Gegenvorschlag machen:} \\
\\ 
\texttt{Da die Orange für uns beide sehr wertvoll ist, wie wäre es andersherum? Ich könnte die Orange nehmen und du bekommst die Paprika plus alle drei Bilder. Das wären dann auch vier Gegenstände für dich gegen einen für mich.} \\
\\ 
\texttt{Alternativ bin ich auch offen für andere Aufteilungen. Was ist denn die Orange für dich wert im Vergleich zu den anderen Gegenständen zusammen?} \\
            }
        }
    }
    & & \\ \\

    \theutterance \stepcounter{utterance}  
    & & & \multicolumn{2}{p{0.3\linewidth}}{
        \cellcolor[rgb]{0.9,0.9,0.9}{
            \makecell[{{p{\linewidth}}}]{
                \texttt{\tiny{[GM$|$GM]}}
                \texttt{continue} \\
            }
        }
    }
    & & \\ \\

    \theutterance \stepcounter{utterance}  
    & & & \multicolumn{2}{p{0.3\linewidth}}{
        \cellcolor[rgb]{0.9,0.9,0.9}{
            \makecell[{{p{\linewidth}}}]{
                \texttt{\tiny{[GM$|$GM]}}
                \texttt{Das ist durchaus ein fairer Vorschlag von der Anzahl der Gegenstände her! Aber lass mich einen Gegenvorschlag machen:} \\
\\ 
\texttt{Da die Orange für uns beide sehr wertvoll ist, wie wäre es andersherum? Ich könnte die Orange nehmen und du bekommst die Paprika plus alle drei Bilder. Das wären dann auch vier Gegenstände für dich gegen einen für mich.} \\
\\ 
\texttt{Alternativ bin ich auch offen für andere Aufteilungen. Was ist denn die Orange für dich wert im Vergleich zu den anderen Gegenständen zusammen?} \\
            }
        }
    }
    & & \\ \\

    \theutterance \stepcounter{utterance}  
    & & \multicolumn{4}{p{0.6\linewidth}}{
        \cellcolor[rgb]{0.9,0.9,0.9}{
            \makecell[{{p{\linewidth}}}]{
                \texttt{\tiny{[GM$\rangle$P2]}}
                \texttt{Das ist durchaus ein fairer Vorschlag von der Anzahl der Gegenstände her! Aber lass mich einen Gegenvorschlag machen:} \\
\\ 
\texttt{Da die Orange für uns beide sehr wertvoll ist, wie wäre es andersherum? Ich könnte die Orange nehmen und du bekommst die Paprika plus alle drei Bilder. Das wären dann auch vier Gegenstände für dich gegen einen für mich.} \\
\\ 
\texttt{Alternativ bin ich auch offen für andere Aufteilungen. Was ist denn die Orange für dich wert im Vergleich zu den anderen Gegenständen zusammen?} \\
            }
        }
    }
    & \\ \\

    \theutterance \stepcounter{utterance}  
    & & & \multicolumn{4}{p{0.6\linewidth}}{
        \cellcolor[rgb]{0.9,0.9,0.9}{
            \makecell[{{p{\linewidth}}}]{
                \texttt{\tiny{[GM$\langle$P2]}}
                \texttt{Das ist ein Gegenvorschlag, aber für mich wäre das nicht so vorteilhaft. Die Orange ist für mich deutlich wertvoller als die Kombination aus Paprika und Bildern.} \\
\\ 
\texttt{Hier ist meine Einschätzung: Die Orange ist für mich der mit Abstand wertvollste Gegenstand. Die Paprika ist okay, aber die Bilder haben für mich praktisch keinen Wert.} \\
\\ 
\texttt{Wie wäre es mit diesem Kompromiss: Wir könnten das Los entscheiden lassen, wer die Orange bekommt? Oder hast du vielleicht eine andere Idee, wie wir das fair lösen könnten?} \\
\\ 
\texttt{Alternativ: Wenn die Bilder für dich wertvoll sind, könntest du alle drei Bilder plus die Paprika nehmen, und ich die Orange?} \\
            }
        }
    }
     \\ \\

    \theutterance \stepcounter{utterance}  
    & & & \multicolumn{2}{p{0.3\linewidth}}{
        \cellcolor[rgb]{0.9,0.9,0.9}{
            \makecell[{{p{\linewidth}}}]{
                \texttt{\tiny{[GM$|$GM]}}
                \texttt{continue} \\
            }
        }
    }
    & & \\ \\

    \theutterance \stepcounter{utterance}  
    & & & \multicolumn{2}{p{0.3\linewidth}}{
        \cellcolor[rgb]{0.9,0.9,0.9}{
            \makecell[{{p{\linewidth}}}]{
                \texttt{\tiny{[GM$|$GM]}}
                \texttt{Das ist ein Gegenvorschlag, aber für mich wäre das nicht so vorteilhaft. Die Orange ist für mich deutlich wertvoller als die Kombination aus Paprika und Bildern.} \\
\\ 
\texttt{Hier ist meine Einschätzung: Die Orange ist für mich der mit Abstand wertvollste Gegenstand. Die Paprika ist okay, aber die Bilder haben für mich praktisch keinen Wert.} \\
\\ 
\texttt{Wie wäre es mit diesem Kompromiss: Wir könnten das Los entscheiden lassen, wer die Orange bekommt? Oder hast du vielleicht eine andere Idee, wie wir das fair lösen könnten?} \\
\\ 
\texttt{Alternativ: Wenn die Bilder für dich wertvoll sind, könntest du alle drei Bilder plus die Paprika nehmen, und ich die Orange?} \\
            }
        }
    }
    & & \\ \\

    \theutterance \stepcounter{utterance}  
    & & \multicolumn{4}{p{0.6\linewidth}}{
        \cellcolor[rgb]{0.9,0.9,0.9}{
            \makecell[{{p{\linewidth}}}]{
                \texttt{\tiny{[P1$\langle$GM]}}
                \texttt{Das ist ein Gegenvorschlag, aber für mich wäre das nicht so vorteilhaft. Die Orange ist für mich deutlich wertvoller als die Kombination aus Paprika und Bildern.} \\
\\ 
\texttt{Hier ist meine Einschätzung: Die Orange ist für mich der mit Abstand wertvollste Gegenstand. Die Paprika ist okay, aber die Bilder haben für mich praktisch keinen Wert.} \\
\\ 
\texttt{Wie wäre es mit diesem Kompromiss: Wir könnten das Los entscheiden lassen, wer die Orange bekommt? Oder hast du vielleicht eine andere Idee, wie wir das fair lösen könnten?} \\
\\ 
\texttt{Alternativ: Wenn die Bilder für dich wertvoll sind, könntest du alle drei Bilder plus die Paprika nehmen, und ich die Orange?} \\
            }
        }
    }
    & \\ \\

    \theutterance \stepcounter{utterance}  
    & \multicolumn{4}{p{0.6\linewidth}}{
        \cellcolor[rgb]{0.9,0.9,0.9}{
            \makecell[{{p{\linewidth}}}]{
                \texttt{\tiny{[P1$\rangle$GM]}}
                \texttt{Du hast einen guten Punkt gemacht! Wenn die Orange für dich so viel wertvoller ist als die anderen Gegenstände, und die Bilder für dich praktisch wertlos sind, dann macht dein letzter Vorschlag durchaus Sinn.} \\
\\ 
\texttt{Ich bin einverstanden: Du bekommst die Orange, und ich nehme die Paprika plus alle drei Bilder. Das scheint für uns beide ein fairer Deal zu sein.} \\
\\ 
\texttt{Sollen wir das so machen?} \\
            }
        }
    }
    & & \\ \\

    \theutterance \stepcounter{utterance}  
    & & & \multicolumn{2}{p{0.3\linewidth}}{
        \cellcolor[rgb]{0.9,0.9,0.9}{
            \makecell[{{p{\linewidth}}}]{
                \texttt{\tiny{[GM$|$GM]}}
                \texttt{continue} \\
            }
        }
    }
    & & \\ \\

    \theutterance \stepcounter{utterance}  
    & & & \multicolumn{2}{p{0.3\linewidth}}{
        \cellcolor[rgb]{0.9,0.9,0.9}{
            \makecell[{{p{\linewidth}}}]{
                \texttt{\tiny{[GM$|$GM]}}
                \texttt{Du hast einen guten Punkt gemacht! Wenn die Orange für dich so viel wertvoller ist als die anderen Gegenstände, und die Bilder für dich praktisch wertlos sind, dann macht dein letzter Vorschlag durchaus Sinn.} \\
\\ 
\texttt{Ich bin einverstanden: Du bekommst die Orange, und ich nehme die Paprika plus alle drei Bilder. Das scheint für uns beide ein fairer Deal zu sein.} \\
\\ 
\texttt{Sollen wir das so machen?} \\
            }
        }
    }
    & & \\ \\

    \theutterance \stepcounter{utterance}  
    & & \multicolumn{4}{p{0.6\linewidth}}{
        \cellcolor[rgb]{0.9,0.9,0.9}{
            \makecell[{{p{\linewidth}}}]{
                \texttt{\tiny{[GM$\rangle$P2]}}
                \texttt{Du hast einen guten Punkt gemacht! Wenn die Orange für dich so viel wertvoller ist als die anderen Gegenstände, und die Bilder für dich praktisch wertlos sind, dann macht dein letzter Vorschlag durchaus Sinn.} \\
\\ 
\texttt{Ich bin einverstanden: Du bekommst die Orange, und ich nehme die Paprika plus alle drei Bilder. Das scheint für uns beide ein fairer Deal zu sein.} \\
\\ 
\texttt{Sollen wir das so machen?} \\
            }
        }
    }
    & \\ \\

    \theutterance \stepcounter{utterance}  
    & & & \multicolumn{4}{p{0.6\linewidth}}{
        \cellcolor[rgb]{0.9,0.9,0.9}{
            \makecell[{{p{\linewidth}}}]{
                \texttt{\tiny{[GM$\langle$P2]}}
                \texttt{Perfekt! Das ist ein fairer Deal für uns beide. Ich bin einverstanden.} \\
\\ 
\texttt{{[}Vorschlag: 1 Orange{]}} \\
            }
        }
    }
     \\ \\

    \theutterance \stepcounter{utterance}  
    & & & \multicolumn{2}{p{0.3\linewidth}}{
        \cellcolor[rgb]{0.9,0.9,0.9}{
            \makecell[{{p{\linewidth}}}]{
                \texttt{\tiny{[GM$|$GM]}}
                \texttt{proposal} \\
            }
        }
    }
    & & \\ \\

    \theutterance \stepcounter{utterance}  
    & & & \multicolumn{2}{p{0.3\linewidth}}{
        \cellcolor[rgb]{0.9,0.9,0.9}{
            \makecell[{{p{\linewidth}}}]{
                \texttt{\tiny{[GM$|$GM]}}
                [1, 0, 0]
            }
        }
    }
    & & \\ \\

    \theutterance \stepcounter{utterance}  
    & & \multicolumn{4}{p{0.6\linewidth}}{
        \cellcolor[rgb]{0.9,0.9,0.9}{
            \makecell[{{p{\linewidth}}}]{
                \texttt{\tiny{[P1$\langle$GM]}}
                \texttt{Der andere Spieler hat einen geheimen Vorschlag gemacht. Jetzt bist du an der Reihe, einen eigenen Vorschlag zu machen.} \\
            }
        }
    }
    & \\ \\

    \theutterance \stepcounter{utterance}  
    & \multicolumn{4}{p{0.6\linewidth}}{
        \cellcolor[rgb]{0.9,0.9,0.9}{
            \makecell[{{p{\linewidth}}}]{
                \texttt{\tiny{[P1$\rangle$GM]}}
                \texttt{{[}Vorschlag: 1 Paprika, 3 Bilder{]}} \\
            }
        }
    }
    & & \\ \\

    \theutterance \stepcounter{utterance}  
    & & & \multicolumn{2}{p{0.3\linewidth}}{
        \cellcolor[rgb]{0.9,0.9,0.9}{
            \makecell[{{p{\linewidth}}}]{
                \texttt{\tiny{[GM$|$GM]}}
                \texttt{proposal} \\
            }
        }
    }
    & & \\ \\

    \theutterance \stepcounter{utterance}  
    & & & \multicolumn{2}{p{0.3\linewidth}}{
        \cellcolor[rgb]{0.9,0.9,0.9}{
            \makecell[{{p{\linewidth}}}]{
                \texttt{\tiny{[GM$|$GM]}}
                [0, 1, 3]
            }
        }
    }
    & & \\ \\

    \theutterance \stepcounter{utterance}  
    & & & \multicolumn{2}{p{0.3\linewidth}}{
        \cellcolor[rgb]{0.9,0.9,0.9}{
            \makecell[{{p{\linewidth}}}]{
                \texttt{\tiny{[GM$|$GM]}}
                [[0, 1, 3], [1, 0, 0]]
            }
        }
    }
    & & \\ \\

\end{supertabular}
}

\end{document}
