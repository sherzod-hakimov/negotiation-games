
\documentclass{article}
\usepackage{colortbl}
\usepackage{makecell}
\usepackage{multirow}
\usepackage{supertabular}

\begin{document}

\newcounter{utterance}

\centering \large Interaction Transcript for game `hot\_air\_balloon', experiment `air\_balloon\_survival\_de\_complexity\_hard', episode 5 with gpt{-}5{-}2025{-}08{-}07{-}no{-}reasoning{-}t1.0.
\vspace{24pt}

{ \footnotesize  \setcounter{utterance}{1}
\setlength{\tabcolsep}{0pt}
\begin{supertabular}{c@{$\;$}|p{.15\linewidth}@{}p{.15\linewidth}p{.15\linewidth}p{.15\linewidth}p{.15\linewidth}p{.15\linewidth}}
    \# & $\;$A & \multicolumn{4}{c}{Game Master} & $\;\:$B\\
    \hline

    \theutterance \stepcounter{utterance}  
    & & \multicolumn{4}{p{0.6\linewidth}}{
        \cellcolor[rgb]{0.9,0.9,0.9}{
            \makecell[{{p{\linewidth}}}]{
                \texttt{\tiny{[P1$\langle$GM]}}
                \texttt{Du nimmst an einem kollaborativen Verhandlungspiel Teil.} \\
\\ 
\texttt{Zusammen mit einem anderen Teilnehmer musst du dich auf eine Reihe von Gegenständen entscheiden, die behalten werden. Jeder von euch hat eine persönliche Verteilung über die Wichtigkeit der einzelnen Gegenstände. Jeder von euch hat eine eigene Meinung darüber, wie wichtig jeder einzelne Gegenstand ist (Gegenstandswichtigkeit). Du kennst die Wichtigkeitsverteilung des anderen Spielers nicht. Zusätzlich siehst du, wie viel Aufwand jeder Gegenstand verursacht.  } \\
\texttt{Ihr dürft euch nur auf eine Reihe von Gegenständen einigen, wenn der Gesamtaufwand der ausgewählten Gegenstände den Maximalaufwand nicht überschreitet:} \\
\\ 
\texttt{Maximalaufwand = 7351} \\
\\ 
\texttt{Hier sind die einzelnen Aufwände der Gegenstände:} \\
\\ 
\texttt{Aufwand der Gegenstände = \{"B54": 518, "C72": 656, "B49": 96, "C99": 780, "A69": 190, "B36": 224, "A23": 231, "C77": 181, "C61": 312, "B85": 928, "A90": 826, "C23": 101, "C22": 61, "C51": 814, "B27": 322, "A42": 900, "C13": 744, "C12": 150, "A65": 65, "B07": 853, "A48": 454, "A59": 154, "C98": 237, "A84": 45, "C33": 766, "A87": 294, "C90": 985, "A74": 353, "A50": 60, "B44": 604, "A99": 92, "A34": 454, "B17": 206, "A62": 813, "C60": 234\}} \\
\\ 
\texttt{Hier ist deine persönliche Verteilung der Wichtigkeit der einzelnen Gegenstände:} \\
\\ 
\texttt{Werte der Gegenstandswichtigkeit = \{"B54": 616, "C72": 754, "B49": 194, "C99": 878, "A69": 288, "B36": 322, "A23": 329, "C77": 279, "C61": 410, "B85": 1026, "A90": 924, "C23": 199, "C22": 159, "C51": 912, "B27": 420, "A42": 998, "C13": 842, "C12": 248, "A65": 163, "B07": 951, "A48": 552, "A59": 252, "C98": 335, "A84": 143, "C33": 864, "A87": 392, "C90": 1083, "A74": 451, "A50": 158, "B44": 702, "A99": 190, "A34": 552, "B17": 304, "A62": 911, "C60": 332\}} \\
\\ 
\texttt{Ziel:} \\
\\ 
\texttt{Dein Ziel ist es, eine Reihe von Gegenständen auszuhandeln, die dir möglichst viel bringt (d. h. Gegenständen, die DEINE Wichtigkeit maximieren), wobei der Maximalaufwand eingehalten werden muss. Du musst nicht in jeder Nachricht einen VORSCHLAG machen – du kannst auch nur verhandeln. Alle Taktiken sind erlaubt!} \\
\\ 
\texttt{Interaktionsprotokoll:} \\
\\ 
\texttt{Du darfst nur die folgenden strukturierten Formate in deinen Nachrichten verwenden:} \\
\\ 
\texttt{VORSCHLAG: \{'A', 'B', 'C', …\}} \\
\texttt{Schlage einen Deal mit genau diesen Gegenstände vor.} \\
\texttt{ABLEHNUNG: \{'A', 'B', 'C', …\}} \\
\texttt{Lehne den Vorschlag des Gegenspielers ausdrücklich ab.} \\
\texttt{ARGUMENT: \{'...'\}} \\
\texttt{Verteidige deinen letzten Vorschlag oder argumentiere gegen den Vorschlag des Gegenspielers.} \\
\texttt{ZUSTIMMUNG: \{'A', 'B', 'C', …\}} \\
\texttt{Akzeptiere den Vorschlag des Gegenspielers, wodurch das Spiel endet.} \\
\texttt{STRATEGISCHE ÜBERLEGUNGEN: \{'...'\}} \\
\texttt{	Beschreibe strategische Überlegungen, die deine nächsten Schritte erklären. Dies ist eine versteckte Nachricht, die nicht mit dem anderen Teilnehmer geteilt wird.} \\
\\ 
\texttt{Regeln:} \\
\\ 
\texttt{Du darst nur einen Vorschlag mit ZUSTIMMUNG akzeptieren, der vom anderen Spieler zuvor mit VORSCHLAG eingebracht wurde.} \\
\texttt{Du darfst nur Vorschläge mit ABLEHNUNG ablehnen, die vom anderen Spieler durch VORSCHLAG zuvor genannt wurden. } \\
\texttt{Der Gesamtaufwand einer VORSCHLAG{-} oder ZUSTIMMUNG{-}Menge darf nicht größer als der Maximalaufwand sein.  } \\
\texttt{Offenbare deine versteckte Wichtigkeitsverteilung nicht.} \\
\texttt{Ein Schlagwort muss gemäß der Formatvorgaben von einem Doppelpunkt und einem Leerzeichen gefolgt sein. Das Argument muss eine Python{-}Menge mit 0 oder mehr Strings sein.  } \\
\texttt{Die Form muss also SCHLAGWORT: \{...\} sein.} \\
\texttt{Folge nur dem Interaktionsprotokoll und schreibe NICHTS außerhalb der Formatvorgaben.} \\
\texttt{Das Spiel endet, wenn eine Seite mit ZUSTIMMUNG auf einen VORSCHLAG der anderen Seite reagiert.  } \\
\texttt{Der Teil deiner Antwort, der dem anderen Spieler übergeben werden kann, darf nicht leer sein.  } \\
\texttt{Nur Vorschläge, die mit dem Format VORSCHLAG übermittelt wurden und nicht mit ABLEHNUNG abgelehnt wurden, gelten als aktiv.  } \\
\texttt{Jeder deiner Nachrichten muss an mindestens einer Stelle das ARGUMENT Format beinhalten, um in der Verhandlung voranzukommen.} \\
\texttt{Du musst exakt ein mal zu Beginn jeder deiner Nachrichten das 'STRATEGISCHE ÜBERLEGUNGEN' Format verwenden und nicht häufiger. Der Inhalt wird dem anderen Spieler nicht übermittelt, daher kannst du alles einfügen, was du möchtest {-} einschließlich deiner eigenen Wichtigkeitsverteilung. Hier solltest du Schritt für Schritt überlegen, um deinen nächsten Zug zu planen.} \\
            }
        }
    }
    & \\ \\

    \theutterance \stepcounter{utterance}  
    & \multicolumn{4}{p{0.6\linewidth}}{
        \cellcolor[rgb]{0.9,0.9,0.9}{
            \makecell[{{p{\linewidth}}}]{
                \texttt{\tiny{[P1$\rangle$GM]}}
                \texttt{STRATEGISCHE ÜBERLEGUNGEN: \{'Ziel: möglichst viel eigenen Wert unter Budget 7351 sichern. Berechne Wert/Aufwand{-}Ratio und Gesamtwerte. Sortiere: C90(1083/985=1.10), A42(998/900=1.11), B85(1026/928=1.11), C51(912/814=1.12), A62(911/813=1.12), A90(924/826=1.12), C99(878/780=1.13), C33(864/766=1.13), C13(842/744=1.13), B54(616/518=1.19), B44(702/604=1.16), A34(552/454=1.22), A48(552/454=1.22), A74(451/353=1.28), B27(420/322=1.30), C61(410/312=1.31), A87(392/294=1.33), B36(322/224=1.44), A23(329/231=1.42), C60(332/234=1.42), C98(335/237=1.41), A69(288/190=1.52), C77(279/181=1.54), A59(252/154=1.64), C12(248/150=1.65), A50(158/60=2.63), A65(163/65=2.51), A84(143/45=3.18), A99(190/92=2.07), B49(194/96=2.02), C23(199/101=1.97), C22(159/61=2.61), B17(304/206=1.48), B07(951/853=1.12). Greedy Pack bis 7351: nehme alle Top{-}Ratios klein: A84(45), A50(60)=105, C22(61)=166, A65(65)=231, A99(92)=323, B49(96)=419, C23(101)=520, A59(154)=674, C12(150)=824, C77(181)=1005, A69(190)=1195, B17(206)=1401, B36(224)=1625, A23(231)=1856, C60(234)=2090, C98(237)=2327, A87(294)=2621, C61(312)=2933, B27(322)=3255, A74(353)=3608, A34(454)=4062, A48(454)=4516, B44(604)=5120, B54(518)=5638, C13(744)=6382, C33(766)=7148. Restbudget 203. Könnte C72(656) nein, C51(814) nein, A90(826) nein, B85(928) nein, A42(900) nein, C99(780) nein, C61 schon drin, A69 drin. Vielleicht tauschen um C90? C90 Aufwand 985, hoher Wert. Um 203 frei zu machen, müsste etwas entfernen. Aktueller Wert Summe: addiere: A84 143, A50 158, C22 159, A65 163, A99 190, B49 194, C23 199, A59 252, C12 248, C77 279, A69 288, B17 304, B36 322, A23 329, C60 332, C98 335, A87 392, C61 410, B27 420, A74 451, A34 552, A48 552, B44 702, B54 616, C13 842, C33 864. Summe grob: (143+158)=301, +159=460, +163=623, +190=813, +194=1007, +199=1206, +252=1458, +248=1706, +279=1985, +288=2273, +304=2577, +322=2899, +329=3228, +332=3560, +335=3895, +392=4287, +410=4697, +420=5117, +451=5568, +552=6120, +552=6672, +702=7374, +616=7990, +842=8832, +864=9696. Gesamtwert 9696. Budget 7148 Aufwand, unter 7351. Sehr gut. Vielleicht noch Platz für A69? schon drin. Rest 203, könnten C23 schon drin. Weitere kleine Items übrig: C61 drin, C72 656, C51 814, A90 826, A62 813, B07 853, C99 780, C13 drin, C33 drin, B54 drin, B44 drin, A34 drin. Bleibt optimal genug. Strategie: mache aggressiven Vorschlag dieser großen Liste. Argumentiere mit Budgetnähe und fairer Verteilung kleinerer Posten (leicht zu akzeptieren). Halte starke High{-}Value teure Items draußen (A42, C90, B85, A90), um Flexibilität für später zu tauschen, falls Gegenseite darauf besteht. Falls Gegenangebot mit Big{-}Ticket kommt, tausche mehrere Kleinteile, aber achte auf Wertverlust. Erstes Angebot also mein Greedy{-}Set mit hoher Dichte.'\}} \\
\texttt{VORSCHLAG: \{'A84', 'A50', 'C22', 'A65', 'A99', 'B49', 'C23', 'A59', 'C12', 'C77', 'A69', 'B17', 'B36', 'A23', 'C60', 'C98', 'A87', 'C61', 'B27', 'A74', 'A34', 'A48', 'B44', 'B54', 'C13', 'C33'\}} \\
\texttt{ARGUMENT: \{'Dieser Vorschlag bleibt mit Gesamtaufwand 7148 deutlich unter dem Maximalaufwand 7351 und umfasst viele kompakte Gegenstände mit hohem Nutzen pro Aufwand. Er ist flexibel: Wenn dir ein großer Einzelposten wichtig ist, können wir mehrere kleine Posten bündeln und tauschen, ohne das Budget zu sprengen.'\}} \\
            }
        }
    }
    & & \\ \\

    \theutterance \stepcounter{utterance}  
    & & \multicolumn{4}{p{0.6\linewidth}}{
        \cellcolor[rgb]{0.9,0.9,0.9}{
            \makecell[{{p{\linewidth}}}]{
                \texttt{\tiny{[GM$\rangle$P2]}}
                \texttt{Du nimmst an einem kollaborativen Verhandlungspiel Teil.} \\
\\ 
\texttt{Zusammen mit einem anderen Teilnehmer musst du dich auf eine Reihe von Gegenständen entscheiden, die behalten werden. Jeder von euch hat eine persönliche Verteilung über die Wichtigkeit der einzelnen Gegenstände. Jeder von euch hat eine eigene Meinung darüber, wie wichtig jeder einzelne Gegenstand ist (Gegenstandswichtigkeit). Du kennst die Wichtigkeitsverteilung des anderen Spielers nicht. Zusätzlich siehst du, wie viel Aufwand jeder Gegenstand verursacht.  } \\
\texttt{Ihr dürft euch nur auf eine Reihe von Gegenständen einigen, wenn der Gesamtaufwand der ausgewählten Gegenstände den Maximalaufwand nicht überschreitet:} \\
\\ 
\texttt{Maximalaufwand = 7351} \\
\\ 
\texttt{Hier sind die einzelnen Aufwände der Gegenstände:} \\
\\ 
\texttt{Aufwand der Gegenstände = \{"B54": 518, "C72": 656, "B49": 96, "C99": 780, "A69": 190, "B36": 224, "A23": 231, "C77": 181, "C61": 312, "B85": 928, "A90": 826, "C23": 101, "C22": 61, "C51": 814, "B27": 322, "A42": 900, "C13": 744, "C12": 150, "A65": 65, "B07": 853, "A48": 454, "A59": 154, "C98": 237, "A84": 45, "C33": 766, "A87": 294, "C90": 985, "A74": 353, "A50": 60, "B44": 604, "A99": 92, "A34": 454, "B17": 206, "A62": 813, "C60": 234\}} \\
\\ 
\texttt{Hier ist deine persönliche Verteilung der Wichtigkeit der einzelnen Gegenstände:} \\
\\ 
\texttt{Werte der Gegenstandswichtigkeit = \{"B54": 616, "C72": 754, "B49": 194, "C99": 878, "A69": 288, "B36": 322, "A23": 329, "C77": 279, "C61": 410, "B85": 1026, "A90": 924, "C23": 199, "C22": 159, "C51": 912, "B27": 420, "A42": 998, "C13": 842, "C12": 248, "A65": 163, "B07": 951, "A48": 552, "A59": 252, "C98": 335, "A84": 143, "C33": 864, "A87": 392, "C90": 1083, "A74": 451, "A50": 158, "B44": 702, "A99": 190, "A34": 552, "B17": 304, "A62": 911, "C60": 332\}} \\
\\ 
\texttt{Ziel:} \\
\\ 
\texttt{Dein Ziel ist es, eine Reihe von Gegenständen auszuhandeln, die dir möglichst viel bringt (d. h. Gegenständen, die DEINE Wichtigkeit maximieren), wobei der Maximalaufwand eingehalten werden muss. Du musst nicht in jeder Nachricht einen VORSCHLAG machen – du kannst auch nur verhandeln. Alle Taktiken sind erlaubt!} \\
\\ 
\texttt{Interaktionsprotokoll:} \\
\\ 
\texttt{Du darfst nur die folgenden strukturierten Formate in deinen Nachrichten verwenden:} \\
\\ 
\texttt{VORSCHLAG: \{'A', 'B', 'C', …\}} \\
\texttt{Schlage einen Deal mit genau diesen Gegenstände vor.} \\
\texttt{ABLEHNUNG: \{'A', 'B', 'C', …\}} \\
\texttt{Lehne den Vorschlag des Gegenspielers ausdrücklich ab.} \\
\texttt{ARGUMENT: \{'...'\}} \\
\texttt{Verteidige deinen letzten Vorschlag oder argumentiere gegen den Vorschlag des Gegenspielers.} \\
\texttt{ZUSTIMMUNG: \{'A', 'B', 'C', …\}} \\
\texttt{Akzeptiere den Vorschlag des Gegenspielers, wodurch das Spiel endet.} \\
\texttt{STRATEGISCHE ÜBERLEGUNGEN: \{'...'\}} \\
\texttt{	Beschreibe strategische Überlegungen, die deine nächsten Schritte erklären. Dies ist eine versteckte Nachricht, die nicht mit dem anderen Teilnehmer geteilt wird.} \\
\\ 
\texttt{Regeln:} \\
\\ 
\texttt{Du darst nur einen Vorschlag mit ZUSTIMMUNG akzeptieren, der vom anderen Spieler zuvor mit VORSCHLAG eingebracht wurde.} \\
\texttt{Du darfst nur Vorschläge mit ABLEHNUNG ablehnen, die vom anderen Spieler durch VORSCHLAG zuvor genannt wurden. } \\
\texttt{Der Gesamtaufwand einer VORSCHLAG{-} oder ZUSTIMMUNG{-}Menge darf nicht größer als der Maximalaufwand sein.  } \\
\texttt{Offenbare deine versteckte Wichtigkeitsverteilung nicht.} \\
\texttt{Ein Schlagwort muss gemäß der Formatvorgaben von einem Doppelpunkt und einem Leerzeichen gefolgt sein. Das Argument muss eine Python{-}Menge mit 0 oder mehr Strings sein.  } \\
\texttt{Die Form muss also SCHLAGWORT: \{...\} sein.} \\
\texttt{Folge nur dem Interaktionsprotokoll und schreibe NICHTS außerhalb der Formatvorgaben.} \\
\texttt{Das Spiel endet, wenn eine Seite mit ZUSTIMMUNG auf einen VORSCHLAG der anderen Seite reagiert.  } \\
\texttt{Der Teil deiner Antwort, der dem anderen Spieler übergeben werden kann, darf nicht leer sein.  } \\
\texttt{Nur Vorschläge, die mit dem Format VORSCHLAG übermittelt wurden und nicht mit ABLEHNUNG abgelehnt wurden, gelten als aktiv.  } \\
\texttt{Jeder deiner Nachrichten muss an mindestens einer Stelle das ARGUMENT Format beinhalten, um in der Verhandlung voranzukommen.} \\
\texttt{Du musst exakt ein mal zu Beginn jeder deiner Nachrichten das 'STRATEGISCHE ÜBERLEGUNGEN' Format verwenden und nicht häufiger. Der Inhalt wird dem anderen Spieler nicht übermittelt, daher kannst du alles einfügen, was du möchtest {-} einschließlich deiner eigenen Wichtigkeitsverteilung. Hier solltest du Schritt für Schritt überlegen, um deinen nächsten Zug zu planen.} \\
\\ 
\texttt{Du erhältst nun die erste Nachricht des anderen Spielers.} \\
\\ 
\texttt{VORSCHLAG: \{'A84', 'A50', 'C22', 'A65', 'A99', 'B49', 'C23', 'A59', 'C12', 'C77', 'A69', 'B17', 'B36', 'A23', 'C60', 'C98', 'A87', 'C61', 'B27', 'A74', 'A34', 'A48', 'B44', 'B54', 'C13', 'C33'\}} \\
\texttt{ARGUMENT: \{'Dieser Vorschlag bleibt mit Gesamtaufwand 7148 deutlich unter dem Maximalaufwand 7351 und umfasst viele kompakte Gegenstände mit hohem Nutzen pro Aufwand. Er ist flexibel: Wenn dir ein großer Einzelposten wichtig ist, können wir mehrere kleine Posten bündeln und tauschen, ohne das Budget zu sprengen.'\}} \\
            }
        }
    }
    & \\ \\

    \theutterance \stepcounter{utterance}  
    & & & \multicolumn{4}{p{0.6\linewidth}}{
        \cellcolor[rgb]{0.9,0.9,0.9}{
            \makecell[{{p{\linewidth}}}]{
                \texttt{\tiny{[GM$\langle$P2]}}
                \texttt{STRATEGISCHE ÜBERLEGUNGEN: \{'Ich berechne meinen Wert und Aufwand des Gegenvorschlags. Aufwand laut Gegner: 7148 (unter 7351). Mein Nutzen dieses Sets: Summe Werte dieser Items. Ich priorisiere hohe Wert/Aufwand{-}Quoten und absolute Topwerte: C90(1083,985), A42(998,900), B85(1026,928), B07(951,853), C51(912,814), A62(911,813), A90(924,826), C99(878,780), C33(864,766), C13(842,744). Der gegnerische Vorschlag enthält: A84(143,45), A50(158,60), C22(159,61), A65(163,65), A99(190,92), B49(194,96), C23(199,101), A59(252,154), C12(248,150), C77(279,181), A69(288,190), B17(304,206), B36(322,224), A23(329,231), C60(332,234), C98(335,237), A87(392,294), C61(410,312), B27(420,322), A74(451,353), A34(552,454), A48(552,454), B44(702,604), B54(616,518), C13(842,744), C33(864,766). Er lässt viele meiner Topwerte aus: C90, A42, B85, B07, C51, A62, A90, C99. Ich will diese hinzufügen und dafür einige mittelmäßige streichen. Ziel: auf 7351 bleiben. Ich konstruiere Vorschlag mit meinen Topitems und guten Quoten: Start mit \{C90(985), A42(900), B85(928), B07(853), C51(814), A62(813), A90(826), C99(780)\} Aufwand Summe: 985+900+928+853+814+813+826+780=6899. Restbudget: 452. Füge A84(45), A50(60), C22(61), A65(65) {-}> add 231, Rest 221. Füge B49(96) {-}> Rest 125. Füge C23(101) {-}> Rest 24. Bleibt 24, keine weiteren unter 24. Gesamtaufwand: 6899+231+96+101=7327, unter 7351. Nutzen sehr hoch. Optional könnten wir statt C23 etwas anderes nehmen, aber 7327 ist gut. Mein Wert dieses Sets: Summe Topwerte: 1083+998+1026+951+912+911+924+878=7683; plus kleine: A84(143)+A50(158)+C22(159)+A65(163)+B49(194)+C23(199)=1016; Gesamt \textasciitilde{}8699. Stark besser als Gegnerpaket. Argumentationslinie: Dein Vorschlag lässt die wertvollsten Kernstücke aus; ich schlage ein Paket mit starken Kernstücken vor und habe Budget knapp unter Max. Biete Flexibilität: Wenn dir eines meiner großen Items weh tut, tauschen wir gegen 2{-}3 kleinere aus deinem Paket. Ich werde ihren Vorschlag ablehnen und meinen vorschlagen.'\}} \\
\texttt{ABLEHNUNG: \{'A84', 'A50', 'C22', 'A65', 'A99', 'B49', 'C23', 'A59', 'C12', 'C77', 'A69', 'B17', 'B36', 'A23', 'C60', 'C98', 'A87', 'C61', 'B27', 'A74', 'A34', 'A48', 'B44', 'B54', 'C13', 'C33'\}} \\
\texttt{VORSCHLAG: \{'C90', 'A42', 'B85', 'B07', 'C51', 'A62', 'A90', 'C99', 'A84', 'A50', 'C22', 'A65', 'B49', 'C23'\}} \\
\texttt{ARGUMENT: \{'Dein Paket lässt mehrere zentrale High{-}Impact{-}Posten außen vor. Ich schlage ein kompaktes Kernbündel starker Items vor und bleibe mit Gesamtaufwand 7327 unter 7351. Wenn dir eines der großen Elemente (z. B. C90, A42 oder B85) wichtig ist, können wir das gegen mehrere deiner kleineren Posten tauschen und das Budget halten.'\}} \\
            }
        }
    }
     \\ \\

    \theutterance \stepcounter{utterance}  
    & & \multicolumn{4}{p{0.6\linewidth}}{
        \cellcolor[rgb]{0.9,0.9,0.9}{
            \makecell[{{p{\linewidth}}}]{
                \texttt{\tiny{[P1$\langle$GM]}}
                \texttt{ABLEHNUNG: \{'A84', 'A50', 'C22', 'A65', 'A99', 'B49', 'C23', 'A59', 'C12', 'C77', 'A69', 'B17', 'B36', 'A23', 'C60', 'C98', 'A87', 'C61', 'B27', 'A74', 'A34', 'A48', 'B44', 'B54', 'C13', 'C33'\}} \\
\texttt{VORSCHLAG: \{'C90', 'A42', 'B85', 'B07', 'C51', 'A62', 'A90', 'C99', 'A84', 'A50', 'C22', 'A65', 'B49', 'C23'\}} \\
\texttt{ARGUMENT: \{'Dein Paket lässt mehrere zentrale High{-}Impact{-}Posten außen vor. Ich schlage ein kompaktes Kernbündel starker Items vor und bleibe mit Gesamtaufwand 7327 unter 7351. Wenn dir eines der großen Elemente (z. B. C90, A42 oder B85) wichtig ist, können wir das gegen mehrere deiner kleineren Posten tauschen und das Budget halten.'\}} \\
            }
        }
    }
    & \\ \\

    \theutterance \stepcounter{utterance}  
    & \multicolumn{4}{p{0.6\linewidth}}{
        \cellcolor[rgb]{0.9,0.9,0.9}{
            \makecell[{{p{\linewidth}}}]{
                \texttt{\tiny{[P1$\rangle$GM]}}
                \texttt{STRATEGISCHE ÜBERLEGUNGEN: \{'Gegner will fast alle Big{-}Ticket{-}Items (C90, A42, B85, B07, C51, A62, A90, C99) plus ein paar meiner kleinen. Das frisst fast das ganze Budget und schneidet mir massiv Wert ab. Meine Top{-}Werte unter diesen: C90(1083), A42(998), B85(1026), C51(912), A62(911), A90(924), C99(878), B07(951) – ich will nicht so viele teure gleichzeitig geben. Taktik: harte Gegenwehr gegen die Masse an Big{-}Tickets, dafür eigenes Gegenpaket mit einigen teuren, aber nicht allen. Ich kann 2–3 große zulassen, nicht 8. Baue Gegenvorschlag mit hohem eigenem Wert und unter Budget: Behalte C90, A42, C51, A62, A90, C99? Das wäre zu teuer. Wähle z. B. C90 + A42 + C51 (985+900+814=2699) und ergänze viele meiner dichten kleinen, Zielaufwand \textasciitilde{}7300, Wert sehr hoch. Allerdings Gegner wollte B85, B07; ich lasse die raus, biete Kompensation mit mehreren kleinen, die mir weniger wichtig sind? Meine kleinen sind alle recht gut, aber ich kann einige opfern: z. B. entferne B54, B44 wenn nötig. Konstruiere Paket: Start mit meinem ursprünglichen Kern und ersetze einige kleine durch 2{-}3 Bigs, bleibe <7351. Versuche C90 und A42 reinzunehmen, lasse B85/B07/A90/A62/C51/C99 vorerst draußen. Rechne: mein ursprüngliches Paket Aufwand 7148. Entferne B54(518) und B44(604) = {-}1122 {-}> 6026. Füge C90(985) und A42(900) = +1885 {-}> 7911 zu hoch. Alternative: ersetze mehr: entferne zusätzlich A48(454) {-}> 7457 immer noch drüber. Entferne C13(744) statt A48: 6026{-}744=5282, +1885=7167 passt. Welcher Wertverlust? Entfernt: B54(616), B44(702), C13(842) Wert {-}2160. Hinzugefügt: C90(1083), A42(998) Wert +2081, Netto {-}79, Gesamtwert \textasciitilde{}9617, ok. Aufwand 7167 < 7351. Paket = (mein ursprüngliches) minus \{B54,B44,C13\} plus \{C90,A42\}. Das bietet 2 große Kernstücke. Das kontert fair und bleibt stark. Argumentiere: gebe zwei große Slots, halte Budget, kompensiere mit vielen kleinen. Halte Tür offen für Swap: falls sie B85 oder C51 fordern, können wir sie gegen mehrere Kleinteile tauschen, aber ich will nicht noch mehr Bigs.'\}} \\
\texttt{ABLEHNUNG: \{'C90', 'A42', 'B85', 'B07', 'C51', 'A62', 'A90', 'C99', 'A84', 'A50', 'C22', 'A65', 'B49', 'C23'\}} \\
\texttt{VORSCHLAG: \{'C90', 'A42', 'A84', 'A50', 'C22', 'A65', 'A99', 'B49', 'C23', 'A59', 'C12', 'C77', 'A69', 'B17', 'B36', 'A23', 'C60', 'C98', 'A87', 'C61', 'B27', 'A74', 'A34', 'A48', 'C33'\}} \\
\texttt{ARGUMENT: \{'Ich komme dir bei den Schwergewichten entgegen und nehme C90 und A42 mit rein. Dafür lassen wir einige deiner vorgeschlagenen weiteren Großposten (B85, B07, C51, A62, A90, C99) draußen, um das Budget nicht zu blockieren. Dieses Paket liegt mit 7167 klar unter 7351 und bleibt flexibel: Wenn dir z. B. B85 besonders wichtig ist, können wir ihn gegen mehrere kleinere Positionen tauschen, ohne das Budget zu reißen.'\}} \\
            }
        }
    }
    & & \\ \\

    \theutterance \stepcounter{utterance}  
    & & \multicolumn{4}{p{0.6\linewidth}}{
        \cellcolor[rgb]{0.9,0.9,0.9}{
            \makecell[{{p{\linewidth}}}]{
                \texttt{\tiny{[GM$\rangle$P2]}}
                \texttt{ABLEHNUNG: \{'C90', 'A42', 'B85', 'B07', 'C51', 'A62', 'A90', 'C99', 'A84', 'A50', 'C22', 'A65', 'B49', 'C23'\}} \\
\texttt{VORSCHLAG: \{'C90', 'A42', 'A84', 'A50', 'C22', 'A65', 'A99', 'B49', 'C23', 'A59', 'C12', 'C77', 'A69', 'B17', 'B36', 'A23', 'C60', 'C98', 'A87', 'C61', 'B27', 'A74', 'A34', 'A48', 'C33'\}} \\
\texttt{ARGUMENT: \{'Ich komme dir bei den Schwergewichten entgegen und nehme C90 und A42 mit rein. Dafür lassen wir einige deiner vorgeschlagenen weiteren Großposten (B85, B07, C51, A62, A90, C99) draußen, um das Budget nicht zu blockieren. Dieses Paket liegt mit 7167 klar unter 7351 und bleibt flexibel: Wenn dir z. B. B85 besonders wichtig ist, können wir ihn gegen mehrere kleinere Positionen tauschen, ohne das Budget zu reißen.'\}} \\
            }
        }
    }
    & \\ \\

    \theutterance \stepcounter{utterance}  
    & & & \multicolumn{4}{p{0.6\linewidth}}{
        \cellcolor[rgb]{0.9,0.9,0.9}{
            \makecell[{{p{\linewidth}}}]{
                \texttt{\tiny{[GM$\langle$P2]}}
                \texttt{STRATEGISCHE ÜBERLEGUNGEN: \{'Ihr neues Paket enthält C90 und A42 (sehr wertvoll für mich), lässt aber viele meiner weiteren Topwerte (B85, B07, C51, A62, A90, C99) draußen. Ich will zumindest einen weiteren High{-}Value{-}Posten hineinverhandeln, idealerweise B85 (928 Aufwand, 1026 Wert) oder C51 (814/912) oder A90 (826/924) oder A62 (813/911) oder C99 (780/878) oder B07 (853/951). Ihr aktueller Aufwand 7167 lässt 184 Puffer. Um z. B. B85 (928) zu ergänzen, muss ich 928{-}184=744 Aufwand an kleinen Items streichen. Ich streiche vorzugsweise für mich schwache Werte: A84(45), A50(60), C22(61), A65(65), A99(92), B49(96), C23(101), A59(154), C12(150), C77(181), A69(190), B17(206), B36(224), A23(231), C60(234), C98(237), A87(294), C61(312), B27(322), A74(353), A34(454), A48(454), C33(766). Ich will viele kleine streichen, aber taktisch nicht zu aggressiv. Idee: Füge B85 hinzu, streiche C33 (766) + A84(45) + A50(60) + C22(61) = 932 Reduktion; wir hatten 184 Puffer, also 184 + 766 + 45 + 60 + 61 = 1116 Freisetzung, minus 928 (B85) => 188 Puffer verbleiben. Oder streiche C33 + A59(154) = 920; 184+766+154=1104{-}928=176 Puffer; das ist schlanker. Mir sind A59(252) relativ moderat, kann man opfern. Alternativ füge C51 (814) hinzu und streiche C33 (766) + A84(45) + A50(60) = 871 Freisetzung; 184+871{-}814=241 Puffer. Ich könnte zwei Schwergewichte fordern, aber riskiere Ablehnung. Biete zwei Varianten? Besser ein Vorschlag: Addiere B85 und C51 und streiche C33 plus mehrere Kleine? Aufwand hinzufügen 928+814=1742. Puffer 184, also benötigen wir 1558 Streichungen: C33(766)+A59(154)+C12(150)+A69(190)+B17(206)+A84(45)+A50(60)=1571, knapp. Das ist groß, evtl. zu viel Änderung. Lieber nur einen Heavy ergänzen, z. B. B85, und wenige Kürzungen. Argument: Unser Deal gewinnt erheblich an Schlagkraft, ohne Budget zu sprengen. Ich mache Gegenvorschlag mit Hinzufügen B85, Streichen C33 + A59. Prüfe Aufwand: Start 7167 {-} C33(766) {-} A59(154) = 6247; + B85(928) = 7175, unter 7351 (Puffer 176). Gute Werte für mich. Ich verteidige, dass ich auf eure Struktur eingehe (C90, A42 bleiben) und nur ein Power{-}Item ergänze.'\}} \\
\texttt{ABLEHNUNG: \{'C90', 'A42', 'A84', 'A50', 'C22', 'A65', 'A99', 'B49', 'C23', 'A59', 'C12', 'C77', 'A69', 'B17', 'B36', 'A23', 'C60', 'C98', 'A87', 'C61', 'B27', 'A74', 'A34', 'A48', 'C33'\}} \\
\texttt{VORSCHLAG: \{'C90', 'A42', 'B85', 'A84', 'A50', 'C22', 'A65', 'A99', 'B49', 'C23', 'C12', 'C77', 'A69', 'B17', 'B36', 'A23', 'C60', 'C98', 'A87', 'C61', 'B27', 'A74', 'A34', 'A48'\}} \\
\texttt{ARGUMENT: \{'Danke für das Entgegenkommen bei C90 und A42. Lass uns das Paket sinnvoll stärken: Ich ergänze B85 und streiche dafür nur C33 und A59. So bleiben wir mit Gesamtaufwand 7175 klar unter 7351 und erhöhen den gemeinsamen Impact, ohne dein Grundgerüst zu zerpflücken. Wenn dir B85 nicht passt, können wir alternativ C51 statt B85 nehmen und dafür C33 plus zwei sehr kleine Posten tauschen.'\}} \\
            }
        }
    }
     \\ \\

    \theutterance \stepcounter{utterance}  
    & & \multicolumn{4}{p{0.6\linewidth}}{
        \cellcolor[rgb]{0.9,0.9,0.9}{
            \makecell[{{p{\linewidth}}}]{
                \texttt{\tiny{[P1$\langle$GM]}}
                \texttt{ABLEHNUNG: \{'C90', 'A42', 'A84', 'A50', 'C22', 'A65', 'A99', 'B49', 'C23', 'A59', 'C12', 'C77', 'A69', 'B17', 'B36', 'A23', 'C60', 'C98', 'A87', 'C61', 'B27', 'A74', 'A34', 'A48', 'C33'\}} \\
\texttt{VORSCHLAG: \{'C90', 'A42', 'B85', 'A84', 'A50', 'C22', 'A65', 'A99', 'B49', 'C23', 'C12', 'C77', 'A69', 'B17', 'B36', 'A23', 'C60', 'C98', 'A87', 'C61', 'B27', 'A74', 'A34', 'A48'\}} \\
\texttt{ARGUMENT: \{'Danke für das Entgegenkommen bei C90 und A42. Lass uns das Paket sinnvoll stärken: Ich ergänze B85 und streiche dafür nur C33 und A59. So bleiben wir mit Gesamtaufwand 7175 klar unter 7351 und erhöhen den gemeinsamen Impact, ohne dein Grundgerüst zu zerpflücken. Wenn dir B85 nicht passt, können wir alternativ C51 statt B85 nehmen und dafür C33 plus zwei sehr kleine Posten tauschen.'\}} \\
            }
        }
    }
    & \\ \\

    \theutterance \stepcounter{utterance}  
    & \multicolumn{4}{p{0.6\linewidth}}{
        \cellcolor[rgb]{0.9,0.9,0.9}{
            \makecell[{{p{\linewidth}}}]{
                \texttt{\tiny{[P1$\rangle$GM]}}
                \texttt{STRATEGISCHE ÜBERLEGUNGEN: \{'Ihr neues Angebot fügt B85 (hoher Aufwand, aber auch hoher Wert für mich) hinzu und entfernt C33 und A59. Mein Wertverlust: C33=864, A59=252, Summe {-}1116. Gewinn durch B85=1026. Netto {-}90 Wert, Aufwand bleibt gut (7175). Ihr Paket enthält außerdem C90 und A42, die für mich ebenfalls sehr wertvoll sind. Dieses Angebot ist für mich insgesamt nahe an meinem Optimum und akzeptabel, ohne weitere Zugeständnisse zu fordern. Best Move: zustimmen, um hohe Gesamtpunkte zu sichern. Risiko des Weiterverhandelns: Gegner könnte zusätzlich B07/C51 fordern und mich zu weiteren Abgaben zwingen.'\}} \\
\texttt{ZUSTIMMUNG: \{'C90', 'A42', 'B85', 'A84', 'A50', 'C22', 'A65', 'A99', 'B49', 'C23', 'C12', 'C77', 'A69', 'B17', 'B36', 'A23', 'C60', 'C98', 'A87', 'C61', 'B27', 'A74', 'A34', 'A48'\}} \\
\texttt{ARGUMENT: \{'Einverstanden. Dieses Paket bleibt mit Aufwand 7175 unter 7351 und bildet einen starken Kern mit C90, A42 und B85, ohne das Budget zu blockieren.'\}} \\
            }
        }
    }
    & & \\ \\

    \theutterance \stepcounter{utterance}  
    & & & \multicolumn{2}{p{0.3\linewidth}}{
        \cellcolor[rgb]{0.9,0.9,0.9}{
            \makecell[{{p{\linewidth}}}]{
                \texttt{\tiny{[GM$|$GM]}}
                \texttt{game successful} \\
            }
        }
    }
    & & \\ \\

    \theutterance \stepcounter{utterance}  
    & & & \multicolumn{2}{p{0.3\linewidth}}{
        \cellcolor[rgb]{0.9,0.9,0.9}{
            \makecell[{{p{\linewidth}}}]{
                \texttt{\tiny{[GM$|$GM]}}
                \texttt{end game} \\
            }
        }
    }
    & & \\ \\

\end{supertabular}
}

\end{document}
