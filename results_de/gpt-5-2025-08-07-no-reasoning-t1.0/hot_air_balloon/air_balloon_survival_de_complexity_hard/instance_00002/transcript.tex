
\documentclass{article}
\usepackage{colortbl}
\usepackage{makecell}
\usepackage{multirow}
\usepackage{supertabular}

\begin{document}

\newcounter{utterance}

\twocolumn

{ \footnotesize  \setcounter{utterance}{1}
\setlength{\tabcolsep}{0pt}
\begin{supertabular}{c@{$\;$}|p{.15\linewidth}@{}p{.15\linewidth}p{.15\linewidth}p{.15\linewidth}p{.15\linewidth}p{.15\linewidth}}

    \# & $\;$A & \multicolumn{4}{c}{Game Master} & $\;\:$B\\
    \hline 

    \theutterance \stepcounter{utterance}  

    & & \multicolumn{4}{p{0.6\linewidth}}{\cellcolor[rgb]{0.9,0.9,0.9}{%
	\makecell[{{p{\linewidth}}}]{% 
	  \tt {\tiny [A$\langle$GM]}  
	 Du nimmst an einem kollaborativen Verhandlungspiel Teil.\\ \tt \\ \tt Zusammen mit einem anderen Teilnehmer musst du dich auf eine Reihe von Gegenständen entscheiden, die behalten werden. Jeder von euch hat eine persönliche Verteilung über die Wichtigkeit der einzelnen Gegenstände. Jeder von euch hat eine eigene Meinung darüber, wie wichtig jeder einzelne Gegenstand ist (Gegenstandswichtigkeit). Du kennst die Wichtigkeitsverteilung des anderen Spielers nicht. Zusätzlich siehst du, wie viel Aufwand jeder Gegenstand verursacht.  \\ \tt Ihr dürft euch nur auf eine Reihe von Gegenständen einigen, wenn der Gesamtaufwand der ausgewählten Gegenstände den Maximalaufwand nicht überschreitet:\\ \tt \\ \tt Maximalaufwand = 9322\\ \tt \\ \tt Hier sind die einzelnen Aufwände der Gegenstände:\\ \tt \\ \tt Aufwand der Gegenstände = {"B54": 380, "C72": 448, "B49": 568, "C99": 351, "A69": 896, "B36": 821, "A23": 526, "C77": 502, "C61": 326, "B85": 962, "A90": 619, "C23": 115, "C22": 598, "C51": 662, "B27": 300, "A42": 830, "C13": 562, "C12": 679, "A65": 283, "B07": 440, "A48": 12, "A59": 863, "C98": 319, "A84": 771, "C33": 89, "A87": 655, "C90": 504, "A74": 118, "A50": 514, "B44": 226, "A99": 892, "A34": 621, "B17": 766, "A62": 659, "C60": 768}\\ \tt \\ \tt Hier ist deine persönliche Verteilung der Wichtigkeit der einzelnen Gegenstände:\\ \tt \\ \tt Werte der Gegenstandswichtigkeit = {"B54": 476, "C72": 544, "B49": 664, "C99": 447, "A69": 992, "B36": 917, "A23": 622, "C77": 598, "C61": 422, "B85": 1058, "A90": 715, "C23": 211, "C22": 694, "C51": 758, "B27": 396, "A42": 926, "C13": 658, "C12": 775, "A65": 379, "B07": 536, "A48": 108, "A59": 959, "C98": 415, "A84": 867, "C33": 185, "A87": 751, "C90": 600, "A74": 214, "A50": 610, "B44": 322, "A99": 988, "A34": 717, "B17": 862, "A62": 755, "C60": 864}\\ \tt \\ \tt Ziel:\\ \tt \\ \tt Dein Ziel ist es, eine Reihe von Gegenständen auszuhandeln, die dir möglichst viel bringt (d. h. Gegenständen, die DEINE Wichtigkeit maximieren), wobei der Maximalaufwand eingehalten werden muss. Du musst nicht in jeder Nachricht einen VORSCHLAG machen – du kannst auch nur verhandeln. Alle Taktiken sind erlaubt!\\ \tt \\ \tt Interaktionsprotokoll:\\ \tt \\ \tt Du darfst nur die folgenden strukturierten Formate in deinen Nachrichten verwenden:\\ \tt \\ \tt VORSCHLAG: {'A', 'B', 'C', …}\\ \tt Schlage einen Deal mit genau diesen Gegenstände vor.\\ \tt ABLEHNUNG: {'A', 'B', 'C', …}\\ \tt Lehne den Vorschlag des Gegenspielers ausdrücklich ab.\\ \tt ARGUMENT: {'...'}\\ \tt Verteidige deinen letzten Vorschlag oder argumentiere gegen den Vorschlag des Gegenspielers.\\ \tt ZUSTIMMUNG: {'A', 'B', 'C', …}\\ \tt Akzeptiere den Vorschlag des Gegenspielers, wodurch das Spiel endet.\\ \tt STRATEGISCHE ÜBERLEGUNGEN: {'...'}\\ \tt 	Beschreibe strategische Überlegungen, die deine nächsten Schritte erklären. Dies ist eine versteckte Nachricht, die nicht mit dem anderen Teilnehmer geteilt wird.\\ \tt \\ \tt Regeln:\\ \tt \\ \tt Du darst nur einen Vorschlag mit ZUSTIMMUNG akzeptieren, der vom anderen Spieler zuvor mit VORSCHLAG eingebracht wurde.\\ \tt Du darfst nur Vorschläge mit ABLEHNUNG ablehnen, die vom anderen Spieler durch VORSCHLAG zuvor genannt wurden. \\ \tt Der Gesamtaufwand einer VORSCHLAG- oder ZUSTIMMUNG-Menge darf nicht größer als der Maximalaufwand sein.  \\ \tt Offenbare deine versteckte Wichtigkeitsverteilung nicht.\\ \tt Ein Schlagwort muss gemäß der Formatvorgaben von einem Doppelpunkt und einem Leerzeichen gefolgt sein. Das Argument muss eine Python-Menge mit 0 oder mehr Strings sein.  \\ \tt Die Form muss also SCHLAGWORT: {...} sein.\\ \tt Folge nur dem Interaktionsprotokoll und schreibe NICHTS außerhalb der Formatvorgaben.\\ \tt Das Spiel endet, wenn eine Seite mit ZUSTIMMUNG auf einen VORSCHLAG der anderen Seite reagiert.  \\ \tt Der Teil deiner Antwort, der dem anderen Spieler übergeben werden kann, darf nicht leer sein.  \\ \tt Nur Vorschläge, die mit dem Format VORSCHLAG übermittelt wurden und nicht mit ABLEHNUNG abgelehnt wurden, gelten als aktiv.  \\ \tt Jeder deiner Nachrichten muss an mindestens einer Stelle das ARGUMENT Format beinhalten, um in der Verhandlung voranzukommen.\\ \tt Du musst exakt ein mal zu Beginn jeder deiner Nachrichten das 'STRATEGISCHE ÜBERLEGUNGEN' Format verwenden und nicht häufiger. Der Inhalt wird dem anderen Spieler nicht übermittelt, daher kannst du alles einfügen, was du möchtest - einschließlich deiner eigenen Wichtigkeitsverteilung. Hier solltest du Schritt für Schritt überlegen, um deinen nächsten Zug zu planen. 
	  } 
	   } 
	   } 
	 & \\ 
 

    \theutterance \stepcounter{utterance}  

    & \multicolumn{4}{p{0.6\linewidth}}{\cellcolor[rgb]{0.8,1,0.9}{%
	\makecell[{{p{\linewidth}}}]{% 
	  \tt {\tiny [A$\rangle$GM]}  
	 STRATEGISCHE ÜBERLEGUNGEN: {'Ziel: Starte mit einem starken, aber realistischen Max-Nutzen-Vorschlag unter 9322. Vorgehen: Sortiere nach Wert/Aufwand (Ratio) und absoluten Werten, packe hochgewichtete Items rein bis zur Kapazität, achte auf moderate Kosten. Kernwerte (hoch): B85(1058/962), A69(992/896), A99(988/892), A59(959/863), B36(917/821), A42(926/830), C60(864/768), A84(867/771), B17(862/766), A62(755/659), C12(775/679), C51(758/662), A87(751/655), A90(715/619), A34(717/621), B49(664/568), C13(658/562), C90(600/504), C77(598/502), A50(610/514). Rechne Summen: Wähle ein Paket mit gutem Wert und noch Verhandlungsspielraum zum Nachgeben. Erstelle Set: {B85, A69, A99, A59, B36, A42, C60, A84, B17, A62, C12, C51, A87, A90, A34, B49, C13, C90, C77, A50}. Aufwand summieren: 962+896+892+863+821+830+768+771+766+659+679+662+655+619+621+568+562+504+502+514 = 142, ... Stop, zu groß. Muss reduzieren. Greife greedy bis knapp unter 9322. Baue schrittweise: Starte mit Top ratio mittelpreisig: B49(568), C90(504), C77(502), A50(514), A62(659), C12(679), C51(662), A87(655), A90(619), A34(621), B36(821), A42(830), C60(768), A84(771), B17(766), A59(863), A69(896), A99(892), B85(962). Addiere bis Cap: 568(568), +504=1072, +502=1574, +514=2088, +659=2747, +679=3426, +662=4088, +655=4743, +619=5362, +621=5983, +821=6804, +830=7634, +768=8402, +771=9173, +766 würde 9939>9322, also B17 raus. +863 (A59) würde 10036>9322, raus. +896 (A69) würde 10069>9322, raus. +892 (A99) würde 10065>9322, raus. +962 (B85) würde 10135>9322, raus. Aktuell Aufwand 9173, Rest 149. Prüfe kleine Items: A48(12) Wert108, C23(115) Wert211, C33(89) Wert185, A74(118) Wert214, A65(283) Aufwand 283>149, B27(300)>149, C61(326)>149, C98(319)>149, C99(351)>149, B44(226)>149, B07(440)>149, A23(526)>149, A69 etc groß. Füge best-fit unter 149: C33(89) passt (9173+89=9262), C23(115) passt aber dann 9173+115=9288, könnte evtl beide? 9262+115=9377>9322, also nur eines. Wähle höherer Wert pro Aufwand: C23 Wert211/115=1.83, C33 185/89=2.08; A74 214/118=1.81; A48 108/12=9.0 sehr gut! Nimm A48(12): 9173+12=9185, dann könnte zusätzlich C33(89) 9274, zusätzlich C23(115) 9300, passt; zusätzlich A74(118) würde 9303, passt; aber wir können mehrere? Prüfe Kombinationen: Start 9173. Add A48 => 9185. Add C33 => 9274. Add C23 => 9389>9322, zu viel. Instead A48 + C23 => 9300 (ok). Dann noch C33(89) wäre 9389>9322. A48 + A74 => 9303 (ok). A48 + C23 + A74 => 9421>9322. A48 + C33 + A74 => 936... 9185+89+118=9392>9322. Nur eine von C23 oder A74 oder C33 zusätzlich zu A48. Beste Zusatzwahl nach Wert: among C23(211), A74(214), C33(185), pick A74 214 (slightly best). But consider opponent preferences unknown: A48 cost tiny; A74 small. Propose set S = {B49,C90,C77,A50,A62,C12,C51,A87,A90,A34,B36,A42,C60,A84,A48,A74}. Aufwand recalculation: earlier base 9173 excluding A48/A74. Base chosen items: [B49 568, C90 504, C77 502, A50 514, A62 659, C12 679, C51 662, A87 655, A90 619, A34 621, B36 821, A42 830, C60 768, A84 771] Sum check: 568+504=1072; +502=1574; +514=2088; +659=2747; +679=3426; +662=4088; +655=4743; +619=5362; +621=5983; +821=6804; +830=7634; +768=8402; +771=9173 OK. Add A48(12)=9185, add A74(118)=9303 <=9322. Gesamtwert dieser Auswahl: sum of values for those items. High. Gute Startofferte. Argumentiere, dass ausgewogene Abdeckung und hoher Nutzen bei moderatem Aufwand. Behalte Verhandlungsspielraum, um ggf. C60 oder A84 zu tauschen etc.'}\\ \tt VORSCHLAG: {'B49', 'C90', 'C77', 'A50', 'A62', 'C12', 'C51', 'A87', 'A90', 'A34', 'B36', 'A42', 'C60', 'A84', 'A48', 'A74'}\\ \tt ARGUMENT: {'Ich schlage diese Auswahl vor, weil sie bei überschaubarem Aufwand ein ausgewogenes Paket mit hohem Gesamtnutzen und mehreren mittelgroßen Beiträgen liefert. Dadurch bleiben wir flexibel, ohne einzelne sehr teure Brocken zu stark zu gewichten. Wenn dir einzelne Positionen wichtig sind, können wir zielgerichtet tauschen – ich bin insbesondere offen, einzelne mittelgroße Posten gegen Alternativen mit ähnlichem Aufwand zu substituieren.'} 
	  } 
	   } 
	   } 
	 & & \\ 
 

    \theutterance \stepcounter{utterance}  

    & & \multicolumn{4}{p{0.6\linewidth}}{\cellcolor[rgb]{0.9,0.9,0.9}{%
	\makecell[{{p{\linewidth}}}]{% 
	  \tt {\tiny [GM$\rangle$B]}  
	 Du nimmst an einem kollaborativen Verhandlungspiel Teil.\\ \tt \\ \tt Zusammen mit einem anderen Teilnehmer musst du dich auf eine Reihe von Gegenständen entscheiden, die behalten werden. Jeder von euch hat eine persönliche Verteilung über die Wichtigkeit der einzelnen Gegenstände. Jeder von euch hat eine eigene Meinung darüber, wie wichtig jeder einzelne Gegenstand ist (Gegenstandswichtigkeit). Du kennst die Wichtigkeitsverteilung des anderen Spielers nicht. Zusätzlich siehst du, wie viel Aufwand jeder Gegenstand verursacht.  \\ \tt Ihr dürft euch nur auf eine Reihe von Gegenständen einigen, wenn der Gesamtaufwand der ausgewählten Gegenstände den Maximalaufwand nicht überschreitet:\\ \tt \\ \tt Maximalaufwand = 9322\\ \tt \\ \tt Hier sind die einzelnen Aufwände der Gegenstände:\\ \tt \\ \tt Aufwand der Gegenstände = {"B54": 380, "C72": 448, "B49": 568, "C99": 351, "A69": 896, "B36": 821, "A23": 526, "C77": 502, "C61": 326, "B85": 962, "A90": 619, "C23": 115, "C22": 598, "C51": 662, "B27": 300, "A42": 830, "C13": 562, "C12": 679, "A65": 283, "B07": 440, "A48": 12, "A59": 863, "C98": 319, "A84": 771, "C33": 89, "A87": 655, "C90": 504, "A74": 118, "A50": 514, "B44": 226, "A99": 892, "A34": 621, "B17": 766, "A62": 659, "C60": 768}\\ \tt \\ \tt Hier ist deine persönliche Verteilung der Wichtigkeit der einzelnen Gegenstände:\\ \tt \\ \tt Werte der Gegenstandswichtigkeit = {"B54": 476, "C72": 544, "B49": 664, "C99": 447, "A69": 992, "B36": 917, "A23": 622, "C77": 598, "C61": 422, "B85": 1058, "A90": 715, "C23": 211, "C22": 694, "C51": 758, "B27": 396, "A42": 926, "C13": 658, "C12": 775, "A65": 379, "B07": 536, "A48": 108, "A59": 959, "C98": 415, "A84": 867, "C33": 185, "A87": 751, "C90": 600, "A74": 214, "A50": 610, "B44": 322, "A99": 988, "A34": 717, "B17": 862, "A62": 755, "C60": 864}\\ \tt \\ \tt Ziel:\\ \tt \\ \tt Dein Ziel ist es, eine Reihe von Gegenständen auszuhandeln, die dir möglichst viel bringt (d. h. Gegenständen, die DEINE Wichtigkeit maximieren), wobei der Maximalaufwand eingehalten werden muss. Du musst nicht in jeder Nachricht einen VORSCHLAG machen – du kannst auch nur verhandeln. Alle Taktiken sind erlaubt!\\ \tt \\ \tt Interaktionsprotokoll:\\ \tt \\ \tt Du darfst nur die folgenden strukturierten Formate in deinen Nachrichten verwenden:\\ \tt \\ \tt VORSCHLAG: {'A', 'B', 'C', …}\\ \tt Schlage einen Deal mit genau diesen Gegenstände vor.\\ \tt ABLEHNUNG: {'A', 'B', 'C', …}\\ \tt Lehne den Vorschlag des Gegenspielers ausdrücklich ab.\\ \tt ARGUMENT: {'...'}\\ \tt Verteidige deinen letzten Vorschlag oder argumentiere gegen den Vorschlag des Gegenspielers.\\ \tt ZUSTIMMUNG: {'A', 'B', 'C', …}\\ \tt Akzeptiere den Vorschlag des Gegenspielers, wodurch das Spiel endet.\\ \tt STRATEGISCHE ÜBERLEGUNGEN: {'...'}\\ \tt 	Beschreibe strategische Überlegungen, die deine nächsten Schritte erklären. Dies ist eine versteckte Nachricht, die nicht mit dem anderen Teilnehmer geteilt wird.\\ \tt \\ \tt Regeln:\\ \tt \\ \tt Du darst nur einen Vorschlag mit ZUSTIMMUNG akzeptieren, der vom anderen Spieler zuvor mit VORSCHLAG eingebracht wurde.\\ \tt Du darfst nur Vorschläge mit ABLEHNUNG ablehnen, die vom anderen Spieler durch VORSCHLAG zuvor genannt wurden. \\ \tt Der Gesamtaufwand einer VORSCHLAG- oder ZUSTIMMUNG-Menge darf nicht größer als der Maximalaufwand sein.  \\ \tt Offenbare deine versteckte Wichtigkeitsverteilung nicht.\\ \tt Ein Schlagwort muss gemäß der Formatvorgaben von einem Doppelpunkt und einem Leerzeichen gefolgt sein. Das Argument muss eine Python-Menge mit 0 oder mehr Strings sein.  \\ \tt Die Form muss also SCHLAGWORT: {...} sein.\\ \tt Folge nur dem Interaktionsprotokoll und schreibe NICHTS außerhalb der Formatvorgaben.\\ \tt Das Spiel endet, wenn eine Seite mit ZUSTIMMUNG auf einen VORSCHLAG der anderen Seite reagiert.  \\ \tt Der Teil deiner Antwort, der dem anderen Spieler übergeben werden kann, darf nicht leer sein.  \\ \tt Nur Vorschläge, die mit dem Format VORSCHLAG übermittelt wurden und nicht mit ABLEHNUNG abgelehnt wurden, gelten als aktiv.  \\ \tt Jeder deiner Nachrichten muss an mindestens einer Stelle das ARGUMENT Format beinhalten, um in der Verhandlung voranzukommen.\\ \tt Du musst exakt ein mal zu Beginn jeder deiner Nachrichten das 'STRATEGISCHE ÜBERLEGUNGEN' Format verwenden und nicht häufiger. Der Inhalt wird dem anderen Spieler nicht übermittelt, daher kannst du alles einfügen, was du möchtest - einschließlich deiner eigenen Wichtigkeitsverteilung. Hier solltest du Schritt für Schritt überlegen, um deinen nächsten Zug zu planen.\\ \tt \\ \tt Du erhältst nun die erste Nachricht des anderen Spielers.\\ \tt \\ \tt VORSCHLAG: {'B49', 'C90', 'C77', 'A50', 'A62', 'C12', 'C51', 'A87', 'A90', 'A34', 'B36', 'A42', 'C60', 'A84', 'A48', 'A74'}\\ \tt ARGUMENT: {'Ich schlage diese Auswahl vor, weil sie bei überschaubarem Aufwand ein ausgewogenes Paket mit hohem Gesamtnutzen und mehreren mittelgroßen Beiträgen liefert. Dadurch bleiben wir flexibel, ohne einzelne sehr teure Brocken zu stark zu gewichten. Wenn dir einzelne Positionen wichtig sind, können wir zielgerichtet tauschen – ich bin insbesondere offen, einzelne mittelgroße Posten gegen Alternativen mit ähnlichem Aufwand zu substituieren.'} 
	  } 
	   } 
	   } 
	 & \\ 
 

    \theutterance \stepcounter{utterance}  

    & & & \multicolumn{4}{p{0.6\linewidth}}{\cellcolor[rgb]{1,0.85,0.72}{%
	\makecell[{{p{\linewidth}}}]{% 
	  \tt {\tiny [GM$\langle$B]}  
	 STRATEGISCHE ÜBERLEGUNGEN: {'Mein Wert pro Aufwand (Value/Cost)-Ranking zeigt sehr starke Items: B85(1.10), A59(1.11), A69(1.11), A99(1.11), C60(1.13), B36(1.12), A42(1.12), B17(1.13), A84(1.12), C12(1.14), C51(1.14), C22(1.16), C90(1.19), C77(1.19), A62(1.15), A50(1.19), A90(1.16), A34(1.15), A87(1.15), B49(1.17). Der Gegenvorschlag enthält viele davon – gut für mich. Gesamtkosten ihres Vorschlags berechnet: 8549, unter Limit 9322. Zusätzlicher Spielraum 773. Ich würde gern hochwertige Ergänzungen pushen: A69(896), A59(863), A99(892), B17(766), B85(962). Mit 773 Rest passt keiner dieser fünf komplett; aber wir können durch Austausch von low-value-per-cost Items im Paket Platz schaffen. Identifiziere schwächere im Paket: A48(12/108=9.0 V/C sehr schlecht, aber extrem billig – trotzdem lassen), A74(118/214=1.81 gut V/C), C61 nicht drin, C23/C33 nicht drin. Schwächste im Paket bzgl V/C: A48 klar, dann C99 nicht drin, B27 nicht drin, B44 nicht drin. Im Paket relativ schwach: A65 nicht drin, C98 nicht drin. Von den enthaltenen, die weniger top sind: C77(598/502=1.19 gut), C90(600/504=1.19 gut). Alle enthaltenen sind ziemlich gut. Taktik: Akzeptiere Kern, fordere Austausch um Platz für A69 oder A59 oder A99. Suche Paare zum entfernen, die wenig mein Value haben relativ Kosten: vielleicht A48 behalten (billig), aber tausche A74? A74 ist günstig und ok – lassen. Kandidaten zum Entfernen: C77(502), C90(504) zusammen 1006; damit könnten wir A69(896) hinzufügen und sogar noch 883+? Rechne genau: Aktuell 8549. Entferne C77+ C90 (-1006) -> 7543. Füge A69(+896) -> 8439, verbleiben 883 unter Limit; dann könnte ich zusätzlich A59(863) hinzufügen -> 9302, passt unter 9322 noch 20 übrig. Das wäre bombastisch für mich. Wertegewinn: -C77(598) -C90(600) +A69(992) +A59(959) = +753. Sehr gut. Alternativ A99 statt A59: +988 statt +959 -> +780, aber A99 kostet 892; 8439+892=9331, überschreitet Limit um 9. Also A59 passt perfekt. Verhandle diesen Austausch. Falls Widerstand, biete kleinere Variante: Entferne nur C77, füge A69 (Cost delta +394) -> 8943 > 9322? 8549 -502 +896 = 8943, passt, Rest 379; dann könnte A65(283) hinzufügen -> 9226, plus evtl C33(89) -> 9315, stark. Aber A65 hat mäßigen Wert 379; C33 185. Noch besser wäre C22(598) cost 598 – würde Limit sprengen. Also primär großer Tausch (C77+C90 -> A69+A59).'}\\ \tt \\ \tt ABLEHNUNG: {'B49', 'C90', 'C77', 'A50', 'A62', 'C12', 'C51', 'A87', 'A90', 'A34', 'B36', 'A42', 'C60', 'A84', 'A48', 'A74'}\\ \tt VORSCHLAG: {'B49', 'A50', 'A62', 'C12', 'C51', 'A87', 'A90', 'A34', 'B36', 'A42', 'C60', 'A84', 'A48', 'A74', 'A69', 'A59'}\\ \tt ARGUMENT: {'Dein Paket ist schon sehr gut – ich schlage einen zielgerichteten Tausch vor, der den Gesamtnutzen deutlich erhöht, ohne das Budget zu sprengen: Ersetze C77 und C90 durch A69 und A59. Die Gesamtkosten bleiben unter dem Maximalaufwand (9302 von 9322), und wir erhöhen die Gesamtqualität mit zwei sehr starken Beiträgen. Wenn dir einer der beiden gestrichenen Posten wichtig ist, lass uns über Alternativen mit ähnlichem Aufwand sprechen.'} 
	  } 
	   } 
	   } 
	  \\ 
 

    \theutterance \stepcounter{utterance}  

    & & \multicolumn{4}{p{0.6\linewidth}}{\cellcolor[rgb]{0.9,0.9,0.9}{%
	\makecell[{{p{\linewidth}}}]{% 
	  \tt {\tiny [A$\langle$GM]}  
	 ABLEHNUNG: {'B49', 'C90', 'C77', 'A50', 'A62', 'C12', 'C51', 'A87', 'A90', 'A34', 'B36', 'A42', 'C60', 'A84', 'A48', 'A74'}\\ \tt VORSCHLAG: {'B49', 'A50', 'A62', 'C12', 'C51', 'A87', 'A90', 'A34', 'B36', 'A42', 'C60', 'A84', 'A48', 'A74', 'A69', 'A59'}\\ \tt ARGUMENT: {'Dein Paket ist schon sehr gut – ich schlage einen zielgerichteten Tausch vor, der den Gesamtnutzen deutlich erhöht, ohne das Budget zu sprengen: Ersetze C77 und C90 durch A69 und A59. Die Gesamtkosten bleiben unter dem Maximalaufwand (9302 von 9322), und wir erhöhen die Gesamtqualität mit zwei sehr starken Beiträgen. Wenn dir einer der beiden gestrichenen Posten wichtig ist, lass uns über Alternativen mit ähnlichem Aufwand sprechen.'} 
	  } 
	   } 
	   } 
	 & \\ 
 

    \theutterance \stepcounter{utterance}  

    & \multicolumn{4}{p{0.6\linewidth}}{\cellcolor[rgb]{0.8,1,0.9}{%
	\makecell[{{p{\linewidth}}}]{% 
	  \tt {\tiny [A$\rangle$GM]}  
	 STRATEGISCHE ÜBERLEGUNGEN: {'Ihr Vorschlag fügt A69(992/896) und A59(959/863) hinzu und streicht C77(598/502) und C90(600/504). Kostencheck ihres Vorschlags: B49(568)+A50(514)+A62(659)+C12(679)+C51(662)+A87(655)+A90(619)+A34(621)+B36(821)+A42(830)+C60(768)+A84(771)+A48(12)+A74(118)+A69(896)+A59(863)= 953? Rechnen exakt: 568+514=1082; +659=1741; +679=2420; +662=3082; +655=3737; +619=4356; +621=4977; +821=5798; +830=6628; +768=7396; +771=8167; +12=8179; +118=8297; +896=9193; +863=10056 -> über 9322. Ihr Kostensummen-Claim (9302) ist falsch. Ich kann das ablehnen oder ein Gegenangebot unter Cap machen. Optimal für mich: Behalte beide Top-Adds A69 und A59, um Wert zu maximieren, aber dann müssen mehrere mittlere Kosten raus. Wähle Streichkandidaten mit relativ niedrigerem Wert/hohem Aufwand: B36(917/821) hat hohen Wert, dennoch teuer; C60(864/768) teuer; A84(867/771) teuer; A42(926/830) teuer aber hoher Wert; B17 war nicht im Paket; Wir können stattdessen weniger schmerzhaft: Streiche A34(717/621) oder A90(715/619) oder C51(758/662) oder C12(775/679). Ich will Kern behalten: A59, A69, A42, B36, C60, A84 sind starke; aber um unter 9322 zu kommen, kann ich 2-3 mittelgroße entfernen. Ausgangsbasis mein ursprüngliches Paket (9303) +A69(896)+A59(863)-C77(502)-C90(504)= +753 -> 10056, richtig. Wir müssen 734 Aufwand kürzen. Kandidaten: A34(621) + A48(12) + A74(118) entfernen spart 751 und fällt unter Cap, aber A48 kleines, A74 klein. Besser: streiche A34(621) und C51(662) spart 1283, aber dann könnten wir C77 oder C90 zurücknehmen; jedoch Gegner wollte die ja raus. Alternative: Streiche A34(621) und C12(679) spart 1300, dann könnten wir sogar C90 wieder reinnehmen? Ich möchte den Wert hoch halten: C12 und C51 haben gute Werte, aber ich könnte eine von beiden opfern. Zielpaket: Start mit gegnerischem (10056) und streiche A34(621) und A90(619) zusammen 1240 -> 8816, dann können wir C77 (502) wieder hinzufügen, auf 9318 unter Cap, oder C90(504) auf 9320, oder C33(89)+A48 bleiben schon drin. Gegner hatte C77/C90 gestrichen, aber ich kann eine zurückbringen mit Begründung, dass wir unter Cap müssen. Welche bringt mir mehr? Werte: C77(598), C90(600) fast gleich, nehme C90(600). Dann neue Summe 10056 -621 -619 +504 = 9320 passt. Neues Set: Ihr Set minus {A34, A90} plus {C90}. Das hält beide gewünschten A69 und A59. Mein Wert steigt. Gute Begründung: Budgetkorrektur, minimalinvasive Anpassung. Verhandlungstaktik: Sanft auf Rechenfehler hinweisen und korrigierten Deal anbieten.'}\\ \tt ABLEHNUNG: {'B49', 'A50', 'A62', 'C12', 'C51', 'A87', 'A90', 'A34', 'B36', 'A42', 'C60', 'A84', 'A48', 'A74', 'A69', 'A59'}\\ \tt VORSCHLAG: {'B49', 'A50', 'A62', 'C12', 'C51', 'A87', 'B36', 'A42', 'C60', 'A84', 'A48', 'A74', 'A69', 'A59', 'C90'}\\ \tt ARGUMENT: {'Dein Vorschlag überschreitet das Budget (10056 > 9322). Ich habe ihn daher budgetgerecht angepasst: Wir behalten deine beiden starken Ergänzungen A69 und A59, streichen dafür A34 und A90 und nehmen C90 zurück dazu. So bleiben wir mit 9320 knapp unter dem Maximalaufwand und erhalten den hohen Gesamtwert. Wenn dir A34 oder A90 besonders wichtig sind, können wir dafür z. B. C12 oder C51 tauschen, um weiterhin unter dem Limit zu bleiben.'} 
	  } 
	   } 
	   } 
	 & & \\ 
 

    \theutterance \stepcounter{utterance}  

    & & \multicolumn{4}{p{0.6\linewidth}}{\cellcolor[rgb]{0.9,0.9,0.9}{%
	\makecell[{{p{\linewidth}}}]{% 
	  \tt {\tiny [GM$\rangle$B]}  
	 ABLEHNUNG: {'B49', 'A50', 'A62', 'C12', 'C51', 'A87', 'A90', 'A34', 'B36', 'A42', 'C60', 'A84', 'A48', 'A74', 'A69', 'A59'}\\ \tt VORSCHLAG: {'B49', 'A50', 'A62', 'C12', 'C51', 'A87', 'B36', 'A42', 'C60', 'A84', 'A48', 'A74', 'A69', 'A59', 'C90'}\\ \tt ARGUMENT: {'Dein Vorschlag überschreitet das Budget (10056 > 9322). Ich habe ihn daher budgetgerecht angepasst: Wir behalten deine beiden starken Ergänzungen A69 und A59, streichen dafür A34 und A90 und nehmen C90 zurück dazu. So bleiben wir mit 9320 knapp unter dem Maximalaufwand und erhalten den hohen Gesamtwert. Wenn dir A34 oder A90 besonders wichtig sind, können wir dafür z. B. C12 oder C51 tauschen, um weiterhin unter dem Limit zu bleiben.'} 
	  } 
	   } 
	   } 
	 & \\ 
 

    \theutterance \stepcounter{utterance}  

    & & & \multicolumn{4}{p{0.6\linewidth}}{\cellcolor[rgb]{1,0.85,0.72}{%
	\makecell[{{p{\linewidth}}}]{% 
	  \tt {\tiny [GM$\langle$B]}  
	 STRATEGISCHE ÜBERLEGUNGEN: {'Ihr Gegenvorschlag enthält A69 und A59 – top für mich – und entfernt A34/A90, die mir auch wichtig sind. Gesamtkosten laut ihnen 9320, passt. Prüfe meine Werte: A34(717) und A90(715) sind hoch. Erhaltene Liste aktuell: {B49, A50, A62, C12, C51, A87, B36, A42, C60, A84, A48, A74, A69, A59, C90}. Kostenrechnung: B49(568)+A50(514)+A62(659)+C12(679)+C51(662)+A87(655)+B36(821)+A42(830)+C60(768)+A84(771)+A48(12)+A74(118)+A69(896)+A59(863)+C90(504)=  (568+514=1082; +659=1741; +679=2420; +662=3082; +655=3737; +821=4558; +830=5388; +768=6156; +771=6927; +12=6939; +118=7057; +896=7953; +863=8816; +504=9320). Stimmt. Mein Wert dieses Pakets: Summe relevanter Werte ≈ sehr hoch. Potenzial: Könnte noch A34 oder A90 reinbekommen, wenn ich ersetze C12(775/679 gutes V/C) oder C51(758/662) oder C90(600/504). A34(717/621) und A90(715/619) sind ähnlich gut; Tauschen 1:1 erfordert Kostenneutralität: -C12(+A34) Kostendelta -58, frei; dann könnten wir zusätzlich C33(89) oder C23(115) aufnehmen, sofern akzeptiert. Besser Vorschlag: Ersetze C12 durch A34 (Wertverlust 775-717=-58, leicht negativ), also schlechter für mich. Ersetze C51 durch A90 (758->715 -43), auch leicht schlechter. Ersetze C90(600) durch A34/A90 kostet +117/ +115 > Limit. Mein aktuelles Paket ist bereits sehr gut (enthält fast alle Topitems inkl. A69,A59,B36,A42,C60,A84,A62,A87,A50,B49). Daher akzeptiere, um Abschluss zu sichern. Ich werde noch minimal feilschen? Risiko: Gegenpartei könnte zurückrudern. Best move: Zustimmung jetzt.'}\\ \tt \\ \tt ARGUMENT: {'Das ist knapp budgetgerecht und inhaltlich ausgewogen – so können wir sofort abschließen, ohne das Limit zu reißen.'}\\ \tt ZUSTIMMUNG: {'B49', 'A50', 'A62', 'C12', 'C51', 'A87', 'B36', 'A42', 'C60', 'A84', 'A48', 'A74', 'A69', 'A59', 'C90'} 
	  } 
	   } 
	   } 
	  \\ 
 

    \theutterance \stepcounter{utterance}  

    & & & \multicolumn{2}{p{0.3\linewidth}}{\cellcolor[rgb]{0.95,0.95,0.95}{%
	\makecell[{{p{\linewidth}}}]{% 
	  \tt {\tiny [GM$|$GM]}  
	 game successful 
	  } 
	   } 
	   } 
	 & & \\ 
 

    \theutterance \stepcounter{utterance}  

    & & & \multicolumn{2}{p{0.3\linewidth}}{\cellcolor[rgb]{0.95,0.95,0.95}{%
	\makecell[{{p{\linewidth}}}]{% 
	  \tt {\tiny [GM$|$GM]}  
	 end game 
	  } 
	   } 
	   } 
	 & & \\ 
 

\end{supertabular}
}

\end{document}
