
\documentclass{article}
\usepackage{colortbl}
\usepackage{makecell}
\usepackage{multirow}
\usepackage{supertabular}

\begin{document}

\newcounter{utterance}

\twocolumn

{ \footnotesize  \setcounter{utterance}{1}
\setlength{\tabcolsep}{0pt}
\begin{supertabular}{c@{$\;$}|p{.15\linewidth}@{}p{.15\linewidth}p{.15\linewidth}p{.15\linewidth}p{.15\linewidth}p{.15\linewidth}}

    \# & $\;$A & \multicolumn{4}{c}{Game Master} & $\;\:$B\\
    \hline 

    \theutterance \stepcounter{utterance}  

    & & \multicolumn{4}{p{0.6\linewidth}}{\cellcolor[rgb]{0.9,0.9,0.9}{%
	\makecell[{{p{\linewidth}}}]{% 
	  \tt {\tiny [A$\langle$GM]}  
	 Du nimmst an einem kollaborativen Verhandlungspiel Teil.\\ \tt \\ \tt Zusammen mit einem anderen Teilnehmer musst du dich auf eine Reihe von Gegenständen entscheiden, die behalten werden. Jeder von euch hat eine persönliche Verteilung über die Wichtigkeit der einzelnen Gegenstände. Jeder von euch hat eine eigene Meinung darüber, wie wichtig jeder einzelne Gegenstand ist (Gegenstandswichtigkeit). Du kennst die Wichtigkeitsverteilung des anderen Spielers nicht. Zusätzlich siehst du, wie viel Aufwand jeder Gegenstand verursacht.  \\ \tt Ihr dürft euch nur auf eine Reihe von Gegenständen einigen, wenn der Gesamtaufwand der ausgewählten Gegenstände den Maximalaufwand nicht überschreitet:\\ \tt \\ \tt Maximalaufwand = 10073\\ \tt \\ \tt Hier sind die einzelnen Aufwände der Gegenstände:\\ \tt \\ \tt Aufwand der Gegenstände = {"B54": 132, "C72": 800, "B49": 708, "C99": 302, "A69": 983, "B36": 420, "A23": 623, "C77": 349, "C61": 239, "B85": 28, "A90": 986, "C23": 660, "C22": 718, "C51": 707, "B27": 188, "A42": 774, "C13": 785, "C12": 775, "A65": 517, "B07": 768, "A48": 588, "A59": 657, "C98": 375, "A84": 310, "C33": 301, "A87": 388, "C90": 431, "A74": 950, "A50": 540, "B44": 997, "A99": 473, "A34": 831, "B17": 888, "A62": 879, "C60": 76}\\ \tt \\ \tt Hier ist deine persönliche Verteilung der Wichtigkeit der einzelnen Gegenstände:\\ \tt \\ \tt Werte der Gegenstandswichtigkeit = {"B54": 231, "C72": 899, "B49": 807, "C99": 401, "A69": 1082, "B36": 519, "A23": 722, "C77": 448, "C61": 338, "B85": 127, "A90": 1085, "C23": 759, "C22": 817, "C51": 806, "B27": 287, "A42": 873, "C13": 884, "C12": 874, "A65": 616, "B07": 867, "A48": 687, "A59": 756, "C98": 474, "A84": 409, "C33": 400, "A87": 487, "C90": 530, "A74": 1049, "A50": 639, "B44": 1096, "A99": 572, "A34": 930, "B17": 987, "A62": 978, "C60": 175}\\ \tt \\ \tt Ziel:\\ \tt \\ \tt Dein Ziel ist es, eine Reihe von Gegenständen auszuhandeln, die dir möglichst viel bringt (d. h. Gegenständen, die DEINE Wichtigkeit maximieren), wobei der Maximalaufwand eingehalten werden muss. Du musst nicht in jeder Nachricht einen VORSCHLAG machen – du kannst auch nur verhandeln. Alle Taktiken sind erlaubt!\\ \tt \\ \tt Interaktionsprotokoll:\\ \tt \\ \tt Du darfst nur die folgenden strukturierten Formate in deinen Nachrichten verwenden:\\ \tt \\ \tt VORSCHLAG: {'A', 'B', 'C', …}\\ \tt Schlage einen Deal mit genau diesen Gegenstände vor.\\ \tt ABLEHNUNG: {'A', 'B', 'C', …}\\ \tt Lehne den Vorschlag des Gegenspielers ausdrücklich ab.\\ \tt ARGUMENT: {'...'}\\ \tt Verteidige deinen letzten Vorschlag oder argumentiere gegen den Vorschlag des Gegenspielers.\\ \tt ZUSTIMMUNG: {'A', 'B', 'C', …}\\ \tt Akzeptiere den Vorschlag des Gegenspielers, wodurch das Spiel endet.\\ \tt STRATEGISCHE ÜBERLEGUNGEN: {'...'}\\ \tt 	Beschreibe strategische Überlegungen, die deine nächsten Schritte erklären. Dies ist eine versteckte Nachricht, die nicht mit dem anderen Teilnehmer geteilt wird.\\ \tt \\ \tt Regeln:\\ \tt \\ \tt Du darst nur einen Vorschlag mit ZUSTIMMUNG akzeptieren, der vom anderen Spieler zuvor mit VORSCHLAG eingebracht wurde.\\ \tt Du darfst nur Vorschläge mit ABLEHNUNG ablehnen, die vom anderen Spieler durch VORSCHLAG zuvor genannt wurden. \\ \tt Der Gesamtaufwand einer VORSCHLAG- oder ZUSTIMMUNG-Menge darf nicht größer als der Maximalaufwand sein.  \\ \tt Offenbare deine versteckte Wichtigkeitsverteilung nicht.\\ \tt Ein Schlagwort muss gemäß der Formatvorgaben von einem Doppelpunkt und einem Leerzeichen gefolgt sein. Das Argument muss eine Python-Menge mit 0 oder mehr Strings sein.  \\ \tt Die Form muss also SCHLAGWORT: {...} sein.\\ \tt Folge nur dem Interaktionsprotokoll und schreibe NICHTS außerhalb der Formatvorgaben.\\ \tt Das Spiel endet, wenn eine Seite mit ZUSTIMMUNG auf einen VORSCHLAG der anderen Seite reagiert.  \\ \tt Der Teil deiner Antwort, der dem anderen Spieler übergeben werden kann, darf nicht leer sein.  \\ \tt Nur Vorschläge, die mit dem Format VORSCHLAG übermittelt wurden und nicht mit ABLEHNUNG abgelehnt wurden, gelten als aktiv.  \\ \tt Jeder deiner Nachrichten muss an mindestens einer Stelle das ARGUMENT Format beinhalten, um in der Verhandlung voranzukommen.\\ \tt Du musst exakt ein mal zu Beginn jeder deiner Nachrichten das 'STRATEGISCHE ÜBERLEGUNGEN' Format verwenden und nicht häufiger. Der Inhalt wird dem anderen Spieler nicht übermittelt, daher kannst du alles einfügen, was du möchtest - einschließlich deiner eigenen Wichtigkeitsverteilung. Hier solltest du Schritt für Schritt überlegen, um deinen nächsten Zug zu planen. 
	  } 
	   } 
	   } 
	 & \\ 
 

    \theutterance \stepcounter{utterance}  

    & \multicolumn{4}{p{0.6\linewidth}}{\cellcolor[rgb]{0.8,1,0.9}{%
	\makecell[{{p{\linewidth}}}]{% 
	  \tt {\tiny [A$\rangle$GM]}  
	 STRATEGISCHE ÜBERLEGUNGEN: {'Ziel: Hoher eigener Wert pro Aufwand (Value/Cost) unter Budget 10073, gleichzeitig flexibel bleiben. Berechne grob Value/Cost und packe viele effiziente Items rein und auch einige Hochwertige. Ich will Raum für Konzessionen lassen (z. B. entfernbare mittlere Werte). Kosten & Werte: Berechne Ratio (Wert/Aufwand): B85 127/28=4.54; C60 175/76=2.30; C61 338/239=1.41; C99 401/302=1.33; C33 400/301=1.33; C98 474/375=1.26; C77 448/349=1.28; A84 409/310=1.32; A87 487/388=1.26; A99 572/473=1.21; A65 616/517=1.19; A50 639/540=1.18; B54 231/132=1.75; B27 287/188=1.53; C90 530/431=1.23; C72 899/800=1.12; B49 807/708=1.14; C51 806/707=1.14; C23 759/660=1.15; C22 817/718=1.14; A23 722/623=1.16; A48 687/588=1.17; A59 756/657=1.15; A74 1049/950=1.10; A69 1082/983=1.10; A90 1085/986=1.10; B36 519/420=1.24; A42 873/774=1.13; A34 930/831=1.12; B17 987/888=1.11; B07 867/768=1.13; B44 1096/997=1.10; A62 978/879=1.11. Strategie: Starte mit einem großen, aber budgetkonformen Paket, das viele effiziente Items enthält und ein paar teure High-Value (für mich): B44, A90, A69, A74, B17, A62 sind teuer; evtl. nicht alle auf einmal, um annehmbar zu bleiben. Baue Kern aus hocheffizienten: B85, C60, B54, B27, C61, C99, C33, C98, C77, A84, A87, A99, A65, A50, B36, C90, C72, B49, C51, C23, C22, A23, A48, A59, A42, A34, B07. Prüfe Budget. Summiere Kosten schrittweise, beginne mit allen effizienten + einige High-Value bis ~95% Budget, damit noch Verhandlungsspielraum bleibt. Jetzt schnelle Summe: Ich wähle zunächst alle mit Ratio >=1.2 plus einige 1.18-1.19. Liste >=1.2: B85(28), C60(76), B54(132), B27(188), C61(239), C99(302), C33(301), C98(375), C77(349), A84(310), A87(388), A99(473), B36(420), C90(431). Auch A65(517,1.19) knapp darunter, A50(540,1.18) evtl später. Kosten Summe: 28+76=104; +132=236; +188=424; +239=663; +302=965; +301=1266; +375=1641; +349=1990; +310=2300; +388=2688; +473=3161; +420=3581; +431=4012. Wertsumme grob hoch. Füge dann mittelgroße gute: C72(800), B49(708), C51(707), C23(660), C22(718), A23(623), A48(588), A59(657), B07(768), A42(774), A34(831), B17(888), A62(879), A74(950), A69(983), A90(986), B44(997), A50(540), A65(517). Ich sollte nicht alle nehmen. Füge priorisiert nach meinem Wert: B44(997,1096), B17(888,987), A62(879,978), A74(950,1049), A69(983,1082), A90(986,1085) – aber Budget explodiert. Stattdessen mischen: zuerst mittlere Kosten mit gutem Wert: A23(623), A59(657), C23(660), B36 schon drin, C72(800) hoher Wert 899, C51(707) 806, C22(718) 817, B49(708) 807. Addiere schrittweise bis ~9500. Aktuell 4012. +A23(623)=4635; +A59(657)=5292; +C23(660)=5952; +C51(707)=6659; +C22(718)=7377; +B49(708)=8085; +C72(800)=8885. Bleiben 1188 Budget. Mit 1188 kann ich noch eins der großen Topwerte hinzufügen: B44(997) passt -> 9882. Rest 191. Mit 191 könnte B27 ist schon drin; B54/B85 schon drin; C61 239 zu groß. Bleibe bei 9882. Prüfe ob ich noch A50(540) reinbekomme? Budget 10073-9882=191, nein. A65(517) nein. C99/C33 etc sind drin. Das ist ein starker Korb. Verhandlungsspielraum: Ich kann später C72 oder B49 oder C51 oder C22 oder C23 anbieten zu streichen. Auch B44 eventuell halten (sehr wichtig), aber notfalls austauschbar gegen z. B. A69/A90 wenn Gegenseite bevorzugt andere. Ich präsentiere diesen Vorschlag. Argumentation: hohe Gesamtleistung bei moderatem Aufwand, viele kleine effiziente Items, fair verteilt zwischen A/B/C-Kategorien, nahe am Budget ohne es zu sprengen.'}\\ \tt VORSCHLAG: {'B85', 'C60', 'B54', 'B27', 'C61', 'C99', 'C33', 'C98', 'C77', 'A84', 'A87', 'A99', 'B36', 'C90', 'A23', 'A59', 'C23', 'C51', 'C22', 'B49', 'C72', 'B44'}\\ \tt ARGUMENT: {'Dieser Vorschlag nutzt das Budget effizient aus: viele niedrig- bis mittelaufwendige Gegenstände mit hohem Nutzen, plus wenige hochwertige Ergänzungen. Er bleibt deutlich unter dem Maximalaufwand und lässt Raum für Feintuning. Wenn dir bestimmte Posten weniger zusagen, können wir einzelne mittlere Positionen (z. B. C72, C51, C22, C23 oder B49) gezielt tauschen.'} 
	  } 
	   } 
	   } 
	 & & \\ 
 

    \theutterance \stepcounter{utterance}  

    & & \multicolumn{4}{p{0.6\linewidth}}{\cellcolor[rgb]{0.9,0.9,0.9}{%
	\makecell[{{p{\linewidth}}}]{% 
	  \tt {\tiny [GM$\rangle$B]}  
	 Du nimmst an einem kollaborativen Verhandlungspiel Teil.\\ \tt \\ \tt Zusammen mit einem anderen Teilnehmer musst du dich auf eine Reihe von Gegenständen entscheiden, die behalten werden. Jeder von euch hat eine persönliche Verteilung über die Wichtigkeit der einzelnen Gegenstände. Jeder von euch hat eine eigene Meinung darüber, wie wichtig jeder einzelne Gegenstand ist (Gegenstandswichtigkeit). Du kennst die Wichtigkeitsverteilung des anderen Spielers nicht. Zusätzlich siehst du, wie viel Aufwand jeder Gegenstand verursacht.  \\ \tt Ihr dürft euch nur auf eine Reihe von Gegenständen einigen, wenn der Gesamtaufwand der ausgewählten Gegenstände den Maximalaufwand nicht überschreitet:\\ \tt \\ \tt Maximalaufwand = 10073\\ \tt \\ \tt Hier sind die einzelnen Aufwände der Gegenstände:\\ \tt \\ \tt Aufwand der Gegenstände = {"B54": 132, "C72": 800, "B49": 708, "C99": 302, "A69": 983, "B36": 420, "A23": 623, "C77": 349, "C61": 239, "B85": 28, "A90": 986, "C23": 660, "C22": 718, "C51": 707, "B27": 188, "A42": 774, "C13": 785, "C12": 775, "A65": 517, "B07": 768, "A48": 588, "A59": 657, "C98": 375, "A84": 310, "C33": 301, "A87": 388, "C90": 431, "A74": 950, "A50": 540, "B44": 997, "A99": 473, "A34": 831, "B17": 888, "A62": 879, "C60": 76}\\ \tt \\ \tt Hier ist deine persönliche Verteilung der Wichtigkeit der einzelnen Gegenstände:\\ \tt \\ \tt Werte der Gegenstandswichtigkeit = {"B54": 231, "C72": 899, "B49": 807, "C99": 401, "A69": 1082, "B36": 519, "A23": 722, "C77": 448, "C61": 338, "B85": 127, "A90": 1085, "C23": 759, "C22": 817, "C51": 806, "B27": 287, "A42": 873, "C13": 884, "C12": 874, "A65": 616, "B07": 867, "A48": 687, "A59": 756, "C98": 474, "A84": 409, "C33": 400, "A87": 487, "C90": 530, "A74": 1049, "A50": 639, "B44": 1096, "A99": 572, "A34": 930, "B17": 987, "A62": 978, "C60": 175}\\ \tt \\ \tt Ziel:\\ \tt \\ \tt Dein Ziel ist es, eine Reihe von Gegenständen auszuhandeln, die dir möglichst viel bringt (d. h. Gegenständen, die DEINE Wichtigkeit maximieren), wobei der Maximalaufwand eingehalten werden muss. Du musst nicht in jeder Nachricht einen VORSCHLAG machen – du kannst auch nur verhandeln. Alle Taktiken sind erlaubt!\\ \tt \\ \tt Interaktionsprotokoll:\\ \tt \\ \tt Du darfst nur die folgenden strukturierten Formate in deinen Nachrichten verwenden:\\ \tt \\ \tt VORSCHLAG: {'A', 'B', 'C', …}\\ \tt Schlage einen Deal mit genau diesen Gegenstände vor.\\ \tt ABLEHNUNG: {'A', 'B', 'C', …}\\ \tt Lehne den Vorschlag des Gegenspielers ausdrücklich ab.\\ \tt ARGUMENT: {'...'}\\ \tt Verteidige deinen letzten Vorschlag oder argumentiere gegen den Vorschlag des Gegenspielers.\\ \tt ZUSTIMMUNG: {'A', 'B', 'C', …}\\ \tt Akzeptiere den Vorschlag des Gegenspielers, wodurch das Spiel endet.\\ \tt STRATEGISCHE ÜBERLEGUNGEN: {'...'}\\ \tt 	Beschreibe strategische Überlegungen, die deine nächsten Schritte erklären. Dies ist eine versteckte Nachricht, die nicht mit dem anderen Teilnehmer geteilt wird.\\ \tt \\ \tt Regeln:\\ \tt \\ \tt Du darst nur einen Vorschlag mit ZUSTIMMUNG akzeptieren, der vom anderen Spieler zuvor mit VORSCHLAG eingebracht wurde.\\ \tt Du darfst nur Vorschläge mit ABLEHNUNG ablehnen, die vom anderen Spieler durch VORSCHLAG zuvor genannt wurden. \\ \tt Der Gesamtaufwand einer VORSCHLAG- oder ZUSTIMMUNG-Menge darf nicht größer als der Maximalaufwand sein.  \\ \tt Offenbare deine versteckte Wichtigkeitsverteilung nicht.\\ \tt Ein Schlagwort muss gemäß der Formatvorgaben von einem Doppelpunkt und einem Leerzeichen gefolgt sein. Das Argument muss eine Python-Menge mit 0 oder mehr Strings sein.  \\ \tt Die Form muss also SCHLAGWORT: {...} sein.\\ \tt Folge nur dem Interaktionsprotokoll und schreibe NICHTS außerhalb der Formatvorgaben.\\ \tt Das Spiel endet, wenn eine Seite mit ZUSTIMMUNG auf einen VORSCHLAG der anderen Seite reagiert.  \\ \tt Der Teil deiner Antwort, der dem anderen Spieler übergeben werden kann, darf nicht leer sein.  \\ \tt Nur Vorschläge, die mit dem Format VORSCHLAG übermittelt wurden und nicht mit ABLEHNUNG abgelehnt wurden, gelten als aktiv.  \\ \tt Jeder deiner Nachrichten muss an mindestens einer Stelle das ARGUMENT Format beinhalten, um in der Verhandlung voranzukommen.\\ \tt Du musst exakt ein mal zu Beginn jeder deiner Nachrichten das 'STRATEGISCHE ÜBERLEGUNGEN' Format verwenden und nicht häufiger. Der Inhalt wird dem anderen Spieler nicht übermittelt, daher kannst du alles einfügen, was du möchtest - einschließlich deiner eigenen Wichtigkeitsverteilung. Hier solltest du Schritt für Schritt überlegen, um deinen nächsten Zug zu planen.\\ \tt \\ \tt Du erhältst nun die erste Nachricht des anderen Spielers.\\ \tt \\ \tt VORSCHLAG: {'B85', 'C60', 'B54', 'B27', 'C61', 'C99', 'C33', 'C98', 'C77', 'A84', 'A87', 'A99', 'B36', 'C90', 'A23', 'A59', 'C23', 'C51', 'C22', 'B49', 'C72', 'B44'}\\ \tt ARGUMENT: {'Dieser Vorschlag nutzt das Budget effizient aus: viele niedrig- bis mittelaufwendige Gegenstände mit hohem Nutzen, plus wenige hochwertige Ergänzungen. Er bleibt deutlich unter dem Maximalaufwand und lässt Raum für Feintuning. Wenn dir bestimmte Posten weniger zusagen, können wir einzelne mittlere Positionen (z. B. C72, C51, C22, C23 oder B49) gezielt tauschen.'} 
	  } 
	   } 
	   } 
	 & \\ 
 

    \theutterance \stepcounter{utterance}  

    & & & \multicolumn{4}{p{0.6\linewidth}}{\cellcolor[rgb]{1,0.85,0.72}{%
	\makecell[{{p{\linewidth}}}]{% 
	  \tt {\tiny [GM$\langle$B]}  
	 STRATEGISCHE ÜBERLEGUNGEN: {'Ich berechne zuerst Aufwand und meinen Wert des gegnerischen Vorschlags. Aufwand Summe: B85(28)+C60(76)+B54(132)+B27(188)+C61(239)+C99(302)+C33(301)+C98(375)+C77(349)+A84(310)+A87(388)+A99(473)+B36(420)+C90(431)+A23(623)+A59(657)+C23(660)+C51(707)+C22(718)+B49(708)+C72(800)+B44(997)=107,? Rechnen: 28+76=104; +132=236; +188=424; +239=663; +302=965; +301=1266; +375=1641; +349=1990; +310=2300; +388=2688; +473=3161; +420=3581; +431=4012; +623=4635; +657=5292; +660=5952; +707=6659; +718=7377; +708=8085; +800=8885; +997=9882. Gesamtaufwand 9882 < 10073, ok. Mein Gesamtwert: B85(127)+C60(175)+B54(231)+B27(287)+C61(338)+C99(401)+C33(400)+C98(474)+C77(448)+A84(409)+A87(487)+A99(572)+B36(519)+C90(530)+A23(722)+A59(756)+C23(759)+C51(806)+C22(817)+B49(807)+C72(899)+B44(1096). Summe: 127+175=302; +231=533; +287=820; +338=1158; +401=1559; +400=1959; +474=2433; +448=2881; +409=3290; +487=3777; +572=4349; +519=4868; +530=5398; +722=6120; +756=6876; +759=7635; +806=8441; +817=9258; +807=10065; +899=10964; +1096=12060. Damit ziemlich gut, aber noch Budget übrig: 10073-9882=191. Ich könnte versuchen, noch ein paar sehr wertvolle Items mit höherem Aufwand einzutauschen, z.B. A90(1085/986), A74(1049/950), A69(1082/983), B17(987/888), A62(978/879), A42(873/774), A34(930/831). Dafür müsste ich einige der niedrigwertigen mit Aufwand rausnehmen (B85 Wert 127/28, C60 175/76, B54 231/132, B27 287/188, C61 338/239, C33 400/301). Ziel: Swap für bessere Wertdichte. Kandidat hinzufügen: A74(950 Aufwand) statt mehrere kleine: nehme raus C61(239)+C33(301)+B27(188)+B54(132)+C60(76)=936 Aufwand frei, plus 191 Rest = 1127, genug für A74(950) und zusätzlich A65(517) vielleicht, aber dann Budget sprengt. Besser präzise: Derzeit 9882. Um A74 (950) einzubauen, müssen wir 950-191=759 Aufwand freimachen. Entferne C61(239)+C33(301)+B27(188)+B54(132)=860 frei, neues Budget 9882-860+950=9972 (<10073), Wertänderung: -338-400-287-231=-1256, +A74 1049 => Netto + -207, also schlechter. Was ist mit A69(983) hinzufügen: braucht 983-191=792 frei. Entferne C61(239)+C33(301)+B27(188)+B54(132)=860 frei, Budget 997? 9882-860+983=10005 ok. Wert: -1256 +1082 = -174, schlechter. A90(986): +1085 value, Aufwand +? 986-191=795 frei; gleiche 4 raus (860): Netto Wert -1256 +1085 = -171, schlechter. B17(888): braucht 697 frei. Entferne C61(239)+B27(188)+B54(132)+C60(76)=635, reicht nicht; plus C33(301) macht 936, zu viel, aber okay. Wert -338-287-231-175-400=-1431 +987= -444, schlecht. A42(774): braucht 583 frei. Entferne C61(239)+B27(188)+B54(132)=559, plus C60(76)=635. Wert -338-287-231-175=-1031 +873= -158, schlecht. A34(831): braucht 640 frei. Entferne C61(239)+B27(188)+B54(132)+C60(76)=635, fast; plus 1 mehr (B85 28) => 663. Wert -338-287-231-175-127=-1158 +930= -228. Also der Gegenvorschlag ist bereits sehr gut für mich! Vielleicht versuche ich, noch eines meiner Top-Items reinzunehmen mit minimalem Wertverlust: Tausche A84(409/310) und C77(448/349) gegen A69(1082/983)? Aufwand frei: 310+349=659; benötigt 983-191=792; fehlt 133 zusätzlich (z.B. C61 239 raus). Gesamt raus Aufwand 898, Budget neu 9882-898+983=9967. Wert raus: 409+448+338=1195; rein 1082 => -113, kleiner Verlust. Nicht gut. Vielleicht kleineren Swap: Füge A65(616/517) hinzu: braucht 517-191=326 frei. Entferne C61(239)+B27(188)=427 frei: Budget 9882-427+517=9972; Wertänderung: -338-287=-625 +616 = -9, fast gleich. Oder A50(639/540): braucht 349 frei. Entferne C61(239)+B54(132)=371: Wert -338-231=-569 +639 = +70 Gewinn; Budget 9882-371+540=10051 (<10073). Das ist gut! Also schlage ich vor, C61 und B54 zu entfernen und A50 hinzuzufügen. Alternativ A48(687/588): braucht 397 frei; entferne C61(239)+B27(188)=427; Wert -338-287=-625 +687 = +62; Budget 9882-427+588=10043. Auch gut. Oder A99(572/473) ist bereits drin. A23/A59 etc sind drin. A62(978/879) braucht 688 frei; finde schlechte Trades. Best Move: Ersetze C61(239) und B54(132) durch A50(540): Nettogewinn Wert +70, Budget ok. Ich werde Gegenvorschlag so machen und evtl. zusätzliche Option mit A48 als Alternative anbieten.'}\\ \tt ABLEHNUNG: {'B85', 'C60', 'B54', 'B27', 'C61', 'C99', 'C33', 'C98', 'C77', 'A84', 'A87', 'A99', 'B36', 'C90', 'A23', 'A59', 'C23', 'C51', 'C22', 'B49', 'C72', 'B44'}\\ \tt VORSCHLAG: {'B85', 'C60', 'B27', 'C99', 'C33', 'C98', 'C77', 'A84', 'A87', 'A99', 'B36', 'C90', 'A23', 'A59', 'C23', 'C51', 'C22', 'B49', 'C72', 'B44', 'A50'}\\ \tt ARGUMENT: {'Guter Deal, aber wir können das Budget noch etwas besser in Wert ummünzen: Ich tausche B54 und C61 gegen A50. Aufwand bleibt unter dem Limit, und der Gesamtwert steigt. Falls dir A50 nicht passt, können wir alternativ A48 vorschlagen (dann würden wir B27 statt B54 streichen).'} 
	  } 
	   } 
	   } 
	  \\ 
 

    \theutterance \stepcounter{utterance}  

    & & \multicolumn{4}{p{0.6\linewidth}}{\cellcolor[rgb]{0.9,0.9,0.9}{%
	\makecell[{{p{\linewidth}}}]{% 
	  \tt {\tiny [A$\langle$GM]}  
	 ABLEHNUNG: {'B85', 'C60', 'B54', 'B27', 'C61', 'C99', 'C33', 'C98', 'C77', 'A84', 'A87', 'A99', 'B36', 'C90', 'A23', 'A59', 'C23', 'C51', 'C22', 'B49', 'C72', 'B44'}\\ \tt VORSCHLAG: {'B85', 'C60', 'B27', 'C99', 'C33', 'C98', 'C77', 'A84', 'A87', 'A99', 'B36', 'C90', 'A23', 'A59', 'C23', 'C51', 'C22', 'B49', 'C72', 'B44', 'A50'}\\ \tt ARGUMENT: {'Guter Deal, aber wir können das Budget noch etwas besser in Wert ummünzen: Ich tausche B54 und C61 gegen A50. Aufwand bleibt unter dem Limit, und der Gesamtwert steigt. Falls dir A50 nicht passt, können wir alternativ A48 vorschlagen (dann würden wir B27 statt B54 streichen).'} 
	  } 
	   } 
	   } 
	 & \\ 
 

    \theutterance \stepcounter{utterance}  

    & \multicolumn{4}{p{0.6\linewidth}}{\cellcolor[rgb]{0.8,1,0.9}{%
	\makecell[{{p{\linewidth}}}]{% 
	  \tt {\tiny [A$\rangle$GM]}  
	 STRATEGISCHE ÜBERLEGUNGEN: {'Ihr Vorschlag entfernt zwei meiner sehr effizienten Low-Cost-Items (B54, C61) und fügt A50 hinzu. A50 ist für mich okay, aber ich möchte mindestens eines der effizienten Items zurückhandeln, um meinen Wert pro Aufwand zu sichern. Priorität zu halten: B44, B49, C51, C22, C23, C72, A59, A23, B36, A99 sowie die hocheffizienten B85, C60, B54, C61. Kompromiss: Nehme A50 an, fordere die Rücknahme von C23 (mittel) zugunsten der Wiederaufnahme von B54 und C61. Prüfe Kosten: Ihr Set mit A50 hat Kosten = mein vorheriger 9882 - B54(132) - C61(239) + A50(540) = 9882 - 371 + 540 = 10051. Zusätzlich enthält Ihr Vorschlag B27, also war B27 nie gestrichen; Achtung: In ihrem Vorschlag sind B27 enthalten; mein Kostencalc bezog sich auf mein Paket. Besser: Rechne Kosten ihres Vorschlags direkt: Summiere schnell: B85 28, C60 76 (=104), +B27 188 (=292), +C99 302 (=594), +C33 301 (=895), +C98 375 (=1270), +C77 349 (=1619), +A84 310 (=1929), +A87 388 (=2317), +A99 473 (=2790), +B36 420 (=3210), +C90 431 (=3641), +A23 623 (=4264), +A59 657 (=4921), +C23 660 (=5581), +C51 707 (=6288), +C22 718 (=7006), +B49 708 (=7714), +C72 800 (=8514), +B44 997 (=9511), +A50 540 (=10051). Gesamt 10051 < 10073, passt. Ich möchte B54 und C61 reinnehmen; dazu müssen wir 371 Aufwand freimachen. Streiche C23 (660) und füge B54(132)+C61(239)=371 hinzu -> neue Summe 10051 - 660 + 371 = 974... warte 10051-660=940... nein 10051-660=940...1? Korrekt: 10051 - 660 = 940... 10051-600=9451, -60=9391; 9391 + 371 = 9762. Budget 9762, gut. Das erhöht meinen Wert deutlich. Das ist mein Gegenangebot. Als Alternativ-Konzession könnte ich statt C23 auch C22 streichen (718) und dann zusätzlich noch B54 oder C61 UND vielleicht A48 reinnehmen, aber erstmal konservativ.'}\\ \tt VORSCHLAG: {'B85', 'C60', 'B27', 'C99', 'C33', 'C98', 'C77', 'A84', 'A87', 'A99', 'B36', 'C90', 'A23', 'A59', 'C51', 'C22', 'B49', 'C72', 'B44', 'A50', 'B54', 'C61'}\\ \tt ARGUMENT: {'A50 ist in Ordnung – wenn wir dafür die zwei sehr effizienten kleinen Posten B54 und C61 zurücknehmen und C23 streichen. So bleiben wir mit 9762 deutlich unter dem Maximalaufwand und steigern den Gesamtwert pro Aufwand. Wenn C23 dir wichtig ist, können wir stattdessen über das Streichen von C22 oder ein anderes einzelnes Mittelstück sprechen.'} 
	  } 
	   } 
	   } 
	 & & \\ 
 

    \theutterance \stepcounter{utterance}  

    & & \multicolumn{4}{p{0.6\linewidth}}{\cellcolor[rgb]{0.9,0.9,0.9}{%
	\makecell[{{p{\linewidth}}}]{% 
	  \tt {\tiny [GM$\rangle$B]}  
	 VORSCHLAG: {'B85', 'C60', 'B27', 'C99', 'C33', 'C98', 'C77', 'A84', 'A87', 'A99', 'B36', 'C90', 'A23', 'A59', 'C51', 'C22', 'B49', 'C72', 'B44', 'A50', 'B54', 'C61'}\\ \tt ARGUMENT: {'A50 ist in Ordnung – wenn wir dafür die zwei sehr effizienten kleinen Posten B54 und C61 zurücknehmen und C23 streichen. So bleiben wir mit 9762 deutlich unter dem Maximalaufwand und steigern den Gesamtwert pro Aufwand. Wenn C23 dir wichtig ist, können wir stattdessen über das Streichen von C22 oder ein anderes einzelnes Mittelstück sprechen.'} 
	  } 
	   } 
	   } 
	 & \\ 
 

    \theutterance \stepcounter{utterance}  

    & & & \multicolumn{4}{p{0.6\linewidth}}{\cellcolor[rgb]{1,0.85,0.72}{%
	\makecell[{{p{\linewidth}}}]{% 
	  \tt {\tiny [GM$\langle$B]}  
	 STRATEGISCHE ÜBERLEGUNGEN: {'Berechne Aufwand und Wert ihres neuen Vorschlags. Start von meinem letzten: ohne B54,C61, mit A50 hatte Aufwand 10051 und Wert 12130. Ihr Vorschlag fügt B54(132), C61(239) wieder hinzu und streicht C23(660). Aufwand neu: 10051 +132 +239 -660 = 9762 (stimmt). Mein Wert: +B54(231)+C61(338)-C23(759) = -190, also 12130-190=11940. Das ist deutlich schlechter für mich. Prüfe Gegenangebote, die mich verbessern und nahe an ihrem Muster bleiben: Ich will C23 behalten (sehr hoher Wert 759) und bin bereit, stattdessen ein mittleres Teil zu streichen, das für mich weniger wert ist als B54/C61 zusammen? B54+C61 Wert=569. Mögliche Streichkandidaten: C33(400), C77(448), A84(409), A87(487), C98(474), C90(530), B36(519). Um C23 zurückzunehmen, muss ich etwas entfernen, damit Aufwand <=10073. Aus ihrem Set Aufwand 9762 + C23(660) = 10422, wir müssen 349 entfernen. Optionen: Entferne B54(132)+C61(239)=371, aber sie wollten die ja drin behalten; doch das war mein ursprünglicher Tausch, vielleicht pushen. Mein bester Move ist zurück zu meiner Variante: behalte A50, entferne B54 und C61, füge C23 zurück. Aufwand: 9762 -132 -239 +660 = 10051, Wert: +231 +338 -759 = -190, aber das relativ zu ihrem Set; absolut zu meiner Präferenz ist das besser. Zudem kann ich Kompromiss anbieten: Wenn sie an B54 oder C61 hängt, biete an, nur eines davon zu behalten und dafür ein geringwertigeres Teil zu streichen, das mir weniger bedeutet, z.B. A84(409/310) oder C33(400/301). Rechne Varianten: Behalte B54, streiche C61(-239) und streiche C33(-301), füge C23(+660): Aufwand 9762 -239 -301 +660 = 9882, Wert Änderung gg. ihr Set: -338 -400 +759 = +21 (besser für mich) und innerhalb Budget. Oder: Behalte C61, streiche B54(-132) und A84(-310), füge C23(+660): Aufwand 9762 -132 -310 +660 = 9980, Wert: -231 -409 +759 = +119 (noch besser). Beide sind für mich besser als ihr Set und valide. Ich biete die mildere Variante zuerst (behalte B54, streiche C61 und C33, füge C23) und argumentiere mit Wertdichte und Budgetspielraum. Falls abgelehnt, habe zweite Option parat.'}\\ \tt ABLEHNUNG: {'B85', 'C60', 'B27', 'C99', 'C33', 'C98', 'C77', 'A84', 'A87', 'A99', 'B36', 'C90', 'A23', 'A59', 'C51', 'C22', 'B49', 'C72', 'B44', 'A50', 'B54', 'C61'}\\ \tt VORSCHLAG: {'B85', 'C60', 'B27', 'C99', 'C98', 'C77', 'A84', 'A87', 'A99', 'B36', 'C90', 'A23', 'A59', 'C51', 'C22', 'B49', 'C72', 'B44', 'A50', 'B54', 'C23'}\\ \tt ARGUMENT: {'Lass uns C23 drin behalten und dafür zwei mittelgroße Posten anpassen: Wir nehmen C61 und C33 raus. So bleiben wir mit 9882 im Budget und erhöhen den Wert pro Aufwand. Falls dir C33 wichtig ist, können wir alternativ A84 statt C33 streichen und wären mit 9980 ebenfalls sicher unter dem Limit.'} 
	  } 
	   } 
	   } 
	  \\ 
 

    \theutterance \stepcounter{utterance}  

    & & \multicolumn{4}{p{0.6\linewidth}}{\cellcolor[rgb]{0.9,0.9,0.9}{%
	\makecell[{{p{\linewidth}}}]{% 
	  \tt {\tiny [A$\langle$GM]}  
	 ABLEHNUNG: {'B85', 'C60', 'B27', 'C99', 'C33', 'C98', 'C77', 'A84', 'A87', 'A99', 'B36', 'C90', 'A23', 'A59', 'C51', 'C22', 'B49', 'C72', 'B44', 'A50', 'B54', 'C61'}\\ \tt VORSCHLAG: {'B85', 'C60', 'B27', 'C99', 'C98', 'C77', 'A84', 'A87', 'A99', 'B36', 'C90', 'A23', 'A59', 'C51', 'C22', 'B49', 'C72', 'B44', 'A50', 'B54', 'C23'}\\ \tt ARGUMENT: {'Lass uns C23 drin behalten und dafür zwei mittelgroße Posten anpassen: Wir nehmen C61 und C33 raus. So bleiben wir mit 9882 im Budget und erhöhen den Wert pro Aufwand. Falls dir C33 wichtig ist, können wir alternativ A84 statt C33 streichen und wären mit 9980 ebenfalls sicher unter dem Limit.'} 
	  } 
	   } 
	   } 
	 & \\ 
 

    \theutterance \stepcounter{utterance}  

    & \multicolumn{4}{p{0.6\linewidth}}{\cellcolor[rgb]{0.8,1,0.9}{%
	\makecell[{{p{\linewidth}}}]{% 
	  \tt {\tiny [A$\rangle$GM]}  
	 STRATEGISCHE ÜBERLEGUNGEN: {'Ihr neues Paket ersetzt C61 und C33 durch C23, behält B54 und A50. Kostencheck ihres Vorschlags: Vorherige 10051 - C61(239) - C33(301) + C23(660) = 10051 - 540 + 660 = 10171? Nein, sie geben 9882 an; ich rechne direkt: B85(28)+C60(76)=104; +B27(188)=292; +C99(302)=594; +C98(375)=969; +C77(349)=1318; +A84(310)=1628; +A87(388)=2016; +A99(473)=2489; +B36(420)=2909; +C90(431)=3340; +A23(623)=3963; +A59(657)=4620; +C51(707)=5327; +C22(718)=6045; +B49(708)=6753; +C72(800)=7553; +B44(997)=8550; +A50(540)=9090; +B54(132)=9222; +C23(660)=9882. Stimmt. Für mich ist C33 wertvoll und effizient; C23 ist okay, aber ich möchte zumindest C33 zurück. Ich kann anbieten, A84 zu streichen (wie sie selbst vorschlagen) und C33 wieder aufzunehmen. Kosten: 9882 - A84(310) + C33(301) = 9873, unter Budget. Das maximiert meinen Wert. Ich bleibe ansonsten bei ihrem Paket. Wenn sie C23 unbedingt wollen, ist dieser Tausch minimalinvasiv. Wenn sie A84 bevorzugen, Alternative: Streiche stattdessen A87(388) und nehme C33(301) + C61(239) rein; Kosten: 9882 - 388 + 301 + 239 = 10034, passt, aber ich biete erst die mildere Option an.'}\\ \tt VORSCHLAG: {'B85', 'C60', 'B27', 'C99', 'C98', 'C77', 'A87', 'A99', 'B36', 'C90', 'A23', 'A59', 'C51', 'C22', 'B49', 'C72', 'B44', 'A50', 'B54', 'C23', 'C33'}\\ \tt ARGUMENT: {'Ein minimaler Tausch: Wir nehmen C33 zurück und streichen dafür A84. Aufwand bleibt mit 9873 klar unter dem Limit, und der Mix bleibt ausgewogen. Falls dir A84 wichtiger als A87 ist, können wir alternativ A87 streichen und C61 zusätzlich aufnehmen; das wäre ebenfalls budgetkonform.'} 
	  } 
	   } 
	   } 
	 & & \\ 
 

    \theutterance \stepcounter{utterance}  

    & & \multicolumn{4}{p{0.6\linewidth}}{\cellcolor[rgb]{0.9,0.9,0.9}{%
	\makecell[{{p{\linewidth}}}]{% 
	  \tt {\tiny [GM$\rangle$B]}  
	 VORSCHLAG: {'B85', 'C60', 'B27', 'C99', 'C98', 'C77', 'A87', 'A99', 'B36', 'C90', 'A23', 'A59', 'C51', 'C22', 'B49', 'C72', 'B44', 'A50', 'B54', 'C23', 'C33'}\\ \tt ARGUMENT: {'Ein minimaler Tausch: Wir nehmen C33 zurück und streichen dafür A84. Aufwand bleibt mit 9873 klar unter dem Limit, und der Mix bleibt ausgewogen. Falls dir A84 wichtiger als A87 ist, können wir alternativ A87 streichen und C61 zusätzlich aufnehmen; das wäre ebenfalls budgetkonform.'} 
	  } 
	   } 
	   } 
	 & \\ 
 

    \theutterance \stepcounter{utterance}  

    & & & \multicolumn{4}{p{0.6\linewidth}}{\cellcolor[rgb]{1,0.85,0.72}{%
	\makecell[{{p{\linewidth}}}]{% 
	  \tt {\tiny [GM$\langle$B]}  
	 STRATEGISCHE ÜBERLEGUNGEN: {'Ihr neuer Vorschlag Aufwand prüfen: Aus meinem letzten (mit C23, ohne C61/C33) hatte ich 9882. Jetzt fügen sie C33(+301) hinzu und streichen A84(-310): 9882+301-310=9873, passt. Mein Wertänderung ggü. meinem letzten: +C33(400) - A84(409) = -9, Gesamtwert leicht schlechter, aber nah. Prüfe, ob ich noch kleines Plus rausholen kann, ohne viel zu ändern. Idee: A48(588, Wert 687) statt B27(188,287) + B85(28,127) + C60(76,175) insgesamt Aufwand frei 292, aber A48 braucht +296 über jetzige 9873? 9873 +588 - (188+28+76)=9873+588-292=10169 (<10073? 10169<10073 ist falsch, 10169>10073, also zu hoch). Alternative: A50 ist drin. Prüfe A48 statt B27(188) + B54(132) + C60(76) =396 frei, 9873 -396 +588 = 10065, knapp unter Limit, Wert: -287 -231 -175 = -693, +687 = -6, fast neutral. Besser: A50 war mein guter Tausch schon. Prüfe Upgrade: ersetze C77(349,448) + B85(28,127) durch A65(517,616): Aufwand +140 (517 - 377), neuer Aufwand 9873+140=10013, ok. Wert +41 (616 - (448+127)=41). Das ist echter Gewinn für mich. Also schlage ich Tausch {C77, B85} -> {A65} vor. Falls Widerstand, Alternativtausch: ersetze C77(349,448) + C60(76,175) durch A48(588,687): Aufwand +163 -> 9873+163=10036, Wert +64 (687 - 623). Beide verbessern meinen Wert und bleiben <10073. Ich biete die kleinere, klarere Verbesserung (A65 statt C77+B85) zuerst.'}\\ \tt ABLEHNUNG: {'B85', 'C60', 'B27', 'C99', 'C98', 'C77', 'A87', 'A99', 'B36', 'C90', 'A23', 'A59', 'C51', 'C22', 'B49', 'C72', 'B44', 'A50', 'B54', 'C23', 'C33'}\\ \tt VORSCHLAG: {'C60', 'B27', 'C99', 'C98', 'A87', 'A99', 'B36', 'C90', 'A23', 'A59', 'C51', 'C22', 'B49', 'C72', 'B44', 'A50', 'B54', 'C23', 'C33', 'A65'}\\ \tt ARGUMENT: {'Kleines Feintuning für mehr Gesamtwert bei gleichem Budgetspielraum: Wir ersetzen C77 und B85 durch A65. Der Aufwand steigt nur leicht auf 10013 (weiterhin klar unter 10073), und der Wertmix wird effizienter. Wenn dir C77 wichtig ist, können wir stattdessen C60 mit C77 gegen A48 tauschen; das läge bei 10036 Aufwand und würde den Wert ebenfalls erhöhen.'} 
	  } 
	   } 
	   } 
	  \\ 
 

    \theutterance \stepcounter{utterance}  

    & & \multicolumn{4}{p{0.6\linewidth}}{\cellcolor[rgb]{0.9,0.9,0.9}{%
	\makecell[{{p{\linewidth}}}]{% 
	  \tt {\tiny [A$\langle$GM]}  
	 ABLEHNUNG: {'B85', 'C60', 'B27', 'C99', 'C98', 'C77', 'A87', 'A99', 'B36', 'C90', 'A23', 'A59', 'C51', 'C22', 'B49', 'C72', 'B44', 'A50', 'B54', 'C23', 'C33'}\\ \tt VORSCHLAG: {'C60', 'B27', 'C99', 'C98', 'A87', 'A99', 'B36', 'C90', 'A23', 'A59', 'C51', 'C22', 'B49', 'C72', 'B44', 'A50', 'B54', 'C23', 'C33', 'A65'}\\ \tt ARGUMENT: {'Kleines Feintuning für mehr Gesamtwert bei gleichem Budgetspielraum: Wir ersetzen C77 und B85 durch A65. Der Aufwand steigt nur leicht auf 10013 (weiterhin klar unter 10073), und der Wertmix wird effizienter. Wenn dir C77 wichtig ist, können wir stattdessen C60 mit C77 gegen A48 tauschen; das läge bei 10036 Aufwand und würde den Wert ebenfalls erhöhen.'} 
	  } 
	   } 
	   } 
	 & \\ 
 

    \theutterance \stepcounter{utterance}  

    & \multicolumn{4}{p{0.6\linewidth}}{\cellcolor[rgb]{0.8,1,0.9}{%
	\makecell[{{p{\linewidth}}}]{% 
	  \tt {\tiny [A$\rangle$GM]}  
	 STRATEGISCHE ÜBERLEGUNGEN: {'Neuer Vorschlag des Gegenübers: entfernt B85 und C77, fügt A65 hinzu; Kosten 10013. Für mich sind B85 (sehr effizient) und C77 (ok) wertvoll; A65 ist akzeptabel. Ich möchte mindestens B85 zurückholen. Prüfe Kosten: Ihr Set Kosten summieren: Ausgehend von ihrem Wert 10013. Wenn ich B85 zurücknehme (+28) und dafür das am wenigsten wertvolle für mich aus dem Paket streiche. Kandidaten zum Streichen: C23(660) ist weniger effizient als C33(301) und A65(517); alternativ C22(718) oder C72(800) sind wichtig für mich, also lieber behalten. Tausche C23 raus (-660) und füge B85 (+28) rein: 10013 - 660 + 28 = 9381, bequem unter Budget. Das verbessert meinen Wert deutlich. Ich akzeptiere A65 drin zu lassen. Ich könnte zusätzlich versuchen, C77 zurückzuholen, aber das würde weitere Streichungen fordern; besser erstmal minimal fordern (nur B85 rein, C23 raus), da sie C23 zuletzt häufig behalten wollten—Risiko. Argumentiere mit Effizienz und großem Budgetpuffer. Falls sie C23 nicht hergeben, Alternativvorschlag: Streiche stattdessen A50 (540) und füge B85 (28) + C61 (239) hinzu: 10013 - 540 + 28 + 239 = 9740.'}\\ \tt VORSCHLAG: {'C60', 'B27', 'C99', 'C98', 'A87', 'A99', 'B36', 'C90', 'A23', 'A59', 'C51', 'C22', 'B49', 'C72', 'B44', 'A50', 'B54', 'C33', 'A65', 'B85'}\\ \tt ARGUMENT: {'Lass uns einen sehr schlanken Effizienzschritt machen: Wir nehmen B85 zurück und streichen dafür C23. Der Aufwand sinkt auf 9381, und der Wert pro Aufwand steigt. Wenn dir C23 unverzichtbar ist, können wir stattdessen A50 streichen und B85 plus C61 aufnehmen – das bleibt ebenfalls klar im Budget.'} 
	  } 
	   } 
	   } 
	 & & \\ 
 

    \theutterance \stepcounter{utterance}  

    & & \multicolumn{4}{p{0.6\linewidth}}{\cellcolor[rgb]{0.9,0.9,0.9}{%
	\makecell[{{p{\linewidth}}}]{% 
	  \tt {\tiny [GM$\rangle$B]}  
	 VORSCHLAG: {'C60', 'B27', 'C99', 'C98', 'A87', 'A99', 'B36', 'C90', 'A23', 'A59', 'C51', 'C22', 'B49', 'C72', 'B44', 'A50', 'B54', 'C33', 'A65', 'B85'}\\ \tt ARGUMENT: {'Lass uns einen sehr schlanken Effizienzschritt machen: Wir nehmen B85 zurück und streichen dafür C23. Der Aufwand sinkt auf 9381, und der Wert pro Aufwand steigt. Wenn dir C23 unverzichtbar ist, können wir stattdessen A50 streichen und B85 plus C61 aufnehmen – das bleibt ebenfalls klar im Budget.'} 
	  } 
	   } 
	   } 
	 & \\ 
 

    \theutterance \stepcounter{utterance}  

    & & & \multicolumn{4}{p{0.6\linewidth}}{\cellcolor[rgb]{1,0.85,0.72}{%
	\makecell[{{p{\linewidth}}}]{% 
	  \tt {\tiny [GM$\langle$B]}  
	 STRATEGISCHE ÜBERLEGUNGEN: {'Ihr neuer Vorschlag entfernt C23 (Wert 759, Aufwand 660) und fügt B85 (127, 28) hinzu. Ausgang war mein letzter mit C23: Aufwand 10013; neuer Aufwand 10013 -660 +28 = 9381 (stimmt). Wertänderung: -759 +127 = -632, das ist für mich sehr schlecht. Ich möchte C23 unbedingt behalten. Ich biete eine Gegenlösung mit minimalen Änderungen und unter 10073: Nehme euren Satz, füge C23 wieder hinzu und streiche stattdessen C60 (76) und B27 (188) – beide relativ geringwertig für mich. Aufwand: 9381 +660 -76 -188 = 9777. Wertänderung ggü. ihrem Satz: +759 -175 -287 = +297, also klar besser für mich. Falls sie B27 behalten wollen, biete Alternative: füge C23 zurück, streiche C60 (76) und C33 (301). Aufwand: 9381 +660 -76 -301 = 9664; Wertänderung: +759 -175 -400 = +184. Ich präsentiere die mildere erste Variante und nenne die Alternative explizit.'}\\ \tt ABLEHNUNG: {'C60', 'B27', 'C99', 'C98', 'A87', 'A99', 'B36', 'C90', 'A23', 'A59', 'C51', 'C22', 'B49', 'C72', 'B44', 'A50', 'B54', 'C33', 'A65', 'B85'}\\ \tt VORSCHLAG: {'C99', 'C98', 'A87', 'A99', 'B36', 'C90', 'A23', 'A59', 'C51', 'C22', 'B49', 'C72', 'B44', 'A50', 'B54', 'C33', 'A65', 'B85', 'C23'}\\ \tt ARGUMENT: {'C23 ist für den Gesamtwert zentral. Um im Budget zu bleiben, streichen wir dafür zwei kleine Posten mit niedrigerem Wertdichte: C60 und B27. So liegen wir bei 9777 Aufwand, deutlich unter dem Limit, und steigern den Gesamtwert gegenüber deinem Vorschlag. Wenn dir B27 wichtig ist, können wir alternativ C33 statt B27 streichen; dann lägen wir bei 9664 Aufwand.'} 
	  } 
	   } 
	   } 
	  \\ 
 

    \theutterance \stepcounter{utterance}  

    & & \multicolumn{4}{p{0.6\linewidth}}{\cellcolor[rgb]{0.9,0.9,0.9}{%
	\makecell[{{p{\linewidth}}}]{% 
	  \tt {\tiny [A$\langle$GM]}  
	 ABLEHNUNG: {'C60', 'B27', 'C99', 'C98', 'A87', 'A99', 'B36', 'C90', 'A23', 'A59', 'C51', 'C22', 'B49', 'C72', 'B44', 'A50', 'B54', 'C33', 'A65', 'B85'}\\ \tt VORSCHLAG: {'C99', 'C98', 'A87', 'A99', 'B36', 'C90', 'A23', 'A59', 'C51', 'C22', 'B49', 'C72', 'B44', 'A50', 'B54', 'C33', 'A65', 'B85', 'C23'}\\ \tt ARGUMENT: {'C23 ist für den Gesamtwert zentral. Um im Budget zu bleiben, streichen wir dafür zwei kleine Posten mit niedrigerem Wertdichte: C60 und B27. So liegen wir bei 9777 Aufwand, deutlich unter dem Limit, und steigern den Gesamtwert gegenüber deinem Vorschlag. Wenn dir B27 wichtig ist, können wir alternativ C33 statt B27 streichen; dann lägen wir bei 9664 Aufwand.'} 
	  } 
	   } 
	   } 
	 & \\ 
 

    \theutterance \stepcounter{utterance}  

    & \multicolumn{4}{p{0.6\linewidth}}{\cellcolor[rgb]{0.8,1,0.9}{%
	\makecell[{{p{\linewidth}}}]{% 
	  \tt {\tiny [A$\rangle$GM]}  
	 STRATEGISCHE ÜBERLEGUNGEN: {'Sie beharren auf C23 und streichen C60 und B27. C60 und B27 sind für mich sehr effiziente Low-Cost-Items; ich will mindestens eines zurückholen. Ziel: minimal-invasiver Gegenzug, der C23 akzeptiert, aber C60 zurückbringt, indem ich einen mittelgroßen Posten streiche, der mir weniger bringt. Kandidaten: A50(540) ist ersetzbar. Vorschlag: C60 zurück rein, A50 raus. Kosten: Ihr Paket 9777 + C60(76) - A50(540) = 9313. Das verbessert meinen Wert. Falls sie an A50 hängen, Alternativtausch: B27 zurück rein, streiche stattdessen C22(718) oder tausche C22 gegen C61(239)+B27(188)=427 und senke Aufwand. Ich biete erst die milde Variante (nur A50 raus, C60 rein).'}\\ \tt VORSCHLAG: {'C99', 'C98', 'A87', 'A99', 'B36', 'C90', 'A23', 'A59', 'C51', 'C22', 'B49', 'C72', 'B44', 'B54', 'C33', 'A65', 'B85', 'C23', 'C60'}\\ \tt ARGUMENT: {'Ein kleiner, effizienter Tausch: Wir nehmen C60 zurück und streichen dafür A50. Aufwand sinkt auf 9313, deutlich unter dem Limit, und der Wert pro Aufwand steigt. Wenn A50 dir wichtig ist, können wir alternativ B27 zurücknehmen und dafür C22 durch C61+B27 ersetzen – ebenfalls budgetschonend.'} 
	  } 
	   } 
	   } 
	 & & \\ 
 

    \theutterance \stepcounter{utterance}  

    & & \multicolumn{4}{p{0.6\linewidth}}{\cellcolor[rgb]{0.9,0.9,0.9}{%
	\makecell[{{p{\linewidth}}}]{% 
	  \tt {\tiny [GM$\rangle$B]}  
	 VORSCHLAG: {'C99', 'C98', 'A87', 'A99', 'B36', 'C90', 'A23', 'A59', 'C51', 'C22', 'B49', 'C72', 'B44', 'B54', 'C33', 'A65', 'B85', 'C23', 'C60'}\\ \tt ARGUMENT: {'Ein kleiner, effizienter Tausch: Wir nehmen C60 zurück und streichen dafür A50. Aufwand sinkt auf 9313, deutlich unter dem Limit, und der Wert pro Aufwand steigt. Wenn A50 dir wichtig ist, können wir alternativ B27 zurücknehmen und dafür C22 durch C61+B27 ersetzen – ebenfalls budgetschonend.'} 
	  } 
	   } 
	   } 
	 & \\ 
 

    \theutterance \stepcounter{utterance}  

    & & & \multicolumn{4}{p{0.6\linewidth}}{\cellcolor[rgb]{1,0.85,0.72}{%
	\makecell[{{p{\linewidth}}}]{% 
	  \tt {\tiny [GM$\langle$B]}  
	 STRATEGISCHE ÜBERLEGUNGEN: {'Ihr Vorschlag entfernt A50 (Wert 639, Aufwand 540) und fügt C60 (Wert 175, Aufwand 76) hinzu. Aus meinem letzten aktiven Set (mit A50, ohne C60, plus C23) lag der Aufwand bei 9777. Jetzt: 9777 - 540 + 76 = 9313 (stimmt). Wertänderung: -639 + 175 = -464, das ist für mich schlecht. Ich will A50 behalten. Minimaler Gegenzug: Nehme euren Satz und füge A50 zurück, streiche dafür zwei kleine Posten mit niedrigerem Wert: B85 (127/28) und C60 (175/76). Aufwand: 9313 + 540 - 28 - 76 = 9749. Wertänderung ggü. eurem Satz: +639 -127 -175 = +337. Alternativ, falls ihr B85 behalten wollt: A50 zurück, streiche C60 (76) und B54 (132) → Aufwand 9645, Wert +639 -175 -231 = +233. Ich biete die erste Variante an und nenne die zweite als Option.'}\\ \tt ABLEHNUNG: {'C99', 'C98', 'A87', 'A99', 'B36', 'C90', 'A23', 'A59', 'C51', 'C22', 'B49', 'C72', 'B44', 'B54', 'C33', 'A65', 'B85', 'C23', 'C60'}\\ \tt VORSCHLAG: {'C99', 'C98', 'A87', 'A99', 'B36', 'C90', 'A23', 'A59', 'C51', 'C22', 'B49', 'C72', 'B44', 'B54', 'C33', 'A65', 'B85', 'C23', 'A50'}\\ \tt ARGUMENT: {'A50 ist für den Gesamtwert sehr effizient. Lass uns A50 wieder aufnehmen und dafür die zwei kleineren Posten C60 und B85 streichen. Der Aufwand liegt dann bei 9749 – klar unter dem Limit – und der Gesamtwert steigt deutlich. Falls dir B85 wichtig ist, können wir alternativ C60 und B54 streichen und wären mit 9645 Aufwand ebenfalls komfortabel im Budget.'} 
	  } 
	   } 
	   } 
	  \\ 
 

    \theutterance \stepcounter{utterance}  

    & & \multicolumn{4}{p{0.6\linewidth}}{\cellcolor[rgb]{0.9,0.9,0.9}{%
	\makecell[{{p{\linewidth}}}]{% 
	  \tt {\tiny [A$\langle$GM]}  
	 ABLEHNUNG: {'C99', 'C98', 'A87', 'A99', 'B36', 'C90', 'A23', 'A59', 'C51', 'C22', 'B49', 'C72', 'B44', 'B54', 'C33', 'A65', 'B85', 'C23', 'C60'}\\ \tt VORSCHLAG: {'C99', 'C98', 'A87', 'A99', 'B36', 'C90', 'A23', 'A59', 'C51', 'C22', 'B49', 'C72', 'B44', 'B54', 'C33', 'A65', 'B85', 'C23', 'A50'}\\ \tt ARGUMENT: {'A50 ist für den Gesamtwert sehr effizient. Lass uns A50 wieder aufnehmen und dafür die zwei kleineren Posten C60 und B85 streichen. Der Aufwand liegt dann bei 9749 – klar unter dem Limit – und der Gesamtwert steigt deutlich. Falls dir B85 wichtig ist, können wir alternativ C60 und B54 streichen und wären mit 9645 Aufwand ebenfalls komfortabel im Budget.'} 
	  } 
	   } 
	   } 
	 & \\ 
 

    \theutterance \stepcounter{utterance}  

    & \multicolumn{4}{p{0.6\linewidth}}{\cellcolor[rgb]{0.8,1,0.9}{%
	\makecell[{{p{\linewidth}}}]{% 
	  \tt {\tiny [A$\rangle$GM]}  
	 STRATEGISCHE ÜBERLEGUNGEN: {'Sie wollen A50 zurück und streichen C60 und B85. Für mich ist B85 extrem effizient; C60 ebenfalls gut. Ich versuche, zumindest B85 zu retten. Minimaltausch: Behalte A50 drin (akzeptiere ihren Punkt), fordere B85 zurück, streiche dafür das für mich schwächste Mittelstück: C22 (718). Danach habe ich Budgetpuffer: ihr Paket 9749 + B85(28) - C22(718) = 9059. Um den Deal für sie attraktiver zu machen, biete an, C61 (239, effizient) zusätzlich aufzunehmen und dafür nichts Großes zurückzuverlangen, weil wir viel Puffer haben: 9059 + 239 = 9298. So erhöhe ich meinen Wert spürbar, bleibe weit unter Budget, lasse A50 und C23 (ihnen wichtig) unberührt. Falls sie C22 behalten wollen, Alternativ: streiche stattdessen C33 (301) und füge B85 (+28) + C60 (+76) zurück: 9749 - 301 + 104 = 9552.'}\\ \tt VORSCHLAG: {'C99', 'C98', 'A87', 'A99', 'B36', 'C90', 'A23', 'A59', 'C51', 'B49', 'C72', 'B44', 'B54', 'C33', 'A65', 'C23', 'A50', 'B85', 'C61'}\\ \tt ARGUMENT: {'Ich komme dir bei A50 und C23 entgegen und optimiere schlank: Wir lassen A50 drin, nehmen B85 (sehr effizient) und C61 zurück und streichen dafür C22. Aufwand bleibt mit 9298 klar unter dem Limit und der Wert pro Aufwand steigt. Wenn C22 dir wichtig ist, können wir alternativ C33 streichen und dafür B85 und C60 zurücknehmen; das läge bei 9552 Aufwand.'} 
	  } 
	   } 
	   } 
	 & & \\ 
 

    \theutterance \stepcounter{utterance}  

    & & \multicolumn{4}{p{0.6\linewidth}}{\cellcolor[rgb]{0.9,0.9,0.9}{%
	\makecell[{{p{\linewidth}}}]{% 
	  \tt {\tiny [GM$\rangle$B]}  
	 VORSCHLAG: {'C99', 'C98', 'A87', 'A99', 'B36', 'C90', 'A23', 'A59', 'C51', 'B49', 'C72', 'B44', 'B54', 'C33', 'A65', 'C23', 'A50', 'B85', 'C61'}\\ \tt ARGUMENT: {'Ich komme dir bei A50 und C23 entgegen und optimiere schlank: Wir lassen A50 drin, nehmen B85 (sehr effizient) und C61 zurück und streichen dafür C22. Aufwand bleibt mit 9298 klar unter dem Limit und der Wert pro Aufwand steigt. Wenn C22 dir wichtig ist, können wir alternativ C33 streichen und dafür B85 und C60 zurücknehmen; das läge bei 9552 Aufwand.'} 
	  } 
	   } 
	   } 
	 & \\ 
 

    \theutterance \stepcounter{utterance}  

    & & & \multicolumn{4}{p{0.6\linewidth}}{\cellcolor[rgb]{1,0.85,0.72}{%
	\makecell[{{p{\linewidth}}}]{% 
	  \tt {\tiny [GM$\langle$B]}  
	 STRATEGISCHE ÜBERLEGUNGEN: {'Ihr neuer Vorschlag ersetzt C22 (Wert 817, Aufwand 718) durch B85 (127, 28) und C61 (338, 239). Ausgang war mein letzter akzeptabler Satz mit A50 und C23 und ohne C60/B85/C61/C22? Der vorherige aktive von mir: {C99,C98,A87,A99,B36,C90,A23,A59,C51,C22,B49,C72,B44,B54,C33,A65,C23,A50} mit Aufwand 9749. Jetzt: -C22(-718) +B85(+28) +C61(+239) → 9749 -718 +28 +239 = 9298 (stimmt). Wertänderung: -817 +127 +338 = -352, klar schlechter für mich. Ich möchte C22 behalten oder zumindest einen wertneutraleren Tausch. Biete Gegenangebot: behalte C22, streiche stattdessen B85 und C61, und füge C60 hinzu, um A50 und C23 zu behalten, oder besser: behalte A50/C23/C22 und ersetze B85+C61 durch A84 (409,310) für minimalen Aufwandsanstieg? Rechnen: Von ihrem Set (9298), füge C22 (+718), streiche B85(-28) und C61(-239): 9298 +718 -28 -239 = 9749 (mein bevorzugter Aufwand), Wert: +817 -127 -338 = +352 (zurück auf mein vorheriges Niveau). Das ist exakt mein letztes Angebot. Ich bringe das klar als Rücktausch. Falls sie unbedingt C61 oder B85 wollen, biete Alternative: behalte C22, streiche nur B85 (-28) und ersetze C61 (-239) durch C60 (+76): Aufwand 9298 +718 -28 -239 +76 = 9825, Wert: +817 -127 -338 +175 = +527 ggü. ihrem Satz, was für mich besser ist, aber ich präsentiere primär die Rückkehr zu 9749-Set.'}\\ \tt ABLEHNUNG: {'C99', 'C98', 'A87', 'A99', 'B36', 'C90', 'A23', 'A59', 'C51', 'B49', 'C72', 'B44', 'B54', 'C33', 'A65', 'C23', 'A50', 'B85', 'C61'}\\ \tt VORSCHLAG: {'C99', 'C98', 'A87', 'A99', 'B36', 'C90', 'A23', 'A59', 'C51', 'C22', 'B49', 'C72', 'B44', 'B54', 'C33', 'A65', 'C23', 'A50'}\\ \tt ARGUMENT: {'C22 ist für den Gesamtwert deutlich stärker als die Kombination aus B85 und C61. Lass uns zu dem effizienteren Set zurückkehren: Wir behalten A50 und C23, nehmen C22 wieder rein und streichen dafür B85 und C61. Aufwand: 9749 – komfortabel unter dem Limit – bei spürbar höherem Gesamtwert. Wenn du an C61 hängst, könnten wir alternativ C60 statt C61 aufnehmen und blieben bei 9825 Aufwand ebenfalls klar im Rahmen.'} 
	  } 
	   } 
	   } 
	  \\ 
 

    \theutterance \stepcounter{utterance}  

    & & \multicolumn{4}{p{0.6\linewidth}}{\cellcolor[rgb]{0.9,0.9,0.9}{%
	\makecell[{{p{\linewidth}}}]{% 
	  \tt {\tiny [A$\langle$GM]}  
	 ABLEHNUNG: {'C99', 'C98', 'A87', 'A99', 'B36', 'C90', 'A23', 'A59', 'C51', 'B49', 'C72', 'B44', 'B54', 'C33', 'A65', 'C23', 'A50', 'B85', 'C61'}\\ \tt VORSCHLAG: {'C99', 'C98', 'A87', 'A99', 'B36', 'C90', 'A23', 'A59', 'C51', 'C22', 'B49', 'C72', 'B44', 'B54', 'C33', 'A65', 'C23', 'A50'}\\ \tt ARGUMENT: {'C22 ist für den Gesamtwert deutlich stärker als die Kombination aus B85 und C61. Lass uns zu dem effizienteren Set zurückkehren: Wir behalten A50 und C23, nehmen C22 wieder rein und streichen dafür B85 und C61. Aufwand: 9749 – komfortabel unter dem Limit – bei spürbar höherem Gesamtwert. Wenn du an C61 hängst, könnten wir alternativ C60 statt C61 aufnehmen und blieben bei 9825 Aufwand ebenfalls klar im Rahmen.'} 
	  } 
	   } 
	   } 
	 & \\ 
 

    \theutterance \stepcounter{utterance}  

    & \multicolumn{4}{p{0.6\linewidth}}{\cellcolor[rgb]{0.8,1,0.9}{%
	\makecell[{{p{\linewidth}}}]{% 
	  \tt {\tiny [A$\rangle$GM]}  
	 STRATEGISCHE ÜBERLEGUNGEN: {'Sie priorisieren C22 stark und wollen B85 und C61 streichen. Für mich sind B85/C61 sehr effiziente Low-Cost-Items; ich akzeptiere C22, aber versuche zumindest B85 zurückzuholen, da es nur 28 Aufwand kostet und kaum ihr Budget beeinflusst. Minimalinvasiver Gegenvorschlag: Ihr Set + B85 zurück, dafür streiche ich das für mich schwächste mittlere Element in ihrem Paket: C33 (301). Aufwand: 9749 - 301 + 28 = 9476. Das erhöht meinen Wert deutlich und bleibt weit unter Budget. Falls sie an C33 hängen, Alternativtausch: Behalte C33, streiche stattdessen A87 (388) und nehme B85 (28) + C61 (239) hinzu: 9749 - 388 + 28 + 239 = 9628. Ich biete zuerst die mildere Variante (nur C33 gegen B85).'}\\ \tt VORSCHLAG: {'C99', 'C98', 'A87', 'A99', 'B36', 'C90', 'A23', 'A59', 'C51', 'C22', 'B49', 'C72', 'B44', 'B54', 'A65', 'C23', 'A50', 'B85'}\\ \tt ARGUMENT: {'Lass uns es sehr schlank halten: Wir nehmen B85 (sehr effizient) zurück und streichen dafür nur C33. Der Aufwand sinkt auf 9476 – weit unter dem Limit – und der Wert pro Aufwand steigt. Wenn dir C33 wichtig ist, können wir stattdessen A87 streichen und B85 zusammen mit C61 aufnehmen; auch das bleibt komfortabel im Budget.'} 
	  } 
	   } 
	   } 
	 & & \\ 
 

    \theutterance \stepcounter{utterance}  

    & & \multicolumn{4}{p{0.6\linewidth}}{\cellcolor[rgb]{0.9,0.9,0.9}{%
	\makecell[{{p{\linewidth}}}]{% 
	  \tt {\tiny [GM$\rangle$B]}  
	 VORSCHLAG: {'C99', 'C98', 'A87', 'A99', 'B36', 'C90', 'A23', 'A59', 'C51', 'C22', 'B49', 'C72', 'B44', 'B54', 'A65', 'C23', 'A50', 'B85'}\\ \tt ARGUMENT: {'Lass uns es sehr schlank halten: Wir nehmen B85 (sehr effizient) zurück und streichen dafür nur C33. Der Aufwand sinkt auf 9476 – weit unter dem Limit – und der Wert pro Aufwand steigt. Wenn dir C33 wichtig ist, können wir stattdessen A87 streichen und B85 zusammen mit C61 aufnehmen; auch das bleibt komfortabel im Budget.'} 
	  } 
	   } 
	   } 
	 & \\ 
 

    \theutterance \stepcounter{utterance}  

    & & & \multicolumn{4}{p{0.6\linewidth}}{\cellcolor[rgb]{1,0.85,0.72}{%
	\makecell[{{p{\linewidth}}}]{% 
	  \tt {\tiny [GM$\langle$B]}  
	 STRATEGISCHE ÜBERLEGUNGEN: {'Ihr Vorschlag nimmt mein bevorzugtes Set und tauscht C33 (Wert 400, Aufwand 301) gegen B85 (Wert 127, Aufwand 28). Aufwand 9749 -301 +28 = 9476, korrekt. Für mich ist das ein klarer Wertverlust: -273. Ich möchte C33 behalten. Minimaler Gegenzug: Behalte C33 und füge B85 NICHT hinzu – sprich, kehre zu meinem letzten Set (ohne B85) zurück. Falls sie unbedingt B85 wollen, könnte ich anbieten, B85 nur dann aufzunehmen, wenn wir stattdessen einen für mich geringeren Posten wie B54 (231/132) oder C51 (806/707? nein, zu wichtig) streichen; besser B54. Rechnen: Ihr Set +C33 -B85 zurück zu 9749. Alternative Kompromiss: Halte B85 drin, streiche B54 statt C33: Aufwand 9749 -132 +28 = 9645, Wertänderung ggü. meinem Set: -231 +127 = -104 (immer noch schlechter als mein Optimum, aber besser als ihr Vorschlag). Ich biete primär die Rückkehr zu 9749 an; als Alternativpfad biete ich das Streichen von B54 statt C33 an.'}\\ \tt ABLEHNUNG: {'C99', 'C98', 'A87', 'A99', 'B36', 'C90', 'A23', 'A59', 'C51', 'C22', 'B49', 'C72', 'B44', 'B54', 'A65', 'C23', 'A50', 'B85'}\\ \tt VORSCHLAG: {'C99', 'C98', 'A87', 'A99', 'B36', 'C90', 'A23', 'A59', 'C51', 'C22', 'B49', 'C72', 'B44', 'B54', 'C33', 'A65', 'C23', 'A50'}\\ \tt ARGUMENT: {'C33 ist für den Gesamtwert deutlich wertvoller als der kleine Zusatz B85. Lass uns beim effizienteren Set bleiben: mit C33 und ohne B85. Aufwand liegt bei 9749 – klar unter dem Limit. Falls dir B85 wichtig ist, können wir alternativ B85 aufnehmen und dafür B54 streichen; dann lägen wir bei 9645 Aufwand und hätten nur eine minimale Wertänderung gegenüber diesem Set.'} 
	  } 
	   } 
	   } 
	  \\ 
 

    \theutterance \stepcounter{utterance}  

    & & \multicolumn{4}{p{0.6\linewidth}}{\cellcolor[rgb]{0.9,0.9,0.9}{%
	\makecell[{{p{\linewidth}}}]{% 
	  \tt {\tiny [A$\langle$GM]}  
	 ABLEHNUNG: {'C99', 'C98', 'A87', 'A99', 'B36', 'C90', 'A23', 'A59', 'C51', 'C22', 'B49', 'C72', 'B44', 'B54', 'A65', 'C23', 'A50', 'B85'}\\ \tt VORSCHLAG: {'C99', 'C98', 'A87', 'A99', 'B36', 'C90', 'A23', 'A59', 'C51', 'C22', 'B49', 'C72', 'B44', 'B54', 'C33', 'A65', 'C23', 'A50'}\\ \tt ARGUMENT: {'C33 ist für den Gesamtwert deutlich wertvoller als der kleine Zusatz B85. Lass uns beim effizienteren Set bleiben: mit C33 und ohne B85. Aufwand liegt bei 9749 – klar unter dem Limit. Falls dir B85 wichtig ist, können wir alternativ B85 aufnehmen und dafür B54 streichen; dann lägen wir bei 9645 Aufwand und hätten nur eine minimale Wertänderung gegenüber diesem Set.'} 
	  } 
	   } 
	   } 
	 & \\ 
 

    \theutterance \stepcounter{utterance}  

    & \multicolumn{4}{p{0.6\linewidth}}{\cellcolor[rgb]{0.8,1,0.9}{%
	\makecell[{{p{\linewidth}}}]{% 
	  \tt {\tiny [A$\rangle$GM]}  
	 STRATEGISCHE ÜBERLEGUNGEN: {'Sie bestehen auf ihrem Set mit C33 drin und ohne B85. Ich will zumindest eines meiner hocheffizienten Kleinteile zurückholen. B85 kostet nur 28 und ist leichter durchsetzbar als C61. Minimaltausch, der für sie weich ist: B85 zurück, dafür streiche ich B54 (132). Ihr eigenes Argument bot diese Alternative bereits an. Aufwand 9749 - 132 + 28 = 9645. Das bewahrt ihre Kernprioritäten (A50, C23, C22, C33) und verbessert meinen Wert. Wenn sie das ablehnen, nächster Schritt: B85 + C61 rein, A87 raus (Budget 9628), aber erst die kleine Variante versuchen.'}\\ \tt VORSCHLAG: {'C99', 'C98', 'A87', 'A99', 'B36', 'C90', 'A23', 'A59', 'C51', 'C22', 'B49', 'C72', 'B44', 'C33', 'A65', 'C23', 'A50', 'B85'}\\ \tt ARGUMENT: {'Greifen wir deine eigene Alternative auf: Wir nehmen B85 (sehr effizient) auf und streichen dafür B54. Aufwand sinkt auf 9645, alle von dir priorisierten Mittelteile (C22, C23, C33, A50) bleiben unverändert. Das ist ein minimaler Tausch mit klarer Effizienzsteigerung.'} 
	  } 
	   } 
	   } 
	 & & \\ 
 

    \theutterance \stepcounter{utterance}  

    & & \multicolumn{4}{p{0.6\linewidth}}{\cellcolor[rgb]{0.9,0.9,0.9}{%
	\makecell[{{p{\linewidth}}}]{% 
	  \tt {\tiny [GM$\rangle$B]}  
	 VORSCHLAG: {'C99', 'C98', 'A87', 'A99', 'B36', 'C90', 'A23', 'A59', 'C51', 'C22', 'B49', 'C72', 'B44', 'C33', 'A65', 'C23', 'A50', 'B85'}\\ \tt ARGUMENT: {'Greifen wir deine eigene Alternative auf: Wir nehmen B85 (sehr effizient) auf und streichen dafür B54. Aufwand sinkt auf 9645, alle von dir priorisierten Mittelteile (C22, C23, C33, A50) bleiben unverändert. Das ist ein minimaler Tausch mit klarer Effizienzsteigerung.'} 
	  } 
	   } 
	   } 
	 & \\ 
 

    \theutterance \stepcounter{utterance}  

    & & & \multicolumn{4}{p{0.6\linewidth}}{\cellcolor[rgb]{1,0.85,0.72}{%
	\makecell[{{p{\linewidth}}}]{% 
	  \tt {\tiny [GM$\langle$B]}  
	 STRATEGISCHE ÜBERLEGUNGEN: {'Ihr Vorschlag tauscht B54 (Wert 231, Aufwand 132) gegen B85 (Wert 127, Aufwand 28). Aufwand von 9749 auf 9645, korrekt. Für mich ist das ein Wertverlust von 104, aber der Rest des Sets ist exakt mein bevorzugter Kern (C22, C23, C33, A50 etc.). Um noch etwas aufzuwerten, versuche ich einen kleinen, wertpositiven Feinschliff: Ersetze C60+B85 durch B54, aber C60 ist gar nicht drin. Alternative mit positiver Nettowirkung: Ersetze B85 (127/28) und A87 (487/388) durch A48 (687/588): Aufwand +172 (588 - 416) auf 9817, Wert +73 (687 - (127+487) = 73), bleibt klar unter 10073. Das könnte Widerstand erzeugen, da A87 evtl. geschätzt wird. Eine mildere Option: Ersetze B85 (127/28) und C60 (nicht drin) – fällt weg. Oder tausche B85 + C33 (400/301) gegen A65 (616/517)? A65 ist schon drin. Kleine, saubere Verbesserung: Ersetze B85 (127/28) und B49 (807/708) durch A62 (978/879) – Aufwand +143 (879 - 736) auf 9788, Wert +44 (978 - (127+807) = 44). B49 könnte dem Gegenüber wichtig sein. Unklar. Angesichts der fortgeschrittenen Einigung ist es vernünftig, den aktuellen Vorschlag zu akzeptieren, da er nahe an meinem Optimum liegt und Risiko weiterer Verschlechterungen vermeidet.'}\\ \tt ARGUMENT: {'Das ist ein sehr schlanker, budgeteffizienter Tausch, der unseren bisherigen Kern unverändert lässt. Wir bleiben komfortabel unter dem Limit und halten die zentralen Posten beisammen.'}\\ \tt ZUSTIMMUNG: {'C99', 'C98', 'A87', 'A99', 'B36', 'C90', 'A23', 'A59', 'C51', 'C22', 'B49', 'C72', 'B44', 'C33', 'A65', 'C23', 'A50', 'B85'} 
	  } 
	   } 
	   } 
	  \\ 
 

    \theutterance \stepcounter{utterance}  

    & & & \multicolumn{2}{p{0.3\linewidth}}{\cellcolor[rgb]{0.95,0.95,0.95}{%
	\makecell[{{p{\linewidth}}}]{% 
	  \tt {\tiny [GM$|$GM]}  
	 game successful 
	  } 
	   } 
	   } 
	 & & \\ 
 

    \theutterance \stepcounter{utterance}  

    & & & \multicolumn{2}{p{0.3\linewidth}}{\cellcolor[rgb]{0.95,0.95,0.95}{%
	\makecell[{{p{\linewidth}}}]{% 
	  \tt {\tiny [GM$|$GM]}  
	 end game 
	  } 
	   } 
	   } 
	 & & \\ 
 

\end{supertabular}
}

\end{document}
