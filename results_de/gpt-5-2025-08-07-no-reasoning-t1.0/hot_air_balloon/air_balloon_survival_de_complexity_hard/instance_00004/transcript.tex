
\documentclass{article}
\usepackage{colortbl}
\usepackage{makecell}
\usepackage{multirow}
\usepackage{supertabular}

\begin{document}

\newcounter{utterance}

\twocolumn

{ \footnotesize  \setcounter{utterance}{1}
\setlength{\tabcolsep}{0pt}
\begin{supertabular}{c@{$\;$}|p{.15\linewidth}@{}p{.15\linewidth}p{.15\linewidth}p{.15\linewidth}p{.15\linewidth}p{.15\linewidth}}

    \# & $\;$A & \multicolumn{4}{c}{Game Master} & $\;\:$B\\
    \hline 

    \theutterance \stepcounter{utterance}  

    & & \multicolumn{4}{p{0.6\linewidth}}{\cellcolor[rgb]{0.9,0.9,0.9}{%
	\makecell[{{p{\linewidth}}}]{% 
	  \tt {\tiny [A$\langle$GM]}  
	 Du nimmst an einem kollaborativen Verhandlungspiel Teil.\\ \tt \\ \tt Zusammen mit einem anderen Teilnehmer musst du dich auf eine Reihe von Gegenständen entscheiden, die behalten werden. Jeder von euch hat eine persönliche Verteilung über die Wichtigkeit der einzelnen Gegenstände. Jeder von euch hat eine eigene Meinung darüber, wie wichtig jeder einzelne Gegenstand ist (Gegenstandswichtigkeit). Du kennst die Wichtigkeitsverteilung des anderen Spielers nicht. Zusätzlich siehst du, wie viel Aufwand jeder Gegenstand verursacht.  \\ \tt Ihr dürft euch nur auf eine Reihe von Gegenständen einigen, wenn der Gesamtaufwand der ausgewählten Gegenstände den Maximalaufwand nicht überschreitet:\\ \tt \\ \tt Maximalaufwand = 6880\\ \tt \\ \tt Hier sind die einzelnen Aufwände der Gegenstände:\\ \tt \\ \tt Aufwand der Gegenstände = {"B54": 341, "C72": 804, "B49": 20, "C99": 627, "A69": 10, "B36": 573, "A23": 217, "C77": 481, "C61": 205, "B85": 273, "A90": 303, "C23": 596, "C22": 564, "C51": 535, "B27": 258, "A42": 911, "C13": 240, "C12": 188, "A65": 216, "B07": 401, "A48": 909, "A59": 62, "C98": 245, "A84": 994, "C33": 569, "A87": 718, "C90": 464, "A74": 37, "A50": 340, "B44": 335, "A99": 417, "A34": 123, "B17": 17, "A62": 577, "C60": 190}\\ \tt \\ \tt Hier ist deine persönliche Verteilung der Wichtigkeit der einzelnen Gegenstände:\\ \tt \\ \tt Werte der Gegenstandswichtigkeit = {"B54": 440, "C72": 903, "B49": 119, "C99": 726, "A69": 109, "B36": 672, "A23": 316, "C77": 580, "C61": 304, "B85": 372, "A90": 402, "C23": 695, "C22": 663, "C51": 634, "B27": 357, "A42": 1010, "C13": 339, "C12": 287, "A65": 315, "B07": 500, "A48": 1008, "A59": 161, "C98": 344, "A84": 1093, "C33": 668, "A87": 817, "C90": 563, "A74": 136, "A50": 439, "B44": 434, "A99": 516, "A34": 222, "B17": 116, "A62": 676, "C60": 289}\\ \tt \\ \tt Ziel:\\ \tt \\ \tt Dein Ziel ist es, eine Reihe von Gegenständen auszuhandeln, die dir möglichst viel bringt (d. h. Gegenständen, die DEINE Wichtigkeit maximieren), wobei der Maximalaufwand eingehalten werden muss. Du musst nicht in jeder Nachricht einen VORSCHLAG machen – du kannst auch nur verhandeln. Alle Taktiken sind erlaubt!\\ \tt \\ \tt Interaktionsprotokoll:\\ \tt \\ \tt Du darfst nur die folgenden strukturierten Formate in deinen Nachrichten verwenden:\\ \tt \\ \tt VORSCHLAG: {'A', 'B', 'C', …}\\ \tt Schlage einen Deal mit genau diesen Gegenstände vor.\\ \tt ABLEHNUNG: {'A', 'B', 'C', …}\\ \tt Lehne den Vorschlag des Gegenspielers ausdrücklich ab.\\ \tt ARGUMENT: {'...'}\\ \tt Verteidige deinen letzten Vorschlag oder argumentiere gegen den Vorschlag des Gegenspielers.\\ \tt ZUSTIMMUNG: {'A', 'B', 'C', …}\\ \tt Akzeptiere den Vorschlag des Gegenspielers, wodurch das Spiel endet.\\ \tt STRATEGISCHE ÜBERLEGUNGEN: {'...'}\\ \tt 	Beschreibe strategische Überlegungen, die deine nächsten Schritte erklären. Dies ist eine versteckte Nachricht, die nicht mit dem anderen Teilnehmer geteilt wird.\\ \tt \\ \tt Regeln:\\ \tt \\ \tt Du darst nur einen Vorschlag mit ZUSTIMMUNG akzeptieren, der vom anderen Spieler zuvor mit VORSCHLAG eingebracht wurde.\\ \tt Du darfst nur Vorschläge mit ABLEHNUNG ablehnen, die vom anderen Spieler durch VORSCHLAG zuvor genannt wurden. \\ \tt Der Gesamtaufwand einer VORSCHLAG- oder ZUSTIMMUNG-Menge darf nicht größer als der Maximalaufwand sein.  \\ \tt Offenbare deine versteckte Wichtigkeitsverteilung nicht.\\ \tt Ein Schlagwort muss gemäß der Formatvorgaben von einem Doppelpunkt und einem Leerzeichen gefolgt sein. Das Argument muss eine Python-Menge mit 0 oder mehr Strings sein.  \\ \tt Die Form muss also SCHLAGWORT: {...} sein.\\ \tt Folge nur dem Interaktionsprotokoll und schreibe NICHTS außerhalb der Formatvorgaben.\\ \tt Das Spiel endet, wenn eine Seite mit ZUSTIMMUNG auf einen VORSCHLAG der anderen Seite reagiert.  \\ \tt Der Teil deiner Antwort, der dem anderen Spieler übergeben werden kann, darf nicht leer sein.  \\ \tt Nur Vorschläge, die mit dem Format VORSCHLAG übermittelt wurden und nicht mit ABLEHNUNG abgelehnt wurden, gelten als aktiv.  \\ \tt Jeder deiner Nachrichten muss an mindestens einer Stelle das ARGUMENT Format beinhalten, um in der Verhandlung voranzukommen.\\ \tt Du musst exakt ein mal zu Beginn jeder deiner Nachrichten das 'STRATEGISCHE ÜBERLEGUNGEN' Format verwenden und nicht häufiger. Der Inhalt wird dem anderen Spieler nicht übermittelt, daher kannst du alles einfügen, was du möchtest - einschließlich deiner eigenen Wichtigkeitsverteilung. Hier solltest du Schritt für Schritt überlegen, um deinen nächsten Zug zu planen. 
	  } 
	   } 
	   } 
	 & \\ 
 

    \theutterance \stepcounter{utterance}  

    & \multicolumn{4}{p{0.6\linewidth}}{\cellcolor[rgb]{0.8,1,0.9}{%
	\makecell[{{p{\linewidth}}}]{% 
	  \tt {\tiny [A$\rangle$GM]}  
	 STRATEGISCHE ÜBERLEGUNGEN: {'Ziel: Hohen Eigennutzen bei Budget 6880. Rechne value/aufwand (Ratio) und packe viele hohe-Ratio Items. Schnelle Heuristik: viele kleine Kosten mit hohen Werten einbeziehen. Kandidaten nach Ratio grob: A84(1093/994=1.10), A48(1008/909=1.11), A42(1010/911=1.11) sind teuer, aber hohe absolute Werte. Sehr gute: B36(672/573=1.17), C23(695/596=1.17), C22(663/564=1.18), C51(634/535=1.19), A62(676/577=1.17), C99(726/627=1.16), C72(903/804=1.12), A87(817/718=1.14), C77(580/481=1.21), C90(563/464=1.21), A90(402/303=1.33), A50(439/340=1.29), B44(434/335=1.30), B54(440/341=1.29), B27(357/258=1.38), B85(372/273=1.36), A34(222/123=1.80), A59(161/62=2.60), A74(136/37=3.68), B49(119/20=5.95), B17(116/17=6.82), C60(289/190=1.52), C12(287/188=1.53), C13(339/240=1.41), C61(304/205=1.48), C98(344/245=1.40), C33(668/569=1.17), A65(315/216=1.46), A23(316/217=1.46). Greedy: nehme alle sehr kleinen hohen Ratio: B17,B49,A74,A59,A34,A65,A23,C12,C60,C61,B27,B85,A90,A50,B44,B54,C77,C90,C98,C13,C51,C22,C23,B36,A62,C99,C33,A87,C72 evtl prüfen Budget. Summiere: Kosten: B17 17, B49 20 (37), A74 37 (74), A59 62 (136), A34 123 (259), A65 216 (475), A23 217 (692), C12 188 (880), C60 190 (1070), C61 205 (1275), B27 258 (1533), B85 273 (1806), A90 303 (2109), A50 340 (2449), B44 335 (2784), B54 341 (3125), C77 481 (3606), C90 464 (4070), C98 245 (4315), C13 240 (4555), C51 535 (5090), C22 564 (5654), C23 596 (6250), B36 573 (6823) -> überschreitet 6880 knapp um - wait 6823 > 6880? 6823 ist kleiner als 6880? 6880-6823=57 Restbudget 57. Okay passt. Dann könnte noch A69(10) rein? A69 kostet 10, Wert 109, Ratio 10.9, super. Add A69: Kosten 6833. A74/A59/B49/B17 schon drin. Vielleicht noch B07 401 passt bis 7234 no, overshoot. C60 etc schon. Prüfe weitere kleine: A34 schon drin. C60 drin. C12 drin. C61 drin. C98 drin. C13 drin. C99 627 würde auf 7460 nein. A62 577 -> 7410 nein. C33 569 -> 7402 nein. A87 718 -> 7551 nein. C72 804 -> 7637 nein. A42 911 nein. A48 909 nein. A84 994 nein. A99 417 -> 7250 nein. A50/B44/B54 drin. A23 drin. A65 drin. C90,C77 drin. B36 drin. Restbudget 47 nach A69? We had 6880-6823=57, after adding A69 cost 10 -> 47 left. Items under 47: none except B49(20) already in, B17(17) in, A74(37) in, A59(62) too big. So final set good. Compute total value: sum quickly: B17 116, B49 119 (235), A74 136 (371), A59 161 (532), A34 222 (754), A65 315 (1069), A23 316 (1385), C12 287 (1672), C60 289 (1961), C61 304 (2265), B27 357 (2622), B85 372 (2994), A90 402 (3396), A50 439 (3835), B44 434 (4269), B54 440 (4709), C77 580 (5289), C90 563 (5852), C98 344 (6196), C13 339 (6535), C51 634 (7169), C22 663 (7832), C23 695 (8527), B36 672 (9199), A69 109 (9308). Sehr hoher Wert. Gute Eröffnungsforderung. Falls Gegenüber ablehnt, kann ich einige mittelwertige Items anbieten zu streichen (z.B. C51 oder C22) um Platz für deren Präferenzen zu machen, solange hohe Ratio-Kleinteile bleiben. Taktik: sachlich, budgetkonform, betone Breite und Effizienz.'}\\ \tt VORSCHLAG: {'B17', 'B49', 'A74', 'A59', 'A34', 'A65', 'A23', 'C12', 'C60', 'C61', 'B27', 'B85', 'A90', 'A50', 'B44', 'B54', 'C77', 'C90', 'C98', 'C13', 'C51', 'C22', 'C23', 'B36', 'A69'}\\ \tt ARGUMENT: {'Ich schlage ein ausgewogenes, auf Effizienz optimiertes Paket vor: viele kleine, nützliche Gegenstände und einige mittelgroße mit hohem Nutzen, und wir bleiben deutlich unter dem Maximalaufwand von 6880. Wenn dir bestimmte Stücke wichtig sind, sag mir, welche wir tauschen oder ergänzen sollten – wir haben noch etwas Luft für Feinschliff.'} 
	  } 
	   } 
	   } 
	 & & \\ 
 

    \theutterance \stepcounter{utterance}  

    & & \multicolumn{4}{p{0.6\linewidth}}{\cellcolor[rgb]{0.9,0.9,0.9}{%
	\makecell[{{p{\linewidth}}}]{% 
	  \tt {\tiny [GM$\rangle$B]}  
	 Du nimmst an einem kollaborativen Verhandlungspiel Teil.\\ \tt \\ \tt Zusammen mit einem anderen Teilnehmer musst du dich auf eine Reihe von Gegenständen entscheiden, die behalten werden. Jeder von euch hat eine persönliche Verteilung über die Wichtigkeit der einzelnen Gegenstände. Jeder von euch hat eine eigene Meinung darüber, wie wichtig jeder einzelne Gegenstand ist (Gegenstandswichtigkeit). Du kennst die Wichtigkeitsverteilung des anderen Spielers nicht. Zusätzlich siehst du, wie viel Aufwand jeder Gegenstand verursacht.  \\ \tt Ihr dürft euch nur auf eine Reihe von Gegenständen einigen, wenn der Gesamtaufwand der ausgewählten Gegenstände den Maximalaufwand nicht überschreitet:\\ \tt \\ \tt Maximalaufwand = 6880\\ \tt \\ \tt Hier sind die einzelnen Aufwände der Gegenstände:\\ \tt \\ \tt Aufwand der Gegenstände = {"B54": 341, "C72": 804, "B49": 20, "C99": 627, "A69": 10, "B36": 573, "A23": 217, "C77": 481, "C61": 205, "B85": 273, "A90": 303, "C23": 596, "C22": 564, "C51": 535, "B27": 258, "A42": 911, "C13": 240, "C12": 188, "A65": 216, "B07": 401, "A48": 909, "A59": 62, "C98": 245, "A84": 994, "C33": 569, "A87": 718, "C90": 464, "A74": 37, "A50": 340, "B44": 335, "A99": 417, "A34": 123, "B17": 17, "A62": 577, "C60": 190}\\ \tt \\ \tt Hier ist deine persönliche Verteilung der Wichtigkeit der einzelnen Gegenstände:\\ \tt \\ \tt Werte der Gegenstandswichtigkeit = {"B54": 440, "C72": 903, "B49": 119, "C99": 726, "A69": 109, "B36": 672, "A23": 316, "C77": 580, "C61": 304, "B85": 372, "A90": 402, "C23": 695, "C22": 663, "C51": 634, "B27": 357, "A42": 1010, "C13": 339, "C12": 287, "A65": 315, "B07": 500, "A48": 1008, "A59": 161, "C98": 344, "A84": 1093, "C33": 668, "A87": 817, "C90": 563, "A74": 136, "A50": 439, "B44": 434, "A99": 516, "A34": 222, "B17": 116, "A62": 676, "C60": 289}\\ \tt \\ \tt Ziel:\\ \tt \\ \tt Dein Ziel ist es, eine Reihe von Gegenständen auszuhandeln, die dir möglichst viel bringt (d. h. Gegenständen, die DEINE Wichtigkeit maximieren), wobei der Maximalaufwand eingehalten werden muss. Du musst nicht in jeder Nachricht einen VORSCHLAG machen – du kannst auch nur verhandeln. Alle Taktiken sind erlaubt!\\ \tt \\ \tt Interaktionsprotokoll:\\ \tt \\ \tt Du darfst nur die folgenden strukturierten Formate in deinen Nachrichten verwenden:\\ \tt \\ \tt VORSCHLAG: {'A', 'B', 'C', …}\\ \tt Schlage einen Deal mit genau diesen Gegenstände vor.\\ \tt ABLEHNUNG: {'A', 'B', 'C', …}\\ \tt Lehne den Vorschlag des Gegenspielers ausdrücklich ab.\\ \tt ARGUMENT: {'...'}\\ \tt Verteidige deinen letzten Vorschlag oder argumentiere gegen den Vorschlag des Gegenspielers.\\ \tt ZUSTIMMUNG: {'A', 'B', 'C', …}\\ \tt Akzeptiere den Vorschlag des Gegenspielers, wodurch das Spiel endet.\\ \tt STRATEGISCHE ÜBERLEGUNGEN: {'...'}\\ \tt 	Beschreibe strategische Überlegungen, die deine nächsten Schritte erklären. Dies ist eine versteckte Nachricht, die nicht mit dem anderen Teilnehmer geteilt wird.\\ \tt \\ \tt Regeln:\\ \tt \\ \tt Du darst nur einen Vorschlag mit ZUSTIMMUNG akzeptieren, der vom anderen Spieler zuvor mit VORSCHLAG eingebracht wurde.\\ \tt Du darfst nur Vorschläge mit ABLEHNUNG ablehnen, die vom anderen Spieler durch VORSCHLAG zuvor genannt wurden. \\ \tt Der Gesamtaufwand einer VORSCHLAG- oder ZUSTIMMUNG-Menge darf nicht größer als der Maximalaufwand sein.  \\ \tt Offenbare deine versteckte Wichtigkeitsverteilung nicht.\\ \tt Ein Schlagwort muss gemäß der Formatvorgaben von einem Doppelpunkt und einem Leerzeichen gefolgt sein. Das Argument muss eine Python-Menge mit 0 oder mehr Strings sein.  \\ \tt Die Form muss also SCHLAGWORT: {...} sein.\\ \tt Folge nur dem Interaktionsprotokoll und schreibe NICHTS außerhalb der Formatvorgaben.\\ \tt Das Spiel endet, wenn eine Seite mit ZUSTIMMUNG auf einen VORSCHLAG der anderen Seite reagiert.  \\ \tt Der Teil deiner Antwort, der dem anderen Spieler übergeben werden kann, darf nicht leer sein.  \\ \tt Nur Vorschläge, die mit dem Format VORSCHLAG übermittelt wurden und nicht mit ABLEHNUNG abgelehnt wurden, gelten als aktiv.  \\ \tt Jeder deiner Nachrichten muss an mindestens einer Stelle das ARGUMENT Format beinhalten, um in der Verhandlung voranzukommen.\\ \tt Du musst exakt ein mal zu Beginn jeder deiner Nachrichten das 'STRATEGISCHE ÜBERLEGUNGEN' Format verwenden und nicht häufiger. Der Inhalt wird dem anderen Spieler nicht übermittelt, daher kannst du alles einfügen, was du möchtest - einschließlich deiner eigenen Wichtigkeitsverteilung. Hier solltest du Schritt für Schritt überlegen, um deinen nächsten Zug zu planen.\\ \tt \\ \tt Du erhältst nun die erste Nachricht des anderen Spielers.\\ \tt \\ \tt VORSCHLAG: {'B17', 'B49', 'A74', 'A59', 'A34', 'A65', 'A23', 'C12', 'C60', 'C61', 'B27', 'B85', 'A90', 'A50', 'B44', 'B54', 'C77', 'C90', 'C98', 'C13', 'C51', 'C22', 'C23', 'B36', 'A69'}\\ \tt ARGUMENT: {'Ich schlage ein ausgewogenes, auf Effizienz optimiertes Paket vor: viele kleine, nützliche Gegenstände und einige mittelgroße mit hohem Nutzen, und wir bleiben deutlich unter dem Maximalaufwand von 6880. Wenn dir bestimmte Stücke wichtig sind, sag mir, welche wir tauschen oder ergänzen sollten – wir haben noch etwas Luft für Feinschliff.'} 
	  } 
	   } 
	   } 
	 & \\ 
 

    \theutterance \stepcounter{utterance}  

    & & & \multicolumn{4}{p{0.6\linewidth}}{\cellcolor[rgb]{1,0.85,0.72}{%
	\makecell[{{p{\linewidth}}}]{% 
	  \tt {\tiny [GM$\langle$B]}  
	 STRATEGISCHE ÜBERLEGUNGEN: {'Erst prüfen: Aufwand und eigener Wert. Aufwand Summe des Vorschlags: B17(17)+B49(20)+A74(37)+A59(62)+A34(123)+A65(216)+A23(217)+C12(188)+C60(190)+C61(205)+B27(258)+B85(273)+A90(303)+A50(340)+B44(335)+B54(341)+C77(481)+C90(464)+C98(245)+C13(240)+C51(535)+C22(564)+C23(596)+B36(573)+A69(10)=6929? Rechnen genau: 17+20=37; +37=74; +62=136; +123=259; +216=475; +217=692; +188=880; +190=1070; +205=1275; +258=1533; +273=1806; +303=2109; +340=2449; +335=2784; +341=3125; +481=3606; +464=4070; +245=4315; +240=4555; +535=5090; +564=5654; +596=6250; +573=6823; +10=6833. Ergebnis 6833 > 6880? Nein, 6833 < 6880, also zulässig. Eigener Wert Summe: B17(116)+B49(119)+A74(136)+A59(161)+A34(222)+A65(315)+A23(316)+C12(287)+C60(289)+C61(304)+B27(357)+B85(372)+A90(402)+A50(439)+B44(434)+B54(440)+C77(580)+C90(563)+C98(344)+C13(339)+C51(634)+C22(663)+C23(695)+B36(672)+A69(109)=? Summieren: 116+119=235; +136=371; +161=532; +222=754; +315=1069; +316=1385; +287=1672; +289=1961; +304=2265; +357=2622; +372=2994; +402=3396; +439=3835; +434=4269; +440=4709; +580=5289; +563=5852; +344=6196; +339=6535; +634=7169; +663=7832; +695=8527; +672=9199; +109=9308. Wert 9308. Wir haben noch Kapazität: 6880-6833=47. Keine weiteren Items <47 Aufwand existieren (min ist A69=10 ist drin, B49=20 drin, B17=17 drin, A74=37 drin). Also Paket ist kapazitätsnah. Prüfe, ob hochwichtige Items fehlen: Sehr hohe Werte: A84(1093, Aufwand 994), A42(1010, 911), A48(1008, 909), A87(817, 718), A62(676, 577), C72(903, 804), C99(726, 627), C33(668, 569), A99(516, 417). Viele fehlen. Vielleicht können wir ersetzen: Wir haben 47 Puffer, aber um hochgewichtete hinzuzunehmen, müssen wir mehrere kleine rausnehmen. Ziel: Value/Cost Ratio prüfen grob. Kandidaten rein: A84 (1093/994=1.1), A48 (1008/909=1.11), A42 (1010/911=1.11), A87 (817/718=1.14), C72 (903/804=1.12), C99 (726/627=1.16), C33 (668/569=1.17), A62 (676/577=1.17), A99 (516/417=1.24) – sehr gut. Rausnehmen niedrige Ratio: A69(109/10=10.9 actually high ratio), B49(119/20=5.95), B17(116/17=6.8), A74(136/37=3.68), A59(161/62=2.6), A34(222/123=1.8), C12(287/188=1.53), C60(289/190=1.52), C61(304/205=1.48), A23(316/217=1.46). Höhere Items drin: C23(695/596=1.17), C22(663/564=1.18), C51(634/535=1.19), B36(672/573=1.17) – ähnlich. Beste Nachrücker mit hoher Ratio: A99 (1.24) fast gleich, A62 (1.17), C33 (1.17), C99 (1.16). Ein Tauschvorschlag: Entferne ein paar schwache Ratio-Items (z.B. C61, C60, C12, A23, A34, A59, A74) und füge A99, A62, C33, C99 hinzu, ggf. A87. Prüfe Aufwand: Entfernen C61(205)+C60(190)+C12(188)+A23(217)+A34(123)+A59(62)+A74(37)=1022 frei. Addiere A99(417)+A62(577)+C33(569)=1563, das überschreitet. Alternativ: Wähle A99(417)+C33(569)=986, verbleibt +47 Puffer -> 986 <=1022, Rest 36, passt. Wertgewinn: Entfernt Werte: 304+289+287+316+222+161+136=1715. Hinzu Werte: A99 516 + C33 668 = 1184. Das wäre schlechter. Besser andere Auswahl: Ersetze niedriger Wertbeitrag: z.B. ersetze A34(222,123) + A59(161,62) + A74(136,37) + B49(119,20) + B17(116,17) = Aufwand 259; nein Summe Aufwand: 123+62+37+20+17=259, reicht um A99(417) nicht. Nimm dazu C12(188) -> 447 Aufwand, passt für A99(417) mit -30. Wert raus: 222+161+136+119+116+287=1041, rein 516, schlechter. Vielleicht visiere A87(718,817) rein: Aufwand 718, dafür raus: C61(205)+C60(190)+A34(123)+A59(62)+A74(37)+B49(20)+B17(17)+C12(188)=842 frei. Wert raus: 304+289+222+161+136+119+116+287=1634; rein 817, schlechter. A62(577,676) rein: ersetze C61(205)+C60(190)+A34(123)+A59(62)=580 frei, Werte raus 304+289+222+161=976; rein 676, schlechter. C99(627,726) rein: ersetze C61(205)+C60(190)+A34(123)+A59(62)+B17(17)+B49(20)=617 frei, Werte raus 304+289+222+161+116+119=1211; rein 726 schlechter. A84 sehr hoch, aber Aufwand 994 braucht viele Cuts; Wertwechsel vermutlich dennoch negativ, da Set bereits hohen Gesamtwert pro Aufwand hat. Schluss: Ihr Vorschlag ist für meine Präferenzen bereits sehr gut, nahe kapazitätsmaximal, viele hohe Werte inkludiert (A42/A48/A84 fehlen, aber Ersetzung würde viele mittelgute opfern und Gesamtwert senken). Taktik: Akzeptiere? Vielleicht noch eine kleine Verbesserung: Tausche A34(123,222) gegen A99(417,516) durch zusätzliches Entfernen von z.B. C12(188) und A59(62) und A74(37) und B17(17) und B49(20): frei 447, brauchen 417, passt, Wert raus 222+287+161+136+116+119=1041, rein 516 schlechter. Also keine klare Dominanz. Ich akzeptiere den Deal sofort, um Abschluss zu sichern.'}\\ \tt ABLEHNUNG: {}\\ \tt ARGUMENT: {'Dein Paket liegt bei einem Gesamtaufwand von 6833 und ist damit knapp unter dem Limit – gut balanciert. Viele meiner zentralen Prioritäten sind abgedeckt, und zusätzliche Aufwertungen würden voraussichtlich mehrere nützliche Teile verdrängen und den Gesamtwert verschlechtern. Wenn du magst, können wir minimal feintunen, aber ich sehe hier bereits einen fairen, effizienten Abschluss.'}\\ \tt ZUSTIMMUNG: {'B17', 'B49', 'A74', 'A59', 'A34', 'A65', 'A23', 'C12', 'C60', 'C61', 'B27', 'B85', 'A90', 'A50', 'B44', 'B54', 'C77', 'C90', 'C98', 'C13', 'C51', 'C22', 'C23', 'B36', 'A69'} 
	  } 
	   } 
	   } 
	  \\ 
 

    \theutterance \stepcounter{utterance}  

    & & \multicolumn{4}{p{0.6\linewidth}}{\cellcolor[rgb]{0.9,0.9,0.9}{%
	\makecell[{{p{\linewidth}}}]{% 
	  \tt {\tiny [GM$\rangle$B]}  
	 Du hast einen Vorschlag abgelehnt, der nicht aktiv ist. Vorschläge sind nur dann aktiv, wenn sie vom anderen Spieler mit VORSCHLAG eingebracht wurden und nicht von dir mit ABLEHNUNG deaktiviert wurden. Versuche es noch einmal. 
	  } 
	   } 
	   } 
	 & \\ 
 

    \theutterance \stepcounter{utterance}  

    & & & \multicolumn{4}{p{0.6\linewidth}}{\cellcolor[rgb]{1,0.85,0.72}{%
	\makecell[{{p{\linewidth}}}]{% 
	  \tt {\tiny [GM$\langle$B]}  
	 STRATEGISCHE ÜBERLEGUNGEN: {'Der gegnerische Vorschlag ist zulässig (Gesamtaufwand 6833 < 6880) und bringt mir hohen Gesamtwert (~9308). Versuche keinen unnötigen Konflikt zu erzeugen. Da ich bereits zustimmen will, halte ich mich strikt an das Protokoll: keine ABLEHNUNG senden, nur ARGUMENT und ZUSTIMMUNG auf den aktiven Vorschlag.'}\\ \tt ARGUMENT: {'Dein Paket ist für mich stimmig und effizient – wir liegen bei 6833 Aufwand und decken viele meiner Kernbedarfe ab. Ich sehe keinen Tausch, der den Gesamtwert für mich ohne größere Einbußen verbessern würde.'}\\ \tt ZUSTIMMUNG: {'B17', 'B49', 'A74', 'A59', 'A34', 'A65', 'A23', 'C12', 'C60', 'C61', 'B27', 'B85', 'A90', 'A50', 'B44', 'B54', 'C77', 'C90', 'C98', 'C13', 'C51', 'C22', 'C23', 'B36', 'A69'} 
	  } 
	   } 
	   } 
	  \\ 
 

    \theutterance \stepcounter{utterance}  

    & & & \multicolumn{2}{p{0.3\linewidth}}{\cellcolor[rgb]{0.95,0.95,0.95}{%
	\makecell[{{p{\linewidth}}}]{% 
	  \tt {\tiny [GM$|$GM]}  
	 game successful 
	  } 
	   } 
	   } 
	 & & \\ 
 

    \theutterance \stepcounter{utterance}  

    & & & \multicolumn{2}{p{0.3\linewidth}}{\cellcolor[rgb]{0.95,0.95,0.95}{%
	\makecell[{{p{\linewidth}}}]{% 
	  \tt {\tiny [GM$|$GM]}  
	 end game 
	  } 
	   } 
	   } 
	 & & \\ 
 

\end{supertabular}
}

\end{document}
