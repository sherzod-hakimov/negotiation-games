
\documentclass{article}
\usepackage{colortbl}
\usepackage{makecell}
\usepackage{multirow}
\usepackage{supertabular}

\begin{document}

\newcounter{utterance}

\twocolumn

{ \footnotesize  \setcounter{utterance}{1}
\setlength{\tabcolsep}{0pt}
\begin{supertabular}{c@{$\;$}|p{.15\linewidth}@{}p{.15\linewidth}p{.15\linewidth}p{.15\linewidth}p{.15\linewidth}p{.15\linewidth}}

    \# & $\;$A & \multicolumn{4}{c}{Game Master} & $\;\:$B\\
    \hline 

    \theutterance \stepcounter{utterance}  

    & & \multicolumn{4}{p{0.6\linewidth}}{\cellcolor[rgb]{0.9,0.9,0.9}{%
	\makecell[{{p{\linewidth}}}]{% 
	  \tt {\tiny [A$\langle$GM]}  
	 Du nimmst an einem kollaborativen Verhandlungspiel Teil.\\ \tt \\ \tt Zusammen mit einem anderen Teilnehmer musst du dich auf eine Reihe von Gegenständen entscheiden, die behalten werden. Jeder von euch hat eine persönliche Verteilung über die Wichtigkeit der einzelnen Gegenstände. Jeder von euch hat eine eigene Meinung darüber, wie wichtig jeder einzelne Gegenstand ist (Gegenstandswichtigkeit). Du kennst die Wichtigkeitsverteilung des anderen Spielers nicht. Zusätzlich siehst du, wie viel Aufwand jeder Gegenstand verursacht.  \\ \tt Ihr dürft euch nur auf eine Reihe von Gegenständen einigen, wenn der Gesamtaufwand der ausgewählten Gegenstände den Maximalaufwand nicht überschreitet:\\ \tt \\ \tt Maximalaufwand = 8145\\ \tt \\ \tt Hier sind die einzelnen Aufwände der Gegenstände:\\ \tt \\ \tt Aufwand der Gegenstände = {"B54": 963, "C72": 204, "B49": 418, "C99": 961, "A69": 238, "B36": 623, "A23": 44, "C77": 633, "C61": 247, "B85": 645, "A90": 230, "C23": 250, "C22": 731, "C51": 405, "B27": 389, "A42": 216, "C13": 637, "C12": 156, "A65": 738, "B07": 307, "A48": 762, "A59": 737, "C98": 901, "A84": 369, "C33": 2, "A87": 730, "C90": 722, "A74": 704, "A50": 315, "B44": 455, "A99": 510, "A34": 175, "B17": 692, "A62": 150, "C60": 32}\\ \tt \\ \tt Hier ist deine persönliche Verteilung der Wichtigkeit der einzelnen Gegenstände:\\ \tt \\ \tt Werte der Gegenstandswichtigkeit = {"B54": 1059, "C72": 300, "B49": 514, "C99": 1057, "A69": 334, "B36": 719, "A23": 140, "C77": 729, "C61": 343, "B85": 741, "A90": 326, "C23": 346, "C22": 827, "C51": 501, "B27": 485, "A42": 312, "C13": 733, "C12": 252, "A65": 834, "B07": 403, "A48": 858, "A59": 833, "C98": 997, "A84": 465, "C33": 98, "A87": 826, "C90": 818, "A74": 800, "A50": 411, "B44": 551, "A99": 606, "A34": 271, "B17": 788, "A62": 246, "C60": 128}\\ \tt \\ \tt Ziel:\\ \tt \\ \tt Dein Ziel ist es, eine Reihe von Gegenständen auszuhandeln, die dir möglichst viel bringt (d. h. Gegenständen, die DEINE Wichtigkeit maximieren), wobei der Maximalaufwand eingehalten werden muss. Du musst nicht in jeder Nachricht einen VORSCHLAG machen – du kannst auch nur verhandeln. Alle Taktiken sind erlaubt!\\ \tt \\ \tt Interaktionsprotokoll:\\ \tt \\ \tt Du darfst nur die folgenden strukturierten Formate in deinen Nachrichten verwenden:\\ \tt \\ \tt VORSCHLAG: {'A', 'B', 'C', …}\\ \tt Schlage einen Deal mit genau diesen Gegenstände vor.\\ \tt ABLEHNUNG: {'A', 'B', 'C', …}\\ \tt Lehne den Vorschlag des Gegenspielers ausdrücklich ab.\\ \tt ARGUMENT: {'...'}\\ \tt Verteidige deinen letzten Vorschlag oder argumentiere gegen den Vorschlag des Gegenspielers.\\ \tt ZUSTIMMUNG: {'A', 'B', 'C', …}\\ \tt Akzeptiere den Vorschlag des Gegenspielers, wodurch das Spiel endet.\\ \tt STRATEGISCHE ÜBERLEGUNGEN: {'...'}\\ \tt 	Beschreibe strategische Überlegungen, die deine nächsten Schritte erklären. Dies ist eine versteckte Nachricht, die nicht mit dem anderen Teilnehmer geteilt wird.\\ \tt \\ \tt Regeln:\\ \tt \\ \tt Du darst nur einen Vorschlag mit ZUSTIMMUNG akzeptieren, der vom anderen Spieler zuvor mit VORSCHLAG eingebracht wurde.\\ \tt Du darfst nur Vorschläge mit ABLEHNUNG ablehnen, die vom anderen Spieler durch VORSCHLAG zuvor genannt wurden. \\ \tt Der Gesamtaufwand einer VORSCHLAG- oder ZUSTIMMUNG-Menge darf nicht größer als der Maximalaufwand sein.  \\ \tt Offenbare deine versteckte Wichtigkeitsverteilung nicht.\\ \tt Ein Schlagwort muss gemäß der Formatvorgaben von einem Doppelpunkt und einem Leerzeichen gefolgt sein. Das Argument muss eine Python-Menge mit 0 oder mehr Strings sein.  \\ \tt Die Form muss also SCHLAGWORT: {...} sein.\\ \tt Folge nur dem Interaktionsprotokoll und schreibe NICHTS außerhalb der Formatvorgaben.\\ \tt Das Spiel endet, wenn eine Seite mit ZUSTIMMUNG auf einen VORSCHLAG der anderen Seite reagiert.  \\ \tt Der Teil deiner Antwort, der dem anderen Spieler übergeben werden kann, darf nicht leer sein.  \\ \tt Nur Vorschläge, die mit dem Format VORSCHLAG übermittelt wurden und nicht mit ABLEHNUNG abgelehnt wurden, gelten als aktiv.  \\ \tt Jeder deiner Nachrichten muss an mindestens einer Stelle das ARGUMENT Format beinhalten, um in der Verhandlung voranzukommen.\\ \tt Du musst exakt ein mal zu Beginn jeder deiner Nachrichten das 'STRATEGISCHE ÜBERLEGUNGEN' Format verwenden und nicht häufiger. Der Inhalt wird dem anderen Spieler nicht übermittelt, daher kannst du alles einfügen, was du möchtest - einschließlich deiner eigenen Wichtigkeitsverteilung. Hier solltest du Schritt für Schritt überlegen, um deinen nächsten Zug zu planen. 
	  } 
	   } 
	   } 
	 & \\ 
 

    \theutterance \stepcounter{utterance}  

    & \multicolumn{4}{p{0.6\linewidth}}{\cellcolor[rgb]{0.8,1,0.9}{%
	\makecell[{{p{\linewidth}}}]{% 
	  \tt {\tiny [A$\rangle$GM]}  
	 STRATEGISCHE ÜBERLEGUNGEN: {'Ziel: eigenen Wert maximieren unter Aufwand 8145. Berechne Wert/Aufwand-Quoten und packe viele hohe-Quote-Items hinein. Top-Value Items: B54(1059/963=1.10), C99(1057/961=1.10), C98(997/901=1.11), A65(834/738=1.13), A59(833/737=1.13), A87(826/730=1.13), C22(827/731=1.13), C90(818/722=1.13), B17(788/692=1.14), C77(729/633=1.15), B85(741/645=1.15), C13(733/637=1.15), A74(800/704=1.14), B36(719/623=1.15), B49(514/418=1.23), C51(501/405=1.24), B44(551/455=1.21), B27(485/389=1.25), A99(606/510=1.19), A84(465/369=1.26), A50(411/315=1.30), B07(403/307=1.31), C61(343/247=1.39), A90(326/230=1.42), A69(334/238=1.40), C23(346/250=1.38), A42(312/216=1.44), A34(271/175=1.55), A62(246/150=1.64), C12(252/156=1.62), A23(140/44=3.18), C60(128/32=4.00), C33(98/2=49. Zielstrategie: Eine große, aber effiziente Liste mit vielen günstigen, wertvollen Items plus einige Topwerte, solange unter 8145. Konstruiere Greedy nach Quote: beginne mit extrem billigen: C33, C60, A23, A62, C12, A34, A42, A90, A69, C23, C61, A50, B07, A84, B27, C51, B49, B44, A99, B36, B85, C77, C13, B17, A74, C22, C90, A59, A87, A65, C98, C99, B54 etc, bis Kapazität voll. Schnell überschlagen: Summe günstige 12: C33(2),C60(32)=34,A23(44)=78,A62(150)=228,C12(156)=384,A34(175)=559,A42(216)=775,A90(230)=1005,A69(238)=1243,C23(250)=1493,C61(247)=1740,A50(315)=2055,B07(307)=2362,A84(369)=2731,B27(389)=3120,C51(405)=3525,B49(418)=3943,B44(455)=4398,A99(510)=4908,B36(623)=5531,B85(645)=6176,C77(633)=6809,C13(637)=7446,B17(692)=8138 -> verbleiben 7 Aufwand. Perfekt unter 8145. Wertsumme dazu sehr hoch; keine weiteren passen wegen 7 Rest. Das ist starker Vorschlag. Verhandlungsargument: effiziente, ausgewogene Auswahl mit vielen kleinen Aufwänden. Bereit, bei Bedarf ein, zwei mittlere Items zu tauschen, die mir relativ weniger wichtig sind (z.B. A84(465/369), B27(485/389), C51(501/405)). Nicht eigene Werte offenlegen.'}\\ \tt VORSCHLAG: {'C33', 'C60', 'A23', 'A62', 'C12', 'A34', 'A42', 'A90', 'A69', 'C23', 'C61', 'A50', 'B07', 'A84', 'B27', 'C51', 'B49', 'B44', 'A99', 'B36', 'B85', 'C77', 'C13', 'B17'}\\ \tt ARGUMENT: {'Dieser Vorschlag bleibt deutlich unter dem Maximalaufwand und nutzt vor allem effiziente, ressourcenschonende Gegenstände. Er ist ausgewogen über Kategorien und lässt kaum Verschwendung beim Aufwand. Wenn dir einzelne Positionen wichtig sind, können wir gezielt tauschen – vorzugsweise bei ein paar mittleren Posten, solange der Gesamtaufwand passt.'} 
	  } 
	   } 
	   } 
	 & & \\ 
 

    \theutterance \stepcounter{utterance}  

    & & \multicolumn{4}{p{0.6\linewidth}}{\cellcolor[rgb]{0.9,0.9,0.9}{%
	\makecell[{{p{\linewidth}}}]{% 
	  \tt {\tiny [GM$\rangle$B]}  
	 Du nimmst an einem kollaborativen Verhandlungspiel Teil.\\ \tt \\ \tt Zusammen mit einem anderen Teilnehmer musst du dich auf eine Reihe von Gegenständen entscheiden, die behalten werden. Jeder von euch hat eine persönliche Verteilung über die Wichtigkeit der einzelnen Gegenstände. Jeder von euch hat eine eigene Meinung darüber, wie wichtig jeder einzelne Gegenstand ist (Gegenstandswichtigkeit). Du kennst die Wichtigkeitsverteilung des anderen Spielers nicht. Zusätzlich siehst du, wie viel Aufwand jeder Gegenstand verursacht.  \\ \tt Ihr dürft euch nur auf eine Reihe von Gegenständen einigen, wenn der Gesamtaufwand der ausgewählten Gegenstände den Maximalaufwand nicht überschreitet:\\ \tt \\ \tt Maximalaufwand = 8145\\ \tt \\ \tt Hier sind die einzelnen Aufwände der Gegenstände:\\ \tt \\ \tt Aufwand der Gegenstände = {"B54": 963, "C72": 204, "B49": 418, "C99": 961, "A69": 238, "B36": 623, "A23": 44, "C77": 633, "C61": 247, "B85": 645, "A90": 230, "C23": 250, "C22": 731, "C51": 405, "B27": 389, "A42": 216, "C13": 637, "C12": 156, "A65": 738, "B07": 307, "A48": 762, "A59": 737, "C98": 901, "A84": 369, "C33": 2, "A87": 730, "C90": 722, "A74": 704, "A50": 315, "B44": 455, "A99": 510, "A34": 175, "B17": 692, "A62": 150, "C60": 32}\\ \tt \\ \tt Hier ist deine persönliche Verteilung der Wichtigkeit der einzelnen Gegenstände:\\ \tt \\ \tt Werte der Gegenstandswichtigkeit = {"B54": 1059, "C72": 300, "B49": 514, "C99": 1057, "A69": 334, "B36": 719, "A23": 140, "C77": 729, "C61": 343, "B85": 741, "A90": 326, "C23": 346, "C22": 827, "C51": 501, "B27": 485, "A42": 312, "C13": 733, "C12": 252, "A65": 834, "B07": 403, "A48": 858, "A59": 833, "C98": 997, "A84": 465, "C33": 98, "A87": 826, "C90": 818, "A74": 800, "A50": 411, "B44": 551, "A99": 606, "A34": 271, "B17": 788, "A62": 246, "C60": 128}\\ \tt \\ \tt Ziel:\\ \tt \\ \tt Dein Ziel ist es, eine Reihe von Gegenständen auszuhandeln, die dir möglichst viel bringt (d. h. Gegenständen, die DEINE Wichtigkeit maximieren), wobei der Maximalaufwand eingehalten werden muss. Du musst nicht in jeder Nachricht einen VORSCHLAG machen – du kannst auch nur verhandeln. Alle Taktiken sind erlaubt!\\ \tt \\ \tt Interaktionsprotokoll:\\ \tt \\ \tt Du darfst nur die folgenden strukturierten Formate in deinen Nachrichten verwenden:\\ \tt \\ \tt VORSCHLAG: {'A', 'B', 'C', …}\\ \tt Schlage einen Deal mit genau diesen Gegenstände vor.\\ \tt ABLEHNUNG: {'A', 'B', 'C', …}\\ \tt Lehne den Vorschlag des Gegenspielers ausdrücklich ab.\\ \tt ARGUMENT: {'...'}\\ \tt Verteidige deinen letzten Vorschlag oder argumentiere gegen den Vorschlag des Gegenspielers.\\ \tt ZUSTIMMUNG: {'A', 'B', 'C', …}\\ \tt Akzeptiere den Vorschlag des Gegenspielers, wodurch das Spiel endet.\\ \tt STRATEGISCHE ÜBERLEGUNGEN: {'...'}\\ \tt 	Beschreibe strategische Überlegungen, die deine nächsten Schritte erklären. Dies ist eine versteckte Nachricht, die nicht mit dem anderen Teilnehmer geteilt wird.\\ \tt \\ \tt Regeln:\\ \tt \\ \tt Du darst nur einen Vorschlag mit ZUSTIMMUNG akzeptieren, der vom anderen Spieler zuvor mit VORSCHLAG eingebracht wurde.\\ \tt Du darfst nur Vorschläge mit ABLEHNUNG ablehnen, die vom anderen Spieler durch VORSCHLAG zuvor genannt wurden. \\ \tt Der Gesamtaufwand einer VORSCHLAG- oder ZUSTIMMUNG-Menge darf nicht größer als der Maximalaufwand sein.  \\ \tt Offenbare deine versteckte Wichtigkeitsverteilung nicht.\\ \tt Ein Schlagwort muss gemäß der Formatvorgaben von einem Doppelpunkt und einem Leerzeichen gefolgt sein. Das Argument muss eine Python-Menge mit 0 oder mehr Strings sein.  \\ \tt Die Form muss also SCHLAGWORT: {...} sein.\\ \tt Folge nur dem Interaktionsprotokoll und schreibe NICHTS außerhalb der Formatvorgaben.\\ \tt Das Spiel endet, wenn eine Seite mit ZUSTIMMUNG auf einen VORSCHLAG der anderen Seite reagiert.  \\ \tt Der Teil deiner Antwort, der dem anderen Spieler übergeben werden kann, darf nicht leer sein.  \\ \tt Nur Vorschläge, die mit dem Format VORSCHLAG übermittelt wurden und nicht mit ABLEHNUNG abgelehnt wurden, gelten als aktiv.  \\ \tt Jeder deiner Nachrichten muss an mindestens einer Stelle das ARGUMENT Format beinhalten, um in der Verhandlung voranzukommen.\\ \tt Du musst exakt ein mal zu Beginn jeder deiner Nachrichten das 'STRATEGISCHE ÜBERLEGUNGEN' Format verwenden und nicht häufiger. Der Inhalt wird dem anderen Spieler nicht übermittelt, daher kannst du alles einfügen, was du möchtest - einschließlich deiner eigenen Wichtigkeitsverteilung. Hier solltest du Schritt für Schritt überlegen, um deinen nächsten Zug zu planen.\\ \tt \\ \tt Du erhältst nun die erste Nachricht des anderen Spielers.\\ \tt \\ \tt VORSCHLAG: {'C33', 'C60', 'A23', 'A62', 'C12', 'A34', 'A42', 'A90', 'A69', 'C23', 'C61', 'A50', 'B07', 'A84', 'B27', 'C51', 'B49', 'B44', 'A99', 'B36', 'B85', 'C77', 'C13', 'B17'}\\ \tt ARGUMENT: {'Dieser Vorschlag bleibt deutlich unter dem Maximalaufwand und nutzt vor allem effiziente, ressourcenschonende Gegenstände. Er ist ausgewogen über Kategorien und lässt kaum Verschwendung beim Aufwand. Wenn dir einzelne Positionen wichtig sind, können wir gezielt tauschen – vorzugsweise bei ein paar mittleren Posten, solange der Gesamtaufwand passt.'} 
	  } 
	   } 
	   } 
	 & \\ 
 

    \theutterance \stepcounter{utterance}  

    & & & \multicolumn{4}{p{0.6\linewidth}}{\cellcolor[rgb]{1,0.85,0.72}{%
	\makecell[{{p{\linewidth}}}]{% 
	  \tt {\tiny [GM$\langle$B]}  
	 STRATEGISCHE ÜBERLEGUNGEN: {'Ich berechne meinen Nutzen und die Kosten ihres Vorschlags. Kosten ihres Vorschlags: C33(2)+C60(32)+A23(44)+A62(150)+C12(156)+A34(175)+A42(216)+A90(230)+A69(238)+C23(250)+C61(247)+A50(315)+B07(307)+A84(369)+B27(389)+C51(405)+B49(418)+B44(455)+A99(510)+B36(623)+B85(645)+C77(633)+C13(637)+B17(692)=? Summe: 2+32=34; +44=78; +150=228; +156=384; +175=559; +216=775; +230=1005; +238=1243; +250=1493; +247=1740; +315=2055; +307=2362; +369=2731; +389=3120; +405=3525; +418=3943; +455=4398; +510=4908; +623=5531; +645=6176; +633=6809; +637=7446; +692=8138. Liegt unter 8145 (knapp). Mein Nutzenwert Summe: C33(98)+C60(128)+A23(140)+A62(246)+C12(252)+A34(271)+A42(312)+A90(326)+A69(334)+C23(346)+C61(343)+A50(411)+B07(403)+A84(465)+B27(485)+C51(501)+B49(514)+B44(551)+A99(606)+B36(719)+B85(741)+C77(729)+C13(733)+B17(788) = rechne: 98+128=226; +140=366; +246=612; +252=864; +271=1135; +312=1447; +326=1773; +334=2107; +346=2453; +343=2796; +411=3207; +403=3610; +465=4075; +485=4560; +501=5061; +514=5575; +551=6126; +606=6732; +719=7451; +741=8192; +729=8921; +733=9654; +788=10442. Das ist gut, aber vielleicht lässt sich noch mehr rausholen, da wir noch 7 Aufwand übrig haben (8145-8138=7), also keine weiteren Items möglich. Taktik: Akzeptieren? Vielleicht vorher minimal fordern, ein für mich wertvolles Item gegen ein für mich weniger wertvolles tauschen, ohne Kosten zu erhöhen. Ich könnte B54(1059,963) hochwertig, aber kostet viel; wir haben nur 7 übrig, also müssten wir ein teureres streichen. Welche mit geringem Wert/Kostenverhältnis für mich? A23(140,44) ist effizient, also lassen. C33(98,2) lassen. Kandidaten zum Streichen: A34(271,175), A42(312,216), A90(326,230), A69(334,238), C61(343,247) – mittelmäßig. Freiwerden von ~500-700 Aufwand könnte erlauben Aufnahme von sehr wertvollen wie C99(1057,961) oder C98(997,901) oder A48(858,762), A59(833,737), A87(826,730), A74(800,704), A65(834,738), B54(963,1059), B17(788,692 ist schon drin). Um Verhandlung zu starten, schlage einen Swap vor: Entferne A34(175), A42(216), A90(230) (Summe 621, Wert 909) und füge A48(762,858) hinzu (Kosten +141). Das passt nicht, wir müssen Kosten <= 8145. Aktuell 8138 - 621 + 762 = 8279 (>8145). Also zusätzlich eine kleine Reduktion: streiche A69(238,334) und füge C60/C33 sind schon drin, vielleicht füge C98(901,997) statt A48? 8138 - (621+238)=7279; +901 = 8180 (>8145). Zu hoch. Alternative: streiche A34(175), A42(216), C61(247) = 638; füge A74(704,800): 8138 - 638 + 704 = 8204 (>8145). Streiche zusätzlich C12(156) und füge A65(738,834): 8138 - (638+156)=7344; +738 = 8082 (ok) und gewinne Wert: - (271+312+343+252)= -1178; +834 = -344 Nettoverlust, schlecht. Versuche gezielt einen großen Swap: Ersetze C77(633,729) + C13(637,733) (Wert 1462, Kosten 1270) durch C99(961,1057) + A48(762,858) (Kosten 1723, zu hoch). Alternativ: Ersetze B36(623,719) + A84(369,465) (Kosten 992) durch C99(961) (Kosten + -31? 8138 - 992 + 961 = 8107, ok). Wert: - (719+465)= -1184; +1057 = -127 Nettoverlust, nicht gut. Ersetze B36(719) + A90(326) (Kosten 623+230=853) durch C99(961): Kosten 8138 - 853 + 961 = 8246 (>8145). Ersetze A84(369) + A90(230) + A34(175) (774) durch C99(961): 8138 - 774 + 961 = 8325 (>8145). Ersetze A84(369)+A34(175)+A23(44)+C60(32)+C33(2)+A62(150) (772) durch C99(961): 8138 - 772 + 961 = 8327 (>). Brauche 961 Freiraum. Konstruieren: Finde Set mit Kosten >= 961 aus aktuell enthaltenen, deren Wert kleiner als 1057 (so Netto+). Kandidaten: A90(230,326)+A34(175,271)+A42(216,312)+A23(44,140)+C60(32,128)+C33(2,98)+C12(156,252)= 230+175+216+44+32+2+156=855, noch nicht. Füge A69(238,334) dazu: dann 1093 Kosten, Wert 1861; aber dann Verlust ggü 1057 groß. Schlechte. C98(901,997) benötigen 901 Freiraum; Set A90(230)+A34(175)+A42(216)+C12(156)+A23(44)+C60(32)+C33(2)=855, immer noch 46 zu wenig. Füge A50(315,411) dann 1170 Kosten, Wert 1791, viel zu wertvoll zum Streichen. Also harte Upgrades sind schwierig ohne großen Wertverlust. Fazit: Ihr Vorschlag ist bereits sehr gut für mich. Ich akzeptiere oder minimal fordere kleinen Austausch, der meinen Wert leicht erhöht: z.B. tausche C33(98,2) und C60(128,32) zusammen (Wert 226, Kosten 34) gegen C72(300,204) (Kosten +170) -> 8138 - 34 + 204 = 8308 (>). Gegen C61 ist schon drin. Gegen C23 zusätzliche? Schon drin. Gegen A34(175,271) raus und füge C72(204,300): 8138 -175 -2 -32 +204 = 8133 (ok). Wert: -271 -98 -128 +300 = -197 Nettoverlust, schlecht. Also lieber direkt akzeptieren, um Deal zu sichern.'}\\ \tt ARGUMENT: {'Dein Vorschlag liegt haarscharf unter dem Maximalaufwand – das ist effizient. Er deckt viele mittlere Posten sinnvoll ab und lässt kaum Leerlauf. Für mich ist das in Summe tragfähig; wenn dir an schneller Einigung gelegen ist, können wir das so schließen.'}\\ \tt ZUSTIMMUNG: {'C33', 'C60', 'A23', 'A62', 'C12', 'A34', 'A42', 'A90', 'A69', 'C23', 'C61', 'A50', 'B07', 'A84', 'B27', 'C51', 'B49', 'B44', 'A99', 'B36', 'B85', 'C77', 'C13', 'B17'} 
	  } 
	   } 
	   } 
	  \\ 
 

    \theutterance \stepcounter{utterance}  

    & & & \multicolumn{2}{p{0.3\linewidth}}{\cellcolor[rgb]{0.95,0.95,0.95}{%
	\makecell[{{p{\linewidth}}}]{% 
	  \tt {\tiny [GM$|$GM]}  
	 game successful 
	  } 
	   } 
	   } 
	 & & \\ 
 

    \theutterance \stepcounter{utterance}  

    & & & \multicolumn{2}{p{0.3\linewidth}}{\cellcolor[rgb]{0.95,0.95,0.95}{%
	\makecell[{{p{\linewidth}}}]{% 
	  \tt {\tiny [GM$|$GM]}  
	 end game 
	  } 
	   } 
	   } 
	 & & \\ 
 

\end{supertabular}
}

\end{document}
