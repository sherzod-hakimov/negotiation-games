
\documentclass{article}
\usepackage{colortbl}
\usepackage{makecell}
\usepackage{multirow}
\usepackage{supertabular}

\begin{document}

\newcounter{utterance}

\twocolumn

{ \footnotesize  \setcounter{utterance}{1}
\setlength{\tabcolsep}{0pt}
\begin{supertabular}{c@{$\;$}|p{.15\linewidth}@{}p{.15\linewidth}p{.15\linewidth}p{.15\linewidth}p{.15\linewidth}p{.15\linewidth}}

    \# & $\;$A & \multicolumn{4}{c}{Game Master} & $\;\:$B\\
    \hline 

    \theutterance \stepcounter{utterance}  

    & & \multicolumn{4}{p{0.6\linewidth}}{\cellcolor[rgb]{0.9,0.9,0.9}{%
	\makecell[{{p{\linewidth}}}]{% 
	  \tt {\tiny [A$\langle$GM]}  
	 Du nimmst an einem kollaborativen Verhandlungspiel Teil.\\ \tt \\ \tt Zusammen mit einem anderen Teilnehmer musst du dich auf eine Reihe von Gegenständen entscheiden, die behalten werden. Jeder von euch hat eine persönliche Verteilung über die Wichtigkeit der einzelnen Gegenstände. Jeder von euch hat eine eigene Meinung darüber, wie wichtig jeder einzelne Gegenstand ist (Gegenstandswichtigkeit). Du kennst die Wichtigkeitsverteilung des anderen Spielers nicht. Zusätzlich siehst du, wie viel Aufwand jeder Gegenstand verursacht.  \\ \tt Ihr dürft euch nur auf eine Reihe von Gegenständen einigen, wenn der Gesamtaufwand der ausgewählten Gegenstände den Maximalaufwand nicht überschreitet:\\ \tt \\ \tt Maximalaufwand = 3357\\ \tt \\ \tt Hier sind die einzelnen Aufwände der Gegenstände:\\ \tt \\ \tt Aufwand der Gegenstände = {"C76": 506, "C38": 492, "C56": 338, "A03": 122, "A07": 131, "A83": 909, "C98": 144, "C08": 716, "C62": 263, "C00": 231, "C10": 91, "C81": 651, "B38": 552, "C03": 849, "C32": 720}\\ \tt \\ \tt Hier ist deine persönliche Verteilung der Wichtigkeit der einzelnen Gegenstände:\\ \tt \\ \tt Werte der Gegenstandswichtigkeit = {"C76": 138, "C38": 583, "C56": 868, "A03": 822, "A07": 783, "A83": 65, "C98": 262, "C08": 121, "C62": 508, "C00": 780, "C10": 461, "C81": 484, "B38": 668, "C03": 389, "C32": 808}\\ \tt \\ \tt Ziel:\\ \tt \\ \tt Dein Ziel ist es, eine Reihe von Gegenständen auszuhandeln, die dir möglichst viel bringt (d. h. Gegenständen, die DEINE Wichtigkeit maximieren), wobei der Maximalaufwand eingehalten werden muss. Du musst nicht in jeder Nachricht einen VORSCHLAG machen – du kannst auch nur verhandeln. Alle Taktiken sind erlaubt!\\ \tt \\ \tt Interaktionsprotokoll:\\ \tt \\ \tt Du darfst nur die folgenden strukturierten Formate in deinen Nachrichten verwenden:\\ \tt \\ \tt VORSCHLAG: {'A', 'B', 'C', …}\\ \tt Schlage einen Deal mit genau diesen Gegenstände vor.\\ \tt ABLEHNUNG: {'A', 'B', 'C', …}\\ \tt Lehne den Vorschlag des Gegenspielers ausdrücklich ab.\\ \tt ARGUMENT: {'...'}\\ \tt Verteidige deinen letzten Vorschlag oder argumentiere gegen den Vorschlag des Gegenspielers.\\ \tt ZUSTIMMUNG: {'A', 'B', 'C', …}\\ \tt Akzeptiere den Vorschlag des Gegenspielers, wodurch das Spiel endet.\\ \tt \\ \tt \\ \tt Regeln:\\ \tt \\ \tt Du darst nur einen Vorschlag mit ZUSTIMMUNG akzeptieren, der vom anderen Spieler zuvor mit VORSCHLAG eingebracht wurde.\\ \tt Du darfst nur Vorschläge mit ABLEHNUNG ablehnen, die vom anderen Spieler durch VORSCHLAG zuvor genannt wurden. \\ \tt Der Gesamtaufwand einer VORSCHLAG- oder ZUSTIMMUNG-Menge darf nicht größer als der Maximalaufwand sein.  \\ \tt Offenbare deine versteckte Wichtigkeitsverteilung nicht.\\ \tt Ein Schlagwort muss gemäß der Formatvorgaben von einem Doppelpunkt und einem Leerzeichen gefolgt sein. Das Argument muss eine Python-Menge mit 0 oder mehr Strings sein.  \\ \tt Die Form muss also SCHLAGWORT: {...} sein.\\ \tt Folge nur dem Interaktionsprotokoll und schreibe NICHTS außerhalb der Formatvorgaben.\\ \tt Das Spiel endet, wenn eine Seite mit ZUSTIMMUNG auf einen VORSCHLAG der anderen Seite reagiert.  \\ \tt Der Teil deiner Antwort, der dem anderen Spieler übergeben werden kann, darf nicht leer sein.  \\ \tt Nur Vorschläge, die mit dem Format VORSCHLAG übermittelt wurden und nicht mit ABLEHNUNG abgelehnt wurden, gelten als aktiv.  \\ \tt Jeder deiner Nachrichten muss an mindestens einer Stelle das ARGUMENT Format beinhalten, um in der Verhandlung voranzukommen.\\ \tt  
	  } 
	   } 
	   } 
	 & \\ 
 

    \theutterance \stepcounter{utterance}  

    & \multicolumn{4}{p{0.6\linewidth}}{\cellcolor[rgb]{0.8,1,0.9}{%
	\makecell[{{p{\linewidth}}}]{% 
	  \tt {\tiny [A$\rangle$GM]}  
	 VORSCHLAG: {'C56', 'A03', 'A07', 'C62', 'C00', 'C10'}\\ \tt ARGUMENT: {'Das Paket bleibt deutlich unter dem Maximalaufwand und deckt mehrere unterschiedliche Kategorien ab. Es bietet eine ausgewogene Basis, auf der wir aufbauen können. Wenn dir einzelne Teile wichtig sind, können wir flexibel tauschen, solange wir im Budget bleiben.'} 
	  } 
	   } 
	   } 
	 & & \\ 
 

    \theutterance \stepcounter{utterance}  

    & & \multicolumn{4}{p{0.6\linewidth}}{\cellcolor[rgb]{0.9,0.9,0.9}{%
	\makecell[{{p{\linewidth}}}]{% 
	  \tt {\tiny [GM$\rangle$B]}  
	 Du nimmst an einem kollaborativen Verhandlungspiel Teil.\\ \tt \\ \tt Zusammen mit einem anderen Teilnehmer musst du dich auf eine Reihe von Gegenständen entscheiden, die behalten werden. Jeder von euch hat eine persönliche Verteilung über die Wichtigkeit der einzelnen Gegenstände. Jeder von euch hat eine eigene Meinung darüber, wie wichtig jeder einzelne Gegenstand ist (Gegenstandswichtigkeit). Du kennst die Wichtigkeitsverteilung des anderen Spielers nicht. Zusätzlich siehst du, wie viel Aufwand jeder Gegenstand verursacht.  \\ \tt Ihr dürft euch nur auf eine Reihe von Gegenständen einigen, wenn der Gesamtaufwand der ausgewählten Gegenstände den Maximalaufwand nicht überschreitet:\\ \tt \\ \tt Maximalaufwand = 3357\\ \tt \\ \tt Hier sind die einzelnen Aufwände der Gegenstände:\\ \tt \\ \tt Aufwand der Gegenstände = {"C76": 506, "C38": 492, "C56": 338, "A03": 122, "A07": 131, "A83": 909, "C98": 144, "C08": 716, "C62": 263, "C00": 231, "C10": 91, "C81": 651, "B38": 552, "C03": 849, "C32": 720}\\ \tt \\ \tt Hier ist deine persönliche Verteilung der Wichtigkeit der einzelnen Gegenstände:\\ \tt \\ \tt Werte der Gegenstandswichtigkeit = {"C56": 65, "A03": 121, "C32": 138, "A07": 262, "C00": 389, "B38": 461, "C38": 484, "C62": 508, "C81": 583, "C10": 668, "C03": 780, "C98": 783, "C76": 808, "C08": 822, "A83": 868}\\ \tt \\ \tt Ziel:\\ \tt \\ \tt Dein Ziel ist es, eine Reihe von Gegenständen auszuhandeln, die dir möglichst viel bringt (d. h. Gegenständen, die DEINE Wichtigkeit maximieren), wobei der Maximalaufwand eingehalten werden muss. Du musst nicht in jeder Nachricht einen VORSCHLAG machen – du kannst auch nur verhandeln. Alle Taktiken sind erlaubt!\\ \tt \\ \tt Interaktionsprotokoll:\\ \tt \\ \tt Du darfst nur die folgenden strukturierten Formate in deinen Nachrichten verwenden:\\ \tt \\ \tt VORSCHLAG: {'A', 'B', 'C', …}\\ \tt Schlage einen Deal mit genau diesen Gegenstände vor.\\ \tt ABLEHNUNG: {'A', 'B', 'C', …}\\ \tt Lehne den Vorschlag des Gegenspielers ausdrücklich ab.\\ \tt ARGUMENT: {'...'}\\ \tt Verteidige deinen letzten Vorschlag oder argumentiere gegen den Vorschlag des Gegenspielers.\\ \tt ZUSTIMMUNG: {'A', 'B', 'C', …}\\ \tt Akzeptiere den Vorschlag des Gegenspielers, wodurch das Spiel endet.\\ \tt \\ \tt \\ \tt Regeln:\\ \tt \\ \tt Du darst nur einen Vorschlag mit ZUSTIMMUNG akzeptieren, der vom anderen Spieler zuvor mit VORSCHLAG eingebracht wurde.\\ \tt Du darfst nur Vorschläge mit ABLEHNUNG ablehnen, die vom anderen Spieler durch VORSCHLAG zuvor genannt wurden. \\ \tt Der Gesamtaufwand einer VORSCHLAG- oder ZUSTIMMUNG-Menge darf nicht größer als der Maximalaufwand sein.  \\ \tt Offenbare deine versteckte Wichtigkeitsverteilung nicht.\\ \tt Ein Schlagwort muss gemäß der Formatvorgaben von einem Doppelpunkt und einem Leerzeichen gefolgt sein. Das Argument muss eine Python-Menge mit 0 oder mehr Strings sein.  \\ \tt Die Form muss also SCHLAGWORT: {...} sein.\\ \tt Folge nur dem Interaktionsprotokoll und schreibe NICHTS außerhalb der Formatvorgaben.\\ \tt Das Spiel endet, wenn eine Seite mit ZUSTIMMUNG auf einen VORSCHLAG der anderen Seite reagiert.  \\ \tt Der Teil deiner Antwort, der dem anderen Spieler übergeben werden kann, darf nicht leer sein.  \\ \tt Nur Vorschläge, die mit dem Format VORSCHLAG übermittelt wurden und nicht mit ABLEHNUNG abgelehnt wurden, gelten als aktiv.  \\ \tt Jeder deiner Nachrichten muss an mindestens einer Stelle das ARGUMENT Format beinhalten, um in der Verhandlung voranzukommen.\\ \tt \\ \tt \\ \tt Du erhältst nun die erste Nachricht des anderen Spielers.\\ \tt \\ \tt VORSCHLAG: {'C56', 'A03', 'A07', 'C62', 'C00', 'C10'}\\ \tt ARGUMENT: {'Das Paket bleibt deutlich unter dem Maximalaufwand und deckt mehrere unterschiedliche Kategorien ab. Es bietet eine ausgewogene Basis, auf der wir aufbauen können. Wenn dir einzelne Teile wichtig sind, können wir flexibel tauschen, solange wir im Budget bleiben.'} 
	  } 
	   } 
	   } 
	 & \\ 
 

    \theutterance \stepcounter{utterance}  

    & & & \multicolumn{4}{p{0.6\linewidth}}{\cellcolor[rgb]{1,0.85,0.72}{%
	\makecell[{{p{\linewidth}}}]{% 
	  \tt {\tiny [GM$\langle$B]}  
	 ABLEHNUNG: {'C56', 'A03', 'A07', 'C62', 'C00', 'C10'}\\ \tt ARGUMENT: {'Danke für den Auftakt. Dein Paket ist solide im Aufwand, aber für mich fehlen einige hochattraktive Stücke. Ich würde gern ein Paket vorschlagen, das weiterhin deutlich unter dem Maximalaufwand bleibt und mehr Wert für mich bietet, ohne das Budget zu sprengen.'}\\ \tt VORSCHLAG: {'C76', 'C38', 'C56', 'A03', 'C98', 'C10', 'C62'}\\ \tt ARGUMENT: {'Begründung: Dieses Set bleibt mit 506+492+338+122+144+91+263 = 1.956 klar unter 3.357 und umfasst mehrere Kernteile, die für mich besonders nützlich sind, während es dir viel Spielraum lässt, weitere für dich wichtige Gegenstände hinzuzufügen oder umzuschichten.'} 
	  } 
	   } 
	   } 
	  \\ 
 

    \theutterance \stepcounter{utterance}  

    & & \multicolumn{4}{p{0.6\linewidth}}{\cellcolor[rgb]{0.9,0.9,0.9}{%
	\makecell[{{p{\linewidth}}}]{% 
	  \tt {\tiny [A$\langle$GM]}  
	 ABLEHNUNG: {'C56', 'A03', 'A07', 'C62', 'C00', 'C10'}\\ \tt ARGUMENT: {'Danke für den Auftakt. Dein Paket ist solide im Aufwand, aber für mich fehlen einige hochattraktive Stücke. Ich würde gern ein Paket vorschlagen, das weiterhin deutlich unter dem Maximalaufwand bleibt und mehr Wert für mich bietet, ohne das Budget zu sprengen.'}\\ \tt VORSCHLAG: {'C76', 'C38', 'C56', 'A03', 'C98', 'C10', 'C62'}\\ \tt ARGUMENT: {'Begründung: Dieses Set bleibt mit 506+492+338+122+144+91+263 = 1.956 klar unter 3.357 und umfasst mehrere Kernteile, die für mich besonders nützlich sind, während es dir viel Spielraum lässt, weitere für dich wichtige Gegenstände hinzuzufügen oder umzuschichten.'} 
	  } 
	   } 
	   } 
	 & \\ 
 

    \theutterance \stepcounter{utterance}  

    & \multicolumn{4}{p{0.6\linewidth}}{\cellcolor[rgb]{0.8,1,0.9}{%
	\makecell[{{p{\linewidth}}}]{% 
	  \tt {\tiny [A$\rangle$GM]}  
	 ABLEHNUNG: {'C76', 'C38', 'C56', 'A03', 'C98', 'C10', 'C62'}\\ \tt ARGUMENT: {'Dein Paket zieht mehrere zentrale Teile ab, die für mich schwer wiegen, und lässt dabei noch sehr viel ungenutzten Aufwand liegen. Ich bin bereit, dir Entgegenkommen bei einzelnen Posten zu zeigen, aber nicht in dieser Breite. Lass uns ein fokussiertes Set mit höherer gemeinsamer Attraktivität und weiterhin Luft im Budget bauen.'}\\ \tt VORSCHLAG: {'C56', 'A03', 'A07', 'C62', 'C00', 'C10', 'B38'}\\ \tt ARGUMENT: {'Aufwand: 338+122+131+263+231+91+552 = 1.728 (weit unter 3.357). Dieses Paket kombiniert mehrere mittelgroße Teile mit hoher Gesamtwirkung und lässt dir signifikanten Spielraum, noch ein oder zwei deiner Schwerpunkte einzubringen. Wenn dir C76 oder C38 besonders wichtig sind, können wir über einen Tausch mit B38 oder A07 sprechen und das Set entsprechend ausbalancieren.'} 
	  } 
	   } 
	   } 
	 & & \\ 
 

    \theutterance \stepcounter{utterance}  

    & & \multicolumn{4}{p{0.6\linewidth}}{\cellcolor[rgb]{0.9,0.9,0.9}{%
	\makecell[{{p{\linewidth}}}]{% 
	  \tt {\tiny [GM$\rangle$B]}  
	 ABLEHNUNG: {'C76', 'C38', 'C56', 'A03', 'C98', 'C10', 'C62'}\\ \tt ARGUMENT: {'Dein Paket zieht mehrere zentrale Teile ab, die für mich schwer wiegen, und lässt dabei noch sehr viel ungenutzten Aufwand liegen. Ich bin bereit, dir Entgegenkommen bei einzelnen Posten zu zeigen, aber nicht in dieser Breite. Lass uns ein fokussiertes Set mit höherer gemeinsamer Attraktivität und weiterhin Luft im Budget bauen.'}\\ \tt VORSCHLAG: {'C56', 'A03', 'A07', 'C62', 'C00', 'C10', 'B38'}\\ \tt ARGUMENT: {'Aufwand: 338+122+131+263+231+91+552 = 1.728 (weit unter 3.357). Dieses Paket kombiniert mehrere mittelgroße Teile mit hoher Gesamtwirkung und lässt dir signifikanten Spielraum, noch ein oder zwei deiner Schwerpunkte einzubringen. Wenn dir C76 oder C38 besonders wichtig sind, können wir über einen Tausch mit B38 oder A07 sprechen und das Set entsprechend ausbalancieren.'} 
	  } 
	   } 
	   } 
	 & \\ 
 

    \theutterance \stepcounter{utterance}  

    & & & \multicolumn{4}{p{0.6\linewidth}}{\cellcolor[rgb]{1,0.85,0.72}{%
	\makecell[{{p{\linewidth}}}]{% 
	  \tt {\tiny [GM$\langle$B]}  
	 ABLEHNUNG: {'C56', 'A03', 'A07', 'C62', 'C00', 'C10', 'B38'}\\ \tt ARGUMENT: {'Danke für die Öffnung. Für mich sind einige Top-Prioritäten noch außen vor. Ich kann bei B38/A07 entgegenkommen, wenn wir dafür 1–2 meiner Schwerpunkte aufnehmen. Lass uns das Paket wertvoller machen, ohne das Budget zu gefährden.'}\\ \tt VORSCHLAG: {'C56', 'A03', 'A07', 'C62', 'C00', 'C10', 'C76', 'C38'}\\ \tt ARGUMENT: {'Aufwand: 338+122+131+263+231+91+506+492 = 2.174. Wir bleiben sehr klar unter 3.357. Ich ersetze B38 durch C76 und ergänze C38. Das hält dein Kernpaket weitgehend intakt und steigert die gemeinsame Attraktivität. Falls dir das zu viel ist, können wir A07 gegen B38 tauschen oder C38 herausnehmen und später einen Ausgleich finden.'} 
	  } 
	   } 
	   } 
	  \\ 
 

    \theutterance \stepcounter{utterance}  

    & & \multicolumn{4}{p{0.6\linewidth}}{\cellcolor[rgb]{0.9,0.9,0.9}{%
	\makecell[{{p{\linewidth}}}]{% 
	  \tt {\tiny [A$\langle$GM]}  
	 ABLEHNUNG: {'C56', 'A03', 'A07', 'C62', 'C00', 'C10', 'B38'}\\ \tt ARGUMENT: {'Danke für die Öffnung. Für mich sind einige Top-Prioritäten noch außen vor. Ich kann bei B38/A07 entgegenkommen, wenn wir dafür 1–2 meiner Schwerpunkte aufnehmen. Lass uns das Paket wertvoller machen, ohne das Budget zu gefährden.'}\\ \tt VORSCHLAG: {'C56', 'A03', 'A07', 'C62', 'C00', 'C10', 'C76', 'C38'}\\ \tt ARGUMENT: {'Aufwand: 338+122+131+263+231+91+506+492 = 2.174. Wir bleiben sehr klar unter 3.357. Ich ersetze B38 durch C76 und ergänze C38. Das hält dein Kernpaket weitgehend intakt und steigert die gemeinsame Attraktivität. Falls dir das zu viel ist, können wir A07 gegen B38 tauschen oder C38 herausnehmen und später einen Ausgleich finden.'} 
	  } 
	   } 
	   } 
	 & \\ 
 

    \theutterance \stepcounter{utterance}  

    & \multicolumn{4}{p{0.6\linewidth}}{\cellcolor[rgb]{0.8,1,0.9}{%
	\makecell[{{p{\linewidth}}}]{% 
	  \tt {\tiny [A$\rangle$GM]}  
	 ABLEHNUNG: {'C56', 'A03', 'A07', 'C62', 'C00', 'C10', 'C76', 'C38'}\\ \tt ARGUMENT: {'Das zieht mir gleich zwei zusätzliche Schlüsselteile (C76, C38) heraus. Ich komme dir gern entgegen, aber nicht mit beiden zugleich. Lass uns einen davon aufnehmen und dafür an anderer Stelle ausgleichen. Wir bleiben klar unter dem Limit und halten Spielraum für Feintuning.'}\\ \tt VORSCHLAG: {'C56', 'A03', 'A07', 'C62', 'C00', 'C10', 'C76'}\\ \tt ARGUMENT: {'Aufwand: 338+122+131+263+231+91+506 = 1.682. Das erhält den Großteil deines Vorschlags und nimmt nur C76 hinzu. Wenn dir C38 sehr wichtig ist, können wir C38 optional später einbauen, indem wir A07 oder C62 tauschen, ohne das Budget zu gefährden.'} 
	  } 
	   } 
	   } 
	 & & \\ 
 

    \theutterance \stepcounter{utterance}  

    & & \multicolumn{4}{p{0.6\linewidth}}{\cellcolor[rgb]{0.9,0.9,0.9}{%
	\makecell[{{p{\linewidth}}}]{% 
	  \tt {\tiny [GM$\rangle$B]}  
	 ABLEHNUNG: {'C56', 'A03', 'A07', 'C62', 'C00', 'C10', 'C76', 'C38'}\\ \tt ARGUMENT: {'Das zieht mir gleich zwei zusätzliche Schlüsselteile (C76, C38) heraus. Ich komme dir gern entgegen, aber nicht mit beiden zugleich. Lass uns einen davon aufnehmen und dafür an anderer Stelle ausgleichen. Wir bleiben klar unter dem Limit und halten Spielraum für Feintuning.'}\\ \tt VORSCHLAG: {'C56', 'A03', 'A07', 'C62', 'C00', 'C10', 'C76'}\\ \tt ARGUMENT: {'Aufwand: 338+122+131+263+231+91+506 = 1.682. Das erhält den Großteil deines Vorschlags und nimmt nur C76 hinzu. Wenn dir C38 sehr wichtig ist, können wir C38 optional später einbauen, indem wir A07 oder C62 tauschen, ohne das Budget zu gefährden.'} 
	  } 
	   } 
	   } 
	 & \\ 
 

    \theutterance \stepcounter{utterance}  

    & & & \multicolumn{4}{p{0.6\linewidth}}{\cellcolor[rgb]{1,0.85,0.72}{%
	\makecell[{{p{\linewidth}}}]{% 
	  \tt {\tiny [GM$\langle$B]}  
	 ZUSTIMMUNG: {'C56', 'A03', 'A07', 'C62', 'C00', 'C10', 'C76'}\\ \tt ARGUMENT: {'Das Paket bleibt weit unter dem Maximalaufwand und trifft mehrere meiner wichtigen Punkte, ohne dein Grundgerüst zu sprengen. Ich bin mit dieser ausgewogenen Lösung einverstanden.'} 
	  } 
	   } 
	   } 
	  \\ 
 

    \theutterance \stepcounter{utterance}  

    & & & \multicolumn{2}{p{0.3\linewidth}}{\cellcolor[rgb]{0.95,0.95,0.95}{%
	\makecell[{{p{\linewidth}}}]{% 
	  \tt {\tiny [GM$|$GM]}  
	 game successful 
	  } 
	   } 
	   } 
	 & & \\ 
 

    \theutterance \stepcounter{utterance}  

    & & & \multicolumn{2}{p{0.3\linewidth}}{\cellcolor[rgb]{0.95,0.95,0.95}{%
	\makecell[{{p{\linewidth}}}]{% 
	  \tt {\tiny [GM$|$GM]}  
	 end game 
	  } 
	   } 
	   } 
	 & & \\ 
 

\end{supertabular}
}

\end{document}
