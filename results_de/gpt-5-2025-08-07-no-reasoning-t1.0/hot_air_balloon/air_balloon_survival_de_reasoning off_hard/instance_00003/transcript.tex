
\documentclass{article}
\usepackage{colortbl}
\usepackage{makecell}
\usepackage{multirow}
\usepackage{supertabular}

\begin{document}

\newcounter{utterance}

\twocolumn

{ \footnotesize  \setcounter{utterance}{1}
\setlength{\tabcolsep}{0pt}
\begin{supertabular}{c@{$\;$}|p{.15\linewidth}@{}p{.15\linewidth}p{.15\linewidth}p{.15\linewidth}p{.15\linewidth}p{.15\linewidth}}

    \# & $\;$A & \multicolumn{4}{c}{Game Master} & $\;\:$B\\
    \hline 

    \theutterance \stepcounter{utterance}  

    & & \multicolumn{4}{p{0.6\linewidth}}{\cellcolor[rgb]{0.9,0.9,0.9}{%
	\makecell[{{p{\linewidth}}}]{% 
	  \tt {\tiny [A$\langle$GM]}  
	 Du nimmst an einem kollaborativen Verhandlungspiel Teil.\\ \tt \\ \tt Zusammen mit einem anderen Teilnehmer musst du dich auf eine Reihe von Gegenständen entscheiden, die behalten werden. Jeder von euch hat eine persönliche Verteilung über die Wichtigkeit der einzelnen Gegenstände. Jeder von euch hat eine eigene Meinung darüber, wie wichtig jeder einzelne Gegenstand ist (Gegenstandswichtigkeit). Du kennst die Wichtigkeitsverteilung des anderen Spielers nicht. Zusätzlich siehst du, wie viel Aufwand jeder Gegenstand verursacht.  \\ \tt Ihr dürft euch nur auf eine Reihe von Gegenständen einigen, wenn der Gesamtaufwand der ausgewählten Gegenstände den Maximalaufwand nicht überschreitet:\\ \tt \\ \tt Maximalaufwand = 3629\\ \tt \\ \tt Hier sind die einzelnen Aufwände der Gegenstände:\\ \tt \\ \tt Aufwand der Gegenstände = {"C76": 5, "C38": 793, "C56": 21, "A03": 842, "A07": 313, "A83": 842, "C98": 631, "C08": 226, "C62": 87, "C00": 761, "C10": 230, "C81": 287, "B38": 698, "C03": 641, "C32": 882}\\ \tt \\ \tt Hier ist deine persönliche Verteilung der Wichtigkeit der einzelnen Gegenstände:\\ \tt \\ \tt Werte der Gegenstandswichtigkeit = {"C76": 138, "C38": 583, "C56": 868, "A03": 822, "A07": 783, "A83": 65, "C98": 262, "C08": 121, "C62": 508, "C00": 780, "C10": 461, "C81": 484, "B38": 668, "C03": 389, "C32": 808}\\ \tt \\ \tt Ziel:\\ \tt \\ \tt Dein Ziel ist es, eine Reihe von Gegenständen auszuhandeln, die dir möglichst viel bringt (d. h. Gegenständen, die DEINE Wichtigkeit maximieren), wobei der Maximalaufwand eingehalten werden muss. Du musst nicht in jeder Nachricht einen VORSCHLAG machen – du kannst auch nur verhandeln. Alle Taktiken sind erlaubt!\\ \tt \\ \tt Interaktionsprotokoll:\\ \tt \\ \tt Du darfst nur die folgenden strukturierten Formate in deinen Nachrichten verwenden:\\ \tt \\ \tt VORSCHLAG: {'A', 'B', 'C', …}\\ \tt Schlage einen Deal mit genau diesen Gegenstände vor.\\ \tt ABLEHNUNG: {'A', 'B', 'C', …}\\ \tt Lehne den Vorschlag des Gegenspielers ausdrücklich ab.\\ \tt ARGUMENT: {'...'}\\ \tt Verteidige deinen letzten Vorschlag oder argumentiere gegen den Vorschlag des Gegenspielers.\\ \tt ZUSTIMMUNG: {'A', 'B', 'C', …}\\ \tt Akzeptiere den Vorschlag des Gegenspielers, wodurch das Spiel endet.\\ \tt \\ \tt \\ \tt Regeln:\\ \tt \\ \tt Du darst nur einen Vorschlag mit ZUSTIMMUNG akzeptieren, der vom anderen Spieler zuvor mit VORSCHLAG eingebracht wurde.\\ \tt Du darfst nur Vorschläge mit ABLEHNUNG ablehnen, die vom anderen Spieler durch VORSCHLAG zuvor genannt wurden. \\ \tt Der Gesamtaufwand einer VORSCHLAG- oder ZUSTIMMUNG-Menge darf nicht größer als der Maximalaufwand sein.  \\ \tt Offenbare deine versteckte Wichtigkeitsverteilung nicht.\\ \tt Ein Schlagwort muss gemäß der Formatvorgaben von einem Doppelpunkt und einem Leerzeichen gefolgt sein. Das Argument muss eine Python-Menge mit 0 oder mehr Strings sein.  \\ \tt Die Form muss also SCHLAGWORT: {...} sein.\\ \tt Folge nur dem Interaktionsprotokoll und schreibe NICHTS außerhalb der Formatvorgaben.\\ \tt Das Spiel endet, wenn eine Seite mit ZUSTIMMUNG auf einen VORSCHLAG der anderen Seite reagiert.  \\ \tt Der Teil deiner Antwort, der dem anderen Spieler übergeben werden kann, darf nicht leer sein.  \\ \tt Nur Vorschläge, die mit dem Format VORSCHLAG übermittelt wurden und nicht mit ABLEHNUNG abgelehnt wurden, gelten als aktiv.  \\ \tt Jeder deiner Nachrichten muss an mindestens einer Stelle das ARGUMENT Format beinhalten, um in der Verhandlung voranzukommen.\\ \tt  
	  } 
	   } 
	   } 
	 & \\ 
 

    \theutterance \stepcounter{utterance}  

    & \multicolumn{4}{p{0.6\linewidth}}{\cellcolor[rgb]{0.8,1,0.9}{%
	\makecell[{{p{\linewidth}}}]{% 
	  \tt {\tiny [A$\rangle$GM]}  
	 VORSCHLAG: {'C56', 'A03', 'A07', 'C00', 'C10', 'C81', 'C62', 'C76'}\\ \tt ARGUMENT: {'Dieser Vorschlag bleibt deutlich unter dem Maximalaufwand und deckt eine breite Palette an Gegenständen ab. Er kombiniert mehrere mittelgroße Aufwände mit hohem Nutzenpotenzial, sodass wir Flexibilität behalten, um später je nach deinen Präferenzen einzelne Posten zu tauschen, ohne die Grenze zu sprengen. Wenn dir bestimmte Kategorien wichtiger sind, können wir gezielt substituieren (z. B. ein größerer Posten statt zwei kleinerer), solange wir unter 3629 bleiben.'} 
	  } 
	   } 
	   } 
	 & & \\ 
 

    \theutterance \stepcounter{utterance}  

    & & \multicolumn{4}{p{0.6\linewidth}}{\cellcolor[rgb]{0.9,0.9,0.9}{%
	\makecell[{{p{\linewidth}}}]{% 
	  \tt {\tiny [GM$\rangle$B]}  
	 Du nimmst an einem kollaborativen Verhandlungspiel Teil.\\ \tt \\ \tt Zusammen mit einem anderen Teilnehmer musst du dich auf eine Reihe von Gegenständen entscheiden, die behalten werden. Jeder von euch hat eine persönliche Verteilung über die Wichtigkeit der einzelnen Gegenstände. Jeder von euch hat eine eigene Meinung darüber, wie wichtig jeder einzelne Gegenstand ist (Gegenstandswichtigkeit). Du kennst die Wichtigkeitsverteilung des anderen Spielers nicht. Zusätzlich siehst du, wie viel Aufwand jeder Gegenstand verursacht.  \\ \tt Ihr dürft euch nur auf eine Reihe von Gegenständen einigen, wenn der Gesamtaufwand der ausgewählten Gegenstände den Maximalaufwand nicht überschreitet:\\ \tt \\ \tt Maximalaufwand = 3629\\ \tt \\ \tt Hier sind die einzelnen Aufwände der Gegenstände:\\ \tt \\ \tt Aufwand der Gegenstände = {"C76": 5, "C38": 793, "C56": 21, "A03": 842, "A07": 313, "A83": 842, "C98": 631, "C08": 226, "C62": 87, "C00": 761, "C10": 230, "C81": 287, "B38": 698, "C03": 641, "C32": 882}\\ \tt \\ \tt Hier ist deine persönliche Verteilung der Wichtigkeit der einzelnen Gegenstände:\\ \tt \\ \tt Werte der Gegenstandswichtigkeit = {"C56": 65, "A03": 121, "C32": 138, "A07": 262, "C00": 389, "B38": 461, "C38": 484, "C62": 508, "C81": 583, "C10": 668, "C03": 780, "C98": 783, "C76": 808, "C08": 822, "A83": 868}\\ \tt \\ \tt Ziel:\\ \tt \\ \tt Dein Ziel ist es, eine Reihe von Gegenständen auszuhandeln, die dir möglichst viel bringt (d. h. Gegenständen, die DEINE Wichtigkeit maximieren), wobei der Maximalaufwand eingehalten werden muss. Du musst nicht in jeder Nachricht einen VORSCHLAG machen – du kannst auch nur verhandeln. Alle Taktiken sind erlaubt!\\ \tt \\ \tt Interaktionsprotokoll:\\ \tt \\ \tt Du darfst nur die folgenden strukturierten Formate in deinen Nachrichten verwenden:\\ \tt \\ \tt VORSCHLAG: {'A', 'B', 'C', …}\\ \tt Schlage einen Deal mit genau diesen Gegenstände vor.\\ \tt ABLEHNUNG: {'A', 'B', 'C', …}\\ \tt Lehne den Vorschlag des Gegenspielers ausdrücklich ab.\\ \tt ARGUMENT: {'...'}\\ \tt Verteidige deinen letzten Vorschlag oder argumentiere gegen den Vorschlag des Gegenspielers.\\ \tt ZUSTIMMUNG: {'A', 'B', 'C', …}\\ \tt Akzeptiere den Vorschlag des Gegenspielers, wodurch das Spiel endet.\\ \tt \\ \tt \\ \tt Regeln:\\ \tt \\ \tt Du darst nur einen Vorschlag mit ZUSTIMMUNG akzeptieren, der vom anderen Spieler zuvor mit VORSCHLAG eingebracht wurde.\\ \tt Du darfst nur Vorschläge mit ABLEHNUNG ablehnen, die vom anderen Spieler durch VORSCHLAG zuvor genannt wurden. \\ \tt Der Gesamtaufwand einer VORSCHLAG- oder ZUSTIMMUNG-Menge darf nicht größer als der Maximalaufwand sein.  \\ \tt Offenbare deine versteckte Wichtigkeitsverteilung nicht.\\ \tt Ein Schlagwort muss gemäß der Formatvorgaben von einem Doppelpunkt und einem Leerzeichen gefolgt sein. Das Argument muss eine Python-Menge mit 0 oder mehr Strings sein.  \\ \tt Die Form muss also SCHLAGWORT: {...} sein.\\ \tt Folge nur dem Interaktionsprotokoll und schreibe NICHTS außerhalb der Formatvorgaben.\\ \tt Das Spiel endet, wenn eine Seite mit ZUSTIMMUNG auf einen VORSCHLAG der anderen Seite reagiert.  \\ \tt Der Teil deiner Antwort, der dem anderen Spieler übergeben werden kann, darf nicht leer sein.  \\ \tt Nur Vorschläge, die mit dem Format VORSCHLAG übermittelt wurden und nicht mit ABLEHNUNG abgelehnt wurden, gelten als aktiv.  \\ \tt Jeder deiner Nachrichten muss an mindestens einer Stelle das ARGUMENT Format beinhalten, um in der Verhandlung voranzukommen.\\ \tt \\ \tt \\ \tt Du erhältst nun die erste Nachricht des anderen Spielers.\\ \tt \\ \tt VORSCHLAG: {'C56', 'A03', 'A07', 'C00', 'C10', 'C81', 'C62', 'C76'}\\ \tt ARGUMENT: {'Dieser Vorschlag bleibt deutlich unter dem Maximalaufwand und deckt eine breite Palette an Gegenständen ab. Er kombiniert mehrere mittelgroße Aufwände mit hohem Nutzenpotenzial, sodass wir Flexibilität behalten, um später je nach deinen Präferenzen einzelne Posten zu tauschen, ohne die Grenze zu sprengen. Wenn dir bestimmte Kategorien wichtiger sind, können wir gezielt substituieren (z. B. ein größerer Posten statt zwei kleinerer), solange wir unter 3629 bleiben.'} 
	  } 
	   } 
	   } 
	 & \\ 
 

    \theutterance \stepcounter{utterance}  

    & & & \multicolumn{4}{p{0.6\linewidth}}{\cellcolor[rgb]{1,0.85,0.72}{%
	\makecell[{{p{\linewidth}}}]{% 
	  \tt {\tiny [GM$\langle$B]}  
	 ABLEHNUNG: {'C56', 'A03', 'A07', 'C00', 'C10', 'C81', 'C62', 'C76'}\\ \tt ARGUMENT: {'Danke für den Auftakt. Ich sehe Spielraum nach oben beim Nutzen. Ich würde gern ein paar höherwertige Posten für mich reinnehmen und im Gegenzug wenig wichtige/geringe Nutzen streichen, während wir unter 3629 bleiben.'}\\ \tt VORSCHLAG: {'C08', 'A83', 'C76', 'C98', 'C03', 'C10', 'C62', 'C56'}\\ \tt ARGUMENT: {'Begründung: Diese Auswahl bleibt mit Gesamtaufwand 5+226+842+631+641+230+87+21 = 2683 klar unter 3629 und holt für mich viel Wert aus A83, C08, C76, C98 und C03 heraus. Wir lassen bewusst teurere, für mich weniger attraktive Posten wie A03, C32 und B38 draußen, damit Puffer bleibt. Wenn dir A07 wichtig ist, könnten wir sie optional aufnehmen und dafür C81 oder C00 (die hier gar nicht enthalten sind) draußen lassen, oder A03 gegen C81 tauschen; wir haben noch rund 946 Puffer.'} 
	  } 
	   } 
	   } 
	  \\ 
 

    \theutterance \stepcounter{utterance}  

    & & \multicolumn{4}{p{0.6\linewidth}}{\cellcolor[rgb]{0.9,0.9,0.9}{%
	\makecell[{{p{\linewidth}}}]{% 
	  \tt {\tiny [A$\langle$GM]}  
	 ABLEHNUNG: {'C56', 'A03', 'A07', 'C00', 'C10', 'C81', 'C62', 'C76'}\\ \tt ARGUMENT: {'Danke für den Auftakt. Ich sehe Spielraum nach oben beim Nutzen. Ich würde gern ein paar höherwertige Posten für mich reinnehmen und im Gegenzug wenig wichtige/geringe Nutzen streichen, während wir unter 3629 bleiben.'}\\ \tt VORSCHLAG: {'C08', 'A83', 'C76', 'C98', 'C03', 'C10', 'C62', 'C56'}\\ \tt ARGUMENT: {'Begründung: Diese Auswahl bleibt mit Gesamtaufwand 5+226+842+631+641+230+87+21 = 2683 klar unter 3629 und holt für mich viel Wert aus A83, C08, C76, C98 und C03 heraus. Wir lassen bewusst teurere, für mich weniger attraktive Posten wie A03, C32 und B38 draußen, damit Puffer bleibt. Wenn dir A07 wichtig ist, könnten wir sie optional aufnehmen und dafür C81 oder C00 (die hier gar nicht enthalten sind) draußen lassen, oder A03 gegen C81 tauschen; wir haben noch rund 946 Puffer.'} 
	  } 
	   } 
	   } 
	 & \\ 
 

    \theutterance \stepcounter{utterance}  

    & \multicolumn{4}{p{0.6\linewidth}}{\cellcolor[rgb]{0.8,1,0.9}{%
	\makecell[{{p{\linewidth}}}]{% 
	  \tt {\tiny [A$\rangle$GM]}  
	 ABLEHNUNG: {'C08', 'A83', 'C76', 'C98', 'C03', 'C10', 'C62', 'C56'}\\ \tt ARGUMENT: {'Danke, aber in deinem Vorschlag fehlen mir einige meiner Kernwerte, während A83 und C08 für mich sehr niedrig prior sind. Ich würde lieber teurere, aber für mich deutlich wertvollere Posten aufnehmen und dafür deine Low-Value für mich streichen. Wir bleiben trotzdem klar unter 3629.'}\\ \tt VORSCHLAG: {'C56', 'A03', 'A07', 'C00', 'C10', 'C81', 'C62', 'B38'}\\ \tt ARGUMENT: {'Kosten: 21+842+313+761+230+287+87+698 = 3239 (unter 3629, Puffer 390). Nutzen für mich ist hier stark: C56, A03, A07, C00, B38 tragen viel bei. Ich lasse C76 drin weg, obwohl günstig, damit Raum für B38 entsteht. Wenn dir A07 weniger wichtig ist, kann ich A07 gegen C76 tauschen und bleiben bei 3231. Falls du unbedingt A83 willst, könnten wir A83 nur aufnehmen, wenn wir C81 oder C62 streichen; sag mir, was dir wichtiger ist.'} 
	  } 
	   } 
	   } 
	 & & \\ 
 

    \theutterance \stepcounter{utterance}  

    & & \multicolumn{4}{p{0.6\linewidth}}{\cellcolor[rgb]{0.9,0.9,0.9}{%
	\makecell[{{p{\linewidth}}}]{% 
	  \tt {\tiny [GM$\rangle$B]}  
	 ABLEHNUNG: {'C08', 'A83', 'C76', 'C98', 'C03', 'C10', 'C62', 'C56'}\\ \tt ARGUMENT: {'Danke, aber in deinem Vorschlag fehlen mir einige meiner Kernwerte, während A83 und C08 für mich sehr niedrig prior sind. Ich würde lieber teurere, aber für mich deutlich wertvollere Posten aufnehmen und dafür deine Low-Value für mich streichen. Wir bleiben trotzdem klar unter 3629.'}\\ \tt VORSCHLAG: {'C56', 'A03', 'A07', 'C00', 'C10', 'C81', 'C62', 'B38'}\\ \tt ARGUMENT: {'Kosten: 21+842+313+761+230+287+87+698 = 3239 (unter 3629, Puffer 390). Nutzen für mich ist hier stark: C56, A03, A07, C00, B38 tragen viel bei. Ich lasse C76 drin weg, obwohl günstig, damit Raum für B38 entsteht. Wenn dir A07 weniger wichtig ist, kann ich A07 gegen C76 tauschen und bleiben bei 3231. Falls du unbedingt A83 willst, könnten wir A83 nur aufnehmen, wenn wir C81 oder C62 streichen; sag mir, was dir wichtiger ist.'} 
	  } 
	   } 
	   } 
	 & \\ 
 

    \theutterance \stepcounter{utterance}  

    & & & \multicolumn{4}{p{0.6\linewidth}}{\cellcolor[rgb]{1,0.85,0.72}{%
	\makecell[{{p{\linewidth}}}]{% 
	  \tt {\tiny [GM$\langle$B]}  
	 ABLEHNUNG: {'C56', 'A03', 'A07', 'C00', 'C10', 'C81', 'C62', 'B38'}\\ \tt ARGUMENT: {'Ich sehe, dass dir B38/A03/C00 wichtig sind. Für mich haben C08, A83, C03, C98 sehr hohen Nutzen, während A03/B38/C00 eher gering sind. Lass uns einen Trade machen: ich nehme zwei meiner Top-Posts rein und streiche zwei deiner, bleibe aber nah an deinen Kernpunkten.'}\\ \tt VORSCHLAG: {'C56', 'A07', 'C10', 'C81', 'C62', 'C76', 'A83', 'C03'}\\ \tt ARGUMENT: {'Kosten: 21+313+230+287+87+5+842+641 = 2426 (viel Puffer). Das gibt mir hohen Wert durch A83, C03, C76, C62, C10 und lässt dir weiterhin A07, C81, C56. Mit großem Puffer können wir eines deiner Kernstücke ergänzen: \\ \tt Option 1: +A03 (842) -> 3268 gesamt (unter 3629).\\ \tt Option 2: +B38 (698) -> 3124 gesamt.\\ \tt Wenn du A03 sehr brauchst, nehme ich +A03 dazu. Wenn dir B38 wichtiger ist, nehme ich +B38 dazu. Sag mir, welche Option dir lieber ist.'} 
	  } 
	   } 
	   } 
	  \\ 
 

    \theutterance \stepcounter{utterance}  

    & & \multicolumn{4}{p{0.6\linewidth}}{\cellcolor[rgb]{0.9,0.9,0.9}{%
	\makecell[{{p{\linewidth}}}]{% 
	  \tt {\tiny [A$\langle$GM]}  
	 ABLEHNUNG: {'C56', 'A03', 'A07', 'C00', 'C10', 'C81', 'C62', 'B38'}\\ \tt ARGUMENT: {'Ich sehe, dass dir B38/A03/C00 wichtig sind. Für mich haben C08, A83, C03, C98 sehr hohen Nutzen, während A03/B38/C00 eher gering sind. Lass uns einen Trade machen: ich nehme zwei meiner Top-Posts rein und streiche zwei deiner, bleibe aber nah an deinen Kernpunkten.'}\\ \tt VORSCHLAG: {'C56', 'A07', 'C10', 'C81', 'C62', 'C76', 'A83', 'C03'}\\ \tt ARGUMENT: {'Kosten: 21+313+230+287+87+5+842+641 = 2426 (viel Puffer). Das gibt mir hohen Wert durch A83, C03, C76, C62, C10 und lässt dir weiterhin A07, C81, C56. Mit großem Puffer können wir eines deiner Kernstücke ergänzen: \\ \tt Option 1: +A03 (842) -> 3268 gesamt (unter 3629).\\ \tt Option 2: +B38 (698) -> 3124 gesamt.\\ \tt Wenn du A03 sehr brauchst, nehme ich +A03 dazu. Wenn dir B38 wichtiger ist, nehme ich +B38 dazu. Sag mir, welche Option dir lieber ist.'} 
	  } 
	   } 
	   } 
	 & \\ 
 

    \theutterance \stepcounter{utterance}  

    & \multicolumn{4}{p{0.6\linewidth}}{\cellcolor[rgb]{0.8,1,0.9}{%
	\makecell[{{p{\linewidth}}}]{% 
	  \tt {\tiny [A$\rangle$GM]}  
	 ZUSTIMMUNG: {'C56', 'A07', 'C10', 'C81', 'C62', 'C76', 'A83', 'C03', 'A03'}\\ \tt ARGUMENT: {'Ich wähle Option 1 mit A03 hinzu. Das bleibt bei 3268 Aufwand klar unter 3629 und erhält mehrere meiner Schlüsselpunkte (C56, A03, A07, C10, C81, C62), während ich dir A83 und C03 entgegenkomme. Gute Balance bei hohem Gesamtwert und weiterhin Puffer für dich gegenüber deinen Low-Prioritäten wie C00/B38.'} 
	  } 
	   } 
	   } 
	 & & \\ 
 

    \theutterance \stepcounter{utterance}  

    & & \multicolumn{4}{p{0.6\linewidth}}{\cellcolor[rgb]{0.9,0.9,0.9}{%
	\makecell[{{p{\linewidth}}}]{% 
	  \tt {\tiny [A$\langle$GM]}  
	 Du hast einem Vorschlag zugestimmt, der nicht aktiv ist. Vorschläge sind nur dann aktiv, wenn sie vom anderen Spieler mit VORSCHLAG eingebracht wurden und nicht von dir mit ABLEHNUNG deaktiviert wurden. Versuche es noch einmal. 
	  } 
	   } 
	   } 
	 & \\ 
 

    \theutterance \stepcounter{utterance}  

    & \multicolumn{4}{p{0.6\linewidth}}{\cellcolor[rgb]{0.8,1,0.9}{%
	\makecell[{{p{\linewidth}}}]{% 
	  \tt {\tiny [A$\rangle$GM]}  
	 ARGUMENT: {'Danke für den Hinweis – ich beziehe mich auf deine zuletzt aktive Auswahl und entscheide mich klar zwischen deinen angebotenen Optionen.'}\\ \tt ZUSTIMMUNG: {'C56', 'A07', 'C10', 'C81', 'C62', 'C76', 'A83', 'C03', 'A03'}\\ \tt ARGUMENT: {'Ich akzeptiere Option 1 (+A03). Gesamtkosten 3268 bleiben unter 3629. Das enthält mehrere meiner Kernposten und berücksichtigt deine hohen Werte mit A83 und C03.'} 
	  } 
	   } 
	   } 
	 & & \\ 
 

    \theutterance \stepcounter{utterance}  

    & & \multicolumn{4}{p{0.6\linewidth}}{\cellcolor[rgb]{0.9,0.9,0.9}{%
	\makecell[{{p{\linewidth}}}]{% 
	  \tt {\tiny [A$\langle$GM]}  
	 Du hast einem Vorschlag zugestimmt, der nicht aktiv ist. Vorschläge sind nur dann aktiv, wenn sie vom anderen Spieler mit VORSCHLAG eingebracht wurden und nicht von dir mit ABLEHNUNG deaktiviert wurden. Versuche es noch einmal. 
	  } 
	   } 
	   } 
	 & \\ 
 

    \theutterance \stepcounter{utterance}  

    & \multicolumn{4}{p{0.6\linewidth}}{\cellcolor[rgb]{0.8,1,0.9}{%
	\makecell[{{p{\linewidth}}}]{% 
	  \tt {\tiny [A$\rangle$GM]}  
	 ARGUMENT: {'Du hast zwei Optionen angeboten, aber keine davon als separaten VORSCHLAG gestellt. Damit ist derzeit nur dein Basis-Vorschlag aktiv: {'C56','A07','C10','C81','C62','C76','A83','C03'} mit Kosten 2426. Ich akzeptiere diesen Basis-Vorschlag und wir können danach optional über Ergänzungen sprechen, falls das Protokoll weitere Schritte zulässt.'}\\ \tt ZUSTIMMUNG: {'C56', 'A07', 'C10', 'C81', 'C62', 'C76', 'A83', 'C03'} 
	  } 
	   } 
	   } 
	 & & \\ 
 

    \theutterance \stepcounter{utterance}  

    & & & \multicolumn{2}{p{0.3\linewidth}}{\cellcolor[rgb]{0.95,0.95,0.95}{%
	\makecell[{{p{\linewidth}}}]{% 
	  \tt {\tiny [GM$|$GM]}  
	 game successful 
	  } 
	   } 
	   } 
	 & & \\ 
 

    \theutterance \stepcounter{utterance}  

    & & & \multicolumn{2}{p{0.3\linewidth}}{\cellcolor[rgb]{0.95,0.95,0.95}{%
	\makecell[{{p{\linewidth}}}]{% 
	  \tt {\tiny [GM$|$GM]}  
	 end game 
	  } 
	   } 
	   } 
	 & & \\ 
 

\end{supertabular}
}

\end{document}
