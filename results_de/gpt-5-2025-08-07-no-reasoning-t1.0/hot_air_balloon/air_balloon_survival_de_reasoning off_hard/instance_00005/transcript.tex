
\documentclass{article}
\usepackage{colortbl}
\usepackage{makecell}
\usepackage{multirow}
\usepackage{supertabular}

\begin{document}

\newcounter{utterance}

\centering \large Interaction Transcript for game `hot\_air\_balloon', experiment `air\_balloon\_survival\_de\_reasoning off\_hard', episode 5 with gpt{-}5{-}2025{-}08{-}07{-}no{-}reasoning{-}t1.0.
\vspace{24pt}

{ \footnotesize  \setcounter{utterance}{1}
\setlength{\tabcolsep}{0pt}
\begin{supertabular}{c@{$\;$}|p{.15\linewidth}@{}p{.15\linewidth}p{.15\linewidth}p{.15\linewidth}p{.15\linewidth}p{.15\linewidth}}
    \# & $\;$A & \multicolumn{4}{c}{Game Master} & $\;\:$B\\
    \hline

    \theutterance \stepcounter{utterance}  
    & & \multicolumn{4}{p{0.6\linewidth}}{
        \cellcolor[rgb]{0.9,0.9,0.9}{
            \makecell[{{p{\linewidth}}}]{
                \texttt{\tiny{[P1$\langle$GM]}}
                \texttt{Du nimmst an einem kollaborativen Verhandlungspiel Teil.} \\
\\ 
\texttt{Zusammen mit einem anderen Teilnehmer musst du dich auf eine Reihe von Gegenständen entscheiden, die behalten werden. Jeder von euch hat eine persönliche Verteilung über die Wichtigkeit der einzelnen Gegenstände. Jeder von euch hat eine eigene Meinung darüber, wie wichtig jeder einzelne Gegenstand ist (Gegenstandswichtigkeit). Du kennst die Wichtigkeitsverteilung des anderen Spielers nicht. Zusätzlich siehst du, wie viel Aufwand jeder Gegenstand verursacht.  } \\
\texttt{Ihr dürft euch nur auf eine Reihe von Gegenständen einigen, wenn der Gesamtaufwand der ausgewählten Gegenstände den Maximalaufwand nicht überschreitet:} \\
\\ 
\texttt{Maximalaufwand = 2307} \\
\\ 
\texttt{Hier sind die einzelnen Aufwände der Gegenstände:} \\
\\ 
\texttt{Aufwand der Gegenstände = \{"C76": 147, "C38": 698, "C56": 588, "A03": 43, "A07": 356, "A83": 80, "C98": 95, "C08": 743, "C62": 106, "C00": 308, "C10": 325, "C81": 255, "B38": 276, "C03": 543, "C32": 51\}} \\
\\ 
\texttt{Hier ist deine persönliche Verteilung der Wichtigkeit der einzelnen Gegenstände:} \\
\\ 
\texttt{Werte der Gegenstandswichtigkeit = \{"C76": 138, "C38": 583, "C56": 65, "A03": 262, "A07": 121, "A83": 508, "C98": 461, "C08": 484, "C62": 668, "C00": 389, "C10": 215, "C81": 97, "B38": 500, "C03": 30, "C32": 400\}} \\
\\ 
\texttt{Ziel:} \\
\\ 
\texttt{Dein Ziel ist es, eine Reihe von Gegenständen auszuhandeln, die dir möglichst viel bringt (d. h. Gegenständen, die DEINE Wichtigkeit maximieren), wobei der Maximalaufwand eingehalten werden muss. Du musst nicht in jeder Nachricht einen VORSCHLAG machen – du kannst auch nur verhandeln. Alle Taktiken sind erlaubt!} \\
\\ 
\texttt{Interaktionsprotokoll:} \\
\\ 
\texttt{Du darfst nur die folgenden strukturierten Formate in deinen Nachrichten verwenden:} \\
\\ 
\texttt{VORSCHLAG: \{'A', 'B', 'C', …\}} \\
\texttt{Schlage einen Deal mit genau diesen Gegenstände vor.} \\
\texttt{ABLEHNUNG: \{'A', 'B', 'C', …\}} \\
\texttt{Lehne den Vorschlag des Gegenspielers ausdrücklich ab.} \\
\texttt{ARGUMENT: \{'...'\}} \\
\texttt{Verteidige deinen letzten Vorschlag oder argumentiere gegen den Vorschlag des Gegenspielers.} \\
\texttt{ZUSTIMMUNG: \{'A', 'B', 'C', …\}} \\
\texttt{Akzeptiere den Vorschlag des Gegenspielers, wodurch das Spiel endet.} \\
\\ 
\\ 
\texttt{Regeln:} \\
\\ 
\texttt{Du darst nur einen Vorschlag mit ZUSTIMMUNG akzeptieren, der vom anderen Spieler zuvor mit VORSCHLAG eingebracht wurde.} \\
\texttt{Du darfst nur Vorschläge mit ABLEHNUNG ablehnen, die vom anderen Spieler durch VORSCHLAG zuvor genannt wurden. } \\
\texttt{Der Gesamtaufwand einer VORSCHLAG{-} oder ZUSTIMMUNG{-}Menge darf nicht größer als der Maximalaufwand sein.  } \\
\texttt{Offenbare deine versteckte Wichtigkeitsverteilung nicht.} \\
\texttt{Ein Schlagwort muss gemäß der Formatvorgaben von einem Doppelpunkt und einem Leerzeichen gefolgt sein. Das Argument muss eine Python{-}Menge mit 0 oder mehr Strings sein.  } \\
\texttt{Die Form muss also SCHLAGWORT: \{...\} sein.} \\
\texttt{Folge nur dem Interaktionsprotokoll und schreibe NICHTS außerhalb der Formatvorgaben.} \\
\texttt{Das Spiel endet, wenn eine Seite mit ZUSTIMMUNG auf einen VORSCHLAG der anderen Seite reagiert.  } \\
\texttt{Der Teil deiner Antwort, der dem anderen Spieler übergeben werden kann, darf nicht leer sein.  } \\
\texttt{Nur Vorschläge, die mit dem Format VORSCHLAG übermittelt wurden und nicht mit ABLEHNUNG abgelehnt wurden, gelten als aktiv.  } \\
\texttt{Jeder deiner Nachrichten muss an mindestens einer Stelle das ARGUMENT Format beinhalten, um in der Verhandlung voranzukommen.} \\
            }
        }
    }
    & \\ \\

    \theutterance \stepcounter{utterance}  
    & \multicolumn{4}{p{0.6\linewidth}}{
        \cellcolor[rgb]{0.9,0.9,0.9}{
            \makecell[{{p{\linewidth}}}]{
                \texttt{\tiny{[P1$\rangle$GM]}}
                \texttt{VORSCHLAG: \{'C62', 'A83', 'B38', 'C98', 'A03', 'C76', 'C32'\}} \\
\texttt{ARGUMENT: \{'Dieser Vorschlag bleibt klar unter dem Maximalaufwand und deckt viele hochwertige Positionen ab. Er kombiniert mehrere mittelgroße Aufwände mit sehr hohem Nutzen und vermeidet teure, ungewisse Posten. Wenn dir ein bestimmter Gegenstand wichtiger ist, sag Bescheid – ich kann eine gezielte Anpassung prüfen, solange wir unter dem Aufwandslimit bleiben.'\}} \\
            }
        }
    }
    & & \\ \\

    \theutterance \stepcounter{utterance}  
    & & \multicolumn{4}{p{0.6\linewidth}}{
        \cellcolor[rgb]{0.9,0.9,0.9}{
            \makecell[{{p{\linewidth}}}]{
                \texttt{\tiny{[GM$\rangle$P2]}}
                \texttt{Du nimmst an einem kollaborativen Verhandlungspiel Teil.} \\
\\ 
\texttt{Zusammen mit einem anderen Teilnehmer musst du dich auf eine Reihe von Gegenständen entscheiden, die behalten werden. Jeder von euch hat eine persönliche Verteilung über die Wichtigkeit der einzelnen Gegenstände. Jeder von euch hat eine eigene Meinung darüber, wie wichtig jeder einzelne Gegenstand ist (Gegenstandswichtigkeit). Du kennst die Wichtigkeitsverteilung des anderen Spielers nicht. Zusätzlich siehst du, wie viel Aufwand jeder Gegenstand verursacht.  } \\
\texttt{Ihr dürft euch nur auf eine Reihe von Gegenständen einigen, wenn der Gesamtaufwand der ausgewählten Gegenstände den Maximalaufwand nicht überschreitet:} \\
\\ 
\texttt{Maximalaufwand = 2307} \\
\\ 
\texttt{Hier sind die einzelnen Aufwände der Gegenstände:} \\
\\ 
\texttt{Aufwand der Gegenstände = \{"C76": 147, "C38": 698, "C56": 588, "A03": 43, "A07": 356, "A83": 80, "C98": 95, "C08": 743, "C62": 106, "C00": 308, "C10": 325, "C81": 255, "B38": 276, "C03": 543, "C32": 51\}} \\
\\ 
\texttt{Hier ist deine persönliche Verteilung der Wichtigkeit der einzelnen Gegenstände:} \\
\\ 
\texttt{Werte der Gegenstandswichtigkeit = \{"C62": 30, "C38": 65, "A83": 97, "B38": 121, "C08": 138, "C98": 215, "C32": 262, "C00": 389, "A03": 400, "C10": 461, "C76": 484, "A07": 500, "C81": 508, "C56": 583, "C03": 668\}} \\
\\ 
\texttt{Ziel:} \\
\\ 
\texttt{Dein Ziel ist es, eine Reihe von Gegenständen auszuhandeln, die dir möglichst viel bringt (d. h. Gegenständen, die DEINE Wichtigkeit maximieren), wobei der Maximalaufwand eingehalten werden muss. Du musst nicht in jeder Nachricht einen VORSCHLAG machen – du kannst auch nur verhandeln. Alle Taktiken sind erlaubt!} \\
\\ 
\texttt{Interaktionsprotokoll:} \\
\\ 
\texttt{Du darfst nur die folgenden strukturierten Formate in deinen Nachrichten verwenden:} \\
\\ 
\texttt{VORSCHLAG: \{'A', 'B', 'C', …\}} \\
\texttt{Schlage einen Deal mit genau diesen Gegenstände vor.} \\
\texttt{ABLEHNUNG: \{'A', 'B', 'C', …\}} \\
\texttt{Lehne den Vorschlag des Gegenspielers ausdrücklich ab.} \\
\texttt{ARGUMENT: \{'...'\}} \\
\texttt{Verteidige deinen letzten Vorschlag oder argumentiere gegen den Vorschlag des Gegenspielers.} \\
\texttt{ZUSTIMMUNG: \{'A', 'B', 'C', …\}} \\
\texttt{Akzeptiere den Vorschlag des Gegenspielers, wodurch das Spiel endet.} \\
\\ 
\\ 
\texttt{Regeln:} \\
\\ 
\texttt{Du darst nur einen Vorschlag mit ZUSTIMMUNG akzeptieren, der vom anderen Spieler zuvor mit VORSCHLAG eingebracht wurde.} \\
\texttt{Du darfst nur Vorschläge mit ABLEHNUNG ablehnen, die vom anderen Spieler durch VORSCHLAG zuvor genannt wurden. } \\
\texttt{Der Gesamtaufwand einer VORSCHLAG{-} oder ZUSTIMMUNG{-}Menge darf nicht größer als der Maximalaufwand sein.  } \\
\texttt{Offenbare deine versteckte Wichtigkeitsverteilung nicht.} \\
\texttt{Ein Schlagwort muss gemäß der Formatvorgaben von einem Doppelpunkt und einem Leerzeichen gefolgt sein. Das Argument muss eine Python{-}Menge mit 0 oder mehr Strings sein.  } \\
\texttt{Die Form muss also SCHLAGWORT: \{...\} sein.} \\
\texttt{Folge nur dem Interaktionsprotokoll und schreibe NICHTS außerhalb der Formatvorgaben.} \\
\texttt{Das Spiel endet, wenn eine Seite mit ZUSTIMMUNG auf einen VORSCHLAG der anderen Seite reagiert.  } \\
\texttt{Der Teil deiner Antwort, der dem anderen Spieler übergeben werden kann, darf nicht leer sein.  } \\
\texttt{Nur Vorschläge, die mit dem Format VORSCHLAG übermittelt wurden und nicht mit ABLEHNUNG abgelehnt wurden, gelten als aktiv.  } \\
\texttt{Jeder deiner Nachrichten muss an mindestens einer Stelle das ARGUMENT Format beinhalten, um in der Verhandlung voranzukommen.} \\
\\ 
\\ 
\texttt{Du erhältst nun die erste Nachricht des anderen Spielers.} \\
\\ 
\texttt{VORSCHLAG: \{'C62', 'A83', 'B38', 'C98', 'A03', 'C76', 'C32'\}} \\
\texttt{ARGUMENT: \{'Dieser Vorschlag bleibt klar unter dem Maximalaufwand und deckt viele hochwertige Positionen ab. Er kombiniert mehrere mittelgroße Aufwände mit sehr hohem Nutzen und vermeidet teure, ungewisse Posten. Wenn dir ein bestimmter Gegenstand wichtiger ist, sag Bescheid – ich kann eine gezielte Anpassung prüfen, solange wir unter dem Aufwandslimit bleiben.'\}} \\
            }
        }
    }
    & \\ \\

    \theutterance \stepcounter{utterance}  
    & & & \multicolumn{4}{p{0.6\linewidth}}{
        \cellcolor[rgb]{0.9,0.9,0.9}{
            \makecell[{{p{\linewidth}}}]{
                \texttt{\tiny{[GM$\langle$P2]}}
                \texttt{ABLEHNUNG: \{'C62', 'A83', 'B38', 'C98', 'A03', 'C76', 'C32'\}} \\
\texttt{ARGUMENT: \{'Danke für den Auftakt! Dein Paket ist solide, aber es lässt mehrere für mich sehr attraktive Posten mit gutem Aufwand{-}Nutzen{-}Verhältnis außen vor (z. B. C10, A07, C56, C03). Ich würde gern mehr Gesamtwert reinholen, ohne das Limit zu reißen.'\}} \\
\texttt{VORSCHLAG: \{'A03', 'A83', 'C98', 'C62', 'C32', 'C10', 'C76', 'A07'\}} \\
\texttt{ARGUMENT: \{'Begründung: Dieses Set bleibt mit 43+80+95+106+51+325+147+356=1203 klar unter 2307 und hebt gezielt starke Positionen bei moderatem Aufwand rein. Es lässt uns Spielraum, später bei Bedarf noch einzelne Posten zu tauschen oder zu ergänzen, falls dir etwas besonders wichtig ist.'\}} \\
            }
        }
    }
     \\ \\

    \theutterance \stepcounter{utterance}  
    & & \multicolumn{4}{p{0.6\linewidth}}{
        \cellcolor[rgb]{0.9,0.9,0.9}{
            \makecell[{{p{\linewidth}}}]{
                \texttt{\tiny{[P1$\langle$GM]}}
                \texttt{ABLEHNUNG: \{'C62', 'A83', 'B38', 'C98', 'A03', 'C76', 'C32'\}} \\
\texttt{ARGUMENT: \{'Danke für den Auftakt! Dein Paket ist solide, aber es lässt mehrere für mich sehr attraktive Posten mit gutem Aufwand{-}Nutzen{-}Verhältnis außen vor (z. B. C10, A07, C56, C03). Ich würde gern mehr Gesamtwert reinholen, ohne das Limit zu reißen.'\}} \\
\texttt{VORSCHLAG: \{'A03', 'A83', 'C98', 'C62', 'C32', 'C10', 'C76', 'A07'\}} \\
\texttt{ARGUMENT: \{'Begründung: Dieses Set bleibt mit 43+80+95+106+51+325+147+356=1203 klar unter 2307 und hebt gezielt starke Positionen bei moderatem Aufwand rein. Es lässt uns Spielraum, später bei Bedarf noch einzelne Posten zu tauschen oder zu ergänzen, falls dir etwas besonders wichtig ist.'\}} \\
            }
        }
    }
    & \\ \\

    \theutterance \stepcounter{utterance}  
    & \multicolumn{4}{p{0.6\linewidth}}{
        \cellcolor[rgb]{0.9,0.9,0.9}{
            \makecell[{{p{\linewidth}}}]{
                \texttt{\tiny{[P1$\rangle$GM]}}
                \texttt{ABLEHNUNG: \{'A03', 'A83', 'C98', 'C62', 'C32', 'C10', 'C76', 'A07'\}} \\
\texttt{ARGUMENT: \{'Dein Set ist aufwandsarm, aber es lässt viel Budget ungenutzt und schiebt zwei für mich schwächere Posten (A07, C10) rein, während sehr wertstarke Alternativen fehlen (z. B. B38, C08, C00). Lass uns das Budget gezielter in höherwertige Stücke lenken.'\}} \\
\texttt{VORSCHLAG: \{'A03', 'A83', 'C98', 'C62', 'C32', 'C76', 'B38', 'C00', 'C08'\}} \\
\texttt{ARGUMENT: \{'Aufwand: 43+80+95+106+51+147+276+308+743=1849 (unter 2307). Das hebt mehrere starke Positionen bei weiterhin solidem Puffer. Falls dir C08 zu teuer ist, können wir stattdessen C10 aufnehmen und C08 rausnehmen; dann landen wir bei 1431 Aufwand und halten weiter viele wertige Posten für beide Seiten offen.'\}} \\
            }
        }
    }
    & & \\ \\

    \theutterance \stepcounter{utterance}  
    & & \multicolumn{4}{p{0.6\linewidth}}{
        \cellcolor[rgb]{0.9,0.9,0.9}{
            \makecell[{{p{\linewidth}}}]{
                \texttt{\tiny{[GM$\rangle$P2]}}
                \texttt{ABLEHNUNG: \{'A03', 'A83', 'C98', 'C62', 'C32', 'C10', 'C76', 'A07'\}} \\
\texttt{ARGUMENT: \{'Dein Set ist aufwandsarm, aber es lässt viel Budget ungenutzt und schiebt zwei für mich schwächere Posten (A07, C10) rein, während sehr wertstarke Alternativen fehlen (z. B. B38, C08, C00). Lass uns das Budget gezielter in höherwertige Stücke lenken.'\}} \\
\texttt{VORSCHLAG: \{'A03', 'A83', 'C98', 'C62', 'C32', 'C76', 'B38', 'C00', 'C08'\}} \\
\texttt{ARGUMENT: \{'Aufwand: 43+80+95+106+51+147+276+308+743=1849 (unter 2307). Das hebt mehrere starke Positionen bei weiterhin solidem Puffer. Falls dir C08 zu teuer ist, können wir stattdessen C10 aufnehmen und C08 rausnehmen; dann landen wir bei 1431 Aufwand und halten weiter viele wertige Posten für beide Seiten offen.'\}} \\
            }
        }
    }
    & \\ \\

    \theutterance \stepcounter{utterance}  
    & & & \multicolumn{4}{p{0.6\linewidth}}{
        \cellcolor[rgb]{0.9,0.9,0.9}{
            \makecell[{{p{\linewidth}}}]{
                \texttt{\tiny{[GM$\langle$P2]}}
                \texttt{ABLEHNUNG: \{'A03', 'A83', 'C98', 'C62', 'C32', 'C76', 'B38', 'C00', 'C08'\}} \\
\texttt{ARGUMENT: \{'Ich schätze die Richtung, aber C08 frisst viel Budget für relativ wenig Zusatznutzen aus meiner Sicht. Wenn wir den Puffer smarter einsetzen, können wir deutlich mehr Gesamtwert reinholen. Ich priorisiere einige andere Posten stärker als B38/C08.'\}} \\
\texttt{VORSCHLAG: \{'A03', 'A83', 'C98', 'C62', 'C32', 'C76', 'C00', 'C10', 'C81', 'C56'\}} \\
\texttt{ARGUMENT: \{'Aufwand: 43+80+95+106+51+147+308+325+255+588 = 1998 (unter 2307). Das ersetzt C08/B38 durch C10/C81/C56 und hebt den Gesamtwert deutlich bei weiterhin gesundem Puffer. Wenn dir ein Tausch nötig ist, können wir C10 gegen B38 tauschen (dann 1949) oder C81 gegen B38 (2022), je nachdem, was dir wichtiger ist.'\}} \\
            }
        }
    }
     \\ \\

    \theutterance \stepcounter{utterance}  
    & & \multicolumn{4}{p{0.6\linewidth}}{
        \cellcolor[rgb]{0.9,0.9,0.9}{
            \makecell[{{p{\linewidth}}}]{
                \texttt{\tiny{[P1$\langle$GM]}}
                \texttt{ABLEHNUNG: \{'A03', 'A83', 'C98', 'C62', 'C32', 'C76', 'B38', 'C00', 'C08'\}} \\
\texttt{ARGUMENT: \{'Ich schätze die Richtung, aber C08 frisst viel Budget für relativ wenig Zusatznutzen aus meiner Sicht. Wenn wir den Puffer smarter einsetzen, können wir deutlich mehr Gesamtwert reinholen. Ich priorisiere einige andere Posten stärker als B38/C08.'\}} \\
\texttt{VORSCHLAG: \{'A03', 'A83', 'C98', 'C62', 'C32', 'C76', 'C00', 'C10', 'C81', 'C56'\}} \\
\texttt{ARGUMENT: \{'Aufwand: 43+80+95+106+51+147+308+325+255+588 = 1998 (unter 2307). Das ersetzt C08/B38 durch C10/C81/C56 und hebt den Gesamtwert deutlich bei weiterhin gesundem Puffer. Wenn dir ein Tausch nötig ist, können wir C10 gegen B38 tauschen (dann 1949) oder C81 gegen B38 (2022), je nachdem, was dir wichtiger ist.'\}} \\
            }
        }
    }
    & \\ \\

    \theutterance \stepcounter{utterance}  
    & \multicolumn{4}{p{0.6\linewidth}}{
        \cellcolor[rgb]{0.9,0.9,0.9}{
            \makecell[{{p{\linewidth}}}]{
                \texttt{\tiny{[P1$\rangle$GM]}}
                \texttt{ABLEHNUNG: \{'A03', 'A83', 'C98', 'C62', 'C32', 'C76', 'C00', 'C10', 'C81', 'C56'\}} \\
\texttt{ARGUMENT: \{'Danke für die Anpassung – deutlich näher dran. Für mich sind C10 und C81 vergleichsweise schwach. Wenn wir dieses Budget auf sehr starke Alternativen legen, bekommen wir mehr gemeinsamen Wert raus. Ich würde B38 klar vor C10/C81 sehen.'\}} \\
\texttt{VORSCHLAG: \{'A03', 'A83', 'C98', 'C62', 'C32', 'C76', 'C00', 'B38', 'C56'\}} \\
\texttt{ARGUMENT: \{'Aufwand: 43+80+95+106+51+147+308+276+588 = 1694 (unter 2307). Das packt die für mich wertstarken Kerne rein und lässt dir 613 Aufwand Puffer für einen gezielten Zusatzposten, der dir wichtig ist. Wenn du einen deiner Prioritäten ergänzen willst, schlag ihn vor – wir können bis zu \textasciitilde{}600 Aufwand ohne Streichung drauflegen.'\}} \\
            }
        }
    }
    & & \\ \\

    \theutterance \stepcounter{utterance}  
    & & \multicolumn{4}{p{0.6\linewidth}}{
        \cellcolor[rgb]{0.9,0.9,0.9}{
            \makecell[{{p{\linewidth}}}]{
                \texttt{\tiny{[GM$\rangle$P2]}}
                \texttt{ABLEHNUNG: \{'A03', 'A83', 'C98', 'C62', 'C32', 'C76', 'C00', 'C10', 'C81', 'C56'\}} \\
\texttt{ARGUMENT: \{'Danke für die Anpassung – deutlich näher dran. Für mich sind C10 und C81 vergleichsweise schwach. Wenn wir dieses Budget auf sehr starke Alternativen legen, bekommen wir mehr gemeinsamen Wert raus. Ich würde B38 klar vor C10/C81 sehen.'\}} \\
\texttt{VORSCHLAG: \{'A03', 'A83', 'C98', 'C62', 'C32', 'C76', 'C00', 'B38', 'C56'\}} \\
\texttt{ARGUMENT: \{'Aufwand: 43+80+95+106+51+147+308+276+588 = 1694 (unter 2307). Das packt die für mich wertstarken Kerne rein und lässt dir 613 Aufwand Puffer für einen gezielten Zusatzposten, der dir wichtig ist. Wenn du einen deiner Prioritäten ergänzen willst, schlag ihn vor – wir können bis zu \textasciitilde{}600 Aufwand ohne Streichung drauflegen.'\}} \\
            }
        }
    }
    & \\ \\

    \theutterance \stepcounter{utterance}  
    & & & \multicolumn{4}{p{0.6\linewidth}}{
        \cellcolor[rgb]{0.9,0.9,0.9}{
            \makecell[{{p{\linewidth}}}]{
                \texttt{\tiny{[GM$\langle$P2]}}
                \texttt{VORSCHLAG: \{'A03', 'A83', 'C98', 'C62', 'C32', 'C76', 'C00', 'B38', 'C56', 'C10'\}} \\
\texttt{ARGUMENT: \{'Ich nehme deinen Kern mit und nutze den von dir gelassenen Puffer sinnvoll aus: Mit C10 landen wir bei 1694+325=2019 Aufwand (unter 2307) und steigern den Gesamtwert spürbar, ohne etwas von deinem bevorzugten Set zu streichen. Falls C10 für dich schwach bleibt, biete ich als Alternativen innerhalb des Puffers an: 1) C81 statt C10 (Gesamt 1949), 2) C62+C98 beibehalten und zusätzlich C76 aufwerten durch C03 gegen B38/C56 (z. B. Tausch B38→C03 ergäbe 1694{-}276+543=1961). Präferenz meinerseits: C10, ansonsten C03 gegen B38 als zweitbeste Option.'\}} \\
            }
        }
    }
     \\ \\

    \theutterance \stepcounter{utterance}  
    & & \multicolumn{4}{p{0.6\linewidth}}{
        \cellcolor[rgb]{0.9,0.9,0.9}{
            \makecell[{{p{\linewidth}}}]{
                \texttt{\tiny{[P1$\langle$GM]}}
                \texttt{VORSCHLAG: \{'A03', 'A83', 'C98', 'C62', 'C32', 'C76', 'C00', 'B38', 'C56', 'C10'\}} \\
\texttt{ARGUMENT: \{'Ich nehme deinen Kern mit und nutze den von dir gelassenen Puffer sinnvoll aus: Mit C10 landen wir bei 1694+325=2019 Aufwand (unter 2307) und steigern den Gesamtwert spürbar, ohne etwas von deinem bevorzugten Set zu streichen. Falls C10 für dich schwach bleibt, biete ich als Alternativen innerhalb des Puffers an: 1) C81 statt C10 (Gesamt 1949), 2) C62+C98 beibehalten und zusätzlich C76 aufwerten durch C03 gegen B38/C56 (z. B. Tausch B38→C03 ergäbe 1694{-}276+543=1961). Präferenz meinerseits: C10, ansonsten C03 gegen B38 als zweitbeste Option.'\}} \\
            }
        }
    }
    & \\ \\

    \theutterance \stepcounter{utterance}  
    & \multicolumn{4}{p{0.6\linewidth}}{
        \cellcolor[rgb]{0.9,0.9,0.9}{
            \makecell[{{p{\linewidth}}}]{
                \texttt{\tiny{[P1$\rangle$GM]}}
                \texttt{ABLEHNUNG: \{'A03', 'A83', 'C98', 'C62', 'C32', 'C76', 'C00', 'B38', 'C56', 'C10'\}} \\
\texttt{ARGUMENT: \{'Danke – guter Schritt. C10 bleibt für mich unterdurchschnittlich, während C03 für mich klar attraktiver ist als B38. Ich kann deinen Vorschlag mit einem gezielten Tausch mitgehen.'\}} \\
\texttt{VORSCHLAG: \{'A03', 'A83', 'C98', 'C62', 'C32', 'C76', 'C00', 'C56', 'C03'\}} \\
\texttt{ARGUMENT: \{'Aufwand: 43+80+95+106+51+147+308+588+543 = 1961 (unter 2307). Das entspricht deinem zweitbesten Pfad (B38→C03), hält deinen Kern intakt und ersetzt nur einen Posten. Wenn du hier mitgehst, sind wir effizient im Budget und bringen für beide Seiten starken Gegenwert rein.'\}} \\
            }
        }
    }
    & & \\ \\

    \theutterance \stepcounter{utterance}  
    & & \multicolumn{4}{p{0.6\linewidth}}{
        \cellcolor[rgb]{0.9,0.9,0.9}{
            \makecell[{{p{\linewidth}}}]{
                \texttt{\tiny{[GM$\rangle$P2]}}
                \texttt{ABLEHNUNG: \{'A03', 'A83', 'C98', 'C62', 'C32', 'C76', 'C00', 'B38', 'C56', 'C10'\}} \\
\texttt{ARGUMENT: \{'Danke – guter Schritt. C10 bleibt für mich unterdurchschnittlich, während C03 für mich klar attraktiver ist als B38. Ich kann deinen Vorschlag mit einem gezielten Tausch mitgehen.'\}} \\
\texttt{VORSCHLAG: \{'A03', 'A83', 'C98', 'C62', 'C32', 'C76', 'C00', 'C56', 'C03'\}} \\
\texttt{ARGUMENT: \{'Aufwand: 43+80+95+106+51+147+308+588+543 = 1961 (unter 2307). Das entspricht deinem zweitbesten Pfad (B38→C03), hält deinen Kern intakt und ersetzt nur einen Posten. Wenn du hier mitgehst, sind wir effizient im Budget und bringen für beide Seiten starken Gegenwert rein.'\}} \\
            }
        }
    }
    & \\ \\

    \theutterance \stepcounter{utterance}  
    & & & \multicolumn{4}{p{0.6\linewidth}}{
        \cellcolor[rgb]{0.9,0.9,0.9}{
            \makecell[{{p{\linewidth}}}]{
                \texttt{\tiny{[GM$\langle$P2]}}
                \texttt{ZUSTIMMUNG: \{'A03', 'A83', 'C98', 'C62', 'C32', 'C76', 'C00', 'C56', 'C03'\}} \\
\texttt{ARGUMENT: \{'Einverstanden. Dieses Paket nutzt das Budget effizient (1961 < 2307) und trifft mehrere für mich hoch priorisierte Posten. Danke für die zielgerichtete Anpassung – das ist für mich ein guter Deal.'\}} \\
            }
        }
    }
     \\ \\

    \theutterance \stepcounter{utterance}  
    & & & \multicolumn{2}{p{0.3\linewidth}}{
        \cellcolor[rgb]{0.9,0.9,0.9}{
            \makecell[{{p{\linewidth}}}]{
                \texttt{\tiny{[GM$|$GM]}}
                \texttt{game successful} \\
            }
        }
    }
    & & \\ \\

    \theutterance \stepcounter{utterance}  
    & & & \multicolumn{2}{p{0.3\linewidth}}{
        \cellcolor[rgb]{0.9,0.9,0.9}{
            \makecell[{{p{\linewidth}}}]{
                \texttt{\tiny{[GM$|$GM]}}
                \texttt{end game} \\
            }
        }
    }
    & & \\ \\

\end{supertabular}
}

\end{document}
