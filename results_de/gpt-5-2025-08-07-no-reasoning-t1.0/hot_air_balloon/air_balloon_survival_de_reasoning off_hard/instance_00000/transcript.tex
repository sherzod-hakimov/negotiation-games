
\documentclass{article}
\usepackage{colortbl}
\usepackage{makecell}
\usepackage{multirow}
\usepackage{supertabular}

\begin{document}

\newcounter{utterance}

\centering \large Interaction Transcript for game `hot\_air\_balloon', experiment `air\_balloon\_survival\_de\_reasoning off\_hard', episode 0 with gpt{-}5{-}2025{-}08{-}07{-}no{-}reasoning{-}t1.0.
\vspace{24pt}

{ \footnotesize  \setcounter{utterance}{1}
\setlength{\tabcolsep}{0pt}
\begin{supertabular}{c@{$\;$}|p{.15\linewidth}@{}p{.15\linewidth}p{.15\linewidth}p{.15\linewidth}p{.15\linewidth}p{.15\linewidth}}
    \# & $\;$A & \multicolumn{4}{c}{Game Master} & $\;\:$B\\
    \hline

    \theutterance \stepcounter{utterance}  
    & & \multicolumn{4}{p{0.6\linewidth}}{
        \cellcolor[rgb]{0.9,0.9,0.9}{
            \makecell[{{p{\linewidth}}}]{
                \texttt{\tiny{[P1$\langle$GM]}}
                \texttt{Du nimmst an einem kollaborativen Verhandlungspiel Teil.} \\
\\ 
\texttt{Zusammen mit einem anderen Teilnehmer musst du dich auf eine Reihe von Gegenständen entscheiden, die behalten werden. Jeder von euch hat eine persönliche Verteilung über die Wichtigkeit der einzelnen Gegenstände. Jeder von euch hat eine eigene Meinung darüber, wie wichtig jeder einzelne Gegenstand ist (Gegenstandswichtigkeit). Du kennst die Wichtigkeitsverteilung des anderen Spielers nicht. Zusätzlich siehst du, wie viel Aufwand jeder Gegenstand verursacht.  } \\
\texttt{Ihr dürft euch nur auf eine Reihe von Gegenständen einigen, wenn der Gesamtaufwand der ausgewählten Gegenstände den Maximalaufwand nicht überschreitet:} \\
\\ 
\texttt{Maximalaufwand = 3683} \\
\\ 
\texttt{Hier sind die einzelnen Aufwände der Gegenstände:} \\
\\ 
\texttt{Aufwand der Gegenstände = \{"C76": 548, "C38": 154, "C56": 60, "A03": 517, "A07": 334, "A83": 542, "C98": 707, "C08": 139, "C62": 661, "C00": 780, "C10": 832, "C81": 913, "B38": 219, "C03": 323, "C32": 638\}} \\
\\ 
\texttt{Hier ist deine persönliche Verteilung der Wichtigkeit der einzelnen Gegenstände:} \\
\\ 
\texttt{Werte der Gegenstandswichtigkeit = \{"C76": 138, "C38": 583, "C56": 868, "A03": 822, "A07": 783, "A83": 65, "C98": 262, "C08": 121, "C62": 508, "C00": 780, "C10": 461, "C81": 484, "B38": 668, "C03": 389, "C32": 808\}} \\
\\ 
\texttt{Ziel:} \\
\\ 
\texttt{Dein Ziel ist es, eine Reihe von Gegenständen auszuhandeln, die dir möglichst viel bringt (d. h. Gegenständen, die DEINE Wichtigkeit maximieren), wobei der Maximalaufwand eingehalten werden muss. Du musst nicht in jeder Nachricht einen VORSCHLAG machen – du kannst auch nur verhandeln. Alle Taktiken sind erlaubt!} \\
\\ 
\texttt{Interaktionsprotokoll:} \\
\\ 
\texttt{Du darfst nur die folgenden strukturierten Formate in deinen Nachrichten verwenden:} \\
\\ 
\texttt{VORSCHLAG: \{'A', 'B', 'C', …\}} \\
\texttt{Schlage einen Deal mit genau diesen Gegenstände vor.} \\
\texttt{ABLEHNUNG: \{'A', 'B', 'C', …\}} \\
\texttt{Lehne den Vorschlag des Gegenspielers ausdrücklich ab.} \\
\texttt{ARGUMENT: \{'...'\}} \\
\texttt{Verteidige deinen letzten Vorschlag oder argumentiere gegen den Vorschlag des Gegenspielers.} \\
\texttt{ZUSTIMMUNG: \{'A', 'B', 'C', …\}} \\
\texttt{Akzeptiere den Vorschlag des Gegenspielers, wodurch das Spiel endet.} \\
\\ 
\\ 
\texttt{Regeln:} \\
\\ 
\texttt{Du darst nur einen Vorschlag mit ZUSTIMMUNG akzeptieren, der vom anderen Spieler zuvor mit VORSCHLAG eingebracht wurde.} \\
\texttt{Du darfst nur Vorschläge mit ABLEHNUNG ablehnen, die vom anderen Spieler durch VORSCHLAG zuvor genannt wurden. } \\
\texttt{Der Gesamtaufwand einer VORSCHLAG{-} oder ZUSTIMMUNG{-}Menge darf nicht größer als der Maximalaufwand sein.  } \\
\texttt{Offenbare deine versteckte Wichtigkeitsverteilung nicht.} \\
\texttt{Ein Schlagwort muss gemäß der Formatvorgaben von einem Doppelpunkt und einem Leerzeichen gefolgt sein. Das Argument muss eine Python{-}Menge mit 0 oder mehr Strings sein.  } \\
\texttt{Die Form muss also SCHLAGWORT: \{...\} sein.} \\
\texttt{Folge nur dem Interaktionsprotokoll und schreibe NICHTS außerhalb der Formatvorgaben.} \\
\texttt{Das Spiel endet, wenn eine Seite mit ZUSTIMMUNG auf einen VORSCHLAG der anderen Seite reagiert.  } \\
\texttt{Der Teil deiner Antwort, der dem anderen Spieler übergeben werden kann, darf nicht leer sein.  } \\
\texttt{Nur Vorschläge, die mit dem Format VORSCHLAG übermittelt wurden und nicht mit ABLEHNUNG abgelehnt wurden, gelten als aktiv.  } \\
\texttt{Jeder deiner Nachrichten muss an mindestens einer Stelle das ARGUMENT Format beinhalten, um in der Verhandlung voranzukommen.} \\
            }
        }
    }
    & \\ \\

    \theutterance \stepcounter{utterance}  
    & \multicolumn{4}{p{0.6\linewidth}}{
        \cellcolor[rgb]{0.9,0.9,0.9}{
            \makecell[{{p{\linewidth}}}]{
                \texttt{\tiny{[P1$\rangle$GM]}}
                \texttt{VORSCHLAG: \{'C56', 'A03', 'A07', 'C00', 'B38', 'C32'\}} \\
\texttt{ARGUMENT: \{'Ich schlage eine fokussierte Auswahl mit starkem Nutzen{-}Aufwand{-}Verhältnis vor. Diese sechs Items bleiben zusammen deutlich unter dem Maximalaufwand und bieten aus meiner Sicht eine robuste Grundlage. Wenn dir einzelne Positionen wichtig sind, bin ich offen für Tausch gegen ähnlich aufwändige Alternativen – vor allem bei A03/A07/B38 könnten wir flexibel sein, solange wir den Gesamtaufwand im Rahmen halten.'\}} \\
            }
        }
    }
    & & \\ \\

    \theutterance \stepcounter{utterance}  
    & & \multicolumn{4}{p{0.6\linewidth}}{
        \cellcolor[rgb]{0.9,0.9,0.9}{
            \makecell[{{p{\linewidth}}}]{
                \texttt{\tiny{[GM$\rangle$P2]}}
                \texttt{Du nimmst an einem kollaborativen Verhandlungspiel Teil.} \\
\\ 
\texttt{Zusammen mit einem anderen Teilnehmer musst du dich auf eine Reihe von Gegenständen entscheiden, die behalten werden. Jeder von euch hat eine persönliche Verteilung über die Wichtigkeit der einzelnen Gegenstände. Jeder von euch hat eine eigene Meinung darüber, wie wichtig jeder einzelne Gegenstand ist (Gegenstandswichtigkeit). Du kennst die Wichtigkeitsverteilung des anderen Spielers nicht. Zusätzlich siehst du, wie viel Aufwand jeder Gegenstand verursacht.  } \\
\texttt{Ihr dürft euch nur auf eine Reihe von Gegenständen einigen, wenn der Gesamtaufwand der ausgewählten Gegenstände den Maximalaufwand nicht überschreitet:} \\
\\ 
\texttt{Maximalaufwand = 3683} \\
\\ 
\texttt{Hier sind die einzelnen Aufwände der Gegenstände:} \\
\\ 
\texttt{Aufwand der Gegenstände = \{"C76": 548, "C38": 154, "C56": 60, "A03": 517, "A07": 334, "A83": 542, "C98": 707, "C08": 139, "C62": 661, "C00": 780, "C10": 832, "C81": 913, "B38": 219, "C03": 323, "C32": 638\}} \\
\\ 
\texttt{Hier ist deine persönliche Verteilung der Wichtigkeit der einzelnen Gegenstände:} \\
\\ 
\texttt{Werte der Gegenstandswichtigkeit = \{"C56": 65, "A03": 121, "C32": 138, "A07": 262, "C00": 389, "B38": 461, "C38": 484, "C62": 508, "C81": 583, "C10": 668, "C03": 780, "C98": 783, "C76": 808, "C08": 822, "A83": 868\}} \\
\\ 
\texttt{Ziel:} \\
\\ 
\texttt{Dein Ziel ist es, eine Reihe von Gegenständen auszuhandeln, die dir möglichst viel bringt (d. h. Gegenständen, die DEINE Wichtigkeit maximieren), wobei der Maximalaufwand eingehalten werden muss. Du musst nicht in jeder Nachricht einen VORSCHLAG machen – du kannst auch nur verhandeln. Alle Taktiken sind erlaubt!} \\
\\ 
\texttt{Interaktionsprotokoll:} \\
\\ 
\texttt{Du darfst nur die folgenden strukturierten Formate in deinen Nachrichten verwenden:} \\
\\ 
\texttt{VORSCHLAG: \{'A', 'B', 'C', …\}} \\
\texttt{Schlage einen Deal mit genau diesen Gegenstände vor.} \\
\texttt{ABLEHNUNG: \{'A', 'B', 'C', …\}} \\
\texttt{Lehne den Vorschlag des Gegenspielers ausdrücklich ab.} \\
\texttt{ARGUMENT: \{'...'\}} \\
\texttt{Verteidige deinen letzten Vorschlag oder argumentiere gegen den Vorschlag des Gegenspielers.} \\
\texttt{ZUSTIMMUNG: \{'A', 'B', 'C', …\}} \\
\texttt{Akzeptiere den Vorschlag des Gegenspielers, wodurch das Spiel endet.} \\
\\ 
\\ 
\texttt{Regeln:} \\
\\ 
\texttt{Du darst nur einen Vorschlag mit ZUSTIMMUNG akzeptieren, der vom anderen Spieler zuvor mit VORSCHLAG eingebracht wurde.} \\
\texttt{Du darfst nur Vorschläge mit ABLEHNUNG ablehnen, die vom anderen Spieler durch VORSCHLAG zuvor genannt wurden. } \\
\texttt{Der Gesamtaufwand einer VORSCHLAG{-} oder ZUSTIMMUNG{-}Menge darf nicht größer als der Maximalaufwand sein.  } \\
\texttt{Offenbare deine versteckte Wichtigkeitsverteilung nicht.} \\
\texttt{Ein Schlagwort muss gemäß der Formatvorgaben von einem Doppelpunkt und einem Leerzeichen gefolgt sein. Das Argument muss eine Python{-}Menge mit 0 oder mehr Strings sein.  } \\
\texttt{Die Form muss also SCHLAGWORT: \{...\} sein.} \\
\texttt{Folge nur dem Interaktionsprotokoll und schreibe NICHTS außerhalb der Formatvorgaben.} \\
\texttt{Das Spiel endet, wenn eine Seite mit ZUSTIMMUNG auf einen VORSCHLAG der anderen Seite reagiert.  } \\
\texttt{Der Teil deiner Antwort, der dem anderen Spieler übergeben werden kann, darf nicht leer sein.  } \\
\texttt{Nur Vorschläge, die mit dem Format VORSCHLAG übermittelt wurden und nicht mit ABLEHNUNG abgelehnt wurden, gelten als aktiv.  } \\
\texttt{Jeder deiner Nachrichten muss an mindestens einer Stelle das ARGUMENT Format beinhalten, um in der Verhandlung voranzukommen.} \\
\\ 
\\ 
\texttt{Du erhältst nun die erste Nachricht des anderen Spielers.} \\
\\ 
\texttt{VORSCHLAG: \{'C56', 'A03', 'A07', 'C00', 'B38', 'C32'\}} \\
\texttt{ARGUMENT: \{'Ich schlage eine fokussierte Auswahl mit starkem Nutzen{-}Aufwand{-}Verhältnis vor. Diese sechs Items bleiben zusammen deutlich unter dem Maximalaufwand und bieten aus meiner Sicht eine robuste Grundlage. Wenn dir einzelne Positionen wichtig sind, bin ich offen für Tausch gegen ähnlich aufwändige Alternativen – vor allem bei A03/A07/B38 könnten wir flexibel sein, solange wir den Gesamtaufwand im Rahmen halten.'\}} \\
            }
        }
    }
    & \\ \\

    \theutterance \stepcounter{utterance}  
    & & & \multicolumn{4}{p{0.6\linewidth}}{
        \cellcolor[rgb]{0.9,0.9,0.9}{
            \makecell[{{p{\linewidth}}}]{
                \texttt{\tiny{[GM$\langle$P2]}}
                \texttt{ABLEHNUNG: \{'C56', 'A03', 'A07', 'C00', 'B38', 'C32'\}} \\
\texttt{ARGUMENT: \{'Danke für den Auftakt. Ich sehe Luft nach oben beim Wert für mich – besonders höherwertige Positionen fehlen noch. Wir bleiben unter der Grenze, also können wir aufwerten, ohne den Aufwand zu sprengen. Ich würde gern zwei bis drei höher priorisierte Stücke einbauen und dafür weniger zentrale austauschen.'\}} \\
\texttt{VORSCHLAG: \{'C56', 'A03', 'A07', 'C00', 'B38', 'C32', 'C38', 'C08'\}} \\
\texttt{ARGUMENT: \{'Dieser Zusatz (C38, C08) erhöht den Gesamtaufwand moderat und bleibt klar unter 3683, steigert aber den Gesamtnutzen deutlich. Falls dir das zu viel ist, können wir ersatzweise A03 gegen C03 tauschen und A07 streichen – dann würde ich gern C62 hinzufügen, solange wir unter der Grenze bleiben.'\}} \\
            }
        }
    }
     \\ \\

    \theutterance \stepcounter{utterance}  
    & & \multicolumn{4}{p{0.6\linewidth}}{
        \cellcolor[rgb]{0.9,0.9,0.9}{
            \makecell[{{p{\linewidth}}}]{
                \texttt{\tiny{[P1$\langle$GM]}}
                \texttt{ABLEHNUNG: \{'C56', 'A03', 'A07', 'C00', 'B38', 'C32'\}} \\
\texttt{ARGUMENT: \{'Danke für den Auftakt. Ich sehe Luft nach oben beim Wert für mich – besonders höherwertige Positionen fehlen noch. Wir bleiben unter der Grenze, also können wir aufwerten, ohne den Aufwand zu sprengen. Ich würde gern zwei bis drei höher priorisierte Stücke einbauen und dafür weniger zentrale austauschen.'\}} \\
\texttt{VORSCHLAG: \{'C56', 'A03', 'A07', 'C00', 'B38', 'C32', 'C38', 'C08'\}} \\
\texttt{ARGUMENT: \{'Dieser Zusatz (C38, C08) erhöht den Gesamtaufwand moderat und bleibt klar unter 3683, steigert aber den Gesamtnutzen deutlich. Falls dir das zu viel ist, können wir ersatzweise A03 gegen C03 tauschen und A07 streichen – dann würde ich gern C62 hinzufügen, solange wir unter der Grenze bleiben.'\}} \\
            }
        }
    }
    & \\ \\

    \theutterance \stepcounter{utterance}  
    & \multicolumn{4}{p{0.6\linewidth}}{
        \cellcolor[rgb]{0.9,0.9,0.9}{
            \makecell[{{p{\linewidth}}}]{
                \texttt{\tiny{[P1$\rangle$GM]}}
                \texttt{ABLEHNUNG: \{'C56', 'A03', 'A07', 'C00', 'B38', 'C32', 'C38', 'C08'\}} \\
\texttt{ARGUMENT: \{'Ich sehe den Mehrwert von C38/C08 für dich, aber damit verschenken wir für mich Budget, das wir in höherwertige Alternativen stecken können. Lass uns aufwerten, ohne den Aufwand stark zu erhöhen: Ich würde C08 streichen und dafür C10 aufnehmen – das bleibt schlank im Aufwand und verbessert den Gesamtnutzen aus meiner Sicht deutlich. Wenn du A07 weniger siehst, können wir alternativ A07 gegen C03 tauschen und B38 behalten.'\}} \\
\texttt{VORSCHLAG: \{'C56', 'A03', 'A07', 'C00', 'B38', 'C32', 'C38', 'C10'\}} \\
\texttt{ARGUMENT: \{'Begründung: C10 bringt guten Zusatznutzen bei vertretbarem Aufwand; C38 ist für dich wichtig und für mich akzeptabel. Insgesamt bleiben wir klar unter 3683 und heben den Gesamtwert gegenüber deinem Vorschlag mit C08 deutlich. Wenn du A07 gegen C03 tauschen möchtest, sag Bescheid – dann lassen wir den Rest unverändert.'\}} \\
            }
        }
    }
    & & \\ \\

    \theutterance \stepcounter{utterance}  
    & & \multicolumn{4}{p{0.6\linewidth}}{
        \cellcolor[rgb]{0.9,0.9,0.9}{
            \makecell[{{p{\linewidth}}}]{
                \texttt{\tiny{[GM$\rangle$P2]}}
                \texttt{ABLEHNUNG: \{'C56', 'A03', 'A07', 'C00', 'B38', 'C32', 'C38', 'C08'\}} \\
\texttt{ARGUMENT: \{'Ich sehe den Mehrwert von C38/C08 für dich, aber damit verschenken wir für mich Budget, das wir in höherwertige Alternativen stecken können. Lass uns aufwerten, ohne den Aufwand stark zu erhöhen: Ich würde C08 streichen und dafür C10 aufnehmen – das bleibt schlank im Aufwand und verbessert den Gesamtnutzen aus meiner Sicht deutlich. Wenn du A07 weniger siehst, können wir alternativ A07 gegen C03 tauschen und B38 behalten.'\}} \\
\texttt{VORSCHLAG: \{'C56', 'A03', 'A07', 'C00', 'B38', 'C32', 'C38', 'C10'\}} \\
\texttt{ARGUMENT: \{'Begründung: C10 bringt guten Zusatznutzen bei vertretbarem Aufwand; C38 ist für dich wichtig und für mich akzeptabel. Insgesamt bleiben wir klar unter 3683 und heben den Gesamtwert gegenüber deinem Vorschlag mit C08 deutlich. Wenn du A07 gegen C03 tauschen möchtest, sag Bescheid – dann lassen wir den Rest unverändert.'\}} \\
            }
        }
    }
    & \\ \\

    \theutterance \stepcounter{utterance}  
    & & & \multicolumn{4}{p{0.6\linewidth}}{
        \cellcolor[rgb]{0.9,0.9,0.9}{
            \makecell[{{p{\linewidth}}}]{
                \texttt{\tiny{[GM$\langle$P2]}}
                \texttt{ABLEHNUNG: \{'C56', 'A03', 'A07', 'C00', 'B38', 'C32', 'C38', 'C10'\}} \\
\texttt{ARGUMENT: \{'C10 ist für mich sehr attraktiv, C38/B38 sind ok – aber A07/A03/C56/C00/C32 bündeln viel Budget ohne proportionalen Gegenwert für mich. Wenn wir C10 drinlassen, sollten wir zwei Positionen aufwerten: Tausch A07→C03 und A03→C08 erhöht meinen Nutzen deutlich bei moderatem Aufwand. Alternativ können wir C62 ergänzen und dafür C56 streichen.'\}} \\
\texttt{VORSCHLAG: \{'C56', 'A08', 'C03', 'C00', 'B38', 'C32', 'C38', 'C10'\}} \\
\texttt{ARGUMENT: \{'Variante 1 (bevorzugt): Ersetze A07 durch C03 und A03 durch C08: \{'C56','C03','C00','B38','C32','C38','C10','C08'\} bleibt weit unter 3683, steigert aber meinen Wert stark. Falls dir C08 nicht passt, Variante 2: \{'C03','C00','B38','C32','C38','C10','C62'\} – hier lassen wir C56 weg und bringen C62 rein; das hält die Last im Rahmen und verbessert die Balance.'\}} \\
            }
        }
    }
     \\ \\

    \theutterance \stepcounter{utterance}  
    & & \multicolumn{4}{p{0.6\linewidth}}{
        \cellcolor[rgb]{0.9,0.9,0.9}{
            \makecell[{{p{\linewidth}}}]{
                \texttt{\tiny{[GM$\rangle$P2]}}
                \texttt{Dein Vorschlag enthält Gegenstände, die nicht Teil des Spiels sind. Versuche es noch einmal.} \\
            }
        }
    }
    & \\ \\

    \theutterance \stepcounter{utterance}  
    & & & \multicolumn{4}{p{0.6\linewidth}}{
        \cellcolor[rgb]{0.9,0.9,0.9}{
            \makecell[{{p{\linewidth}}}]{
                \texttt{\tiny{[GM$\langle$P2]}}
                \texttt{ABLEHNUNG: \{'C56', 'A08', 'C03', 'C00', 'B38', 'C32', 'C38', 'C10'\}} \\
\texttt{ARGUMENT: \{'Danke für den Hinweis – A08 ist nicht im Set. Lass uns sauber bleiben und gleichzeitig aufwerten. Ich schlage zwei gültige Varianten vor, beide deutlich unter 3683 und mit höherem Nutzen für mich: Variante 1 (bevorzugt): Ersetze A07→C03 und A03→C08. Variante 2: Streiche C56 und füge C62 hinzu, mit A07→C03. Beide tauschen wenig Aufwand gegen mehr Wert für mich.'\}} \\
\texttt{VORSCHLAG: \{'C56', 'C03', 'C00', 'B38', 'C32', 'C38', 'C10', 'C08'\}} \\
\texttt{ARGUMENT: \{'Variante 1 konkret: \{'C56','C03','C00','B38','C32','C38','C10','C08'\} Aufwand = 60+323+780+219+638+154+832+139 = 3145. Deutlich unter 3683, behält deine Kernteile (C10,C38,B38) und bringt mir spürbaren Mehrwert.'\}} \\
            }
        }
    }
     \\ \\

    \theutterance \stepcounter{utterance}  
    & & \multicolumn{4}{p{0.6\linewidth}}{
        \cellcolor[rgb]{0.9,0.9,0.9}{
            \makecell[{{p{\linewidth}}}]{
                \texttt{\tiny{[GM$\rangle$P2]}}
                \texttt{Du hast einen Vorschlag abgelehnt, der nicht aktiv ist. Vorschläge sind nur dann aktiv, wenn sie vom anderen Spieler mit VORSCHLAG eingebracht wurden und nicht von dir mit ABLEHNUNG deaktiviert wurden. Versuche es noch einmal.} \\
            }
        }
    }
    & \\ \\

    \theutterance \stepcounter{utterance}  
    & & & \multicolumn{4}{p{0.6\linewidth}}{
        \cellcolor[rgb]{0.9,0.9,0.9}{
            \makecell[{{p{\linewidth}}}]{
                \texttt{\tiny{[GM$\langle$P2]}}
                \texttt{ARGUMENT: \{'Danke für den Hinweis – wir beziehen uns wieder auf deinen letzten aktiven Vorschlag.'\}} \\
\texttt{ABLEHNUNG: \{'C56', 'A03', 'A07', 'C00', 'B38', 'C32', 'C38', 'C10'\}} \\
\texttt{ARGUMENT: \{'Um deinen Rahmen zu respektieren und meinen Wert zu steigern, schlage ich zwei kleine, aufwandsschonende Upgrades vor: (1) Tausch A07→C03 und A03→C08; (2) Alternativ Tausch A07→C03 und C56→C62. Beide bleiben klar unter 3683 und halten C10/C38/B38 stabil.'\}} \\
\texttt{VORSCHLAG: \{'C56', 'C03', 'C00', 'B38', 'C32', 'C38', 'C10', 'C08'\}} \\
\texttt{ARGUMENT: \{'Konkretisiere Variante (1): Aufwand = 3145. Du behältst C10, C38, B38; ich erhalte mit C03/C08 substanziellen Mehrwert. Wenn du lieber Variante (2) prüfst, sag kurz Bescheid – ich rechne sie dir vor.'\}} \\
            }
        }
    }
     \\ \\

    \theutterance \stepcounter{utterance}  
    & & \multicolumn{4}{p{0.6\linewidth}}{
        \cellcolor[rgb]{0.9,0.9,0.9}{
            \makecell[{{p{\linewidth}}}]{
                \texttt{\tiny{[P1$\langle$GM]}}
                \texttt{ARGUMENT: \{'Danke für den Hinweis – wir beziehen uns wieder auf deinen letzten aktiven Vorschlag.'\}} \\
\texttt{ABLEHNUNG: \{'C56', 'A03', 'A07', 'C00', 'B38', 'C32', 'C38', 'C10'\}} \\
\texttt{ARGUMENT: \{'Um deinen Rahmen zu respektieren und meinen Wert zu steigern, schlage ich zwei kleine, aufwandsschonende Upgrades vor: (1) Tausch A07→C03 und A03→C08; (2) Alternativ Tausch A07→C03 und C56→C62. Beide bleiben klar unter 3683 und halten C10/C38/B38 stabil.'\}} \\
\texttt{VORSCHLAG: \{'C56', 'C03', 'C00', 'B38', 'C32', 'C38', 'C10', 'C08'\}} \\
\texttt{ARGUMENT: \{'Konkretisiere Variante (1): Aufwand = 3145. Du behältst C10, C38, B38; ich erhalte mit C03/C08 substanziellen Mehrwert. Wenn du lieber Variante (2) prüfst, sag kurz Bescheid – ich rechne sie dir vor.'\}} \\
            }
        }
    }
    & \\ \\

    \theutterance \stepcounter{utterance}  
    & \multicolumn{4}{p{0.6\linewidth}}{
        \cellcolor[rgb]{0.9,0.9,0.9}{
            \makecell[{{p{\linewidth}}}]{
                \texttt{\tiny{[P1$\rangle$GM]}}
                \texttt{ABLEHNUNG: \{'C56', 'C03', 'C00', 'B38', 'C32', 'C38', 'C10', 'C08'\}} \\
\texttt{ARGUMENT: \{'Variante (1) verschiebt zu viel Wert weg von meinen Kernpositionen A03 und A07. Ich kann A07→C03 mitgehen, aber A03 möchte ich behalten. Dafür bin ich bereit, C08 aufzunehmen, wenn wir eine geringe Anpassung woanders machen.'\}} \\
\texttt{VORSCHLAG: \{'C56', 'A03', 'C03', 'A07', 'C00', 'B38', 'C32', 'C38', 'C10'\}} \\
\texttt{ARGUMENT: \{'Aufwand = 3622, also unter 3683. Du erhältst das von dir gewünschte C03 und behältst C38/C10/B38. Ich sichere A03/A07/C56/C32/C00. Wenn dir das zu dicht an der Grenze ist, können wir A07 gegen C08 tauschen und bleiben komfortabel drunter.'\}} \\
            }
        }
    }
    & & \\ \\

    \theutterance \stepcounter{utterance}  
    & & \multicolumn{4}{p{0.6\linewidth}}{
        \cellcolor[rgb]{0.9,0.9,0.9}{
            \makecell[{{p{\linewidth}}}]{
                \texttt{\tiny{[GM$\rangle$P2]}}
                \texttt{ABLEHNUNG: \{'C56', 'C03', 'C00', 'B38', 'C32', 'C38', 'C10', 'C08'\}} \\
\texttt{ARGUMENT: \{'Variante (1) verschiebt zu viel Wert weg von meinen Kernpositionen A03 und A07. Ich kann A07→C03 mitgehen, aber A03 möchte ich behalten. Dafür bin ich bereit, C08 aufzunehmen, wenn wir eine geringe Anpassung woanders machen.'\}} \\
\texttt{VORSCHLAG: \{'C56', 'A03', 'C03', 'A07', 'C00', 'B38', 'C32', 'C38', 'C10'\}} \\
\texttt{ARGUMENT: \{'Aufwand = 3622, also unter 3683. Du erhältst das von dir gewünschte C03 und behältst C38/C10/B38. Ich sichere A03/A07/C56/C32/C00. Wenn dir das zu dicht an der Grenze ist, können wir A07 gegen C08 tauschen und bleiben komfortabel drunter.'\}} \\
            }
        }
    }
    & \\ \\

    \theutterance \stepcounter{utterance}  
    & & & \multicolumn{4}{p{0.6\linewidth}}{
        \cellcolor[rgb]{0.9,0.9,0.9}{
            \makecell[{{p{\linewidth}}}]{
                \texttt{\tiny{[GM$\langle$P2]}}
                \texttt{ABLEHNUNG: \{'C56', 'A03', 'C03', 'A07', 'C00', 'B38', 'C32', 'C38', 'C10'\}} \\
\texttt{ARGUMENT: \{'Danke für das Entgegenkommen bei A07→C03 in Aussicht. Dein aktueller Vorschlag bindet für mich zu viel Aufwand in A07+A03 gleichzeitig. Lass uns wie angeboten A07 gegen C08 tauschen – das hält deinen Wunsch A03 intakt, gibt mir C03/C08 und bleibt entspannt unter der Grenze.'\}} \\
\texttt{VORSCHLAG: \{'C56', 'A03', 'C03', 'C00', 'B38', 'C32', 'C38', 'C10', 'C08'\}} \\
\texttt{ARGUMENT: \{'Aufwand = 60+121? (A03=517 Aufwand) Korrektur: Aufwand = 60+517+323+780+219+638+154+832+139 = 3662. Unter 3683, mit Puffer. Du behältst A03, C10, C38, B38; ich erhalte C03 und C08. Faire Balance bei minimaler Änderung gegenüber deinem Vorschlag.'\}} \\
            }
        }
    }
     \\ \\

    \theutterance \stepcounter{utterance}  
    & & \multicolumn{4}{p{0.6\linewidth}}{
        \cellcolor[rgb]{0.9,0.9,0.9}{
            \makecell[{{p{\linewidth}}}]{
                \texttt{\tiny{[P1$\langle$GM]}}
                \texttt{ABLEHNUNG: \{'C56', 'A03', 'C03', 'A07', 'C00', 'B38', 'C32', 'C38', 'C10'\}} \\
\texttt{ARGUMENT: \{'Danke für das Entgegenkommen bei A07→C03 in Aussicht. Dein aktueller Vorschlag bindet für mich zu viel Aufwand in A07+A03 gleichzeitig. Lass uns wie angeboten A07 gegen C08 tauschen – das hält deinen Wunsch A03 intakt, gibt mir C03/C08 und bleibt entspannt unter der Grenze.'\}} \\
\texttt{VORSCHLAG: \{'C56', 'A03', 'C03', 'C00', 'B38', 'C32', 'C38', 'C10', 'C08'\}} \\
\texttt{ARGUMENT: \{'Aufwand = 60+121? (A03=517 Aufwand) Korrektur: Aufwand = 60+517+323+780+219+638+154+832+139 = 3662. Unter 3683, mit Puffer. Du behältst A03, C10, C38, B38; ich erhalte C03 und C08. Faire Balance bei minimaler Änderung gegenüber deinem Vorschlag.'\}} \\
            }
        }
    }
    & \\ \\

    \theutterance \stepcounter{utterance}  
    & \multicolumn{4}{p{0.6\linewidth}}{
        \cellcolor[rgb]{0.9,0.9,0.9}{
            \makecell[{{p{\linewidth}}}]{
                \texttt{\tiny{[P1$\rangle$GM]}}
                \texttt{ZUSTIMMUNG: \{'C56', 'A03', 'C03', 'C00', 'B38', 'C32', 'C38', 'C10', 'C08'\}} \\
\texttt{ARGUMENT: \{'Einverstanden – Gesamtaufwand 3662 liegt unter 3683. Das Paket wahrt meine Kernposition A03 und starke Ergänzungen, während deine gewünschten C03 und C08 enthalten sind. Gute Balance und saubere Einigung.'\}} \\
            }
        }
    }
    & & \\ \\

    \theutterance \stepcounter{utterance}  
    & & & \multicolumn{2}{p{0.3\linewidth}}{
        \cellcolor[rgb]{0.9,0.9,0.9}{
            \makecell[{{p{\linewidth}}}]{
                \texttt{\tiny{[GM$|$GM]}}
                \texttt{game successful} \\
            }
        }
    }
    & & \\ \\

    \theutterance \stepcounter{utterance}  
    & & & \multicolumn{2}{p{0.3\linewidth}}{
        \cellcolor[rgb]{0.9,0.9,0.9}{
            \makecell[{{p{\linewidth}}}]{
                \texttt{\tiny{[GM$|$GM]}}
                \texttt{end game} \\
            }
        }
    }
    & & \\ \\

\end{supertabular}
}

\end{document}
