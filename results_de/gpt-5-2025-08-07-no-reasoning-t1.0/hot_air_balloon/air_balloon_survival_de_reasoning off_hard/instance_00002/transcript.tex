
\documentclass{article}
\usepackage{colortbl}
\usepackage{makecell}
\usepackage{multirow}
\usepackage{supertabular}

\begin{document}

\newcounter{utterance}

\twocolumn

{ \footnotesize  \setcounter{utterance}{1}
\setlength{\tabcolsep}{0pt}
\begin{supertabular}{c@{$\;$}|p{.15\linewidth}@{}p{.15\linewidth}p{.15\linewidth}p{.15\linewidth}p{.15\linewidth}p{.15\linewidth}}

    \# & $\;$A & \multicolumn{4}{c}{Game Master} & $\;\:$B\\
    \hline 

    \theutterance \stepcounter{utterance}  

    & & \multicolumn{4}{p{0.6\linewidth}}{\cellcolor[rgb]{0.9,0.9,0.9}{%
	\makecell[{{p{\linewidth}}}]{% 
	  \tt {\tiny [A$\langle$GM]}  
	 Du nimmst an einem kollaborativen Verhandlungspiel Teil.\\ \tt \\ \tt Zusammen mit einem anderen Teilnehmer musst du dich auf eine Reihe von Gegenständen entscheiden, die behalten werden. Jeder von euch hat eine persönliche Verteilung über die Wichtigkeit der einzelnen Gegenstände. Jeder von euch hat eine eigene Meinung darüber, wie wichtig jeder einzelne Gegenstand ist (Gegenstandswichtigkeit). Du kennst die Wichtigkeitsverteilung des anderen Spielers nicht. Zusätzlich siehst du, wie viel Aufwand jeder Gegenstand verursacht.  \\ \tt Ihr dürft euch nur auf eine Reihe von Gegenständen einigen, wenn der Gesamtaufwand der ausgewählten Gegenstände den Maximalaufwand nicht überschreitet:\\ \tt \\ \tt Maximalaufwand = 3203\\ \tt \\ \tt Hier sind die einzelnen Aufwände der Gegenstände:\\ \tt \\ \tt Aufwand der Gegenstände = {"C76": 52, "C38": 577, "C56": 177, "A03": 702, "A07": 119, "A83": 232, "C98": 577, "C08": 205, "C62": 516, "C00": 582, "C10": 676, "C81": 906, "B38": 316, "C03": 433, "C32": 336}\\ \tt \\ \tt Hier ist deine persönliche Verteilung der Wichtigkeit der einzelnen Gegenstände:\\ \tt \\ \tt Werte der Gegenstandswichtigkeit = {"C76": 138, "C38": 583, "C56": 868, "A03": 822, "A07": 783, "A83": 65, "C98": 262, "C08": 121, "C62": 508, "C00": 780, "C10": 461, "C81": 484, "B38": 668, "C03": 389, "C32": 808}\\ \tt \\ \tt Ziel:\\ \tt \\ \tt Dein Ziel ist es, eine Reihe von Gegenständen auszuhandeln, die dir möglichst viel bringt (d. h. Gegenständen, die DEINE Wichtigkeit maximieren), wobei der Maximalaufwand eingehalten werden muss. Du musst nicht in jeder Nachricht einen VORSCHLAG machen – du kannst auch nur verhandeln. Alle Taktiken sind erlaubt!\\ \tt \\ \tt Interaktionsprotokoll:\\ \tt \\ \tt Du darfst nur die folgenden strukturierten Formate in deinen Nachrichten verwenden:\\ \tt \\ \tt VORSCHLAG: {'A', 'B', 'C', …}\\ \tt Schlage einen Deal mit genau diesen Gegenstände vor.\\ \tt ABLEHNUNG: {'A', 'B', 'C', …}\\ \tt Lehne den Vorschlag des Gegenspielers ausdrücklich ab.\\ \tt ARGUMENT: {'...'}\\ \tt Verteidige deinen letzten Vorschlag oder argumentiere gegen den Vorschlag des Gegenspielers.\\ \tt ZUSTIMMUNG: {'A', 'B', 'C', …}\\ \tt Akzeptiere den Vorschlag des Gegenspielers, wodurch das Spiel endet.\\ \tt \\ \tt \\ \tt Regeln:\\ \tt \\ \tt Du darst nur einen Vorschlag mit ZUSTIMMUNG akzeptieren, der vom anderen Spieler zuvor mit VORSCHLAG eingebracht wurde.\\ \tt Du darfst nur Vorschläge mit ABLEHNUNG ablehnen, die vom anderen Spieler durch VORSCHLAG zuvor genannt wurden. \\ \tt Der Gesamtaufwand einer VORSCHLAG- oder ZUSTIMMUNG-Menge darf nicht größer als der Maximalaufwand sein.  \\ \tt Offenbare deine versteckte Wichtigkeitsverteilung nicht.\\ \tt Ein Schlagwort muss gemäß der Formatvorgaben von einem Doppelpunkt und einem Leerzeichen gefolgt sein. Das Argument muss eine Python-Menge mit 0 oder mehr Strings sein.  \\ \tt Die Form muss also SCHLAGWORT: {...} sein.\\ \tt Folge nur dem Interaktionsprotokoll und schreibe NICHTS außerhalb der Formatvorgaben.\\ \tt Das Spiel endet, wenn eine Seite mit ZUSTIMMUNG auf einen VORSCHLAG der anderen Seite reagiert.  \\ \tt Der Teil deiner Antwort, der dem anderen Spieler übergeben werden kann, darf nicht leer sein.  \\ \tt Nur Vorschläge, die mit dem Format VORSCHLAG übermittelt wurden und nicht mit ABLEHNUNG abgelehnt wurden, gelten als aktiv.  \\ \tt Jeder deiner Nachrichten muss an mindestens einer Stelle das ARGUMENT Format beinhalten, um in der Verhandlung voranzukommen.\\ \tt  
	  } 
	   } 
	   } 
	 & \\ 
 

    \theutterance \stepcounter{utterance}  

    & \multicolumn{4}{p{0.6\linewidth}}{\cellcolor[rgb]{0.8,1,0.9}{%
	\makecell[{{p{\linewidth}}}]{% 
	  \tt {\tiny [A$\rangle$GM]}  
	 VORSCHLAG: {'C56', 'A03', 'A07', 'C00', 'B38', 'C32'}\\ \tt ARGUMENT: {'Ich schlage diese Kombination vor, weil sie einen ausgewogenen Aufwand hat und mehrere zentrale Kategorien abdeckt. Wir bleiben deutlich unter dem Maximalaufwand und sichern uns eine starke Basis an wichtigen Gegenständen, ohne unflexibel zu werden. Falls dir ein Tausch wichtig ist, können wir über Alternativen wie den Ersatz von A03 durch C10 oder das Hinzufügen eines mittelaufwändigen Postens sprechen, solange wir im Budget bleiben.'} 
	  } 
	   } 
	   } 
	 & & \\ 
 

    \theutterance \stepcounter{utterance}  

    & & \multicolumn{4}{p{0.6\linewidth}}{\cellcolor[rgb]{0.9,0.9,0.9}{%
	\makecell[{{p{\linewidth}}}]{% 
	  \tt {\tiny [GM$\rangle$B]}  
	 Du nimmst an einem kollaborativen Verhandlungspiel Teil.\\ \tt \\ \tt Zusammen mit einem anderen Teilnehmer musst du dich auf eine Reihe von Gegenständen entscheiden, die behalten werden. Jeder von euch hat eine persönliche Verteilung über die Wichtigkeit der einzelnen Gegenstände. Jeder von euch hat eine eigene Meinung darüber, wie wichtig jeder einzelne Gegenstand ist (Gegenstandswichtigkeit). Du kennst die Wichtigkeitsverteilung des anderen Spielers nicht. Zusätzlich siehst du, wie viel Aufwand jeder Gegenstand verursacht.  \\ \tt Ihr dürft euch nur auf eine Reihe von Gegenständen einigen, wenn der Gesamtaufwand der ausgewählten Gegenstände den Maximalaufwand nicht überschreitet:\\ \tt \\ \tt Maximalaufwand = 3203\\ \tt \\ \tt Hier sind die einzelnen Aufwände der Gegenstände:\\ \tt \\ \tt Aufwand der Gegenstände = {"C76": 52, "C38": 577, "C56": 177, "A03": 702, "A07": 119, "A83": 232, "C98": 577, "C08": 205, "C62": 516, "C00": 582, "C10": 676, "C81": 906, "B38": 316, "C03": 433, "C32": 336}\\ \tt \\ \tt Hier ist deine persönliche Verteilung der Wichtigkeit der einzelnen Gegenstände:\\ \tt \\ \tt Werte der Gegenstandswichtigkeit = {"C56": 65, "A03": 121, "C32": 138, "A07": 262, "C00": 389, "B38": 461, "C38": 484, "C62": 508, "C81": 583, "C10": 668, "C03": 780, "C98": 783, "C76": 808, "C08": 822, "A83": 868}\\ \tt \\ \tt Ziel:\\ \tt \\ \tt Dein Ziel ist es, eine Reihe von Gegenständen auszuhandeln, die dir möglichst viel bringt (d. h. Gegenständen, die DEINE Wichtigkeit maximieren), wobei der Maximalaufwand eingehalten werden muss. Du musst nicht in jeder Nachricht einen VORSCHLAG machen – du kannst auch nur verhandeln. Alle Taktiken sind erlaubt!\\ \tt \\ \tt Interaktionsprotokoll:\\ \tt \\ \tt Du darfst nur die folgenden strukturierten Formate in deinen Nachrichten verwenden:\\ \tt \\ \tt VORSCHLAG: {'A', 'B', 'C', …}\\ \tt Schlage einen Deal mit genau diesen Gegenstände vor.\\ \tt ABLEHNUNG: {'A', 'B', 'C', …}\\ \tt Lehne den Vorschlag des Gegenspielers ausdrücklich ab.\\ \tt ARGUMENT: {'...'}\\ \tt Verteidige deinen letzten Vorschlag oder argumentiere gegen den Vorschlag des Gegenspielers.\\ \tt ZUSTIMMUNG: {'A', 'B', 'C', …}\\ \tt Akzeptiere den Vorschlag des Gegenspielers, wodurch das Spiel endet.\\ \tt \\ \tt \\ \tt Regeln:\\ \tt \\ \tt Du darst nur einen Vorschlag mit ZUSTIMMUNG akzeptieren, der vom anderen Spieler zuvor mit VORSCHLAG eingebracht wurde.\\ \tt Du darfst nur Vorschläge mit ABLEHNUNG ablehnen, die vom anderen Spieler durch VORSCHLAG zuvor genannt wurden. \\ \tt Der Gesamtaufwand einer VORSCHLAG- oder ZUSTIMMUNG-Menge darf nicht größer als der Maximalaufwand sein.  \\ \tt Offenbare deine versteckte Wichtigkeitsverteilung nicht.\\ \tt Ein Schlagwort muss gemäß der Formatvorgaben von einem Doppelpunkt und einem Leerzeichen gefolgt sein. Das Argument muss eine Python-Menge mit 0 oder mehr Strings sein.  \\ \tt Die Form muss also SCHLAGWORT: {...} sein.\\ \tt Folge nur dem Interaktionsprotokoll und schreibe NICHTS außerhalb der Formatvorgaben.\\ \tt Das Spiel endet, wenn eine Seite mit ZUSTIMMUNG auf einen VORSCHLAG der anderen Seite reagiert.  \\ \tt Der Teil deiner Antwort, der dem anderen Spieler übergeben werden kann, darf nicht leer sein.  \\ \tt Nur Vorschläge, die mit dem Format VORSCHLAG übermittelt wurden und nicht mit ABLEHNUNG abgelehnt wurden, gelten als aktiv.  \\ \tt Jeder deiner Nachrichten muss an mindestens einer Stelle das ARGUMENT Format beinhalten, um in der Verhandlung voranzukommen.\\ \tt \\ \tt \\ \tt Du erhältst nun die erste Nachricht des anderen Spielers.\\ \tt \\ \tt VORSCHLAG: {'C56', 'A03', 'A07', 'C00', 'B38', 'C32'}\\ \tt ARGUMENT: {'Ich schlage diese Kombination vor, weil sie einen ausgewogenen Aufwand hat und mehrere zentrale Kategorien abdeckt. Wir bleiben deutlich unter dem Maximalaufwand und sichern uns eine starke Basis an wichtigen Gegenständen, ohne unflexibel zu werden. Falls dir ein Tausch wichtig ist, können wir über Alternativen wie den Ersatz von A03 durch C10 oder das Hinzufügen eines mittelaufwändigen Postens sprechen, solange wir im Budget bleiben.'} 
	  } 
	   } 
	   } 
	 & \\ 
 

    \theutterance \stepcounter{utterance}  

    & & & \multicolumn{4}{p{0.6\linewidth}}{\cellcolor[rgb]{1,0.85,0.72}{%
	\makecell[{{p{\linewidth}}}]{% 
	  \tt {\tiny [GM$\langle$B]}  
	 ABLEHNUNG: {'C56', 'A03', 'A07', 'C00', 'B38', 'C32'}\\ \tt ARGUMENT: {'Danke für den Auftakt! Euer Vorschlag liegt zwar im Budget, lässt aber aus meiner Sicht zu viel Wert auf dem Tisch. Wir können mit demselben oder geringerem Aufwand deutlich mehr Gesamtnutzen erzielen, wenn wir ein paar höherwertige Posten priorisieren und teurere, weniger attraktive ersetzen.'}\\ \tt VORSCHLAG: {'C56', 'A07', 'A83', 'C38', 'C62', 'C98', 'C08'}\\ \tt ARGUMENT: {'Begründung: Diese Auswahl bleibt mit Aufwand 2485 klar unter 3203 und fokussiert mehrere für mich sehr wertvolle Posten (u. a. C98, C62, C38, A83, C08), während wir günstige Ergänzungen (C56, A07) mitnehmen. Dadurch maximieren wir den gemeinsamen Handlungsspielraum für spätere Anpassungen. Wenn dir ein Tausch wichtig ist, können wir A83 oder C38 gegen C10 prüfen; alternativ könnten wir C98 durch C10 ersetzen und bleiben dennoch im Budget.'} 
	  } 
	   } 
	   } 
	  \\ 
 

    \theutterance \stepcounter{utterance}  

    & & \multicolumn{4}{p{0.6\linewidth}}{\cellcolor[rgb]{0.9,0.9,0.9}{%
	\makecell[{{p{\linewidth}}}]{% 
	  \tt {\tiny [A$\langle$GM]}  
	 ABLEHNUNG: {'C56', 'A03', 'A07', 'C00', 'B38', 'C32'}\\ \tt ARGUMENT: {'Danke für den Auftakt! Euer Vorschlag liegt zwar im Budget, lässt aber aus meiner Sicht zu viel Wert auf dem Tisch. Wir können mit demselben oder geringerem Aufwand deutlich mehr Gesamtnutzen erzielen, wenn wir ein paar höherwertige Posten priorisieren und teurere, weniger attraktive ersetzen.'}\\ \tt VORSCHLAG: {'C56', 'A07', 'A83', 'C38', 'C62', 'C98', 'C08'}\\ \tt ARGUMENT: {'Begründung: Diese Auswahl bleibt mit Aufwand 2485 klar unter 3203 und fokussiert mehrere für mich sehr wertvolle Posten (u. a. C98, C62, C38, A83, C08), während wir günstige Ergänzungen (C56, A07) mitnehmen. Dadurch maximieren wir den gemeinsamen Handlungsspielraum für spätere Anpassungen. Wenn dir ein Tausch wichtig ist, können wir A83 oder C38 gegen C10 prüfen; alternativ könnten wir C98 durch C10 ersetzen und bleiben dennoch im Budget.'} 
	  } 
	   } 
	   } 
	 & \\ 
 

    \theutterance \stepcounter{utterance}  

    & \multicolumn{4}{p{0.6\linewidth}}{\cellcolor[rgb]{0.8,1,0.9}{%
	\makecell[{{p{\linewidth}}}]{% 
	  \tt {\tiny [A$\rangle$GM]}  
	 ABLEHNUNG: {'C56', 'A07', 'A83', 'C38', 'C62', 'C98', 'C08'}\\ \tt ARGUMENT: {'Ich sehe, dass du damit Spielraum lässt, aber aus meiner Sicht fehlen hier mehrere sehr wertvolle Kernposten. Insbesondere A03, C00, C32 und B38 liefern mir erheblich mehr Nutzen pro Aufwand als A83, C98 oder C08. Wenn wir die genannten drei ersetzen, steigt unser Gesamtnutzen deutlich, ohne das Budget zu sprengen.'}\\ \tt VORSCHLAG: {'C56', 'A03', 'A07', 'C00', 'B38', 'C32', 'C38'}\\ \tt ARGUMENT: {'Aufwand gesamt: 3225 – das ist 22 zu viel. Vorschlag zur Korrektur: Ersetze C38 durch C10 oder C03. Mit C10 sind wir bei 3324 (zu hoch), mit C03 landen wir bei 3081 und bleiben komfortabel drin. Diese Auswahl (C56, A03, A07, C00, B38, C32, C03) priorisiert starke Wertbringer und verzichtet auf für mich schwächere Posten wie A83/C98/C08.'} 
	  } 
	   } 
	   } 
	 & & \\ 
 

    \theutterance \stepcounter{utterance}  

    & & \multicolumn{4}{p{0.6\linewidth}}{\cellcolor[rgb]{0.9,0.9,0.9}{%
	\makecell[{{p{\linewidth}}}]{% 
	  \tt {\tiny [GM$\rangle$B]}  
	 ABLEHNUNG: {'C56', 'A07', 'A83', 'C38', 'C62', 'C98', 'C08'}\\ \tt ARGUMENT: {'Ich sehe, dass du damit Spielraum lässt, aber aus meiner Sicht fehlen hier mehrere sehr wertvolle Kernposten. Insbesondere A03, C00, C32 und B38 liefern mir erheblich mehr Nutzen pro Aufwand als A83, C98 oder C08. Wenn wir die genannten drei ersetzen, steigt unser Gesamtnutzen deutlich, ohne das Budget zu sprengen.'}\\ \tt VORSCHLAG: {'C56', 'A03', 'A07', 'C00', 'B38', 'C32', 'C38'}\\ \tt ARGUMENT: {'Aufwand gesamt: 3225 – das ist 22 zu viel. Vorschlag zur Korrektur: Ersetze C38 durch C10 oder C03. Mit C10 sind wir bei 3324 (zu hoch), mit C03 landen wir bei 3081 und bleiben komfortabel drin. Diese Auswahl (C56, A03, A07, C00, B38, C32, C03) priorisiert starke Wertbringer und verzichtet auf für mich schwächere Posten wie A83/C98/C08.'} 
	  } 
	   } 
	   } 
	 & \\ 
 

    \theutterance \stepcounter{utterance}  

    & & & \multicolumn{4}{p{0.6\linewidth}}{\cellcolor[rgb]{1,0.85,0.72}{%
	\makecell[{{p{\linewidth}}}]{% 
	  \tt {\tiny [GM$\langle$B]}  
	 ABLEHNUNG: {'C56', 'A03', 'A07', 'C00', 'B38', 'C32', 'C03'}\\ \tt ARGUMENT: {'Dein korrigierter Vorschlag ist budgetkonform, lässt aber mehrere für mich sehr starke Posten außen vor. Besonders C76, C08, A83 und C98 sind für mich zentrale Treiber. Ich kann entgegenkommen, indem ich zwei deiner Kernposten (A03, C00) mitnehme, möchte dafür aber mindestens zwei meiner Top-Items integrieren.'}\\ \tt VORSCHLAG: {'C56', 'A03', 'A07', 'C00', 'B38', 'C32', 'C08', 'C76'}\\ \tt ARGUMENT: {'Aufwand gesamt: 2309 – klar unter 3203. Diese Auswahl übernimmt fünf deiner Kerne (C56, A03, A07, C00, B38, C32) und ergänzt zwei meiner wichtigsten mit sehr gutem Aufwand-Nutzen (C08, C76). Wir lassen bewusst hochaufwändige Streckposten draußen, um später optional noch aufzurunden, falls nötig. Wenn du zwingend C03 willst, könnten wir C56 oder A07 dafür tauschen und bleiben trotzdem im Budget.'} 
	  } 
	   } 
	   } 
	  \\ 
 

    \theutterance \stepcounter{utterance}  

    & & \multicolumn{4}{p{0.6\linewidth}}{\cellcolor[rgb]{0.9,0.9,0.9}{%
	\makecell[{{p{\linewidth}}}]{% 
	  \tt {\tiny [GM$\rangle$B]}  
	 Du hast einen Vorschlag abgelehnt, der nicht aktiv ist. Vorschläge sind nur dann aktiv, wenn sie vom anderen Spieler mit VORSCHLAG eingebracht wurden und nicht von dir mit ABLEHNUNG deaktiviert wurden. Versuche es noch einmal. 
	  } 
	   } 
	   } 
	 & \\ 
 

    \theutterance \stepcounter{utterance}  

    & & & \multicolumn{4}{p{0.6\linewidth}}{\cellcolor[rgb]{1,0.85,0.72}{%
	\makecell[{{p{\linewidth}}}]{% 
	  \tt {\tiny [GM$\langle$B]}  
	 ARGUMENT: {'Danke für den Hinweis – lass uns sauber weitermachen. Dein aktuell aktiver Vorschlag ist: {C56, A03, A07, C00, B38, C32, C38} mit 3225 Aufwand (22 über Budget) und dein eigener Korrekturvorschlag war {C56, A03, A07, C00, B38, C32, C03} mit 3081 Aufwand. Ich adressiere beides: Ich schlage eine budgetkonforme Alternative vor, die deine Kernposten weitgehend respektiert und zwei meiner Top-Prioritäten integriert, um den gemeinsamen Nutzen zu erhöhen.'}\\ \tt VORSCHLAG: {'C56', 'A03', 'A07', 'C00', 'B38', 'C32', 'C08', 'C76'}\\ \tt ARGUMENT: {'Aufwand gesamt: 2309. Begründung: Du bekommst sechs von dir priorisierten Items (C56, A03, A07, C00, B38, C32). Ich integriere zwei meiner zentralen Werttreiber (C08, C76) mit sehr gutem Aufwand-Nutzen. Wir bleiben weit unter 3203 und halten Spielraum für Anpassungen. Wenn dir C03 besonders wichtig ist, kann ich C56 gegen C03 tauschen; Aufwand läge dann bei 2740. Alternativ: C08 gegen C03 tauschen; Aufwand 2634. Sag mir, welche Variante dir lieber ist.'} 
	  } 
	   } 
	   } 
	  \\ 
 

    \theutterance \stepcounter{utterance}  

    & & \multicolumn{4}{p{0.6\linewidth}}{\cellcolor[rgb]{0.9,0.9,0.9}{%
	\makecell[{{p{\linewidth}}}]{% 
	  \tt {\tiny [A$\langle$GM]}  
	 ARGUMENT: {'Danke für den Hinweis – lass uns sauber weitermachen. Dein aktuell aktiver Vorschlag ist: {C56, A03, A07, C00, B38, C32, C38} mit 3225 Aufwand (22 über Budget) und dein eigener Korrekturvorschlag war {C56, A03, A07, C00, B38, C32, C03} mit 3081 Aufwand. Ich adressiere beides: Ich schlage eine budgetkonforme Alternative vor, die deine Kernposten weitgehend respektiert und zwei meiner Top-Prioritäten integriert, um den gemeinsamen Nutzen zu erhöhen.'}\\ \tt VORSCHLAG: {'C56', 'A03', 'A07', 'C00', 'B38', 'C32', 'C08', 'C76'}\\ \tt ARGUMENT: {'Aufwand gesamt: 2309. Begründung: Du bekommst sechs von dir priorisierten Items (C56, A03, A07, C00, B38, C32). Ich integriere zwei meiner zentralen Werttreiber (C08, C76) mit sehr gutem Aufwand-Nutzen. Wir bleiben weit unter 3203 und halten Spielraum für Anpassungen. Wenn dir C03 besonders wichtig ist, kann ich C56 gegen C03 tauschen; Aufwand läge dann bei 2740. Alternativ: C08 gegen C03 tauschen; Aufwand 2634. Sag mir, welche Variante dir lieber ist.'} 
	  } 
	   } 
	   } 
	 & \\ 
 

    \theutterance \stepcounter{utterance}  

    & \multicolumn{4}{p{0.6\linewidth}}{\cellcolor[rgb]{0.8,1,0.9}{%
	\makecell[{{p{\linewidth}}}]{% 
	  \tt {\tiny [A$\rangle$GM]}  
	 ABLEHNUNG: {'C56', 'A03', 'A07', 'C00', 'B38', 'C32', 'C08', 'C76'}\\ \tt ARGUMENT: {'Danke für die Bereitschaft, meine Kernposten zu respektieren. Für mich sind C03 und C10 deutlich wertvoller als C08 und C76. Wenn wir diese beiden integrieren, steigt mein Gesamtnutzen erheblich, ohne das Budget zu reißen. Ich bin bereit, dir dafür den großen Spielraum zu lassen und nichts Teures zusätzlich zu drücken.'}\\ \tt VORSCHLAG: {'C56', 'A03', 'A07', 'C00', 'B38', 'C32', 'C03', 'C10'}\\ \tt ARGUMENT: {'Aufwand gesamt: 3081 + 676 = 3757 – das überschreitet das Budget. Korrigierte, budgetkonforme Variante: Ersetze A07 (119) durch C76 (52). Dann liegen wir bei 3690 – immer noch zu hoch. Nächster Schritt: Wir behalten C03, verzichten aber auf C10. Ergebnis: {C56, A03, A07, C00, B38, C32, C03} mit 3081 Aufwand – klar im Rahmen, deckt deine sechs Kernposten plus C03 ab und verzichtet nur auf die für mich schwächeren C08/C76. Wenn dir C10 sehr wichtig ist, können wir stattdessen C10 aufnehmen und A07 sowie C56 gemeinsam herausnehmen; Aufwand läge dann bei 3081 - 119 - 177 + 676 = 3461 (zu hoch). Alternative: C10 rein, A07 raus, C56 behalten, und dafür B38 durch C38 tauschen. Aufwand: 3081 - 119 - 316 + 577 = 3223 (20 über). Wenn du stattdessen B38 durch C03+ C76 ersetzt, bleiben wir bei 3081 - 316 + 52 = 2817, könnten dann C10 hinzufügen und landen bei 3493 (zu hoch). Kurz: Mit C10 wird es sehr eng; deshalb schlage ich zunächst die saubere 3081-Variante vor.'} 
	  } 
	   } 
	   } 
	 & & \\ 
 

    \theutterance \stepcounter{utterance}  

    & & \multicolumn{4}{p{0.6\linewidth}}{\cellcolor[rgb]{0.9,0.9,0.9}{%
	\makecell[{{p{\linewidth}}}]{% 
	  \tt {\tiny [GM$\rangle$B]}  
	 ABLEHNUNG: {'C56', 'A03', 'A07', 'C00', 'B38', 'C32', 'C08', 'C76'}\\ \tt ARGUMENT: {'Danke für die Bereitschaft, meine Kernposten zu respektieren. Für mich sind C03 und C10 deutlich wertvoller als C08 und C76. Wenn wir diese beiden integrieren, steigt mein Gesamtnutzen erheblich, ohne das Budget zu reißen. Ich bin bereit, dir dafür den großen Spielraum zu lassen und nichts Teures zusätzlich zu drücken.'}\\ \tt VORSCHLAG: {'C56', 'A03', 'A07', 'C00', 'B38', 'C32', 'C03', 'C10'}\\ \tt ARGUMENT: {'Aufwand gesamt: 3081 + 676 = 3757 – das überschreitet das Budget. Korrigierte, budgetkonforme Variante: Ersetze A07 (119) durch C76 (52). Dann liegen wir bei 3690 – immer noch zu hoch. Nächster Schritt: Wir behalten C03, verzichten aber auf C10. Ergebnis: {C56, A03, A07, C00, B38, C32, C03} mit 3081 Aufwand – klar im Rahmen, deckt deine sechs Kernposten plus C03 ab und verzichtet nur auf die für mich schwächeren C08/C76. Wenn dir C10 sehr wichtig ist, können wir stattdessen C10 aufnehmen und A07 sowie C56 gemeinsam herausnehmen; Aufwand läge dann bei 3081 - 119 - 177 + 676 = 3461 (zu hoch). Alternative: C10 rein, A07 raus, C56 behalten, und dafür B38 durch C38 tauschen. Aufwand: 3081 - 119 - 316 + 577 = 3223 (20 über). Wenn du stattdessen B38 durch C03+ C76 ersetzt, bleiben wir bei 3081 - 316 + 52 = 2817, könnten dann C10 hinzufügen und landen bei 3493 (zu hoch). Kurz: Mit C10 wird es sehr eng; deshalb schlage ich zunächst die saubere 3081-Variante vor.'} 
	  } 
	   } 
	   } 
	 & \\ 
 

    \theutterance \stepcounter{utterance}  

    & & & \multicolumn{4}{p{0.6\linewidth}}{\cellcolor[rgb]{1,0.85,0.72}{%
	\makecell[{{p{\linewidth}}}]{% 
	  \tt {\tiny [GM$\langle$B]}  
	 ABLEHNUNG: {'C56', 'A03', 'A07', 'C00', 'B38', 'C32', 'C03', 'C10'}\\ \tt ARGUMENT: {'Diese Kombination ist über Budget und priorisiert C10, der für mich nur dann akzeptabel ist, wenn wir dafür wirklich schwache Posten herausnehmen. Ich möchte C03 erhalten (kann ich mitgehen), aber C10 passt nur, wenn wir zwei andere Items streichen. Da das mehrfach knapp scheitert, fokussiere ich eine klare, budgetkonforme Lösung, die deine Kernposten weitgehend wahrt und meine starken Treiber berücksichtigt.'}\\ \tt VORSCHLAG: {'C56', 'A03', 'A07', 'C00', 'B38', 'C32', 'C03', 'C76'}\\ \tt ARGUMENT: {'Aufwand gesamt: 3133 (<= 3203). Du bekommst deine sechs Kernposten plus den für dich wichtigen C03. Ich nehme im Gegenzug das sehr günstige C76 auf. Wir verzichten auf C10, damit das Budget sauber bleibt. Wenn C10 für dich zwingend ist, schlage ich als Tausch C10 rein, A07 und C56 raus vor; dann brauchen wir zusätzlich den Ersatz von B38 durch C38, landen aber über Budget. Alternative budgetkonforme Option mit C10: C10 rein, A07 raus, B38 durch C38 ersetzen, und C76 behalten – Aufwand 3225 (knapp zu hoch). Daher ist die vorliegende 3133-Variante die stabile, faire Lösung.'} 
	  } 
	   } 
	   } 
	  \\ 
 

    \theutterance \stepcounter{utterance}  

    & & \multicolumn{4}{p{0.6\linewidth}}{\cellcolor[rgb]{0.9,0.9,0.9}{%
	\makecell[{{p{\linewidth}}}]{% 
	  \tt {\tiny [A$\langle$GM]}  
	 ABLEHNUNG: {'C56', 'A03', 'A07', 'C00', 'B38', 'C32', 'C03', 'C10'}\\ \tt ARGUMENT: {'Diese Kombination ist über Budget und priorisiert C10, der für mich nur dann akzeptabel ist, wenn wir dafür wirklich schwache Posten herausnehmen. Ich möchte C03 erhalten (kann ich mitgehen), aber C10 passt nur, wenn wir zwei andere Items streichen. Da das mehrfach knapp scheitert, fokussiere ich eine klare, budgetkonforme Lösung, die deine Kernposten weitgehend wahrt und meine starken Treiber berücksichtigt.'}\\ \tt VORSCHLAG: {'C56', 'A03', 'A07', 'C00', 'B38', 'C32', 'C03', 'C76'}\\ \tt ARGUMENT: {'Aufwand gesamt: 3133 (<= 3203). Du bekommst deine sechs Kernposten plus den für dich wichtigen C03. Ich nehme im Gegenzug das sehr günstige C76 auf. Wir verzichten auf C10, damit das Budget sauber bleibt. Wenn C10 für dich zwingend ist, schlage ich als Tausch C10 rein, A07 und C56 raus vor; dann brauchen wir zusätzlich den Ersatz von B38 durch C38, landen aber über Budget. Alternative budgetkonforme Option mit C10: C10 rein, A07 raus, B38 durch C38 ersetzen, und C76 behalten – Aufwand 3225 (knapp zu hoch). Daher ist die vorliegende 3133-Variante die stabile, faire Lösung.'} 
	  } 
	   } 
	   } 
	 & \\ 
 

    \theutterance \stepcounter{utterance}  

    & \multicolumn{4}{p{0.6\linewidth}}{\cellcolor[rgb]{0.8,1,0.9}{%
	\makecell[{{p{\linewidth}}}]{% 
	  \tt {\tiny [A$\rangle$GM]}  
	 VORSCHLAG: {'C56', 'A03', 'A07', 'C00', 'B38', 'C32', 'C03', 'C76'}\\ \tt ARGUMENT: {'Das ist eine tragfähige Basis: Sie bleibt mit 3133 klar im Budget und enthält meine wesentlichen Kernposten plus C03, während du C76 als günstigen Treiber bekommst. Um dir noch entgegenzukommen, biete ich zwei strikt budgetkonforme Alternativen an: (1) C10 ist dir wichtig? Dann C10 rein und dafür A07 und C76 raus; zusätzlich B38 -> C38 tauschen. Ergebnis: 3133 - 119 - 52 - 316 + 577 + 676 = 3899 (zu hoch) – also nicht machbar. (2) Realistische Option: Ersetze B38 durch C38 und streiche A07; dafür C10 rein. Aufwand: 3133 - 316 - 119 + 577 + 676 = 3871 (ebenfalls zu hoch). Fazit: C10 verzerrt das Budget zu stark in jeder sinnvollen Konfiguration. Deshalb ist der aktuelle Vorschlag die faire, stabile Lösung, die deinen Wunsch nach C03 berücksichtigt und breit Nutzen stiftet.'} 
	  } 
	   } 
	   } 
	 & & \\ 
 

    \theutterance \stepcounter{utterance}  

    & & \multicolumn{4}{p{0.6\linewidth}}{\cellcolor[rgb]{0.9,0.9,0.9}{%
	\makecell[{{p{\linewidth}}}]{% 
	  \tt {\tiny [GM$\rangle$B]}  
	 VORSCHLAG: {'C56', 'A03', 'A07', 'C00', 'B38', 'C32', 'C03', 'C76'}\\ \tt ARGUMENT: {'Das ist eine tragfähige Basis: Sie bleibt mit 3133 klar im Budget und enthält meine wesentlichen Kernposten plus C03, während du C76 als günstigen Treiber bekommst. Um dir noch entgegenzukommen, biete ich zwei strikt budgetkonforme Alternativen an: (1) C10 ist dir wichtig? Dann C10 rein und dafür A07 und C76 raus; zusätzlich B38 -> C38 tauschen. Ergebnis: 3133 - 119 - 52 - 316 + 577 + 676 = 3899 (zu hoch) – also nicht machbar. (2) Realistische Option: Ersetze B38 durch C38 und streiche A07; dafür C10 rein. Aufwand: 3133 - 316 - 119 + 577 + 676 = 3871 (ebenfalls zu hoch). Fazit: C10 verzerrt das Budget zu stark in jeder sinnvollen Konfiguration. Deshalb ist der aktuelle Vorschlag die faire, stabile Lösung, die deinen Wunsch nach C03 berücksichtigt und breit Nutzen stiftet.'} 
	  } 
	   } 
	   } 
	 & \\ 
 

    \theutterance \stepcounter{utterance}  

    & & & \multicolumn{4}{p{0.6\linewidth}}{\cellcolor[rgb]{1,0.85,0.72}{%
	\makecell[{{p{\linewidth}}}]{% 
	  \tt {\tiny [GM$\langle$B]}  
	 ZUSTIMMUNG: {'C56', 'A03', 'A07', 'C00', 'B38', 'C32', 'C03', 'C76'}\\ \tt ARGUMENT: {'Einverstanden. Diese Auswahl ist budgetkonform (3133 <= 3203) und deckt deine Kernposten inklusive C03 ab, während ich mit C76 eine wichtige Ergänzung erhalte. Gute, faire Lösung.'} 
	  } 
	   } 
	   } 
	  \\ 
 

    \theutterance \stepcounter{utterance}  

    & & & \multicolumn{2}{p{0.3\linewidth}}{\cellcolor[rgb]{0.95,0.95,0.95}{%
	\makecell[{{p{\linewidth}}}]{% 
	  \tt {\tiny [GM$|$GM]}  
	 game successful 
	  } 
	   } 
	   } 
	 & & \\ 
 

    \theutterance \stepcounter{utterance}  

    & & & \multicolumn{2}{p{0.3\linewidth}}{\cellcolor[rgb]{0.95,0.95,0.95}{%
	\makecell[{{p{\linewidth}}}]{% 
	  \tt {\tiny [GM$|$GM]}  
	 end game 
	  } 
	   } 
	   } 
	 & & \\ 
 

\end{supertabular}
}

\end{document}
