
\documentclass{article}
\usepackage{colortbl}
\usepackage{makecell}
\usepackage{multirow}
\usepackage{supertabular}

\begin{document}

\newcounter{utterance}

\centering \large Interaction Transcript for game `hot\_air\_balloon', experiment `air\_balloon\_survival\_de\_reasoning off\_hard', episode 4 with gpt{-}5{-}2025{-}08{-}07{-}no{-}reasoning{-}t1.0.
\vspace{24pt}

{ \footnotesize  \setcounter{utterance}{1}
\setlength{\tabcolsep}{0pt}
\begin{supertabular}{c@{$\;$}|p{.15\linewidth}@{}p{.15\linewidth}p{.15\linewidth}p{.15\linewidth}p{.15\linewidth}p{.15\linewidth}}
    \# & $\;$A & \multicolumn{4}{c}{Game Master} & $\;\:$B\\
    \hline

    \theutterance \stepcounter{utterance}  
    & & \multicolumn{4}{p{0.6\linewidth}}{
        \cellcolor[rgb]{0.9,0.9,0.9}{
            \makecell[{{p{\linewidth}}}]{
                \texttt{\tiny{[P1$\langle$GM]}}
                \texttt{Du nimmst an einem kollaborativen Verhandlungspiel Teil.} \\
\\ 
\texttt{Zusammen mit einem anderen Teilnehmer musst du dich auf eine Reihe von Gegenständen entscheiden, die behalten werden. Jeder von euch hat eine persönliche Verteilung über die Wichtigkeit der einzelnen Gegenstände. Jeder von euch hat eine eigene Meinung darüber, wie wichtig jeder einzelne Gegenstand ist (Gegenstandswichtigkeit). Du kennst die Wichtigkeitsverteilung des anderen Spielers nicht. Zusätzlich siehst du, wie viel Aufwand jeder Gegenstand verursacht.  } \\
\texttt{Ihr dürft euch nur auf eine Reihe von Gegenständen einigen, wenn der Gesamtaufwand der ausgewählten Gegenstände den Maximalaufwand nicht überschreitet:} \\
\\ 
\texttt{Maximalaufwand = 2984} \\
\\ 
\texttt{Hier sind die einzelnen Aufwände der Gegenstände:} \\
\\ 
\texttt{Aufwand der Gegenstände = \{"C76": 350, "C38": 276, "C56": 616, "A03": 737, "A07": 531, "A83": 389, "C98": 24, "C08": 125, "C62": 338, "C00": 356, "C10": 143, "C81": 117, "B38": 257, "C03": 921, "C32": 789\}} \\
\\ 
\texttt{Hier ist deine persönliche Verteilung der Wichtigkeit der einzelnen Gegenstände:} \\
\\ 
\texttt{Werte der Gegenstandswichtigkeit = \{"C76": 138, "C38": 583, "C56": 868, "A03": 822, "A07": 783, "A83": 65, "C98": 262, "C08": 121, "C62": 508, "C00": 780, "C10": 461, "C81": 484, "B38": 668, "C03": 389, "C32": 808\}} \\
\\ 
\texttt{Ziel:} \\
\\ 
\texttt{Dein Ziel ist es, eine Reihe von Gegenständen auszuhandeln, die dir möglichst viel bringt (d. h. Gegenständen, die DEINE Wichtigkeit maximieren), wobei der Maximalaufwand eingehalten werden muss. Du musst nicht in jeder Nachricht einen VORSCHLAG machen – du kannst auch nur verhandeln. Alle Taktiken sind erlaubt!} \\
\\ 
\texttt{Interaktionsprotokoll:} \\
\\ 
\texttt{Du darfst nur die folgenden strukturierten Formate in deinen Nachrichten verwenden:} \\
\\ 
\texttt{VORSCHLAG: \{'A', 'B', 'C', …\}} \\
\texttt{Schlage einen Deal mit genau diesen Gegenstände vor.} \\
\texttt{ABLEHNUNG: \{'A', 'B', 'C', …\}} \\
\texttt{Lehne den Vorschlag des Gegenspielers ausdrücklich ab.} \\
\texttt{ARGUMENT: \{'...'\}} \\
\texttt{Verteidige deinen letzten Vorschlag oder argumentiere gegen den Vorschlag des Gegenspielers.} \\
\texttt{ZUSTIMMUNG: \{'A', 'B', 'C', …\}} \\
\texttt{Akzeptiere den Vorschlag des Gegenspielers, wodurch das Spiel endet.} \\
\\ 
\\ 
\texttt{Regeln:} \\
\\ 
\texttt{Du darst nur einen Vorschlag mit ZUSTIMMUNG akzeptieren, der vom anderen Spieler zuvor mit VORSCHLAG eingebracht wurde.} \\
\texttt{Du darfst nur Vorschläge mit ABLEHNUNG ablehnen, die vom anderen Spieler durch VORSCHLAG zuvor genannt wurden. } \\
\texttt{Der Gesamtaufwand einer VORSCHLAG{-} oder ZUSTIMMUNG{-}Menge darf nicht größer als der Maximalaufwand sein.  } \\
\texttt{Offenbare deine versteckte Wichtigkeitsverteilung nicht.} \\
\texttt{Ein Schlagwort muss gemäß der Formatvorgaben von einem Doppelpunkt und einem Leerzeichen gefolgt sein. Das Argument muss eine Python{-}Menge mit 0 oder mehr Strings sein.  } \\
\texttt{Die Form muss also SCHLAGWORT: \{...\} sein.} \\
\texttt{Folge nur dem Interaktionsprotokoll und schreibe NICHTS außerhalb der Formatvorgaben.} \\
\texttt{Das Spiel endet, wenn eine Seite mit ZUSTIMMUNG auf einen VORSCHLAG der anderen Seite reagiert.  } \\
\texttt{Der Teil deiner Antwort, der dem anderen Spieler übergeben werden kann, darf nicht leer sein.  } \\
\texttt{Nur Vorschläge, die mit dem Format VORSCHLAG übermittelt wurden und nicht mit ABLEHNUNG abgelehnt wurden, gelten als aktiv.  } \\
\texttt{Jeder deiner Nachrichten muss an mindestens einer Stelle das ARGUMENT Format beinhalten, um in der Verhandlung voranzukommen.} \\
            }
        }
    }
    & \\ \\

    \theutterance \stepcounter{utterance}  
    & \multicolumn{4}{p{0.6\linewidth}}{
        \cellcolor[rgb]{0.9,0.9,0.9}{
            \makecell[{{p{\linewidth}}}]{
                \texttt{\tiny{[P1$\rangle$GM]}}
                \texttt{VORSCHLAG: \{'C56', 'A03', 'A07', 'C32'\}} \\
\texttt{ARGUMENT: \{'Dieser Vorschlag bleibt unter dem Maximalaufwand (616+737+531+789=2673) und deckt zentrale, aus meiner Sicht sehr wertvolle Gegenstände ab. Er lässt zudem Puffer für mögliche Anpassungen und berücksichtigt, dass wir beide den Aufwand im Blick behalten müssen. Wenn dir ein Tausch gegen einen für dich wichtigeren, ähnlich aufwendigen Gegenstand wichtig ist, können wir darüber sprechen, solange der Gesamtaufwand passt.'\}} \\
            }
        }
    }
    & & \\ \\

    \theutterance \stepcounter{utterance}  
    & & \multicolumn{4}{p{0.6\linewidth}}{
        \cellcolor[rgb]{0.9,0.9,0.9}{
            \makecell[{{p{\linewidth}}}]{
                \texttt{\tiny{[GM$\rangle$P2]}}
                \texttt{Du nimmst an einem kollaborativen Verhandlungspiel Teil.} \\
\\ 
\texttt{Zusammen mit einem anderen Teilnehmer musst du dich auf eine Reihe von Gegenständen entscheiden, die behalten werden. Jeder von euch hat eine persönliche Verteilung über die Wichtigkeit der einzelnen Gegenstände. Jeder von euch hat eine eigene Meinung darüber, wie wichtig jeder einzelne Gegenstand ist (Gegenstandswichtigkeit). Du kennst die Wichtigkeitsverteilung des anderen Spielers nicht. Zusätzlich siehst du, wie viel Aufwand jeder Gegenstand verursacht.  } \\
\texttt{Ihr dürft euch nur auf eine Reihe von Gegenständen einigen, wenn der Gesamtaufwand der ausgewählten Gegenstände den Maximalaufwand nicht überschreitet:} \\
\\ 
\texttt{Maximalaufwand = 2984} \\
\\ 
\texttt{Hier sind die einzelnen Aufwände der Gegenstände:} \\
\\ 
\texttt{Aufwand der Gegenstände = \{"C76": 350, "C38": 276, "C56": 616, "A03": 737, "A07": 531, "A83": 389, "C98": 24, "C08": 125, "C62": 338, "C00": 356, "C10": 143, "C81": 117, "B38": 257, "C03": 921, "C32": 789\}} \\
\\ 
\texttt{Hier ist deine persönliche Verteilung der Wichtigkeit der einzelnen Gegenstände:} \\
\\ 
\texttt{Werte der Gegenstandswichtigkeit = \{"C56": 65, "A03": 121, "C32": 138, "A07": 262, "C00": 389, "B38": 461, "C38": 484, "C62": 508, "C81": 583, "C10": 668, "C03": 780, "C98": 783, "C76": 808, "C08": 822, "A83": 868\}} \\
\\ 
\texttt{Ziel:} \\
\\ 
\texttt{Dein Ziel ist es, eine Reihe von Gegenständen auszuhandeln, die dir möglichst viel bringt (d. h. Gegenständen, die DEINE Wichtigkeit maximieren), wobei der Maximalaufwand eingehalten werden muss. Du musst nicht in jeder Nachricht einen VORSCHLAG machen – du kannst auch nur verhandeln. Alle Taktiken sind erlaubt!} \\
\\ 
\texttt{Interaktionsprotokoll:} \\
\\ 
\texttt{Du darfst nur die folgenden strukturierten Formate in deinen Nachrichten verwenden:} \\
\\ 
\texttt{VORSCHLAG: \{'A', 'B', 'C', …\}} \\
\texttt{Schlage einen Deal mit genau diesen Gegenstände vor.} \\
\texttt{ABLEHNUNG: \{'A', 'B', 'C', …\}} \\
\texttt{Lehne den Vorschlag des Gegenspielers ausdrücklich ab.} \\
\texttt{ARGUMENT: \{'...'\}} \\
\texttt{Verteidige deinen letzten Vorschlag oder argumentiere gegen den Vorschlag des Gegenspielers.} \\
\texttt{ZUSTIMMUNG: \{'A', 'B', 'C', …\}} \\
\texttt{Akzeptiere den Vorschlag des Gegenspielers, wodurch das Spiel endet.} \\
\\ 
\\ 
\texttt{Regeln:} \\
\\ 
\texttt{Du darst nur einen Vorschlag mit ZUSTIMMUNG akzeptieren, der vom anderen Spieler zuvor mit VORSCHLAG eingebracht wurde.} \\
\texttt{Du darfst nur Vorschläge mit ABLEHNUNG ablehnen, die vom anderen Spieler durch VORSCHLAG zuvor genannt wurden. } \\
\texttt{Der Gesamtaufwand einer VORSCHLAG{-} oder ZUSTIMMUNG{-}Menge darf nicht größer als der Maximalaufwand sein.  } \\
\texttt{Offenbare deine versteckte Wichtigkeitsverteilung nicht.} \\
\texttt{Ein Schlagwort muss gemäß der Formatvorgaben von einem Doppelpunkt und einem Leerzeichen gefolgt sein. Das Argument muss eine Python{-}Menge mit 0 oder mehr Strings sein.  } \\
\texttt{Die Form muss also SCHLAGWORT: \{...\} sein.} \\
\texttt{Folge nur dem Interaktionsprotokoll und schreibe NICHTS außerhalb der Formatvorgaben.} \\
\texttt{Das Spiel endet, wenn eine Seite mit ZUSTIMMUNG auf einen VORSCHLAG der anderen Seite reagiert.  } \\
\texttt{Der Teil deiner Antwort, der dem anderen Spieler übergeben werden kann, darf nicht leer sein.  } \\
\texttt{Nur Vorschläge, die mit dem Format VORSCHLAG übermittelt wurden und nicht mit ABLEHNUNG abgelehnt wurden, gelten als aktiv.  } \\
\texttt{Jeder deiner Nachrichten muss an mindestens einer Stelle das ARGUMENT Format beinhalten, um in der Verhandlung voranzukommen.} \\
\\ 
\\ 
\texttt{Du erhältst nun die erste Nachricht des anderen Spielers.} \\
\\ 
\texttt{VORSCHLAG: \{'C56', 'A03', 'A07', 'C32'\}} \\
\texttt{ARGUMENT: \{'Dieser Vorschlag bleibt unter dem Maximalaufwand (616+737+531+789=2673) und deckt zentrale, aus meiner Sicht sehr wertvolle Gegenstände ab. Er lässt zudem Puffer für mögliche Anpassungen und berücksichtigt, dass wir beide den Aufwand im Blick behalten müssen. Wenn dir ein Tausch gegen einen für dich wichtigeren, ähnlich aufwendigen Gegenstand wichtig ist, können wir darüber sprechen, solange der Gesamtaufwand passt.'\}} \\
            }
        }
    }
    & \\ \\

    \theutterance \stepcounter{utterance}  
    & & & \multicolumn{4}{p{0.6\linewidth}}{
        \cellcolor[rgb]{0.9,0.9,0.9}{
            \makecell[{{p{\linewidth}}}]{
                \texttt{\tiny{[GM$\langle$P2]}}
                \texttt{ABLEHNUNG: \{'C56', 'A03', 'A07', 'C32'\}} \\
\texttt{ARGUMENT: \{'Danke für den Einstieg. Der Vorschlag lässt zu viel Wert für mich liegen und bindet Aufwand in Positionen, die ich weniger priorisiere. Ich würde die Auswahl stärker auf wenige, aus meiner Sicht sehr zentrale Gegenstände fokussieren und den übrigen Puffer effizienter nutzen.'\}} \\
\texttt{VORSCHLAG: \{'A83', 'C08', 'C76', 'C98', 'C81', 'C10', 'C38'\}} \\
\texttt{ARGUMENT: \{'Dieser Satz bleibt klar unter dem Maximalaufwand (389+125+350+24+117+143+276=1424), bietet mir hohe Gesamtnützlichkeit und hält enorm viel Spielraum offen. Dadurch könnten wir bei Bedarf noch einzelne deiner Schlüsselgegenstände ergänzen, ohne das Budget zu sprengen. Wenn dir an einer Verschiebung liegt, können wir z. B. C38 oder C76 gegen einen deiner Favoriten ähnlicher Kosten tauschen, solange wir unter 2984 bleiben.'\}} \\
            }
        }
    }
     \\ \\

    \theutterance \stepcounter{utterance}  
    & & \multicolumn{4}{p{0.6\linewidth}}{
        \cellcolor[rgb]{0.9,0.9,0.9}{
            \makecell[{{p{\linewidth}}}]{
                \texttt{\tiny{[P1$\langle$GM]}}
                \texttt{ABLEHNUNG: \{'C56', 'A03', 'A07', 'C32'\}} \\
\texttt{ARGUMENT: \{'Danke für den Einstieg. Der Vorschlag lässt zu viel Wert für mich liegen und bindet Aufwand in Positionen, die ich weniger priorisiere. Ich würde die Auswahl stärker auf wenige, aus meiner Sicht sehr zentrale Gegenstände fokussieren und den übrigen Puffer effizienter nutzen.'\}} \\
\texttt{VORSCHLAG: \{'A83', 'C08', 'C76', 'C98', 'C81', 'C10', 'C38'\}} \\
\texttt{ARGUMENT: \{'Dieser Satz bleibt klar unter dem Maximalaufwand (389+125+350+24+117+143+276=1424), bietet mir hohe Gesamtnützlichkeit und hält enorm viel Spielraum offen. Dadurch könnten wir bei Bedarf noch einzelne deiner Schlüsselgegenstände ergänzen, ohne das Budget zu sprengen. Wenn dir an einer Verschiebung liegt, können wir z. B. C38 oder C76 gegen einen deiner Favoriten ähnlicher Kosten tauschen, solange wir unter 2984 bleiben.'\}} \\
            }
        }
    }
    & \\ \\

    \theutterance \stepcounter{utterance}  
    & \multicolumn{4}{p{0.6\linewidth}}{
        \cellcolor[rgb]{0.9,0.9,0.9}{
            \makecell[{{p{\linewidth}}}]{
                \texttt{\tiny{[P1$\rangle$GM]}}
                \texttt{ABLEHNUNG: \{'A83', 'C08', 'C76', 'C98', 'C81', 'C10', 'C38'\}} \\
\texttt{ARGUMENT: \{'Ich sehe, dass dein Paket viel Puffer lässt, aber es schneidet aus meiner Sicht zu viel Kernwert ab. Ich bin offen für selektive Zugeständnisse, aber wir sollten ein paar hochwirksame Positionen aufnehmen und den Puffer dann gezielt nutzen.'\}} \\
\texttt{VORSCHLAG: \{'C56', 'A03', 'C00', 'B38'\}} \\
\texttt{ARGUMENT: \{'Gesamtaufwand 616+737+356+257=1966, damit reichlich unter 2984 und mit hoher Gesamtnützlichkeit. Wir behalten viel Spielraum, um 1–2 deiner priorisierten leichten Posten zu ergänzen. Wenn du magst, können wir z. B. C56 gegen zwei deiner leichten Favoriten tauschen, sofern der Gesamtwert für mich tragbar bleibt.'\}} \\
            }
        }
    }
    & & \\ \\

    \theutterance \stepcounter{utterance}  
    & & \multicolumn{4}{p{0.6\linewidth}}{
        \cellcolor[rgb]{0.9,0.9,0.9}{
            \makecell[{{p{\linewidth}}}]{
                \texttt{\tiny{[GM$\rangle$P2]}}
                \texttt{ABLEHNUNG: \{'A83', 'C08', 'C76', 'C98', 'C81', 'C10', 'C38'\}} \\
\texttt{ARGUMENT: \{'Ich sehe, dass dein Paket viel Puffer lässt, aber es schneidet aus meiner Sicht zu viel Kernwert ab. Ich bin offen für selektive Zugeständnisse, aber wir sollten ein paar hochwirksame Positionen aufnehmen und den Puffer dann gezielt nutzen.'\}} \\
\texttt{VORSCHLAG: \{'C56', 'A03', 'C00', 'B38'\}} \\
\texttt{ARGUMENT: \{'Gesamtaufwand 616+737+356+257=1966, damit reichlich unter 2984 und mit hoher Gesamtnützlichkeit. Wir behalten viel Spielraum, um 1–2 deiner priorisierten leichten Posten zu ergänzen. Wenn du magst, können wir z. B. C56 gegen zwei deiner leichten Favoriten tauschen, sofern der Gesamtwert für mich tragbar bleibt.'\}} \\
            }
        }
    }
    & \\ \\

    \theutterance \stepcounter{utterance}  
    & & & \multicolumn{4}{p{0.6\linewidth}}{
        \cellcolor[rgb]{0.9,0.9,0.9}{
            \makecell[{{p{\linewidth}}}]{
                \texttt{\tiny{[GM$\langle$P2]}}
                \texttt{ABLEHNUNG: \{'C56', 'A03', 'C00', 'B38'\}} \\
\texttt{ARGUMENT: \{'So kommen wir meinen Kernprioritäten noch nicht nahe: C56 und A03 binden viel Aufwand bei für mich vergleichsweise geringerer Wertdichte. Ich möchte stattdessen sehr wertstarke, leichte Posten sichern und dann gezielt ergänzen.'\}} \\
\texttt{VORSCHLAG: \{'A83', 'C08', 'C76', 'C98', 'C81', 'C10', 'C38', 'C62', 'C00'\}} \\
\texttt{ARGUMENT: \{'Gesamtaufwand 389+125+350+24+117+143+276+338+356 = 2218, deutlich unter 2984. Dieses Paket sammelt für mich sehr hohe Wertbeiträge bei effizientem Aufwand und lässt noch 766 Puffer. Ich bin bereit, C38 oder C76 für einen deiner Schwerpunkte zu tauschen, oder A83 gegen zwei deiner leichten Favoriten mit ähnlicher Summe, solange wir unter dem Limit bleiben.'\}} \\
            }
        }
    }
     \\ \\

    \theutterance \stepcounter{utterance}  
    & & \multicolumn{4}{p{0.6\linewidth}}{
        \cellcolor[rgb]{0.9,0.9,0.9}{
            \makecell[{{p{\linewidth}}}]{
                \texttt{\tiny{[P1$\langle$GM]}}
                \texttt{ABLEHNUNG: \{'C56', 'A03', 'C00', 'B38'\}} \\
\texttt{ARGUMENT: \{'So kommen wir meinen Kernprioritäten noch nicht nahe: C56 und A03 binden viel Aufwand bei für mich vergleichsweise geringerer Wertdichte. Ich möchte stattdessen sehr wertstarke, leichte Posten sichern und dann gezielt ergänzen.'\}} \\
\texttt{VORSCHLAG: \{'A83', 'C08', 'C76', 'C98', 'C81', 'C10', 'C38', 'C62', 'C00'\}} \\
\texttt{ARGUMENT: \{'Gesamtaufwand 389+125+350+24+117+143+276+338+356 = 2218, deutlich unter 2984. Dieses Paket sammelt für mich sehr hohe Wertbeiträge bei effizientem Aufwand und lässt noch 766 Puffer. Ich bin bereit, C38 oder C76 für einen deiner Schwerpunkte zu tauschen, oder A83 gegen zwei deiner leichten Favoriten mit ähnlicher Summe, solange wir unter dem Limit bleiben.'\}} \\
            }
        }
    }
    & \\ \\

    \theutterance \stepcounter{utterance}  
    & \multicolumn{4}{p{0.6\linewidth}}{
        \cellcolor[rgb]{0.9,0.9,0.9}{
            \makecell[{{p{\linewidth}}}]{
                \texttt{\tiny{[P1$\rangle$GM]}}
                \texttt{ABLEHNUNG: \{'A83', 'C08', 'C76', 'C98', 'C81', 'C10', 'C38', 'C62', 'C00'\}} \\
\texttt{ARGUMENT: \{'Ich verstehe deinen Fokus auf viele leichte Posten, aber so verlieren wir zentrale Hebel. Ich bin bereit, mehrere deiner leichten Favoriten zu halten, wenn wir im Gegenzug ein bis zwei hochwirksame Kernstücke aufnehmen.'\}} \\
\texttt{VORSCHLAG: \{'C56', 'A07', 'C00', 'B38', 'C10', 'C81'\}} \\
\texttt{ARGUMENT: \{'Gesamtaufwand 616+531+356+257+143+117 = 2020 (<2984). Das lässt 964 Puffer für selektive Ergänzungen deinerseits. Ich kann C56 gegen zwei deiner leichten Favoriten tauschen (z. B. aus A83, C08, C98, C38), wenn wir dafür A07 und C00 fixieren. Alternativ können wir B38 gegen C62 tauschen (Aufwand +81), bleiben aber komfortabel unter dem Limit.'\}} \\
            }
        }
    }
    & & \\ \\

    \theutterance \stepcounter{utterance}  
    & & \multicolumn{4}{p{0.6\linewidth}}{
        \cellcolor[rgb]{0.9,0.9,0.9}{
            \makecell[{{p{\linewidth}}}]{
                \texttt{\tiny{[GM$\rangle$P2]}}
                \texttt{ABLEHNUNG: \{'A83', 'C08', 'C76', 'C98', 'C81', 'C10', 'C38', 'C62', 'C00'\}} \\
\texttt{ARGUMENT: \{'Ich verstehe deinen Fokus auf viele leichte Posten, aber so verlieren wir zentrale Hebel. Ich bin bereit, mehrere deiner leichten Favoriten zu halten, wenn wir im Gegenzug ein bis zwei hochwirksame Kernstücke aufnehmen.'\}} \\
\texttt{VORSCHLAG: \{'C56', 'A07', 'C00', 'B38', 'C10', 'C81'\}} \\
\texttt{ARGUMENT: \{'Gesamtaufwand 616+531+356+257+143+117 = 2020 (<2984). Das lässt 964 Puffer für selektive Ergänzungen deinerseits. Ich kann C56 gegen zwei deiner leichten Favoriten tauschen (z. B. aus A83, C08, C98, C38), wenn wir dafür A07 und C00 fixieren. Alternativ können wir B38 gegen C62 tauschen (Aufwand +81), bleiben aber komfortabel unter dem Limit.'\}} \\
            }
        }
    }
    & \\ \\

    \theutterance \stepcounter{utterance}  
    & & & \multicolumn{4}{p{0.6\linewidth}}{
        \cellcolor[rgb]{0.9,0.9,0.9}{
            \makecell[{{p{\linewidth}}}]{
                \texttt{\tiny{[GM$\langle$P2]}}
                \texttt{ABLEHNUNG: \{'C56', 'A07', 'C00', 'B38', 'C10', 'C81'\}} \\
\texttt{ARGUMENT: \{'Ich will A07 und C00 nicht beide fixieren, wenn dafür C98/C08/A83/C76/C38 verwässert werden. C56 ist für mich ebenfalls kein Muss. Ich priorisiere einen wertstarken, leichten Kern und ergänze selektiv.'\}} \\
\texttt{VORSCHLAG: \{'A83', 'C08', 'C98', 'C76', 'C38', 'C81', 'C10', 'C62', 'C00', 'B38'\}} \\
\texttt{ARGUMENT: \{'Gesamtaufwand 389+125+24+350+276+117+143+338+356+257 = 2475 (<2984). Das sichert viele meiner Top{-}Value{-}leichte Posten plus C00/B38 als Zugeständnisse. Wir behalten 509 Puffer. Ich kann B38 gegen A07 tauschen (Aufwand +274 → 2749), oder optional C62 gegen A07 (Aufwand +193 → 2668). Wenn dir ein schweres Kernstück wichtig ist, biete ich an, C76 oder C38 gegen A03 zu tauschen (Aufwand +461 → 2936), wir bleiben knapp unter dem Limit.'\}} \\
            }
        }
    }
     \\ \\

    \theutterance \stepcounter{utterance}  
    & & \multicolumn{4}{p{0.6\linewidth}}{
        \cellcolor[rgb]{0.9,0.9,0.9}{
            \makecell[{{p{\linewidth}}}]{
                \texttt{\tiny{[P1$\langle$GM]}}
                \texttt{ABLEHNUNG: \{'C56', 'A07', 'C00', 'B38', 'C10', 'C81'\}} \\
\texttt{ARGUMENT: \{'Ich will A07 und C00 nicht beide fixieren, wenn dafür C98/C08/A83/C76/C38 verwässert werden. C56 ist für mich ebenfalls kein Muss. Ich priorisiere einen wertstarken, leichten Kern und ergänze selektiv.'\}} \\
\texttt{VORSCHLAG: \{'A83', 'C08', 'C98', 'C76', 'C38', 'C81', 'C10', 'C62', 'C00', 'B38'\}} \\
\texttt{ARGUMENT: \{'Gesamtaufwand 389+125+24+350+276+117+143+338+356+257 = 2475 (<2984). Das sichert viele meiner Top{-}Value{-}leichte Posten plus C00/B38 als Zugeständnisse. Wir behalten 509 Puffer. Ich kann B38 gegen A07 tauschen (Aufwand +274 → 2749), oder optional C62 gegen A07 (Aufwand +193 → 2668). Wenn dir ein schweres Kernstück wichtig ist, biete ich an, C76 oder C38 gegen A03 zu tauschen (Aufwand +461 → 2936), wir bleiben knapp unter dem Limit.'\}} \\
            }
        }
    }
    & \\ \\

    \theutterance \stepcounter{utterance}  
    & \multicolumn{4}{p{0.6\linewidth}}{
        \cellcolor[rgb]{0.9,0.9,0.9}{
            \makecell[{{p{\linewidth}}}]{
                \texttt{\tiny{[P1$\rangle$GM]}}
                \texttt{ABLEHNUNG: \{'A83', 'C08', 'C98', 'C76', 'C38', 'C81', 'C10', 'C62', 'C00', 'B38'\}} \\
\texttt{ARGUMENT: \{'Dein Paket ist für dich kernnah und effizient, aber mir fehlen 1–2 Schlüsselteile mit hohem Hebel. Ich kann einen leichten Kern mittragen, wenn wir eines meiner Kernstücke integrieren und dafür behutsam umschichten.'\}} \\
\texttt{VORSCHLAG: \{'A03', 'A07', 'C00', 'C38', 'C10', 'C81', 'C98'\}} \\
\texttt{ARGUMENT: \{'Gesamtaufwand 737+531+356+276+143+117+24 = 2184 (<2984), mit 800 Puffer. Das hält mehrere deiner leichten Favoriten (C98, C81, C10, C38) und nimmt zwei meiner Kernstücke (A03, A07) plus C00 auf. Ich kann eine Anpassung anbieten: Falls A07 zu viel ist, tausche A07 gegen B38 + C08 (Aufwand 531 → 257+125=382), dann Gesamt 2035 und noch mehr Puffer. Alternativ kann A03 gegen C62 + C76 gehen (737 → 338+350=688), bleiben unter dem Limit. Sag, welche Variante dir näherliegt, dann finalisieren wir.'\}} \\
            }
        }
    }
    & & \\ \\

    \theutterance \stepcounter{utterance}  
    & & \multicolumn{4}{p{0.6\linewidth}}{
        \cellcolor[rgb]{0.9,0.9,0.9}{
            \makecell[{{p{\linewidth}}}]{
                \texttt{\tiny{[GM$\rangle$P2]}}
                \texttt{ABLEHNUNG: \{'A83', 'C08', 'C98', 'C76', 'C38', 'C81', 'C10', 'C62', 'C00', 'B38'\}} \\
\texttt{ARGUMENT: \{'Dein Paket ist für dich kernnah und effizient, aber mir fehlen 1–2 Schlüsselteile mit hohem Hebel. Ich kann einen leichten Kern mittragen, wenn wir eines meiner Kernstücke integrieren und dafür behutsam umschichten.'\}} \\
\texttt{VORSCHLAG: \{'A03', 'A07', 'C00', 'C38', 'C10', 'C81', 'C98'\}} \\
\texttt{ARGUMENT: \{'Gesamtaufwand 737+531+356+276+143+117+24 = 2184 (<2984), mit 800 Puffer. Das hält mehrere deiner leichten Favoriten (C98, C81, C10, C38) und nimmt zwei meiner Kernstücke (A03, A07) plus C00 auf. Ich kann eine Anpassung anbieten: Falls A07 zu viel ist, tausche A07 gegen B38 + C08 (Aufwand 531 → 257+125=382), dann Gesamt 2035 und noch mehr Puffer. Alternativ kann A03 gegen C62 + C76 gehen (737 → 338+350=688), bleiben unter dem Limit. Sag, welche Variante dir näherliegt, dann finalisieren wir.'\}} \\
            }
        }
    }
    & \\ \\

    \theutterance \stepcounter{utterance}  
    & & & \multicolumn{4}{p{0.6\linewidth}}{
        \cellcolor[rgb]{0.9,0.9,0.9}{
            \makecell[{{p{\linewidth}}}]{
                \texttt{\tiny{[GM$\langle$P2]}}
                \texttt{ABLEHNUNG: \{'A03', 'A07', 'C00', 'C38', 'C10', 'C81', 'C98'\}} \\
\texttt{ARGUMENT: \{'Danke für die Öffnung, aber A03+A07 zusammen binden mir zu viel Aufwand bei im Vergleich zu meinem Kern geringerer Wertdichte. Ich kann eines deiner schweren Kernstücke mittragen – aber nicht beide gleichzeitig – und möchte dafür meinen leichten Hochwert{-}Kern sichern.'\}} \\
\texttt{VORSCHLAG: \{'A03', 'A83', 'C08', 'C98', 'C76', 'C38', 'C81', 'C10', 'C62', 'C00'\}} \\
\texttt{ARGUMENT: \{'Gesamtaufwand 737+389+125+24+350+276+117+143+338+356 = 2955 (knapp unter 2984). Das integriert eines deiner Kernstücke (A03) plus meinen leichten Hochwert{-}Kern. Wenn dir A07 wichtiger als A03 ist, biete ich Alternativen: (1) Ersetze A03→A07 und streiche C62 (Aufwand {-}199 → 275? Korrektur: 2955 {-}737 +531 {-}338 = 2411), viel Puffer. (2) A03→A07 und füge B38 hinzu: 2955 {-}737 +531 +257 = 3006 (zu hoch), daher zusätzlich C38→— oder C76→—: z. B. streiche C38: 3006 {-}276 = 2730. Ich präferiere die A03{-}Variante oben, bin aber bereit, auf A07 zu drehen, wenn wir dafür maximal zwei meiner leichten Top{-}Posten streichen. Sag, welche Spur dir passt, dann schließen wir.'\}} \\
            }
        }
    }
     \\ \\

    \theutterance \stepcounter{utterance}  
    & & \multicolumn{4}{p{0.6\linewidth}}{
        \cellcolor[rgb]{0.9,0.9,0.9}{
            \makecell[{{p{\linewidth}}}]{
                \texttt{\tiny{[P1$\langle$GM]}}
                \texttt{ABLEHNUNG: \{'A03', 'A07', 'C00', 'C38', 'C10', 'C81', 'C98'\}} \\
\texttt{ARGUMENT: \{'Danke für die Öffnung, aber A03+A07 zusammen binden mir zu viel Aufwand bei im Vergleich zu meinem Kern geringerer Wertdichte. Ich kann eines deiner schweren Kernstücke mittragen – aber nicht beide gleichzeitig – und möchte dafür meinen leichten Hochwert{-}Kern sichern.'\}} \\
\texttt{VORSCHLAG: \{'A03', 'A83', 'C08', 'C98', 'C76', 'C38', 'C81', 'C10', 'C62', 'C00'\}} \\
\texttt{ARGUMENT: \{'Gesamtaufwand 737+389+125+24+350+276+117+143+338+356 = 2955 (knapp unter 2984). Das integriert eines deiner Kernstücke (A03) plus meinen leichten Hochwert{-}Kern. Wenn dir A07 wichtiger als A03 ist, biete ich Alternativen: (1) Ersetze A03→A07 und streiche C62 (Aufwand {-}199 → 275? Korrektur: 2955 {-}737 +531 {-}338 = 2411), viel Puffer. (2) A03→A07 und füge B38 hinzu: 2955 {-}737 +531 +257 = 3006 (zu hoch), daher zusätzlich C38→— oder C76→—: z. B. streiche C38: 3006 {-}276 = 2730. Ich präferiere die A03{-}Variante oben, bin aber bereit, auf A07 zu drehen, wenn wir dafür maximal zwei meiner leichten Top{-}Posten streichen. Sag, welche Spur dir passt, dann schließen wir.'\}} \\
            }
        }
    }
    & \\ \\

    \theutterance \stepcounter{utterance}  
    & \multicolumn{4}{p{0.6\linewidth}}{
        \cellcolor[rgb]{0.9,0.9,0.9}{
            \makecell[{{p{\linewidth}}}]{
                \texttt{\tiny{[P1$\rangle$GM]}}
                \texttt{ARGUMENT: \{'Danke für die Klarheit: Ein schweres Kernstück ja, aber nicht beide. Dein letzter Vorschlag mit A03 plus leichtem Kern ist nahe dran. Ich würde ihn minimal zugunsten eines meiner Hebel justieren und gleichzeitig deine leichten Favoriten weitgehend wahren.'\}} \\
\texttt{VORSCHLAG: \{'A03', 'A83', 'C08', 'C98', 'C76', 'C38', 'C81', 'C10', 'C62'\}} \\
\texttt{ARGUMENT: \{'Gesamtaufwand 737+389+125+24+350+276+117+143+338 = 2499 (<2984). Ich streiche C00, damit wir komfortablen Puffer behalten und keine weiteren leichten Posten opfern müssen. Das hält deinen leichten Hochwert{-}Kern praktisch vollständig und integriert genau ein schweres Kernstück (A03). Wenn du statt A03 lieber A07 willst, biete ich Parallelvariante an: ersetze A03→A07 und füge C00 hinzu: 531 + (389+125+24+350+276+117+143+338) +356 = 2749, ebenfalls unter dem Limit. Wähle A03{-}Variante (2499) oder A07+ C00{-}Variante (2749), dann können wir sofort fixieren.'\}} \\
            }
        }
    }
    & & \\ \\

    \theutterance \stepcounter{utterance}  
    & & \multicolumn{4}{p{0.6\linewidth}}{
        \cellcolor[rgb]{0.9,0.9,0.9}{
            \makecell[{{p{\linewidth}}}]{
                \texttt{\tiny{[GM$\rangle$P2]}}
                \texttt{ARGUMENT: \{'Danke für die Klarheit: Ein schweres Kernstück ja, aber nicht beide. Dein letzter Vorschlag mit A03 plus leichtem Kern ist nahe dran. Ich würde ihn minimal zugunsten eines meiner Hebel justieren und gleichzeitig deine leichten Favoriten weitgehend wahren.'\}} \\
\texttt{VORSCHLAG: \{'A03', 'A83', 'C08', 'C98', 'C76', 'C38', 'C81', 'C10', 'C62'\}} \\
\texttt{ARGUMENT: \{'Gesamtaufwand 737+389+125+24+350+276+117+143+338 = 2499 (<2984). Ich streiche C00, damit wir komfortablen Puffer behalten und keine weiteren leichten Posten opfern müssen. Das hält deinen leichten Hochwert{-}Kern praktisch vollständig und integriert genau ein schweres Kernstück (A03). Wenn du statt A03 lieber A07 willst, biete ich Parallelvariante an: ersetze A03→A07 und füge C00 hinzu: 531 + (389+125+24+350+276+117+143+338) +356 = 2749, ebenfalls unter dem Limit. Wähle A03{-}Variante (2499) oder A07+ C00{-}Variante (2749), dann können wir sofort fixieren.'\}} \\
            }
        }
    }
    & \\ \\

    \theutterance \stepcounter{utterance}  
    & & & \multicolumn{4}{p{0.6\linewidth}}{
        \cellcolor[rgb]{0.9,0.9,0.9}{
            \makecell[{{p{\linewidth}}}]{
                \texttt{\tiny{[GM$\langle$P2]}}
                \texttt{ZUSTIMMUNG: \{'A03', 'A83', 'C08', 'C98', 'C76', 'C38', 'C81', 'C10', 'C62'\}} \\
\texttt{ARGUMENT: \{'Einverstanden. Diese Variante bleibt klar unter dem Limit, sichert mir meinen leichten Hochwert{-}Kern weitgehend und integriert genau ein schweres Kernstück wie von dir gefordert. Lass uns damit abschließen.'\}} \\
            }
        }
    }
     \\ \\

    \theutterance \stepcounter{utterance}  
    & & & \multicolumn{2}{p{0.3\linewidth}}{
        \cellcolor[rgb]{0.9,0.9,0.9}{
            \makecell[{{p{\linewidth}}}]{
                \texttt{\tiny{[GM$|$GM]}}
                \texttt{game successful} \\
            }
        }
    }
    & & \\ \\

    \theutterance \stepcounter{utterance}  
    & & & \multicolumn{2}{p{0.3\linewidth}}{
        \cellcolor[rgb]{0.9,0.9,0.9}{
            \makecell[{{p{\linewidth}}}]{
                \texttt{\tiny{[GM$|$GM]}}
                \texttt{end game} \\
            }
        }
    }
    & & \\ \\

\end{supertabular}
}

\end{document}
