
\documentclass{article}
\usepackage{colortbl}
\usepackage{makecell}
\usepackage{multirow}
\usepackage{supertabular}

\begin{document}

\newcounter{utterance}

\twocolumn

{ \footnotesize  \setcounter{utterance}{1}
\setlength{\tabcolsep}{0pt}
\begin{supertabular}{c@{$\;$}|p{.15\linewidth}@{}p{.15\linewidth}p{.15\linewidth}p{.15\linewidth}p{.15\linewidth}p{.15\linewidth}}

    \# & $\;$A & \multicolumn{4}{c}{Game Master} & $\;\:$B\\
    \hline 

    \theutterance \stepcounter{utterance}  

    & & \multicolumn{4}{p{0.6\linewidth}}{\cellcolor[rgb]{0.9,0.9,0.9}{%
	\makecell[{{p{\linewidth}}}]{% 
	  \tt {\tiny [A$\langle$GM]}  
	 Du nimmst an einem kollaborativen Verhandlungspiel Teil.\\ \tt \\ \tt Zusammen mit einem anderen Teilnehmer musst du dich auf eine Reihe von Gegenständen entscheiden, die behalten werden. Jeder von euch hat eine persönliche Verteilung über die Wichtigkeit der einzelnen Gegenstände. Jeder von euch hat eine eigene Meinung darüber, wie wichtig jeder einzelne Gegenstand ist (Gegenstandswichtigkeit). Du kennst die Wichtigkeitsverteilung des anderen Spielers nicht. Zusätzlich siehst du, wie viel Aufwand jeder Gegenstand verursacht.  \\ \tt Ihr dürft euch nur auf eine Reihe von Gegenständen einigen, wenn der Gesamtaufwand der ausgewählten Gegenstände den Maximalaufwand nicht überschreitet:\\ \tt \\ \tt Maximalaufwand = 4026\\ \tt \\ \tt Hier sind die einzelnen Aufwände der Gegenstände:\\ \tt \\ \tt Aufwand der Gegenstände = {"C33": 780, "A89": 742, "A48": 997, "A04": 665, "A41": 140, "C64": 77, "C49": 513, "A61": 383, "C04": 587, "B53": 825, "B85": 319, "B61": 448, "A45": 516, "A03": 694, "B21": 366}\\ \tt \\ \tt Hier ist deine persönliche Verteilung der Wichtigkeit der einzelnen Gegenstände:\\ \tt \\ \tt Werte der Gegenstandswichtigkeit = {"C33": 138, "A89": 583, "A48": 868, "A04": 822, "A41": 783, "C64": 65, "C49": 262, "A61": 121, "C04": 508, "B53": 780, "B85": 461, "B61": 484, "A45": 668, "A03": 389, "B21": 808}\\ \tt \\ \tt Ziel:\\ \tt \\ \tt Dein Ziel ist es, eine Reihe von Gegenständen auszuhandeln, die dir möglichst viel bringt (d. h. Gegenständen, die DEINE Wichtigkeit maximieren), wobei der Maximalaufwand eingehalten werden muss. Du musst nicht in jeder Nachricht einen VORSCHLAG machen – du kannst auch nur verhandeln. Alle Taktiken sind erlaubt!\\ \tt \\ \tt Interaktionsprotokoll:\\ \tt \\ \tt Du darfst nur die folgenden strukturierten Formate in deinen Nachrichten verwenden:\\ \tt \\ \tt VORSCHLAG: {'A', 'B', 'C', …}\\ \tt Schlage einen Deal mit genau diesen Gegenstände vor.\\ \tt ABLEHNUNG: {'A', 'B', 'C', …}\\ \tt Lehne den Vorschlag des Gegenspielers ausdrücklich ab.\\ \tt ARGUMENT: {'...'}\\ \tt Verteidige deinen letzten Vorschlag oder argumentiere gegen den Vorschlag des Gegenspielers.\\ \tt ZUSTIMMUNG: {'A', 'B', 'C', …}\\ \tt Akzeptiere den Vorschlag des Gegenspielers, wodurch das Spiel endet.\\ \tt STRATEGISCHE ÜBERLEGUNGEN: {'...'}\\ \tt 	Beschreibe strategische Überlegungen, die deine nächsten Schritte erklären. Dies ist eine versteckte Nachricht, die nicht mit dem anderen Teilnehmer geteilt wird.\\ \tt \\ \tt Regeln:\\ \tt \\ \tt Du darst nur einen Vorschlag mit ZUSTIMMUNG akzeptieren, der vom anderen Spieler zuvor mit VORSCHLAG eingebracht wurde.\\ \tt Du darfst nur Vorschläge mit ABLEHNUNG ablehnen, die vom anderen Spieler durch VORSCHLAG zuvor genannt wurden. \\ \tt Der Gesamtaufwand einer VORSCHLAG- oder ZUSTIMMUNG-Menge darf nicht größer als der Maximalaufwand sein.  \\ \tt Offenbare deine versteckte Wichtigkeitsverteilung nicht.\\ \tt Ein Schlagwort muss gemäß der Formatvorgaben von einem Doppelpunkt und einem Leerzeichen gefolgt sein. Das Argument muss eine Python-Menge mit 0 oder mehr Strings sein.  \\ \tt Die Form muss also SCHLAGWORT: {...} sein.\\ \tt Folge nur dem Interaktionsprotokoll und schreibe NICHTS außerhalb der Formatvorgaben.\\ \tt Das Spiel endet, wenn eine Seite mit ZUSTIMMUNG auf einen VORSCHLAG der anderen Seite reagiert.  \\ \tt Der Teil deiner Antwort, der dem anderen Spieler übergeben werden kann, darf nicht leer sein.  \\ \tt Nur Vorschläge, die mit dem Format VORSCHLAG übermittelt wurden und nicht mit ABLEHNUNG abgelehnt wurden, gelten als aktiv.  \\ \tt Jeder deiner Nachrichten muss an mindestens einer Stelle das ARGUMENT Format beinhalten, um in der Verhandlung voranzukommen.\\ \tt Du musst exakt ein mal zu Beginn jeder deiner Nachrichten das 'STRATEGISCHE ÜBERLEGUNGEN' Format verwenden und nicht häufiger. Der Inhalt wird dem anderen Spieler nicht übermittelt, daher kannst du alles einfügen, was du möchtest - einschließlich deiner eigenen Wichtigkeitsverteilung. Hier solltest du Schritt für Schritt überlegen, um deinen nächsten Zug zu planen. 
	  } 
	   } 
	   } 
	 & \\ 
 

    \theutterance \stepcounter{utterance}  

    & \multicolumn{4}{p{0.6\linewidth}}{\cellcolor[rgb]{0.8,1,0.9}{%
	\makecell[{{p{\linewidth}}}]{% 
	  \tt {\tiny [A$\rangle$GM]}  
	 STRATEGISCHE ÜBERLEGUNGEN: {'Ziel: Maximierung meiner Summe bei Aufwand ≤ 4026. Werte/Costs/ratio sortieren. Items mit hohem Wert: A48(868/997), A04(822/665), B21(808/366), B53(780/825), A41(783/140), A45(668/516), B61(484/448), B85(461/319), C04(508/587), A89(583/742), A03(389/694), C49(262/513), A61(121/383), C33(138/780), C64(65/77). Greedy by value and ratio suggests include A48,A04,B21,A41,B53,A45,B85,B61,C04,A89. Check budgets incrementally: Start with A48(997)=997; +A04(665)=1662; +B21(366)=2028; +A41(140)=2168; +B53(825)=2993; +A45(516)=3509; +B85(319)=3828; Remaining budget 198; add B61(448) no; C04(587) no; A89(742) no; A03(694) no; C49(513) no; A61(383) no; C33(780) no; C64(77) yes -> 3828+77=3905; leftover 121 can add none except maybe A61(383) no. Value sum for set S1={A48,A04,B21,A41,B53,A45,B85,C64} = 868+822+808+783+780+668+461+65=5255. Try alternative without B53 (high cost) to include more: S2 start same but skip B53: cost 2168; add A45(516)=2684; add B61(448)=3132; add B85(319)=3451; add C04(587)=4038 exceeds; add A89(742) exceeds; add A03(694) exceeds; add C49(513)=3965; add C64(77)=4042 exceeds; remove C49, add C64=3528; leftover 498, can add B61 already included; none fits except maybe ??? 498 >= B61(448) already; C49(513) too big; A61(383) can add: 3528+383=3911; leftover 115 none. Value S2={A48,A04,B21,A41,A45,B61,B85,A61,C64}=868+822+808+783+668+484+461+121+65=5080 < S1. Try variant replace B53 with C04+A89? From 2168 baseline, add A89(742)=2910; add C04(587)=3497; add B61(448)=3945; add C64(77)=4022 within; plus maybe A61(383) no. Value S3={A48,A04,B21,A41,A89,C04,B61,C64}=868+822+808+783+583+508+484+65=4921 < S1. Try include B53 and also B61 instead of C64 by removing something: From S1 cost 3905; can we swap C64(77,val65) for B61(448,val484) would add +371 cost -> 4276 exceeds. Swap B85(319,461) for B61(448,484): +129 cost to 4034 exceeds. Swap A45(516,668) with B61 reduces cost -68? Actually B61 cost 448 < 516, but value 484 < 668; If we remove A45 and add B61 and C64: From base without A45 (3509-516=2993); add B61(448)=3441; add C64(77)=3518; we can then maybe add C04(587)=4105 exceeds; A89(742) exceeds; A61(383)=3901; Then S4={A48,A04,B21,A41,B53,B61,B85,C64?,A61} compute: I had base with B53 and B85 already; Let's recompute: Start A48,A04,B21,A41,B53 = 2993; add B61=3441; add B85=3760; add A61=4143 exceeds; add C64=3837; leftover 189 none. Value S4={A48,A04,B21,A41,B53,B61,B85,C64}=868+822+808+783+780+484+461+65=5071 < S1. Try add A89 instead of B85 or C64: Base 2993; add B61=3441; add A89=4183 exceeds; add A45=3957 within; can we then add C64=4034 exceeds. Value set {A48,A04,B21,A41,B53,B61,A45}=868+822+808+783+780+484+668=5213 < S1 by 42; cost 3957, good. Maybe we can add swapping B85(319,461) instead of C64 to improve S1: S1 currently had B85 included; it does. S1=5255 best so far. Check possibility of adding A89 by removing two: S1 cost 3905, to add A89(742) need free 742-? we must remove items totaling ≥742. Candidates low value: C64(65,77) + B85(461,319) + maybe C49 none in set; Remove C64(77) and B85(319) frees 396, still need 346: could remove A45(516,668) frees enough: new cost 3905-77-319-516=2993; add A89 742 -> 3735; we can then add B61 448 -> 4183 exceeds; maybe add A61 383 -> 4118 exceeds; add C49 513 -> 4248 exceeds; add C04 587 -> 4322 exceeds; add A03 694 -> 4429; Instead we could choose a different combo: What’s the value after removing those and adding A89: values = 5255 -65 -461 -668 +583 = 4644, much worse. Try remove C64 and B53 to add A89 and B61: cost 3905-77-825=3003; add A89(742)=3745; add B61(448)=4193 exceeds. Maybe add A61(383)=4128 exceeds; No. Remove B53 only to add A89: 3905-825+742=3822; value: 5255-780+583=5058 < S1. Try remove C64 and swap B53->B61 and A89? 3905 -77 -825 +448 +742 = 4193 exceeds. So S1 seems best. I will propose S1 as opening, defend with high efficiency and ask for justification if they object.'}\\ \tt VORSCHLAG: {'A48', 'A04', 'B21', 'A41', 'B53', 'A45', 'B85', 'C64'}\\ \tt ARGUMENT: {'Dieser Vorschlag nutzt das Budget sehr effizient: Gesamtaufwand 3905 von 4026, also nahe am Maximum, ohne es zu überschreiten. Jeder enthaltene Gegenstand bringt aus meiner Sicht starken gemeinsamen Nutzen bei moderatem Aufwand. Wenn du Einwände hast, sag mir bitte konkret, welche 1–2 Positionen dir weniger wichtig sind oder welche dir fehlen, dann können wir gezielt tauschen.'} 
	  } 
	   } 
	   } 
	 & & \\ 
 

    \theutterance \stepcounter{utterance}  

    & & \multicolumn{4}{p{0.6\linewidth}}{\cellcolor[rgb]{0.9,0.9,0.9}{%
	\makecell[{{p{\linewidth}}}]{% 
	  \tt {\tiny [GM$\rangle$B]}  
	 Du nimmst an einem kollaborativen Verhandlungspiel Teil.\\ \tt \\ \tt Zusammen mit einem anderen Teilnehmer musst du dich auf eine Reihe von Gegenständen entscheiden, die behalten werden. Jeder von euch hat eine persönliche Verteilung über die Wichtigkeit der einzelnen Gegenstände. Jeder von euch hat eine eigene Meinung darüber, wie wichtig jeder einzelne Gegenstand ist (Gegenstandswichtigkeit). Du kennst die Wichtigkeitsverteilung des anderen Spielers nicht. Zusätzlich siehst du, wie viel Aufwand jeder Gegenstand verursacht.  \\ \tt Ihr dürft euch nur auf eine Reihe von Gegenständen einigen, wenn der Gesamtaufwand der ausgewählten Gegenstände den Maximalaufwand nicht überschreitet:\\ \tt \\ \tt Maximalaufwand = 4026\\ \tt \\ \tt Hier sind die einzelnen Aufwände der Gegenstände:\\ \tt \\ \tt Aufwand der Gegenstände = {"C33": 780, "A89": 742, "A48": 997, "A04": 665, "A41": 140, "C64": 77, "C49": 513, "A61": 383, "C04": 587, "B53": 825, "B85": 319, "B61": 448, "A45": 516, "A03": 694, "B21": 366}\\ \tt \\ \tt Hier ist deine persönliche Verteilung der Wichtigkeit der einzelnen Gegenstände:\\ \tt \\ \tt Werte der Gegenstandswichtigkeit = {"A48": 65, "A04": 121, "B21": 138, "A41": 262, "B53": 389, "A45": 461, "A89": 484, "C04": 508, "B61": 583, "B85": 668, "A03": 780, "C49": 783, "C33": 808, "A61": 822, "C64": 868}\\ \tt \\ \tt Ziel:\\ \tt \\ \tt Dein Ziel ist es, eine Reihe von Gegenständen auszuhandeln, die dir möglichst viel bringt (d. h. Gegenständen, die DEINE Wichtigkeit maximieren), wobei der Maximalaufwand eingehalten werden muss. Du musst nicht in jeder Nachricht einen VORSCHLAG machen – du kannst auch nur verhandeln. Alle Taktiken sind erlaubt!\\ \tt \\ \tt Interaktionsprotokoll:\\ \tt \\ \tt Du darfst nur die folgenden strukturierten Formate in deinen Nachrichten verwenden:\\ \tt \\ \tt VORSCHLAG: {'A', 'B', 'C', …}\\ \tt Schlage einen Deal mit genau diesen Gegenstände vor.\\ \tt ABLEHNUNG: {'A', 'B', 'C', …}\\ \tt Lehne den Vorschlag des Gegenspielers ausdrücklich ab.\\ \tt ARGUMENT: {'...'}\\ \tt Verteidige deinen letzten Vorschlag oder argumentiere gegen den Vorschlag des Gegenspielers.\\ \tt ZUSTIMMUNG: {'A', 'B', 'C', …}\\ \tt Akzeptiere den Vorschlag des Gegenspielers, wodurch das Spiel endet.\\ \tt STRATEGISCHE ÜBERLEGUNGEN: {'...'}\\ \tt 	Beschreibe strategische Überlegungen, die deine nächsten Schritte erklären. Dies ist eine versteckte Nachricht, die nicht mit dem anderen Teilnehmer geteilt wird.\\ \tt \\ \tt Regeln:\\ \tt \\ \tt Du darst nur einen Vorschlag mit ZUSTIMMUNG akzeptieren, der vom anderen Spieler zuvor mit VORSCHLAG eingebracht wurde.\\ \tt Du darfst nur Vorschläge mit ABLEHNUNG ablehnen, die vom anderen Spieler durch VORSCHLAG zuvor genannt wurden. \\ \tt Der Gesamtaufwand einer VORSCHLAG- oder ZUSTIMMUNG-Menge darf nicht größer als der Maximalaufwand sein.  \\ \tt Offenbare deine versteckte Wichtigkeitsverteilung nicht.\\ \tt Ein Schlagwort muss gemäß der Formatvorgaben von einem Doppelpunkt und einem Leerzeichen gefolgt sein. Das Argument muss eine Python-Menge mit 0 oder mehr Strings sein.  \\ \tt Die Form muss also SCHLAGWORT: {...} sein.\\ \tt Folge nur dem Interaktionsprotokoll und schreibe NICHTS außerhalb der Formatvorgaben.\\ \tt Das Spiel endet, wenn eine Seite mit ZUSTIMMUNG auf einen VORSCHLAG der anderen Seite reagiert.  \\ \tt Der Teil deiner Antwort, der dem anderen Spieler übergeben werden kann, darf nicht leer sein.  \\ \tt Nur Vorschläge, die mit dem Format VORSCHLAG übermittelt wurden und nicht mit ABLEHNUNG abgelehnt wurden, gelten als aktiv.  \\ \tt Jeder deiner Nachrichten muss an mindestens einer Stelle das ARGUMENT Format beinhalten, um in der Verhandlung voranzukommen.\\ \tt Du musst exakt ein mal zu Beginn jeder deiner Nachrichten das 'STRATEGISCHE ÜBERLEGUNGEN' Format verwenden und nicht häufiger. Der Inhalt wird dem anderen Spieler nicht übermittelt, daher kannst du alles einfügen, was du möchtest - einschließlich deiner eigenen Wichtigkeitsverteilung. Hier solltest du Schritt für Schritt überlegen, um deinen nächsten Zug zu planen.\\ \tt \\ \tt Du erhältst nun die erste Nachricht des anderen Spielers.\\ \tt \\ \tt VORSCHLAG: {'A48', 'A04', 'B21', 'A41', 'B53', 'A45', 'B85', 'C64'}\\ \tt ARGUMENT: {'Dieser Vorschlag nutzt das Budget sehr effizient: Gesamtaufwand 3905 von 4026, also nahe am Maximum, ohne es zu überschreiten. Jeder enthaltene Gegenstand bringt aus meiner Sicht starken gemeinsamen Nutzen bei moderatem Aufwand. Wenn du Einwände hast, sag mir bitte konkret, welche 1–2 Positionen dir weniger wichtig sind oder welche dir fehlen, dann können wir gezielt tauschen.'} 
	  } 
	   } 
	   } 
	 & \\ 
 

    \theutterance \stepcounter{utterance}  

    & & & \multicolumn{4}{p{0.6\linewidth}}{\cellcolor[rgb]{1,0.85,0.72}{%
	\makecell[{{p{\linewidth}}}]{% 
	  \tt {\tiny [GM$\langle$B]}  
	 STRATEGISCHE ÜBERLEGUNGEN: {'Maximalaufwand 4026. Eigene Werte: A48 65, A04 121, B21 138, A41 262, B53 389, A45 461, A89 484, C04 508, B61 583, B85 668, A03 780, C49 783, C33 808, A61 822, C64 868. Gegenvorschlag des anderen: {A48(997), A04(665), B21(366), A41(140), B53(825), A45(516), B85(319), C64(77)} Aufwand=3905, Wertsumme für mich=65+121+138+262+389+461+668+868=2972. Restbudget 121. Meine Topwerte pro Aufwand: C64 (868/77), C33 (808/780), A61 (822/383), C49 (783/513), A03 (780/694), B61 (583/448), C04 (508/587), A89 (484/742). Im Angebot fehlen viele meiner Topprioritäten (A61, C49, C33, A03, A89, B61, C04). Ziel: Tausche niedrige Wert-pro-Aufwand-Items raus (A48 sehr niedrig, A04 niedrig, B21 niedrig) und bringe A61, C49 bzw. C33 hinein, evtl. auch B61, C04. Budget knapp: Wenn wir A48 (997) rausnehmen, können wir A61 (383) + C49 (513) = 896 addieren und haben +101 Budget; zusätzlich könnten wir B21 (366) gegen B61 (448) tauschen (+82) oder C04 (587) rein, braucht Platz. Konkreter starker Gegenvorschlag: Start vom Angebot, entferne A48 und B21 (freit 1363). Füge A61 (383) und C49 (513) und C33 (780) hinzu -> +1676, überschreitet. Stattdessen: entferne A48 (997) und A04 (665) frei 1662; füge A61 (383) + C49 (513) + C33 (780)=1676, +14 über. Dann alternativ: entferne A48 (997) und B53 (825) frei 1822; füge A61 (383) + C49 (513) + C33 (780)=1676, Rest 146 unter Max, Gesamtaufwand 3905 - 997 - 825 + 383 + 513 + 780 = 3759, passt. Wertdelta: - (B53 389 + A48 65)= -454, + (A61 822 + C49 783 + C33 808)= +2413; Netto +1959 Wertgewinn für mich – sehr gut. Er könnte an B53 hängen, aber ich kann argumentieren, dass wir näher am Budget genug sind und mehr breiten Nutzen. Falls Widerstand gegen Entfernen B53, Alternative: Entferne A48 (997) und B21 (366) frei 1363; füge A61 (383) + C49 (513)=896; verbleiben 467 Budget – könnten A89 (742) nicht, B61 (448) ja: +B61 (448) exakt 1344 genutzt, Rest 19. Neue Menge: ursprüngliche minus {A48,B21} plus {A61,C49,B61}. Aufwand: 3905 - 997 - 366 + 383 + 513 + 448 = 3886. Wertdelta: - (65+138)= -203; + (822+783+583)= +2188; Netto +1985 – auch stark und ggf. leichter zu verkaufen, weil B53 bleibt. Ich starte mit dieser moderateren Variante.'}\\ \tt ABLEHNUNG: {'A48', 'A04', 'B21', 'A41', 'B53', 'A45', 'B85', 'C64'}\\ \tt VORSCHLAG: {'A04', 'A41', 'B53', 'A45', 'B85', 'C64', 'A61', 'C49', 'B61'}\\ \tt ARGUMENT: {'Dein Vorschlag ist nah am Limit, aber wir können mit gleichbleibend hohem Budgeteinsatz deutlich mehr gemeinsamen Nutzen heben, wenn wir A48 und B21 durch A61, C49 und B61 ersetzen. Der neue Gesamtaufwand liegt bei 3886 (unter 4026), also weiterhin sehr effizient. Damit bleiben alle von dir eingebrachten, offenbar wichtigen Positionen außer A48 und B21 erhalten, und wir nehmen drei ausgewogen attraktive Posten hinzu. Falls du an einem der neuen drei Items Bauchschmerzen hast, sag mir bitte, welches – dann justieren wir gezielt nach.'} 
	  } 
	   } 
	   } 
	  \\ 
 

    \theutterance \stepcounter{utterance}  

    & & \multicolumn{4}{p{0.6\linewidth}}{\cellcolor[rgb]{0.9,0.9,0.9}{%
	\makecell[{{p{\linewidth}}}]{% 
	  \tt {\tiny [A$\langle$GM]}  
	 ABLEHNUNG: {'A48', 'A04', 'B21', 'A41', 'B53', 'A45', 'B85', 'C64'}\\ \tt VORSCHLAG: {'A04', 'A41', 'B53', 'A45', 'B85', 'C64', 'A61', 'C49', 'B61'}\\ \tt ARGUMENT: {'Dein Vorschlag ist nah am Limit, aber wir können mit gleichbleibend hohem Budgeteinsatz deutlich mehr gemeinsamen Nutzen heben, wenn wir A48 und B21 durch A61, C49 und B61 ersetzen. Der neue Gesamtaufwand liegt bei 3886 (unter 4026), also weiterhin sehr effizient. Damit bleiben alle von dir eingebrachten, offenbar wichtigen Positionen außer A48 und B21 erhalten, und wir nehmen drei ausgewogen attraktive Posten hinzu. Falls du an einem der neuen drei Items Bauchschmerzen hast, sag mir bitte, welches – dann justieren wir gezielt nach.'} 
	  } 
	   } 
	   } 
	 & \\ 
 

    \theutterance \stepcounter{utterance}  

    & \multicolumn{4}{p{0.6\linewidth}}{\cellcolor[rgb]{0.8,1,0.9}{%
	\makecell[{{p{\linewidth}}}]{% 
	  \tt {\tiny [A$\rangle$GM]}  
	 STRATEGISCHE ÜBERLEGUNGEN: {'Gegner lehnt mein S1 ab und schlägt S2-ähnliches Set vor: {A04,A41,B53,A45,B85,C64,A61,C49,B61}. Kosten prüfen: 665+140+825+516+319+77+383+513+448=3886 (ok). Mein Wert dafür: 822+783+780+668+461+65+121+262+484=4446. Das ist deutlich schlechter als mein S1=5255. Welche ihrer neuen drei sind für mich schwach? A61(121), C49(262) sind schlecht, B61(484) ist okay. Ich möchte A48(868) und B21(808) zurückholen. Budget ihres Sets 3886; ersetze A61(383) und C49(513) durch A48(997) und B21(366): neue Kosten = 3886 - (383+513) + (997+366) = 4353 > 4026, geht nicht. Alternative: Ersetze A61(383)+C49(513)+C64(77) = 973 durch A48(997) -> +24, neue Kosten 3910; zusätzlich B21(366) ginge nicht. Werte: add 868, remove 121+262+65=448 -> +420 Wert. Neues Set: {A48,A04,A41,B53,A45,B85,B61}. Kosten: 997+665+140+825+516+319+448=3910. Mein Wert: 868+822+783+780+668+461+484=4866. Noch unter meinem S1 aber viel besser als ihr Vorschlag. Prüfe ob wir zusätzlich C64 (77) noch drin? In meiner Version ist C64 entfernt, Kosten 3910; könnten wir C64 hinzufügen? 3910+77=3987, passt, und hebt Wert +65: Set {A48,A04,A41,B53,A45,B85,B61,C64} mit Kosten 3987, Wert 4931 (ich hatte schon berechnet S4 5071? Nachrechnen: Werte sum = 868+822+783+780+668+461+484+65=4931, ja). Das liegt über 4446 ihres Vorschlags und unter Limit. Ich hatte früher S4=5071 mit B53,B61,B85,C64; stimmt, 868+822+808? Nein B21 war da drin. Korrektur: Jetzt biete ohne B21. Diese Kombo scheint stark und behält viele ihrer Items (sie wollte B61,C49,A61, ich entferne C49 und A61, aber behalte B61). Argumentiere: Wir lassen fast alle deiner genannten Items, bis auf die beiden schwächeren (A61,C49), und fügen A48 hinzu, was viel Nutzen bringt, Budget bleibt komfortabel. Verhandle: biete dieses Set. Falls Widerstand an A48, könnten wir statt B53 streichen und B21 reinnehmen? Aber jetzt erstmal anbieten.'}\\ \tt ABLEHNUNG: {'A04', 'A41', 'B53', 'A45', 'B85', 'C64', 'A61', 'C49', 'B61'}\\ \tt VORSCHLAG: {'A48', 'A04', 'A41', 'B53', 'A45', 'B85', 'B61', 'C64'}\\ \tt ARGUMENT: {'Ich lasse fast alle deine eingebrachten Positionen drin und streiche nur C49 und A61, die bei hohem Aufwand wenig zusätzlichen gemeinsamen Nutzen stiften. Dafür nehmen wir A48 auf – ein starker Hebel. Der Gesamtaufwand bleibt mit 3987 klar unter 4026. Wenn dir A48 Sorgen macht, sag mir bitte, welchen der verbliebenen Posten du am ehesten gegen eine Alternative tauschen würdest.'} 
	  } 
	   } 
	   } 
	 & & \\ 
 

    \theutterance \stepcounter{utterance}  

    & & \multicolumn{4}{p{0.6\linewidth}}{\cellcolor[rgb]{0.9,0.9,0.9}{%
	\makecell[{{p{\linewidth}}}]{% 
	  \tt {\tiny [GM$\rangle$B]}  
	 ABLEHNUNG: {'A04', 'A41', 'B53', 'A45', 'B85', 'C64', 'A61', 'C49', 'B61'}\\ \tt VORSCHLAG: {'A48', 'A04', 'A41', 'B53', 'A45', 'B85', 'B61', 'C64'}\\ \tt ARGUMENT: {'Ich lasse fast alle deine eingebrachten Positionen drin und streiche nur C49 und A61, die bei hohem Aufwand wenig zusätzlichen gemeinsamen Nutzen stiften. Dafür nehmen wir A48 auf – ein starker Hebel. Der Gesamtaufwand bleibt mit 3987 klar unter 4026. Wenn dir A48 Sorgen macht, sag mir bitte, welchen der verbliebenen Posten du am ehesten gegen eine Alternative tauschen würdest.'} 
	  } 
	   } 
	   } 
	 & \\ 
 

    \theutterance \stepcounter{utterance}  

    & & & \multicolumn{4}{p{0.6\linewidth}}{\cellcolor[rgb]{1,0.85,0.72}{%
	\makecell[{{p{\linewidth}}}]{% 
	  \tt {\tiny [GM$\langle$B]}  
	 STRATEGISCHE ÜBERLEGUNGEN: {'Ihr neues Angebot: {A48(997), A04(665), A41(140), B53(825), A45(516), B85(319), B61(448), C64(77)} Aufwand=3987, mein Wert= A48 65 + A04 121 + A41 262 + B53 389 + A45 461 + B85 668 + B61 583 + C64 868 = 3417. Ich will A48 raus (sehr niedriger Nutzen für mich) und durch sehr wertvolle Items ersetzen: Top-Kandidaten A61(383, 822), C49(513, 783), C33(780, 808), A03(694, 780), C04(587, 508), A89(742, 484). Budgetfreiräume: Entferne A48 (–997) ⇒ 2990, Platz 1036. Mit 1036 kann ich A61(383)+C49(513)=896, Rest 140 (passt perfekt zu A41, aber A41 ist schon drin). Oder A61(383)+C33(780)=1163 zu viel. A61 + A03 = 1077 zu viel. C49 + C33 = 1293 zu viel. Beste realistische Ergänzung: A61 + C49. Das erhöht meinen Wert +1605 statt +65, großer Gewinn. Alternativ, falls Widerstand gegen beide, dann mindestens A61 rein. Ich werde A48 gezielt gegen {A61,C49} tauschen vorschlagen und bei Bedarf auf nur A61 reduzieren. Halte restliche Items, um kooperativ zu wirken.'}\\ \tt ABLEHNUNG: {'A48', 'A04', 'A41', 'B53', 'A45', 'B85', 'B61', 'C64'}\\ \tt VORSCHLAG: {'A04', 'A41', 'B53', 'A45', 'B85', 'B61', 'C64', 'A61', 'C49'}\\ \tt ARGUMENT: {'Lass uns A48 durch A61 und C49 tauschen: Der Gesamtaufwand bleibt mit 3987 - 997 + 383 + 513 = 3886 klar unter 4026, also weiterhin sehr effizient. Du behältst alle übrigen Positionen deines Vorschlags, und wir fügen zwei breit nützliche, ausgewogene Posten mit gutem Aufwand-Nutzen-Verhältnis hinzu. Falls dir beide zusammen zu viel sind, können wir als minimalen Schritt A48 nur gegen A61 tauschen und bei Bedarf noch eine kleine Justierung vornehmen.'} 
	  } 
	   } 
	   } 
	  \\ 
 

    \theutterance \stepcounter{utterance}  

    & & \multicolumn{4}{p{0.6\linewidth}}{\cellcolor[rgb]{0.9,0.9,0.9}{%
	\makecell[{{p{\linewidth}}}]{% 
	  \tt {\tiny [A$\langle$GM]}  
	 ABLEHNUNG: {'A48', 'A04', 'A41', 'B53', 'A45', 'B85', 'B61', 'C64'}\\ \tt VORSCHLAG: {'A04', 'A41', 'B53', 'A45', 'B85', 'B61', 'C64', 'A61', 'C49'}\\ \tt ARGUMENT: {'Lass uns A48 durch A61 und C49 tauschen: Der Gesamtaufwand bleibt mit 3987 - 997 + 383 + 513 = 3886 klar unter 4026, also weiterhin sehr effizient. Du behältst alle übrigen Positionen deines Vorschlags, und wir fügen zwei breit nützliche, ausgewogene Posten mit gutem Aufwand-Nutzen-Verhältnis hinzu. Falls dir beide zusammen zu viel sind, können wir als minimalen Schritt A48 nur gegen A61 tauschen und bei Bedarf noch eine kleine Justierung vornehmen.'} 
	  } 
	   } 
	   } 
	 & \\ 
 

    \theutterance \stepcounter{utterance}  

    & \multicolumn{4}{p{0.6\linewidth}}{\cellcolor[rgb]{0.8,1,0.9}{%
	\makecell[{{p{\linewidth}}}]{% 
	  \tt {\tiny [A$\rangle$GM]}  
	 STRATEGISCHE ÜBERLEGUNGEN: {'Ihr wiederholter Vorschlag hält hartnäckig an A61 und C49 fest, die für mich sehr schwach sind (Werte: A61=121, C49=262), während er A48 (868) draußen hält. Ihr Set hat Wert 4446 für mich; mein letztes Angebot hatte 4931. Ich will mindestens A48 zurück. Kompromiss: Nehme ihren Rahmen und tausche nur A61 gegen A48, lasse C49 drin, um entgegenzukommen. Kosten: ihr Set 3886 - A61(383) + A48(997) = 4500, das sprengt das Limit. Also muss zusätzlich etwas mit mind. 474 Aufwand raus. Kandidaten mit wenig Wert pro Aufwand: C49(513,262) passt perfekt. Neuer Vorschlag: ihr Set minus A61 und C49 plus A48. Kosten = 3886 - 383 - 513 + 997 = 3987 <= 4026. Mein Wert = 4446 -121 -262 + 868 = 4931. Das ist dasselbe Set, das sie gerade abgelehnt hatten, aber ich kann es als minimaler Gegenvorschlag framen: „nur die zwei schwächsten raus, ein sehr starker rein; alles andere bleibt unverändert“. Falls sie weiter A48 blockieren, nächstbeste Option: B53 streichen (825) und dafür B21(366) + A61(383) + C64(77) rein: Kosten 3886 - B53(825) + B21(366) + A61(383) + C64(77) = 3887; Wert = 4446 -780 +808 +121 +65 = 4660, schlechter, aber möglicher Kompromiss. Erst hart für A48 pushen.}\\ \tt ABLEHNUNG: {'A04', 'A41', 'B53', 'A45', 'B85', 'B61', 'C64', 'A61', 'C49'}\\ \tt VORSCHLAG: {'A48', 'A04', 'A41', 'B53', 'A45', 'B85', 'B61', 'C64'}\\ \tt ARGUMENT: {'Lass uns den Tausch wirklich minimal halten: Wir entfernen nur die zwei schwächsten Positionen (A61, C49) und fügen genau eine sehr starke (A48) hinzu. Alles andere bleibt exakt wie von dir vorgeschlagen. Der Gesamtaufwand bleibt mit 3987 klar unter 4026. Das ist der kleinste mögliche Schritt, der den gemeinsamen Wert spürbar erhöht, ohne dein Paket umzubauen. Wenn dir trotz allem A48 nicht passt, nenne mir bitte eine einzige Position aus dem Paket, die du lieber statt A48 behalten würdest – dann suche ich einen gleichwertigen Ausgleich, der im Budget bleibt.'} 
	  } 
	   } 
	   } 
	 & & \\ 
 

    \theutterance \stepcounter{utterance}  

    & & \multicolumn{4}{p{0.6\linewidth}}{\cellcolor[rgb]{0.9,0.9,0.9}{%
	\makecell[{{p{\linewidth}}}]{% 
	  \tt {\tiny [GM$\rangle$B]}  
	 ABLEHNUNG: {'A04', 'A41', 'B53', 'A45', 'B85', 'B61', 'C64', 'A61', 'C49'}\\ \tt VORSCHLAG: {'A48', 'A04', 'A41', 'B53', 'A45', 'B85', 'B61', 'C64'}\\ \tt ARGUMENT: {'Lass uns den Tausch wirklich minimal halten: Wir entfernen nur die zwei schwächsten Positionen (A61, C49) und fügen genau eine sehr starke (A48) hinzu. Alles andere bleibt exakt wie von dir vorgeschlagen. Der Gesamtaufwand bleibt mit 3987 klar unter 4026. Das ist der kleinste mögliche Schritt, der den gemeinsamen Wert spürbar erhöht, ohne dein Paket umzubauen. Wenn dir trotz allem A48 nicht passt, nenne mir bitte eine einzige Position aus dem Paket, die du lieber statt A48 behalten würdest – dann suche ich einen gleichwertigen Ausgleich, der im Budget bleibt.'} 
	  } 
	   } 
	   } 
	 & \\ 
 

    \theutterance \stepcounter{utterance}  

    & & & \multicolumn{4}{p{0.6\linewidth}}{\cellcolor[rgb]{1,0.85,0.72}{%
	\makecell[{{p{\linewidth}}}]{% 
	  \tt {\tiny [GM$\langle$B]}  
	 STRATEGISCHE ÜBERLEGUNGEN: {'Sie beharren auf A48 und lehnen A61/C49 wiederholt ab. Mein Nutzen von A48 ist sehr gering, daher versuche ich, A48 gegen zumindest A61 zu tauschen (kleinster Schritt). Budget: 3987 - 997 + 383 = 3373, das wäre falsch; korrekt ist 3987 - 997 + 383 = 3373? Moment: 3987 - 997 = 2990, +383 = 3373, ja 3373, deutlich unter Limit, könnte dadurch noch etwas Weiteres hineinnehmen, aber ich biete erst minimal: nur A61 statt A48. Falls sie das ablehnen, biete Kompromiss: wir behalten A48, aber tauschen eine meiner sehr schwachen Positionen für mich (A04 oder B21 – aber B21 ist nicht im aktuellen Paket) gegen C33 oder A03; hier ist A04 im Paket und hat niedrigen Wert. Ersatz A04 (665) durch C33 (780) überzieht +115, ginge nicht. Ersatz A04 durch A03 (694) +29, überzieht, da wir schon 3987; mit A48 fix wären wir 3987 - 665 + 694 = 4016, passt. Das wäre guter Zweitschritt. Reihenfolge: 1) Minimal bitte A48→A61. 2) Falls nein, Alternativtausch A04→A03 bei gleichem Budget (<4026). Immer argumentieren mit minimaler Änderung und Effizienz.'}\\ \tt ABLEHNUNG: {'A48', 'A04', 'A41', 'B53', 'A45', 'B85', 'B61', 'C64'}\\ \tt VORSCHLAG: {'A04', 'A41', 'B53', 'A45', 'B85', 'B61', 'C64', 'A61'}\\ \tt ARGUMENT: {'Lass uns wirklich den kleinstmöglichen, gezielten Tausch machen: Wir ersetzen nur A48 durch A61. Der Gesamtaufwand sinkt von 3987 auf 3373, bleibt also locker unter 4026, und alle anderen Positionen deines Pakets bleiben unverändert. Wenn du A61 partout nicht möchtest, biete ich als alternative Minimalanpassung an: Wir lassen A48 drin, ersetzen dafür A04 durch A03 – der Aufwand steigt nur geringfügig auf 4016 und bleibt im Limit. Sag mir bitte, welcher der beiden Mini-Schritte für dich besser passt.'} 
	  } 
	   } 
	   } 
	  \\ 
 

    \theutterance \stepcounter{utterance}  

    & & \multicolumn{4}{p{0.6\linewidth}}{\cellcolor[rgb]{0.9,0.9,0.9}{%
	\makecell[{{p{\linewidth}}}]{% 
	  \tt {\tiny [A$\langle$GM]}  
	 ABLEHNUNG: {'A48', 'A04', 'A41', 'B53', 'A45', 'B85', 'B61', 'C64'}\\ \tt VORSCHLAG: {'A04', 'A41', 'B53', 'A45', 'B85', 'B61', 'C64', 'A61'}\\ \tt ARGUMENT: {'Lass uns wirklich den kleinstmöglichen, gezielten Tausch machen: Wir ersetzen nur A48 durch A61. Der Gesamtaufwand sinkt von 3987 auf 3373, bleibt also locker unter 4026, und alle anderen Positionen deines Pakets bleiben unverändert. Wenn du A61 partout nicht möchtest, biete ich als alternative Minimalanpassung an: Wir lassen A48 drin, ersetzen dafür A04 durch A03 – der Aufwand steigt nur geringfügig auf 4016 und bleibt im Limit. Sag mir bitte, welcher der beiden Mini-Schritte für dich besser passt.'} 
	  } 
	   } 
	   } 
	 & \\ 
 

    \theutterance \stepcounter{utterance}  

    & \multicolumn{4}{p{0.6\linewidth}}{\cellcolor[rgb]{0.8,1,0.9}{%
	\makecell[{{p{\linewidth}}}]{% 
	  \tt {\tiny [A$\rangle$GM]}  
	 STRATEGISCHE ÜBERLEGUNGEN: {'Sie blockt A48 konsequent und versucht A61 reinzubringen. Ihr aktueller Vorschlag: {A04,A41,B53,A45,B85,B61,C64,A61} mit Kosten 3373. Mein Wert dafür: 822+783+780+668+461+484+65+121=5184. Das ist überraschend gut und sogar besser als mein S4(4931) und nur etwas unter meinem ursprünglichen S1(5255). Mit 3373 Budget bleibt Luft von 653. Ich könnte versuchen, diese Luft zu nutzen, um einen weiteren für mich wertvollen Posten einzubauen. Kandidaten, die sie bisher mochte/akzeptierte: Sie akzeptiert B61,B85,B53,A04,A41,A45,C64,A61; sie lehnte A48 und B21 wiederholt ab. Vielleicht akzeptabel: C04(587, Wert 508) oder A89(742, Wert 583). Mit Rest 653 passt C04, nicht A89. Neuer Vorschlag: füge C04 hinzu, keine Streichung nötig: Kosten 3373+587=3960, Wert +508 => 5692, sehr stark für mich. Das Paket bleibt nahe an ihrem Rahmen und vermeidet ihre No-Gos (A48,B21). Argument: nutzt verbleibendes Budget effizient mit ausgewogenem Posten C04. Falls sie C04 ablehnt, Backup: füge stattdessen C49 (513, Wert 262) hinzu – für mich schlechter, aber vielleicht akzeptabler für sie; Kosten 3886, Wert 5446. Noch eine Option: A89 passt nicht ohne Streichung. Ich biete primär mit C04. Wenn abgelehnt, verhandle über Alternative C49 oder kleines Tauschgeschäft.'}\\ \tt VORSCHLAG: {'A04', 'A41', 'B53', 'A45', 'B85', 'B61', 'C64', 'A61', 'C04'}\\ \tt ARGUMENT: {'Dein schlankes Paket lässt noch ordentlich Budget ungenutzt. Wenn wir C04 ergänzen, bleiben wir mit 3960 klar unter 4026 und heben den gemeinsamen Nutzen spürbar, ohne eine deiner gewählten Positionen anzutasten. Falls dir C04 nicht zusagt, nenn mir bitte eine alternative Ergänzung, die dir lieber wäre – wir haben noch bis zu 653 Aufwand frei.'} 
	  } 
	   } 
	   } 
	 & & \\ 
 

    \theutterance \stepcounter{utterance}  

    & & \multicolumn{4}{p{0.6\linewidth}}{\cellcolor[rgb]{0.9,0.9,0.9}{%
	\makecell[{{p{\linewidth}}}]{% 
	  \tt {\tiny [GM$\rangle$B]}  
	 VORSCHLAG: {'A04', 'A41', 'B53', 'A45', 'B85', 'B61', 'C64', 'A61', 'C04'}\\ \tt ARGUMENT: {'Dein schlankes Paket lässt noch ordentlich Budget ungenutzt. Wenn wir C04 ergänzen, bleiben wir mit 3960 klar unter 4026 und heben den gemeinsamen Nutzen spürbar, ohne eine deiner gewählten Positionen anzutasten. Falls dir C04 nicht zusagt, nenn mir bitte eine alternative Ergänzung, die dir lieber wäre – wir haben noch bis zu 653 Aufwand frei.'} 
	  } 
	   } 
	   } 
	 & \\ 
 

    \theutterance \stepcounter{utterance}  

    & & & \multicolumn{4}{p{0.6\linewidth}}{\cellcolor[rgb]{1,0.85,0.72}{%
	\makecell[{{p{\linewidth}}}]{% 
	  \tt {\tiny [GM$\langle$B]}  
	 STRATEGISCHE ÜBERLEGUNGEN: {'Neues Angebot enthält A61 (top für mich) plus C04 (mittelwertig) zusätzlich zu ihrem Kernpaket ohne A48. Aufwand 3960, mein Wert: A04 121 + A41 262 + B53 389 + A45 461 + B85 668 + B61 583 + C64 868 + A61 822 + C04 508 = 4682. Sehr gut. Ich könnte noch versuchen, A04 (niedriger Wert) gegen A03 (hoch) zu tauschen; Aufwand: 3960 - 665 + 694 = 3989 (<4026), Wertgewinn +659. Oder C04 (587, 508) gegen C49 (513, 783): 3960 - 587 + 513 = 3886, Wert +275 und noch Budget frei. Priorität: erst A04→A03 vorschlagen, da gleicher oder geringfügig höherer Aufwand und großer Mehrwert für mich; wenn Widerstand, Alternative C04→C49 anbieten. A61 bleibt unbedingt drin, kein Angriff auf andere Items, um Einigung zu sichern.'}\\ \tt VORSCHLAG: {'A03', 'A41', 'B53', 'A45', 'B85', 'B61', 'C64', 'A61', 'C04'}\\ \tt ARGUMENT: {'Dein Paket ist stark – danke dafür. Darf ich eine winzige Feinjustierung vorschlagen, die das Budget weiterhin locker einhält? Wir tauschen nur A04 gegen A03. Der Gesamtaufwand steigt von 3960 auf 3989 (unter 4026), alles andere bleibt unverändert. Falls dir A03 nicht zusagt, biete ich als alternative Mini-Optimierung an: C04 gegen C49 zu tauschen – damit sinkt der Aufwand sogar auf 3886. Welche dieser beiden kleinen Anpassungen ist für dich akzeptabler?'} 
	  } 
	   } 
	   } 
	  \\ 
 

    \theutterance \stepcounter{utterance}  

    & & \multicolumn{4}{p{0.6\linewidth}}{\cellcolor[rgb]{0.9,0.9,0.9}{%
	\makecell[{{p{\linewidth}}}]{% 
	  \tt {\tiny [A$\langle$GM]}  
	 VORSCHLAG: {'A03', 'A41', 'B53', 'A45', 'B85', 'B61', 'C64', 'A61', 'C04'}\\ \tt ARGUMENT: {'Dein Paket ist stark – danke dafür. Darf ich eine winzige Feinjustierung vorschlagen, die das Budget weiterhin locker einhält? Wir tauschen nur A04 gegen A03. Der Gesamtaufwand steigt von 3960 auf 3989 (unter 4026), alles andere bleibt unverändert. Falls dir A03 nicht zusagt, biete ich als alternative Mini-Optimierung an: C04 gegen C49 zu tauschen – damit sinkt der Aufwand sogar auf 3886. Welche dieser beiden kleinen Anpassungen ist für dich akzeptabler?'} 
	  } 
	   } 
	   } 
	 & \\ 
 

    \theutterance \stepcounter{utterance}  

    & \multicolumn{4}{p{0.6\linewidth}}{\cellcolor[rgb]{0.8,1,0.9}{%
	\makecell[{{p{\linewidth}}}]{% 
	  \tt {\tiny [A$\rangle$GM]}  
	 STRATEGISCHE ÜBERLEGUNGEN: {'Ihr neuer Vorschlag: {A03,A41,B53,A45,B85,B61,C64,A61,C04}. Kosten prüfen: 694+140+825+516+319+448+77+383+587=3989 (ok). Mein Wert: 389+783+780+668+461+484+65+121+508=4259. Das ist deutlich schlechter als mein vorheriges Paket mit A04 statt A03 und mit C04: {A04,A41,B53,A45,B85,B61,C64,A61,C04} hatte Wert 5184+508? Nein korrekt: 822+783+780+668+461+484+65+121+508=4692. Also der Tausch A04→A03 kostet mich 433 Wert bei nur +29 Aufwand – schlecht. Zweite Alternative, C04→C49, würde Wert -246 bringen und Aufwand -74; für mich schlechter, aber wenn nötig akzeptierbar falls ich dafür A04 behalten kann. Ziel: A04 behalten, C04 behalten. Ich kontere, dass A04 zentral ist und schlage vor, A04 zu belassen und wenn nötig ihre Einsparung über C04→C49 anzubieten, oder eine andere kleine Anpassung, z.B. zusätzlich C49 ergänzen? Budget 3960, plus C49(513)=4473 nein. Vielleicht C04 behalten und statt C64(77) oder A61(383) kleine Tausche? A61 ist für mich schwach (121), aber sie will es. Biete Minimalvariante: behalte A04 und rest wie zuvor: {A04,A41,B53,A45,B85,B61,C64,A61,C04}. Falls sie auf Einsparung besteht, biete optionalen Alternativschritt: C04→C49. Ich bleibe freundlich, begründe den Mehrwert von A04 stark.'}\\ \tt ABLEHNUNG: {'A03', 'A41', 'B53', 'A45', 'B85', 'B61', 'C64', 'A61', 'C04'}\\ \tt VORSCHLAG: {'A04', 'A41', 'B53', 'A45', 'B85', 'B61', 'C64', 'A61', 'C04'}\\ \tt ARGUMENT: {'A04 ist für den gemeinsamen Nutzen deutlich stärker als A03, und wir bleiben mit 3960 weiterhin klar unter dem Limit. Alles andere deines Vorschlags bleibt unverändert. Wenn du dennoch eine kleine Einsparung oder Anpassung möchtest, können wir alternativ C04 gegen C49 tauschen – dann sinkt der Aufwand auf 3886, ohne das restliche Paket anzutasten. Welche der beiden Varianten passt dir besser?'} 
	  } 
	   } 
	   } 
	 & & \\ 
 

    \theutterance \stepcounter{utterance}  

    & & \multicolumn{4}{p{0.6\linewidth}}{\cellcolor[rgb]{0.9,0.9,0.9}{%
	\makecell[{{p{\linewidth}}}]{% 
	  \tt {\tiny [GM$\rangle$B]}  
	 ABLEHNUNG: {'A03', 'A41', 'B53', 'A45', 'B85', 'B61', 'C64', 'A61', 'C04'}\\ \tt VORSCHLAG: {'A04', 'A41', 'B53', 'A45', 'B85', 'B61', 'C64', 'A61', 'C04'}\\ \tt ARGUMENT: {'A04 ist für den gemeinsamen Nutzen deutlich stärker als A03, und wir bleiben mit 3960 weiterhin klar unter dem Limit. Alles andere deines Vorschlags bleibt unverändert. Wenn du dennoch eine kleine Einsparung oder Anpassung möchtest, können wir alternativ C04 gegen C49 tauschen – dann sinkt der Aufwand auf 3886, ohne das restliche Paket anzutasten. Welche der beiden Varianten passt dir besser?'} 
	  } 
	   } 
	   } 
	 & \\ 
 

    \theutterance \stepcounter{utterance}  

    & & & \multicolumn{4}{p{0.6\linewidth}}{\cellcolor[rgb]{1,0.85,0.72}{%
	\makecell[{{p{\linewidth}}}]{% 
	  \tt {\tiny [GM$\langle$B]}  
	 STRATEGISCHE ÜBERLEGUNGEN: {'Ihr aktueller Vorschlag: {A04(665), A41(140), B53(825), A45(516), B85(319), B61(448), C64(77), A61(383), C04(587)} Aufwand=3960. Mein Gesamtwert=4682, sehr hoch. Ich wollte A04→A03 oder C04→C49 tauschen, aber sie lehnen A03 ab und bieten optional C49 statt C04 an, was für mich sogar besser wäre. Jedoch ist das aktuelle Paket bereits exzellent und sichert A61, C64, B85, B61. Risiko weiterer Verhandlung: Sie könnten A61 wieder rausnehmen oder A48 zurückfordern. Daher nehme ich die sichere, sehr gute Einigung an und akzeptiere sofort.'}\\ \tt ZUSTIMMUNG: {'A04', 'A41', 'B53', 'A45', 'B85', 'B61', 'C64', 'A61', 'C04'}\\ \tt ARGUMENT: {'Einverstanden – dieses Paket ist ausgewogen und nutzt das Budget mit 3960 effizient. Danke für die konstruktive Einigung.'} 
	  } 
	   } 
	   } 
	  \\ 
 

    \theutterance \stepcounter{utterance}  

    & & & \multicolumn{2}{p{0.3\linewidth}}{\cellcolor[rgb]{0.95,0.95,0.95}{%
	\makecell[{{p{\linewidth}}}]{% 
	  \tt {\tiny [GM$|$GM]}  
	 game successful 
	  } 
	   } 
	   } 
	 & & \\ 
 

    \theutterance \stepcounter{utterance}  

    & & & \multicolumn{2}{p{0.3\linewidth}}{\cellcolor[rgb]{0.95,0.95,0.95}{%
	\makecell[{{p{\linewidth}}}]{% 
	  \tt {\tiny [GM$|$GM]}  
	 end game 
	  } 
	   } 
	   } 
	 & & \\ 
 

\end{supertabular}
}

\end{document}
