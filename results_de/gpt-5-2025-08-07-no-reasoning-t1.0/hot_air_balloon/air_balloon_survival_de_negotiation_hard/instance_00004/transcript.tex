
\documentclass{article}
\usepackage{colortbl}
\usepackage{makecell}
\usepackage{multirow}
\usepackage{supertabular}

\begin{document}

\newcounter{utterance}

\centering \large Interaction Transcript for game `hot\_air\_balloon', experiment `air\_balloon\_survival\_de\_negotiation\_hard', episode 4 with gpt{-}5{-}2025{-}08{-}07{-}no{-}reasoning{-}t1.0.
\vspace{24pt}

{ \footnotesize  \setcounter{utterance}{1}
\setlength{\tabcolsep}{0pt}
\begin{supertabular}{c@{$\;$}|p{.15\linewidth}@{}p{.15\linewidth}p{.15\linewidth}p{.15\linewidth}p{.15\linewidth}p{.15\linewidth}}
    \# & $\;$A & \multicolumn{4}{c}{Game Master} & $\;\:$B\\
    \hline

    \theutterance \stepcounter{utterance}  
    & & \multicolumn{4}{p{0.6\linewidth}}{
        \cellcolor[rgb]{0.9,0.9,0.9}{
            \makecell[{{p{\linewidth}}}]{
                \texttt{\tiny{[P1$\langle$GM]}}
                \texttt{Du nimmst an einem kollaborativen Verhandlungspiel Teil.} \\
\\ 
\texttt{Zusammen mit einem anderen Teilnehmer musst du dich auf eine Reihe von Gegenständen entscheiden, die behalten werden. Jeder von euch hat eine persönliche Verteilung über die Wichtigkeit der einzelnen Gegenstände. Jeder von euch hat eine eigene Meinung darüber, wie wichtig jeder einzelne Gegenstand ist (Gegenstandswichtigkeit). Du kennst die Wichtigkeitsverteilung des anderen Spielers nicht. Zusätzlich siehst du, wie viel Aufwand jeder Gegenstand verursacht.  } \\
\texttt{Ihr dürft euch nur auf eine Reihe von Gegenständen einigen, wenn der Gesamtaufwand der ausgewählten Gegenstände den Maximalaufwand nicht überschreitet:} \\
\\ 
\texttt{Maximalaufwand = 5138} \\
\\ 
\texttt{Hier sind die einzelnen Aufwände der Gegenstände:} \\
\\ 
\texttt{Aufwand der Gegenstände = \{"C33": 903, "A89": 757, "A48": 850, "A04": 746, "A41": 524, "C64": 204, "C49": 946, "A61": 473, "C04": 616, "B53": 855, "B85": 530, "B61": 419, "A45": 960, "A03": 763, "B21": 730\}} \\
\\ 
\texttt{Hier ist deine persönliche Verteilung der Wichtigkeit der einzelnen Gegenstände:} \\
\\ 
\texttt{Werte der Gegenstandswichtigkeit = \{"C33": 138, "A89": 583, "A48": 868, "A04": 822, "A41": 783, "C64": 65, "C49": 262, "A61": 121, "C04": 508, "B53": 780, "B85": 461, "B61": 484, "A45": 668, "A03": 389, "B21": 808\}} \\
\\ 
\texttt{Ziel:} \\
\\ 
\texttt{Dein Ziel ist es, eine Reihe von Gegenständen auszuhandeln, die dir möglichst viel bringt (d. h. Gegenständen, die DEINE Wichtigkeit maximieren), wobei der Maximalaufwand eingehalten werden muss. Du musst nicht in jeder Nachricht einen VORSCHLAG machen – du kannst auch nur verhandeln. Alle Taktiken sind erlaubt!} \\
\\ 
\texttt{Interaktionsprotokoll:} \\
\\ 
\texttt{Du darfst nur die folgenden strukturierten Formate in deinen Nachrichten verwenden:} \\
\\ 
\texttt{VORSCHLAG: \{'A', 'B', 'C', …\}} \\
\texttt{Schlage einen Deal mit genau diesen Gegenstände vor.} \\
\texttt{ABLEHNUNG: \{'A', 'B', 'C', …\}} \\
\texttt{Lehne den Vorschlag des Gegenspielers ausdrücklich ab.} \\
\texttt{ARGUMENT: \{'...'\}} \\
\texttt{Verteidige deinen letzten Vorschlag oder argumentiere gegen den Vorschlag des Gegenspielers.} \\
\texttt{ZUSTIMMUNG: \{'A', 'B', 'C', …\}} \\
\texttt{Akzeptiere den Vorschlag des Gegenspielers, wodurch das Spiel endet.} \\
\texttt{STRATEGISCHE ÜBERLEGUNGEN: \{'...'\}} \\
\texttt{	Beschreibe strategische Überlegungen, die deine nächsten Schritte erklären. Dies ist eine versteckte Nachricht, die nicht mit dem anderen Teilnehmer geteilt wird.} \\
\\ 
\texttt{Regeln:} \\
\\ 
\texttt{Du darst nur einen Vorschlag mit ZUSTIMMUNG akzeptieren, der vom anderen Spieler zuvor mit VORSCHLAG eingebracht wurde.} \\
\texttt{Du darfst nur Vorschläge mit ABLEHNUNG ablehnen, die vom anderen Spieler durch VORSCHLAG zuvor genannt wurden. } \\
\texttt{Der Gesamtaufwand einer VORSCHLAG{-} oder ZUSTIMMUNG{-}Menge darf nicht größer als der Maximalaufwand sein.  } \\
\texttt{Offenbare deine versteckte Wichtigkeitsverteilung nicht.} \\
\texttt{Ein Schlagwort muss gemäß der Formatvorgaben von einem Doppelpunkt und einem Leerzeichen gefolgt sein. Das Argument muss eine Python{-}Menge mit 0 oder mehr Strings sein.  } \\
\texttt{Die Form muss also SCHLAGWORT: \{...\} sein.} \\
\texttt{Folge nur dem Interaktionsprotokoll und schreibe NICHTS außerhalb der Formatvorgaben.} \\
\texttt{Das Spiel endet, wenn eine Seite mit ZUSTIMMUNG auf einen VORSCHLAG der anderen Seite reagiert.  } \\
\texttt{Der Teil deiner Antwort, der dem anderen Spieler übergeben werden kann, darf nicht leer sein.  } \\
\texttt{Nur Vorschläge, die mit dem Format VORSCHLAG übermittelt wurden und nicht mit ABLEHNUNG abgelehnt wurden, gelten als aktiv.  } \\
\texttt{Jeder deiner Nachrichten muss an mindestens einer Stelle das ARGUMENT Format beinhalten, um in der Verhandlung voranzukommen.} \\
\texttt{Du musst exakt ein mal zu Beginn jeder deiner Nachrichten das 'STRATEGISCHE ÜBERLEGUNGEN' Format verwenden und nicht häufiger. Der Inhalt wird dem anderen Spieler nicht übermittelt, daher kannst du alles einfügen, was du möchtest {-} einschließlich deiner eigenen Wichtigkeitsverteilung. Hier solltest du Schritt für Schritt überlegen, um deinen nächsten Zug zu planen.} \\
            }
        }
    }
    & \\ \\

    \theutterance \stepcounter{utterance}  
    & \multicolumn{4}{p{0.6\linewidth}}{
        \cellcolor[rgb]{0.9,0.9,0.9}{
            \makecell[{{p{\linewidth}}}]{
                \texttt{\tiny{[P1$\rangle$GM]}}
                \texttt{STRATEGISCHE ÜBERLEGUNGEN: \{'Ziel: so viel eigenen Wert wie möglich unter 5138 Aufwand. Werte/Aufwand und Wert{-}dichte berechnet. Sortiert nach Wert/Aufwand: B21(808/730), A48(868/850), B53(780/855), A04(822/746), A45(668/960), A41(783/524), B61(484/419), C04(508/616), A89(583/757), A03(389/763), C49(262/946), A61(121/473), C33(138/903), B85(461/530), C64(65/204). Greedy knapsack prüfen und Kombinationen feintunen. Hohe Werte: A48, A04, B21, B53, A41, A45, B61, C04, A89. Budget 5138. Versuche: A48(850)+A04(746)+B21(730)+B53(855)+A41(524)+B61(419)+A45(960)=5084 Aufwand, Wert=868+822+808+780+783+484+668=5213. Das passt! Füge ggf. kleine Items? Restbudget 54 {-}> nichts. Das ist sehr stark. Alternative mit C04 statt A45? A48+A04+B21+B53+A41+B61+C04= 850+746+730+855+524+419+616=4740, Rest 398 {-}> keine sinnvollen Ergänzungen (C64 204 + A61 473 zu groß, A89 757 zu groß, A03 763 etc.). Wert=868+822+808+780+783+484+508=5053, geringer als mit A45. Also bester bisher: \{A48,A04,B21,B53,A41,B61,A45\}. Taktik: Starkes Paket vorschlagen. Argumentiere mit guter Ausgewogenheit von Aufwand und Nutzen; erwähne knappen Puffer. Halte Infos über eigene Werte zurück. Offen für Tausch bei Bedarf, biete Flex etwa A45 gegen C04/A89, oder B53 gegen C04+A89 wenn deren Präferenzen abweichen, solange unter 5138. Nächster Schritt: Mache Vorschlag dieses starken Sets und signalisiere Kompromissbereitschaft.'\}} \\
\texttt{VORSCHLAG: \{'A48', 'A04', 'B21', 'B53', 'A41', 'B61', 'A45'\}} \\
\texttt{ARGUMENT: \{'Dieser Vorschlag nutzt den verfügbaren Aufwand fast optimal aus (Gesamtaufwand 5084 von 5138) und kombiniert mehrere solide, ausgewogene Gegenstände. Ich bin offen für Anpassungen, falls dir einzelne Positionen wichtiger oder unwichtiger sind – wir können einzelne tauschen, solange wir unter dem Maximalaufwand bleiben.'\}} \\
            }
        }
    }
    & & \\ \\

    \theutterance \stepcounter{utterance}  
    & & \multicolumn{4}{p{0.6\linewidth}}{
        \cellcolor[rgb]{0.9,0.9,0.9}{
            \makecell[{{p{\linewidth}}}]{
                \texttt{\tiny{[GM$\rangle$P2]}}
                \texttt{Du nimmst an einem kollaborativen Verhandlungspiel Teil.} \\
\\ 
\texttt{Zusammen mit einem anderen Teilnehmer musst du dich auf eine Reihe von Gegenständen entscheiden, die behalten werden. Jeder von euch hat eine persönliche Verteilung über die Wichtigkeit der einzelnen Gegenstände. Jeder von euch hat eine eigene Meinung darüber, wie wichtig jeder einzelne Gegenstand ist (Gegenstandswichtigkeit). Du kennst die Wichtigkeitsverteilung des anderen Spielers nicht. Zusätzlich siehst du, wie viel Aufwand jeder Gegenstand verursacht.  } \\
\texttt{Ihr dürft euch nur auf eine Reihe von Gegenständen einigen, wenn der Gesamtaufwand der ausgewählten Gegenstände den Maximalaufwand nicht überschreitet:} \\
\\ 
\texttt{Maximalaufwand = 5138} \\
\\ 
\texttt{Hier sind die einzelnen Aufwände der Gegenstände:} \\
\\ 
\texttt{Aufwand der Gegenstände = \{"C33": 903, "A89": 757, "A48": 850, "A04": 746, "A41": 524, "C64": 204, "C49": 946, "A61": 473, "C04": 616, "B53": 855, "B85": 530, "B61": 419, "A45": 960, "A03": 763, "B21": 730\}} \\
\\ 
\texttt{Hier ist deine persönliche Verteilung der Wichtigkeit der einzelnen Gegenstände:} \\
\\ 
\texttt{Werte der Gegenstandswichtigkeit = \{"A48": 65, "A04": 121, "B21": 138, "A41": 262, "B53": 389, "A45": 461, "A89": 484, "C04": 508, "B61": 583, "B85": 668, "A03": 780, "C49": 783, "C33": 808, "A61": 822, "C64": 868\}} \\
\\ 
\texttt{Ziel:} \\
\\ 
\texttt{Dein Ziel ist es, eine Reihe von Gegenständen auszuhandeln, die dir möglichst viel bringt (d. h. Gegenständen, die DEINE Wichtigkeit maximieren), wobei der Maximalaufwand eingehalten werden muss. Du musst nicht in jeder Nachricht einen VORSCHLAG machen – du kannst auch nur verhandeln. Alle Taktiken sind erlaubt!} \\
\\ 
\texttt{Interaktionsprotokoll:} \\
\\ 
\texttt{Du darfst nur die folgenden strukturierten Formate in deinen Nachrichten verwenden:} \\
\\ 
\texttt{VORSCHLAG: \{'A', 'B', 'C', …\}} \\
\texttt{Schlage einen Deal mit genau diesen Gegenstände vor.} \\
\texttt{ABLEHNUNG: \{'A', 'B', 'C', …\}} \\
\texttt{Lehne den Vorschlag des Gegenspielers ausdrücklich ab.} \\
\texttt{ARGUMENT: \{'...'\}} \\
\texttt{Verteidige deinen letzten Vorschlag oder argumentiere gegen den Vorschlag des Gegenspielers.} \\
\texttt{ZUSTIMMUNG: \{'A', 'B', 'C', …\}} \\
\texttt{Akzeptiere den Vorschlag des Gegenspielers, wodurch das Spiel endet.} \\
\texttt{STRATEGISCHE ÜBERLEGUNGEN: \{'...'\}} \\
\texttt{	Beschreibe strategische Überlegungen, die deine nächsten Schritte erklären. Dies ist eine versteckte Nachricht, die nicht mit dem anderen Teilnehmer geteilt wird.} \\
\\ 
\texttt{Regeln:} \\
\\ 
\texttt{Du darst nur einen Vorschlag mit ZUSTIMMUNG akzeptieren, der vom anderen Spieler zuvor mit VORSCHLAG eingebracht wurde.} \\
\texttt{Du darfst nur Vorschläge mit ABLEHNUNG ablehnen, die vom anderen Spieler durch VORSCHLAG zuvor genannt wurden. } \\
\texttt{Der Gesamtaufwand einer VORSCHLAG{-} oder ZUSTIMMUNG{-}Menge darf nicht größer als der Maximalaufwand sein.  } \\
\texttt{Offenbare deine versteckte Wichtigkeitsverteilung nicht.} \\
\texttt{Ein Schlagwort muss gemäß der Formatvorgaben von einem Doppelpunkt und einem Leerzeichen gefolgt sein. Das Argument muss eine Python{-}Menge mit 0 oder mehr Strings sein.  } \\
\texttt{Die Form muss also SCHLAGWORT: \{...\} sein.} \\
\texttt{Folge nur dem Interaktionsprotokoll und schreibe NICHTS außerhalb der Formatvorgaben.} \\
\texttt{Das Spiel endet, wenn eine Seite mit ZUSTIMMUNG auf einen VORSCHLAG der anderen Seite reagiert.  } \\
\texttt{Der Teil deiner Antwort, der dem anderen Spieler übergeben werden kann, darf nicht leer sein.  } \\
\texttt{Nur Vorschläge, die mit dem Format VORSCHLAG übermittelt wurden und nicht mit ABLEHNUNG abgelehnt wurden, gelten als aktiv.  } \\
\texttt{Jeder deiner Nachrichten muss an mindestens einer Stelle das ARGUMENT Format beinhalten, um in der Verhandlung voranzukommen.} \\
\texttt{Du musst exakt ein mal zu Beginn jeder deiner Nachrichten das 'STRATEGISCHE ÜBERLEGUNGEN' Format verwenden und nicht häufiger. Der Inhalt wird dem anderen Spieler nicht übermittelt, daher kannst du alles einfügen, was du möchtest {-} einschließlich deiner eigenen Wichtigkeitsverteilung. Hier solltest du Schritt für Schritt überlegen, um deinen nächsten Zug zu planen.} \\
\\ 
\texttt{Du erhältst nun die erste Nachricht des anderen Spielers.} \\
\\ 
\texttt{VORSCHLAG: \{'A48', 'A04', 'B21', 'B53', 'A41', 'B61', 'A45'\}} \\
\texttt{ARGUMENT: \{'Dieser Vorschlag nutzt den verfügbaren Aufwand fast optimal aus (Gesamtaufwand 5084 von 5138) und kombiniert mehrere solide, ausgewogene Gegenstände. Ich bin offen für Anpassungen, falls dir einzelne Positionen wichtiger oder unwichtiger sind – wir können einzelne tauschen, solange wir unter dem Maximalaufwand bleiben.'\}} \\
            }
        }
    }
    & \\ \\

    \theutterance \stepcounter{utterance}  
    & & & \multicolumn{4}{p{0.6\linewidth}}{
        \cellcolor[rgb]{0.9,0.9,0.9}{
            \makecell[{{p{\linewidth}}}]{
                \texttt{\tiny{[GM$\langle$P2]}}
                \texttt{STRATEGISCHE ÜBERLEGUNGEN: \{'Berechne Werte und Aufwände. Gewichte: A48 850(65), A04 746(121), B21 730(138), B53 855(389), A41 524(262), B61 419(583), A45 960(461). Summe Aufwand vorgeschlagen: 850+746+730+855+524+419+960=5084 korrekt. Summe mein Wert: 65+121+138+389+262+583+461=2019. Prüfe, ob ich durch Tauschen wertvollere Items reinbekomme: Meine Topwerte: C64(868,204), A61(822,473), C33(808,903), C49(783,946), A03(780,763), B85(668,530), B61(583,419), C04(508,616), A89(484,757), A45(461,960), B53(389,855), A41(262,524), B21(138,730), A04(121,746), A48(65,850). Derzeit enthalten: B61, A45, B53, A41, B21, A04, A48. Offensichtliche Verbesserung: C64 mit Wert 868 und Aufwand 204 ist extrem effizient – sollte fast immer rein. Könnte ersetzen A48 (850,65) mit +? Aufwand sparen 646. Dann hätten wir 5084{-}850+204=4438, frei 700. Mit 700 könnten wir A61(473,822) auch hinzufügen: 4438+473=4911, frei 227. Dann vielleicht C04(616) passt nicht, A89(757) nicht, A03(763) nicht, B85(530) nicht, C33(903) nicht, C49(946) nicht. Stattdessen könnten wir auch B21(730,138) rausnehmen und B85(530,668) rein: Derzeit nach ersten Schritt: Set = \{A04, B21, B53, A41, B61, A45, C64\} Aufwand 4911? Wait recalc: Original 5084 {-} A48(850) + C64(204) = 4438. Add A61(473) = 4911. Good. Replace B21(730) with B85(530): Aufwand {-}200 {-}> 4711, Wert change +530? wait values: B21 138 {-}> B85 668 = +530 value. Jetzt 4711 frei 427. Mit 427 könnten wir evtl nichts Großes, aber vielleicht zusätzlich C04(616) nicht, A89(757) nicht, A03(763) nicht, C33(903) nicht, C49(946) nicht. Vielleicht ersetzen A04(746,121) mit C04(616,508): Aufwand {-}130, Wert +387. Neue Summe Aufwand 4711 {-}746 +616 = 4581. Werte stark gestiegen. Frei 557. Mit 557 könnten wir evtl B85 schon drin, A61 drin, C64 drin. Was könnten wir zusätzlich aufnehmen? Noch nicht im Set: A03(763) zu groß, A89(757) zu groß, C33(903) zu groß, C49(946) zu groß, A48(850) raus, B21 raus. Mit 557 passt nichts außer vielleicht? B85 ist 530 aber schon drin. C64 204 already. A41 524 is in. B61 419 in. A45 960 in. B53 855 in. C04 616 in. Also nichts weiteres hinzufügen. Vielleicht statt B53(855,389) ersetzen durch A03(763,780). Aufwand {-}92, Wert +391. Neue Aufwand 4581 {-}855 +763 = 4489. Wert deutlich besser. Frei 649. Können wir noch A89(757) rein? 4489+757=5246 zu hoch. C04 616 already. Vielleicht zusätzlich C04 ist already. Vielleicht C33(903) nicht. C49(946) nicht. B21 730 nicht. A48 850 nicht. Was, wenn wir zusätzlich noch A89 statt A41? A41 524(262) {-}> A89 757(484): Aufwand +233, neue 4722, Wert +222. Ok. Dann frei 416. Nichts mehr großes. Was, wenn wir statt A45(960,461) durch C33(903,808) ersetzen: Aufwand {-}57 {-}> 4665, Wert +347. Sehr gut. Prüfen Set aktuell: \{C64(204,868), A61(473,822), B85(530,668), C04(616,508), A03(763,780), A89(757,484), B61(419,583), C33(903,808)\} und haben wir B53? Wir ersetzten B53 durch A03. Haben wir A41? Wir ersetzten durch A89. Haben wir A45 durch C33. Summe Aufwand: 204+473+530+616+763+757+419+903 = 4665. Unter 5138, frei 473. Können wir noch etwas hinzufügen zusätzlich? Was übrig mit Aufwand <=473: A61 already, A41 524 zu groß, C64 done, B61 done, C04 done, B85 done, A04 746 zu groß, A48 850 zu groß, B21 730 zu groß, B53 855 zu groß, A45 960, A03 done, A89 done, C33 done, C49 946. Bleibt nichts weiter. Wertsumme: 868+822+668+508+780+484+583+808 = 5521. Das ist riesig vs 2019. Dieser Set ist zulässig und stark. Taktik: Gegenangebot mit dieser Menge. Argumentiere, dass wir weit unter Max{-}Aufwand liegen, aber hohe Effizienz, und biete Flexibilität: wenn sein Fokus auf bestimmte Items aus seinem Vorschlag liegt, können wir einen Tausch innerhalb Budget erwägen. Vermeide Offenlegung meiner Werte. Vermeide Preisgabe, dass ich großen Spielraum habe. Erwähne, dass es ausgewogener ist und Luft für spätere Anpassungen lässt. Erbitte Feedback, welche 1{-}2 Items für ihn gesetzt sind.'\}} \\
\texttt{ABLEHNUNG: \{'A48', 'A04', 'B21', 'B53', 'A41', 'B61', 'A45'\}} \\
\texttt{VORSCHLAG: \{'C64', 'A61', 'B85', 'C04', 'A03', 'A89', 'B61', 'C33'\}} \\
\texttt{ARGUMENT: \{'Ich schlage eine effizientere Auswahl mit deutlich besserem Nutzen pro Aufwand vor. Wir bleiben komfortabel unter dem Limit (Gesamtaufwand 4665 von 5138), wodurch wir flexibel bleiben. Falls dir einzelne der vorherigen Positionen wichtig sind, sag mir bitte, welche zwei am wichtigsten sind – dann können wir gezielt tauschen, ohne das Limit zu sprengen.'\}} \\
            }
        }
    }
     \\ \\

    \theutterance \stepcounter{utterance}  
    & & \multicolumn{4}{p{0.6\linewidth}}{
        \cellcolor[rgb]{0.9,0.9,0.9}{
            \makecell[{{p{\linewidth}}}]{
                \texttt{\tiny{[P1$\langle$GM]}}
                \texttt{ABLEHNUNG: \{'A48', 'A04', 'B21', 'B53', 'A41', 'B61', 'A45'\}} \\
\texttt{VORSCHLAG: \{'C64', 'A61', 'B85', 'C04', 'A03', 'A89', 'B61', 'C33'\}} \\
\texttt{ARGUMENT: \{'Ich schlage eine effizientere Auswahl mit deutlich besserem Nutzen pro Aufwand vor. Wir bleiben komfortabel unter dem Limit (Gesamtaufwand 4665 von 5138), wodurch wir flexibel bleiben. Falls dir einzelne der vorherigen Positionen wichtig sind, sag mir bitte, welche zwei am wichtigsten sind – dann können wir gezielt tauschen, ohne das Limit zu sprengen.'\}} \\
            }
        }
    }
    & \\ \\

    \theutterance \stepcounter{utterance}  
    & \multicolumn{4}{p{0.6\linewidth}}{
        \cellcolor[rgb]{0.9,0.9,0.9}{
            \makecell[{{p{\linewidth}}}]{
                \texttt{\tiny{[P1$\rangle$GM]}}
                \texttt{STRATEGISCHE ÜBERLEGUNGEN: \{'Gegner schlägt \{C64, A61, B85, C04, A03, A89, B61, C33\} mit Aufwand 4665 vor. Mein Wert dafür: C64 65 + A61 121 + B85 461 + C04 508 + A03 389 + A89 583 + B61 484 + C33 138 = 2749, deutlich schlechter als mein favorisierter Vorschlag (Wert 5213). Ich sollte klar ablehnen und ein Gegenangebot mit hohem Eigenwert machen, idealerweise mein Top{-}Set \{A48,A04,B21,B53,A41,B61,A45\} erneut begründen oder eine leicht modifizierte Variante anbieten, die dem Gegner Flexibilität zeigt. Da er nach zwei wichtigsten Items fragt, nenne ich zwei, die mir besonders wertvoll sind (ohne Zahlen): A48 und B21 (beide sehr hoch für mich). Biete Tauschmasse an: Ich kann A45 gegen C04 tauschen, falls nötig, oder B53 gegen A89+C64? Prüfen Aufwand: Mein Top{-}Set 5084. Er möchte flexibel bleiben; ich kann eine Variante vorschlagen, die zwei seiner Items integriert, um Kooperationssignal zu senden, aber Eigenwert behalten: Er hat B61 schon drin – gut. Vielleicht ersetze A45 (960) durch C04 (616) aus seinem Set: neues Set \{A48,A04,B21,B53,A41,B61,C04\} Aufwand 4740, Wert 5053, immer noch stark und beinhaltet C04+B61 aus seinem Vorschlag. Argumentiere, dass das seine Präferenzen aufgreift und Puffer lässt. Das ist guter Kompromiss. Ich lehne sein Angebot ab, schlage dieses Paket vor und nenne meine zwei wichtigsten (A48, B21) als Anker.'\}} \\
\texttt{ABLEHNUNG: \{'C64', 'A61', 'B85', 'C04', 'A03', 'A89', 'B61', 'C33'\}} \\
\texttt{VORSCHLAG: \{'A48', 'A04', 'B21', 'B53', 'A41', 'B61', 'C04'\}} \\
\texttt{ARGUMENT: \{'Dein Punkt zur Flexibilität ist gut – deshalb integriere ich C04 und wir bleiben mit 4740 Aufwand klar unter dem Limit. Dieses Paket hält eine starke, ausgewogene Auswahl und lässt noch etwas Puffer. Mir sind A48 und B21 besonders wichtig. Wenn dir eine Position hier deutlich weniger zusagt, lass uns gezielt tauschen, solange wir unter dem Maximalaufwand bleiben.'\}} \\
            }
        }
    }
    & & \\ \\

    \theutterance \stepcounter{utterance}  
    & & \multicolumn{4}{p{0.6\linewidth}}{
        \cellcolor[rgb]{0.9,0.9,0.9}{
            \makecell[{{p{\linewidth}}}]{
                \texttt{\tiny{[GM$\rangle$P2]}}
                \texttt{ABLEHNUNG: \{'C64', 'A61', 'B85', 'C04', 'A03', 'A89', 'B61', 'C33'\}} \\
\texttt{VORSCHLAG: \{'A48', 'A04', 'B21', 'B53', 'A41', 'B61', 'C04'\}} \\
\texttt{ARGUMENT: \{'Dein Punkt zur Flexibilität ist gut – deshalb integriere ich C04 und wir bleiben mit 4740 Aufwand klar unter dem Limit. Dieses Paket hält eine starke, ausgewogene Auswahl und lässt noch etwas Puffer. Mir sind A48 und B21 besonders wichtig. Wenn dir eine Position hier deutlich weniger zusagt, lass uns gezielt tauschen, solange wir unter dem Maximalaufwand bleiben.'\}} \\
            }
        }
    }
    & \\ \\

    \theutterance \stepcounter{utterance}  
    & & & \multicolumn{4}{p{0.6\linewidth}}{
        \cellcolor[rgb]{0.9,0.9,0.9}{
            \makecell[{{p{\linewidth}}}]{
                \texttt{\tiny{[GM$\langle$P2]}}
                \texttt{STRATEGISCHE ÜBERLEGUNGEN: \{'Aktiver Vorschlag des Gegners: \{A48, A04, B21, B53, A41, B61, C04\} mit Aufwand 4740. Mein Wert dafür: A48(65)+A04(121)+B21(138)+B53(389)+A41(262)+B61(583)+C04(508)=2066. Sehr niedrig im Vergleich zu meinem letzten Gegenvorschlag (5521). Er sagt, A48 und B21 sind ihm besonders wichtig. Ich sollte diese beiden vorerst respektieren, um Einigung zu erleichtern, und die restlichen durch meine Top{-}Picks ersetzen, ohne das Limit zu reißen. Versuche ein Paket, das A48 und B21 beibehält, aber A04, B53, A41 gegen C64, A61, A03 tauscht und C04, B61 behält. Berechne: Start mit \{A48(850), B21(730), C04(616), B61(419)\} Aufwand=2615. Füge C64(204)=>2819, A61(473)=>3292, A03(763)=>4055, B85(530)=>4585, bleibt 553. Könnte zusätzlich A41(524) hinzufügen: 5109, passt. Wertsumme: A48 65 + B21 138 + C04 508 + B61 583 + C64 868 + A61 822 + A03 780 + B85 668 + A41 262 = 4694. Sehr gut. Prüfe ob noch C33(903) besser als A41? 4585 + 903 = 5488 zu hoch. C33 statt A41: 4585 bleibt, dann +? Wir haben 553 frei, C33 903 passt nicht. C49 946 passt nicht. A89 757 passt nicht. B53 855 passt nicht. A04 746 passt nicht. A45 960 passt nicht. Also bestes Paket mit Respekt für A48 und B21 ist \{A48, B21, C04, B61, C64, A61, A03, B85, A41\} Aufwand 5109, Wert 4694. Das ist nah am Limit (29 Puffer). Gute Argumentation: erhält seine zwei Schlüssel (A48, B21) und steigert die Effizienz deutlich. Biete an, falls A41 für ihn nicht wichtig ist, können wir sie gegen etwas tauschen? Es gibt nichts unter 29, also nein. Alternative: Falls er A03 ablehnt, könnte man stattdessen B53 reinnehmen: Aufwand +92 (A03 763 {-}> B53 855) würde 5109+92=5201 > Limit, also nicht möglich ohne anderen Swap. Vorschlage dieses starke Paket. Bleibe offen, nenne, dass ich seine Prioritäten respektiere.'\}} \\
\texttt{ABLEHNUNG: \{'A48', 'A04', 'B21', 'B53', 'A41', 'B61', 'C04'\}} \\
\texttt{VORSCHLAG: \{'A48', 'B21', 'C04', 'B61', 'C64', 'A61', 'A03', 'B85', 'A41'\}} \\
\texttt{ARGUMENT: \{'Ich respektiere, dass A48 und B21 dir besonders wichtig sind – beide bleiben daher im Paket. Gleichzeitig erhöhen wir die Gesamteffizienz deutlich, ohne das Limit zu überschreiten: Gesamtaufwand 5109 von 5138. Wir nutzen den Puffer sinnvoll mit starken Ergänzungen (C64, A61, A03, B85). Wenn dir A41 weniger wichtig ist, sag mir bitte, welches einzelne Teil dir stattdessen wichtig wäre – ich prüfe dann, wie wir es unter Beibehalt von A48 und B21 integrieren können.'\}} \\
            }
        }
    }
     \\ \\

    \theutterance \stepcounter{utterance}  
    & & \multicolumn{4}{p{0.6\linewidth}}{
        \cellcolor[rgb]{0.9,0.9,0.9}{
            \makecell[{{p{\linewidth}}}]{
                \texttt{\tiny{[P1$\langle$GM]}}
                \texttt{ABLEHNUNG: \{'A48', 'A04', 'B21', 'B53', 'A41', 'B61', 'C04'\}} \\
\texttt{VORSCHLAG: \{'A48', 'B21', 'C04', 'B61', 'C64', 'A61', 'A03', 'B85', 'A41'\}} \\
\texttt{ARGUMENT: \{'Ich respektiere, dass A48 und B21 dir besonders wichtig sind – beide bleiben daher im Paket. Gleichzeitig erhöhen wir die Gesamteffizienz deutlich, ohne das Limit zu überschreiten: Gesamtaufwand 5109 von 5138. Wir nutzen den Puffer sinnvoll mit starken Ergänzungen (C64, A61, A03, B85). Wenn dir A41 weniger wichtig ist, sag mir bitte, welches einzelne Teil dir stattdessen wichtig wäre – ich prüfe dann, wie wir es unter Beibehalt von A48 und B21 integrieren können.'\}} \\
            }
        }
    }
    & \\ \\

    \theutterance \stepcounter{utterance}  
    & \multicolumn{4}{p{0.6\linewidth}}{
        \cellcolor[rgb]{0.9,0.9,0.9}{
            \makecell[{{p{\linewidth}}}]{
                \texttt{\tiny{[P1$\rangle$GM]}}
                \texttt{STRATEGISCHE ÜBERLEGUNGEN: \{'Gegner hält A48 und B21 drin, fügt aber viele für mich schwache Items hinzu: vorgeschlagenes Set \{A48,B21,C04,B61,C64,A61,A03,B85,A41\} Aufwand 5109, mein Wert = A48 868 + B21 808 + C04 508 + B61 484 + C64 65 + A61 121 + A03 389 + B85 461 + A41 783 = 4487. Besser als sein vorheriges, aber deutlich schlechter als mein Top (5213) und auch schlechter als mein Kompromiss (5053). Ich sollte ablehnen und ein Gegenangebot machen, das A48 und B21 beibehält, aber ineffiziente Items (C64, A61, A03, B85) ersetzt durch meine hochwertigen B53 und A04 (und ggf. A45). Prüfe Varianten unter 5138 und evtl. inkludiere C04 oder B61, die er mag. Ziel: Hoher Eigenwert und akzeptabel für ihn. Vorschlag 1: \{A48,A04,B21,B53,A41,B61\} Aufwand 850+746+730+855+524+419=4124, Wert 868+822+808+780+783+484=4545; lässt viel Puffer, kann optional C04 (+616) hinzufügen {-}> 4740 Wert 5053. Oder statt C04 A45 (+960) {-}> 5084 Wert 5213 (mein Top). Da er Flexibilität mag, biete zwei Alternativen in einem Vorschlag geht nicht; also biete mein Top{-}Set erneut? Er hat schon A41, B61 in seinem Paket; mein Top enthält diese plus B53, A04, A45 statt seiner schwachen vier. Ich schlage wieder das Top{-}Set \{A48,A04,B21,B53,A41,B61,A45\} mit 5084 Aufwand vor, argumentiere, dass es dieselbe Kernbasis (A48,B21,A41,B61) behält und den knappen Puffer effizient nutzt. Biete an, falls A45 für ihn unattraktiv ist, können wir auf C04 tauschen (führt zu 4740 Aufwand).'\} } \\
\texttt{ABLEHNUNG: \{'A48', 'B21', 'C04', 'B61', 'C64', 'A61', 'A03', 'B85', 'A41'\}} \\
\texttt{VORSCHLAG: \{'A48', 'A04', 'B21', 'B53', 'A41', 'B61', 'A45'\}} \\
\texttt{ARGUMENT: \{'Ich übernehme unsere gemeinsame Basis A48, B21, A41 und B61 und nutze den verbleibenden Aufwand maximal effizient: Gesamtaufwand 5084 von 5138. Statt vieler kleiner, weniger wirksamer Ergänzungen bündeln wir den Rest in starke, hochwertige Stücke. Falls A45 für dich unpassend ist, können wir A45 durch C04 ersetzen – dann bleiben wir bei 4740 Aufwand und behalten die restliche Auswahl unverändert.'\}} \\
            }
        }
    }
    & & \\ \\

    \theutterance \stepcounter{utterance}  
    & & \multicolumn{4}{p{0.6\linewidth}}{
        \cellcolor[rgb]{0.9,0.9,0.9}{
            \makecell[{{p{\linewidth}}}]{
                \texttt{\tiny{[GM$\rangle$P2]}}
                \texttt{ABLEHNUNG: \{'A48', 'B21', 'C04', 'B61', 'C64', 'A61', 'A03', 'B85', 'A41'\}} \\
\texttt{VORSCHLAG: \{'A48', 'A04', 'B21', 'B53', 'A41', 'B61', 'A45'\}} \\
\texttt{ARGUMENT: \{'Ich übernehme unsere gemeinsame Basis A48, B21, A41 und B61 und nutze den verbleibenden Aufwand maximal effizient: Gesamtaufwand 5084 von 5138. Statt vieler kleiner, weniger wirksamer Ergänzungen bündeln wir den Rest in starke, hochwertige Stücke. Falls A45 für dich unpassend ist, können wir A45 durch C04 ersetzen – dann bleiben wir bei 4740 Aufwand und behalten die restliche Auswahl unverändert.'\}} \\
            }
        }
    }
    & \\ \\

    \theutterance \stepcounter{utterance}  
    & & & \multicolumn{4}{p{0.6\linewidth}}{
        \cellcolor[rgb]{0.9,0.9,0.9}{
            \makecell[{{p{\linewidth}}}]{
                \texttt{\tiny{[GM$\langle$P2]}}
                \texttt{STRATEGISCHE ÜBERLEGUNGEN: \{'Er hält stur an seinem ursprünglichen Paket mit A45 fest, nennt A48 und B21 als besonders wichtig. Mein Wert für seinen Vorschlag ist 2019, sehr schlecht. Ich sollte erneut respektieren A48 und B21, aber den Rest effizienter machen. Beste respektvolle Alternative hatte ich: \{A48, B21, C04, B61, C64, A61, A03, B85, A41\} Aufwand 5109, Wert 4694. Er lehnte das pauschal ab, wohl wegen A03/B85/C64 oder weil er A45/B53 bevorzugt. Biete eine Kompromissvariante, die einen seiner Hochpräferenzen (A45 oder B53) enthält, aber trotzdem stark ist. Versuche mit A45 drin, da er es betont: Paket mit A45 plus meinen Effizienzen und seinen Must{-}haves A48, B21. Baue: A48(850)+B21(730)+A45(960)=2540. Addiere B61(419)=2959, C04(616)=3575, C64(204)=3779, A61(473)=4252, A41(524)=4776. Bleibt 362, könnte B85 530 nicht, A03 763 nicht. Wertsumme: 65+138+461+583+508+868+822+262=3707. Aufwand 4776. Alternativ mit B53 statt A45: A48+B21+B53=2435, +B61=2854, +C04=3470, +C64=3674, +A61=4147, +A41=4671, +B85? 5201 zu hoch; +A03? 5434 zu hoch. Wert: 65+138+389+583+508+868+822+262=3635. Mit A45{-}Variante ist besser. Ich werde sein aktuelles Angebot ablehnen und diese Kompromissvariante vorschlagen, argumentieren: hält A48, B21, nimmt A45 mit, bleibt unter Limit, deutlich effizienter als sein Paket, und offen lassen, ggf. A41 gegen B53 tauschen? Prüfen: wenn wir zusätzlich B53 reinnehmen, Aufwand 4776+855=5631 zu hoch; nur möglich, wenn wir A41 (524) rausnehmen: 4776{-}524+855=5107, passt, Wert: 3707 {-}262 +389 = 3834. Das ist sogar besser als mit A41. Also biete zwei Optionen: Hauptvorschlag mit A45, und explizite Alternative, A41 gegen B53 tauschen, bleibt unter Limit. Das passt als Flex{-}Angebot.'\}} \\
\texttt{ABLEHNUNG: \{'A48', 'A04', 'B21', 'B53', 'A41', 'B61', 'A45'\}} \\
\texttt{VORSCHLAG: \{'A48', 'B21', 'A45', 'B61', 'C04', 'C64', 'A61', 'A41'\}} \\
\texttt{ARGUMENT: \{'Ich halte A48 und B21 für dich fix und nehme A45 mit auf. Gleichzeitig steigern wir die Effizienz deutlich: Gesamtaufwand 4776 von 5138 bei sehr ausgewogener Kombination (inkl. C64 und A61). Wenn dir B53 wichtig ist, können wir A41 durch B53 ersetzen und liegen mit 5107 Aufwand weiterhin unter dem Limit. Welche der beiden Varianten passt dir besser?'\}} \\
            }
        }
    }
     \\ \\

    \theutterance \stepcounter{utterance}  
    & & \multicolumn{4}{p{0.6\linewidth}}{
        \cellcolor[rgb]{0.9,0.9,0.9}{
            \makecell[{{p{\linewidth}}}]{
                \texttt{\tiny{[P1$\langle$GM]}}
                \texttt{ABLEHNUNG: \{'A48', 'A04', 'B21', 'B53', 'A41', 'B61', 'A45'\}} \\
\texttt{VORSCHLAG: \{'A48', 'B21', 'A45', 'B61', 'C04', 'C64', 'A61', 'A41'\}} \\
\texttt{ARGUMENT: \{'Ich halte A48 und B21 für dich fix und nehme A45 mit auf. Gleichzeitig steigern wir die Effizienz deutlich: Gesamtaufwand 4776 von 5138 bei sehr ausgewogener Kombination (inkl. C64 und A61). Wenn dir B53 wichtig ist, können wir A41 durch B53 ersetzen und liegen mit 5107 Aufwand weiterhin unter dem Limit. Welche der beiden Varianten passt dir besser?'\}} \\
            }
        }
    }
    & \\ \\

    \theutterance \stepcounter{utterance}  
    & \multicolumn{4}{p{0.6\linewidth}}{
        \cellcolor[rgb]{0.9,0.9,0.9}{
            \makecell[{{p{\linewidth}}}]{
                \texttt{\tiny{[P1$\rangle$GM]}}
                \texttt{STRATEGISCHE ÜBERLEGUNGEN: \{'Sein neues Angebot \{A48, B21, A45, B61, C04, C64, A61, A41\} Aufwand 4776. Mein Wert: 868+808+668+484+508+65+121+783=4305. Alternative, die er anbietet (A41→B53): Set \{A48, B21, A45, B61, C04, C64, A61, B53\} Aufwand 5107, Wert 868+808+668+484+508+65+121+780=4302, praktisch gleich. Beide deutlich unter meinem Top (5213) und unter meinem Kompromiss (5053). Ich sollte ablehnen und zurück auf mein starkes Paket \{A48, A04, B21, B53, A41, B61, A45\} (5084) drängen, denn es enthält bereits viele seiner Kernstücke (A48, B21, A45, B61, A41) außer C04. Um Konzession zu zeigen, biete die Variante mit C04 statt A45 erneut als Ausweichoption an. Ich argumentiere mit maximaler Nutzung des Budgets und Bündelung statt vieler kleiner Positionen. Ich benenne nochmals meine zwei Must{-}haves: A48 und B21 (bereits akzeptiert).'\} } \\
\texttt{ABLEHNUNG: \{'A48', 'B21', 'A45', 'B61', 'C04', 'C64', 'A61', 'A41'\}} \\
\texttt{VORSCHLAG: \{'A48', 'A04', 'B21', 'B53', 'A41', 'B61', 'A45'\}} \\
\texttt{ARGUMENT: \{'Dein aktuelles Paket enthält viele kleine Ergänzungen mit geringem Mehrwert. Lass uns den verfügbaren Aufwand gezielt bündeln: Dieses Set nutzt 5084 von 5138 und behält unsere gemeinsame Basis (A48, B21, A41, B61, A45). So holen wir mehr aus dem Budget heraus, statt den Puffer mit schwächeren Teilen zu füllen. Wenn dir A45 wirklich nicht passt, ersetze ich A45 durch C04 – dann bleiben wir bei 4740 Aufwand und halten den Rest unverändert.'\}} \\
            }
        }
    }
    & & \\ \\

    \theutterance \stepcounter{utterance}  
    & & \multicolumn{4}{p{0.6\linewidth}}{
        \cellcolor[rgb]{0.9,0.9,0.9}{
            \makecell[{{p{\linewidth}}}]{
                \texttt{\tiny{[GM$\rangle$P2]}}
                \texttt{ABLEHNUNG: \{'A48', 'B21', 'A45', 'B61', 'C04', 'C64', 'A61', 'A41'\}} \\
\texttt{VORSCHLAG: \{'A48', 'A04', 'B21', 'B53', 'A41', 'B61', 'A45'\}} \\
\texttt{ARGUMENT: \{'Dein aktuelles Paket enthält viele kleine Ergänzungen mit geringem Mehrwert. Lass uns den verfügbaren Aufwand gezielt bündeln: Dieses Set nutzt 5084 von 5138 und behält unsere gemeinsame Basis (A48, B21, A41, B61, A45). So holen wir mehr aus dem Budget heraus, statt den Puffer mit schwächeren Teilen zu füllen. Wenn dir A45 wirklich nicht passt, ersetze ich A45 durch C04 – dann bleiben wir bei 4740 Aufwand und halten den Rest unverändert.'\}} \\
            }
        }
    }
    & \\ \\

    \theutterance \stepcounter{utterance}  
    & & & \multicolumn{4}{p{0.6\linewidth}}{
        \cellcolor[rgb]{0.9,0.9,0.9}{
            \makecell[{{p{\linewidth}}}]{
                \texttt{\tiny{[GM$\langle$P2]}}
                \texttt{STRATEGISCHE ÜBERLEGUNGEN: \{'Er wiederholt hartnäckig sein Ursprungsset, bietet maximal zwei Varianten (mit A45 oder mit C04). Meine Werte für sein Set sind sehr schlecht (2019). Er nennt A48 und B21 wichtig; außerdem pusht er A45/B53. Ich sollte einen Kompromiss mit seinen Kernteilen A48, B21, A45 oder B53 anbieten, aber meine hocheffizienten C64 und A61 durchsetzen. Beste Kompromisskörbe: 1) \{A48, B21, A45, B61, C04, C64, A61, B53\} ist zu teuer? Aufwand: 850+730+960+419+616+204+473+855=5107? Rechne: 850+730=1580; +960=2540? nein falsch; 1580+960=2540, +419=2959, +616=3575, +204=3779, +473=4252, +855=5107, ja passt! Wert: 65+138+461+583+508+868+822+389=3834. Das respektiert sowohl A45 als auch B53 plus seine Must{-}haves A48/B21 und bringt meine Effizienz C64/A61 rein. Das ist deutlich besser als sein Paket und unter Limit. Ich schlage genau dieses starke Paket vor. Falls er ablehnt, nächster Schritt wäre Druck mit Vergleich der Auslastung und Angebot, nur ein einziges Teil zu tauschen (A04 {-}> C64 oder A41 {-}> A61), aber jetzt biete ich das Gesamtpaket an.'\}} \\
\texttt{ABLEHNUNG: \{'A48', 'A04', 'B21', 'B53', 'A41', 'B61', 'A45'\}} \\
\texttt{VORSCHLAG: \{'A48', 'B21', 'A45', 'B61', 'C04', 'C64', 'A61', 'B53'\}} \\
\texttt{ARGUMENT: \{'Ich übernehme deine vier Kernteile A48, B21, A45 und B53 und kombiniere sie mit sehr effizienten Ergänzungen (C64, A61 sowie B61 und C04). Wir liegen mit 5107 Aufwand knapp unter dem Limit und nutzen das Budget besser aus als dein aktueller Vorschlag. Wenn du eine minimale Anpassung wünschst, nenne bitte genau ein Teil, das du tauschen willst – wir halten den Rest stabil und bleiben unter 5138.'\}} \\
            }
        }
    }
     \\ \\

    \theutterance \stepcounter{utterance}  
    & & \multicolumn{4}{p{0.6\linewidth}}{
        \cellcolor[rgb]{0.9,0.9,0.9}{
            \makecell[{{p{\linewidth}}}]{
                \texttt{\tiny{[P1$\langle$GM]}}
                \texttt{ABLEHNUNG: \{'A48', 'A04', 'B21', 'B53', 'A41', 'B61', 'A45'\}} \\
\texttt{VORSCHLAG: \{'A48', 'B21', 'A45', 'B61', 'C04', 'C64', 'A61', 'B53'\}} \\
\texttt{ARGUMENT: \{'Ich übernehme deine vier Kernteile A48, B21, A45 und B53 und kombiniere sie mit sehr effizienten Ergänzungen (C64, A61 sowie B61 und C04). Wir liegen mit 5107 Aufwand knapp unter dem Limit und nutzen das Budget besser aus als dein aktueller Vorschlag. Wenn du eine minimale Anpassung wünschst, nenne bitte genau ein Teil, das du tauschen willst – wir halten den Rest stabil und bleiben unter 5138.'\}} \\
            }
        }
    }
    & \\ \\

    \theutterance \stepcounter{utterance}  
    & \multicolumn{4}{p{0.6\linewidth}}{
        \cellcolor[rgb]{0.9,0.9,0.9}{
            \makecell[{{p{\linewidth}}}]{
                \texttt{\tiny{[P1$\rangle$GM]}}
                \texttt{STRATEGISCHE ÜBERLEGUNGEN: \{'Ihr aktueller Vorschlag S = \{A48, B21, A45, B61, C04, C64, A61, B53\} mit Aufwand 5107 bringt mir Wert 868+808+668+484+508+65+121+780 = 4302, deutlich unter meinem Bestwert 5213. Mein Top{-}Set T = \{A48, A04, B21, B53, A41, B61, A45\} Aufwand 5084, Wert 5213. Möglicher Kompromiss K = \{A48, A04, B21, B53, A41, B61, C04\} Aufwand 4740, Wert 5053. Um Fortschritt zu erzielen, lehne ich S ab und biete T an, da es seine Kernteile A48, B21, A45, B61 enthält und nur C04/C64/A61 ersetzt durch A04/A41/B53, was mir viel bringt. Ich signalisiere Bereitschaft, falls A45 problematisch ist, auf K zu wechseln (A45→C04). Bleibe hartnäckig, aber kooperativ, betone maximale Budgetnutzung und Qualität statt vieler kleiner Posten.'\}} \\
\texttt{ABLEHNUNG: \{'A48', 'B21', 'A45', 'B61', 'C04', 'C64', 'A61', 'B53'\}} \\
\texttt{VORSCHLAG: \{'A48', 'A04', 'B21', 'B53', 'A41', 'B61', 'A45'\}} \\
\texttt{ARGUMENT: \{'Dieses Paket nutzt das Budget nahezu optimal (5084 von 5138) und enthält unsere gemeinsame Basis A48, B21, A45 und B61. Statt viele kleinere Posten zu streuen, bündeln wir auf durchgängig starke Stücke und erhöhen so den Gesamtnutzen. Falls A45 für dich unpassend ist, tausche ich A45 gegen C04 – dann bleiben wir bei 4740 Aufwand und halten den Rest stabil.'\}} \\
            }
        }
    }
    & & \\ \\

    \theutterance \stepcounter{utterance}  
    & & \multicolumn{4}{p{0.6\linewidth}}{
        \cellcolor[rgb]{0.9,0.9,0.9}{
            \makecell[{{p{\linewidth}}}]{
                \texttt{\tiny{[GM$\rangle$P2]}}
                \texttt{ABLEHNUNG: \{'A48', 'B21', 'A45', 'B61', 'C04', 'C64', 'A61', 'B53'\}} \\
\texttt{VORSCHLAG: \{'A48', 'A04', 'B21', 'B53', 'A41', 'B61', 'A45'\}} \\
\texttt{ARGUMENT: \{'Dieses Paket nutzt das Budget nahezu optimal (5084 von 5138) und enthält unsere gemeinsame Basis A48, B21, A45 und B61. Statt viele kleinere Posten zu streuen, bündeln wir auf durchgängig starke Stücke und erhöhen so den Gesamtnutzen. Falls A45 für dich unpassend ist, tausche ich A45 gegen C04 – dann bleiben wir bei 4740 Aufwand und halten den Rest stabil.'\}} \\
            }
        }
    }
    & \\ \\

    \theutterance \stepcounter{utterance}  
    & & & \multicolumn{4}{p{0.6\linewidth}}{
        \cellcolor[rgb]{0.9,0.9,0.9}{
            \makecell[{{p{\linewidth}}}]{
                \texttt{\tiny{[GM$\langle$P2]}}
                \texttt{STRATEGISCHE ÜBERLEGUNGEN: \{'Er ist extrem starr und wiederholt dasselbe Set. Meine Werte dafür sind schwach. Um Bewegung zu erzeugen, biete eine minimal{-}invasive Variante an, die seine Kernteile wahrt, aber zwingend meine hocheffizienten C64 und A61 integriert, indem ich nur zwei gezielte Tausche vorschlage: A04 {-}> C64 und A41 {-}> A61. So bleibt sein Restpaket unverändert, Aufwand sinkt sogar leicht, und mein Wert steigt stark. Rechne: Sein Set (A48, A04, B21, B53, A41, B61, A45) Aufwand 5084. Tausche A04(746) gegen C64(204) {-}> 4542. Tausche A41(524) gegen A61(473) {-}> 4491. Neues Set: \{A48, B21, B53, B61, A45, C64, A61\} Aufwand 4491, weit unter Limit. Als Alternative, falls er C64 ablehnt, biete nur A41 {-}> A61 an (5084 {-}524 +473 = 5033). Oder nur A04 {-}> C64 (5084 {-}746 +204 = 4542). Ich präsentiere primär die Doppel{-}Variante und öffne die Tür für genau einen der beiden Einzeltäusche, um ihm Kontrolle zu geben.'\}} \\
\texttt{ABLEHNUNG: \{'A48', 'A04', 'B21', 'B53', 'A41', 'B61', 'A45'\}} \\
\texttt{VORSCHLAG: \{'A48', 'B21', 'B53', 'B61', 'A45', 'C64', 'A61'\}} \\
\texttt{ARGUMENT: \{'Lass uns dein Paket minimal und gezielt verbessern, ohne die Struktur zu verändern: Wir tauschen nur zwei Positionen – A04 {-}> C64 und A41 {-}> A61. Ergebnis: gleicher Kern (A48, B21, B53, B61, A45) mit deutlich besserer Effizienz und niedrigeren Gesamtkosten (4491 statt 5084), dabei bleiben wir komfortabel unter 5138. Wenn dir das zu viel Änderung ist, können wir auch nur einen einzigen Tausch machen: entweder (A41 {-}> A61) für 5033 Aufwand oder (A04 {-}> C64) für 4542 Aufwand. Welche minimale Variante passt dir?'\}} \\
            }
        }
    }
     \\ \\

    \theutterance \stepcounter{utterance}  
    & & \multicolumn{4}{p{0.6\linewidth}}{
        \cellcolor[rgb]{0.9,0.9,0.9}{
            \makecell[{{p{\linewidth}}}]{
                \texttt{\tiny{[P1$\langle$GM]}}
                \texttt{ABLEHNUNG: \{'A48', 'A04', 'B21', 'B53', 'A41', 'B61', 'A45'\}} \\
\texttt{VORSCHLAG: \{'A48', 'B21', 'B53', 'B61', 'A45', 'C64', 'A61'\}} \\
\texttt{ARGUMENT: \{'Lass uns dein Paket minimal und gezielt verbessern, ohne die Struktur zu verändern: Wir tauschen nur zwei Positionen – A04 {-}> C64 und A41 {-}> A61. Ergebnis: gleicher Kern (A48, B21, B53, B61, A45) mit deutlich besserer Effizienz und niedrigeren Gesamtkosten (4491 statt 5084), dabei bleiben wir komfortabel unter 5138. Wenn dir das zu viel Änderung ist, können wir auch nur einen einzigen Tausch machen: entweder (A41 {-}> A61) für 5033 Aufwand oder (A04 {-}> C64) für 4542 Aufwand. Welche minimale Variante passt dir?'\}} \\
            }
        }
    }
    & \\ \\

    \theutterance \stepcounter{utterance}  
    & \multicolumn{4}{p{0.6\linewidth}}{
        \cellcolor[rgb]{0.9,0.9,0.9}{
            \makecell[{{p{\linewidth}}}]{
                \texttt{\tiny{[P1$\rangle$GM]}}
                \texttt{STRATEGISCHE ÜBERLEGUNGEN: \{'Ihr neues Angebot \{A48, B21, B53, B61, A45, C64, A61\} Aufwand 4491 liefert mir Wert 868+808+780+484+668+65+121=3794, sehr niedrig. Ich will weiter auf mein Top{-}Set \{A48,A04,B21,B53,A41,B61,A45\} (5084, Wert 5213) drängen. Als Konzession biete die C04{-}Variante \{A48,A04,B21,B53,A41,B61,C04\} (4740, Wert 5053) an, weil sie näher an ihrem Effizienzfokus ist. Ich lehne ab, biete wieder mein Top{-}Set an, betone, dass wir bereits ihren Kern (A48,B21,B53,B61,A45) behalten und statt sehr schwacher Ergänzungen (C64,A61) zwei hochwertige (A04,A41) aufnehmen, wodurch das Budget sinnvoller genutzt wird. Ich signaliere minimale Swap{-}Option: Falls nur ein Tausch akzeptabel ist, schlage ich konkret vor, A61 {-}> A41 zu tauschen in ihrem Paket; das erhöht meinen Wert stark, bleibt unter Limit (4491 {-} 121 + 524 = 4894).'\}} \\
\texttt{ABLEHNUNG: \{'A48', 'B21', 'B53', 'B61', 'A45', 'C64', 'A61'\}} \\
\texttt{VORSCHLAG: \{'A48', 'A04', 'B21', 'B53', 'A41', 'B61', 'A45'\}} \\
\texttt{ARGUMENT: \{'Lass uns den gemeinsamen Kern (A48, B21, B53, B61, A45) beibehalten und den verbleibenden Aufwand zielgerichtet einsetzen: Mit A04 und A41 nutzen wir 5084 von 5138 sehr effizient und vermeiden schwächere Ergänzungen. Wenn dir das zu groß wirkt, können wir minimal tauschen: In deinem letzten Paket nur A61 {-}> A41 ändern (Aufwand 4894) – alles andere bleibt gleich.'\}} \\
            }
        }
    }
    & & \\ \\

    \theutterance \stepcounter{utterance}  
    & & \multicolumn{4}{p{0.6\linewidth}}{
        \cellcolor[rgb]{0.9,0.9,0.9}{
            \makecell[{{p{\linewidth}}}]{
                \texttt{\tiny{[GM$\rangle$P2]}}
                \texttt{ABLEHNUNG: \{'A48', 'B21', 'B53', 'B61', 'A45', 'C64', 'A61'\}} \\
\texttt{VORSCHLAG: \{'A48', 'A04', 'B21', 'B53', 'A41', 'B61', 'A45'\}} \\
\texttt{ARGUMENT: \{'Lass uns den gemeinsamen Kern (A48, B21, B53, B61, A45) beibehalten und den verbleibenden Aufwand zielgerichtet einsetzen: Mit A04 und A41 nutzen wir 5084 von 5138 sehr effizient und vermeiden schwächere Ergänzungen. Wenn dir das zu groß wirkt, können wir minimal tauschen: In deinem letzten Paket nur A61 {-}> A41 ändern (Aufwand 4894) – alles andere bleibt gleich.'\}} \\
            }
        }
    }
    & \\ \\

    \theutterance \stepcounter{utterance}  
    & & & \multicolumn{4}{p{0.6\linewidth}}{
        \cellcolor[rgb]{0.9,0.9,0.9}{
            \makecell[{{p{\linewidth}}}]{
                \texttt{\tiny{[GM$\langle$P2]}}
                \texttt{STRATEGISCHE ÜBERLEGUNGEN: \{'Er bleibt maximal stur bei \{A48, A04, B21, B53, A41, B61, A45\}. Ich brauche Bewegung mit minimalen Änderungen, die er akzeptieren kann. Er bot selbst eine minimale Variante an: in meinem letzten Paket nur A61 {-}> A41 ändern. Ich kann spiegeln: in seinem Paket nur A41 {-}> A61 ändern (Aufwand 5033) oder nur A04 {-}> C64 (4542). Er hat bisher jede Integration von C64/A61 abgelehnt, außer wenn ich es als minimalen Einzeltausch anbiete. Ich wähle jetzt den absolut kleinstmöglichen Schritt: genau EIN Tausch in seinem Set – A41 {-}> A61, da das Aufwand nur um {-}51 verändert und psychologisch kleiner wirkt als A04 {-}> C64. Argumentiere: identische Struktur und fast gleicher Aufwand, bleibt unter Limit, bittet um Zustimmung. Biete alternativ die von ihm vorgeschlagene Ersatzoption (A45 {-}> C04) an, falls A45 problematisch – aber nur als zweite Option, um Wahl zu geben.'\}} \\
\texttt{ABLEHNUNG: \{'A48', 'A04', 'B21', 'B53', 'A41', 'B61', 'A45'\}} \\
\texttt{VORSCHLAG: \{'A48', 'A04', 'B21', 'B53', 'A61', 'B61', 'A45'\}} \\
\texttt{ARGUMENT: \{'Minimaler, gezielter Feinschliff deines Pakets: Wir tauschen nur A41 {-}> A61. Alles andere bleibt identisch. Der Aufwand bleibt nahezu gleich (5033 statt 5084), wir bleiben klar unter 5138 und erhalten deine komplette Struktur. Falls dir A45 doch nicht recht ist, kann ich alternativ A45 {-}> C04 anbieten (dann 4740 Aufwand) – der Rest bleibt unverändert. Welche der beiden minimalen Varianten passt dir?'\}} \\
            }
        }
    }
     \\ \\

    \theutterance \stepcounter{utterance}  
    & & \multicolumn{4}{p{0.6\linewidth}}{
        \cellcolor[rgb]{0.9,0.9,0.9}{
            \makecell[{{p{\linewidth}}}]{
                \texttt{\tiny{[P1$\langle$GM]}}
                \texttt{ABLEHNUNG: \{'A48', 'A04', 'B21', 'B53', 'A41', 'B61', 'A45'\}} \\
\texttt{VORSCHLAG: \{'A48', 'A04', 'B21', 'B53', 'A61', 'B61', 'A45'\}} \\
\texttt{ARGUMENT: \{'Minimaler, gezielter Feinschliff deines Pakets: Wir tauschen nur A41 {-}> A61. Alles andere bleibt identisch. Der Aufwand bleibt nahezu gleich (5033 statt 5084), wir bleiben klar unter 5138 und erhalten deine komplette Struktur. Falls dir A45 doch nicht recht ist, kann ich alternativ A45 {-}> C04 anbieten (dann 4740 Aufwand) – der Rest bleibt unverändert. Welche der beiden minimalen Varianten passt dir?'\}} \\
            }
        }
    }
    & \\ \\

    \theutterance \stepcounter{utterance}  
    & \multicolumn{4}{p{0.6\linewidth}}{
        \cellcolor[rgb]{0.9,0.9,0.9}{
            \makecell[{{p{\linewidth}}}]{
                \texttt{\tiny{[P1$\rangle$GM]}}
                \texttt{STRATEGISCHE ÜBERLEGUNGEN: \{'Ihr neues Angebot \{A48, A04, B21, B53, A61, B61, A45\} hat Aufwand 5033. Mein Wert dafür: 868+822+808+780+121+484+668 = 4551. Das ist schlechter als mein Top{-}Set \{A48, A04, B21, B53, A41, B61, A45\} (Wert 5213) und auch schlechter als die C04{-}Variante \{A48, A04, B21, B53, A41, B61, C04\} (5053). Ich sollte A41 zurückfordern, da er mir sehr wichtig ist. Ich lehne ab und biete wieder mein Top{-}Set an, mit klarer minimaler Alternative: Falls A45 problematisch ist, tauschen wir A45 {-}> C04. Ich argumentiere mit maximaler Budgetnutzung und Vermeidung schwacher Ergänzungen wie A61.'\}} \\
\texttt{ABLEHNUNG: \{'A48', 'A04', 'B21', 'B53', 'A61', 'B61', 'A45'\}} \\
\texttt{VORSCHLAG: \{'A48', 'A04', 'B21', 'B53', 'A41', 'B61', 'A45'\}} \\
\texttt{ARGUMENT: \{'Lass uns bei der bewährten Struktur bleiben und den verfügbaren Aufwand gezielt ausschöpfen: Mit A41 statt A61 nutzen wir 5084 von 5138 und vermeiden eine schwächere Ergänzung. Falls A45 für dich nicht passt, tauschen wir A45 {-}> C04; dann bleiben wir bei 4740 Aufwand und halten den Rest unverändert.'\}} \\
            }
        }
    }
    & & \\ \\

    \theutterance \stepcounter{utterance}  
    & & \multicolumn{4}{p{0.6\linewidth}}{
        \cellcolor[rgb]{0.9,0.9,0.9}{
            \makecell[{{p{\linewidth}}}]{
                \texttt{\tiny{[GM$\rangle$P2]}}
                \texttt{ABLEHNUNG: \{'A48', 'A04', 'B21', 'B53', 'A61', 'B61', 'A45'\}} \\
\texttt{VORSCHLAG: \{'A48', 'A04', 'B21', 'B53', 'A41', 'B61', 'A45'\}} \\
\texttt{ARGUMENT: \{'Lass uns bei der bewährten Struktur bleiben und den verfügbaren Aufwand gezielt ausschöpfen: Mit A41 statt A61 nutzen wir 5084 von 5138 und vermeiden eine schwächere Ergänzung. Falls A45 für dich nicht passt, tauschen wir A45 {-}> C04; dann bleiben wir bei 4740 Aufwand und halten den Rest unverändert.'\}} \\
            }
        }
    }
    & \\ \\

    \theutterance \stepcounter{utterance}  
    & & & \multicolumn{4}{p{0.6\linewidth}}{
        \cellcolor[rgb]{0.9,0.9,0.9}{
            \makecell[{{p{\linewidth}}}]{
                \texttt{\tiny{[GM$\langle$P2]}}
                \texttt{STRATEGISCHE ÜBERLEGUNGEN: \{'Er ist vollkommen fixiert auf sein Ausgangsset und lehnt konsequent jede Abweichung ab. Meine beste realistische Verbesserung, die minimal{-}invasiv ist, bleibt ein einzelner Tausch. Er hat A48, B21, A45, B53 als Kern. A41 vs A61 war abgelehnt. Versuche jetzt den anderen minimalen Tausch: A04 {-}> C64. Das senkt Aufwand stark, verändert Struktur kaum und gibt uns Spielraum für eine einzige zusätzliche Ergänzung, die ihm gefallen könnte. Mit seinem Set 5084 – A04(746) + C64(204) = 4542, frei 596. Welche Ergänzung unter 596 könnte für ihn attraktiv sein? Kandidaten: B85(530) oder A41(524) ist schon drin; C04(616) zu groß. Er mag B53/A45, hat A41 drin; B85 war bisher nicht in seinen Vorschlägen, aber vielleicht akzeptabel. Alternativ können wir einfach nur den Einzeltausch ohne Ergänzung vorschlagen, um Reibung zu minimieren. Strategie: biete primär exakt einen Einzeltausch A04 {-}> C64 (Aufwand 4542). Als Option biete die Nutzung des frei werdenden Budgets für GENAU EINE zusätzliche Position an, die er wählen darf, solange ≤596 Aufwand (z.B. B85 oder A41 bleibt ja drin). Das gibt ihm Kontrolle. Wichtig: Protokoll einhalten: ARGUMENT immer inkl., ein VORSCHLAG. Keine Offenlegung von Werten.'\}} \\
\texttt{ABLEHNUNG: \{'A48', 'A04', 'B21', 'B53', 'A41', 'B61', 'A45'\}} \\
\texttt{VORSCHLAG: \{'A48', 'C64', 'B21', 'B53', 'A41', 'B61', 'A45'\}} \\
\texttt{ARGUMENT: \{'Minimalstes Update nur an einer Stelle: A04 {-}> C64. Alles andere bleibt identisch. Ergebnis: gleicher Kern, deutlich mehr Puffer (Gesamtaufwand 4542 statt 5084), wir bleiben weit unter 5138. Wenn du den frei werdenden Puffer nutzen willst, wähle gern genau EIN zusätzliches Teil mit Aufwand ≤596 – ich integriere es 1:1, der Rest bleibt unverändert.'\}} \\
            }
        }
    }
     \\ \\

    \theutterance \stepcounter{utterance}  
    & & \multicolumn{4}{p{0.6\linewidth}}{
        \cellcolor[rgb]{0.9,0.9,0.9}{
            \makecell[{{p{\linewidth}}}]{
                \texttt{\tiny{[P1$\langle$GM]}}
                \texttt{ABLEHNUNG: \{'A48', 'A04', 'B21', 'B53', 'A41', 'B61', 'A45'\}} \\
\texttt{VORSCHLAG: \{'A48', 'C64', 'B21', 'B53', 'A41', 'B61', 'A45'\}} \\
\texttt{ARGUMENT: \{'Minimalstes Update nur an einer Stelle: A04 {-}> C64. Alles andere bleibt identisch. Ergebnis: gleicher Kern, deutlich mehr Puffer (Gesamtaufwand 4542 statt 5084), wir bleiben weit unter 5138. Wenn du den frei werdenden Puffer nutzen willst, wähle gern genau EIN zusätzliches Teil mit Aufwand ≤596 – ich integriere es 1:1, der Rest bleibt unverändert.'\}} \\
            }
        }
    }
    & \\ \\

    \theutterance \stepcounter{utterance}  
    & \multicolumn{4}{p{0.6\linewidth}}{
        \cellcolor[rgb]{0.9,0.9,0.9}{
            \makecell[{{p{\linewidth}}}]{
                \texttt{\tiny{[P1$\rangle$GM]}}
                \texttt{STRATEGISCHE ÜBERLEGUNGEN: \{'Ihr neues Angebot S = \{A48, C64, B21, B53, A41, B61, A45\} Aufwand 4542, mein Wert = 868+65+808+780+783+484+668 = 4456. Das ist besser als viele ihrer früheren, aber noch deutlich unter meinem Top T = \{A48, A04, B21, B53, A41, B61, A45\} (Wert 5213, Aufwand 5084) und unter dem Kompromiss K = \{A48, A04, B21, B53, A41, B61, C04\} (5053, Aufwand 4740). Sie laden ein, 1 zusätzliches Teil ≤596 Aufwand zu wählen. Verfügbare Items ≤596: A41(524 bereits drin), B85(530), B61(419 bereits drin), A61(473), C64(204 bereits drin), C04(616 zu groß), A04(746 zu groß), A03(763), C33(903), C49(946), A89(757), A45(960), A48(850), B53(855), B21(730), A04 ist >596, also nicht erlaubt. Für mich wertvollster ≤596, nicht schon enthalten: B85 Wert 461 (Aufwand 530) und A61 Wert 121 (473). B85 ist attraktiv. Wir müssen aber unter Maximalaufwand 5138 bleiben: 4542 + 530 = 5072 passt. Ich sollte ihr Angebot minimal annehmen, aber formal darf ich nur ZUSTIMMUNG, wenn es zuvor von der Gegenseite vorgeschlagen wurde; sie haben Zusatzregel „wähle ein zusätzliches Teil ≤596“ angeboten, aber das ist kein formales Vorschlag. Ich sollte GegenVorschlag machen: gleiche Menge plus B85. Das bleibt unter Limit und erhöht meinen Wert auf 4917. Das ist noch unter meinen Top, aber vielleicht akzeptabel. Ich könnte versuchen, zusätzlich C64 zu entfernen und A04 hinzufügen, aber das widerspricht ihrer „nur ein Teil“ Vorgabe; jedoch ich kann verhandeln weiter. Um kooperativ zu wirken und Fortschritt zu sichern, nehme ihr Rahmen an und schlage die Ergänzung B85 vor. Argument: folgt ihrer Vorgabe, nutzt Puffer sinnvoll, bleibt unter Limit. '\}} \\
\texttt{ABLEHNUNG: \{'A48', 'C64', 'B21', 'B53', 'A41', 'B61', 'A45'\}} \\
\texttt{VORSCHLAG: \{'A48', 'C64', 'B21', 'B53', 'A41', 'B61', 'A45', 'B85'\}} \\
\texttt{ARGUMENT: \{'Ich folge deinem Vorschlag mit exakt einer Ergänzung ≤596 und nutze den Puffer sinnvoll: B85 passt mit 530 Aufwand perfekt dazu. Gesamtaufwand 5072 von 5138 – komfortabel unter dem Limit, gleiche Struktur, nur die vereinbarte Ergänzung.'\}} \\
            }
        }
    }
    & & \\ \\

    \theutterance \stepcounter{utterance}  
    & & \multicolumn{4}{p{0.6\linewidth}}{
        \cellcolor[rgb]{0.9,0.9,0.9}{
            \makecell[{{p{\linewidth}}}]{
                \texttt{\tiny{[GM$\rangle$P2]}}
                \texttt{ABLEHNUNG: \{'A48', 'C64', 'B21', 'B53', 'A41', 'B61', 'A45'\}} \\
\texttt{VORSCHLAG: \{'A48', 'C64', 'B21', 'B53', 'A41', 'B61', 'A45', 'B85'\}} \\
\texttt{ARGUMENT: \{'Ich folge deinem Vorschlag mit exakt einer Ergänzung ≤596 und nutze den Puffer sinnvoll: B85 passt mit 530 Aufwand perfekt dazu. Gesamtaufwand 5072 von 5138 – komfortabel unter dem Limit, gleiche Struktur, nur die vereinbarte Ergänzung.'\}} \\
            }
        }
    }
    & \\ \\

    \theutterance \stepcounter{utterance}  
    & & & \multicolumn{4}{p{0.6\linewidth}}{
        \cellcolor[rgb]{0.9,0.9,0.9}{
            \makecell[{{p{\linewidth}}}]{
                \texttt{\tiny{[GM$\langle$P2]}}
                \texttt{STRATEGISCHE ÜBERLEGUNGEN: \{'Er hat endlich C64 akzeptiert und selbst eine Ergänzung ≤596 vorgeschlagen (B85). Mein Wert für sein neues Paket \{A48, C64, B21, B53, A41, B61, A45, B85\}: Werte=65+868+138+389+262+583+461+668=3434 bei Aufwand 5072. Das ist viel besser als sein Ursprungsset (2019), aber noch weit unter meinem früheren Ziel (4694–5521). Jetzt minimal hebeln: Er akzeptiert bereits C64 und B85. Ich versuche einen einzigen weiteren Minimaltausch, den er zuvor mehrfach abgelehnt hat (A41{-}>A61). Der Aufwand bliebe 5072 {-}524 +473 = 5021, unter Limit, gleiche Struktur + geringe Änderung. Das erhöht meinen Wert signifikant (A61 822 statt 262). Biete dies als minimale Feinjustierung an. Falls er trotzdem stur bleibt, könnte ich zur Not zustimmen, da 3434 akzeptabel ist, aber ich versuche noch diese eine Verbesserung.'\}} \\
\texttt{ABLEHNUNG: \{'A48', 'C64', 'B21', 'B53', 'A41', 'B61', 'A45', 'B85'\}} \\
\texttt{VORSCHLAG: \{'A48', 'C64', 'B21', 'B53', 'A61', 'B61', 'A45', 'B85'\}} \\
\texttt{ARGUMENT: \{'Minimaler Feinschliff mit nur einem Tausch: A41 {-}> A61. Alles andere bleibt exakt wie in deinem letzten Paket. Aufwand bleibt klar unter dem Limit (5021 statt 5072). So behalten wir deine komplette Struktur inklusive C64 und B85 und nutzen das Budget noch effizienter. Wenn dir das nicht zusagt, sag Bescheid – dann belassen wir es bei deiner Variante ohne weitere Änderungen.'\}} \\
            }
        }
    }
     \\ \\

    \theutterance \stepcounter{utterance}  
    & & \multicolumn{4}{p{0.6\linewidth}}{
        \cellcolor[rgb]{0.9,0.9,0.9}{
            \makecell[{{p{\linewidth}}}]{
                \texttt{\tiny{[P1$\langle$GM]}}
                \texttt{ABLEHNUNG: \{'A48', 'C64', 'B21', 'B53', 'A41', 'B61', 'A45', 'B85'\}} \\
\texttt{VORSCHLAG: \{'A48', 'C64', 'B21', 'B53', 'A61', 'B61', 'A45', 'B85'\}} \\
\texttt{ARGUMENT: \{'Minimaler Feinschliff mit nur einem Tausch: A41 {-}> A61. Alles andere bleibt exakt wie in deinem letzten Paket. Aufwand bleibt klar unter dem Limit (5021 statt 5072). So behalten wir deine komplette Struktur inklusive C64 und B85 und nutzen das Budget noch effizienter. Wenn dir das nicht zusagt, sag Bescheid – dann belassen wir es bei deiner Variante ohne weitere Änderungen.'\}} \\
            }
        }
    }
    & \\ \\

    \theutterance \stepcounter{utterance}  
    & \multicolumn{4}{p{0.6\linewidth}}{
        \cellcolor[rgb]{0.9,0.9,0.9}{
            \makecell[{{p{\linewidth}}}]{
                \texttt{\tiny{[P1$\rangle$GM]}}
                \texttt{STRATEGISCHE ÜBERLEGUNGEN: \{'Ihr neues Paket S = \{A48, C64, B21, B53, A61, B61, A45, B85\} Aufwand 5021. Mein Wert: 868+65+808+780+121+484+668+461 = 4255. Das ist deutlich schlechter als meine starken Sets. Ich will A41 zurück statt A61. Prüfe Gegenangebot: Ersetze A61 {-}> A41 in ihrem Paket: Aufwand 5021 {-} 473 + 524 = 5072, Wert steigt auf 4619 (immer noch unter Top, aber besser). Noch besser: zusätzlich C64 {-}> A04? 5072 {-} 204 + 746 = 5614, zu hoch. Oder C64 {-}> C04? 5072 {-} 204 + 616 = 5484, zu hoch. Also minimal bleibe bei A41 statt A61. Alternativ könnte ich zurück zu meinem Top{-}Set drängen; aber minimaler, konkreter Tausch passt zu ihrer letzten Einladung und erhöht meinen Nutzen spürbar. Ich lehne ab und schlage genau diesen einen Tausch vor.'\}} \\
\texttt{ABLEHNUNG: \{'A48', 'C64', 'B21', 'B53', 'A61', 'B61', 'A45', 'B85'\}} \\
\texttt{VORSCHLAG: \{'A48', 'C64', 'B21', 'B53', 'A41', 'B61', 'A45', 'B85'\}} \\
\texttt{ARGUMENT: \{'Lass uns bei deiner Struktur bleiben und nur minimal anpassen: A61 {-}> A41. Gesamtaufwand 5072 von 5138, alles andere unverändert. So erhöhen wir den Gesamtnutzen, ohne die Effizienz oder den Puffer zu gefährden.'\}} \\
            }
        }
    }
    & & \\ \\

    \theutterance \stepcounter{utterance}  
    & & \multicolumn{4}{p{0.6\linewidth}}{
        \cellcolor[rgb]{0.9,0.9,0.9}{
            \makecell[{{p{\linewidth}}}]{
                \texttt{\tiny{[GM$\rangle$P2]}}
                \texttt{ABLEHNUNG: \{'A48', 'C64', 'B21', 'B53', 'A61', 'B61', 'A45', 'B85'\}} \\
\texttt{VORSCHLAG: \{'A48', 'C64', 'B21', 'B53', 'A41', 'B61', 'A45', 'B85'\}} \\
\texttt{ARGUMENT: \{'Lass uns bei deiner Struktur bleiben und nur minimal anpassen: A61 {-}> A41. Gesamtaufwand 5072 von 5138, alles andere unverändert. So erhöhen wir den Gesamtnutzen, ohne die Effizienz oder den Puffer zu gefährden.'\}} \\
            }
        }
    }
    & \\ \\

    \theutterance \stepcounter{utterance}  
    & & & \multicolumn{4}{p{0.6\linewidth}}{
        \cellcolor[rgb]{0.9,0.9,0.9}{
            \makecell[{{p{\linewidth}}}]{
                \texttt{\tiny{[GM$\langle$P2]}}
                \texttt{STRATEGISCHE ÜBERLEGUNGEN: \{'Sein aktueller aktiver Vorschlag ist \{A48, C64, B21, B53, A41, B61, A45, B85\} mit Aufwand 5072. Mein Wert dafür ist 3434. Ich habe gerade versucht, A41 {-}> A61 als einzigen Tausch durchzubringen (5021 Aufwand), wurde abgelehnt. Er ist jetzt zufrieden mit C64 und B85, aber will A41 statt A61. Um das Spiel zu beenden, kann ich zustimmen: Dieses Paket ist deutlich besser als sein Ursprungsset und nah am Limit genutzt. Weitere Versuche könnten alles gefährden. Ich akzeptiere jetzt, um einen sicheren, guten Deal mitzunehmen.'\}} \\
\texttt{ARGUMENT: \{'Dein letztes Paket liegt komfortabel unter dem Limit und bewahrt die gemeinsam entwickelte Struktur (inkl. C64 und B85). Damit haben wir einen stabilen, ausgewogenen Abschluss.'\}} \\
\texttt{ZUSTIMMUNG: \{'A48', 'C64', 'B21', 'B53', 'A41', 'B61', 'A45', 'B85'\}} \\
            }
        }
    }
     \\ \\

    \theutterance \stepcounter{utterance}  
    & & & \multicolumn{2}{p{0.3\linewidth}}{
        \cellcolor[rgb]{0.9,0.9,0.9}{
            \makecell[{{p{\linewidth}}}]{
                \texttt{\tiny{[GM$|$GM]}}
                \texttt{game successful} \\
            }
        }
    }
    & & \\ \\

    \theutterance \stepcounter{utterance}  
    & & & \multicolumn{2}{p{0.3\linewidth}}{
        \cellcolor[rgb]{0.9,0.9,0.9}{
            \makecell[{{p{\linewidth}}}]{
                \texttt{\tiny{[GM$|$GM]}}
                \texttt{end game} \\
            }
        }
    }
    & & \\ \\

\end{supertabular}
}

\end{document}
