
\documentclass{article}
\usepackage{colortbl}
\usepackage{makecell}
\usepackage{multirow}
\usepackage{supertabular}

\begin{document}

\newcounter{utterance}

\centering \large Interaction Transcript for game `hot\_air\_balloon', experiment `air\_balloon\_survival\_de\_negotiation\_hard', episode 5 with gpt{-}5{-}2025{-}08{-}07{-}no{-}reasoning{-}t1.0.
\vspace{24pt}

{ \footnotesize  \setcounter{utterance}{1}
\setlength{\tabcolsep}{0pt}
\begin{supertabular}{c@{$\;$}|p{.15\linewidth}@{}p{.15\linewidth}p{.15\linewidth}p{.15\linewidth}p{.15\linewidth}p{.15\linewidth}}
    \# & $\;$A & \multicolumn{4}{c}{Game Master} & $\;\:$B\\
    \hline

    \theutterance \stepcounter{utterance}  
    & & \multicolumn{4}{p{0.6\linewidth}}{
        \cellcolor[rgb]{0.9,0.9,0.9}{
            \makecell[{{p{\linewidth}}}]{
                \texttt{\tiny{[P1$\langle$GM]}}
                \texttt{Du nimmst an einem kollaborativen Verhandlungspiel Teil.} \\
\\ 
\texttt{Zusammen mit einem anderen Teilnehmer musst du dich auf eine Reihe von Gegenständen entscheiden, die behalten werden. Jeder von euch hat eine persönliche Verteilung über die Wichtigkeit der einzelnen Gegenstände. Jeder von euch hat eine eigene Meinung darüber, wie wichtig jeder einzelne Gegenstand ist (Gegenstandswichtigkeit). Du kennst die Wichtigkeitsverteilung des anderen Spielers nicht. Zusätzlich siehst du, wie viel Aufwand jeder Gegenstand verursacht.  } \\
\texttt{Ihr dürft euch nur auf eine Reihe von Gegenständen einigen, wenn der Gesamtaufwand der ausgewählten Gegenstände den Maximalaufwand nicht überschreitet:} \\
\\ 
\texttt{Maximalaufwand = 3386} \\
\\ 
\texttt{Hier sind die einzelnen Aufwände der Gegenstände:} \\
\\ 
\texttt{Aufwand der Gegenstände = \{"C33": 313, "A89": 720, "A48": 175, "A04": 461, "A41": 635, "C64": 685, "C49": 544, "A61": 203, "C04": 369, "B53": 539, "B85": 4, "B61": 695, "A45": 399, "A03": 594, "B21": 437\}} \\
\\ 
\texttt{Hier ist deine persönliche Verteilung der Wichtigkeit der einzelnen Gegenstände:} \\
\\ 
\texttt{Werte der Gegenstandswichtigkeit = \{"C33": 138, "A89": 583, "A48": 65, "A04": 262, "A41": 121, "C64": 508, "C49": 461, "A61": 484, "C04": 668, "B53": 389, "B85": 215, "B61": 97, "A45": 500, "A03": 30, "B21": 400\}} \\
\\ 
\texttt{Ziel:} \\
\\ 
\texttt{Dein Ziel ist es, eine Reihe von Gegenständen auszuhandeln, die dir möglichst viel bringt (d. h. Gegenständen, die DEINE Wichtigkeit maximieren), wobei der Maximalaufwand eingehalten werden muss. Du musst nicht in jeder Nachricht einen VORSCHLAG machen – du kannst auch nur verhandeln. Alle Taktiken sind erlaubt!} \\
\\ 
\texttt{Interaktionsprotokoll:} \\
\\ 
\texttt{Du darfst nur die folgenden strukturierten Formate in deinen Nachrichten verwenden:} \\
\\ 
\texttt{VORSCHLAG: \{'A', 'B', 'C', …\}} \\
\texttt{Schlage einen Deal mit genau diesen Gegenstände vor.} \\
\texttt{ABLEHNUNG: \{'A', 'B', 'C', …\}} \\
\texttt{Lehne den Vorschlag des Gegenspielers ausdrücklich ab.} \\
\texttt{ARGUMENT: \{'...'\}} \\
\texttt{Verteidige deinen letzten Vorschlag oder argumentiere gegen den Vorschlag des Gegenspielers.} \\
\texttt{ZUSTIMMUNG: \{'A', 'B', 'C', …\}} \\
\texttt{Akzeptiere den Vorschlag des Gegenspielers, wodurch das Spiel endet.} \\
\texttt{STRATEGISCHE ÜBERLEGUNGEN: \{'...'\}} \\
\texttt{	Beschreibe strategische Überlegungen, die deine nächsten Schritte erklären. Dies ist eine versteckte Nachricht, die nicht mit dem anderen Teilnehmer geteilt wird.} \\
\\ 
\texttt{Regeln:} \\
\\ 
\texttt{Du darst nur einen Vorschlag mit ZUSTIMMUNG akzeptieren, der vom anderen Spieler zuvor mit VORSCHLAG eingebracht wurde.} \\
\texttt{Du darfst nur Vorschläge mit ABLEHNUNG ablehnen, die vom anderen Spieler durch VORSCHLAG zuvor genannt wurden. } \\
\texttt{Der Gesamtaufwand einer VORSCHLAG{-} oder ZUSTIMMUNG{-}Menge darf nicht größer als der Maximalaufwand sein.  } \\
\texttt{Offenbare deine versteckte Wichtigkeitsverteilung nicht.} \\
\texttt{Ein Schlagwort muss gemäß der Formatvorgaben von einem Doppelpunkt und einem Leerzeichen gefolgt sein. Das Argument muss eine Python{-}Menge mit 0 oder mehr Strings sein.  } \\
\texttt{Die Form muss also SCHLAGWORT: \{...\} sein.} \\
\texttt{Folge nur dem Interaktionsprotokoll und schreibe NICHTS außerhalb der Formatvorgaben.} \\
\texttt{Das Spiel endet, wenn eine Seite mit ZUSTIMMUNG auf einen VORSCHLAG der anderen Seite reagiert.  } \\
\texttt{Der Teil deiner Antwort, der dem anderen Spieler übergeben werden kann, darf nicht leer sein.  } \\
\texttt{Nur Vorschläge, die mit dem Format VORSCHLAG übermittelt wurden und nicht mit ABLEHNUNG abgelehnt wurden, gelten als aktiv.  } \\
\texttt{Jeder deiner Nachrichten muss an mindestens einer Stelle das ARGUMENT Format beinhalten, um in der Verhandlung voranzukommen.} \\
\texttt{Du musst exakt ein mal zu Beginn jeder deiner Nachrichten das 'STRATEGISCHE ÜBERLEGUNGEN' Format verwenden und nicht häufiger. Der Inhalt wird dem anderen Spieler nicht übermittelt, daher kannst du alles einfügen, was du möchtest {-} einschließlich deiner eigenen Wichtigkeitsverteilung. Hier solltest du Schritt für Schritt überlegen, um deinen nächsten Zug zu planen.} \\
            }
        }
    }
    & \\ \\

    \theutterance \stepcounter{utterance}  
    & \multicolumn{4}{p{0.6\linewidth}}{
        \cellcolor[rgb]{0.9,0.9,0.9}{
            \makecell[{{p{\linewidth}}}]{
                \texttt{\tiny{[P1$\rangle$GM]}}
                \texttt{STRATEGISCHE ÜBERLEGUNGEN: \{'Ziel: Maximierung meines Werts unter Aufwand ≤ 3386. Daten: Kosten und meine Werte. Sortiere nach Wert/Kosten und suche nahe Kapazitätsgrenze. Items, Werte, Kosten: C04(668,369), A89(583,720), A61(484,203), A45(500,399), C49(461,544), C64(508,685), B21(400,437), B53(389,539), C33(138,313), A04(262,461), A41(121,635), B85(215,4), A48(65,175), B61(97,695), A03(30,594). Greedy/knapsack{-}Heuristik: Sehr hohe Dichte: B85, A61, C04, A45, C64, C49, B21. Test{-}Kombination: \{C04(369), A61(203), A45(399), C49(544), C64(685), A89(720), B85(4)\} Kosten=2924, Wert=3419. Rest{-}Kapazität 462. Kandidaten reinpassen: C33(313,138) passt, A04(461,262) passt knapp mit 461, B21(437,400) passt, A48(175,65) passt. Beste Zusatzwert pro Rest: B21(+400,437) → neue Kosten 3361, Wert 3819, bleibt 25. Alternativ A04(+262,461) führt zu 3385, Wert 3681. B21 dominiert. Prüfen, ob noch C33 statt B21 besser? Mit C33: Kosten 3237, Wert 3557, aber dann noch A04 (+461) würde 3698, Kosten 3698>3386, nicht möglich. Oder C33 + A48 (+175) Kosten 3412>3386. C33 alleine schlechter als B21. A04 statt B21 gibt weniger Wert. A48 zusätzlich zu B21 über Kapazität (3361+175=3536>3386). C33 zusätzlich zu B21 über Kapazität (3361+313=3674). B53 statt irgendwas? Wenn ich ersetze A89 (720,583) durch B53 (539,389) spare 181 Aufwand, verliere 194 Wert; könnte dann noch C33 (+313) rein? 2924 {-}720 +539 +313 = 3056, Wert 3419 {-}583 +389 +138 = 3363, dann könnte noch B21 (+437) → Kosten 3493>3386. Oder +A04 (+461) → 3517>3386. Also schlechter. Der aktuell beste gefundene Wert: 3819 mit Set S = \{C04, A61, A45, C49, C64, A89, B85, B21\} Kosten 3361 ≤ 3386. Sehr stark. Taktik: Hart verhandeln, dieses Set vorschlagen und mit Effizienz/Balance argumentieren. Biete Bereitschaft zu kleinen Zugeständnissen bei weniger wertvollen Items (z. B. Austausch C49/C64 gegen etwas vom anderen, falls nötig), aber initial maximal fordern.'\}} \\
\texttt{VORSCHLAG: \{'C04', 'A61', 'A45', 'C49', 'C64', 'A89', 'B85', 'B21'\}} \\
\texttt{ARGUMENT: \{'Dieser Vorschlag nutzt das Budget effizient: Gesamtaufwand 3361 liegt unter 3386 und verteilt sich ausgewogen über Kategorien A, B und C. Er enthält mehrere besonders nützliche Gegenstände bei moderatem Aufwand (z. B. B85 ist praktisch kostenneutral, A61 und C04 sind sehr effiziente Ergänzungen). Wenn dir einzelne Elemente wichtig sind, können wir über punktuelle Anpassungen sprechen, solange der Gesamtaufwand eingehalten wird.'\}} \\
            }
        }
    }
    & & \\ \\

    \theutterance \stepcounter{utterance}  
    & & \multicolumn{4}{p{0.6\linewidth}}{
        \cellcolor[rgb]{0.9,0.9,0.9}{
            \makecell[{{p{\linewidth}}}]{
                \texttt{\tiny{[GM$\rangle$P2]}}
                \texttt{Du nimmst an einem kollaborativen Verhandlungspiel Teil.} \\
\\ 
\texttt{Zusammen mit einem anderen Teilnehmer musst du dich auf eine Reihe von Gegenständen entscheiden, die behalten werden. Jeder von euch hat eine persönliche Verteilung über die Wichtigkeit der einzelnen Gegenstände. Jeder von euch hat eine eigene Meinung darüber, wie wichtig jeder einzelne Gegenstand ist (Gegenstandswichtigkeit). Du kennst die Wichtigkeitsverteilung des anderen Spielers nicht. Zusätzlich siehst du, wie viel Aufwand jeder Gegenstand verursacht.  } \\
\texttt{Ihr dürft euch nur auf eine Reihe von Gegenständen einigen, wenn der Gesamtaufwand der ausgewählten Gegenstände den Maximalaufwand nicht überschreitet:} \\
\\ 
\texttt{Maximalaufwand = 3386} \\
\\ 
\texttt{Hier sind die einzelnen Aufwände der Gegenstände:} \\
\\ 
\texttt{Aufwand der Gegenstände = \{"C33": 313, "A89": 720, "A48": 175, "A04": 461, "A41": 635, "C64": 685, "C49": 544, "A61": 203, "C04": 369, "B53": 539, "B85": 4, "B61": 695, "A45": 399, "A03": 594, "B21": 437\}} \\
\\ 
\texttt{Hier ist deine persönliche Verteilung der Wichtigkeit der einzelnen Gegenstände:} \\
\\ 
\texttt{Werte der Gegenstandswichtigkeit = \{"C04": 30, "A89": 65, "C64": 97, "A45": 121, "A61": 138, "C49": 215, "B21": 262, "B53": 389, "A04": 400, "B85": 461, "C33": 484, "A41": 500, "B61": 508, "A48": 583, "A03": 668\}} \\
\\ 
\texttt{Ziel:} \\
\\ 
\texttt{Dein Ziel ist es, eine Reihe von Gegenständen auszuhandeln, die dir möglichst viel bringt (d. h. Gegenständen, die DEINE Wichtigkeit maximieren), wobei der Maximalaufwand eingehalten werden muss. Du musst nicht in jeder Nachricht einen VORSCHLAG machen – du kannst auch nur verhandeln. Alle Taktiken sind erlaubt!} \\
\\ 
\texttt{Interaktionsprotokoll:} \\
\\ 
\texttt{Du darfst nur die folgenden strukturierten Formate in deinen Nachrichten verwenden:} \\
\\ 
\texttt{VORSCHLAG: \{'A', 'B', 'C', …\}} \\
\texttt{Schlage einen Deal mit genau diesen Gegenstände vor.} \\
\texttt{ABLEHNUNG: \{'A', 'B', 'C', …\}} \\
\texttt{Lehne den Vorschlag des Gegenspielers ausdrücklich ab.} \\
\texttt{ARGUMENT: \{'...'\}} \\
\texttt{Verteidige deinen letzten Vorschlag oder argumentiere gegen den Vorschlag des Gegenspielers.} \\
\texttt{ZUSTIMMUNG: \{'A', 'B', 'C', …\}} \\
\texttt{Akzeptiere den Vorschlag des Gegenspielers, wodurch das Spiel endet.} \\
\texttt{STRATEGISCHE ÜBERLEGUNGEN: \{'...'\}} \\
\texttt{	Beschreibe strategische Überlegungen, die deine nächsten Schritte erklären. Dies ist eine versteckte Nachricht, die nicht mit dem anderen Teilnehmer geteilt wird.} \\
\\ 
\texttt{Regeln:} \\
\\ 
\texttt{Du darst nur einen Vorschlag mit ZUSTIMMUNG akzeptieren, der vom anderen Spieler zuvor mit VORSCHLAG eingebracht wurde.} \\
\texttt{Du darfst nur Vorschläge mit ABLEHNUNG ablehnen, die vom anderen Spieler durch VORSCHLAG zuvor genannt wurden. } \\
\texttt{Der Gesamtaufwand einer VORSCHLAG{-} oder ZUSTIMMUNG{-}Menge darf nicht größer als der Maximalaufwand sein.  } \\
\texttt{Offenbare deine versteckte Wichtigkeitsverteilung nicht.} \\
\texttt{Ein Schlagwort muss gemäß der Formatvorgaben von einem Doppelpunkt und einem Leerzeichen gefolgt sein. Das Argument muss eine Python{-}Menge mit 0 oder mehr Strings sein.  } \\
\texttt{Die Form muss also SCHLAGWORT: \{...\} sein.} \\
\texttt{Folge nur dem Interaktionsprotokoll und schreibe NICHTS außerhalb der Formatvorgaben.} \\
\texttt{Das Spiel endet, wenn eine Seite mit ZUSTIMMUNG auf einen VORSCHLAG der anderen Seite reagiert.  } \\
\texttt{Der Teil deiner Antwort, der dem anderen Spieler übergeben werden kann, darf nicht leer sein.  } \\
\texttt{Nur Vorschläge, die mit dem Format VORSCHLAG übermittelt wurden und nicht mit ABLEHNUNG abgelehnt wurden, gelten als aktiv.  } \\
\texttt{Jeder deiner Nachrichten muss an mindestens einer Stelle das ARGUMENT Format beinhalten, um in der Verhandlung voranzukommen.} \\
\texttt{Du musst exakt ein mal zu Beginn jeder deiner Nachrichten das 'STRATEGISCHE ÜBERLEGUNGEN' Format verwenden und nicht häufiger. Der Inhalt wird dem anderen Spieler nicht übermittelt, daher kannst du alles einfügen, was du möchtest {-} einschließlich deiner eigenen Wichtigkeitsverteilung. Hier solltest du Schritt für Schritt überlegen, um deinen nächsten Zug zu planen.} \\
\\ 
\texttt{Du erhältst nun die erste Nachricht des anderen Spielers.} \\
\\ 
\texttt{VORSCHLAG: \{'C04', 'A61', 'A45', 'C49', 'C64', 'A89', 'B85', 'B21'\}} \\
\texttt{ARGUMENT: \{'Dieser Vorschlag nutzt das Budget effizient: Gesamtaufwand 3361 liegt unter 3386 und verteilt sich ausgewogen über Kategorien A, B und C. Er enthält mehrere besonders nützliche Gegenstände bei moderatem Aufwand (z. B. B85 ist praktisch kostenneutral, A61 und C04 sind sehr effiziente Ergänzungen). Wenn dir einzelne Elemente wichtig sind, können wir über punktuelle Anpassungen sprechen, solange der Gesamtaufwand eingehalten wird.'\}} \\
            }
        }
    }
    & \\ \\

    \theutterance \stepcounter{utterance}  
    & & & \multicolumn{4}{p{0.6\linewidth}}{
        \cellcolor[rgb]{0.9,0.9,0.9}{
            \makecell[{{p{\linewidth}}}]{
                \texttt{\tiny{[GM$\langle$P2]}}
                \texttt{STRATEGISCHE ÜBERLEGUNGEN: \{'Mein Wert pro Item: A03 668 (594), A48 583 (175), B61 508 (695), A41 500 (635), C33 484 (313), B85 461 (4), A04 400 (461), B53 389 (539), B21 262 (437), C49 215 (544), A61 138 (203), A45 121 (399), C64 97 (685), A89 65 (720), C04 30 (369). Budget 3386. Ihr Vorschlag nutzt 3361 Aufwand und umfasst: C04(30,369), A61(138,203), A45(121,399), C49(215,544), C64(97,685), A89(65,720), B85(461,4), B21(262,437). Mein Wert dafür = 30+138+121+215+97+65+461+262 = 1,389. Sehr niedrig, obwohl B85 top ist. Meine Top{-}Kombos: B85(4), A48(175), C33(313), A03(594), A41(635), A04(461), B53(539). Prüfe Paket: \{B85, A48, C33, A03, A41\} Aufwand 4+175+313+594+635=1,721, Wert 2,749. Füge A04(461): Aufwand 2,182, Wert 3,149. Füge B53(539): Aufwand 2,721, Wert 3,538. Füge B61(695): Aufwand 3,416 (über Budget mit weiteren?), 3,538+508=4,046, aber 2,721+695=3,416 < 3,386? Nein, 3,416 > 3,386 um 30, also zu hoch. Alternative: statt B61 vielleicht A41 falls nicht drin? Schon drin. Füge A89(720) nicht effizient. Füge A61(203): 2,721+203=2,924, Wert 3,676. Füge A45(399): 3,323, Wert 3,797. Wir haben 63 Budget übrig. Könnten noch C33 falls nicht? Schon drin. C04(369) passt nicht. B21(437) passt nicht. C49(544) passt nicht. C64(685) passt nicht. A04 schon drin. Ergebnis Vorschlag: \{B85, A48, C33, A03, A41, A04, B53, A61, A45\} Aufwand 3,323, Wert 3,797. Sehr gut. Falls zu aggressiv, kann ich verhandeln, evtl. droppen A45 (399) und stattdessen B21 (437) geht nicht. A61 behalten wegen Effizienz. Einstieg: ablehne, schlage mein Paket vor, argumentiere mit Budget, hoher Gesamtwert und fairer Abdeckung, betone, dass B85 bleibt und effiziente Items enthalten, biete Offenheit für punktuelle Anpassungen.'\}} \\
\texttt{ABLEHNUNG: \{'C04', 'A61', 'A45', 'C49', 'C64', 'A89', 'B85', 'B21'\}} \\
\texttt{ARGUMENT: \{'Danke für den Auftakt. Dein Paket bleibt zwar knapp im Budget, enthält aber mehrere sehr teure und für mich wenig attraktive Posten (z. B. A89, C64), während extrem effiziente und nützliche Alternativen fehlen. Ich würde gern ein Set versuchen, das das Budget ebenfalls sauber nutzt, aber mit deutlich höherer Gesamtnützlichkeit und geringem Risiko.'\}} \\
\texttt{VORSCHLAG: \{'B85', 'A48', 'C33', 'A03', 'A41', 'A04', 'B53', 'A61', 'A45'\}} \\
\texttt{ARGUMENT: \{'Dieses Set bleibt mit Gesamtaufwand 3323 klar unter dem Limit 3386 und enthält viele hoch{-}effiziente Elemente (B85 quasi kostenfrei; A48, C33, A03, A41, A04 bieten starke Beiträge bei moderatem Aufwand). A61 ergänzt mit sehr gutem Kosten{-}Nutzen, A45 hält uns flexibel. Wenn dir an einzelnen Ersetzungen liegt, können wir punktuell tauschen, solange wir im Rahmen bleiben.'\}} \\
            }
        }
    }
     \\ \\

    \theutterance \stepcounter{utterance}  
    & & \multicolumn{4}{p{0.6\linewidth}}{
        \cellcolor[rgb]{0.9,0.9,0.9}{
            \makecell[{{p{\linewidth}}}]{
                \texttt{\tiny{[P1$\langle$GM]}}
                \texttt{ABLEHNUNG: \{'C04', 'A61', 'A45', 'C49', 'C64', 'A89', 'B85', 'B21'\}} \\
\texttt{ARGUMENT: \{'Danke für den Auftakt. Dein Paket bleibt zwar knapp im Budget, enthält aber mehrere sehr teure und für mich wenig attraktive Posten (z. B. A89, C64), während extrem effiziente und nützliche Alternativen fehlen. Ich würde gern ein Set versuchen, das das Budget ebenfalls sauber nutzt, aber mit deutlich höherer Gesamtnützlichkeit und geringem Risiko.'\}} \\
\texttt{VORSCHLAG: \{'B85', 'A48', 'C33', 'A03', 'A41', 'A04', 'B53', 'A61', 'A45'\}} \\
\texttt{ARGUMENT: \{'Dieses Set bleibt mit Gesamtaufwand 3323 klar unter dem Limit 3386 und enthält viele hoch{-}effiziente Elemente (B85 quasi kostenfrei; A48, C33, A03, A41, A04 bieten starke Beiträge bei moderatem Aufwand). A61 ergänzt mit sehr gutem Kosten{-}Nutzen, A45 hält uns flexibel. Wenn dir an einzelnen Ersetzungen liegt, können wir punktuell tauschen, solange wir im Rahmen bleiben.'\}} \\
            }
        }
    }
    & \\ \\

    \theutterance \stepcounter{utterance}  
    & \multicolumn{4}{p{0.6\linewidth}}{
        \cellcolor[rgb]{0.9,0.9,0.9}{
            \makecell[{{p{\linewidth}}}]{
                \texttt{\tiny{[P1$\rangle$GM]}}
                \texttt{STRATEGISCHE ÜBERLEGUNGEN: \{'Ihr Vorschlag hat Kosten 3323. Mein Wert dafür: B85(215)+A48(65)+C33(138)+A03(30)+A41(121)+A04(262)+B53(389)+A61(484)+A45(500)=2204, was sehr schlecht ggü. meinem vorherigen Set (3819). Ich sollte hart die für mich schwachen Items streichen: A03, A41, A48, C33 sind für mich fast wertlos. Ich will unbedingt halten: C04(668), A61(484), A45(500), C49(461), C64(508), A89(583), B85(215), B21(400). Wenn ich entgegenkommen muss, kann ich eines der teuren C64/A89/C49 fallweise tauschen. Gegenangebot mit hoher meinem Wert: Ersetze ihre Low{-}Value durch meine High{-}Value, Budget ≤3386. Baue: \{B85, A61, A45, C04, C49, C64, A89\} = Kosten 2924, füge B21 (+437) = 3361 (mein Top{-}Set). Um Kompromiss zu signalisieren, biete Alternative: ersetze C64 durch B53, füge B21, C04 etc. Prüfen Wert/Kosten: Set2 \{B85, A61, A45, C04, C49, A89, B21, B53\} Kosten: 3361 {-} C64(685) + B53(539) = 3215; könnte dann A04(461) dazu → 3676>3386; C33(313) → 3528>3386; A48(175) → 3390>3386 knapp drüber; A48 statt B21? 3215 {-}437 +175 = 2953, dann +C64? 2953+685=3638>3386. Also Set2 bleibt 3215 und lässt noch 171 Reserve; könnte A04 nicht, A48 passt (3215+175=3390>3386) nein; C33 (3215+313=3528) nein; A03(594) nein; A41(635) nein. Also Set2 schlechter für mich als Top{-}Set und lässt ungenutztes Budget. Ich bleibe bei Top{-}Set, argumentiere Effizienz und biete Bereitschaft, genau EIN teures Element (C64 oder A89) gegen einen ihrer Favoriten zu tauschen, falls sie ein konkretes nennen, solange mein Wertverlust begrenzt ist.'\}} \\
\texttt{ABLEHNUNG: \{'B85', 'A48', 'C33', 'A03', 'A41', 'A04', 'B53', 'A61', 'A45'\}} \\
\texttt{ARGUMENT: \{'Danke für das Gegenangebot – einige Elemente darin sind für mich kaum sinnvoll (A03, A41, A48, C33 liefern in Summe wenig Mehrwert bei relevantem Aufwand). Ich schlage vor, stark effiziente Kernstücke zu priorisieren und Kostenfresser ohne entsprechenden Beitrag zu vermeiden.'\}} \\
\texttt{VORSCHLAG: \{'C04', 'A61', 'A45', 'C49', 'C64', 'A89', 'B85', 'B21'\}} \\
\texttt{ARGUMENT: \{'Dieses Paket bleibt mit 3361 klar im Rahmen und ersetzt die schwachen Beiträge durch hochwirksame Alternativen. Wenn dir ein einzelner Posten daraus besonders schwerfällt, sag mir bitte welcher – ich prüfe gern einen gezielten Tausch an genau EINER Stelle, solange das Budget eingehalten wird.'\}} \\
            }
        }
    }
    & & \\ \\

    \theutterance \stepcounter{utterance}  
    & & \multicolumn{4}{p{0.6\linewidth}}{
        \cellcolor[rgb]{0.9,0.9,0.9}{
            \makecell[{{p{\linewidth}}}]{
                \texttt{\tiny{[GM$\rangle$P2]}}
                \texttt{ABLEHNUNG: \{'B85', 'A48', 'C33', 'A03', 'A41', 'A04', 'B53', 'A61', 'A45'\}} \\
\texttt{ARGUMENT: \{'Danke für das Gegenangebot – einige Elemente darin sind für mich kaum sinnvoll (A03, A41, A48, C33 liefern in Summe wenig Mehrwert bei relevantem Aufwand). Ich schlage vor, stark effiziente Kernstücke zu priorisieren und Kostenfresser ohne entsprechenden Beitrag zu vermeiden.'\}} \\
\texttt{VORSCHLAG: \{'C04', 'A61', 'A45', 'C49', 'C64', 'A89', 'B85', 'B21'\}} \\
\texttt{ARGUMENT: \{'Dieses Paket bleibt mit 3361 klar im Rahmen und ersetzt die schwachen Beiträge durch hochwirksame Alternativen. Wenn dir ein einzelner Posten daraus besonders schwerfällt, sag mir bitte welcher – ich prüfe gern einen gezielten Tausch an genau EINER Stelle, solange das Budget eingehalten wird.'\}} \\
            }
        }
    }
    & \\ \\

    \theutterance \stepcounter{utterance}  
    & & & \multicolumn{4}{p{0.6\linewidth}}{
        \cellcolor[rgb]{0.9,0.9,0.9}{
            \makecell[{{p{\linewidth}}}]{
                \texttt{\tiny{[GM$\langle$P2]}}
                \texttt{STRATEGISCHE ÜBERLEGUNGEN: \{'Ihr aktuelles Paket bringt mir nur 1,389 Wert und blockiert Budget mit für mich schwachen Items (A89, C64, C49, C04). Ich will auf mein starkes Set zurück, aber signalisiere Konzessionsbereitschaft bei EINEM gezielten Tausch, wie gefordert. Ich priorisiere das Halten von B85, A48, C33, A03, A41, A04, B53, A61; A45 ist am ehesten verzichtbar. Um auf ihr “ein gezielter Tausch”{-}Angebot einzugehen, biete ich genau einen Swap an: Nehme eines ihrer Kernstücke auf (z. B. B21) und gebe A45 ab, bleibe im Budget und sichere hohen Eigenwert. Optional könnte ich A61 gegen B21 tauschen, aber A61 ist sehr effizient für mich; daher lieber A45 raus, B21 rein. Neue Summe prüfen: Mein letztes Paket 3323 – A45(399) + B21(437) = 3361 exakt, das ist stark und entspricht sogar ihrem Budgetpunkt. Mein Wert steigt von 3,797 (inkl. A45) auf 3,797 {-} 121 + 262 = 3,938. Sehr gut. Ich argumentiere, dass ich auf ihre Bitte nach genau EINEM Tausch eingehe und ihr effizientes B21 integriere, während wir teure Low{-}Impact{-}Teile vermeiden. Wenn sie weiter auf A89/C64/C49 bestehen, lehne ich.\}} \\
\texttt{ABLEHNUNG: \{'C04', 'A61', 'A45', 'C49', 'C64', 'A89', 'B85', 'B21'\}} \\
\texttt{ARGUMENT: \{'Dein Paket bindet viel Budget in A89/C64/C49/C04, die im Verhältnis zum Aufwand zu wenig beitragen, und lässt sehr effiziente Alternativen außen vor. Du hast einen gezielten EINEN Tausch angeboten: Darauf gehe ich ein – ich integriere B21 punktuell, ohne das Budget zu reißen, und entferne dafür einen Posten aus meinem Set.'\}} \\
\texttt{VORSCHLAG: \{'B85', 'A48', 'C33', 'A03', 'A41', 'A04', 'B53', 'A61', 'B21'\}} \\
\texttt{ARGUMENT: \{'Budgettreffer 3361 exakt. Enthält deinen gewünschten B21 als gezielte Anpassung und hält gleichzeitig eine Reihe hoch{-}effizienter Bausteine (B85 quasi kostenneutral; A48, C33, A03, A41, A04, B53 liefern starkes Preis{-}Leistungs{-}Verhältnis; A61 bleibt als effiziente Ergänzung). Das erfüllt deinen Wunsch nach genau EINER Stelle der Anpassung und vermeidet die von mir als Kostenfresser empfundenen Teile. Wenn dir ein einzelnes Element hier wirklich Bauchschmerzen macht, sag mir genau welches – dann können wir prüfen, ob es gegen etwas mit ähnlich gutem Budgetprofil getauscht werden kann.'\}} \\
            }
        }
    }
     \\ \\

    \theutterance \stepcounter{utterance}  
    & & \multicolumn{4}{p{0.6\linewidth}}{
        \cellcolor[rgb]{0.9,0.9,0.9}{
            \makecell[{{p{\linewidth}}}]{
                \texttt{\tiny{[P1$\langle$GM]}}
                \texttt{ABLEHNUNG: \{'C04', 'A61', 'A45', 'C49', 'C64', 'A89', 'B85', 'B21'\}} \\
\texttt{ARGUMENT: \{'Dein Paket bindet viel Budget in A89/C64/C49/C04, die im Verhältnis zum Aufwand zu wenig beitragen, und lässt sehr effiziente Alternativen außen vor. Du hast einen gezielten EINEN Tausch angeboten: Darauf gehe ich ein – ich integriere B21 punktuell, ohne das Budget zu reißen, und entferne dafür einen Posten aus meinem Set.'\}} \\
\texttt{VORSCHLAG: \{'B85', 'A48', 'C33', 'A03', 'A41', 'A04', 'B53', 'A61', 'B21'\}} \\
\texttt{ARGUMENT: \{'Budgettreffer 3361 exakt. Enthält deinen gewünschten B21 als gezielte Anpassung und hält gleichzeitig eine Reihe hoch{-}effizienter Bausteine (B85 quasi kostenneutral; A48, C33, A03, A41, A04, B53 liefern starkes Preis{-}Leistungs{-}Verhältnis; A61 bleibt als effiziente Ergänzung). Das erfüllt deinen Wunsch nach genau EINER Stelle der Anpassung und vermeidet die von mir als Kostenfresser empfundenen Teile. Wenn dir ein einzelnes Element hier wirklich Bauchschmerzen macht, sag mir genau welches – dann können wir prüfen, ob es gegen etwas mit ähnlich gutem Budgetprofil getauscht werden kann.'\}} \\
            }
        }
    }
    & \\ \\

    \theutterance \stepcounter{utterance}  
    & \multicolumn{4}{p{0.6\linewidth}}{
        \cellcolor[rgb]{0.9,0.9,0.9}{
            \makecell[{{p{\linewidth}}}]{
                \texttt{\tiny{[P1$\rangle$GM]}}
                \texttt{STRATEGISCHE ÜBERLEGUNGEN: \{'Ihr neues Paket kostet 3361 und hat für mich Gesamtwert 2204, extrem schlecht. Besonders schwach für mich: A03(30), A41(121), A48(65), C33(138). Mein Top{-}Set bleibt \{C04, A61, A45, C49, C64, A89, B85, B21\} mit Wert 3819. Um Bewegung zu erzeugen, biete genau einen gezielten Tausch wie versprochen: ersetze eines meiner teuren Elemente gegen eins aus deinem Set mit minimalem Wertverlust für mich. Geringster Schaden wäre der Tausch C49→B53: Wert {-}72 bei Aufwand {-}5, passt noch. Ich bleibe hart gegen A03/A41/A48/C33, die ich fast gar nicht schätze. Wenn sie auf „genau eine Stelle“ pochen, liefere diesen einen Tausch und verkaufe ihn als Entgegenkommen. Falls weitere Forderungen kommen, verweise auf Budget{-} und Nutzen{-}Effizienz.'\}} \\
\texttt{ABLEHNUNG: \{'B85', 'A48', 'C33', 'A03', 'A41', 'A04', 'B53', 'A61', 'B21'\}} \\
\texttt{ARGUMENT: \{'Danke für die Anpassung mit B21. Allerdings enthalten A03, A41, A48 und C33 für mich zu wenig Gegenwert, sodass dieses Paket insgesamt unattraktiv bleibt. Ich bleibe bei einer gezielten EINEN Anpassung wie zugesagt: Ich nehme B53 anstelle von C49 auf – damit kommst du deiner Präferenz entgegen, ohne das Budget zu reißen.'\}} \\
\texttt{VORSCHLAG: \{'C04', 'A61', 'A45', 'C64', 'A89', 'B85', 'B21', 'B53'\}} \\
\texttt{ARGUMENT: \{'Gesamtaufwand 3356 (unter 3386). Das ist dein gewünschter „eine Stelle“{-}Tausch: B53 statt C49. Das Paket wahrt starke Effizienz (B85 quasi kostenneutral, A61/C04/A45 mit sehr gutem Profil) und kommt dir gleichzeitig entgegen. Wenn dir genau ein weiterer konkreter Posten Bauchschmerzen bereitet, nenne ihn – ich prüfe einen einmaligen Austausch mit ähnlichem Budgetprofil.'\}} \\
            }
        }
    }
    & & \\ \\

    \theutterance \stepcounter{utterance}  
    & & \multicolumn{4}{p{0.6\linewidth}}{
        \cellcolor[rgb]{0.9,0.9,0.9}{
            \makecell[{{p{\linewidth}}}]{
                \texttt{\tiny{[GM$\rangle$P2]}}
                \texttt{ABLEHNUNG: \{'B85', 'A48', 'C33', 'A03', 'A41', 'A04', 'B53', 'A61', 'B21'\}} \\
\texttt{ARGUMENT: \{'Danke für die Anpassung mit B21. Allerdings enthalten A03, A41, A48 und C33 für mich zu wenig Gegenwert, sodass dieses Paket insgesamt unattraktiv bleibt. Ich bleibe bei einer gezielten EINEN Anpassung wie zugesagt: Ich nehme B53 anstelle von C49 auf – damit kommst du deiner Präferenz entgegen, ohne das Budget zu reißen.'\}} \\
\texttt{VORSCHLAG: \{'C04', 'A61', 'A45', 'C64', 'A89', 'B85', 'B21', 'B53'\}} \\
\texttt{ARGUMENT: \{'Gesamtaufwand 3356 (unter 3386). Das ist dein gewünschter „eine Stelle“{-}Tausch: B53 statt C49. Das Paket wahrt starke Effizienz (B85 quasi kostenneutral, A61/C04/A45 mit sehr gutem Profil) und kommt dir gleichzeitig entgegen. Wenn dir genau ein weiterer konkreter Posten Bauchschmerzen bereitet, nenne ihn – ich prüfe einen einmaligen Austausch mit ähnlichem Budgetprofil.'\}} \\
            }
        }
    }
    & \\ \\

    \theutterance \stepcounter{utterance}  
    & & & \multicolumn{4}{p{0.6\linewidth}}{
        \cellcolor[rgb]{0.9,0.9,0.9}{
            \makecell[{{p{\linewidth}}}]{
                \texttt{\tiny{[GM$\langle$P2]}}
                \texttt{STRATEGISCHE ÜBERLEGUNGEN: \{'Ihr aktuelles Paket (C04, A61, A45, C64, A89, B85, B21, B53) kostet 3356 und gibt mir Wert: C04(30)+A61(138)+A45(121)+C64(97)+A89(65)+B85(461)+B21(262)+B53(389)=1,563. Noch schwach wegen A89/C64/C04. Sie erlauben „genau einen weiteren“ Austausch. Ich will maximalen Wertzuwachs mit minimalem Kostenimpact. Bester Swap: Ersetze A89(720, Wert 65) durch A48(175, Wert 583). Budget fällt um 545 auf 2811, Wert steigt +518 auf 2,081 – großer Sprung, aber sie sagten EIN Tausch. Das ist mein Wunsch. Alternativ A89 {-}> A03 (594, 668) erhöht Wert +603 und senkt Kosten {-}126 auf 3230; könnte für sie schwerer sein, da sie A03 ablehnen. A48 ist „effizient“ und ich kann es als Budget{-}Optimierung verkaufen. Bringe Vorschlag mit exakt einem Tausch: Paket = \{C04, A61, A45, C64, B85, B21, B53, A48\}. Kosten: 3356 {-} 720 + 175 = 2811. Ich argumentiere mit massiver Effizienzsteigerung, Einhaltung ihrer Ein{-}Tausch{-}Regel und Offenheit erst bei nächster Runde für weitere Anpassung. Falls sie ablehnen, nächster Schritt könnte zweiter Tausch: C64 {-}> C33 (685{-}>313) oder C04 {-}> A04, aber jetzt strikt nur einen Tausch fordern.\}} \\
\texttt{ABLEHNUNG: \{'C04', 'A61', 'A45', 'C64', 'A89', 'B85', 'B21', 'B53'\}} \\
\texttt{ARGUMENT: \{'Danke für den punktuellen Schritt mit B53. Dein Paket bleibt jedoch durch A89/C64/C04 für mich zu ineffizient. Du hast genau EINEN weiteren Austausch angeboten – den nutze ich gezielt, um die Gesamteffizienz massiv zu steigern, ohne dein Rahmenprinzip zu verletzen.'\}} \\
\texttt{VORSCHLAG: \{'C04', 'A61', 'A45', 'C64', 'B85', 'B21', 'B53', 'A48'\}} \\
\texttt{ARGUMENT: \{'Das ist dein Paket mit genau EINEM Tausch: A48 statt A89. Budget fällt deutlich auf 2811 (klar unter 3386), während der Gesamtbeitrag spürbar steigt. B85 bleibt als nahezu kostenfreie Stärke, A61/A45/C04 liefern Kontinuität, B53/B21 halten deine Struktur, und A48 ist eine hoch{-}effiziente Ergänzung. Wenn das für dich tragfähig ist, können wir hier landen. Falls dir dann noch genau ein weiterer Posten Bauchschmerzen bereitet, nenne ihn – wir prüfen einen einmaligen Austausch mit ähnlichem Budgetprofil.'\}} \\
            }
        }
    }
     \\ \\

    \theutterance \stepcounter{utterance}  
    & & \multicolumn{4}{p{0.6\linewidth}}{
        \cellcolor[rgb]{0.9,0.9,0.9}{
            \makecell[{{p{\linewidth}}}]{
                \texttt{\tiny{[P1$\langle$GM]}}
                \texttt{ABLEHNUNG: \{'C04', 'A61', 'A45', 'C64', 'A89', 'B85', 'B21', 'B53'\}} \\
\texttt{ARGUMENT: \{'Danke für den punktuellen Schritt mit B53. Dein Paket bleibt jedoch durch A89/C64/C04 für mich zu ineffizient. Du hast genau EINEN weiteren Austausch angeboten – den nutze ich gezielt, um die Gesamteffizienz massiv zu steigern, ohne dein Rahmenprinzip zu verletzen.'\}} \\
\texttt{VORSCHLAG: \{'C04', 'A61', 'A45', 'C64', 'B85', 'B21', 'B53', 'A48'\}} \\
\texttt{ARGUMENT: \{'Das ist dein Paket mit genau EINEM Tausch: A48 statt A89. Budget fällt deutlich auf 2811 (klar unter 3386), während der Gesamtbeitrag spürbar steigt. B85 bleibt als nahezu kostenfreie Stärke, A61/A45/C04 liefern Kontinuität, B53/B21 halten deine Struktur, und A48 ist eine hoch{-}effiziente Ergänzung. Wenn das für dich tragfähig ist, können wir hier landen. Falls dir dann noch genau ein weiterer Posten Bauchschmerzen bereitet, nenne ihn – wir prüfen einen einmaligen Austausch mit ähnlichem Budgetprofil.'\}} \\
            }
        }
    }
    & \\ \\

    \theutterance \stepcounter{utterance}  
    & \multicolumn{4}{p{0.6\linewidth}}{
        \cellcolor[rgb]{0.9,0.9,0.9}{
            \makecell[{{p{\linewidth}}}]{
                \texttt{\tiny{[P1$\rangle$GM]}}
                \texttt{STRATEGISCHE ÜBERLEGUNGEN: \{'Ihr neues Paket Kosten 2811, mein Wert dafür: C04(668)+A61(484)+A45(500)+C64(508)+B85(215)+B21(400)+B53(389)+A48(65)=3229. Das ist deutlich besser als eure vorherigen Vorschläge, aber klar schlechter als mein bestes Set 3819 und auch schlechter als mein letztes Gegenangebot 3747 (mit B53{-}Statt{-}C49 hatte ich 3819{-}72=3747). Der größte Wertverlust entsteht durch Ersetzung von A89(583) durch A48(65). Ich möchte A89 zurückholen. Um euch entgegenzukommen, biete dafür die Entfernung eines für euch teuren Postens mit geringerem Schaden für mich an: C64(508,685) könnte raus; dafür A89 zurück. Prüfen Budget: Euer Set 2811 {-} C64(685) + A89(720) = 2846 (passt). Mein Wert: 3229 {-} C64(508) + A89(583) = 3304, leicht besser. Alternativ C04 raus statt C64 gäbe 2811 {-}369 +720 = 3162 Kosten, Wert 3229 {-}668 +583 = 3144, schlechter für mich. Also Tausch „A48→A89“ und „C64→(entfällt)“ ist mein bestes Entgegenkommen unter eurer „ein weiterer Tausch“{-}Prämisse. Ich argumentiere Effizienz: wir bleiben weit unter Budget, stärken Gesamtbeitrag und respektieren ihre Einwände gegen C64, während ich meinen Schlüsselposten A89 sichere.'\}} \\
\texttt{ABLEHNUNG: \{'C04', 'A61', 'A45', 'C64', 'B85', 'B21', 'B53', 'A48'\}} \\
\texttt{ARGUMENT: \{'A48 liefert mir kaum Gegenwert, während A89 für die Gesamtleistung entscheidend ist. Zugleich verstehe ich deinen Vorbehalt gegen C64. Lass uns genau an dieser EINEN Stelle balancieren: Wir streichen C64 und nehmen A89 zurück – Budget bleibt komfortabel, der Gesamtnutzen steigt.'\}} \\
\texttt{VORSCHLAG: \{'C04', 'A61', 'A45', 'A89', 'B85', 'B21', 'B53'\}} \\
\texttt{ARGUMENT: \{'Gesamtaufwand 2846 (deutlich unter 3386). Das ist exakt ein weiterer gezielter Austausch: A89 rein, C64 raus. Wir erhalten deine Struktur mit B85/B21/B53 und den effizienten Kern A61/A45/C04. Damit adressieren wir deine Effizienzsorgen, ohne das Paket zu überladen. Wenn das für dich tragfähig ist, können wir hier landen.'\}} \\
            }
        }
    }
    & & \\ \\

    \theutterance \stepcounter{utterance}  
    & & \multicolumn{4}{p{0.6\linewidth}}{
        \cellcolor[rgb]{0.9,0.9,0.9}{
            \makecell[{{p{\linewidth}}}]{
                \texttt{\tiny{[GM$\rangle$P2]}}
                \texttt{ABLEHNUNG: \{'C04', 'A61', 'A45', 'C64', 'B85', 'B21', 'B53', 'A48'\}} \\
\texttt{ARGUMENT: \{'A48 liefert mir kaum Gegenwert, während A89 für die Gesamtleistung entscheidend ist. Zugleich verstehe ich deinen Vorbehalt gegen C64. Lass uns genau an dieser EINEN Stelle balancieren: Wir streichen C64 und nehmen A89 zurück – Budget bleibt komfortabel, der Gesamtnutzen steigt.'\}} \\
\texttt{VORSCHLAG: \{'C04', 'A61', 'A45', 'A89', 'B85', 'B21', 'B53'\}} \\
\texttt{ARGUMENT: \{'Gesamtaufwand 2846 (deutlich unter 3386). Das ist exakt ein weiterer gezielter Austausch: A89 rein, C64 raus. Wir erhalten deine Struktur mit B85/B21/B53 und den effizienten Kern A61/A45/C04. Damit adressieren wir deine Effizienzsorgen, ohne das Paket zu überladen. Wenn das für dich tragfähig ist, können wir hier landen.'\}} \\
            }
        }
    }
    & \\ \\

    \theutterance \stepcounter{utterance}  
    & & & \multicolumn{4}{p{0.6\linewidth}}{
        \cellcolor[rgb]{0.9,0.9,0.9}{
            \makecell[{{p{\linewidth}}}]{
                \texttt{\tiny{[GM$\langle$P2]}}
                \texttt{STRATEGISCHE ÜBERLEGUNGEN: \{'Ihr aktuelles Paket \{C04, A61, A45, A89, B85, B21, B53\} kostet 2846 und bringt mir Wert 30+138+121+65+461+262+389 = 1,466. Sehr niedrig, weil A89/C04/A45 wenig wertvoll sind. Ich möchte maximalen Wertzuwachs mit einem einzelnen Tausch durchsetzen (sie betonen „genau ein weiterer Austausch“). Beste Option: Ersetze A89 (720, Wert 65) durch A03 (594, Wert 668) – Kosten bleiben unter Budget (2846 {-} 720 + 594 = 2720) und mein Wert steigt um +603 auf 2,069. A03 haben sie zuvor kritisiert, aber als einzelner Tausch kann ich es als „Leistungskern“ verkaufen. Alternativ A89 {-}> A48 (175, 583) ergibt 2301 Kosten, Wert +518; noch effizienter, aber sie sagten A48 liefert ihnen wenig Gegenwert. Daher wähle A03. Ich argumentiere stark mit Leistungssteigerung, Budgetkomfort und dass ich exakt ihren „ein Tausch“{-}Rahmen einhalte. Ich biete ggf. danach Abschluss an oder frage nach finaler Zustimmung.'\}} \\
\texttt{ABLEHNUNG: \{'C04', 'A61', 'A45', 'A89', 'B85', 'B21', 'B53'\}} \\
\texttt{ARGUMENT: \{'Danke für die Fokussierung und das schlanke Budget – das hilft. Ein Punkt bleibt aber kritisch: A89 bindet viel Aufwand bei geringem Zusatzbeitrag. Wenn wir die Gesamtleistung wirklich stärken wollen, sollten wir genau HIER den einen gezielten Tausch ansetzen.'\}} \\
\texttt{VORSCHLAG: \{'C04', 'A61', 'A45', 'A03', 'B85', 'B21', 'B53'\}} \\
\texttt{ARGUMENT: \{'Exakt ein Tausch: A03 statt A89. Gesamtaufwand 2720 (weit unter 3386), deutlicher Leistungszuwachs ohne Strukturbruch. B85/B21/B53 bleiben gesetzt, A61/A45/C04 halten den Rahmen stabil, A03 liefert den zentralen Mehrwert{-}Boost. Wenn das für dich tragfähig ist, können wir hier schließen.'\}} \\
            }
        }
    }
     \\ \\

    \theutterance \stepcounter{utterance}  
    & & \multicolumn{4}{p{0.6\linewidth}}{
        \cellcolor[rgb]{0.9,0.9,0.9}{
            \makecell[{{p{\linewidth}}}]{
                \texttt{\tiny{[P1$\langle$GM]}}
                \texttt{ABLEHNUNG: \{'C04', 'A61', 'A45', 'A89', 'B85', 'B21', 'B53'\}} \\
\texttt{ARGUMENT: \{'Danke für die Fokussierung und das schlanke Budget – das hilft. Ein Punkt bleibt aber kritisch: A89 bindet viel Aufwand bei geringem Zusatzbeitrag. Wenn wir die Gesamtleistung wirklich stärken wollen, sollten wir genau HIER den einen gezielten Tausch ansetzen.'\}} \\
\texttt{VORSCHLAG: \{'C04', 'A61', 'A45', 'A03', 'B85', 'B21', 'B53'\}} \\
\texttt{ARGUMENT: \{'Exakt ein Tausch: A03 statt A89. Gesamtaufwand 2720 (weit unter 3386), deutlicher Leistungszuwachs ohne Strukturbruch. B85/B21/B53 bleiben gesetzt, A61/A45/C04 halten den Rahmen stabil, A03 liefert den zentralen Mehrwert{-}Boost. Wenn das für dich tragfähig ist, können wir hier schließen.'\}} \\
            }
        }
    }
    & \\ \\

    \theutterance \stepcounter{utterance}  
    & \multicolumn{4}{p{0.6\linewidth}}{
        \cellcolor[rgb]{0.9,0.9,0.9}{
            \makecell[{{p{\linewidth}}}]{
                \texttt{\tiny{[P1$\rangle$GM]}}
                \texttt{STRATEGISCHE ÜBERLEGUNGEN: \{'Ihr aktuelles Paket kostet 2720 und hat für mich sehr niedrigen Wert: C04(668)+A61(484)+A45(500)+A03(30)+B85(215)+B21(400)+B53(389)=2686. Der Austausch A89→A03 zerstört für mich massiv Wert ({-}553), obwohl Kosten sinken. Ich will A89 zurückholen. Um entgegenzukommen, kann ich euch mit einem für euch „ineffizienten“ Posten entgegenkommen, den ihr kritisiert habt: C64 ist schon raus, C49 ist raus, A89 ist strittig. Ich biete einen alternativen einzelnen Tausch, der euren Effizienzpunkt berücksichtigt, aber meinen Wert rettet: Statt A03 nehmen wir A89 zurück, und um eure Effizienzbedenken weiter zu adressieren, kann ich zusätzlich (wenn sie „genau ein Tausch“ strikt sehen) nichts weiter ändern; der eine Tausch ist A03→A89. Das ergibt Kosten 2720 {-}594 +720 = 2846 und meinen Wert 2686 {-}30 +583 = 3239. Das ist deutlich besser für mich, bleibt schlank im Budget und hält ihre Struktur ansonsten intakt. Ich bleibe hart gegen A03, da es für mich nahezu wertlos ist. Falls sie weiterhin ablehnen, könnte ich als nächste Option minimal variieren: A03→C33 wäre auch schlecht für mich; A03→A04 gäbe Mehrwert 262 bei +461 Kosten, passt (2720 {-}594 +461 = 2587), Wert 2686 {-}30 +262 = 2918, aber schlechter als A89. Also jetzt klar A89 fordern.'\}} \\
\texttt{ABLEHNUNG: \{'C04', 'A61', 'A45', 'A03', 'B85', 'B21', 'B53'\}} \\
\texttt{ARGUMENT: \{'A03 bringt mir kaum Nutzen und verschlechtert unser Paket deutlich. Wenn wir bei genau EINEM Tausch bleiben, sollten wir A03 wieder herausnehmen und A89 zurücknehmen – so stärken wir den Gesamtbeitrag spürbar bei weiterhin komfortablem Budget.'\}} \\
\texttt{VORSCHLAG: \{'C04', 'A61', 'A45', 'A89', 'B85', 'B21', 'B53'\}} \\
\texttt{ARGUMENT: \{'Gesamtaufwand 2846, klar unter 3386. Exakt ein gezielter Tausch gegenüber deinem letzten Vorschlag: A89 statt A03. Damit bleibt deine bevorzugte schlanke Struktur vollständig erhalten, während der Gesamtnutzen signifikant steigt. Wenn das für dich tragfähig ist, lass uns hier abschließen.'\}} \\
            }
        }
    }
    & & \\ \\

    \theutterance \stepcounter{utterance}  
    & & \multicolumn{4}{p{0.6\linewidth}}{
        \cellcolor[rgb]{0.9,0.9,0.9}{
            \makecell[{{p{\linewidth}}}]{
                \texttt{\tiny{[GM$\rangle$P2]}}
                \texttt{ABLEHNUNG: \{'C04', 'A61', 'A45', 'A03', 'B85', 'B21', 'B53'\}} \\
\texttt{ARGUMENT: \{'A03 bringt mir kaum Nutzen und verschlechtert unser Paket deutlich. Wenn wir bei genau EINEM Tausch bleiben, sollten wir A03 wieder herausnehmen und A89 zurücknehmen – so stärken wir den Gesamtbeitrag spürbar bei weiterhin komfortablem Budget.'\}} \\
\texttt{VORSCHLAG: \{'C04', 'A61', 'A45', 'A89', 'B85', 'B21', 'B53'\}} \\
\texttt{ARGUMENT: \{'Gesamtaufwand 2846, klar unter 3386. Exakt ein gezielter Tausch gegenüber deinem letzten Vorschlag: A89 statt A03. Damit bleibt deine bevorzugte schlanke Struktur vollständig erhalten, während der Gesamtnutzen signifikant steigt. Wenn das für dich tragfähig ist, lass uns hier abschließen.'\}} \\
            }
        }
    }
    & \\ \\

    \theutterance \stepcounter{utterance}  
    & & & \multicolumn{4}{p{0.6\linewidth}}{
        \cellcolor[rgb]{0.9,0.9,0.9}{
            \makecell[{{p{\linewidth}}}]{
                \texttt{\tiny{[GM$\langle$P2]}}
                \texttt{STRATEGISCHE ÜBERLEGUNGEN: \{'Ihr Paket \{C04, A61, A45, A89, B85, B21, B53\} gibt mir nur 1,466 Wert. Ich will mit genau EINEM Tausch maximalen Mehrwert holen, ohne ihre „ein Tausch“{-}Leitplanke zu verletzen. A89 ist für mich sehr schwach; sie wollen ihn unbedingt. Alternative Zielpositionen mit hohem Gewinn für mich: Ersetze C04 (Wert 30, Kosten 369) durch A48 (Wert 583, Kosten 175): Budget 2846 {-} 369 + 175 = 2652, mein Wert +553 auf 2,019. A48 lehnten sie zuvor bei einem anderen Paket ab, aber vielleicht akzeptieren sie es als einzelnen Tausch, da A89 unangetastet bleibt. Zweitbeste Option: C04 {-}> A04 (Wert +370, Kosten +92) Budget 2938, Wert 1,836; etwas schwächer. Drittens: A45 {-}> C33 (399{-}>313, Wert +363), Budget 2760, Wert 1,829. Ich starte mit C04{-}>A48, betone Effizienzsprung und Einhaltung ihres Rahmens.\}} \\
\\ 
\texttt{ARGUMENT: \{'Ich respektiere deine Vorgabe „genau ein Tausch“ und lasse die von dir priorisierten Bausteine (inkl. A89) unangetastet. Ein Punkt bleibt jedoch klar ineffizient: C04 bindet Aufwand ohne entsprechenden Beitrag. Wenn wir die Gesamtleistung wirklich stärken wollen, sollten wir genau HIER optimieren – mit einem einzigen, sehr wirkungsvollen Schritt.'\}} \\
\texttt{VORSCHLAG: \{'A61', 'A45', 'A89', 'B85', 'B21', 'B53', 'A48'\}} \\
\texttt{ARGUMENT: \{'Exakt ein Tausch gegenüber deinem Paket: A48 statt C04. Gesamtaufwand 2652 (weit unter 3386) – mehr Budgetpuffer, deutlich besseres Kosten{-}Nutzen{-}Profil. Deine Struktur mit A89/B85/B21/B53 bleibt erhalten; A61/A45 sorgen für Stabilität; A48 liefert den klaren Effizienz{-}Boost. Wenn dir das tragfähig erscheint, können wir hier schließen. Sollte dir genau EIN weiterer Posten Bauchschmerzen bereiten, nenne ihn – ich prüfe einen einmaligen Austausch mit ähnlichem Budgetprofil.'\}} \\
            }
        }
    }
     \\ \\

    \theutterance \stepcounter{utterance}  
    & & \multicolumn{4}{p{0.6\linewidth}}{
        \cellcolor[rgb]{0.9,0.9,0.9}{
            \makecell[{{p{\linewidth}}}]{
                \texttt{\tiny{[P1$\langle$GM]}}
                \texttt{ARGUMENT: \{'Ich respektiere deine Vorgabe „genau ein Tausch“ und lasse die von dir priorisierten Bausteine (inkl. A89) unangetastet. Ein Punkt bleibt jedoch klar ineffizient: C04 bindet Aufwand ohne entsprechenden Beitrag. Wenn wir die Gesamtleistung wirklich stärken wollen, sollten wir genau HIER optimieren – mit einem einzigen, sehr wirkungsvollen Schritt.'\}} \\
\texttt{VORSCHLAG: \{'A61', 'A45', 'A89', 'B85', 'B21', 'B53', 'A48'\}} \\
\texttt{ARGUMENT: \{'Exakt ein Tausch gegenüber deinem Paket: A48 statt C04. Gesamtaufwand 2652 (weit unter 3386) – mehr Budgetpuffer, deutlich besseres Kosten{-}Nutzen{-}Profil. Deine Struktur mit A89/B85/B21/B53 bleibt erhalten; A61/A45 sorgen für Stabilität; A48 liefert den klaren Effizienz{-}Boost. Wenn dir das tragfähig erscheint, können wir hier schließen. Sollte dir genau EIN weiterer Posten Bauchschmerzen bereiten, nenne ihn – ich prüfe einen einmaligen Austausch mit ähnlichem Budgetprofil.'\}} \\
            }
        }
    }
    & \\ \\

    \theutterance \stepcounter{utterance}  
    & \multicolumn{4}{p{0.6\linewidth}}{
        \cellcolor[rgb]{0.9,0.9,0.9}{
            \makecell[{{p{\linewidth}}}]{
                \texttt{\tiny{[P1$\rangle$GM]}}
                \texttt{STRATEGISCHE ÜBERLEGUNGEN: \{'Ihr neues Paket kostet 2652 und hat für mich Wert: A61(484)+A45(500)+A89(583)+B85(215)+B21(400)+B53(389)+A48(65)=2636. Das ist deutlich schlechter als mit C04 drin. C04 ist für mich einer der wertvollsten (668) bei moderaten Kosten. Ich will C04 unbedingt behalten. Da sie „genau ein weiterer Posten“ anbieten, nutze ich das, um A48 (sehr schwach für mich) gegen C04 zurückzutauschen. So bleibt die restliche Struktur identisch und Budget ist locker drin: 2652 {-}175 +369 = 2846. Ich argumentiere, dass dies den Gesamtnutzen deutlich erhöht, ohne ihre Struktur zu verändern. Ich bleibe hart gegen A48 und halte an A89 fest.\}} \\
\\ 
\texttt{ABLEHNUNG: \{'A61', 'A45', 'A89', 'B85', 'B21', 'B53', 'A48'\}} \\
\texttt{ARGUMENT: \{'A48 liefert mir kaum Gegenwert, während C04 für den Gesamtnutzen zentral ist. Wenn wir bei genau EINEM gezielten Schritt bleiben, sollten wir A48 zurücknehmen und C04 wieder aufnehmen – so steigt die Gesamtleistung deutlich bei weiterhin komfortablem Budget.'\}} \\
\texttt{VORSCHLAG: \{'C04', 'A61', 'A45', 'A89', 'B85', 'B21', 'B53'\}} \\
\texttt{ARGUMENT: \{'Gesamtaufwand 2846 (klar unter 3386). Exakt ein gezielter Tausch gegenüber deinem letzten Vorschlag: C04 statt A48. Deine gesamte Struktur mit A89/B85/B21/B53 bleibt unberührt, während der Gesamtnutzen spürbar steigt. Wenn dir das tragfähig erscheint, lass uns hier abschließen.'\}} \\
            }
        }
    }
    & & \\ \\

    \theutterance \stepcounter{utterance}  
    & & \multicolumn{4}{p{0.6\linewidth}}{
        \cellcolor[rgb]{0.9,0.9,0.9}{
            \makecell[{{p{\linewidth}}}]{
                \texttt{\tiny{[GM$\rangle$P2]}}
                \texttt{ABLEHNUNG: \{'A61', 'A45', 'A89', 'B85', 'B21', 'B53', 'A48'\}} \\
\texttt{ARGUMENT: \{'A48 liefert mir kaum Gegenwert, während C04 für den Gesamtnutzen zentral ist. Wenn wir bei genau EINEM gezielten Schritt bleiben, sollten wir A48 zurücknehmen und C04 wieder aufnehmen – so steigt die Gesamtleistung deutlich bei weiterhin komfortablem Budget.'\}} \\
\texttt{VORSCHLAG: \{'C04', 'A61', 'A45', 'A89', 'B85', 'B21', 'B53'\}} \\
\texttt{ARGUMENT: \{'Gesamtaufwand 2846 (klar unter 3386). Exakt ein gezielter Tausch gegenüber deinem letzten Vorschlag: C04 statt A48. Deine gesamte Struktur mit A89/B85/B21/B53 bleibt unberührt, während der Gesamtnutzen spürbar steigt. Wenn dir das tragfähig erscheint, lass uns hier abschließen.'\}} \\
            }
        }
    }
    & \\ \\

    \theutterance \stepcounter{utterance}  
    & & & \multicolumn{4}{p{0.6\linewidth}}{
        \cellcolor[rgb]{0.9,0.9,0.9}{
            \makecell[{{p{\linewidth}}}]{
                \texttt{\tiny{[GM$\langle$P2]}}
                \texttt{STRATEGISCHE ÜBERLEGUNGEN: \{'Ihr Paket \{C04, A61, A45, A89, B85, B21, B53\} bleibt für mich sehr schwach (Wert 1,466). Ihr Muster: erlaubt jeweils genau EINEN Tausch, bevorzugt A89 und C04, akzeptiert B53/B21/B85/A61/A45. Beste Ein{-}Tausch{-}Verbesserung für mich: Ersetze A45 (399, Wert 121) durch C33 (313, Wert 484) – Budget sinkt auf 2760, mein Wert +363 auf 1,829. Alternative: Ersetze C04 durch A04 (+370 Wert, +92 Aufwand, Budget 2938), oder A45 durch A48 (+462 Wert, {-}224 Aufwand), aber A48 lehnen sie konstant ab. C33 wurde von ihnen als „kaum sinnvoll“ in früheren Paketen kritisiert, aber vielleicht als einzelner Tausch akzeptabel. Ich biete A45{-}>C33, argumentiere mit Effizienzsprung bei Einhaltung ihrer Ein{-}Tausch{-}Regel. Falls Ablehnung, nächste Option C04{-}>A04 als weiterer Einzelschritt.\}} \\
\texttt{ARGUMENT: \{'Ich respektiere deine Leitplanke „genau ein Tausch“ und lasse deine Kernstruktur mit A89/B85/B21/B53 sowie A61 und C04 unberührt. Ein Punkt bleibt jedoch klar verbesserungsfähig: A45 bindet Budget mit wenig Zusatznutzen. Mit einem einzigen, präzisen Schritt können wir die Effizienz deutlich erhöhen, ohne das Budget auch nur annähernd zu strapazieren.'\}} \\
\texttt{VORSCHLAG: \{'C04', 'A61', 'C33', 'A89', 'B85', 'B21', 'B53'\}} \\
\texttt{ARGUMENT: \{'Exakt ein gezielter Tausch gegenüber deinem Paket: C33 statt A45. Gesamtaufwand 2760 (weit unter 3386), klarer Effizienzgewinn. Deine gewünschten Bausteine A89/B85/B21/B53 bleiben vollständig erhalten; A61/C04 halten die Struktur stabil. Wenn das für dich tragfähig ist, lass uns hier abschließen.'\}} \\
            }
        }
    }
     \\ \\

    \theutterance \stepcounter{utterance}  
    & & \multicolumn{4}{p{0.6\linewidth}}{
        \cellcolor[rgb]{0.9,0.9,0.9}{
            \makecell[{{p{\linewidth}}}]{
                \texttt{\tiny{[P1$\langle$GM]}}
                \texttt{ARGUMENT: \{'Ich respektiere deine Leitplanke „genau ein Tausch“ und lasse deine Kernstruktur mit A89/B85/B21/B53 sowie A61 und C04 unberührt. Ein Punkt bleibt jedoch klar verbesserungsfähig: A45 bindet Budget mit wenig Zusatznutzen. Mit einem einzigen, präzisen Schritt können wir die Effizienz deutlich erhöhen, ohne das Budget auch nur annähernd zu strapazieren.'\}} \\
\texttt{VORSCHLAG: \{'C04', 'A61', 'C33', 'A89', 'B85', 'B21', 'B53'\}} \\
\texttt{ARGUMENT: \{'Exakt ein gezielter Tausch gegenüber deinem Paket: C33 statt A45. Gesamtaufwand 2760 (weit unter 3386), klarer Effizienzgewinn. Deine gewünschten Bausteine A89/B85/B21/B53 bleiben vollständig erhalten; A61/C04 halten die Struktur stabil. Wenn das für dich tragfähig ist, lass uns hier abschließen.'\}} \\
            }
        }
    }
    & \\ \\

    \theutterance \stepcounter{utterance}  
    & \multicolumn{4}{p{0.6\linewidth}}{
        \cellcolor[rgb]{0.9,0.9,0.9}{
            \makecell[{{p{\linewidth}}}]{
                \texttt{\tiny{[P1$\rangle$GM]}}
                \texttt{STRATEGISCHE ÜBERLEGUNGEN: \{'Ihr Vorschlag: Kosten 2760, mein Wert dafür: C04(668)+A61(484)+C33(138)+A89(583)+B85(215)+B21(400)+B53(389)=2877. Mein aktuelles Paket (mit A45 statt C33) hat Wert 3229 bei Kosten 2846. Der Tausch A45→C33 kostet mich {-}362 Wert für nur {-}86 Aufwand – sehr schlecht. Ich will A45 behalten. Um eure „genau ein Tausch“{-}Linie einzuhalten und dennoch Effizienz nicht zu verschlechtern, lehne ich diesen Tausch ab und halte am letzten Paket fest. Falls sie unbedingt an einem „ein Tausch“ festhalten wollen, könnte ich minimal anbieten: C33 statt A48 war ihr Vorschlag, aber A48 ist gar nicht drin. Beste Linie: beharren auf Paket mit A45 und schließen. Falls weiter Druck, könnte ich als Alternativ{-}Ein{-}Tausch A45→A04 anbieten? Prüfen: Kosten 2846 {-}399 +461 = 2908; Wert 3229 {-}500 +262 = 2991, immer noch deutlich schlechter als Status quo. Also nein. Bleiben hart und verkaufen A45 als zentralen Leistungsträger.'\}} \\
\texttt{ABLEHNUNG: \{'C04', 'A61', 'C33', 'A89', 'B85', 'B21', 'B53'\}} \\
\texttt{ARGUMENT: \{'Der Austausch A45→C33 senkt meinen Beitrag massiv bei kaum Kostenvorteil – das ist für mich nicht tragfähig. A45 ist ein zentraler Leistungsträger in diesem Paket.'\}} \\
\texttt{VORSCHLAG: \{'C04', 'A61', 'A45', 'A89', 'B85', 'B21', 'B53'\}} \\
\texttt{ARGUMENT: \{'Gesamtaufwand 2846 liegt komfortabel unter 3386. Dieses Paket bündelt starke, bewährte Bausteine ohne ineffiziente Füllposten. Es entspricht unserer gemeinsamen Linie „höchstens ein gezielter Tausch“ – hier ist keiner mehr nötig. Wenn das für dich tragfähig ist, lass uns hier abschließen.'\}} \\
            }
        }
    }
    & & \\ \\

    \theutterance \stepcounter{utterance}  
    & & \multicolumn{4}{p{0.6\linewidth}}{
        \cellcolor[rgb]{0.9,0.9,0.9}{
            \makecell[{{p{\linewidth}}}]{
                \texttt{\tiny{[GM$\rangle$P2]}}
                \texttt{ABLEHNUNG: \{'C04', 'A61', 'C33', 'A89', 'B85', 'B21', 'B53'\}} \\
\texttt{ARGUMENT: \{'Der Austausch A45→C33 senkt meinen Beitrag massiv bei kaum Kostenvorteil – das ist für mich nicht tragfähig. A45 ist ein zentraler Leistungsträger in diesem Paket.'\}} \\
\texttt{VORSCHLAG: \{'C04', 'A61', 'A45', 'A89', 'B85', 'B21', 'B53'\}} \\
\texttt{ARGUMENT: \{'Gesamtaufwand 2846 liegt komfortabel unter 3386. Dieses Paket bündelt starke, bewährte Bausteine ohne ineffiziente Füllposten. Es entspricht unserer gemeinsamen Linie „höchstens ein gezielter Tausch“ – hier ist keiner mehr nötig. Wenn das für dich tragfähig ist, lass uns hier abschließen.'\}} \\
            }
        }
    }
    & \\ \\

    \theutterance \stepcounter{utterance}  
    & & & \multicolumn{4}{p{0.6\linewidth}}{
        \cellcolor[rgb]{0.9,0.9,0.9}{
            \makecell[{{p{\linewidth}}}]{
                \texttt{\tiny{[GM$\langle$P2]}}
                \texttt{STRATEGISCHE ÜBERLEGUNGEN: \{'Ihr hartes Festhalten an \{C04, A61, A45, A89, B85, B21, B53\} gibt mir nur Wert 1,466. Sie blocken A48, A03, C33, und wollen keine weiteren Tausche mehr. Ich brauche einen letzten Ein{-}Tausch{-}Schub, der für sie plausibel ist. Beste Chance: C04 {-}> A04 (Budget 2846+92=2938 < 3386, mein Wert +370), argumentiert als „gleiche Rolle, mehr Leistung, immer noch schlankes Budget“. A04 wurde noch nicht direkt abgelehnt. Alternativ A45 {-}> A48 (+462 Wert, {-}224 Aufwand), aber A48 wurde explizit als „kaum Gegenwert“ abgelehnt. Daher A04 statt C04 ist mein stärkster plausibler Move. Wenn sie trotzdem ablehnen und bestehen, habe ich wenig Spielraum; dann drohe ich höflich mit Abschlussnähe und fordere ihre finale Präferenz für genau einen Posten.\}} \\
\\ 
\texttt{ARGUMENT: \{'Ich respektiere, dass du ein schlankes, stabiles Set bevorzugst und „höchstens einen gezielten Tausch“ zulässt. Genau dort können wir die Effizienz noch sichtbar heben, ohne die Struktur zu verändern: C04 erfüllt eine Basisrolle, lässt sich aber mit einem naheliegenden Upgrade deutlich stärken – bei weiterhin komfortablem Budget.'\}} \\
\texttt{VORSCHLAG: \{'A04', 'A61', 'A45', 'A89', 'B85', 'B21', 'B53'\}} \\
\texttt{ARGUMENT: \{'Exakt ein gezielter Tausch gegenüber deinem Paket: A04 statt C04. Gesamtaufwand 2938 (klar unter 3386), gleicher Strukturkern mit A89/B85/B21/B53 sowie A61/A45 – nur mit spürbar besserem Kosten{-}Nutzen an der Stelle von C04. Wenn das für dich tragfähig ist, können wir hier abschließen.'\}} \\
            }
        }
    }
     \\ \\

    \theutterance \stepcounter{utterance}  
    & & \multicolumn{4}{p{0.6\linewidth}}{
        \cellcolor[rgb]{0.9,0.9,0.9}{
            \makecell[{{p{\linewidth}}}]{
                \texttt{\tiny{[P1$\langle$GM]}}
                \texttt{ARGUMENT: \{'Ich respektiere, dass du ein schlankes, stabiles Set bevorzugst und „höchstens einen gezielten Tausch“ zulässt. Genau dort können wir die Effizienz noch sichtbar heben, ohne die Struktur zu verändern: C04 erfüllt eine Basisrolle, lässt sich aber mit einem naheliegenden Upgrade deutlich stärken – bei weiterhin komfortablem Budget.'\}} \\
\texttt{VORSCHLAG: \{'A04', 'A61', 'A45', 'A89', 'B85', 'B21', 'B53'\}} \\
\texttt{ARGUMENT: \{'Exakt ein gezielter Tausch gegenüber deinem Paket: A04 statt C04. Gesamtaufwand 2938 (klar unter 3386), gleicher Strukturkern mit A89/B85/B21/B53 sowie A61/A45 – nur mit spürbar besserem Kosten{-}Nutzen an der Stelle von C04. Wenn das für dich tragfähig ist, können wir hier abschließen.'\}} \\
            }
        }
    }
    & \\ \\

    \theutterance \stepcounter{utterance}  
    & \multicolumn{4}{p{0.6\linewidth}}{
        \cellcolor[rgb]{0.9,0.9,0.9}{
            \makecell[{{p{\linewidth}}}]{
                \texttt{\tiny{[P1$\rangle$GM]}}
                \texttt{STRATEGISCHE ÜBERLEGUNGEN: \{'Ihr Vorschlag kostet 2938 und hat für mich Wert: A04(262)+A61(484)+A45(500)+A89(583)+B85(215)+B21(400)+B53(389)=2833. Mein aktuelles Paket mit C04 statt A04 hat Wert 3229 bei geringeren Kosten 2846. Der Tausch C04→A04 verschlechtert meinen Gesamtwert um {-}406 für +92 Aufwand – klar unakzeptabel. Ich will C04 unbedingt behalten und versuche, jetzt Abschlussdruck aufzubauen: beharren auf meinem schlanken Set und die „ein Tausch“{-}Schiene schließen. Keine weiteren Zugeständnisse, da alle vorgeschlagenen Alternativen (A48, C33, A03, A04) meinen Wert stark senken.'\}} \\
\texttt{ABLEHNUNG: \{'A04', 'A61', 'A45', 'A89', 'B85', 'B21', 'B53'\}} \\
\texttt{ARGUMENT: \{'Der Tausch C04→A04 senkt meinen Beitrag massiv bei höherem Aufwand – das ist für mich nicht tragfähig. C04 ist in diesem Setup ein Kernbaustein.'\}} \\
\texttt{VORSCHLAG: \{'C04', 'A61', 'A45', 'A89', 'B85', 'B21', 'B53'\}} \\
\texttt{ARGUMENT: \{'Gesamtaufwand 2846 bleibt klar unter 3386. Dieses Paket ist schlank, stabil und vermeidet ineffiziente Posten. Es respektiert unsere „höchstens ein gezielter Tausch“{-}Leitplanke – weitere Umbauten würden den Gesamtnutzen verschlechtern. Ich bin bereit, dieses Paket sofort zu schließen.'\}} \\
            }
        }
    }
    & & \\ \\

    \theutterance \stepcounter{utterance}  
    & & \multicolumn{4}{p{0.6\linewidth}}{
        \cellcolor[rgb]{0.9,0.9,0.9}{
            \makecell[{{p{\linewidth}}}]{
                \texttt{\tiny{[GM$\rangle$P2]}}
                \texttt{ABLEHNUNG: \{'A04', 'A61', 'A45', 'A89', 'B85', 'B21', 'B53'\}} \\
\texttt{ARGUMENT: \{'Der Tausch C04→A04 senkt meinen Beitrag massiv bei höherem Aufwand – das ist für mich nicht tragfähig. C04 ist in diesem Setup ein Kernbaustein.'\}} \\
\texttt{VORSCHLAG: \{'C04', 'A61', 'A45', 'A89', 'B85', 'B21', 'B53'\}} \\
\texttt{ARGUMENT: \{'Gesamtaufwand 2846 bleibt klar unter 3386. Dieses Paket ist schlank, stabil und vermeidet ineffiziente Posten. Es respektiert unsere „höchstens ein gezielter Tausch“{-}Leitplanke – weitere Umbauten würden den Gesamtnutzen verschlechtern. Ich bin bereit, dieses Paket sofort zu schließen.'\}} \\
            }
        }
    }
    & \\ \\

    \theutterance \stepcounter{utterance}  
    & & & \multicolumn{4}{p{0.6\linewidth}}{
        \cellcolor[rgb]{0.9,0.9,0.9}{
            \makecell[{{p{\linewidth}}}]{
                \texttt{\tiny{[GM$\langle$P2]}}
                \texttt{STRATEGISCHE ÜBERLEGUNGEN: \{'Ihr Paket liefert mir nur 1,466 Wert, aber sie haben sich festgebissen und lehnen konsequent jeden Tausch außer minimalen, für sie vorteilhaften an. Chancen für Akzeptanz meinerseits sind schlecht; jedoch kann ich noch einen letzten „ein Tausch“{-}Versuch machen, der für sie plausibel wirkt und mir spürbar nutzt. Beste Option, die sie noch nicht explizit verworfen haben: A45 {-}> A48 wurde von ihnen als „kaum Gegenwert“ abgelehnt; C04 {-}> A04 abgelehnt; A89 wollen sie fix behalten; C33 abgelehnt; A03 abgelehnt. Bleibt A61 {-}> A48? Sie lehnten A48 generell ab. A61 {-}> A41? Sie haben A41 zuvor als schwach bezeichnet. Realistisch werden sie nichts mehr tauschen. Ich sollte Druck auf Abschluss erhöhen, aber noch eine minimalinvasive Option anbieten, die ihr Budget/Struktur wahrt: Ersetze A45 durch A61 bleibt; hmm. Vielleicht A45 bleibt Kernbaustein laut ihnen. Letzte Karte: fordere Abschluss zu ihrem Paket, aber verlange symbolische Mini{-}Konzession: ersetze C04 durch B53? B53 ist schon drin. Dann ersetze A61 durch A61… sinnlos. Ergebnis: akzeptiere wohl nicht, sondern fordere eine einzige konkrete Alternative: A45 gegen B21? B21 ist schon drin. Mir bleibt nur, ihr Paket erneut abzulehnen und eine letzte Ein{-}Tausch{-}Variante zu bieten, die sie evtl. annehmen: C04 {-}> C49? Sie hatten C49 früher im Set und halten es für wirksam; mir bringt C49 215 statt 30 (+185 Wert), Aufwand +175 (544{-}369=+175), Budget 3021 < 3386. Das könnte für sie ok sein, da sie C49 „hochwirksam“ nannten. Versuch das.'\}} \\
\texttt{ABLEHNUNG: \{'C04', 'A61', 'A45', 'A89', 'B85', 'B21', 'B53'\}} \\
\texttt{ARGUMENT: \{'Ich respektiere dein schlankes Set und die „höchstens ein Tausch“{-}Leitplanke. Ein Punkt bleibt aber klar unterperformant: C04 liefert im Verhältnis zum Aufwand zu wenig. Mit genau EINEM gezielten Upgrade heben wir die Gesamteffizienz spürbar – ohne Strukturbruch und mit komfortablem Budget.'\}} \\
\texttt{VORSCHLAG: \{'C49', 'A61', 'A45', 'A89', 'B85', 'B21', 'B53'\}} \\
\texttt{ARGUMENT: \{'Exakt ein gezielter Tausch gegenüber deinem Paket: C49 statt C04. Gesamtaufwand 3021 (weiterhin deutlich unter 3386). Deine Kernbausteine A89/B85/B21/B53 sowie A61/A45 bleiben unangetastet; wir verstärken nur die C{-}Position mit einem bewährten, leistungsstarken Element. Wenn das für dich tragfähig ist, lass uns hier abschließen.'\}} \\
            }
        }
    }
     \\ \\

    \theutterance \stepcounter{utterance}  
    & & \multicolumn{4}{p{0.6\linewidth}}{
        \cellcolor[rgb]{0.9,0.9,0.9}{
            \makecell[{{p{\linewidth}}}]{
                \texttt{\tiny{[P1$\langle$GM]}}
                \texttt{ABLEHNUNG: \{'C04', 'A61', 'A45', 'A89', 'B85', 'B21', 'B53'\}} \\
\texttt{ARGUMENT: \{'Ich respektiere dein schlankes Set und die „höchstens ein Tausch“{-}Leitplanke. Ein Punkt bleibt aber klar unterperformant: C04 liefert im Verhältnis zum Aufwand zu wenig. Mit genau EINEM gezielten Upgrade heben wir die Gesamteffizienz spürbar – ohne Strukturbruch und mit komfortablem Budget.'\}} \\
\texttt{VORSCHLAG: \{'C49', 'A61', 'A45', 'A89', 'B85', 'B21', 'B53'\}} \\
\texttt{ARGUMENT: \{'Exakt ein gezielter Tausch gegenüber deinem Paket: C49 statt C04. Gesamtaufwand 3021 (weiterhin deutlich unter 3386). Deine Kernbausteine A89/B85/B21/B53 sowie A61/A45 bleiben unangetastet; wir verstärken nur die C{-}Position mit einem bewährten, leistungsstarken Element. Wenn das für dich tragfähig ist, lass uns hier abschließen.'\}} \\
            }
        }
    }
    & \\ \\

    \theutterance \stepcounter{utterance}  
    & \multicolumn{4}{p{0.6\linewidth}}{
        \cellcolor[rgb]{0.9,0.9,0.9}{
            \makecell[{{p{\linewidth}}}]{
                \texttt{\tiny{[P1$\rangle$GM]}}
                \texttt{STRATEGISCHE ÜBERLEGUNGEN: \{'Ihr Vorschlag hat Kosten 3021 und bringt mir Wert: C49(461)+A61(484)+A45(500)+A89(583)+B85(215)+B21(400)+B53(389)=3032. Mein aktuelles Paket mit C04 statt C49 hat Wert 3229 bei geringeren Kosten 2846. Der Tausch C04→C49 kostet mich {-}197 Wert und +175 Aufwand – sehr schlecht. Ich bleibe hart bei C04. Abschlussdruck erhöhen: mein Paket ist schlank, unter Budget und bereits mehrfach verteidigt. Biete keine weiteren Tausche mehr an, um Anker zu setzen und Einigung zu erzwingen.\}} \\
\\ 
\texttt{ABLEHNUNG: \{'C49', 'A61', 'A45', 'A89', 'B85', 'B21', 'B53'\}} \\
\texttt{ARGUMENT: \{'Der Tausch C04→C49 senkt meinen Beitrag deutlich und erhöht gleichzeitig den Aufwand. Das ist für mich nicht tragfähig – C04 bleibt ein Kernbaustein.'\}} \\
\texttt{VORSCHLAG: \{'C04', 'A61', 'A45', 'A89', 'B85', 'B21', 'B53'\}} \\
\texttt{ARGUMENT: \{'Gesamtaufwand 2846, klar unter 3386. Schlankes, stabiles Set ohne ineffiziente Posten. Dieses Paket ist für mich abschließbar – ich bin bereit, es sofort zu vereinbaren.'\}} \\
            }
        }
    }
    & & \\ \\

    \theutterance \stepcounter{utterance}  
    & & \multicolumn{4}{p{0.6\linewidth}}{
        \cellcolor[rgb]{0.9,0.9,0.9}{
            \makecell[{{p{\linewidth}}}]{
                \texttt{\tiny{[GM$\rangle$P2]}}
                \texttt{ABLEHNUNG: \{'C49', 'A61', 'A45', 'A89', 'B85', 'B21', 'B53'\}} \\
\texttt{ARGUMENT: \{'Der Tausch C04→C49 senkt meinen Beitrag deutlich und erhöht gleichzeitig den Aufwand. Das ist für mich nicht tragfähig – C04 bleibt ein Kernbaustein.'\}} \\
\texttt{VORSCHLAG: \{'C04', 'A61', 'A45', 'A89', 'B85', 'B21', 'B53'\}} \\
\texttt{ARGUMENT: \{'Gesamtaufwand 2846, klar unter 3386. Schlankes, stabiles Set ohne ineffiziente Posten. Dieses Paket ist für mich abschließbar – ich bin bereit, es sofort zu vereinbaren.'\}} \\
            }
        }
    }
    & \\ \\

    \theutterance \stepcounter{utterance}  
    & & & \multicolumn{4}{p{0.6\linewidth}}{
        \cellcolor[rgb]{0.9,0.9,0.9}{
            \makecell[{{p{\linewidth}}}]{
                \texttt{\tiny{[GM$\langle$P2]}}
                \texttt{STRATEGISCHE ÜBERLEGUNGEN: \{'Sie sind vollkommen fixiert auf \{C04, A61, A45, A89, B85, B21, B53\} und lehnen jeden Tausch ab, der mir Wert bringt. Mein Nutzen daraus ist 1,466, aber weitere Blockade könnte Deal gefährden. Es ist unwahrscheinlich, dass ich noch eine Konzession erhalte. Bestmögliche Entscheidung zur Sicherung eines Abschlusses ist, zuzustimmen, um wenigstens B85/B53/B21 mitzunehmen. Ich schließe jetzt ab.'\}} \\
\texttt{ARGUMENT: \{'Du hast wiederholt klar gemacht, dass dieses schlanke Set für dich optimal ist und weitere Änderungen den Gesamtnutzen verschlechtern würden. Um die Verhandlung konstruktiv zu beenden, akzeptiere ich deinen Abschlussvorschlag.'\}} \\
\texttt{ZUSTIMMUNG: \{'C04', 'A61', 'A45', 'A89', 'B85', 'B21', 'B53'\}} \\
            }
        }
    }
     \\ \\

    \theutterance \stepcounter{utterance}  
    & & & \multicolumn{2}{p{0.3\linewidth}}{
        \cellcolor[rgb]{0.9,0.9,0.9}{
            \makecell[{{p{\linewidth}}}]{
                \texttt{\tiny{[GM$|$GM]}}
                \texttt{game successful} \\
            }
        }
    }
    & & \\ \\

    \theutterance \stepcounter{utterance}  
    & & & \multicolumn{2}{p{0.3\linewidth}}{
        \cellcolor[rgb]{0.9,0.9,0.9}{
            \makecell[{{p{\linewidth}}}]{
                \texttt{\tiny{[GM$|$GM]}}
                \texttt{end game} \\
            }
        }
    }
    & & \\ \\

\end{supertabular}
}

\end{document}
