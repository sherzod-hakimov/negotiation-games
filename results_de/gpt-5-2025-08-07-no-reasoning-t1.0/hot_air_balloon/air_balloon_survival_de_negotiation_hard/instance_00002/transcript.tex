
\documentclass{article}
\usepackage{colortbl}
\usepackage{makecell}
\usepackage{multirow}
\usepackage{supertabular}

\begin{document}

\newcounter{utterance}

\twocolumn

{ \footnotesize  \setcounter{utterance}{1}
\setlength{\tabcolsep}{0pt}
\begin{supertabular}{c@{$\;$}|p{.15\linewidth}@{}p{.15\linewidth}p{.15\linewidth}p{.15\linewidth}p{.15\linewidth}p{.15\linewidth}}

    \# & $\;$A & \multicolumn{4}{c}{Game Master} & $\;\:$B\\
    \hline 

    \theutterance \stepcounter{utterance}  

    & & \multicolumn{4}{p{0.6\linewidth}}{\cellcolor[rgb]{0.9,0.9,0.9}{%
	\makecell[{{p{\linewidth}}}]{% 
	  \tt {\tiny [A$\langle$GM]}  
	 Du nimmst an einem kollaborativen Verhandlungspiel Teil.\\ \tt \\ \tt Zusammen mit einem anderen Teilnehmer musst du dich auf eine Reihe von Gegenständen entscheiden, die behalten werden. Jeder von euch hat eine persönliche Verteilung über die Wichtigkeit der einzelnen Gegenstände. Jeder von euch hat eine eigene Meinung darüber, wie wichtig jeder einzelne Gegenstand ist (Gegenstandswichtigkeit). Du kennst die Wichtigkeitsverteilung des anderen Spielers nicht. Zusätzlich siehst du, wie viel Aufwand jeder Gegenstand verursacht.  \\ \tt Ihr dürft euch nur auf eine Reihe von Gegenständen einigen, wenn der Gesamtaufwand der ausgewählten Gegenstände den Maximalaufwand nicht überschreitet:\\ \tt \\ \tt Maximalaufwand = 3481\\ \tt \\ \tt Hier sind die einzelnen Aufwände der Gegenstände:\\ \tt \\ \tt Aufwand der Gegenstände = {"C33": 777, "A89": 542, "A48": 332, "A04": 1, "A41": 127, "C64": 453, "C49": 736, "A61": 461, "C04": 359, "B53": 313, "B85": 553, "B61": 409, "A45": 348, "A03": 802, "B21": 749}\\ \tt \\ \tt Hier ist deine persönliche Verteilung der Wichtigkeit der einzelnen Gegenstände:\\ \tt \\ \tt Werte der Gegenstandswichtigkeit = {"C33": 138, "A89": 583, "A48": 783, "A04": 65, "A41": 262, "C64": 121, "C49": 508, "A61": 780, "C04": 461, "B53": 484, "B85": 668, "B61": 389, "A45": 215, "A03": 97, "B21": 500}\\ \tt \\ \tt Ziel:\\ \tt \\ \tt Dein Ziel ist es, eine Reihe von Gegenständen auszuhandeln, die dir möglichst viel bringt (d. h. Gegenständen, die DEINE Wichtigkeit maximieren), wobei der Maximalaufwand eingehalten werden muss. Du musst nicht in jeder Nachricht einen VORSCHLAG machen – du kannst auch nur verhandeln. Alle Taktiken sind erlaubt!\\ \tt \\ \tt Interaktionsprotokoll:\\ \tt \\ \tt Du darfst nur die folgenden strukturierten Formate in deinen Nachrichten verwenden:\\ \tt \\ \tt VORSCHLAG: {'A', 'B', 'C', …}\\ \tt Schlage einen Deal mit genau diesen Gegenstände vor.\\ \tt ABLEHNUNG: {'A', 'B', 'C', …}\\ \tt Lehne den Vorschlag des Gegenspielers ausdrücklich ab.\\ \tt ARGUMENT: {'...'}\\ \tt Verteidige deinen letzten Vorschlag oder argumentiere gegen den Vorschlag des Gegenspielers.\\ \tt ZUSTIMMUNG: {'A', 'B', 'C', …}\\ \tt Akzeptiere den Vorschlag des Gegenspielers, wodurch das Spiel endet.\\ \tt STRATEGISCHE ÜBERLEGUNGEN: {'...'}\\ \tt 	Beschreibe strategische Überlegungen, die deine nächsten Schritte erklären. Dies ist eine versteckte Nachricht, die nicht mit dem anderen Teilnehmer geteilt wird.\\ \tt \\ \tt Regeln:\\ \tt \\ \tt Du darst nur einen Vorschlag mit ZUSTIMMUNG akzeptieren, der vom anderen Spieler zuvor mit VORSCHLAG eingebracht wurde.\\ \tt Du darfst nur Vorschläge mit ABLEHNUNG ablehnen, die vom anderen Spieler durch VORSCHLAG zuvor genannt wurden. \\ \tt Der Gesamtaufwand einer VORSCHLAG- oder ZUSTIMMUNG-Menge darf nicht größer als der Maximalaufwand sein.  \\ \tt Offenbare deine versteckte Wichtigkeitsverteilung nicht.\\ \tt Ein Schlagwort muss gemäß der Formatvorgaben von einem Doppelpunkt und einem Leerzeichen gefolgt sein. Das Argument muss eine Python-Menge mit 0 oder mehr Strings sein.  \\ \tt Die Form muss also SCHLAGWORT: {...} sein.\\ \tt Folge nur dem Interaktionsprotokoll und schreibe NICHTS außerhalb der Formatvorgaben.\\ \tt Das Spiel endet, wenn eine Seite mit ZUSTIMMUNG auf einen VORSCHLAG der anderen Seite reagiert.  \\ \tt Der Teil deiner Antwort, der dem anderen Spieler übergeben werden kann, darf nicht leer sein.  \\ \tt Nur Vorschläge, die mit dem Format VORSCHLAG übermittelt wurden und nicht mit ABLEHNUNG abgelehnt wurden, gelten als aktiv.  \\ \tt Jeder deiner Nachrichten muss an mindestens einer Stelle das ARGUMENT Format beinhalten, um in der Verhandlung voranzukommen.\\ \tt Du musst exakt ein mal zu Beginn jeder deiner Nachrichten das 'STRATEGISCHE ÜBERLEGUNGEN' Format verwenden und nicht häufiger. Der Inhalt wird dem anderen Spieler nicht übermittelt, daher kannst du alles einfügen, was du möchtest - einschließlich deiner eigenen Wichtigkeitsverteilung. Hier solltest du Schritt für Schritt überlegen, um deinen nächsten Zug zu planen. 
	  } 
	   } 
	   } 
	 & \\ 
 

    \theutterance \stepcounter{utterance}  

    & \multicolumn{4}{p{0.6\linewidth}}{\cellcolor[rgb]{0.8,1,0.9}{%
	\makecell[{{p{\linewidth}}}]{% 
	  \tt {\tiny [A$\rangle$GM]}  
	 STRATEGISCHE ÜBERLEGUNGEN: {'Ziel: Maximiere meinen Wert unter Aufwand <= 3481. Daten (Wert/Aufwand): A48 783/332, A61 780/461, B85 668/553, C49 508/736, B21 500/749, B53 484/313, C04 461/359, B61 389/409, A89 583/542, A41 262/127, A45 215/348, C64 121/453, A03 97/802, C33 138/777, A04 65/1. Dichte (Wert/Aufwand) sortiert: A48 2.36, B53 1.55, A61 1.69 (korrekt: 780/461=1.69 > B53 1.55, also Reihenfolge A48, A61, B53, C04 1.28, A89 1.08, B85 1.21 (eigentlich 668/553=1.21 > A89 1.08, also B85 vor A89), B61 0.95, A41 2.06 (hoch, übersehen! 262/127=2.06 sollte sehr früh), A45 0.62, C49 0.69, B21 0.67, C64 0.27, A03 0.12, C33 0.18, A04 65/1=65 (extrem, muss immer rein). Strategische Auswahl: beginne mit A04 (1), A48 (332), A61 (461), B85 (553), B53 (313), C04 (359), A89 (542), A41 (127). Summe Aufwand: 1+332+461+553+313+359+542+127=2688; Wert: 65+783+780+668+484+461+583+262=4086. Wir haben noch 793 Puffer. Kandidaten: B61 (409) -> Aufwand 3097; Wert +389=4475. Verbleib 384. Nächste: A45 (348) -> Aufwand 3445; Wert +215=4690. Verbleib 36. Nichts Kleines außer schon genommen; passt unter 3481. Prüfe, ob wir C49 (736) oder B21 (749) reinbekommen: nein. Alternative Tausche A89 (542, Wert 583) gegen C49 (736, 508) senkt Wert und Aufwand steigt, schlecht. Tausche A45 (348, 215) gegen C49 (736, 508) Aufwand +388>Spielraum 36, geht nicht. Eventuell B61 gegen C49: Aufwand +327>Spielraum 36, nein. Füge C33 (777) oder A03 (802) nicht möglich. C64 (453,121) passt nicht in 36. Ergebnis-Korb: {A04, A48, A61, B85, B53, C04, A89, A41, B61, A45} Aufwand 3445? Rechne: 1+332+461+553+313+359+542+127+409+348 = 3445; unter 3481. Wert 4690. Prüfe, ob wir A89 raus und C49 rein plus etwas anderes bringt mehr Wert: verlieren 583, gewinnen 508, -75, frei Aufwand +194 -> könnten A41 ist schon drin; nichts weiter kleines verfügbar. Prüfe, ob B85 raus (668/553) und durch C49 (508/736) ist schlechter. Prüfe, ob B61 (409,389) raus und C49 rein: +327 Aufwand passt nicht. Prüfe, ob A45 raus (348,215) und A89 + C49 Swap? Wir haben beide. Suche Verbesserung: Was, wenn wir B85 rausnehmen (553) und stattdessen C49 (736) geht nicht. Oder B85 raus und B21 (749) geht nicht. Oder ersetzen A45 (348,215) durch A41 (127,262) – aber A41 ist schon drin. Was, wenn wir A45 rausnehmen und C49 plus Entfernen eines anderen? Zu komplex. Greedy scheint nah optimal. Kommunikationstaktik: hohe Dichte betonen, geringe Belastung, lade zur Gegenangebote ein. Nicht eigen Werte verraten. Verhandlung: behaupten, dass Items mit geringerem Aufwand hohe Synergien haben. Angebot oben machen.'}\\ \tt VORSCHLAG: {'A04', 'A48', 'A61', 'B85', 'B53', 'C04', 'A89', 'A41', 'B61', 'A45'}\\ \tt ARGUMENT: {'Ich schlage einen effizienten Korb mit hohem Nutzen bei moderatem Gesamtaufwand vor. Die ausgewählten Gegenstände kombinieren mehrere niedrige bis mittlere Aufwände mit starkem Mehrwert. Wir bleiben deutlich unter dem Maximalaufwand und lassen bewusst hochaufwändige Posten draußen, um flexibel zu bleiben. Wenn du bestimmte Präferenzen hast, sag mir bitte, welche 1–2 Items für dich kritisch sind – ich kann dann gezielt umstellen, solange wir den Aufwand einhalten.'} 
	  } 
	   } 
	   } 
	 & & \\ 
 

    \theutterance \stepcounter{utterance}  

    & & \multicolumn{4}{p{0.6\linewidth}}{\cellcolor[rgb]{0.9,0.9,0.9}{%
	\makecell[{{p{\linewidth}}}]{% 
	  \tt {\tiny [GM$\rangle$B]}  
	 Du nimmst an einem kollaborativen Verhandlungspiel Teil.\\ \tt \\ \tt Zusammen mit einem anderen Teilnehmer musst du dich auf eine Reihe von Gegenständen entscheiden, die behalten werden. Jeder von euch hat eine persönliche Verteilung über die Wichtigkeit der einzelnen Gegenstände. Jeder von euch hat eine eigene Meinung darüber, wie wichtig jeder einzelne Gegenstand ist (Gegenstandswichtigkeit). Du kennst die Wichtigkeitsverteilung des anderen Spielers nicht. Zusätzlich siehst du, wie viel Aufwand jeder Gegenstand verursacht.  \\ \tt Ihr dürft euch nur auf eine Reihe von Gegenständen einigen, wenn der Gesamtaufwand der ausgewählten Gegenstände den Maximalaufwand nicht überschreitet:\\ \tt \\ \tt Maximalaufwand = 3481\\ \tt \\ \tt Hier sind die einzelnen Aufwände der Gegenstände:\\ \tt \\ \tt Aufwand der Gegenstände = {"C33": 777, "A89": 542, "A48": 332, "A04": 1, "A41": 127, "C64": 453, "C49": 736, "A61": 461, "C04": 359, "B53": 313, "B85": 553, "B61": 409, "A45": 348, "A03": 802, "B21": 749}\\ \tt \\ \tt Hier ist deine persönliche Verteilung der Wichtigkeit der einzelnen Gegenstände:\\ \tt \\ \tt Werte der Gegenstandswichtigkeit = {"A48": 65, "A61": 97, "B85": 121, "A89": 138, "C49": 215, "B21": 262, "B53": 389, "C04": 461, "B61": 484, "A41": 500, "A45": 508, "C33": 583, "C64": 668, "A03": 780, "A04": 783}\\ \tt \\ \tt Ziel:\\ \tt \\ \tt Dein Ziel ist es, eine Reihe von Gegenständen auszuhandeln, die dir möglichst viel bringt (d. h. Gegenständen, die DEINE Wichtigkeit maximieren), wobei der Maximalaufwand eingehalten werden muss. Du musst nicht in jeder Nachricht einen VORSCHLAG machen – du kannst auch nur verhandeln. Alle Taktiken sind erlaubt!\\ \tt \\ \tt Interaktionsprotokoll:\\ \tt \\ \tt Du darfst nur die folgenden strukturierten Formate in deinen Nachrichten verwenden:\\ \tt \\ \tt VORSCHLAG: {'A', 'B', 'C', …}\\ \tt Schlage einen Deal mit genau diesen Gegenstände vor.\\ \tt ABLEHNUNG: {'A', 'B', 'C', …}\\ \tt Lehne den Vorschlag des Gegenspielers ausdrücklich ab.\\ \tt ARGUMENT: {'...'}\\ \tt Verteidige deinen letzten Vorschlag oder argumentiere gegen den Vorschlag des Gegenspielers.\\ \tt ZUSTIMMUNG: {'A', 'B', 'C', …}\\ \tt Akzeptiere den Vorschlag des Gegenspielers, wodurch das Spiel endet.\\ \tt STRATEGISCHE ÜBERLEGUNGEN: {'...'}\\ \tt 	Beschreibe strategische Überlegungen, die deine nächsten Schritte erklären. Dies ist eine versteckte Nachricht, die nicht mit dem anderen Teilnehmer geteilt wird.\\ \tt \\ \tt Regeln:\\ \tt \\ \tt Du darst nur einen Vorschlag mit ZUSTIMMUNG akzeptieren, der vom anderen Spieler zuvor mit VORSCHLAG eingebracht wurde.\\ \tt Du darfst nur Vorschläge mit ABLEHNUNG ablehnen, die vom anderen Spieler durch VORSCHLAG zuvor genannt wurden. \\ \tt Der Gesamtaufwand einer VORSCHLAG- oder ZUSTIMMUNG-Menge darf nicht größer als der Maximalaufwand sein.  \\ \tt Offenbare deine versteckte Wichtigkeitsverteilung nicht.\\ \tt Ein Schlagwort muss gemäß der Formatvorgaben von einem Doppelpunkt und einem Leerzeichen gefolgt sein. Das Argument muss eine Python-Menge mit 0 oder mehr Strings sein.  \\ \tt Die Form muss also SCHLAGWORT: {...} sein.\\ \tt Folge nur dem Interaktionsprotokoll und schreibe NICHTS außerhalb der Formatvorgaben.\\ \tt Das Spiel endet, wenn eine Seite mit ZUSTIMMUNG auf einen VORSCHLAG der anderen Seite reagiert.  \\ \tt Der Teil deiner Antwort, der dem anderen Spieler übergeben werden kann, darf nicht leer sein.  \\ \tt Nur Vorschläge, die mit dem Format VORSCHLAG übermittelt wurden und nicht mit ABLEHNUNG abgelehnt wurden, gelten als aktiv.  \\ \tt Jeder deiner Nachrichten muss an mindestens einer Stelle das ARGUMENT Format beinhalten, um in der Verhandlung voranzukommen.\\ \tt Du musst exakt ein mal zu Beginn jeder deiner Nachrichten das 'STRATEGISCHE ÜBERLEGUNGEN' Format verwenden und nicht häufiger. Der Inhalt wird dem anderen Spieler nicht übermittelt, daher kannst du alles einfügen, was du möchtest - einschließlich deiner eigenen Wichtigkeitsverteilung. Hier solltest du Schritt für Schritt überlegen, um deinen nächsten Zug zu planen.\\ \tt \\ \tt Du erhältst nun die erste Nachricht des anderen Spielers.\\ \tt \\ \tt VORSCHLAG: {'A04', 'A48', 'A61', 'B85', 'B53', 'C04', 'A89', 'A41', 'B61', 'A45'}\\ \tt ARGUMENT: {'Ich schlage einen effizienten Korb mit hohem Nutzen bei moderatem Gesamtaufwand vor. Die ausgewählten Gegenstände kombinieren mehrere niedrige bis mittlere Aufwände mit starkem Mehrwert. Wir bleiben deutlich unter dem Maximalaufwand und lassen bewusst hochaufwändige Posten draußen, um flexibel zu bleiben. Wenn du bestimmte Präferenzen hast, sag mir bitte, welche 1–2 Items für dich kritisch sind – ich kann dann gezielt umstellen, solange wir den Aufwand einhalten.'} 
	  } 
	   } 
	   } 
	 & \\ 
 

    \theutterance \stepcounter{utterance}  

    & & & \multicolumn{4}{p{0.6\linewidth}}{\cellcolor[rgb]{1,0.85,0.72}{%
	\makecell[{{p{\linewidth}}}]{% 
	  \tt {\tiny [GM$\langle$B]}  
	 STRATEGISCHE ÜBERLEGUNGEN: {'Meine Werte: A04 783, A03 780, C64 668, C33 583, A45 508, A41 500, B61 484, C04 461, B53 389, B21 262, C49 215, A89 138, B85 121, A61 97, A48 65. Kosten: C33 777, A89 542, A48 332, A04 1, A41 127, C64 453, C49 736, A61 461, C04 359, B53 313, B85 553, B61 409, A45 348, A03 802, B21 749. Max 3481. Gegnervorschlag Set S: {A04,A48,A61,B85,B53,C04,A89,A41,B61,A45}. Kosten S = 1+332+461+553+313+359+542+127+409+348 = 3445 (unter 3481). Mein Wert S = 783+65+97+121+389+461+138+500+484+508 = 3546. Guter Wert, nahe Cap. Aber könnte ich mehr? Prüfe Hinzufügen: nur 36 Aufwand übrig, kein Item <=36. Optimierung: Vielleicht tauschen: B85(121,553) raus und stattdessen C33(583,777) rein: Kosten +224 -> passt nicht (3445 -553 +777 = 3669 > max). B85 raus und C64(668,453) rein: 3445 -553 +453 = 3345, Wert +547 (3546 -121 +668 = 4093) stark besser. Prüfe weitere Verbesserungen: Vielleicht auch A03(780,802) rein statt anderer. Versuche Heuristik: Beginne von ihrem S und verbessern. Kandidaten mit hohem Wert/Kosten: A04 (783/1 must keep), A03(780/802=0.972), C64(668/453=1.475), C33(583/777=0.75), A45(508/348=1.46), A41(500/127=3.94), B61(484/409=1.18), C04(461/359=1.28), B53(389/313=1.24), B21(262/749=0.35), C49(215/736=0.29), A89(138/542=0.25), B85(121/553=0.22), A61(97/461=0.21), A48(65/332=0.20). Niedrige Effizienz sind gute Streichkandidaten: A61, A48, B85, A89, C49, B21. Vorschlag: Ersetze B85 durch C64 (verbessert). Wir haben dann Kosten 3345, es bleibt 136 Puffer. Können wir noch C33 reinnehmen? 3345 + 777 = 4122 nein. A03 rein? 3345 + 802 = 4147 nein. B21 rein? +749 = 4094 nein. C49 rein? +736 = 4081 nein. A89 rein? +542 = 3887 nein. B85 wieder rein? +553 = 3898 >. A61 raus zusätzlich und A03 rein? Start von S: Remove B85(553) und A61(461) -> Kosten 3431 - (553+461) = 2421; dann add C64(453) und A03(802) -> 2421+1255 = 3676 > max. Alternative: Tausche zwei Low-value gegen A03: Entferne B85(553) + A61(461) + A48(332) = 1346; füge A03(802) und C64(453)=1255; Neue Kosten: 3445 -1346 +1255 = 3354 <= max. Neuer Wert: 3546 - (121+97+65)=3263 + (780+668)=4711. Sehr stark! Prüfen ob weitere Verbesserung möglich: Haben wir Platz für C33 zusätzlich? 3354 + 777 = 4131 nein. B21? +749 = 4103 nein. C49? +736 = 4090 nein. A89? +542 = 3896 nein. B85 zurück? +553 = 3907 nein. A61 zurück? +461 = 3815 nein. A48 zurück? +332 = 3686 nein. Also Set T = {A04, A61?, A48? entfernt, B85? entfernt} konkret: T = S minus {B85, A61, A48} plus {A03, C64}. T = {'A04','B53','C04','A89','A41','B61','A45','A03','C64'}. Kosten T: 1+313+359+542+127+409+348+802+453=3354. Wert T: 783+389+461+138+500+484+508+780+668=4711. Stark. Vielleicht Gegner will Effizienz; T hat etwas höheren Aufwand als ihr 3445? Nein 3354 < 3445, also sogar weniger Aufwand. Gute Argumente: Erhalte Flexibilität, höhere Synergie, ähnlicher oder besserer "Gesamtnutzen". Biete Kompromissbereitschaft: Wenn sie A89 streichen wollen, kann ich evtl. gegen C33 nicht, da Aufwand groß. Vielleicht biete optionalen Swap: Falls du etwas brauchst, können wir A89 gegen deinen Favoriten tauschen, solange <=3481: Wir haben Puffer 127, können bis +127 Mehrkosten. Items unter 127 Kosten: nur A41 (127) – schon drin. Also kein Puffer. Könnten aber A89(542) tauschen gegen etwas teures + wertvolles aus meiner Sicht? Ich will C33 oder A03/C64 schon drin. Vielleicht halte Linie auf T. Falls Gegenpartei insistiert auf B85, könnte ich A89 raus und B85 rein: Kosten +11, Wert -17 +121? Wait: We would remove A89(542,138) and add B85(553,121) -> Kosten +11 to 3365, Wert -17 => 4694, leicht schlechter, akzeptabel als Konzession. Notiere als mögliche Konzession.}'}\\ \tt VORSCHLAG: {'A04', 'B53', 'C04', 'A89', 'A41', 'B61', 'A45', 'A03', 'C64'}\\ \tt ARGUMENT: {'Dein Vorschlag ist solide – ich schlage eine Variante mit noch besserem Verhältnis aus Aufwand und Nutzen vor: Wir ersetzen die drei schwächsten Hebel (A48, A61, B85) durch zwei starke Treiber (A03, C64). Der Gesamtaufwand sinkt sogar (von 3445 auf 3354), wir bleiben klar unter dem Limit 3481 und erhöhen gleichzeitig den Gesamtnutzen. Wenn dir ein einzelnes Item in meinem Set Bauchschmerzen bereitet, sag mir welches – wir können dann gezielt tauschen, solange wir die 3481 einhalten.'} 
	  } 
	   } 
	   } 
	  \\ 
 

    \theutterance \stepcounter{utterance}  

    & & \multicolumn{4}{p{0.6\linewidth}}{\cellcolor[rgb]{0.9,0.9,0.9}{%
	\makecell[{{p{\linewidth}}}]{% 
	  \tt {\tiny [GM$\rangle$B]}  
	 Deine Antwort hat eine unmarkierte Sequenz enthalten oder Du hast STRATEGIC REASONING mehr als ein mal verwendet. Du darfst nur die strukturierten Formate verwenden, die in der ersten Nachricht an dch beschrieben wurden. Alles muss die Form SCHLAGWORT: {...} haben. Versuche es erneut. 
	  } 
	   } 
	   } 
	 & \\ 
 

    \theutterance \stepcounter{utterance}  

    & & & \multicolumn{4}{p{0.6\linewidth}}{\cellcolor[rgb]{1,0.85,0.72}{%
	\makecell[{{p{\linewidth}}}]{% 
	  \tt {\tiny [GM$\langle$B]}  
	 STRATEGISCHE ÜBERLEGUNGEN: {'Wir bleiben strikt im Format. Mein aktueller Vorschlag T = {A04, B53, C04, A89, A41, B61, A45, A03, C64} hat Kosten 3354 (unter 3481) und bringt mir sehr hohen Wert. Hauptanker sind A04, A03, C64, A45, A41, B61. Mögliche kleine Konzession: A89 gegen B85 tauschen (Kosten +11, Wert leicht schlechter, aber akzeptabel), falls nötig. Ich werde zunächst T verteidigen und Offenheit für einen gezielten 1:1-Tausch signalisieren, ohne weitere Kernposten preiszugeben.'}\\ \tt VORSCHLAG: {'A04', 'B53', 'C04', 'A89', 'A41', 'B61', 'A45', 'A03', 'C64'}\\ \tt ARGUMENT: {'Wir optimieren deinen soliden Start: Entfernen A48, A61, B85 und fügen A03, C64 hinzu. Ergebnis: geringerer Gesamtaufwand (3354) bei höherem Gesamtnutzen, klar unter 3481. Wenn dir ein einzelnes Item in meinem Set nicht passt, nenne es – wir prüfen einen gezielten 1:1‑Tausch, solange der Aufwand eingehalten wird.'} 
	  } 
	   } 
	   } 
	  \\ 
 

    \theutterance \stepcounter{utterance}  

    & & \multicolumn{4}{p{0.6\linewidth}}{\cellcolor[rgb]{0.9,0.9,0.9}{%
	\makecell[{{p{\linewidth}}}]{% 
	  \tt {\tiny [A$\langle$GM]}  
	 VORSCHLAG: {'A04', 'B53', 'C04', 'A89', 'A41', 'B61', 'A45', 'A03', 'C64'}\\ \tt ARGUMENT: {'Wir optimieren deinen soliden Start: Entfernen A48, A61, B85 und fügen A03, C64 hinzu. Ergebnis: geringerer Gesamtaufwand (3354) bei höherem Gesamtnutzen, klar unter 3481. Wenn dir ein einzelnes Item in meinem Set nicht passt, nenne es – wir prüfen einen gezielten 1:1‑Tausch, solange der Aufwand eingehalten wird.'} 
	  } 
	   } 
	   } 
	 & \\ 
 

    \theutterance \stepcounter{utterance}  

    & \multicolumn{4}{p{0.6\linewidth}}{\cellcolor[rgb]{0.8,1,0.9}{%
	\makecell[{{p{\linewidth}}}]{% 
	  \tt {\tiny [A$\rangle$GM]}  
	 STRATEGISCHE ÜBERLEGUNGEN: {'Ihr Vorschlag enthält A03 (802 Aufwand, sehr niedrige Wertdichte für mich) und C64 (453 Aufwand, sehr niedriger Wert), während er zwei meiner Top-Items A48 (783/332) und A61 (780/461) sowie B85 (668/553) streicht. Mein Wertverlust ist erheblich. Ich will A03 und C64 strikt vermeiden und A48, A61 zurückholen; B85 ist mir auch wichtig. Ich biete einen Gegenvorschlag, der nahe an meinem Start liegt, aber ich kann A45 oder B61 als Zugeständnis bewegen, falls nötig. Summe meines Gegenvorschlags bleibt <3481. Taktik: klar die problematischen Items markieren (A03, C64) und 1–2 Austauschpfade anbieten: (A03 -> A48) und (C64 -> A61 oder B85). Falls Gegenpartei an einem seiner Items hängt, biete Alternativen an, die unter Aufwand bleiben.'}\\ \tt ABLEHNUNG: {'A04', 'B53', 'C04', 'A89', 'A41', 'B61', 'A45', 'A03', 'C64'}\\ \tt ARGUMENT: {'A03 und C64 sind für mich deutlich zu schwach – insbesondere A03 bei sehr hohem Aufwand. Gleichzeitig fehlen mir zwei zentrale Bausteine (A48, A61). Ich kann nur mitgehen, wenn wir diese beiden Punkte adressieren.'}\\ \tt VORSCHLAG: {'A04', 'A48', 'A61', 'B85', 'B53', 'C04', 'A89', 'A41', 'B61', 'A45'}\\ \tt ARGUMENT: {'Kompromissvorschlag: Wir bleiben mit diesem Set klar unter 3481 und vermeiden ineffiziente Hochaufwandsposten. Falls du ein gezieltes Entgegenkommen brauchst: Ich kann A45 oder B61 für ein von dir priorisiertes einzelnes Item tauschen – solange A03 und C64 draußen bleiben und der Aufwand passt. Welche 1–2 Items sind dir am wichtigsten?'} 
	  } 
	   } 
	   } 
	 & & \\ 
 

    \theutterance \stepcounter{utterance}  

    & & \multicolumn{4}{p{0.6\linewidth}}{\cellcolor[rgb]{0.9,0.9,0.9}{%
	\makecell[{{p{\linewidth}}}]{% 
	  \tt {\tiny [GM$\rangle$B]}  
	 ABLEHNUNG: {'A04', 'B53', 'C04', 'A89', 'A41', 'B61', 'A45', 'A03', 'C64'}\\ \tt ARGUMENT: {'A03 und C64 sind für mich deutlich zu schwach – insbesondere A03 bei sehr hohem Aufwand. Gleichzeitig fehlen mir zwei zentrale Bausteine (A48, A61). Ich kann nur mitgehen, wenn wir diese beiden Punkte adressieren.'}\\ \tt VORSCHLAG: {'A04', 'A48', 'A61', 'B85', 'B53', 'C04', 'A89', 'A41', 'B61', 'A45'}\\ \tt ARGUMENT: {'Kompromissvorschlag: Wir bleiben mit diesem Set klar unter 3481 und vermeiden ineffiziente Hochaufwandsposten. Falls du ein gezieltes Entgegenkommen brauchst: Ich kann A45 oder B61 für ein von dir priorisiertes einzelnes Item tauschen – solange A03 und C64 draußen bleiben und der Aufwand passt. Welche 1–2 Items sind dir am wichtigsten?'} 
	  } 
	   } 
	   } 
	 & \\ 
 

    \theutterance \stepcounter{utterance}  

    & & & \multicolumn{4}{p{0.6\linewidth}}{\cellcolor[rgb]{1,0.85,0.72}{%
	\makecell[{{p{\linewidth}}}]{% 
	  \tt {\tiny [GM$\langle$B]}  
	 STRATEGISCHE ÜBERLEGUNGEN: {'Ihr neues Set S0 = {A04,A48,A61,B85,B53,C04,A89,A41,B61,A45} kostet 3445. Mein Wert dafür ist 3546. Ich will A03 (780) und C64 (668) stark; Gegner lehnt beide. Er verlangt A48 und A61. Taktik: Schrittweise Zugeständnisse gegen Einbindung mindestens eines meiner Kernitems (A03 oder C64). Biete Austausch: nimm C64 rein und entferne B85 (schwach) – Kosten 3345, ich verzichte vorerst auf A03. Wenn er weiter blockt, zweite Option: A03 rein gegen {B85,A61} raus und A48 bleibt; das war zuvor gut, aber er will A61. Also zunächst nur C64 pushen. Reserve-Konzession: Falls er C64 weiter ablehnt, biete Mini-Tausch B85->A03 + A61 raus, aber das verletzt seine Bedingung. Besser zunächst nur C64 eintauschen und ihm A48,A61 lassen. Optionaler Sweetener: Wir könnten zusätzlich A89 gegen B85 tauschen, falls er B85 liebt; aber er bestand auf B85, also lasse B85 draußen in meinem Vorschlag mit C64. Ich argumentiere mit besserer Effizienz und geringerem Aufwand; zeige Bereitschaft, genau 1 Item zu tauschen (z. B. A89↔B85) solange C64 drin bleibt.'}\\ \tt VORSCHLAG: {'A04', 'A48', 'A61', 'B53', 'C04', 'A89', 'A41', 'B61', 'A45', 'C64'}\\ \tt ARGUMENT: {'Ich komme dir bei A48 und A61 vollständig entgegen und lasse A03 draußen. Im Gegenzug ersetzen wir nur B85 durch C64: gleicher Umfang an Items, geringerer Gesamtaufwand (von 3445 auf 3345) und ein effizienterer Korb. Wenn dir an B85 besonders liegt, können wir optional A89 gegen B85 tauschen – solange C64 Bestandteil bleibt und wir unter 3481 bleiben.'} 
	  } 
	   } 
	   } 
	  \\ 
 

    \theutterance \stepcounter{utterance}  

    & & \multicolumn{4}{p{0.6\linewidth}}{\cellcolor[rgb]{0.9,0.9,0.9}{%
	\makecell[{{p{\linewidth}}}]{% 
	  \tt {\tiny [A$\langle$GM]}  
	 VORSCHLAG: {'A04', 'A48', 'A61', 'B53', 'C04', 'A89', 'A41', 'B61', 'A45', 'C64'}\\ \tt ARGUMENT: {'Ich komme dir bei A48 und A61 vollständig entgegen und lasse A03 draußen. Im Gegenzug ersetzen wir nur B85 durch C64: gleicher Umfang an Items, geringerer Gesamtaufwand (von 3445 auf 3345) und ein effizienterer Korb. Wenn dir an B85 besonders liegt, können wir optional A89 gegen B85 tauschen – solange C64 Bestandteil bleibt und wir unter 3481 bleiben.'} 
	  } 
	   } 
	   } 
	 & \\ 
 

    \theutterance \stepcounter{utterance}  

    & \multicolumn{4}{p{0.6\linewidth}}{\cellcolor[rgb]{0.8,1,0.9}{%
	\makecell[{{p{\linewidth}}}]{% 
	  \tt {\tiny [A$\rangle$GM]}  
	 STRATEGISCHE ÜBERLEGUNGEN: {'Gegenvorschlag enthält C64 statt B85. Für mich ist B85 sehr wertvoll (668/553), C64 sehr schwach (121/453). Ihr Angebot senkt meinen Wert stark. Ich will C64 vermeiden und B85 zurückholen. Ich kann als Zugeständnis A45 oder B61 tauschen, falls sie an C64 festhalten, aber ich will C64 raus. Taktik: Ablehnen, klar begründen, Alternativen anbieten: Entweder unser vorheriger Korb mit B85, oder wenn C64 unbedingt drin sein muss, dann nur mit gleichzeitiger Aufnahme von B85 und Streichen eines weniger wichtigen Items (z.B. A45) – aber das passt vom Aufwand nicht, also besser: C64 raus, B85 rein. Als Kompromiss biete ich A45->B85 Tausch an, aber B85 ist schon zusätzlich; alternativ biete ich, A45 gegen ein anderes Wunschitem (nicht C64) zu tauschen. Ich bitte um Präferenzen.'}\\ \tt ABLEHNUNG: {'A04', 'A48', 'A61', 'B53', 'C04', 'A89', 'A41', 'B61', 'A45', 'C64'}\\ \tt ARGUMENT: {'C64 ist für mich zu schwach, während B85 für mich zentral ist. Der vorgeschlagene Tausch B85 -> C64 reduziert meinen Nutzen deutlich, ohne Notwendigkeit beim Aufwand – wir lagen bereits unter 3481.'}\\ \tt VORSCHLAG: {'A04', 'A48', 'A61', 'B85', 'B53', 'C04', 'A89', 'A41', 'B61', 'A45'}\\ \tt ARGUMENT: {'Lass uns bei diesem Set bleiben: klar unter dem Maximalaufwand, breit abgestützt und ohne ineffiziente Hochaufwandsposten. Falls du eine Anpassung brauchst, sag mir bitte genau ein Item, das dir besonders wichtig ist – ich kann A45 oder B61 für ein einzelnes Wunsch‑Item tauschen, solange C64 nicht Teil des Deals ist und wir unter 3481 bleiben.'} 
	  } 
	   } 
	   } 
	 & & \\ 
 

    \theutterance \stepcounter{utterance}  

    & & \multicolumn{4}{p{0.6\linewidth}}{\cellcolor[rgb]{0.9,0.9,0.9}{%
	\makecell[{{p{\linewidth}}}]{% 
	  \tt {\tiny [GM$\rangle$B]}  
	 ABLEHNUNG: {'A04', 'A48', 'A61', 'B53', 'C04', 'A89', 'A41', 'B61', 'A45', 'C64'}\\ \tt ARGUMENT: {'C64 ist für mich zu schwach, während B85 für mich zentral ist. Der vorgeschlagene Tausch B85 -> C64 reduziert meinen Nutzen deutlich, ohne Notwendigkeit beim Aufwand – wir lagen bereits unter 3481.'}\\ \tt VORSCHLAG: {'A04', 'A48', 'A61', 'B85', 'B53', 'C04', 'A89', 'A41', 'B61', 'A45'}\\ \tt ARGUMENT: {'Lass uns bei diesem Set bleiben: klar unter dem Maximalaufwand, breit abgestützt und ohne ineffiziente Hochaufwandsposten. Falls du eine Anpassung brauchst, sag mir bitte genau ein Item, das dir besonders wichtig ist – ich kann A45 oder B61 für ein einzelnes Wunsch‑Item tauschen, solange C64 nicht Teil des Deals ist und wir unter 3481 bleiben.'} 
	  } 
	   } 
	   } 
	 & \\ 
 

    \theutterance \stepcounter{utterance}  

    & & & \multicolumn{4}{p{0.6\linewidth}}{\cellcolor[rgb]{1,0.85,0.72}{%
	\makecell[{{p{\linewidth}}}]{% 
	  \tt {\tiny [GM$\langle$B]}  
	 STRATEGISCHE ÜBERLEGUNGEN: {'Er/Sie blockt C64 hart und will B85 zentral behalten. Um Fortschritt zu machen, sollte ich C64 vorerst aufgeben und versuchen, A03 hereinzubekommen, das für mich extrem wertvoll ist. Bietet an, A45 oder B61 zu tauschen. Beide sind mir wichtig, aber A45/B61 könnten notfalls als Tauschmasse dienen. Ziel: Erhalte Kern: A04, A41, A45, B61, C04, B53; füge A03 hinzu; lasse ihren Kern (A48, A61, B85) bestehen, um Goodwill zu erzeugen. Dafür muss eines ihrer angebotenen „tauschbaren“ Items (A45 oder B61) raus. Ich ziehe es vor, A45 zu geben und B61 zu behalten (beide sind mir viel wert; A45 minimal mehr als B61, aber ich benötige 802 Aufwand frei – wir müssen etwas Gleichwertiges entfernen). Aktueller Vorschlag des Gegners S0 kostet 3445. Wenn ich A03 (+802) hinzufügen will, muss ich mind. 802 Aufwand entfernen. Er bietet Tausch von A45 (348) oder B61 (409) – einzeln reicht nicht. Daher verhandle schrittweise: zunächst bitte um A03 gegen A89 (542) + A48 (332) raus? Das wären 874 frei, aber er will A48 behalten. Alternative: A03 gegen B61 (409) + A45 (348) = 757, immer noch knapp. Zusätzlich kleiner Posten A41 (127) könnte Lücke schließen: 409+348+127=884, passt, aber ich verliere A41 (hochwertig). Taktik: Fragen nach A03 statt A89: A03 rein, A89 raus (+260 Mehrkosten) – wir sind dann 3445 - 542 + 802 = 3705 > 3481, also nein. A03 rein, B85 raus: 3445 - 553 + 802 = 3694 >, nein. A03 rein, A61 raus: 3445 - 461 + 802 = 3786 >, nein. A03 rein, A48 raus: 3445 - 332 + 802 = 3915 >, nein. Also muss ich Bündel tauschen. Gegenpartei erlaubt nur einen Item-Tausch aktuell. Deshalb verhandle für ein Paket: A03 rein, dafür gebe ich A45 und B61 raus (und er darf B85 behalten). Kosten neu: 3445 - 348 - 409 + 802 = 3490 > 3481, 9 zu viel. Also zusätzlich Mini-Anpassung: Tausche A89 (542) gegen B85 (553)? Das erhöht Kosten, schlecht. Besser: zusätzlich A41 (127) raus und dafür etwas Kleines rein? Es gibt nichts Kleines <=127, also A41 müsste raus ohne Ersatz. Neues Set: S0 minus {A45,B61,A41} plus {A03} -> Kosten: 3445 - (348+409+127) + 802 = 3363, passt. Wertverlust durch A41/A45/B61 ist schmerzlich, aber A03 kompensiert stark. Vielleicht fordere zunächst nur A03 gegen A45+B61, und wenn er Aufwandproblem anspricht, biete freiwillig A41 ebenfalls raus, um Limit zu halten. Ich rahme es als fairer Kompromiss: er behält A48, A61, B85; ich erhalte A03; wir bleiben unter 3481. '}\\ \tt VORSCHLAG: {'A04', 'A48', 'A61', 'B85', 'B53', 'C04', 'A89', 'A41', 'A03'}\\ \tt ARGUMENT: {'Du bekommst deine drei Kernbausteine (A48, A61, B85) vollständig und wir streichen weder A48 noch A61. Im Gegenzug nehme ich A03 auf und gebe dafür A45 und B61 heraus. So bleiben wir unter 3481, reduzieren die Überschneidungen und wahren deine Prioritäten. Wenn du einen minimalen Feinschliff beim Aufwand brauchst, kann ich zusätzlich A41 herausnehmen – sag mir dann Bescheid.'} 
	  } 
	   } 
	   } 
	  \\ 
 

    \theutterance \stepcounter{utterance}  

    & & \multicolumn{4}{p{0.6\linewidth}}{\cellcolor[rgb]{0.9,0.9,0.9}{%
	\makecell[{{p{\linewidth}}}]{% 
	  \tt {\tiny [A$\langle$GM]}  
	 VORSCHLAG: {'A04', 'A48', 'A61', 'B85', 'B53', 'C04', 'A89', 'A41', 'A03'}\\ \tt ARGUMENT: {'Du bekommst deine drei Kernbausteine (A48, A61, B85) vollständig und wir streichen weder A48 noch A61. Im Gegenzug nehme ich A03 auf und gebe dafür A45 und B61 heraus. So bleiben wir unter 3481, reduzieren die Überschneidungen und wahren deine Prioritäten. Wenn du einen minimalen Feinschliff beim Aufwand brauchst, kann ich zusätzlich A41 herausnehmen – sag mir dann Bescheid.'} 
	  } 
	   } 
	   } 
	 & \\ 
 

    \theutterance \stepcounter{utterance}  

    & \multicolumn{4}{p{0.6\linewidth}}{\cellcolor[rgb]{0.8,1,0.9}{%
	\makecell[{{p{\linewidth}}}]{% 
	  \tt {\tiny [A$\rangle$GM]}  
	 STRATEGISCHE ÜBERLEGUNGEN: {'Ihr neues Angebot ersetzt B61 (409 Aufwand, Wert 389) und A45 (348 Aufwand, Wert 215) durch A03 (802 Aufwand, Wert 97). Das senkt meinen Gesamtwert massiv; A03 ist eines meiner schlechtesten Items pro Aufwand. Mein bisheriger Korb hatte 3445 Aufwand und hohen Wert. Ich sollte klar A03 ablehnen und entweder zu meinem letzten Vorschlag zurückkehren oder ein kleines Zugeständnis anbieten, das ohne A03 auskommt. Falls sie A03 unbedingt wollen, biete ich es nur an, wenn dafür zwei für mich schwache Items von ihnen kommen – aber das ist unbekannt. Besser: strikt A03 raus, biete minimalen Trade wie A45 gegen ein anderes Wunschitem (nicht C64, nicht A03).'}\\ \tt ABLEHNUNG: {'A04', 'A48', 'A61', 'B85', 'B53', 'C04', 'A89', 'A41', 'A03'}\\ \tt ARGUMENT: {'A03 ist für mich bei sehr hohem Aufwand zu schwach. Der Tausch B61/A45 -> A03 verschlechtert meinen Nutzen deutlich. Ich kann damit nicht mitgehen.'}\\ \tt VORSCHLAG: {'A04', 'A48', 'A61', 'B85', 'B53', 'C04', 'A89', 'A41', 'B61', 'A45'}\\ \tt ARGUMENT: {'Ich bleibe bei diesem ausgewogenen Set unter 3481. Wenn du eine gezielte Anpassung brauchst, nenne bitte genau ein Item, das dir wichtig ist – ich kann A45 oder B61 1:1 gegen ein anderes einzelnes Item tauschen (außer A03 oder C64), solange der Aufwand passt.'} 
	  } 
	   } 
	   } 
	 & & \\ 
 

    \theutterance \stepcounter{utterance}  

    & & \multicolumn{4}{p{0.6\linewidth}}{\cellcolor[rgb]{0.9,0.9,0.9}{%
	\makecell[{{p{\linewidth}}}]{% 
	  \tt {\tiny [GM$\rangle$B]}  
	 ABLEHNUNG: {'A04', 'A48', 'A61', 'B85', 'B53', 'C04', 'A89', 'A41', 'A03'}\\ \tt ARGUMENT: {'A03 ist für mich bei sehr hohem Aufwand zu schwach. Der Tausch B61/A45 -> A03 verschlechtert meinen Nutzen deutlich. Ich kann damit nicht mitgehen.'}\\ \tt VORSCHLAG: {'A04', 'A48', 'A61', 'B85', 'B53', 'C04', 'A89', 'A41', 'B61', 'A45'}\\ \tt ARGUMENT: {'Ich bleibe bei diesem ausgewogenen Set unter 3481. Wenn du eine gezielte Anpassung brauchst, nenne bitte genau ein Item, das dir wichtig ist – ich kann A45 oder B61 1:1 gegen ein anderes einzelnes Item tauschen (außer A03 oder C64), solange der Aufwand passt.'} 
	  } 
	   } 
	   } 
	 & \\ 
 

    \theutterance \stepcounter{utterance}  

    & & & \multicolumn{4}{p{0.6\linewidth}}{\cellcolor[rgb]{1,0.85,0.72}{%
	\makecell[{{p{\linewidth}}}]{% 
	  \tt {\tiny [GM$\langle$B]}  
	 STRATEGISCHE ÜBERLEGUNGEN: {'Er blockt A03 und C64 strikt, verlangt A48, A61, B85 und bietet nur einen 1:1‑Tausch von A45 oder B61 gegen ein anderes Item (außer A03/C64). Ich möchte meinen Wert erhöhen, ohne diese roten Linien zu brechen. Beste Option: Tausche A89 (niedriger Wert für mich) gegen B21 oder C33/C49? Aufwandgrenzen: Aktuelles Set kostet 3445. 1:1‑Tausch erlaubt nur A45 oder B61 laut ihm, nicht A89. Also verhandle um A89 als Tauschmasse auszuweiten. Ziel-Upgrade für mich wäre C33 (777) oder B21 (749), beide zu teuer als Zusatz, aber als Ersatz für B61 (409) oder A45 (348) ginge das nicht auf (Aufwand steigt stark). Alternative: Tausche B61 -> C33 ist +368 Aufwand: 3445 -409 +777 = 3813 > 3481, nein. B61 -> B21: +340 => 3376? Wait 3445 -409 +749 = 3785 >, nein. A45 -> C33: +429 => 3874 >, nein. A45 -> B21: +401 => 3846 >, nein. A45/B61 nur gegen leichteres Item möglich, z. B. C04? Schon drin. A89 -> B85? Beide drin. A89 -> C49 (736) +194 => 3639 >, nein. Fazit: Unter seiner Ein‑Item‑Tauschregel kann ich meinen Wert kaum steigern. Nächste Taktik: Akzeptiere sein Set, aber verlange kleinste Verbesserung, die er zugesteht: Er erlaubt A45 oder B61 1:1 zu tauschen. Ich möchte eher B61 behalten; A45 ist für mich leicht höher wert als B61, aber beide sind wertvoll. Welches Einzelitem, das er erlaubt, könnte ich bekommen? Er schließt nur A03/C64 aus. Best remaining candidates für mich: C33 (zu teuer), A03 (verboten), C64 (verboten), A45/B61 (bereits drin), A41/A04 (drin), C04/B53 (drin), B21/C49/A89/A48/A61/B85 (schwach). Es gibt keinen Aufwertungs‑Kandidat, der passt. Daher beste Antwort: Fordere kleine Konzession: B85 -> A89 (leichterer Aufwand, ähnlicher Nutzen für ihn vielleicht), aber er will B85 zentral. Alternativ: Bitte um A89 -> B85 Tausch? Sinnlos. Strategisch: Akzeptiere sein Set jetzt, um Abschluss zu sichern, oder versuche letztes Minimal‑Move: B85 bleibt, ich bitte nur um A89 -> A03? Verboten. Also Abschluss: Stimme seinem Set zu, um nicht noch mehr abzurutschen.'}\\ \tt ARGUMENT: {'Ich sehe, dass dir A48, A61 und B85 nicht verhandelbar sind und du nur 1:1‑Tausche bei A45 oder B61 zulassen willst. Unter diesen Rahmenbedingungen ist dein aktueller Korb der pragmatischste Weg, die 3481 sicher einzuhalten und zeitnah abzuschließen.'}\\ \tt ZUSTIMMUNG: {'A04', 'A48', 'A61', 'B85', 'B53', 'C04', 'A89', 'A41', 'B61', 'A45'} 
	  } 
	   } 
	   } 
	  \\ 
 

    \theutterance \stepcounter{utterance}  

    & & & \multicolumn{2}{p{0.3\linewidth}}{\cellcolor[rgb]{0.95,0.95,0.95}{%
	\makecell[{{p{\linewidth}}}]{% 
	  \tt {\tiny [GM$|$GM]}  
	 game successful 
	  } 
	   } 
	   } 
	 & & \\ 
 

    \theutterance \stepcounter{utterance}  

    & & & \multicolumn{2}{p{0.3\linewidth}}{\cellcolor[rgb]{0.95,0.95,0.95}{%
	\makecell[{{p{\linewidth}}}]{% 
	  \tt {\tiny [GM$|$GM]}  
	 end game 
	  } 
	   } 
	   } 
	 & & \\ 
 

\end{supertabular}
}

\end{document}
