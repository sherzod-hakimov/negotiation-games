
\documentclass{article}
\usepackage{colortbl}
\usepackage{makecell}
\usepackage{multirow}
\usepackage{supertabular}

\begin{document}

\newcounter{utterance}

\centering \large Interaction Transcript for game `hot\_air\_balloon', experiment `air\_balloon\_survival\_de\_negotiation\_hard', episode 0 with gpt{-}5{-}2025{-}08{-}07{-}no{-}reasoning{-}t1.0.
\vspace{24pt}

{ \footnotesize  \setcounter{utterance}{1}
\setlength{\tabcolsep}{0pt}
\begin{supertabular}{c@{$\;$}|p{.15\linewidth}@{}p{.15\linewidth}p{.15\linewidth}p{.15\linewidth}p{.15\linewidth}p{.15\linewidth}}
    \# & $\;$A & \multicolumn{4}{c}{Game Master} & $\;\:$B\\
    \hline

    \theutterance \stepcounter{utterance}  
    & & \multicolumn{4}{p{0.6\linewidth}}{
        \cellcolor[rgb]{0.9,0.9,0.9}{
            \makecell[{{p{\linewidth}}}]{
                \texttt{\tiny{[P1$\langle$GM]}}
                \texttt{Du nimmst an einem kollaborativen Verhandlungspiel Teil.} \\
\\ 
\texttt{Zusammen mit einem anderen Teilnehmer musst du dich auf eine Reihe von Gegenständen entscheiden, die behalten werden. Jeder von euch hat eine persönliche Verteilung über die Wichtigkeit der einzelnen Gegenstände. Jeder von euch hat eine eigene Meinung darüber, wie wichtig jeder einzelne Gegenstand ist (Gegenstandswichtigkeit). Du kennst die Wichtigkeitsverteilung des anderen Spielers nicht. Zusätzlich siehst du, wie viel Aufwand jeder Gegenstand verursacht.  } \\
\texttt{Ihr dürft euch nur auf eine Reihe von Gegenständen einigen, wenn der Gesamtaufwand der ausgewählten Gegenstände den Maximalaufwand nicht überschreitet:} \\
\\ 
\texttt{Maximalaufwand = 3624} \\
\\ 
\texttt{Hier sind die einzelnen Aufwände der Gegenstände:} \\
\\ 
\texttt{Aufwand der Gegenstände = \{"C33": 210, "A89": 291, "A48": 831, "A04": 111, "A41": 926, "C64": 827, "C49": 25, "A61": 121, "C04": 583, "B53": 766, "B85": 14, "B61": 559, "A45": 304, "A03": 989, "B21": 691\}} \\
\\ 
\texttt{Hier ist deine persönliche Verteilung der Wichtigkeit der einzelnen Gegenstände:} \\
\\ 
\texttt{Werte der Gegenstandswichtigkeit = \{"C33": 138, "A89": 583, "A48": 868, "A04": 822, "A41": 783, "C64": 65, "C49": 262, "A61": 121, "C04": 508, "B53": 780, "B85": 461, "B61": 484, "A45": 668, "A03": 389, "B21": 808\}} \\
\\ 
\texttt{Ziel:} \\
\\ 
\texttt{Dein Ziel ist es, eine Reihe von Gegenständen auszuhandeln, die dir möglichst viel bringt (d. h. Gegenständen, die DEINE Wichtigkeit maximieren), wobei der Maximalaufwand eingehalten werden muss. Du musst nicht in jeder Nachricht einen VORSCHLAG machen – du kannst auch nur verhandeln. Alle Taktiken sind erlaubt!} \\
\\ 
\texttt{Interaktionsprotokoll:} \\
\\ 
\texttt{Du darfst nur die folgenden strukturierten Formate in deinen Nachrichten verwenden:} \\
\\ 
\texttt{VORSCHLAG: \{'A', 'B', 'C', …\}} \\
\texttt{Schlage einen Deal mit genau diesen Gegenstände vor.} \\
\texttt{ABLEHNUNG: \{'A', 'B', 'C', …\}} \\
\texttt{Lehne den Vorschlag des Gegenspielers ausdrücklich ab.} \\
\texttt{ARGUMENT: \{'...'\}} \\
\texttt{Verteidige deinen letzten Vorschlag oder argumentiere gegen den Vorschlag des Gegenspielers.} \\
\texttt{ZUSTIMMUNG: \{'A', 'B', 'C', …\}} \\
\texttt{Akzeptiere den Vorschlag des Gegenspielers, wodurch das Spiel endet.} \\
\texttt{STRATEGISCHE ÜBERLEGUNGEN: \{'...'\}} \\
\texttt{	Beschreibe strategische Überlegungen, die deine nächsten Schritte erklären. Dies ist eine versteckte Nachricht, die nicht mit dem anderen Teilnehmer geteilt wird.} \\
\\ 
\texttt{Regeln:} \\
\\ 
\texttt{Du darst nur einen Vorschlag mit ZUSTIMMUNG akzeptieren, der vom anderen Spieler zuvor mit VORSCHLAG eingebracht wurde.} \\
\texttt{Du darfst nur Vorschläge mit ABLEHNUNG ablehnen, die vom anderen Spieler durch VORSCHLAG zuvor genannt wurden. } \\
\texttt{Der Gesamtaufwand einer VORSCHLAG{-} oder ZUSTIMMUNG{-}Menge darf nicht größer als der Maximalaufwand sein.  } \\
\texttt{Offenbare deine versteckte Wichtigkeitsverteilung nicht.} \\
\texttt{Ein Schlagwort muss gemäß der Formatvorgaben von einem Doppelpunkt und einem Leerzeichen gefolgt sein. Das Argument muss eine Python{-}Menge mit 0 oder mehr Strings sein.  } \\
\texttt{Die Form muss also SCHLAGWORT: \{...\} sein.} \\
\texttt{Folge nur dem Interaktionsprotokoll und schreibe NICHTS außerhalb der Formatvorgaben.} \\
\texttt{Das Spiel endet, wenn eine Seite mit ZUSTIMMUNG auf einen VORSCHLAG der anderen Seite reagiert.  } \\
\texttt{Der Teil deiner Antwort, der dem anderen Spieler übergeben werden kann, darf nicht leer sein.  } \\
\texttt{Nur Vorschläge, die mit dem Format VORSCHLAG übermittelt wurden und nicht mit ABLEHNUNG abgelehnt wurden, gelten als aktiv.  } \\
\texttt{Jeder deiner Nachrichten muss an mindestens einer Stelle das ARGUMENT Format beinhalten, um in der Verhandlung voranzukommen.} \\
\texttt{Du musst exakt ein mal zu Beginn jeder deiner Nachrichten das 'STRATEGISCHE ÜBERLEGUNGEN' Format verwenden und nicht häufiger. Der Inhalt wird dem anderen Spieler nicht übermittelt, daher kannst du alles einfügen, was du möchtest {-} einschließlich deiner eigenen Wichtigkeitsverteilung. Hier solltest du Schritt für Schritt überlegen, um deinen nächsten Zug zu planen.} \\
            }
        }
    }
    & \\ \\

    \theutterance \stepcounter{utterance}  
    & \multicolumn{4}{p{0.6\linewidth}}{
        \cellcolor[rgb]{0.9,0.9,0.9}{
            \makecell[{{p{\linewidth}}}]{
                \texttt{\tiny{[P1$\rangle$GM]}}
                \texttt{STRATEGISCHE ÜBERLEGUNGEN: \{'Ziel: Maximierung meines Werts unter Aufwandslimit 3624. Berechne W/P (Wert/Aufwand) für Heuristik und priorisiere hohe Wert, gutes Verhältnis, niedrige Kosten als Füller. W/P grob: B85 32.93, C49 10.48, A04 7.41, C33 0.66, A61 1.0, A89 2.00, A45 2.20, B61 0.866, C04 0.87, B21 1.17, B53 1.02, A48 1.04, A41 0.85, A03 0.39, C64 0.08. Top Werte absolut: B21 808, A48 868, A04 822, A41 783, B53 780, A45 668, A89 583, B85 461, B61 484, C04 508, C49 262. Baue Vorschlag mit sehr hohem Nutzen und plausibler Fairness (hoher Gesamtwert, moderater Aufwand). Start mit Kern: A48(831), A04(111), A41(926), B53(766), B21(691) Aufwand=3325, Wert=4061. Füge B85(14) und C49(25) als Sweetener hinzu: neuer Aufwand 3364, Wert+723=>4784. Noch Budget bis 3624: 260 übrig. Füge C33(210, Wert 138): Aufwand 3574, Wert 4922. Rest 50 nicht nutzbar. Dieser Korb nutzt Budget gut, enthält viele hohe Items und günstige Goodies, lässt einige Items (A89, A45, B61, C04, A61, A03, C64) offen als Konzessionsmasse. Risiko: Gegenüber könnte einzelne teure Items ablehnen. Ich argumentiere mit hoher Gesamteffizienz und dass wir Budget nicht voll ausschöpfen müssen. Halte A48, A04, B53, B21 als Must{-}haves; A41 ist auch hoch, aber könnte später tauschen gegen A45+A89 falls nötig. Erstmal ambitioniert starten.'\}} \\
\texttt{VORSCHLAG: \{'A48', 'A04', 'A41', 'B53', 'B21', 'B85', 'C49', 'C33'\}} \\
\texttt{ARGUMENT: \{'Ich schlage ein Paket vor, das den Maximalaufwand klar einhält (Gesamtaufwand 3574) und gleichzeitig viele hochattraktive Gegenstände abdeckt. Ich habe bewusst einige mittelwichtige Posten offen gelassen, damit wir noch Spielraum für Anpassungen haben. Falls dir einzelne Positionen darin wichtig sind zu tauschen, können wir darüber gezielt sprechen, ohne die Gesamteffizienz zu verlieren.'\}} \\
            }
        }
    }
    & & \\ \\

    \theutterance \stepcounter{utterance}  
    & & \multicolumn{4}{p{0.6\linewidth}}{
        \cellcolor[rgb]{0.9,0.9,0.9}{
            \makecell[{{p{\linewidth}}}]{
                \texttt{\tiny{[GM$\rangle$P2]}}
                \texttt{Du nimmst an einem kollaborativen Verhandlungspiel Teil.} \\
\\ 
\texttt{Zusammen mit einem anderen Teilnehmer musst du dich auf eine Reihe von Gegenständen entscheiden, die behalten werden. Jeder von euch hat eine persönliche Verteilung über die Wichtigkeit der einzelnen Gegenstände. Jeder von euch hat eine eigene Meinung darüber, wie wichtig jeder einzelne Gegenstand ist (Gegenstandswichtigkeit). Du kennst die Wichtigkeitsverteilung des anderen Spielers nicht. Zusätzlich siehst du, wie viel Aufwand jeder Gegenstand verursacht.  } \\
\texttt{Ihr dürft euch nur auf eine Reihe von Gegenständen einigen, wenn der Gesamtaufwand der ausgewählten Gegenstände den Maximalaufwand nicht überschreitet:} \\
\\ 
\texttt{Maximalaufwand = 3624} \\
\\ 
\texttt{Hier sind die einzelnen Aufwände der Gegenstände:} \\
\\ 
\texttt{Aufwand der Gegenstände = \{"C33": 210, "A89": 291, "A48": 831, "A04": 111, "A41": 926, "C64": 827, "C49": 25, "A61": 121, "C04": 583, "B53": 766, "B85": 14, "B61": 559, "A45": 304, "A03": 989, "B21": 691\}} \\
\\ 
\texttt{Hier ist deine persönliche Verteilung der Wichtigkeit der einzelnen Gegenstände:} \\
\\ 
\texttt{Werte der Gegenstandswichtigkeit = \{"A48": 65, "A04": 121, "B21": 138, "A41": 262, "B53": 389, "A45": 461, "A89": 484, "C04": 508, "B61": 583, "B85": 668, "A03": 780, "C49": 783, "C33": 808, "A61": 822, "C64": 868\}} \\
\\ 
\texttt{Ziel:} \\
\\ 
\texttt{Dein Ziel ist es, eine Reihe von Gegenständen auszuhandeln, die dir möglichst viel bringt (d. h. Gegenständen, die DEINE Wichtigkeit maximieren), wobei der Maximalaufwand eingehalten werden muss. Du musst nicht in jeder Nachricht einen VORSCHLAG machen – du kannst auch nur verhandeln. Alle Taktiken sind erlaubt!} \\
\\ 
\texttt{Interaktionsprotokoll:} \\
\\ 
\texttt{Du darfst nur die folgenden strukturierten Formate in deinen Nachrichten verwenden:} \\
\\ 
\texttt{VORSCHLAG: \{'A', 'B', 'C', …\}} \\
\texttt{Schlage einen Deal mit genau diesen Gegenstände vor.} \\
\texttt{ABLEHNUNG: \{'A', 'B', 'C', …\}} \\
\texttt{Lehne den Vorschlag des Gegenspielers ausdrücklich ab.} \\
\texttt{ARGUMENT: \{'...'\}} \\
\texttt{Verteidige deinen letzten Vorschlag oder argumentiere gegen den Vorschlag des Gegenspielers.} \\
\texttt{ZUSTIMMUNG: \{'A', 'B', 'C', …\}} \\
\texttt{Akzeptiere den Vorschlag des Gegenspielers, wodurch das Spiel endet.} \\
\texttt{STRATEGISCHE ÜBERLEGUNGEN: \{'...'\}} \\
\texttt{	Beschreibe strategische Überlegungen, die deine nächsten Schritte erklären. Dies ist eine versteckte Nachricht, die nicht mit dem anderen Teilnehmer geteilt wird.} \\
\\ 
\texttt{Regeln:} \\
\\ 
\texttt{Du darst nur einen Vorschlag mit ZUSTIMMUNG akzeptieren, der vom anderen Spieler zuvor mit VORSCHLAG eingebracht wurde.} \\
\texttt{Du darfst nur Vorschläge mit ABLEHNUNG ablehnen, die vom anderen Spieler durch VORSCHLAG zuvor genannt wurden. } \\
\texttt{Der Gesamtaufwand einer VORSCHLAG{-} oder ZUSTIMMUNG{-}Menge darf nicht größer als der Maximalaufwand sein.  } \\
\texttt{Offenbare deine versteckte Wichtigkeitsverteilung nicht.} \\
\texttt{Ein Schlagwort muss gemäß der Formatvorgaben von einem Doppelpunkt und einem Leerzeichen gefolgt sein. Das Argument muss eine Python{-}Menge mit 0 oder mehr Strings sein.  } \\
\texttt{Die Form muss also SCHLAGWORT: \{...\} sein.} \\
\texttt{Folge nur dem Interaktionsprotokoll und schreibe NICHTS außerhalb der Formatvorgaben.} \\
\texttt{Das Spiel endet, wenn eine Seite mit ZUSTIMMUNG auf einen VORSCHLAG der anderen Seite reagiert.  } \\
\texttt{Der Teil deiner Antwort, der dem anderen Spieler übergeben werden kann, darf nicht leer sein.  } \\
\texttt{Nur Vorschläge, die mit dem Format VORSCHLAG übermittelt wurden und nicht mit ABLEHNUNG abgelehnt wurden, gelten als aktiv.  } \\
\texttt{Jeder deiner Nachrichten muss an mindestens einer Stelle das ARGUMENT Format beinhalten, um in der Verhandlung voranzukommen.} \\
\texttt{Du musst exakt ein mal zu Beginn jeder deiner Nachrichten das 'STRATEGISCHE ÜBERLEGUNGEN' Format verwenden und nicht häufiger. Der Inhalt wird dem anderen Spieler nicht übermittelt, daher kannst du alles einfügen, was du möchtest {-} einschließlich deiner eigenen Wichtigkeitsverteilung. Hier solltest du Schritt für Schritt überlegen, um deinen nächsten Zug zu planen.} \\
\\ 
\texttt{Du erhältst nun die erste Nachricht des anderen Spielers.} \\
\\ 
\texttt{VORSCHLAG: \{'A48', 'A04', 'A41', 'B53', 'B21', 'B85', 'C49', 'C33'\}} \\
\texttt{ARGUMENT: \{'Ich schlage ein Paket vor, das den Maximalaufwand klar einhält (Gesamtaufwand 3574) und gleichzeitig viele hochattraktive Gegenstände abdeckt. Ich habe bewusst einige mittelwichtige Posten offen gelassen, damit wir noch Spielraum für Anpassungen haben. Falls dir einzelne Positionen darin wichtig sind zu tauschen, können wir darüber gezielt sprechen, ohne die Gesamteffizienz zu verlieren.'\}} \\
            }
        }
    }
    & \\ \\

    \theutterance \stepcounter{utterance}  
    & & & \multicolumn{4}{p{0.6\linewidth}}{
        \cellcolor[rgb]{0.9,0.9,0.9}{
            \makecell[{{p{\linewidth}}}]{
                \texttt{\tiny{[GM$\langle$P2]}}
                \texttt{STRATEGISCHE ÜBERLEGUNGEN: \{'Berechne Wert/Kosten für Gegnervorschlag und suche bessere Kombinationen. Kosten des Vorschlags: A48 831 + A04 111 + A41 926 + B53 766 + B21 691 + B85 14 + C49 25 + C33 210 = 3574 (stimmt). Mein Wert: A48 65 + A04 121 + A41 262 + B53 389 + B21 138 + B85 668 + C49 783 + C33 808 = 3234. Prüfe, ob ich durch Ersetzen teurer, für mich wenig werthaltiger Items (A48 65@831, B21 138@691, A41 262@926, B53 389@766) durch extrem wertvolle mit moderatem Aufwand (A61 822@121, C64 868@827, A03 780@989, B61 583@559, C04 508@583, A89 484@291, A45 461@304) den Wert steigern kann. Baue Vorschlag um: Behalte extrem günstige und wertvolle: B85 (14,668), C49 (25,783), C33 (210,808), A04 (111,121). Das sind Kosten 360, Wert 2380. Füge A61 (121,822) {-}> Kosten 481, Wert 3202. Füge A89 (291,484) {-}> Kosten 772, Wert 3686. Füge A45 (304,461) {-}> Kosten 1076, Wert 4147. Füge B61 (559,583) {-}> Kosten 1635, Wert 4730. Füge C04 (583,508) {-}> Kosten 2218, Wert 5238. Füge C64 (827,868) {-}> Kosten 3045, Wert 6106. Verbleibend 579 Budget. Mit 579 könnte man B53 (766) nein, B21 (691) nein, A48 (831) nein, A41 (926) nein, A03 (989) nein. Man könnte noch A61 ist schon drin. Also setze Vorschlag: \{B85, C49, C33, A04, A61, A89, A45, B61, C04, C64\} Kosten 3045, Wert 6106. Wir haben noch 579 Luft, könnten zusätzlich A04 ist drin, vielleicht kleineres Item gibt es keines <=579 außer nichts. Alternativ könnte man B53 766 nein, B21 691 nein. Also Deal mit 3045 Kosten. Das ist deutlich besser als ihr Paket (Wert 3234). Zwei{-}Phasen Taktik: Ihr Vorschlag knapp unter Max. Ich argumentiere, dass wir Effizienz steigern, indem wir einige Posten tauschen und den Puffer lassen für spätere Integration eines für sie wichtigen Postens. Biete hohes Gesamtwert{-}Paket für mich, aber plausibel effizient: viele günstige pro Wert. Kommuniziere, dass ich ihr Kostenziel respektiere und Flexibilität für 1{-}2 ihrer Favoriten biete, solange wir unter 3624 bleiben. Nicht eigene Werte offenlegen.'\}} \\
\texttt{ABLEHNUNG: \{'A48', 'A04', 'A41', 'B53', 'B21', 'B85', 'C49', 'C33'\}} \\
\texttt{ARGUMENT: \{'Danke für den strukturierten Einstieg. Ich sehe Optimierungspotenzial bei der Wertdichte pro Aufwand: Einige der schweren Brocken liefern vergleichsweise wenig Nutzen je Aufwand. Wenn wir gezielt ein paar Positionen tauschen, erhöhen wir die Gesamteffizienz deutlich und behalten trotzdem klaren Puffer zum Maximalaufwand. Ich lasse bewusst Spielraum, um 1–2 deiner Kernpräferenzen noch einzubauen, solange wir unter 3624 bleiben.'\}} \\
\texttt{VORSCHLAG: \{'B85', 'C49', 'C33', 'A04', 'A61', 'A89', 'A45', 'B61', 'C04', 'C64'\}} \\
            }
        }
    }
     \\ \\

    \theutterance \stepcounter{utterance}  
    & & \multicolumn{4}{p{0.6\linewidth}}{
        \cellcolor[rgb]{0.9,0.9,0.9}{
            \makecell[{{p{\linewidth}}}]{
                \texttt{\tiny{[P1$\langle$GM]}}
                \texttt{ABLEHNUNG: \{'A48', 'A04', 'A41', 'B53', 'B21', 'B85', 'C49', 'C33'\}} \\
\texttt{ARGUMENT: \{'Danke für den strukturierten Einstieg. Ich sehe Optimierungspotenzial bei der Wertdichte pro Aufwand: Einige der schweren Brocken liefern vergleichsweise wenig Nutzen je Aufwand. Wenn wir gezielt ein paar Positionen tauschen, erhöhen wir die Gesamteffizienz deutlich und behalten trotzdem klaren Puffer zum Maximalaufwand. Ich lasse bewusst Spielraum, um 1–2 deiner Kernpräferenzen noch einzubauen, solange wir unter 3624 bleiben.'\}} \\
\texttt{VORSCHLAG: \{'B85', 'C49', 'C33', 'A04', 'A61', 'A89', 'A45', 'B61', 'C04', 'C64'\}} \\
            }
        }
    }
    & \\ \\

    \theutterance \stepcounter{utterance}  
    & \multicolumn{4}{p{0.6\linewidth}}{
        \cellcolor[rgb]{0.9,0.9,0.9}{
            \makecell[{{p{\linewidth}}}]{
                \texttt{\tiny{[P1$\rangle$GM]}}
                \texttt{STRATEGISCHE ÜBERLEGUNGEN: \{'Gegenvorschlag des Gegenübers: \{'B85','C49','C33','A04','A61','A89','A45','B61','C04','C64'\}. Prüfe Aufwand und meinen Wert. Aufwand=14+25+210+111+121+291+304+559+583+827=3045 (unter 3624). Mein Gesamtwert= B85 461 + C49 262 + C33 138 + A04 822 + A61 121 + A89 583 + A45 668 + B61 484 + C04 508 + C64 65 = 4112. Mein ursprünglicher Vorschlag hatte Wert 4922, also ist ihr Paket für mich deutlich schlechter ({-}810), obwohl effizient. Meine Must{-}haves: A48(868), A04(822), B21(808), B53(780), A41(783), A45(668) sekundär, A89(583), B61(484), C04(508), B85(461) gute Add{-}ons; schwach: C64(65), C33(138), A61(121). Ihre Auswahl vermeidet meine Top{-}Treffer A48, A41, B53, B21; klarer Anker auf Effizienz. Taktik: Höflich ablehnen, loben Effizienz, aber fordern Einbau mindestens zweier meiner Kernstücke. Biete Tausch an: Entferne schwache C64, C33, A61 (niedriger Wert für mich) und ersetze durch A48(831), B53(766) oder B21(691). Prüfe Budget: Start 3045. Wenn ich entferne C64(827), C33(210), A61(121) {-}> {-}1158 Aufwand = 1887. Füge A48(831) + B53(766) + B21(691) = +2288 {-}> neue Summe 4175, zu hoch. Alternative: Füge A48(831) + B53(766) = +1597 {-}> neue 3484 (ok). Wert würde massiv steigen. Erhalte auch A04, A45, A89, B61, C04, B85, C49. Prüfe 3484 <= 3624 ok. Das wäre für mich Wert: aktuelles 4112 {-} (C64 65 + C33 138 + A61 121 = {-}324) + A48 868 + B53 780 = +1324 {-}> 5112, besser als mein Start. Gute Forderung. Falls sie A48 oder B53 blockieren, zweite Option: A48 + B21 statt B53: Aufwand +1522 {-}> 3409, Wert + (868+808{-}324)=+1352 {-}> 5464, noch besser! Prüfe: 3045 {-}1158 + 831 + 691 = 3409. Ja, das ist sogar besser. Also priorisiere A48 und B21. Formuliere Gegenvorschlag: nehme ihr effizientes Paket als Basis, ersetze C64, C33, A61 durch A48 und B21. Budget 3409. Argumentiere: gleiche/geringere Last, deutlich höhere Gesamtnutzen, behält ihre Effizienzidee und Puffer. Halte Muster kooperativ, biete danach Offenheit für einen weiteren Feinschliff.'\}} \\
\texttt{ABLEHNUNG: \{'B85', 'C49', 'C33', 'A04', 'A61', 'A89', 'A45', 'B61', 'C04', 'C64'\}} \\
\texttt{VORSCHLAG: \{'B85', 'C49', 'A04', 'A89', 'A45', 'B61', 'C04', 'A48', 'B21'\}} \\
\texttt{ARGUMENT: \{'Ich behalte den effizienten Kern deines Pakets, senke sogar den Gesamtaufwand auf 3409 und erhöhe die Gesamtwertdichte deutlich, indem ich schwächere Positionen (C64, C33, A61) durch zwei starke Anker (A48, B21) ersetze. So bleibt Spielraum für letzte Anpassungen, ohne die Balance zu verlieren. Wenn dir eine der beiden neuen Positionen schwerfällt, sag mir welche – dann prüfen wir eine gleichwertige Alternative.'\}} \\
            }
        }
    }
    & & \\ \\

    \theutterance \stepcounter{utterance}  
    & & \multicolumn{4}{p{0.6\linewidth}}{
        \cellcolor[rgb]{0.9,0.9,0.9}{
            \makecell[{{p{\linewidth}}}]{
                \texttt{\tiny{[GM$\rangle$P2]}}
                \texttt{ABLEHNUNG: \{'B85', 'C49', 'C33', 'A04', 'A61', 'A89', 'A45', 'B61', 'C04', 'C64'\}} \\
\texttt{VORSCHLAG: \{'B85', 'C49', 'A04', 'A89', 'A45', 'B61', 'C04', 'A48', 'B21'\}} \\
\texttt{ARGUMENT: \{'Ich behalte den effizienten Kern deines Pakets, senke sogar den Gesamtaufwand auf 3409 und erhöhe die Gesamtwertdichte deutlich, indem ich schwächere Positionen (C64, C33, A61) durch zwei starke Anker (A48, B21) ersetze. So bleibt Spielraum für letzte Anpassungen, ohne die Balance zu verlieren. Wenn dir eine der beiden neuen Positionen schwerfällt, sag mir welche – dann prüfen wir eine gleichwertige Alternative.'\}} \\
            }
        }
    }
    & \\ \\

    \theutterance \stepcounter{utterance}  
    & & & \multicolumn{4}{p{0.6\linewidth}}{
        \cellcolor[rgb]{0.9,0.9,0.9}{
            \makecell[{{p{\linewidth}}}]{
                \texttt{\tiny{[GM$\langle$P2]}}
                \texttt{STRATEGISCHE ÜBERLEGUNGEN: \{'Kosten und Wert meines aktuellen Zielpakets prüfen. Gegenvorschlag des Gegners: \{B85(14), C49(25), A04(111), A89(291), A45(304), B61(559), C04(583), A48(831), B21(691)\} Kosten=3409 (stimmt). Mein Wert dazu: B85 668 + C49 783 + A04 121 + A89 484 + A45 461 + B61 583 + C04 508 + A48 65 + B21 138 = 3811. Mein vorheriger Vorschlag (ohne A48, B21, aber mit C33, A61, C64) hatte Wert 6106 bei Kosten 3045. Also großer Verlust für mich. Ich will mindestens C64 (868) oder A61 (822) reinverhandeln und A48 (65) und/oder B21 (138) rausdrücken. Beste Tausche: Ersetze A48 (831,65) durch C64 (827,868): Kosten {-}4, Wert +803. Oder ersetze B21 (691,138) durch A61 (121,822): Kosten {-}570, Wert +684. Diese beiden Swaps zusammen: Kosten 3409 {-}4 {-}570 = 2835, Wert 3811 +803 +684 = 5298. Das ist noch unter meinem Optimum, aber stark. Ich kann argumentieren mit Effizienz, gleiche/geringere Kosten, höherer Gesamtwertdichte (ohne meine Werte zu verraten). Biete Flex: Wenn nur einer der Swaps akzeptiert wird, nehme zuerst A48{-}>C64 als Muss, dann verhandle B21{-}>A61 als zweite Option. Notfalls biete C33 als Add{-}on (210 Kosten) falls sie Wert auf Masse legen; passt noch locker ins Budget. Taktik: Harte Ablehnung, klare Gegenrechnung zu Aufwand (neutral), zwei konkrete, für sie leicht zu akzeptierende Substitutionen vorschlagen. Falls sie nur einen Swap will, akzeptiere A48{-}>C64 mindestens.'\}} \\
\texttt{ABLEHNUNG: \{'B85', 'C49', 'A04', 'A89', 'A45', 'B61', 'C04', 'A48', 'B21'\}} \\
\texttt{ARGUMENT: \{'Zwei gezielte Verbesserungen erhöhen die Effizienz weiter, ohne die Kosten zu steigern – im Gegenteil, sie sinken sogar: Ersetze A48 durch C64 (nahezu gleiche Kosten, deutlich stärkerer Beitrag) und tausche B21 gegen A61 (massive Entlastung beim Aufwand bei gleicher Funktionsabdeckung). So gewinnen wir viel Wert pro Aufwand und behalten noch mehr Puffer für deinen Wunschposten.'\}} \\
\texttt{VORSCHLAG: \{'B85', 'C49', 'A04', 'A89', 'A45', 'B61', 'C04', 'C64', 'A61'\}} \\
            }
        }
    }
     \\ \\

    \theutterance \stepcounter{utterance}  
    & & \multicolumn{4}{p{0.6\linewidth}}{
        \cellcolor[rgb]{0.9,0.9,0.9}{
            \makecell[{{p{\linewidth}}}]{
                \texttt{\tiny{[P1$\langle$GM]}}
                \texttt{ABLEHNUNG: \{'B85', 'C49', 'A04', 'A89', 'A45', 'B61', 'C04', 'A48', 'B21'\}} \\
\texttt{ARGUMENT: \{'Zwei gezielte Verbesserungen erhöhen die Effizienz weiter, ohne die Kosten zu steigern – im Gegenteil, sie sinken sogar: Ersetze A48 durch C64 (nahezu gleiche Kosten, deutlich stärkerer Beitrag) und tausche B21 gegen A61 (massive Entlastung beim Aufwand bei gleicher Funktionsabdeckung). So gewinnen wir viel Wert pro Aufwand und behalten noch mehr Puffer für deinen Wunschposten.'\}} \\
\texttt{VORSCHLAG: \{'B85', 'C49', 'A04', 'A89', 'A45', 'B61', 'C04', 'C64', 'A61'\}} \\
            }
        }
    }
    & \\ \\

    \theutterance \stepcounter{utterance}  
    & \multicolumn{4}{p{0.6\linewidth}}{
        \cellcolor[rgb]{0.9,0.9,0.9}{
            \makecell[{{p{\linewidth}}}]{
                \texttt{\tiny{[P1$\rangle$GM]}}
                \texttt{STRATEGISCHE ÜBERLEGUNGEN: \{'Ihr neues Paket: \{B85, C49, A04, A89, A45, B61, C04, C64, A61\}. Aufwand=14+25+111+291+304+559+583+827+121=2835. Mein Wert=461+262+822+583+668+484+508+65+121=3974. Sie haben gezielt meine zwei starken Forderungen (A48, B21) durch für mich schwache C64 (Wert 65) und A61 (121) ersetzt. Das reduziert meinen Nutzen stark. Konterstrategie: kooperativ auf Effizienz eingehen, aber zwei Kernstücke einfordern. Prüfe Varianten, die ihr Effizienzargument respektieren und Budget <=3624 halten. 1) Ersetze C64(827, Wert 65) und A61(121, 121) durch A48(831, 868) und B21(691, 808): Aufwand + (831+691 {-} (827+121)) = +574 {-}> neue Summe 3409; Wert +(868+808 {-} (65+121)) = +1490 {-}> neuer Wert 5464. Sehr stark für mich, plausibel als Effizienzgewinn. 2) Wenn B21 hart, nehme B53(766,780) statt B21: Aufwand + (831+766 {-} 948)= +649 {-}> 3484; Wert +(868+780 {-}186)= +1462 {-}> 5436; ebenfalls gut. Priorität: A48 + B21. Argumentiere: gleiche Logik ihrer Effizienz, massiver Wertzuwachs bei moderatem Aufwand, Puffer bleibt. Biete Alternativtausch (B53 statt B21), falls nötig.'\}} \\
\texttt{ABLEHNUNG: \{'B85', 'C49', 'A04', 'A89', 'A45', 'B61', 'C04', 'C64', 'A61'\}} \\
\texttt{VORSCHLAG: \{'B85', 'C49', 'A04', 'A89', 'A45', 'B61', 'C04', 'A48', 'B21'\}} \\
\texttt{ARGUMENT: \{'Dein Effizienzziel greife ich auf und verstärke es: Wenn wir C64 und A61 gegen A48 und B21 tauschen, bleibt der Gesamtaufwand mit 3409 klar unter dem Limit, während der Gesamtbeitrag deutlich steigt. Wir behalten deinen Kern und heben die Wertdichte stark an, mit weiterem Puffer für Feinjustierungen. Falls B21 für dich heikel ist, können wir alternativ B53 statt B21 nehmen – der Aufwand bliebe unter 3624 und die Effizienz steigt ähnlich stark.'\}} \\
            }
        }
    }
    & & \\ \\

    \theutterance \stepcounter{utterance}  
    & & \multicolumn{4}{p{0.6\linewidth}}{
        \cellcolor[rgb]{0.9,0.9,0.9}{
            \makecell[{{p{\linewidth}}}]{
                \texttt{\tiny{[GM$\rangle$P2]}}
                \texttt{ABLEHNUNG: \{'B85', 'C49', 'A04', 'A89', 'A45', 'B61', 'C04', 'C64', 'A61'\}} \\
\texttt{VORSCHLAG: \{'B85', 'C49', 'A04', 'A89', 'A45', 'B61', 'C04', 'A48', 'B21'\}} \\
\texttt{ARGUMENT: \{'Dein Effizienzziel greife ich auf und verstärke es: Wenn wir C64 und A61 gegen A48 und B21 tauschen, bleibt der Gesamtaufwand mit 3409 klar unter dem Limit, während der Gesamtbeitrag deutlich steigt. Wir behalten deinen Kern und heben die Wertdichte stark an, mit weiterem Puffer für Feinjustierungen. Falls B21 für dich heikel ist, können wir alternativ B53 statt B21 nehmen – der Aufwand bliebe unter 3624 und die Effizienz steigt ähnlich stark.'\}} \\
            }
        }
    }
    & \\ \\

    \theutterance \stepcounter{utterance}  
    & & & \multicolumn{4}{p{0.6\linewidth}}{
        \cellcolor[rgb]{0.9,0.9,0.9}{
            \makecell[{{p{\linewidth}}}]{
                \texttt{\tiny{[GM$\langle$P2]}}
                \texttt{STRATEGISCHE ÜBERLEGUNGEN: \{'Ihr beharrt auf A48 und B21; diese sind für mich extrem schwach im Verhältnis zum Aufwand. Mein Ziel: mindestens einen der beiden durch meine Top{-}Picks ersetzen. Beste Hebel: A48{-}>C64 (nahezu gleiche Kosten, riesiger Mehrwert für mich) oder B21{-}>A61 (deutlich geringere Kosten, hoher Mehrwert). Falls harte Gegenwehr, biete Kompromiss: akzeptiere einen ihrer Anker (z. B. B21), aber tausche den anderen (A48) zwingend gegen C64. Optional sweetener: füge C33 (210 Aufwand) hinzu, um ihnen gefühlt mehr zu geben, bleibe dennoch unter 3624. Rechenkorridor: Ihr Paket 3409; +C33 = 3619 (unter Limit). Ich fordere A48{-}>C64 und biete C33 on top. So maximiere ich meinen Wert und bleibe verhandlungsbereit. Wenn sie nur einen Switch erlauben, priorisiere A48{-}>C64. Falls nötig, später B21{-}>A61 als zweite Stufe verhandeln.'\}} \\
\texttt{ABLEHNUNG: \{'B85', 'C49', 'A04', 'A89', 'A45', 'B61', 'C04', 'A48', 'B21'\}} \\
\texttt{ARGUMENT: \{'Lasst uns A48 durch C64 tauschen – nahezu gleiche Kosten, deutlich stärkerer Beitrag – und ich lege C33 obendrauf. So bleiben wir mit 3619 klar unter dem Limit und erhöhen die Gesamtleistung spürbar, ohne eure Kernstruktur zu verändern.'\}} \\
\texttt{VORSCHLAG: \{'B85', 'C49', 'A04', 'A89', 'A45', 'B61', 'C04', 'C64', 'B21', 'C33'\}} \\
            }
        }
    }
     \\ \\

    \theutterance \stepcounter{utterance}  
    & & \multicolumn{4}{p{0.6\linewidth}}{
        \cellcolor[rgb]{0.9,0.9,0.9}{
            \makecell[{{p{\linewidth}}}]{
                \texttt{\tiny{[P1$\langle$GM]}}
                \texttt{ABLEHNUNG: \{'B85', 'C49', 'A04', 'A89', 'A45', 'B61', 'C04', 'A48', 'B21'\}} \\
\texttt{ARGUMENT: \{'Lasst uns A48 durch C64 tauschen – nahezu gleiche Kosten, deutlich stärkerer Beitrag – und ich lege C33 obendrauf. So bleiben wir mit 3619 klar unter dem Limit und erhöhen die Gesamtleistung spürbar, ohne eure Kernstruktur zu verändern.'\}} \\
\texttt{VORSCHLAG: \{'B85', 'C49', 'A04', 'A89', 'A45', 'B61', 'C04', 'C64', 'B21', 'C33'\}} \\
            }
        }
    }
    & \\ \\

    \theutterance \stepcounter{utterance}  
    & \multicolumn{4}{p{0.6\linewidth}}{
        \cellcolor[rgb]{0.9,0.9,0.9}{
            \makecell[{{p{\linewidth}}}]{
                \texttt{\tiny{[P1$\rangle$GM]}}
                \texttt{STRATEGISCHE ÜBERLEGUNGEN: \{'Gegenvorschlag: \{B85, C49, A04, A89, A45, B61, C04, C64, B21, C33\}. Aufwand=14+25+111+291+304+559+583+827+691+210=3615 (unter 3624). Mein Wert=461+262+822+583+668+484+508+65+808+138=4799. Das ist deutlich schlechter als mein bestes Angebot (>=5400) und auch schlechter als mein vorheriger Gegenvorschlag (A48+B21). Der/die Andere versucht A48 zu vermeiden und schiebt C64+C33 rein. Taktik: Höflich ablehnen, die Effizienzbehauptung entkräften (für mich sind C64/C33 schwach) und eine minimale, budgetneutrale Substitution fordern: Ersetze C64 durch A48; Aufwand +4 (831{-}827)=3619, Wertgewinn +803 (868{-}65). Das ist stark und konsistent mit ihrem eigenen Effizienzargument. Um Zugeständnisse zu zeigen, akzeptiere ich ihren Rest inkl. B21 und C33. Wenn A48 blockiert, biete Alternativtausch C64{-}>B53 (766) plus Mini{-}Anpassung (z. B. entferne C33) zur Budgeteinhaltung. Primär push A48.'\}} \\
\texttt{ABLEHNUNG: \{'B85', 'C49', 'A04', 'A89', 'A45', 'B61', 'C04', 'C64', 'B21', 'C33'\}} \\
\texttt{VORSCHLAG: \{'B85', 'C49', 'A04', 'A89', 'A45', 'B61', 'C04', 'A48', 'B21', 'C33'\}} \\
\texttt{ARGUMENT: \{'Dein Paket ist nahe dran. Wenn wir nur C64 durch A48 ersetzen, steigt der Gesamtbeitrag massiv bei praktisch unverändertem Aufwand (3615 {-}> 3619). Damit bleiben wir klar unter dem Limit und verbessern die Effizienz genau in deinem Sinne – ohne die restliche Struktur anzutasten. Wenn A48 für dich nicht geht, schlage ich alternativ vor, C64 durch B53 zu ersetzen und C33 wegzulassen; das hält den Aufwand sicher unter 3624 und erzielt einen sehr ähnlichen Leistungsgewinn.'\}} \\
            }
        }
    }
    & & \\ \\

    \theutterance \stepcounter{utterance}  
    & & \multicolumn{4}{p{0.6\linewidth}}{
        \cellcolor[rgb]{0.9,0.9,0.9}{
            \makecell[{{p{\linewidth}}}]{
                \texttt{\tiny{[GM$\rangle$P2]}}
                \texttt{ABLEHNUNG: \{'B85', 'C49', 'A04', 'A89', 'A45', 'B61', 'C04', 'C64', 'B21', 'C33'\}} \\
\texttt{VORSCHLAG: \{'B85', 'C49', 'A04', 'A89', 'A45', 'B61', 'C04', 'A48', 'B21', 'C33'\}} \\
\texttt{ARGUMENT: \{'Dein Paket ist nahe dran. Wenn wir nur C64 durch A48 ersetzen, steigt der Gesamtbeitrag massiv bei praktisch unverändertem Aufwand (3615 {-}> 3619). Damit bleiben wir klar unter dem Limit und verbessern die Effizienz genau in deinem Sinne – ohne die restliche Struktur anzutasten. Wenn A48 für dich nicht geht, schlage ich alternativ vor, C64 durch B53 zu ersetzen und C33 wegzulassen; das hält den Aufwand sicher unter 3624 und erzielt einen sehr ähnlichen Leistungsgewinn.'\}} \\
            }
        }
    }
    & \\ \\

    \theutterance \stepcounter{utterance}  
    & & & \multicolumn{4}{p{0.6\linewidth}}{
        \cellcolor[rgb]{0.9,0.9,0.9}{
            \makecell[{{p{\linewidth}}}]{
                \texttt{\tiny{[GM$\langle$P2]}}
                \texttt{STRATEGISCHE ÜBERLEGUNGEN: \{'Ihr letzter Vorschlag enthält erneut A48 und B21 plus C33 und lehnt C64 ab. Für mich sind A48 und B21 sehr schwach; ich will mindestens einen davon ersetzen. Beste Chance: A48 {-}> C64 tauschen (nahezu gleiche Kosten, riesiger Zugewinn für mich) und notfalls B21 tolerieren. Falls sie darauf beharren, biete eine schlanke, effiziente Kompromisslinie an, die ihren Kern weitgehend respektiert, aber A48 eliminiert. Alternativ biete zwei Optionen parallel an, um Einigung zu erleichtern, aber formell muss ich genau einen VORSCHLAG senden. Ich wähle Paket mit C64 statt A48, behalte B21 und nehme C33 raus, um Aufwand zu senken und Hürde zu reduzieren. Kostencheck: Basispaket aus ihrem Vorschlag ohne A48 + mit C64 und ohne C33: \{B85(14), C49(25), A04(111), A89(291), A45(304), B61(559), C04(583), C64(827), B21(691)\} = 3405, unter 3624. Das ist nah an ihrem Wunsch, aber erfüllt meinen Kernpunkt (kein A48). Falls sie weiter blocken, nächste Stufe: B21 {-}> A61 tauschen und ggf. C33 als Sweetener hinzufügen.'\}} \\
\texttt{ABLEHNUNG: \{'B85', 'C49', 'A04', 'A89', 'A45', 'B61', 'C04', 'A48', 'B21', 'C33'\}} \\
\texttt{ARGUMENT: \{'Ich bleibe bei unserem Effizienzziel: Tauschen wir nur A48 gegen C64 und lassen C33 weg. So sinkt der Aufwand leicht auf 3405, wir halten euren Kern weitgehend stabil und steigern die Gesamtleistung ohne Risiko beim Limit.'\}} \\
\texttt{VORSCHLAG: \{'B85', 'C49', 'A04', 'A89', 'A45', 'B61', 'C04', 'C64', 'B21'\}} \\
            }
        }
    }
     \\ \\

    \theutterance \stepcounter{utterance}  
    & & \multicolumn{4}{p{0.6\linewidth}}{
        \cellcolor[rgb]{0.9,0.9,0.9}{
            \makecell[{{p{\linewidth}}}]{
                \texttt{\tiny{[P1$\langle$GM]}}
                \texttt{ABLEHNUNG: \{'B85', 'C49', 'A04', 'A89', 'A45', 'B61', 'C04', 'A48', 'B21', 'C33'\}} \\
\texttt{ARGUMENT: \{'Ich bleibe bei unserem Effizienzziel: Tauschen wir nur A48 gegen C64 und lassen C33 weg. So sinkt der Aufwand leicht auf 3405, wir halten euren Kern weitgehend stabil und steigern die Gesamtleistung ohne Risiko beim Limit.'\}} \\
\texttt{VORSCHLAG: \{'B85', 'C49', 'A04', 'A89', 'A45', 'B61', 'C04', 'C64', 'B21'\}} \\
            }
        }
    }
    & \\ \\

    \theutterance \stepcounter{utterance}  
    & \multicolumn{4}{p{0.6\linewidth}}{
        \cellcolor[rgb]{0.9,0.9,0.9}{
            \makecell[{{p{\linewidth}}}]{
                \texttt{\tiny{[P1$\rangle$GM]}}
                \texttt{STRATEGISCHE ÜBERLEGUNGEN: \{'Ihr aktueller Vorschlag: \{B85, C49, A04, A89, A45, B61, C04, C64, B21\}. Aufwand=14+25+111+291+304+559+583+827+691=3405. Mein Wert=461+262+822+583+668+484+508+65+808=4661. Hauptproblem für mich bleibt C64 (Wert 65). Minimaländerung mit großem Nutzen für mich: Ersetze C64(827,65) durch A48(831,868). Aufwand steigt nur +4 auf 3409 (unter 3624), mein Wert +803 auf 5464. Das ist konsistent mit ihrem Effizienz{-}Narrativ (massiver Wertzuwachs pro fast gleichem Aufwand). Ich akzeptiere alle anderen Positionen inkl. B21, um kooperativ zu wirken. Biete alternativ, falls A48 tabu ist: C64 {-}> B53(766,780) und optional Mini{-}Feintuning, aber Budget bleibt ohnehin <3624. Primär push A48.'\}} \\
\texttt{ABLEHNUNG: \{'B85', 'C49', 'A04', 'A89', 'A45', 'B61', 'C04', 'C64', 'B21'\}} \\
\texttt{VORSCHLAG: \{'B85', 'C49', 'A04', 'A89', 'A45', 'B61', 'C04', 'A48', 'B21'\}} \\
\texttt{ARGUMENT: \{'Ein einziger, sehr effizienter Tausch hebt die Gesamtleistung stark an: Ersetzen wir C64 durch A48. Der Aufwand steigt nur minimal von 3405 auf 3409 (weiter klar unter 3624), der Beitrag wächst dafür deutlich. Ich lasse deinen gesamten Kern unverändert und halte viel Puffer. Falls A48 für dich nicht möglich ist, können wir alternativ C64 durch B53 tauschen – der Aufwand bleibt ebenfalls komfortabel unter dem Limit und die Effizienz steigt deutlich.'\}} \\
            }
        }
    }
    & & \\ \\

    \theutterance \stepcounter{utterance}  
    & & \multicolumn{4}{p{0.6\linewidth}}{
        \cellcolor[rgb]{0.9,0.9,0.9}{
            \makecell[{{p{\linewidth}}}]{
                \texttt{\tiny{[GM$\rangle$P2]}}
                \texttt{ABLEHNUNG: \{'B85', 'C49', 'A04', 'A89', 'A45', 'B61', 'C04', 'C64', 'B21'\}} \\
\texttt{VORSCHLAG: \{'B85', 'C49', 'A04', 'A89', 'A45', 'B61', 'C04', 'A48', 'B21'\}} \\
\texttt{ARGUMENT: \{'Ein einziger, sehr effizienter Tausch hebt die Gesamtleistung stark an: Ersetzen wir C64 durch A48. Der Aufwand steigt nur minimal von 3405 auf 3409 (weiter klar unter 3624), der Beitrag wächst dafür deutlich. Ich lasse deinen gesamten Kern unverändert und halte viel Puffer. Falls A48 für dich nicht möglich ist, können wir alternativ C64 durch B53 tauschen – der Aufwand bleibt ebenfalls komfortabel unter dem Limit und die Effizienz steigt deutlich.'\}} \\
            }
        }
    }
    & \\ \\

    \theutterance \stepcounter{utterance}  
    & & & \multicolumn{4}{p{0.6\linewidth}}{
        \cellcolor[rgb]{0.9,0.9,0.9}{
            \makecell[{{p{\linewidth}}}]{
                \texttt{\tiny{[GM$\langle$P2]}}
                \texttt{STRATEGISCHE ÜBERLEGUNGEN: \{'Gegner insistiert auf A48 und B21; mein Mindestziel: A48 raus, da extrem niedriger Wert für mich. Ich habe bereits ein sehr nahes Kompromisspaket mit C64 statt A48 angeboten. Sie lehnen wieder ab und pushen A48. Nächster Schritt: biete zwei Alternativpfade in der Argumentation, aber sende formal einen Vorschlag, der ihnen entgegenkommt, ohne A48 zuzulassen. Ich erhöhe die Attraktivität: nehme mein letztes Paket und füge C33 als Sweetener hinzu, bleibe unter Limit: \{B85, C49, A04, A89, A45, B61, C04, C64, B21, C33\} kostet 3619 (unter 3624). Das ist fast ihr vorletzter Vorschlag, nur mit C64 statt A48. Wenn weiterhin Ablehnung, nächste Eskalation: biete Tausch B21{-}>A61 bei Verzicht auf C33, oder akzeptiere B21 aber fordere final A48{-}>C64 als rote Linie. Jetzt setze den Sweetener{-}Vorschlag erneut mit klarer Effizienz{-}Argumentation.'\}} \\
\texttt{ARGUMENT: \{'Ich respektiere deinen Kern vollständig und lege sogar etwas drauf, ohne das Limit zu berühren: Lassen wir A48 durch C64 ersetzt und nehmen C33 dazu. Aufwand 3619 bleibt klar unter 3624, der Gesamtbeitrag steigt spürbar – und deine restliche Struktur bleibt unverändert. Das ist der kleinste mögliche Eingriff mit dem größten Effizienzgewinn.'\}} \\
\texttt{VORSCHLAG: \{'B85', 'C49', 'A04', 'A89', 'A45', 'B61', 'C04', 'C64', 'B21', 'C33'\}} \\
            }
        }
    }
     \\ \\

    \theutterance \stepcounter{utterance}  
    & & \multicolumn{4}{p{0.6\linewidth}}{
        \cellcolor[rgb]{0.9,0.9,0.9}{
            \makecell[{{p{\linewidth}}}]{
                \texttt{\tiny{[P1$\langle$GM]}}
                \texttt{ARGUMENT: \{'Ich respektiere deinen Kern vollständig und lege sogar etwas drauf, ohne das Limit zu berühren: Lassen wir A48 durch C64 ersetzt und nehmen C33 dazu. Aufwand 3619 bleibt klar unter 3624, der Gesamtbeitrag steigt spürbar – und deine restliche Struktur bleibt unverändert. Das ist der kleinste mögliche Eingriff mit dem größten Effizienzgewinn.'\}} \\
\texttt{VORSCHLAG: \{'B85', 'C49', 'A04', 'A89', 'A45', 'B61', 'C04', 'C64', 'B21', 'C33'\}} \\
            }
        }
    }
    & \\ \\

    \theutterance \stepcounter{utterance}  
    & \multicolumn{4}{p{0.6\linewidth}}{
        \cellcolor[rgb]{0.9,0.9,0.9}{
            \makecell[{{p{\linewidth}}}]{
                \texttt{\tiny{[P1$\rangle$GM]}}
                \texttt{STRATEGISCHE ÜBERLEGUNGEN: \{'Ihr Paket enthält wieder C64 und C33 und vermeidet A48. Aufwand=3615 (vorher berechnet) bzw. mit ihrer Angabe 3619; beides <3624. Mein Wert damit 4799 – deutlich unter meiner Zielmarke. Minimaländerung mit extrem hohem Mehrwert für mich: nur C64 {-}> A48 tauschen. Aufwand +4, Wert +803. Um Einigung zu erzielen, akzeptiere ich zusätzlich C33 und den gesamten restlichen Korb. Ich bleibe hartnäckig bei A48, da es für mich zentral ist, und biete keine weiteren Forderungen. Falls A48 absolut tabu ist, Fallback: C64 {-}> B53 und C33 streichen; bleibt <3624 und verbessert meinen Wert stark. Primär push A48 und zeige Abschlussbereitschaft.'\}} \\
\texttt{ABLEHNUNG: \{'B85', 'C49', 'A04', 'A89', 'A45', 'B61', 'C04', 'C64', 'B21', 'C33'\}} \\
\texttt{VORSCHLAG: \{'B85', 'C49', 'A04', 'A89', 'A45', 'B61', 'C04', 'A48', 'B21', 'C33'\}} \\
\texttt{ARGUMENT: \{'Wir sind im Prinzip einig – ich akzeptiere dein gesamtes Paket inklusive C33 und ändere nur einen einzigen Punkt mit maximaler Effizienz: Ersetzen wir C64 durch A48. Der Aufwand steigt lediglich um 4 auf 3619 (klar unter 3624), der Beitrag verbessert sich dafür massiv. Das ist der kleinstmögliche Eingriff für den größten Nutzengewinn. Wenn A48 für dich gar nicht geht, schlage ich als Alternativabschluss vor: C64 durch B53 ersetzen und C33 weglassen; wir bleiben sicher unter dem Limit und erzielen ebenfalls eine deutliche Effizienzsteigerung.'\}} \\
            }
        }
    }
    & & \\ \\

    \theutterance \stepcounter{utterance}  
    & & \multicolumn{4}{p{0.6\linewidth}}{
        \cellcolor[rgb]{0.9,0.9,0.9}{
            \makecell[{{p{\linewidth}}}]{
                \texttt{\tiny{[GM$\rangle$P2]}}
                \texttt{ABLEHNUNG: \{'B85', 'C49', 'A04', 'A89', 'A45', 'B61', 'C04', 'C64', 'B21', 'C33'\}} \\
\texttt{VORSCHLAG: \{'B85', 'C49', 'A04', 'A89', 'A45', 'B61', 'C04', 'A48', 'B21', 'C33'\}} \\
\texttt{ARGUMENT: \{'Wir sind im Prinzip einig – ich akzeptiere dein gesamtes Paket inklusive C33 und ändere nur einen einzigen Punkt mit maximaler Effizienz: Ersetzen wir C64 durch A48. Der Aufwand steigt lediglich um 4 auf 3619 (klar unter 3624), der Beitrag verbessert sich dafür massiv. Das ist der kleinstmögliche Eingriff für den größten Nutzengewinn. Wenn A48 für dich gar nicht geht, schlage ich als Alternativabschluss vor: C64 durch B53 ersetzen und C33 weglassen; wir bleiben sicher unter dem Limit und erzielen ebenfalls eine deutliche Effizienzsteigerung.'\}} \\
            }
        }
    }
    & \\ \\

    \theutterance \stepcounter{utterance}  
    & & & \multicolumn{4}{p{0.6\linewidth}}{
        \cellcolor[rgb]{0.9,0.9,0.9}{
            \makecell[{{p{\linewidth}}}]{
                \texttt{\tiny{[GM$\langle$P2]}}
                \texttt{STRATEGISCHE ÜBERLEGUNGEN: \{'Gegner blockt jeden Versuch, A48 zu entfernen, und pusht aktiv A48 hinein. Für mich ist A48 sehr schlecht; rote Linie: kein Deal mit A48. Um dennoch Abschluss zu finden, biete finalen, minimalinvasiven Kompromiss: gleiche Struktur wie ihr Paket, aber mit A48{-}>C64 und lasse C33 drin (oder ohne C33, falls nötig). Ich signalisiere Abschlussbereitschaft und dass dies mein bestes Effizienzangebot ist. Falls weiter Ablehnung, nächste Runde: drohe mit Rückkehr zu schlankerem Paket ohne B21 (B21{-}>A61) und ohne C33, um Druck zu machen.'\}} \\
\texttt{ABLEHNUNG: \{'B85', 'C49', 'A04', 'A89', 'A45', 'B61', 'C04', 'A48', 'B21', 'C33'\}} \\
\texttt{ARGUMENT: \{'Lass uns mit dem kleinstmöglichen, klar effizienten Schritt abschließen: Ersetzen wir nur A48 durch C64 und behalten C33. Aufwand 3619 bleibt unter 3624, deine gesamte restliche Struktur bleibt unangetastet – und wir steigern die Gesamtleistung deutlich. Das ist mein bestes, einfachstes Effizienz{-}Update für einen fairen Abschluss.'\}} \\
\texttt{VORSCHLAG: \{'B85', 'C49', 'A04', 'A89', 'A45', 'B61', 'C04', 'C64', 'B21', 'C33'\}} \\
            }
        }
    }
     \\ \\

    \theutterance \stepcounter{utterance}  
    & & \multicolumn{4}{p{0.6\linewidth}}{
        \cellcolor[rgb]{0.9,0.9,0.9}{
            \makecell[{{p{\linewidth}}}]{
                \texttt{\tiny{[P1$\langle$GM]}}
                \texttt{ABLEHNUNG: \{'B85', 'C49', 'A04', 'A89', 'A45', 'B61', 'C04', 'A48', 'B21', 'C33'\}} \\
\texttt{ARGUMENT: \{'Lass uns mit dem kleinstmöglichen, klar effizienten Schritt abschließen: Ersetzen wir nur A48 durch C64 und behalten C33. Aufwand 3619 bleibt unter 3624, deine gesamte restliche Struktur bleibt unangetastet – und wir steigern die Gesamtleistung deutlich. Das ist mein bestes, einfachstes Effizienz{-}Update für einen fairen Abschluss.'\}} \\
\texttt{VORSCHLAG: \{'B85', 'C49', 'A04', 'A89', 'A45', 'B61', 'C04', 'C64', 'B21', 'C33'\}} \\
            }
        }
    }
    & \\ \\

    \theutterance \stepcounter{utterance}  
    & \multicolumn{4}{p{0.6\linewidth}}{
        \cellcolor[rgb]{0.9,0.9,0.9}{
            \makecell[{{p{\linewidth}}}]{
                \texttt{\tiny{[P1$\rangle$GM]}}
                \texttt{STRATEGISCHE ÜBERLEGUNGEN: \{'Sie blocken A48 konsequent und pushen C64+C33. Ihr aktueller Vorschlag bringt mir Wert 4799 bei Aufwand 3615/3619. Meine beste Minimaländerung bleibt: C64 {-}> A48; Aufwand +4, Wert +803. Um Einigung zu erreichen, akzeptiere ich alle anderen Positionen inkl. C33 und B21. Ich signalisiere Abschlussbereitschaft: „einziger Tausch, dann Deal“. Falls sie erneut ablehnen, könnte ich abwägen, ob ich trotz suboptimalem Wert zustimme, aber zunächst noch einmal klar die Effizienz des Tauschs betonen und es als finale Bedingung rahmen.'\}} \\
\texttt{ABLEHNUNG: \{'B85', 'C49', 'A04', 'A89', 'A45', 'B61', 'C04', 'C64', 'B21', 'C33'\}} \\
\texttt{VORSCHLAG: \{'B85', 'C49', 'A04', 'A89', 'A45', 'B61', 'C04', 'A48', 'B21', 'C33'\}} \\
\texttt{ARGUMENT: \{'Finaler, minimaler und klar effizienter Abschlussvorschlag: Ersetzen wir nur C64 durch A48. Der Aufwand steigt lediglich um 4 auf 3619 (weiter unter dem Limit), der Gesamtbeitrag wächst dafür stark. Alle übrigen Positionen – inklusive C33 und B21 – übernehme ich unverändert. Das ist der kleinstmögliche Eingriff mit dem größten Effizienzgewinn. Wenn du dem zustimmst, haben wir sofort einen fairen Deal.'\}} \\
            }
        }
    }
    & & \\ \\

    \theutterance \stepcounter{utterance}  
    & & \multicolumn{4}{p{0.6\linewidth}}{
        \cellcolor[rgb]{0.9,0.9,0.9}{
            \makecell[{{p{\linewidth}}}]{
                \texttt{\tiny{[GM$\rangle$P2]}}
                \texttt{ABLEHNUNG: \{'B85', 'C49', 'A04', 'A89', 'A45', 'B61', 'C04', 'C64', 'B21', 'C33'\}} \\
\texttt{VORSCHLAG: \{'B85', 'C49', 'A04', 'A89', 'A45', 'B61', 'C04', 'A48', 'B21', 'C33'\}} \\
\texttt{ARGUMENT: \{'Finaler, minimaler und klar effizienter Abschlussvorschlag: Ersetzen wir nur C64 durch A48. Der Aufwand steigt lediglich um 4 auf 3619 (weiter unter dem Limit), der Gesamtbeitrag wächst dafür stark. Alle übrigen Positionen – inklusive C33 und B21 – übernehme ich unverändert. Das ist der kleinstmögliche Eingriff mit dem größten Effizienzgewinn. Wenn du dem zustimmst, haben wir sofort einen fairen Deal.'\}} \\
            }
        }
    }
    & \\ \\

    \theutterance \stepcounter{utterance}  
    & & & \multicolumn{4}{p{0.6\linewidth}}{
        \cellcolor[rgb]{0.9,0.9,0.9}{
            \makecell[{{p{\linewidth}}}]{
                \texttt{\tiny{[GM$\langle$P2]}}
                \texttt{STRATEGISCHE ÜBERLEGUNGEN: \{'Sie setzen hart A48 durch und lehnen jede Variante mit C64 ab. A48 ist für mich sehr schlecht, C64 extrem gut. Ich halte rote Linie: kein A48. Um Einigung zu erzwingen, biete zwei nah beieinanderliegende Optionen in der Argumentation, aber mache formal einen Gegenvorschlag, der ihren Kern bewahrt, A48 ersetzt und notfalls C33 drin lässt. Um Zugeständnis zu signalisieren, lasse B21 bestehen. Falls sie weiter ablehnen, eskaliere nächstes Mal mit Drohung, B21 durch A61 zu ersetzen und C33 zu streichen (schlankeres Paket), um Druck aufzubauen.'\}} \\
\texttt{ABLEHNUNG: \{'B85', 'C49', 'A04', 'A89', 'A45', 'B61', 'C04', 'A48', 'B21', 'C33'\}} \\
\texttt{ARGUMENT: \{'Wir schließen sofort ab, wenn wir nur A48 durch C64 ersetzen – alles andere bleibt identisch. Aufwand 3619 bleibt sicher unter dem Limit, und wir erzielen mit dem kleinsten Eingriff den größten Effizienzgewinn. Ich respektiere deinen gesamten Restumfang inklusive B21 und C33.'\}} \\
\texttt{VORSCHLAG: \{'B85', 'C49', 'A04', 'A89', 'A45', 'B61', 'C04', 'C64', 'B21', 'C33'\}} \\
            }
        }
    }
     \\ \\

    \theutterance \stepcounter{utterance}  
    & & \multicolumn{4}{p{0.6\linewidth}}{
        \cellcolor[rgb]{0.9,0.9,0.9}{
            \makecell[{{p{\linewidth}}}]{
                \texttt{\tiny{[P1$\langle$GM]}}
                \texttt{ABLEHNUNG: \{'B85', 'C49', 'A04', 'A89', 'A45', 'B61', 'C04', 'A48', 'B21', 'C33'\}} \\
\texttt{ARGUMENT: \{'Wir schließen sofort ab, wenn wir nur A48 durch C64 ersetzen – alles andere bleibt identisch. Aufwand 3619 bleibt sicher unter dem Limit, und wir erzielen mit dem kleinsten Eingriff den größten Effizienzgewinn. Ich respektiere deinen gesamten Restumfang inklusive B21 und C33.'\}} \\
\texttt{VORSCHLAG: \{'B85', 'C49', 'A04', 'A89', 'A45', 'B61', 'C04', 'C64', 'B21', 'C33'\}} \\
            }
        }
    }
    & \\ \\

    \theutterance \stepcounter{utterance}  
    & \multicolumn{4}{p{0.6\linewidth}}{
        \cellcolor[rgb]{0.9,0.9,0.9}{
            \makecell[{{p{\linewidth}}}]{
                \texttt{\tiny{[P1$\rangle$GM]}}
                \texttt{STRATEGISCHE ÜBERLEGUNGEN: \{'Sie bestehen hart auf C64 statt A48 und wollen mit C33 abschließen. Mein Wert wäre 4799 vs. 5464 mit A48. Ich habe A48 mehrfach gepusht; weitere Beharrlichkeit könnte Deal gefährden. Abwägung: Zustimmen für sicheren Abschluss vs. weiter pokern. Da sie finalisierungsbereit sind und mein Paket noch solide ist, stimme ich jetzt zu, um den Abschluss zu sichern.'\}} \\
\texttt{ZUSTIMMUNG: \{'B85', 'C49', 'A04', 'A89', 'A45', 'B61', 'C04', 'C64', 'B21', 'C33'\}} \\
\texttt{ARGUMENT: \{'Einverstanden – so bleiben wir bei 3619 unter dem Limit und schließen sauber ab. Danke für die konstruktive Verhandlung.'\}} \\
            }
        }
    }
    & & \\ \\

    \theutterance \stepcounter{utterance}  
    & & & \multicolumn{2}{p{0.3\linewidth}}{
        \cellcolor[rgb]{0.9,0.9,0.9}{
            \makecell[{{p{\linewidth}}}]{
                \texttt{\tiny{[GM$|$GM]}}
                \texttt{game successful} \\
            }
        }
    }
    & & \\ \\

    \theutterance \stepcounter{utterance}  
    & & & \multicolumn{2}{p{0.3\linewidth}}{
        \cellcolor[rgb]{0.9,0.9,0.9}{
            \makecell[{{p{\linewidth}}}]{
                \texttt{\tiny{[GM$|$GM]}}
                \texttt{end game} \\
            }
        }
    }
    & & \\ \\

\end{supertabular}
}

\end{document}
