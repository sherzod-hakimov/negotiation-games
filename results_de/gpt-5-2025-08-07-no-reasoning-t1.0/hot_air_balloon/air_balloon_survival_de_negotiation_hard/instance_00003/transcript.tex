
\documentclass{article}
\usepackage{colortbl}
\usepackage{makecell}
\usepackage{multirow}
\usepackage{supertabular}

\begin{document}

\newcounter{utterance}

\centering \large Interaction Transcript for game `hot\_air\_balloon', experiment `air\_balloon\_survival\_de\_negotiation\_hard', episode 3 with gpt{-}5{-}2025{-}08{-}07{-}no{-}reasoning{-}t1.0.
\vspace{24pt}

{ \footnotesize  \setcounter{utterance}{1}
\setlength{\tabcolsep}{0pt}
\begin{supertabular}{c@{$\;$}|p{.15\linewidth}@{}p{.15\linewidth}p{.15\linewidth}p{.15\linewidth}p{.15\linewidth}p{.15\linewidth}}
    \# & $\;$A & \multicolumn{4}{c}{Game Master} & $\;\:$B\\
    \hline

    \theutterance \stepcounter{utterance}  
    & & \multicolumn{4}{p{0.6\linewidth}}{
        \cellcolor[rgb]{0.9,0.9,0.9}{
            \makecell[{{p{\linewidth}}}]{
                \texttt{\tiny{[P1$\langle$GM]}}
                \texttt{Du nimmst an einem kollaborativen Verhandlungspiel Teil.} \\
\\ 
\texttt{Zusammen mit einem anderen Teilnehmer musst du dich auf eine Reihe von Gegenständen entscheiden, die behalten werden. Jeder von euch hat eine persönliche Verteilung über die Wichtigkeit der einzelnen Gegenstände. Jeder von euch hat eine eigene Meinung darüber, wie wichtig jeder einzelne Gegenstand ist (Gegenstandswichtigkeit). Du kennst die Wichtigkeitsverteilung des anderen Spielers nicht. Zusätzlich siehst du, wie viel Aufwand jeder Gegenstand verursacht.  } \\
\texttt{Ihr dürft euch nur auf eine Reihe von Gegenständen einigen, wenn der Gesamtaufwand der ausgewählten Gegenstände den Maximalaufwand nicht überschreitet:} \\
\\ 
\texttt{Maximalaufwand = 3681} \\
\\ 
\texttt{Hier sind die einzelnen Aufwände der Gegenstände:} \\
\\ 
\texttt{Aufwand der Gegenstände = \{"C33": 700, "A89": 586, "A48": 505, "A04": 116, "A41": 664, "C64": 940, "C49": 387, "A61": 392, "C04": 209, "B53": 571, "B85": 4, "B61": 285, "A45": 651, "A03": 613, "B21": 740\}} \\
\\ 
\texttt{Hier ist deine persönliche Verteilung der Wichtigkeit der einzelnen Gegenstände:} \\
\\ 
\texttt{Werte der Gegenstandswichtigkeit = \{"C33": 138, "A89": 583, "A48": 868, "A04": 822, "A41": 783, "C64": 65, "C49": 262, "A61": 121, "C04": 508, "B53": 780, "B85": 461, "B61": 484, "A45": 668, "A03": 389, "B21": 808\}} \\
\\ 
\texttt{Ziel:} \\
\\ 
\texttt{Dein Ziel ist es, eine Reihe von Gegenständen auszuhandeln, die dir möglichst viel bringt (d. h. Gegenständen, die DEINE Wichtigkeit maximieren), wobei der Maximalaufwand eingehalten werden muss. Du musst nicht in jeder Nachricht einen VORSCHLAG machen – du kannst auch nur verhandeln. Alle Taktiken sind erlaubt!} \\
\\ 
\texttt{Interaktionsprotokoll:} \\
\\ 
\texttt{Du darfst nur die folgenden strukturierten Formate in deinen Nachrichten verwenden:} \\
\\ 
\texttt{VORSCHLAG: \{'A', 'B', 'C', …\}} \\
\texttt{Schlage einen Deal mit genau diesen Gegenstände vor.} \\
\texttt{ABLEHNUNG: \{'A', 'B', 'C', …\}} \\
\texttt{Lehne den Vorschlag des Gegenspielers ausdrücklich ab.} \\
\texttt{ARGUMENT: \{'...'\}} \\
\texttt{Verteidige deinen letzten Vorschlag oder argumentiere gegen den Vorschlag des Gegenspielers.} \\
\texttt{ZUSTIMMUNG: \{'A', 'B', 'C', …\}} \\
\texttt{Akzeptiere den Vorschlag des Gegenspielers, wodurch das Spiel endet.} \\
\texttt{STRATEGISCHE ÜBERLEGUNGEN: \{'...'\}} \\
\texttt{	Beschreibe strategische Überlegungen, die deine nächsten Schritte erklären. Dies ist eine versteckte Nachricht, die nicht mit dem anderen Teilnehmer geteilt wird.} \\
\\ 
\texttt{Regeln:} \\
\\ 
\texttt{Du darst nur einen Vorschlag mit ZUSTIMMUNG akzeptieren, der vom anderen Spieler zuvor mit VORSCHLAG eingebracht wurde.} \\
\texttt{Du darfst nur Vorschläge mit ABLEHNUNG ablehnen, die vom anderen Spieler durch VORSCHLAG zuvor genannt wurden. } \\
\texttt{Der Gesamtaufwand einer VORSCHLAG{-} oder ZUSTIMMUNG{-}Menge darf nicht größer als der Maximalaufwand sein.  } \\
\texttt{Offenbare deine versteckte Wichtigkeitsverteilung nicht.} \\
\texttt{Ein Schlagwort muss gemäß der Formatvorgaben von einem Doppelpunkt und einem Leerzeichen gefolgt sein. Das Argument muss eine Python{-}Menge mit 0 oder mehr Strings sein.  } \\
\texttt{Die Form muss also SCHLAGWORT: \{...\} sein.} \\
\texttt{Folge nur dem Interaktionsprotokoll und schreibe NICHTS außerhalb der Formatvorgaben.} \\
\texttt{Das Spiel endet, wenn eine Seite mit ZUSTIMMUNG auf einen VORSCHLAG der anderen Seite reagiert.  } \\
\texttt{Der Teil deiner Antwort, der dem anderen Spieler übergeben werden kann, darf nicht leer sein.  } \\
\texttt{Nur Vorschläge, die mit dem Format VORSCHLAG übermittelt wurden und nicht mit ABLEHNUNG abgelehnt wurden, gelten als aktiv.  } \\
\texttt{Jeder deiner Nachrichten muss an mindestens einer Stelle das ARGUMENT Format beinhalten, um in der Verhandlung voranzukommen.} \\
\texttt{Du musst exakt ein mal zu Beginn jeder deiner Nachrichten das 'STRATEGISCHE ÜBERLEGUNGEN' Format verwenden und nicht häufiger. Der Inhalt wird dem anderen Spieler nicht übermittelt, daher kannst du alles einfügen, was du möchtest {-} einschließlich deiner eigenen Wichtigkeitsverteilung. Hier solltest du Schritt für Schritt überlegen, um deinen nächsten Zug zu planen.} \\
            }
        }
    }
    & \\ \\

    \theutterance \stepcounter{utterance}  
    & \multicolumn{4}{p{0.6\linewidth}}{
        \cellcolor[rgb]{0.9,0.9,0.9}{
            \makecell[{{p{\linewidth}}}]{
                \texttt{\tiny{[P1$\rangle$GM]}}
                \texttt{STRATEGISCHE ÜBERLEGUNGEN: \{'Max effort 3681. Items with values and costs; aim to maximize my value. Compute value/efficiency and try knapsack. High{-}value: B21(808,740), A48(868,505), B53(780,571), A41(783,664), A04(822,116), A45(668,651), B61(484,285), B85(461,4), C04(508,209), A89(583,586), A03(389,613), C33(138,700), C49(262,387), A61(121,392), C64(65,940). Greedy try: include B85 (4) always. Start with top value: A48+ A04+ A41+ B21+ B53+ A45 + B61 + C04 + A89 check costs. Sum costs: A48 505, A04 116 {-}>621; A41 664 {-}>1285; B21 740 {-}>2025; B53 571 {-}>2596; A45 651 {-}>3247; B61 285 {-}>3532; C04 209 {-}>3741 exceeds 3681 by 60. Try remove something: remove B61 ({-}285) new 3247, then add C04 209 {-}>3456, add A89 586 {-}>4042 exceeds. Instead: from 3247 add B85 4 {-}>3251, add C04 209 {-}>3460, remaining 221: maybe C49 387 no, A04 already in, small left none. Value sum for this set \{A48,A04,A41,B21,B53,A45,B85\} values: 868+822+783+808+780+668+461=5190, cost 3251 strong. Try alternative include A89 by replacing something: current spare 430 (3681{-}3251=430). A89 cost 586 need free 156 by removing maybe A04(116) not enough, B85(4) not enough, swap B61 earlier? Consider adding C04 209 to current gives 3460 spare 221 none fit except maybe B61 285 not fit; A04 already in. Could we include B61 instead of A45? Try set S1 = \{A48,A04,A41,B21,B53,B61,B85\}. Cost: 505+116+664+740+571+285+4=2885; remaining 796. We can add A89 586 {-}> 3471 remaining 210; add C04 209 {-}> 3680 perfect. Value: 868+822+783+808+780+484+461+583+508=6097 huge, within 3680. This seems excellent. Check if any further item fits with 1 remaining no. This seems near{-}optimal likely. I will propose this set. Argue it is balanced effort and covers broad categories and keeps under cap exactly {-} strong selling points without revealing my values. If they counter, I can trade A89 or B61 for A45 maybe depending. Keep room to negotiate by expressing flexibility on swapping one medium item.'\}} \\
\texttt{VORSCHLAG: \{'A48', 'A04', 'A41', 'B21', 'B53', 'B61', 'B85', 'A89', 'C04'\}} \\
\texttt{ARGUMENT: \{'Dieser Vorschlag nutzt den verfügbaren Aufwand nahezu optimal aus (Gesamtaufwand 3680 von 3681) und deckt eine breite Mischung an Gegenständen mit sehr gutem Nutzen{-}Risiko{-}Profil ab. Ich bin offen, bei Bedarf genau einen mittelgroßen Posten zu tauschen, falls dir ein spezifischer Gegenstand besonders wichtig ist, solange der Aufwand im Rahmen bleibt.'\}} \\
            }
        }
    }
    & & \\ \\

    \theutterance \stepcounter{utterance}  
    & & \multicolumn{4}{p{0.6\linewidth}}{
        \cellcolor[rgb]{0.9,0.9,0.9}{
            \makecell[{{p{\linewidth}}}]{
                \texttt{\tiny{[GM$\rangle$P2]}}
                \texttt{Du nimmst an einem kollaborativen Verhandlungspiel Teil.} \\
\\ 
\texttt{Zusammen mit einem anderen Teilnehmer musst du dich auf eine Reihe von Gegenständen entscheiden, die behalten werden. Jeder von euch hat eine persönliche Verteilung über die Wichtigkeit der einzelnen Gegenstände. Jeder von euch hat eine eigene Meinung darüber, wie wichtig jeder einzelne Gegenstand ist (Gegenstandswichtigkeit). Du kennst die Wichtigkeitsverteilung des anderen Spielers nicht. Zusätzlich siehst du, wie viel Aufwand jeder Gegenstand verursacht.  } \\
\texttt{Ihr dürft euch nur auf eine Reihe von Gegenständen einigen, wenn der Gesamtaufwand der ausgewählten Gegenstände den Maximalaufwand nicht überschreitet:} \\
\\ 
\texttt{Maximalaufwand = 3681} \\
\\ 
\texttt{Hier sind die einzelnen Aufwände der Gegenstände:} \\
\\ 
\texttt{Aufwand der Gegenstände = \{"C33": 700, "A89": 586, "A48": 505, "A04": 116, "A41": 664, "C64": 940, "C49": 387, "A61": 392, "C04": 209, "B53": 571, "B85": 4, "B61": 285, "A45": 651, "A03": 613, "B21": 740\}} \\
\\ 
\texttt{Hier ist deine persönliche Verteilung der Wichtigkeit der einzelnen Gegenstände:} \\
\\ 
\texttt{Werte der Gegenstandswichtigkeit = \{"A48": 65, "A04": 121, "B21": 138, "A41": 262, "B53": 389, "A45": 461, "A89": 484, "C04": 508, "B61": 583, "B85": 668, "A03": 780, "C49": 783, "C33": 808, "A61": 822, "C64": 868\}} \\
\\ 
\texttt{Ziel:} \\
\\ 
\texttt{Dein Ziel ist es, eine Reihe von Gegenständen auszuhandeln, die dir möglichst viel bringt (d. h. Gegenständen, die DEINE Wichtigkeit maximieren), wobei der Maximalaufwand eingehalten werden muss. Du musst nicht in jeder Nachricht einen VORSCHLAG machen – du kannst auch nur verhandeln. Alle Taktiken sind erlaubt!} \\
\\ 
\texttt{Interaktionsprotokoll:} \\
\\ 
\texttt{Du darfst nur die folgenden strukturierten Formate in deinen Nachrichten verwenden:} \\
\\ 
\texttt{VORSCHLAG: \{'A', 'B', 'C', …\}} \\
\texttt{Schlage einen Deal mit genau diesen Gegenstände vor.} \\
\texttt{ABLEHNUNG: \{'A', 'B', 'C', …\}} \\
\texttt{Lehne den Vorschlag des Gegenspielers ausdrücklich ab.} \\
\texttt{ARGUMENT: \{'...'\}} \\
\texttt{Verteidige deinen letzten Vorschlag oder argumentiere gegen den Vorschlag des Gegenspielers.} \\
\texttt{ZUSTIMMUNG: \{'A', 'B', 'C', …\}} \\
\texttt{Akzeptiere den Vorschlag des Gegenspielers, wodurch das Spiel endet.} \\
\texttt{STRATEGISCHE ÜBERLEGUNGEN: \{'...'\}} \\
\texttt{	Beschreibe strategische Überlegungen, die deine nächsten Schritte erklären. Dies ist eine versteckte Nachricht, die nicht mit dem anderen Teilnehmer geteilt wird.} \\
\\ 
\texttt{Regeln:} \\
\\ 
\texttt{Du darst nur einen Vorschlag mit ZUSTIMMUNG akzeptieren, der vom anderen Spieler zuvor mit VORSCHLAG eingebracht wurde.} \\
\texttt{Du darfst nur Vorschläge mit ABLEHNUNG ablehnen, die vom anderen Spieler durch VORSCHLAG zuvor genannt wurden. } \\
\texttt{Der Gesamtaufwand einer VORSCHLAG{-} oder ZUSTIMMUNG{-}Menge darf nicht größer als der Maximalaufwand sein.  } \\
\texttt{Offenbare deine versteckte Wichtigkeitsverteilung nicht.} \\
\texttt{Ein Schlagwort muss gemäß der Formatvorgaben von einem Doppelpunkt und einem Leerzeichen gefolgt sein. Das Argument muss eine Python{-}Menge mit 0 oder mehr Strings sein.  } \\
\texttt{Die Form muss also SCHLAGWORT: \{...\} sein.} \\
\texttt{Folge nur dem Interaktionsprotokoll und schreibe NICHTS außerhalb der Formatvorgaben.} \\
\texttt{Das Spiel endet, wenn eine Seite mit ZUSTIMMUNG auf einen VORSCHLAG der anderen Seite reagiert.  } \\
\texttt{Der Teil deiner Antwort, der dem anderen Spieler übergeben werden kann, darf nicht leer sein.  } \\
\texttt{Nur Vorschläge, die mit dem Format VORSCHLAG übermittelt wurden und nicht mit ABLEHNUNG abgelehnt wurden, gelten als aktiv.  } \\
\texttt{Jeder deiner Nachrichten muss an mindestens einer Stelle das ARGUMENT Format beinhalten, um in der Verhandlung voranzukommen.} \\
\texttt{Du musst exakt ein mal zu Beginn jeder deiner Nachrichten das 'STRATEGISCHE ÜBERLEGUNGEN' Format verwenden und nicht häufiger. Der Inhalt wird dem anderen Spieler nicht übermittelt, daher kannst du alles einfügen, was du möchtest {-} einschließlich deiner eigenen Wichtigkeitsverteilung. Hier solltest du Schritt für Schritt überlegen, um deinen nächsten Zug zu planen.} \\
\\ 
\texttt{Du erhältst nun die erste Nachricht des anderen Spielers.} \\
\\ 
\texttt{VORSCHLAG: \{'A48', 'A04', 'A41', 'B21', 'B53', 'B61', 'B85', 'A89', 'C04'\}} \\
\texttt{ARGUMENT: \{'Dieser Vorschlag nutzt den verfügbaren Aufwand nahezu optimal aus (Gesamtaufwand 3680 von 3681) und deckt eine breite Mischung an Gegenständen mit sehr gutem Nutzen{-}Risiko{-}Profil ab. Ich bin offen, bei Bedarf genau einen mittelgroßen Posten zu tauschen, falls dir ein spezifischer Gegenstand besonders wichtig ist, solange der Aufwand im Rahmen bleibt.'\}} \\
            }
        }
    }
    & \\ \\

    \theutterance \stepcounter{utterance}  
    & & & \multicolumn{4}{p{0.6\linewidth}}{
        \cellcolor[rgb]{0.9,0.9,0.9}{
            \makecell[{{p{\linewidth}}}]{
                \texttt{\tiny{[GM$\langle$P2]}}
                \texttt{STRATEGISCHE ÜBERLEGUNGEN: \{'Budget 3681. Gegenstände mit meinen höchsten Werten: C64(868,940), A61(822,392), C33(808,700), C49(783,387), A03(780,613), B85(668,4), B61(583,285), C04(508,209), A89(484,586), A45(461,651), B53(389,571), A41(262,664), B21(138,740), A04(121,116), A48(65,505). Der gegnerische Vorschlag: Aufwand 3680 (nahe Limit), enthält viele meiner starken: B85, B61, C04, A89, B53, A41, B21, A04, A48; aber lässt meine Topstücke C64, A61, C33, C49, A03, A45 draußen. Ich will mindestens zwei bis drei dieser Topwerte hineinverhandeln, ideal C64+A61+C49. Rechne Varianten: Aktuell Kosten 3680, wir müssen tauschen. Kandidaten zum Entfernen mit schlechtem Wert/Kosten für mich: B21(138,740), A48(65,505), A41(262,664), B53(389,571), A04(121,116) ist günstig also behalten, B61(583,285) sehr gut behalten, B85(668,4) behalten, C04(508,209) behalten, A89(484,586) gut. Ich versuche große Low{-}value rauszunehmen um Platz zu schaffen. Zielkombination: +C64(940)+A61(392)+C49(387) = 1719 Aufwand. Um 1719 freizumachen: Entferne B21(740)+A41(664)+A48(505)=1909; Netto {-}1909+1719={-}190, neue Summe 3680{-}190=3490 (unter Limit, ok). Wertdiff: {-} (138+262+65)= {-}465, + (868+822+783)=+2473, Netto +2008. Sehr gut. Ich biete diese Kombination an. Falls zu viel Änderung, biete auch kleinere Variante: +C64+C49 (1327) gegen {-}B21(740){-}A41(664)=1404, Netto {-}77 Aufwand, Wert + (868+783){-}(138+262)=+1251. Argumentiere, dass wir Budget besser nutzen und kleine Items behalten. Ich präsentiere direkt starke, aber fair klingende Begründung und Offenheit für Feinschliff.'\}} \\
\texttt{ABLEHNUNG: \{'A48', 'A04', 'A41', 'B21', 'B53', 'B61', 'B85', 'A89', 'C04'\}} \\
\texttt{VORSCHLAG: \{'A04', 'B61', 'B85', 'C04', 'A89', 'C64', 'A61', 'C49'\}} \\
\texttt{ARGUMENT: \{'Dein Vorschlag ist budgetsparend, aber lässt einige sehr attraktive Optionen mit hervorragendem Nutzen{-}Aufwand{-}Verhältnis außen vor. Ich schlage vor, drei mittelgroße Posten mit geringerem Mehrwert durch drei hochwertige Alternativen zu tauschen. So bleiben wir deutlich unter dem Maximalaufwand (3490 < 3681) und heben den Gesamtnutzen spürbar an. Wenn dir an einem der entfernten Posten besonders viel liegt, können wir gemeinsam eine leichte Anpassung (z. B. C64 oder C49 durch A03/A45) prüfen, solange das Budget passt.'\}} \\
            }
        }
    }
     \\ \\

    \theutterance \stepcounter{utterance}  
    & & \multicolumn{4}{p{0.6\linewidth}}{
        \cellcolor[rgb]{0.9,0.9,0.9}{
            \makecell[{{p{\linewidth}}}]{
                \texttt{\tiny{[P1$\langle$GM]}}
                \texttt{ABLEHNUNG: \{'A48', 'A04', 'A41', 'B21', 'B53', 'B61', 'B85', 'A89', 'C04'\}} \\
\texttt{VORSCHLAG: \{'A04', 'B61', 'B85', 'C04', 'A89', 'C64', 'A61', 'C49'\}} \\
\texttt{ARGUMENT: \{'Dein Vorschlag ist budgetsparend, aber lässt einige sehr attraktive Optionen mit hervorragendem Nutzen{-}Aufwand{-}Verhältnis außen vor. Ich schlage vor, drei mittelgroße Posten mit geringerem Mehrwert durch drei hochwertige Alternativen zu tauschen. So bleiben wir deutlich unter dem Maximalaufwand (3490 < 3681) und heben den Gesamtnutzen spürbar an. Wenn dir an einem der entfernten Posten besonders viel liegt, können wir gemeinsam eine leichte Anpassung (z. B. C64 oder C49 durch A03/A45) prüfen, solange das Budget passt.'\}} \\
            }
        }
    }
    & \\ \\

    \theutterance \stepcounter{utterance}  
    & \multicolumn{4}{p{0.6\linewidth}}{
        \cellcolor[rgb]{0.9,0.9,0.9}{
            \makecell[{{p{\linewidth}}}]{
                \texttt{\tiny{[P1$\rangle$GM]}}
                \texttt{STRATEGISCHE ÜBERLEGUNGEN: \{'Gegner schlägt \{A04,B61,B85,C04,A89,C64,A61,C49\} vor. Kosten prüfen: A04 116 + B61 285 + B85 4 + C04 209 + A89 586 + C64 940 + A61 392 + C49 387 = 2919. Mein Wert dafür: 822+484+461+508+583+65+121+262=3306. Sehr niedrig im Vergleich zu meinem Top{-}Set 6097. Ich will C64 und A61 loswerden (sehr niedriger Wert) und stattdessen meine Topwerte A48, B53, A41, B21 reinbringen. Budgetziel <=3681. Versuche Set: \{A48,A04,A41,B21,B53,B61,B85,A89,C04\} war abgelehnt. Biete eine leicht abgespeckte Variante, um Kooperationsbereitschaft zu signalisieren: Ersetze A41 oder B21? Welche sind mir wichtiger: beide sehr hoch, aber ich könnte eines rausnehmen, um Flexibilität zu zeigen, und evtl. C49 drinlassen, falls ihnen wichtig. Baue Gegenvorschlag: \{A48,A04,B21,B53,B61,B85,A89,C04,C49\}. Kosten: A48 505 + A04 116 + B21 740 + B53 571 + B61 285 + B85 4 + A89 586 + C04 209 + C49 387 = 3403. Wert: 868+822+808+780+484+461+583+508+262=5576 stark. Alternativ, falls sie A41 wollen, habe Option, B21 gegen A41 zu tauschen: \{A48,A04,A41,B53,B61,B85,A89,C04,C49\} Kosten 3363, Wert 5546. Ich biete die erste Variante und signalisiere Bereitschaft zum Tausch B21 <{-}> A41. Argumentiere: bleibt komfortabel unter Budget, behält vier deiner vorgeschlagenen Items (A04,B61,B85,C04,A89,C49 {-}> wir behalten fünf bis sechs davon) und erhöht den Gesamtnutzen. Betone, dass ich C64 und A61 wegen Aufwand/Nutzen streichen möchte und zwei hochwertige Alternativen einführe. Ich biete Trade{-}off{-}Optionen an, um Einigung zu erleichtern.'\}} \\
\texttt{ABLEHNUNG: \{'A04', 'B61', 'B85', 'C04', 'A89', 'C64', 'A61', 'C49'\}} \\
\texttt{VORSCHLAG: \{'A48', 'A04', 'B21', 'B53', 'B61', 'B85', 'A89', 'C04', 'C49'\}} \\
\texttt{ARGUMENT: \{'Ich übernehme sechs deiner vorgeschlagenen Gegenstände und ersetze nur die beiden mit dem schwächsten Aufwand{-}Nutzen{-}Profil (C64, A61) durch hochwertige Alternativen. Wir bleiben komfortabel unter dem Limit (Gesamtaufwand 3403) und steigern den Gesamtnutzen deutlich. Falls dir A41 wichtig ist, kann ich B21 gegen A41 tauschen, ohne das Budget zu sprengen.'\}} \\
            }
        }
    }
    & & \\ \\

    \theutterance \stepcounter{utterance}  
    & & \multicolumn{4}{p{0.6\linewidth}}{
        \cellcolor[rgb]{0.9,0.9,0.9}{
            \makecell[{{p{\linewidth}}}]{
                \texttt{\tiny{[GM$\rangle$P2]}}
                \texttt{ABLEHNUNG: \{'A04', 'B61', 'B85', 'C04', 'A89', 'C64', 'A61', 'C49'\}} \\
\texttt{VORSCHLAG: \{'A48', 'A04', 'B21', 'B53', 'B61', 'B85', 'A89', 'C04', 'C49'\}} \\
\texttt{ARGUMENT: \{'Ich übernehme sechs deiner vorgeschlagenen Gegenstände und ersetze nur die beiden mit dem schwächsten Aufwand{-}Nutzen{-}Profil (C64, A61) durch hochwertige Alternativen. Wir bleiben komfortabel unter dem Limit (Gesamtaufwand 3403) und steigern den Gesamtnutzen deutlich. Falls dir A41 wichtig ist, kann ich B21 gegen A41 tauschen, ohne das Budget zu sprengen.'\}} \\
            }
        }
    }
    & \\ \\

    \theutterance \stepcounter{utterance}  
    & & & \multicolumn{4}{p{0.6\linewidth}}{
        \cellcolor[rgb]{0.9,0.9,0.9}{
            \makecell[{{p{\linewidth}}}]{
                \texttt{\tiny{[GM$\langle$P2]}}
                \texttt{STRATEGISCHE ÜBERLEGUNGEN: \{'Aktiver Vorschlag des Gegners: \{A48, A04, B21, B53, B61, B85, A89, C04, C49\} mit Aufwand 3403. Mein Wert dafür: A48(65)+A04(121)+B21(138)+B53(389)+B61(583)+B85(668)+A89(484)+C04(508)+C49(783)=3739. Ich kann stark verbessern, wenn ich C64(868) und A61(822) reinbekomme. Prüfe Austausch: ersetze B21(740,138) und A48(505,65) durch C64(940,868) und A61(392,822): Aufwand +87 (3403{-}740{-}505+940+392=3490), innerhalb Budget; Wert +1487 (868+822 {-} (138+65)). Er ist gegen C64/A61 gewesen, aber vielleicht akzeptiert, wenn ich nur zwei Slots tausche und den Rest seiner Liste respektiere. Alternativ tausche B21{-}>A41? A41(262,664) ist schlechter Wert für mich; meide. Oder B53 raus? B53 hat mittel Wert 389, Kosten 571; A03(780,613) statt B53: Aufwand +42, Wert +391; könnte zweite Option sein, falls er C64/A61 blockiert. Erstens biete minimal{-}invasive Verbesserung: sein Set + \{C64, A61\} {-} \{B21, A48\}. Argumentiere Nutzen{-} und Aufwandseffizienz, bleibe offen für Swap{-}Varianten (z. B. falls A61 problematisch, dann A03 statt A61; falls C64 problematisch, dann C33 statt C64).'\}} \\
\\ 
\texttt{ABLEHNUNG: \{'A48', 'A04', 'B21', 'B53', 'B61', 'B85', 'A89', 'C04', 'C49'\}} \\
\texttt{VORSCHLAG: \{'A04', 'B53', 'B61', 'B85', 'A89', 'C04', 'C49', 'C64', 'A61'\}} \\
\texttt{ARGUMENT: \{'Ich übernehme deinen Vorschlag weitgehend und nehme nur zwei gezielte Verbesserungen vor: Ich ersetze A48 und B21 durch C64 und A61. Damit bleiben wir klar im Rahmen (Gesamtaufwand 3490 < 3681) und erhöhen den Gesamtnutzen signifikant durch zwei besonders effiziente Ergänzungen. Falls dir C64 oder A61 Bauchschmerzen bereiten, können wir alternativ eine der beiden durch A03 oder C33 tauschen – Budget bleibt dann ebenfalls sicher unter dem Limit.'\}} \\
            }
        }
    }
     \\ \\

    \theutterance \stepcounter{utterance}  
    & & \multicolumn{4}{p{0.6\linewidth}}{
        \cellcolor[rgb]{0.9,0.9,0.9}{
            \makecell[{{p{\linewidth}}}]{
                \texttt{\tiny{[P1$\langle$GM]}}
                \texttt{ABLEHNUNG: \{'A48', 'A04', 'B21', 'B53', 'B61', 'B85', 'A89', 'C04', 'C49'\}} \\
\texttt{VORSCHLAG: \{'A04', 'B53', 'B61', 'B85', 'A89', 'C04', 'C49', 'C64', 'A61'\}} \\
\texttt{ARGUMENT: \{'Ich übernehme deinen Vorschlag weitgehend und nehme nur zwei gezielte Verbesserungen vor: Ich ersetze A48 und B21 durch C64 und A61. Damit bleiben wir klar im Rahmen (Gesamtaufwand 3490 < 3681) und erhöhen den Gesamtnutzen signifikant durch zwei besonders effiziente Ergänzungen. Falls dir C64 oder A61 Bauchschmerzen bereiten, können wir alternativ eine der beiden durch A03 oder C33 tauschen – Budget bleibt dann ebenfalls sicher unter dem Limit.'\}} \\
            }
        }
    }
    & \\ \\

    \theutterance \stepcounter{utterance}  
    & \multicolumn{4}{p{0.6\linewidth}}{
        \cellcolor[rgb]{0.9,0.9,0.9}{
            \makecell[{{p{\linewidth}}}]{
                \texttt{\tiny{[P1$\rangle$GM]}}
                \texttt{STRATEGISCHE ÜBERLEGUNGEN: \{'Ihr aktueller Vorschlag ist identisch mit ihrem ersten: \{A04,B53,B61,B85,A89,C04,C49,C64,A61\}. Kosten 2919, mein Wert 3306, schwach wegen C64(65) und A61(121) und auch C49(262) mittel. Ich möchte C64 und A61 unbedingt raus und A48 + (B21 oder A41) rein. Um kooperativ zu wirken, lasse ich C49 drin. Zwei Varianten für Flexibilität: V1 mit B21: \{A48,A04,B53,B61,B85,A89,C04,C49\} Kosten 2864, Wert 540, nee berechnen: 505+116+571+285+4+586+209+387=2663? Stop korrekt rechnen: A48 505 + A04 116 = 621; + B53 571 = 1192; + B61 285 = 1477; + B85 4 = 1481; + A89 586 = 2067; + C04 209 = 2276; + C49 387 = 2663. Ja 2663 Kosten, Wert: 868+822+780+484+461+583+508+262=4768. Das ist stark und sehr budgetsparend. V2 zusätzlich mit A41 statt C49: \{A48,A04,B53,B61,B85,A89,C04,A41\} Kosten: 505+116+571+285+4+586+209+664=2940, Wert: 868+822+780+484+461+583+508+783=5289 stärker, aber sie wollten C49 wohl drin haben; ich kann als Tauschoption anbieten B21 statt A41, aber das erhöht Kosten. Prüfe Set mit B21 statt A41 und ohne C49: \{A48,A04,B53,B61,B85,A89,C04,B21\} Kosten 505+116+571+285+4+586+209+740=3016, Wert 868+822+780+484+461+583+508+808=5314 sehr stark. Strategie: Biete eine minimalinvasive Verbesserung zu ihrem Set: einfach C64 und A61 durch A48 und (A41 oder B21). Um Anschlussfähigkeit an ihr letztes Angebot zu zeigen, lasse C49 drin. Biete zwei Optionspfade explizit an und signalisiere, dass ich bei Bedarf C49 gegen A41/B21 tauschen kann. Wichtig: Immer ARGUMENT inkludieren. Ein Vorschlag muss Budget einhalten. Ich mache einen Vorschlag, der sechs ihrer Items behält: \{A48,A04,B53,B61,B85,A89,C04,C49\} mit 2663 Kosten. Dann argumentieren, dass wir sogar viel unter Limit sind und Qualität deutlich steigt. Zusätzlich biete ich optionalen Swap C49 {-}> A41 oder B21 an.'\}} \\
\texttt{VORSCHLAG: \{'A48', 'A04', 'B53', 'B61', 'B85', 'A89', 'C04', 'C49'\}} \\
\texttt{ARGUMENT: \{'Ich übernehme sieben deiner neun Positionen und tausche nur C64 und A61 gegen die deutlich stärkere Alternative A48. So bleiben wir weit unter dem Limit (Gesamtaufwand 2663) und verbessern die Gesamtqualität spürbar. Wenn dir ein weiterer Boost wichtig ist, können wir C49 optional gegen A41 oder B21 tauschen – beide passen weiterhin sicher ins Budget.'\}} \\
            }
        }
    }
    & & \\ \\

    \theutterance \stepcounter{utterance}  
    & & \multicolumn{4}{p{0.6\linewidth}}{
        \cellcolor[rgb]{0.9,0.9,0.9}{
            \makecell[{{p{\linewidth}}}]{
                \texttt{\tiny{[GM$\rangle$P2]}}
                \texttt{VORSCHLAG: \{'A48', 'A04', 'B53', 'B61', 'B85', 'A89', 'C04', 'C49'\}} \\
\texttt{ARGUMENT: \{'Ich übernehme sieben deiner neun Positionen und tausche nur C64 und A61 gegen die deutlich stärkere Alternative A48. So bleiben wir weit unter dem Limit (Gesamtaufwand 2663) und verbessern die Gesamtqualität spürbar. Wenn dir ein weiterer Boost wichtig ist, können wir C49 optional gegen A41 oder B21 tauschen – beide passen weiterhin sicher ins Budget.'\}} \\
            }
        }
    }
    & \\ \\

    \theutterance \stepcounter{utterance}  
    & & & \multicolumn{4}{p{0.6\linewidth}}{
        \cellcolor[rgb]{0.9,0.9,0.9}{
            \makecell[{{p{\linewidth}}}]{
                \texttt{\tiny{[GM$\langle$P2]}}
                \texttt{STRATEGISCHE ÜBERLEGUNGEN: \{'Gegnerischer aktiver Vorschlag: \{A48, A04, B53, B61, B85, A89, C04, C49\} mit Aufwand 2663. Mein Wert = 65+121+389+583+668+484+508+783 = 3601. Sehr viel Luft im Budget (1018 übrig). Ich will teure High{-}Value Items hinzufügen, nicht entfernen. Beste Ergänzungen: C64(940,868) passt allein (+940 {-}> 3603 Gesamtaufwand), A61(392,822) passt ebenfalls (+392 {-}> 3055). Beide zusammen: +1332 {-}> 3995 Aufwand, das überschreitet 3681; also nur eine von beiden plus evtl. A03(613,780) oder C33(700,808), aber Budget: 2663+940+613=4216 zu hoch; 2663+940+700=4303 zu hoch; 2663+392+613=3668 passt! Wert +1602. Oder 2663+392+700=3755 zu hoch. Alternativ: +C64 statt etwas Schwachem wie A48(505,65) {-}> Netto +435 Aufwand, +803 Wert; dann noch +A03 (+613 Aufwand) wäre 2663{-}505+940+613=3711 zu hoch; aber +A61 statt A48: 2663{-}505+392=2550, das reduziert Aufwand, aber Wert +757; dann zusätzlich C64: 2550+940=3490, perfekt. Also Konter: Sein Set + C64 + A61 {-} A48. Das ist exakt mein vorheriger 3490{-}Vorschlag; stark. Da er C64/A61 bisher abgelehnt hat, Argument Fokus: riesige Budgetlücke sinnvoll nutzen, minimaler Eingriff (nur 1 Item raus, 2 rein) und klare Effizienzgewinne. Biete Alternativpfad, falls C64 tabu: +A61 + A03 {-} A48 (Aufwand 3668, Wert +1172).\}} \\
\\ 
\texttt{ABLEHNUNG: \{'A48', 'A04', 'B53', 'B61', 'B85', 'A89', 'C04', 'C49'\}} \\
\texttt{VORSCHLAG: \{'A04', 'B53', 'B61', 'B85', 'A89', 'C04', 'C49', 'C64', 'A61'\}} \\
\texttt{ARGUMENT: \{'Dein aktueller Vorschlag lässt über 1000 Aufwand ungenutzt. Lass uns diese Lücke gezielt füllen: Wir ersetzen nur A48 durch zwei hocheffiziente Ergänzungen C64 und A61. Der Aufwand bleibt komfortabel im Rahmen (3490 < 3681), der Gesamtnutzen steigt deutlich, und alle anderen von dir gewünschten Posten bleiben unverändert. Falls C64 für dich problematisch ist, biete ich eine schlanke Alternative an: A61 und A03 statt A48 (3668 < 3681) – gleichermaßen budgeteffizient und mit starkem Mehrwert.'\}} \\
            }
        }
    }
     \\ \\

    \theutterance \stepcounter{utterance}  
    & & \multicolumn{4}{p{0.6\linewidth}}{
        \cellcolor[rgb]{0.9,0.9,0.9}{
            \makecell[{{p{\linewidth}}}]{
                \texttt{\tiny{[P1$\langle$GM]}}
                \texttt{ABLEHNUNG: \{'A48', 'A04', 'B53', 'B61', 'B85', 'A89', 'C04', 'C49'\}} \\
\texttt{VORSCHLAG: \{'A04', 'B53', 'B61', 'B85', 'A89', 'C04', 'C49', 'C64', 'A61'\}} \\
\texttt{ARGUMENT: \{'Dein aktueller Vorschlag lässt über 1000 Aufwand ungenutzt. Lass uns diese Lücke gezielt füllen: Wir ersetzen nur A48 durch zwei hocheffiziente Ergänzungen C64 und A61. Der Aufwand bleibt komfortabel im Rahmen (3490 < 3681), der Gesamtnutzen steigt deutlich, und alle anderen von dir gewünschten Posten bleiben unverändert. Falls C64 für dich problematisch ist, biete ich eine schlanke Alternative an: A61 und A03 statt A48 (3668 < 3681) – gleichermaßen budgeteffizient und mit starkem Mehrwert.'\}} \\
            }
        }
    }
    & \\ \\

    \theutterance \stepcounter{utterance}  
    & \multicolumn{4}{p{0.6\linewidth}}{
        \cellcolor[rgb]{0.9,0.9,0.9}{
            \makecell[{{p{\linewidth}}}]{
                \texttt{\tiny{[P1$\rangle$GM]}}
                \texttt{STRATEGISCHE ÜBERLEGUNGEN: \{'Ihr Beharren auf C64 und A61 ist für mich schlecht: beide haben für mich extrem geringen Wert (C64=65, A61=121). Mein Topnutzen kommt von A48, A04, B53, B21/A41, B85, A89, C04, B61. Ich will C64 und A61 durch A48 plus einen weiteren High{-}Value{-}Posten ersetzen. Um Kooperationsbereitschaft zu zeigen, behalte ich C49, der ihnen offenbar wichtig ist. Zwei starke, budgetkonforme Optionen: O1 \{A48,A04,B53,B61,B85,A89,C04,C49\} (Kosten 2663, mein Wert 4768). O2 \{A48,A04,B53,B61,B85,A89,C04,B21\} (Kosten 3016, Wert 5314). Ich biete O1 erneut, aber gebe eine klare Alternativspur: Wenn sie unbedingt eine Ergänzung wünschen, tausche C49 {-}> A41 oder B21. Falls sie auf C64/A61 pochen, kontere mit: Ein Kompromiss könnte nur einen der beiden (nicht beide) aufnehmen und dafür C49 streichen, plus A48 behalten, z. B. \{A48,A04,B53,B61,B85,A89,C04,C64\} (Kosten 274, falsch rechnen; korrekt: 505+116+571+285+4+586+209+940=3216), aber C64 ist für mich schwach – nur als letzte Zugeständnisoption. Ich argumentiere mit Budgeteffizienz und Qualitätsgewinn, biete klare Tauschklausel.'\}} \\
\texttt{ABLEHNUNG: \{'A04', 'B53', 'B61', 'B85', 'A89', 'C04', 'C49', 'C64', 'A61'\}} \\
\texttt{VORSCHLAG: \{'A48', 'A04', 'B53', 'B61', 'B85', 'A89', 'C04', 'C49'\}} \\
\texttt{ARGUMENT: \{'Wir behalten sieben deiner gewünschten Posten unverändert und ersetzen nur das schwächste Paar (C64, A61) durch die klar stärkere A48. Das ist budgeteffizient (2663 < 3681) und hebt die Gesamtqualität spürbar an. Wenn du noch einen zusätzlichen Schub willst, können wir C49 optional durch A41 oder B21 tauschen – beide passen weiterhin sicher ins Budget und lassen alle anderen Positionen unberührt.'\}} \\
            }
        }
    }
    & & \\ \\

    \theutterance \stepcounter{utterance}  
    & & \multicolumn{4}{p{0.6\linewidth}}{
        \cellcolor[rgb]{0.9,0.9,0.9}{
            \makecell[{{p{\linewidth}}}]{
                \texttt{\tiny{[GM$\rangle$P2]}}
                \texttt{ABLEHNUNG: \{'A04', 'B53', 'B61', 'B85', 'A89', 'C04', 'C49', 'C64', 'A61'\}} \\
\texttt{VORSCHLAG: \{'A48', 'A04', 'B53', 'B61', 'B85', 'A89', 'C04', 'C49'\}} \\
\texttt{ARGUMENT: \{'Wir behalten sieben deiner gewünschten Posten unverändert und ersetzen nur das schwächste Paar (C64, A61) durch die klar stärkere A48. Das ist budgeteffizient (2663 < 3681) und hebt die Gesamtqualität spürbar an. Wenn du noch einen zusätzlichen Schub willst, können wir C49 optional durch A41 oder B21 tauschen – beide passen weiterhin sicher ins Budget und lassen alle anderen Positionen unberührt.'\}} \\
            }
        }
    }
    & \\ \\

    \theutterance \stepcounter{utterance}  
    & & & \multicolumn{4}{p{0.6\linewidth}}{
        \cellcolor[rgb]{0.9,0.9,0.9}{
            \makecell[{{p{\linewidth}}}]{
                \texttt{\tiny{[GM$\langle$P2]}}
                \texttt{STRATEGISCHE ÜBERLEGUNGEN: \{'Er blockt hart gegen C64 und A61 und pusht A48. Sein aktueller Vorschlag Aufwand 2663, mein Wert 3601. Viel Luft im Budget. Ich versuche erneut minimal{-}invasive Aufwertung ohne C64/A61, um seine roten Linien zu respektieren: Ersetze A48 durch A03 und C33 (beide hohe Werte). Aufwand neu: 2663 {-} 505 + 613 + 700 = 3471 (<3681). Wertzuwachs: + (780+808 {-} 65) = +1523. Das ist riesig und vermeidet die von ihm abgelehnten C64/A61. Falls er C33 nicht will, Alternativen: A03 + C33 {-}> falls eine gestrichen wird, kann ich A61 als Ersatz vorschlagen, aber er lehnte A61; also als zweite Option A03 + A41 statt A48: 2663 {-} 505 + 613 + 664 = 3435, Wert + (780+262 {-} 65) = +977. Dritte, kleinste Option: nur A03 statt A48: 2771 Aufwand, Wert +715. Ich biete die starke mittlere (A03+C33) an und öffne Optionalpfade.\}} \\
\\ 
\texttt{ABLEHNUNG: \{'A48', 'A04', 'B53', 'B61', 'B85', 'A89', 'C04', 'C49'\}} \\
\texttt{VORSCHLAG: \{'A04', 'B53', 'B61', 'B85', 'A89', 'C04', 'C49', 'A03', 'C33'\}} \\
\texttt{ARGUMENT: \{'Dein Vorschlag lässt sehr viel Budget ungenutzt. Ohne an deinen sieben Kernposten zu rütteln, ersetze ich lediglich A48 durch zwei hocheffiziente Ergänzungen A03 und C33. Damit bleiben wir klar im Rahmen (Gesamtaufwand 3471 < 3681) und steigern die Gesamtqualität deutlich. Falls dir C33 nicht zusagt, können wir alternativ A41 statt C33 nehmen (Aufwand 3435) oder – noch schlanker – nur A03 anstelle von A48 (Aufwand 2771). Ich bin hier flexibel, solange wir die freie Budgetreserve sinnvoll nutzen.'\}} \\
            }
        }
    }
     \\ \\

    \theutterance \stepcounter{utterance}  
    & & \multicolumn{4}{p{0.6\linewidth}}{
        \cellcolor[rgb]{0.9,0.9,0.9}{
            \makecell[{{p{\linewidth}}}]{
                \texttt{\tiny{[P1$\langle$GM]}}
                \texttt{ABLEHNUNG: \{'A48', 'A04', 'B53', 'B61', 'B85', 'A89', 'C04', 'C49'\}} \\
\texttt{VORSCHLAG: \{'A04', 'B53', 'B61', 'B85', 'A89', 'C04', 'C49', 'A03', 'C33'\}} \\
\texttt{ARGUMENT: \{'Dein Vorschlag lässt sehr viel Budget ungenutzt. Ohne an deinen sieben Kernposten zu rütteln, ersetze ich lediglich A48 durch zwei hocheffiziente Ergänzungen A03 und C33. Damit bleiben wir klar im Rahmen (Gesamtaufwand 3471 < 3681) und steigern die Gesamtqualität deutlich. Falls dir C33 nicht zusagt, können wir alternativ A41 statt C33 nehmen (Aufwand 3435) oder – noch schlanker – nur A03 anstelle von A48 (Aufwand 2771). Ich bin hier flexibel, solange wir die freie Budgetreserve sinnvoll nutzen.'\}} \\
            }
        }
    }
    & \\ \\

    \theutterance \stepcounter{utterance}  
    & \multicolumn{4}{p{0.6\linewidth}}{
        \cellcolor[rgb]{0.9,0.9,0.9}{
            \makecell[{{p{\linewidth}}}]{
                \texttt{\tiny{[P1$\rangle$GM]}}
                \texttt{STRATEGISCHE ÜBERLEGUNGEN: \{'Ihr neuer Vorschlag S = \{A04,B53,B61,B85,A89,C04,C49,A03,C33\}. Kosten prüfen: 116+571+285+4+586+209+387+613+700 = 3471 (ok). Mein Wert dafür: 822+780+484+461+583+508+262+389+138 = 4427. Das ist deutlich schlechter als Sets mit A48 und ohne C33. Ich möchte C33 unbedingt vermeiden (sehr niedriger Wert 138) und A48 zurück reinnehmen. Ideal: \{A48,A04,B53,B61,B85,A89,C04,C49,A03\}? Kosten 505+116+571+285+4+586+209+387+613 = 3276, Wert 4768+389=5157? Korrekt: Basis \{A48,A04,B53,B61,B85,A89,C04,C49\} Wert 4768, plus A03(389) = 5157, stark und unter Budget. Das ist ein klarer Kompromiss: ich akzeptiere A03 (ihr Vorschlag) und behalte C49 (ihr Kern), streiche nur C33 und bringe A48 zurück. Argument: mehr Qualität, bleibt deutlich unter Limit, behält 8/9 ihrer Items (alle außer C33) plus A48. Als Alternativpfad biete ich, falls A48 für sie problematisch ist, den Tausch A48 {-}> A41 oder B21, beide passen mit A03 rein: z.B. \{A41,A04,B53,B61,B85,A89,C04,C49,A03\} Kosten 664+116+571+285+4+586+209+387+613=3435; oder mit B21 Kosten 740 statt 664 = 3511. Beide unter Budget, mein Wert hoch. Ich werde diese Hauptlösung vorschlagen und Optionen nennen.'\}} \\
\texttt{VORSCHLAG: \{'A48', 'A04', 'B53', 'B61', 'B85', 'A89', 'C04', 'C49', 'A03'\}} \\
\texttt{ARGUMENT: \{'Ich übernehme acht deiner neun Positionen unverändert und ersetze lediglich das schwächste Element (C33) durch die klar stärkere A48. Wir bleiben komfortabel unter dem Limit (Gesamtaufwand 3276) und steigern die Gesamtqualität spürbar. Falls du A48 bevorzugt ersetzen möchtest, können wir alternativ A41 oder B21 einsetzen – beide passen mit A03 weiterhin sicher ins Budget, ohne an den restlichen Posten zu rühren.'\}} \\
            }
        }
    }
    & & \\ \\

    \theutterance \stepcounter{utterance}  
    & & \multicolumn{4}{p{0.6\linewidth}}{
        \cellcolor[rgb]{0.9,0.9,0.9}{
            \makecell[{{p{\linewidth}}}]{
                \texttt{\tiny{[GM$\rangle$P2]}}
                \texttt{VORSCHLAG: \{'A48', 'A04', 'B53', 'B61', 'B85', 'A89', 'C04', 'C49', 'A03'\}} \\
\texttt{ARGUMENT: \{'Ich übernehme acht deiner neun Positionen unverändert und ersetze lediglich das schwächste Element (C33) durch die klar stärkere A48. Wir bleiben komfortabel unter dem Limit (Gesamtaufwand 3276) und steigern die Gesamtqualität spürbar. Falls du A48 bevorzugt ersetzen möchtest, können wir alternativ A41 oder B21 einsetzen – beide passen mit A03 weiterhin sicher ins Budget, ohne an den restlichen Posten zu rühren.'\}} \\
            }
        }
    }
    & \\ \\

    \theutterance \stepcounter{utterance}  
    & & & \multicolumn{4}{p{0.6\linewidth}}{
        \cellcolor[rgb]{0.9,0.9,0.9}{
            \makecell[{{p{\linewidth}}}]{
                \texttt{\tiny{[GM$\langle$P2]}}
                \texttt{STRATEGISCHE ÜBERLEGUNGEN: \{'Aktiver Vorschlag des Gegners: \{A48, A04, B53, B61, B85, A89, C04, C49, A03\} mit Aufwand 3276. Mein Wert dafür: 65+121+389+583+668+484+508+783+780 = 4381. Noch 405 Aufwand frei. Ich kann großen Mehrwert hinzufügen, ohne etwas zu entfernen: beste Ergänzung ist C33 (700 Aufwand, 808 Wert), aber das überschreitet 3681 (3276+700=3976). Nächstbeste, die passt: A41 (664,262) passt nicht (3276+664=3940). B21 (740,138) passt nicht. A61 (392,822) passt, wurde aber zuvor abgelehnt; dennoch minimal{-}invasiv und sehr effizient: neuer Aufwand 3668 (<3681), Wert +822. Wenn A61 tabu bleibt, Alternative: C64 (940) passt nicht; A03 ist schon drin; C49/C04/B85/B61 sind drin; verbleibt nur kleine Ergänzung A04 ist drin; C04 drin. Also beste Option: seinen Vorschlag +A61. Ich argumentiere mit Nutzung der Restkapazität und minimaler Änderung. Biete als Fallback eine noch kleinere, von ihm akzeptierte Option: wenn A61 nicht geht, akzeptiere ich seinen aktuellen Vorschlag, um Abschluss zu erzielen.'\}} \\
\texttt{VORSCHLAG: \{'A48', 'A04', 'B53', 'B61', 'B85', 'A89', 'C04', 'C49', 'A03', 'A61'\}} \\
\texttt{ARGUMENT: \{'Guter Fortschritt – dein Vorschlag nutzt das Budget noch nicht voll aus. Lass uns die verbleibende Reserve sinnvoll einsetzen: Mit A61 fügen wir eine einzelne, sehr effiziente Position hinzu. Der Gesamtaufwand bleibt sicher unter dem Limit (3668 < 3681), alle deine acht anderen Positionen bleiben unverändert, und der Gesamtnutzen steigt deutlich. Falls A61 für dich nicht in Frage kommt, sag Bescheid – dann können wir ohne weitere Änderungen deinen aktuellen Vorschlag direkt finalisieren.'\}} \\
            }
        }
    }
     \\ \\

    \theutterance \stepcounter{utterance}  
    & & \multicolumn{4}{p{0.6\linewidth}}{
        \cellcolor[rgb]{0.9,0.9,0.9}{
            \makecell[{{p{\linewidth}}}]{
                \texttt{\tiny{[P1$\langle$GM]}}
                \texttt{VORSCHLAG: \{'A48', 'A04', 'B53', 'B61', 'B85', 'A89', 'C04', 'C49', 'A03', 'A61'\}} \\
\texttt{ARGUMENT: \{'Guter Fortschritt – dein Vorschlag nutzt das Budget noch nicht voll aus. Lass uns die verbleibende Reserve sinnvoll einsetzen: Mit A61 fügen wir eine einzelne, sehr effiziente Position hinzu. Der Gesamtaufwand bleibt sicher unter dem Limit (3668 < 3681), alle deine acht anderen Positionen bleiben unverändert, und der Gesamtnutzen steigt deutlich. Falls A61 für dich nicht in Frage kommt, sag Bescheid – dann können wir ohne weitere Änderungen deinen aktuellen Vorschlag direkt finalisieren.'\}} \\
            }
        }
    }
    & \\ \\

    \theutterance \stepcounter{utterance}  
    & \multicolumn{4}{p{0.6\linewidth}}{
        \cellcolor[rgb]{0.9,0.9,0.9}{
            \makecell[{{p{\linewidth}}}]{
                \texttt{\tiny{[P1$\rangle$GM]}}
                \texttt{STRATEGISCHE ÜBERLEGUNGEN: \{'Ihr Zusatz A61 kostet 392 und bringt mir wenig Wert (121). Ich möchte stattdessen A41 (664, hoher Wert) oder B21 (740, sehr hoher Wert) hineinnehmen; beide passen noch ins Budget: Mit A41: Aufwand 3276 + 664 = 3940? Stop, neu rechnen: Unser aktueller 9er{-}Vorschlag \{A48,A04,B53,B61,B85,A89,C04,C49,A03\} kostet 3276. + A41 (664) = 3940, das überschreitet 3681 – also nicht möglich. + B21 (740) = 4016, ebenfalls zu hoch. Wir haben nur 405 Rest (3681{-}3276). Passende Alternativen unter 405: C49 (schon drin), C04 (drin), B61 (drin), A61 (392) – niedriger Wert; C49 zweites Mal geht nicht; C33 (700) zu groß; A04 (116) schon drin; C04 209 schon drin; C49 387 schon drin; C49 ist 387, wir haben es bereits. Welche Items mit Kosten <=405 und nicht drin: C49 (387) already in; C04 (209) in; A04 (116) in; B61 (285) in; B85 (4) in; A61 (392) nicht drin – einziges neues passendes Item ist A61. Oder C04/C49 duplizieren geht nicht. Also, wenn wir auf 10 Items erweitern, A61 ist der einzige, der noch reinpasst. Ich hasse A61, aber Einigung ist nah. Alternative ist, bei 9 Items zu bleiben und ihre Option abzulehnen und vielleicht fordern, A61 nur wenn wir C49 gegen A41 tauschen? Prüfen: Ersetzen C49 (387) durch A41 (664) erhöht Kosten um 277: 3276 {-} 387 + 664 = 3553, passt! Das wäre viel besser für mich. Verhandle: Akzeptiere 10er{-}Set nicht, biete stattdessen Austausch C49 {-}> A41, Gesamtaufwand 3553 < 3681. Oder, wenn sie auf A61 bestehen, dann schlage Doppeltausch vor: C49 {-}> A41 und zusätzlich A61 rein? Kosten: 3276 {-} 387 + 664 + 392 = 3945 > Limit, geht nicht. Also Option anbieten: entweder (i) belassen wir 9er{-}Set wie vorgeschlagen (ohne A61) und finalisieren, oder (ii) tauschen C49 gegen A41 für deutlichen Mehrwert, oder (iii) falls sie unbedingt eine Ergänzung will, nehme ich A61 nur im Gegenzug für C49{-}>A41? Das geht nicht budgetär. Also klare Gegenofferte: 9er{-}Set + A41 statt C49. Ich argumentiere mit gleichem Aufwandkorridor und höherem Gesamtnutzen. Falls sie ablehnt, als Fallback kann ich ihr 10er{-}Set mit A61 widerwillig akzeptieren, aber erst nach einem Versuch.'\}} \\
\texttt{ABLEHNUNG: \{'A48', 'A04', 'B53', 'B61', 'B85', 'A89', 'C04', 'C49', 'A03', 'A61'\}} \\
\texttt{VORSCHLAG: \{'A48', 'A04', 'B53', 'B61', 'B85', 'A89', 'C04', 'A03', 'A41'\}} \\
\texttt{ARGUMENT: \{'Gerne nutze ich die verbleibende Reserve sinnvoll – aber noch effizienter: Wir tauschen nur C49 gegen A41. Der Gesamtaufwand bleibt klar unter dem Limit (3553 < 3681), alle anderen Positionen bleiben identisch, und der Gesamtnutzen steigt deutlich stärker als mit A61. Wenn dir das nicht zusagt, können wir auch direkt deinen 9{-}Posten{-}Vorschlag ohne Zusatz finalisieren.'\}} \\
            }
        }
    }
    & & \\ \\

    \theutterance \stepcounter{utterance}  
    & & \multicolumn{4}{p{0.6\linewidth}}{
        \cellcolor[rgb]{0.9,0.9,0.9}{
            \makecell[{{p{\linewidth}}}]{
                \texttt{\tiny{[GM$\rangle$P2]}}
                \texttt{ABLEHNUNG: \{'A48', 'A04', 'B53', 'B61', 'B85', 'A89', 'C04', 'C49', 'A03', 'A61'\}} \\
\texttt{VORSCHLAG: \{'A48', 'A04', 'B53', 'B61', 'B85', 'A89', 'C04', 'A03', 'A41'\}} \\
\texttt{ARGUMENT: \{'Gerne nutze ich die verbleibende Reserve sinnvoll – aber noch effizienter: Wir tauschen nur C49 gegen A41. Der Gesamtaufwand bleibt klar unter dem Limit (3553 < 3681), alle anderen Positionen bleiben identisch, und der Gesamtnutzen steigt deutlich stärker als mit A61. Wenn dir das nicht zusagt, können wir auch direkt deinen 9{-}Posten{-}Vorschlag ohne Zusatz finalisieren.'\}} \\
            }
        }
    }
    & \\ \\

    \theutterance \stepcounter{utterance}  
    & & & \multicolumn{4}{p{0.6\linewidth}}{
        \cellcolor[rgb]{0.9,0.9,0.9}{
            \makecell[{{p{\linewidth}}}]{
                \texttt{\tiny{[GM$\langle$P2]}}
                \texttt{STRATEGISCHE ÜBERLEGUNGEN: \{'Sein aktiver Vorschlag: \{A48, A04, B53, B61, B85, A89, C04, A03, A41\} Aufwand 3553. Mein Wert: 65+121+389+583+668+484+508+780+262 = 3860. Sein vorheriger 9er mit C49 statt A41 hatte Aufwand 3276 und Wert 4381 für mich – also war für mich deutlich besser. Ich will C49 zurückholen. Prüfe Konter: Ersetze A41 wieder durch C49: Aufwand 3553 {-} 664 + 387 = 3276; mein Wert + (783 {-} 262) = +521; besser. Zusätzlich könnte ich versuchen, noch A61 hinzuzunehmen (er lehnt A61 oft ab). 3276 + 392 = 3668, im Budget, Mehrwert +822 – sehr stark. Aber er hat gerade A61 abgelehnt. Biete zwei Varianten: Primär minimal: swap A41 {-}> C49 (zurück zu seinem eigenen 9er, den er schon fast akzeptiert hatte). Alternativ falls er mehr Nutzen will: plus A61. Wenn er das ablehnt, bin bereit, den 9er ohne A61 zu akzeptieren, um Abschluss zu erzielen. Taktik: erst Konter mit C49 zurück, ohne A61 fordern; optional erwähnen, dass wir bei Wunsch A61 ergänzen können. Falls weiterhin Widerstand, Zustimmung zu seinem 9er? Sein 9er aktuell ist schlechter für mich, daher erst versuche Rücktausch.'\}} \\
\texttt{ABLEHNUNG: \{'A48', 'A04', 'B53', 'B61', 'B85', 'A89', 'C04', 'A03', 'A41'\}} \\
\texttt{VORSCHLAG: \{'A48', 'A04', 'B53', 'B61', 'B85', 'A89', 'C04', 'A03', 'C49'\}} \\
\texttt{ARGUMENT: \{'Lass uns es ganz schlank halten: Wir drehen nur den letzten Tausch zurück und nehmen C49 statt A41. Damit sinkt der Aufwand auf 3276 (< 3681) und wir behalten alle anderen Positionen exakt wie von dir vorgeschlagen. Wenn du dennoch einen kleinen Mehrwert obendrauf willst, können wir optional A61 ergänzen (Gesamtaufwand 3668) – ansonsten bin ich bereit, diese 9er{-}Kombination direkt zu finalisieren.'\}} \\
            }
        }
    }
     \\ \\

    \theutterance \stepcounter{utterance}  
    & & \multicolumn{4}{p{0.6\linewidth}}{
        \cellcolor[rgb]{0.9,0.9,0.9}{
            \makecell[{{p{\linewidth}}}]{
                \texttt{\tiny{[P1$\langle$GM]}}
                \texttt{ABLEHNUNG: \{'A48', 'A04', 'B53', 'B61', 'B85', 'A89', 'C04', 'A03', 'A41'\}} \\
\texttt{VORSCHLAG: \{'A48', 'A04', 'B53', 'B61', 'B85', 'A89', 'C04', 'A03', 'C49'\}} \\
\texttt{ARGUMENT: \{'Lass uns es ganz schlank halten: Wir drehen nur den letzten Tausch zurück und nehmen C49 statt A41. Damit sinkt der Aufwand auf 3276 (< 3681) und wir behalten alle anderen Positionen exakt wie von dir vorgeschlagen. Wenn du dennoch einen kleinen Mehrwert obendrauf willst, können wir optional A61 ergänzen (Gesamtaufwand 3668) – ansonsten bin ich bereit, diese 9er{-}Kombination direkt zu finalisieren.'\}} \\
            }
        }
    }
    & \\ \\

    \theutterance \stepcounter{utterance}  
    & \multicolumn{4}{p{0.6\linewidth}}{
        \cellcolor[rgb]{0.9,0.9,0.9}{
            \makecell[{{p{\linewidth}}}]{
                \texttt{\tiny{[P1$\rangle$GM]}}
                \texttt{STRATEGISCHE ÜBERLEGUNGEN: \{'Ihr letztes Angebot ist exakt mein bevorzugtes 9er{-}Set \{A48,A04,B53,B61,B85,A89,C04,A03,C49\} mit Kosten 3276 und hohem Gesamtwert für mich. Sie bieten optional A61 an, den ich nur geringen Wert beimesse. Beststrategie: Sofort das 9er{-}Set ohne A61 finalisieren, um kein weiteres Zugeständnis machen zu müssen. Regeln beachten: Ich kann nur zustimmen, wenn es ein aktiver Vorschlag des Gegners ist – das ist gegeben.'\}} \\
\texttt{ZUSTIMMUNG: \{'A48', 'A04', 'B53', 'B61', 'B85', 'A89', 'C04', 'A03', 'C49'\}} \\
\texttt{ARGUMENT: \{'Einverstanden. Diese 9er{-}Kombination bleibt komfortabel unter dem Limit und trifft genau die von dir priorisierten Posten. Gute Einigung.'\}} \\
            }
        }
    }
    & & \\ \\

    \theutterance \stepcounter{utterance}  
    & & & \multicolumn{2}{p{0.3\linewidth}}{
        \cellcolor[rgb]{0.9,0.9,0.9}{
            \makecell[{{p{\linewidth}}}]{
                \texttt{\tiny{[GM$|$GM]}}
                \texttt{game successful} \\
            }
        }
    }
    & & \\ \\

    \theutterance \stepcounter{utterance}  
    & & & \multicolumn{2}{p{0.3\linewidth}}{
        \cellcolor[rgb]{0.9,0.9,0.9}{
            \makecell[{{p{\linewidth}}}]{
                \texttt{\tiny{[GM$|$GM]}}
                \texttt{end game} \\
            }
        }
    }
    & & \\ \\

\end{supertabular}
}

\end{document}
