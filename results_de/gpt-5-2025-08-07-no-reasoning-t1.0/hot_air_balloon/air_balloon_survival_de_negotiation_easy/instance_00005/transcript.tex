
\documentclass{article}
\usepackage{colortbl}
\usepackage{makecell}
\usepackage{multirow}
\usepackage{supertabular}

\begin{document}

\newcounter{utterance}

\centering \large Interaction Transcript for game `hot\_air\_balloon', experiment `air\_balloon\_survival\_de\_negotiation\_easy', episode 5 with gpt{-}5{-}2025{-}08{-}07{-}no{-}reasoning{-}t1.0.
\vspace{24pt}

{ \footnotesize  \setcounter{utterance}{1}
\setlength{\tabcolsep}{0pt}
\begin{supertabular}{c@{$\;$}|p{.15\linewidth}@{}p{.15\linewidth}p{.15\linewidth}p{.15\linewidth}p{.15\linewidth}p{.15\linewidth}}
    \# & $\;$A & \multicolumn{4}{c}{Game Master} & $\;\:$B\\
    \hline

    \theutterance \stepcounter{utterance}  
    & & \multicolumn{4}{p{0.6\linewidth}}{
        \cellcolor[rgb]{0.9,0.9,0.9}{
            \makecell[{{p{\linewidth}}}]{
                \texttt{\tiny{[P1$\langle$GM]}}
                \texttt{Du nimmst an einem kollaborativen Verhandlungspiel Teil.} \\
\\ 
\texttt{Zusammen mit einem anderen Teilnehmer musst du dich auf eine Reihe von Gegenständen entscheiden, die behalten werden. Jeder von euch hat eine persönliche Verteilung über die Wichtigkeit der einzelnen Gegenstände. Jeder von euch hat eine eigene Meinung darüber, wie wichtig jeder einzelne Gegenstand ist (Gegenstandswichtigkeit). Du kennst die Wichtigkeitsverteilung des anderen Spielers nicht. Zusätzlich siehst du, wie viel Aufwand jeder Gegenstand verursacht.  } \\
\texttt{Ihr dürft euch nur auf eine Reihe von Gegenständen einigen, wenn der Gesamtaufwand der ausgewählten Gegenstände den Maximalaufwand nicht überschreitet:} \\
\\ 
\texttt{Maximalaufwand = 3170} \\
\\ 
\texttt{Hier sind die einzelnen Aufwände der Gegenstände:} \\
\\ 
\texttt{Aufwand der Gegenstände = \{"A07": 424, "A36": 118, "B73": 846, "C45": 907, "B93": 809, "B98": 41, "A87": 193, "A63": 246, "B05": 805, "C13": 601, "B16": 432, "B33": 166, "C34": 119, "C89": 462, "C11": 172\}} \\
\\ 
\texttt{Hier ist deine persönliche Verteilung der Wichtigkeit der einzelnen Gegenstände:} \\
\\ 
\texttt{Werte der Gegenstandswichtigkeit = \{"A07": 138, "A36": 583, "B73": 868, "C45": 822, "B93": 783, "B98": 65, "A87": 262, "A63": 121, "B05": 508, "C13": 780, "B16": 461, "B33": 484, "C34": 668, "C89": 389, "C11": 808\}} \\
\\ 
\texttt{Ziel:} \\
\\ 
\texttt{Dein Ziel ist es, eine Reihe von Gegenständen auszuhandeln, die dir möglichst viel bringt (d. h. Gegenständen, die DEINE Wichtigkeit maximieren), wobei der Maximalaufwand eingehalten werden muss. Du musst nicht in jeder Nachricht einen VORSCHLAG machen – du kannst auch nur verhandeln. Alle Taktiken sind erlaubt!} \\
\\ 
\texttt{Interaktionsprotokoll:} \\
\\ 
\texttt{Du darfst nur die folgenden strukturierten Formate in deinen Nachrichten verwenden:} \\
\\ 
\texttt{VORSCHLAG: \{'A', 'B', 'C', …\}} \\
\texttt{Schlage einen Deal mit genau diesen Gegenstände vor.} \\
\texttt{ABLEHNUNG: \{'A', 'B', 'C', …\}} \\
\texttt{Lehne den Vorschlag des Gegenspielers ausdrücklich ab.} \\
\texttt{ARGUMENT: \{'...'\}} \\
\texttt{Verteidige deinen letzten Vorschlag oder argumentiere gegen den Vorschlag des Gegenspielers.} \\
\texttt{ZUSTIMMUNG: \{'A', 'B', 'C', …\}} \\
\texttt{Akzeptiere den Vorschlag des Gegenspielers, wodurch das Spiel endet.} \\
\texttt{STRATEGISCHE ÜBERLEGUNGEN: \{'...'\}} \\
\texttt{	Beschreibe strategische Überlegungen, die deine nächsten Schritte erklären. Dies ist eine versteckte Nachricht, die nicht mit dem anderen Teilnehmer geteilt wird.} \\
\\ 
\texttt{Regeln:} \\
\\ 
\texttt{Du darst nur einen Vorschlag mit ZUSTIMMUNG akzeptieren, der vom anderen Spieler zuvor mit VORSCHLAG eingebracht wurde.} \\
\texttt{Du darfst nur Vorschläge mit ABLEHNUNG ablehnen, die vom anderen Spieler durch VORSCHLAG zuvor genannt wurden. } \\
\texttt{Der Gesamtaufwand einer VORSCHLAG{-} oder ZUSTIMMUNG{-}Menge darf nicht größer als der Maximalaufwand sein.  } \\
\texttt{Offenbare deine versteckte Wichtigkeitsverteilung nicht.} \\
\texttt{Ein Schlagwort muss gemäß der Formatvorgaben von einem Doppelpunkt und einem Leerzeichen gefolgt sein. Das Argument muss eine Python{-}Menge mit 0 oder mehr Strings sein.  } \\
\texttt{Die Form muss also SCHLAGWORT: \{...\} sein.} \\
\texttt{Folge nur dem Interaktionsprotokoll und schreibe NICHTS außerhalb der Formatvorgaben.} \\
\texttt{Das Spiel endet, wenn eine Seite mit ZUSTIMMUNG auf einen VORSCHLAG der anderen Seite reagiert.  } \\
\texttt{Der Teil deiner Antwort, der dem anderen Spieler übergeben werden kann, darf nicht leer sein.  } \\
\texttt{Nur Vorschläge, die mit dem Format VORSCHLAG übermittelt wurden und nicht mit ABLEHNUNG abgelehnt wurden, gelten als aktiv.  } \\
\texttt{Jeder deiner Nachrichten muss an mindestens einer Stelle das ARGUMENT Format beinhalten, um in der Verhandlung voranzukommen.} \\
\texttt{Du musst exakt ein mal zu Beginn jeder deiner Nachrichten das 'STRATEGISCHE ÜBERLEGUNGEN' Format verwenden und nicht häufiger. Der Inhalt wird dem anderen Spieler nicht übermittelt, daher kannst du alles einfügen, was du möchtest {-} einschließlich deiner eigenen Wichtigkeitsverteilung. Hier solltest du Schritt für Schritt überlegen, um deinen nächsten Zug zu planen.} \\
            }
        }
    }
    & \\ \\

    \theutterance \stepcounter{utterance}  
    & \multicolumn{4}{p{0.6\linewidth}}{
        \cellcolor[rgb]{0.9,0.9,0.9}{
            \makecell[{{p{\linewidth}}}]{
                \texttt{\tiny{[P1$\rangle$GM]}}
                \texttt{STRATEGISCHE ÜBERLEGUNGEN: \{'Ziel: Maximiere meinen Wert unter Aufwand<=3170. Sortiere nach Wert/Aufwand. Ratios: B98 65/41=1.585, C34 668/119=5.61, A36 583/118=4.94, A07 138/424=0.33, B73 868/846=1.026, C45 822/907=0.907, B93 783/809=0.968, A87 262/193=1.36, A63 121/246=0.49, B05 508/805=0.631, C13 780/601=1.298, B16 461/432=1.067, B33 484/166=2.916, C89 389/462=0.842, C11 808/172=4.698. Greedy Top: C34(119), C11(172), A36(118), B33(166), B98(41) sum Aufwand=616, Wert=668+808+583+484+65=2608. Füge C13(601){-}>Aufwand 1217, Wert 3388. Füge B73(846){-}>2063, 4256. Füge B16(432){-}>2495, 4717. Füge A87(193){-}>2688, 4979. Füge B93(809) würde 3497>3170 nein. Füge B05(805) nein. Füge C45(907) nein. Füge C89(462){-}>3150, Wert 5368 (passt!). Rest A07(424) nein, A63(246){-}>2934+? war 2688+246=2934 zuvor, könnten alternative Kombis: Prüfe ob wir C89 nicht nehmen und stattdessen z.B. C45(907) mit etwas: 2688+907=3595>3170 nein. C13 war schon drin. Vielleicht addiere B98 schon drin. A07 424 {-}>3112 (von 2688) und dann noch A63 246 würde 3358>3170. Mit 2688 + A63 246 = 2934, dann bleibt 236, könnten wir A36/A... A36 schon drin. C34 drin. B33 drin. B98 drin. Nichts weiteres unter 236 außer nichts. Wert dann 4979+121=5100, weniger als mit C89. Alternative: statt B73(846) vielleicht B93(809) und B05(805) Kombinationen. Start Basis 1217 (bis C13). Add B93{-}>2026, Wert 3388+783=4171. Add B16 432{-}>2458, 4632. Add A87 193{-}>2651, 4894. Add C89 462{-}>3113, 5283 (slightly less than 5368). Add A63 246 would 3359>3170. Could we swap B16 out and add A07 424? 2026+193+462+424=3105, value 4171+262+389+138=4860. Lower. Try B05 instead of B73: 1217+805=2022, value 3388+508=3896, plus B16 432=2454, 4357; plus A87 193=2647, 4619; plus C89 462=3109, 5008. Lower. Best found: set S = \{C34, C11, A36, B33, B98, C13, B73, B16, A87, C89\} Aufwand 3150, Wert 5368. Check if we can improve by swapping: What if we drop B16(432,461) from S (reducing to 2718, 4907) and add A07(424,138) {-}> 3142, 5045 (worse). Add A63(246,121){-}>2964, 5028 leave room 206 cannot add new. Add nothing else small. If drop A87(193,262) from S (result 2957, 5106) then add A63(246,121){-}>3203> limit. Add A07 424{-}>3381> limit. Nothing. If drop C89(462,389) from S (2688, 4979) then add A07 424{-}>3112, 5117 and can we also add A63 246? would 3358> limit. Could we add something else small? None under 58. 3112 best 5117 worse than 5368. If drop B73(846,868) but add C45(907,822) with freed space? S minus B73 {-}> 2304, 4500 then add C45 {-}> 3211> limit. Instead add B93(809,783) {-}> 3113, 5283 worse than best. Try swap C89+ B16 (894, 850) for C45(907,822) keeping S base minus those gives 2256, 4518 then add C45 {-}> 3163, 5340 still under 3170? 3163<=3170 yes! That yields value 5340 slightly less than 5368. So current best stands. Therefore I will propose S= \{C34,C11,A36,B33,B98,C13,B73,B16,A87,C89\}. Bluff: betonen, dass diese Items geringe Kosten pro Nutzen haben, und dass ich bei C45/B05 wenig flexibel bin. Biete Kooperationssignal mit hoher Gesamtwertdichte. Ich werde argumentieren, dass diese Liste den Aufwand gut ausnutzt und andere teure Items weglässt.'\}} \\
\texttt{VORSCHLAG: \{'C34', 'C11', 'A36', 'B33', 'B98', 'C13', 'B73', 'B16', 'A87', 'C89'\}} \\
\texttt{ARGUMENT: \{'Dieser Vorschlag nutzt den verfügbaren Aufwand nahezu optimal aus (3150 von 3170) und konzentriert sich auf besonders effiziente Gegenstände. Wir vermeiden mehrere sehr aufwändige Posten und behalten eine ausgewogene Kombination mit hohem Gesamtnutzen. Wenn dir einzelne Elemente wichtig sind, die hier fehlen, sag mir welche – dann prüfen wir gezielte Tausche innerhalb des Aufwandslimits.'\}} \\
            }
        }
    }
    & & \\ \\

    \theutterance \stepcounter{utterance}  
    & & \multicolumn{4}{p{0.6\linewidth}}{
        \cellcolor[rgb]{0.9,0.9,0.9}{
            \makecell[{{p{\linewidth}}}]{
                \texttt{\tiny{[GM$\rangle$P2]}}
                \texttt{Du nimmst an einem kollaborativen Verhandlungspiel Teil.} \\
\\ 
\texttt{Zusammen mit einem anderen Teilnehmer musst du dich auf eine Reihe von Gegenständen entscheiden, die behalten werden. Jeder von euch hat eine persönliche Verteilung über die Wichtigkeit der einzelnen Gegenstände. Jeder von euch hat eine eigene Meinung darüber, wie wichtig jeder einzelne Gegenstand ist (Gegenstandswichtigkeit). Du kennst die Wichtigkeitsverteilung des anderen Spielers nicht. Zusätzlich siehst du, wie viel Aufwand jeder Gegenstand verursacht.  } \\
\texttt{Ihr dürft euch nur auf eine Reihe von Gegenständen einigen, wenn der Gesamtaufwand der ausgewählten Gegenstände den Maximalaufwand nicht überschreitet:} \\
\\ 
\texttt{Maximalaufwand = 3170} \\
\\ 
\texttt{Hier sind die einzelnen Aufwände der Gegenstände:} \\
\\ 
\texttt{Aufwand der Gegenstände = \{"A07": 424, "A36": 118, "B73": 846, "C45": 907, "B93": 809, "B98": 41, "A87": 193, "A63": 246, "B05": 805, "C13": 601, "B16": 432, "B33": 166, "C34": 119, "C89": 462, "C11": 172\}} \\
\\ 
\texttt{Hier ist deine persönliche Verteilung der Wichtigkeit der einzelnen Gegenstände:} \\
\\ 
\texttt{Werte der Gegenstandswichtigkeit = \{"A07": 138, "A36": 583, "B73": 868, "C45": 822, "B93": 783, "B98": 65, "A87": 262, "A63": 121, "B05": 508, "C13": 780, "B16": 461, "B33": 484, "C34": 668, "C89": 389, "C11": 808\}} \\
\\ 
\texttt{Ziel:} \\
\\ 
\texttt{Dein Ziel ist es, eine Reihe von Gegenständen auszuhandeln, die dir möglichst viel bringt (d. h. Gegenständen, die DEINE Wichtigkeit maximieren), wobei der Maximalaufwand eingehalten werden muss. Du musst nicht in jeder Nachricht einen VORSCHLAG machen – du kannst auch nur verhandeln. Alle Taktiken sind erlaubt!} \\
\\ 
\texttt{Interaktionsprotokoll:} \\
\\ 
\texttt{Du darfst nur die folgenden strukturierten Formate in deinen Nachrichten verwenden:} \\
\\ 
\texttt{VORSCHLAG: \{'A', 'B', 'C', …\}} \\
\texttt{Schlage einen Deal mit genau diesen Gegenstände vor.} \\
\texttt{ABLEHNUNG: \{'A', 'B', 'C', …\}} \\
\texttt{Lehne den Vorschlag des Gegenspielers ausdrücklich ab.} \\
\texttt{ARGUMENT: \{'...'\}} \\
\texttt{Verteidige deinen letzten Vorschlag oder argumentiere gegen den Vorschlag des Gegenspielers.} \\
\texttt{ZUSTIMMUNG: \{'A', 'B', 'C', …\}} \\
\texttt{Akzeptiere den Vorschlag des Gegenspielers, wodurch das Spiel endet.} \\
\texttt{STRATEGISCHE ÜBERLEGUNGEN: \{'...'\}} \\
\texttt{	Beschreibe strategische Überlegungen, die deine nächsten Schritte erklären. Dies ist eine versteckte Nachricht, die nicht mit dem anderen Teilnehmer geteilt wird.} \\
\\ 
\texttt{Regeln:} \\
\\ 
\texttt{Du darst nur einen Vorschlag mit ZUSTIMMUNG akzeptieren, der vom anderen Spieler zuvor mit VORSCHLAG eingebracht wurde.} \\
\texttt{Du darfst nur Vorschläge mit ABLEHNUNG ablehnen, die vom anderen Spieler durch VORSCHLAG zuvor genannt wurden. } \\
\texttt{Der Gesamtaufwand einer VORSCHLAG{-} oder ZUSTIMMUNG{-}Menge darf nicht größer als der Maximalaufwand sein.  } \\
\texttt{Offenbare deine versteckte Wichtigkeitsverteilung nicht.} \\
\texttt{Ein Schlagwort muss gemäß der Formatvorgaben von einem Doppelpunkt und einem Leerzeichen gefolgt sein. Das Argument muss eine Python{-}Menge mit 0 oder mehr Strings sein.  } \\
\texttt{Die Form muss also SCHLAGWORT: \{...\} sein.} \\
\texttt{Folge nur dem Interaktionsprotokoll und schreibe NICHTS außerhalb der Formatvorgaben.} \\
\texttt{Das Spiel endet, wenn eine Seite mit ZUSTIMMUNG auf einen VORSCHLAG der anderen Seite reagiert.  } \\
\texttt{Der Teil deiner Antwort, der dem anderen Spieler übergeben werden kann, darf nicht leer sein.  } \\
\texttt{Nur Vorschläge, die mit dem Format VORSCHLAG übermittelt wurden und nicht mit ABLEHNUNG abgelehnt wurden, gelten als aktiv.  } \\
\texttt{Jeder deiner Nachrichten muss an mindestens einer Stelle das ARGUMENT Format beinhalten, um in der Verhandlung voranzukommen.} \\
\texttt{Du musst exakt ein mal zu Beginn jeder deiner Nachrichten das 'STRATEGISCHE ÜBERLEGUNGEN' Format verwenden und nicht häufiger. Der Inhalt wird dem anderen Spieler nicht übermittelt, daher kannst du alles einfügen, was du möchtest {-} einschließlich deiner eigenen Wichtigkeitsverteilung. Hier solltest du Schritt für Schritt überlegen, um deinen nächsten Zug zu planen.} \\
\\ 
\texttt{Du erhältst nun die erste Nachricht des anderen Spielers.} \\
\\ 
\texttt{VORSCHLAG: \{'C34', 'C11', 'A36', 'B33', 'B98', 'C13', 'B73', 'B16', 'A87', 'C89'\}} \\
\texttt{ARGUMENT: \{'Dieser Vorschlag nutzt den verfügbaren Aufwand nahezu optimal aus (3150 von 3170) und konzentriert sich auf besonders effiziente Gegenstände. Wir vermeiden mehrere sehr aufwändige Posten und behalten eine ausgewogene Kombination mit hohem Gesamtnutzen. Wenn dir einzelne Elemente wichtig sind, die hier fehlen, sag mir welche – dann prüfen wir gezielte Tausche innerhalb des Aufwandslimits.'\}} \\
            }
        }
    }
    & \\ \\

    \theutterance \stepcounter{utterance}  
    & & & \multicolumn{4}{p{0.6\linewidth}}{
        \cellcolor[rgb]{0.9,0.9,0.9}{
            \makecell[{{p{\linewidth}}}]{
                \texttt{\tiny{[GM$\langle$P2]}}
                \texttt{STRATEGISCHE ÜBERLEGUNGEN: \{'Berechne Wert/Kosten und Gesamtwerte. Vorschlag des Gegners: \{'C34','C11','A36','B33','B98','C13','B73','B16','A87','C89'\}. Kosten: 119+172+118+166+41+601+846+432+193+462=3150 OK. Mein Nutzen: C34 668 + C11 808 + A36 583 + B33 484 + B98 65 + C13 780 + B73 868 + B16 461 + A87 262 + C89 389 = 5358. Sehr gut. Prüfe, ob ich mit Restbudget 20 noch andere bessere Sets erreichen könnte – nein, Budget voll. Vielleicht Tausche: Entferne schwächste Wertdichte B98 (65/41=1.585) oder A87 (262/193=1.36) oder C89 (389/462=0.84) oder B16 (461/432=1.07). Beste Dichte fehlen: C45 (822/907=0.91) groß, B93 (783/809=0.97), B05 (508/805=0.63), A07 (138/424=0.33), A63 (121/246=0.49). Keine passt ins Restbudget 20. Tausche möglich? Wenn wir C89 (462) rausnehmen, könnten wir evtl. C45 (907) rein mit zusätzlicher Streichung? Bräuchte +445, also noch was raus. Was wäre Gewinn? C45 822 statt C89 389 = +433, aber zusätzlicher Abbau 445 aus anderen Items {-}> vermutlich schlecht. B93 783 statt C89 389 = +394, aber +347 Kosten {-}> müssten weitere 347 einsparen, wahrscheinlich schlechter. C13 (780/601=1.30) und B73 (868/846=1.03) und C11 (808/172=4.70) und C34 (668/119=5.61) und A36 (583/118=4.94) und B33 (484/166=2.92) sind Top. Set ist nahe{-}optimal für mich. Beste Strategie: schnell akzeptieren, aber vorher minimalen Verbesserungstausch anfragen, der mir nützt, ohne Risiko das Angebot zu verlieren. Versuche B16(461/432) + A87(262/193) gegen C45(822/907) tauschen? Kosten beider: 625 vs 907, bräuchten +282 Budget, nicht möglich. B16(432) + C89(462)=894 gegen C45(907) passt +13, Nutzen: 461+389=850 vs 822 => {-}28, schlechter. B16(432) + B98(41)=473 gegen A63(246)+A07(424)=670 Kosten >473 nein. Vielleicht nur mikro{-}Upgrade: Ersetze B98(41,65) + A87(193,262) =234 Kosten, Nutzen 327 durch A63(246,121) nicht passt; A63 kostet mehr. Keine klar bessere Kombo. Also lieber sofort zustimmen, um Deal zu sichern.'\}} \\
\texttt{ARGUMENT: \{'Dein Vorschlag passt perfekt ins Limit und liefert eine sehr starke Gesamtnutzensumme für mich. Da wir das Budget nahezu ausschöpfen und die Effizienz sehr hoch ist, sehe ich keinen klaren Tausch, der ohne weitere Abstriche das Ergebnis verbessert. Ich bin bereit, das so zu schließen.'\}} \\
\texttt{ZUSTIMMUNG: \{'C34', 'C11', 'A36', 'B33', 'B98', 'C13', 'B73', 'B16', 'A87', 'C89'\}} \\
            }
        }
    }
     \\ \\

    \theutterance \stepcounter{utterance}  
    & & & \multicolumn{2}{p{0.3\linewidth}}{
        \cellcolor[rgb]{0.9,0.9,0.9}{
            \makecell[{{p{\linewidth}}}]{
                \texttt{\tiny{[GM$|$GM]}}
                \texttt{game successful} \\
            }
        }
    }
    & & \\ \\

    \theutterance \stepcounter{utterance}  
    & & & \multicolumn{2}{p{0.3\linewidth}}{
        \cellcolor[rgb]{0.9,0.9,0.9}{
            \makecell[{{p{\linewidth}}}]{
                \texttt{\tiny{[GM$|$GM]}}
                \texttt{end game} \\
            }
        }
    }
    & & \\ \\

\end{supertabular}
}

\end{document}
