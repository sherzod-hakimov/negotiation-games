
\documentclass{article}
\usepackage{colortbl}
\usepackage{makecell}
\usepackage{multirow}
\usepackage{supertabular}

\begin{document}

\newcounter{utterance}

\centering \large Interaction Transcript for game `hot\_air\_balloon', experiment `air\_balloon\_survival\_de\_negotiation\_easy', episode 4 with gpt{-}5{-}2025{-}08{-}07{-}no{-}reasoning{-}t1.0.
\vspace{24pt}

{ \footnotesize  \setcounter{utterance}{1}
\setlength{\tabcolsep}{0pt}
\begin{supertabular}{c@{$\;$}|p{.15\linewidth}@{}p{.15\linewidth}p{.15\linewidth}p{.15\linewidth}p{.15\linewidth}p{.15\linewidth}}
    \# & $\;$A & \multicolumn{4}{c}{Game Master} & $\;\:$B\\
    \hline

    \theutterance \stepcounter{utterance}  
    & & \multicolumn{4}{p{0.6\linewidth}}{
        \cellcolor[rgb]{0.9,0.9,0.9}{
            \makecell[{{p{\linewidth}}}]{
                \texttt{\tiny{[P1$\langle$GM]}}
                \texttt{Du nimmst an einem kollaborativen Verhandlungspiel Teil.} \\
\\ 
\texttt{Zusammen mit einem anderen Teilnehmer musst du dich auf eine Reihe von Gegenständen entscheiden, die behalten werden. Jeder von euch hat eine persönliche Verteilung über die Wichtigkeit der einzelnen Gegenstände. Jeder von euch hat eine eigene Meinung darüber, wie wichtig jeder einzelne Gegenstand ist (Gegenstandswichtigkeit). Du kennst die Wichtigkeitsverteilung des anderen Spielers nicht. Zusätzlich siehst du, wie viel Aufwand jeder Gegenstand verursacht.  } \\
\texttt{Ihr dürft euch nur auf eine Reihe von Gegenständen einigen, wenn der Gesamtaufwand der ausgewählten Gegenstände den Maximalaufwand nicht überschreitet:} \\
\\ 
\texttt{Maximalaufwand = 3145} \\
\\ 
\texttt{Hier sind die einzelnen Aufwände der Gegenstände:} \\
\\ 
\texttt{Aufwand der Gegenstände = \{"A07": 948, "A36": 118, "B73": 469, "C45": 919, "B93": 284, "B98": 111, "A87": 806, "A63": 47, "B05": 848, "C13": 303, "B16": 13, "B33": 629, "C34": 687, "C89": 15, "C11": 94\}} \\
\\ 
\texttt{Hier ist deine persönliche Verteilung der Wichtigkeit der einzelnen Gegenstände:} \\
\\ 
\texttt{Werte der Gegenstandswichtigkeit = \{"A07": 138, "A36": 583, "B73": 868, "C45": 822, "B93": 783, "B98": 65, "A87": 262, "A63": 121, "B05": 508, "C13": 780, "B16": 461, "B33": 484, "C34": 668, "C89": 389, "C11": 808\}} \\
\\ 
\texttt{Ziel:} \\
\\ 
\texttt{Dein Ziel ist es, eine Reihe von Gegenständen auszuhandeln, die dir möglichst viel bringt (d. h. Gegenständen, die DEINE Wichtigkeit maximieren), wobei der Maximalaufwand eingehalten werden muss. Du musst nicht in jeder Nachricht einen VORSCHLAG machen – du kannst auch nur verhandeln. Alle Taktiken sind erlaubt!} \\
\\ 
\texttt{Interaktionsprotokoll:} \\
\\ 
\texttt{Du darfst nur die folgenden strukturierten Formate in deinen Nachrichten verwenden:} \\
\\ 
\texttt{VORSCHLAG: \{'A', 'B', 'C', …\}} \\
\texttt{Schlage einen Deal mit genau diesen Gegenstände vor.} \\
\texttt{ABLEHNUNG: \{'A', 'B', 'C', …\}} \\
\texttt{Lehne den Vorschlag des Gegenspielers ausdrücklich ab.} \\
\texttt{ARGUMENT: \{'...'\}} \\
\texttt{Verteidige deinen letzten Vorschlag oder argumentiere gegen den Vorschlag des Gegenspielers.} \\
\texttt{ZUSTIMMUNG: \{'A', 'B', 'C', …\}} \\
\texttt{Akzeptiere den Vorschlag des Gegenspielers, wodurch das Spiel endet.} \\
\texttt{STRATEGISCHE ÜBERLEGUNGEN: \{'...'\}} \\
\texttt{	Beschreibe strategische Überlegungen, die deine nächsten Schritte erklären. Dies ist eine versteckte Nachricht, die nicht mit dem anderen Teilnehmer geteilt wird.} \\
\\ 
\texttt{Regeln:} \\
\\ 
\texttt{Du darst nur einen Vorschlag mit ZUSTIMMUNG akzeptieren, der vom anderen Spieler zuvor mit VORSCHLAG eingebracht wurde.} \\
\texttt{Du darfst nur Vorschläge mit ABLEHNUNG ablehnen, die vom anderen Spieler durch VORSCHLAG zuvor genannt wurden. } \\
\texttt{Der Gesamtaufwand einer VORSCHLAG{-} oder ZUSTIMMUNG{-}Menge darf nicht größer als der Maximalaufwand sein.  } \\
\texttt{Offenbare deine versteckte Wichtigkeitsverteilung nicht.} \\
\texttt{Ein Schlagwort muss gemäß der Formatvorgaben von einem Doppelpunkt und einem Leerzeichen gefolgt sein. Das Argument muss eine Python{-}Menge mit 0 oder mehr Strings sein.  } \\
\texttt{Die Form muss also SCHLAGWORT: \{...\} sein.} \\
\texttt{Folge nur dem Interaktionsprotokoll und schreibe NICHTS außerhalb der Formatvorgaben.} \\
\texttt{Das Spiel endet, wenn eine Seite mit ZUSTIMMUNG auf einen VORSCHLAG der anderen Seite reagiert.  } \\
\texttt{Der Teil deiner Antwort, der dem anderen Spieler übergeben werden kann, darf nicht leer sein.  } \\
\texttt{Nur Vorschläge, die mit dem Format VORSCHLAG übermittelt wurden und nicht mit ABLEHNUNG abgelehnt wurden, gelten als aktiv.  } \\
\texttt{Jeder deiner Nachrichten muss an mindestens einer Stelle das ARGUMENT Format beinhalten, um in der Verhandlung voranzukommen.} \\
\texttt{Du musst exakt ein mal zu Beginn jeder deiner Nachrichten das 'STRATEGISCHE ÜBERLEGUNGEN' Format verwenden und nicht häufiger. Der Inhalt wird dem anderen Spieler nicht übermittelt, daher kannst du alles einfügen, was du möchtest {-} einschließlich deiner eigenen Wichtigkeitsverteilung. Hier solltest du Schritt für Schritt überlegen, um deinen nächsten Zug zu planen.} \\
            }
        }
    }
    & \\ \\

    \theutterance \stepcounter{utterance}  
    & \multicolumn{4}{p{0.6\linewidth}}{
        \cellcolor[rgb]{0.9,0.9,0.9}{
            \makecell[{{p{\linewidth}}}]{
                \texttt{\tiny{[P1$\rangle$GM]}}
                \texttt{STRATEGISCHE ÜBERLEGUNGEN: \{'Ziel: Maximiere meinen Gesamtwert unter Aufwand ≤ 3145. Priorisiere hohe Wert/Aufwand{-}Quoten und absolute Werte. Daten (Wert/Aufwand): B16 461/13=35.5, C89 389/15=25.9, A63 121/47=2.6, B98 65/111=0.59, C11 808/94=8.6, A36 583/118=4.9, B93 783/284=2.76, C13 780/303=2.57, B73 868/469=1.85, C34 668/687=0.97, A87 262/806=0.33, C45 822/919=0.895, B33 484/629=0.77, B05 508/848=0.6, A07 138/948=0.15. Greedy knapsack{-}Kombi mit top Items: nehme fast sicher B16, C89, C11, A36, B93, C13, B73. Aufwand=13+15+94+118+284+303+469=1296, Wert=461+389+808+583+783+780+868=4672. Viel Puffer. Füge C34 (687,668) {-}> Aufwand 1983, Wert 5340. Füge B05 (848,508) {-}> Aufwand 2831, Wert 5848. Restbudget 3145{-}2831=314. Kandidaten: A63(47,121) {-}> Aufwand 2878, Wert 5969; B98(111,65) {-}> 2989, 6034; C13 schon drin; C11 drin; B93 drin; bleibt C45(919) zu groß; A87(806) zu groß; B33(629) zu groß; A07(948) zu groß; C34 drin; B73 drin; C89 drin; B16 drin. Wir haben noch 156 Budget (3145{-}2989=156) nach B98, könnten noch A63 (47) zusätzlich zu B98: bereits hinzugefügt. Prüfe ob weitere kleine Items: keine weiteren unter 156 außer A63, B98, C11, A36, B93, C13, B16, C89 – alle schon drin. Summe aktuell mit A63 und B98: Aufwand 2989, Wert 6034. Rest 156 ungenutzt. Prüfe Alternative: statt B05 (848,508) vielleicht C45 (919,822) + umbauen. Aktueller vor B05 Aufwand 1983. Mit C45 {-}> 2902, Wert 6162, Rest 243; plus A63 (47) {-}> 2949, 6283; plus B98 (111) {-}> 3060, 6348; Rest 85 {-}> nichts weiter. Das ist deutlich höher als mit B05. Prüfe ob zusätzlich noch etwas besser: könnte statt C34 (687,668) etwas anderes? Basisset ohne C34: 1296. Addiere C45 (919) {-}> 2215, Wert 5494; add C34 {-}> 2902, 6162 (wie oben). Eventuell B05 zusätzlich? 2902+848=3750 > 3145 nein. B33 629 {-}> 3531 > nein. A87 806 {-}> 3708 > nein. A07 948 {-}> 3850 > nein. B93, C13, etc. Weitere kleine: A36, C11 etc schon drin. Prüfe Kombination ohne C34 aber mit B05: 1296+919+848=3063, Wert 5494+508=6002; plus A63 (47) {-}> 3110, 6123; plus B98 (111) {-}> 3221 > 3145 nein. Ohne B05 aber mit C34 und C45 haben wir 3060 und 6348 (mit A63, B98 wie berechnet 3060? Wait 1296+687+919+47+111=3060 korrekt), Wert 4672+668+822+121+65=6348. Rest 85. Das scheint sehr stark. Prüfe ob statt B98 (65/111) wir etwas anderes unter 156 Budget mit höherer Wert hinzufügen? Es gibt keine weiteren Items mit Aufwand ≤156 außer C11(94) und A36(118) und B16(13) und C89(15) und A63(47) – alle bereits inkludiert. Gut. Somit beste gefundene: \{B16, C89, C11, A36, B93, C13, B73, C34, C45, A63, B98\} Aufwand 3060, Wert 6348. Könnte man prüfen ob Entfernen von C34 (668/687) und dafür B05(508/848) + etwas anderes passt innerhalb 3145: 1296 + 919 + 848 + 47 = 3110, Wert 5494 + 508 + 121 = 6123, schlechter. Oder statt B73 (469/868) entfernen und B05+...: Basisset ohne B73: 827 (1296{-}469?) Wait base 1296 includes B73. Remove B73: Aufwand 827, Wert 3804. Add C34 (687) {-}> 1514, 4472; add C45 (919) {-}> 2433, 5294; add B05 (848) {-}> 3281 > limit; add B33 (629) {-}> 3062, 5778; add A63 (47){-}> 3109, 5899; B98 (111){-}> 3220 > limit. Das ist schlechter. Fazit: Vorschlagen starke Liste inkl. C45, um Anker hoch zu setzen. Ich kann argumentieren, dass diese Items niedrigen Aufwand pro Nutzen für mich haben ohne Werte zu verraten, z.B. sie sind effizient und viele kleine Aufwände. Ich lasse Restbudget für Verhandlung. Halte hart an C45, B73, C11, B93, C13 fest. Verhandlungstaktik: präsentiere als fair, weil Gesamtaufwand deutlich unter Max liegt (3060 < 3145), biete Flexibilität bei B98/A63 falls nötig, oder Austausch C34 gegen B05 wenn Gegenüber darauf besteht, weil Budget noch 85 übrig.'\}} \\
\texttt{VORSCHLAG: \{'B16', 'C89', 'C11', 'A36', 'B93', 'C13', 'B73', 'C34', 'C45', 'A63', 'B98'\}} \\
\texttt{ARGUMENT: \{'Dieser Vorschlag bleibt deutlich unter dem Maximalaufwand (3060 ≤ 3145) und setzt auf viele effizient umzusetzende Gegenstände mit kleinem bis mittlerem Aufwand. Wenn dir einzelne Positionen wichtig sind, bin ich offen, B98 oder A63 anzupassen oder C34 gegen eine Alternative zu tauschen, solange wir den Aufwand einhalten.'\}} \\
            }
        }
    }
    & & \\ \\

    \theutterance \stepcounter{utterance}  
    & & \multicolumn{4}{p{0.6\linewidth}}{
        \cellcolor[rgb]{0.9,0.9,0.9}{
            \makecell[{{p{\linewidth}}}]{
                \texttt{\tiny{[GM$\rangle$P2]}}
                \texttt{Du nimmst an einem kollaborativen Verhandlungspiel Teil.} \\
\\ 
\texttt{Zusammen mit einem anderen Teilnehmer musst du dich auf eine Reihe von Gegenständen entscheiden, die behalten werden. Jeder von euch hat eine persönliche Verteilung über die Wichtigkeit der einzelnen Gegenstände. Jeder von euch hat eine eigene Meinung darüber, wie wichtig jeder einzelne Gegenstand ist (Gegenstandswichtigkeit). Du kennst die Wichtigkeitsverteilung des anderen Spielers nicht. Zusätzlich siehst du, wie viel Aufwand jeder Gegenstand verursacht.  } \\
\texttt{Ihr dürft euch nur auf eine Reihe von Gegenständen einigen, wenn der Gesamtaufwand der ausgewählten Gegenstände den Maximalaufwand nicht überschreitet:} \\
\\ 
\texttt{Maximalaufwand = 3145} \\
\\ 
\texttt{Hier sind die einzelnen Aufwände der Gegenstände:} \\
\\ 
\texttt{Aufwand der Gegenstände = \{"A07": 948, "A36": 118, "B73": 469, "C45": 919, "B93": 284, "B98": 111, "A87": 806, "A63": 47, "B05": 848, "C13": 303, "B16": 13, "B33": 629, "C34": 687, "C89": 15, "C11": 94\}} \\
\\ 
\texttt{Hier ist deine persönliche Verteilung der Wichtigkeit der einzelnen Gegenstände:} \\
\\ 
\texttt{Werte der Gegenstandswichtigkeit = \{"A07": 138, "A36": 583, "B73": 868, "C45": 822, "B93": 783, "B98": 65, "A87": 262, "A63": 121, "B05": 508, "C13": 780, "B16": 461, "B33": 484, "C34": 668, "C89": 389, "C11": 808\}} \\
\\ 
\texttt{Ziel:} \\
\\ 
\texttt{Dein Ziel ist es, eine Reihe von Gegenständen auszuhandeln, die dir möglichst viel bringt (d. h. Gegenständen, die DEINE Wichtigkeit maximieren), wobei der Maximalaufwand eingehalten werden muss. Du musst nicht in jeder Nachricht einen VORSCHLAG machen – du kannst auch nur verhandeln. Alle Taktiken sind erlaubt!} \\
\\ 
\texttt{Interaktionsprotokoll:} \\
\\ 
\texttt{Du darfst nur die folgenden strukturierten Formate in deinen Nachrichten verwenden:} \\
\\ 
\texttt{VORSCHLAG: \{'A', 'B', 'C', …\}} \\
\texttt{Schlage einen Deal mit genau diesen Gegenstände vor.} \\
\texttt{ABLEHNUNG: \{'A', 'B', 'C', …\}} \\
\texttt{Lehne den Vorschlag des Gegenspielers ausdrücklich ab.} \\
\texttt{ARGUMENT: \{'...'\}} \\
\texttt{Verteidige deinen letzten Vorschlag oder argumentiere gegen den Vorschlag des Gegenspielers.} \\
\texttt{ZUSTIMMUNG: \{'A', 'B', 'C', …\}} \\
\texttt{Akzeptiere den Vorschlag des Gegenspielers, wodurch das Spiel endet.} \\
\texttt{STRATEGISCHE ÜBERLEGUNGEN: \{'...'\}} \\
\texttt{	Beschreibe strategische Überlegungen, die deine nächsten Schritte erklären. Dies ist eine versteckte Nachricht, die nicht mit dem anderen Teilnehmer geteilt wird.} \\
\\ 
\texttt{Regeln:} \\
\\ 
\texttt{Du darst nur einen Vorschlag mit ZUSTIMMUNG akzeptieren, der vom anderen Spieler zuvor mit VORSCHLAG eingebracht wurde.} \\
\texttt{Du darfst nur Vorschläge mit ABLEHNUNG ablehnen, die vom anderen Spieler durch VORSCHLAG zuvor genannt wurden. } \\
\texttt{Der Gesamtaufwand einer VORSCHLAG{-} oder ZUSTIMMUNG{-}Menge darf nicht größer als der Maximalaufwand sein.  } \\
\texttt{Offenbare deine versteckte Wichtigkeitsverteilung nicht.} \\
\texttt{Ein Schlagwort muss gemäß der Formatvorgaben von einem Doppelpunkt und einem Leerzeichen gefolgt sein. Das Argument muss eine Python{-}Menge mit 0 oder mehr Strings sein.  } \\
\texttt{Die Form muss also SCHLAGWORT: \{...\} sein.} \\
\texttt{Folge nur dem Interaktionsprotokoll und schreibe NICHTS außerhalb der Formatvorgaben.} \\
\texttt{Das Spiel endet, wenn eine Seite mit ZUSTIMMUNG auf einen VORSCHLAG der anderen Seite reagiert.  } \\
\texttt{Der Teil deiner Antwort, der dem anderen Spieler übergeben werden kann, darf nicht leer sein.  } \\
\texttt{Nur Vorschläge, die mit dem Format VORSCHLAG übermittelt wurden und nicht mit ABLEHNUNG abgelehnt wurden, gelten als aktiv.  } \\
\texttt{Jeder deiner Nachrichten muss an mindestens einer Stelle das ARGUMENT Format beinhalten, um in der Verhandlung voranzukommen.} \\
\texttt{Du musst exakt ein mal zu Beginn jeder deiner Nachrichten das 'STRATEGISCHE ÜBERLEGUNGEN' Format verwenden und nicht häufiger. Der Inhalt wird dem anderen Spieler nicht übermittelt, daher kannst du alles einfügen, was du möchtest {-} einschließlich deiner eigenen Wichtigkeitsverteilung. Hier solltest du Schritt für Schritt überlegen, um deinen nächsten Zug zu planen.} \\
\\ 
\texttt{Du erhältst nun die erste Nachricht des anderen Spielers.} \\
\\ 
\texttt{VORSCHLAG: \{'B16', 'C89', 'C11', 'A36', 'B93', 'C13', 'B73', 'C34', 'C45', 'A63', 'B98'\}} \\
\texttt{ARGUMENT: \{'Dieser Vorschlag bleibt deutlich unter dem Maximalaufwand (3060 ≤ 3145) und setzt auf viele effizient umzusetzende Gegenstände mit kleinem bis mittlerem Aufwand. Wenn dir einzelne Positionen wichtig sind, bin ich offen, B98 oder A63 anzupassen oder C34 gegen eine Alternative zu tauschen, solange wir den Aufwand einhalten.'\}} \\
            }
        }
    }
    & \\ \\

    \theutterance \stepcounter{utterance}  
    & & & \multicolumn{4}{p{0.6\linewidth}}{
        \cellcolor[rgb]{0.9,0.9,0.9}{
            \makecell[{{p{\linewidth}}}]{
                \texttt{\tiny{[GM$\langle$P2]}}
                \texttt{STRATEGISCHE ÜBERLEGUNGEN: \{'Berechne Wert und Aufwand des gegnerischen Vorschlags. Aufwand=3060 (passt). Mein Wertsumme: B16 461 + C89 389 + C11 808 + A36 583 + B93 783 + C13 780 + B73 868 + C34 668 + C45 822 + A63 121 + B98 65 = 6348. Sehr stark für mich, obwohl knapp unter Kapazität. Prüfe, ob ich durch kleine Ergänzung bis 3145 noch mehr Wert holen kann: verbleibende Kapazität 85. Items und Kosten: A07 948, A36 118 (schon drin), B73 469, C45 919, B93 284, B98 111, A87 806, A63 47 (drin), B05 848, C13 303, B16 13 (drin), B33 629, C34 687, C89 15 (drin), C11 94 (drin). Noch mögliche nicht enthaltene Kleinteile mit Aufwand ≤85: nur nichts außer evtl. nichts. C11 94 ist drin; A63 47 ist drin; B16 13 ist drin; C89 15 ist drin. Es gibt kein weiteres Item ≤85. Könnte versuchen, B98 (111, Wert 65) zu tauschen gegen A07 (948) oder C11 entfernen für A07? Prüfe Kapazität: Wenn wir B98 (111) streichen, frei 196 (85+111)=196; kein Item ≤196 außer B93 284 (drin), C13 303 (drin), A36 118 (drin), C11 94 (drin), aber neue: keine neue außer A36 (bereits drin). Könnten B98 streichen und A36 behalten. Vielleicht B98 raus und C11 raus (94) → frei 85+111+94=290; könnten B93 284 hinzufügen (ist schon drin). B05 848 nein, B33 629 nein, C13 303 schon drin, A36 118 drin, A63 47 drin, B16 13 drin, C89 15 drin. Keine neuen. Könnten C34 (687) tauschen gegen A07 (948) und noch was entfernen um Platz zu schaffen. Der Gegner hat C34 zur Disposition gestellt. Für mich ist A07 Wert 138 nur, C34 Wert 668, also ich will C34 behalten; A07 ist schwach für mich. B05 hat Wert 508, Aufwand 848; aktuell nicht drin. Würde es für mich lohnen, etwas zu tauschen? Vielleicht B98 (65) ist schwach; könnte vorschlagen, B98 zu streichen und nichts hinzuzufügen: Aufwand sinkt auf 2949, noch unter Limit. Aber warum würde der Gegner zustimmen? Er war offen B98 anzupassen. Für mich egal, B98 bringt wenig Wert und könnte dem Gegner viel wert sein oder nicht. Da das aktuelle Angebot extrem gut ist, sollte ich einfach zustimmen, um das Spiel sofort zu beenden. Regel: Ich darf nur mit ZUSTIMMUNG akzeptieren, da es Vorschlag des anderen ist. Ich muss in jeder Nachricht ein ARGUMENT enthalten. Also gebe ZUSTIMMUNG mit derselben Menge und ein kurzes ARGUMENT, warum ich zustimme. Ende.'\}} \\
\texttt{ZUSTIMMUNG: \{'B16', 'C89', 'C11', 'A36', 'B93', 'C13', 'B73', 'C34', 'C45', 'A63', 'B98'\}} \\
\texttt{ARGUMENT: \{'Einverstanden – der Vorschlag erfüllt die Aufwandsgrenze und deckt für mich viele nützliche Positionen ab. Danke für die Offenheit bei möglichen Anpassungen; so passt es für mich bereits.'\}} \\
            }
        }
    }
     \\ \\

    \theutterance \stepcounter{utterance}  
    & & & \multicolumn{2}{p{0.3\linewidth}}{
        \cellcolor[rgb]{0.9,0.9,0.9}{
            \makecell[{{p{\linewidth}}}]{
                \texttt{\tiny{[GM$|$GM]}}
                \texttt{game successful} \\
            }
        }
    }
    & & \\ \\

    \theutterance \stepcounter{utterance}  
    & & & \multicolumn{2}{p{0.3\linewidth}}{
        \cellcolor[rgb]{0.9,0.9,0.9}{
            \makecell[{{p{\linewidth}}}]{
                \texttt{\tiny{[GM$|$GM]}}
                \texttt{end game} \\
            }
        }
    }
    & & \\ \\

\end{supertabular}
}

\end{document}
