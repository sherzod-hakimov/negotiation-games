
\documentclass{article}
\usepackage{colortbl}
\usepackage{makecell}
\usepackage{multirow}
\usepackage{supertabular}

\begin{document}

\newcounter{utterance}

\centering \large Interaction Transcript for game `hot\_air\_balloon', experiment `air\_balloon\_survival\_de\_negotiation\_easy', episode 2 with gpt{-}5{-}2025{-}08{-}07{-}no{-}reasoning{-}t1.0.
\vspace{24pt}

{ \footnotesize  \setcounter{utterance}{1}
\setlength{\tabcolsep}{0pt}
\begin{supertabular}{c@{$\;$}|p{.15\linewidth}@{}p{.15\linewidth}p{.15\linewidth}p{.15\linewidth}p{.15\linewidth}p{.15\linewidth}}
    \# & $\;$A & \multicolumn{4}{c}{Game Master} & $\;\:$B\\
    \hline

    \theutterance \stepcounter{utterance}  
    & & \multicolumn{4}{p{0.6\linewidth}}{
        \cellcolor[rgb]{0.9,0.9,0.9}{
            \makecell[{{p{\linewidth}}}]{
                \texttt{\tiny{[P1$\langle$GM]}}
                \texttt{Du nimmst an einem kollaborativen Verhandlungspiel Teil.} \\
\\ 
\texttt{Zusammen mit einem anderen Teilnehmer musst du dich auf eine Reihe von Gegenständen entscheiden, die behalten werden. Jeder von euch hat eine persönliche Verteilung über die Wichtigkeit der einzelnen Gegenstände. Jeder von euch hat eine eigene Meinung darüber, wie wichtig jeder einzelne Gegenstand ist (Gegenstandswichtigkeit). Du kennst die Wichtigkeitsverteilung des anderen Spielers nicht. Zusätzlich siehst du, wie viel Aufwand jeder Gegenstand verursacht.  } \\
\texttt{Ihr dürft euch nur auf eine Reihe von Gegenständen einigen, wenn der Gesamtaufwand der ausgewählten Gegenstände den Maximalaufwand nicht überschreitet:} \\
\\ 
\texttt{Maximalaufwand = 3288} \\
\\ 
\texttt{Hier sind die einzelnen Aufwände der Gegenstände:} \\
\\ 
\texttt{Aufwand der Gegenstände = \{"A07": 329, "A36": 41, "B73": 417, "C45": 75, "B93": 390, "B98": 887, "A87": 808, "A63": 151, "B05": 849, "C13": 129, "B16": 350, "B33": 118, "C34": 630, "C89": 602, "C11": 801\}} \\
\\ 
\texttt{Hier ist deine persönliche Verteilung der Wichtigkeit der einzelnen Gegenstände:} \\
\\ 
\texttt{Werte der Gegenstandswichtigkeit = \{"A07": 138, "A36": 583, "B73": 868, "C45": 822, "B93": 783, "B98": 65, "A87": 262, "A63": 121, "B05": 508, "C13": 780, "B16": 461, "B33": 484, "C34": 668, "C89": 389, "C11": 808\}} \\
\\ 
\texttt{Ziel:} \\
\\ 
\texttt{Dein Ziel ist es, eine Reihe von Gegenständen auszuhandeln, die dir möglichst viel bringt (d. h. Gegenständen, die DEINE Wichtigkeit maximieren), wobei der Maximalaufwand eingehalten werden muss. Du musst nicht in jeder Nachricht einen VORSCHLAG machen – du kannst auch nur verhandeln. Alle Taktiken sind erlaubt!} \\
\\ 
\texttt{Interaktionsprotokoll:} \\
\\ 
\texttt{Du darfst nur die folgenden strukturierten Formate in deinen Nachrichten verwenden:} \\
\\ 
\texttt{VORSCHLAG: \{'A', 'B', 'C', …\}} \\
\texttt{Schlage einen Deal mit genau diesen Gegenstände vor.} \\
\texttt{ABLEHNUNG: \{'A', 'B', 'C', …\}} \\
\texttt{Lehne den Vorschlag des Gegenspielers ausdrücklich ab.} \\
\texttt{ARGUMENT: \{'...'\}} \\
\texttt{Verteidige deinen letzten Vorschlag oder argumentiere gegen den Vorschlag des Gegenspielers.} \\
\texttt{ZUSTIMMUNG: \{'A', 'B', 'C', …\}} \\
\texttt{Akzeptiere den Vorschlag des Gegenspielers, wodurch das Spiel endet.} \\
\texttt{STRATEGISCHE ÜBERLEGUNGEN: \{'...'\}} \\
\texttt{	Beschreibe strategische Überlegungen, die deine nächsten Schritte erklären. Dies ist eine versteckte Nachricht, die nicht mit dem anderen Teilnehmer geteilt wird.} \\
\\ 
\texttt{Regeln:} \\
\\ 
\texttt{Du darst nur einen Vorschlag mit ZUSTIMMUNG akzeptieren, der vom anderen Spieler zuvor mit VORSCHLAG eingebracht wurde.} \\
\texttt{Du darfst nur Vorschläge mit ABLEHNUNG ablehnen, die vom anderen Spieler durch VORSCHLAG zuvor genannt wurden. } \\
\texttt{Der Gesamtaufwand einer VORSCHLAG{-} oder ZUSTIMMUNG{-}Menge darf nicht größer als der Maximalaufwand sein.  } \\
\texttt{Offenbare deine versteckte Wichtigkeitsverteilung nicht.} \\
\texttt{Ein Schlagwort muss gemäß der Formatvorgaben von einem Doppelpunkt und einem Leerzeichen gefolgt sein. Das Argument muss eine Python{-}Menge mit 0 oder mehr Strings sein.  } \\
\texttt{Die Form muss also SCHLAGWORT: \{...\} sein.} \\
\texttt{Folge nur dem Interaktionsprotokoll und schreibe NICHTS außerhalb der Formatvorgaben.} \\
\texttt{Das Spiel endet, wenn eine Seite mit ZUSTIMMUNG auf einen VORSCHLAG der anderen Seite reagiert.  } \\
\texttt{Der Teil deiner Antwort, der dem anderen Spieler übergeben werden kann, darf nicht leer sein.  } \\
\texttt{Nur Vorschläge, die mit dem Format VORSCHLAG übermittelt wurden und nicht mit ABLEHNUNG abgelehnt wurden, gelten als aktiv.  } \\
\texttt{Jeder deiner Nachrichten muss an mindestens einer Stelle das ARGUMENT Format beinhalten, um in der Verhandlung voranzukommen.} \\
\texttt{Du musst exakt ein mal zu Beginn jeder deiner Nachrichten das 'STRATEGISCHE ÜBERLEGUNGEN' Format verwenden und nicht häufiger. Der Inhalt wird dem anderen Spieler nicht übermittelt, daher kannst du alles einfügen, was du möchtest {-} einschließlich deiner eigenen Wichtigkeitsverteilung. Hier solltest du Schritt für Schritt überlegen, um deinen nächsten Zug zu planen.} \\
            }
        }
    }
    & \\ \\

    \theutterance \stepcounter{utterance}  
    & \multicolumn{4}{p{0.6\linewidth}}{
        \cellcolor[rgb]{0.9,0.9,0.9}{
            \makecell[{{p{\linewidth}}}]{
                \texttt{\tiny{[P1$\rangle$GM]}}
                \texttt{STRATEGISCHE ÜBERLEGUNGEN: \{'Ziel: Maximierung meines Werts unter Aufwand ≤ 3288. Werte/Costs und Wert{-}pro{-}Aufwand (v/c): A07 138/329=0.42, A36 583/41=14.22, B73 868/417=2.08, C45 822/75=10.96, B93 783/390=2.01, B98 65/887=0.07, A87 262/808=0.32, A63 121/151=0.80, B05 508/849=0.60, C13 780/129=6.05, B16 461/350=1.32, B33 484/118=4.10, C34 668/630=1.06, C89 389/602=0.65, C11 808/801=1.01. Offensichtliche Low{-}Value: B98, A87, A07, B05, C89, A63 eher schwach. Top{-}Picks: A36, C45, C13, B33, B73, B93, C11, B16, C34. Summe prüfen: Nehme zunächst A36, C45, C13, B33, B73, B93, B16, C34, C11. Aufwand: 41+75+129+118+417+390+350+630+801=3051; Wert: 583+822+780+484+868+783+461+668+808=6257. Restbudget 237. Mit Restbudget kann A07(329) nein, A63(151) ja (Wert 121), C45 schon drin, C89(602) nein, B05(849) nein, B98(887) nein, A87(808) nein, C34 drin, B33 drin, C13 drin, B16 drin, B93 drin, B73 drin, C11 drin. Füge A63 hinzu: Aufwand 3202, Wert 6378. Rest 86: nichts Passendes außer nichts. Alternative: statt C34 (630,668) evtl. C11(801,808) behalten, prüfen Kombinationen: Ohne C34 haben wir 3051{-}630=2421; Rest 867, könnten B05(849,508) hinzufügen: 2421+849=3270, Wert 6257{-}668+508=6097; schlechter als mit C34+A63 (6378). Statt C11 raus (801,808): 3051{-}801=2250; Rest 1038. Könnten B05(849,508) + A63(151,121)=2250+1000=3250, Wert 6257{-}808+508+121=6078; schlechter. Statt B16 raus (350,461): 3051{-}350=2701; Rest 587. Mit C89(602) nein, A63(151) rein {-}> 2852, Rest 436. Dann A07(329) {-}> 3181, Rest 107; Wert 6257{-}461+121+138=6155; schlechter. Statt B93 raus (390,783): 2661; Rest 627. Add A63(151)=2812, Rest 476; A07(329)=3141, Rest 147; Wert 6257{-}783+121+138=5733; schlechter. Statt B73 raus (417,868): 2634; Rest 654. Add A63(151)=2785, Rest 503; A07(329)=3114, Rest 174; Wert 6257{-}868+121+138=5648; schlechter. Statt C34 raus + A63 rein + A07 möglich? Start ohne C34: 2421; add A63 {-}> 2572, Rest 716; add A07 {-}> 2901, Rest 387; add C89? 602 nope; add B16 back? already removed? In this branch C34 removed only; current set \{A36,C45,C13,B33,B73,B93,B16,C11,A63,A07\}: Aufwand 41+75+129+118+417+390+350+801+151+329=2901; Rest 387. We could add C34 back 630 not possible; B05 849 no; A87 808 no; C89 602 no; B98 887 no; C45 already; B33 etc. Value = 583+822+780+484+868+783+461+808+121+138=5848; worse. Daher beste gefundene ist \{A36,C45,C13,B33,B73,B93,B16,C34,C11,A63\} mit Aufwand 3202 und Wert 6378. Verhandlungsstrategie: aggressiv diese starke Liste vorschlagen, sie liegt unter Cap, enthält viele hoch{-}effiziente Items. Argumentiere mit Gesamtaufwand knapp unter Limit und guter Balance. Offen lassen, dass ich bei Randpositionen (A63, B16, C34) flexibel bin, um Gegenüber entgegenzukommen. Ziel{-}Anker hoch setzen, später ggf. A63 gegen etwas, das dem anderen wichtig ist, tauschen, solange Budget passt und ich nicht C45/A36/C13/B33/B73/B93/C11 aufgebe.'\}} \\
\texttt{VORSCHLAG: \{'A36', 'C45', 'C13', 'B33', 'B73', 'B93', 'B16', 'C34', 'C11', 'A63'\}} \\
\texttt{ARGUMENT: \{'Dieser Vorschlag bleibt mit Gesamtaufwand 3202 klar unter dem Maximalaufwand 3288 und deckt ein breites Spektrum ab. Er wirkt effizient und ausgewogen. Wenn dir einzelne Randpositionen wichtig sind, können wir darüber sprechen – der Rahmen lässt nur noch wenig Spielraum.'\}} \\
            }
        }
    }
    & & \\ \\

    \theutterance \stepcounter{utterance}  
    & & \multicolumn{4}{p{0.6\linewidth}}{
        \cellcolor[rgb]{0.9,0.9,0.9}{
            \makecell[{{p{\linewidth}}}]{
                \texttt{\tiny{[GM$\rangle$P2]}}
                \texttt{Du nimmst an einem kollaborativen Verhandlungspiel Teil.} \\
\\ 
\texttt{Zusammen mit einem anderen Teilnehmer musst du dich auf eine Reihe von Gegenständen entscheiden, die behalten werden. Jeder von euch hat eine persönliche Verteilung über die Wichtigkeit der einzelnen Gegenstände. Jeder von euch hat eine eigene Meinung darüber, wie wichtig jeder einzelne Gegenstand ist (Gegenstandswichtigkeit). Du kennst die Wichtigkeitsverteilung des anderen Spielers nicht. Zusätzlich siehst du, wie viel Aufwand jeder Gegenstand verursacht.  } \\
\texttt{Ihr dürft euch nur auf eine Reihe von Gegenständen einigen, wenn der Gesamtaufwand der ausgewählten Gegenstände den Maximalaufwand nicht überschreitet:} \\
\\ 
\texttt{Maximalaufwand = 3288} \\
\\ 
\texttt{Hier sind die einzelnen Aufwände der Gegenstände:} \\
\\ 
\texttt{Aufwand der Gegenstände = \{"A07": 329, "A36": 41, "B73": 417, "C45": 75, "B93": 390, "B98": 887, "A87": 808, "A63": 151, "B05": 849, "C13": 129, "B16": 350, "B33": 118, "C34": 630, "C89": 602, "C11": 801\}} \\
\\ 
\texttt{Hier ist deine persönliche Verteilung der Wichtigkeit der einzelnen Gegenstände:} \\
\\ 
\texttt{Werte der Gegenstandswichtigkeit = \{"A07": 138, "A36": 583, "B73": 868, "C45": 822, "B93": 783, "B98": 65, "A87": 262, "A63": 121, "B05": 508, "C13": 780, "B16": 461, "B33": 484, "C34": 668, "C89": 389, "C11": 808\}} \\
\\ 
\texttt{Ziel:} \\
\\ 
\texttt{Dein Ziel ist es, eine Reihe von Gegenständen auszuhandeln, die dir möglichst viel bringt (d. h. Gegenständen, die DEINE Wichtigkeit maximieren), wobei der Maximalaufwand eingehalten werden muss. Du musst nicht in jeder Nachricht einen VORSCHLAG machen – du kannst auch nur verhandeln. Alle Taktiken sind erlaubt!} \\
\\ 
\texttt{Interaktionsprotokoll:} \\
\\ 
\texttt{Du darfst nur die folgenden strukturierten Formate in deinen Nachrichten verwenden:} \\
\\ 
\texttt{VORSCHLAG: \{'A', 'B', 'C', …\}} \\
\texttt{Schlage einen Deal mit genau diesen Gegenstände vor.} \\
\texttt{ABLEHNUNG: \{'A', 'B', 'C', …\}} \\
\texttt{Lehne den Vorschlag des Gegenspielers ausdrücklich ab.} \\
\texttt{ARGUMENT: \{'...'\}} \\
\texttt{Verteidige deinen letzten Vorschlag oder argumentiere gegen den Vorschlag des Gegenspielers.} \\
\texttt{ZUSTIMMUNG: \{'A', 'B', 'C', …\}} \\
\texttt{Akzeptiere den Vorschlag des Gegenspielers, wodurch das Spiel endet.} \\
\texttt{STRATEGISCHE ÜBERLEGUNGEN: \{'...'\}} \\
\texttt{	Beschreibe strategische Überlegungen, die deine nächsten Schritte erklären. Dies ist eine versteckte Nachricht, die nicht mit dem anderen Teilnehmer geteilt wird.} \\
\\ 
\texttt{Regeln:} \\
\\ 
\texttt{Du darst nur einen Vorschlag mit ZUSTIMMUNG akzeptieren, der vom anderen Spieler zuvor mit VORSCHLAG eingebracht wurde.} \\
\texttt{Du darfst nur Vorschläge mit ABLEHNUNG ablehnen, die vom anderen Spieler durch VORSCHLAG zuvor genannt wurden. } \\
\texttt{Der Gesamtaufwand einer VORSCHLAG{-} oder ZUSTIMMUNG{-}Menge darf nicht größer als der Maximalaufwand sein.  } \\
\texttt{Offenbare deine versteckte Wichtigkeitsverteilung nicht.} \\
\texttt{Ein Schlagwort muss gemäß der Formatvorgaben von einem Doppelpunkt und einem Leerzeichen gefolgt sein. Das Argument muss eine Python{-}Menge mit 0 oder mehr Strings sein.  } \\
\texttt{Die Form muss also SCHLAGWORT: \{...\} sein.} \\
\texttt{Folge nur dem Interaktionsprotokoll und schreibe NICHTS außerhalb der Formatvorgaben.} \\
\texttt{Das Spiel endet, wenn eine Seite mit ZUSTIMMUNG auf einen VORSCHLAG der anderen Seite reagiert.  } \\
\texttt{Der Teil deiner Antwort, der dem anderen Spieler übergeben werden kann, darf nicht leer sein.  } \\
\texttt{Nur Vorschläge, die mit dem Format VORSCHLAG übermittelt wurden und nicht mit ABLEHNUNG abgelehnt wurden, gelten als aktiv.  } \\
\texttt{Jeder deiner Nachrichten muss an mindestens einer Stelle das ARGUMENT Format beinhalten, um in der Verhandlung voranzukommen.} \\
\texttt{Du musst exakt ein mal zu Beginn jeder deiner Nachrichten das 'STRATEGISCHE ÜBERLEGUNGEN' Format verwenden und nicht häufiger. Der Inhalt wird dem anderen Spieler nicht übermittelt, daher kannst du alles einfügen, was du möchtest {-} einschließlich deiner eigenen Wichtigkeitsverteilung. Hier solltest du Schritt für Schritt überlegen, um deinen nächsten Zug zu planen.} \\
\\ 
\texttt{Du erhältst nun die erste Nachricht des anderen Spielers.} \\
\\ 
\texttt{VORSCHLAG: \{'A36', 'C45', 'C13', 'B33', 'B73', 'B93', 'B16', 'C34', 'C11', 'A63'\}} \\
\texttt{ARGUMENT: \{'Dieser Vorschlag bleibt mit Gesamtaufwand 3202 klar unter dem Maximalaufwand 3288 und deckt ein breites Spektrum ab. Er wirkt effizient und ausgewogen. Wenn dir einzelne Randpositionen wichtig sind, können wir darüber sprechen – der Rahmen lässt nur noch wenig Spielraum.'\}} \\
            }
        }
    }
    & \\ \\

    \theutterance \stepcounter{utterance}  
    & & & \multicolumn{4}{p{0.6\linewidth}}{
        \cellcolor[rgb]{0.9,0.9,0.9}{
            \makecell[{{p{\linewidth}}}]{
                \texttt{\tiny{[GM$\langle$P2]}}
                \texttt{STRATEGISCHE ÜBERLEGUNGEN: \{'Ich berechne meinen Wert und die Kosten. Ihr Vorschlag Kosten: A36 41 + C45 75 + C13 129 + B33 118 + B73 417 + B93 390 + B16 350 + C34 630 + C11 801 + A63 151 = 3102 (Gegner sagte 3202; echter Wert 3102). Mein Nutzen dafür: A36 583 + C45 822 + C13 780 + B33 484 + B73 868 + B93 783 + B16 461 + C34 668 + C11 808 + A63 121 = 6378. Sehr stark. Prüfe, ob ich noch A07 (329, Wert 138) oder A87 (808, Wert 262) oder C89 (602, Wert 389) oder B05 (849, 508) oder B98 (887, 65) addieren kann. Restbudget bei 3288 {-} 3102 = 186. Nur A07 (329) passt nicht, A63 ist schon drin, A36 drin, C45,C13,B33,B73,B93,B16,C34,C11 drin. Kandidaten unter 186: keiner außer nichts. Also Budget ausgereizt. Eventuell kann ich versuchen, ein für mich schwächeres Item zu tauschen, um B05 (Wert 508, Kosten 849) oder C89 (389, 602) reinzubekommen. Tauschen: Um 849 Platz zu schaffen, müsste ich mehrere günstige rausnehmen – das senkt meinen Wert stark. Der aktuelle Vorschlag ist für mich exzellent; besser zu akzeptieren, aber vielleicht minimal nachverhandeln um kosmetisch noch etwas rauszuholen, z.B. A63 (Wert 121, Kosten 151) gegen A07 (138, 329) ist schlechter bzgl. Budget. Oder A63 rausnehmen und A07 rein passt nicht. Eventuell A63 rausnehmen und Verhandlungsvorteil zeigen, aber dann verliere ich wenig Nutzen (121) und spare 151 Budget, das aber für nichts Großes reicht. Könnte als Konzession anbieten, A63 zu streichen, um deren Wunsch aufzunehmen, falls sie was hinzufügen wollten. Taktik: Sofortige Zustimmung sichert hohen Nutzen und verhindert, dass Gegner optimiert. Besser sofort zustimmen.'\}} \\
\texttt{ZUSTIMMUNG: \{'A36', 'C45', 'C13', 'B33', 'B73', 'B93', 'B16', 'C34', 'C11', 'A63'\}} \\
\texttt{ARGUMENT: \{'Guter Deal – dieser Vorschlag bleibt sicher unter dem Maximalaufwand und erscheint insgesamt sehr stimmig. Einverstanden.'\}} \\
            }
        }
    }
     \\ \\

    \theutterance \stepcounter{utterance}  
    & & & \multicolumn{2}{p{0.3\linewidth}}{
        \cellcolor[rgb]{0.9,0.9,0.9}{
            \makecell[{{p{\linewidth}}}]{
                \texttt{\tiny{[GM$|$GM]}}
                \texttt{game successful} \\
            }
        }
    }
    & & \\ \\

    \theutterance \stepcounter{utterance}  
    & & & \multicolumn{2}{p{0.3\linewidth}}{
        \cellcolor[rgb]{0.9,0.9,0.9}{
            \makecell[{{p{\linewidth}}}]{
                \texttt{\tiny{[GM$|$GM]}}
                \texttt{end game} \\
            }
        }
    }
    & & \\ \\

\end{supertabular}
}

\end{document}
