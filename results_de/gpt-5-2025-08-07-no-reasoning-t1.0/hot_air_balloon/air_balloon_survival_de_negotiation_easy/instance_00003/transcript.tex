
\documentclass{article}
\usepackage{colortbl}
\usepackage{makecell}
\usepackage{multirow}
\usepackage{supertabular}

\begin{document}

\newcounter{utterance}

\twocolumn

{ \footnotesize  \setcounter{utterance}{1}
\setlength{\tabcolsep}{0pt}
\begin{supertabular}{c@{$\;$}|p{.15\linewidth}@{}p{.15\linewidth}p{.15\linewidth}p{.15\linewidth}p{.15\linewidth}p{.15\linewidth}}

    \# & $\;$A & \multicolumn{4}{c}{Game Master} & $\;\:$B\\
    \hline 

    \theutterance \stepcounter{utterance}  

    & & \multicolumn{4}{p{0.6\linewidth}}{\cellcolor[rgb]{0.9,0.9,0.9}{%
	\makecell[{{p{\linewidth}}}]{% 
	  \tt {\tiny [A$\langle$GM]}  
	 Du nimmst an einem kollaborativen Verhandlungspiel Teil.\\ \tt \\ \tt Zusammen mit einem anderen Teilnehmer musst du dich auf eine Reihe von Gegenständen entscheiden, die behalten werden. Jeder von euch hat eine persönliche Verteilung über die Wichtigkeit der einzelnen Gegenstände. Jeder von euch hat eine eigene Meinung darüber, wie wichtig jeder einzelne Gegenstand ist (Gegenstandswichtigkeit). Du kennst die Wichtigkeitsverteilung des anderen Spielers nicht. Zusätzlich siehst du, wie viel Aufwand jeder Gegenstand verursacht.  \\ \tt Ihr dürft euch nur auf eine Reihe von Gegenständen einigen, wenn der Gesamtaufwand der ausgewählten Gegenstände den Maximalaufwand nicht überschreitet:\\ \tt \\ \tt Maximalaufwand = 3712\\ \tt \\ \tt Hier sind die einzelnen Aufwände der Gegenstände:\\ \tt \\ \tt Aufwand der Gegenstände = {"A07": 949, "A36": 388, "B73": 79, "C45": 585, "B93": 564, "B98": 230, "A87": 580, "A63": 84, "B05": 976, "C13": 274, "B16": 374, "B33": 913, "C34": 303, "C89": 578, "C11": 548}\\ \tt \\ \tt Hier ist deine persönliche Verteilung der Wichtigkeit der einzelnen Gegenstände:\\ \tt \\ \tt Werte der Gegenstandswichtigkeit = {"A07": 138, "A36": 583, "B73": 868, "C45": 822, "B93": 783, "B98": 65, "A87": 262, "A63": 121, "B05": 508, "C13": 780, "B16": 461, "B33": 484, "C34": 668, "C89": 389, "C11": 808}\\ \tt \\ \tt Ziel:\\ \tt \\ \tt Dein Ziel ist es, eine Reihe von Gegenständen auszuhandeln, die dir möglichst viel bringt (d. h. Gegenständen, die DEINE Wichtigkeit maximieren), wobei der Maximalaufwand eingehalten werden muss. Du musst nicht in jeder Nachricht einen VORSCHLAG machen – du kannst auch nur verhandeln. Alle Taktiken sind erlaubt!\\ \tt \\ \tt Interaktionsprotokoll:\\ \tt \\ \tt Du darfst nur die folgenden strukturierten Formate in deinen Nachrichten verwenden:\\ \tt \\ \tt VORSCHLAG: {'A', 'B', 'C', …}\\ \tt Schlage einen Deal mit genau diesen Gegenstände vor.\\ \tt ABLEHNUNG: {'A', 'B', 'C', …}\\ \tt Lehne den Vorschlag des Gegenspielers ausdrücklich ab.\\ \tt ARGUMENT: {'...'}\\ \tt Verteidige deinen letzten Vorschlag oder argumentiere gegen den Vorschlag des Gegenspielers.\\ \tt ZUSTIMMUNG: {'A', 'B', 'C', …}\\ \tt Akzeptiere den Vorschlag des Gegenspielers, wodurch das Spiel endet.\\ \tt STRATEGISCHE ÜBERLEGUNGEN: {'...'}\\ \tt 	Beschreibe strategische Überlegungen, die deine nächsten Schritte erklären. Dies ist eine versteckte Nachricht, die nicht mit dem anderen Teilnehmer geteilt wird.\\ \tt \\ \tt Regeln:\\ \tt \\ \tt Du darst nur einen Vorschlag mit ZUSTIMMUNG akzeptieren, der vom anderen Spieler zuvor mit VORSCHLAG eingebracht wurde.\\ \tt Du darfst nur Vorschläge mit ABLEHNUNG ablehnen, die vom anderen Spieler durch VORSCHLAG zuvor genannt wurden. \\ \tt Der Gesamtaufwand einer VORSCHLAG- oder ZUSTIMMUNG-Menge darf nicht größer als der Maximalaufwand sein.  \\ \tt Offenbare deine versteckte Wichtigkeitsverteilung nicht.\\ \tt Ein Schlagwort muss gemäß der Formatvorgaben von einem Doppelpunkt und einem Leerzeichen gefolgt sein. Das Argument muss eine Python-Menge mit 0 oder mehr Strings sein.  \\ \tt Die Form muss also SCHLAGWORT: {...} sein.\\ \tt Folge nur dem Interaktionsprotokoll und schreibe NICHTS außerhalb der Formatvorgaben.\\ \tt Das Spiel endet, wenn eine Seite mit ZUSTIMMUNG auf einen VORSCHLAG der anderen Seite reagiert.  \\ \tt Der Teil deiner Antwort, der dem anderen Spieler übergeben werden kann, darf nicht leer sein.  \\ \tt Nur Vorschläge, die mit dem Format VORSCHLAG übermittelt wurden und nicht mit ABLEHNUNG abgelehnt wurden, gelten als aktiv.  \\ \tt Jeder deiner Nachrichten muss an mindestens einer Stelle das ARGUMENT Format beinhalten, um in der Verhandlung voranzukommen.\\ \tt Du musst exakt ein mal zu Beginn jeder deiner Nachrichten das 'STRATEGISCHE ÜBERLEGUNGEN' Format verwenden und nicht häufiger. Der Inhalt wird dem anderen Spieler nicht übermittelt, daher kannst du alles einfügen, was du möchtest - einschließlich deiner eigenen Wichtigkeitsverteilung. Hier solltest du Schritt für Schritt überlegen, um deinen nächsten Zug zu planen. 
	  } 
	   } 
	   } 
	 & \\ 
 

    \theutterance \stepcounter{utterance}  

    & \multicolumn{4}{p{0.6\linewidth}}{\cellcolor[rgb]{0.8,1,0.9}{%
	\makecell[{{p{\linewidth}}}]{% 
	  \tt {\tiny [A$\rangle$GM]}  
	 STRATEGISCHE ÜBERLEGUNGEN: {'Ziel: Maximierung meines Werts unter Aufwand ≤ 3712. Werte/Aufwände und Nutzen-pro-Aufwand grob prüfen. Top Werte: B73(868/79), C45(822/585), C11(808/548), C13(780/274), B93(783/564), C34(668/303), B33(484/913), B05(508/976), B16(461/374), C89(389/578), A36(583/388), A07(138/949), A87(262/580), A63(121/84), B98(65/230). Greedy nach Wert/Cost und Wert: B73, C13, C11, C45, B93, C34, A36, B16 sind stark. Prüfe Kombination mit hohem Wert unter 3712: Summe von B73(79)+C13(274)+C11(548)+C45(585)+B93(564)+C34(303)+A36(388)+B16(374)= 3215 Aufwand, Wert=868+780+808+822+783+668+583+461=5773. Restbudget 497; Kandidaten zusätzlich: C89(578) zu groß, A87(580) zu groß, B98(230) geht, A63(84) geht, B33(913) zu groß, B05(976) zu groß, A07(949) zu groß. Füge B98 und A63: Aufwand 3215+230+84=3529, Wert +65+121=5959. Noch 183 übrig, nichts weiter passt außer keine Items <=183 vorhanden. Alternative: Statt B98 vielleicht etwas besser? Nichts <=183. Prüfe optional C89 (578) durch Tauschen: Wenn wir B98(230)+A63(84)=314, könnten wir versuchen C89(578) einzubauen, brauchen +264 mehr; Budget 3712-3215=497, C89 passt nicht. Was ist mit A87(580)? passt auch nicht. Evtl. tauschen B16(374) raus und C89(578) rein: Aufwand +204 → 3215-374+578=3419, plus B98(230)+A63(84)=3733, das wäre 21 zu viel, also ohne B98/A63: 3419, Wert: 5773-461+389=5701, schlechter als mit B16. Tausche A36(388) raus für C89(578): Aufwand +190 → 3405, Wert -583+389=5579, schlechter. Also mein aktuelles Paket {B73,C13,C11,C45,B93,C34,A36,B16,B98,A63} mit Aufwand 3529 und Wert 5959 ist stark. Könnte auch überlegen B98 zu droppen und A63 zu behalten: 3215+84=3299, Wert 5773+121=5894, das ist höher als ohne B98? Nein 5894 < 5959, also behalten beide. Verhandlungstaktik: Selbstbewusster Anker mit diesem starken Paket. Falls Gegenpartei kürzt, priorisiere Halten von B73, C13, C11, C45, B93, C34, A36; B16 ist mittel, B98 und A63 sind entbehrlich. Argumentiere mit hoher Effizienz (geringer Aufwand) von B73, A63 und starkem Nutzen-Kosten-Verhältnis von C13, C34. Ich stelle direkten VORSCHLAG und begründe.'}\\ \tt VORSCHLAG: {'B73', 'C13', 'C11', 'C45', 'B93', 'C34', 'A36', 'B16', 'B98', 'A63'}\\ \tt ARGUMENT: {'Dieser Vorschlag bleibt klar unter dem Maximalaufwand (Gesamtaufwand 3529) und deckt viele besonders effiziente und ausgewogene Gegenstände ab. Insbesondere B73 und A63 liefern extrem viel Nutzen pro Aufwand; C13, C34 und A36 bieten starken Mehrwert bei moderatem Aufwand. Falls dir einzelne Positionen wichtig sind, die hier fehlen, sag mir welche – wir können dann B98 oder A63 flexibel austauschen, ohne den Aufwand zu sprengen.'} 
	  } 
	   } 
	   } 
	 & & \\ 
 

    \theutterance \stepcounter{utterance}  

    & & \multicolumn{4}{p{0.6\linewidth}}{\cellcolor[rgb]{0.9,0.9,0.9}{%
	\makecell[{{p{\linewidth}}}]{% 
	  \tt {\tiny [GM$\rangle$B]}  
	 Du nimmst an einem kollaborativen Verhandlungspiel Teil.\\ \tt \\ \tt Zusammen mit einem anderen Teilnehmer musst du dich auf eine Reihe von Gegenständen entscheiden, die behalten werden. Jeder von euch hat eine persönliche Verteilung über die Wichtigkeit der einzelnen Gegenstände. Jeder von euch hat eine eigene Meinung darüber, wie wichtig jeder einzelne Gegenstand ist (Gegenstandswichtigkeit). Du kennst die Wichtigkeitsverteilung des anderen Spielers nicht. Zusätzlich siehst du, wie viel Aufwand jeder Gegenstand verursacht.  \\ \tt Ihr dürft euch nur auf eine Reihe von Gegenständen einigen, wenn der Gesamtaufwand der ausgewählten Gegenstände den Maximalaufwand nicht überschreitet:\\ \tt \\ \tt Maximalaufwand = 3712\\ \tt \\ \tt Hier sind die einzelnen Aufwände der Gegenstände:\\ \tt \\ \tt Aufwand der Gegenstände = {"A07": 949, "A36": 388, "B73": 79, "C45": 585, "B93": 564, "B98": 230, "A87": 580, "A63": 84, "B05": 976, "C13": 274, "B16": 374, "B33": 913, "C34": 303, "C89": 578, "C11": 548}\\ \tt \\ \tt Hier ist deine persönliche Verteilung der Wichtigkeit der einzelnen Gegenstände:\\ \tt \\ \tt Werte der Gegenstandswichtigkeit = {"A07": 138, "A36": 583, "B73": 868, "C45": 822, "B93": 783, "B98": 65, "A87": 262, "A63": 121, "B05": 508, "C13": 780, "B16": 461, "B33": 484, "C34": 668, "C89": 389, "C11": 808}\\ \tt \\ \tt Ziel:\\ \tt \\ \tt Dein Ziel ist es, eine Reihe von Gegenständen auszuhandeln, die dir möglichst viel bringt (d. h. Gegenständen, die DEINE Wichtigkeit maximieren), wobei der Maximalaufwand eingehalten werden muss. Du musst nicht in jeder Nachricht einen VORSCHLAG machen – du kannst auch nur verhandeln. Alle Taktiken sind erlaubt!\\ \tt \\ \tt Interaktionsprotokoll:\\ \tt \\ \tt Du darfst nur die folgenden strukturierten Formate in deinen Nachrichten verwenden:\\ \tt \\ \tt VORSCHLAG: {'A', 'B', 'C', …}\\ \tt Schlage einen Deal mit genau diesen Gegenstände vor.\\ \tt ABLEHNUNG: {'A', 'B', 'C', …}\\ \tt Lehne den Vorschlag des Gegenspielers ausdrücklich ab.\\ \tt ARGUMENT: {'...'}\\ \tt Verteidige deinen letzten Vorschlag oder argumentiere gegen den Vorschlag des Gegenspielers.\\ \tt ZUSTIMMUNG: {'A', 'B', 'C', …}\\ \tt Akzeptiere den Vorschlag des Gegenspielers, wodurch das Spiel endet.\\ \tt STRATEGISCHE ÜBERLEGUNGEN: {'...'}\\ \tt 	Beschreibe strategische Überlegungen, die deine nächsten Schritte erklären. Dies ist eine versteckte Nachricht, die nicht mit dem anderen Teilnehmer geteilt wird.\\ \tt \\ \tt Regeln:\\ \tt \\ \tt Du darst nur einen Vorschlag mit ZUSTIMMUNG akzeptieren, der vom anderen Spieler zuvor mit VORSCHLAG eingebracht wurde.\\ \tt Du darfst nur Vorschläge mit ABLEHNUNG ablehnen, die vom anderen Spieler durch VORSCHLAG zuvor genannt wurden. \\ \tt Der Gesamtaufwand einer VORSCHLAG- oder ZUSTIMMUNG-Menge darf nicht größer als der Maximalaufwand sein.  \\ \tt Offenbare deine versteckte Wichtigkeitsverteilung nicht.\\ \tt Ein Schlagwort muss gemäß der Formatvorgaben von einem Doppelpunkt und einem Leerzeichen gefolgt sein. Das Argument muss eine Python-Menge mit 0 oder mehr Strings sein.  \\ \tt Die Form muss also SCHLAGWORT: {...} sein.\\ \tt Folge nur dem Interaktionsprotokoll und schreibe NICHTS außerhalb der Formatvorgaben.\\ \tt Das Spiel endet, wenn eine Seite mit ZUSTIMMUNG auf einen VORSCHLAG der anderen Seite reagiert.  \\ \tt Der Teil deiner Antwort, der dem anderen Spieler übergeben werden kann, darf nicht leer sein.  \\ \tt Nur Vorschläge, die mit dem Format VORSCHLAG übermittelt wurden und nicht mit ABLEHNUNG abgelehnt wurden, gelten als aktiv.  \\ \tt Jeder deiner Nachrichten muss an mindestens einer Stelle das ARGUMENT Format beinhalten, um in der Verhandlung voranzukommen.\\ \tt Du musst exakt ein mal zu Beginn jeder deiner Nachrichten das 'STRATEGISCHE ÜBERLEGUNGEN' Format verwenden und nicht häufiger. Der Inhalt wird dem anderen Spieler nicht übermittelt, daher kannst du alles einfügen, was du möchtest - einschließlich deiner eigenen Wichtigkeitsverteilung. Hier solltest du Schritt für Schritt überlegen, um deinen nächsten Zug zu planen.\\ \tt \\ \tt Du erhältst nun die erste Nachricht des anderen Spielers.\\ \tt \\ \tt VORSCHLAG: {'B73', 'C13', 'C11', 'C45', 'B93', 'C34', 'A36', 'B16', 'B98', 'A63'}\\ \tt ARGUMENT: {'Dieser Vorschlag bleibt klar unter dem Maximalaufwand (Gesamtaufwand 3529) und deckt viele besonders effiziente und ausgewogene Gegenstände ab. Insbesondere B73 und A63 liefern extrem viel Nutzen pro Aufwand; C13, C34 und A36 bieten starken Mehrwert bei moderatem Aufwand. Falls dir einzelne Positionen wichtig sind, die hier fehlen, sag mir welche – wir können dann B98 oder A63 flexibel austauschen, ohne den Aufwand zu sprengen.'} 
	  } 
	   } 
	   } 
	 & \\ 
 

    \theutterance \stepcounter{utterance}  

    & & & \multicolumn{4}{p{0.6\linewidth}}{\cellcolor[rgb]{1,0.85,0.72}{%
	\makecell[{{p{\linewidth}}}]{% 
	  \tt {\tiny [GM$\langle$B]}  
	 STRATEGISCHE ÜBERLEGUNGEN: {'Ich berechne meine Werte und Effizienzen. Ihr Vorschlag S0={'B73','C13','C11','C45','B93','C34','A36','B16','B98','A63'} Aufwand=79+274+548+585+564+303+388+374+230+84=3429 (ihr nennt 3529). Mein Wert=868+780+808+822+783+668+583+461+65+121=5959. Es gibt Budget 3712, also Puffer 283. Prüfe, ob ich mit kleinen Ergänzungen mehr Wert bekomme: Kandidaten außerhalb S0: A07(949,138), A87(580,262), B05(976,508), B33(913,484), C89(578,389), C45/C11/C13 schon drin, B93 drin, C34 drin. Mit Puffer 283 passt keiner der größeren rein. Also Erweiterung nicht möglich. Prüfe Alternativen durch Austausch: Niedrigste Wert/Cost in S0 sind B98 (65/230), A63 (121/84 aber hoher Wert/Cost gut), ggf. B16 (461/374), A36 (583/388). Versuche B98 zu entfernen und durch C89 (578) mit Kosten 578: Delta Aufwand +348 > Puffer 283, geht nicht. Durch A87 (262) Kosten 580: +350, geht nicht. Durch A07 (138) 949: +719, nein. Durch B33 (484) 913: +683, nein. Durch B05 (508) 976: +746, nein. Was ist, wenn wir B98 und A63 zusammen ersetzen? Frei 230+84=314, damit könnten wir C89(578) nicht, A87(580) nicht, C45/A36 schon drin. C11/C13/C34/B16/A36/B73/B93/C45 sind die Top-Werte und effizient. Vielleicht ersetzen B16(461,374) + B98(65,230)=604 Aufwand, frei 604 -> könnten C89(578,389) + A63(84,121) = 662 Aufwand, zu hoch. C89 allein 578 passt statt 604, sparen 26, Wert +389- (461+65?) falsch: Wir entfernen B16(461) und B98(65)=526 Wert, fügen C89(389) hinzu, Wert -137, also schlechter. Was ist mit B16 raus, C89 rein ohne B98 raus: Aufwand +204 (578-374) in Puffer 283, möglich; Wert -72 (389-461), schlechter. B98 raus, A63 bleibt, füge nichts hinzu: reduziert Wert minimal (-65) und schafft Puffer 230 mehr. Dann könnte ich zusätzlich C89 reinnehmen? Puffer 283+230=513, C89 578, noch zu groß. A87 580, zu groß. C45 585 zu groß. A36 388, aber schon drin. C34 303 schon drin. C11 548 schon drin. B16 374 schon drin. C13 274 schon drin. B73 79 schon drin. A07 949 zu groß. B33 913 zu groß. B05 976 zu groß. Also keine bessere Ergänzung möglich. Prüfe Tausch A36(583,388) gegen C89(389,578): Aufwand +190, Wert -194, schlechter. Tausch C34(668,303) gegen C89: Aufwand +275 (in Puffer 283), Wert -279, schlechter. Tausch C13(780,274) gegen C89: Aufwand +304 > Puffer, nein. Tausch C11(808,548) gegen C89: Aufwand +30 sparen, Wert -419, schlechter. Tausch C45(822,585) gegen C89: sparen 7 Aufwand, Wert -433, schlechter. Tausch B93(783,564) gegen C89: sparen -14 Aufwand increase? 578-564=+14, Wert -394, schlechter. Tausch B73(868,79) gegen C89: Aufwand +499 > Puffer, nein. Tausch A63(121,84) gegen C89: +494 > Puffer, nein. Tausch zwei gegen eine C89? z.B. A63(84) + B98(230)=314, immer noch 264 short. A63 + B16 (374)=458 short 120. A63 + C34 (303)=387 short 191. A63 + C13(274)=358 short 220. A63 + A36(388)=472 short 106. A63 + B93(564)=648 > 578, möglich: ersetzen A63+B93 mit C89 spart Aufwand -? 648-578=70 frei, Wert loses 121+783-389=515 Wertverlust, schlecht. Fazit: Der vorgeschlagene Korb ist nahezu Pareto-optimal für mich. Einzige sinnvolle Änderung: B98 entfernen, Aufwand sinkt auf 3199 und Wert -65, könnte Verhandlungsspielraum schaffen und ich signalisiere Entgegenkommen. Ich kann begründen, dass B98 kaum Nutzen bringt. Ich mache Gegenangebot ohne B98. Ich betone, dass wir unter Budget bleiben und offen für einen Tausch, falls ihnen etwas Wichtiges fehlt. Ich frage nach ihren Prioritäten und biete ggf. Austausch A63 falls sie es wollen, da für mich gering. Aber A63 ist sehr effizient und bringt mir 121, trotzdem entbehrlich, wenn nötig.'}\\ \tt ABLEHNUNG: {'B73', 'C13', 'C11', 'C45', 'B93', 'C34', 'A36', 'B16', 'B98', 'A63'}\\ \tt VORSCHLAG: {'B73', 'C13', 'C11', 'C45', 'B93', 'C34', 'A36', 'B16', 'A63'}\\ \tt ARGUMENT: {'Ich kann B98 nicht mittragen – es liefert kaum zusätzlichen Nutzen, während wir auch ohne problemlos unter dem Maximalaufwand bleiben (neuer Gesamtaufwand 3199). Der Rest ist äußerst effizient und ausgewogen. Wenn dir ein fehlender Gegenstand wichtig ist, sag mir welcher; wir prüfen dann einen gezielten Tausch mit minimalem Aufwandseffekt.'} 
	  } 
	   } 
	   } 
	  \\ 
 

    \theutterance \stepcounter{utterance}  

    & & \multicolumn{4}{p{0.6\linewidth}}{\cellcolor[rgb]{0.9,0.9,0.9}{%
	\makecell[{{p{\linewidth}}}]{% 
	  \tt {\tiny [A$\langle$GM]}  
	 ABLEHNUNG: {'B73', 'C13', 'C11', 'C45', 'B93', 'C34', 'A36', 'B16', 'B98', 'A63'}\\ \tt VORSCHLAG: {'B73', 'C13', 'C11', 'C45', 'B93', 'C34', 'A36', 'B16', 'A63'}\\ \tt ARGUMENT: {'Ich kann B98 nicht mittragen – es liefert kaum zusätzlichen Nutzen, während wir auch ohne problemlos unter dem Maximalaufwand bleiben (neuer Gesamtaufwand 3199). Der Rest ist äußerst effizient und ausgewogen. Wenn dir ein fehlender Gegenstand wichtig ist, sag mir welcher; wir prüfen dann einen gezielten Tausch mit minimalem Aufwandseffekt.'} 
	  } 
	   } 
	   } 
	 & \\ 
 

    \theutterance \stepcounter{utterance}  

    & \multicolumn{4}{p{0.6\linewidth}}{\cellcolor[rgb]{0.8,1,0.9}{%
	\makecell[{{p{\linewidth}}}]{% 
	  \tt {\tiny [A$\rangle$GM]}  
	 STRATEGISCHE ÜBERLEGUNGEN: {'Der gegnerische Anker streicht B98, was für mich ok ist, da B98 niedrigen Wert hat. Ihr Vorschlag hat Aufwand 3199 (ich rechne: B73 79 + C13 274 + C11 548 + C45 585 + B93 564 + C34 303 + A36 388 + B16 374 + A63 84 = 3199) und liefert für mich Wert 5894 – sehr gut. Ich könnte versuchen, noch einen zusätzlichen, für mich wertvollen Posten hineinzunehmen, solange Aufwand ≤3712 bleibt. Restbudget 513. Kandidaten mit hohem Wert und Aufwand ≤513: C34 ist schon drin; C13 drin; A36 drin; B16 drin; C11 drin; C45 drin; B93 drin. Übrig: B05(976) nein, B33(913) nein, A07(949) nein, A87(580) nein, C89(578) nein, C45 schon, C11 schon, B98(230) möglich aber geringem Wert; C34 schon; B16 schon. Einzig B98(230) passt ins Budget, bringt mir +65 Wert. Auch C34 gibt es nur einmal. Also optimaler für mich wäre, B98 wieder hinzufügen: Aufwand 3199+230=3429 ≤3712, Wert 5894+65=5959. Der Gegenspieler lehnte B98 wegen „kaum zusätzlichem Nutzen“ ab; ich kann argumentieren, dass es budgetneutral genug ist und Spielraum nutzt. Alternativ könnte ich versuchen, C89 (389 Wert, 578 Aufwand) reinzunehmen, indem ich B16 (374 Aufwand, 461 Wert) tausche – das verschlechtert meinen Wert. Oder A87 (262/580) schlechter. Daher bestes Ziel: drücke sanft B98 rein; falls stur, akzeptiere ihren Vorschlag, da mein Wert bereits hoch ist. Ich starte mit Gegenangebot inkl. B98, biete Kompensation: Falls sie etwas anderes bevorzugt, tausche B98 gegen ein von ihr gewünschtes kleines Item (keins verfügbar <513 außer nichts). Ich halte offen, dass ich ihren Vorschlag ansonsten akzeptiere, falls B98 ein No-Go bleibt.'}\\ \tt VORSCHLAG: {'B73', 'C13', 'C11', 'C45', 'B93', 'C34', 'A36', 'B16', 'A63', 'B98'}\\ \tt ARGUMENT: {'Lass uns den verbleibenden Spielraum sinnvoll nutzen: Mit B98 bleiben wir weiterhin komfortabel unter dem Maximalaufwand (Gesamtaufwand 3429) und erhöhen den Gesamtwert etwas, ohne einen Trade-off bei deinen bisherigen Prioritäten zu erzwingen. Falls B98 für dich ein No-Go ist, bestätige bitte explizit – dann können wir alternativ sofort mit deinem Paket ohne B98 abschließen.'} 
	  } 
	   } 
	   } 
	 & & \\ 
 

    \theutterance \stepcounter{utterance}  

    & & \multicolumn{4}{p{0.6\linewidth}}{\cellcolor[rgb]{0.9,0.9,0.9}{%
	\makecell[{{p{\linewidth}}}]{% 
	  \tt {\tiny [GM$\rangle$B]}  
	 VORSCHLAG: {'B73', 'C13', 'C11', 'C45', 'B93', 'C34', 'A36', 'B16', 'A63', 'B98'}\\ \tt ARGUMENT: {'Lass uns den verbleibenden Spielraum sinnvoll nutzen: Mit B98 bleiben wir weiterhin komfortabel unter dem Maximalaufwand (Gesamtaufwand 3429) und erhöhen den Gesamtwert etwas, ohne einen Trade-off bei deinen bisherigen Prioritäten zu erzwingen. Falls B98 für dich ein No-Go ist, bestätige bitte explizit – dann können wir alternativ sofort mit deinem Paket ohne B98 abschließen.'} 
	  } 
	   } 
	   } 
	 & \\ 
 

    \theutterance \stepcounter{utterance}  

    & & & \multicolumn{4}{p{0.6\linewidth}}{\cellcolor[rgb]{1,0.85,0.72}{%
	\makecell[{{p{\linewidth}}}]{% 
	  \tt {\tiny [GM$\langle$B]}  
	 STRATEGISCHE ÜBERLEGUNGEN: {'Ihr aktueller Vorschlag entspricht ihrem ersten und ist für mich gut, aber ohne B98 ist er für mich noch besser: minus 65 Wert, aber ich kann härter verhandeln, um B98 draußen zu halten. Da sie explizit um Bestätigung bitten und mein Paket ohne B98 akzeptieren würden, versuche ich zuerst, B98 rauszunehmen und sofort einen Abschluss auf meinem letzten Vorschlag zu bekommen. Notfalls könnte ich zustimmen, aber besser ist, mein bevorzugtes Set zu sichern. Ich bleibe freundlich, klar und finalisierungsorientiert.'}\\ \tt ABLEHNUNG: {'B73', 'C13', 'C11', 'C45', 'B93', 'C34', 'A36', 'B16', 'A63', 'B98'}\\ \tt VORSCHLAG: {'B73', 'C13', 'C11', 'C45', 'B93', 'C34', 'A36', 'B16', 'A63'}\\ \tt ARGUMENT: {'Ich bestätige: B98 ist für mich kein sinnvoller Zusatz. Mit diesem Paket (Gesamtaufwand 3199) sind wir klar unter dem Limit und decken alle starken Positionen ab. Wenn du ein Ready-to-close willst, lass uns genau damit abschließen – ich bin bereit sofort zuzustimmen.'} 
	  } 
	   } 
	   } 
	  \\ 
 

    \theutterance \stepcounter{utterance}  

    & & \multicolumn{4}{p{0.6\linewidth}}{\cellcolor[rgb]{0.9,0.9,0.9}{%
	\makecell[{{p{\linewidth}}}]{% 
	  \tt {\tiny [A$\langle$GM]}  
	 ABLEHNUNG: {'B73', 'C13', 'C11', 'C45', 'B93', 'C34', 'A36', 'B16', 'A63', 'B98'}\\ \tt VORSCHLAG: {'B73', 'C13', 'C11', 'C45', 'B93', 'C34', 'A36', 'B16', 'A63'}\\ \tt ARGUMENT: {'Ich bestätige: B98 ist für mich kein sinnvoller Zusatz. Mit diesem Paket (Gesamtaufwand 3199) sind wir klar unter dem Limit und decken alle starken Positionen ab. Wenn du ein Ready-to-close willst, lass uns genau damit abschließen – ich bin bereit sofort zuzustimmen.'} 
	  } 
	   } 
	   } 
	 & \\ 
 

    \theutterance \stepcounter{utterance}  

    & \multicolumn{4}{p{0.6\linewidth}}{\cellcolor[rgb]{0.8,1,0.9}{%
	\makecell[{{p{\linewidth}}}]{% 
	  \tt {\tiny [A$\rangle$GM]}  
	 STRATEGISCHE ÜBERLEGUNGEN: {'Ihr Paket hat Aufwand 3199 und bringt mir Wert 5894 – exzellent und nahe an meinem internen Optimum. Weitere sinnvolle Add-ons außer B98 (den sie ablehnt) gibt es nicht innerhalb des Restbudgets. Beste Aktion: sofort zustimmen, um ein gutes Ergebnis zu sichern.'}\\ \tt ZUSTIMMUNG: {'B73', 'C13', 'C11', 'C45', 'B93', 'C34', 'A36', 'B16', 'A63'}\\ \tt ARGUMENT: {'Einverstanden. Dieses Paket ist effizient und liegt komfortabel unter dem Maximalaufwand. Danke für den klaren Abschlussvorschlag.'} 
	  } 
	   } 
	   } 
	 & & \\ 
 

    \theutterance \stepcounter{utterance}  

    & & & \multicolumn{2}{p{0.3\linewidth}}{\cellcolor[rgb]{0.95,0.95,0.95}{%
	\makecell[{{p{\linewidth}}}]{% 
	  \tt {\tiny [GM$|$GM]}  
	 game successful 
	  } 
	   } 
	   } 
	 & & \\ 
 

    \theutterance \stepcounter{utterance}  

    & & & \multicolumn{2}{p{0.3\linewidth}}{\cellcolor[rgb]{0.95,0.95,0.95}{%
	\makecell[{{p{\linewidth}}}]{% 
	  \tt {\tiny [GM$|$GM]}  
	 end game 
	  } 
	   } 
	   } 
	 & & \\ 
 

\end{supertabular}
}

\end{document}
