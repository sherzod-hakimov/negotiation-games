
\documentclass{article}
\usepackage{colortbl}
\usepackage{makecell}
\usepackage{multirow}
\usepackage{supertabular}

\begin{document}

\newcounter{utterance}

\twocolumn

{ \footnotesize  \setcounter{utterance}{1}
\setlength{\tabcolsep}{0pt}
\begin{supertabular}{c@{$\;$}|p{.15\linewidth}@{}p{.15\linewidth}p{.15\linewidth}p{.15\linewidth}p{.15\linewidth}p{.15\linewidth}}

    \# & $\;$A & \multicolumn{4}{c}{Game Master} & $\;\:$B\\
    \hline 

    \theutterance \stepcounter{utterance}  

    & & \multicolumn{4}{p{0.6\linewidth}}{\cellcolor[rgb]{0.9,0.9,0.9}{%
	\makecell[{{p{\linewidth}}}]{% 
	  \tt {\tiny [A$\langle$GM]}  
	 Du nimmst an einem kollaborativen Verhandlungspiel Teil.\\ \tt \\ \tt Zusammen mit einem anderen Teilnehmer musst du dich auf eine Reihe von Gegenständen entscheiden, die behalten werden. Jeder von euch hat eine persönliche Verteilung über die Wichtigkeit der einzelnen Gegenstände. Jeder von euch hat eine eigene Meinung darüber, wie wichtig jeder einzelne Gegenstand ist (Gegenstandswichtigkeit). Du kennst die Wichtigkeitsverteilung des anderen Spielers nicht. Zusätzlich siehst du, wie viel Aufwand jeder Gegenstand verursacht.  \\ \tt Ihr dürft euch nur auf eine Reihe von Gegenständen einigen, wenn der Gesamtaufwand der ausgewählten Gegenstände den Maximalaufwand nicht überschreitet:\\ \tt \\ \tt Maximalaufwand = 3358\\ \tt \\ \tt Hier sind die einzelnen Aufwände der Gegenstände:\\ \tt \\ \tt Aufwand der Gegenstände = {"A07": 171, "A36": 933, "B73": 552, "C45": 219, "B93": 275, "B98": 778, "A87": 341, "A63": 615, "B05": 519, "C13": 862, "B16": 262, "B33": 377, "C34": 347, "C89": 349, "C11": 117}\\ \tt \\ \tt Hier ist deine persönliche Verteilung der Wichtigkeit der einzelnen Gegenstände:\\ \tt \\ \tt Werte der Gegenstandswichtigkeit = {"A07": 138, "A36": 583, "B73": 868, "C45": 822, "B93": 783, "B98": 65, "A87": 262, "A63": 121, "B05": 508, "C13": 780, "B16": 461, "B33": 484, "C34": 668, "C89": 389, "C11": 808}\\ \tt \\ \tt Ziel:\\ \tt \\ \tt Dein Ziel ist es, eine Reihe von Gegenständen auszuhandeln, die dir möglichst viel bringt (d. h. Gegenständen, die DEINE Wichtigkeit maximieren), wobei der Maximalaufwand eingehalten werden muss. Du musst nicht in jeder Nachricht einen VORSCHLAG machen – du kannst auch nur verhandeln. Alle Taktiken sind erlaubt!\\ \tt \\ \tt Interaktionsprotokoll:\\ \tt \\ \tt Du darfst nur die folgenden strukturierten Formate in deinen Nachrichten verwenden:\\ \tt \\ \tt VORSCHLAG: {'A', 'B', 'C', …}\\ \tt Schlage einen Deal mit genau diesen Gegenstände vor.\\ \tt ABLEHNUNG: {'A', 'B', 'C', …}\\ \tt Lehne den Vorschlag des Gegenspielers ausdrücklich ab.\\ \tt ARGUMENT: {'...'}\\ \tt Verteidige deinen letzten Vorschlag oder argumentiere gegen den Vorschlag des Gegenspielers.\\ \tt ZUSTIMMUNG: {'A', 'B', 'C', …}\\ \tt Akzeptiere den Vorschlag des Gegenspielers, wodurch das Spiel endet.\\ \tt STRATEGISCHE ÜBERLEGUNGEN: {'...'}\\ \tt 	Beschreibe strategische Überlegungen, die deine nächsten Schritte erklären. Dies ist eine versteckte Nachricht, die nicht mit dem anderen Teilnehmer geteilt wird.\\ \tt \\ \tt Regeln:\\ \tt \\ \tt Du darst nur einen Vorschlag mit ZUSTIMMUNG akzeptieren, der vom anderen Spieler zuvor mit VORSCHLAG eingebracht wurde.\\ \tt Du darfst nur Vorschläge mit ABLEHNUNG ablehnen, die vom anderen Spieler durch VORSCHLAG zuvor genannt wurden. \\ \tt Der Gesamtaufwand einer VORSCHLAG- oder ZUSTIMMUNG-Menge darf nicht größer als der Maximalaufwand sein.  \\ \tt Offenbare deine versteckte Wichtigkeitsverteilung nicht.\\ \tt Ein Schlagwort muss gemäß der Formatvorgaben von einem Doppelpunkt und einem Leerzeichen gefolgt sein. Das Argument muss eine Python-Menge mit 0 oder mehr Strings sein.  \\ \tt Die Form muss also SCHLAGWORT: {...} sein.\\ \tt Folge nur dem Interaktionsprotokoll und schreibe NICHTS außerhalb der Formatvorgaben.\\ \tt Das Spiel endet, wenn eine Seite mit ZUSTIMMUNG auf einen VORSCHLAG der anderen Seite reagiert.  \\ \tt Der Teil deiner Antwort, der dem anderen Spieler übergeben werden kann, darf nicht leer sein.  \\ \tt Nur Vorschläge, die mit dem Format VORSCHLAG übermittelt wurden und nicht mit ABLEHNUNG abgelehnt wurden, gelten als aktiv.  \\ \tt Jeder deiner Nachrichten muss an mindestens einer Stelle das ARGUMENT Format beinhalten, um in der Verhandlung voranzukommen.\\ \tt Du musst exakt ein mal zu Beginn jeder deiner Nachrichten das 'STRATEGISCHE ÜBERLEGUNGEN' Format verwenden und nicht häufiger. Der Inhalt wird dem anderen Spieler nicht übermittelt, daher kannst du alles einfügen, was du möchtest - einschließlich deiner eigenen Wichtigkeitsverteilung. Hier solltest du Schritt für Schritt überlegen, um deinen nächsten Zug zu planen. 
	  } 
	   } 
	   } 
	 & \\ 
 

    \theutterance \stepcounter{utterance}  

    & \multicolumn{4}{p{0.6\linewidth}}{\cellcolor[rgb]{0.8,1,0.9}{%
	\makecell[{{p{\linewidth}}}]{% 
	  \tt {\tiny [A$\rangle$GM]}  
	 STRATEGISCHE ÜBERLEGUNGEN: {'Kapazität 3358. Meine Werte/Costs; berechne Wert/Kosten-Quoten: B73 868/552=1.57, C45 822/219=3.76, B93 783/275=2.85, B98 65/778=0.08, A87 262/341=0.77, A63 121/615=0.20, B05 508/519=0.98, C13 780/862=0.90, B16 461/262=1.76, B33 484/377=1.28, C34 668/347=1.93, C89 389/349=1.11, C11 808/117=6.91, A07 138/171=0.81, A36 583/933=0.62. Top Items: C11, C45, B93, C34, B16, B73, B33, C89, B05. Build knapsack: Start sum cost 0. Add C11(117), C45(219)=336, B93(275)=611, C34(347)=958, B16(262)=1220, B73(552)=1772, B33(377)=2149, C89(349)=2498, B05(519)=3017, remaining 341; can add A87(341)=3358 exact fit. Total value: 808+822+783+668+461+868+484+389+508+262=6053. Perfect fill. This seems excellent. I will open with this maximal bundle. Argue that es nutzt Budget exakt aus und enthält viele effiziente Items. Leave room: could drop A87 or B05 if needed. Keep stance firm on C11, C45, B93 as must-haves. Offer rationale focusing on Aufwandseffizienz.'}\\ \tt VORSCHLAG: {'C11', 'C45', 'B93', 'C34', 'B16', 'B73', 'B33', 'C89', 'B05', 'A87'}\\ \tt ARGUMENT: {'Ich schlage dieses Paket vor, weil es den Maximalaufwand exakt ausschöpft und überwiegend Gegenstände mit sehr gutem Nutzen-Aufwand-Verhältnis enthält. Insbesondere C11, C45 und B93 sind extrem effizient. Wenn dir einzelne Positionen wichtig sind, können wir über Austausch sprechen, aber diese drei sollten unbedingt bleiben.'} 
	  } 
	   } 
	   } 
	 & & \\ 
 

    \theutterance \stepcounter{utterance}  

    & & \multicolumn{4}{p{0.6\linewidth}}{\cellcolor[rgb]{0.9,0.9,0.9}{%
	\makecell[{{p{\linewidth}}}]{% 
	  \tt {\tiny [GM$\rangle$B]}  
	 Du nimmst an einem kollaborativen Verhandlungspiel Teil.\\ \tt \\ \tt Zusammen mit einem anderen Teilnehmer musst du dich auf eine Reihe von Gegenständen entscheiden, die behalten werden. Jeder von euch hat eine persönliche Verteilung über die Wichtigkeit der einzelnen Gegenstände. Jeder von euch hat eine eigene Meinung darüber, wie wichtig jeder einzelne Gegenstand ist (Gegenstandswichtigkeit). Du kennst die Wichtigkeitsverteilung des anderen Spielers nicht. Zusätzlich siehst du, wie viel Aufwand jeder Gegenstand verursacht.  \\ \tt Ihr dürft euch nur auf eine Reihe von Gegenständen einigen, wenn der Gesamtaufwand der ausgewählten Gegenstände den Maximalaufwand nicht überschreitet:\\ \tt \\ \tt Maximalaufwand = 3358\\ \tt \\ \tt Hier sind die einzelnen Aufwände der Gegenstände:\\ \tt \\ \tt Aufwand der Gegenstände = {"A07": 171, "A36": 933, "B73": 552, "C45": 219, "B93": 275, "B98": 778, "A87": 341, "A63": 615, "B05": 519, "C13": 862, "B16": 262, "B33": 377, "C34": 347, "C89": 349, "C11": 117}\\ \tt \\ \tt Hier ist deine persönliche Verteilung der Wichtigkeit der einzelnen Gegenstände:\\ \tt \\ \tt Werte der Gegenstandswichtigkeit = {"A07": 138, "A36": 583, "B73": 868, "C45": 822, "B93": 783, "B98": 65, "A87": 262, "A63": 121, "B05": 508, "C13": 780, "B16": 461, "B33": 484, "C34": 668, "C89": 389, "C11": 808}\\ \tt \\ \tt Ziel:\\ \tt \\ \tt Dein Ziel ist es, eine Reihe von Gegenständen auszuhandeln, die dir möglichst viel bringt (d. h. Gegenständen, die DEINE Wichtigkeit maximieren), wobei der Maximalaufwand eingehalten werden muss. Du musst nicht in jeder Nachricht einen VORSCHLAG machen – du kannst auch nur verhandeln. Alle Taktiken sind erlaubt!\\ \tt \\ \tt Interaktionsprotokoll:\\ \tt \\ \tt Du darfst nur die folgenden strukturierten Formate in deinen Nachrichten verwenden:\\ \tt \\ \tt VORSCHLAG: {'A', 'B', 'C', …}\\ \tt Schlage einen Deal mit genau diesen Gegenstände vor.\\ \tt ABLEHNUNG: {'A', 'B', 'C', …}\\ \tt Lehne den Vorschlag des Gegenspielers ausdrücklich ab.\\ \tt ARGUMENT: {'...'}\\ \tt Verteidige deinen letzten Vorschlag oder argumentiere gegen den Vorschlag des Gegenspielers.\\ \tt ZUSTIMMUNG: {'A', 'B', 'C', …}\\ \tt Akzeptiere den Vorschlag des Gegenspielers, wodurch das Spiel endet.\\ \tt STRATEGISCHE ÜBERLEGUNGEN: {'...'}\\ \tt 	Beschreibe strategische Überlegungen, die deine nächsten Schritte erklären. Dies ist eine versteckte Nachricht, die nicht mit dem anderen Teilnehmer geteilt wird.\\ \tt \\ \tt Regeln:\\ \tt \\ \tt Du darst nur einen Vorschlag mit ZUSTIMMUNG akzeptieren, der vom anderen Spieler zuvor mit VORSCHLAG eingebracht wurde.\\ \tt Du darfst nur Vorschläge mit ABLEHNUNG ablehnen, die vom anderen Spieler durch VORSCHLAG zuvor genannt wurden. \\ \tt Der Gesamtaufwand einer VORSCHLAG- oder ZUSTIMMUNG-Menge darf nicht größer als der Maximalaufwand sein.  \\ \tt Offenbare deine versteckte Wichtigkeitsverteilung nicht.\\ \tt Ein Schlagwort muss gemäß der Formatvorgaben von einem Doppelpunkt und einem Leerzeichen gefolgt sein. Das Argument muss eine Python-Menge mit 0 oder mehr Strings sein.  \\ \tt Die Form muss also SCHLAGWORT: {...} sein.\\ \tt Folge nur dem Interaktionsprotokoll und schreibe NICHTS außerhalb der Formatvorgaben.\\ \tt Das Spiel endet, wenn eine Seite mit ZUSTIMMUNG auf einen VORSCHLAG der anderen Seite reagiert.  \\ \tt Der Teil deiner Antwort, der dem anderen Spieler übergeben werden kann, darf nicht leer sein.  \\ \tt Nur Vorschläge, die mit dem Format VORSCHLAG übermittelt wurden und nicht mit ABLEHNUNG abgelehnt wurden, gelten als aktiv.  \\ \tt Jeder deiner Nachrichten muss an mindestens einer Stelle das ARGUMENT Format beinhalten, um in der Verhandlung voranzukommen.\\ \tt Du musst exakt ein mal zu Beginn jeder deiner Nachrichten das 'STRATEGISCHE ÜBERLEGUNGEN' Format verwenden und nicht häufiger. Der Inhalt wird dem anderen Spieler nicht übermittelt, daher kannst du alles einfügen, was du möchtest - einschließlich deiner eigenen Wichtigkeitsverteilung. Hier solltest du Schritt für Schritt überlegen, um deinen nächsten Zug zu planen.\\ \tt \\ \tt Du erhältst nun die erste Nachricht des anderen Spielers.\\ \tt \\ \tt VORSCHLAG: {'C11', 'C45', 'B93', 'C34', 'B16', 'B73', 'B33', 'C89', 'B05', 'A87'}\\ \tt ARGUMENT: {'Ich schlage dieses Paket vor, weil es den Maximalaufwand exakt ausschöpft und überwiegend Gegenstände mit sehr gutem Nutzen-Aufwand-Verhältnis enthält. Insbesondere C11, C45 und B93 sind extrem effizient. Wenn dir einzelne Positionen wichtig sind, können wir über Austausch sprechen, aber diese drei sollten unbedingt bleiben.'} 
	  } 
	   } 
	   } 
	 & \\ 
 

    \theutterance \stepcounter{utterance}  

    & & & \multicolumn{4}{p{0.6\linewidth}}{\cellcolor[rgb]{1,0.85,0.72}{%
	\makecell[{{p{\linewidth}}}]{% 
	  \tt {\tiny [GM$\langle$B]}  
	 STRATEGISCHE ÜBERLEGUNGEN: {'Berechne Werte, Kosten, Nutzen/Kosten. Werte: C11=808/117=6.91, C45=822/219=3.75, B93=783/275=2.85, B73=868/552=1.57, C34=668/347=1.92, B16=461/262=1.76, B33=484/377=1.28, C89=389/349=1.11, B05=508/519=0.98, A87=262/341=0.77. Summe Kosten ihres Pakets: 117+219+275+552+347+262+377+349+519+341=3358 genau Max. Summe Wert für mich: 608? Rechnen: 808+822=1630, +783=2413, +868=3281, +668=3949, +461=4410, +484=4894, +389=5283, +508=5791, +262=6053. Sehr gut. Prüfe ob ich durch Austausch B98 (wert 65, schlecht) A36 (583/933=0.62) A07 (138/171=0.81) A63 (121/615=0.20) C13 (780/862=0.90) verbessern kann? Unwahrscheinlich, da Paket schon sehr gut. Aber ich könnte versuchen noch C13 (780) reinzunehmen, braucht 862 Kosten, müssten andere raus. Welche rauswerfen, um meinen Wert zu steigern: Ersetze z.B. B05 (508,519) + A87 (262,341) =860 Kosten frei, fast genug; plus 2 mehr -> Tausche gegen C13 (780). Wertänderung: -508-262+780=+10 bei +2 Kosten -> unmöglich wegen Budget. Oder tausche nur B05 (508) und etwas kleines: B05(519) + C11(117) =636 Kosten -> zu wenig. B33(377)+B05(519)=896, könnte für C13(862) passen und 34 frei. Wertänderung: -484-508+780=-212, schlechter. B16(262)+B05(519)=781 Kosten, passt nicht. B05(519)+C89(349)=868 -> Kosten +6 über C13 862, aber wir würden 6 sparen, okay: Replace B05 and C89 (cost 868) with C13 (862): frees 6. Value change: -508-389+780=-117 (worse). B73(552)+A87(341)=893 -> for C13(862) and +31 spare; value change: -868-262+780=-350 (worse). C34(347)+A87(341)=688 -> not enough. C45 or C11 are must keep for me; opponent claimed must keep C11,C45,B93 which also align with me; good to agree quickly. Maybe negotiate minor swap to appear cooperative but keep core. I could ask to swap A87 (low ROI) for A07 (138/171) if capacity: Replace A87(341) with A07(171): frees 170 cost, but then total drops to 3188, under max; could add something else: with 170 freed, we could add A36(933) no, A63(615) no, B98(778) no, C13(862) no, B73 already in, B05 already in, B16 in, B33 in, C34 in, C89 in, C11 in. Remaining candidates: A07 (used), A36, A63, B93(in), B98, C45(in), C11(in). None fit <=170. So swap is wasteful. Maybe swap A87(341) + free 0 to include A07 as additional if we can free 171 by removing something with low value/cost: replace B05(519) by A07(171) and add something else? That would free 348 cost; we could add nothing else small; but value loss: -508+138=-370, bad. Current package seems near-optimal for me; accept quickly to end. Must include ARGUMENT and respond with ZUSTIMMUNG. Ensure formatting.'}\\ \tt ARGUMENT: {'C11, C45 und B93 sind auch für mich äußerst effizient – da sind wir völlig aligned. Das Gesamtpaket trifft aus meiner Sicht ein sehr starkes Nutzen-Aufwand-Profil, ohne offensichtliche dominated items. Von meiner Seite passt das so.'}\\ \tt ZUSTIMMUNG: {'C11', 'C45', 'B93', 'C34', 'B16', 'B73', 'B33', 'C89', 'B05', 'A87'} 
	  } 
	   } 
	   } 
	  \\ 
 

    \theutterance \stepcounter{utterance}  

    & & & \multicolumn{2}{p{0.3\linewidth}}{\cellcolor[rgb]{0.95,0.95,0.95}{%
	\makecell[{{p{\linewidth}}}]{% 
	  \tt {\tiny [GM$|$GM]}  
	 game successful 
	  } 
	   } 
	   } 
	 & & \\ 
 

    \theutterance \stepcounter{utterance}  

    & & & \multicolumn{2}{p{0.3\linewidth}}{\cellcolor[rgb]{0.95,0.95,0.95}{%
	\makecell[{{p{\linewidth}}}]{% 
	  \tt {\tiny [GM$|$GM]}  
	 end game 
	  } 
	   } 
	   } 
	 & & \\ 
 

\end{supertabular}
}

\end{document}
