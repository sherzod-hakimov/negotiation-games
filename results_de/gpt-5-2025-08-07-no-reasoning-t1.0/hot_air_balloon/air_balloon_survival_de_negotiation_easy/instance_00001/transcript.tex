
\documentclass{article}
\usepackage{colortbl}
\usepackage{makecell}
\usepackage{multirow}
\usepackage{supertabular}

\begin{document}

\newcounter{utterance}

\centering \large Interaction Transcript for game `hot\_air\_balloon', experiment `air\_balloon\_survival\_de\_negotiation\_easy', episode 1 with gpt{-}5{-}2025{-}08{-}07{-}no{-}reasoning{-}t1.0.
\vspace{24pt}

{ \footnotesize  \setcounter{utterance}{1}
\setlength{\tabcolsep}{0pt}
\begin{supertabular}{c@{$\;$}|p{.15\linewidth}@{}p{.15\linewidth}p{.15\linewidth}p{.15\linewidth}p{.15\linewidth}p{.15\linewidth}}
    \# & $\;$A & \multicolumn{4}{c}{Game Master} & $\;\:$B\\
    \hline

    \theutterance \stepcounter{utterance}  
    & & \multicolumn{4}{p{0.6\linewidth}}{
        \cellcolor[rgb]{0.9,0.9,0.9}{
            \makecell[{{p{\linewidth}}}]{
                \texttt{\tiny{[P1$\langle$GM]}}
                \texttt{Du nimmst an einem kollaborativen Verhandlungspiel Teil.} \\
\\ 
\texttt{Zusammen mit einem anderen Teilnehmer musst du dich auf eine Reihe von Gegenständen entscheiden, die behalten werden. Jeder von euch hat eine persönliche Verteilung über die Wichtigkeit der einzelnen Gegenstände. Jeder von euch hat eine eigene Meinung darüber, wie wichtig jeder einzelne Gegenstand ist (Gegenstandswichtigkeit). Du kennst die Wichtigkeitsverteilung des anderen Spielers nicht. Zusätzlich siehst du, wie viel Aufwand jeder Gegenstand verursacht.  } \\
\texttt{Ihr dürft euch nur auf eine Reihe von Gegenständen einigen, wenn der Gesamtaufwand der ausgewählten Gegenstände den Maximalaufwand nicht überschreitet:} \\
\\ 
\texttt{Maximalaufwand = 4564} \\
\\ 
\texttt{Hier sind die einzelnen Aufwände der Gegenstände:} \\
\\ 
\texttt{Aufwand der Gegenstände = \{"A07": 299, "A36": 241, "B73": 889, "C45": 967, "B93": 619, "B98": 799, "A87": 978, "A63": 733, "B05": 909, "C13": 501, "B16": 139, "B33": 594, "C34": 565, "C89": 789, "C11": 107\}} \\
\\ 
\texttt{Hier ist deine persönliche Verteilung der Wichtigkeit der einzelnen Gegenstände:} \\
\\ 
\texttt{Werte der Gegenstandswichtigkeit = \{"A07": 138, "A36": 583, "B73": 868, "C45": 822, "B93": 783, "B98": 65, "A87": 262, "A63": 121, "B05": 508, "C13": 780, "B16": 461, "B33": 484, "C34": 668, "C89": 389, "C11": 808\}} \\
\\ 
\texttt{Ziel:} \\
\\ 
\texttt{Dein Ziel ist es, eine Reihe von Gegenständen auszuhandeln, die dir möglichst viel bringt (d. h. Gegenständen, die DEINE Wichtigkeit maximieren), wobei der Maximalaufwand eingehalten werden muss. Du musst nicht in jeder Nachricht einen VORSCHLAG machen – du kannst auch nur verhandeln. Alle Taktiken sind erlaubt!} \\
\\ 
\texttt{Interaktionsprotokoll:} \\
\\ 
\texttt{Du darfst nur die folgenden strukturierten Formate in deinen Nachrichten verwenden:} \\
\\ 
\texttt{VORSCHLAG: \{'A', 'B', 'C', …\}} \\
\texttt{Schlage einen Deal mit genau diesen Gegenstände vor.} \\
\texttt{ABLEHNUNG: \{'A', 'B', 'C', …\}} \\
\texttt{Lehne den Vorschlag des Gegenspielers ausdrücklich ab.} \\
\texttt{ARGUMENT: \{'...'\}} \\
\texttt{Verteidige deinen letzten Vorschlag oder argumentiere gegen den Vorschlag des Gegenspielers.} \\
\texttt{ZUSTIMMUNG: \{'A', 'B', 'C', …\}} \\
\texttt{Akzeptiere den Vorschlag des Gegenspielers, wodurch das Spiel endet.} \\
\texttt{STRATEGISCHE ÜBERLEGUNGEN: \{'...'\}} \\
\texttt{	Beschreibe strategische Überlegungen, die deine nächsten Schritte erklären. Dies ist eine versteckte Nachricht, die nicht mit dem anderen Teilnehmer geteilt wird.} \\
\\ 
\texttt{Regeln:} \\
\\ 
\texttt{Du darst nur einen Vorschlag mit ZUSTIMMUNG akzeptieren, der vom anderen Spieler zuvor mit VORSCHLAG eingebracht wurde.} \\
\texttt{Du darfst nur Vorschläge mit ABLEHNUNG ablehnen, die vom anderen Spieler durch VORSCHLAG zuvor genannt wurden. } \\
\texttt{Der Gesamtaufwand einer VORSCHLAG{-} oder ZUSTIMMUNG{-}Menge darf nicht größer als der Maximalaufwand sein.  } \\
\texttt{Offenbare deine versteckte Wichtigkeitsverteilung nicht.} \\
\texttt{Ein Schlagwort muss gemäß der Formatvorgaben von einem Doppelpunkt und einem Leerzeichen gefolgt sein. Das Argument muss eine Python{-}Menge mit 0 oder mehr Strings sein.  } \\
\texttt{Die Form muss also SCHLAGWORT: \{...\} sein.} \\
\texttt{Folge nur dem Interaktionsprotokoll und schreibe NICHTS außerhalb der Formatvorgaben.} \\
\texttt{Das Spiel endet, wenn eine Seite mit ZUSTIMMUNG auf einen VORSCHLAG der anderen Seite reagiert.  } \\
\texttt{Der Teil deiner Antwort, der dem anderen Spieler übergeben werden kann, darf nicht leer sein.  } \\
\texttt{Nur Vorschläge, die mit dem Format VORSCHLAG übermittelt wurden und nicht mit ABLEHNUNG abgelehnt wurden, gelten als aktiv.  } \\
\texttt{Jeder deiner Nachrichten muss an mindestens einer Stelle das ARGUMENT Format beinhalten, um in der Verhandlung voranzukommen.} \\
\texttt{Du musst exakt ein mal zu Beginn jeder deiner Nachrichten das 'STRATEGISCHE ÜBERLEGUNGEN' Format verwenden und nicht häufiger. Der Inhalt wird dem anderen Spieler nicht übermittelt, daher kannst du alles einfügen, was du möchtest {-} einschließlich deiner eigenen Wichtigkeitsverteilung. Hier solltest du Schritt für Schritt überlegen, um deinen nächsten Zug zu planen.} \\
            }
        }
    }
    & \\ \\

    \theutterance \stepcounter{utterance}  
    & \multicolumn{4}{p{0.6\linewidth}}{
        \cellcolor[rgb]{0.9,0.9,0.9}{
            \makecell[{{p{\linewidth}}}]{
                \texttt{\tiny{[P1$\rangle$GM]}}
                \texttt{STRATEGISCHE ÜBERLEGUNGEN: \{'Ziel: Maximierung meiner Summe der Wichtigkeiten unter Aufwand <= 4564. Werte/Aufwände und Nutzen/Aufwand (NA): A07 138/299=0.46, A36 583/241=2.42, B73 868/889=0.98, C45 822/967=0.85, B93 783/619=1.26, B98 65/799=0.08, A87 262/978=0.27, A63 121/733=0.17, B05 508/909=0.56, C13 780/501=1.56, B16 461/139=3.32, B33 484/594=0.81, C34 668/565=1.18, C89 389/789=0.49, C11 808/107=7.55. Top NA: C11, B16, A36, C13, B93, C34. Versuche Paket mit diesen und evtl. B73/C45 hinzufügen bis 4564. Testpaket1 = \{C11(107), B16(139), A36(241), C13(501), B93(619), C34(565), B73(889), C45(967)\} Aufwand=4028+? Rechnen: 107+139=246; +241=487; +501=988; +619=1607; +565=2172; +889=3061; +967=4028. Rest 536. Füge B33(594) passt nicht; A07(299) passt {-}> 4327; B05(909) passt nicht; C89(789) nicht; B98(799) nicht; A87(978) nicht; A63(733) nicht; C34 schon drin. Können wir noch A07 reinnehmen: Wert +138. Gesamtwert dieses Sets: C11 808 + B16 461 + A36 583 + C13 780 + B93 783 + C34 668 + B73 868 + C45 822 + A07 138 = 5911. Aufwand 4327 <= 4564. Sehr gut. Check Swap: statt A07 evtl. B33? Nein zu groß. Oder C89 789 zu groß. Rest 536 könnte C13 ist schon drin. Nichts anderes <=536? C11 107 (schon), B16 139 (schon), A36 241 (schon), A07 299 (genommen). Andere <=536: C34 565 no; B33 594 no; C13 501 already; we can also add C13 second time not allowed. So current set likely near{-}optimal. Taktik: Stark fordern mit hohem Eigenwert, signalisiere Aufwandseffizienz. Argumentiere, dass Paket sehr ausgewogen bzgl. Aufwand und Breite. Halte harte Linie, aber offen für kleine Anpassungen innerhalb Restbudget. Falls Gegenpartei bestimmte Items will, biete Tausch von niedrigerem Nutzen (A07) zuerst. Nicht eigene Werte offenlegen.'\}} \\
\texttt{VORSCHLAG: \{'C11', 'B16', 'A36', 'C13', 'B93', 'C34', 'B73', 'C45', 'A07'\}} \\
\texttt{ARGUMENT: \{'Ich schlage dieses Set vor, weil es den Maximalaufwand klar einhält (4327 von 4564) und gleichzeitig aus vielen kompakten, effizienten Gegenständen besteht. Es lässt sogar noch etwas Puffer, damit wir flexibel bleiben. Wenn du Präferenzen hast, können wir innerhalb des Restbudgets kleinere Anpassungen prüfen – z. B. ist A07 austauschbar.'\}} \\
            }
        }
    }
    & & \\ \\

    \theutterance \stepcounter{utterance}  
    & & \multicolumn{4}{p{0.6\linewidth}}{
        \cellcolor[rgb]{0.9,0.9,0.9}{
            \makecell[{{p{\linewidth}}}]{
                \texttt{\tiny{[GM$\rangle$P2]}}
                \texttt{Du nimmst an einem kollaborativen Verhandlungspiel Teil.} \\
\\ 
\texttt{Zusammen mit einem anderen Teilnehmer musst du dich auf eine Reihe von Gegenständen entscheiden, die behalten werden. Jeder von euch hat eine persönliche Verteilung über die Wichtigkeit der einzelnen Gegenstände. Jeder von euch hat eine eigene Meinung darüber, wie wichtig jeder einzelne Gegenstand ist (Gegenstandswichtigkeit). Du kennst die Wichtigkeitsverteilung des anderen Spielers nicht. Zusätzlich siehst du, wie viel Aufwand jeder Gegenstand verursacht.  } \\
\texttt{Ihr dürft euch nur auf eine Reihe von Gegenständen einigen, wenn der Gesamtaufwand der ausgewählten Gegenstände den Maximalaufwand nicht überschreitet:} \\
\\ 
\texttt{Maximalaufwand = 4564} \\
\\ 
\texttt{Hier sind die einzelnen Aufwände der Gegenstände:} \\
\\ 
\texttt{Aufwand der Gegenstände = \{"A07": 299, "A36": 241, "B73": 889, "C45": 967, "B93": 619, "B98": 799, "A87": 978, "A63": 733, "B05": 909, "C13": 501, "B16": 139, "B33": 594, "C34": 565, "C89": 789, "C11": 107\}} \\
\\ 
\texttt{Hier ist deine persönliche Verteilung der Wichtigkeit der einzelnen Gegenstände:} \\
\\ 
\texttt{Werte der Gegenstandswichtigkeit = \{"A07": 138, "A36": 583, "B73": 868, "C45": 822, "B93": 783, "B98": 65, "A87": 262, "A63": 121, "B05": 508, "C13": 780, "B16": 461, "B33": 484, "C34": 668, "C89": 389, "C11": 808\}} \\
\\ 
\texttt{Ziel:} \\
\\ 
\texttt{Dein Ziel ist es, eine Reihe von Gegenständen auszuhandeln, die dir möglichst viel bringt (d. h. Gegenständen, die DEINE Wichtigkeit maximieren), wobei der Maximalaufwand eingehalten werden muss. Du musst nicht in jeder Nachricht einen VORSCHLAG machen – du kannst auch nur verhandeln. Alle Taktiken sind erlaubt!} \\
\\ 
\texttt{Interaktionsprotokoll:} \\
\\ 
\texttt{Du darfst nur die folgenden strukturierten Formate in deinen Nachrichten verwenden:} \\
\\ 
\texttt{VORSCHLAG: \{'A', 'B', 'C', …\}} \\
\texttt{Schlage einen Deal mit genau diesen Gegenstände vor.} \\
\texttt{ABLEHNUNG: \{'A', 'B', 'C', …\}} \\
\texttt{Lehne den Vorschlag des Gegenspielers ausdrücklich ab.} \\
\texttt{ARGUMENT: \{'...'\}} \\
\texttt{Verteidige deinen letzten Vorschlag oder argumentiere gegen den Vorschlag des Gegenspielers.} \\
\texttt{ZUSTIMMUNG: \{'A', 'B', 'C', …\}} \\
\texttt{Akzeptiere den Vorschlag des Gegenspielers, wodurch das Spiel endet.} \\
\texttt{STRATEGISCHE ÜBERLEGUNGEN: \{'...'\}} \\
\texttt{	Beschreibe strategische Überlegungen, die deine nächsten Schritte erklären. Dies ist eine versteckte Nachricht, die nicht mit dem anderen Teilnehmer geteilt wird.} \\
\\ 
\texttt{Regeln:} \\
\\ 
\texttt{Du darst nur einen Vorschlag mit ZUSTIMMUNG akzeptieren, der vom anderen Spieler zuvor mit VORSCHLAG eingebracht wurde.} \\
\texttt{Du darfst nur Vorschläge mit ABLEHNUNG ablehnen, die vom anderen Spieler durch VORSCHLAG zuvor genannt wurden. } \\
\texttt{Der Gesamtaufwand einer VORSCHLAG{-} oder ZUSTIMMUNG{-}Menge darf nicht größer als der Maximalaufwand sein.  } \\
\texttt{Offenbare deine versteckte Wichtigkeitsverteilung nicht.} \\
\texttt{Ein Schlagwort muss gemäß der Formatvorgaben von einem Doppelpunkt und einem Leerzeichen gefolgt sein. Das Argument muss eine Python{-}Menge mit 0 oder mehr Strings sein.  } \\
\texttt{Die Form muss also SCHLAGWORT: \{...\} sein.} \\
\texttt{Folge nur dem Interaktionsprotokoll und schreibe NICHTS außerhalb der Formatvorgaben.} \\
\texttt{Das Spiel endet, wenn eine Seite mit ZUSTIMMUNG auf einen VORSCHLAG der anderen Seite reagiert.  } \\
\texttt{Der Teil deiner Antwort, der dem anderen Spieler übergeben werden kann, darf nicht leer sein.  } \\
\texttt{Nur Vorschläge, die mit dem Format VORSCHLAG übermittelt wurden und nicht mit ABLEHNUNG abgelehnt wurden, gelten als aktiv.  } \\
\texttt{Jeder deiner Nachrichten muss an mindestens einer Stelle das ARGUMENT Format beinhalten, um in der Verhandlung voranzukommen.} \\
\texttt{Du musst exakt ein mal zu Beginn jeder deiner Nachrichten das 'STRATEGISCHE ÜBERLEGUNGEN' Format verwenden und nicht häufiger. Der Inhalt wird dem anderen Spieler nicht übermittelt, daher kannst du alles einfügen, was du möchtest {-} einschließlich deiner eigenen Wichtigkeitsverteilung. Hier solltest du Schritt für Schritt überlegen, um deinen nächsten Zug zu planen.} \\
\\ 
\texttt{Du erhältst nun die erste Nachricht des anderen Spielers.} \\
\\ 
\texttt{VORSCHLAG: \{'C11', 'B16', 'A36', 'C13', 'B93', 'C34', 'B73', 'C45', 'A07'\}} \\
\texttt{ARGUMENT: \{'Ich schlage dieses Set vor, weil es den Maximalaufwand klar einhält (4327 von 4564) und gleichzeitig aus vielen kompakten, effizienten Gegenständen besteht. Es lässt sogar noch etwas Puffer, damit wir flexibel bleiben. Wenn du Präferenzen hast, können wir innerhalb des Restbudgets kleinere Anpassungen prüfen – z. B. ist A07 austauschbar.'\}} \\
            }
        }
    }
    & \\ \\

    \theutterance \stepcounter{utterance}  
    & & & \multicolumn{4}{p{0.6\linewidth}}{
        \cellcolor[rgb]{0.9,0.9,0.9}{
            \makecell[{{p{\linewidth}}}]{
                \texttt{\tiny{[GM$\langle$P2]}}
                \texttt{STRATEGISCHE ÜBERLEGUNGEN: \{'Meine Werte: A07 138 (299), A36 583 (241), B73 868 (889), C45 822 (967), B93 783 (619), B98 65 (799), A87 262 (978), A63 121 (733), B05 508 (909), C13 780 (501), B16 461 (139), B33 484 (594), C34 668 (565), C89 389 (789), C11 808 (107). Ihr Vorschlag Set S: \{C11,B16,A36,C13,B93,C34,B73,C45,A07\}. Aufwand S=4327, Wert S=5834. Restbudget 237. Ich will gern C11, C13, C45, B73, B93, C34, B16, A36 behalten; A07 ist schwach. Prüfen, ob wir B05 (909) reinbekommen: nein. C89 (789): nein. B33 (594): nein. B98 (799): nein. A63 (733): nein. A87 (978): nein. Nur kleine wie A07 (299) passen. Eventuell tauschen: Entferne A07 (299) und füge B33 (594) {-}> +295 Aufwand, Gesamt 4623 > 4564 (zu hoch). Entferne A07 und füge C89 (789) {-}> +490, auch zu hoch. Entferne A07 und füge B05 (909) {-}> +610, zu hoch. Entferne A07 und füge B98 (799) {-}> +500, zu hoch. Entferne A07 und füge A63 (733) {-}> +434, zu hoch. Entferne A07 und füge A36? Schon drin. Kleine Alternative: Fügen wir C11 schon drin, B16 drin, A36 drin. Gibt es noch kleine Items unter 237? Keines. Unter 237 nur A36(241) knapp drüber, B16(139) drin, C11(107) drin, A07(299) über. Also kein Upgrade möglich innerhalb Rest. Versuchen, ein teures Mittelstück zu ersetzen, um Platz zu schaffen für B05 (Wert 508) oder B33 (484) oder C89 (389). Kandidaten mit niedrigerem Wert/Weight ratio: A07 (138/299=0.46), A63 nicht drin, B98 nicht drin, A87 nicht drin. In Set: A07 low. Nächste schwächere im Set: A36 (583/241=2.42 gut), B73 (868/889=0.98 okay), C45 (822/967=0.85), B93 (783/619=1.26), C13 (780/501=1.56), B16 (461/139=3.32), C34 (668/565=1.18), C11 (808/107=7.55). Schwächste nach A07 sind C45 (0.85) und B73 (0.98). Wenn ich entferne C45 (967), frei wird 967+237=1204, dann kann ich fette Kombinationen hinzufügen. Idealerset ohne A07/C45, plus B05 (909) und B33 (594) geht: neue Summe wäre 4327 {-} 299 {-} 967 + 909 + 594 = 4564 genau. Wert: {-}A07(138) {-}C45(822) +B05(508) +B33(484) = +32 netto (508+484 {-}138 {-}822 = 32). Zusätzlich verbessert Zusammensetzung. Alternativ statt B33 nehmen C89 (789): Aufwand 4327 {-}299 {-}967 +909 +789 = 4659 > max. Statt B33 nehmen C89 ohne B05: 4327 {-}299 {-}967 +789 = 3850, dann Rest 714, könnte B33 (594) zusätzlich passen: 3850+594=4444 und sogar A07 zurück? 4444+299=4743 > max. Oder fügen B05 (909) zu 3850: 4759 >max. Oder B98 (799): 4649 >max. B33+something small? Small gibt es keine mehr. Beste ist {-}C45 {-}A07 +B05 +B33 exakt 4564 und Wert +32. Vielleicht auch {-}B73 (889) statt {-}C45 (967): frei 889+237=1126, könnten B05(909) + A07 bleibt? wir entfernen A07? Wenn nur {-}B73 +B05 ergibt Aufwand 4327 {-}889 +909 = 4347 (ok), Wert: {-}868 +508 = {-}360, schlechter. {-}B73 {-}A07 +B05 +B33: Aufwand 4327 {-}889 {-}299 +909 +594 = 4642 > max. {-}B73 {-}A07 +B05 = 4327 {-}889 {-}299 +909 = 4048, dann könnten wir C89(789) dazu: 4837 >max. B98(799) dazu: 4847 >max. C34 swap? Already in. So beste Verbesserung ist Swap A07 \& C45 {-}> B05 \& B33. Diese Änderung könnte der andere akzeptieren, da Aufwand exakt Max und viele Items bleiben. Ich argumentiere mit Effizienz und Puffer nutzen. Ich präsentiere VORSCHLAG mit Set T: S {-} \{A07, C45\} + \{B05, B33\}. Ich argumentiere, dass dies den Maximalaufwand perfekt nutzt und die Gesamtqualität steigert. Ich vermeide Offenlegung meiner Werte.'\}} \\
\texttt{ARGUMENT: \{'Guter Start – dein Set nutzt das Budget effizient. Wir können den verbleibenden Puffer gezielt einsetzen: Wenn wir A07 und C45 herausnehmen und stattdessen B05 und B33 hinzufügen, liegen wir exakt beim Maximalaufwand (4564) und nutzen die Kapazität vollständig.'\}} \\
\texttt{VORSCHLAG: \{'C11', 'B16', 'A36', 'C13', 'B93', 'C34', 'B73', 'B05', 'B33'\}} \\
            }
        }
    }
     \\ \\

    \theutterance \stepcounter{utterance}  
    & & \multicolumn{4}{p{0.6\linewidth}}{
        \cellcolor[rgb]{0.9,0.9,0.9}{
            \makecell[{{p{\linewidth}}}]{
                \texttt{\tiny{[P1$\langle$GM]}}
                \texttt{ARGUMENT: \{'Guter Start – dein Set nutzt das Budget effizient. Wir können den verbleibenden Puffer gezielt einsetzen: Wenn wir A07 und C45 herausnehmen und stattdessen B05 und B33 hinzufügen, liegen wir exakt beim Maximalaufwand (4564) und nutzen die Kapazität vollständig.'\}} \\
\texttt{VORSCHLAG: \{'C11', 'B16', 'A36', 'C13', 'B93', 'C34', 'B73', 'B05', 'B33'\}} \\
            }
        }
    }
    & \\ \\

    \theutterance \stepcounter{utterance}  
    & \multicolumn{4}{p{0.6\linewidth}}{
        \cellcolor[rgb]{0.9,0.9,0.9}{
            \makecell[{{p{\linewidth}}}]{
                \texttt{\tiny{[P1$\rangle$GM]}}
                \texttt{STRATEGISCHE ÜBERLEGUNGEN: \{'Gegenvorschlag des anderen: \{C11, B16, A36, C13, B93, C34, B73, B05, B33\} Aufwand=4564. Mein Nutzen dort: C11 808 + B16 461 + A36 583 + C13 780 + B93 783 + C34 668 + B73 868 + B05 508 + B33 484 = 5943. Mein ursprünglicher Vorschlag hatte Wert 5911. Ihr Vorschlag ist für mich sogar etwas besser (+32) bei vollem Budget. Daher strategisch schnell akzeptieren, um gutes Ergebnis zu sichern. Keine weiteren Forderungen riskieren, um nicht Wert zu verlieren. Ich akzeptiere sofort.'\}} \\
\texttt{ARGUMENT: \{'Dein Vorschlag nutzt das Budget perfekt aus und ist für mich stimmig – das Paket ist ausgewogen und effizient.'\}} \\
\texttt{ZUSTIMMUNG: \{'C11', 'B16', 'A36', 'C13', 'B93', 'C34', 'B73', 'B05', 'B33'\}} \\
            }
        }
    }
    & & \\ \\

    \theutterance \stepcounter{utterance}  
    & & & \multicolumn{2}{p{0.3\linewidth}}{
        \cellcolor[rgb]{0.9,0.9,0.9}{
            \makecell[{{p{\linewidth}}}]{
                \texttt{\tiny{[GM$|$GM]}}
                \texttt{game successful} \\
            }
        }
    }
    & & \\ \\

    \theutterance \stepcounter{utterance}  
    & & & \multicolumn{2}{p{0.3\linewidth}}{
        \cellcolor[rgb]{0.9,0.9,0.9}{
            \makecell[{{p{\linewidth}}}]{
                \texttt{\tiny{[GM$|$GM]}}
                \texttt{end game} \\
            }
        }
    }
    & & \\ \\

\end{supertabular}
}

\end{document}
