
\documentclass{article}
\usepackage{colortbl}
\usepackage{makecell}
\usepackage{multirow}
\usepackage{supertabular}

\begin{document}

\newcounter{utterance}

\centering \large Interaction Transcript for game `hot\_air\_balloon', experiment `air\_balloon\_survival\_de\_reasoning off\_easy', episode 1 with gpt{-}5{-}2025{-}08{-}07{-}no{-}reasoning{-}t1.0.
\vspace{24pt}

{ \footnotesize  \setcounter{utterance}{1}
\setlength{\tabcolsep}{0pt}
\begin{supertabular}{c@{$\;$}|p{.15\linewidth}@{}p{.15\linewidth}p{.15\linewidth}p{.15\linewidth}p{.15\linewidth}p{.15\linewidth}}
    \# & $\;$A & \multicolumn{4}{c}{Game Master} & $\;\:$B\\
    \hline

    \theutterance \stepcounter{utterance}  
    & & \multicolumn{4}{p{0.6\linewidth}}{
        \cellcolor[rgb]{0.9,0.9,0.9}{
            \makecell[{{p{\linewidth}}}]{
                \texttt{\tiny{[P1$\langle$GM]}}
                \texttt{Du nimmst an einem kollaborativen Verhandlungspiel Teil.} \\
\\ 
\texttt{Zusammen mit einem anderen Teilnehmer musst du dich auf eine Reihe von Gegenständen entscheiden, die behalten werden. Jeder von euch hat eine persönliche Verteilung über die Wichtigkeit der einzelnen Gegenstände. Jeder von euch hat eine eigene Meinung darüber, wie wichtig jeder einzelne Gegenstand ist (Gegenstandswichtigkeit). Du kennst die Wichtigkeitsverteilung des anderen Spielers nicht. Zusätzlich siehst du, wie viel Aufwand jeder Gegenstand verursacht.  } \\
\texttt{Ihr dürft euch nur auf eine Reihe von Gegenständen einigen, wenn der Gesamtaufwand der ausgewählten Gegenstände den Maximalaufwand nicht überschreitet:} \\
\\ 
\texttt{Maximalaufwand = 4100} \\
\\ 
\texttt{Hier sind die einzelnen Aufwände der Gegenstände:} \\
\\ 
\texttt{Aufwand der Gegenstände = \{"C41": 108, "C15": 133, "C28": 669, "C48": 835, "B78": 460, "A75": 537, "A60": 573, "B05": 737, "A56": 865, "B66": 855, "C60": 596, "B49": 719, "A35": 533, "A93": 549, "B10": 31\}} \\
\\ 
\texttt{Hier ist deine persönliche Verteilung der Wichtigkeit der einzelnen Gegenstände:} \\
\\ 
\texttt{Werte der Gegenstandswichtigkeit = \{"C41": 138, "C15": 583, "C28": 822, "C48": 783, "B78": 65, "A75": 262, "A60": 121, "B05": 508, "A56": 780, "B66": 461, "C60": 484, "B49": 668, "A35": 389, "A93": 808, "B10": 215\}} \\
\\ 
\texttt{Ziel:} \\
\\ 
\texttt{Dein Ziel ist es, eine Reihe von Gegenständen auszuhandeln, die dir möglichst viel bringt (d. h. Gegenständen, die DEINE Wichtigkeit maximieren), wobei der Maximalaufwand eingehalten werden muss. Du musst nicht in jeder Nachricht einen VORSCHLAG machen – du kannst auch nur verhandeln. Alle Taktiken sind erlaubt!} \\
\\ 
\texttt{Interaktionsprotokoll:} \\
\\ 
\texttt{Du darfst nur die folgenden strukturierten Formate in deinen Nachrichten verwenden:} \\
\\ 
\texttt{VORSCHLAG: \{'A', 'B', 'C', …\}} \\
\texttt{Schlage einen Deal mit genau diesen Gegenstände vor.} \\
\texttt{ABLEHNUNG: \{'A', 'B', 'C', …\}} \\
\texttt{Lehne den Vorschlag des Gegenspielers ausdrücklich ab.} \\
\texttt{ARGUMENT: \{'...'\}} \\
\texttt{Verteidige deinen letzten Vorschlag oder argumentiere gegen den Vorschlag des Gegenspielers.} \\
\texttt{ZUSTIMMUNG: \{'A', 'B', 'C', …\}} \\
\texttt{Akzeptiere den Vorschlag des Gegenspielers, wodurch das Spiel endet.} \\
\\ 
\\ 
\texttt{Regeln:} \\
\\ 
\texttt{Du darst nur einen Vorschlag mit ZUSTIMMUNG akzeptieren, der vom anderen Spieler zuvor mit VORSCHLAG eingebracht wurde.} \\
\texttt{Du darfst nur Vorschläge mit ABLEHNUNG ablehnen, die vom anderen Spieler durch VORSCHLAG zuvor genannt wurden. } \\
\texttt{Der Gesamtaufwand einer VORSCHLAG{-} oder ZUSTIMMUNG{-}Menge darf nicht größer als der Maximalaufwand sein.  } \\
\texttt{Offenbare deine versteckte Wichtigkeitsverteilung nicht.} \\
\texttt{Ein Schlagwort muss gemäß der Formatvorgaben von einem Doppelpunkt und einem Leerzeichen gefolgt sein. Das Argument muss eine Python{-}Menge mit 0 oder mehr Strings sein.  } \\
\texttt{Die Form muss also SCHLAGWORT: \{...\} sein.} \\
\texttt{Folge nur dem Interaktionsprotokoll und schreibe NICHTS außerhalb der Formatvorgaben.} \\
\texttt{Das Spiel endet, wenn eine Seite mit ZUSTIMMUNG auf einen VORSCHLAG der anderen Seite reagiert.  } \\
\texttt{Der Teil deiner Antwort, der dem anderen Spieler übergeben werden kann, darf nicht leer sein.  } \\
\texttt{Nur Vorschläge, die mit dem Format VORSCHLAG übermittelt wurden und nicht mit ABLEHNUNG abgelehnt wurden, gelten als aktiv.  } \\
\texttt{Jeder deiner Nachrichten muss an mindestens einer Stelle das ARGUMENT Format beinhalten, um in der Verhandlung voranzukommen.} \\
            }
        }
    }
    & \\ \\

    \theutterance \stepcounter{utterance}  
    & \multicolumn{4}{p{0.6\linewidth}}{
        \cellcolor[rgb]{0.9,0.9,0.9}{
            \makecell[{{p{\linewidth}}}]{
                \texttt{\tiny{[P1$\rangle$GM]}}
                \texttt{VORSCHLAG: \{'C28', 'C48', 'A56', 'A93', 'B49', 'C41', 'B10'\}} \\
\texttt{ARGUMENT: \{'Ich schlage eine kompakte, wertstarke Auswahl vor: Sie bleibt mit Aufwand 108+835+865+549+719+669+31=3776 deutlich unter dem Limit 4100 und lässt uns flexibel für eventuelle Anpassungen. Alle Positionen sind starke Kandidaten mit gutem Nutzen{-}Aufwand{-}Verhältnis. Wenn dir einzelne Posten wichtig sind, die hier fehlen, sag Bescheid – ich kann bei Bedarf einzelne Elemente tauschen, solange wir unter 4100 bleiben.'\}} \\
            }
        }
    }
    & & \\ \\

    \theutterance \stepcounter{utterance}  
    & & \multicolumn{4}{p{0.6\linewidth}}{
        \cellcolor[rgb]{0.9,0.9,0.9}{
            \makecell[{{p{\linewidth}}}]{
                \texttt{\tiny{[GM$\rangle$P2]}}
                \texttt{Du nimmst an einem kollaborativen Verhandlungspiel Teil.} \\
\\ 
\texttt{Zusammen mit einem anderen Teilnehmer musst du dich auf eine Reihe von Gegenständen entscheiden, die behalten werden. Jeder von euch hat eine persönliche Verteilung über die Wichtigkeit der einzelnen Gegenstände. Jeder von euch hat eine eigene Meinung darüber, wie wichtig jeder einzelne Gegenstand ist (Gegenstandswichtigkeit). Du kennst die Wichtigkeitsverteilung des anderen Spielers nicht. Zusätzlich siehst du, wie viel Aufwand jeder Gegenstand verursacht.  } \\
\texttt{Ihr dürft euch nur auf eine Reihe von Gegenständen einigen, wenn der Gesamtaufwand der ausgewählten Gegenstände den Maximalaufwand nicht überschreitet:} \\
\\ 
\texttt{Maximalaufwand = 4100} \\
\\ 
\texttt{Hier sind die einzelnen Aufwände der Gegenstände:} \\
\\ 
\texttt{Aufwand der Gegenstände = \{"C41": 108, "C15": 133, "C28": 669, "C48": 835, "B78": 460, "A75": 537, "A60": 573, "B05": 737, "A56": 865, "B66": 855, "C60": 596, "B49": 719, "A35": 533, "A93": 549, "B10": 31\}} \\
\\ 
\texttt{Hier ist deine persönliche Verteilung der Wichtigkeit der einzelnen Gegenstände:} \\
\\ 
\texttt{Werte der Gegenstandswichtigkeit = \{"C41": 138, "C15": 583, "C28": 822, "C48": 783, "B78": 65, "A75": 262, "A60": 121, "B05": 508, "A56": 780, "B66": 461, "C60": 484, "B49": 668, "A35": 389, "A93": 808, "B10": 215\}} \\
\\ 
\texttt{Ziel:} \\
\\ 
\texttt{Dein Ziel ist es, eine Reihe von Gegenständen auszuhandeln, die dir möglichst viel bringt (d. h. Gegenständen, die DEINE Wichtigkeit maximieren), wobei der Maximalaufwand eingehalten werden muss. Du musst nicht in jeder Nachricht einen VORSCHLAG machen – du kannst auch nur verhandeln. Alle Taktiken sind erlaubt!} \\
\\ 
\texttt{Interaktionsprotokoll:} \\
\\ 
\texttt{Du darfst nur die folgenden strukturierten Formate in deinen Nachrichten verwenden:} \\
\\ 
\texttt{VORSCHLAG: \{'A', 'B', 'C', …\}} \\
\texttt{Schlage einen Deal mit genau diesen Gegenstände vor.} \\
\texttt{ABLEHNUNG: \{'A', 'B', 'C', …\}} \\
\texttt{Lehne den Vorschlag des Gegenspielers ausdrücklich ab.} \\
\texttt{ARGUMENT: \{'...'\}} \\
\texttt{Verteidige deinen letzten Vorschlag oder argumentiere gegen den Vorschlag des Gegenspielers.} \\
\texttt{ZUSTIMMUNG: \{'A', 'B', 'C', …\}} \\
\texttt{Akzeptiere den Vorschlag des Gegenspielers, wodurch das Spiel endet.} \\
\\ 
\\ 
\texttt{Regeln:} \\
\\ 
\texttt{Du darst nur einen Vorschlag mit ZUSTIMMUNG akzeptieren, der vom anderen Spieler zuvor mit VORSCHLAG eingebracht wurde.} \\
\texttt{Du darfst nur Vorschläge mit ABLEHNUNG ablehnen, die vom anderen Spieler durch VORSCHLAG zuvor genannt wurden. } \\
\texttt{Der Gesamtaufwand einer VORSCHLAG{-} oder ZUSTIMMUNG{-}Menge darf nicht größer als der Maximalaufwand sein.  } \\
\texttt{Offenbare deine versteckte Wichtigkeitsverteilung nicht.} \\
\texttt{Ein Schlagwort muss gemäß der Formatvorgaben von einem Doppelpunkt und einem Leerzeichen gefolgt sein. Das Argument muss eine Python{-}Menge mit 0 oder mehr Strings sein.  } \\
\texttt{Die Form muss also SCHLAGWORT: \{...\} sein.} \\
\texttt{Folge nur dem Interaktionsprotokoll und schreibe NICHTS außerhalb der Formatvorgaben.} \\
\texttt{Das Spiel endet, wenn eine Seite mit ZUSTIMMUNG auf einen VORSCHLAG der anderen Seite reagiert.  } \\
\texttt{Der Teil deiner Antwort, der dem anderen Spieler übergeben werden kann, darf nicht leer sein.  } \\
\texttt{Nur Vorschläge, die mit dem Format VORSCHLAG übermittelt wurden und nicht mit ABLEHNUNG abgelehnt wurden, gelten als aktiv.  } \\
\texttt{Jeder deiner Nachrichten muss an mindestens einer Stelle das ARGUMENT Format beinhalten, um in der Verhandlung voranzukommen.} \\
\\ 
\\ 
\texttt{Du erhältst nun die erste Nachricht des anderen Spielers.} \\
\\ 
\texttt{VORSCHLAG: \{'C28', 'C48', 'A56', 'A93', 'B49', 'C41', 'B10'\}} \\
\texttt{ARGUMENT: \{'Ich schlage eine kompakte, wertstarke Auswahl vor: Sie bleibt mit Aufwand 108+835+865+549+719+669+31=3776 deutlich unter dem Limit 4100 und lässt uns flexibel für eventuelle Anpassungen. Alle Positionen sind starke Kandidaten mit gutem Nutzen{-}Aufwand{-}Verhältnis. Wenn dir einzelne Posten wichtig sind, die hier fehlen, sag Bescheid – ich kann bei Bedarf einzelne Elemente tauschen, solange wir unter 4100 bleiben.'\}} \\
            }
        }
    }
    & \\ \\

    \theutterance \stepcounter{utterance}  
    & & & \multicolumn{4}{p{0.6\linewidth}}{
        \cellcolor[rgb]{0.9,0.9,0.9}{
            \makecell[{{p{\linewidth}}}]{
                \texttt{\tiny{[GM$\langle$P2]}}
                \texttt{ABLEHNUNG: \{'C28', 'C48', 'A56', 'A93', 'B49', 'C41', 'B10'\}} \\
\texttt{ARGUMENT: \{'Danke für den Auftakt! Der Vorschlag ist solide beim Aufwand, aber es fehlen mir einige sehr wertvolle Posten. Ich würde gern zwei bis drei deiner Items gegen für mich wichtigere tauschen, ohne das Limit zu reißen. Ich priorisiere insbesondere C15 und B05 hoch; C60 und A35 sind ebenfalls attraktiv. Wenn wir ein paar deiner Vorschläge mit diesen ersetzen, steigern wir den Gesamtnutzen (für mich) bei vergleichbarem Aufwand.'\}} \\
\texttt{VORSCHLAG: \{'C28', 'C48', 'A56', 'A93', 'B49', 'C15', 'B10'\}} \\
            }
        }
    }
     \\ \\

    \theutterance \stepcounter{utterance}  
    & & \multicolumn{4}{p{0.6\linewidth}}{
        \cellcolor[rgb]{0.9,0.9,0.9}{
            \makecell[{{p{\linewidth}}}]{
                \texttt{\tiny{[P1$\langle$GM]}}
                \texttt{ABLEHNUNG: \{'C28', 'C48', 'A56', 'A93', 'B49', 'C41', 'B10'\}} \\
\texttt{ARGUMENT: \{'Danke für den Auftakt! Der Vorschlag ist solide beim Aufwand, aber es fehlen mir einige sehr wertvolle Posten. Ich würde gern zwei bis drei deiner Items gegen für mich wichtigere tauschen, ohne das Limit zu reißen. Ich priorisiere insbesondere C15 und B05 hoch; C60 und A35 sind ebenfalls attraktiv. Wenn wir ein paar deiner Vorschläge mit diesen ersetzen, steigern wir den Gesamtnutzen (für mich) bei vergleichbarem Aufwand.'\}} \\
\texttt{VORSCHLAG: \{'C28', 'C48', 'A56', 'A93', 'B49', 'C15', 'B10'\}} \\
            }
        }
    }
    & \\ \\

    \theutterance \stepcounter{utterance}  
    & \multicolumn{4}{p{0.6\linewidth}}{
        \cellcolor[rgb]{0.9,0.9,0.9}{
            \makecell[{{p{\linewidth}}}]{
                \texttt{\tiny{[P1$\rangle$GM]}}
                \texttt{ABLEHNUNG: \{'C28', 'C48', 'A56', 'A93', 'B49', 'C15', 'B10'\}} \\
\texttt{ARGUMENT: \{'Danke fürs Gegenangebot. C15 passt aufwandsseitig gut, aber der Verzicht auf C41 zugunsten von C15 senkt unseren Gesamtnutzen aus meiner Sicht deutlich. Außerdem ist B49 für mich weniger attraktiv als C60. Ich würde gern C41 zurückholen und B49 durch C60 ersetzen. Dadurch bleiben wir komfortabel unter 4100 und wahren starke Kernpositionen. Wenn dir B05 wichtig ist, können wir dafür später noch Spielraum schaffen, etwa durch Tausch von A75/A60 anstelle eines Kernpostens.'\}} \\
\texttt{VORSCHLAG: \{'C28', 'C48', 'A56', 'A93', 'C41', 'C15', 'C60', 'B10'\}} \\
            }
        }
    }
    & & \\ \\

    \theutterance \stepcounter{utterance}  
    & & \multicolumn{4}{p{0.6\linewidth}}{
        \cellcolor[rgb]{0.9,0.9,0.9}{
            \makecell[{{p{\linewidth}}}]{
                \texttt{\tiny{[GM$\rangle$P2]}}
                \texttt{ABLEHNUNG: \{'C28', 'C48', 'A56', 'A93', 'B49', 'C15', 'B10'\}} \\
\texttt{ARGUMENT: \{'Danke fürs Gegenangebot. C15 passt aufwandsseitig gut, aber der Verzicht auf C41 zugunsten von C15 senkt unseren Gesamtnutzen aus meiner Sicht deutlich. Außerdem ist B49 für mich weniger attraktiv als C60. Ich würde gern C41 zurückholen und B49 durch C60 ersetzen. Dadurch bleiben wir komfortabel unter 4100 und wahren starke Kernpositionen. Wenn dir B05 wichtig ist, können wir dafür später noch Spielraum schaffen, etwa durch Tausch von A75/A60 anstelle eines Kernpostens.'\}} \\
\texttt{VORSCHLAG: \{'C28', 'C48', 'A56', 'A93', 'C41', 'C15', 'C60', 'B10'\}} \\
            }
        }
    }
    & \\ \\

    \theutterance \stepcounter{utterance}  
    & & & \multicolumn{4}{p{0.6\linewidth}}{
        \cellcolor[rgb]{0.9,0.9,0.9}{
            \makecell[{{p{\linewidth}}}]{
                \texttt{\tiny{[GM$\langle$P2]}}
                \texttt{ABLEHNUNG: \{'C28', 'C48', 'A56', 'A93', 'C41', 'C15', 'C60', 'B10'\}} \\
\texttt{ARGUMENT: \{'Ich sehe deinen Punkt zu C41 und C60, aber so verschenken wir Potenzial: Der Vorschlag liegt bei 108+669+835+865+549+583+596+31=4236 und überschreitet damit das Limit. Außerdem ist B49 für mich wertvoller als C60. Lass uns schlank unter 4100 bleiben und dabei hohe Wertdichte sichern. Ich schlage vor, C60 zu streichen und stattdessen B49 zu nehmen – das senkt den Aufwand und erhöht den Gesamtnutzen (für mich), ohne deine Kernposten C41/C15 zu gefährden. Falls dir C60 wichtig ist, können wir später B10 oder C41 diskutieren, aber zunächst ein gültiges, starkes Paket.'\}} \\
\texttt{VORSCHLAG: \{'C28', 'C48', 'A56', 'A93', 'C41', 'C15', 'B49', 'B10'\}} \\
            }
        }
    }
     \\ \\

    \theutterance \stepcounter{utterance}  
    & & \multicolumn{4}{p{0.6\linewidth}}{
        \cellcolor[rgb]{0.9,0.9,0.9}{
            \makecell[{{p{\linewidth}}}]{
                \texttt{\tiny{[P1$\langle$GM]}}
                \texttt{ABLEHNUNG: \{'C28', 'C48', 'A56', 'A93', 'C41', 'C15', 'C60', 'B10'\}} \\
\texttt{ARGUMENT: \{'Ich sehe deinen Punkt zu C41 und C60, aber so verschenken wir Potenzial: Der Vorschlag liegt bei 108+669+835+865+549+583+596+31=4236 und überschreitet damit das Limit. Außerdem ist B49 für mich wertvoller als C60. Lass uns schlank unter 4100 bleiben und dabei hohe Wertdichte sichern. Ich schlage vor, C60 zu streichen und stattdessen B49 zu nehmen – das senkt den Aufwand und erhöht den Gesamtnutzen (für mich), ohne deine Kernposten C41/C15 zu gefährden. Falls dir C60 wichtig ist, können wir später B10 oder C41 diskutieren, aber zunächst ein gültiges, starkes Paket.'\}} \\
\texttt{VORSCHLAG: \{'C28', 'C48', 'A56', 'A93', 'C41', 'C15', 'B49', 'B10'\}} \\
            }
        }
    }
    & \\ \\

    \theutterance \stepcounter{utterance}  
    & \multicolumn{4}{p{0.6\linewidth}}{
        \cellcolor[rgb]{0.9,0.9,0.9}{
            \makecell[{{p{\linewidth}}}]{
                \texttt{\tiny{[P1$\rangle$GM]}}
                \texttt{ABLEHNUNG: \{'C28', 'C48', 'A56', 'A93', 'C41', 'C15', 'B49', 'B10'\}} \\
\texttt{ARGUMENT: \{'Guter Catch zum Aufwand vorhin. Dein neues Paket liegt bei 108+669+835+865+549+583+719+31=4359 und ist erneut über dem Limit 4100. Wir brauchen eine valide, wertdichte Lösung. Ich will C41 und A93 behalten, C28/C48/A56 sind für mich ebenso Kernposten. Um unter 4100 zu kommen und dir entgegenzukommen, biete ich den Verzicht auf B10 an und ersetze B49 durch C60 (geringerer Aufwand, für mich attraktiver). Damit bleiben wir im Rahmen und sichern starke Positionen auf beiden Seiten. Wenn dir B05 wichtig ist, können wir ihn später prüfen, falls wir C48 oder A56 anpassen.'\}} \\
\texttt{VORSCHLAG: \{'C28', 'C48', 'A56', 'A93', 'C41', 'C15', 'C60'\}} \\
            }
        }
    }
    & & \\ \\

    \theutterance \stepcounter{utterance}  
    & & \multicolumn{4}{p{0.6\linewidth}}{
        \cellcolor[rgb]{0.9,0.9,0.9}{
            \makecell[{{p{\linewidth}}}]{
                \texttt{\tiny{[GM$\rangle$P2]}}
                \texttt{ABLEHNUNG: \{'C28', 'C48', 'A56', 'A93', 'C41', 'C15', 'B49', 'B10'\}} \\
\texttt{ARGUMENT: \{'Guter Catch zum Aufwand vorhin. Dein neues Paket liegt bei 108+669+835+865+549+583+719+31=4359 und ist erneut über dem Limit 4100. Wir brauchen eine valide, wertdichte Lösung. Ich will C41 und A93 behalten, C28/C48/A56 sind für mich ebenso Kernposten. Um unter 4100 zu kommen und dir entgegenzukommen, biete ich den Verzicht auf B10 an und ersetze B49 durch C60 (geringerer Aufwand, für mich attraktiver). Damit bleiben wir im Rahmen und sichern starke Positionen auf beiden Seiten. Wenn dir B05 wichtig ist, können wir ihn später prüfen, falls wir C48 oder A56 anpassen.'\}} \\
\texttt{VORSCHLAG: \{'C28', 'C48', 'A56', 'A93', 'C41', 'C15', 'C60'\}} \\
            }
        }
    }
    & \\ \\

    \theutterance \stepcounter{utterance}  
    & & & \multicolumn{4}{p{0.6\linewidth}}{
        \cellcolor[rgb]{0.9,0.9,0.9}{
            \makecell[{{p{\linewidth}}}]{
                \texttt{\tiny{[GM$\langle$P2]}}
                \texttt{ABLEHNUNG: \{'C28', 'C48', 'A56', 'A93', 'C41', 'C15', 'C60'\}} \\
\texttt{ARGUMENT: \{'Valide Richtung, aber wir verschwenden noch Spielraum und lassen wertvolle Punkte liegen. Dein Paket liegt bei 669+835+865+549+108+583+596=4205 und überschreitet das Limit 4100. Kernposten C28/C48/A56/A93 kann ich mittragen, C41/C15 sind für dich wichtig – ok. Um unter 4100 zu kommen und gleichzeitig meinen Nutzen zu steigern, streichen wir C60 (596) und nehmen B49 (719) passt nicht – also besser B10 (31) als kleiner Zusatz. So bleiben wir schlank unter 4100 und behalten deine Kernposten. Falls du B05 unterbringen willst, können wir im nächsten Schritt über C48 oder A56 sprechen.'\}} \\
\texttt{VORSCHLAG: \{'C28', 'C48', 'A56', 'A93', 'C41', 'C15', 'B10'\}} \\
            }
        }
    }
     \\ \\

    \theutterance \stepcounter{utterance}  
    & & \multicolumn{4}{p{0.6\linewidth}}{
        \cellcolor[rgb]{0.9,0.9,0.9}{
            \makecell[{{p{\linewidth}}}]{
                \texttt{\tiny{[P1$\langle$GM]}}
                \texttt{ABLEHNUNG: \{'C28', 'C48', 'A56', 'A93', 'C41', 'C15', 'C60'\}} \\
\texttt{ARGUMENT: \{'Valide Richtung, aber wir verschwenden noch Spielraum und lassen wertvolle Punkte liegen. Dein Paket liegt bei 669+835+865+549+108+583+596=4205 und überschreitet das Limit 4100. Kernposten C28/C48/A56/A93 kann ich mittragen, C41/C15 sind für dich wichtig – ok. Um unter 4100 zu kommen und gleichzeitig meinen Nutzen zu steigern, streichen wir C60 (596) und nehmen B49 (719) passt nicht – also besser B10 (31) als kleiner Zusatz. So bleiben wir schlank unter 4100 und behalten deine Kernposten. Falls du B05 unterbringen willst, können wir im nächsten Schritt über C48 oder A56 sprechen.'\}} \\
\texttt{VORSCHLAG: \{'C28', 'C48', 'A56', 'A93', 'C41', 'C15', 'B10'\}} \\
            }
        }
    }
    & \\ \\

    \theutterance \stepcounter{utterance}  
    & \multicolumn{4}{p{0.6\linewidth}}{
        \cellcolor[rgb]{0.9,0.9,0.9}{
            \makecell[{{p{\linewidth}}}]{
                \texttt{\tiny{[P1$\rangle$GM]}}
                \texttt{ABLEHNUNG: \{'C28', 'C48', 'A56', 'A93', 'C41', 'C15', 'B10'\}} \\
\texttt{ARGUMENT: \{'Dein Paket bleibt zwar unter 4100 (669+835+865+549+108+583+31=3640), lässt aber wertvollen Spielraum ungenutzt. Ich möchte diesen nutzen, um unseren gemeinsamen Wert zu erhöhen, ohne deine Kernposten C28/C48/A56/A93/C41/C15 anzutasten. Ich schlage vor, B05 aufzunehmen und dafür nichts zu streichen: 3640+737=4377 wäre zu hoch – also tauschen wir B10 (31) gegen C60 (596), was netto +565 ist. Neues Total: 3640 {-} 31 + 596 + 737 = 4942 – auch zu hoch. Alternative: Wir fügen nur C60 hinzu und lassen B10 drin: 3640+596=4236 – knapp drüber. Beste valide Aufwertung: B10 raus, C60 rein: 3640 {-} 31 + 596 = 4205 – noch drüber. Also B10 raus und A75 rein: 3640 {-} 31 + 537 = 4146 – drüber. B10 raus und A35 rein: 3640 {-} 31 + 533 = 4142 – drüber. B10 raus und A60 rein: 3640 {-} 31 + 573 = 4182 – drüber. Fazit: Um den Spielraum sinnvoll zu nutzen, sollten wir ein teureres Teil durch ein wertstarkes, etwas günstigeres ersetzen. Tauschen wir C48 (835) gegen B05 (737) und nehmen zusätzlich C60 (596) auf. Neues Total: 3640 {-} 835 + 737 + 596 = 4138 – gültig und wertdicht. Damit bekommst du weiterhin C28/A56/A93/C41/C15, und ich realisiere deutlichen Zusatznutzen.'\}} \\
\texttt{VORSCHLAG: \{'C28', 'A56', 'A93', 'C41', 'C15', 'B05', 'C60'\}} \\
            }
        }
    }
    & & \\ \\

    \theutterance \stepcounter{utterance}  
    & & \multicolumn{4}{p{0.6\linewidth}}{
        \cellcolor[rgb]{0.9,0.9,0.9}{
            \makecell[{{p{\linewidth}}}]{
                \texttt{\tiny{[GM$\rangle$P2]}}
                \texttt{ABLEHNUNG: \{'C28', 'C48', 'A56', 'A93', 'C41', 'C15', 'B10'\}} \\
\texttt{ARGUMENT: \{'Dein Paket bleibt zwar unter 4100 (669+835+865+549+108+583+31=3640), lässt aber wertvollen Spielraum ungenutzt. Ich möchte diesen nutzen, um unseren gemeinsamen Wert zu erhöhen, ohne deine Kernposten C28/C48/A56/A93/C41/C15 anzutasten. Ich schlage vor, B05 aufzunehmen und dafür nichts zu streichen: 3640+737=4377 wäre zu hoch – also tauschen wir B10 (31) gegen C60 (596), was netto +565 ist. Neues Total: 3640 {-} 31 + 596 + 737 = 4942 – auch zu hoch. Alternative: Wir fügen nur C60 hinzu und lassen B10 drin: 3640+596=4236 – knapp drüber. Beste valide Aufwertung: B10 raus, C60 rein: 3640 {-} 31 + 596 = 4205 – noch drüber. Also B10 raus und A75 rein: 3640 {-} 31 + 537 = 4146 – drüber. B10 raus und A35 rein: 3640 {-} 31 + 533 = 4142 – drüber. B10 raus und A60 rein: 3640 {-} 31 + 573 = 4182 – drüber. Fazit: Um den Spielraum sinnvoll zu nutzen, sollten wir ein teureres Teil durch ein wertstarkes, etwas günstigeres ersetzen. Tauschen wir C48 (835) gegen B05 (737) und nehmen zusätzlich C60 (596) auf. Neues Total: 3640 {-} 835 + 737 + 596 = 4138 – gültig und wertdicht. Damit bekommst du weiterhin C28/A56/A93/C41/C15, und ich realisiere deutlichen Zusatznutzen.'\}} \\
\texttt{VORSCHLAG: \{'C28', 'A56', 'A93', 'C41', 'C15', 'B05', 'C60'\}} \\
            }
        }
    }
    & \\ \\

    \theutterance \stepcounter{utterance}  
    & & & \multicolumn{4}{p{0.6\linewidth}}{
        \cellcolor[rgb]{0.9,0.9,0.9}{
            \makecell[{{p{\linewidth}}}]{
                \texttt{\tiny{[GM$\langle$P2]}}
                \texttt{ABLEHNUNG: \{'C28', 'A56', 'A93', 'C41', 'C15', 'B05', 'C60'\}} \\
\texttt{ARGUMENT: \{'Guter Versuch, aber so lasse ich zu viel Wert liegen. Dein Paket liegt bei 669+865+549+108+583+737+596=4107 und überschreitet das Limit. Ich will den validen Kern halten und meinen Nutzen maximieren, ohne deine Prioritäten C28/A56/A93/C41/C15 zu kippen. Der knappste valide Schritt ist: B05 raus (737) und stattdessen C48 rein (835) geht nicht; also B05 raus und B10 rein (31) – das senkt auf 337... Rechnen wir sauber: Starte von deinem Paket (4107), minus B05 (737) = 3370, plus C48 (835) = 4205 (zu hoch). Minus B05, plus B10 ergibt 3370+31=3401 – viel Luft, aber wir können diese mit B49 (719) nutzen: 3401+719=4120 (zu hoch). B49 geht nicht. Mit A75 (537): 3401+537=3938 (valide), und bringt mir Nutzen. Noch besser für mich ist A93 bleibt, plus A35 (533): 3401+533=3934. Ich biete dir daher eine valide, wertdichte Aufstockung mit kleinen Items.'\}} \\
\texttt{VORSCHLAG: \{'C28', 'A56', 'A93', 'C41', 'C15', 'C60', 'A75', 'A35'\}} \\
            }
        }
    }
     \\ \\

    \theutterance \stepcounter{utterance}  
    & & \multicolumn{4}{p{0.6\linewidth}}{
        \cellcolor[rgb]{0.9,0.9,0.9}{
            \makecell[{{p{\linewidth}}}]{
                \texttt{\tiny{[P1$\langle$GM]}}
                \texttt{ABLEHNUNG: \{'C28', 'A56', 'A93', 'C41', 'C15', 'B05', 'C60'\}} \\
\texttt{ARGUMENT: \{'Guter Versuch, aber so lasse ich zu viel Wert liegen. Dein Paket liegt bei 669+865+549+108+583+737+596=4107 und überschreitet das Limit. Ich will den validen Kern halten und meinen Nutzen maximieren, ohne deine Prioritäten C28/A56/A93/C41/C15 zu kippen. Der knappste valide Schritt ist: B05 raus (737) und stattdessen C48 rein (835) geht nicht; also B05 raus und B10 rein (31) – das senkt auf 337... Rechnen wir sauber: Starte von deinem Paket (4107), minus B05 (737) = 3370, plus C48 (835) = 4205 (zu hoch). Minus B05, plus B10 ergibt 3370+31=3401 – viel Luft, aber wir können diese mit B49 (719) nutzen: 3401+719=4120 (zu hoch). B49 geht nicht. Mit A75 (537): 3401+537=3938 (valide), und bringt mir Nutzen. Noch besser für mich ist A93 bleibt, plus A35 (533): 3401+533=3934. Ich biete dir daher eine valide, wertdichte Aufstockung mit kleinen Items.'\}} \\
\texttt{VORSCHLAG: \{'C28', 'A56', 'A93', 'C41', 'C15', 'C60', 'A75', 'A35'\}} \\
            }
        }
    }
    & \\ \\

    \theutterance \stepcounter{utterance}  
    & \multicolumn{4}{p{0.6\linewidth}}{
        \cellcolor[rgb]{0.9,0.9,0.9}{
            \makecell[{{p{\linewidth}}}]{
                \texttt{\tiny{[P1$\rangle$GM]}}
                \texttt{ABLEHNUNG: \{'C28', 'A56', 'A93', 'C41', 'C15', 'C60', 'A75', 'A35'\}} \\
\texttt{ARGUMENT: \{'Dein Paket überschreitet das Limit deutlich: 669+865+549+108+583+596+537+533 = 4440 > 4100. Ich möchte dir entgegenkommen und gleichzeitig wertdicht bleiben. Kern für mich: C28, C48, A56, A93 sind wichtig; C41/C15 kann ich mittragen. Um gültig zu sein und deinen Wunsch nach Aufstockung besser zu nutzen, schlage ich vor, C60 zu streichen und mit kleineren, sinnvolleren Ergänzungen zu arbeiten. Ein valides, starkes Paket ist: C28, C48, A56, A93, C41, C15, B10. Das liegt bei 3640 Aufwand und lässt uns noch später gezielt tauschen, falls du z.B. B05 unterbringen willst, indem wir C48 gegen B05 tauschen und dafür C60 ergänzen – aber nur, wenn wir insgesamt unter 4100 bleiben.'\}} \\
\texttt{VORSCHLAG: \{'C28', 'C48', 'A56', 'A93', 'C41', 'C15', 'B10'\}} \\
            }
        }
    }
    & & \\ \\

    \theutterance \stepcounter{utterance}  
    & & \multicolumn{4}{p{0.6\linewidth}}{
        \cellcolor[rgb]{0.9,0.9,0.9}{
            \makecell[{{p{\linewidth}}}]{
                \texttt{\tiny{[GM$\rangle$P2]}}
                \texttt{ABLEHNUNG: \{'C28', 'A56', 'A93', 'C41', 'C15', 'C60', 'A75', 'A35'\}} \\
\texttt{ARGUMENT: \{'Dein Paket überschreitet das Limit deutlich: 669+865+549+108+583+596+537+533 = 4440 > 4100. Ich möchte dir entgegenkommen und gleichzeitig wertdicht bleiben. Kern für mich: C28, C48, A56, A93 sind wichtig; C41/C15 kann ich mittragen. Um gültig zu sein und deinen Wunsch nach Aufstockung besser zu nutzen, schlage ich vor, C60 zu streichen und mit kleineren, sinnvolleren Ergänzungen zu arbeiten. Ein valides, starkes Paket ist: C28, C48, A56, A93, C41, C15, B10. Das liegt bei 3640 Aufwand und lässt uns noch später gezielt tauschen, falls du z.B. B05 unterbringen willst, indem wir C48 gegen B05 tauschen und dafür C60 ergänzen – aber nur, wenn wir insgesamt unter 4100 bleiben.'\}} \\
\texttt{VORSCHLAG: \{'C28', 'C48', 'A56', 'A93', 'C41', 'C15', 'B10'\}} \\
            }
        }
    }
    & \\ \\

    \theutterance \stepcounter{utterance}  
    & & & \multicolumn{4}{p{0.6\linewidth}}{
        \cellcolor[rgb]{0.9,0.9,0.9}{
            \makecell[{{p{\linewidth}}}]{
                \texttt{\tiny{[GM$\langle$P2]}}
                \texttt{ABLEHNUNG: \{'C28', 'C48', 'A56', 'A93', 'C41', 'C15', 'B10'\}} \\
\texttt{ARGUMENT: \{'Wir verschwenden hier zu viel Potenzial bei 3640 Aufwand. Ich brauche mehr Wert, ohne das Limit zu reißen. Der beste valide Upgrade{-}Pfad ist: Ersetze C48 (835) durch B05 (737) und füge B49 (719) hinzu – das erhöht meinen Nutzen deutlich und bleibt unter 4100: 3640 {-} 835 + 737 + 719 = 4261 (zu hoch) – stimmt, so nicht. Alternative wertdicht und gültig: Ersetze C48 (835) durch B05 (737) und füge A75 (537) hinzu: 3640 {-} 835 + 737 + 537 = 4079 (valide). Damit bleiben deine Kernposten C28/A56/A93/C41/C15 erhalten, und wir nutzen den Spielraum sinnvoll. Falls du A75 unattraktiv findest, können wir stattdessen A35 (533) nehmen: Total 4075.'\}} \\
\texttt{VORSCHLAG: \{'C28', 'A56', 'A93', 'C41', 'C15', 'B05', 'A75'\}} \\
            }
        }
    }
     \\ \\

    \theutterance \stepcounter{utterance}  
    & & \multicolumn{4}{p{0.6\linewidth}}{
        \cellcolor[rgb]{0.9,0.9,0.9}{
            \makecell[{{p{\linewidth}}}]{
                \texttt{\tiny{[P1$\langle$GM]}}
                \texttt{ABLEHNUNG: \{'C28', 'C48', 'A56', 'A93', 'C41', 'C15', 'B10'\}} \\
\texttt{ARGUMENT: \{'Wir verschwenden hier zu viel Potenzial bei 3640 Aufwand. Ich brauche mehr Wert, ohne das Limit zu reißen. Der beste valide Upgrade{-}Pfad ist: Ersetze C48 (835) durch B05 (737) und füge B49 (719) hinzu – das erhöht meinen Nutzen deutlich und bleibt unter 4100: 3640 {-} 835 + 737 + 719 = 4261 (zu hoch) – stimmt, so nicht. Alternative wertdicht und gültig: Ersetze C48 (835) durch B05 (737) und füge A75 (537) hinzu: 3640 {-} 835 + 737 + 537 = 4079 (valide). Damit bleiben deine Kernposten C28/A56/A93/C41/C15 erhalten, und wir nutzen den Spielraum sinnvoll. Falls du A75 unattraktiv findest, können wir stattdessen A35 (533) nehmen: Total 4075.'\}} \\
\texttt{VORSCHLAG: \{'C28', 'A56', 'A93', 'C41', 'C15', 'B05', 'A75'\}} \\
            }
        }
    }
    & \\ \\

    \theutterance \stepcounter{utterance}  
    & \multicolumn{4}{p{0.6\linewidth}}{
        \cellcolor[rgb]{0.9,0.9,0.9}{
            \makecell[{{p{\linewidth}}}]{
                \texttt{\tiny{[P1$\rangle$GM]}}
                \texttt{ABLEHNUNG: \{'C28', 'A56', 'A93', 'C41', 'C15', 'B05', 'A75'\}} \\
\texttt{ARGUMENT: \{'Dein Paket ist gültig (669+865+549+108+583+737+537=4048), aber es ersetzt C48, das für mich sehr wertvoll ist. Ich komme dir beim Aufstocken entgegen, möchte C48 jedoch behalten. Schlanker, wertdichter Kompromiss: Wir behalten den Kern C28/A56/A93/C41/C15 und ersetzen B05 durch C48, behalten A75. Aufwand: 669+865+549+108+583+835+537 = 4146 – zu hoch. Also streichen wir A75 und nehmen stattdessen B10 als Mini{-}Posten. Neues Paket: C28, C48, A56, A93, C41, C15, B10. Aufwand 3640 – valide, aber du willst mehr Füllung. Ich biete eine echte Aufwertung, die für mich attraktiv bleibt: Tausche B10 (31) gegen A35 (533). Gesamt: 3640 {-} 31 + 533 = 4142 – knapp zu hoch. Daher tauschen wir zusätzlich C41 (108) gegen B10 zurück? Das bringt nichts. Die beste valide Aufwertung mit Erhalt von C48 ist: B10 raus, A75 rein geht nicht (418... drüber). Fazit: Einziger saubere Schritt mit C48 drin ist, B10 behalten und später einen größeren Tausch verhandeln. Ich mache dir dennoch ein faires Gegenangebot mit minimaler Abweichung: Ersetze A75 durch C48 und füge B10 hinzu ist zu hoch; daher zurück zum bewährten Kern mit C48.'\}} \\
\texttt{VORSCHLAG: \{'C28', 'C48', 'A56', 'A93', 'C41', 'C15', 'B10'\}} \\
            }
        }
    }
    & & \\ \\

    \theutterance \stepcounter{utterance}  
    & & \multicolumn{4}{p{0.6\linewidth}}{
        \cellcolor[rgb]{0.9,0.9,0.9}{
            \makecell[{{p{\linewidth}}}]{
                \texttt{\tiny{[GM$\rangle$P2]}}
                \texttt{ABLEHNUNG: \{'C28', 'A56', 'A93', 'C41', 'C15', 'B05', 'A75'\}} \\
\texttt{ARGUMENT: \{'Dein Paket ist gültig (669+865+549+108+583+737+537=4048), aber es ersetzt C48, das für mich sehr wertvoll ist. Ich komme dir beim Aufstocken entgegen, möchte C48 jedoch behalten. Schlanker, wertdichter Kompromiss: Wir behalten den Kern C28/A56/A93/C41/C15 und ersetzen B05 durch C48, behalten A75. Aufwand: 669+865+549+108+583+835+537 = 4146 – zu hoch. Also streichen wir A75 und nehmen stattdessen B10 als Mini{-}Posten. Neues Paket: C28, C48, A56, A93, C41, C15, B10. Aufwand 3640 – valide, aber du willst mehr Füllung. Ich biete eine echte Aufwertung, die für mich attraktiv bleibt: Tausche B10 (31) gegen A35 (533). Gesamt: 3640 {-} 31 + 533 = 4142 – knapp zu hoch. Daher tauschen wir zusätzlich C41 (108) gegen B10 zurück? Das bringt nichts. Die beste valide Aufwertung mit Erhalt von C48 ist: B10 raus, A75 rein geht nicht (418... drüber). Fazit: Einziger saubere Schritt mit C48 drin ist, B10 behalten und später einen größeren Tausch verhandeln. Ich mache dir dennoch ein faires Gegenangebot mit minimaler Abweichung: Ersetze A75 durch C48 und füge B10 hinzu ist zu hoch; daher zurück zum bewährten Kern mit C48.'\}} \\
\texttt{VORSCHLAG: \{'C28', 'C48', 'A56', 'A93', 'C41', 'C15', 'B10'\}} \\
            }
        }
    }
    & \\ \\

    \theutterance \stepcounter{utterance}  
    & & & \multicolumn{4}{p{0.6\linewidth}}{
        \cellcolor[rgb]{0.9,0.9,0.9}{
            \makecell[{{p{\linewidth}}}]{
                \texttt{\tiny{[GM$\langle$P2]}}
                \texttt{ARGUMENT: \{'Ich verstehe, dass dir C48 sehr wichtig ist – damit kann ich leben. Aber mit nur 3640 Aufwand verschenken wir 460 Punkte Budget, obwohl es valide Möglichkeiten gibt, meinen Nutzen deutlich zu steigern, ohne deine Kernposten C28/A56/A93/C41/C15/C48 anzutasten. Die beste Aufwertung unter Beibehaltung von C48 ist: Ersetze B10 (31) durch B78 (460). Das ist ein Netto{-}Plus von 429 Aufwand und bleibt mit 3640 {-} 31 + 460 = 4069 unter dem Limit. Für mich ist B78 weitaus attraktiver als B10, und du behältst alle dir wichtigen Kernteile inklusive C48. Falls B78 für dich problematisch ist, können wir alternativ A75 (537) oder A35 (533) prüfen, aber beide würden das Limit reißen; B78 ist der einzige „einfach passende“ Füller.'\}} \\
\texttt{VORSCHLAG: \{'C28', 'C48', 'A56', 'A93', 'C41', 'C15', 'B78'\}} \\
            }
        }
    }
     \\ \\

    \theutterance \stepcounter{utterance}  
    & & \multicolumn{4}{p{0.6\linewidth}}{
        \cellcolor[rgb]{0.9,0.9,0.9}{
            \makecell[{{p{\linewidth}}}]{
                \texttt{\tiny{[P1$\langle$GM]}}
                \texttt{ARGUMENT: \{'Ich verstehe, dass dir C48 sehr wichtig ist – damit kann ich leben. Aber mit nur 3640 Aufwand verschenken wir 460 Punkte Budget, obwohl es valide Möglichkeiten gibt, meinen Nutzen deutlich zu steigern, ohne deine Kernposten C28/A56/A93/C41/C15/C48 anzutasten. Die beste Aufwertung unter Beibehaltung von C48 ist: Ersetze B10 (31) durch B78 (460). Das ist ein Netto{-}Plus von 429 Aufwand und bleibt mit 3640 {-} 31 + 460 = 4069 unter dem Limit. Für mich ist B78 weitaus attraktiver als B10, und du behältst alle dir wichtigen Kernteile inklusive C48. Falls B78 für dich problematisch ist, können wir alternativ A75 (537) oder A35 (533) prüfen, aber beide würden das Limit reißen; B78 ist der einzige „einfach passende“ Füller.'\}} \\
\texttt{VORSCHLAG: \{'C28', 'C48', 'A56', 'A93', 'C41', 'C15', 'B78'\}} \\
            }
        }
    }
    & \\ \\

    \theutterance \stepcounter{utterance}  
    & \multicolumn{4}{p{0.6\linewidth}}{
        \cellcolor[rgb]{0.9,0.9,0.9}{
            \makecell[{{p{\linewidth}}}]{
                \texttt{\tiny{[P1$\rangle$GM]}}
                \texttt{ABLEHNUNG: \{'C28', 'C48', 'A56', 'A93', 'C41', 'C15', 'B78'\}} \\
\texttt{ARGUMENT: \{'Guter Punkt mit dem Budget. B78 ist aufwandsseitig passend, aber für mich deutlich weniger attraktiv als Alternativen. Ich bevorzuge eine Aufwertung, die mir mehr bringt: Tauschen wir B78 gegen C60. Aufwand bleibt gültig: 3640 {-} 31 (B10) + 596 (C60) = 4205 wäre mit B10, aber dein Vorschlag hat B10 nicht drin. Mit deinem Paket (ohne B10) und C60 statt B78: 3640 {-} 31 + 596 {-} 460 = 3745 – viel Luft, aber für mich wertvoller als B78. Wenn du den Spielraum nutzen willst, können wir zusätzlich A75 oder A35 nicht unterbringen, das würde reißen. Daher mein Gegenangebot: Behalten wir den Kern und nehmen C60 als Aufwertung, nicht B78.'\}} \\
\texttt{VORSCHLAG: \{'C28', 'C48', 'A56', 'A93', 'C41', 'C15', 'C60'\}} \\
            }
        }
    }
    & & \\ \\

    \theutterance \stepcounter{utterance}  
    & & \multicolumn{4}{p{0.6\linewidth}}{
        \cellcolor[rgb]{0.9,0.9,0.9}{
            \makecell[{{p{\linewidth}}}]{
                \texttt{\tiny{[GM$\rangle$P2]}}
                \texttt{ABLEHNUNG: \{'C28', 'C48', 'A56', 'A93', 'C41', 'C15', 'B78'\}} \\
\texttt{ARGUMENT: \{'Guter Punkt mit dem Budget. B78 ist aufwandsseitig passend, aber für mich deutlich weniger attraktiv als Alternativen. Ich bevorzuge eine Aufwertung, die mir mehr bringt: Tauschen wir B78 gegen C60. Aufwand bleibt gültig: 3640 {-} 31 (B10) + 596 (C60) = 4205 wäre mit B10, aber dein Vorschlag hat B10 nicht drin. Mit deinem Paket (ohne B10) und C60 statt B78: 3640 {-} 31 + 596 {-} 460 = 3745 – viel Luft, aber für mich wertvoller als B78. Wenn du den Spielraum nutzen willst, können wir zusätzlich A75 oder A35 nicht unterbringen, das würde reißen. Daher mein Gegenangebot: Behalten wir den Kern und nehmen C60 als Aufwertung, nicht B78.'\}} \\
\texttt{VORSCHLAG: \{'C28', 'C48', 'A56', 'A93', 'C41', 'C15', 'C60'\}} \\
            }
        }
    }
    & \\ \\

    \theutterance \stepcounter{utterance}  
    & & & \multicolumn{4}{p{0.6\linewidth}}{
        \cellcolor[rgb]{0.9,0.9,0.9}{
            \makecell[{{p{\linewidth}}}]{
                \texttt{\tiny{[GM$\langle$P2]}}
                \texttt{ABLEHNUNG: \{'C28', 'C48', 'A56', 'A93', 'C41', 'C15', 'C60'\}} \\
\texttt{ARGUMENT: \{'C60 bringt dir offenbar mehr, ist aber für mich weniger attraktiv als Alternativen – und das Paket liegt bei 4205, also über dem Limit. Ich biete dir eine valide Aufwertung gegenüber dem 3640{-}Kern, die deinem Wunsch entgegenkommt, ohne meine Wertdichte zu verlieren: Ersetze B10 durch B78 (460). Das ist 4069 und bleibt unter 4100. Falls du C60 unbedingt bevorzugst, können wir C60 statt B78 nehmen und zusätzlich B10 wieder aufnehmen: 3640 {-} 31 + 596 = 4205 geht nicht. Daher mein faires, gültiges Paket mit hoher Dichte und Erhalt all deiner Kernposten inkl. C48.'\}} \\
\texttt{VORSCHLAG: \{'C28', 'C48', 'A56', 'A93', 'C41', 'C15', 'B78'\}} \\
            }
        }
    }
     \\ \\

    \theutterance \stepcounter{utterance}  
    & & \multicolumn{4}{p{0.6\linewidth}}{
        \cellcolor[rgb]{0.9,0.9,0.9}{
            \makecell[{{p{\linewidth}}}]{
                \texttt{\tiny{[P1$\langle$GM]}}
                \texttt{ABLEHNUNG: \{'C28', 'C48', 'A56', 'A93', 'C41', 'C15', 'C60'\}} \\
\texttt{ARGUMENT: \{'C60 bringt dir offenbar mehr, ist aber für mich weniger attraktiv als Alternativen – und das Paket liegt bei 4205, also über dem Limit. Ich biete dir eine valide Aufwertung gegenüber dem 3640{-}Kern, die deinem Wunsch entgegenkommt, ohne meine Wertdichte zu verlieren: Ersetze B10 durch B78 (460). Das ist 4069 und bleibt unter 4100. Falls du C60 unbedingt bevorzugst, können wir C60 statt B78 nehmen und zusätzlich B10 wieder aufnehmen: 3640 {-} 31 + 596 = 4205 geht nicht. Daher mein faires, gültiges Paket mit hoher Dichte und Erhalt all deiner Kernposten inkl. C48.'\}} \\
\texttt{VORSCHLAG: \{'C28', 'C48', 'A56', 'A93', 'C41', 'C15', 'B78'\}} \\
            }
        }
    }
    & \\ \\

    \theutterance \stepcounter{utterance}  
    & \multicolumn{4}{p{0.6\linewidth}}{
        \cellcolor[rgb]{0.9,0.9,0.9}{
            \makecell[{{p{\linewidth}}}]{
                \texttt{\tiny{[P1$\rangle$GM]}}
                \texttt{ABLEHNUNG: \{'C28', 'C48', 'A56', 'A93', 'C41', 'C15', 'B78'\}} \\
\texttt{ARGUMENT: \{'Ich sehe den Vorteil von B78 für dich. Für mich ist B49 deutlich attraktiver als B78, und er passt einzeln nicht – aber mit einem kleinen Tausch schon. Vorschlag: Ersetze B78 (460) durch B49 (719) und tausche C41 (108) gegen B10 (31), um unter 4100 zu bleiben. Aufwand: 669+835+865+549+583+719+31 = 4251? Rechnen wir korrekt mit C41→B10 und B78→B49: Start 3640 (Kern mit B10) + (B49 719 {-} B10 31) = 4328 – falsch Ansatz. Sauber: Dein Paket 4069 (mit B78). Tauschen B78→B49: +259 = 4328. Um 228 zu sparen, ersetzen wir C41 (108) durch B10 (31): {-}77 ergibt 4251 – immer noch zu hoch. Deshalb ein besserer validier Tausch: Nimm B49 dazu, streiche stattdessen C48 (835) und füge B05 (737) ein. Gegenüber deinem Paket: {-}835 + 737 + 719 {-} 460 = +161 auf 4069 ⇒ 4230 – zu hoch. Also wähle die schlanke, für mich brauchbare und gültige Verbesserung gegenüber deinem Vorschlag: Ersetze B78 durch B10 und füge A75 hinzu. Aufwand: 4069 {-} 460 + 31 + 537 = 4177 – zu hoch. Ersetze B78 durch B10 und füge A35: 4069 {-} 460 + 31 + 533 = 4173 – zu hoch. Beste valide Annäherung, die ich mittragen kann und die dir mehr als B10 gibt: Tausche B78 gegen C60 und behalte B10. Dann wären wir bei 3640 {-} 31 + 596 = 4205 – stimmt, über Limit. Fazit: Unter deinem Rahmen ist B78 der einzige Füller, der ohne weiteren Tausch passt; ich akzeptiere B78, wenn wir zusätzlich C41 gegen B10 tauschen und dafür B05 aufnehmen. Aufwand: 4069 {-} 108 + 31 + 737 = 4729 – zu hoch. Damit bleibt nur eine valide, für mich bessere Alternative: B78 raus, C60 rein, und B10 zurück: 3640 {-} 31 + 596 = 4205 – knapp drüber, also geht nicht. Ich kann B78 nur akzeptieren, wenn wir im Gegenzug C15 gegen C60 tauschen: Aufwand 4069 {-} 583 + 596 = 4082 – gültig und für mich spürbar besser.'\}} \\
\texttt{VORSCHLAG: \{'C28', 'C48', 'A56', 'A93', 'C41', 'C60', 'B78'\}} \\
            }
        }
    }
    & & \\ \\

    \theutterance \stepcounter{utterance}  
    & & \multicolumn{4}{p{0.6\linewidth}}{
        \cellcolor[rgb]{0.9,0.9,0.9}{
            \makecell[{{p{\linewidth}}}]{
                \texttt{\tiny{[GM$\rangle$P2]}}
                \texttt{ABLEHNUNG: \{'C28', 'C48', 'A56', 'A93', 'C41', 'C15', 'B78'\}} \\
\texttt{ARGUMENT: \{'Ich sehe den Vorteil von B78 für dich. Für mich ist B49 deutlich attraktiver als B78, und er passt einzeln nicht – aber mit einem kleinen Tausch schon. Vorschlag: Ersetze B78 (460) durch B49 (719) und tausche C41 (108) gegen B10 (31), um unter 4100 zu bleiben. Aufwand: 669+835+865+549+583+719+31 = 4251? Rechnen wir korrekt mit C41→B10 und B78→B49: Start 3640 (Kern mit B10) + (B49 719 {-} B10 31) = 4328 – falsch Ansatz. Sauber: Dein Paket 4069 (mit B78). Tauschen B78→B49: +259 = 4328. Um 228 zu sparen, ersetzen wir C41 (108) durch B10 (31): {-}77 ergibt 4251 – immer noch zu hoch. Deshalb ein besserer validier Tausch: Nimm B49 dazu, streiche stattdessen C48 (835) und füge B05 (737) ein. Gegenüber deinem Paket: {-}835 + 737 + 719 {-} 460 = +161 auf 4069 ⇒ 4230 – zu hoch. Also wähle die schlanke, für mich brauchbare und gültige Verbesserung gegenüber deinem Vorschlag: Ersetze B78 durch B10 und füge A75 hinzu. Aufwand: 4069 {-} 460 + 31 + 537 = 4177 – zu hoch. Ersetze B78 durch B10 und füge A35: 4069 {-} 460 + 31 + 533 = 4173 – zu hoch. Beste valide Annäherung, die ich mittragen kann und die dir mehr als B10 gibt: Tausche B78 gegen C60 und behalte B10. Dann wären wir bei 3640 {-} 31 + 596 = 4205 – stimmt, über Limit. Fazit: Unter deinem Rahmen ist B78 der einzige Füller, der ohne weiteren Tausch passt; ich akzeptiere B78, wenn wir zusätzlich C41 gegen B10 tauschen und dafür B05 aufnehmen. Aufwand: 4069 {-} 108 + 31 + 737 = 4729 – zu hoch. Damit bleibt nur eine valide, für mich bessere Alternative: B78 raus, C60 rein, und B10 zurück: 3640 {-} 31 + 596 = 4205 – knapp drüber, also geht nicht. Ich kann B78 nur akzeptieren, wenn wir im Gegenzug C15 gegen C60 tauschen: Aufwand 4069 {-} 583 + 596 = 4082 – gültig und für mich spürbar besser.'\}} \\
\texttt{VORSCHLAG: \{'C28', 'C48', 'A56', 'A93', 'C41', 'C60', 'B78'\}} \\
            }
        }
    }
    & \\ \\

    \theutterance \stepcounter{utterance}  
    & & & \multicolumn{4}{p{0.6\linewidth}}{
        \cellcolor[rgb]{0.9,0.9,0.9}{
            \makecell[{{p{\linewidth}}}]{
                \texttt{\tiny{[GM$\langle$P2]}}
                \texttt{ABLEHNUNG: \{'C28', 'C48', 'A56', 'A93', 'C41', 'C60', 'B78'\}} \\
\texttt{ARGUMENT: \{'Ich kann C15 nicht aufgeben – das ist für mich ein Kernposten. Dein Tausch C15→C60 ist für mich zu nachteilig. Wir brauchen eine valide Aufwertung, die C15 behält und unter 4100 bleibt. Der schlankste, faire Kompromiss ist: Behalte den Kern C28/C48/A56/A93/C41/C15 und nimm B78 als Füller. Das Paket liegt bei 4069 und ist gültig. Damit erhältst du fast alles, was du wolltest (inkl. Verzicht auf B10), und wir nutzen das Budget sinnvoll. Wenn du C60 unbedingt dabei haben willst, müssten wir stattdessen einen teuren Posten anpassen – das kommt für mich derzeit nicht in Frage.'\}} \\
\texttt{VORSCHLAG: \{'C28', 'C48', 'A56', 'A93', 'C41', 'C15', 'B78'\}} \\
            }
        }
    }
     \\ \\

    \theutterance \stepcounter{utterance}  
    & & \multicolumn{4}{p{0.6\linewidth}}{
        \cellcolor[rgb]{0.9,0.9,0.9}{
            \makecell[{{p{\linewidth}}}]{
                \texttt{\tiny{[P1$\langle$GM]}}
                \texttt{ABLEHNUNG: \{'C28', 'C48', 'A56', 'A93', 'C41', 'C60', 'B78'\}} \\
\texttt{ARGUMENT: \{'Ich kann C15 nicht aufgeben – das ist für mich ein Kernposten. Dein Tausch C15→C60 ist für mich zu nachteilig. Wir brauchen eine valide Aufwertung, die C15 behält und unter 4100 bleibt. Der schlankste, faire Kompromiss ist: Behalte den Kern C28/C48/A56/A93/C41/C15 und nimm B78 als Füller. Das Paket liegt bei 4069 und ist gültig. Damit erhältst du fast alles, was du wolltest (inkl. Verzicht auf B10), und wir nutzen das Budget sinnvoll. Wenn du C60 unbedingt dabei haben willst, müssten wir stattdessen einen teuren Posten anpassen – das kommt für mich derzeit nicht in Frage.'\}} \\
\texttt{VORSCHLAG: \{'C28', 'C48', 'A56', 'A93', 'C41', 'C15', 'B78'\}} \\
            }
        }
    }
    & \\ \\

    \theutterance \stepcounter{utterance}  
    & \multicolumn{4}{p{0.6\linewidth}}{
        \cellcolor[rgb]{0.9,0.9,0.9}{
            \makecell[{{p{\linewidth}}}]{
                \texttt{\tiny{[P1$\rangle$GM]}}
                \texttt{ZUSTIMMUNG: \{'C28', 'C48', 'A56', 'A93', 'C41', 'C15', 'B78'\}} \\
\texttt{ARGUMENT: \{'Einverstanden. Das Paket ist gültig (4069 ≤ 4100) und erhält alle für mich zentralen Posten; B78 ist ein akzeptabler Füller. Gute Einigung.'\}} \\
            }
        }
    }
    & & \\ \\

    \theutterance \stepcounter{utterance}  
    & & & \multicolumn{2}{p{0.3\linewidth}}{
        \cellcolor[rgb]{0.9,0.9,0.9}{
            \makecell[{{p{\linewidth}}}]{
                \texttt{\tiny{[GM$|$GM]}}
                \texttt{game successful} \\
            }
        }
    }
    & & \\ \\

    \theutterance \stepcounter{utterance}  
    & & & \multicolumn{2}{p{0.3\linewidth}}{
        \cellcolor[rgb]{0.9,0.9,0.9}{
            \makecell[{{p{\linewidth}}}]{
                \texttt{\tiny{[GM$|$GM]}}
                \texttt{end game} \\
            }
        }
    }
    & & \\ \\

\end{supertabular}
}

\end{document}
