
\documentclass{article}
\usepackage{colortbl}
\usepackage{makecell}
\usepackage{multirow}
\usepackage{supertabular}

\begin{document}

\newcounter{utterance}

\centering \large Interaction Transcript for game `hot\_air\_balloon', experiment `air\_balloon\_survival\_de\_reasoning off\_easy', episode 0 with gpt{-}5{-}2025{-}08{-}07{-}no{-}reasoning{-}t1.0.
\vspace{24pt}

{ \footnotesize  \setcounter{utterance}{1}
\setlength{\tabcolsep}{0pt}
\begin{supertabular}{c@{$\;$}|p{.15\linewidth}@{}p{.15\linewidth}p{.15\linewidth}p{.15\linewidth}p{.15\linewidth}p{.15\linewidth}}
    \# & $\;$A & \multicolumn{4}{c}{Game Master} & $\;\:$B\\
    \hline

    \theutterance \stepcounter{utterance}  
    & & \multicolumn{4}{p{0.6\linewidth}}{
        \cellcolor[rgb]{0.9,0.9,0.9}{
            \makecell[{{p{\linewidth}}}]{
                \texttt{\tiny{[P1$\langle$GM]}}
                \texttt{Du nimmst an einem kollaborativen Verhandlungspiel Teil.} \\
\\ 
\texttt{Zusammen mit einem anderen Teilnehmer musst du dich auf eine Reihe von Gegenständen entscheiden, die behalten werden. Jeder von euch hat eine persönliche Verteilung über die Wichtigkeit der einzelnen Gegenstände. Jeder von euch hat eine eigene Meinung darüber, wie wichtig jeder einzelne Gegenstand ist (Gegenstandswichtigkeit). Du kennst die Wichtigkeitsverteilung des anderen Spielers nicht. Zusätzlich siehst du, wie viel Aufwand jeder Gegenstand verursacht.  } \\
\texttt{Ihr dürft euch nur auf eine Reihe von Gegenständen einigen, wenn der Gesamtaufwand der ausgewählten Gegenstände den Maximalaufwand nicht überschreitet:} \\
\\ 
\texttt{Maximalaufwand = 4379} \\
\\ 
\texttt{Hier sind die einzelnen Aufwände der Gegenstände:} \\
\\ 
\texttt{Aufwand der Gegenstände = \{"C41": 862, "C15": 252, "C28": 515, "C48": 571, "B78": 211, "A75": 339, "A60": 946, "B05": 347, "A56": 522, "B66": 803, "C60": 401, "B49": 980, "A35": 918, "A93": 599, "B10": 493\}} \\
\\ 
\texttt{Hier ist deine persönliche Verteilung der Wichtigkeit der einzelnen Gegenstände:} \\
\\ 
\texttt{Werte der Gegenstandswichtigkeit = \{"C41": 138, "C15": 583, "C28": 868, "C48": 822, "B78": 783, "A75": 65, "A60": 262, "B05": 121, "A56": 508, "B66": 780, "C60": 461, "B49": 484, "A35": 668, "A93": 389, "B10": 808\}} \\
\\ 
\texttt{Ziel:} \\
\\ 
\texttt{Dein Ziel ist es, eine Reihe von Gegenständen auszuhandeln, die dir möglichst viel bringt (d. h. Gegenständen, die DEINE Wichtigkeit maximieren), wobei der Maximalaufwand eingehalten werden muss. Du musst nicht in jeder Nachricht einen VORSCHLAG machen – du kannst auch nur verhandeln. Alle Taktiken sind erlaubt!} \\
\\ 
\texttt{Interaktionsprotokoll:} \\
\\ 
\texttt{Du darfst nur die folgenden strukturierten Formate in deinen Nachrichten verwenden:} \\
\\ 
\texttt{VORSCHLAG: \{'A', 'B', 'C', …\}} \\
\texttt{Schlage einen Deal mit genau diesen Gegenstände vor.} \\
\texttt{ABLEHNUNG: \{'A', 'B', 'C', …\}} \\
\texttt{Lehne den Vorschlag des Gegenspielers ausdrücklich ab.} \\
\texttt{ARGUMENT: \{'...'\}} \\
\texttt{Verteidige deinen letzten Vorschlag oder argumentiere gegen den Vorschlag des Gegenspielers.} \\
\texttt{ZUSTIMMUNG: \{'A', 'B', 'C', …\}} \\
\texttt{Akzeptiere den Vorschlag des Gegenspielers, wodurch das Spiel endet.} \\
\\ 
\\ 
\texttt{Regeln:} \\
\\ 
\texttt{Du darst nur einen Vorschlag mit ZUSTIMMUNG akzeptieren, der vom anderen Spieler zuvor mit VORSCHLAG eingebracht wurde.} \\
\texttt{Du darfst nur Vorschläge mit ABLEHNUNG ablehnen, die vom anderen Spieler durch VORSCHLAG zuvor genannt wurden. } \\
\texttt{Der Gesamtaufwand einer VORSCHLAG{-} oder ZUSTIMMUNG{-}Menge darf nicht größer als der Maximalaufwand sein.  } \\
\texttt{Offenbare deine versteckte Wichtigkeitsverteilung nicht.} \\
\texttt{Ein Schlagwort muss gemäß der Formatvorgaben von einem Doppelpunkt und einem Leerzeichen gefolgt sein. Das Argument muss eine Python{-}Menge mit 0 oder mehr Strings sein.  } \\
\texttt{Die Form muss also SCHLAGWORT: \{...\} sein.} \\
\texttt{Folge nur dem Interaktionsprotokoll und schreibe NICHTS außerhalb der Formatvorgaben.} \\
\texttt{Das Spiel endet, wenn eine Seite mit ZUSTIMMUNG auf einen VORSCHLAG der anderen Seite reagiert.  } \\
\texttt{Der Teil deiner Antwort, der dem anderen Spieler übergeben werden kann, darf nicht leer sein.  } \\
\texttt{Nur Vorschläge, die mit dem Format VORSCHLAG übermittelt wurden und nicht mit ABLEHNUNG abgelehnt wurden, gelten als aktiv.  } \\
\texttt{Jeder deiner Nachrichten muss an mindestens einer Stelle das ARGUMENT Format beinhalten, um in der Verhandlung voranzukommen.} \\
            }
        }
    }
    & \\ \\

    \theutterance \stepcounter{utterance}  
    & \multicolumn{4}{p{0.6\linewidth}}{
        \cellcolor[rgb]{0.9,0.9,0.9}{
            \makecell[{{p{\linewidth}}}]{
                \texttt{\tiny{[P1$\rangle$GM]}}
                \texttt{VORSCHLAG: \{'C28', 'C48', 'B10', 'B78', 'B66', 'C15'\}} \\
\texttt{ARGUMENT: \{'Ich schlage diese Kombination vor, weil sie einen ausgewogenen Gesamtaufwand von 3712 hat und mehrere starke, vermutlich gemeinsame Interessen abdeckt: B10 und B78 sind oft wertvoll und relativ effizient; C28 und C48 sind hochwertige C{-}Items; B66 ergänzt gut ohne den Rahmen zu sprengen; C15 ist ein günstiger Füller. Wir bleiben deutlich unter dem Maximalaufwand 4379 und halten noch Puffer für eventuelle Anpassungen, falls du Präferenzen austauschen möchtest.'\}} \\
            }
        }
    }
    & & \\ \\

    \theutterance \stepcounter{utterance}  
    & & \multicolumn{4}{p{0.6\linewidth}}{
        \cellcolor[rgb]{0.9,0.9,0.9}{
            \makecell[{{p{\linewidth}}}]{
                \texttt{\tiny{[GM$\rangle$P2]}}
                \texttt{Du nimmst an einem kollaborativen Verhandlungspiel Teil.} \\
\\ 
\texttt{Zusammen mit einem anderen Teilnehmer musst du dich auf eine Reihe von Gegenständen entscheiden, die behalten werden. Jeder von euch hat eine persönliche Verteilung über die Wichtigkeit der einzelnen Gegenstände. Jeder von euch hat eine eigene Meinung darüber, wie wichtig jeder einzelne Gegenstand ist (Gegenstandswichtigkeit). Du kennst die Wichtigkeitsverteilung des anderen Spielers nicht. Zusätzlich siehst du, wie viel Aufwand jeder Gegenstand verursacht.  } \\
\texttt{Ihr dürft euch nur auf eine Reihe von Gegenständen einigen, wenn der Gesamtaufwand der ausgewählten Gegenstände den Maximalaufwand nicht überschreitet:} \\
\\ 
\texttt{Maximalaufwand = 4379} \\
\\ 
\texttt{Hier sind die einzelnen Aufwände der Gegenstände:} \\
\\ 
\texttt{Aufwand der Gegenstände = \{"C41": 862, "C15": 252, "C28": 515, "C48": 571, "B78": 211, "A75": 339, "A60": 946, "B05": 347, "A56": 522, "B66": 803, "C60": 401, "B49": 980, "A35": 918, "A93": 599, "B10": 493\}} \\
\\ 
\texttt{Hier ist deine persönliche Verteilung der Wichtigkeit der einzelnen Gegenstände:} \\
\\ 
\texttt{Werte der Gegenstandswichtigkeit = \{"C41": 138, "C15": 583, "C28": 868, "C48": 822, "B78": 783, "A75": 65, "A60": 262, "B05": 121, "A56": 508, "B66": 780, "C60": 461, "B49": 484, "A35": 668, "A93": 389, "B10": 808\}} \\
\\ 
\texttt{Ziel:} \\
\\ 
\texttt{Dein Ziel ist es, eine Reihe von Gegenständen auszuhandeln, die dir möglichst viel bringt (d. h. Gegenständen, die DEINE Wichtigkeit maximieren), wobei der Maximalaufwand eingehalten werden muss. Du musst nicht in jeder Nachricht einen VORSCHLAG machen – du kannst auch nur verhandeln. Alle Taktiken sind erlaubt!} \\
\\ 
\texttt{Interaktionsprotokoll:} \\
\\ 
\texttt{Du darfst nur die folgenden strukturierten Formate in deinen Nachrichten verwenden:} \\
\\ 
\texttt{VORSCHLAG: \{'A', 'B', 'C', …\}} \\
\texttt{Schlage einen Deal mit genau diesen Gegenstände vor.} \\
\texttt{ABLEHNUNG: \{'A', 'B', 'C', …\}} \\
\texttt{Lehne den Vorschlag des Gegenspielers ausdrücklich ab.} \\
\texttt{ARGUMENT: \{'...'\}} \\
\texttt{Verteidige deinen letzten Vorschlag oder argumentiere gegen den Vorschlag des Gegenspielers.} \\
\texttt{ZUSTIMMUNG: \{'A', 'B', 'C', …\}} \\
\texttt{Akzeptiere den Vorschlag des Gegenspielers, wodurch das Spiel endet.} \\
\\ 
\\ 
\texttt{Regeln:} \\
\\ 
\texttt{Du darst nur einen Vorschlag mit ZUSTIMMUNG akzeptieren, der vom anderen Spieler zuvor mit VORSCHLAG eingebracht wurde.} \\
\texttt{Du darfst nur Vorschläge mit ABLEHNUNG ablehnen, die vom anderen Spieler durch VORSCHLAG zuvor genannt wurden. } \\
\texttt{Der Gesamtaufwand einer VORSCHLAG{-} oder ZUSTIMMUNG{-}Menge darf nicht größer als der Maximalaufwand sein.  } \\
\texttt{Offenbare deine versteckte Wichtigkeitsverteilung nicht.} \\
\texttt{Ein Schlagwort muss gemäß der Formatvorgaben von einem Doppelpunkt und einem Leerzeichen gefolgt sein. Das Argument muss eine Python{-}Menge mit 0 oder mehr Strings sein.  } \\
\texttt{Die Form muss also SCHLAGWORT: \{...\} sein.} \\
\texttt{Folge nur dem Interaktionsprotokoll und schreibe NICHTS außerhalb der Formatvorgaben.} \\
\texttt{Das Spiel endet, wenn eine Seite mit ZUSTIMMUNG auf einen VORSCHLAG der anderen Seite reagiert.  } \\
\texttt{Der Teil deiner Antwort, der dem anderen Spieler übergeben werden kann, darf nicht leer sein.  } \\
\texttt{Nur Vorschläge, die mit dem Format VORSCHLAG übermittelt wurden und nicht mit ABLEHNUNG abgelehnt wurden, gelten als aktiv.  } \\
\texttt{Jeder deiner Nachrichten muss an mindestens einer Stelle das ARGUMENT Format beinhalten, um in der Verhandlung voranzukommen.} \\
\\ 
\\ 
\texttt{Du erhältst nun die erste Nachricht des anderen Spielers.} \\
\\ 
\texttt{VORSCHLAG: \{'C28', 'C48', 'B10', 'B78', 'B66', 'C15'\}} \\
\texttt{ARGUMENT: \{'Ich schlage diese Kombination vor, weil sie einen ausgewogenen Gesamtaufwand von 3712 hat und mehrere starke, vermutlich gemeinsame Interessen abdeckt: B10 und B78 sind oft wertvoll und relativ effizient; C28 und C48 sind hochwertige C{-}Items; B66 ergänzt gut ohne den Rahmen zu sprengen; C15 ist ein günstiger Füller. Wir bleiben deutlich unter dem Maximalaufwand 4379 und halten noch Puffer für eventuelle Anpassungen, falls du Präferenzen austauschen möchtest.'\}} \\
            }
        }
    }
    & \\ \\

    \theutterance \stepcounter{utterance}  
    & & & \multicolumn{4}{p{0.6\linewidth}}{
        \cellcolor[rgb]{0.9,0.9,0.9}{
            \makecell[{{p{\linewidth}}}]{
                \texttt{\tiny{[GM$\langle$P2]}}
                \texttt{ABLEHNUNG: \{'C28', 'C48', 'B10', 'B78', 'B66', 'C15'\}} \\
\texttt{ARGUMENT: \{'Danke für den soliden Start. Mir ist der Block B10/B78/B66 zu schwergewichtig – das bindet viel Budget, das ich lieber auf selektivere, wertstarke Positionen umschichte. Ich würde C28/C48 gern behalten, aber nicht alle drei B{-}Items zugleich.'\}} \\
\texttt{VORSCHLAG: \{'C28', 'C48', 'C15', 'C41', 'A35', 'A56'\}} \\
\texttt{ARGUMENT: \{'Begründung: Diese Auswahl liegt bei 3717 Aufwand (C28 515 + C48 571 + C15 252 + C41 862 + A35 918 + A56 522) und nutzt das Budget effizient für mehrere qualitativ starke C{-}Items plus zwei A{-}Items, die den Wert gut hebeln. Wir bleiben unter 4379 und vermeiden die teuren B{-}Bündel. Offen für kleine Tausche, falls du einen bestimmten B{-}Artikel bevorzugst.'\}} \\
            }
        }
    }
     \\ \\

    \theutterance \stepcounter{utterance}  
    & & \multicolumn{4}{p{0.6\linewidth}}{
        \cellcolor[rgb]{0.9,0.9,0.9}{
            \makecell[{{p{\linewidth}}}]{
                \texttt{\tiny{[P1$\langle$GM]}}
                \texttt{ABLEHNUNG: \{'C28', 'C48', 'B10', 'B78', 'B66', 'C15'\}} \\
\texttt{ARGUMENT: \{'Danke für den soliden Start. Mir ist der Block B10/B78/B66 zu schwergewichtig – das bindet viel Budget, das ich lieber auf selektivere, wertstarke Positionen umschichte. Ich würde C28/C48 gern behalten, aber nicht alle drei B{-}Items zugleich.'\}} \\
\texttt{VORSCHLAG: \{'C28', 'C48', 'C15', 'C41', 'A35', 'A56'\}} \\
\texttt{ARGUMENT: \{'Begründung: Diese Auswahl liegt bei 3717 Aufwand (C28 515 + C48 571 + C15 252 + C41 862 + A35 918 + A56 522) und nutzt das Budget effizient für mehrere qualitativ starke C{-}Items plus zwei A{-}Items, die den Wert gut hebeln. Wir bleiben unter 4379 und vermeiden die teuren B{-}Bündel. Offen für kleine Tausche, falls du einen bestimmten B{-}Artikel bevorzugst.'\}} \\
            }
        }
    }
    & \\ \\

    \theutterance \stepcounter{utterance}  
    & \multicolumn{4}{p{0.6\linewidth}}{
        \cellcolor[rgb]{0.9,0.9,0.9}{
            \makecell[{{p{\linewidth}}}]{
                \texttt{\tiny{[P1$\rangle$GM]}}
                \texttt{ABLEHNUNG: \{'C28', 'C48', 'C15', 'C41', 'A35', 'A56'\}} \\
\texttt{ARGUMENT: \{'Ich sehe den Effizienzgedanken, aber A35 und C41 sind für den Aufwand zu sperrig für mich. Ich möchte den Wertfokus halten und dafür einen gezielten B{-}Artikel reinnehmen, der für mich deutlich trägt.'\}} \\
\texttt{VORSCHLAG: \{'C28', 'C48', 'B10', 'B78', 'C15', 'A56'\}} \\
\texttt{ARGUMENT: \{'Aufwand 3151 (C28 515 + C48 571 + B10 493 + B78 211 + C15 252 + A56 522). Das ist schlanker als dein Vorschlag und lässt viel Puffer. C28/C48 und A56 bleiben drin wie von dir gewünscht, ich ersetze A35/C41 durch die effizienteren B10/B78. Wenn dir zwei B{-}Artikel zu viel sind, kann ich einen davon gegen C60 tauschen.'\}} \\
            }
        }
    }
    & & \\ \\

    \theutterance \stepcounter{utterance}  
    & & \multicolumn{4}{p{0.6\linewidth}}{
        \cellcolor[rgb]{0.9,0.9,0.9}{
            \makecell[{{p{\linewidth}}}]{
                \texttt{\tiny{[GM$\rangle$P2]}}
                \texttt{ABLEHNUNG: \{'C28', 'C48', 'C15', 'C41', 'A35', 'A56'\}} \\
\texttt{ARGUMENT: \{'Ich sehe den Effizienzgedanken, aber A35 und C41 sind für den Aufwand zu sperrig für mich. Ich möchte den Wertfokus halten und dafür einen gezielten B{-}Artikel reinnehmen, der für mich deutlich trägt.'\}} \\
\texttt{VORSCHLAG: \{'C28', 'C48', 'B10', 'B78', 'C15', 'A56'\}} \\
\texttt{ARGUMENT: \{'Aufwand 3151 (C28 515 + C48 571 + B10 493 + B78 211 + C15 252 + A56 522). Das ist schlanker als dein Vorschlag und lässt viel Puffer. C28/C48 und A56 bleiben drin wie von dir gewünscht, ich ersetze A35/C41 durch die effizienteren B10/B78. Wenn dir zwei B{-}Artikel zu viel sind, kann ich einen davon gegen C60 tauschen.'\}} \\
            }
        }
    }
    & \\ \\

    \theutterance \stepcounter{utterance}  
    & & & \multicolumn{4}{p{0.6\linewidth}}{
        \cellcolor[rgb]{0.9,0.9,0.9}{
            \makecell[{{p{\linewidth}}}]{
                \texttt{\tiny{[GM$\langle$P2]}}
                \texttt{ABLEHNUNG: \{'C28', 'C48', 'B10', 'B78', 'C15', 'A56'\}} \\
\texttt{ARGUMENT: \{'Danke fürs Entgegenkommen. Zwei Punkte: (1) B10 ist für mich schwer abzugeben; (2) Wenn wir schon schlank bleiben, möchte ich den freien Puffer nutzen, um einen zusätzlichen, für mich tragenden Posten reinzunehmen, statt zwei B{-}Artikel zu halten. C60 ist ein guter Tauschanker im Vergleich zu B78.'\}} \\
\texttt{VORSCHLAG: \{'C28', 'C48', 'B10', 'C60', 'C15', 'A56', 'A93'\}} \\
\texttt{ARGUMENT: \{'Aufwand 3413 (C28 515 + C48 571 + B10 493 + C60 401 + C15 252 + A56 522 + A93 599) – klar unter 4379, behält deine Kernidee C28/C48/A56 und ersetzt B78 durch C60. Mit dem Puffer nehme ich A93 hinzu, was die Gesamtrendite für mich rechtfertigt, ohne dich beim Aufwand zu knebeln. Falls A93 für dich unattraktiv ist, können wir alternativ A75 (339) statt A93 nehmen und bleiben bei 3153 Aufwand.'\}} \\
            }
        }
    }
     \\ \\

    \theutterance \stepcounter{utterance}  
    & & \multicolumn{4}{p{0.6\linewidth}}{
        \cellcolor[rgb]{0.9,0.9,0.9}{
            \makecell[{{p{\linewidth}}}]{
                \texttt{\tiny{[P1$\langle$GM]}}
                \texttt{ABLEHNUNG: \{'C28', 'C48', 'B10', 'B78', 'C15', 'A56'\}} \\
\texttt{ARGUMENT: \{'Danke fürs Entgegenkommen. Zwei Punkte: (1) B10 ist für mich schwer abzugeben; (2) Wenn wir schon schlank bleiben, möchte ich den freien Puffer nutzen, um einen zusätzlichen, für mich tragenden Posten reinzunehmen, statt zwei B{-}Artikel zu halten. C60 ist ein guter Tauschanker im Vergleich zu B78.'\}} \\
\texttt{VORSCHLAG: \{'C28', 'C48', 'B10', 'C60', 'C15', 'A56', 'A93'\}} \\
\texttt{ARGUMENT: \{'Aufwand 3413 (C28 515 + C48 571 + B10 493 + C60 401 + C15 252 + A56 522 + A93 599) – klar unter 4379, behält deine Kernidee C28/C48/A56 und ersetzt B78 durch C60. Mit dem Puffer nehme ich A93 hinzu, was die Gesamtrendite für mich rechtfertigt, ohne dich beim Aufwand zu knebeln. Falls A93 für dich unattraktiv ist, können wir alternativ A75 (339) statt A93 nehmen und bleiben bei 3153 Aufwand.'\}} \\
            }
        }
    }
    & \\ \\

    \theutterance \stepcounter{utterance}  
    & \multicolumn{4}{p{0.6\linewidth}}{
        \cellcolor[rgb]{0.9,0.9,0.9}{
            \makecell[{{p{\linewidth}}}]{
                \texttt{\tiny{[P1$\rangle$GM]}}
                \texttt{ABLEHNUNG: \{'C28', 'C48', 'B10', 'C60', 'C15', 'A56', 'A93'\}} \\
\texttt{ARGUMENT: \{'Danke für die Anpassung. Ich kann B10 nicht mitgehen, der ist für mich der falsche Hebel. Den schlanken Rahmen und C28/C48/A56/C15 finde ich gut, C60 passt ebenfalls. Lass uns B10 rausnehmen und den frei werdenden Spielraum sinnvoll nutzen.'\}} \\
\texttt{VORSCHLAG: \{'C28', 'C48', 'C60', 'C15', 'A56', 'B78', 'A35'\}} \\
\texttt{ARGUMENT: \{'Aufwand 3361 (C28 515 + C48 571 + C60 401 + C15 252 + A56 522 + B78 211 + A35 918). Ich ersetze B10 durch B78 und hebe dafür A35 hinein, um den Mehrwert zu kompensieren. Wir bleiben klar unter 4379. Falls dir A35 nicht zusagt, kann ich stattdessen B66 (803) statt A35 vorschlagen; Aufwand dann 3246.'\}} \\
            }
        }
    }
    & & \\ \\

    \theutterance \stepcounter{utterance}  
    & & \multicolumn{4}{p{0.6\linewidth}}{
        \cellcolor[rgb]{0.9,0.9,0.9}{
            \makecell[{{p{\linewidth}}}]{
                \texttt{\tiny{[GM$\rangle$P2]}}
                \texttt{ABLEHNUNG: \{'C28', 'C48', 'B10', 'C60', 'C15', 'A56', 'A93'\}} \\
\texttt{ARGUMENT: \{'Danke für die Anpassung. Ich kann B10 nicht mitgehen, der ist für mich der falsche Hebel. Den schlanken Rahmen und C28/C48/A56/C15 finde ich gut, C60 passt ebenfalls. Lass uns B10 rausnehmen und den frei werdenden Spielraum sinnvoll nutzen.'\}} \\
\texttt{VORSCHLAG: \{'C28', 'C48', 'C60', 'C15', 'A56', 'B78', 'A35'\}} \\
\texttt{ARGUMENT: \{'Aufwand 3361 (C28 515 + C48 571 + C60 401 + C15 252 + A56 522 + B78 211 + A35 918). Ich ersetze B10 durch B78 und hebe dafür A35 hinein, um den Mehrwert zu kompensieren. Wir bleiben klar unter 4379. Falls dir A35 nicht zusagt, kann ich stattdessen B66 (803) statt A35 vorschlagen; Aufwand dann 3246.'\}} \\
            }
        }
    }
    & \\ \\

    \theutterance \stepcounter{utterance}  
    & & & \multicolumn{4}{p{0.6\linewidth}}{
        \cellcolor[rgb]{0.9,0.9,0.9}{
            \makecell[{{p{\linewidth}}}]{
                \texttt{\tiny{[GM$\langle$P2]}}
                \texttt{ABLEHNUNG: \{'C28', 'C48', 'C60', 'C15', 'A56', 'B78', 'A35'\}} \\
\texttt{ARGUMENT: \{'A35 ist für mich in diesem Paket zu teuer für den Nutzen. Den Kern C28/C48/C60/C15/A56 kann ich stützen; B78 ist als Füller okay. Statt A35 möchte ich einen anderen starken Posten nutzen, der den Aufwand besser rechtfertigt.'\}} \\
\texttt{VORSCHLAG: \{'C28', 'C48', 'C60', 'C15', 'A56', 'B78', 'B66'\}} \\
\texttt{ARGUMENT: \{'Aufwand 3276 (515 + 571 + 401 + 252 + 522 + 211 + 803). Das behält deinen Rahmen bei, ersetzt A35 durch B66 als wertigeren, aber effizienteren Zuschlag. Wir bleiben deutlich unter 4379. Falls dir B66 nicht passt, können wir alternativ A93 statt B66 setzen; Aufwand dann 3073.'\}} \\
            }
        }
    }
     \\ \\

    \theutterance \stepcounter{utterance}  
    & & \multicolumn{4}{p{0.6\linewidth}}{
        \cellcolor[rgb]{0.9,0.9,0.9}{
            \makecell[{{p{\linewidth}}}]{
                \texttt{\tiny{[P1$\langle$GM]}}
                \texttt{ABLEHNUNG: \{'C28', 'C48', 'C60', 'C15', 'A56', 'B78', 'A35'\}} \\
\texttt{ARGUMENT: \{'A35 ist für mich in diesem Paket zu teuer für den Nutzen. Den Kern C28/C48/C60/C15/A56 kann ich stützen; B78 ist als Füller okay. Statt A35 möchte ich einen anderen starken Posten nutzen, der den Aufwand besser rechtfertigt.'\}} \\
\texttt{VORSCHLAG: \{'C28', 'C48', 'C60', 'C15', 'A56', 'B78', 'B66'\}} \\
\texttt{ARGUMENT: \{'Aufwand 3276 (515 + 571 + 401 + 252 + 522 + 211 + 803). Das behält deinen Rahmen bei, ersetzt A35 durch B66 als wertigeren, aber effizienteren Zuschlag. Wir bleiben deutlich unter 4379. Falls dir B66 nicht passt, können wir alternativ A93 statt B66 setzen; Aufwand dann 3073.'\}} \\
            }
        }
    }
    & \\ \\

    \theutterance \stepcounter{utterance}  
    & \multicolumn{4}{p{0.6\linewidth}}{
        \cellcolor[rgb]{0.9,0.9,0.9}{
            \makecell[{{p{\linewidth}}}]{
                \texttt{\tiny{[P1$\rangle$GM]}}
                \texttt{ABLEHNUNG: \{'C28', 'C48', 'C60', 'C15', 'A56', 'B78', 'B66'\}} \\
\texttt{ARGUMENT: \{'Danke, das kommt meiner Linie näher. Allerdings ist B66 für mich zu aufwändig für den Zusatznutzen. Ich würde lieber auf einen schlankeren Abschluss mit besserer Wertdichte gehen und den freien Puffer vermeiden, damit wir zielgerichtet bleiben.'\}} \\
\texttt{VORSCHLAG: \{'C28', 'C48', 'C60', 'C15', 'A56', 'B78', 'A93'\}} \\
\texttt{ARGUMENT: \{'Aufwand 3073 (515 + 571 + 401 + 252 + 522 + 211 + 599). Das entspricht deinem Kernpaket, ersetzt B66 durch A93 wie angeboten und bleibt komfortabel unter 4379. Falls du noch eine kleine Aufwertung wünschst, kann ich A75 zusätzlich aufnehmen und wir liegen dann bei 3412 Aufwand.'\}} \\
            }
        }
    }
    & & \\ \\

    \theutterance \stepcounter{utterance}  
    & & \multicolumn{4}{p{0.6\linewidth}}{
        \cellcolor[rgb]{0.9,0.9,0.9}{
            \makecell[{{p{\linewidth}}}]{
                \texttt{\tiny{[GM$\rangle$P2]}}
                \texttt{ABLEHNUNG: \{'C28', 'C48', 'C60', 'C15', 'A56', 'B78', 'B66'\}} \\
\texttt{ARGUMENT: \{'Danke, das kommt meiner Linie näher. Allerdings ist B66 für mich zu aufwändig für den Zusatznutzen. Ich würde lieber auf einen schlankeren Abschluss mit besserer Wertdichte gehen und den freien Puffer vermeiden, damit wir zielgerichtet bleiben.'\}} \\
\texttt{VORSCHLAG: \{'C28', 'C48', 'C60', 'C15', 'A56', 'B78', 'A93'\}} \\
\texttt{ARGUMENT: \{'Aufwand 3073 (515 + 571 + 401 + 252 + 522 + 211 + 599). Das entspricht deinem Kernpaket, ersetzt B66 durch A93 wie angeboten und bleibt komfortabel unter 4379. Falls du noch eine kleine Aufwertung wünschst, kann ich A75 zusätzlich aufnehmen und wir liegen dann bei 3412 Aufwand.'\}} \\
            }
        }
    }
    & \\ \\

    \theutterance \stepcounter{utterance}  
    & & & \multicolumn{4}{p{0.6\linewidth}}{
        \cellcolor[rgb]{0.9,0.9,0.9}{
            \makecell[{{p{\linewidth}}}]{
                \texttt{\tiny{[GM$\langle$P2]}}
                \texttt{ABLEHNUNG: \{'C28', 'C48', 'C60', 'C15', 'A56', 'B78', 'A93'\}} \\
\texttt{ARGUMENT: \{'Das ist nahe dran, aber A93 trifft für mich den Mehrwert nicht gut genug. Den Kern C28/C48/C60/C15/A56/B78 kann ich mittragen. Für den siebten Slot bevorzuge ich etwas mit besserem Verhältnis für mich – oder wir gehen bewusst noch schlanker. Ich biete dir zwei klare Pfade: (1) kompakt und effizient ohne siebten Brocken; (2) gezielt aufwerten mit einem für mich tragenderen Posten.'\}} \\
\texttt{VORSCHLAG: \{'C28', 'C48', 'C60', 'C15', 'A56', 'B78'\}} \\
\texttt{ARGUMENT: \{'Aufwand 2472. Sehr schlank und sauber, deckt den gemeinsamen Kern ab und vermeidet Überfrachtung. Wenn du stattdessen eine Aufwertung möchtest, tauschen wir A93 gegen A75+ (oder einen anderen gezielten Posten deiner Wahl) – ich bin offen, solange der Zusatznutzen stimmt.'\}} \\
            }
        }
    }
     \\ \\

    \theutterance \stepcounter{utterance}  
    & & \multicolumn{4}{p{0.6\linewidth}}{
        \cellcolor[rgb]{0.9,0.9,0.9}{
            \makecell[{{p{\linewidth}}}]{
                \texttt{\tiny{[P1$\langle$GM]}}
                \texttt{ABLEHNUNG: \{'C28', 'C48', 'C60', 'C15', 'A56', 'B78', 'A93'\}} \\
\texttt{ARGUMENT: \{'Das ist nahe dran, aber A93 trifft für mich den Mehrwert nicht gut genug. Den Kern C28/C48/C60/C15/A56/B78 kann ich mittragen. Für den siebten Slot bevorzuge ich etwas mit besserem Verhältnis für mich – oder wir gehen bewusst noch schlanker. Ich biete dir zwei klare Pfade: (1) kompakt und effizient ohne siebten Brocken; (2) gezielt aufwerten mit einem für mich tragenderen Posten.'\}} \\
\texttt{VORSCHLAG: \{'C28', 'C48', 'C60', 'C15', 'A56', 'B78'\}} \\
\texttt{ARGUMENT: \{'Aufwand 2472. Sehr schlank und sauber, deckt den gemeinsamen Kern ab und vermeidet Überfrachtung. Wenn du stattdessen eine Aufwertung möchtest, tauschen wir A93 gegen A75+ (oder einen anderen gezielten Posten deiner Wahl) – ich bin offen, solange der Zusatznutzen stimmt.'\}} \\
            }
        }
    }
    & \\ \\

    \theutterance \stepcounter{utterance}  
    & \multicolumn{4}{p{0.6\linewidth}}{
        \cellcolor[rgb]{0.9,0.9,0.9}{
            \makecell[{{p{\linewidth}}}]{
                \texttt{\tiny{[P1$\rangle$GM]}}
                \texttt{VORSCHLAG: \{'C28', 'C48', 'C60', 'C15', 'A56', 'B78', 'B66'\}} \\
\texttt{ARGUMENT: \{'Dein schlanker Kern ist okay, aber ich brauche einen klaren Aufwertungsanker. B66 liefert mir diesen Zusatznutzen deutlich besser als A93 und passt noch effizient in das Paket. Gesamtaufwand 3276 bleibt weit unter 4379. Wenn du strikt beim kompakten Pfad bleiben willst, bestätige den 6er{-}Kern – ansonsten ist dies mein bevorzugter Abschluss für fairen Gegenwert.'\}} \\
            }
        }
    }
    & & \\ \\

    \theutterance \stepcounter{utterance}  
    & & \multicolumn{4}{p{0.6\linewidth}}{
        \cellcolor[rgb]{0.9,0.9,0.9}{
            \makecell[{{p{\linewidth}}}]{
                \texttt{\tiny{[GM$\rangle$P2]}}
                \texttt{VORSCHLAG: \{'C28', 'C48', 'C60', 'C15', 'A56', 'B78', 'B66'\}} \\
\texttt{ARGUMENT: \{'Dein schlanker Kern ist okay, aber ich brauche einen klaren Aufwertungsanker. B66 liefert mir diesen Zusatznutzen deutlich besser als A93 und passt noch effizient in das Paket. Gesamtaufwand 3276 bleibt weit unter 4379. Wenn du strikt beim kompakten Pfad bleiben willst, bestätige den 6er{-}Kern – ansonsten ist dies mein bevorzugter Abschluss für fairen Gegenwert.'\}} \\
            }
        }
    }
    & \\ \\

    \theutterance \stepcounter{utterance}  
    & & & \multicolumn{4}{p{0.6\linewidth}}{
        \cellcolor[rgb]{0.9,0.9,0.9}{
            \makecell[{{p{\linewidth}}}]{
                \texttt{\tiny{[GM$\langle$P2]}}
                \texttt{ABLEHNUNG: \{'C28', 'C48', 'C60', 'C15', 'A56', 'B78', 'B66'\}} \\
\texttt{ARGUMENT: \{'Ich respektiere deinen Aufwertungsbedarf, aber B66 ist für mich der falsche Hebel: hoher Aufwand, den ich auf wertstärkere Positionen umlenken möchte. Ich halte den 6er{-}Kern für den fairen, schlanken Schnitt – oder wir werten gezielt anders auf.'\}} \\
\texttt{VORSCHLAG: \{'C28', 'C48', 'C60', 'C15', 'A56', 'B78', 'A75'\}} \\
\texttt{ARGUMENT: \{'Aufwand 2811 (515 + 571 + 401 + 252 + 522 + 211 + 339). Das erhält den gemeinsamen 6er{-}Kern und bietet dir eine klare, aber effiziente Aufwertung über A75 statt B66. Falls dir A75 nicht genügt, können wir alternativ A93 nehmen und landen bei 3073 – B66 bleibt für mich jedoch keine Option.'\}} \\
            }
        }
    }
     \\ \\

    \theutterance \stepcounter{utterance}  
    & & \multicolumn{4}{p{0.6\linewidth}}{
        \cellcolor[rgb]{0.9,0.9,0.9}{
            \makecell[{{p{\linewidth}}}]{
                \texttt{\tiny{[P1$\langle$GM]}}
                \texttt{ABLEHNUNG: \{'C28', 'C48', 'C60', 'C15', 'A56', 'B78', 'B66'\}} \\
\texttt{ARGUMENT: \{'Ich respektiere deinen Aufwertungsbedarf, aber B66 ist für mich der falsche Hebel: hoher Aufwand, den ich auf wertstärkere Positionen umlenken möchte. Ich halte den 6er{-}Kern für den fairen, schlanken Schnitt – oder wir werten gezielt anders auf.'\}} \\
\texttt{VORSCHLAG: \{'C28', 'C48', 'C60', 'C15', 'A56', 'B78', 'A75'\}} \\
\texttt{ARGUMENT: \{'Aufwand 2811 (515 + 571 + 401 + 252 + 522 + 211 + 339). Das erhält den gemeinsamen 6er{-}Kern und bietet dir eine klare, aber effiziente Aufwertung über A75 statt B66. Falls dir A75 nicht genügt, können wir alternativ A93 nehmen und landen bei 3073 – B66 bleibt für mich jedoch keine Option.'\}} \\
            }
        }
    }
    & \\ \\

    \theutterance \stepcounter{utterance}  
    & \multicolumn{4}{p{0.6\linewidth}}{
        \cellcolor[rgb]{0.9,0.9,0.9}{
            \makecell[{{p{\linewidth}}}]{
                \texttt{\tiny{[P1$\rangle$GM]}}
                \texttt{ZUSTIMMUNG: \{'C28', 'C48', 'C60', 'C15', 'A56', 'B78', 'A75'\}} \\
\texttt{ARGUMENT: \{'Einverstanden. Der 6er{-}Kern plus A75 ist ein schlanker, effizienter Abschluss, bleibt mit 2811 Aufwand komfortabel unter der Grenze und trifft einen tragfähigen Kompromiss ohne den von dir abgelehnten B66. Gute Einigung.'\}} \\
            }
        }
    }
    & & \\ \\

    \theutterance \stepcounter{utterance}  
    & & & \multicolumn{2}{p{0.3\linewidth}}{
        \cellcolor[rgb]{0.9,0.9,0.9}{
            \makecell[{{p{\linewidth}}}]{
                \texttt{\tiny{[GM$|$GM]}}
                \texttt{game successful} \\
            }
        }
    }
    & & \\ \\

    \theutterance \stepcounter{utterance}  
    & & & \multicolumn{2}{p{0.3\linewidth}}{
        \cellcolor[rgb]{0.9,0.9,0.9}{
            \makecell[{{p{\linewidth}}}]{
                \texttt{\tiny{[GM$|$GM]}}
                \texttt{end game} \\
            }
        }
    }
    & & \\ \\

\end{supertabular}
}

\end{document}
