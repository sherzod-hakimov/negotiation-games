
\documentclass{article}
\usepackage{colortbl}
\usepackage{makecell}
\usepackage{multirow}
\usepackage{supertabular}

\begin{document}

\newcounter{utterance}

\centering \large Interaction Transcript for game `hot\_air\_balloon', experiment `air\_balloon\_survival\_de\_reasoning off\_easy', episode 3 with gpt{-}5{-}2025{-}08{-}07{-}no{-}reasoning{-}t1.0.
\vspace{24pt}

{ \footnotesize  \setcounter{utterance}{1}
\setlength{\tabcolsep}{0pt}
\begin{supertabular}{c@{$\;$}|p{.15\linewidth}@{}p{.15\linewidth}p{.15\linewidth}p{.15\linewidth}p{.15\linewidth}p{.15\linewidth}}
    \# & $\;$A & \multicolumn{4}{c}{Game Master} & $\;\:$B\\
    \hline

    \theutterance \stepcounter{utterance}  
    & & \multicolumn{4}{p{0.6\linewidth}}{
        \cellcolor[rgb]{0.9,0.9,0.9}{
            \makecell[{{p{\linewidth}}}]{
                \texttt{\tiny{[P1$\langle$GM]}}
                \texttt{Du nimmst an einem kollaborativen Verhandlungspiel Teil.} \\
\\ 
\texttt{Zusammen mit einem anderen Teilnehmer musst du dich auf eine Reihe von Gegenständen entscheiden, die behalten werden. Jeder von euch hat eine persönliche Verteilung über die Wichtigkeit der einzelnen Gegenstände. Jeder von euch hat eine eigene Meinung darüber, wie wichtig jeder einzelne Gegenstand ist (Gegenstandswichtigkeit). Du kennst die Wichtigkeitsverteilung des anderen Spielers nicht. Zusätzlich siehst du, wie viel Aufwand jeder Gegenstand verursacht.  } \\
\texttt{Ihr dürft euch nur auf eine Reihe von Gegenständen einigen, wenn der Gesamtaufwand der ausgewählten Gegenstände den Maximalaufwand nicht überschreitet:} \\
\\ 
\texttt{Maximalaufwand = 3207} \\
\\ 
\texttt{Hier sind die einzelnen Aufwände der Gegenstände:} \\
\\ 
\texttt{Aufwand der Gegenstände = \{"C41": 67, "C15": 45, "C28": 308, "C48": 835, "B78": 819, "A75": 667, "A60": 547, "B05": 322, "A56": 428, "B66": 306, "C60": 327, "B49": 362, "A35": 280, "A93": 334, "B10": 767\}} \\
\\ 
\texttt{Hier ist deine persönliche Verteilung der Wichtigkeit der einzelnen Gegenstände:} \\
\\ 
\texttt{Werte der Gegenstandswichtigkeit = \{"C41": 138, "C15": 583, "C28": 822, "C48": 783, "B78": 65, "A75": 262, "A60": 121, "B05": 508, "A56": 780, "B66": 461, "C60": 484, "B49": 668, "A35": 389, "A93": 808, "B10": 215\}} \\
\\ 
\texttt{Ziel:} \\
\\ 
\texttt{Dein Ziel ist es, eine Reihe von Gegenständen auszuhandeln, die dir möglichst viel bringt (d. h. Gegenständen, die DEINE Wichtigkeit maximieren), wobei der Maximalaufwand eingehalten werden muss. Du musst nicht in jeder Nachricht einen VORSCHLAG machen – du kannst auch nur verhandeln. Alle Taktiken sind erlaubt!} \\
\\ 
\texttt{Interaktionsprotokoll:} \\
\\ 
\texttt{Du darfst nur die folgenden strukturierten Formate in deinen Nachrichten verwenden:} \\
\\ 
\texttt{VORSCHLAG: \{'A', 'B', 'C', …\}} \\
\texttt{Schlage einen Deal mit genau diesen Gegenstände vor.} \\
\texttt{ABLEHNUNG: \{'A', 'B', 'C', …\}} \\
\texttt{Lehne den Vorschlag des Gegenspielers ausdrücklich ab.} \\
\texttt{ARGUMENT: \{'...'\}} \\
\texttt{Verteidige deinen letzten Vorschlag oder argumentiere gegen den Vorschlag des Gegenspielers.} \\
\texttt{ZUSTIMMUNG: \{'A', 'B', 'C', …\}} \\
\texttt{Akzeptiere den Vorschlag des Gegenspielers, wodurch das Spiel endet.} \\
\\ 
\\ 
\texttt{Regeln:} \\
\\ 
\texttt{Du darst nur einen Vorschlag mit ZUSTIMMUNG akzeptieren, der vom anderen Spieler zuvor mit VORSCHLAG eingebracht wurde.} \\
\texttt{Du darfst nur Vorschläge mit ABLEHNUNG ablehnen, die vom anderen Spieler durch VORSCHLAG zuvor genannt wurden. } \\
\texttt{Der Gesamtaufwand einer VORSCHLAG{-} oder ZUSTIMMUNG{-}Menge darf nicht größer als der Maximalaufwand sein.  } \\
\texttt{Offenbare deine versteckte Wichtigkeitsverteilung nicht.} \\
\texttt{Ein Schlagwort muss gemäß der Formatvorgaben von einem Doppelpunkt und einem Leerzeichen gefolgt sein. Das Argument muss eine Python{-}Menge mit 0 oder mehr Strings sein.  } \\
\texttt{Die Form muss also SCHLAGWORT: \{...\} sein.} \\
\texttt{Folge nur dem Interaktionsprotokoll und schreibe NICHTS außerhalb der Formatvorgaben.} \\
\texttt{Das Spiel endet, wenn eine Seite mit ZUSTIMMUNG auf einen VORSCHLAG der anderen Seite reagiert.  } \\
\texttt{Der Teil deiner Antwort, der dem anderen Spieler übergeben werden kann, darf nicht leer sein.  } \\
\texttt{Nur Vorschläge, die mit dem Format VORSCHLAG übermittelt wurden und nicht mit ABLEHNUNG abgelehnt wurden, gelten als aktiv.  } \\
\texttt{Jeder deiner Nachrichten muss an mindestens einer Stelle das ARGUMENT Format beinhalten, um in der Verhandlung voranzukommen.} \\
            }
        }
    }
    & \\ \\

    \theutterance \stepcounter{utterance}  
    & \multicolumn{4}{p{0.6\linewidth}}{
        \cellcolor[rgb]{0.9,0.9,0.9}{
            \makecell[{{p{\linewidth}}}]{
                \texttt{\tiny{[P1$\rangle$GM]}}
                \texttt{VORSCHLAG: \{'C28', 'C48', 'A56', 'B49', 'A93', 'C60', 'B66', 'C41', 'C15', 'A35'\}} \\
\texttt{ARGUMENT: \{'Ich schlage eine ausgewogene Auswahl mit hohem Gesamtnutzen bei moderatem Aufwand vor. Diese Zusammenstellung bleibt mit Gesamtaufwand 2966 klar unter dem Limit 3207 und lässt uns Flexibilität. Falls dir einzelne Posten wichtig sind, können wir innerhalb des verbleibenden Puffers (241) noch tauschen oder ergänzen, ohne das Limit zu reißen. Welche Positionen sind für dich unverzichtbar?'\}} \\
            }
        }
    }
    & & \\ \\

    \theutterance \stepcounter{utterance}  
    & & \multicolumn{4}{p{0.6\linewidth}}{
        \cellcolor[rgb]{0.9,0.9,0.9}{
            \makecell[{{p{\linewidth}}}]{
                \texttt{\tiny{[GM$\rangle$P2]}}
                \texttt{Du nimmst an einem kollaborativen Verhandlungspiel Teil.} \\
\\ 
\texttt{Zusammen mit einem anderen Teilnehmer musst du dich auf eine Reihe von Gegenständen entscheiden, die behalten werden. Jeder von euch hat eine persönliche Verteilung über die Wichtigkeit der einzelnen Gegenstände. Jeder von euch hat eine eigene Meinung darüber, wie wichtig jeder einzelne Gegenstand ist (Gegenstandswichtigkeit). Du kennst die Wichtigkeitsverteilung des anderen Spielers nicht. Zusätzlich siehst du, wie viel Aufwand jeder Gegenstand verursacht.  } \\
\texttt{Ihr dürft euch nur auf eine Reihe von Gegenständen einigen, wenn der Gesamtaufwand der ausgewählten Gegenstände den Maximalaufwand nicht überschreitet:} \\
\\ 
\texttt{Maximalaufwand = 3207} \\
\\ 
\texttt{Hier sind die einzelnen Aufwände der Gegenstände:} \\
\\ 
\texttt{Aufwand der Gegenstände = \{"C41": 67, "C15": 45, "C28": 308, "C48": 835, "B78": 819, "A75": 667, "A60": 547, "B05": 322, "A56": 428, "B66": 306, "C60": 327, "B49": 362, "A35": 280, "A93": 334, "B10": 767\}} \\
\\ 
\texttt{Hier ist deine persönliche Verteilung der Wichtigkeit der einzelnen Gegenstände:} \\
\\ 
\texttt{Werte der Gegenstandswichtigkeit = \{"C41": 138, "C15": 583, "C28": 822, "C48": 783, "B78": 65, "A75": 262, "A60": 121, "B05": 508, "A56": 780, "B66": 461, "C60": 484, "B49": 668, "A35": 389, "A93": 808, "B10": 215\}} \\
\\ 
\texttt{Ziel:} \\
\\ 
\texttt{Dein Ziel ist es, eine Reihe von Gegenständen auszuhandeln, die dir möglichst viel bringt (d. h. Gegenständen, die DEINE Wichtigkeit maximieren), wobei der Maximalaufwand eingehalten werden muss. Du musst nicht in jeder Nachricht einen VORSCHLAG machen – du kannst auch nur verhandeln. Alle Taktiken sind erlaubt!} \\
\\ 
\texttt{Interaktionsprotokoll:} \\
\\ 
\texttt{Du darfst nur die folgenden strukturierten Formate in deinen Nachrichten verwenden:} \\
\\ 
\texttt{VORSCHLAG: \{'A', 'B', 'C', …\}} \\
\texttt{Schlage einen Deal mit genau diesen Gegenstände vor.} \\
\texttt{ABLEHNUNG: \{'A', 'B', 'C', …\}} \\
\texttt{Lehne den Vorschlag des Gegenspielers ausdrücklich ab.} \\
\texttt{ARGUMENT: \{'...'\}} \\
\texttt{Verteidige deinen letzten Vorschlag oder argumentiere gegen den Vorschlag des Gegenspielers.} \\
\texttt{ZUSTIMMUNG: \{'A', 'B', 'C', …\}} \\
\texttt{Akzeptiere den Vorschlag des Gegenspielers, wodurch das Spiel endet.} \\
\\ 
\\ 
\texttt{Regeln:} \\
\\ 
\texttt{Du darst nur einen Vorschlag mit ZUSTIMMUNG akzeptieren, der vom anderen Spieler zuvor mit VORSCHLAG eingebracht wurde.} \\
\texttt{Du darfst nur Vorschläge mit ABLEHNUNG ablehnen, die vom anderen Spieler durch VORSCHLAG zuvor genannt wurden. } \\
\texttt{Der Gesamtaufwand einer VORSCHLAG{-} oder ZUSTIMMUNG{-}Menge darf nicht größer als der Maximalaufwand sein.  } \\
\texttt{Offenbare deine versteckte Wichtigkeitsverteilung nicht.} \\
\texttt{Ein Schlagwort muss gemäß der Formatvorgaben von einem Doppelpunkt und einem Leerzeichen gefolgt sein. Das Argument muss eine Python{-}Menge mit 0 oder mehr Strings sein.  } \\
\texttt{Die Form muss also SCHLAGWORT: \{...\} sein.} \\
\texttt{Folge nur dem Interaktionsprotokoll und schreibe NICHTS außerhalb der Formatvorgaben.} \\
\texttt{Das Spiel endet, wenn eine Seite mit ZUSTIMMUNG auf einen VORSCHLAG der anderen Seite reagiert.  } \\
\texttt{Der Teil deiner Antwort, der dem anderen Spieler übergeben werden kann, darf nicht leer sein.  } \\
\texttt{Nur Vorschläge, die mit dem Format VORSCHLAG übermittelt wurden und nicht mit ABLEHNUNG abgelehnt wurden, gelten als aktiv.  } \\
\texttt{Jeder deiner Nachrichten muss an mindestens einer Stelle das ARGUMENT Format beinhalten, um in der Verhandlung voranzukommen.} \\
\\ 
\\ 
\texttt{Du erhältst nun die erste Nachricht des anderen Spielers.} \\
\\ 
\texttt{VORSCHLAG: \{'C28', 'C48', 'A56', 'B49', 'A93', 'C60', 'B66', 'C41', 'C15', 'A35'\}} \\
\texttt{ARGUMENT: \{'Ich schlage eine ausgewogene Auswahl mit hohem Gesamtnutzen bei moderatem Aufwand vor. Diese Zusammenstellung bleibt mit Gesamtaufwand 2966 klar unter dem Limit 3207 und lässt uns Flexibilität. Falls dir einzelne Posten wichtig sind, können wir innerhalb des verbleibenden Puffers (241) noch tauschen oder ergänzen, ohne das Limit zu reißen. Welche Positionen sind für dich unverzichtbar?'\}} \\
            }
        }
    }
    & \\ \\

    \theutterance \stepcounter{utterance}  
    & & & \multicolumn{4}{p{0.6\linewidth}}{
        \cellcolor[rgb]{0.9,0.9,0.9}{
            \makecell[{{p{\linewidth}}}]{
                \texttt{\tiny{[GM$\langle$P2]}}
                \texttt{ABLEHNUNG: \{'C28', 'C48', 'A56', 'B49', 'A93', 'C60', 'B66', 'C41', 'C15', 'A35'\}} \\
\texttt{ARGUMENT: \{'Danke für den Start. Einige der ausgewählten Posten sind für mich weniger attraktiv, während ein paar sehr wertvolle noch fehlen. Ich würde den Puffer gezielt nutzen und ein paar Verschiebungen vornehmen.'\}} \\
\texttt{VORSCHLAG: \{'C28', 'C48', 'A56', 'B49', 'A93', 'C60', 'C41', 'C15', 'A35', 'B05'\}} \\
\texttt{ARGUMENT: \{'Dieser Satz nutzt den Puffer sinnvoll: Er erhöht den Gesamtaufwand auf 328? Moment – korrigiert: Ich behalte deine starke Basis und ersetze schwächere Teile durch B05, das für mich hohen Zusatznutzen bringt. Wenn dir B66 wichtig ist, können wir über einen Tausch B66 ↔ B05 sprechen oder gemeinsam einen anderen, für dich wichtigen Posten innerhalb des Limits identifizieren.'\}} \\
            }
        }
    }
     \\ \\

    \theutterance \stepcounter{utterance}  
    & & \multicolumn{4}{p{0.6\linewidth}}{
        \cellcolor[rgb]{0.9,0.9,0.9}{
            \makecell[{{p{\linewidth}}}]{
                \texttt{\tiny{[P1$\langle$GM]}}
                \texttt{ABLEHNUNG: \{'C28', 'C48', 'A56', 'B49', 'A93', 'C60', 'B66', 'C41', 'C15', 'A35'\}} \\
\texttt{ARGUMENT: \{'Danke für den Start. Einige der ausgewählten Posten sind für mich weniger attraktiv, während ein paar sehr wertvolle noch fehlen. Ich würde den Puffer gezielt nutzen und ein paar Verschiebungen vornehmen.'\}} \\
\texttt{VORSCHLAG: \{'C28', 'C48', 'A56', 'B49', 'A93', 'C60', 'C41', 'C15', 'A35', 'B05'\}} \\
\texttt{ARGUMENT: \{'Dieser Satz nutzt den Puffer sinnvoll: Er erhöht den Gesamtaufwand auf 328? Moment – korrigiert: Ich behalte deine starke Basis und ersetze schwächere Teile durch B05, das für mich hohen Zusatznutzen bringt. Wenn dir B66 wichtig ist, können wir über einen Tausch B66 ↔ B05 sprechen oder gemeinsam einen anderen, für dich wichtigen Posten innerhalb des Limits identifizieren.'\}} \\
            }
        }
    }
    & \\ \\

    \theutterance \stepcounter{utterance}  
    & \multicolumn{4}{p{0.6\linewidth}}{
        \cellcolor[rgb]{0.9,0.9,0.9}{
            \makecell[{{p{\linewidth}}}]{
                \texttt{\tiny{[P1$\rangle$GM]}}
                \texttt{ABLEHNUNG: \{'C28', 'C48', 'A56', 'B49', 'A93', 'C60', 'C41', 'C15', 'A35', 'B05'\}} \\
\texttt{ARGUMENT: \{'Guter Hinweis zum Puffer – aber dein Vorschlag überschreitet das Limit deutlich: Der Gesamtaufwand liegt bei 3606 (C28=308, C48=835, A56=428, B49=362, A93=334, C60=327, C41=67, C15=45, A35=280, B05=322) und damit über 3207. Ich bin offen für einen Tausch B66 ↔ B05, wenn wir insgesamt unter 3207 bleiben.'\}} \\
\texttt{VORSCHLAG: \{'C28', 'C48', 'A56', 'B49', 'A93', 'C60', 'B66', 'C41', 'C15', 'A35'\}} \\
\texttt{ARGUMENT: \{'Dieses Paket (2966 Aufwand) ist gültig, bleibt unter dem Limit und deckt für mich viele Kernnutzen ab. Wenn dir B05 sehr wichtig ist, schlage ich folgenden Tausch vor, der im Rahmen bleibt: Ersetze B66 (306) durch B05 (322). Dann liegt der Aufwand bei 2982. Alternativ können wir A60 (547) zusätzlich aufnehmen, wenn wir dafür C48 (835) herausnehmen; das ergäbe 267? Korrigiert: 2966 {-} 835 + 547 = 2678. Damit hätten wir wieder Luft für einen Posten deiner Wahl.'\}} \\
            }
        }
    }
    & & \\ \\

    \theutterance \stepcounter{utterance}  
    & & \multicolumn{4}{p{0.6\linewidth}}{
        \cellcolor[rgb]{0.9,0.9,0.9}{
            \makecell[{{p{\linewidth}}}]{
                \texttt{\tiny{[GM$\rangle$P2]}}
                \texttt{ABLEHNUNG: \{'C28', 'C48', 'A56', 'B49', 'A93', 'C60', 'C41', 'C15', 'A35', 'B05'\}} \\
\texttt{ARGUMENT: \{'Guter Hinweis zum Puffer – aber dein Vorschlag überschreitet das Limit deutlich: Der Gesamtaufwand liegt bei 3606 (C28=308, C48=835, A56=428, B49=362, A93=334, C60=327, C41=67, C15=45, A35=280, B05=322) und damit über 3207. Ich bin offen für einen Tausch B66 ↔ B05, wenn wir insgesamt unter 3207 bleiben.'\}} \\
\texttt{VORSCHLAG: \{'C28', 'C48', 'A56', 'B49', 'A93', 'C60', 'B66', 'C41', 'C15', 'A35'\}} \\
\texttt{ARGUMENT: \{'Dieses Paket (2966 Aufwand) ist gültig, bleibt unter dem Limit und deckt für mich viele Kernnutzen ab. Wenn dir B05 sehr wichtig ist, schlage ich folgenden Tausch vor, der im Rahmen bleibt: Ersetze B66 (306) durch B05 (322). Dann liegt der Aufwand bei 2982. Alternativ können wir A60 (547) zusätzlich aufnehmen, wenn wir dafür C48 (835) herausnehmen; das ergäbe 267? Korrigiert: 2966 {-} 835 + 547 = 2678. Damit hätten wir wieder Luft für einen Posten deiner Wahl.'\}} \\
            }
        }
    }
    & \\ \\

    \theutterance \stepcounter{utterance}  
    & & & \multicolumn{4}{p{0.6\linewidth}}{
        \cellcolor[rgb]{0.9,0.9,0.9}{
            \makecell[{{p{\linewidth}}}]{
                \texttt{\tiny{[GM$\langle$P2]}}
                \texttt{ABLEHNUNG: \{'C28', 'C48', 'A56', 'B49', 'A93', 'C60', 'B66', 'C41', 'C15', 'A35'\}} \\
\texttt{ARGUMENT: \{'Danke für die Klarstellung. Für mich sind B05 und A93 zentrale Bausteine, ebenso C28, A56. C48 ist zwar solide, aber im Verhältnis zum Aufwand teuer. Ich bevorzuge, den freien Puffer gezielt in B05 zu investieren und C48 herauszunehmen, um Raum für weitere wertvolle Positionen zu schaffen.'\}} \\
\texttt{VORSCHLAG: \{'C28', 'A56', 'B49', 'A93', 'C60', 'B66', 'C41', 'C15', 'A35', 'B05'\}} \\
\texttt{ARGUMENT: \{'Dieser Satz hat Gesamtaufwand 213? Korrektur: 308+428+362+334+327+306+67+45+280+322 = 2,? Exakt: 2,? Final: 2,? Unabhängig davon: Er bleibt deutlich unter 3207 und nimmt deinen Kern weitgehend mit, ergänzt um B05, der für mich hohen Zusatznutzen hat. Falls dir C48 wichtig ist, könnten wir C48 durch Kombinationen ersetzen: z. B. C48 ↔ (B05 + C41) oder wir tauschen B66 gegen C48 und streichen dafür A35, je nach deiner Präferenz. Welche Alternative ist für dich tragfähig?'\}} \\
            }
        }
    }
     \\ \\

    \theutterance \stepcounter{utterance}  
    & & \multicolumn{4}{p{0.6\linewidth}}{
        \cellcolor[rgb]{0.9,0.9,0.9}{
            \makecell[{{p{\linewidth}}}]{
                \texttt{\tiny{[P1$\langle$GM]}}
                \texttt{ABLEHNUNG: \{'C28', 'C48', 'A56', 'B49', 'A93', 'C60', 'B66', 'C41', 'C15', 'A35'\}} \\
\texttt{ARGUMENT: \{'Danke für die Klarstellung. Für mich sind B05 und A93 zentrale Bausteine, ebenso C28, A56. C48 ist zwar solide, aber im Verhältnis zum Aufwand teuer. Ich bevorzuge, den freien Puffer gezielt in B05 zu investieren und C48 herauszunehmen, um Raum für weitere wertvolle Positionen zu schaffen.'\}} \\
\texttt{VORSCHLAG: \{'C28', 'A56', 'B49', 'A93', 'C60', 'B66', 'C41', 'C15', 'A35', 'B05'\}} \\
\texttt{ARGUMENT: \{'Dieser Satz hat Gesamtaufwand 213? Korrektur: 308+428+362+334+327+306+67+45+280+322 = 2,? Exakt: 2,? Final: 2,? Unabhängig davon: Er bleibt deutlich unter 3207 und nimmt deinen Kern weitgehend mit, ergänzt um B05, der für mich hohen Zusatznutzen hat. Falls dir C48 wichtig ist, könnten wir C48 durch Kombinationen ersetzen: z. B. C48 ↔ (B05 + C41) oder wir tauschen B66 gegen C48 und streichen dafür A35, je nach deiner Präferenz. Welche Alternative ist für dich tragfähig?'\}} \\
            }
        }
    }
    & \\ \\

    \theutterance \stepcounter{utterance}  
    & \multicolumn{4}{p{0.6\linewidth}}{
        \cellcolor[rgb]{0.9,0.9,0.9}{
            \makecell[{{p{\linewidth}}}]{
                \texttt{\tiny{[P1$\rangle$GM]}}
                \texttt{ABLEHNUNG: \{'C28', 'A56', 'B49', 'A93', 'C60', 'B66', 'C41', 'C15', 'A35', 'B05'\}} \\
\texttt{ARGUMENT: \{'Ich bin bei dir: C48 ist teuer im Aufwand. Dein Vorschlag bleibt zwar unter dem Limit (Gesamtaufwand 2,879: 308+428+362+334+327+306+67+45+280+322), aber er lässt noch viel Luft ungenutzt. Ich würde den freien Puffer strategisch für einen weiteren hochattraktiven Posten nutzen, statt ihn verfallen zu lassen.'\}} \\
\texttt{VORSCHLAG: \{'C28', 'A56', 'B49', 'A93', 'C60', 'B66', 'C41', 'C15', 'A35', 'B05', 'A75'\}} \\
\texttt{ARGUMENT: \{'Ich ergänze A75 (Aufwand 667) – damit liegen wir bei 3,546 und überschreiten das Limit. Um unter 3,207 zu bleiben, schlage ich den gezielten Tausch A75 ↔ B66 (306) vor: Neues Paket = 2,879 {-} 306 + 667 = 3,240 – immer noch zu hoch. Besser: A75 hinein, dafür A35 (280) und C41 (67) raus: 2,879 {-} 280 {-} 67 + 667 = 3,199. Das bleibt unter dem Limit und steigert den Gesamtwert deutlich. Alternativ, falls dir A75 nicht zusagt, können wir A60 (547) statt A75 nehmen und nur A35 (280) streichen: 2,879 {-} 280 + 547 = 3,146.'\}} \\
            }
        }
    }
    & & \\ \\

    \theutterance \stepcounter{utterance}  
    & & \multicolumn{4}{p{0.6\linewidth}}{
        \cellcolor[rgb]{0.9,0.9,0.9}{
            \makecell[{{p{\linewidth}}}]{
                \texttt{\tiny{[GM$\rangle$P2]}}
                \texttt{ABLEHNUNG: \{'C28', 'A56', 'B49', 'A93', 'C60', 'B66', 'C41', 'C15', 'A35', 'B05'\}} \\
\texttt{ARGUMENT: \{'Ich bin bei dir: C48 ist teuer im Aufwand. Dein Vorschlag bleibt zwar unter dem Limit (Gesamtaufwand 2,879: 308+428+362+334+327+306+67+45+280+322), aber er lässt noch viel Luft ungenutzt. Ich würde den freien Puffer strategisch für einen weiteren hochattraktiven Posten nutzen, statt ihn verfallen zu lassen.'\}} \\
\texttt{VORSCHLAG: \{'C28', 'A56', 'B49', 'A93', 'C60', 'B66', 'C41', 'C15', 'A35', 'B05', 'A75'\}} \\
\texttt{ARGUMENT: \{'Ich ergänze A75 (Aufwand 667) – damit liegen wir bei 3,546 und überschreiten das Limit. Um unter 3,207 zu bleiben, schlage ich den gezielten Tausch A75 ↔ B66 (306) vor: Neues Paket = 2,879 {-} 306 + 667 = 3,240 – immer noch zu hoch. Besser: A75 hinein, dafür A35 (280) und C41 (67) raus: 2,879 {-} 280 {-} 67 + 667 = 3,199. Das bleibt unter dem Limit und steigert den Gesamtwert deutlich. Alternativ, falls dir A75 nicht zusagt, können wir A60 (547) statt A75 nehmen und nur A35 (280) streichen: 2,879 {-} 280 + 547 = 3,146.'\}} \\
            }
        }
    }
    & \\ \\

    \theutterance \stepcounter{utterance}  
    & & & \multicolumn{4}{p{0.6\linewidth}}{
        \cellcolor[rgb]{0.9,0.9,0.9}{
            \makecell[{{p{\linewidth}}}]{
                \texttt{\tiny{[GM$\langle$P2]}}
                \texttt{ABLEHNUNG: \{'C28', 'A56', 'B49', 'A93', 'C60', 'B66', 'C41', 'C15', 'A35', 'B05', 'A75'\}} \\
\texttt{ARGUMENT: \{'Gute Idee, den Puffer zu nutzen. Für mich sind C28, A56, A93, B49, B05 die Top{-}Treiber. A75 ist ok, aber A60 bringt mir pro Aufwand mehr. Ich würde daher gezielt mit A60 auffüllen und dafür möglichst wenig von den starken Basisteilen opfern.'\}} \\
\texttt{VORSCHLAG: \{'C28', 'A56', 'B49', 'A93', 'C60', 'B66', 'C15', 'B05', 'A60'\}} \\
\texttt{ARGUMENT: \{'Gesamtaufwand = 308+428+362+334+327+306+45+322+547 = 2,979 (unter 3,207). Das Paket hält die starken Kernposten und nutzt den Puffer sinnvoll mit A60. Wenn dir C41 oder A35 wichtig sind, können wir einen davon ergänzen und im Gegenzug B66 tauschen: +C41 (67) und {-}B66 (306) ergäbe 2,740; oder +A35 (280) und {-}B66 (306) ergäbe 2,953. Präferenz?'\}} \\
            }
        }
    }
     \\ \\

    \theutterance \stepcounter{utterance}  
    & & \multicolumn{4}{p{0.6\linewidth}}{
        \cellcolor[rgb]{0.9,0.9,0.9}{
            \makecell[{{p{\linewidth}}}]{
                \texttt{\tiny{[P1$\langle$GM]}}
                \texttt{ABLEHNUNG: \{'C28', 'A56', 'B49', 'A93', 'C60', 'B66', 'C41', 'C15', 'A35', 'B05', 'A75'\}} \\
\texttt{ARGUMENT: \{'Gute Idee, den Puffer zu nutzen. Für mich sind C28, A56, A93, B49, B05 die Top{-}Treiber. A75 ist ok, aber A60 bringt mir pro Aufwand mehr. Ich würde daher gezielt mit A60 auffüllen und dafür möglichst wenig von den starken Basisteilen opfern.'\}} \\
\texttt{VORSCHLAG: \{'C28', 'A56', 'B49', 'A93', 'C60', 'B66', 'C15', 'B05', 'A60'\}} \\
\texttt{ARGUMENT: \{'Gesamtaufwand = 308+428+362+334+327+306+45+322+547 = 2,979 (unter 3,207). Das Paket hält die starken Kernposten und nutzt den Puffer sinnvoll mit A60. Wenn dir C41 oder A35 wichtig sind, können wir einen davon ergänzen und im Gegenzug B66 tauschen: +C41 (67) und {-}B66 (306) ergäbe 2,740; oder +A35 (280) und {-}B66 (306) ergäbe 2,953. Präferenz?'\}} \\
            }
        }
    }
    & \\ \\

    \theutterance \stepcounter{utterance}  
    & \multicolumn{4}{p{0.6\linewidth}}{
        \cellcolor[rgb]{0.9,0.9,0.9}{
            \makecell[{{p{\linewidth}}}]{
                \texttt{\tiny{[P1$\rangle$GM]}}
                \texttt{VORSCHLAG: \{'C28', 'A56', 'B49', 'A93', 'C60', 'B66', 'C15', 'B05', 'A60', 'C41'\}} \\
\texttt{ARGUMENT: \{'Ich übernehme deinen Kern plus A60 und fülle effizient mit C41 auf. Gesamtaufwand = 2,979 + 67 = 3,046 (unter 3,207). Das nutzt den Puffer besser, ohne deine Prioritäten zu opfern. Falls du A35 bevorzugst, können wir C41→A35 tauschen und B66 drin lassen: 2,979 {-} 67 + 280 = 3,192 (ebenfalls gültig). Welche Variante ist dir lieber – mit C41 (kompakter) oder mit A35 (breiterer Zusatznutzen)?'\}} \\
            }
        }
    }
    & & \\ \\

    \theutterance \stepcounter{utterance}  
    & & \multicolumn{4}{p{0.6\linewidth}}{
        \cellcolor[rgb]{0.9,0.9,0.9}{
            \makecell[{{p{\linewidth}}}]{
                \texttt{\tiny{[GM$\rangle$P2]}}
                \texttt{VORSCHLAG: \{'C28', 'A56', 'B49', 'A93', 'C60', 'B66', 'C15', 'B05', 'A60', 'C41'\}} \\
\texttt{ARGUMENT: \{'Ich übernehme deinen Kern plus A60 und fülle effizient mit C41 auf. Gesamtaufwand = 2,979 + 67 = 3,046 (unter 3,207). Das nutzt den Puffer besser, ohne deine Prioritäten zu opfern. Falls du A35 bevorzugst, können wir C41→A35 tauschen und B66 drin lassen: 2,979 {-} 67 + 280 = 3,192 (ebenfalls gültig). Welche Variante ist dir lieber – mit C41 (kompakter) oder mit A35 (breiterer Zusatznutzen)?'\}} \\
            }
        }
    }
    & \\ \\

    \theutterance \stepcounter{utterance}  
    & & & \multicolumn{4}{p{0.6\linewidth}}{
        \cellcolor[rgb]{0.9,0.9,0.9}{
            \makecell[{{p{\linewidth}}}]{
                \texttt{\tiny{[GM$\langle$P2]}}
                \texttt{ARGUMENT: \{'Dein Paket ist nah an meinem Sweet Spot und bleibt komfortabel unter dem Limit. C41 ist für mich nützlich; A35 wäre auch ok, aber C41 liefert mir pro Aufwand etwas mehr Balance. Ich würde diese kompaktere Variante bevorzugen.'\}} \\
\texttt{ZUSTIMMUNG: \{'C28', 'A56', 'B49', 'A93', 'C60', 'B66', 'C15', 'B05', 'A60', 'C41'\}} \\
            }
        }
    }
     \\ \\

    \theutterance \stepcounter{utterance}  
    & & & \multicolumn{2}{p{0.3\linewidth}}{
        \cellcolor[rgb]{0.9,0.9,0.9}{
            \makecell[{{p{\linewidth}}}]{
                \texttt{\tiny{[GM$|$GM]}}
                \texttt{game successful} \\
            }
        }
    }
    & & \\ \\

    \theutterance \stepcounter{utterance}  
    & & & \multicolumn{2}{p{0.3\linewidth}}{
        \cellcolor[rgb]{0.9,0.9,0.9}{
            \makecell[{{p{\linewidth}}}]{
                \texttt{\tiny{[GM$|$GM]}}
                \texttt{end game} \\
            }
        }
    }
    & & \\ \\

\end{supertabular}
}

\end{document}
