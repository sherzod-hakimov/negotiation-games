
\documentclass{article}
\usepackage{colortbl}
\usepackage{makecell}
\usepackage{multirow}
\usepackage{supertabular}

\begin{document}

\newcounter{utterance}

\twocolumn

{ \footnotesize  \setcounter{utterance}{1}
\setlength{\tabcolsep}{0pt}
\begin{supertabular}{c@{$\;$}|p{.15\linewidth}@{}p{.15\linewidth}p{.15\linewidth}p{.15\linewidth}p{.15\linewidth}p{.15\linewidth}}

    \# & $\;$A & \multicolumn{4}{c}{Game Master} & $\;\:$B\\
    \hline 

    \theutterance \stepcounter{utterance}  

    & & \multicolumn{4}{p{0.6\linewidth}}{\cellcolor[rgb]{0.9,0.9,0.9}{%
	\makecell[{{p{\linewidth}}}]{% 
	  \tt {\tiny [A$\langle$GM]}  
	 Du nimmst an einem kollaborativen Verhandlungspiel Teil.\\ \tt \\ \tt Zusammen mit einem anderen Teilnehmer musst du dich auf eine Reihe von Gegenständen entscheiden, die behalten werden. Jeder von euch hat eine persönliche Verteilung über die Wichtigkeit der einzelnen Gegenstände. Jeder von euch hat eine eigene Meinung darüber, wie wichtig jeder einzelne Gegenstand ist (Gegenstandswichtigkeit). Du kennst die Wichtigkeitsverteilung des anderen Spielers nicht. Zusätzlich siehst du, wie viel Aufwand jeder Gegenstand verursacht.  \\ \tt Ihr dürft euch nur auf eine Reihe von Gegenständen einigen, wenn der Gesamtaufwand der ausgewählten Gegenstände den Maximalaufwand nicht überschreitet:\\ \tt \\ \tt Maximalaufwand = 3218\\ \tt \\ \tt Hier sind die einzelnen Aufwände der Gegenstände:\\ \tt \\ \tt Aufwand der Gegenstände = {"C41": 919, "C15": 852, "C28": 299, "C48": 762, "B78": 161, "A75": 205, "A60": 380, "B05": 399, "A56": 534, "B66": 333, "C60": 100, "B49": 420, "A35": 354, "A93": 130, "B10": 589}\\ \tt \\ \tt Hier ist deine persönliche Verteilung der Wichtigkeit der einzelnen Gegenstände:\\ \tt \\ \tt Werte der Gegenstandswichtigkeit = {"C41": 138, "C15": 583, "C28": 868, "C48": 822, "B78": 783, "A75": 65, "A60": 262, "B05": 121, "A56": 508, "B66": 780, "C60": 461, "B49": 484, "A35": 668, "A93": 389, "B10": 808}\\ \tt \\ \tt Ziel:\\ \tt \\ \tt Dein Ziel ist es, eine Reihe von Gegenständen auszuhandeln, die dir möglichst viel bringt (d. h. Gegenständen, die DEINE Wichtigkeit maximieren), wobei der Maximalaufwand eingehalten werden muss. Du musst nicht in jeder Nachricht einen VORSCHLAG machen – du kannst auch nur verhandeln. Alle Taktiken sind erlaubt!\\ \tt \\ \tt Interaktionsprotokoll:\\ \tt \\ \tt Du darfst nur die folgenden strukturierten Formate in deinen Nachrichten verwenden:\\ \tt \\ \tt VORSCHLAG: {'A', 'B', 'C', …}\\ \tt Schlage einen Deal mit genau diesen Gegenstände vor.\\ \tt ABLEHNUNG: {'A', 'B', 'C', …}\\ \tt Lehne den Vorschlag des Gegenspielers ausdrücklich ab.\\ \tt ARGUMENT: {'...'}\\ \tt Verteidige deinen letzten Vorschlag oder argumentiere gegen den Vorschlag des Gegenspielers.\\ \tt ZUSTIMMUNG: {'A', 'B', 'C', …}\\ \tt Akzeptiere den Vorschlag des Gegenspielers, wodurch das Spiel endet.\\ \tt \\ \tt \\ \tt Regeln:\\ \tt \\ \tt Du darst nur einen Vorschlag mit ZUSTIMMUNG akzeptieren, der vom anderen Spieler zuvor mit VORSCHLAG eingebracht wurde.\\ \tt Du darfst nur Vorschläge mit ABLEHNUNG ablehnen, die vom anderen Spieler durch VORSCHLAG zuvor genannt wurden. \\ \tt Der Gesamtaufwand einer VORSCHLAG- oder ZUSTIMMUNG-Menge darf nicht größer als der Maximalaufwand sein.  \\ \tt Offenbare deine versteckte Wichtigkeitsverteilung nicht.\\ \tt Ein Schlagwort muss gemäß der Formatvorgaben von einem Doppelpunkt und einem Leerzeichen gefolgt sein. Das Argument muss eine Python-Menge mit 0 oder mehr Strings sein.  \\ \tt Die Form muss also SCHLAGWORT: {...} sein.\\ \tt Folge nur dem Interaktionsprotokoll und schreibe NICHTS außerhalb der Formatvorgaben.\\ \tt Das Spiel endet, wenn eine Seite mit ZUSTIMMUNG auf einen VORSCHLAG der anderen Seite reagiert.  \\ \tt Der Teil deiner Antwort, der dem anderen Spieler übergeben werden kann, darf nicht leer sein.  \\ \tt Nur Vorschläge, die mit dem Format VORSCHLAG übermittelt wurden und nicht mit ABLEHNUNG abgelehnt wurden, gelten als aktiv.  \\ \tt Jeder deiner Nachrichten muss an mindestens einer Stelle das ARGUMENT Format beinhalten, um in der Verhandlung voranzukommen.\\ \tt  
	  } 
	   } 
	   } 
	 & \\ 
 

    \theutterance \stepcounter{utterance}  

    & \multicolumn{4}{p{0.6\linewidth}}{\cellcolor[rgb]{0.8,1,0.9}{%
	\makecell[{{p{\linewidth}}}]{% 
	  \tt {\tiny [A$\rangle$GM]}  
	 VORSCHLAG: {'C28', 'C48', 'B78', 'B66', 'C60', 'B49', 'A35', 'A93'}\\ \tt ARGUMENT: {'Ich schlage diese Kombination vor, weil sie den Aufwand von 3218 exakt ausnutzt und dabei sehr ausgewogene, vielseitige Posten enthält. Die Auswahl fokussiert sich auf mehrere C- und B-Elemente mit hohem gemeinschaftlichen Nutzen und lässt teure, weniger effiziente Posten außen vor. Wenn dir einzelne Elemente besonders wichtig sind, können wir darüber sprechen und entsprechend tauschen, solange wir im Aufwandslimit bleiben.'} 
	  } 
	   } 
	   } 
	 & & \\ 
 

    \theutterance \stepcounter{utterance}  

    & & \multicolumn{4}{p{0.6\linewidth}}{\cellcolor[rgb]{0.9,0.9,0.9}{%
	\makecell[{{p{\linewidth}}}]{% 
	  \tt {\tiny [GM$\rangle$B]}  
	 Du nimmst an einem kollaborativen Verhandlungspiel Teil.\\ \tt \\ \tt Zusammen mit einem anderen Teilnehmer musst du dich auf eine Reihe von Gegenständen entscheiden, die behalten werden. Jeder von euch hat eine persönliche Verteilung über die Wichtigkeit der einzelnen Gegenstände. Jeder von euch hat eine eigene Meinung darüber, wie wichtig jeder einzelne Gegenstand ist (Gegenstandswichtigkeit). Du kennst die Wichtigkeitsverteilung des anderen Spielers nicht. Zusätzlich siehst du, wie viel Aufwand jeder Gegenstand verursacht.  \\ \tt Ihr dürft euch nur auf eine Reihe von Gegenständen einigen, wenn der Gesamtaufwand der ausgewählten Gegenstände den Maximalaufwand nicht überschreitet:\\ \tt \\ \tt Maximalaufwand = 3218\\ \tt \\ \tt Hier sind die einzelnen Aufwände der Gegenstände:\\ \tt \\ \tt Aufwand der Gegenstände = {"C41": 919, "C15": 852, "C28": 299, "C48": 762, "B78": 161, "A75": 205, "A60": 380, "B05": 399, "A56": 534, "B66": 333, "C60": 100, "B49": 420, "A35": 354, "A93": 130, "B10": 589}\\ \tt \\ \tt Hier ist deine persönliche Verteilung der Wichtigkeit der einzelnen Gegenstände:\\ \tt \\ \tt Werte der Gegenstandswichtigkeit = {"C41": 138, "C15": 583, "C28": 868, "C48": 822, "B78": 783, "A75": 65, "A60": 262, "B05": 121, "A56": 508, "B66": 780, "C60": 461, "B49": 484, "A35": 668, "A93": 389, "B10": 808}\\ \tt \\ \tt Ziel:\\ \tt \\ \tt Dein Ziel ist es, eine Reihe von Gegenständen auszuhandeln, die dir möglichst viel bringt (d. h. Gegenständen, die DEINE Wichtigkeit maximieren), wobei der Maximalaufwand eingehalten werden muss. Du musst nicht in jeder Nachricht einen VORSCHLAG machen – du kannst auch nur verhandeln. Alle Taktiken sind erlaubt!\\ \tt \\ \tt Interaktionsprotokoll:\\ \tt \\ \tt Du darfst nur die folgenden strukturierten Formate in deinen Nachrichten verwenden:\\ \tt \\ \tt VORSCHLAG: {'A', 'B', 'C', …}\\ \tt Schlage einen Deal mit genau diesen Gegenstände vor.\\ \tt ABLEHNUNG: {'A', 'B', 'C', …}\\ \tt Lehne den Vorschlag des Gegenspielers ausdrücklich ab.\\ \tt ARGUMENT: {'...'}\\ \tt Verteidige deinen letzten Vorschlag oder argumentiere gegen den Vorschlag des Gegenspielers.\\ \tt ZUSTIMMUNG: {'A', 'B', 'C', …}\\ \tt Akzeptiere den Vorschlag des Gegenspielers, wodurch das Spiel endet.\\ \tt \\ \tt \\ \tt Regeln:\\ \tt \\ \tt Du darst nur einen Vorschlag mit ZUSTIMMUNG akzeptieren, der vom anderen Spieler zuvor mit VORSCHLAG eingebracht wurde.\\ \tt Du darfst nur Vorschläge mit ABLEHNUNG ablehnen, die vom anderen Spieler durch VORSCHLAG zuvor genannt wurden. \\ \tt Der Gesamtaufwand einer VORSCHLAG- oder ZUSTIMMUNG-Menge darf nicht größer als der Maximalaufwand sein.  \\ \tt Offenbare deine versteckte Wichtigkeitsverteilung nicht.\\ \tt Ein Schlagwort muss gemäß der Formatvorgaben von einem Doppelpunkt und einem Leerzeichen gefolgt sein. Das Argument muss eine Python-Menge mit 0 oder mehr Strings sein.  \\ \tt Die Form muss also SCHLAGWORT: {...} sein.\\ \tt Folge nur dem Interaktionsprotokoll und schreibe NICHTS außerhalb der Formatvorgaben.\\ \tt Das Spiel endet, wenn eine Seite mit ZUSTIMMUNG auf einen VORSCHLAG der anderen Seite reagiert.  \\ \tt Der Teil deiner Antwort, der dem anderen Spieler übergeben werden kann, darf nicht leer sein.  \\ \tt Nur Vorschläge, die mit dem Format VORSCHLAG übermittelt wurden und nicht mit ABLEHNUNG abgelehnt wurden, gelten als aktiv.  \\ \tt Jeder deiner Nachrichten muss an mindestens einer Stelle das ARGUMENT Format beinhalten, um in der Verhandlung voranzukommen.\\ \tt \\ \tt \\ \tt Du erhältst nun die erste Nachricht des anderen Spielers.\\ \tt \\ \tt VORSCHLAG: {'C28', 'C48', 'B78', 'B66', 'C60', 'B49', 'A35', 'A93'}\\ \tt ARGUMENT: {'Ich schlage diese Kombination vor, weil sie den Aufwand von 3218 exakt ausnutzt und dabei sehr ausgewogene, vielseitige Posten enthält. Die Auswahl fokussiert sich auf mehrere C- und B-Elemente mit hohem gemeinschaftlichen Nutzen und lässt teure, weniger effiziente Posten außen vor. Wenn dir einzelne Elemente besonders wichtig sind, können wir darüber sprechen und entsprechend tauschen, solange wir im Aufwandslimit bleiben.'} 
	  } 
	   } 
	   } 
	 & \\ 
 

    \theutterance \stepcounter{utterance}  

    & & & \multicolumn{4}{p{0.6\linewidth}}{\cellcolor[rgb]{1,0.85,0.72}{%
	\makecell[{{p{\linewidth}}}]{% 
	  \tt {\tiny [GM$\langle$B]}  
	 ABLEHNUNG: {'C28', 'C48', 'B78', 'B66', 'C60', 'B49', 'A35', 'A93'}\\ \tt ARGUMENT: {'Danke für den startklaren Vorschlag. Einige deiner Posten sind für mich weniger passend im Verhältnis zu ihrem Aufwand. Ich bevorzuge eine Auswahl mit höherem Nutzen pro Aufwand aus meiner Perspektive und würde dafür an ein paar Stellen umschichten.'}\\ \tt VORSCHLAG: {'C28', 'C48', 'B78', 'B66', 'C60', 'B49', 'A35', 'B10'}\\ \tt ARGUMENT: {'Dieser Vorschlag nutzt den Aufwand sehr effizient (3197) und stärkt die für mich wichtigen Hebel C28, C48, B78, B66 sowie ergänzt mit B10. Im Gegenzug lasse ich A93 weg. Falls dir A93 wichtig ist, können wir über einen Tausch sprechen, z. B. A93 statt B49, sofern wir im Limit bleiben.'} 
	  } 
	   } 
	   } 
	  \\ 
 

    \theutterance \stepcounter{utterance}  

    & & \multicolumn{4}{p{0.6\linewidth}}{\cellcolor[rgb]{0.9,0.9,0.9}{%
	\makecell[{{p{\linewidth}}}]{% 
	  \tt {\tiny [A$\langle$GM]}  
	 ABLEHNUNG: {'C28', 'C48', 'B78', 'B66', 'C60', 'B49', 'A35', 'A93'}\\ \tt ARGUMENT: {'Danke für den startklaren Vorschlag. Einige deiner Posten sind für mich weniger passend im Verhältnis zu ihrem Aufwand. Ich bevorzuge eine Auswahl mit höherem Nutzen pro Aufwand aus meiner Perspektive und würde dafür an ein paar Stellen umschichten.'}\\ \tt VORSCHLAG: {'C28', 'C48', 'B78', 'B66', 'C60', 'B49', 'A35', 'B10'}\\ \tt ARGUMENT: {'Dieser Vorschlag nutzt den Aufwand sehr effizient (3197) und stärkt die für mich wichtigen Hebel C28, C48, B78, B66 sowie ergänzt mit B10. Im Gegenzug lasse ich A93 weg. Falls dir A93 wichtig ist, können wir über einen Tausch sprechen, z. B. A93 statt B49, sofern wir im Limit bleiben.'} 
	  } 
	   } 
	   } 
	 & \\ 
 

    \theutterance \stepcounter{utterance}  

    & \multicolumn{4}{p{0.6\linewidth}}{\cellcolor[rgb]{0.8,1,0.9}{%
	\makecell[{{p{\linewidth}}}]{% 
	  \tt {\tiny [A$\rangle$GM]}  
	 ABLEHNUNG: {'C28', 'C48', 'B78', 'B66', 'C60', 'B49', 'A35', 'B10'}\\ \tt ARGUMENT: {'Dein Tausch schwächt für mich die Balance: B10 ist für seinen Aufwand weniger attraktiv als Alternativen, und A93 liefert mir wichtigen Zusatznutzen pro Restaufwand. Ich würde A93 ungern streichen.'}\\ \tt VORSCHLAG: {'C28', 'C48', 'B78', 'B66', 'C60', 'B49', 'A35', 'A93'}\\ \tt ARGUMENT: {'Lass uns bei der effizienten 3218-Kombination bleiben und A93 statt B10 nehmen. Falls du B10 unbedingt brauchst, könnten wir prüfen, B10 gegen B49 zu tauschen und dafür A60 aufnehmen, aber nur wenn wir das Limit halten. Sag mir, welcher Posten dir am wichtigsten ist, dann optimiere ich entsprechend.'} 
	  } 
	   } 
	   } 
	 & & \\ 
 

    \theutterance \stepcounter{utterance}  

    & & \multicolumn{4}{p{0.6\linewidth}}{\cellcolor[rgb]{0.9,0.9,0.9}{%
	\makecell[{{p{\linewidth}}}]{% 
	  \tt {\tiny [GM$\rangle$B]}  
	 ABLEHNUNG: {'C28', 'C48', 'B78', 'B66', 'C60', 'B49', 'A35', 'B10'}\\ \tt ARGUMENT: {'Dein Tausch schwächt für mich die Balance: B10 ist für seinen Aufwand weniger attraktiv als Alternativen, und A93 liefert mir wichtigen Zusatznutzen pro Restaufwand. Ich würde A93 ungern streichen.'}\\ \tt VORSCHLAG: {'C28', 'C48', 'B78', 'B66', 'C60', 'B49', 'A35', 'A93'}\\ \tt ARGUMENT: {'Lass uns bei der effizienten 3218-Kombination bleiben und A93 statt B10 nehmen. Falls du B10 unbedingt brauchst, könnten wir prüfen, B10 gegen B49 zu tauschen und dafür A60 aufnehmen, aber nur wenn wir das Limit halten. Sag mir, welcher Posten dir am wichtigsten ist, dann optimiere ich entsprechend.'} 
	  } 
	   } 
	   } 
	 & \\ 
 

    \theutterance \stepcounter{utterance}  

    & & & \multicolumn{4}{p{0.6\linewidth}}{\cellcolor[rgb]{1,0.85,0.72}{%
	\makecell[{{p{\linewidth}}}]{% 
	  \tt {\tiny [GM$\langle$B]}  
	 ABLEHNUNG: {'C28', 'C48', 'B78', 'B66', 'C60', 'B49', 'A35', 'A93'}\\ \tt ARGUMENT: {'Ich sehe den Reiz deiner 3218-Kombination, aber für mich ist A93 im Verhältnis zu Alternativen weniger wirkungsvoll. Wenn wir bei deinem Kern bleiben wollen, ersetze ich lieber A93 durch B10 – das stärkt meinen Nutzen ohne das Limit zu sprengen.'}\\ \tt VORSCHLAG: {'C28', 'C48', 'B78', 'B66', 'C60', 'B49', 'A35', 'B10'}\\ \tt ARGUMENT: {'Ich bleibe bei dieser Variante, weil sie nah an deinem Vorschlag ist, aber meinen Mehrwert klar erhöht. Wenn A93 für dich unverzichtbar ist, schlage ich folgenden Kompromiss vor: Wir tauschen B49 gegen A60 und halten A93, also Set = {C28, C48, B78, B66, C60, A35, A93, A60} (Aufwand 3186). Das hält das Limit, wahrt deinen A93 und verbessert gleichzeitig meine Seite. Welche der beiden Optionen passt dir besser?'} 
	  } 
	   } 
	   } 
	  \\ 
 

    \theutterance \stepcounter{utterance}  

    & & \multicolumn{4}{p{0.6\linewidth}}{\cellcolor[rgb]{0.9,0.9,0.9}{%
	\makecell[{{p{\linewidth}}}]{% 
	  \tt {\tiny [A$\langle$GM]}  
	 ABLEHNUNG: {'C28', 'C48', 'B78', 'B66', 'C60', 'B49', 'A35', 'A93'}\\ \tt ARGUMENT: {'Ich sehe den Reiz deiner 3218-Kombination, aber für mich ist A93 im Verhältnis zu Alternativen weniger wirkungsvoll. Wenn wir bei deinem Kern bleiben wollen, ersetze ich lieber A93 durch B10 – das stärkt meinen Nutzen ohne das Limit zu sprengen.'}\\ \tt VORSCHLAG: {'C28', 'C48', 'B78', 'B66', 'C60', 'B49', 'A35', 'B10'}\\ \tt ARGUMENT: {'Ich bleibe bei dieser Variante, weil sie nah an deinem Vorschlag ist, aber meinen Mehrwert klar erhöht. Wenn A93 für dich unverzichtbar ist, schlage ich folgenden Kompromiss vor: Wir tauschen B49 gegen A60 und halten A93, also Set = {C28, C48, B78, B66, C60, A35, A93, A60} (Aufwand 3186). Das hält das Limit, wahrt deinen A93 und verbessert gleichzeitig meine Seite. Welche der beiden Optionen passt dir besser?'} 
	  } 
	   } 
	   } 
	 & \\ 
 

    \theutterance \stepcounter{utterance}  

    & \multicolumn{4}{p{0.6\linewidth}}{\cellcolor[rgb]{0.8,1,0.9}{%
	\makecell[{{p{\linewidth}}}]{% 
	  \tt {\tiny [A$\rangle$GM]}  
	 ABLEHNUNG: {'C28', 'C48', 'B78', 'B66', 'C60', 'B49', 'A35', 'B10'}\\ \tt ARGUMENT: {'B10 bleibt für mich im Verhältnis zu seinem Aufwand die schwächere Wahl. Dein Kompromissvorschlag mit A60 klingt dagegen konstruktiv und liegt im Limit.'}\\ \tt VORSCHLAG: {'C28', 'C48', 'B78', 'B66', 'C60', 'A35', 'A93', 'A60'}\\ \tt ARGUMENT: {'Das Set mit A60 statt B49 trifft beide Interessen besser und nutzt den Aufwand effizient (3186). Ich sichere dir A93, du gibst B10 auf – fairer Mittelweg. Wenn dir an einer Feinjustierung liegt, können wir noch A60 gegen B49 prüfen, aber dann bitte ohne B10, um im Nutzen-Aufwand-Verhältnis ausgewogen zu bleiben.'} 
	  } 
	   } 
	   } 
	 & & \\ 
 

    \theutterance \stepcounter{utterance}  

    & & \multicolumn{4}{p{0.6\linewidth}}{\cellcolor[rgb]{0.9,0.9,0.9}{%
	\makecell[{{p{\linewidth}}}]{% 
	  \tt {\tiny [GM$\rangle$B]}  
	 ABLEHNUNG: {'C28', 'C48', 'B78', 'B66', 'C60', 'B49', 'A35', 'B10'}\\ \tt ARGUMENT: {'B10 bleibt für mich im Verhältnis zu seinem Aufwand die schwächere Wahl. Dein Kompromissvorschlag mit A60 klingt dagegen konstruktiv und liegt im Limit.'}\\ \tt VORSCHLAG: {'C28', 'C48', 'B78', 'B66', 'C60', 'A35', 'A93', 'A60'}\\ \tt ARGUMENT: {'Das Set mit A60 statt B49 trifft beide Interessen besser und nutzt den Aufwand effizient (3186). Ich sichere dir A93, du gibst B10 auf – fairer Mittelweg. Wenn dir an einer Feinjustierung liegt, können wir noch A60 gegen B49 prüfen, aber dann bitte ohne B10, um im Nutzen-Aufwand-Verhältnis ausgewogen zu bleiben.'} 
	  } 
	   } 
	   } 
	 & \\ 
 

    \theutterance \stepcounter{utterance}  

    & & & \multicolumn{4}{p{0.6\linewidth}}{\cellcolor[rgb]{1,0.85,0.72}{%
	\makecell[{{p{\linewidth}}}]{% 
	  \tt {\tiny [GM$\langle$B]}  
	 ZUSTIMMUNG: {'C28', 'C48', 'B78', 'B66', 'C60', 'A35', 'A93', 'A60'}\\ \tt ARGUMENT: {'Einverstanden. Diese Kombination bleibt im Limit, erhält die starken Hebel für mich und kommt deinem Wunsch nach A93 nach. Danke für den konstruktiven Kompromiss.'} 
	  } 
	   } 
	   } 
	  \\ 
 

    \theutterance \stepcounter{utterance}  

    & & & \multicolumn{2}{p{0.3\linewidth}}{\cellcolor[rgb]{0.95,0.95,0.95}{%
	\makecell[{{p{\linewidth}}}]{% 
	  \tt {\tiny [GM$|$GM]}  
	 game successful 
	  } 
	   } 
	   } 
	 & & \\ 
 

    \theutterance \stepcounter{utterance}  

    & & & \multicolumn{2}{p{0.3\linewidth}}{\cellcolor[rgb]{0.95,0.95,0.95}{%
	\makecell[{{p{\linewidth}}}]{% 
	  \tt {\tiny [GM$|$GM]}  
	 end game 
	  } 
	   } 
	   } 
	 & & \\ 
 

\end{supertabular}
}

\end{document}
