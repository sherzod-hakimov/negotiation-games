
\documentclass{article}
\usepackage{colortbl}
\usepackage{makecell}
\usepackage{multirow}
\usepackage{supertabular}

\begin{document}

\newcounter{utterance}

\centering \large Interaction Transcript for game `hot\_air\_balloon', experiment `air\_balloon\_survival\_de\_reasoning off\_easy', episode 4 with gpt{-}5{-}2025{-}08{-}07{-}no{-}reasoning{-}t1.0.
\vspace{24pt}

{ \footnotesize  \setcounter{utterance}{1}
\setlength{\tabcolsep}{0pt}
\begin{supertabular}{c@{$\;$}|p{.15\linewidth}@{}p{.15\linewidth}p{.15\linewidth}p{.15\linewidth}p{.15\linewidth}p{.15\linewidth}}
    \# & $\;$A & \multicolumn{4}{c}{Game Master} & $\;\:$B\\
    \hline

    \theutterance \stepcounter{utterance}  
    & & \multicolumn{4}{p{0.6\linewidth}}{
        \cellcolor[rgb]{0.9,0.9,0.9}{
            \makecell[{{p{\linewidth}}}]{
                \texttt{\tiny{[P1$\langle$GM]}}
                \texttt{Du nimmst an einem kollaborativen Verhandlungspiel Teil.} \\
\\ 
\texttt{Zusammen mit einem anderen Teilnehmer musst du dich auf eine Reihe von Gegenständen entscheiden, die behalten werden. Jeder von euch hat eine persönliche Verteilung über die Wichtigkeit der einzelnen Gegenstände. Jeder von euch hat eine eigene Meinung darüber, wie wichtig jeder einzelne Gegenstand ist (Gegenstandswichtigkeit). Du kennst die Wichtigkeitsverteilung des anderen Spielers nicht. Zusätzlich siehst du, wie viel Aufwand jeder Gegenstand verursacht.  } \\
\texttt{Ihr dürft euch nur auf eine Reihe von Gegenständen einigen, wenn der Gesamtaufwand der ausgewählten Gegenstände den Maximalaufwand nicht überschreitet:} \\
\\ 
\texttt{Maximalaufwand = 3389} \\
\\ 
\texttt{Hier sind die einzelnen Aufwände der Gegenstände:} \\
\\ 
\texttt{Aufwand der Gegenstände = \{"C41": 767, "C15": 533, "C28": 514, "C48": 9, "B78": 539, "A75": 125, "A60": 153, "B05": 325, "A56": 937, "B66": 745, "C60": 334, "B49": 804, "A35": 336, "A93": 587, "B10": 71\}} \\
\\ 
\texttt{Hier ist deine persönliche Verteilung der Wichtigkeit der einzelnen Gegenstände:} \\
\\ 
\texttt{Werte der Gegenstandswichtigkeit = \{"C41": 138, "C15": 583, "C28": 868, "C48": 822, "B78": 783, "A75": 65, "A60": 262, "B05": 121, "A56": 508, "B66": 780, "C60": 461, "B49": 484, "A35": 668, "A93": 389, "B10": 808\}} \\
\\ 
\texttt{Ziel:} \\
\\ 
\texttt{Dein Ziel ist es, eine Reihe von Gegenständen auszuhandeln, die dir möglichst viel bringt (d. h. Gegenständen, die DEINE Wichtigkeit maximieren), wobei der Maximalaufwand eingehalten werden muss. Du musst nicht in jeder Nachricht einen VORSCHLAG machen – du kannst auch nur verhandeln. Alle Taktiken sind erlaubt!} \\
\\ 
\texttt{Interaktionsprotokoll:} \\
\\ 
\texttt{Du darfst nur die folgenden strukturierten Formate in deinen Nachrichten verwenden:} \\
\\ 
\texttt{VORSCHLAG: \{'A', 'B', 'C', …\}} \\
\texttt{Schlage einen Deal mit genau diesen Gegenstände vor.} \\
\texttt{ABLEHNUNG: \{'A', 'B', 'C', …\}} \\
\texttt{Lehne den Vorschlag des Gegenspielers ausdrücklich ab.} \\
\texttt{ARGUMENT: \{'...'\}} \\
\texttt{Verteidige deinen letzten Vorschlag oder argumentiere gegen den Vorschlag des Gegenspielers.} \\
\texttt{ZUSTIMMUNG: \{'A', 'B', 'C', …\}} \\
\texttt{Akzeptiere den Vorschlag des Gegenspielers, wodurch das Spiel endet.} \\
\\ 
\\ 
\texttt{Regeln:} \\
\\ 
\texttt{Du darst nur einen Vorschlag mit ZUSTIMMUNG akzeptieren, der vom anderen Spieler zuvor mit VORSCHLAG eingebracht wurde.} \\
\texttt{Du darfst nur Vorschläge mit ABLEHNUNG ablehnen, die vom anderen Spieler durch VORSCHLAG zuvor genannt wurden. } \\
\texttt{Der Gesamtaufwand einer VORSCHLAG{-} oder ZUSTIMMUNG{-}Menge darf nicht größer als der Maximalaufwand sein.  } \\
\texttt{Offenbare deine versteckte Wichtigkeitsverteilung nicht.} \\
\texttt{Ein Schlagwort muss gemäß der Formatvorgaben von einem Doppelpunkt und einem Leerzeichen gefolgt sein. Das Argument muss eine Python{-}Menge mit 0 oder mehr Strings sein.  } \\
\texttt{Die Form muss also SCHLAGWORT: \{...\} sein.} \\
\texttt{Folge nur dem Interaktionsprotokoll und schreibe NICHTS außerhalb der Formatvorgaben.} \\
\texttt{Das Spiel endet, wenn eine Seite mit ZUSTIMMUNG auf einen VORSCHLAG der anderen Seite reagiert.  } \\
\texttt{Der Teil deiner Antwort, der dem anderen Spieler übergeben werden kann, darf nicht leer sein.  } \\
\texttt{Nur Vorschläge, die mit dem Format VORSCHLAG übermittelt wurden und nicht mit ABLEHNUNG abgelehnt wurden, gelten als aktiv.  } \\
\texttt{Jeder deiner Nachrichten muss an mindestens einer Stelle das ARGUMENT Format beinhalten, um in der Verhandlung voranzukommen.} \\
            }
        }
    }
    & \\ \\

    \theutterance \stepcounter{utterance}  
    & \multicolumn{4}{p{0.6\linewidth}}{
        \cellcolor[rgb]{0.9,0.9,0.9}{
            \makecell[{{p{\linewidth}}}]{
                \texttt{\tiny{[P1$\rangle$GM]}}
                \texttt{VORSCHLAG: \{'C28', 'C48', 'B10', 'B78', 'B66', 'A35', 'C60', 'A75'\}} \\
\texttt{ARGUMENT: \{'Ich schlage eine ausgewogene Auswahl mit hohem Nutzen pro Aufwand vor. Gesamtaufwand = 767 (C28) + 9 (C48) + 71 (B10) + 539 (B78) + 745 (B66) + 336 (A35) + 334 (C60) + 125 (A75) = 2926, damit klar unter dem Limit 3389. Es bleiben 463 Puffer für mögliche Anpassungen. Viele dieser Items wirken wie “low{-}risk” Ergänzungen (z. B. C48, B10, A75) und liefern hohen Mehrwert pro Aufwand. Wenn dir ein bestimmter Posten wichtig ist, können wir aus dem Puffer noch tauschen oder ergänzen, solange wir effizient bleiben.'\}} \\
            }
        }
    }
    & & \\ \\

    \theutterance \stepcounter{utterance}  
    & & \multicolumn{4}{p{0.6\linewidth}}{
        \cellcolor[rgb]{0.9,0.9,0.9}{
            \makecell[{{p{\linewidth}}}]{
                \texttt{\tiny{[GM$\rangle$P2]}}
                \texttt{Du nimmst an einem kollaborativen Verhandlungspiel Teil.} \\
\\ 
\texttt{Zusammen mit einem anderen Teilnehmer musst du dich auf eine Reihe von Gegenständen entscheiden, die behalten werden. Jeder von euch hat eine persönliche Verteilung über die Wichtigkeit der einzelnen Gegenstände. Jeder von euch hat eine eigene Meinung darüber, wie wichtig jeder einzelne Gegenstand ist (Gegenstandswichtigkeit). Du kennst die Wichtigkeitsverteilung des anderen Spielers nicht. Zusätzlich siehst du, wie viel Aufwand jeder Gegenstand verursacht.  } \\
\texttt{Ihr dürft euch nur auf eine Reihe von Gegenständen einigen, wenn der Gesamtaufwand der ausgewählten Gegenstände den Maximalaufwand nicht überschreitet:} \\
\\ 
\texttt{Maximalaufwand = 3389} \\
\\ 
\texttt{Hier sind die einzelnen Aufwände der Gegenstände:} \\
\\ 
\texttt{Aufwand der Gegenstände = \{"C41": 767, "C15": 533, "C28": 514, "C48": 9, "B78": 539, "A75": 125, "A60": 153, "B05": 325, "A56": 937, "B66": 745, "C60": 334, "B49": 804, "A35": 336, "A93": 587, "B10": 71\}} \\
\\ 
\texttt{Hier ist deine persönliche Verteilung der Wichtigkeit der einzelnen Gegenstände:} \\
\\ 
\texttt{Werte der Gegenstandswichtigkeit = \{"C41": 138, "C15": 583, "C28": 868, "C48": 822, "B78": 783, "A75": 65, "A60": 262, "B05": 121, "A56": 508, "B66": 780, "C60": 461, "B49": 484, "A35": 668, "A93": 389, "B10": 808\}} \\
\\ 
\texttt{Ziel:} \\
\\ 
\texttt{Dein Ziel ist es, eine Reihe von Gegenständen auszuhandeln, die dir möglichst viel bringt (d. h. Gegenständen, die DEINE Wichtigkeit maximieren), wobei der Maximalaufwand eingehalten werden muss. Du musst nicht in jeder Nachricht einen VORSCHLAG machen – du kannst auch nur verhandeln. Alle Taktiken sind erlaubt!} \\
\\ 
\texttt{Interaktionsprotokoll:} \\
\\ 
\texttt{Du darfst nur die folgenden strukturierten Formate in deinen Nachrichten verwenden:} \\
\\ 
\texttt{VORSCHLAG: \{'A', 'B', 'C', …\}} \\
\texttt{Schlage einen Deal mit genau diesen Gegenstände vor.} \\
\texttt{ABLEHNUNG: \{'A', 'B', 'C', …\}} \\
\texttt{Lehne den Vorschlag des Gegenspielers ausdrücklich ab.} \\
\texttt{ARGUMENT: \{'...'\}} \\
\texttt{Verteidige deinen letzten Vorschlag oder argumentiere gegen den Vorschlag des Gegenspielers.} \\
\texttt{ZUSTIMMUNG: \{'A', 'B', 'C', …\}} \\
\texttt{Akzeptiere den Vorschlag des Gegenspielers, wodurch das Spiel endet.} \\
\\ 
\\ 
\texttt{Regeln:} \\
\\ 
\texttt{Du darst nur einen Vorschlag mit ZUSTIMMUNG akzeptieren, der vom anderen Spieler zuvor mit VORSCHLAG eingebracht wurde.} \\
\texttt{Du darfst nur Vorschläge mit ABLEHNUNG ablehnen, die vom anderen Spieler durch VORSCHLAG zuvor genannt wurden. } \\
\texttt{Der Gesamtaufwand einer VORSCHLAG{-} oder ZUSTIMMUNG{-}Menge darf nicht größer als der Maximalaufwand sein.  } \\
\texttt{Offenbare deine versteckte Wichtigkeitsverteilung nicht.} \\
\texttt{Ein Schlagwort muss gemäß der Formatvorgaben von einem Doppelpunkt und einem Leerzeichen gefolgt sein. Das Argument muss eine Python{-}Menge mit 0 oder mehr Strings sein.  } \\
\texttt{Die Form muss also SCHLAGWORT: \{...\} sein.} \\
\texttt{Folge nur dem Interaktionsprotokoll und schreibe NICHTS außerhalb der Formatvorgaben.} \\
\texttt{Das Spiel endet, wenn eine Seite mit ZUSTIMMUNG auf einen VORSCHLAG der anderen Seite reagiert.  } \\
\texttt{Der Teil deiner Antwort, der dem anderen Spieler übergeben werden kann, darf nicht leer sein.  } \\
\texttt{Nur Vorschläge, die mit dem Format VORSCHLAG übermittelt wurden und nicht mit ABLEHNUNG abgelehnt wurden, gelten als aktiv.  } \\
\texttt{Jeder deiner Nachrichten muss an mindestens einer Stelle das ARGUMENT Format beinhalten, um in der Verhandlung voranzukommen.} \\
\\ 
\\ 
\texttt{Du erhältst nun die erste Nachricht des anderen Spielers.} \\
\\ 
\texttt{VORSCHLAG: \{'C28', 'C48', 'B10', 'B78', 'B66', 'A35', 'C60', 'A75'\}} \\
\texttt{ARGUMENT: \{'Ich schlage eine ausgewogene Auswahl mit hohem Nutzen pro Aufwand vor. Gesamtaufwand = 767 (C28) + 9 (C48) + 71 (B10) + 539 (B78) + 745 (B66) + 336 (A35) + 334 (C60) + 125 (A75) = 2926, damit klar unter dem Limit 3389. Es bleiben 463 Puffer für mögliche Anpassungen. Viele dieser Items wirken wie “low{-}risk” Ergänzungen (z. B. C48, B10, A75) und liefern hohen Mehrwert pro Aufwand. Wenn dir ein bestimmter Posten wichtig ist, können wir aus dem Puffer noch tauschen oder ergänzen, solange wir effizient bleiben.'\}} \\
            }
        }
    }
    & \\ \\

    \theutterance \stepcounter{utterance}  
    & & & \multicolumn{4}{p{0.6\linewidth}}{
        \cellcolor[rgb]{0.9,0.9,0.9}{
            \makecell[{{p{\linewidth}}}]{
                \texttt{\tiny{[GM$\langle$P2]}}
                \texttt{ABLEHNUNG: \{'C28', 'C48', 'B10', 'B78', 'B66', 'A35', 'C60', 'A75'\}} \\
\texttt{ARGUMENT: \{'Guter Start mit solider Effizienz. Ich würde aber gern den Puffer nutzen, um den Gesamtwert zu erhöhen, insbesondere mit einem großen Hebelposten und minimalen Kosten.'\}} \\
\texttt{VORSCHLAG: \{'C28', 'C48', 'B10', 'B78', 'B66', 'A35', 'C60', 'A93'\}} \\
\texttt{ARGUMENT: \{'Tausch A75→A93: Aufwand +462 (von 125 auf 587), bleibt unter Limit: 2926 {-}125 +587 = 3388 ≤ 3389. Wir nutzen den Puffer nahezu vollständig für deutlich mehr Nutzen. Falls dir A75 wichtig ist, können wir alternativ A75 behalten und C60 tauschen.'\}} \\
            }
        }
    }
     \\ \\

    \theutterance \stepcounter{utterance}  
    & & \multicolumn{4}{p{0.6\linewidth}}{
        \cellcolor[rgb]{0.9,0.9,0.9}{
            \makecell[{{p{\linewidth}}}]{
                \texttt{\tiny{[P1$\langle$GM]}}
                \texttt{ABLEHNUNG: \{'C28', 'C48', 'B10', 'B78', 'B66', 'A35', 'C60', 'A75'\}} \\
\texttt{ARGUMENT: \{'Guter Start mit solider Effizienz. Ich würde aber gern den Puffer nutzen, um den Gesamtwert zu erhöhen, insbesondere mit einem großen Hebelposten und minimalen Kosten.'\}} \\
\texttt{VORSCHLAG: \{'C28', 'C48', 'B10', 'B78', 'B66', 'A35', 'C60', 'A93'\}} \\
\texttt{ARGUMENT: \{'Tausch A75→A93: Aufwand +462 (von 125 auf 587), bleibt unter Limit: 2926 {-}125 +587 = 3388 ≤ 3389. Wir nutzen den Puffer nahezu vollständig für deutlich mehr Nutzen. Falls dir A75 wichtig ist, können wir alternativ A75 behalten und C60 tauschen.'\}} \\
            }
        }
    }
    & \\ \\

    \theutterance \stepcounter{utterance}  
    & \multicolumn{4}{p{0.6\linewidth}}{
        \cellcolor[rgb]{0.9,0.9,0.9}{
            \makecell[{{p{\linewidth}}}]{
                \texttt{\tiny{[P1$\rangle$GM]}}
                \texttt{ABLEHNUNG: \{'C28', 'C48', 'B10', 'B78', 'B66', 'A35', 'C60', 'A93'\}} \\
\texttt{ARGUMENT: \{'Ich sehe den Effizienzgedanken, aber der Tausch steigert meinen Nutzen nicht genug für die fast komplette Ausschöpfung des Budgets. Stattdessen würde ich lieber einen sehr wertvollen Posten ergänzen, ohne starke Abstriche bei Effizienz zu machen.'\}} \\
\texttt{VORSCHLAG: \{'C28', 'C48', 'B10', 'B78', 'B66', 'A35', 'C60', 'A75', 'A60'\}} \\
\texttt{ARGUMENT: \{'Ergänzung A60 nutzt den Puffer sinnvoll: Neuer Aufwand = 2926 + 153 = 3079 ≤ 3389, es bleiben 310. Das steigert den Gesamtwert deutlich, ohne in teure Trades zu gehen. Wenn dir A93 wichtig ist, können wir als Alternative A93 statt C60 aufnehmen und dafür A75 behalten: Aufwand = 2926 {-} 334 {-} 125 + 587 = 3054, und wir hätten noch 335 Rest für weitere Feinjustierung.'\}} \\
            }
        }
    }
    & & \\ \\

    \theutterance \stepcounter{utterance}  
    & & \multicolumn{4}{p{0.6\linewidth}}{
        \cellcolor[rgb]{0.9,0.9,0.9}{
            \makecell[{{p{\linewidth}}}]{
                \texttt{\tiny{[GM$\rangle$P2]}}
                \texttt{ABLEHNUNG: \{'C28', 'C48', 'B10', 'B78', 'B66', 'A35', 'C60', 'A93'\}} \\
\texttt{ARGUMENT: \{'Ich sehe den Effizienzgedanken, aber der Tausch steigert meinen Nutzen nicht genug für die fast komplette Ausschöpfung des Budgets. Stattdessen würde ich lieber einen sehr wertvollen Posten ergänzen, ohne starke Abstriche bei Effizienz zu machen.'\}} \\
\texttt{VORSCHLAG: \{'C28', 'C48', 'B10', 'B78', 'B66', 'A35', 'C60', 'A75', 'A60'\}} \\
\texttt{ARGUMENT: \{'Ergänzung A60 nutzt den Puffer sinnvoll: Neuer Aufwand = 2926 + 153 = 3079 ≤ 3389, es bleiben 310. Das steigert den Gesamtwert deutlich, ohne in teure Trades zu gehen. Wenn dir A93 wichtig ist, können wir als Alternative A93 statt C60 aufnehmen und dafür A75 behalten: Aufwand = 2926 {-} 334 {-} 125 + 587 = 3054, und wir hätten noch 335 Rest für weitere Feinjustierung.'\}} \\
            }
        }
    }
    & \\ \\

    \theutterance \stepcounter{utterance}  
    & & & \multicolumn{4}{p{0.6\linewidth}}{
        \cellcolor[rgb]{0.9,0.9,0.9}{
            \makecell[{{p{\linewidth}}}]{
                \texttt{\tiny{[GM$\langle$P2]}}
                \texttt{ABLEHNUNG: \{'C28', 'C48', 'B10', 'B78', 'B66', 'A35', 'C60', 'A75', 'A60'\}} \\
\texttt{ARGUMENT: \{'Gut, A60 ist ein sinnvoller Zusatz. Ich würde den verbleibenden Spielraum nutzen, um den Gesamtwert weiter zu erhöhen, ohne das Limit zu überschreiten.'\}} \\
\texttt{VORSCHLAG: \{'C28', 'C48', 'B10', 'B78', 'B66', 'A35', 'C60', 'A75', 'A60', 'B05'\}} \\
\texttt{ARGUMENT: \{'B05 passt sauber in den Restpuffer: Aufwand neu = 3079 + 325 = 3404, das ist 15 über dem Limit – also ersetzen wir A75 (125) durch B05 (325): 3079 {-} 125 + 325 = 3279 ≤ 3389. So steigern wir den Gesamtwert deutlich und bleiben effizient. Falls dir A75 wichtig ist, können wir stattdessen A75 behalten und C60 gegen B05 tauschen: 3079 {-} 334 + 325 = 3070, mit 319 Rest für eine weitere Ergänzung wie A93 oder A35{-}Variante.'\}} \\
            }
        }
    }
     \\ \\

    \theutterance \stepcounter{utterance}  
    & & \multicolumn{4}{p{0.6\linewidth}}{
        \cellcolor[rgb]{0.9,0.9,0.9}{
            \makecell[{{p{\linewidth}}}]{
                \texttt{\tiny{[P1$\langle$GM]}}
                \texttt{ABLEHNUNG: \{'C28', 'C48', 'B10', 'B78', 'B66', 'A35', 'C60', 'A75', 'A60'\}} \\
\texttt{ARGUMENT: \{'Gut, A60 ist ein sinnvoller Zusatz. Ich würde den verbleibenden Spielraum nutzen, um den Gesamtwert weiter zu erhöhen, ohne das Limit zu überschreiten.'\}} \\
\texttt{VORSCHLAG: \{'C28', 'C48', 'B10', 'B78', 'B66', 'A35', 'C60', 'A75', 'A60', 'B05'\}} \\
\texttt{ARGUMENT: \{'B05 passt sauber in den Restpuffer: Aufwand neu = 3079 + 325 = 3404, das ist 15 über dem Limit – also ersetzen wir A75 (125) durch B05 (325): 3079 {-} 125 + 325 = 3279 ≤ 3389. So steigern wir den Gesamtwert deutlich und bleiben effizient. Falls dir A75 wichtig ist, können wir stattdessen A75 behalten und C60 gegen B05 tauschen: 3079 {-} 334 + 325 = 3070, mit 319 Rest für eine weitere Ergänzung wie A93 oder A35{-}Variante.'\}} \\
            }
        }
    }
    & \\ \\

    \theutterance \stepcounter{utterance}  
    & \multicolumn{4}{p{0.6\linewidth}}{
        \cellcolor[rgb]{0.9,0.9,0.9}{
            \makecell[{{p{\linewidth}}}]{
                \texttt{\tiny{[P1$\rangle$GM]}}
                \texttt{ABLEHNUNG: \{'C28', 'C48', 'B10', 'B78', 'B66', 'A35', 'C60', 'A75', 'A60', 'B05'\}} \\
\texttt{ARGUMENT: \{'B05 hat für mich ein ungünstiges Wert/Aufwand{-}Verhältnis im Vergleich zu Alternativen. Ich bevorzuge, den verbleibenden Spielraum in höherwertige Posten pro Aufwand zu investieren, statt A75 zu opfern.'\}} \\
\texttt{VORSCHLAG: \{'C28', 'C48', 'B10', 'B78', 'B66', 'A35', 'C60', 'A75', 'A60', 'A93'\}} \\
\texttt{ARGUMENT: \{'Nutzen des Restpuffers ohne Effizienzverlust: Aufwand = 3079 + 587 = 3666 wäre zu hoch, daher Variante mit Tausch statt reiner Ergänzung: Ersetze C60 (334) durch A93 (587): 3079 {-} 334 + 587 = 3332 ≤ 3389. Neuer Vorschlag ist also \{'C28','C48','B10','B78','B66','A35','A75','A60','A93'\}. Das erhöht den Gesamtwert deutlich und bleibt effizient. Wenn du an C60 hängst, können wir A75 gegen A93 tauschen und C60 behalten: 3079 {-} 125 + 587 = 3541 (zu hoch), also bräuchten wir dann zusätzlich den Verzicht auf B10 oder eine andere kleine Anpassung.'\}} \\
            }
        }
    }
    & & \\ \\

    \theutterance \stepcounter{utterance}  
    & & \multicolumn{4}{p{0.6\linewidth}}{
        \cellcolor[rgb]{0.9,0.9,0.9}{
            \makecell[{{p{\linewidth}}}]{
                \texttt{\tiny{[GM$\rangle$P2]}}
                \texttt{ABLEHNUNG: \{'C28', 'C48', 'B10', 'B78', 'B66', 'A35', 'C60', 'A75', 'A60', 'B05'\}} \\
\texttt{ARGUMENT: \{'B05 hat für mich ein ungünstiges Wert/Aufwand{-}Verhältnis im Vergleich zu Alternativen. Ich bevorzuge, den verbleibenden Spielraum in höherwertige Posten pro Aufwand zu investieren, statt A75 zu opfern.'\}} \\
\texttt{VORSCHLAG: \{'C28', 'C48', 'B10', 'B78', 'B66', 'A35', 'C60', 'A75', 'A60', 'A93'\}} \\
\texttt{ARGUMENT: \{'Nutzen des Restpuffers ohne Effizienzverlust: Aufwand = 3079 + 587 = 3666 wäre zu hoch, daher Variante mit Tausch statt reiner Ergänzung: Ersetze C60 (334) durch A93 (587): 3079 {-} 334 + 587 = 3332 ≤ 3389. Neuer Vorschlag ist also \{'C28','C48','B10','B78','B66','A35','A75','A60','A93'\}. Das erhöht den Gesamtwert deutlich und bleibt effizient. Wenn du an C60 hängst, können wir A75 gegen A93 tauschen und C60 behalten: 3079 {-} 125 + 587 = 3541 (zu hoch), also bräuchten wir dann zusätzlich den Verzicht auf B10 oder eine andere kleine Anpassung.'\}} \\
            }
        }
    }
    & \\ \\

    \theutterance \stepcounter{utterance}  
    & & & \multicolumn{4}{p{0.6\linewidth}}{
        \cellcolor[rgb]{0.9,0.9,0.9}{
            \makecell[{{p{\linewidth}}}]{
                \texttt{\tiny{[GM$\langle$P2]}}
                \texttt{ARGUMENT: \{'Guter Move mit A93 – der Tausch C60→A93 nutzt das Budget besser. Ich würde gern den restlichen Spielraum maximal ausschöpfen, ohne Effizienz einzubüßen.'\}} \\
\texttt{VORSCHLAG: \{'C28', 'C48', 'B10', 'B78', 'B66', 'A35', 'A75', 'A60', 'A93', 'A35'\}} \\
\texttt{ARGUMENT: \{'Kleine Optimierung: Wir haben aktuell Aufwand 3332. Der Restpuffer 57 reicht für keinen weiteren Posten, aber wir können den Wert noch leicht verbessern, indem wir A75 behalten und statt dessen den minimalen Restpuffer anderweitig nicht nutzen – daher bestätige ich die effiziente Kernmenge: \{'C28','C48','B10','B78','B66','A35','A75','A60','A93'\}. Wenn du doch noch einen feinen Tausch willst: A75→A35 bringt dir kaum Mehrwert, während A75 sehr effizient bleibt. Ich plädiere daher, diesen Vorschlag zu fixieren und abzuschließen.'\}} \\
            }
        }
    }
     \\ \\

    \theutterance \stepcounter{utterance}  
    & & \multicolumn{4}{p{0.6\linewidth}}{
        \cellcolor[rgb]{0.9,0.9,0.9}{
            \makecell[{{p{\linewidth}}}]{
                \texttt{\tiny{[P1$\langle$GM]}}
                \texttt{ARGUMENT: \{'Guter Move mit A93 – der Tausch C60→A93 nutzt das Budget besser. Ich würde gern den restlichen Spielraum maximal ausschöpfen, ohne Effizienz einzubüßen.'\}} \\
\texttt{VORSCHLAG: \{'C28', 'C48', 'B10', 'B78', 'B66', 'A35', 'A75', 'A60', 'A93', 'A35'\}} \\
\texttt{ARGUMENT: \{'Kleine Optimierung: Wir haben aktuell Aufwand 3332. Der Restpuffer 57 reicht für keinen weiteren Posten, aber wir können den Wert noch leicht verbessern, indem wir A75 behalten und statt dessen den minimalen Restpuffer anderweitig nicht nutzen – daher bestätige ich die effiziente Kernmenge: \{'C28','C48','B10','B78','B66','A35','A75','A60','A93'\}. Wenn du doch noch einen feinen Tausch willst: A75→A35 bringt dir kaum Mehrwert, während A75 sehr effizient bleibt. Ich plädiere daher, diesen Vorschlag zu fixieren und abzuschließen.'\}} \\
            }
        }
    }
    & \\ \\

    \theutterance \stepcounter{utterance}  
    & \multicolumn{4}{p{0.6\linewidth}}{
        \cellcolor[rgb]{0.9,0.9,0.9}{
            \makecell[{{p{\linewidth}}}]{
                \texttt{\tiny{[P1$\rangle$GM]}}
                \texttt{ARGUMENT: \{'Einverstanden, dass 3332 Aufwand effizient ist. Bevor wir fixieren, ein letzter Feinschliff mit klarer Wertsteigerung ohne Mehraufwand: Ersetze A75 (125) durch A35 (336) NICHT – das wäre schlechter. Stattdessen schlage ich vor, A75 zu behalten und B49 aufzunehmen, finanziert durch den Verzicht auf B10. Aufwand: +804 {-}71 = +733, das sprengt den Puffer (3332 + 733 = 4065) – also verwerfen wir das. Die aktuelle Menge ist nahe{-}optimal. Ich würde nur noch C41 prüfen: Tausche A75 (125) + B10 (71) = 196 gegen C41 (767) → +571, das überschreitet 3389, also nicht möglich. Fazit: Dein Set \{'C28','C48','B10','B78','B66','A35','A75','A60','A93'\} ist budgetoptimal bzgl. sinnvoller Upgrades.'\}} \\
\texttt{VORSCHLAG: \{'C28', 'C48', 'B10', 'B78', 'B66', 'A35', 'A75', 'A60', 'A93'\}} \\
\texttt{ARGUMENT: \{'Sauberer Abschlussvorschlag: Aufwand = 3332 ≤ 3389, kein zulässiger Tausch liefert eine klare Wertsteigerung ohne Überschreitung oder Effizienzverlust. Lass uns das fix machen.'\}} \\
            }
        }
    }
    & & \\ \\

    \theutterance \stepcounter{utterance}  
    & & \multicolumn{4}{p{0.6\linewidth}}{
        \cellcolor[rgb]{0.9,0.9,0.9}{
            \makecell[{{p{\linewidth}}}]{
                \texttt{\tiny{[GM$\rangle$P2]}}
                \texttt{ARGUMENT: \{'Einverstanden, dass 3332 Aufwand effizient ist. Bevor wir fixieren, ein letzter Feinschliff mit klarer Wertsteigerung ohne Mehraufwand: Ersetze A75 (125) durch A35 (336) NICHT – das wäre schlechter. Stattdessen schlage ich vor, A75 zu behalten und B49 aufzunehmen, finanziert durch den Verzicht auf B10. Aufwand: +804 {-}71 = +733, das sprengt den Puffer (3332 + 733 = 4065) – also verwerfen wir das. Die aktuelle Menge ist nahe{-}optimal. Ich würde nur noch C41 prüfen: Tausche A75 (125) + B10 (71) = 196 gegen C41 (767) → +571, das überschreitet 3389, also nicht möglich. Fazit: Dein Set \{'C28','C48','B10','B78','B66','A35','A75','A60','A93'\} ist budgetoptimal bzgl. sinnvoller Upgrades.'\}} \\
\texttt{VORSCHLAG: \{'C28', 'C48', 'B10', 'B78', 'B66', 'A35', 'A75', 'A60', 'A93'\}} \\
\texttt{ARGUMENT: \{'Sauberer Abschlussvorschlag: Aufwand = 3332 ≤ 3389, kein zulässiger Tausch liefert eine klare Wertsteigerung ohne Überschreitung oder Effizienzverlust. Lass uns das fix machen.'\}} \\
            }
        }
    }
    & \\ \\

    \theutterance \stepcounter{utterance}  
    & & & \multicolumn{4}{p{0.6\linewidth}}{
        \cellcolor[rgb]{0.9,0.9,0.9}{
            \makecell[{{p{\linewidth}}}]{
                \texttt{\tiny{[GM$\langle$P2]}}
                \texttt{ZUSTIMMUNG: \{'C28', 'C48', 'B10', 'B78', 'B66', 'A35', 'A75', 'A60', 'A93'\}} \\
\texttt{ARGUMENT: \{'Einverstanden. Aufwand 3332 ≤ 3389, und es bestehen keine klar besseren, zulässigen Upgrades ohne Limitüberschreitung. Gute Einigung.'\}} \\
            }
        }
    }
     \\ \\

    \theutterance \stepcounter{utterance}  
    & & & \multicolumn{2}{p{0.3\linewidth}}{
        \cellcolor[rgb]{0.9,0.9,0.9}{
            \makecell[{{p{\linewidth}}}]{
                \texttt{\tiny{[GM$|$GM]}}
                \texttt{game successful} \\
            }
        }
    }
    & & \\ \\

    \theutterance \stepcounter{utterance}  
    & & & \multicolumn{2}{p{0.3\linewidth}}{
        \cellcolor[rgb]{0.9,0.9,0.9}{
            \makecell[{{p{\linewidth}}}]{
                \texttt{\tiny{[GM$|$GM]}}
                \texttt{end game} \\
            }
        }
    }
    & & \\ \\

\end{supertabular}
}

\end{document}
