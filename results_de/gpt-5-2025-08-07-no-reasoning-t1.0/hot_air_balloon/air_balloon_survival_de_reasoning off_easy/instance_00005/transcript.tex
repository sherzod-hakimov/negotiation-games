
\documentclass{article}
\usepackage{colortbl}
\usepackage{makecell}
\usepackage{multirow}
\usepackage{supertabular}

\begin{document}

\newcounter{utterance}

\twocolumn

{ \footnotesize  \setcounter{utterance}{1}
\setlength{\tabcolsep}{0pt}
\begin{supertabular}{c@{$\;$}|p{.15\linewidth}@{}p{.15\linewidth}p{.15\linewidth}p{.15\linewidth}p{.15\linewidth}p{.15\linewidth}}

    \# & $\;$A & \multicolumn{4}{c}{Game Master} & $\;\:$B\\
    \hline 

    \theutterance \stepcounter{utterance}  

    & & \multicolumn{4}{p{0.6\linewidth}}{\cellcolor[rgb]{0.9,0.9,0.9}{%
	\makecell[{{p{\linewidth}}}]{% 
	  \tt {\tiny [A$\langle$GM]}  
	 Du nimmst an einem kollaborativen Verhandlungspiel Teil.\\ \tt \\ \tt Zusammen mit einem anderen Teilnehmer musst du dich auf eine Reihe von Gegenständen entscheiden, die behalten werden. Jeder von euch hat eine persönliche Verteilung über die Wichtigkeit der einzelnen Gegenstände. Jeder von euch hat eine eigene Meinung darüber, wie wichtig jeder einzelne Gegenstand ist (Gegenstandswichtigkeit). Du kennst die Wichtigkeitsverteilung des anderen Spielers nicht. Zusätzlich siehst du, wie viel Aufwand jeder Gegenstand verursacht.  \\ \tt Ihr dürft euch nur auf eine Reihe von Gegenständen einigen, wenn der Gesamtaufwand der ausgewählten Gegenstände den Maximalaufwand nicht überschreitet:\\ \tt \\ \tt Maximalaufwand = 4915\\ \tt \\ \tt Hier sind die einzelnen Aufwände der Gegenstände:\\ \tt \\ \tt Aufwand der Gegenstände = {"C41": 463, "C15": 287, "C28": 492, "C48": 466, "B78": 935, "A75": 373, "A60": 950, "B05": 760, "A56": 994, "B66": 390, "C60": 836, "B49": 911, "A35": 948, "A93": 81, "B10": 945}\\ \tt \\ \tt Hier ist deine persönliche Verteilung der Wichtigkeit der einzelnen Gegenstände:\\ \tt \\ \tt Werte der Gegenstandswichtigkeit = {"C41": 138, "C15": 583, "C28": 868, "C48": 822, "B78": 783, "A75": 65, "A60": 262, "B05": 121, "A56": 508, "B66": 780, "C60": 461, "B49": 484, "A35": 668, "A93": 389, "B10": 808}\\ \tt \\ \tt Ziel:\\ \tt \\ \tt Dein Ziel ist es, eine Reihe von Gegenständen auszuhandeln, die dir möglichst viel bringt (d. h. Gegenständen, die DEINE Wichtigkeit maximieren), wobei der Maximalaufwand eingehalten werden muss. Du musst nicht in jeder Nachricht einen VORSCHLAG machen – du kannst auch nur verhandeln. Alle Taktiken sind erlaubt!\\ \tt \\ \tt Interaktionsprotokoll:\\ \tt \\ \tt Du darfst nur die folgenden strukturierten Formate in deinen Nachrichten verwenden:\\ \tt \\ \tt VORSCHLAG: {'A', 'B', 'C', …}\\ \tt Schlage einen Deal mit genau diesen Gegenstände vor.\\ \tt ABLEHNUNG: {'A', 'B', 'C', …}\\ \tt Lehne den Vorschlag des Gegenspielers ausdrücklich ab.\\ \tt ARGUMENT: {'...'}\\ \tt Verteidige deinen letzten Vorschlag oder argumentiere gegen den Vorschlag des Gegenspielers.\\ \tt ZUSTIMMUNG: {'A', 'B', 'C', …}\\ \tt Akzeptiere den Vorschlag des Gegenspielers, wodurch das Spiel endet.\\ \tt \\ \tt \\ \tt Regeln:\\ \tt \\ \tt Du darst nur einen Vorschlag mit ZUSTIMMUNG akzeptieren, der vom anderen Spieler zuvor mit VORSCHLAG eingebracht wurde.\\ \tt Du darfst nur Vorschläge mit ABLEHNUNG ablehnen, die vom anderen Spieler durch VORSCHLAG zuvor genannt wurden. \\ \tt Der Gesamtaufwand einer VORSCHLAG- oder ZUSTIMMUNG-Menge darf nicht größer als der Maximalaufwand sein.  \\ \tt Offenbare deine versteckte Wichtigkeitsverteilung nicht.\\ \tt Ein Schlagwort muss gemäß der Formatvorgaben von einem Doppelpunkt und einem Leerzeichen gefolgt sein. Das Argument muss eine Python-Menge mit 0 oder mehr Strings sein.  \\ \tt Die Form muss also SCHLAGWORT: {...} sein.\\ \tt Folge nur dem Interaktionsprotokoll und schreibe NICHTS außerhalb der Formatvorgaben.\\ \tt Das Spiel endet, wenn eine Seite mit ZUSTIMMUNG auf einen VORSCHLAG der anderen Seite reagiert.  \\ \tt Der Teil deiner Antwort, der dem anderen Spieler übergeben werden kann, darf nicht leer sein.  \\ \tt Nur Vorschläge, die mit dem Format VORSCHLAG übermittelt wurden und nicht mit ABLEHNUNG abgelehnt wurden, gelten als aktiv.  \\ \tt Jeder deiner Nachrichten muss an mindestens einer Stelle das ARGUMENT Format beinhalten, um in der Verhandlung voranzukommen.\\ \tt  
	  } 
	   } 
	   } 
	 & \\ 
 

    \theutterance \stepcounter{utterance}  

    & \multicolumn{4}{p{0.6\linewidth}}{\cellcolor[rgb]{0.8,1,0.9}{%
	\makecell[{{p{\linewidth}}}]{% 
	  \tt {\tiny [A$\rangle$GM]}  
	 VORSCHLAG: {'C28', 'C48', 'B78', 'B66', 'A35', 'A93'}\\ \tt ARGUMENT: {'Ich schlage diese Auswahl vor, weil sie einen starken Gesamtwert liefert und den Maximalaufwand einhält. Die Kombination deckt mehrere hochwertige C- und B-Items ab und beinhaltet A35 als wertvolles A-Item bei moderatem Gesamtaufwand. Falls dir einzelne Positionen wichtig sind, kann ich über Anpassungen sprechen, solange der Aufwandrahmen gewahrt bleibt.'} 
	  } 
	   } 
	   } 
	 & & \\ 
 

    \theutterance \stepcounter{utterance}  

    & & \multicolumn{4}{p{0.6\linewidth}}{\cellcolor[rgb]{0.9,0.9,0.9}{%
	\makecell[{{p{\linewidth}}}]{% 
	  \tt {\tiny [GM$\rangle$B]}  
	 Du nimmst an einem kollaborativen Verhandlungspiel Teil.\\ \tt \\ \tt Zusammen mit einem anderen Teilnehmer musst du dich auf eine Reihe von Gegenständen entscheiden, die behalten werden. Jeder von euch hat eine persönliche Verteilung über die Wichtigkeit der einzelnen Gegenstände. Jeder von euch hat eine eigene Meinung darüber, wie wichtig jeder einzelne Gegenstand ist (Gegenstandswichtigkeit). Du kennst die Wichtigkeitsverteilung des anderen Spielers nicht. Zusätzlich siehst du, wie viel Aufwand jeder Gegenstand verursacht.  \\ \tt Ihr dürft euch nur auf eine Reihe von Gegenständen einigen, wenn der Gesamtaufwand der ausgewählten Gegenstände den Maximalaufwand nicht überschreitet:\\ \tt \\ \tt Maximalaufwand = 4915\\ \tt \\ \tt Hier sind die einzelnen Aufwände der Gegenstände:\\ \tt \\ \tt Aufwand der Gegenstände = {"C41": 463, "C15": 287, "C28": 492, "C48": 466, "B78": 935, "A75": 373, "A60": 950, "B05": 760, "A56": 994, "B66": 390, "C60": 836, "B49": 911, "A35": 948, "A93": 81, "B10": 945}\\ \tt \\ \tt Hier ist deine persönliche Verteilung der Wichtigkeit der einzelnen Gegenstände:\\ \tt \\ \tt Werte der Gegenstandswichtigkeit = {"C41": 138, "C15": 583, "C28": 868, "C48": 822, "B78": 783, "A75": 65, "A60": 262, "B05": 121, "A56": 508, "B66": 780, "C60": 461, "B49": 484, "A35": 668, "A93": 389, "B10": 808}\\ \tt \\ \tt Ziel:\\ \tt \\ \tt Dein Ziel ist es, eine Reihe von Gegenständen auszuhandeln, die dir möglichst viel bringt (d. h. Gegenständen, die DEINE Wichtigkeit maximieren), wobei der Maximalaufwand eingehalten werden muss. Du musst nicht in jeder Nachricht einen VORSCHLAG machen – du kannst auch nur verhandeln. Alle Taktiken sind erlaubt!\\ \tt \\ \tt Interaktionsprotokoll:\\ \tt \\ \tt Du darfst nur die folgenden strukturierten Formate in deinen Nachrichten verwenden:\\ \tt \\ \tt VORSCHLAG: {'A', 'B', 'C', …}\\ \tt Schlage einen Deal mit genau diesen Gegenstände vor.\\ \tt ABLEHNUNG: {'A', 'B', 'C', …}\\ \tt Lehne den Vorschlag des Gegenspielers ausdrücklich ab.\\ \tt ARGUMENT: {'...'}\\ \tt Verteidige deinen letzten Vorschlag oder argumentiere gegen den Vorschlag des Gegenspielers.\\ \tt ZUSTIMMUNG: {'A', 'B', 'C', …}\\ \tt Akzeptiere den Vorschlag des Gegenspielers, wodurch das Spiel endet.\\ \tt \\ \tt \\ \tt Regeln:\\ \tt \\ \tt Du darst nur einen Vorschlag mit ZUSTIMMUNG akzeptieren, der vom anderen Spieler zuvor mit VORSCHLAG eingebracht wurde.\\ \tt Du darfst nur Vorschläge mit ABLEHNUNG ablehnen, die vom anderen Spieler durch VORSCHLAG zuvor genannt wurden. \\ \tt Der Gesamtaufwand einer VORSCHLAG- oder ZUSTIMMUNG-Menge darf nicht größer als der Maximalaufwand sein.  \\ \tt Offenbare deine versteckte Wichtigkeitsverteilung nicht.\\ \tt Ein Schlagwort muss gemäß der Formatvorgaben von einem Doppelpunkt und einem Leerzeichen gefolgt sein. Das Argument muss eine Python-Menge mit 0 oder mehr Strings sein.  \\ \tt Die Form muss also SCHLAGWORT: {...} sein.\\ \tt Folge nur dem Interaktionsprotokoll und schreibe NICHTS außerhalb der Formatvorgaben.\\ \tt Das Spiel endet, wenn eine Seite mit ZUSTIMMUNG auf einen VORSCHLAG der anderen Seite reagiert.  \\ \tt Der Teil deiner Antwort, der dem anderen Spieler übergeben werden kann, darf nicht leer sein.  \\ \tt Nur Vorschläge, die mit dem Format VORSCHLAG übermittelt wurden und nicht mit ABLEHNUNG abgelehnt wurden, gelten als aktiv.  \\ \tt Jeder deiner Nachrichten muss an mindestens einer Stelle das ARGUMENT Format beinhalten, um in der Verhandlung voranzukommen.\\ \tt \\ \tt \\ \tt Du erhältst nun die erste Nachricht des anderen Spielers.\\ \tt \\ \tt VORSCHLAG: {'C28', 'C48', 'B78', 'B66', 'A35', 'A93'}\\ \tt ARGUMENT: {'Ich schlage diese Auswahl vor, weil sie einen starken Gesamtwert liefert und den Maximalaufwand einhält. Die Kombination deckt mehrere hochwertige C- und B-Items ab und beinhaltet A35 als wertvolles A-Item bei moderatem Gesamtaufwand. Falls dir einzelne Positionen wichtig sind, kann ich über Anpassungen sprechen, solange der Aufwandrahmen gewahrt bleibt.'} 
	  } 
	   } 
	   } 
	 & \\ 
 

    \theutterance \stepcounter{utterance}  

    & & & \multicolumn{4}{p{0.6\linewidth}}{\cellcolor[rgb]{1,0.85,0.72}{%
	\makecell[{{p{\linewidth}}}]{% 
	  \tt {\tiny [GM$\langle$B]}  
	 ABLEHNUNG: {'C28', 'C48', 'B78', 'B66', 'A35', 'A93'}\\ \tt ARGUMENT: {'Danke für den Auftakt. Deine Auswahl ist für mich zu teuer im Gegenwert, weil sie mehrere Posten mit für mich geringerer Priorität bindet und wenig Spielraum lässt. Ich würde gern den Fokus stärker auf Items mit höherem gemeinsamen Nutzen legen und dabei den Aufwand effizienter nutzen.'}\\ \tt VORSCHLAG: {'C28', 'C48', 'B66', 'B10', 'A93'}\\ \tt ARGUMENT: {'Dieser Vorschlag bleibt deutlich unter dem Maximalaufwand und bündelt aus meiner Sicht sehr starke Positionen: C28, C48 und B66 sind hochwertige Eckpfeiler; B10 ergänzt mit solidem Mehrwert, A93 ist ein kostengünstiger Füller. Durch den freien Aufwandsspielraum können wir bei Bedarf noch feinjustieren (z. B. Tausch von B10 gegen A35 oder Hinzunahme eines kleinen Postens), ohne das Limit zu gefährden.'} 
	  } 
	   } 
	   } 
	  \\ 
 

    \theutterance \stepcounter{utterance}  

    & & \multicolumn{4}{p{0.6\linewidth}}{\cellcolor[rgb]{0.9,0.9,0.9}{%
	\makecell[{{p{\linewidth}}}]{% 
	  \tt {\tiny [A$\langle$GM]}  
	 ABLEHNUNG: {'C28', 'C48', 'B78', 'B66', 'A35', 'A93'}\\ \tt ARGUMENT: {'Danke für den Auftakt. Deine Auswahl ist für mich zu teuer im Gegenwert, weil sie mehrere Posten mit für mich geringerer Priorität bindet und wenig Spielraum lässt. Ich würde gern den Fokus stärker auf Items mit höherem gemeinsamen Nutzen legen und dabei den Aufwand effizienter nutzen.'}\\ \tt VORSCHLAG: {'C28', 'C48', 'B66', 'B10', 'A93'}\\ \tt ARGUMENT: {'Dieser Vorschlag bleibt deutlich unter dem Maximalaufwand und bündelt aus meiner Sicht sehr starke Positionen: C28, C48 und B66 sind hochwertige Eckpfeiler; B10 ergänzt mit solidem Mehrwert, A93 ist ein kostengünstiger Füller. Durch den freien Aufwandsspielraum können wir bei Bedarf noch feinjustieren (z. B. Tausch von B10 gegen A35 oder Hinzunahme eines kleinen Postens), ohne das Limit zu gefährden.'} 
	  } 
	   } 
	   } 
	 & \\ 
 

    \theutterance \stepcounter{utterance}  

    & \multicolumn{4}{p{0.6\linewidth}}{\cellcolor[rgb]{0.8,1,0.9}{%
	\makecell[{{p{\linewidth}}}]{% 
	  \tt {\tiny [A$\rangle$GM]}  
	 ABLEHNUNG: {'C28', 'C48', 'B66', 'B10', 'A93'}\\ \tt ARGUMENT: {'Ich sehe den Kern aus C28, C48 und B66 ebenfalls als stark an. B10 passt für mich jedoch schlechter als eine Alternative mit ähnlichem Aufwand und höherem gemeinsamen Potenzial. Ich würde B10 gegen A35 tauschen und den freien Spielraum für einen weiteren gut passenden Posten nutzen.'}\\ \tt VORSCHLAG: {'C28', 'C48', 'B66', 'A35', 'A93'}\\ \tt ARGUMENT: {'Diese Auswahl bleibt komfortabel unter dem Maximalaufwand, hält die starken Eckpfeiler und ersetzt B10 durch A35, was aus meiner Sicht den Gesamtnutzen verbessert. Mit dem verbleibenden Aufwand können wir bei Bedarf noch einen passenden Zusatzposten prüfen, sofern er für dich attraktiv ist.'} 
	  } 
	   } 
	   } 
	 & & \\ 
 

    \theutterance \stepcounter{utterance}  

    & & \multicolumn{4}{p{0.6\linewidth}}{\cellcolor[rgb]{0.9,0.9,0.9}{%
	\makecell[{{p{\linewidth}}}]{% 
	  \tt {\tiny [GM$\rangle$B]}  
	 ABLEHNUNG: {'C28', 'C48', 'B66', 'B10', 'A93'}\\ \tt ARGUMENT: {'Ich sehe den Kern aus C28, C48 und B66 ebenfalls als stark an. B10 passt für mich jedoch schlechter als eine Alternative mit ähnlichem Aufwand und höherem gemeinsamen Potenzial. Ich würde B10 gegen A35 tauschen und den freien Spielraum für einen weiteren gut passenden Posten nutzen.'}\\ \tt VORSCHLAG: {'C28', 'C48', 'B66', 'A35', 'A93'}\\ \tt ARGUMENT: {'Diese Auswahl bleibt komfortabel unter dem Maximalaufwand, hält die starken Eckpfeiler und ersetzt B10 durch A35, was aus meiner Sicht den Gesamtnutzen verbessert. Mit dem verbleibenden Aufwand können wir bei Bedarf noch einen passenden Zusatzposten prüfen, sofern er für dich attraktiv ist.'} 
	  } 
	   } 
	   } 
	 & \\ 
 

    \theutterance \stepcounter{utterance}  

    & & & \multicolumn{4}{p{0.6\linewidth}}{\cellcolor[rgb]{1,0.85,0.72}{%
	\makecell[{{p{\linewidth}}}]{% 
	  \tt {\tiny [GM$\langle$B]}  
	 ABLEHNUNG: {'C28', 'C48', 'B66', 'A35', 'A93'}\\ \tt ARGUMENT: {'Der Kern C28, C48, B66 ist für mich gesetzt. A35 ist für mich im Verhältnis zum Aufwand weniger attraktiv, während A93 als günstiger Füller gut passt. Ich sehe mehr gemeinsamen Spielraum, wenn wir A35 weglassen und stattdessen einen B-/C-Posten mit besserem Verhältnis ergänzen.'}\\ \tt VORSCHLAG: {'C28', 'C48', 'B66', 'B78', 'A93'}\\ \tt ARGUMENT: {'Diese Variante hält die drei Eckpfeiler, behält A93 als effizienten Zusatz und ersetzt A35 durch B78, was den Gesamtnutzen aus meiner Sicht stärker steigert. Wir bleiben unter dem Maximalaufwand und nutzen das Budget deutlich sinnvoller.'} 
	  } 
	   } 
	   } 
	  \\ 
 

    \theutterance \stepcounter{utterance}  

    & & \multicolumn{4}{p{0.6\linewidth}}{\cellcolor[rgb]{0.9,0.9,0.9}{%
	\makecell[{{p{\linewidth}}}]{% 
	  \tt {\tiny [A$\langle$GM]}  
	 ABLEHNUNG: {'C28', 'C48', 'B66', 'A35', 'A93'}\\ \tt ARGUMENT: {'Der Kern C28, C48, B66 ist für mich gesetzt. A35 ist für mich im Verhältnis zum Aufwand weniger attraktiv, während A93 als günstiger Füller gut passt. Ich sehe mehr gemeinsamen Spielraum, wenn wir A35 weglassen und stattdessen einen B-/C-Posten mit besserem Verhältnis ergänzen.'}\\ \tt VORSCHLAG: {'C28', 'C48', 'B66', 'B78', 'A93'}\\ \tt ARGUMENT: {'Diese Variante hält die drei Eckpfeiler, behält A93 als effizienten Zusatz und ersetzt A35 durch B78, was den Gesamtnutzen aus meiner Sicht stärker steigert. Wir bleiben unter dem Maximalaufwand und nutzen das Budget deutlich sinnvoller.'} 
	  } 
	   } 
	   } 
	 & \\ 
 

    \theutterance \stepcounter{utterance}  

    & \multicolumn{4}{p{0.6\linewidth}}{\cellcolor[rgb]{0.8,1,0.9}{%
	\makecell[{{p{\linewidth}}}]{% 
	  \tt {\tiny [A$\rangle$GM]}  
	 ABLEHNUNG: {'C28', 'C48', 'B66', 'B78', 'A93'}\\ \tt ARGUMENT: {'Den Kern C28, C48, B66 unterstütze ich weiterhin. B78 ist für mich im Vergleich zum Aufwand weniger attraktiv als eine Alternative mit ähnlichem oder besserem Nutzenprofil. Ich würde stattdessen einen C-Posten mit starkem Verhältnis ergänzen und effizient bleiben.'}\\ \tt VORSCHLAG: {'C28', 'C48', 'B66', 'C41', 'A93'}\\ \tt ARGUMENT: {'Diese Zusammenstellung hält die drei Eckpfeiler und ersetzt B78 durch C41. Wir bleiben klar unter dem Maximalaufwand, erhöhen die Effizienz und lassen noch etwas Luft für minimale Anpassungen, falls dir eine kleine Verschiebung wichtig ist.'} 
	  } 
	   } 
	   } 
	 & & \\ 
 

    \theutterance \stepcounter{utterance}  

    & & \multicolumn{4}{p{0.6\linewidth}}{\cellcolor[rgb]{0.9,0.9,0.9}{%
	\makecell[{{p{\linewidth}}}]{% 
	  \tt {\tiny [GM$\rangle$B]}  
	 ABLEHNUNG: {'C28', 'C48', 'B66', 'B78', 'A93'}\\ \tt ARGUMENT: {'Den Kern C28, C48, B66 unterstütze ich weiterhin. B78 ist für mich im Vergleich zum Aufwand weniger attraktiv als eine Alternative mit ähnlichem oder besserem Nutzenprofil. Ich würde stattdessen einen C-Posten mit starkem Verhältnis ergänzen und effizient bleiben.'}\\ \tt VORSCHLAG: {'C28', 'C48', 'B66', 'C41', 'A93'}\\ \tt ARGUMENT: {'Diese Zusammenstellung hält die drei Eckpfeiler und ersetzt B78 durch C41. Wir bleiben klar unter dem Maximalaufwand, erhöhen die Effizienz und lassen noch etwas Luft für minimale Anpassungen, falls dir eine kleine Verschiebung wichtig ist.'} 
	  } 
	   } 
	   } 
	 & \\ 
 

    \theutterance \stepcounter{utterance}  

    & & & \multicolumn{4}{p{0.6\linewidth}}{\cellcolor[rgb]{1,0.85,0.72}{%
	\makecell[{{p{\linewidth}}}]{% 
	  \tt {\tiny [GM$\langle$B]}  
	 ABLEHNUNG: {'C28', 'C48', 'B66', 'C41', 'A93'}\\ \tt ARGUMENT: {'C41 ist für mich im Verhältnis zum Aufwand weniger attraktiv als eine andere Option. Ich möchte den Kern C28, C48, B66 behalten und den freien Rahmen nutzen, um mit einem starken B-Posten deutlicheren Mehrwert zu erzielen.'}\\ \tt VORSCHLAG: {'C28', 'C48', 'B66', 'B49', 'A93'}\\ \tt ARGUMENT: {'Diese Auswahl hält unseren gemeinsamen Kern, ersetzt C41 durch B49 und bleibt deutlich unter dem Maximalaufwand. Aus meiner Sicht liefert B49 in Kombination mit C28/C48/B66 ein runderes Gesamtpaket. Wenn dir an einem kleinen C- oder A-Zusatz liegt, können wir prüfen, ob noch ein sehr günstiger Posten hineinpasst, ohne das Budget zu sprengen.'} 
	  } 
	   } 
	   } 
	  \\ 
 

    \theutterance \stepcounter{utterance}  

    & & \multicolumn{4}{p{0.6\linewidth}}{\cellcolor[rgb]{0.9,0.9,0.9}{%
	\makecell[{{p{\linewidth}}}]{% 
	  \tt {\tiny [A$\langle$GM]}  
	 ABLEHNUNG: {'C28', 'C48', 'B66', 'C41', 'A93'}\\ \tt ARGUMENT: {'C41 ist für mich im Verhältnis zum Aufwand weniger attraktiv als eine andere Option. Ich möchte den Kern C28, C48, B66 behalten und den freien Rahmen nutzen, um mit einem starken B-Posten deutlicheren Mehrwert zu erzielen.'}\\ \tt VORSCHLAG: {'C28', 'C48', 'B66', 'B49', 'A93'}\\ \tt ARGUMENT: {'Diese Auswahl hält unseren gemeinsamen Kern, ersetzt C41 durch B49 und bleibt deutlich unter dem Maximalaufwand. Aus meiner Sicht liefert B49 in Kombination mit C28/C48/B66 ein runderes Gesamtpaket. Wenn dir an einem kleinen C- oder A-Zusatz liegt, können wir prüfen, ob noch ein sehr günstiger Posten hineinpasst, ohne das Budget zu sprengen.'} 
	  } 
	   } 
	   } 
	 & \\ 
 

    \theutterance \stepcounter{utterance}  

    & \multicolumn{4}{p{0.6\linewidth}}{\cellcolor[rgb]{0.8,1,0.9}{%
	\makecell[{{p{\linewidth}}}]{% 
	  \tt {\tiny [A$\rangle$GM]}  
	 ABLEHNUNG: {'C28', 'C48', 'B66', 'B49', 'A93'}\\ \tt ARGUMENT: {'Den Kern C28, C48, B66 trage ich weiterhin mit. B49 ist für mich im Verhältnis zum Aufwand weniger attraktiv als eine schlankere C-Option, die das Gesamtpaket effizienter macht und Raum für eventuelle Feinjustierung lässt. Ich schlage vor, bei einem C-Posten zu bleiben, der besser skaliert.'}\\ \tt VORSCHLAG: {'C28', 'C48', 'B66', 'C60', 'A93'}\\ \tt ARGUMENT: {'Diese Variante hält unseren gemeinsamen Kern, ersetzt B49 durch C60 und bleibt bequem unter dem Maximalaufwand. So erzielen wir hohen Gesamtwert bei solider Effizienz. Falls dir ein anderer C-Posten lieber ist, können wir C60 gegen deine bevorzugte C-Alternative mit ähnlichem Aufwand tauschen.'} 
	  } 
	   } 
	   } 
	 & & \\ 
 

    \theutterance \stepcounter{utterance}  

    & & \multicolumn{4}{p{0.6\linewidth}}{\cellcolor[rgb]{0.9,0.9,0.9}{%
	\makecell[{{p{\linewidth}}}]{% 
	  \tt {\tiny [GM$\rangle$B]}  
	 ABLEHNUNG: {'C28', 'C48', 'B66', 'B49', 'A93'}\\ \tt ARGUMENT: {'Den Kern C28, C48, B66 trage ich weiterhin mit. B49 ist für mich im Verhältnis zum Aufwand weniger attraktiv als eine schlankere C-Option, die das Gesamtpaket effizienter macht und Raum für eventuelle Feinjustierung lässt. Ich schlage vor, bei einem C-Posten zu bleiben, der besser skaliert.'}\\ \tt VORSCHLAG: {'C28', 'C48', 'B66', 'C60', 'A93'}\\ \tt ARGUMENT: {'Diese Variante hält unseren gemeinsamen Kern, ersetzt B49 durch C60 und bleibt bequem unter dem Maximalaufwand. So erzielen wir hohen Gesamtwert bei solider Effizienz. Falls dir ein anderer C-Posten lieber ist, können wir C60 gegen deine bevorzugte C-Alternative mit ähnlichem Aufwand tauschen.'} 
	  } 
	   } 
	   } 
	 & \\ 
 

    \theutterance \stepcounter{utterance}  

    & & & \multicolumn{4}{p{0.6\linewidth}}{\cellcolor[rgb]{1,0.85,0.72}{%
	\makecell[{{p{\linewidth}}}]{% 
	  \tt {\tiny [GM$\langle$B]}  
	 ABLEHNUNG: {'C28', 'C48', 'B66', 'C60', 'A93'}\\ \tt ARGUMENT: {'Danke für die Flexibilität beim C-Posten. C60 ist für mich im Verhältnis zum Aufwand weniger attraktiv als eine andere C-Alternative mit stärkerem Mehrwert. Ich möchte den Kern C28, C48, B66 halten und den C-Slot auf eine Option mit besserem Verhältnis setzen.'}\\ \tt VORSCHLAG: {'C28', 'C48', 'B66', 'C15', 'A93'}\\ \tt ARGUMENT: {'Diese Auswahl bleibt klar unter dem Maximalaufwand und hält unseren gemeinsamen Kern. C15 ist für mich die deutlich attraktivere C-Alternative gegenüber C60, und A93 bleibt als effizienter Füller. Wenn dir ein minimaler Tausch innerhalb der C-Items wichtig ist, können wir über C41 statt C15 sprechen, aber C15 ist mein klarer Favorit.'} 
	  } 
	   } 
	   } 
	  \\ 
 

    \theutterance \stepcounter{utterance}  

    & & \multicolumn{4}{p{0.6\linewidth}}{\cellcolor[rgb]{0.9,0.9,0.9}{%
	\makecell[{{p{\linewidth}}}]{% 
	  \tt {\tiny [A$\langle$GM]}  
	 ABLEHNUNG: {'C28', 'C48', 'B66', 'C60', 'A93'}\\ \tt ARGUMENT: {'Danke für die Flexibilität beim C-Posten. C60 ist für mich im Verhältnis zum Aufwand weniger attraktiv als eine andere C-Alternative mit stärkerem Mehrwert. Ich möchte den Kern C28, C48, B66 halten und den C-Slot auf eine Option mit besserem Verhältnis setzen.'}\\ \tt VORSCHLAG: {'C28', 'C48', 'B66', 'C15', 'A93'}\\ \tt ARGUMENT: {'Diese Auswahl bleibt klar unter dem Maximalaufwand und hält unseren gemeinsamen Kern. C15 ist für mich die deutlich attraktivere C-Alternative gegenüber C60, und A93 bleibt als effizienter Füller. Wenn dir ein minimaler Tausch innerhalb der C-Items wichtig ist, können wir über C41 statt C15 sprechen, aber C15 ist mein klarer Favorit.'} 
	  } 
	   } 
	   } 
	 & \\ 
 

    \theutterance \stepcounter{utterance}  

    & \multicolumn{4}{p{0.6\linewidth}}{\cellcolor[rgb]{0.8,1,0.9}{%
	\makecell[{{p{\linewidth}}}]{% 
	  \tt {\tiny [A$\rangle$GM]}  
	 ZUSTIMMUNG: {'C28', 'C48', 'B66', 'C15', 'A93'}\\ \tt ARGUMENT: {'Einverstanden – dieser Deal hält unseren gemeinsamen Kern, ist effizient im Aufwand und trifft für mich ein sehr gutes Nutzenverhältnis. Danke für die konstruktive Einigung.'} 
	  } 
	   } 
	   } 
	 & & \\ 
 

    \theutterance \stepcounter{utterance}  

    & & & \multicolumn{2}{p{0.3\linewidth}}{\cellcolor[rgb]{0.95,0.95,0.95}{%
	\makecell[{{p{\linewidth}}}]{% 
	  \tt {\tiny [GM$|$GM]}  
	 game successful 
	  } 
	   } 
	   } 
	 & & \\ 
 

    \theutterance \stepcounter{utterance}  

    & & & \multicolumn{2}{p{0.3\linewidth}}{\cellcolor[rgb]{0.95,0.95,0.95}{%
	\makecell[{{p{\linewidth}}}]{% 
	  \tt {\tiny [GM$|$GM]}  
	 end game 
	  } 
	   } 
	   } 
	 & & \\ 
 

\end{supertabular}
}

\end{document}
