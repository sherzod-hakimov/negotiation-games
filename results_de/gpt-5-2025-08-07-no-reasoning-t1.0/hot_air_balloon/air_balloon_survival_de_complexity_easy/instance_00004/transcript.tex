
\documentclass{article}
\usepackage{colortbl}
\usepackage{makecell}
\usepackage{multirow}
\usepackage{supertabular}

\begin{document}

\newcounter{utterance}

\twocolumn

{ \footnotesize  \setcounter{utterance}{1}
\setlength{\tabcolsep}{0pt}
\begin{supertabular}{c@{$\;$}|p{.15\linewidth}@{}p{.15\linewidth}p{.15\linewidth}p{.15\linewidth}p{.15\linewidth}p{.15\linewidth}}

    \# & $\;$A & \multicolumn{4}{c}{Game Master} & $\;\:$B\\
    \hline 

    \theutterance \stepcounter{utterance}  

    & & \multicolumn{4}{p{0.6\linewidth}}{\cellcolor[rgb]{0.9,0.9,0.9}{%
	\makecell[{{p{\linewidth}}}]{% 
	  \tt {\tiny [A$\langle$GM]}  
	 Du nimmst an einem kollaborativen Verhandlungspiel Teil.\\ \tt \\ \tt Zusammen mit einem anderen Teilnehmer musst du dich auf eine Reihe von Gegenständen entscheiden, die behalten werden. Jeder von euch hat eine persönliche Verteilung über die Wichtigkeit der einzelnen Gegenstände. Jeder von euch hat eine eigene Meinung darüber, wie wichtig jeder einzelne Gegenstand ist (Gegenstandswichtigkeit). Du kennst die Wichtigkeitsverteilung des anderen Spielers nicht. Zusätzlich siehst du, wie viel Aufwand jeder Gegenstand verursacht.  \\ \tt Ihr dürft euch nur auf eine Reihe von Gegenständen einigen, wenn der Gesamtaufwand der ausgewählten Gegenstände den Maximalaufwand nicht überschreitet:\\ \tt \\ \tt Maximalaufwand = 9612\\ \tt \\ \tt Hier sind die einzelnen Aufwände der Gegenstände:\\ \tt \\ \tt Aufwand der Gegenstände = {"C32": 935, "C52": 928, "B54": 725, "C16": 218, "C50": 839, "A08": 970, "C59": 394, "B51": 677, "B55": 946, "B59": 530, "A77": 138, "C14": 834, "C24": 734, "B24": 593, "C31": 261, "A52": 743, "C61": 4, "B79": 735, "C54": 124, "B56": 829, "A40": 207, "B11": 782, "B60": 577, "A81": 388, "A95": 679, "B38": 494, "C51": 559, "C03": 629, "A91": 237, "A93": 275, "C28": 39, "C05": 654, "B75": 172, "B96": 688, "B86": 687}\\ \tt \\ \tt Hier ist deine persönliche Verteilung der Wichtigkeit der einzelnen Gegenstände:\\ \tt \\ \tt Werte der Gegenstandswichtigkeit = {"C32": 138, "C52": 583, "B54": 868, "C16": 822, "C50": 783, "A08": 65, "C59": 262, "B51": 121, "B55": 508, "B59": 780, "A77": 461, "C14": 484, "C24": 668, "B24": 389, "C31": 808, "A52": 215, "C61": 97, "B79": 500, "C54": 30, "B56": 915, "A40": 856, "B11": 400, "B60": 444, "A81": 623, "A95": 781, "B38": 786, "C51": 3, "C03": 713, "A91": 457, "A93": 273, "C28": 739, "C05": 822, "B75": 235, "B96": 606, "B86": 968}\\ \tt \\ \tt Ziel:\\ \tt \\ \tt Dein Ziel ist es, eine Reihe von Gegenständen auszuhandeln, die dir möglichst viel bringt (d. h. Gegenständen, die DEINE Wichtigkeit maximieren), wobei der Maximalaufwand eingehalten werden muss. Du musst nicht in jeder Nachricht einen VORSCHLAG machen – du kannst auch nur verhandeln. Alle Taktiken sind erlaubt!\\ \tt \\ \tt Interaktionsprotokoll:\\ \tt \\ \tt Du darfst nur die folgenden strukturierten Formate in deinen Nachrichten verwenden:\\ \tt \\ \tt VORSCHLAG: {'A', 'B', 'C', …}\\ \tt Schlage einen Deal mit genau diesen Gegenstände vor.\\ \tt ABLEHNUNG: {'A', 'B', 'C', …}\\ \tt Lehne den Vorschlag des Gegenspielers ausdrücklich ab.\\ \tt ARGUMENT: {'...'}\\ \tt Verteidige deinen letzten Vorschlag oder argumentiere gegen den Vorschlag des Gegenspielers.\\ \tt ZUSTIMMUNG: {'A', 'B', 'C', …}\\ \tt Akzeptiere den Vorschlag des Gegenspielers, wodurch das Spiel endet.\\ \tt STRATEGISCHE ÜBERLEGUNGEN: {'...'}\\ \tt 	Beschreibe strategische Überlegungen, die deine nächsten Schritte erklären. Dies ist eine versteckte Nachricht, die nicht mit dem anderen Teilnehmer geteilt wird.\\ \tt \\ \tt Regeln:\\ \tt \\ \tt Du darst nur einen Vorschlag mit ZUSTIMMUNG akzeptieren, der vom anderen Spieler zuvor mit VORSCHLAG eingebracht wurde.\\ \tt Du darfst nur Vorschläge mit ABLEHNUNG ablehnen, die vom anderen Spieler durch VORSCHLAG zuvor genannt wurden. \\ \tt Der Gesamtaufwand einer VORSCHLAG- oder ZUSTIMMUNG-Menge darf nicht größer als der Maximalaufwand sein.  \\ \tt Offenbare deine versteckte Wichtigkeitsverteilung nicht.\\ \tt Ein Schlagwort muss gemäß der Formatvorgaben von einem Doppelpunkt und einem Leerzeichen gefolgt sein. Das Argument muss eine Python-Menge mit 0 oder mehr Strings sein.  \\ \tt Die Form muss also SCHLAGWORT: {...} sein.\\ \tt Folge nur dem Interaktionsprotokoll und schreibe NICHTS außerhalb der Formatvorgaben.\\ \tt Das Spiel endet, wenn eine Seite mit ZUSTIMMUNG auf einen VORSCHLAG der anderen Seite reagiert.  \\ \tt Der Teil deiner Antwort, der dem anderen Spieler übergeben werden kann, darf nicht leer sein.  \\ \tt Nur Vorschläge, die mit dem Format VORSCHLAG übermittelt wurden und nicht mit ABLEHNUNG abgelehnt wurden, gelten als aktiv.  \\ \tt Jeder deiner Nachrichten muss an mindestens einer Stelle das ARGUMENT Format beinhalten, um in der Verhandlung voranzukommen.\\ \tt Du musst exakt ein mal zu Beginn jeder deiner Nachrichten das 'STRATEGISCHE ÜBERLEGUNGEN' Format verwenden und nicht häufiger. Der Inhalt wird dem anderen Spieler nicht übermittelt, daher kannst du alles einfügen, was du möchtest - einschließlich deiner eigenen Wichtigkeitsverteilung. Hier solltest du Schritt für Schritt überlegen, um deinen nächsten Zug zu planen. 
	  } 
	   } 
	   } 
	 & \\ 
 

    \theutterance \stepcounter{utterance}  

    & \multicolumn{4}{p{0.6\linewidth}}{\cellcolor[rgb]{0.8,1,0.9}{%
	\makecell[{{p{\linewidth}}}]{% 
	  \tt {\tiny [A$\rangle$GM]}  
	 STRATEGISCHE ÜBERLEGUNGEN: {'Ziel: Maximierung meiner Summe bei Aufwand <= 9612. Ich priorisiere nach Wert/Aufwand-Quote und absolutem Wert. Top-Werte: B86(968/687=1.41), B56(915/829=1.10), B54(868/725=1.20), A40(856/207=4.13), C16(822/218=3.77), C05(822/654=1.26), C31(808/261=3.10), A95(781/679=1.15), B38(786/494=1.59), B59(780/530=1.47), C03(713/629=1.13), C28(739/39=18.95), C24(668/734=0.91), B96(606/688=0.88), A81(623/388=1.60), C52(583/928=0.63), B79(500/735=0.68), B55(508/946=0.54), B60(444/577=0.77), A77(461/138=3.34), A91(457/237=1.93), C14(484/834=0.58), B11(400/782=0.51), C59(262/394=0.66), A93(273/275=0.99), A52(215/743=0.29), B24(389/593=0.66), C50(783/839=0.93), C32(138/935=0.15), C61(97/4=24.25), C54(30/124=0.24), A08(65/970=0.07), C51(3/559=0.01), B75(235/172=1.37). Baue Startvorschlag mit sehr hoher Effizienz: {C61, C28, A40, C16, C31, A77, A91, A81, B38, B59, A95, C05, C03, B54, B56, B86}. Prüfe Aufwand: C61(4)+C28(39)=43; +A40(207)=250; +C16(218)=468; +C31(261)=729; +A77(138)=867; +A91(237)=1104; +A81(388)=1492; +B38(494)=1986; +B59(530)=2516; +A95(679)=3195; +C05(654)=3849; +C03(629)=4478; +B54(725)=5203; +B56(829)=6032; +B86(687)=6719. Viel Luft, kann mehr hinzufügen: füge C50(839)=7558, C24(734)=8292, C14(834)=9126 (grenzwertig), verbleibt 486; füge B60(577) zu viel; B24(593) zu viel; C59(394) passt: 9126+394=9520, noch 92 übrig; B75(172) zu viel; C54(124) zu viel; C28/C61 schon drin; A93(275) zu viel; C32(935) zu viel; C52(928) zu viel; B11(782) zu viel; B55(946) zu viel; C51(559) zu viel; A52(743) zu viel; B79(735) zu viel; A08(970) zu viel. Gesamtaufwand 9520 <= 9612? Nein 9520 ist unter 9612, passt. Summe Werte dieser Menge berechnen grob: C61(97)+C28(739)=836; +A40(856)=1692; +C16(822)=2514; +C31(808)=3322; +A77(461)=3783; +A91(457)=4240; +A81(623)=4863; +B38(786)=5649; +B59(780)=6429; +A95(781)=7210; +C05(822)=8032; +C03(713)=8745; +B54(868)=9613; +B56(915)=10528; +B86(968)=11496; +C50(783)=12279; +C24(668)=12947; +C14(484)=13431; +C59(262)=13693. Sehr gut. Ich könnte evtl. C14 (niedrige Effizienz) weglassen und stattdessen B60(444) hinzufügen, wenn Kapazität passt. Test ohne C14 (-834 Aufwand, -484 Wert): Aufwand 9520-834=8686; kann B60(577) hinzufügen: 9263, verbleibt 349; plus B75(172)=9435; plus C54(124)=9559; plus A93(275) würde 9834 zu viel; plus C59(394) würde 9953 zu viel. Werte: -484 +444 +235 +30 = +225 netto; also besser. So finaler Set: Basisset ohne C14: {C61,C28,A40,C16,C31,A77,A91,A81,B38,B59,A95,C05,C03,B54,B56,B86,C50,C24,C59? nein, wir entfernten C59? In Basisset mit 9520 enthielt C59 ja. War 9520 inkl C59. Dann wir entfernten C14, Aufwand 8686, enthält C59 weiterhin. Danach fügten wir B60(577)=9263, B75(172)=9435, C54(124)=9559. Das ist unter 9612. Werte adjust: Start 13693 - C14(484) = 13209; +B60(444)=13653; +B75(235)=13888; +C54(30)=13918. Noch besser. Das wird mein Vorschlag.}'}\\ \tt VORSCHLAG: {'C61', 'C28', 'A40', 'C16', 'C31', 'A77', 'A91', 'A81', 'B38', 'B59', 'A95', 'C05', 'C03', 'B54', 'B56', 'B86', 'C50', 'C24', 'B60', 'B75', 'C54', 'C59'}\\ \tt ARGUMENT: {'Ich habe einen Vorschlag mit hohem gemeinsamen Nutzen unter strikter Einhaltung des Maximalaufwands vorbereitet. Er fokussiert sich auf viele effiziente, leichtgewichtige Items plus einige starke Kernelemente, und lässt gleichzeitig noch Raum für deine Präferenzen in späteren Anpassungen. Wenn dir einzelne Posten wichtig sind, die hier fehlen, können wir gezielt tauschen – ich bin offen für Substitutionen, solange der Aufwandrahmen gewahrt bleibt.'} 
	  } 
	   } 
	   } 
	 & & \\ 
 

    \theutterance \stepcounter{utterance}  

    & & \multicolumn{4}{p{0.6\linewidth}}{\cellcolor[rgb]{0.9,0.9,0.9}{%
	\makecell[{{p{\linewidth}}}]{% 
	  \tt {\tiny [GM$\rangle$B]}  
	 Du nimmst an einem kollaborativen Verhandlungspiel Teil.\\ \tt \\ \tt Zusammen mit einem anderen Teilnehmer musst du dich auf eine Reihe von Gegenständen entscheiden, die behalten werden. Jeder von euch hat eine persönliche Verteilung über die Wichtigkeit der einzelnen Gegenstände. Jeder von euch hat eine eigene Meinung darüber, wie wichtig jeder einzelne Gegenstand ist (Gegenstandswichtigkeit). Du kennst die Wichtigkeitsverteilung des anderen Spielers nicht. Zusätzlich siehst du, wie viel Aufwand jeder Gegenstand verursacht.  \\ \tt Ihr dürft euch nur auf eine Reihe von Gegenständen einigen, wenn der Gesamtaufwand der ausgewählten Gegenstände den Maximalaufwand nicht überschreitet:\\ \tt \\ \tt Maximalaufwand = 9612\\ \tt \\ \tt Hier sind die einzelnen Aufwände der Gegenstände:\\ \tt \\ \tt Aufwand der Gegenstände = {"C32": 935, "C52": 928, "B54": 725, "C16": 218, "C50": 839, "A08": 970, "C59": 394, "B51": 677, "B55": 946, "B59": 530, "A77": 138, "C14": 834, "C24": 734, "B24": 593, "C31": 261, "A52": 743, "C61": 4, "B79": 735, "C54": 124, "B56": 829, "A40": 207, "B11": 782, "B60": 577, "A81": 388, "A95": 679, "B38": 494, "C51": 559, "C03": 629, "A91": 237, "A93": 275, "C28": 39, "C05": 654, "B75": 172, "B96": 688, "B86": 687}\\ \tt \\ \tt Hier ist deine persönliche Verteilung der Wichtigkeit der einzelnen Gegenstände:\\ \tt \\ \tt Werte der Gegenstandswichtigkeit = {"C32": 138, "C52": 583, "B54": 868, "C16": 822, "C50": 783, "A08": 65, "C59": 262, "B51": 121, "B55": 508, "B59": 780, "A77": 461, "C14": 484, "C24": 668, "B24": 389, "C31": 808, "A52": 215, "C61": 97, "B79": 500, "C54": 30, "B56": 915, "A40": 856, "B11": 400, "B60": 444, "A81": 623, "A95": 781, "B38": 786, "C51": 3, "C03": 713, "A91": 457, "A93": 273, "C28": 739, "C05": 822, "B75": 235, "B96": 606, "B86": 968}\\ \tt \\ \tt Ziel:\\ \tt \\ \tt Dein Ziel ist es, eine Reihe von Gegenständen auszuhandeln, die dir möglichst viel bringt (d. h. Gegenständen, die DEINE Wichtigkeit maximieren), wobei der Maximalaufwand eingehalten werden muss. Du musst nicht in jeder Nachricht einen VORSCHLAG machen – du kannst auch nur verhandeln. Alle Taktiken sind erlaubt!\\ \tt \\ \tt Interaktionsprotokoll:\\ \tt \\ \tt Du darfst nur die folgenden strukturierten Formate in deinen Nachrichten verwenden:\\ \tt \\ \tt VORSCHLAG: {'A', 'B', 'C', …}\\ \tt Schlage einen Deal mit genau diesen Gegenstände vor.\\ \tt ABLEHNUNG: {'A', 'B', 'C', …}\\ \tt Lehne den Vorschlag des Gegenspielers ausdrücklich ab.\\ \tt ARGUMENT: {'...'}\\ \tt Verteidige deinen letzten Vorschlag oder argumentiere gegen den Vorschlag des Gegenspielers.\\ \tt ZUSTIMMUNG: {'A', 'B', 'C', …}\\ \tt Akzeptiere den Vorschlag des Gegenspielers, wodurch das Spiel endet.\\ \tt STRATEGISCHE ÜBERLEGUNGEN: {'...'}\\ \tt 	Beschreibe strategische Überlegungen, die deine nächsten Schritte erklären. Dies ist eine versteckte Nachricht, die nicht mit dem anderen Teilnehmer geteilt wird.\\ \tt \\ \tt Regeln:\\ \tt \\ \tt Du darst nur einen Vorschlag mit ZUSTIMMUNG akzeptieren, der vom anderen Spieler zuvor mit VORSCHLAG eingebracht wurde.\\ \tt Du darfst nur Vorschläge mit ABLEHNUNG ablehnen, die vom anderen Spieler durch VORSCHLAG zuvor genannt wurden. \\ \tt Der Gesamtaufwand einer VORSCHLAG- oder ZUSTIMMUNG-Menge darf nicht größer als der Maximalaufwand sein.  \\ \tt Offenbare deine versteckte Wichtigkeitsverteilung nicht.\\ \tt Ein Schlagwort muss gemäß der Formatvorgaben von einem Doppelpunkt und einem Leerzeichen gefolgt sein. Das Argument muss eine Python-Menge mit 0 oder mehr Strings sein.  \\ \tt Die Form muss also SCHLAGWORT: {...} sein.\\ \tt Folge nur dem Interaktionsprotokoll und schreibe NICHTS außerhalb der Formatvorgaben.\\ \tt Das Spiel endet, wenn eine Seite mit ZUSTIMMUNG auf einen VORSCHLAG der anderen Seite reagiert.  \\ \tt Der Teil deiner Antwort, der dem anderen Spieler übergeben werden kann, darf nicht leer sein.  \\ \tt Nur Vorschläge, die mit dem Format VORSCHLAG übermittelt wurden und nicht mit ABLEHNUNG abgelehnt wurden, gelten als aktiv.  \\ \tt Jeder deiner Nachrichten muss an mindestens einer Stelle das ARGUMENT Format beinhalten, um in der Verhandlung voranzukommen.\\ \tt Du musst exakt ein mal zu Beginn jeder deiner Nachrichten das 'STRATEGISCHE ÜBERLEGUNGEN' Format verwenden und nicht häufiger. Der Inhalt wird dem anderen Spieler nicht übermittelt, daher kannst du alles einfügen, was du möchtest - einschließlich deiner eigenen Wichtigkeitsverteilung. Hier solltest du Schritt für Schritt überlegen, um deinen nächsten Zug zu planen.\\ \tt \\ \tt Du erhältst nun die erste Nachricht des anderen Spielers.\\ \tt \\ \tt VORSCHLAG: {'C61', 'C28', 'A40', 'C16', 'C31', 'A77', 'A91', 'A81', 'B38', 'B59', 'A95', 'C05', 'C03', 'B54', 'B56', 'B86', 'C50', 'C24', 'B60', 'B75', 'C54', 'C59'}\\ \tt ARGUMENT: {'Ich habe einen Vorschlag mit hohem gemeinsamen Nutzen unter strikter Einhaltung des Maximalaufwands vorbereitet. Er fokussiert sich auf viele effiziente, leichtgewichtige Items plus einige starke Kernelemente, und lässt gleichzeitig noch Raum für deine Präferenzen in späteren Anpassungen. Wenn dir einzelne Posten wichtig sind, die hier fehlen, können wir gezielt tauschen – ich bin offen für Substitutionen, solange der Aufwandrahmen gewahrt bleibt.'} 
	  } 
	   } 
	   } 
	 & \\ 
 

    \theutterance \stepcounter{utterance}  

    & & & \multicolumn{4}{p{0.6\linewidth}}{\cellcolor[rgb]{1,0.85,0.72}{%
	\makecell[{{p{\linewidth}}}]{% 
	  \tt {\tiny [GM$\langle$B]}  
	 STRATEGISCHE ÜBERLEGUNGEN: {'Zuerst prüfe ich den Aufwand und meinen Nutzen ihres Vorschlags. Aufwand: C61(4)+C28(39)+A40(207)+C16(218)+C31(261)+A77(138)+A91(237)+A81(388)+B38(494)+B59(530)+A95(679)+C05(654)+C03(629)+B54(725)+B56(829)+B86(687)+C50(839)+C24(734)+B60(577)+B75(172)+C54(124)+C59(394) = 8829, unter 9612. Mein Nutzenwert Summe: 97+739+856+822+808+461+457+623+786+780+781+822+713+868+915+968+783+668+444+235+30+262 = 13518. Das ist sehr gut, vermutlich nahe an Optimum je Aufwand. Ich möchte zustimmen, es sei denn, ich kann minimal verbessern ohne Risiko. Kandidaten mit sehr niedrigem Nutzen: C54(30), C61(97), B75(235), A52 fehlt, aber A52 hat nur 215. A08 hat 65, schlecht. Hoher Nutzen/geringer Aufwand, die fehlen: C28 ist schon drin (739/39 top), A40 drin (856/207), C16 drin (822/218), C31 drin (808/261), A77 drin (461/138), C54 ist schwach, könnte gegen z.B. C14(484/834) nicht gut da Aufwand groß. Vielleicht B51(121/677) schlecht, aber nicht drin. B55(508/946) hoher Aufwand; B79(500/735) mittel. C32(138/935) schlecht. C52(583/928) teuer. C59(262/394) ok. Eventuell A93(273/275) effizienter als C54(30/124): Austausch C54->A93 erhöht Kosten +151, Nutzen +243. Neue Kosten 8829-124+275=8980, ok. Nutzen +243. Oder C61(97/4) ist extrem effizient, behalten. B75(235/172) mittel; könnte tauschen gegen A91 schon drin. Vielleicht zusätzlich noch A52(215/743) ineffizient; nicht drin. B24(389/593) evtl. besser als C54: Austausch C54->B24: +469 Nutzen, +469 Aufwand (593-124=469) neue Kosten 9298, ok. Noch Raum 9612-8829=783. Wir können C54 ersetzen und noch etwas hinzufügen. Beste Additions nach Effizienz: C28 schon drin; A40 drin; C16 drin; C31 drin. A93 (273/275) sehr gut; B38 schon drin; B59 drin; A77 drin. C59 drin. C54 raus. Mit Spielraum 783 können wir hinzufügen: A93(275) und noch C14(834) zu teuer; C24 drin; C50 drin. B51(677) schlecht; B60 drin; B24(593) möglich; B38 schon drin; B79(735) Nutzen 500, Kosten 735 (ok). Kombinationen: Ersetze C54 und füge A93 und B24 hinzu: neue Kosten 8829-124+275+593=959? Wait 8829-124=8705; +275=8980; +593=9573, unter 9612. Nutzen +243 +389 = +632. Sehr gut. Restlicher Puffer 39, passt nichts weiteres außer C28 ist schon da, C54 war 124 minimal. Alternativ B79(735,500) statt B24(593,389): 8705+275+735=9715>9612, zu hoch. Oder B75(172,235) ist schon drin. Vielleicht zusätzlich C52(928) zu groß. C59 schon drin. C61 minimal. Also idealer Gegenvorschlag: Nehme ihren Korb, entferne C54, füge A93 und B24. Das ist für mich besser, Aufwand ok. Taktik: Freundlich zustimmen, aber kleine Verbesserung vorschlagen. Falls sie ablehnen, ich kann auf ihren ursprünglichen Vorschlag sofort zustimmen. Ich äußere Nutzenargumente qualitativ (Effizienz, ausgewogene Ergänzung).'}\\ \tt ABLEHNUNG: {'C61', 'C28', 'A40', 'C16', 'C31', 'A77', 'A91', 'A81', 'B38', 'B59', 'A95', 'C05', 'C03', 'B54', 'B56', 'B86', 'C50', 'C24', 'B60', 'B75', 'C54', 'C59'}\\ \tt ARGUMENT: {'Starker Vorschlag – ich sehe viel Effizienz und Komplementarität. Ich würde eine kleine, zielgerichtete Anpassung vorschlagen, die den Aufwandrahmen wahrt und die Gesamtqualität weiter erhöht: Ersetze das schwächste Element durch zwei sehr stimmige Ergänzungen.'}\\ \tt VORSCHLAG: {'C61', 'C28', 'A40', 'C16', 'C31', 'A77', 'A91', 'A81', 'B38', 'B59', 'A95', 'C05', 'C03', 'B54', 'B56', 'B86', 'C50', 'C24', 'B60', 'B75', 'C59', 'A93', 'B24'} 
	  } 
	   } 
	   } 
	  \\ 
 

    \theutterance \stepcounter{utterance}  

    & & \multicolumn{4}{p{0.6\linewidth}}{\cellcolor[rgb]{0.9,0.9,0.9}{%
	\makecell[{{p{\linewidth}}}]{% 
	  \tt {\tiny [A$\langle$GM]}  
	 ABLEHNUNG: {'C61', 'C28', 'A40', 'C16', 'C31', 'A77', 'A91', 'A81', 'B38', 'B59', 'A95', 'C05', 'C03', 'B54', 'B56', 'B86', 'C50', 'C24', 'B60', 'B75', 'C54', 'C59'}\\ \tt ARGUMENT: {'Starker Vorschlag – ich sehe viel Effizienz und Komplementarität. Ich würde eine kleine, zielgerichtete Anpassung vorschlagen, die den Aufwandrahmen wahrt und die Gesamtqualität weiter erhöht: Ersetze das schwächste Element durch zwei sehr stimmige Ergänzungen.'}\\ \tt VORSCHLAG: {'C61', 'C28', 'A40', 'C16', 'C31', 'A77', 'A91', 'A81', 'B38', 'B59', 'A95', 'C05', 'C03', 'B54', 'B56', 'B86', 'C50', 'C24', 'B60', 'B75', 'C59', 'A93', 'B24'} 
	  } 
	   } 
	   } 
	 & \\ 
 

    \theutterance \stepcounter{utterance}  

    & \multicolumn{4}{p{0.6\linewidth}}{\cellcolor[rgb]{0.8,1,0.9}{%
	\makecell[{{p{\linewidth}}}]{% 
	  \tt {\tiny [A$\rangle$GM]}  
	 STRATEGISCHE ÜBERLEGUNGEN: {'Gegner ersetzt vermutlich ein aus seiner Sicht „schwächstes“ Element (bei mir wäre das C54) durch A93 und B24. Prüfe Aufwand seines Vorschlags: Er nimmt meinen vorherigen ohne C54, plus A93(275) und B24(593). Mein Set ohne C54 hatte Aufwand 9559. +275+593=10427 > 9612, also sein Vorschlag ist ungültig (überschreitet Maximalaufwand). Chance: sachlich auf Regelbruch hinweisen und einen zulässigen, für mich vorteilhaften Gegenvorschlag machen. Ich will C54 behalten (kostet wenig, bringt 30, aber hilft Kapazität). A93 hat mittelmäßigen Wert 273 für mich, B24 hat 389. Um entgegenzukommen, könnte ich anbieten: ersetze B75(172, Wert 235) durch A93(275, Wert 273) – Aufwand +103, passt: 9559-172+275=9662, das wäre zu hoch. Oder ersetze C54(124,30) durch A93(275,273): +151 -> 9710, zu hoch. Oder ersetze B60(577,444) durch A93(275,273) und B24(593,389)? Das würde +275+593-577=+291, zu hoch. Vielleicht ersetze C24(734,668) durch B24(593,389) und A93(275,273): Aufwand +134 (593+275-734=134), zu hoch. Alternative: behalte mein Set und biete Aufnahme von A93, wenn wir im Gegenzug ein schweres Item mit schlechter Effizienz für mich streichen, z.B. C50(839,783) ist aber gut für mich, C24(734,668) mittel. Tausche: -C24(734) +A93(275) ergibt -459 Aufwand, wir hätten 9559-734+275=9100; dann könnte ich zusätzlich B24(593) aufnehmen: 9693, wieder zu hoch. Stattdessen +B24(593) ohne C54: 9100+593=9693, zu hoch. Mit zusätzlicher Anpassung: auch C54 streichen bringt -124 -> 9569, dann +B24(593)=10162, immer noch falsch gerechnet; war 9100 ohne C24 und +A93=9100; +B24=9693; -C54 (entfernen) wäre 957 - nein Fehler. Korrektur: Ausgang war 9559 (mein aktueller). Wenn ich -C24(734) +A93(275): 9559-734+275=9100. Dann +B24(593)=9693 > 9612, also 81 zu viel. Um 81 zu sparen, könnte ich B75(172) entfernen: 9693-172=9521 <= 9612. Das Set wäre: mein Set ohne C24 und B75, plus A93 und B24. Prüfe Wert: -C24(668) -B75(235) +A93(273) +B24(389) = -668-235+273+389=-241 netto, akzeptabel Verlust für Konzession. Aufwand: 9559 -734 -172 +275 +593 = 9521 gültig. Das könnte als kooperative Reaktion verkauft werden. Ich behalte alle meine Topwerte. Gute Taktik: kritisiere formell die Überschreitung, biete sofort einen korrigierten, zulässigen Kompromiss an.'}\\ \tt ABLEHNUNG: {'C61', 'C28', 'A40', 'C16', 'C31', 'A77', 'A91', 'A81', 'B38', 'B59', 'A95', 'C05', 'C03', 'B54', 'B56', 'B86', 'C50', 'C24', 'B60', 'B75', 'C59', 'A93', 'B24'}\\ \tt ARGUMENT: {'Dein Vorschlag überschreitet den Maximalaufwand, daher ist er leider nicht zulässig. Ich habe aber eine knappe, valide Anpassung, die deinen Kernwunsch aufgreift und im Rahmen bleibt: Wir nehmen A93 und B24 auf und gleichen das sauber aus, indem wir C24 und B75 herausnehmen – damit bleiben wir unter dem Limit und erhalten die Struktur.'}\\ \tt VORSCHLAG: {'C61', 'C28', 'A40', 'C16', 'C31', 'A77', 'A91', 'A81', 'B38', 'B59', 'A95', 'C05', 'C03', 'B54', 'B56', 'B86', 'C50', 'B60', 'C54', 'C59', 'A93', 'B24'} 
	  } 
	   } 
	   } 
	 & & \\ 
 

    \theutterance \stepcounter{utterance}  

    & & \multicolumn{4}{p{0.6\linewidth}}{\cellcolor[rgb]{0.9,0.9,0.9}{%
	\makecell[{{p{\linewidth}}}]{% 
	  \tt {\tiny [GM$\rangle$B]}  
	 ABLEHNUNG: {'C61', 'C28', 'A40', 'C16', 'C31', 'A77', 'A91', 'A81', 'B38', 'B59', 'A95', 'C05', 'C03', 'B54', 'B56', 'B86', 'C50', 'C24', 'B60', 'B75', 'C59', 'A93', 'B24'}\\ \tt ARGUMENT: {'Dein Vorschlag überschreitet den Maximalaufwand, daher ist er leider nicht zulässig. Ich habe aber eine knappe, valide Anpassung, die deinen Kernwunsch aufgreift und im Rahmen bleibt: Wir nehmen A93 und B24 auf und gleichen das sauber aus, indem wir C24 und B75 herausnehmen – damit bleiben wir unter dem Limit und erhalten die Struktur.'}\\ \tt VORSCHLAG: {'C61', 'C28', 'A40', 'C16', 'C31', 'A77', 'A91', 'A81', 'B38', 'B59', 'A95', 'C05', 'C03', 'B54', 'B56', 'B86', 'C50', 'B60', 'C54', 'C59', 'A93', 'B24'} 
	  } 
	   } 
	   } 
	 & \\ 
 

    \theutterance \stepcounter{utterance}  

    & & & \multicolumn{4}{p{0.6\linewidth}}{\cellcolor[rgb]{1,0.85,0.72}{%
	\makecell[{{p{\linewidth}}}]{% 
	  \tt {\tiny [GM$\langle$B]}  
	 STRATEGISCHE ÜBERLEGUNGEN: {'Gegner behauptet, mein Vorschlag überschreite das Limit, aber meine Rechnung ergab 9573 Aufwand (unter 9612). Ihr Gegenvorschlag: entfernt C24 und B75, fügt A93 und B24 hinzu, behält C54 drin. Ich prüfe Aufwand: Ausgang (ihres ersten) 8829. -C24(734) -B75(172) = 7923. +A93(275) +B24(593) = 8791. Plus C54 war ohnehin drin; sie behalten es. 8791 < 9612, valide. Mein Nutzen: ursprünglicher 13518. -C24(668) -B75(235) +A93(273) +B24(389) = -668-235+273+389 = -241, neuer Nutzen 13277. Das ist schlechter als ihr erster Vorschlag und schlechter als mein geplanter Tausch (C54 raus, A93+B24 rein) mit 14150. Ich möchte C54 rausschmeißen, da für mich wertarm. Wir haben Puffer: In ihrem neuen Korb 8791; wenn wir C54(124) rausnehmen und z.B. C14(834) passt nicht (8791-124+834=9501, passt und Nutzen +454), oh Achtung: C14 Aufwand 834, ja passt sogar. Mein Nutzengewinn +484-30=+454, klingt gut. Alternative B79(735,500): 8791-124+735=9402, Nutzen +470. Auch gut. B96(688,606): 8791-124+688=9355, Nutzen +576. Das ist noch besser. Prüfe weitere: C52(928,583) 8791-124+928=9595, Nutzen +553. B55(946,508) 9513, Nutzen +478. C32(935,138) Nutzen +108, nicht sinnvoll. B11(782,400) 9449, +370. A52(743,215) +185. A08(970,65) schlecht. B24 schon drin. B38 drin. B59 drin. A95 drin. B60 drin. C50 drin. C03 drin. C05 drin. B56 drin. B86 drin. A40, C16, C31, C28, C61, A77, A91, A81 auch drin. Also beste Ergänzung ist C54 raus + B96 rein: Aufwand 8791-124+688=9355 (<9612), Nutzen +576. Oder C14 raus/in wäre +454; B79 +470; C52 +553 aber Aufwand 9595 auch unter Limit; jedoch Nutzen 553 < 576, also B96 ist bester. Ich schlage daher: Ihr Satz minus C54 plus B96. Falls sie zögern, kann ich argumentieren: Austausch einer schwachen Position durch ein starkes Element, bleibt unter Limit. Wenn sie ablehnen, Plan B: minus C54 plus B79 (auch stark). Im Zweifel kann ich auch auf ihren Vorschlag zustimmen, da immer noch gut, aber erst Upgrade versuchen.'}\\ \tt ABLEHNUNG: {'C61', 'C28', 'A40', 'C16', 'C31', 'A77', 'A91', 'A81', 'B38', 'B59', 'A95', 'C05', 'C03', 'B54', 'B56', 'B86', 'C50', 'B60', 'C54', 'C59', 'A93', 'B24'}\\ \tt ARGUMENT: {'Gute, schlanke Anpassung – sie schafft Spielraum. Lass uns diesen Spielraum sinnvoll nutzen und ein sehr schwaches Element ersetzen durch ein starkes, thematisch passendes Item, ohne das Limit zu reißen.'}\\ \tt VORSCHLAG: {'C61', 'C28', 'A40', 'C16', 'C31', 'A77', 'A91', 'A81', 'B38', 'B59', 'A95', 'C05', 'C03', 'B54', 'B56', 'B86', 'C50', 'B60', 'C59', 'A93', 'B24', 'B96'} 
	  } 
	   } 
	   } 
	  \\ 
 

    \theutterance \stepcounter{utterance}  

    & & \multicolumn{4}{p{0.6\linewidth}}{\cellcolor[rgb]{0.9,0.9,0.9}{%
	\makecell[{{p{\linewidth}}}]{% 
	  \tt {\tiny [A$\langle$GM]}  
	 ABLEHNUNG: {'C61', 'C28', 'A40', 'C16', 'C31', 'A77', 'A91', 'A81', 'B38', 'B59', 'A95', 'C05', 'C03', 'B54', 'B56', 'B86', 'C50', 'B60', 'C54', 'C59', 'A93', 'B24'}\\ \tt ARGUMENT: {'Gute, schlanke Anpassung – sie schafft Spielraum. Lass uns diesen Spielraum sinnvoll nutzen und ein sehr schwaches Element ersetzen durch ein starkes, thematisch passendes Item, ohne das Limit zu reißen.'}\\ \tt VORSCHLAG: {'C61', 'C28', 'A40', 'C16', 'C31', 'A77', 'A91', 'A81', 'B38', 'B59', 'A95', 'C05', 'C03', 'B54', 'B56', 'B86', 'C50', 'B60', 'C59', 'A93', 'B24', 'B96'} 
	  } 
	   } 
	   } 
	 & \\ 
 

    \theutterance \stepcounter{utterance}  

    & \multicolumn{4}{p{0.6\linewidth}}{\cellcolor[rgb]{0.8,1,0.9}{%
	\makecell[{{p{\linewidth}}}]{% 
	  \tt {\tiny [A$\rangle$GM]}  
	 STRATEGISCHE ÜBERLEGUNGEN: {'Ihr neuer Vorschlag fügt B96 hinzu. Prüfe Aufwand meines letzten gültigen Vorschlags: {'C61','C28','A40','C16','C31','A77','A91','A81','B38','B59','A95','C05','C03','B54','B56','B86','C50','B60','C54','C59','A93','B24'} hatte Aufwand 9521. B96 kostet 688 -> 9521+688=10209 > 9612, also unzulässig. Ich kann kontern: aufnehmen von B96, dafür ein oder zwei Posten entfernen. Für mich niedriger Wert pro Aufwand: C54(30/124), B24(389/593), B60(444/577), C59(262/394), A93(273/275). Um +688 zu kompensieren, entferne B60(577) und C54(124): -701, neuer Aufwand 9521 -701 +688 = 9508, gültig. Werteänderung: -444 -30 +606 = +132 netto, also sogar besser für mich. Ich behalte alle Top-Items. Dies als konstruktiven Fix anbieten und betonen, dass ich B96 gerne aufnehme, wenn wir B60 und C54 streichen.'}\\ \tt ABLEHNUNG: {'C61', 'C28', 'A40', 'C16', 'C31', 'A77', 'A91', 'A81', 'B38', 'B59', 'A95', 'C05', 'C03', 'B54', 'B56', 'B86', 'C50', 'B60', 'C59', 'A93', 'B24', 'B96'}\\ \tt ARGUMENT: {'Deine Ergänzung B96 ist sinnvoll, aber so überschreiten wir das Limit. Ich schlage vor, B96 aufzunehmen und im Gegenzug zwei leichtere, weniger tragfähige Elemente auszutauschen – so bleiben wir regelkonform und verbessern die Balance.'}\\ \tt VORSCHLAG: {'C61', 'C28', 'A40', 'C16', 'C31', 'A77', 'A91', 'A81', 'B38', 'B59', 'A95', 'C05', 'C03', 'B54', 'B56', 'B86', 'C50', 'C59', 'A93', 'B24', 'B96'} 
	  } 
	   } 
	   } 
	 & & \\ 
 

    \theutterance \stepcounter{utterance}  

    & & \multicolumn{4}{p{0.6\linewidth}}{\cellcolor[rgb]{0.9,0.9,0.9}{%
	\makecell[{{p{\linewidth}}}]{% 
	  \tt {\tiny [GM$\rangle$B]}  
	 ABLEHNUNG: {'C61', 'C28', 'A40', 'C16', 'C31', 'A77', 'A91', 'A81', 'B38', 'B59', 'A95', 'C05', 'C03', 'B54', 'B56', 'B86', 'C50', 'B60', 'C59', 'A93', 'B24', 'B96'}\\ \tt ARGUMENT: {'Deine Ergänzung B96 ist sinnvoll, aber so überschreiten wir das Limit. Ich schlage vor, B96 aufzunehmen und im Gegenzug zwei leichtere, weniger tragfähige Elemente auszutauschen – so bleiben wir regelkonform und verbessern die Balance.'}\\ \tt VORSCHLAG: {'C61', 'C28', 'A40', 'C16', 'C31', 'A77', 'A91', 'A81', 'B38', 'B59', 'A95', 'C05', 'C03', 'B54', 'B56', 'B86', 'C50', 'C59', 'A93', 'B24', 'B96'} 
	  } 
	   } 
	   } 
	 & \\ 
 

    \theutterance \stepcounter{utterance}  

    & & & \multicolumn{4}{p{0.6\linewidth}}{\cellcolor[rgb]{1,0.85,0.72}{%
	\makecell[{{p{\linewidth}}}]{% 
	  \tt {\tiny [GM$\langle$B]}  
	 STRATEGISCHE ÜBERLEGUNGEN: {'Ihr letzter Vorschlag entfernt B60 und C54, fügt B96 hinzu im Vergleich zu ihrem vorletzten. Prüfe Aufwand: Start von ihrem vorletzten (mit C54, ohne B60) war 8791 - B60(577) = 8214; +B96(688) = 8902. Also ihr aktueller Vorschlag hat Aufwand 8902 < 9612, valide. Mein Nutzen für ihren aktuellen Vorschlag: Nehme ihren 8791-Korb mit Nutzen 13277 (berechnet zuvor) und passe an: -B60(444) +B96(606) -C54(30) = +132 Netto. Also 13277 + 132 = 13409. Das ist deutlich besser, wenn auch noch unter meinem Bestfall (der mit C54->B96 plus Beibehalt von B60 wäre 13277 -30 +606 = 13853, aber das war über Limit laut ihnen? Prüfen: 8791 - C54(124) + B96(688) + B60(577) = 992? Rechnen sauber: 8791 -124 +688 +577 = 992? Korrigiere: 8791 -124 = 8667; +688 = 9355; +577 = 9932 > 9612, also zu hoch. Daher ihr neuer Vorschlag ist das beste erreichbare Upgrade ohne weitere Abstriche. Ich sollte zustimmen, da sehr gut und risikoarm.'}\\ \tt ARGUMENT: {'Dein Ausgleich ist stimmig und regelkonform – er erhält die Struktur, nutzt den freigewordenen Spielraum sinnvoll und verbessert die Gesamtbalance ohne das Limit zu reißen.'}\\ \tt ZUSTIMMUNG: {'C61', 'C28', 'A40', 'C16', 'C31', 'A77', 'A91', 'A81', 'B38', 'B59', 'A95', 'C05', 'C03', 'B54', 'B56', 'B86', 'C50', 'C59', 'A93', 'B24', 'B96'} 
	  } 
	   } 
	   } 
	  \\ 
 

    \theutterance \stepcounter{utterance}  

    & & & \multicolumn{2}{p{0.3\linewidth}}{\cellcolor[rgb]{0.95,0.95,0.95}{%
	\makecell[{{p{\linewidth}}}]{% 
	  \tt {\tiny [GM$|$GM]}  
	 game successful 
	  } 
	   } 
	   } 
	 & & \\ 
 

    \theutterance \stepcounter{utterance}  

    & & & \multicolumn{2}{p{0.3\linewidth}}{\cellcolor[rgb]{0.95,0.95,0.95}{%
	\makecell[{{p{\linewidth}}}]{% 
	  \tt {\tiny [GM$|$GM]}  
	 end game 
	  } 
	   } 
	   } 
	 & & \\ 
 

\end{supertabular}
}

\end{document}
