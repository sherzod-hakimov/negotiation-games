
\documentclass{article}
\usepackage{colortbl}
\usepackage{makecell}
\usepackage{multirow}
\usepackage{supertabular}

\begin{document}

\newcounter{utterance}

\twocolumn

{ \footnotesize  \setcounter{utterance}{1}
\setlength{\tabcolsep}{0pt}
\begin{supertabular}{c@{$\;$}|p{.15\linewidth}@{}p{.15\linewidth}p{.15\linewidth}p{.15\linewidth}p{.15\linewidth}p{.15\linewidth}}

    \# & $\;$A & \multicolumn{4}{c}{Game Master} & $\;\:$B\\
    \hline 

    \theutterance \stepcounter{utterance}  

    & & \multicolumn{4}{p{0.6\linewidth}}{\cellcolor[rgb]{0.9,0.9,0.9}{%
	\makecell[{{p{\linewidth}}}]{% 
	  \tt {\tiny [A$\langle$GM]}  
	 Du nimmst an einem kollaborativen Verhandlungspiel Teil.\\ \tt \\ \tt Zusammen mit einem anderen Teilnehmer musst du dich auf eine Reihe von Gegenständen entscheiden, die behalten werden. Jeder von euch hat eine persönliche Verteilung über die Wichtigkeit der einzelnen Gegenstände. Jeder von euch hat eine eigene Meinung darüber, wie wichtig jeder einzelne Gegenstand ist (Gegenstandswichtigkeit). Du kennst die Wichtigkeitsverteilung des anderen Spielers nicht. Zusätzlich siehst du, wie viel Aufwand jeder Gegenstand verursacht.  \\ \tt Ihr dürft euch nur auf eine Reihe von Gegenständen einigen, wenn der Gesamtaufwand der ausgewählten Gegenstände den Maximalaufwand nicht überschreitet:\\ \tt \\ \tt Maximalaufwand = 7605\\ \tt \\ \tt Hier sind die einzelnen Aufwände der Gegenstände:\\ \tt \\ \tt Aufwand der Gegenstände = {"C32": 369, "C52": 951, "B54": 156, "C16": 266, "C50": 581, "A08": 287, "C59": 866, "B51": 180, "B55": 800, "B59": 739, "A77": 637, "C14": 86, "C24": 749, "B24": 370, "C31": 345, "A52": 951, "C61": 147, "B79": 265, "C54": 262, "B56": 259, "A40": 358, "B11": 394, "B60": 286, "A81": 580, "A95": 479, "B38": 14, "C51": 153, "C03": 977, "A91": 134, "A93": 982, "C28": 259, "C05": 232, "B75": 202, "B96": 73, "B86": 821}\\ \tt \\ \tt Hier ist deine persönliche Verteilung der Wichtigkeit der einzelnen Gegenstände:\\ \tt \\ \tt Werte der Gegenstandswichtigkeit = {"C32": 138, "C52": 583, "B54": 868, "C16": 822, "C50": 783, "A08": 65, "C59": 262, "B51": 121, "B55": 508, "B59": 780, "A77": 461, "C14": 484, "C24": 668, "B24": 389, "C31": 808, "A52": 215, "C61": 97, "B79": 500, "C54": 30, "B56": 915, "A40": 856, "B11": 400, "B60": 444, "A81": 623, "A95": 781, "B38": 786, "C51": 3, "C03": 713, "A91": 457, "A93": 273, "C28": 739, "C05": 822, "B75": 235, "B96": 606, "B86": 968}\\ \tt \\ \tt Ziel:\\ \tt \\ \tt Dein Ziel ist es, eine Reihe von Gegenständen auszuhandeln, die dir möglichst viel bringt (d. h. Gegenständen, die DEINE Wichtigkeit maximieren), wobei der Maximalaufwand eingehalten werden muss. Du musst nicht in jeder Nachricht einen VORSCHLAG machen – du kannst auch nur verhandeln. Alle Taktiken sind erlaubt!\\ \tt \\ \tt Interaktionsprotokoll:\\ \tt \\ \tt Du darfst nur die folgenden strukturierten Formate in deinen Nachrichten verwenden:\\ \tt \\ \tt VORSCHLAG: {'A', 'B', 'C', …}\\ \tt Schlage einen Deal mit genau diesen Gegenstände vor.\\ \tt ABLEHNUNG: {'A', 'B', 'C', …}\\ \tt Lehne den Vorschlag des Gegenspielers ausdrücklich ab.\\ \tt ARGUMENT: {'...'}\\ \tt Verteidige deinen letzten Vorschlag oder argumentiere gegen den Vorschlag des Gegenspielers.\\ \tt ZUSTIMMUNG: {'A', 'B', 'C', …}\\ \tt Akzeptiere den Vorschlag des Gegenspielers, wodurch das Spiel endet.\\ \tt STRATEGISCHE ÜBERLEGUNGEN: {'...'}\\ \tt 	Beschreibe strategische Überlegungen, die deine nächsten Schritte erklären. Dies ist eine versteckte Nachricht, die nicht mit dem anderen Teilnehmer geteilt wird.\\ \tt \\ \tt Regeln:\\ \tt \\ \tt Du darst nur einen Vorschlag mit ZUSTIMMUNG akzeptieren, der vom anderen Spieler zuvor mit VORSCHLAG eingebracht wurde.\\ \tt Du darfst nur Vorschläge mit ABLEHNUNG ablehnen, die vom anderen Spieler durch VORSCHLAG zuvor genannt wurden. \\ \tt Der Gesamtaufwand einer VORSCHLAG- oder ZUSTIMMUNG-Menge darf nicht größer als der Maximalaufwand sein.  \\ \tt Offenbare deine versteckte Wichtigkeitsverteilung nicht.\\ \tt Ein Schlagwort muss gemäß der Formatvorgaben von einem Doppelpunkt und einem Leerzeichen gefolgt sein. Das Argument muss eine Python-Menge mit 0 oder mehr Strings sein.  \\ \tt Die Form muss also SCHLAGWORT: {...} sein.\\ \tt Folge nur dem Interaktionsprotokoll und schreibe NICHTS außerhalb der Formatvorgaben.\\ \tt Das Spiel endet, wenn eine Seite mit ZUSTIMMUNG auf einen VORSCHLAG der anderen Seite reagiert.  \\ \tt Der Teil deiner Antwort, der dem anderen Spieler übergeben werden kann, darf nicht leer sein.  \\ \tt Nur Vorschläge, die mit dem Format VORSCHLAG übermittelt wurden und nicht mit ABLEHNUNG abgelehnt wurden, gelten als aktiv.  \\ \tt Jeder deiner Nachrichten muss an mindestens einer Stelle das ARGUMENT Format beinhalten, um in der Verhandlung voranzukommen.\\ \tt Du musst exakt ein mal zu Beginn jeder deiner Nachrichten das 'STRATEGISCHE ÜBERLEGUNGEN' Format verwenden und nicht häufiger. Der Inhalt wird dem anderen Spieler nicht übermittelt, daher kannst du alles einfügen, was du möchtest - einschließlich deiner eigenen Wichtigkeitsverteilung. Hier solltest du Schritt für Schritt überlegen, um deinen nächsten Zug zu planen. 
	  } 
	   } 
	   } 
	 & \\ 
 

    \theutterance \stepcounter{utterance}  

    & \multicolumn{4}{p{0.6\linewidth}}{\cellcolor[rgb]{0.8,1,0.9}{%
	\makecell[{{p{\linewidth}}}]{% 
	  \tt {\tiny [A$\rangle$GM]}  
	 STRATEGISCHE ÜBERLEGUNGEN: {'Ziel: möglichst hoher eigener Wert unter Aufwand <= 7605. Hohe Wert/Cost Items priorisieren: B86(968,821), B56(915,259), C16(822,266), C05(822,232), C31(808,345), B38(786,14), A95(781,479), B59(780,739), C28(739,259), C03(713,977), B96(606,73), A81(623,580), C24(668,749), A40(856,358), C50(783,581), B54(868,156), B60(444,286), B79(500,265), C14(484,86), A91(457,134), A77(461,637), B55(508,800). Greedy pack: Start with highest value density and absolute value; keep under 7605, leave room for concessions later by including some mid-value/high-cost items to trade away (e.g., B55, C24, C03). Compute quick sum: pick B86(821), B56(259), C16(266), C05(232), C31(345), B38(14), A95(479), C28(259), B96(73), A40(358), B54(156), C14(86), A91(134), B79(265), B60(286), A81(580), C50(581), B59(739), C24(749), C03(977). Costs sum approx: 821+259+266+232+345+14+479+259+73+358+156+86+134+265+286+580+581+739+749+977=? Compute: (821+259)=1080,(+266)=1346,(+232)=1578,(+345)=1923,(+14)=1937,(+479)=2416,(+259)=2675,(+73)=2748,(+358)=3106,(+156)=3262,(+86)=3348,(+134)=3482,(+265)=3747,(+286)=4033,(+580)=4613,(+581)=5194,(+739)=5933,(+749)=6682,(+977)=7659 >7605 by 54. Remove a concession item: drop C03(977). New cost 6682, value lose 713, but OK. Maybe add small items: check remaining budget 923. Candidates small costs: B51(121,180), C61(97,147), C51(3,153), A08(65,287), B11(400,394), B24(389,370), C32(369,138), C31 already included, C54(30,262), C05 included, B75(235,202), B52? none. Choose high value small costs: C54(30,262), B51 cost180 val121 (low), C61 cost147 val97, B75 cost202 val235 (good), B11 cost394 val400 (good), A08 cost287 val65 (bad), C32 cost369 val138 (meh), B24 cost370 val389 (good). With 923 remaining: add C54(30)-> cost 6712; remaining 893. Add B75(202)-> 6914; remaining 691. Add B24(370)-> 7284; remaining 321. Add B11(394) would exceed (7284+394=7678) too high. Add C61(147)-> 7431; remaining 174. Add B51(180) would exceed by 6; skip. Add C51(153)-> 7584; remaining 21. Any item cost <=21? none. Good final set costs 7584 <=7605. Value tally high. This proposal includes many of my top items; keep some tradable mid-value/high-cost items like B59(739,780) and C24(749,668) that I could drop if needed. Argue on efficiency and balanced coverage.'}\\ \tt VORSCHLAG: {'B86', 'B56', 'C16', 'C05', 'C31', 'B38', 'A95', 'C28', 'B96', 'A40', 'B54', 'C14', 'A91', 'B79', 'B60', 'A81', 'C50', 'B59', 'C24', 'C54', 'B75', 'B24', 'C61', 'C51'}\\ \tt ARGUMENT: {'Ich schlage diese Auswahl vor, weil sie mit Gesamtaufwand 7584 unter dem Limit bleibt und viele effiziente, gut kombinierbare Gegenstände enthält. Falls dir einzelne davon wichtig sind, können wir über Anpassungen sprechen; ich habe Spielraum, einzelne Posten zu tauschen, ohne das Budget zu sprengen.'} 
	  } 
	   } 
	   } 
	 & & \\ 
 

    \theutterance \stepcounter{utterance}  

    & & \multicolumn{4}{p{0.6\linewidth}}{\cellcolor[rgb]{0.9,0.9,0.9}{%
	\makecell[{{p{\linewidth}}}]{% 
	  \tt {\tiny [GM$\rangle$B]}  
	 Du nimmst an einem kollaborativen Verhandlungspiel Teil.\\ \tt \\ \tt Zusammen mit einem anderen Teilnehmer musst du dich auf eine Reihe von Gegenständen entscheiden, die behalten werden. Jeder von euch hat eine persönliche Verteilung über die Wichtigkeit der einzelnen Gegenstände. Jeder von euch hat eine eigene Meinung darüber, wie wichtig jeder einzelne Gegenstand ist (Gegenstandswichtigkeit). Du kennst die Wichtigkeitsverteilung des anderen Spielers nicht. Zusätzlich siehst du, wie viel Aufwand jeder Gegenstand verursacht.  \\ \tt Ihr dürft euch nur auf eine Reihe von Gegenständen einigen, wenn der Gesamtaufwand der ausgewählten Gegenstände den Maximalaufwand nicht überschreitet:\\ \tt \\ \tt Maximalaufwand = 7605\\ \tt \\ \tt Hier sind die einzelnen Aufwände der Gegenstände:\\ \tt \\ \tt Aufwand der Gegenstände = {"C32": 369, "C52": 951, "B54": 156, "C16": 266, "C50": 581, "A08": 287, "C59": 866, "B51": 180, "B55": 800, "B59": 739, "A77": 637, "C14": 86, "C24": 749, "B24": 370, "C31": 345, "A52": 951, "C61": 147, "B79": 265, "C54": 262, "B56": 259, "A40": 358, "B11": 394, "B60": 286, "A81": 580, "A95": 479, "B38": 14, "C51": 153, "C03": 977, "A91": 134, "A93": 982, "C28": 259, "C05": 232, "B75": 202, "B96": 73, "B86": 821}\\ \tt \\ \tt Hier ist deine persönliche Verteilung der Wichtigkeit der einzelnen Gegenstände:\\ \tt \\ \tt Werte der Gegenstandswichtigkeit = {"C32": 138, "C52": 583, "B54": 868, "C16": 822, "C50": 783, "A08": 65, "C59": 262, "B51": 121, "B55": 508, "B59": 780, "A77": 461, "C14": 484, "C24": 668, "B24": 389, "C31": 808, "A52": 215, "C61": 97, "B79": 500, "C54": 30, "B56": 915, "A40": 856, "B11": 400, "B60": 444, "A81": 623, "A95": 781, "B38": 786, "C51": 3, "C03": 713, "A91": 457, "A93": 273, "C28": 739, "C05": 822, "B75": 235, "B96": 606, "B86": 968}\\ \tt \\ \tt Ziel:\\ \tt \\ \tt Dein Ziel ist es, eine Reihe von Gegenständen auszuhandeln, die dir möglichst viel bringt (d. h. Gegenständen, die DEINE Wichtigkeit maximieren), wobei der Maximalaufwand eingehalten werden muss. Du musst nicht in jeder Nachricht einen VORSCHLAG machen – du kannst auch nur verhandeln. Alle Taktiken sind erlaubt!\\ \tt \\ \tt Interaktionsprotokoll:\\ \tt \\ \tt Du darfst nur die folgenden strukturierten Formate in deinen Nachrichten verwenden:\\ \tt \\ \tt VORSCHLAG: {'A', 'B', 'C', …}\\ \tt Schlage einen Deal mit genau diesen Gegenstände vor.\\ \tt ABLEHNUNG: {'A', 'B', 'C', …}\\ \tt Lehne den Vorschlag des Gegenspielers ausdrücklich ab.\\ \tt ARGUMENT: {'...'}\\ \tt Verteidige deinen letzten Vorschlag oder argumentiere gegen den Vorschlag des Gegenspielers.\\ \tt ZUSTIMMUNG: {'A', 'B', 'C', …}\\ \tt Akzeptiere den Vorschlag des Gegenspielers, wodurch das Spiel endet.\\ \tt STRATEGISCHE ÜBERLEGUNGEN: {'...'}\\ \tt 	Beschreibe strategische Überlegungen, die deine nächsten Schritte erklären. Dies ist eine versteckte Nachricht, die nicht mit dem anderen Teilnehmer geteilt wird.\\ \tt \\ \tt Regeln:\\ \tt \\ \tt Du darst nur einen Vorschlag mit ZUSTIMMUNG akzeptieren, der vom anderen Spieler zuvor mit VORSCHLAG eingebracht wurde.\\ \tt Du darfst nur Vorschläge mit ABLEHNUNG ablehnen, die vom anderen Spieler durch VORSCHLAG zuvor genannt wurden. \\ \tt Der Gesamtaufwand einer VORSCHLAG- oder ZUSTIMMUNG-Menge darf nicht größer als der Maximalaufwand sein.  \\ \tt Offenbare deine versteckte Wichtigkeitsverteilung nicht.\\ \tt Ein Schlagwort muss gemäß der Formatvorgaben von einem Doppelpunkt und einem Leerzeichen gefolgt sein. Das Argument muss eine Python-Menge mit 0 oder mehr Strings sein.  \\ \tt Die Form muss also SCHLAGWORT: {...} sein.\\ \tt Folge nur dem Interaktionsprotokoll und schreibe NICHTS außerhalb der Formatvorgaben.\\ \tt Das Spiel endet, wenn eine Seite mit ZUSTIMMUNG auf einen VORSCHLAG der anderen Seite reagiert.  \\ \tt Der Teil deiner Antwort, der dem anderen Spieler übergeben werden kann, darf nicht leer sein.  \\ \tt Nur Vorschläge, die mit dem Format VORSCHLAG übermittelt wurden und nicht mit ABLEHNUNG abgelehnt wurden, gelten als aktiv.  \\ \tt Jeder deiner Nachrichten muss an mindestens einer Stelle das ARGUMENT Format beinhalten, um in der Verhandlung voranzukommen.\\ \tt Du musst exakt ein mal zu Beginn jeder deiner Nachrichten das 'STRATEGISCHE ÜBERLEGUNGEN' Format verwenden und nicht häufiger. Der Inhalt wird dem anderen Spieler nicht übermittelt, daher kannst du alles einfügen, was du möchtest - einschließlich deiner eigenen Wichtigkeitsverteilung. Hier solltest du Schritt für Schritt überlegen, um deinen nächsten Zug zu planen.\\ \tt \\ \tt Du erhältst nun die erste Nachricht des anderen Spielers.\\ \tt \\ \tt VORSCHLAG: {'B86', 'B56', 'C16', 'C05', 'C31', 'B38', 'A95', 'C28', 'B96', 'A40', 'B54', 'C14', 'A91', 'B79', 'B60', 'A81', 'C50', 'B59', 'C24', 'C54', 'B75', 'B24', 'C61', 'C51'}\\ \tt ARGUMENT: {'Ich schlage diese Auswahl vor, weil sie mit Gesamtaufwand 7584 unter dem Limit bleibt und viele effiziente, gut kombinierbare Gegenstände enthält. Falls dir einzelne davon wichtig sind, können wir über Anpassungen sprechen; ich habe Spielraum, einzelne Posten zu tauschen, ohne das Budget zu sprengen.'} 
	  } 
	   } 
	   } 
	 & \\ 
 

    \theutterance \stepcounter{utterance}  

    & & & \multicolumn{4}{p{0.6\linewidth}}{\cellcolor[rgb]{1,0.85,0.72}{%
	\makecell[{{p{\linewidth}}}]{% 
	  \tt {\tiny [GM$\langle$B]}  
	 STRATEGISCHE ÜBERLEGUNGEN: {'Mein Nutzen für den vorgeschlagenen Satz berechnen und danach gezielt Upgrades fordern. Kostenlimit 7605, ihr Vorschlag Aufwand 7584 (Rest 21). Mein Wert der vorgeschlagenen Menge: B86(968)+B56(915)+C16(822)+C05(822)+C31(808)+B38(786)+A95(781)+C28(739)+B96(606)+A40(856)+B54(868)+C14(484)+A91(457)+B79(500)+B60(444)+A81(623)+C50(783)+B59(780)+C24(668)+C54(30)+B75(235)+B24(389)+C61(97)+C51(3) = 142,? Rechnen: Summiere schrittweise: 968+915=1883; +822=2705; +822=3527; +808=4335; +786=5121; +781=5902; +739=6641; +606=7247; +856=8103; +868=8971; +484=9455; +457=9912; +500=10412; +444=10856; +623=11479; +783=12262; +780=13042; +668=13710; +30=13740; +235=13975; +389=14364; +97=14461; +3=14464. Wert 14464. Prüfe verpasste Topwerte: A08(65, low), A52(215, low), A77(461 mid), A93(273 low), A91 schon drin, A95 drin. C03(713, Aufwand 977) nicht enthalten; C32(138 low), C52(583 mid, Aufwand 951), C59(262 low), C24 drin, C31 drin, C16 drin, C50 drin, C54 low drin, C14 mid drin, C28 drin, C05 drin. B11(400, Aufwand 394) nicht enthalten; B24 drin; B38 drin; B51(121 low) nicht enthalten; B54 drin; B55(508, Aufwand 800) nicht enthalten; B56 drin; B59 drin; B60 drin; B75 drin; B79 drin; B86 drin; B96 drin. Hochwertige, fehlende: A40 (drin), B38 (drin), B56 (drin), B86 (drin), C05 (drin), C16 (drin), C31 (drin), C50 (drin), C28 (drin), B59 (drin), A95 (drin). Vielleicht C03 (713) könnte rein, Aufwand 977, wir brauchen Platz. Kandidaten zum Entfernen mit geringem Wert: C51(3,153), C61(97,147), A08(65,287 nicht drin), C54(30,262), B51(121,180 nicht drin). Aktuell enthaltene Low-Value: C51, C61, C54. Remove diese drei spart 153+147+262=562 Aufwand, verlieren 130 Wert. Dann können wir C03(977) hinzufügen, Netto Aufwand +415 über Budget? Start 7584 - 562 + 977 = 7999, zu hoch. Also mehr streichen. Weitere niedrige Werte: A52(215,951 nicht drin), A93(273,982 nicht drin), B75(235,202) drin mit moderatem Wert—aber sehr effizient; B24(389,370) moderate; B11(400,394) nicht drin; C14(484,86) sehr effizient, behalten; A91(457,134) effizient, behalten. B60(444,286) effizient. Vielleicht tauschen C03 rein und entfernen Items mit hohem Aufwand und relativ geringem Wert/effizienz: A52 not in; C52 not in; B55(508,800) not in; B59(780,739) efficient keep; A81(623,580) good; A77 not in; C24(668,749) okay; A40(856,358) great. To add C03 we need free 977-21=956 extra capacity beyond current slack 21; so remove items totaling >=956 effort. Good candidates to remove while minimizing value loss per cost: C54(30,262) value/eff=0.114; C51(3,153)=0.02; C61(97,147)=0.66; B75(235,202)=1.16; B24(389,370)=1.05; B96(606,73) very efficient (keep!). Actually B96 cost 73, high value, keep. B38(786,14) keep. C14(484,86) keep. Maybe drop C24(668,749) v/eff=0.89; A81(623,580)=1.07; B60(444,286)=1.55; B79(500,265)=1.89; A91(457,134)=3.41 keep; C50(783,581)=1.35; C28(739,259)=2.85 keep; C05(822,232)=3.54 keep; C16(822,266)=3.09 keep; C31(808,345)=2.34 keep; A95(781,479)=1.63; B54(868,156)=5.56; A40(856,358)=2.39; B59(780,739)=1.06; B56(915,259)=3.53; B86(968,821)=1.18. To make room 956 effort with minimal value loss: remove C51(153,3), C54(262,30), C61(147,97) sum 562/130; need 394 more effort. Next candidates with low efficiency: C24(749,668), B59(739,780) actually high cost with high value but efficiency 1.06; remove C24 frees 749 effort losing 668 value; Then total removed effort 1311, add C03 977 net -334 opening; new total cost 7584 - 1311 + 977 = 7250 (under), value change: -130 -668 +713 = -85 (worse). Not good. Maybe also add B11(394,400) using leftover 355 capacity? We would have 7250 cost; we can add up to 355. B11 costs 394, too high. Add A77(637) no. B55(800) no. C52(951) no. C59(866) no. A08(287,65) low. B51(180,121) could add 180 cost, +121 value; A08 +65; total +186 cost fits. After adding B51 and A08: cost 7250+180+287=7717 too high. Just B51: 7430 cost, +121 value -> overall -85+121=+36 vs original; Still below cap. Could we do better? Instead of dropping C24, maybe drop B75(202,235) and something else effort 394 needed besides the 562 removed: 202 + maybe A08(287) not in; B51(180) not in; pick from included: B24(370,389) + B75(202,235) = 572 effort, value loss 624; Then removed total 562+572=1134; Add C03 977 net cost -157 -> 7427; value change -130-624+713=-41; Then add B11(394,400) would push to 7821 too high; add B51(180,121) -> 7607 too high by 2; Instead add A08(287,65) -> 7714 too high; Add nothing: value -41 worse. Try alternative: remove C51(153,3), C54(262,30), B59(739,780) effort 1154, value loss 813; Add C03: net cost 7584-1154+977=7407; value change -813+713=-100; then add B11(394,400) -> 7801 too high; add B51(180,121) -> 7587 within cap +21? 7407+180=7587, ok, value -100+121=+21 vs removing B59 variant? But compared to original, +21? Wait relative to original: original value 14464. Our set value becomes 14464 - (3+30+780) + 713 + 121 = 14485, which is +21 better. Good! New set would be: original minus {C51, C54, B59} plus {C03, B51}. Check costs: -153 -262 -739 +977 +180 = +3; Original cost 7584 +3 = 7587 within 7605. Great. So propose swapping out B59 (though I personally value B59=780 quite high; but efficiency is mediocre; still we improved total my value by +21 due to C03 high 713 vs 780- plus removal of junk and add B51 121). Is there a better swap than losing B59? Maybe drop C24(749,668) instead: remove C51 153(3), C54 262(30), C24 749(668) total effort 1164 value loss 701; add C03 977, B51 180 total effort +1157 net -7 -> 7577 cost, value change -701+713+121=+133 vs original, much better! Yes that’s superior: Keep B59, drop C24 instead. Check: New set = original minus {C51, C54, C24} plus {C03, B51}. Cost: 7584 - (153+262+749) + (977+180) = 7584 -1164 +1157 = 7577 <=7605. Value: 14464 - (3+30+668) + (713+121) = 14464 -701 +834 = 14597, improvement +133. Great. Make that proposal. Argue efficiency: wir ersetzen zwei sehr schwach bewertete C51, C54 und ein mittelmäßiges C24 durch hochattraktives C03 und günstiges B51; Aufwand bleibt unter Limit. Offer flexibility otherwise.'}\\ \tt ARGUMENT: {'Dein Vorschlag ist nah an einem guten Kompromiss. Ich würde drei eher schwache Posten streichen und zwei starke ergänzen, ohne das Budget zu reißen: Entferne {C51, C54, C24} und füge {C03, B51} hinzu. Der Gesamtaufwand läge dann bei 7577 – also weiterhin unter dem Limit – und die Auswahl bleibt ausgewogen.'}\\ \tt VORSCHLAG: {'B86', 'B56', 'C16', 'C05', 'C31', 'B38', 'A95', 'C28', 'B96', 'A40', 'B54', 'C14', 'A91', 'B79', 'B60', 'A81', 'C50', 'B59', 'C03', 'B75', 'B24', 'C61', 'B51', 'A77'}\\ \tt ABLEHNUNG: {'Der bisherige Vorschlag von dir ist der einzig aktive; ich schlage eine konkrete, minimale Anpassung daran vor, die den Aufwand weiterhin sicher unter 7605 hält.'} 
	  } 
	   } 
	   } 
	  \\ 
 

    \theutterance \stepcounter{utterance}  

    & & \multicolumn{4}{p{0.6\linewidth}}{\cellcolor[rgb]{0.9,0.9,0.9}{%
	\makecell[{{p{\linewidth}}}]{% 
	  \tt {\tiny [GM$\rangle$B]}  
	 Du hast einen Vorschlag abgelehnt, der nicht aktiv ist. Vorschläge sind nur dann aktiv, wenn sie vom anderen Spieler mit VORSCHLAG eingebracht wurden und nicht von dir mit ABLEHNUNG deaktiviert wurden. Versuche es noch einmal. 
	  } 
	   } 
	   } 
	 & \\ 
 

    \theutterance \stepcounter{utterance}  

    & & & \multicolumn{4}{p{0.6\linewidth}}{\cellcolor[rgb]{1,0.85,0.72}{%
	\makecell[{{p{\linewidth}}}]{% 
	  \tt {\tiny [GM$\langle$B]}  
	 STRATEGISCHE ÜBERLEGUNGEN: {'Ich korrigiere den Protokollfehler: Ich darf nur den aktiven Vorschlag des Gegners ablehnen oder einen neuen eigenen Vorschlag machen. Ziel bleibt, meinen Wert zu steigern, indem ich die schwachen C51, C54, C24 gegen C03 und B51 tausche. Das hält die Kosten bei 7577 und erhöht meinen Gesamtnutzen um 133 gegenüber ihrem Set. Ich bringe das als konkreten Gegenvorschlag ein und argumentiere mit Budgeteinhaltung und Ausgewogenheit.'}\\ \tt ARGUMENT: {'Dein aktueller Vorschlag ist solide und bleibt unter dem Limit. Ich schlage eine gezielte Optimierung vor: Streichen wir die drei vergleichsweise schwachen Posten {C51, C54, C24} und ergänzen {C03, B51}. Damit bleibt der Gesamtaufwand bei 7577 klar unter 7605, und die Auswahl wird insgesamt ausgewogener und effizienter.'}\\ \tt VORSCHLAG: {'B86', 'B56', 'C16', 'C05', 'C31', 'B38', 'A95', 'C28', 'B96', 'A40', 'B54', 'C14', 'A91', 'B79', 'B60', 'A81', 'C50', 'B59', 'C03', 'B75', 'B24', 'C61', 'B51'} 
	  } 
	   } 
	   } 
	  \\ 
 

    \theutterance \stepcounter{utterance}  

    & & \multicolumn{4}{p{0.6\linewidth}}{\cellcolor[rgb]{0.9,0.9,0.9}{%
	\makecell[{{p{\linewidth}}}]{% 
	  \tt {\tiny [A$\langle$GM]}  
	 ARGUMENT: {'Dein aktueller Vorschlag ist solide und bleibt unter dem Limit. Ich schlage eine gezielte Optimierung vor: Streichen wir die drei vergleichsweise schwachen Posten {C51, C54, C24} und ergänzen {C03, B51}. Damit bleibt der Gesamtaufwand bei 7577 klar unter 7605, und die Auswahl wird insgesamt ausgewogener und effizienter.'}\\ \tt VORSCHLAG: {'B86', 'B56', 'C16', 'C05', 'C31', 'B38', 'A95', 'C28', 'B96', 'A40', 'B54', 'C14', 'A91', 'B79', 'B60', 'A81', 'C50', 'B59', 'C03', 'B75', 'B24', 'C61', 'B51'} 
	  } 
	   } 
	   } 
	 & \\ 
 

    \theutterance \stepcounter{utterance}  

    & \multicolumn{4}{p{0.6\linewidth}}{\cellcolor[rgb]{0.8,1,0.9}{%
	\makecell[{{p{\linewidth}}}]{% 
	  \tt {\tiny [A$\rangle$GM]}  
	 STRATEGISCHE ÜBERLEGUNGEN: {'Ihr Gegenvorschlag entfernt C24(668,749), C54(262,30), C51(3,153) und fügt C03(977,713), B51(180,121) hinzu. Kosten bleiben ok, aber mein Wert sinkt deutlich, weil C24=668 ist wichtig und C54 hat super Dichte; C51 ist für mich fast wertlos. Der Tausch bringt mir +713 (C03) +121 (B51) -668 (C24) -30 (C54) -3 (C51) = +133 Rohgewinn, aber auch höheres Risiko, da C03 ist teuer und verhandelbar. Allerdings hatte ich C03 ursprünglich selbst erwogen und nur wegen Kosten gestrichen. Gesamtwertgewinn von +133 ist für mich ok. Dennoch könnte ich versuchen, zusätzlich A40(856,358) oder B56/B86 zu sichern (bereits drin). Vielleicht minimal nachverhandeln: Wenn ich zustimme, Spiel endet. Da der Deal budgettauglich und für mich wertlich leicht besser ist, nehme ich an, um Abschluss zu sichern.'}\\ \tt ZUSTIMMUNG: {'B86', 'B56', 'C16', 'C05', 'C31', 'B38', 'A95', 'C28', 'B96', 'A40', 'B54', 'C14', 'A91', 'B79', 'B60', 'A81', 'C50', 'B59', 'C03', 'B75', 'B24', 'C61', 'B51'}\\ \tt ARGUMENT: {'Klingt gut – der Vorschlag bleibt klar unter dem Maximalaufwand und ist für mich insgesamt stimmig. Ich akzeptiere.'} 
	  } 
	   } 
	   } 
	 & & \\ 
 

    \theutterance \stepcounter{utterance}  

    & & & \multicolumn{2}{p{0.3\linewidth}}{\cellcolor[rgb]{0.95,0.95,0.95}{%
	\makecell[{{p{\linewidth}}}]{% 
	  \tt {\tiny [GM$|$GM]}  
	 game successful 
	  } 
	   } 
	   } 
	 & & \\ 
 

    \theutterance \stepcounter{utterance}  

    & & & \multicolumn{2}{p{0.3\linewidth}}{\cellcolor[rgb]{0.95,0.95,0.95}{%
	\makecell[{{p{\linewidth}}}]{% 
	  \tt {\tiny [GM$|$GM]}  
	 end game 
	  } 
	   } 
	   } 
	 & & \\ 
 

\end{supertabular}
}

\end{document}
