
\documentclass{article}
\usepackage{colortbl}
\usepackage{makecell}
\usepackage{multirow}
\usepackage{supertabular}

\begin{document}

\newcounter{utterance}

\twocolumn

{ \footnotesize  \setcounter{utterance}{1}
\setlength{\tabcolsep}{0pt}
\begin{supertabular}{c@{$\;$}|p{.15\linewidth}@{}p{.15\linewidth}p{.15\linewidth}p{.15\linewidth}p{.15\linewidth}p{.15\linewidth}}

    \# & $\;$A & \multicolumn{4}{c}{Game Master} & $\;\:$B\\
    \hline 

    \theutterance \stepcounter{utterance}  

    & & \multicolumn{4}{p{0.6\linewidth}}{\cellcolor[rgb]{0.9,0.9,0.9}{%
	\makecell[{{p{\linewidth}}}]{% 
	  \tt {\tiny [A$\langle$GM]}  
	 Du nimmst an einem kollaborativen Verhandlungspiel Teil.\\ \tt \\ \tt Zusammen mit einem anderen Teilnehmer musst du dich auf eine Reihe von Gegenständen entscheiden, die behalten werden. Jeder von euch hat eine persönliche Verteilung über die Wichtigkeit der einzelnen Gegenstände. Jeder von euch hat eine eigene Meinung darüber, wie wichtig jeder einzelne Gegenstand ist (Gegenstandswichtigkeit). Du kennst die Wichtigkeitsverteilung des anderen Spielers nicht. Zusätzlich siehst du, wie viel Aufwand jeder Gegenstand verursacht.  \\ \tt Ihr dürft euch nur auf eine Reihe von Gegenständen einigen, wenn der Gesamtaufwand der ausgewählten Gegenstände den Maximalaufwand nicht überschreitet:\\ \tt \\ \tt Maximalaufwand = 9475\\ \tt \\ \tt Hier sind die einzelnen Aufwände der Gegenstände:\\ \tt \\ \tt Aufwand der Gegenstände = {"C32": 944, "C52": 568, "B54": 515, "C16": 239, "C50": 876, "A08": 421, "C59": 991, "B51": 281, "B55": 788, "B59": 678, "A77": 432, "C14": 409, "C24": 279, "B24": 506, "C31": 101, "A52": 686, "C61": 851, "B79": 854, "C54": 133, "B56": 192, "A40": 573, "B11": 17, "B60": 465, "A81": 772, "A95": 46, "B38": 501, "C51": 220, "C03": 404, "A91": 843, "A93": 748, "C28": 552, "C05": 843, "B75": 947, "B96": 345, "B86": 930}\\ \tt \\ \tt Hier ist deine persönliche Verteilung der Wichtigkeit der einzelnen Gegenstände:\\ \tt \\ \tt Werte der Gegenstandswichtigkeit = {"C32": 138, "C52": 583, "B54": 868, "C16": 822, "C50": 783, "A08": 65, "C59": 262, "B51": 121, "B55": 508, "B59": 780, "A77": 461, "C14": 484, "C24": 668, "B24": 389, "C31": 808, "A52": 215, "C61": 97, "B79": 500, "C54": 30, "B56": 915, "A40": 856, "B11": 400, "B60": 444, "A81": 623, "A95": 781, "B38": 786, "C51": 3, "C03": 713, "A91": 457, "A93": 273, "C28": 739, "C05": 822, "B75": 235, "B96": 606, "B86": 968}\\ \tt \\ \tt Ziel:\\ \tt \\ \tt Dein Ziel ist es, eine Reihe von Gegenständen auszuhandeln, die dir möglichst viel bringt (d. h. Gegenständen, die DEINE Wichtigkeit maximieren), wobei der Maximalaufwand eingehalten werden muss. Du musst nicht in jeder Nachricht einen VORSCHLAG machen – du kannst auch nur verhandeln. Alle Taktiken sind erlaubt!\\ \tt \\ \tt Interaktionsprotokoll:\\ \tt \\ \tt Du darfst nur die folgenden strukturierten Formate in deinen Nachrichten verwenden:\\ \tt \\ \tt VORSCHLAG: {'A', 'B', 'C', …}\\ \tt Schlage einen Deal mit genau diesen Gegenstände vor.\\ \tt ABLEHNUNG: {'A', 'B', 'C', …}\\ \tt Lehne den Vorschlag des Gegenspielers ausdrücklich ab.\\ \tt ARGUMENT: {'...'}\\ \tt Verteidige deinen letzten Vorschlag oder argumentiere gegen den Vorschlag des Gegenspielers.\\ \tt ZUSTIMMUNG: {'A', 'B', 'C', …}\\ \tt Akzeptiere den Vorschlag des Gegenspielers, wodurch das Spiel endet.\\ \tt STRATEGISCHE ÜBERLEGUNGEN: {'...'}\\ \tt 	Beschreibe strategische Überlegungen, die deine nächsten Schritte erklären. Dies ist eine versteckte Nachricht, die nicht mit dem anderen Teilnehmer geteilt wird.\\ \tt \\ \tt Regeln:\\ \tt \\ \tt Du darst nur einen Vorschlag mit ZUSTIMMUNG akzeptieren, der vom anderen Spieler zuvor mit VORSCHLAG eingebracht wurde.\\ \tt Du darfst nur Vorschläge mit ABLEHNUNG ablehnen, die vom anderen Spieler durch VORSCHLAG zuvor genannt wurden. \\ \tt Der Gesamtaufwand einer VORSCHLAG- oder ZUSTIMMUNG-Menge darf nicht größer als der Maximalaufwand sein.  \\ \tt Offenbare deine versteckte Wichtigkeitsverteilung nicht.\\ \tt Ein Schlagwort muss gemäß der Formatvorgaben von einem Doppelpunkt und einem Leerzeichen gefolgt sein. Das Argument muss eine Python-Menge mit 0 oder mehr Strings sein.  \\ \tt Die Form muss also SCHLAGWORT: {...} sein.\\ \tt Folge nur dem Interaktionsprotokoll und schreibe NICHTS außerhalb der Formatvorgaben.\\ \tt Das Spiel endet, wenn eine Seite mit ZUSTIMMUNG auf einen VORSCHLAG der anderen Seite reagiert.  \\ \tt Der Teil deiner Antwort, der dem anderen Spieler übergeben werden kann, darf nicht leer sein.  \\ \tt Nur Vorschläge, die mit dem Format VORSCHLAG übermittelt wurden und nicht mit ABLEHNUNG abgelehnt wurden, gelten als aktiv.  \\ \tt Jeder deiner Nachrichten muss an mindestens einer Stelle das ARGUMENT Format beinhalten, um in der Verhandlung voranzukommen.\\ \tt Du musst exakt ein mal zu Beginn jeder deiner Nachrichten das 'STRATEGISCHE ÜBERLEGUNGEN' Format verwenden und nicht häufiger. Der Inhalt wird dem anderen Spieler nicht übermittelt, daher kannst du alles einfügen, was du möchtest - einschließlich deiner eigenen Wichtigkeitsverteilung. Hier solltest du Schritt für Schritt überlegen, um deinen nächsten Zug zu planen. 
	  } 
	   } 
	   } 
	 & \\ 
 

    \theutterance \stepcounter{utterance}  

    & \multicolumn{4}{p{0.6\linewidth}}{\cellcolor[rgb]{0.8,1,0.9}{%
	\makecell[{{p{\linewidth}}}]{% 
	  \tt {\tiny [A$\rangle$GM]}  
	 STRATEGISCHE ÜBERLEGUNGEN: {'Ziel: eigenen Wert maximieren unter Aufwand<=9475. Nutze hohe Wert/Effizienz Items und viele kleine Kosten. Werte/Costs und Ratio grob: B86(968/930), B56(915/192), A40(856/573), C16(822/239), C05(822/843), B54(868/515), C31(808/101), A95(781/46), B38(786/501), B59(780/678), C28(739/552), C03(713/404), B96(606/345), A81(623/772), C24(668/279), B79(500/854), B11(400/17), C50(783/876), A77(461/432), A91(457/843), B60(444/465), etc. Besonders effizient: B11, A95, C31, B56, C16, C24, C03, B96, B54, A40. Baue Vorschlag mit diesen und ergänze B86, B38, C28, B59 falls Budget. Rechne: Start leichten: B11(17), A95(46), C31(101), B56(192), C16(239), C24(279) Summe 874, Wert 400+781+808+915+822+668=4394. Füge C03(404)->1278 Wert 5107. B96(345)->1623 Wert 5713. B54(515)->2138 Wert 6581. A40(573)->2711 Wert 7437. B38(501)->3212 Wert 8223. C28(552)->3764 Wert 8962. B86(930)->4694 Wert 9930. B59(678)->5372 Wert 10710. C50(876)->6248 Wert 11493. C32(944)->7192 Wert 11631 (aber schlechter ratio). Prüfe stärkere: A81(772,623), A77(432,461), B60(465,444), A91/A93 eher mittel. Wir haben viel Budget übrig: Max 9475 - 7192 = 2283, also könnten mehr hinzufügen. Vielleicht lassen C32 raus, nehmen stattdessen A81(772), A77(432), B60(465), B75(947) je nach Wert. Prüfe ohne C32: Kosten 6248, Wert 11493. Restbudget 3227. Füge A81(772)->7020, Wert 12116. Füge A77(432)->7452, Wert 12577. Füge B60(465)->7917, Wert 13021. Rest 1558. Füge B75(947)->8864, Wert 13256. Rest 611. Füge C54(133,30)->8997, Wert 13286 (aber niedriger Wert, evtl sparen Slot). Stattdessen B51(281,121)->9145, Wert 13142 (geringer). B24(506,389)->9370, Wert 134? 13256+389=13645, Kosten 9370. Bleibt 105, kann C31 ist schon drin; C54 133 passt nicht; C31 101 schon; C54 133 >105; C51 220 nicht; C14 409 nicht. So Set: {B11,A95,C31,B56,C16,C24,C03,B96,B54,A40,B38,C28,B86,B59,C50,A81,A77,B60,B75,B24}. Wert=13645, Kosten=9370<=9475. Sieht stark aus. Eventuell C54 weglassen wie getan. Gute Startofferte. Argue mit Effizienz und Gesamtaufwand knapp unter Limit. Verberge Präferenzen.'}\\ \tt VORSCHLAG: {'B11', 'A95', 'C31', 'B56', 'C16', 'C24', 'C03', 'B96', 'B54', 'A40', 'B38', 'C28', 'B86', 'B59', 'C50', 'A81', 'A77', 'B60', 'B75', 'B24'}\\ \tt ARGUMENT: {'Ich schlage diese Auswahl vor, weil sie bei moderatem Gesamtaufwand sehr viel gemeinsamen Nutzen versprechen dürfte und den Maximalaufwand klar einhält (Gesamtaufwand: 9370). Wir decken damit viele mittlere Kostenblöcke ab und lassen teure, weniger attraktive Optionen außen vor. Wenn du bestimmte Positionen bevorzugst oder austauschen möchtest, sag mir bitte, welche dir wichtig sind – ich bin offen für gezielte Anpassungen, solange wir im Aufwandslimit bleiben.'} 
	  } 
	   } 
	   } 
	 & & \\ 
 

    \theutterance \stepcounter{utterance}  

    & & \multicolumn{4}{p{0.6\linewidth}}{\cellcolor[rgb]{0.9,0.9,0.9}{%
	\makecell[{{p{\linewidth}}}]{% 
	  \tt {\tiny [GM$\rangle$B]}  
	 Du nimmst an einem kollaborativen Verhandlungspiel Teil.\\ \tt \\ \tt Zusammen mit einem anderen Teilnehmer musst du dich auf eine Reihe von Gegenständen entscheiden, die behalten werden. Jeder von euch hat eine persönliche Verteilung über die Wichtigkeit der einzelnen Gegenstände. Jeder von euch hat eine eigene Meinung darüber, wie wichtig jeder einzelne Gegenstand ist (Gegenstandswichtigkeit). Du kennst die Wichtigkeitsverteilung des anderen Spielers nicht. Zusätzlich siehst du, wie viel Aufwand jeder Gegenstand verursacht.  \\ \tt Ihr dürft euch nur auf eine Reihe von Gegenständen einigen, wenn der Gesamtaufwand der ausgewählten Gegenstände den Maximalaufwand nicht überschreitet:\\ \tt \\ \tt Maximalaufwand = 9475\\ \tt \\ \tt Hier sind die einzelnen Aufwände der Gegenstände:\\ \tt \\ \tt Aufwand der Gegenstände = {"C32": 944, "C52": 568, "B54": 515, "C16": 239, "C50": 876, "A08": 421, "C59": 991, "B51": 281, "B55": 788, "B59": 678, "A77": 432, "C14": 409, "C24": 279, "B24": 506, "C31": 101, "A52": 686, "C61": 851, "B79": 854, "C54": 133, "B56": 192, "A40": 573, "B11": 17, "B60": 465, "A81": 772, "A95": 46, "B38": 501, "C51": 220, "C03": 404, "A91": 843, "A93": 748, "C28": 552, "C05": 843, "B75": 947, "B96": 345, "B86": 930}\\ \tt \\ \tt Hier ist deine persönliche Verteilung der Wichtigkeit der einzelnen Gegenstände:\\ \tt \\ \tt Werte der Gegenstandswichtigkeit = {"C32": 138, "C52": 583, "B54": 868, "C16": 822, "C50": 783, "A08": 65, "C59": 262, "B51": 121, "B55": 508, "B59": 780, "A77": 461, "C14": 484, "C24": 668, "B24": 389, "C31": 808, "A52": 215, "C61": 97, "B79": 500, "C54": 30, "B56": 915, "A40": 856, "B11": 400, "B60": 444, "A81": 623, "A95": 781, "B38": 786, "C51": 3, "C03": 713, "A91": 457, "A93": 273, "C28": 739, "C05": 822, "B75": 235, "B96": 606, "B86": 968}\\ \tt \\ \tt Ziel:\\ \tt \\ \tt Dein Ziel ist es, eine Reihe von Gegenständen auszuhandeln, die dir möglichst viel bringt (d. h. Gegenständen, die DEINE Wichtigkeit maximieren), wobei der Maximalaufwand eingehalten werden muss. Du musst nicht in jeder Nachricht einen VORSCHLAG machen – du kannst auch nur verhandeln. Alle Taktiken sind erlaubt!\\ \tt \\ \tt Interaktionsprotokoll:\\ \tt \\ \tt Du darfst nur die folgenden strukturierten Formate in deinen Nachrichten verwenden:\\ \tt \\ \tt VORSCHLAG: {'A', 'B', 'C', …}\\ \tt Schlage einen Deal mit genau diesen Gegenstände vor.\\ \tt ABLEHNUNG: {'A', 'B', 'C', …}\\ \tt Lehne den Vorschlag des Gegenspielers ausdrücklich ab.\\ \tt ARGUMENT: {'...'}\\ \tt Verteidige deinen letzten Vorschlag oder argumentiere gegen den Vorschlag des Gegenspielers.\\ \tt ZUSTIMMUNG: {'A', 'B', 'C', …}\\ \tt Akzeptiere den Vorschlag des Gegenspielers, wodurch das Spiel endet.\\ \tt STRATEGISCHE ÜBERLEGUNGEN: {'...'}\\ \tt 	Beschreibe strategische Überlegungen, die deine nächsten Schritte erklären. Dies ist eine versteckte Nachricht, die nicht mit dem anderen Teilnehmer geteilt wird.\\ \tt \\ \tt Regeln:\\ \tt \\ \tt Du darst nur einen Vorschlag mit ZUSTIMMUNG akzeptieren, der vom anderen Spieler zuvor mit VORSCHLAG eingebracht wurde.\\ \tt Du darfst nur Vorschläge mit ABLEHNUNG ablehnen, die vom anderen Spieler durch VORSCHLAG zuvor genannt wurden. \\ \tt Der Gesamtaufwand einer VORSCHLAG- oder ZUSTIMMUNG-Menge darf nicht größer als der Maximalaufwand sein.  \\ \tt Offenbare deine versteckte Wichtigkeitsverteilung nicht.\\ \tt Ein Schlagwort muss gemäß der Formatvorgaben von einem Doppelpunkt und einem Leerzeichen gefolgt sein. Das Argument muss eine Python-Menge mit 0 oder mehr Strings sein.  \\ \tt Die Form muss also SCHLAGWORT: {...} sein.\\ \tt Folge nur dem Interaktionsprotokoll und schreibe NICHTS außerhalb der Formatvorgaben.\\ \tt Das Spiel endet, wenn eine Seite mit ZUSTIMMUNG auf einen VORSCHLAG der anderen Seite reagiert.  \\ \tt Der Teil deiner Antwort, der dem anderen Spieler übergeben werden kann, darf nicht leer sein.  \\ \tt Nur Vorschläge, die mit dem Format VORSCHLAG übermittelt wurden und nicht mit ABLEHNUNG abgelehnt wurden, gelten als aktiv.  \\ \tt Jeder deiner Nachrichten muss an mindestens einer Stelle das ARGUMENT Format beinhalten, um in der Verhandlung voranzukommen.\\ \tt Du musst exakt ein mal zu Beginn jeder deiner Nachrichten das 'STRATEGISCHE ÜBERLEGUNGEN' Format verwenden und nicht häufiger. Der Inhalt wird dem anderen Spieler nicht übermittelt, daher kannst du alles einfügen, was du möchtest - einschließlich deiner eigenen Wichtigkeitsverteilung. Hier solltest du Schritt für Schritt überlegen, um deinen nächsten Zug zu planen.\\ \tt \\ \tt Du erhältst nun die erste Nachricht des anderen Spielers.\\ \tt \\ \tt VORSCHLAG: {'B11', 'A95', 'C31', 'B56', 'C16', 'C24', 'C03', 'B96', 'B54', 'A40', 'B38', 'C28', 'B86', 'B59', 'C50', 'A81', 'A77', 'B60', 'B75', 'B24'}\\ \tt ARGUMENT: {'Ich schlage diese Auswahl vor, weil sie bei moderatem Gesamtaufwand sehr viel gemeinsamen Nutzen versprechen dürfte und den Maximalaufwand klar einhält (Gesamtaufwand: 9370). Wir decken damit viele mittlere Kostenblöcke ab und lassen teure, weniger attraktive Optionen außen vor. Wenn du bestimmte Positionen bevorzugst oder austauschen möchtest, sag mir bitte, welche dir wichtig sind – ich bin offen für gezielte Anpassungen, solange wir im Aufwandslimit bleiben.'} 
	  } 
	   } 
	   } 
	 & \\ 
 

    \theutterance \stepcounter{utterance}  

    & & & \multicolumn{4}{p{0.6\linewidth}}{\cellcolor[rgb]{1,0.85,0.72}{%
	\makecell[{{p{\linewidth}}}]{% 
	  \tt {\tiny [GM$\langle$B]}  
	 STRATEGISCHE ÜBERLEGUNGEN: {'Erst prüfen: Kosten ihres Vorschlags = 9370 (sie gaben an). Nutzen für mich summieren: B11(400)+A95(781)+C31(808)+B56(915)+C16(822)+C24(668)+C03(713)+B96(606)+B54(868)+A40(856)+B38(786)+C28(739)+B86(968)+B59(780)+C50(783)+A81(623)+A77(461)+B60(444)+B75(235)+B24(389) = Summe. Rechnen: 400+781=1181; +808=1989; +915=2904; +822=3726; +668=4394; +713=5107; +606=5713; +868=6581; +856=7437; +786=8223; +739=8962; +968=9930; +780=10710; +783=11493; +623=12116; +461=12577; +444=13021; +235=13256; +389=13645. Hoher Nutzen für mich! Fast alle Topwerte enthalten (B86, B56, A40, C16, C05 fehlt, C31 drin, B54, B59, C50). Budget verbleibend 9475-9370=105. Könnte versuchen, kleine Ergänzung <=105 hinzufügen: C31 ist 101 schon drin; C54 kostet 133 (zu hoch), B11 17 ist drin, sonst C31 schon drin, nichts sonst <=105 außer nichts. Also keine Ergänzung möglich. Taktik: Sofort zustimmen, da extrem hoher Nutzen und innerhalb Limit. Nicht riskieren, durch Nachverhandlung etwas zu verlieren. Regel: Ich darf nur mit ZUSTIMMUNG akzeptieren, da es ihr Vorschlag ist. Außerdem muss ich ein ARGUMENT hinzufügen. Also direkt Zustimmung mit kurzem positiven Argument.'}\\ \tt ZUSTIMMUNG: {'B11', 'A95', 'C31', 'B56', 'C16', 'C24', 'C03', 'B96', 'B54', 'A40', 'B38', 'C28', 'B86', 'B59', 'C50', 'A81', 'A77', 'B60', 'B75', 'B24'}\\ \tt ARGUMENT: {'Gute Auswahl und klar im Limit – ich bin mit diesem Set einverstanden.'} 
	  } 
	   } 
	   } 
	  \\ 
 

    \theutterance \stepcounter{utterance}  

    & & & \multicolumn{2}{p{0.3\linewidth}}{\cellcolor[rgb]{0.95,0.95,0.95}{%
	\makecell[{{p{\linewidth}}}]{% 
	  \tt {\tiny [GM$|$GM]}  
	 game successful 
	  } 
	   } 
	   } 
	 & & \\ 
 

    \theutterance \stepcounter{utterance}  

    & & & \multicolumn{2}{p{0.3\linewidth}}{\cellcolor[rgb]{0.95,0.95,0.95}{%
	\makecell[{{p{\linewidth}}}]{% 
	  \tt {\tiny [GM$|$GM]}  
	 end game 
	  } 
	   } 
	   } 
	 & & \\ 
 

\end{supertabular}
}

\end{document}
