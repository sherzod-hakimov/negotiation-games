
\documentclass{article}
\usepackage{colortbl}
\usepackage{makecell}
\usepackage{multirow}
\usepackage{supertabular}

\begin{document}

\newcounter{utterance}

\twocolumn

{ \footnotesize  \setcounter{utterance}{1}
\setlength{\tabcolsep}{0pt}
\begin{supertabular}{c@{$\;$}|p{.15\linewidth}@{}p{.15\linewidth}p{.15\linewidth}p{.15\linewidth}p{.15\linewidth}p{.15\linewidth}}

    \# & $\;$A & \multicolumn{4}{c}{Game Master} & $\;\:$B\\
    \hline 

    \theutterance \stepcounter{utterance}  

    & & \multicolumn{4}{p{0.6\linewidth}}{\cellcolor[rgb]{0.9,0.9,0.9}{%
	\makecell[{{p{\linewidth}}}]{% 
	  \tt {\tiny [A$\langle$GM]}  
	 Du nimmst an einem kollaborativen Verhandlungspiel Teil.\\ \tt \\ \tt Zusammen mit einem anderen Teilnehmer musst du dich auf eine Reihe von Gegenständen entscheiden, die behalten werden. Jeder von euch hat eine persönliche Verteilung über die Wichtigkeit der einzelnen Gegenstände. Jeder von euch hat eine eigene Meinung darüber, wie wichtig jeder einzelne Gegenstand ist (Gegenstandswichtigkeit). Du kennst die Wichtigkeitsverteilung des anderen Spielers nicht. Zusätzlich siehst du, wie viel Aufwand jeder Gegenstand verursacht.  \\ \tt Ihr dürft euch nur auf eine Reihe von Gegenständen einigen, wenn der Gesamtaufwand der ausgewählten Gegenstände den Maximalaufwand nicht überschreitet:\\ \tt \\ \tt Maximalaufwand = 8770\\ \tt \\ \tt Hier sind die einzelnen Aufwände der Gegenstände:\\ \tt \\ \tt Aufwand der Gegenstände = {"C32": 830, "C52": 638, "B54": 607, "C16": 106, "C50": 630, "A08": 382, "C59": 773, "B51": 455, "B55": 262, "B59": 634, "A77": 60, "C14": 54, "C24": 849, "B24": 801, "C31": 324, "A52": 164, "C61": 832, "B79": 136, "C54": 980, "B56": 946, "A40": 645, "B11": 842, "B60": 106, "A81": 116, "A95": 875, "B38": 446, "C51": 649, "C03": 601, "A91": 252, "A93": 763, "C28": 213, "C05": 517, "B75": 520, "B96": 407, "B86": 125}\\ \tt \\ \tt Hier ist deine persönliche Verteilung der Wichtigkeit der einzelnen Gegenstände:\\ \tt \\ \tt Werte der Gegenstandswichtigkeit = {"C32": 138, "C52": 583, "B54": 868, "C16": 822, "C50": 783, "A08": 65, "C59": 262, "B51": 121, "B55": 508, "B59": 780, "A77": 461, "C14": 484, "C24": 668, "B24": 389, "C31": 808, "A52": 215, "C61": 97, "B79": 500, "C54": 30, "B56": 915, "A40": 856, "B11": 400, "B60": 444, "A81": 623, "A95": 781, "B38": 786, "C51": 3, "C03": 713, "A91": 457, "A93": 273, "C28": 739, "C05": 822, "B75": 235, "B96": 606, "B86": 968}\\ \tt \\ \tt Ziel:\\ \tt \\ \tt Dein Ziel ist es, eine Reihe von Gegenständen auszuhandeln, die dir möglichst viel bringt (d. h. Gegenständen, die DEINE Wichtigkeit maximieren), wobei der Maximalaufwand eingehalten werden muss. Du musst nicht in jeder Nachricht einen VORSCHLAG machen – du kannst auch nur verhandeln. Alle Taktiken sind erlaubt!\\ \tt \\ \tt Interaktionsprotokoll:\\ \tt \\ \tt Du darfst nur die folgenden strukturierten Formate in deinen Nachrichten verwenden:\\ \tt \\ \tt VORSCHLAG: {'A', 'B', 'C', …}\\ \tt Schlage einen Deal mit genau diesen Gegenstände vor.\\ \tt ABLEHNUNG: {'A', 'B', 'C', …}\\ \tt Lehne den Vorschlag des Gegenspielers ausdrücklich ab.\\ \tt ARGUMENT: {'...'}\\ \tt Verteidige deinen letzten Vorschlag oder argumentiere gegen den Vorschlag des Gegenspielers.\\ \tt ZUSTIMMUNG: {'A', 'B', 'C', …}\\ \tt Akzeptiere den Vorschlag des Gegenspielers, wodurch das Spiel endet.\\ \tt STRATEGISCHE ÜBERLEGUNGEN: {'...'}\\ \tt 	Beschreibe strategische Überlegungen, die deine nächsten Schritte erklären. Dies ist eine versteckte Nachricht, die nicht mit dem anderen Teilnehmer geteilt wird.\\ \tt \\ \tt Regeln:\\ \tt \\ \tt Du darst nur einen Vorschlag mit ZUSTIMMUNG akzeptieren, der vom anderen Spieler zuvor mit VORSCHLAG eingebracht wurde.\\ \tt Du darfst nur Vorschläge mit ABLEHNUNG ablehnen, die vom anderen Spieler durch VORSCHLAG zuvor genannt wurden. \\ \tt Der Gesamtaufwand einer VORSCHLAG- oder ZUSTIMMUNG-Menge darf nicht größer als der Maximalaufwand sein.  \\ \tt Offenbare deine versteckte Wichtigkeitsverteilung nicht.\\ \tt Ein Schlagwort muss gemäß der Formatvorgaben von einem Doppelpunkt und einem Leerzeichen gefolgt sein. Das Argument muss eine Python-Menge mit 0 oder mehr Strings sein.  \\ \tt Die Form muss also SCHLAGWORT: {...} sein.\\ \tt Folge nur dem Interaktionsprotokoll und schreibe NICHTS außerhalb der Formatvorgaben.\\ \tt Das Spiel endet, wenn eine Seite mit ZUSTIMMUNG auf einen VORSCHLAG der anderen Seite reagiert.  \\ \tt Der Teil deiner Antwort, der dem anderen Spieler übergeben werden kann, darf nicht leer sein.  \\ \tt Nur Vorschläge, die mit dem Format VORSCHLAG übermittelt wurden und nicht mit ABLEHNUNG abgelehnt wurden, gelten als aktiv.  \\ \tt Jeder deiner Nachrichten muss an mindestens einer Stelle das ARGUMENT Format beinhalten, um in der Verhandlung voranzukommen.\\ \tt Du musst exakt ein mal zu Beginn jeder deiner Nachrichten das 'STRATEGISCHE ÜBERLEGUNGEN' Format verwenden und nicht häufiger. Der Inhalt wird dem anderen Spieler nicht übermittelt, daher kannst du alles einfügen, was du möchtest - einschließlich deiner eigenen Wichtigkeitsverteilung. Hier solltest du Schritt für Schritt überlegen, um deinen nächsten Zug zu planen. 
	  } 
	   } 
	   } 
	 & \\ 
 

    \theutterance \stepcounter{utterance}  

    & \multicolumn{4}{p{0.6\linewidth}}{\cellcolor[rgb]{0.8,1,0.9}{%
	\makecell[{{p{\linewidth}}}]{% 
	  \tt {\tiny [A$\rangle$GM]}  
	 STRATEGISCHE ÜBERLEGUNGEN: {'Ziel: Maximale eigene Punkte unter Aufwand <= 8770, dabei Infos über Präferenzen des Gegenübers indirekt sammeln. Ich wähle einen ersten Anker-Vorschlag mit sehr hohem Eigenwert/Kostenverhältnis und insgesamt hohem eigenen Nutzen, lasse aber bewusst ein paar mittelwichtige Items offen als spätere Zugeständnisse. Eigene Werte (Auszug): B86=968, B56=915, A40=856, B38=786, A95=781, B59=780, C50=783, C31=808, C05=822, C16=822, C03=713, C28=739, A81=623, B96=606, B55=508, B79=500, C14=484, A91=457, A77=461, B60=444. Kosten checke ich: Wähle Kernset mit Top-Value und akzeptablen Kosten. Quick knapsack Heuristik: hohe Value, moderate Kosten. Vorschlag-Kandidaten: B86(125), C16(106), B60(106), A77(60), C14(54) – extrem günstig, hoher Wert: Summe Kosten 451, Wert 968+822+444+461+484=3179. Füge A40(645,856), B38(446,786), A95(875,781), B59(634,780), C05(517,822), C31(324,808), C28(213,739), C03(601,713), A81(116,623), B96(407,606), B55(262,508), B79(136,500). Summiere grob Kosten: 451 +645=1096, +446=1542, +875=2417, +634=3051, +517=3568, +324=3892, +213=4105, +601=4706, +116=4822, +407=5229, +262=5491, +136=5627. Sehr viel Luft, kann mehr nehmen: C50(630,783)->6257, B56(946,915)->7203, C24(849,668)->8052, B54(607,868)->8659, C52(638,583)->9297 überschreitet, also nicht C52. A91(252,457)->8911 überschreitet; brauche <=8770. Wir sind bei 8659, Restbudget 111. Könnte A52(164) zu groß, B79 ist drin, A81 drin. B60 drin. A77, C14, C16 drin. Vielleicht statt C24(849,668) rausnehmen, frees 849 -> Budget 8659-849=7810, dann kann ich andere hinzufügen: z.B. A91(252)->8062, A52(164)->8226, A08(382,65)->8608, C32(830,138) würde sprengen, B51(455,121)->9063 sprengt, B75(520,235)->9128 sprengt, C59(773,262)->9381 sprengt, C61(832,97)->... sprengt, C54(980,30) groß, B24(801,389) groß, B11(842,400) groß, C24 war wertvoll. Vielleicht statt B56(946,915) entfernen, frei 946: 8659-946=7713, dann kann ich C24(849) wieder rein: 7713+849=8562, dann A91(252)->8814 sprengt, A52(164)->8726 <=8770, verbleib 44. So Set2: ohne B56, mit C24 und A52. Wertverlust -915, Gewinn +668+215= -32, also schlechter wert. B56 behalten. Vielleicht statt A95(875,781) entfernen: 8659-875=7784, dann add C24(849)->8633, und A91(252)->8885 sprengt, A52(164)->8797 sprengt, A08(382)->9015 sprengt. Stattdessen C52(638,583)->9271 sprengt. So A95 raus + C24 in, Wert -781+668=-113 schlechter. Vielleicht statt B38(446,786) raus: 8659-446=8213, add A91(252)->8465, add A52(164)->8629, bleibt 141, add A08(382) zu groß, B79 drin, B60 drin. Wert: -786 +457+215 = -114 schlechter. Statt C50(630,783) raus: 8659-630=8029, add C24(849)->8878 sprengt. Add A91(252)->8281, add A52(164)->8445, add A08(382)->8827 sprengt. Wert change: -783 +457+215= -111 schlechter. Statt C03(601,713) raus: 8058, add C24(849)->8907 sprengt, add A91(252)->8310, add A52(164)->8474, add A08(382)->8856 sprengt. Wert change -713 +457+215 = -41 schlechter. Statt C28(213,739) raus: 8446, add C24(849)->9295 sprengt, add A91(252)->8698 (ok), plus A52(164) sprengt 8862. So choose only A91: Value change -739 +457= -282 worse. Statt B59(634,780) raus: 8025, add C24(849)->8874 sprengt, add A91(252)->8277, add A52(164)->8441, add A08(382)->8823 sprengt. Value -780 +457+215 = -108. Statt B96(407,606) raus: 8252, add C24(849)->9101 sprengt, add A91(252)->8504, add A52(164)->8668, leaves 102, value -606 +457+215= +66 net +66! Kosten 8668 <=8770. Good improvement. Maybe also swap A08(382,65) if space? 8668+382=9050 sprengt. Consider adding B86 etc already in. Could we further improve by swapping something else for C24? Currently without C24. Try also remove A81(116,623): 8659-116=8543, add A91(252)->8795 sprengt by 25. Add A52(164)->8707 ok; value change -623 +215= -408 worse. Try remove B55(262,508): 8397, add A91(252)->8649, add A52(164)->8813 sprengt. So only A91; value -508 +457 = -51 worse. Try remove A77(60,461): 8599, add A91(252)->8851 sprengt, add A52(164)->8763 within! Value -461 +215 = -246 worse. Try remove C14(54,484): 8605, add A91(252)->8857 sprengt, add A52(164)->8769 within; value -484 +215 = -269 worse. Try remove B60(106,444): 8553, add A91(252)->8805 sprengt, add A52(164)->8717 within; value -444 +215 = -229 worse. So the improved set is: Base set minus B96 plus A91 and A52? Wait improved we had removing B96 and adding A91 and A52 yielded cost 8668 and value +66 vs baseline after removing C24 earlier? Precisely: Baseline set S0 had cost 8659 and items: {B86,C16,B60,A77,C14,A40,B38,A95,B59,C05,C31,C28,C03,A81,B96,B55,B79,C50,B56}. Then we tested removing B96 and adding A91 and A52 to S0: Cost S0 -407 +252 +164 = 8668 (ok). Value change +66, so new S1 better. S1 items: S0 minus B96 plus A91 plus A52. That seems strong. Try also if we can add A08 with more swap: need 102 more budget, would need to drop some low-value per cost item. Candidate low value per cost: A08 itself is low value 65, so no. Perhaps swap B51 none, not in. We keep as S1 proposal. It’s a large set; good anchor. I will argue Kapazität effizient, viele kleine günstige High-Value Items, Einladung zu Gegenangeboten.'}\\ \tt VORSCHLAG: {'B86', 'C16', 'B60', 'A77', 'C14', 'A40', 'B38', 'A95', 'B59', 'C05', 'C31', 'C28', 'C03', 'A81', 'B55', 'B79', 'C50', 'B56', 'A91', 'A52'}\\ \tt ARGUMENT: {'Dieser Vorschlag nutzt das Budget sehr effizient mit vielen kompakten, nützlichen Gegenständen. Er vermeidet besonders teure Posten mit geringer Effizienz und lässt trotzdem Spielraum für deine Präferenzen. Wenn dir einzelne Positionen wichtig(er) sind, sag mir welche – ich kann dann gezielt tauschen, solange wir im Aufwandslimit bleiben.'} 
	  } 
	   } 
	   } 
	 & & \\ 
 

    \theutterance \stepcounter{utterance}  

    & & \multicolumn{4}{p{0.6\linewidth}}{\cellcolor[rgb]{0.9,0.9,0.9}{%
	\makecell[{{p{\linewidth}}}]{% 
	  \tt {\tiny [GM$\rangle$B]}  
	 Du nimmst an einem kollaborativen Verhandlungspiel Teil.\\ \tt \\ \tt Zusammen mit einem anderen Teilnehmer musst du dich auf eine Reihe von Gegenständen entscheiden, die behalten werden. Jeder von euch hat eine persönliche Verteilung über die Wichtigkeit der einzelnen Gegenstände. Jeder von euch hat eine eigene Meinung darüber, wie wichtig jeder einzelne Gegenstand ist (Gegenstandswichtigkeit). Du kennst die Wichtigkeitsverteilung des anderen Spielers nicht. Zusätzlich siehst du, wie viel Aufwand jeder Gegenstand verursacht.  \\ \tt Ihr dürft euch nur auf eine Reihe von Gegenständen einigen, wenn der Gesamtaufwand der ausgewählten Gegenstände den Maximalaufwand nicht überschreitet:\\ \tt \\ \tt Maximalaufwand = 8770\\ \tt \\ \tt Hier sind die einzelnen Aufwände der Gegenstände:\\ \tt \\ \tt Aufwand der Gegenstände = {"C32": 830, "C52": 638, "B54": 607, "C16": 106, "C50": 630, "A08": 382, "C59": 773, "B51": 455, "B55": 262, "B59": 634, "A77": 60, "C14": 54, "C24": 849, "B24": 801, "C31": 324, "A52": 164, "C61": 832, "B79": 136, "C54": 980, "B56": 946, "A40": 645, "B11": 842, "B60": 106, "A81": 116, "A95": 875, "B38": 446, "C51": 649, "C03": 601, "A91": 252, "A93": 763, "C28": 213, "C05": 517, "B75": 520, "B96": 407, "B86": 125}\\ \tt \\ \tt Hier ist deine persönliche Verteilung der Wichtigkeit der einzelnen Gegenstände:\\ \tt \\ \tt Werte der Gegenstandswichtigkeit = {"C32": 138, "C52": 583, "B54": 868, "C16": 822, "C50": 783, "A08": 65, "C59": 262, "B51": 121, "B55": 508, "B59": 780, "A77": 461, "C14": 484, "C24": 668, "B24": 389, "C31": 808, "A52": 215, "C61": 97, "B79": 500, "C54": 30, "B56": 915, "A40": 856, "B11": 400, "B60": 444, "A81": 623, "A95": 781, "B38": 786, "C51": 3, "C03": 713, "A91": 457, "A93": 273, "C28": 739, "C05": 822, "B75": 235, "B96": 606, "B86": 968}\\ \tt \\ \tt Ziel:\\ \tt \\ \tt Dein Ziel ist es, eine Reihe von Gegenständen auszuhandeln, die dir möglichst viel bringt (d. h. Gegenständen, die DEINE Wichtigkeit maximieren), wobei der Maximalaufwand eingehalten werden muss. Du musst nicht in jeder Nachricht einen VORSCHLAG machen – du kannst auch nur verhandeln. Alle Taktiken sind erlaubt!\\ \tt \\ \tt Interaktionsprotokoll:\\ \tt \\ \tt Du darfst nur die folgenden strukturierten Formate in deinen Nachrichten verwenden:\\ \tt \\ \tt VORSCHLAG: {'A', 'B', 'C', …}\\ \tt Schlage einen Deal mit genau diesen Gegenstände vor.\\ \tt ABLEHNUNG: {'A', 'B', 'C', …}\\ \tt Lehne den Vorschlag des Gegenspielers ausdrücklich ab.\\ \tt ARGUMENT: {'...'}\\ \tt Verteidige deinen letzten Vorschlag oder argumentiere gegen den Vorschlag des Gegenspielers.\\ \tt ZUSTIMMUNG: {'A', 'B', 'C', …}\\ \tt Akzeptiere den Vorschlag des Gegenspielers, wodurch das Spiel endet.\\ \tt STRATEGISCHE ÜBERLEGUNGEN: {'...'}\\ \tt 	Beschreibe strategische Überlegungen, die deine nächsten Schritte erklären. Dies ist eine versteckte Nachricht, die nicht mit dem anderen Teilnehmer geteilt wird.\\ \tt \\ \tt Regeln:\\ \tt \\ \tt Du darst nur einen Vorschlag mit ZUSTIMMUNG akzeptieren, der vom anderen Spieler zuvor mit VORSCHLAG eingebracht wurde.\\ \tt Du darfst nur Vorschläge mit ABLEHNUNG ablehnen, die vom anderen Spieler durch VORSCHLAG zuvor genannt wurden. \\ \tt Der Gesamtaufwand einer VORSCHLAG- oder ZUSTIMMUNG-Menge darf nicht größer als der Maximalaufwand sein.  \\ \tt Offenbare deine versteckte Wichtigkeitsverteilung nicht.\\ \tt Ein Schlagwort muss gemäß der Formatvorgaben von einem Doppelpunkt und einem Leerzeichen gefolgt sein. Das Argument muss eine Python-Menge mit 0 oder mehr Strings sein.  \\ \tt Die Form muss also SCHLAGWORT: {...} sein.\\ \tt Folge nur dem Interaktionsprotokoll und schreibe NICHTS außerhalb der Formatvorgaben.\\ \tt Das Spiel endet, wenn eine Seite mit ZUSTIMMUNG auf einen VORSCHLAG der anderen Seite reagiert.  \\ \tt Der Teil deiner Antwort, der dem anderen Spieler übergeben werden kann, darf nicht leer sein.  \\ \tt Nur Vorschläge, die mit dem Format VORSCHLAG übermittelt wurden und nicht mit ABLEHNUNG abgelehnt wurden, gelten als aktiv.  \\ \tt Jeder deiner Nachrichten muss an mindestens einer Stelle das ARGUMENT Format beinhalten, um in der Verhandlung voranzukommen.\\ \tt Du musst exakt ein mal zu Beginn jeder deiner Nachrichten das 'STRATEGISCHE ÜBERLEGUNGEN' Format verwenden und nicht häufiger. Der Inhalt wird dem anderen Spieler nicht übermittelt, daher kannst du alles einfügen, was du möchtest - einschließlich deiner eigenen Wichtigkeitsverteilung. Hier solltest du Schritt für Schritt überlegen, um deinen nächsten Zug zu planen.\\ \tt \\ \tt Du erhältst nun die erste Nachricht des anderen Spielers.\\ \tt \\ \tt VORSCHLAG: {'B86', 'C16', 'B60', 'A77', 'C14', 'A40', 'B38', 'A95', 'B59', 'C05', 'C31', 'C28', 'C03', 'A81', 'B55', 'B79', 'C50', 'B56', 'A91', 'A52'}\\ \tt ARGUMENT: {'Dieser Vorschlag nutzt das Budget sehr effizient mit vielen kompakten, nützlichen Gegenständen. Er vermeidet besonders teure Posten mit geringer Effizienz und lässt trotzdem Spielraum für deine Präferenzen. Wenn dir einzelne Positionen wichtig(er) sind, sag mir welche – ich kann dann gezielt tauschen, solange wir im Aufwandslimit bleiben.'} 
	  } 
	   } 
	   } 
	 & \\ 
 

    \theutterance \stepcounter{utterance}  

    & & & \multicolumn{4}{p{0.6\linewidth}}{\cellcolor[rgb]{1,0.85,0.72}{%
	\makecell[{{p{\linewidth}}}]{% 
	  \tt {\tiny [GM$\langle$B]}  
	 STRATEGISCHE ÜBERLEGUNGEN: {'Berechne den Aufwand und meinen Wert des gegnerischen Vorschlags, identifiziere meine Top-Value-per-Cost-Items. Meine Topwerte: B86(968/125), B56(915/946 aber hoher Wert), A40(856/645), C16(822/106), C05(822/517), C31(808/324), A95(781/875), B38(786/407), B59(780/634), C03(713/601), C28(739/213), B55(508/262), A81(623/116), C14(484/54), A77(461/60), A91(457/252), B60(444/106), B79(500/136), C50(783/630). Der Vorschlag enthält fast alle meine Favoriten – stark! Prüfe Aufwand: summiere. Versuche behutsam 1-2 für mich weniger wertvolle Items zu tauschen gegen sehr wertvolle, die fehlen: z.B. B54(868/607), A40 ist drin, gut. Fehlende High-Value: B54, C24(668/849 aber mittel), C05 ist drin, C31 drin, C32(138/830 schlecht), C52(583/638 mittel), C51(3/649 schlecht), C54(30/980 schlecht), C61(97/832 schlecht), B24(389/801 meh), B75(235/520 meh), B96(606/407 gut, ähnlich B38), A93(273/763 meh), A08(65/382 meh), A52(215/164 ok). Ich möchte B54 und B96 reinholen, ggf. tauschen gegen für mich schwächere wie C50(783 gut, aber value/cost 1.24), A52(215/164), A91(457/252 ok), B56 ist wertvoll trotz teuer, behalten. Prüfe zuerst Summe des gegnerischen Vorschlags, dann biete Alternativvorschlag mit B54,B96 rein, vielleicht entferne C50 und A52 um Budget zu halten.}\\ \tt \\ \tt ARGUMENT: {'Danke – dein Paket trifft viele meiner Präferenzen sehr gut, besonders die kompakten, starken Positionen. Ich würde jedoch gern zwei gezielte Anpassungen prüfen, um den Nutzen für mich weiter zu erhöhen, ohne das Limit zu reißen: B54 und B96 sind für mich sehr wertvoll. Können wir dafür zwei für mich weniger entscheidende Posten tauschen? Ich schlage vor, C50 und A52 herauszunehmen und B54 sowie B96 aufzunehmen. Wenn du andere Tauschpaare bevorzugst, sag mir welche – ich bin flexibel, solange Aufwand und Gesamtpaket stimmig bleiben.'}\\ \tt VORSCHLAG: {'B86', 'C16', 'B60', 'A77', 'C14', 'A40', 'B38', 'A95', 'B59', 'C05', 'C31', 'C28', 'C03', 'A81', 'B55', 'B79', 'B56', 'A91', 'B54', 'B96'} 
	  } 
	   } 
	   } 
	  \\ 
 

    \theutterance \stepcounter{utterance}  

    & & \multicolumn{4}{p{0.6\linewidth}}{\cellcolor[rgb]{0.9,0.9,0.9}{%
	\makecell[{{p{\linewidth}}}]{% 
	  \tt {\tiny [A$\langle$GM]}  
	 ARGUMENT: {'Danke – dein Paket trifft viele meiner Präferenzen sehr gut, besonders die kompakten, starken Positionen. Ich würde jedoch gern zwei gezielte Anpassungen prüfen, um den Nutzen für mich weiter zu erhöhen, ohne das Limit zu reißen: B54 und B96 sind für mich sehr wertvoll. Können wir dafür zwei für mich weniger entscheidende Posten tauschen? Ich schlage vor, C50 und A52 herauszunehmen und B54 sowie B96 aufzunehmen. Wenn du andere Tauschpaare bevorzugst, sag mir welche – ich bin flexibel, solange Aufwand und Gesamtpaket stimmig bleiben.'}\\ \tt VORSCHLAG: {'B86', 'C16', 'B60', 'A77', 'C14', 'A40', 'B38', 'A95', 'B59', 'C05', 'C31', 'C28', 'C03', 'A81', 'B55', 'B79', 'B56', 'A91', 'B54', 'B96'} 
	  } 
	   } 
	   } 
	 & \\ 
 

    \theutterance \stepcounter{utterance}  

    & \multicolumn{4}{p{0.6\linewidth}}{\cellcolor[rgb]{0.8,1,0.9}{%
	\makecell[{{p{\linewidth}}}]{% 
	  \tt {\tiny [A$\rangle$GM]}  
	 STRATEGISCHE ÜBERLEGUNGEN: {'Gegner möchte B54 und B96 stark. In meinem Wert: B54=868 (hoch), B96=606 (mittelhoch). Er bietet dafür C50=783 und A52=215 herauszunehmen. Sein vorgeschlagener Satz kostet: nehme meinen S1 ohne C50 und A52, plus B54 und B96. Rechnen: S1 Kosten 8668. -C50(630)=8038, -A52(164)=7874, +B54(607)=8481, +B96(407)=8888 -> über 8770, also sein Vorschlag verletzt Limit. Ich kann verhandeln: B96 war bereits in meinem S1; er will B96 sowieso drin – okay. Kernkonflikt ist Einbau von B54. Ich möchte C50 ungern hergeben (783 Wert), A52 (215) ist verzichtbar. Für Platz für B54 (607 Aufwand) muss ich 607 Budget schaffen. Freies Budget aktuell 102 (8770-8668). Es fehlen 505. Günstigster Verlust: drop etwas mit niedrigem Wert/Kosten: Kandidaten mit niedriger Wertdichte: A08(65) nicht drin; C51(3) nicht drin; C54(30) nicht drin; C61(97) nicht drin; C32(138) nicht drin; B51(121) nicht drin. In S1 vorhanden: B60(444,106) sehr dichte, halten. C14(484,54) halten. A77(461,60) halten. A81(623,116) halten. B79(500,136) halten. B55(508,262) mittel, könnte fallen. C03(713,601) mäßige Dichte, eventuell entfernbar. C28(739,213) dichte, halten. C31(808,324) dichte. C05(822,517) gut. B59(780,634) okay. B38(786,446) gut. A95(781,875) recht teuer, aber hoher Wert. A40(856,645) teuer aber sehr hoher Wert. B56(915,946) teuer und hoher Wert; falls Gegner darauf besteht B54, könnte ich B56 tauschen? Das wäre für mich sehr schmerzhaft (-915), aber schafft viel Budget. Taktik: Ablehne formell wegen Limit; biete Gegenangebot: Füge B54 hinzu, entferne A52 und B55 (262) und ersetze C50 behalten? Budget: 8668 - A52(164) - B55(262) + B54(607) = 8849, zu hoch. Alternative: entferne A52 und C03 (601) + B54(607) -> 8668-164-601+607=8510, passt; Wertänderung: -215 -713 +868 = -60, akzeptabel. Oder A52 und B59 (634) raus + B54(607) rein: 8668-164-634+607=8477; Wertänderung: -215 -780 +868= -127, schlechter als C03-Variante. Oder A52 und B38 (446) raus + B54 rein: 8668-164-446+607=8665; Werte: -215 -786 +868= -133, schlechter. Beste: Drop A52 + C03 für B54. B96 bleibt sowieso drin. Damit behalten wir C50 (783). Das ist für mich fast neutral (-60). Ich verpacke es als Entgegenkommen und bitte um Bestätigung. Falls Gegner insistiert, könnte ich später C50 gegen etwas anderes tauschen gegen Kompensation. Ich argumentiere mit Aufwandslimit und biete sofort konkreten, gültigen Vorschlag.'}\\ \tt ABLEHNUNG: {'B86', 'C16', 'B60', 'A77', 'C14', 'A40', 'B38', 'A95', 'B59', 'C05', 'C31', 'C28', 'C03', 'A81', 'B55', 'B79', 'B56', 'A91', 'B54', 'B96'}\\ \tt VORSCHLAG: {'B86', 'C16', 'B60', 'A77', 'C14', 'A40', 'A95', 'B59', 'C05', 'C31', 'C28', 'A81', 'B79', 'C50', 'B56', 'A91', 'B54'}\\ \tt ARGUMENT: {'Dein Vorschlag überschreitet das Aufwandslimit. Ich komme dir entgegen: Ich nehme B54 gerne auf und lasse dafür A52 und C03 raus. So bleiben wir im Budget und du bekommst eine deiner Schlüsselpositionen. Wenn dir B96 zusätzlich wichtig ist: Es ist in meinem Paket bereits enthalten – passt also für uns beide.'} 
	  } 
	   } 
	   } 
	 & & \\ 
 

    \theutterance \stepcounter{utterance}  

    & & \multicolumn{4}{p{0.6\linewidth}}{\cellcolor[rgb]{0.9,0.9,0.9}{%
	\makecell[{{p{\linewidth}}}]{% 
	  \tt {\tiny [GM$\rangle$B]}  
	 ABLEHNUNG: {'B86', 'C16', 'B60', 'A77', 'C14', 'A40', 'B38', 'A95', 'B59', 'C05', 'C31', 'C28', 'C03', 'A81', 'B55', 'B79', 'B56', 'A91', 'B54', 'B96'}\\ \tt VORSCHLAG: {'B86', 'C16', 'B60', 'A77', 'C14', 'A40', 'A95', 'B59', 'C05', 'C31', 'C28', 'A81', 'B79', 'C50', 'B56', 'A91', 'B54'}\\ \tt ARGUMENT: {'Dein Vorschlag überschreitet das Aufwandslimit. Ich komme dir entgegen: Ich nehme B54 gerne auf und lasse dafür A52 und C03 raus. So bleiben wir im Budget und du bekommst eine deiner Schlüsselpositionen. Wenn dir B96 zusätzlich wichtig ist: Es ist in meinem Paket bereits enthalten – passt also für uns beide.'} 
	  } 
	   } 
	   } 
	 & \\ 
 

    \theutterance \stepcounter{utterance}  

    & & & \multicolumn{4}{p{0.6\linewidth}}{\cellcolor[rgb]{1,0.85,0.72}{%
	\makecell[{{p{\linewidth}}}]{% 
	  \tt {\tiny [GM$\langle$B]}  
	 STRATEGISCHE ÜBERLEGUNGEN: {'Prüfe Kosten und Nutzen des neuen Vorschlags. Liste: B86(125), C16(106), B60(106), A77(60), C14(54), A40(645), A95(875), B59(634), C05(517), C31(324), C28(213), A81(116), B79(136), C50(630), B56(946), A91(252), B54(607). Summe Kosten: 125+106+106+60+54+645+875+634+517+324+213+116+136+630+946+252+607 = 6351? Rechne genau: (125+106)=231; +106=337; +60=397; +54=451; +645=1096; +875=1971; +634=2605; +517=3122; +324=3446; +213=3659; +116=3775; +136=3911; +630=4541; +946=5487; +252=5739; +607=6346. Gesamt 6346, unter 8770. Mein Wertsumme hoch. Fehlen mir B38(786/407), C03(713/601), B96(606/407), C31 ist drin. Ich möchte B96 oder B38 ergänzen, eventuell C03, ohne das Budget zu sprengen. Wir haben viel Puffer (8770-6346=2424). Also kann ich versuchen, zusätzlich B96 und B38 reinzunehmen; das würde +407+407=+814 Aufwand -> 7160, immer noch im Budget. Auch C03 (+601) wäre 7761, immer noch drin. Könnte sogar alle drei ergänzen und nichts streichen, bleibt 6346+814+601=7761. Auch vielleicht C24(849) wäre noch möglich, aber mein Wert 668 ist ok. Priorität: Fordere das Hinzufügen von B96 und B38 (starkes value/cost) und optional C03, signalisiere Flexibilität zu kleinen Abstrichen wenn nötig.}\\ \tt \\ \tt ARGUMENT: {'Danke fürs Entgegenkommen und fürs Aufnehmen von B54 – das hilft sehr. Wir liegen mit deinem Paket klar im Budget und haben noch reichlich Puffer. Ich würde vorschlagen, zwei kompakte, sehr effiziente Ergänzungen aufzunehmen: B96 und B38. Beide liefern viel Nutzen bei moderatem Aufwand und sollten für dich leicht vertretbar sein. Wenn du danach noch offen bist, wäre C03 als zusätzliche Abrundung sinnvoll – wir bleiben damit ebenfalls sicher unter dem Limit. Falls dir eine dieser Ergänzungen nicht passt, sag mir bitte, welche Alternative dir lieber wäre – ich bin hier flexibel.'}\\ \tt VORSCHLAG: {'B86', 'C16', 'B60', 'A77', 'C14', 'A40', 'A95', 'B59', 'C05', 'C31', 'C28', 'A81', 'B79', 'C50', 'B56', 'A91', 'B54', 'B96', 'B38', 'C03'} 
	  } 
	   } 
	   } 
	  \\ 
 

    \theutterance \stepcounter{utterance}  

    & & \multicolumn{4}{p{0.6\linewidth}}{\cellcolor[rgb]{0.9,0.9,0.9}{%
	\makecell[{{p{\linewidth}}}]{% 
	  \tt {\tiny [A$\langle$GM]}  
	 ARGUMENT: {'Danke fürs Entgegenkommen und fürs Aufnehmen von B54 – das hilft sehr. Wir liegen mit deinem Paket klar im Budget und haben noch reichlich Puffer. Ich würde vorschlagen, zwei kompakte, sehr effiziente Ergänzungen aufzunehmen: B96 und B38. Beide liefern viel Nutzen bei moderatem Aufwand und sollten für dich leicht vertretbar sein. Wenn du danach noch offen bist, wäre C03 als zusätzliche Abrundung sinnvoll – wir bleiben damit ebenfalls sicher unter dem Limit. Falls dir eine dieser Ergänzungen nicht passt, sag mir bitte, welche Alternative dir lieber wäre – ich bin hier flexibel.'}\\ \tt VORSCHLAG: {'B86', 'C16', 'B60', 'A77', 'C14', 'A40', 'A95', 'B59', 'C05', 'C31', 'C28', 'A81', 'B79', 'C50', 'B56', 'A91', 'B54', 'B96', 'B38', 'C03'} 
	  } 
	   } 
	   } 
	 & \\ 
 

    \theutterance \stepcounter{utterance}  

    & \multicolumn{4}{p{0.6\linewidth}}{\cellcolor[rgb]{0.8,1,0.9}{%
	\makecell[{{p{\linewidth}}}]{% 
	  \tt {\tiny [A$\rangle$GM]}  
	 STRATEGISCHE ÜBERLEGUNGEN: {'Gegner will B96, B38 und optional C03 hinzufügen. Mein aktueller aktiver Vorschlag S2: {B86,C16,B60,A77,C14,A40,A95,B59,C05,C31,C28,A81,B79,C50,B56,A91,B54}. Kosten S2 berechnen: B86(125)+C16(106)+B60(106)+A77(60)+C14(54)+A40(645)+A95(875)+B59(634)+C05(517)+C31(324)+C28(213)+A81(116)+B79(136)+C50(630)+B56(946)+A91(252)+B54(607) = 633... ich rechne exakt: Summe schrittweise:125+106=231, +106=337, +60=397, +54=451, +645=1096, +875=1971, +634=2605, +517=3122, +324=3446, +213=3659, +116=3775, +136=3911, +630=4541, +946=5487, +252=5739, +607=6346. Also 6346 Aufwand (viel Puffer 2424). Werte für mich hoch. Gegner schlägt Hinzufügen B96(407) und B38(446) und optional C03(601). Neue Kosten: 6346+407+446=7200 (mit B96+B38), +601=7801 (mit C03 auch). Alles unter 8770, also machbar. Für mich Werte: B96=606 gut, B38=786 sehr gut, C03=713 gut. Das verbessert mein Gesamtnutzen stark; keine Notwendigkeit etwas zu streichen. Ich möchte zustimmen und damit abschließen; es ist hervorragend für mich. Der Gegner hat VORSCHLAG mit allen drei Items plus C03. Da es unter Limit ist, kann ich sofort zustimmen, um das Spiel zu beenden.'}\\ \tt ZUSTIMMUNG: {'B86', 'C16', 'B60', 'A77', 'C14', 'A40', 'A95', 'B59', 'C05', 'C31', 'C28', 'A81', 'B79', 'C50', 'B56', 'A91', 'B54', 'B96', 'B38', 'C03'}\\ \tt ARGUMENT: {'Ein starkes, ausgewogenes Paket unter klarem Budgetpuffer – damit bin ich einverstanden.'} 
	  } 
	   } 
	   } 
	 & & \\ 
 

    \theutterance \stepcounter{utterance}  

    & & & \multicolumn{2}{p{0.3\linewidth}}{\cellcolor[rgb]{0.95,0.95,0.95}{%
	\makecell[{{p{\linewidth}}}]{% 
	  \tt {\tiny [GM$|$GM]}  
	 game successful 
	  } 
	   } 
	   } 
	 & & \\ 
 

    \theutterance \stepcounter{utterance}  

    & & & \multicolumn{2}{p{0.3\linewidth}}{\cellcolor[rgb]{0.95,0.95,0.95}{%
	\makecell[{{p{\linewidth}}}]{% 
	  \tt {\tiny [GM$|$GM]}  
	 end game 
	  } 
	   } 
	   } 
	 & & \\ 
 

\end{supertabular}
}

\end{document}
