
\documentclass{article}
\usepackage{colortbl}
\usepackage{makecell}
\usepackage{multirow}
\usepackage{supertabular}

\begin{document}

\newcounter{utterance}

\twocolumn

{ \footnotesize  \setcounter{utterance}{1}
\setlength{\tabcolsep}{0pt}
\begin{supertabular}{c@{$\;$}|p{.15\linewidth}@{}p{.15\linewidth}p{.15\linewidth}p{.15\linewidth}p{.15\linewidth}p{.15\linewidth}}

    \# & $\;$A & \multicolumn{4}{c}{Game Master} & $\;\:$B\\
    \hline 

    \theutterance \stepcounter{utterance}  

    & & \multicolumn{4}{p{0.6\linewidth}}{\cellcolor[rgb]{0.9,0.9,0.9}{%
	\makecell[{{p{\linewidth}}}]{% 
	  \tt {\tiny [A$\langle$GM]}  
	 Ich bin der Spielleiter, und du spielst ein kooperatives Spiel mit dem folgenden Raster als Spielbrett:\\ \tt \\ \tt ```\\ \tt  1234567\\ \tt ╔═══╤═╤═╗\\ \tt ║CH◌│W│◌║ 1\\ \tt ╟─┬─┼─┴─╢ 2\\ \tt ║◌│◌│◌◌◌║ 3\\ \tt ║◌├─┤◌┌─╢ 4\\ \tt ╟─┘◌│◌│◌║ 5\\ \tt ║T◌◌└┬┴─╢ 6\\ \tt ║◌◌◌◌│◌I║ 7\\ \tt ╚════╧══╝\\ \tt \\ \tt ```\\ \tt \\ \tt * Am oberen Rand sind die x-Koordinaten angegeben, die nach rechts hin zunehmen, und am rechten Rand die y-Koordinaten, die nach unten zunehmen.\\ \tt * Die folgenden Objekte sind zufällig auf deinem Raster platziert: 'W', 'I', 'T', 'C', 'H'.\\ \tt \\ \tt \\ \tt Dein Mitspieler sieht eine Variation des Spielbretts, auf dem dieselben Objekte an anderen zufälligen Positionen platziert sind. Du kannst das Spielbrett des anderen Spielers nicht sehen, und er kann deins nicht sehen.\\ \tt \\ \tt \\ \tt **Ziel des Spiels:**\\ \tt \\ \tt Beide Spieler müssen die Objekte auf ihrem jeweiligen Spielbrett so bewegen, dass identische Objekte am Ende an denselben Koordinaten liegen. Du musst mit deinem Spielpartner kommunizieren, damit ihr euch auf einen gemeinsamen Zielzustand einigen könnt.\\ \tt \\ \tt \\ \tt **Spielregeln:**\\ \tt \\ \tt * In jedem Spielzug kannst du genau eine der folgenden zwei Aktionen ausführen:\\ \tt 1. `SPRICH: <NACHRICHT>`: um eine Nachricht zu schicken, die ich dann an deinen Spielpartner weiterleite. Als Nachricht gilt alles bis zum nächsten Zeilenumbruch.\\ \tt 2. `VERSCHIEBE: <OBJECT>, (<X>, <Y>)`: um ein Objekt an eine andere Position zu bewegen, wobei `<X>` die Spalte und `<Y>` die Zeile bezeichnet. Ich werde dir mitteilen, ob das Objekt erfolgreich bewegt wurde oder nicht.\\ \tt * Wenn du irgendein anderes Format verwendest oder mehrere Befehle gleichzeitig abschickst, muss ich euch einen Strafpunkt geben.\\ \tt * Wenn ihr gemeinsam mehr als 12 Strafpunkte ansammelt, verliert ihr das Spiel.\\ \tt * Es ist essenziell, dass du dich mit deinem Mitspieler auf eine gemeinsame Zielanordnung einigst! Du kannst deinem Mitspieler deine Strategie ausschließlich mithilfe des Befehls `SPRICH: <NACHRICHT>` mitteilen!\\ \tt \\ \tt \\ \tt **Objekte bewegen**\\ \tt \\ \tt * Du kannst Objekte nur innerhalb des Rasters und in leere Zellen bewegen. Leere Zellen erkennst am Zeichen '◌'.\\ \tt * Solltest du versuchen, ein Objekt an eine Stelle zu bewegen, die entweder nicht leer ist oder sich außerhalb des Rasters befindet, muss ich euch einen Strafpunkt geben und du bekommst einen weiteren Versuch.\\ \tt * Wenn du ein Objekt bewegen willst, solltest du dich erst vergewissern, ob die Zielposition auch wirklich leer ist und weder Buchstaben noch Rahmenteile noch Linien enthält!\\ \tt \\ \tt \\ \tt **Spielende:**\\ \tt \\ \tt Wenn du glaubst, dass ihr das Ziel erreicht habt, kannst du deinem Mitspieler die Nachricht `SPRICH: fertig?` schicken. Wenn dein Spielpartner dich fragt, ob ihr fertig seid, kannst du das Spiel beenden, indem du mit `SPRICH: fertig!` antwortest.\\ \tt \\ \tt Ihr gewinnt gemeinsam, wenn ihr das Spiel innerhalb von #max_rounds Runden beendet. Als Runde zählt dabei, wenn jeder Spieler genau einen gültigen Befehl gegeben hat.\\ \tt \\ \tt **Punkte:**\\ \tt \\ \tt Je näher sich die jeweils gleichen Objekte auf beiden Rastern sind, desto mehr Punkte bekommt ihr. Für Strafpunkte gibt es Punktabzug. Könnt ihr den Rekord brechen?\\ \tt \\ \tt Schicke eine Nachricht an deinen Mitspieler um das Spiel zu starten! 
	  } 
	   } 
	   } 
	 & \\ 
 

    \theutterance \stepcounter{utterance}  

    & \multicolumn{4}{p{0.6\linewidth}}{\cellcolor[rgb]{0.8,1,0.9}{%
	\makecell[{{p{\linewidth}}}]{% 
	  \tt {\tiny [A$\rangle$GM]}  
	 SPRICH: Hallo! Lass uns zuerst ein gemeinsames Ziel festlegen: Ich schlage vor, wir ordnen die Buchstaben so, dass sie das Wort “WITCH” von links nach rechts in Zeile 4 bilden: W@x=1,4 | I@2,4 | T@3,4 | C@4,4 | H@5,4. Wenn bei dir an diesen Koordinaten etwas blockiert ist, sag mir bitte, welche Spalten in Zeile 4 frei sind, oder schlage eine alternative Zielzeile vor. 
	  } 
	   } 
	   } 
	 & & \\ 
 

    \theutterance \stepcounter{utterance}  

    & & & \multicolumn{2}{p{0.3\linewidth}}{\cellcolor[rgb]{0.95,0.95,0.95}{%
	\makecell[{{p{\linewidth}}}]{% 
	  \tt {\tiny [GM$|$GM]}  
	 SPRICH: Hallo! Lass uns zuerst ein gemeinsames Ziel festlegen: Ich schlage vor, wir ordnen die Buchstaben so, dass sie das Wort “WITCH” von links nach rechts in Zeile 4 bilden: W@x=1,4 | I@2,4 | T@3,4 | C@4,4 | H@5,4. Wenn bei dir an diesen Koordinaten etwas blockiert ist, sag mir bitte, welche Spalten in Zeile 4 frei sind, oder schlage eine alternative Zielzeile vor. 
	  } 
	   } 
	   } 
	 & & \\ 
 

    \theutterance \stepcounter{utterance}  

    & & \multicolumn{4}{p{0.6\linewidth}}{\cellcolor[rgb]{0.9,0.9,0.9}{%
	\makecell[{{p{\linewidth}}}]{% 
	  \tt {\tiny [GM$\rangle$B]}  
	 Ich bin der Spielleiter, und du spielst ein kooperatives Spiel mit dem folgenden Raster als Spielbrett:\\ \tt \\ \tt ```\\ \tt  1234567\\ \tt ╔═══╤═╤═╗\\ \tt ║◌W◌│I│◌║ 1\\ \tt ╟─┬─┼─┴─╢ 2\\ \tt ║◌│C│◌◌H║ 3\\ \tt ║◌├─┤◌┌─╢ 4\\ \tt ╟─┘◌│◌│◌║ 5\\ \tt ║◌◌T└┬┴─╢ 6\\ \tt ║◌◌◌◌│◌◌║ 7\\ \tt ╚════╧══╝\\ \tt \\ \tt ```\\ \tt \\ \tt * Am oberen Rand sind die x-Koordinaten angegeben, die nach rechts hin zunehmen, und am rechten Rand die y-Koordinaten, die nach unten zunehmen.\\ \tt * Die folgenden Objekte sind zufällig auf deinem Raster platziert: 'W', 'I', 'T', 'C', 'H'.\\ \tt \\ \tt \\ \tt Dein Mitspieler sieht eine Variation des Spielbretts, auf dem dieselben Objekte an anderen zufälligen Positionen platziert sind. Du kannst das Spielbrett des anderen Spielers nicht sehen, und er kann deins nicht sehen.\\ \tt \\ \tt \\ \tt **Ziel des Spiels:**\\ \tt \\ \tt Beide Spieler müssen die Objekte auf ihrem jeweiligen Spielbrett so bewegen, dass identische Objekte am Ende an denselben Koordinaten liegen. Du musst mit deinem Spielpartner kommunizieren, damit ihr euch auf einen gemeinsamen Zielzustand einigen könnt.\\ \tt \\ \tt \\ \tt **Spielregeln:**\\ \tt \\ \tt * In jedem Spielzug kannst du genau eine der folgenden zwei Aktionen ausführen:\\ \tt 1. `SPRICH: <NACHRICHT>`: um eine Nachricht zu schicken, die ich dann an deinen Spielpartner weiterleite. Als Nachricht gilt alles bis zum nächsten Zeilenumbruch.\\ \tt 2. `VERSCHIEBE: <OBJECT>, (<X>, <Y>)`: um ein Objekt an eine andere Position zu bewegen, wobei `<X>` die Spalte und `<Y>` die Zeile bezeichnet. Ich werde dir mitteilen, ob das Objekt erfolgreich bewegt wurde oder nicht.\\ \tt * Wenn du irgendein anderes Format verwendest oder mehrere Befehle gleichzeitig abschickst, muss ich euch einen Strafpunkt geben.\\ \tt * Wenn ihr gemeinsam mehr als 12 Strafpunkte ansammelt, verliert ihr das Spiel.\\ \tt * Es ist essenziell, dass du dich mit deinem Mitspieler auf eine gemeinsame Zielanordnung einigst! Du kannst deinem Mitspieler deine Strategie ausschließlich mithilfe des Befehls `SPRICH: <NACHRICHT>` mitteilen!\\ \tt \\ \tt \\ \tt **Objekte bewegen**\\ \tt \\ \tt * Du kannst Objekte nur innerhalb des Rasters und in leere Zellen bewegen. Leere Zellen erkennst am Zeichen '◌'.\\ \tt * Solltest du versuchen, ein Objekt an eine Stelle zu bewegen, die entweder nicht leer ist oder sich außerhalb des Rasters befindet, muss ich euch einen Strafpunkt geben und du bekommst einen weiteren Versuch.\\ \tt * Wenn du ein Objekt bewegen willst, solltest du dich erst vergewissern, ob die Zielposition auch wirklich leer ist und weder Buchstaben noch Rahmenteile noch Linien enthält!\\ \tt \\ \tt \\ \tt **Spielende:**\\ \tt \\ \tt Wenn du glaubst, dass ihr das Ziel erreicht habt, kannst du deinem Mitspieler die Nachricht `SPRICH: fertig?` schicken. Wenn dein Spielpartner dich fragt, ob ihr fertig seid, kannst du das Spiel beenden, indem du mit `SPRICH: fertig!` antwortest.\\ \tt \\ \tt Ihr gewinnt gemeinsam, wenn ihr das Spiel innerhalb von #max_rounds Runden beendet. Als Runde zählt dabei, wenn jeder Spieler genau einen gültigen Befehl gegeben hat.\\ \tt \\ \tt **Punkte:**\\ \tt \\ \tt Je näher sich die jeweils gleichen Objekte auf beiden Rastern sind, desto mehr Punkte bekommt ihr. Für Strafpunkte gibt es Punktabzug. Könnt ihr den Rekord brechen?\\ \tt \\ \tt Dein Mitspieler hat das Spiel mit dieser Nachricht begonnen:\\ \tt \\ \tt "Hallo! Lass uns zuerst ein gemeinsames Ziel festlegen: Ich schlage vor, wir ordnen die Buchstaben so, dass sie das Wort “WITCH” von links nach rechts in Zeile 4 bilden: W@x=1,4 | I@2,4 | T@3,4 | C@4,4 | H@5,4. Wenn bei dir an diesen Koordinaten etwas blockiert ist, sag mir bitte, welche Spalten in Zeile 4 frei sind, oder schlage eine alternative Zielzeile vor."\\ \tt \\ \tt Wie lautet dein erster Befehl? 
	  } 
	   } 
	   } 
	 & \\ 
 

    \theutterance \stepcounter{utterance}  

    & & & \multicolumn{4}{p{0.6\linewidth}}{\cellcolor[rgb]{1,0.85,0.72}{%
	\makecell[{{p{\linewidth}}}]{% 
	  \tt {\tiny [GM$\langle$B]}  
	 SPRICH: Gute Idee! In meiner Zeile 4 sind die Spalten 1,2,4,5 frei; Spalte 3 ist durch eine vertikale Linie blockiert. Alternative: Wir könnten “WITCH” in Zeile 6 linksbündig setzen: W@1,6 I@2,6 T@3,6 C@4,6 H@5,6 – dort sind 1–5 frei. Sag mir, welche der beiden Optionen bei dir passt. 
	  } 
	   } 
	   } 
	  \\ 
 

    \theutterance \stepcounter{utterance}  

    & & & \multicolumn{2}{p{0.3\linewidth}}{\cellcolor[rgb]{0.95,0.95,0.95}{%
	\makecell[{{p{\linewidth}}}]{% 
	  \tt {\tiny [GM$|$GM]}  
	 SPRICH: Gute Idee! In meiner Zeile 4 sind die Spalten 1,2,4,5 frei; Spalte 3 ist durch eine vertikale Linie blockiert. Alternative: Wir könnten “WITCH” in Zeile 6 linksbündig setzen: W@1,6 I@2,6 T@3,6 C@4,6 H@5,6 – dort sind 1–5 frei. Sag mir, welche der beiden Optionen bei dir passt. 
	  } 
	   } 
	   } 
	 & & \\ 
 

    \theutterance \stepcounter{utterance}  

    & & \multicolumn{4}{p{0.6\linewidth}}{\cellcolor[rgb]{0.9,0.9,0.9}{%
	\makecell[{{p{\linewidth}}}]{% 
	  \tt {\tiny [A$\langle$GM]}  
	 Deine Nachricht wurde an deinen Mitspieler übermittelt.\\ \tt \\ \tt \\ \tt Ihr seid gerade bei Runde 1 von maximal 20.\\ \tt Ihr habt gemeinsam 0 von maximal 12 Strafpunkten angesammelt.\\ \tt Dein Mitspieler hat dir diese Nachricht geschickt:\\ \tt \\ \tt \\ \tt "Gute Idee! In meiner Zeile 4 sind die Spalten 1,2,4,5 frei; Spalte 3 ist durch eine vertikale Linie blockiert. Alternative: Wir könnten “WITCH” in Zeile 6 linksbündig setzen: W@1,6 I@2,6 T@3,6 C@4,6 H@5,6 – dort sind 1–5 frei. Sag mir, welche der beiden Optionen bei dir passt."\\ \tt \\ \tt \\ \tt Wie lautet dein nächster Befehl? 
	  } 
	   } 
	   } 
	 & \\ 
 

    \theutterance \stepcounter{utterance}  

    & \multicolumn{4}{p{0.6\linewidth}}{\cellcolor[rgb]{0.8,1,0.9}{%
	\makecell[{{p{\linewidth}}}]{% 
	  \tt {\tiny [A$\rangle$GM]}  
	 SPRICH: Bei mir ist Zeile 4 blockiert (Linien bei x=2–3 und x=5–7). Zeile 6 linksbündig passt perfekt: W@1,6 I@2,6 T@3,6 C@4,6 H@5,6. Ich beginne mit W nach (1,6) zu ziehen. Bitte bestätige, dass du dasselbe Ziel setzt. 
	  } 
	   } 
	   } 
	 & & \\ 
 

    \theutterance \stepcounter{utterance}  

    & & & \multicolumn{2}{p{0.3\linewidth}}{\cellcolor[rgb]{0.95,0.95,0.95}{%
	\makecell[{{p{\linewidth}}}]{% 
	  \tt {\tiny [GM$|$GM]}  
	 SPRICH: Bei mir ist Zeile 4 blockiert (Linien bei x=2–3 und x=5–7). Zeile 6 linksbündig passt perfekt: W@1,6 I@2,6 T@3,6 C@4,6 H@5,6. Ich beginne mit W nach (1,6) zu ziehen. Bitte bestätige, dass du dasselbe Ziel setzt. 
	  } 
	   } 
	   } 
	 & & \\ 
 

    \theutterance \stepcounter{utterance}  

    & & \multicolumn{4}{p{0.6\linewidth}}{\cellcolor[rgb]{0.9,0.9,0.9}{%
	\makecell[{{p{\linewidth}}}]{% 
	  \tt {\tiny [GM$\rangle$B]}  
	 Deine Nachricht wurde an deinen Mitspieler übermittelt.\\ \tt \\ \tt \\ \tt Ihr seid gerade bei Runde 2 von maximal 20.\\ \tt Ihr habt gemeinsam 0 von maximal 12 Strafpunkten angesammelt.\\ \tt Dein Mitspieler hat dir diese Nachricht geschickt:\\ \tt \\ \tt \\ \tt "Bei mir ist Zeile 4 blockiert (Linien bei x=2–3 und x=5–7). Zeile 6 linksbündig passt perfekt: W@1,6 I@2,6 T@3,6 C@4,6 H@5,6. Ich beginne mit W nach (1,6) zu ziehen. Bitte bestätige, dass du dasselbe Ziel setzt."\\ \tt \\ \tt \\ \tt Wie lautet dein nächster Befehl? 
	  } 
	   } 
	   } 
	 & \\ 
 

    \theutterance \stepcounter{utterance}  

    & & & \multicolumn{4}{p{0.6\linewidth}}{\cellcolor[rgb]{1,0.85,0.72}{%
	\makecell[{{p{\linewidth}}}]{% 
	  \tt {\tiny [GM$\langle$B]}  
	 SPRICH: Bestätigt, gemeinsames Ziel ist WITCH in Zeile 6 linksbündig: W@1,6 I@2,6 T@3,6 C@4,6 H@5,6. Ich beginne mit W nach (1,6) zu ziehen. 
	  } 
	   } 
	   } 
	  \\ 
 

    \theutterance \stepcounter{utterance}  

    & & & \multicolumn{2}{p{0.3\linewidth}}{\cellcolor[rgb]{0.95,0.95,0.95}{%
	\makecell[{{p{\linewidth}}}]{% 
	  \tt {\tiny [GM$|$GM]}  
	 SPRICH: Bestätigt, gemeinsames Ziel ist WITCH in Zeile 6 linksbündig: W@1,6 I@2,6 T@3,6 C@4,6 H@5,6. Ich beginne mit W nach (1,6) zu ziehen. 
	  } 
	   } 
	   } 
	 & & \\ 
 

    \theutterance \stepcounter{utterance}  

    & & \multicolumn{4}{p{0.6\linewidth}}{\cellcolor[rgb]{0.9,0.9,0.9}{%
	\makecell[{{p{\linewidth}}}]{% 
	  \tt {\tiny [A$\langle$GM]}  
	 Deine Nachricht wurde an deinen Mitspieler übermittelt.\\ \tt \\ \tt \\ \tt Ihr seid gerade bei Runde 2 von maximal 20.\\ \tt Ihr habt gemeinsam 0 von maximal 12 Strafpunkten angesammelt.\\ \tt Dein Mitspieler hat dir diese Nachricht geschickt:\\ \tt \\ \tt \\ \tt "Bestätigt, gemeinsames Ziel ist WITCH in Zeile 6 linksbündig: W@1,6 I@2,6 T@3,6 C@4,6 H@5,6. Ich beginne mit W nach (1,6) zu ziehen."\\ \tt \\ \tt \\ \tt Wie lautet dein nächster Befehl? 
	  } 
	   } 
	   } 
	 & \\ 
 

    \theutterance \stepcounter{utterance}  

    & \multicolumn{4}{p{0.6\linewidth}}{\cellcolor[rgb]{0.8,1,0.9}{%
	\makecell[{{p{\linewidth}}}]{% 
	  \tt {\tiny [A$\rangle$GM]}  
	 VERSCHIEBE: W, (1, 6) 
	  } 
	   } 
	   } 
	 & & \\ 
 

    \theutterance \stepcounter{utterance}  

    & & & \multicolumn{2}{p{0.3\linewidth}}{\cellcolor[rgb]{0.95,0.95,0.95}{%
	\makecell[{{p{\linewidth}}}]{% 
	  \tt {\tiny [GM$|$GM]}  
	 VERSCHIEBE: W, (1, 6) 
	  } 
	   } 
	   } 
	 & & \\ 
 

    \theutterance \stepcounter{utterance}  

    & & & \multicolumn{2}{p{0.3\linewidth}}{\cellcolor[rgb]{0.95,0.95,0.95}{%
	\makecell[{{p{\linewidth}}}]{% 
	  \tt {\tiny [GM$|$GM]}  
	 Strafpunkt: (1,6) ist nicht leer, sondern enthält 'T'.\\ \tt \\ \tt Ihr habt gemeinsam 1 von maximal 12 Strafpunkten angesammelt.\\ \tt Bitte versuche es noch mal! 
	  } 
	   } 
	   } 
	 & & \\ 
 

    \theutterance \stepcounter{utterance}  

    & & \multicolumn{4}{p{0.6\linewidth}}{\cellcolor[rgb]{0.9,0.9,0.9}{%
	\makecell[{{p{\linewidth}}}]{% 
	  \tt {\tiny [A$\langle$GM]}  
	 Strafpunkt: (1,6) ist nicht leer, sondern enthält 'T'.\\ \tt \\ \tt Ihr habt gemeinsam 1 von maximal 12 Strafpunkten angesammelt.\\ \tt Bitte versuche es noch mal! 
	  } 
	   } 
	   } 
	 & \\ 
 

    \theutterance \stepcounter{utterance}  

    & \multicolumn{4}{p{0.6\linewidth}}{\cellcolor[rgb]{0.8,1,0.9}{%
	\makecell[{{p{\linewidth}}}]{% 
	  \tt {\tiny [A$\rangle$GM]}  
	 SPRICH: Danke für den Hinweis. Bei mir steht T bereits auf (1,6). Aktuelle Positionen: T@1,6; C@1,1; H@2,1; W@2,1; I@7,7. Ich werde zuerst T nach (3,6) verschieben, dann C→4,6; H→5,6; W→1,6; I→2,6. Passt das für dich? 
	  } 
	   } 
	   } 
	 & & \\ 
 

    \theutterance \stepcounter{utterance}  

    & & & \multicolumn{2}{p{0.3\linewidth}}{\cellcolor[rgb]{0.95,0.95,0.95}{%
	\makecell[{{p{\linewidth}}}]{% 
	  \tt {\tiny [GM$|$GM]}  
	 SPRICH: Danke für den Hinweis. Bei mir steht T bereits auf (1,6). Aktuelle Positionen: T@1,6; C@1,1; H@2,1; W@2,1; I@7,7. Ich werde zuerst T nach (3,6) verschieben, dann C→4,6; H→5,6; W→1,6; I→2,6. Passt das für dich? 
	  } 
	   } 
	   } 
	 & & \\ 
 

    \theutterance \stepcounter{utterance}  

    & & \multicolumn{4}{p{0.6\linewidth}}{\cellcolor[rgb]{0.9,0.9,0.9}{%
	\makecell[{{p{\linewidth}}}]{% 
	  \tt {\tiny [GM$\rangle$B]}  
	 Deine Nachricht wurde an deinen Mitspieler übermittelt.\\ \tt \\ \tt \\ \tt Ihr seid gerade bei Runde 3 von maximal 20.\\ \tt Ihr habt gemeinsam 1 von maximal 12 Strafpunkten angesammelt.\\ \tt Dein Mitspieler hat dir diese Nachricht geschickt:\\ \tt \\ \tt \\ \tt "Danke für den Hinweis. Bei mir steht T bereits auf (1,6). Aktuelle Positionen: T@1,6; C@1,1; H@2,1; W@2,1; I@7,7. Ich werde zuerst T nach (3,6) verschieben, dann C→4,6; H→5,6; W→1,6; I→2,6. Passt das für dich?"\\ \tt \\ \tt \\ \tt Wie lautet dein nächster Befehl? 
	  } 
	   } 
	   } 
	 & \\ 
 

    \theutterance \stepcounter{utterance}  

    & & & \multicolumn{4}{p{0.6\linewidth}}{\cellcolor[rgb]{1,0.85,0.72}{%
	\makecell[{{p{\linewidth}}}]{% 
	  \tt {\tiny [GM$\langle$B]}  
	 SPRICH: Passt. Meine aktuellen Positionen: W@2,1; I@4,1; T@3,6; C@2,3; H@5,3. Ich ziehe jetzt W nach (1,6). 
	  } 
	   } 
	   } 
	  \\ 
 

    \theutterance \stepcounter{utterance}  

    & & & \multicolumn{2}{p{0.3\linewidth}}{\cellcolor[rgb]{0.95,0.95,0.95}{%
	\makecell[{{p{\linewidth}}}]{% 
	  \tt {\tiny [GM$|$GM]}  
	 SPRICH: Passt. Meine aktuellen Positionen: W@2,1; I@4,1; T@3,6; C@2,3; H@5,3. Ich ziehe jetzt W nach (1,6). 
	  } 
	   } 
	   } 
	 & & \\ 
 

    \theutterance \stepcounter{utterance}  

    & & \multicolumn{4}{p{0.6\linewidth}}{\cellcolor[rgb]{0.9,0.9,0.9}{%
	\makecell[{{p{\linewidth}}}]{% 
	  \tt {\tiny [A$\langle$GM]}  
	 Deine Nachricht wurde an deinen Mitspieler übermittelt.\\ \tt \\ \tt \\ \tt Ihr seid gerade bei Runde 3 von maximal 20.\\ \tt Ihr habt gemeinsam 1 von maximal 12 Strafpunkten angesammelt.\\ \tt Dein Mitspieler hat dir diese Nachricht geschickt:\\ \tt \\ \tt \\ \tt "Passt. Meine aktuellen Positionen: W@2,1; I@4,1; T@3,6; C@2,3; H@5,3. Ich ziehe jetzt W nach (1,6)."\\ \tt \\ \tt \\ \tt Wie lautet dein nächster Befehl? 
	  } 
	   } 
	   } 
	 & \\ 
 

    \theutterance \stepcounter{utterance}  

    & \multicolumn{4}{p{0.6\linewidth}}{\cellcolor[rgb]{0.8,1,0.9}{%
	\makecell[{{p{\linewidth}}}]{% 
	  \tt {\tiny [A$\rangle$GM]}  
	 VERSCHIEBE: T, (3, 6) 
	  } 
	   } 
	   } 
	 & & \\ 
 

    \theutterance \stepcounter{utterance}  

    & & & \multicolumn{2}{p{0.3\linewidth}}{\cellcolor[rgb]{0.95,0.95,0.95}{%
	\makecell[{{p{\linewidth}}}]{% 
	  \tt {\tiny [GM$|$GM]}  
	 VERSCHIEBE: T, (3, 6) 
	  } 
	   } 
	   } 
	 & & \\ 
 

    \theutterance \stepcounter{utterance}  

    & & & \multicolumn{2}{p{0.3\linewidth}}{\cellcolor[rgb]{0.95,0.95,0.95}{%
	\makecell[{{p{\linewidth}}}]{% 
	  \tt {\tiny [GM$|$GM]}  
	 'T' erfolgreich nach (3,6) verschoben. Dein Raster sieht jetzt so aus:\\ \tt ```\\ \tt  1234567\\ \tt ╔═══╤═╤═╗\\ \tt ║CH◌│W│◌║ 1\\ \tt ╟─┬─┼─┴─╢ 2\\ \tt ║◌│◌│◌◌◌║ 3\\ \tt ║◌├─┤◌┌─╢ 4\\ \tt ╟─┘◌│◌│◌║ 5\\ \tt ║◌◌T└┬┴─╢ 6\\ \tt ║◌◌◌◌│◌I║ 7\\ \tt ╚════╧══╝\\ \tt ```\\ \tt  
	  } 
	   } 
	   } 
	 & & \\ 
 

    \theutterance \stepcounter{utterance}  

    & & \multicolumn{4}{p{0.6\linewidth}}{\cellcolor[rgb]{0.9,0.9,0.9}{%
	\makecell[{{p{\linewidth}}}]{% 
	  \tt {\tiny [GM$\rangle$B]}  
	 Deine Nachricht wurde an deinen Mitspieler übermittelt.\\ \tt \\ \tt \\ \tt Ihr seid gerade bei Runde 4 von maximal 20.\\ \tt Ihr habt gemeinsam 1 von maximal 12 Strafpunkten angesammelt.\\ \tt Dein Mitspieler hat ein Objekt auf seinem Raster bewegt.\\ \tt \\ \tt \\ \tt Wie lautet dein nächster Befehl? 
	  } 
	   } 
	   } 
	 & \\ 
 

    \theutterance \stepcounter{utterance}  

    & & & \multicolumn{4}{p{0.6\linewidth}}{\cellcolor[rgb]{1,0.85,0.72}{%
	\makecell[{{p{\linewidth}}}]{% 
	  \tt {\tiny [GM$\langle$B]}  
	 VERSCHIEBE: W, (1, 6) 
	  } 
	   } 
	   } 
	  \\ 
 

    \theutterance \stepcounter{utterance}  

    & & & \multicolumn{2}{p{0.3\linewidth}}{\cellcolor[rgb]{0.95,0.95,0.95}{%
	\makecell[{{p{\linewidth}}}]{% 
	  \tt {\tiny [GM$|$GM]}  
	 VERSCHIEBE: W, (1, 6) 
	  } 
	   } 
	   } 
	 & & \\ 
 

    \theutterance \stepcounter{utterance}  

    & & & \multicolumn{2}{p{0.3\linewidth}}{\cellcolor[rgb]{0.95,0.95,0.95}{%
	\makecell[{{p{\linewidth}}}]{% 
	  \tt {\tiny [GM$|$GM]}  
	 'W' erfolgreich nach (1,6) verschoben. Dein Raster sieht jetzt so aus:\\ \tt ```\\ \tt  1234567\\ \tt ╔═══╤═╤═╗\\ \tt ║◌◌◌│I│◌║ 1\\ \tt ╟─┬─┼─┴─╢ 2\\ \tt ║◌│C│◌◌H║ 3\\ \tt ║◌├─┤◌┌─╢ 4\\ \tt ╟─┘◌│◌│◌║ 5\\ \tt ║W◌T└┬┴─╢ 6\\ \tt ║◌◌◌◌│◌◌║ 7\\ \tt ╚════╧══╝\\ \tt ```\\ \tt  
	  } 
	   } 
	   } 
	 & & \\ 
 

    \theutterance \stepcounter{utterance}  

    & & \multicolumn{4}{p{0.6\linewidth}}{\cellcolor[rgb]{0.9,0.9,0.9}{%
	\makecell[{{p{\linewidth}}}]{% 
	  \tt {\tiny [A$\langle$GM]}  
	 'T' erfolgreich nach (3,6) verschoben. Dein Raster sieht jetzt so aus:\\ \tt ```\\ \tt  1234567\\ \tt ╔═══╤═╤═╗\\ \tt ║CH◌│W│◌║ 1\\ \tt ╟─┬─┼─┴─╢ 2\\ \tt ║◌│◌│◌◌◌║ 3\\ \tt ║◌├─┤◌┌─╢ 4\\ \tt ╟─┘◌│◌│◌║ 5\\ \tt ║◌◌T└┬┴─╢ 6\\ \tt ║◌◌◌◌│◌I║ 7\\ \tt ╚════╧══╝\\ \tt ```\\ \tt Ihr seid gerade bei Runde 4 von maximal 20.\\ \tt Ihr habt gemeinsam 1 von maximal 12 Strafpunkten angesammelt.\\ \tt Dein Mitspieler hat ein Objekt auf seinem Raster bewegt.\\ \tt \\ \tt \\ \tt Wie lautet dein nächster Befehl? 
	  } 
	   } 
	   } 
	 & \\ 
 

    \theutterance \stepcounter{utterance}  

    & \multicolumn{4}{p{0.6\linewidth}}{\cellcolor[rgb]{0.8,1,0.9}{%
	\makecell[{{p{\linewidth}}}]{% 
	  \tt {\tiny [A$\rangle$GM]}  
	 VERSCHIEBE: C, (4, 6) 
	  } 
	   } 
	   } 
	 & & \\ 
 

    \theutterance \stepcounter{utterance}  

    & & & \multicolumn{2}{p{0.3\linewidth}}{\cellcolor[rgb]{0.95,0.95,0.95}{%
	\makecell[{{p{\linewidth}}}]{% 
	  \tt {\tiny [GM$|$GM]}  
	 VERSCHIEBE: C, (4, 6) 
	  } 
	   } 
	   } 
	 & & \\ 
 

    \theutterance \stepcounter{utterance}  

    & & & \multicolumn{2}{p{0.3\linewidth}}{\cellcolor[rgb]{0.95,0.95,0.95}{%
	\makecell[{{p{\linewidth}}}]{% 
	  \tt {\tiny [GM$|$GM]}  
	 Strafpunkt: (4,6) ist nicht leer, sondern enthält '└'.\\ \tt \\ \tt Ihr habt gemeinsam 2 von maximal 12 Strafpunkten angesammelt.\\ \tt Bitte versuche es noch mal! 
	  } 
	   } 
	   } 
	 & & \\ 
 

    \theutterance \stepcounter{utterance}  

    & & \multicolumn{4}{p{0.6\linewidth}}{\cellcolor[rgb]{0.9,0.9,0.9}{%
	\makecell[{{p{\linewidth}}}]{% 
	  \tt {\tiny [A$\langle$GM]}  
	 Strafpunkt: (4,6) ist nicht leer, sondern enthält '└'.\\ \tt \\ \tt Ihr habt gemeinsam 2 von maximal 12 Strafpunkten angesammelt.\\ \tt Bitte versuche es noch mal! 
	  } 
	   } 
	   } 
	 & \\ 
 

    \theutterance \stepcounter{utterance}  

    & \multicolumn{4}{p{0.6\linewidth}}{\cellcolor[rgb]{0.8,1,0.9}{%
	\makecell[{{p{\linewidth}}}]{% 
	  \tt {\tiny [A$\rangle$GM]}  
	 SPRICH: Danke, (4,6) ist blockiert. Ich plane neue Zielkoordinaten in Zeile 6: W@1,6; I@2,6; T@3,6 bleibt; C@4,6 ist blockiert, daher C@6,6; H@5,6 bleibt Ziel. Ich verschiebe als Nächstes C nach (6,6), dann H nach (5,6), zuletzt W@1,6 und I@2,6. Ist (6,6) bei dir frei, und passt dir diese Anpassung? 
	  } 
	   } 
	   } 
	 & & \\ 
 

    \theutterance \stepcounter{utterance}  

    & & & \multicolumn{2}{p{0.3\linewidth}}{\cellcolor[rgb]{0.95,0.95,0.95}{%
	\makecell[{{p{\linewidth}}}]{% 
	  \tt {\tiny [GM$|$GM]}  
	 SPRICH: Danke, (4,6) ist blockiert. Ich plane neue Zielkoordinaten in Zeile 6: W@1,6; I@2,6; T@3,6 bleibt; C@4,6 ist blockiert, daher C@6,6; H@5,6 bleibt Ziel. Ich verschiebe als Nächstes C nach (6,6), dann H nach (5,6), zuletzt W@1,6 und I@2,6. Ist (6,6) bei dir frei, und passt dir diese Anpassung? 
	  } 
	   } 
	   } 
	 & & \\ 
 

    \theutterance \stepcounter{utterance}  

    & & \multicolumn{4}{p{0.6\linewidth}}{\cellcolor[rgb]{0.9,0.9,0.9}{%
	\makecell[{{p{\linewidth}}}]{% 
	  \tt {\tiny [GM$\rangle$B]}  
	 'W' erfolgreich nach (1,6) verschoben. Dein Raster sieht jetzt so aus:\\ \tt ```\\ \tt  1234567\\ \tt ╔═══╤═╤═╗\\ \tt ║◌◌◌│I│◌║ 1\\ \tt ╟─┬─┼─┴─╢ 2\\ \tt ║◌│C│◌◌H║ 3\\ \tt ║◌├─┤◌┌─╢ 4\\ \tt ╟─┘◌│◌│◌║ 5\\ \tt ║W◌T└┬┴─╢ 6\\ \tt ║◌◌◌◌│◌◌║ 7\\ \tt ╚════╧══╝\\ \tt ```\\ \tt Ihr seid gerade bei Runde 5 von maximal 20.\\ \tt Ihr habt gemeinsam 2 von maximal 12 Strafpunkten angesammelt.\\ \tt Dein Mitspieler hat dir diese Nachricht geschickt:\\ \tt \\ \tt \\ \tt "Danke, (4,6) ist blockiert. Ich plane neue Zielkoordinaten in Zeile 6: W@1,6; I@2,6; T@3,6 bleibt; C@4,6 ist blockiert, daher C@6,6; H@5,6 bleibt Ziel. Ich verschiebe als Nächstes C nach (6,6), dann H nach (5,6), zuletzt W@1,6 und I@2,6. Ist (6,6) bei dir frei, und passt dir diese Anpassung?"\\ \tt \\ \tt \\ \tt Wie lautet dein nächster Befehl? 
	  } 
	   } 
	   } 
	 & \\ 
 

    \theutterance \stepcounter{utterance}  

    & & & \multicolumn{4}{p{0.6\linewidth}}{\cellcolor[rgb]{1,0.85,0.72}{%
	\makecell[{{p{\linewidth}}}]{% 
	  \tt {\tiny [GM$\langle$B]}  
	 SPRICH: Ja, (6,6) ist bei mir frei. Aktualisiertes gemeinsames Ziel für Zeile 6: W@1,6; I@2,6; T@3,6; C@6,6; H@5,6. Ich verschiebe als Nächstes I nach (2,6), danach H nach (5,6) und C nach (6,6). Passt? 
	  } 
	   } 
	   } 
	  \\ 
 

    \theutterance \stepcounter{utterance}  

    & & & \multicolumn{2}{p{0.3\linewidth}}{\cellcolor[rgb]{0.95,0.95,0.95}{%
	\makecell[{{p{\linewidth}}}]{% 
	  \tt {\tiny [GM$|$GM]}  
	 SPRICH: Ja, (6,6) ist bei mir frei. Aktualisiertes gemeinsames Ziel für Zeile 6: W@1,6; I@2,6; T@3,6; C@6,6; H@5,6. Ich verschiebe als Nächstes I nach (2,6), danach H nach (5,6) und C nach (6,6). Passt? 
	  } 
	   } 
	   } 
	 & & \\ 
 

    \theutterance \stepcounter{utterance}  

    & & \multicolumn{4}{p{0.6\linewidth}}{\cellcolor[rgb]{0.9,0.9,0.9}{%
	\makecell[{{p{\linewidth}}}]{% 
	  \tt {\tiny [A$\langle$GM]}  
	 Deine Nachricht wurde an deinen Mitspieler übermittelt.\\ \tt \\ \tt \\ \tt Ihr seid gerade bei Runde 5 von maximal 20.\\ \tt Ihr habt gemeinsam 2 von maximal 12 Strafpunkten angesammelt.\\ \tt Dein Mitspieler hat dir diese Nachricht geschickt:\\ \tt \\ \tt \\ \tt "Ja, (6,6) ist bei mir frei. Aktualisiertes gemeinsames Ziel für Zeile 6: W@1,6; I@2,6; T@3,6; C@6,6; H@5,6. Ich verschiebe als Nächstes I nach (2,6), danach H nach (5,6) und C nach (6,6). Passt?"\\ \tt \\ \tt \\ \tt Wie lautet dein nächster Befehl? 
	  } 
	   } 
	   } 
	 & \\ 
 

    \theutterance \stepcounter{utterance}  

    & \multicolumn{4}{p{0.6\linewidth}}{\cellcolor[rgb]{0.8,1,0.9}{%
	\makecell[{{p{\linewidth}}}]{% 
	  \tt {\tiny [A$\rangle$GM]}  
	 VERSCHIEBE: C, (6, 6) 
	  } 
	   } 
	   } 
	 & & \\ 
 

    \theutterance \stepcounter{utterance}  

    & & & \multicolumn{2}{p{0.3\linewidth}}{\cellcolor[rgb]{0.95,0.95,0.95}{%
	\makecell[{{p{\linewidth}}}]{% 
	  \tt {\tiny [GM$|$GM]}  
	 VERSCHIEBE: C, (6, 6) 
	  } 
	   } 
	   } 
	 & & \\ 
 

    \theutterance \stepcounter{utterance}  

    & & & \multicolumn{2}{p{0.3\linewidth}}{\cellcolor[rgb]{0.95,0.95,0.95}{%
	\makecell[{{p{\linewidth}}}]{% 
	  \tt {\tiny [GM$|$GM]}  
	 Strafpunkt: (6,6) ist nicht leer, sondern enthält '┴'.\\ \tt \\ \tt Ihr habt gemeinsam 3 von maximal 12 Strafpunkten angesammelt.\\ \tt Bitte versuche es noch mal! 
	  } 
	   } 
	   } 
	 & & \\ 
 

    \theutterance \stepcounter{utterance}  

    & & \multicolumn{4}{p{0.6\linewidth}}{\cellcolor[rgb]{0.9,0.9,0.9}{%
	\makecell[{{p{\linewidth}}}]{% 
	  \tt {\tiny [A$\langle$GM]}  
	 Strafpunkt: (6,6) ist nicht leer, sondern enthält '┴'.\\ \tt \\ \tt Ihr habt gemeinsam 3 von maximal 12 Strafpunkten angesammelt.\\ \tt Bitte versuche es noch mal! 
	  } 
	   } 
	   } 
	 & \\ 
 

    \theutterance \stepcounter{utterance}  

    & \multicolumn{4}{p{0.6\linewidth}}{\cellcolor[rgb]{0.8,1,0.9}{%
	\makecell[{{p{\linewidth}}}]{% 
	  \tt {\tiny [A$\rangle$GM]}  
	 SPRICH: Bei mir sind in Zeile 6 die Spalten 4 (└) und 6 (┴) blockiert; frei sind 1–3 und 5 sowie 7. Vorschlag für neues gemeinsames Ziel in Zeile 6: W@1,6; I@2,6; T@3,6; H@5,6; C@7,6. Ich würde als Nächstes H nach (5,6) verschieben und C danach nach (7,6). Passt das bei dir? 
	  } 
	   } 
	   } 
	 & & \\ 
 

    \theutterance \stepcounter{utterance}  

    & & & \multicolumn{2}{p{0.3\linewidth}}{\cellcolor[rgb]{0.95,0.95,0.95}{%
	\makecell[{{p{\linewidth}}}]{% 
	  \tt {\tiny [GM$|$GM]}  
	 SPRICH: Bei mir sind in Zeile 6 die Spalten 4 (└) und 6 (┴) blockiert; frei sind 1–3 und 5 sowie 7. Vorschlag für neues gemeinsames Ziel in Zeile 6: W@1,6; I@2,6; T@3,6; H@5,6; C@7,6. Ich würde als Nächstes H nach (5,6) verschieben und C danach nach (7,6). Passt das bei dir? 
	  } 
	   } 
	   } 
	 & & \\ 
 

    \theutterance \stepcounter{utterance}  

    & & \multicolumn{4}{p{0.6\linewidth}}{\cellcolor[rgb]{0.9,0.9,0.9}{%
	\makecell[{{p{\linewidth}}}]{% 
	  \tt {\tiny [GM$\rangle$B]}  
	 Deine Nachricht wurde an deinen Mitspieler übermittelt.\\ \tt \\ \tt \\ \tt Ihr seid gerade bei Runde 6 von maximal 20.\\ \tt Ihr habt gemeinsam 3 von maximal 12 Strafpunkten angesammelt.\\ \tt Dein Mitspieler hat dir diese Nachricht geschickt:\\ \tt \\ \tt \\ \tt "Bei mir sind in Zeile 6 die Spalten 4 (└) und 6 (┴) blockiert; frei sind 1–3 und 5 sowie 7. Vorschlag für neues gemeinsames Ziel in Zeile 6: W@1,6; I@2,6; T@3,6; H@5,6; C@7,6. Ich würde als Nächstes H nach (5,6) verschieben und C danach nach (7,6). Passt das bei dir?"\\ \tt \\ \tt \\ \tt Wie lautet dein nächster Befehl? 
	  } 
	   } 
	   } 
	 & \\ 
 

    \theutterance \stepcounter{utterance}  

    & & & \multicolumn{4}{p{0.6\linewidth}}{\cellcolor[rgb]{1,0.85,0.72}{%
	\makecell[{{p{\linewidth}}}]{% 
	  \tt {\tiny [GM$\langle$B]}  
	 SPRICH: Passt. Gemeinsames Ziel Zeile 6: W@1,6; I@2,6; T@3,6; H@5,6; C@7,6. Bei mir ist (7,6) frei. Ich ziehe jetzt I nach (2,6), danach H nach (5,6) und C nach (7,6). 
	  } 
	   } 
	   } 
	  \\ 
 

    \theutterance \stepcounter{utterance}  

    & & & \multicolumn{2}{p{0.3\linewidth}}{\cellcolor[rgb]{0.95,0.95,0.95}{%
	\makecell[{{p{\linewidth}}}]{% 
	  \tt {\tiny [GM$|$GM]}  
	 SPRICH: Passt. Gemeinsames Ziel Zeile 6: W@1,6; I@2,6; T@3,6; H@5,6; C@7,6. Bei mir ist (7,6) frei. Ich ziehe jetzt I nach (2,6), danach H nach (5,6) und C nach (7,6). 
	  } 
	   } 
	   } 
	 & & \\ 
 

    \theutterance \stepcounter{utterance}  

    & & \multicolumn{4}{p{0.6\linewidth}}{\cellcolor[rgb]{0.9,0.9,0.9}{%
	\makecell[{{p{\linewidth}}}]{% 
	  \tt {\tiny [A$\langle$GM]}  
	 Deine Nachricht wurde an deinen Mitspieler übermittelt.\\ \tt \\ \tt \\ \tt Ihr seid gerade bei Runde 6 von maximal 20.\\ \tt Ihr habt gemeinsam 3 von maximal 12 Strafpunkten angesammelt.\\ \tt Dein Mitspieler hat dir diese Nachricht geschickt:\\ \tt \\ \tt \\ \tt "Passt. Gemeinsames Ziel Zeile 6: W@1,6; I@2,6; T@3,6; H@5,6; C@7,6. Bei mir ist (7,6) frei. Ich ziehe jetzt I nach (2,6), danach H nach (5,6) und C nach (7,6)."\\ \tt \\ \tt \\ \tt Wie lautet dein nächster Befehl? 
	  } 
	   } 
	   } 
	 & \\ 
 

    \theutterance \stepcounter{utterance}  

    & \multicolumn{4}{p{0.6\linewidth}}{\cellcolor[rgb]{0.8,1,0.9}{%
	\makecell[{{p{\linewidth}}}]{% 
	  \tt {\tiny [A$\rangle$GM]}  
	 VERSCHIEBE: H, (5, 6) 
	  } 
	   } 
	   } 
	 & & \\ 
 

    \theutterance \stepcounter{utterance}  

    & & & \multicolumn{2}{p{0.3\linewidth}}{\cellcolor[rgb]{0.95,0.95,0.95}{%
	\makecell[{{p{\linewidth}}}]{% 
	  \tt {\tiny [GM$|$GM]}  
	 VERSCHIEBE: H, (5, 6) 
	  } 
	   } 
	   } 
	 & & \\ 
 

    \theutterance \stepcounter{utterance}  

    & & & \multicolumn{2}{p{0.3\linewidth}}{\cellcolor[rgb]{0.95,0.95,0.95}{%
	\makecell[{{p{\linewidth}}}]{% 
	  \tt {\tiny [GM$|$GM]}  
	 Strafpunkt: (5,6) ist nicht leer, sondern enthält '┬'.\\ \tt \\ \tt Ihr habt gemeinsam 4 von maximal 12 Strafpunkten angesammelt.\\ \tt Bitte versuche es noch mal! 
	  } 
	   } 
	   } 
	 & & \\ 
 

    \theutterance \stepcounter{utterance}  

    & & \multicolumn{4}{p{0.6\linewidth}}{\cellcolor[rgb]{0.9,0.9,0.9}{%
	\makecell[{{p{\linewidth}}}]{% 
	  \tt {\tiny [A$\langle$GM]}  
	 Strafpunkt: (5,6) ist nicht leer, sondern enthält '┬'.\\ \tt \\ \tt Ihr habt gemeinsam 4 von maximal 12 Strafpunkten angesammelt.\\ \tt Bitte versuche es noch mal! 
	  } 
	   } 
	   } 
	 & \\ 
 

    \theutterance \stepcounter{utterance}  

    & \multicolumn{4}{p{0.6\linewidth}}{\cellcolor[rgb]{0.8,1,0.9}{%
	\makecell[{{p{\linewidth}}}]{% 
	  \tt {\tiny [A$\rangle$GM]}  
	 SPRICH: Danke, verstanden. Bei mir sind in Zeile 6 nur 1–3 frei; 4=└, 5=┬, 6=┴, 7=─ (blockiert). Vorschlag: Ziel in Zeile 6 nur für W,I,T: W@1,6; I@2,6; T@3,6. Für C und H weichen wir auf Zeile 7 aus: H@5,7; C@6,7 (beide frei bei mir). Ich würde als Nächstes C nach (6,7) ziehen, danach H nach (5,7). Ist das bei dir frei und okay? 
	  } 
	   } 
	   } 
	 & & \\ 
 

    \theutterance \stepcounter{utterance}  

    & & & \multicolumn{2}{p{0.3\linewidth}}{\cellcolor[rgb]{0.95,0.95,0.95}{%
	\makecell[{{p{\linewidth}}}]{% 
	  \tt {\tiny [GM$|$GM]}  
	 SPRICH: Danke, verstanden. Bei mir sind in Zeile 6 nur 1–3 frei; 4=└, 5=┬, 6=┴, 7=─ (blockiert). Vorschlag: Ziel in Zeile 6 nur für W,I,T: W@1,6; I@2,6; T@3,6. Für C und H weichen wir auf Zeile 7 aus: H@5,7; C@6,7 (beide frei bei mir). Ich würde als Nächstes C nach (6,7) ziehen, danach H nach (5,7). Ist das bei dir frei und okay? 
	  } 
	   } 
	   } 
	 & & \\ 
 

    \theutterance \stepcounter{utterance}  

    & & \multicolumn{4}{p{0.6\linewidth}}{\cellcolor[rgb]{0.9,0.9,0.9}{%
	\makecell[{{p{\linewidth}}}]{% 
	  \tt {\tiny [GM$\rangle$B]}  
	 Deine Nachricht wurde an deinen Mitspieler übermittelt.\\ \tt \\ \tt \\ \tt Ihr seid gerade bei Runde 7 von maximal 20.\\ \tt Ihr habt gemeinsam 4 von maximal 12 Strafpunkten angesammelt.\\ \tt Dein Mitspieler hat dir diese Nachricht geschickt:\\ \tt \\ \tt \\ \tt "Danke, verstanden. Bei mir sind in Zeile 6 nur 1–3 frei; 4=└, 5=┬, 6=┴, 7=─ (blockiert). Vorschlag: Ziel in Zeile 6 nur für W,I,T: W@1,6; I@2,6; T@3,6. Für C und H weichen wir auf Zeile 7 aus: H@5,7; C@6,7 (beide frei bei mir). Ich würde als Nächstes C nach (6,7) ziehen, danach H nach (5,7). Ist das bei dir frei und okay?"\\ \tt \\ \tt \\ \tt Wie lautet dein nächster Befehl? 
	  } 
	   } 
	   } 
	 & \\ 
 

    \theutterance \stepcounter{utterance}  

    & & & \multicolumn{4}{p{0.6\linewidth}}{\cellcolor[rgb]{1,0.85,0.72}{%
	\makecell[{{p{\linewidth}}}]{% 
	  \tt {\tiny [GM$\langle$B]}  
	 SPRICH: Einverstanden. Neues Ziel: W@1,6; I@2,6; T@3,6; H@5,7; C@6,7. Beide Ziele (5,7) und (6,7) sind bei mir frei. Ich ziehe jetzt I nach (2,6), danach H nach (5,7) und C nach (6,7). Passt? 
	  } 
	   } 
	   } 
	  \\ 
 

    \theutterance \stepcounter{utterance}  

    & & & \multicolumn{2}{p{0.3\linewidth}}{\cellcolor[rgb]{0.95,0.95,0.95}{%
	\makecell[{{p{\linewidth}}}]{% 
	  \tt {\tiny [GM$|$GM]}  
	 SPRICH: Einverstanden. Neues Ziel: W@1,6; I@2,6; T@3,6; H@5,7; C@6,7. Beide Ziele (5,7) und (6,7) sind bei mir frei. Ich ziehe jetzt I nach (2,6), danach H nach (5,7) und C nach (6,7). Passt? 
	  } 
	   } 
	   } 
	 & & \\ 
 

    \theutterance \stepcounter{utterance}  

    & & \multicolumn{4}{p{0.6\linewidth}}{\cellcolor[rgb]{0.9,0.9,0.9}{%
	\makecell[{{p{\linewidth}}}]{% 
	  \tt {\tiny [A$\langle$GM]}  
	 Deine Nachricht wurde an deinen Mitspieler übermittelt.\\ \tt \\ \tt \\ \tt Ihr seid gerade bei Runde 7 von maximal 20.\\ \tt Ihr habt gemeinsam 4 von maximal 12 Strafpunkten angesammelt.\\ \tt Dein Mitspieler hat dir diese Nachricht geschickt:\\ \tt \\ \tt \\ \tt "Einverstanden. Neues Ziel: W@1,6; I@2,6; T@3,6; H@5,7; C@6,7. Beide Ziele (5,7) und (6,7) sind bei mir frei. Ich ziehe jetzt I nach (2,6), danach H nach (5,7) und C nach (6,7). Passt?"\\ \tt \\ \tt \\ \tt Wie lautet dein nächster Befehl? 
	  } 
	   } 
	   } 
	 & \\ 
 

    \theutterance \stepcounter{utterance}  

    & \multicolumn{4}{p{0.6\linewidth}}{\cellcolor[rgb]{0.8,1,0.9}{%
	\makecell[{{p{\linewidth}}}]{% 
	  \tt {\tiny [A$\rangle$GM]}  
	 VERSCHIEBE: C, (6, 7) 
	  } 
	   } 
	   } 
	 & & \\ 
 

    \theutterance \stepcounter{utterance}  

    & & & \multicolumn{2}{p{0.3\linewidth}}{\cellcolor[rgb]{0.95,0.95,0.95}{%
	\makecell[{{p{\linewidth}}}]{% 
	  \tt {\tiny [GM$|$GM]}  
	 VERSCHIEBE: C, (6, 7) 
	  } 
	   } 
	   } 
	 & & \\ 
 

    \theutterance \stepcounter{utterance}  

    & & & \multicolumn{2}{p{0.3\linewidth}}{\cellcolor[rgb]{0.95,0.95,0.95}{%
	\makecell[{{p{\linewidth}}}]{% 
	  \tt {\tiny [GM$|$GM]}  
	 'C' erfolgreich nach (6,7) verschoben. Dein Raster sieht jetzt so aus:\\ \tt ```\\ \tt  1234567\\ \tt ╔═══╤═╤═╗\\ \tt ║◌H◌│W│◌║ 1\\ \tt ╟─┬─┼─┴─╢ 2\\ \tt ║◌│◌│◌◌◌║ 3\\ \tt ║◌├─┤◌┌─╢ 4\\ \tt ╟─┘◌│◌│◌║ 5\\ \tt ║◌◌T└┬┴─╢ 6\\ \tt ║◌◌◌◌│CI║ 7\\ \tt ╚════╧══╝\\ \tt ```\\ \tt  
	  } 
	   } 
	   } 
	 & & \\ 
 

    \theutterance \stepcounter{utterance}  

    & & \multicolumn{4}{p{0.6\linewidth}}{\cellcolor[rgb]{0.9,0.9,0.9}{%
	\makecell[{{p{\linewidth}}}]{% 
	  \tt {\tiny [GM$\rangle$B]}  
	 Deine Nachricht wurde an deinen Mitspieler übermittelt.\\ \tt \\ \tt \\ \tt Ihr seid gerade bei Runde 8 von maximal 20.\\ \tt Ihr habt gemeinsam 4 von maximal 12 Strafpunkten angesammelt.\\ \tt Dein Mitspieler hat ein Objekt auf seinem Raster bewegt.\\ \tt \\ \tt \\ \tt Wie lautet dein nächster Befehl? 
	  } 
	   } 
	   } 
	 & \\ 
 

    \theutterance \stepcounter{utterance}  

    & & & \multicolumn{4}{p{0.6\linewidth}}{\cellcolor[rgb]{1,0.85,0.72}{%
	\makecell[{{p{\linewidth}}}]{% 
	  \tt {\tiny [GM$\langle$B]}  
	 VERSCHIEBE: I, (2, 6) 
	  } 
	   } 
	   } 
	  \\ 
 

    \theutterance \stepcounter{utterance}  

    & & & \multicolumn{2}{p{0.3\linewidth}}{\cellcolor[rgb]{0.95,0.95,0.95}{%
	\makecell[{{p{\linewidth}}}]{% 
	  \tt {\tiny [GM$|$GM]}  
	 VERSCHIEBE: I, (2, 6) 
	  } 
	   } 
	   } 
	 & & \\ 
 

    \theutterance \stepcounter{utterance}  

    & & & \multicolumn{2}{p{0.3\linewidth}}{\cellcolor[rgb]{0.95,0.95,0.95}{%
	\makecell[{{p{\linewidth}}}]{% 
	  \tt {\tiny [GM$|$GM]}  
	 'I' erfolgreich nach (2,6) verschoben. Dein Raster sieht jetzt so aus:\\ \tt ```\\ \tt  1234567\\ \tt ╔═══╤═╤═╗\\ \tt ║◌◌◌│◌│◌║ 1\\ \tt ╟─┬─┼─┴─╢ 2\\ \tt ║◌│C│◌◌H║ 3\\ \tt ║◌├─┤◌┌─╢ 4\\ \tt ╟─┘◌│◌│◌║ 5\\ \tt ║WIT└┬┴─╢ 6\\ \tt ║◌◌◌◌│◌◌║ 7\\ \tt ╚════╧══╝\\ \tt ```\\ \tt  
	  } 
	   } 
	   } 
	 & & \\ 
 

    \theutterance \stepcounter{utterance}  

    & & \multicolumn{4}{p{0.6\linewidth}}{\cellcolor[rgb]{0.9,0.9,0.9}{%
	\makecell[{{p{\linewidth}}}]{% 
	  \tt {\tiny [A$\langle$GM]}  
	 'C' erfolgreich nach (6,7) verschoben. Dein Raster sieht jetzt so aus:\\ \tt ```\\ \tt  1234567\\ \tt ╔═══╤═╤═╗\\ \tt ║◌H◌│W│◌║ 1\\ \tt ╟─┬─┼─┴─╢ 2\\ \tt ║◌│◌│◌◌◌║ 3\\ \tt ║◌├─┤◌┌─╢ 4\\ \tt ╟─┘◌│◌│◌║ 5\\ \tt ║◌◌T└┬┴─╢ 6\\ \tt ║◌◌◌◌│CI║ 7\\ \tt ╚════╧══╝\\ \tt ```\\ \tt Ihr seid gerade bei Runde 8 von maximal 20.\\ \tt Ihr habt gemeinsam 4 von maximal 12 Strafpunkten angesammelt.\\ \tt Dein Mitspieler hat ein Objekt auf seinem Raster bewegt.\\ \tt \\ \tt \\ \tt Wie lautet dein nächster Befehl? 
	  } 
	   } 
	   } 
	 & \\ 
 

    \theutterance \stepcounter{utterance}  

    & \multicolumn{4}{p{0.6\linewidth}}{\cellcolor[rgb]{0.8,1,0.9}{%
	\makecell[{{p{\linewidth}}}]{% 
	  \tt {\tiny [A$\rangle$GM]}  
	 VERSCHIEBE: H, (5, 7) 
	  } 
	   } 
	   } 
	 & & \\ 
 

    \theutterance \stepcounter{utterance}  

    & & & \multicolumn{2}{p{0.3\linewidth}}{\cellcolor[rgb]{0.95,0.95,0.95}{%
	\makecell[{{p{\linewidth}}}]{% 
	  \tt {\tiny [GM$|$GM]}  
	 VERSCHIEBE: H, (5, 7) 
	  } 
	   } 
	   } 
	 & & \\ 
 

    \theutterance \stepcounter{utterance}  

    & & & \multicolumn{2}{p{0.3\linewidth}}{\cellcolor[rgb]{0.95,0.95,0.95}{%
	\makecell[{{p{\linewidth}}}]{% 
	  \tt {\tiny [GM$|$GM]}  
	 Strafpunkt: (5,7) ist nicht leer, sondern enthält '│'.\\ \tt \\ \tt Ihr habt gemeinsam 5 von maximal 12 Strafpunkten angesammelt.\\ \tt Bitte versuche es noch mal! 
	  } 
	   } 
	   } 
	 & & \\ 
 

    \theutterance \stepcounter{utterance}  

    & & \multicolumn{4}{p{0.6\linewidth}}{\cellcolor[rgb]{0.9,0.9,0.9}{%
	\makecell[{{p{\linewidth}}}]{% 
	  \tt {\tiny [A$\langle$GM]}  
	 Strafpunkt: (5,7) ist nicht leer, sondern enthält '│'.\\ \tt \\ \tt Ihr habt gemeinsam 5 von maximal 12 Strafpunkten angesammelt.\\ \tt Bitte versuche es noch mal! 
	  } 
	   } 
	   } 
	 & \\ 
 

    \theutterance \stepcounter{utterance}  

    & \multicolumn{4}{p{0.6\linewidth}}{\cellcolor[rgb]{0.8,1,0.9}{%
	\makecell[{{p{\linewidth}}}]{% 
	  \tt {\tiny [A$\rangle$GM]}  
	 SPRICH: Bei mir ist (5,7) durch eine vertikale Linie blockiert. Freie Felder in Zeile 7 sind: 1–4 und 7; 5 ist │, 6 ist von C belegt. Vorschlag für finales gemeinsames Ziel:\\ \tt - W@1,6; I@2,6; T@3,6 (bleibt)\\ \tt - H@4,7; C@6,7 (bereits)\\ \tt Ist H@4,7 bei dir frei? Wenn ja, ziehe ich H als Nächstes nach (4,7). 
	  } 
	   } 
	   } 
	 & & \\ 
 

    \theutterance \stepcounter{utterance}  

    & & & \multicolumn{2}{p{0.3\linewidth}}{\cellcolor[rgb]{0.95,0.95,0.95}{%
	\makecell[{{p{\linewidth}}}]{% 
	  \tt {\tiny [GM$|$GM]}  
	 SPRICH: Bei mir ist (5,7) durch eine vertikale Linie blockiert. Freie Felder in Zeile 7 sind: 1–4 und 7; 5 ist │, 6 ist von C belegt. Vorschlag für finales gemeinsames Ziel:\\ \tt - W@1,6; I@2,6; T@3,6 (bleibt)\\ \tt - H@4,7; C@6,7 (bereits)\\ \tt Ist H@4,7 bei dir frei? Wenn ja, ziehe ich H als Nächstes nach (4,7). 
	  } 
	   } 
	   } 
	 & & \\ 
 

    \theutterance \stepcounter{utterance}  

    & & & \multicolumn{2}{p{0.3\linewidth}}{\cellcolor[rgb]{0.95,0.95,0.95}{%
	\makecell[{{p{\linewidth}}}]{% 
	  \tt {\tiny [GM$|$GM]}  
	 Invalid format: tail is not empty: '\\ \tt - W@1,6; I@2,6; T@3,6 (bleibt)\\ \tt - H@4,7; C@6,7 (bereits)\\ \tt Ist H@4,7 bei dir frei? Wenn ja, ziehe ich H als Nächstes nach (4,7). 
	  } 
	   } 
	   } 
	 & & \\ 
 

    \theutterance \stepcounter{utterance}  

    & & \multicolumn{4}{p{0.6\linewidth}}{\cellcolor[rgb]{0.9,0.9,0.9}{%
	\makecell[{{p{\linewidth}}}]{% 
	  \tt {\tiny [A$\langle$GM]}  
	 Strafpunkt: Deine Nachricht darf nichts nach dem Befehl enthalten! Stelle sicher, dass deine Antwort ausschließlich einen der beiden Befehle `SPRICH: <NACHRICHT>` und `VERSCHIEBE: <OBJECT>, (<X>, <Y>)` und sonst nichts enthält!\\ \tt \\ \tt Ihr habt gemeinsam 6 von maximal 12 Strafpunkten angesammelt.\\ \tt Bitte versuche es noch mal! 
	  } 
	   } 
	   } 
	 & \\ 
 

    \theutterance \stepcounter{utterance}  

    & \multicolumn{4}{p{0.6\linewidth}}{\cellcolor[rgb]{0.8,1,0.9}{%
	\makecell[{{p{\linewidth}}}]{% 
	  \tt {\tiny [A$\rangle$GM]}  
	 SPRICH: Bei mir ist (5,7) blockiert. Freie Felder in Zeile 7: 1–4 und 7; C steht auf (6,7). Vorschlag neues Ziel: W@1,6; I@2,6; T@3,6; H@4,7; C@6,7. Ist H@4,7 bei dir frei? Ich würde dann H nach (4,7) ziehen. 
	  } 
	   } 
	   } 
	 & & \\ 
 

    \theutterance \stepcounter{utterance}  

    & & & \multicolumn{2}{p{0.3\linewidth}}{\cellcolor[rgb]{0.95,0.95,0.95}{%
	\makecell[{{p{\linewidth}}}]{% 
	  \tt {\tiny [GM$|$GM]}  
	 SPRICH: Bei mir ist (5,7) blockiert. Freie Felder in Zeile 7: 1–4 und 7; C steht auf (6,7). Vorschlag neues Ziel: W@1,6; I@2,6; T@3,6; H@4,7; C@6,7. Ist H@4,7 bei dir frei? Ich würde dann H nach (4,7) ziehen. 
	  } 
	   } 
	   } 
	 & & \\ 
 

    \theutterance \stepcounter{utterance}  

    & & \multicolumn{4}{p{0.6\linewidth}}{\cellcolor[rgb]{0.9,0.9,0.9}{%
	\makecell[{{p{\linewidth}}}]{% 
	  \tt {\tiny [GM$\rangle$B]}  
	 'I' erfolgreich nach (2,6) verschoben. Dein Raster sieht jetzt so aus:\\ \tt ```\\ \tt  1234567\\ \tt ╔═══╤═╤═╗\\ \tt ║◌◌◌│◌│◌║ 1\\ \tt ╟─┬─┼─┴─╢ 2\\ \tt ║◌│C│◌◌H║ 3\\ \tt ║◌├─┤◌┌─╢ 4\\ \tt ╟─┘◌│◌│◌║ 5\\ \tt ║WIT└┬┴─╢ 6\\ \tt ║◌◌◌◌│◌◌║ 7\\ \tt ╚════╧══╝\\ \tt ```\\ \tt Ihr seid gerade bei Runde 9 von maximal 20.\\ \tt Ihr habt gemeinsam 6 von maximal 12 Strafpunkten angesammelt.\\ \tt Dein Mitspieler hat dir diese Nachricht geschickt:\\ \tt \\ \tt \\ \tt "Bei mir ist (5,7) blockiert. Freie Felder in Zeile 7: 1–4 und 7; C steht auf (6,7). Vorschlag neues Ziel: W@1,6; I@2,6; T@3,6; H@4,7; C@6,7. Ist H@4,7 bei dir frei? Ich würde dann H nach (4,7) ziehen."\\ \tt \\ \tt \\ \tt Wie lautet dein nächster Befehl? 
	  } 
	   } 
	   } 
	 & \\ 
 

    \theutterance \stepcounter{utterance}  

    & & & \multicolumn{4}{p{0.6\linewidth}}{\cellcolor[rgb]{1,0.85,0.72}{%
	\makecell[{{p{\linewidth}}}]{% 
	  \tt {\tiny [GM$\langle$B]}  
	 SPRICH: H@4,7 ist bei mir frei. Einverstanden mit Ziel: W@1,6; I@2,6; T@3,6; H@4,7; C@6,7. Ich ziehe als Nächstes H nach (4,7), danach C nach (6,7). Passt? 
	  } 
	   } 
	   } 
	  \\ 
 

    \theutterance \stepcounter{utterance}  

    & & & \multicolumn{2}{p{0.3\linewidth}}{\cellcolor[rgb]{0.95,0.95,0.95}{%
	\makecell[{{p{\linewidth}}}]{% 
	  \tt {\tiny [GM$|$GM]}  
	 SPRICH: H@4,7 ist bei mir frei. Einverstanden mit Ziel: W@1,6; I@2,6; T@3,6; H@4,7; C@6,7. Ich ziehe als Nächstes H nach (4,7), danach C nach (6,7). Passt? 
	  } 
	   } 
	   } 
	 & & \\ 
 

    \theutterance \stepcounter{utterance}  

    & & \multicolumn{4}{p{0.6\linewidth}}{\cellcolor[rgb]{0.9,0.9,0.9}{%
	\makecell[{{p{\linewidth}}}]{% 
	  \tt {\tiny [A$\langle$GM]}  
	 Deine Nachricht wurde an deinen Mitspieler übermittelt.\\ \tt \\ \tt \\ \tt Ihr seid gerade bei Runde 9 von maximal 20.\\ \tt Ihr habt gemeinsam 6 von maximal 12 Strafpunkten angesammelt.\\ \tt Dein Mitspieler hat dir diese Nachricht geschickt:\\ \tt \\ \tt \\ \tt "H@4,7 ist bei mir frei. Einverstanden mit Ziel: W@1,6; I@2,6; T@3,6; H@4,7; C@6,7. Ich ziehe als Nächstes H nach (4,7), danach C nach (6,7). Passt?"\\ \tt \\ \tt \\ \tt Wie lautet dein nächster Befehl? 
	  } 
	   } 
	   } 
	 & \\ 
 

    \theutterance \stepcounter{utterance}  

    & \multicolumn{4}{p{0.6\linewidth}}{\cellcolor[rgb]{0.8,1,0.9}{%
	\makecell[{{p{\linewidth}}}]{% 
	  \tt {\tiny [A$\rangle$GM]}  
	 VERSCHIEBE: H, (4, 7) 
	  } 
	   } 
	   } 
	 & & \\ 
 

    \theutterance \stepcounter{utterance}  

    & & & \multicolumn{2}{p{0.3\linewidth}}{\cellcolor[rgb]{0.95,0.95,0.95}{%
	\makecell[{{p{\linewidth}}}]{% 
	  \tt {\tiny [GM$|$GM]}  
	 VERSCHIEBE: H, (4, 7) 
	  } 
	   } 
	   } 
	 & & \\ 
 

    \theutterance \stepcounter{utterance}  

    & & & \multicolumn{2}{p{0.3\linewidth}}{\cellcolor[rgb]{0.95,0.95,0.95}{%
	\makecell[{{p{\linewidth}}}]{% 
	  \tt {\tiny [GM$|$GM]}  
	 'H' erfolgreich nach (4,7) verschoben. Dein Raster sieht jetzt so aus:\\ \tt ```\\ \tt  1234567\\ \tt ╔═══╤═╤═╗\\ \tt ║◌◌◌│W│◌║ 1\\ \tt ╟─┬─┼─┴─╢ 2\\ \tt ║◌│◌│◌◌◌║ 3\\ \tt ║◌├─┤◌┌─╢ 4\\ \tt ╟─┘◌│◌│◌║ 5\\ \tt ║◌◌T└┬┴─╢ 6\\ \tt ║◌◌◌H│CI║ 7\\ \tt ╚════╧══╝\\ \tt ```\\ \tt  
	  } 
	   } 
	   } 
	 & & \\ 
 

    \theutterance \stepcounter{utterance}  

    & & \multicolumn{4}{p{0.6\linewidth}}{\cellcolor[rgb]{0.9,0.9,0.9}{%
	\makecell[{{p{\linewidth}}}]{% 
	  \tt {\tiny [GM$\rangle$B]}  
	 Deine Nachricht wurde an deinen Mitspieler übermittelt.\\ \tt \\ \tt \\ \tt Ihr seid gerade bei Runde 10 von maximal 20.\\ \tt Ihr habt gemeinsam 6 von maximal 12 Strafpunkten angesammelt.\\ \tt Dein Mitspieler hat ein Objekt auf seinem Raster bewegt.\\ \tt \\ \tt \\ \tt Wie lautet dein nächster Befehl? 
	  } 
	   } 
	   } 
	 & \\ 
 

    \theutterance \stepcounter{utterance}  

    & & & \multicolumn{4}{p{0.6\linewidth}}{\cellcolor[rgb]{1,0.85,0.72}{%
	\makecell[{{p{\linewidth}}}]{% 
	  \tt {\tiny [GM$\langle$B]}  
	 VERSCHIEBE: H, (4, 7) 
	  } 
	   } 
	   } 
	  \\ 
 

    \theutterance \stepcounter{utterance}  

    & & & \multicolumn{2}{p{0.3\linewidth}}{\cellcolor[rgb]{0.95,0.95,0.95}{%
	\makecell[{{p{\linewidth}}}]{% 
	  \tt {\tiny [GM$|$GM]}  
	 VERSCHIEBE: H, (4, 7) 
	  } 
	   } 
	   } 
	 & & \\ 
 

    \theutterance \stepcounter{utterance}  

    & & & \multicolumn{2}{p{0.3\linewidth}}{\cellcolor[rgb]{0.95,0.95,0.95}{%
	\makecell[{{p{\linewidth}}}]{% 
	  \tt {\tiny [GM$|$GM]}  
	 'H' erfolgreich nach (4,7) verschoben. Dein Raster sieht jetzt so aus:\\ \tt ```\\ \tt  1234567\\ \tt ╔═══╤═╤═╗\\ \tt ║◌◌◌│◌│◌║ 1\\ \tt ╟─┬─┼─┴─╢ 2\\ \tt ║◌│C│◌◌◌║ 3\\ \tt ║◌├─┤◌┌─╢ 4\\ \tt ╟─┘◌│◌│◌║ 5\\ \tt ║WIT└┬┴─╢ 6\\ \tt ║◌◌◌H│◌◌║ 7\\ \tt ╚════╧══╝\\ \tt ```\\ \tt  
	  } 
	   } 
	   } 
	 & & \\ 
 

    \theutterance \stepcounter{utterance}  

    & & \multicolumn{4}{p{0.6\linewidth}}{\cellcolor[rgb]{0.9,0.9,0.9}{%
	\makecell[{{p{\linewidth}}}]{% 
	  \tt {\tiny [A$\langle$GM]}  
	 'H' erfolgreich nach (4,7) verschoben. Dein Raster sieht jetzt so aus:\\ \tt ```\\ \tt  1234567\\ \tt ╔═══╤═╤═╗\\ \tt ║◌◌◌│W│◌║ 1\\ \tt ╟─┬─┼─┴─╢ 2\\ \tt ║◌│◌│◌◌◌║ 3\\ \tt ║◌├─┤◌┌─╢ 4\\ \tt ╟─┘◌│◌│◌║ 5\\ \tt ║◌◌T└┬┴─╢ 6\\ \tt ║◌◌◌H│CI║ 7\\ \tt ╚════╧══╝\\ \tt ```\\ \tt Ihr seid gerade bei Runde 10 von maximal 20.\\ \tt Ihr habt gemeinsam 6 von maximal 12 Strafpunkten angesammelt.\\ \tt Dein Mitspieler hat ein Objekt auf seinem Raster bewegt.\\ \tt \\ \tt \\ \tt Wie lautet dein nächster Befehl? 
	  } 
	   } 
	   } 
	 & \\ 
 

    \theutterance \stepcounter{utterance}  

    & \multicolumn{4}{p{0.6\linewidth}}{\cellcolor[rgb]{0.8,1,0.9}{%
	\makecell[{{p{\linewidth}}}]{% 
	  \tt {\tiny [A$\rangle$GM]}  
	 SPRICH: Super. Mein aktueller Stand: W@4,1; I@7,7; T@3,6; H@4,7; C@6,7. Ziel bleibt: W@1,6; I@2,6; T@3,6; H@4,7; C@6,7. Ich ziehe als Nächstes W nach (1,6), danach I nach (2,6). Ist das bei dir okay? 
	  } 
	   } 
	   } 
	 & & \\ 
 

    \theutterance \stepcounter{utterance}  

    & & & \multicolumn{2}{p{0.3\linewidth}}{\cellcolor[rgb]{0.95,0.95,0.95}{%
	\makecell[{{p{\linewidth}}}]{% 
	  \tt {\tiny [GM$|$GM]}  
	 SPRICH: Super. Mein aktueller Stand: W@4,1; I@7,7; T@3,6; H@4,7; C@6,7. Ziel bleibt: W@1,6; I@2,6; T@3,6; H@4,7; C@6,7. Ich ziehe als Nächstes W nach (1,6), danach I nach (2,6). Ist das bei dir okay? 
	  } 
	   } 
	   } 
	 & & \\ 
 

    \theutterance \stepcounter{utterance}  

    & & \multicolumn{4}{p{0.6\linewidth}}{\cellcolor[rgb]{0.9,0.9,0.9}{%
	\makecell[{{p{\linewidth}}}]{% 
	  \tt {\tiny [GM$\rangle$B]}  
	 'H' erfolgreich nach (4,7) verschoben. Dein Raster sieht jetzt so aus:\\ \tt ```\\ \tt  1234567\\ \tt ╔═══╤═╤═╗\\ \tt ║◌◌◌│◌│◌║ 1\\ \tt ╟─┬─┼─┴─╢ 2\\ \tt ║◌│C│◌◌◌║ 3\\ \tt ║◌├─┤◌┌─╢ 4\\ \tt ╟─┘◌│◌│◌║ 5\\ \tt ║WIT└┬┴─╢ 6\\ \tt ║◌◌◌H│◌◌║ 7\\ \tt ╚════╧══╝\\ \tt ```\\ \tt Ihr seid gerade bei Runde 11 von maximal 20.\\ \tt Ihr habt gemeinsam 6 von maximal 12 Strafpunkten angesammelt.\\ \tt Dein Mitspieler hat dir diese Nachricht geschickt:\\ \tt \\ \tt \\ \tt "Super. Mein aktueller Stand: W@4,1; I@7,7; T@3,6; H@4,7; C@6,7. Ziel bleibt: W@1,6; I@2,6; T@3,6; H@4,7; C@6,7. Ich ziehe als Nächstes W nach (1,6), danach I nach (2,6). Ist das bei dir okay?"\\ \tt \\ \tt \\ \tt Wie lautet dein nächster Befehl? 
	  } 
	   } 
	   } 
	 & \\ 
 

    \theutterance \stepcounter{utterance}  

    & & & \multicolumn{4}{p{0.6\linewidth}}{\cellcolor[rgb]{1,0.85,0.72}{%
	\makecell[{{p{\linewidth}}}]{% 
	  \tt {\tiny [GM$\langle$B]}  
	 SPRICH: Passt. Ich ziehe jetzt C nach (6,7). Danach melde ich „fertig?“, sobald du W@1,6 und I@2,6 erreicht hast. 
	  } 
	   } 
	   } 
	  \\ 
 

    \theutterance \stepcounter{utterance}  

    & & & \multicolumn{2}{p{0.3\linewidth}}{\cellcolor[rgb]{0.95,0.95,0.95}{%
	\makecell[{{p{\linewidth}}}]{% 
	  \tt {\tiny [GM$|$GM]}  
	 SPRICH: Passt. Ich ziehe jetzt C nach (6,7). Danach melde ich „fertig?“, sobald du W@1,6 und I@2,6 erreicht hast. 
	  } 
	   } 
	   } 
	 & & \\ 
 

    \theutterance \stepcounter{utterance}  

    & & \multicolumn{4}{p{0.6\linewidth}}{\cellcolor[rgb]{0.9,0.9,0.9}{%
	\makecell[{{p{\linewidth}}}]{% 
	  \tt {\tiny [A$\langle$GM]}  
	 Deine Nachricht wurde an deinen Mitspieler übermittelt.\\ \tt \\ \tt \\ \tt Ihr seid gerade bei Runde 11 von maximal 20.\\ \tt Ihr habt gemeinsam 6 von maximal 12 Strafpunkten angesammelt.\\ \tt Dein Mitspieler hat dir diese Nachricht geschickt:\\ \tt \\ \tt \\ \tt "Passt. Ich ziehe jetzt C nach (6,7). Danach melde ich „fertig?“, sobald du W@1,6 und I@2,6 erreicht hast."\\ \tt \\ \tt \\ \tt Wie lautet dein nächster Befehl? 
	  } 
	   } 
	   } 
	 & \\ 
 

    \theutterance \stepcounter{utterance}  

    & \multicolumn{4}{p{0.6\linewidth}}{\cellcolor[rgb]{0.8,1,0.9}{%
	\makecell[{{p{\linewidth}}}]{% 
	  \tt {\tiny [A$\rangle$GM]}  
	 VERSCHIEBE: W, (1, 6) 
	  } 
	   } 
	   } 
	 & & \\ 
 

    \theutterance \stepcounter{utterance}  

    & & & \multicolumn{2}{p{0.3\linewidth}}{\cellcolor[rgb]{0.95,0.95,0.95}{%
	\makecell[{{p{\linewidth}}}]{% 
	  \tt {\tiny [GM$|$GM]}  
	 VERSCHIEBE: W, (1, 6) 
	  } 
	   } 
	   } 
	 & & \\ 
 

    \theutterance \stepcounter{utterance}  

    & & & \multicolumn{2}{p{0.3\linewidth}}{\cellcolor[rgb]{0.95,0.95,0.95}{%
	\makecell[{{p{\linewidth}}}]{% 
	  \tt {\tiny [GM$|$GM]}  
	 'W' erfolgreich nach (1,6) verschoben. Dein Raster sieht jetzt so aus:\\ \tt ```\\ \tt  1234567\\ \tt ╔═══╤═╤═╗\\ \tt ║◌◌◌│◌│◌║ 1\\ \tt ╟─┬─┼─┴─╢ 2\\ \tt ║◌│◌│◌◌◌║ 3\\ \tt ║◌├─┤◌┌─╢ 4\\ \tt ╟─┘◌│◌│◌║ 5\\ \tt ║W◌T└┬┴─╢ 6\\ \tt ║◌◌◌H│CI║ 7\\ \tt ╚════╧══╝\\ \tt ```\\ \tt  
	  } 
	   } 
	   } 
	 & & \\ 
 

    \theutterance \stepcounter{utterance}  

    & & \multicolumn{4}{p{0.6\linewidth}}{\cellcolor[rgb]{0.9,0.9,0.9}{%
	\makecell[{{p{\linewidth}}}]{% 
	  \tt {\tiny [GM$\rangle$B]}  
	 Deine Nachricht wurde an deinen Mitspieler übermittelt.\\ \tt \\ \tt \\ \tt Ihr seid gerade bei Runde 12 von maximal 20.\\ \tt Ihr habt gemeinsam 6 von maximal 12 Strafpunkten angesammelt.\\ \tt Dein Mitspieler hat ein Objekt auf seinem Raster bewegt.\\ \tt \\ \tt \\ \tt Wie lautet dein nächster Befehl? 
	  } 
	   } 
	   } 
	 & \\ 
 

    \theutterance \stepcounter{utterance}  

    & & & \multicolumn{4}{p{0.6\linewidth}}{\cellcolor[rgb]{1,0.85,0.72}{%
	\makecell[{{p{\linewidth}}}]{% 
	  \tt {\tiny [GM$\langle$B]}  
	 VERSCHIEBE: C, (6, 7) 
	  } 
	   } 
	   } 
	  \\ 
 

    \theutterance \stepcounter{utterance}  

    & & & \multicolumn{2}{p{0.3\linewidth}}{\cellcolor[rgb]{0.95,0.95,0.95}{%
	\makecell[{{p{\linewidth}}}]{% 
	  \tt {\tiny [GM$|$GM]}  
	 VERSCHIEBE: C, (6, 7) 
	  } 
	   } 
	   } 
	 & & \\ 
 

    \theutterance \stepcounter{utterance}  

    & & & \multicolumn{2}{p{0.3\linewidth}}{\cellcolor[rgb]{0.95,0.95,0.95}{%
	\makecell[{{p{\linewidth}}}]{% 
	  \tt {\tiny [GM$|$GM]}  
	 'C' erfolgreich nach (6,7) verschoben. Dein Raster sieht jetzt so aus:\\ \tt ```\\ \tt  1234567\\ \tt ╔═══╤═╤═╗\\ \tt ║◌◌◌│◌│◌║ 1\\ \tt ╟─┬─┼─┴─╢ 2\\ \tt ║◌│◌│◌◌◌║ 3\\ \tt ║◌├─┤◌┌─╢ 4\\ \tt ╟─┘◌│◌│◌║ 5\\ \tt ║WIT└┬┴─╢ 6\\ \tt ║◌◌◌H│C◌║ 7\\ \tt ╚════╧══╝\\ \tt ```\\ \tt  
	  } 
	   } 
	   } 
	 & & \\ 
 

    \theutterance \stepcounter{utterance}  

    & & \multicolumn{4}{p{0.6\linewidth}}{\cellcolor[rgb]{0.9,0.9,0.9}{%
	\makecell[{{p{\linewidth}}}]{% 
	  \tt {\tiny [A$\langle$GM]}  
	 'W' erfolgreich nach (1,6) verschoben. Dein Raster sieht jetzt so aus:\\ \tt ```\\ \tt  1234567\\ \tt ╔═══╤═╤═╗\\ \tt ║◌◌◌│◌│◌║ 1\\ \tt ╟─┬─┼─┴─╢ 2\\ \tt ║◌│◌│◌◌◌║ 3\\ \tt ║◌├─┤◌┌─╢ 4\\ \tt ╟─┘◌│◌│◌║ 5\\ \tt ║W◌T└┬┴─╢ 6\\ \tt ║◌◌◌H│CI║ 7\\ \tt ╚════╧══╝\\ \tt ```\\ \tt Ihr seid gerade bei Runde 12 von maximal 20.\\ \tt Ihr habt gemeinsam 6 von maximal 12 Strafpunkten angesammelt.\\ \tt Dein Mitspieler hat ein Objekt auf seinem Raster bewegt.\\ \tt \\ \tt \\ \tt Wie lautet dein nächster Befehl? 
	  } 
	   } 
	   } 
	 & \\ 
 

    \theutterance \stepcounter{utterance}  

    & \multicolumn{4}{p{0.6\linewidth}}{\cellcolor[rgb]{0.8,1,0.9}{%
	\makecell[{{p{\linewidth}}}]{% 
	  \tt {\tiny [A$\rangle$GM]}  
	 VERSCHIEBE: I, (2, 6) 
	  } 
	   } 
	   } 
	 & & \\ 
 

    \theutterance \stepcounter{utterance}  

    & & & \multicolumn{2}{p{0.3\linewidth}}{\cellcolor[rgb]{0.95,0.95,0.95}{%
	\makecell[{{p{\linewidth}}}]{% 
	  \tt {\tiny [GM$|$GM]}  
	 VERSCHIEBE: I, (2, 6) 
	  } 
	   } 
	   } 
	 & & \\ 
 

    \theutterance \stepcounter{utterance}  

    & & & \multicolumn{2}{p{0.3\linewidth}}{\cellcolor[rgb]{0.95,0.95,0.95}{%
	\makecell[{{p{\linewidth}}}]{% 
	  \tt {\tiny [GM$|$GM]}  
	 'I' erfolgreich nach (2,6) verschoben. Dein Raster sieht jetzt so aus:\\ \tt ```\\ \tt  1234567\\ \tt ╔═══╤═╤═╗\\ \tt ║◌◌◌│◌│◌║ 1\\ \tt ╟─┬─┼─┴─╢ 2\\ \tt ║◌│◌│◌◌◌║ 3\\ \tt ║◌├─┤◌┌─╢ 4\\ \tt ╟─┘◌│◌│◌║ 5\\ \tt ║WIT└┬┴─╢ 6\\ \tt ║◌◌◌H│C◌║ 7\\ \tt ╚════╧══╝\\ \tt ```\\ \tt  
	  } 
	   } 
	   } 
	 & & \\ 
 

    \theutterance \stepcounter{utterance}  

    & & \multicolumn{4}{p{0.6\linewidth}}{\cellcolor[rgb]{0.9,0.9,0.9}{%
	\makecell[{{p{\linewidth}}}]{% 
	  \tt {\tiny [GM$\rangle$B]}  
	 'C' erfolgreich nach (6,7) verschoben. Dein Raster sieht jetzt so aus:\\ \tt ```\\ \tt  1234567\\ \tt ╔═══╤═╤═╗\\ \tt ║◌◌◌│◌│◌║ 1\\ \tt ╟─┬─┼─┴─╢ 2\\ \tt ║◌│◌│◌◌◌║ 3\\ \tt ║◌├─┤◌┌─╢ 4\\ \tt ╟─┘◌│◌│◌║ 5\\ \tt ║WIT└┬┴─╢ 6\\ \tt ║◌◌◌H│C◌║ 7\\ \tt ╚════╧══╝\\ \tt ```\\ \tt Ihr seid gerade bei Runde 13 von maximal 20.\\ \tt Ihr habt gemeinsam 6 von maximal 12 Strafpunkten angesammelt.\\ \tt Dein Mitspieler hat ein Objekt auf seinem Raster bewegt.\\ \tt \\ \tt \\ \tt Wie lautet dein nächster Befehl? 
	  } 
	   } 
	   } 
	 & \\ 
 

    \theutterance \stepcounter{utterance}  

    & & & \multicolumn{4}{p{0.6\linewidth}}{\cellcolor[rgb]{1,0.85,0.72}{%
	\makecell[{{p{\linewidth}}}]{% 
	  \tt {\tiny [GM$\langle$B]}  
	 SPRICH: fertig? 
	  } 
	   } 
	   } 
	  \\ 
 

    \theutterance \stepcounter{utterance}  

    & & & \multicolumn{2}{p{0.3\linewidth}}{\cellcolor[rgb]{0.95,0.95,0.95}{%
	\makecell[{{p{\linewidth}}}]{% 
	  \tt {\tiny [GM$|$GM]}  
	 SPRICH: fertig? 
	  } 
	   } 
	   } 
	 & & \\ 
 

    \theutterance \stepcounter{utterance}  

    & & \multicolumn{4}{p{0.6\linewidth}}{\cellcolor[rgb]{0.9,0.9,0.9}{%
	\makecell[{{p{\linewidth}}}]{% 
	  \tt {\tiny [A$\langle$GM]}  
	 'I' erfolgreich nach (2,6) verschoben. Dein Raster sieht jetzt so aus:\\ \tt ```\\ \tt  1234567\\ \tt ╔═══╤═╤═╗\\ \tt ║◌◌◌│◌│◌║ 1\\ \tt ╟─┬─┼─┴─╢ 2\\ \tt ║◌│◌│◌◌◌║ 3\\ \tt ║◌├─┤◌┌─╢ 4\\ \tt ╟─┘◌│◌│◌║ 5\\ \tt ║WIT└┬┴─╢ 6\\ \tt ║◌◌◌H│C◌║ 7\\ \tt ╚════╧══╝\\ \tt ```\\ \tt Ihr seid gerade bei Runde 13 von maximal 20.\\ \tt Ihr habt gemeinsam 6 von maximal 12 Strafpunkten angesammelt.\\ \tt Dein Mitspieler hat dir diese Nachricht geschickt:\\ \tt \\ \tt \\ \tt "fertig?"\\ \tt \\ \tt \\ \tt Wie lautet dein nächster Befehl? 
	  } 
	   } 
	   } 
	 & \\ 
 

    \theutterance \stepcounter{utterance}  

    & \multicolumn{4}{p{0.6\linewidth}}{\cellcolor[rgb]{0.8,1,0.9}{%
	\makecell[{{p{\linewidth}}}]{% 
	  \tt {\tiny [A$\rangle$GM]}  
	 SPRICH: fertig! 
	  } 
	   } 
	   } 
	 & & \\ 
 

    \theutterance \stepcounter{utterance}  

    & & & \multicolumn{2}{p{0.3\linewidth}}{\cellcolor[rgb]{0.95,0.95,0.95}{%
	\makecell[{{p{\linewidth}}}]{% 
	  \tt {\tiny [GM$|$GM]}  
	 SPRICH: fertig! 
	  } 
	   } 
	   } 
	 & & \\ 
 

    \theutterance \stepcounter{utterance}  

    & & & \multicolumn{2}{p{0.3\linewidth}}{\cellcolor[rgb]{0.95,0.95,0.95}{%
	\makecell[{{p{\linewidth}}}]{% 
	  \tt {\tiny [GM$|$GM]}  
	 true 
	  } 
	   } 
	   } 
	 & & \\ 
 

    \theutterance \stepcounter{utterance}  

    & & & \multicolumn{2}{p{0.3\linewidth}}{\cellcolor[rgb]{0.95,0.95,0.95}{%
	\makecell[{{p{\linewidth}}}]{% 
	  \tt {\tiny [GM$|$GM]}  
	 * success: True\\ \tt * lose: False\\ \tt * aborted: False\\ \tt -------\\ \tt * Shifts: 7.00\\ \tt * Max Shifts: 8.00\\ \tt * Min Shifts: 4.00\\ \tt * End Distance Sum: 0.00\\ \tt * Init Distance Sum: 19.54\\ \tt * Expected Distance Sum: 20.95\\ \tt * Penalties: 6.00\\ \tt * Max Penalties: 12.00\\ \tt * Rounds: 13.00\\ \tt * Max Rounds: 20.00\\ \tt * Object Count: 5.00\\ \tt  
	  } 
	   } 
	   } 
	 & & \\ 
 

    \theutterance \stepcounter{utterance}  

    & & & \multicolumn{2}{p{0.3\linewidth}}{\cellcolor[rgb]{0.95,0.95,0.95}{%
	\makecell[{{p{\linewidth}}}]{% 
	  \tt {\tiny [GM$|$GM]}  
	 * Main Score: 40.62\\ \tt \\ \tt -------\\ \tt * Distance Score: 1.00\\ \tt * Consistency Score: 0.40\\ \tt * Coverage Score: 0.83\\ \tt * Penalty Score: 0.64\\ \tt * Alternative Penalty Score: 0.78\\ \tt * Alternative Main Score: 77.78\\ \tt \\ \tt -------\\ \tt * Shifts: 7.00\\ \tt * Max Shifts: 8.00\\ \tt * Min Shifts: 4.00\\ \tt * End Distance Sum: 0.00\\ \tt * Init Distance Sum: 19.54\\ \tt * Expected Distance Sum: 20.95\\ \tt * Penalties: 6.00\\ \tt * Max Penalties: 12.00\\ \tt * Rounds: 13.00\\ \tt * Max Rounds: 20.00\\ \tt * Object Count: 5.00\\ \tt  
	  } 
	   } 
	   } 
	 & & \\ 
 

\end{supertabular}
}

\end{document}
