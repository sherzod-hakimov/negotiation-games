
\documentclass{article}
\usepackage{colortbl}
\usepackage{makecell}
\usepackage{multirow}
\usepackage{supertabular}

\begin{document}

\newcounter{utterance}

\twocolumn

{ \footnotesize  \setcounter{utterance}{1}
\setlength{\tabcolsep}{0pt}
\begin{supertabular}{c@{$\;$}|p{.15\linewidth}@{}p{.15\linewidth}p{.15\linewidth}p{.15\linewidth}p{.15\linewidth}p{.15\linewidth}}

    \# & $\;$A & \multicolumn{4}{c}{Game Master} & $\;\:$B\\
    \hline 

    \theutterance \stepcounter{utterance}  

    & & \multicolumn{4}{p{0.6\linewidth}}{\cellcolor[rgb]{0.9,0.9,0.9}{%
	\makecell[{{p{\linewidth}}}]{% 
	  \tt {\tiny [A$\langle$GM]}  
	 Ich bin der Spielleiter, und du spielst ein kooperatives Spiel mit dem folgenden Raster als Spielbrett:\\ \tt \\ \tt ```\\ \tt  1234567\\ \tt ╔═══════╗\\ \tt ║◌◌P◌◌◌◌║ 1\\ \tt ║◌◌◌◌◌◌◌║ 2\\ \tt ║◌◌◌◌D┌─╢ 3\\ \tt ║◌TM┌─┤◌║ 4\\ \tt ║◌◌O│◌├─╢ 5\\ \tt ╟───┼─┘◌║ 6\\ \tt ║◌◌◌│◌AS║ 7\\ \tt ╚═══╧═══╝\\ \tt \\ \tt ```\\ \tt \\ \tt * Am oberen Rand sind die x-Koordinaten angegeben, die nach rechts hin zunehmen, und am rechten Rand die y-Koordinaten, die nach unten zunehmen.\\ \tt * Die folgenden Objekte sind zufällig auf deinem Raster platziert: 'P', 'O', 'T', 'S', 'D', 'A', 'M'.\\ \tt \\ \tt \\ \tt Dein Mitspieler sieht eine Variation des Spielbretts, auf dem dieselben Objekte an anderen zufälligen Positionen platziert sind. Du kannst das Spielbrett des anderen Spielers nicht sehen, und er kann deins nicht sehen.\\ \tt \\ \tt \\ \tt **Ziel des Spiels:**\\ \tt \\ \tt Beide Spieler müssen die Objekte auf ihrem jeweiligen Spielbrett so bewegen, dass identische Objekte am Ende an denselben Koordinaten liegen. Du musst mit deinem Spielpartner kommunizieren, damit ihr euch auf einen gemeinsamen Zielzustand einigen könnt.\\ \tt \\ \tt \\ \tt **Spielregeln:**\\ \tt \\ \tt * In jedem Spielzug kannst du genau eine der folgenden zwei Aktionen ausführen:\\ \tt 1. `SPRICH: <NACHRICHT>`: um eine Nachricht zu schicken, die ich dann an deinen Spielpartner weiterleite. Als Nachricht gilt alles bis zum nächsten Zeilenumbruch.\\ \tt 2. `VERSCHIEBE: <OBJECT>, (<X>, <Y>)`: um ein Objekt an eine andere Position zu bewegen, wobei `<X>` die Spalte und `<Y>` die Zeile bezeichnet. Ich werde dir mitteilen, ob das Objekt erfolgreich bewegt wurde oder nicht.\\ \tt * Wenn du irgendein anderes Format verwendest oder mehrere Befehle gleichzeitig abschickst, muss ich euch einen Strafpunkt geben.\\ \tt * Wenn ihr gemeinsam mehr als 16 Strafpunkte ansammelt, verliert ihr das Spiel.\\ \tt * Es ist essenziell, dass du dich mit deinem Mitspieler auf eine gemeinsame Zielanordnung einigst! Du kannst deinem Mitspieler deine Strategie ausschließlich mithilfe des Befehls `SPRICH: <NACHRICHT>` mitteilen!\\ \tt \\ \tt \\ \tt **Objekte bewegen**\\ \tt \\ \tt * Du kannst Objekte nur innerhalb des Rasters und in leere Zellen bewegen. Leere Zellen erkennst am Zeichen '◌'.\\ \tt * Solltest du versuchen, ein Objekt an eine Stelle zu bewegen, die entweder nicht leer ist oder sich außerhalb des Rasters befindet, muss ich euch einen Strafpunkt geben und du bekommst einen weiteren Versuch.\\ \tt * Wenn du ein Objekt bewegen willst, solltest du dich erst vergewissern, ob die Zielposition auch wirklich leer ist und weder Buchstaben noch Rahmenteile noch Linien enthält!\\ \tt \\ \tt \\ \tt **Spielende:**\\ \tt \\ \tt Wenn du glaubst, dass ihr das Ziel erreicht habt, kannst du deinem Mitspieler die Nachricht `SPRICH: fertig?` schicken. Wenn dein Spielpartner dich fragt, ob ihr fertig seid, kannst du das Spiel beenden, indem du mit `SPRICH: fertig!` antwortest.\\ \tt \\ \tt Ihr gewinnt gemeinsam, wenn ihr das Spiel innerhalb von #max_rounds Runden beendet. Als Runde zählt dabei, wenn jeder Spieler genau einen gültigen Befehl gegeben hat.\\ \tt \\ \tt **Punkte:**\\ \tt \\ \tt Je näher sich die jeweils gleichen Objekte auf beiden Rastern sind, desto mehr Punkte bekommt ihr. Für Strafpunkte gibt es Punktabzug. Könnt ihr den Rekord brechen?\\ \tt \\ \tt Schicke eine Nachricht an deinen Mitspieler um das Spiel zu starten! 
	  } 
	   } 
	   } 
	 & \\ 
 

    \theutterance \stepcounter{utterance}  

    & \multicolumn{4}{p{0.6\linewidth}}{\cellcolor[rgb]{0.8,1,0.9}{%
	\makecell[{{p{\linewidth}}}]{% 
	  \tt {\tiny [A$\rangle$GM]}  
	 SPRICH: Hallo! Lass uns zuerst die Startkoordinaten unserer Objekte austauschen. Bitte sende mir für jedes Objekt die Koordinaten im Format OBJ: x,y. Meine sind:\\ \tt P: 3,1\\ \tt O: 3,5\\ \tt T: 2,4\\ \tt S: 6,7\\ \tt D: 5,3\\ \tt A: 5,7\\ \tt M: 3,4\\ \tt Wenn wir beide Listen haben, wählen wir gemeinsame Zielkoordinaten (am besten deine oder meine aktuelle, um Moves zu sparen). 
	  } 
	   } 
	   } 
	 & & \\ 
 

    \theutterance \stepcounter{utterance}  

    & & & \multicolumn{2}{p{0.3\linewidth}}{\cellcolor[rgb]{0.95,0.95,0.95}{%
	\makecell[{{p{\linewidth}}}]{% 
	  \tt {\tiny [GM$|$GM]}  
	 SPRICH: Hallo! Lass uns zuerst die Startkoordinaten unserer Objekte austauschen. Bitte sende mir für jedes Objekt die Koordinaten im Format OBJ: x,y. Meine sind:\\ \tt P: 3,1\\ \tt O: 3,5\\ \tt T: 2,4\\ \tt S: 6,7\\ \tt D: 5,3\\ \tt A: 5,7\\ \tt M: 3,4\\ \tt Wenn wir beide Listen haben, wählen wir gemeinsame Zielkoordinaten (am besten deine oder meine aktuelle, um Moves zu sparen). 
	  } 
	   } 
	   } 
	 & & \\ 
 

    \theutterance \stepcounter{utterance}  

    & & & \multicolumn{2}{p{0.3\linewidth}}{\cellcolor[rgb]{0.95,0.95,0.95}{%
	\makecell[{{p{\linewidth}}}]{% 
	  \tt {\tiny [GM$|$GM]}  
	 Invalid format: tail is not empty: '\\ \tt P: 3,1\\ \tt O: 3,5\\ \tt T: 2,4\\ \tt S: 6,7\\ \tt D: 5,3\\ \tt A: 5,7\\ \tt M: 3,4\\ \tt Wenn wir beide Listen haben, wählen wir gemeinsame Zielkoordinaten (am besten deine oder meine aktuelle, um Moves zu sparen). 
	  } 
	   } 
	   } 
	 & & \\ 
 

    \theutterance \stepcounter{utterance}  

    & & \multicolumn{4}{p{0.6\linewidth}}{\cellcolor[rgb]{0.9,0.9,0.9}{%
	\makecell[{{p{\linewidth}}}]{% 
	  \tt {\tiny [A$\langle$GM]}  
	 Strafpunkt: Deine Nachricht darf nichts nach dem Befehl enthalten! Stelle sicher, dass deine Antwort ausschließlich einen der beiden Befehle `SPRICH: <NACHRICHT>` und `VERSCHIEBE: <OBJECT>, (<X>, <Y>)` und sonst nichts enthält!\\ \tt \\ \tt Ihr habt gemeinsam 1 von maximal 16 Strafpunkten angesammelt.\\ \tt Bitte versuche es noch mal! 
	  } 
	   } 
	   } 
	 & \\ 
 

    \theutterance \stepcounter{utterance}  

    & \multicolumn{4}{p{0.6\linewidth}}{\cellcolor[rgb]{0.8,1,0.9}{%
	\makecell[{{p{\linewidth}}}]{% 
	  \tt {\tiny [A$\rangle$GM]}  
	 SPRICH: Hallo! Bitte sende mir die Koordinaten aller Objekte im Format OBJ:x,y. Meine sind P:3,1 O:3,5 T:2,4 S:6,7 D:5,3 A:5,7 M:3,4 
	  } 
	   } 
	   } 
	 & & \\ 
 

    \theutterance \stepcounter{utterance}  

    & & & \multicolumn{2}{p{0.3\linewidth}}{\cellcolor[rgb]{0.95,0.95,0.95}{%
	\makecell[{{p{\linewidth}}}]{% 
	  \tt {\tiny [GM$|$GM]}  
	 SPRICH: Hallo! Bitte sende mir die Koordinaten aller Objekte im Format OBJ:x,y. Meine sind P:3,1 O:3,5 T:2,4 S:6,7 D:5,3 A:5,7 M:3,4 
	  } 
	   } 
	   } 
	 & & \\ 
 

    \theutterance \stepcounter{utterance}  

    & & \multicolumn{4}{p{0.6\linewidth}}{\cellcolor[rgb]{0.9,0.9,0.9}{%
	\makecell[{{p{\linewidth}}}]{% 
	  \tt {\tiny [GM$\rangle$B]}  
	 Ich bin der Spielleiter, und du spielst ein kooperatives Spiel mit dem folgenden Raster als Spielbrett:\\ \tt \\ \tt ```\\ \tt  1234567\\ \tt ╔═══════╗\\ \tt ║T◌◌◌S◌◌║ 1\\ \tt ║◌◌◌D◌◌◌║ 2\\ \tt ║◌◌O◌◌┌─╢ 3\\ \tt ║◌M◌┌─┤P║ 4\\ \tt ║◌◌◌│◌├─╢ 5\\ \tt ╟───┼─┘◌║ 6\\ \tt ║◌◌◌│◌A◌║ 7\\ \tt ╚═══╧═══╝\\ \tt \\ \tt ```\\ \tt \\ \tt * Am oberen Rand sind die x-Koordinaten angegeben, die nach rechts hin zunehmen, und am rechten Rand die y-Koordinaten, die nach unten zunehmen.\\ \tt * Die folgenden Objekte sind zufällig auf deinem Raster platziert: 'P', 'O', 'T', 'S', 'D', 'A', 'M'.\\ \tt \\ \tt \\ \tt Dein Mitspieler sieht eine Variation des Spielbretts, auf dem dieselben Objekte an anderen zufälligen Positionen platziert sind. Du kannst das Spielbrett des anderen Spielers nicht sehen, und er kann deins nicht sehen.\\ \tt \\ \tt \\ \tt **Ziel des Spiels:**\\ \tt \\ \tt Beide Spieler müssen die Objekte auf ihrem jeweiligen Spielbrett so bewegen, dass identische Objekte am Ende an denselben Koordinaten liegen. Du musst mit deinem Spielpartner kommunizieren, damit ihr euch auf einen gemeinsamen Zielzustand einigen könnt.\\ \tt \\ \tt \\ \tt **Spielregeln:**\\ \tt \\ \tt * In jedem Spielzug kannst du genau eine der folgenden zwei Aktionen ausführen:\\ \tt 1. `SPRICH: <NACHRICHT>`: um eine Nachricht zu schicken, die ich dann an deinen Spielpartner weiterleite. Als Nachricht gilt alles bis zum nächsten Zeilenumbruch.\\ \tt 2. `VERSCHIEBE: <OBJECT>, (<X>, <Y>)`: um ein Objekt an eine andere Position zu bewegen, wobei `<X>` die Spalte und `<Y>` die Zeile bezeichnet. Ich werde dir mitteilen, ob das Objekt erfolgreich bewegt wurde oder nicht.\\ \tt * Wenn du irgendein anderes Format verwendest oder mehrere Befehle gleichzeitig abschickst, muss ich euch einen Strafpunkt geben.\\ \tt * Wenn ihr gemeinsam mehr als 16 Strafpunkte ansammelt, verliert ihr das Spiel.\\ \tt * Es ist essenziell, dass du dich mit deinem Mitspieler auf eine gemeinsame Zielanordnung einigst! Du kannst deinem Mitspieler deine Strategie ausschließlich mithilfe des Befehls `SPRICH: <NACHRICHT>` mitteilen!\\ \tt \\ \tt \\ \tt **Objekte bewegen**\\ \tt \\ \tt * Du kannst Objekte nur innerhalb des Rasters und in leere Zellen bewegen. Leere Zellen erkennst am Zeichen '◌'.\\ \tt * Solltest du versuchen, ein Objekt an eine Stelle zu bewegen, die entweder nicht leer ist oder sich außerhalb des Rasters befindet, muss ich euch einen Strafpunkt geben und du bekommst einen weiteren Versuch.\\ \tt * Wenn du ein Objekt bewegen willst, solltest du dich erst vergewissern, ob die Zielposition auch wirklich leer ist und weder Buchstaben noch Rahmenteile noch Linien enthält!\\ \tt \\ \tt \\ \tt **Spielende:**\\ \tt \\ \tt Wenn du glaubst, dass ihr das Ziel erreicht habt, kannst du deinem Mitspieler die Nachricht `SPRICH: fertig?` schicken. Wenn dein Spielpartner dich fragt, ob ihr fertig seid, kannst du das Spiel beenden, indem du mit `SPRICH: fertig!` antwortest.\\ \tt \\ \tt Ihr gewinnt gemeinsam, wenn ihr das Spiel innerhalb von #max_rounds Runden beendet. Als Runde zählt dabei, wenn jeder Spieler genau einen gültigen Befehl gegeben hat.\\ \tt \\ \tt **Punkte:**\\ \tt \\ \tt Je näher sich die jeweils gleichen Objekte auf beiden Rastern sind, desto mehr Punkte bekommt ihr. Für Strafpunkte gibt es Punktabzug. Könnt ihr den Rekord brechen?\\ \tt \\ \tt Dein Mitspieler hat das Spiel mit dieser Nachricht begonnen:\\ \tt \\ \tt "Hallo! Bitte sende mir die Koordinaten aller Objekte im Format OBJ:x,y. Meine sind P:3,1 O:3,5 T:2,4 S:6,7 D:5,3 A:5,7 M:3,4"\\ \tt \\ \tt Wie lautet dein erster Befehl? 
	  } 
	   } 
	   } 
	 & \\ 
 

    \theutterance \stepcounter{utterance}  

    & & & \multicolumn{4}{p{0.6\linewidth}}{\cellcolor[rgb]{1,0.85,0.72}{%
	\makecell[{{p{\linewidth}}}]{% 
	  \tt {\tiny [GM$\langle$B]}  
	 SPRICH: Hallo! Meine Koordinaten sind P:7,4 O:3,3 T:1,1 S:4,1 D:4,2 A:6,7 M:2,4 
	  } 
	   } 
	   } 
	  \\ 
 

    \theutterance \stepcounter{utterance}  

    & & & \multicolumn{2}{p{0.3\linewidth}}{\cellcolor[rgb]{0.95,0.95,0.95}{%
	\makecell[{{p{\linewidth}}}]{% 
	  \tt {\tiny [GM$|$GM]}  
	 SPRICH: Hallo! Meine Koordinaten sind P:7,4 O:3,3 T:1,1 S:4,1 D:4,2 A:6,7 M:2,4 
	  } 
	   } 
	   } 
	 & & \\ 
 

    \theutterance \stepcounter{utterance}  

    & & \multicolumn{4}{p{0.6\linewidth}}{\cellcolor[rgb]{0.9,0.9,0.9}{%
	\makecell[{{p{\linewidth}}}]{% 
	  \tt {\tiny [A$\langle$GM]}  
	 Deine Nachricht wurde an deinen Mitspieler übermittelt.\\ \tt \\ \tt \\ \tt Ihr seid gerade bei Runde 1 von maximal 28.\\ \tt Ihr habt gemeinsam 1 von maximal 16 Strafpunkten angesammelt.\\ \tt Dein Mitspieler hat dir diese Nachricht geschickt:\\ \tt \\ \tt \\ \tt "Hallo! Meine Koordinaten sind P:7,4 O:3,3 T:1,1 S:4,1 D:4,2 A:6,7 M:2,4"\\ \tt \\ \tt \\ \tt Wie lautet dein nächster Befehl? 
	  } 
	   } 
	   } 
	 & \\ 
 

    \theutterance \stepcounter{utterance}  

    & \multicolumn{4}{p{0.6\linewidth}}{\cellcolor[rgb]{0.8,1,0.9}{%
	\makecell[{{p{\linewidth}}}]{% 
	  \tt {\tiny [A$\rangle$GM]}  
	 SPRICH: Vorschlag: Wir übernehmen möglichst eure Positionen als Ziel, außer wo ich schon dort bin. Gemeinsames Ziel: P:7,4 O:3,3 T:1,1 S:4,1 D:4,2 A:6,7 M:2,4. Ich stehe bereits bei A:5,7 statt 6,7 und S:6,7 statt 4,1 etc., also muss ich viel bewegen; willst du stattdessen meine Positionen übernehmen? Meine: P:3,1 O:3,5 T:2,4 S:6,7 D:5,3 A:5,7 M:3,4. Wähle bitte, welche Liste wir nehmen, oder schlage eine Mischliste mit minimalen Zügen vor. 
	  } 
	   } 
	   } 
	 & & \\ 
 

    \theutterance \stepcounter{utterance}  

    & & & \multicolumn{2}{p{0.3\linewidth}}{\cellcolor[rgb]{0.95,0.95,0.95}{%
	\makecell[{{p{\linewidth}}}]{% 
	  \tt {\tiny [GM$|$GM]}  
	 SPRICH: Vorschlag: Wir übernehmen möglichst eure Positionen als Ziel, außer wo ich schon dort bin. Gemeinsames Ziel: P:7,4 O:3,3 T:1,1 S:4,1 D:4,2 A:6,7 M:2,4. Ich stehe bereits bei A:5,7 statt 6,7 und S:6,7 statt 4,1 etc., also muss ich viel bewegen; willst du stattdessen meine Positionen übernehmen? Meine: P:3,1 O:3,5 T:2,4 S:6,7 D:5,3 A:5,7 M:3,4. Wähle bitte, welche Liste wir nehmen, oder schlage eine Mischliste mit minimalen Zügen vor. 
	  } 
	   } 
	   } 
	 & & \\ 
 

    \theutterance \stepcounter{utterance}  

    & & \multicolumn{4}{p{0.6\linewidth}}{\cellcolor[rgb]{0.9,0.9,0.9}{%
	\makecell[{{p{\linewidth}}}]{% 
	  \tt {\tiny [GM$\rangle$B]}  
	 Deine Nachricht wurde an deinen Mitspieler übermittelt.\\ \tt \\ \tt \\ \tt Ihr seid gerade bei Runde 2 von maximal 28.\\ \tt Ihr habt gemeinsam 1 von maximal 16 Strafpunkten angesammelt.\\ \tt Dein Mitspieler hat dir diese Nachricht geschickt:\\ \tt \\ \tt \\ \tt "Vorschlag: Wir übernehmen möglichst eure Positionen als Ziel, außer wo ich schon dort bin. Gemeinsames Ziel: P:7,4 O:3,3 T:1,1 S:4,1 D:4,2 A:6,7 M:2,4. Ich stehe bereits bei A:5,7 statt 6,7 und S:6,7 statt 4,1 etc., also muss ich viel bewegen; willst du stattdessen meine Positionen übernehmen? Meine: P:3,1 O:3,5 T:2,4 S:6,7 D:5,3 A:5,7 M:3,4. Wähle bitte, welche Liste wir nehmen, oder schlage eine Mischliste mit minimalen Zügen vor."\\ \tt \\ \tt \\ \tt Wie lautet dein nächster Befehl? 
	  } 
	   } 
	   } 
	 & \\ 
 

    \theutterance \stepcounter{utterance}  

    & & & \multicolumn{4}{p{0.6\linewidth}}{\cellcolor[rgb]{1,0.85,0.72}{%
	\makecell[{{p{\linewidth}}}]{% 
	  \tt {\tiny [GM$\langle$B]}  
	 SPRICH: Lass uns deine Liste als Ziel nehmen, da du sonst viel bewegen müsstest. Gemeinsames Ziel: P:3,1 O:3,5 T:2,4 S:6,7 D:5,3 A:5,7 M:3,4. Bitte bestätige. 
	  } 
	   } 
	   } 
	  \\ 
 

    \theutterance \stepcounter{utterance}  

    & & & \multicolumn{2}{p{0.3\linewidth}}{\cellcolor[rgb]{0.95,0.95,0.95}{%
	\makecell[{{p{\linewidth}}}]{% 
	  \tt {\tiny [GM$|$GM]}  
	 SPRICH: Lass uns deine Liste als Ziel nehmen, da du sonst viel bewegen müsstest. Gemeinsames Ziel: P:3,1 O:3,5 T:2,4 S:6,7 D:5,3 A:5,7 M:3,4. Bitte bestätige. 
	  } 
	   } 
	   } 
	 & & \\ 
 

    \theutterance \stepcounter{utterance}  

    & & \multicolumn{4}{p{0.6\linewidth}}{\cellcolor[rgb]{0.9,0.9,0.9}{%
	\makecell[{{p{\linewidth}}}]{% 
	  \tt {\tiny [A$\langle$GM]}  
	 Deine Nachricht wurde an deinen Mitspieler übermittelt.\\ \tt \\ \tt \\ \tt Ihr seid gerade bei Runde 2 von maximal 28.\\ \tt Ihr habt gemeinsam 1 von maximal 16 Strafpunkten angesammelt.\\ \tt Dein Mitspieler hat dir diese Nachricht geschickt:\\ \tt \\ \tt \\ \tt "Lass uns deine Liste als Ziel nehmen, da du sonst viel bewegen müsstest. Gemeinsames Ziel: P:3,1 O:3,5 T:2,4 S:6,7 D:5,3 A:5,7 M:3,4. Bitte bestätige."\\ \tt \\ \tt \\ \tt Wie lautet dein nächster Befehl? 
	  } 
	   } 
	   } 
	 & \\ 
 

    \theutterance \stepcounter{utterance}  

    & \multicolumn{4}{p{0.6\linewidth}}{\cellcolor[rgb]{0.8,1,0.9}{%
	\makecell[{{p{\linewidth}}}]{% 
	  \tt {\tiny [A$\rangle$GM]}  
	 SPRICH: Bestätigt. Gemeinsames Ziel: P:3,1 O:3,5 T:2,4 S:6,7 D:5,3 A:5,7 M:3,4. Bitte sag Bescheid, wenn du soweit bist, dann starte ich mit Anpassungen. 
	  } 
	   } 
	   } 
	 & & \\ 
 

    \theutterance \stepcounter{utterance}  

    & & & \multicolumn{2}{p{0.3\linewidth}}{\cellcolor[rgb]{0.95,0.95,0.95}{%
	\makecell[{{p{\linewidth}}}]{% 
	  \tt {\tiny [GM$|$GM]}  
	 SPRICH: Bestätigt. Gemeinsames Ziel: P:3,1 O:3,5 T:2,4 S:6,7 D:5,3 A:5,7 M:3,4. Bitte sag Bescheid, wenn du soweit bist, dann starte ich mit Anpassungen. 
	  } 
	   } 
	   } 
	 & & \\ 
 

    \theutterance \stepcounter{utterance}  

    & & \multicolumn{4}{p{0.6\linewidth}}{\cellcolor[rgb]{0.9,0.9,0.9}{%
	\makecell[{{p{\linewidth}}}]{% 
	  \tt {\tiny [GM$\rangle$B]}  
	 Deine Nachricht wurde an deinen Mitspieler übermittelt.\\ \tt \\ \tt \\ \tt Ihr seid gerade bei Runde 3 von maximal 28.\\ \tt Ihr habt gemeinsam 1 von maximal 16 Strafpunkten angesammelt.\\ \tt Dein Mitspieler hat dir diese Nachricht geschickt:\\ \tt \\ \tt \\ \tt "Bestätigt. Gemeinsames Ziel: P:3,1 O:3,5 T:2,4 S:6,7 D:5,3 A:5,7 M:3,4. Bitte sag Bescheid, wenn du soweit bist, dann starte ich mit Anpassungen."\\ \tt \\ \tt \\ \tt Wie lautet dein nächster Befehl? 
	  } 
	   } 
	   } 
	 & \\ 
 

    \theutterance \stepcounter{utterance}  

    & & & \multicolumn{4}{p{0.6\linewidth}}{\cellcolor[rgb]{1,0.85,0.72}{%
	\makecell[{{p{\linewidth}}}]{% 
	  \tt {\tiny [GM$\langle$B]}  
	 SPRICH: Bereit. Ich beginne mit Verschiebungen gemäß Ziel P:3,1 O:3,5 T:2,4 S:6,7 D:5,3 A:5,7 M:3,4. Starte du auch. 
	  } 
	   } 
	   } 
	  \\ 
 

    \theutterance \stepcounter{utterance}  

    & & & \multicolumn{2}{p{0.3\linewidth}}{\cellcolor[rgb]{0.95,0.95,0.95}{%
	\makecell[{{p{\linewidth}}}]{% 
	  \tt {\tiny [GM$|$GM]}  
	 SPRICH: Bereit. Ich beginne mit Verschiebungen gemäß Ziel P:3,1 O:3,5 T:2,4 S:6,7 D:5,3 A:5,7 M:3,4. Starte du auch. 
	  } 
	   } 
	   } 
	 & & \\ 
 

    \theutterance \stepcounter{utterance}  

    & & \multicolumn{4}{p{0.6\linewidth}}{\cellcolor[rgb]{0.9,0.9,0.9}{%
	\makecell[{{p{\linewidth}}}]{% 
	  \tt {\tiny [A$\langle$GM]}  
	 Deine Nachricht wurde an deinen Mitspieler übermittelt.\\ \tt \\ \tt \\ \tt Ihr seid gerade bei Runde 3 von maximal 28.\\ \tt Ihr habt gemeinsam 1 von maximal 16 Strafpunkten angesammelt.\\ \tt Dein Mitspieler hat dir diese Nachricht geschickt:\\ \tt \\ \tt \\ \tt "Bereit. Ich beginne mit Verschiebungen gemäß Ziel P:3,1 O:3,5 T:2,4 S:6,7 D:5,3 A:5,7 M:3,4. Starte du auch."\\ \tt \\ \tt \\ \tt Wie lautet dein nächster Befehl? 
	  } 
	   } 
	   } 
	 & \\ 
 

    \theutterance \stepcounter{utterance}  

    & \multicolumn{4}{p{0.6\linewidth}}{\cellcolor[rgb]{0.8,1,0.9}{%
	\makecell[{{p{\linewidth}}}]{% 
	  \tt {\tiny [A$\rangle$GM]}  
	 VERSCHIEBE: A, (5,7) 
	  } 
	   } 
	   } 
	 & & \\ 
 

    \theutterance \stepcounter{utterance}  

    & & & \multicolumn{2}{p{0.3\linewidth}}{\cellcolor[rgb]{0.95,0.95,0.95}{%
	\makecell[{{p{\linewidth}}}]{% 
	  \tt {\tiny [GM$|$GM]}  
	 VERSCHIEBE: A, (5,7) 
	  } 
	   } 
	   } 
	 & & \\ 
 

    \theutterance \stepcounter{utterance}  

    & & & \multicolumn{2}{p{0.3\linewidth}}{\cellcolor[rgb]{0.95,0.95,0.95}{%
	\makecell[{{p{\linewidth}}}]{% 
	  \tt {\tiny [GM$|$GM]}  
	 'A' erfolgreich nach (5,7) verschoben. Dein Raster sieht jetzt so aus:\\ \tt ```\\ \tt  1234567\\ \tt ╔═══════╗\\ \tt ║◌◌P◌◌◌◌║ 1\\ \tt ║◌◌◌◌◌◌◌║ 2\\ \tt ║◌◌◌◌D┌─╢ 3\\ \tt ║◌TM┌─┤◌║ 4\\ \tt ║◌◌O│◌├─╢ 5\\ \tt ╟───┼─┘◌║ 6\\ \tt ║◌◌◌│A◌S║ 7\\ \tt ╚═══╧═══╝\\ \tt ```\\ \tt  
	  } 
	   } 
	   } 
	 & & \\ 
 

    \theutterance \stepcounter{utterance}  

    & & \multicolumn{4}{p{0.6\linewidth}}{\cellcolor[rgb]{0.9,0.9,0.9}{%
	\makecell[{{p{\linewidth}}}]{% 
	  \tt {\tiny [GM$\rangle$B]}  
	 Deine Nachricht wurde an deinen Mitspieler übermittelt.\\ \tt \\ \tt \\ \tt Ihr seid gerade bei Runde 4 von maximal 28.\\ \tt Ihr habt gemeinsam 1 von maximal 16 Strafpunkten angesammelt.\\ \tt Dein Mitspieler hat ein Objekt auf seinem Raster bewegt.\\ \tt \\ \tt \\ \tt Wie lautet dein nächster Befehl? 
	  } 
	   } 
	   } 
	 & \\ 
 

    \theutterance \stepcounter{utterance}  

    & & & \multicolumn{4}{p{0.6\linewidth}}{\cellcolor[rgb]{1,0.85,0.72}{%
	\makecell[{{p{\linewidth}}}]{% 
	  \tt {\tiny [GM$\langle$B]}  
	 VERSCHIEBE: A, (5, 7) 
	  } 
	   } 
	   } 
	  \\ 
 

    \theutterance \stepcounter{utterance}  

    & & & \multicolumn{2}{p{0.3\linewidth}}{\cellcolor[rgb]{0.95,0.95,0.95}{%
	\makecell[{{p{\linewidth}}}]{% 
	  \tt {\tiny [GM$|$GM]}  
	 VERSCHIEBE: A, (5, 7) 
	  } 
	   } 
	   } 
	 & & \\ 
 

    \theutterance \stepcounter{utterance}  

    & & & \multicolumn{2}{p{0.3\linewidth}}{\cellcolor[rgb]{0.95,0.95,0.95}{%
	\makecell[{{p{\linewidth}}}]{% 
	  \tt {\tiny [GM$|$GM]}  
	 'A' erfolgreich nach (5,7) verschoben. Dein Raster sieht jetzt so aus:\\ \tt ```\\ \tt  1234567\\ \tt ╔═══════╗\\ \tt ║T◌◌◌S◌◌║ 1\\ \tt ║◌◌◌D◌◌◌║ 2\\ \tt ║◌◌O◌◌┌─╢ 3\\ \tt ║◌M◌┌─┤P║ 4\\ \tt ║◌◌◌│◌├─╢ 5\\ \tt ╟───┼─┘◌║ 6\\ \tt ║◌◌◌│A◌◌║ 7\\ \tt ╚═══╧═══╝\\ \tt ```\\ \tt  
	  } 
	   } 
	   } 
	 & & \\ 
 

    \theutterance \stepcounter{utterance}  

    & & \multicolumn{4}{p{0.6\linewidth}}{\cellcolor[rgb]{0.9,0.9,0.9}{%
	\makecell[{{p{\linewidth}}}]{% 
	  \tt {\tiny [A$\langle$GM]}  
	 'A' erfolgreich nach (5,7) verschoben. Dein Raster sieht jetzt so aus:\\ \tt ```\\ \tt  1234567\\ \tt ╔═══════╗\\ \tt ║◌◌P◌◌◌◌║ 1\\ \tt ║◌◌◌◌◌◌◌║ 2\\ \tt ║◌◌◌◌D┌─╢ 3\\ \tt ║◌TM┌─┤◌║ 4\\ \tt ║◌◌O│◌├─╢ 5\\ \tt ╟───┼─┘◌║ 6\\ \tt ║◌◌◌│A◌S║ 7\\ \tt ╚═══╧═══╝\\ \tt ```\\ \tt Ihr seid gerade bei Runde 4 von maximal 28.\\ \tt Ihr habt gemeinsam 1 von maximal 16 Strafpunkten angesammelt.\\ \tt Dein Mitspieler hat ein Objekt auf seinem Raster bewegt.\\ \tt \\ \tt \\ \tt Wie lautet dein nächster Befehl? 
	  } 
	   } 
	   } 
	 & \\ 
 

    \theutterance \stepcounter{utterance}  

    & \multicolumn{4}{p{0.6\linewidth}}{\cellcolor[rgb]{0.8,1,0.9}{%
	\makecell[{{p{\linewidth}}}]{% 
	  \tt {\tiny [A$\rangle$GM]}  
	 VERSCHIEBE: S, (6,7) 
	  } 
	   } 
	   } 
	 & & \\ 
 

    \theutterance \stepcounter{utterance}  

    & & & \multicolumn{2}{p{0.3\linewidth}}{\cellcolor[rgb]{0.95,0.95,0.95}{%
	\makecell[{{p{\linewidth}}}]{% 
	  \tt {\tiny [GM$|$GM]}  
	 VERSCHIEBE: S, (6,7) 
	  } 
	   } 
	   } 
	 & & \\ 
 

    \theutterance \stepcounter{utterance}  

    & & & \multicolumn{2}{p{0.3\linewidth}}{\cellcolor[rgb]{0.95,0.95,0.95}{%
	\makecell[{{p{\linewidth}}}]{% 
	  \tt {\tiny [GM$|$GM]}  
	 'S' erfolgreich nach (6,7) verschoben. Dein Raster sieht jetzt so aus:\\ \tt ```\\ \tt  1234567\\ \tt ╔═══════╗\\ \tt ║◌◌P◌◌◌◌║ 1\\ \tt ║◌◌◌◌◌◌◌║ 2\\ \tt ║◌◌◌◌D┌─╢ 3\\ \tt ║◌TM┌─┤◌║ 4\\ \tt ║◌◌O│◌├─╢ 5\\ \tt ╟───┼─┘◌║ 6\\ \tt ║◌◌◌│AS◌║ 7\\ \tt ╚═══╧═══╝\\ \tt ```\\ \tt  
	  } 
	   } 
	   } 
	 & & \\ 
 

    \theutterance \stepcounter{utterance}  

    & & \multicolumn{4}{p{0.6\linewidth}}{\cellcolor[rgb]{0.9,0.9,0.9}{%
	\makecell[{{p{\linewidth}}}]{% 
	  \tt {\tiny [GM$\rangle$B]}  
	 'A' erfolgreich nach (5,7) verschoben. Dein Raster sieht jetzt so aus:\\ \tt ```\\ \tt  1234567\\ \tt ╔═══════╗\\ \tt ║T◌◌◌S◌◌║ 1\\ \tt ║◌◌◌D◌◌◌║ 2\\ \tt ║◌◌O◌◌┌─╢ 3\\ \tt ║◌M◌┌─┤P║ 4\\ \tt ║◌◌◌│◌├─╢ 5\\ \tt ╟───┼─┘◌║ 6\\ \tt ║◌◌◌│A◌◌║ 7\\ \tt ╚═══╧═══╝\\ \tt ```\\ \tt Ihr seid gerade bei Runde 5 von maximal 28.\\ \tt Ihr habt gemeinsam 1 von maximal 16 Strafpunkten angesammelt.\\ \tt Dein Mitspieler hat ein Objekt auf seinem Raster bewegt.\\ \tt \\ \tt \\ \tt Wie lautet dein nächster Befehl? 
	  } 
	   } 
	   } 
	 & \\ 
 

    \theutterance \stepcounter{utterance}  

    & & & \multicolumn{4}{p{0.6\linewidth}}{\cellcolor[rgb]{1,0.85,0.72}{%
	\makecell[{{p{\linewidth}}}]{% 
	  \tt {\tiny [GM$\langle$B]}  
	 VERSCHIEBE: P, (3, 1) 
	  } 
	   } 
	   } 
	  \\ 
 

    \theutterance \stepcounter{utterance}  

    & & & \multicolumn{2}{p{0.3\linewidth}}{\cellcolor[rgb]{0.95,0.95,0.95}{%
	\makecell[{{p{\linewidth}}}]{% 
	  \tt {\tiny [GM$|$GM]}  
	 VERSCHIEBE: P, (3, 1) 
	  } 
	   } 
	   } 
	 & & \\ 
 

    \theutterance \stepcounter{utterance}  

    & & & \multicolumn{2}{p{0.3\linewidth}}{\cellcolor[rgb]{0.95,0.95,0.95}{%
	\makecell[{{p{\linewidth}}}]{% 
	  \tt {\tiny [GM$|$GM]}  
	 'P' erfolgreich nach (3,1) verschoben. Dein Raster sieht jetzt so aus:\\ \tt ```\\ \tt  1234567\\ \tt ╔═══════╗\\ \tt ║T◌P◌S◌◌║ 1\\ \tt ║◌◌◌D◌◌◌║ 2\\ \tt ║◌◌O◌◌┌─╢ 3\\ \tt ║◌M◌┌─┤◌║ 4\\ \tt ║◌◌◌│◌├─╢ 5\\ \tt ╟───┼─┘◌║ 6\\ \tt ║◌◌◌│A◌◌║ 7\\ \tt ╚═══╧═══╝\\ \tt ```\\ \tt  
	  } 
	   } 
	   } 
	 & & \\ 
 

    \theutterance \stepcounter{utterance}  

    & & \multicolumn{4}{p{0.6\linewidth}}{\cellcolor[rgb]{0.9,0.9,0.9}{%
	\makecell[{{p{\linewidth}}}]{% 
	  \tt {\tiny [A$\langle$GM]}  
	 'S' erfolgreich nach (6,7) verschoben. Dein Raster sieht jetzt so aus:\\ \tt ```\\ \tt  1234567\\ \tt ╔═══════╗\\ \tt ║◌◌P◌◌◌◌║ 1\\ \tt ║◌◌◌◌◌◌◌║ 2\\ \tt ║◌◌◌◌D┌─╢ 3\\ \tt ║◌TM┌─┤◌║ 4\\ \tt ║◌◌O│◌├─╢ 5\\ \tt ╟───┼─┘◌║ 6\\ \tt ║◌◌◌│AS◌║ 7\\ \tt ╚═══╧═══╝\\ \tt ```\\ \tt Ihr seid gerade bei Runde 5 von maximal 28.\\ \tt Ihr habt gemeinsam 1 von maximal 16 Strafpunkten angesammelt.\\ \tt Dein Mitspieler hat ein Objekt auf seinem Raster bewegt.\\ \tt \\ \tt \\ \tt Wie lautet dein nächster Befehl? 
	  } 
	   } 
	   } 
	 & \\ 
 

    \theutterance \stepcounter{utterance}  

    & \multicolumn{4}{p{0.6\linewidth}}{\cellcolor[rgb]{0.8,1,0.9}{%
	\makecell[{{p{\linewidth}}}]{% 
	  \tt {\tiny [A$\rangle$GM]}  
	 VERSCHIEBE: P, (3,1) 
	  } 
	   } 
	   } 
	 & & \\ 
 

    \theutterance \stepcounter{utterance}  

    & & & \multicolumn{2}{p{0.3\linewidth}}{\cellcolor[rgb]{0.95,0.95,0.95}{%
	\makecell[{{p{\linewidth}}}]{% 
	  \tt {\tiny [GM$|$GM]}  
	 VERSCHIEBE: P, (3,1) 
	  } 
	   } 
	   } 
	 & & \\ 
 

    \theutterance \stepcounter{utterance}  

    & & & \multicolumn{2}{p{0.3\linewidth}}{\cellcolor[rgb]{0.95,0.95,0.95}{%
	\makecell[{{p{\linewidth}}}]{% 
	  \tt {\tiny [GM$|$GM]}  
	 Strafpunkt: (3,1) ist nicht leer, sondern enthält 'P'.\\ \tt \\ \tt Ihr habt gemeinsam 2 von maximal 16 Strafpunkten angesammelt.\\ \tt Bitte versuche es noch mal! 
	  } 
	   } 
	   } 
	 & & \\ 
 

    \theutterance \stepcounter{utterance}  

    & & \multicolumn{4}{p{0.6\linewidth}}{\cellcolor[rgb]{0.9,0.9,0.9}{%
	\makecell[{{p{\linewidth}}}]{% 
	  \tt {\tiny [A$\langle$GM]}  
	 Strafpunkt: (3,1) ist nicht leer, sondern enthält 'P'.\\ \tt \\ \tt Ihr habt gemeinsam 2 von maximal 16 Strafpunkten angesammelt.\\ \tt Bitte versuche es noch mal! 
	  } 
	   } 
	   } 
	 & \\ 
 

    \theutterance \stepcounter{utterance}  

    & \multicolumn{4}{p{0.6\linewidth}}{\cellcolor[rgb]{0.8,1,0.9}{%
	\makecell[{{p{\linewidth}}}]{% 
	  \tt {\tiny [A$\rangle$GM]}  
	 SPRICH: Mein Fehler mit P. Ich fahre fort. Nächste Ziele unverändert: P:3,1 O:3,5 T:2,4 S:6,7 D:5,3 A:5,7 M:3,4. Sag Bescheid, wenn eins bei dir blockiert ist. 
	  } 
	   } 
	   } 
	 & & \\ 
 

    \theutterance \stepcounter{utterance}  

    & & & \multicolumn{2}{p{0.3\linewidth}}{\cellcolor[rgb]{0.95,0.95,0.95}{%
	\makecell[{{p{\linewidth}}}]{% 
	  \tt {\tiny [GM$|$GM]}  
	 SPRICH: Mein Fehler mit P. Ich fahre fort. Nächste Ziele unverändert: P:3,1 O:3,5 T:2,4 S:6,7 D:5,3 A:5,7 M:3,4. Sag Bescheid, wenn eins bei dir blockiert ist. 
	  } 
	   } 
	   } 
	 & & \\ 
 

    \theutterance \stepcounter{utterance}  

    & & \multicolumn{4}{p{0.6\linewidth}}{\cellcolor[rgb]{0.9,0.9,0.9}{%
	\makecell[{{p{\linewidth}}}]{% 
	  \tt {\tiny [GM$\rangle$B]}  
	 'P' erfolgreich nach (3,1) verschoben. Dein Raster sieht jetzt so aus:\\ \tt ```\\ \tt  1234567\\ \tt ╔═══════╗\\ \tt ║T◌P◌S◌◌║ 1\\ \tt ║◌◌◌D◌◌◌║ 2\\ \tt ║◌◌O◌◌┌─╢ 3\\ \tt ║◌M◌┌─┤◌║ 4\\ \tt ║◌◌◌│◌├─╢ 5\\ \tt ╟───┼─┘◌║ 6\\ \tt ║◌◌◌│A◌◌║ 7\\ \tt ╚═══╧═══╝\\ \tt ```\\ \tt Ihr seid gerade bei Runde 6 von maximal 28.\\ \tt Ihr habt gemeinsam 2 von maximal 16 Strafpunkten angesammelt.\\ \tt Dein Mitspieler hat dir diese Nachricht geschickt:\\ \tt \\ \tt \\ \tt "Mein Fehler mit P. Ich fahre fort. Nächste Ziele unverändert: P:3,1 O:3,5 T:2,4 S:6,7 D:5,3 A:5,7 M:3,4. Sag Bescheid, wenn eins bei dir blockiert ist."\\ \tt \\ \tt \\ \tt Wie lautet dein nächster Befehl? 
	  } 
	   } 
	   } 
	 & \\ 
 

    \theutterance \stepcounter{utterance}  

    & & & \multicolumn{4}{p{0.6\linewidth}}{\cellcolor[rgb]{1,0.85,0.72}{%
	\makecell[{{p{\linewidth}}}]{% 
	  \tt {\tiny [GM$\langle$B]}  
	 VERSCHIEBE: M, (3, 4) 
	  } 
	   } 
	   } 
	  \\ 
 

    \theutterance \stepcounter{utterance}  

    & & & \multicolumn{2}{p{0.3\linewidth}}{\cellcolor[rgb]{0.95,0.95,0.95}{%
	\makecell[{{p{\linewidth}}}]{% 
	  \tt {\tiny [GM$|$GM]}  
	 VERSCHIEBE: M, (3, 4) 
	  } 
	   } 
	   } 
	 & & \\ 
 

    \theutterance \stepcounter{utterance}  

    & & & \multicolumn{2}{p{0.3\linewidth}}{\cellcolor[rgb]{0.95,0.95,0.95}{%
	\makecell[{{p{\linewidth}}}]{% 
	  \tt {\tiny [GM$|$GM]}  
	 'M' erfolgreich nach (3,4) verschoben. Dein Raster sieht jetzt so aus:\\ \tt ```\\ \tt  1234567\\ \tt ╔═══════╗\\ \tt ║T◌P◌S◌◌║ 1\\ \tt ║◌◌◌D◌◌◌║ 2\\ \tt ║◌◌O◌◌┌─╢ 3\\ \tt ║◌◌M┌─┤◌║ 4\\ \tt ║◌◌◌│◌├─╢ 5\\ \tt ╟───┼─┘◌║ 6\\ \tt ║◌◌◌│A◌◌║ 7\\ \tt ╚═══╧═══╝\\ \tt ```\\ \tt  
	  } 
	   } 
	   } 
	 & & \\ 
 

    \theutterance \stepcounter{utterance}  

    & & \multicolumn{4}{p{0.6\linewidth}}{\cellcolor[rgb]{0.9,0.9,0.9}{%
	\makecell[{{p{\linewidth}}}]{% 
	  \tt {\tiny [A$\langle$GM]}  
	 Deine Nachricht wurde an deinen Mitspieler übermittelt.\\ \tt \\ \tt \\ \tt Ihr seid gerade bei Runde 6 von maximal 28.\\ \tt Ihr habt gemeinsam 2 von maximal 16 Strafpunkten angesammelt.\\ \tt Dein Mitspieler hat ein Objekt auf seinem Raster bewegt.\\ \tt \\ \tt \\ \tt Wie lautet dein nächster Befehl? 
	  } 
	   } 
	   } 
	 & \\ 
 

    \theutterance \stepcounter{utterance}  

    & \multicolumn{4}{p{0.6\linewidth}}{\cellcolor[rgb]{0.8,1,0.9}{%
	\makecell[{{p{\linewidth}}}]{% 
	  \tt {\tiny [A$\rangle$GM]}  
	 VERSCHIEBE: O, (3,5) 
	  } 
	   } 
	   } 
	 & & \\ 
 

    \theutterance \stepcounter{utterance}  

    & & & \multicolumn{2}{p{0.3\linewidth}}{\cellcolor[rgb]{0.95,0.95,0.95}{%
	\makecell[{{p{\linewidth}}}]{% 
	  \tt {\tiny [GM$|$GM]}  
	 VERSCHIEBE: O, (3,5) 
	  } 
	   } 
	   } 
	 & & \\ 
 

    \theutterance \stepcounter{utterance}  

    & & & \multicolumn{2}{p{0.3\linewidth}}{\cellcolor[rgb]{0.95,0.95,0.95}{%
	\makecell[{{p{\linewidth}}}]{% 
	  \tt {\tiny [GM$|$GM]}  
	 Strafpunkt: (3,5) ist nicht leer, sondern enthält 'O'.\\ \tt \\ \tt Ihr habt gemeinsam 3 von maximal 16 Strafpunkten angesammelt.\\ \tt Bitte versuche es noch mal! 
	  } 
	   } 
	   } 
	 & & \\ 
 

    \theutterance \stepcounter{utterance}  

    & & \multicolumn{4}{p{0.6\linewidth}}{\cellcolor[rgb]{0.9,0.9,0.9}{%
	\makecell[{{p{\linewidth}}}]{% 
	  \tt {\tiny [A$\langle$GM]}  
	 Strafpunkt: (3,5) ist nicht leer, sondern enthält 'O'.\\ \tt \\ \tt Ihr habt gemeinsam 3 von maximal 16 Strafpunkten angesammelt.\\ \tt Bitte versuche es noch mal! 
	  } 
	   } 
	   } 
	 & \\ 
 

    \theutterance \stepcounter{utterance}  

    & \multicolumn{4}{p{0.6\linewidth}}{\cellcolor[rgb]{0.8,1,0.9}{%
	\makecell[{{p{\linewidth}}}]{% 
	  \tt {\tiny [A$\rangle$GM]}  
	 VERSCHIEBE: T, (2,4) 
	  } 
	   } 
	   } 
	 & & \\ 
 

    \theutterance \stepcounter{utterance}  

    & & & \multicolumn{2}{p{0.3\linewidth}}{\cellcolor[rgb]{0.95,0.95,0.95}{%
	\makecell[{{p{\linewidth}}}]{% 
	  \tt {\tiny [GM$|$GM]}  
	 VERSCHIEBE: T, (2,4) 
	  } 
	   } 
	   } 
	 & & \\ 
 

    \theutterance \stepcounter{utterance}  

    & & & \multicolumn{2}{p{0.3\linewidth}}{\cellcolor[rgb]{0.95,0.95,0.95}{%
	\makecell[{{p{\linewidth}}}]{% 
	  \tt {\tiny [GM$|$GM]}  
	 Strafpunkt: (2,4) ist nicht leer, sondern enthält 'T'.\\ \tt \\ \tt Ihr habt gemeinsam 4 von maximal 16 Strafpunkten angesammelt.\\ \tt Bitte versuche es noch mal! 
	  } 
	   } 
	   } 
	 & & \\ 
 

    \theutterance \stepcounter{utterance}  

    & & \multicolumn{4}{p{0.6\linewidth}}{\cellcolor[rgb]{0.9,0.9,0.9}{%
	\makecell[{{p{\linewidth}}}]{% 
	  \tt {\tiny [A$\langle$GM]}  
	 Strafpunkt: (2,4) ist nicht leer, sondern enthält 'T'.\\ \tt \\ \tt Ihr habt gemeinsam 4 von maximal 16 Strafpunkten angesammelt.\\ \tt Bitte versuche es noch mal! 
	  } 
	   } 
	   } 
	 & \\ 
 

    \theutterance \stepcounter{utterance}  

    & \multicolumn{4}{p{0.6\linewidth}}{\cellcolor[rgb]{0.8,1,0.9}{%
	\makecell[{{p{\linewidth}}}]{% 
	  \tt {\tiny [A$\rangle$GM]}  
	 VERSCHIEBE: D, (5,3) 
	  } 
	   } 
	   } 
	 & & \\ 
 

    \theutterance \stepcounter{utterance}  

    & & & \multicolumn{2}{p{0.3\linewidth}}{\cellcolor[rgb]{0.95,0.95,0.95}{%
	\makecell[{{p{\linewidth}}}]{% 
	  \tt {\tiny [GM$|$GM]}  
	 VERSCHIEBE: D, (5,3) 
	  } 
	   } 
	   } 
	 & & \\ 
 

    \theutterance \stepcounter{utterance}  

    & & & \multicolumn{2}{p{0.3\linewidth}}{\cellcolor[rgb]{0.95,0.95,0.95}{%
	\makecell[{{p{\linewidth}}}]{% 
	  \tt {\tiny [GM$|$GM]}  
	 Strafpunkt: (5,3) ist nicht leer, sondern enthält 'D'.\\ \tt \\ \tt Ihr habt gemeinsam 5 von maximal 16 Strafpunkten angesammelt.\\ \tt Bitte versuche es noch mal! 
	  } 
	   } 
	   } 
	 & & \\ 
 

    \theutterance \stepcounter{utterance}  

    & & \multicolumn{4}{p{0.6\linewidth}}{\cellcolor[rgb]{0.9,0.9,0.9}{%
	\makecell[{{p{\linewidth}}}]{% 
	  \tt {\tiny [A$\langle$GM]}  
	 Strafpunkt: (5,3) ist nicht leer, sondern enthält 'D'.\\ \tt \\ \tt Ihr habt gemeinsam 5 von maximal 16 Strafpunkten angesammelt.\\ \tt Bitte versuche es noch mal! 
	  } 
	   } 
	   } 
	 & \\ 
 

    \theutterance \stepcounter{utterance}  

    & \multicolumn{4}{p{0.6\linewidth}}{\cellcolor[rgb]{0.8,1,0.9}{%
	\makecell[{{p{\linewidth}}}]{% 
	  \tt {\tiny [A$\rangle$GM]}  
	 VERSCHIEBE: M, (3,4) 
	  } 
	   } 
	   } 
	 & & \\ 
 

    \theutterance \stepcounter{utterance}  

    & & & \multicolumn{2}{p{0.3\linewidth}}{\cellcolor[rgb]{0.95,0.95,0.95}{%
	\makecell[{{p{\linewidth}}}]{% 
	  \tt {\tiny [GM$|$GM]}  
	 VERSCHIEBE: M, (3,4) 
	  } 
	   } 
	   } 
	 & & \\ 
 

    \theutterance \stepcounter{utterance}  

    & & & \multicolumn{2}{p{0.3\linewidth}}{\cellcolor[rgb]{0.95,0.95,0.95}{%
	\makecell[{{p{\linewidth}}}]{% 
	  \tt {\tiny [GM$|$GM]}  
	 Strafpunkt: (3,4) ist nicht leer, sondern enthält 'M'.\\ \tt \\ \tt Ihr habt gemeinsam 6 von maximal 16 Strafpunkten angesammelt.\\ \tt Bitte versuche es noch mal! 
	  } 
	   } 
	   } 
	 & & \\ 
 

    \theutterance \stepcounter{utterance}  

    & & \multicolumn{4}{p{0.6\linewidth}}{\cellcolor[rgb]{0.9,0.9,0.9}{%
	\makecell[{{p{\linewidth}}}]{% 
	  \tt {\tiny [A$\langle$GM]}  
	 Strafpunkt: (3,4) ist nicht leer, sondern enthält 'M'.\\ \tt \\ \tt Ihr habt gemeinsam 6 von maximal 16 Strafpunkten angesammelt.\\ \tt Bitte versuche es noch mal! 
	  } 
	   } 
	   } 
	 & \\ 
 

    \theutterance \stepcounter{utterance}  

    & \multicolumn{4}{p{0.6\linewidth}}{\cellcolor[rgb]{0.8,1,0.9}{%
	\makecell[{{p{\linewidth}}}]{% 
	  \tt {\tiny [A$\rangle$GM]}  
	 SPRICH: Ich bin an allen Zielkoordinaten. Bitte arbeite du weiter auf unsere Zielpositionen hin und gib Bescheid, wenn du fertig bist, dann bestätige ich mit „fertig!“. 
	  } 
	   } 
	   } 
	 & & \\ 
 

    \theutterance \stepcounter{utterance}  

    & & & \multicolumn{2}{p{0.3\linewidth}}{\cellcolor[rgb]{0.95,0.95,0.95}{%
	\makecell[{{p{\linewidth}}}]{% 
	  \tt {\tiny [GM$|$GM]}  
	 SPRICH: Ich bin an allen Zielkoordinaten. Bitte arbeite du weiter auf unsere Zielpositionen hin und gib Bescheid, wenn du fertig bist, dann bestätige ich mit „fertig!“. 
	  } 
	   } 
	   } 
	 & & \\ 
 

    \theutterance \stepcounter{utterance}  

    & & \multicolumn{4}{p{0.6\linewidth}}{\cellcolor[rgb]{0.9,0.9,0.9}{%
	\makecell[{{p{\linewidth}}}]{% 
	  \tt {\tiny [GM$\rangle$B]}  
	 'M' erfolgreich nach (3,4) verschoben. Dein Raster sieht jetzt so aus:\\ \tt ```\\ \tt  1234567\\ \tt ╔═══════╗\\ \tt ║T◌P◌S◌◌║ 1\\ \tt ║◌◌◌D◌◌◌║ 2\\ \tt ║◌◌O◌◌┌─╢ 3\\ \tt ║◌◌M┌─┤◌║ 4\\ \tt ║◌◌◌│◌├─╢ 5\\ \tt ╟───┼─┘◌║ 6\\ \tt ║◌◌◌│A◌◌║ 7\\ \tt ╚═══╧═══╝\\ \tt ```\\ \tt Ihr seid gerade bei Runde 7 von maximal 28.\\ \tt Ihr habt gemeinsam 6 von maximal 16 Strafpunkten angesammelt.\\ \tt Dein Mitspieler hat dir diese Nachricht geschickt:\\ \tt \\ \tt \\ \tt "Ich bin an allen Zielkoordinaten. Bitte arbeite du weiter auf unsere Zielpositionen hin und gib Bescheid, wenn du fertig bist, dann bestätige ich mit „fertig!“."\\ \tt \\ \tt \\ \tt Wie lautet dein nächster Befehl? 
	  } 
	   } 
	   } 
	 & \\ 
 

    \theutterance \stepcounter{utterance}  

    & & & \multicolumn{4}{p{0.6\linewidth}}{\cellcolor[rgb]{1,0.85,0.72}{%
	\makecell[{{p{\linewidth}}}]{% 
	  \tt {\tiny [GM$\langle$B]}  
	 VERSCHIEBE: O, (3, 5) 
	  } 
	   } 
	   } 
	  \\ 
 

    \theutterance \stepcounter{utterance}  

    & & & \multicolumn{2}{p{0.3\linewidth}}{\cellcolor[rgb]{0.95,0.95,0.95}{%
	\makecell[{{p{\linewidth}}}]{% 
	  \tt {\tiny [GM$|$GM]}  
	 VERSCHIEBE: O, (3, 5) 
	  } 
	   } 
	   } 
	 & & \\ 
 

    \theutterance \stepcounter{utterance}  

    & & & \multicolumn{2}{p{0.3\linewidth}}{\cellcolor[rgb]{0.95,0.95,0.95}{%
	\makecell[{{p{\linewidth}}}]{% 
	  \tt {\tiny [GM$|$GM]}  
	 'O' erfolgreich nach (3,5) verschoben. Dein Raster sieht jetzt so aus:\\ \tt ```\\ \tt  1234567\\ \tt ╔═══════╗\\ \tt ║T◌P◌S◌◌║ 1\\ \tt ║◌◌◌D◌◌◌║ 2\\ \tt ║◌◌◌◌◌┌─╢ 3\\ \tt ║◌◌M┌─┤◌║ 4\\ \tt ║◌◌O│◌├─╢ 5\\ \tt ╟───┼─┘◌║ 6\\ \tt ║◌◌◌│A◌◌║ 7\\ \tt ╚═══╧═══╝\\ \tt ```\\ \tt  
	  } 
	   } 
	   } 
	 & & \\ 
 

    \theutterance \stepcounter{utterance}  

    & & \multicolumn{4}{p{0.6\linewidth}}{\cellcolor[rgb]{0.9,0.9,0.9}{%
	\makecell[{{p{\linewidth}}}]{% 
	  \tt {\tiny [A$\langle$GM]}  
	 Deine Nachricht wurde an deinen Mitspieler übermittelt.\\ \tt \\ \tt \\ \tt Ihr seid gerade bei Runde 7 von maximal 28.\\ \tt Ihr habt gemeinsam 6 von maximal 16 Strafpunkten angesammelt.\\ \tt Dein Mitspieler hat ein Objekt auf seinem Raster bewegt.\\ \tt \\ \tt \\ \tt Wie lautet dein nächster Befehl? 
	  } 
	   } 
	   } 
	 & \\ 
 

    \theutterance \stepcounter{utterance}  

    & \multicolumn{4}{p{0.6\linewidth}}{\cellcolor[rgb]{0.8,1,0.9}{%
	\makecell[{{p{\linewidth}}}]{% 
	  \tt {\tiny [A$\rangle$GM]}  
	 SPRICH: Ich warte. Sag bitte „fertig?“, wenn du deine Zielpositionen erreicht hast. 
	  } 
	   } 
	   } 
	 & & \\ 
 

    \theutterance \stepcounter{utterance}  

    & & & \multicolumn{2}{p{0.3\linewidth}}{\cellcolor[rgb]{0.95,0.95,0.95}{%
	\makecell[{{p{\linewidth}}}]{% 
	  \tt {\tiny [GM$|$GM]}  
	 SPRICH: Ich warte. Sag bitte „fertig?“, wenn du deine Zielpositionen erreicht hast. 
	  } 
	   } 
	   } 
	 & & \\ 
 

    \theutterance \stepcounter{utterance}  

    & & \multicolumn{4}{p{0.6\linewidth}}{\cellcolor[rgb]{0.9,0.9,0.9}{%
	\makecell[{{p{\linewidth}}}]{% 
	  \tt {\tiny [GM$\rangle$B]}  
	 'O' erfolgreich nach (3,5) verschoben. Dein Raster sieht jetzt so aus:\\ \tt ```\\ \tt  1234567\\ \tt ╔═══════╗\\ \tt ║T◌P◌S◌◌║ 1\\ \tt ║◌◌◌D◌◌◌║ 2\\ \tt ║◌◌◌◌◌┌─╢ 3\\ \tt ║◌◌M┌─┤◌║ 4\\ \tt ║◌◌O│◌├─╢ 5\\ \tt ╟───┼─┘◌║ 6\\ \tt ║◌◌◌│A◌◌║ 7\\ \tt ╚═══╧═══╝\\ \tt ```\\ \tt Ihr seid gerade bei Runde 8 von maximal 28.\\ \tt Ihr habt gemeinsam 6 von maximal 16 Strafpunkten angesammelt.\\ \tt Dein Mitspieler hat dir diese Nachricht geschickt:\\ \tt \\ \tt \\ \tt "Ich warte. Sag bitte „fertig?“, wenn du deine Zielpositionen erreicht hast."\\ \tt \\ \tt \\ \tt Wie lautet dein nächster Befehl? 
	  } 
	   } 
	   } 
	 & \\ 
 

    \theutterance \stepcounter{utterance}  

    & & & \multicolumn{4}{p{0.6\linewidth}}{\cellcolor[rgb]{1,0.85,0.72}{%
	\makecell[{{p{\linewidth}}}]{% 
	  \tt {\tiny [GM$\langle$B]}  
	 VERSCHIEBE: T, (2, 4) 
	  } 
	   } 
	   } 
	  \\ 
 

    \theutterance \stepcounter{utterance}  

    & & & \multicolumn{2}{p{0.3\linewidth}}{\cellcolor[rgb]{0.95,0.95,0.95}{%
	\makecell[{{p{\linewidth}}}]{% 
	  \tt {\tiny [GM$|$GM]}  
	 VERSCHIEBE: T, (2, 4) 
	  } 
	   } 
	   } 
	 & & \\ 
 

    \theutterance \stepcounter{utterance}  

    & & & \multicolumn{2}{p{0.3\linewidth}}{\cellcolor[rgb]{0.95,0.95,0.95}{%
	\makecell[{{p{\linewidth}}}]{% 
	  \tt {\tiny [GM$|$GM]}  
	 'T' erfolgreich nach (2,4) verschoben. Dein Raster sieht jetzt so aus:\\ \tt ```\\ \tt  1234567\\ \tt ╔═══════╗\\ \tt ║◌◌P◌S◌◌║ 1\\ \tt ║◌◌◌D◌◌◌║ 2\\ \tt ║◌◌◌◌◌┌─╢ 3\\ \tt ║◌TM┌─┤◌║ 4\\ \tt ║◌◌O│◌├─╢ 5\\ \tt ╟───┼─┘◌║ 6\\ \tt ║◌◌◌│A◌◌║ 7\\ \tt ╚═══╧═══╝\\ \tt ```\\ \tt  
	  } 
	   } 
	   } 
	 & & \\ 
 

    \theutterance \stepcounter{utterance}  

    & & \multicolumn{4}{p{0.6\linewidth}}{\cellcolor[rgb]{0.9,0.9,0.9}{%
	\makecell[{{p{\linewidth}}}]{% 
	  \tt {\tiny [A$\langle$GM]}  
	 Deine Nachricht wurde an deinen Mitspieler übermittelt.\\ \tt \\ \tt \\ \tt Ihr seid gerade bei Runde 8 von maximal 28.\\ \tt Ihr habt gemeinsam 6 von maximal 16 Strafpunkten angesammelt.\\ \tt Dein Mitspieler hat ein Objekt auf seinem Raster bewegt.\\ \tt \\ \tt \\ \tt Wie lautet dein nächster Befehl? 
	  } 
	   } 
	   } 
	 & \\ 
 

    \theutterance \stepcounter{utterance}  

    & \multicolumn{4}{p{0.6\linewidth}}{\cellcolor[rgb]{0.8,1,0.9}{%
	\makecell[{{p{\linewidth}}}]{% 
	  \tt {\tiny [A$\rangle$GM]}  
	 SPRICH: Status-Check: Wie weit bist du? Wenn du fertig bist, sag bitte „fertig?“. 
	  } 
	   } 
	   } 
	 & & \\ 
 

    \theutterance \stepcounter{utterance}  

    & & & \multicolumn{2}{p{0.3\linewidth}}{\cellcolor[rgb]{0.95,0.95,0.95}{%
	\makecell[{{p{\linewidth}}}]{% 
	  \tt {\tiny [GM$|$GM]}  
	 SPRICH: Status-Check: Wie weit bist du? Wenn du fertig bist, sag bitte „fertig?“. 
	  } 
	   } 
	   } 
	 & & \\ 
 

    \theutterance \stepcounter{utterance}  

    & & \multicolumn{4}{p{0.6\linewidth}}{\cellcolor[rgb]{0.9,0.9,0.9}{%
	\makecell[{{p{\linewidth}}}]{% 
	  \tt {\tiny [GM$\rangle$B]}  
	 'T' erfolgreich nach (2,4) verschoben. Dein Raster sieht jetzt so aus:\\ \tt ```\\ \tt  1234567\\ \tt ╔═══════╗\\ \tt ║◌◌P◌S◌◌║ 1\\ \tt ║◌◌◌D◌◌◌║ 2\\ \tt ║◌◌◌◌◌┌─╢ 3\\ \tt ║◌TM┌─┤◌║ 4\\ \tt ║◌◌O│◌├─╢ 5\\ \tt ╟───┼─┘◌║ 6\\ \tt ║◌◌◌│A◌◌║ 7\\ \tt ╚═══╧═══╝\\ \tt ```\\ \tt Ihr seid gerade bei Runde 9 von maximal 28.\\ \tt Ihr habt gemeinsam 6 von maximal 16 Strafpunkten angesammelt.\\ \tt Dein Mitspieler hat dir diese Nachricht geschickt:\\ \tt \\ \tt \\ \tt "Status-Check: Wie weit bist du? Wenn du fertig bist, sag bitte „fertig?“."\\ \tt \\ \tt \\ \tt Wie lautet dein nächster Befehl? 
	  } 
	   } 
	   } 
	 & \\ 
 

    \theutterance \stepcounter{utterance}  

    & & & \multicolumn{4}{p{0.6\linewidth}}{\cellcolor[rgb]{1,0.85,0.72}{%
	\makecell[{{p{\linewidth}}}]{% 
	  \tt {\tiny [GM$\langle$B]}  
	 VERSCHIEBE: S, (6, 7) 
	  } 
	   } 
	   } 
	  \\ 
 

    \theutterance \stepcounter{utterance}  

    & & & \multicolumn{2}{p{0.3\linewidth}}{\cellcolor[rgb]{0.95,0.95,0.95}{%
	\makecell[{{p{\linewidth}}}]{% 
	  \tt {\tiny [GM$|$GM]}  
	 VERSCHIEBE: S, (6, 7) 
	  } 
	   } 
	   } 
	 & & \\ 
 

    \theutterance \stepcounter{utterance}  

    & & & \multicolumn{2}{p{0.3\linewidth}}{\cellcolor[rgb]{0.95,0.95,0.95}{%
	\makecell[{{p{\linewidth}}}]{% 
	  \tt {\tiny [GM$|$GM]}  
	 'S' erfolgreich nach (6,7) verschoben. Dein Raster sieht jetzt so aus:\\ \tt ```\\ \tt  1234567\\ \tt ╔═══════╗\\ \tt ║◌◌P◌◌◌◌║ 1\\ \tt ║◌◌◌D◌◌◌║ 2\\ \tt ║◌◌◌◌◌┌─╢ 3\\ \tt ║◌TM┌─┤◌║ 4\\ \tt ║◌◌O│◌├─╢ 5\\ \tt ╟───┼─┘◌║ 6\\ \tt ║◌◌◌│AS◌║ 7\\ \tt ╚═══╧═══╝\\ \tt ```\\ \tt  
	  } 
	   } 
	   } 
	 & & \\ 
 

    \theutterance \stepcounter{utterance}  

    & & \multicolumn{4}{p{0.6\linewidth}}{\cellcolor[rgb]{0.9,0.9,0.9}{%
	\makecell[{{p{\linewidth}}}]{% 
	  \tt {\tiny [A$\langle$GM]}  
	 Deine Nachricht wurde an deinen Mitspieler übermittelt.\\ \tt \\ \tt \\ \tt Ihr seid gerade bei Runde 9 von maximal 28.\\ \tt Ihr habt gemeinsam 6 von maximal 16 Strafpunkten angesammelt.\\ \tt Dein Mitspieler hat ein Objekt auf seinem Raster bewegt.\\ \tt \\ \tt \\ \tt Wie lautet dein nächster Befehl? 
	  } 
	   } 
	   } 
	 & \\ 
 

    \theutterance \stepcounter{utterance}  

    & \multicolumn{4}{p{0.6\linewidth}}{\cellcolor[rgb]{0.8,1,0.9}{%
	\makecell[{{p{\linewidth}}}]{% 
	  \tt {\tiny [A$\rangle$GM]}  
	 SPRICH: Alles klar. Ich bleibe bereit. Melde dich mit „fertig?“, sobald du alle Ziele erreicht hast. 
	  } 
	   } 
	   } 
	 & & \\ 
 

    \theutterance \stepcounter{utterance}  

    & & & \multicolumn{2}{p{0.3\linewidth}}{\cellcolor[rgb]{0.95,0.95,0.95}{%
	\makecell[{{p{\linewidth}}}]{% 
	  \tt {\tiny [GM$|$GM]}  
	 SPRICH: Alles klar. Ich bleibe bereit. Melde dich mit „fertig?“, sobald du alle Ziele erreicht hast. 
	  } 
	   } 
	   } 
	 & & \\ 
 

    \theutterance \stepcounter{utterance}  

    & & \multicolumn{4}{p{0.6\linewidth}}{\cellcolor[rgb]{0.9,0.9,0.9}{%
	\makecell[{{p{\linewidth}}}]{% 
	  \tt {\tiny [GM$\rangle$B]}  
	 'S' erfolgreich nach (6,7) verschoben. Dein Raster sieht jetzt so aus:\\ \tt ```\\ \tt  1234567\\ \tt ╔═══════╗\\ \tt ║◌◌P◌◌◌◌║ 1\\ \tt ║◌◌◌D◌◌◌║ 2\\ \tt ║◌◌◌◌◌┌─╢ 3\\ \tt ║◌TM┌─┤◌║ 4\\ \tt ║◌◌O│◌├─╢ 5\\ \tt ╟───┼─┘◌║ 6\\ \tt ║◌◌◌│AS◌║ 7\\ \tt ╚═══╧═══╝\\ \tt ```\\ \tt Ihr seid gerade bei Runde 10 von maximal 28.\\ \tt Ihr habt gemeinsam 6 von maximal 16 Strafpunkten angesammelt.\\ \tt Dein Mitspieler hat dir diese Nachricht geschickt:\\ \tt \\ \tt \\ \tt "Alles klar. Ich bleibe bereit. Melde dich mit „fertig?“, sobald du alle Ziele erreicht hast."\\ \tt \\ \tt \\ \tt Wie lautet dein nächster Befehl? 
	  } 
	   } 
	   } 
	 & \\ 
 

    \theutterance \stepcounter{utterance}  

    & & & \multicolumn{4}{p{0.6\linewidth}}{\cellcolor[rgb]{1,0.85,0.72}{%
	\makecell[{{p{\linewidth}}}]{% 
	  \tt {\tiny [GM$\langle$B]}  
	 VERSCHIEBE: D, (5, 3) 
	  } 
	   } 
	   } 
	  \\ 
 

    \theutterance \stepcounter{utterance}  

    & & & \multicolumn{2}{p{0.3\linewidth}}{\cellcolor[rgb]{0.95,0.95,0.95}{%
	\makecell[{{p{\linewidth}}}]{% 
	  \tt {\tiny [GM$|$GM]}  
	 VERSCHIEBE: D, (5, 3) 
	  } 
	   } 
	   } 
	 & & \\ 
 

    \theutterance \stepcounter{utterance}  

    & & & \multicolumn{2}{p{0.3\linewidth}}{\cellcolor[rgb]{0.95,0.95,0.95}{%
	\makecell[{{p{\linewidth}}}]{% 
	  \tt {\tiny [GM$|$GM]}  
	 'D' erfolgreich nach (5,3) verschoben. Dein Raster sieht jetzt so aus:\\ \tt ```\\ \tt  1234567\\ \tt ╔═══════╗\\ \tt ║◌◌P◌◌◌◌║ 1\\ \tt ║◌◌◌◌◌◌◌║ 2\\ \tt ║◌◌◌◌D┌─╢ 3\\ \tt ║◌TM┌─┤◌║ 4\\ \tt ║◌◌O│◌├─╢ 5\\ \tt ╟───┼─┘◌║ 6\\ \tt ║◌◌◌│AS◌║ 7\\ \tt ╚═══╧═══╝\\ \tt ```\\ \tt  
	  } 
	   } 
	   } 
	 & & \\ 
 

    \theutterance \stepcounter{utterance}  

    & & \multicolumn{4}{p{0.6\linewidth}}{\cellcolor[rgb]{0.9,0.9,0.9}{%
	\makecell[{{p{\linewidth}}}]{% 
	  \tt {\tiny [A$\langle$GM]}  
	 Deine Nachricht wurde an deinen Mitspieler übermittelt.\\ \tt \\ \tt \\ \tt Ihr seid gerade bei Runde 10 von maximal 28.\\ \tt Ihr habt gemeinsam 6 von maximal 16 Strafpunkten angesammelt.\\ \tt Dein Mitspieler hat ein Objekt auf seinem Raster bewegt.\\ \tt \\ \tt \\ \tt Wie lautet dein nächster Befehl? 
	  } 
	   } 
	   } 
	 & \\ 
 

    \theutterance \stepcounter{utterance}  

    & \multicolumn{4}{p{0.6\linewidth}}{\cellcolor[rgb]{0.8,1,0.9}{%
	\makecell[{{p{\linewidth}}}]{% 
	  \tt {\tiny [A$\rangle$GM]}  
	 SPRICH: fertig? 
	  } 
	   } 
	   } 
	 & & \\ 
 

    \theutterance \stepcounter{utterance}  

    & & & \multicolumn{2}{p{0.3\linewidth}}{\cellcolor[rgb]{0.95,0.95,0.95}{%
	\makecell[{{p{\linewidth}}}]{% 
	  \tt {\tiny [GM$|$GM]}  
	 SPRICH: fertig? 
	  } 
	   } 
	   } 
	 & & \\ 
 

    \theutterance \stepcounter{utterance}  

    & & \multicolumn{4}{p{0.6\linewidth}}{\cellcolor[rgb]{0.9,0.9,0.9}{%
	\makecell[{{p{\linewidth}}}]{% 
	  \tt {\tiny [GM$\rangle$B]}  
	 'D' erfolgreich nach (5,3) verschoben. Dein Raster sieht jetzt so aus:\\ \tt ```\\ \tt  1234567\\ \tt ╔═══════╗\\ \tt ║◌◌P◌◌◌◌║ 1\\ \tt ║◌◌◌◌◌◌◌║ 2\\ \tt ║◌◌◌◌D┌─╢ 3\\ \tt ║◌TM┌─┤◌║ 4\\ \tt ║◌◌O│◌├─╢ 5\\ \tt ╟───┼─┘◌║ 6\\ \tt ║◌◌◌│AS◌║ 7\\ \tt ╚═══╧═══╝\\ \tt ```\\ \tt Ihr seid gerade bei Runde 11 von maximal 28.\\ \tt Ihr habt gemeinsam 6 von maximal 16 Strafpunkten angesammelt.\\ \tt Dein Mitspieler hat dir diese Nachricht geschickt:\\ \tt \\ \tt \\ \tt "fertig?"\\ \tt \\ \tt \\ \tt Wie lautet dein nächster Befehl? 
	  } 
	   } 
	   } 
	 & \\ 
 

    \theutterance \stepcounter{utterance}  

    & & & \multicolumn{4}{p{0.6\linewidth}}{\cellcolor[rgb]{1,0.85,0.72}{%
	\makecell[{{p{\linewidth}}}]{% 
	  \tt {\tiny [GM$\langle$B]}  
	 SPRICH: fertig! 
	  } 
	   } 
	   } 
	  \\ 
 

    \theutterance \stepcounter{utterance}  

    & & & \multicolumn{2}{p{0.3\linewidth}}{\cellcolor[rgb]{0.95,0.95,0.95}{%
	\makecell[{{p{\linewidth}}}]{% 
	  \tt {\tiny [GM$|$GM]}  
	 SPRICH: fertig! 
	  } 
	   } 
	   } 
	 & & \\ 
 

    \theutterance \stepcounter{utterance}  

    & & & \multicolumn{2}{p{0.3\linewidth}}{\cellcolor[rgb]{0.95,0.95,0.95}{%
	\makecell[{{p{\linewidth}}}]{% 
	  \tt {\tiny [GM$|$GM]}  
	 true 
	  } 
	   } 
	   } 
	 & & \\ 
 

    \theutterance \stepcounter{utterance}  

    & & & \multicolumn{2}{p{0.3\linewidth}}{\cellcolor[rgb]{0.95,0.95,0.95}{%
	\makecell[{{p{\linewidth}}}]{% 
	  \tt {\tiny [GM$|$GM]}  
	 * success: True\\ \tt * lose: False\\ \tt * aborted: False\\ \tt -------\\ \tt * Shifts: 7.00\\ \tt * Max Shifts: 12.00\\ \tt * Min Shifts: 6.00\\ \tt * End Distance Sum: 0.00\\ \tt * Init Distance Sum: 18.90\\ \tt * Expected Distance Sum: 29.33\\ \tt * Penalties: 6.00\\ \tt * Max Penalties: 16.00\\ \tt * Rounds: 11.00\\ \tt * Max Rounds: 28.00\\ \tt * Object Count: 7.00\\ \tt  
	  } 
	   } 
	   } 
	 & & \\ 
 

    \theutterance \stepcounter{utterance}  

    & & & \multicolumn{2}{p{0.3\linewidth}}{\cellcolor[rgb]{0.95,0.95,0.95}{%
	\makecell[{{p{\linewidth}}}]{% 
	  \tt {\tiny [GM$|$GM]}  
	 * Main Score: 45.52\\ \tt \\ \tt -------\\ \tt * Distance Score: 1.00\\ \tt * Consistency Score: 0.86\\ \tt * Coverage Score: 0.38\\ \tt * Penalty Score: 0.73\\ \tt * Alternative Penalty Score: 0.85\\ \tt * Alternative Main Score: 84.62\\ \tt \\ \tt -------\\ \tt * Shifts: 7.00\\ \tt * Max Shifts: 12.00\\ \tt * Min Shifts: 6.00\\ \tt * End Distance Sum: 0.00\\ \tt * Init Distance Sum: 18.90\\ \tt * Expected Distance Sum: 29.33\\ \tt * Penalties: 6.00\\ \tt * Max Penalties: 16.00\\ \tt * Rounds: 11.00\\ \tt * Max Rounds: 28.00\\ \tt * Object Count: 7.00\\ \tt  
	  } 
	   } 
	   } 
	 & & \\ 
 

\end{supertabular}
}

\end{document}
