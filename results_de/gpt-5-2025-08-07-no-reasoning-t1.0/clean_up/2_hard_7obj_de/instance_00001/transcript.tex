
\documentclass{article}
\usepackage{colortbl}
\usepackage{makecell}
\usepackage{multirow}
\usepackage{supertabular}

\begin{document}

\newcounter{utterance}

\centering \large Interaction Transcript for game `clean\_up', experiment `2\_hard\_7obj\_de', episode 1 with gpt{-}5{-}2025{-}08{-}07{-}no{-}reasoning{-}t1.0.
\vspace{24pt}

{ \footnotesize  \setcounter{utterance}{1}
\setlength{\tabcolsep}{0pt}
\begin{supertabular}{c@{$\;$}|p{.15\linewidth}@{}p{.15\linewidth}p{.15\linewidth}p{.15\linewidth}p{.15\linewidth}p{.15\linewidth}}
    \# & $\;$A & \multicolumn{4}{c}{Game Master} & $\;\:$B\\
    \hline

    \theutterance \stepcounter{utterance}  
    & & \multicolumn{4}{p{0.6\linewidth}}{
        \cellcolor[rgb]{0.9,0.9,0.9}{
            \makecell[{{p{\linewidth}}}]{
                \texttt{\tiny{[P1$\langle$GM]}}
                \texttt{Ich bin der Spielleiter, und du spielst ein kooperatives Spiel mit dem folgenden Raster als Spielbrett:} \\
\\ 
\texttt{```} \\
\texttt{ 1234567} \\
\texttt{╔═══╤═╤═╗} \\
\texttt{║◌S◌│◌│A║ 1} \\
\texttt{╟─┬─┼─┴─╢ 2} \\
\texttt{║◌│◌│◌◌◌║ 3} \\
\texttt{║P├─┤D┌─╢ 4} \\
\texttt{╟─┘◌│◌│M║ 5} \\
\texttt{║◌◌◌└┬┴─╢ 6} \\
\texttt{║◌OT◌│◌◌║ 7} \\
\texttt{╚════╧══╝} \\
\\ 
\texttt{```} \\
\\ 
\texttt{* Am oberen Rand sind die x{-}Koordinaten angegeben, die nach rechts hin zunehmen, und am rechten Rand die y{-}Koordinaten, die nach unten zunehmen.} \\
\texttt{* Die folgenden Objekte sind zufällig auf deinem Raster platziert: 'P', 'O', 'T', 'S', 'D', 'A', 'M'.} \\
\\ 
\\ 
\texttt{Dein Mitspieler sieht eine Variation des Spielbretts, auf dem dieselben Objekte an anderen zufälligen Positionen platziert sind. Du kannst das Spielbrett des anderen Spielers nicht sehen, und er kann deins nicht sehen.} \\
\\ 
\\ 
\texttt{**Ziel des Spiels:**} \\
\\ 
\texttt{Beide Spieler müssen die Objekte auf ihrem jeweiligen Spielbrett so bewegen, dass identische Objekte am Ende an denselben Koordinaten liegen. Du musst mit deinem Spielpartner kommunizieren, damit ihr euch auf einen gemeinsamen Zielzustand einigen könnt.} \\
\\ 
\\ 
\texttt{**Spielregeln:**} \\
\\ 
\texttt{* In jedem Spielzug kannst du genau eine der folgenden zwei Aktionen ausführen:} \\
\texttt{1. `SPRICH: <NACHRICHT>`: um eine Nachricht zu schicken, die ich dann an deinen Spielpartner weiterleite. Als Nachricht gilt alles bis zum nächsten Zeilenumbruch.} \\
\texttt{2. `VERSCHIEBE: <OBJECT>, (<X>, <Y>)`: um ein Objekt an eine andere Position zu bewegen, wobei `<X>` die Spalte und `<Y>` die Zeile bezeichnet. Ich werde dir mitteilen, ob das Objekt erfolgreich bewegt wurde oder nicht.} \\
\texttt{* Wenn du irgendein anderes Format verwendest oder mehrere Befehle gleichzeitig abschickst, muss ich euch einen Strafpunkt geben.} \\
\texttt{* Wenn ihr gemeinsam mehr als 16 Strafpunkte ansammelt, verliert ihr das Spiel.} \\
\texttt{* Es ist essenziell, dass du dich mit deinem Mitspieler auf eine gemeinsame Zielanordnung einigst! Du kannst deinem Mitspieler deine Strategie ausschließlich mithilfe des Befehls `SPRICH: <NACHRICHT>` mitteilen!} \\
\\ 
\\ 
\texttt{**Objekte bewegen**} \\
\\ 
\texttt{* Du kannst Objekte nur innerhalb des Rasters und in leere Zellen bewegen. Leere Zellen erkennst am Zeichen '◌'.} \\
\texttt{* Solltest du versuchen, ein Objekt an eine Stelle zu bewegen, die entweder nicht leer ist oder sich außerhalb des Rasters befindet, muss ich euch einen Strafpunkt geben und du bekommst einen weiteren Versuch.} \\
\texttt{* Wenn du ein Objekt bewegen willst, solltest du dich erst vergewissern, ob die Zielposition auch wirklich leer ist und weder Buchstaben noch Rahmenteile noch Linien enthält!} \\
\\ 
\\ 
\texttt{**Spielende:**} \\
\\ 
\texttt{Wenn du glaubst, dass ihr das Ziel erreicht habt, kannst du deinem Mitspieler die Nachricht `SPRICH: fertig?` schicken. Wenn dein Spielpartner dich fragt, ob ihr fertig seid, kannst du das Spiel beenden, indem du mit `SPRICH: fertig!` antwortest.} \\
\\ 
\texttt{Ihr gewinnt gemeinsam, wenn ihr das Spiel innerhalb von \#max\_rounds Runden beendet. Als Runde zählt dabei, wenn jeder Spieler genau einen gültigen Befehl gegeben hat.} \\
\\ 
\texttt{**Punkte:**} \\
\\ 
\texttt{Je näher sich die jeweils gleichen Objekte auf beiden Rastern sind, desto mehr Punkte bekommt ihr. Für Strafpunkte gibt es Punktabzug. Könnt ihr den Rekord brechen?} \\
\\ 
\texttt{Schicke eine Nachricht an deinen Mitspieler um das Spiel zu starten!} \\
            }
        }
    }
    & \\ \\

    \theutterance \stepcounter{utterance}  
    & \multicolumn{4}{p{0.6\linewidth}}{
        \cellcolor[rgb]{0.9,0.9,0.9}{
            \makecell[{{p{\linewidth}}}]{
                \texttt{\tiny{[P1$\rangle$GM]}}
                \texttt{SPRICH: Hallo! Lass uns eine gemeinsame Zielanordnung festlegen. Vorschlag: Ordnen wir die Buchstaben alphabetisch in einer Linie an: A bei (1,1), D bei (2,1), M bei (3,1), O bei (4,1), P bei (5,1), S bei (6,1), T bei (7,1). Wenn du einverstanden bist, bestätige bitte, dann beginne ich mit Verschiebungen. Wenn nicht, schlage eine alternative, einfache Ordnung vor (z. B. alphabetisch in Spalte 1 von oben nach unten).} \\
            }
        }
    }
    & & \\ \\

    \theutterance \stepcounter{utterance}  
    & & & \multicolumn{2}{p{0.3\linewidth}}{
        \cellcolor[rgb]{0.9,0.9,0.9}{
            \makecell[{{p{\linewidth}}}]{
                \texttt{\tiny{[GM$|$GM]}}
                \texttt{SPRICH: Hallo! Lass uns eine gemeinsame Zielanordnung festlegen. Vorschlag: Ordnen wir die Buchstaben alphabetisch in einer Linie an: A bei (1,1), D bei (2,1), M bei (3,1), O bei (4,1), P bei (5,1), S bei (6,1), T bei (7,1). Wenn du einverstanden bist, bestätige bitte, dann beginne ich mit Verschiebungen. Wenn nicht, schlage eine alternative, einfache Ordnung vor (z. B. alphabetisch in Spalte 1 von oben nach unten).} \\
            }
        }
    }
    & & \\ \\

    \theutterance \stepcounter{utterance}  
    & & \multicolumn{4}{p{0.6\linewidth}}{
        \cellcolor[rgb]{0.9,0.9,0.9}{
            \makecell[{{p{\linewidth}}}]{
                \texttt{\tiny{[GM$\rangle$P2]}}
                \texttt{Ich bin der Spielleiter, und du spielst ein kooperatives Spiel mit dem folgenden Raster als Spielbrett:} \\
\\ 
\texttt{```} \\
\texttt{ 1234567} \\
\texttt{╔═══╤═╤═╗} \\
\texttt{║P◌A│◌│◌║ 1} \\
\texttt{╟─┬─┼─┴─╢ 2} \\
\texttt{║◌│D│◌◌◌║ 3} \\
\texttt{║◌├─┤◌┌─╢ 4} \\
\texttt{╟─┘◌│◌│O║ 5} \\
\texttt{║◌◌◌└┬┴─╢ 6} \\
\texttt{║◌S◌T│◌M║ 7} \\
\texttt{╚════╧══╝} \\
\\ 
\texttt{```} \\
\\ 
\texttt{* Am oberen Rand sind die x{-}Koordinaten angegeben, die nach rechts hin zunehmen, und am rechten Rand die y{-}Koordinaten, die nach unten zunehmen.} \\
\texttt{* Die folgenden Objekte sind zufällig auf deinem Raster platziert: 'P', 'O', 'T', 'S', 'D', 'A', 'M'.} \\
\\ 
\\ 
\texttt{Dein Mitspieler sieht eine Variation des Spielbretts, auf dem dieselben Objekte an anderen zufälligen Positionen platziert sind. Du kannst das Spielbrett des anderen Spielers nicht sehen, und er kann deins nicht sehen.} \\
\\ 
\\ 
\texttt{**Ziel des Spiels:**} \\
\\ 
\texttt{Beide Spieler müssen die Objekte auf ihrem jeweiligen Spielbrett so bewegen, dass identische Objekte am Ende an denselben Koordinaten liegen. Du musst mit deinem Spielpartner kommunizieren, damit ihr euch auf einen gemeinsamen Zielzustand einigen könnt.} \\
\\ 
\\ 
\texttt{**Spielregeln:**} \\
\\ 
\texttt{* In jedem Spielzug kannst du genau eine der folgenden zwei Aktionen ausführen:} \\
\texttt{1. `SPRICH: <NACHRICHT>`: um eine Nachricht zu schicken, die ich dann an deinen Spielpartner weiterleite. Als Nachricht gilt alles bis zum nächsten Zeilenumbruch.} \\
\texttt{2. `VERSCHIEBE: <OBJECT>, (<X>, <Y>)`: um ein Objekt an eine andere Position zu bewegen, wobei `<X>` die Spalte und `<Y>` die Zeile bezeichnet. Ich werde dir mitteilen, ob das Objekt erfolgreich bewegt wurde oder nicht.} \\
\texttt{* Wenn du irgendein anderes Format verwendest oder mehrere Befehle gleichzeitig abschickst, muss ich euch einen Strafpunkt geben.} \\
\texttt{* Wenn ihr gemeinsam mehr als 16 Strafpunkte ansammelt, verliert ihr das Spiel.} \\
\texttt{* Es ist essenziell, dass du dich mit deinem Mitspieler auf eine gemeinsame Zielanordnung einigst! Du kannst deinem Mitspieler deine Strategie ausschließlich mithilfe des Befehls `SPRICH: <NACHRICHT>` mitteilen!} \\
\\ 
\\ 
\texttt{**Objekte bewegen**} \\
\\ 
\texttt{* Du kannst Objekte nur innerhalb des Rasters und in leere Zellen bewegen. Leere Zellen erkennst am Zeichen '◌'.} \\
\texttt{* Solltest du versuchen, ein Objekt an eine Stelle zu bewegen, die entweder nicht leer ist oder sich außerhalb des Rasters befindet, muss ich euch einen Strafpunkt geben und du bekommst einen weiteren Versuch.} \\
\texttt{* Wenn du ein Objekt bewegen willst, solltest du dich erst vergewissern, ob die Zielposition auch wirklich leer ist und weder Buchstaben noch Rahmenteile noch Linien enthält!} \\
\\ 
\\ 
\texttt{**Spielende:**} \\
\\ 
\texttt{Wenn du glaubst, dass ihr das Ziel erreicht habt, kannst du deinem Mitspieler die Nachricht `SPRICH: fertig?` schicken. Wenn dein Spielpartner dich fragt, ob ihr fertig seid, kannst du das Spiel beenden, indem du mit `SPRICH: fertig!` antwortest.} \\
\\ 
\texttt{Ihr gewinnt gemeinsam, wenn ihr das Spiel innerhalb von \#max\_rounds Runden beendet. Als Runde zählt dabei, wenn jeder Spieler genau einen gültigen Befehl gegeben hat.} \\
\\ 
\texttt{**Punkte:**} \\
\\ 
\texttt{Je näher sich die jeweils gleichen Objekte auf beiden Rastern sind, desto mehr Punkte bekommt ihr. Für Strafpunkte gibt es Punktabzug. Könnt ihr den Rekord brechen?} \\
\\ 
\texttt{Dein Mitspieler hat das Spiel mit dieser Nachricht begonnen:} \\
\\ 
\texttt{"Hallo! Lass uns eine gemeinsame Zielanordnung festlegen. Vorschlag: Ordnen wir die Buchstaben alphabetisch in einer Linie an: A bei (1,1), D bei (2,1), M bei (3,1), O bei (4,1), P bei (5,1), S bei (6,1), T bei (7,1). Wenn du einverstanden bist, bestätige bitte, dann beginne ich mit Verschiebungen. Wenn nicht, schlage eine alternative, einfache Ordnung vor (z. B. alphabetisch in Spalte 1 von oben nach unten)."} \\
\\ 
\texttt{Wie lautet dein erster Befehl?} \\
            }
        }
    }
    & \\ \\

    \theutterance \stepcounter{utterance}  
    & & & \multicolumn{4}{p{0.6\linewidth}}{
        \cellcolor[rgb]{0.9,0.9,0.9}{
            \makecell[{{p{\linewidth}}}]{
                \texttt{\tiny{[GM$\langle$P2]}}
                \texttt{SPRICH: Einverstanden mit alphabetischer Linie in Zeile 1: A(1,1), D(2,1), M(3,1), O(4,1), P(5,1), S(6,1), T(7,1). Ich beginne bei mir mit Verschiebungen, sag Bescheid, wenn du startest.} \\
            }
        }
    }
     \\ \\

    \theutterance \stepcounter{utterance}  
    & & & \multicolumn{2}{p{0.3\linewidth}}{
        \cellcolor[rgb]{0.9,0.9,0.9}{
            \makecell[{{p{\linewidth}}}]{
                \texttt{\tiny{[GM$|$GM]}}
                \texttt{SPRICH: Einverstanden mit alphabetischer Linie in Zeile 1: A(1,1), D(2,1), M(3,1), O(4,1), P(5,1), S(6,1), T(7,1). Ich beginne bei mir mit Verschiebungen, sag Bescheid, wenn du startest.} \\
            }
        }
    }
    & & \\ \\

    \theutterance \stepcounter{utterance}  
    & & \multicolumn{4}{p{0.6\linewidth}}{
        \cellcolor[rgb]{0.9,0.9,0.9}{
            \makecell[{{p{\linewidth}}}]{
                \texttt{\tiny{[P1$\langle$GM]}}
                \texttt{Deine Nachricht wurde an deinen Mitspieler übermittelt.} \\
\\ 
\\ 
\texttt{Ihr seid gerade bei Runde 1 von maximal 28.} \\
\texttt{Ihr habt gemeinsam 0 von maximal 16 Strafpunkten angesammelt.} \\
\texttt{Dein Mitspieler hat dir diese Nachricht geschickt:} \\
\\ 
\\ 
\texttt{"Einverstanden mit alphabetischer Linie in Zeile 1: A(1,1), D(2,1), M(3,1), O(4,1), P(5,1), S(6,1), T(7,1). Ich beginne bei mir mit Verschiebungen, sag Bescheid, wenn du startest."} \\
\\ 
\\ 
\texttt{Wie lautet dein nächster Befehl?} \\
            }
        }
    }
    & \\ \\

    \theutterance \stepcounter{utterance}  
    & \multicolumn{4}{p{0.6\linewidth}}{
        \cellcolor[rgb]{0.9,0.9,0.9}{
            \makecell[{{p{\linewidth}}}]{
                \texttt{\tiny{[P1$\rangle$GM]}}
                \texttt{VERSCHIEBE: A, (7,1)} \\
            }
        }
    }
    & & \\ \\

    \theutterance \stepcounter{utterance}  
    & & & \multicolumn{2}{p{0.3\linewidth}}{
        \cellcolor[rgb]{0.9,0.9,0.9}{
            \makecell[{{p{\linewidth}}}]{
                \texttt{\tiny{[GM$|$GM]}}
                \texttt{VERSCHIEBE: A, (7,1)} \\
            }
        }
    }
    & & \\ \\

    \theutterance \stepcounter{utterance}  
    & & & \multicolumn{2}{p{0.3\linewidth}}{
        \cellcolor[rgb]{0.9,0.9,0.9}{
            \makecell[{{p{\linewidth}}}]{
                \texttt{\tiny{[GM$|$GM]}}
                \texttt{Strafpunkt: (7,1) ist nicht leer, sondern enthält 'A'.} \\
\\ 
\texttt{Ihr habt gemeinsam 1 von maximal 16 Strafpunkten angesammelt.} \\
\texttt{Bitte versuche es noch mal!} \\
            }
        }
    }
    & & \\ \\

    \theutterance \stepcounter{utterance}  
    & & \multicolumn{4}{p{0.6\linewidth}}{
        \cellcolor[rgb]{0.9,0.9,0.9}{
            \makecell[{{p{\linewidth}}}]{
                \texttt{\tiny{[P1$\langle$GM]}}
                \texttt{Strafpunkt: (7,1) ist nicht leer, sondern enthält 'A'.} \\
\\ 
\texttt{Ihr habt gemeinsam 1 von maximal 16 Strafpunkten angesammelt.} \\
\texttt{Bitte versuche es noch mal!} \\
            }
        }
    }
    & \\ \\

    \theutterance \stepcounter{utterance}  
    & \multicolumn{4}{p{0.6\linewidth}}{
        \cellcolor[rgb]{0.9,0.9,0.9}{
            \makecell[{{p{\linewidth}}}]{
                \texttt{\tiny{[P1$\rangle$GM]}}
                \texttt{SPRICH: Danke, ich starte jetzt. Mein aktuelles Board in Zeile 1 ist: (2,1)=S und (6,1)=A, Rest leer. Ziel laut Plan: A(1,1), D(2,1), M(3,1), O(4,1), P(5,1), S(6,1), T(7,1). Ich verschiebe als Nächstes A nach (1,1), dann S bleibt auf (6,1). Sag mir bitte, falls du Kollisionen vermeiden musst.} \\
            }
        }
    }
    & & \\ \\

    \theutterance \stepcounter{utterance}  
    & & & \multicolumn{2}{p{0.3\linewidth}}{
        \cellcolor[rgb]{0.9,0.9,0.9}{
            \makecell[{{p{\linewidth}}}]{
                \texttt{\tiny{[GM$|$GM]}}
                \texttt{SPRICH: Danke, ich starte jetzt. Mein aktuelles Board in Zeile 1 ist: (2,1)=S und (6,1)=A, Rest leer. Ziel laut Plan: A(1,1), D(2,1), M(3,1), O(4,1), P(5,1), S(6,1), T(7,1). Ich verschiebe als Nächstes A nach (1,1), dann S bleibt auf (6,1). Sag mir bitte, falls du Kollisionen vermeiden musst.} \\
            }
        }
    }
    & & \\ \\

    \theutterance \stepcounter{utterance}  
    & & \multicolumn{4}{p{0.6\linewidth}}{
        \cellcolor[rgb]{0.9,0.9,0.9}{
            \makecell[{{p{\linewidth}}}]{
                \texttt{\tiny{[GM$\rangle$P2]}}
                \texttt{Deine Nachricht wurde an deinen Mitspieler übermittelt.} \\
\\ 
\\ 
\texttt{Ihr seid gerade bei Runde 2 von maximal 28.} \\
\texttt{Ihr habt gemeinsam 1 von maximal 16 Strafpunkten angesammelt.} \\
\texttt{Dein Mitspieler hat dir diese Nachricht geschickt:} \\
\\ 
\\ 
\texttt{"Danke, ich starte jetzt. Mein aktuelles Board in Zeile 1 ist: (2,1)=S und (6,1)=A, Rest leer. Ziel laut Plan: A(1,1), D(2,1), M(3,1), O(4,1), P(5,1), S(6,1), T(7,1). Ich verschiebe als Nächstes A nach (1,1), dann S bleibt auf (6,1). Sag mir bitte, falls du Kollisionen vermeiden musst."} \\
\\ 
\\ 
\texttt{Wie lautet dein nächster Befehl?} \\
            }
        }
    }
    & \\ \\

    \theutterance \stepcounter{utterance}  
    & & & \multicolumn{4}{p{0.6\linewidth}}{
        \cellcolor[rgb]{0.9,0.9,0.9}{
            \makecell[{{p{\linewidth}}}]{
                \texttt{\tiny{[GM$\langle$P2]}}
                \texttt{SPRICH: Danke! Ich starte auch. Mein aktueller Stand in Zeile 1: P(1,1), A(3,1), sonst leer. Ich verschiebe als Nächstes P nach (5,1) und danach A nach (1,1), sofern frei. Bitte sag Bescheid, wenn (5,1) oder (1,1) bei dir kurzzeitig belegt sein müssen.} \\
            }
        }
    }
     \\ \\

    \theutterance \stepcounter{utterance}  
    & & & \multicolumn{2}{p{0.3\linewidth}}{
        \cellcolor[rgb]{0.9,0.9,0.9}{
            \makecell[{{p{\linewidth}}}]{
                \texttt{\tiny{[GM$|$GM]}}
                \texttt{SPRICH: Danke! Ich starte auch. Mein aktueller Stand in Zeile 1: P(1,1), A(3,1), sonst leer. Ich verschiebe als Nächstes P nach (5,1) und danach A nach (1,1), sofern frei. Bitte sag Bescheid, wenn (5,1) oder (1,1) bei dir kurzzeitig belegt sein müssen.} \\
            }
        }
    }
    & & \\ \\

    \theutterance \stepcounter{utterance}  
    & & \multicolumn{4}{p{0.6\linewidth}}{
        \cellcolor[rgb]{0.9,0.9,0.9}{
            \makecell[{{p{\linewidth}}}]{
                \texttt{\tiny{[P1$\langle$GM]}}
                \texttt{Deine Nachricht wurde an deinen Mitspieler übermittelt.} \\
\\ 
\\ 
\texttt{Ihr seid gerade bei Runde 2 von maximal 28.} \\
\texttt{Ihr habt gemeinsam 1 von maximal 16 Strafpunkten angesammelt.} \\
\texttt{Dein Mitspieler hat dir diese Nachricht geschickt:} \\
\\ 
\\ 
\texttt{"Danke! Ich starte auch. Mein aktueller Stand in Zeile 1: P(1,1), A(3,1), sonst leer. Ich verschiebe als Nächstes P nach (5,1) und danach A nach (1,1), sofern frei. Bitte sag Bescheid, wenn (5,1) oder (1,1) bei dir kurzzeitig belegt sein müssen."} \\
\\ 
\\ 
\texttt{Wie lautet dein nächster Befehl?} \\
            }
        }
    }
    & \\ \\

    \theutterance \stepcounter{utterance}  
    & \multicolumn{4}{p{0.6\linewidth}}{
        \cellcolor[rgb]{0.9,0.9,0.9}{
            \makecell[{{p{\linewidth}}}]{
                \texttt{\tiny{[P1$\rangle$GM]}}
                \texttt{VERSCHIEBE: A, (1,1)} \\
            }
        }
    }
    & & \\ \\

    \theutterance \stepcounter{utterance}  
    & & & \multicolumn{2}{p{0.3\linewidth}}{
        \cellcolor[rgb]{0.9,0.9,0.9}{
            \makecell[{{p{\linewidth}}}]{
                \texttt{\tiny{[GM$|$GM]}}
                \texttt{VERSCHIEBE: A, (1,1)} \\
            }
        }
    }
    & & \\ \\

    \theutterance \stepcounter{utterance}  
    & & & \multicolumn{2}{p{0.3\linewidth}}{
        \cellcolor[rgb]{0.9,0.9,0.9}{
            \makecell[{{p{\linewidth}}}]{
                \texttt{\tiny{[GM$|$GM]}}
                \texttt{'A' erfolgreich nach (1,1) verschoben. Dein Raster sieht jetzt so aus:} \\
\texttt{```} \\
\texttt{ 1234567} \\
\texttt{╔═══╤═╤═╗} \\
\texttt{║AS◌│◌│◌║ 1} \\
\texttt{╟─┬─┼─┴─╢ 2} \\
\texttt{║◌│◌│◌◌◌║ 3} \\
\texttt{║P├─┤D┌─╢ 4} \\
\texttt{╟─┘◌│◌│M║ 5} \\
\texttt{║◌◌◌└┬┴─╢ 6} \\
\texttt{║◌OT◌│◌◌║ 7} \\
\texttt{╚════╧══╝} \\
\texttt{```} \\
            }
        }
    }
    & & \\ \\

    \theutterance \stepcounter{utterance}  
    & & \multicolumn{4}{p{0.6\linewidth}}{
        \cellcolor[rgb]{0.9,0.9,0.9}{
            \makecell[{{p{\linewidth}}}]{
                \texttt{\tiny{[GM$\rangle$P2]}}
                \texttt{Deine Nachricht wurde an deinen Mitspieler übermittelt.} \\
\\ 
\\ 
\texttt{Ihr seid gerade bei Runde 3 von maximal 28.} \\
\texttt{Ihr habt gemeinsam 1 von maximal 16 Strafpunkten angesammelt.} \\
\texttt{Dein Mitspieler hat ein Objekt auf seinem Raster bewegt.} \\
\\ 
\\ 
\texttt{Wie lautet dein nächster Befehl?} \\
            }
        }
    }
    & \\ \\

    \theutterance \stepcounter{utterance}  
    & & & \multicolumn{4}{p{0.6\linewidth}}{
        \cellcolor[rgb]{0.9,0.9,0.9}{
            \makecell[{{p{\linewidth}}}]{
                \texttt{\tiny{[GM$\langle$P2]}}
                \texttt{VERSCHIEBE: P, (5, 1)} \\
            }
        }
    }
     \\ \\

    \theutterance \stepcounter{utterance}  
    & & & \multicolumn{2}{p{0.3\linewidth}}{
        \cellcolor[rgb]{0.9,0.9,0.9}{
            \makecell[{{p{\linewidth}}}]{
                \texttt{\tiny{[GM$|$GM]}}
                \texttt{VERSCHIEBE: P, (5, 1)} \\
            }
        }
    }
    & & \\ \\

    \theutterance \stepcounter{utterance}  
    & & & \multicolumn{2}{p{0.3\linewidth}}{
        \cellcolor[rgb]{0.9,0.9,0.9}{
            \makecell[{{p{\linewidth}}}]{
                \texttt{\tiny{[GM$|$GM]}}
                \texttt{'P' erfolgreich nach (5,1) verschoben. Dein Raster sieht jetzt so aus:} \\
\texttt{```} \\
\texttt{ 1234567} \\
\texttt{╔═══╤═╤═╗} \\
\texttt{║◌◌A│P│◌║ 1} \\
\texttt{╟─┬─┼─┴─╢ 2} \\
\texttt{║◌│D│◌◌◌║ 3} \\
\texttt{║◌├─┤◌┌─╢ 4} \\
\texttt{╟─┘◌│◌│O║ 5} \\
\texttt{║◌◌◌└┬┴─╢ 6} \\
\texttt{║◌S◌T│◌M║ 7} \\
\texttt{╚════╧══╝} \\
\texttt{```} \\
            }
        }
    }
    & & \\ \\

    \theutterance \stepcounter{utterance}  
    & & \multicolumn{4}{p{0.6\linewidth}}{
        \cellcolor[rgb]{0.9,0.9,0.9}{
            \makecell[{{p{\linewidth}}}]{
                \texttt{\tiny{[P1$\langle$GM]}}
                \texttt{'A' erfolgreich nach (1,1) verschoben. Dein Raster sieht jetzt so aus:} \\
\texttt{```} \\
\texttt{ 1234567} \\
\texttt{╔═══╤═╤═╗} \\
\texttt{║AS◌│◌│◌║ 1} \\
\texttt{╟─┬─┼─┴─╢ 2} \\
\texttt{║◌│◌│◌◌◌║ 3} \\
\texttt{║P├─┤D┌─╢ 4} \\
\texttt{╟─┘◌│◌│M║ 5} \\
\texttt{║◌◌◌└┬┴─╢ 6} \\
\texttt{║◌OT◌│◌◌║ 7} \\
\texttt{╚════╧══╝} \\
\texttt{```} \\
\texttt{Ihr seid gerade bei Runde 3 von maximal 28.} \\
\texttt{Ihr habt gemeinsam 1 von maximal 16 Strafpunkten angesammelt.} \\
\texttt{Dein Mitspieler hat ein Objekt auf seinem Raster bewegt.} \\
\\ 
\\ 
\texttt{Wie lautet dein nächster Befehl?} \\
            }
        }
    }
    & \\ \\

    \theutterance \stepcounter{utterance}  
    & \multicolumn{4}{p{0.6\linewidth}}{
        \cellcolor[rgb]{0.9,0.9,0.9}{
            \makecell[{{p{\linewidth}}}]{
                \texttt{\tiny{[P1$\rangle$GM]}}
                \texttt{SPRICH: Super, A ist jetzt bei mir auf (1,1) und S steht bereits korrekt auf (6,1). Ich plane als Nächstes D nach (2,1) zu bringen. Bitte gib Bescheid, falls du kurzfristig (2,1) brauchst; ansonsten verschiebe ich gleich D zu (2,1).} \\
            }
        }
    }
    & & \\ \\

    \theutterance \stepcounter{utterance}  
    & & & \multicolumn{2}{p{0.3\linewidth}}{
        \cellcolor[rgb]{0.9,0.9,0.9}{
            \makecell[{{p{\linewidth}}}]{
                \texttt{\tiny{[GM$|$GM]}}
                \texttt{SPRICH: Super, A ist jetzt bei mir auf (1,1) und S steht bereits korrekt auf (6,1). Ich plane als Nächstes D nach (2,1) zu bringen. Bitte gib Bescheid, falls du kurzfristig (2,1) brauchst; ansonsten verschiebe ich gleich D zu (2,1).} \\
            }
        }
    }
    & & \\ \\

    \theutterance \stepcounter{utterance}  
    & & \multicolumn{4}{p{0.6\linewidth}}{
        \cellcolor[rgb]{0.9,0.9,0.9}{
            \makecell[{{p{\linewidth}}}]{
                \texttt{\tiny{[GM$\rangle$P2]}}
                \texttt{'P' erfolgreich nach (5,1) verschoben. Dein Raster sieht jetzt so aus:} \\
\texttt{```} \\
\texttt{ 1234567} \\
\texttt{╔═══╤═╤═╗} \\
\texttt{║◌◌A│P│◌║ 1} \\
\texttt{╟─┬─┼─┴─╢ 2} \\
\texttt{║◌│D│◌◌◌║ 3} \\
\texttt{║◌├─┤◌┌─╢ 4} \\
\texttt{╟─┘◌│◌│O║ 5} \\
\texttt{║◌◌◌└┬┴─╢ 6} \\
\texttt{║◌S◌T│◌M║ 7} \\
\texttt{╚════╧══╝} \\
\texttt{```} \\
\texttt{Ihr seid gerade bei Runde 4 von maximal 28.} \\
\texttt{Ihr habt gemeinsam 1 von maximal 16 Strafpunkten angesammelt.} \\
\texttt{Dein Mitspieler hat dir diese Nachricht geschickt:} \\
\\ 
\\ 
\texttt{"Super, A ist jetzt bei mir auf (1,1) und S steht bereits korrekt auf (6,1). Ich plane als Nächstes D nach (2,1) zu bringen. Bitte gib Bescheid, falls du kurzfristig (2,1) brauchst; ansonsten verschiebe ich gleich D zu (2,1)."} \\
\\ 
\\ 
\texttt{Wie lautet dein nächster Befehl?} \\
            }
        }
    }
    & \\ \\

    \theutterance \stepcounter{utterance}  
    & & & \multicolumn{4}{p{0.6\linewidth}}{
        \cellcolor[rgb]{0.9,0.9,0.9}{
            \makecell[{{p{\linewidth}}}]{
                \texttt{\tiny{[GM$\langle$P2]}}
                \texttt{VERSCHIEBE: A, (1, 1)} \\
            }
        }
    }
     \\ \\

    \theutterance \stepcounter{utterance}  
    & & & \multicolumn{2}{p{0.3\linewidth}}{
        \cellcolor[rgb]{0.9,0.9,0.9}{
            \makecell[{{p{\linewidth}}}]{
                \texttt{\tiny{[GM$|$GM]}}
                \texttt{VERSCHIEBE: A, (1, 1)} \\
            }
        }
    }
    & & \\ \\

    \theutterance \stepcounter{utterance}  
    & & & \multicolumn{2}{p{0.3\linewidth}}{
        \cellcolor[rgb]{0.9,0.9,0.9}{
            \makecell[{{p{\linewidth}}}]{
                \texttt{\tiny{[GM$|$GM]}}
                \texttt{'A' erfolgreich nach (1,1) verschoben. Dein Raster sieht jetzt so aus:} \\
\texttt{```} \\
\texttt{ 1234567} \\
\texttt{╔═══╤═╤═╗} \\
\texttt{║A◌◌│P│◌║ 1} \\
\texttt{╟─┬─┼─┴─╢ 2} \\
\texttt{║◌│D│◌◌◌║ 3} \\
\texttt{║◌├─┤◌┌─╢ 4} \\
\texttt{╟─┘◌│◌│O║ 5} \\
\texttt{║◌◌◌└┬┴─╢ 6} \\
\texttt{║◌S◌T│◌M║ 7} \\
\texttt{╚════╧══╝} \\
\texttt{```} \\
            }
        }
    }
    & & \\ \\

    \theutterance \stepcounter{utterance}  
    & & \multicolumn{4}{p{0.6\linewidth}}{
        \cellcolor[rgb]{0.9,0.9,0.9}{
            \makecell[{{p{\linewidth}}}]{
                \texttt{\tiny{[P1$\langle$GM]}}
                \texttt{Deine Nachricht wurde an deinen Mitspieler übermittelt.} \\
\\ 
\\ 
\texttt{Ihr seid gerade bei Runde 4 von maximal 28.} \\
\texttt{Ihr habt gemeinsam 1 von maximal 16 Strafpunkten angesammelt.} \\
\texttt{Dein Mitspieler hat ein Objekt auf seinem Raster bewegt.} \\
\\ 
\\ 
\texttt{Wie lautet dein nächster Befehl?} \\
            }
        }
    }
    & \\ \\

    \theutterance \stepcounter{utterance}  
    & \multicolumn{4}{p{0.6\linewidth}}{
        \cellcolor[rgb]{0.9,0.9,0.9}{
            \makecell[{{p{\linewidth}}}]{
                \texttt{\tiny{[P1$\rangle$GM]}}
                \texttt{VERSCHIEBE: D, (2,1)} \\
            }
        }
    }
    & & \\ \\

    \theutterance \stepcounter{utterance}  
    & & & \multicolumn{2}{p{0.3\linewidth}}{
        \cellcolor[rgb]{0.9,0.9,0.9}{
            \makecell[{{p{\linewidth}}}]{
                \texttt{\tiny{[GM$|$GM]}}
                \texttt{VERSCHIEBE: D, (2,1)} \\
            }
        }
    }
    & & \\ \\

    \theutterance \stepcounter{utterance}  
    & & & \multicolumn{2}{p{0.3\linewidth}}{
        \cellcolor[rgb]{0.9,0.9,0.9}{
            \makecell[{{p{\linewidth}}}]{
                \texttt{\tiny{[GM$|$GM]}}
                \texttt{Strafpunkt: (2,1) ist nicht leer, sondern enthält 'S'.} \\
\\ 
\texttt{Ihr habt gemeinsam 2 von maximal 16 Strafpunkten angesammelt.} \\
\texttt{Bitte versuche es noch mal!} \\
            }
        }
    }
    & & \\ \\

    \theutterance \stepcounter{utterance}  
    & & \multicolumn{4}{p{0.6\linewidth}}{
        \cellcolor[rgb]{0.9,0.9,0.9}{
            \makecell[{{p{\linewidth}}}]{
                \texttt{\tiny{[P1$\langle$GM]}}
                \texttt{Strafpunkt: (2,1) ist nicht leer, sondern enthält 'S'.} \\
\\ 
\texttt{Ihr habt gemeinsam 2 von maximal 16 Strafpunkten angesammelt.} \\
\texttt{Bitte versuche es noch mal!} \\
            }
        }
    }
    & \\ \\

    \theutterance \stepcounter{utterance}  
    & \multicolumn{4}{p{0.6\linewidth}}{
        \cellcolor[rgb]{0.9,0.9,0.9}{
            \makecell[{{p{\linewidth}}}]{
                \texttt{\tiny{[P1$\rangle$GM]}}
                \texttt{SPRICH: Mein Fehler, (2,1) ist bei mir mit S belegt. Neuer Plan: Ich verschiebe zuerst S von (2,1) nach (6,1), dann D nach (2,1). Bitte sag mir, ob es okay ist, wenn ich S jetzt nach (6,1) setze, oder ob du (6,1) kurzfristig brauchst.} \\
            }
        }
    }
    & & \\ \\

    \theutterance \stepcounter{utterance}  
    & & & \multicolumn{2}{p{0.3\linewidth}}{
        \cellcolor[rgb]{0.9,0.9,0.9}{
            \makecell[{{p{\linewidth}}}]{
                \texttt{\tiny{[GM$|$GM]}}
                \texttt{SPRICH: Mein Fehler, (2,1) ist bei mir mit S belegt. Neuer Plan: Ich verschiebe zuerst S von (2,1) nach (6,1), dann D nach (2,1). Bitte sag mir, ob es okay ist, wenn ich S jetzt nach (6,1) setze, oder ob du (6,1) kurzfristig brauchst.} \\
            }
        }
    }
    & & \\ \\

    \theutterance \stepcounter{utterance}  
    & & \multicolumn{4}{p{0.6\linewidth}}{
        \cellcolor[rgb]{0.9,0.9,0.9}{
            \makecell[{{p{\linewidth}}}]{
                \texttt{\tiny{[GM$\rangle$P2]}}
                \texttt{'A' erfolgreich nach (1,1) verschoben. Dein Raster sieht jetzt so aus:} \\
\texttt{```} \\
\texttt{ 1234567} \\
\texttt{╔═══╤═╤═╗} \\
\texttt{║A◌◌│P│◌║ 1} \\
\texttt{╟─┬─┼─┴─╢ 2} \\
\texttt{║◌│D│◌◌◌║ 3} \\
\texttt{║◌├─┤◌┌─╢ 4} \\
\texttt{╟─┘◌│◌│O║ 5} \\
\texttt{║◌◌◌└┬┴─╢ 6} \\
\texttt{║◌S◌T│◌M║ 7} \\
\texttt{╚════╧══╝} \\
\texttt{```} \\
\texttt{Ihr seid gerade bei Runde 5 von maximal 28.} \\
\texttt{Ihr habt gemeinsam 2 von maximal 16 Strafpunkten angesammelt.} \\
\texttt{Dein Mitspieler hat dir diese Nachricht geschickt:} \\
\\ 
\\ 
\texttt{"Mein Fehler, (2,1) ist bei mir mit S belegt. Neuer Plan: Ich verschiebe zuerst S von (2,1) nach (6,1), dann D nach (2,1). Bitte sag mir, ob es okay ist, wenn ich S jetzt nach (6,1) setze, oder ob du (6,1) kurzfristig brauchst."} \\
\\ 
\\ 
\texttt{Wie lautet dein nächster Befehl?} \\
            }
        }
    }
    & \\ \\

    \theutterance \stepcounter{utterance}  
    & & & \multicolumn{4}{p{0.6\linewidth}}{
        \cellcolor[rgb]{0.9,0.9,0.9}{
            \makecell[{{p{\linewidth}}}]{
                \texttt{\tiny{[GM$\langle$P2]}}
                \texttt{SPRICH: Alles gut, (6,1) brauche ich nicht. Bei mir sind A(1,1) und P(5,1) bereits korrekt. Ich plane als Nächstes D auf (2,1), O auf (4,1), M auf (3,1), S auf (6,1), T auf (7,1). Gib mir kurz Bescheid, wenn du (2,1) oder (4,1) zeitweise blockierst.} \\
            }
        }
    }
     \\ \\

    \theutterance \stepcounter{utterance}  
    & & & \multicolumn{2}{p{0.3\linewidth}}{
        \cellcolor[rgb]{0.9,0.9,0.9}{
            \makecell[{{p{\linewidth}}}]{
                \texttt{\tiny{[GM$|$GM]}}
                \texttt{SPRICH: Alles gut, (6,1) brauche ich nicht. Bei mir sind A(1,1) und P(5,1) bereits korrekt. Ich plane als Nächstes D auf (2,1), O auf (4,1), M auf (3,1), S auf (6,1), T auf (7,1). Gib mir kurz Bescheid, wenn du (2,1) oder (4,1) zeitweise blockierst.} \\
            }
        }
    }
    & & \\ \\

    \theutterance \stepcounter{utterance}  
    & & \multicolumn{4}{p{0.6\linewidth}}{
        \cellcolor[rgb]{0.9,0.9,0.9}{
            \makecell[{{p{\linewidth}}}]{
                \texttt{\tiny{[P1$\langle$GM]}}
                \texttt{Deine Nachricht wurde an deinen Mitspieler übermittelt.} \\
\\ 
\\ 
\texttt{Ihr seid gerade bei Runde 5 von maximal 28.} \\
\texttt{Ihr habt gemeinsam 2 von maximal 16 Strafpunkten angesammelt.} \\
\texttt{Dein Mitspieler hat dir diese Nachricht geschickt:} \\
\\ 
\\ 
\texttt{"Alles gut, (6,1) brauche ich nicht. Bei mir sind A(1,1) und P(5,1) bereits korrekt. Ich plane als Nächstes D auf (2,1), O auf (4,1), M auf (3,1), S auf (6,1), T auf (7,1). Gib mir kurz Bescheid, wenn du (2,1) oder (4,1) zeitweise blockierst."} \\
\\ 
\\ 
\texttt{Wie lautet dein nächster Befehl?} \\
            }
        }
    }
    & \\ \\

    \theutterance \stepcounter{utterance}  
    & \multicolumn{4}{p{0.6\linewidth}}{
        \cellcolor[rgb]{0.9,0.9,0.9}{
            \makecell[{{p{\linewidth}}}]{
                \texttt{\tiny{[P1$\rangle$GM]}}
                \texttt{VERSCHIEBE: S, (6,1)} \\
            }
        }
    }
    & & \\ \\

    \theutterance \stepcounter{utterance}  
    & & & \multicolumn{2}{p{0.3\linewidth}}{
        \cellcolor[rgb]{0.9,0.9,0.9}{
            \makecell[{{p{\linewidth}}}]{
                \texttt{\tiny{[GM$|$GM]}}
                \texttt{VERSCHIEBE: S, (6,1)} \\
            }
        }
    }
    & & \\ \\

    \theutterance \stepcounter{utterance}  
    & & & \multicolumn{2}{p{0.3\linewidth}}{
        \cellcolor[rgb]{0.9,0.9,0.9}{
            \makecell[{{p{\linewidth}}}]{
                \texttt{\tiny{[GM$|$GM]}}
                \texttt{Strafpunkt: (6,1) ist nicht leer, sondern enthält '│'.} \\
\\ 
\texttt{Ihr habt gemeinsam 3 von maximal 16 Strafpunkten angesammelt.} \\
\texttt{Bitte versuche es noch mal!} \\
            }
        }
    }
    & & \\ \\

    \theutterance \stepcounter{utterance}  
    & & \multicolumn{4}{p{0.6\linewidth}}{
        \cellcolor[rgb]{0.9,0.9,0.9}{
            \makecell[{{p{\linewidth}}}]{
                \texttt{\tiny{[P1$\langle$GM]}}
                \texttt{Strafpunkt: (6,1) ist nicht leer, sondern enthält '│'.} \\
\\ 
\texttt{Ihr habt gemeinsam 3 von maximal 16 Strafpunkten angesammelt.} \\
\texttt{Bitte versuche es noch mal!} \\
            }
        }
    }
    & \\ \\

    \theutterance \stepcounter{utterance}  
    & \multicolumn{4}{p{0.6\linewidth}}{
        \cellcolor[rgb]{0.9,0.9,0.9}{
            \makecell[{{p{\linewidth}}}]{
                \texttt{\tiny{[P1$\rangle$GM]}}
                \texttt{SPRICH: Danke für den Hinweis. Bei mir ist (6,1) durch eine Linie blockiert, daher kann S dort nicht stehen. Vorschlag: Wir wechseln auf Zeile 7 als Zielreihe: A(1,7), D(2,7), M(3,7), O(4,7), P(5,7), S(6,7), T(7,7). Bei mir ist Zeile 7 frei. Bist du einverstanden?} \\
            }
        }
    }
    & & \\ \\

    \theutterance \stepcounter{utterance}  
    & & & \multicolumn{2}{p{0.3\linewidth}}{
        \cellcolor[rgb]{0.9,0.9,0.9}{
            \makecell[{{p{\linewidth}}}]{
                \texttt{\tiny{[GM$|$GM]}}
                \texttt{SPRICH: Danke für den Hinweis. Bei mir ist (6,1) durch eine Linie blockiert, daher kann S dort nicht stehen. Vorschlag: Wir wechseln auf Zeile 7 als Zielreihe: A(1,7), D(2,7), M(3,7), O(4,7), P(5,7), S(6,7), T(7,7). Bei mir ist Zeile 7 frei. Bist du einverstanden?} \\
            }
        }
    }
    & & \\ \\

    \theutterance \stepcounter{utterance}  
    & & \multicolumn{4}{p{0.6\linewidth}}{
        \cellcolor[rgb]{0.9,0.9,0.9}{
            \makecell[{{p{\linewidth}}}]{
                \texttt{\tiny{[GM$\rangle$P2]}}
                \texttt{Deine Nachricht wurde an deinen Mitspieler übermittelt.} \\
\\ 
\\ 
\texttt{Ihr seid gerade bei Runde 6 von maximal 28.} \\
\texttt{Ihr habt gemeinsam 3 von maximal 16 Strafpunkten angesammelt.} \\
\texttt{Dein Mitspieler hat dir diese Nachricht geschickt:} \\
\\ 
\\ 
\texttt{"Danke für den Hinweis. Bei mir ist (6,1) durch eine Linie blockiert, daher kann S dort nicht stehen. Vorschlag: Wir wechseln auf Zeile 7 als Zielreihe: A(1,7), D(2,7), M(3,7), O(4,7), P(5,7), S(6,7), T(7,7). Bei mir ist Zeile 7 frei. Bist du einverstanden?"} \\
\\ 
\\ 
\texttt{Wie lautet dein nächster Befehl?} \\
            }
        }
    }
    & \\ \\

    \theutterance \stepcounter{utterance}  
    & & & \multicolumn{4}{p{0.6\linewidth}}{
        \cellcolor[rgb]{0.9,0.9,0.9}{
            \makecell[{{p{\linewidth}}}]{
                \texttt{\tiny{[GM$\langle$P2]}}
                \texttt{SPRICH: Einverstanden mit Zeile 7 als Ziel: A(1,7), D(2,7), M(3,7), O(4,7), P(5,7), S(6,7), T(7,7). Ich beginne mit P nach (5,7) zu bewegen; prüfe vorher, ob frei. Falls blockiert, sag mir bitte eine Ausweichkoordinate in Zeile 7.} \\
            }
        }
    }
     \\ \\

    \theutterance \stepcounter{utterance}  
    & & & \multicolumn{2}{p{0.3\linewidth}}{
        \cellcolor[rgb]{0.9,0.9,0.9}{
            \makecell[{{p{\linewidth}}}]{
                \texttt{\tiny{[GM$|$GM]}}
                \texttt{SPRICH: Einverstanden mit Zeile 7 als Ziel: A(1,7), D(2,7), M(3,7), O(4,7), P(5,7), S(6,7), T(7,7). Ich beginne mit P nach (5,7) zu bewegen; prüfe vorher, ob frei. Falls blockiert, sag mir bitte eine Ausweichkoordinate in Zeile 7.} \\
            }
        }
    }
    & & \\ \\

    \theutterance \stepcounter{utterance}  
    & & \multicolumn{4}{p{0.6\linewidth}}{
        \cellcolor[rgb]{0.9,0.9,0.9}{
            \makecell[{{p{\linewidth}}}]{
                \texttt{\tiny{[P1$\langle$GM]}}
                \texttt{Deine Nachricht wurde an deinen Mitspieler übermittelt.} \\
\\ 
\\ 
\texttt{Ihr seid gerade bei Runde 6 von maximal 28.} \\
\texttt{Ihr habt gemeinsam 3 von maximal 16 Strafpunkten angesammelt.} \\
\texttt{Dein Mitspieler hat dir diese Nachricht geschickt:} \\
\\ 
\\ 
\texttt{"Einverstanden mit Zeile 7 als Ziel: A(1,7), D(2,7), M(3,7), O(4,7), P(5,7), S(6,7), T(7,7). Ich beginne mit P nach (5,7) zu bewegen; prüfe vorher, ob frei. Falls blockiert, sag mir bitte eine Ausweichkoordinate in Zeile 7."} \\
\\ 
\\ 
\texttt{Wie lautet dein nächster Befehl?} \\
            }
        }
    }
    & \\ \\

    \theutterance \stepcounter{utterance}  
    & \multicolumn{4}{p{0.6\linewidth}}{
        \cellcolor[rgb]{0.9,0.9,0.9}{
            \makecell[{{p{\linewidth}}}]{
                \texttt{\tiny{[P1$\rangle$GM]}}
                \texttt{SPRICH: Bestätigt. Bei mir ist (5,7) frei, du kannst P dorthin setzen. Ich beginne nun, meine Steine Richtung Zeile 7 zu bewegen. Ich starte mit O nach (4,7), falls frei bei mir. Sag bitte Bescheid, falls du kurzfristig (4,7) brauchst.} \\
            }
        }
    }
    & & \\ \\

    \theutterance \stepcounter{utterance}  
    & & & \multicolumn{2}{p{0.3\linewidth}}{
        \cellcolor[rgb]{0.9,0.9,0.9}{
            \makecell[{{p{\linewidth}}}]{
                \texttt{\tiny{[GM$|$GM]}}
                \texttt{SPRICH: Bestätigt. Bei mir ist (5,7) frei, du kannst P dorthin setzen. Ich beginne nun, meine Steine Richtung Zeile 7 zu bewegen. Ich starte mit O nach (4,7), falls frei bei mir. Sag bitte Bescheid, falls du kurzfristig (4,7) brauchst.} \\
            }
        }
    }
    & & \\ \\

    \theutterance \stepcounter{utterance}  
    & & \multicolumn{4}{p{0.6\linewidth}}{
        \cellcolor[rgb]{0.9,0.9,0.9}{
            \makecell[{{p{\linewidth}}}]{
                \texttt{\tiny{[GM$\rangle$P2]}}
                \texttt{Deine Nachricht wurde an deinen Mitspieler übermittelt.} \\
\\ 
\\ 
\texttt{Ihr seid gerade bei Runde 7 von maximal 28.} \\
\texttt{Ihr habt gemeinsam 3 von maximal 16 Strafpunkten angesammelt.} \\
\texttt{Dein Mitspieler hat dir diese Nachricht geschickt:} \\
\\ 
\\ 
\texttt{"Bestätigt. Bei mir ist (5,7) frei, du kannst P dorthin setzen. Ich beginne nun, meine Steine Richtung Zeile 7 zu bewegen. Ich starte mit O nach (4,7), falls frei bei mir. Sag bitte Bescheid, falls du kurzfristig (4,7) brauchst."} \\
\\ 
\\ 
\texttt{Wie lautet dein nächster Befehl?} \\
            }
        }
    }
    & \\ \\

    \theutterance \stepcounter{utterance}  
    & & & \multicolumn{4}{p{0.6\linewidth}}{
        \cellcolor[rgb]{0.9,0.9,0.9}{
            \makecell[{{p{\linewidth}}}]{
                \texttt{\tiny{[GM$\langle$P2]}}
                \texttt{VERSCHIEBE: P, (5, 7)} \\
            }
        }
    }
     \\ \\

    \theutterance \stepcounter{utterance}  
    & & & \multicolumn{2}{p{0.3\linewidth}}{
        \cellcolor[rgb]{0.9,0.9,0.9}{
            \makecell[{{p{\linewidth}}}]{
                \texttt{\tiny{[GM$|$GM]}}
                \texttt{VERSCHIEBE: P, (5, 7)} \\
            }
        }
    }
    & & \\ \\

    \theutterance \stepcounter{utterance}  
    & & & \multicolumn{2}{p{0.3\linewidth}}{
        \cellcolor[rgb]{0.9,0.9,0.9}{
            \makecell[{{p{\linewidth}}}]{
                \texttt{\tiny{[GM$|$GM]}}
                \texttt{Strafpunkt: (5,7) ist nicht leer, sondern enthält '│'.} \\
\\ 
\texttt{Ihr habt gemeinsam 4 von maximal 16 Strafpunkten angesammelt.} \\
\texttt{Bitte versuche es noch mal!} \\
            }
        }
    }
    & & \\ \\

    \theutterance \stepcounter{utterance}  
    & & \multicolumn{4}{p{0.6\linewidth}}{
        \cellcolor[rgb]{0.9,0.9,0.9}{
            \makecell[{{p{\linewidth}}}]{
                \texttt{\tiny{[GM$\rangle$P2]}}
                \texttt{Strafpunkt: (5,7) ist nicht leer, sondern enthält '│'.} \\
\\ 
\texttt{Ihr habt gemeinsam 4 von maximal 16 Strafpunkten angesammelt.} \\
\texttt{Bitte versuche es noch mal!} \\
            }
        }
    }
    & \\ \\

    \theutterance \stepcounter{utterance}  
    & & & \multicolumn{4}{p{0.6\linewidth}}{
        \cellcolor[rgb]{0.9,0.9,0.9}{
            \makecell[{{p{\linewidth}}}]{
                \texttt{\tiny{[GM$\langle$P2]}}
                \texttt{SPRICH: Bei mir ist (5,7) durch eine Linie blockiert. Prüfen wir alternative freie Ziele in Zeile 7: Ich sehe S auf (2,7), T auf (4,7) und M auf (7,7). Prüfe bitte bei dir, ob folgende Endpositionen frei sind: A(1,7), D(2,7), M(3,7), O(4,7), P(5,7), S(6,7), T(7,7). Wenn (5,7) oder andere blockiert sind, schlage bitte eine durchgehend freie Reihe/Spalte vor (z. B. Zeile 3 oder Spalte 3). Ich warte mit Verschiebungen bis zur Bestätigung.} \\
            }
        }
    }
     \\ \\

    \theutterance \stepcounter{utterance}  
    & & & \multicolumn{2}{p{0.3\linewidth}}{
        \cellcolor[rgb]{0.9,0.9,0.9}{
            \makecell[{{p{\linewidth}}}]{
                \texttt{\tiny{[GM$|$GM]}}
                \texttt{SPRICH: Bei mir ist (5,7) durch eine Linie blockiert. Prüfen wir alternative freie Ziele in Zeile 7: Ich sehe S auf (2,7), T auf (4,7) und M auf (7,7). Prüfe bitte bei dir, ob folgende Endpositionen frei sind: A(1,7), D(2,7), M(3,7), O(4,7), P(5,7), S(6,7), T(7,7). Wenn (5,7) oder andere blockiert sind, schlage bitte eine durchgehend freie Reihe/Spalte vor (z. B. Zeile 3 oder Spalte 3). Ich warte mit Verschiebungen bis zur Bestätigung.} \\
            }
        }
    }
    & & \\ \\

    \theutterance \stepcounter{utterance}  
    & & \multicolumn{4}{p{0.6\linewidth}}{
        \cellcolor[rgb]{0.9,0.9,0.9}{
            \makecell[{{p{\linewidth}}}]{
                \texttt{\tiny{[P1$\langle$GM]}}
                \texttt{Deine Nachricht wurde an deinen Mitspieler übermittelt.} \\
\\ 
\\ 
\texttt{Ihr seid gerade bei Runde 7 von maximal 28.} \\
\texttt{Ihr habt gemeinsam 4 von maximal 16 Strafpunkten angesammelt.} \\
\texttt{Dein Mitspieler hat dir diese Nachricht geschickt:} \\
\\ 
\\ 
\texttt{"Bei mir ist (5,7) durch eine Linie blockiert. Prüfen wir alternative freie Ziele in Zeile 7: Ich sehe S auf (2,7), T auf (4,7) und M auf (7,7). Prüfe bitte bei dir, ob folgende Endpositionen frei sind: A(1,7), D(2,7), M(3,7), O(4,7), P(5,7), S(6,7), T(7,7). Wenn (5,7) oder andere blockiert sind, schlage bitte eine durchgehend freie Reihe/Spalte vor (z. B. Zeile 3 oder Spalte 3). Ich warte mit Verschiebungen bis zur Bestätigung."} \\
\\ 
\\ 
\texttt{Wie lautet dein nächster Befehl?} \\
            }
        }
    }
    & \\ \\

    \theutterance \stepcounter{utterance}  
    & \multicolumn{4}{p{0.6\linewidth}}{
        \cellcolor[rgb]{0.9,0.9,0.9}{
            \makecell[{{p{\linewidth}}}]{
                \texttt{\tiny{[P1$\rangle$GM]}}
                \texttt{SPRICH: Danke für die Prüfung. Auf meinem Board sind in Zeile 7 die Plätze (1,7)=◌, (2,7)=O, (3,7)=T, (4,7)=◌, (5,7)=◌, (6,7)=◌, (7,7)=◌. Zeile 7 ist also nicht komplett frei. Ich schlage stattdessen Zeile 3 als durchgehend freie Zielreihe vor: A(1,3), D(2,3), M(3,3), O(4,3), P(5,3), S(6,3), T(7,3). Bei mir ist Zeile 3 vollständig frei. Einverstanden?} \\
            }
        }
    }
    & & \\ \\

    \theutterance \stepcounter{utterance}  
    & & & \multicolumn{2}{p{0.3\linewidth}}{
        \cellcolor[rgb]{0.9,0.9,0.9}{
            \makecell[{{p{\linewidth}}}]{
                \texttt{\tiny{[GM$|$GM]}}
                \texttt{SPRICH: Danke für die Prüfung. Auf meinem Board sind in Zeile 7 die Plätze (1,7)=◌, (2,7)=O, (3,7)=T, (4,7)=◌, (5,7)=◌, (6,7)=◌, (7,7)=◌. Zeile 7 ist also nicht komplett frei. Ich schlage stattdessen Zeile 3 als durchgehend freie Zielreihe vor: A(1,3), D(2,3), M(3,3), O(4,3), P(5,3), S(6,3), T(7,3). Bei mir ist Zeile 3 vollständig frei. Einverstanden?} \\
            }
        }
    }
    & & \\ \\

    \theutterance \stepcounter{utterance}  
    & & \multicolumn{4}{p{0.6\linewidth}}{
        \cellcolor[rgb]{0.9,0.9,0.9}{
            \makecell[{{p{\linewidth}}}]{
                \texttt{\tiny{[GM$\rangle$P2]}}
                \texttt{Deine Nachricht wurde an deinen Mitspieler übermittelt.} \\
\\ 
\\ 
\texttt{Ihr seid gerade bei Runde 8 von maximal 28.} \\
\texttt{Ihr habt gemeinsam 4 von maximal 16 Strafpunkten angesammelt.} \\
\texttt{Dein Mitspieler hat dir diese Nachricht geschickt:} \\
\\ 
\\ 
\texttt{"Danke für die Prüfung. Auf meinem Board sind in Zeile 7 die Plätze (1,7)=◌, (2,7)=O, (3,7)=T, (4,7)=◌, (5,7)=◌, (6,7)=◌, (7,7)=◌. Zeile 7 ist also nicht komplett frei. Ich schlage stattdessen Zeile 3 als durchgehend freie Zielreihe vor: A(1,3), D(2,3), M(3,3), O(4,3), P(5,3), S(6,3), T(7,3). Bei mir ist Zeile 3 vollständig frei. Einverstanden?"} \\
\\ 
\\ 
\texttt{Wie lautet dein nächster Befehl?} \\
            }
        }
    }
    & \\ \\

    \theutterance \stepcounter{utterance}  
    & & & \multicolumn{4}{p{0.6\linewidth}}{
        \cellcolor[rgb]{0.9,0.9,0.9}{
            \makecell[{{p{\linewidth}}}]{
                \texttt{\tiny{[GM$\langle$P2]}}
                \texttt{SPRICH: Einverstanden mit Zeile 3: A(1,3), D(2,3), M(3,3), O(4,3), P(5,3), S(6,3), T(7,3). Bei mir ist (1,3) bereits D belegt; ich plane D nach (2,3) zu verschieben, dann A nach (1,3), P nach (5,3), O nach (4,3), M nach (3,3), S nach (6,3), T nach (7,3). Bitte sag Bescheid, falls du (2,3) oder (5,3) kurzfristig brauchst.} \\
            }
        }
    }
     \\ \\

    \theutterance \stepcounter{utterance}  
    & & & \multicolumn{2}{p{0.3\linewidth}}{
        \cellcolor[rgb]{0.9,0.9,0.9}{
            \makecell[{{p{\linewidth}}}]{
                \texttt{\tiny{[GM$|$GM]}}
                \texttt{SPRICH: Einverstanden mit Zeile 3: A(1,3), D(2,3), M(3,3), O(4,3), P(5,3), S(6,3), T(7,3). Bei mir ist (1,3) bereits D belegt; ich plane D nach (2,3) zu verschieben, dann A nach (1,3), P nach (5,3), O nach (4,3), M nach (3,3), S nach (6,3), T nach (7,3). Bitte sag Bescheid, falls du (2,3) oder (5,3) kurzfristig brauchst.} \\
            }
        }
    }
    & & \\ \\

    \theutterance \stepcounter{utterance}  
    & & \multicolumn{4}{p{0.6\linewidth}}{
        \cellcolor[rgb]{0.9,0.9,0.9}{
            \makecell[{{p{\linewidth}}}]{
                \texttt{\tiny{[P1$\langle$GM]}}
                \texttt{Deine Nachricht wurde an deinen Mitspieler übermittelt.} \\
\\ 
\\ 
\texttt{Ihr seid gerade bei Runde 8 von maximal 28.} \\
\texttt{Ihr habt gemeinsam 4 von maximal 16 Strafpunkten angesammelt.} \\
\texttt{Dein Mitspieler hat dir diese Nachricht geschickt:} \\
\\ 
\\ 
\texttt{"Einverstanden mit Zeile 3: A(1,3), D(2,3), M(3,3), O(4,3), P(5,3), S(6,3), T(7,3). Bei mir ist (1,3) bereits D belegt; ich plane D nach (2,3) zu verschieben, dann A nach (1,3), P nach (5,3), O nach (4,3), M nach (3,3), S nach (6,3), T nach (7,3). Bitte sag Bescheid, falls du (2,3) oder (5,3) kurzfristig brauchst."} \\
\\ 
\\ 
\texttt{Wie lautet dein nächster Befehl?} \\
            }
        }
    }
    & \\ \\

    \theutterance \stepcounter{utterance}  
    & \multicolumn{4}{p{0.6\linewidth}}{
        \cellcolor[rgb]{0.9,0.9,0.9}{
            \makecell[{{p{\linewidth}}}]{
                \texttt{\tiny{[P1$\rangle$GM]}}
                \texttt{SPRICH: Bestätigt. Zeile 3 ist bei mir komplett frei, du kannst (2,3) und (5,3) nutzen. Ich beginne ebenfalls: Ich verschiebe gleich O nach (4,3), danach T nach (7,3), dann D nach (2,3), M nach (3,3), P nach (5,3), S nach (6,3), A nach (1,3). Sag Bescheid, falls du kurzfristig (4,3) oder (7,3) brauchst.} \\
            }
        }
    }
    & & \\ \\

    \theutterance \stepcounter{utterance}  
    & & & \multicolumn{2}{p{0.3\linewidth}}{
        \cellcolor[rgb]{0.9,0.9,0.9}{
            \makecell[{{p{\linewidth}}}]{
                \texttt{\tiny{[GM$|$GM]}}
                \texttt{SPRICH: Bestätigt. Zeile 3 ist bei mir komplett frei, du kannst (2,3) und (5,3) nutzen. Ich beginne ebenfalls: Ich verschiebe gleich O nach (4,3), danach T nach (7,3), dann D nach (2,3), M nach (3,3), P nach (5,3), S nach (6,3), A nach (1,3). Sag Bescheid, falls du kurzfristig (4,3) oder (7,3) brauchst.} \\
            }
        }
    }
    & & \\ \\

    \theutterance \stepcounter{utterance}  
    & & \multicolumn{4}{p{0.6\linewidth}}{
        \cellcolor[rgb]{0.9,0.9,0.9}{
            \makecell[{{p{\linewidth}}}]{
                \texttt{\tiny{[GM$\rangle$P2]}}
                \texttt{Deine Nachricht wurde an deinen Mitspieler übermittelt.} \\
\\ 
\\ 
\texttt{Ihr seid gerade bei Runde 9 von maximal 28.} \\
\texttt{Ihr habt gemeinsam 4 von maximal 16 Strafpunkten angesammelt.} \\
\texttt{Dein Mitspieler hat dir diese Nachricht geschickt:} \\
\\ 
\\ 
\texttt{"Bestätigt. Zeile 3 ist bei mir komplett frei, du kannst (2,3) und (5,3) nutzen. Ich beginne ebenfalls: Ich verschiebe gleich O nach (4,3), danach T nach (7,3), dann D nach (2,3), M nach (3,3), P nach (5,3), S nach (6,3), A nach (1,3). Sag Bescheid, falls du kurzfristig (4,3) oder (7,3) brauchst."} \\
\\ 
\\ 
\texttt{Wie lautet dein nächster Befehl?} \\
            }
        }
    }
    & \\ \\

    \theutterance \stepcounter{utterance}  
    & & & \multicolumn{4}{p{0.6\linewidth}}{
        \cellcolor[rgb]{0.9,0.9,0.9}{
            \makecell[{{p{\linewidth}}}]{
                \texttt{\tiny{[GM$\langle$P2]}}
                \texttt{VERSCHIEBE: D, (2, 3)} \\
            }
        }
    }
     \\ \\

    \theutterance \stepcounter{utterance}  
    & & & \multicolumn{2}{p{0.3\linewidth}}{
        \cellcolor[rgb]{0.9,0.9,0.9}{
            \makecell[{{p{\linewidth}}}]{
                \texttt{\tiny{[GM$|$GM]}}
                \texttt{VERSCHIEBE: D, (2, 3)} \\
            }
        }
    }
    & & \\ \\

    \theutterance \stepcounter{utterance}  
    & & & \multicolumn{2}{p{0.3\linewidth}}{
        \cellcolor[rgb]{0.9,0.9,0.9}{
            \makecell[{{p{\linewidth}}}]{
                \texttt{\tiny{[GM$|$GM]}}
                \texttt{Strafpunkt: (2,3) ist nicht leer, sondern enthält '│'.} \\
\\ 
\texttt{Ihr habt gemeinsam 5 von maximal 16 Strafpunkten angesammelt.} \\
\texttt{Bitte versuche es noch mal!} \\
            }
        }
    }
    & & \\ \\

    \theutterance \stepcounter{utterance}  
    & & \multicolumn{4}{p{0.6\linewidth}}{
        \cellcolor[rgb]{0.9,0.9,0.9}{
            \makecell[{{p{\linewidth}}}]{
                \texttt{\tiny{[GM$\rangle$P2]}}
                \texttt{Strafpunkt: (2,3) ist nicht leer, sondern enthält '│'.} \\
\\ 
\texttt{Ihr habt gemeinsam 5 von maximal 16 Strafpunkten angesammelt.} \\
\texttt{Bitte versuche es noch mal!} \\
            }
        }
    }
    & \\ \\

    \theutterance \stepcounter{utterance}  
    & & & \multicolumn{4}{p{0.6\linewidth}}{
        \cellcolor[rgb]{0.9,0.9,0.9}{
            \makecell[{{p{\linewidth}}}]{
                \texttt{\tiny{[GM$\langle$P2]}}
                \texttt{SPRICH: Bei mir ist (2,3) durch eine Linie blockiert. Ich suche freie Felder in Zeile 3: Frei sehe ich (3,3), (4,3), (5,3), (6,3), (7,3). Ich würde Reihenfolge ändern: P→(5,3), O→(4,3), M→(3,3), S→(6,3), T→(7,3), dann A→(1,3), D→(2,3) zuletzt oder alternative Spalte, falls (1,3)/(2,3) blockiert bleiben. Passt das für dich?} \\
            }
        }
    }
     \\ \\

    \theutterance \stepcounter{utterance}  
    & & & \multicolumn{2}{p{0.3\linewidth}}{
        \cellcolor[rgb]{0.9,0.9,0.9}{
            \makecell[{{p{\linewidth}}}]{
                \texttt{\tiny{[GM$|$GM]}}
                \texttt{SPRICH: Bei mir ist (2,3) durch eine Linie blockiert. Ich suche freie Felder in Zeile 3: Frei sehe ich (3,3), (4,3), (5,3), (6,3), (7,3). Ich würde Reihenfolge ändern: P→(5,3), O→(4,3), M→(3,3), S→(6,3), T→(7,3), dann A→(1,3), D→(2,3) zuletzt oder alternative Spalte, falls (1,3)/(2,3) blockiert bleiben. Passt das für dich?} \\
            }
        }
    }
    & & \\ \\

    \theutterance \stepcounter{utterance}  
    & & \multicolumn{4}{p{0.6\linewidth}}{
        \cellcolor[rgb]{0.9,0.9,0.9}{
            \makecell[{{p{\linewidth}}}]{
                \texttt{\tiny{[P1$\langle$GM]}}
                \texttt{Deine Nachricht wurde an deinen Mitspieler übermittelt.} \\
\\ 
\\ 
\texttt{Ihr seid gerade bei Runde 9 von maximal 28.} \\
\texttt{Ihr habt gemeinsam 5 von maximal 16 Strafpunkten angesammelt.} \\
\texttt{Dein Mitspieler hat dir diese Nachricht geschickt:} \\
\\ 
\\ 
\texttt{"Bei mir ist (2,3) durch eine Linie blockiert. Ich suche freie Felder in Zeile 3: Frei sehe ich (3,3), (4,3), (5,3), (6,3), (7,3). Ich würde Reihenfolge ändern: P→(5,3), O→(4,3), M→(3,3), S→(6,3), T→(7,3), dann A→(1,3), D→(2,3) zuletzt oder alternative Spalte, falls (1,3)/(2,3) blockiert bleiben. Passt das für dich?"} \\
\\ 
\\ 
\texttt{Wie lautet dein nächster Befehl?} \\
            }
        }
    }
    & \\ \\

    \theutterance \stepcounter{utterance}  
    & \multicolumn{4}{p{0.6\linewidth}}{
        \cellcolor[rgb]{0.9,0.9,0.9}{
            \makecell[{{p{\linewidth}}}]{
                \texttt{\tiny{[P1$\rangle$GM]}}
                \texttt{SPRICH: Passt. Ich beginne jetzt mit O nach (4,3), danach T→(7,3), M→(3,3), P→(5,3), S→(6,3), A→(1,3), D→(2,3) zum Schluss. Bei mir sind (4,3) und (7,3) frei. Sag Bescheid, falls du eines davon kurzfristig brauchst.} \\
            }
        }
    }
    & & \\ \\

    \theutterance \stepcounter{utterance}  
    & & & \multicolumn{2}{p{0.3\linewidth}}{
        \cellcolor[rgb]{0.9,0.9,0.9}{
            \makecell[{{p{\linewidth}}}]{
                \texttt{\tiny{[GM$|$GM]}}
                \texttt{SPRICH: Passt. Ich beginne jetzt mit O nach (4,3), danach T→(7,3), M→(3,3), P→(5,3), S→(6,3), A→(1,3), D→(2,3) zum Schluss. Bei mir sind (4,3) und (7,3) frei. Sag Bescheid, falls du eines davon kurzfristig brauchst.} \\
            }
        }
    }
    & & \\ \\

    \theutterance \stepcounter{utterance}  
    & & \multicolumn{4}{p{0.6\linewidth}}{
        \cellcolor[rgb]{0.9,0.9,0.9}{
            \makecell[{{p{\linewidth}}}]{
                \texttt{\tiny{[GM$\rangle$P2]}}
                \texttt{Deine Nachricht wurde an deinen Mitspieler übermittelt.} \\
\\ 
\\ 
\texttt{Ihr seid gerade bei Runde 10 von maximal 28.} \\
\texttt{Ihr habt gemeinsam 5 von maximal 16 Strafpunkten angesammelt.} \\
\texttt{Dein Mitspieler hat dir diese Nachricht geschickt:} \\
\\ 
\\ 
\texttt{"Passt. Ich beginne jetzt mit O nach (4,3), danach T→(7,3), M→(3,3), P→(5,3), S→(6,3), A→(1,3), D→(2,3) zum Schluss. Bei mir sind (4,3) und (7,3) frei. Sag Bescheid, falls du eines davon kurzfristig brauchst."} \\
\\ 
\\ 
\texttt{Wie lautet dein nächster Befehl?} \\
            }
        }
    }
    & \\ \\

    \theutterance \stepcounter{utterance}  
    & & & \multicolumn{4}{p{0.6\linewidth}}{
        \cellcolor[rgb]{0.9,0.9,0.9}{
            \makecell[{{p{\linewidth}}}]{
                \texttt{\tiny{[GM$\langle$P2]}}
                \texttt{VERSCHIEBE: P, (5, 3)} \\
            }
        }
    }
     \\ \\

    \theutterance \stepcounter{utterance}  
    & & & \multicolumn{2}{p{0.3\linewidth}}{
        \cellcolor[rgb]{0.9,0.9,0.9}{
            \makecell[{{p{\linewidth}}}]{
                \texttt{\tiny{[GM$|$GM]}}
                \texttt{VERSCHIEBE: P, (5, 3)} \\
            }
        }
    }
    & & \\ \\

    \theutterance \stepcounter{utterance}  
    & & & \multicolumn{2}{p{0.3\linewidth}}{
        \cellcolor[rgb]{0.9,0.9,0.9}{
            \makecell[{{p{\linewidth}}}]{
                \texttt{\tiny{[GM$|$GM]}}
                \texttt{'P' erfolgreich nach (5,3) verschoben. Dein Raster sieht jetzt so aus:} \\
\texttt{```} \\
\texttt{ 1234567} \\
\texttt{╔═══╤═╤═╗} \\
\texttt{║A◌◌│◌│◌║ 1} \\
\texttt{╟─┬─┼─┴─╢ 2} \\
\texttt{║◌│D│P◌◌║ 3} \\
\texttt{║◌├─┤◌┌─╢ 4} \\
\texttt{╟─┘◌│◌│O║ 5} \\
\texttt{║◌◌◌└┬┴─╢ 6} \\
\texttt{║◌S◌T│◌M║ 7} \\
\texttt{╚════╧══╝} \\
\texttt{```} \\
            }
        }
    }
    & & \\ \\

    \theutterance \stepcounter{utterance}  
    & & \multicolumn{4}{p{0.6\linewidth}}{
        \cellcolor[rgb]{0.9,0.9,0.9}{
            \makecell[{{p{\linewidth}}}]{
                \texttt{\tiny{[P1$\langle$GM]}}
                \texttt{Deine Nachricht wurde an deinen Mitspieler übermittelt.} \\
\\ 
\\ 
\texttt{Ihr seid gerade bei Runde 10 von maximal 28.} \\
\texttt{Ihr habt gemeinsam 5 von maximal 16 Strafpunkten angesammelt.} \\
\texttt{Dein Mitspieler hat ein Objekt auf seinem Raster bewegt.} \\
\\ 
\\ 
\texttt{Wie lautet dein nächster Befehl?} \\
            }
        }
    }
    & \\ \\

    \theutterance \stepcounter{utterance}  
    & \multicolumn{4}{p{0.6\linewidth}}{
        \cellcolor[rgb]{0.9,0.9,0.9}{
            \makecell[{{p{\linewidth}}}]{
                \texttt{\tiny{[P1$\rangle$GM]}}
                \texttt{VERSCHIEBE: O, (4,3)} \\
            }
        }
    }
    & & \\ \\

    \theutterance \stepcounter{utterance}  
    & & & \multicolumn{2}{p{0.3\linewidth}}{
        \cellcolor[rgb]{0.9,0.9,0.9}{
            \makecell[{{p{\linewidth}}}]{
                \texttt{\tiny{[GM$|$GM]}}
                \texttt{VERSCHIEBE: O, (4,3)} \\
            }
        }
    }
    & & \\ \\

    \theutterance \stepcounter{utterance}  
    & & & \multicolumn{2}{p{0.3\linewidth}}{
        \cellcolor[rgb]{0.9,0.9,0.9}{
            \makecell[{{p{\linewidth}}}]{
                \texttt{\tiny{[GM$|$GM]}}
                \texttt{Strafpunkt: (4,3) ist nicht leer, sondern enthält '│'.} \\
\\ 
\texttt{Ihr habt gemeinsam 6 von maximal 16 Strafpunkten angesammelt.} \\
\texttt{Bitte versuche es noch mal!} \\
            }
        }
    }
    & & \\ \\

    \theutterance \stepcounter{utterance}  
    & & \multicolumn{4}{p{0.6\linewidth}}{
        \cellcolor[rgb]{0.9,0.9,0.9}{
            \makecell[{{p{\linewidth}}}]{
                \texttt{\tiny{[P1$\langle$GM]}}
                \texttt{Strafpunkt: (4,3) ist nicht leer, sondern enthält '│'.} \\
\\ 
\texttt{Ihr habt gemeinsam 6 von maximal 16 Strafpunkten angesammelt.} \\
\texttt{Bitte versuche es noch mal!} \\
            }
        }
    }
    & \\ \\

    \theutterance \stepcounter{utterance}  
    & \multicolumn{4}{p{0.6\linewidth}}{
        \cellcolor[rgb]{0.9,0.9,0.9}{
            \makecell[{{p{\linewidth}}}]{
                \texttt{\tiny{[P1$\rangle$GM]}}
                \texttt{SPRICH: Bei mir ist (4,3) durch eine Linie blockiert. Ich prüfe freie Felder in Zeile 3: (1,3)=◌, (2,3)=◌, (3,3)=◌, (4,3)=│ blockiert, (5,3)=◌, (6,3)=◌, (7,3)=◌. Vorschlag: Wir bleiben bei Zeile 3, aber ich setze zuerst T→(7,3), dann M→(3,3), P→(5,3), S→(6,3), A→(1,3), und wir verschieben O/D zuletzt auf freie Plätze (z. B. O→(4,2) oder (4,4) Zwischenpark, dann final, falls (4,3) frei ist). Einverstanden?} \\
            }
        }
    }
    & & \\ \\

    \theutterance \stepcounter{utterance}  
    & & & \multicolumn{2}{p{0.3\linewidth}}{
        \cellcolor[rgb]{0.9,0.9,0.9}{
            \makecell[{{p{\linewidth}}}]{
                \texttt{\tiny{[GM$|$GM]}}
                \texttt{SPRICH: Bei mir ist (4,3) durch eine Linie blockiert. Ich prüfe freie Felder in Zeile 3: (1,3)=◌, (2,3)=◌, (3,3)=◌, (4,3)=│ blockiert, (5,3)=◌, (6,3)=◌, (7,3)=◌. Vorschlag: Wir bleiben bei Zeile 3, aber ich setze zuerst T→(7,3), dann M→(3,3), P→(5,3), S→(6,3), A→(1,3), und wir verschieben O/D zuletzt auf freie Plätze (z. B. O→(4,2) oder (4,4) Zwischenpark, dann final, falls (4,3) frei ist). Einverstanden?} \\
            }
        }
    }
    & & \\ \\

    \theutterance \stepcounter{utterance}  
    & & \multicolumn{4}{p{0.6\linewidth}}{
        \cellcolor[rgb]{0.9,0.9,0.9}{
            \makecell[{{p{\linewidth}}}]{
                \texttt{\tiny{[GM$\rangle$P2]}}
                \texttt{'P' erfolgreich nach (5,3) verschoben. Dein Raster sieht jetzt so aus:} \\
\texttt{```} \\
\texttt{ 1234567} \\
\texttt{╔═══╤═╤═╗} \\
\texttt{║A◌◌│◌│◌║ 1} \\
\texttt{╟─┬─┼─┴─╢ 2} \\
\texttt{║◌│D│P◌◌║ 3} \\
\texttt{║◌├─┤◌┌─╢ 4} \\
\texttt{╟─┘◌│◌│O║ 5} \\
\texttt{║◌◌◌└┬┴─╢ 6} \\
\texttt{║◌S◌T│◌M║ 7} \\
\texttt{╚════╧══╝} \\
\texttt{```} \\
\texttt{Ihr seid gerade bei Runde 11 von maximal 28.} \\
\texttt{Ihr habt gemeinsam 6 von maximal 16 Strafpunkten angesammelt.} \\
\texttt{Dein Mitspieler hat dir diese Nachricht geschickt:} \\
\\ 
\\ 
\texttt{"Bei mir ist (4,3) durch eine Linie blockiert. Ich prüfe freie Felder in Zeile 3: (1,3)=◌, (2,3)=◌, (3,3)=◌, (4,3)=│ blockiert, (5,3)=◌, (6,3)=◌, (7,3)=◌. Vorschlag: Wir bleiben bei Zeile 3, aber ich setze zuerst T→(7,3), dann M→(3,3), P→(5,3), S→(6,3), A→(1,3), und wir verschieben O/D zuletzt auf freie Plätze (z. B. O→(4,2) oder (4,4) Zwischenpark, dann final, falls (4,3) frei ist). Einverstanden?"} \\
\\ 
\\ 
\texttt{Wie lautet dein nächster Befehl?} \\
            }
        }
    }
    & \\ \\

    \theutterance \stepcounter{utterance}  
    & & & \multicolumn{4}{p{0.6\linewidth}}{
        \cellcolor[rgb]{0.9,0.9,0.9}{
            \makecell[{{p{\linewidth}}}]{
                \texttt{\tiny{[GM$\langle$P2]}}
                \texttt{SPRICH: Einverstanden. Bei mir ist (4,3) ebenfalls blockiert. Ich parke O vorerst anderswo und fülle die freien Ziele: Als Nächstes setze ich M→(3,3), dann S→(6,3), T→(7,3). A bleibt einstweilen auf (1,1), D kann nicht nach (2,3) wegen Linie. Sag bitte Bescheid, falls du (3,3) oder (6,3) kurzfristig brauchst.} \\
            }
        }
    }
     \\ \\

    \theutterance \stepcounter{utterance}  
    & & & \multicolumn{2}{p{0.3\linewidth}}{
        \cellcolor[rgb]{0.9,0.9,0.9}{
            \makecell[{{p{\linewidth}}}]{
                \texttt{\tiny{[GM$|$GM]}}
                \texttt{SPRICH: Einverstanden. Bei mir ist (4,3) ebenfalls blockiert. Ich parke O vorerst anderswo und fülle die freien Ziele: Als Nächstes setze ich M→(3,3), dann S→(6,3), T→(7,3). A bleibt einstweilen auf (1,1), D kann nicht nach (2,3) wegen Linie. Sag bitte Bescheid, falls du (3,3) oder (6,3) kurzfristig brauchst.} \\
            }
        }
    }
    & & \\ \\

    \theutterance \stepcounter{utterance}  
    & & \multicolumn{4}{p{0.6\linewidth}}{
        \cellcolor[rgb]{0.9,0.9,0.9}{
            \makecell[{{p{\linewidth}}}]{
                \texttt{\tiny{[P1$\langle$GM]}}
                \texttt{Deine Nachricht wurde an deinen Mitspieler übermittelt.} \\
\\ 
\\ 
\texttt{Ihr seid gerade bei Runde 11 von maximal 28.} \\
\texttt{Ihr habt gemeinsam 6 von maximal 16 Strafpunkten angesammelt.} \\
\texttt{Dein Mitspieler hat dir diese Nachricht geschickt:} \\
\\ 
\\ 
\texttt{"Einverstanden. Bei mir ist (4,3) ebenfalls blockiert. Ich parke O vorerst anderswo und fülle die freien Ziele: Als Nächstes setze ich M→(3,3), dann S→(6,3), T→(7,3). A bleibt einstweilen auf (1,1), D kann nicht nach (2,3) wegen Linie. Sag bitte Bescheid, falls du (3,3) oder (6,3) kurzfristig brauchst."} \\
\\ 
\\ 
\texttt{Wie lautet dein nächster Befehl?} \\
            }
        }
    }
    & \\ \\

    \theutterance \stepcounter{utterance}  
    & \multicolumn{4}{p{0.6\linewidth}}{
        \cellcolor[rgb]{0.9,0.9,0.9}{
            \makecell[{{p{\linewidth}}}]{
                \texttt{\tiny{[P1$\rangle$GM]}}
                \texttt{VERSCHIEBE: T, (7,3)} \\
            }
        }
    }
    & & \\ \\

    \theutterance \stepcounter{utterance}  
    & & & \multicolumn{2}{p{0.3\linewidth}}{
        \cellcolor[rgb]{0.9,0.9,0.9}{
            \makecell[{{p{\linewidth}}}]{
                \texttt{\tiny{[GM$|$GM]}}
                \texttt{VERSCHIEBE: T, (7,3)} \\
            }
        }
    }
    & & \\ \\

    \theutterance \stepcounter{utterance}  
    & & & \multicolumn{2}{p{0.3\linewidth}}{
        \cellcolor[rgb]{0.9,0.9,0.9}{
            \makecell[{{p{\linewidth}}}]{
                \texttt{\tiny{[GM$|$GM]}}
                \texttt{'T' erfolgreich nach (7,3) verschoben. Dein Raster sieht jetzt so aus:} \\
\texttt{```} \\
\texttt{ 1234567} \\
\texttt{╔═══╤═╤═╗} \\
\texttt{║AS◌│◌│◌║ 1} \\
\texttt{╟─┬─┼─┴─╢ 2} \\
\texttt{║◌│◌│◌◌T║ 3} \\
\texttt{║P├─┤D┌─╢ 4} \\
\texttt{╟─┘◌│◌│M║ 5} \\
\texttt{║◌◌◌└┬┴─╢ 6} \\
\texttt{║◌O◌◌│◌◌║ 7} \\
\texttt{╚════╧══╝} \\
\texttt{```} \\
            }
        }
    }
    & & \\ \\

    \theutterance \stepcounter{utterance}  
    & & \multicolumn{4}{p{0.6\linewidth}}{
        \cellcolor[rgb]{0.9,0.9,0.9}{
            \makecell[{{p{\linewidth}}}]{
                \texttt{\tiny{[GM$\rangle$P2]}}
                \texttt{Deine Nachricht wurde an deinen Mitspieler übermittelt.} \\
\\ 
\\ 
\texttt{Ihr seid gerade bei Runde 12 von maximal 28.} \\
\texttt{Ihr habt gemeinsam 6 von maximal 16 Strafpunkten angesammelt.} \\
\texttt{Dein Mitspieler hat ein Objekt auf seinem Raster bewegt.} \\
\\ 
\\ 
\texttt{Wie lautet dein nächster Befehl?} \\
            }
        }
    }
    & \\ \\

    \theutterance \stepcounter{utterance}  
    & & & \multicolumn{4}{p{0.6\linewidth}}{
        \cellcolor[rgb]{0.9,0.9,0.9}{
            \makecell[{{p{\linewidth}}}]{
                \texttt{\tiny{[GM$\langle$P2]}}
                \texttt{VERSCHIEBE: M, (3, 3)} \\
            }
        }
    }
     \\ \\

    \theutterance \stepcounter{utterance}  
    & & & \multicolumn{2}{p{0.3\linewidth}}{
        \cellcolor[rgb]{0.9,0.9,0.9}{
            \makecell[{{p{\linewidth}}}]{
                \texttt{\tiny{[GM$|$GM]}}
                \texttt{VERSCHIEBE: M, (3, 3)} \\
            }
        }
    }
    & & \\ \\

    \theutterance \stepcounter{utterance}  
    & & & \multicolumn{2}{p{0.3\linewidth}}{
        \cellcolor[rgb]{0.9,0.9,0.9}{
            \makecell[{{p{\linewidth}}}]{
                \texttt{\tiny{[GM$|$GM]}}
                \texttt{Strafpunkt: (3,3) ist nicht leer, sondern enthält 'D'.} \\
\\ 
\texttt{Ihr habt gemeinsam 7 von maximal 16 Strafpunkten angesammelt.} \\
\texttt{Bitte versuche es noch mal!} \\
            }
        }
    }
    & & \\ \\

    \theutterance \stepcounter{utterance}  
    & & \multicolumn{4}{p{0.6\linewidth}}{
        \cellcolor[rgb]{0.9,0.9,0.9}{
            \makecell[{{p{\linewidth}}}]{
                \texttt{\tiny{[GM$\rangle$P2]}}
                \texttt{Strafpunkt: (3,3) ist nicht leer, sondern enthält 'D'.} \\
\\ 
\texttt{Ihr habt gemeinsam 7 von maximal 16 Strafpunkten angesammelt.} \\
\texttt{Bitte versuche es noch mal!} \\
            }
        }
    }
    & \\ \\

    \theutterance \stepcounter{utterance}  
    & & & \multicolumn{4}{p{0.6\linewidth}}{
        \cellcolor[rgb]{0.9,0.9,0.9}{
            \makecell[{{p{\linewidth}}}]{
                \texttt{\tiny{[GM$\langle$P2]}}
                \texttt{SPRICH: Korrektur: Bei mir steht D bereits auf (2,3) und blockiert angrenzend, M steht auf (7,7). Freie Ziele in Zeile 3 sind aktuell (1,3), (6,3), (7,3); (3,3) ist belegt, (4,3) blockiert, (5,3) ist P. Ich setze als Nächstes S→(6,3), dann T→(7,3), danach suche ich eine freie Route für M→(3,3). Bitte sag Bescheid, falls du (6,3) oder (7,3) brauchst.} \\
            }
        }
    }
     \\ \\

    \theutterance \stepcounter{utterance}  
    & & & \multicolumn{2}{p{0.3\linewidth}}{
        \cellcolor[rgb]{0.9,0.9,0.9}{
            \makecell[{{p{\linewidth}}}]{
                \texttt{\tiny{[GM$|$GM]}}
                \texttt{SPRICH: Korrektur: Bei mir steht D bereits auf (2,3) und blockiert angrenzend, M steht auf (7,7). Freie Ziele in Zeile 3 sind aktuell (1,3), (6,3), (7,3); (3,3) ist belegt, (4,3) blockiert, (5,3) ist P. Ich setze als Nächstes S→(6,3), dann T→(7,3), danach suche ich eine freie Route für M→(3,3). Bitte sag Bescheid, falls du (6,3) oder (7,3) brauchst.} \\
            }
        }
    }
    & & \\ \\

    \theutterance \stepcounter{utterance}  
    & & \multicolumn{4}{p{0.6\linewidth}}{
        \cellcolor[rgb]{0.9,0.9,0.9}{
            \makecell[{{p{\linewidth}}}]{
                \texttt{\tiny{[P1$\langle$GM]}}
                \texttt{'T' erfolgreich nach (7,3) verschoben. Dein Raster sieht jetzt so aus:} \\
\texttt{```} \\
\texttt{ 1234567} \\
\texttt{╔═══╤═╤═╗} \\
\texttt{║AS◌│◌│◌║ 1} \\
\texttt{╟─┬─┼─┴─╢ 2} \\
\texttt{║◌│◌│◌◌T║ 3} \\
\texttt{║P├─┤D┌─╢ 4} \\
\texttt{╟─┘◌│◌│M║ 5} \\
\texttt{║◌◌◌└┬┴─╢ 6} \\
\texttt{║◌O◌◌│◌◌║ 7} \\
\texttt{╚════╧══╝} \\
\texttt{```} \\
\texttt{Ihr seid gerade bei Runde 12 von maximal 28.} \\
\texttt{Ihr habt gemeinsam 7 von maximal 16 Strafpunkten angesammelt.} \\
\texttt{Dein Mitspieler hat dir diese Nachricht geschickt:} \\
\\ 
\\ 
\texttt{"Korrektur: Bei mir steht D bereits auf (2,3) und blockiert angrenzend, M steht auf (7,7). Freie Ziele in Zeile 3 sind aktuell (1,3), (6,3), (7,3); (3,3) ist belegt, (4,3) blockiert, (5,3) ist P. Ich setze als Nächstes S→(6,3), dann T→(7,3), danach suche ich eine freie Route für M→(3,3). Bitte sag Bescheid, falls du (6,3) oder (7,3) brauchst."} \\
\\ 
\\ 
\texttt{Wie lautet dein nächster Befehl?} \\
            }
        }
    }
    & \\ \\

    \theutterance \stepcounter{utterance}  
    & \multicolumn{4}{p{0.6\linewidth}}{
        \cellcolor[rgb]{0.9,0.9,0.9}{
            \makecell[{{p{\linewidth}}}]{
                \texttt{\tiny{[P1$\rangle$GM]}}
                \texttt{SPRICH: Danke für das Update. Bei mir ist (7,3) bereits mit T belegt und bleibt dort korrekt. (6,3) ist frei; setz S gern dorthin. Ich mache weiter mit M→(3,3) als Nächstes, danach P→(5,3) und A→(1,3). O bleibt vorerst geparkt, da (4,3) blockiert. Sag Bescheid, falls du kurzfristig (3,3) oder (5,3) brauchst.} \\
            }
        }
    }
    & & \\ \\

    \theutterance \stepcounter{utterance}  
    & & & \multicolumn{2}{p{0.3\linewidth}}{
        \cellcolor[rgb]{0.9,0.9,0.9}{
            \makecell[{{p{\linewidth}}}]{
                \texttt{\tiny{[GM$|$GM]}}
                \texttt{SPRICH: Danke für das Update. Bei mir ist (7,3) bereits mit T belegt und bleibt dort korrekt. (6,3) ist frei; setz S gern dorthin. Ich mache weiter mit M→(3,3) als Nächstes, danach P→(5,3) und A→(1,3). O bleibt vorerst geparkt, da (4,3) blockiert. Sag Bescheid, falls du kurzfristig (3,3) oder (5,3) brauchst.} \\
            }
        }
    }
    & & \\ \\

    \theutterance \stepcounter{utterance}  
    & & \multicolumn{4}{p{0.6\linewidth}}{
        \cellcolor[rgb]{0.9,0.9,0.9}{
            \makecell[{{p{\linewidth}}}]{
                \texttt{\tiny{[GM$\rangle$P2]}}
                \texttt{Deine Nachricht wurde an deinen Mitspieler übermittelt.} \\
\\ 
\\ 
\texttt{Ihr seid gerade bei Runde 13 von maximal 28.} \\
\texttt{Ihr habt gemeinsam 7 von maximal 16 Strafpunkten angesammelt.} \\
\texttt{Dein Mitspieler hat dir diese Nachricht geschickt:} \\
\\ 
\\ 
\texttt{"Danke für das Update. Bei mir ist (7,3) bereits mit T belegt und bleibt dort korrekt. (6,3) ist frei; setz S gern dorthin. Ich mache weiter mit M→(3,3) als Nächstes, danach P→(5,3) und A→(1,3). O bleibt vorerst geparkt, da (4,3) blockiert. Sag Bescheid, falls du kurzfristig (3,3) oder (5,3) brauchst."} \\
\\ 
\\ 
\texttt{Wie lautet dein nächster Befehl?} \\
            }
        }
    }
    & \\ \\

    \theutterance \stepcounter{utterance}  
    & & & \multicolumn{4}{p{0.6\linewidth}}{
        \cellcolor[rgb]{0.9,0.9,0.9}{
            \makecell[{{p{\linewidth}}}]{
                \texttt{\tiny{[GM$\langle$P2]}}
                \texttt{VERSCHIEBE: S, (6, 3)} \\
            }
        }
    }
     \\ \\

    \theutterance \stepcounter{utterance}  
    & & & \multicolumn{2}{p{0.3\linewidth}}{
        \cellcolor[rgb]{0.9,0.9,0.9}{
            \makecell[{{p{\linewidth}}}]{
                \texttt{\tiny{[GM$|$GM]}}
                \texttt{VERSCHIEBE: S, (6, 3)} \\
            }
        }
    }
    & & \\ \\

    \theutterance \stepcounter{utterance}  
    & & & \multicolumn{2}{p{0.3\linewidth}}{
        \cellcolor[rgb]{0.9,0.9,0.9}{
            \makecell[{{p{\linewidth}}}]{
                \texttt{\tiny{[GM$|$GM]}}
                \texttt{'S' erfolgreich nach (6,3) verschoben. Dein Raster sieht jetzt so aus:} \\
\texttt{```} \\
\texttt{ 1234567} \\
\texttt{╔═══╤═╤═╗} \\
\texttt{║A◌◌│◌│◌║ 1} \\
\texttt{╟─┬─┼─┴─╢ 2} \\
\texttt{║◌│D│PS◌║ 3} \\
\texttt{║◌├─┤◌┌─╢ 4} \\
\texttt{╟─┘◌│◌│O║ 5} \\
\texttt{║◌◌◌└┬┴─╢ 6} \\
\texttt{║◌◌◌T│◌M║ 7} \\
\texttt{╚════╧══╝} \\
\texttt{```} \\
            }
        }
    }
    & & \\ \\

    \theutterance \stepcounter{utterance}  
    & & \multicolumn{4}{p{0.6\linewidth}}{
        \cellcolor[rgb]{0.9,0.9,0.9}{
            \makecell[{{p{\linewidth}}}]{
                \texttt{\tiny{[P1$\langle$GM]}}
                \texttt{Deine Nachricht wurde an deinen Mitspieler übermittelt.} \\
\\ 
\\ 
\texttt{Ihr seid gerade bei Runde 13 von maximal 28.} \\
\texttt{Ihr habt gemeinsam 7 von maximal 16 Strafpunkten angesammelt.} \\
\texttt{Dein Mitspieler hat ein Objekt auf seinem Raster bewegt.} \\
\\ 
\\ 
\texttt{Wie lautet dein nächster Befehl?} \\
            }
        }
    }
    & \\ \\

    \theutterance \stepcounter{utterance}  
    & \multicolumn{4}{p{0.6\linewidth}}{
        \cellcolor[rgb]{0.9,0.9,0.9}{
            \makecell[{{p{\linewidth}}}]{
                \texttt{\tiny{[P1$\rangle$GM]}}
                \texttt{VERSCHIEBE: M, (3,3)} \\
            }
        }
    }
    & & \\ \\

    \theutterance \stepcounter{utterance}  
    & & & \multicolumn{2}{p{0.3\linewidth}}{
        \cellcolor[rgb]{0.9,0.9,0.9}{
            \makecell[{{p{\linewidth}}}]{
                \texttt{\tiny{[GM$|$GM]}}
                \texttt{VERSCHIEBE: M, (3,3)} \\
            }
        }
    }
    & & \\ \\

    \theutterance \stepcounter{utterance}  
    & & & \multicolumn{2}{p{0.3\linewidth}}{
        \cellcolor[rgb]{0.9,0.9,0.9}{
            \makecell[{{p{\linewidth}}}]{
                \texttt{\tiny{[GM$|$GM]}}
                \texttt{'M' erfolgreich nach (3,3) verschoben. Dein Raster sieht jetzt so aus:} \\
\texttt{```} \\
\texttt{ 1234567} \\
\texttt{╔═══╤═╤═╗} \\
\texttt{║AS◌│◌│◌║ 1} \\
\texttt{╟─┬─┼─┴─╢ 2} \\
\texttt{║◌│M│◌◌T║ 3} \\
\texttt{║P├─┤D┌─╢ 4} \\
\texttt{╟─┘◌│◌│◌║ 5} \\
\texttt{║◌◌◌└┬┴─╢ 6} \\
\texttt{║◌O◌◌│◌◌║ 7} \\
\texttt{╚════╧══╝} \\
\texttt{```} \\
            }
        }
    }
    & & \\ \\

    \theutterance \stepcounter{utterance}  
    & & \multicolumn{4}{p{0.6\linewidth}}{
        \cellcolor[rgb]{0.9,0.9,0.9}{
            \makecell[{{p{\linewidth}}}]{
                \texttt{\tiny{[GM$\rangle$P2]}}
                \texttt{'S' erfolgreich nach (6,3) verschoben. Dein Raster sieht jetzt so aus:} \\
\texttt{```} \\
\texttt{ 1234567} \\
\texttt{╔═══╤═╤═╗} \\
\texttt{║A◌◌│◌│◌║ 1} \\
\texttt{╟─┬─┼─┴─╢ 2} \\
\texttt{║◌│D│PS◌║ 3} \\
\texttt{║◌├─┤◌┌─╢ 4} \\
\texttt{╟─┘◌│◌│O║ 5} \\
\texttt{║◌◌◌└┬┴─╢ 6} \\
\texttt{║◌◌◌T│◌M║ 7} \\
\texttt{╚════╧══╝} \\
\texttt{```} \\
\texttt{Ihr seid gerade bei Runde 14 von maximal 28.} \\
\texttt{Ihr habt gemeinsam 7 von maximal 16 Strafpunkten angesammelt.} \\
\texttt{Dein Mitspieler hat ein Objekt auf seinem Raster bewegt.} \\
\\ 
\\ 
\texttt{Wie lautet dein nächster Befehl?} \\
            }
        }
    }
    & \\ \\

    \theutterance \stepcounter{utterance}  
    & & & \multicolumn{4}{p{0.6\linewidth}}{
        \cellcolor[rgb]{0.9,0.9,0.9}{
            \makecell[{{p{\linewidth}}}]{
                \texttt{\tiny{[GM$\langle$P2]}}
                \texttt{VERSCHIEBE: T, (7, 3)} \\
            }
        }
    }
     \\ \\

    \theutterance \stepcounter{utterance}  
    & & & \multicolumn{2}{p{0.3\linewidth}}{
        \cellcolor[rgb]{0.9,0.9,0.9}{
            \makecell[{{p{\linewidth}}}]{
                \texttt{\tiny{[GM$|$GM]}}
                \texttt{VERSCHIEBE: T, (7, 3)} \\
            }
        }
    }
    & & \\ \\

    \theutterance \stepcounter{utterance}  
    & & & \multicolumn{2}{p{0.3\linewidth}}{
        \cellcolor[rgb]{0.9,0.9,0.9}{
            \makecell[{{p{\linewidth}}}]{
                \texttt{\tiny{[GM$|$GM]}}
                \texttt{'T' erfolgreich nach (7,3) verschoben. Dein Raster sieht jetzt so aus:} \\
\texttt{```} \\
\texttt{ 1234567} \\
\texttt{╔═══╤═╤═╗} \\
\texttt{║A◌◌│◌│◌║ 1} \\
\texttt{╟─┬─┼─┴─╢ 2} \\
\texttt{║◌│D│PST║ 3} \\
\texttt{║◌├─┤◌┌─╢ 4} \\
\texttt{╟─┘◌│◌│O║ 5} \\
\texttt{║◌◌◌└┬┴─╢ 6} \\
\texttt{║◌◌◌◌│◌M║ 7} \\
\texttt{╚════╧══╝} \\
\texttt{```} \\
            }
        }
    }
    & & \\ \\

    \theutterance \stepcounter{utterance}  
    & & \multicolumn{4}{p{0.6\linewidth}}{
        \cellcolor[rgb]{0.9,0.9,0.9}{
            \makecell[{{p{\linewidth}}}]{
                \texttt{\tiny{[P1$\langle$GM]}}
                \texttt{'M' erfolgreich nach (3,3) verschoben. Dein Raster sieht jetzt so aus:} \\
\texttt{```} \\
\texttt{ 1234567} \\
\texttt{╔═══╤═╤═╗} \\
\texttt{║AS◌│◌│◌║ 1} \\
\texttt{╟─┬─┼─┴─╢ 2} \\
\texttt{║◌│M│◌◌T║ 3} \\
\texttt{║P├─┤D┌─╢ 4} \\
\texttt{╟─┘◌│◌│◌║ 5} \\
\texttt{║◌◌◌└┬┴─╢ 6} \\
\texttt{║◌O◌◌│◌◌║ 7} \\
\texttt{╚════╧══╝} \\
\texttt{```} \\
\texttt{Ihr seid gerade bei Runde 14 von maximal 28.} \\
\texttt{Ihr habt gemeinsam 7 von maximal 16 Strafpunkten angesammelt.} \\
\texttt{Dein Mitspieler hat ein Objekt auf seinem Raster bewegt.} \\
\\ 
\\ 
\texttt{Wie lautet dein nächster Befehl?} \\
            }
        }
    }
    & \\ \\

    \theutterance \stepcounter{utterance}  
    & \multicolumn{4}{p{0.6\linewidth}}{
        \cellcolor[rgb]{0.9,0.9,0.9}{
            \makecell[{{p{\linewidth}}}]{
                \texttt{\tiny{[P1$\rangle$GM]}}
                \texttt{VERSCHIEBE: P, (5,3)} \\
            }
        }
    }
    & & \\ \\

    \theutterance \stepcounter{utterance}  
    & & & \multicolumn{2}{p{0.3\linewidth}}{
        \cellcolor[rgb]{0.9,0.9,0.9}{
            \makecell[{{p{\linewidth}}}]{
                \texttt{\tiny{[GM$|$GM]}}
                \texttt{VERSCHIEBE: P, (5,3)} \\
            }
        }
    }
    & & \\ \\

    \theutterance \stepcounter{utterance}  
    & & & \multicolumn{2}{p{0.3\linewidth}}{
        \cellcolor[rgb]{0.9,0.9,0.9}{
            \makecell[{{p{\linewidth}}}]{
                \texttt{\tiny{[GM$|$GM]}}
                \texttt{'P' erfolgreich nach (5,3) verschoben. Dein Raster sieht jetzt so aus:} \\
\texttt{```} \\
\texttt{ 1234567} \\
\texttt{╔═══╤═╤═╗} \\
\texttt{║AS◌│◌│◌║ 1} \\
\texttt{╟─┬─┼─┴─╢ 2} \\
\texttt{║◌│M│P◌T║ 3} \\
\texttt{║◌├─┤D┌─╢ 4} \\
\texttt{╟─┘◌│◌│◌║ 5} \\
\texttt{║◌◌◌└┬┴─╢ 6} \\
\texttt{║◌O◌◌│◌◌║ 7} \\
\texttt{╚════╧══╝} \\
\texttt{```} \\
            }
        }
    }
    & & \\ \\

    \theutterance \stepcounter{utterance}  
    & & \multicolumn{4}{p{0.6\linewidth}}{
        \cellcolor[rgb]{0.9,0.9,0.9}{
            \makecell[{{p{\linewidth}}}]{
                \texttt{\tiny{[GM$\rangle$P2]}}
                \texttt{'T' erfolgreich nach (7,3) verschoben. Dein Raster sieht jetzt so aus:} \\
\texttt{```} \\
\texttt{ 1234567} \\
\texttt{╔═══╤═╤═╗} \\
\texttt{║A◌◌│◌│◌║ 1} \\
\texttt{╟─┬─┼─┴─╢ 2} \\
\texttt{║◌│D│PST║ 3} \\
\texttt{║◌├─┤◌┌─╢ 4} \\
\texttt{╟─┘◌│◌│O║ 5} \\
\texttt{║◌◌◌└┬┴─╢ 6} \\
\texttt{║◌◌◌◌│◌M║ 7} \\
\texttt{╚════╧══╝} \\
\texttt{```} \\
\texttt{Ihr seid gerade bei Runde 15 von maximal 28.} \\
\texttt{Ihr habt gemeinsam 7 von maximal 16 Strafpunkten angesammelt.} \\
\texttt{Dein Mitspieler hat ein Objekt auf seinem Raster bewegt.} \\
\\ 
\\ 
\texttt{Wie lautet dein nächster Befehl?} \\
            }
        }
    }
    & \\ \\

    \theutterance \stepcounter{utterance}  
    & & & \multicolumn{4}{p{0.6\linewidth}}{
        \cellcolor[rgb]{0.9,0.9,0.9}{
            \makecell[{{p{\linewidth}}}]{
                \texttt{\tiny{[GM$\langle$P2]}}
                \texttt{SPRICH: Update: Bei mir stehen jetzt P(5,3), S(6,3), T(7,3). (4,3) ist weiterhin blockiert, (3,3) ist frei, (1,3) ist frei, (2,3) blockiert. Ich setze als Nächstes A→(1,3). Danach plane ich M→(3,3). Brauchst du kurzfristig (1,3) oder (3,3)?} \\
            }
        }
    }
     \\ \\

    \theutterance \stepcounter{utterance}  
    & & & \multicolumn{2}{p{0.3\linewidth}}{
        \cellcolor[rgb]{0.9,0.9,0.9}{
            \makecell[{{p{\linewidth}}}]{
                \texttt{\tiny{[GM$|$GM]}}
                \texttt{SPRICH: Update: Bei mir stehen jetzt P(5,3), S(6,3), T(7,3). (4,3) ist weiterhin blockiert, (3,3) ist frei, (1,3) ist frei, (2,3) blockiert. Ich setze als Nächstes A→(1,3). Danach plane ich M→(3,3). Brauchst du kurzfristig (1,3) oder (3,3)?} \\
            }
        }
    }
    & & \\ \\

    \theutterance \stepcounter{utterance}  
    & & \multicolumn{4}{p{0.6\linewidth}}{
        \cellcolor[rgb]{0.9,0.9,0.9}{
            \makecell[{{p{\linewidth}}}]{
                \texttt{\tiny{[P1$\langle$GM]}}
                \texttt{'P' erfolgreich nach (5,3) verschoben. Dein Raster sieht jetzt so aus:} \\
\texttt{```} \\
\texttt{ 1234567} \\
\texttt{╔═══╤═╤═╗} \\
\texttt{║AS◌│◌│◌║ 1} \\
\texttt{╟─┬─┼─┴─╢ 2} \\
\texttt{║◌│M│P◌T║ 3} \\
\texttt{║◌├─┤D┌─╢ 4} \\
\texttt{╟─┘◌│◌│◌║ 5} \\
\texttt{║◌◌◌└┬┴─╢ 6} \\
\texttt{║◌O◌◌│◌◌║ 7} \\
\texttt{╚════╧══╝} \\
\texttt{```} \\
\texttt{Ihr seid gerade bei Runde 15 von maximal 28.} \\
\texttt{Ihr habt gemeinsam 7 von maximal 16 Strafpunkten angesammelt.} \\
\texttt{Dein Mitspieler hat dir diese Nachricht geschickt:} \\
\\ 
\\ 
\texttt{"Update: Bei mir stehen jetzt P(5,3), S(6,3), T(7,3). (4,3) ist weiterhin blockiert, (3,3) ist frei, (1,3) ist frei, (2,3) blockiert. Ich setze als Nächstes A→(1,3). Danach plane ich M→(3,3). Brauchst du kurzfristig (1,3) oder (3,3)?"} \\
\\ 
\\ 
\texttt{Wie lautet dein nächster Befehl?} \\
            }
        }
    }
    & \\ \\

    \theutterance \stepcounter{utterance}  
    & \multicolumn{4}{p{0.6\linewidth}}{
        \cellcolor[rgb]{0.9,0.9,0.9}{
            \makecell[{{p{\linewidth}}}]{
                \texttt{\tiny{[P1$\rangle$GM]}}
                \texttt{VERSCHIEBE: A, (1,3)} \\
            }
        }
    }
    & & \\ \\

    \theutterance \stepcounter{utterance}  
    & & & \multicolumn{2}{p{0.3\linewidth}}{
        \cellcolor[rgb]{0.9,0.9,0.9}{
            \makecell[{{p{\linewidth}}}]{
                \texttt{\tiny{[GM$|$GM]}}
                \texttt{VERSCHIEBE: A, (1,3)} \\
            }
        }
    }
    & & \\ \\

    \theutterance \stepcounter{utterance}  
    & & & \multicolumn{2}{p{0.3\linewidth}}{
        \cellcolor[rgb]{0.9,0.9,0.9}{
            \makecell[{{p{\linewidth}}}]{
                \texttt{\tiny{[GM$|$GM]}}
                \texttt{'A' erfolgreich nach (1,3) verschoben. Dein Raster sieht jetzt so aus:} \\
\texttt{```} \\
\texttt{ 1234567} \\
\texttt{╔═══╤═╤═╗} \\
\texttt{║◌S◌│◌│◌║ 1} \\
\texttt{╟─┬─┼─┴─╢ 2} \\
\texttt{║A│M│P◌T║ 3} \\
\texttt{║◌├─┤D┌─╢ 4} \\
\texttt{╟─┘◌│◌│◌║ 5} \\
\texttt{║◌◌◌└┬┴─╢ 6} \\
\texttt{║◌O◌◌│◌◌║ 7} \\
\texttt{╚════╧══╝} \\
\texttt{```} \\
            }
        }
    }
    & & \\ \\

    \theutterance \stepcounter{utterance}  
    & & \multicolumn{4}{p{0.6\linewidth}}{
        \cellcolor[rgb]{0.9,0.9,0.9}{
            \makecell[{{p{\linewidth}}}]{
                \texttt{\tiny{[GM$\rangle$P2]}}
                \texttt{Deine Nachricht wurde an deinen Mitspieler übermittelt.} \\
\\ 
\\ 
\texttt{Ihr seid gerade bei Runde 16 von maximal 28.} \\
\texttt{Ihr habt gemeinsam 7 von maximal 16 Strafpunkten angesammelt.} \\
\texttt{Dein Mitspieler hat ein Objekt auf seinem Raster bewegt.} \\
\\ 
\\ 
\texttt{Wie lautet dein nächster Befehl?} \\
            }
        }
    }
    & \\ \\

    \theutterance \stepcounter{utterance}  
    & & & \multicolumn{4}{p{0.6\linewidth}}{
        \cellcolor[rgb]{0.9,0.9,0.9}{
            \makecell[{{p{\linewidth}}}]{
                \texttt{\tiny{[GM$\langle$P2]}}
                \texttt{VERSCHIEBE: A, (1, 3)} \\
            }
        }
    }
     \\ \\

    \theutterance \stepcounter{utterance}  
    & & & \multicolumn{2}{p{0.3\linewidth}}{
        \cellcolor[rgb]{0.9,0.9,0.9}{
            \makecell[{{p{\linewidth}}}]{
                \texttt{\tiny{[GM$|$GM]}}
                \texttt{VERSCHIEBE: A, (1, 3)} \\
            }
        }
    }
    & & \\ \\

    \theutterance \stepcounter{utterance}  
    & & & \multicolumn{2}{p{0.3\linewidth}}{
        \cellcolor[rgb]{0.9,0.9,0.9}{
            \makecell[{{p{\linewidth}}}]{
                \texttt{\tiny{[GM$|$GM]}}
                \texttt{'A' erfolgreich nach (1,3) verschoben. Dein Raster sieht jetzt so aus:} \\
\texttt{```} \\
\texttt{ 1234567} \\
\texttt{╔═══╤═╤═╗} \\
\texttt{║◌◌◌│◌│◌║ 1} \\
\texttt{╟─┬─┼─┴─╢ 2} \\
\texttt{║A│D│PST║ 3} \\
\texttt{║◌├─┤◌┌─╢ 4} \\
\texttt{╟─┘◌│◌│O║ 5} \\
\texttt{║◌◌◌└┬┴─╢ 6} \\
\texttt{║◌◌◌◌│◌M║ 7} \\
\texttt{╚════╧══╝} \\
\texttt{```} \\
            }
        }
    }
    & & \\ \\

    \theutterance \stepcounter{utterance}  
    & & \multicolumn{4}{p{0.6\linewidth}}{
        \cellcolor[rgb]{0.9,0.9,0.9}{
            \makecell[{{p{\linewidth}}}]{
                \texttt{\tiny{[P1$\langle$GM]}}
                \texttt{'A' erfolgreich nach (1,3) verschoben. Dein Raster sieht jetzt so aus:} \\
\texttt{```} \\
\texttt{ 1234567} \\
\texttt{╔═══╤═╤═╗} \\
\texttt{║◌S◌│◌│◌║ 1} \\
\texttt{╟─┬─┼─┴─╢ 2} \\
\texttt{║A│M│P◌T║ 3} \\
\texttt{║◌├─┤D┌─╢ 4} \\
\texttt{╟─┘◌│◌│◌║ 5} \\
\texttt{║◌◌◌└┬┴─╢ 6} \\
\texttt{║◌O◌◌│◌◌║ 7} \\
\texttt{╚════╧══╝} \\
\texttt{```} \\
\texttt{Ihr seid gerade bei Runde 16 von maximal 28.} \\
\texttt{Ihr habt gemeinsam 7 von maximal 16 Strafpunkten angesammelt.} \\
\texttt{Dein Mitspieler hat ein Objekt auf seinem Raster bewegt.} \\
\\ 
\\ 
\texttt{Wie lautet dein nächster Befehl?} \\
            }
        }
    }
    & \\ \\

    \theutterance \stepcounter{utterance}  
    & \multicolumn{4}{p{0.6\linewidth}}{
        \cellcolor[rgb]{0.9,0.9,0.9}{
            \makecell[{{p{\linewidth}}}]{
                \texttt{\tiny{[P1$\rangle$GM]}}
                \texttt{SPRICH: Bei mir stehen jetzt A(1,3), M(3,3), P(5,3), T(7,3); (4,3) bleibt blockiert. Ich setze als Nächstes S→(6,3). Danach bleibt nur O, das wir wegen (4,3) blockiert später klären. Ist (6,3) bei dir weiterhin frei?} \\
            }
        }
    }
    & & \\ \\

    \theutterance \stepcounter{utterance}  
    & & & \multicolumn{2}{p{0.3\linewidth}}{
        \cellcolor[rgb]{0.9,0.9,0.9}{
            \makecell[{{p{\linewidth}}}]{
                \texttt{\tiny{[GM$|$GM]}}
                \texttt{SPRICH: Bei mir stehen jetzt A(1,3), M(3,3), P(5,3), T(7,3); (4,3) bleibt blockiert. Ich setze als Nächstes S→(6,3). Danach bleibt nur O, das wir wegen (4,3) blockiert später klären. Ist (6,3) bei dir weiterhin frei?} \\
            }
        }
    }
    & & \\ \\

    \theutterance \stepcounter{utterance}  
    & & \multicolumn{4}{p{0.6\linewidth}}{
        \cellcolor[rgb]{0.9,0.9,0.9}{
            \makecell[{{p{\linewidth}}}]{
                \texttt{\tiny{[GM$\rangle$P2]}}
                \texttt{'A' erfolgreich nach (1,3) verschoben. Dein Raster sieht jetzt so aus:} \\
\texttt{```} \\
\texttt{ 1234567} \\
\texttt{╔═══╤═╤═╗} \\
\texttt{║◌◌◌│◌│◌║ 1} \\
\texttt{╟─┬─┼─┴─╢ 2} \\
\texttt{║A│D│PST║ 3} \\
\texttt{║◌├─┤◌┌─╢ 4} \\
\texttt{╟─┘◌│◌│O║ 5} \\
\texttt{║◌◌◌└┬┴─╢ 6} \\
\texttt{║◌◌◌◌│◌M║ 7} \\
\texttt{╚════╧══╝} \\
\texttt{```} \\
\texttt{Ihr seid gerade bei Runde 17 von maximal 28.} \\
\texttt{Ihr habt gemeinsam 7 von maximal 16 Strafpunkten angesammelt.} \\
\texttt{Dein Mitspieler hat dir diese Nachricht geschickt:} \\
\\ 
\\ 
\texttt{"Bei mir stehen jetzt A(1,3), M(3,3), P(5,3), T(7,3); (4,3) bleibt blockiert. Ich setze als Nächstes S→(6,3). Danach bleibt nur O, das wir wegen (4,3) blockiert später klären. Ist (6,3) bei dir weiterhin frei?"} \\
\\ 
\\ 
\texttt{Wie lautet dein nächster Befehl?} \\
            }
        }
    }
    & \\ \\

    \theutterance \stepcounter{utterance}  
    & & & \multicolumn{4}{p{0.6\linewidth}}{
        \cellcolor[rgb]{0.9,0.9,0.9}{
            \makecell[{{p{\linewidth}}}]{
                \texttt{\tiny{[GM$\langle$P2]}}
                \texttt{SPRICH: Ja, (6,3) ist bei mir bereits mit S belegt und passt. Bei mir stehen nun A(1,3), P(5,3), S(6,3), T(7,3); (3,3) ist frei für M, (4,3) bleibt blockiert. Ich setze als Nächstes M→(3,3). Falls okay, sag Bescheid, wann wir O gemeinsam lösen (z. B. späterer gemeinsamer Wechsel der Zielposition für O).} \\
            }
        }
    }
     \\ \\

    \theutterance \stepcounter{utterance}  
    & & & \multicolumn{2}{p{0.3\linewidth}}{
        \cellcolor[rgb]{0.9,0.9,0.9}{
            \makecell[{{p{\linewidth}}}]{
                \texttt{\tiny{[GM$|$GM]}}
                \texttt{SPRICH: Ja, (6,3) ist bei mir bereits mit S belegt und passt. Bei mir stehen nun A(1,3), P(5,3), S(6,3), T(7,3); (3,3) ist frei für M, (4,3) bleibt blockiert. Ich setze als Nächstes M→(3,3). Falls okay, sag Bescheid, wann wir O gemeinsam lösen (z. B. späterer gemeinsamer Wechsel der Zielposition für O).} \\
            }
        }
    }
    & & \\ \\

    \theutterance \stepcounter{utterance}  
    & & \multicolumn{4}{p{0.6\linewidth}}{
        \cellcolor[rgb]{0.9,0.9,0.9}{
            \makecell[{{p{\linewidth}}}]{
                \texttt{\tiny{[P1$\langle$GM]}}
                \texttt{Deine Nachricht wurde an deinen Mitspieler übermittelt.} \\
\\ 
\\ 
\texttt{Ihr seid gerade bei Runde 17 von maximal 28.} \\
\texttt{Ihr habt gemeinsam 7 von maximal 16 Strafpunkten angesammelt.} \\
\texttt{Dein Mitspieler hat dir diese Nachricht geschickt:} \\
\\ 
\\ 
\texttt{"Ja, (6,3) ist bei mir bereits mit S belegt und passt. Bei mir stehen nun A(1,3), P(5,3), S(6,3), T(7,3); (3,3) ist frei für M, (4,3) bleibt blockiert. Ich setze als Nächstes M→(3,3). Falls okay, sag Bescheid, wann wir O gemeinsam lösen (z. B. späterer gemeinsamer Wechsel der Zielposition für O)."} \\
\\ 
\\ 
\texttt{Wie lautet dein nächster Befehl?} \\
            }
        }
    }
    & \\ \\

    \theutterance \stepcounter{utterance}  
    & \multicolumn{4}{p{0.6\linewidth}}{
        \cellcolor[rgb]{0.9,0.9,0.9}{
            \makecell[{{p{\linewidth}}}]{
                \texttt{\tiny{[P1$\rangle$GM]}}
                \texttt{VERSCHIEBE: S, (6,3)} \\
            }
        }
    }
    & & \\ \\

    \theutterance \stepcounter{utterance}  
    & & & \multicolumn{2}{p{0.3\linewidth}}{
        \cellcolor[rgb]{0.9,0.9,0.9}{
            \makecell[{{p{\linewidth}}}]{
                \texttt{\tiny{[GM$|$GM]}}
                \texttt{VERSCHIEBE: S, (6,3)} \\
            }
        }
    }
    & & \\ \\

    \theutterance \stepcounter{utterance}  
    & & & \multicolumn{2}{p{0.3\linewidth}}{
        \cellcolor[rgb]{0.9,0.9,0.9}{
            \makecell[{{p{\linewidth}}}]{
                \texttt{\tiny{[GM$|$GM]}}
                \texttt{'S' erfolgreich nach (6,3) verschoben. Dein Raster sieht jetzt so aus:} \\
\texttt{```} \\
\texttt{ 1234567} \\
\texttt{╔═══╤═╤═╗} \\
\texttt{║◌◌◌│◌│◌║ 1} \\
\texttt{╟─┬─┼─┴─╢ 2} \\
\texttt{║A│M│PST║ 3} \\
\texttt{║◌├─┤D┌─╢ 4} \\
\texttt{╟─┘◌│◌│◌║ 5} \\
\texttt{║◌◌◌└┬┴─╢ 6} \\
\texttt{║◌O◌◌│◌◌║ 7} \\
\texttt{╚════╧══╝} \\
\texttt{```} \\
            }
        }
    }
    & & \\ \\

    \theutterance \stepcounter{utterance}  
    & & \multicolumn{4}{p{0.6\linewidth}}{
        \cellcolor[rgb]{0.9,0.9,0.9}{
            \makecell[{{p{\linewidth}}}]{
                \texttt{\tiny{[GM$\rangle$P2]}}
                \texttt{Deine Nachricht wurde an deinen Mitspieler übermittelt.} \\
\\ 
\\ 
\texttt{Ihr seid gerade bei Runde 18 von maximal 28.} \\
\texttt{Ihr habt gemeinsam 7 von maximal 16 Strafpunkten angesammelt.} \\
\texttt{Dein Mitspieler hat ein Objekt auf seinem Raster bewegt.} \\
\\ 
\\ 
\texttt{Wie lautet dein nächster Befehl?} \\
            }
        }
    }
    & \\ \\

    \theutterance \stepcounter{utterance}  
    & & & \multicolumn{4}{p{0.6\linewidth}}{
        \cellcolor[rgb]{0.9,0.9,0.9}{
            \makecell[{{p{\linewidth}}}]{
                \texttt{\tiny{[GM$\langle$P2]}}
                \texttt{VERSCHIEBE: M, (3, 3)} \\
            }
        }
    }
     \\ \\

    \theutterance \stepcounter{utterance}  
    & & & \multicolumn{2}{p{0.3\linewidth}}{
        \cellcolor[rgb]{0.9,0.9,0.9}{
            \makecell[{{p{\linewidth}}}]{
                \texttt{\tiny{[GM$|$GM]}}
                \texttt{VERSCHIEBE: M, (3, 3)} \\
            }
        }
    }
    & & \\ \\

    \theutterance \stepcounter{utterance}  
    & & & \multicolumn{2}{p{0.3\linewidth}}{
        \cellcolor[rgb]{0.9,0.9,0.9}{
            \makecell[{{p{\linewidth}}}]{
                \texttt{\tiny{[GM$|$GM]}}
                \texttt{Strafpunkt: (3,3) ist nicht leer, sondern enthält 'D'.} \\
\\ 
\texttt{Ihr habt gemeinsam 8 von maximal 16 Strafpunkten angesammelt.} \\
\texttt{Bitte versuche es noch mal!} \\
            }
        }
    }
    & & \\ \\

    \theutterance \stepcounter{utterance}  
    & & \multicolumn{4}{p{0.6\linewidth}}{
        \cellcolor[rgb]{0.9,0.9,0.9}{
            \makecell[{{p{\linewidth}}}]{
                \texttt{\tiny{[GM$\rangle$P2]}}
                \texttt{Strafpunkt: (3,3) ist nicht leer, sondern enthält 'D'.} \\
\\ 
\texttt{Ihr habt gemeinsam 8 von maximal 16 Strafpunkten angesammelt.} \\
\texttt{Bitte versuche es noch mal!} \\
            }
        }
    }
    & \\ \\

    \theutterance \stepcounter{utterance}  
    & & & \multicolumn{4}{p{0.6\linewidth}}{
        \cellcolor[rgb]{0.9,0.9,0.9}{
            \makecell[{{p{\linewidth}}}]{
                \texttt{\tiny{[GM$\langle$P2]}}
                \texttt{SPRICH: Korrektur: Bei mir steht D dauerhaft bei (2,3) und blockiert daneben; (3,3) ist also nicht frei. Aktueller Stand: A(1,3), D(2,3), P(5,3), S(6,3), T(7,3); frei sind (3,3) nicht, (4,3) blockiert. Vorschlag: Wir ändern nur O’s Ziel auf (4,2) als gemeinsamen Fixpunkt (da (4,3) blockiert bleibt) und lassen D bei mir auf (2,3). Bei dir könntest du O ebenfalls auf (4,2) setzen, falls frei; sonst sag eine alternative gemeinsame Koordinate für O (z. B. (4,4)). Ich warte mit weiteren Zügen auf deine Bestätigung.} \\
            }
        }
    }
     \\ \\

    \theutterance \stepcounter{utterance}  
    & & & \multicolumn{2}{p{0.3\linewidth}}{
        \cellcolor[rgb]{0.9,0.9,0.9}{
            \makecell[{{p{\linewidth}}}]{
                \texttt{\tiny{[GM$|$GM]}}
                \texttt{SPRICH: Korrektur: Bei mir steht D dauerhaft bei (2,3) und blockiert daneben; (3,3) ist also nicht frei. Aktueller Stand: A(1,3), D(2,3), P(5,3), S(6,3), T(7,3); frei sind (3,3) nicht, (4,3) blockiert. Vorschlag: Wir ändern nur O’s Ziel auf (4,2) als gemeinsamen Fixpunkt (da (4,3) blockiert bleibt) und lassen D bei mir auf (2,3). Bei dir könntest du O ebenfalls auf (4,2) setzen, falls frei; sonst sag eine alternative gemeinsame Koordinate für O (z. B. (4,4)). Ich warte mit weiteren Zügen auf deine Bestätigung.} \\
            }
        }
    }
    & & \\ \\

    \theutterance \stepcounter{utterance}  
    & & \multicolumn{4}{p{0.6\linewidth}}{
        \cellcolor[rgb]{0.9,0.9,0.9}{
            \makecell[{{p{\linewidth}}}]{
                \texttt{\tiny{[P1$\langle$GM]}}
                \texttt{'S' erfolgreich nach (6,3) verschoben. Dein Raster sieht jetzt so aus:} \\
\texttt{```} \\
\texttt{ 1234567} \\
\texttt{╔═══╤═╤═╗} \\
\texttt{║◌◌◌│◌│◌║ 1} \\
\texttt{╟─┬─┼─┴─╢ 2} \\
\texttt{║A│M│PST║ 3} \\
\texttt{║◌├─┤D┌─╢ 4} \\
\texttt{╟─┘◌│◌│◌║ 5} \\
\texttt{║◌◌◌└┬┴─╢ 6} \\
\texttt{║◌O◌◌│◌◌║ 7} \\
\texttt{╚════╧══╝} \\
\texttt{```} \\
\texttt{Ihr seid gerade bei Runde 18 von maximal 28.} \\
\texttt{Ihr habt gemeinsam 8 von maximal 16 Strafpunkten angesammelt.} \\
\texttt{Dein Mitspieler hat dir diese Nachricht geschickt:} \\
\\ 
\\ 
\texttt{"Korrektur: Bei mir steht D dauerhaft bei (2,3) und blockiert daneben; (3,3) ist also nicht frei. Aktueller Stand: A(1,3), D(2,3), P(5,3), S(6,3), T(7,3); frei sind (3,3) nicht, (4,3) blockiert. Vorschlag: Wir ändern nur O’s Ziel auf (4,2) als gemeinsamen Fixpunkt (da (4,3) blockiert bleibt) und lassen D bei mir auf (2,3). Bei dir könntest du O ebenfalls auf (4,2) setzen, falls frei; sonst sag eine alternative gemeinsame Koordinate für O (z. B. (4,4)). Ich warte mit weiteren Zügen auf deine Bestätigung."} \\
\\ 
\\ 
\texttt{Wie lautet dein nächster Befehl?} \\
            }
        }
    }
    & \\ \\

    \theutterance \stepcounter{utterance}  
    & \multicolumn{4}{p{0.6\linewidth}}{
        \cellcolor[rgb]{0.9,0.9,0.9}{
            \makecell[{{p{\linewidth}}}]{
                \texttt{\tiny{[P1$\rangle$GM]}}
                \texttt{SPRICH: Einverstanden, setzen wir O gemeinsam auf (4,2) als Ziel. Bei mir ist (4,2) frei. Ich werde O dorthin verschieben. Bitte bestätige, ob (4,2) bei dir ebenfalls frei ist bzw. ob du O schon dort platzierst. Sobald O steht, prüfen wir den Endstand: A(1,3), D(2,3), M(3,3) bei mir ist ok, P(5,3), S(6,3), T(7,3), und O(4,2). Wenn das bei dir matcht (mit D fix auf (2,3)), können wir aufräumen und „fertig?“ prüfen.} \\
            }
        }
    }
    & & \\ \\

    \theutterance \stepcounter{utterance}  
    & & & \multicolumn{2}{p{0.3\linewidth}}{
        \cellcolor[rgb]{0.9,0.9,0.9}{
            \makecell[{{p{\linewidth}}}]{
                \texttt{\tiny{[GM$|$GM]}}
                \texttt{SPRICH: Einverstanden, setzen wir O gemeinsam auf (4,2) als Ziel. Bei mir ist (4,2) frei. Ich werde O dorthin verschieben. Bitte bestätige, ob (4,2) bei dir ebenfalls frei ist bzw. ob du O schon dort platzierst. Sobald O steht, prüfen wir den Endstand: A(1,3), D(2,3), M(3,3) bei mir ist ok, P(5,3), S(6,3), T(7,3), und O(4,2). Wenn das bei dir matcht (mit D fix auf (2,3)), können wir aufräumen und „fertig?“ prüfen.} \\
            }
        }
    }
    & & \\ \\

    \theutterance \stepcounter{utterance}  
    & & \multicolumn{4}{p{0.6\linewidth}}{
        \cellcolor[rgb]{0.9,0.9,0.9}{
            \makecell[{{p{\linewidth}}}]{
                \texttt{\tiny{[GM$\rangle$P2]}}
                \texttt{Deine Nachricht wurde an deinen Mitspieler übermittelt.} \\
\\ 
\\ 
\texttt{Ihr seid gerade bei Runde 19 von maximal 28.} \\
\texttt{Ihr habt gemeinsam 8 von maximal 16 Strafpunkten angesammelt.} \\
\texttt{Dein Mitspieler hat dir diese Nachricht geschickt:} \\
\\ 
\\ 
\texttt{"Einverstanden, setzen wir O gemeinsam auf (4,2) als Ziel. Bei mir ist (4,2) frei. Ich werde O dorthin verschieben. Bitte bestätige, ob (4,2) bei dir ebenfalls frei ist bzw. ob du O schon dort platzierst. Sobald O steht, prüfen wir den Endstand: A(1,3), D(2,3), M(3,3) bei mir ist ok, P(5,3), S(6,3), T(7,3), und O(4,2). Wenn das bei dir matcht (mit D fix auf (2,3)), können wir aufräumen und „fertig?“ prüfen."} \\
\\ 
\\ 
\texttt{Wie lautet dein nächster Befehl?} \\
            }
        }
    }
    & \\ \\

    \theutterance \stepcounter{utterance}  
    & & & \multicolumn{4}{p{0.6\linewidth}}{
        \cellcolor[rgb]{0.9,0.9,0.9}{
            \makecell[{{p{\linewidth}}}]{
                \texttt{\tiny{[GM$\langle$P2]}}
                \texttt{SPRICH: Bestätigt, (4,2) ist bei mir frei. Ich setze O jetzt auf (4,2). Danach ist mein Stand: A(1,3), D(2,3), P(5,3), S(6,3), T(7,3), O(4,2); M bleibt abweichend, da (3,3) bei mir blockiert ist. Passt für dich, wenn wir M als einzigen Unterschied akzeptieren, oder willst du eine alternative gemeinsame Koordinate für M vorschlagen?} \\
            }
        }
    }
     \\ \\

    \theutterance \stepcounter{utterance}  
    & & & \multicolumn{2}{p{0.3\linewidth}}{
        \cellcolor[rgb]{0.9,0.9,0.9}{
            \makecell[{{p{\linewidth}}}]{
                \texttt{\tiny{[GM$|$GM]}}
                \texttt{SPRICH: Bestätigt, (4,2) ist bei mir frei. Ich setze O jetzt auf (4,2). Danach ist mein Stand: A(1,3), D(2,3), P(5,3), S(6,3), T(7,3), O(4,2); M bleibt abweichend, da (3,3) bei mir blockiert ist. Passt für dich, wenn wir M als einzigen Unterschied akzeptieren, oder willst du eine alternative gemeinsame Koordinate für M vorschlagen?} \\
            }
        }
    }
    & & \\ \\

    \theutterance \stepcounter{utterance}  
    & & \multicolumn{4}{p{0.6\linewidth}}{
        \cellcolor[rgb]{0.9,0.9,0.9}{
            \makecell[{{p{\linewidth}}}]{
                \texttt{\tiny{[P1$\langle$GM]}}
                \texttt{Deine Nachricht wurde an deinen Mitspieler übermittelt.} \\
\\ 
\\ 
\texttt{Ihr seid gerade bei Runde 19 von maximal 28.} \\
\texttt{Ihr habt gemeinsam 8 von maximal 16 Strafpunkten angesammelt.} \\
\texttt{Dein Mitspieler hat dir diese Nachricht geschickt:} \\
\\ 
\\ 
\texttt{"Bestätigt, (4,2) ist bei mir frei. Ich setze O jetzt auf (4,2). Danach ist mein Stand: A(1,3), D(2,3), P(5,3), S(6,3), T(7,3), O(4,2); M bleibt abweichend, da (3,3) bei mir blockiert ist. Passt für dich, wenn wir M als einzigen Unterschied akzeptieren, oder willst du eine alternative gemeinsame Koordinate für M vorschlagen?"} \\
\\ 
\\ 
\texttt{Wie lautet dein nächster Befehl?} \\
            }
        }
    }
    & \\ \\

    \theutterance \stepcounter{utterance}  
    & \multicolumn{4}{p{0.6\linewidth}}{
        \cellcolor[rgb]{0.9,0.9,0.9}{
            \makecell[{{p{\linewidth}}}]{
                \texttt{\tiny{[P1$\rangle$GM]}}
                \texttt{VERSCHIEBE: O, (4,2)} \\
            }
        }
    }
    & & \\ \\

    \theutterance \stepcounter{utterance}  
    & & & \multicolumn{2}{p{0.3\linewidth}}{
        \cellcolor[rgb]{0.9,0.9,0.9}{
            \makecell[{{p{\linewidth}}}]{
                \texttt{\tiny{[GM$|$GM]}}
                \texttt{VERSCHIEBE: O, (4,2)} \\
            }
        }
    }
    & & \\ \\

    \theutterance \stepcounter{utterance}  
    & & & \multicolumn{2}{p{0.3\linewidth}}{
        \cellcolor[rgb]{0.9,0.9,0.9}{
            \makecell[{{p{\linewidth}}}]{
                \texttt{\tiny{[GM$|$GM]}}
                \texttt{Strafpunkt: (4,2) ist nicht leer, sondern enthält '┼'.} \\
\\ 
\texttt{Ihr habt gemeinsam 9 von maximal 16 Strafpunkten angesammelt.} \\
\texttt{Bitte versuche es noch mal!} \\
            }
        }
    }
    & & \\ \\

    \theutterance \stepcounter{utterance}  
    & & \multicolumn{4}{p{0.6\linewidth}}{
        \cellcolor[rgb]{0.9,0.9,0.9}{
            \makecell[{{p{\linewidth}}}]{
                \texttt{\tiny{[P1$\langle$GM]}}
                \texttt{Strafpunkt: (4,2) ist nicht leer, sondern enthält '┼'.} \\
\\ 
\texttt{Ihr habt gemeinsam 9 von maximal 16 Strafpunkten angesammelt.} \\
\texttt{Bitte versuche es noch mal!} \\
            }
        }
    }
    & \\ \\

    \theutterance \stepcounter{utterance}  
    & \multicolumn{4}{p{0.6\linewidth}}{
        \cellcolor[rgb]{0.9,0.9,0.9}{
            \makecell[{{p{\linewidth}}}]{
                \texttt{\tiny{[P1$\rangle$GM]}}
                \texttt{SPRICH: Bei mir ist (4,2) durch ein Kreuzungszeichen blockiert. Vorschlag: Alternativkoordinate für O: (4,4) ist bei mir frei. Prüfe bitte, ob (4,4) bei dir ebenfalls frei ist. Falls nicht, nenne mir eine freie gemeinsame Alternative für O (z. B. (1,2) oder (5,2)), die bei dir frei ist und bei mir keine Linien hat.} \\
            }
        }
    }
    & & \\ \\

    \theutterance \stepcounter{utterance}  
    & & & \multicolumn{2}{p{0.3\linewidth}}{
        \cellcolor[rgb]{0.9,0.9,0.9}{
            \makecell[{{p{\linewidth}}}]{
                \texttt{\tiny{[GM$|$GM]}}
                \texttt{SPRICH: Bei mir ist (4,2) durch ein Kreuzungszeichen blockiert. Vorschlag: Alternativkoordinate für O: (4,4) ist bei mir frei. Prüfe bitte, ob (4,4) bei dir ebenfalls frei ist. Falls nicht, nenne mir eine freie gemeinsame Alternative für O (z. B. (1,2) oder (5,2)), die bei dir frei ist und bei mir keine Linien hat.} \\
            }
        }
    }
    & & \\ \\

    \theutterance \stepcounter{utterance}  
    & & \multicolumn{4}{p{0.6\linewidth}}{
        \cellcolor[rgb]{0.9,0.9,0.9}{
            \makecell[{{p{\linewidth}}}]{
                \texttt{\tiny{[GM$\rangle$P2]}}
                \texttt{Deine Nachricht wurde an deinen Mitspieler übermittelt.} \\
\\ 
\\ 
\texttt{Ihr seid gerade bei Runde 20 von maximal 28.} \\
\texttt{Ihr habt gemeinsam 9 von maximal 16 Strafpunkten angesammelt.} \\
\texttt{Dein Mitspieler hat dir diese Nachricht geschickt:} \\
\\ 
\\ 
\texttt{"Bei mir ist (4,2) durch ein Kreuzungszeichen blockiert. Vorschlag: Alternativkoordinate für O: (4,4) ist bei mir frei. Prüfe bitte, ob (4,4) bei dir ebenfalls frei ist. Falls nicht, nenne mir eine freie gemeinsame Alternative für O (z. B. (1,2) oder (5,2)), die bei dir frei ist und bei mir keine Linien hat."} \\
\\ 
\\ 
\texttt{Wie lautet dein nächster Befehl?} \\
            }
        }
    }
    & \\ \\

    \theutterance \stepcounter{utterance}  
    & & & \multicolumn{4}{p{0.6\linewidth}}{
        \cellcolor[rgb]{0.9,0.9,0.9}{
            \makecell[{{p{\linewidth}}}]{
                \texttt{\tiny{[GM$\langle$P2]}}
                \texttt{SPRICH: Bei mir ist (4,4) frei. Einverstanden: Setzen wir O auf (4,4) als gemeinsamen Zielpunkt. Ich verschiebe O als Nächstes zu (4,4). Aktueller Stand bei mir danach: A(1,3), D(2,3), P(5,3), S(6,3), T(7,3), O(4,4); M bleibt noch offen. Prüfe bitte, ob wir für M eine gemeinsame Alternative zu (3,3) wählen sollen (z. B. (3,2) oder (3,4)), und sag mir, welche dieser Koordinaten bei dir frei ist.} \\
            }
        }
    }
     \\ \\

    \theutterance \stepcounter{utterance}  
    & & & \multicolumn{2}{p{0.3\linewidth}}{
        \cellcolor[rgb]{0.9,0.9,0.9}{
            \makecell[{{p{\linewidth}}}]{
                \texttt{\tiny{[GM$|$GM]}}
                \texttt{SPRICH: Bei mir ist (4,4) frei. Einverstanden: Setzen wir O auf (4,4) als gemeinsamen Zielpunkt. Ich verschiebe O als Nächstes zu (4,4). Aktueller Stand bei mir danach: A(1,3), D(2,3), P(5,3), S(6,3), T(7,3), O(4,4); M bleibt noch offen. Prüfe bitte, ob wir für M eine gemeinsame Alternative zu (3,3) wählen sollen (z. B. (3,2) oder (3,4)), und sag mir, welche dieser Koordinaten bei dir frei ist.} \\
            }
        }
    }
    & & \\ \\

    \theutterance \stepcounter{utterance}  
    & & \multicolumn{4}{p{0.6\linewidth}}{
        \cellcolor[rgb]{0.9,0.9,0.9}{
            \makecell[{{p{\linewidth}}}]{
                \texttt{\tiny{[P1$\langle$GM]}}
                \texttt{Deine Nachricht wurde an deinen Mitspieler übermittelt.} \\
\\ 
\\ 
\texttt{Ihr seid gerade bei Runde 20 von maximal 28.} \\
\texttt{Ihr habt gemeinsam 9 von maximal 16 Strafpunkten angesammelt.} \\
\texttt{Dein Mitspieler hat dir diese Nachricht geschickt:} \\
\\ 
\\ 
\texttt{"Bei mir ist (4,4) frei. Einverstanden: Setzen wir O auf (4,4) als gemeinsamen Zielpunkt. Ich verschiebe O als Nächstes zu (4,4). Aktueller Stand bei mir danach: A(1,3), D(2,3), P(5,3), S(6,3), T(7,3), O(4,4); M bleibt noch offen. Prüfe bitte, ob wir für M eine gemeinsame Alternative zu (3,3) wählen sollen (z. B. (3,2) oder (3,4)), und sag mir, welche dieser Koordinaten bei dir frei ist."} \\
\\ 
\\ 
\texttt{Wie lautet dein nächster Befehl?} \\
            }
        }
    }
    & \\ \\

    \theutterance \stepcounter{utterance}  
    & \multicolumn{4}{p{0.6\linewidth}}{
        \cellcolor[rgb]{0.9,0.9,0.9}{
            \makecell[{{p{\linewidth}}}]{
                \texttt{\tiny{[P1$\rangle$GM]}}
                \texttt{VERSCHIEBE: O, (4,4)} \\
            }
        }
    }
    & & \\ \\

    \theutterance \stepcounter{utterance}  
    & & & \multicolumn{2}{p{0.3\linewidth}}{
        \cellcolor[rgb]{0.9,0.9,0.9}{
            \makecell[{{p{\linewidth}}}]{
                \texttt{\tiny{[GM$|$GM]}}
                \texttt{VERSCHIEBE: O, (4,4)} \\
            }
        }
    }
    & & \\ \\

    \theutterance \stepcounter{utterance}  
    & & & \multicolumn{2}{p{0.3\linewidth}}{
        \cellcolor[rgb]{0.9,0.9,0.9}{
            \makecell[{{p{\linewidth}}}]{
                \texttt{\tiny{[GM$|$GM]}}
                \texttt{Strafpunkt: (4,4) ist nicht leer, sondern enthält '┤'.} \\
\\ 
\texttt{Ihr habt gemeinsam 10 von maximal 16 Strafpunkten angesammelt.} \\
\texttt{Bitte versuche es noch mal!} \\
            }
        }
    }
    & & \\ \\

    \theutterance \stepcounter{utterance}  
    & & \multicolumn{4}{p{0.6\linewidth}}{
        \cellcolor[rgb]{0.9,0.9,0.9}{
            \makecell[{{p{\linewidth}}}]{
                \texttt{\tiny{[P1$\langle$GM]}}
                \texttt{Strafpunkt: (4,4) ist nicht leer, sondern enthält '┤'.} \\
\\ 
\texttt{Ihr habt gemeinsam 10 von maximal 16 Strafpunkten angesammelt.} \\
\texttt{Bitte versuche es noch mal!} \\
            }
        }
    }
    & \\ \\

    \theutterance \stepcounter{utterance}  
    & \multicolumn{4}{p{0.6\linewidth}}{
        \cellcolor[rgb]{0.9,0.9,0.9}{
            \makecell[{{p{\linewidth}}}]{
                \texttt{\tiny{[P1$\rangle$GM]}}
                \texttt{SPRICH: Bei mir sind an Spalte 4 viele Linien: (4,2)=┼, (4,3)=│ blockiert, (4,4)=┤ blockiert. Ich schlage für O eine neue gemeinsame Koordinate vor, die bei mir frei ist: (1,2) oder (5,2) oder (1,6). Bitte prüfe, welche davon bei dir frei ist, und nenne die erste gemeinsame freie, dann setze ich O dorthin. Für M als Alternative zu (3,3) kann ich (3,2) oder (3,4) anbieten; beide sind bei mir frei. Welche ist bei dir frei?} \\
            }
        }
    }
    & & \\ \\

    \theutterance \stepcounter{utterance}  
    & & & \multicolumn{2}{p{0.3\linewidth}}{
        \cellcolor[rgb]{0.9,0.9,0.9}{
            \makecell[{{p{\linewidth}}}]{
                \texttt{\tiny{[GM$|$GM]}}
                \texttt{SPRICH: Bei mir sind an Spalte 4 viele Linien: (4,2)=┼, (4,3)=│ blockiert, (4,4)=┤ blockiert. Ich schlage für O eine neue gemeinsame Koordinate vor, die bei mir frei ist: (1,2) oder (5,2) oder (1,6). Bitte prüfe, welche davon bei dir frei ist, und nenne die erste gemeinsame freie, dann setze ich O dorthin. Für M als Alternative zu (3,3) kann ich (3,2) oder (3,4) anbieten; beide sind bei mir frei. Welche ist bei dir frei?} \\
            }
        }
    }
    & & \\ \\

    \theutterance \stepcounter{utterance}  
    & & \multicolumn{4}{p{0.6\linewidth}}{
        \cellcolor[rgb]{0.9,0.9,0.9}{
            \makecell[{{p{\linewidth}}}]{
                \texttt{\tiny{[GM$\rangle$P2]}}
                \texttt{Deine Nachricht wurde an deinen Mitspieler übermittelt.} \\
\\ 
\\ 
\texttt{Ihr seid gerade bei Runde 21 von maximal 28.} \\
\texttt{Ihr habt gemeinsam 10 von maximal 16 Strafpunkten angesammelt.} \\
\texttt{Dein Mitspieler hat dir diese Nachricht geschickt:} \\
\\ 
\\ 
\texttt{"Bei mir sind an Spalte 4 viele Linien: (4,2)=┼, (4,3)=│ blockiert, (4,4)=┤ blockiert. Ich schlage für O eine neue gemeinsame Koordinate vor, die bei mir frei ist: (1,2) oder (5,2) oder (1,6). Bitte prüfe, welche davon bei dir frei ist, und nenne die erste gemeinsame freie, dann setze ich O dorthin. Für M als Alternative zu (3,3) kann ich (3,2) oder (3,4) anbieten; beide sind bei mir frei. Welche ist bei dir frei?"} \\
\\ 
\\ 
\texttt{Wie lautet dein nächster Befehl?} \\
            }
        }
    }
    & \\ \\

    \theutterance \stepcounter{utterance}  
    & & & \multicolumn{4}{p{0.6\linewidth}}{
        \cellcolor[rgb]{0.9,0.9,0.9}{
            \makecell[{{p{\linewidth}}}]{
                \texttt{\tiny{[GM$\langle$P2]}}
                \texttt{SPRICH: Prüfergebnis bei mir: O{-}Ziele — (1,2) frei, (5,2) frei, (1,6) belegt durch Linie. Nehmen wir O→(1,2) als gemeinsamen Punkt. Für M: (3,2) ist frei, (3,4) ist frei; ich schlage M→(3,2) vor. Ich setze als Nächstes O→(1,2), danach M→(3,2), sofern du bestätigst.} \\
            }
        }
    }
     \\ \\

    \theutterance \stepcounter{utterance}  
    & & & \multicolumn{2}{p{0.3\linewidth}}{
        \cellcolor[rgb]{0.9,0.9,0.9}{
            \makecell[{{p{\linewidth}}}]{
                \texttt{\tiny{[GM$|$GM]}}
                \texttt{SPRICH: Prüfergebnis bei mir: O{-}Ziele — (1,2) frei, (5,2) frei, (1,6) belegt durch Linie. Nehmen wir O→(1,2) als gemeinsamen Punkt. Für M: (3,2) ist frei, (3,4) ist frei; ich schlage M→(3,2) vor. Ich setze als Nächstes O→(1,2), danach M→(3,2), sofern du bestätigst.} \\
            }
        }
    }
    & & \\ \\

    \theutterance \stepcounter{utterance}  
    & & \multicolumn{4}{p{0.6\linewidth}}{
        \cellcolor[rgb]{0.9,0.9,0.9}{
            \makecell[{{p{\linewidth}}}]{
                \texttt{\tiny{[P1$\langle$GM]}}
                \texttt{Deine Nachricht wurde an deinen Mitspieler übermittelt.} \\
\\ 
\\ 
\texttt{Ihr seid gerade bei Runde 21 von maximal 28.} \\
\texttt{Ihr habt gemeinsam 10 von maximal 16 Strafpunkten angesammelt.} \\
\texttt{Dein Mitspieler hat dir diese Nachricht geschickt:} \\
\\ 
\\ 
\texttt{"Prüfergebnis bei mir: O{-}Ziele — (1,2) frei, (5,2) frei, (1,6) belegt durch Linie. Nehmen wir O→(1,2) als gemeinsamen Punkt. Für M: (3,2) ist frei, (3,4) ist frei; ich schlage M→(3,2) vor. Ich setze als Nächstes O→(1,2), danach M→(3,2), sofern du bestätigst."} \\
\\ 
\\ 
\texttt{Wie lautet dein nächster Befehl?} \\
            }
        }
    }
    & \\ \\

    \theutterance \stepcounter{utterance}  
    & \multicolumn{4}{p{0.6\linewidth}}{
        \cellcolor[rgb]{0.9,0.9,0.9}{
            \makecell[{{p{\linewidth}}}]{
                \texttt{\tiny{[P1$\rangle$GM]}}
                \texttt{SPRICH: Bestätigt: O→(1,2) als gemeinsamer Punkt und M→(3,2) als gemeinsamer Punkt. Ich setze jetzt O nach (1,2) und danach M nach (3,2). Sag Bescheid, falls du (1,2) kurzfristig blockierst; ansonsten verschiebe ich gleich.} \\
            }
        }
    }
    & & \\ \\

    \theutterance \stepcounter{utterance}  
    & & & \multicolumn{2}{p{0.3\linewidth}}{
        \cellcolor[rgb]{0.9,0.9,0.9}{
            \makecell[{{p{\linewidth}}}]{
                \texttt{\tiny{[GM$|$GM]}}
                \texttt{SPRICH: Bestätigt: O→(1,2) als gemeinsamer Punkt und M→(3,2) als gemeinsamer Punkt. Ich setze jetzt O nach (1,2) und danach M nach (3,2). Sag Bescheid, falls du (1,2) kurzfristig blockierst; ansonsten verschiebe ich gleich.} \\
            }
        }
    }
    & & \\ \\

    \theutterance \stepcounter{utterance}  
    & & \multicolumn{4}{p{0.6\linewidth}}{
        \cellcolor[rgb]{0.9,0.9,0.9}{
            \makecell[{{p{\linewidth}}}]{
                \texttt{\tiny{[GM$\rangle$P2]}}
                \texttt{Deine Nachricht wurde an deinen Mitspieler übermittelt.} \\
\\ 
\\ 
\texttt{Ihr seid gerade bei Runde 22 von maximal 28.} \\
\texttt{Ihr habt gemeinsam 10 von maximal 16 Strafpunkten angesammelt.} \\
\texttt{Dein Mitspieler hat dir diese Nachricht geschickt:} \\
\\ 
\\ 
\texttt{"Bestätigt: O→(1,2) als gemeinsamer Punkt und M→(3,2) als gemeinsamer Punkt. Ich setze jetzt O nach (1,2) und danach M nach (3,2). Sag Bescheid, falls du (1,2) kurzfristig blockierst; ansonsten verschiebe ich gleich."} \\
\\ 
\\ 
\texttt{Wie lautet dein nächster Befehl?} \\
            }
        }
    }
    & \\ \\

    \theutterance \stepcounter{utterance}  
    & & & \multicolumn{4}{p{0.6\linewidth}}{
        \cellcolor[rgb]{0.9,0.9,0.9}{
            \makecell[{{p{\linewidth}}}]{
                \texttt{\tiny{[GM$\langle$P2]}}
                \texttt{VERSCHIEBE: O, (1, 2)} \\
            }
        }
    }
     \\ \\

    \theutterance \stepcounter{utterance}  
    & & & \multicolumn{2}{p{0.3\linewidth}}{
        \cellcolor[rgb]{0.9,0.9,0.9}{
            \makecell[{{p{\linewidth}}}]{
                \texttt{\tiny{[GM$|$GM]}}
                \texttt{VERSCHIEBE: O, (1, 2)} \\
            }
        }
    }
    & & \\ \\

    \theutterance \stepcounter{utterance}  
    & & & \multicolumn{2}{p{0.3\linewidth}}{
        \cellcolor[rgb]{0.9,0.9,0.9}{
            \makecell[{{p{\linewidth}}}]{
                \texttt{\tiny{[GM$|$GM]}}
                \texttt{Strafpunkt: (1,2) ist nicht leer, sondern enthält '─'.} \\
\\ 
\texttt{Ihr habt gemeinsam 11 von maximal 16 Strafpunkten angesammelt.} \\
\texttt{Bitte versuche es noch mal!} \\
            }
        }
    }
    & & \\ \\

    \theutterance \stepcounter{utterance}  
    & & \multicolumn{4}{p{0.6\linewidth}}{
        \cellcolor[rgb]{0.9,0.9,0.9}{
            \makecell[{{p{\linewidth}}}]{
                \texttt{\tiny{[GM$\rangle$P2]}}
                \texttt{Strafpunkt: (1,2) ist nicht leer, sondern enthält '─'.} \\
\\ 
\texttt{Ihr habt gemeinsam 11 von maximal 16 Strafpunkten angesammelt.} \\
\texttt{Bitte versuche es noch mal!} \\
            }
        }
    }
    & \\ \\

    \theutterance \stepcounter{utterance}  
    & & & \multicolumn{4}{p{0.6\linewidth}}{
        \cellcolor[rgb]{0.9,0.9,0.9}{
            \makecell[{{p{\linewidth}}}]{
                \texttt{\tiny{[GM$\langle$P2]}}
                \texttt{SPRICH: Korrektur: Bei mir ist (1,2) durch eine Linie blockiert. Alternative gemeinsame O{-}Position: (5,2) ist bei mir frei. Können wir O→(5,2) setzen? Für M bleibt (3,2) bei mir frei. Bitte bestätigen, dann verschiebe ich entsprechend.} \\
            }
        }
    }
     \\ \\

    \theutterance \stepcounter{utterance}  
    & & & \multicolumn{2}{p{0.3\linewidth}}{
        \cellcolor[rgb]{0.9,0.9,0.9}{
            \makecell[{{p{\linewidth}}}]{
                \texttt{\tiny{[GM$|$GM]}}
                \texttt{SPRICH: Korrektur: Bei mir ist (1,2) durch eine Linie blockiert. Alternative gemeinsame O{-}Position: (5,2) ist bei mir frei. Können wir O→(5,2) setzen? Für M bleibt (3,2) bei mir frei. Bitte bestätigen, dann verschiebe ich entsprechend.} \\
            }
        }
    }
    & & \\ \\

    \theutterance \stepcounter{utterance}  
    & & \multicolumn{4}{p{0.6\linewidth}}{
        \cellcolor[rgb]{0.9,0.9,0.9}{
            \makecell[{{p{\linewidth}}}]{
                \texttt{\tiny{[P1$\langle$GM]}}
                \texttt{Deine Nachricht wurde an deinen Mitspieler übermittelt.} \\
\\ 
\\ 
\texttt{Ihr seid gerade bei Runde 22 von maximal 28.} \\
\texttt{Ihr habt gemeinsam 11 von maximal 16 Strafpunkten angesammelt.} \\
\texttt{Dein Mitspieler hat dir diese Nachricht geschickt:} \\
\\ 
\\ 
\texttt{"Korrektur: Bei mir ist (1,2) durch eine Linie blockiert. Alternative gemeinsame O{-}Position: (5,2) ist bei mir frei. Können wir O→(5,2) setzen? Für M bleibt (3,2) bei mir frei. Bitte bestätigen, dann verschiebe ich entsprechend."} \\
\\ 
\\ 
\texttt{Wie lautet dein nächster Befehl?} \\
            }
        }
    }
    & \\ \\

    \theutterance \stepcounter{utterance}  
    & \multicolumn{4}{p{0.6\linewidth}}{
        \cellcolor[rgb]{0.9,0.9,0.9}{
            \makecell[{{p{\linewidth}}}]{
                \texttt{\tiny{[P1$\rangle$GM]}}
                \texttt{SPRICH: Bestätigt: O→(5,2) als gemeinsamer Punkt und M→(3,2) bleibt. Bei mir ist (5,2) frei. Ich verschiebe als Nächstes O nach (5,2), danach M nach (3,2). Sag Bescheid, falls du (5,2) kurzfristig brauchst.} \\
            }
        }
    }
    & & \\ \\

    \theutterance \stepcounter{utterance}  
    & & & \multicolumn{2}{p{0.3\linewidth}}{
        \cellcolor[rgb]{0.9,0.9,0.9}{
            \makecell[{{p{\linewidth}}}]{
                \texttt{\tiny{[GM$|$GM]}}
                \texttt{SPRICH: Bestätigt: O→(5,2) als gemeinsamer Punkt und M→(3,2) bleibt. Bei mir ist (5,2) frei. Ich verschiebe als Nächstes O nach (5,2), danach M nach (3,2). Sag Bescheid, falls du (5,2) kurzfristig brauchst.} \\
            }
        }
    }
    & & \\ \\

    \theutterance \stepcounter{utterance}  
    & & \multicolumn{4}{p{0.6\linewidth}}{
        \cellcolor[rgb]{0.9,0.9,0.9}{
            \makecell[{{p{\linewidth}}}]{
                \texttt{\tiny{[GM$\rangle$P2]}}
                \texttt{Deine Nachricht wurde an deinen Mitspieler übermittelt.} \\
\\ 
\\ 
\texttt{Ihr seid gerade bei Runde 23 von maximal 28.} \\
\texttt{Ihr habt gemeinsam 11 von maximal 16 Strafpunkten angesammelt.} \\
\texttt{Dein Mitspieler hat dir diese Nachricht geschickt:} \\
\\ 
\\ 
\texttt{"Bestätigt: O→(5,2) als gemeinsamer Punkt und M→(3,2) bleibt. Bei mir ist (5,2) frei. Ich verschiebe als Nächstes O nach (5,2), danach M nach (3,2). Sag Bescheid, falls du (5,2) kurzfristig brauchst."} \\
\\ 
\\ 
\texttt{Wie lautet dein nächster Befehl?} \\
            }
        }
    }
    & \\ \\

    \theutterance \stepcounter{utterance}  
    & & & \multicolumn{4}{p{0.6\linewidth}}{
        \cellcolor[rgb]{0.9,0.9,0.9}{
            \makecell[{{p{\linewidth}}}]{
                \texttt{\tiny{[GM$\langle$P2]}}
                \texttt{VERSCHIEBE: O, (5, 2)} \\
            }
        }
    }
     \\ \\

    \theutterance \stepcounter{utterance}  
    & & & \multicolumn{2}{p{0.3\linewidth}}{
        \cellcolor[rgb]{0.9,0.9,0.9}{
            \makecell[{{p{\linewidth}}}]{
                \texttt{\tiny{[GM$|$GM]}}
                \texttt{VERSCHIEBE: O, (5, 2)} \\
            }
        }
    }
    & & \\ \\

    \theutterance \stepcounter{utterance}  
    & & & \multicolumn{2}{p{0.3\linewidth}}{
        \cellcolor[rgb]{0.9,0.9,0.9}{
            \makecell[{{p{\linewidth}}}]{
                \texttt{\tiny{[GM$|$GM]}}
                \texttt{Strafpunkt: (5,2) ist nicht leer, sondern enthält '─'.} \\
\\ 
\texttt{Ihr habt gemeinsam 12 von maximal 16 Strafpunkten angesammelt.} \\
\texttt{Bitte versuche es noch mal!} \\
            }
        }
    }
    & & \\ \\

    \theutterance \stepcounter{utterance}  
    & & \multicolumn{4}{p{0.6\linewidth}}{
        \cellcolor[rgb]{0.9,0.9,0.9}{
            \makecell[{{p{\linewidth}}}]{
                \texttt{\tiny{[GM$\rangle$P2]}}
                \texttt{Strafpunkt: (5,2) ist nicht leer, sondern enthält '─'.} \\
\\ 
\texttt{Ihr habt gemeinsam 12 von maximal 16 Strafpunkten angesammelt.} \\
\texttt{Bitte versuche es noch mal!} \\
            }
        }
    }
    & \\ \\

    \theutterance \stepcounter{utterance}  
    & & & \multicolumn{4}{p{0.6\linewidth}}{
        \cellcolor[rgb]{0.9,0.9,0.9}{
            \makecell[{{p{\linewidth}}}]{
                \texttt{\tiny{[GM$\langle$P2]}}
                \texttt{SPRICH: Sorry, auch (5,2) ist bei mir durch eine Linie blockiert. Neue Vorschläge für O, die bei mir frei sind: (2,1) frei, (6,2) frei, (7,2) frei. Bitte wähle eine davon, die bei dir ebenfalls frei ist. Für M bleibt (3,2) bei mir frei.} \\
            }
        }
    }
     \\ \\

    \theutterance \stepcounter{utterance}  
    & & & \multicolumn{2}{p{0.3\linewidth}}{
        \cellcolor[rgb]{0.9,0.9,0.9}{
            \makecell[{{p{\linewidth}}}]{
                \texttt{\tiny{[GM$|$GM]}}
                \texttt{SPRICH: Sorry, auch (5,2) ist bei mir durch eine Linie blockiert. Neue Vorschläge für O, die bei mir frei sind: (2,1) frei, (6,2) frei, (7,2) frei. Bitte wähle eine davon, die bei dir ebenfalls frei ist. Für M bleibt (3,2) bei mir frei.} \\
            }
        }
    }
    & & \\ \\

    \theutterance \stepcounter{utterance}  
    & & \multicolumn{4}{p{0.6\linewidth}}{
        \cellcolor[rgb]{0.9,0.9,0.9}{
            \makecell[{{p{\linewidth}}}]{
                \texttt{\tiny{[P1$\langle$GM]}}
                \texttt{Deine Nachricht wurde an deinen Mitspieler übermittelt.} \\
\\ 
\\ 
\texttt{Ihr seid gerade bei Runde 23 von maximal 28.} \\
\texttt{Ihr habt gemeinsam 12 von maximal 16 Strafpunkten angesammelt.} \\
\texttt{Dein Mitspieler hat dir diese Nachricht geschickt:} \\
\\ 
\\ 
\texttt{"Sorry, auch (5,2) ist bei mir durch eine Linie blockiert. Neue Vorschläge für O, die bei mir frei sind: (2,1) frei, (6,2) frei, (7,2) frei. Bitte wähle eine davon, die bei dir ebenfalls frei ist. Für M bleibt (3,2) bei mir frei."} \\
\\ 
\\ 
\texttt{Wie lautet dein nächster Befehl?} \\
            }
        }
    }
    & \\ \\

    \theutterance \stepcounter{utterance}  
    & \multicolumn{4}{p{0.6\linewidth}}{
        \cellcolor[rgb]{0.9,0.9,0.9}{
            \makecell[{{p{\linewidth}}}]{
                \texttt{\tiny{[P1$\rangle$GM]}}
                \texttt{SPRICH: Bei mir ist (2,1) belegt, (6,2) ist frei, (7,2) ist frei. Lass uns O→(6,2) als gemeinsamen Punkt nehmen. M→(3,2) bleibt wie besprochen. Ich verschiebe dann O nach (6,2) und anschließend M nach (3,2), sofern du zustimmst.} \\
            }
        }
    }
    & & \\ \\

    \theutterance \stepcounter{utterance}  
    & & & \multicolumn{2}{p{0.3\linewidth}}{
        \cellcolor[rgb]{0.9,0.9,0.9}{
            \makecell[{{p{\linewidth}}}]{
                \texttt{\tiny{[GM$|$GM]}}
                \texttt{SPRICH: Bei mir ist (2,1) belegt, (6,2) ist frei, (7,2) ist frei. Lass uns O→(6,2) als gemeinsamen Punkt nehmen. M→(3,2) bleibt wie besprochen. Ich verschiebe dann O nach (6,2) und anschließend M nach (3,2), sofern du zustimmst.} \\
            }
        }
    }
    & & \\ \\

    \theutterance \stepcounter{utterance}  
    & & \multicolumn{4}{p{0.6\linewidth}}{
        \cellcolor[rgb]{0.9,0.9,0.9}{
            \makecell[{{p{\linewidth}}}]{
                \texttt{\tiny{[GM$\rangle$P2]}}
                \texttt{Deine Nachricht wurde an deinen Mitspieler übermittelt.} \\
\\ 
\\ 
\texttt{Ihr seid gerade bei Runde 24 von maximal 28.} \\
\texttt{Ihr habt gemeinsam 12 von maximal 16 Strafpunkten angesammelt.} \\
\texttt{Dein Mitspieler hat dir diese Nachricht geschickt:} \\
\\ 
\\ 
\texttt{"Bei mir ist (2,1) belegt, (6,2) ist frei, (7,2) ist frei. Lass uns O→(6,2) als gemeinsamen Punkt nehmen. M→(3,2) bleibt wie besprochen. Ich verschiebe dann O nach (6,2) und anschließend M nach (3,2), sofern du zustimmst."} \\
\\ 
\\ 
\texttt{Wie lautet dein nächster Befehl?} \\
            }
        }
    }
    & \\ \\

    \theutterance \stepcounter{utterance}  
    & & & \multicolumn{4}{p{0.6\linewidth}}{
        \cellcolor[rgb]{0.9,0.9,0.9}{
            \makecell[{{p{\linewidth}}}]{
                \texttt{\tiny{[GM$\langle$P2]}}
                \texttt{SPRICH: Einverstanden: O→(6,2) und M→(3,2). Ich setze jetzt O nicht, da (6,2) bei mir frei ist, aber ich verschiebe zuerst M→(3,2), danach O→(6,2), falls weiterhin frei. Sag bitte Bescheid, wenn du (3,2) kurzfristig brauchst; ansonsten lege ich los.} \\
            }
        }
    }
     \\ \\

    \theutterance \stepcounter{utterance}  
    & & & \multicolumn{2}{p{0.3\linewidth}}{
        \cellcolor[rgb]{0.9,0.9,0.9}{
            \makecell[{{p{\linewidth}}}]{
                \texttt{\tiny{[GM$|$GM]}}
                \texttt{SPRICH: Einverstanden: O→(6,2) und M→(3,2). Ich setze jetzt O nicht, da (6,2) bei mir frei ist, aber ich verschiebe zuerst M→(3,2), danach O→(6,2), falls weiterhin frei. Sag bitte Bescheid, wenn du (3,2) kurzfristig brauchst; ansonsten lege ich los.} \\
            }
        }
    }
    & & \\ \\

    \theutterance \stepcounter{utterance}  
    & & \multicolumn{4}{p{0.6\linewidth}}{
        \cellcolor[rgb]{0.9,0.9,0.9}{
            \makecell[{{p{\linewidth}}}]{
                \texttt{\tiny{[P1$\langle$GM]}}
                \texttt{Deine Nachricht wurde an deinen Mitspieler übermittelt.} \\
\\ 
\\ 
\texttt{Ihr seid gerade bei Runde 24 von maximal 28.} \\
\texttt{Ihr habt gemeinsam 12 von maximal 16 Strafpunkten angesammelt.} \\
\texttt{Dein Mitspieler hat dir diese Nachricht geschickt:} \\
\\ 
\\ 
\texttt{"Einverstanden: O→(6,2) und M→(3,2). Ich setze jetzt O nicht, da (6,2) bei mir frei ist, aber ich verschiebe zuerst M→(3,2), danach O→(6,2), falls weiterhin frei. Sag bitte Bescheid, wenn du (3,2) kurzfristig brauchst; ansonsten lege ich los."} \\
\\ 
\\ 
\texttt{Wie lautet dein nächster Befehl?} \\
            }
        }
    }
    & \\ \\

    \theutterance \stepcounter{utterance}  
    & \multicolumn{4}{p{0.6\linewidth}}{
        \cellcolor[rgb]{0.9,0.9,0.9}{
            \makecell[{{p{\linewidth}}}]{
                \texttt{\tiny{[P1$\rangle$GM]}}
                \texttt{VERSCHIEBE: M, (3,2)} \\
            }
        }
    }
    & & \\ \\

    \theutterance \stepcounter{utterance}  
    & & & \multicolumn{2}{p{0.3\linewidth}}{
        \cellcolor[rgb]{0.9,0.9,0.9}{
            \makecell[{{p{\linewidth}}}]{
                \texttt{\tiny{[GM$|$GM]}}
                \texttt{VERSCHIEBE: M, (3,2)} \\
            }
        }
    }
    & & \\ \\

    \theutterance \stepcounter{utterance}  
    & & & \multicolumn{2}{p{0.3\linewidth}}{
        \cellcolor[rgb]{0.9,0.9,0.9}{
            \makecell[{{p{\linewidth}}}]{
                \texttt{\tiny{[GM$|$GM]}}
                \texttt{Strafpunkt: (3,2) ist nicht leer, sondern enthält '─'.} \\
\\ 
\texttt{Ihr habt gemeinsam 13 von maximal 16 Strafpunkten angesammelt.} \\
\texttt{Bitte versuche es noch mal!} \\
            }
        }
    }
    & & \\ \\

    \theutterance \stepcounter{utterance}  
    & & \multicolumn{4}{p{0.6\linewidth}}{
        \cellcolor[rgb]{0.9,0.9,0.9}{
            \makecell[{{p{\linewidth}}}]{
                \texttt{\tiny{[P1$\langle$GM]}}
                \texttt{Strafpunkt: (3,2) ist nicht leer, sondern enthält '─'.} \\
\\ 
\texttt{Ihr habt gemeinsam 13 von maximal 16 Strafpunkten angesammelt.} \\
\texttt{Bitte versuche es noch mal!} \\
            }
        }
    }
    & \\ \\

    \theutterance \stepcounter{utterance}  
    & \multicolumn{4}{p{0.6\linewidth}}{
        \cellcolor[rgb]{0.9,0.9,0.9}{
            \makecell[{{p{\linewidth}}}]{
                \texttt{\tiny{[P1$\rangle$GM]}}
                \texttt{SPRICH: Bei mir ist (3,2) durch eine Linie blockiert. Alternative gemeinsame M{-}Positionen, die bei mir frei sind: (3,4) oder (2,2) oder (2,4). Für O bleibt (6,2) bei mir frei. Bitte prüfe: Welche dieser M{-}Koordinaten sind bei dir frei? Wenn z. B. (3,4) bei dir frei ist, nehmen wir M→(3,4) und O→(6,2).} \\
            }
        }
    }
    & & \\ \\

    \theutterance \stepcounter{utterance}  
    & & & \multicolumn{2}{p{0.3\linewidth}}{
        \cellcolor[rgb]{0.9,0.9,0.9}{
            \makecell[{{p{\linewidth}}}]{
                \texttt{\tiny{[GM$|$GM]}}
                \texttt{SPRICH: Bei mir ist (3,2) durch eine Linie blockiert. Alternative gemeinsame M{-}Positionen, die bei mir frei sind: (3,4) oder (2,2) oder (2,4). Für O bleibt (6,2) bei mir frei. Bitte prüfe: Welche dieser M{-}Koordinaten sind bei dir frei? Wenn z. B. (3,4) bei dir frei ist, nehmen wir M→(3,4) und O→(6,2).} \\
            }
        }
    }
    & & \\ \\

    \theutterance \stepcounter{utterance}  
    & & \multicolumn{4}{p{0.6\linewidth}}{
        \cellcolor[rgb]{0.9,0.9,0.9}{
            \makecell[{{p{\linewidth}}}]{
                \texttt{\tiny{[GM$\rangle$P2]}}
                \texttt{Deine Nachricht wurde an deinen Mitspieler übermittelt.} \\
\\ 
\\ 
\texttt{Ihr seid gerade bei Runde 25 von maximal 28.} \\
\texttt{Ihr habt gemeinsam 13 von maximal 16 Strafpunkten angesammelt.} \\
\texttt{Dein Mitspieler hat dir diese Nachricht geschickt:} \\
\\ 
\\ 
\texttt{"Bei mir ist (3,2) durch eine Linie blockiert. Alternative gemeinsame M{-}Positionen, die bei mir frei sind: (3,4) oder (2,2) oder (2,4). Für O bleibt (6,2) bei mir frei. Bitte prüfe: Welche dieser M{-}Koordinaten sind bei dir frei? Wenn z. B. (3,4) bei dir frei ist, nehmen wir M→(3,4) und O→(6,2)."} \\
\\ 
\\ 
\texttt{Wie lautet dein nächster Befehl?} \\
            }
        }
    }
    & \\ \\

    \theutterance \stepcounter{utterance}  
    & & & \multicolumn{4}{p{0.6\linewidth}}{
        \cellcolor[rgb]{0.9,0.9,0.9}{
            \makecell[{{p{\linewidth}}}]{
                \texttt{\tiny{[GM$\langle$P2]}}
                \texttt{SPRICH: Prüfung bei mir: M{-}Alternativen — (3,4) ist frei, (2,2) ist frei, (2,4) ist frei. Vorschlag: M→(3,4) als gemeinsamer Punkt und O→(6,2) bleibt. Ich verschiebe als Nächstes M nach (3,4), danach O nach (6,2), sofern du bestätigst.} \\
            }
        }
    }
     \\ \\

    \theutterance \stepcounter{utterance}  
    & & & \multicolumn{2}{p{0.3\linewidth}}{
        \cellcolor[rgb]{0.9,0.9,0.9}{
            \makecell[{{p{\linewidth}}}]{
                \texttt{\tiny{[GM$|$GM]}}
                \texttt{SPRICH: Prüfung bei mir: M{-}Alternativen — (3,4) ist frei, (2,2) ist frei, (2,4) ist frei. Vorschlag: M→(3,4) als gemeinsamer Punkt und O→(6,2) bleibt. Ich verschiebe als Nächstes M nach (3,4), danach O nach (6,2), sofern du bestätigst.} \\
            }
        }
    }
    & & \\ \\

    \theutterance \stepcounter{utterance}  
    & & \multicolumn{4}{p{0.6\linewidth}}{
        \cellcolor[rgb]{0.9,0.9,0.9}{
            \makecell[{{p{\linewidth}}}]{
                \texttt{\tiny{[P1$\langle$GM]}}
                \texttt{Deine Nachricht wurde an deinen Mitspieler übermittelt.} \\
\\ 
\\ 
\texttt{Ihr seid gerade bei Runde 25 von maximal 28.} \\
\texttt{Ihr habt gemeinsam 13 von maximal 16 Strafpunkten angesammelt.} \\
\texttt{Dein Mitspieler hat dir diese Nachricht geschickt:} \\
\\ 
\\ 
\texttt{"Prüfung bei mir: M{-}Alternativen — (3,4) ist frei, (2,2) ist frei, (2,4) ist frei. Vorschlag: M→(3,4) als gemeinsamer Punkt und O→(6,2) bleibt. Ich verschiebe als Nächstes M nach (3,4), danach O nach (6,2), sofern du bestätigst."} \\
\\ 
\\ 
\texttt{Wie lautet dein nächster Befehl?} \\
            }
        }
    }
    & \\ \\

    \theutterance \stepcounter{utterance}  
    & \multicolumn{4}{p{0.6\linewidth}}{
        \cellcolor[rgb]{0.9,0.9,0.9}{
            \makecell[{{p{\linewidth}}}]{
                \texttt{\tiny{[P1$\rangle$GM]}}
                \texttt{SPRICH: Bestätigt: M→(3,4) als gemeinsamer Punkt und O→(6,2) bleibt. Bei mir ist (3,4) frei und (6,2) frei. Ich verschiebe als Nächstes O nach (6,2) und danach, falls nötig, passe ich nichts weiter an. Sag Bescheid, wenn du umreihst.} \\
            }
        }
    }
    & & \\ \\

    \theutterance \stepcounter{utterance}  
    & & & \multicolumn{2}{p{0.3\linewidth}}{
        \cellcolor[rgb]{0.9,0.9,0.9}{
            \makecell[{{p{\linewidth}}}]{
                \texttt{\tiny{[GM$|$GM]}}
                \texttt{SPRICH: Bestätigt: M→(3,4) als gemeinsamer Punkt und O→(6,2) bleibt. Bei mir ist (3,4) frei und (6,2) frei. Ich verschiebe als Nächstes O nach (6,2) und danach, falls nötig, passe ich nichts weiter an. Sag Bescheid, wenn du umreihst.} \\
            }
        }
    }
    & & \\ \\

    \theutterance \stepcounter{utterance}  
    & & \multicolumn{4}{p{0.6\linewidth}}{
        \cellcolor[rgb]{0.9,0.9,0.9}{
            \makecell[{{p{\linewidth}}}]{
                \texttt{\tiny{[GM$\rangle$P2]}}
                \texttt{Deine Nachricht wurde an deinen Mitspieler übermittelt.} \\
\\ 
\\ 
\texttt{Ihr seid gerade bei Runde 26 von maximal 28.} \\
\texttt{Ihr habt gemeinsam 13 von maximal 16 Strafpunkten angesammelt.} \\
\texttt{Dein Mitspieler hat dir diese Nachricht geschickt:} \\
\\ 
\\ 
\texttt{"Bestätigt: M→(3,4) als gemeinsamer Punkt und O→(6,2) bleibt. Bei mir ist (3,4) frei und (6,2) frei. Ich verschiebe als Nächstes O nach (6,2) und danach, falls nötig, passe ich nichts weiter an. Sag Bescheid, wenn du umreihst."} \\
\\ 
\\ 
\texttt{Wie lautet dein nächster Befehl?} \\
            }
        }
    }
    & \\ \\

    \theutterance \stepcounter{utterance}  
    & & & \multicolumn{4}{p{0.6\linewidth}}{
        \cellcolor[rgb]{0.9,0.9,0.9}{
            \makecell[{{p{\linewidth}}}]{
                \texttt{\tiny{[GM$\langle$P2]}}
                \texttt{VERSCHIEBE: M, (3, 4)} \\
            }
        }
    }
     \\ \\

    \theutterance \stepcounter{utterance}  
    & & & \multicolumn{2}{p{0.3\linewidth}}{
        \cellcolor[rgb]{0.9,0.9,0.9}{
            \makecell[{{p{\linewidth}}}]{
                \texttt{\tiny{[GM$|$GM]}}
                \texttt{VERSCHIEBE: M, (3, 4)} \\
            }
        }
    }
    & & \\ \\

    \theutterance \stepcounter{utterance}  
    & & & \multicolumn{2}{p{0.3\linewidth}}{
        \cellcolor[rgb]{0.9,0.9,0.9}{
            \makecell[{{p{\linewidth}}}]{
                \texttt{\tiny{[GM$|$GM]}}
                \texttt{Strafpunkt: (3,4) ist nicht leer, sondern enthält '─'.} \\
\\ 
\texttt{Ihr habt gemeinsam 14 von maximal 16 Strafpunkten angesammelt.} \\
\texttt{Bitte versuche es noch mal!} \\
            }
        }
    }
    & & \\ \\

    \theutterance \stepcounter{utterance}  
    & & \multicolumn{4}{p{0.6\linewidth}}{
        \cellcolor[rgb]{0.9,0.9,0.9}{
            \makecell[{{p{\linewidth}}}]{
                \texttt{\tiny{[GM$\rangle$P2]}}
                \texttt{Strafpunkt: (3,4) ist nicht leer, sondern enthält '─'.} \\
\\ 
\texttt{Ihr habt gemeinsam 14 von maximal 16 Strafpunkten angesammelt.} \\
\texttt{Bitte versuche es noch mal!} \\
            }
        }
    }
    & \\ \\

    \theutterance \stepcounter{utterance}  
    & & & \multicolumn{4}{p{0.6\linewidth}}{
        \cellcolor[rgb]{0.9,0.9,0.9}{
            \makecell[{{p{\linewidth}}}]{
                \texttt{\tiny{[GM$\langle$P2]}}
                \texttt{SPRICH: Korrektur: Bei mir ist (3,4) doch durch eine Linie blockiert. Alternativen für M, die bei mir frei sind: (2,2) und (2,4). Können wir M→(2,4) festlegen? O bleibt O→(6,2). Wenn du bestätigst, verschiebe ich anschließend.} \\
            }
        }
    }
     \\ \\

    \theutterance \stepcounter{utterance}  
    & & & \multicolumn{2}{p{0.3\linewidth}}{
        \cellcolor[rgb]{0.9,0.9,0.9}{
            \makecell[{{p{\linewidth}}}]{
                \texttt{\tiny{[GM$|$GM]}}
                \texttt{SPRICH: Korrektur: Bei mir ist (3,4) doch durch eine Linie blockiert. Alternativen für M, die bei mir frei sind: (2,2) und (2,4). Können wir M→(2,4) festlegen? O bleibt O→(6,2). Wenn du bestätigst, verschiebe ich anschließend.} \\
            }
        }
    }
    & & \\ \\

    \theutterance \stepcounter{utterance}  
    & & \multicolumn{4}{p{0.6\linewidth}}{
        \cellcolor[rgb]{0.9,0.9,0.9}{
            \makecell[{{p{\linewidth}}}]{
                \texttt{\tiny{[P1$\langle$GM]}}
                \texttt{Deine Nachricht wurde an deinen Mitspieler übermittelt.} \\
\\ 
\\ 
\texttt{Ihr seid gerade bei Runde 26 von maximal 28.} \\
\texttt{Ihr habt gemeinsam 14 von maximal 16 Strafpunkten angesammelt.} \\
\texttt{Dein Mitspieler hat dir diese Nachricht geschickt:} \\
\\ 
\\ 
\texttt{"Korrektur: Bei mir ist (3,4) doch durch eine Linie blockiert. Alternativen für M, die bei mir frei sind: (2,2) und (2,4). Können wir M→(2,4) festlegen? O bleibt O→(6,2). Wenn du bestätigst, verschiebe ich anschließend."} \\
\\ 
\\ 
\texttt{Wie lautet dein nächster Befehl?} \\
            }
        }
    }
    & \\ \\

    \theutterance \stepcounter{utterance}  
    & \multicolumn{4}{p{0.6\linewidth}}{
        \cellcolor[rgb]{0.9,0.9,0.9}{
            \makecell[{{p{\linewidth}}}]{
                \texttt{\tiny{[P1$\rangle$GM]}}
                \texttt{SPRICH: Bestätigt: M→(2,4) als gemeinsamer Punkt und O→(6,2) bleibt. Bei mir sind (2,4) und (6,2) frei. Ich verschiebe als Nächstes O nach (6,2). Sag Bescheid, wenn du bereit bist, dann koordiniere ich den Rest und wir prüfen „fertig?“.} \\
            }
        }
    }
    & & \\ \\

    \theutterance \stepcounter{utterance}  
    & & & \multicolumn{2}{p{0.3\linewidth}}{
        \cellcolor[rgb]{0.9,0.9,0.9}{
            \makecell[{{p{\linewidth}}}]{
                \texttt{\tiny{[GM$|$GM]}}
                \texttt{SPRICH: Bestätigt: M→(2,4) als gemeinsamer Punkt und O→(6,2) bleibt. Bei mir sind (2,4) und (6,2) frei. Ich verschiebe als Nächstes O nach (6,2). Sag Bescheid, wenn du bereit bist, dann koordiniere ich den Rest und wir prüfen „fertig?“.} \\
            }
        }
    }
    & & \\ \\

    \theutterance \stepcounter{utterance}  
    & & \multicolumn{4}{p{0.6\linewidth}}{
        \cellcolor[rgb]{0.9,0.9,0.9}{
            \makecell[{{p{\linewidth}}}]{
                \texttt{\tiny{[GM$\rangle$P2]}}
                \texttt{Deine Nachricht wurde an deinen Mitspieler übermittelt.} \\
\\ 
\\ 
\texttt{Ihr seid gerade bei Runde 27 von maximal 28.} \\
\texttt{Ihr habt gemeinsam 14 von maximal 16 Strafpunkten angesammelt.} \\
\texttt{Dein Mitspieler hat dir diese Nachricht geschickt:} \\
\\ 
\\ 
\texttt{"Bestätigt: M→(2,4) als gemeinsamer Punkt und O→(6,2) bleibt. Bei mir sind (2,4) und (6,2) frei. Ich verschiebe als Nächstes O nach (6,2). Sag Bescheid, wenn du bereit bist, dann koordiniere ich den Rest und wir prüfen „fertig?“."} \\
\\ 
\\ 
\texttt{Wie lautet dein nächster Befehl?} \\
            }
        }
    }
    & \\ \\

    \theutterance \stepcounter{utterance}  
    & & & \multicolumn{4}{p{0.6\linewidth}}{
        \cellcolor[rgb]{0.9,0.9,0.9}{
            \makecell[{{p{\linewidth}}}]{
                \texttt{\tiny{[GM$\langle$P2]}}
                \texttt{VERSCHIEBE: M, (2, 4)} \\
            }
        }
    }
     \\ \\

    \theutterance \stepcounter{utterance}  
    & & & \multicolumn{2}{p{0.3\linewidth}}{
        \cellcolor[rgb]{0.9,0.9,0.9}{
            \makecell[{{p{\linewidth}}}]{
                \texttt{\tiny{[GM$|$GM]}}
                \texttt{VERSCHIEBE: M, (2, 4)} \\
            }
        }
    }
    & & \\ \\

    \theutterance \stepcounter{utterance}  
    & & & \multicolumn{2}{p{0.3\linewidth}}{
        \cellcolor[rgb]{0.9,0.9,0.9}{
            \makecell[{{p{\linewidth}}}]{
                \texttt{\tiny{[GM$|$GM]}}
                \texttt{Strafpunkt: (2,4) ist nicht leer, sondern enthält '├'.} \\
\\ 
\texttt{Ihr habt gemeinsam 15 von maximal 16 Strafpunkten angesammelt.} \\
\texttt{Bitte versuche es noch mal!} \\
            }
        }
    }
    & & \\ \\

    \theutterance \stepcounter{utterance}  
    & & \multicolumn{4}{p{0.6\linewidth}}{
        \cellcolor[rgb]{0.9,0.9,0.9}{
            \makecell[{{p{\linewidth}}}]{
                \texttt{\tiny{[GM$\rangle$P2]}}
                \texttt{Strafpunkt: (2,4) ist nicht leer, sondern enthält '├'.} \\
\\ 
\texttt{Ihr habt gemeinsam 15 von maximal 16 Strafpunkten angesammelt.} \\
\texttt{Bitte versuche es noch mal!} \\
            }
        }
    }
    & \\ \\

    \theutterance \stepcounter{utterance}  
    & & & \multicolumn{4}{p{0.6\linewidth}}{
        \cellcolor[rgb]{0.9,0.9,0.9}{
            \makecell[{{p{\linewidth}}}]{
                \texttt{\tiny{[GM$\langle$P2]}}
                \texttt{SPRICH: Ich bin fast blockiert von Linien: Für M finde ich aktuell keine der vorgeschlagenen Ziele frei. Freie Felder für M bei mir: (1,4), (1,5), (3,5). Für O ist (6,2) bei mir frei. Vorschlag: O→(6,2) beibehalten; M→(1,5) als gemeinsamer Punkt (oder alternativ (3,5), falls bei dir besser). Bitte bestätige eines davon, dann verschiebe ich sofort.} \\
            }
        }
    }
     \\ \\

    \theutterance \stepcounter{utterance}  
    & & & \multicolumn{2}{p{0.3\linewidth}}{
        \cellcolor[rgb]{0.9,0.9,0.9}{
            \makecell[{{p{\linewidth}}}]{
                \texttt{\tiny{[GM$|$GM]}}
                \texttt{SPRICH: Ich bin fast blockiert von Linien: Für M finde ich aktuell keine der vorgeschlagenen Ziele frei. Freie Felder für M bei mir: (1,4), (1,5), (3,5). Für O ist (6,2) bei mir frei. Vorschlag: O→(6,2) beibehalten; M→(1,5) als gemeinsamer Punkt (oder alternativ (3,5), falls bei dir besser). Bitte bestätige eines davon, dann verschiebe ich sofort.} \\
            }
        }
    }
    & & \\ \\

    \theutterance \stepcounter{utterance}  
    & & \multicolumn{4}{p{0.6\linewidth}}{
        \cellcolor[rgb]{0.9,0.9,0.9}{
            \makecell[{{p{\linewidth}}}]{
                \texttt{\tiny{[P1$\langle$GM]}}
                \texttt{Deine Nachricht wurde an deinen Mitspieler übermittelt.} \\
\\ 
\\ 
\texttt{Ihr seid gerade bei Runde 27 von maximal 28.} \\
\texttt{Ihr habt gemeinsam 15 von maximal 16 Strafpunkten angesammelt.} \\
\texttt{Dein Mitspieler hat dir diese Nachricht geschickt:} \\
\\ 
\\ 
\texttt{"Ich bin fast blockiert von Linien: Für M finde ich aktuell keine der vorgeschlagenen Ziele frei. Freie Felder für M bei mir: (1,4), (1,5), (3,5). Für O ist (6,2) bei mir frei. Vorschlag: O→(6,2) beibehalten; M→(1,5) als gemeinsamer Punkt (oder alternativ (3,5), falls bei dir besser). Bitte bestätige eines davon, dann verschiebe ich sofort."} \\
\\ 
\\ 
\texttt{Wie lautet dein nächster Befehl?} \\
            }
        }
    }
    & \\ \\

    \theutterance \stepcounter{utterance}  
    & \multicolumn{4}{p{0.6\linewidth}}{
        \cellcolor[rgb]{0.9,0.9,0.9}{
            \makecell[{{p{\linewidth}}}]{
                \texttt{\tiny{[P1$\rangle$GM]}}
                \texttt{SPRICH: Bestätigt: O→(6,2) bleibt. Für M nehme ich (1,5) als gemeinsamen Punkt; bei mir ist (1,5) frei. Ich verschiebe als Nächstes O nach (6,2). Wenn du M auf (1,5) gesetzt hast, sag Bescheid, dann prüfen wir „fertig?“.} \\
            }
        }
    }
    & & \\ \\

    \theutterance \stepcounter{utterance}  
    & & & \multicolumn{2}{p{0.3\linewidth}}{
        \cellcolor[rgb]{0.9,0.9,0.9}{
            \makecell[{{p{\linewidth}}}]{
                \texttt{\tiny{[GM$|$GM]}}
                \texttt{SPRICH: Bestätigt: O→(6,2) bleibt. Für M nehme ich (1,5) als gemeinsamen Punkt; bei mir ist (1,5) frei. Ich verschiebe als Nächstes O nach (6,2). Wenn du M auf (1,5) gesetzt hast, sag Bescheid, dann prüfen wir „fertig?“.} \\
            }
        }
    }
    & & \\ \\

    \theutterance \stepcounter{utterance}  
    & & \multicolumn{4}{p{0.6\linewidth}}{
        \cellcolor[rgb]{0.9,0.9,0.9}{
            \makecell[{{p{\linewidth}}}]{
                \texttt{\tiny{[GM$\rangle$P2]}}
                \texttt{Deine Nachricht wurde an deinen Mitspieler übermittelt.} \\
\\ 
\\ 
\texttt{Ihr seid gerade bei Runde 28 von maximal 28.} \\
\texttt{Ihr habt gemeinsam 15 von maximal 16 Strafpunkten angesammelt.} \\
\texttt{Dein Mitspieler hat dir diese Nachricht geschickt:} \\
\\ 
\\ 
\texttt{"Bestätigt: O→(6,2) bleibt. Für M nehme ich (1,5) als gemeinsamen Punkt; bei mir ist (1,5) frei. Ich verschiebe als Nächstes O nach (6,2). Wenn du M auf (1,5) gesetzt hast, sag Bescheid, dann prüfen wir „fertig?“."} \\
\\ 
\\ 
\texttt{Wie lautet dein nächster Befehl?} \\
            }
        }
    }
    & \\ \\

    \theutterance \stepcounter{utterance}  
    & & & \multicolumn{4}{p{0.6\linewidth}}{
        \cellcolor[rgb]{0.9,0.9,0.9}{
            \makecell[{{p{\linewidth}}}]{
                \texttt{\tiny{[GM$\langle$P2]}}
                \texttt{VERSCHIEBE: M, (1, 5)} \\
            }
        }
    }
     \\ \\

    \theutterance \stepcounter{utterance}  
    & & & \multicolumn{2}{p{0.3\linewidth}}{
        \cellcolor[rgb]{0.9,0.9,0.9}{
            \makecell[{{p{\linewidth}}}]{
                \texttt{\tiny{[GM$|$GM]}}
                \texttt{VERSCHIEBE: M, (1, 5)} \\
            }
        }
    }
    & & \\ \\

    \theutterance \stepcounter{utterance}  
    & & & \multicolumn{2}{p{0.3\linewidth}}{
        \cellcolor[rgb]{0.9,0.9,0.9}{
            \makecell[{{p{\linewidth}}}]{
                \texttt{\tiny{[GM$|$GM]}}
                \texttt{Strafpunkt: (1,5) ist nicht leer, sondern enthält '─'.} \\
\\ 
\texttt{Ihr habt gemeinsam 16 von maximal 16 Strafpunkten angesammelt.} \\
\texttt{Bitte versuche es noch mal!} \\
            }
        }
    }
    & & \\ \\

    \theutterance \stepcounter{utterance}  
    & & \multicolumn{4}{p{0.6\linewidth}}{
        \cellcolor[rgb]{0.9,0.9,0.9}{
            \makecell[{{p{\linewidth}}}]{
                \texttt{\tiny{[GM$\rangle$P2]}}
                \texttt{Strafpunkt: (1,5) ist nicht leer, sondern enthält '─'.} \\
\\ 
\texttt{Ihr habt gemeinsam 16 von maximal 16 Strafpunkten angesammelt.} \\
\texttt{Bitte versuche es noch mal!} \\
            }
        }
    }
    & \\ \\

    \theutterance \stepcounter{utterance}  
    & & & \multicolumn{4}{p{0.6\linewidth}}{
        \cellcolor[rgb]{0.9,0.9,0.9}{
            \makecell[{{p{\linewidth}}}]{
                \texttt{\tiny{[GM$\langle$P2]}}
                \texttt{SPRICH: Ich finde bei mir kein freies Ziel für M mehr; (1,5) ist auch blockiert. Letzter Rettungsversuch: Für M ist (3,5) frei; können wir M→(3,5) als gemeinsamen Punkt nehmen? O bleibt O→(6,2). Wenn das passt, setze ich M sofort auf (3,5).} \\
            }
        }
    }
     \\ \\

    \theutterance \stepcounter{utterance}  
    & & & \multicolumn{2}{p{0.3\linewidth}}{
        \cellcolor[rgb]{0.9,0.9,0.9}{
            \makecell[{{p{\linewidth}}}]{
                \texttt{\tiny{[GM$|$GM]}}
                \texttt{SPRICH: Ich finde bei mir kein freies Ziel für M mehr; (1,5) ist auch blockiert. Letzter Rettungsversuch: Für M ist (3,5) frei; können wir M→(3,5) als gemeinsamen Punkt nehmen? O bleibt O→(6,2). Wenn das passt, setze ich M sofort auf (3,5).} \\
            }
        }
    }
    & & \\ \\

    \theutterance \stepcounter{utterance}  
    & & & \multicolumn{2}{p{0.3\linewidth}}{
        \cellcolor[rgb]{0.9,0.9,0.9}{
            \makecell[{{p{\linewidth}}}]{
                \texttt{\tiny{[GM$|$GM]}}
                \texttt{Maximum number of rounds reached} \\
            }
        }
    }
    & & \\ \\

    \theutterance \stepcounter{utterance}  
    & & & \multicolumn{2}{p{0.3\linewidth}}{
        \cellcolor[rgb]{0.9,0.9,0.9}{
            \makecell[{{p{\linewidth}}}]{
                \texttt{\tiny{[GM$|$GM]}}
                \texttt{* success: True} \\
\texttt{* lose: False} \\
\texttt{* aborted: False} \\
\texttt{{-}{-}{-}{-}{-}{-}{-}} \\
\texttt{* Shifts: 10.00} \\
\texttt{* Max Shifts: 12.00} \\
\texttt{* Min Shifts: 6.00} \\
\texttt{* End Distance Sum: 13.28} \\
\texttt{* Init Distance Sum: 23.62} \\
\texttt{* Expected Distance Sum: 29.33} \\
\texttt{* Penalties: 16.00} \\
\texttt{* Max Penalties: 16.00} \\
\texttt{* Rounds: 28.00} \\
\texttt{* Max Rounds: 28.00} \\
\texttt{* Object Count: 7.00} \\
            }
        }
    }
    & & \\ \\

    \theutterance \stepcounter{utterance}  
    & & & \multicolumn{2}{p{0.3\linewidth}}{
        \cellcolor[rgb]{0.9,0.9,0.9}{
            \makecell[{{p{\linewidth}}}]{
                \texttt{\tiny{[GM$|$GM]}}
                \texttt{Initial Grids:} \\
\texttt{```} \\
\texttt{Player 1:     Player 2:} \\
\texttt{ 1234567       1234567} \\
\texttt{╔═══╤═╤═╗     ╔═══╤═╤═╗} \\
\texttt{║◌S◌│◌│A║ 1   ║P◌A│◌│◌║ 1} \\
\texttt{╟─┬─┼─┴─╢ 2   ╟─┬─┼─┴─╢ 2} \\
\texttt{║◌│◌│◌◌◌║ 3   ║◌│D│◌◌◌║ 3} \\
\texttt{║P├─┤D┌─╢ 4   ║◌├─┤◌┌─╢ 4} \\
\texttt{╟─┘◌│◌│M║ 5   ╟─┘◌│◌│O║ 5} \\
\texttt{║◌◌◌└┬┴─╢ 6   ║◌◌◌└┬┴─╢ 6} \\
\texttt{║◌OT◌│◌◌║ 7   ║◌S◌T│◌M║ 7} \\
\texttt{╚════╧══╝     ╚════╧══╝} \\
\texttt{```} \\
\\ 
\texttt{Final Grids:} \\
\texttt{```} \\
\texttt{Player 1:     Player 2:} \\
\texttt{ 1234567       1234567} \\
\texttt{╔═══╤═╤═╗     ╔═══╤═╤═╗} \\
\texttt{║◌◌◌│◌│◌║ 1   ║◌◌◌│◌│◌║ 1} \\
\texttt{╟─┬─┼─┴─╢ 2   ╟─┬─┼─┴─╢ 2} \\
\texttt{║A│M│PST║ 3   ║A│D│PST║ 3} \\
\texttt{║◌├─┤D┌─╢ 4   ║◌├─┤◌┌─╢ 4} \\
\texttt{╟─┘◌│◌│◌║ 5   ╟─┘◌│◌│O║ 5} \\
\texttt{║◌◌◌└┬┴─╢ 6   ║◌◌◌└┬┴─╢ 6} \\
\texttt{║◌O◌◌│◌◌║ 7   ║◌◌◌◌│◌M║ 7} \\
\texttt{╚════╧══╝     ╚════╧══╝} \\
\texttt{```} \\
            }
        }
    }
    & & \\ \\

\end{supertabular}
}

\end{document}
