
\documentclass{article}
\usepackage{colortbl}
\usepackage{makecell}
\usepackage{multirow}
\usepackage{supertabular}

\begin{document}

\newcounter{utterance}

\centering \large Interaction Transcript for game `clean\_up', experiment `2\_hard\_7obj\_de', episode 0 with gpt{-}5{-}2025{-}08{-}07{-}no{-}reasoning{-}t1.0.
\vspace{24pt}

{ \footnotesize  \setcounter{utterance}{1}
\setlength{\tabcolsep}{0pt}
\begin{supertabular}{c@{$\;$}|p{.15\linewidth}@{}p{.15\linewidth}p{.15\linewidth}p{.15\linewidth}p{.15\linewidth}p{.15\linewidth}}
    \# & $\;$A & \multicolumn{4}{c}{Game Master} & $\;\:$B\\
    \hline

    \theutterance \stepcounter{utterance}  
    & & \multicolumn{4}{p{0.6\linewidth}}{
        \cellcolor[rgb]{0.9,0.9,0.9}{
            \makecell[{{p{\linewidth}}}]{
                \texttt{\tiny{[P1$\langle$GM]}}
                \texttt{Ich bin der Spielleiter, und du spielst ein kooperatives Spiel mit dem folgenden Raster als Spielbrett:} \\
\\ 
\texttt{```} \\
\texttt{ 1234567} \\
\texttt{╔══╤════╗} \\
\texttt{║◌T│D◌◌◌║ 1} \\
\texttt{╟──┘◌◌┌─╢ 2} \\
\texttt{║SA◌◌◌│◌║ 3} \\
\texttt{╟─┐◌┌─┼─╢ 4} \\
\texttt{║◌├─┤◌│O║ 5} \\
\texttt{╟─┤◌├─┤◌║ 6} \\
\texttt{║◌│P│M│◌║ 7} \\
\texttt{╚═╧═╧═╧═╝} \\
\\ 
\texttt{```} \\
\\ 
\texttt{* Am oberen Rand sind die x{-}Koordinaten angegeben, die nach rechts hin zunehmen, und am rechten Rand die y{-}Koordinaten, die nach unten zunehmen.} \\
\texttt{* Die folgenden Objekte sind zufällig auf deinem Raster platziert: 'P', 'O', 'T', 'S', 'D', 'A', 'M'.} \\
\\ 
\\ 
\texttt{Dein Mitspieler sieht eine Variation des Spielbretts, auf dem dieselben Objekte an anderen zufälligen Positionen platziert sind. Du kannst das Spielbrett des anderen Spielers nicht sehen, und er kann deins nicht sehen.} \\
\\ 
\\ 
\texttt{**Ziel des Spiels:**} \\
\\ 
\texttt{Beide Spieler müssen die Objekte auf ihrem jeweiligen Spielbrett so bewegen, dass identische Objekte am Ende an denselben Koordinaten liegen. Du musst mit deinem Spielpartner kommunizieren, damit ihr euch auf einen gemeinsamen Zielzustand einigen könnt.} \\
\\ 
\\ 
\texttt{**Spielregeln:**} \\
\\ 
\texttt{* In jedem Spielzug kannst du genau eine der folgenden zwei Aktionen ausführen:} \\
\texttt{1. `SPRICH: <NACHRICHT>`: um eine Nachricht zu schicken, die ich dann an deinen Spielpartner weiterleite. Als Nachricht gilt alles bis zum nächsten Zeilenumbruch.} \\
\texttt{2. `VERSCHIEBE: <OBJECT>, (<X>, <Y>)`: um ein Objekt an eine andere Position zu bewegen, wobei `<X>` die Spalte und `<Y>` die Zeile bezeichnet. Ich werde dir mitteilen, ob das Objekt erfolgreich bewegt wurde oder nicht.} \\
\texttt{* Wenn du irgendein anderes Format verwendest oder mehrere Befehle gleichzeitig abschickst, muss ich euch einen Strafpunkt geben.} \\
\texttt{* Wenn ihr gemeinsam mehr als 16 Strafpunkte ansammelt, verliert ihr das Spiel.} \\
\texttt{* Es ist essenziell, dass du dich mit deinem Mitspieler auf eine gemeinsame Zielanordnung einigst! Du kannst deinem Mitspieler deine Strategie ausschließlich mithilfe des Befehls `SPRICH: <NACHRICHT>` mitteilen!} \\
\\ 
\\ 
\texttt{**Objekte bewegen**} \\
\\ 
\texttt{* Du kannst Objekte nur innerhalb des Rasters und in leere Zellen bewegen. Leere Zellen erkennst am Zeichen '◌'.} \\
\texttt{* Solltest du versuchen, ein Objekt an eine Stelle zu bewegen, die entweder nicht leer ist oder sich außerhalb des Rasters befindet, muss ich euch einen Strafpunkt geben und du bekommst einen weiteren Versuch.} \\
\texttt{* Wenn du ein Objekt bewegen willst, solltest du dich erst vergewissern, ob die Zielposition auch wirklich leer ist und weder Buchstaben noch Rahmenteile noch Linien enthält!} \\
\\ 
\\ 
\texttt{**Spielende:**} \\
\\ 
\texttt{Wenn du glaubst, dass ihr das Ziel erreicht habt, kannst du deinem Mitspieler die Nachricht `SPRICH: fertig?` schicken. Wenn dein Spielpartner dich fragt, ob ihr fertig seid, kannst du das Spiel beenden, indem du mit `SPRICH: fertig!` antwortest.} \\
\\ 
\texttt{Ihr gewinnt gemeinsam, wenn ihr das Spiel innerhalb von \#max\_rounds Runden beendet. Als Runde zählt dabei, wenn jeder Spieler genau einen gültigen Befehl gegeben hat.} \\
\\ 
\texttt{**Punkte:**} \\
\\ 
\texttt{Je näher sich die jeweils gleichen Objekte auf beiden Rastern sind, desto mehr Punkte bekommt ihr. Für Strafpunkte gibt es Punktabzug. Könnt ihr den Rekord brechen?} \\
\\ 
\texttt{Schicke eine Nachricht an deinen Mitspieler um das Spiel zu starten!} \\
            }
        }
    }
    & \\ \\

    \theutterance \stepcounter{utterance}  
    & \multicolumn{4}{p{0.6\linewidth}}{
        \cellcolor[rgb]{0.9,0.9,0.9}{
            \makecell[{{p{\linewidth}}}]{
                \texttt{\tiny{[P1$\rangle$GM]}}
                \texttt{SPRICH: Hallo! Lass uns eine gemeinsame Zielanordnung festlegen. Vorschlag: Wir ordnen die Buchstaben in alphabetischer Reihenfolge in der unteren Reihe an, von links nach rechts: A bei (1,7), D bei (2,7), M bei (3,7), O bei (4,7), P bei (5,7), S bei (6,7), T bei (7,7). Wenn du zustimmst, bestätige bitte und beginne, deine Objekte entsprechend zu verschieben. Falls eine Position blockiert ist, sag mir, welche Alternativen du brauchst.} \\
            }
        }
    }
    & & \\ \\

    \theutterance \stepcounter{utterance}  
    & & & \multicolumn{2}{p{0.3\linewidth}}{
        \cellcolor[rgb]{0.9,0.9,0.9}{
            \makecell[{{p{\linewidth}}}]{
                \texttt{\tiny{[GM$|$GM]}}
                \texttt{SPRICH: Hallo! Lass uns eine gemeinsame Zielanordnung festlegen. Vorschlag: Wir ordnen die Buchstaben in alphabetischer Reihenfolge in der unteren Reihe an, von links nach rechts: A bei (1,7), D bei (2,7), M bei (3,7), O bei (4,7), P bei (5,7), S bei (6,7), T bei (7,7). Wenn du zustimmst, bestätige bitte und beginne, deine Objekte entsprechend zu verschieben. Falls eine Position blockiert ist, sag mir, welche Alternativen du brauchst.} \\
            }
        }
    }
    & & \\ \\

    \theutterance \stepcounter{utterance}  
    & & \multicolumn{4}{p{0.6\linewidth}}{
        \cellcolor[rgb]{0.9,0.9,0.9}{
            \makecell[{{p{\linewidth}}}]{
                \texttt{\tiny{[GM$\rangle$P2]}}
                \texttt{Ich bin der Spielleiter, und du spielst ein kooperatives Spiel mit dem folgenden Raster als Spielbrett:} \\
\\ 
\texttt{```} \\
\texttt{ 1234567} \\
\texttt{╔══╤════╗} \\
\texttt{║◌S│◌◌◌◌║ 1} \\
\texttt{╟──┘◌◌┌─╢ 2} \\
\texttt{║A◌O◌◌│◌║ 3} \\
\texttt{╟─┐◌┌─┼─╢ 4} \\
\texttt{║◌├─┤T│D║ 5} \\
\texttt{╟─┤◌├─┤◌║ 6} \\
\texttt{║P│◌│M│◌║ 7} \\
\texttt{╚═╧═╧═╧═╝} \\
\\ 
\texttt{```} \\
\\ 
\texttt{* Am oberen Rand sind die x{-}Koordinaten angegeben, die nach rechts hin zunehmen, und am rechten Rand die y{-}Koordinaten, die nach unten zunehmen.} \\
\texttt{* Die folgenden Objekte sind zufällig auf deinem Raster platziert: 'P', 'O', 'T', 'S', 'D', 'A', 'M'.} \\
\\ 
\\ 
\texttt{Dein Mitspieler sieht eine Variation des Spielbretts, auf dem dieselben Objekte an anderen zufälligen Positionen platziert sind. Du kannst das Spielbrett des anderen Spielers nicht sehen, und er kann deins nicht sehen.} \\
\\ 
\\ 
\texttt{**Ziel des Spiels:**} \\
\\ 
\texttt{Beide Spieler müssen die Objekte auf ihrem jeweiligen Spielbrett so bewegen, dass identische Objekte am Ende an denselben Koordinaten liegen. Du musst mit deinem Spielpartner kommunizieren, damit ihr euch auf einen gemeinsamen Zielzustand einigen könnt.} \\
\\ 
\\ 
\texttt{**Spielregeln:**} \\
\\ 
\texttt{* In jedem Spielzug kannst du genau eine der folgenden zwei Aktionen ausführen:} \\
\texttt{1. `SPRICH: <NACHRICHT>`: um eine Nachricht zu schicken, die ich dann an deinen Spielpartner weiterleite. Als Nachricht gilt alles bis zum nächsten Zeilenumbruch.} \\
\texttt{2. `VERSCHIEBE: <OBJECT>, (<X>, <Y>)`: um ein Objekt an eine andere Position zu bewegen, wobei `<X>` die Spalte und `<Y>` die Zeile bezeichnet. Ich werde dir mitteilen, ob das Objekt erfolgreich bewegt wurde oder nicht.} \\
\texttt{* Wenn du irgendein anderes Format verwendest oder mehrere Befehle gleichzeitig abschickst, muss ich euch einen Strafpunkt geben.} \\
\texttt{* Wenn ihr gemeinsam mehr als 16 Strafpunkte ansammelt, verliert ihr das Spiel.} \\
\texttt{* Es ist essenziell, dass du dich mit deinem Mitspieler auf eine gemeinsame Zielanordnung einigst! Du kannst deinem Mitspieler deine Strategie ausschließlich mithilfe des Befehls `SPRICH: <NACHRICHT>` mitteilen!} \\
\\ 
\\ 
\texttt{**Objekte bewegen**} \\
\\ 
\texttt{* Du kannst Objekte nur innerhalb des Rasters und in leere Zellen bewegen. Leere Zellen erkennst am Zeichen '◌'.} \\
\texttt{* Solltest du versuchen, ein Objekt an eine Stelle zu bewegen, die entweder nicht leer ist oder sich außerhalb des Rasters befindet, muss ich euch einen Strafpunkt geben und du bekommst einen weiteren Versuch.} \\
\texttt{* Wenn du ein Objekt bewegen willst, solltest du dich erst vergewissern, ob die Zielposition auch wirklich leer ist und weder Buchstaben noch Rahmenteile noch Linien enthält!} \\
\\ 
\\ 
\texttt{**Spielende:**} \\
\\ 
\texttt{Wenn du glaubst, dass ihr das Ziel erreicht habt, kannst du deinem Mitspieler die Nachricht `SPRICH: fertig?` schicken. Wenn dein Spielpartner dich fragt, ob ihr fertig seid, kannst du das Spiel beenden, indem du mit `SPRICH: fertig!` antwortest.} \\
\\ 
\texttt{Ihr gewinnt gemeinsam, wenn ihr das Spiel innerhalb von \#max\_rounds Runden beendet. Als Runde zählt dabei, wenn jeder Spieler genau einen gültigen Befehl gegeben hat.} \\
\\ 
\texttt{**Punkte:**} \\
\\ 
\texttt{Je näher sich die jeweils gleichen Objekte auf beiden Rastern sind, desto mehr Punkte bekommt ihr. Für Strafpunkte gibt es Punktabzug. Könnt ihr den Rekord brechen?} \\
\\ 
\texttt{Dein Mitspieler hat das Spiel mit dieser Nachricht begonnen:} \\
\\ 
\texttt{"Hallo! Lass uns eine gemeinsame Zielanordnung festlegen. Vorschlag: Wir ordnen die Buchstaben in alphabetischer Reihenfolge in der unteren Reihe an, von links nach rechts: A bei (1,7), D bei (2,7), M bei (3,7), O bei (4,7), P bei (5,7), S bei (6,7), T bei (7,7). Wenn du zustimmst, bestätige bitte und beginne, deine Objekte entsprechend zu verschieben. Falls eine Position blockiert ist, sag mir, welche Alternativen du brauchst."} \\
\\ 
\texttt{Wie lautet dein erster Befehl?} \\
            }
        }
    }
    & \\ \\

    \theutterance \stepcounter{utterance}  
    & & & \multicolumn{4}{p{0.6\linewidth}}{
        \cellcolor[rgb]{0.9,0.9,0.9}{
            \makecell[{{p{\linewidth}}}]{
                \texttt{\tiny{[GM$\langle$P2]}}
                \texttt{SPRICH: Einverstanden mit deinem Vorschlag. Ich beginne zu verschieben. Prüfe bei mir zuerst: A(1,7), D(2,7), M(3,7), O(4,7), P(5,7), S(6,7), T(7,7). Ich starte mit A nach (1,7).} \\
            }
        }
    }
     \\ \\

    \theutterance \stepcounter{utterance}  
    & & & \multicolumn{2}{p{0.3\linewidth}}{
        \cellcolor[rgb]{0.9,0.9,0.9}{
            \makecell[{{p{\linewidth}}}]{
                \texttt{\tiny{[GM$|$GM]}}
                \texttt{SPRICH: Einverstanden mit deinem Vorschlag. Ich beginne zu verschieben. Prüfe bei mir zuerst: A(1,7), D(2,7), M(3,7), O(4,7), P(5,7), S(6,7), T(7,7). Ich starte mit A nach (1,7).} \\
            }
        }
    }
    & & \\ \\

    \theutterance \stepcounter{utterance}  
    & & \multicolumn{4}{p{0.6\linewidth}}{
        \cellcolor[rgb]{0.9,0.9,0.9}{
            \makecell[{{p{\linewidth}}}]{
                \texttt{\tiny{[P1$\langle$GM]}}
                \texttt{Deine Nachricht wurde an deinen Mitspieler übermittelt.} \\
\\ 
\\ 
\texttt{Ihr seid gerade bei Runde 1 von maximal 28.} \\
\texttt{Ihr habt gemeinsam 0 von maximal 16 Strafpunkten angesammelt.} \\
\texttt{Dein Mitspieler hat dir diese Nachricht geschickt:} \\
\\ 
\\ 
\texttt{"Einverstanden mit deinem Vorschlag. Ich beginne zu verschieben. Prüfe bei mir zuerst: A(1,7), D(2,7), M(3,7), O(4,7), P(5,7), S(6,7), T(7,7). Ich starte mit A nach (1,7)."} \\
\\ 
\\ 
\texttt{Wie lautet dein nächster Befehl?} \\
            }
        }
    }
    & \\ \\

    \theutterance \stepcounter{utterance}  
    & \multicolumn{4}{p{0.6\linewidth}}{
        \cellcolor[rgb]{0.9,0.9,0.9}{
            \makecell[{{p{\linewidth}}}]{
                \texttt{\tiny{[P1$\rangle$GM]}}
                \texttt{SPRICH: Bestätigt. Ich übernehme dieselbe Zielreihe. Meine aktuellen Positionen: P(2,7), O(6,5), T(2,1), S(1,3), D(2,1? nein, D ist (2,1) belegt von T) Korrektur: D steht bei (2,1)? Ich sehe D bei (2,1) und T bei (2,1) geht nicht – ich präzisiere meine Sicht: T bei (2,1), D bei (2,3)? Warte, ich beschreibe exakt: T(2,1), D(2,1?) Ich stoppe und frage: Bestätige du erst deine Züge. Ich verschiebe zunächst O nach (4,7), wenn frei, dann P nach (5,7). Sag Bescheid, wenn du eine andere Reihenfolge brauchst.} \\
            }
        }
    }
    & & \\ \\

    \theutterance \stepcounter{utterance}  
    & & & \multicolumn{2}{p{0.3\linewidth}}{
        \cellcolor[rgb]{0.9,0.9,0.9}{
            \makecell[{{p{\linewidth}}}]{
                \texttt{\tiny{[GM$|$GM]}}
                \texttt{SPRICH: Bestätigt. Ich übernehme dieselbe Zielreihe. Meine aktuellen Positionen: P(2,7), O(6,5), T(2,1), S(1,3), D(2,1? nein, D ist (2,1) belegt von T) Korrektur: D steht bei (2,1)? Ich sehe D bei (2,1) und T bei (2,1) geht nicht – ich präzisiere meine Sicht: T bei (2,1), D bei (2,3)? Warte, ich beschreibe exakt: T(2,1), D(2,1?) Ich stoppe und frage: Bestätige du erst deine Züge. Ich verschiebe zunächst O nach (4,7), wenn frei, dann P nach (5,7). Sag Bescheid, wenn du eine andere Reihenfolge brauchst.} \\
            }
        }
    }
    & & \\ \\

    \theutterance \stepcounter{utterance}  
    & & \multicolumn{4}{p{0.6\linewidth}}{
        \cellcolor[rgb]{0.9,0.9,0.9}{
            \makecell[{{p{\linewidth}}}]{
                \texttt{\tiny{[GM$\rangle$P2]}}
                \texttt{Deine Nachricht wurde an deinen Mitspieler übermittelt.} \\
\\ 
\\ 
\texttt{Ihr seid gerade bei Runde 2 von maximal 28.} \\
\texttt{Ihr habt gemeinsam 0 von maximal 16 Strafpunkten angesammelt.} \\
\texttt{Dein Mitspieler hat dir diese Nachricht geschickt:} \\
\\ 
\\ 
\texttt{"Bestätigt. Ich übernehme dieselbe Zielreihe. Meine aktuellen Positionen: P(2,7), O(6,5), T(2,1), S(1,3), D(2,1? nein, D ist (2,1) belegt von T) Korrektur: D steht bei (2,1)? Ich sehe D bei (2,1) und T bei (2,1) geht nicht – ich präzisiere meine Sicht: T bei (2,1), D bei (2,3)? Warte, ich beschreibe exakt: T(2,1), D(2,1?) Ich stoppe und frage: Bestätige du erst deine Züge. Ich verschiebe zunächst O nach (4,7), wenn frei, dann P nach (5,7). Sag Bescheid, wenn du eine andere Reihenfolge brauchst."} \\
\\ 
\\ 
\texttt{Wie lautet dein nächster Befehl?} \\
            }
        }
    }
    & \\ \\

    \theutterance \stepcounter{utterance}  
    & & & \multicolumn{4}{p{0.6\linewidth}}{
        \cellcolor[rgb]{0.9,0.9,0.9}{
            \makecell[{{p{\linewidth}}}]{
                \texttt{\tiny{[GM$\langle$P2]}}
                \texttt{VERSCHIEBE: A, (1,7)} \\
            }
        }
    }
     \\ \\

    \theutterance \stepcounter{utterance}  
    & & & \multicolumn{2}{p{0.3\linewidth}}{
        \cellcolor[rgb]{0.9,0.9,0.9}{
            \makecell[{{p{\linewidth}}}]{
                \texttt{\tiny{[GM$|$GM]}}
                \texttt{VERSCHIEBE: A, (1,7)} \\
            }
        }
    }
    & & \\ \\

    \theutterance \stepcounter{utterance}  
    & & & \multicolumn{2}{p{0.3\linewidth}}{
        \cellcolor[rgb]{0.9,0.9,0.9}{
            \makecell[{{p{\linewidth}}}]{
                \texttt{\tiny{[GM$|$GM]}}
                \texttt{Strafpunkt: (1,7) ist nicht leer, sondern enthält 'P'.} \\
\\ 
\texttt{Ihr habt gemeinsam 1 von maximal 16 Strafpunkten angesammelt.} \\
\texttt{Bitte versuche es noch mal!} \\
            }
        }
    }
    & & \\ \\

    \theutterance \stepcounter{utterance}  
    & & \multicolumn{4}{p{0.6\linewidth}}{
        \cellcolor[rgb]{0.9,0.9,0.9}{
            \makecell[{{p{\linewidth}}}]{
                \texttt{\tiny{[GM$\rangle$P2]}}
                \texttt{Strafpunkt: (1,7) ist nicht leer, sondern enthält 'P'.} \\
\\ 
\texttt{Ihr habt gemeinsam 1 von maximal 16 Strafpunkten angesammelt.} \\
\texttt{Bitte versuche es noch mal!} \\
            }
        }
    }
    & \\ \\

    \theutterance \stepcounter{utterance}  
    & & & \multicolumn{4}{p{0.6\linewidth}}{
        \cellcolor[rgb]{0.9,0.9,0.9}{
            \makecell[{{p{\linewidth}}}]{
                \texttt{\tiny{[GM$\langle$P2]}}
                \texttt{SPRICH: Danke für die Info. Meine Startpositionen: P(1,7), O(3,3), T(5,5), S(2,1), D(7,5), A(1,3), M(3,7). Ich schlage vor, erst P nach (5,7), M nach (3,7 bleibt), O nach (4,7), T nach (7,7), S nach (6,7), A nach (1,7), D nach (2,7). Ich beginne mit: Ich räume zuerst P von (1,7) weg, dann setze A. Sag Bescheid, falls du eine andere Reihenfolge brauchst.} \\
            }
        }
    }
     \\ \\

    \theutterance \stepcounter{utterance}  
    & & & \multicolumn{2}{p{0.3\linewidth}}{
        \cellcolor[rgb]{0.9,0.9,0.9}{
            \makecell[{{p{\linewidth}}}]{
                \texttt{\tiny{[GM$|$GM]}}
                \texttt{SPRICH: Danke für die Info. Meine Startpositionen: P(1,7), O(3,3), T(5,5), S(2,1), D(7,5), A(1,3), M(3,7). Ich schlage vor, erst P nach (5,7), M nach (3,7 bleibt), O nach (4,7), T nach (7,7), S nach (6,7), A nach (1,7), D nach (2,7). Ich beginne mit: Ich räume zuerst P von (1,7) weg, dann setze A. Sag Bescheid, falls du eine andere Reihenfolge brauchst.} \\
            }
        }
    }
    & & \\ \\

    \theutterance \stepcounter{utterance}  
    & & \multicolumn{4}{p{0.6\linewidth}}{
        \cellcolor[rgb]{0.9,0.9,0.9}{
            \makecell[{{p{\linewidth}}}]{
                \texttt{\tiny{[P1$\langle$GM]}}
                \texttt{Deine Nachricht wurde an deinen Mitspieler übermittelt.} \\
\\ 
\\ 
\texttt{Ihr seid gerade bei Runde 2 von maximal 28.} \\
\texttt{Ihr habt gemeinsam 1 von maximal 16 Strafpunkten angesammelt.} \\
\texttt{Dein Mitspieler hat dir diese Nachricht geschickt:} \\
\\ 
\\ 
\texttt{"Danke für die Info. Meine Startpositionen: P(1,7), O(3,3), T(5,5), S(2,1), D(7,5), A(1,3), M(3,7). Ich schlage vor, erst P nach (5,7), M nach (3,7 bleibt), O nach (4,7), T nach (7,7), S nach (6,7), A nach (1,7), D nach (2,7). Ich beginne mit: Ich räume zuerst P von (1,7) weg, dann setze A. Sag Bescheid, falls du eine andere Reihenfolge brauchst."} \\
\\ 
\\ 
\texttt{Wie lautet dein nächster Befehl?} \\
            }
        }
    }
    & \\ \\

    \theutterance \stepcounter{utterance}  
    & \multicolumn{4}{p{0.6\linewidth}}{
        \cellcolor[rgb]{0.9,0.9,0.9}{
            \makecell[{{p{\linewidth}}}]{
                \texttt{\tiny{[P1$\rangle$GM]}}
                \texttt{SPRICH: Einverstanden mit deiner Reihenfolge. Meine Startpositionen sind: P(2,7), O(6,5), T(2,1), S(1,3), D(2,1? korrigiert gleich nach Sichtprüfung), A(2,3), M(3,7). Ich richte ebenfalls die Zielreihe ein: A(1,7), D(2,7), M(3,7), O(4,7), P(5,7), S(6,7), T(7,7). Ich beginne mit: VERSCHIEBE O nach (4,7), dann P nach (5,7), A nach (1,7), D nach (2,7), S nach (6,7), T nach (7,7). Starte jetzt mit O→(4,7). Bitte sag, falls du (4,7) schon belegt hast.} \\
            }
        }
    }
    & & \\ \\

    \theutterance \stepcounter{utterance}  
    & & & \multicolumn{2}{p{0.3\linewidth}}{
        \cellcolor[rgb]{0.9,0.9,0.9}{
            \makecell[{{p{\linewidth}}}]{
                \texttt{\tiny{[GM$|$GM]}}
                \texttt{SPRICH: Einverstanden mit deiner Reihenfolge. Meine Startpositionen sind: P(2,7), O(6,5), T(2,1), S(1,3), D(2,1? korrigiert gleich nach Sichtprüfung), A(2,3), M(3,7). Ich richte ebenfalls die Zielreihe ein: A(1,7), D(2,7), M(3,7), O(4,7), P(5,7), S(6,7), T(7,7). Ich beginne mit: VERSCHIEBE O nach (4,7), dann P nach (5,7), A nach (1,7), D nach (2,7), S nach (6,7), T nach (7,7). Starte jetzt mit O→(4,7). Bitte sag, falls du (4,7) schon belegt hast.} \\
            }
        }
    }
    & & \\ \\

    \theutterance \stepcounter{utterance}  
    & & \multicolumn{4}{p{0.6\linewidth}}{
        \cellcolor[rgb]{0.9,0.9,0.9}{
            \makecell[{{p{\linewidth}}}]{
                \texttt{\tiny{[GM$\rangle$P2]}}
                \texttt{Deine Nachricht wurde an deinen Mitspieler übermittelt.} \\
\\ 
\\ 
\texttt{Ihr seid gerade bei Runde 3 von maximal 28.} \\
\texttt{Ihr habt gemeinsam 1 von maximal 16 Strafpunkten angesammelt.} \\
\texttt{Dein Mitspieler hat dir diese Nachricht geschickt:} \\
\\ 
\\ 
\texttt{"Einverstanden mit deiner Reihenfolge. Meine Startpositionen sind: P(2,7), O(6,5), T(2,1), S(1,3), D(2,1? korrigiert gleich nach Sichtprüfung), A(2,3), M(3,7). Ich richte ebenfalls die Zielreihe ein: A(1,7), D(2,7), M(3,7), O(4,7), P(5,7), S(6,7), T(7,7). Ich beginne mit: VERSCHIEBE O nach (4,7), dann P nach (5,7), A nach (1,7), D nach (2,7), S nach (6,7), T nach (7,7). Starte jetzt mit O→(4,7). Bitte sag, falls du (4,7) schon belegt hast."} \\
\\ 
\\ 
\texttt{Wie lautet dein nächster Befehl?} \\
            }
        }
    }
    & \\ \\

    \theutterance \stepcounter{utterance}  
    & & & \multicolumn{4}{p{0.6\linewidth}}{
        \cellcolor[rgb]{0.9,0.9,0.9}{
            \makecell[{{p{\linewidth}}}]{
                \texttt{\tiny{[GM$\langle$P2]}}
                \texttt{VERSCHIEBE: P, (5,7)} \\
            }
        }
    }
     \\ \\

    \theutterance \stepcounter{utterance}  
    & & & \multicolumn{2}{p{0.3\linewidth}}{
        \cellcolor[rgb]{0.9,0.9,0.9}{
            \makecell[{{p{\linewidth}}}]{
                \texttt{\tiny{[GM$|$GM]}}
                \texttt{VERSCHIEBE: P, (5,7)} \\
            }
        }
    }
    & & \\ \\

    \theutterance \stepcounter{utterance}  
    & & & \multicolumn{2}{p{0.3\linewidth}}{
        \cellcolor[rgb]{0.9,0.9,0.9}{
            \makecell[{{p{\linewidth}}}]{
                \texttt{\tiny{[GM$|$GM]}}
                \texttt{Strafpunkt: (5,7) ist nicht leer, sondern enthält 'M'.} \\
\\ 
\texttt{Ihr habt gemeinsam 2 von maximal 16 Strafpunkten angesammelt.} \\
\texttt{Bitte versuche es noch mal!} \\
            }
        }
    }
    & & \\ \\

    \theutterance \stepcounter{utterance}  
    & & \multicolumn{4}{p{0.6\linewidth}}{
        \cellcolor[rgb]{0.9,0.9,0.9}{
            \makecell[{{p{\linewidth}}}]{
                \texttt{\tiny{[GM$\rangle$P2]}}
                \texttt{Strafpunkt: (5,7) ist nicht leer, sondern enthält 'M'.} \\
\\ 
\texttt{Ihr habt gemeinsam 2 von maximal 16 Strafpunkten angesammelt.} \\
\texttt{Bitte versuche es noch mal!} \\
            }
        }
    }
    & \\ \\

    \theutterance \stepcounter{utterance}  
    & & & \multicolumn{4}{p{0.6\linewidth}}{
        \cellcolor[rgb]{0.9,0.9,0.9}{
            \makecell[{{p{\linewidth}}}]{
                \texttt{\tiny{[GM$\langle$P2]}}
                \texttt{SPRICH: Danke, (5,7) ist bei mir von M belegt. Aktueller Plan angepasst: Ich verschiebe zuerst O von (3,3) nach (4,7), dann T von (5,5) nach (7,7), dann S von (2,1) nach (6,7), dann P von (1,7) nach (5,7), danach A nach (1,7) und D nach (2,7). Ist (4,7) bei dir frei? Ich würde jetzt O→(4,7) machen.} \\
            }
        }
    }
     \\ \\

    \theutterance \stepcounter{utterance}  
    & & & \multicolumn{2}{p{0.3\linewidth}}{
        \cellcolor[rgb]{0.9,0.9,0.9}{
            \makecell[{{p{\linewidth}}}]{
                \texttt{\tiny{[GM$|$GM]}}
                \texttt{SPRICH: Danke, (5,7) ist bei mir von M belegt. Aktueller Plan angepasst: Ich verschiebe zuerst O von (3,3) nach (4,7), dann T von (5,5) nach (7,7), dann S von (2,1) nach (6,7), dann P von (1,7) nach (5,7), danach A nach (1,7) und D nach (2,7). Ist (4,7) bei dir frei? Ich würde jetzt O→(4,7) machen.} \\
            }
        }
    }
    & & \\ \\

    \theutterance \stepcounter{utterance}  
    & & \multicolumn{4}{p{0.6\linewidth}}{
        \cellcolor[rgb]{0.9,0.9,0.9}{
            \makecell[{{p{\linewidth}}}]{
                \texttt{\tiny{[P1$\langle$GM]}}
                \texttt{Deine Nachricht wurde an deinen Mitspieler übermittelt.} \\
\\ 
\\ 
\texttt{Ihr seid gerade bei Runde 3 von maximal 28.} \\
\texttt{Ihr habt gemeinsam 2 von maximal 16 Strafpunkten angesammelt.} \\
\texttt{Dein Mitspieler hat dir diese Nachricht geschickt:} \\
\\ 
\\ 
\texttt{"Danke, (5,7) ist bei mir von M belegt. Aktueller Plan angepasst: Ich verschiebe zuerst O von (3,3) nach (4,7), dann T von (5,5) nach (7,7), dann S von (2,1) nach (6,7), dann P von (1,7) nach (5,7), danach A nach (1,7) und D nach (2,7). Ist (4,7) bei dir frei? Ich würde jetzt O→(4,7) machen."} \\
\\ 
\\ 
\texttt{Wie lautet dein nächster Befehl?} \\
            }
        }
    }
    & \\ \\

    \theutterance \stepcounter{utterance}  
    & \multicolumn{4}{p{0.6\linewidth}}{
        \cellcolor[rgb]{0.9,0.9,0.9}{
            \makecell[{{p{\linewidth}}}]{
                \texttt{\tiny{[P1$\rangle$GM]}}
                \texttt{SPRICH: Ja, (4,7) ist bei mir frei. Ich ziehe in gleicher Reihenfolge: O→(4,7), P→(5,7), A→(1,7), D→(2,7), S→(6,7), T→(7,7). Ich warte, bis du O gesetzt hast, dann starte ich mit O→(4,7).} \\
            }
        }
    }
    & & \\ \\

    \theutterance \stepcounter{utterance}  
    & & & \multicolumn{2}{p{0.3\linewidth}}{
        \cellcolor[rgb]{0.9,0.9,0.9}{
            \makecell[{{p{\linewidth}}}]{
                \texttt{\tiny{[GM$|$GM]}}
                \texttt{SPRICH: Ja, (4,7) ist bei mir frei. Ich ziehe in gleicher Reihenfolge: O→(4,7), P→(5,7), A→(1,7), D→(2,7), S→(6,7), T→(7,7). Ich warte, bis du O gesetzt hast, dann starte ich mit O→(4,7).} \\
            }
        }
    }
    & & \\ \\

    \theutterance \stepcounter{utterance}  
    & & \multicolumn{4}{p{0.6\linewidth}}{
        \cellcolor[rgb]{0.9,0.9,0.9}{
            \makecell[{{p{\linewidth}}}]{
                \texttt{\tiny{[GM$\rangle$P2]}}
                \texttt{Deine Nachricht wurde an deinen Mitspieler übermittelt.} \\
\\ 
\\ 
\texttt{Ihr seid gerade bei Runde 4 von maximal 28.} \\
\texttt{Ihr habt gemeinsam 2 von maximal 16 Strafpunkten angesammelt.} \\
\texttt{Dein Mitspieler hat dir diese Nachricht geschickt:} \\
\\ 
\\ 
\texttt{"Ja, (4,7) ist bei mir frei. Ich ziehe in gleicher Reihenfolge: O→(4,7), P→(5,7), A→(1,7), D→(2,7), S→(6,7), T→(7,7). Ich warte, bis du O gesetzt hast, dann starte ich mit O→(4,7)."} \\
\\ 
\\ 
\texttt{Wie lautet dein nächster Befehl?} \\
            }
        }
    }
    & \\ \\

    \theutterance \stepcounter{utterance}  
    & & & \multicolumn{4}{p{0.6\linewidth}}{
        \cellcolor[rgb]{0.9,0.9,0.9}{
            \makecell[{{p{\linewidth}}}]{
                \texttt{\tiny{[GM$\langle$P2]}}
                \texttt{VERSCHIEBE: O, (4,7)} \\
            }
        }
    }
     \\ \\

    \theutterance \stepcounter{utterance}  
    & & & \multicolumn{2}{p{0.3\linewidth}}{
        \cellcolor[rgb]{0.9,0.9,0.9}{
            \makecell[{{p{\linewidth}}}]{
                \texttt{\tiny{[GM$|$GM]}}
                \texttt{VERSCHIEBE: O, (4,7)} \\
            }
        }
    }
    & & \\ \\

    \theutterance \stepcounter{utterance}  
    & & & \multicolumn{2}{p{0.3\linewidth}}{
        \cellcolor[rgb]{0.9,0.9,0.9}{
            \makecell[{{p{\linewidth}}}]{
                \texttt{\tiny{[GM$|$GM]}}
                \texttt{Strafpunkt: (4,7) ist nicht leer, sondern enthält '│'.} \\
\\ 
\texttt{Ihr habt gemeinsam 3 von maximal 16 Strafpunkten angesammelt.} \\
\texttt{Bitte versuche es noch mal!} \\
            }
        }
    }
    & & \\ \\

    \theutterance \stepcounter{utterance}  
    & & \multicolumn{4}{p{0.6\linewidth}}{
        \cellcolor[rgb]{0.9,0.9,0.9}{
            \makecell[{{p{\linewidth}}}]{
                \texttt{\tiny{[GM$\rangle$P2]}}
                \texttt{Strafpunkt: (4,7) ist nicht leer, sondern enthält '│'.} \\
\\ 
\texttt{Ihr habt gemeinsam 3 von maximal 16 Strafpunkten angesammelt.} \\
\texttt{Bitte versuche es noch mal!} \\
            }
        }
    }
    & \\ \\

    \theutterance \stepcounter{utterance}  
    & & & \multicolumn{4}{p{0.6\linewidth}}{
        \cellcolor[rgb]{0.9,0.9,0.9}{
            \makecell[{{p{\linewidth}}}]{
                \texttt{\tiny{[GM$\langle$P2]}}
                \texttt{SPRICH: Bei mir ist (4,7) blockiert durch eine Linie. Ich brauche alternative Zielreihe. Vorschlag: gleiche Reihenfolge, aber in Reihe 6: A(1,6), D(2,6), M(3,6), O(4,6), P(5,6), S(6,6), T(7,6). Kannst du das auch nutzen? Ich beginne dann mit O→(4,6).} \\
            }
        }
    }
     \\ \\

    \theutterance \stepcounter{utterance}  
    & & & \multicolumn{2}{p{0.3\linewidth}}{
        \cellcolor[rgb]{0.9,0.9,0.9}{
            \makecell[{{p{\linewidth}}}]{
                \texttt{\tiny{[GM$|$GM]}}
                \texttt{SPRICH: Bei mir ist (4,7) blockiert durch eine Linie. Ich brauche alternative Zielreihe. Vorschlag: gleiche Reihenfolge, aber in Reihe 6: A(1,6), D(2,6), M(3,6), O(4,6), P(5,6), S(6,6), T(7,6). Kannst du das auch nutzen? Ich beginne dann mit O→(4,6).} \\
            }
        }
    }
    & & \\ \\

    \theutterance \stepcounter{utterance}  
    & & \multicolumn{4}{p{0.6\linewidth}}{
        \cellcolor[rgb]{0.9,0.9,0.9}{
            \makecell[{{p{\linewidth}}}]{
                \texttt{\tiny{[P1$\langle$GM]}}
                \texttt{Deine Nachricht wurde an deinen Mitspieler übermittelt.} \\
\\ 
\\ 
\texttt{Ihr seid gerade bei Runde 4 von maximal 28.} \\
\texttt{Ihr habt gemeinsam 3 von maximal 16 Strafpunkten angesammelt.} \\
\texttt{Dein Mitspieler hat dir diese Nachricht geschickt:} \\
\\ 
\\ 
\texttt{"Bei mir ist (4,7) blockiert durch eine Linie. Ich brauche alternative Zielreihe. Vorschlag: gleiche Reihenfolge, aber in Reihe 6: A(1,6), D(2,6), M(3,6), O(4,6), P(5,6), S(6,6), T(7,6). Kannst du das auch nutzen? Ich beginne dann mit O→(4,6)."} \\
\\ 
\\ 
\texttt{Wie lautet dein nächster Befehl?} \\
            }
        }
    }
    & \\ \\

    \theutterance \stepcounter{utterance}  
    & \multicolumn{4}{p{0.6\linewidth}}{
        \cellcolor[rgb]{0.9,0.9,0.9}{
            \makecell[{{p{\linewidth}}}]{
                \texttt{\tiny{[P1$\rangle$GM]}}
                \texttt{SPRICH: Einverstanden, Zielreihe y=6 ist gut. Meine Ziele: A(1,6), D(2,6), M(3,6), O(4,6), P(5,6), S(6,6), T(7,6). Ich starte synchron mit O→(4,6). Bitte gib Bescheid, wenn (4,6) bei dir belegt/blockiert ist.} \\
            }
        }
    }
    & & \\ \\

    \theutterance \stepcounter{utterance}  
    & & & \multicolumn{2}{p{0.3\linewidth}}{
        \cellcolor[rgb]{0.9,0.9,0.9}{
            \makecell[{{p{\linewidth}}}]{
                \texttt{\tiny{[GM$|$GM]}}
                \texttt{SPRICH: Einverstanden, Zielreihe y=6 ist gut. Meine Ziele: A(1,6), D(2,6), M(3,6), O(4,6), P(5,6), S(6,6), T(7,6). Ich starte synchron mit O→(4,6). Bitte gib Bescheid, wenn (4,6) bei dir belegt/blockiert ist.} \\
            }
        }
    }
    & & \\ \\

    \theutterance \stepcounter{utterance}  
    & & \multicolumn{4}{p{0.6\linewidth}}{
        \cellcolor[rgb]{0.9,0.9,0.9}{
            \makecell[{{p{\linewidth}}}]{
                \texttt{\tiny{[GM$\rangle$P2]}}
                \texttt{Deine Nachricht wurde an deinen Mitspieler übermittelt.} \\
\\ 
\\ 
\texttt{Ihr seid gerade bei Runde 5 von maximal 28.} \\
\texttt{Ihr habt gemeinsam 3 von maximal 16 Strafpunkten angesammelt.} \\
\texttt{Dein Mitspieler hat dir diese Nachricht geschickt:} \\
\\ 
\\ 
\texttt{"Einverstanden, Zielreihe y=6 ist gut. Meine Ziele: A(1,6), D(2,6), M(3,6), O(4,6), P(5,6), S(6,6), T(7,6). Ich starte synchron mit O→(4,6). Bitte gib Bescheid, wenn (4,6) bei dir belegt/blockiert ist."} \\
\\ 
\\ 
\texttt{Wie lautet dein nächster Befehl?} \\
            }
        }
    }
    & \\ \\

    \theutterance \stepcounter{utterance}  
    & & & \multicolumn{4}{p{0.6\linewidth}}{
        \cellcolor[rgb]{0.9,0.9,0.9}{
            \makecell[{{p{\linewidth}}}]{
                \texttt{\tiny{[GM$\langle$P2]}}
                \texttt{VERSCHIEBE: O, (4,6)} \\
            }
        }
    }
     \\ \\

    \theutterance \stepcounter{utterance}  
    & & & \multicolumn{2}{p{0.3\linewidth}}{
        \cellcolor[rgb]{0.9,0.9,0.9}{
            \makecell[{{p{\linewidth}}}]{
                \texttt{\tiny{[GM$|$GM]}}
                \texttt{VERSCHIEBE: O, (4,6)} \\
            }
        }
    }
    & & \\ \\

    \theutterance \stepcounter{utterance}  
    & & & \multicolumn{2}{p{0.3\linewidth}}{
        \cellcolor[rgb]{0.9,0.9,0.9}{
            \makecell[{{p{\linewidth}}}]{
                \texttt{\tiny{[GM$|$GM]}}
                \texttt{Strafpunkt: (4,6) ist nicht leer, sondern enthält '├'.} \\
\\ 
\texttt{Ihr habt gemeinsam 4 von maximal 16 Strafpunkten angesammelt.} \\
\texttt{Bitte versuche es noch mal!} \\
            }
        }
    }
    & & \\ \\

    \theutterance \stepcounter{utterance}  
    & & \multicolumn{4}{p{0.6\linewidth}}{
        \cellcolor[rgb]{0.9,0.9,0.9}{
            \makecell[{{p{\linewidth}}}]{
                \texttt{\tiny{[GM$\rangle$P2]}}
                \texttt{Strafpunkt: (4,6) ist nicht leer, sondern enthält '├'.} \\
\\ 
\texttt{Ihr habt gemeinsam 4 von maximal 16 Strafpunkten angesammelt.} \\
\texttt{Bitte versuche es noch mal!} \\
            }
        }
    }
    & \\ \\

    \theutterance \stepcounter{utterance}  
    & & & \multicolumn{4}{p{0.6\linewidth}}{
        \cellcolor[rgb]{0.9,0.9,0.9}{
            \makecell[{{p{\linewidth}}}]{
                \texttt{\tiny{[GM$\langle$P2]}}
                \texttt{SPRICH: Bei mir sind in Reihe 6 mehrere Linien/Blocker. Ich schlage stattdessen Reihe 4 vor: A(1,4), D(2,4), M(3,4), O(4,4), P(5,4), S(6,4), T(7,4). Bitte bestätige, ob y=4 bei dir frei genug ist. Wenn ja, beginne ich mit O→(4,4). Wenn nicht, nenne mir eine freie Reihe ohne Linien bei dir.} \\
            }
        }
    }
     \\ \\

    \theutterance \stepcounter{utterance}  
    & & & \multicolumn{2}{p{0.3\linewidth}}{
        \cellcolor[rgb]{0.9,0.9,0.9}{
            \makecell[{{p{\linewidth}}}]{
                \texttt{\tiny{[GM$|$GM]}}
                \texttt{SPRICH: Bei mir sind in Reihe 6 mehrere Linien/Blocker. Ich schlage stattdessen Reihe 4 vor: A(1,4), D(2,4), M(3,4), O(4,4), P(5,4), S(6,4), T(7,4). Bitte bestätige, ob y=4 bei dir frei genug ist. Wenn ja, beginne ich mit O→(4,4). Wenn nicht, nenne mir eine freie Reihe ohne Linien bei dir.} \\
            }
        }
    }
    & & \\ \\

    \theutterance \stepcounter{utterance}  
    & & \multicolumn{4}{p{0.6\linewidth}}{
        \cellcolor[rgb]{0.9,0.9,0.9}{
            \makecell[{{p{\linewidth}}}]{
                \texttt{\tiny{[P1$\langle$GM]}}
                \texttt{Deine Nachricht wurde an deinen Mitspieler übermittelt.} \\
\\ 
\\ 
\texttt{Ihr seid gerade bei Runde 5 von maximal 28.} \\
\texttt{Ihr habt gemeinsam 4 von maximal 16 Strafpunkten angesammelt.} \\
\texttt{Dein Mitspieler hat dir diese Nachricht geschickt:} \\
\\ 
\\ 
\texttt{"Bei mir sind in Reihe 6 mehrere Linien/Blocker. Ich schlage stattdessen Reihe 4 vor: A(1,4), D(2,4), M(3,4), O(4,4), P(5,4), S(6,4), T(7,4). Bitte bestätige, ob y=4 bei dir frei genug ist. Wenn ja, beginne ich mit O→(4,4). Wenn nicht, nenne mir eine freie Reihe ohne Linien bei dir."} \\
\\ 
\\ 
\texttt{Wie lautet dein nächster Befehl?} \\
            }
        }
    }
    & \\ \\

    \theutterance \stepcounter{utterance}  
    & \multicolumn{4}{p{0.6\linewidth}}{
        \cellcolor[rgb]{0.9,0.9,0.9}{
            \makecell[{{p{\linewidth}}}]{
                \texttt{\tiny{[P1$\rangle$GM]}}
                \texttt{SPRICH: Bei mir ist Reihe 4 frei. Einverstanden mit Ziel y=4: A(1,4), D(2,4), M(3,4), O(4,4), P(5,4), S(6,4), T(7,4). Starte du mit O→(4,4); ich ziehe danach synchron meine Objekte in diese Reihe.} \\
            }
        }
    }
    & & \\ \\

    \theutterance \stepcounter{utterance}  
    & & & \multicolumn{2}{p{0.3\linewidth}}{
        \cellcolor[rgb]{0.9,0.9,0.9}{
            \makecell[{{p{\linewidth}}}]{
                \texttt{\tiny{[GM$|$GM]}}
                \texttt{SPRICH: Bei mir ist Reihe 4 frei. Einverstanden mit Ziel y=4: A(1,4), D(2,4), M(3,4), O(4,4), P(5,4), S(6,4), T(7,4). Starte du mit O→(4,4); ich ziehe danach synchron meine Objekte in diese Reihe.} \\
            }
        }
    }
    & & \\ \\

    \theutterance \stepcounter{utterance}  
    & & \multicolumn{4}{p{0.6\linewidth}}{
        \cellcolor[rgb]{0.9,0.9,0.9}{
            \makecell[{{p{\linewidth}}}]{
                \texttt{\tiny{[GM$\rangle$P2]}}
                \texttt{Deine Nachricht wurde an deinen Mitspieler übermittelt.} \\
\\ 
\\ 
\texttt{Ihr seid gerade bei Runde 6 von maximal 28.} \\
\texttt{Ihr habt gemeinsam 4 von maximal 16 Strafpunkten angesammelt.} \\
\texttt{Dein Mitspieler hat dir diese Nachricht geschickt:} \\
\\ 
\\ 
\texttt{"Bei mir ist Reihe 4 frei. Einverstanden mit Ziel y=4: A(1,4), D(2,4), M(3,4), O(4,4), P(5,4), S(6,4), T(7,4). Starte du mit O→(4,4); ich ziehe danach synchron meine Objekte in diese Reihe."} \\
\\ 
\\ 
\texttt{Wie lautet dein nächster Befehl?} \\
            }
        }
    }
    & \\ \\

    \theutterance \stepcounter{utterance}  
    & & & \multicolumn{4}{p{0.6\linewidth}}{
        \cellcolor[rgb]{0.9,0.9,0.9}{
            \makecell[{{p{\linewidth}}}]{
                \texttt{\tiny{[GM$\langle$P2]}}
                \texttt{SPRICH: Bevor ich ziehe: Bei mir ist (4,4) frei? Ich sehe dort eine Kreuzung '┼' bei (4,4). Reihe 4 hat Linien: z.B. (2,4) '┐', (3,4) '◌', (4,4) '┼', (5,4) '─'. Ich brauche eine wirklich freie Reihe ohne Linien bei mir. Freie Kandidaten laut Sicht: y=3 wirkt größtenteils frei (A bei (1,3), O bei (3,3) stehen dort), y=5 hat viele Wände, y=7 hat Rahmen/Spaltenlinien. Vorschlag: Zielreihe y=3: A(1,3), D(2,3), M(3,3), O(4,3), P(5,3), S(6,3), T(7,3). Ich kann dann Objekte innerhalb von y=3 anordnen. Bestätige y=3 oder schlage eine alternative freie Reihe vor.} \\
            }
        }
    }
     \\ \\

    \theutterance \stepcounter{utterance}  
    & & & \multicolumn{2}{p{0.3\linewidth}}{
        \cellcolor[rgb]{0.9,0.9,0.9}{
            \makecell[{{p{\linewidth}}}]{
                \texttt{\tiny{[GM$|$GM]}}
                \texttt{SPRICH: Bevor ich ziehe: Bei mir ist (4,4) frei? Ich sehe dort eine Kreuzung '┼' bei (4,4). Reihe 4 hat Linien: z.B. (2,4) '┐', (3,4) '◌', (4,4) '┼', (5,4) '─'. Ich brauche eine wirklich freie Reihe ohne Linien bei mir. Freie Kandidaten laut Sicht: y=3 wirkt größtenteils frei (A bei (1,3), O bei (3,3) stehen dort), y=5 hat viele Wände, y=7 hat Rahmen/Spaltenlinien. Vorschlag: Zielreihe y=3: A(1,3), D(2,3), M(3,3), O(4,3), P(5,3), S(6,3), T(7,3). Ich kann dann Objekte innerhalb von y=3 anordnen. Bestätige y=3 oder schlage eine alternative freie Reihe vor.} \\
            }
        }
    }
    & & \\ \\

    \theutterance \stepcounter{utterance}  
    & & \multicolumn{4}{p{0.6\linewidth}}{
        \cellcolor[rgb]{0.9,0.9,0.9}{
            \makecell[{{p{\linewidth}}}]{
                \texttt{\tiny{[P1$\langle$GM]}}
                \texttt{Deine Nachricht wurde an deinen Mitspieler übermittelt.} \\
\\ 
\\ 
\texttt{Ihr seid gerade bei Runde 6 von maximal 28.} \\
\texttt{Ihr habt gemeinsam 4 von maximal 16 Strafpunkten angesammelt.} \\
\texttt{Dein Mitspieler hat dir diese Nachricht geschickt:} \\
\\ 
\\ 
\texttt{"Bevor ich ziehe: Bei mir ist (4,4) frei? Ich sehe dort eine Kreuzung '┼' bei (4,4). Reihe 4 hat Linien: z.B. (2,4) '┐', (3,4) '◌', (4,4) '┼', (5,4) '─'. Ich brauche eine wirklich freie Reihe ohne Linien bei mir. Freie Kandidaten laut Sicht: y=3 wirkt größtenteils frei (A bei (1,3), O bei (3,3) stehen dort), y=5 hat viele Wände, y=7 hat Rahmen/Spaltenlinien. Vorschlag: Zielreihe y=3: A(1,3), D(2,3), M(3,3), O(4,3), P(5,3), S(6,3), T(7,3). Ich kann dann Objekte innerhalb von y=3 anordnen. Bestätige y=3 oder schlage eine alternative freie Reihe vor."} \\
\\ 
\\ 
\texttt{Wie lautet dein nächster Befehl?} \\
            }
        }
    }
    & \\ \\

    \theutterance \stepcounter{utterance}  
    & \multicolumn{4}{p{0.6\linewidth}}{
        \cellcolor[rgb]{0.9,0.9,0.9}{
            \makecell[{{p{\linewidth}}}]{
                \texttt{\tiny{[P1$\rangle$GM]}}
                \texttt{SPRICH: Bestätige Zielreihe y=3: A(1,3), D(2,3), M(3,3), O(4,3), P(5,3), S(6,3), T(7,3). Ich beginne nach deiner Bestätigung mit O→(4,3). Ist (4,3) bei dir frei?} \\
            }
        }
    }
    & & \\ \\

    \theutterance \stepcounter{utterance}  
    & & & \multicolumn{2}{p{0.3\linewidth}}{
        \cellcolor[rgb]{0.9,0.9,0.9}{
            \makecell[{{p{\linewidth}}}]{
                \texttt{\tiny{[GM$|$GM]}}
                \texttt{SPRICH: Bestätige Zielreihe y=3: A(1,3), D(2,3), M(3,3), O(4,3), P(5,3), S(6,3), T(7,3). Ich beginne nach deiner Bestätigung mit O→(4,3). Ist (4,3) bei dir frei?} \\
            }
        }
    }
    & & \\ \\

    \theutterance \stepcounter{utterance}  
    & & \multicolumn{4}{p{0.6\linewidth}}{
        \cellcolor[rgb]{0.9,0.9,0.9}{
            \makecell[{{p{\linewidth}}}]{
                \texttt{\tiny{[GM$\rangle$P2]}}
                \texttt{Deine Nachricht wurde an deinen Mitspieler übermittelt.} \\
\\ 
\\ 
\texttt{Ihr seid gerade bei Runde 7 von maximal 28.} \\
\texttt{Ihr habt gemeinsam 4 von maximal 16 Strafpunkten angesammelt.} \\
\texttt{Dein Mitspieler hat dir diese Nachricht geschickt:} \\
\\ 
\\ 
\texttt{"Bestätige Zielreihe y=3: A(1,3), D(2,3), M(3,3), O(4,3), P(5,3), S(6,3), T(7,3). Ich beginne nach deiner Bestätigung mit O→(4,3). Ist (4,3) bei dir frei?"} \\
\\ 
\\ 
\texttt{Wie lautet dein nächster Befehl?} \\
            }
        }
    }
    & \\ \\

    \theutterance \stepcounter{utterance}  
    & & & \multicolumn{4}{p{0.6\linewidth}}{
        \cellcolor[rgb]{0.9,0.9,0.9}{
            \makecell[{{p{\linewidth}}}]{
                \texttt{\tiny{[GM$\langle$P2]}}
                \texttt{SPRICH: Ja, (4,3) ist bei mir frei. Ich starte jetzt mit O→(4,3), danach P→(5,3), S→(6,3), T→(7,3), A→(1,3), D→(2,3), M bleibt (3,3). Bitte starte ebenfalls mit O→(4,3).} \\
            }
        }
    }
     \\ \\

    \theutterance \stepcounter{utterance}  
    & & & \multicolumn{2}{p{0.3\linewidth}}{
        \cellcolor[rgb]{0.9,0.9,0.9}{
            \makecell[{{p{\linewidth}}}]{
                \texttt{\tiny{[GM$|$GM]}}
                \texttt{SPRICH: Ja, (4,3) ist bei mir frei. Ich starte jetzt mit O→(4,3), danach P→(5,3), S→(6,3), T→(7,3), A→(1,3), D→(2,3), M bleibt (3,3). Bitte starte ebenfalls mit O→(4,3).} \\
            }
        }
    }
    & & \\ \\

    \theutterance \stepcounter{utterance}  
    & & \multicolumn{4}{p{0.6\linewidth}}{
        \cellcolor[rgb]{0.9,0.9,0.9}{
            \makecell[{{p{\linewidth}}}]{
                \texttt{\tiny{[P1$\langle$GM]}}
                \texttt{Deine Nachricht wurde an deinen Mitspieler übermittelt.} \\
\\ 
\\ 
\texttt{Ihr seid gerade bei Runde 7 von maximal 28.} \\
\texttt{Ihr habt gemeinsam 4 von maximal 16 Strafpunkten angesammelt.} \\
\texttt{Dein Mitspieler hat dir diese Nachricht geschickt:} \\
\\ 
\\ 
\texttt{"Ja, (4,3) ist bei mir frei. Ich starte jetzt mit O→(4,3), danach P→(5,3), S→(6,3), T→(7,3), A→(1,3), D→(2,3), M bleibt (3,3). Bitte starte ebenfalls mit O→(4,3)."} \\
\\ 
\\ 
\texttt{Wie lautet dein nächster Befehl?} \\
            }
        }
    }
    & \\ \\

    \theutterance \stepcounter{utterance}  
    & \multicolumn{4}{p{0.6\linewidth}}{
        \cellcolor[rgb]{0.9,0.9,0.9}{
            \makecell[{{p{\linewidth}}}]{
                \texttt{\tiny{[P1$\rangle$GM]}}
                \texttt{VERSCHIEBE: O, (4,3)} \\
            }
        }
    }
    & & \\ \\

    \theutterance \stepcounter{utterance}  
    & & & \multicolumn{2}{p{0.3\linewidth}}{
        \cellcolor[rgb]{0.9,0.9,0.9}{
            \makecell[{{p{\linewidth}}}]{
                \texttt{\tiny{[GM$|$GM]}}
                \texttt{VERSCHIEBE: O, (4,3)} \\
            }
        }
    }
    & & \\ \\

    \theutterance \stepcounter{utterance}  
    & & & \multicolumn{2}{p{0.3\linewidth}}{
        \cellcolor[rgb]{0.9,0.9,0.9}{
            \makecell[{{p{\linewidth}}}]{
                \texttt{\tiny{[GM$|$GM]}}
                \texttt{'O' erfolgreich nach (4,3) verschoben. Dein Raster sieht jetzt so aus:} \\
\texttt{```} \\
\texttt{ 1234567} \\
\texttt{╔══╤════╗} \\
\texttt{║◌T│D◌◌◌║ 1} \\
\texttt{╟──┘◌◌┌─╢ 2} \\
\texttt{║SA◌O◌│◌║ 3} \\
\texttt{╟─┐◌┌─┼─╢ 4} \\
\texttt{║◌├─┤◌│◌║ 5} \\
\texttt{╟─┤◌├─┤◌║ 6} \\
\texttt{║◌│P│M│◌║ 7} \\
\texttt{╚═╧═╧═╧═╝} \\
\texttt{```} \\
            }
        }
    }
    & & \\ \\

    \theutterance \stepcounter{utterance}  
    & & \multicolumn{4}{p{0.6\linewidth}}{
        \cellcolor[rgb]{0.9,0.9,0.9}{
            \makecell[{{p{\linewidth}}}]{
                \texttt{\tiny{[GM$\rangle$P2]}}
                \texttt{Deine Nachricht wurde an deinen Mitspieler übermittelt.} \\
\\ 
\\ 
\texttt{Ihr seid gerade bei Runde 8 von maximal 28.} \\
\texttt{Ihr habt gemeinsam 4 von maximal 16 Strafpunkten angesammelt.} \\
\texttt{Dein Mitspieler hat ein Objekt auf seinem Raster bewegt.} \\
\\ 
\\ 
\texttt{Wie lautet dein nächster Befehl?} \\
            }
        }
    }
    & \\ \\

    \theutterance \stepcounter{utterance}  
    & & & \multicolumn{4}{p{0.6\linewidth}}{
        \cellcolor[rgb]{0.9,0.9,0.9}{
            \makecell[{{p{\linewidth}}}]{
                \texttt{\tiny{[GM$\langle$P2]}}
                \texttt{VERSCHIEBE: O, (4,3)} \\
            }
        }
    }
     \\ \\

    \theutterance \stepcounter{utterance}  
    & & & \multicolumn{2}{p{0.3\linewidth}}{
        \cellcolor[rgb]{0.9,0.9,0.9}{
            \makecell[{{p{\linewidth}}}]{
                \texttt{\tiny{[GM$|$GM]}}
                \texttt{VERSCHIEBE: O, (4,3)} \\
            }
        }
    }
    & & \\ \\

    \theutterance \stepcounter{utterance}  
    & & & \multicolumn{2}{p{0.3\linewidth}}{
        \cellcolor[rgb]{0.9,0.9,0.9}{
            \makecell[{{p{\linewidth}}}]{
                \texttt{\tiny{[GM$|$GM]}}
                \texttt{'O' erfolgreich nach (4,3) verschoben. Dein Raster sieht jetzt so aus:} \\
\texttt{```} \\
\texttt{ 1234567} \\
\texttt{╔══╤════╗} \\
\texttt{║◌S│◌◌◌◌║ 1} \\
\texttt{╟──┘◌◌┌─╢ 2} \\
\texttt{║A◌◌O◌│◌║ 3} \\
\texttt{╟─┐◌┌─┼─╢ 4} \\
\texttt{║◌├─┤T│D║ 5} \\
\texttt{╟─┤◌├─┤◌║ 6} \\
\texttt{║P│◌│M│◌║ 7} \\
\texttt{╚═╧═╧═╧═╝} \\
\texttt{```} \\
            }
        }
    }
    & & \\ \\

    \theutterance \stepcounter{utterance}  
    & & \multicolumn{4}{p{0.6\linewidth}}{
        \cellcolor[rgb]{0.9,0.9,0.9}{
            \makecell[{{p{\linewidth}}}]{
                \texttt{\tiny{[P1$\langle$GM]}}
                \texttt{'O' erfolgreich nach (4,3) verschoben. Dein Raster sieht jetzt so aus:} \\
\texttt{```} \\
\texttt{ 1234567} \\
\texttt{╔══╤════╗} \\
\texttt{║◌T│D◌◌◌║ 1} \\
\texttt{╟──┘◌◌┌─╢ 2} \\
\texttt{║SA◌O◌│◌║ 3} \\
\texttt{╟─┐◌┌─┼─╢ 4} \\
\texttt{║◌├─┤◌│◌║ 5} \\
\texttt{╟─┤◌├─┤◌║ 6} \\
\texttt{║◌│P│M│◌║ 7} \\
\texttt{╚═╧═╧═╧═╝} \\
\texttt{```} \\
\texttt{Ihr seid gerade bei Runde 8 von maximal 28.} \\
\texttt{Ihr habt gemeinsam 4 von maximal 16 Strafpunkten angesammelt.} \\
\texttt{Dein Mitspieler hat ein Objekt auf seinem Raster bewegt.} \\
\\ 
\\ 
\texttt{Wie lautet dein nächster Befehl?} \\
            }
        }
    }
    & \\ \\

    \theutterance \stepcounter{utterance}  
    & \multicolumn{4}{p{0.6\linewidth}}{
        \cellcolor[rgb]{0.9,0.9,0.9}{
            \makecell[{{p{\linewidth}}}]{
                \texttt{\tiny{[P1$\rangle$GM]}}
                \texttt{VERSCHIEBE: P, (5,3)} \\
            }
        }
    }
    & & \\ \\

    \theutterance \stepcounter{utterance}  
    & & & \multicolumn{2}{p{0.3\linewidth}}{
        \cellcolor[rgb]{0.9,0.9,0.9}{
            \makecell[{{p{\linewidth}}}]{
                \texttt{\tiny{[GM$|$GM]}}
                \texttt{VERSCHIEBE: P, (5,3)} \\
            }
        }
    }
    & & \\ \\

    \theutterance \stepcounter{utterance}  
    & & & \multicolumn{2}{p{0.3\linewidth}}{
        \cellcolor[rgb]{0.9,0.9,0.9}{
            \makecell[{{p{\linewidth}}}]{
                \texttt{\tiny{[GM$|$GM]}}
                \texttt{'P' erfolgreich nach (5,3) verschoben. Dein Raster sieht jetzt so aus:} \\
\texttt{```} \\
\texttt{ 1234567} \\
\texttt{╔══╤════╗} \\
\texttt{║◌T│D◌◌◌║ 1} \\
\texttt{╟──┘◌◌┌─╢ 2} \\
\texttt{║SA◌OP│◌║ 3} \\
\texttt{╟─┐◌┌─┼─╢ 4} \\
\texttt{║◌├─┤◌│◌║ 5} \\
\texttt{╟─┤◌├─┤◌║ 6} \\
\texttt{║◌│◌│M│◌║ 7} \\
\texttt{╚═╧═╧═╧═╝} \\
\texttt{```} \\
            }
        }
    }
    & & \\ \\

    \theutterance \stepcounter{utterance}  
    & & \multicolumn{4}{p{0.6\linewidth}}{
        \cellcolor[rgb]{0.9,0.9,0.9}{
            \makecell[{{p{\linewidth}}}]{
                \texttt{\tiny{[GM$\rangle$P2]}}
                \texttt{'O' erfolgreich nach (4,3) verschoben. Dein Raster sieht jetzt so aus:} \\
\texttt{```} \\
\texttt{ 1234567} \\
\texttt{╔══╤════╗} \\
\texttt{║◌S│◌◌◌◌║ 1} \\
\texttt{╟──┘◌◌┌─╢ 2} \\
\texttt{║A◌◌O◌│◌║ 3} \\
\texttt{╟─┐◌┌─┼─╢ 4} \\
\texttt{║◌├─┤T│D║ 5} \\
\texttt{╟─┤◌├─┤◌║ 6} \\
\texttt{║P│◌│M│◌║ 7} \\
\texttt{╚═╧═╧═╧═╝} \\
\texttt{```} \\
\texttt{Ihr seid gerade bei Runde 9 von maximal 28.} \\
\texttt{Ihr habt gemeinsam 4 von maximal 16 Strafpunkten angesammelt.} \\
\texttt{Dein Mitspieler hat ein Objekt auf seinem Raster bewegt.} \\
\\ 
\\ 
\texttt{Wie lautet dein nächster Befehl?} \\
            }
        }
    }
    & \\ \\

    \theutterance \stepcounter{utterance}  
    & & & \multicolumn{4}{p{0.6\linewidth}}{
        \cellcolor[rgb]{0.9,0.9,0.9}{
            \makecell[{{p{\linewidth}}}]{
                \texttt{\tiny{[GM$\langle$P2]}}
                \texttt{VERSCHIEBE: P, (5,3)} \\
            }
        }
    }
     \\ \\

    \theutterance \stepcounter{utterance}  
    & & & \multicolumn{2}{p{0.3\linewidth}}{
        \cellcolor[rgb]{0.9,0.9,0.9}{
            \makecell[{{p{\linewidth}}}]{
                \texttt{\tiny{[GM$|$GM]}}
                \texttt{VERSCHIEBE: P, (5,3)} \\
            }
        }
    }
    & & \\ \\

    \theutterance \stepcounter{utterance}  
    & & & \multicolumn{2}{p{0.3\linewidth}}{
        \cellcolor[rgb]{0.9,0.9,0.9}{
            \makecell[{{p{\linewidth}}}]{
                \texttt{\tiny{[GM$|$GM]}}
                \texttt{'P' erfolgreich nach (5,3) verschoben. Dein Raster sieht jetzt so aus:} \\
\texttt{```} \\
\texttt{ 1234567} \\
\texttt{╔══╤════╗} \\
\texttt{║◌S│◌◌◌◌║ 1} \\
\texttt{╟──┘◌◌┌─╢ 2} \\
\texttt{║A◌◌OP│◌║ 3} \\
\texttt{╟─┐◌┌─┼─╢ 4} \\
\texttt{║◌├─┤T│D║ 5} \\
\texttt{╟─┤◌├─┤◌║ 6} \\
\texttt{║◌│◌│M│◌║ 7} \\
\texttt{╚═╧═╧═╧═╝} \\
\texttt{```} \\
            }
        }
    }
    & & \\ \\

    \theutterance \stepcounter{utterance}  
    & & \multicolumn{4}{p{0.6\linewidth}}{
        \cellcolor[rgb]{0.9,0.9,0.9}{
            \makecell[{{p{\linewidth}}}]{
                \texttt{\tiny{[P1$\langle$GM]}}
                \texttt{'P' erfolgreich nach (5,3) verschoben. Dein Raster sieht jetzt so aus:} \\
\texttt{```} \\
\texttt{ 1234567} \\
\texttt{╔══╤════╗} \\
\texttt{║◌T│D◌◌◌║ 1} \\
\texttt{╟──┘◌◌┌─╢ 2} \\
\texttt{║SA◌OP│◌║ 3} \\
\texttt{╟─┐◌┌─┼─╢ 4} \\
\texttt{║◌├─┤◌│◌║ 5} \\
\texttt{╟─┤◌├─┤◌║ 6} \\
\texttt{║◌│◌│M│◌║ 7} \\
\texttt{╚═╧═╧═╧═╝} \\
\texttt{```} \\
\texttt{Ihr seid gerade bei Runde 9 von maximal 28.} \\
\texttt{Ihr habt gemeinsam 4 von maximal 16 Strafpunkten angesammelt.} \\
\texttt{Dein Mitspieler hat ein Objekt auf seinem Raster bewegt.} \\
\\ 
\\ 
\texttt{Wie lautet dein nächster Befehl?} \\
            }
        }
    }
    & \\ \\

    \theutterance \stepcounter{utterance}  
    & \multicolumn{4}{p{0.6\linewidth}}{
        \cellcolor[rgb]{0.9,0.9,0.9}{
            \makecell[{{p{\linewidth}}}]{
                \texttt{\tiny{[P1$\rangle$GM]}}
                \texttt{VERSCHIEBE: S, (6,3)} \\
            }
        }
    }
    & & \\ \\

    \theutterance \stepcounter{utterance}  
    & & & \multicolumn{2}{p{0.3\linewidth}}{
        \cellcolor[rgb]{0.9,0.9,0.9}{
            \makecell[{{p{\linewidth}}}]{
                \texttt{\tiny{[GM$|$GM]}}
                \texttt{VERSCHIEBE: S, (6,3)} \\
            }
        }
    }
    & & \\ \\

    \theutterance \stepcounter{utterance}  
    & & & \multicolumn{2}{p{0.3\linewidth}}{
        \cellcolor[rgb]{0.9,0.9,0.9}{
            \makecell[{{p{\linewidth}}}]{
                \texttt{\tiny{[GM$|$GM]}}
                \texttt{Strafpunkt: (6,3) ist nicht leer, sondern enthält '│'.} \\
\\ 
\texttt{Ihr habt gemeinsam 5 von maximal 16 Strafpunkten angesammelt.} \\
\texttt{Bitte versuche es noch mal!} \\
            }
        }
    }
    & & \\ \\

    \theutterance \stepcounter{utterance}  
    & & \multicolumn{4}{p{0.6\linewidth}}{
        \cellcolor[rgb]{0.9,0.9,0.9}{
            \makecell[{{p{\linewidth}}}]{
                \texttt{\tiny{[P1$\langle$GM]}}
                \texttt{Strafpunkt: (6,3) ist nicht leer, sondern enthält '│'.} \\
\\ 
\texttt{Ihr habt gemeinsam 5 von maximal 16 Strafpunkten angesammelt.} \\
\texttt{Bitte versuche es noch mal!} \\
            }
        }
    }
    & \\ \\

    \theutterance \stepcounter{utterance}  
    & \multicolumn{4}{p{0.6\linewidth}}{
        \cellcolor[rgb]{0.9,0.9,0.9}{
            \makecell[{{p{\linewidth}}}]{
                \texttt{\tiny{[P1$\rangle$GM]}}
                \texttt{SPRICH: Fehler bei mir: (6,3) ist blockiert. Alternative Anordnung für y=3 vorgeschlagen: A(1,3), D(2,3), M(3,3), O(4,3), P(5,3), T(7,3) und S weichen wir auf y=2 nach (6,2) aus. Kannst du S auf (6,2) legen? Ich setze S dann auf (6,2), T auf (7,3). Bitte bestätigen.} \\
            }
        }
    }
    & & \\ \\

    \theutterance \stepcounter{utterance}  
    & & & \multicolumn{2}{p{0.3\linewidth}}{
        \cellcolor[rgb]{0.9,0.9,0.9}{
            \makecell[{{p{\linewidth}}}]{
                \texttt{\tiny{[GM$|$GM]}}
                \texttt{SPRICH: Fehler bei mir: (6,3) ist blockiert. Alternative Anordnung für y=3 vorgeschlagen: A(1,3), D(2,3), M(3,3), O(4,3), P(5,3), T(7,3) und S weichen wir auf y=2 nach (6,2) aus. Kannst du S auf (6,2) legen? Ich setze S dann auf (6,2), T auf (7,3). Bitte bestätigen.} \\
            }
        }
    }
    & & \\ \\

    \theutterance \stepcounter{utterance}  
    & & \multicolumn{4}{p{0.6\linewidth}}{
        \cellcolor[rgb]{0.9,0.9,0.9}{
            \makecell[{{p{\linewidth}}}]{
                \texttt{\tiny{[GM$\rangle$P2]}}
                \texttt{'P' erfolgreich nach (5,3) verschoben. Dein Raster sieht jetzt so aus:} \\
\texttt{```} \\
\texttt{ 1234567} \\
\texttt{╔══╤════╗} \\
\texttt{║◌S│◌◌◌◌║ 1} \\
\texttt{╟──┘◌◌┌─╢ 2} \\
\texttt{║A◌◌OP│◌║ 3} \\
\texttt{╟─┐◌┌─┼─╢ 4} \\
\texttt{║◌├─┤T│D║ 5} \\
\texttt{╟─┤◌├─┤◌║ 6} \\
\texttt{║◌│◌│M│◌║ 7} \\
\texttt{╚═╧═╧═╧═╝} \\
\texttt{```} \\
\texttt{Ihr seid gerade bei Runde 10 von maximal 28.} \\
\texttt{Ihr habt gemeinsam 5 von maximal 16 Strafpunkten angesammelt.} \\
\texttt{Dein Mitspieler hat dir diese Nachricht geschickt:} \\
\\ 
\\ 
\texttt{"Fehler bei mir: (6,3) ist blockiert. Alternative Anordnung für y=3 vorgeschlagen: A(1,3), D(2,3), M(3,3), O(4,3), P(5,3), T(7,3) und S weichen wir auf y=2 nach (6,2) aus. Kannst du S auf (6,2) legen? Ich setze S dann auf (6,2), T auf (7,3). Bitte bestätigen."} \\
\\ 
\\ 
\texttt{Wie lautet dein nächster Befehl?} \\
            }
        }
    }
    & \\ \\

    \theutterance \stepcounter{utterance}  
    & & & \multicolumn{4}{p{0.6\linewidth}}{
        \cellcolor[rgb]{0.9,0.9,0.9}{
            \makecell[{{p{\linewidth}}}]{
                \texttt{\tiny{[GM$\langle$P2]}}
                \texttt{SPRICH: Einverstanden. Neue Ziele: A(1,3), D(2,3), M(3,3), O(4,3), P(5,3), S(6,2), T(7,3). Ich setze jetzt S→(6,2), danach T→(7,3). Bitte bestätige, wenn du synchron ziehst.} \\
            }
        }
    }
     \\ \\

    \theutterance \stepcounter{utterance}  
    & & & \multicolumn{2}{p{0.3\linewidth}}{
        \cellcolor[rgb]{0.9,0.9,0.9}{
            \makecell[{{p{\linewidth}}}]{
                \texttt{\tiny{[GM$|$GM]}}
                \texttt{SPRICH: Einverstanden. Neue Ziele: A(1,3), D(2,3), M(3,3), O(4,3), P(5,3), S(6,2), T(7,3). Ich setze jetzt S→(6,2), danach T→(7,3). Bitte bestätige, wenn du synchron ziehst.} \\
            }
        }
    }
    & & \\ \\

    \theutterance \stepcounter{utterance}  
    & & \multicolumn{4}{p{0.6\linewidth}}{
        \cellcolor[rgb]{0.9,0.9,0.9}{
            \makecell[{{p{\linewidth}}}]{
                \texttt{\tiny{[P1$\langle$GM]}}
                \texttt{Deine Nachricht wurde an deinen Mitspieler übermittelt.} \\
\\ 
\\ 
\texttt{Ihr seid gerade bei Runde 10 von maximal 28.} \\
\texttt{Ihr habt gemeinsam 5 von maximal 16 Strafpunkten angesammelt.} \\
\texttt{Dein Mitspieler hat dir diese Nachricht geschickt:} \\
\\ 
\\ 
\texttt{"Einverstanden. Neue Ziele: A(1,3), D(2,3), M(3,3), O(4,3), P(5,3), S(6,2), T(7,3). Ich setze jetzt S→(6,2), danach T→(7,3). Bitte bestätige, wenn du synchron ziehst."} \\
\\ 
\\ 
\texttt{Wie lautet dein nächster Befehl?} \\
            }
        }
    }
    & \\ \\

    \theutterance \stepcounter{utterance}  
    & \multicolumn{4}{p{0.6\linewidth}}{
        \cellcolor[rgb]{0.9,0.9,0.9}{
            \makecell[{{p{\linewidth}}}]{
                \texttt{\tiny{[P1$\rangle$GM]}}
                \texttt{VERSCHIEBE: S, (6,2)} \\
            }
        }
    }
    & & \\ \\

    \theutterance \stepcounter{utterance}  
    & & & \multicolumn{2}{p{0.3\linewidth}}{
        \cellcolor[rgb]{0.9,0.9,0.9}{
            \makecell[{{p{\linewidth}}}]{
                \texttt{\tiny{[GM$|$GM]}}
                \texttt{VERSCHIEBE: S, (6,2)} \\
            }
        }
    }
    & & \\ \\

    \theutterance \stepcounter{utterance}  
    & & & \multicolumn{2}{p{0.3\linewidth}}{
        \cellcolor[rgb]{0.9,0.9,0.9}{
            \makecell[{{p{\linewidth}}}]{
                \texttt{\tiny{[GM$|$GM]}}
                \texttt{Strafpunkt: (6,2) ist nicht leer, sondern enthält '┌'.} \\
\\ 
\texttt{Ihr habt gemeinsam 6 von maximal 16 Strafpunkten angesammelt.} \\
\texttt{Bitte versuche es noch mal!} \\
            }
        }
    }
    & & \\ \\

    \theutterance \stepcounter{utterance}  
    & & \multicolumn{4}{p{0.6\linewidth}}{
        \cellcolor[rgb]{0.9,0.9,0.9}{
            \makecell[{{p{\linewidth}}}]{
                \texttt{\tiny{[P1$\langle$GM]}}
                \texttt{Strafpunkt: (6,2) ist nicht leer, sondern enthält '┌'.} \\
\\ 
\texttt{Ihr habt gemeinsam 6 von maximal 16 Strafpunkten angesammelt.} \\
\texttt{Bitte versuche es noch mal!} \\
            }
        }
    }
    & \\ \\

    \theutterance \stepcounter{utterance}  
    & \multicolumn{4}{p{0.6\linewidth}}{
        \cellcolor[rgb]{0.9,0.9,0.9}{
            \makecell[{{p{\linewidth}}}]{
                \texttt{\tiny{[P1$\rangle$GM]}}
                \texttt{SPRICH: Bei mir ist (6,2) blockiert durch '┌'. Vorschlag: Lassen wir S auf y=5 setzen: (6,5) ist bei mir frei? Nein, dort stand vorher O. Aktuell frei in meiner Nähe: (1,5), (4,5), (7,5) scheinen frei. Alternative Ziele:} \\
\texttt{{-} Behalte y=3 für A(1,3), D(2,3), M(3,3), O(4,3), P(5,3), T(7,3)} \\
\texttt{{-} Setze S nach (4,5) bei mir; kannst du S bei dir auf (4,5) legen? Wenn nicht, nenne mir eine freie S{-}Position bei dir außerhalb y=3, die ich auch frei habe: Kandidaten bei mir: (1,5) und (7,5) sind frei. Welche davon passt bei dir?} \\
            }
        }
    }
    & & \\ \\

    \theutterance \stepcounter{utterance}  
    & & & \multicolumn{2}{p{0.3\linewidth}}{
        \cellcolor[rgb]{0.9,0.9,0.9}{
            \makecell[{{p{\linewidth}}}]{
                \texttt{\tiny{[GM$|$GM]}}
                \texttt{SPRICH: Bei mir ist (6,2) blockiert durch '┌'. Vorschlag: Lassen wir S auf y=5 setzen: (6,5) ist bei mir frei? Nein, dort stand vorher O. Aktuell frei in meiner Nähe: (1,5), (4,5), (7,5) scheinen frei. Alternative Ziele:} \\
\texttt{{-} Behalte y=3 für A(1,3), D(2,3), M(3,3), O(4,3), P(5,3), T(7,3)} \\
\texttt{{-} Setze S nach (4,5) bei mir; kannst du S bei dir auf (4,5) legen? Wenn nicht, nenne mir eine freie S{-}Position bei dir außerhalb y=3, die ich auch frei habe: Kandidaten bei mir: (1,5) und (7,5) sind frei. Welche davon passt bei dir?} \\
            }
        }
    }
    & & \\ \\

    \theutterance \stepcounter{utterance}  
    & & & \multicolumn{2}{p{0.3\linewidth}}{
        \cellcolor[rgb]{0.9,0.9,0.9}{
            \makecell[{{p{\linewidth}}}]{
                \texttt{\tiny{[GM$|$GM]}}
                \texttt{Invalid format: tail is not empty: '} \\
\texttt{{-} Behalte y=3 für A(1,3), D(2,3), M(3,3), O(4,3), P(5,3), T(7,3)} \\
\texttt{{-} Setze S nach (4,5) bei mir; kannst du S bei dir auf (4,5) legen? Wenn nicht, nenne mir eine freie S{-}Position bei dir außerhalb y=3, die ich auch frei habe: Kandidaten bei mir: (1,5) und (7,5) sind frei. Welche davon passt bei dir?} \\
            }
        }
    }
    & & \\ \\

    \theutterance \stepcounter{utterance}  
    & & \multicolumn{4}{p{0.6\linewidth}}{
        \cellcolor[rgb]{0.9,0.9,0.9}{
            \makecell[{{p{\linewidth}}}]{
                \texttt{\tiny{[P1$\langle$GM]}}
                \texttt{Strafpunkt: Deine Nachricht darf nichts nach dem Befehl enthalten! Stelle sicher, dass deine Antwort ausschließlich einen der beiden Befehle `SPRICH: <NACHRICHT>` und `VERSCHIEBE: <OBJECT>, (<X>, <Y>)` und sonst nichts enthält!} \\
\\ 
\texttt{Ihr habt gemeinsam 7 von maximal 16 Strafpunkten angesammelt.} \\
\texttt{Bitte versuche es noch mal!} \\
            }
        }
    }
    & \\ \\

    \theutterance \stepcounter{utterance}  
    & \multicolumn{4}{p{0.6\linewidth}}{
        \cellcolor[rgb]{0.9,0.9,0.9}{
            \makecell[{{p{\linewidth}}}]{
                \texttt{\tiny{[P1$\rangle$GM]}}
                \texttt{SPRICH: (6,2) ist bei mir blockiert. Vorschlag: Behalte y=3 für A(1,3), D(2,3), M(3,3), O(4,3), P(5,3), T(7,3). Für S schlage ich (1,5) vor. Ist (1,5) bei dir frei? Wenn nicht, Alternative (7,5).} \\
            }
        }
    }
    & & \\ \\

    \theutterance \stepcounter{utterance}  
    & & & \multicolumn{2}{p{0.3\linewidth}}{
        \cellcolor[rgb]{0.9,0.9,0.9}{
            \makecell[{{p{\linewidth}}}]{
                \texttt{\tiny{[GM$|$GM]}}
                \texttt{SPRICH: (6,2) ist bei mir blockiert. Vorschlag: Behalte y=3 für A(1,3), D(2,3), M(3,3), O(4,3), P(5,3), T(7,3). Für S schlage ich (1,5) vor. Ist (1,5) bei dir frei? Wenn nicht, Alternative (7,5).} \\
            }
        }
    }
    & & \\ \\

    \theutterance \stepcounter{utterance}  
    & & \multicolumn{4}{p{0.6\linewidth}}{
        \cellcolor[rgb]{0.9,0.9,0.9}{
            \makecell[{{p{\linewidth}}}]{
                \texttt{\tiny{[GM$\rangle$P2]}}
                \texttt{Deine Nachricht wurde an deinen Mitspieler übermittelt.} \\
\\ 
\\ 
\texttt{Ihr seid gerade bei Runde 11 von maximal 28.} \\
\texttt{Ihr habt gemeinsam 7 von maximal 16 Strafpunkten angesammelt.} \\
\texttt{Dein Mitspieler hat dir diese Nachricht geschickt:} \\
\\ 
\\ 
\texttt{"(6,2) ist bei mir blockiert. Vorschlag: Behalte y=3 für A(1,3), D(2,3), M(3,3), O(4,3), P(5,3), T(7,3). Für S schlage ich (1,5) vor. Ist (1,5) bei dir frei? Wenn nicht, Alternative (7,5)."} \\
\\ 
\\ 
\texttt{Wie lautet dein nächster Befehl?} \\
            }
        }
    }
    & \\ \\

    \theutterance \stepcounter{utterance}  
    & & & \multicolumn{4}{p{0.6\linewidth}}{
        \cellcolor[rgb]{0.9,0.9,0.9}{
            \makecell[{{p{\linewidth}}}]{
                \texttt{\tiny{[GM$\langle$P2]}}
                \texttt{SPRICH: Bei mir ist (1,5) frei. (7,5) ist belegt von D. Einverstanden mit S→(1,5). Ziele damit: A(1,3), D(2,3), M(3,3), O(4,3), P(5,3), T(7,3), S(1,5). Ich setze jetzt S→(1,5).} \\
            }
        }
    }
     \\ \\

    \theutterance \stepcounter{utterance}  
    & & & \multicolumn{2}{p{0.3\linewidth}}{
        \cellcolor[rgb]{0.9,0.9,0.9}{
            \makecell[{{p{\linewidth}}}]{
                \texttt{\tiny{[GM$|$GM]}}
                \texttt{SPRICH: Bei mir ist (1,5) frei. (7,5) ist belegt von D. Einverstanden mit S→(1,5). Ziele damit: A(1,3), D(2,3), M(3,3), O(4,3), P(5,3), T(7,3), S(1,5). Ich setze jetzt S→(1,5).} \\
            }
        }
    }
    & & \\ \\

    \theutterance \stepcounter{utterance}  
    & & \multicolumn{4}{p{0.6\linewidth}}{
        \cellcolor[rgb]{0.9,0.9,0.9}{
            \makecell[{{p{\linewidth}}}]{
                \texttt{\tiny{[P1$\langle$GM]}}
                \texttt{Deine Nachricht wurde an deinen Mitspieler übermittelt.} \\
\\ 
\\ 
\texttt{Ihr seid gerade bei Runde 11 von maximal 28.} \\
\texttt{Ihr habt gemeinsam 7 von maximal 16 Strafpunkten angesammelt.} \\
\texttt{Dein Mitspieler hat dir diese Nachricht geschickt:} \\
\\ 
\\ 
\texttt{"Bei mir ist (1,5) frei. (7,5) ist belegt von D. Einverstanden mit S→(1,5). Ziele damit: A(1,3), D(2,3), M(3,3), O(4,3), P(5,3), T(7,3), S(1,5). Ich setze jetzt S→(1,5)."} \\
\\ 
\\ 
\texttt{Wie lautet dein nächster Befehl?} \\
            }
        }
    }
    & \\ \\

    \theutterance \stepcounter{utterance}  
    & \multicolumn{4}{p{0.6\linewidth}}{
        \cellcolor[rgb]{0.9,0.9,0.9}{
            \makecell[{{p{\linewidth}}}]{
                \texttt{\tiny{[P1$\rangle$GM]}}
                \texttt{VERSCHIEBE: S, (1,5)} \\
            }
        }
    }
    & & \\ \\

    \theutterance \stepcounter{utterance}  
    & & & \multicolumn{2}{p{0.3\linewidth}}{
        \cellcolor[rgb]{0.9,0.9,0.9}{
            \makecell[{{p{\linewidth}}}]{
                \texttt{\tiny{[GM$|$GM]}}
                \texttt{VERSCHIEBE: S, (1,5)} \\
            }
        }
    }
    & & \\ \\

    \theutterance \stepcounter{utterance}  
    & & & \multicolumn{2}{p{0.3\linewidth}}{
        \cellcolor[rgb]{0.9,0.9,0.9}{
            \makecell[{{p{\linewidth}}}]{
                \texttt{\tiny{[GM$|$GM]}}
                \texttt{'S' erfolgreich nach (1,5) verschoben. Dein Raster sieht jetzt so aus:} \\
\texttt{```} \\
\texttt{ 1234567} \\
\texttt{╔══╤════╗} \\
\texttt{║◌T│D◌◌◌║ 1} \\
\texttt{╟──┘◌◌┌─╢ 2} \\
\texttt{║◌A◌OP│◌║ 3} \\
\texttt{╟─┐◌┌─┼─╢ 4} \\
\texttt{║S├─┤◌│◌║ 5} \\
\texttt{╟─┤◌├─┤◌║ 6} \\
\texttt{║◌│◌│M│◌║ 7} \\
\texttt{╚═╧═╧═╧═╝} \\
\texttt{```} \\
            }
        }
    }
    & & \\ \\

    \theutterance \stepcounter{utterance}  
    & & \multicolumn{4}{p{0.6\linewidth}}{
        \cellcolor[rgb]{0.9,0.9,0.9}{
            \makecell[{{p{\linewidth}}}]{
                \texttt{\tiny{[GM$\rangle$P2]}}
                \texttt{Deine Nachricht wurde an deinen Mitspieler übermittelt.} \\
\\ 
\\ 
\texttt{Ihr seid gerade bei Runde 12 von maximal 28.} \\
\texttt{Ihr habt gemeinsam 7 von maximal 16 Strafpunkten angesammelt.} \\
\texttt{Dein Mitspieler hat ein Objekt auf seinem Raster bewegt.} \\
\\ 
\\ 
\texttt{Wie lautet dein nächster Befehl?} \\
            }
        }
    }
    & \\ \\

    \theutterance \stepcounter{utterance}  
    & & & \multicolumn{4}{p{0.6\linewidth}}{
        \cellcolor[rgb]{0.9,0.9,0.9}{
            \makecell[{{p{\linewidth}}}]{
                \texttt{\tiny{[GM$\langle$P2]}}
                \texttt{VERSCHIEBE: S, (1,5)} \\
            }
        }
    }
     \\ \\

    \theutterance \stepcounter{utterance}  
    & & & \multicolumn{2}{p{0.3\linewidth}}{
        \cellcolor[rgb]{0.9,0.9,0.9}{
            \makecell[{{p{\linewidth}}}]{
                \texttt{\tiny{[GM$|$GM]}}
                \texttt{VERSCHIEBE: S, (1,5)} \\
            }
        }
    }
    & & \\ \\

    \theutterance \stepcounter{utterance}  
    & & & \multicolumn{2}{p{0.3\linewidth}}{
        \cellcolor[rgb]{0.9,0.9,0.9}{
            \makecell[{{p{\linewidth}}}]{
                \texttt{\tiny{[GM$|$GM]}}
                \texttt{'S' erfolgreich nach (1,5) verschoben. Dein Raster sieht jetzt so aus:} \\
\texttt{```} \\
\texttt{ 1234567} \\
\texttt{╔══╤════╗} \\
\texttt{║◌◌│◌◌◌◌║ 1} \\
\texttt{╟──┘◌◌┌─╢ 2} \\
\texttt{║A◌◌OP│◌║ 3} \\
\texttt{╟─┐◌┌─┼─╢ 4} \\
\texttt{║S├─┤T│D║ 5} \\
\texttt{╟─┤◌├─┤◌║ 6} \\
\texttt{║◌│◌│M│◌║ 7} \\
\texttt{╚═╧═╧═╧═╝} \\
\texttt{```} \\
            }
        }
    }
    & & \\ \\

    \theutterance \stepcounter{utterance}  
    & & \multicolumn{4}{p{0.6\linewidth}}{
        \cellcolor[rgb]{0.9,0.9,0.9}{
            \makecell[{{p{\linewidth}}}]{
                \texttt{\tiny{[P1$\langle$GM]}}
                \texttt{'S' erfolgreich nach (1,5) verschoben. Dein Raster sieht jetzt so aus:} \\
\texttt{```} \\
\texttt{ 1234567} \\
\texttt{╔══╤════╗} \\
\texttt{║◌T│D◌◌◌║ 1} \\
\texttt{╟──┘◌◌┌─╢ 2} \\
\texttt{║◌A◌OP│◌║ 3} \\
\texttt{╟─┐◌┌─┼─╢ 4} \\
\texttt{║S├─┤◌│◌║ 5} \\
\texttt{╟─┤◌├─┤◌║ 6} \\
\texttt{║◌│◌│M│◌║ 7} \\
\texttt{╚═╧═╧═╧═╝} \\
\texttt{```} \\
\texttt{Ihr seid gerade bei Runde 12 von maximal 28.} \\
\texttt{Ihr habt gemeinsam 7 von maximal 16 Strafpunkten angesammelt.} \\
\texttt{Dein Mitspieler hat ein Objekt auf seinem Raster bewegt.} \\
\\ 
\\ 
\texttt{Wie lautet dein nächster Befehl?} \\
            }
        }
    }
    & \\ \\

    \theutterance \stepcounter{utterance}  
    & \multicolumn{4}{p{0.6\linewidth}}{
        \cellcolor[rgb]{0.9,0.9,0.9}{
            \makecell[{{p{\linewidth}}}]{
                \texttt{\tiny{[P1$\rangle$GM]}}
                \texttt{VERSCHIEBE: T, (7,3)} \\
            }
        }
    }
    & & \\ \\

    \theutterance \stepcounter{utterance}  
    & & & \multicolumn{2}{p{0.3\linewidth}}{
        \cellcolor[rgb]{0.9,0.9,0.9}{
            \makecell[{{p{\linewidth}}}]{
                \texttt{\tiny{[GM$|$GM]}}
                \texttt{VERSCHIEBE: T, (7,3)} \\
            }
        }
    }
    & & \\ \\

    \theutterance \stepcounter{utterance}  
    & & & \multicolumn{2}{p{0.3\linewidth}}{
        \cellcolor[rgb]{0.9,0.9,0.9}{
            \makecell[{{p{\linewidth}}}]{
                \texttt{\tiny{[GM$|$GM]}}
                \texttt{'T' erfolgreich nach (7,3) verschoben. Dein Raster sieht jetzt so aus:} \\
\texttt{```} \\
\texttt{ 1234567} \\
\texttt{╔══╤════╗} \\
\texttt{║◌◌│D◌◌◌║ 1} \\
\texttt{╟──┘◌◌┌─╢ 2} \\
\texttt{║◌A◌OP│T║ 3} \\
\texttt{╟─┐◌┌─┼─╢ 4} \\
\texttt{║S├─┤◌│◌║ 5} \\
\texttt{╟─┤◌├─┤◌║ 6} \\
\texttt{║◌│◌│M│◌║ 7} \\
\texttt{╚═╧═╧═╧═╝} \\
\texttt{```} \\
            }
        }
    }
    & & \\ \\

    \theutterance \stepcounter{utterance}  
    & & \multicolumn{4}{p{0.6\linewidth}}{
        \cellcolor[rgb]{0.9,0.9,0.9}{
            \makecell[{{p{\linewidth}}}]{
                \texttt{\tiny{[GM$\rangle$P2]}}
                \texttt{'S' erfolgreich nach (1,5) verschoben. Dein Raster sieht jetzt so aus:} \\
\texttt{```} \\
\texttt{ 1234567} \\
\texttt{╔══╤════╗} \\
\texttt{║◌◌│◌◌◌◌║ 1} \\
\texttt{╟──┘◌◌┌─╢ 2} \\
\texttt{║A◌◌OP│◌║ 3} \\
\texttt{╟─┐◌┌─┼─╢ 4} \\
\texttt{║S├─┤T│D║ 5} \\
\texttt{╟─┤◌├─┤◌║ 6} \\
\texttt{║◌│◌│M│◌║ 7} \\
\texttt{╚═╧═╧═╧═╝} \\
\texttt{```} \\
\texttt{Ihr seid gerade bei Runde 13 von maximal 28.} \\
\texttt{Ihr habt gemeinsam 7 von maximal 16 Strafpunkten angesammelt.} \\
\texttt{Dein Mitspieler hat ein Objekt auf seinem Raster bewegt.} \\
\\ 
\\ 
\texttt{Wie lautet dein nächster Befehl?} \\
            }
        }
    }
    & \\ \\

    \theutterance \stepcounter{utterance}  
    & & & \multicolumn{4}{p{0.6\linewidth}}{
        \cellcolor[rgb]{0.9,0.9,0.9}{
            \makecell[{{p{\linewidth}}}]{
                \texttt{\tiny{[GM$\langle$P2]}}
                \texttt{VERSCHIEBE: T, (7,3)} \\
            }
        }
    }
     \\ \\

    \theutterance \stepcounter{utterance}  
    & & & \multicolumn{2}{p{0.3\linewidth}}{
        \cellcolor[rgb]{0.9,0.9,0.9}{
            \makecell[{{p{\linewidth}}}]{
                \texttt{\tiny{[GM$|$GM]}}
                \texttt{VERSCHIEBE: T, (7,3)} \\
            }
        }
    }
    & & \\ \\

    \theutterance \stepcounter{utterance}  
    & & & \multicolumn{2}{p{0.3\linewidth}}{
        \cellcolor[rgb]{0.9,0.9,0.9}{
            \makecell[{{p{\linewidth}}}]{
                \texttt{\tiny{[GM$|$GM]}}
                \texttt{'T' erfolgreich nach (7,3) verschoben. Dein Raster sieht jetzt so aus:} \\
\texttt{```} \\
\texttt{ 1234567} \\
\texttt{╔══╤════╗} \\
\texttt{║◌◌│◌◌◌◌║ 1} \\
\texttt{╟──┘◌◌┌─╢ 2} \\
\texttt{║A◌◌OP│T║ 3} \\
\texttt{╟─┐◌┌─┼─╢ 4} \\
\texttt{║S├─┤◌│D║ 5} \\
\texttt{╟─┤◌├─┤◌║ 6} \\
\texttt{║◌│◌│M│◌║ 7} \\
\texttt{╚═╧═╧═╧═╝} \\
\texttt{```} \\
            }
        }
    }
    & & \\ \\

    \theutterance \stepcounter{utterance}  
    & & \multicolumn{4}{p{0.6\linewidth}}{
        \cellcolor[rgb]{0.9,0.9,0.9}{
            \makecell[{{p{\linewidth}}}]{
                \texttt{\tiny{[P1$\langle$GM]}}
                \texttt{'T' erfolgreich nach (7,3) verschoben. Dein Raster sieht jetzt so aus:} \\
\texttt{```} \\
\texttt{ 1234567} \\
\texttt{╔══╤════╗} \\
\texttt{║◌◌│D◌◌◌║ 1} \\
\texttt{╟──┘◌◌┌─╢ 2} \\
\texttt{║◌A◌OP│T║ 3} \\
\texttt{╟─┐◌┌─┼─╢ 4} \\
\texttt{║S├─┤◌│◌║ 5} \\
\texttt{╟─┤◌├─┤◌║ 6} \\
\texttt{║◌│◌│M│◌║ 7} \\
\texttt{╚═╧═╧═╧═╝} \\
\texttt{```} \\
\texttt{Ihr seid gerade bei Runde 13 von maximal 28.} \\
\texttt{Ihr habt gemeinsam 7 von maximal 16 Strafpunkten angesammelt.} \\
\texttt{Dein Mitspieler hat ein Objekt auf seinem Raster bewegt.} \\
\\ 
\\ 
\texttt{Wie lautet dein nächster Befehl?} \\
            }
        }
    }
    & \\ \\

    \theutterance \stepcounter{utterance}  
    & \multicolumn{4}{p{0.6\linewidth}}{
        \cellcolor[rgb]{0.9,0.9,0.9}{
            \makecell[{{p{\linewidth}}}]{
                \texttt{\tiny{[P1$\rangle$GM]}}
                \texttt{SPRICH: Aktueller Status bei mir: A(2,3), D unbekannt, M(4,7), O(4,3), P(5,3), S(1,5), T(7,3). Ziel bleibt: A(1,3), D(2,3), M(3,3), O(4,3), P(5,3), S(1,5), T(7,3). Ich verschiebe als Nächstes A→(1,3). Bitte sag, ob das bei dir passt und ob du D auf (2,3) setzen kannst.} \\
            }
        }
    }
    & & \\ \\

    \theutterance \stepcounter{utterance}  
    & & & \multicolumn{2}{p{0.3\linewidth}}{
        \cellcolor[rgb]{0.9,0.9,0.9}{
            \makecell[{{p{\linewidth}}}]{
                \texttt{\tiny{[GM$|$GM]}}
                \texttt{SPRICH: Aktueller Status bei mir: A(2,3), D unbekannt, M(4,7), O(4,3), P(5,3), S(1,5), T(7,3). Ziel bleibt: A(1,3), D(2,3), M(3,3), O(4,3), P(5,3), S(1,5), T(7,3). Ich verschiebe als Nächstes A→(1,3). Bitte sag, ob das bei dir passt und ob du D auf (2,3) setzen kannst.} \\
            }
        }
    }
    & & \\ \\

    \theutterance \stepcounter{utterance}  
    & & \multicolumn{4}{p{0.6\linewidth}}{
        \cellcolor[rgb]{0.9,0.9,0.9}{
            \makecell[{{p{\linewidth}}}]{
                \texttt{\tiny{[GM$\rangle$P2]}}
                \texttt{'T' erfolgreich nach (7,3) verschoben. Dein Raster sieht jetzt so aus:} \\
\texttt{```} \\
\texttt{ 1234567} \\
\texttt{╔══╤════╗} \\
\texttt{║◌◌│◌◌◌◌║ 1} \\
\texttt{╟──┘◌◌┌─╢ 2} \\
\texttt{║A◌◌OP│T║ 3} \\
\texttt{╟─┐◌┌─┼─╢ 4} \\
\texttt{║S├─┤◌│D║ 5} \\
\texttt{╟─┤◌├─┤◌║ 6} \\
\texttt{║◌│◌│M│◌║ 7} \\
\texttt{╚═╧═╧═╧═╝} \\
\texttt{```} \\
\texttt{Ihr seid gerade bei Runde 14 von maximal 28.} \\
\texttt{Ihr habt gemeinsam 7 von maximal 16 Strafpunkten angesammelt.} \\
\texttt{Dein Mitspieler hat dir diese Nachricht geschickt:} \\
\\ 
\\ 
\texttt{"Aktueller Status bei mir: A(2,3), D unbekannt, M(4,7), O(4,3), P(5,3), S(1,5), T(7,3). Ziel bleibt: A(1,3), D(2,3), M(3,3), O(4,3), P(5,3), S(1,5), T(7,3). Ich verschiebe als Nächstes A→(1,3). Bitte sag, ob das bei dir passt und ob du D auf (2,3) setzen kannst."} \\
\\ 
\\ 
\texttt{Wie lautet dein nächster Befehl?} \\
            }
        }
    }
    & \\ \\

    \theutterance \stepcounter{utterance}  
    & & & \multicolumn{4}{p{0.6\linewidth}}{
        \cellcolor[rgb]{0.9,0.9,0.9}{
            \makecell[{{p{\linewidth}}}]{
                \texttt{\tiny{[GM$\langle$P2]}}
                \texttt{SPRICH: Passt. Mein aktueller Stand: A(1,3), D(7,5), M(4,7), O(4,3), P(5,3), S(1,5), T(7,3). Ich kann D auf (2,3) setzen, aber (2,3) muss frei sein. Bei mir ist (2,3) frei. Ich ziehe als Nächstes D→(2,3). Bestätige, wenn du A→(1,3) setzt.} \\
            }
        }
    }
     \\ \\

    \theutterance \stepcounter{utterance}  
    & & & \multicolumn{2}{p{0.3\linewidth}}{
        \cellcolor[rgb]{0.9,0.9,0.9}{
            \makecell[{{p{\linewidth}}}]{
                \texttt{\tiny{[GM$|$GM]}}
                \texttt{SPRICH: Passt. Mein aktueller Stand: A(1,3), D(7,5), M(4,7), O(4,3), P(5,3), S(1,5), T(7,3). Ich kann D auf (2,3) setzen, aber (2,3) muss frei sein. Bei mir ist (2,3) frei. Ich ziehe als Nächstes D→(2,3). Bestätige, wenn du A→(1,3) setzt.} \\
            }
        }
    }
    & & \\ \\

    \theutterance \stepcounter{utterance}  
    & & \multicolumn{4}{p{0.6\linewidth}}{
        \cellcolor[rgb]{0.9,0.9,0.9}{
            \makecell[{{p{\linewidth}}}]{
                \texttt{\tiny{[P1$\langle$GM]}}
                \texttt{Deine Nachricht wurde an deinen Mitspieler übermittelt.} \\
\\ 
\\ 
\texttt{Ihr seid gerade bei Runde 14 von maximal 28.} \\
\texttt{Ihr habt gemeinsam 7 von maximal 16 Strafpunkten angesammelt.} \\
\texttt{Dein Mitspieler hat dir diese Nachricht geschickt:} \\
\\ 
\\ 
\texttt{"Passt. Mein aktueller Stand: A(1,3), D(7,5), M(4,7), O(4,3), P(5,3), S(1,5), T(7,3). Ich kann D auf (2,3) setzen, aber (2,3) muss frei sein. Bei mir ist (2,3) frei. Ich ziehe als Nächstes D→(2,3). Bestätige, wenn du A→(1,3) setzt."} \\
\\ 
\\ 
\texttt{Wie lautet dein nächster Befehl?} \\
            }
        }
    }
    & \\ \\

    \theutterance \stepcounter{utterance}  
    & \multicolumn{4}{p{0.6\linewidth}}{
        \cellcolor[rgb]{0.9,0.9,0.9}{
            \makecell[{{p{\linewidth}}}]{
                \texttt{\tiny{[P1$\rangle$GM]}}
                \texttt{VERSCHIEBE: A, (1,3)} \\
            }
        }
    }
    & & \\ \\

    \theutterance \stepcounter{utterance}  
    & & & \multicolumn{2}{p{0.3\linewidth}}{
        \cellcolor[rgb]{0.9,0.9,0.9}{
            \makecell[{{p{\linewidth}}}]{
                \texttt{\tiny{[GM$|$GM]}}
                \texttt{VERSCHIEBE: A, (1,3)} \\
            }
        }
    }
    & & \\ \\

    \theutterance \stepcounter{utterance}  
    & & & \multicolumn{2}{p{0.3\linewidth}}{
        \cellcolor[rgb]{0.9,0.9,0.9}{
            \makecell[{{p{\linewidth}}}]{
                \texttt{\tiny{[GM$|$GM]}}
                \texttt{'A' erfolgreich nach (1,3) verschoben. Dein Raster sieht jetzt so aus:} \\
\texttt{```} \\
\texttt{ 1234567} \\
\texttt{╔══╤════╗} \\
\texttt{║◌◌│D◌◌◌║ 1} \\
\texttt{╟──┘◌◌┌─╢ 2} \\
\texttt{║A◌◌OP│T║ 3} \\
\texttt{╟─┐◌┌─┼─╢ 4} \\
\texttt{║S├─┤◌│◌║ 5} \\
\texttt{╟─┤◌├─┤◌║ 6} \\
\texttt{║◌│◌│M│◌║ 7} \\
\texttt{╚═╧═╧═╧═╝} \\
\texttt{```} \\
            }
        }
    }
    & & \\ \\

    \theutterance \stepcounter{utterance}  
    & & \multicolumn{4}{p{0.6\linewidth}}{
        \cellcolor[rgb]{0.9,0.9,0.9}{
            \makecell[{{p{\linewidth}}}]{
                \texttt{\tiny{[GM$\rangle$P2]}}
                \texttt{Deine Nachricht wurde an deinen Mitspieler übermittelt.} \\
\\ 
\\ 
\texttt{Ihr seid gerade bei Runde 15 von maximal 28.} \\
\texttt{Ihr habt gemeinsam 7 von maximal 16 Strafpunkten angesammelt.} \\
\texttt{Dein Mitspieler hat ein Objekt auf seinem Raster bewegt.} \\
\\ 
\\ 
\texttt{Wie lautet dein nächster Befehl?} \\
            }
        }
    }
    & \\ \\

    \theutterance \stepcounter{utterance}  
    & & & \multicolumn{4}{p{0.6\linewidth}}{
        \cellcolor[rgb]{0.9,0.9,0.9}{
            \makecell[{{p{\linewidth}}}]{
                \texttt{\tiny{[GM$\langle$P2]}}
                \texttt{VERSCHIEBE: D, (2,3)} \\
            }
        }
    }
     \\ \\

    \theutterance \stepcounter{utterance}  
    & & & \multicolumn{2}{p{0.3\linewidth}}{
        \cellcolor[rgb]{0.9,0.9,0.9}{
            \makecell[{{p{\linewidth}}}]{
                \texttt{\tiny{[GM$|$GM]}}
                \texttt{VERSCHIEBE: D, (2,3)} \\
            }
        }
    }
    & & \\ \\

    \theutterance \stepcounter{utterance}  
    & & & \multicolumn{2}{p{0.3\linewidth}}{
        \cellcolor[rgb]{0.9,0.9,0.9}{
            \makecell[{{p{\linewidth}}}]{
                \texttt{\tiny{[GM$|$GM]}}
                \texttt{'D' erfolgreich nach (2,3) verschoben. Dein Raster sieht jetzt so aus:} \\
\texttt{```} \\
\texttt{ 1234567} \\
\texttt{╔══╤════╗} \\
\texttt{║◌◌│◌◌◌◌║ 1} \\
\texttt{╟──┘◌◌┌─╢ 2} \\
\texttt{║AD◌OP│T║ 3} \\
\texttt{╟─┐◌┌─┼─╢ 4} \\
\texttt{║S├─┤◌│◌║ 5} \\
\texttt{╟─┤◌├─┤◌║ 6} \\
\texttt{║◌│◌│M│◌║ 7} \\
\texttt{╚═╧═╧═╧═╝} \\
\texttt{```} \\
            }
        }
    }
    & & \\ \\

    \theutterance \stepcounter{utterance}  
    & & \multicolumn{4}{p{0.6\linewidth}}{
        \cellcolor[rgb]{0.9,0.9,0.9}{
            \makecell[{{p{\linewidth}}}]{
                \texttt{\tiny{[P1$\langle$GM]}}
                \texttt{'A' erfolgreich nach (1,3) verschoben. Dein Raster sieht jetzt so aus:} \\
\texttt{```} \\
\texttt{ 1234567} \\
\texttt{╔══╤════╗} \\
\texttt{║◌◌│D◌◌◌║ 1} \\
\texttt{╟──┘◌◌┌─╢ 2} \\
\texttt{║A◌◌OP│T║ 3} \\
\texttt{╟─┐◌┌─┼─╢ 4} \\
\texttt{║S├─┤◌│◌║ 5} \\
\texttt{╟─┤◌├─┤◌║ 6} \\
\texttt{║◌│◌│M│◌║ 7} \\
\texttt{╚═╧═╧═╧═╝} \\
\texttt{```} \\
\texttt{Ihr seid gerade bei Runde 15 von maximal 28.} \\
\texttt{Ihr habt gemeinsam 7 von maximal 16 Strafpunkten angesammelt.} \\
\texttt{Dein Mitspieler hat ein Objekt auf seinem Raster bewegt.} \\
\\ 
\\ 
\texttt{Wie lautet dein nächster Befehl?} \\
            }
        }
    }
    & \\ \\

    \theutterance \stepcounter{utterance}  
    & \multicolumn{4}{p{0.6\linewidth}}{
        \cellcolor[rgb]{0.9,0.9,0.9}{
            \makecell[{{p{\linewidth}}}]{
                \texttt{\tiny{[P1$\rangle$GM]}}
                \texttt{SPRICH: Bestätigt. A→(1,3) ist gesetzt. Ich setze als Nächstes D→(2,3). Bitte bestätige, wenn du D ebenfalls auf (2,3) gelegt hast, und ob M→(3,3) bei dir möglich ist.} \\
            }
        }
    }
    & & \\ \\

    \theutterance \stepcounter{utterance}  
    & & & \multicolumn{2}{p{0.3\linewidth}}{
        \cellcolor[rgb]{0.9,0.9,0.9}{
            \makecell[{{p{\linewidth}}}]{
                \texttt{\tiny{[GM$|$GM]}}
                \texttt{SPRICH: Bestätigt. A→(1,3) ist gesetzt. Ich setze als Nächstes D→(2,3). Bitte bestätige, wenn du D ebenfalls auf (2,3) gelegt hast, und ob M→(3,3) bei dir möglich ist.} \\
            }
        }
    }
    & & \\ \\

    \theutterance \stepcounter{utterance}  
    & & \multicolumn{4}{p{0.6\linewidth}}{
        \cellcolor[rgb]{0.9,0.9,0.9}{
            \makecell[{{p{\linewidth}}}]{
                \texttt{\tiny{[GM$\rangle$P2]}}
                \texttt{'D' erfolgreich nach (2,3) verschoben. Dein Raster sieht jetzt so aus:} \\
\texttt{```} \\
\texttt{ 1234567} \\
\texttt{╔══╤════╗} \\
\texttt{║◌◌│◌◌◌◌║ 1} \\
\texttt{╟──┘◌◌┌─╢ 2} \\
\texttt{║AD◌OP│T║ 3} \\
\texttt{╟─┐◌┌─┼─╢ 4} \\
\texttt{║S├─┤◌│◌║ 5} \\
\texttt{╟─┤◌├─┤◌║ 6} \\
\texttt{║◌│◌│M│◌║ 7} \\
\texttt{╚═╧═╧═╧═╝} \\
\texttt{```} \\
\texttt{Ihr seid gerade bei Runde 16 von maximal 28.} \\
\texttt{Ihr habt gemeinsam 7 von maximal 16 Strafpunkten angesammelt.} \\
\texttt{Dein Mitspieler hat dir diese Nachricht geschickt:} \\
\\ 
\\ 
\texttt{"Bestätigt. A→(1,3) ist gesetzt. Ich setze als Nächstes D→(2,3). Bitte bestätige, wenn du D ebenfalls auf (2,3) gelegt hast, und ob M→(3,3) bei dir möglich ist."} \\
\\ 
\\ 
\texttt{Wie lautet dein nächster Befehl?} \\
            }
        }
    }
    & \\ \\

    \theutterance \stepcounter{utterance}  
    & & & \multicolumn{4}{p{0.6\linewidth}}{
        \cellcolor[rgb]{0.9,0.9,0.9}{
            \makecell[{{p{\linewidth}}}]{
                \texttt{\tiny{[GM$\langle$P2]}}
                \texttt{SPRICH: Bestätigt, D steht bei mir auf (2,3). M auf (3,3) ist möglich; Zielzelle ist frei. Ich ziehe jetzt M→(3,3). Wenn du soweit bist, sag Bescheid für Abschluss{-}Check danach.} \\
            }
        }
    }
     \\ \\

    \theutterance \stepcounter{utterance}  
    & & & \multicolumn{2}{p{0.3\linewidth}}{
        \cellcolor[rgb]{0.9,0.9,0.9}{
            \makecell[{{p{\linewidth}}}]{
                \texttt{\tiny{[GM$|$GM]}}
                \texttt{SPRICH: Bestätigt, D steht bei mir auf (2,3). M auf (3,3) ist möglich; Zielzelle ist frei. Ich ziehe jetzt M→(3,3). Wenn du soweit bist, sag Bescheid für Abschluss{-}Check danach.} \\
            }
        }
    }
    & & \\ \\

    \theutterance \stepcounter{utterance}  
    & & \multicolumn{4}{p{0.6\linewidth}}{
        \cellcolor[rgb]{0.9,0.9,0.9}{
            \makecell[{{p{\linewidth}}}]{
                \texttt{\tiny{[P1$\langle$GM]}}
                \texttt{Deine Nachricht wurde an deinen Mitspieler übermittelt.} \\
\\ 
\\ 
\texttt{Ihr seid gerade bei Runde 16 von maximal 28.} \\
\texttt{Ihr habt gemeinsam 7 von maximal 16 Strafpunkten angesammelt.} \\
\texttt{Dein Mitspieler hat dir diese Nachricht geschickt:} \\
\\ 
\\ 
\texttt{"Bestätigt, D steht bei mir auf (2,3). M auf (3,3) ist möglich; Zielzelle ist frei. Ich ziehe jetzt M→(3,3). Wenn du soweit bist, sag Bescheid für Abschluss{-}Check danach."} \\
\\ 
\\ 
\texttt{Wie lautet dein nächster Befehl?} \\
            }
        }
    }
    & \\ \\

    \theutterance \stepcounter{utterance}  
    & \multicolumn{4}{p{0.6\linewidth}}{
        \cellcolor[rgb]{0.9,0.9,0.9}{
            \makecell[{{p{\linewidth}}}]{
                \texttt{\tiny{[P1$\rangle$GM]}}
                \texttt{VERSCHIEBE: D, (2,3)} \\
            }
        }
    }
    & & \\ \\

    \theutterance \stepcounter{utterance}  
    & & & \multicolumn{2}{p{0.3\linewidth}}{
        \cellcolor[rgb]{0.9,0.9,0.9}{
            \makecell[{{p{\linewidth}}}]{
                \texttt{\tiny{[GM$|$GM]}}
                \texttt{VERSCHIEBE: D, (2,3)} \\
            }
        }
    }
    & & \\ \\

    \theutterance \stepcounter{utterance}  
    & & & \multicolumn{2}{p{0.3\linewidth}}{
        \cellcolor[rgb]{0.9,0.9,0.9}{
            \makecell[{{p{\linewidth}}}]{
                \texttt{\tiny{[GM$|$GM]}}
                \texttt{'D' erfolgreich nach (2,3) verschoben. Dein Raster sieht jetzt so aus:} \\
\texttt{```} \\
\texttt{ 1234567} \\
\texttt{╔══╤════╗} \\
\texttt{║◌◌│◌◌◌◌║ 1} \\
\texttt{╟──┘◌◌┌─╢ 2} \\
\texttt{║AD◌OP│T║ 3} \\
\texttt{╟─┐◌┌─┼─╢ 4} \\
\texttt{║S├─┤◌│◌║ 5} \\
\texttt{╟─┤◌├─┤◌║ 6} \\
\texttt{║◌│◌│M│◌║ 7} \\
\texttt{╚═╧═╧═╧═╝} \\
\texttt{```} \\
            }
        }
    }
    & & \\ \\

    \theutterance \stepcounter{utterance}  
    & & \multicolumn{4}{p{0.6\linewidth}}{
        \cellcolor[rgb]{0.9,0.9,0.9}{
            \makecell[{{p{\linewidth}}}]{
                \texttt{\tiny{[GM$\rangle$P2]}}
                \texttt{Deine Nachricht wurde an deinen Mitspieler übermittelt.} \\
\\ 
\\ 
\texttt{Ihr seid gerade bei Runde 17 von maximal 28.} \\
\texttt{Ihr habt gemeinsam 7 von maximal 16 Strafpunkten angesammelt.} \\
\texttt{Dein Mitspieler hat ein Objekt auf seinem Raster bewegt.} \\
\\ 
\\ 
\texttt{Wie lautet dein nächster Befehl?} \\
            }
        }
    }
    & \\ \\

    \theutterance \stepcounter{utterance}  
    & & & \multicolumn{4}{p{0.6\linewidth}}{
        \cellcolor[rgb]{0.9,0.9,0.9}{
            \makecell[{{p{\linewidth}}}]{
                \texttt{\tiny{[GM$\langle$P2]}}
                \texttt{VERSCHIEBE: M, (3,3)} \\
            }
        }
    }
     \\ \\

    \theutterance \stepcounter{utterance}  
    & & & \multicolumn{2}{p{0.3\linewidth}}{
        \cellcolor[rgb]{0.9,0.9,0.9}{
            \makecell[{{p{\linewidth}}}]{
                \texttt{\tiny{[GM$|$GM]}}
                \texttt{VERSCHIEBE: M, (3,3)} \\
            }
        }
    }
    & & \\ \\

    \theutterance \stepcounter{utterance}  
    & & & \multicolumn{2}{p{0.3\linewidth}}{
        \cellcolor[rgb]{0.9,0.9,0.9}{
            \makecell[{{p{\linewidth}}}]{
                \texttt{\tiny{[GM$|$GM]}}
                \texttt{'M' erfolgreich nach (3,3) verschoben. Dein Raster sieht jetzt so aus:} \\
\texttt{```} \\
\texttt{ 1234567} \\
\texttt{╔══╤════╗} \\
\texttt{║◌◌│◌◌◌◌║ 1} \\
\texttt{╟──┘◌◌┌─╢ 2} \\
\texttt{║ADMOP│T║ 3} \\
\texttt{╟─┐◌┌─┼─╢ 4} \\
\texttt{║S├─┤◌│◌║ 5} \\
\texttt{╟─┤◌├─┤◌║ 6} \\
\texttt{║◌│◌│◌│◌║ 7} \\
\texttt{╚═╧═╧═╧═╝} \\
\texttt{```} \\
            }
        }
    }
    & & \\ \\

    \theutterance \stepcounter{utterance}  
    & & \multicolumn{4}{p{0.6\linewidth}}{
        \cellcolor[rgb]{0.9,0.9,0.9}{
            \makecell[{{p{\linewidth}}}]{
                \texttt{\tiny{[P1$\langle$GM]}}
                \texttt{'D' erfolgreich nach (2,3) verschoben. Dein Raster sieht jetzt so aus:} \\
\texttt{```} \\
\texttt{ 1234567} \\
\texttt{╔══╤════╗} \\
\texttt{║◌◌│◌◌◌◌║ 1} \\
\texttt{╟──┘◌◌┌─╢ 2} \\
\texttt{║AD◌OP│T║ 3} \\
\texttt{╟─┐◌┌─┼─╢ 4} \\
\texttt{║S├─┤◌│◌║ 5} \\
\texttt{╟─┤◌├─┤◌║ 6} \\
\texttt{║◌│◌│M│◌║ 7} \\
\texttt{╚═╧═╧═╧═╝} \\
\texttt{```} \\
\texttt{Ihr seid gerade bei Runde 17 von maximal 28.} \\
\texttt{Ihr habt gemeinsam 7 von maximal 16 Strafpunkten angesammelt.} \\
\texttt{Dein Mitspieler hat ein Objekt auf seinem Raster bewegt.} \\
\\ 
\\ 
\texttt{Wie lautet dein nächster Befehl?} \\
            }
        }
    }
    & \\ \\

    \theutterance \stepcounter{utterance}  
    & \multicolumn{4}{p{0.6\linewidth}}{
        \cellcolor[rgb]{0.9,0.9,0.9}{
            \makecell[{{p{\linewidth}}}]{
                \texttt{\tiny{[P1$\rangle$GM]}}
                \texttt{SPRICH: Ich bin bereit für den Abschluss{-}Check. Aktueller Stand bei mir: A(1,3), D(2,3), M(4,7), O(4,3), P(5,3), S(1,5), T(7,3). Ich kann M→(3,3) setzen. Ziehst du M jetzt auf (3,3)? Danach prüfen wir und sagen „fertig?“.} \\
            }
        }
    }
    & & \\ \\

    \theutterance \stepcounter{utterance}  
    & & & \multicolumn{2}{p{0.3\linewidth}}{
        \cellcolor[rgb]{0.9,0.9,0.9}{
            \makecell[{{p{\linewidth}}}]{
                \texttt{\tiny{[GM$|$GM]}}
                \texttt{SPRICH: Ich bin bereit für den Abschluss{-}Check. Aktueller Stand bei mir: A(1,3), D(2,3), M(4,7), O(4,3), P(5,3), S(1,5), T(7,3). Ich kann M→(3,3) setzen. Ziehst du M jetzt auf (3,3)? Danach prüfen wir und sagen „fertig?“.} \\
            }
        }
    }
    & & \\ \\

    \theutterance \stepcounter{utterance}  
    & & \multicolumn{4}{p{0.6\linewidth}}{
        \cellcolor[rgb]{0.9,0.9,0.9}{
            \makecell[{{p{\linewidth}}}]{
                \texttt{\tiny{[GM$\rangle$P2]}}
                \texttt{'M' erfolgreich nach (3,3) verschoben. Dein Raster sieht jetzt so aus:} \\
\texttt{```} \\
\texttt{ 1234567} \\
\texttt{╔══╤════╗} \\
\texttt{║◌◌│◌◌◌◌║ 1} \\
\texttt{╟──┘◌◌┌─╢ 2} \\
\texttt{║ADMOP│T║ 3} \\
\texttt{╟─┐◌┌─┼─╢ 4} \\
\texttt{║S├─┤◌│◌║ 5} \\
\texttt{╟─┤◌├─┤◌║ 6} \\
\texttt{║◌│◌│◌│◌║ 7} \\
\texttt{╚═╧═╧═╧═╝} \\
\texttt{```} \\
\texttt{Ihr seid gerade bei Runde 18 von maximal 28.} \\
\texttt{Ihr habt gemeinsam 7 von maximal 16 Strafpunkten angesammelt.} \\
\texttt{Dein Mitspieler hat dir diese Nachricht geschickt:} \\
\\ 
\\ 
\texttt{"Ich bin bereit für den Abschluss{-}Check. Aktueller Stand bei mir: A(1,3), D(2,3), M(4,7), O(4,3), P(5,3), S(1,5), T(7,3). Ich kann M→(3,3) setzen. Ziehst du M jetzt auf (3,3)? Danach prüfen wir und sagen „fertig?“."} \\
\\ 
\\ 
\texttt{Wie lautet dein nächster Befehl?} \\
            }
        }
    }
    & \\ \\

    \theutterance \stepcounter{utterance}  
    & & & \multicolumn{4}{p{0.6\linewidth}}{
        \cellcolor[rgb]{0.9,0.9,0.9}{
            \makecell[{{p{\linewidth}}}]{
                \texttt{\tiny{[GM$\langle$P2]}}
                \texttt{SPRICH: Bestätigt: Bei mir steht jetzt A(1,3), D(2,3), M(3,3), O(4,3), P(5,3), S(1,5), T(7,3). Wenn das bei dir genauso ist, sag: fertig?} \\
            }
        }
    }
     \\ \\

    \theutterance \stepcounter{utterance}  
    & & & \multicolumn{2}{p{0.3\linewidth}}{
        \cellcolor[rgb]{0.9,0.9,0.9}{
            \makecell[{{p{\linewidth}}}]{
                \texttt{\tiny{[GM$|$GM]}}
                \texttt{SPRICH: Bestätigt: Bei mir steht jetzt A(1,3), D(2,3), M(3,3), O(4,3), P(5,3), S(1,5), T(7,3). Wenn das bei dir genauso ist, sag: fertig?} \\
            }
        }
    }
    & & \\ \\

    \theutterance \stepcounter{utterance}  
    & & \multicolumn{4}{p{0.6\linewidth}}{
        \cellcolor[rgb]{0.9,0.9,0.9}{
            \makecell[{{p{\linewidth}}}]{
                \texttt{\tiny{[P1$\langle$GM]}}
                \texttt{Deine Nachricht wurde an deinen Mitspieler übermittelt.} \\
\\ 
\\ 
\texttt{Ihr seid gerade bei Runde 18 von maximal 28.} \\
\texttt{Ihr habt gemeinsam 7 von maximal 16 Strafpunkten angesammelt.} \\
\texttt{Dein Mitspieler hat dir diese Nachricht geschickt:} \\
\\ 
\\ 
\texttt{"Bestätigt: Bei mir steht jetzt A(1,3), D(2,3), M(3,3), O(4,3), P(5,3), S(1,5), T(7,3). Wenn das bei dir genauso ist, sag: fertig?"} \\
\\ 
\\ 
\texttt{Wie lautet dein nächster Befehl?} \\
            }
        }
    }
    & \\ \\

    \theutterance \stepcounter{utterance}  
    & \multicolumn{4}{p{0.6\linewidth}}{
        \cellcolor[rgb]{0.9,0.9,0.9}{
            \makecell[{{p{\linewidth}}}]{
                \texttt{\tiny{[P1$\rangle$GM]}}
                \texttt{VERSCHIEBE: M, (3,3)} \\
            }
        }
    }
    & & \\ \\

    \theutterance \stepcounter{utterance}  
    & & & \multicolumn{2}{p{0.3\linewidth}}{
        \cellcolor[rgb]{0.9,0.9,0.9}{
            \makecell[{{p{\linewidth}}}]{
                \texttt{\tiny{[GM$|$GM]}}
                \texttt{VERSCHIEBE: M, (3,3)} \\
            }
        }
    }
    & & \\ \\

    \theutterance \stepcounter{utterance}  
    & & & \multicolumn{2}{p{0.3\linewidth}}{
        \cellcolor[rgb]{0.9,0.9,0.9}{
            \makecell[{{p{\linewidth}}}]{
                \texttt{\tiny{[GM$|$GM]}}
                \texttt{'M' erfolgreich nach (3,3) verschoben. Dein Raster sieht jetzt so aus:} \\
\texttt{```} \\
\texttt{ 1234567} \\
\texttt{╔══╤════╗} \\
\texttt{║◌◌│◌◌◌◌║ 1} \\
\texttt{╟──┘◌◌┌─╢ 2} \\
\texttt{║ADMOP│T║ 3} \\
\texttt{╟─┐◌┌─┼─╢ 4} \\
\texttt{║S├─┤◌│◌║ 5} \\
\texttt{╟─┤◌├─┤◌║ 6} \\
\texttt{║◌│◌│◌│◌║ 7} \\
\texttt{╚═╧═╧═╧═╝} \\
\texttt{```} \\
            }
        }
    }
    & & \\ \\

    \theutterance \stepcounter{utterance}  
    & & \multicolumn{4}{p{0.6\linewidth}}{
        \cellcolor[rgb]{0.9,0.9,0.9}{
            \makecell[{{p{\linewidth}}}]{
                \texttt{\tiny{[GM$\rangle$P2]}}
                \texttt{Deine Nachricht wurde an deinen Mitspieler übermittelt.} \\
\\ 
\\ 
\texttt{Ihr seid gerade bei Runde 19 von maximal 28.} \\
\texttt{Ihr habt gemeinsam 7 von maximal 16 Strafpunkten angesammelt.} \\
\texttt{Dein Mitspieler hat ein Objekt auf seinem Raster bewegt.} \\
\\ 
\\ 
\texttt{Wie lautet dein nächster Befehl?} \\
            }
        }
    }
    & \\ \\

    \theutterance \stepcounter{utterance}  
    & & & \multicolumn{4}{p{0.6\linewidth}}{
        \cellcolor[rgb]{0.9,0.9,0.9}{
            \makecell[{{p{\linewidth}}}]{
                \texttt{\tiny{[GM$\langle$P2]}}
                \texttt{SPRICH: fertig?} \\
            }
        }
    }
     \\ \\

    \theutterance \stepcounter{utterance}  
    & & & \multicolumn{2}{p{0.3\linewidth}}{
        \cellcolor[rgb]{0.9,0.9,0.9}{
            \makecell[{{p{\linewidth}}}]{
                \texttt{\tiny{[GM$|$GM]}}
                \texttt{SPRICH: fertig?} \\
            }
        }
    }
    & & \\ \\

    \theutterance \stepcounter{utterance}  
    & & \multicolumn{4}{p{0.6\linewidth}}{
        \cellcolor[rgb]{0.9,0.9,0.9}{
            \makecell[{{p{\linewidth}}}]{
                \texttt{\tiny{[P1$\langle$GM]}}
                \texttt{'M' erfolgreich nach (3,3) verschoben. Dein Raster sieht jetzt so aus:} \\
\texttt{```} \\
\texttt{ 1234567} \\
\texttt{╔══╤════╗} \\
\texttt{║◌◌│◌◌◌◌║ 1} \\
\texttt{╟──┘◌◌┌─╢ 2} \\
\texttt{║ADMOP│T║ 3} \\
\texttt{╟─┐◌┌─┼─╢ 4} \\
\texttt{║S├─┤◌│◌║ 5} \\
\texttt{╟─┤◌├─┤◌║ 6} \\
\texttt{║◌│◌│◌│◌║ 7} \\
\texttt{╚═╧═╧═╧═╝} \\
\texttt{```} \\
\texttt{Ihr seid gerade bei Runde 19 von maximal 28.} \\
\texttt{Ihr habt gemeinsam 7 von maximal 16 Strafpunkten angesammelt.} \\
\texttt{Dein Mitspieler hat dir diese Nachricht geschickt:} \\
\\ 
\\ 
\texttt{"fertig?"} \\
\\ 
\\ 
\texttt{Wie lautet dein nächster Befehl?} \\
            }
        }
    }
    & \\ \\

    \theutterance \stepcounter{utterance}  
    & \multicolumn{4}{p{0.6\linewidth}}{
        \cellcolor[rgb]{0.9,0.9,0.9}{
            \makecell[{{p{\linewidth}}}]{
                \texttt{\tiny{[P1$\rangle$GM]}}
                \texttt{SPRICH: fertig!} \\
            }
        }
    }
    & & \\ \\

    \theutterance \stepcounter{utterance}  
    & & & \multicolumn{2}{p{0.3\linewidth}}{
        \cellcolor[rgb]{0.9,0.9,0.9}{
            \makecell[{{p{\linewidth}}}]{
                \texttt{\tiny{[GM$|$GM]}}
                \texttt{SPRICH: fertig!} \\
            }
        }
    }
    & & \\ \\

    \theutterance \stepcounter{utterance}  
    & & & \multicolumn{2}{p{0.3\linewidth}}{
        \cellcolor[rgb]{0.9,0.9,0.9}{
            \makecell[{{p{\linewidth}}}]{
                \texttt{\tiny{[GM$|$GM]}}
                \texttt{true} \\
            }
        }
    }
    & & \\ \\

    \theutterance \stepcounter{utterance}  
    & & & \multicolumn{2}{p{0.3\linewidth}}{
        \cellcolor[rgb]{0.9,0.9,0.9}{
            \makecell[{{p{\linewidth}}}]{
                \texttt{\tiny{[GM$|$GM]}}
                \texttt{* success: True} \\
\texttt{* lose: False} \\
\texttt{* aborted: False} \\
\texttt{{-}{-}{-}{-}{-}{-}{-}} \\
\texttt{* Shifts: 6.00} \\
\texttt{* Max Shifts: 12.00} \\
\texttt{* Min Shifts: 6.00} \\
\texttt{* End Distance Sum: 0.00} \\
\texttt{* Init Distance Sum: 19.71} \\
\texttt{* Expected Distance Sum: 29.33} \\
\texttt{* Penalties: 7.00} \\
\texttt{* Max Penalties: 16.00} \\
\texttt{* Rounds: 19.00} \\
\texttt{* Max Rounds: 28.00} \\
\texttt{* Object Count: 7.00} \\
            }
        }
    }
    & & \\ \\

    \theutterance \stepcounter{utterance}  
    & & & \multicolumn{2}{p{0.3\linewidth}}{
        \cellcolor[rgb]{0.9,0.9,0.9}{
            \makecell[{{p{\linewidth}}}]{
                \texttt{\tiny{[GM$|$GM]}}
                \texttt{Initial Grids:} \\
\texttt{```} \\
\texttt{Player 1:     Player 2:} \\
\texttt{ 1234567       1234567} \\
\texttt{╔══╤════╗     ╔══╤════╗} \\
\texttt{║◌T│D◌◌◌║ 1   ║◌S│◌◌◌◌║ 1} \\
\texttt{╟──┘◌◌┌─╢ 2   ╟──┘◌◌┌─╢ 2} \\
\texttt{║SA◌◌◌│◌║ 3   ║A◌O◌◌│◌║ 3} \\
\texttt{╟─┐◌┌─┼─╢ 4   ╟─┐◌┌─┼─╢ 4} \\
\texttt{║◌├─┤◌│O║ 5   ║◌├─┤T│D║ 5} \\
\texttt{╟─┤◌├─┤◌║ 6   ╟─┤◌├─┤◌║ 6} \\
\texttt{║◌│P│M│◌║ 7   ║P│◌│M│◌║ 7} \\
\texttt{╚═╧═╧═╧═╝     ╚═╧═╧═╧═╝} \\
\texttt{```} \\
\\ 
\texttt{Final Grids:} \\
\texttt{```} \\
\texttt{Player 1:     Player 2:} \\
\texttt{ 1234567       1234567} \\
\texttt{╔══╤════╗     ╔══╤════╗} \\
\texttt{║◌◌│◌◌◌◌║ 1   ║◌◌│◌◌◌◌║ 1} \\
\texttt{╟──┘◌◌┌─╢ 2   ╟──┘◌◌┌─╢ 2} \\
\texttt{║ADMOP│T║ 3   ║ADMOP│T║ 3} \\
\texttt{╟─┐◌┌─┼─╢ 4   ╟─┐◌┌─┼─╢ 4} \\
\texttt{║S├─┤◌│◌║ 5   ║S├─┤◌│◌║ 5} \\
\texttt{╟─┤◌├─┤◌║ 6   ╟─┤◌├─┤◌║ 6} \\
\texttt{║◌│◌│◌│◌║ 7   ║◌│◌│◌│◌║ 7} \\
\texttt{╚═╧═╧═╧═╝     ╚═╧═╧═╧═╝} \\
\texttt{```} \\
            }
        }
    }
    & & \\ \\

\end{supertabular}
}

\end{document}
