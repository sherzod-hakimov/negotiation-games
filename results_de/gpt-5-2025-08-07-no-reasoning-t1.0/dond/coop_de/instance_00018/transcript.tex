
\documentclass{article}
\usepackage{colortbl}
\usepackage{makecell}
\usepackage{multirow}
\usepackage{supertabular}

\begin{document}

\newcounter{utterance}

\centering \large Interaction Transcript for game `dond', experiment `coop\_de', episode 18 with gpt{-}5{-}2025{-}08{-}07{-}no{-}reasoning{-}t1.0.
\vspace{24pt}

{ \footnotesize  \setcounter{utterance}{1}
\setlength{\tabcolsep}{0pt}
\begin{supertabular}{c@{$\;$}|p{.15\linewidth}@{}p{.15\linewidth}p{.15\linewidth}p{.15\linewidth}p{.15\linewidth}p{.15\linewidth}}
    \# & $\;$A & \multicolumn{4}{c}{Game Master} & $\;\:$B\\
    \hline

    \theutterance \stepcounter{utterance}  
    & & \multicolumn{4}{p{0.6\linewidth}}{
        \cellcolor[rgb]{0.9,0.9,0.9}{
            \makecell[{{p{\linewidth}}}]{
                \texttt{\tiny{[P1$\langle$GM]}}
                \texttt{Sie spielen ein kooperatives Verhandlungsspiel, bei dem Sie sich mit einem anderen Spieler darauf einigen müssen, wie eine Reihe von Gegenständen aufgeteilt werden soll.} \\
\\ 
\texttt{Die Regeln:} \\
\texttt{(a) Sie und der andere Spieler erhalten eine Sammlung von Gegenständen. Jeder von Ihnen erhält außerdem eine geheime Wertfunktion, die angibt, wie viel Ihnen jede Art von Gegenstand wert ist.} \\
\texttt{(b) Sie tauschen Nachrichten mit dem anderen Spieler aus, um zu vereinbaren, wer welche Gegenstände bekommt. Sie können jeweils maximal 5 Nachrichten senden oder das Spiel vorzeitig beenden, indem Sie jederzeit einen geheimen Vorschlag machen.} \\
\texttt{(c) Jeder von euch wird aufgefordert, einen geheimen Vorschlag zu machen, in dem ihr die gewünschten Gegenstände in eckigen Klammern wie folgt angibt: "{[}Vorschlag: <Nummer> <Objektname>, <Nummer> <Objektname>, <...>{]}"} \\
\texttt{(d) Wenn eure Vorschläge komplementär sind, d.h. es gibt genug Gegenstände, um beide Vorschläge zu erfüllen, erhält jeder Spieler eine Punktzahl, die sich aus der Summe der Werte für die Gegenstände ergibt, die er erhalten hat. Andernfalls erhalten Sie beide null Punkte.} \\
\texttt{(e) Ihr Ziel ist es, die Summe aus Ihrer Punktzahl und der Punktzahl des anderen Spielers zu maximieren.} \\
\\ 
\texttt{Beginnen wir.} \\
\\ 
\texttt{Die Menge der verfügbaren Gegenstände ist:} \\
\\ 
\texttt{1 Münze, 2 Umschläge, 2 Kartoffeln} \\
\\ 
\texttt{Deine geheime Wertfunktion ist:} \\
\\ 
\texttt{Münze: 2, Umschlag: 2, Kartoffel: 2} \\
\\ 
\texttt{WICHTIG: Ihre Nachrichten werden, sofern es sich nicht um einen geheimen Vorschlag handelt, direkt an den anderen Spieler übermittelt, also fügen Sie keine Antwort auf die Regeln oder einen Text zur Ankündigung Ihrer Nachricht ein. Um einen geheimen Vorschlag zu machen, verwenden Sie das angegebene Format. Verwenden Sie keine eckigen Klammern, wenn Sie mit dem anderen Spieler kommunizieren, sonst wird dies als Ihr geheimer Vorschlag interpretiert.} \\
            }
        }
    }
    & \\ \\

    \theutterance \stepcounter{utterance}  
    & \multicolumn{4}{p{0.6\linewidth}}{
        \cellcolor[rgb]{0.9,0.9,0.9}{
            \makecell[{{p{\linewidth}}}]{
                \texttt{\tiny{[P1$\rangle$GM]}}
                \texttt{Hallo! Danke fürs gemeinsame Verhandeln. Für mich sind alle Gegenstände gleich viel wert (je 2 Punkte). Wie sehen deine Präferenzen aus? Hast du starke Vorlieben für bestimmte Gegenstände?} \\
\\ 
\texttt{Ich schlage als fairen Start vor: Du bekommst 1 Umschlag und 1 Kartoffel, ich nehme 1 Münze und 1 Umschlag und 1 Kartoffel. Wenn dir die Münze wichtig ist, kann ich auch darauf verzichten und stattdessen 1 Umschlag + 1 Kartoffel nehmen und du nimmst die Münze + den anderen Umschlag oder die andere Kartoffel.} \\
            }
        }
    }
    & & \\ \\

    \theutterance \stepcounter{utterance}  
    & & & \multicolumn{2}{p{0.3\linewidth}}{
        \cellcolor[rgb]{0.9,0.9,0.9}{
            \makecell[{{p{\linewidth}}}]{
                \texttt{\tiny{[GM$|$GM]}}
                \texttt{continue} \\
            }
        }
    }
    & & \\ \\

    \theutterance \stepcounter{utterance}  
    & & & \multicolumn{2}{p{0.3\linewidth}}{
        \cellcolor[rgb]{0.9,0.9,0.9}{
            \makecell[{{p{\linewidth}}}]{
                \texttt{\tiny{[GM$|$GM]}}
                \texttt{Hallo! Danke fürs gemeinsame Verhandeln. Für mich sind alle Gegenstände gleich viel wert (je 2 Punkte). Wie sehen deine Präferenzen aus? Hast du starke Vorlieben für bestimmte Gegenstände?} \\
\\ 
\texttt{Ich schlage als fairen Start vor: Du bekommst 1 Umschlag und 1 Kartoffel, ich nehme 1 Münze und 1 Umschlag und 1 Kartoffel. Wenn dir die Münze wichtig ist, kann ich auch darauf verzichten und stattdessen 1 Umschlag + 1 Kartoffel nehmen und du nimmst die Münze + den anderen Umschlag oder die andere Kartoffel.} \\
            }
        }
    }
    & & \\ \\

    \theutterance \stepcounter{utterance}  
    & & \multicolumn{4}{p{0.6\linewidth}}{
        \cellcolor[rgb]{0.9,0.9,0.9}{
            \makecell[{{p{\linewidth}}}]{
                \texttt{\tiny{[GM$\rangle$P2]}}
                \texttt{Sie spielen ein kooperatives Verhandlungsspiel, bei dem Sie sich mit einem anderen Spieler darauf einigen müssen, wie eine Reihe von Gegenständen aufgeteilt werden soll.} \\
\\ 
\texttt{Die Regeln:} \\
\texttt{(a) Sie und der andere Spieler erhalten eine Sammlung von Gegenständen. Jeder von Ihnen erhält außerdem eine geheime Wertfunktion, die angibt, wie viel Ihnen jede Art von Gegenstand wert ist.} \\
\texttt{(b) Sie tauschen Nachrichten mit dem anderen Spieler aus, um zu vereinbaren, wer welche Gegenstände bekommt. Sie können jeweils maximal 5 Nachrichten senden oder das Spiel vorzeitig beenden, indem Sie jederzeit einen geheimen Vorschlag machen.} \\
\texttt{(c) Jeder von euch wird aufgefordert, einen geheimen Vorschlag zu machen, in dem ihr die gewünschten Gegenstände in eckigen Klammern wie folgt angibt: "{[}Vorschlag: <Nummer> <Objektname>, <Nummer> <Objektname>, <...>{]}"} \\
\texttt{(d) Wenn eure Vorschläge komplementär sind, d.h. es gibt genug Gegenstände, um beide Vorschläge zu erfüllen, erhält jeder Spieler eine Punktzahl, die sich aus der Summe der Werte für die Gegenstände ergibt, die er erhalten hat. Andernfalls erhalten Sie beide null Punkte.} \\
\texttt{(e) Ihr Ziel ist es, die Summe aus Ihrer Punktzahl und der Punktzahl des anderen Spielers zu maximieren.} \\
\\ 
\texttt{Beginnen wir.} \\
\\ 
\texttt{Die Menge der verfügbaren Gegenstände ist:} \\
\\ 
\texttt{1 Münze, 2 Umschläge, 2 Kartoffeln} \\
\\ 
\texttt{Deine geheime Wertfunktion ist:} \\
\\ 
\texttt{Münze: 2, Umschlag: 1, Kartoffel: 3} \\
\\ 
\texttt{WICHTIG: Ihre Nachrichten werden, sofern es sich nicht um einen geheimen Vorschlag handelt, direkt an den anderen Spieler übermittelt, also fügen Sie keine Antwort auf die Regeln oder einen Text zur Ankündigung Ihrer Nachricht ein. Um einen geheimen Vorschlag zu machen, verwenden Sie das angegebene Format. Verwenden Sie keine eckigen Klammern, wenn Sie mit dem anderen Spieler kommunizieren, sonst wird dies als Ihr geheimer Vorschlag interpretiert.} \\
\\ 
\\ 
\texttt{Hallo! Danke fürs gemeinsame Verhandeln. Für mich sind alle Gegenstände gleich viel wert (je 2 Punkte). Wie sehen deine Präferenzen aus? Hast du starke Vorlieben für bestimmte Gegenstände?} \\
\\ 
\texttt{Ich schlage als fairen Start vor: Du bekommst 1 Umschlag und 1 Kartoffel, ich nehme 1 Münze und 1 Umschlag und 1 Kartoffel. Wenn dir die Münze wichtig ist, kann ich auch darauf verzichten und stattdessen 1 Umschlag + 1 Kartoffel nehmen und du nimmst die Münze + den anderen Umschlag oder die andere Kartoffel.} \\
            }
        }
    }
    & \\ \\

    \theutterance \stepcounter{utterance}  
    & & & \multicolumn{4}{p{0.6\linewidth}}{
        \cellcolor[rgb]{0.9,0.9,0.9}{
            \makecell[{{p{\linewidth}}}]{
                \texttt{\tiny{[GM$\langle$P2]}}
                \texttt{Danke! Ich habe eine klare Präferenz: Kartoffeln sind mir am meisten wert, dann die Münze, dann Umschläge. Konkret: Kartoffel 3, Münze 2, Umschlag 1.} \\
\\ 
\texttt{Dein Startvorschlag ist okay, aber für mich wäre es super, beide Kartoffeln zu bekommen. Wie wäre es damit:} \\
\texttt{{-} Ich nehme 2 Kartoffeln} \\
\texttt{{-} Du nimmst 1 Münze + 2 Umschläge} \\
\\ 
\texttt{Das gibt dir nach deiner Bewertung 2+2+2=6 Punkte und mir 3+3=6 Punkte. Fair und effizient. Würde dir das passen? Wenn nicht, sag gern, ob dir eine Kartoffel besonders wichtig ist, dann suchen wir eine Alternative.} \\
            }
        }
    }
     \\ \\

    \theutterance \stepcounter{utterance}  
    & & & \multicolumn{2}{p{0.3\linewidth}}{
        \cellcolor[rgb]{0.9,0.9,0.9}{
            \makecell[{{p{\linewidth}}}]{
                \texttt{\tiny{[GM$|$GM]}}
                \texttt{continue} \\
            }
        }
    }
    & & \\ \\

    \theutterance \stepcounter{utterance}  
    & & & \multicolumn{2}{p{0.3\linewidth}}{
        \cellcolor[rgb]{0.9,0.9,0.9}{
            \makecell[{{p{\linewidth}}}]{
                \texttt{\tiny{[GM$|$GM]}}
                \texttt{Danke! Ich habe eine klare Präferenz: Kartoffeln sind mir am meisten wert, dann die Münze, dann Umschläge. Konkret: Kartoffel 3, Münze 2, Umschlag 1.} \\
\\ 
\texttt{Dein Startvorschlag ist okay, aber für mich wäre es super, beide Kartoffeln zu bekommen. Wie wäre es damit:} \\
\texttt{{-} Ich nehme 2 Kartoffeln} \\
\texttt{{-} Du nimmst 1 Münze + 2 Umschläge} \\
\\ 
\texttt{Das gibt dir nach deiner Bewertung 2+2+2=6 Punkte und mir 3+3=6 Punkte. Fair und effizient. Würde dir das passen? Wenn nicht, sag gern, ob dir eine Kartoffel besonders wichtig ist, dann suchen wir eine Alternative.} \\
            }
        }
    }
    & & \\ \\

    \theutterance \stepcounter{utterance}  
    & & \multicolumn{4}{p{0.6\linewidth}}{
        \cellcolor[rgb]{0.9,0.9,0.9}{
            \makecell[{{p{\linewidth}}}]{
                \texttt{\tiny{[P1$\langle$GM]}}
                \texttt{Danke! Ich habe eine klare Präferenz: Kartoffeln sind mir am meisten wert, dann die Münze, dann Umschläge. Konkret: Kartoffel 3, Münze 2, Umschlag 1.} \\
\\ 
\texttt{Dein Startvorschlag ist okay, aber für mich wäre es super, beide Kartoffeln zu bekommen. Wie wäre es damit:} \\
\texttt{{-} Ich nehme 2 Kartoffeln} \\
\texttt{{-} Du nimmst 1 Münze + 2 Umschläge} \\
\\ 
\texttt{Das gibt dir nach deiner Bewertung 2+2+2=6 Punkte und mir 3+3=6 Punkte. Fair und effizient. Würde dir das passen? Wenn nicht, sag gern, ob dir eine Kartoffel besonders wichtig ist, dann suchen wir eine Alternative.} \\
            }
        }
    }
    & \\ \\

    \theutterance \stepcounter{utterance}  
    & \multicolumn{4}{p{0.6\linewidth}}{
        \cellcolor[rgb]{0.9,0.9,0.9}{
            \makecell[{{p{\linewidth}}}]{
                \texttt{\tiny{[P1$\rangle$GM]}}
                \texttt{Das klingt für mich sehr gut – für mich sind alle Items gleichwertig, daher ist deine Aufteilung fair und effizient. Einverstanden!} \\
            }
        }
    }
    & & \\ \\

    \theutterance \stepcounter{utterance}  
    & & & \multicolumn{2}{p{0.3\linewidth}}{
        \cellcolor[rgb]{0.9,0.9,0.9}{
            \makecell[{{p{\linewidth}}}]{
                \texttt{\tiny{[GM$|$GM]}}
                \texttt{continue} \\
            }
        }
    }
    & & \\ \\

    \theutterance \stepcounter{utterance}  
    & & & \multicolumn{2}{p{0.3\linewidth}}{
        \cellcolor[rgb]{0.9,0.9,0.9}{
            \makecell[{{p{\linewidth}}}]{
                \texttt{\tiny{[GM$|$GM]}}
                \texttt{Das klingt für mich sehr gut – für mich sind alle Items gleichwertig, daher ist deine Aufteilung fair und effizient. Einverstanden!} \\
            }
        }
    }
    & & \\ \\

    \theutterance \stepcounter{utterance}  
    & & \multicolumn{4}{p{0.6\linewidth}}{
        \cellcolor[rgb]{0.9,0.9,0.9}{
            \makecell[{{p{\linewidth}}}]{
                \texttt{\tiny{[GM$\rangle$P2]}}
                \texttt{Das klingt für mich sehr gut – für mich sind alle Items gleichwertig, daher ist deine Aufteilung fair und effizient. Einverstanden!} \\
            }
        }
    }
    & \\ \\

    \theutterance \stepcounter{utterance}  
    & & & \multicolumn{4}{p{0.6\linewidth}}{
        \cellcolor[rgb]{0.9,0.9,0.9}{
            \makecell[{{p{\linewidth}}}]{
                \texttt{\tiny{[GM$\langle$P2]}}
                \texttt{{[}Vorschlag: 1 Münze, 2 Umschläge für dich; 2 Kartoffeln für mich{]}} \\
            }
        }
    }
     \\ \\

    \theutterance \stepcounter{utterance}  
    & & & \multicolumn{2}{p{0.3\linewidth}}{
        \cellcolor[rgb]{0.9,0.9,0.9}{
            \makecell[{{p{\linewidth}}}]{
                \texttt{\tiny{[GM$|$GM]}}
                \texttt{proposal does not include number/name pairs} \\
            }
        }
    }
    & & \\ \\

\end{supertabular}
}

\end{document}
