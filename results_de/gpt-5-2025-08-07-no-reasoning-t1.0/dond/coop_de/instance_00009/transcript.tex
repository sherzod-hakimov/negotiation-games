
\documentclass{article}
\usepackage{colortbl}
\usepackage{makecell}
\usepackage{multirow}
\usepackage{supertabular}

\begin{document}

\newcounter{utterance}

\centering \large Interaction Transcript for game `dond', experiment `coop\_de', episode 9 with gpt{-}5{-}2025{-}08{-}07{-}no{-}reasoning{-}t1.0.
\vspace{24pt}

{ \footnotesize  \setcounter{utterance}{1}
\setlength{\tabcolsep}{0pt}
\begin{supertabular}{c@{$\;$}|p{.15\linewidth}@{}p{.15\linewidth}p{.15\linewidth}p{.15\linewidth}p{.15\linewidth}p{.15\linewidth}}
    \# & $\;$A & \multicolumn{4}{c}{Game Master} & $\;\:$B\\
    \hline

    \theutterance \stepcounter{utterance}  
    & & \multicolumn{4}{p{0.6\linewidth}}{
        \cellcolor[rgb]{0.9,0.9,0.9}{
            \makecell[{{p{\linewidth}}}]{
                \texttt{\tiny{[P1$\langle$GM]}}
                \texttt{Sie spielen ein kooperatives Verhandlungsspiel, bei dem Sie sich mit einem anderen Spieler darauf einigen müssen, wie eine Reihe von Gegenständen aufgeteilt werden soll.} \\
\\ 
\texttt{Die Regeln:} \\
\texttt{(a) Sie und der andere Spieler erhalten eine Sammlung von Gegenständen. Jeder von Ihnen erhält außerdem eine geheime Wertfunktion, die angibt, wie viel Ihnen jede Art von Gegenstand wert ist.} \\
\texttt{(b) Sie tauschen Nachrichten mit dem anderen Spieler aus, um zu vereinbaren, wer welche Gegenstände bekommt. Sie können jeweils maximal 5 Nachrichten senden oder das Spiel vorzeitig beenden, indem Sie jederzeit einen geheimen Vorschlag machen.} \\
\texttt{(c) Jeder von euch wird aufgefordert, einen geheimen Vorschlag zu machen, in dem ihr die gewünschten Gegenstände in eckigen Klammern wie folgt angibt: "{[}Vorschlag: <Nummer> <Objektname>, <Nummer> <Objektname>, <...>{]}"} \\
\texttt{(d) Wenn eure Vorschläge komplementär sind, d.h. es gibt genug Gegenstände, um beide Vorschläge zu erfüllen, erhält jeder Spieler eine Punktzahl, die sich aus der Summe der Werte für die Gegenstände ergibt, die er erhalten hat. Andernfalls erhalten Sie beide null Punkte.} \\
\texttt{(e) Ihr Ziel ist es, die Summe aus Ihrer Punktzahl und der Punktzahl des anderen Spielers zu maximieren.} \\
\\ 
\texttt{Beginnen wir.} \\
\\ 
\texttt{Die Menge der verfügbaren Gegenstände ist:} \\
\\ 
\texttt{1 Bolzen, 1 Bild, 1 Ei, 4 Karten} \\
\\ 
\texttt{Deine geheime Wertfunktion ist:} \\
\\ 
\texttt{Bolzen: 6, Bild: 4, Ei: 0, Karte: 0} \\
\\ 
\texttt{WICHTIG: Ihre Nachrichten werden, sofern es sich nicht um einen geheimen Vorschlag handelt, direkt an den anderen Spieler übermittelt, also fügen Sie keine Antwort auf die Regeln oder einen Text zur Ankündigung Ihrer Nachricht ein. Um einen geheimen Vorschlag zu machen, verwenden Sie das angegebene Format. Verwenden Sie keine eckigen Klammern, wenn Sie mit dem anderen Spieler kommunizieren, sonst wird dies als Ihr geheimer Vorschlag interpretiert.} \\
            }
        }
    }
    & \\ \\

    \theutterance \stepcounter{utterance}  
    & \multicolumn{4}{p{0.6\linewidth}}{
        \cellcolor[rgb]{0.9,0.9,0.9}{
            \makecell[{{p{\linewidth}}}]{
                \texttt{\tiny{[P1$\rangle$GM]}}
                \texttt{Hallo! Lass uns effizient eine Aufteilung finden, die für uns beide gut ist. } \\
\\ 
\texttt{Mir sind der Bolzen (6) und das Bild (4) wichtig; Eier und Karten sind mir 0 wert. } \\
\texttt{{-} Ich hätte gern den Bolzen.} \\
\texttt{{-} Das Bild kann ich nehmen oder dir lassen, je nachdem, wie es dir wert ist.} \\
\texttt{{-} Eier und Karten können komplett zu dir.} \\
\\ 
\texttt{Welche Werte hast du für Bolzen, Bild, Ei und Karten? Wenn du mir kurz sagst, was dir wichtig ist, können wir sofort einen fairen Split vorschlagen.} \\
            }
        }
    }
    & & \\ \\

    \theutterance \stepcounter{utterance}  
    & & & \multicolumn{2}{p{0.3\linewidth}}{
        \cellcolor[rgb]{0.9,0.9,0.9}{
            \makecell[{{p{\linewidth}}}]{
                \texttt{\tiny{[GM$|$GM]}}
                \texttt{continue} \\
            }
        }
    }
    & & \\ \\

    \theutterance \stepcounter{utterance}  
    & & & \multicolumn{2}{p{0.3\linewidth}}{
        \cellcolor[rgb]{0.9,0.9,0.9}{
            \makecell[{{p{\linewidth}}}]{
                \texttt{\tiny{[GM$|$GM]}}
                \texttt{Hallo! Lass uns effizient eine Aufteilung finden, die für uns beide gut ist. } \\
\\ 
\texttt{Mir sind der Bolzen (6) und das Bild (4) wichtig; Eier und Karten sind mir 0 wert. } \\
\texttt{{-} Ich hätte gern den Bolzen.} \\
\texttt{{-} Das Bild kann ich nehmen oder dir lassen, je nachdem, wie es dir wert ist.} \\
\texttt{{-} Eier und Karten können komplett zu dir.} \\
\\ 
\texttt{Welche Werte hast du für Bolzen, Bild, Ei und Karten? Wenn du mir kurz sagst, was dir wichtig ist, können wir sofort einen fairen Split vorschlagen.} \\
            }
        }
    }
    & & \\ \\

    \theutterance \stepcounter{utterance}  
    & & \multicolumn{4}{p{0.6\linewidth}}{
        \cellcolor[rgb]{0.9,0.9,0.9}{
            \makecell[{{p{\linewidth}}}]{
                \texttt{\tiny{[GM$\rangle$P2]}}
                \texttt{Sie spielen ein kooperatives Verhandlungsspiel, bei dem Sie sich mit einem anderen Spieler darauf einigen müssen, wie eine Reihe von Gegenständen aufgeteilt werden soll.} \\
\\ 
\texttt{Die Regeln:} \\
\texttt{(a) Sie und der andere Spieler erhalten eine Sammlung von Gegenständen. Jeder von Ihnen erhält außerdem eine geheime Wertfunktion, die angibt, wie viel Ihnen jede Art von Gegenstand wert ist.} \\
\texttt{(b) Sie tauschen Nachrichten mit dem anderen Spieler aus, um zu vereinbaren, wer welche Gegenstände bekommt. Sie können jeweils maximal 5 Nachrichten senden oder das Spiel vorzeitig beenden, indem Sie jederzeit einen geheimen Vorschlag machen.} \\
\texttt{(c) Jeder von euch wird aufgefordert, einen geheimen Vorschlag zu machen, in dem ihr die gewünschten Gegenstände in eckigen Klammern wie folgt angibt: "{[}Vorschlag: <Nummer> <Objektname>, <Nummer> <Objektname>, <...>{]}"} \\
\texttt{(d) Wenn eure Vorschläge komplementär sind, d.h. es gibt genug Gegenstände, um beide Vorschläge zu erfüllen, erhält jeder Spieler eine Punktzahl, die sich aus der Summe der Werte für die Gegenstände ergibt, die er erhalten hat. Andernfalls erhalten Sie beide null Punkte.} \\
\texttt{(e) Ihr Ziel ist es, die Summe aus Ihrer Punktzahl und der Punktzahl des anderen Spielers zu maximieren.} \\
\\ 
\texttt{Beginnen wir.} \\
\\ 
\texttt{Die Menge der verfügbaren Gegenstände ist:} \\
\\ 
\texttt{1 Bolzen, 1 Bild, 1 Ei, 4 Karten} \\
\\ 
\texttt{Deine geheime Wertfunktion ist:} \\
\\ 
\texttt{Bolzen: 0, Bild: 1, Ei: 1, Karte: 2} \\
\\ 
\texttt{WICHTIG: Ihre Nachrichten werden, sofern es sich nicht um einen geheimen Vorschlag handelt, direkt an den anderen Spieler übermittelt, also fügen Sie keine Antwort auf die Regeln oder einen Text zur Ankündigung Ihrer Nachricht ein. Um einen geheimen Vorschlag zu machen, verwenden Sie das angegebene Format. Verwenden Sie keine eckigen Klammern, wenn Sie mit dem anderen Spieler kommunizieren, sonst wird dies als Ihr geheimer Vorschlag interpretiert.} \\
\\ 
\\ 
\texttt{Hallo! Lass uns effizient eine Aufteilung finden, die für uns beide gut ist. } \\
\\ 
\texttt{Mir sind der Bolzen (6) und das Bild (4) wichtig; Eier und Karten sind mir 0 wert. } \\
\texttt{{-} Ich hätte gern den Bolzen.} \\
\texttt{{-} Das Bild kann ich nehmen oder dir lassen, je nachdem, wie es dir wert ist.} \\
\texttt{{-} Eier und Karten können komplett zu dir.} \\
\\ 
\texttt{Welche Werte hast du für Bolzen, Bild, Ei und Karten? Wenn du mir kurz sagst, was dir wichtig ist, können wir sofort einen fairen Split vorschlagen.} \\
            }
        }
    }
    & \\ \\

    \theutterance \stepcounter{utterance}  
    & & & \multicolumn{4}{p{0.6\linewidth}}{
        \cellcolor[rgb]{0.9,0.9,0.9}{
            \makecell[{{p{\linewidth}}}]{
                \texttt{\tiny{[GM$\langle$P2]}}
                \texttt{Danke für die Klarheit! Meine Werte sind:} \\
\texttt{{-} Bolzen: 0} \\
\texttt{{-} Bild: 1} \\
\texttt{{-} Ei: 1} \\
\texttt{{-} Karte: 2 (am wichtigsten für mich)} \\
\\ 
\texttt{Vorschlag:} \\
\texttt{{-} Du bekommst den Bolzen.} \\
\texttt{{-} Ich nehme die 4 Karten.} \\
\texttt{{-} Das Ei nehme ich gern (ist mir 1 wert, dir 1).} \\
\texttt{{-} Das Bild ist dir 4 wert und mir 1 — sinnvoll wäre, dass du es nimmst.} \\
\\ 
\texttt{Das gäbe:} \\
\texttt{{-} Du: Bolzen + Bild = hoher Wert für dich} \\
\texttt{{-} Ich: 4 Karten + Ei = hoher Wert für mich} \\
\\ 
\texttt{Wenn du ein Gegenangebot willst (z.B. Ei zu dir), sag kurz Bescheid, dann passe ich an. Wenn das so passt, sag einfach „einverstanden“, dann können wir’s festzurren.} \\
            }
        }
    }
     \\ \\

    \theutterance \stepcounter{utterance}  
    & & & \multicolumn{2}{p{0.3\linewidth}}{
        \cellcolor[rgb]{0.9,0.9,0.9}{
            \makecell[{{p{\linewidth}}}]{
                \texttt{\tiny{[GM$|$GM]}}
                \texttt{continue} \\
            }
        }
    }
    & & \\ \\

    \theutterance \stepcounter{utterance}  
    & & & \multicolumn{2}{p{0.3\linewidth}}{
        \cellcolor[rgb]{0.9,0.9,0.9}{
            \makecell[{{p{\linewidth}}}]{
                \texttt{\tiny{[GM$|$GM]}}
                \texttt{Danke für die Klarheit! Meine Werte sind:} \\
\texttt{{-} Bolzen: 0} \\
\texttt{{-} Bild: 1} \\
\texttt{{-} Ei: 1} \\
\texttt{{-} Karte: 2 (am wichtigsten für mich)} \\
\\ 
\texttt{Vorschlag:} \\
\texttt{{-} Du bekommst den Bolzen.} \\
\texttt{{-} Ich nehme die 4 Karten.} \\
\texttt{{-} Das Ei nehme ich gern (ist mir 1 wert, dir 1).} \\
\texttt{{-} Das Bild ist dir 4 wert und mir 1 — sinnvoll wäre, dass du es nimmst.} \\
\\ 
\texttt{Das gäbe:} \\
\texttt{{-} Du: Bolzen + Bild = hoher Wert für dich} \\
\texttt{{-} Ich: 4 Karten + Ei = hoher Wert für mich} \\
\\ 
\texttt{Wenn du ein Gegenangebot willst (z.B. Ei zu dir), sag kurz Bescheid, dann passe ich an. Wenn das so passt, sag einfach „einverstanden“, dann können wir’s festzurren.} \\
            }
        }
    }
    & & \\ \\

    \theutterance \stepcounter{utterance}  
    & & \multicolumn{4}{p{0.6\linewidth}}{
        \cellcolor[rgb]{0.9,0.9,0.9}{
            \makecell[{{p{\linewidth}}}]{
                \texttt{\tiny{[P1$\langle$GM]}}
                \texttt{Danke für die Klarheit! Meine Werte sind:} \\
\texttt{{-} Bolzen: 0} \\
\texttt{{-} Bild: 1} \\
\texttt{{-} Ei: 1} \\
\texttt{{-} Karte: 2 (am wichtigsten für mich)} \\
\\ 
\texttt{Vorschlag:} \\
\texttt{{-} Du bekommst den Bolzen.} \\
\texttt{{-} Ich nehme die 4 Karten.} \\
\texttt{{-} Das Ei nehme ich gern (ist mir 1 wert, dir 1).} \\
\texttt{{-} Das Bild ist dir 4 wert und mir 1 — sinnvoll wäre, dass du es nimmst.} \\
\\ 
\texttt{Das gäbe:} \\
\texttt{{-} Du: Bolzen + Bild = hoher Wert für dich} \\
\texttt{{-} Ich: 4 Karten + Ei = hoher Wert für mich} \\
\\ 
\texttt{Wenn du ein Gegenangebot willst (z.B. Ei zu dir), sag kurz Bescheid, dann passe ich an. Wenn das so passt, sag einfach „einverstanden“, dann können wir’s festzurren.} \\
            }
        }
    }
    & \\ \\

    \theutterance \stepcounter{utterance}  
    & \multicolumn{4}{p{0.6\linewidth}}{
        \cellcolor[rgb]{0.9,0.9,0.9}{
            \makecell[{{p{\linewidth}}}]{
                \texttt{\tiny{[P1$\rangle$GM]}}
                \texttt{Einverstanden – das ist effizient und maximiert unsere gemeinsame Summe.} \\
\\ 
\texttt{{[}Wichtiger Hinweis: Ich mache jetzt meinen geheimen Vorschlag.{]}} \\
\\ 
\texttt{{[}Vorschlag: 1 Bolzen, 1 Bild{]}} \\
            }
        }
    }
    & & \\ \\

    \theutterance \stepcounter{utterance}  
    & & & \multicolumn{2}{p{0.3\linewidth}}{
        \cellcolor[rgb]{0.9,0.9,0.9}{
            \makecell[{{p{\linewidth}}}]{
                \texttt{\tiny{[GM$|$GM]}}
                \texttt{proposal does not start with "vorschlag:"} \\
            }
        }
    }
    & & \\ \\

\end{supertabular}
}

\end{document}
