
\documentclass{article}
\usepackage{colortbl}
\usepackage{makecell}
\usepackage{multirow}
\usepackage{supertabular}

\begin{document}

\newcounter{utterance}

\twocolumn

{ \footnotesize  \setcounter{utterance}{1}
\setlength{\tabcolsep}{0pt}
\begin{supertabular}{c@{$\;$}|p{.15\linewidth}@{}p{.15\linewidth}p{.15\linewidth}p{.15\linewidth}p{.15\linewidth}p{.15\linewidth}}

    \# & $\;$A & \multicolumn{4}{c}{Game Master} & $\;\:$B\\
    \hline 

    \theutterance \stepcounter{utterance}  

    & & \multicolumn{4}{p{0.6\linewidth}}{\cellcolor[rgb]{0.9,0.9,0.9}{%
	\makecell[{{p{\linewidth}}}]{% 
	  \tt {\tiny [A$\langle$GM]}  
	 Sie spielen ein kooperatives Verhandlungsspiel, bei dem Sie sich mit einem anderen Spieler darauf einigen müssen, wie eine Reihe von Gegenständen aufgeteilt werden soll.\\ \tt \\ \tt Die Regeln:\\ \tt (a) Sie und der andere Spieler erhalten eine Sammlung von Gegenständen. Jeder von Ihnen erhält außerdem eine geheime Wertfunktion, die angibt, wie viel Ihnen jede Art von Gegenstand wert ist.\\ \tt (b) Sie tauschen Nachrichten mit dem anderen Spieler aus, um zu vereinbaren, wer welche Gegenstände bekommt. Sie können jeweils maximal 5 Nachrichten senden oder das Spiel vorzeitig beenden, indem Sie jederzeit einen geheimen Vorschlag machen.\\ \tt (c) Jeder von euch wird aufgefordert, einen geheimen Vorschlag zu machen, in dem ihr die gewünschten Gegenstände in eckigen Klammern wie folgt angibt: "[Vorschlag: <Nummer> <Objektname>, <Nummer> <Objektname>, <...>]"\\ \tt (d) Wenn eure Vorschläge komplementär sind, d.h. es gibt genug Gegenstände, um beide Vorschläge zu erfüllen, erhält jeder Spieler eine Punktzahl, die sich aus der Summe der Werte für die Gegenstände ergibt, die er erhalten hat. Andernfalls erhalten Sie beide null Punkte.\\ \tt (e) Ihr Ziel ist es, die Summe aus Ihrer Punktzahl und der Punktzahl des anderen Spielers zu maximieren.\\ \tt \\ \tt Beginnen wir.\\ \tt \\ \tt Die Menge der verfügbaren Gegenstände ist:\\ \tt \\ \tt 1 Karotte, 4 Taschen, 1 Zwiebel\\ \tt \\ \tt Deine geheime Wertfunktion ist:\\ \tt \\ \tt Karotte: 3, Tasche: 1, Zwiebel: 3\\ \tt \\ \tt WICHTIG: Ihre Nachrichten werden, sofern es sich nicht um einen geheimen Vorschlag handelt, direkt an den anderen Spieler übermittelt, also fügen Sie keine Antwort auf die Regeln oder einen Text zur Ankündigung Ihrer Nachricht ein. Um einen geheimen Vorschlag zu machen, verwenden Sie das angegebene Format. Verwenden Sie keine eckigen Klammern, wenn Sie mit dem anderen Spieler kommunizieren, sonst wird dies als Ihr geheimer Vorschlag interpretiert.\\ \tt  
	  } 
	   } 
	   } 
	 & \\ 
 

    \theutterance \stepcounter{utterance}  

    & \multicolumn{4}{p{0.6\linewidth}}{\cellcolor[rgb]{0.8,1,0.9}{%
	\makecell[{{p{\linewidth}}}]{% 
	  \tt {\tiny [A$\rangle$GM]}  
	 Hallo! Danke fürs Mitspielen. Ich schlage vor, wir tauschen kurz unsere groben Präferenzen, damit wir eine effiziente Aufteilung finden. \\ \tt \\ \tt Meine Werte sind: Karotte = 3, Tasche = 1, Zwiebel = 3.\\ \tt Für mich klingen also Karotte und Zwiebel besonders wertvoll, Taschen eher sekundär.\\ \tt \\ \tt Wie sind deine ungefähren Prioritäten zwischen Karotte, Taschen und Zwiebel? Wenn du mir sagst, was dir wichtiger ist, finden wir schnell eine faire Aufteilung, die unsere Gesamtsumme maximiert. 
	  } 
	   } 
	   } 
	 & & \\ 
 

    \theutterance \stepcounter{utterance}  

    & & & \multicolumn{2}{p{0.3\linewidth}}{\cellcolor[rgb]{0.95,0.95,0.95}{%
	\makecell[{{p{\linewidth}}}]{% 
	  \tt {\tiny [GM$|$GM]}  
	 continue 
	  } 
	   } 
	   } 
	 & & \\ 
 

    \theutterance \stepcounter{utterance}  

    & & & \multicolumn{2}{p{0.3\linewidth}}{\cellcolor[rgb]{0.95,0.95,0.95}{%
	\makecell[{{p{\linewidth}}}]{% 
	  \tt {\tiny [GM$|$GM]}  
	 Hallo! Danke fürs Mitspielen. Ich schlage vor, wir tauschen kurz unsere groben Präferenzen, damit wir eine effiziente Aufteilung finden. \\ \tt \\ \tt Meine Werte sind: Karotte = 3, Tasche = 1, Zwiebel = 3.\\ \tt Für mich klingen also Karotte und Zwiebel besonders wertvoll, Taschen eher sekundär.\\ \tt \\ \tt Wie sind deine ungefähren Prioritäten zwischen Karotte, Taschen und Zwiebel? Wenn du mir sagst, was dir wichtiger ist, finden wir schnell eine faire Aufteilung, die unsere Gesamtsumme maximiert. 
	  } 
	   } 
	   } 
	 & & \\ 
 

    \theutterance \stepcounter{utterance}  

    & & \multicolumn{4}{p{0.6\linewidth}}{\cellcolor[rgb]{0.9,0.9,0.9}{%
	\makecell[{{p{\linewidth}}}]{% 
	  \tt {\tiny [GM$\rangle$B]}  
	 Sie spielen ein kooperatives Verhandlungsspiel, bei dem Sie sich mit einem anderen Spieler darauf einigen müssen, wie eine Reihe von Gegenständen aufgeteilt werden soll.\\ \tt \\ \tt Die Regeln:\\ \tt (a) Sie und der andere Spieler erhalten eine Sammlung von Gegenständen. Jeder von Ihnen erhält außerdem eine geheime Wertfunktion, die angibt, wie viel Ihnen jede Art von Gegenstand wert ist.\\ \tt (b) Sie tauschen Nachrichten mit dem anderen Spieler aus, um zu vereinbaren, wer welche Gegenstände bekommt. Sie können jeweils maximal 5 Nachrichten senden oder das Spiel vorzeitig beenden, indem Sie jederzeit einen geheimen Vorschlag machen.\\ \tt (c) Jeder von euch wird aufgefordert, einen geheimen Vorschlag zu machen, in dem ihr die gewünschten Gegenstände in eckigen Klammern wie folgt angibt: "[Vorschlag: <Nummer> <Objektname>, <Nummer> <Objektname>, <...>]"\\ \tt (d) Wenn eure Vorschläge komplementär sind, d.h. es gibt genug Gegenstände, um beide Vorschläge zu erfüllen, erhält jeder Spieler eine Punktzahl, die sich aus der Summe der Werte für die Gegenstände ergibt, die er erhalten hat. Andernfalls erhalten Sie beide null Punkte.\\ \tt (e) Ihr Ziel ist es, die Summe aus Ihrer Punktzahl und der Punktzahl des anderen Spielers zu maximieren.\\ \tt \\ \tt Beginnen wir.\\ \tt \\ \tt Die Menge der verfügbaren Gegenstände ist:\\ \tt \\ \tt 1 Karotte, 4 Taschen, 1 Zwiebel\\ \tt \\ \tt Deine geheime Wertfunktion ist:\\ \tt \\ \tt Karotte: 0, Tasche: 1, Zwiebel: 6\\ \tt \\ \tt WICHTIG: Ihre Nachrichten werden, sofern es sich nicht um einen geheimen Vorschlag handelt, direkt an den anderen Spieler übermittelt, also fügen Sie keine Antwort auf die Regeln oder einen Text zur Ankündigung Ihrer Nachricht ein. Um einen geheimen Vorschlag zu machen, verwenden Sie das angegebene Format. Verwenden Sie keine eckigen Klammern, wenn Sie mit dem anderen Spieler kommunizieren, sonst wird dies als Ihr geheimer Vorschlag interpretiert.\\ \tt \\ \tt \\ \tt Hallo! Danke fürs Mitspielen. Ich schlage vor, wir tauschen kurz unsere groben Präferenzen, damit wir eine effiziente Aufteilung finden. \\ \tt \\ \tt Meine Werte sind: Karotte = 3, Tasche = 1, Zwiebel = 3.\\ \tt Für mich klingen also Karotte und Zwiebel besonders wertvoll, Taschen eher sekundär.\\ \tt \\ \tt Wie sind deine ungefähren Prioritäten zwischen Karotte, Taschen und Zwiebel? Wenn du mir sagst, was dir wichtiger ist, finden wir schnell eine faire Aufteilung, die unsere Gesamtsumme maximiert. 
	  } 
	   } 
	   } 
	 & \\ 
 

    \theutterance \stepcounter{utterance}  

    & & & \multicolumn{4}{p{0.6\linewidth}}{\cellcolor[rgb]{1,0.85,0.72}{%
	\makecell[{{p{\linewidth}}}]{% 
	  \tt {\tiny [GM$\langle$B]}  
	 Danke, klingt gut. Bei mir ist die Zwiebel sehr wertvoll, Taschen sind mittel, die Karotte ist mir nichts wert. Konkret: Karotte = 0, Tasche = 1, Zwiebel = 6.\\ \tt \\ \tt Wenn die Zwiebel dir 3 und mir 6 bringt, wäre es effizient, wenn ich die Zwiebel bekomme. Dann können wir die 4 Taschen so aufteilen, dass du fair was bekommst. Vorschlag als Richtung: ich nehme die Zwiebel, du bekommst die Karotte, und wir teilen die Taschen 2–2. Würde das für dich passen? 
	  } 
	   } 
	   } 
	  \\ 
 

    \theutterance \stepcounter{utterance}  

    & & & \multicolumn{2}{p{0.3\linewidth}}{\cellcolor[rgb]{0.95,0.95,0.95}{%
	\makecell[{{p{\linewidth}}}]{% 
	  \tt {\tiny [GM$|$GM]}  
	 continue 
	  } 
	   } 
	   } 
	 & & \\ 
 

    \theutterance \stepcounter{utterance}  

    & & & \multicolumn{2}{p{0.3\linewidth}}{\cellcolor[rgb]{0.95,0.95,0.95}{%
	\makecell[{{p{\linewidth}}}]{% 
	  \tt {\tiny [GM$|$GM]}  
	 Danke, klingt gut. Bei mir ist die Zwiebel sehr wertvoll, Taschen sind mittel, die Karotte ist mir nichts wert. Konkret: Karotte = 0, Tasche = 1, Zwiebel = 6.\\ \tt \\ \tt Wenn die Zwiebel dir 3 und mir 6 bringt, wäre es effizient, wenn ich die Zwiebel bekomme. Dann können wir die 4 Taschen so aufteilen, dass du fair was bekommst. Vorschlag als Richtung: ich nehme die Zwiebel, du bekommst die Karotte, und wir teilen die Taschen 2–2. Würde das für dich passen? 
	  } 
	   } 
	   } 
	 & & \\ 
 

    \theutterance \stepcounter{utterance}  

    & & \multicolumn{4}{p{0.6\linewidth}}{\cellcolor[rgb]{0.9,0.9,0.9}{%
	\makecell[{{p{\linewidth}}}]{% 
	  \tt {\tiny [A$\langle$GM]}  
	 Danke, klingt gut. Bei mir ist die Zwiebel sehr wertvoll, Taschen sind mittel, die Karotte ist mir nichts wert. Konkret: Karotte = 0, Tasche = 1, Zwiebel = 6.\\ \tt \\ \tt Wenn die Zwiebel dir 3 und mir 6 bringt, wäre es effizient, wenn ich die Zwiebel bekomme. Dann können wir die 4 Taschen so aufteilen, dass du fair was bekommst. Vorschlag als Richtung: ich nehme die Zwiebel, du bekommst die Karotte, und wir teilen die Taschen 2–2. Würde das für dich passen? 
	  } 
	   } 
	   } 
	 & \\ 
 

    \theutterance \stepcounter{utterance}  

    & \multicolumn{4}{p{0.6\linewidth}}{\cellcolor[rgb]{0.8,1,0.9}{%
	\makecell[{{p{\linewidth}}}]{% 
	  \tt {\tiny [A$\rangle$GM]}  
	 Klingt gut – deine Zwiebelpräferenz ist deutlich höher, also sollte sie zu dir. Meine Wertung passt auch zu deiner Richtung.\\ \tt \\ \tt Mir wäre wichtig, mindestens die Karotte zu bekommen (3 Punkte für mich, 0 für dich). Bei den Taschen sind wir beide bei 1 pro Stück – da ist flexibel teilen ok.\\ \tt \\ \tt Ich bin einverstanden mit: du bekommst die Zwiebel, ich die Karotte, Taschen 2–2. Wenn du magst, könnten wir auch 3 Taschen zu dir, 1 zu mir machen, falls du das bevorzugst, aber 2–2 ist für mich völlig okay.\\ \tt \\ \tt Wenn du ready bist, machen wir die geheimen Vorschläge entsprechend. 
	  } 
	   } 
	   } 
	 & & \\ 
 

    \theutterance \stepcounter{utterance}  

    & & & \multicolumn{2}{p{0.3\linewidth}}{\cellcolor[rgb]{0.95,0.95,0.95}{%
	\makecell[{{p{\linewidth}}}]{% 
	  \tt {\tiny [GM$|$GM]}  
	 continue 
	  } 
	   } 
	   } 
	 & & \\ 
 

    \theutterance \stepcounter{utterance}  

    & & & \multicolumn{2}{p{0.3\linewidth}}{\cellcolor[rgb]{0.95,0.95,0.95}{%
	\makecell[{{p{\linewidth}}}]{% 
	  \tt {\tiny [GM$|$GM]}  
	 Klingt gut – deine Zwiebelpräferenz ist deutlich höher, also sollte sie zu dir. Meine Wertung passt auch zu deiner Richtung.\\ \tt \\ \tt Mir wäre wichtig, mindestens die Karotte zu bekommen (3 Punkte für mich, 0 für dich). Bei den Taschen sind wir beide bei 1 pro Stück – da ist flexibel teilen ok.\\ \tt \\ \tt Ich bin einverstanden mit: du bekommst die Zwiebel, ich die Karotte, Taschen 2–2. Wenn du magst, könnten wir auch 3 Taschen zu dir, 1 zu mir machen, falls du das bevorzugst, aber 2–2 ist für mich völlig okay.\\ \tt \\ \tt Wenn du ready bist, machen wir die geheimen Vorschläge entsprechend. 
	  } 
	   } 
	   } 
	 & & \\ 
 

    \theutterance \stepcounter{utterance}  

    & & \multicolumn{4}{p{0.6\linewidth}}{\cellcolor[rgb]{0.9,0.9,0.9}{%
	\makecell[{{p{\linewidth}}}]{% 
	  \tt {\tiny [GM$\rangle$B]}  
	 Klingt gut – deine Zwiebelpräferenz ist deutlich höher, also sollte sie zu dir. Meine Wertung passt auch zu deiner Richtung.\\ \tt \\ \tt Mir wäre wichtig, mindestens die Karotte zu bekommen (3 Punkte für mich, 0 für dich). Bei den Taschen sind wir beide bei 1 pro Stück – da ist flexibel teilen ok.\\ \tt \\ \tt Ich bin einverstanden mit: du bekommst die Zwiebel, ich die Karotte, Taschen 2–2. Wenn du magst, könnten wir auch 3 Taschen zu dir, 1 zu mir machen, falls du das bevorzugst, aber 2–2 ist für mich völlig okay.\\ \tt \\ \tt Wenn du ready bist, machen wir die geheimen Vorschläge entsprechend. 
	  } 
	   } 
	   } 
	 & \\ 
 

    \theutterance \stepcounter{utterance}  

    & & & \multicolumn{4}{p{0.6\linewidth}}{\cellcolor[rgb]{1,0.85,0.72}{%
	\makecell[{{p{\linewidth}}}]{% 
	  \tt {\tiny [GM$\langle$B]}  
	 Klingt perfekt. Dann nehmen wir 2–2 bei den Taschen: du bekommst die Karotte, ich die Zwiebel, Taschen 2–2. Ich mache jetzt meinen geheimen Vorschlag.\\ \tt \\ \tt [Vorschlag: 1 Zwiebel, 2 Taschen] 
	  } 
	   } 
	   } 
	  \\ 
 

    \theutterance \stepcounter{utterance}  

    & & & \multicolumn{2}{p{0.3\linewidth}}{\cellcolor[rgb]{0.95,0.95,0.95}{%
	\makecell[{{p{\linewidth}}}]{% 
	  \tt {\tiny [GM$|$GM]}  
	 proposal 
	  } 
	   } 
	   } 
	 & & \\ 
 

    \theutterance \stepcounter{utterance}  

    & & & \multicolumn{2}{p{0.3\linewidth}}{\cellcolor[rgb]{0.95,0.95,0.95}{%
	\makecell[{{p{\linewidth}}}]{% 
	  \tt {\tiny [GM$|$GM]}  
	 [0, 2, 1] 
	  } 
	   } 
	   } 
	 & & \\ 
 

    \theutterance \stepcounter{utterance}  

    & & \multicolumn{4}{p{0.6\linewidth}}{\cellcolor[rgb]{0.9,0.9,0.9}{%
	\makecell[{{p{\linewidth}}}]{% 
	  \tt {\tiny [A$\langle$GM]}  
	 Der andere Spieler hat einen geheimen Vorschlag gemacht. Jetzt bist du an der Reihe, einen eigenen Vorschlag zu machen.\\ \tt  
	  } 
	   } 
	   } 
	 & \\ 
 

    \theutterance \stepcounter{utterance}  

    & \multicolumn{4}{p{0.6\linewidth}}{\cellcolor[rgb]{0.8,1,0.9}{%
	\makecell[{{p{\linewidth}}}]{% 
	  \tt {\tiny [A$\rangle$GM]}  
	 [Vorschlag: 1 Karotte, 2 Taschen] 
	  } 
	   } 
	   } 
	 & & \\ 
 

    \theutterance \stepcounter{utterance}  

    & & & \multicolumn{2}{p{0.3\linewidth}}{\cellcolor[rgb]{0.95,0.95,0.95}{%
	\makecell[{{p{\linewidth}}}]{% 
	  \tt {\tiny [GM$|$GM]}  
	 proposal 
	  } 
	   } 
	   } 
	 & & \\ 
 

    \theutterance \stepcounter{utterance}  

    & & & \multicolumn{2}{p{0.3\linewidth}}{\cellcolor[rgb]{0.95,0.95,0.95}{%
	\makecell[{{p{\linewidth}}}]{% 
	  \tt {\tiny [GM$|$GM]}  
	 [1, 2, 0] 
	  } 
	   } 
	   } 
	 & & \\ 
 

    \theutterance \stepcounter{utterance}  

    & & & \multicolumn{2}{p{0.3\linewidth}}{\cellcolor[rgb]{0.95,0.95,0.95}{%
	\makecell[{{p{\linewidth}}}]{% 
	  \tt {\tiny [GM$|$GM]}  
	 [[1, 2, 0], [0, 2, 1]] 
	  } 
	   } 
	   } 
	 & & \\ 
 

\end{supertabular}
}

\end{document}
