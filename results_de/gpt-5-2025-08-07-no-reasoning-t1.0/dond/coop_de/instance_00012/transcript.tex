
\documentclass{article}
\usepackage{colortbl}
\usepackage{makecell}
\usepackage{multirow}
\usepackage{supertabular}

\begin{document}

\newcounter{utterance}

\centering \large Interaction Transcript for game `dond', experiment `coop\_de', episode 12 with gpt{-}5{-}2025{-}08{-}07{-}no{-}reasoning{-}t1.0.
\vspace{24pt}

{ \footnotesize  \setcounter{utterance}{1}
\setlength{\tabcolsep}{0pt}
\begin{supertabular}{c@{$\;$}|p{.15\linewidth}@{}p{.15\linewidth}p{.15\linewidth}p{.15\linewidth}p{.15\linewidth}p{.15\linewidth}}
    \# & $\;$A & \multicolumn{4}{c}{Game Master} & $\;\:$B\\
    \hline

    \theutterance \stepcounter{utterance}  
    & & \multicolumn{4}{p{0.6\linewidth}}{
        \cellcolor[rgb]{0.9,0.9,0.9}{
            \makecell[{{p{\linewidth}}}]{
                \texttt{\tiny{[P1$\langle$GM]}}
                \texttt{Sie spielen ein kooperatives Verhandlungsspiel, bei dem Sie sich mit einem anderen Spieler darauf einigen müssen, wie eine Reihe von Gegenständen aufgeteilt werden soll.} \\
\\ 
\texttt{Die Regeln:} \\
\texttt{(a) Sie und der andere Spieler erhalten eine Sammlung von Gegenständen. Jeder von Ihnen erhält außerdem eine geheime Wertfunktion, die angibt, wie viel Ihnen jede Art von Gegenstand wert ist.} \\
\texttt{(b) Sie tauschen Nachrichten mit dem anderen Spieler aus, um zu vereinbaren, wer welche Gegenstände bekommt. Sie können jeweils maximal 5 Nachrichten senden oder das Spiel vorzeitig beenden, indem Sie jederzeit einen geheimen Vorschlag machen.} \\
\texttt{(c) Jeder von euch wird aufgefordert, einen geheimen Vorschlag zu machen, in dem ihr die gewünschten Gegenstände in eckigen Klammern wie folgt angibt: "{[}Vorschlag: <Nummer> <Objektname>, <Nummer> <Objektname>, <...>{]}"} \\
\texttt{(d) Wenn eure Vorschläge komplementär sind, d.h. es gibt genug Gegenstände, um beide Vorschläge zu erfüllen, erhält jeder Spieler eine Punktzahl, die sich aus der Summe der Werte für die Gegenstände ergibt, die er erhalten hat. Andernfalls erhalten Sie beide null Punkte.} \\
\texttt{(e) Ihr Ziel ist es, die Summe aus Ihrer Punktzahl und der Punktzahl des anderen Spielers zu maximieren.} \\
\\ 
\texttt{Beginnen wir.} \\
\\ 
\texttt{Die Menge der verfügbaren Gegenstände ist:} \\
\\ 
\texttt{3 Lineale, 2 Eier, 1 Batterie} \\
\\ 
\texttt{Deine geheime Wertfunktion ist:} \\
\\ 
\texttt{Lineal: 1, Ei: 2, Batterie: 3} \\
\\ 
\texttt{WICHTIG: Ihre Nachrichten werden, sofern es sich nicht um einen geheimen Vorschlag handelt, direkt an den anderen Spieler übermittelt, also fügen Sie keine Antwort auf die Regeln oder einen Text zur Ankündigung Ihrer Nachricht ein. Um einen geheimen Vorschlag zu machen, verwenden Sie das angegebene Format. Verwenden Sie keine eckigen Klammern, wenn Sie mit dem anderen Spieler kommunizieren, sonst wird dies als Ihr geheimer Vorschlag interpretiert.} \\
            }
        }
    }
    & \\ \\

    \theutterance \stepcounter{utterance}  
    & \multicolumn{4}{p{0.6\linewidth}}{
        \cellcolor[rgb]{0.9,0.9,0.9}{
            \makecell[{{p{\linewidth}}}]{
                \texttt{\tiny{[P1$\rangle$GM]}}
                \texttt{Hallo! Lass uns effizient eine Pareto{-}optimale Aufteilung finden. Meine Präferenzen sind recht ausgeglichen: Batterie ist mir am meisten wert, Eier mittel, Lineale am wenigsten. Wie sind deine groben Prioritäten zwischen Batterie, Eiern und Linealen? Ich kann flexibel bei Linealen und Eiern sein, wenn du die Batterie stark willst.} \\
            }
        }
    }
    & & \\ \\

    \theutterance \stepcounter{utterance}  
    & & & \multicolumn{2}{p{0.3\linewidth}}{
        \cellcolor[rgb]{0.9,0.9,0.9}{
            \makecell[{{p{\linewidth}}}]{
                \texttt{\tiny{[GM$|$GM]}}
                \texttt{continue} \\
            }
        }
    }
    & & \\ \\

    \theutterance \stepcounter{utterance}  
    & & & \multicolumn{2}{p{0.3\linewidth}}{
        \cellcolor[rgb]{0.9,0.9,0.9}{
            \makecell[{{p{\linewidth}}}]{
                \texttt{\tiny{[GM$|$GM]}}
                \texttt{Hallo! Lass uns effizient eine Pareto{-}optimale Aufteilung finden. Meine Präferenzen sind recht ausgeglichen: Batterie ist mir am meisten wert, Eier mittel, Lineale am wenigsten. Wie sind deine groben Prioritäten zwischen Batterie, Eiern und Linealen? Ich kann flexibel bei Linealen und Eiern sein, wenn du die Batterie stark willst.} \\
            }
        }
    }
    & & \\ \\

    \theutterance \stepcounter{utterance}  
    & & \multicolumn{4}{p{0.6\linewidth}}{
        \cellcolor[rgb]{0.9,0.9,0.9}{
            \makecell[{{p{\linewidth}}}]{
                \texttt{\tiny{[GM$\rangle$P2]}}
                \texttt{Sie spielen ein kooperatives Verhandlungsspiel, bei dem Sie sich mit einem anderen Spieler darauf einigen müssen, wie eine Reihe von Gegenständen aufgeteilt werden soll.} \\
\\ 
\texttt{Die Regeln:} \\
\texttt{(a) Sie und der andere Spieler erhalten eine Sammlung von Gegenständen. Jeder von Ihnen erhält außerdem eine geheime Wertfunktion, die angibt, wie viel Ihnen jede Art von Gegenstand wert ist.} \\
\texttt{(b) Sie tauschen Nachrichten mit dem anderen Spieler aus, um zu vereinbaren, wer welche Gegenstände bekommt. Sie können jeweils maximal 5 Nachrichten senden oder das Spiel vorzeitig beenden, indem Sie jederzeit einen geheimen Vorschlag machen.} \\
\texttt{(c) Jeder von euch wird aufgefordert, einen geheimen Vorschlag zu machen, in dem ihr die gewünschten Gegenstände in eckigen Klammern wie folgt angibt: "{[}Vorschlag: <Nummer> <Objektname>, <Nummer> <Objektname>, <...>{]}"} \\
\texttt{(d) Wenn eure Vorschläge komplementär sind, d.h. es gibt genug Gegenstände, um beide Vorschläge zu erfüllen, erhält jeder Spieler eine Punktzahl, die sich aus der Summe der Werte für die Gegenstände ergibt, die er erhalten hat. Andernfalls erhalten Sie beide null Punkte.} \\
\texttt{(e) Ihr Ziel ist es, die Summe aus Ihrer Punktzahl und der Punktzahl des anderen Spielers zu maximieren.} \\
\\ 
\texttt{Beginnen wir.} \\
\\ 
\texttt{Die Menge der verfügbaren Gegenstände ist:} \\
\\ 
\texttt{3 Lineale, 2 Eier, 1 Batterie} \\
\\ 
\texttt{Deine geheime Wertfunktion ist:} \\
\\ 
\texttt{Lineal: 3, Ei: 0, Batterie: 1} \\
\\ 
\texttt{WICHTIG: Ihre Nachrichten werden, sofern es sich nicht um einen geheimen Vorschlag handelt, direkt an den anderen Spieler übermittelt, also fügen Sie keine Antwort auf die Regeln oder einen Text zur Ankündigung Ihrer Nachricht ein. Um einen geheimen Vorschlag zu machen, verwenden Sie das angegebene Format. Verwenden Sie keine eckigen Klammern, wenn Sie mit dem anderen Spieler kommunizieren, sonst wird dies als Ihr geheimer Vorschlag interpretiert.} \\
\\ 
\\ 
\texttt{Hallo! Lass uns effizient eine Pareto{-}optimale Aufteilung finden. Meine Präferenzen sind recht ausgeglichen: Batterie ist mir am meisten wert, Eier mittel, Lineale am wenigsten. Wie sind deine groben Prioritäten zwischen Batterie, Eiern und Linealen? Ich kann flexibel bei Linealen und Eiern sein, wenn du die Batterie stark willst.} \\
            }
        }
    }
    & \\ \\

    \theutterance \stepcounter{utterance}  
    & & & \multicolumn{4}{p{0.6\linewidth}}{
        \cellcolor[rgb]{0.9,0.9,0.9}{
            \makecell[{{p{\linewidth}}}]{
                \texttt{\tiny{[GM$\langle$P2]}}
                \texttt{Danke für die Offenheit! Meine Werte: Lineale sind mir am wichtigsten (3 Punkte pro Stück), Batterie ist gering wertvoll (1 Punkt), Eier sind mir nichts wert (0 Punkte). Ich würde also gern viele Lineale nehmen und bin bei Eiern sehr flexibel. Wie wichtig sind dir die Lineale relativ zu den Eiern? Wenn dir die Batterie sehr wichtig ist, kann ich sie dir überlassen.} \\
            }
        }
    }
     \\ \\

    \theutterance \stepcounter{utterance}  
    & & & \multicolumn{2}{p{0.3\linewidth}}{
        \cellcolor[rgb]{0.9,0.9,0.9}{
            \makecell[{{p{\linewidth}}}]{
                \texttt{\tiny{[GM$|$GM]}}
                \texttt{continue} \\
            }
        }
    }
    & & \\ \\

    \theutterance \stepcounter{utterance}  
    & & & \multicolumn{2}{p{0.3\linewidth}}{
        \cellcolor[rgb]{0.9,0.9,0.9}{
            \makecell[{{p{\linewidth}}}]{
                \texttt{\tiny{[GM$|$GM]}}
                \texttt{Danke für die Offenheit! Meine Werte: Lineale sind mir am wichtigsten (3 Punkte pro Stück), Batterie ist gering wertvoll (1 Punkt), Eier sind mir nichts wert (0 Punkte). Ich würde also gern viele Lineale nehmen und bin bei Eiern sehr flexibel. Wie wichtig sind dir die Lineale relativ zu den Eiern? Wenn dir die Batterie sehr wichtig ist, kann ich sie dir überlassen.} \\
            }
        }
    }
    & & \\ \\

    \theutterance \stepcounter{utterance}  
    & & \multicolumn{4}{p{0.6\linewidth}}{
        \cellcolor[rgb]{0.9,0.9,0.9}{
            \makecell[{{p{\linewidth}}}]{
                \texttt{\tiny{[P1$\langle$GM]}}
                \texttt{Danke für die Offenheit! Meine Werte: Lineale sind mir am wichtigsten (3 Punkte pro Stück), Batterie ist gering wertvoll (1 Punkt), Eier sind mir nichts wert (0 Punkte). Ich würde also gern viele Lineale nehmen und bin bei Eiern sehr flexibel. Wie wichtig sind dir die Lineale relativ zu den Eiern? Wenn dir die Batterie sehr wichtig ist, kann ich sie dir überlassen.} \\
            }
        }
    }
    & \\ \\

    \theutterance \stepcounter{utterance}  
    & \multicolumn{4}{p{0.6\linewidth}}{
        \cellcolor[rgb]{0.9,0.9,0.9}{
            \makecell[{{p{\linewidth}}}]{
                \texttt{\tiny{[P1$\rangle$GM]}}
                \texttt{Danke, das hilft. Mir ist die Batterie am wichtigsten (3), Eier mittel (2), Lineale am wenigsten (1). Da du Lineale liebst und Eier dir nichts bringen, schlage ich vor:} \\
\\ 
\texttt{{-} Du bekommst alle 3 Lineale (für dich 9, für mich 3 entgangene Punkte sind ok).} \\
\texttt{{-} Ich bekomme die 2 Eier (für mich 4, für dich 0) und die 1 Batterie (für mich 3, für dich 1 gering).} \\
\\ 
\texttt{Gesamtnutzen: 9 (du) + 7 (ich) = 16, was sehr gut ist. Passt dir das? Wenn ja, sag Bescheid, dann machen wir gleich die passenden geheimen Vorschläge.} \\
            }
        }
    }
    & & \\ \\

    \theutterance \stepcounter{utterance}  
    & & & \multicolumn{2}{p{0.3\linewidth}}{
        \cellcolor[rgb]{0.9,0.9,0.9}{
            \makecell[{{p{\linewidth}}}]{
                \texttt{\tiny{[GM$|$GM]}}
                \texttt{continue} \\
            }
        }
    }
    & & \\ \\

    \theutterance \stepcounter{utterance}  
    & & & \multicolumn{2}{p{0.3\linewidth}}{
        \cellcolor[rgb]{0.9,0.9,0.9}{
            \makecell[{{p{\linewidth}}}]{
                \texttt{\tiny{[GM$|$GM]}}
                \texttt{Danke, das hilft. Mir ist die Batterie am wichtigsten (3), Eier mittel (2), Lineale am wenigsten (1). Da du Lineale liebst und Eier dir nichts bringen, schlage ich vor:} \\
\\ 
\texttt{{-} Du bekommst alle 3 Lineale (für dich 9, für mich 3 entgangene Punkte sind ok).} \\
\texttt{{-} Ich bekomme die 2 Eier (für mich 4, für dich 0) und die 1 Batterie (für mich 3, für dich 1 gering).} \\
\\ 
\texttt{Gesamtnutzen: 9 (du) + 7 (ich) = 16, was sehr gut ist. Passt dir das? Wenn ja, sag Bescheid, dann machen wir gleich die passenden geheimen Vorschläge.} \\
            }
        }
    }
    & & \\ \\

    \theutterance \stepcounter{utterance}  
    & & \multicolumn{4}{p{0.6\linewidth}}{
        \cellcolor[rgb]{0.9,0.9,0.9}{
            \makecell[{{p{\linewidth}}}]{
                \texttt{\tiny{[GM$\rangle$P2]}}
                \texttt{Danke, das hilft. Mir ist die Batterie am wichtigsten (3), Eier mittel (2), Lineale am wenigsten (1). Da du Lineale liebst und Eier dir nichts bringen, schlage ich vor:} \\
\\ 
\texttt{{-} Du bekommst alle 3 Lineale (für dich 9, für mich 3 entgangene Punkte sind ok).} \\
\texttt{{-} Ich bekomme die 2 Eier (für mich 4, für dich 0) und die 1 Batterie (für mich 3, für dich 1 gering).} \\
\\ 
\texttt{Gesamtnutzen: 9 (du) + 7 (ich) = 16, was sehr gut ist. Passt dir das? Wenn ja, sag Bescheid, dann machen wir gleich die passenden geheimen Vorschläge.} \\
            }
        }
    }
    & \\ \\

    \theutterance \stepcounter{utterance}  
    & & & \multicolumn{4}{p{0.6\linewidth}}{
        \cellcolor[rgb]{0.9,0.9,0.9}{
            \makecell[{{p{\linewidth}}}]{
                \texttt{\tiny{[GM$\langle$P2]}}
                \texttt{Klingt perfekt für mich. Ich nehme gerne alle 3 Lineale, und du bekommst die 2 Eier und die Batterie. Lass uns das so festhalten und die geheimen Vorschläge abgeben.} \\
\\ 
\texttt{{[}Warte auf dein Go, dann gebe ich meinen geheimen Vorschlag ab.{]}} \\
            }
        }
    }
     \\ \\

    \theutterance \stepcounter{utterance}  
    & & & \multicolumn{2}{p{0.3\linewidth}}{
        \cellcolor[rgb]{0.9,0.9,0.9}{
            \makecell[{{p{\linewidth}}}]{
                \texttt{\tiny{[GM$|$GM]}}
                \texttt{proposal does not start with "vorschlag:"} \\
            }
        }
    }
    & & \\ \\

\end{supertabular}
}

\end{document}
