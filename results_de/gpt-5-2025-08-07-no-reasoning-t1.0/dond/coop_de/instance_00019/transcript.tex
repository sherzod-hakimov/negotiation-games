
\documentclass{article}
\usepackage{colortbl}
\usepackage{makecell}
\usepackage{multirow}
\usepackage{supertabular}

\begin{document}

\newcounter{utterance}

\twocolumn

{ \footnotesize  \setcounter{utterance}{1}
\setlength{\tabcolsep}{0pt}
\begin{supertabular}{c@{$\;$}|p{.15\linewidth}@{}p{.15\linewidth}p{.15\linewidth}p{.15\linewidth}p{.15\linewidth}p{.15\linewidth}}

    \# & $\;$A & \multicolumn{4}{c}{Game Master} & $\;\:$B\\
    \hline 

    \theutterance \stepcounter{utterance}  

    & & \multicolumn{4}{p{0.6\linewidth}}{\cellcolor[rgb]{0.9,0.9,0.9}{%
	\makecell[{{p{\linewidth}}}]{% 
	  \tt {\tiny [A$\langle$GM]}  
	 Sie spielen ein kooperatives Verhandlungsspiel, bei dem Sie sich mit einem anderen Spieler darauf einigen müssen, wie eine Reihe von Gegenständen aufgeteilt werden soll.\\ \tt \\ \tt Die Regeln:\\ \tt (a) Sie und der andere Spieler erhalten eine Sammlung von Gegenständen. Jeder von Ihnen erhält außerdem eine geheime Wertfunktion, die angibt, wie viel Ihnen jede Art von Gegenstand wert ist.\\ \tt (b) Sie tauschen Nachrichten mit dem anderen Spieler aus, um zu vereinbaren, wer welche Gegenstände bekommt. Sie können jeweils maximal 5 Nachrichten senden oder das Spiel vorzeitig beenden, indem Sie jederzeit einen geheimen Vorschlag machen.\\ \tt (c) Jeder von euch wird aufgefordert, einen geheimen Vorschlag zu machen, in dem ihr die gewünschten Gegenstände in eckigen Klammern wie folgt angibt: "[Vorschlag: <Nummer> <Objektname>, <Nummer> <Objektname>, <...>]"\\ \tt (d) Wenn eure Vorschläge komplementär sind, d.h. es gibt genug Gegenstände, um beide Vorschläge zu erfüllen, erhält jeder Spieler eine Punktzahl, die sich aus der Summe der Werte für die Gegenstände ergibt, die er erhalten hat. Andernfalls erhalten Sie beide null Punkte.\\ \tt (e) Ihr Ziel ist es, die Summe aus Ihrer Punktzahl und der Punktzahl des anderen Spielers zu maximieren.\\ \tt \\ \tt Beginnen wir.\\ \tt \\ \tt Die Menge der verfügbaren Gegenstände ist:\\ \tt \\ \tt 2 Haken, 1 Blume, 1 Karte, 1 Radiergummi, 1 Schwamm\\ \tt \\ \tt Deine geheime Wertfunktion ist:\\ \tt \\ \tt Haken: 3, Blume: 3, Karte: 0, Radiergummi: 1, Schwamm: 0\\ \tt \\ \tt WICHTIG: Ihre Nachrichten werden, sofern es sich nicht um einen geheimen Vorschlag handelt, direkt an den anderen Spieler übermittelt, also fügen Sie keine Antwort auf die Regeln oder einen Text zur Ankündigung Ihrer Nachricht ein. Um einen geheimen Vorschlag zu machen, verwenden Sie das angegebene Format. Verwenden Sie keine eckigen Klammern, wenn Sie mit dem anderen Spieler kommunizieren, sonst wird dies als Ihr geheimer Vorschlag interpretiert.\\ \tt  
	  } 
	   } 
	   } 
	 & \\ 
 

    \theutterance \stepcounter{utterance}  

    & \multicolumn{4}{p{0.6\linewidth}}{\cellcolor[rgb]{0.8,1,0.9}{%
	\makecell[{{p{\linewidth}}}]{% 
	  \tt {\tiny [A$\rangle$GM]}  
	 Hi! Lass uns gemeinsam den Gesamtwert maximieren. Meine Präferenzen: Haken und Blume sind mir am meisten wert, Radiergummi etwas, Karte/Schwamm sind mir egal. Wie sind deine Werte?\\ \tt \\ \tt Vorschlag zur Orientierung: Wenn du hohe Werte auf Karte/Schwamm hast, könntest du die nehmen. Ich würde gern möglichst die Blume und einen (oder zwei) Haken bekommen. Sag mir, was dir am wichtigsten ist, dann finden wir eine Aufteilung, die beide maximiert.\\ \tt \\ \tt Erster konkreter Split-Vorschlag (flexibel):\\ \tt - Ich: Blume, 1 Haken\\ \tt - Du: 1 Haken, Karte, Radiergummi, Schwamm\\ \tt \\ \tt Wenn das für dich schlecht ist, sag mir deine Top-2-Gegenstände, dann passe ich an. 
	  } 
	   } 
	   } 
	 & & \\ 
 

    \theutterance \stepcounter{utterance}  

    & & & \multicolumn{2}{p{0.3\linewidth}}{\cellcolor[rgb]{0.95,0.95,0.95}{%
	\makecell[{{p{\linewidth}}}]{% 
	  \tt {\tiny [GM$|$GM]}  
	 continue 
	  } 
	   } 
	   } 
	 & & \\ 
 

    \theutterance \stepcounter{utterance}  

    & & & \multicolumn{2}{p{0.3\linewidth}}{\cellcolor[rgb]{0.95,0.95,0.95}{%
	\makecell[{{p{\linewidth}}}]{% 
	  \tt {\tiny [GM$|$GM]}  
	 Hi! Lass uns gemeinsam den Gesamtwert maximieren. Meine Präferenzen: Haken und Blume sind mir am meisten wert, Radiergummi etwas, Karte/Schwamm sind mir egal. Wie sind deine Werte?\\ \tt \\ \tt Vorschlag zur Orientierung: Wenn du hohe Werte auf Karte/Schwamm hast, könntest du die nehmen. Ich würde gern möglichst die Blume und einen (oder zwei) Haken bekommen. Sag mir, was dir am wichtigsten ist, dann finden wir eine Aufteilung, die beide maximiert.\\ \tt \\ \tt Erster konkreter Split-Vorschlag (flexibel):\\ \tt - Ich: Blume, 1 Haken\\ \tt - Du: 1 Haken, Karte, Radiergummi, Schwamm\\ \tt \\ \tt Wenn das für dich schlecht ist, sag mir deine Top-2-Gegenstände, dann passe ich an. 
	  } 
	   } 
	   } 
	 & & \\ 
 

    \theutterance \stepcounter{utterance}  

    & & \multicolumn{4}{p{0.6\linewidth}}{\cellcolor[rgb]{0.9,0.9,0.9}{%
	\makecell[{{p{\linewidth}}}]{% 
	  \tt {\tiny [GM$\rangle$B]}  
	 Sie spielen ein kooperatives Verhandlungsspiel, bei dem Sie sich mit einem anderen Spieler darauf einigen müssen, wie eine Reihe von Gegenständen aufgeteilt werden soll.\\ \tt \\ \tt Die Regeln:\\ \tt (a) Sie und der andere Spieler erhalten eine Sammlung von Gegenständen. Jeder von Ihnen erhält außerdem eine geheime Wertfunktion, die angibt, wie viel Ihnen jede Art von Gegenstand wert ist.\\ \tt (b) Sie tauschen Nachrichten mit dem anderen Spieler aus, um zu vereinbaren, wer welche Gegenstände bekommt. Sie können jeweils maximal 5 Nachrichten senden oder das Spiel vorzeitig beenden, indem Sie jederzeit einen geheimen Vorschlag machen.\\ \tt (c) Jeder von euch wird aufgefordert, einen geheimen Vorschlag zu machen, in dem ihr die gewünschten Gegenstände in eckigen Klammern wie folgt angibt: "[Vorschlag: <Nummer> <Objektname>, <Nummer> <Objektname>, <...>]"\\ \tt (d) Wenn eure Vorschläge komplementär sind, d.h. es gibt genug Gegenstände, um beide Vorschläge zu erfüllen, erhält jeder Spieler eine Punktzahl, die sich aus der Summe der Werte für die Gegenstände ergibt, die er erhalten hat. Andernfalls erhalten Sie beide null Punkte.\\ \tt (e) Ihr Ziel ist es, die Summe aus Ihrer Punktzahl und der Punktzahl des anderen Spielers zu maximieren.\\ \tt \\ \tt Beginnen wir.\\ \tt \\ \tt Die Menge der verfügbaren Gegenstände ist:\\ \tt \\ \tt 2 Haken, 1 Blume, 1 Karte, 1 Radiergummi, 1 Schwamm\\ \tt \\ \tt Deine geheime Wertfunktion ist:\\ \tt \\ \tt Haken: 0, Blume: 2, Karte: 1, Radiergummi: 4, Schwamm: 3\\ \tt \\ \tt WICHTIG: Ihre Nachrichten werden, sofern es sich nicht um einen geheimen Vorschlag handelt, direkt an den anderen Spieler übermittelt, also fügen Sie keine Antwort auf die Regeln oder einen Text zur Ankündigung Ihrer Nachricht ein. Um einen geheimen Vorschlag zu machen, verwenden Sie das angegebene Format. Verwenden Sie keine eckigen Klammern, wenn Sie mit dem anderen Spieler kommunizieren, sonst wird dies als Ihr geheimer Vorschlag interpretiert.\\ \tt \\ \tt \\ \tt Hi! Lass uns gemeinsam den Gesamtwert maximieren. Meine Präferenzen: Haken und Blume sind mir am meisten wert, Radiergummi etwas, Karte/Schwamm sind mir egal. Wie sind deine Werte?\\ \tt \\ \tt Vorschlag zur Orientierung: Wenn du hohe Werte auf Karte/Schwamm hast, könntest du die nehmen. Ich würde gern möglichst die Blume und einen (oder zwei) Haken bekommen. Sag mir, was dir am wichtigsten ist, dann finden wir eine Aufteilung, die beide maximiert.\\ \tt \\ \tt Erster konkreter Split-Vorschlag (flexibel):\\ \tt - Ich: Blume, 1 Haken\\ \tt - Du: 1 Haken, Karte, Radiergummi, Schwamm\\ \tt \\ \tt Wenn das für dich schlecht ist, sag mir deine Top-2-Gegenstände, dann passe ich an. 
	  } 
	   } 
	   } 
	 & \\ 
 

    \theutterance \stepcounter{utterance}  

    & & & \multicolumn{4}{p{0.6\linewidth}}{\cellcolor[rgb]{1,0.85,0.72}{%
	\makecell[{{p{\linewidth}}}]{% 
	  \tt {\tiny [GM$\langle$B]}  
	 Danke für die Offenheit! Meine Werte sind:\\ \tt - Haken: 0\\ \tt - Blume: 2\\ \tt - Karte: 1\\ \tt - Radiergummi: 4\\ \tt - Schwamm: 3\\ \tt \\ \tt Dein Vorschlag passt schon ziemlich gut zu meinen Werten, weil ich Radiergummi und Schwamm sehr hoch schätze und Haken mir nichts bringen. Konkret hätte ich folgenden Wunsch:\\ \tt \\ \tt - Du: Blume, 1 Haken\\ \tt - Ich: 1 Haken, Karte, Radiergummi, Schwamm\\ \tt \\ \tt Das ist exakt dein Split und für mich stark. Wenn du lieber beide Haken willst, könnten wir auch so gehen:\\ \tt - Du: Blume, 2 Haken\\ \tt - Ich: Karte, Radiergummi, Schwamm\\ \tt \\ \tt Gib Bescheid, welche Variante du bevorzugst. Beide sind für mich gut. 
	  } 
	   } 
	   } 
	  \\ 
 

    \theutterance \stepcounter{utterance}  

    & & & \multicolumn{2}{p{0.3\linewidth}}{\cellcolor[rgb]{0.95,0.95,0.95}{%
	\makecell[{{p{\linewidth}}}]{% 
	  \tt {\tiny [GM$|$GM]}  
	 continue 
	  } 
	   } 
	   } 
	 & & \\ 
 

    \theutterance \stepcounter{utterance}  

    & & & \multicolumn{2}{p{0.3\linewidth}}{\cellcolor[rgb]{0.95,0.95,0.95}{%
	\makecell[{{p{\linewidth}}}]{% 
	  \tt {\tiny [GM$|$GM]}  
	 Danke für die Offenheit! Meine Werte sind:\\ \tt - Haken: 0\\ \tt - Blume: 2\\ \tt - Karte: 1\\ \tt - Radiergummi: 4\\ \tt - Schwamm: 3\\ \tt \\ \tt Dein Vorschlag passt schon ziemlich gut zu meinen Werten, weil ich Radiergummi und Schwamm sehr hoch schätze und Haken mir nichts bringen. Konkret hätte ich folgenden Wunsch:\\ \tt \\ \tt - Du: Blume, 1 Haken\\ \tt - Ich: 1 Haken, Karte, Radiergummi, Schwamm\\ \tt \\ \tt Das ist exakt dein Split und für mich stark. Wenn du lieber beide Haken willst, könnten wir auch so gehen:\\ \tt - Du: Blume, 2 Haken\\ \tt - Ich: Karte, Radiergummi, Schwamm\\ \tt \\ \tt Gib Bescheid, welche Variante du bevorzugst. Beide sind für mich gut. 
	  } 
	   } 
	   } 
	 & & \\ 
 

    \theutterance \stepcounter{utterance}  

    & & \multicolumn{4}{p{0.6\linewidth}}{\cellcolor[rgb]{0.9,0.9,0.9}{%
	\makecell[{{p{\linewidth}}}]{% 
	  \tt {\tiny [A$\langle$GM]}  
	 Danke für die Offenheit! Meine Werte sind:\\ \tt - Haken: 0\\ \tt - Blume: 2\\ \tt - Karte: 1\\ \tt - Radiergummi: 4\\ \tt - Schwamm: 3\\ \tt \\ \tt Dein Vorschlag passt schon ziemlich gut zu meinen Werten, weil ich Radiergummi und Schwamm sehr hoch schätze und Haken mir nichts bringen. Konkret hätte ich folgenden Wunsch:\\ \tt \\ \tt - Du: Blume, 1 Haken\\ \tt - Ich: 1 Haken, Karte, Radiergummi, Schwamm\\ \tt \\ \tt Das ist exakt dein Split und für mich stark. Wenn du lieber beide Haken willst, könnten wir auch so gehen:\\ \tt - Du: Blume, 2 Haken\\ \tt - Ich: Karte, Radiergummi, Schwamm\\ \tt \\ \tt Gib Bescheid, welche Variante du bevorzugst. Beide sind für mich gut. 
	  } 
	   } 
	   } 
	 & \\ 
 

    \theutterance \stepcounter{utterance}  

    & \multicolumn{4}{p{0.6\linewidth}}{\cellcolor[rgb]{0.8,1,0.9}{%
	\makecell[{{p{\linewidth}}}]{% 
	  \tt {\tiny [A$\rangle$GM]}  
	 Super, danke für die klaren Werte! Beide Varianten maximieren gemeinsam 3+3+1+4+3 = 14 Gesamtpunkte, also top.\\ \tt \\ \tt Für mich ist Variante 2 minimal besser (ich bekomme Blume+2 Haken = 9 statt 6), während du dort 8 statt 11 bekommst. Da wir die Summe maximieren wollen und du beide gut findest, würde ich fairerweise die erste, ausgewogenere Variante nehmen:\\ \tt \\ \tt - Du: Blume, 1 Haken\\ \tt - Du: 6 Punkte\\ \tt - Ich: 1 Haken, Karte, Radiergummi, Schwamm\\ \tt - Ich: 8 Punkte\\ \tt \\ \tt Wenn das passt, lass uns es so fix machen.\\ \tt \\ \tt [Vorschlag: 1 Haken, 1 Blume] 
	  } 
	   } 
	   } 
	 & & \\ 
 

    \theutterance \stepcounter{utterance}  

    & & & \multicolumn{2}{p{0.3\linewidth}}{\cellcolor[rgb]{0.95,0.95,0.95}{%
	\makecell[{{p{\linewidth}}}]{% 
	  \tt {\tiny [GM$|$GM]}  
	 proposal 
	  } 
	   } 
	   } 
	 & & \\ 
 

    \theutterance \stepcounter{utterance}  

    & & & \multicolumn{2}{p{0.3\linewidth}}{\cellcolor[rgb]{0.95,0.95,0.95}{%
	\makecell[{{p{\linewidth}}}]{% 
	  \tt {\tiny [GM$|$GM]}  
	 [1, 1, 0, 0, 0] 
	  } 
	   } 
	   } 
	 & & \\ 
 

    \theutterance \stepcounter{utterance}  

    & & \multicolumn{4}{p{0.6\linewidth}}{\cellcolor[rgb]{0.9,0.9,0.9}{%
	\makecell[{{p{\linewidth}}}]{% 
	  \tt {\tiny [GM$\rangle$B]}  
	 Der andere Spieler hat einen geheimen Vorschlag gemacht. Jetzt bist du an der Reihe, einen eigenen Vorschlag zu machen.\\ \tt  
	  } 
	   } 
	   } 
	 & \\ 
 

    \theutterance \stepcounter{utterance}  

    & & & \multicolumn{4}{p{0.6\linewidth}}{\cellcolor[rgb]{1,0.85,0.72}{%
	\makecell[{{p{\linewidth}}}]{% 
	  \tt {\tiny [GM$\langle$B]}  
	 [Vorschlag: 1 Haken, 1 Blume, 1 Karte, 1 Radiergummi, 1 Schwamm] 
	  } 
	   } 
	   } 
	  \\ 
 

    \theutterance \stepcounter{utterance}  

    & & & \multicolumn{2}{p{0.3\linewidth}}{\cellcolor[rgb]{0.95,0.95,0.95}{%
	\makecell[{{p{\linewidth}}}]{% 
	  \tt {\tiny [GM$|$GM]}  
	 proposal 
	  } 
	   } 
	   } 
	 & & \\ 
 

    \theutterance \stepcounter{utterance}  

    & & & \multicolumn{2}{p{0.3\linewidth}}{\cellcolor[rgb]{0.95,0.95,0.95}{%
	\makecell[{{p{\linewidth}}}]{% 
	  \tt {\tiny [GM$|$GM]}  
	 [1, 1, 1, 1, 1] 
	  } 
	   } 
	   } 
	 & & \\ 
 

    \theutterance \stepcounter{utterance}  

    & & & \multicolumn{2}{p{0.3\linewidth}}{\cellcolor[rgb]{0.95,0.95,0.95}{%
	\makecell[{{p{\linewidth}}}]{% 
	  \tt {\tiny [GM$|$GM]}  
	 [[1, 1, 0, 0, 0], [1, 1, 1, 1, 1]] 
	  } 
	   } 
	   } 
	 & & \\ 
 

\end{supertabular}
}

\end{document}
