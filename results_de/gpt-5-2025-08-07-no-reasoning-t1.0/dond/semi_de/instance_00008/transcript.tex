
\documentclass{article}
\usepackage{colortbl}
\usepackage{makecell}
\usepackage{multirow}
\usepackage{supertabular}

\begin{document}

\newcounter{utterance}

\twocolumn

{ \footnotesize  \setcounter{utterance}{1}
\setlength{\tabcolsep}{0pt}
\begin{supertabular}{c@{$\;$}|p{.15\linewidth}@{}p{.15\linewidth}p{.15\linewidth}p{.15\linewidth}p{.15\linewidth}p{.15\linewidth}}

    \# & $\;$A & \multicolumn{4}{c}{Game Master} & $\;\:$B\\
    \hline 

    \theutterance \stepcounter{utterance}  

    & & \multicolumn{4}{p{0.6\linewidth}}{\cellcolor[rgb]{0.9,0.9,0.9}{%
	\makecell[{{p{\linewidth}}}]{% 
	  \tt {\tiny [A$\langle$GM]}  
	 Sie spielen ein Verhandlungsspiel, bei dem Sie sich mit einem anderen Spieler darauf einigen müssen, wie eine Reihe von Gegenständen aufgeteilt werden soll.\\ \tt \\ \tt Die Regeln:\\ \tt (a) Sie und der andere Spieler erhalten eine Sammlung von Gegenständen. Jeder von Ihnen erhält außerdem eine geheime Wertfunktion, die angibt, wie viel Ihnen jede Art von Gegenstand wert ist.\\ \tt (b) Sie tauschen Nachrichten mit dem anderen Spieler aus, um zu vereinbaren, wer welche Gegenstände bekommt. Sie können jeweils maximal 5 Nachrichten senden oder das Spiel vorzeitig beenden, indem Sie jederzeit einen geheimen Vorschlag machen.\\ \tt (c) Jeder von euch wird aufgefordert, einen geheimen Vorschlag zu machen, in dem ihr die gewünschten Gegenstände in eckigen Klammern wie folgt angibt: "[Vorschlag: <Nummer> <Objektname>, <Nummer> <Objektname>, <...>]"\\ \tt (d) Wenn eure Vorschläge komplementär sind, d.h. es gibt genug Gegenstände, um beide Vorschläge zu erfüllen, erhält jeder Spieler eine Punktzahl, die sich aus der Summe der Werte für die Gegenstände ergibt, die er erhalten hat. Andernfalls erhalten Sie beide null Punkte.\\ \tt (e) Ihr Ziel ist es, eine möglichst hohe Punktzahl zu erreichen.\\ \tt \\ \tt Beginnen wir.\\ \tt \\ \tt Die Menge der verfügbaren Gegenstände ist:\\ \tt \\ \tt 3 Schaufeln, 2 Bilder, 1 Muschel\\ \tt \\ \tt Deine geheime Wertfunktion ist:\\ \tt \\ \tt Schaufel: 0, Bild: 0, Muschel: 10\\ \tt \\ \tt WICHTIG: Ihre Nachrichten werden, sofern es sich nicht um einen geheimen Vorschlag handelt, direkt an den anderen Spieler übermittelt, also fügen Sie keine Antwort auf die Regeln oder einen Text zur Ankündigung Ihrer Nachricht ein. Um einen geheimen Vorschlag zu machen, verwenden Sie das angegebene Format. Verwenden Sie keine eckigen Klammern, wenn Sie mit dem anderen Spieler kommunizieren, sonst wird dies als Ihr geheimer Vorschlag interpretiert.\\ \tt  
	  } 
	   } 
	   } 
	 & \\ 
 

    \theutterance \stepcounter{utterance}  

    & \multicolumn{4}{p{0.6\linewidth}}{\cellcolor[rgb]{0.8,1,0.9}{%
	\makecell[{{p{\linewidth}}}]{% 
	  \tt {\tiny [A$\rangle$GM]}  
	 Hallo! Danke für die Zusammenarbeit. Für mich ist vor allem ein einzelner Gegenstand besonders wertvoll, der Rest ist mir ziemlich egal. Wie sieht es bei dir aus? Welche Gegenstände sind dir wichtig, und brauchst du mehrere Exemplare davon oder nur bestimmte? 
	  } 
	   } 
	   } 
	 & & \\ 
 

    \theutterance \stepcounter{utterance}  

    & & & \multicolumn{2}{p{0.3\linewidth}}{\cellcolor[rgb]{0.95,0.95,0.95}{%
	\makecell[{{p{\linewidth}}}]{% 
	  \tt {\tiny [GM$|$GM]}  
	 continue 
	  } 
	   } 
	   } 
	 & & \\ 
 

    \theutterance \stepcounter{utterance}  

    & & & \multicolumn{2}{p{0.3\linewidth}}{\cellcolor[rgb]{0.95,0.95,0.95}{%
	\makecell[{{p{\linewidth}}}]{% 
	  \tt {\tiny [GM$|$GM]}  
	 Hallo! Danke für die Zusammenarbeit. Für mich ist vor allem ein einzelner Gegenstand besonders wertvoll, der Rest ist mir ziemlich egal. Wie sieht es bei dir aus? Welche Gegenstände sind dir wichtig, und brauchst du mehrere Exemplare davon oder nur bestimmte? 
	  } 
	   } 
	   } 
	 & & \\ 
 

    \theutterance \stepcounter{utterance}  

    & & \multicolumn{4}{p{0.6\linewidth}}{\cellcolor[rgb]{0.9,0.9,0.9}{%
	\makecell[{{p{\linewidth}}}]{% 
	  \tt {\tiny [GM$\rangle$B]}  
	 Sie spielen ein Verhandlungsspiel, bei dem Sie sich mit einem anderen Spieler darauf einigen müssen, wie eine Reihe von Gegenständen aufgeteilt werden soll.\\ \tt \\ \tt Die Regeln:\\ \tt (a) Sie und der andere Spieler erhalten eine Sammlung von Gegenständen. Jeder von Ihnen erhält außerdem eine geheime Wertfunktion, die angibt, wie viel Ihnen jede Art von Gegenstand wert ist.\\ \tt (b) Sie tauschen Nachrichten mit dem anderen Spieler aus, um zu vereinbaren, wer welche Gegenstände bekommt. Sie können jeweils maximal 5 Nachrichten senden oder das Spiel vorzeitig beenden, indem Sie jederzeit einen geheimen Vorschlag machen.\\ \tt (c) Jeder von euch wird aufgefordert, einen geheimen Vorschlag zu machen, in dem ihr die gewünschten Gegenstände in eckigen Klammern wie folgt angibt: "[Vorschlag: <Nummer> <Objektname>, <Nummer> <Objektname>, <...>]"\\ \tt (d) Wenn eure Vorschläge komplementär sind, d.h. es gibt genug Gegenstände, um beide Vorschläge zu erfüllen, erhält jeder Spieler eine Punktzahl, die sich aus der Summe der Werte für die Gegenstände ergibt, die er erhalten hat. Andernfalls erhalten Sie beide null Punkte.\\ \tt (e) Ihr Ziel ist es, eine möglichst hohe Punktzahl zu erreichen.\\ \tt \\ \tt Beginnen wir.\\ \tt \\ \tt Die Menge der verfügbaren Gegenstände ist:\\ \tt \\ \tt 3 Schaufeln, 2 Bilder, 1 Muschel\\ \tt \\ \tt Deine geheime Wertfunktion ist:\\ \tt \\ \tt Schaufel: 1, Bild: 1, Muschel: 5\\ \tt \\ \tt WICHTIG: Ihre Nachrichten werden, sofern es sich nicht um einen geheimen Vorschlag handelt, direkt an den anderen Spieler übermittelt, also fügen Sie keine Antwort auf die Regeln oder einen Text zur Ankündigung Ihrer Nachricht ein. Um einen geheimen Vorschlag zu machen, verwenden Sie das angegebene Format. Verwenden Sie keine eckigen Klammern, wenn Sie mit dem anderen Spieler kommunizieren, sonst wird dies als Ihr geheimer Vorschlag interpretiert.\\ \tt \\ \tt \\ \tt Hallo! Danke für die Zusammenarbeit. Für mich ist vor allem ein einzelner Gegenstand besonders wertvoll, der Rest ist mir ziemlich egal. Wie sieht es bei dir aus? Welche Gegenstände sind dir wichtig, und brauchst du mehrere Exemplare davon oder nur bestimmte? 
	  } 
	   } 
	   } 
	 & \\ 
 

    \theutterance \stepcounter{utterance}  

    & & & \multicolumn{4}{p{0.6\linewidth}}{\cellcolor[rgb]{1,0.85,0.72}{%
	\makecell[{{p{\linewidth}}}]{% 
	  \tt {\tiny [GM$\langle$B]}  
	 Hallo! Danke dir. Mir ist eine Sorte deutlich wichtiger als die anderen, und von den übrigen nehme ich gerne etwas mit, aber die sind für mich eher austauschbar. Ich brauche nicht unbedingt mehrere Exemplare derselben Sache, wenn ich dafür den wichtigsten Gegenstand für mich bekommen kann.\\ \tt \\ \tt Magst du sagen, ob du eher an mehreren identischen Stücken interessiert bist oder an einem speziellen Einzelstück? Und gibt es etwas, das dir gar nicht wichtig ist? 
	  } 
	   } 
	   } 
	  \\ 
 

    \theutterance \stepcounter{utterance}  

    & & & \multicolumn{2}{p{0.3\linewidth}}{\cellcolor[rgb]{0.95,0.95,0.95}{%
	\makecell[{{p{\linewidth}}}]{% 
	  \tt {\tiny [GM$|$GM]}  
	 continue 
	  } 
	   } 
	   } 
	 & & \\ 
 

    \theutterance \stepcounter{utterance}  

    & & & \multicolumn{2}{p{0.3\linewidth}}{\cellcolor[rgb]{0.95,0.95,0.95}{%
	\makecell[{{p{\linewidth}}}]{% 
	  \tt {\tiny [GM$|$GM]}  
	 Hallo! Danke dir. Mir ist eine Sorte deutlich wichtiger als die anderen, und von den übrigen nehme ich gerne etwas mit, aber die sind für mich eher austauschbar. Ich brauche nicht unbedingt mehrere Exemplare derselben Sache, wenn ich dafür den wichtigsten Gegenstand für mich bekommen kann.\\ \tt \\ \tt Magst du sagen, ob du eher an mehreren identischen Stücken interessiert bist oder an einem speziellen Einzelstück? Und gibt es etwas, das dir gar nicht wichtig ist? 
	  } 
	   } 
	   } 
	 & & \\ 
 

    \theutterance \stepcounter{utterance}  

    & & \multicolumn{4}{p{0.6\linewidth}}{\cellcolor[rgb]{0.9,0.9,0.9}{%
	\makecell[{{p{\linewidth}}}]{% 
	  \tt {\tiny [A$\langle$GM]}  
	 Hallo! Danke dir. Mir ist eine Sorte deutlich wichtiger als die anderen, und von den übrigen nehme ich gerne etwas mit, aber die sind für mich eher austauschbar. Ich brauche nicht unbedingt mehrere Exemplare derselben Sache, wenn ich dafür den wichtigsten Gegenstand für mich bekommen kann.\\ \tt \\ \tt Magst du sagen, ob du eher an mehreren identischen Stücken interessiert bist oder an einem speziellen Einzelstück? Und gibt es etwas, das dir gar nicht wichtig ist? 
	  } 
	   } 
	   } 
	 & \\ 
 

    \theutterance \stepcounter{utterance}  

    & \multicolumn{4}{p{0.6\linewidth}}{\cellcolor[rgb]{0.8,1,0.9}{%
	\makecell[{{p{\linewidth}}}]{% 
	  \tt {\tiny [A$\rangle$GM]}  
	 Danke für die Offenheit. Für mich ist im Prinzip nur ein einzelnes, spezielles Einzelstück wirklich wichtig; der Rest hat für mich praktisch keinen Wert. Mehrere Exemplare brauche ich nicht.\\ \tt \\ \tt Klingt so, als hättest du eine andere Sorte als Top-Priorität und wärst bei den übrigen flexibel. Wenn du mir dein wichtigstes Stück überlässt, kann ich dir gern den Großteil oder sogar alle der anderen Sachen überlassen. Wäre das für dich grundsätzlich attraktiv? 
	  } 
	   } 
	   } 
	 & & \\ 
 

    \theutterance \stepcounter{utterance}  

    & & & \multicolumn{2}{p{0.3\linewidth}}{\cellcolor[rgb]{0.95,0.95,0.95}{%
	\makecell[{{p{\linewidth}}}]{% 
	  \tt {\tiny [GM$|$GM]}  
	 continue 
	  } 
	   } 
	   } 
	 & & \\ 
 

    \theutterance \stepcounter{utterance}  

    & & & \multicolumn{2}{p{0.3\linewidth}}{\cellcolor[rgb]{0.95,0.95,0.95}{%
	\makecell[{{p{\linewidth}}}]{% 
	  \tt {\tiny [GM$|$GM]}  
	 Danke für die Offenheit. Für mich ist im Prinzip nur ein einzelnes, spezielles Einzelstück wirklich wichtig; der Rest hat für mich praktisch keinen Wert. Mehrere Exemplare brauche ich nicht.\\ \tt \\ \tt Klingt so, als hättest du eine andere Sorte als Top-Priorität und wärst bei den übrigen flexibel. Wenn du mir dein wichtigstes Stück überlässt, kann ich dir gern den Großteil oder sogar alle der anderen Sachen überlassen. Wäre das für dich grundsätzlich attraktiv? 
	  } 
	   } 
	   } 
	 & & \\ 
 

    \theutterance \stepcounter{utterance}  

    & & \multicolumn{4}{p{0.6\linewidth}}{\cellcolor[rgb]{0.9,0.9,0.9}{%
	\makecell[{{p{\linewidth}}}]{% 
	  \tt {\tiny [GM$\rangle$B]}  
	 Danke für die Offenheit. Für mich ist im Prinzip nur ein einzelnes, spezielles Einzelstück wirklich wichtig; der Rest hat für mich praktisch keinen Wert. Mehrere Exemplare brauche ich nicht.\\ \tt \\ \tt Klingt so, als hättest du eine andere Sorte als Top-Priorität und wärst bei den übrigen flexibel. Wenn du mir dein wichtigstes Stück überlässt, kann ich dir gern den Großteil oder sogar alle der anderen Sachen überlassen. Wäre das für dich grundsätzlich attraktiv? 
	  } 
	   } 
	   } 
	 & \\ 
 

    \theutterance \stepcounter{utterance}  

    & & & \multicolumn{4}{p{0.6\linewidth}}{\cellcolor[rgb]{1,0.85,0.72}{%
	\makecell[{{p{\linewidth}}}]{% 
	  \tt {\tiny [GM$\langle$B]}  
	 Ja, das klingt für mich grundsätzlich attraktiv. Wenn du dieses eine spezielle Einzelstück bekommst, kann ich mit den übrigen gut leben.\\ \tt \\ \tt Magst du sagen, ob du eher an der Muschel oder an einem der anderen Dinge interessiert bist? Wenn du mir verrätst, welches dein wichtigstes Stück ist, kann ich dir vorschlagen, wie wir den Rest sinnvoll aufteilen, damit es für uns beide gut passt. 
	  } 
	   } 
	   } 
	  \\ 
 

    \theutterance \stepcounter{utterance}  

    & & & \multicolumn{2}{p{0.3\linewidth}}{\cellcolor[rgb]{0.95,0.95,0.95}{%
	\makecell[{{p{\linewidth}}}]{% 
	  \tt {\tiny [GM$|$GM]}  
	 continue 
	  } 
	   } 
	   } 
	 & & \\ 
 

    \theutterance \stepcounter{utterance}  

    & & & \multicolumn{2}{p{0.3\linewidth}}{\cellcolor[rgb]{0.95,0.95,0.95}{%
	\makecell[{{p{\linewidth}}}]{% 
	  \tt {\tiny [GM$|$GM]}  
	 Ja, das klingt für mich grundsätzlich attraktiv. Wenn du dieses eine spezielle Einzelstück bekommst, kann ich mit den übrigen gut leben.\\ \tt \\ \tt Magst du sagen, ob du eher an der Muschel oder an einem der anderen Dinge interessiert bist? Wenn du mir verrätst, welches dein wichtigstes Stück ist, kann ich dir vorschlagen, wie wir den Rest sinnvoll aufteilen, damit es für uns beide gut passt. 
	  } 
	   } 
	   } 
	 & & \\ 
 

    \theutterance \stepcounter{utterance}  

    & & \multicolumn{4}{p{0.6\linewidth}}{\cellcolor[rgb]{0.9,0.9,0.9}{%
	\makecell[{{p{\linewidth}}}]{% 
	  \tt {\tiny [A$\langle$GM]}  
	 Ja, das klingt für mich grundsätzlich attraktiv. Wenn du dieses eine spezielle Einzelstück bekommst, kann ich mit den übrigen gut leben.\\ \tt \\ \tt Magst du sagen, ob du eher an der Muschel oder an einem der anderen Dinge interessiert bist? Wenn du mir verrätst, welches dein wichtigstes Stück ist, kann ich dir vorschlagen, wie wir den Rest sinnvoll aufteilen, damit es für uns beide gut passt. 
	  } 
	   } 
	   } 
	 & \\ 
 

    \theutterance \stepcounter{utterance}  

    & \multicolumn{4}{p{0.6\linewidth}}{\cellcolor[rgb]{0.8,1,0.9}{%
	\makecell[{{p{\linewidth}}}]{% 
	  \tt {\tiny [A$\rangle$GM]}  
	 Ich bin an genau einem Einzelstück interessiert, nicht an mehreren gleichen – und die anderen Dinge sind mir praktisch egal. Wenn du mir dein wichtigstes Stück nennst, schlage ich vor, dass du dieses bekommst und ich dafür das für mich wichtige Einzelstück erhalte, und du zusätzlich den Großteil (oder alle) der restlichen Dinge. So maximieren wir vermutlich beide unseren Wert. 
	  } 
	   } 
	   } 
	 & & \\ 
 

    \theutterance \stepcounter{utterance}  

    & & & \multicolumn{2}{p{0.3\linewidth}}{\cellcolor[rgb]{0.95,0.95,0.95}{%
	\makecell[{{p{\linewidth}}}]{% 
	  \tt {\tiny [GM$|$GM]}  
	 continue 
	  } 
	   } 
	   } 
	 & & \\ 
 

    \theutterance \stepcounter{utterance}  

    & & & \multicolumn{2}{p{0.3\linewidth}}{\cellcolor[rgb]{0.95,0.95,0.95}{%
	\makecell[{{p{\linewidth}}}]{% 
	  \tt {\tiny [GM$|$GM]}  
	 Ich bin an genau einem Einzelstück interessiert, nicht an mehreren gleichen – und die anderen Dinge sind mir praktisch egal. Wenn du mir dein wichtigstes Stück nennst, schlage ich vor, dass du dieses bekommst und ich dafür das für mich wichtige Einzelstück erhalte, und du zusätzlich den Großteil (oder alle) der restlichen Dinge. So maximieren wir vermutlich beide unseren Wert. 
	  } 
	   } 
	   } 
	 & & \\ 
 

    \theutterance \stepcounter{utterance}  

    & & \multicolumn{4}{p{0.6\linewidth}}{\cellcolor[rgb]{0.9,0.9,0.9}{%
	\makecell[{{p{\linewidth}}}]{% 
	  \tt {\tiny [GM$\rangle$B]}  
	 Ich bin an genau einem Einzelstück interessiert, nicht an mehreren gleichen – und die anderen Dinge sind mir praktisch egal. Wenn du mir dein wichtigstes Stück nennst, schlage ich vor, dass du dieses bekommst und ich dafür das für mich wichtige Einzelstück erhalte, und du zusätzlich den Großteil (oder alle) der restlichen Dinge. So maximieren wir vermutlich beide unseren Wert. 
	  } 
	   } 
	   } 
	 & \\ 
 

    \theutterance \stepcounter{utterance}  

    & & & \multicolumn{4}{p{0.6\linewidth}}{\cellcolor[rgb]{1,0.85,0.72}{%
	\makecell[{{p{\linewidth}}}]{% 
	  \tt {\tiny [GM$\langle$B]}  
	 Klingt gut. Mir ist vor allem eine Sorte wichtig, und von den anderen nehme ich gern den Großteil.\\ \tt \\ \tt Mein wichtigstes Stück ist nicht das Einzelstück – ich vermute, deins ist das Einzelstück (die Muschel). Vorschlag zur Aufteilung: Du bekommst die Muschel, ich bekomme die meisten der restlichen Sachen. Zum Beispiel: du nimmst die Muschel, ich nehme zwei der Schaufeln und ein Bild, und das verbleibende Bild und die letzte Schaufel gehen an dich. Passt dir das so grundsätzlich, oder willst du lieber alle restlichen Dinge außer einem kleinen Teil? Wenn du magst, formuliere ich das gleich als Vorschlag. 
	  } 
	   } 
	   } 
	  \\ 
 

    \theutterance \stepcounter{utterance}  

    & & & \multicolumn{2}{p{0.3\linewidth}}{\cellcolor[rgb]{0.95,0.95,0.95}{%
	\makecell[{{p{\linewidth}}}]{% 
	  \tt {\tiny [GM$|$GM]}  
	 continue 
	  } 
	   } 
	   } 
	 & & \\ 
 

    \theutterance \stepcounter{utterance}  

    & & & \multicolumn{2}{p{0.3\linewidth}}{\cellcolor[rgb]{0.95,0.95,0.95}{%
	\makecell[{{p{\linewidth}}}]{% 
	  \tt {\tiny [GM$|$GM]}  
	 Klingt gut. Mir ist vor allem eine Sorte wichtig, und von den anderen nehme ich gern den Großteil.\\ \tt \\ \tt Mein wichtigstes Stück ist nicht das Einzelstück – ich vermute, deins ist das Einzelstück (die Muschel). Vorschlag zur Aufteilung: Du bekommst die Muschel, ich bekomme die meisten der restlichen Sachen. Zum Beispiel: du nimmst die Muschel, ich nehme zwei der Schaufeln und ein Bild, und das verbleibende Bild und die letzte Schaufel gehen an dich. Passt dir das so grundsätzlich, oder willst du lieber alle restlichen Dinge außer einem kleinen Teil? Wenn du magst, formuliere ich das gleich als Vorschlag. 
	  } 
	   } 
	   } 
	 & & \\ 
 

    \theutterance \stepcounter{utterance}  

    & & \multicolumn{4}{p{0.6\linewidth}}{\cellcolor[rgb]{0.9,0.9,0.9}{%
	\makecell[{{p{\linewidth}}}]{% 
	  \tt {\tiny [A$\langle$GM]}  
	 Klingt gut. Mir ist vor allem eine Sorte wichtig, und von den anderen nehme ich gern den Großteil.\\ \tt \\ \tt Mein wichtigstes Stück ist nicht das Einzelstück – ich vermute, deins ist das Einzelstück (die Muschel). Vorschlag zur Aufteilung: Du bekommst die Muschel, ich bekomme die meisten der restlichen Sachen. Zum Beispiel: du nimmst die Muschel, ich nehme zwei der Schaufeln und ein Bild, und das verbleibende Bild und die letzte Schaufel gehen an dich. Passt dir das so grundsätzlich, oder willst du lieber alle restlichen Dinge außer einem kleinen Teil? Wenn du magst, formuliere ich das gleich als Vorschlag. 
	  } 
	   } 
	   } 
	 & \\ 
 

    \theutterance \stepcounter{utterance}  

    & \multicolumn{4}{p{0.6\linewidth}}{\cellcolor[rgb]{0.8,1,0.9}{%
	\makecell[{{p{\linewidth}}}]{% 
	  \tt {\tiny [A$\rangle$GM]}  
	 Das passt für mich sehr gut. Mir ist wirklich nur die Muschel wichtig; die restlichen Dinge kannst du gern größtenteils haben.\\ \tt \\ \tt Wenn es für dich okay ist, mache ich dir ein einfaches, klares Paket: Ich nehme die Muschel, und du bekommst alle restlichen Gegenstände (3 Schaufeln und 2 Bilder). Das maximiert vermutlich deinen Wert bei minimalem Konflikt.\\ \tt \\ \tt Wenn du lieber an deinem Aufteilungsvorschlag festhalten willst (ich: Muschel + 1 Schaufel + 1 Bild; du: 2 Schaufeln + 1 Bild), ist das für mich ebenfalls akzeptabel. Sag mir kurz, welche Variante du bevorzugst, dann machen wir es offiziell. 
	  } 
	   } 
	   } 
	 & & \\ 
 

    \theutterance \stepcounter{utterance}  

    & & & \multicolumn{2}{p{0.3\linewidth}}{\cellcolor[rgb]{0.95,0.95,0.95}{%
	\makecell[{{p{\linewidth}}}]{% 
	  \tt {\tiny [GM$|$GM]}  
	 continue 
	  } 
	   } 
	   } 
	 & & \\ 
 

    \theutterance \stepcounter{utterance}  

    & & & \multicolumn{2}{p{0.3\linewidth}}{\cellcolor[rgb]{0.95,0.95,0.95}{%
	\makecell[{{p{\linewidth}}}]{% 
	  \tt {\tiny [GM$|$GM]}  
	 Das passt für mich sehr gut. Mir ist wirklich nur die Muschel wichtig; die restlichen Dinge kannst du gern größtenteils haben.\\ \tt \\ \tt Wenn es für dich okay ist, mache ich dir ein einfaches, klares Paket: Ich nehme die Muschel, und du bekommst alle restlichen Gegenstände (3 Schaufeln und 2 Bilder). Das maximiert vermutlich deinen Wert bei minimalem Konflikt.\\ \tt \\ \tt Wenn du lieber an deinem Aufteilungsvorschlag festhalten willst (ich: Muschel + 1 Schaufel + 1 Bild; du: 2 Schaufeln + 1 Bild), ist das für mich ebenfalls akzeptabel. Sag mir kurz, welche Variante du bevorzugst, dann machen wir es offiziell. 
	  } 
	   } 
	   } 
	 & & \\ 
 

    \theutterance \stepcounter{utterance}  

    & & \multicolumn{4}{p{0.6\linewidth}}{\cellcolor[rgb]{0.9,0.9,0.9}{%
	\makecell[{{p{\linewidth}}}]{% 
	  \tt {\tiny [GM$\rangle$B]}  
	 Das passt für mich sehr gut. Mir ist wirklich nur die Muschel wichtig; die restlichen Dinge kannst du gern größtenteils haben.\\ \tt \\ \tt Wenn es für dich okay ist, mache ich dir ein einfaches, klares Paket: Ich nehme die Muschel, und du bekommst alle restlichen Gegenstände (3 Schaufeln und 2 Bilder). Das maximiert vermutlich deinen Wert bei minimalem Konflikt.\\ \tt \\ \tt Wenn du lieber an deinem Aufteilungsvorschlag festhalten willst (ich: Muschel + 1 Schaufel + 1 Bild; du: 2 Schaufeln + 1 Bild), ist das für mich ebenfalls akzeptabel. Sag mir kurz, welche Variante du bevorzugst, dann machen wir es offiziell. 
	  } 
	   } 
	   } 
	 & \\ 
 

    \theutterance \stepcounter{utterance}  

    & & & \multicolumn{4}{p{0.6\linewidth}}{\cellcolor[rgb]{1,0.85,0.72}{%
	\makecell[{{p{\linewidth}}}]{% 
	  \tt {\tiny [GM$\langle$B]}  
	 Dein erstes Paket ist für mich attraktiv und macht’s einfach: Du nimmst die Muschel, ich bekomme alle restlichen Gegenstände (3 Schaufeln, 2 Bilder). Lass uns das so festhalten.\\ \tt \\ \tt [Vorschlag: 3 Schaufel, 2 Bild] 
	  } 
	   } 
	   } 
	  \\ 
 

    \theutterance \stepcounter{utterance}  

    & & & \multicolumn{2}{p{0.3\linewidth}}{\cellcolor[rgb]{0.95,0.95,0.95}{%
	\makecell[{{p{\linewidth}}}]{% 
	  \tt {\tiny [GM$|$GM]}  
	 proposal 
	  } 
	   } 
	   } 
	 & & \\ 
 

    \theutterance \stepcounter{utterance}  

    & & & \multicolumn{2}{p{0.3\linewidth}}{\cellcolor[rgb]{0.95,0.95,0.95}{%
	\makecell[{{p{\linewidth}}}]{% 
	  \tt {\tiny [GM$|$GM]}  
	 [3, 2, 0] 
	  } 
	   } 
	   } 
	 & & \\ 
 

    \theutterance \stepcounter{utterance}  

    & & \multicolumn{4}{p{0.6\linewidth}}{\cellcolor[rgb]{0.9,0.9,0.9}{%
	\makecell[{{p{\linewidth}}}]{% 
	  \tt {\tiny [A$\langle$GM]}  
	 Der andere Spieler hat einen geheimen Vorschlag gemacht. Jetzt bist du an der Reihe, einen eigenen Vorschlag zu machen.\\ \tt  
	  } 
	   } 
	   } 
	 & \\ 
 

    \theutterance \stepcounter{utterance}  

    & \multicolumn{4}{p{0.6\linewidth}}{\cellcolor[rgb]{0.8,1,0.9}{%
	\makecell[{{p{\linewidth}}}]{% 
	  \tt {\tiny [A$\rangle$GM]}  
	 [Vorschlag: 1 Muschel] 
	  } 
	   } 
	   } 
	 & & \\ 
 

    \theutterance \stepcounter{utterance}  

    & & & \multicolumn{2}{p{0.3\linewidth}}{\cellcolor[rgb]{0.95,0.95,0.95}{%
	\makecell[{{p{\linewidth}}}]{% 
	  \tt {\tiny [GM$|$GM]}  
	 proposal 
	  } 
	   } 
	   } 
	 & & \\ 
 

    \theutterance \stepcounter{utterance}  

    & & & \multicolumn{2}{p{0.3\linewidth}}{\cellcolor[rgb]{0.95,0.95,0.95}{%
	\makecell[{{p{\linewidth}}}]{% 
	  \tt {\tiny [GM$|$GM]}  
	 [0, 0, 1] 
	  } 
	   } 
	   } 
	 & & \\ 
 

    \theutterance \stepcounter{utterance}  

    & & & \multicolumn{2}{p{0.3\linewidth}}{\cellcolor[rgb]{0.95,0.95,0.95}{%
	\makecell[{{p{\linewidth}}}]{% 
	  \tt {\tiny [GM$|$GM]}  
	 [[0, 0, 1], [3, 2, 0]] 
	  } 
	   } 
	   } 
	 & & \\ 
 

\end{supertabular}
}

\end{document}
