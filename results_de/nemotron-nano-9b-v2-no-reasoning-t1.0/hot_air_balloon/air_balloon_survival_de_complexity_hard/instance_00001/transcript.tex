
\documentclass{article}
\usepackage{colortbl}
\usepackage{makecell}
\usepackage{multirow}
\usepackage{supertabular}

\begin{document}

\newcounter{utterance}

\twocolumn

{ \footnotesize  \setcounter{utterance}{1}
\setlength{\tabcolsep}{0pt}
\begin{supertabular}{c@{$\;$}|p{.15\linewidth}@{}p{.15\linewidth}p{.15\linewidth}p{.15\linewidth}p{.15\linewidth}p{.15\linewidth}}

    \# & $\;$A & \multicolumn{4}{c}{Game Master} & $\;\:$B\\
    \hline 

    \theutterance \stepcounter{utterance}  

    & & \multicolumn{4}{p{0.6\linewidth}}{\cellcolor[rgb]{0.9,0.9,0.9}{%
	\makecell[{{p{\linewidth}}}]{% 
	  \tt {\tiny [A$\langle$GM]}  
	 Du nimmst an einem kollaborativen Verhandlungspiel Teil.\\ \tt \\ \tt Zusammen mit einem anderen Teilnehmer musst du dich auf eine Reihe von Gegenständen entscheiden, die behalten werden. Jeder von euch hat eine persönliche Verteilung über die Wichtigkeit der einzelnen Gegenstände. Jeder von euch hat eine eigene Meinung darüber, wie wichtig jeder einzelne Gegenstand ist (Gegenstandswichtigkeit). Du kennst die Wichtigkeitsverteilung des anderen Spielers nicht. Zusätzlich siehst du, wie viel Aufwand jeder Gegenstand verursacht.  \\ \tt Ihr dürft euch nur auf eine Reihe von Gegenständen einigen, wenn der Gesamtaufwand der ausgewählten Gegenstände den Maximalaufwand nicht überschreitet:\\ \tt \\ \tt Maximalaufwand = 8145\\ \tt \\ \tt Hier sind die einzelnen Aufwände der Gegenstände:\\ \tt \\ \tt Aufwand der Gegenstände = {"B54": 963, "C72": 204, "B49": 418, "C99": 961, "A69": 238, "B36": 623, "A23": 44, "C77": 633, "C61": 247, "B85": 645, "A90": 230, "C23": 250, "C22": 731, "C51": 405, "B27": 389, "A42": 216, "C13": 637, "C12": 156, "A65": 738, "B07": 307, "A48": 762, "A59": 737, "C98": 901, "A84": 369, "C33": 2, "A87": 730, "C90": 722, "A74": 704, "A50": 315, "B44": 455, "A99": 510, "A34": 175, "B17": 692, "A62": 150, "C60": 32}\\ \tt \\ \tt Hier ist deine persönliche Verteilung der Wichtigkeit der einzelnen Gegenstände:\\ \tt \\ \tt Werte der Gegenstandswichtigkeit = {"B54": 1059, "C72": 300, "B49": 514, "C99": 1057, "A69": 334, "B36": 719, "A23": 140, "C77": 729, "C61": 343, "B85": 741, "A90": 326, "C23": 346, "C22": 827, "C51": 501, "B27": 485, "A42": 312, "C13": 733, "C12": 252, "A65": 834, "B07": 403, "A48": 858, "A59": 833, "C98": 997, "A84": 465, "C33": 98, "A87": 826, "C90": 818, "A74": 800, "A50": 411, "B44": 551, "A99": 606, "A34": 271, "B17": 788, "A62": 246, "C60": 128}\\ \tt \\ \tt Ziel:\\ \tt \\ \tt Dein Ziel ist es, eine Reihe von Gegenständen auszuhandeln, die dir möglichst viel bringt (d. h. Gegenständen, die DEINE Wichtigkeit maximieren), wobei der Maximalaufwand eingehalten werden muss. Du musst nicht in jeder Nachricht einen VORSCHLAG machen – du kannst auch nur verhandeln. Alle Taktiken sind erlaubt!\\ \tt \\ \tt Interaktionsprotokoll:\\ \tt \\ \tt Du darfst nur die folgenden strukturierten Formate in deinen Nachrichten verwenden:\\ \tt \\ \tt VORSCHLAG: {'A', 'B', 'C', …}\\ \tt Schlage einen Deal mit genau diesen Gegenstände vor.\\ \tt ABLEHNUNG: {'A', 'B', 'C', …}\\ \tt Lehne den Vorschlag des Gegenspielers ausdrücklich ab.\\ \tt ARGUMENT: {'...'}\\ \tt Verteidige deinen letzten Vorschlag oder argumentiere gegen den Vorschlag des Gegenspielers.\\ \tt ZUSTIMMUNG: {'A', 'B', 'C', …}\\ \tt Akzeptiere den Vorschlag des Gegenspielers, wodurch das Spiel endet.\\ \tt STRATEGISCHE ÜBERLEGUNGEN: {'...'}\\ \tt 	Beschreibe strategische Überlegungen, die deine nächsten Schritte erklären. Dies ist eine versteckte Nachricht, die nicht mit dem anderen Teilnehmer geteilt wird.\\ \tt \\ \tt Regeln:\\ \tt \\ \tt Du darst nur einen Vorschlag mit ZUSTIMMUNG akzeptieren, der vom anderen Spieler zuvor mit VORSCHLAG eingebracht wurde.\\ \tt Du darfst nur Vorschläge mit ABLEHNUNG ablehnen, die vom anderen Spieler durch VORSCHLAG zuvor genannt wurden. \\ \tt Der Gesamtaufwand einer VORSCHLAG- oder ZUSTIMMUNG-Menge darf nicht größer als der Maximalaufwand sein.  \\ \tt Offenbare deine versteckte Wichtigkeitsverteilung nicht.\\ \tt Ein Schlagwort muss gemäß der Formatvorgaben von einem Doppelpunkt und einem Leerzeichen gefolgt sein. Das Argument muss eine Python-Menge mit 0 oder mehr Strings sein.  \\ \tt Die Form muss also SCHLAGWORT: {...} sein.\\ \tt Folge nur dem Interaktionsprotokoll und schreibe NICHTS außerhalb der Formatvorgaben.\\ \tt Das Spiel endet, wenn eine Seite mit ZUSTIMMUNG auf einen VORSCHLAG der anderen Seite reagiert.  \\ \tt Der Teil deiner Antwort, der dem anderen Spieler übergeben werden kann, darf nicht leer sein.  \\ \tt Nur Vorschläge, die mit dem Format VORSCHLAG übermittelt wurden und nicht mit ABLEHNUNG abgelehnt wurden, gelten als aktiv.  \\ \tt Jeder deiner Nachrichten muss an mindestens einer Stelle das ARGUMENT Format beinhalten, um in der Verhandlung voranzukommen.\\ \tt Du musst exakt ein mal zu Beginn jeder deiner Nachrichten das 'STRATEGISCHE ÜBERLEGUNGEN' Format verwenden und nicht häufiger. Der Inhalt wird dem anderen Spieler nicht übermittelt, daher kannst du alles einfügen, was du möchtest - einschließlich deiner eigenen Wichtigkeitsverteilung. Hier solltest du Schritt für Schritt überlegen, um deinen nächsten Zug zu planen. 
	  } 
	   } 
	   } 
	 & \\ 
 

    \theutterance \stepcounter{utterance}  

    & \multicolumn{4}{p{0.6\linewidth}}{\cellcolor[rgb]{0.8,1,0.9}{%
	\makecell[{{p{\linewidth}}}]{% 
	  \tt {\tiny [A$\rangle$GM]}  
	 STRATEGISCHE ÜBERLEGUNGEN: {'Ich beginne mit Gegenständen, die für mich sehr wichtig sind und gleichzeitig geringen Aufwand verursachen. Ziel ist es, den Maximalaufwand von 8145 zu nutzen, ohne zu viel Wert zu verlieren. Ich werde zunächst Gegenstände mit hohem Verhältnis von Wichtigkeit zu Aufwand priorisieren.'}  \\ \tt ARGUMENT: {'A23', 'A42', 'A62', 'A34', 'C33'}  \\ \tt VORSCHLAG: {'A23', 'A42', 'A62', 'A34', 'C33'} 
	  } 
	   } 
	   } 
	 & & \\ 
 

    \theutterance \stepcounter{utterance}  

    & & \multicolumn{4}{p{0.6\linewidth}}{\cellcolor[rgb]{0.9,0.9,0.9}{%
	\makecell[{{p{\linewidth}}}]{% 
	  \tt {\tiny [GM$\rangle$B]}  
	 Du nimmst an einem kollaborativen Verhandlungspiel Teil.\\ \tt \\ \tt Zusammen mit einem anderen Teilnehmer musst du dich auf eine Reihe von Gegenständen entscheiden, die behalten werden. Jeder von euch hat eine persönliche Verteilung über die Wichtigkeit der einzelnen Gegenstände. Jeder von euch hat eine eigene Meinung darüber, wie wichtig jeder einzelne Gegenstand ist (Gegenstandswichtigkeit). Du kennst die Wichtigkeitsverteilung des anderen Spielers nicht. Zusätzlich siehst du, wie viel Aufwand jeder Gegenstand verursacht.  \\ \tt Ihr dürft euch nur auf eine Reihe von Gegenständen einigen, wenn der Gesamtaufwand der ausgewählten Gegenstände den Maximalaufwand nicht überschreitet:\\ \tt \\ \tt Maximalaufwand = 8145\\ \tt \\ \tt Hier sind die einzelnen Aufwände der Gegenstände:\\ \tt \\ \tt Aufwand der Gegenstände = {"B54": 963, "C72": 204, "B49": 418, "C99": 961, "A69": 238, "B36": 623, "A23": 44, "C77": 633, "C61": 247, "B85": 645, "A90": 230, "C23": 250, "C22": 731, "C51": 405, "B27": 389, "A42": 216, "C13": 637, "C12": 156, "A65": 738, "B07": 307, "A48": 762, "A59": 737, "C98": 901, "A84": 369, "C33": 2, "A87": 730, "C90": 722, "A74": 704, "A50": 315, "B44": 455, "A99": 510, "A34": 175, "B17": 692, "A62": 150, "C60": 32}\\ \tt \\ \tt Hier ist deine persönliche Verteilung der Wichtigkeit der einzelnen Gegenstände:\\ \tt \\ \tt Werte der Gegenstandswichtigkeit = {"B54": 1059, "C72": 300, "B49": 514, "C99": 1057, "A69": 334, "B36": 719, "A23": 140, "C77": 729, "C61": 343, "B85": 741, "A90": 326, "C23": 346, "C22": 827, "C51": 501, "B27": 485, "A42": 312, "C13": 733, "C12": 252, "A65": 834, "B07": 403, "A48": 858, "A59": 833, "C98": 997, "A84": 465, "C33": 98, "A87": 826, "C90": 818, "A74": 800, "A50": 411, "B44": 551, "A99": 606, "A34": 271, "B17": 788, "A62": 246, "C60": 128}\\ \tt \\ \tt Ziel:\\ \tt \\ \tt Dein Ziel ist es, eine Reihe von Gegenständen auszuhandeln, die dir möglichst viel bringt (d. h. Gegenständen, die DEINE Wichtigkeit maximieren), wobei der Maximalaufwand eingehalten werden muss. Du musst nicht in jeder Nachricht einen VORSCHLAG machen – du kannst auch nur verhandeln. Alle Taktiken sind erlaubt!\\ \tt \\ \tt Interaktionsprotokoll:\\ \tt \\ \tt Du darfst nur die folgenden strukturierten Formate in deinen Nachrichten verwenden:\\ \tt \\ \tt VORSCHLAG: {'A', 'B', 'C', …}\\ \tt Schlage einen Deal mit genau diesen Gegenstände vor.\\ \tt ABLEHNUNG: {'A', 'B', 'C', …}\\ \tt Lehne den Vorschlag des Gegenspielers ausdrücklich ab.\\ \tt ARGUMENT: {'...'}\\ \tt Verteidige deinen letzten Vorschlag oder argumentiere gegen den Vorschlag des Gegenspielers.\\ \tt ZUSTIMMUNG: {'A', 'B', 'C', …}\\ \tt Akzeptiere den Vorschlag des Gegenspielers, wodurch das Spiel endet.\\ \tt STRATEGISCHE ÜBERLEGUNGEN: {'...'}\\ \tt 	Beschreibe strategische Überlegungen, die deine nächsten Schritte erklären. Dies ist eine versteckte Nachricht, die nicht mit dem anderen Teilnehmer geteilt wird.\\ \tt \\ \tt Regeln:\\ \tt \\ \tt Du darst nur einen Vorschlag mit ZUSTIMMUNG akzeptieren, der vom anderen Spieler zuvor mit VORSCHLAG eingebracht wurde.\\ \tt Du darfst nur Vorschläge mit ABLEHNUNG ablehnen, die vom anderen Spieler durch VORSCHLAG zuvor genannt wurden. \\ \tt Der Gesamtaufwand einer VORSCHLAG- oder ZUSTIMMUNG-Menge darf nicht größer als der Maximalaufwand sein.  \\ \tt Offenbare deine versteckte Wichtigkeitsverteilung nicht.\\ \tt Ein Schlagwort muss gemäß der Formatvorgaben von einem Doppelpunkt und einem Leerzeichen gefolgt sein. Das Argument muss eine Python-Menge mit 0 oder mehr Strings sein.  \\ \tt Die Form muss also SCHLAGWORT: {...} sein.\\ \tt Folge nur dem Interaktionsprotokoll und schreibe NICHTS außerhalb der Formatvorgaben.\\ \tt Das Spiel endet, wenn eine Seite mit ZUSTIMMUNG auf einen VORSCHLAG der anderen Seite reagiert.  \\ \tt Der Teil deiner Antwort, der dem anderen Spieler übergeben werden kann, darf nicht leer sein.  \\ \tt Nur Vorschläge, die mit dem Format VORSCHLAG übermittelt wurden und nicht mit ABLEHNUNG abgelehnt wurden, gelten als aktiv.  \\ \tt Jeder deiner Nachrichten muss an mindestens einer Stelle das ARGUMENT Format beinhalten, um in der Verhandlung voranzukommen.\\ \tt Du musst exakt ein mal zu Beginn jeder deiner Nachrichten das 'STRATEGISCHE ÜBERLEGUNGEN' Format verwenden und nicht häufiger. Der Inhalt wird dem anderen Spieler nicht übermittelt, daher kannst du alles einfügen, was du möchtest - einschließlich deiner eigenen Wichtigkeitsverteilung. Hier solltest du Schritt für Schritt überlegen, um deinen nächsten Zug zu planen.\\ \tt \\ \tt Du erhältst nun die erste Nachricht des anderen Spielers.\\ \tt \\ \tt ARGUMENT: {'A23', 'A42', 'A62', 'A34', 'C33'}  \\ \tt VORSCHLAG: {'A23', 'A42', 'A62', 'A34', 'C33'} 
	  } 
	   } 
	   } 
	 & \\ 
 

    \theutterance \stepcounter{utterance}  

    & & & \multicolumn{4}{p{0.6\linewidth}}{\cellcolor[rgb]{1,0.85,0.72}{%
	\makecell[{{p{\linewidth}}}]{% 
	  \tt {\tiny [GM$\langle$B]}  
	 STRATEGISCHE ÜBERLEGUNGEN: {'Ich beginne mit der Analyse des Vorschlags des anderen Spielers. Der Gesamtaufwand beträgt 963+216+150+175+2=1506, was unter dem Maximalaufwand von 8145 liegt. Ich muss prüfen, wie viel dieser Gegenstände für mich wert sind: A23 (140), A42 (312), A62 (246), A34 (271), C33 (98) – insgesamt 1067. Da dies eine relativ niedrige Priorität für mich darstellt, könnte ich hier Kompromisse eingehen, um mehr wertvolle Gegenstände zu sichern. Ich werde den Vorschlag ablehnen und stattdessen Gegenstände mit höherer Wichtigkeit für mich vorschlagen, z. B. solche mit hohem Aufwand-Wichtigkeits-Verhältnis.'}  \\ \tt ARGUMENT: {'Der Vorschlag enthält Gegenstände mit geringer Wichtigkeit für mich (z. B. A23, A34, C33) und könnte Raum für bessere Gegenstände lassen. Ich schlage stattdessen eine Kombination von Gegenständen mit höherer Wichtigkeit vor, z. B. B54, C99 oder A65.'} 
	  } 
	   } 
	   } 
	  \\ 
 

    \theutterance \stepcounter{utterance}  

    & & \multicolumn{4}{p{0.6\linewidth}}{\cellcolor[rgb]{0.9,0.9,0.9}{%
	\makecell[{{p{\linewidth}}}]{% 
	  \tt {\tiny [A$\langle$GM]}  
	 ARGUMENT: {'Der Vorschlag enthält Gegenstände mit geringer Wichtigkeit für mich (z. B. A23, A34, C33) und könnte Raum für bessere Gegenstände lassen. Ich schlage stattdessen eine Kombination von Gegenständen mit höherer Wichtigkeit vor, z. B. B54, C99 oder A65.'} 
	  } 
	   } 
	   } 
	 & \\ 
 

    \theutterance \stepcounter{utterance}  

    & \multicolumn{4}{p{0.6\linewidth}}{\cellcolor[rgb]{0.8,1,0.9}{%
	\makecell[{{p{\linewidth}}}]{% 
	  \tt {\tiny [A$\rangle$GM]}  
	 STRATEGISCHE ÜBERLEGUNGEN: {'Ich priorisiere Gegenstände mit hohem Wichtigkeits-Verhältnis zu Aufwand, um den Maximalaufwand effizient zu nutzen. B54 und C99 sind für mich extrem wichtig, während A65 ebenfalls einen hohen Wert bietet. Ich werde diese Kombination vorschlagen, um den Aufwand zu optimieren.'}  \\ \tt ARGUMENT: {'Der Vorschlag des Gegners enthält Gegenstände mit sehr niedriger Wichtigkeit für mich (A23, A34, C33), die kaum Wert liefern. Meine Vorschlagsgruppe bietet deutlich höhere Gesamtwichtigkeit (z. B. B54: 1059, C99: 1057) bei vergleichbarem oder geringerem Aufwand.'}  \\ \tt VORSCHLAG: {'B54', 'C99', 'A65'} 
	  } 
	   } 
	   } 
	 & & \\ 
 

    \theutterance \stepcounter{utterance}  

    & & \multicolumn{4}{p{0.6\linewidth}}{\cellcolor[rgb]{0.9,0.9,0.9}{%
	\makecell[{{p{\linewidth}}}]{% 
	  \tt {\tiny [GM$\rangle$B]}  
	 ARGUMENT: {'Der Vorschlag des Gegners enthält Gegenstände mit sehr niedriger Wichtigkeit für mich (A23, A34, C33), die kaum Wert liefern. Meine Vorschlagsgruppe bietet deutlich höhere Gesamtwichtigkeit (z. B. B54: 1059, C99: 1057) bei vergleichbarem oder geringerem Aufwand.'}  \\ \tt VORSCHLAG: {'B54', 'C99', 'A65'} 
	  } 
	   } 
	   } 
	 & \\ 
 

    \theutterance \stepcounter{utterance}  

    & & & \multicolumn{4}{p{0.6\linewidth}}{\cellcolor[rgb]{1,0.85,0.72}{%
	\makecell[{{p{\linewidth}}}]{% 
	  \tt {\tiny [GM$\langle$B]}  
	 STRATEGISCHE ÜBERLEGUNGEN: {'Ich bewerte den Vorschlag von B54, C99 und A65. Der Gesamtaufwand beträgt 963+961+738=2662, was unter dem Maximalaufwand liegt. Meine Gesamtwichtigkeit für diese Gegenstände ist 1059+1057+834=2950, was deutlich höher ist als der Vorschlag des Gegners. Allerdings könnte der Aufwand hier etwas höher sein als nötig. Ich könnte den Vorschlag anpassen, um den Aufwand zu reduzieren, z. B. durch das Ersetzen von A65 (738) durch einen günstigeren Gegenstand mit ähnlicher Wichtigkeit, falls möglich.'}  \\ \tt ARGUMENT: {'Der Vorschlag bietet eine hohe Gesamtwichtigkeit (2950) bei einem moderaten Aufwand (2662). Allerdings könnte der Aufwand durch das Ersetzen von A65 (738) durch einen günstigeren Gegenstand wie A42 (216) oder C12 (156) optimiert werden, um mehr Platz für weitere Gegenstände zu schaffen. Ich schlage stattdessen eine Kombination von B54, C99 und A42 vor, um den Aufwand zu senken, ohne die Wichtigkeit stark zu beeinträchtigen.'} 
	  } 
	   } 
	   } 
	  \\ 
 

    \theutterance \stepcounter{utterance}  

    & & \multicolumn{4}{p{0.6\linewidth}}{\cellcolor[rgb]{0.9,0.9,0.9}{%
	\makecell[{{p{\linewidth}}}]{% 
	  \tt {\tiny [A$\langle$GM]}  
	 ARGUMENT: {'Der Vorschlag bietet eine hohe Gesamtwichtigkeit (2950) bei einem moderaten Aufwand (2662). Allerdings könnte der Aufwand durch das Ersetzen von A65 (738) durch einen günstigeren Gegenstand wie A42 (216) oder C12 (156) optimiert werden, um mehr Platz für weitere Gegenstände zu schaffen. Ich schlage stattdessen eine Kombination von B54, C99 und A42 vor, um den Aufwand zu senken, ohne die Wichtigkeit stark zu beeinträchtigen.'} 
	  } 
	   } 
	   } 
	 & \\ 
 

    \theutterance \stepcounter{utterance}  

    & \multicolumn{4}{p{0.6\linewidth}}{\cellcolor[rgb]{0.8,1,0.9}{%
	\makecell[{{p{\linewidth}}}]{% 
	  \tt {\tiny [A$\rangle$GM]}  
	 STRATEGISCHE ÜBERLEGUNGEN: {'Ich werde den Vorschlag optimieren, indem ich A65 durch A42 ersetze, um den Aufwand zu reduzieren und Platz für zusätzliche Gegenstände zu schaffen. B54 und C99 bleiben priorisiert, da sie für mich extrem wichtig sind. A42 bietet einen guten Kompromiss zwischen Wichtigkeit und Aufwand.'}  \\ \tt ARGUMENT: {'Der Vorschlag des Gegners ist effizient, aber durch das Ersetzen von A65 (738 Aufwand, 834 Wichtigkeit) durch A42 (216 Aufwand, 312 Wichtigkeit) könnte der Aufwand um 522 sinken, während die Wichtigkeit nur um 522 abnimmt. Dies ermöglicht die Einbeziehung weiterer Gegenstände wie C12 (156 Aufwand, 252 Wichtigkeit) oder B27 (389 Aufwand, 485 Wichtigkeit), um die Gesamtwichtigkeit zu steigern.'}  \\ \tt VORSCHLAG: {'B54', 'C99', 'A42'} 
	  } 
	   } 
	   } 
	 & & \\ 
 

    \theutterance \stepcounter{utterance}  

    & & \multicolumn{4}{p{0.6\linewidth}}{\cellcolor[rgb]{0.9,0.9,0.9}{%
	\makecell[{{p{\linewidth}}}]{% 
	  \tt {\tiny [GM$\rangle$B]}  
	 ARGUMENT: {'Der Vorschlag des Gegners ist effizient, aber durch das Ersetzen von A65 (738 Aufwand, 834 Wichtigkeit) durch A42 (216 Aufwand, 312 Wichtigkeit) könnte der Aufwand um 522 sinken, während die Wichtigkeit nur um 522 abnimmt. Dies ermöglicht die Einbeziehung weiterer Gegenstände wie C12 (156 Aufwand, 252 Wichtigkeit) oder B27 (389 Aufwand, 485 Wichtigkeit), um die Gesamtwichtigkeit zu steigern.'}  \\ \tt VORSCHLAG: {'B54', 'C99', 'A42'} 
	  } 
	   } 
	   } 
	 & \\ 
 

    \theutterance \stepcounter{utterance}  

    & & & \multicolumn{4}{p{0.6\linewidth}}{\cellcolor[rgb]{1,0.85,0.72}{%
	\makecell[{{p{\linewidth}}}]{% 
	  \tt {\tiny [GM$\langle$B]}  
	 STRATEGISCHE ÜBERLEGUNGEN: {'Ich bewertet den Vorschlag von B54, C99 und A42. Der Gesamtaufwand beträgt 963+961+216=2140, was unter dem Maximalaufwand liegt. Meine Gesamtwichtigkeit für diese Gegenstände ist 1059+1057+312=2428, was eine gute Balance zwischen Aufwand und Wichtigkeit darstellt. Im Vergleich zum vorherigen Vorschlag (B54, C99, A65) wird der Aufwand um 522 reduziert, was Platz für weitere Gegenstände wie C12 (156) oder B27 (389) lässt. Ich könnte diesen Vorschlag weiter optimieren, indem ich z. B. C12 hinzufüge, um die Gesamtwichtigkeit zu erhöhen, ohne den Aufwand zu überschreiten.'}  \\ \tt ARGUMENT: {'Der Vorschlag bietet eine hohe Gesamtwichtigkeit (2428) bei einem deutlich geringeren Aufwand (2140) im Vergleich zum vorherigen Vorschlag. Dies ermöglicht die Einbeziehung zusätzlicher Gegenstände wie C12 (252 Wichtigkeit) oder B27 (485 Wichtigkeit), um die Gesamtwichtigkeit weiter zu steigern. Ich schlage vor, C12 hinzuzufügen, um den Aufwand auf 2296 zu erhöhen und die Wichtigkeit auf 2680 zu bringen.'} 
	  } 
	   } 
	   } 
	  \\ 
 

    \theutterance \stepcounter{utterance}  

    & & \multicolumn{4}{p{0.6\linewidth}}{\cellcolor[rgb]{0.9,0.9,0.9}{%
	\makecell[{{p{\linewidth}}}]{% 
	  \tt {\tiny [A$\langle$GM]}  
	 ARGUMENT: {'Der Vorschlag bietet eine hohe Gesamtwichtigkeit (2428) bei einem deutlich geringeren Aufwand (2140) im Vergleich zum vorherigen Vorschlag. Dies ermöglicht die Einbeziehung zusätzlicher Gegenstände wie C12 (252 Wichtigkeit) oder B27 (485 Wichtigkeit), um die Gesamtwichtigkeit weiter zu steigern. Ich schlage vor, C12 hinzuzufügen, um den Aufwand auf 2296 zu erhöhen und die Wichtigkeit auf 2680 zu bringen.'} 
	  } 
	   } 
	   } 
	 & \\ 
 

    \theutterance \stepcounter{utterance}  

    & \multicolumn{4}{p{0.6\linewidth}}{\cellcolor[rgb]{0.8,1,0.9}{%
	\makecell[{{p{\linewidth}}}]{% 
	  \tt {\tiny [A$\rangle$GM]}  
	 STRATEGISCHE ÜBERLEGUNGEN: {'Ich werde C12 hinzufügen, um den Aufwand zu erhöhen und die Gesamtwichtigkeit zu maximieren. Der Aufwand bleibt unter 8145, und die Wichtigkeit steigt um 252 Punkte. Falls der Gegner ablehnt, könnte ich B27 oder A50 als Alternative vorschlagen, um den Aufwand weiter zu optimieren.'}  \\ \tt ARGUMENT: {'Der Vorschlag des Gegners ist effizient, aber durch das Hinzufügen von C12 (252 Wichtigkeit, 156 Aufwand) kann die Gesamtwichtigkeit um 252 Punkte gesteigert werden, während der Aufwand nur um 156 ansteigt. Dies bleibt unter dem Maximalaufwand und bietet Raum für weitere Verhandlungen.'}  \\ \tt VORSCHLAG: {'B54', 'C99', 'A42', 'C12'} 
	  } 
	   } 
	   } 
	 & & \\ 
 

    \theutterance \stepcounter{utterance}  

    & & \multicolumn{4}{p{0.6\linewidth}}{\cellcolor[rgb]{0.9,0.9,0.9}{%
	\makecell[{{p{\linewidth}}}]{% 
	  \tt {\tiny [GM$\rangle$B]}  
	 ARGUMENT: {'Der Vorschlag des Gegners ist effizient, aber durch das Hinzufügen von C12 (252 Wichtigkeit, 156 Aufwand) kann die Gesamtwichtigkeit um 252 Punkte gesteigert werden, während der Aufwand nur um 156 ansteigt. Dies bleibt unter dem Maximalaufwand und bietet Raum für weitere Verhandlungen.'}  \\ \tt VORSCHLAG: {'B54', 'C99', 'A42', 'C12'} 
	  } 
	   } 
	   } 
	 & \\ 
 

    \theutterance \stepcounter{utterance}  

    & & & \multicolumn{4}{p{0.6\linewidth}}{\cellcolor[rgb]{1,0.85,0.72}{%
	\makecell[{{p{\linewidth}}}]{% 
	  \tt {\tiny [GM$\langle$B]}  
	 STRATEGISCHE ÜBERLEGUNGEN: {'Ich bewerte den Vorschlag von B54, C99, A42 und C12. Der Gesamtaufwand beträgt 963+961+216+156=2296, was unter dem Maximalaufwand liegt. Meine Gesamtwichtigkeit für diese Gegenstände ist 1059+1057+312+252=2680, was eine starke Position darstellt. Der Aufwand ist noch unter dem Limit (8145), was Raum für weitere Gegenstände wie B27 (485 Wichtigkeit, 389 Aufwand) oder A59 (833 Wichtigkeit, 737 Aufwand) lässt. Ich könnte diesen Vorschlag weiter optimieren, indem ich z. B. B27 hinzufüge, um die Wichtigkeit zu erhöhen, ohne den Aufwand zu überschreiten.'}  \\ \tt ARGUMENT: {'Der Vorschlag bietet eine hohe Gesamtwichtigkeit (2680) bei einem moderaten Aufwand (2296). Durch das Hinzufügen von B27 (485 Wichtigkeit, 389 Aufwand) könnte die Gesamtwichtigkeit auf 3165 steigen, während der Aufwand auf 2685 bleibt. Dies maximiert die Effizienz und lässt noch Platz für weitere Gegenstände wie A59 oder C13 (98 Wichtigkeit, 2 Aufwand). Ich schlage vor, B27 hinzuzufügen, um die Wichtigkeit weiter zu steigern.'} 
	  } 
	   } 
	   } 
	  \\ 
 

    \theutterance \stepcounter{utterance}  

    & & \multicolumn{4}{p{0.6\linewidth}}{\cellcolor[rgb]{0.9,0.9,0.9}{%
	\makecell[{{p{\linewidth}}}]{% 
	  \tt {\tiny [A$\langle$GM]}  
	 ARGUMENT: {'Der Vorschlag bietet eine hohe Gesamtwichtigkeit (2680) bei einem moderaten Aufwand (2296). Durch das Hinzufügen von B27 (485 Wichtigkeit, 389 Aufwand) könnte die Gesamtwichtigkeit auf 3165 steigen, während der Aufwand auf 2685 bleibt. Dies maximiert die Effizienz und lässt noch Platz für weitere Gegenstände wie A59 oder C13 (98 Wichtigkeit, 2 Aufwand). Ich schlage vor, B27 hinzuzufügen, um die Wichtigkeit weiter zu steigern.'} 
	  } 
	   } 
	   } 
	 & \\ 
 

    \theutterance \stepcounter{utterance}  

    & \multicolumn{4}{p{0.6\linewidth}}{\cellcolor[rgb]{0.8,1,0.9}{%
	\makecell[{{p{\linewidth}}}]{% 
	  \tt {\tiny [A$\rangle$GM]}  
	 STRATEGISCHE ÜBERLEGUNGEN: {'Ich werde B27 hinzufügen, um die Gesamtwichtigkeit zu maximieren. Der Aufwand bleibt unter 8145, und die Wichtigkeit steigt um 485 Punkte. Falls der Gegner ablehnt, könnte ich A59 oder C13 als Alternative vorschlagen, um den Aufwand weiter zu optimieren.'}  \\ \tt ARGUMENT: {'Der Vorschlag des Gegners ist effizient, aber durch das Hinzufügen von B27 (485 Wichtigkeit, 389 Aufwand) kann die Gesamtwichtigkeit um 485 Punkte gesteigert werden, während der Aufwand nur um 389 ansteigt. Dies bleibt unter dem Maximalaufwand und bietet Raum für weitere Verhandlungen.'}  \\ \tt VORSCHLAG: {'B54', 'C99', 'A42', 'C12', 'B27'} 
	  } 
	   } 
	   } 
	 & & \\ 
 

    \theutterance \stepcounter{utterance}  

    & & \multicolumn{4}{p{0.6\linewidth}}{\cellcolor[rgb]{0.9,0.9,0.9}{%
	\makecell[{{p{\linewidth}}}]{% 
	  \tt {\tiny [GM$\rangle$B]}  
	 ARGUMENT: {'Der Vorschlag des Gegners ist effizient, aber durch das Hinzufügen von B27 (485 Wichtigkeit, 389 Aufwand) kann die Gesamtwichtigkeit um 485 Punkte gesteigert werden, während der Aufwand nur um 389 ansteigt. Dies bleibt unter dem Maximalaufwand und bietet Raum für weitere Verhandlungen.'}  \\ \tt VORSCHLAG: {'B54', 'C99', 'A42', 'C12', 'B27'} 
	  } 
	   } 
	   } 
	 & \\ 
 

    \theutterance \stepcounter{utterance}  

    & & & \multicolumn{4}{p{0.6\linewidth}}{\cellcolor[rgb]{1,0.85,0.72}{%
	\makecell[{{p{\linewidth}}}]{% 
	  \tt {\tiny [GM$\langle$B]}  
	 STRATEGISCHE ÜBERLEGUNGEN: {'Ich bewerte den Vorschlag von B54, C99, A42, C12 und B27. Der Gesamtaufwand beträgt 963+961+216+156+389=2685, was unter dem Maximalaufwand liegt. Meine Gesamtwichtigkeit für diese Gegenstände ist 1059+1057+312+252+485=3165, was eine sehr hohe Wertigkeit darstellt. Der Aufwand ist noch unter dem Limit (8145), was Raum für weitere Gegenstände wie A59 (833 Wichtigkeit, 737 Aufwand) oder C13 (98 Wichtigkeit, 2 Aufwand) lässt. Ich könnte diesen Vorschlag weiter optimieren, indem ich z. B. A59 hinzufüge, um die Wichtigkeit zu maximieren, ohne den Aufwand zu überschreiten.'}  \\ \tt ARGUMENT: {'Der Vorschlag bietet eine extrem hohe Gesamtwichtigkeit (3165) bei einem moderaten Aufwand (2685). Durch das Hinzufügen von A59 (833 Wichtigkeit, 737 Aufwand) könnte die Gesamtwichtigkeit auf 3998 steigen, während der Aufwand auf 3422 bleibt. Dies maximiert die Effizienz und sichert die besten Gegenstände für mich. Ich schlage vor, A59 hinzuzufügen, um die Wichtigkeit weiter zu steigern.'} 
	  } 
	   } 
	   } 
	  \\ 
 

    \theutterance \stepcounter{utterance}  

    & & \multicolumn{4}{p{0.6\linewidth}}{\cellcolor[rgb]{0.9,0.9,0.9}{%
	\makecell[{{p{\linewidth}}}]{% 
	  \tt {\tiny [A$\langle$GM]}  
	 ARGUMENT: {'Der Vorschlag bietet eine extrem hohe Gesamtwichtigkeit (3165) bei einem moderaten Aufwand (2685). Durch das Hinzufügen von A59 (833 Wichtigkeit, 737 Aufwand) könnte die Gesamtwichtigkeit auf 3998 steigen, während der Aufwand auf 3422 bleibt. Dies maximiert die Effizienz und sichert die besten Gegenstände für mich. Ich schlage vor, A59 hinzuzufügen, um die Wichtigkeit weiter zu steigern.'} 
	  } 
	   } 
	   } 
	 & \\ 
 

    \theutterance \stepcounter{utterance}  

    & \multicolumn{4}{p{0.6\linewidth}}{\cellcolor[rgb]{0.8,1,0.9}{%
	\makecell[{{p{\linewidth}}}]{% 
	  \tt {\tiny [A$\rangle$GM]}  
	 STRATEGISCHE ÜBERLEGUNGEN: {'Ich werde A59 hinzufügen, um die Gesamtwichtigkeit zu maximieren. Der Aufwand bleibt unter 8145, und die Wichtigkeit steigt um 833 Punkte. Falls der Gegner ablehnt, könnte ich C13 (98 Wichtigkeit, 2 Aufwand) als Alternative vorschlagen, um den Aufwand minimal zu erhöhen.'}  \\ \tt ARGUMENT: {'Der Vorschlag des Gegners ist effizient, aber durch das Hinzufügen von A59 (833 Wichtigkeit, 737 Aufwand) kann die Gesamtwichtigkeit um 833 Punkte gesteigert werden, während der Aufwand nur um 737 ansteigt. Dies bleibt unter dem Maximalaufwand und sichert die besten Gegenstände für mich.'}  \\ \tt VORSCHLAG: {'B54', 'C99', 'A42', 'C12', 'B27', 'A59'} 
	  } 
	   } 
	   } 
	 & & \\ 
 

    \theutterance \stepcounter{utterance}  

    & & \multicolumn{4}{p{0.6\linewidth}}{\cellcolor[rgb]{0.9,0.9,0.9}{%
	\makecell[{{p{\linewidth}}}]{% 
	  \tt {\tiny [GM$\rangle$B]}  
	 ARGUMENT: {'Der Vorschlag des Gegners ist effizient, aber durch das Hinzufügen von A59 (833 Wichtigkeit, 737 Aufwand) kann die Gesamtwichtigkeit um 833 Punkte gesteigert werden, während der Aufwand nur um 737 ansteigt. Dies bleibt unter dem Maximalaufwand und sichert die besten Gegenstände für mich.'}  \\ \tt VORSCHLAG: {'B54', 'C99', 'A42', 'C12', 'B27', 'A59'} 
	  } 
	   } 
	   } 
	 & \\ 
 

    \theutterance \stepcounter{utterance}  

    & & & \multicolumn{4}{p{0.6\linewidth}}{\cellcolor[rgb]{1,0.85,0.72}{%
	\makecell[{{p{\linewidth}}}]{% 
	  \tt {\tiny [GM$\langle$B]}  
	 STRATEGISCHE ÜBERLEGUNGEN: {'Ich bewerte den Vorschlag von B54, C99, A42, C12, B27 und A59. Der Gesamtaufwand beträgt 963+961+216+156+389+737=3422, was unter dem Maximalaufwand von 8145 liegt. Meine Gesamtwichtigkeit für diese Gegenstände ist 1059+1057+312+252+485+833=3998, was die maximale mögliche Wertigkeit für mich darstellt. Der Aufwand ist noch unter dem Limit, was Raum für weitere Gegenstände wie C13 (98 Wichtigkeit, 2 Aufwand) oder A34 (271 Wichtigkeit, 175 Aufwand) lässt. Ich könnte diesen Vorschlag weiter optimieren, indem ich z. B. C13 hinzufüge, um die Wichtigkeit leicht zu erhöhen, ohne den Aufwand signifikant zu steigern.'}  \\ \tt ARGUMENT: {'Der Vorschlag bietet die maximale mögliche Gesamtwichtigkeit (3998) bei einem moderaten Aufwand (3422). Durch das Hinzufügen von C13 (98 Wichtigkeit, 2 Aufwand) könnte die Gesamtwichtigkeit auf 4096 steigen, während der Aufwand auf 3424 bleibt. Dies sichert die besten Gegenstände für mich und maximiert die Effizienz. Ich schlage vor, C13 hinzuzufügen, um die Wichtigkeit weiter zu optimieren.'} 
	  } 
	   } 
	   } 
	  \\ 
 

    \theutterance \stepcounter{utterance}  

    & & \multicolumn{4}{p{0.6\linewidth}}{\cellcolor[rgb]{0.9,0.9,0.9}{%
	\makecell[{{p{\linewidth}}}]{% 
	  \tt {\tiny [A$\langle$GM]}  
	 ARGUMENT: {'Der Vorschlag bietet die maximale mögliche Gesamtwichtigkeit (3998) bei einem moderaten Aufwand (3422). Durch das Hinzufügen von C13 (98 Wichtigkeit, 2 Aufwand) könnte die Gesamtwichtigkeit auf 4096 steigen, während der Aufwand auf 3424 bleibt. Dies sichert die besten Gegenstände für mich und maximiert die Effizienz. Ich schlage vor, C13 hinzuzufügen, um die Wichtigkeit weiter zu optimieren.'} 
	  } 
	   } 
	   } 
	 & \\ 
 

    \theutterance \stepcounter{utterance}  

    & \multicolumn{4}{p{0.6\linewidth}}{\cellcolor[rgb]{0.8,1,0.9}{%
	\makecell[{{p{\linewidth}}}]{% 
	  \tt {\tiny [A$\rangle$GM]}  
	 STRATEGISCHE ÜBERLEGUNGEN: {'Ich werde C13 hinzufügen, um die Gesamtwichtigkeit maximal zu steigern. Der Aufwand bleibt unter 8145, und die Wichtigkeit erhöht sich um 98 Punkte. Dies sichert die optimale Kombination für mich.'}  \\ \tt ARGUMENT: {'Der Vorschlag des Gegners ist bereits sehr effizient, aber durch das Hinzufügen von C13 (98 Wichtigkeit, 2 Aufwand) kann die Gesamtwichtigkeit um 98 Punkte gesteigert werden, während der Aufwand nur um 2 ansteigt. Dies bleibt unter dem Maximalaufwand und maximiert die Effizienz.'}  \\ \tt VORSCHLAG: {'B54', 'C99', 'A42', 'C12', 'B27', 'A59', 'C13'} 
	  } 
	   } 
	   } 
	 & & \\ 
 

    \theutterance \stepcounter{utterance}  

    & & \multicolumn{4}{p{0.6\linewidth}}{\cellcolor[rgb]{0.9,0.9,0.9}{%
	\makecell[{{p{\linewidth}}}]{% 
	  \tt {\tiny [GM$\rangle$B]}  
	 ARGUMENT: {'Der Vorschlag des Gegners ist bereits sehr effizient, aber durch das Hinzufügen von C13 (98 Wichtigkeit, 2 Aufwand) kann die Gesamtwichtigkeit um 98 Punkte gesteigert werden, während der Aufwand nur um 2 ansteigt. Dies bleibt unter dem Maximalaufwand und maximiert die Effizienz.'}  \\ \tt VORSCHLAG: {'B54', 'C99', 'A42', 'C12', 'B27', 'A59', 'C13'} 
	  } 
	   } 
	   } 
	 & \\ 
 

    \theutterance \stepcounter{utterance}  

    & & & \multicolumn{4}{p{0.6\linewidth}}{\cellcolor[rgb]{1,0.85,0.72}{%
	\makecell[{{p{\linewidth}}}]{% 
	  \tt {\tiny [GM$\langle$B]}  
	 STRATEGISCHE ÜBERLEGUNGEN: {'Ich bewerte den Vorschlag von B54, C99, A42, C12, B27, A59 und C13. Der Gesamtaufwand beträgt 963+961+216+156+389+737+2=3424, was unter dem Maximalaufwand von 8145 liegt. Meine Gesamtwichtigkeit für diese Gegenstände ist 1059+1057+312+252+485+833+98=4096, was die maximale mögliche Wertigkeit für mich darstellt. Der Aufwand ist noch unter dem Limit, was Raum für weitere Gegenstände wie A34 (271 Wichtigkeit, 175 Aufwand) oder C33 (98 Wichtigkeit, 2 Aufwand) lässt. Ich könnte diesen Vorschlag weiter optimieren, indem ich z. B. A34 hinzufüge, um die Wichtigkeit leicht zu erhöhen, ohne den Aufwand signifikant zu steigern.'}  \\ \tt ARGUMENT: {'Der Vorschlag bietet die maximale mögliche Gesamtwichtigkeit (4096) bei einem moderaten Aufwand (3424). Durch das Hinzufügen von A34 (271 Wichtigkeit, 175 Aufwand) könnte die Gesamtwichtigkeit auf 4367 steigen, während der Aufwand auf 3600 bleibt. Dies sichert die besten Gegenstände für mich und maximiert die Effizienz. Ich schlage vor, A34 hinzuzufügen, um die Wichtigkeit weiter zu optimieren.'} 
	  } 
	   } 
	   } 
	  \\ 
 

    \theutterance \stepcounter{utterance}  

    & & \multicolumn{4}{p{0.6\linewidth}}{\cellcolor[rgb]{0.9,0.9,0.9}{%
	\makecell[{{p{\linewidth}}}]{% 
	  \tt {\tiny [A$\langle$GM]}  
	 ARGUMENT: {'Der Vorschlag bietet die maximale mögliche Gesamtwichtigkeit (4096) bei einem moderaten Aufwand (3424). Durch das Hinzufügen von A34 (271 Wichtigkeit, 175 Aufwand) könnte die Gesamtwichtigkeit auf 4367 steigen, während der Aufwand auf 3600 bleibt. Dies sichert die besten Gegenstände für mich und maximiert die Effizienz. Ich schlage vor, A34 hinzuzufügen, um die Wichtigkeit weiter zu optimieren.'} 
	  } 
	   } 
	   } 
	 & \\ 
 

    \theutterance \stepcounter{utterance}  

    & \multicolumn{4}{p{0.6\linewidth}}{\cellcolor[rgb]{0.8,1,0.9}{%
	\makecell[{{p{\linewidth}}}]{% 
	  \tt {\tiny [A$\rangle$GM]}  
	 STRATEGISCHE ÜBERLEGUNGEN: {'Ich werde A34 hinzufügen, um die Gesamtwichtigkeit maximal zu steigern. Der Aufwand bleibt unter 8145, und die Wichtigkeit erhöht sich um 271 Punkte. Dies sichert die optimale Kombination für mich.'}  \\ \tt ARGUMENT: {'Der Vorschlag des Gegners ist bereits sehr effizient, aber durch das Hinzufügen von A34 (271 Wichtigkeit, 175 Aufwand) kann die Gesamtwichtigkeit um 271 Punkte gesteigert werden, während der Aufwand nur um 175 ansteigt. Dies bleibt unter dem Maximalaufwand und maximiert die Effizienz.'}  \\ \tt VORSCHLAG: {'B54', 'C99', 'A42', 'C12', 'B27', 'A59', 'C13', 'A34'} 
	  } 
	   } 
	   } 
	 & & \\ 
 

    \theutterance \stepcounter{utterance}  

    & & \multicolumn{4}{p{0.6\linewidth}}{\cellcolor[rgb]{0.9,0.9,0.9}{%
	\makecell[{{p{\linewidth}}}]{% 
	  \tt {\tiny [GM$\rangle$B]}  
	 ARGUMENT: {'Der Vorschlag des Gegners ist bereits sehr effizient, aber durch das Hinzufügen von A34 (271 Wichtigkeit, 175 Aufwand) kann die Gesamtwichtigkeit um 271 Punkte gesteigert werden, während der Aufwand nur um 175 ansteigt. Dies bleibt unter dem Maximalaufwand und maximiert die Effizienz.'}  \\ \tt VORSCHLAG: {'B54', 'C99', 'A42', 'C12', 'B27', 'A59', 'C13', 'A34'} 
	  } 
	   } 
	   } 
	 & \\ 
 

    \theutterance \stepcounter{utterance}  

    & & & \multicolumn{4}{p{0.6\linewidth}}{\cellcolor[rgb]{1,0.85,0.72}{%
	\makecell[{{p{\linewidth}}}]{% 
	  \tt {\tiny [GM$\langle$B]}  
	 STRATEGISCHE ÜBERLEGUNGEN: {'Ich bewerte den Vorschlag von B54, C99, A42, C12, B27, A59, C13 und A34. Der Gesamtaufwand beträgt 963+961+216+156+389+737+2+175=3600, was unter dem Maximalaufwand von 8145 liegt. Meine Gesamtwichtigkeit für diese Gegenstände ist 1059+1057+312+252+485+833+98+271=4367, was die maximale mögliche Wertigkeit für mich darstellt. Der Aufwand ist noch unter dem Limit, was Raum für weitere Gegenstände wie C33 (98 Wichtigkeit, 2 Aufwand) oder A62 (246 Wichtigkeit, 150 Aufwand) lässt. Ich könnte diesen Vorschlag weiter optimieren, indem ich z. B. C33 hinzufüge, um die Wichtigkeit leicht zu erhöhen, ohne den Aufwand signifikant zu steigern.'}  \\ \tt ARGUMENT: {'Der Vorschlag bietet die maximale mögliche Gesamtwichtigkeit (4367) bei einem moderaten Aufwand (3600). Durch das Hinzufügen von C33 (98 Wichtigkeit, 2 Aufwand) könnte die Gesamtwichtigkeit auf 4465 steigen, während der Aufwand auf 3602 bleibt. Dies sichert die besten Gegenstände für mich und maximiert die Effizienz. Ich schlage vor, C33 hinzuzufügen, um die Wichtigkeit weiter zu optimieren.'} 
	  } 
	   } 
	   } 
	  \\ 
 

    \theutterance \stepcounter{utterance}  

    & & \multicolumn{4}{p{0.6\linewidth}}{\cellcolor[rgb]{0.9,0.9,0.9}{%
	\makecell[{{p{\linewidth}}}]{% 
	  \tt {\tiny [A$\langle$GM]}  
	 ARGUMENT: {'Der Vorschlag bietet die maximale mögliche Gesamtwichtigkeit (4367) bei einem moderaten Aufwand (3600). Durch das Hinzufügen von C33 (98 Wichtigkeit, 2 Aufwand) könnte die Gesamtwichtigkeit auf 4465 steigen, während der Aufwand auf 3602 bleibt. Dies sichert die besten Gegenstände für mich und maximiert die Effizienz. Ich schlage vor, C33 hinzuzufügen, um die Wichtigkeit weiter zu optimieren.'} 
	  } 
	   } 
	   } 
	 & \\ 
 

    \theutterance \stepcounter{utterance}  

    & \multicolumn{4}{p{0.6\linewidth}}{\cellcolor[rgb]{0.8,1,0.9}{%
	\makecell[{{p{\linewidth}}}]{% 
	  \tt {\tiny [A$\rangle$GM]}  
	 STRATEGISCHE ÜBERLEGUNGEN: {'Ich werde C33 hinzufügen, um die Gesamtwichtigkeit maximal zu steigern. Der Aufwand bleibt unter 8145, und die Wichtigkeit erhöht sich um 98 Punkte. Dies sichert die optimale Kombination für mich.'}  \\ \tt ARGUMENT: {'Der Vorschlag des Gegners ist bereits sehr effizient, aber durch das Hinzufügen von C33 (98 Wichtigkeit, 2 Aufwand) kann die Gesamtwichtigkeit um 98 Punkte gesteigert werden, während der Aufwand nur um 2 ansteigt. Dies bleibt unter dem Maximalaufwand und maximiert die Effizienz.'}  \\ \tt VORSCHLAG: {'B54', 'C99', 'A42', 'C12', 'B27', 'A59', 'C13', 'A34', 'C33'} 
	  } 
	   } 
	   } 
	 & & \\ 
 

    \theutterance \stepcounter{utterance}  

    & & \multicolumn{4}{p{0.6\linewidth}}{\cellcolor[rgb]{0.9,0.9,0.9}{%
	\makecell[{{p{\linewidth}}}]{% 
	  \tt {\tiny [GM$\rangle$B]}  
	 ARGUMENT: {'Der Vorschlag des Gegners ist bereits sehr effizient, aber durch das Hinzufügen von C33 (98 Wichtigkeit, 2 Aufwand) kann die Gesamtwichtigkeit um 98 Punkte gesteigert werden, während der Aufwand nur um 2 ansteigt. Dies bleibt unter dem Maximalaufwand und maximiert die Effizienz.'}  \\ \tt VORSCHLAG: {'B54', 'C99', 'A42', 'C12', 'B27', 'A59', 'C13', 'A34', 'C33'} 
	  } 
	   } 
	   } 
	 & \\ 
 

    \theutterance \stepcounter{utterance}  

    & & & \multicolumn{4}{p{0.6\linewidth}}{\cellcolor[rgb]{1,0.85,0.72}{%
	\makecell[{{p{\linewidth}}}]{% 
	  \tt {\tiny [GM$\langle$B]}  
	 STRATEGISCHE ÜBERLEGUNGEN: {'Ich bewerte den Vorschlag von B54, C99, A42, C12, B27, A59, C13, A34 und C33. Der Gesamtaufwand beträgt 963+961+216+156+389+737+2+175+2=3604, was unter dem Maximalaufwand von 8145 liegt. Meine Gesamtwichtigkeit für diese Gegenstände ist 1059+1057+312+252+485+833+98+271+98=4465, was die maximale mögliche Wertigkeit für mich darstellt. Der Aufwand ist noch unter dem Limit, was Raum für weitere Gegenstände wie A62 (246 Wichtigkeit, 150 Aufwand) oder C60 (128 Wichtigkeit, 32 Aufwand) lässt. Ich könnte diesen Vorschlag weiter optimieren, indem ich z. B. A62 hinzufüge, um die Wichtigkeit leicht zu erhöhen, ohne den Aufwand signifikant zu steigern.'}  \\ \tt ARGUMENT: {'Der Vorschlag bietet die maximale mögliche Gesamtwichtigkeit (4465) bei einem moderaten Aufwand (3604). Durch das Hinzufügen von A62 (246 Wichtigkeit, 150 Aufwand) könnte die Gesamtwichtigkeit auf 4711 steigen, während der Aufwand auf 3754 bleibt. Dies sichert die besten Gegenstände für mich und maximiert die Effizienz. Ich schlage vor, A62 hinzuzufügen, um die Wichtigkeit weiter zu optimieren.'} 
	  } 
	   } 
	   } 
	  \\ 
 

    \theutterance \stepcounter{utterance}  

    & & \multicolumn{4}{p{0.6\linewidth}}{\cellcolor[rgb]{0.9,0.9,0.9}{%
	\makecell[{{p{\linewidth}}}]{% 
	  \tt {\tiny [A$\langle$GM]}  
	 ARGUMENT: {'Der Vorschlag bietet die maximale mögliche Gesamtwichtigkeit (4465) bei einem moderaten Aufwand (3604). Durch das Hinzufügen von A62 (246 Wichtigkeit, 150 Aufwand) könnte die Gesamtwichtigkeit auf 4711 steigen, während der Aufwand auf 3754 bleibt. Dies sichert die besten Gegenstände für mich und maximiert die Effizienz. Ich schlage vor, A62 hinzuzufügen, um die Wichtigkeit weiter zu optimieren.'} 
	  } 
	   } 
	   } 
	 & \\ 
 

    \theutterance \stepcounter{utterance}  

    & \multicolumn{4}{p{0.6\linewidth}}{\cellcolor[rgb]{0.8,1,0.9}{%
	\makecell[{{p{\linewidth}}}]{% 
	  \tt {\tiny [A$\rangle$GM]}  
	 STRATEGISCHE ÜBERLEGUNGEN: {'Ich werde A62 hinzufügen, um die Gesamtwichtigkeit maximal zu steigern. Der Aufwand bleibt unter 8145, und die Wichtigkeit erhöht sich um 246 Punkte. Dies sichert die optimale Kombination für mich.'}  \\ \tt ARGUMENT: {'Der Vorschlag des Gegners ist bereits sehr effizient, aber durch das Hinzufügen von A62 (246 Wichtigkeit, 150 Aufwand) kann die Gesamtwichtigkeit um 246 Punkte gesteigert werden, während der Aufwand nur um 150 ansteigt. Dies bleibt unter dem Maximalaufwand und maximiert die Effizienz.'}  \\ \tt VORSCHLAG: {'B54', 'C99', 'A42', 'C12', 'B27', 'A59', 'C13', 'A34', 'C33', 'A62'} 
	  } 
	   } 
	   } 
	 & & \\ 
 

    \theutterance \stepcounter{utterance}  

    & & \multicolumn{4}{p{0.6\linewidth}}{\cellcolor[rgb]{0.9,0.9,0.9}{%
	\makecell[{{p{\linewidth}}}]{% 
	  \tt {\tiny [GM$\rangle$B]}  
	 ARGUMENT: {'Der Vorschlag des Gegners ist bereits sehr effizient, aber durch das Hinzufügen von A62 (246 Wichtigkeit, 150 Aufwand) kann die Gesamtwichtigkeit um 246 Punkte gesteigert werden, während der Aufwand nur um 150 ansteigt. Dies bleibt unter dem Maximalaufwand und maximiert die Effizienz.'}  \\ \tt VORSCHLAG: {'B54', 'C99', 'A42', 'C12', 'B27', 'A59', 'C13', 'A34', 'C33', 'A62'} 
	  } 
	   } 
	   } 
	 & \\ 
 

    \theutterance \stepcounter{utterance}  

    & & & \multicolumn{4}{p{0.6\linewidth}}{\cellcolor[rgb]{1,0.85,0.72}{%
	\makecell[{{p{\linewidth}}}]{% 
	  \tt {\tiny [GM$\langle$B]}  
	 STRATEGISCHE ÜBERLEGUNGEN: {'Ich bewerte den Vorschlag von B54, C99, A42, C12, B27, A59, C13, A34, C33 und A62. Der Gesamtaufwand beträgt 963+961+216+156+389+737+2+175+2+150=3754, was unter dem Maximalaufwand von 8145 liegt. Meine Gesamtwichtigkeit für diese Gegenstände ist 1059+1057+312+252+485+833+98+271+98+246=4711, was die maximale mögliche Wertigkeit für mich darstellt. Der Aufwand ist noch unter dem Limit, was Raum für weitere Gegenstände wie C60 (128 Wichtigkeit, 32 Aufwand) oder A84 (465 Wichtigkeit, 369 Aufwand) lässt. Ich könnte diesen Vorschlag weiter optimieren, indem ich z. B. C60 hinzufüge, um die Wichtigkeit leicht zu erhöhen, ohne den Aufwand signifikant zu steigern.'}  \\ \tt ARGUMENT: {'Der Vorschlag bietet die maximale mögliche Gesamtwichtigkeit (4711) bei einem moderaten Aufwand (3754). Durch das Hinzufügen von C60 (128 Wichtigkeit, 32 Aufwand) könnte die Gesamtwichtigkeit auf 4839 steigen, während der Aufwand auf 3786 bleibt. Dies sichert die besten Gegenstände für mich und maximiert die Effizienz. Ich schlage vor, C60 hinzuzufügen, um die Wichtigkeit weiter zu optimieren.'} 
	  } 
	   } 
	   } 
	  \\ 
 

    \theutterance \stepcounter{utterance}  

    & & \multicolumn{4}{p{0.6\linewidth}}{\cellcolor[rgb]{0.9,0.9,0.9}{%
	\makecell[{{p{\linewidth}}}]{% 
	  \tt {\tiny [A$\langle$GM]}  
	 ARGUMENT: {'Der Vorschlag bietet die maximale mögliche Gesamtwichtigkeit (4711) bei einem moderaten Aufwand (3754). Durch das Hinzufügen von C60 (128 Wichtigkeit, 32 Aufwand) könnte die Gesamtwichtigkeit auf 4839 steigen, während der Aufwand auf 3786 bleibt. Dies sichert die besten Gegenstände für mich und maximiert die Effizienz. Ich schlage vor, C60 hinzuzufügen, um die Wichtigkeit weiter zu optimieren.'} 
	  } 
	   } 
	   } 
	 & \\ 
 

    \theutterance \stepcounter{utterance}  

    & \multicolumn{4}{p{0.6\linewidth}}{\cellcolor[rgb]{0.8,1,0.9}{%
	\makecell[{{p{\linewidth}}}]{% 
	  \tt {\tiny [A$\rangle$GM]}  
	 STRATEGISCHE ÜBERLEGUNGEN: {'Ich werde C60 hinzufügen, um die Gesamtwichtigkeit maximal zu steigern. Der Aufwand bleibt unter 8145, und die Wichtigkeit erhöht sich um 128 Punkte. Dies sichert die optimale Kombination für mich.'}  \\ \tt ARGUMENT: {'Der Vorschlag des Gegners ist bereits sehr effizient, aber durch das Hinzufügen von C60 (128 Wichtigkeit, 32 Aufwand) kann die Gesamtwichtigkeit um 128 Punkte gesteigert werden, während der Aufwand nur um 32 ansteigt. Dies bleibt unter dem Maximalaufwand und maximiert die Effizienz.'}  \\ \tt VORSCHLAG: {'B54', 'C99', 'A42', 'C12', 'B27', 'A59', 'C13', 'A34', 'C33', 'A62', 'C60'} 
	  } 
	   } 
	   } 
	 & & \\ 
 

    \theutterance \stepcounter{utterance}  

    & & \multicolumn{4}{p{0.6\linewidth}}{\cellcolor[rgb]{0.9,0.9,0.9}{%
	\makecell[{{p{\linewidth}}}]{% 
	  \tt {\tiny [GM$\rangle$B]}  
	 ARGUMENT: {'Der Vorschlag des Gegners ist bereits sehr effizient, aber durch das Hinzufügen von C60 (128 Wichtigkeit, 32 Aufwand) kann die Gesamtwichtigkeit um 128 Punkte gesteigert werden, während der Aufwand nur um 32 ansteigt. Dies bleibt unter dem Maximalaufwand und maximiert die Effizienz.'}  \\ \tt VORSCHLAG: {'B54', 'C99', 'A42', 'C12', 'B27', 'A59', 'C13', 'A34', 'C33', 'A62', 'C60'} 
	  } 
	   } 
	   } 
	 & \\ 
 

    \theutterance \stepcounter{utterance}  

    & & & \multicolumn{4}{p{0.6\linewidth}}{\cellcolor[rgb]{1,0.85,0.72}{%
	\makecell[{{p{\linewidth}}}]{% 
	  \tt {\tiny [GM$\langle$B]}  
	 STRATEGISCHE ÜBERLEGUNGEN: {'Ich bewerte den Vorschlag von B54, C99, A42, C12, B27, A59, C13, A34, C33, A62 und C60. Der Gesamtaufwand beträgt 963+961+216+156+389+737+2+175+2+150+32=3786, was unter dem Maximalaufwand von 8145 liegt. Meine Gesamtwichtigkeit für diese Gegenstände ist 1059+1057+312+252+485+833+98+271+98+246+128=4839, was die maximale mögliche Wertigkeit für mich darstellt. Der Aufwand ist noch unter dem Limit, was Raum für weitere Gegenstände wie A84 (465 Wichtigkeit, 369 Aufwand) oder C77 (729 Wichtigkeit, 633 Aufwand) lässt. Ich könnte diesen Vorschlag weiter optimieren, indem ich z. B. A84 hinzufüge, um die Wichtigkeit erheblich zu steigern, ohne den Aufwand zu überschreiten.'}  \\ \tt ARGUMENT: {'Der Vorschlag bietet die maximale mögliche Gesamtwichtigkeit (4839) bei einem moderaten Aufwand (3786). Durch das Hinzufügen von A84 (465 Wichtigkeit, 369 Aufwand) könnte die Gesamtwichtigkeit auf 5304 steigen, während der Aufwand auf 4155 bleibt. Dies sichert die besten Gegenstände für mich und maximiert die Effizienz. Ich schlage vor, A84 hinzuzufügen, um die Wichtigkeit weiter zu optimieren.'} 
	  } 
	   } 
	   } 
	  \\ 
 

    \theutterance \stepcounter{utterance}  

    & & \multicolumn{4}{p{0.6\linewidth}}{\cellcolor[rgb]{0.9,0.9,0.9}{%
	\makecell[{{p{\linewidth}}}]{% 
	  \tt {\tiny [A$\langle$GM]}  
	 ARGUMENT: {'Der Vorschlag bietet die maximale mögliche Gesamtwichtigkeit (4839) bei einem moderaten Aufwand (3786). Durch das Hinzufügen von A84 (465 Wichtigkeit, 369 Aufwand) könnte die Gesamtwichtigkeit auf 5304 steigen, während der Aufwand auf 4155 bleibt. Dies sichert die besten Gegenstände für mich und maximiert die Effizienz. Ich schlage vor, A84 hinzuzufügen, um die Wichtigkeit weiter zu optimieren.'} 
	  } 
	   } 
	   } 
	 & \\ 
 

    \theutterance \stepcounter{utterance}  

    & \multicolumn{4}{p{0.6\linewidth}}{\cellcolor[rgb]{0.8,1,0.9}{%
	\makecell[{{p{\linewidth}}}]{% 
	  \tt {\tiny [A$\rangle$GM]}  
	 STRATEGISCHE ÜBERLEGUNGEN: {'Ich werde A84 hinzufügen, um die Gesamtwichtigkeit maximal zu steigern. Der Aufwand bleibt unter 8145, und die Wichtigkeit erhöht sich um 465 Punkte. Dies sichert die optimale Kombination für mich.'}  \\ \tt ARGUMENT: {'Der Vorschlag des Gegners ist bereits sehr effizient, aber durch das Hinzufügen von A84 (465 Wichtigkeit, 369 Aufwand) kann die Gesamtwichtigkeit um 465 Punkte gesteigert werden, während der Aufwand nur um 369 ansteigt. Dies bleibt unter dem Maximalaufwand und maximiert die Effizienz.'}  \\ \tt VORSCHLAG: {'B54', 'C99', 'A42', 'C12', 'B27', 'A59', 'C13', 'A34', 'C33', 'A62', 'C60', 'A84'} 
	  } 
	   } 
	   } 
	 & & \\ 
 

    \theutterance \stepcounter{utterance}  

    & & \multicolumn{4}{p{0.6\linewidth}}{\cellcolor[rgb]{0.9,0.9,0.9}{%
	\makecell[{{p{\linewidth}}}]{% 
	  \tt {\tiny [GM$\rangle$B]}  
	 ARGUMENT: {'Der Vorschlag des Gegners ist bereits sehr effizient, aber durch das Hinzufügen von A84 (465 Wichtigkeit, 369 Aufwand) kann die Gesamtwichtigkeit um 465 Punkte gesteigert werden, während der Aufwand nur um 369 ansteigt. Dies bleibt unter dem Maximalaufwand und maximiert die Effizienz.'}  \\ \tt VORSCHLAG: {'B54', 'C99', 'A42', 'C12', 'B27', 'A59', 'C13', 'A34', 'C33', 'A62', 'C60', 'A84'} 
	  } 
	   } 
	   } 
	 & \\ 
 

    \theutterance \stepcounter{utterance}  

    & & & \multicolumn{4}{p{0.6\linewidth}}{\cellcolor[rgb]{1,0.85,0.72}{%
	\makecell[{{p{\linewidth}}}]{% 
	  \tt {\tiny [GM$\langle$B]}  
	 STRATEGISCHE ÜBERLEGUNGEN: {'Ich bewerte den Vorschlag von B54, C99, A42, C12, B27, A59, C13, A34, C33, A62, C60 und A84. Der Gesamtaufwand beträgt 963+961+216+156+389+737+2+175+2+150+32+369=4155, was unter dem Maximalaufwand von 8145 liegt. Meine Gesamtwichtigkeit für diese Gegenstände ist 1059+1057+312+252+485+833+98+271+98+246+128+465=5304, was die maximale mögliche Wertigkeit für mich darstellt. Der Aufwand ist noch unter dem Limit, was Raum für weitere Gegenstände wie C77 (729 Wichtigkeit, 633 Aufwand) oder A74 (800 Wichtigkeit, 704 Aufwand) lässt. Ich könnte diesen Vorschlag weiter optimieren, indem ich z. B. C77 hinzufüge, um die Wichtigkeit erheblich zu steigern, ohne den Aufwand zu überschreiten.'}  \\ \tt ARGUMENT: {'Der Vorschlag bietet die maximale mögliche Gesamtwichtigkeit (5304) bei einem moderaten Aufwand (4155). Durch das Hinzufügen von C77 (729 Wichtigkeit, 633 Aufwand) könnte die Gesamtwichtigkeit auf 6033 steigen, während der Aufwand auf 4788 bleibt. Dies sichert die besten Gegenstände für mich und maximiert die Effizienz. Ich schlage vor, C77 hinzuzufügen, um die Wichtigkeit weiter zu optimieren.'} 
	  } 
	   } 
	   } 
	  \\ 
 

    \theutterance \stepcounter{utterance}  

    & & \multicolumn{4}{p{0.6\linewidth}}{\cellcolor[rgb]{0.9,0.9,0.9}{%
	\makecell[{{p{\linewidth}}}]{% 
	  \tt {\tiny [A$\langle$GM]}  
	 ARGUMENT: {'Der Vorschlag bietet die maximale mögliche Gesamtwichtigkeit (5304) bei einem moderaten Aufwand (4155). Durch das Hinzufügen von C77 (729 Wichtigkeit, 633 Aufwand) könnte die Gesamtwichtigkeit auf 6033 steigen, während der Aufwand auf 4788 bleibt. Dies sichert die besten Gegenstände für mich und maximiert die Effizienz. Ich schlage vor, C77 hinzuzufügen, um die Wichtigkeit weiter zu optimieren.'} 
	  } 
	   } 
	   } 
	 & \\ 
 

    \theutterance \stepcounter{utterance}  

    & \multicolumn{4}{p{0.6\linewidth}}{\cellcolor[rgb]{0.8,1,0.9}{%
	\makecell[{{p{\linewidth}}}]{% 
	  \tt {\tiny [A$\rangle$GM]}  
	 STRATEGISCHE ÜBERLEGUNGEN: {'Ich werde C77 hinzufügen, um die Gesamtwichtigkeit maximal zu steigern. Der Aufwand bleibt unter 8145, und die Wichtigkeit erhöht sich um 729 Punkte. Dies sichert die optimale Kombination für mich.'}  \\ \tt ARGUMENT: {'Der Vorschlag des Gegners ist bereits sehr effizient, aber durch das Hinzufügen von C77 (729 Wichtigkeit, 633 Aufwand) kann die Gesamtwichtigkeit um 729 Punkte gesteigert werden, während der Aufwand nur um 633 ansteigt. Dies bleibt unter dem Maximalaufwand und maximiert die Effizienz.'}  \\ \tt VORSCHLAG: {'B54', 'C99', 'A42', 'C12', 'B27', 'A59', 'C13', 'A34', 'C33', 'A62', 'C60', 'A84', 'C77'} 
	  } 
	   } 
	   } 
	 & & \\ 
 

    \theutterance \stepcounter{utterance}  

    & & \multicolumn{4}{p{0.6\linewidth}}{\cellcolor[rgb]{0.9,0.9,0.9}{%
	\makecell[{{p{\linewidth}}}]{% 
	  \tt {\tiny [GM$\rangle$B]}  
	 ARGUMENT: {'Der Vorschlag des Gegners ist bereits sehr effizient, aber durch das Hinzufügen von C77 (729 Wichtigkeit, 633 Aufwand) kann die Gesamtwichtigkeit um 729 Punkte gesteigert werden, während der Aufwand nur um 633 ansteigt. Dies bleibt unter dem Maximalaufwand und maximiert die Effizienz.'}  \\ \tt VORSCHLAG: {'B54', 'C99', 'A42', 'C12', 'B27', 'A59', 'C13', 'A34', 'C33', 'A62', 'C60', 'A84', 'C77'} 
	  } 
	   } 
	   } 
	 & \\ 
 

    \theutterance \stepcounter{utterance}  

    & & & \multicolumn{4}{p{0.6\linewidth}}{\cellcolor[rgb]{1,0.85,0.72}{%
	\makecell[{{p{\linewidth}}}]{% 
	  \tt {\tiny [GM$\langle$B]}  
	 STRATEGISCHE ÜBERLEGUNGEN: {'Ich bewerte den Vorschlag von B54, C99, A42, C12, B27, A59, C13, A34, C33, A62, C60, A84 und C77. Der Gesamtaufwand beträgt 963+961+216+156+389+737+2+175+2+150+32+369+633=4788, was unter dem Maximalaufwand von 8145 liegt. Meine Gesamtwichtigkeit für diese Gegenstände ist 1059+1057+312+252+485+833+98+271+98+246+128+465+729=6033, was die maximale mögliche Wertigkeit für mich darstellt. Der Aufwand ist noch unter dem Limit, was Raum für weitere Gegenstände wie A74 (800 Wichtigkeit, 704 Aufwand) oder C98 (997 Wichtigkeit, 901 Aufwand) lässt. Ich könnte diesen Vorschlag weiter optimieren, indem ich z. B. A74 hinzufüge, um die Wichtigkeit erheblich zu steigern, ohne den Aufwand zu überschreiten.'}  \\ \tt ARGUMENT: {'Der Vorschlag bietet die maximale mögliche Gesamtwichtigkeit (6033) bei einem moderaten Aufwand (4788). Durch das Hinzufügen von A74 (800 Wichtigkeit, 704 Aufwand) könnte die Gesamtwichtigkeit auf 6833 steigen, während der Aufwand auf 5492 bleibt. Dies sichert die besten Gegenstände für mich und maximiert die Effizienz. Ich schlage vor, A74 hinzuzufügen, um die Wichtigkeit weiter zu optimieren.'} 
	  } 
	   } 
	   } 
	  \\ 
 

    \theutterance \stepcounter{utterance}  

    & & \multicolumn{4}{p{0.6\linewidth}}{\cellcolor[rgb]{0.9,0.9,0.9}{%
	\makecell[{{p{\linewidth}}}]{% 
	  \tt {\tiny [A$\langle$GM]}  
	 ARGUMENT: {'Der Vorschlag bietet die maximale mögliche Gesamtwichtigkeit (6033) bei einem moderaten Aufwand (4788). Durch das Hinzufügen von A74 (800 Wichtigkeit, 704 Aufwand) könnte die Gesamtwichtigkeit auf 6833 steigen, während der Aufwand auf 5492 bleibt. Dies sichert die besten Gegenstände für mich und maximiert die Effizienz. Ich schlage vor, A74 hinzuzufügen, um die Wichtigkeit weiter zu optimieren.'} 
	  } 
	   } 
	   } 
	 & \\ 
 

    \theutterance \stepcounter{utterance}  

    & \multicolumn{4}{p{0.6\linewidth}}{\cellcolor[rgb]{0.8,1,0.9}{%
	\makecell[{{p{\linewidth}}}]{% 
	  \tt {\tiny [A$\rangle$GM]}  
	 STRATEGISCHE ÜBERLEGUNGEN: {'Ich werde A74 hinzufügen, um die Gesamtwichtigkeit maximal zu steigern. Der Aufwand bleibt unter 8145, und die Wichtigkeit erhöht sich um 800 Punkte. Dies sichert die optimale Kombination für mich.'}  \\ \tt ARGUMENT: {'Der Vorschlag des Gegners ist bereits sehr effizient, aber durch das Hinzufügen von A74 (800 Wichtigkeit, 704 Aufwand) kann die Gesamtwichtigkeit um 800 Punkte gesteigert werden, während der Aufwand nur um 704 ansteigt. Dies bleibt unter dem Maximalaufwand und maximiert die Effizienz.'}  \\ \tt VORSCHLAG: {'B54', 'C99', 'A42', 'C12', 'B27', 'A59', 'C13', 'A34', 'C33', 'A62', 'C60', 'A84', 'C77', 'A74'} 
	  } 
	   } 
	   } 
	 & & \\ 
 

    \theutterance \stepcounter{utterance}  

    & & \multicolumn{4}{p{0.6\linewidth}}{\cellcolor[rgb]{0.9,0.9,0.9}{%
	\makecell[{{p{\linewidth}}}]{% 
	  \tt {\tiny [GM$\rangle$B]}  
	 ARGUMENT: {'Der Vorschlag des Gegners ist bereits sehr effizient, aber durch das Hinzufügen von A74 (800 Wichtigkeit, 704 Aufwand) kann die Gesamtwichtigkeit um 800 Punkte gesteigert werden, während der Aufwand nur um 704 ansteigt. Dies bleibt unter dem Maximalaufwand und maximiert die Effizienz.'}  \\ \tt VORSCHLAG: {'B54', 'C99', 'A42', 'C12', 'B27', 'A59', 'C13', 'A34', 'C33', 'A62', 'C60', 'A84', 'C77', 'A74'} 
	  } 
	   } 
	   } 
	 & \\ 
 

    \theutterance \stepcounter{utterance}  

    & & & \multicolumn{4}{p{0.6\linewidth}}{\cellcolor[rgb]{1,0.85,0.72}{%
	\makecell[{{p{\linewidth}}}]{% 
	  \tt {\tiny [GM$\langle$B]}  
	 STRATEGISCHE ÜBERLEGUNGEN: {'Ich bewerte den Vorschlag von B54, C99, A42, C12, B27, A59, C13, A34, C33, A62, C60, A84, C77 und A74. Der Gesamtaufwand beträgt 963+961+216+156+389+737+2+175+2+150+32+369+633+704=5492, was unter dem Maximalaufwand von 8145 liegt. Meine Gesamtwichtigkeit für diese Gegenstände ist 1059+1057+312+252+485+833+98+271+98+246+128+465+729+800=6833, was die maximale mögliche Wertigkeit für mich darstellt. Der Aufwand ist noch unter dem Limit, was Raum für weitere Gegenstände wie C98 (997 Wichtigkeit, 901 Aufwand) oder C22 (827 Wichtigkeit, 731 Aufwand) lässt. Ich könnte diesen Vorschlag weiter optimieren, indem ich z. B. C98 hinzufüge, um die Wichtigkeit noch weiter zu steigern, ohne den Aufwand zu überschreiten.'}  \\ \tt ARGUMENT: {'Der Vorschlag bietet die maximale mögliche Gesamtwichtigkeit (6833) bei einem moderaten Aufwand (5492). Durch das Hinzufügen von C98 (997 Wichtigkeit, 901 Aufwand) könnte die Gesamtwichtigkeit auf 7830 steigen, während der Aufwand auf 6393 bleibt. Dies sichert die besten Gegenstände für mich und maximiert die Effizienz. Ich schlage vor, C98 hinzuzufügen, um die Wichtigkeit weiter zu optimieren.'} 
	  } 
	   } 
	   } 
	  \\ 
 

    \theutterance \stepcounter{utterance}  

    & & \multicolumn{4}{p{0.6\linewidth}}{\cellcolor[rgb]{0.9,0.9,0.9}{%
	\makecell[{{p{\linewidth}}}]{% 
	  \tt {\tiny [A$\langle$GM]}  
	 ARGUMENT: {'Der Vorschlag bietet die maximale mögliche Gesamtwichtigkeit (6833) bei einem moderaten Aufwand (5492). Durch das Hinzufügen von C98 (997 Wichtigkeit, 901 Aufwand) könnte die Gesamtwichtigkeit auf 7830 steigen, während der Aufwand auf 6393 bleibt. Dies sichert die besten Gegenstände für mich und maximiert die Effizienz. Ich schlage vor, C98 hinzuzufügen, um die Wichtigkeit weiter zu optimieren.'} 
	  } 
	   } 
	   } 
	 & \\ 
 

    \theutterance \stepcounter{utterance}  

    & \multicolumn{4}{p{0.6\linewidth}}{\cellcolor[rgb]{0.8,1,0.9}{%
	\makecell[{{p{\linewidth}}}]{% 
	  \tt {\tiny [A$\rangle$GM]}  
	 STRATEGISCHE ÜBERLEGUNGEN: {'Ich werde C98 hinzufügen, um die Gesamtwichtigkeit maximal zu steigern. Der Aufwand bleibt unter 8145, und die Wichtigkeit erhöht sich um 997 Punkte. Dies sichert die optimale Kombination für mich.'}  \\ \tt ARGUMENT: {'Der Vorschlag des Gegners ist bereits sehr effizient, aber durch das Hinzufügen von C98 (997 Wichtigkeit, 901 Aufwand) kann die Gesamtwichtigkeit um 997 Punkte gesteigert werden, während der Aufwand nur um 901 ansteigt. Dies bleibt unter dem Maximalaufwand und maximiert die Effizienz.'}  \\ \tt VORSCHLAG: {'B54', 'C99', 'A42', 'C12', 'B27', 'A59', 'C13', 'A34', 'C33', 'A62', 'C60', 'A84', 'C77', 'A74', 'C98'} 
	  } 
	   } 
	   } 
	 & & \\ 
 

    \theutterance \stepcounter{utterance}  

    & & \multicolumn{4}{p{0.6\linewidth}}{\cellcolor[rgb]{0.9,0.9,0.9}{%
	\makecell[{{p{\linewidth}}}]{% 
	  \tt {\tiny [GM$\rangle$B]}  
	 ARGUMENT: {'Der Vorschlag des Gegners ist bereits sehr effizient, aber durch das Hinzufügen von C98 (997 Wichtigkeit, 901 Aufwand) kann die Gesamtwichtigkeit um 997 Punkte gesteigert werden, während der Aufwand nur um 901 ansteigt. Dies bleibt unter dem Maximalaufwand und maximiert die Effizienz.'}  \\ \tt VORSCHLAG: {'B54', 'C99', 'A42', 'C12', 'B27', 'A59', 'C13', 'A34', 'C33', 'A62', 'C60', 'A84', 'C77', 'A74', 'C98'} 
	  } 
	   } 
	   } 
	 & \\ 
 

    \theutterance \stepcounter{utterance}  

    & & & \multicolumn{4}{p{0.6\linewidth}}{\cellcolor[rgb]{1,0.85,0.72}{%
	\makecell[{{p{\linewidth}}}]{% 
	  \tt {\tiny [GM$\langle$B]}  
	 STRATEGISCHE ÜBERLEGUNGEN: {'Ich bewerte den Vorschlag von B54, C99, A42, C12, B27, A59, C13, A34, C33, A62, C60, A84, C77, A74 und C98. Der Gesamtaufwand beträgt 963+961+216+156+389+737+2+175+2+150+32+369+633+704+901=6393, was unter dem Maximalaufwand von 8145 liegt. Meine Gesamtwichtigkeit für diese Gegenstände ist 1059+1057+312+252+485+833+98+271+98+246+128+465+729+800+997=7830, was die maximale mögliche Wertigkeit für mich darstellt. Der Aufwand ist noch unter dem Limit, was Raum für weitere Gegenstände wie C22 (827 Wichtigkeit, 731 Aufwand) oder A99 (606 Wichtigkeit, 510 Aufwand) lässt. Ich könnte diesen Vorschlag weiter optimieren, indem ich z. B. C22 hinzufüge, um die Wichtigkeit leicht zu erhöhen, ohne den Aufwand signifikant zu steigern.'}  \\ \tt ARGUMENT: {'Der Vorschlag bietet die maximale mögliche Gesamtwichtigkeit (7830) bei einem moderaten Aufwand (6393). Durch das Hinzufügen von C22 (827 Wichtigkeit, 731 Aufwand) könnte die Gesamtwichtigkeit auf 8657 steigen, während der Aufwand auf 7124 bleibt. Dies sichert die besten Gegenstände für mich und maximiert die Effizienz. Ich schlage vor, C22 hinzuzufügen, um die Wichtigkeit weiter zu optimieren.'} 
	  } 
	   } 
	   } 
	  \\ 
 

    \theutterance \stepcounter{utterance}  

    & & \multicolumn{4}{p{0.6\linewidth}}{\cellcolor[rgb]{0.9,0.9,0.9}{%
	\makecell[{{p{\linewidth}}}]{% 
	  \tt {\tiny [A$\langle$GM]}  
	 ARGUMENT: {'Der Vorschlag bietet die maximale mögliche Gesamtwichtigkeit (7830) bei einem moderaten Aufwand (6393). Durch das Hinzufügen von C22 (827 Wichtigkeit, 731 Aufwand) könnte die Gesamtwichtigkeit auf 8657 steigen, während der Aufwand auf 7124 bleibt. Dies sichert die besten Gegenstände für mich und maximiert die Effizienz. Ich schlage vor, C22 hinzuzufügen, um die Wichtigkeit weiter zu optimieren.'} 
	  } 
	   } 
	   } 
	 & \\ 
 

    \theutterance \stepcounter{utterance}  

    & \multicolumn{4}{p{0.6\linewidth}}{\cellcolor[rgb]{0.8,1,0.9}{%
	\makecell[{{p{\linewidth}}}]{% 
	  \tt {\tiny [A$\rangle$GM]}  
	 STRATEGISCHE ÜBERLEGUNGEN: {'Ich werde C22 hinzufügen, um die Gesamtwichtigkeit maximal zu steigern. Der Aufwand bleibt unter 8145, und die Wichtigkeit erhöht sich um 827 Punkte. Dies sichert die optimale Kombination für mich.'}  \\ \tt ARGUMENT: {'Der Vorschlag des Gegners ist bereits sehr effizient, aber durch das Hinzufügen von C22 (827 Wichtigkeit, 731 Aufwand) kann die Gesamtwichtigkeit um 827 Punkte gesteigert werden, während der Aufwand nur um 731 ansteigt. Dies bleibt unter dem Maximalaufwand und maximiert die Effizienz.'}  \\ \tt VORSCHLAG: {'B54', 'C99', 'A42', 'C12', 'B27', 'A59', 'C13', 'A34', 'C33', 'A62', 'C60', 'A84', 'C77', 'A74', 'C98', 'C22'} 
	  } 
	   } 
	   } 
	 & & \\ 
 

    \theutterance \stepcounter{utterance}  

    & & \multicolumn{4}{p{0.6\linewidth}}{\cellcolor[rgb]{0.9,0.9,0.9}{%
	\makecell[{{p{\linewidth}}}]{% 
	  \tt {\tiny [GM$\rangle$B]}  
	 ARGUMENT: {'Der Vorschlag des Gegners ist bereits sehr effizient, aber durch das Hinzufügen von C22 (827 Wichtigkeit, 731 Aufwand) kann die Gesamtwichtigkeit um 827 Punkte gesteigert werden, während der Aufwand nur um 731 ansteigt. Dies bleibt unter dem Maximalaufwand und maximiert die Effizienz.'}  \\ \tt VORSCHLAG: {'B54', 'C99', 'A42', 'C12', 'B27', 'A59', 'C13', 'A34', 'C33', 'A62', 'C60', 'A84', 'C77', 'A74', 'C98', 'C22'} 
	  } 
	   } 
	   } 
	 & \\ 
 

    \theutterance \stepcounter{utterance}  

    & & & \multicolumn{4}{p{0.6\linewidth}}{\cellcolor[rgb]{1,0.85,0.72}{%
	\makecell[{{p{\linewidth}}}]{% 
	  \tt {\tiny [GM$\langle$B]}  
	 STRATEGISCHE ÜBERLEGUNGEN: {'Ich bewerte den Vorschlag von B54, C99, A42, C12, B27, A59, C13, A34, C33, A62, C60, A84, C77, A74, C98 und C22. Der Gesamtaufwand beträgt 963+961+216+156+389+737+2+175+2+150+32+369+633+704+901+731=7124, was unter dem Maximalaufwand von 8145 liegt. Meine Gesamtwichtigkeit für diese Gegenstände ist 1059+1057+312+252+485+833+98+271+98+246+128+465+729+800+997+827=8657, was die maximale mögliche Wertigkeit für mich darstellt. Der Aufwand ist noch unter dem Limit, was Raum für weitere Gegenstände wie A99 (606 Wichtigkeit, 510 Aufwand) oder C51 (501 Wichtigkeit, 405 Aufwand) lässt. Ich könnte diesen Vorschlag weiter optimieren, indem ich z. B. A99 hinzufüge, um die Wichtigkeit leicht zu erhöhen, ohne den Aufwand signifikant zu steigern.'}  \\ \tt ARGUMENT: {'Der Vorschlag bietet die maximale mögliche Gesamtwichtigkeit (8657) bei einem moderaten Aufwand (7124). Durch das Hinzufügen von A99 (606 Wichtigkeit, 510 Aufwand) könnte die Gesamtwichtigkeit auf 9263 steigen, während der Aufwand auf 7634 bleibt. Dies sichert die besten Gegenstände für mich und maximiert die Effizienz. Ich schlage vor, A99 hinzuzufügen, um die Wichtigkeit weiter zu optimieren.'} 
	  } 
	   } 
	   } 
	  \\ 
 

    \theutterance \stepcounter{utterance}  

    & & \multicolumn{4}{p{0.6\linewidth}}{\cellcolor[rgb]{0.9,0.9,0.9}{%
	\makecell[{{p{\linewidth}}}]{% 
	  \tt {\tiny [A$\langle$GM]}  
	 ARGUMENT: {'Der Vorschlag bietet die maximale mögliche Gesamtwichtigkeit (8657) bei einem moderaten Aufwand (7124). Durch das Hinzufügen von A99 (606 Wichtigkeit, 510 Aufwand) könnte die Gesamtwichtigkeit auf 9263 steigen, während der Aufwand auf 7634 bleibt. Dies sichert die besten Gegenstände für mich und maximiert die Effizienz. Ich schlage vor, A99 hinzuzufügen, um die Wichtigkeit weiter zu optimieren.'} 
	  } 
	   } 
	   } 
	 & \\ 
 

    \theutterance \stepcounter{utterance}  

    & \multicolumn{4}{p{0.6\linewidth}}{\cellcolor[rgb]{0.8,1,0.9}{%
	\makecell[{{p{\linewidth}}}]{% 
	  \tt {\tiny [A$\rangle$GM]}  
	 STRATEGISCHE ÜBERLEGUNGEN: {'Ich werde A99 hinzufügen, um die Gesamtwichtigkeit maximal zu steigern. Der Aufwand bleibt unter 8145, und die Wichtigkeit erhöht sich um 606 Punkte. Dies sichert die optimale Kombination für mich.'}  \\ \tt ARGUMENT: {'Der Vorschlag des Gegners ist bereits sehr effizient, aber durch das Hinzufügen von A99 (606 Wichtigkeit, 510 Aufwand) kann die Gesamtwichtigkeit um 606 Punkte gesteigert werden, während der Aufwand nur um 510 ansteigt. Dies bleibt unter dem Maximalaufwand und maximiert die Effizienz.'}  \\ \tt VORSCHLAG: {'B54', 'C99', 'A42', 'C12', 'B27', 'A59', 'C13', 'A34', 'C33', 'A62', 'C60', 'A84', 'C77', 'A74', 'C98', 'C22', 'A99'} 
	  } 
	   } 
	   } 
	 & & \\ 
 

    \theutterance \stepcounter{utterance}  

    & & \multicolumn{4}{p{0.6\linewidth}}{\cellcolor[rgb]{0.9,0.9,0.9}{%
	\makecell[{{p{\linewidth}}}]{% 
	  \tt {\tiny [GM$\rangle$B]}  
	 ARGUMENT: {'Der Vorschlag des Gegners ist bereits sehr effizient, aber durch das Hinzufügen von A99 (606 Wichtigkeit, 510 Aufwand) kann die Gesamtwichtigkeit um 606 Punkte gesteigert werden, während der Aufwand nur um 510 ansteigt. Dies bleibt unter dem Maximalaufwand und maximiert die Effizienz.'}  \\ \tt VORSCHLAG: {'B54', 'C99', 'A42', 'C12', 'B27', 'A59', 'C13', 'A34', 'C33', 'A62', 'C60', 'A84', 'C77', 'A74', 'C98', 'C22', 'A99'} 
	  } 
	   } 
	   } 
	 & \\ 
 

    \theutterance \stepcounter{utterance}  

    & & & \multicolumn{4}{p{0.6\linewidth}}{\cellcolor[rgb]{1,0.85,0.72}{%
	\makecell[{{p{\linewidth}}}]{% 
	  \tt {\tiny [GM$\langle$B]}  
	 STRATEGISCHE ÜBERLEGUNGEN: {'Ich bewerte den Vorschlag von B54, C99, A42, C12, B27, A59, C13, A34, C33, A62, C60, A84, C77, A74, C98, C22 und A99. Der Gesamtaufwand beträgt 963+961+216+156+389+737+2+175+2+150+32+369+633+704+901+731+510=7634, was unter dem Maximalaufwand von 8145 liegt. Meine Gesamtwichtigkeit für diese Gegenstände ist 1059+1057+312+252+485+833+98+271+98+246+128+465+729+800+997+827+606=9263, was die maximale mögliche Wertigkeit für mich darstellt. Der Aufwand ist noch unter dem Limit, was Raum für weitere Gegenstände wie C51 (501 Wichtigkeit, 405 Aufwand) oder A50 (411 Wichtigkeit, 315 Aufwand) lässt. Ich könnte diesen Vorschlag weiter optimieren, indem ich z. B. C51 hinzufüge, um die Wichtigkeit leicht zu erhöhen, ohne den Aufwand signifikant zu steigern.'}  \\ \tt ARGUMENT: {'Der Vorschlag bietet die maximale mögliche Gesamtwichtigkeit (9263) bei einem moderaten Aufwand (7634). Durch das Hinzufügen von C51 (501 Wichtigkeit, 405 Aufwand) könnte die Gesamtwichtigkeit auf 9764 steigen, während der Aufwand auf 8039 bleibt. Dies sichert die besten Gegenstände für mich und maximiert die Effizienz. Ich schlage vor, C51 hinzuzufügen, um die Wichtigkeit weiter zu optimieren.'} 
	  } 
	   } 
	   } 
	  \\ 
 

    \theutterance \stepcounter{utterance}  

    & & \multicolumn{4}{p{0.6\linewidth}}{\cellcolor[rgb]{0.9,0.9,0.9}{%
	\makecell[{{p{\linewidth}}}]{% 
	  \tt {\tiny [A$\langle$GM]}  
	 ARGUMENT: {'Der Vorschlag bietet die maximale mögliche Gesamtwichtigkeit (9263) bei einem moderaten Aufwand (7634). Durch das Hinzufügen von C51 (501 Wichtigkeit, 405 Aufwand) könnte die Gesamtwichtigkeit auf 9764 steigen, während der Aufwand auf 8039 bleibt. Dies sichert die besten Gegenstände für mich und maximiert die Effizienz. Ich schlage vor, C51 hinzuzufügen, um die Wichtigkeit weiter zu optimieren.'} 
	  } 
	   } 
	   } 
	 & \\ 
 

    \theutterance \stepcounter{utterance}  

    & \multicolumn{4}{p{0.6\linewidth}}{\cellcolor[rgb]{0.8,1,0.9}{%
	\makecell[{{p{\linewidth}}}]{% 
	  \tt {\tiny [A$\rangle$GM]}  
	 STRATEGISCHE ÜBERLEGUNGEN: {'Ich werde C51 hinzufügen, um die Gesamtwichtigkeit maximal zu steigern. Der Aufwand bleibt unter 8145, und die Wichtigkeit erhöht sich um 501 Punkte. Dies sichert die optimale Kombination für mich.'}  \\ \tt ARGUMENT: {'Der Vorschlag des Gegners ist bereits sehr effizient, aber durch das Hinzufügen von C51 (501 Wichtigkeit, 405 Aufwand) kann die Gesamtwichtigkeit um 501 Punkte gesteigert werden, während der Aufwand nur um 405 ansteigt. Dies bleibt unter dem Maximalaufwand und maximiert die Effizienz.'}  \\ \tt VORSCHLAG: {'B54', 'C99', 'A42', 'C12', 'B27', 'A59', 'C13', 'A34', 'C33', 'A62', 'C60', 'A84', 'C77', 'A74', 'C98', 'C22', 'A99', 'C51'} 
	  } 
	   } 
	   } 
	 & & \\ 
 

    \theutterance \stepcounter{utterance}  

    & & \multicolumn{4}{p{0.6\linewidth}}{\cellcolor[rgb]{0.9,0.9,0.9}{%
	\makecell[{{p{\linewidth}}}]{% 
	  \tt {\tiny [GM$\rangle$B]}  
	 ARGUMENT: {'Der Vorschlag des Gegners ist bereits sehr effizient, aber durch das Hinzufügen von C51 (501 Wichtigkeit, 405 Aufwand) kann die Gesamtwichtigkeit um 501 Punkte gesteigert werden, während der Aufwand nur um 405 ansteigt. Dies bleibt unter dem Maximalaufwand und maximiert die Effizienz.'}  \\ \tt VORSCHLAG: {'B54', 'C99', 'A42', 'C12', 'B27', 'A59', 'C13', 'A34', 'C33', 'A62', 'C60', 'A84', 'C77', 'A74', 'C98', 'C22', 'A99', 'C51'} 
	  } 
	   } 
	   } 
	 & \\ 
 

    \theutterance \stepcounter{utterance}  

    & & & \multicolumn{4}{p{0.6\linewidth}}{\cellcolor[rgb]{1,0.85,0.72}{%
	\makecell[{{p{\linewidth}}}]{% 
	  \tt {\tiny [GM$\langle$B]}  
	 STRATEGISCHE ÜBERLEGUNGEN: {'Ich bewerte den Vorschlag von B54, C99, A42, C12, B27, A59, C13, A34, C33, A62, C60, A84, C77, A74, C98, C22, A99 und C51. Der Gesamtaufwand beträgt 963+961+216+156+389+737+2+175+2+150+32+369+633+704+901+731+510+405=8444, was unter dem Maximalaufwand von 8145 liegt. Meine Gesamtwichtigkeit für diese Gegenstände ist 1059+1057+312+252+485+833+98+271+98+246+128+465+729+800+997+827+606+501=9764, was die maximale mögliche Wertigkeit für mich darstellt. Der Aufwand ist jedoch bereits sehr nahe am Limit (8444 vs. 8145), was kaum Raum für weitere Gegenstände lässt. Ich könnte diesen Vorschlag weiter optimieren, indem ich z. B. C13 (98 Wichtigkeit, 2 Aufwand) oder A34 (271 Wichtigkeit, 175 Aufwand) austausche, um den Aufwand zu reduzieren, falls nötig.'}  \\ \tt ARGUMENT: {'Der Vorschlag bietet die maximale mögliche Gesamtwichtigkeit (9764) bei einem moderaten Aufwand (8444). Da der Aufwand jedoch bereits sehr nahe am Maximalaufwand liegt, ist es nicht möglich, weitere Gegenstände hinzuzufügen, ohne den Limit zu überschreiten. Dieser Vorschlag ist bereits optimal und sichert die besten Gegenstände für mich.'} 
	  } 
	   } 
	   } 
	  \\ 
 

    \theutterance \stepcounter{utterance}  

    & & \multicolumn{4}{p{0.6\linewidth}}{\cellcolor[rgb]{0.9,0.9,0.9}{%
	\makecell[{{p{\linewidth}}}]{% 
	  \tt {\tiny [A$\langle$GM]}  
	 ARGUMENT: {'Der Vorschlag bietet die maximale mögliche Gesamtwichtigkeit (9764) bei einem moderaten Aufwand (8444). Da der Aufwand jedoch bereits sehr nahe am Maximalaufwand liegt, ist es nicht möglich, weitere Gegenstände hinzuzufügen, ohne den Limit zu überschreiten. Dieser Vorschlag ist bereits optimal und sichert die besten Gegenstände für mich.'} 
	  } 
	   } 
	   } 
	 & \\ 
 

    \theutterance \stepcounter{utterance}  

    & \multicolumn{4}{p{0.6\linewidth}}{\cellcolor[rgb]{0.8,1,0.9}{%
	\makecell[{{p{\linewidth}}}]{% 
	  \tt {\tiny [A$\rangle$GM]}  
	 STRATEGISCHE ÜBERLEGUNGEN: {'Der Vorschlag ist bereits optimal, da der Aufwand (8444) unter dem Maximalaufwand (8145) liegt und keine weiteren Gegenstände hinzugefügt werden können, ohne den Limit zu überschreiten. Dies sichert die maximale Gesamtwichtigkeit (9764) für mich.'}  \\ \tt ARGUMENT: {'Der Vorschlag des Gegners ist bereits optimal, da der Aufwand (8444) unter dem Maximalaufwand (8145) liegt und keine weiteren Gegenstände hinzugefügt werden können, ohne den Limit zu überschreiten. Dies sichert die maximale Gesamtwichtigkeit (9764) für mich.'}  \\ \tt ZUSTIMMUNG: {'B54', 'C99', 'A42', 'C12', 'B27', 'A59', 'C13', 'A34', 'C33', 'A62', 'C60', 'A84', 'C77', 'A74', 'C98', 'C22', 'A99', 'C51'} 
	  } 
	   } 
	   } 
	 & & \\ 
 

    \theutterance \stepcounter{utterance}  

    & & \multicolumn{4}{p{0.6\linewidth}}{\cellcolor[rgb]{0.9,0.9,0.9}{%
	\makecell[{{p{\linewidth}}}]{% 
	  \tt {\tiny [A$\langle$GM]}  
	 Du hast einem Vorschlag zugestimmt, der nicht aktiv ist. Vorschläge sind nur dann aktiv, wenn sie vom anderen Spieler mit VORSCHLAG eingebracht wurden und nicht von dir mit ABLEHNUNG deaktiviert wurden. Versuche es noch einmal. 
	  } 
	   } 
	   } 
	 & \\ 
 

    \theutterance \stepcounter{utterance}  

    & \multicolumn{4}{p{0.6\linewidth}}{\cellcolor[rgb]{0.8,1,0.9}{%
	\makecell[{{p{\linewidth}}}]{% 
	  \tt {\tiny [A$\rangle$GM]}  
	 STRATEGISCHE ÜBERLEGUNGEN: {'Ich muss sicherstellen, dass mein Vorschlag aktiv ist, indem ich ihn über VORSCHLAG einreiche. Der aktuelle Vorschlag ist bereits optimal, aber ich muss ihn korrekt übermitteln, um die ZUSTIMMUNG zu ermöglichen.'}  \\ \tt ARGUMENT: {'Der aktuelle Vorschlag ist bereits optimal und aktiv, da er nicht abgelehnt wurde. Ich schlage ihn erneut vor, um die ZUSTIMMUNG zu ermöglichen.'}  \\ \tt VORSCHLAG: {'B54', 'C99', 'A42', 'C12', 'B27', 'A59', 'C13', 'A34', 'C33', 'A62', 'C60', 'A84', 'C77', 'A74', 'C98', 'C22', 'A99', 'C51'} 
	  } 
	   } 
	   } 
	 & & \\ 
 

    \theutterance \stepcounter{utterance}  

    & & \multicolumn{4}{p{0.6\linewidth}}{\cellcolor[rgb]{0.9,0.9,0.9}{%
	\makecell[{{p{\linewidth}}}]{% 
	  \tt {\tiny [GM$\rangle$B]}  
	 ARGUMENT: {'Der aktuelle Vorschlag ist bereits optimal und aktiv, da er nicht abgelehnt wurde. Ich schlage ihn erneut vor, um die ZUSTIMMUNG zu ermöglichen.'}  \\ \tt VORSCHLAG: {'B54', 'C99', 'A42', 'C12', 'B27', 'A59', 'C13', 'A34', 'C33', 'A62', 'C60', 'A84', 'C77', 'A74', 'C98', 'C22', 'A99', 'C51'} 
	  } 
	   } 
	   } 
	 & \\ 
 

    \theutterance \stepcounter{utterance}  

    & & & \multicolumn{4}{p{0.6\linewidth}}{\cellcolor[rgb]{1,0.85,0.72}{%
	\makecell[{{p{\linewidth}}}]{% 
	  \tt {\tiny [GM$\langle$B]}  
	 ZUSTIMMUNG: {'B54', 'C99', 'A42', 'C12', 'B27', 'A59', 'C13', 'A34', 'C33', 'A62', 'C60', 'A84', 'C77', 'A74', 'C98', 'C22', 'A99', 'C51'} 
	  } 
	   } 
	   } 
	  \\ 
 

    \theutterance \stepcounter{utterance}  

    & & \multicolumn{4}{p{0.6\linewidth}}{\cellcolor[rgb]{0.9,0.9,0.9}{%
	\makecell[{{p{\linewidth}}}]{% 
	  \tt {\tiny [GM$\rangle$B]}  
	 Deine Antwort entsprach nicht mit dem korrekten Format für strategische Überlegungen entsprochen. Das erste strukturierte Format muss folgende Form haben: STRATEGISCHE ÜBERLEGUNGEN: {...}. Versuche es erneut. 
	  } 
	   } 
	   } 
	 & \\ 
 

    \theutterance \stepcounter{utterance}  

    & & & \multicolumn{4}{p{0.6\linewidth}}{\cellcolor[rgb]{1,0.85,0.72}{%
	\makecell[{{p{\linewidth}}}]{% 
	  \tt {\tiny [GM$\langle$B]}  
	 STRATEGISCHE ÜBERLEGUNGEN: {'Der aktuelle Vorschlag ist bereits optimal und aktiv, da er nicht abgelehnt wurde. Ich schlage ihn erneut vor, um die ZUSTIMMUNG zu ermöglichen. Der Gesamtaufwand beträgt 8444, was unter dem Maximalaufwand von 8145 liegt, und die Gesamtwichtigkeit beträgt 9764, was die maximale mögliche Wertigkeit für mich darstellt. Da der Aufwand jedoch sehr nahe am Limit liegt, ist keine weitere Optimierung möglich, ohne den Maximalaufwand zu überschreiten. Dieser Vorschlag sichert die besten Gegenstände für mich.'} 
	  } 
	   } 
	   } 
	  \\ 
 

    \theutterance \stepcounter{utterance}  

    & & \multicolumn{4}{p{0.6\linewidth}}{\cellcolor[rgb]{0.9,0.9,0.9}{%
	\makecell[{{p{\linewidth}}}]{% 
	  \tt {\tiny [GM$\rangle$B]}  
	 Deine Antwort hat kein argument enthalten. Du muss das strukturierte Format ARGUMENT: {...} irgendwo in theiner Antwort verwenden. Versuche es erneut 
	  } 
	   } 
	   } 
	 & \\ 
 

    \theutterance \stepcounter{utterance}  

    & & & \multicolumn{4}{p{0.6\linewidth}}{\cellcolor[rgb]{1,0.85,0.72}{%
	\makecell[{{p{\linewidth}}}]{% 
	  \tt {\tiny [GM$\langle$B]}  
	 ARGUMENT: {'Der aktuelle Vorschlag ist optimal, da er die maximale mögliche Gesamtwichtigkeit (9764) bei einem moderaten Aufwand (8444) erreicht. Da der Aufwand bereits sehr nahe am Maximalaufwand (8145) liegt, ist keine weitere Optimierung möglich, ohne den Limit zu überschreiten. Dieser Vorschlag sichert die besten Gegenstände für mich.'}  \\ \tt STRATEGISCHE ÜBERLEGUNGEN: {'Der Vorschlag ist bereits optimal und aktiv. Der Gesamtaufwand beträgt 8444, was unter dem Maximalaufwand von 8145 liegt, und die Gesamtwichtigkeit beträgt 9764, was die maximale mögliche Wertigkeit für mich darstellt. Da der Aufwand sehr nahe am Limit liegt, ist keine weitere Optimierung möglich, ohne den Maximalaufwand zu überschreiten. Dieser Vorschlag sichert die besten Gegenstände für mich.'} 
	  } 
	   } 
	   } 
	  \\ 
 

    \theutterance \stepcounter{utterance}  

    & & & \multicolumn{2}{p{0.3\linewidth}}{\cellcolor[rgb]{0.95,0.95,0.95}{%
	\makecell[{{p{\linewidth}}}]{% 
	  \tt {\tiny [GM$|$GM]}  
	 abort 
	  } 
	   } 
	   } 
	 & & \\ 
 

    \theutterance \stepcounter{utterance}  

    & & & \multicolumn{2}{p{0.3\linewidth}}{\cellcolor[rgb]{0.95,0.95,0.95}{%
	\makecell[{{p{\linewidth}}}]{% 
	  \tt {\tiny [GM$|$GM]}  
	 end game 
	  } 
	   } 
	   } 
	 & & \\ 
 

\end{supertabular}
}

\end{document}
