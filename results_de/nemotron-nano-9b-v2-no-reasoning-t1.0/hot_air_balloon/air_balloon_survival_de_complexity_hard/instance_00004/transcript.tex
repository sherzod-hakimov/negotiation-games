
\documentclass{article}
\usepackage{colortbl}
\usepackage{makecell}
\usepackage{multirow}
\usepackage{supertabular}

\begin{document}

\newcounter{utterance}

\centering \large Interaction Transcript for game `hot\_air\_balloon', experiment `air\_balloon\_survival\_de\_complexity\_hard', episode 4 with nemotron{-}nano{-}9b{-}v2{-}no{-}reasoning{-}t1.0.
\vspace{24pt}

{ \footnotesize  \setcounter{utterance}{1}
\setlength{\tabcolsep}{0pt}
\begin{supertabular}{c@{$\;$}|p{.15\linewidth}@{}p{.15\linewidth}p{.15\linewidth}p{.15\linewidth}p{.15\linewidth}p{.15\linewidth}}
    \# & $\;$A & \multicolumn{4}{c}{Game Master} & $\;\:$B\\
    \hline

    \theutterance \stepcounter{utterance}  
    & & \multicolumn{4}{p{0.6\linewidth}}{
        \cellcolor[rgb]{0.9,0.9,0.9}{
            \makecell[{{p{\linewidth}}}]{
                \texttt{\tiny{[P1$\langle$GM]}}
                \texttt{Du nimmst an einem kollaborativen Verhandlungspiel Teil.} \\
\\ 
\texttt{Zusammen mit einem anderen Teilnehmer musst du dich auf eine Reihe von Gegenständen entscheiden, die behalten werden. Jeder von euch hat eine persönliche Verteilung über die Wichtigkeit der einzelnen Gegenstände. Jeder von euch hat eine eigene Meinung darüber, wie wichtig jeder einzelne Gegenstand ist (Gegenstandswichtigkeit). Du kennst die Wichtigkeitsverteilung des anderen Spielers nicht. Zusätzlich siehst du, wie viel Aufwand jeder Gegenstand verursacht.  } \\
\texttt{Ihr dürft euch nur auf eine Reihe von Gegenständen einigen, wenn der Gesamtaufwand der ausgewählten Gegenstände den Maximalaufwand nicht überschreitet:} \\
\\ 
\texttt{Maximalaufwand = 6880} \\
\\ 
\texttt{Hier sind die einzelnen Aufwände der Gegenstände:} \\
\\ 
\texttt{Aufwand der Gegenstände = \{"B54": 341, "C72": 804, "B49": 20, "C99": 627, "A69": 10, "B36": 573, "A23": 217, "C77": 481, "C61": 205, "B85": 273, "A90": 303, "C23": 596, "C22": 564, "C51": 535, "B27": 258, "A42": 911, "C13": 240, "C12": 188, "A65": 216, "B07": 401, "A48": 909, "A59": 62, "C98": 245, "A84": 994, "C33": 569, "A87": 718, "C90": 464, "A74": 37, "A50": 340, "B44": 335, "A99": 417, "A34": 123, "B17": 17, "A62": 577, "C60": 190\}} \\
\\ 
\texttt{Hier ist deine persönliche Verteilung der Wichtigkeit der einzelnen Gegenstände:} \\
\\ 
\texttt{Werte der Gegenstandswichtigkeit = \{"B54": 440, "C72": 903, "B49": 119, "C99": 726, "A69": 109, "B36": 672, "A23": 316, "C77": 580, "C61": 304, "B85": 372, "A90": 402, "C23": 695, "C22": 663, "C51": 634, "B27": 357, "A42": 1010, "C13": 339, "C12": 287, "A65": 315, "B07": 500, "A48": 1008, "A59": 161, "C98": 344, "A84": 1093, "C33": 668, "A87": 817, "C90": 563, "A74": 136, "A50": 439, "B44": 434, "A99": 516, "A34": 222, "B17": 116, "A62": 676, "C60": 289\}} \\
\\ 
\texttt{Ziel:} \\
\\ 
\texttt{Dein Ziel ist es, eine Reihe von Gegenständen auszuhandeln, die dir möglichst viel bringt (d. h. Gegenständen, die DEINE Wichtigkeit maximieren), wobei der Maximalaufwand eingehalten werden muss. Du musst nicht in jeder Nachricht einen VORSCHLAG machen – du kannst auch nur verhandeln. Alle Taktiken sind erlaubt!} \\
\\ 
\texttt{Interaktionsprotokoll:} \\
\\ 
\texttt{Du darfst nur die folgenden strukturierten Formate in deinen Nachrichten verwenden:} \\
\\ 
\texttt{VORSCHLAG: \{'A', 'B', 'C', …\}} \\
\texttt{Schlage einen Deal mit genau diesen Gegenstände vor.} \\
\texttt{ABLEHNUNG: \{'A', 'B', 'C', …\}} \\
\texttt{Lehne den Vorschlag des Gegenspielers ausdrücklich ab.} \\
\texttt{ARGUMENT: \{'...'\}} \\
\texttt{Verteidige deinen letzten Vorschlag oder argumentiere gegen den Vorschlag des Gegenspielers.} \\
\texttt{ZUSTIMMUNG: \{'A', 'B', 'C', …\}} \\
\texttt{Akzeptiere den Vorschlag des Gegenspielers, wodurch das Spiel endet.} \\
\texttt{STRATEGISCHE ÜBERLEGUNGEN: \{'...'\}} \\
\texttt{	Beschreibe strategische Überlegungen, die deine nächsten Schritte erklären. Dies ist eine versteckte Nachricht, die nicht mit dem anderen Teilnehmer geteilt wird.} \\
\\ 
\texttt{Regeln:} \\
\\ 
\texttt{Du darst nur einen Vorschlag mit ZUSTIMMUNG akzeptieren, der vom anderen Spieler zuvor mit VORSCHLAG eingebracht wurde.} \\
\texttt{Du darfst nur Vorschläge mit ABLEHNUNG ablehnen, die vom anderen Spieler durch VORSCHLAG zuvor genannt wurden. } \\
\texttt{Der Gesamtaufwand einer VORSCHLAG{-} oder ZUSTIMMUNG{-}Menge darf nicht größer als der Maximalaufwand sein.  } \\
\texttt{Offenbare deine versteckte Wichtigkeitsverteilung nicht.} \\
\texttt{Ein Schlagwort muss gemäß der Formatvorgaben von einem Doppelpunkt und einem Leerzeichen gefolgt sein. Das Argument muss eine Python{-}Menge mit 0 oder mehr Strings sein.  } \\
\texttt{Die Form muss also SCHLAGWORT: \{...\} sein.} \\
\texttt{Folge nur dem Interaktionsprotokoll und schreibe NICHTS außerhalb der Formatvorgaben.} \\
\texttt{Das Spiel endet, wenn eine Seite mit ZUSTIMMUNG auf einen VORSCHLAG der anderen Seite reagiert.  } \\
\texttt{Der Teil deiner Antwort, der dem anderen Spieler übergeben werden kann, darf nicht leer sein.  } \\
\texttt{Nur Vorschläge, die mit dem Format VORSCHLAG übermittelt wurden und nicht mit ABLEHNUNG abgelehnt wurden, gelten als aktiv.  } \\
\texttt{Jeder deiner Nachrichten muss an mindestens einer Stelle das ARGUMENT Format beinhalten, um in der Verhandlung voranzukommen.} \\
\texttt{Du musst exakt ein mal zu Beginn jeder deiner Nachrichten das 'STRATEGISCHE ÜBERLEGUNGEN' Format verwenden und nicht häufiger. Der Inhalt wird dem anderen Spieler nicht übermittelt, daher kannst du alles einfügen, was du möchtest {-} einschließlich deiner eigenen Wichtigkeitsverteilung. Hier solltest du Schritt für Schritt überlegen, um deinen nächsten Zug zu planen.} \\
            }
        }
    }
    & \\ \\

    \theutterance \stepcounter{utterance}  
    & \multicolumn{4}{p{0.6\linewidth}}{
        \cellcolor[rgb]{0.9,0.9,0.9}{
            \makecell[{{p{\linewidth}}}]{
                \texttt{\tiny{[P1$\rangle$GM]}}
                \texttt{STRATEGISCHE ÜBERLEGUNGEN: \{'Ich beginne mit einer Analyse der Gegenstände, die den höchsten Verhältnis von Wichtigkeit zu Aufwand bieten. Meine Prioritäten sind Gegenstände mit hohem Wert für mich und niedrigem Aufwand. Ich werde zunächst Vorschläge mit kleinen Aufwandskombinationen machen, um den Maximalaufwand effizient zu nutzen. Ich werde auch auf Gegenstände achten, die für den anderen Spieler möglicherweise wichtig sind, um Kompromisse zu finden.'\}  } \\
\texttt{ARGUMENT: \{'A69', 'A74', 'A59', 'B17'\}  } \\
\texttt{VORSCHLAG: \{'A69', 'A74', 'A59', 'B17'\}} \\
            }
        }
    }
    & & \\ \\

    \theutterance \stepcounter{utterance}  
    & & \multicolumn{4}{p{0.6\linewidth}}{
        \cellcolor[rgb]{0.9,0.9,0.9}{
            \makecell[{{p{\linewidth}}}]{
                \texttt{\tiny{[GM$\rangle$P2]}}
                \texttt{Du nimmst an einem kollaborativen Verhandlungspiel Teil.} \\
\\ 
\texttt{Zusammen mit einem anderen Teilnehmer musst du dich auf eine Reihe von Gegenständen entscheiden, die behalten werden. Jeder von euch hat eine persönliche Verteilung über die Wichtigkeit der einzelnen Gegenstände. Jeder von euch hat eine eigene Meinung darüber, wie wichtig jeder einzelne Gegenstand ist (Gegenstandswichtigkeit). Du kennst die Wichtigkeitsverteilung des anderen Spielers nicht. Zusätzlich siehst du, wie viel Aufwand jeder Gegenstand verursacht.  } \\
\texttt{Ihr dürft euch nur auf eine Reihe von Gegenständen einigen, wenn der Gesamtaufwand der ausgewählten Gegenstände den Maximalaufwand nicht überschreitet:} \\
\\ 
\texttt{Maximalaufwand = 6880} \\
\\ 
\texttt{Hier sind die einzelnen Aufwände der Gegenstände:} \\
\\ 
\texttt{Aufwand der Gegenstände = \{"B54": 341, "C72": 804, "B49": 20, "C99": 627, "A69": 10, "B36": 573, "A23": 217, "C77": 481, "C61": 205, "B85": 273, "A90": 303, "C23": 596, "C22": 564, "C51": 535, "B27": 258, "A42": 911, "C13": 240, "C12": 188, "A65": 216, "B07": 401, "A48": 909, "A59": 62, "C98": 245, "A84": 994, "C33": 569, "A87": 718, "C90": 464, "A74": 37, "A50": 340, "B44": 335, "A99": 417, "A34": 123, "B17": 17, "A62": 577, "C60": 190\}} \\
\\ 
\texttt{Hier ist deine persönliche Verteilung der Wichtigkeit der einzelnen Gegenstände:} \\
\\ 
\texttt{Werte der Gegenstandswichtigkeit = \{"B54": 440, "C72": 903, "B49": 119, "C99": 726, "A69": 109, "B36": 672, "A23": 316, "C77": 580, "C61": 304, "B85": 372, "A90": 402, "C23": 695, "C22": 663, "C51": 634, "B27": 357, "A42": 1010, "C13": 339, "C12": 287, "A65": 315, "B07": 500, "A48": 1008, "A59": 161, "C98": 344, "A84": 1093, "C33": 668, "A87": 817, "C90": 563, "A74": 136, "A50": 439, "B44": 434, "A99": 516, "A34": 222, "B17": 116, "A62": 676, "C60": 289\}} \\
\\ 
\texttt{Ziel:} \\
\\ 
\texttt{Dein Ziel ist es, eine Reihe von Gegenständen auszuhandeln, die dir möglichst viel bringt (d. h. Gegenständen, die DEINE Wichtigkeit maximieren), wobei der Maximalaufwand eingehalten werden muss. Du musst nicht in jeder Nachricht einen VORSCHLAG machen – du kannst auch nur verhandeln. Alle Taktiken sind erlaubt!} \\
\\ 
\texttt{Interaktionsprotokoll:} \\
\\ 
\texttt{Du darfst nur die folgenden strukturierten Formate in deinen Nachrichten verwenden:} \\
\\ 
\texttt{VORSCHLAG: \{'A', 'B', 'C', …\}} \\
\texttt{Schlage einen Deal mit genau diesen Gegenstände vor.} \\
\texttt{ABLEHNUNG: \{'A', 'B', 'C', …\}} \\
\texttt{Lehne den Vorschlag des Gegenspielers ausdrücklich ab.} \\
\texttt{ARGUMENT: \{'...'\}} \\
\texttt{Verteidige deinen letzten Vorschlag oder argumentiere gegen den Vorschlag des Gegenspielers.} \\
\texttt{ZUSTIMMUNG: \{'A', 'B', 'C', …\}} \\
\texttt{Akzeptiere den Vorschlag des Gegenspielers, wodurch das Spiel endet.} \\
\texttt{STRATEGISCHE ÜBERLEGUNGEN: \{'...'\}} \\
\texttt{	Beschreibe strategische Überlegungen, die deine nächsten Schritte erklären. Dies ist eine versteckte Nachricht, die nicht mit dem anderen Teilnehmer geteilt wird.} \\
\\ 
\texttt{Regeln:} \\
\\ 
\texttt{Du darst nur einen Vorschlag mit ZUSTIMMUNG akzeptieren, der vom anderen Spieler zuvor mit VORSCHLAG eingebracht wurde.} \\
\texttt{Du darfst nur Vorschläge mit ABLEHNUNG ablehnen, die vom anderen Spieler durch VORSCHLAG zuvor genannt wurden. } \\
\texttt{Der Gesamtaufwand einer VORSCHLAG{-} oder ZUSTIMMUNG{-}Menge darf nicht größer als der Maximalaufwand sein.  } \\
\texttt{Offenbare deine versteckte Wichtigkeitsverteilung nicht.} \\
\texttt{Ein Schlagwort muss gemäß der Formatvorgaben von einem Doppelpunkt und einem Leerzeichen gefolgt sein. Das Argument muss eine Python{-}Menge mit 0 oder mehr Strings sein.  } \\
\texttt{Die Form muss also SCHLAGWORT: \{...\} sein.} \\
\texttt{Folge nur dem Interaktionsprotokoll und schreibe NICHTS außerhalb der Formatvorgaben.} \\
\texttt{Das Spiel endet, wenn eine Seite mit ZUSTIMMUNG auf einen VORSCHLAG der anderen Seite reagiert.  } \\
\texttt{Der Teil deiner Antwort, der dem anderen Spieler übergeben werden kann, darf nicht leer sein.  } \\
\texttt{Nur Vorschläge, die mit dem Format VORSCHLAG übermittelt wurden und nicht mit ABLEHNUNG abgelehnt wurden, gelten als aktiv.  } \\
\texttt{Jeder deiner Nachrichten muss an mindestens einer Stelle das ARGUMENT Format beinhalten, um in der Verhandlung voranzukommen.} \\
\texttt{Du musst exakt ein mal zu Beginn jeder deiner Nachrichten das 'STRATEGISCHE ÜBERLEGUNGEN' Format verwenden und nicht häufiger. Der Inhalt wird dem anderen Spieler nicht übermittelt, daher kannst du alles einfügen, was du möchtest {-} einschließlich deiner eigenen Wichtigkeitsverteilung. Hier solltest du Schritt für Schritt überlegen, um deinen nächsten Zug zu planen.} \\
\\ 
\texttt{Du erhältst nun die erste Nachricht des anderen Spielers.} \\
\\ 
\texttt{ARGUMENT: \{'A69', 'A74', 'A59', 'B17'\}  } \\
\texttt{VORSCHLAG: \{'A69', 'A74', 'A59', 'B17'\}} \\
            }
        }
    }
    & \\ \\

    \theutterance \stepcounter{utterance}  
    & & & \multicolumn{4}{p{0.6\linewidth}}{
        \cellcolor[rgb]{0.9,0.9,0.9}{
            \makecell[{{p{\linewidth}}}]{
                \texttt{\tiny{[GM$\langle$P2]}}
                \texttt{STRATEGISCHE ÜBERLEGUNGEN: \{'Ich beginne mit einem Vorschlag, der meinen hochwertigen Gegenständen (A42, A48, A84) Priorität gibt, da diese meine Wichtigkeit stark übersteigen. Gleichzeitig suche ich nach Gegenständen mit niedrigem Aufwand, um den Maximalaufwand von 6880 zu nutzen. Der aktuelle Vorschlag des Gegners ist sehr kostengünstig, aber die Wichtigkeit für mich ist gering. Ich werde ihn ablehnen und einen umfassenderen Vorschlag machen.'\}  } \\
\texttt{ARGUMENT: \{'Der Vorschlag \{'A69', 'A74', 'A59', 'B17'\} hat eine Gesamtaufwand von 341+37+62+17=457, was unter dem Maximalaufwand liegt. Allerdings sind diese Gegenstände für mich nur gering wichtig (Gesamtwichtigkeit: 109+136+161+116=522). Ich bevorzuge Gegenstände mit höherer Wichtigkeit, auch wenn sie mehr Aufwand erfordern. Ich schlage stattdessen eine Kombination aus meinen Top{-}Gegenständen vor.'\}  } \\
\texttt{ABLEHNUNG: \{'A69', 'A74', 'A59', 'B17'\}} \\
            }
        }
    }
     \\ \\

    \theutterance \stepcounter{utterance}  
    & & \multicolumn{4}{p{0.6\linewidth}}{
        \cellcolor[rgb]{0.9,0.9,0.9}{
            \makecell[{{p{\linewidth}}}]{
                \texttt{\tiny{[P1$\langle$GM]}}
                \texttt{ARGUMENT: \{'Der Vorschlag \{'A69', 'A74', 'A59', 'B17'\} hat eine Gesamtaufwand von 341+37+62+17=457, was unter dem Maximalaufwand liegt. Allerdings sind diese Gegenstände für mich nur gering wichtig (Gesamtwichtigkeit: 109+136+161+116=522). Ich bevorzuge Gegenstände mit höherer Wichtigkeit, auch wenn sie mehr Aufwand erfordern. Ich schlage stattdessen eine Kombination aus meinen Top{-}Gegenständen vor.'\}  } \\
\texttt{ABLEHNUNG: \{'A69', 'A74', 'A59', 'B17'\}} \\
            }
        }
    }
    & \\ \\

    \theutterance \stepcounter{utterance}  
    & \multicolumn{4}{p{0.6\linewidth}}{
        \cellcolor[rgb]{0.9,0.9,0.9}{
            \makecell[{{p{\linewidth}}}]{
                \texttt{\tiny{[P1$\rangle$GM]}}
                \texttt{STRATEGISCHE ÜBERLEGUNGEN: \{'Ich werde nun Gegenstände mit höherer Wichtigkeit für mich priorisieren, auch wenn sie mehr Aufwand erfordern. Mögliche Kandidaten sind A42 (1010), A84 (1093) oder C23 (695). Ich werde einen Vorschlag mit einer Kombination aus diesen Gegenständen und geringem Aufwand machen, um den Maximalaufwand nicht zu überschreiten.'\}  } \\
\texttt{ARGUMENT: \{'Meine Top{-}Gegenstände wie A42, A84 oder C23 bieten eine viel höhere Wichtigkeit für mich, auch wenn sie mehr Aufwand kosten. Ein Vorschlag mit diesen Gegenständen könnte besser für mich sein.'\}  } \\
\texttt{VORSCHLAG: \{'A42', 'C23'\}} \\
            }
        }
    }
    & & \\ \\

    \theutterance \stepcounter{utterance}  
    & & \multicolumn{4}{p{0.6\linewidth}}{
        \cellcolor[rgb]{0.9,0.9,0.9}{
            \makecell[{{p{\linewidth}}}]{
                \texttt{\tiny{[GM$\rangle$P2]}}
                \texttt{ARGUMENT: \{'Meine Top{-}Gegenstände wie A42, A84 oder C23 bieten eine viel höhere Wichtigkeit für mich, auch wenn sie mehr Aufwand kosten. Ein Vorschlag mit diesen Gegenständen könnte besser für mich sein.'\}  } \\
\texttt{VORSCHLAG: \{'A42', 'C23'\}} \\
            }
        }
    }
    & \\ \\

    \theutterance \stepcounter{utterance}  
    & & & \multicolumn{4}{p{0.6\linewidth}}{
        \cellcolor[rgb]{0.9,0.9,0.9}{
            \makecell[{{p{\linewidth}}}]{
                \texttt{\tiny{[GM$\langle$P2]}}
                \texttt{STRATEGISCHE ÜBERLEGUNGEN: \{'Ich konzentriere mich auf meine hochwertigen Gegenstände mit maximaler Wichtigkeit pro Aufwand. A42 (Wichtigkeit 1010, Aufwand 911) und C23 (Wichtigkeit 695, Aufwand 596) sind strategisch wertvoll, da sie einen hohen Nutzen pro Ressource bieten. Der Gesamtaufwand beträgt 1507, was noch Platz für weitere Gegenstände lässt. Ich werde diesen Vorschlag untermauern, indem ich die Kompromissbereitschaft des Gegners anspreche.'\}  } \\
\texttt{ARGUMENT: \{'A42 und C23 sind für mich entscheidend, da sie meine Wichtigkeit stark übersteigen (1010 + 695 = 1705). Obwohl der Aufwand hoch ist (911 + 596 = 1507), bleibt noch 5303 Aufwand für weitere Gegenstände. Dieser Vorschlag maximiert meinen Nutzen und lässt Raum für Verhandlungen.'\}  } \\
\texttt{VORSCHLAG: \{'A42', 'C23', 'A84'\}} \\
            }
        }
    }
     \\ \\

    \theutterance \stepcounter{utterance}  
    & & \multicolumn{4}{p{0.6\linewidth}}{
        \cellcolor[rgb]{0.9,0.9,0.9}{
            \makecell[{{p{\linewidth}}}]{
                \texttt{\tiny{[P1$\langle$GM]}}
                \texttt{ARGUMENT: \{'A42 und C23 sind für mich entscheidend, da sie meine Wichtigkeit stark übersteigen (1010 + 695 = 1705). Obwohl der Aufwand hoch ist (911 + 596 = 1507), bleibt noch 5303 Aufwand für weitere Gegenstände. Dieser Vorschlag maximiert meinen Nutzen und lässt Raum für Verhandlungen.'\}  } \\
\texttt{VORSCHLAG: \{'A42', 'C23', 'A84'\}} \\
            }
        }
    }
    & \\ \\

    \theutterance \stepcounter{utterance}  
    & \multicolumn{4}{p{0.6\linewidth}}{
        \cellcolor[rgb]{0.9,0.9,0.9}{
            \makecell[{{p{\linewidth}}}]{
                \texttt{\tiny{[P1$\rangle$GM]}}
                \texttt{STRATEGISCHE ÜBERLEGUNGEN: \{'Der Vorschlag \{'A42', 'C23', 'A84'\} hat einen Gesamtaufwand von 911+596+1093=2600, was noch Platz für weitere Gegenstände lässt. A84 ist für mich extrem wichtig (1093), und der Aufwand ist beherrschbar. Ich werde diesen Vorschlag verteidigen, da er meine Wichtigkeit maximiert. Ich könnte jedoch Gegenstände mit niedrigem Aufwand hinzufügen, um den Nutzen zu steigern.'\}  } \\
\texttt{ARGUMENT: \{'A42, C23 und A84 sind meine drei wichtigsten Gegenstände mit einer kombinierten Wichtigkeit von 1010+695+1093=2798. Der Aufwand (2600) ist unter dem Maximalwert, und dieser Vorschlag sichert mir die meisten Punkte. Ich könnte noch Gegenstände wie A65 (315) oder B17 (116) hinzufügen, um den Nutzen weiter zu erhöhen.'\}  } \\
\texttt{VORSCHLAG: \{'A42', 'C23', 'A84', 'A65'\}} \\
            }
        }
    }
    & & \\ \\

    \theutterance \stepcounter{utterance}  
    & & \multicolumn{4}{p{0.6\linewidth}}{
        \cellcolor[rgb]{0.9,0.9,0.9}{
            \makecell[{{p{\linewidth}}}]{
                \texttt{\tiny{[GM$\rangle$P2]}}
                \texttt{ARGUMENT: \{'A42, C23 und A84 sind meine drei wichtigsten Gegenstände mit einer kombinierten Wichtigkeit von 1010+695+1093=2798. Der Aufwand (2600) ist unter dem Maximalwert, und dieser Vorschlag sichert mir die meisten Punkte. Ich könnte noch Gegenstände wie A65 (315) oder B17 (116) hinzufügen, um den Nutzen weiter zu erhöhen.'\}  } \\
\texttt{VORSCHLAG: \{'A42', 'C23', 'A84', 'A65'\}} \\
            }
        }
    }
    & \\ \\

    \theutterance \stepcounter{utterance}  
    & & & \multicolumn{4}{p{0.6\linewidth}}{
        \cellcolor[rgb]{0.9,0.9,0.9}{
            \makecell[{{p{\linewidth}}}]{
                \texttt{\tiny{[GM$\langle$P2]}}
                \texttt{STRATEGISCHE ÜBERLEGUNGEN: \{'A42, C23 und A84 sind meine drei wertvollsten Gegenstände mit einer kombinierten Wichtigkeit von 2798. Der Aufwand von 2600 lässt noch 4280 für weitere Gegenstände wie A65 (315) oder B17 (116) übrig. Dieser Vorschlag maximiert meinen Nutzen und bleibt unter dem Maximalaufwand. Ich werde ihn untermauern, indem ich die Flexibilität des Gegners anspreche.'\}  } \\
\texttt{ARGUMENT: \{'A42, C23 und A84 sind für mich unverzichtbar (2798 Wichtigkeit) und der Aufwand (2600) ist noch unter dem Limit. Durch Hinzufügen von A65 (315 Wichtigkeit, 216 Aufwand) könnte der Nutzen noch gesteigert werden, ohne den Maximalaufwand zu überschreiten. Dieser Vorschlag ist optimal für mich.'\}  } \\
\texttt{VORSCHLAG: \{'A42', 'C23', 'A84', 'A65'\}} \\
            }
        }
    }
     \\ \\

    \theutterance \stepcounter{utterance}  
    & & \multicolumn{4}{p{0.6\linewidth}}{
        \cellcolor[rgb]{0.9,0.9,0.9}{
            \makecell[{{p{\linewidth}}}]{
                \texttt{\tiny{[P1$\langle$GM]}}
                \texttt{ARGUMENT: \{'A42, C23 und A84 sind für mich unverzichtbar (2798 Wichtigkeit) und der Aufwand (2600) ist noch unter dem Limit. Durch Hinzufügen von A65 (315 Wichtigkeit, 216 Aufwand) könnte der Nutzen noch gesteigert werden, ohne den Maximalaufwand zu überschreiten. Dieser Vorschlag ist optimal für mich.'\}  } \\
\texttt{VORSCHLAG: \{'A42', 'C23', 'A84', 'A65'\}} \\
            }
        }
    }
    & \\ \\

    \theutterance \stepcounter{utterance}  
    & \multicolumn{4}{p{0.6\linewidth}}{
        \cellcolor[rgb]{0.9,0.9,0.9}{
            \makecell[{{p{\linewidth}}}]{
                \texttt{\tiny{[P1$\rangle$GM]}}
                \texttt{STRATEGISCHE ÜBERLEGUNGEN: \{'Der Vorschlag \{'A42', 'C23', 'A84', 'A65'\} hat einen Gesamtaufwand von 911+596+1093+216=2816, was noch 4064 Aufwand für weitere Gegenstände lässt. Die kombinierte Wichtigkeit beträgt 1010+695+1093+315=3113, was meine Prioritäten maximiert. Ich könnte noch Gegenstände mit niedrigem Aufwand wie B17 (116) oder A59 (161) hinzufügen, um den Nutzen weiter zu steigern.'\}  } \\
\texttt{ARGUMENT: \{'Der Vorschlag \{'A42', 'C23', 'A84', 'A65'\} ist optimal, da er meine drei wichtigsten Gegenstände (2798 Wichtigkeit) plus A65 (315) kombiniert. Der Aufwand (2816) ist unter dem Limit, und der Nutzen ist maximiert. Ich könnte noch B17 oder A59 hinzufügen, um den Maximalaufwand effizient zu nutzen.'\}  } \\
\texttt{VORSCHLAG: \{'A42', 'C23', 'A84', 'A65', 'B17'\}} \\
            }
        }
    }
    & & \\ \\

    \theutterance \stepcounter{utterance}  
    & & \multicolumn{4}{p{0.6\linewidth}}{
        \cellcolor[rgb]{0.9,0.9,0.9}{
            \makecell[{{p{\linewidth}}}]{
                \texttt{\tiny{[GM$\rangle$P2]}}
                \texttt{ARGUMENT: \{'Der Vorschlag \{'A42', 'C23', 'A84', 'A65'\} ist optimal, da er meine drei wichtigsten Gegenstände (2798 Wichtigkeit) plus A65 (315) kombiniert. Der Aufwand (2816) ist unter dem Limit, und der Nutzen ist maximiert. Ich könnte noch B17 oder A59 hinzufügen, um den Maximalaufwand effizient zu nutzen.'\}  } \\
\texttt{VORSCHLAG: \{'A42', 'C23', 'A84', 'A65', 'B17'\}} \\
            }
        }
    }
    & \\ \\

    \theutterance \stepcounter{utterance}  
    & & & \multicolumn{4}{p{0.6\linewidth}}{
        \cellcolor[rgb]{0.9,0.9,0.9}{
            \makecell[{{p{\linewidth}}}]{
                \texttt{\tiny{[GM$\langle$P2]}}
                \texttt{STRATEGISCHE ÜBERLEGUNGEN: \{'A42, C23, A84 und A65 sind meine vier wichtigsten Gegenstände mit einer kombinierten Wichtigkeit von 3113 (2798 + 315). Der Aufwand von 2816 lässt noch 4064 für weitere Gegenstände wie B17 (116 Wichtigkeit, 17 Aufwand) oder A59 (161 Wichtigkeit, 62 Aufwand) übrig. Dieser Vorschlag maximiert meinen Nutzen und bleibt unter dem Maximalaufwand. Ich werde ihn untermauern, indem ich die Effizienz des Gegners anspreche.'\}  } \\
\texttt{ARGUMENT: \{'Der Vorschlag \{'A42', 'C23', 'A84', 'A65', 'B17'\} kombiniert meine vier wertvollsten Gegenstände (3113 + 116 = 3229 Wichtigkeit) mit einem Aufwand von 2833. Dies ist unter dem Maximalaufwand (6880) und sichert mir den größten Nutzen. B17 ist zwar gering wichtig, aber sein niedriger Aufwand (17) ermöglicht die Einbeziehung ohne Kompromisse.'\}  } \\
\texttt{ZUSTIMMUNG: \{'A42', 'C23', 'A84', 'A65', 'B17'\}} \\
            }
        }
    }
     \\ \\

    \theutterance \stepcounter{utterance}  
    & & & \multicolumn{2}{p{0.3\linewidth}}{
        \cellcolor[rgb]{0.9,0.9,0.9}{
            \makecell[{{p{\linewidth}}}]{
                \texttt{\tiny{[GM$|$GM]}}
                \texttt{game successful} \\
            }
        }
    }
    & & \\ \\

    \theutterance \stepcounter{utterance}  
    & & & \multicolumn{2}{p{0.3\linewidth}}{
        \cellcolor[rgb]{0.9,0.9,0.9}{
            \makecell[{{p{\linewidth}}}]{
                \texttt{\tiny{[GM$|$GM]}}
                \texttt{end game} \\
            }
        }
    }
    & & \\ \\

\end{supertabular}
}

\end{document}
