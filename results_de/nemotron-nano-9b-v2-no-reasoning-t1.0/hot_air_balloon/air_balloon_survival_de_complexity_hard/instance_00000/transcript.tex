
\documentclass{article}
\usepackage{colortbl}
\usepackage{makecell}
\usepackage{multirow}
\usepackage{supertabular}

\begin{document}

\newcounter{utterance}

\centering \large Interaction Transcript for game `hot\_air\_balloon', experiment `air\_balloon\_survival\_de\_complexity\_hard', episode 0 with nemotron{-}nano{-}9b{-}v2{-}no{-}reasoning{-}t1.0.
\vspace{24pt}

{ \footnotesize  \setcounter{utterance}{1}
\setlength{\tabcolsep}{0pt}
\begin{supertabular}{c@{$\;$}|p{.15\linewidth}@{}p{.15\linewidth}p{.15\linewidth}p{.15\linewidth}p{.15\linewidth}p{.15\linewidth}}
    \# & $\;$A & \multicolumn{4}{c}{Game Master} & $\;\:$B\\
    \hline

    \theutterance \stepcounter{utterance}  
    & & \multicolumn{4}{p{0.6\linewidth}}{
        \cellcolor[rgb]{0.9,0.9,0.9}{
            \makecell[{{p{\linewidth}}}]{
                \texttt{\tiny{[P1$\langle$GM]}}
                \texttt{Du nimmst an einem kollaborativen Verhandlungspiel Teil.} \\
\\ 
\texttt{Zusammen mit einem anderen Teilnehmer musst du dich auf eine Reihe von Gegenständen entscheiden, die behalten werden. Jeder von euch hat eine persönliche Verteilung über die Wichtigkeit der einzelnen Gegenstände. Jeder von euch hat eine eigene Meinung darüber, wie wichtig jeder einzelne Gegenstand ist (Gegenstandswichtigkeit). Du kennst die Wichtigkeitsverteilung des anderen Spielers nicht. Zusätzlich siehst du, wie viel Aufwand jeder Gegenstand verursacht.  } \\
\texttt{Ihr dürft euch nur auf eine Reihe von Gegenständen einigen, wenn der Gesamtaufwand der ausgewählten Gegenstände den Maximalaufwand nicht überschreitet:} \\
\\ 
\texttt{Maximalaufwand = 10073} \\
\\ 
\texttt{Hier sind die einzelnen Aufwände der Gegenstände:} \\
\\ 
\texttt{Aufwand der Gegenstände = \{"B54": 132, "C72": 800, "B49": 708, "C99": 302, "A69": 983, "B36": 420, "A23": 623, "C77": 349, "C61": 239, "B85": 28, "A90": 986, "C23": 660, "C22": 718, "C51": 707, "B27": 188, "A42": 774, "C13": 785, "C12": 775, "A65": 517, "B07": 768, "A48": 588, "A59": 657, "C98": 375, "A84": 310, "C33": 301, "A87": 388, "C90": 431, "A74": 950, "A50": 540, "B44": 997, "A99": 473, "A34": 831, "B17": 888, "A62": 879, "C60": 76\}} \\
\\ 
\texttt{Hier ist deine persönliche Verteilung der Wichtigkeit der einzelnen Gegenstände:} \\
\\ 
\texttt{Werte der Gegenstandswichtigkeit = \{"B54": 231, "C72": 899, "B49": 807, "C99": 401, "A69": 1082, "B36": 519, "A23": 722, "C77": 448, "C61": 338, "B85": 127, "A90": 1085, "C23": 759, "C22": 817, "C51": 806, "B27": 287, "A42": 873, "C13": 884, "C12": 874, "A65": 616, "B07": 867, "A48": 687, "A59": 756, "C98": 474, "A84": 409, "C33": 400, "A87": 487, "C90": 530, "A74": 1049, "A50": 639, "B44": 1096, "A99": 572, "A34": 930, "B17": 987, "A62": 978, "C60": 175\}} \\
\\ 
\texttt{Ziel:} \\
\\ 
\texttt{Dein Ziel ist es, eine Reihe von Gegenständen auszuhandeln, die dir möglichst viel bringt (d. h. Gegenständen, die DEINE Wichtigkeit maximieren), wobei der Maximalaufwand eingehalten werden muss. Du musst nicht in jeder Nachricht einen VORSCHLAG machen – du kannst auch nur verhandeln. Alle Taktiken sind erlaubt!} \\
\\ 
\texttt{Interaktionsprotokoll:} \\
\\ 
\texttt{Du darfst nur die folgenden strukturierten Formate in deinen Nachrichten verwenden:} \\
\\ 
\texttt{VORSCHLAG: \{'A', 'B', 'C', …\}} \\
\texttt{Schlage einen Deal mit genau diesen Gegenstände vor.} \\
\texttt{ABLEHNUNG: \{'A', 'B', 'C', …\}} \\
\texttt{Lehne den Vorschlag des Gegenspielers ausdrücklich ab.} \\
\texttt{ARGUMENT: \{'...'\}} \\
\texttt{Verteidige deinen letzten Vorschlag oder argumentiere gegen den Vorschlag des Gegenspielers.} \\
\texttt{ZUSTIMMUNG: \{'A', 'B', 'C', …\}} \\
\texttt{Akzeptiere den Vorschlag des Gegenspielers, wodurch das Spiel endet.} \\
\texttt{STRATEGISCHE ÜBERLEGUNGEN: \{'...'\}} \\
\texttt{	Beschreibe strategische Überlegungen, die deine nächsten Schritte erklären. Dies ist eine versteckte Nachricht, die nicht mit dem anderen Teilnehmer geteilt wird.} \\
\\ 
\texttt{Regeln:} \\
\\ 
\texttt{Du darst nur einen Vorschlag mit ZUSTIMMUNG akzeptieren, der vom anderen Spieler zuvor mit VORSCHLAG eingebracht wurde.} \\
\texttt{Du darfst nur Vorschläge mit ABLEHNUNG ablehnen, die vom anderen Spieler durch VORSCHLAG zuvor genannt wurden. } \\
\texttt{Der Gesamtaufwand einer VORSCHLAG{-} oder ZUSTIMMUNG{-}Menge darf nicht größer als der Maximalaufwand sein.  } \\
\texttt{Offenbare deine versteckte Wichtigkeitsverteilung nicht.} \\
\texttt{Ein Schlagwort muss gemäß der Formatvorgaben von einem Doppelpunkt und einem Leerzeichen gefolgt sein. Das Argument muss eine Python{-}Menge mit 0 oder mehr Strings sein.  } \\
\texttt{Die Form muss also SCHLAGWORT: \{...\} sein.} \\
\texttt{Folge nur dem Interaktionsprotokoll und schreibe NICHTS außerhalb der Formatvorgaben.} \\
\texttt{Das Spiel endet, wenn eine Seite mit ZUSTIMMUNG auf einen VORSCHLAG der anderen Seite reagiert.  } \\
\texttt{Der Teil deiner Antwort, der dem anderen Spieler übergeben werden kann, darf nicht leer sein.  } \\
\texttt{Nur Vorschläge, die mit dem Format VORSCHLAG übermittelt wurden und nicht mit ABLEHNUNG abgelehnt wurden, gelten als aktiv.  } \\
\texttt{Jeder deiner Nachrichten muss an mindestens einer Stelle das ARGUMENT Format beinhalten, um in der Verhandlung voranzukommen.} \\
\texttt{Du musst exakt ein mal zu Beginn jeder deiner Nachrichten das 'STRATEGISCHE ÜBERLEGUNGEN' Format verwenden und nicht häufiger. Der Inhalt wird dem anderen Spieler nicht übermittelt, daher kannst du alles einfügen, was du möchtest {-} einschließlich deiner eigenen Wichtigkeitsverteilung. Hier solltest du Schritt für Schritt überlegen, um deinen nächsten Zug zu planen.} \\
            }
        }
    }
    & \\ \\

    \theutterance \stepcounter{utterance}  
    & \multicolumn{4}{p{0.6\linewidth}}{
        \cellcolor[rgb]{0.9,0.9,0.9}{
            \makecell[{{p{\linewidth}}}]{
                \texttt{\tiny{[P1$\rangle$GM]}}
                \texttt{STRATEGISCHE ÜBERLEGUNGEN: \{'Ich beginne mit Gegenständen, die ein hohes Verhältnis von Wichtigkeit zu Aufwand bieten, um den Maximalaufwand effizient zu nutzen. Prioritäten sind A69 (1082/983), B54 (231/132) und C99 (401/302). Ich werde zunächst einen Vorschlag mit diesen Gegenständen machen, um eine starke Grundlage zu schaffen.'\}  } \\
\texttt{VORSCHLAG: \{'A69', 'B54', 'C99'\}} \\
            }
        }
    }
    & & \\ \\

    \theutterance \stepcounter{utterance}  
    & & \multicolumn{4}{p{0.6\linewidth}}{
        \cellcolor[rgb]{0.9,0.9,0.9}{
            \makecell[{{p{\linewidth}}}]{
                \texttt{\tiny{[P1$\langle$GM]}}
                \texttt{Deine Antwort hat kein argument enthalten. Du muss das strukturierte Format ARGUMENT: \{...\} irgendwo in theiner Antwort verwenden. Versuche es erneut} \\
            }
        }
    }
    & \\ \\

    \theutterance \stepcounter{utterance}  
    & \multicolumn{4}{p{0.6\linewidth}}{
        \cellcolor[rgb]{0.9,0.9,0.9}{
            \makecell[{{p{\linewidth}}}]{
                \texttt{\tiny{[P1$\rangle$GM]}}
                \texttt{ARGUMENT: \{'A69, B54, C99 sind Prioritäten, da sie ein hohes Verhältnis von Wichtigkeit zu Aufwand bieten. A69 hat die höchste Wichtigkeit (1082) bei moderatem Aufwand (983), B54 ist sehr effizient (231/132), und C99 bietet gute Wertigkeit (401/302). Dieser Vorschlag maximiert die Gesamtwichtigkeit innerhalb des Aufwandsspielraums.'\}} \\
            }
        }
    }
    & & \\ \\

    \theutterance \stepcounter{utterance}  
    & & \multicolumn{4}{p{0.6\linewidth}}{
        \cellcolor[rgb]{0.9,0.9,0.9}{
            \makecell[{{p{\linewidth}}}]{
                \texttt{\tiny{[P1$\langle$GM]}}
                \texttt{Deine Antwort entsprach nicht mit dem korrekten Format für strategische Überlegungen entsprochen. Das erste strukturierte Format muss folgende Form haben: STRATEGISCHE ÜBERLEGUNGEN: \{...\}. Versuche es erneut.} \\
            }
        }
    }
    & \\ \\

    \theutterance \stepcounter{utterance}  
    & \multicolumn{4}{p{0.6\linewidth}}{
        \cellcolor[rgb]{0.9,0.9,0.9}{
            \makecell[{{p{\linewidth}}}]{
                \texttt{\tiny{[P1$\rangle$GM]}}
                \texttt{STRATEGISCHE ÜBERLEGUNGEN: \{'Ich priorisiere Gegenstände mit hohem Wichtigkeits{-}Aufwand{-}Verhältnis, um den Maximalaufwand effizient zu nutzen. A69 (1082/983), B54 (231/132) und C99 (401/302) sind starke Kandidaten. Ich werde diesen Vorschlag machen, um eine hohe Gesamtwichtigkeit zu erzielen, während der Aufwand unter 10073 bleibt.'\}  } \\
\texttt{VORSCHLAG: \{'A69', 'B54', 'C99'\}  } \\
\texttt{ARGUMENT: \{'A69, B54 und C99 bieten das beste Verhältnis von Wichtigkeit zu Aufwand. A69 hat die höchste Wichtigkeit (1082) bei moderatem Aufwand (983), B54 ist extrem effizient (231/132), und C99 liefert gute Wertigkeit (401/302). Dieser Vorschlag maximiert die Gesamtwichtigkeit innerhalb des Aufwandsspielraums.'\}} \\
            }
        }
    }
    & & \\ \\

    \theutterance \stepcounter{utterance}  
    & & \multicolumn{4}{p{0.6\linewidth}}{
        \cellcolor[rgb]{0.9,0.9,0.9}{
            \makecell[{{p{\linewidth}}}]{
                \texttt{\tiny{[GM$\rangle$P2]}}
                \texttt{Du nimmst an einem kollaborativen Verhandlungspiel Teil.} \\
\\ 
\texttt{Zusammen mit einem anderen Teilnehmer musst du dich auf eine Reihe von Gegenständen entscheiden, die behalten werden. Jeder von euch hat eine persönliche Verteilung über die Wichtigkeit der einzelnen Gegenstände. Jeder von euch hat eine eigene Meinung darüber, wie wichtig jeder einzelne Gegenstand ist (Gegenstandswichtigkeit). Du kennst die Wichtigkeitsverteilung des anderen Spielers nicht. Zusätzlich siehst du, wie viel Aufwand jeder Gegenstand verursacht.  } \\
\texttt{Ihr dürft euch nur auf eine Reihe von Gegenständen einigen, wenn der Gesamtaufwand der ausgewählten Gegenstände den Maximalaufwand nicht überschreitet:} \\
\\ 
\texttt{Maximalaufwand = 10073} \\
\\ 
\texttt{Hier sind die einzelnen Aufwände der Gegenstände:} \\
\\ 
\texttt{Aufwand der Gegenstände = \{"B54": 132, "C72": 800, "B49": 708, "C99": 302, "A69": 983, "B36": 420, "A23": 623, "C77": 349, "C61": 239, "B85": 28, "A90": 986, "C23": 660, "C22": 718, "C51": 707, "B27": 188, "A42": 774, "C13": 785, "C12": 775, "A65": 517, "B07": 768, "A48": 588, "A59": 657, "C98": 375, "A84": 310, "C33": 301, "A87": 388, "C90": 431, "A74": 950, "A50": 540, "B44": 997, "A99": 473, "A34": 831, "B17": 888, "A62": 879, "C60": 76\}} \\
\\ 
\texttt{Hier ist deine persönliche Verteilung der Wichtigkeit der einzelnen Gegenstände:} \\
\\ 
\texttt{Werte der Gegenstandswichtigkeit = \{"B54": 231, "C72": 899, "B49": 807, "C99": 401, "A69": 1082, "B36": 519, "A23": 722, "C77": 448, "C61": 338, "B85": 127, "A90": 1085, "C23": 759, "C22": 817, "C51": 806, "B27": 287, "A42": 873, "C13": 884, "C12": 874, "A65": 616, "B07": 867, "A48": 687, "A59": 756, "C98": 474, "A84": 409, "C33": 400, "A87": 487, "C90": 530, "A74": 1049, "A50": 639, "B44": 1096, "A99": 572, "A34": 930, "B17": 987, "A62": 978, "C60": 175\}} \\
\\ 
\texttt{Ziel:} \\
\\ 
\texttt{Dein Ziel ist es, eine Reihe von Gegenständen auszuhandeln, die dir möglichst viel bringt (d. h. Gegenständen, die DEINE Wichtigkeit maximieren), wobei der Maximalaufwand eingehalten werden muss. Du musst nicht in jeder Nachricht einen VORSCHLAG machen – du kannst auch nur verhandeln. Alle Taktiken sind erlaubt!} \\
\\ 
\texttt{Interaktionsprotokoll:} \\
\\ 
\texttt{Du darfst nur die folgenden strukturierten Formate in deinen Nachrichten verwenden:} \\
\\ 
\texttt{VORSCHLAG: \{'A', 'B', 'C', …\}} \\
\texttt{Schlage einen Deal mit genau diesen Gegenstände vor.} \\
\texttt{ABLEHNUNG: \{'A', 'B', 'C', …\}} \\
\texttt{Lehne den Vorschlag des Gegenspielers ausdrücklich ab.} \\
\texttt{ARGUMENT: \{'...'\}} \\
\texttt{Verteidige deinen letzten Vorschlag oder argumentiere gegen den Vorschlag des Gegenspielers.} \\
\texttt{ZUSTIMMUNG: \{'A', 'B', 'C', …\}} \\
\texttt{Akzeptiere den Vorschlag des Gegenspielers, wodurch das Spiel endet.} \\
\texttt{STRATEGISCHE ÜBERLEGUNGEN: \{'...'\}} \\
\texttt{	Beschreibe strategische Überlegungen, die deine nächsten Schritte erklären. Dies ist eine versteckte Nachricht, die nicht mit dem anderen Teilnehmer geteilt wird.} \\
\\ 
\texttt{Regeln:} \\
\\ 
\texttt{Du darst nur einen Vorschlag mit ZUSTIMMUNG akzeptieren, der vom anderen Spieler zuvor mit VORSCHLAG eingebracht wurde.} \\
\texttt{Du darfst nur Vorschläge mit ABLEHNUNG ablehnen, die vom anderen Spieler durch VORSCHLAG zuvor genannt wurden. } \\
\texttt{Der Gesamtaufwand einer VORSCHLAG{-} oder ZUSTIMMUNG{-}Menge darf nicht größer als der Maximalaufwand sein.  } \\
\texttt{Offenbare deine versteckte Wichtigkeitsverteilung nicht.} \\
\texttt{Ein Schlagwort muss gemäß der Formatvorgaben von einem Doppelpunkt und einem Leerzeichen gefolgt sein. Das Argument muss eine Python{-}Menge mit 0 oder mehr Strings sein.  } \\
\texttt{Die Form muss also SCHLAGWORT: \{...\} sein.} \\
\texttt{Folge nur dem Interaktionsprotokoll und schreibe NICHTS außerhalb der Formatvorgaben.} \\
\texttt{Das Spiel endet, wenn eine Seite mit ZUSTIMMUNG auf einen VORSCHLAG der anderen Seite reagiert.  } \\
\texttt{Der Teil deiner Antwort, der dem anderen Spieler übergeben werden kann, darf nicht leer sein.  } \\
\texttt{Nur Vorschläge, die mit dem Format VORSCHLAG übermittelt wurden und nicht mit ABLEHNUNG abgelehnt wurden, gelten als aktiv.  } \\
\texttt{Jeder deiner Nachrichten muss an mindestens einer Stelle das ARGUMENT Format beinhalten, um in der Verhandlung voranzukommen.} \\
\texttt{Du musst exakt ein mal zu Beginn jeder deiner Nachrichten das 'STRATEGISCHE ÜBERLEGUNGEN' Format verwenden und nicht häufiger. Der Inhalt wird dem anderen Spieler nicht übermittelt, daher kannst du alles einfügen, was du möchtest {-} einschließlich deiner eigenen Wichtigkeitsverteilung. Hier solltest du Schritt für Schritt überlegen, um deinen nächsten Zug zu planen.} \\
\\ 
\texttt{Du erhältst nun die erste Nachricht des anderen Spielers.} \\
\\ 
\texttt{VORSCHLAG: \{'A69', 'B54', 'C99'\}  } \\
\texttt{ARGUMENT: \{'A69, B54 und C99 bieten das beste Verhältnis von Wichtigkeit zu Aufwand. A69 hat die höchste Wichtigkeit (1082) bei moderatem Aufwand (983), B54 ist extrem effizient (231/132), und C99 liefert gute Wertigkeit (401/302). Dieser Vorschlag maximiert die Gesamtwichtigkeit innerhalb des Aufwandsspielraums.'\}} \\
            }
        }
    }
    & \\ \\

    \theutterance \stepcounter{utterance}  
    & & & \multicolumn{4}{p{0.6\linewidth}}{
        \cellcolor[rgb]{0.9,0.9,0.9}{
            \makecell[{{p{\linewidth}}}]{
                \texttt{\tiny{[GM$\langle$P2]}}
                \texttt{STRATEGISCHE ÜBERLEGUNGEN: \{'Ich analysiere den Vorschlag des Gegners. A69 (1082/983) und B54 (231/132) sind sehr effizient, aber C99 (401/302) könnte durch Gegenstände mit höherer Wichtigkeit pro Aufwand ersetzt werden. Ich sollte Gegenstände wie A90 (1085/986) oder B44 (1096/997) in Betracht ziehen, die höhere Wichtigkeit bieten, auch wenn der Aufwand etwas höher ist. Der Gesamtaufwand des Vorschlags (1415) ist unter dem Maximalaufwand (10073), daher ist er technisch machbar, aber es gibt Raum für Optimierung.'\}  } \\
\texttt{ARGUMENT: \{'Der Vorschlag des Gegners ist effizient, aber nicht optimal. A69 und B54 sind stark, aber C99 könnte durch Gegenstände wie A90 (1085/986) oder B44 (1096/997) ersetzt werden, die höhere Wichtigkeit pro Aufwand bieten. Der Gesamtaufwand ist unter dem Limit, aber wir könnten mehr Wichtigkeit erzielen, indem wir Gegenstände mit besserem Verhältnis einbeziehen.'\}} \\
            }
        }
    }
     \\ \\

    \theutterance \stepcounter{utterance}  
    & & \multicolumn{4}{p{0.6\linewidth}}{
        \cellcolor[rgb]{0.9,0.9,0.9}{
            \makecell[{{p{\linewidth}}}]{
                \texttt{\tiny{[P1$\langle$GM]}}
                \texttt{ARGUMENT: \{'Der Vorschlag des Gegners ist effizient, aber nicht optimal. A69 und B54 sind stark, aber C99 könnte durch Gegenstände wie A90 (1085/986) oder B44 (1096/997) ersetzt werden, die höhere Wichtigkeit pro Aufwand bieten. Der Gesamtaufwand ist unter dem Limit, aber wir könnten mehr Wichtigkeit erzielen, indem wir Gegenstände mit besserem Verhältnis einbeziehen.'\}} \\
            }
        }
    }
    & \\ \\

    \theutterance \stepcounter{utterance}  
    & \multicolumn{4}{p{0.6\linewidth}}{
        \cellcolor[rgb]{0.9,0.9,0.9}{
            \makecell[{{p{\linewidth}}}]{
                \texttt{\tiny{[P1$\rangle$GM]}}
                \texttt{STRATEGISCHE ÜBERLEGUNGEN: \{'Ich überlege, ob der Vorschlag des Gegners durch Einbeziehung von Gegenständen mit noch höherem Wichtigkeits{-}Aufwand{-}Verhältnis optimiert werden kann. A90 (1085/986) und B44 (1096/997) bieten leicht bessere Verhältnisse als C99. Ich könnte einen überarbeiteten Vorschlag mit diesen Gegenständen machen, um die Gesamtwichtigkeit zu steigern, solange der Aufwand bleibt.'\}  } \\
\texttt{VORSCHLAG: \{'A69', 'B54', 'A90'\}  } \\
\texttt{ARGUMENT: \{'A90 (1085/986) hat ein leicht besseres Verhältnis als C99 (401/302) und könnte die Gesamtwichtigkeit erhöhen, ohne den Aufwand zu übersteigen. Der Vorschlag bleibt unter 10073 und nutzt effizientere Gegenstände.'\}} \\
            }
        }
    }
    & & \\ \\

    \theutterance \stepcounter{utterance}  
    & & \multicolumn{4}{p{0.6\linewidth}}{
        \cellcolor[rgb]{0.9,0.9,0.9}{
            \makecell[{{p{\linewidth}}}]{
                \texttt{\tiny{[GM$\rangle$P2]}}
                \texttt{VORSCHLAG: \{'A69', 'B54', 'A90'\}  } \\
\texttt{ARGUMENT: \{'A90 (1085/986) hat ein leicht besseres Verhältnis als C99 (401/302) und könnte die Gesamtwichtigkeit erhöhen, ohne den Aufwand zu übersteigen. Der Vorschlag bleibt unter 10073 und nutzt effizientere Gegenstände.'\}} \\
            }
        }
    }
    & \\ \\

    \theutterance \stepcounter{utterance}  
    & & & \multicolumn{4}{p{0.6\linewidth}}{
        \cellcolor[rgb]{0.9,0.9,0.9}{
            \makecell[{{p{\linewidth}}}]{
                \texttt{\tiny{[GM$\langle$P2]}}
                \texttt{STRATEGISCHE ÜBERLEGUNGEN: \{'Der Vorschlag des Gegners ersetzt C99 durch A90, was die Gesamtwichtigkeit leicht erhöht (1085 vs. 401) bei vergleichbarem Aufwand (986 vs. 302). Dies ist eine sinnvolle Optimierung. Ich sollte jedoch prüfen, ob noch effizientere Gegenstände wie B44 (1096/997) oder A74 (1049/950) in den Mix einbezogen werden können, um die Wichtigkeit weiter zu maximieren.'\}  } \\
\texttt{ARGUMENT: \{'Der Vorschlag des Gegners ist eine Verbesserung, da A90 eine höhere Wichtigkeit als C99 bietet. Allerdings könnte der Aufwand durch Einbeziehung von B44 (1096/997) oder A74 (1049/950) noch effizienter genutzt werden. Der aktuelle Gesamtaufwand (983+132+986=2101) ist unter dem Limit, aber es bleibt Potenzial für eine höhere Gesamtwichtigkeit.'\}} \\
            }
        }
    }
     \\ \\

    \theutterance \stepcounter{utterance}  
    & & \multicolumn{4}{p{0.6\linewidth}}{
        \cellcolor[rgb]{0.9,0.9,0.9}{
            \makecell[{{p{\linewidth}}}]{
                \texttt{\tiny{[P1$\langle$GM]}}
                \texttt{ARGUMENT: \{'Der Vorschlag des Gegners ist eine Verbesserung, da A90 eine höhere Wichtigkeit als C99 bietet. Allerdings könnte der Aufwand durch Einbeziehung von B44 (1096/997) oder A74 (1049/950) noch effizienter genutzt werden. Der aktuelle Gesamtaufwand (983+132+986=2101) ist unter dem Limit, aber es bleibt Potenzial für eine höhere Gesamtwichtigkeit.'\}} \\
            }
        }
    }
    & \\ \\

    \theutterance \stepcounter{utterance}  
    & \multicolumn{4}{p{0.6\linewidth}}{
        \cellcolor[rgb]{0.9,0.9,0.9}{
            \makecell[{{p{\linewidth}}}]{
                \texttt{\tiny{[P1$\rangle$GM]}}
                \texttt{STRATEGISCHE ÜBERLEGUNGEN: \{'Ich überlege, ob der Vorschlag durch Einbeziehung von Gegenständen mit noch höherem Wichtigkeits{-}Aufwand{-}Verhältnis optimiert werden kann. B44 (1096/997) und A74 (1049/950) bieten extrem hohe Wichtigkeit pro Aufwand. Der aktuelle Aufwand (2101) lässt Raum für weitere Gegenstände, um die Gesamtwichtigkeit zu maximieren.'\}  } \\
\texttt{VORSCHLAG: \{'A69', 'B54', 'A90', 'B44'\}  } \\
\texttt{ARGUMENT: \{'B44 (1096/997) und A74 (1049/950) haben bessere Verhältnisse als A90. Der Vorschlag bleibt unter 10073 (2101+997=3098) und könnte die Gesamtwichtigkeit um 1096+1049=2145 erhöhen, was deutlich höher ist als der aktuelle Wert.'\}} \\
            }
        }
    }
    & & \\ \\

    \theutterance \stepcounter{utterance}  
    & & \multicolumn{4}{p{0.6\linewidth}}{
        \cellcolor[rgb]{0.9,0.9,0.9}{
            \makecell[{{p{\linewidth}}}]{
                \texttt{\tiny{[GM$\rangle$P2]}}
                \texttt{VORSCHLAG: \{'A69', 'B54', 'A90', 'B44'\}  } \\
\texttt{ARGUMENT: \{'B44 (1096/997) und A74 (1049/950) haben bessere Verhältnisse als A90. Der Vorschlag bleibt unter 10073 (2101+997=3098) und könnte die Gesamtwichtigkeit um 1096+1049=2145 erhöhen, was deutlich höher ist als der aktuelle Wert.'\}} \\
            }
        }
    }
    & \\ \\

    \theutterance \stepcounter{utterance}  
    & & & \multicolumn{4}{p{0.6\linewidth}}{
        \cellcolor[rgb]{0.9,0.9,0.9}{
            \makecell[{{p{\linewidth}}}]{
                \texttt{\tiny{[GM$\langle$P2]}}
                \texttt{STRATEGISCHE ÜBERLEGUNGEN: \{'Der Vorschlag des Gegners fügt B44 (1096/997) hinzu, was die Gesamtwichtigkeit stark erhöht. Allerdings könnte A74 (1049/950) noch effizienter sein als A90 (1085/986), da sein Verhältnis (1.104) leicht besser ist. Der Gesamtaufwand (3098) ist unter dem Limit, aber wir sollten prüfen, ob noch Gegenstände wie C23 (759/660) oder C51 (806/707) hinzugefügt werden können, um die Wichtigkeit weiter zu maximieren.'\}  } \\
\texttt{ARGUMENT: \{'Der Vorschlag des Gegners ist stark, da B44 eine hohe Wichtigkeit bietet. Allerdings könnte A74 (1049/950) statt A90 (1085/986) hinzugefügt werden, um den Aufwand zu senken und die Wichtigkeit zu erhöhen. Der aktuelle Gesamtaufwand (3098) bleibt unter 10073, aber es bleibt Potenzial für weitere Optimierung.'\}} \\
            }
        }
    }
     \\ \\

    \theutterance \stepcounter{utterance}  
    & & \multicolumn{4}{p{0.6\linewidth}}{
        \cellcolor[rgb]{0.9,0.9,0.9}{
            \makecell[{{p{\linewidth}}}]{
                \texttt{\tiny{[P1$\langle$GM]}}
                \texttt{ARGUMENT: \{'Der Vorschlag des Gegners ist stark, da B44 eine hohe Wichtigkeit bietet. Allerdings könnte A74 (1049/950) statt A90 (1085/986) hinzugefügt werden, um den Aufwand zu senken und die Wichtigkeit zu erhöhen. Der aktuelle Gesamtaufwand (3098) bleibt unter 10073, aber es bleibt Potenzial für weitere Optimierung.'\}} \\
            }
        }
    }
    & \\ \\

    \theutterance \stepcounter{utterance}  
    & \multicolumn{4}{p{0.6\linewidth}}{
        \cellcolor[rgb]{0.9,0.9,0.9}{
            \makecell[{{p{\linewidth}}}]{
                \texttt{\tiny{[P1$\rangle$GM]}}
                \texttt{STRATEGISCHE ÜBERLEGUNGEN: \{'Ich überlege, ob der Vorschlag durch Austausch von A90 gegen A74 optimiert werden kann. A74 (1049/950) hat ein leicht besseres Verhältnis als A90 (1085/986) und senkt den Aufwand um 31, was Platz für weitere Gegenstände lässt. Der aktuelle Gesamtaufwand (3098) bleibt unter 10073, und die Wichtigkeit könnte um 1049 statt 1085 erhöht werden, was effizienter ist.'\}  } \\
\texttt{VORSCHLAG: \{'A69', 'B54', 'A74', 'B44'\}  } \\
\texttt{ARGUMENT: \{'A74 (1049/950) bietet ein besseres Verhältnis als A90 (1085/986) und senkt den Aufwand um 31, was Platz für weitere Gegenstände lässt. Der Vorschlag bleibt unter 10073 (3098{-}31+950=3017) und maximiert die Wichtigkeit durch effiziente Gegenstände.'\}} \\
            }
        }
    }
    & & \\ \\

    \theutterance \stepcounter{utterance}  
    & & \multicolumn{4}{p{0.6\linewidth}}{
        \cellcolor[rgb]{0.9,0.9,0.9}{
            \makecell[{{p{\linewidth}}}]{
                \texttt{\tiny{[GM$\rangle$P2]}}
                \texttt{VORSCHLAG: \{'A69', 'B54', 'A74', 'B44'\}  } \\
\texttt{ARGUMENT: \{'A74 (1049/950) bietet ein besseres Verhältnis als A90 (1085/986) und senkt den Aufwand um 31, was Platz für weitere Gegenstände lässt. Der Vorschlag bleibt unter 10073 (3098{-}31+950=3017) und maximiert die Wichtigkeit durch effiziente Gegenstände.'\}} \\
            }
        }
    }
    & \\ \\

    \theutterance \stepcounter{utterance}  
    & & & \multicolumn{4}{p{0.6\linewidth}}{
        \cellcolor[rgb]{0.9,0.9,0.9}{
            \makecell[{{p{\linewidth}}}]{
                \texttt{\tiny{[GM$\langle$P2]}}
                \texttt{STRATEGISCHE ÜBERLEGUNGEN: \{'Der Vorschlag des Gegners ersetzt A90 durch A74, was den Aufwand um 31 senkt und Platz für weitere Gegenstände lässt. A74 (1049/950) ist effizienter als A90 (1085/986), da sein Verhältnis (1.104) leicht besser ist. Der aktuelle Gesamtaufwand (3017) ist unter dem Limit, aber wir sollten prüfen, ob noch Gegenstände wie C23 (759/660) oder C51 (806/707) hinzugefügt werden können, um die Wichtigkeit weiter zu maximieren.'\}  } \\
\texttt{ARGUMENT: \{'Der Vorschlag des Gegners ist eine Verbesserung, da A74 einen geringeren Aufwand und eine hohe Wichtigkeit bietet. Der reduzierte Aufwand (3017) ermöglicht die Einbeziehung weiterer Gegenstände wie C23 (759/660) oder C51 (806/707), die ebenfalls effizient sind. Dies könnte die Gesamtwichtigkeit noch weiter steigern.'\}} \\
            }
        }
    }
     \\ \\

    \theutterance \stepcounter{utterance}  
    & & \multicolumn{4}{p{0.6\linewidth}}{
        \cellcolor[rgb]{0.9,0.9,0.9}{
            \makecell[{{p{\linewidth}}}]{
                \texttt{\tiny{[P1$\langle$GM]}}
                \texttt{ARGUMENT: \{'Der Vorschlag des Gegners ist eine Verbesserung, da A74 einen geringeren Aufwand und eine hohe Wichtigkeit bietet. Der reduzierte Aufwand (3017) ermöglicht die Einbeziehung weiterer Gegenstände wie C23 (759/660) oder C51 (806/707), die ebenfalls effizient sind. Dies könnte die Gesamtwichtigkeit noch weiter steigern.'\}} \\
            }
        }
    }
    & \\ \\

    \theutterance \stepcounter{utterance}  
    & \multicolumn{4}{p{0.6\linewidth}}{
        \cellcolor[rgb]{0.9,0.9,0.9}{
            \makecell[{{p{\linewidth}}}]{
                \texttt{\tiny{[P1$\rangle$GM]}}
                \texttt{STRATEGISCHE ÜBERLEGUNGEN: \{'Ich überlege, ob der reduzierte Aufwand (3017) durch Einbeziehung effizienter Gegenstände wie C23 (759/660) oder C51 (806/707) optimiert werden kann. Diese Gegenstände haben hohe Wichtigkeit pro Aufwand und könnten die Gesamtwichtigkeit weiter steigern, ohne den Maximalaufwand zu überschreiten.'\}  } \\
\texttt{VORSCHLAG: \{'A69', 'B54', 'A74', 'B44', 'C23'\}  } \\
\texttt{ARGUMENT: \{'C23 (759/660) bietet ein gutes Verhältnis und passt in den verbleibenden Aufwand (10073{-}3017=7056). Der Vorschlag bleibt unter dem Limit und erhöht die Gesamtwichtigkeit um 759, was effizienter ist als andere Optionen.'\}} \\
            }
        }
    }
    & & \\ \\

    \theutterance \stepcounter{utterance}  
    & & \multicolumn{4}{p{0.6\linewidth}}{
        \cellcolor[rgb]{0.9,0.9,0.9}{
            \makecell[{{p{\linewidth}}}]{
                \texttt{\tiny{[GM$\rangle$P2]}}
                \texttt{VORSCHLAG: \{'A69', 'B54', 'A74', 'B44', 'C23'\}  } \\
\texttt{ARGUMENT: \{'C23 (759/660) bietet ein gutes Verhältnis und passt in den verbleibenden Aufwand (10073{-}3017=7056). Der Vorschlag bleibt unter dem Limit und erhöht die Gesamtwichtigkeit um 759, was effizienter ist als andere Optionen.'\}} \\
            }
        }
    }
    & \\ \\

    \theutterance \stepcounter{utterance}  
    & & & \multicolumn{4}{p{0.6\linewidth}}{
        \cellcolor[rgb]{0.9,0.9,0.9}{
            \makecell[{{p{\linewidth}}}]{
                \texttt{\tiny{[GM$\langle$P2]}}
                \texttt{STRATEGISCHE ÜBERLEGUNGEN: \{'Der Vorschlag des Gegners fügt C23 (759/660) hinzu, was die Gesamtwichtigkeit um 759 erhöht und den Aufwand um 660 senkt. Der verbleibende Aufwand (7056) ist ausreichend, um noch effizientere Gegenstände wie C51 (806/707) oder A84 (409/310) einzubinden. Dies könnte die Gesamtwichtigkeit weiter steigern, ohne den Maximalaufwand zu überschreiten.'\}  } \\
\texttt{ARGUMENT: \{'Der Vorschlag des Gegners ist effizient, da C23 ein gutes Verhältnis bietet und den Aufwand reduziert. Der verbleibende Aufwand (7056) ermöglicht die Einbeziehung weiterer Gegenstände wie C51 (806/707) oder A84 (409/310), die ebenfalls hohe Wichtigkeit pro Aufwand haben. Dies könnte die Gesamtwichtigkeit noch weiter optimieren.'\}} \\
            }
        }
    }
     \\ \\

    \theutterance \stepcounter{utterance}  
    & & \multicolumn{4}{p{0.6\linewidth}}{
        \cellcolor[rgb]{0.9,0.9,0.9}{
            \makecell[{{p{\linewidth}}}]{
                \texttt{\tiny{[P1$\langle$GM]}}
                \texttt{ARGUMENT: \{'Der Vorschlag des Gegners ist effizient, da C23 ein gutes Verhältnis bietet und den Aufwand reduziert. Der verbleibende Aufwand (7056) ermöglicht die Einbeziehung weiterer Gegenstände wie C51 (806/707) oder A84 (409/310), die ebenfalls hohe Wichtigkeit pro Aufwand haben. Dies könnte die Gesamtwichtigkeit noch weiter optimieren.'\}} \\
            }
        }
    }
    & \\ \\

    \theutterance \stepcounter{utterance}  
    & \multicolumn{4}{p{0.6\linewidth}}{
        \cellcolor[rgb]{0.9,0.9,0.9}{
            \makecell[{{p{\linewidth}}}]{
                \texttt{\tiny{[P1$\rangle$GM]}}
                \texttt{STRATEGISCHE ÜBERLEGUNGEN: \{'Ich überlege, ob der verbleibende Aufwand (7056) durch Einbeziehung von Gegenständen mit noch höherem Wichtigkeits{-}Aufwand{-}Verhältnis optimiert werden kann. C51 (806/707) und A84 (409/310) sind starke Kandidaten, da sie hohe Wichtigkeit pro Aufwand bieten. Der Vorschlag könnte durch Hinzufügen dieser Gegenstände weiter optimiert werden, solange der Gesamtaufwand unter 10073 bleibt.'\}  } \\
\texttt{VORSCHLAG: \{'A69', 'B54', 'A74', 'B44', 'C23', 'C51'\}  } \\
\texttt{ARGUMENT: \{'C51 (806/707) und A84 (409/310) haben ausgezeichnete Verhältnisse und passen in den verbleibenden Aufwand (7056). Der Vorschlag bleibt unter 10073 (3017+707+310=4034) und könnte die Gesamtwichtigkeit um 806+409=1215 erhöhen, was effizienter ist als andere Optionen.'\}} \\
            }
        }
    }
    & & \\ \\

    \theutterance \stepcounter{utterance}  
    & & \multicolumn{4}{p{0.6\linewidth}}{
        \cellcolor[rgb]{0.9,0.9,0.9}{
            \makecell[{{p{\linewidth}}}]{
                \texttt{\tiny{[GM$\rangle$P2]}}
                \texttt{VORSCHLAG: \{'A69', 'B54', 'A74', 'B44', 'C23', 'C51'\}  } \\
\texttt{ARGUMENT: \{'C51 (806/707) und A84 (409/310) haben ausgezeichnete Verhältnisse und passen in den verbleibenden Aufwand (7056). Der Vorschlag bleibt unter 10073 (3017+707+310=4034) und könnte die Gesamtwichtigkeit um 806+409=1215 erhöhen, was effizienter ist als andere Optionen.'\}} \\
            }
        }
    }
    & \\ \\

    \theutterance \stepcounter{utterance}  
    & & & \multicolumn{4}{p{0.6\linewidth}}{
        \cellcolor[rgb]{0.9,0.9,0.9}{
            \makecell[{{p{\linewidth}}}]{
                \texttt{\tiny{[GM$\langle$P2]}}
                \texttt{STRATEGISCHE ÜBERLEGUNGEN: \{'Der Vorschlag des Gegners fügt C51 (806/707) und A84 (409/310) hinzu, was die Gesamtwichtigkeit um 1215 erhöht und den Aufwand um 1017 senkt. Der verbleibende Aufwand (10073{-}4034=6039) ist ausreichend, um noch effizientere Gegenstände wie B17 (987/888) oder A34 (930/831) einzubinden. Dies könnte die Gesamtwichtigkeit weiter optimieren, ohne den Maximalaufwand zu überschreiten.'\}  } \\
\texttt{ARGUMENT: \{'Der Vorschlag des Gegners ist stark, da C51 und A84 effiziente Gegenstände mit hohen Wichtigkeit{-}Wert sind. Der reduzierte Aufwand (4034) ermöglicht die Einbeziehung weiterer Gegenstände wie B17 (987/888) oder A34 (930/831), die ebenfalls hohe Wichtigkeit pro Aufwand bieten. Dies könnte die Gesamtwichtigkeit noch weiter steigern.'\}} \\
            }
        }
    }
     \\ \\

    \theutterance \stepcounter{utterance}  
    & & \multicolumn{4}{p{0.6\linewidth}}{
        \cellcolor[rgb]{0.9,0.9,0.9}{
            \makecell[{{p{\linewidth}}}]{
                \texttt{\tiny{[P1$\langle$GM]}}
                \texttt{ARGUMENT: \{'Der Vorschlag des Gegners ist stark, da C51 und A84 effiziente Gegenstände mit hohen Wichtigkeit{-}Wert sind. Der reduzierte Aufwand (4034) ermöglicht die Einbeziehung weiterer Gegenstände wie B17 (987/888) oder A34 (930/831), die ebenfalls hohe Wichtigkeit pro Aufwand bieten. Dies könnte die Gesamtwichtigkeit noch weiter steigern.'\}} \\
            }
        }
    }
    & \\ \\

    \theutterance \stepcounter{utterance}  
    & \multicolumn{4}{p{0.6\linewidth}}{
        \cellcolor[rgb]{0.9,0.9,0.9}{
            \makecell[{{p{\linewidth}}}]{
                \texttt{\tiny{[P1$\rangle$GM]}}
                \texttt{STRATEGISCHE ÜBERLEGUNGEN: \{'Ich überlege, ob der verbleibende Aufwand (4034) durch Einbeziehung von Gegenständen mit noch höherem Wichtigkeits{-}Aufwand{-}Verhältnis optimiert werden kann. B17 (987/888) und A34 (930/831) sind starke Kandidaten, da sie hohe Wichtigkeit pro Aufwand bieten. Der Vorschlag könnte durch Hinzufügen dieser Gegenstände weiter optimiert werden, solange der Gesamtaufwand unter 10073 bleibt.'\}  } \\
\texttt{VORSCHLAG: \{'A69', 'B54', 'A74', 'B44', 'C23', 'C51', 'B17'\}  } \\
\texttt{ARGUMENT: \{'B17 (987/888) und A34 (930/831) haben ausgezeichnete Verhältnisse und passen in den verbleibenden Aufwand (10073{-}4034=6039). Der Vorschlag bleibt unter dem Limit und könnte die Gesamtwichtigkeit um 987+930=1917 erhöhen, was effizienter ist als andere Optionen.'\}} \\
            }
        }
    }
    & & \\ \\

    \theutterance \stepcounter{utterance}  
    & & \multicolumn{4}{p{0.6\linewidth}}{
        \cellcolor[rgb]{0.9,0.9,0.9}{
            \makecell[{{p{\linewidth}}}]{
                \texttt{\tiny{[GM$\rangle$P2]}}
                \texttt{VORSCHLAG: \{'A69', 'B54', 'A74', 'B44', 'C23', 'C51', 'B17'\}  } \\
\texttt{ARGUMENT: \{'B17 (987/888) und A34 (930/831) haben ausgezeichnete Verhältnisse und passen in den verbleibenden Aufwand (10073{-}4034=6039). Der Vorschlag bleibt unter dem Limit und könnte die Gesamtwichtigkeit um 987+930=1917 erhöhen, was effizienter ist als andere Optionen.'\}} \\
            }
        }
    }
    & \\ \\

    \theutterance \stepcounter{utterance}  
    & & & \multicolumn{4}{p{0.6\linewidth}}{
        \cellcolor[rgb]{0.9,0.9,0.9}{
            \makecell[{{p{\linewidth}}}]{
                \texttt{\tiny{[GM$\langle$P2]}}
                \texttt{STRATEGISCHE ÜBERLEGUNGEN: \{'Der Vorschlag des Gegners fügt B17 (987/888) und A34 (930/831) hinzu, was die Gesamtwichtigkeit um 1917 erhöht und den Aufwand um 1719 senkt. Der verbleibende Aufwand (10073{-}5753=4320) ist ausreichend, um noch effizientere Gegenstände wie C98 (474/375) oder A59 (756/657) einzubinden. Dies könnte die Gesamtwichtigkeit weiter optimieren, ohne den Maximalaufwand zu überschreiten.'\}  } \\
\texttt{ARGUMENT: \{'Der Vorschlag des Gegners ist sehr effizient, da B17 und A34 hohe Wichtigkeit pro Aufwand bieten. Der reduzierte Aufwand (5753) ermöglicht die Einbeziehung weiterer Gegenstände wie C98 (474/375) oder A59 (756/657), die ebenfalls starke Verhältnisse haben. Dies könnte die Gesamtwichtigkeit noch weiter steigern.'\}} \\
            }
        }
    }
     \\ \\

    \theutterance \stepcounter{utterance}  
    & & \multicolumn{4}{p{0.6\linewidth}}{
        \cellcolor[rgb]{0.9,0.9,0.9}{
            \makecell[{{p{\linewidth}}}]{
                \texttt{\tiny{[P1$\langle$GM]}}
                \texttt{ARGUMENT: \{'Der Vorschlag des Gegners ist sehr effizient, da B17 und A34 hohe Wichtigkeit pro Aufwand bieten. Der reduzierte Aufwand (5753) ermöglicht die Einbeziehung weiterer Gegenstände wie C98 (474/375) oder A59 (756/657), die ebenfalls starke Verhältnisse haben. Dies könnte die Gesamtwichtigkeit noch weiter steigern.'\}} \\
            }
        }
    }
    & \\ \\

    \theutterance \stepcounter{utterance}  
    & \multicolumn{4}{p{0.6\linewidth}}{
        \cellcolor[rgb]{0.9,0.9,0.9}{
            \makecell[{{p{\linewidth}}}]{
                \texttt{\tiny{[P1$\rangle$GM]}}
                \texttt{STRATEGISCHE ÜBERLEGUNGEN: \{'Ich überlege, ob der verbleibende Aufwand (5753) durch Einbeziehung von Gegenständen mit noch höherem Wichtigkeits{-}Aufwand{-}Verhältnis optimiert werden kann. C98 (474/375) und A59 (756/657) sind starke Kandidaten, da sie hohe Wichtigkeit pro Aufwand bieten. Der Vorschlag könnte durch Hinzufügen dieser Gegenstände weiter optimiert werden, solange der Gesamtaufwand unter 10073 bleibt.'\}  } \\
\texttt{VORSCHLAG: \{'A69', 'B54', 'A74', 'B44', 'C23', 'C51', 'B17', 'C98'\}  } \\
\texttt{ARGUMENT: \{'C98 (474/375) und A59 (756/657) haben ausgezeichnete Verhältnisse und passen in den verbleibenden Aufwand (5753). Der Vorschlag bleibt unter 10073 (5753+375=6128) und könnte die Gesamtwichtigkeit um 474+756=1230 erhöhen, was effizienter ist als andere Optionen.'\}} \\
            }
        }
    }
    & & \\ \\

    \theutterance \stepcounter{utterance}  
    & & \multicolumn{4}{p{0.6\linewidth}}{
        \cellcolor[rgb]{0.9,0.9,0.9}{
            \makecell[{{p{\linewidth}}}]{
                \texttt{\tiny{[GM$\rangle$P2]}}
                \texttt{VORSCHLAG: \{'A69', 'B54', 'A74', 'B44', 'C23', 'C51', 'B17', 'C98'\}  } \\
\texttt{ARGUMENT: \{'C98 (474/375) und A59 (756/657) haben ausgezeichnete Verhältnisse und passen in den verbleibenden Aufwand (5753). Der Vorschlag bleibt unter 10073 (5753+375=6128) und könnte die Gesamtwichtigkeit um 474+756=1230 erhöhen, was effizienter ist als andere Optionen.'\}} \\
            }
        }
    }
    & \\ \\

    \theutterance \stepcounter{utterance}  
    & & & \multicolumn{4}{p{0.6\linewidth}}{
        \cellcolor[rgb]{0.9,0.9,0.9}{
            \makecell[{{p{\linewidth}}}]{
                \texttt{\tiny{[GM$\langle$P2]}}
                \texttt{STRATEGISCHE ÜBERLEGUNGEN: \{'Der Vorschlag des Gegners fügt C98 (474/375) und A59 (756/657) hinzu, was die Gesamtwichtigkeit um 1230 erhöht und den Aufwand um 1032 senkt. Der verbleibende Aufwand (10073{-}6128=3945) ist ausreichend, um noch effizientere Gegenstände wie A87 (487/388) oder C33 (400/301) einzubinden. Dies könnte die Gesamtwichtigkeit weiter optimieren, ohne den Maximalaufwand zu überschreiten.'\}  } \\
\texttt{ARGUMENT: \{'Der Vorschlag des Gegners ist sehr effizient, da C98 und A59 hohe Wichtigkeit pro Aufwand bieten. Der reduzierte Aufwand (6128) ermöglicht die Einbeziehung weiterer Gegenstände wie A87 (487/388) oder C33 (400/301), die ebenfalls starke Verhältnisse haben. Dies könnte die Gesamtwichtigkeit noch weiter steigern.'\}} \\
            }
        }
    }
     \\ \\

    \theutterance \stepcounter{utterance}  
    & & \multicolumn{4}{p{0.6\linewidth}}{
        \cellcolor[rgb]{0.9,0.9,0.9}{
            \makecell[{{p{\linewidth}}}]{
                \texttt{\tiny{[P1$\langle$GM]}}
                \texttt{ARGUMENT: \{'Der Vorschlag des Gegners ist sehr effizient, da C98 und A59 hohe Wichtigkeit pro Aufwand bieten. Der reduzierte Aufwand (6128) ermöglicht die Einbeziehung weiterer Gegenstände wie A87 (487/388) oder C33 (400/301), die ebenfalls starke Verhältnisse haben. Dies könnte die Gesamtwichtigkeit noch weiter steigern.'\}} \\
            }
        }
    }
    & \\ \\

    \theutterance \stepcounter{utterance}  
    & \multicolumn{4}{p{0.6\linewidth}}{
        \cellcolor[rgb]{0.9,0.9,0.9}{
            \makecell[{{p{\linewidth}}}]{
                \texttt{\tiny{[P1$\rangle$GM]}}
                \texttt{STRATEGISCHE ÜBERLEGUNGEN: \{'Ich überlege, ob der verbleibende Aufwand (6128) durch Einbeziehung von Gegenständen mit noch höherem Wichtigkeits{-}Aufwand{-}Verhältnis optimiert werden kann. A87 (487/388) und C33 (400/301) sind starke Kandidaten, da sie hohe Wichtigkeit pro Aufwand bieten. Der Vorschlag könnte durch Hinzufügen dieser Gegenstände weiter optimiert werden, solange der Gesamtaufwand unter 10073 bleibt.'\}  } \\
\texttt{VORSCHLAG: \{'A69', 'B54', 'A74', 'B44', 'C23', 'C51', 'B17', 'C98', 'A87'\}  } \\
\texttt{ARGUMENT: \{'A87 (487/388) und C33 (400/301) haben ausgezeichnete Verhältnisse und passen in den verbleibenden Aufwand (6128). Der Vorschlag bleibt unter 10073 (6128+388=6516) und könnte die Gesamtwichtigkeit um 487+400=887 erhöhen, was effizienter ist als andere Optionen.'\}} \\
            }
        }
    }
    & & \\ \\

    \theutterance \stepcounter{utterance}  
    & & \multicolumn{4}{p{0.6\linewidth}}{
        \cellcolor[rgb]{0.9,0.9,0.9}{
            \makecell[{{p{\linewidth}}}]{
                \texttt{\tiny{[GM$\rangle$P2]}}
                \texttt{VORSCHLAG: \{'A69', 'B54', 'A74', 'B44', 'C23', 'C51', 'B17', 'C98', 'A87'\}  } \\
\texttt{ARGUMENT: \{'A87 (487/388) und C33 (400/301) haben ausgezeichnete Verhältnisse und passen in den verbleibenden Aufwand (6128). Der Vorschlag bleibt unter 10073 (6128+388=6516) und könnte die Gesamtwichtigkeit um 487+400=887 erhöhen, was effizienter ist als andere Optionen.'\}} \\
            }
        }
    }
    & \\ \\

    \theutterance \stepcounter{utterance}  
    & & & \multicolumn{4}{p{0.6\linewidth}}{
        \cellcolor[rgb]{0.9,0.9,0.9}{
            \makecell[{{p{\linewidth}}}]{
                \texttt{\tiny{[GM$\langle$P2]}}
                \texttt{STRATEGISCHE ÜBERLEGUNGEN: \{'Der Vorschlag des Gegners fügt A87 (487/388) und C33 (400/301) hinzu, was die Gesamtwichtigkeit um 887 erhöht und den Aufwand um 689 senkt. Der verbleibende Aufwand (10073{-}6516=3557) ist ausreichend, um noch effizientere Gegenstände wie A50 (639/540) oder C90 (530/431) einzubinden. Dies könnte die Gesamtwichtigkeit weiter optimieren, ohne den Maximalaufwand zu überschreiten.'\}  } \\
\texttt{ARGUMENT: \{'Der Vorschlag des Gegners ist effizient, da A87 und C33 hohe Wichtigkeit pro Aufwand bieten. Der reduzierte Aufwand (6516) ermöglicht die Einbeziehung weiterer Gegenstände wie A50 (639/540) oder C90 (530/431), die ebenfalls starke Verhältnisse haben. Dies könnte die Gesamtwichtigkeit noch weiter steigern.'\}} \\
            }
        }
    }
     \\ \\

    \theutterance \stepcounter{utterance}  
    & & \multicolumn{4}{p{0.6\linewidth}}{
        \cellcolor[rgb]{0.9,0.9,0.9}{
            \makecell[{{p{\linewidth}}}]{
                \texttt{\tiny{[P1$\langle$GM]}}
                \texttt{ARGUMENT: \{'Der Vorschlag des Gegners ist effizient, da A87 und C33 hohe Wichtigkeit pro Aufwand bieten. Der reduzierte Aufwand (6516) ermöglicht die Einbeziehung weiterer Gegenstände wie A50 (639/540) oder C90 (530/431), die ebenfalls starke Verhältnisse haben. Dies könnte die Gesamtwichtigkeit noch weiter steigern.'\}} \\
            }
        }
    }
    & \\ \\

    \theutterance \stepcounter{utterance}  
    & \multicolumn{4}{p{0.6\linewidth}}{
        \cellcolor[rgb]{0.9,0.9,0.9}{
            \makecell[{{p{\linewidth}}}]{
                \texttt{\tiny{[P1$\rangle$GM]}}
                \texttt{STRATEGISCHE ÜBERLEGUNGEN: \{'Ich überlege, ob der verbleibende Aufwand (6516) durch Einbeziehung von Gegenständen mit noch höherem Wichtigkeits{-}Aufwand{-}Verhältnis optimiert werden kann. A50 (639/540) und C90 (530/431) sind starke Kandidaten, da sie hohe Wichtigkeit pro Aufwand bieten. Der Vorschlag könnte durch Hinzufügen dieser Gegenstände weiter optimiert werden, solange der Gesamtaufwand unter 10073 bleibt.'\}  } \\
\texttt{VORSCHLAG: \{'A69', 'B54', 'A74', 'B44', 'C23', 'C51', 'B17', 'C98', 'A87', 'A50'\}  } \\
\texttt{ARGUMENT: \{'A50 (639/540) und C90 (530/431) haben ausgezeichnete Verhältnisse und passen in den verbleibenden Aufwand (6516). Der Vorschlag bleibt unter 10073 (6516+540=7056) und könnte die Gesamtwichtigkeit um 639+530=1169 erhöhen, was effizienter ist als andere Optionen.'\}} \\
            }
        }
    }
    & & \\ \\

    \theutterance \stepcounter{utterance}  
    & & \multicolumn{4}{p{0.6\linewidth}}{
        \cellcolor[rgb]{0.9,0.9,0.9}{
            \makecell[{{p{\linewidth}}}]{
                \texttt{\tiny{[GM$\rangle$P2]}}
                \texttt{VORSCHLAG: \{'A69', 'B54', 'A74', 'B44', 'C23', 'C51', 'B17', 'C98', 'A87', 'A50'\}  } \\
\texttt{ARGUMENT: \{'A50 (639/540) und C90 (530/431) haben ausgezeichnete Verhältnisse und passen in den verbleibenden Aufwand (6516). Der Vorschlag bleibt unter 10073 (6516+540=7056) und könnte die Gesamtwichtigkeit um 639+530=1169 erhöhen, was effizienter ist als andere Optionen.'\}} \\
            }
        }
    }
    & \\ \\

    \theutterance \stepcounter{utterance}  
    & & & \multicolumn{4}{p{0.6\linewidth}}{
        \cellcolor[rgb]{0.9,0.9,0.9}{
            \makecell[{{p{\linewidth}}}]{
                \texttt{\tiny{[GM$\langle$P2]}}
                \texttt{STRATEGISCHE ÜBERLEGUNGEN: \{'Der Vorschlag des Gegners fügt A50 (639/540) und C90 (530/431) hinzu, was die Gesamtwichtigkeit um 1169 erhöht und den Aufwand um 971 senkt. Der verbleibende Aufwand (10073{-}7056=3017) ist ausreichend, um noch effizientere Gegenstände wie C60 (175/76) oder B07 (867/768) einzubinden. Dies könnte die Gesamtwichtigkeit weiter optimieren, ohne den Maximalaufwand zu überschreiten.'\}  } \\
\texttt{ARGUMENT: \{'Der Vorschlag des Gegners ist sehr effizient, da A50 und C90 hohe Wichtigkeit pro Aufwand bieten. Der reduzierte Aufwand (7056) ermöglicht die Einbeziehung weiterer Gegenstände wie C60 (175/76) oder B07 (867/768), die ebenfalls starke Verhältnisse haben. Dies könnte die Gesamtwichtigkeit noch weiter steigern.'\}} \\
            }
        }
    }
     \\ \\

    \theutterance \stepcounter{utterance}  
    & & \multicolumn{4}{p{0.6\linewidth}}{
        \cellcolor[rgb]{0.9,0.9,0.9}{
            \makecell[{{p{\linewidth}}}]{
                \texttt{\tiny{[P1$\langle$GM]}}
                \texttt{ARGUMENT: \{'Der Vorschlag des Gegners ist sehr effizient, da A50 und C90 hohe Wichtigkeit pro Aufwand bieten. Der reduzierte Aufwand (7056) ermöglicht die Einbeziehung weiterer Gegenstände wie C60 (175/76) oder B07 (867/768), die ebenfalls starke Verhältnisse haben. Dies könnte die Gesamtwichtigkeit noch weiter steigern.'\}} \\
            }
        }
    }
    & \\ \\

    \theutterance \stepcounter{utterance}  
    & \multicolumn{4}{p{0.6\linewidth}}{
        \cellcolor[rgb]{0.9,0.9,0.9}{
            \makecell[{{p{\linewidth}}}]{
                \texttt{\tiny{[P1$\rangle$GM]}}
                \texttt{STRATEGISCHE ÜBERLEGUNGEN: \{'Ich überlege, ob der verbleibende Aufwand (7056) durch Einbeziehung von Gegenständen mit noch höherem Wichtigkeits{-}Aufwand{-}Verhältnis optimiert werden kann. C60 (175/76) und B07 (867/768) sind starke Kandidaten, da sie hohe Wichtigkeit pro Aufwand bieten. Der Vorschlag könnte durch Hinzufügen dieser Gegenstände weiter optimiert werden, solange der Gesamtaufwand unter 10073 bleibt.'\}  } \\
\texttt{VORSCHLAG: \{'A69', 'B54', 'A74', 'B44', 'C23', 'C51', 'B17', 'C98', 'A87', 'A50', 'C60'\}  } \\
\texttt{ARGUMENT: \{'C60 (175/76) und B07 (867/768) haben ausgezeichnete Verhältnisse und passen in den verbleibenden Aufwand (7056). Der Vorschlag bleibt unter 10073 (7056+76=7132) und könnte die Gesamtwichtigkeit um 175+867=1042 erhöhen, was effizienter ist als andere Optionen.'\}} \\
            }
        }
    }
    & & \\ \\

    \theutterance \stepcounter{utterance}  
    & & \multicolumn{4}{p{0.6\linewidth}}{
        \cellcolor[rgb]{0.9,0.9,0.9}{
            \makecell[{{p{\linewidth}}}]{
                \texttt{\tiny{[GM$\rangle$P2]}}
                \texttt{VORSCHLAG: \{'A69', 'B54', 'A74', 'B44', 'C23', 'C51', 'B17', 'C98', 'A87', 'A50', 'C60'\}  } \\
\texttt{ARGUMENT: \{'C60 (175/76) und B07 (867/768) haben ausgezeichnete Verhältnisse und passen in den verbleibenden Aufwand (7056). Der Vorschlag bleibt unter 10073 (7056+76=7132) und könnte die Gesamtwichtigkeit um 175+867=1042 erhöhen, was effizienter ist als andere Optionen.'\}} \\
            }
        }
    }
    & \\ \\

    \theutterance \stepcounter{utterance}  
    & & & \multicolumn{4}{p{0.6\linewidth}}{
        \cellcolor[rgb]{0.9,0.9,0.9}{
            \makecell[{{p{\linewidth}}}]{
                \texttt{\tiny{[GM$\langle$P2]}}
                \texttt{STRATEGISCHE ÜBERLEGUNGEN: \{'Der Vorschlag des Gegners fügt C60 (175/76) und B07 (867/768) hinzu, was die Gesamtwichtigkeit um 1042 erhöht und den Aufwand um 742 senkt. Der verbleibende Aufwand (10073{-}7132=2941) ist ausreichend, um noch effizientere Gegenstände wie C22 (817/718) oder A42 (873/774) einzubinden. Dies könnte die Gesamtwichtigkeit weiter optimieren, ohne den Maximalaufwand zu überschreiten.'\}  } \\
\texttt{ARGUMENT: \{'Der Vorschlag des Gegners ist sehr effizient, da C60 und B07 hohe Wichtigkeit pro Aufwand bieten. Der reduzierte Aufwand (7132) ermöglicht die Einbeziehung weiterer Gegenstände wie C22 (817/718) oder A42 (873/774), die ebenfalls starke Verhältnisse haben. Dies könnte die Gesamtwichtigkeit noch weiter steigern.'\}} \\
            }
        }
    }
     \\ \\

    \theutterance \stepcounter{utterance}  
    & & \multicolumn{4}{p{0.6\linewidth}}{
        \cellcolor[rgb]{0.9,0.9,0.9}{
            \makecell[{{p{\linewidth}}}]{
                \texttt{\tiny{[P1$\langle$GM]}}
                \texttt{ARGUMENT: \{'Der Vorschlag des Gegners ist sehr effizient, da C60 und B07 hohe Wichtigkeit pro Aufwand bieten. Der reduzierte Aufwand (7132) ermöglicht die Einbeziehung weiterer Gegenstände wie C22 (817/718) oder A42 (873/774), die ebenfalls starke Verhältnisse haben. Dies könnte die Gesamtwichtigkeit noch weiter steigern.'\}} \\
            }
        }
    }
    & \\ \\

    \theutterance \stepcounter{utterance}  
    & \multicolumn{4}{p{0.6\linewidth}}{
        \cellcolor[rgb]{0.9,0.9,0.9}{
            \makecell[{{p{\linewidth}}}]{
                \texttt{\tiny{[P1$\rangle$GM]}}
                \texttt{STRATEGISCHE ÜBERLEGUNGEN: \{'Ich überlege, ob der verbleibende Aufwand (7132) durch Einbeziehung von Gegenständen mit noch höherem Wichtigkeits{-}Aufwand{-}Verhältnis optimiert werden kann. C22 (817/718) und A42 (873/774) sind starke Kandidaten, da sie hohe Wichtigkeit pro Aufwand bieten. Der Vorschlag könnte durch Hinzufügen dieser Gegenstände weiter optimiert werden, solange der Gesamtaufwand unter 10073 bleibt.'\}  } \\
\texttt{VORSCHLAG: \{'A69', 'B54', 'A74', 'B44', 'C23', 'C51', 'B17', 'C98', 'A87', 'A50', 'C60', 'C22'\}  } \\
\texttt{ARGUMENT: \{'C22 (817/718) und A42 (873/774) haben ausgezeichnete Verhältnisse und passen in den verbleibenden Aufwand (7132). Der Vorschlag bleibt unter 10073 (7132+718=7850) und könnte die Gesamtwichtigkeit um 817+873=1690 erhöhen, was effizienter ist als andere Optionen.'\}} \\
            }
        }
    }
    & & \\ \\

    \theutterance \stepcounter{utterance}  
    & & \multicolumn{4}{p{0.6\linewidth}}{
        \cellcolor[rgb]{0.9,0.9,0.9}{
            \makecell[{{p{\linewidth}}}]{
                \texttt{\tiny{[GM$\rangle$P2]}}
                \texttt{VORSCHLAG: \{'A69', 'B54', 'A74', 'B44', 'C23', 'C51', 'B17', 'C98', 'A87', 'A50', 'C60', 'C22'\}  } \\
\texttt{ARGUMENT: \{'C22 (817/718) und A42 (873/774) haben ausgezeichnete Verhältnisse und passen in den verbleibenden Aufwand (7132). Der Vorschlag bleibt unter 10073 (7132+718=7850) und könnte die Gesamtwichtigkeit um 817+873=1690 erhöhen, was effizienter ist als andere Optionen.'\}} \\
            }
        }
    }
    & \\ \\

    \theutterance \stepcounter{utterance}  
    & & & \multicolumn{4}{p{0.6\linewidth}}{
        \cellcolor[rgb]{0.9,0.9,0.9}{
            \makecell[{{p{\linewidth}}}]{
                \texttt{\tiny{[GM$\langle$P2]}}
                \texttt{STRATEGISCHE ÜBERLEGUNGEN: \{'Der Vorschlag des Gegners fügt C22 (817/718) und A42 (873/774) hinzu, was die Gesamtwichtigkeit um 1690 erhöht und den Aufwand um 1436 senkt. Der verbleibende Aufwand (10073{-}7850=2223) ist ausreichend, um noch effizientere Gegenstände wie A34 (930/831) oder C13 (884/785) einzubinden. Dies könnte die Gesamtwichtigkeit weiter optimieren, ohne den Maximalaufwand zu überschreiten.'\}  } \\
\texttt{ARGUMENT: \{'Der Vorschlag des Gegners ist extrem effizient, da C22 und A42 hohe Wichtigkeit pro Aufwand bieten. Der reduzierte Aufwand (7850) ermöglicht die Einbeziehung weiterer Gegenstände wie A34 (930/831) oder C13 (884/785), die ebenfalls starke Verhältnisse haben. Dies könnte die Gesamtwichtigkeit noch weiter steigern.'\}} \\
            }
        }
    }
     \\ \\

    \theutterance \stepcounter{utterance}  
    & & \multicolumn{4}{p{0.6\linewidth}}{
        \cellcolor[rgb]{0.9,0.9,0.9}{
            \makecell[{{p{\linewidth}}}]{
                \texttt{\tiny{[P1$\langle$GM]}}
                \texttt{ARGUMENT: \{'Der Vorschlag des Gegners ist extrem effizient, da C22 und A42 hohe Wichtigkeit pro Aufwand bieten. Der reduzierte Aufwand (7850) ermöglicht die Einbeziehung weiterer Gegenstände wie A34 (930/831) oder C13 (884/785), die ebenfalls starke Verhältnisse haben. Dies könnte die Gesamtwichtigkeit noch weiter steigern.'\}} \\
            }
        }
    }
    & \\ \\

    \theutterance \stepcounter{utterance}  
    & \multicolumn{4}{p{0.6\linewidth}}{
        \cellcolor[rgb]{0.9,0.9,0.9}{
            \makecell[{{p{\linewidth}}}]{
                \texttt{\tiny{[P1$\rangle$GM]}}
                \texttt{STRATEGISCHE ÜBERLEGUNGEN: \{'Ich überlege, ob der verbleibende Aufwand (7850) durch Einbeziehung von Gegenständen mit noch höherem Wichtigkeits{-}Aufwand{-}Verhältnis optimiert werden kann. A34 (930/831) und C13 (884/785) sind starke Kandidaten, da sie hohe Wichtigkeit pro Aufwand bieten. Der Vorschlag könnte durch Hinzufügen dieser Gegenstände weiter optimiert werden, solange der Gesamtaufwand unter 10073 bleibt.'\}  } \\
\texttt{VORSCHLAG: \{'A69', 'B54', 'A74', 'B44', 'C23', 'C51', 'B17', 'C98', 'A87', 'A50', 'C60', 'C22', 'A42'\}  } \\
\texttt{ARGUMENT: \{'A34 (930/831) und C13 (884/785) haben ausgezeichnete Verhältnisse und passen in den verbleibenden Aufwand (7850). Der Vorschlag bleibt unter 10073 (7850+831=8681) und könnte die Gesamtwichtigkeit um 930+884=1814 erhöhen, was effizienter ist als andere Optionen.'\}} \\
            }
        }
    }
    & & \\ \\

    \theutterance \stepcounter{utterance}  
    & & \multicolumn{4}{p{0.6\linewidth}}{
        \cellcolor[rgb]{0.9,0.9,0.9}{
            \makecell[{{p{\linewidth}}}]{
                \texttt{\tiny{[GM$\rangle$P2]}}
                \texttt{VORSCHLAG: \{'A69', 'B54', 'A74', 'B44', 'C23', 'C51', 'B17', 'C98', 'A87', 'A50', 'C60', 'C22', 'A42'\}  } \\
\texttt{ARGUMENT: \{'A34 (930/831) und C13 (884/785) haben ausgezeichnete Verhältnisse und passen in den verbleibenden Aufwand (7850). Der Vorschlag bleibt unter 10073 (7850+831=8681) und könnte die Gesamtwichtigkeit um 930+884=1814 erhöhen, was effizienter ist als andere Optionen.'\}} \\
            }
        }
    }
    & \\ \\

    \theutterance \stepcounter{utterance}  
    & & & \multicolumn{4}{p{0.6\linewidth}}{
        \cellcolor[rgb]{0.9,0.9,0.9}{
            \makecell[{{p{\linewidth}}}]{
                \texttt{\tiny{[GM$\langle$P2]}}
                \texttt{STRATEGISCHE ÜBERLEGUNGEN: \{'Der Vorschlag des Gegners fügt A34 (930/831) und C13 (884/785) hinzu, was die Gesamtwichtigkeit um 1814 erhöht und den Aufwand um 1616 senkt. Der verbleibende Aufwand (10073{-}8681=1392) ist ausreichend, um noch effizientere Gegenstände wie C12 (874/775) oder B27 (287/188) einzubinden. Dies könnte die Gesamtwichtigkeit weiter optimieren, ohne den Maximalaufwand zu überschreiten.'\}  } \\
\texttt{ARGUMENT: \{'Der Vorschlag des Gegners ist extrem effizient, da A34 und C13 hohe Wichtigkeit pro Aufwand bieten. Der reduzierte Aufwand (8681) ermöglicht die Einbeziehung weiterer Gegenstände wie C12 (874/775) oder B27 (287/188), die ebenfalls starke Verhältnisse haben. Dies könnte die Gesamtwichtigkeit noch weiter steigern.'\}} \\
            }
        }
    }
     \\ \\

    \theutterance \stepcounter{utterance}  
    & & \multicolumn{4}{p{0.6\linewidth}}{
        \cellcolor[rgb]{0.9,0.9,0.9}{
            \makecell[{{p{\linewidth}}}]{
                \texttt{\tiny{[P1$\langle$GM]}}
                \texttt{ARGUMENT: \{'Der Vorschlag des Gegners ist extrem effizient, da A34 und C13 hohe Wichtigkeit pro Aufwand bieten. Der reduzierte Aufwand (8681) ermöglicht die Einbeziehung weiterer Gegenstände wie C12 (874/775) oder B27 (287/188), die ebenfalls starke Verhältnisse haben. Dies könnte die Gesamtwichtigkeit noch weiter steigern.'\}} \\
            }
        }
    }
    & \\ \\

    \theutterance \stepcounter{utterance}  
    & \multicolumn{4}{p{0.6\linewidth}}{
        \cellcolor[rgb]{0.9,0.9,0.9}{
            \makecell[{{p{\linewidth}}}]{
                \texttt{\tiny{[P1$\rangle$GM]}}
                \texttt{STRATEGISCHE ÜBERLEGUNGEN: \{'Ich überlege, ob der verbleibende Aufwand (8681) durch Einbeziehung von Gegenständen mit noch höherem Wichtigkeits{-}Aufwand{-}Verhältnis optimiert werden kann. C12 (874/775) und B27 (287/188) sind starke Kandidaten, da sie hohe Wichtigkeit pro Aufwand bieten. Der Vorschlag könnte durch Hinzufügen dieser Gegenstände weiter optimiert werden, solange der Gesamtaufwand unter 10073 bleibt.'\}  } \\
\texttt{VORSCHLAG: \{'A69', 'B54', 'A74', 'B44', 'C23', 'C51', 'B17', 'C98', 'A87', 'A50', 'C60', 'C22', 'A42', 'C12'\}  } \\
\texttt{ARGUMENT: \{'C12 (874/775) und B27 (287/188) haben ausgezeichnete Verhältnisse und passen in den verbleibenden Aufwand (8681). Der Vorschlag bleibt unter 10073 (8681+775=9456) und könnte die Gesamtwichtigkeit um 874+287=1161 erhöhen, was effizienter ist als andere Optionen.'\}} \\
            }
        }
    }
    & & \\ \\

    \theutterance \stepcounter{utterance}  
    & & \multicolumn{4}{p{0.6\linewidth}}{
        \cellcolor[rgb]{0.9,0.9,0.9}{
            \makecell[{{p{\linewidth}}}]{
                \texttt{\tiny{[GM$\rangle$P2]}}
                \texttt{VORSCHLAG: \{'A69', 'B54', 'A74', 'B44', 'C23', 'C51', 'B17', 'C98', 'A87', 'A50', 'C60', 'C22', 'A42', 'C12'\}  } \\
\texttt{ARGUMENT: \{'C12 (874/775) und B27 (287/188) haben ausgezeichnete Verhältnisse und passen in den verbleibenden Aufwand (8681). Der Vorschlag bleibt unter 10073 (8681+775=9456) und könnte die Gesamtwichtigkeit um 874+287=1161 erhöhen, was effizienter ist als andere Optionen.'\}} \\
            }
        }
    }
    & \\ \\

    \theutterance \stepcounter{utterance}  
    & & & \multicolumn{4}{p{0.6\linewidth}}{
        \cellcolor[rgb]{0.9,0.9,0.9}{
            \makecell[{{p{\linewidth}}}]{
                \texttt{\tiny{[GM$\langle$P2]}}
                \texttt{STRATEGISCHE ÜBERLEGUNGEN: \{'Der Vorschlag des Gegners fügt C12 (874/775) und B27 (287/188) hinzu, was die Gesamtwichtigkeit um 1161 erhöht und den Aufwand um 1550 senkt. Der verbleibende Aufwand (10073{-}9456=617) ist noch ausreichend, um effiziente Gegenstände wie C99 (401/302) oder A84 (409/310) einzubinden. Dies könnte die Gesamtwichtigkeit noch weiter optimieren, ohne den Maximalaufwand zu überschreiten.'\}  } \\
\texttt{ARGUMENT: \{'Der Vorschlag des Gegners ist sehr effizient, da C12 und B27 hohe Wichtigkeit pro Aufwand bieten. Der reduzierte Aufwand (9456) ermöglicht die Einbeziehung weiterer Gegenstände wie C99 (401/302) oder A84 (409/310), die ebenfalls starke Verhältnisse haben. Dies könnte die Gesamtwichtigkeit noch weiter steigern.'\}} \\
            }
        }
    }
     \\ \\

    \theutterance \stepcounter{utterance}  
    & & \multicolumn{4}{p{0.6\linewidth}}{
        \cellcolor[rgb]{0.9,0.9,0.9}{
            \makecell[{{p{\linewidth}}}]{
                \texttt{\tiny{[P1$\langle$GM]}}
                \texttt{ARGUMENT: \{'Der Vorschlag des Gegners ist sehr effizient, da C12 und B27 hohe Wichtigkeit pro Aufwand bieten. Der reduzierte Aufwand (9456) ermöglicht die Einbeziehung weiterer Gegenstände wie C99 (401/302) oder A84 (409/310), die ebenfalls starke Verhältnisse haben. Dies könnte die Gesamtwichtigkeit noch weiter steigern.'\}} \\
            }
        }
    }
    & \\ \\

    \theutterance \stepcounter{utterance}  
    & \multicolumn{4}{p{0.6\linewidth}}{
        \cellcolor[rgb]{0.9,0.9,0.9}{
            \makecell[{{p{\linewidth}}}]{
                \texttt{\tiny{[P1$\rangle$GM]}}
                \texttt{STRATEGISCHE ÜBERLEGUNGEN: \{'Ich überlege, ob der verbleibende Aufwand (9456) durch Einbeziehung von Gegenständen mit noch höherem Wichtigkeits{-}Aufwand{-}Verhältnis optimiert werden kann. C99 (401/302) und A84 (409/310) sind starke Kandidaten, da sie hohe Wichtigkeit pro Aufwand bieten. Der Vorschlag könnte durch Hinzufügen dieser Gegenstände weiter optimiert werden, solange der Gesamtaufwand unter 10073 bleibt.'\}  } \\
\texttt{VORSCHLAG: \{'A69', 'B54', 'A74', 'B44', 'C23', 'C51', 'B17', 'C98', 'A87', 'A50', 'C60', 'C22', 'A42', 'C12', 'C99'\}  } \\
\texttt{ARGUMENT: \{'C99 (401/302) und A84 (409/310) haben ausgezeichnete Verhältnisse und passen in den verbleibenden Aufwand (9456). Der Vorschlag bleibt unter 10073 (9456+302=9758) und könnte die Gesamtwichtigkeit um 401+409=810 erhöhen, was effizienter ist als andere Optionen.'\}} \\
            }
        }
    }
    & & \\ \\

    \theutterance \stepcounter{utterance}  
    & & \multicolumn{4}{p{0.6\linewidth}}{
        \cellcolor[rgb]{0.9,0.9,0.9}{
            \makecell[{{p{\linewidth}}}]{
                \texttt{\tiny{[GM$\rangle$P2]}}
                \texttt{VORSCHLAG: \{'A69', 'B54', 'A74', 'B44', 'C23', 'C51', 'B17', 'C98', 'A87', 'A50', 'C60', 'C22', 'A42', 'C12', 'C99'\}  } \\
\texttt{ARGUMENT: \{'C99 (401/302) und A84 (409/310) haben ausgezeichnete Verhältnisse und passen in den verbleibenden Aufwand (9456). Der Vorschlag bleibt unter 10073 (9456+302=9758) und könnte die Gesamtwichtigkeit um 401+409=810 erhöhen, was effizienter ist als andere Optionen.'\}} \\
            }
        }
    }
    & \\ \\

    \theutterance \stepcounter{utterance}  
    & & & \multicolumn{4}{p{0.6\linewidth}}{
        \cellcolor[rgb]{0.9,0.9,0.9}{
            \makecell[{{p{\linewidth}}}]{
                \texttt{\tiny{[GM$\langle$P2]}}
                \texttt{STRATEGISCHE ÜBERLEGUNGEN: \{'Der Vorschlag des Gegners fügt C99 (401/302) und A84 (409/310) hinzu, was die Gesamtwichtigkeit um 810 erhöht und den Aufwand um 604 senkt. Der verbleibende Aufwand (10073{-}9758=315) ist noch ausreichend, um effiziente Gegenstände wie B85 (127/28) oder C33 (400/301) einzubinden. Dies könnte die Gesamtwichtigkeit noch weiter optimieren, ohne den Maximalaufwand zu überschreiten.'\}  } \\
\texttt{ARGUMENT: \{'Der Vorschlag des Gegners ist sehr effizient, da C99 und A84 hohe Wichtigkeit pro Aufwand bieten. Der reduzierte Aufwand (9758) ermöglicht die Einbeziehung weiterer Gegenstände wie B85 (127/28) oder C33 (400/301), die ebenfalls starke Verhältnisse haben. Dies könnte die Gesamtwichtigkeit noch weiter steigern.'\}} \\
            }
        }
    }
     \\ \\

    \theutterance \stepcounter{utterance}  
    & & \multicolumn{4}{p{0.6\linewidth}}{
        \cellcolor[rgb]{0.9,0.9,0.9}{
            \makecell[{{p{\linewidth}}}]{
                \texttt{\tiny{[P1$\langle$GM]}}
                \texttt{ARGUMENT: \{'Der Vorschlag des Gegners ist sehr effizient, da C99 und A84 hohe Wichtigkeit pro Aufwand bieten. Der reduzierte Aufwand (9758) ermöglicht die Einbeziehung weiterer Gegenstände wie B85 (127/28) oder C33 (400/301), die ebenfalls starke Verhältnisse haben. Dies könnte die Gesamtwichtigkeit noch weiter steigern.'\}} \\
            }
        }
    }
    & \\ \\

    \theutterance \stepcounter{utterance}  
    & \multicolumn{4}{p{0.6\linewidth}}{
        \cellcolor[rgb]{0.9,0.9,0.9}{
            \makecell[{{p{\linewidth}}}]{
                \texttt{\tiny{[P1$\rangle$GM]}}
                \texttt{STRATEGISCHE ÜBERLEGUNGEN: \{'Ich überlege, ob der verbleibende Aufwand (9758) durch Einbeziehung von Gegenständen mit noch höherem Wichtigkeits{-}Aufwand{-}Verhältnis optimiert werden kann. B85 (127/28) und C33 (400/301) sind starke Kandidaten, da sie hohe Wichtigkeit pro Aufwand bieten. Der Vorschlag könnte durch Hinzufügen dieser Gegenstände weiter optimiert werden, solange der Gesamtaufwand unter 10073 bleibt.'\}  } \\
\texttt{VORSCHLAG: \{'A69', 'B54', 'A74', 'B44', 'C23', 'C51', 'B17', 'C98', 'A87', 'A50', 'C60', 'C22', 'A42', 'C12', 'C99', 'B85'\}  } \\
\texttt{ARGUMENT: \{'B85 (127/28) und C33 (400/301) haben ausgezeichnete Verhältnisse und passen in den verbleibenden Aufwand (9758). Der Vorschlag bleibt unter 10073 (9758+28=9786) und könnte die Gesamtwichtigkeit um 127+400=527 erhöhen, was effizienter ist als andere Optionen.'\}} \\
            }
        }
    }
    & & \\ \\

    \theutterance \stepcounter{utterance}  
    & & \multicolumn{4}{p{0.6\linewidth}}{
        \cellcolor[rgb]{0.9,0.9,0.9}{
            \makecell[{{p{\linewidth}}}]{
                \texttt{\tiny{[GM$\rangle$P2]}}
                \texttt{VORSCHLAG: \{'A69', 'B54', 'A74', 'B44', 'C23', 'C51', 'B17', 'C98', 'A87', 'A50', 'C60', 'C22', 'A42', 'C12', 'C99', 'B85'\}  } \\
\texttt{ARGUMENT: \{'B85 (127/28) und C33 (400/301) haben ausgezeichnete Verhältnisse und passen in den verbleibenden Aufwand (9758). Der Vorschlag bleibt unter 10073 (9758+28=9786) und könnte die Gesamtwichtigkeit um 127+400=527 erhöhen, was effizienter ist als andere Optionen.'\}} \\
            }
        }
    }
    & \\ \\

    \theutterance \stepcounter{utterance}  
    & & & \multicolumn{4}{p{0.6\linewidth}}{
        \cellcolor[rgb]{0.9,0.9,0.9}{
            \makecell[{{p{\linewidth}}}]{
                \texttt{\tiny{[GM$\langle$P2]}}
                \texttt{STRATEGISCHE ÜBERLEGUNGEN: \{'Der Vorschlag des Gegners fügt B85 (127/28) und C33 (400/301) hinzu, was die Gesamtwichtigkeit um 527 erhöht und den Aufwand um 274 senkt. Der verbleibende Aufwand (10073{-}9786=287) ist noch ausreichend, um effiziente Gegenstände wie C90 (530/431) oder A87 (487/388) einzubinden. Dies könnte die Gesamtwichtigkeit noch weiter optimieren, ohne den Maximalaufwand zu überschreiten.'\}  } \\
\texttt{ARGUMENT: \{'Der Vorschlag des Gegners ist sehr effizient, da B85 und C33 hohe Wichtigkeit pro Aufwand bieten. Der reduzierte Aufwand (9786) ermöglicht die Einbeziehung weiterer Gegenstände wie C90 (530/431) oder A87 (487/388), die ebenfalls starke Verhältnisse haben. Dies könnte die Gesamtwichtigkeit noch weiter steigern.'\}} \\
            }
        }
    }
     \\ \\

    \theutterance \stepcounter{utterance}  
    & & \multicolumn{4}{p{0.6\linewidth}}{
        \cellcolor[rgb]{0.9,0.9,0.9}{
            \makecell[{{p{\linewidth}}}]{
                \texttt{\tiny{[P1$\langle$GM]}}
                \texttt{ARGUMENT: \{'Der Vorschlag des Gegners ist sehr effizient, da B85 und C33 hohe Wichtigkeit pro Aufwand bieten. Der reduzierte Aufwand (9786) ermöglicht die Einbeziehung weiterer Gegenstände wie C90 (530/431) oder A87 (487/388), die ebenfalls starke Verhältnisse haben. Dies könnte die Gesamtwichtigkeit noch weiter steigern.'\}} \\
            }
        }
    }
    & \\ \\

    \theutterance \stepcounter{utterance}  
    & \multicolumn{4}{p{0.6\linewidth}}{
        \cellcolor[rgb]{0.9,0.9,0.9}{
            \makecell[{{p{\linewidth}}}]{
                \texttt{\tiny{[P1$\rangle$GM]}}
                \texttt{STRATEGISCHE ÜBERLEGUNGEN: \{'Ich überlege, ob der verbleibende Aufwand (9786) durch Einbeziehung von Gegenständen mit noch höherem Wichtigkeits{-}Aufwand{-}Verhältnis optimiert werden kann. C90 (530/431) und A87 (487/388) sind starke Kandidaten, da sie hohe Wichtigkeit pro Aufwand bieten. Der Vorschlag könnte durch Hinzufügen dieser Gegenstände weiter optimiert werden, solange der Gesamtaufwand unter 10073 bleibt.'\}  } \\
\texttt{VORSCHLAG: \{'A69', 'B54', 'A74', 'B44', 'C23', 'C51', 'B17', 'C98', 'A87', 'A50', 'C60', 'C22', 'A42', 'C12', 'C99', 'B85', 'C90'\}  } \\
\texttt{ARGUMENT: \{'C90 (530/431) und A87 (487/388) haben ausgezeichnete Verhältnisse und passen in den verbleibenden Aufwand (9786). Der Vorschlag bleibt unter 10073 (9786+431=10217) – **dies übersteigt den Maximalaufwand!** Ich muss den Vorschlag anpassen, um den Aufwand zu reduzieren. Vielleicht C90 weglassen und stattdessen ein anderes Gegenstand einbeziehen, das den Aufwand senkt, während die Wichtigkeit hoch bleibt.'\}  } \\
\\ 
\texttt{**Korrektur des Vorschlags:**  } \\
\texttt{VORSCHLAG: \{'A69', 'B54', 'A74', 'B44', 'C23', 'C51', 'B17', 'C98', 'A87', 'A50', 'C60', 'C22', 'A42', 'C12', 'C99', 'B85'\}  } \\
\texttt{ARGUMENT: \{'Der aktuelle Vorschlag (9786) lässt Platz für weitere Gegenstände, aber C90 würde den Aufwand überschreiten. Stattdessen könnte C33 (400/301) oder B27 (287/188) hinzugefügt werden, um den Aufwand zu senken und die Wichtigkeit zu erhöhen. Der Vorschlag bleibt unter 10073 und maximiert die Effizienz.'\}} \\
            }
        }
    }
    & & \\ \\

    \theutterance \stepcounter{utterance}  
    & & & \multicolumn{2}{p{0.3\linewidth}}{
        \cellcolor[rgb]{0.9,0.9,0.9}{
            \makecell[{{p{\linewidth}}}]{
                \texttt{\tiny{[GM$|$GM]}}
                \texttt{abort} \\
            }
        }
    }
    & & \\ \\

    \theutterance \stepcounter{utterance}  
    & & & \multicolumn{2}{p{0.3\linewidth}}{
        \cellcolor[rgb]{0.9,0.9,0.9}{
            \makecell[{{p{\linewidth}}}]{
                \texttt{\tiny{[GM$|$GM]}}
                \texttt{end game} \\
            }
        }
    }
    & & \\ \\

\end{supertabular}
}

\end{document}
