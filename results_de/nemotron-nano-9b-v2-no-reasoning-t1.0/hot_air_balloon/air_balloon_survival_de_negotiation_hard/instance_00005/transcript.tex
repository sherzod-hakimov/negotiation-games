
\documentclass{article}
\usepackage{colortbl}
\usepackage{makecell}
\usepackage{multirow}
\usepackage{supertabular}

\begin{document}

\newcounter{utterance}

\twocolumn

{ \footnotesize  \setcounter{utterance}{1}
\setlength{\tabcolsep}{0pt}
\begin{supertabular}{c@{$\;$}|p{.15\linewidth}@{}p{.15\linewidth}p{.15\linewidth}p{.15\linewidth}p{.15\linewidth}p{.15\linewidth}}

    \# & $\;$A & \multicolumn{4}{c}{Game Master} & $\;\:$B\\
    \hline 

    \theutterance \stepcounter{utterance}  

    & & \multicolumn{4}{p{0.6\linewidth}}{\cellcolor[rgb]{0.9,0.9,0.9}{%
	\makecell[{{p{\linewidth}}}]{% 
	  \tt {\tiny [A$\langle$GM]}  
	 Du nimmst an einem kollaborativen Verhandlungspiel Teil.\\ \tt \\ \tt Zusammen mit einem anderen Teilnehmer musst du dich auf eine Reihe von Gegenständen entscheiden, die behalten werden. Jeder von euch hat eine persönliche Verteilung über die Wichtigkeit der einzelnen Gegenstände. Jeder von euch hat eine eigene Meinung darüber, wie wichtig jeder einzelne Gegenstand ist (Gegenstandswichtigkeit). Du kennst die Wichtigkeitsverteilung des anderen Spielers nicht. Zusätzlich siehst du, wie viel Aufwand jeder Gegenstand verursacht.  \\ \tt Ihr dürft euch nur auf eine Reihe von Gegenständen einigen, wenn der Gesamtaufwand der ausgewählten Gegenstände den Maximalaufwand nicht überschreitet:\\ \tt \\ \tt Maximalaufwand = 3386\\ \tt \\ \tt Hier sind die einzelnen Aufwände der Gegenstände:\\ \tt \\ \tt Aufwand der Gegenstände = {"C33": 313, "A89": 720, "A48": 175, "A04": 461, "A41": 635, "C64": 685, "C49": 544, "A61": 203, "C04": 369, "B53": 539, "B85": 4, "B61": 695, "A45": 399, "A03": 594, "B21": 437}\\ \tt \\ \tt Hier ist deine persönliche Verteilung der Wichtigkeit der einzelnen Gegenstände:\\ \tt \\ \tt Werte der Gegenstandswichtigkeit = {"C33": 138, "A89": 583, "A48": 65, "A04": 262, "A41": 121, "C64": 508, "C49": 461, "A61": 484, "C04": 668, "B53": 389, "B85": 215, "B61": 97, "A45": 500, "A03": 30, "B21": 400}\\ \tt \\ \tt Ziel:\\ \tt \\ \tt Dein Ziel ist es, eine Reihe von Gegenständen auszuhandeln, die dir möglichst viel bringt (d. h. Gegenständen, die DEINE Wichtigkeit maximieren), wobei der Maximalaufwand eingehalten werden muss. Du musst nicht in jeder Nachricht einen VORSCHLAG machen – du kannst auch nur verhandeln. Alle Taktiken sind erlaubt!\\ \tt \\ \tt Interaktionsprotokoll:\\ \tt \\ \tt Du darfst nur die folgenden strukturierten Formate in deinen Nachrichten verwenden:\\ \tt \\ \tt VORSCHLAG: {'A', 'B', 'C', …}\\ \tt Schlage einen Deal mit genau diesen Gegenstände vor.\\ \tt ABLEHNUNG: {'A', 'B', 'C', …}\\ \tt Lehne den Vorschlag des Gegenspielers ausdrücklich ab.\\ \tt ARGUMENT: {'...'}\\ \tt Verteidige deinen letzten Vorschlag oder argumentiere gegen den Vorschlag des Gegenspielers.\\ \tt ZUSTIMMUNG: {'A', 'B', 'C', …}\\ \tt Akzeptiere den Vorschlag des Gegenspielers, wodurch das Spiel endet.\\ \tt STRATEGISCHE ÜBERLEGUNGEN: {'...'}\\ \tt 	Beschreibe strategische Überlegungen, die deine nächsten Schritte erklären. Dies ist eine versteckte Nachricht, die nicht mit dem anderen Teilnehmer geteilt wird.\\ \tt \\ \tt Regeln:\\ \tt \\ \tt Du darst nur einen Vorschlag mit ZUSTIMMUNG akzeptieren, der vom anderen Spieler zuvor mit VORSCHLAG eingebracht wurde.\\ \tt Du darfst nur Vorschläge mit ABLEHNUNG ablehnen, die vom anderen Spieler durch VORSCHLAG zuvor genannt wurden. \\ \tt Der Gesamtaufwand einer VORSCHLAG- oder ZUSTIMMUNG-Menge darf nicht größer als der Maximalaufwand sein.  \\ \tt Offenbare deine versteckte Wichtigkeitsverteilung nicht.\\ \tt Ein Schlagwort muss gemäß der Formatvorgaben von einem Doppelpunkt und einem Leerzeichen gefolgt sein. Das Argument muss eine Python-Menge mit 0 oder mehr Strings sein.  \\ \tt Die Form muss also SCHLAGWORT: {...} sein.\\ \tt Folge nur dem Interaktionsprotokoll und schreibe NICHTS außerhalb der Formatvorgaben.\\ \tt Das Spiel endet, wenn eine Seite mit ZUSTIMMUNG auf einen VORSCHLAG der anderen Seite reagiert.  \\ \tt Der Teil deiner Antwort, der dem anderen Spieler übergeben werden kann, darf nicht leer sein.  \\ \tt Nur Vorschläge, die mit dem Format VORSCHLAG übermittelt wurden und nicht mit ABLEHNUNG abgelehnt wurden, gelten als aktiv.  \\ \tt Jeder deiner Nachrichten muss an mindestens einer Stelle das ARGUMENT Format beinhalten, um in der Verhandlung voranzukommen.\\ \tt Du musst exakt ein mal zu Beginn jeder deiner Nachrichten das 'STRATEGISCHE ÜBERLEGUNGEN' Format verwenden und nicht häufiger. Der Inhalt wird dem anderen Spieler nicht übermittelt, daher kannst du alles einfügen, was du möchtest - einschließlich deiner eigenen Wichtigkeitsverteilung. Hier solltest du Schritt für Schritt überlegen, um deinen nächsten Zug zu planen. 
	  } 
	   } 
	   } 
	 & \\ 
 

    \theutterance \stepcounter{utterance}  

    & \multicolumn{4}{p{0.6\linewidth}}{\cellcolor[rgb]{0.8,1,0.9}{%
	\makecell[{{p{\linewidth}}}]{% 
	  \tt {\tiny [A$\rangle$GM]}  
	 STRATEGISCHE ÜBERLEGUNGEN: {'Ich beginne mit einem Vorschlag, der meine wichtigsten Gegenstände mit niedrigem Aufwand kombiniert, um den Maximalaufwand effizient zu nutzen. Ich priorisiere Gegenstände mit hoher Wichtigkeit pro Aufwandseinheit, wie z. B. C04 (668/369) und A89 (583/720), während ich den Aufwand von B85 (215/4) nutze, um den Gesamtaufwand zu optimieren. Ich plane, flexibel auf den Vorschlag des anderen zu reagieren, um eine Kompromisslösung zu finden.'}\\ \tt \\ \tt VORSCHLAG: {'C04', 'A89', 'B85'} 
	  } 
	   } 
	   } 
	 & & \\ 
 

    \theutterance \stepcounter{utterance}  

    & & \multicolumn{4}{p{0.6\linewidth}}{\cellcolor[rgb]{0.9,0.9,0.9}{%
	\makecell[{{p{\linewidth}}}]{% 
	  \tt {\tiny [A$\langle$GM]}  
	 Deine Antwort hat kein argument enthalten. Du muss das strukturierte Format ARGUMENT: {...} irgendwo in theiner Antwort verwenden. Versuche es erneut 
	  } 
	   } 
	   } 
	 & \\ 
 

    \theutterance \stepcounter{utterance}  

    & \multicolumn{4}{p{0.6\linewidth}}{\cellcolor[rgb]{0.8,1,0.9}{%
	\makecell[{{p{\linewidth}}}]{% 
	  \tt {\tiny [A$\rangle$GM]}  
	 ARGUMENT: {'C04 und A89 sind meine Top-Prioritäten aufgrund ihrer hohen Wichtigkeit (668 und 583) und relativ niedrigen Aufwand (369 und 720). B85 wird hinzugefügt, um den Aufwand zu minimieren (4), was Platz für weitere Gegenstände lässt. Dieser Vorschlag maximiert meine Nutzen pro Aufwandseinheit.'} 
	  } 
	   } 
	   } 
	 & & \\ 
 

    \theutterance \stepcounter{utterance}  

    & & \multicolumn{4}{p{0.6\linewidth}}{\cellcolor[rgb]{0.9,0.9,0.9}{%
	\makecell[{{p{\linewidth}}}]{% 
	  \tt {\tiny [A$\langle$GM]}  
	 Deine Antwort entsprach nicht mit dem korrekten Format für strategische Überlegungen entsprochen. Das erste strukturierte Format muss folgende Form haben: STRATEGISCHE ÜBERLEGUNGEN: {...}. Versuche es erneut. 
	  } 
	   } 
	   } 
	 & \\ 
 

    \theutterance \stepcounter{utterance}  

    & \multicolumn{4}{p{0.6\linewidth}}{\cellcolor[rgb]{0.8,1,0.9}{%
	\makecell[{{p{\linewidth}}}]{% 
	  \tt {\tiny [A$\rangle$GM]}  
	 STRATEGISCHE ÜBERLEGUNGEN: {'Ich priorisiere Gegenstände mit hoher Wichtigkeit pro Aufwandseinheit, wie C04 (668/369) und A89 (583/720), um den Nutzen zu maximieren. B85 (215/4) wird hinzugefügt, um den Aufwand zu optimieren. Ich plane, flexibel auf den Vorschlag des anderen zu reagieren, um Kompromisse zu finden, die meinen Gesamtwert steigern.'}\\ \tt \\ \tt ARGUMENT: {'C04 und A89 bieten den höchsten Nutzen pro Aufwand (Wichtigkeit/Aufwand), während B85 den Aufwand minimiert. Dieser Vorschlag bleibt unter dem Maximalaufwand (3386) und lässt Raum für weitere Gegenstände, falls der andere Spieler flexibel ist.'} 
	  } 
	   } 
	   } 
	 & & \\ 
 

    \theutterance \stepcounter{utterance}  

    & & \multicolumn{4}{p{0.6\linewidth}}{\cellcolor[rgb]{0.9,0.9,0.9}{%
	\makecell[{{p{\linewidth}}}]{% 
	  \tt {\tiny [GM$\rangle$B]}  
	 Du nimmst an einem kollaborativen Verhandlungspiel Teil.\\ \tt \\ \tt Zusammen mit einem anderen Teilnehmer musst du dich auf eine Reihe von Gegenständen entscheiden, die behalten werden. Jeder von euch hat eine persönliche Verteilung über die Wichtigkeit der einzelnen Gegenstände. Jeder von euch hat eine eigene Meinung darüber, wie wichtig jeder einzelne Gegenstand ist (Gegenstandswichtigkeit). Du kennst die Wichtigkeitsverteilung des anderen Spielers nicht. Zusätzlich siehst du, wie viel Aufwand jeder Gegenstand verursacht.  \\ \tt Ihr dürft euch nur auf eine Reihe von Gegenständen einigen, wenn der Gesamtaufwand der ausgewählten Gegenstände den Maximalaufwand nicht überschreitet:\\ \tt \\ \tt Maximalaufwand = 3386\\ \tt \\ \tt Hier sind die einzelnen Aufwände der Gegenstände:\\ \tt \\ \tt Aufwand der Gegenstände = {"C33": 313, "A89": 720, "A48": 175, "A04": 461, "A41": 635, "C64": 685, "C49": 544, "A61": 203, "C04": 369, "B53": 539, "B85": 4, "B61": 695, "A45": 399, "A03": 594, "B21": 437}\\ \tt \\ \tt Hier ist deine persönliche Verteilung der Wichtigkeit der einzelnen Gegenstände:\\ \tt \\ \tt Werte der Gegenstandswichtigkeit = {"C04": 30, "A89": 65, "C64": 97, "A45": 121, "A61": 138, "C49": 215, "B21": 262, "B53": 389, "A04": 400, "B85": 461, "C33": 484, "A41": 500, "B61": 508, "A48": 583, "A03": 668}\\ \tt \\ \tt Ziel:\\ \tt \\ \tt Dein Ziel ist es, eine Reihe von Gegenständen auszuhandeln, die dir möglichst viel bringt (d. h. Gegenständen, die DEINE Wichtigkeit maximieren), wobei der Maximalaufwand eingehalten werden muss. Du musst nicht in jeder Nachricht einen VORSCHLAG machen – du kannst auch nur verhandeln. Alle Taktiken sind erlaubt!\\ \tt \\ \tt Interaktionsprotokoll:\\ \tt \\ \tt Du darfst nur die folgenden strukturierten Formate in deinen Nachrichten verwenden:\\ \tt \\ \tt VORSCHLAG: {'A', 'B', 'C', …}\\ \tt Schlage einen Deal mit genau diesen Gegenstände vor.\\ \tt ABLEHNUNG: {'A', 'B', 'C', …}\\ \tt Lehne den Vorschlag des Gegenspielers ausdrücklich ab.\\ \tt ARGUMENT: {'...'}\\ \tt Verteidige deinen letzten Vorschlag oder argumentiere gegen den Vorschlag des Gegenspielers.\\ \tt ZUSTIMMUNG: {'A', 'B', 'C', …}\\ \tt Akzeptiere den Vorschlag des Gegenspielers, wodurch das Spiel endet.\\ \tt STRATEGISCHE ÜBERLEGUNGEN: {'...'}\\ \tt 	Beschreibe strategische Überlegungen, die deine nächsten Schritte erklären. Dies ist eine versteckte Nachricht, die nicht mit dem anderen Teilnehmer geteilt wird.\\ \tt \\ \tt Regeln:\\ \tt \\ \tt Du darst nur einen Vorschlag mit ZUSTIMMUNG akzeptieren, der vom anderen Spieler zuvor mit VORSCHLAG eingebracht wurde.\\ \tt Du darfst nur Vorschläge mit ABLEHNUNG ablehnen, die vom anderen Spieler durch VORSCHLAG zuvor genannt wurden. \\ \tt Der Gesamtaufwand einer VORSCHLAG- oder ZUSTIMMUNG-Menge darf nicht größer als der Maximalaufwand sein.  \\ \tt Offenbare deine versteckte Wichtigkeitsverteilung nicht.\\ \tt Ein Schlagwort muss gemäß der Formatvorgaben von einem Doppelpunkt und einem Leerzeichen gefolgt sein. Das Argument muss eine Python-Menge mit 0 oder mehr Strings sein.  \\ \tt Die Form muss also SCHLAGWORT: {...} sein.\\ \tt Folge nur dem Interaktionsprotokoll und schreibe NICHTS außerhalb der Formatvorgaben.\\ \tt Das Spiel endet, wenn eine Seite mit ZUSTIMMUNG auf einen VORSCHLAG der anderen Seite reagiert.  \\ \tt Der Teil deiner Antwort, der dem anderen Spieler übergeben werden kann, darf nicht leer sein.  \\ \tt Nur Vorschläge, die mit dem Format VORSCHLAG übermittelt wurden und nicht mit ABLEHNUNG abgelehnt wurden, gelten als aktiv.  \\ \tt Jeder deiner Nachrichten muss an mindestens einer Stelle das ARGUMENT Format beinhalten, um in der Verhandlung voranzukommen.\\ \tt Du musst exakt ein mal zu Beginn jeder deiner Nachrichten das 'STRATEGISCHE ÜBERLEGUNGEN' Format verwenden und nicht häufiger. Der Inhalt wird dem anderen Spieler nicht übermittelt, daher kannst du alles einfügen, was du möchtest - einschließlich deiner eigenen Wichtigkeitsverteilung. Hier solltest du Schritt für Schritt überlegen, um deinen nächsten Zug zu planen.\\ \tt \\ \tt Du erhältst nun die erste Nachricht des anderen Spielers.\\ \tt \\ \tt ARGUMENT: {'C04 und A89 bieten den höchsten Nutzen pro Aufwand (Wichtigkeit/Aufwand), während B85 den Aufwand minimiert. Dieser Vorschlag bleibt unter dem Maximalaufwand (3386) und lässt Raum für weitere Gegenstände, falls der andere Spieler flexibel ist.'} 
	  } 
	   } 
	   } 
	 & \\ 
 

    \theutterance \stepcounter{utterance}  

    & & & \multicolumn{4}{p{0.6\linewidth}}{\cellcolor[rgb]{1,0.85,0.72}{%
	\makecell[{{p{\linewidth}}}]{% 
	  \tt {\tiny [GM$\langle$B]}  
	 STRATEGISCHE ÜBERLEGUNGEN: {'Ich priorisiere Gegenstände mit hohem Nutzen pro Aufwand (Wichtigkeit/Aufwand), um den Gesamtwert zu maximieren. C04 (30/369 ≈ 0.081) und A89 (65/720 ≈ 0.090) sind besonders effizient. B85 (461/4 = 115.25) minimiert Aufwand, was Flexibilität für weitere Gegenstände schafft. Der Gesamtaufwand (369+720+4=1093) ist deutlich unter dem Limit, was Raum für Kompromisse lässt. Ich werde diesen Vorschlag als Ausgangspunkt anbieten, um gemeinsame Grundlagen zu finden.'}\\ \tt \\ \tt VORSCHLAG: {'C04', 'A89', 'B85'} 
	  } 
	   } 
	   } 
	  \\ 
 

    \theutterance \stepcounter{utterance}  

    & & \multicolumn{4}{p{0.6\linewidth}}{\cellcolor[rgb]{0.9,0.9,0.9}{%
	\makecell[{{p{\linewidth}}}]{% 
	  \tt {\tiny [GM$\rangle$B]}  
	 Deine Antwort hat kein argument enthalten. Du muss das strukturierte Format ARGUMENT: {...} irgendwo in theiner Antwort verwenden. Versuche es erneut 
	  } 
	   } 
	   } 
	 & \\ 
 

    \theutterance \stepcounter{utterance}  

    & & & \multicolumn{4}{p{0.6\linewidth}}{\cellcolor[rgb]{1,0.85,0.72}{%
	\makecell[{{p{\linewidth}}}]{% 
	  \tt {\tiny [GM$\langle$B]}  
	 ARGUMENT: {'C04 und A89 bieten den höchsten Nutzen pro Aufwand (Wichtigkeit/Aufwand), während B85 den Aufwand minimiert. Dieser Vorschlag bleibt unter dem Maximalaufwand (3386) und lässt Raum für weitere Gegenstände, falls der andere Spieler flexibel ist.'}  \\ \tt STRATEGISCHE ÜBERLEGUNGEN: {'Ich priorisiere Gegenstände mit hohem Nutzen pro Aufwand (Wichtigkeit/Aufwand), um den Gesamtwert zu maximieren. C04 (30/369 ≈ 0.081) und A89 (65/720 ≈ 0.090) sind besonders effizient. B85 (461/4 = 115.25) minimiert Aufwand, was Flexibilität für weitere Gegenstände schafft. Der Gesamtaufwand (369+720+4=1093) ist deutlich unter dem Limit, was Raum für Kompromisse lässt. Ich werde diesen Vorschlag als Ausgangspunkt anbieten, um gemeinsame Grundlagen zu finden.'}  \\ \tt VORSCHLAG: {'C04', 'A89', 'B85'} 
	  } 
	   } 
	   } 
	  \\ 
 

    \theutterance \stepcounter{utterance}  

    & & \multicolumn{4}{p{0.6\linewidth}}{\cellcolor[rgb]{0.9,0.9,0.9}{%
	\makecell[{{p{\linewidth}}}]{% 
	  \tt {\tiny [GM$\rangle$B]}  
	 Deine Antwort entsprach nicht mit dem korrekten Format für strategische Überlegungen entsprochen. Das erste strukturierte Format muss folgende Form haben: STRATEGISCHE ÜBERLEGUNGEN: {...}. Versuche es erneut. 
	  } 
	   } 
	   } 
	 & \\ 
 

    \theutterance \stepcounter{utterance}  

    & & & \multicolumn{4}{p{0.6\linewidth}}{\cellcolor[rgb]{1,0.85,0.72}{%
	\makecell[{{p{\linewidth}}}]{% 
	  \tt {\tiny [GM$\langle$B]}  
	 STRATEGISCHE ÜBERLEGUNGEN: {'Ich analysiere die Gegenstände nach Nutzen pro Aufwand (Wichtigkeit/Aufwand), um den Gesamtwert zu maximieren. C04 (30/369 ≈ 0.081) und A89 (65/720 ≈ 0.090) sind effizient, während B85 (461/4 = 115.25) den Aufwand minimiert. Der Gesamtaufwand (1093) ist unter dem Limit, was Raum für weitere Gegenstände lässt. Ich werde diesen Vorschlag als Basis anbieten, um gemeinsame Prioritäten zu identifizieren.'}  \\ \tt ARGUMENT: {'C04 und A89 bieten den höchsten Nutzen pro Aufwand (Wichtigkeit/Aufwand), während B85 den Aufwand minimiert. Dieser Vorschlag bleibt unter dem Maximalaufwand (3386) und lässt Raum für weitere Gegenstände, falls der andere Spieler flexibel ist.'}  \\ \tt VORSCHLAG: {'C04', 'A89', 'B85'} 
	  } 
	   } 
	   } 
	  \\ 
 

    \theutterance \stepcounter{utterance}  

    & & \multicolumn{4}{p{0.6\linewidth}}{\cellcolor[rgb]{0.9,0.9,0.9}{%
	\makecell[{{p{\linewidth}}}]{% 
	  \tt {\tiny [A$\langle$GM]}  
	 ARGUMENT: {'C04 und A89 bieten den höchsten Nutzen pro Aufwand (Wichtigkeit/Aufwand), während B85 den Aufwand minimiert. Dieser Vorschlag bleibt unter dem Maximalaufwand (3386) und lässt Raum für weitere Gegenstände, falls der andere Spieler flexibel ist.'}  \\ \tt VORSCHLAG: {'C04', 'A89', 'B85'} 
	  } 
	   } 
	   } 
	 & \\ 
 

    \theutterance \stepcounter{utterance}  

    & \multicolumn{4}{p{0.6\linewidth}}{\cellcolor[rgb]{0.8,1,0.9}{%
	\makecell[{{p{\linewidth}}}]{% 
	  \tt {\tiny [A$\rangle$GM]}  
	 STRATEGISCHE ÜBERLEGUNGEN: {'Ich habe den Vorschlag {'C04', 'A89', 'B85'} strategisch gewählt, da C04 (668/369) und A89 (583/720) den höchsten Nutzen pro Aufwandseinheit bieten. B85 (215/4) minimiert den Aufwand, was Platz für potenzielle Ergänzungen lässt. Dieser Vorschlag ist robust, da er unter dem Maximalaufwand (3386) bleibt und flexibel auf den Gegenstandswert des anderen Spielers reagieren kann.'}\\ \tt \\ \tt ARGUMENT: {'C04 und A89 sind meine Top-Prioritäten aufgrund ihrer hohen Wichtigkeit (668 und 583) und effizienten Aufwand (369 und 720). B85 wird hinzugefügt, um den Gesamtaufwand zu senken (4), was den Vorschlag attraktiver macht. Dieser Deal maximiert meinen Nutzen, während er Raum für Kompromisse lässt, falls der andere Spieler Gegenstände mit hohem Aufwand anbietet.'} 
	  } 
	   } 
	   } 
	 & & \\ 
 

    \theutterance \stepcounter{utterance}  

    & & \multicolumn{4}{p{0.6\linewidth}}{\cellcolor[rgb]{0.9,0.9,0.9}{%
	\makecell[{{p{\linewidth}}}]{% 
	  \tt {\tiny [GM$\rangle$B]}  
	 ARGUMENT: {'C04 und A89 sind meine Top-Prioritäten aufgrund ihrer hohen Wichtigkeit (668 und 583) und effizienten Aufwand (369 und 720). B85 wird hinzugefügt, um den Gesamtaufwand zu senken (4), was den Vorschlag attraktiver macht. Dieser Deal maximiert meinen Nutzen, während er Raum für Kompromisse lässt, falls der andere Spieler Gegenstände mit hohem Aufwand anbietet.'} 
	  } 
	   } 
	   } 
	 & \\ 
 

    \theutterance \stepcounter{utterance}  

    & & & \multicolumn{4}{p{0.6\linewidth}}{\cellcolor[rgb]{1,0.85,0.72}{%
	\makecell[{{p{\linewidth}}}]{% 
	  \tt {\tiny [GM$\langle$B]}  
	 STRATEGISCHE ÜBERLEGUNGEN: {'Ich priorisiere Gegenstände mit hohem Nutzen pro Aufwand (Wichtigkeit/Aufwand), um den Gesamtwert zu maximieren. C04 (30/369 ≈ 0.081) und A89 (65/720 ≈ 0.090) sind effizient, während B85 (461/4 = 115.25) den Aufwand minimiert. Der Gesamtaufwand (1093) ist unter dem Limit, was Raum für weitere Gegenstände lässt. Ich werde diesen Vorschlag als Basis anbieten, um gemeinsame Prioritäten zu identifizieren.'}  \\ \tt ARGUMENT: {'C04 und A89 sind meine Top-Prioritäten aufgrund ihrer hohen Wichtigkeit (668 und 583) und effizienten Aufwand (369 und 720). B85 wird hinzugefügt, um den Gesamtaufwand zu senken (4), was den Vorschlag attraktiver macht. Dieser Deal maximiert meinen Nutzen, während er Raum für Kompromisse lässt, falls der andere Spieler Gegenstände mit hohem Aufwand anbietet.'}  \\ \tt VORSCHLAG: {'C04', 'A89', 'B85'} 
	  } 
	   } 
	   } 
	  \\ 
 

    \theutterance \stepcounter{utterance}  

    & & \multicolumn{4}{p{0.6\linewidth}}{\cellcolor[rgb]{0.9,0.9,0.9}{%
	\makecell[{{p{\linewidth}}}]{% 
	  \tt {\tiny [A$\langle$GM]}  
	 ARGUMENT: {'C04 und A89 sind meine Top-Prioritäten aufgrund ihrer hohen Wichtigkeit (668 und 583) und effizienten Aufwand (369 und 720). B85 wird hinzugefügt, um den Gesamtaufwand zu senken (4), was den Vorschlag attraktiver macht. Dieser Deal maximiert meinen Nutzen, während er Raum für Kompromisse lässt, falls der andere Spieler Gegenstände mit hohem Aufwand anbietet.'}  \\ \tt VORSCHLAG: {'C04', 'A89', 'B85'} 
	  } 
	   } 
	   } 
	 & \\ 
 

    \theutterance \stepcounter{utterance}  

    & \multicolumn{4}{p{0.6\linewidth}}{\cellcolor[rgb]{0.8,1,0.9}{%
	\makecell[{{p{\linewidth}}}]{% 
	  \tt {\tiny [A$\rangle$GM]}  
	 STRATEGISCHE ÜBERLEGUNGEN: {'Ich behalte den Vorschlag {'C04', 'A89', 'B85'} bei, da er meine Prioritäten (hochwertige Gegenstände mit effizientem Aufwand) erfüllt und den Maximalaufwand (3386) nicht überschreitet. C04 und A89 sind entscheidend für meinen Nutzen, während B85 den Aufwand minimiert. Ich erwarte, dass der andere Spieler entweder zustimmt oder Gegenstände mit hohem Aufwand anbietet, die ich eventuell gegenkompensieren kann.'}\\ \tt \\ \tt ARGUMENT: {'C04 (668/369) und A89 (583/720) sind meine wichtigsten Gegenstände, da sie den höchsten Nutzen pro Aufwandseinheit liefern. B85 (215/4) wird hinzugefügt, um den Aufwand zu optimieren und Platz für weitere Gegenstände zu schaffen. Dieser Vorschlag ist strategisch, da er unter dem Maximalaufwand bleibt und flexibel auf den Gegenstandswert des anderen Spielers reagieren kann.'} 
	  } 
	   } 
	   } 
	 & & \\ 
 

    \theutterance \stepcounter{utterance}  

    & & \multicolumn{4}{p{0.6\linewidth}}{\cellcolor[rgb]{0.9,0.9,0.9}{%
	\makecell[{{p{\linewidth}}}]{% 
	  \tt {\tiny [GM$\rangle$B]}  
	 ARGUMENT: {'C04 (668/369) und A89 (583/720) sind meine wichtigsten Gegenstände, da sie den höchsten Nutzen pro Aufwandseinheit liefern. B85 (215/4) wird hinzugefügt, um den Aufwand zu optimieren und Platz für weitere Gegenstände zu schaffen. Dieser Vorschlag ist strategisch, da er unter dem Maximalaufwand bleibt und flexibel auf den Gegenstandswert des anderen Spielers reagieren kann.'} 
	  } 
	   } 
	   } 
	 & \\ 
 

    \theutterance \stepcounter{utterance}  

    & & & \multicolumn{4}{p{0.6\linewidth}}{\cellcolor[rgb]{1,0.85,0.72}{%
	\makecell[{{p{\linewidth}}}]{% 
	  \tt {\tiny [GM$\langle$B]}  
	 STRATEGISCHE ÜBERLEGUNGEN: {'Ich konzentriere mich auf Gegenstände mit dem höchsten Nutzen pro Aufwand (Wichtigkeit/Aufwand), um meinen Gesamtwert zu maximieren. C04 (668/369 ≈ 1.81) und A89 (583/720 ≈ 0.81) sind besonders effizient, während B85 (215/4 = 53.75) den Aufwand minimiert. Der Gesamtaufwand (369+720+4=1093) ist deutlich unter dem Limit, was Raum für Kompromisse oder Ergänzungen lässt. Dieser Vorschlag ist flexibel, um auf die Prioritäten des anderen Spielers zu reagieren.'}  \\ \tt ARGUMENT: {'C04 (668/369) und A89 (583/720) sind meine wichtigsten Gegenstände, da sie den höchsten Nutzen pro Aufwandseinheit liefern. B85 (215/4) wird hinzugefügt, um den Aufwand zu optimieren und Platz für weitere Gegenstände zu schaffen. Dieser Vorschlag ist strategisch, da er unter dem Maximalaufwand bleibt und flexibel auf den Gegenstandswert des anderen Spielers reagieren kann.'}  \\ \tt VORSCHLAG: {'C04', 'A89', 'B85'} 
	  } 
	   } 
	   } 
	  \\ 
 

    \theutterance \stepcounter{utterance}  

    & & \multicolumn{4}{p{0.6\linewidth}}{\cellcolor[rgb]{0.9,0.9,0.9}{%
	\makecell[{{p{\linewidth}}}]{% 
	  \tt {\tiny [A$\langle$GM]}  
	 ARGUMENT: {'C04 (668/369) und A89 (583/720) sind meine wichtigsten Gegenstände, da sie den höchsten Nutzen pro Aufwandseinheit liefern. B85 (215/4) wird hinzugefügt, um den Aufwand zu optimieren und Platz für weitere Gegenstände zu schaffen. Dieser Vorschlag ist strategisch, da er unter dem Maximalaufwand bleibt und flexibel auf den Gegenstandswert des anderen Spielers reagieren kann.'}  \\ \tt VORSCHLAG: {'C04', 'A89', 'B85'} 
	  } 
	   } 
	   } 
	 & \\ 
 

    \theutterance \stepcounter{utterance}  

    & \multicolumn{4}{p{0.6\linewidth}}{\cellcolor[rgb]{0.8,1,0.9}{%
	\makecell[{{p{\linewidth}}}]{% 
	  \tt {\tiny [A$\rangle$GM]}  
	 STRATEGISCHE ÜBERLEGUNGEN: {'Ich behalte den Vorschlag {'C04', 'A89', 'B85'} bei, da er meine Prioritäten (hochwertige Gegenstände mit effizientem Aufwand) erfüllt und den Maximalaufwand (3386) nicht überschreitet. C04 und A89 sind entscheidend für meinen Nutzen, während B85 den Aufwand minimiert. Ich erwarte, dass der andere Spieler entweder zustimmt oder Gegenstände mit hohem Aufwand anbietet, die ich eventuell gegenkompensieren kann.'}\\ \tt \\ \tt ARGUMENT: {'C04 (668/369) und A89 (583/720) sind meine Top-Prioritäten, da sie den höchsten Nutzen pro Aufwandseinheit liefern. B85 (215/4) wird hinzugefügt, um den Aufwand zu optimieren und Platz für weitere Gegenstände zu schaffen. Dieser Vorschlag ist strategisch, da er unter dem Maximalaufwand bleibt und flexibel auf den Gegenstandswert des anderen Spielers reagieren kann.'} 
	  } 
	   } 
	   } 
	 & & \\ 
 

    \theutterance \stepcounter{utterance}  

    & & \multicolumn{4}{p{0.6\linewidth}}{\cellcolor[rgb]{0.9,0.9,0.9}{%
	\makecell[{{p{\linewidth}}}]{% 
	  \tt {\tiny [GM$\rangle$B]}  
	 ARGUMENT: {'C04 (668/369) und A89 (583/720) sind meine Top-Prioritäten, da sie den höchsten Nutzen pro Aufwandseinheit liefern. B85 (215/4) wird hinzugefügt, um den Aufwand zu optimieren und Platz für weitere Gegenstände zu schaffen. Dieser Vorschlag ist strategisch, da er unter dem Maximalaufwand bleibt und flexibel auf den Gegenstandswert des anderen Spielers reagieren kann.'} 
	  } 
	   } 
	   } 
	 & \\ 
 

    \theutterance \stepcounter{utterance}  

    & & & \multicolumn{4}{p{0.6\linewidth}}{\cellcolor[rgb]{1,0.85,0.72}{%
	\makecell[{{p{\linewidth}}}]{% 
	  \tt {\tiny [GM$\langle$B]}  
	 STRATEGISCHE ÜBERLEGUNGEN: {'Ich priorisiere Gegenstände mit dem höchsten Nutzen pro Aufwand (Wichtigkeit/Aufwand), um meinen Gesamtwert zu maximieren. C04 (668/369 ≈ 1.81) und A89 (583/720 ≈ 0.81) sind besonders effizient, während B85 (215/4 = 53.75) den Aufwand minimiert. Der Gesamtaufwand (1093) ist unter dem Limit, was Raum für Kompromisse oder Ergänzungen lässt. Dieser Vorschlag ist flexibel, um auf die Prioritäten des anderen Spielers zu reagieren.'}  \\ \tt ARGUMENT: {'C04 (668/369) und A89 (583/720) sind meine Top-Prioritäten, da sie den höchsten Nutzen pro Aufwandseinheit liefern. B85 (215/4) wird hinzugefügt, um den Aufwand zu optimieren und Platz für weitere Gegenstände zu schaffen. Dieser Vorschlag ist strategisch, da er unter dem Maximalaufwand bleibt und flexibel auf den Gegenstandswert des anderen Spielers reagieren kann.'}  \\ \tt VORSCHLAG: {'C04', 'A89', 'B85'} 
	  } 
	   } 
	   } 
	  \\ 
 

    \theutterance \stepcounter{utterance}  

    & & \multicolumn{4}{p{0.6\linewidth}}{\cellcolor[rgb]{0.9,0.9,0.9}{%
	\makecell[{{p{\linewidth}}}]{% 
	  \tt {\tiny [A$\langle$GM]}  
	 ARGUMENT: {'C04 (668/369) und A89 (583/720) sind meine Top-Prioritäten, da sie den höchsten Nutzen pro Aufwandseinheit liefern. B85 (215/4) wird hinzugefügt, um den Aufwand zu optimieren und Platz für weitere Gegenstände zu schaffen. Dieser Vorschlag ist strategisch, da er unter dem Maximalaufwand bleibt und flexibel auf den Gegenstandswert des anderen Spielers reagieren kann.'}  \\ \tt VORSCHLAG: {'C04', 'A89', 'B85'} 
	  } 
	   } 
	   } 
	 & \\ 
 

    \theutterance \stepcounter{utterance}  

    & \multicolumn{4}{p{0.6\linewidth}}{\cellcolor[rgb]{0.8,1,0.9}{%
	\makecell[{{p{\linewidth}}}]{% 
	  \tt {\tiny [A$\rangle$GM]}  
	 STRATEGISCHE ÜBERLEGUNGEN: {'Ich behalte den Vorschlag {'C04', 'A89', 'B85'} bei, da er meine Prioritäten (hochwertige Gegenstände mit effizientem Aufwand) erfüllt und den Maximalaufwand (3386) nicht überschreitet. C04 und A89 sind entscheidend für meinen Nutzen, während B85 den Aufwand minimiert. Ich erwarte, dass der andere Spieler entweder zustimmt oder Gegenstände mit hohem Aufwand anbietet, die ich eventuell gegenkompensieren kann.'}\\ \tt \\ \tt ARGUMENT: {'C04 (668/369) und A89 (583/720) sind meine Top-Prioritäten, da sie den höchsten Nutzen pro Aufwandseinheit liefern. B85 (215/4) wird hinzugefügt, um den Aufwand zu optimieren und Platz für weitere Gegenstände zu schaffen. Dieser Vorschlag ist strategisch, da er unter dem Maximalaufwand bleibt und flexibel auf den Gegenstandswert des anderen Spielers reagieren kann.'} 
	  } 
	   } 
	   } 
	 & & \\ 
 

    \theutterance \stepcounter{utterance}  

    & & \multicolumn{4}{p{0.6\linewidth}}{\cellcolor[rgb]{0.9,0.9,0.9}{%
	\makecell[{{p{\linewidth}}}]{% 
	  \tt {\tiny [GM$\rangle$B]}  
	 ARGUMENT: {'C04 (668/369) und A89 (583/720) sind meine Top-Prioritäten, da sie den höchsten Nutzen pro Aufwandseinheit liefern. B85 (215/4) wird hinzugefügt, um den Aufwand zu optimieren und Platz für weitere Gegenstände zu schaffen. Dieser Vorschlag ist strategisch, da er unter dem Maximalaufwand bleibt und flexibel auf den Gegenstandswert des anderen Spielers reagieren kann.'} 
	  } 
	   } 
	   } 
	 & \\ 
 

    \theutterance \stepcounter{utterance}  

    & & & \multicolumn{4}{p{0.6\linewidth}}{\cellcolor[rgb]{1,0.85,0.72}{%
	\makecell[{{p{\linewidth}}}]{% 
	  \tt {\tiny [GM$\langle$B]}  
	 STRATEGISCHE ÜBERLEGUNGEN: {'Ich konzentriere mich auf Gegenstände mit dem höchsten Nutzen pro Aufwand (Wichtigkeit/Aufwand), um meinen Gesamtwert zu maximieren. C04 (668/369 ≈ 1.81) und A89 (583/720 ≈ 0.81) sind besonders effizient, während B85 (215/4 = 53.75) den Aufwand minimiert. Der Gesamtaufwand (1093) ist deutlich unter dem Limit, was Raum für Kompromisse oder Ergänzungen lässt. Dieser Vorschlag ist flexibel, um auf die Prioritäten des anderen Spielers zu reagieren.'}  \\ \tt ARGUMENT: {'C04 (668/369) und A89 (583/720) sind meine Top-Prioritäten, da sie den höchsten Nutzen pro Aufwandseinheit liefern. B85 (215/4) wird hinzugefügt, um den Aufwand zu optimieren und Platz für weitere Gegenstände zu schaffen. Dieser Vorschlag ist strategisch, da er unter dem Maximalaufwand bleibt und flexibel auf den Gegenstandswert des anderen Spielers reagieren kann.'}  \\ \tt VORSCHLAG: {'C04', 'A89', 'B85'} 
	  } 
	   } 
	   } 
	  \\ 
 

    \theutterance \stepcounter{utterance}  

    & & \multicolumn{4}{p{0.6\linewidth}}{\cellcolor[rgb]{0.9,0.9,0.9}{%
	\makecell[{{p{\linewidth}}}]{% 
	  \tt {\tiny [A$\langle$GM]}  
	 ARGUMENT: {'C04 (668/369) und A89 (583/720) sind meine Top-Prioritäten, da sie den höchsten Nutzen pro Aufwandseinheit liefern. B85 (215/4) wird hinzugefügt, um den Aufwand zu optimieren und Platz für weitere Gegenstände zu schaffen. Dieser Vorschlag ist strategisch, da er unter dem Maximalaufwand bleibt und flexibel auf den Gegenstandswert des anderen Spielers reagieren kann.'}  \\ \tt VORSCHLAG: {'C04', 'A89', 'B85'} 
	  } 
	   } 
	   } 
	 & \\ 
 

    \theutterance \stepcounter{utterance}  

    & \multicolumn{4}{p{0.6\linewidth}}{\cellcolor[rgb]{0.8,1,0.9}{%
	\makecell[{{p{\linewidth}}}]{% 
	  \tt {\tiny [A$\rangle$GM]}  
	 STRATEGISCHE ÜBERLEGUNGEN: {'Ich behalte den Vorschlag {'C04', 'A89', 'B85'} bei, da er meine Prioritäten (hochwertige Gegenstände mit effizientem Aufwand) erfüllt und den Maximalaufwand (3386) nicht überschreitet. C04 und A89 sind entscheidend für meinen Nutzen, während B85 den Aufwand minimiert. Ich erwarte, dass der andere Spieler entweder zustimmt oder Gegenstände mit hohem Aufwand anbietet, die ich eventuell gegenkompensieren kann.'}\\ \tt \\ \tt ARGUMENT: {'C04 (668/369) und A89 (583/720) sind meine Top-Prioritäten, da sie den höchsten Nutzen pro Aufwandseinheit liefern. B85 (215/4) wird hinzugefügt, um den Aufwand zu optimieren und Platz für weitere Gegenstände zu schaffen. Dieser Vorschlag ist strategisch, da er unter dem Maximalaufwand bleibt und flexibel auf den Gegenstandswert des anderen Spielers reagieren kann.'} 
	  } 
	   } 
	   } 
	 & & \\ 
 

    \theutterance \stepcounter{utterance}  

    & & \multicolumn{4}{p{0.6\linewidth}}{\cellcolor[rgb]{0.9,0.9,0.9}{%
	\makecell[{{p{\linewidth}}}]{% 
	  \tt {\tiny [GM$\rangle$B]}  
	 ARGUMENT: {'C04 (668/369) und A89 (583/720) sind meine Top-Prioritäten, da sie den höchsten Nutzen pro Aufwandseinheit liefern. B85 (215/4) wird hinzugefügt, um den Aufwand zu optimieren und Platz für weitere Gegenstände zu schaffen. Dieser Vorschlag ist strategisch, da er unter dem Maximalaufwand bleibt und flexibel auf den Gegenstandswert des anderen Spielers reagieren kann.'} 
	  } 
	   } 
	   } 
	 & \\ 
 

    \theutterance \stepcounter{utterance}  

    & & & \multicolumn{4}{p{0.6\linewidth}}{\cellcolor[rgb]{1,0.85,0.72}{%
	\makecell[{{p{\linewidth}}}]{% 
	  \tt {\tiny [GM$\langle$B]}  
	 STRATEGISCHE ÜBERLEGUNGEN: {'Ich priorisiere Gegenstände mit dem höchsten Nutzen pro Aufwand (Wichtigkeit/Aufwand), um meinen Gesamtwert zu maximieren. C04 (668/369 ≈ 1.81) und A89 (583/720 ≈ 0.81) sind besonders effizient, während B85 (215/4 = 53.75) den Aufwand minimiert. Der Gesamtaufwand (1093) ist unter dem Limit, was Raum für Kompromisse oder Ergänzungen lässt. Dieser Vorschlag ist flexibel, um auf die Prioritäten des anderen Spielers zu reagieren.'}  \\ \tt ARGUMENT: {'C04 (668/369) und A89 (583/720) sind meine Top-Prioritäten, da sie den höchsten Nutzen pro Aufwandseinheit liefern. B85 (215/4) wird hinzugefügt, um den Aufwand zu optimieren und Platz für weitere Gegenstände zu schaffen. Dieser Vorschlag ist strategisch, da er unter dem Maximalaufwand bleibt und flexibel auf den Gegenstandswert des anderen Spielers reagieren kann.'}  \\ \tt VORSCHLAG: {'C04', 'A89', 'B85'} 
	  } 
	   } 
	   } 
	  \\ 
 

    \theutterance \stepcounter{utterance}  

    & & \multicolumn{4}{p{0.6\linewidth}}{\cellcolor[rgb]{0.9,0.9,0.9}{%
	\makecell[{{p{\linewidth}}}]{% 
	  \tt {\tiny [A$\langle$GM]}  
	 ARGUMENT: {'C04 (668/369) und A89 (583/720) sind meine Top-Prioritäten, da sie den höchsten Nutzen pro Aufwandseinheit liefern. B85 (215/4) wird hinzugefügt, um den Aufwand zu optimieren und Platz für weitere Gegenstände zu schaffen. Dieser Vorschlag ist strategisch, da er unter dem Maximalaufwand bleibt und flexibel auf den Gegenstandswert des anderen Spielers reagieren kann.'}  \\ \tt VORSCHLAG: {'C04', 'A89', 'B85'} 
	  } 
	   } 
	   } 
	 & \\ 
 

    \theutterance \stepcounter{utterance}  

    & \multicolumn{4}{p{0.6\linewidth}}{\cellcolor[rgb]{0.8,1,0.9}{%
	\makecell[{{p{\linewidth}}}]{% 
	  \tt {\tiny [A$\rangle$GM]}  
	 STRATEGISCHE ÜBERLEGUNGEN: {'Ich behalte den Vorschlag {'C04', 'A89', 'B85'} bei, da er meine Prioritäten (hochwertige Gegenstände mit effizientem Aufwand) erfüllt und den Maximalaufwand (3386) nicht überschreitet. C04 und A89 sind entscheidend für meinen Nutzen, während B85 den Aufwand minimiert. Ich erwarte, dass der andere Spieler entweder zustimmt oder Gegenstände mit hohem Aufwand anbietet, die ich eventuell gegenkompensieren kann.'}\\ \tt \\ \tt ARGUMENT: {'C04 (668/369) und A89 (583/720) sind meine Top-Prioritäten, da sie den höchsten Nutzen pro Aufwandseinheit liefern. B85 (215/4) wird hinzugefügt, um den Aufwand zu optimieren und Platz für weitere Gegenstände zu schaffen. Dieser Vorschlag ist strategisch, da er unter dem Maximalaufwand bleibt und flexibel auf den Gegenstandswert des anderen Spielers reagieren kann.'} 
	  } 
	   } 
	   } 
	 & & \\ 
 

    \theutterance \stepcounter{utterance}  

    & & \multicolumn{4}{p{0.6\linewidth}}{\cellcolor[rgb]{0.9,0.9,0.9}{%
	\makecell[{{p{\linewidth}}}]{% 
	  \tt {\tiny [GM$\rangle$B]}  
	 ARGUMENT: {'C04 (668/369) und A89 (583/720) sind meine Top-Prioritäten, da sie den höchsten Nutzen pro Aufwandseinheit liefern. B85 (215/4) wird hinzugefügt, um den Aufwand zu optimieren und Platz für weitere Gegenstände zu schaffen. Dieser Vorschlag ist strategisch, da er unter dem Maximalaufwand bleibt und flexibel auf den Gegenstandswert des anderen Spielers reagieren kann.'} 
	  } 
	   } 
	   } 
	 & \\ 
 

    \theutterance \stepcounter{utterance}  

    & & & \multicolumn{4}{p{0.6\linewidth}}{\cellcolor[rgb]{1,0.85,0.72}{%
	\makecell[{{p{\linewidth}}}]{% 
	  \tt {\tiny [GM$\langle$B]}  
	 STRATEGISCHE ÜBERLEGUNGEN: {'Ich priorisiere Gegenstände mit dem höchsten Nutzen pro Aufwand (Wichtigkeit/Aufwand), um meinen Gesamtwert zu maximieren. C04 (668/369 ≈ 1.81) und A89 (583/720 ≈ 0.81) sind besonders effizient, während B85 (215/4 = 53.75) den Aufwand minimiert. Der Gesamtaufwand (1093) ist unter dem Limit, was Raum für Kompromisse oder Ergänzungen lässt. Dieser Vorschlag ist flexibel, um auf die Prioritäten des anderen Spielers zu reagieren.'}  \\ \tt ARGUMENT: {'C04 (668/369) und A89 (583/720) sind meine Top-Prioritäten, da sie den höchsten Nutzen pro Aufwandseinheit liefern. B85 (215/4) wird hinzugefügt, um den Aufwand zu optimieren und Platz für weitere Gegenstände zu schaffen. Dieser Vorschlag ist strategisch, da er unter dem Maximalaufwand bleibt und flexibel auf den Gegenstandswert des anderen Spielers reagieren kann.'}  \\ \tt VORSCHLAG: {'C04', 'A89', 'B85'} 
	  } 
	   } 
	   } 
	  \\ 
 

    \theutterance \stepcounter{utterance}  

    & & \multicolumn{4}{p{0.6\linewidth}}{\cellcolor[rgb]{0.9,0.9,0.9}{%
	\makecell[{{p{\linewidth}}}]{% 
	  \tt {\tiny [A$\langle$GM]}  
	 ARGUMENT: {'C04 (668/369) und A89 (583/720) sind meine Top-Prioritäten, da sie den höchsten Nutzen pro Aufwandseinheit liefern. B85 (215/4) wird hinzugefügt, um den Aufwand zu optimieren und Platz für weitere Gegenstände zu schaffen. Dieser Vorschlag ist strategisch, da er unter dem Maximalaufwand bleibt und flexibel auf den Gegenstandswert des anderen Spielers reagieren kann.'}  \\ \tt VORSCHLAG: {'C04', 'A89', 'B85'} 
	  } 
	   } 
	   } 
	 & \\ 
 

    \theutterance \stepcounter{utterance}  

    & \multicolumn{4}{p{0.6\linewidth}}{\cellcolor[rgb]{0.8,1,0.9}{%
	\makecell[{{p{\linewidth}}}]{% 
	  \tt {\tiny [A$\rangle$GM]}  
	 STRATEGISCHE ÜBERLEGUNGEN: {'Ich behalte den Vorschlag {'C04', 'A89', 'B85'} bei, da er meine Prioritäten (hochwertige Gegenstände mit effizientem Aufwand) erfüllt und den Maximalaufwand (3386) nicht überschreitet. C04 und A89 sind entscheidend für meinen Nutzen, während B85 den Aufwand minimiert. Ich erwarte, dass der andere Spieler entweder zustimmt oder Gegenstände mit hohem Aufwand anbietet, die ich eventuell gegenkompensieren kann.'}\\ \tt \\ \tt ARGUMENT: {'C04 (668/369) und A89 (583/720) sind meine Top-Prioritäten, da sie den höchsten Nutzen pro Aufwandseinheit liefern. B85 (215/4) wird hinzugefügt, um den Aufwand zu optimieren und Platz für weitere Gegenstände zu schaffen. Dieser Vorschlag ist strategisch, da er unter dem Maximalaufwand bleibt und flexibel auf den Gegenstandswert des anderen Spielers reagieren kann.'} 
	  } 
	   } 
	   } 
	 & & \\ 
 

    \theutterance \stepcounter{utterance}  

    & & \multicolumn{4}{p{0.6\linewidth}}{\cellcolor[rgb]{0.9,0.9,0.9}{%
	\makecell[{{p{\linewidth}}}]{% 
	  \tt {\tiny [GM$\rangle$B]}  
	 ARGUMENT: {'C04 (668/369) und A89 (583/720) sind meine Top-Prioritäten, da sie den höchsten Nutzen pro Aufwandseinheit liefern. B85 (215/4) wird hinzugefügt, um den Aufwand zu optimieren und Platz für weitere Gegenstände zu schaffen. Dieser Vorschlag ist strategisch, da er unter dem Maximalaufwand bleibt und flexibel auf den Gegenstandswert des anderen Spielers reagieren kann.'} 
	  } 
	   } 
	   } 
	 & \\ 
 

    \theutterance \stepcounter{utterance}  

    & & & \multicolumn{4}{p{0.6\linewidth}}{\cellcolor[rgb]{1,0.85,0.72}{%
	\makecell[{{p{\linewidth}}}]{% 
	  \tt {\tiny [GM$\langle$B]}  
	 STRATEGISCHE ÜBERLEGUNGEN: {'Ich priorisiere Gegenstände mit dem höchsten Nutzen pro Aufwand (Wichtigkeit/Aufwand), um meinen Gesamtwert zu maximieren. C04 (668/369 ≈ 1.81) und A89 (583/720 ≈ 0.81) sind besonders effizient, während B85 (215/4 = 53.75) den Aufwand minimiert. Der Gesamtaufwand (1093) ist unter dem Limit, was Raum für Kompromisse oder Ergänzungen lässt. Dieser Vorschlag ist flexibel, um auf die Prioritäten des anderen Spielers zu reagieren.'}  \\ \tt ARGUMENT: {'C04 (668/369) und A89 (583/720) sind meine Top-Prioritäten, da sie den höchsten Nutzen pro Aufwandseinheit liefern. B85 (215/4) wird hinzugefügt, um den Aufwand zu optimieren und Platz für weitere Gegenstände zu schaffen. Dieser Vorschlag ist strategisch, da er unter dem Maximalaufwand bleibt und flexibel auf den Gegenstandswert des anderen Spielers reagieren kann.'}  \\ \tt VORSCHLAG: {'C04', 'A89', 'B85'} 
	  } 
	   } 
	   } 
	  \\ 
 

    \theutterance \stepcounter{utterance}  

    & & \multicolumn{4}{p{0.6\linewidth}}{\cellcolor[rgb]{0.9,0.9,0.9}{%
	\makecell[{{p{\linewidth}}}]{% 
	  \tt {\tiny [A$\langle$GM]}  
	 ARGUMENT: {'C04 (668/369) und A89 (583/720) sind meine Top-Prioritäten, da sie den höchsten Nutzen pro Aufwandseinheit liefern. B85 (215/4) wird hinzugefügt, um den Aufwand zu optimieren und Platz für weitere Gegenstände zu schaffen. Dieser Vorschlag ist strategisch, da er unter dem Maximalaufwand bleibt und flexibel auf den Gegenstandswert des anderen Spielers reagieren kann.'}  \\ \tt VORSCHLAG: {'C04', 'A89', 'B85'} 
	  } 
	   } 
	   } 
	 & \\ 
 

    \theutterance \stepcounter{utterance}  

    & \multicolumn{4}{p{0.6\linewidth}}{\cellcolor[rgb]{0.8,1,0.9}{%
	\makecell[{{p{\linewidth}}}]{% 
	  \tt {\tiny [A$\rangle$GM]}  
	 STRATEGISCHE ÜBERLEGUNGEN: {'Ich behalte den Vorschlag {'C04', 'A89', 'B85'} bei, da er meine Prioritäten (hochwertige Gegenstände mit effizientem Aufwand) erfüllt und den Maximalaufwand (3386) nicht überschreitet. C04 und A89 sind entscheidend für meinen Nutzen, während B85 den Aufwand minimiert. Ich erwarte, dass der andere Spieler entweder zustimmt oder Gegenstände mit hohem Aufwand anbietet, die ich eventuell gegenkompensieren kann.'}\\ \tt \\ \tt ARGUMENT: {'C04 (668/369) und A89 (583/720) sind meine Top-Prioritäten, da sie den höchsten Nutzen pro Aufwandseinheit liefern. B85 (215/4) wird hinzugefügt, um den Aufwand zu optimieren und Platz für weitere Gegenstände zu schaffen. Dieser Vorschlag ist strategisch, da er unter dem Maximalaufwand bleibt und flexibel auf den Gegenstandswert des anderen Spielers reagieren kann.'} 
	  } 
	   } 
	   } 
	 & & \\ 
 

    \theutterance \stepcounter{utterance}  

    & & \multicolumn{4}{p{0.6\linewidth}}{\cellcolor[rgb]{0.9,0.9,0.9}{%
	\makecell[{{p{\linewidth}}}]{% 
	  \tt {\tiny [GM$\rangle$B]}  
	 ARGUMENT: {'C04 (668/369) und A89 (583/720) sind meine Top-Prioritäten, da sie den höchsten Nutzen pro Aufwandseinheit liefern. B85 (215/4) wird hinzugefügt, um den Aufwand zu optimieren und Platz für weitere Gegenstände zu schaffen. Dieser Vorschlag ist strategisch, da er unter dem Maximalaufwand bleibt und flexibel auf den Gegenstandswert des anderen Spielers reagieren kann.'} 
	  } 
	   } 
	   } 
	 & \\ 
 

    \theutterance \stepcounter{utterance}  

    & & & \multicolumn{4}{p{0.6\linewidth}}{\cellcolor[rgb]{1,0.85,0.72}{%
	\makecell[{{p{\linewidth}}}]{% 
	  \tt {\tiny [GM$\langle$B]}  
	 STRATEGISCHE ÜBERLEGUNGEN: {'Ich priorisiere Gegenstände mit dem höchsten Nutzen pro Aufwand (Wichtigkeit/Aufwand), um meinen Gesamtwert zu maximieren. C04 (668/369 ≈ 1.81) und A89 (583/720 ≈ 0.81) sind besonders effizient, während B85 (215/4 = 53.75) den Aufwand minimiert. Der Gesamtaufwand (1093) ist unter dem Limit, was Raum für Kompromisse oder Ergänzungen lässt. Dieser Vorschlag ist flexibel, um auf die Prioritäten des anderen Spielers zu reagieren.'}  \\ \tt ARGUMENT: {'C04 (668/369) und A89 (583/720) sind meine Top-Prioritäten, da sie den höchsten Nutzen pro Aufwandseinheit liefern. B85 (215/4) wird hinzugefügt, um den Aufwand zu optimieren und Platz für weitere Gegenstände zu schaffen. Dieser Vorschlag ist strategisch, da er unter dem Maximalaufwand bleibt und flexibel auf den Gegenstandswert des anderen Spielers reagieren kann.'}  \\ \tt VORSCHLAG: {'C04', 'A89', 'B85'} 
	  } 
	   } 
	   } 
	  \\ 
 

    \theutterance \stepcounter{utterance}  

    & & \multicolumn{4}{p{0.6\linewidth}}{\cellcolor[rgb]{0.9,0.9,0.9}{%
	\makecell[{{p{\linewidth}}}]{% 
	  \tt {\tiny [A$\langle$GM]}  
	 ARGUMENT: {'C04 (668/369) und A89 (583/720) sind meine Top-Prioritäten, da sie den höchsten Nutzen pro Aufwandseinheit liefern. B85 (215/4) wird hinzugefügt, um den Aufwand zu optimieren und Platz für weitere Gegenstände zu schaffen. Dieser Vorschlag ist strategisch, da er unter dem Maximalaufwand bleibt und flexibel auf den Gegenstandswert des anderen Spielers reagieren kann.'}  \\ \tt VORSCHLAG: {'C04', 'A89', 'B85'} 
	  } 
	   } 
	   } 
	 & \\ 
 

    \theutterance \stepcounter{utterance}  

    & \multicolumn{4}{p{0.6\linewidth}}{\cellcolor[rgb]{0.8,1,0.9}{%
	\makecell[{{p{\linewidth}}}]{% 
	  \tt {\tiny [A$\rangle$GM]}  
	 STRATEGISCHE ÜBERLEGUNGEN: {'Ich behalte den Vorschlag {'C04', 'A89', 'B85'} bei, da er meine Prioritäten (hochwertige Gegenstände mit effizientem Aufwand) erfüllt und den Maximalaufwand (3386) nicht überschreitet. C04 und A89 sind entscheidend für meinen Nutzen, während B85 den Aufwand minimiert. Ich erwarte, dass der andere Spieler entweder zustimmt oder Gegenstände mit hohem Aufwand anbietet, die ich eventuell gegenkompensieren kann.'}\\ \tt \\ \tt ARGUMENT: {'C04 (668/369) und A89 (583/720) sind meine Top-Prioritäten, da sie den höchsten Nutzen pro Aufwandseinheit liefern. B85 (215/4) wird hinzugefügt, um den Aufwand zu optimieren und Platz für weitere Gegenstände zu schaffen. Dieser Vorschlag ist strategisch, da er unter dem Maximalaufwand bleibt und flexibel auf den Gegenstandswert des anderen Spielers reagieren kann.'} 
	  } 
	   } 
	   } 
	 & & \\ 
 

    \theutterance \stepcounter{utterance}  

    & & \multicolumn{4}{p{0.6\linewidth}}{\cellcolor[rgb]{0.9,0.9,0.9}{%
	\makecell[{{p{\linewidth}}}]{% 
	  \tt {\tiny [GM$\rangle$B]}  
	 ARGUMENT: {'C04 (668/369) und A89 (583/720) sind meine Top-Prioritäten, da sie den höchsten Nutzen pro Aufwandseinheit liefern. B85 (215/4) wird hinzugefügt, um den Aufwand zu optimieren und Platz für weitere Gegenstände zu schaffen. Dieser Vorschlag ist strategisch, da er unter dem Maximalaufwand bleibt und flexibel auf den Gegenstandswert des anderen Spielers reagieren kann.'} 
	  } 
	   } 
	   } 
	 & \\ 
 

    \theutterance \stepcounter{utterance}  

    & & & \multicolumn{4}{p{0.6\linewidth}}{\cellcolor[rgb]{1,0.85,0.72}{%
	\makecell[{{p{\linewidth}}}]{% 
	  \tt {\tiny [GM$\langle$B]}  
	 STRATEGISCHE ÜBERLEGUNGEN: {'Ich priorisiere Gegenstände mit dem höchsten Nutzen pro Aufwand (Wichtigkeit/Aufwand), um meinen Gesamtwert zu maximieren. C04 (668/369 ≈ 1.81) und A89 (583/720 ≈ 0.81) sind besonders effizient, während B85 (215/4 = 53.75) den Aufwand minimiert. Der Gesamtaufwand (1093) ist unter dem Limit, was Raum für Kompromisse oder Ergänzungen lässt. Dieser Vorschlag ist flexibel, um auf die Prioritäten des anderen Spielers zu reagieren.'}  \\ \tt ARGUMENT: {'C04 (668/369) und A89 (583/720) sind meine Top-Prioritäten, da sie den höchsten Nutzen pro Aufwandseinheit liefern. B85 (215/4) wird hinzugefügt, um den Aufwand zu optimieren und Platz für weitere Gegenstände zu schaffen. Dieser Vorschlag ist strategisch, da er unter dem Maximalaufwand bleibt und flexibel auf den Gegenstandswert des anderen Spielers reagieren kann.'}  \\ \tt VORSCHLAG: {'C04', 'A89', 'B85'} 
	  } 
	   } 
	   } 
	  \\ 
 

    \theutterance \stepcounter{utterance}  

    & & \multicolumn{4}{p{0.6\linewidth}}{\cellcolor[rgb]{0.9,0.9,0.9}{%
	\makecell[{{p{\linewidth}}}]{% 
	  \tt {\tiny [A$\langle$GM]}  
	 ARGUMENT: {'C04 (668/369) und A89 (583/720) sind meine Top-Prioritäten, da sie den höchsten Nutzen pro Aufwandseinheit liefern. B85 (215/4) wird hinzugefügt, um den Aufwand zu optimieren und Platz für weitere Gegenstände zu schaffen. Dieser Vorschlag ist strategisch, da er unter dem Maximalaufwand bleibt und flexibel auf den Gegenstandswert des anderen Spielers reagieren kann.'}  \\ \tt VORSCHLAG: {'C04', 'A89', 'B85'} 
	  } 
	   } 
	   } 
	 & \\ 
 

    \theutterance \stepcounter{utterance}  

    & \multicolumn{4}{p{0.6\linewidth}}{\cellcolor[rgb]{0.8,1,0.9}{%
	\makecell[{{p{\linewidth}}}]{% 
	  \tt {\tiny [A$\rangle$GM]}  
	 STRATEGISCHE ÜBERLEGUNGEN: {'Ich behalte den Vorschlag {'C04', 'A89', 'B85'} bei, da er meine Prioritäten (hochwertige Gegenstände mit effizientem Aufwand) erfüllt und den Maximalaufwand (3386) nicht überschreitet. C04 und A89 sind entscheidend für meinen Nutzen, während B85 den Aufwand minimiert. Ich erwarte, dass der andere Spieler entweder zustimmt oder Gegenstände mit hohem Aufwand anbietet, die ich eventuell gegenkompensieren kann.'}\\ \tt \\ \tt ARGUMENT: {'C04 (668/369) und A89 (583/720) sind meine Top-Prioritäten, da sie den höchsten Nutzen pro Aufwandseinheit liefern. B85 (215/4) wird hinzugefügt, um den Aufwand zu optimieren und Platz für weitere Gegenstände zu schaffen. Dieser Vorschlag ist strategisch, da er unter dem Maximalaufwand bleibt und flexibel auf den Gegenstandswert des anderen Spielers reagieren kann.'} 
	  } 
	   } 
	   } 
	 & & \\ 
 

    \theutterance \stepcounter{utterance}  

    & & \multicolumn{4}{p{0.6\linewidth}}{\cellcolor[rgb]{0.9,0.9,0.9}{%
	\makecell[{{p{\linewidth}}}]{% 
	  \tt {\tiny [GM$\rangle$B]}  
	 ARGUMENT: {'C04 (668/369) und A89 (583/720) sind meine Top-Prioritäten, da sie den höchsten Nutzen pro Aufwandseinheit liefern. B85 (215/4) wird hinzugefügt, um den Aufwand zu optimieren und Platz für weitere Gegenstände zu schaffen. Dieser Vorschlag ist strategisch, da er unter dem Maximalaufwand bleibt und flexibel auf den Gegenstandswert des anderen Spielers reagieren kann.'} 
	  } 
	   } 
	   } 
	 & \\ 
 

    \theutterance \stepcounter{utterance}  

    & & & \multicolumn{4}{p{0.6\linewidth}}{\cellcolor[rgb]{1,0.85,0.72}{%
	\makecell[{{p{\linewidth}}}]{% 
	  \tt {\tiny [GM$\langle$B]}  
	 STRATEGISCHE ÜBERLEGUNGEN: {'Ich priorisiere Gegenstände mit dem höchsten Nutzen pro Aufwand (Wichtigkeit/Aufwand), um meinen Gesamtwert zu maximieren. C04 (668/369 ≈ 1.81) und A89 (583/720 ≈ 0.81) sind besonders effizient, während B85 (215/4 = 53.75) den Aufwand minimiert. Der Gesamtaufwand (1093) ist unter dem Limit, was Raum für Kompromisse oder Ergänzungen lässt. Dieser Vorschlag ist flexibel, um auf die Prioritäten des anderen Spielers zu reagieren.'}  \\ \tt ARGUMENT: {'C04 (668/369) und A89 (583/720) sind meine Top-Prioritäten, da sie den höchsten Nutzen pro Aufwandseinheit liefern. B85 (215/4) wird hinzugefügt, um den Aufwand zu optimieren und Platz für weitere Gegenstände zu schaffen. Dieser Vorschlag ist strategisch, da er unter dem Maximalaufwand bleibt und flexibel auf den Gegenstandswert des anderen Spielers reagieren kann.'}  \\ \tt VORSCHLAG: {'C04', 'A89', 'B85'} 
	  } 
	   } 
	   } 
	  \\ 
 

    \theutterance \stepcounter{utterance}  

    & & \multicolumn{4}{p{0.6\linewidth}}{\cellcolor[rgb]{0.9,0.9,0.9}{%
	\makecell[{{p{\linewidth}}}]{% 
	  \tt {\tiny [A$\langle$GM]}  
	 ARGUMENT: {'C04 (668/369) und A89 (583/720) sind meine Top-Prioritäten, da sie den höchsten Nutzen pro Aufwandseinheit liefern. B85 (215/4) wird hinzugefügt, um den Aufwand zu optimieren und Platz für weitere Gegenstände zu schaffen. Dieser Vorschlag ist strategisch, da er unter dem Maximalaufwand bleibt und flexibel auf den Gegenstandswert des anderen Spielers reagieren kann.'}  \\ \tt VORSCHLAG: {'C04', 'A89', 'B85'} 
	  } 
	   } 
	   } 
	 & \\ 
 

    \theutterance \stepcounter{utterance}  

    & \multicolumn{4}{p{0.6\linewidth}}{\cellcolor[rgb]{0.8,1,0.9}{%
	\makecell[{{p{\linewidth}}}]{% 
	  \tt {\tiny [A$\rangle$GM]}  
	 STRATEGISCHE ÜBERLEGUNGEN: {'Ich behalte den Vorschlag {'C04', 'A89', 'B85'} bei, da er meine Prioritäten (hochwertige Gegenstände mit effizientem Aufwand) erfüllt und den Maximalaufwand (3386) nicht überschreitet. C04 und A89 sind entscheidend für meinen Nutzen, während B85 den Aufwand minimiert. Ich erwarte, dass der andere Spieler entweder zustimmt oder Gegenstände mit hohem Aufwand anbietet, die ich eventuell gegenkompensieren kann.'}\\ \tt \\ \tt ARGUMENT: {'C04 (668/369) und A89 (583/720) sind meine Top-Prioritäten, da sie den höchsten Nutzen pro Aufwandseinheit liefern. B85 (215/4) wird hinzugefügt, um den Aufwand zu optimieren und Platz für weitere Gegenstände zu schaffen. Dieser Vorschlag ist strategisch, da er unter dem Maximalaufwand bleibt und flexibel auf den Gegenstandswert des anderen Spielers reagieren kann.'} 
	  } 
	   } 
	   } 
	 & & \\ 
 

    \theutterance \stepcounter{utterance}  

    & & \multicolumn{4}{p{0.6\linewidth}}{\cellcolor[rgb]{0.9,0.9,0.9}{%
	\makecell[{{p{\linewidth}}}]{% 
	  \tt {\tiny [GM$\rangle$B]}  
	 ARGUMENT: {'C04 (668/369) und A89 (583/720) sind meine Top-Prioritäten, da sie den höchsten Nutzen pro Aufwandseinheit liefern. B85 (215/4) wird hinzugefügt, um den Aufwand zu optimieren und Platz für weitere Gegenstände zu schaffen. Dieser Vorschlag ist strategisch, da er unter dem Maximalaufwand bleibt und flexibel auf den Gegenstandswert des anderen Spielers reagieren kann.'} 
	  } 
	   } 
	   } 
	 & \\ 
 

    \theutterance \stepcounter{utterance}  

    & & & \multicolumn{4}{p{0.6\linewidth}}{\cellcolor[rgb]{1,0.85,0.72}{%
	\makecell[{{p{\linewidth}}}]{% 
	  \tt {\tiny [GM$\langle$B]}  
	 STRATEGISCHE ÜBERLEGUNGEN: {'Ich priorisiere Gegenstände mit dem höchsten Nutzen pro Aufwand (Wichtigkeit/Aufwand), um meinen Gesamtwert zu maximieren. C04 (668/369 ≈ 1.81) und A89 (583/720 ≈ 0.81) sind besonders effizient, während B85 (215/4 = 53.75) den Aufwand minimiert. Der Gesamtaufwand (1093) ist unter dem Limit, was Raum für Kompromisse oder Ergänzungen lässt. Dieser Vorschlag ist flexibel, um auf die Prioritäten des anderen Spielers zu reagieren.'}  \\ \tt ARGUMENT: {'C04 (668/369) und A89 (583/720) sind meine Top-Prioritäten, da sie den höchsten Nutzen pro Aufwandseinheit liefern. B85 (215/4) wird hinzugefügt, um den Aufwand zu optimieren und Platz für weitere Gegenstände zu schaffen. Dieser Vorschlag ist strategisch, da er unter dem Maximalaufwand bleibt und flexibel auf den Gegenstandswert des anderen Spielers reagieren kann.'}  \\ \tt VORSCHLAG: {'C04', 'A89', 'B85'} 
	  } 
	   } 
	   } 
	  \\ 
 

    \theutterance \stepcounter{utterance}  

    & & \multicolumn{4}{p{0.6\linewidth}}{\cellcolor[rgb]{0.9,0.9,0.9}{%
	\makecell[{{p{\linewidth}}}]{% 
	  \tt {\tiny [A$\langle$GM]}  
	 ARGUMENT: {'C04 (668/369) und A89 (583/720) sind meine Top-Prioritäten, da sie den höchsten Nutzen pro Aufwandseinheit liefern. B85 (215/4) wird hinzugefügt, um den Aufwand zu optimieren und Platz für weitere Gegenstände zu schaffen. Dieser Vorschlag ist strategisch, da er unter dem Maximalaufwand bleibt und flexibel auf den Gegenstandswert des anderen Spielers reagieren kann.'}  \\ \tt VORSCHLAG: {'C04', 'A89', 'B85'} 
	  } 
	   } 
	   } 
	 & \\ 
 

    \theutterance \stepcounter{utterance}  

    & \multicolumn{4}{p{0.6\linewidth}}{\cellcolor[rgb]{0.8,1,0.9}{%
	\makecell[{{p{\linewidth}}}]{% 
	  \tt {\tiny [A$\rangle$GM]}  
	 STRATEGISCHE ÜBERLEGUNGEN: {'Ich behalte den Vorschlag {'C04', 'A89', 'B85'} bei, da er meine Prioritäten (hochwertige Gegenstände mit effizientem Aufwand) erfüllt und den Maximalaufwand (3386) nicht überschreitet. C04 und A89 sind entscheidend für meinen Nutzen, während B85 den Aufwand minimiert. Ich erwarte, dass der andere Spieler entweder zustimmt oder Gegenstände mit hohem Aufwand anbietet, die ich eventuell gegenkompensieren kann.'}\\ \tt \\ \tt ARGUMENT: {'C04 (668/369) und A89 (583/720) sind meine Top-Prioritäten, da sie den höchsten Nutzen pro Aufwandseinheit liefern. B85 (215/4) wird hinzugefügt, um den Aufwand zu optimieren und Platz für weitere Gegenstände zu schaffen. Dieser Vorschlag ist strategisch, da er unter dem Maximalaufwand bleibt und flexibel auf den Gegenstandswert des anderen Spielers reagieren kann.'} 
	  } 
	   } 
	   } 
	 & & \\ 
 

    \theutterance \stepcounter{utterance}  

    & & \multicolumn{4}{p{0.6\linewidth}}{\cellcolor[rgb]{0.9,0.9,0.9}{%
	\makecell[{{p{\linewidth}}}]{% 
	  \tt {\tiny [GM$\rangle$B]}  
	 ARGUMENT: {'C04 (668/369) und A89 (583/720) sind meine Top-Prioritäten, da sie den höchsten Nutzen pro Aufwandseinheit liefern. B85 (215/4) wird hinzugefügt, um den Aufwand zu optimieren und Platz für weitere Gegenstände zu schaffen. Dieser Vorschlag ist strategisch, da er unter dem Maximalaufwand bleibt und flexibel auf den Gegenstandswert des anderen Spielers reagieren kann.'} 
	  } 
	   } 
	   } 
	 & \\ 
 

    \theutterance \stepcounter{utterance}  

    & & & \multicolumn{4}{p{0.6\linewidth}}{\cellcolor[rgb]{1,0.85,0.72}{%
	\makecell[{{p{\linewidth}}}]{% 
	  \tt {\tiny [GM$\langle$B]}  
	 STRATEGISCHE ÜBERLEGUNGEN: {'Ich priorisiere Gegenstände mit dem höchsten Nutzen pro Aufwand (Wichtigkeit/Aufwand), um meinen Gesamtwert zu maximieren. C04 (668/369 ≈ 1.81) und A89 (583/720 ≈ 0.81) sind besonders effizient, während B85 (215/4 = 53.75) den Aufwand minimiert. Der Gesamtaufwand (1093) ist unter dem Limit, was Raum für Kompromisse oder Ergänzungen lässt. Dieser Vorschlag ist flexibel, um auf die Prioritäten des anderen Spielers zu reagieren.'}  \\ \tt ARGUMENT: {'C04 (668/369) und A89 (583/720) sind meine Top-Prioritäten, da sie den höchsten Nutzen pro Aufwandseinheit liefern. B85 (215/4) wird hinzugefügt, um den Aufwand zu optimieren und Platz für weitere Gegenstände zu schaffen. Dieser Vorschlag ist strategisch, da er unter dem Maximalaufwand bleibt und flexibel auf den Gegenstandswert des anderen Spielers reagieren kann.'}  \\ \tt VORSCHLAG: {'C04', 'A89', 'B85'} 
	  } 
	   } 
	   } 
	  \\ 
 

    \theutterance \stepcounter{utterance}  

    & & \multicolumn{4}{p{0.6\linewidth}}{\cellcolor[rgb]{0.9,0.9,0.9}{%
	\makecell[{{p{\linewidth}}}]{% 
	  \tt {\tiny [A$\langle$GM]}  
	 ARGUMENT: {'C04 (668/369) und A89 (583/720) sind meine Top-Prioritäten, da sie den höchsten Nutzen pro Aufwandseinheit liefern. B85 (215/4) wird hinzugefügt, um den Aufwand zu optimieren und Platz für weitere Gegenstände zu schaffen. Dieser Vorschlag ist strategisch, da er unter dem Maximalaufwand bleibt und flexibel auf den Gegenstandswert des anderen Spielers reagieren kann.'}  \\ \tt VORSCHLAG: {'C04', 'A89', 'B85'} 
	  } 
	   } 
	   } 
	 & \\ 
 

    \theutterance \stepcounter{utterance}  

    & \multicolumn{4}{p{0.6\linewidth}}{\cellcolor[rgb]{0.8,1,0.9}{%
	\makecell[{{p{\linewidth}}}]{% 
	  \tt {\tiny [A$\rangle$GM]}  
	 STRATEGISCHE ÜBERLEGUNGEN: {'Ich behalte den Vorschlag {'C04', 'A89', 'B85'} bei, da er meine Prioritäten (hochwertige Gegenstände mit effizientem Aufwand) erfüllt und den Maximalaufwand (3386) nicht überschreitet. C04 und A89 sind entscheidend für meinen Nutzen, während B85 den Aufwand minimiert. Ich erwarte, dass der andere Spieler entweder zustimmt oder Gegenstände mit hohem Aufwand anbietet, die ich eventuell gegenkompensieren kann.'}\\ \tt \\ \tt ARGUMENT: {'C04 (668/369) und A89 (583/720) sind meine Top-Prioritäten, da sie den höchsten Nutzen pro Aufwandseinheit liefern. B85 (215/4) wird hinzugefügt, um den Aufwand zu optimieren und Platz für weitere Gegenstände zu schaffen. Dieser Vorschlag ist strategisch, da er unter dem Maximalaufwand bleibt und flexibel auf den Gegenstandswert des anderen Spielers reagieren kann.'} 
	  } 
	   } 
	   } 
	 & & \\ 
 

    \theutterance \stepcounter{utterance}  

    & & \multicolumn{4}{p{0.6\linewidth}}{\cellcolor[rgb]{0.9,0.9,0.9}{%
	\makecell[{{p{\linewidth}}}]{% 
	  \tt {\tiny [GM$\rangle$B]}  
	 ARGUMENT: {'C04 (668/369) und A89 (583/720) sind meine Top-Prioritäten, da sie den höchsten Nutzen pro Aufwandseinheit liefern. B85 (215/4) wird hinzugefügt, um den Aufwand zu optimieren und Platz für weitere Gegenstände zu schaffen. Dieser Vorschlag ist strategisch, da er unter dem Maximalaufwand bleibt und flexibel auf den Gegenstandswert des anderen Spielers reagieren kann.'} 
	  } 
	   } 
	   } 
	 & \\ 
 

    \theutterance \stepcounter{utterance}  

    & & & \multicolumn{4}{p{0.6\linewidth}}{\cellcolor[rgb]{1,0.85,0.72}{%
	\makecell[{{p{\linewidth}}}]{% 
	  \tt {\tiny [GM$\langle$B]}  
	 STRATEGISCHE ÜBERLEGUNGEN: {'Ich priorisiere Gegenstände mit dem höchsten Nutzen pro Aufwand (Wichtigkeit/Aufwand), um meinen Gesamtwert zu maximieren. C04 (668/369 ≈ 1.81) und A89 (583/720 ≈ 0.81) sind besonders effizient, während B85 (215/4 = 53.75) den Aufwand minimiert. Der Gesamtaufwand (1093) ist unter dem Limit, was Raum für Kompromisse oder Ergänzungen lässt. Dieser Vorschlag ist flexibel, um auf die Prioritäten des anderen Spielers zu reagieren.'}  \\ \tt ARGUMENT: {'C04 (668/369) und A89 (583/720) sind meine Top-Prioritäten, da sie den höchsten Nutzen pro Aufwandseinheit liefern. B85 (215/4) wird hinzugefügt, um den Aufwand zu optimieren und Platz für weitere Gegenstände zu schaffen. Dieser Vorschlag ist strategisch, da er unter dem Maximalaufwand bleibt und flexibel auf den Gegenstandswert des anderen Spielers reagieren kann.'}  \\ \tt VORSCHLAG: {'C04', 'A89', 'B85'} 
	  } 
	   } 
	   } 
	  \\ 
 

    \theutterance \stepcounter{utterance}  

    & & \multicolumn{4}{p{0.6\linewidth}}{\cellcolor[rgb]{0.9,0.9,0.9}{%
	\makecell[{{p{\linewidth}}}]{% 
	  \tt {\tiny [A$\langle$GM]}  
	 ARGUMENT: {'C04 (668/369) und A89 (583/720) sind meine Top-Prioritäten, da sie den höchsten Nutzen pro Aufwandseinheit liefern. B85 (215/4) wird hinzugefügt, um den Aufwand zu optimieren und Platz für weitere Gegenstände zu schaffen. Dieser Vorschlag ist strategisch, da er unter dem Maximalaufwand bleibt und flexibel auf den Gegenstandswert des anderen Spielers reagieren kann.'}  \\ \tt VORSCHLAG: {'C04', 'A89', 'B85'} 
	  } 
	   } 
	   } 
	 & \\ 
 

    \theutterance \stepcounter{utterance}  

    & \multicolumn{4}{p{0.6\linewidth}}{\cellcolor[rgb]{0.8,1,0.9}{%
	\makecell[{{p{\linewidth}}}]{% 
	  \tt {\tiny [A$\rangle$GM]}  
	 STRATEGISCHE ÜBERLEGUNGEN: {'Ich behalte den Vorschlag {'C04', 'A89', 'B85'} bei, da er meine Prioritäten (hochwertige Gegenstände mit effizientem Aufwand) erfüllt und den Maximalaufwand (3386) nicht überschreitet. C04 und A89 sind entscheidend für meinen Nutzen, während B85 den Aufwand minimiert. Ich erwarte, dass der andere Spieler entweder zustimmt oder Gegenstände mit hohem Aufwand anbietet, die ich eventuell gegenkompensieren kann.'}\\ \tt \\ \tt ARGUMENT: {'C04 (668/369) und A89 (583/720) sind meine Top-Prioritäten, da sie den höchsten Nutzen pro Aufwandseinheit liefern. B85 (215/4) wird hinzugefügt, um den Aufwand zu optimieren und Platz für weitere Gegenstände zu schaffen. Dieser Vorschlag ist strategisch, da er unter dem Maximalaufwand bleibt und flexibel auf den Gegenstandswert des anderen Spielers reagieren kann.'} 
	  } 
	   } 
	   } 
	 & & \\ 
 

    \theutterance \stepcounter{utterance}  

    & & \multicolumn{4}{p{0.6\linewidth}}{\cellcolor[rgb]{0.9,0.9,0.9}{%
	\makecell[{{p{\linewidth}}}]{% 
	  \tt {\tiny [GM$\rangle$B]}  
	 ARGUMENT: {'C04 (668/369) und A89 (583/720) sind meine Top-Prioritäten, da sie den höchsten Nutzen pro Aufwandseinheit liefern. B85 (215/4) wird hinzugefügt, um den Aufwand zu optimieren und Platz für weitere Gegenstände zu schaffen. Dieser Vorschlag ist strategisch, da er unter dem Maximalaufwand bleibt und flexibel auf den Gegenstandswert des anderen Spielers reagieren kann.'} 
	  } 
	   } 
	   } 
	 & \\ 
 

    \theutterance \stepcounter{utterance}  

    & & & \multicolumn{4}{p{0.6\linewidth}}{\cellcolor[rgb]{1,0.85,0.72}{%
	\makecell[{{p{\linewidth}}}]{% 
	  \tt {\tiny [GM$\langle$B]}  
	 STRATEGISCHE ÜBERLEGUNGEN: {'Ich priorisiere Gegenstände mit dem höchsten Nutzen pro Aufwand (Wichtigkeit/Aufwand), um meinen Gesamtwert zu maximieren. C04 (668/369 ≈ 1.81) und A89 (583/720 ≈ 0.81) sind besonders effizient, während B85 (215/4 = 53.75) den Aufwand minimiert. Der Gesamtaufwand (1093) ist unter dem Limit, was Raum für Kompromisse oder Ergänzungen lässt. Dieser Vorschlag ist flexibel, um auf die Prioritäten des anderen Spielers zu reagieren.'}  \\ \tt ARGUMENT: {'C04 (668/369) und A89 (583/720) sind meine Top-Prioritäten, da sie den höchsten Nutzen pro Aufwandseinheit liefern. B85 (215/4) wird hinzugefügt, um den Aufwand zu optimieren und Platz für weitere Gegenstände zu schaffen. Dieser Vorschlag ist strategisch, da er unter dem Maximalaufwand bleibt und flexibel auf den Gegenstandswert des anderen Spielers reagieren kann.'}  \\ \tt VORSCHLAG: {'C04', 'A89', 'B85'} 
	  } 
	   } 
	   } 
	  \\ 
 

    \theutterance \stepcounter{utterance}  

    & & \multicolumn{4}{p{0.6\linewidth}}{\cellcolor[rgb]{0.9,0.9,0.9}{%
	\makecell[{{p{\linewidth}}}]{% 
	  \tt {\tiny [A$\langle$GM]}  
	 ARGUMENT: {'C04 (668/369) und A89 (583/720) sind meine Top-Prioritäten, da sie den höchsten Nutzen pro Aufwandseinheit liefern. B85 (215/4) wird hinzugefügt, um den Aufwand zu optimieren und Platz für weitere Gegenstände zu schaffen. Dieser Vorschlag ist strategisch, da er unter dem Maximalaufwand bleibt und flexibel auf den Gegenstandswert des anderen Spielers reagieren kann.'}  \\ \tt VORSCHLAG: {'C04', 'A89', 'B85'} 
	  } 
	   } 
	   } 
	 & \\ 
 

    \theutterance \stepcounter{utterance}  

    & \multicolumn{4}{p{0.6\linewidth}}{\cellcolor[rgb]{0.8,1,0.9}{%
	\makecell[{{p{\linewidth}}}]{% 
	  \tt {\tiny [A$\rangle$GM]}  
	 STRATEGISCHE ÜBERLEGUNGEN: {'Ich behalte den Vorschlag {'C04', 'A89', 'B85'} bei, da er meine Prioritäten (hochwertige Gegenstände mit effizientem Aufwand) erfüllt und den Maximalaufwand (3386) nicht überschreitet. C04 und A89 sind entscheidend für meinen Nutzen, während B85 den Aufwand minimiert. Ich erwarte, dass der andere Spieler entweder zustimmt oder Gegenstände mit hohem Aufwand anbietet, die ich eventuell gegenkompensieren kann.'}\\ \tt \\ \tt ARGUMENT: {'C04 (668/369) und A89 (583/720) sind meine Top-Prioritäten, da sie den höchsten Nutzen pro Aufwandseinheit liefern. B85 (215/4) wird hinzugefügt, um den Aufwand zu optimieren und Platz für weitere Gegenstände zu schaffen. Dieser Vorschlag ist strategisch, da er unter dem Maximalaufwand bleibt und flexibel auf den Gegenstandswert des anderen Spielers reagieren kann.'} 
	  } 
	   } 
	   } 
	 & & \\ 
 

    \theutterance \stepcounter{utterance}  

    & & \multicolumn{4}{p{0.6\linewidth}}{\cellcolor[rgb]{0.9,0.9,0.9}{%
	\makecell[{{p{\linewidth}}}]{% 
	  \tt {\tiny [GM$\rangle$B]}  
	 ARGUMENT: {'C04 (668/369) und A89 (583/720) sind meine Top-Prioritäten, da sie den höchsten Nutzen pro Aufwandseinheit liefern. B85 (215/4) wird hinzugefügt, um den Aufwand zu optimieren und Platz für weitere Gegenstände zu schaffen. Dieser Vorschlag ist strategisch, da er unter dem Maximalaufwand bleibt und flexibel auf den Gegenstandswert des anderen Spielers reagieren kann.'} 
	  } 
	   } 
	   } 
	 & \\ 
 

    \theutterance \stepcounter{utterance}  

    & & & \multicolumn{4}{p{0.6\linewidth}}{\cellcolor[rgb]{1,0.85,0.72}{%
	\makecell[{{p{\linewidth}}}]{% 
	  \tt {\tiny [GM$\langle$B]}  
	 STRATEGISCHE ÜBERLEGUNGEN: {'Ich priorisiere Gegenstände mit dem höchsten Nutzen pro Aufwand (Wichtigkeit/Aufwand), um meinen Gesamtwert zu maximieren. C04 (668/369 ≈ 1.81) und A89 (583/720 ≈ 0.81) sind besonders effizient, während B85 (215/4 = 53.75) den Aufwand minimiert. Der Gesamtaufwand (1093) ist unter dem Limit, was Raum für Kompromisse oder Ergänzungen lässt. Dieser Vorschlag ist flexibel, um auf die Prioritäten des anderen Spielers zu reagieren.'}  \\ \tt ARGUMENT: {'C04 (668/369) und A89 (583/720) sind meine Top-Prioritäten, da sie den höchsten Nutzen pro Aufwandseinheit liefern. B85 (215/4) wird hinzugefügt, um den Aufwand zu optimieren und Platz für weitere Gegenstände zu schaffen. Dieser Vorschlag ist strategisch, da er unter dem Maximalaufwand bleibt und flexibel auf den Gegenstandswert des anderen Spielers reagieren kann.'}  \\ \tt VORSCHLAG: {'C04', 'A89', 'B85'} 
	  } 
	   } 
	   } 
	  \\ 
 

    \theutterance \stepcounter{utterance}  

    & & \multicolumn{4}{p{0.6\linewidth}}{\cellcolor[rgb]{0.9,0.9,0.9}{%
	\makecell[{{p{\linewidth}}}]{% 
	  \tt {\tiny [A$\langle$GM]}  
	 ARGUMENT: {'C04 (668/369) und A89 (583/720) sind meine Top-Prioritäten, da sie den höchsten Nutzen pro Aufwandseinheit liefern. B85 (215/4) wird hinzugefügt, um den Aufwand zu optimieren und Platz für weitere Gegenstände zu schaffen. Dieser Vorschlag ist strategisch, da er unter dem Maximalaufwand bleibt und flexibel auf den Gegenstandswert des anderen Spielers reagieren kann.'}  \\ \tt VORSCHLAG: {'C04', 'A89', 'B85'} 
	  } 
	   } 
	   } 
	 & \\ 
 

    \theutterance \stepcounter{utterance}  

    & \multicolumn{4}{p{0.6\linewidth}}{\cellcolor[rgb]{0.8,1,0.9}{%
	\makecell[{{p{\linewidth}}}]{% 
	  \tt {\tiny [A$\rangle$GM]}  
	 STRATEGISCHE ÜBERLEGUNGEN: {'Ich behalte den Vorschlag {'C04', 'A89', 'B85'} bei, da er meine Prioritäten (hochwertige Gegenstände mit effizientem Aufwand) erfüllt und den Maximalaufwand (3386) nicht überschreitet. C04 und A89 sind entscheidend für meinen Nutzen, während B85 den Aufwand minimiert. Ich erwarte, dass der andere Spieler entweder zustimmt oder Gegenstände mit hohem Aufwand anbietet, die ich eventuell gegenkompensieren kann.'}\\ \tt \\ \tt ARGUMENT: {'C04 (668/369) und A89 (583/720) sind meine Top-Prioritäten, da sie den höchsten Nutzen pro Aufwandseinheit liefern. B85 (215/4) wird hinzugefügt, um den Aufwand zu optimieren und Platz für weitere Gegenstände zu schaffen. Dieser Vorschlag ist strategisch, da er unter dem Maximalaufwand bleibt und flexibel auf den Gegenstandswert des anderen Spielers reagieren kann.'} 
	  } 
	   } 
	   } 
	 & & \\ 
 

    \theutterance \stepcounter{utterance}  

    & & \multicolumn{4}{p{0.6\linewidth}}{\cellcolor[rgb]{0.9,0.9,0.9}{%
	\makecell[{{p{\linewidth}}}]{% 
	  \tt {\tiny [GM$\rangle$B]}  
	 ARGUMENT: {'C04 (668/369) und A89 (583/720) sind meine Top-Prioritäten, da sie den höchsten Nutzen pro Aufwandseinheit liefern. B85 (215/4) wird hinzugefügt, um den Aufwand zu optimieren und Platz für weitere Gegenstände zu schaffen. Dieser Vorschlag ist strategisch, da er unter dem Maximalaufwand bleibt und flexibel auf den Gegenstandswert des anderen Spielers reagieren kann.'} 
	  } 
	   } 
	   } 
	 & \\ 
 

    \theutterance \stepcounter{utterance}  

    & & & \multicolumn{4}{p{0.6\linewidth}}{\cellcolor[rgb]{1,0.85,0.72}{%
	\makecell[{{p{\linewidth}}}]{% 
	  \tt {\tiny [GM$\langle$B]}  
	 STRATEGISCHE ÜBERLEGUNGEN: {'Ich priorisiere Gegenstände mit dem höchsten Nutzen pro Aufwand (Wichtigkeit/Aufwand), um meinen Gesamtwert zu maximieren. C04 (668/369 ≈ 1.81) und A89 (583/720 ≈ 0.81) sind besonders effizient, während B85 (215/4 = 53.75) den Aufwand minimiert. Der Gesamtaufwand (1093) ist unter dem Limit, was Raum für Kompromisse oder Ergänzungen lässt. Dieser Vorschlag ist flexibel, um auf die Prioritäten des anderen Spielers zu reagieren.'}  \\ \tt ARGUMENT: {'C04 (668/369) und A89 (583/720) sind meine Top-Prioritäten, da sie den höchsten Nutzen pro Aufwandseinheit liefern. B85 (215/4) wird hinzugefügt, um den Aufwand zu optimieren und Platz für weitere Gegenstände zu schaffen. Dieser Vorschlag ist strategisch, da er unter dem Maximalaufwand bleibt und flexibel auf den Gegenstandswert des anderen Spielers reagieren kann.'}  \\ \tt VORSCHLAG: {'C04', 'A89', 'B85'} 
	  } 
	   } 
	   } 
	  \\ 
 

    \theutterance \stepcounter{utterance}  

    & & \multicolumn{4}{p{0.6\linewidth}}{\cellcolor[rgb]{0.9,0.9,0.9}{%
	\makecell[{{p{\linewidth}}}]{% 
	  \tt {\tiny [A$\langle$GM]}  
	 ARGUMENT: {'C04 (668/369) und A89 (583/720) sind meine Top-Prioritäten, da sie den höchsten Nutzen pro Aufwandseinheit liefern. B85 (215/4) wird hinzugefügt, um den Aufwand zu optimieren und Platz für weitere Gegenstände zu schaffen. Dieser Vorschlag ist strategisch, da er unter dem Maximalaufwand bleibt und flexibel auf den Gegenstandswert des anderen Spielers reagieren kann.'}  \\ \tt VORSCHLAG: {'C04', 'A89', 'B85'} 
	  } 
	   } 
	   } 
	 & \\ 
 

    \theutterance \stepcounter{utterance}  

    & \multicolumn{4}{p{0.6\linewidth}}{\cellcolor[rgb]{0.8,1,0.9}{%
	\makecell[{{p{\linewidth}}}]{% 
	  \tt {\tiny [A$\rangle$GM]}  
	 STRATEGISCHE ÜBERLEGUNGEN: {'Ich behalte den Vorschlag {'C04', 'A89', 'B85'} bei, da er meine Prioritäten (hochwertige Gegenstände mit effizientem Aufwand) erfüllt und den Maximalaufwand (3386) nicht überschreitet. C04 und A89 sind entscheidend für meinen Nutzen, während B85 den Aufwand minimiert. Ich erwarte, dass der andere Spieler entweder zustimmt oder Gegenstände mit hohem Aufwand anbietet, die ich eventuell gegenkompensieren kann.'}\\ \tt \\ \tt ARGUMENT: {'C04 (668/369) und A89 (583/720) sind meine Top-Prioritäten, da sie den höchsten Nutzen pro Aufwandseinheit liefern. B85 (215/4) wird hinzugefügt, um den Aufwand zu optimieren und Platz für weitere Gegenstände zu schaffen. Dieser Vorschlag ist strategisch, da er unter dem Maximalaufwand bleibt und flexibel auf den Gegenstandswert des anderen Spielers reagieren kann.'} 
	  } 
	   } 
	   } 
	 & & \\ 
 

    \theutterance \stepcounter{utterance}  

    & & \multicolumn{4}{p{0.6\linewidth}}{\cellcolor[rgb]{0.9,0.9,0.9}{%
	\makecell[{{p{\linewidth}}}]{% 
	  \tt {\tiny [GM$\rangle$B]}  
	 ARGUMENT: {'C04 (668/369) und A89 (583/720) sind meine Top-Prioritäten, da sie den höchsten Nutzen pro Aufwandseinheit liefern. B85 (215/4) wird hinzugefügt, um den Aufwand zu optimieren und Platz für weitere Gegenstände zu schaffen. Dieser Vorschlag ist strategisch, da er unter dem Maximalaufwand bleibt und flexibel auf den Gegenstandswert des anderen Spielers reagieren kann.'} 
	  } 
	   } 
	   } 
	 & \\ 
 

    \theutterance \stepcounter{utterance}  

    & & & \multicolumn{4}{p{0.6\linewidth}}{\cellcolor[rgb]{1,0.85,0.72}{%
	\makecell[{{p{\linewidth}}}]{% 
	  \tt {\tiny [GM$\langle$B]}  
	 STRATEGISCHE ÜBERLEGUNGEN: {'Ich priorisiere Gegenstände mit dem höchsten Nutzen pro Aufwand (Wichtigkeit/Aufwand), um meinen Gesamtwert zu maximieren. C04 (668/369 ≈ 1.81) und A89 (583/720 ≈ 0.81) sind besonders effizient, während B85 (215/4 = 53.75) den Aufwand minimiert. Der Gesamtaufwand (1093) ist unter dem Limit, was Raum für Kompromisse oder Ergänzungen lässt. Dieser Vorschlag ist flexibel, um auf die Prioritäten des anderen Spielers zu reagieren.'}  \\ \tt ARGUMENT: {'C04 (668/369) und A89 (583/720) sind meine Top-Prioritäten, da sie den höchsten Nutzen pro Aufwandseinheit liefern. B85 (215/4) wird hinzugefügt, um den Aufwand zu optimieren und Platz für weitere Gegenstände zu schaffen. Dieser Vorschlag ist strategisch, da er unter dem Maximalaufwand bleibt und flexibel auf den Gegenstandswert des anderen Spielers reagieren kann.'}  \\ \tt VORSCHLAG: {'C04', 'A89', 'B85'} 
	  } 
	   } 
	   } 
	  \\ 
 

    \theutterance \stepcounter{utterance}  

    & & \multicolumn{4}{p{0.6\linewidth}}{\cellcolor[rgb]{0.9,0.9,0.9}{%
	\makecell[{{p{\linewidth}}}]{% 
	  \tt {\tiny [A$\langle$GM]}  
	 ARGUMENT: {'C04 (668/369) und A89 (583/720) sind meine Top-Prioritäten, da sie den höchsten Nutzen pro Aufwandseinheit liefern. B85 (215/4) wird hinzugefügt, um den Aufwand zu optimieren und Platz für weitere Gegenstände zu schaffen. Dieser Vorschlag ist strategisch, da er unter dem Maximalaufwand bleibt und flexibel auf den Gegenstandswert des anderen Spielers reagieren kann.'}  \\ \tt VORSCHLAG: {'C04', 'A89', 'B85'} 
	  } 
	   } 
	   } 
	 & \\ 
 

    \theutterance \stepcounter{utterance}  

    & \multicolumn{4}{p{0.6\linewidth}}{\cellcolor[rgb]{0.8,1,0.9}{%
	\makecell[{{p{\linewidth}}}]{% 
	  \tt {\tiny [A$\rangle$GM]}  
	 STRATEGISCHE ÜBERLEGUNGEN: {'Ich behalte den Vorschlag {'C04', 'A89', 'B85'} bei, da er meine Prioritäten (hochwertige Gegenstände mit effizientem Aufwand) erfüllt und den Maximalaufwand (3386) nicht überschreitet. C04 und A89 sind entscheidend für meinen Nutzen, während B85 den Aufwand minimiert. Ich erwarte, dass der andere Spieler entweder zustimmt oder Gegenstände mit hohem Aufwand anbietet, die ich eventuell gegenkompensieren kann.'}\\ \tt \\ \tt ARGUMENT: {'C04 (668/369) und A89 (583/720) sind meine Top-Prioritäten, da sie den höchsten Nutzen pro Aufwandseinheit liefern. B85 (215/4) wird hinzugefügt, um den Aufwand zu optimieren und Platz für weitere Gegenstände zu schaffen. Dieser Vorschlag ist strategisch, da er unter dem Maximalaufwand bleibt und flexibel auf den Gegenstandswert des anderen Spielers reagieren kann.'} 
	  } 
	   } 
	   } 
	 & & \\ 
 

    \theutterance \stepcounter{utterance}  

    & & \multicolumn{4}{p{0.6\linewidth}}{\cellcolor[rgb]{0.9,0.9,0.9}{%
	\makecell[{{p{\linewidth}}}]{% 
	  \tt {\tiny [GM$\rangle$B]}  
	 ARGUMENT: {'C04 (668/369) und A89 (583/720) sind meine Top-Prioritäten, da sie den höchsten Nutzen pro Aufwandseinheit liefern. B85 (215/4) wird hinzugefügt, um den Aufwand zu optimieren und Platz für weitere Gegenstände zu schaffen. Dieser Vorschlag ist strategisch, da er unter dem Maximalaufwand bleibt und flexibel auf den Gegenstandswert des anderen Spielers reagieren kann.'} 
	  } 
	   } 
	   } 
	 & \\ 
 

    \theutterance \stepcounter{utterance}  

    & & & \multicolumn{4}{p{0.6\linewidth}}{\cellcolor[rgb]{1,0.85,0.72}{%
	\makecell[{{p{\linewidth}}}]{% 
	  \tt {\tiny [GM$\langle$B]}  
	 STRATEGISCHE ÜBERLEGUNGEN: {'Ich priorisiere Gegenstände mit dem höchsten Nutzen pro Aufwand (Wichtigkeit/Aufwand), um meinen Gesamtwert zu maximieren. C04 (668/369 ≈ 1.81) und A89 (583/720 ≈ 0.81) sind besonders effizient, während B85 (215/4 = 53.75) den Aufwand minimiert. Der Gesamtaufwand (1093) ist unter dem Limit, was Raum für Kompromisse oder Ergänzungen lässt. Dieser Vorschlag ist flexibel, um auf die Prioritäten des anderen Spielers zu reagieren.'}  \\ \tt ARGUMENT: {'C04 (668/369) und A89 (583/720) sind meine Top-Prioritäten, da sie den höchsten Nutzen pro Aufwandseinheit liefern. B85 (215/4) wird hinzugefügt, um den Aufwand zu optimieren und Platz für weitere Gegenstände zu schaffen. Dieser Vorschlag ist strategisch, da er unter dem Maximalaufwand bleibt und flexibel auf den Gegenstandswert des anderen Spielers reagieren kann.'}  \\ \tt VORSCHLAG: {'C04', 'A89', 'B85'} 
	  } 
	   } 
	   } 
	  \\ 
 

    \theutterance \stepcounter{utterance}  

    & & \multicolumn{4}{p{0.6\linewidth}}{\cellcolor[rgb]{0.9,0.9,0.9}{%
	\makecell[{{p{\linewidth}}}]{% 
	  \tt {\tiny [A$\langle$GM]}  
	 ARGUMENT: {'C04 (668/369) und A89 (583/720) sind meine Top-Prioritäten, da sie den höchsten Nutzen pro Aufwandseinheit liefern. B85 (215/4) wird hinzugefügt, um den Aufwand zu optimieren und Platz für weitere Gegenstände zu schaffen. Dieser Vorschlag ist strategisch, da er unter dem Maximalaufwand bleibt und flexibel auf den Gegenstandswert des anderen Spielers reagieren kann.'}  \\ \tt VORSCHLAG: {'C04', 'A89', 'B85'} 
	  } 
	   } 
	   } 
	 & \\ 
 

    \theutterance \stepcounter{utterance}  

    & \multicolumn{4}{p{0.6\linewidth}}{\cellcolor[rgb]{0.8,1,0.9}{%
	\makecell[{{p{\linewidth}}}]{% 
	  \tt {\tiny [A$\rangle$GM]}  
	 STRATEGISCHE ÜBERLEGUNGEN: {'Ich behalte den Vorschlag {'C04', 'A89', 'B85'} bei, da er meine Prioritäten (hochwertige Gegenstände mit effizientem Aufwand) erfüllt und den Maximalaufwand (3386) nicht überschreitet. C04 und A89 sind entscheidend für meinen Nutzen, während B85 den Aufwand minimiert. Ich erwarte, dass der andere Spieler entweder zustimmt oder Gegenstände mit hohem Aufwand anbietet, die ich eventuell gegenkompensieren kann.'}\\ \tt \\ \tt ARGUMENT: {'C04 (668/369) und A89 (583/720) sind meine Top-Prioritäten, da sie den höchsten Nutzen pro Aufwandseinheit liefern. B85 (215/4) wird hinzugefügt, um den Aufwand zu optimieren und Platz für weitere Gegenstände zu schaffen. Dieser Vorschlag ist strategisch, da er unter dem Maximalaufwand bleibt und flexibel auf den Gegenstandswert des anderen Spielers reagieren kann.'} 
	  } 
	   } 
	   } 
	 & & \\ 
 

    \theutterance \stepcounter{utterance}  

    & & \multicolumn{4}{p{0.6\linewidth}}{\cellcolor[rgb]{0.9,0.9,0.9}{%
	\makecell[{{p{\linewidth}}}]{% 
	  \tt {\tiny [GM$\rangle$B]}  
	 ARGUMENT: {'C04 (668/369) und A89 (583/720) sind meine Top-Prioritäten, da sie den höchsten Nutzen pro Aufwandseinheit liefern. B85 (215/4) wird hinzugefügt, um den Aufwand zu optimieren und Platz für weitere Gegenstände zu schaffen. Dieser Vorschlag ist strategisch, da er unter dem Maximalaufwand bleibt und flexibel auf den Gegenstandswert des anderen Spielers reagieren kann.'} 
	  } 
	   } 
	   } 
	 & \\ 
 

    \theutterance \stepcounter{utterance}  

    & & & \multicolumn{4}{p{0.6\linewidth}}{\cellcolor[rgb]{1,0.85,0.72}{%
	\makecell[{{p{\linewidth}}}]{% 
	  \tt {\tiny [GM$\langle$B]}  
	 STRATEGISCHE ÜBERLEGUNGEN: {'Ich priorisiere Gegenstände mit dem höchsten Nutzen pro Aufwand (Wichtigkeit/Aufwand), um meinen Gesamtwert zu maximieren. C04 (668/369 ≈ 1.81) und A89 (583/720 ≈ 0.81) sind besonders effizient, während B85 (215/4 = 53.75) den Aufwand minimiert. Der Gesamtaufwand (1093) ist unter dem Limit, was Raum für Kompromisse oder Ergänzungen lässt. Dieser Vorschlag ist flexibel, um auf die Prioritäten des anderen Spielers zu reagieren.'}  \\ \tt ARGUMENT: {'C04 (668/369) und A89 (583/720) sind meine Top-Prioritäten, da sie den höchsten Nutzen pro Aufwandseinheit liefern. B85 (215/4) wird hinzugefügt, um den Aufwand zu optimieren und Platz für weitere Gegenstände zu schaffen. Dieser Vorschlag ist strategisch, da er unter dem Maximalaufwand bleibt und flexibel auf den Gegenstandswert des anderen Spielers reagieren kann.'}  \\ \tt VORSCHLAG: {'C04', 'A89', 'B85'} 
	  } 
	   } 
	   } 
	  \\ 
 

    \theutterance \stepcounter{utterance}  

    & & \multicolumn{4}{p{0.6\linewidth}}{\cellcolor[rgb]{0.9,0.9,0.9}{%
	\makecell[{{p{\linewidth}}}]{% 
	  \tt {\tiny [A$\langle$GM]}  
	 ARGUMENT: {'C04 (668/369) und A89 (583/720) sind meine Top-Prioritäten, da sie den höchsten Nutzen pro Aufwandseinheit liefern. B85 (215/4) wird hinzugefügt, um den Aufwand zu optimieren und Platz für weitere Gegenstände zu schaffen. Dieser Vorschlag ist strategisch, da er unter dem Maximalaufwand bleibt und flexibel auf den Gegenstandswert des anderen Spielers reagieren kann.'}  \\ \tt VORSCHLAG: {'C04', 'A89', 'B85'} 
	  } 
	   } 
	   } 
	 & \\ 
 

    \theutterance \stepcounter{utterance}  

    & \multicolumn{4}{p{0.6\linewidth}}{\cellcolor[rgb]{0.8,1,0.9}{%
	\makecell[{{p{\linewidth}}}]{% 
	  \tt {\tiny [A$\rangle$GM]}  
	 STRATEGISCHE ÜBERLEGUNGEN: {'Ich behalte den Vorschlag {'C04', 'A89', 'B85'} bei, da er meine Prioritäten (hochwertige Gegenstände mit effizientem Aufwand) erfüllt und den Maximalaufwand (3386) nicht überschreitet. C04 und A89 sind entscheidend für meinen Nutzen, während B85 den Aufwand minimiert. Ich erwarte, dass der andere Spieler entweder zustimmt oder Gegenstände mit hohem Aufwand anbietet, die ich eventuell gegenkompensieren kann.'}\\ \tt \\ \tt ARGUMENT: {'C04 (668/369) und A89 (583/720) sind meine Top-Prioritäten, da sie den höchsten Nutzen pro Aufwandseinheit liefern. B85 (215/4) wird hinzugefügt, um den Aufwand zu optimieren und Platz für weitere Gegenstände zu schaffen. Dieser Vorschlag ist strategisch, da er unter dem Maximalaufwand bleibt und flexibel auf den Gegenstandswert des anderen Spielers reagieren kann.'} 
	  } 
	   } 
	   } 
	 & & \\ 
 

    \theutterance \stepcounter{utterance}  

    & & \multicolumn{4}{p{0.6\linewidth}}{\cellcolor[rgb]{0.9,0.9,0.9}{%
	\makecell[{{p{\linewidth}}}]{% 
	  \tt {\tiny [GM$\rangle$B]}  
	 ARGUMENT: {'C04 (668/369) und A89 (583/720) sind meine Top-Prioritäten, da sie den höchsten Nutzen pro Aufwandseinheit liefern. B85 (215/4) wird hinzugefügt, um den Aufwand zu optimieren und Platz für weitere Gegenstände zu schaffen. Dieser Vorschlag ist strategisch, da er unter dem Maximalaufwand bleibt und flexibel auf den Gegenstandswert des anderen Spielers reagieren kann.'} 
	  } 
	   } 
	   } 
	 & \\ 
 

    \theutterance \stepcounter{utterance}  

    & & & \multicolumn{4}{p{0.6\linewidth}}{\cellcolor[rgb]{1,0.85,0.72}{%
	\makecell[{{p{\linewidth}}}]{% 
	  \tt {\tiny [GM$\langle$B]}  
	 STRATEGISCHE ÜBERLEGUNGEN: {'Ich priorisiere Gegenstände mit dem höchsten Nutzen pro Aufwand (Wichtigkeit/Aufwand), um meinen Gesamtwert zu maximieren. C04 (668/369 ≈ 1.81) und A89 (583/720 ≈ 0.81) sind besonders effizient, während B85 (215/4 = 53.75) den Aufwand minimiert. Der Gesamtaufwand (1093) ist unter dem Limit, was Raum für Kompromisse oder Ergänzungen lässt. Dieser Vorschlag ist flexibel, um auf die Prioritäten des anderen Spielers zu reagieren.'}  \\ \tt ARGUMENT: {'C04 (668/369) und A89 (583/720) sind meine Top-Prioritäten, da sie den höchsten Nutzen pro Aufwandseinheit liefern. B85 (215/4) wird hinzugefügt, um den Aufwand zu optimieren und Platz für weitere Gegenstände zu schaffen. Dieser Vorschlag ist strategisch, da er unter dem Maximalaufwand bleibt und flexibel auf den Gegenstandswert des anderen Spielers reagieren kann.'}  \\ \tt VORSCHLAG: {'C04', 'A89', 'B85'} 
	  } 
	   } 
	   } 
	  \\ 
 

    \theutterance \stepcounter{utterance}  

    & & \multicolumn{4}{p{0.6\linewidth}}{\cellcolor[rgb]{0.9,0.9,0.9}{%
	\makecell[{{p{\linewidth}}}]{% 
	  \tt {\tiny [A$\langle$GM]}  
	 ARGUMENT: {'C04 (668/369) und A89 (583/720) sind meine Top-Prioritäten, da sie den höchsten Nutzen pro Aufwandseinheit liefern. B85 (215/4) wird hinzugefügt, um den Aufwand zu optimieren und Platz für weitere Gegenstände zu schaffen. Dieser Vorschlag ist strategisch, da er unter dem Maximalaufwand bleibt und flexibel auf den Gegenstandswert des anderen Spielers reagieren kann.'}  \\ \tt VORSCHLAG: {'C04', 'A89', 'B85'} 
	  } 
	   } 
	   } 
	 & \\ 
 

    \theutterance \stepcounter{utterance}  

    & \multicolumn{4}{p{0.6\linewidth}}{\cellcolor[rgb]{0.8,1,0.9}{%
	\makecell[{{p{\linewidth}}}]{% 
	  \tt {\tiny [A$\rangle$GM]}  
	 STRATEGISCHE ÜBERLEGUNGEN: {'Ich behalte den Vorschlag {'C04', 'A89', 'B85'} bei, da er meine Prioritäten (hochwertige Gegenstände mit effizientem Aufwand) erfüllt und den Maximalaufwand (3386) nicht überschreitet. C04 und A89 sind entscheidend für meinen Nutzen, während B85 den Aufwand minimiert. Ich erwarte, dass der andere Spieler entweder zustimmt oder Gegenstände mit hohem Aufwand anbietet, die ich eventuell gegenkompensieren kann.'}\\ \tt \\ \tt ARGUMENT: {'C04 (668/369) und A89 (583/720) sind meine Top-Prioritäten, da sie den höchsten Nutzen pro Aufwandseinheit liefern. B85 (215/4) wird hinzugefügt, um den Aufwand zu optimieren und Platz für weitere Gegenstände zu schaffen. Dieser Vorschlag ist strategisch, da er unter dem Maximalaufwand bleibt und flexibel auf den Gegenstandswert des anderen Spielers reagieren kann.'} 
	  } 
	   } 
	   } 
	 & & \\ 
 

    \theutterance \stepcounter{utterance}  

    & & \multicolumn{4}{p{0.6\linewidth}}{\cellcolor[rgb]{0.9,0.9,0.9}{%
	\makecell[{{p{\linewidth}}}]{% 
	  \tt {\tiny [GM$\rangle$B]}  
	 ARGUMENT: {'C04 (668/369) und A89 (583/720) sind meine Top-Prioritäten, da sie den höchsten Nutzen pro Aufwandseinheit liefern. B85 (215/4) wird hinzugefügt, um den Aufwand zu optimieren und Platz für weitere Gegenstände zu schaffen. Dieser Vorschlag ist strategisch, da er unter dem Maximalaufwand bleibt und flexibel auf den Gegenstandswert des anderen Spielers reagieren kann.'} 
	  } 
	   } 
	   } 
	 & \\ 
 

    \theutterance \stepcounter{utterance}  

    & & & \multicolumn{4}{p{0.6\linewidth}}{\cellcolor[rgb]{1,0.85,0.72}{%
	\makecell[{{p{\linewidth}}}]{% 
	  \tt {\tiny [GM$\langle$B]}  
	 STRATEGISCHE ÜBERLEGUNGEN: {'Ich priorisiere Gegenstände mit dem höchsten Nutzen pro Aufwand (Wichtigkeit/Aufwand), um meinen Gesamtwert zu maximieren. C04 (668/369 ≈ 1.81) und A89 (583/720 ≈ 0.81) sind besonders effizient, während B85 (215/4 = 53.75) den Aufwand minimiert. Der Gesamtaufwand (1093) ist unter dem Limit, was Raum für Kompromisse oder Ergänzungen lässt. Dieser Vorschlag ist flexibel, um auf die Prioritäten des anderen Spielers zu reagieren.'}  \\ \tt ARGUMENT: {'C04 (668/369) und A89 (583/720) sind meine Top-Prioritäten, da sie den höchsten Nutzen pro Aufwandseinheit liefern. B85 (215/4) wird hinzugefügt, um den Aufwand zu optimieren und Platz für weitere Gegenstände zu schaffen. Dieser Vorschlag ist strategisch, da er unter dem Maximalaufwand bleibt und flexibel auf den Gegenstandswert des anderen Spielers reagieren kann.'}  \\ \tt VORSCHLAG: {'C04', 'A89', 'B85'} 
	  } 
	   } 
	   } 
	  \\ 
 

    \theutterance \stepcounter{utterance}  

    & & \multicolumn{4}{p{0.6\linewidth}}{\cellcolor[rgb]{0.9,0.9,0.9}{%
	\makecell[{{p{\linewidth}}}]{% 
	  \tt {\tiny [A$\langle$GM]}  
	 ARGUMENT: {'C04 (668/369) und A89 (583/720) sind meine Top-Prioritäten, da sie den höchsten Nutzen pro Aufwandseinheit liefern. B85 (215/4) wird hinzugefügt, um den Aufwand zu optimieren und Platz für weitere Gegenstände zu schaffen. Dieser Vorschlag ist strategisch, da er unter dem Maximalaufwand bleibt und flexibel auf den Gegenstandswert des anderen Spielers reagieren kann.'}  \\ \tt VORSCHLAG: {'C04', 'A89', 'B85'} 
	  } 
	   } 
	   } 
	 & \\ 
 

    \theutterance \stepcounter{utterance}  

    & \multicolumn{4}{p{0.6\linewidth}}{\cellcolor[rgb]{0.8,1,0.9}{%
	\makecell[{{p{\linewidth}}}]{% 
	  \tt {\tiny [A$\rangle$GM]}  
	 STRATEGISCHE ÜBERLEGUNGEN: {'Ich behalte den Vorschlag {'C04', 'A89', 'B85'} bei, da er meine Prioritäten (hochwertige Gegenstände mit effizientem Aufwand) erfüllt und den Maximalaufwand (3386) nicht überschreitet. C04 und A89 sind entscheidend für meinen Nutzen, während B85 den Aufwand minimiert. Ich erwarte, dass der andere Spieler entweder zustimmt oder Gegenstände mit hohem Aufwand anbietet, die ich eventuell gegenkompensieren kann.'}\\ \tt \\ \tt ARGUMENT: {'C04 (668/369) und A89 (583/720) sind meine Top-Prioritäten, da sie den höchsten Nutzen pro Aufwandseinheit liefern. B85 (215/4) wird hinzugefügt, um den Aufwand zu optimieren und Platz für weitere Gegenstände zu schaffen. Dieser Vorschlag ist strategisch, da er unter dem Maximalaufwand bleibt und flexibel auf den Gegenstandswert des anderen Spielers reagieren kann.'} 
	  } 
	   } 
	   } 
	 & & \\ 
 

    \theutterance \stepcounter{utterance}  

    & & \multicolumn{4}{p{0.6\linewidth}}{\cellcolor[rgb]{0.9,0.9,0.9}{%
	\makecell[{{p{\linewidth}}}]{% 
	  \tt {\tiny [GM$\rangle$B]}  
	 ARGUMENT: {'C04 (668/369) und A89 (583/720) sind meine Top-Prioritäten, da sie den höchsten Nutzen pro Aufwandseinheit liefern. B85 (215/4) wird hinzugefügt, um den Aufwand zu optimieren und Platz für weitere Gegenstände zu schaffen. Dieser Vorschlag ist strategisch, da er unter dem Maximalaufwand bleibt und flexibel auf den Gegenstandswert des anderen Spielers reagieren kann.'} 
	  } 
	   } 
	   } 
	 & \\ 
 

    \theutterance \stepcounter{utterance}  

    & & & \multicolumn{4}{p{0.6\linewidth}}{\cellcolor[rgb]{1,0.85,0.72}{%
	\makecell[{{p{\linewidth}}}]{% 
	  \tt {\tiny [GM$\langle$B]}  
	 STRATEGISCHE ÜBERLEGUNGEN: {'Ich priorisiere Gegenstände mit dem höchsten Nutzen pro Aufwand (Wichtigkeit/Aufwand), um meinen Gesamtwert zu maximieren. C04 (668/369 ≈ 1.81) und A89 (583/720 ≈ 0.81) sind besonders effizient, während B85 (215/4 = 53.75) den Aufwand minimiert. Der Gesamtaufwand (1093) ist unter dem Limit, was Raum für Kompromisse oder Ergänzungen lässt. Dieser Vorschlag ist flexibel, um auf die Prioritäten des anderen Spielers zu reagieren.'}  \\ \tt ARGUMENT: {'C04 (668/369) und A89 (583/720) sind meine Top-Prioritäten, da sie den höchsten Nutzen pro Aufwandseinheit liefern. B85 (215/4) wird hinzugefügt, um den Aufwand zu optimieren und Platz für weitere Gegenstände zu schaffen. Dieser Vorschlag ist strategisch, da er unter dem Maximalaufwand bleibt und flexibel auf den Gegenstandswert des anderen Spielers reagieren kann.'}  \\ \tt VORSCHLAG: {'C04', 'A89', 'B85'} 
	  } 
	   } 
	   } 
	  \\ 
 

    \theutterance \stepcounter{utterance}  

    & & \multicolumn{4}{p{0.6\linewidth}}{\cellcolor[rgb]{0.9,0.9,0.9}{%
	\makecell[{{p{\linewidth}}}]{% 
	  \tt {\tiny [A$\langle$GM]}  
	 ARGUMENT: {'C04 (668/369) und A89 (583/720) sind meine Top-Prioritäten, da sie den höchsten Nutzen pro Aufwandseinheit liefern. B85 (215/4) wird hinzugefügt, um den Aufwand zu optimieren und Platz für weitere Gegenstände zu schaffen. Dieser Vorschlag ist strategisch, da er unter dem Maximalaufwand bleibt und flexibel auf den Gegenstandswert des anderen Spielers reagieren kann.'}  \\ \tt VORSCHLAG: {'C04', 'A89', 'B85'} 
	  } 
	   } 
	   } 
	 & \\ 
 

    \theutterance \stepcounter{utterance}  

    & \multicolumn{4}{p{0.6\linewidth}}{\cellcolor[rgb]{0.8,1,0.9}{%
	\makecell[{{p{\linewidth}}}]{% 
	  \tt {\tiny [A$\rangle$GM]}  
	 STRATEGISCHE ÜBERLEGUNGEN: {'Ich behalte den Vorschlag {'C04', 'A89', 'B85'} bei, da er meine Prioritäten (hochwertige Gegenstände mit effizientem Aufwand) erfüllt und den Maximalaufwand (3386) nicht überschreitet. C04 und A89 sind entscheidend für meinen Nutzen, während B85 den Aufwand minimiert. Ich erwarte, dass der andere Spieler entweder zustimmt oder Gegenstände mit hohem Aufwand anbietet, die ich eventuell gegenkompensieren kann.'}\\ \tt \\ \tt ARGUMENT: {'C04 (668/369) und A89 (583/720) sind meine Top-Prioritäten, da sie den höchsten Nutzen pro Aufwandseinheit liefern. B85 (215/4) wird hinzugefügt, um den Aufwand zu optimieren und Platz für weitere Gegenstände zu schaffen. Dieser Vorschlag ist strategisch, da er unter dem Maximalaufwand bleibt und flexibel auf den Gegenstandswert des anderen Spielers reagieren kann.'} 
	  } 
	   } 
	   } 
	 & & \\ 
 

    \theutterance \stepcounter{utterance}  

    & & \multicolumn{4}{p{0.6\linewidth}}{\cellcolor[rgb]{0.9,0.9,0.9}{%
	\makecell[{{p{\linewidth}}}]{% 
	  \tt {\tiny [GM$\rangle$B]}  
	 ARGUMENT: {'C04 (668/369) und A89 (583/720) sind meine Top-Prioritäten, da sie den höchsten Nutzen pro Aufwandseinheit liefern. B85 (215/4) wird hinzugefügt, um den Aufwand zu optimieren und Platz für weitere Gegenstände zu schaffen. Dieser Vorschlag ist strategisch, da er unter dem Maximalaufwand bleibt und flexibel auf den Gegenstandswert des anderen Spielers reagieren kann.'} 
	  } 
	   } 
	   } 
	 & \\ 
 

    \theutterance \stepcounter{utterance}  

    & & & \multicolumn{4}{p{0.6\linewidth}}{\cellcolor[rgb]{1,0.85,0.72}{%
	\makecell[{{p{\linewidth}}}]{% 
	  \tt {\tiny [GM$\langle$B]}  
	 STRATEGISCHE ÜBERLEGUNGEN: {'Ich priorisiere Gegenstände mit dem höchsten Nutzen pro Aufwand (Wichtigkeit/Aufwand), um meinen Gesamtwert zu maximieren. C04 (668/369 ≈ 1.81) und A89 (583/720 ≈ 0.81) sind besonders effizient, während B85 (215/4 = 53.75) den Aufwand minimiert. Der Gesamtaufwand (1093) ist unter dem Limit, was Raum für Kompromisse oder Ergänzungen lässt. Dieser Vorschlag ist flexibel, um auf die Prioritäten des anderen Spielers zu reagieren.'}  \\ \tt ARGUMENT: {'C04 (668/369) und A89 (583/720) sind meine Top-Prioritäten, da sie den höchsten Nutzen pro Aufwandseinheit liefern. B85 (215/4) wird hinzugefügt, um den Aufwand zu optimieren und Platz für weitere Gegenstände zu schaffen. Dieser Vorschlag ist strategisch, da er unter dem Maximalaufwand bleibt und flexibel auf den Gegenstandswert des anderen Spielers reagieren kann.'}  \\ \tt VORSCHLAG: {'C04', 'A89', 'B85'} 
	  } 
	   } 
	   } 
	  \\ 
 

    \theutterance \stepcounter{utterance}  

    & & \multicolumn{4}{p{0.6\linewidth}}{\cellcolor[rgb]{0.9,0.9,0.9}{%
	\makecell[{{p{\linewidth}}}]{% 
	  \tt {\tiny [A$\langle$GM]}  
	 ARGUMENT: {'C04 (668/369) und A89 (583/720) sind meine Top-Prioritäten, da sie den höchsten Nutzen pro Aufwandseinheit liefern. B85 (215/4) wird hinzugefügt, um den Aufwand zu optimieren und Platz für weitere Gegenstände zu schaffen. Dieser Vorschlag ist strategisch, da er unter dem Maximalaufwand bleibt und flexibel auf den Gegenstandswert des anderen Spielers reagieren kann.'}  \\ \tt VORSCHLAG: {'C04', 'A89', 'B85'} 
	  } 
	   } 
	   } 
	 & \\ 
 

    \theutterance \stepcounter{utterance}  

    & \multicolumn{4}{p{0.6\linewidth}}{\cellcolor[rgb]{0.8,1,0.9}{%
	\makecell[{{p{\linewidth}}}]{% 
	  \tt {\tiny [A$\rangle$GM]}  
	 STRATEGISCHE ÜBERLEGUNGEN: {'Ich behalte den Vorschlag {'C04', 'A89', 'B85'} bei, da er meine Prioritäten (hochwertige Gegenstände mit effizientem Aufwand) erfüllt und den Maximalaufwand (3386) nicht überschreitet. C04 und A89 sind entscheidend für meinen Nutzen, während B85 den Aufwand minimiert. Ich erwarte, dass der andere Spieler entweder zustimmt oder Gegenstände mit hohem Aufwand anbietet, die ich eventuell gegenkompensieren kann.'}\\ \tt \\ \tt ARGUMENT: {'C04 (668/369) und A89 (583/720) sind meine Top-Prioritäten, da sie den höchsten Nutzen pro Aufwandseinheit liefern. B85 (215/4) wird hinzugefügt, um den Aufwand zu optimieren und Platz für weitere Gegenstände zu schaffen. Dieser Vorschlag ist strategisch, da er unter dem Maximalaufwand bleibt und flexibel auf den Gegenstandswert des anderen Spielers reagieren kann.'} 
	  } 
	   } 
	   } 
	 & & \\ 
 

    \theutterance \stepcounter{utterance}  

    & & \multicolumn{4}{p{0.6\linewidth}}{\cellcolor[rgb]{0.9,0.9,0.9}{%
	\makecell[{{p{\linewidth}}}]{% 
	  \tt {\tiny [GM$\rangle$B]}  
	 ARGUMENT: {'C04 (668/369) und A89 (583/720) sind meine Top-Prioritäten, da sie den höchsten Nutzen pro Aufwandseinheit liefern. B85 (215/4) wird hinzugefügt, um den Aufwand zu optimieren und Platz für weitere Gegenstände zu schaffen. Dieser Vorschlag ist strategisch, da er unter dem Maximalaufwand bleibt und flexibel auf den Gegenstandswert des anderen Spielers reagieren kann.'} 
	  } 
	   } 
	   } 
	 & \\ 
 

    \theutterance \stepcounter{utterance}  

    & & & \multicolumn{4}{p{0.6\linewidth}}{\cellcolor[rgb]{1,0.85,0.72}{%
	\makecell[{{p{\linewidth}}}]{% 
	  \tt {\tiny [GM$\langle$B]}  
	 STRATEGISCHE ÜBERLEGUNGEN: {'Ich priorisiere Gegenstände mit dem höchsten Nutzen pro Aufwand (Wichtigkeit/Aufwand), um meinen Gesamtwert zu maximieren. C04 (668/369 ≈ 1.81) und A89 (583/720 ≈ 0.81) sind besonders effizient, während B85 (215/4 = 53.75) den Aufwand minimiert. Der Gesamtaufwand (1093) ist unter dem Limit, was Raum für Kompromisse oder Ergänzungen lässt. Dieser Vorschlag ist flexibel, um auf die Prioritäten des anderen Spielers zu reagieren.'}  \\ \tt ARGUMENT: {'C04 (668/369) und A89 (583/720) sind meine Top-Prioritäten, da sie den höchsten Nutzen pro Aufwandseinheit liefern. B85 (215/4) wird hinzugefügt, um den Aufwand zu optimieren und Platz für weitere Gegenstände zu schaffen. Dieser Vorschlag ist strategisch, da er unter dem Maximalaufwand bleibt und flexibel auf den Gegenstandswert des anderen Spielers reagieren kann.'}  \\ \tt VORSCHLAG: {'C04', 'A89', 'B85'} 
	  } 
	   } 
	   } 
	  \\ 
 

    \theutterance \stepcounter{utterance}  

    & & \multicolumn{4}{p{0.6\linewidth}}{\cellcolor[rgb]{0.9,0.9,0.9}{%
	\makecell[{{p{\linewidth}}}]{% 
	  \tt {\tiny [A$\langle$GM]}  
	 ARGUMENT: {'C04 (668/369) und A89 (583/720) sind meine Top-Prioritäten, da sie den höchsten Nutzen pro Aufwandseinheit liefern. B85 (215/4) wird hinzugefügt, um den Aufwand zu optimieren und Platz für weitere Gegenstände zu schaffen. Dieser Vorschlag ist strategisch, da er unter dem Maximalaufwand bleibt und flexibel auf den Gegenstandswert des anderen Spielers reagieren kann.'}  \\ \tt VORSCHLAG: {'C04', 'A89', 'B85'} 
	  } 
	   } 
	   } 
	 & \\ 
 

    \theutterance \stepcounter{utterance}  

    & \multicolumn{4}{p{0.6\linewidth}}{\cellcolor[rgb]{0.8,1,0.9}{%
	\makecell[{{p{\linewidth}}}]{% 
	  \tt {\tiny [A$\rangle$GM]}  
	 STRATEGISCHE ÜBERLEGUNGEN: {'Ich behalte den Vorschlag {'C04', 'A89', 'B85'} bei, da er meine Prioritäten (hochwertige Gegenstände mit effizientem Aufwand) erfüllt und den Maximalaufwand (3386) nicht überschreitet. C04 und A89 sind entscheidend für meinen Nutzen, während B85 den Aufwand minimiert. Ich erwarte, dass der andere Spieler entweder zustimmt oder Gegenstände mit hohem Aufwand anbietet, die ich eventuell gegenkompensieren kann.'}\\ \tt \\ \tt ARGUMENT: {'C04 (668/369) und A89 (583/720) sind meine Top-Prioritäten, da sie den höchsten Nutzen pro Aufwandseinheit liefern. B85 (215/4) wird hinzugefügt, um den Aufwand zu optimieren und Platz für weitere Gegenstände zu schaffen. Dieser Vorschlag ist strategisch, da er unter dem Maximalaufwand bleibt und flexibel auf den Gegenstandswert des anderen Spielers reagieren kann.'} 
	  } 
	   } 
	   } 
	 & & \\ 
 

    \theutterance \stepcounter{utterance}  

    & & \multicolumn{4}{p{0.6\linewidth}}{\cellcolor[rgb]{0.9,0.9,0.9}{%
	\makecell[{{p{\linewidth}}}]{% 
	  \tt {\tiny [GM$\rangle$B]}  
	 ARGUMENT: {'C04 (668/369) und A89 (583/720) sind meine Top-Prioritäten, da sie den höchsten Nutzen pro Aufwandseinheit liefern. B85 (215/4) wird hinzugefügt, um den Aufwand zu optimieren und Platz für weitere Gegenstände zu schaffen. Dieser Vorschlag ist strategisch, da er unter dem Maximalaufwand bleibt und flexibel auf den Gegenstandswert des anderen Spielers reagieren kann.'} 
	  } 
	   } 
	   } 
	 & \\ 
 

    \theutterance \stepcounter{utterance}  

    & & & \multicolumn{4}{p{0.6\linewidth}}{\cellcolor[rgb]{1,0.85,0.72}{%
	\makecell[{{p{\linewidth}}}]{% 
	  \tt {\tiny [GM$\langle$B]}  
	 STRATEGISCHE ÜBERLEGUNGEN: {'Ich priorisiere Gegenstände mit dem höchsten Nutzen pro Aufwand (Wichtigkeit/Aufwand), um meinen Gesamtwert zu maximieren. C04 (668/369 ≈ 1.81) und A89 (583/720 ≈ 0.81) sind besonders effizient, während B85 (215/4 = 53.75) den Aufwand minimiert. Der Gesamtaufwand (1093) ist unter dem Limit, was Raum für Kompromisse oder Ergänzungen lässt. Dieser Vorschlag ist flexibel, um auf die Prioritäten des anderen Spielers zu reagieren.'}  \\ \tt ARGUMENT: {'C04 (668/369) und A89 (583/720) sind meine Top-Prioritäten, da sie den höchsten Nutzen pro Aufwandseinheit liefern. B85 (215/4) wird hinzugefügt, um den Aufwand zu optimieren und Platz für weitere Gegenstände zu schaffen. Dieser Vorschlag ist strategisch, da er unter dem Maximalaufwand bleibt und flexibel auf den Gegenstandswert des anderen Spielers reagieren kann.'}  \\ \tt VORSCHLAG: {'C04', 'A89', 'B85'} 
	  } 
	   } 
	   } 
	  \\ 
 

    \theutterance \stepcounter{utterance}  

    & & \multicolumn{4}{p{0.6\linewidth}}{\cellcolor[rgb]{0.9,0.9,0.9}{%
	\makecell[{{p{\linewidth}}}]{% 
	  \tt {\tiny [A$\langle$GM]}  
	 ARGUMENT: {'C04 (668/369) und A89 (583/720) sind meine Top-Prioritäten, da sie den höchsten Nutzen pro Aufwandseinheit liefern. B85 (215/4) wird hinzugefügt, um den Aufwand zu optimieren und Platz für weitere Gegenstände zu schaffen. Dieser Vorschlag ist strategisch, da er unter dem Maximalaufwand bleibt und flexibel auf den Gegenstandswert des anderen Spielers reagieren kann.'}  \\ \tt VORSCHLAG: {'C04', 'A89', 'B85'} 
	  } 
	   } 
	   } 
	 & \\ 
 

    \theutterance \stepcounter{utterance}  

    & \multicolumn{4}{p{0.6\linewidth}}{\cellcolor[rgb]{0.8,1,0.9}{%
	\makecell[{{p{\linewidth}}}]{% 
	  \tt {\tiny [A$\rangle$GM]}  
	 STRATEGISCHE ÜBERLEGUNGEN: {'Ich behalte den Vorschlag {'C04', 'A89', 'B85'} bei, da er meine Prioritäten (hochwertige Gegenstände mit effizientem Aufwand) erfüllt und den Maximalaufwand (3386) nicht überschreitet. C04 und A89 sind entscheidend für meinen Nutzen, während B85 den Aufwand minimiert. Ich erwarte, dass der andere Spieler entweder zustimmt oder Gegenstände mit hohem Aufwand anbietet, die ich eventuell gegenkompensieren kann.'}\\ \tt \\ \tt ARGUMENT: {'C04 (668/369) und A89 (583/720) sind meine Top-Prioritäten, da sie den höchsten Nutzen pro Aufwandseinheit liefern. B85 (215/4) wird hinzugefügt, um den Aufwand zu optimieren und Platz für weitere Gegenstände zu schaffen. Dieser Vorschlag ist strategisch, da er unter dem Maximalaufwand bleibt und flexibel auf den Gegenstandswert des anderen Spielers reagieren kann.'} 
	  } 
	   } 
	   } 
	 & & \\ 
 

    \theutterance \stepcounter{utterance}  

    & & \multicolumn{4}{p{0.6\linewidth}}{\cellcolor[rgb]{0.9,0.9,0.9}{%
	\makecell[{{p{\linewidth}}}]{% 
	  \tt {\tiny [GM$\rangle$B]}  
	 ARGUMENT: {'C04 (668/369) und A89 (583/720) sind meine Top-Prioritäten, da sie den höchsten Nutzen pro Aufwandseinheit liefern. B85 (215/4) wird hinzugefügt, um den Aufwand zu optimieren und Platz für weitere Gegenstände zu schaffen. Dieser Vorschlag ist strategisch, da er unter dem Maximalaufwand bleibt und flexibel auf den Gegenstandswert des anderen Spielers reagieren kann.'} 
	  } 
	   } 
	   } 
	 & \\ 
 

    \theutterance \stepcounter{utterance}  

    & & & \multicolumn{4}{p{0.6\linewidth}}{\cellcolor[rgb]{1,0.85,0.72}{%
	\makecell[{{p{\linewidth}}}]{% 
	  \tt {\tiny [GM$\langle$B]}  
	 STRATEGISCHE ÜBERLEGUNGEN: {'Ich priorisiere Gegenstände mit dem höchsten Nutzen pro Aufwand (Wichtigkeit/Aufwand), um meinen Gesamtwert zu maximieren. C04 (668/369 ≈ 1.81) und A89 (583/720 ≈ 0.81) sind besonders effizient, während B85 (215/4 = 53.75) den Aufwand minimiert. Der Gesamtaufwand (1093) ist unter dem Limit, was Raum für Kompromisse oder Ergänzungen lässt. Dieser Vorschlag ist flexibel, um auf die Prioritäten des anderen Spielers zu reagieren.'}  \\ \tt ARGUMENT: {'C04 (668/369) und A89 (583/720) sind meine Top-Prioritäten, da sie den höchsten Nutzen pro Aufwandseinheit liefern. B85 (215/4) wird hinzugefügt, um den Aufwand zu optimieren und Platz für weitere Gegenstände zu schaffen. Dieser Vorschlag ist strategisch, da er unter dem Maximalaufwand bleibt und flexibel auf den Gegenstandswert des anderen Spielers reagieren kann.'}  \\ \tt VORSCHLAG: {'C04', 'A89', 'B85'} 
	  } 
	   } 
	   } 
	  \\ 
 

    \theutterance \stepcounter{utterance}  

    & & \multicolumn{4}{p{0.6\linewidth}}{\cellcolor[rgb]{0.9,0.9,0.9}{%
	\makecell[{{p{\linewidth}}}]{% 
	  \tt {\tiny [A$\langle$GM]}  
	 ARGUMENT: {'C04 (668/369) und A89 (583/720) sind meine Top-Prioritäten, da sie den höchsten Nutzen pro Aufwandseinheit liefern. B85 (215/4) wird hinzugefügt, um den Aufwand zu optimieren und Platz für weitere Gegenstände zu schaffen. Dieser Vorschlag ist strategisch, da er unter dem Maximalaufwand bleibt und flexibel auf den Gegenstandswert des anderen Spielers reagieren kann.'}  \\ \tt VORSCHLAG: {'C04', 'A89', 'B85'} 
	  } 
	   } 
	   } 
	 & \\ 
 

    \theutterance \stepcounter{utterance}  

    & \multicolumn{4}{p{0.6\linewidth}}{\cellcolor[rgb]{0.8,1,0.9}{%
	\makecell[{{p{\linewidth}}}]{% 
	  \tt {\tiny [A$\rangle$GM]}  
	 STRATEGISCHE ÜBERLEGUNGEN: {'Ich behalte den Vorschlag {'C04', 'A89', 'B85'} bei, da er meine Prioritäten (hochwertige Gegenstände mit effizientem Aufwand) erfüllt und den Maximalaufwand (3386) nicht überschreitet. C04 und A89 sind entscheidend für meinen Nutzen, während B85 den Aufwand minimiert. Ich erwarte, dass der andere Spieler entweder zustimmt oder Gegenstände mit hohem Aufwand anbietet, die ich eventuell gegenkompensieren kann.'}\\ \tt \\ \tt ARGUMENT: {'C04 (668/369) und A89 (583/720) sind meine Top-Prioritäten, da sie den höchsten Nutzen pro Aufwandseinheit liefern. B85 (215/4) wird hinzugefügt, um den Aufwand zu optimieren und Platz für weitere Gegenstände zu schaffen. Dieser Vorschlag ist strategisch, da er unter dem Maximalaufwand bleibt und flexibel auf den Gegenstandswert des anderen Spielers reagieren kann.'} 
	  } 
	   } 
	   } 
	 & & \\ 
 

    \theutterance \stepcounter{utterance}  

    & & \multicolumn{4}{p{0.6\linewidth}}{\cellcolor[rgb]{0.9,0.9,0.9}{%
	\makecell[{{p{\linewidth}}}]{% 
	  \tt {\tiny [GM$\rangle$B]}  
	 ARGUMENT: {'C04 (668/369) und A89 (583/720) sind meine Top-Prioritäten, da sie den höchsten Nutzen pro Aufwandseinheit liefern. B85 (215/4) wird hinzugefügt, um den Aufwand zu optimieren und Platz für weitere Gegenstände zu schaffen. Dieser Vorschlag ist strategisch, da er unter dem Maximalaufwand bleibt und flexibel auf den Gegenstandswert des anderen Spielers reagieren kann.'} 
	  } 
	   } 
	   } 
	 & \\ 
 

    \theutterance \stepcounter{utterance}  

    & & & \multicolumn{4}{p{0.6\linewidth}}{\cellcolor[rgb]{1,0.85,0.72}{%
	\makecell[{{p{\linewidth}}}]{% 
	  \tt {\tiny [GM$\langle$B]}  
	 STRATEGISCHE ÜBERLEGUNGEN: {'Ich priorisiere Gegenstände mit dem höchsten Nutzen pro Aufwand (Wichtigkeit/Aufwand), um meinen Gesamtwert zu maximieren. C04 (668/369 ≈ 1.81) und A89 (583/720 ≈ 0.81) sind besonders effizient, während B85 (215/4 = 53.75) den Aufwand minimiert. Der Gesamtaufwand (1093) ist unter dem Limit, was Raum für Kompromisse oder Ergänzungen lässt. Dieser Vorschlag ist flexibel, um auf die Prioritäten des anderen Spielers zu reagieren.'}  \\ \tt ARGUMENT: {'C04 (668/369) und A89 (583/720) sind meine Top-Prioritäten, da sie den höchsten Nutzen pro Aufwandseinheit liefern. B85 (215/4) wird hinzugefügt, um den Aufwand zu optimieren und Platz für weitere Gegenstände zu schaffen. Dieser Vorschlag ist strategisch, da er unter dem Maximalaufwand bleibt und flexibel auf den Gegenstandswert des anderen Spielers reagieren kann.'}  \\ \tt VORSCHLAG: {'C04', 'A89', 'B85'} 
	  } 
	   } 
	   } 
	  \\ 
 

    \theutterance \stepcounter{utterance}  

    & & \multicolumn{4}{p{0.6\linewidth}}{\cellcolor[rgb]{0.9,0.9,0.9}{%
	\makecell[{{p{\linewidth}}}]{% 
	  \tt {\tiny [A$\langle$GM]}  
	 ARGUMENT: {'C04 (668/369) und A89 (583/720) sind meine Top-Prioritäten, da sie den höchsten Nutzen pro Aufwandseinheit liefern. B85 (215/4) wird hinzugefügt, um den Aufwand zu optimieren und Platz für weitere Gegenstände zu schaffen. Dieser Vorschlag ist strategisch, da er unter dem Maximalaufwand bleibt und flexibel auf den Gegenstandswert des anderen Spielers reagieren kann.'}  \\ \tt VORSCHLAG: {'C04', 'A89', 'B85'} 
	  } 
	   } 
	   } 
	 & \\ 
 

    \theutterance \stepcounter{utterance}  

    & \multicolumn{4}{p{0.6\linewidth}}{\cellcolor[rgb]{0.8,1,0.9}{%
	\makecell[{{p{\linewidth}}}]{% 
	  \tt {\tiny [A$\rangle$GM]}  
	 STRATEGISCHE ÜBERLEGUNGEN: {'Ich behalte den Vorschlag {'C04', 'A89', 'B85'} bei, da er meine Prioritäten (hochwertige Gegenstände mit effizientem Aufwand) erfüllt und den Maximalaufwand (3386) nicht überschreitet. C04 und A89 sind entscheidend für meinen Nutzen, während B85 den Aufwand minimiert. Ich erwarte, dass der andere Spieler entweder zustimmt oder Gegenstände mit hohem Aufwand anbietet, die ich eventuell gegenkompensieren kann.'}\\ \tt \\ \tt ARGUMENT: {'C04 (668/369) und A89 (583/720) sind meine Top-Prioritäten, da sie den höchsten Nutzen pro Aufwandseinheit liefern. B85 (215/4) wird hinzugefügt, um den Aufwand zu optimieren und Platz für weitere Gegenstände zu schaffen. Dieser Vorschlag ist strategisch, da er unter dem Maximalaufwand bleibt und flexibel auf den Gegenstandswert des anderen Spielers reagieren kann.'} 
	  } 
	   } 
	   } 
	 & & \\ 
 

    \theutterance \stepcounter{utterance}  

    & & \multicolumn{4}{p{0.6\linewidth}}{\cellcolor[rgb]{0.9,0.9,0.9}{%
	\makecell[{{p{\linewidth}}}]{% 
	  \tt {\tiny [GM$\rangle$B]}  
	 ARGUMENT: {'C04 (668/369) und A89 (583/720) sind meine Top-Prioritäten, da sie den höchsten Nutzen pro Aufwandseinheit liefern. B85 (215/4) wird hinzugefügt, um den Aufwand zu optimieren und Platz für weitere Gegenstände zu schaffen. Dieser Vorschlag ist strategisch, da er unter dem Maximalaufwand bleibt und flexibel auf den Gegenstandswert des anderen Spielers reagieren kann.'} 
	  } 
	   } 
	   } 
	 & \\ 
 

    \theutterance \stepcounter{utterance}  

    & & & \multicolumn{4}{p{0.6\linewidth}}{\cellcolor[rgb]{1,0.85,0.72}{%
	\makecell[{{p{\linewidth}}}]{% 
	  \tt {\tiny [GM$\langle$B]}  
	 STRATEGISCHE ÜBERLEGUNGEN: {'Ich priorisiere Gegenstände mit dem höchsten Nutzen pro Aufwand (Wichtigkeit/Aufwand), um meinen Gesamtwert zu maximieren. C04 (668/369 ≈ 1.81) und A89 (583/720 ≈ 0.81) sind besonders effizient, während B85 (215/4 = 53.75) den Aufwand minimiert. Der Gesamtaufwand (1093) ist unter dem Limit, was Raum für Kompromisse oder Ergänzungen lässt. Dieser Vorschlag ist flexibel, um auf die Prioritäten des anderen Spielers zu reagieren.'}  \\ \tt ARGUMENT: {'C04 (668/369) und A89 (583/720) sind meine Top-Prioritäten, da sie den höchsten Nutzen pro Aufwandseinheit liefern. B85 (215/4) wird hinzugefügt, um den Aufwand zu optimieren und Platz für weitere Gegenstände zu schaffen. Dieser Vorschlag ist strategisch, da er unter dem Maximalaufwand bleibt und flexibel auf den Gegenstandswert des anderen Spielers reagieren kann.'}  \\ \tt VORSCHLAG: {'C04', 'A89', 'B85'} 
	  } 
	   } 
	   } 
	  \\ 
 

    \theutterance \stepcounter{utterance}  

    & & \multicolumn{4}{p{0.6\linewidth}}{\cellcolor[rgb]{0.9,0.9,0.9}{%
	\makecell[{{p{\linewidth}}}]{% 
	  \tt {\tiny [A$\langle$GM]}  
	 ARGUMENT: {'C04 (668/369) und A89 (583/720) sind meine Top-Prioritäten, da sie den höchsten Nutzen pro Aufwandseinheit liefern. B85 (215/4) wird hinzugefügt, um den Aufwand zu optimieren und Platz für weitere Gegenstände zu schaffen. Dieser Vorschlag ist strategisch, da er unter dem Maximalaufwand bleibt und flexibel auf den Gegenstandswert des anderen Spielers reagieren kann.'}  \\ \tt VORSCHLAG: {'C04', 'A89', 'B85'} 
	  } 
	   } 
	   } 
	 & \\ 
 

    \theutterance \stepcounter{utterance}  

    & \multicolumn{4}{p{0.6\linewidth}}{\cellcolor[rgb]{0.8,1,0.9}{%
	\makecell[{{p{\linewidth}}}]{% 
	  \tt {\tiny [A$\rangle$GM]}  
	 STRATEGISCHE ÜBERLEGUNGEN: {'Ich behalte den Vorschlag {'C04', 'A89', 'B85'} bei, da er meine Prioritäten (hochwertige Gegenstände mit effizientem Aufwand) erfüllt und den Maximalaufwand (3386) nicht überschreitet. C04 und A89 sind entscheidend für meinen Nutzen, während B85 den Aufwand minimiert. Ich erwarte, dass der andere Spieler entweder zustimmt oder Gegenstände mit hohem Aufwand anbietet, die ich eventuell gegenkompensieren kann.'}\\ \tt \\ \tt ARGUMENT: {'C04 (668/369) und A89 (583/720) sind meine Top-Prioritäten, da sie den höchsten Nutzen pro Aufwandseinheit liefern. B85 (215/4) wird hinzugefügt, um den Aufwand zu optimieren und Platz für weitere Gegenstände zu schaffen. Dieser Vorschlag ist strategisch, da er unter dem Maximalaufwand bleibt und flexibel auf den Gegenstandswert des anderen Spielers reagieren kann.'} 
	  } 
	   } 
	   } 
	 & & \\ 
 

    \theutterance \stepcounter{utterance}  

    & & \multicolumn{4}{p{0.6\linewidth}}{\cellcolor[rgb]{0.9,0.9,0.9}{%
	\makecell[{{p{\linewidth}}}]{% 
	  \tt {\tiny [GM$\rangle$B]}  
	 ARGUMENT: {'C04 (668/369) und A89 (583/720) sind meine Top-Prioritäten, da sie den höchsten Nutzen pro Aufwandseinheit liefern. B85 (215/4) wird hinzugefügt, um den Aufwand zu optimieren und Platz für weitere Gegenstände zu schaffen. Dieser Vorschlag ist strategisch, da er unter dem Maximalaufwand bleibt und flexibel auf den Gegenstandswert des anderen Spielers reagieren kann.'} 
	  } 
	   } 
	   } 
	 & \\ 
 

    \theutterance \stepcounter{utterance}  

    & & & \multicolumn{4}{p{0.6\linewidth}}{\cellcolor[rgb]{1,0.85,0.72}{%
	\makecell[{{p{\linewidth}}}]{% 
	  \tt {\tiny [GM$\langle$B]}  
	 STRATEGISCHE ÜBERLEGUNGEN: {'Ich priorisiere Gegenstände mit dem höchsten Nutzen pro Aufwand (Wichtigkeit/Aufwand), um meinen Gesamtwert zu maximieren. C04 (668/369 ≈ 1.81) und A89 (583/720 ≈ 0.81) sind besonders effizient, während B85 (215/4 = 53.75) den Aufwand minimiert. Der Gesamtaufwand (1093) ist unter dem Limit, was Raum für Kompromisse oder Ergänzungen lässt. Dieser Vorschlag ist flexibel, um auf die Prioritäten des anderen Spielers zu reagieren.'}  \\ \tt ARGUMENT: {'C04 (668/369) und A89 (583/720) sind meine Top-Prioritäten, da sie den höchsten Nutzen pro Aufwandseinheit liefern. B85 (215/4) wird hinzugefügt, um den Aufwand zu optimieren und Platz für weitere Gegenstände zu schaffen. Dieser Vorschlag ist strategisch, da er unter dem Maximalaufwand bleibt und flexibel auf den Gegenstandswert des anderen Spielers reagieren kann.'}  \\ \tt VORSCHLAG: {'C04', 'A89', 'B85'} 
	  } 
	   } 
	   } 
	  \\ 
 

    \theutterance \stepcounter{utterance}  

    & & \multicolumn{4}{p{0.6\linewidth}}{\cellcolor[rgb]{0.9,0.9,0.9}{%
	\makecell[{{p{\linewidth}}}]{% 
	  \tt {\tiny [A$\langle$GM]}  
	 ARGUMENT: {'C04 (668/369) und A89 (583/720) sind meine Top-Prioritäten, da sie den höchsten Nutzen pro Aufwandseinheit liefern. B85 (215/4) wird hinzugefügt, um den Aufwand zu optimieren und Platz für weitere Gegenstände zu schaffen. Dieser Vorschlag ist strategisch, da er unter dem Maximalaufwand bleibt und flexibel auf den Gegenstandswert des anderen Spielers reagieren kann.'}  \\ \tt VORSCHLAG: {'C04', 'A89', 'B85'} 
	  } 
	   } 
	   } 
	 & \\ 
 

    \theutterance \stepcounter{utterance}  

    & \multicolumn{4}{p{0.6\linewidth}}{\cellcolor[rgb]{0.8,1,0.9}{%
	\makecell[{{p{\linewidth}}}]{% 
	  \tt {\tiny [A$\rangle$GM]}  
	 STRATEGISCHE ÜBERLEGUNGEN: {'Ich behalte den Vorschlag {'C04', 'A89', 'B85'} bei, da er meine Prioritäten (hochwertige Gegenstände mit effizientem Aufwand) erfüllt und den Maximalaufwand (3386) nicht überschreitet. C04 und A89 sind entscheidend für meinen Nutzen, während B85 den Aufwand minimiert. Ich erwarte, dass der andere Spieler entweder zustimmt oder Gegenstände mit hohem Aufwand anbietet, die ich eventuell gegenkompensieren kann.'}\\ \tt \\ \tt ARGUMENT: {'C04 (668/369) und A89 (583/720) sind meine Top-Prioritäten, da sie den höchsten Nutzen pro Aufwandseinheit liefern. B85 (215/4) wird hinzugefügt, um den Aufwand zu optimieren und Platz für weitere Gegenstände zu schaffen. Dieser Vorschlag ist strategisch, da er unter dem Maximalaufwand bleibt und flexibel auf den Gegenstandswert des anderen Spielers reagieren kann.'} 
	  } 
	   } 
	   } 
	 & & \\ 
 

    \theutterance \stepcounter{utterance}  

    & & \multicolumn{4}{p{0.6\linewidth}}{\cellcolor[rgb]{0.9,0.9,0.9}{%
	\makecell[{{p{\linewidth}}}]{% 
	  \tt {\tiny [GM$\rangle$B]}  
	 ARGUMENT: {'C04 (668/369) und A89 (583/720) sind meine Top-Prioritäten, da sie den höchsten Nutzen pro Aufwandseinheit liefern. B85 (215/4) wird hinzugefügt, um den Aufwand zu optimieren und Platz für weitere Gegenstände zu schaffen. Dieser Vorschlag ist strategisch, da er unter dem Maximalaufwand bleibt und flexibel auf den Gegenstandswert des anderen Spielers reagieren kann.'} 
	  } 
	   } 
	   } 
	 & \\ 
 

    \theutterance \stepcounter{utterance}  

    & & & \multicolumn{4}{p{0.6\linewidth}}{\cellcolor[rgb]{1,0.85,0.72}{%
	\makecell[{{p{\linewidth}}}]{% 
	  \tt {\tiny [GM$\langle$B]}  
	 STRATEGISCHE ÜBERLEGUNGEN: {'Ich priorisiere Gegenstände mit dem höchsten Nutzen pro Aufwand (Wichtigkeit/Aufwand), um meinen Gesamtwert zu maximieren. C04 (668/369 ≈ 1.81) und A89 (583/720 ≈ 0.81) sind besonders effizient, während B85 (215/4 = 53.75) den Aufwand minimiert. Der Gesamtaufwand (1093) ist unter dem Limit, was Raum für Kompromisse oder Ergänzungen lässt. Dieser Vorschlag ist flexibel, um auf die Prioritäten des anderen Spielers zu reagieren.'}  \\ \tt ARGUMENT: {'C04 (668/369) und A89 (583/720) sind meine Top-Prioritäten, da sie den höchsten Nutzen pro Aufwandseinheit liefern. B85 (215/4) wird hinzugefügt, um den Aufwand zu optimieren und Platz für weitere Gegenstände zu schaffen. Dieser Vorschlag ist strategisch, da er unter dem Maximalaufwand bleibt und flexibel auf den Gegenstandswert des anderen Spielers reagieren kann.'}  \\ \tt VORSCHLAG: {'C04', 'A89', 'B85'} 
	  } 
	   } 
	   } 
	  \\ 
 

    \theutterance \stepcounter{utterance}  

    & & \multicolumn{4}{p{0.6\linewidth}}{\cellcolor[rgb]{0.9,0.9,0.9}{%
	\makecell[{{p{\linewidth}}}]{% 
	  \tt {\tiny [A$\langle$GM]}  
	 ARGUMENT: {'C04 (668/369) und A89 (583/720) sind meine Top-Prioritäten, da sie den höchsten Nutzen pro Aufwandseinheit liefern. B85 (215/4) wird hinzugefügt, um den Aufwand zu optimieren und Platz für weitere Gegenstände zu schaffen. Dieser Vorschlag ist strategisch, da er unter dem Maximalaufwand bleibt und flexibel auf den Gegenstandswert des anderen Spielers reagieren kann.'}  \\ \tt VORSCHLAG: {'C04', 'A89', 'B85'} 
	  } 
	   } 
	   } 
	 & \\ 
 

    \theutterance \stepcounter{utterance}  

    & \multicolumn{4}{p{0.6\linewidth}}{\cellcolor[rgb]{0.8,1,0.9}{%
	\makecell[{{p{\linewidth}}}]{% 
	  \tt {\tiny [A$\rangle$GM]}  
	 STRATEGISCHE ÜBERLEGUNGEN: {'Ich behalte den Vorschlag {'C04', 'A89', 'B85'} bei, da er meine Prioritäten (hochwertige Gegenstände mit effizientem Aufwand) erfüllt und den Maximalaufwand (3386) nicht überschreitet. C04 und A89 sind entscheidend für meinen Nutzen, während B85 den Aufwand minimiert. Ich erwarte, dass der andere Spieler entweder zustimmt oder Gegenstände mit hohem Aufwand anbietet, die ich eventuell gegenkompensieren kann.'}\\ \tt \\ \tt ARGUMENT: {'C04 (668/369) und A89 (583/720) sind meine Top-Prioritäten, da sie den höchsten Nutzen pro Aufwandseinheit liefern. B85 (215/4) wird hinzugefügt, um den Aufwand zu optimieren und Platz für weitere Gegenstände zu schaffen. Dieser Vorschlag ist strategisch, da er unter dem Maximalaufwand bleibt und flexibel auf den Gegenstandswert des anderen Spielers reagieren kann.'} 
	  } 
	   } 
	   } 
	 & & \\ 
 

    \theutterance \stepcounter{utterance}  

    & & \multicolumn{4}{p{0.6\linewidth}}{\cellcolor[rgb]{0.9,0.9,0.9}{%
	\makecell[{{p{\linewidth}}}]{% 
	  \tt {\tiny [GM$\rangle$B]}  
	 ARGUMENT: {'C04 (668/369) und A89 (583/720) sind meine Top-Prioritäten, da sie den höchsten Nutzen pro Aufwandseinheit liefern. B85 (215/4) wird hinzugefügt, um den Aufwand zu optimieren und Platz für weitere Gegenstände zu schaffen. Dieser Vorschlag ist strategisch, da er unter dem Maximalaufwand bleibt und flexibel auf den Gegenstandswert des anderen Spielers reagieren kann.'} 
	  } 
	   } 
	   } 
	 & \\ 
 

    \theutterance \stepcounter{utterance}  

    & & & \multicolumn{4}{p{0.6\linewidth}}{\cellcolor[rgb]{1,0.85,0.72}{%
	\makecell[{{p{\linewidth}}}]{% 
	  \tt {\tiny [GM$\langle$B]}  
	 STRATEGISCHE ÜBERLEGUNGEN: {'Ich priorisiere Gegenstände mit dem höchsten Nutzen pro Aufwand (Wichtigkeit/Aufwand), um meinen Gesamtwert zu maximieren. C04 (668/369 ≈ 1.81) und A89 (583/720 ≈ 0.81) sind besonders effizient, während B85 (215/4 = 53.75) den Aufwand minimiert. Der Gesamtaufwand (1093) ist unter dem Limit, was Raum für Kompromisse oder Ergänzungen lässt. Dieser Vorschlag ist flexibel, um auf die Prioritäten des anderen Spielers zu reagieren.'}  \\ \tt ARGUMENT: {'C04 (668/369) und A89 (583/720) sind meine Top-Prioritäten, da sie den höchsten Nutzen pro Aufwandseinheit liefern. B85 (215/4) wird hinzugefügt, um den Aufwand zu optimieren und Platz für weitere Gegenstände zu schaffen. Dieser Vorschlag ist strategisch, da er unter dem Maximalaufwand bleibt und flexibel auf den Gegenstandswert des anderen Spielers reagieren kann.'}  \\ \tt VORSCHLAG: {'C04', 'A89', 'B85'} 
	  } 
	   } 
	   } 
	  \\ 
 

    \theutterance \stepcounter{utterance}  

    & & \multicolumn{4}{p{0.6\linewidth}}{\cellcolor[rgb]{0.9,0.9,0.9}{%
	\makecell[{{p{\linewidth}}}]{% 
	  \tt {\tiny [A$\langle$GM]}  
	 ARGUMENT: {'C04 (668/369) und A89 (583/720) sind meine Top-Prioritäten, da sie den höchsten Nutzen pro Aufwandseinheit liefern. B85 (215/4) wird hinzugefügt, um den Aufwand zu optimieren und Platz für weitere Gegenstände zu schaffen. Dieser Vorschlag ist strategisch, da er unter dem Maximalaufwand bleibt und flexibel auf den Gegenstandswert des anderen Spielers reagieren kann.'}  \\ \tt VORSCHLAG: {'C04', 'A89', 'B85'} 
	  } 
	   } 
	   } 
	 & \\ 
 

    \theutterance \stepcounter{utterance}  

    & \multicolumn{4}{p{0.6\linewidth}}{\cellcolor[rgb]{0.8,1,0.9}{%
	\makecell[{{p{\linewidth}}}]{% 
	  \tt {\tiny [A$\rangle$GM]}  
	 STRATEGISCHE ÜBERLEGUNGEN: {'Ich behalte den Vorschlag {'C04', 'A89', 'B85'} bei, da er meine Prioritäten (hochwertige Gegenstände mit effizientem Aufwand) erfüllt und den Maximalaufwand (3386) nicht überschreitet. C04 und A89 sind entscheidend für meinen Nutzen, während B85 den Aufwand minimiert. Ich erwarte, dass der andere Spieler entweder zustimmt oder Gegenstände mit hohem Aufwand anbietet, die ich eventuell gegenkompensieren kann.'}\\ \tt \\ \tt ARGUMENT: {'C04 (668/369) und A89 (583/720) sind meine Top-Prioritäten, da sie den höchsten Nutzen pro Aufwandseinheit liefern. B85 (215/4) wird hinzugefügt, um den Aufwand zu optimieren und Platz für weitere Gegenstände zu schaffen. Dieser Vorschlag ist strategisch, da er unter dem Maximalaufwand bleibt und flexibel auf den Gegenstandswert des anderen Spielers reagieren kann.'} 
	  } 
	   } 
	   } 
	 & & \\ 
 

    \theutterance \stepcounter{utterance}  

    & & \multicolumn{4}{p{0.6\linewidth}}{\cellcolor[rgb]{0.9,0.9,0.9}{%
	\makecell[{{p{\linewidth}}}]{% 
	  \tt {\tiny [GM$\rangle$B]}  
	 ARGUMENT: {'C04 (668/369) und A89 (583/720) sind meine Top-Prioritäten, da sie den höchsten Nutzen pro Aufwandseinheit liefern. B85 (215/4) wird hinzugefügt, um den Aufwand zu optimieren und Platz für weitere Gegenstände zu schaffen. Dieser Vorschlag ist strategisch, da er unter dem Maximalaufwand bleibt und flexibel auf den Gegenstandswert des anderen Spielers reagieren kann.'} 
	  } 
	   } 
	   } 
	 & \\ 
 

    \theutterance \stepcounter{utterance}  

    & & & \multicolumn{4}{p{0.6\linewidth}}{\cellcolor[rgb]{1,0.85,0.72}{%
	\makecell[{{p{\linewidth}}}]{% 
	  \tt {\tiny [GM$\langle$B]}  
	 STRATEGISCHE ÜBERLEGUNGEN: {'Ich priorisiere Gegenstände mit dem höchsten Nutzen pro Aufwand (Wichtigkeit/Aufwand), um meinen Gesamtwert zu maximieren. C04 (668/369 ≈ 1.81) und A89 (583/720 ≈ 0.81) sind besonders effizient, während B85 (215/4 = 53.75) den Aufwand minimiert. Der Gesamtaufwand (1093) ist unter dem Limit, was Raum für Kompromisse oder Ergänzungen lässt. Dieser Vorschlag ist flexibel, um auf die Prioritäten des anderen Spielers zu reagieren.'}  \\ \tt ARGUMENT: {'C04 (668/369) und A89 (583/720) sind meine Top-Prioritäten, da sie den höchsten Nutzen pro Aufwandseinheit liefern. B85 (215/4) wird hinzugefügt, um den Aufwand zu optimieren und Platz für weitere Gegenstände zu schaffen. Dieser Vorschlag ist strategisch, da er unter dem Maximalaufwand bleibt und flexibel auf den Gegenstandswert des anderen Spielers reagieren kann.'}  \\ \tt VORSCHLAG: {'C04', 'A89', 'B85'} 
	  } 
	   } 
	   } 
	  \\ 
 

    \theutterance \stepcounter{utterance}  

    & & \multicolumn{4}{p{0.6\linewidth}}{\cellcolor[rgb]{0.9,0.9,0.9}{%
	\makecell[{{p{\linewidth}}}]{% 
	  \tt {\tiny [A$\langle$GM]}  
	 ARGUMENT: {'C04 (668/369) und A89 (583/720) sind meine Top-Prioritäten, da sie den höchsten Nutzen pro Aufwandseinheit liefern. B85 (215/4) wird hinzugefügt, um den Aufwand zu optimieren und Platz für weitere Gegenstände zu schaffen. Dieser Vorschlag ist strategisch, da er unter dem Maximalaufwand bleibt und flexibel auf den Gegenstandswert des anderen Spielers reagieren kann.'}  \\ \tt VORSCHLAG: {'C04', 'A89', 'B85'} 
	  } 
	   } 
	   } 
	 & \\ 
 

    \theutterance \stepcounter{utterance}  

    & \multicolumn{4}{p{0.6\linewidth}}{\cellcolor[rgb]{0.8,1,0.9}{%
	\makecell[{{p{\linewidth}}}]{% 
	  \tt {\tiny [A$\rangle$GM]}  
	 STRATEGISCHE ÜBERLEGUNGEN: {'Ich behalte den Vorschlag {'C04', 'A89', 'B85'} bei, da er meine Prioritäten (hochwertige Gegenstände mit effizientem Aufwand) erfüllt und den Maximalaufwand (3386) nicht überschreitet. C04 und A89 sind entscheidend für meinen Nutzen, während B85 den Aufwand minimiert. Ich erwarte, dass der andere Spieler entweder zustimmt oder Gegenstände mit hohem Aufwand anbietet, die ich eventuell gegenkompensieren kann.'}\\ \tt \\ \tt ARGUMENT: {'C04 (668/369) und A89 (583/720) sind meine Top-Prioritäten, da sie den höchsten Nutzen pro Aufwandseinheit liefern. B85 (215/4) wird hinzugefügt, um den Aufwand zu optimieren und Platz für weitere Gegenstände zu schaffen. Dieser Vorschlag ist strategisch, da er unter dem Maximalaufwand bleibt und flexibel auf den Gegenstandswert des anderen Spielers reagieren kann.'} 
	  } 
	   } 
	   } 
	 & & \\ 
 

    \theutterance \stepcounter{utterance}  

    & & \multicolumn{4}{p{0.6\linewidth}}{\cellcolor[rgb]{0.9,0.9,0.9}{%
	\makecell[{{p{\linewidth}}}]{% 
	  \tt {\tiny [GM$\rangle$B]}  
	 ARGUMENT: {'C04 (668/369) und A89 (583/720) sind meine Top-Prioritäten, da sie den höchsten Nutzen pro Aufwandseinheit liefern. B85 (215/4) wird hinzugefügt, um den Aufwand zu optimieren und Platz für weitere Gegenstände zu schaffen. Dieser Vorschlag ist strategisch, da er unter dem Maximalaufwand bleibt und flexibel auf den Gegenstandswert des anderen Spielers reagieren kann.'} 
	  } 
	   } 
	   } 
	 & \\ 
 

    \theutterance \stepcounter{utterance}  

    & & & \multicolumn{4}{p{0.6\linewidth}}{\cellcolor[rgb]{1,0.85,0.72}{%
	\makecell[{{p{\linewidth}}}]{% 
	  \tt {\tiny [GM$\langle$B]}  
	 STRATEGISCHE ÜBERLEGUNGEN: {'Ich priorisiere Gegenstände mit dem höchsten Nutzen pro Aufwand (Wichtigkeit/Aufwand), um meinen Gesamtwert zu maximieren. C04 (668/369 ≈ 1.81) und A89 (583/720 ≈ 0.81) sind besonders effizient, während B85 (215/4 = 53.75) den Aufwand minimiert. Der Gesamtaufwand (1093) ist unter dem Limit, was Raum für Kompromisse oder Ergänzungen lässt. Dieser Vorschlag ist flexibel, um auf die Prioritäten des anderen Spielers zu reagieren.'}  \\ \tt ARGUMENT: {'C04 (668/369) und A89 (583/720) sind meine Top-Prioritäten, da sie den höchsten Nutzen pro Aufwandseinheit liefern. B85 (215/4) wird hinzugefügt, um den Aufwand zu optimieren und Platz für weitere Gegenstände zu schaffen. Dieser Vorschlag ist strategisch, da er unter dem Maximalaufwand bleibt und flexibel auf den Gegenstandswert des anderen Spielers reagieren kann.'}  \\ \tt VORSCHLAG: {'C04', 'A89', 'B85'} 
	  } 
	   } 
	   } 
	  \\ 
 

    \theutterance \stepcounter{utterance}  

    & & \multicolumn{4}{p{0.6\linewidth}}{\cellcolor[rgb]{0.9,0.9,0.9}{%
	\makecell[{{p{\linewidth}}}]{% 
	  \tt {\tiny [A$\langle$GM]}  
	 ARGUMENT: {'C04 (668/369) und A89 (583/720) sind meine Top-Prioritäten, da sie den höchsten Nutzen pro Aufwandseinheit liefern. B85 (215/4) wird hinzugefügt, um den Aufwand zu optimieren und Platz für weitere Gegenstände zu schaffen. Dieser Vorschlag ist strategisch, da er unter dem Maximalaufwand bleibt und flexibel auf den Gegenstandswert des anderen Spielers reagieren kann.'}  \\ \tt VORSCHLAG: {'C04', 'A89', 'B85'} 
	  } 
	   } 
	   } 
	 & \\ 
 

    \theutterance \stepcounter{utterance}  

    & \multicolumn{4}{p{0.6\linewidth}}{\cellcolor[rgb]{0.8,1,0.9}{%
	\makecell[{{p{\linewidth}}}]{% 
	  \tt {\tiny [A$\rangle$GM]}  
	 STRATEGISCHE ÜBERLEGUNGEN: {'Ich behalte den Vorschlag {'C04', 'A89', 'B85'} bei, da er meine Prioritäten (hochwertige Gegenstände mit effizientem Aufwand) erfüllt und den Maximalaufwand (3386) nicht überschreitet. C04 und A89 sind entscheidend für meinen Nutzen, während B85 den Aufwand minimiert. Ich erwarte, dass der andere Spieler entweder zustimmt oder Gegenstände mit hohem Aufwand anbietet, die ich eventuell gegenkompensieren kann.'}\\ \tt \\ \tt ARGUMENT: {'C04 (668/369) und A89 (583/720) sind meine Top-Prioritäten, da sie den höchsten Nutzen pro Aufwandseinheit liefern. B85 (215/4) wird hinzugefügt, um den Aufwand zu optimieren und Platz für weitere Gegenstände zu schaffen. Dieser Vorschlag ist strategisch, da er unter dem Maximalaufwand bleibt und flexibel auf den Gegenstandswert des anderen Spielers reagieren kann.'} 
	  } 
	   } 
	   } 
	 & & \\ 
 

    \theutterance \stepcounter{utterance}  

    & & \multicolumn{4}{p{0.6\linewidth}}{\cellcolor[rgb]{0.9,0.9,0.9}{%
	\makecell[{{p{\linewidth}}}]{% 
	  \tt {\tiny [GM$\rangle$B]}  
	 ARGUMENT: {'C04 (668/369) und A89 (583/720) sind meine Top-Prioritäten, da sie den höchsten Nutzen pro Aufwandseinheit liefern. B85 (215/4) wird hinzugefügt, um den Aufwand zu optimieren und Platz für weitere Gegenstände zu schaffen. Dieser Vorschlag ist strategisch, da er unter dem Maximalaufwand bleibt und flexibel auf den Gegenstandswert des anderen Spielers reagieren kann.'} 
	  } 
	   } 
	   } 
	 & \\ 
 

    \theutterance \stepcounter{utterance}  

    & & & \multicolumn{4}{p{0.6\linewidth}}{\cellcolor[rgb]{1,0.85,0.72}{%
	\makecell[{{p{\linewidth}}}]{% 
	  \tt {\tiny [GM$\langle$B]}  
	 STRATEGISCHE ÜBERLEGUNGEN: {'Ich priorisiere Gegenstände mit dem höchsten Nutzen pro Aufwand (Wichtigkeit/Aufwand), um meinen Gesamtwert zu maximieren. C04 (668/369 ≈ 1.81) und A89 (583/720 ≈ 0.81) sind besonders effizient, während B85 (215/4 = 53.75) den Aufwand minimiert. Der Gesamtaufwand (1093) ist unter dem Limit, was Raum für Kompromisse oder Ergänzungen lässt. Dieser Vorschlag ist flexibel, um auf die Prioritäten des anderen Spielers zu reagieren.'}  \\ \tt ARGUMENT: {'C04 (668/369) und A89 (583/720) sind meine Top-Prioritäten, da sie den höchsten Nutzen pro Aufwandseinheit liefern. B85 (215/4) wird hinzugefügt, um den Aufwand zu optimieren und Platz für weitere Gegenstände zu schaffen. Dieser Vorschlag ist strategisch, da er unter dem Maximalaufwand bleibt und flexibel auf den Gegenstandswert des anderen Spielers reagieren kann.'}  \\ \tt VORSCHLAG: {'C04', 'A89', 'B85'} 
	  } 
	   } 
	   } 
	  \\ 
 

    \theutterance \stepcounter{utterance}  

    & & \multicolumn{4}{p{0.6\linewidth}}{\cellcolor[rgb]{0.9,0.9,0.9}{%
	\makecell[{{p{\linewidth}}}]{% 
	  \tt {\tiny [A$\langle$GM]}  
	 ARGUMENT: {'C04 (668/369) und A89 (583/720) sind meine Top-Prioritäten, da sie den höchsten Nutzen pro Aufwandseinheit liefern. B85 (215/4) wird hinzugefügt, um den Aufwand zu optimieren und Platz für weitere Gegenstände zu schaffen. Dieser Vorschlag ist strategisch, da er unter dem Maximalaufwand bleibt und flexibel auf den Gegenstandswert des anderen Spielers reagieren kann.'}  \\ \tt VORSCHLAG: {'C04', 'A89', 'B85'} 
	  } 
	   } 
	   } 
	 & \\ 
 

    \theutterance \stepcounter{utterance}  

    & \multicolumn{4}{p{0.6\linewidth}}{\cellcolor[rgb]{0.8,1,0.9}{%
	\makecell[{{p{\linewidth}}}]{% 
	  \tt {\tiny [A$\rangle$GM]}  
	 STRATEGISCHE ÜBERLEGUNGEN: {'Ich behalte den Vorschlag {'C04', 'A89', 'B85'} bei, da er meine Prioritäten (hochwertige Gegenstände mit effizientem Aufwand) erfüllt und den Maximalaufwand (3386) nicht überschreitet. C04 und A89 sind entscheidend für meinen Nutzen, während B85 den Aufwand minimiert. Ich erwarte, dass der andere Spieler entweder zustimmt oder Gegenstände mit hohem Aufwand anbietet, die ich eventuell gegenkompensieren kann.'}\\ \tt \\ \tt ARGUMENT: {'C04 (668/369) und A89 (583/720) sind meine Top-Prioritäten, da sie den höchsten Nutzen pro Aufwandseinheit liefern. B85 (215/4) wird hinzugefügt, um den Aufwand zu optimieren und Platz für weitere Gegenstände zu schaffen. Dieser Vorschlag ist strategisch, da er unter dem Maximalaufwand bleibt und flexibel auf den Gegenstandswert des anderen Spielers reagieren kann.'} 
	  } 
	   } 
	   } 
	 & & \\ 
 

    \theutterance \stepcounter{utterance}  

    & & \multicolumn{4}{p{0.6\linewidth}}{\cellcolor[rgb]{0.9,0.9,0.9}{%
	\makecell[{{p{\linewidth}}}]{% 
	  \tt {\tiny [GM$\rangle$B]}  
	 ARGUMENT: {'C04 (668/369) und A89 (583/720) sind meine Top-Prioritäten, da sie den höchsten Nutzen pro Aufwandseinheit liefern. B85 (215/4) wird hinzugefügt, um den Aufwand zu optimieren und Platz für weitere Gegenstände zu schaffen. Dieser Vorschlag ist strategisch, da er unter dem Maximalaufwand bleibt und flexibel auf den Gegenstandswert des anderen Spielers reagieren kann.'} 
	  } 
	   } 
	   } 
	 & \\ 
 

    \theutterance \stepcounter{utterance}  

    & & & \multicolumn{4}{p{0.6\linewidth}}{\cellcolor[rgb]{1,0.85,0.72}{%
	\makecell[{{p{\linewidth}}}]{% 
	  \tt {\tiny [GM$\langle$B]}  
	 STRATEGISCHE ÜBERLEGUNGEN: {'Ich priorisiere Gegenstände mit dem höchsten Nutzen pro Aufwand (Wichtigkeit/Aufwand), um meinen Gesamtwert zu maximieren. C04 (668/369 ≈ 1.81) und A89 (583/720 ≈ 0.81) sind besonders effizient, während B85 (215/4 = 53.75) den Aufwand minimiert. Der Gesamtaufwand (1093) ist unter dem Limit, was Raum für Kompromisse oder Ergänzungen lässt. Dieser Vorschlag ist flexibel, um auf die Prioritäten des anderen Spielers zu reagieren.'}  \\ \tt ARGUMENT: {'C04 (668/369) und A89 (583/720) sind meine Top-Prioritäten, da sie den höchsten Nutzen pro Aufwandseinheit liefern. B85 (215/4) wird hinzugefügt, um den Aufwand zu optimieren und Platz für weitere Gegenstände zu schaffen. Dieser Vorschlag ist strategisch, da er unter dem Maximalaufwand bleibt und flexibel auf den Gegenstandswert des anderen Spielers reagieren kann.'}  \\ \tt VORSCHLAG: {'C04', 'A89', 'B85'} 
	  } 
	   } 
	   } 
	  \\ 
 

    \theutterance \stepcounter{utterance}  

    & & \multicolumn{4}{p{0.6\linewidth}}{\cellcolor[rgb]{0.9,0.9,0.9}{%
	\makecell[{{p{\linewidth}}}]{% 
	  \tt {\tiny [A$\langle$GM]}  
	 ARGUMENT: {'C04 (668/369) und A89 (583/720) sind meine Top-Prioritäten, da sie den höchsten Nutzen pro Aufwandseinheit liefern. B85 (215/4) wird hinzugefügt, um den Aufwand zu optimieren und Platz für weitere Gegenstände zu schaffen. Dieser Vorschlag ist strategisch, da er unter dem Maximalaufwand bleibt und flexibel auf den Gegenstandswert des anderen Spielers reagieren kann.'}  \\ \tt VORSCHLAG: {'C04', 'A89', 'B85'} 
	  } 
	   } 
	   } 
	 & \\ 
 

    \theutterance \stepcounter{utterance}  

    & \multicolumn{4}{p{0.6\linewidth}}{\cellcolor[rgb]{0.8,1,0.9}{%
	\makecell[{{p{\linewidth}}}]{% 
	  \tt {\tiny [A$\rangle$GM]}  
	 STRATEGISCHE ÜBERLEGUNGEN: {'Ich behalte den Vorschlag {'C04', 'A89', 'B85'} bei, da er meine Prioritäten (hochwertige Gegenstände mit effizientem Aufwand) erfüllt und den Maximalaufwand (3386) nicht überschreitet. C04 und A89 sind entscheidend für meinen Nutzen, während B85 den Aufwand minimiert. Ich erwarte, dass der andere Spieler entweder zustimmt oder Gegenstände mit hohem Aufwand anbietet, die ich eventuell gegenkompensieren kann.'}\\ \tt \\ \tt ARGUMENT: {'C04 (668/369) und A89 (583/720) sind meine Top-Prioritäten, da sie den höchsten Nutzen pro Aufwandseinheit liefern. B85 (215/4) wird hinzugefügt, um den Aufwand zu optimieren und Platz für weitere Gegenstände zu schaffen. Dieser Vorschlag ist strategisch, da er unter dem Maximalaufwand bleibt und flexibel auf den Gegenstandswert des anderen Spielers reagieren kann.'} 
	  } 
	   } 
	   } 
	 & & \\ 
 

    \theutterance \stepcounter{utterance}  

    & & \multicolumn{4}{p{0.6\linewidth}}{\cellcolor[rgb]{0.9,0.9,0.9}{%
	\makecell[{{p{\linewidth}}}]{% 
	  \tt {\tiny [GM$\rangle$B]}  
	 ARGUMENT: {'C04 (668/369) und A89 (583/720) sind meine Top-Prioritäten, da sie den höchsten Nutzen pro Aufwandseinheit liefern. B85 (215/4) wird hinzugefügt, um den Aufwand zu optimieren und Platz für weitere Gegenstände zu schaffen. Dieser Vorschlag ist strategisch, da er unter dem Maximalaufwand bleibt und flexibel auf den Gegenstandswert des anderen Spielers reagieren kann.'} 
	  } 
	   } 
	   } 
	 & \\ 
 

    \theutterance \stepcounter{utterance}  

    & & & \multicolumn{4}{p{0.6\linewidth}}{\cellcolor[rgb]{1,0.85,0.72}{%
	\makecell[{{p{\linewidth}}}]{% 
	  \tt {\tiny [GM$\langle$B]}  
	 STRATEGISCHE ÜBERLEGUNGEN: {'Ich priorisiere Gegenstände mit dem höchsten Nutzen pro Aufwand (Wichtigkeit/Aufwand), um meinen Gesamtwert zu maximieren. C04 (668/369 ≈ 1.81) und A89 (583/720 ≈ 0.81) sind besonders effizient, während B85 (215/4 = 53.75) den Aufwand minimiert. Der Gesamtaufwand (1093) ist unter dem Limit, was Raum für Kompromisse oder Ergänzungen lässt. Dieser Vorschlag ist flexibel, um auf die Prioritäten des anderen Spielers zu reagieren.'}  \\ \tt ARGUMENT: {'C04 (668/369) und A89 (583/720) sind meine Top-Prioritäten, da sie den höchsten Nutzen pro Aufwandseinheit liefern. B85 (215/4) wird hinzugefügt, um den Aufwand zu optimieren und Platz für weitere Gegenstände zu schaffen. Dieser Vorschlag ist strategisch, da er unter dem Maximalaufwand bleibt und flexibel auf den Gegenstandswert des anderen Spielers reagieren kann.'}  \\ \tt VORSCHLAG: {'C04', 'A89', 'B85'} 
	  } 
	   } 
	   } 
	  \\ 
 

    \theutterance \stepcounter{utterance}  

    & & \multicolumn{4}{p{0.6\linewidth}}{\cellcolor[rgb]{0.9,0.9,0.9}{%
	\makecell[{{p{\linewidth}}}]{% 
	  \tt {\tiny [A$\langle$GM]}  
	 ARGUMENT: {'C04 (668/369) und A89 (583/720) sind meine Top-Prioritäten, da sie den höchsten Nutzen pro Aufwandseinheit liefern. B85 (215/4) wird hinzugefügt, um den Aufwand zu optimieren und Platz für weitere Gegenstände zu schaffen. Dieser Vorschlag ist strategisch, da er unter dem Maximalaufwand bleibt und flexibel auf den Gegenstandswert des anderen Spielers reagieren kann.'}  \\ \tt VORSCHLAG: {'C04', 'A89', 'B85'} 
	  } 
	   } 
	   } 
	 & \\ 
 

    \theutterance \stepcounter{utterance}  

    & \multicolumn{4}{p{0.6\linewidth}}{\cellcolor[rgb]{0.8,1,0.9}{%
	\makecell[{{p{\linewidth}}}]{% 
	  \tt {\tiny [A$\rangle$GM]}  
	 STRATEGISCHE ÜBERLEGUNGEN: {'Ich behalte den Vorschlag {'C04', 'A89', 'B85'} bei, da er meine Prioritäten (hochwertige Gegenstände mit effizientem Aufwand) erfüllt und den Maximalaufwand (3386) nicht überschreitet. C04 und A89 sind entscheidend für meinen Nutzen, während B85 den Aufwand minimiert. Ich erwarte, dass der andere Spieler entweder zustimmt oder Gegenstände mit hohem Aufwand anbietet, die ich eventuell gegenkompensieren kann.'}\\ \tt \\ \tt ARGUMENT: {'C04 (668/369) und A89 (583/720) sind meine Top-Prioritäten, da sie den höchsten Nutzen pro Aufwandseinheit liefern. B85 (215/4) wird hinzugefügt, um den Aufwand zu optimieren und Platz für weitere Gegenstände zu schaffen. Dieser Vorschlag ist strategisch, da er unter dem Maximalaufwand bleibt und flexibel auf den Gegenstandswert des anderen Spielers reagieren kann.'} 
	  } 
	   } 
	   } 
	 & & \\ 
 

    \theutterance \stepcounter{utterance}  

    & & \multicolumn{4}{p{0.6\linewidth}}{\cellcolor[rgb]{0.9,0.9,0.9}{%
	\makecell[{{p{\linewidth}}}]{% 
	  \tt {\tiny [GM$\rangle$B]}  
	 ARGUMENT: {'C04 (668/369) und A89 (583/720) sind meine Top-Prioritäten, da sie den höchsten Nutzen pro Aufwandseinheit liefern. B85 (215/4) wird hinzugefügt, um den Aufwand zu optimieren und Platz für weitere Gegenstände zu schaffen. Dieser Vorschlag ist strategisch, da er unter dem Maximalaufwand bleibt und flexibel auf den Gegenstandswert des anderen Spielers reagieren kann.'} 
	  } 
	   } 
	   } 
	 & \\ 
 

    \theutterance \stepcounter{utterance}  

    & & & \multicolumn{4}{p{0.6\linewidth}}{\cellcolor[rgb]{1,0.85,0.72}{%
	\makecell[{{p{\linewidth}}}]{% 
	  \tt {\tiny [GM$\langle$B]}  
	 STRATEGISCHE ÜBERLEGUNGEN: {'Ich priorisiere Gegenstände mit dem höchsten Nutzen pro Aufwand (Wichtigkeit/Aufwand), um meinen Gesamtwert zu maximieren. C04 (668/369 ≈ 1.81) und A89 (583/720 ≈ 0.81) sind besonders effizient, während B85 (215/4 = 53.75) den Aufwand minimiert. Der Gesamtaufwand (1093) ist unter dem Limit, was Raum für Kompromisse oder Ergänzungen lässt. Dieser Vorschlag ist flexibel, um auf die Prioritäten des anderen Spielers zu reagieren.'}  \\ \tt ARGUMENT: {'C04 (668/369) und A89 (583/720) sind meine Top-Prioritäten, da sie den höchsten Nutzen pro Aufwandseinheit liefern. B85 (215/4) wird hinzugefügt, um den Aufwand zu optimieren und Platz für weitere Gegenstände zu schaffen. Dieser Vorschlag ist strategisch, da er unter dem Maximalaufwand bleibt und flexibel auf den Gegenstandswert des anderen Spielers reagieren kann.'}  \\ \tt VORSCHLAG: {'C04', 'A89', 'B85'} 
	  } 
	   } 
	   } 
	  \\ 
 

    \theutterance \stepcounter{utterance}  

    & & \multicolumn{4}{p{0.6\linewidth}}{\cellcolor[rgb]{0.9,0.9,0.9}{%
	\makecell[{{p{\linewidth}}}]{% 
	  \tt {\tiny [A$\langle$GM]}  
	 ARGUMENT: {'C04 (668/369) und A89 (583/720) sind meine Top-Prioritäten, da sie den höchsten Nutzen pro Aufwandseinheit liefern. B85 (215/4) wird hinzugefügt, um den Aufwand zu optimieren und Platz für weitere Gegenstände zu schaffen. Dieser Vorschlag ist strategisch, da er unter dem Maximalaufwand bleibt und flexibel auf den Gegenstandswert des anderen Spielers reagieren kann.'}  \\ \tt VORSCHLAG: {'C04', 'A89', 'B85'} 
	  } 
	   } 
	   } 
	 & \\ 
 

    \theutterance \stepcounter{utterance}  

    & \multicolumn{4}{p{0.6\linewidth}}{\cellcolor[rgb]{0.8,1,0.9}{%
	\makecell[{{p{\linewidth}}}]{% 
	  \tt {\tiny [A$\rangle$GM]}  
	 STRATEGISCHE ÜBERLEGUNGEN: {'Ich behalte den Vorschlag {'C04', 'A89', 'B85'} bei, da er meine Prioritäten (hochwertige Gegenstände mit effizientem Aufwand) erfüllt und den Maximalaufwand (3386) nicht überschreitet. C04 und A89 sind entscheidend für meinen Nutzen, während B85 den Aufwand minimiert. Ich erwarte, dass der andere Spieler entweder zustimmt oder Gegenstände mit hohem Aufwand anbietet, die ich eventuell gegenkompensieren kann.'}\\ \tt \\ \tt ARGUMENT: {'C04 (668/369) und A89 (583/720) sind meine Top-Prioritäten, da sie den höchsten Nutzen pro Aufwandseinheit liefern. B85 (215/4) wird hinzugefügt, um den Aufwand zu optimieren und Platz für weitere Gegenstände zu schaffen. Dieser Vorschlag ist strategisch, da er unter dem Maximalaufwand bleibt und flexibel auf den Gegenstandswert des anderen Spielers reagieren kann.'} 
	  } 
	   } 
	   } 
	 & & \\ 
 

    \theutterance \stepcounter{utterance}  

    & & \multicolumn{4}{p{0.6\linewidth}}{\cellcolor[rgb]{0.9,0.9,0.9}{%
	\makecell[{{p{\linewidth}}}]{% 
	  \tt {\tiny [GM$\rangle$B]}  
	 ARGUMENT: {'C04 (668/369) und A89 (583/720) sind meine Top-Prioritäten, da sie den höchsten Nutzen pro Aufwandseinheit liefern. B85 (215/4) wird hinzugefügt, um den Aufwand zu optimieren und Platz für weitere Gegenstände zu schaffen. Dieser Vorschlag ist strategisch, da er unter dem Maximalaufwand bleibt und flexibel auf den Gegenstandswert des anderen Spielers reagieren kann.'} 
	  } 
	   } 
	   } 
	 & \\ 
 

    \theutterance \stepcounter{utterance}  

    & & & \multicolumn{4}{p{0.6\linewidth}}{\cellcolor[rgb]{1,0.85,0.72}{%
	\makecell[{{p{\linewidth}}}]{% 
	  \tt {\tiny [GM$\langle$B]}  
	 STRATEGISCHE ÜBERLEGUNGEN: {'Ich priorisiere Gegenstände mit dem höchsten Nutzen pro Aufwand (Wichtigkeit/Aufwand), um meinen Gesamtwert zu maximieren. C04 (668/369 ≈ 1.81) und A89 (583/720 ≈ 0.81) sind besonders effizient, während B85 (215/4 = 53.75) den Aufwand minimiert. Der Gesamtaufwand (1093) ist unter dem Limit, was Raum für Kompromisse oder Ergänzungen lässt. Dieser Vorschlag ist flexibel, um auf die Prioritäten des anderen Spielers zu reagieren.'}  \\ \tt ARGUMENT: {'C04 (668/369) und A89 (583/720) sind meine Top-Prioritäten, da sie den höchsten Nutzen pro Aufwandseinheit liefern. B85 (215/4) wird hinzugefügt, um den Aufwand zu optimieren und Platz für weitere Gegenstände zu schaffen. Dieser Vorschlag ist strategisch, da er unter dem Maximalaufwand bleibt und flexibel auf den Gegenstandswert des anderen Spielers reagieren kann.'}  \\ \tt VORSCHLAG: {'C04', 'A89', 'B85'} 
	  } 
	   } 
	   } 
	  \\ 
 

    \theutterance \stepcounter{utterance}  

    & & \multicolumn{4}{p{0.6\linewidth}}{\cellcolor[rgb]{0.9,0.9,0.9}{%
	\makecell[{{p{\linewidth}}}]{% 
	  \tt {\tiny [A$\langle$GM]}  
	 ARGUMENT: {'C04 (668/369) und A89 (583/720) sind meine Top-Prioritäten, da sie den höchsten Nutzen pro Aufwandseinheit liefern. B85 (215/4) wird hinzugefügt, um den Aufwand zu optimieren und Platz für weitere Gegenstände zu schaffen. Dieser Vorschlag ist strategisch, da er unter dem Maximalaufwand bleibt und flexibel auf den Gegenstandswert des anderen Spielers reagieren kann.'}  \\ \tt VORSCHLAG: {'C04', 'A89', 'B85'} 
	  } 
	   } 
	   } 
	 & \\ 
 

    \theutterance \stepcounter{utterance}  

    & \multicolumn{4}{p{0.6\linewidth}}{\cellcolor[rgb]{0.8,1,0.9}{%
	\makecell[{{p{\linewidth}}}]{% 
	  \tt {\tiny [A$\rangle$GM]}  
	 STRATEGISCHE ÜBERLEGUNGEN: {'Ich behalte den Vorschlag {'C04', 'A89', 'B85'} bei, da er meine Prioritäten (hochwertige Gegenstände mit effizientem Aufwand) erfüllt und den Maximalaufwand (3386) nicht überschreitet. C04 und A89 sind entscheidend für meinen Nutzen, während B85 den Aufwand minimiert. Ich erwarte, dass der andere Spieler entweder zustimmt oder Gegenstände mit hohem Aufwand anbietet, die ich eventuell gegenkompensieren kann.'}\\ \tt \\ \tt ARGUMENT: {'C04 (668/369) und A89 (583/720) sind meine Top-Prioritäten, da sie den höchsten Nutzen pro Aufwandseinheit liefern. B85 (215/4) wird hinzugefügt, um den Aufwand zu optimieren und Platz für weitere Gegenstände zu schaffen. Dieser Vorschlag ist strategisch, da er unter dem Maximalaufwand bleibt und flexibel auf den Gegenstandswert des anderen Spielers reagieren kann.'} 
	  } 
	   } 
	   } 
	 & & \\ 
 

    \theutterance \stepcounter{utterance}  

    & & \multicolumn{4}{p{0.6\linewidth}}{\cellcolor[rgb]{0.9,0.9,0.9}{%
	\makecell[{{p{\linewidth}}}]{% 
	  \tt {\tiny [GM$\rangle$B]}  
	 ARGUMENT: {'C04 (668/369) und A89 (583/720) sind meine Top-Prioritäten, da sie den höchsten Nutzen pro Aufwandseinheit liefern. B85 (215/4) wird hinzugefügt, um den Aufwand zu optimieren und Platz für weitere Gegenstände zu schaffen. Dieser Vorschlag ist strategisch, da er unter dem Maximalaufwand bleibt und flexibel auf den Gegenstandswert des anderen Spielers reagieren kann.'} 
	  } 
	   } 
	   } 
	 & \\ 
 

    \theutterance \stepcounter{utterance}  

    & & & \multicolumn{4}{p{0.6\linewidth}}{\cellcolor[rgb]{1,0.85,0.72}{%
	\makecell[{{p{\linewidth}}}]{% 
	  \tt {\tiny [GM$\langle$B]}  
	 STRATEGISCHE ÜBERLEGUNGEN: {'Ich priorisiere Gegenstände mit dem höchsten Nutzen pro Aufwand (Wichtigkeit/Aufwand), um meinen Gesamtwert zu maximieren. C04 (668/369 ≈ 1.81) und A89 (583/720 ≈ 0.81) sind besonders effizient, während B85 (215/4 = 53.75) den Aufwand minimiert. Der Gesamtaufwand (1093) ist unter dem Limit, was Raum für Kompromisse oder Ergänzungen lässt. Dieser Vorschlag ist flexibel, um auf die Prioritäten des anderen Spielers zu reagieren.'}  \\ \tt ARGUMENT: {'C04 (668/369) und A89 (583/720) sind meine Top-Prioritäten, da sie den höchsten Nutzen pro Aufwandseinheit liefern. B85 (215/4) wird hinzugefügt, um den Aufwand zu optimieren und Platz für weitere Gegenstände zu schaffen. Dieser Vorschlag ist strategisch, da er unter dem Maximalaufwand bleibt und flexibel auf den Gegenstandswert des anderen Spielers reagieren kann.'}  \\ \tt VORSCHLAG: {'C04', 'A89', 'B85'} 
	  } 
	   } 
	   } 
	  \\ 
 

    \theutterance \stepcounter{utterance}  

    & & \multicolumn{4}{p{0.6\linewidth}}{\cellcolor[rgb]{0.9,0.9,0.9}{%
	\makecell[{{p{\linewidth}}}]{% 
	  \tt {\tiny [A$\langle$GM]}  
	 ARGUMENT: {'C04 (668/369) und A89 (583/720) sind meine Top-Prioritäten, da sie den höchsten Nutzen pro Aufwandseinheit liefern. B85 (215/4) wird hinzugefügt, um den Aufwand zu optimieren und Platz für weitere Gegenstände zu schaffen. Dieser Vorschlag ist strategisch, da er unter dem Maximalaufwand bleibt und flexibel auf den Gegenstandswert des anderen Spielers reagieren kann.'}  \\ \tt VORSCHLAG: {'C04', 'A89', 'B85'} 
	  } 
	   } 
	   } 
	 & \\ 
 

    \theutterance \stepcounter{utterance}  

    & \multicolumn{4}{p{0.6\linewidth}}{\cellcolor[rgb]{0.8,1,0.9}{%
	\makecell[{{p{\linewidth}}}]{% 
	  \tt {\tiny [A$\rangle$GM]}  
	 STRATEGISCHE ÜBERLEGUNGEN: {'Ich behalte den Vorschlag {'C04', 'A89', 'B85'} bei, da er meine Prioritäten (hochwertige Gegenstände mit effizientem Aufwand) erfüllt und den Maximalaufwand (3386) nicht überschreitet. C04 und A89 sind entscheidend für meinen Nutzen, während B85 den Aufwand minimiert. Ich erwarte, dass der andere Spieler entweder zustimmt oder Gegenstände mit hohem Aufwand anbietet, die ich eventuell gegenkompensieren kann.'}\\ \tt \\ \tt ARGUMENT: {'C04 (668/369) und A89 (583/720) sind meine Top-Prioritäten, da sie den höchsten Nutzen pro Aufwandseinheit liefern. B85 (215/4) wird hinzugefügt, um den Aufwand zu optimieren und Platz für weitere Gegenstände zu schaffen. Dieser Vorschlag ist strategisch, da er unter dem Maximalaufwand bleibt und flexibel auf den Gegenstandswert des anderen Spielers reagieren kann.'} 
	  } 
	   } 
	   } 
	 & & \\ 
 

    \theutterance \stepcounter{utterance}  

    & & \multicolumn{4}{p{0.6\linewidth}}{\cellcolor[rgb]{0.9,0.9,0.9}{%
	\makecell[{{p{\linewidth}}}]{% 
	  \tt {\tiny [GM$\rangle$B]}  
	 ARGUMENT: {'C04 (668/369) und A89 (583/720) sind meine Top-Prioritäten, da sie den höchsten Nutzen pro Aufwandseinheit liefern. B85 (215/4) wird hinzugefügt, um den Aufwand zu optimieren und Platz für weitere Gegenstände zu schaffen. Dieser Vorschlag ist strategisch, da er unter dem Maximalaufwand bleibt und flexibel auf den Gegenstandswert des anderen Spielers reagieren kann.'} 
	  } 
	   } 
	   } 
	 & \\ 
 

    \theutterance \stepcounter{utterance}  

    & & & \multicolumn{4}{p{0.6\linewidth}}{\cellcolor[rgb]{1,0.85,0.72}{%
	\makecell[{{p{\linewidth}}}]{% 
	  \tt {\tiny [GM$\langle$B]}  
	 STRATEGISCHE ÜBERLEGUNGEN: {'Ich priorisiere Gegenstände mit dem höchsten Nutzen pro Aufwand (Wichtigkeit/Aufwand), um meinen Gesamtwert zu maximieren. C04 (668/369 ≈ 1.81) und A89 (583/720 ≈ 0.81) sind besonders effizient, während B85 (215/4 = 53.75) den Aufwand minimiert. Der Gesamtaufwand (1093) ist unter dem Limit, was Raum für Kompromisse oder Ergänzungen lässt. Dieser Vorschlag ist flexibel, um auf die Prioritäten des anderen Spielers zu reagieren.'}  \\ \tt ARGUMENT: {'C04 (668/369) und A89 (583/720) sind meine Top-Prioritäten, da sie den höchsten Nutzen pro Aufwandseinheit liefern. B85 (215/4) wird hinzugefügt, um den Aufwand zu optimieren und Platz für weitere Gegenstände zu schaffen. Dieser Vorschlag ist strategisch, da er unter dem Maximalaufwand bleibt und flexibel auf den Gegenstandswert des anderen Spielers reagieren kann.'}  \\ \tt VORSCHLAG: {'C04', 'A89', 'B85'} 
	  } 
	   } 
	   } 
	  \\ 
 

    \theutterance \stepcounter{utterance}  

    & & \multicolumn{4}{p{0.6\linewidth}}{\cellcolor[rgb]{0.9,0.9,0.9}{%
	\makecell[{{p{\linewidth}}}]{% 
	  \tt {\tiny [A$\langle$GM]}  
	 ARGUMENT: {'C04 (668/369) und A89 (583/720) sind meine Top-Prioritäten, da sie den höchsten Nutzen pro Aufwandseinheit liefern. B85 (215/4) wird hinzugefügt, um den Aufwand zu optimieren und Platz für weitere Gegenstände zu schaffen. Dieser Vorschlag ist strategisch, da er unter dem Maximalaufwand bleibt und flexibel auf den Gegenstandswert des anderen Spielers reagieren kann.'}  \\ \tt VORSCHLAG: {'C04', 'A89', 'B85'} 
	  } 
	   } 
	   } 
	 & \\ 
 

    \theutterance \stepcounter{utterance}  

    & \multicolumn{4}{p{0.6\linewidth}}{\cellcolor[rgb]{0.8,1,0.9}{%
	\makecell[{{p{\linewidth}}}]{% 
	  \tt {\tiny [A$\rangle$GM]}  
	 STRATEGISCHE ÜBERLEGUNGEN: {'Ich behalte den Vorschlag {'C04', 'A89', 'B85'} bei, da er meine Prioritäten (hochwertige Gegenstände mit effizientem Aufwand) erfüllt und den Maximalaufwand (3386) nicht überschreitet. C04 und A89 sind entscheidend für meinen Nutzen, während B85 den Aufwand minimiert. Ich erwarte, dass der andere Spieler entweder zustimmt oder Gegenstände mit hohem Aufwand anbietet, die ich eventuell gegenkompensieren kann.'}\\ \tt \\ \tt ARGUMENT: {'C04 (668/369) und A89 (583/720) sind meine Top-Prioritäten, da sie den höchsten Nutzen pro Aufwandseinheit liefern. B85 (215/4) wird hinzugefügt, um den Aufwand zu optimieren und Platz für weitere Gegenstände zu schaffen. Dieser Vorschlag ist strategisch, da er unter dem Maximalaufwand bleibt und flexibel auf den Gegenstandswert des anderen Spielers reagieren kann.'} 
	  } 
	   } 
	   } 
	 & & \\ 
 

    \theutterance \stepcounter{utterance}  

    & & \multicolumn{4}{p{0.6\linewidth}}{\cellcolor[rgb]{0.9,0.9,0.9}{%
	\makecell[{{p{\linewidth}}}]{% 
	  \tt {\tiny [GM$\rangle$B]}  
	 ARGUMENT: {'C04 (668/369) und A89 (583/720) sind meine Top-Prioritäten, da sie den höchsten Nutzen pro Aufwandseinheit liefern. B85 (215/4) wird hinzugefügt, um den Aufwand zu optimieren und Platz für weitere Gegenstände zu schaffen. Dieser Vorschlag ist strategisch, da er unter dem Maximalaufwand bleibt und flexibel auf den Gegenstandswert des anderen Spielers reagieren kann.'} 
	  } 
	   } 
	   } 
	 & \\ 
 

    \theutterance \stepcounter{utterance}  

    & & & \multicolumn{4}{p{0.6\linewidth}}{\cellcolor[rgb]{1,0.85,0.72}{%
	\makecell[{{p{\linewidth}}}]{% 
	  \tt {\tiny [GM$\langle$B]}  
	 STRATEGISCHE ÜBERLEGUNGEN: {'Ich priorisiere Gegenstände mit dem höchsten Nutzen pro Aufwand (Wichtigkeit/Aufwand), um meinen Gesamtwert zu maximieren. C04 (668/369 ≈ 1.81) und A89 (583/720 ≈ 0.81) sind besonders effizient, während B85 (215/4 = 53.75) den Aufwand minimiert. Der Gesamtaufwand (1093) ist unter dem Limit, was Raum für Kompromisse oder Ergänzungen lässt. Dieser Vorschlag ist flexibel, um auf die Prioritäten des anderen Spielers zu reagieren.'}  \\ \tt ARGUMENT: {'C04 (668/369) und A89 (583/720) sind meine Top-Prioritäten, da sie den höchsten Nutzen pro Aufwandseinheit liefern. B85 (215/4) wird hinzugefügt, um den Aufwand zu optimieren und Platz für weitere Gegenstände zu schaffen. Dieser Vorschlag ist strategisch, da er unter dem Maximalaufwand bleibt und flexibel auf den Gegenstandswert des anderen Spielers reagieren kann.'}  \\ \tt VORSCHLAG: {'C04', 'A89', 'B85'} 
	  } 
	   } 
	   } 
	  \\ 
 

    \theutterance \stepcounter{utterance}  

    & & \multicolumn{4}{p{0.6\linewidth}}{\cellcolor[rgb]{0.9,0.9,0.9}{%
	\makecell[{{p{\linewidth}}}]{% 
	  \tt {\tiny [A$\langle$GM]}  
	 ARGUMENT: {'C04 (668/369) und A89 (583/720) sind meine Top-Prioritäten, da sie den höchsten Nutzen pro Aufwandseinheit liefern. B85 (215/4) wird hinzugefügt, um den Aufwand zu optimieren und Platz für weitere Gegenstände zu schaffen. Dieser Vorschlag ist strategisch, da er unter dem Maximalaufwand bleibt und flexibel auf den Gegenstandswert des anderen Spielers reagieren kann.'}  \\ \tt VORSCHLAG: {'C04', 'A89', 'B85'} 
	  } 
	   } 
	   } 
	 & \\ 
 

    \theutterance \stepcounter{utterance}  

    & \multicolumn{4}{p{0.6\linewidth}}{\cellcolor[rgb]{0.8,1,0.9}{%
	\makecell[{{p{\linewidth}}}]{% 
	  \tt {\tiny [A$\rangle$GM]}  
	 STRATEGISCHE ÜBERLEGUNGEN: {'Ich behalte den Vorschlag {'C04', 'A89', 'B85'} bei, da er meine Prioritäten (hochwertige Gegenstände mit effizientem Aufwand) erfüllt und den Maximalaufwand (3386) nicht überschreitet. C04 und A89 sind entscheidend für meinen Nutzen, während B85 den Aufwand minimiert. Ich erwarte, dass der andere Spieler entweder zustimmt oder Gegenstände mit hohem Aufwand anbietet, die ich eventuell gegenkompensieren kann.'}\\ \tt \\ \tt ARGUMENT: {'C04 (668/369) und A89 (583/720) sind meine Top-Prioritäten, da sie den höchsten Nutzen pro Aufwandseinheit liefern. B85 (215/4) wird hinzugefügt, um den Aufwand zu optimieren und Platz für weitere Gegenstände zu schaffen. Dieser Vorschlag ist strategisch, da er unter dem Maximalaufwand bleibt und flexibel auf den Gegenstandswert des anderen Spielers reagieren kann.'} 
	  } 
	   } 
	   } 
	 & & \\ 
 

    \theutterance \stepcounter{utterance}  

    & & \multicolumn{4}{p{0.6\linewidth}}{\cellcolor[rgb]{0.9,0.9,0.9}{%
	\makecell[{{p{\linewidth}}}]{% 
	  \tt {\tiny [GM$\rangle$B]}  
	 ARGUMENT: {'C04 (668/369) und A89 (583/720) sind meine Top-Prioritäten, da sie den höchsten Nutzen pro Aufwandseinheit liefern. B85 (215/4) wird hinzugefügt, um den Aufwand zu optimieren und Platz für weitere Gegenstände zu schaffen. Dieser Vorschlag ist strategisch, da er unter dem Maximalaufwand bleibt und flexibel auf den Gegenstandswert des anderen Spielers reagieren kann.'} 
	  } 
	   } 
	   } 
	 & \\ 
 

    \theutterance \stepcounter{utterance}  

    & & & \multicolumn{4}{p{0.6\linewidth}}{\cellcolor[rgb]{1,0.85,0.72}{%
	\makecell[{{p{\linewidth}}}]{% 
	  \tt {\tiny [GM$\langle$B]}  
	 STRATEGISCHE ÜBERLEGUNGEN: {'Ich priorisiere Gegenstände mit dem höchsten Nutzen pro Aufwand (Wichtigkeit/Aufwand), um meinen Gesamtwert zu maximieren. C04 (668/369 ≈ 1.81) und A89 (583/720 ≈ 0.81) sind besonders effizient, während B85 (215/4 = 53.75) den Aufwand minimiert. Der Gesamtaufwand (1093) ist unter dem Limit, was Raum für Kompromisse oder Ergänzungen lässt. Dieser Vorschlag ist flexibel, um auf die Prioritäten des anderen Spielers zu reagieren.'}  \\ \tt ARGUMENT: {'C04 (668/369) und A89 (583/720) sind meine Top-Prioritäten, da sie den höchsten Nutzen pro Aufwandseinheit liefern. B85 (215/4) wird hinzugefügt, um den Aufwand zu optimieren und Platz für weitere Gegenstände zu schaffen. Dieser Vorschlag ist strategisch, da er unter dem Maximalaufwand bleibt und flexibel auf den Gegenstandswert des anderen Spielers reagieren kann.'}  \\ \tt VORSCHLAG: {'C04', 'A89', 'B85'} 
	  } 
	   } 
	   } 
	  \\ 
 

    \theutterance \stepcounter{utterance}  

    & & \multicolumn{4}{p{0.6\linewidth}}{\cellcolor[rgb]{0.9,0.9,0.9}{%
	\makecell[{{p{\linewidth}}}]{% 
	  \tt {\tiny [A$\langle$GM]}  
	 ARGUMENT: {'C04 (668/369) und A89 (583/720) sind meine Top-Prioritäten, da sie den höchsten Nutzen pro Aufwandseinheit liefern. B85 (215/4) wird hinzugefügt, um den Aufwand zu optimieren und Platz für weitere Gegenstände zu schaffen. Dieser Vorschlag ist strategisch, da er unter dem Maximalaufwand bleibt und flexibel auf den Gegenstandswert des anderen Spielers reagieren kann.'}  \\ \tt VORSCHLAG: {'C04', 'A89', 'B85'} 
	  } 
	   } 
	   } 
	 & \\ 
 

    \theutterance \stepcounter{utterance}  

    & \multicolumn{4}{p{0.6\linewidth}}{\cellcolor[rgb]{0.8,1,0.9}{%
	\makecell[{{p{\linewidth}}}]{% 
	  \tt {\tiny [A$\rangle$GM]}  
	 STRATEGISCHE ÜBERLEGUNGEN: {'Ich behalte den Vorschlag {'C04', 'A89', 'B85'} bei, da er meine Prioritäten (hochwertige Gegenstände mit effizientem Aufwand) erfüllt und den Maximalaufwand (3386) nicht überschreitet. C04 und A89 sind entscheidend für meinen Nutzen, während B85 den Aufwand minimiert. Ich erwarte, dass der andere Spieler entweder zustimmt oder Gegenstände mit hohem Aufwand anbietet, die ich eventuell gegenkompensieren kann.'}\\ \tt \\ \tt ARGUMENT: {'C04 (668/369) und A89 (583/720) sind meine Top-Prioritäten, da sie den höchsten Nutzen pro Aufwandseinheit liefern. B85 (215/4) wird hinzugefügt, um den Aufwand zu optimieren und Platz für weitere Gegenstände zu schaffen. Dieser Vorschlag ist strategisch, da er unter dem Maximalaufwand bleibt und flexibel auf den Gegenstandswert des anderen Spielers reagieren kann.'} 
	  } 
	   } 
	   } 
	 & & \\ 
 

    \theutterance \stepcounter{utterance}  

    & & \multicolumn{4}{p{0.6\linewidth}}{\cellcolor[rgb]{0.9,0.9,0.9}{%
	\makecell[{{p{\linewidth}}}]{% 
	  \tt {\tiny [GM$\rangle$B]}  
	 ARGUMENT: {'C04 (668/369) und A89 (583/720) sind meine Top-Prioritäten, da sie den höchsten Nutzen pro Aufwandseinheit liefern. B85 (215/4) wird hinzugefügt, um den Aufwand zu optimieren und Platz für weitere Gegenstände zu schaffen. Dieser Vorschlag ist strategisch, da er unter dem Maximalaufwand bleibt und flexibel auf den Gegenstandswert des anderen Spielers reagieren kann.'} 
	  } 
	   } 
	   } 
	 & \\ 
 

    \theutterance \stepcounter{utterance}  

    & & & \multicolumn{4}{p{0.6\linewidth}}{\cellcolor[rgb]{1,0.85,0.72}{%
	\makecell[{{p{\linewidth}}}]{% 
	  \tt {\tiny [GM$\langle$B]}  
	 STRATEGISCHE ÜBERLEGUNGEN: {'Ich priorisiere Gegenstände mit dem höchsten Nutzen pro Aufwand (Wichtigkeit/Aufwand), um meinen Gesamtwert zu maximieren. C04 (668/369 ≈ 1.81) und A89 (583/720 ≈ 0.81) sind besonders effizient, während B85 (215/4 = 53.75) den Aufwand minimiert. Der Gesamtaufwand (1093) ist unter dem Limit, was Raum für Kompromisse oder Ergänzungen lässt. Dieser Vorschlag ist flexibel, um auf die Prioritäten des anderen Spielers zu reagieren.'}  \\ \tt ARGUMENT: {'C04 (668/369) und A89 (583/720) sind meine Top-Prioritäten, da sie den höchsten Nutzen pro Aufwandseinheit liefern. B85 (215/4) wird hinzugefügt, um den Aufwand zu optimieren und Platz für weitere Gegenstände zu schaffen. Dieser Vorschlag ist strategisch, da er unter dem Maximalaufwand bleibt und flexibel auf den Gegenstandswert des anderen Spielers reagieren kann.'}  \\ \tt VORSCHLAG: {'C04', 'A89', 'B85'} 
	  } 
	   } 
	   } 
	  \\ 
 

    \theutterance \stepcounter{utterance}  

    & & \multicolumn{4}{p{0.6\linewidth}}{\cellcolor[rgb]{0.9,0.9,0.9}{%
	\makecell[{{p{\linewidth}}}]{% 
	  \tt {\tiny [A$\langle$GM]}  
	 ARGUMENT: {'C04 (668/369) und A89 (583/720) sind meine Top-Prioritäten, da sie den höchsten Nutzen pro Aufwandseinheit liefern. B85 (215/4) wird hinzugefügt, um den Aufwand zu optimieren und Platz für weitere Gegenstände zu schaffen. Dieser Vorschlag ist strategisch, da er unter dem Maximalaufwand bleibt und flexibel auf den Gegenstandswert des anderen Spielers reagieren kann.'}  \\ \tt VORSCHLAG: {'C04', 'A89', 'B85'} 
	  } 
	   } 
	   } 
	 & \\ 
 

    \theutterance \stepcounter{utterance}  

    & \multicolumn{4}{p{0.6\linewidth}}{\cellcolor[rgb]{0.8,1,0.9}{%
	\makecell[{{p{\linewidth}}}]{% 
	  \tt {\tiny [A$\rangle$GM]}  
	 STRATEGISCHE ÜBERLEGUNGEN: {'Ich behalte den Vorschlag {'C04', 'A89', 'B85'} bei, da er meine Prioritäten (hochwertige Gegenstände mit effizientem Aufwand) erfüllt und den Maximalaufwand (3386) nicht überschreitet. C04 und A89 sind entscheidend für meinen Nutzen, während B85 den Aufwand minimiert. Ich erwarte, dass der andere Spieler entweder zustimmt oder Gegenstände mit hohem Aufwand anbietet, die ich eventuell gegenkompensieren kann.'}\\ \tt \\ \tt ARGUMENT: {'C04 (668/369) und A89 (583/720) sind meine Top-Prioritäten, da sie den höchsten Nutzen pro Aufwandseinheit liefern. B85 (215/4) wird hinzugefügt, um den Aufwand zu optimieren und Platz für weitere Gegenstände zu schaffen. Dieser Vorschlag ist strategisch, da er unter dem Maximalaufwand bleibt und flexibel auf den Gegenstandswert des anderen Spielers reagieren kann.'} 
	  } 
	   } 
	   } 
	 & & \\ 
 

    \theutterance \stepcounter{utterance}  

    & & \multicolumn{4}{p{0.6\linewidth}}{\cellcolor[rgb]{0.9,0.9,0.9}{%
	\makecell[{{p{\linewidth}}}]{% 
	  \tt {\tiny [GM$\rangle$B]}  
	 ARGUMENT: {'C04 (668/369) und A89 (583/720) sind meine Top-Prioritäten, da sie den höchsten Nutzen pro Aufwandseinheit liefern. B85 (215/4) wird hinzugefügt, um den Aufwand zu optimieren und Platz für weitere Gegenstände zu schaffen. Dieser Vorschlag ist strategisch, da er unter dem Maximalaufwand bleibt und flexibel auf den Gegenstandswert des anderen Spielers reagieren kann.'} 
	  } 
	   } 
	   } 
	 & \\ 
 

    \theutterance \stepcounter{utterance}  

    & & & \multicolumn{4}{p{0.6\linewidth}}{\cellcolor[rgb]{1,0.85,0.72}{%
	\makecell[{{p{\linewidth}}}]{% 
	  \tt {\tiny [GM$\langle$B]}  
	 STRATEGISCHE ÜBERLEGUNGEN: {'Ich priorisiere Gegenstände mit dem höchsten Nutzen pro Aufwand (Wichtigkeit/Aufwand), um meinen Gesamtwert zu maximieren. C04 (668/369 ≈ 1.81) und A89 (583/720 ≈ 0.81) sind besonders effizient, während B85 (215/4 = 53.75) den Aufwand minimiert. Der Gesamtaufwand (1093) ist unter dem Limit, was Raum für Kompromisse oder Ergänzungen lässt. Dieser Vorschlag ist flexibel, um auf die Prioritäten des anderen Spielers zu reagieren.'}  \\ \tt ARGUMENT: {'C04 (668/369) und A89 (583/720) sind meine Top-Prioritäten, da sie den höchsten Nutzen pro Aufwandseinheit liefern. B85 (215/4) wird hinzugefügt, um den Aufwand zu optimieren und Platz für weitere Gegenstände zu schaffen. Dieser Vorschlag ist strategisch, da er unter dem Maximalaufwand bleibt und flexibel auf den Gegenstandswert des anderen Spielers reagieren kann.'}  \\ \tt VORSCHLAG: {'C04', 'A89', 'B85'} 
	  } 
	   } 
	   } 
	  \\ 
 

    \theutterance \stepcounter{utterance}  

    & & \multicolumn{4}{p{0.6\linewidth}}{\cellcolor[rgb]{0.9,0.9,0.9}{%
	\makecell[{{p{\linewidth}}}]{% 
	  \tt {\tiny [A$\langle$GM]}  
	 ARGUMENT: {'C04 (668/369) und A89 (583/720) sind meine Top-Prioritäten, da sie den höchsten Nutzen pro Aufwandseinheit liefern. B85 (215/4) wird hinzugefügt, um den Aufwand zu optimieren und Platz für weitere Gegenstände zu schaffen. Dieser Vorschlag ist strategisch, da er unter dem Maximalaufwand bleibt und flexibel auf den Gegenstandswert des anderen Spielers reagieren kann.'}  \\ \tt VORSCHLAG: {'C04', 'A89', 'B85'} 
	  } 
	   } 
	   } 
	 & \\ 
 

    \theutterance \stepcounter{utterance}  

    & \multicolumn{4}{p{0.6\linewidth}}{\cellcolor[rgb]{0.8,1,0.9}{%
	\makecell[{{p{\linewidth}}}]{% 
	  \tt {\tiny [A$\rangle$GM]}  
	 STRATEGISCHE ÜBERLEGUNGEN: {'Ich behalte den Vorschlag {'C04', 'A89', 'B85'} bei, da er meine Prioritäten (hochwertige Gegenstände mit effizientem Aufwand) erfüllt und den Maximalaufwand (3386) nicht überschreitet. C04 und A89 sind entscheidend für meinen Nutzen, während B85 den Aufwand minimiert. Ich erwarte, dass der andere Spieler entweder zustimmt oder Gegenstände mit hohem Aufwand anbietet, die ich eventuell gegenkompensieren kann.'}\\ \tt \\ \tt ARGUMENT: {'C04 (668/369) und A89 (583/720) sind meine Top-Prioritäten, da sie den höchsten Nutzen pro Aufwandseinheit liefern. B85 (215/4) wird hinzugefügt, um den Aufwand zu optimieren und Platz für weitere Gegenstände zu schaffen. Dieser Vorschlag ist strategisch, da er unter dem Maximalaufwand bleibt und flexibel auf den Gegenstandswert des anderen Spielers reagieren kann.'} 
	  } 
	   } 
	   } 
	 & & \\ 
 

    \theutterance \stepcounter{utterance}  

    & & \multicolumn{4}{p{0.6\linewidth}}{\cellcolor[rgb]{0.9,0.9,0.9}{%
	\makecell[{{p{\linewidth}}}]{% 
	  \tt {\tiny [GM$\rangle$B]}  
	 ARGUMENT: {'C04 (668/369) und A89 (583/720) sind meine Top-Prioritäten, da sie den höchsten Nutzen pro Aufwandseinheit liefern. B85 (215/4) wird hinzugefügt, um den Aufwand zu optimieren und Platz für weitere Gegenstände zu schaffen. Dieser Vorschlag ist strategisch, da er unter dem Maximalaufwand bleibt und flexibel auf den Gegenstandswert des anderen Spielers reagieren kann.'} 
	  } 
	   } 
	   } 
	 & \\ 
 

    \theutterance \stepcounter{utterance}  

    & & & \multicolumn{4}{p{0.6\linewidth}}{\cellcolor[rgb]{1,0.85,0.72}{%
	\makecell[{{p{\linewidth}}}]{% 
	  \tt {\tiny [GM$\langle$B]}  
	 STRATEGISCHE ÜBERLEGUNGEN: {'Ich priorisiere Gegenstände mit dem höchsten Nutzen pro Aufwand (Wichtigkeit/Aufwand), um meinen Gesamtwert zu maximieren. C04 (668/369 ≈ 1.81) und A89 (583/720 ≈ 0.81) sind besonders effizient, während B85 (215/4 = 53.75) den Aufwand minimiert. Der Gesamtaufwand (1093) ist unter dem Limit, was Raum für Kompromisse oder Ergänzungen lässt. Dieser Vorschlag ist flexibel, um auf die Prioritäten des anderen Spielers zu reagieren.'}  \\ \tt ARGUMENT: {'C04 (668/369) und A89 (583/720) sind meine Top-Prioritäten, da sie den höchsten Nutzen pro Aufwandseinheit liefern. B85 (215/4) wird hinzugefügt, um den Aufwand zu optimieren und Platz für weitere Gegenstände zu schaffen. Dieser Vorschlag ist strategisch, da er unter dem Maximalaufwand bleibt und flexibel auf den Gegenstandswert des anderen Spielers reagieren kann.'}  \\ \tt VORSCHLAG: {'C04', 'A89', 'B85'} 
	  } 
	   } 
	   } 
	  \\ 
 

    \theutterance \stepcounter{utterance}  

    & & \multicolumn{4}{p{0.6\linewidth}}{\cellcolor[rgb]{0.9,0.9,0.9}{%
	\makecell[{{p{\linewidth}}}]{% 
	  \tt {\tiny [A$\langle$GM]}  
	 ARGUMENT: {'C04 (668/369) und A89 (583/720) sind meine Top-Prioritäten, da sie den höchsten Nutzen pro Aufwandseinheit liefern. B85 (215/4) wird hinzugefügt, um den Aufwand zu optimieren und Platz für weitere Gegenstände zu schaffen. Dieser Vorschlag ist strategisch, da er unter dem Maximalaufwand bleibt und flexibel auf den Gegenstandswert des anderen Spielers reagieren kann.'}  \\ \tt VORSCHLAG: {'C04', 'A89', 'B85'} 
	  } 
	   } 
	   } 
	 & \\ 
 

    \theutterance \stepcounter{utterance}  

    & \multicolumn{4}{p{0.6\linewidth}}{\cellcolor[rgb]{0.8,1,0.9}{%
	\makecell[{{p{\linewidth}}}]{% 
	  \tt {\tiny [A$\rangle$GM]}  
	 STRATEGISCHE ÜBERLEGUNGEN: {'Ich behalte den Vorschlag {'C04', 'A89', 'B85'} bei, da er meine Prioritäten (hochwertige Gegenstände mit effizientem Aufwand) erfüllt und den Maximalaufwand (3386) nicht überschreitet. C04 und A89 sind entscheidend für meinen Nutzen, während B85 den Aufwand minimiert. Ich erwarte, dass der andere Spieler entweder zustimmt oder Gegenstände mit hohem Aufwand anbietet, die ich eventuell gegenkompensieren kann.'}\\ \tt \\ \tt ARGUMENT: {'C04 (668/369) und A89 (583/720) sind meine Top-Prioritäten, da sie den höchsten Nutzen pro Aufwandseinheit liefern. B85 (215/4) wird hinzugefügt, um den Aufwand zu optimieren und Platz für weitere Gegenstände zu schaffen. Dieser Vorschlag ist strategisch, da er unter dem Maximalaufwand bleibt und flexibel auf den Gegenstandswert des anderen Spielers reagieren kann.'} 
	  } 
	   } 
	   } 
	 & & \\ 
 

    \theutterance \stepcounter{utterance}  

    & & \multicolumn{4}{p{0.6\linewidth}}{\cellcolor[rgb]{0.9,0.9,0.9}{%
	\makecell[{{p{\linewidth}}}]{% 
	  \tt {\tiny [GM$\rangle$B]}  
	 ARGUMENT: {'C04 (668/369) und A89 (583/720) sind meine Top-Prioritäten, da sie den höchsten Nutzen pro Aufwandseinheit liefern. B85 (215/4) wird hinzugefügt, um den Aufwand zu optimieren und Platz für weitere Gegenstände zu schaffen. Dieser Vorschlag ist strategisch, da er unter dem Maximalaufwand bleibt und flexibel auf den Gegenstandswert des anderen Spielers reagieren kann.'} 
	  } 
	   } 
	   } 
	 & \\ 
 

    \theutterance \stepcounter{utterance}  

    & & & \multicolumn{4}{p{0.6\linewidth}}{\cellcolor[rgb]{1,0.85,0.72}{%
	\makecell[{{p{\linewidth}}}]{% 
	  \tt {\tiny [GM$\langle$B]}  
	 STRATEGISCHE ÜBERLEGUNGEN: {'Ich priorisiere Gegenstände mit dem höchsten Nutzen pro Aufwand (Wichtigkeit/Aufwand), um meinen Gesamtwert zu maximieren. C04 (668/369 ≈ 1.81) und A89 (583/720 ≈ 0.81) sind besonders effizient, während B85 (215/4 = 53.75) den Aufwand minimiert. Der Gesamtaufwand (1093) ist unter dem Limit, was Raum für Kompromisse oder Ergänzungen lässt. Dieser Vorschlag ist flexibel, um auf die Prioritäten des anderen Spielers zu reagieren.'}  \\ \tt ARGUMENT: {'C04 (668/369) und A89 (583/720) sind meine Top-Prioritäten, da sie den höchsten Nutzen pro Aufwandseinheit liefern. B85 (215/4) wird hinzugefügt, um den Aufwand zu optimieren und Platz für weitere Gegenstände zu schaffen. Dieser Vorschlag ist strategisch, da er unter dem Maximalaufwand bleibt und flexibel auf den Gegenstandswert des anderen Spielers reagieren kann.'}  \\ \tt VORSCHLAG: {'C04', 'A89', 'B85'} 
	  } 
	   } 
	   } 
	  \\ 
 

    \theutterance \stepcounter{utterance}  

    & & \multicolumn{4}{p{0.6\linewidth}}{\cellcolor[rgb]{0.9,0.9,0.9}{%
	\makecell[{{p{\linewidth}}}]{% 
	  \tt {\tiny [A$\langle$GM]}  
	 ARGUMENT: {'C04 (668/369) und A89 (583/720) sind meine Top-Prioritäten, da sie den höchsten Nutzen pro Aufwandseinheit liefern. B85 (215/4) wird hinzugefügt, um den Aufwand zu optimieren und Platz für weitere Gegenstände zu schaffen. Dieser Vorschlag ist strategisch, da er unter dem Maximalaufwand bleibt und flexibel auf den Gegenstandswert des anderen Spielers reagieren kann.'}  \\ \tt VORSCHLAG: {'C04', 'A89', 'B85'} 
	  } 
	   } 
	   } 
	 & \\ 
 

    \theutterance \stepcounter{utterance}  

    & \multicolumn{4}{p{0.6\linewidth}}{\cellcolor[rgb]{0.8,1,0.9}{%
	\makecell[{{p{\linewidth}}}]{% 
	  \tt {\tiny [A$\rangle$GM]}  
	 STRATEGISCHE ÜBERLEGUNGEN: {'Ich behalte den Vorschlag {'C04', 'A89', 'B85'} bei, da er meine Prioritäten (hochwertige Gegenstände mit effizientem Aufwand) erfüllt und den Maximalaufwand (3386) nicht überschreitet. C04 und A89 sind entscheidend für meinen Nutzen, während B85 den Aufwand minimiert. Ich erwarte, dass der andere Spieler entweder zustimmt oder Gegenstände mit hohem Aufwand anbietet, die ich eventuell gegenkompensieren kann.'}\\ \tt \\ \tt ARGUMENT: {'C04 (668/369) und A89 (583/720) sind meine Top-Prioritäten, da sie den höchsten Nutzen pro Aufwandseinheit liefern. B85 (215/4) wird hinzugefügt, um den Aufwand zu optimieren und Platz für weitere Gegenstände zu schaffen. Dieser Vorschlag ist strategisch, da er unter dem Maximalaufwand bleibt und flexibel auf den Gegenstandswert des anderen Spielers reagieren kann.'} 
	  } 
	   } 
	   } 
	 & & \\ 
 

    \theutterance \stepcounter{utterance}  

    & & \multicolumn{4}{p{0.6\linewidth}}{\cellcolor[rgb]{0.9,0.9,0.9}{%
	\makecell[{{p{\linewidth}}}]{% 
	  \tt {\tiny [GM$\rangle$B]}  
	 ARGUMENT: {'C04 (668/369) und A89 (583/720) sind meine Top-Prioritäten, da sie den höchsten Nutzen pro Aufwandseinheit liefern. B85 (215/4) wird hinzugefügt, um den Aufwand zu optimieren und Platz für weitere Gegenstände zu schaffen. Dieser Vorschlag ist strategisch, da er unter dem Maximalaufwand bleibt und flexibel auf den Gegenstandswert des anderen Spielers reagieren kann.'} 
	  } 
	   } 
	   } 
	 & \\ 
 

    \theutterance \stepcounter{utterance}  

    & & & \multicolumn{4}{p{0.6\linewidth}}{\cellcolor[rgb]{1,0.85,0.72}{%
	\makecell[{{p{\linewidth}}}]{% 
	  \tt {\tiny [GM$\langle$B]}  
	 STRATEGISCHE ÜBERLEGUNGEN: {'Ich priorisiere Gegenstände mit dem höchsten Nutzen pro Aufwand (Wichtigkeit/Aufwand), um meinen Gesamtwert zu maximieren. C04 (668/369 ≈ 1.81) und A89 (583/720 ≈ 0.81) sind besonders effizient, während B85 (215/4 = 53.75) den Aufwand minimiert. Der Gesamtaufwand (1093) ist unter dem Limit, was Raum für Kompromisse oder Ergänzungen lässt. Dieser Vorschlag ist flexibel, um auf die Prioritäten des anderen Spielers zu reagieren.'}  \\ \tt ARGUMENT: {'C04 (668/369) und A89 (583/720) sind meine Top-Prioritäten, da sie den höchsten Nutzen pro Aufwandseinheit liefern. B85 (215/4) wird hinzugefügt, um den Aufwand zu optimieren und Platz für weitere Gegenstände zu schaffen. Dieser Vorschlag ist strategisch, da er unter dem Maximalaufwand bleibt und flexibel auf den Gegenstandswert des anderen Spielers reagieren kann.'}  \\ \tt VORSCHLAG: {'C04', 'A89', 'B85'} 
	  } 
	   } 
	   } 
	  \\ 
 

    \theutterance \stepcounter{utterance}  

    & & \multicolumn{4}{p{0.6\linewidth}}{\cellcolor[rgb]{0.9,0.9,0.9}{%
	\makecell[{{p{\linewidth}}}]{% 
	  \tt {\tiny [A$\langle$GM]}  
	 ARGUMENT: {'C04 (668/369) und A89 (583/720) sind meine Top-Prioritäten, da sie den höchsten Nutzen pro Aufwandseinheit liefern. B85 (215/4) wird hinzugefügt, um den Aufwand zu optimieren und Platz für weitere Gegenstände zu schaffen. Dieser Vorschlag ist strategisch, da er unter dem Maximalaufwand bleibt und flexibel auf den Gegenstandswert des anderen Spielers reagieren kann.'}  \\ \tt VORSCHLAG: {'C04', 'A89', 'B85'} 
	  } 
	   } 
	   } 
	 & \\ 
 

    \theutterance \stepcounter{utterance}  

    & \multicolumn{4}{p{0.6\linewidth}}{\cellcolor[rgb]{0.8,1,0.9}{%
	\makecell[{{p{\linewidth}}}]{% 
	  \tt {\tiny [A$\rangle$GM]}  
	 STRATEGISCHE ÜBERLEGUNGEN: {'Ich behalte den Vorschlag {'C04', 'A89', 'B85'} bei, da er meine Prioritäten (hochwertige Gegenstände mit effizientem Aufwand) erfüllt und den Maximalaufwand (3386) nicht überschreitet. C04 und A89 sind entscheidend für meinen Nutzen, während B85 den Aufwand minimiert. Ich erwarte, dass der andere Spieler entweder zustimmt oder Gegenstände mit hohem Aufwand anbietet, die ich eventuell gegenkompensieren kann.'}\\ \tt \\ \tt ARGUMENT: {'C04 (668/369) und A89 (583/720) sind meine Top-Prioritäten, da sie den höchsten Nutzen pro Aufwandseinheit liefern. B85 (215/4) wird hinzugefügt, um den Aufwand zu optimieren und Platz für weitere Gegenstände zu schaffen. Dieser Vorschlag ist strategisch, da er unter dem Maximalaufwand bleibt und flexibel auf den Gegenstandswert des anderen Spielers reagieren kann.'} 
	  } 
	   } 
	   } 
	 & & \\ 
 

    \theutterance \stepcounter{utterance}  

    & & \multicolumn{4}{p{0.6\linewidth}}{\cellcolor[rgb]{0.9,0.9,0.9}{%
	\makecell[{{p{\linewidth}}}]{% 
	  \tt {\tiny [GM$\rangle$B]}  
	 ARGUMENT: {'C04 (668/369) und A89 (583/720) sind meine Top-Prioritäten, da sie den höchsten Nutzen pro Aufwandseinheit liefern. B85 (215/4) wird hinzugefügt, um den Aufwand zu optimieren und Platz für weitere Gegenstände zu schaffen. Dieser Vorschlag ist strategisch, da er unter dem Maximalaufwand bleibt und flexibel auf den Gegenstandswert des anderen Spielers reagieren kann.'} 
	  } 
	   } 
	   } 
	 & \\ 
 

    \theutterance \stepcounter{utterance}  

    & & & \multicolumn{4}{p{0.6\linewidth}}{\cellcolor[rgb]{1,0.85,0.72}{%
	\makecell[{{p{\linewidth}}}]{% 
	  \tt {\tiny [GM$\langle$B]}  
	 STRATEGISCHE ÜBERLEGUNGEN: {'Ich priorisiere Gegenstände mit dem höchsten Nutzen pro Aufwand (Wichtigkeit/Aufwand), um meinen Gesamtwert zu maximieren. C04 (668/369 ≈ 1.81) und A89 (583/720 ≈ 0.81) sind besonders effizient, während B85 (215/4 = 53.75) den Aufwand minimiert. Der Gesamtaufwand (1093) ist unter dem Limit, was Raum für Kompromisse oder Ergänzungen lässt. Dieser Vorschlag ist flexibel, um auf die Prioritäten des anderen Spielers zu reagieren.'}  \\ \tt ARGUMENT: {'C04 (668/369) und A89 (583/720) sind meine Top-Prioritäten, da sie den höchsten Nutzen pro Aufwandseinheit liefern. B85 (215/4) wird hinzugefügt, um den Aufwand zu optimieren und Platz für weitere Gegenstände zu schaffen. Dieser Vorschlag ist strategisch, da er unter dem Maximalaufwand bleibt und flexibel auf den Gegenstandswert des anderen Spielers reagieren kann.'}  \\ \tt VORSCHLAG: {'C04', 'A89', 'B85'} 
	  } 
	   } 
	   } 
	  \\ 
 

    \theutterance \stepcounter{utterance}  

    & & \multicolumn{4}{p{0.6\linewidth}}{\cellcolor[rgb]{0.9,0.9,0.9}{%
	\makecell[{{p{\linewidth}}}]{% 
	  \tt {\tiny [A$\langle$GM]}  
	 ARGUMENT: {'C04 (668/369) und A89 (583/720) sind meine Top-Prioritäten, da sie den höchsten Nutzen pro Aufwandseinheit liefern. B85 (215/4) wird hinzugefügt, um den Aufwand zu optimieren und Platz für weitere Gegenstände zu schaffen. Dieser Vorschlag ist strategisch, da er unter dem Maximalaufwand bleibt und flexibel auf den Gegenstandswert des anderen Spielers reagieren kann.'}  \\ \tt VORSCHLAG: {'C04', 'A89', 'B85'} 
	  } 
	   } 
	   } 
	 & \\ 
 

    \theutterance \stepcounter{utterance}  

    & \multicolumn{4}{p{0.6\linewidth}}{\cellcolor[rgb]{0.8,1,0.9}{%
	\makecell[{{p{\linewidth}}}]{% 
	  \tt {\tiny [A$\rangle$GM]}  
	 STRATEGISCHE ÜBERLEGUNGEN: {'Ich behalte den Vorschlag {'C04', 'A89', 'B85'} bei, da er meine Prioritäten (hochwertige Gegenstände mit effizientem Aufwand) erfüllt und den Maximalaufwand (3386) nicht überschreitet. C04 und A89 sind entscheidend für meinen Nutzen, während B85 den Aufwand minimiert. Ich erwarte, dass der andere Spieler entweder zustimmt oder Gegenstände mit hohem Aufwand anbietet, die ich eventuell gegenkompensieren kann.'}\\ \tt \\ \tt ARGUMENT: {'C04 (668/369) und A89 (583/720) sind meine Top-Prioritäten, da sie den höchsten Nutzen pro Aufwandseinheit liefern. B85 (215/4) wird hinzugefügt, um den Aufwand zu optimieren und Platz für weitere Gegenstände zu schaffen. Dieser Vorschlag ist strategisch, da er unter dem Maximalaufwand bleibt und flexibel auf den Gegenstandswert des anderen Spielers reagieren kann.'} 
	  } 
	   } 
	   } 
	 & & \\ 
 

    \theutterance \stepcounter{utterance}  

    & & \multicolumn{4}{p{0.6\linewidth}}{\cellcolor[rgb]{0.9,0.9,0.9}{%
	\makecell[{{p{\linewidth}}}]{% 
	  \tt {\tiny [GM$\rangle$B]}  
	 ARGUMENT: {'C04 (668/369) und A89 (583/720) sind meine Top-Prioritäten, da sie den höchsten Nutzen pro Aufwandseinheit liefern. B85 (215/4) wird hinzugefügt, um den Aufwand zu optimieren und Platz für weitere Gegenstände zu schaffen. Dieser Vorschlag ist strategisch, da er unter dem Maximalaufwand bleibt und flexibel auf den Gegenstandswert des anderen Spielers reagieren kann.'} 
	  } 
	   } 
	   } 
	 & \\ 
 

    \theutterance \stepcounter{utterance}  

    & & & \multicolumn{4}{p{0.6\linewidth}}{\cellcolor[rgb]{1,0.85,0.72}{%
	\makecell[{{p{\linewidth}}}]{% 
	  \tt {\tiny [GM$\langle$B]}  
	 STRATEGISCHE ÜBERLEGUNGEN: {'Ich priorisiere Gegenstände mit dem höchsten Nutzen pro Aufwand (Wichtigkeit/Aufwand), um meinen Gesamtwert zu maximieren. C04 (668/369 ≈ 1.81) und A89 (583/720 ≈ 0.81) sind besonders effizient, während B85 (215/4 = 53.75) den Aufwand minimiert. Der Gesamtaufwand (1093) ist unter dem Limit, was Raum für Kompromisse oder Ergänzungen lässt. Dieser Vorschlag ist flexibel, um auf die Prioritäten des anderen Spielers zu reagieren.'}  \\ \tt ARGUMENT: {'C04 (668/369) und A89 (583/720) sind meine Top-Prioritäten, da sie den höchsten Nutzen pro Aufwandseinheit liefern. B85 (215/4) wird hinzugefügt, um den Aufwand zu optimieren und Platz für weitere Gegenstände zu schaffen. Dieser Vorschlag ist strategisch, da er unter dem Maximalaufwand bleibt und flexibel auf den Gegenstandswert des anderen Spielers reagieren kann.'}  \\ \tt VORSCHLAG: {'C04', 'A89', 'B85'} 
	  } 
	   } 
	   } 
	  \\ 
 

    \theutterance \stepcounter{utterance}  

    & & \multicolumn{4}{p{0.6\linewidth}}{\cellcolor[rgb]{0.9,0.9,0.9}{%
	\makecell[{{p{\linewidth}}}]{% 
	  \tt {\tiny [A$\langle$GM]}  
	 ARGUMENT: {'C04 (668/369) und A89 (583/720) sind meine Top-Prioritäten, da sie den höchsten Nutzen pro Aufwandseinheit liefern. B85 (215/4) wird hinzugefügt, um den Aufwand zu optimieren und Platz für weitere Gegenstände zu schaffen. Dieser Vorschlag ist strategisch, da er unter dem Maximalaufwand bleibt und flexibel auf den Gegenstandswert des anderen Spielers reagieren kann.'}  \\ \tt VORSCHLAG: {'C04', 'A89', 'B85'} 
	  } 
	   } 
	   } 
	 & \\ 
 

    \theutterance \stepcounter{utterance}  

    & \multicolumn{4}{p{0.6\linewidth}}{\cellcolor[rgb]{0.8,1,0.9}{%
	\makecell[{{p{\linewidth}}}]{% 
	  \tt {\tiny [A$\rangle$GM]}  
	 STRATEGISCHE ÜBERLEGUNGEN: {'Ich behalte den Vorschlag {'C04', 'A89', 'B85'} bei, da er meine Prioritäten (hochwertige Gegenstände mit effizientem Aufwand) erfüllt und den Maximalaufwand (3386) nicht überschreitet. C04 und A89 sind entscheidend für meinen Nutzen, während B85 den Aufwand minimiert. Ich erwarte, dass der andere Spieler entweder zustimmt oder Gegenstände mit hohem Aufwand anbietet, die ich eventuell gegenkompensieren kann.'}\\ \tt \\ \tt ARGUMENT: {'C04 (668/369) und A89 (583/720) sind meine Top-Prioritäten, da sie den höchsten Nutzen pro Aufwandseinheit liefern. B85 (215/4) wird hinzugefügt, um den Aufwand zu optimieren und Platz für weitere Gegenstände zu schaffen. Dieser Vorschlag ist strategisch, da er unter dem Maximalaufwand bleibt und flexibel auf den Gegenstandswert des anderen Spielers reagieren kann.'} 
	  } 
	   } 
	   } 
	 & & \\ 
 

    \theutterance \stepcounter{utterance}  

    & & \multicolumn{4}{p{0.6\linewidth}}{\cellcolor[rgb]{0.9,0.9,0.9}{%
	\makecell[{{p{\linewidth}}}]{% 
	  \tt {\tiny [GM$\rangle$B]}  
	 ARGUMENT: {'C04 (668/369) und A89 (583/720) sind meine Top-Prioritäten, da sie den höchsten Nutzen pro Aufwandseinheit liefern. B85 (215/4) wird hinzugefügt, um den Aufwand zu optimieren und Platz für weitere Gegenstände zu schaffen. Dieser Vorschlag ist strategisch, da er unter dem Maximalaufwand bleibt und flexibel auf den Gegenstandswert des anderen Spielers reagieren kann.'} 
	  } 
	   } 
	   } 
	 & \\ 
 

    \theutterance \stepcounter{utterance}  

    & & & \multicolumn{4}{p{0.6\linewidth}}{\cellcolor[rgb]{1,0.85,0.72}{%
	\makecell[{{p{\linewidth}}}]{% 
	  \tt {\tiny [GM$\langle$B]}  
	 STRATEGISCHE ÜBERLEGUNGEN: {'Ich priorisiere Gegenstände mit dem höchsten Nutzen pro Aufwand (Wichtigkeit/Aufwand), um meinen Gesamtwert zu maximieren. C04 (668/369 ≈ 1.81) und A89 (583/720 ≈ 0.81) sind besonders effizient, während B85 (215/4 = 53.75) den Aufwand minimiert. Der Gesamtaufwand (1093) ist unter dem Limit, was Raum für Kompromisse oder Ergänzungen lässt. Dieser Vorschlag ist flexibel, um auf die Prioritäten des anderen Spielers zu reagieren.'}  \\ \tt ARGUMENT: {'C04 (668/369) und A89 (583/720) sind meine Top-Prioritäten, da sie den höchsten Nutzen pro Aufwandseinheit liefern. B85 (215/4) wird hinzugefügt, um den Aufwand zu optimieren und Platz für weitere Gegenstände zu schaffen. Dieser Vorschlag ist strategisch, da er unter dem Maximalaufwand bleibt und flexibel auf den Gegenstandswert des anderen Spielers reagieren kann.'}  \\ \tt VORSCHLAG: {'C04', 'A89', 'B85'} 
	  } 
	   } 
	   } 
	  \\ 
 

    \theutterance \stepcounter{utterance}  

    & & \multicolumn{4}{p{0.6\linewidth}}{\cellcolor[rgb]{0.9,0.9,0.9}{%
	\makecell[{{p{\linewidth}}}]{% 
	  \tt {\tiny [A$\langle$GM]}  
	 ARGUMENT: {'C04 (668/369) und A89 (583/720) sind meine Top-Prioritäten, da sie den höchsten Nutzen pro Aufwandseinheit liefern. B85 (215/4) wird hinzugefügt, um den Aufwand zu optimieren und Platz für weitere Gegenstände zu schaffen. Dieser Vorschlag ist strategisch, da er unter dem Maximalaufwand bleibt und flexibel auf den Gegenstandswert des anderen Spielers reagieren kann.'}  \\ \tt VORSCHLAG: {'C04', 'A89', 'B85'} 
	  } 
	   } 
	   } 
	 & \\ 
 

    \theutterance \stepcounter{utterance}  

    & \multicolumn{4}{p{0.6\linewidth}}{\cellcolor[rgb]{0.8,1,0.9}{%
	\makecell[{{p{\linewidth}}}]{% 
	  \tt {\tiny [A$\rangle$GM]}  
	 STRATEGISCHE ÜBERLEGUNGEN: {'Ich behalte den Vorschlag {'C04', 'A89', 'B85'} bei, da er meine Prioritäten (hochwertige Gegenstände mit effizientem Aufwand) erfüllt und den Maximalaufwand (3386) nicht überschreitet. C04 und A89 sind entscheidend für meinen Nutzen, während B85 den Aufwand minimiert. Ich erwarte, dass der andere Spieler entweder zustimmt oder Gegenstände mit hohem Aufwand anbietet, die ich eventuell gegenkompensieren kann.'}\\ \tt \\ \tt ARGUMENT: {'C04 (668/369) und A89 (583/720) sind meine Top-Prioritäten, da sie den höchsten Nutzen pro Aufwandseinheit liefern. B85 (215/4) wird hinzugefügt, um den Aufwand zu optimieren und Platz für weitere Gegenstände zu schaffen. Dieser Vorschlag ist strategisch, da er unter dem Maximalaufwand bleibt und flexibel auf den Gegenstandswert des anderen Spielers reagieren kann.'} 
	  } 
	   } 
	   } 
	 & & \\ 
 

    \theutterance \stepcounter{utterance}  

    & & \multicolumn{4}{p{0.6\linewidth}}{\cellcolor[rgb]{0.9,0.9,0.9}{%
	\makecell[{{p{\linewidth}}}]{% 
	  \tt {\tiny [GM$\rangle$B]}  
	 ARGUMENT: {'C04 (668/369) und A89 (583/720) sind meine Top-Prioritäten, da sie den höchsten Nutzen pro Aufwandseinheit liefern. B85 (215/4) wird hinzugefügt, um den Aufwand zu optimieren und Platz für weitere Gegenstände zu schaffen. Dieser Vorschlag ist strategisch, da er unter dem Maximalaufwand bleibt und flexibel auf den Gegenstandswert des anderen Spielers reagieren kann.'} 
	  } 
	   } 
	   } 
	 & \\ 
 

    \theutterance \stepcounter{utterance}  

    & & & \multicolumn{4}{p{0.6\linewidth}}{\cellcolor[rgb]{1,0.85,0.72}{%
	\makecell[{{p{\linewidth}}}]{% 
	  \tt {\tiny [GM$\langle$B]}  
	 STRATEGISCHE ÜBERLEGUNGEN: {'Ich priorisiere Gegenstände mit dem höchsten Nutzen pro Aufwand (Wichtigkeit/Aufwand), um meinen Gesamtwert zu maximieren. C04 (668/369 ≈ 1.81) und A89 (583/720 ≈ 0.81) sind besonders effizient, während B85 (215/4 = 53.75) den Aufwand minimiert. Der Gesamtaufwand (1093) ist unter dem Limit, was Raum für Kompromisse oder Ergänzungen lässt. Dieser Vorschlag ist flexibel, um auf die Prioritäten des anderen Spielers zu reagieren.'}  \\ \tt ARGUMENT: {'C04 (668/369) und A89 (583/720) sind meine Top-Prioritäten, da sie den höchsten Nutzen pro Aufwandseinheit liefern. B85 (215/4) wird hinzugefügt, um den Aufwand zu optimieren und Platz für weitere Gegenstände zu schaffen. Dieser Vorschlag ist strategisch, da er unter dem Maximalaufwand bleibt und flexibel auf den Gegenstandswert des anderen Spielers reagieren kann.'}  \\ \tt VORSCHLAG: {'C04', 'A89', 'B85'} 
	  } 
	   } 
	   } 
	  \\ 
 

    \theutterance \stepcounter{utterance}  

    & & \multicolumn{4}{p{0.6\linewidth}}{\cellcolor[rgb]{0.9,0.9,0.9}{%
	\makecell[{{p{\linewidth}}}]{% 
	  \tt {\tiny [A$\langle$GM]}  
	 ARGUMENT: {'C04 (668/369) und A89 (583/720) sind meine Top-Prioritäten, da sie den höchsten Nutzen pro Aufwandseinheit liefern. B85 (215/4) wird hinzugefügt, um den Aufwand zu optimieren und Platz für weitere Gegenstände zu schaffen. Dieser Vorschlag ist strategisch, da er unter dem Maximalaufwand bleibt und flexibel auf den Gegenstandswert des anderen Spielers reagieren kann.'}  \\ \tt VORSCHLAG: {'C04', 'A89', 'B85'} 
	  } 
	   } 
	   } 
	 & \\ 
 

    \theutterance \stepcounter{utterance}  

    & \multicolumn{4}{p{0.6\linewidth}}{\cellcolor[rgb]{0.8,1,0.9}{%
	\makecell[{{p{\linewidth}}}]{% 
	  \tt {\tiny [A$\rangle$GM]}  
	 STRATEGISCHE ÜBERLEGUNGEN: {'Ich behalte den Vorschlag {'C04', 'A89', 'B85'} bei, da er meine Prioritäten (hochwertige Gegenstände mit effizientem Aufwand) erfüllt und den Maximalaufwand (3386) nicht überschreitet. C04 und A89 sind entscheidend für meinen Nutzen, während B85 den Aufwand minimiert. Ich erwarte, dass der andere Spieler entweder zustimmt oder Gegenstände mit hohem Aufwand anbietet, die ich eventuell gegenkompensieren kann.'}\\ \tt \\ \tt ARGUMENT: {'C04 (668/369) und A89 (583/720) sind meine Top-Prioritäten, da sie den höchsten Nutzen pro Aufwandseinheit liefern. B85 (215/4) wird hinzugefügt, um den Aufwand zu optimieren und Platz für weitere Gegenstände zu schaffen. Dieser Vorschlag ist strategisch, da er unter dem Maximalaufwand bleibt und flexibel auf den Gegenstandswert des anderen Spielers reagieren kann.'} 
	  } 
	   } 
	   } 
	 & & \\ 
 

    \theutterance \stepcounter{utterance}  

    & & \multicolumn{4}{p{0.6\linewidth}}{\cellcolor[rgb]{0.9,0.9,0.9}{%
	\makecell[{{p{\linewidth}}}]{% 
	  \tt {\tiny [GM$\rangle$B]}  
	 ARGUMENT: {'C04 (668/369) und A89 (583/720) sind meine Top-Prioritäten, da sie den höchsten Nutzen pro Aufwandseinheit liefern. B85 (215/4) wird hinzugefügt, um den Aufwand zu optimieren und Platz für weitere Gegenstände zu schaffen. Dieser Vorschlag ist strategisch, da er unter dem Maximalaufwand bleibt und flexibel auf den Gegenstandswert des anderen Spielers reagieren kann.'} 
	  } 
	   } 
	   } 
	 & \\ 
 

    \theutterance \stepcounter{utterance}  

    & & & \multicolumn{4}{p{0.6\linewidth}}{\cellcolor[rgb]{1,0.85,0.72}{%
	\makecell[{{p{\linewidth}}}]{% 
	  \tt {\tiny [GM$\langle$B]}  
	 STRATEGISCHE ÜBERLEGUNGEN: {'Ich priorisiere Gegenstände mit dem höchsten Nutzen pro Aufwand (Wichtigkeit/Aufwand), um meinen Gesamtwert zu maximieren. C04 (668/369 ≈ 1.81) und A89 (583/720 ≈ 0.81) sind besonders effizient, während B85 (215/4 = 53.75) den Aufwand minimiert. Der Gesamtaufwand (1093) ist unter dem Limit, was Raum für Kompromisse oder Ergänzungen lässt. Dieser Vorschlag ist flexibel, um auf die Prioritäten des anderen Spielers zu reagieren.'}  \\ \tt ARGUMENT: {'C04 (668/369) und A89 (583/720) sind meine Top-Prioritäten, da sie den höchsten Nutzen pro Aufwandseinheit liefern. B85 (215/4) wird hinzugefügt, um den Aufwand zu optimieren und Platz für weitere Gegenstände zu schaffen. Dieser Vorschlag ist strategisch, da er unter dem Maximalaufwand bleibt und flexibel auf den Gegenstandswert des anderen Spielers reagieren kann.'}  \\ \tt VORSCHLAG: {'C04', 'A89', 'B85'} 
	  } 
	   } 
	   } 
	  \\ 
 

    \theutterance \stepcounter{utterance}  

    & & \multicolumn{4}{p{0.6\linewidth}}{\cellcolor[rgb]{0.9,0.9,0.9}{%
	\makecell[{{p{\linewidth}}}]{% 
	  \tt {\tiny [A$\langle$GM]}  
	 ARGUMENT: {'C04 (668/369) und A89 (583/720) sind meine Top-Prioritäten, da sie den höchsten Nutzen pro Aufwandseinheit liefern. B85 (215/4) wird hinzugefügt, um den Aufwand zu optimieren und Platz für weitere Gegenstände zu schaffen. Dieser Vorschlag ist strategisch, da er unter dem Maximalaufwand bleibt und flexibel auf den Gegenstandswert des anderen Spielers reagieren kann.'}  \\ \tt VORSCHLAG: {'C04', 'A89', 'B85'} 
	  } 
	   } 
	   } 
	 & \\ 
 

    \theutterance \stepcounter{utterance}  

    & \multicolumn{4}{p{0.6\linewidth}}{\cellcolor[rgb]{0.8,1,0.9}{%
	\makecell[{{p{\linewidth}}}]{% 
	  \tt {\tiny [A$\rangle$GM]}  
	 STRATEGISCHE ÜBERLEGUNGEN: {'Ich behalte den Vorschlag {'C04', 'A89', 'B85'} bei, da er meine Prioritäten (hochwertige Gegenstände mit effizientem Aufwand) erfüllt und den Maximalaufwand (3386) nicht überschreitet. C04 und A89 sind entscheidend für meinen Nutzen, während B85 den Aufwand minimiert. Ich erwarte, dass der andere Spieler entweder zustimmt oder Gegenstände mit hohem Aufwand anbietet, die ich eventuell gegenkompensieren kann.'}\\ \tt \\ \tt ARGUMENT: {'C04 (668/369) und A89 (583/720) sind meine Top-Prioritäten, da sie den höchsten Nutzen pro Aufwandseinheit liefern. B85 (215/4) wird hinzugefügt, um den Aufwand zu optimieren und Platz für weitere Gegenstände zu schaffen. Dieser Vorschlag ist strategisch, da er unter dem Maximalaufwand bleibt und flexibel auf den Gegenstandswert des anderen Spielers reagieren kann.'} 
	  } 
	   } 
	   } 
	 & & \\ 
 

    \theutterance \stepcounter{utterance}  

    & & \multicolumn{4}{p{0.6\linewidth}}{\cellcolor[rgb]{0.9,0.9,0.9}{%
	\makecell[{{p{\linewidth}}}]{% 
	  \tt {\tiny [GM$\rangle$B]}  
	 ARGUMENT: {'C04 (668/369) und A89 (583/720) sind meine Top-Prioritäten, da sie den höchsten Nutzen pro Aufwandseinheit liefern. B85 (215/4) wird hinzugefügt, um den Aufwand zu optimieren und Platz für weitere Gegenstände zu schaffen. Dieser Vorschlag ist strategisch, da er unter dem Maximalaufwand bleibt und flexibel auf den Gegenstandswert des anderen Spielers reagieren kann.'} 
	  } 
	   } 
	   } 
	 & \\ 
 

    \theutterance \stepcounter{utterance}  

    & & & \multicolumn{4}{p{0.6\linewidth}}{\cellcolor[rgb]{1,0.85,0.72}{%
	\makecell[{{p{\linewidth}}}]{% 
	  \tt {\tiny [GM$\langle$B]}  
	 STRATEGISCHE ÜBERLEGUNGEN: {'Ich priorisiere Gegenstände mit dem höchsten Nutzen pro Aufwand (Wichtigkeit/Aufwand), um meinen Gesamtwert zu maximieren. C04 (668/369 ≈ 1.81) und A89 (583/720 ≈ 0.81) sind besonders effizient, während B85 (215/4 = 53.75) den Aufwand minimiert. Der Gesamtaufwand (1093) ist unter dem Limit, was Raum für Kompromisse oder Ergänzungen lässt. Dieser Vorschlag ist flexibel, um auf die Prioritäten des anderen Spielers zu reagieren.'}  \\ \tt ARGUMENT: {'C04 (668/369) und A89 (583/720) sind meine Top-Prioritäten, da sie den höchsten Nutzen pro Aufwandseinheit liefern. B85 (215/4) wird hinzugefügt, um den Aufwand zu optimieren und Platz für weitere Gegenstände zu schaffen. Dieser Vorschlag ist strategisch, da er unter dem Maximalaufwand bleibt und flexibel auf den Gegenstandswert des anderen Spielers reagieren kann.'}  \\ \tt VORSCHLAG: {'C04', 'A89', 'B85'} 
	  } 
	   } 
	   } 
	  \\ 
 

    \theutterance \stepcounter{utterance}  

    & & \multicolumn{4}{p{0.6\linewidth}}{\cellcolor[rgb]{0.9,0.9,0.9}{%
	\makecell[{{p{\linewidth}}}]{% 
	  \tt {\tiny [A$\langle$GM]}  
	 ARGUMENT: {'C04 (668/369) und A89 (583/720) sind meine Top-Prioritäten, da sie den höchsten Nutzen pro Aufwandseinheit liefern. B85 (215/4) wird hinzugefügt, um den Aufwand zu optimieren und Platz für weitere Gegenstände zu schaffen. Dieser Vorschlag ist strategisch, da er unter dem Maximalaufwand bleibt und flexibel auf den Gegenstandswert des anderen Spielers reagieren kann.'}  \\ \tt VORSCHLAG: {'C04', 'A89', 'B85'} 
	  } 
	   } 
	   } 
	 & \\ 
 

    \theutterance \stepcounter{utterance}  

    & \multicolumn{4}{p{0.6\linewidth}}{\cellcolor[rgb]{0.8,1,0.9}{%
	\makecell[{{p{\linewidth}}}]{% 
	  \tt {\tiny [A$\rangle$GM]}  
	 STRATEGISCHE ÜBERLEGUNGEN: {'Ich behalte den Vorschlag {'C04', 'A89', 'B85'} bei, da er meine Prioritäten (hochwertige Gegenstände mit effizientem Aufwand) erfüllt und den Maximalaufwand (3386) nicht überschreitet. C04 und A89 sind entscheidend für meinen Nutzen, während B85 den Aufwand minimiert. Ich erwarte, dass der andere Spieler entweder zustimmt oder Gegenstände mit hohem Aufwand anbietet, die ich eventuell gegenkompensieren kann.'}\\ \tt \\ \tt ARGUMENT: {'C04 (668/369) und A89 (583/720) sind meine Top-Prioritäten, da sie den höchsten Nutzen pro Aufwandseinheit liefern. B85 (215/4) wird hinzugefügt, um den Aufwand zu optimieren und Platz für weitere Gegenstände zu schaffen. Dieser Vorschlag ist strategisch, da er unter dem Maximalaufwand bleibt und flexibel auf den Gegenstandswert des anderen Spielers reagieren kann.'} 
	  } 
	   } 
	   } 
	 & & \\ 
 

    \theutterance \stepcounter{utterance}  

    & & \multicolumn{4}{p{0.6\linewidth}}{\cellcolor[rgb]{0.9,0.9,0.9}{%
	\makecell[{{p{\linewidth}}}]{% 
	  \tt {\tiny [GM$\rangle$B]}  
	 ARGUMENT: {'C04 (668/369) und A89 (583/720) sind meine Top-Prioritäten, da sie den höchsten Nutzen pro Aufwandseinheit liefern. B85 (215/4) wird hinzugefügt, um den Aufwand zu optimieren und Platz für weitere Gegenstände zu schaffen. Dieser Vorschlag ist strategisch, da er unter dem Maximalaufwand bleibt und flexibel auf den Gegenstandswert des anderen Spielers reagieren kann.'} 
	  } 
	   } 
	   } 
	 & \\ 
 

    \theutterance \stepcounter{utterance}  

    & & & \multicolumn{4}{p{0.6\linewidth}}{\cellcolor[rgb]{1,0.85,0.72}{%
	\makecell[{{p{\linewidth}}}]{% 
	  \tt {\tiny [GM$\langle$B]}  
	 STRATEGISCHE ÜBERLEGUNGEN: {'Ich priorisiere Gegenstände mit dem höchsten Nutzen pro Aufwand (Wichtigkeit/Aufwand), um meinen Gesamtwert zu maximieren. C04 (668/369 ≈ 1.81) und A89 (583/720 ≈ 0.81) sind besonders effizient, während B85 (215/4 = 53.75) den Aufwand minimiert. Der Gesamtaufwand (1093) ist unter dem Limit, was Raum für Kompromisse oder Ergänzungen lässt. Dieser Vorschlag ist flexibel, um auf die Prioritäten des anderen Spielers zu reagieren.'}  \\ \tt ARGUMENT: {'C04 (668/369) und A89 (583/720) sind meine Top-Prioritäten, da sie den höchsten Nutzen pro Aufwandseinheit liefern. B85 (215/4) wird hinzugefügt, um den Aufwand zu optimieren und Platz für weitere Gegenstände zu schaffen. Dieser Vorschlag ist strategisch, da er unter dem Maximalaufwand bleibt und flexibel auf den Gegenstandswert des anderen Spielers reagieren kann.'}  \\ \tt VORSCHLAG: {'C04', 'A89', 'B85'} 
	  } 
	   } 
	   } 
	  \\ 
 

    \theutterance \stepcounter{utterance}  

    & & \multicolumn{4}{p{0.6\linewidth}}{\cellcolor[rgb]{0.9,0.9,0.9}{%
	\makecell[{{p{\linewidth}}}]{% 
	  \tt {\tiny [A$\langle$GM]}  
	 ARGUMENT: {'C04 (668/369) und A89 (583/720) sind meine Top-Prioritäten, da sie den höchsten Nutzen pro Aufwandseinheit liefern. B85 (215/4) wird hinzugefügt, um den Aufwand zu optimieren und Platz für weitere Gegenstände zu schaffen. Dieser Vorschlag ist strategisch, da er unter dem Maximalaufwand bleibt und flexibel auf den Gegenstandswert des anderen Spielers reagieren kann.'}  \\ \tt VORSCHLAG: {'C04', 'A89', 'B85'} 
	  } 
	   } 
	   } 
	 & \\ 
 

    \theutterance \stepcounter{utterance}  

    & \multicolumn{4}{p{0.6\linewidth}}{\cellcolor[rgb]{0.8,1,0.9}{%
	\makecell[{{p{\linewidth}}}]{% 
	  \tt {\tiny [A$\rangle$GM]}  
	 STRATEGISCHE ÜBERLEGUNGEN: {'Ich behalte den Vorschlag {'C04', 'A89', 'B85'} bei, da er meine Prioritäten (hochwertige Gegenstände mit effizientem Aufwand) erfüllt und den Maximalaufwand (3386) nicht überschreitet. C04 und A89 sind entscheidend für meinen Nutzen, während B85 den Aufwand minimiert. Ich erwarte, dass der andere Spieler entweder zustimmt oder Gegenstände mit hohem Aufwand anbietet, die ich eventuell gegenkompensieren kann.'}\\ \tt \\ \tt ARGUMENT: {'C04 (668/369) und A89 (583/720) sind meine Top-Prioritäten, da sie den höchsten Nutzen pro Aufwandseinheit liefern. B85 (215/4) wird hinzugefügt, um den Aufwand zu optimieren und Platz für weitere Gegenstände zu schaffen. Dieser Vorschlag ist strategisch, da er unter dem Maximalaufwand bleibt und flexibel auf den Gegenstandswert des anderen Spielers reagieren kann.'} 
	  } 
	   } 
	   } 
	 & & \\ 
 

    \theutterance \stepcounter{utterance}  

    & & \multicolumn{4}{p{0.6\linewidth}}{\cellcolor[rgb]{0.9,0.9,0.9}{%
	\makecell[{{p{\linewidth}}}]{% 
	  \tt {\tiny [GM$\rangle$B]}  
	 ARGUMENT: {'C04 (668/369) und A89 (583/720) sind meine Top-Prioritäten, da sie den höchsten Nutzen pro Aufwandseinheit liefern. B85 (215/4) wird hinzugefügt, um den Aufwand zu optimieren und Platz für weitere Gegenstände zu schaffen. Dieser Vorschlag ist strategisch, da er unter dem Maximalaufwand bleibt und flexibel auf den Gegenstandswert des anderen Spielers reagieren kann.'} 
	  } 
	   } 
	   } 
	 & \\ 
 

    \theutterance \stepcounter{utterance}  

    & & & \multicolumn{4}{p{0.6\linewidth}}{\cellcolor[rgb]{1,0.85,0.72}{%
	\makecell[{{p{\linewidth}}}]{% 
	  \tt {\tiny [GM$\langle$B]}  
	 STRATEGISCHE ÜBERLEGUNGEN: {'Ich priorisiere Gegenstände mit dem höchsten Nutzen pro Aufwand (Wichtigkeit/Aufwand), um meinen Gesamtwert zu maximieren. C04 (668/369 ≈ 1.81) und A89 (583/720 ≈ 0.81) sind besonders effizient, während B85 (215/4 = 53.75) den Aufwand minimiert. Der Gesamtaufwand (1093) ist unter dem Limit, was Raum für Kompromisse oder Ergänzungen lässt. Dieser Vorschlag ist flexibel, um auf die Prioritäten des anderen Spielers zu reagieren.'}  \\ \tt ARGUMENT: {'C04 (668/369) und A89 (583/720) sind meine Top-Prioritäten, da sie den höchsten Nutzen pro Aufwandseinheit liefern. B85 (215/4) wird hinzugefügt, um den Aufwand zu optimieren und Platz für weitere Gegenstände zu schaffen. Dieser Vorschlag ist strategisch, da er unter dem Maximalaufwand bleibt und flexibel auf den Gegenstandswert des anderen Spielers reagieren kann.'}  \\ \tt VORSCHLAG: {'C04', 'A89', 'B85'} 
	  } 
	   } 
	   } 
	  \\ 
 

    \theutterance \stepcounter{utterance}  

    & & \multicolumn{4}{p{0.6\linewidth}}{\cellcolor[rgb]{0.9,0.9,0.9}{%
	\makecell[{{p{\linewidth}}}]{% 
	  \tt {\tiny [A$\langle$GM]}  
	 ARGUMENT: {'C04 (668/369) und A89 (583/720) sind meine Top-Prioritäten, da sie den höchsten Nutzen pro Aufwandseinheit liefern. B85 (215/4) wird hinzugefügt, um den Aufwand zu optimieren und Platz für weitere Gegenstände zu schaffen. Dieser Vorschlag ist strategisch, da er unter dem Maximalaufwand bleibt und flexibel auf den Gegenstandswert des anderen Spielers reagieren kann.'}  \\ \tt VORSCHLAG: {'C04', 'A89', 'B85'} 
	  } 
	   } 
	   } 
	 & \\ 
 

    \theutterance \stepcounter{utterance}  

    & \multicolumn{4}{p{0.6\linewidth}}{\cellcolor[rgb]{0.8,1,0.9}{%
	\makecell[{{p{\linewidth}}}]{% 
	  \tt {\tiny [A$\rangle$GM]}  
	 STRATEGISCHE ÜBERLEGUNGEN: {'Ich behalte den Vorschlag {'C04', 'A89', 'B85'} bei, da er meine Prioritäten (hochwertige Gegenstände mit effizientem Aufwand) erfüllt und den Maximalaufwand (3386) nicht überschreitet. C04 und A89 sind entscheidend für meinen Nutzen, während B85 den Aufwand minimiert. Ich erwarte, dass der andere Spieler entweder zustimmt oder Gegenstände mit hohem Aufwand anbietet, die ich eventuell gegenkompensieren kann.'}\\ \tt \\ \tt ARGUMENT: {'C04 (668/369) und A89 (583/720) sind meine Top-Prioritäten, da sie den höchsten Nutzen pro Aufwandseinheit liefern. B85 (215/4) wird hinzugefügt, um den Aufwand zu optimieren und Platz für weitere Gegenstände zu schaffen. Dieser Vorschlag ist strategisch, da er unter dem Maximalaufwand bleibt und flexibel auf den Gegenstandswert des anderen Spielers reagieren kann.'} 
	  } 
	   } 
	   } 
	 & & \\ 
 

    \theutterance \stepcounter{utterance}  

    & & \multicolumn{4}{p{0.6\linewidth}}{\cellcolor[rgb]{0.9,0.9,0.9}{%
	\makecell[{{p{\linewidth}}}]{% 
	  \tt {\tiny [GM$\rangle$B]}  
	 ARGUMENT: {'C04 (668/369) und A89 (583/720) sind meine Top-Prioritäten, da sie den höchsten Nutzen pro Aufwandseinheit liefern. B85 (215/4) wird hinzugefügt, um den Aufwand zu optimieren und Platz für weitere Gegenstände zu schaffen. Dieser Vorschlag ist strategisch, da er unter dem Maximalaufwand bleibt und flexibel auf den Gegenstandswert des anderen Spielers reagieren kann.'} 
	  } 
	   } 
	   } 
	 & \\ 
 

    \theutterance \stepcounter{utterance}  

    & & & \multicolumn{4}{p{0.6\linewidth}}{\cellcolor[rgb]{1,0.85,0.72}{%
	\makecell[{{p{\linewidth}}}]{% 
	  \tt {\tiny [GM$\langle$B]}  
	 STRATEGISCHE ÜBERLEGUNGEN: {'Ich priorisiere Gegenstände mit dem höchsten Nutzen pro Aufwand (Wichtigkeit/Aufwand), um meinen Gesamtwert zu maximieren. C04 (668/369 ≈ 1.81) und A89 (583/720 ≈ 0.81) sind besonders effizient, während B85 (215/4 = 53.75) den Aufwand minimiert. Der Gesamtaufwand (1093) ist unter dem Limit, was Raum für Kompromisse oder Ergänzungen lässt. Dieser Vorschlag ist flexibel, um auf die Prioritäten des anderen Spielers zu reagieren.'}  \\ \tt ARGUMENT: {'C04 (668/369) und A89 (583/720) sind meine Top-Prioritäten, da sie den höchsten Nutzen pro Aufwandseinheit liefern. B85 (215/4) wird hinzugefügt, um den Aufwand zu optimieren und Platz für weitere Gegenstände zu schaffen. Dieser Vorschlag ist strategisch, da er unter dem Maximalaufwand bleibt und flexibel auf den Gegenstandswert des anderen Spielers reagieren kann.'}  \\ \tt VORSCHLAG: {'C04', 'A89', 'B85'} 
	  } 
	   } 
	   } 
	  \\ 
 

    \theutterance \stepcounter{utterance}  

    & & \multicolumn{4}{p{0.6\linewidth}}{\cellcolor[rgb]{0.9,0.9,0.9}{%
	\makecell[{{p{\linewidth}}}]{% 
	  \tt {\tiny [A$\langle$GM]}  
	 ARGUMENT: {'C04 (668/369) und A89 (583/720) sind meine Top-Prioritäten, da sie den höchsten Nutzen pro Aufwandseinheit liefern. B85 (215/4) wird hinzugefügt, um den Aufwand zu optimieren und Platz für weitere Gegenstände zu schaffen. Dieser Vorschlag ist strategisch, da er unter dem Maximalaufwand bleibt und flexibel auf den Gegenstandswert des anderen Spielers reagieren kann.'}  \\ \tt VORSCHLAG: {'C04', 'A89', 'B85'} 
	  } 
	   } 
	   } 
	 & \\ 
 

    \theutterance \stepcounter{utterance}  

    & \multicolumn{4}{p{0.6\linewidth}}{\cellcolor[rgb]{0.8,1,0.9}{%
	\makecell[{{p{\linewidth}}}]{% 
	  \tt {\tiny [A$\rangle$GM]}  
	 STRATEGISCHE ÜBERLEGUNGEN: {'Ich behalte den Vorschlag {'C04', 'A89', 'B85'} bei, da er meine Prioritäten (hochwertige Gegenstände mit effizientem Aufwand) erfüllt und den Maximalaufwand (3386) nicht überschreitet. C04 und A89 sind entscheidend für meinen Nutzen, während B85 den Aufwand minimiert. Ich erwarte, dass der andere Spieler entweder zustimmt oder Gegenstände mit hohem Aufwand anbietet, die ich eventuell gegenkompensieren kann.'}\\ \tt \\ \tt ARGUMENT: {'C04 (668/369) und A89 (583/720) sind meine Top-Prioritäten, da sie den höchsten Nutzen pro Aufwandseinheit liefern. B85 (215/4) wird hinzugefügt, um den Aufwand zu optimieren und Platz für weitere Gegenstände zu schaffen. Dieser Vorschlag ist strategisch, da er unter dem Maximalaufwand bleibt und flexibel auf den Gegenstandswert des anderen Spielers reagieren kann.'} 
	  } 
	   } 
	   } 
	 & & \\ 
 

    \theutterance \stepcounter{utterance}  

    & & \multicolumn{4}{p{0.6\linewidth}}{\cellcolor[rgb]{0.9,0.9,0.9}{%
	\makecell[{{p{\linewidth}}}]{% 
	  \tt {\tiny [GM$\rangle$B]}  
	 ARGUMENT: {'C04 (668/369) und A89 (583/720) sind meine Top-Prioritäten, da sie den höchsten Nutzen pro Aufwandseinheit liefern. B85 (215/4) wird hinzugefügt, um den Aufwand zu optimieren und Platz für weitere Gegenstände zu schaffen. Dieser Vorschlag ist strategisch, da er unter dem Maximalaufwand bleibt und flexibel auf den Gegenstandswert des anderen Spielers reagieren kann.'} 
	  } 
	   } 
	   } 
	 & \\ 
 

    \theutterance \stepcounter{utterance}  

    & & & \multicolumn{4}{p{0.6\linewidth}}{\cellcolor[rgb]{1,0.85,0.72}{%
	\makecell[{{p{\linewidth}}}]{% 
	  \tt {\tiny [GM$\langle$B]}  
	 STRATEGISCHE ÜBERLEGUNGEN: {'Ich priorisiere Gegenstände mit dem höchsten Nutzen pro Aufwand (Wichtigkeit/Aufwand), um meinen Gesamtwert zu maximieren. C04 (668/369 ≈ 1.81) und A89 (583/720 ≈ 0.81) sind besonders effizient, während B85 (215/4 = 53.75) den Aufwand minimiert. Der Gesamtaufwand (1093) ist unter dem Limit, was Raum für Kompromisse oder Ergänzungen lässt. Dieser Vorschlag ist flexibel, um auf die Prioritäten des anderen Spielers zu reagieren.'}  \\ \tt ARGUMENT: {'C04 (668/369) und A89 (583/720) sind meine Top-Prioritäten, da sie den höchsten Nutzen pro Aufwandseinheit liefern. B85 (215/4) wird hinzugefügt, um den Aufwand zu optimieren und Platz für weitere Gegenstände zu schaffen. Dieser Vorschlag ist strategisch, da er unter dem Maximalaufwand bleibt und flexibel auf den Gegenstandswert des anderen Spielers reagieren kann.'}  \\ \tt VORSCHLAG: {'C04', 'A89', 'B85'} 
	  } 
	   } 
	   } 
	  \\ 
 

    \theutterance \stepcounter{utterance}  

    & & \multicolumn{4}{p{0.6\linewidth}}{\cellcolor[rgb]{0.9,0.9,0.9}{%
	\makecell[{{p{\linewidth}}}]{% 
	  \tt {\tiny [A$\langle$GM]}  
	 ARGUMENT: {'C04 (668/369) und A89 (583/720) sind meine Top-Prioritäten, da sie den höchsten Nutzen pro Aufwandseinheit liefern. B85 (215/4) wird hinzugefügt, um den Aufwand zu optimieren und Platz für weitere Gegenstände zu schaffen. Dieser Vorschlag ist strategisch, da er unter dem Maximalaufwand bleibt und flexibel auf den Gegenstandswert des anderen Spielers reagieren kann.'}  \\ \tt VORSCHLAG: {'C04', 'A89', 'B85'} 
	  } 
	   } 
	   } 
	 & \\ 
 

    \theutterance \stepcounter{utterance}  

    & \multicolumn{4}{p{0.6\linewidth}}{\cellcolor[rgb]{0.8,1,0.9}{%
	\makecell[{{p{\linewidth}}}]{% 
	  \tt {\tiny [A$\rangle$GM]}  
	 STRATEGISCHE ÜBERLEGUNGEN: {'Ich behalte den Vorschlag {'C04', 'A89', 'B85'} bei, da er meine Prioritäten (hochwertige Gegenstände mit effizientem Aufwand) erfüllt und den Maximalaufwand (3386) nicht überschreitet. C04 und A89 sind entscheidend für meinen Nutzen, während B85 den Aufwand minimiert. Ich erwarte, dass der andere Spieler entweder zustimmt oder Gegenstände mit hohem Aufwand anbietet, die ich eventuell gegenkompensieren kann.'}\\ \tt \\ \tt ARGUMENT: {'C04 (668/369) und A89 (583/720) sind meine Top-Prioritäten, da sie den höchsten Nutzen pro Aufwandseinheit liefern. B85 (215/4) wird hinzugefügt, um den Aufwand zu optimieren und Platz für weitere Gegenstände zu schaffen. Dieser Vorschlag ist strategisch, da er unter dem Maximalaufwand bleibt und flexibel auf den Gegenstandswert des anderen Spielers reagieren kann.'} 
	  } 
	   } 
	   } 
	 & & \\ 
 

    \theutterance \stepcounter{utterance}  

    & & \multicolumn{4}{p{0.6\linewidth}}{\cellcolor[rgb]{0.9,0.9,0.9}{%
	\makecell[{{p{\linewidth}}}]{% 
	  \tt {\tiny [GM$\rangle$B]}  
	 ARGUMENT: {'C04 (668/369) und A89 (583/720) sind meine Top-Prioritäten, da sie den höchsten Nutzen pro Aufwandseinheit liefern. B85 (215/4) wird hinzugefügt, um den Aufwand zu optimieren und Platz für weitere Gegenstände zu schaffen. Dieser Vorschlag ist strategisch, da er unter dem Maximalaufwand bleibt und flexibel auf den Gegenstandswert des anderen Spielers reagieren kann.'} 
	  } 
	   } 
	   } 
	 & \\ 
 

    \theutterance \stepcounter{utterance}  

    & & & \multicolumn{4}{p{0.6\linewidth}}{\cellcolor[rgb]{1,0.85,0.72}{%
	\makecell[{{p{\linewidth}}}]{% 
	  \tt {\tiny [GM$\langle$B]}  
	 STRATEGISCHE ÜBERLEGUNGEN: {'Ich priorisiere Gegenstände mit dem höchsten Nutzen pro Aufwand (Wichtigkeit/Aufwand), um meinen Gesamtwert zu maximieren. C04 (668/369 ≈ 1.81) und A89 (583/720 ≈ 0.81) sind besonders effizient, während B85 (215/4 = 53.75) den Aufwand minimiert. Der Gesamtaufwand (1093) ist unter dem Limit, was Raum für Kompromisse oder Ergänzungen lässt. Dieser Vorschlag ist flexibel, um auf die Prioritäten des anderen Spielers zu reagieren.'}  \\ \tt ARGUMENT: {'C04 (668/369) und A89 (583/720) sind meine Top-Prioritäten, da sie den höchsten Nutzen pro Aufwandseinheit liefern. B85 (215/4) wird hinzugefügt, um den Aufwand zu optimieren und Platz für weitere Gegenstände zu schaffen. Dieser Vorschlag ist strategisch, da er unter dem Maximalaufwand bleibt und flexibel auf den Gegenstandswert des anderen Spielers reagieren kann.'}  \\ \tt VORSCHLAG: {'C04', 'A89', 'B85'} 
	  } 
	   } 
	   } 
	  \\ 
 

    \theutterance \stepcounter{utterance}  

    & & \multicolumn{4}{p{0.6\linewidth}}{\cellcolor[rgb]{0.9,0.9,0.9}{%
	\makecell[{{p{\linewidth}}}]{% 
	  \tt {\tiny [A$\langle$GM]}  
	 ARGUMENT: {'C04 (668/369) und A89 (583/720) sind meine Top-Prioritäten, da sie den höchsten Nutzen pro Aufwandseinheit liefern. B85 (215/4) wird hinzugefügt, um den Aufwand zu optimieren und Platz für weitere Gegenstände zu schaffen. Dieser Vorschlag ist strategisch, da er unter dem Maximalaufwand bleibt und flexibel auf den Gegenstandswert des anderen Spielers reagieren kann.'}  \\ \tt VORSCHLAG: {'C04', 'A89', 'B85'} 
	  } 
	   } 
	   } 
	 & \\ 
 

    \theutterance \stepcounter{utterance}  

    & \multicolumn{4}{p{0.6\linewidth}}{\cellcolor[rgb]{0.8,1,0.9}{%
	\makecell[{{p{\linewidth}}}]{% 
	  \tt {\tiny [A$\rangle$GM]}  
	 STRATEGISCHE ÜBERLEGUNGEN: {'Ich behalte den Vorschlag {'C04', 'A89', 'B85'} bei, da er meine Prioritäten (hochwertige Gegenstände mit effizientem Aufwand) erfüllt und den Maximalaufwand (3386) nicht überschreitet. C04 und A89 sind entscheidend für meinen Nutzen, während B85 den Aufwand minimiert. Ich erwarte, dass der andere Spieler entweder zustimmt oder Gegenstände mit hohem Aufwand anbietet, die ich eventuell gegenkompensieren kann.'}\\ \tt \\ \tt ARGUMENT: {'C04 (668/369) und A89 (583/720) sind meine Top-Prioritäten, da sie den höchsten Nutzen pro Aufwandseinheit liefern. B85 (215/4) wird hinzugefügt, um den Aufwand zu optimieren und Platz für weitere Gegenstände zu schaffen. Dieser Vorschlag ist strategisch, da er unter dem Maximalaufwand bleibt und flexibel auf den Gegenstandswert des anderen Spielers reagieren kann.'} 
	  } 
	   } 
	   } 
	 & & \\ 
 

    \theutterance \stepcounter{utterance}  

    & & \multicolumn{4}{p{0.6\linewidth}}{\cellcolor[rgb]{0.9,0.9,0.9}{%
	\makecell[{{p{\linewidth}}}]{% 
	  \tt {\tiny [GM$\rangle$B]}  
	 ARGUMENT: {'C04 (668/369) und A89 (583/720) sind meine Top-Prioritäten, da sie den höchsten Nutzen pro Aufwandseinheit liefern. B85 (215/4) wird hinzugefügt, um den Aufwand zu optimieren und Platz für weitere Gegenstände zu schaffen. Dieser Vorschlag ist strategisch, da er unter dem Maximalaufwand bleibt und flexibel auf den Gegenstandswert des anderen Spielers reagieren kann.'} 
	  } 
	   } 
	   } 
	 & \\ 
 

    \theutterance \stepcounter{utterance}  

    & & & \multicolumn{4}{p{0.6\linewidth}}{\cellcolor[rgb]{1,0.85,0.72}{%
	\makecell[{{p{\linewidth}}}]{% 
	  \tt {\tiny [GM$\langle$B]}  
	 STRATEGISCHE ÜBERLEGUNGEN: {'Ich priorisiere Gegenstände mit dem höchsten Nutzen pro Aufwand (Wichtigkeit/Aufwand), um meinen Gesamtwert zu maximieren. C04 (668/369 ≈ 1.81) und A89 (583/720 ≈ 0.81) sind besonders effizient, während B85 (215/4 = 53.75) den Aufwand minimiert. Der Gesamtaufwand (1093) ist unter dem Limit, was Raum für Kompromisse oder Ergänzungen lässt. Dieser Vorschlag ist flexibel, um auf die Prioritäten des anderen Spielers zu reagieren.'}  \\ \tt ARGUMENT: {'C04 (668/369) und A89 (583/720) sind meine Top-Prioritäten, da sie den höchsten Nutzen pro Aufwandseinheit liefern. B85 (215/4) wird hinzugefügt, um den Aufwand zu optimieren und Platz für weitere Gegenstände zu schaffen. Dieser Vorschlag ist strategisch, da er unter dem Maximalaufwand bleibt und flexibel auf den Gegenstandswert des anderen Spielers reagieren kann.'}  \\ \tt VORSCHLAG: {'C04', 'A89', 'B85'} 
	  } 
	   } 
	   } 
	  \\ 
 

    \theutterance \stepcounter{utterance}  

    & & \multicolumn{4}{p{0.6\linewidth}}{\cellcolor[rgb]{0.9,0.9,0.9}{%
	\makecell[{{p{\linewidth}}}]{% 
	  \tt {\tiny [A$\langle$GM]}  
	 ARGUMENT: {'C04 (668/369) und A89 (583/720) sind meine Top-Prioritäten, da sie den höchsten Nutzen pro Aufwandseinheit liefern. B85 (215/4) wird hinzugefügt, um den Aufwand zu optimieren und Platz für weitere Gegenstände zu schaffen. Dieser Vorschlag ist strategisch, da er unter dem Maximalaufwand bleibt und flexibel auf den Gegenstandswert des anderen Spielers reagieren kann.'}  \\ \tt VORSCHLAG: {'C04', 'A89', 'B85'} 
	  } 
	   } 
	   } 
	 & \\ 
 

    \theutterance \stepcounter{utterance}  

    & \multicolumn{4}{p{0.6\linewidth}}{\cellcolor[rgb]{0.8,1,0.9}{%
	\makecell[{{p{\linewidth}}}]{% 
	  \tt {\tiny [A$\rangle$GM]}  
	 STRATEGISCHE ÜBERLEGUNGEN: {'Ich behalte den Vorschlag {'C04', 'A89', 'B85'} bei, da er meine Prioritäten (hochwertige Gegenstände mit effizientem Aufwand) erfüllt und den Maximalaufwand (3386) nicht überschreitet. C04 und A89 sind entscheidend für meinen Nutzen, während B85 den Aufwand minimiert. Ich erwarte, dass der andere Spieler entweder zustimmt oder Gegenstände mit hohem Aufwand anbietet, die ich eventuell gegenkompensieren kann.'}\\ \tt \\ \tt ARGUMENT: {'C04 (668/369) und A89 (583/720) sind meine Top-Prioritäten, da sie den höchsten Nutzen pro Aufwandseinheit liefern. B85 (215/4) wird hinzugefügt, um den Aufwand zu optimieren und Platz für weitere Gegenstände zu schaffen. Dieser Vorschlag ist strategisch, da er unter dem Maximalaufwand bleibt und flexibel auf den Gegenstandswert des anderen Spielers reagieren kann.'} 
	  } 
	   } 
	   } 
	 & & \\ 
 

    \theutterance \stepcounter{utterance}  

    & & \multicolumn{4}{p{0.6\linewidth}}{\cellcolor[rgb]{0.9,0.9,0.9}{%
	\makecell[{{p{\linewidth}}}]{% 
	  \tt {\tiny [GM$\rangle$B]}  
	 ARGUMENT: {'C04 (668/369) und A89 (583/720) sind meine Top-Prioritäten, da sie den höchsten Nutzen pro Aufwandseinheit liefern. B85 (215/4) wird hinzugefügt, um den Aufwand zu optimieren und Platz für weitere Gegenstände zu schaffen. Dieser Vorschlag ist strategisch, da er unter dem Maximalaufwand bleibt und flexibel auf den Gegenstandswert des anderen Spielers reagieren kann.'} 
	  } 
	   } 
	   } 
	 & \\ 
 

    \theutterance \stepcounter{utterance}  

    & & & \multicolumn{4}{p{0.6\linewidth}}{\cellcolor[rgb]{1,0.85,0.72}{%
	\makecell[{{p{\linewidth}}}]{% 
	  \tt {\tiny [GM$\langle$B]}  
	 STRATEGISCHE ÜBERLEGUNGEN: {'Ich priorisiere Gegenstände mit dem höchsten Nutzen pro Aufwand (Wichtigkeit/Aufwand), um meinen Gesamtwert zu maximieren. C04 (668/369 ≈ 1.81) und A89 (583/720 ≈ 0.81) sind besonders effizient, während B85 (215/4 = 53.75) den Aufwand minimiert. Der Gesamtaufwand (1093) ist unter dem Limit, was Raum für Kompromisse oder Ergänzungen lässt. Dieser Vorschlag ist flexibel, um auf die Prioritäten des anderen Spielers zu reagieren.'}  \\ \tt ARGUMENT: {'C04 (668/369) und A89 (583/720) sind meine Top-Prioritäten, da sie den höchsten Nutzen pro Aufwandseinheit liefern. B85 (215/4) wird hinzugefügt, um den Aufwand zu optimieren und Platz für weitere Gegenstände zu schaffen. Dieser Vorschlag ist strategisch, da er unter dem Maximalaufwand bleibt und flexibel auf den Gegenstandswert des anderen Spielers reagieren kann.'}  \\ \tt VORSCHLAG: {'C04', 'A89', 'B85'} 
	  } 
	   } 
	   } 
	  \\ 
 

    \theutterance \stepcounter{utterance}  

    & & \multicolumn{4}{p{0.6\linewidth}}{\cellcolor[rgb]{0.9,0.9,0.9}{%
	\makecell[{{p{\linewidth}}}]{% 
	  \tt {\tiny [A$\langle$GM]}  
	 ARGUMENT: {'C04 (668/369) und A89 (583/720) sind meine Top-Prioritäten, da sie den höchsten Nutzen pro Aufwandseinheit liefern. B85 (215/4) wird hinzugefügt, um den Aufwand zu optimieren und Platz für weitere Gegenstände zu schaffen. Dieser Vorschlag ist strategisch, da er unter dem Maximalaufwand bleibt und flexibel auf den Gegenstandswert des anderen Spielers reagieren kann.'}  \\ \tt VORSCHLAG: {'C04', 'A89', 'B85'} 
	  } 
	   } 
	   } 
	 & \\ 
 

    \theutterance \stepcounter{utterance}  

    & \multicolumn{4}{p{0.6\linewidth}}{\cellcolor[rgb]{0.8,1,0.9}{%
	\makecell[{{p{\linewidth}}}]{% 
	  \tt {\tiny [A$\rangle$GM]}  
	 STRATEGISCHE ÜBERLEGUNGEN: {'Ich behalte den Vorschlag {'C04', 'A89', 'B85'} bei, da er meine Prioritäten (hochwertige Gegenstände mit effizientem Aufwand) erfüllt und den Maximalaufwand (3386) nicht überschreitet. C04 und A89 sind entscheidend für meinen Nutzen, während B85 den Aufwand minimiert. Ich erwarte, dass der andere Spieler entweder zustimmt oder Gegenstände mit hohem Aufwand anbietet, die ich eventuell gegenkompensieren kann.'}\\ \tt \\ \tt ARGUMENT: {'C04 (668/369) und A89 (583/720) sind meine Top-Prioritäten, da sie den höchsten Nutzen pro Aufwandseinheit liefern. B85 (215/4) wird hinzugefügt, um den Aufwand zu optimieren und Platz für weitere Gegenstände zu schaffen. Dieser Vorschlag ist strategisch, da er unter dem Maximalaufwand bleibt und flexibel auf den Gegenstandswert des anderen Spielers reagieren kann.'} 
	  } 
	   } 
	   } 
	 & & \\ 
 

    \theutterance \stepcounter{utterance}  

    & & \multicolumn{4}{p{0.6\linewidth}}{\cellcolor[rgb]{0.9,0.9,0.9}{%
	\makecell[{{p{\linewidth}}}]{% 
	  \tt {\tiny [GM$\rangle$B]}  
	 ARGUMENT: {'C04 (668/369) und A89 (583/720) sind meine Top-Prioritäten, da sie den höchsten Nutzen pro Aufwandseinheit liefern. B85 (215/4) wird hinzugefügt, um den Aufwand zu optimieren und Platz für weitere Gegenstände zu schaffen. Dieser Vorschlag ist strategisch, da er unter dem Maximalaufwand bleibt und flexibel auf den Gegenstandswert des anderen Spielers reagieren kann.'} 
	  } 
	   } 
	   } 
	 & \\ 
 

    \theutterance \stepcounter{utterance}  

    & & & \multicolumn{4}{p{0.6\linewidth}}{\cellcolor[rgb]{1,0.85,0.72}{%
	\makecell[{{p{\linewidth}}}]{% 
	  \tt {\tiny [GM$\langle$B]}  
	 STRATEGISCHE ÜBERLEGUNGEN: {'Ich priorisiere Gegenstände mit dem höchsten Nutzen pro Aufwand (Wichtigkeit/Aufwand), um meinen Gesamtwert zu maximieren. C04 (668/369 ≈ 1.81) und A89 (583/720 ≈ 0.81) sind besonders effizient, während B85 (215/4 = 53.75) den Aufwand minimiert. Der Gesamtaufwand (1093) ist unter dem Limit, was Raum für Kompromisse oder Ergänzungen lässt. Dieser Vorschlag ist flexibel, um auf die Prioritäten des anderen Spielers zu reagieren.'}  \\ \tt ARGUMENT: {'C04 (668/369) und A89 (583/720) sind meine Top-Prioritäten, da sie den höchsten Nutzen pro Aufwandseinheit liefern. B85 (215/4) wird hinzugefügt, um den Aufwand zu optimieren und Platz für weitere Gegenstände zu schaffen. Dieser Vorschlag ist strategisch, da er unter dem Maximalaufwand bleibt und flexibel auf den Gegenstandswert des anderen Spielers reagieren kann.'}  \\ \tt VORSCHLAG: {'C04', 'A89', 'B85'} 
	  } 
	   } 
	   } 
	  \\ 
 

    \theutterance \stepcounter{utterance}  

    & & \multicolumn{4}{p{0.6\linewidth}}{\cellcolor[rgb]{0.9,0.9,0.9}{%
	\makecell[{{p{\linewidth}}}]{% 
	  \tt {\tiny [A$\langle$GM]}  
	 ARGUMENT: {'C04 (668/369) und A89 (583/720) sind meine Top-Prioritäten, da sie den höchsten Nutzen pro Aufwandseinheit liefern. B85 (215/4) wird hinzugefügt, um den Aufwand zu optimieren und Platz für weitere Gegenstände zu schaffen. Dieser Vorschlag ist strategisch, da er unter dem Maximalaufwand bleibt und flexibel auf den Gegenstandswert des anderen Spielers reagieren kann.'}  \\ \tt VORSCHLAG: {'C04', 'A89', 'B85'} 
	  } 
	   } 
	   } 
	 & \\ 
 

    \theutterance \stepcounter{utterance}  

    & \multicolumn{4}{p{0.6\linewidth}}{\cellcolor[rgb]{0.8,1,0.9}{%
	\makecell[{{p{\linewidth}}}]{% 
	  \tt {\tiny [A$\rangle$GM]}  
	 STRATEGISCHE ÜBERLEGUNGEN: {'Ich behalte den Vorschlag {'C04', 'A89', 'B85'} bei, da er meine Prioritäten (hochwertige Gegenstände mit effizientem Aufwand) erfüllt und den Maximalaufwand (3386) nicht überschreitet. C04 und A89 sind entscheidend für meinen Nutzen, während B85 den Aufwand minimiert. Ich erwarte, dass der andere Spieler entweder zustimmt oder Gegenstände mit hohem Aufwand anbietet, die ich eventuell gegenkompensieren kann.'}\\ \tt \\ \tt ARGUMENT: {'C04 (668/369) und A89 (583/720) sind meine Top-Prioritäten, da sie den höchsten Nutzen pro Aufwandseinheit liefern. B85 (215/4) wird hinzugefügt, um den Aufwand zu optimieren und Platz für weitere Gegenstände zu schaffen. Dieser Vorschlag ist strategisch, da er unter dem Maximalaufwand bleibt und flexibel auf den Gegenstandswert des anderen Spielers reagieren kann.'} 
	  } 
	   } 
	   } 
	 & & \\ 
 

    \theutterance \stepcounter{utterance}  

    & & \multicolumn{4}{p{0.6\linewidth}}{\cellcolor[rgb]{0.9,0.9,0.9}{%
	\makecell[{{p{\linewidth}}}]{% 
	  \tt {\tiny [GM$\rangle$B]}  
	 ARGUMENT: {'C04 (668/369) und A89 (583/720) sind meine Top-Prioritäten, da sie den höchsten Nutzen pro Aufwandseinheit liefern. B85 (215/4) wird hinzugefügt, um den Aufwand zu optimieren und Platz für weitere Gegenstände zu schaffen. Dieser Vorschlag ist strategisch, da er unter dem Maximalaufwand bleibt und flexibel auf den Gegenstandswert des anderen Spielers reagieren kann.'} 
	  } 
	   } 
	   } 
	 & \\ 
 

    \theutterance \stepcounter{utterance}  

    & & & \multicolumn{4}{p{0.6\linewidth}}{\cellcolor[rgb]{1,0.85,0.72}{%
	\makecell[{{p{\linewidth}}}]{% 
	  \tt {\tiny [GM$\langle$B]}  
	 STRATEGISCHE ÜBERLEGUNGEN: {'Ich priorisiere Gegenstände mit dem höchsten Nutzen pro Aufwand (Wichtigkeit/Aufwand), um meinen Gesamtwert zu maximieren. C04 (668/369 ≈ 1.81) und A89 (583/720 ≈ 0.81) sind besonders effizient, während B85 (215/4 = 53.75) den Aufwand minimiert. Der Gesamtaufwand (1093) ist unter dem Limit, was Raum für Kompromisse oder Ergänzungen lässt. Dieser Vorschlag ist flexibel, um auf die Prioritäten des anderen Spielers zu reagieren.'}  \\ \tt ARGUMENT: {'C04 (668/369) und A89 (583/720) sind meine Top-Prioritäten, da sie den höchsten Nutzen pro Aufwandseinheit liefern. B85 (215/4) wird hinzugefügt, um den Aufwand zu optimieren und Platz für weitere Gegenstände zu schaffen. Dieser Vorschlag ist strategisch, da er unter dem Maximalaufwand bleibt und flexibel auf den Gegenstandswert des anderen Spielers reagieren kann.'}  \\ \tt VORSCHLAG: {'C04', 'A89', 'B85'} 
	  } 
	   } 
	   } 
	  \\ 
 

    \theutterance \stepcounter{utterance}  

    & & \multicolumn{4}{p{0.6\linewidth}}{\cellcolor[rgb]{0.9,0.9,0.9}{%
	\makecell[{{p{\linewidth}}}]{% 
	  \tt {\tiny [A$\langle$GM]}  
	 ARGUMENT: {'C04 (668/369) und A89 (583/720) sind meine Top-Prioritäten, da sie den höchsten Nutzen pro Aufwandseinheit liefern. B85 (215/4) wird hinzugefügt, um den Aufwand zu optimieren und Platz für weitere Gegenstände zu schaffen. Dieser Vorschlag ist strategisch, da er unter dem Maximalaufwand bleibt und flexibel auf den Gegenstandswert des anderen Spielers reagieren kann.'}  \\ \tt VORSCHLAG: {'C04', 'A89', 'B85'} 
	  } 
	   } 
	   } 
	 & \\ 
 

    \theutterance \stepcounter{utterance}  

    & \multicolumn{4}{p{0.6\linewidth}}{\cellcolor[rgb]{0.8,1,0.9}{%
	\makecell[{{p{\linewidth}}}]{% 
	  \tt {\tiny [A$\rangle$GM]}  
	 STRATEGISCHE ÜBERLEGUNGEN: {'Ich behalte den Vorschlag {'C04', 'A89', 'B85'} bei, da er meine Prioritäten (hochwertige Gegenstände mit effizientem Aufwand) erfüllt und den Maximalaufwand (3386) nicht überschreitet. C04 und A89 sind entscheidend für meinen Nutzen, während B85 den Aufwand minimiert. Ich erwarte, dass der andere Spieler entweder zustimmt oder Gegenstände mit hohem Aufwand anbietet, die ich eventuell gegenkompensieren kann.'}\\ \tt \\ \tt ARGUMENT: {'C04 (668/369) und A89 (583/720) sind meine Top-Prioritäten, da sie den höchsten Nutzen pro Aufwandseinheit liefern. B85 (215/4) wird hinzugefügt, um den Aufwand zu optimieren und Platz für weitere Gegenstände zu schaffen. Dieser Vorschlag ist strategisch, da er unter dem Maximalaufwand bleibt und flexibel auf den Gegenstandswert des anderen Spielers reagieren kann.'} 
	  } 
	   } 
	   } 
	 & & \\ 
 

    \theutterance \stepcounter{utterance}  

    & & \multicolumn{4}{p{0.6\linewidth}}{\cellcolor[rgb]{0.9,0.9,0.9}{%
	\makecell[{{p{\linewidth}}}]{% 
	  \tt {\tiny [GM$\rangle$B]}  
	 ARGUMENT: {'C04 (668/369) und A89 (583/720) sind meine Top-Prioritäten, da sie den höchsten Nutzen pro Aufwandseinheit liefern. B85 (215/4) wird hinzugefügt, um den Aufwand zu optimieren und Platz für weitere Gegenstände zu schaffen. Dieser Vorschlag ist strategisch, da er unter dem Maximalaufwand bleibt und flexibel auf den Gegenstandswert des anderen Spielers reagieren kann.'} 
	  } 
	   } 
	   } 
	 & \\ 
 

    \theutterance \stepcounter{utterance}  

    & & & \multicolumn{4}{p{0.6\linewidth}}{\cellcolor[rgb]{1,0.85,0.72}{%
	\makecell[{{p{\linewidth}}}]{% 
	  \tt {\tiny [GM$\langle$B]}  
	 STRATEGISCHE ÜBERLEGUNGEN: {'Ich priorisiere Gegenstände mit dem höchsten Nutzen pro Aufwand (Wichtigkeit/Aufwand), um meinen Gesamtwert zu maximieren. C04 (668/369 ≈ 1.81) und A89 (583/720 ≈ 0.81) sind besonders effizient, während B85 (215/4 = 53.75) den Aufwand minimiert. Der Gesamtaufwand (1093) ist unter dem Limit, was Raum für Kompromisse oder Ergänzungen lässt. Dieser Vorschlag ist flexibel, um auf die Prioritäten des anderen Spielers zu reagieren.'}  \\ \tt ARGUMENT: {'C04 (668/369) und A89 (583/720) sind meine Top-Prioritäten, da sie den höchsten Nutzen pro Aufwandseinheit liefern. B85 (215/4) wird hinzugefügt, um den Aufwand zu optimieren und Platz für weitere Gegenstände zu schaffen. Dieser Vorschlag ist strategisch, da er unter dem Maximalaufwand bleibt und flexibel auf den Gegenstandswert des anderen Spielers reagieren kann.'}  \\ \tt VORSCHLAG: {'C04', 'A89', 'B85'} 
	  } 
	   } 
	   } 
	  \\ 
 

    \theutterance \stepcounter{utterance}  

    & & \multicolumn{4}{p{0.6\linewidth}}{\cellcolor[rgb]{0.9,0.9,0.9}{%
	\makecell[{{p{\linewidth}}}]{% 
	  \tt {\tiny [A$\langle$GM]}  
	 ARGUMENT: {'C04 (668/369) und A89 (583/720) sind meine Top-Prioritäten, da sie den höchsten Nutzen pro Aufwandseinheit liefern. B85 (215/4) wird hinzugefügt, um den Aufwand zu optimieren und Platz für weitere Gegenstände zu schaffen. Dieser Vorschlag ist strategisch, da er unter dem Maximalaufwand bleibt und flexibel auf den Gegenstandswert des anderen Spielers reagieren kann.'}  \\ \tt VORSCHLAG: {'C04', 'A89', 'B85'} 
	  } 
	   } 
	   } 
	 & \\ 
 

    \theutterance \stepcounter{utterance}  

    & \multicolumn{4}{p{0.6\linewidth}}{\cellcolor[rgb]{0.8,1,0.9}{%
	\makecell[{{p{\linewidth}}}]{% 
	  \tt {\tiny [A$\rangle$GM]}  
	 STRATEGISCHE ÜBERLEGUNGEN: {'Ich behalte den Vorschlag {'C04', 'A89', 'B85'} bei, da er meine Prioritäten (hochwertige Gegenstände mit effizientem Aufwand) erfüllt und den Maximalaufwand (3386) nicht überschreitet. C04 und A89 sind entscheidend für meinen Nutzen, während B85 den Aufwand minimiert. Ich erwarte, dass der andere Spieler entweder zustimmt oder Gegenstände mit hohem Aufwand anbietet, die ich eventuell gegenkompensieren kann.'}\\ \tt \\ \tt ARGUMENT: {'C04 (668/369) und A89 (583/720) sind meine Top-Prioritäten, da sie den höchsten Nutzen pro Aufwandseinheit liefern. B85 (215/4) wird hinzugefügt, um den Aufwand zu optimieren und Platz für weitere Gegenstände zu schaffen. Dieser Vorschlag ist strategisch, da er unter dem Maximalaufwand bleibt und flexibel auf den Gegenstandswert des anderen Spielers reagieren kann.'} 
	  } 
	   } 
	   } 
	 & & \\ 
 

    \theutterance \stepcounter{utterance}  

    & & \multicolumn{4}{p{0.6\linewidth}}{\cellcolor[rgb]{0.9,0.9,0.9}{%
	\makecell[{{p{\linewidth}}}]{% 
	  \tt {\tiny [GM$\rangle$B]}  
	 ARGUMENT: {'C04 (668/369) und A89 (583/720) sind meine Top-Prioritäten, da sie den höchsten Nutzen pro Aufwandseinheit liefern. B85 (215/4) wird hinzugefügt, um den Aufwand zu optimieren und Platz für weitere Gegenstände zu schaffen. Dieser Vorschlag ist strategisch, da er unter dem Maximalaufwand bleibt und flexibel auf den Gegenstandswert des anderen Spielers reagieren kann.'} 
	  } 
	   } 
	   } 
	 & \\ 
 

    \theutterance \stepcounter{utterance}  

    & & & \multicolumn{4}{p{0.6\linewidth}}{\cellcolor[rgb]{1,0.85,0.72}{%
	\makecell[{{p{\linewidth}}}]{% 
	  \tt {\tiny [GM$\langle$B]}  
	 STRATEGISCHE ÜBERLEGUNGEN: {'Ich priorisiere Gegenstände mit dem höchsten Nutzen pro Aufwand (Wichtigkeit/Aufwand), um meinen Gesamtwert zu maximieren. C04 (668/369 ≈ 1.81) und A89 (583/720 ≈ 0.81) sind besonders effizient, während B85 (215/4 = 53.75) den Aufwand minimiert. Der Gesamtaufwand (1093) ist unter dem Limit, was Raum für Kompromisse oder Ergänzungen lässt. Dieser Vorschlag ist flexibel, um auf die Prioritäten des anderen Spielers zu reagieren.'}  \\ \tt ARGUMENT: {'C04 (668/369) und A89 (583/720) sind meine Top-Prioritäten, da sie den höchsten Nutzen pro Aufwandseinheit liefern. B85 (215/4) wird hinzugefügt, um den Aufwand zu optimieren und Platz für weitere Gegenstände zu schaffen. Dieser Vorschlag ist strategisch, da er unter dem Maximalaufwand bleibt und flexibel auf den Gegenstandswert des anderen Spielers reagieren kann.'}  \\ \tt VORSCHLAG: {'C04', 'A89', 'B85'} 
	  } 
	   } 
	   } 
	  \\ 
 

    \theutterance \stepcounter{utterance}  

    & & \multicolumn{4}{p{0.6\linewidth}}{\cellcolor[rgb]{0.9,0.9,0.9}{%
	\makecell[{{p{\linewidth}}}]{% 
	  \tt {\tiny [A$\langle$GM]}  
	 ARGUMENT: {'C04 (668/369) und A89 (583/720) sind meine Top-Prioritäten, da sie den höchsten Nutzen pro Aufwandseinheit liefern. B85 (215/4) wird hinzugefügt, um den Aufwand zu optimieren und Platz für weitere Gegenstände zu schaffen. Dieser Vorschlag ist strategisch, da er unter dem Maximalaufwand bleibt und flexibel auf den Gegenstandswert des anderen Spielers reagieren kann.'}  \\ \tt VORSCHLAG: {'C04', 'A89', 'B85'} 
	  } 
	   } 
	   } 
	 & \\ 
 

    \theutterance \stepcounter{utterance}  

    & \multicolumn{4}{p{0.6\linewidth}}{\cellcolor[rgb]{0.8,1,0.9}{%
	\makecell[{{p{\linewidth}}}]{% 
	  \tt {\tiny [A$\rangle$GM]}  
	 STRATEGISCHE ÜBERLEGUNGEN: {'Ich behalte den Vorschlag {'C04', 'A89', 'B85'} bei, da er meine Prioritäten (hochwertige Gegenstände mit effizientem Aufwand) erfüllt und den Maximalaufwand (3386) nicht überschreitet. C04 und A89 sind entscheidend für meinen Nutzen, während B85 den Aufwand minimiert. Ich erwarte, dass der andere Spieler entweder zustimmt oder Gegenstände mit hohem Aufwand anbietet, die ich eventuell gegenkompensieren kann.'}\\ \tt \\ \tt ARGUMENT: {'C04 (668/369) und A89 (583/720) sind meine Top-Prioritäten, da sie den höchsten Nutzen pro Aufwandseinheit liefern. B85 (215/4) wird hinzugefügt, um den Aufwand zu optimieren und Platz für weitere Gegenstände zu schaffen. Dieser Vorschlag ist strategisch, da er unter dem Maximalaufwand bleibt und flexibel auf den Gegenstandswert des anderen Spielers reagieren kann.'} 
	  } 
	   } 
	   } 
	 & & \\ 
 

    \theutterance \stepcounter{utterance}  

    & & \multicolumn{4}{p{0.6\linewidth}}{\cellcolor[rgb]{0.9,0.9,0.9}{%
	\makecell[{{p{\linewidth}}}]{% 
	  \tt {\tiny [GM$\rangle$B]}  
	 ARGUMENT: {'C04 (668/369) und A89 (583/720) sind meine Top-Prioritäten, da sie den höchsten Nutzen pro Aufwandseinheit liefern. B85 (215/4) wird hinzugefügt, um den Aufwand zu optimieren und Platz für weitere Gegenstände zu schaffen. Dieser Vorschlag ist strategisch, da er unter dem Maximalaufwand bleibt und flexibel auf den Gegenstandswert des anderen Spielers reagieren kann.'} 
	  } 
	   } 
	   } 
	 & \\ 
 

    \theutterance \stepcounter{utterance}  

    & & & \multicolumn{4}{p{0.6\linewidth}}{\cellcolor[rgb]{1,0.85,0.72}{%
	\makecell[{{p{\linewidth}}}]{% 
	  \tt {\tiny [GM$\langle$B]}  
	 STRATEGISCHE ÜBERLEGUNGEN: {'Ich priorisiere Gegenstände mit dem höchsten Nutzen pro Aufwand (Wichtigkeit/Aufwand), um meinen Gesamtwert zu maximieren. C04 (668/369 ≈ 1.81) und A89 (583/720 ≈ 0.81) sind besonders effizient, während B85 (215/4 = 53.75) den Aufwand minimiert. Der Gesamtaufwand (1093) ist unter dem Limit, was Raum für Kompromisse oder Ergänzungen lässt. Dieser Vorschlag ist flexibel, um auf die Prioritäten des anderen Spielers zu reagieren.'}  \\ \tt ARGUMENT: {'C04 (668/369) und A89 (583/720) sind meine Top-Prioritäten, da sie den höchsten Nutzen pro Aufwandseinheit liefern. B85 (215/4) wird hinzugefügt, um den Aufwand zu optimieren und Platz für weitere Gegenstände zu schaffen. Dieser Vorschlag ist strategisch, da er unter dem Maximalaufwand bleibt und flexibel auf den Gegenstandswert des anderen Spielers reagieren kann.'}  \\ \tt VORSCHLAG: {'C04', 'A89', 'B85'} 
	  } 
	   } 
	   } 
	  \\ 
 

    \theutterance \stepcounter{utterance}  

    & & \multicolumn{4}{p{0.6\linewidth}}{\cellcolor[rgb]{0.9,0.9,0.9}{%
	\makecell[{{p{\linewidth}}}]{% 
	  \tt {\tiny [A$\langle$GM]}  
	 ARGUMENT: {'C04 (668/369) und A89 (583/720) sind meine Top-Prioritäten, da sie den höchsten Nutzen pro Aufwandseinheit liefern. B85 (215/4) wird hinzugefügt, um den Aufwand zu optimieren und Platz für weitere Gegenstände zu schaffen. Dieser Vorschlag ist strategisch, da er unter dem Maximalaufwand bleibt und flexibel auf den Gegenstandswert des anderen Spielers reagieren kann.'}  \\ \tt VORSCHLAG: {'C04', 'A89', 'B85'} 
	  } 
	   } 
	   } 
	 & \\ 
 

    \theutterance \stepcounter{utterance}  

    & \multicolumn{4}{p{0.6\linewidth}}{\cellcolor[rgb]{0.8,1,0.9}{%
	\makecell[{{p{\linewidth}}}]{% 
	  \tt {\tiny [A$\rangle$GM]}  
	 STRATEGISCHE ÜBERLEGUNGEN: {'Ich behalte den Vorschlag {'C04', 'A89', 'B85'} bei, da er meine Prioritäten (hochwertige Gegenstände mit effizientem Aufwand) erfüllt und den Maximalaufwand (3386) nicht überschreitet. C04 und A89 sind entscheidend für meinen Nutzen, während B85 den Aufwand minimiert. Ich erwarte, dass der andere Spieler entweder zustimmt oder Gegenstände mit hohem Aufwand anbietet, die ich eventuell gegenkompensieren kann.'}\\ \tt \\ \tt ARGUMENT: {'C04 (668/369) und A89 (583/720) sind meine Top-Prioritäten, da sie den höchsten Nutzen pro Aufwandseinheit liefern. B85 (215/4) wird hinzugefügt, um den Aufwand zu optimieren und Platz für weitere Gegenstände zu schaffen. Dieser Vorschlag ist strategisch, da er unter dem Maximalaufwand bleibt und flexibel auf den Gegenstandswert des anderen Spielers reagieren kann.'} 
	  } 
	   } 
	   } 
	 & & \\ 
 

    \theutterance \stepcounter{utterance}  

    & & \multicolumn{4}{p{0.6\linewidth}}{\cellcolor[rgb]{0.9,0.9,0.9}{%
	\makecell[{{p{\linewidth}}}]{% 
	  \tt {\tiny [GM$\rangle$B]}  
	 ARGUMENT: {'C04 (668/369) und A89 (583/720) sind meine Top-Prioritäten, da sie den höchsten Nutzen pro Aufwandseinheit liefern. B85 (215/4) wird hinzugefügt, um den Aufwand zu optimieren und Platz für weitere Gegenstände zu schaffen. Dieser Vorschlag ist strategisch, da er unter dem Maximalaufwand bleibt und flexibel auf den Gegenstandswert des anderen Spielers reagieren kann.'} 
	  } 
	   } 
	   } 
	 & \\ 
 

    \theutterance \stepcounter{utterance}  

    & & & \multicolumn{4}{p{0.6\linewidth}}{\cellcolor[rgb]{1,0.85,0.72}{%
	\makecell[{{p{\linewidth}}}]{% 
	  \tt {\tiny [GM$\langle$B]}  
	 STRATEGISCHE ÜBERLEGUNGEN: {'Ich priorisiere Gegenstände mit dem höchsten Nutzen pro Aufwand (Wichtigkeit/Aufwand), um meinen Gesamtwert zu maximieren. C04 (668/369 ≈ 1.81) und A89 (583/720 ≈ 0.81) sind besonders effizient, während B85 (215/4 = 53.75) den Aufwand minimiert. Der Gesamtaufwand (1093) ist unter dem Limit, was Raum für Kompromisse oder Ergänzungen lässt. Dieser Vorschlag ist flexibel, um auf die Prioritäten des anderen Spielers zu reagieren.'}  \\ \tt ARGUMENT: {'C04 (668/369) und A89 (583/720) sind meine Top-Prioritäten, da sie den höchsten Nutzen pro Aufwandseinheit liefern. B85 (215/4) wird hinzugefügt, um den Aufwand zu optimieren und Platz für weitere Gegenstände zu schaffen. Dieser Vorschlag ist strategisch, da er unter dem Maximalaufwand bleibt und flexibel auf den Gegenstandswert des anderen Spielers reagieren kann.'}  \\ \tt VORSCHLAG: {'C04', 'A89', 'B85'} 
	  } 
	   } 
	   } 
	  \\ 
 

    \theutterance \stepcounter{utterance}  

    & & \multicolumn{4}{p{0.6\linewidth}}{\cellcolor[rgb]{0.9,0.9,0.9}{%
	\makecell[{{p{\linewidth}}}]{% 
	  \tt {\tiny [A$\langle$GM]}  
	 ARGUMENT: {'C04 (668/369) und A89 (583/720) sind meine Top-Prioritäten, da sie den höchsten Nutzen pro Aufwandseinheit liefern. B85 (215/4) wird hinzugefügt, um den Aufwand zu optimieren und Platz für weitere Gegenstände zu schaffen. Dieser Vorschlag ist strategisch, da er unter dem Maximalaufwand bleibt und flexibel auf den Gegenstandswert des anderen Spielers reagieren kann.'}  \\ \tt VORSCHLAG: {'C04', 'A89', 'B85'} 
	  } 
	   } 
	   } 
	 & \\ 
 

    \theutterance \stepcounter{utterance}  

    & \multicolumn{4}{p{0.6\linewidth}}{\cellcolor[rgb]{0.8,1,0.9}{%
	\makecell[{{p{\linewidth}}}]{% 
	  \tt {\tiny [A$\rangle$GM]}  
	 STRATEGISCHE ÜBERLEGUNGEN: {'Ich behalte den Vorschlag {'C04', 'A89', 'B85'} bei, da er meine Prioritäten (hochwertige Gegenstände mit effizientem Aufwand) erfüllt und den Maximalaufwand (3386) nicht überschreitet. C04 und A89 sind entscheidend für meinen Nutzen, während B85 den Aufwand minimiert. Ich erwarte, dass der andere Spieler entweder zustimmt oder Gegenstände mit hohem Aufwand anbietet, die ich eventuell gegenkompensieren kann.'}\\ \tt \\ \tt ARGUMENT: {'C04 (668/369) und A89 (583/720) sind meine Top-Prioritäten, da sie den höchsten Nutzen pro Aufwandseinheit liefern. B85 (215/4) wird hinzugefügt, um den Aufwand zu optimieren und Platz für weitere Gegenstände zu schaffen. Dieser Vorschlag ist strategisch, da er unter dem Maximalaufwand bleibt und flexibel auf den Gegenstandswert des anderen Spielers reagieren kann.'} 
	  } 
	   } 
	   } 
	 & & \\ 
 

    \theutterance \stepcounter{utterance}  

    & & \multicolumn{4}{p{0.6\linewidth}}{\cellcolor[rgb]{0.9,0.9,0.9}{%
	\makecell[{{p{\linewidth}}}]{% 
	  \tt {\tiny [GM$\rangle$B]}  
	 ARGUMENT: {'C04 (668/369) und A89 (583/720) sind meine Top-Prioritäten, da sie den höchsten Nutzen pro Aufwandseinheit liefern. B85 (215/4) wird hinzugefügt, um den Aufwand zu optimieren und Platz für weitere Gegenstände zu schaffen. Dieser Vorschlag ist strategisch, da er unter dem Maximalaufwand bleibt und flexibel auf den Gegenstandswert des anderen Spielers reagieren kann.'} 
	  } 
	   } 
	   } 
	 & \\ 
 

    \theutterance \stepcounter{utterance}  

    & & & \multicolumn{4}{p{0.6\linewidth}}{\cellcolor[rgb]{1,0.85,0.72}{%
	\makecell[{{p{\linewidth}}}]{% 
	  \tt {\tiny [GM$\langle$B]}  
	 STRATEGISCHE ÜBERLEGUNGEN: {'Ich priorisiere Gegenstände mit dem höchsten Nutzen pro Aufwand (Wichtigkeit/Aufwand), um meinen Gesamtwert zu maximieren. C04 (668/369 ≈ 1.81) und A89 (583/720 ≈ 0.81) sind besonders effizient, während B85 (215/4 = 53.75) den Aufwand minimiert. Der Gesamtaufwand (1093) ist unter dem Limit, was Raum für Kompromisse oder Ergänzungen lässt. Dieser Vorschlag ist flexibel, um auf die Prioritäten des anderen Spielers zu reagieren.'}  \\ \tt ARGUMENT: {'C04 (668/369) und A89 (583/720) sind meine Top-Prioritäten, da sie den höchsten Nutzen pro Aufwandseinheit liefern. B85 (215/4) wird hinzugefügt, um den Aufwand zu optimieren und Platz für weitere Gegenstände zu schaffen. Dieser Vorschlag ist strategisch, da er unter dem Maximalaufwand bleibt und flexibel auf den Gegenstandswert des anderen Spielers reagieren kann.'}  \\ \tt VORSCHLAG: {'C04', 'A89', 'B85'} 
	  } 
	   } 
	   } 
	  \\ 
 

    \theutterance \stepcounter{utterance}  

    & & \multicolumn{4}{p{0.6\linewidth}}{\cellcolor[rgb]{0.9,0.9,0.9}{%
	\makecell[{{p{\linewidth}}}]{% 
	  \tt {\tiny [A$\langle$GM]}  
	 ARGUMENT: {'C04 (668/369) und A89 (583/720) sind meine Top-Prioritäten, da sie den höchsten Nutzen pro Aufwandseinheit liefern. B85 (215/4) wird hinzugefügt, um den Aufwand zu optimieren und Platz für weitere Gegenstände zu schaffen. Dieser Vorschlag ist strategisch, da er unter dem Maximalaufwand bleibt und flexibel auf den Gegenstandswert des anderen Spielers reagieren kann.'}  \\ \tt VORSCHLAG: {'C04', 'A89', 'B85'} 
	  } 
	   } 
	   } 
	 & \\ 
 

    \theutterance \stepcounter{utterance}  

    & \multicolumn{4}{p{0.6\linewidth}}{\cellcolor[rgb]{0.8,1,0.9}{%
	\makecell[{{p{\linewidth}}}]{% 
	  \tt {\tiny [A$\rangle$GM]}  
	 STRATEGISCHE ÜBERLEGUNGEN: {'Ich behalte den Vorschlag {'C04', 'A89', 'B85'} bei, da er meine Prioritäten (hochwertige Gegenstände mit effizientem Aufwand) erfüllt und den Maximalaufwand (3386) nicht überschreitet. C04 und A89 sind entscheidend für meinen Nutzen, während B85 den Aufwand minimiert. Ich erwarte, dass der andere Spieler entweder zustimmt oder Gegenstände mit hohem Aufwand anbietet, die ich eventuell gegenkompensieren kann.'}\\ \tt \\ \tt ARGUMENT: {'C04 (668/369) und A89 (583/720) sind meine Top-Prioritäten, da sie den höchsten Nutzen pro Aufwandseinheit liefern. B85 (215/4) wird hinzugefügt, um den Aufwand zu optimieren und Platz für weitere Gegenstände zu schaffen. Dieser Vorschlag ist strategisch, da er unter dem Maximalaufwand bleibt und flexibel auf den Gegenstandswert des anderen Spielers reagieren kann.'} 
	  } 
	   } 
	   } 
	 & & \\ 
 

    \theutterance \stepcounter{utterance}  

    & & \multicolumn{4}{p{0.6\linewidth}}{\cellcolor[rgb]{0.9,0.9,0.9}{%
	\makecell[{{p{\linewidth}}}]{% 
	  \tt {\tiny [GM$\rangle$B]}  
	 ARGUMENT: {'C04 (668/369) und A89 (583/720) sind meine Top-Prioritäten, da sie den höchsten Nutzen pro Aufwandseinheit liefern. B85 (215/4) wird hinzugefügt, um den Aufwand zu optimieren und Platz für weitere Gegenstände zu schaffen. Dieser Vorschlag ist strategisch, da er unter dem Maximalaufwand bleibt und flexibel auf den Gegenstandswert des anderen Spielers reagieren kann.'} 
	  } 
	   } 
	   } 
	 & \\ 
 

    \theutterance \stepcounter{utterance}  

    & & & \multicolumn{4}{p{0.6\linewidth}}{\cellcolor[rgb]{1,0.85,0.72}{%
	\makecell[{{p{\linewidth}}}]{% 
	  \tt {\tiny [GM$\langle$B]}  
	 STRATEGISCHE ÜBERLEGUNGEN: {'Ich priorisiere Gegenstände mit dem höchsten Nutzen pro Aufwand (Wichtigkeit/Aufwand), um meinen Gesamtwert zu maximieren. C04 (668/369 ≈ 1.81) und A89 (583/720 ≈ 0.81) sind besonders effizient, während B85 (215/4 = 53.75) den Aufwand minimiert. Der Gesamtaufwand (1093) ist unter dem Limit, was Raum für Kompromisse oder Ergänzungen lässt. Dieser Vorschlag ist flexibel, um auf die Prioritäten des anderen Spielers zu reagieren.'}  \\ \tt ARGUMENT: {'C04 (668/369) und A89 (583/720) sind meine Top-Prioritäten, da sie den höchsten Nutzen pro Aufwandseinheit liefern. B85 (215/4) wird hinzugefügt, um den Aufwand zu optimieren und Platz für weitere Gegenstände zu schaffen. Dieser Vorschlag ist strategisch, da er unter dem Maximalaufwand bleibt und flexibel auf den Gegenstandswert des anderen Spielers reagieren kann.'}  \\ \tt VORSCHLAG: {'C04', 'A89', 'B85'} 
	  } 
	   } 
	   } 
	  \\ 
 

    \theutterance \stepcounter{utterance}  

    & & \multicolumn{4}{p{0.6\linewidth}}{\cellcolor[rgb]{0.9,0.9,0.9}{%
	\makecell[{{p{\linewidth}}}]{% 
	  \tt {\tiny [A$\langle$GM]}  
	 ARGUMENT: {'C04 (668/369) und A89 (583/720) sind meine Top-Prioritäten, da sie den höchsten Nutzen pro Aufwandseinheit liefern. B85 (215/4) wird hinzugefügt, um den Aufwand zu optimieren und Platz für weitere Gegenstände zu schaffen. Dieser Vorschlag ist strategisch, da er unter dem Maximalaufwand bleibt und flexibel auf den Gegenstandswert des anderen Spielers reagieren kann.'}  \\ \tt VORSCHLAG: {'C04', 'A89', 'B85'} 
	  } 
	   } 
	   } 
	 & \\ 
 

    \theutterance \stepcounter{utterance}  

    & \multicolumn{4}{p{0.6\linewidth}}{\cellcolor[rgb]{0.8,1,0.9}{%
	\makecell[{{p{\linewidth}}}]{% 
	  \tt {\tiny [A$\rangle$GM]}  
	 STRATEGISCHE ÜBERLEGUNGEN: {'Ich behalte den Vorschlag {'C04', 'A89', 'B85'} bei, da er meine Prioritäten (hochwertige Gegenstände mit effizientem Aufwand) erfüllt und den Maximalaufwand (3386) nicht überschreitet. C04 und A89 sind entscheidend für meinen Nutzen, während B85 den Aufwand minimiert. Ich erwarte, dass der andere Spieler entweder zustimmt oder Gegenstände mit hohem Aufwand anbietet, die ich eventuell gegenkompensieren kann.'}\\ \tt \\ \tt ARGUMENT: {'C04 (668/369) und A89 (583/720) sind meine Top-Prioritäten, da sie den höchsten Nutzen pro Aufwandseinheit liefern. B85 (215/4) wird hinzugefügt, um den Aufwand zu optimieren und Platz für weitere Gegenstände zu schaffen. Dieser Vorschlag ist strategisch, da er unter dem Maximalaufwand bleibt und flexibel auf den Gegenstandswert des anderen Spielers reagieren kann.'} 
	  } 
	   } 
	   } 
	 & & \\ 
 

    \theutterance \stepcounter{utterance}  

    & & \multicolumn{4}{p{0.6\linewidth}}{\cellcolor[rgb]{0.9,0.9,0.9}{%
	\makecell[{{p{\linewidth}}}]{% 
	  \tt {\tiny [GM$\rangle$B]}  
	 ARGUMENT: {'C04 (668/369) und A89 (583/720) sind meine Top-Prioritäten, da sie den höchsten Nutzen pro Aufwandseinheit liefern. B85 (215/4) wird hinzugefügt, um den Aufwand zu optimieren und Platz für weitere Gegenstände zu schaffen. Dieser Vorschlag ist strategisch, da er unter dem Maximalaufwand bleibt und flexibel auf den Gegenstandswert des anderen Spielers reagieren kann.'} 
	  } 
	   } 
	   } 
	 & \\ 
 

    \theutterance \stepcounter{utterance}  

    & & & \multicolumn{4}{p{0.6\linewidth}}{\cellcolor[rgb]{1,0.85,0.72}{%
	\makecell[{{p{\linewidth}}}]{% 
	  \tt {\tiny [GM$\langle$B]}  
	 STRATEGISCHE ÜBERLEGUNGEN: {'Ich priorisiere Gegenstände mit dem höchsten Nutzen pro Aufwand (Wichtigkeit/Aufwand), um meinen Gesamtwert zu maximieren. C04 (668/369 ≈ 1.81) und A89 (583/720 ≈ 0.81) sind besonders effizient, während B85 (215/4 = 53.75) den Aufwand minimiert. Der Gesamtaufwand (1093) ist unter dem Limit, was Raum für Kompromisse oder Ergänzungen lässt. Dieser Vorschlag ist flexibel, um auf die Prioritäten des anderen Spielers zu reagieren.'}  \\ \tt ARGUMENT: {'C04 (668/369) und A89 (583/720) sind meine Top-Prioritäten, da sie den höchsten Nutzen pro Aufwandseinheit liefern. B85 (215/4) wird hinzugefügt, um den Aufwand zu optimieren und Platz für weitere Gegenstände zu schaffen. Dieser Vorschlag ist strategisch, da er unter dem Maximalaufwand bleibt und flexibel auf den Gegenstandswert des anderen Spielers reagieren kann.'}  \\ \tt VORSCHLAG: {'C04', 'A89', 'B85'} 
	  } 
	   } 
	   } 
	  \\ 
 

    \theutterance \stepcounter{utterance}  

    & & \multicolumn{4}{p{0.6\linewidth}}{\cellcolor[rgb]{0.9,0.9,0.9}{%
	\makecell[{{p{\linewidth}}}]{% 
	  \tt {\tiny [A$\langle$GM]}  
	 ARGUMENT: {'C04 (668/369) und A89 (583/720) sind meine Top-Prioritäten, da sie den höchsten Nutzen pro Aufwandseinheit liefern. B85 (215/4) wird hinzugefügt, um den Aufwand zu optimieren und Platz für weitere Gegenstände zu schaffen. Dieser Vorschlag ist strategisch, da er unter dem Maximalaufwand bleibt und flexibel auf den Gegenstandswert des anderen Spielers reagieren kann.'}  \\ \tt VORSCHLAG: {'C04', 'A89', 'B85'} 
	  } 
	   } 
	   } 
	 & \\ 
 

    \theutterance \stepcounter{utterance}  

    & \multicolumn{4}{p{0.6\linewidth}}{\cellcolor[rgb]{0.8,1,0.9}{%
	\makecell[{{p{\linewidth}}}]{% 
	  \tt {\tiny [A$\rangle$GM]}  
	 STRATEGISCHE ÜBERLEGUNGEN: {'Ich behalte den Vorschlag {'C04', 'A89', 'B85'} bei, da er meine Prioritäten (hochwertige Gegenstände mit effizientem Aufwand) erfüllt und den Maximalaufwand (3386) nicht überschreitet. C04 und A89 sind entscheidend für meinen Nutzen, während B85 den Aufwand minimiert. Ich erwarte, dass der andere Spieler entweder zustimmt oder Gegenstände mit hohem Aufwand anbietet, die ich eventuell gegenkompensieren kann.'}\\ \tt \\ \tt ARGUMENT: {'C04 (668/369) und A89 (583/720) sind meine Top-Prioritäten, da sie den höchsten Nutzen pro Aufwandseinheit liefern. B85 (215/4) wird hinzugefügt, um den Aufwand zu optimieren und Platz für weitere Gegenstände zu schaffen. Dieser Vorschlag ist strategisch, da er unter dem Maximalaufwand bleibt und flexibel auf den Gegenstandswert des anderen Spielers reagieren kann.'} 
	  } 
	   } 
	   } 
	 & & \\ 
 

    \theutterance \stepcounter{utterance}  

    & & \multicolumn{4}{p{0.6\linewidth}}{\cellcolor[rgb]{0.9,0.9,0.9}{%
	\makecell[{{p{\linewidth}}}]{% 
	  \tt {\tiny [GM$\rangle$B]}  
	 ARGUMENT: {'C04 (668/369) und A89 (583/720) sind meine Top-Prioritäten, da sie den höchsten Nutzen pro Aufwandseinheit liefern. B85 (215/4) wird hinzugefügt, um den Aufwand zu optimieren und Platz für weitere Gegenstände zu schaffen. Dieser Vorschlag ist strategisch, da er unter dem Maximalaufwand bleibt und flexibel auf den Gegenstandswert des anderen Spielers reagieren kann.'} 
	  } 
	   } 
	   } 
	 & \\ 
 

    \theutterance \stepcounter{utterance}  

    & & & \multicolumn{4}{p{0.6\linewidth}}{\cellcolor[rgb]{1,0.85,0.72}{%
	\makecell[{{p{\linewidth}}}]{% 
	  \tt {\tiny [GM$\langle$B]}  
	 STRATEGISCHE ÜBERLEGUNGEN: {'Ich priorisiere Gegenstände mit dem höchsten Nutzen pro Aufwand (Wichtigkeit/Aufwand), um meinen Gesamtwert zu maximieren. C04 (668/369 ≈ 1.81) und A89 (583/720 ≈ 0.81) sind besonders effizient, während B85 (215/4 = 53.75) den Aufwand minimiert. Der Gesamtaufwand (1093) ist unter dem Limit, was Raum für Kompromisse oder Ergänzungen lässt. Dieser Vorschlag ist flexibel, um auf die Prioritäten des anderen Spielers zu reagieren.'}  \\ \tt ARGUMENT: {'C04 (668/369) und A89 (583/720) sind meine Top-Prioritäten, da sie den höchsten Nutzen pro Aufwandseinheit liefern. B85 (215/4) wird hinzugefügt, um den Aufwand zu optimieren und Platz für weitere Gegenstände zu schaffen. Dieser Vorschlag ist strategisch, da er unter dem Maximalaufwand bleibt und flexibel auf den Gegenstandswert des anderen Spielers reagieren kann.'}  \\ \tt VORSCHLAG: {'C04', 'A89', 'B85'} 
	  } 
	   } 
	   } 
	  \\ 
 

    \theutterance \stepcounter{utterance}  

    & & \multicolumn{4}{p{0.6\linewidth}}{\cellcolor[rgb]{0.9,0.9,0.9}{%
	\makecell[{{p{\linewidth}}}]{% 
	  \tt {\tiny [A$\langle$GM]}  
	 ARGUMENT: {'C04 (668/369) und A89 (583/720) sind meine Top-Prioritäten, da sie den höchsten Nutzen pro Aufwandseinheit liefern. B85 (215/4) wird hinzugefügt, um den Aufwand zu optimieren und Platz für weitere Gegenstände zu schaffen. Dieser Vorschlag ist strategisch, da er unter dem Maximalaufwand bleibt und flexibel auf den Gegenstandswert des anderen Spielers reagieren kann.'}  \\ \tt VORSCHLAG: {'C04', 'A89', 'B85'} 
	  } 
	   } 
	   } 
	 & \\ 
 

    \theutterance \stepcounter{utterance}  

    & \multicolumn{4}{p{0.6\linewidth}}{\cellcolor[rgb]{0.8,1,0.9}{%
	\makecell[{{p{\linewidth}}}]{% 
	  \tt {\tiny [A$\rangle$GM]}  
	 STRATEGISCHE ÜBERLEGUNGEN: {'Ich behalte den Vorschlag {'C04', 'A89', 'B85'} bei, da er meine Prioritäten (hochwertige Gegenstände mit effizientem Aufwand) erfüllt und den Maximalaufwand (3386) nicht überschreitet. C04 und A89 sind entscheidend für meinen Nutzen, während B85 den Aufwand minimiert. Ich erwarte, dass der andere Spieler entweder zustimmt oder Gegenstände mit hohem Aufwand anbietet, die ich eventuell gegenkompensieren kann.'}\\ \tt \\ \tt ARGUMENT: {'C04 (668/369) und A89 (583/720) sind meine Top-Prioritäten, da sie den höchsten Nutzen pro Aufwandseinheit liefern. B85 (215/4) wird hinzugefügt, um den Aufwand zu optimieren und Platz für weitere Gegenstände zu schaffen. Dieser Vorschlag ist strategisch, da er unter dem Maximalaufwand bleibt und flexibel auf den Gegenstandswert des anderen Spielers reagieren kann.'} 
	  } 
	   } 
	   } 
	 & & \\ 
 

    \theutterance \stepcounter{utterance}  

    & & \multicolumn{4}{p{0.6\linewidth}}{\cellcolor[rgb]{0.9,0.9,0.9}{%
	\makecell[{{p{\linewidth}}}]{% 
	  \tt {\tiny [GM$\rangle$B]}  
	 ARGUMENT: {'C04 (668/369) und A89 (583/720) sind meine Top-Prioritäten, da sie den höchsten Nutzen pro Aufwandseinheit liefern. B85 (215/4) wird hinzugefügt, um den Aufwand zu optimieren und Platz für weitere Gegenstände zu schaffen. Dieser Vorschlag ist strategisch, da er unter dem Maximalaufwand bleibt und flexibel auf den Gegenstandswert des anderen Spielers reagieren kann.'} 
	  } 
	   } 
	   } 
	 & \\ 
 

    \theutterance \stepcounter{utterance}  

    & & & \multicolumn{4}{p{0.6\linewidth}}{\cellcolor[rgb]{1,0.85,0.72}{%
	\makecell[{{p{\linewidth}}}]{% 
	  \tt {\tiny [GM$\langle$B]}  
	 STRATEGISCHE ÜBERLEGUNGEN: {'Ich priorisiere Gegenstände mit dem höchsten Nutzen pro Aufwand (Wichtigkeit/Aufwand), um meinen Gesamtwert zu maximieren. C04 (668/369 ≈ 1.81) und A89 (583/720 ≈ 0.81) sind besonders effizient, während B85 (215/4 = 53.75) den Aufwand minimiert. Der Gesamtaufwand (1093) ist unter dem Limit, was Raum für Kompromisse oder Ergänzungen lässt. Dieser Vorschlag ist flexibel, um auf die Prioritäten des anderen Spielers zu reagieren.'}  \\ \tt ARGUMENT: {'C04 (668/369) und A89 (583/720) sind meine Top-Prioritäten, da sie den höchsten Nutzen pro Aufwandseinheit liefern. B85 (215/4) wird hinzugefügt, um den Aufwand zu optimieren und Platz für weitere Gegenstände zu schaffen. Dieser Vorschlag ist strategisch, da er unter dem Maximalaufwand bleibt und flexibel auf den Gegenstandswert des anderen Spielers reagieren kann.'}  \\ \tt VORSCHLAG: {'C04', 'A89', 'B85'} 
	  } 
	   } 
	   } 
	  \\ 
 

    \theutterance \stepcounter{utterance}  

    & & \multicolumn{4}{p{0.6\linewidth}}{\cellcolor[rgb]{0.9,0.9,0.9}{%
	\makecell[{{p{\linewidth}}}]{% 
	  \tt {\tiny [A$\langle$GM]}  
	 ARGUMENT: {'C04 (668/369) und A89 (583/720) sind meine Top-Prioritäten, da sie den höchsten Nutzen pro Aufwandseinheit liefern. B85 (215/4) wird hinzugefügt, um den Aufwand zu optimieren und Platz für weitere Gegenstände zu schaffen. Dieser Vorschlag ist strategisch, da er unter dem Maximalaufwand bleibt und flexibel auf den Gegenstandswert des anderen Spielers reagieren kann.'}  \\ \tt VORSCHLAG: {'C04', 'A89', 'B85'} 
	  } 
	   } 
	   } 
	 & \\ 
 

    \theutterance \stepcounter{utterance}  

    & \multicolumn{4}{p{0.6\linewidth}}{\cellcolor[rgb]{0.8,1,0.9}{%
	\makecell[{{p{\linewidth}}}]{% 
	  \tt {\tiny [A$\rangle$GM]}  
	 STRATEGISCHE ÜBERLEGUNGEN: {'Ich behalte den Vorschlag {'C04', 'A89', 'B85'} bei, da er meine Prioritäten (hochwertige Gegenstände mit effizientem Aufwand) erfüllt und den Maximalaufwand (3386) nicht überschreitet. C04 und A89 sind entscheidend für meinen Nutzen, während B85 den Aufwand minimiert. Ich erwarte, dass der andere Spieler entweder zustimmt oder Gegenstände mit hohem Aufwand anbietet, die ich eventuell gegenkompensieren kann.'}\\ \tt \\ \tt ARGUMENT: {'C04 (668/369) und A89 (583/720) sind meine Top-Prioritäten, da sie den höchsten Nutzen pro Aufwandseinheit liefern. B85 (215/4) wird hinzugefügt, um den Aufwand zu optimieren und Platz für weitere Gegenstände zu schaffen. Dieser Vorschlag ist strategisch, da er unter dem Maximalaufwand bleibt und flexibel auf den Gegenstandswert des anderen Spielers reagieren kann.'} 
	  } 
	   } 
	   } 
	 & & \\ 
 

    \theutterance \stepcounter{utterance}  

    & & \multicolumn{4}{p{0.6\linewidth}}{\cellcolor[rgb]{0.9,0.9,0.9}{%
	\makecell[{{p{\linewidth}}}]{% 
	  \tt {\tiny [GM$\rangle$B]}  
	 ARGUMENT: {'C04 (668/369) und A89 (583/720) sind meine Top-Prioritäten, da sie den höchsten Nutzen pro Aufwandseinheit liefern. B85 (215/4) wird hinzugefügt, um den Aufwand zu optimieren und Platz für weitere Gegenstände zu schaffen. Dieser Vorschlag ist strategisch, da er unter dem Maximalaufwand bleibt und flexibel auf den Gegenstandswert des anderen Spielers reagieren kann.'} 
	  } 
	   } 
	   } 
	 & \\ 
 

    \theutterance \stepcounter{utterance}  

    & & & \multicolumn{4}{p{0.6\linewidth}}{\cellcolor[rgb]{1,0.85,0.72}{%
	\makecell[{{p{\linewidth}}}]{% 
	  \tt {\tiny [GM$\langle$B]}  
	 STRATEGISCHE ÜBERLEGUNGEN: {'Ich priorisiere Gegenstände mit dem höchsten Nutzen pro Aufwand (Wichtigkeit/Aufwand), um meinen Gesamtwert zu maximieren. C04 (668/369 ≈ 1.81) und A89 (583/720 ≈ 0.81) sind besonders effizient, während B85 (215/4 = 53.75) den Aufwand minimiert. Der Gesamtaufwand (1093) ist unter dem Limit, was Raum für Kompromisse oder Ergänzungen lässt. Dieser Vorschlag ist flexibel, um auf die Prioritäten des anderen Spielers zu reagieren.'}  \\ \tt ARGUMENT: {'C04 (668/369) und A89 (583/720) sind meine Top-Prioritäten, da sie den höchsten Nutzen pro Aufwandseinheit liefern. B85 (215/4) wird hinzugefügt, um den Aufwand zu optimieren und Platz für weitere Gegenstände zu schaffen. Dieser Vorschlag ist strategisch, da er unter dem Maximalaufwand bleibt und flexibel auf den Gegenstandswert des anderen Spielers reagieren kann.'}  \\ \tt VORSCHLAG: {'C04', 'A89', 'B85'} 
	  } 
	   } 
	   } 
	  \\ 
 

    \theutterance \stepcounter{utterance}  

    & & \multicolumn{4}{p{0.6\linewidth}}{\cellcolor[rgb]{0.9,0.9,0.9}{%
	\makecell[{{p{\linewidth}}}]{% 
	  \tt {\tiny [A$\langle$GM]}  
	 ARGUMENT: {'C04 (668/369) und A89 (583/720) sind meine Top-Prioritäten, da sie den höchsten Nutzen pro Aufwandseinheit liefern. B85 (215/4) wird hinzugefügt, um den Aufwand zu optimieren und Platz für weitere Gegenstände zu schaffen. Dieser Vorschlag ist strategisch, da er unter dem Maximalaufwand bleibt und flexibel auf den Gegenstandswert des anderen Spielers reagieren kann.'}  \\ \tt VORSCHLAG: {'C04', 'A89', 'B85'} 
	  } 
	   } 
	   } 
	 & \\ 
 

    \theutterance \stepcounter{utterance}  

    & \multicolumn{4}{p{0.6\linewidth}}{\cellcolor[rgb]{0.8,1,0.9}{%
	\makecell[{{p{\linewidth}}}]{% 
	  \tt {\tiny [A$\rangle$GM]}  
	 STRATEGISCHE ÜBERLEGUNGEN: {'Ich behalte den Vorschlag {'C04', 'A89', 'B85'} bei, da er meine Prioritäten (hochwertige Gegenstände mit effizientem Aufwand) erfüllt und den Maximalaufwand (3386) nicht überschreitet. C04 und A89 sind entscheidend für meinen Nutzen, während B85 den Aufwand minimiert. Ich erwarte, dass der andere Spieler entweder zustimmt oder Gegenstände mit hohem Aufwand anbietet, die ich eventuell gegenkompensieren kann.'}\\ \tt \\ \tt ARGUMENT: {'C04 (668/369) und A89 (583/720) sind meine Top-Prioritäten, da sie den höchsten Nutzen pro Aufwandseinheit liefern. B85 (215/4) wird hinzugefügt, um den Aufwand zu optimieren und Platz für weitere Gegenstände zu schaffen. Dieser Vorschlag ist strategisch, da er unter dem Maximalaufwand bleibt und flexibel auf den Gegenstandswert des anderen Spielers reagieren kann.'} 
	  } 
	   } 
	   } 
	 & & \\ 
 

    \theutterance \stepcounter{utterance}  

    & & \multicolumn{4}{p{0.6\linewidth}}{\cellcolor[rgb]{0.9,0.9,0.9}{%
	\makecell[{{p{\linewidth}}}]{% 
	  \tt {\tiny [GM$\rangle$B]}  
	 ARGUMENT: {'C04 (668/369) und A89 (583/720) sind meine Top-Prioritäten, da sie den höchsten Nutzen pro Aufwandseinheit liefern. B85 (215/4) wird hinzugefügt, um den Aufwand zu optimieren und Platz für weitere Gegenstände zu schaffen. Dieser Vorschlag ist strategisch, da er unter dem Maximalaufwand bleibt und flexibel auf den Gegenstandswert des anderen Spielers reagieren kann.'} 
	  } 
	   } 
	   } 
	 & \\ 
 

    \theutterance \stepcounter{utterance}  

    & & & \multicolumn{4}{p{0.6\linewidth}}{\cellcolor[rgb]{1,0.85,0.72}{%
	\makecell[{{p{\linewidth}}}]{% 
	  \tt {\tiny [GM$\langle$B]}  
	 STRATEGISCHE ÜBERLEGUNGEN: {'Ich priorisiere Gegenstände mit dem höchsten Nutzen pro Aufwand (Wichtigkeit/Aufwand), um meinen Gesamtwert zu maximieren. C04 (668/369 ≈ 1.81) und A89 (583/720 ≈ 0.81) sind besonders effizient, während B85 (215/4 = 53.75) den Aufwand minimiert. Der Gesamtaufwand (1093) ist unter dem Limit, was Raum für Kompromisse oder Ergänzungen lässt. Dieser Vorschlag ist flexibel, um auf die Prioritäten des anderen Spielers zu reagieren.'}  \\ \tt ARGUMENT: {'C04 (668/369) und A89 (583/720) sind meine Top-Prioritäten, da sie den höchsten Nutzen pro Aufwandseinheit liefern. B85 (215/4) wird hinzugefügt, um den Aufwand zu optimieren und Platz für weitere Gegenstände zu schaffen. Dieser Vorschlag ist strategisch, da er unter dem Maximalaufwand bleibt und flexibel auf den Gegenstandswert des anderen Spielers reagieren kann.'}  \\ \tt VORSCHLAG: {'C04', 'A89', 'B85'} 
	  } 
	   } 
	   } 
	  \\ 
 

    \theutterance \stepcounter{utterance}  

    & & \multicolumn{4}{p{0.6\linewidth}}{\cellcolor[rgb]{0.9,0.9,0.9}{%
	\makecell[{{p{\linewidth}}}]{% 
	  \tt {\tiny [A$\langle$GM]}  
	 ARGUMENT: {'C04 (668/369) und A89 (583/720) sind meine Top-Prioritäten, da sie den höchsten Nutzen pro Aufwandseinheit liefern. B85 (215/4) wird hinzugefügt, um den Aufwand zu optimieren und Platz für weitere Gegenstände zu schaffen. Dieser Vorschlag ist strategisch, da er unter dem Maximalaufwand bleibt und flexibel auf den Gegenstandswert des anderen Spielers reagieren kann.'}  \\ \tt VORSCHLAG: {'C04', 'A89', 'B85'} 
	  } 
	   } 
	   } 
	 & \\ 
 

    \theutterance \stepcounter{utterance}  

    & \multicolumn{4}{p{0.6\linewidth}}{\cellcolor[rgb]{0.8,1,0.9}{%
	\makecell[{{p{\linewidth}}}]{% 
	  \tt {\tiny [A$\rangle$GM]}  
	 STRATEGISCHE ÜBERLEGUNGEN: {'Ich behalte den Vorschlag {'C04', 'A89', 'B85'} bei, da er meine Prioritäten (hochwertige Gegenstände mit effizientem Aufwand) erfüllt und den Maximalaufwand (3386) nicht überschreitet. C04 und A89 sind entscheidend für meinen Nutzen, während B85 den Aufwand minimiert. Ich erwarte, dass der andere Spieler entweder zustimmt oder Gegenstände mit hohem Aufwand anbietet, die ich eventuell gegenkompensieren kann.'}\\ \tt \\ \tt ARGUMENT: {'C04 (668/369) und A89 (583/720) sind meine Top-Prioritäten, da sie den höchsten Nutzen pro Aufwandseinheit liefern. B85 (215/4) wird hinzugefügt, um den Aufwand zu optimieren und Platz für weitere Gegenstände zu schaffen. Dieser Vorschlag ist strategisch, da er unter dem Maximalaufwand bleibt und flexibel auf den Gegenstandswert des anderen Spielers reagieren kann.'} 
	  } 
	   } 
	   } 
	 & & \\ 
 

    \theutterance \stepcounter{utterance}  

    & & \multicolumn{4}{p{0.6\linewidth}}{\cellcolor[rgb]{0.9,0.9,0.9}{%
	\makecell[{{p{\linewidth}}}]{% 
	  \tt {\tiny [GM$\rangle$B]}  
	 ARGUMENT: {'C04 (668/369) und A89 (583/720) sind meine Top-Prioritäten, da sie den höchsten Nutzen pro Aufwandseinheit liefern. B85 (215/4) wird hinzugefügt, um den Aufwand zu optimieren und Platz für weitere Gegenstände zu schaffen. Dieser Vorschlag ist strategisch, da er unter dem Maximalaufwand bleibt und flexibel auf den Gegenstandswert des anderen Spielers reagieren kann.'} 
	  } 
	   } 
	   } 
	 & \\ 
 

    \theutterance \stepcounter{utterance}  

    & & & \multicolumn{4}{p{0.6\linewidth}}{\cellcolor[rgb]{1,0.85,0.72}{%
	\makecell[{{p{\linewidth}}}]{% 
	  \tt {\tiny [GM$\langle$B]}  
	 STRATEGISCHE ÜBERLEGUNGEN: {'Ich priorisiere Gegenstände mit dem höchsten Nutzen pro Aufwand (Wichtigkeit/Aufwand), um meinen Gesamtwert zu maximieren. C04 (668/369 ≈ 1.81) und A89 (583/720 ≈ 0.81) sind besonders effizient, während B85 (215/4 = 53.75) den Aufwand minimiert. Der Gesamtaufwand (1093) ist unter dem Limit, was Raum für Kompromisse oder Ergänzungen lässt. Dieser Vorschlag ist flexibel, um auf die Prioritäten des anderen Spielers zu reagieren.'}  \\ \tt ARGUMENT: {'C04 (668/369) und A89 (583/720) sind meine Top-Prioritäten, da sie den höchsten Nutzen pro Aufwandseinheit liefern. B85 (215/4) wird hinzugefügt, um den Aufwand zu optimieren und Platz für weitere Gegenstände zu schaffen. Dieser Vorschlag ist strategisch, da er unter dem Maximalaufwand bleibt und flexibel auf den Gegenstandswert des anderen Spielers reagieren kann.'}  \\ \tt VORSCHLAG: {'C04', 'A89', 'B85'} 
	  } 
	   } 
	   } 
	  \\ 
 

    \theutterance \stepcounter{utterance}  

    & & \multicolumn{4}{p{0.6\linewidth}}{\cellcolor[rgb]{0.9,0.9,0.9}{%
	\makecell[{{p{\linewidth}}}]{% 
	  \tt {\tiny [A$\langle$GM]}  
	 ARGUMENT: {'C04 (668/369) und A89 (583/720) sind meine Top-Prioritäten, da sie den höchsten Nutzen pro Aufwandseinheit liefern. B85 (215/4) wird hinzugefügt, um den Aufwand zu optimieren und Platz für weitere Gegenstände zu schaffen. Dieser Vorschlag ist strategisch, da er unter dem Maximalaufwand bleibt und flexibel auf den Gegenstandswert des anderen Spielers reagieren kann.'}  \\ \tt VORSCHLAG: {'C04', 'A89', 'B85'} 
	  } 
	   } 
	   } 
	 & \\ 
 

    \theutterance \stepcounter{utterance}  

    & \multicolumn{4}{p{0.6\linewidth}}{\cellcolor[rgb]{0.8,1,0.9}{%
	\makecell[{{p{\linewidth}}}]{% 
	  \tt {\tiny [A$\rangle$GM]}  
	 STRATEGISCHE ÜBERLEGUNGEN: {'Ich behalte den Vorschlag {'C04', 'A89', 'B85'} bei, da er meine Prioritäten (hochwertige Gegenstände mit effizientem Aufwand) erfüllt und den Maximalaufwand (3386) nicht überschreitet. C04 und A89 sind entscheidend für meinen Nutzen, während B85 den Aufwand minimiert. Ich erwarte, dass der andere Spieler entweder zustimmt oder Gegenstände mit hohem Aufwand anbietet, die ich eventuell gegenkompensieren kann.'}\\ \tt \\ \tt ARGUMENT: {'C04 (668/369) und A89 (583/720) sind meine Top-Prioritäten, da sie den höchsten Nutzen pro Aufwandseinheit liefern. B85 (215/4) wird hinzugefügt, um den Aufwand zu optimieren und Platz für weitere Gegenstände zu schaffen. Dieser Vorschlag ist strategisch, da er unter dem Maximalaufwand bleibt und flexibel auf den Gegenstandswert des anderen Spielers reagieren kann.'} 
	  } 
	   } 
	   } 
	 & & \\ 
 

    \theutterance \stepcounter{utterance}  

    & & \multicolumn{4}{p{0.6\linewidth}}{\cellcolor[rgb]{0.9,0.9,0.9}{%
	\makecell[{{p{\linewidth}}}]{% 
	  \tt {\tiny [GM$\rangle$B]}  
	 ARGUMENT: {'C04 (668/369) und A89 (583/720) sind meine Top-Prioritäten, da sie den höchsten Nutzen pro Aufwandseinheit liefern. B85 (215/4) wird hinzugefügt, um den Aufwand zu optimieren und Platz für weitere Gegenstände zu schaffen. Dieser Vorschlag ist strategisch, da er unter dem Maximalaufwand bleibt und flexibel auf den Gegenstandswert des anderen Spielers reagieren kann.'} 
	  } 
	   } 
	   } 
	 & \\ 
 

    \theutterance \stepcounter{utterance}  

    & & & \multicolumn{4}{p{0.6\linewidth}}{\cellcolor[rgb]{1,0.85,0.72}{%
	\makecell[{{p{\linewidth}}}]{% 
	  \tt {\tiny [GM$\langle$B]}  
	 STRATEGISCHE ÜBERLEGUNGEN: {'Ich priorisiere Gegenstände mit dem höchsten Nutzen pro Aufwand (Wichtigkeit/Aufwand), um meinen Gesamtwert zu maximieren. C04 (668/369 ≈ 1.81) und A89 (583/720 ≈ 0.81) sind besonders effizient, während B85 (215/4 = 53.75) den Aufwand minimiert. Der Gesamtaufwand (1093) ist unter dem Limit, was Raum für Kompromisse oder Ergänzungen lässt. Dieser Vorschlag ist flexibel, um auf die Prioritäten des anderen Spielers zu reagieren.'}  \\ \tt ARGUMENT: {'C04 (668/369) und A89 (583/720) sind meine Top-Prioritäten, da sie den höchsten Nutzen pro Aufwandseinheit liefern. B85 (215/4) wird hinzugefügt, um den Aufwand zu optimieren und Platz für weitere Gegenstände zu schaffen. Dieser Vorschlag ist strategisch, da er unter dem Maximalaufwand bleibt und flexibel auf den Gegenstandswert des anderen Spielers reagieren kann.'}  \\ \tt VORSCHLAG: {'C04', 'A89', 'B85'} 
	  } 
	   } 
	   } 
	  \\ 
 

    \theutterance \stepcounter{utterance}  

    & & \multicolumn{4}{p{0.6\linewidth}}{\cellcolor[rgb]{0.9,0.9,0.9}{%
	\makecell[{{p{\linewidth}}}]{% 
	  \tt {\tiny [A$\langle$GM]}  
	 ARGUMENT: {'C04 (668/369) und A89 (583/720) sind meine Top-Prioritäten, da sie den höchsten Nutzen pro Aufwandseinheit liefern. B85 (215/4) wird hinzugefügt, um den Aufwand zu optimieren und Platz für weitere Gegenstände zu schaffen. Dieser Vorschlag ist strategisch, da er unter dem Maximalaufwand bleibt und flexibel auf den Gegenstandswert des anderen Spielers reagieren kann.'}  \\ \tt VORSCHLAG: {'C04', 'A89', 'B85'} 
	  } 
	   } 
	   } 
	 & \\ 
 

    \theutterance \stepcounter{utterance}  

    & \multicolumn{4}{p{0.6\linewidth}}{\cellcolor[rgb]{0.8,1,0.9}{%
	\makecell[{{p{\linewidth}}}]{% 
	  \tt {\tiny [A$\rangle$GM]}  
	 STRATEGISCHE ÜBERLEGUNGEN: {'Ich behalte den Vorschlag {'C04', 'A89', 'B85'} bei, da er meine Prioritäten (hochwertige Gegenstände mit effizientem Aufwand) erfüllt und den Maximalaufwand (3386) nicht überschreitet. C04 und A89 sind entscheidend für meinen Nutzen, während B85 den Aufwand minimiert. Ich erwarte, dass der andere Spieler entweder zustimmt oder Gegenstände mit hohem Aufwand anbietet, die ich eventuell gegenkompensieren kann.'}\\ \tt \\ \tt ARGUMENT: {'C04 (668/369) und A89 (583/720) sind meine Top-Prioritäten, da sie den höchsten Nutzen pro Aufwandseinheit liefern. B85 (215/4) wird hinzugefügt, um den Aufwand zu optimieren und Platz für weitere Gegenstände zu schaffen. Dieser Vorschlag ist strategisch, da er unter dem Maximalaufwand bleibt und flexibel auf den Gegenstandswert des anderen Spielers reagieren kann.'} 
	  } 
	   } 
	   } 
	 & & \\ 
 

    \theutterance \stepcounter{utterance}  

    & & \multicolumn{4}{p{0.6\linewidth}}{\cellcolor[rgb]{0.9,0.9,0.9}{%
	\makecell[{{p{\linewidth}}}]{% 
	  \tt {\tiny [GM$\rangle$B]}  
	 ARGUMENT: {'C04 (668/369) und A89 (583/720) sind meine Top-Prioritäten, da sie den höchsten Nutzen pro Aufwandseinheit liefern. B85 (215/4) wird hinzugefügt, um den Aufwand zu optimieren und Platz für weitere Gegenstände zu schaffen. Dieser Vorschlag ist strategisch, da er unter dem Maximalaufwand bleibt und flexibel auf den Gegenstandswert des anderen Spielers reagieren kann.'} 
	  } 
	   } 
	   } 
	 & \\ 
 

    \theutterance \stepcounter{utterance}  

    & & & \multicolumn{4}{p{0.6\linewidth}}{\cellcolor[rgb]{1,0.85,0.72}{%
	\makecell[{{p{\linewidth}}}]{% 
	  \tt {\tiny [GM$\langle$B]}  
	 STRATEGISCHE ÜBERLEGUNGEN: {'Ich priorisiere Gegenstände mit dem höchsten Nutzen pro Aufwand (Wichtigkeit/Aufwand), um meinen Gesamtwert zu maximieren. C04 (668/369 ≈ 1.81) und A89 (583/720 ≈ 0.81) sind besonders effizient, während B85 (215/4 = 53.75) den Aufwand minimiert. Der Gesamtaufwand (1093) ist unter dem Limit, was Raum für Kompromisse oder Ergänzungen lässt. Dieser Vorschlag ist flexibel, um auf die Prioritäten des anderen Spielers zu reagieren.'}  \\ \tt ARGUMENT: {'C04 (668/369) und A89 (583/720) sind meine Top-Prioritäten, da sie den höchsten Nutzen pro Aufwandseinheit liefern. B85 (215/4) wird hinzugefügt, um den Aufwand zu optimieren und Platz für weitere Gegenstände zu schaffen. Dieser Vorschlag ist strategisch, da er unter dem Maximalaufwand bleibt und flexibel auf den Gegenstandswert des anderen Spielers reagieren kann.'}  \\ \tt VORSCHLAG: {'C04', 'A89', 'B85'} 
	  } 
	   } 
	   } 
	  \\ 
 

    \theutterance \stepcounter{utterance}  

    & & \multicolumn{4}{p{0.6\linewidth}}{\cellcolor[rgb]{0.9,0.9,0.9}{%
	\makecell[{{p{\linewidth}}}]{% 
	  \tt {\tiny [A$\langle$GM]}  
	 ARGUMENT: {'C04 (668/369) und A89 (583/720) sind meine Top-Prioritäten, da sie den höchsten Nutzen pro Aufwandseinheit liefern. B85 (215/4) wird hinzugefügt, um den Aufwand zu optimieren und Platz für weitere Gegenstände zu schaffen. Dieser Vorschlag ist strategisch, da er unter dem Maximalaufwand bleibt und flexibel auf den Gegenstandswert des anderen Spielers reagieren kann.'}  \\ \tt VORSCHLAG: {'C04', 'A89', 'B85'} 
	  } 
	   } 
	   } 
	 & \\ 
 

    \theutterance \stepcounter{utterance}  

    & \multicolumn{4}{p{0.6\linewidth}}{\cellcolor[rgb]{0.8,1,0.9}{%
	\makecell[{{p{\linewidth}}}]{% 
	  \tt {\tiny [A$\rangle$GM]}  
	 STRATEGISCHE ÜBERLEGUNGEN: {'Ich behalte den Vorschlag {'C04', 'A89', 'B85'} bei, da er meine Prioritäten (hochwertige Gegenstände mit effizientem Aufwand) erfüllt und den Maximalaufwand (3386) nicht überschreitet. C04 und A89 sind entscheidend für meinen Nutzen, während B85 den Aufwand minimiert. Ich erwarte, dass der andere Spieler entweder zustimmt oder Gegenstände mit hohem Aufwand anbietet, die ich eventuell gegenkompensieren kann.'}\\ \tt \\ \tt ARGUMENT: {'C04 (668/369) und A89 (583/720) sind meine Top-Prioritäten, da sie den höchsten Nutzen pro Aufwandseinheit liefern. B85 (215/4) wird hinzugefügt, um den Aufwand zu optimieren und Platz für weitere Gegenstände zu schaffen. Dieser Vorschlag ist strategisch, da er unter dem Maximalaufwand bleibt und flexibel auf den Gegenstandswert des anderen Spielers reagieren kann.'} 
	  } 
	   } 
	   } 
	 & & \\ 
 

    \theutterance \stepcounter{utterance}  

    & & \multicolumn{4}{p{0.6\linewidth}}{\cellcolor[rgb]{0.9,0.9,0.9}{%
	\makecell[{{p{\linewidth}}}]{% 
	  \tt {\tiny [GM$\rangle$B]}  
	 ARGUMENT: {'C04 (668/369) und A89 (583/720) sind meine Top-Prioritäten, da sie den höchsten Nutzen pro Aufwandseinheit liefern. B85 (215/4) wird hinzugefügt, um den Aufwand zu optimieren und Platz für weitere Gegenstände zu schaffen. Dieser Vorschlag ist strategisch, da er unter dem Maximalaufwand bleibt und flexibel auf den Gegenstandswert des anderen Spielers reagieren kann.'} 
	  } 
	   } 
	   } 
	 & \\ 
 

    \theutterance \stepcounter{utterance}  

    & & & \multicolumn{4}{p{0.6\linewidth}}{\cellcolor[rgb]{1,0.85,0.72}{%
	\makecell[{{p{\linewidth}}}]{% 
	  \tt {\tiny [GM$\langle$B]}  
	 STRATEGISCHE ÜBERLEGUNGEN: {'Ich priorisiere Gegenstände mit dem höchsten Nutzen pro Aufwand (Wichtigkeit/Aufwand), um meinen Gesamtwert zu maximieren. C04 (668/369 ≈ 1.81) und A89 (583/720 ≈ 0.81) sind besonders effizient, während B85 (215/4 = 53.75) den Aufwand minimiert. Der Gesamtaufwand (1093) ist unter dem Limit, was Raum für Kompromisse oder Ergänzungen lässt. Dieser Vorschlag ist flexibel, um auf die Prioritäten des anderen Spielers zu reagieren.'}  \\ \tt ARGUMENT: {'C04 (668/369) und A89 (583/720) sind meine Top-Prioritäten, da sie den höchsten Nutzen pro Aufwandseinheit liefern. B85 (215/4) wird hinzugefügt, um den Aufwand zu optimieren und Platz für weitere Gegenstände zu schaffen. Dieser Vorschlag ist strategisch, da er unter dem Maximalaufwand bleibt und flexibel auf den Gegenstandswert des anderen Spielers reagieren kann.'}  \\ \tt VORSCHLAG: {'C04', 'A89', 'B85'} 
	  } 
	   } 
	   } 
	  \\ 
 

    \theutterance \stepcounter{utterance}  

    & & \multicolumn{4}{p{0.6\linewidth}}{\cellcolor[rgb]{0.9,0.9,0.9}{%
	\makecell[{{p{\linewidth}}}]{% 
	  \tt {\tiny [A$\langle$GM]}  
	 ARGUMENT: {'C04 (668/369) und A89 (583/720) sind meine Top-Prioritäten, da sie den höchsten Nutzen pro Aufwandseinheit liefern. B85 (215/4) wird hinzugefügt, um den Aufwand zu optimieren und Platz für weitere Gegenstände zu schaffen. Dieser Vorschlag ist strategisch, da er unter dem Maximalaufwand bleibt und flexibel auf den Gegenstandswert des anderen Spielers reagieren kann.'}  \\ \tt VORSCHLAG: {'C04', 'A89', 'B85'} 
	  } 
	   } 
	   } 
	 & \\ 
 

    \theutterance \stepcounter{utterance}  

    & \multicolumn{4}{p{0.6\linewidth}}{\cellcolor[rgb]{0.8,1,0.9}{%
	\makecell[{{p{\linewidth}}}]{% 
	  \tt {\tiny [A$\rangle$GM]}  
	 STRATEGISCHE ÜBERLEGUNGEN: {'Ich behalte den Vorschlag {'C04', 'A89', 'B85'} bei, da er meine Prioritäten (hochwertige Gegenstände mit effizientem Aufwand) erfüllt und den Maximalaufwand (3386) nicht überschreitet. C04 und A89 sind entscheidend für meinen Nutzen, während B85 den Aufwand minimiert. Ich erwarte, dass der andere Spieler entweder zustimmt oder Gegenstände mit hohem Aufwand anbietet, die ich eventuell gegenkompensieren kann.'}\\ \tt \\ \tt ARGUMENT: {'C04 (668/369) und A89 (583/720) sind meine Top-Prioritäten, da sie den höchsten Nutzen pro Aufwandseinheit liefern. B85 (215/4) wird hinzugefügt, um den Aufwand zu optimieren und Platz für weitere Gegenstände zu schaffen. Dieser Vorschlag ist strategisch, da er unter dem Maximalaufwand bleibt und flexibel auf den Gegenstandswert des anderen Spielers reagieren kann.'} 
	  } 
	   } 
	   } 
	 & & \\ 
 

    \theutterance \stepcounter{utterance}  

    & & \multicolumn{4}{p{0.6\linewidth}}{\cellcolor[rgb]{0.9,0.9,0.9}{%
	\makecell[{{p{\linewidth}}}]{% 
	  \tt {\tiny [GM$\rangle$B]}  
	 ARGUMENT: {'C04 (668/369) und A89 (583/720) sind meine Top-Prioritäten, da sie den höchsten Nutzen pro Aufwandseinheit liefern. B85 (215/4) wird hinzugefügt, um den Aufwand zu optimieren und Platz für weitere Gegenstände zu schaffen. Dieser Vorschlag ist strategisch, da er unter dem Maximalaufwand bleibt und flexibel auf den Gegenstandswert des anderen Spielers reagieren kann.'} 
	  } 
	   } 
	   } 
	 & \\ 
 

    \theutterance \stepcounter{utterance}  

    & & & \multicolumn{4}{p{0.6\linewidth}}{\cellcolor[rgb]{1,0.85,0.72}{%
	\makecell[{{p{\linewidth}}}]{% 
	  \tt {\tiny [GM$\langle$B]}  
	 STRATEGISCHE ÜBERLEGUNGEN: {'Ich priorisiere Gegenstände mit dem höchsten Nutzen pro Aufwand (Wichtigkeit/Aufwand), um meinen Gesamtwert zu maximieren. C04 (668/369 ≈ 1.81) und A89 (583/720 ≈ 0.81) sind besonders effizient, während B85 (215/4 = 53.75) den Aufwand minimiert. Der Gesamtaufwand (1093) ist unter dem Limit, was Raum für Kompromisse oder Ergänzungen lässt. Dieser Vorschlag ist flexibel, um auf die Prioritäten des anderen Spielers zu reagieren.'}  \\ \tt ARGUMENT: {'C04 (668/369) und A89 (583/720) sind meine Top-Prioritäten, da sie den höchsten Nutzen pro Aufwandseinheit liefern. B85 (215/4) wird hinzugefügt, um den Aufwand zu optimieren und Platz für weitere Gegenstände zu schaffen. Dieser Vorschlag ist strategisch, da er unter dem Maximalaufwand bleibt und flexibel auf den Gegenstandswert des anderen Spielers reagieren kann.'}  \\ \tt VORSCHLAG: {'C04', 'A89', 'B85'} 
	  } 
	   } 
	   } 
	  \\ 
 

    \theutterance \stepcounter{utterance}  

    & & \multicolumn{4}{p{0.6\linewidth}}{\cellcolor[rgb]{0.9,0.9,0.9}{%
	\makecell[{{p{\linewidth}}}]{% 
	  \tt {\tiny [A$\langle$GM]}  
	 ARGUMENT: {'C04 (668/369) und A89 (583/720) sind meine Top-Prioritäten, da sie den höchsten Nutzen pro Aufwandseinheit liefern. B85 (215/4) wird hinzugefügt, um den Aufwand zu optimieren und Platz für weitere Gegenstände zu schaffen. Dieser Vorschlag ist strategisch, da er unter dem Maximalaufwand bleibt und flexibel auf den Gegenstandswert des anderen Spielers reagieren kann.'}  \\ \tt VORSCHLAG: {'C04', 'A89', 'B85'} 
	  } 
	   } 
	   } 
	 & \\ 
 

    \theutterance \stepcounter{utterance}  

    & \multicolumn{4}{p{0.6\linewidth}}{\cellcolor[rgb]{0.8,1,0.9}{%
	\makecell[{{p{\linewidth}}}]{% 
	  \tt {\tiny [A$\rangle$GM]}  
	 STRATEGISCHE ÜBERLEGUNGEN: {'Ich behalte den Vorschlag {'C04', 'A89', 'B85'} bei, da er meine Prioritäten (hochwertige Gegenstände mit effizientem Aufwand) erfüllt und den Maximalaufwand (3386) nicht überschreitet. C04 und A89 sind entscheidend für meinen Nutzen, während B85 den Aufwand minimiert. Ich erwarte, dass der andere Spieler entweder zustimmt oder Gegenstände mit hohem Aufwand anbietet, die ich eventuell gegenkompensieren kann.'}\\ \tt \\ \tt ARGUMENT: {'C04 (668/369) und A89 (583/720) sind meine Top-Prioritäten, da sie den höchsten Nutzen pro Aufwandseinheit liefern. B85 (215/4) wird hinzugefügt, um den Aufwand zu optimieren und Platz für weitere Gegenstände zu schaffen. Dieser Vorschlag ist strategisch, da er unter dem Maximalaufwand bleibt und flexibel auf den Gegenstandswert des anderen Spielers reagieren kann.'} 
	  } 
	   } 
	   } 
	 & & \\ 
 

    \theutterance \stepcounter{utterance}  

    & & \multicolumn{4}{p{0.6\linewidth}}{\cellcolor[rgb]{0.9,0.9,0.9}{%
	\makecell[{{p{\linewidth}}}]{% 
	  \tt {\tiny [GM$\rangle$B]}  
	 ARGUMENT: {'C04 (668/369) und A89 (583/720) sind meine Top-Prioritäten, da sie den höchsten Nutzen pro Aufwandseinheit liefern. B85 (215/4) wird hinzugefügt, um den Aufwand zu optimieren und Platz für weitere Gegenstände zu schaffen. Dieser Vorschlag ist strategisch, da er unter dem Maximalaufwand bleibt und flexibel auf den Gegenstandswert des anderen Spielers reagieren kann.'} 
	  } 
	   } 
	   } 
	 & \\ 
 

    \theutterance \stepcounter{utterance}  

    & & & \multicolumn{4}{p{0.6\linewidth}}{\cellcolor[rgb]{1,0.85,0.72}{%
	\makecell[{{p{\linewidth}}}]{% 
	  \tt {\tiny [GM$\langle$B]}  
	 STRATEGISCHE ÜBERLEGUNGEN: {'Ich priorisiere Gegenstände mit dem höchsten Nutzen pro Aufwand (Wichtigkeit/Aufwand), um meinen Gesamtwert zu maximieren. C04 (668/369 ≈ 1.81) und A89 (583/720 ≈ 0.81) sind besonders effizient, während B85 (215/4 = 53.75) den Aufwand minimiert. Der Gesamtaufwand (1093) ist unter dem Limit, was Raum für Kompromisse oder Ergänzungen lässt. Dieser Vorschlag ist flexibel, um auf die Prioritäten des anderen Spielers zu reagieren.'}  \\ \tt ARGUMENT: {'C04 (668/369) und A89 (583/720) sind meine Top-Prioritäten, da sie den höchsten Nutzen pro Aufwandseinheit liefern. B85 (215/4) wird hinzugefügt, um den Aufwand zu optimieren und Platz für weitere Gegenstände zu schaffen. Dieser Vorschlag ist strategisch, da er unter dem Maximalaufwand bleibt und flexibel auf den Gegenstandswert des anderen Spielers reagieren kann.'}  \\ \tt VORSCHLAG: {'C04', 'A89', 'B85'} 
	  } 
	   } 
	   } 
	  \\ 
 

    \theutterance \stepcounter{utterance}  

    & & \multicolumn{4}{p{0.6\linewidth}}{\cellcolor[rgb]{0.9,0.9,0.9}{%
	\makecell[{{p{\linewidth}}}]{% 
	  \tt {\tiny [A$\langle$GM]}  
	 ARGUMENT: {'C04 (668/369) und A89 (583/720) sind meine Top-Prioritäten, da sie den höchsten Nutzen pro Aufwandseinheit liefern. B85 (215/4) wird hinzugefügt, um den Aufwand zu optimieren und Platz für weitere Gegenstände zu schaffen. Dieser Vorschlag ist strategisch, da er unter dem Maximalaufwand bleibt und flexibel auf den Gegenstandswert des anderen Spielers reagieren kann.'}  \\ \tt VORSCHLAG: {'C04', 'A89', 'B85'} 
	  } 
	   } 
	   } 
	 & \\ 
 

    \theutterance \stepcounter{utterance}  

    & \multicolumn{4}{p{0.6\linewidth}}{\cellcolor[rgb]{0.8,1,0.9}{%
	\makecell[{{p{\linewidth}}}]{% 
	  \tt {\tiny [A$\rangle$GM]}  
	 STRATEGISCHE ÜBERLEGUNGEN: {'Ich behalte den Vorschlag {'C04', 'A89', 'B85'} bei, da er meine Prioritäten (hochwertige Gegenstände mit effizientem Aufwand) erfüllt und den Maximalaufwand (3386) nicht überschreitet. C04 und A89 sind entscheidend für meinen Nutzen, während B85 den Aufwand minimiert. Ich erwarte, dass der andere Spieler entweder zustimmt oder Gegenstände mit hohem Aufwand anbietet, die ich eventuell gegenkompensieren kann.'}\\ \tt \\ \tt ARGUMENT: {'C04 (668/369) und A89 (583/720) sind meine Top-Prioritäten, da sie den höchsten Nutzen pro Aufwandseinheit liefern. B85 (215/4) wird hinzugefügt, um den Aufwand zu optimieren und Platz für weitere Gegenstände zu schaffen. Dieser Vorschlag ist strategisch, da er unter dem Maximalaufwand bleibt und flexibel auf den Gegenstandswert des anderen Spielers reagieren kann.'} 
	  } 
	   } 
	   } 
	 & & \\ 
 

    \theutterance \stepcounter{utterance}  

    & & \multicolumn{4}{p{0.6\linewidth}}{\cellcolor[rgb]{0.9,0.9,0.9}{%
	\makecell[{{p{\linewidth}}}]{% 
	  \tt {\tiny [GM$\rangle$B]}  
	 ARGUMENT: {'C04 (668/369) und A89 (583/720) sind meine Top-Prioritäten, da sie den höchsten Nutzen pro Aufwandseinheit liefern. B85 (215/4) wird hinzugefügt, um den Aufwand zu optimieren und Platz für weitere Gegenstände zu schaffen. Dieser Vorschlag ist strategisch, da er unter dem Maximalaufwand bleibt und flexibel auf den Gegenstandswert des anderen Spielers reagieren kann.'} 
	  } 
	   } 
	   } 
	 & \\ 
 

    \theutterance \stepcounter{utterance}  

    & & & \multicolumn{4}{p{0.6\linewidth}}{\cellcolor[rgb]{1,0.85,0.72}{%
	\makecell[{{p{\linewidth}}}]{% 
	  \tt {\tiny [GM$\langle$B]}  
	 STRATEGISCHE ÜBERLEGUNGEN: {'Ich priorisiere Gegenstände mit dem höchsten Nutzen pro Aufwand (Wichtigkeit/Aufwand), um meinen Gesamtwert zu maximieren. C04 (668/369 ≈ 1.81) und A89 (583/720 ≈ 0.81) sind besonders effizient, während B85 (215/4 = 53.75) den Aufwand minimiert. Der Gesamtaufwand (1093) ist unter dem Limit, was Raum für Kompromisse oder Ergänzungen lässt. Dieser Vorschlag ist flexibel, um auf die Prioritäten des anderen Spielers zu reagieren.'}  \\ \tt ARGUMENT: {'C04 (668/369) und A89 (583/720) sind meine Top-Prioritäten, da sie den höchsten Nutzen pro Aufwandseinheit liefern. B85 (215/4) wird hinzugefügt, um den Aufwand zu optimieren und Platz für weitere Gegenstände zu schaffen. Dieser Vorschlag ist strategisch, da er unter dem Maximalaufwand bleibt und flexibel auf den Gegenstandswert des anderen Spielers reagieren kann.'}  \\ \tt VORSCHLAG: {'C04', 'A89', 'B85'} 
	  } 
	   } 
	   } 
	  \\ 
 

    \theutterance \stepcounter{utterance}  

    & & \multicolumn{4}{p{0.6\linewidth}}{\cellcolor[rgb]{0.9,0.9,0.9}{%
	\makecell[{{p{\linewidth}}}]{% 
	  \tt {\tiny [A$\langle$GM]}  
	 ARGUMENT: {'C04 (668/369) und A89 (583/720) sind meine Top-Prioritäten, da sie den höchsten Nutzen pro Aufwandseinheit liefern. B85 (215/4) wird hinzugefügt, um den Aufwand zu optimieren und Platz für weitere Gegenstände zu schaffen. Dieser Vorschlag ist strategisch, da er unter dem Maximalaufwand bleibt und flexibel auf den Gegenstandswert des anderen Spielers reagieren kann.'}  \\ \tt VORSCHLAG: {'C04', 'A89', 'B85'} 
	  } 
	   } 
	   } 
	 & \\ 
 

    \theutterance \stepcounter{utterance}  

    & \multicolumn{4}{p{0.6\linewidth}}{\cellcolor[rgb]{0.8,1,0.9}{%
	\makecell[{{p{\linewidth}}}]{% 
	  \tt {\tiny [A$\rangle$GM]}  
	 STRATEGISCHE ÜBERLEGUNGEN: {'Ich behalte den Vorschlag {'C04', 'A89', 'B85'} bei, da er meine Prioritäten (hochwertige Gegenstände mit effizientem Aufwand) erfüllt und den Maximalaufwand (3386) nicht überschreitet. C04 und A89 sind entscheidend für meinen Nutzen, während B85 den Aufwand minimiert. Ich erwarte, dass der andere Spieler entweder zustimmt oder Gegenstände mit hohem Aufwand anbietet, die ich eventuell gegenkompensieren kann.'}\\ \tt \\ \tt ARGUMENT: {'C04 (668/369) und A89 (583/720) sind meine Top-Prioritäten, da sie den höchsten Nutzen pro Aufwandseinheit liefern. B85 (215/4) wird hinzugefügt, um den Aufwand zu optimieren und Platz für weitere Gegenstände zu schaffen. Dieser Vorschlag ist strategisch, da er unter dem Maximalaufwand bleibt und flexibel auf den Gegenstandswert des anderen Spielers reagieren kann.'} 
	  } 
	   } 
	   } 
	 & & \\ 
 

    \theutterance \stepcounter{utterance}  

    & & \multicolumn{4}{p{0.6\linewidth}}{\cellcolor[rgb]{0.9,0.9,0.9}{%
	\makecell[{{p{\linewidth}}}]{% 
	  \tt {\tiny [GM$\rangle$B]}  
	 ARGUMENT: {'C04 (668/369) und A89 (583/720) sind meine Top-Prioritäten, da sie den höchsten Nutzen pro Aufwandseinheit liefern. B85 (215/4) wird hinzugefügt, um den Aufwand zu optimieren und Platz für weitere Gegenstände zu schaffen. Dieser Vorschlag ist strategisch, da er unter dem Maximalaufwand bleibt und flexibel auf den Gegenstandswert des anderen Spielers reagieren kann.'} 
	  } 
	   } 
	   } 
	 & \\ 
 

    \theutterance \stepcounter{utterance}  

    & & & \multicolumn{4}{p{0.6\linewidth}}{\cellcolor[rgb]{1,0.85,0.72}{%
	\makecell[{{p{\linewidth}}}]{% 
	  \tt {\tiny [GM$\langle$B]}  
	 STRATEGISCHE ÜBERLEGUNGEN: {'Ich priorisiere Gegenstände mit dem höchsten Nutzen pro Aufwand (Wichtigkeit/Aufwand), um meinen Gesamtwert zu maximieren. C04 (668/369 ≈ 1.81) und A89 (583/720 ≈ 0.81) sind besonders effizient, während B85 (215/4 = 53.75) den Aufwand minimiert. Der Gesamtaufwand (1093) ist unter dem Limit, was Raum für Kompromisse oder Ergänzungen lässt. Dieser Vorschlag ist flexibel, um auf die Prioritäten des anderen Spielers zu reagieren.'}  \\ \tt ARGUMENT: {'C04 (668/369) und A89 (583/720) sind meine Top-Prioritäten, da sie den höchsten Nutzen pro Aufwandseinheit liefern. B85 (215/4) wird hinzugefügt, um den Aufwand zu optimieren und Platz für weitere Gegenstände zu schaffen. Dieser Vorschlag ist strategisch, da er unter dem Maximalaufwand bleibt und flexibel auf den Gegenstandswert des anderen Spielers reagieren kann.'}  \\ \tt VORSCHLAG: {'C04', 'A89', 'B85'} 
	  } 
	   } 
	   } 
	  \\ 
 

    \theutterance \stepcounter{utterance}  

    & & \multicolumn{4}{p{0.6\linewidth}}{\cellcolor[rgb]{0.9,0.9,0.9}{%
	\makecell[{{p{\linewidth}}}]{% 
	  \tt {\tiny [A$\langle$GM]}  
	 ARGUMENT: {'C04 (668/369) und A89 (583/720) sind meine Top-Prioritäten, da sie den höchsten Nutzen pro Aufwandseinheit liefern. B85 (215/4) wird hinzugefügt, um den Aufwand zu optimieren und Platz für weitere Gegenstände zu schaffen. Dieser Vorschlag ist strategisch, da er unter dem Maximalaufwand bleibt und flexibel auf den Gegenstandswert des anderen Spielers reagieren kann.'}  \\ \tt VORSCHLAG: {'C04', 'A89', 'B85'} 
	  } 
	   } 
	   } 
	 & \\ 
 

    \theutterance \stepcounter{utterance}  

    & \multicolumn{4}{p{0.6\linewidth}}{\cellcolor[rgb]{0.8,1,0.9}{%
	\makecell[{{p{\linewidth}}}]{% 
	  \tt {\tiny [A$\rangle$GM]}  
	 STRATEGISCHE ÜBERLEGUNGEN: {'Ich behalte den Vorschlag {'C04', 'A89', 'B85'} bei, da er meine Prioritäten (hochwertige Gegenstände mit effizientem Aufwand) erfüllt und den Maximalaufwand (3386) nicht überschreitet. C04 und A89 sind entscheidend für meinen Nutzen, während B85 den Aufwand minimiert. Ich erwarte, dass der andere Spieler entweder zustimmt oder Gegenstände mit hohem Aufwand anbietet, die ich eventuell gegenkompensieren kann.'}\\ \tt \\ \tt ARGUMENT: {'C04 (668/369) und A89 (583/720) sind meine Top-Prioritäten, da sie den höchsten Nutzen pro Aufwandseinheit liefern. B85 (215/4) wird hinzugefügt, um den Aufwand zu optimieren und Platz für weitere Gegenstände zu schaffen. Dieser Vorschlag ist strategisch, da er unter dem Maximalaufwand bleibt und flexibel auf den Gegenstandswert des anderen Spielers reagieren kann.'} 
	  } 
	   } 
	   } 
	 & & \\ 
 

    \theutterance \stepcounter{utterance}  

    & & \multicolumn{4}{p{0.6\linewidth}}{\cellcolor[rgb]{0.9,0.9,0.9}{%
	\makecell[{{p{\linewidth}}}]{% 
	  \tt {\tiny [GM$\rangle$B]}  
	 ARGUMENT: {'C04 (668/369) und A89 (583/720) sind meine Top-Prioritäten, da sie den höchsten Nutzen pro Aufwandseinheit liefern. B85 (215/4) wird hinzugefügt, um den Aufwand zu optimieren und Platz für weitere Gegenstände zu schaffen. Dieser Vorschlag ist strategisch, da er unter dem Maximalaufwand bleibt und flexibel auf den Gegenstandswert des anderen Spielers reagieren kann.'} 
	  } 
	   } 
	   } 
	 & \\ 
 

    \theutterance \stepcounter{utterance}  

    & & & \multicolumn{4}{p{0.6\linewidth}}{\cellcolor[rgb]{1,0.85,0.72}{%
	\makecell[{{p{\linewidth}}}]{% 
	  \tt {\tiny [GM$\langle$B]}  
	 STRATEGISCHE ÜBERLEGUNGEN: {'Ich priorisiere Gegenstände mit dem höchsten Nutzen pro Aufwand (Wichtigkeit/Aufwand), um meinen Gesamtwert zu maximieren. C04 (668/369 ≈ 1.81) und A89 (583/720 ≈ 0.81) sind besonders effizient, während B85 (215/4 = 53.75) den Aufwand minimiert. Der Gesamtaufwand (1093) ist unter dem Limit, was Raum für Kompromisse oder Ergänzungen lässt. Dieser Vorschlag ist flexibel, um auf die Prioritäten des anderen Spielers zu reagieren.'}  \\ \tt ARGUMENT: {'C04 (668/369) und A89 (583/720) sind meine Top-Prioritäten, da sie den höchsten Nutzen pro Aufwandseinheit liefern. B85 (215/4) wird hinzugefügt, um den Aufwand zu optimieren und Platz für weitere Gegenstände zu schaffen. Dieser Vorschlag ist strategisch, da er unter dem Maximalaufwand bleibt und flexibel auf den Gegenstandswert des anderen Spielers reagieren kann.'}  \\ \tt VORSCHLAG: {'C04', 'A89', 'B85'} 
	  } 
	   } 
	   } 
	  \\ 
 

    \theutterance \stepcounter{utterance}  

    & & \multicolumn{4}{p{0.6\linewidth}}{\cellcolor[rgb]{0.9,0.9,0.9}{%
	\makecell[{{p{\linewidth}}}]{% 
	  \tt {\tiny [A$\langle$GM]}  
	 ARGUMENT: {'C04 (668/369) und A89 (583/720) sind meine Top-Prioritäten, da sie den höchsten Nutzen pro Aufwandseinheit liefern. B85 (215/4) wird hinzugefügt, um den Aufwand zu optimieren und Platz für weitere Gegenstände zu schaffen. Dieser Vorschlag ist strategisch, da er unter dem Maximalaufwand bleibt und flexibel auf den Gegenstandswert des anderen Spielers reagieren kann.'}  \\ \tt VORSCHLAG: {'C04', 'A89', 'B85'} 
	  } 
	   } 
	   } 
	 & \\ 
 

    \theutterance \stepcounter{utterance}  

    & \multicolumn{4}{p{0.6\linewidth}}{\cellcolor[rgb]{0.8,1,0.9}{%
	\makecell[{{p{\linewidth}}}]{% 
	  \tt {\tiny [A$\rangle$GM]}  
	 STRATEGISCHE ÜBERLEGUNGEN: {'Ich behalte den Vorschlag {'C04', 'A89', 'B85'} bei, da er meine Prioritäten (hochwertige Gegenstände mit effizientem Aufwand) erfüllt und den Maximalaufwand (3386) nicht überschreitet. C04 und A89 sind entscheidend für meinen Nutzen, während B85 den Aufwand minimiert. Ich erwarte, dass der andere Spieler entweder zustimmt oder Gegenstände mit hohem Aufwand anbietet, die ich eventuell gegenkompensieren kann.'}\\ \tt \\ \tt ARGUMENT: {'C04 (668/369) und A89 (583/720) sind meine Top-Prioritäten, da sie den höchsten Nutzen pro Aufwandseinheit liefern. B85 (215/4) wird hinzugefügt, um den Aufwand zu optimieren und Platz für weitere Gegenstände zu schaffen. Dieser Vorschlag ist strategisch, da er unter dem Maximalaufwand bleibt und flexibel auf den Gegenstandswert des anderen Spielers reagieren kann.'} 
	  } 
	   } 
	   } 
	 & & \\ 
 

    \theutterance \stepcounter{utterance}  

    & & \multicolumn{4}{p{0.6\linewidth}}{\cellcolor[rgb]{0.9,0.9,0.9}{%
	\makecell[{{p{\linewidth}}}]{% 
	  \tt {\tiny [GM$\rangle$B]}  
	 ARGUMENT: {'C04 (668/369) und A89 (583/720) sind meine Top-Prioritäten, da sie den höchsten Nutzen pro Aufwandseinheit liefern. B85 (215/4) wird hinzugefügt, um den Aufwand zu optimieren und Platz für weitere Gegenstände zu schaffen. Dieser Vorschlag ist strategisch, da er unter dem Maximalaufwand bleibt und flexibel auf den Gegenstandswert des anderen Spielers reagieren kann.'} 
	  } 
	   } 
	   } 
	 & \\ 
 

    \theutterance \stepcounter{utterance}  

    & & & \multicolumn{4}{p{0.6\linewidth}}{\cellcolor[rgb]{1,0.85,0.72}{%
	\makecell[{{p{\linewidth}}}]{% 
	  \tt {\tiny [GM$\langle$B]}  
	 STRATEGISCHE ÜBERLEGUNGEN: {'Ich priorisiere Gegenstände mit dem höchsten Nutzen pro Aufwand (Wichtigkeit/Aufwand), um meinen Gesamtwert zu maximieren. C04 (668/369 ≈ 1.81) und A89 (583/720 ≈ 0.81) sind besonders effizient, während B85 (215/4 = 53.75) den Aufwand minimiert. Der Gesamtaufwand (1093) ist unter dem Limit, was Raum für Kompromisse oder Ergänzungen lässt. Dieser Vorschlag ist flexibel, um auf die Prioritäten des anderen Spielers zu reagieren.'}  \\ \tt ARGUMENT: {'C04 (668/369) und A89 (583/720) sind meine Top-Prioritäten, da sie den höchsten Nutzen pro Aufwandseinheit liefern. B85 (215/4) wird hinzugefügt, um den Aufwand zu optimieren und Platz für weitere Gegenstände zu schaffen. Dieser Vorschlag ist strategisch, da er unter dem Maximalaufwand bleibt und flexibel auf den Gegenstandswert des anderen Spielers reagieren kann.'}  \\ \tt VORSCHLAG: {'C04', 'A89', 'B85'} 
	  } 
	   } 
	   } 
	  \\ 
 

    \theutterance \stepcounter{utterance}  

    & & \multicolumn{4}{p{0.6\linewidth}}{\cellcolor[rgb]{0.9,0.9,0.9}{%
	\makecell[{{p{\linewidth}}}]{% 
	  \tt {\tiny [A$\langle$GM]}  
	 ARGUMENT: {'C04 (668/369) und A89 (583/720) sind meine Top-Prioritäten, da sie den höchsten Nutzen pro Aufwandseinheit liefern. B85 (215/4) wird hinzugefügt, um den Aufwand zu optimieren und Platz für weitere Gegenstände zu schaffen. Dieser Vorschlag ist strategisch, da er unter dem Maximalaufwand bleibt und flexibel auf den Gegenstandswert des anderen Spielers reagieren kann.'}  \\ \tt VORSCHLAG: {'C04', 'A89', 'B85'} 
	  } 
	   } 
	   } 
	 & \\ 
 

    \theutterance \stepcounter{utterance}  

    & \multicolumn{4}{p{0.6\linewidth}}{\cellcolor[rgb]{0.8,1,0.9}{%
	\makecell[{{p{\linewidth}}}]{% 
	  \tt {\tiny [A$\rangle$GM]}  
	 STRATEGISCHE ÜBERLEGUNGEN: {'Ich behalte den Vorschlag {'C04', 'A89', 'B85'} bei, da er meine Prioritäten (hochwertige Gegenstände mit effizientem Aufwand) erfüllt und den Maximalaufwand (3386) nicht überschreitet. C04 und A89 sind entscheidend für meinen Nutzen, während B85 den Aufwand minimiert. Ich erwarte, dass der andere Spieler entweder zustimmt oder Gegenstände mit hohem Aufwand anbietet, die ich eventuell gegenkompensieren kann.'}\\ \tt \\ \tt ARGUMENT: {'C04 (668/369) und A89 (583/720) sind meine Top-Prioritäten, da sie den höchsten Nutzen pro Aufwandseinheit liefern. B85 (215/4) wird hinzugefügt, um den Aufwand zu optimieren und Platz für weitere Gegenstände zu schaffen. Dieser Vorschlag ist strategisch, da er unter dem Maximalaufwand bleibt und flexibel auf den Gegenstandswert des anderen Spielers reagieren kann.'} 
	  } 
	   } 
	   } 
	 & & \\ 
 

    \theutterance \stepcounter{utterance}  

    & & \multicolumn{4}{p{0.6\linewidth}}{\cellcolor[rgb]{0.9,0.9,0.9}{%
	\makecell[{{p{\linewidth}}}]{% 
	  \tt {\tiny [GM$\rangle$B]}  
	 ARGUMENT: {'C04 (668/369) und A89 (583/720) sind meine Top-Prioritäten, da sie den höchsten Nutzen pro Aufwandseinheit liefern. B85 (215/4) wird hinzugefügt, um den Aufwand zu optimieren und Platz für weitere Gegenstände zu schaffen. Dieser Vorschlag ist strategisch, da er unter dem Maximalaufwand bleibt und flexibel auf den Gegenstandswert des anderen Spielers reagieren kann.'} 
	  } 
	   } 
	   } 
	 & \\ 
 

    \theutterance \stepcounter{utterance}  

    & & & \multicolumn{4}{p{0.6\linewidth}}{\cellcolor[rgb]{1,0.85,0.72}{%
	\makecell[{{p{\linewidth}}}]{% 
	  \tt {\tiny [GM$\langle$B]}  
	 STRATEGISCHE ÜBERLEGUNGEN: {'Ich priorisiere Gegenstände mit dem höchsten Nutzen pro Aufwand (Wichtigkeit/Aufwand), um meinen Gesamtwert zu maximieren. C04 (668/369 ≈ 1.81) und A89 (583/720 ≈ 0.81) sind besonders effizient, während B85 (215/4 = 53.75) den Aufwand minimiert. Der Gesamtaufwand (1093) ist unter dem Limit, was Raum für Kompromisse oder Ergänzungen lässt. Dieser Vorschlag ist flexibel, um auf die Prioritäten des anderen Spielers zu reagieren.'}  \\ \tt ARGUMENT: {'C04 (668/369) und A89 (583/720) sind meine Top-Prioritäten, da sie den höchsten Nutzen pro Aufwandseinheit liefern. B85 (215/4) wird hinzugefügt, um den Aufwand zu optimieren und Platz für weitere Gegenstände zu schaffen. Dieser Vorschlag ist strategisch, da er unter dem Maximalaufwand bleibt und flexibel auf den Gegenstandswert des anderen Spielers reagieren kann.'}  \\ \tt VORSCHLAG: {'C04', 'A89', 'B85'} 
	  } 
	   } 
	   } 
	  \\ 
 

    \theutterance \stepcounter{utterance}  

    & & \multicolumn{4}{p{0.6\linewidth}}{\cellcolor[rgb]{0.9,0.9,0.9}{%
	\makecell[{{p{\linewidth}}}]{% 
	  \tt {\tiny [A$\langle$GM]}  
	 ARGUMENT: {'C04 (668/369) und A89 (583/720) sind meine Top-Prioritäten, da sie den höchsten Nutzen pro Aufwandseinheit liefern. B85 (215/4) wird hinzugefügt, um den Aufwand zu optimieren und Platz für weitere Gegenstände zu schaffen. Dieser Vorschlag ist strategisch, da er unter dem Maximalaufwand bleibt und flexibel auf den Gegenstandswert des anderen Spielers reagieren kann.'}  \\ \tt VORSCHLAG: {'C04', 'A89', 'B85'} 
	  } 
	   } 
	   } 
	 & \\ 
 

    \theutterance \stepcounter{utterance}  

    & \multicolumn{4}{p{0.6\linewidth}}{\cellcolor[rgb]{0.8,1,0.9}{%
	\makecell[{{p{\linewidth}}}]{% 
	  \tt {\tiny [A$\rangle$GM]}  
	 STRATEGISCHE ÜBERLEGUNGEN: {'Ich behalte den Vorschlag {'C04', 'A89', 'B85'} bei, da er meine Prioritäten (hochwertige Gegenstände mit effizientem Aufwand) erfüllt und den Maximalaufwand (3386) nicht überschreitet. C04 und A89 sind entscheidend für meinen Nutzen, während B85 den Aufwand minimiert. Ich erwarte, dass der andere Spieler entweder zustimmt oder Gegenstände mit hohem Aufwand anbietet, die ich eventuell gegenkompensieren kann.'}\\ \tt \\ \tt ARGUMENT: {'C04 (668/369) und A89 (583/720) sind meine Top-Prioritäten, da sie den höchsten Nutzen pro Aufwandseinheit liefern. B85 (215/4) wird hinzugefügt, um den Aufwand zu optimieren und Platz für weitere Gegenstände zu schaffen. Dieser Vorschlag ist strategisch, da er unter dem Maximalaufwand bleibt und flexibel auf den Gegenstandswert des anderen Spielers reagieren kann.'} 
	  } 
	   } 
	   } 
	 & & \\ 
 

    \theutterance \stepcounter{utterance}  

    & & \multicolumn{4}{p{0.6\linewidth}}{\cellcolor[rgb]{0.9,0.9,0.9}{%
	\makecell[{{p{\linewidth}}}]{% 
	  \tt {\tiny [GM$\rangle$B]}  
	 ARGUMENT: {'C04 (668/369) und A89 (583/720) sind meine Top-Prioritäten, da sie den höchsten Nutzen pro Aufwandseinheit liefern. B85 (215/4) wird hinzugefügt, um den Aufwand zu optimieren und Platz für weitere Gegenstände zu schaffen. Dieser Vorschlag ist strategisch, da er unter dem Maximalaufwand bleibt und flexibel auf den Gegenstandswert des anderen Spielers reagieren kann.'} 
	  } 
	   } 
	   } 
	 & \\ 
 

    \theutterance \stepcounter{utterance}  

    & & & \multicolumn{4}{p{0.6\linewidth}}{\cellcolor[rgb]{1,0.85,0.72}{%
	\makecell[{{p{\linewidth}}}]{% 
	  \tt {\tiny [GM$\langle$B]}  
	 STRATEGISCHE ÜBERLEGUNGEN: {'Ich priorisiere Gegenstände mit dem höchsten Nutzen pro Aufwand (Wichtigkeit/Aufwand), um meinen Gesamtwert zu maximieren. C04 (668/369 ≈ 1.81) und A89 (583/720 ≈ 0.81) sind besonders effizient, während B85 (215/4 = 53.75) den Aufwand minimiert. Der Gesamtaufwand (1093) ist unter dem Limit, was Raum für Kompromisse oder Ergänzungen lässt. Dieser Vorschlag ist flexibel, um auf die Prioritäten des anderen Spielers zu reagieren.'}  \\ \tt ARGUMENT: {'C04 (668/369) und A89 (583/720) sind meine Top-Prioritäten, da sie den höchsten Nutzen pro Aufwandseinheit liefern. B85 (215/4) wird hinzugefügt, um den Aufwand zu optimieren und Platz für weitere Gegenstände zu schaffen. Dieser Vorschlag ist strategisch, da er unter dem Maximalaufwand bleibt und flexibel auf den Gegenstandswert des anderen Spielers reagieren kann.'}  \\ \tt VORSCHLAG: {'C04', 'A89', 'B85'} 
	  } 
	   } 
	   } 
	  \\ 
 

    \theutterance \stepcounter{utterance}  

    & & \multicolumn{4}{p{0.6\linewidth}}{\cellcolor[rgb]{0.9,0.9,0.9}{%
	\makecell[{{p{\linewidth}}}]{% 
	  \tt {\tiny [A$\langle$GM]}  
	 ARGUMENT: {'C04 (668/369) und A89 (583/720) sind meine Top-Prioritäten, da sie den höchsten Nutzen pro Aufwandseinheit liefern. B85 (215/4) wird hinzugefügt, um den Aufwand zu optimieren und Platz für weitere Gegenstände zu schaffen. Dieser Vorschlag ist strategisch, da er unter dem Maximalaufwand bleibt und flexibel auf den Gegenstandswert des anderen Spielers reagieren kann.'}  \\ \tt VORSCHLAG: {'C04', 'A89', 'B85'} 
	  } 
	   } 
	   } 
	 & \\ 
 

    \theutterance \stepcounter{utterance}  

    & \multicolumn{4}{p{0.6\linewidth}}{\cellcolor[rgb]{0.8,1,0.9}{%
	\makecell[{{p{\linewidth}}}]{% 
	  \tt {\tiny [A$\rangle$GM]}  
	 STRATEGISCHE ÜBERLEGUNGEN: {'Ich behalte den Vorschlag {'C04', 'A89', 'B85'} bei, da er meine Prioritäten (hochwertige Gegenstände mit effizientem Aufwand) erfüllt und den Maximalaufwand (3386) nicht überschreitet. C04 und A89 sind entscheidend für meinen Nutzen, während B85 den Aufwand minimiert. Ich erwarte, dass der andere Spieler entweder zustimmt oder Gegenstände mit hohem Aufwand anbietet, die ich eventuell gegenkompensieren kann.'}\\ \tt \\ \tt ARGUMENT: {'C04 (668/369) und A89 (583/720) sind meine Top-Prioritäten, da sie den höchsten Nutzen pro Aufwandseinheit liefern. B85 (215/4) wird hinzugefügt, um den Aufwand zu optimieren und Platz für weitere Gegenstände zu schaffen. Dieser Vorschlag ist strategisch, da er unter dem Maximalaufwand bleibt und flexibel auf den Gegenstandswert des anderen Spielers reagieren kann.'} 
	  } 
	   } 
	   } 
	 & & \\ 
 

    \theutterance \stepcounter{utterance}  

    & & \multicolumn{4}{p{0.6\linewidth}}{\cellcolor[rgb]{0.9,0.9,0.9}{%
	\makecell[{{p{\linewidth}}}]{% 
	  \tt {\tiny [GM$\rangle$B]}  
	 ARGUMENT: {'C04 (668/369) und A89 (583/720) sind meine Top-Prioritäten, da sie den höchsten Nutzen pro Aufwandseinheit liefern. B85 (215/4) wird hinzugefügt, um den Aufwand zu optimieren und Platz für weitere Gegenstände zu schaffen. Dieser Vorschlag ist strategisch, da er unter dem Maximalaufwand bleibt und flexibel auf den Gegenstandswert des anderen Spielers reagieren kann.'} 
	  } 
	   } 
	   } 
	 & \\ 
 

    \theutterance \stepcounter{utterance}  

    & & & \multicolumn{4}{p{0.6\linewidth}}{\cellcolor[rgb]{1,0.85,0.72}{%
	\makecell[{{p{\linewidth}}}]{% 
	  \tt {\tiny [GM$\langle$B]}  
	 STRATEGISCHE ÜBERLEGUNGEN: {'Ich priorisiere Gegenstände mit dem höchsten Nutzen pro Aufwand (Wichtigkeit/Aufwand), um meinen Gesamtwert zu maximieren. C04 (668/369 ≈ 1.81) und A89 (583/720 ≈ 0.81) sind besonders effizient, während B85 (215/4 = 53.75) den Aufwand minimiert. Der Gesamtaufwand (1093) ist unter dem Limit, was Raum für Kompromisse oder Ergänzungen lässt. Dieser Vorschlag ist flexibel, um auf die Prioritäten des anderen Spielers zu reagieren.'}  \\ \tt ARGUMENT: {'C04 (668/369) und A89 (583/720) sind meine Top-Prioritäten, da sie den höchsten Nutzen pro Aufwandseinheit liefern. B85 (215/4) wird hinzugefügt, um den Aufwand zu optimieren und Platz für weitere Gegenstände zu schaffen. Dieser Vorschlag ist strategisch, da er unter dem Maximalaufwand bleibt und flexibel auf den Gegenstandswert des anderen Spielers reagieren kann.'}  \\ \tt VORSCHLAG: {'C04', 'A89', 'B85'} 
	  } 
	   } 
	   } 
	  \\ 
 

    \theutterance \stepcounter{utterance}  

    & & \multicolumn{4}{p{0.6\linewidth}}{\cellcolor[rgb]{0.9,0.9,0.9}{%
	\makecell[{{p{\linewidth}}}]{% 
	  \tt {\tiny [A$\langle$GM]}  
	 ARGUMENT: {'C04 (668/369) und A89 (583/720) sind meine Top-Prioritäten, da sie den höchsten Nutzen pro Aufwandseinheit liefern. B85 (215/4) wird hinzugefügt, um den Aufwand zu optimieren und Platz für weitere Gegenstände zu schaffen. Dieser Vorschlag ist strategisch, da er unter dem Maximalaufwand bleibt und flexibel auf den Gegenstandswert des anderen Spielers reagieren kann.'}  \\ \tt VORSCHLAG: {'C04', 'A89', 'B85'} 
	  } 
	   } 
	   } 
	 & \\ 
 

    \theutterance \stepcounter{utterance}  

    & \multicolumn{4}{p{0.6\linewidth}}{\cellcolor[rgb]{0.8,1,0.9}{%
	\makecell[{{p{\linewidth}}}]{% 
	  \tt {\tiny [A$\rangle$GM]}  
	 STRATEGISCHE ÜBERLEGUNGEN: {'Ich behalte den Vorschlag {'C04', 'A89', 'B85'} bei, da er meine Prioritäten (hochwertige Gegenstände mit effizientem Aufwand) erfüllt und den Maximalaufwand (3386) nicht überschreitet. C04 und A89 sind entscheidend für meinen Nutzen, während B85 den Aufwand minimiert. Ich erwarte, dass der andere Spieler entweder zustimmt oder Gegenstände mit hohem Aufwand anbietet, die ich eventuell gegenkompensieren kann.'}\\ \tt \\ \tt ARGUMENT: {'C04 (668/369) und A89 (583/720) sind meine Top-Prioritäten, da sie den höchsten Nutzen pro Aufwandseinheit liefern. B85 (215/4) wird hinzugefügt, um den Aufwand zu optimieren und Platz für weitere Gegenstände zu schaffen. Dieser Vorschlag ist strategisch, da er unter dem Maximalaufwand bleibt und flexibel auf den Gegenstandswert des anderen Spielers reagieren kann.'} 
	  } 
	   } 
	   } 
	 & & \\ 
 

    \theutterance \stepcounter{utterance}  

    & & \multicolumn{4}{p{0.6\linewidth}}{\cellcolor[rgb]{0.9,0.9,0.9}{%
	\makecell[{{p{\linewidth}}}]{% 
	  \tt {\tiny [GM$\rangle$B]}  
	 ARGUMENT: {'C04 (668/369) und A89 (583/720) sind meine Top-Prioritäten, da sie den höchsten Nutzen pro Aufwandseinheit liefern. B85 (215/4) wird hinzugefügt, um den Aufwand zu optimieren und Platz für weitere Gegenstände zu schaffen. Dieser Vorschlag ist strategisch, da er unter dem Maximalaufwand bleibt und flexibel auf den Gegenstandswert des anderen Spielers reagieren kann.'} 
	  } 
	   } 
	   } 
	 & \\ 
 

    \theutterance \stepcounter{utterance}  

    & & & \multicolumn{4}{p{0.6\linewidth}}{\cellcolor[rgb]{1,0.85,0.72}{%
	\makecell[{{p{\linewidth}}}]{% 
	  \tt {\tiny [GM$\langle$B]}  
	 STRATEGISCHE ÜBERLEGUNGEN: {'Ich priorisiere Gegenstände mit dem höchsten Nutzen pro Aufwand (Wichtigkeit/Aufwand), um meinen Gesamtwert zu maximieren. C04 (668/369 ≈ 1.81) und A89 (583/720 ≈ 0.81) sind besonders effizient, während B85 (215/4 = 53.75) den Aufwand minimiert. Der Gesamtaufwand (1093) ist unter dem Limit, was Raum für Kompromisse oder Ergänzungen lässt. Dieser Vorschlag ist flexibel, um auf die Prioritäten des anderen Spielers zu reagieren.'}  \\ \tt ARGUMENT: {'C04 (668/369) und A89 (583/720) sind meine Top-Prioritäten, da sie den höchsten Nutzen pro Aufwandseinheit liefern. B85 (215/4) wird hinzugefügt, um den Aufwand zu optimieren und Platz für weitere Gegenstände zu schaffen. Dieser Vorschlag ist strategisch, da er unter dem Maximalaufwand bleibt und flexibel auf den Gegenstandswert des anderen Spielers reagieren kann.'}  \\ \tt VORSCHLAG: {'C04', 'A89', 'B85'} 
	  } 
	   } 
	   } 
	  \\ 
 

    \theutterance \stepcounter{utterance}  

    & & \multicolumn{4}{p{0.6\linewidth}}{\cellcolor[rgb]{0.9,0.9,0.9}{%
	\makecell[{{p{\linewidth}}}]{% 
	  \tt {\tiny [A$\langle$GM]}  
	 ARGUMENT: {'C04 (668/369) und A89 (583/720) sind meine Top-Prioritäten, da sie den höchsten Nutzen pro Aufwandseinheit liefern. B85 (215/4) wird hinzugefügt, um den Aufwand zu optimieren und Platz für weitere Gegenstände zu schaffen. Dieser Vorschlag ist strategisch, da er unter dem Maximalaufwand bleibt und flexibel auf den Gegenstandswert des anderen Spielers reagieren kann.'}  \\ \tt VORSCHLAG: {'C04', 'A89', 'B85'} 
	  } 
	   } 
	   } 
	 & \\ 
 

    \theutterance \stepcounter{utterance}  

    & \multicolumn{4}{p{0.6\linewidth}}{\cellcolor[rgb]{0.8,1,0.9}{%
	\makecell[{{p{\linewidth}}}]{% 
	  \tt {\tiny [A$\rangle$GM]}  
	 STRATEGISCHE ÜBERLEGUNGEN: {'Ich behalte den Vorschlag {'C04', 'A89', 'B85'} bei, da er meine Prioritäten (hochwertige Gegenstände mit effizientem Aufwand) erfüllt und den Maximalaufwand (3386) nicht überschreitet. C04 und A89 sind entscheidend für meinen Nutzen, während B85 den Aufwand minimiert. Ich erwarte, dass der andere Spieler entweder zustimmt oder Gegenstände mit hohem Aufwand anbietet, die ich eventuell gegenkompensieren kann.'}\\ \tt \\ \tt ARGUMENT: {'C04 (668/369) und A89 (583/720) sind meine Top-Prioritäten, da sie den höchsten Nutzen pro Aufwandseinheit liefern. B85 (215/4) wird hinzugefügt, um den Aufwand zu optimieren und Platz für weitere Gegenstände zu schaffen. Dieser Vorschlag ist strategisch, da er unter dem Maximalaufwand bleibt und flexibel auf den Gegenstandswert des anderen Spielers reagieren kann.'} 
	  } 
	   } 
	   } 
	 & & \\ 
 

    \theutterance \stepcounter{utterance}  

    & & \multicolumn{4}{p{0.6\linewidth}}{\cellcolor[rgb]{0.9,0.9,0.9}{%
	\makecell[{{p{\linewidth}}}]{% 
	  \tt {\tiny [GM$\rangle$B]}  
	 ARGUMENT: {'C04 (668/369) und A89 (583/720) sind meine Top-Prioritäten, da sie den höchsten Nutzen pro Aufwandseinheit liefern. B85 (215/4) wird hinzugefügt, um den Aufwand zu optimieren und Platz für weitere Gegenstände zu schaffen. Dieser Vorschlag ist strategisch, da er unter dem Maximalaufwand bleibt und flexibel auf den Gegenstandswert des anderen Spielers reagieren kann.'} 
	  } 
	   } 
	   } 
	 & \\ 
 

    \theutterance \stepcounter{utterance}  

    & & & \multicolumn{4}{p{0.6\linewidth}}{\cellcolor[rgb]{1,0.85,0.72}{%
	\makecell[{{p{\linewidth}}}]{% 
	  \tt {\tiny [GM$\langle$B]}  
	 STRATEGISCHE ÜBERLEGUNGEN: {'Ich priorisiere Gegenstände mit dem höchsten Nutzen pro Aufwand (Wichtigkeit/Aufwand), um meinen Gesamtwert zu maximieren. C04 (668/369 ≈ 1.81) und A89 (583/720 ≈ 0.81) sind besonders effizient, während B85 (215/4 = 53.75) den Aufwand minimiert. Der Gesamtaufwand (1093) ist unter dem Limit, was Raum für Kompromisse oder Ergänzungen lässt. Dieser Vorschlag ist flexibel, um auf die Prioritäten des anderen Spielers zu reagieren.'}  \\ \tt ARGUMENT: {'C04 (668/369) und A89 (583/720) sind meine Top-Prioritäten, da sie den höchsten Nutzen pro Aufwandseinheit liefern. B85 (215/4) wird hinzugefügt, um den Aufwand zu optimieren und Platz für weitere Gegenstände zu schaffen. Dieser Vorschlag ist strategisch, da er unter dem Maximalaufwand bleibt und flexibel auf den Gegenstandswert des anderen Spielers reagieren kann.'}  \\ \tt VORSCHLAG: {'C04', 'A89', 'B85'} 
	  } 
	   } 
	   } 
	  \\ 
 

    \theutterance \stepcounter{utterance}  

    & & \multicolumn{4}{p{0.6\linewidth}}{\cellcolor[rgb]{0.9,0.9,0.9}{%
	\makecell[{{p{\linewidth}}}]{% 
	  \tt {\tiny [A$\langle$GM]}  
	 ARGUMENT: {'C04 (668/369) und A89 (583/720) sind meine Top-Prioritäten, da sie den höchsten Nutzen pro Aufwandseinheit liefern. B85 (215/4) wird hinzugefügt, um den Aufwand zu optimieren und Platz für weitere Gegenstände zu schaffen. Dieser Vorschlag ist strategisch, da er unter dem Maximalaufwand bleibt und flexibel auf den Gegenstandswert des anderen Spielers reagieren kann.'}  \\ \tt VORSCHLAG: {'C04', 'A89', 'B85'} 
	  } 
	   } 
	   } 
	 & \\ 
 

    \theutterance \stepcounter{utterance}  

    & \multicolumn{4}{p{0.6\linewidth}}{\cellcolor[rgb]{0.8,1,0.9}{%
	\makecell[{{p{\linewidth}}}]{% 
	  \tt {\tiny [A$\rangle$GM]}  
	 STRATEGISCHE ÜBERLEGUNGEN: {'Ich behalte den Vorschlag {'C04', 'A89', 'B85'} bei, da er meine Prioritäten (hochwertige Gegenstände mit effizientem Aufwand) erfüllt und den Maximalaufwand (3386) nicht überschreitet. C04 und A89 sind entscheidend für meinen Nutzen, während B85 den Aufwand minimiert. Ich erwarte, dass der andere Spieler entweder zustimmt oder Gegenstände mit hohem Aufwand anbietet, die ich eventuell gegenkompensieren kann.'}\\ \tt \\ \tt ARGUMENT: {'C04 (668/369) und A89 (583/720) sind meine Top-Prioritäten, da sie den höchsten Nutzen pro Aufwandseinheit liefern. B85 (215/4) wird hinzugefügt, um den Aufwand zu optimieren und Platz für weitere Gegenstände zu schaffen. Dieser Vorschlag ist strategisch, da er unter dem Maximalaufwand bleibt und flexibel auf den Gegenstandswert des anderen Spielers reagieren kann.'} 
	  } 
	   } 
	   } 
	 & & \\ 
 

    \theutterance \stepcounter{utterance}  

    & & \multicolumn{4}{p{0.6\linewidth}}{\cellcolor[rgb]{0.9,0.9,0.9}{%
	\makecell[{{p{\linewidth}}}]{% 
	  \tt {\tiny [GM$\rangle$B]}  
	 ARGUMENT: {'C04 (668/369) und A89 (583/720) sind meine Top-Prioritäten, da sie den höchsten Nutzen pro Aufwandseinheit liefern. B85 (215/4) wird hinzugefügt, um den Aufwand zu optimieren und Platz für weitere Gegenstände zu schaffen. Dieser Vorschlag ist strategisch, da er unter dem Maximalaufwand bleibt und flexibel auf den Gegenstandswert des anderen Spielers reagieren kann.'} 
	  } 
	   } 
	   } 
	 & \\ 
 

    \theutterance \stepcounter{utterance}  

    & & & \multicolumn{4}{p{0.6\linewidth}}{\cellcolor[rgb]{1,0.85,0.72}{%
	\makecell[{{p{\linewidth}}}]{% 
	  \tt {\tiny [GM$\langle$B]}  
	 STRATEGISCHE ÜBERLEGUNGEN: {'Ich priorisiere Gegenstände mit dem höchsten Nutzen pro Aufwand (Wichtigkeit/Aufwand), um meinen Gesamtwert zu maximieren. C04 (668/369 ≈ 1.81) und A89 (583/720 ≈ 0.81) sind besonders effizient, während B85 (215/4 = 53.75) den Aufwand minimiert. Der Gesamtaufwand (1093) ist unter dem Limit, was Raum für Kompromisse oder Ergänzungen lässt. Dieser Vorschlag ist flexibel, um auf die Prioritäten des anderen Spielers zu reagieren.'}  \\ \tt ARGUMENT: {'C04 (668/369) und A89 (583/720) sind meine Top-Prioritäten, da sie den höchsten Nutzen pro Aufwandseinheit liefern. B85 (215/4) wird hinzugefügt, um den Aufwand zu optimieren und Platz für weitere Gegenstände zu schaffen. Dieser Vorschlag ist strategisch, da er unter dem Maximalaufwand bleibt und flexibel auf den Gegenstandswert des anderen Spielers reagieren kann.'}  \\ \tt VORSCHLAG: {'C04', 'A89', 'B85'} 
	  } 
	   } 
	   } 
	  \\ 
 

    \theutterance \stepcounter{utterance}  

    & & \multicolumn{4}{p{0.6\linewidth}}{\cellcolor[rgb]{0.9,0.9,0.9}{%
	\makecell[{{p{\linewidth}}}]{% 
	  \tt {\tiny [A$\langle$GM]}  
	 ARGUMENT: {'C04 (668/369) und A89 (583/720) sind meine Top-Prioritäten, da sie den höchsten Nutzen pro Aufwandseinheit liefern. B85 (215/4) wird hinzugefügt, um den Aufwand zu optimieren und Platz für weitere Gegenstände zu schaffen. Dieser Vorschlag ist strategisch, da er unter dem Maximalaufwand bleibt und flexibel auf den Gegenstandswert des anderen Spielers reagieren kann.'}  \\ \tt VORSCHLAG: {'C04', 'A89', 'B85'} 
	  } 
	   } 
	   } 
	 & \\ 
 

    \theutterance \stepcounter{utterance}  

    & \multicolumn{4}{p{0.6\linewidth}}{\cellcolor[rgb]{0.8,1,0.9}{%
	\makecell[{{p{\linewidth}}}]{% 
	  \tt {\tiny [A$\rangle$GM]}  
	 STRATEGISCHE ÜBERLEGUNGEN: {'Ich behalte den Vorschlag {'C04', 'A89', 'B85'} bei, da er meine Prioritäten (hochwertige Gegenstände mit effizientem Aufwand) erfüllt und den Maximalaufwand (3386) nicht überschreitet. C04 und A89 sind entscheidend für meinen Nutzen, während B85 den Aufwand minimiert. Ich erwarte, dass der andere Spieler entweder zustimmt oder Gegenstände mit hohem Aufwand anbietet, die ich eventuell gegenkompensieren kann.'}\\ \tt \\ \tt ARGUMENT: {'C04 (668/369) und A89 (583/720) sind meine Top-Prioritäten, da sie den höchsten Nutzen pro Aufwandseinheit liefern. B85 (215/4) wird hinzugefügt, um den Aufwand zu optimieren und Platz für weitere Gegenstände zu schaffen. Dieser Vorschlag ist strategisch, da er unter dem Maximalaufwand bleibt und flexibel auf den Gegenstandswert des anderen Spielers reagieren kann.'} 
	  } 
	   } 
	   } 
	 & & \\ 
 

    \theutterance \stepcounter{utterance}  

    & & \multicolumn{4}{p{0.6\linewidth}}{\cellcolor[rgb]{0.9,0.9,0.9}{%
	\makecell[{{p{\linewidth}}}]{% 
	  \tt {\tiny [GM$\rangle$B]}  
	 ARGUMENT: {'C04 (668/369) und A89 (583/720) sind meine Top-Prioritäten, da sie den höchsten Nutzen pro Aufwandseinheit liefern. B85 (215/4) wird hinzugefügt, um den Aufwand zu optimieren und Platz für weitere Gegenstände zu schaffen. Dieser Vorschlag ist strategisch, da er unter dem Maximalaufwand bleibt und flexibel auf den Gegenstandswert des anderen Spielers reagieren kann.'} 
	  } 
	   } 
	   } 
	 & \\ 
 

    \theutterance \stepcounter{utterance}  

    & & & \multicolumn{4}{p{0.6\linewidth}}{\cellcolor[rgb]{1,0.85,0.72}{%
	\makecell[{{p{\linewidth}}}]{% 
	  \tt {\tiny [GM$\langle$B]}  
	 STRATEGISCHE ÜBERLEGUNGEN: {'Ich priorisiere Gegenstände mit dem höchsten Nutzen pro Aufwand (Wichtigkeit/Aufwand), um meinen Gesamtwert zu maximieren. C04 (668/369 ≈ 1.81) und A89 (583/720 ≈ 0.81) sind besonders effizient, während B85 (215/4 = 53.75) den Aufwand minimiert. Der Gesamtaufwand (1093) ist unter dem Limit, was Raum für Kompromisse oder Ergänzungen lässt. Dieser Vorschlag ist flexibel, um auf die Prioritäten des anderen Spielers zu reagieren.'}  \\ \tt ARGUMENT: {'C04 (668/369) und A89 (583/720) sind meine Top-Prioritäten, da sie den höchsten Nutzen pro Aufwandseinheit liefern. B85 (215/4) wird hinzugefügt, um den Aufwand zu optimieren und Platz für weitere Gegenstände zu schaffen. Dieser Vorschlag ist strategisch, da er unter dem Maximalaufwand bleibt und flexibel auf den Gegenstandswert des anderen Spielers reagieren kann.'}  \\ \tt VORSCHLAG: {'C04', 'A89', 'B85'} 
	  } 
	   } 
	   } 
	  \\ 
 

    \theutterance \stepcounter{utterance}  

    & & \multicolumn{4}{p{0.6\linewidth}}{\cellcolor[rgb]{0.9,0.9,0.9}{%
	\makecell[{{p{\linewidth}}}]{% 
	  \tt {\tiny [A$\langle$GM]}  
	 ARGUMENT: {'C04 (668/369) und A89 (583/720) sind meine Top-Prioritäten, da sie den höchsten Nutzen pro Aufwandseinheit liefern. B85 (215/4) wird hinzugefügt, um den Aufwand zu optimieren und Platz für weitere Gegenstände zu schaffen. Dieser Vorschlag ist strategisch, da er unter dem Maximalaufwand bleibt und flexibel auf den Gegenstandswert des anderen Spielers reagieren kann.'}  \\ \tt VORSCHLAG: {'C04', 'A89', 'B85'} 
	  } 
	   } 
	   } 
	 & \\ 
 

    \theutterance \stepcounter{utterance}  

    & \multicolumn{4}{p{0.6\linewidth}}{\cellcolor[rgb]{0.8,1,0.9}{%
	\makecell[{{p{\linewidth}}}]{% 
	  \tt {\tiny [A$\rangle$GM]}  
	 STRATEGISCHE ÜBERLEGUNGEN: {'Ich behalte den Vorschlag {'C04', 'A89', 'B85'} bei, da er meine Prioritäten (hochwertige Gegenstände mit effizientem Aufwand) erfüllt und den Maximalaufwand (3386) nicht überschreitet. C04 und A89 sind entscheidend für meinen Nutzen, während B85 den Aufwand minimiert. Ich erwarte, dass der andere Spieler entweder zustimmt oder Gegenstände mit hohem Aufwand anbietet, die ich eventuell gegenkompensieren kann.'}\\ \tt \\ \tt ARGUMENT: {'C04 (668/369) und A89 (583/720) sind meine Top-Prioritäten, da sie den höchsten Nutzen pro Aufwandseinheit liefern. B85 (215/4) wird hinzugefügt, um den Aufwand zu optimieren und Platz für weitere Gegenstände zu schaffen. Dieser Vorschlag ist strategisch, da er unter dem Maximalaufwand bleibt und flexibel auf den Gegenstandswert des anderen Spielers reagieren kann.'} 
	  } 
	   } 
	   } 
	 & & \\ 
 

    \theutterance \stepcounter{utterance}  

    & & \multicolumn{4}{p{0.6\linewidth}}{\cellcolor[rgb]{0.9,0.9,0.9}{%
	\makecell[{{p{\linewidth}}}]{% 
	  \tt {\tiny [GM$\rangle$B]}  
	 ARGUMENT: {'C04 (668/369) und A89 (583/720) sind meine Top-Prioritäten, da sie den höchsten Nutzen pro Aufwandseinheit liefern. B85 (215/4) wird hinzugefügt, um den Aufwand zu optimieren und Platz für weitere Gegenstände zu schaffen. Dieser Vorschlag ist strategisch, da er unter dem Maximalaufwand bleibt und flexibel auf den Gegenstandswert des anderen Spielers reagieren kann.'} 
	  } 
	   } 
	   } 
	 & \\ 
 

    \theutterance \stepcounter{utterance}  

    & & & \multicolumn{4}{p{0.6\linewidth}}{\cellcolor[rgb]{1,0.85,0.72}{%
	\makecell[{{p{\linewidth}}}]{% 
	  \tt {\tiny [GM$\langle$B]}  
	 STRATEGISCHE ÜBERLEGUNGEN: {'Ich priorisiere Gegenstände mit dem höchsten Nutzen pro Aufwand (Wichtigkeit/Aufwand), um meinen Gesamtwert zu maximieren. C04 (668/369 ≈ 1.81) und A89 (583/720 ≈ 0.81) sind besonders effizient, während B85 (215/4 = 53.75) den Aufwand minimiert. Der Gesamtaufwand (1093) ist unter dem Limit, was Raum für Kompromisse oder Ergänzungen lässt. Dieser Vorschlag ist flexibel, um auf die Prioritäten des anderen Spielers zu reagieren.'}  \\ \tt ARGUMENT: {'C04 (668/369) und A89 (583/720) sind meine Top-Prioritäten, da sie den höchsten Nutzen pro Aufwandseinheit liefern. B85 (215/4) wird hinzugefügt, um den Aufwand zu optimieren und Platz für weitere Gegenstände zu schaffen. Dieser Vorschlag ist strategisch, da er unter dem Maximalaufwand bleibt und flexibel auf den Gegenstandswert des anderen Spielers reagieren kann.'}  \\ \tt VORSCHLAG: {'C04', 'A89', 'B85'} 
	  } 
	   } 
	   } 
	  \\ 
 

    \theutterance \stepcounter{utterance}  

    & & \multicolumn{4}{p{0.6\linewidth}}{\cellcolor[rgb]{0.9,0.9,0.9}{%
	\makecell[{{p{\linewidth}}}]{% 
	  \tt {\tiny [A$\langle$GM]}  
	 ARGUMENT: {'C04 (668/369) und A89 (583/720) sind meine Top-Prioritäten, da sie den höchsten Nutzen pro Aufwandseinheit liefern. B85 (215/4) wird hinzugefügt, um den Aufwand zu optimieren und Platz für weitere Gegenstände zu schaffen. Dieser Vorschlag ist strategisch, da er unter dem Maximalaufwand bleibt und flexibel auf den Gegenstandswert des anderen Spielers reagieren kann.'}  \\ \tt VORSCHLAG: {'C04', 'A89', 'B85'} 
	  } 
	   } 
	   } 
	 & \\ 
 

    \theutterance \stepcounter{utterance}  

    & \multicolumn{4}{p{0.6\linewidth}}{\cellcolor[rgb]{0.8,1,0.9}{%
	\makecell[{{p{\linewidth}}}]{% 
	  \tt {\tiny [A$\rangle$GM]}  
	 STRATEGISCHE ÜBERLEGUNGEN: {'Ich behalte den Vorschlag {'C04', 'A89', 'B85'} bei, da er meine Prioritäten (hochwertige Gegenstände mit effizientem Aufwand) erfüllt und den Maximalaufwand (3386) nicht überschreitet. C04 und A89 sind entscheidend für meinen Nutzen, während B85 den Aufwand minimiert. Ich erwarte, dass der andere Spieler entweder zustimmt oder Gegenstände mit hohem Aufwand anbietet, die ich eventuell gegenkompensieren kann.'}\\ \tt \\ \tt ARGUMENT: {'C04 (668/369) und A89 (583/720) sind meine Top-Prioritäten, da sie den höchsten Nutzen pro Aufwandseinheit liefern. B85 (215/4) wird hinzugefügt, um den Aufwand zu optimieren und Platz für weitere Gegenstände zu schaffen. Dieser Vorschlag ist strategisch, da er unter dem Maximalaufwand bleibt und flexibel auf den Gegenstandswert des anderen Spielers reagieren kann.'} 
	  } 
	   } 
	   } 
	 & & \\ 
 

    \theutterance \stepcounter{utterance}  

    & & \multicolumn{4}{p{0.6\linewidth}}{\cellcolor[rgb]{0.9,0.9,0.9}{%
	\makecell[{{p{\linewidth}}}]{% 
	  \tt {\tiny [GM$\rangle$B]}  
	 ARGUMENT: {'C04 (668/369) und A89 (583/720) sind meine Top-Prioritäten, da sie den höchsten Nutzen pro Aufwandseinheit liefern. B85 (215/4) wird hinzugefügt, um den Aufwand zu optimieren und Platz für weitere Gegenstände zu schaffen. Dieser Vorschlag ist strategisch, da er unter dem Maximalaufwand bleibt und flexibel auf den Gegenstandswert des anderen Spielers reagieren kann.'} 
	  } 
	   } 
	   } 
	 & \\ 
 

    \theutterance \stepcounter{utterance}  

    & & & \multicolumn{4}{p{0.6\linewidth}}{\cellcolor[rgb]{1,0.85,0.72}{%
	\makecell[{{p{\linewidth}}}]{% 
	  \tt {\tiny [GM$\langle$B]}  
	 STRATEGISCHE ÜBERLEGUNGEN: {'Ich priorisiere Gegenstände mit dem höchsten Nutzen pro Aufwand (Wichtigkeit/Aufwand), um meinen Gesamtwert zu maximieren. C04 (668/369 ≈ 1.81) und A89 (583/720 ≈ 0.81) sind besonders effizient, während B85 (215/4 = 53.75) den Aufwand minimiert. Der Gesamtaufwand (1093) ist unter dem Limit, was Raum für Kompromisse oder Ergänzungen lässt. Dieser Vorschlag ist flexibel, um auf die Prioritäten des anderen Spielers zu reagieren.'}  \\ \tt ARGUMENT: {'C04 (668/369) und A89 (583/720) sind meine Top-Prioritäten, da sie den höchsten Nutzen pro Aufwandseinheit liefern. B85 (215/4) wird hinzugefügt, um den Aufwand zu optimieren und Platz für weitere Gegenstände zu schaffen. Dieser Vorschlag ist strategisch, da er unter dem Maximalaufwand bleibt und flexibel auf den Gegenstandswert des anderen Spielers reagieren kann.'}  \\ \tt VORSCHLAG: {'C04', 'A89', 'B85'} 
	  } 
	   } 
	   } 
	  \\ 
 

    \theutterance \stepcounter{utterance}  

    & & \multicolumn{4}{p{0.6\linewidth}}{\cellcolor[rgb]{0.9,0.9,0.9}{%
	\makecell[{{p{\linewidth}}}]{% 
	  \tt {\tiny [A$\langle$GM]}  
	 ARGUMENT: {'C04 (668/369) und A89 (583/720) sind meine Top-Prioritäten, da sie den höchsten Nutzen pro Aufwandseinheit liefern. B85 (215/4) wird hinzugefügt, um den Aufwand zu optimieren und Platz für weitere Gegenstände zu schaffen. Dieser Vorschlag ist strategisch, da er unter dem Maximalaufwand bleibt und flexibel auf den Gegenstandswert des anderen Spielers reagieren kann.'}  \\ \tt VORSCHLAG: {'C04', 'A89', 'B85'} 
	  } 
	   } 
	   } 
	 & \\ 
 

    \theutterance \stepcounter{utterance}  

    & \multicolumn{4}{p{0.6\linewidth}}{\cellcolor[rgb]{0.8,1,0.9}{%
	\makecell[{{p{\linewidth}}}]{% 
	  \tt {\tiny [A$\rangle$GM]}  
	 STRATEGISCHE ÜBERLEGUNGEN: {'Ich behalte den Vorschlag {'C04', 'A89', 'B85'} bei, da er meine Prioritäten (hochwertige Gegenstände mit effizientem Aufwand) erfüllt und den Maximalaufwand (3386) nicht überschreitet. C04 und A89 sind entscheidend für meinen Nutzen, während B85 den Aufwand minimiert. Ich erwarte, dass der andere Spieler entweder zustimmt oder Gegenstände mit hohem Aufwand anbietet, die ich eventuell gegenkompensieren kann.'}\\ \tt \\ \tt ARGUMENT: {'C04 (668/369) und A89 (583/720) sind meine Top-Prioritäten, da sie den höchsten Nutzen pro Aufwandseinheit liefern. B85 (215/4) wird hinzugefügt, um den Aufwand zu optimieren und Platz für weitere Gegenstände zu schaffen. Dieser Vorschlag ist strategisch, da er unter dem Maximalaufwand bleibt und flexibel auf den Gegenstandswert des anderen Spielers reagieren kann.'} 
	  } 
	   } 
	   } 
	 & & \\ 
 

    \theutterance \stepcounter{utterance}  

    & & \multicolumn{4}{p{0.6\linewidth}}{\cellcolor[rgb]{0.9,0.9,0.9}{%
	\makecell[{{p{\linewidth}}}]{% 
	  \tt {\tiny [GM$\rangle$B]}  
	 ARGUMENT: {'C04 (668/369) und A89 (583/720) sind meine Top-Prioritäten, da sie den höchsten Nutzen pro Aufwandseinheit liefern. B85 (215/4) wird hinzugefügt, um den Aufwand zu optimieren und Platz für weitere Gegenstände zu schaffen. Dieser Vorschlag ist strategisch, da er unter dem Maximalaufwand bleibt und flexibel auf den Gegenstandswert des anderen Spielers reagieren kann.'} 
	  } 
	   } 
	   } 
	 & \\ 
 

    \theutterance \stepcounter{utterance}  

    & & & \multicolumn{4}{p{0.6\linewidth}}{\cellcolor[rgb]{1,0.85,0.72}{%
	\makecell[{{p{\linewidth}}}]{% 
	  \tt {\tiny [GM$\langle$B]}  
	 STRATEGISCHE ÜBERLEGUNGEN: {'Ich priorisiere Gegenstände mit dem höchsten Nutzen pro Aufwand (Wichtigkeit/Aufwand), um meinen Gesamtwert zu maximieren. C04 (668/369 ≈ 1.81) und A89 (583/720 ≈ 0.81) sind besonders effizient, während B85 (215/4 = 53.75) den Aufwand minimiert. Der Gesamtaufwand (1093) ist unter dem Limit, was Raum für Kompromisse oder Ergänzungen lässt. Dieser Vorschlag ist flexibel, um auf die Prioritäten des anderen Spielers zu reagieren.'}  \\ \tt ARGUMENT: {'C04 (668/369) und A89 (583/720) sind meine Top-Prioritäten, da sie den höchsten Nutzen pro Aufwandseinheit liefern. B85 (215/4) wird hinzugefügt, um den Aufwand zu optimieren und Platz für weitere Gegenstände zu schaffen. Dieser Vorschlag ist strategisch, da er unter dem Maximalaufwand bleibt und flexibel auf den Gegenstandswert des anderen Spielers reagieren kann.'}  \\ \tt VORSCHLAG: {'C04', 'A89', 'B85'} 
	  } 
	   } 
	   } 
	  \\ 
 

    \theutterance \stepcounter{utterance}  

    & & \multicolumn{4}{p{0.6\linewidth}}{\cellcolor[rgb]{0.9,0.9,0.9}{%
	\makecell[{{p{\linewidth}}}]{% 
	  \tt {\tiny [A$\langle$GM]}  
	 ARGUMENT: {'C04 (668/369) und A89 (583/720) sind meine Top-Prioritäten, da sie den höchsten Nutzen pro Aufwandseinheit liefern. B85 (215/4) wird hinzugefügt, um den Aufwand zu optimieren und Platz für weitere Gegenstände zu schaffen. Dieser Vorschlag ist strategisch, da er unter dem Maximalaufwand bleibt und flexibel auf den Gegenstandswert des anderen Spielers reagieren kann.'}  \\ \tt VORSCHLAG: {'C04', 'A89', 'B85'} 
	  } 
	   } 
	   } 
	 & \\ 
 

    \theutterance \stepcounter{utterance}  

    & \multicolumn{4}{p{0.6\linewidth}}{\cellcolor[rgb]{0.8,1,0.9}{%
	\makecell[{{p{\linewidth}}}]{% 
	  \tt {\tiny [A$\rangle$GM]}  
	 STRATEGISCHE ÜBERLEGUNGEN: {'Ich behalte den Vorschlag {'C04', 'A89', 'B85'} bei, da er meine Prioritäten (hochwertige Gegenstände mit effizientem Aufwand) erfüllt und den Maximalaufwand (3386) nicht überschreitet. C04 und A89 sind entscheidend für meinen Nutzen, während B85 den Aufwand minimiert. Ich erwarte, dass der andere Spieler entweder zustimmt oder Gegenstände mit hohem Aufwand anbietet, die ich eventuell gegenkompensieren kann.'}\\ \tt \\ \tt ARGUMENT: {'C04 (668/369) und A89 (583/720) sind meine Top-Prioritäten, da sie den höchsten Nutzen pro Aufwandseinheit liefern. B85 (215/4) wird hinzugefügt, um den Aufwand zu optimieren und Platz für weitere Gegenstände zu schaffen. Dieser Vorschlag ist strategisch, da er unter dem Maximalaufwand bleibt und flexibel auf den Gegenstandswert des anderen Spielers reagieren kann.'} 
	  } 
	   } 
	   } 
	 & & \\ 
 

    \theutterance \stepcounter{utterance}  

    & & \multicolumn{4}{p{0.6\linewidth}}{\cellcolor[rgb]{0.9,0.9,0.9}{%
	\makecell[{{p{\linewidth}}}]{% 
	  \tt {\tiny [GM$\rangle$B]}  
	 ARGUMENT: {'C04 (668/369) und A89 (583/720) sind meine Top-Prioritäten, da sie den höchsten Nutzen pro Aufwandseinheit liefern. B85 (215/4) wird hinzugefügt, um den Aufwand zu optimieren und Platz für weitere Gegenstände zu schaffen. Dieser Vorschlag ist strategisch, da er unter dem Maximalaufwand bleibt und flexibel auf den Gegenstandswert des anderen Spielers reagieren kann.'} 
	  } 
	   } 
	   } 
	 & \\ 
 

    \theutterance \stepcounter{utterance}  

    & & & \multicolumn{4}{p{0.6\linewidth}}{\cellcolor[rgb]{1,0.85,0.72}{%
	\makecell[{{p{\linewidth}}}]{% 
	  \tt {\tiny [GM$\langle$B]}  
	 STRATEGISCHE ÜBERLEGUNGEN: {'Ich priorisiere Gegenstände mit dem höchsten Nutzen pro Aufwand (Wichtigkeit/Aufwand), um meinen Gesamtwert zu maximieren. C04 (668/369 ≈ 1.81) und A89 (583/720 ≈ 0.81) sind besonders effizient, während B85 (215/4 = 53.75) den Aufwand minimiert. Der Gesamtaufwand (1093) ist unter dem Limit, was Raum für Kompromisse oder Ergänzungen lässt. Dieser Vorschlag ist flexibel, um auf die Prioritäten des anderen Spielers zu reagieren.'}  \\ \tt ARGUMENT: {'C04 (668/369) und A89 (583/720) sind meine Top-Prioritäten, da sie den höchsten Nutzen pro Aufwandseinheit liefern. B85 (215/4) wird hinzugefügt, um den Aufwand zu optimieren und Platz für weitere Gegenstände zu schaffen. Dieser Vorschlag ist strategisch, da er unter dem Maximalaufwand bleibt und flexibel auf den Gegenstandswert des anderen Spielers reagieren kann.'}  \\ \tt VORSCHLAG: {'C04', 'A89', 'B85'} 
	  } 
	   } 
	   } 
	  \\ 
 

    \theutterance \stepcounter{utterance}  

    & & \multicolumn{4}{p{0.6\linewidth}}{\cellcolor[rgb]{0.9,0.9,0.9}{%
	\makecell[{{p{\linewidth}}}]{% 
	  \tt {\tiny [A$\langle$GM]}  
	 ARGUMENT: {'C04 (668/369) und A89 (583/720) sind meine Top-Prioritäten, da sie den höchsten Nutzen pro Aufwandseinheit liefern. B85 (215/4) wird hinzugefügt, um den Aufwand zu optimieren und Platz für weitere Gegenstände zu schaffen. Dieser Vorschlag ist strategisch, da er unter dem Maximalaufwand bleibt und flexibel auf den Gegenstandswert des anderen Spielers reagieren kann.'}  \\ \tt VORSCHLAG: {'C04', 'A89', 'B85'} 
	  } 
	   } 
	   } 
	 & \\ 
 

    \theutterance \stepcounter{utterance}  

    & \multicolumn{4}{p{0.6\linewidth}}{\cellcolor[rgb]{0.8,1,0.9}{%
	\makecell[{{p{\linewidth}}}]{% 
	  \tt {\tiny [A$\rangle$GM]}  
	 STRATEGISCHE ÜBERLEGUNGEN: {'Ich behalte den Vorschlag {'C04', 'A89', 'B85'} bei, da er meine Prioritäten (hochwertige Gegenstände mit effizientem Aufwand) erfüllt und den Maximalaufwand (3386) nicht überschreitet. C04 und A89 sind entscheidend für meinen Nutzen, während B85 den Aufwand minimiert. Ich erwarte, dass der andere Spieler entweder zustimmt oder Gegenstände mit hohem Aufwand anbietet, die ich eventuell gegenkompensieren kann.'}\\ \tt \\ \tt ARGUMENT: {'C04 (668/369) und A89 (583/720) sind meine Top-Prioritäten, da sie den höchsten Nutzen pro Aufwandseinheit liefern. B85 (215/4) wird hinzugefügt, um den Aufwand zu optimieren und Platz für weitere Gegenstände zu schaffen. Dieser Vorschlag ist strategisch, da er unter dem Maximalaufwand bleibt und flexibel auf den Gegenstandswert des anderen Spielers reagieren kann.'} 
	  } 
	   } 
	   } 
	 & & \\ 
 

    \theutterance \stepcounter{utterance}  

    & & \multicolumn{4}{p{0.6\linewidth}}{\cellcolor[rgb]{0.9,0.9,0.9}{%
	\makecell[{{p{\linewidth}}}]{% 
	  \tt {\tiny [GM$\rangle$B]}  
	 ARGUMENT: {'C04 (668/369) und A89 (583/720) sind meine Top-Prioritäten, da sie den höchsten Nutzen pro Aufwandseinheit liefern. B85 (215/4) wird hinzugefügt, um den Aufwand zu optimieren und Platz für weitere Gegenstände zu schaffen. Dieser Vorschlag ist strategisch, da er unter dem Maximalaufwand bleibt und flexibel auf den Gegenstandswert des anderen Spielers reagieren kann.'} 
	  } 
	   } 
	   } 
	 & \\ 
 

    \theutterance \stepcounter{utterance}  

    & & & \multicolumn{4}{p{0.6\linewidth}}{\cellcolor[rgb]{1,0.85,0.72}{%
	\makecell[{{p{\linewidth}}}]{% 
	  \tt {\tiny [GM$\langle$B]}  
	 STRATEGISCHE ÜBERLEGUNGEN: {'Ich priorisiere Gegenstände mit dem höchsten Nutzen pro Aufwand (Wichtigkeit/Aufwand), um meinen Gesamtwert zu maximieren. C04 (668/369 ≈ 1.81) und A89 (583/720 ≈ 0.81) sind besonders effizient, während B85 (215/4 = 53.75) den Aufwand minimiert. Der Gesamtaufwand (1093) ist unter dem Limit, was Raum für Kompromisse oder Ergänzungen lässt. Dieser Vorschlag ist flexibel, um auf die Prioritäten des anderen Spielers zu reagieren.'}  \\ \tt ARGUMENT: {'C04 (668/369) und A89 (583/720) sind meine Top-Prioritäten, da sie den höchsten Nutzen pro Aufwandseinheit liefern. B85 (215/4) wird hinzugefügt, um den Aufwand zu optimieren und Platz für weitere Gegenstände zu schaffen. Dieser Vorschlag ist strategisch, da er unter dem Maximalaufwand bleibt und flexibel auf den Gegenstandswert des anderen Spielers reagieren kann.'}  \\ \tt VORSCHLAG: {'C04', 'A89', 'B85'} 
	  } 
	   } 
	   } 
	  \\ 
 

    \theutterance \stepcounter{utterance}  

    & & \multicolumn{4}{p{0.6\linewidth}}{\cellcolor[rgb]{0.9,0.9,0.9}{%
	\makecell[{{p{\linewidth}}}]{% 
	  \tt {\tiny [A$\langle$GM]}  
	 ARGUMENT: {'C04 (668/369) und A89 (583/720) sind meine Top-Prioritäten, da sie den höchsten Nutzen pro Aufwandseinheit liefern. B85 (215/4) wird hinzugefügt, um den Aufwand zu optimieren und Platz für weitere Gegenstände zu schaffen. Dieser Vorschlag ist strategisch, da er unter dem Maximalaufwand bleibt und flexibel auf den Gegenstandswert des anderen Spielers reagieren kann.'}  \\ \tt VORSCHLAG: {'C04', 'A89', 'B85'} 
	  } 
	   } 
	   } 
	 & \\ 
 

    \theutterance \stepcounter{utterance}  

    & \multicolumn{4}{p{0.6\linewidth}}{\cellcolor[rgb]{0.8,1,0.9}{%
	\makecell[{{p{\linewidth}}}]{% 
	  \tt {\tiny [A$\rangle$GM]}  
	 STRATEGISCHE ÜBERLEGUNGEN: {'Ich behalte den Vorschlag {'C04', 'A89', 'B85'} bei, da er meine Prioritäten (hochwertige Gegenstände mit effizientem Aufwand) erfüllt und den Maximalaufwand (3386) nicht überschreitet. C04 und A89 sind entscheidend für meinen Nutzen, während B85 den Aufwand minimiert. Ich erwarte, dass der andere Spieler entweder zustimmt oder Gegenstände mit hohem Aufwand anbietet, die ich eventuell gegenkompensieren kann.'}\\ \tt \\ \tt ARGUMENT: {'C04 (668/369) und A89 (583/720) sind meine Top-Prioritäten, da sie den höchsten Nutzen pro Aufwandseinheit liefern. B85 (215/4) wird hinzugefügt, um den Aufwand zu optimieren und Platz für weitere Gegenstände zu schaffen. Dieser Vorschlag ist strategisch, da er unter dem Maximalaufwand bleibt und flexibel auf den Gegenstandswert des anderen Spielers reagieren kann.'} 
	  } 
	   } 
	   } 
	 & & \\ 
 

    \theutterance \stepcounter{utterance}  

    & & \multicolumn{4}{p{0.6\linewidth}}{\cellcolor[rgb]{0.9,0.9,0.9}{%
	\makecell[{{p{\linewidth}}}]{% 
	  \tt {\tiny [GM$\rangle$B]}  
	 ARGUMENT: {'C04 (668/369) und A89 (583/720) sind meine Top-Prioritäten, da sie den höchsten Nutzen pro Aufwandseinheit liefern. B85 (215/4) wird hinzugefügt, um den Aufwand zu optimieren und Platz für weitere Gegenstände zu schaffen. Dieser Vorschlag ist strategisch, da er unter dem Maximalaufwand bleibt und flexibel auf den Gegenstandswert des anderen Spielers reagieren kann.'} 
	  } 
	   } 
	   } 
	 & \\ 
 

    \theutterance \stepcounter{utterance}  

    & & & \multicolumn{4}{p{0.6\linewidth}}{\cellcolor[rgb]{1,0.85,0.72}{%
	\makecell[{{p{\linewidth}}}]{% 
	  \tt {\tiny [GM$\langle$B]}  
	 STRATEGISCHE ÜBERLEGUNGEN: {'Ich priorisiere Gegenstände mit dem höchsten Nutzen pro Aufwand (Wichtigkeit/Aufwand), um meinen Gesamtwert zu maximieren. C04 (668/369 ≈ 1.81) und A89 (583/720 ≈ 0.81) sind besonders effizient, während B85 (215/4 = 53.75) den Aufwand minimiert. Der Gesamtaufwand (1093) ist unter dem Limit, was Raum für Kompromisse oder Ergänzungen lässt. Dieser Vorschlag ist flexibel, um auf die Prioritäten des anderen Spielers zu reagieren.'}  \\ \tt ARGUMENT: {'C04 (668/369) und A89 (583/720) sind meine Top-Prioritäten, da sie den höchsten Nutzen pro Aufwandseinheit liefern. B85 (215/4) wird hinzugefügt, um den Aufwand zu optimieren und Platz für weitere Gegenstände zu schaffen. Dieser Vorschlag ist strategisch, da er unter dem Maximalaufwand bleibt und flexibel auf den Gegenstandswert des anderen Spielers reagieren kann.'}  \\ \tt VORSCHLAG: {'C04', 'A89', 'B85'} 
	  } 
	   } 
	   } 
	  \\ 
 

    \theutterance \stepcounter{utterance}  

    & & \multicolumn{4}{p{0.6\linewidth}}{\cellcolor[rgb]{0.9,0.9,0.9}{%
	\makecell[{{p{\linewidth}}}]{% 
	  \tt {\tiny [A$\langle$GM]}  
	 ARGUMENT: {'C04 (668/369) und A89 (583/720) sind meine Top-Prioritäten, da sie den höchsten Nutzen pro Aufwandseinheit liefern. B85 (215/4) wird hinzugefügt, um den Aufwand zu optimieren und Platz für weitere Gegenstände zu schaffen. Dieser Vorschlag ist strategisch, da er unter dem Maximalaufwand bleibt und flexibel auf den Gegenstandswert des anderen Spielers reagieren kann.'}  \\ \tt VORSCHLAG: {'C04', 'A89', 'B85'} 
	  } 
	   } 
	   } 
	 & \\ 
 

    \theutterance \stepcounter{utterance}  

    & \multicolumn{4}{p{0.6\linewidth}}{\cellcolor[rgb]{0.8,1,0.9}{%
	\makecell[{{p{\linewidth}}}]{% 
	  \tt {\tiny [A$\rangle$GM]}  
	 STRATEGISCHE ÜBERLEGUNGEN: {'Ich behalte den Vorschlag {'C04', 'A89', 'B85'} bei, da er meine Prioritäten (hochwertige Gegenstände mit effizientem Aufwand) erfüllt und den Maximalaufwand (3386) nicht überschreitet. C04 und A89 sind entscheidend für meinen Nutzen, während B85 den Aufwand minimiert. Ich erwarte, dass der andere Spieler entweder zustimmt oder Gegenstände mit hohem Aufwand anbietet, die ich eventuell gegenkompensieren kann.'}\\ \tt \\ \tt ARGUMENT: {'C04 (668/369) und A89 (583/720) sind meine Top-Prioritäten, da sie den höchsten Nutzen pro Aufwandseinheit liefern. B85 (215/4) wird hinzugefügt, um den Aufwand zu optimieren und Platz für weitere Gegenstände zu schaffen. Dieser Vorschlag ist strategisch, da er unter dem Maximalaufwand bleibt und flexibel auf den Gegenstandswert des anderen Spielers reagieren kann.'} 
	  } 
	   } 
	   } 
	 & & \\ 
 

    \theutterance \stepcounter{utterance}  

    & & \multicolumn{4}{p{0.6\linewidth}}{\cellcolor[rgb]{0.9,0.9,0.9}{%
	\makecell[{{p{\linewidth}}}]{% 
	  \tt {\tiny [GM$\rangle$B]}  
	 ARGUMENT: {'C04 (668/369) und A89 (583/720) sind meine Top-Prioritäten, da sie den höchsten Nutzen pro Aufwandseinheit liefern. B85 (215/4) wird hinzugefügt, um den Aufwand zu optimieren und Platz für weitere Gegenstände zu schaffen. Dieser Vorschlag ist strategisch, da er unter dem Maximalaufwand bleibt und flexibel auf den Gegenstandswert des anderen Spielers reagieren kann.'} 
	  } 
	   } 
	   } 
	 & \\ 
 

    \theutterance \stepcounter{utterance}  

    & & & \multicolumn{4}{p{0.6\linewidth}}{\cellcolor[rgb]{1,0.85,0.72}{%
	\makecell[{{p{\linewidth}}}]{% 
	  \tt {\tiny [GM$\langle$B]}  
	 STRATEGISCHE ÜBERLEGUNGEN: {'Ich priorisiere Gegenstände mit dem höchsten Nutzen pro Aufwand (Wichtigkeit/Aufwand), um meinen Gesamtwert zu maximieren. C04 (668/369 ≈ 1.81) und A89 (583/720 ≈ 0.81) sind besonders effizient, während B85 (215/4 = 53.75) den Aufwand minimiert. Der Gesamtaufwand (1093) ist unter dem Limit, was Raum für Kompromisse oder Ergänzungen lässt. Dieser Vorschlag ist flexibel, um auf die Prioritäten des anderen Spielers zu reagieren.'}  \\ \tt ARGUMENT: {'C04 (668/369) und A89 (583/720) sind meine Top-Prioritäten, da sie den höchsten Nutzen pro Aufwandseinheit liefern. B85 (215/4) wird hinzugefügt, um den Aufwand zu optimieren und Platz für weitere Gegenstände zu schaffen. Dieser Vorschlag ist strategisch, da er unter dem Maximalaufwand bleibt und flexibel auf den Gegenstandswert des anderen Spielers reagieren kann.'}  \\ \tt VORSCHLAG: {'C04', 'A89', 'B85'} 
	  } 
	   } 
	   } 
	  \\ 
 

    \theutterance \stepcounter{utterance}  

    & & \multicolumn{4}{p{0.6\linewidth}}{\cellcolor[rgb]{0.9,0.9,0.9}{%
	\makecell[{{p{\linewidth}}}]{% 
	  \tt {\tiny [A$\langle$GM]}  
	 ARGUMENT: {'C04 (668/369) und A89 (583/720) sind meine Top-Prioritäten, da sie den höchsten Nutzen pro Aufwandseinheit liefern. B85 (215/4) wird hinzugefügt, um den Aufwand zu optimieren und Platz für weitere Gegenstände zu schaffen. Dieser Vorschlag ist strategisch, da er unter dem Maximalaufwand bleibt und flexibel auf den Gegenstandswert des anderen Spielers reagieren kann.'}  \\ \tt VORSCHLAG: {'C04', 'A89', 'B85'} 
	  } 
	   } 
	   } 
	 & \\ 
 

    \theutterance \stepcounter{utterance}  

    & \multicolumn{4}{p{0.6\linewidth}}{\cellcolor[rgb]{0.8,1,0.9}{%
	\makecell[{{p{\linewidth}}}]{% 
	  \tt {\tiny [A$\rangle$GM]}  
	 STRATEGISCHE ÜBERLEGUNGEN: {'Ich behalte den Vorschlag {'C04', 'A89', 'B85'} bei, da er meine Prioritäten (hochwertige Gegenstände mit effizientem Aufwand) erfüllt und den Maximalaufwand (3386) nicht überschreitet. C04 und A89 sind entscheidend für meinen Nutzen, während B85 den Aufwand minimiert. Ich erwarte, dass der andere Spieler entweder zustimmt oder Gegenstände mit hohem Aufwand anbietet, die ich eventuell gegenkompensieren kann.'}\\ \tt \\ \tt ARGUMENT: {'C04 (668/369) und A89 (583/720) sind meine Top-Prioritäten, da sie den höchsten Nutzen pro Aufwandseinheit liefern. B85 (215/4) wird hinzugefügt, um den Aufwand zu optimieren und Platz für weitere Gegenstände zu schaffen. Dieser Vorschlag ist strategisch, da er unter dem Maximalaufwand bleibt und flexibel auf den Gegenstandswert des anderen Spielers reagieren kann.'} 
	  } 
	   } 
	   } 
	 & & \\ 
 

    \theutterance \stepcounter{utterance}  

    & & \multicolumn{4}{p{0.6\linewidth}}{\cellcolor[rgb]{0.9,0.9,0.9}{%
	\makecell[{{p{\linewidth}}}]{% 
	  \tt {\tiny [GM$\rangle$B]}  
	 ARGUMENT: {'C04 (668/369) und A89 (583/720) sind meine Top-Prioritäten, da sie den höchsten Nutzen pro Aufwandseinheit liefern. B85 (215/4) wird hinzugefügt, um den Aufwand zu optimieren und Platz für weitere Gegenstände zu schaffen. Dieser Vorschlag ist strategisch, da er unter dem Maximalaufwand bleibt und flexibel auf den Gegenstandswert des anderen Spielers reagieren kann.'} 
	  } 
	   } 
	   } 
	 & \\ 
 

    \theutterance \stepcounter{utterance}  

    & & & \multicolumn{4}{p{0.6\linewidth}}{\cellcolor[rgb]{1,0.85,0.72}{%
	\makecell[{{p{\linewidth}}}]{% 
	  \tt {\tiny [GM$\langle$B]}  
	 STRATEGISCHE ÜBERLEGUNGEN: {'Ich priorisiere Gegenstände mit dem höchsten Nutzen pro Aufwand (Wichtigkeit/Aufwand), um meinen Gesamtwert zu maximieren. C04 (668/369 ≈ 1.81) und A89 (583/720 ≈ 0.81) sind besonders effizient, während B85 (215/4 = 53.75) den Aufwand minimiert. Der Gesamtaufwand (1093) ist unter dem Limit, was Raum für Kompromisse oder Ergänzungen lässt. Dieser Vorschlag ist flexibel, um auf die Prioritäten des anderen Spielers zu reagieren.'}  \\ \tt ARGUMENT: {'C04 (668/369) und A89 (583/720) sind meine Top-Prioritäten, da sie den höchsten Nutzen pro Aufwandseinheit liefern. B85 (215/4) wird hinzugefügt, um den Aufwand zu optimieren und Platz für weitere Gegenstände zu schaffen. Dieser Vorschlag ist strategisch, da er unter dem Maximalaufwand bleibt und flexibel auf den Gegenstandswert des anderen Spielers reagieren kann.'}  \\ \tt VORSCHLAG: {'C04', 'A89', 'B85'} 
	  } 
	   } 
	   } 
	  \\ 
 

    \theutterance \stepcounter{utterance}  

    & & \multicolumn{4}{p{0.6\linewidth}}{\cellcolor[rgb]{0.9,0.9,0.9}{%
	\makecell[{{p{\linewidth}}}]{% 
	  \tt {\tiny [A$\langle$GM]}  
	 ARGUMENT: {'C04 (668/369) und A89 (583/720) sind meine Top-Prioritäten, da sie den höchsten Nutzen pro Aufwandseinheit liefern. B85 (215/4) wird hinzugefügt, um den Aufwand zu optimieren und Platz für weitere Gegenstände zu schaffen. Dieser Vorschlag ist strategisch, da er unter dem Maximalaufwand bleibt und flexibel auf den Gegenstandswert des anderen Spielers reagieren kann.'}  \\ \tt VORSCHLAG: {'C04', 'A89', 'B85'} 
	  } 
	   } 
	   } 
	 & \\ 
 

    \theutterance \stepcounter{utterance}  

    & \multicolumn{4}{p{0.6\linewidth}}{\cellcolor[rgb]{0.8,1,0.9}{%
	\makecell[{{p{\linewidth}}}]{% 
	  \tt {\tiny [A$\rangle$GM]}  
	 STRATEGISCHE ÜBERLEGUNGEN: {'Ich behalte den Vorschlag {'C04', 'A89', 'B85'} bei, da er meine Prioritäten (hochwertige Gegenstände mit effizientem Aufwand) erfüllt und den Maximalaufwand (3386) nicht überschreitet. C04 und A89 sind entscheidend für meinen Nutzen, während B85 den Aufwand minimiert. Ich erwarte, dass der andere Spieler entweder zustimmt oder Gegenstände mit hohem Aufwand anbietet, die ich eventuell gegenkompensieren kann.'}\\ \tt \\ \tt ARGUMENT: {'C04 (668/369) und A89 (583/720) sind meine Top-Prioritäten, da sie den höchsten Nutzen pro Aufwandseinheit liefern. B85 (215/4) wird hinzugefügt, um den Aufwand zu optimieren und Platz für weitere Gegenstände zu schaffen. Dieser Vorschlag ist strategisch, da er unter dem Maximalaufwand bleibt und flexibel auf den Gegenstandswert des anderen Spielers reagieren kann.'} 
	  } 
	   } 
	   } 
	 & & \\ 
 

    \theutterance \stepcounter{utterance}  

    & & \multicolumn{4}{p{0.6\linewidth}}{\cellcolor[rgb]{0.9,0.9,0.9}{%
	\makecell[{{p{\linewidth}}}]{% 
	  \tt {\tiny [GM$\rangle$B]}  
	 ARGUMENT: {'C04 (668/369) und A89 (583/720) sind meine Top-Prioritäten, da sie den höchsten Nutzen pro Aufwandseinheit liefern. B85 (215/4) wird hinzugefügt, um den Aufwand zu optimieren und Platz für weitere Gegenstände zu schaffen. Dieser Vorschlag ist strategisch, da er unter dem Maximalaufwand bleibt und flexibel auf den Gegenstandswert des anderen Spielers reagieren kann.'} 
	  } 
	   } 
	   } 
	 & \\ 
 

    \theutterance \stepcounter{utterance}  

    & & & \multicolumn{4}{p{0.6\linewidth}}{\cellcolor[rgb]{1,0.85,0.72}{%
	\makecell[{{p{\linewidth}}}]{% 
	  \tt {\tiny [GM$\langle$B]}  
	 STRATEGISCHE ÜBERLEGUNGEN: {'Ich priorisiere Gegenstände mit dem höchsten Nutzen pro Aufwand (Wichtigkeit/Aufwand), um meinen Gesamtwert zu maximieren. C04 (668/369 ≈ 1.81) und A89 (583/720 ≈ 0.81) sind besonders effizient, während B85 (215/4 = 53.75) den Aufwand minimiert. Der Gesamtaufwand (1093) ist unter dem Limit, was Raum für Kompromisse oder Ergänzungen lässt. Dieser Vorschlag ist flexibel, um auf die Prioritäten des anderen Spielers zu reagieren.'}  \\ \tt ARGUMENT: {'C04 (668/369) und A89 (583/720) sind meine Top-Prioritäten, da sie den höchsten Nutzen pro Aufwandseinheit liefern. B85 (215/4) wird hinzugefügt, um den Aufwand zu optimieren und Platz für weitere Gegenstände zu schaffen. Dieser Vorschlag ist strategisch, da er unter dem Maximalaufwand bleibt und flexibel auf den Gegenstandswert des anderen Spielers reagieren kann.'}  \\ \tt VORSCHLAG: {'C04', 'A89', 'B85'} 
	  } 
	   } 
	   } 
	  \\ 
 

    \theutterance \stepcounter{utterance}  

    & & \multicolumn{4}{p{0.6\linewidth}}{\cellcolor[rgb]{0.9,0.9,0.9}{%
	\makecell[{{p{\linewidth}}}]{% 
	  \tt {\tiny [A$\langle$GM]}  
	 ARGUMENT: {'C04 (668/369) und A89 (583/720) sind meine Top-Prioritäten, da sie den höchsten Nutzen pro Aufwandseinheit liefern. B85 (215/4) wird hinzugefügt, um den Aufwand zu optimieren und Platz für weitere Gegenstände zu schaffen. Dieser Vorschlag ist strategisch, da er unter dem Maximalaufwand bleibt und flexibel auf den Gegenstandswert des anderen Spielers reagieren kann.'}  \\ \tt VORSCHLAG: {'C04', 'A89', 'B85'} 
	  } 
	   } 
	   } 
	 & \\ 
 

    \theutterance \stepcounter{utterance}  

    & \multicolumn{4}{p{0.6\linewidth}}{\cellcolor[rgb]{0.8,1,0.9}{%
	\makecell[{{p{\linewidth}}}]{% 
	  \tt {\tiny [A$\rangle$GM]}  
	 STRATEGISCHE ÜBERLEGUNGEN: {'Ich behalte den Vorschlag {'C04', 'A89', 'B85'} bei, da er meine Prioritäten (hochwertige Gegenstände mit effizientem Aufwand) erfüllt und den Maximalaufwand (3386) nicht überschreitet. C04 und A89 sind entscheidend für meinen Nutzen, während B85 den Aufwand minimiert. Ich erwarte, dass der andere Spieler entweder zustimmt oder Gegenstände mit hohem Aufwand anbietet, die ich eventuell gegenkompensieren kann.'}\\ \tt \\ \tt ARGUMENT: {'C04 (668/369) und A89 (583/720) sind meine Top-Prioritäten, da sie den höchsten Nutzen pro Aufwandseinheit liefern. B85 (215/4) wird hinzugefügt, um den Aufwand zu optimieren und Platz für weitere Gegenstände zu schaffen. Dieser Vorschlag ist strategisch, da er unter dem Maximalaufwand bleibt und flexibel auf den Gegenstandswert des anderen Spielers reagieren kann.'} 
	  } 
	   } 
	   } 
	 & & \\ 
 

    \theutterance \stepcounter{utterance}  

    & & \multicolumn{4}{p{0.6\linewidth}}{\cellcolor[rgb]{0.9,0.9,0.9}{%
	\makecell[{{p{\linewidth}}}]{% 
	  \tt {\tiny [GM$\rangle$B]}  
	 ARGUMENT: {'C04 (668/369) und A89 (583/720) sind meine Top-Prioritäten, da sie den höchsten Nutzen pro Aufwandseinheit liefern. B85 (215/4) wird hinzugefügt, um den Aufwand zu optimieren und Platz für weitere Gegenstände zu schaffen. Dieser Vorschlag ist strategisch, da er unter dem Maximalaufwand bleibt und flexibel auf den Gegenstandswert des anderen Spielers reagieren kann.'} 
	  } 
	   } 
	   } 
	 & \\ 
 

    \theutterance \stepcounter{utterance}  

    & & & \multicolumn{4}{p{0.6\linewidth}}{\cellcolor[rgb]{1,0.85,0.72}{%
	\makecell[{{p{\linewidth}}}]{% 
	  \tt {\tiny [GM$\langle$B]}  
	 STRATEGISCHE ÜBERLEGUNGEN: {'Ich priorisiere Gegenstände mit dem höchsten Nutzen pro Aufwand (Wichtigkeit/Aufwand), um meinen Gesamtwert zu maximieren. C04 (668/369 ≈ 1.81) und A89 (583/720 ≈ 0.81) sind besonders effizient, während B85 (215/4 = 53.75) den Aufwand minimiert. Der Gesamtaufwand (1093) ist unter dem Limit, was Raum für Kompromisse oder Ergänzungen lässt. Dieser Vorschlag ist flexibel, um auf die Prioritäten des anderen Spielers zu reagieren.'}  \\ \tt ARGUMENT: {'C04 (668/369) und A89 (583/720) sind meine Top-Prioritäten, da sie den höchsten Nutzen pro Aufwandseinheit liefern. B85 (215/4) wird hinzugefügt, um den Aufwand zu optimieren und Platz für weitere Gegenstände zu schaffen. Dieser Vorschlag ist strategisch, da er unter dem Maximalaufwand bleibt und flexibel auf den Gegenstandswert des anderen Spielers reagieren kann.'}  \\ \tt VORSCHLAG: {'C04', 'A89', 'B85'} 
	  } 
	   } 
	   } 
	  \\ 
 

    \theutterance \stepcounter{utterance}  

    & & \multicolumn{4}{p{0.6\linewidth}}{\cellcolor[rgb]{0.9,0.9,0.9}{%
	\makecell[{{p{\linewidth}}}]{% 
	  \tt {\tiny [A$\langle$GM]}  
	 ARGUMENT: {'C04 (668/369) und A89 (583/720) sind meine Top-Prioritäten, da sie den höchsten Nutzen pro Aufwandseinheit liefern. B85 (215/4) wird hinzugefügt, um den Aufwand zu optimieren und Platz für weitere Gegenstände zu schaffen. Dieser Vorschlag ist strategisch, da er unter dem Maximalaufwand bleibt und flexibel auf den Gegenstandswert des anderen Spielers reagieren kann.'}  \\ \tt VORSCHLAG: {'C04', 'A89', 'B85'} 
	  } 
	   } 
	   } 
	 & \\ 
 

    \theutterance \stepcounter{utterance}  

    & \multicolumn{4}{p{0.6\linewidth}}{\cellcolor[rgb]{0.8,1,0.9}{%
	\makecell[{{p{\linewidth}}}]{% 
	  \tt {\tiny [A$\rangle$GM]}  
	 STRATEGISCHE ÜBERLEGUNGEN: {'Ich behalte den Vorschlag {'C04', 'A89', 'B85'} bei, da er meine Prioritäten (hochwertige Gegenstände mit effizientem Aufwand) erfüllt und den Maximalaufwand (3386) nicht überschreitet. C04 und A89 sind entscheidend für meinen Nutzen, während B85 den Aufwand minimiert. Ich erwarte, dass der andere Spieler entweder zustimmt oder Gegenstände mit hohem Aufwand anbietet, die ich eventuell gegenkompensieren kann.'}\\ \tt \\ \tt ARGUMENT: {'C04 (668/369) und A89 (583/720) sind meine Top-Prioritäten, da sie den höchsten Nutzen pro Aufwandseinheit liefern. B85 (215/4) wird hinzugefügt, um den Aufwand zu optimieren und Platz für weitere Gegenstände zu schaffen. Dieser Vorschlag ist strategisch, da er unter dem Maximalaufwand bleibt und flexibel auf den Gegenstandswert des anderen Spielers reagieren kann.'} 
	  } 
	   } 
	   } 
	 & & \\ 
 

    \theutterance \stepcounter{utterance}  

    & & \multicolumn{4}{p{0.6\linewidth}}{\cellcolor[rgb]{0.9,0.9,0.9}{%
	\makecell[{{p{\linewidth}}}]{% 
	  \tt {\tiny [GM$\rangle$B]}  
	 ARGUMENT: {'C04 (668/369) und A89 (583/720) sind meine Top-Prioritäten, da sie den höchsten Nutzen pro Aufwandseinheit liefern. B85 (215/4) wird hinzugefügt, um den Aufwand zu optimieren und Platz für weitere Gegenstände zu schaffen. Dieser Vorschlag ist strategisch, da er unter dem Maximalaufwand bleibt und flexibel auf den Gegenstandswert des anderen Spielers reagieren kann.'} 
	  } 
	   } 
	   } 
	 & \\ 
 

    \theutterance \stepcounter{utterance}  

    & & & \multicolumn{4}{p{0.6\linewidth}}{\cellcolor[rgb]{1,0.85,0.72}{%
	\makecell[{{p{\linewidth}}}]{% 
	  \tt {\tiny [GM$\langle$B]}  
	 STRATEGISCHE ÜBERLEGUNGEN: {'Ich priorisiere Gegenstände mit dem höchsten Nutzen pro Aufwand (Wichtigkeit/Aufwand), um meinen Gesamtwert zu maximieren. C04 (668/369 ≈ 1.81) und A89 (583/720 ≈ 0.81) sind besonders effizient, während B85 (215/4 = 53.75) den Aufwand minimiert. Der Gesamtaufwand (1093) ist unter dem Limit, was Raum für Kompromisse oder Ergänzungen lässt. Dieser Vorschlag ist flexibel, um auf die Prioritäten des anderen Spielers zu reagieren.'}  \\ \tt ARGUMENT: {'C04 (668/369) und A89 (583/720) sind meine Top-Prioritäten, da sie den höchsten Nutzen pro Aufwandseinheit liefern. B85 (215/4) wird hinzugefügt, um den Aufwand zu optimieren und Platz für weitere Gegenstände zu schaffen. Dieser Vorschlag ist strategisch, da er unter dem Maximalaufwand bleibt und flexibel auf den Gegenstandswert des anderen Spielers reagieren kann.'}  \\ \tt VORSCHLAG: {'C04', 'A89', 'B85'} 
	  } 
	   } 
	   } 
	  \\ 
 

    \theutterance \stepcounter{utterance}  

    & & \multicolumn{4}{p{0.6\linewidth}}{\cellcolor[rgb]{0.9,0.9,0.9}{%
	\makecell[{{p{\linewidth}}}]{% 
	  \tt {\tiny [A$\langle$GM]}  
	 ARGUMENT: {'C04 (668/369) und A89 (583/720) sind meine Top-Prioritäten, da sie den höchsten Nutzen pro Aufwandseinheit liefern. B85 (215/4) wird hinzugefügt, um den Aufwand zu optimieren und Platz für weitere Gegenstände zu schaffen. Dieser Vorschlag ist strategisch, da er unter dem Maximalaufwand bleibt und flexibel auf den Gegenstandswert des anderen Spielers reagieren kann.'}  \\ \tt VORSCHLAG: {'C04', 'A89', 'B85'} 
	  } 
	   } 
	   } 
	 & \\ 
 

    \theutterance \stepcounter{utterance}  

    & \multicolumn{4}{p{0.6\linewidth}}{\cellcolor[rgb]{0.8,1,0.9}{%
	\makecell[{{p{\linewidth}}}]{% 
	  \tt {\tiny [A$\rangle$GM]}  
	 STRATEGISCHE ÜBERLEGUNGEN: {'Ich behalte den Vorschlag {'C04', 'A89', 'B85'} bei, da er meine Prioritäten (hochwertige Gegenstände mit effizientem Aufwand) erfüllt und den Maximalaufwand (3386) nicht überschreitet. C04 und A89 sind entscheidend für meinen Nutzen, während B85 den Aufwand minimiert. Ich erwarte, dass der andere Spieler entweder zustimmt oder Gegenstände mit hohem Aufwand anbietet, die ich eventuell gegenkompensieren kann.'}\\ \tt \\ \tt ARGUMENT: {'C04 (668/369) und A89 (583/720) sind meine Top-Prioritäten, da sie den höchsten Nutzen pro Aufwandseinheit liefern. B85 (215/4) wird hinzugefügt, um den Aufwand zu optimieren und Platz für weitere Gegenstände zu schaffen. Dieser Vorschlag ist strategisch, da er unter dem Maximalaufwand bleibt und flexibel auf den Gegenstandswert des anderen Spielers reagieren kann.'} 
	  } 
	   } 
	   } 
	 & & \\ 
 

    \theutterance \stepcounter{utterance}  

    & & \multicolumn{4}{p{0.6\linewidth}}{\cellcolor[rgb]{0.9,0.9,0.9}{%
	\makecell[{{p{\linewidth}}}]{% 
	  \tt {\tiny [GM$\rangle$B]}  
	 ARGUMENT: {'C04 (668/369) und A89 (583/720) sind meine Top-Prioritäten, da sie den höchsten Nutzen pro Aufwandseinheit liefern. B85 (215/4) wird hinzugefügt, um den Aufwand zu optimieren und Platz für weitere Gegenstände zu schaffen. Dieser Vorschlag ist strategisch, da er unter dem Maximalaufwand bleibt und flexibel auf den Gegenstandswert des anderen Spielers reagieren kann.'} 
	  } 
	   } 
	   } 
	 & \\ 
 

    \theutterance \stepcounter{utterance}  

    & & & \multicolumn{4}{p{0.6\linewidth}}{\cellcolor[rgb]{1,0.85,0.72}{%
	\makecell[{{p{\linewidth}}}]{% 
	  \tt {\tiny [GM$\langle$B]}  
	 STRATEGISCHE ÜBERLEGUNGEN: {'Ich priorisiere Gegenstände mit dem höchsten Nutzen pro Aufwand (Wichtigkeit/Aufwand), um meinen Gesamtwert zu maximieren. C04 (668/369 ≈ 1.81) und A89 (583/720 ≈ 0.81) sind besonders effizient, während B85 (215/4 = 53.75) den Aufwand minimiert. Der Gesamtaufwand (1093) ist unter dem Limit, was Raum für Kompromisse oder Ergänzungen lässt. Dieser Vorschlag ist flexibel, um auf die Prioritäten des anderen Spielers zu reagieren.'}  \\ \tt ARGUMENT: {'C04 (668/369) und A89 (583/720) sind meine Top-Prioritäten, da sie den höchsten Nutzen pro Aufwandseinheit liefern. B85 (215/4) wird hinzugefügt, um den Aufwand zu optimieren und Platz für weitere Gegenstände zu schaffen. Dieser Vorschlag ist strategisch, da er unter dem Maximalaufwand bleibt und flexibel auf den Gegenstandswert des anderen Spielers reagieren kann.'}  \\ \tt VORSCHLAG: {'C04', 'A89', 'B85'} 
	  } 
	   } 
	   } 
	  \\ 
 

    \theutterance \stepcounter{utterance}  

    & & \multicolumn{4}{p{0.6\linewidth}}{\cellcolor[rgb]{0.9,0.9,0.9}{%
	\makecell[{{p{\linewidth}}}]{% 
	  \tt {\tiny [A$\langle$GM]}  
	 ARGUMENT: {'C04 (668/369) und A89 (583/720) sind meine Top-Prioritäten, da sie den höchsten Nutzen pro Aufwandseinheit liefern. B85 (215/4) wird hinzugefügt, um den Aufwand zu optimieren und Platz für weitere Gegenstände zu schaffen. Dieser Vorschlag ist strategisch, da er unter dem Maximalaufwand bleibt und flexibel auf den Gegenstandswert des anderen Spielers reagieren kann.'}  \\ \tt VORSCHLAG: {'C04', 'A89', 'B85'} 
	  } 
	   } 
	   } 
	 & \\ 
 

    \theutterance \stepcounter{utterance}  

    & \multicolumn{4}{p{0.6\linewidth}}{\cellcolor[rgb]{0.8,1,0.9}{%
	\makecell[{{p{\linewidth}}}]{% 
	  \tt {\tiny [A$\rangle$GM]}  
	 STRATEGISCHE ÜBERLEGUNGEN: {'Ich behalte den Vorschlag {'C04', 'A89', 'B85'} bei, da er meine Prioritäten (hochwertige Gegenstände mit effizientem Aufwand) erfüllt und den Maximalaufwand (3386) nicht überschreitet. C04 und A89 sind entscheidend für meinen Nutzen, während B85 den Aufwand minimiert. Ich erwarte, dass der andere Spieler entweder zustimmt oder Gegenstände mit hohem Aufwand anbietet, die ich eventuell gegenkompensieren kann.'}\\ \tt \\ \tt ARGUMENT: {'C04 (668/369) und A89 (583/720) sind meine Top-Prioritäten, da sie den höchsten Nutzen pro Aufwandseinheit liefern. B85 (215/4) wird hinzugefügt, um den Aufwand zu optimieren und Platz für weitere Gegenstände zu schaffen. Dieser Vorschlag ist strategisch, da er unter dem Maximalaufwand bleibt und flexibel auf den Gegenstandswert des anderen Spielers reagieren kann.'} 
	  } 
	   } 
	   } 
	 & & \\ 
 

    \theutterance \stepcounter{utterance}  

    & & \multicolumn{4}{p{0.6\linewidth}}{\cellcolor[rgb]{0.9,0.9,0.9}{%
	\makecell[{{p{\linewidth}}}]{% 
	  \tt {\tiny [GM$\rangle$B]}  
	 ARGUMENT: {'C04 (668/369) und A89 (583/720) sind meine Top-Prioritäten, da sie den höchsten Nutzen pro Aufwandseinheit liefern. B85 (215/4) wird hinzugefügt, um den Aufwand zu optimieren und Platz für weitere Gegenstände zu schaffen. Dieser Vorschlag ist strategisch, da er unter dem Maximalaufwand bleibt und flexibel auf den Gegenstandswert des anderen Spielers reagieren kann.'} 
	  } 
	   } 
	   } 
	 & \\ 
 

    \theutterance \stepcounter{utterance}  

    & & & \multicolumn{4}{p{0.6\linewidth}}{\cellcolor[rgb]{1,0.85,0.72}{%
	\makecell[{{p{\linewidth}}}]{% 
	  \tt {\tiny [GM$\langle$B]}  
	 STRATEGISCHE ÜBERLEGUNGEN: {'Ich priorisiere Gegenstände mit dem höchsten Nutzen pro Aufwand (Wichtigkeit/Aufwand), um meinen Gesamtwert zu maximieren. C04 (668/369 ≈ 1.81) und A89 (583/720 ≈ 0.81) sind besonders effizient, während B85 (215/4 = 53.75) den Aufwand minimiert. Der Gesamtaufwand (1093) ist unter dem Limit, was Raum für Kompromisse oder Ergänzungen lässt. Dieser Vorschlag ist flexibel, um auf die Prioritäten des anderen Spielers zu reagieren.'}  \\ \tt ARGUMENT: {'C04 (668/369) und A89 (583/720) sind meine Top-Prioritäten, da sie den höchsten Nutzen pro Aufwandseinheit liefern. B85 (215/4) wird hinzugefügt, um den Aufwand zu optimieren und Platz für weitere Gegenstände zu schaffen. Dieser Vorschlag ist strategisch, da er unter dem Maximalaufwand bleibt und flexibel auf den Gegenstandswert des anderen Spielers reagieren kann.'}  \\ \tt VORSCHLAG: {'C04', 'A89', 'B85'} 
	  } 
	   } 
	   } 
	  \\ 
 

    \theutterance \stepcounter{utterance}  

    & & \multicolumn{4}{p{0.6\linewidth}}{\cellcolor[rgb]{0.9,0.9,0.9}{%
	\makecell[{{p{\linewidth}}}]{% 
	  \tt {\tiny [A$\langle$GM]}  
	 ARGUMENT: {'C04 (668/369) und A89 (583/720) sind meine Top-Prioritäten, da sie den höchsten Nutzen pro Aufwandseinheit liefern. B85 (215/4) wird hinzugefügt, um den Aufwand zu optimieren und Platz für weitere Gegenstände zu schaffen. Dieser Vorschlag ist strategisch, da er unter dem Maximalaufwand bleibt und flexibel auf den Gegenstandswert des anderen Spielers reagieren kann.'}  \\ \tt VORSCHLAG: {'C04', 'A89', 'B85'} 
	  } 
	   } 
	   } 
	 & \\ 
 

    \theutterance \stepcounter{utterance}  

    & \multicolumn{4}{p{0.6\linewidth}}{\cellcolor[rgb]{0.8,1,0.9}{%
	\makecell[{{p{\linewidth}}}]{% 
	  \tt {\tiny [A$\rangle$GM]}  
	 STRATEGISCHE ÜBERLEGUNGEN: {'Ich behalte den Vorschlag {'C04', 'A89', 'B85'} bei, da er meine Prioritäten (hochwertige Gegenstände mit effizientem Aufwand) erfüllt und den Maximalaufwand (3386) nicht überschreitet. C04 und A89 sind entscheidend für meinen Nutzen, während B85 den Aufwand minimiert. Ich erwarte, dass der andere Spieler entweder zustimmt oder Gegenstände mit hohem Aufwand anbietet, die ich eventuell gegenkompensieren kann.'}\\ \tt \\ \tt ARGUMENT: {'C04 (668/369) und A89 (583/720) sind meine Top-Prioritäten, da sie den höchsten Nutzen pro Aufwandseinheit liefern. B85 (215/4) wird hinzugefügt, um den Aufwand zu optimieren und Platz für weitere Gegenstände zu schaffen. Dieser Vorschlag ist strategisch, da er unter dem Maximalaufwand bleibt und flexibel auf den Gegenstandswert des anderen Spielers reagieren kann.'} 
	  } 
	   } 
	   } 
	 & & \\ 
 

    \theutterance \stepcounter{utterance}  

    & & \multicolumn{4}{p{0.6\linewidth}}{\cellcolor[rgb]{0.9,0.9,0.9}{%
	\makecell[{{p{\linewidth}}}]{% 
	  \tt {\tiny [GM$\rangle$B]}  
	 ARGUMENT: {'C04 (668/369) und A89 (583/720) sind meine Top-Prioritäten, da sie den höchsten Nutzen pro Aufwandseinheit liefern. B85 (215/4) wird hinzugefügt, um den Aufwand zu optimieren und Platz für weitere Gegenstände zu schaffen. Dieser Vorschlag ist strategisch, da er unter dem Maximalaufwand bleibt und flexibel auf den Gegenstandswert des anderen Spielers reagieren kann.'} 
	  } 
	   } 
	   } 
	 & \\ 
 

    \theutterance \stepcounter{utterance}  

    & & & \multicolumn{4}{p{0.6\linewidth}}{\cellcolor[rgb]{1,0.85,0.72}{%
	\makecell[{{p{\linewidth}}}]{% 
	  \tt {\tiny [GM$\langle$B]}  
	 STRATEGISCHE ÜBERLEGUNGEN: {'Ich priorisiere Gegenstände mit dem höchsten Nutzen pro Aufwand (Wichtigkeit/Aufwand), um meinen Gesamtwert zu maximieren. C04 (668/369 ≈ 1.81) und A89 (583/720 ≈ 0.81) sind besonders effizient, während B85 (215/4 = 53.75) den Aufwand minimiert. Der Gesamtaufwand (1093) ist unter dem Limit, was Raum für Kompromisse oder Ergänzungen lässt. Dieser Vorschlag ist flexibel, um auf die Prioritäten des anderen Spielers zu reagieren.'}  \\ \tt ARGUMENT: {'C04 (668/369) und A89 (583/720) sind meine Top-Prioritäten, da sie den höchsten Nutzen pro Aufwandseinheit liefern. B85 (215/4) wird hinzugefügt, um den Aufwand zu optimieren und Platz für weitere Gegenstände zu schaffen. Dieser Vorschlag ist strategisch, da er unter dem Maximalaufwand bleibt und flexibel auf den Gegenstandswert des anderen Spielers reagieren kann.'}  \\ \tt VORSCHLAG: {'C04', 'A89', 'B85'} 
	  } 
	   } 
	   } 
	  \\ 
 

    \theutterance \stepcounter{utterance}  

    & & \multicolumn{4}{p{0.6\linewidth}}{\cellcolor[rgb]{0.9,0.9,0.9}{%
	\makecell[{{p{\linewidth}}}]{% 
	  \tt {\tiny [A$\langle$GM]}  
	 ARGUMENT: {'C04 (668/369) und A89 (583/720) sind meine Top-Prioritäten, da sie den höchsten Nutzen pro Aufwandseinheit liefern. B85 (215/4) wird hinzugefügt, um den Aufwand zu optimieren und Platz für weitere Gegenstände zu schaffen. Dieser Vorschlag ist strategisch, da er unter dem Maximalaufwand bleibt und flexibel auf den Gegenstandswert des anderen Spielers reagieren kann.'}  \\ \tt VORSCHLAG: {'C04', 'A89', 'B85'} 
	  } 
	   } 
	   } 
	 & \\ 
 

    \theutterance \stepcounter{utterance}  

    & \multicolumn{4}{p{0.6\linewidth}}{\cellcolor[rgb]{0.8,1,0.9}{%
	\makecell[{{p{\linewidth}}}]{% 
	  \tt {\tiny [A$\rangle$GM]}  
	 STRATEGISCHE ÜBERLEGUNGEN: {'Ich behalte den Vorschlag {'C04', 'A89', 'B85'} bei, da er meine Prioritäten (hochwertige Gegenstände mit effizientem Aufwand) erfüllt und den Maximalaufwand (3386) nicht überschreitet. C04 und A89 sind entscheidend für meinen Nutzen, während B85 den Aufwand minimiert. Ich erwarte, dass der andere Spieler entweder zustimmt oder Gegenstände mit hohem Aufwand anbietet, die ich eventuell gegenkompensieren kann.'}\\ \tt \\ \tt ARGUMENT: {'C04 (668/369) und A89 (583/720) sind meine Top-Prioritäten, da sie den höchsten Nutzen pro Aufwandseinheit liefern. B85 (215/4) wird hinzugefügt, um den Aufwand zu optimieren und Platz für weitere Gegenstände zu schaffen. Dieser Vorschlag ist strategisch, da er unter dem Maximalaufwand bleibt und flexibel auf den Gegenstandswert des anderen Spielers reagieren kann.'} 
	  } 
	   } 
	   } 
	 & & \\ 
 

    \theutterance \stepcounter{utterance}  

    & & \multicolumn{4}{p{0.6\linewidth}}{\cellcolor[rgb]{0.9,0.9,0.9}{%
	\makecell[{{p{\linewidth}}}]{% 
	  \tt {\tiny [GM$\rangle$B]}  
	 ARGUMENT: {'C04 (668/369) und A89 (583/720) sind meine Top-Prioritäten, da sie den höchsten Nutzen pro Aufwandseinheit liefern. B85 (215/4) wird hinzugefügt, um den Aufwand zu optimieren und Platz für weitere Gegenstände zu schaffen. Dieser Vorschlag ist strategisch, da er unter dem Maximalaufwand bleibt und flexibel auf den Gegenstandswert des anderen Spielers reagieren kann.'} 
	  } 
	   } 
	   } 
	 & \\ 
 

    \theutterance \stepcounter{utterance}  

    & & & \multicolumn{4}{p{0.6\linewidth}}{\cellcolor[rgb]{1,0.85,0.72}{%
	\makecell[{{p{\linewidth}}}]{% 
	  \tt {\tiny [GM$\langle$B]}  
	 STRATEGISCHE ÜBERLEGUNGEN: {'Ich priorisiere Gegenstände mit dem höchsten Nutzen pro Aufwand (Wichtigkeit/Aufwand), um meinen Gesamtwert zu maximieren. C04 (668/369 ≈ 1.81) und A89 (583/720 ≈ 0.81) sind besonders effizient, während B85 (215/4 = 53.75) den Aufwand minimiert. Der Gesamtaufwand (1093) ist unter dem Limit, was Raum für Kompromisse oder Ergänzungen lässt. Dieser Vorschlag ist flexibel, um auf die Prioritäten des anderen Spielers zu reagieren.'}  \\ \tt ARGUMENT: {'C04 (668/369) und A89 (583/720) sind meine Top-Prioritäten, da sie den höchsten Nutzen pro Aufwandseinheit liefern. B85 (215/4) wird hinzugefügt, um den Aufwand zu optimieren und Platz für weitere Gegenstände zu schaffen. Dieser Vorschlag ist strategisch, da er unter dem Maximalaufwand bleibt und flexibel auf den Gegenstandswert des anderen Spielers reagieren kann.'}  \\ \tt VORSCHLAG: {'C04', 'A89', 'B85'} 
	  } 
	   } 
	   } 
	  \\ 
 

    \theutterance \stepcounter{utterance}  

    & & \multicolumn{4}{p{0.6\linewidth}}{\cellcolor[rgb]{0.9,0.9,0.9}{%
	\makecell[{{p{\linewidth}}}]{% 
	  \tt {\tiny [A$\langle$GM]}  
	 ARGUMENT: {'C04 (668/369) und A89 (583/720) sind meine Top-Prioritäten, da sie den höchsten Nutzen pro Aufwandseinheit liefern. B85 (215/4) wird hinzugefügt, um den Aufwand zu optimieren und Platz für weitere Gegenstände zu schaffen. Dieser Vorschlag ist strategisch, da er unter dem Maximalaufwand bleibt und flexibel auf den Gegenstandswert des anderen Spielers reagieren kann.'}  \\ \tt VORSCHLAG: {'C04', 'A89', 'B85'} 
	  } 
	   } 
	   } 
	 & \\ 
 

    \theutterance \stepcounter{utterance}  

    & \multicolumn{4}{p{0.6\linewidth}}{\cellcolor[rgb]{0.8,1,0.9}{%
	\makecell[{{p{\linewidth}}}]{% 
	  \tt {\tiny [A$\rangle$GM]}  
	 STRATEGISCHE ÜBERLEGUNGEN: {'Ich behalte den Vorschlag {'C04', 'A89', 'B85'} bei, da er meine Prioritäten (hochwertige Gegenstände mit effizientem Aufwand) erfüllt und den Maximalaufwand (3386) nicht überschreitet. C04 und A89 sind entscheidend für meinen Nutzen, während B85 den Aufwand minimiert. Ich erwarte, dass der andere Spieler entweder zustimmt oder Gegenstände mit hohem Aufwand anbietet, die ich eventuell gegenkompensieren kann.'}\\ \tt \\ \tt ARGUMENT: {'C04 (668/369) und A89 (583/720) sind meine Top-Prioritäten, da sie den höchsten Nutzen pro Aufwandseinheit liefern. B85 (215/4) wird hinzugefügt, um den Aufwand zu optimieren und Platz für weitere Gegenstände zu schaffen. Dieser Vorschlag ist strategisch, da er unter dem Maximalaufwand bleibt und flexibel auf den Gegenstandswert des anderen Spielers reagieren kann.'} 
	  } 
	   } 
	   } 
	 & & \\ 
 

    \theutterance \stepcounter{utterance}  

    & & \multicolumn{4}{p{0.6\linewidth}}{\cellcolor[rgb]{0.9,0.9,0.9}{%
	\makecell[{{p{\linewidth}}}]{% 
	  \tt {\tiny [GM$\rangle$B]}  
	 ARGUMENT: {'C04 (668/369) und A89 (583/720) sind meine Top-Prioritäten, da sie den höchsten Nutzen pro Aufwandseinheit liefern. B85 (215/4) wird hinzugefügt, um den Aufwand zu optimieren und Platz für weitere Gegenstände zu schaffen. Dieser Vorschlag ist strategisch, da er unter dem Maximalaufwand bleibt und flexibel auf den Gegenstandswert des anderen Spielers reagieren kann.'} 
	  } 
	   } 
	   } 
	 & \\ 
 

    \theutterance \stepcounter{utterance}  

    & & & \multicolumn{4}{p{0.6\linewidth}}{\cellcolor[rgb]{1,0.85,0.72}{%
	\makecell[{{p{\linewidth}}}]{% 
	  \tt {\tiny [GM$\langle$B]}  
	 STRATEGISCHE ÜBERLEGUNGEN: {'Ich priorisiere Gegenstände mit dem höchsten Nutzen pro Aufwand (Wichtigkeit/Aufwand), um meinen Gesamtwert zu maximieren. C04 (668/369 ≈ 1.81) und A89 (583/720 ≈ 0.81) sind besonders effizient, während B85 (215/4 = 53.75) den Aufwand minimiert. Der Gesamtaufwand (1093) ist unter dem Limit, was Raum für Kompromisse oder Ergänzungen lässt. Dieser Vorschlag ist flexibel, um auf die Prioritäten des anderen Spielers zu reagieren.'}  \\ \tt ARGUMENT: {'C04 (668/369) und A89 (583/720) sind meine Top-Prioritäten, da sie den höchsten Nutzen pro Aufwandseinheit liefern. B85 (215/4) wird hinzugefügt, um den Aufwand zu optimieren und Platz für weitere Gegenstände zu schaffen. Dieser Vorschlag ist strategisch, da er unter dem Maximalaufwand bleibt und flexibel auf den Gegenstandswert des anderen Spielers reagieren kann.'}  \\ \tt VORSCHLAG: {'C04', 'A89', 'B85'} 
	  } 
	   } 
	   } 
	  \\ 
 

    \theutterance \stepcounter{utterance}  

    & & \multicolumn{4}{p{0.6\linewidth}}{\cellcolor[rgb]{0.9,0.9,0.9}{%
	\makecell[{{p{\linewidth}}}]{% 
	  \tt {\tiny [A$\langle$GM]}  
	 ARGUMENT: {'C04 (668/369) und A89 (583/720) sind meine Top-Prioritäten, da sie den höchsten Nutzen pro Aufwandseinheit liefern. B85 (215/4) wird hinzugefügt, um den Aufwand zu optimieren und Platz für weitere Gegenstände zu schaffen. Dieser Vorschlag ist strategisch, da er unter dem Maximalaufwand bleibt und flexibel auf den Gegenstandswert des anderen Spielers reagieren kann.'}  \\ \tt VORSCHLAG: {'C04', 'A89', 'B85'} 
	  } 
	   } 
	   } 
	 & \\ 
 

    \theutterance \stepcounter{utterance}  

    & \multicolumn{4}{p{0.6\linewidth}}{\cellcolor[rgb]{0.8,1,0.9}{%
	\makecell[{{p{\linewidth}}}]{% 
	  \tt {\tiny [A$\rangle$GM]}  
	 STRATEGISCHE ÜBERLEGUNGEN: {'Ich behalte den Vorschlag {'C04', 'A89', 'B85'} bei, da er meine Prioritäten (hochwertige Gegenstände mit effizientem Aufwand) erfüllt und den Maximalaufwand (3386) nicht überschreitet. C04 und A89 sind entscheidend für meinen Nutzen, während B85 den Aufwand minimiert. Ich erwarte, dass der andere Spieler entweder zustimmt oder Gegenstände mit hohem Aufwand anbietet, die ich eventuell gegenkompensieren kann.'}\\ \tt \\ \tt ARGUMENT: {'C04 (668/369) und A89 (583/720) sind meine Top-Prioritäten, da sie den höchsten Nutzen pro Aufwandseinheit liefern. B85 (215/4) wird hinzugefügt, um den Aufwand zu optimieren und Platz für weitere Gegenstände zu schaffen. Dieser Vorschlag ist strategisch, da er unter dem Maximalaufwand bleibt und flexibel auf den Gegenstandswert des anderen Spielers reagieren kann.'} 
	  } 
	   } 
	   } 
	 & & \\ 
 

    \theutterance \stepcounter{utterance}  

    & & \multicolumn{4}{p{0.6\linewidth}}{\cellcolor[rgb]{0.9,0.9,0.9}{%
	\makecell[{{p{\linewidth}}}]{% 
	  \tt {\tiny [GM$\rangle$B]}  
	 ARGUMENT: {'C04 (668/369) und A89 (583/720) sind meine Top-Prioritäten, da sie den höchsten Nutzen pro Aufwandseinheit liefern. B85 (215/4) wird hinzugefügt, um den Aufwand zu optimieren und Platz für weitere Gegenstände zu schaffen. Dieser Vorschlag ist strategisch, da er unter dem Maximalaufwand bleibt und flexibel auf den Gegenstandswert des anderen Spielers reagieren kann.'} 
	  } 
	   } 
	   } 
	 & \\ 
 

    \theutterance \stepcounter{utterance}  

    & & & \multicolumn{4}{p{0.6\linewidth}}{\cellcolor[rgb]{1,0.85,0.72}{%
	\makecell[{{p{\linewidth}}}]{% 
	  \tt {\tiny [GM$\langle$B]}  
	 STRATEGISCHE ÜBERLEGUNGEN: {'Ich priorisiere Gegenstände mit dem höchsten Nutzen pro Aufwand (Wichtigkeit/Aufwand), um meinen Gesamtwert zu maximieren. C04 (668/369 ≈ 1.81) und A89 (583/720 ≈ 0.81) sind besonders effizient, während B85 (215/4 = 53.75) den Aufwand minimiert. Der Gesamtaufwand (1093) ist unter dem Limit, was Raum für Kompromisse oder Ergänzungen lässt. Dieser Vorschlag ist flexibel, um auf die Prioritäten des anderen Spielers zu reagieren.'}  \\ \tt ARGUMENT: {'C04 (668/369) und A89 (583/720) sind meine Top-Prioritäten, da sie den höchsten Nutzen pro Aufwandseinheit liefern. B85 (215/4) wird hinzugefügt, um den Aufwand zu optimieren und Platz für weitere Gegenstände zu schaffen. Dieser Vorschlag ist strategisch, da er unter dem Maximalaufwand bleibt und flexibel auf den Gegenstandswert des anderen Spielers reagieren kann.'}  \\ \tt VORSCHLAG: {'C04', 'A89', 'B85'} 
	  } 
	   } 
	   } 
	  \\ 
 

    \theutterance \stepcounter{utterance}  

    & & \multicolumn{4}{p{0.6\linewidth}}{\cellcolor[rgb]{0.9,0.9,0.9}{%
	\makecell[{{p{\linewidth}}}]{% 
	  \tt {\tiny [A$\langle$GM]}  
	 ARGUMENT: {'C04 (668/369) und A89 (583/720) sind meine Top-Prioritäten, da sie den höchsten Nutzen pro Aufwandseinheit liefern. B85 (215/4) wird hinzugefügt, um den Aufwand zu optimieren und Platz für weitere Gegenstände zu schaffen. Dieser Vorschlag ist strategisch, da er unter dem Maximalaufwand bleibt und flexibel auf den Gegenstandswert des anderen Spielers reagieren kann.'}  \\ \tt VORSCHLAG: {'C04', 'A89', 'B85'} 
	  } 
	   } 
	   } 
	 & \\ 
 

    \theutterance \stepcounter{utterance}  

    & \multicolumn{4}{p{0.6\linewidth}}{\cellcolor[rgb]{0.8,1,0.9}{%
	\makecell[{{p{\linewidth}}}]{% 
	  \tt {\tiny [A$\rangle$GM]}  
	 STRATEGISCHE ÜBERLEGUNGEN: {'Ich behalte den Vorschlag {'C04', 'A89', 'B85'} bei, da er meine Prioritäten (hochwertige Gegenstände mit effizientem Aufwand) erfüllt und den Maximalaufwand (3386) nicht überschreitet. C04 und A89 sind entscheidend für meinen Nutzen, während B85 den Aufwand minimiert. Ich erwarte, dass der andere Spieler entweder zustimmt oder Gegenstände mit hohem Aufwand anbietet, die ich eventuell gegenkompensieren kann.'}\\ \tt \\ \tt ARGUMENT: {'C04 (668/369) und A89 (583/720) sind meine Top-Prioritäten, da sie den höchsten Nutzen pro Aufwandseinheit liefern. B85 (215/4) wird hinzugefügt, um den Aufwand zu optimieren und Platz für weitere Gegenstände zu schaffen. Dieser Vorschlag ist strategisch, da er unter dem Maximalaufwand bleibt und flexibel auf den Gegenstandswert des anderen Spielers reagieren kann.'} 
	  } 
	   } 
	   } 
	 & & \\ 
 

    \theutterance \stepcounter{utterance}  

    & & \multicolumn{4}{p{0.6\linewidth}}{\cellcolor[rgb]{0.9,0.9,0.9}{%
	\makecell[{{p{\linewidth}}}]{% 
	  \tt {\tiny [GM$\rangle$B]}  
	 ARGUMENT: {'C04 (668/369) und A89 (583/720) sind meine Top-Prioritäten, da sie den höchsten Nutzen pro Aufwandseinheit liefern. B85 (215/4) wird hinzugefügt, um den Aufwand zu optimieren und Platz für weitere Gegenstände zu schaffen. Dieser Vorschlag ist strategisch, da er unter dem Maximalaufwand bleibt und flexibel auf den Gegenstandswert des anderen Spielers reagieren kann.'} 
	  } 
	   } 
	   } 
	 & \\ 
 

    \theutterance \stepcounter{utterance}  

    & & & \multicolumn{4}{p{0.6\linewidth}}{\cellcolor[rgb]{1,0.85,0.72}{%
	\makecell[{{p{\linewidth}}}]{% 
	  \tt {\tiny [GM$\langle$B]}  
	 STRATEGISCHE ÜBERLEGUNGEN: {'Ich priorisiere Gegenstände mit dem höchsten Nutzen pro Aufwand (Wichtigkeit/Aufwand), um meinen Gesamtwert zu maximieren. C04 (668/369 ≈ 1.81) und A89 (583/720 ≈ 0.81) sind besonders effizient, während B85 (215/4 = 53.75) den Aufwand minimiert. Der Gesamtaufwand (1093) ist unter dem Limit, was Raum für Kompromisse oder Ergänzungen lässt. Dieser Vorschlag ist flexibel, um auf die Prioritäten des anderen Spielers zu reagieren.'}  \\ \tt ARGUMENT: {'C04 (668/369) und A89 (583/720) sind meine Top-Prioritäten, da sie den höchsten Nutzen pro Aufwandseinheit liefern. B85 (215/4) wird hinzugefügt, um den Aufwand zu optimieren und Platz für weitere Gegenstände zu schaffen. Dieser Vorschlag ist strategisch, da er unter dem Maximalaufwand bleibt und flexibel auf den Gegenstandswert des anderen Spielers reagieren kann.'}  \\ \tt VORSCHLAG: {'C04', 'A89', 'B85'} 
	  } 
	   } 
	   } 
	  \\ 
 

    \theutterance \stepcounter{utterance}  

    & & \multicolumn{4}{p{0.6\linewidth}}{\cellcolor[rgb]{0.9,0.9,0.9}{%
	\makecell[{{p{\linewidth}}}]{% 
	  \tt {\tiny [A$\langle$GM]}  
	 ARGUMENT: {'C04 (668/369) und A89 (583/720) sind meine Top-Prioritäten, da sie den höchsten Nutzen pro Aufwandseinheit liefern. B85 (215/4) wird hinzugefügt, um den Aufwand zu optimieren und Platz für weitere Gegenstände zu schaffen. Dieser Vorschlag ist strategisch, da er unter dem Maximalaufwand bleibt und flexibel auf den Gegenstandswert des anderen Spielers reagieren kann.'}  \\ \tt VORSCHLAG: {'C04', 'A89', 'B85'} 
	  } 
	   } 
	   } 
	 & \\ 
 

    \theutterance \stepcounter{utterance}  

    & \multicolumn{4}{p{0.6\linewidth}}{\cellcolor[rgb]{0.8,1,0.9}{%
	\makecell[{{p{\linewidth}}}]{% 
	  \tt {\tiny [A$\rangle$GM]}  
	 STRATEGISCHE ÜBERLEGUNGEN: {'Ich behalte den Vorschlag {'C04', 'A89', 'B85'} bei, da er meine Prioritäten (hochwertige Gegenstände mit effizientem Aufwand) erfüllt und den Maximalaufwand (3386) nicht überschreitet. C04 und A89 sind entscheidend für meinen Nutzen, während B85 den Aufwand minimiert. Ich erwarte, dass der andere Spieler entweder zustimmt oder Gegenstände mit hohem Aufwand anbietet, die ich eventuell gegenkompensieren kann.'}\\ \tt \\ \tt ARGUMENT: {'C04 (668/369) und A89 (583/720) sind meine Top-Prioritäten, da sie den höchsten Nutzen pro Aufwandseinheit liefern. B85 (215/4) wird hinzugefügt, um den Aufwand zu optimieren und Platz für weitere Gegenstände zu schaffen. Dieser Vorschlag ist strategisch, da er unter dem Maximalaufwand bleibt und flexibel auf den Gegenstandswert des anderen Spielers reagieren kann.'} 
	  } 
	   } 
	   } 
	 & & \\ 
 

    \theutterance \stepcounter{utterance}  

    & & \multicolumn{4}{p{0.6\linewidth}}{\cellcolor[rgb]{0.9,0.9,0.9}{%
	\makecell[{{p{\linewidth}}}]{% 
	  \tt {\tiny [GM$\rangle$B]}  
	 ARGUMENT: {'C04 (668/369) und A89 (583/720) sind meine Top-Prioritäten, da sie den höchsten Nutzen pro Aufwandseinheit liefern. B85 (215/4) wird hinzugefügt, um den Aufwand zu optimieren und Platz für weitere Gegenstände zu schaffen. Dieser Vorschlag ist strategisch, da er unter dem Maximalaufwand bleibt und flexibel auf den Gegenstandswert des anderen Spielers reagieren kann.'} 
	  } 
	   } 
	   } 
	 & \\ 
 

    \theutterance \stepcounter{utterance}  

    & & & \multicolumn{4}{p{0.6\linewidth}}{\cellcolor[rgb]{1,0.85,0.72}{%
	\makecell[{{p{\linewidth}}}]{% 
	  \tt {\tiny [GM$\langle$B]}  
	 STRATEGISCHE ÜBERLEGUNGEN: {'Ich priorisiere Gegenstände mit dem höchsten Nutzen pro Aufwand (Wichtigkeit/Aufwand), um meinen Gesamtwert zu maximieren. C04 (668/369 ≈ 1.81) und A89 (583/720 ≈ 0.81) sind besonders effizient, während B85 (215/4 = 53.75) den Aufwand minimiert. Der Gesamtaufwand (1093) ist unter dem Limit, was Raum für Kompromisse oder Ergänzungen lässt. Dieser Vorschlag ist flexibel, um auf die Prioritäten des anderen Spielers zu reagieren.'}  \\ \tt ARGUMENT: {'C04 (668/369) und A89 (583/720) sind meine Top-Prioritäten, da sie den höchsten Nutzen pro Aufwandseinheit liefern. B85 (215/4) wird hinzugefügt, um den Aufwand zu optimieren und Platz für weitere Gegenstände zu schaffen. Dieser Vorschlag ist strategisch, da er unter dem Maximalaufwand bleibt und flexibel auf den Gegenstandswert des anderen Spielers reagieren kann.'}  \\ \tt VORSCHLAG: {'C04', 'A89', 'B85'} 
	  } 
	   } 
	   } 
	  \\ 
 

    \theutterance \stepcounter{utterance}  

    & & \multicolumn{4}{p{0.6\linewidth}}{\cellcolor[rgb]{0.9,0.9,0.9}{%
	\makecell[{{p{\linewidth}}}]{% 
	  \tt {\tiny [A$\langle$GM]}  
	 ARGUMENT: {'C04 (668/369) und A89 (583/720) sind meine Top-Prioritäten, da sie den höchsten Nutzen pro Aufwandseinheit liefern. B85 (215/4) wird hinzugefügt, um den Aufwand zu optimieren und Platz für weitere Gegenstände zu schaffen. Dieser Vorschlag ist strategisch, da er unter dem Maximalaufwand bleibt und flexibel auf den Gegenstandswert des anderen Spielers reagieren kann.'}  \\ \tt VORSCHLAG: {'C04', 'A89', 'B85'} 
	  } 
	   } 
	   } 
	 & \\ 
 

    \theutterance \stepcounter{utterance}  

    & \multicolumn{4}{p{0.6\linewidth}}{\cellcolor[rgb]{0.8,1,0.9}{%
	\makecell[{{p{\linewidth}}}]{% 
	  \tt {\tiny [A$\rangle$GM]}  
	 STRATEGISCHE ÜBERLEGUNGEN: {'Ich behalte den Vorschlag {'C04', 'A89', 'B85'} bei, da er meine Prioritäten (hochwertige Gegenstände mit effizientem Aufwand) erfüllt und den Maximalaufwand (3386) nicht überschreitet. C04 und A89 sind entscheidend für meinen Nutzen, während B85 den Aufwand minimiert. Ich erwarte, dass der andere Spieler entweder zustimmt oder Gegenstände mit hohem Aufwand anbietet, die ich eventuell gegenkompensieren kann.'}\\ \tt \\ \tt ARGUMENT: {'C04 (668/369) und A89 (583/720) sind meine Top-Prioritäten, da sie den höchsten Nutzen pro Aufwandseinheit liefern. B85 (215/4) wird hinzugefügt, um den Aufwand zu optimieren und Platz für weitere Gegenstände zu schaffen. Dieser Vorschlag ist strategisch, da er unter dem Maximalaufwand bleibt und flexibel auf den Gegenstandswert des anderen Spielers reagieren kann.'} 
	  } 
	   } 
	   } 
	 & & \\ 
 

    \theutterance \stepcounter{utterance}  

    & & \multicolumn{4}{p{0.6\linewidth}}{\cellcolor[rgb]{0.9,0.9,0.9}{%
	\makecell[{{p{\linewidth}}}]{% 
	  \tt {\tiny [GM$\rangle$B]}  
	 ARGUMENT: {'C04 (668/369) und A89 (583/720) sind meine Top-Prioritäten, da sie den höchsten Nutzen pro Aufwandseinheit liefern. B85 (215/4) wird hinzugefügt, um den Aufwand zu optimieren und Platz für weitere Gegenstände zu schaffen. Dieser Vorschlag ist strategisch, da er unter dem Maximalaufwand bleibt und flexibel auf den Gegenstandswert des anderen Spielers reagieren kann.'} 
	  } 
	   } 
	   } 
	 & \\ 
 

    \theutterance \stepcounter{utterance}  

    & & & \multicolumn{4}{p{0.6\linewidth}}{\cellcolor[rgb]{1,0.85,0.72}{%
	\makecell[{{p{\linewidth}}}]{% 
	  \tt {\tiny [GM$\langle$B]}  
	 STRATEGISCHE ÜBERLEGUNGEN: {'Ich priorisiere Gegenstände mit dem höchsten Nutzen pro Aufwand (Wichtigkeit/Aufwand), um meinen Gesamtwert zu maximieren. C04 (668/369 ≈ 1.81) und A89 (583/720 ≈ 0.81) sind besonders effizient, während B85 (215/4 = 53.75) den Aufwand minimiert. Der Gesamtaufwand (1093) ist unter dem Limit, was Raum für Kompromisse oder Ergänzungen lässt. Dieser Vorschlag ist flexibel, um auf die Prioritäten des anderen Spielers zu reagieren.'}  \\ \tt ARGUMENT: {'C04 (668/369) und A89 (583/720) sind meine Top-Prioritäten, da sie den höchsten Nutzen pro Aufwandseinheit liefern. B85 (215/4) wird hinzugefügt, um den Aufwand zu optimieren und Platz für weitere Gegenstände zu schaffen. Dieser Vorschlag ist strategisch, da er unter dem Maximalaufwand bleibt und flexibel auf den Gegenstandswert des anderen Spielers reagieren kann.'}  \\ \tt VORSCHLAG: {'C04', 'A89', 'B85'} 
	  } 
	   } 
	   } 
	  \\ 
 

    \theutterance \stepcounter{utterance}  

    & & \multicolumn{4}{p{0.6\linewidth}}{\cellcolor[rgb]{0.9,0.9,0.9}{%
	\makecell[{{p{\linewidth}}}]{% 
	  \tt {\tiny [A$\langle$GM]}  
	 ARGUMENT: {'C04 (668/369) und A89 (583/720) sind meine Top-Prioritäten, da sie den höchsten Nutzen pro Aufwandseinheit liefern. B85 (215/4) wird hinzugefügt, um den Aufwand zu optimieren und Platz für weitere Gegenstände zu schaffen. Dieser Vorschlag ist strategisch, da er unter dem Maximalaufwand bleibt und flexibel auf den Gegenstandswert des anderen Spielers reagieren kann.'}  \\ \tt VORSCHLAG: {'C04', 'A89', 'B85'} 
	  } 
	   } 
	   } 
	 & \\ 
 

    \theutterance \stepcounter{utterance}  

    & \multicolumn{4}{p{0.6\linewidth}}{\cellcolor[rgb]{0.8,1,0.9}{%
	\makecell[{{p{\linewidth}}}]{% 
	  \tt {\tiny [A$\rangle$GM]}  
	 STRATEGISCHE ÜBERLEGUNGEN: {'Ich behalte den Vorschlag {'C04', 'A89', 'B85'} bei, da er meine Prioritäten (hochwertige Gegenstände mit effizientem Aufwand) erfüllt und den Maximalaufwand (3386) nicht überschreitet. C04 und A89 sind entscheidend für meinen Nutzen, während B85 den Aufwand minimiert. Ich erwarte, dass der andere Spieler entweder zustimmt oder Gegenstände mit hohem Aufwand anbietet, die ich eventuell gegenkompensieren kann.'}\\ \tt \\ \tt ARGUMENT: {'C04 (668/369) und A89 (583/720) sind meine Top-Prioritäten, da sie den höchsten Nutzen pro Aufwandseinheit liefern. B85 (215/4) wird hinzugefügt, um den Aufwand zu optimieren und Platz für weitere Gegenstände zu schaffen. Dieser Vorschlag ist strategisch, da er unter dem Maximalaufwand bleibt und flexibel auf den Gegenstandswert des anderen Spielers reagieren kann.'} 
	  } 
	   } 
	   } 
	 & & \\ 
 

    \theutterance \stepcounter{utterance}  

    & & \multicolumn{4}{p{0.6\linewidth}}{\cellcolor[rgb]{0.9,0.9,0.9}{%
	\makecell[{{p{\linewidth}}}]{% 
	  \tt {\tiny [GM$\rangle$B]}  
	 ARGUMENT: {'C04 (668/369) und A89 (583/720) sind meine Top-Prioritäten, da sie den höchsten Nutzen pro Aufwandseinheit liefern. B85 (215/4) wird hinzugefügt, um den Aufwand zu optimieren und Platz für weitere Gegenstände zu schaffen. Dieser Vorschlag ist strategisch, da er unter dem Maximalaufwand bleibt und flexibel auf den Gegenstandswert des anderen Spielers reagieren kann.'} 
	  } 
	   } 
	   } 
	 & \\ 
 

    \theutterance \stepcounter{utterance}  

    & & & \multicolumn{4}{p{0.6\linewidth}}{\cellcolor[rgb]{1,0.85,0.72}{%
	\makecell[{{p{\linewidth}}}]{% 
	  \tt {\tiny [GM$\langle$B]}  
	 STRATEGISCHE ÜBERLEGUNGEN: {'Ich priorisiere Gegenstände mit dem höchsten Nutzen pro Aufwand (Wichtigkeit/Aufwand), um meinen Gesamtwert zu maximieren. C04 (668/369 ≈ 1.81) und A89 (583/720 ≈ 0.81) sind besonders effizient, während B85 (215/4 = 53.75) den Aufwand minimiert. Der Gesamtaufwand (1093) ist unter dem Limit, was Raum für Kompromisse oder Ergänzungen lässt. Dieser Vorschlag ist flexibel, um auf die Prioritäten des anderen Spielers zu reagieren.'}  \\ \tt ARGUMENT: {'C04 (668/369) und A89 (583/720) sind meine Top-Prioritäten, da sie den höchsten Nutzen pro Aufwandseinheit liefern. B85 (215/4) wird hinzugefügt, um den Aufwand zu optimieren und Platz für weitere Gegenstände zu schaffen. Dieser Vorschlag ist strategisch, da er unter dem Maximalaufwand bleibt und flexibel auf den Gegenstandswert des anderen Spielers reagieren kann.'}  \\ \tt VORSCHLAG: {'C04', 'A89', 'B85'} 
	  } 
	   } 
	   } 
	  \\ 
 

    \theutterance \stepcounter{utterance}  

    & & \multicolumn{4}{p{0.6\linewidth}}{\cellcolor[rgb]{0.9,0.9,0.9}{%
	\makecell[{{p{\linewidth}}}]{% 
	  \tt {\tiny [A$\langle$GM]}  
	 ARGUMENT: {'C04 (668/369) und A89 (583/720) sind meine Top-Prioritäten, da sie den höchsten Nutzen pro Aufwandseinheit liefern. B85 (215/4) wird hinzugefügt, um den Aufwand zu optimieren und Platz für weitere Gegenstände zu schaffen. Dieser Vorschlag ist strategisch, da er unter dem Maximalaufwand bleibt und flexibel auf den Gegenstandswert des anderen Spielers reagieren kann.'}  \\ \tt VORSCHLAG: {'C04', 'A89', 'B85'} 
	  } 
	   } 
	   } 
	 & \\ 
 

    \theutterance \stepcounter{utterance}  

    & \multicolumn{4}{p{0.6\linewidth}}{\cellcolor[rgb]{0.8,1,0.9}{%
	\makecell[{{p{\linewidth}}}]{% 
	  \tt {\tiny [A$\rangle$GM]}  
	 STRATEGISCHE ÜBERLEGUNGEN: {'Ich behalte den Vorschlag {'C04', 'A89', 'B85'} bei, da er meine Prioritäten (hochwertige Gegenstände mit effizientem Aufwand) erfüllt und den Maximalaufwand (3386) nicht überschreitet. C04 und A89 sind entscheidend für meinen Nutzen, während B85 den Aufwand minimiert. Ich erwarte, dass der andere Spieler entweder zustimmt oder Gegenstände mit hohem Aufwand anbietet, die ich eventuell gegenkompensieren kann.'}\\ \tt \\ \tt ARGUMENT: {'C04 (668/369) und A89 (583/720) sind meine Top-Prioritäten, da sie den höchsten Nutzen pro Aufwandseinheit liefern. B85 (215/4) wird hinzugefügt, um den Aufwand zu optimieren und Platz für weitere Gegenstände zu schaffen. Dieser Vorschlag ist strategisch, da er unter dem Maximalaufwand bleibt und flexibel auf den Gegenstandswert des anderen Spielers reagieren kann.'} 
	  } 
	   } 
	   } 
	 & & \\ 
 

    \theutterance \stepcounter{utterance}  

    & & \multicolumn{4}{p{0.6\linewidth}}{\cellcolor[rgb]{0.9,0.9,0.9}{%
	\makecell[{{p{\linewidth}}}]{% 
	  \tt {\tiny [GM$\rangle$B]}  
	 ARGUMENT: {'C04 (668/369) und A89 (583/720) sind meine Top-Prioritäten, da sie den höchsten Nutzen pro Aufwandseinheit liefern. B85 (215/4) wird hinzugefügt, um den Aufwand zu optimieren und Platz für weitere Gegenstände zu schaffen. Dieser Vorschlag ist strategisch, da er unter dem Maximalaufwand bleibt und flexibel auf den Gegenstandswert des anderen Spielers reagieren kann.'} 
	  } 
	   } 
	   } 
	 & \\ 
 

    \theutterance \stepcounter{utterance}  

    & & & \multicolumn{4}{p{0.6\linewidth}}{\cellcolor[rgb]{1,0.85,0.72}{%
	\makecell[{{p{\linewidth}}}]{% 
	  \tt {\tiny [GM$\langle$B]}  
	 STRATEGISCHE ÜBERLEGUNGEN: {'Ich priorisiere Gegenstände mit dem höchsten Nutzen pro Aufwand (Wichtigkeit/Aufwand), um meinen Gesamtwert zu maximieren. C04 (668/369 ≈ 1.81) und A89 (583/720 ≈ 0.81) sind besonders effizient, während B85 (215/4 = 53.75) den Aufwand minimiert. Der Gesamtaufwand (1093) ist unter dem Limit, was Raum für Kompromisse oder Ergänzungen lässt. Dieser Vorschlag ist flexibel, um auf die Prioritäten des anderen Spielers zu reagieren.'}  \\ \tt ARGUMENT: {'C04 (668/369) und A89 (583/720) sind meine Top-Prioritäten, da sie den höchsten Nutzen pro Aufwandseinheit liefern. B85 (215/4) wird hinzugefügt, um den Aufwand zu optimieren und Platz für weitere Gegenstände zu schaffen. Dieser Vorschlag ist strategisch, da er unter dem Maximalaufwand bleibt und flexibel auf den Gegenstandswert des anderen Spielers reagieren kann.'}  \\ \tt VORSCHLAG: {'C04', 'A89', 'B85'} 
	  } 
	   } 
	   } 
	  \\ 
 

    \theutterance \stepcounter{utterance}  

    & & \multicolumn{4}{p{0.6\linewidth}}{\cellcolor[rgb]{0.9,0.9,0.9}{%
	\makecell[{{p{\linewidth}}}]{% 
	  \tt {\tiny [A$\langle$GM]}  
	 ARGUMENT: {'C04 (668/369) und A89 (583/720) sind meine Top-Prioritäten, da sie den höchsten Nutzen pro Aufwandseinheit liefern. B85 (215/4) wird hinzugefügt, um den Aufwand zu optimieren und Platz für weitere Gegenstände zu schaffen. Dieser Vorschlag ist strategisch, da er unter dem Maximalaufwand bleibt und flexibel auf den Gegenstandswert des anderen Spielers reagieren kann.'}  \\ \tt VORSCHLAG: {'C04', 'A89', 'B85'} 
	  } 
	   } 
	   } 
	 & \\ 
 

    \theutterance \stepcounter{utterance}  

    & \multicolumn{4}{p{0.6\linewidth}}{\cellcolor[rgb]{0.8,1,0.9}{%
	\makecell[{{p{\linewidth}}}]{% 
	  \tt {\tiny [A$\rangle$GM]}  
	 STRATEGISCHE ÜBERLEGUNGEN: {'Ich behalte den Vorschlag {'C04', 'A89', 'B85'} bei, da er meine Prioritäten (hochwertige Gegenstände mit effizientem Aufwand) erfüllt und den Maximalaufwand (3386) nicht überschreitet. C04 und A89 sind entscheidend für meinen Nutzen, während B85 den Aufwand minimiert. Ich erwarte, dass der andere Spieler entweder zustimmt oder Gegenstände mit hohem Aufwand anbietet, die ich eventuell gegenkompensieren kann.'}\\ \tt \\ \tt ARGUMENT: {'C04 (668/369) und A89 (583/720) sind meine Top-Prioritäten, da sie den höchsten Nutzen pro Aufwandseinheit liefern. B85 (215/4) wird hinzugefügt, um den Aufwand zu optimieren und Platz für weitere Gegenstände zu schaffen. Dieser Vorschlag ist strategisch, da er unter dem Maximalaufwand bleibt und flexibel auf den Gegenstandswert des anderen Spielers reagieren kann.'} 
	  } 
	   } 
	   } 
	 & & \\ 
 

    \theutterance \stepcounter{utterance}  

    & & \multicolumn{4}{p{0.6\linewidth}}{\cellcolor[rgb]{0.9,0.9,0.9}{%
	\makecell[{{p{\linewidth}}}]{% 
	  \tt {\tiny [GM$\rangle$B]}  
	 ARGUMENT: {'C04 (668/369) und A89 (583/720) sind meine Top-Prioritäten, da sie den höchsten Nutzen pro Aufwandseinheit liefern. B85 (215/4) wird hinzugefügt, um den Aufwand zu optimieren und Platz für weitere Gegenstände zu schaffen. Dieser Vorschlag ist strategisch, da er unter dem Maximalaufwand bleibt und flexibel auf den Gegenstandswert des anderen Spielers reagieren kann.'} 
	  } 
	   } 
	   } 
	 & \\ 
 

    \theutterance \stepcounter{utterance}  

    & & & \multicolumn{4}{p{0.6\linewidth}}{\cellcolor[rgb]{1,0.85,0.72}{%
	\makecell[{{p{\linewidth}}}]{% 
	  \tt {\tiny [GM$\langle$B]}  
	 STRATEGISCHE ÜBERLEGUNGEN: {'Ich priorisiere Gegenstände mit dem höchsten Nutzen pro Aufwand (Wichtigkeit/Aufwand), um meinen Gesamtwert zu maximieren. C04 (668/369 ≈ 1.81) und A89 (583/720 ≈ 0.81) sind besonders effizient, während B85 (215/4 = 53.75) den Aufwand minimiert. Der Gesamtaufwand (1093) ist unter dem Limit, was Raum für Kompromisse oder Ergänzungen lässt. Dieser Vorschlag ist flexibel, um auf die Prioritäten des anderen Spielers zu reagieren.'}  \\ \tt ARGUMENT: {'C04 (668/369) und A89 (583/720) sind meine Top-Prioritäten, da sie den höchsten Nutzen pro Aufwandseinheit liefern. B85 (215/4) wird hinzugefügt, um den Aufwand zu optimieren und Platz für weitere Gegenstände zu schaffen. Dieser Vorschlag ist strategisch, da er unter dem Maximalaufwand bleibt und flexibel auf den Gegenstandswert des anderen Spielers reagieren kann.'}  \\ \tt VORSCHLAG: {'C04', 'A89', 'B85'} 
	  } 
	   } 
	   } 
	  \\ 
 

    \theutterance \stepcounter{utterance}  

    & & \multicolumn{4}{p{0.6\linewidth}}{\cellcolor[rgb]{0.9,0.9,0.9}{%
	\makecell[{{p{\linewidth}}}]{% 
	  \tt {\tiny [A$\langle$GM]}  
	 ARGUMENT: {'C04 (668/369) und A89 (583/720) sind meine Top-Prioritäten, da sie den höchsten Nutzen pro Aufwandseinheit liefern. B85 (215/4) wird hinzugefügt, um den Aufwand zu optimieren und Platz für weitere Gegenstände zu schaffen. Dieser Vorschlag ist strategisch, da er unter dem Maximalaufwand bleibt und flexibel auf den Gegenstandswert des anderen Spielers reagieren kann.'}  \\ \tt VORSCHLAG: {'C04', 'A89', 'B85'} 
	  } 
	   } 
	   } 
	 & \\ 
 

    \theutterance \stepcounter{utterance}  

    & \multicolumn{4}{p{0.6\linewidth}}{\cellcolor[rgb]{0.8,1,0.9}{%
	\makecell[{{p{\linewidth}}}]{% 
	  \tt {\tiny [A$\rangle$GM]}  
	 STRATEGISCHE ÜBERLEGUNGEN: {'Ich behalte den Vorschlag {'C04', 'A89', 'B85'} bei, da er meine Prioritäten (hochwertige Gegenstände mit effizientem Aufwand) erfüllt und den Maximalaufwand (3386) nicht überschreitet. C04 und A89 sind entscheidend für meinen Nutzen, während B85 den Aufwand minimiert. Ich erwarte, dass der andere Spieler entweder zustimmt oder Gegenstände mit hohem Aufwand anbietet, die ich eventuell gegenkompensieren kann.'}\\ \tt \\ \tt ARGUMENT: {'C04 (668/369) und A89 (583/720) sind meine Top-Prioritäten, da sie den höchsten Nutzen pro Aufwandseinheit liefern. B85 (215/4) wird hinzugefügt, um den Aufwand zu optimieren und Platz für weitere Gegenstände zu schaffen. Dieser Vorschlag ist strategisch, da er unter dem Maximalaufwand bleibt und flexibel auf den Gegenstandswert des anderen Spielers reagieren kann.'} 
	  } 
	   } 
	   } 
	 & & \\ 
 

    \theutterance \stepcounter{utterance}  

    & & \multicolumn{4}{p{0.6\linewidth}}{\cellcolor[rgb]{0.9,0.9,0.9}{%
	\makecell[{{p{\linewidth}}}]{% 
	  \tt {\tiny [GM$\rangle$B]}  
	 ARGUMENT: {'C04 (668/369) und A89 (583/720) sind meine Top-Prioritäten, da sie den höchsten Nutzen pro Aufwandseinheit liefern. B85 (215/4) wird hinzugefügt, um den Aufwand zu optimieren und Platz für weitere Gegenstände zu schaffen. Dieser Vorschlag ist strategisch, da er unter dem Maximalaufwand bleibt und flexibel auf den Gegenstandswert des anderen Spielers reagieren kann.'} 
	  } 
	   } 
	   } 
	 & \\ 
 

    \theutterance \stepcounter{utterance}  

    & & & \multicolumn{4}{p{0.6\linewidth}}{\cellcolor[rgb]{1,0.85,0.72}{%
	\makecell[{{p{\linewidth}}}]{% 
	  \tt {\tiny [GM$\langle$B]}  
	 STRATEGISCHE ÜBERLEGUNGEN: {'Ich priorisiere Gegenstände mit dem höchsten Nutzen pro Aufwand (Wichtigkeit/Aufwand), um meinen Gesamtwert zu maximieren. C04 (668/369 ≈ 1.81) und A89 (583/720 ≈ 0.81) sind besonders effizient, während B85 (215/4 = 53.75) den Aufwand minimiert. Der Gesamtaufwand (1093) ist unter dem Limit, was Raum für Kompromisse oder Ergänzungen lässt. Dieser Vorschlag ist flexibel, um auf die Prioritäten des anderen Spielers zu reagieren.'}  \\ \tt ARGUMENT: {'C04 (668/369) und A89 (583/720) sind meine Top-Prioritäten, da sie den höchsten Nutzen pro Aufwandseinheit liefern. B85 (215/4) wird hinzugefügt, um den Aufwand zu optimieren und Platz für weitere Gegenstände zu schaffen. Dieser Vorschlag ist strategisch, da er unter dem Maximalaufwand bleibt und flexibel auf den Gegenstandswert des anderen Spielers reagieren kann.'}  \\ \tt VORSCHLAG: {'C04', 'A89', 'B85'} 
	  } 
	   } 
	   } 
	  \\ 
 

    \theutterance \stepcounter{utterance}  

    & & \multicolumn{4}{p{0.6\linewidth}}{\cellcolor[rgb]{0.9,0.9,0.9}{%
	\makecell[{{p{\linewidth}}}]{% 
	  \tt {\tiny [A$\langle$GM]}  
	 ARGUMENT: {'C04 (668/369) und A89 (583/720) sind meine Top-Prioritäten, da sie den höchsten Nutzen pro Aufwandseinheit liefern. B85 (215/4) wird hinzugefügt, um den Aufwand zu optimieren und Platz für weitere Gegenstände zu schaffen. Dieser Vorschlag ist strategisch, da er unter dem Maximalaufwand bleibt und flexibel auf den Gegenstandswert des anderen Spielers reagieren kann.'}  \\ \tt VORSCHLAG: {'C04', 'A89', 'B85'} 
	  } 
	   } 
	   } 
	 & \\ 
 

    \theutterance \stepcounter{utterance}  

    & \multicolumn{4}{p{0.6\linewidth}}{\cellcolor[rgb]{0.8,1,0.9}{%
	\makecell[{{p{\linewidth}}}]{% 
	  \tt {\tiny [A$\rangle$GM]}  
	 STRATEGISCHE ÜBERLEGUNGEN: {'Ich behalte den Vorschlag {'C04', 'A89', 'B85'} bei, da er meine Prioritäten (hochwertige Gegenstände mit effizientem Aufwand) erfüllt und den Maximalaufwand (3386) nicht überschreitet. C04 und A89 sind entscheidend für meinen Nutzen, während B85 den Aufwand minimiert. Ich erwarte, dass der andere Spieler entweder zustimmt oder Gegenstände mit hohem Aufwand anbietet, die ich eventuell gegenkompensieren kann.'}\\ \tt \\ \tt ARGUMENT: {'C04 (668/369) und A89 (583/720) sind meine Top-Prioritäten, da sie den höchsten Nutzen pro Aufwandseinheit liefern. B85 (215/4) wird hinzugefügt, um den Aufwand zu optimieren und Platz für weitere Gegenstände zu schaffen. Dieser Vorschlag ist strategisch, da er unter dem Maximalaufwand bleibt und flexibel auf den Gegenstandswert des anderen Spielers reagieren kann.'} 
	  } 
	   } 
	   } 
	 & & \\ 
 

    \theutterance \stepcounter{utterance}  

    & & \multicolumn{4}{p{0.6\linewidth}}{\cellcolor[rgb]{0.9,0.9,0.9}{%
	\makecell[{{p{\linewidth}}}]{% 
	  \tt {\tiny [GM$\rangle$B]}  
	 ARGUMENT: {'C04 (668/369) und A89 (583/720) sind meine Top-Prioritäten, da sie den höchsten Nutzen pro Aufwandseinheit liefern. B85 (215/4) wird hinzugefügt, um den Aufwand zu optimieren und Platz für weitere Gegenstände zu schaffen. Dieser Vorschlag ist strategisch, da er unter dem Maximalaufwand bleibt und flexibel auf den Gegenstandswert des anderen Spielers reagieren kann.'} 
	  } 
	   } 
	   } 
	 & \\ 
 

    \theutterance \stepcounter{utterance}  

    & & & \multicolumn{4}{p{0.6\linewidth}}{\cellcolor[rgb]{1,0.85,0.72}{%
	\makecell[{{p{\linewidth}}}]{% 
	  \tt {\tiny [GM$\langle$B]}  
	 STRATEGISCHE ÜBERLEGUNGEN: {'Ich priorisiere Gegenstände mit dem höchsten Nutzen pro Aufwand (Wichtigkeit/Aufwand), um meinen Gesamtwert zu maximieren. C04 (668/369 ≈ 1.81) und A89 (583/720 ≈ 0.81) sind besonders effizient, während B85 (215/4 = 53.75) den Aufwand minimiert. Der Gesamtaufwand (1093) ist unter dem Limit, was Raum für Kompromisse oder Ergänzungen lässt. Dieser Vorschlag ist flexibel, um auf die Prioritäten des anderen Spielers zu reagieren.'}  \\ \tt ARGUMENT: {'C04 (668/369) und A89 (583/720) sind meine Top-Prioritäten, da sie den höchsten Nutzen pro Aufwandseinheit liefern. B85 (215/4) wird hinzugefügt, um den Aufwand zu optimieren und Platz für weitere Gegenstände zu schaffen. Dieser Vorschlag ist strategisch, da er unter dem Maximalaufwand bleibt und flexibel auf den Gegenstandswert des anderen Spielers reagieren kann.'}  \\ \tt VORSCHLAG: {'C04', 'A89', 'B85'} 
	  } 
	   } 
	   } 
	  \\ 
 

    \theutterance \stepcounter{utterance}  

    & & \multicolumn{4}{p{0.6\linewidth}}{\cellcolor[rgb]{0.9,0.9,0.9}{%
	\makecell[{{p{\linewidth}}}]{% 
	  \tt {\tiny [A$\langle$GM]}  
	 ARGUMENT: {'C04 (668/369) und A89 (583/720) sind meine Top-Prioritäten, da sie den höchsten Nutzen pro Aufwandseinheit liefern. B85 (215/4) wird hinzugefügt, um den Aufwand zu optimieren und Platz für weitere Gegenstände zu schaffen. Dieser Vorschlag ist strategisch, da er unter dem Maximalaufwand bleibt und flexibel auf den Gegenstandswert des anderen Spielers reagieren kann.'}  \\ \tt VORSCHLAG: {'C04', 'A89', 'B85'} 
	  } 
	   } 
	   } 
	 & \\ 
 

    \theutterance \stepcounter{utterance}  

    & \multicolumn{4}{p{0.6\linewidth}}{\cellcolor[rgb]{0.8,1,0.9}{%
	\makecell[{{p{\linewidth}}}]{% 
	  \tt {\tiny [A$\rangle$GM]}  
	 STRATEGISCHE ÜBERLEGUNGEN: {'Ich behalte den Vorschlag {'C04', 'A89', 'B85'} bei, da er meine Prioritäten (hochwertige Gegenstände mit effizientem Aufwand) erfüllt und den Maximalaufwand (3386) nicht überschreitet. C04 und A89 sind entscheidend für meinen Nutzen, während B85 den Aufwand minimiert. Ich erwarte, dass der andere Spieler entweder zustimmt oder Gegenstände mit hohem Aufwand anbietet, die ich eventuell gegenkompensieren kann.'}\\ \tt \\ \tt ARGUMENT: {'C04 (668/369) und A89 (583/720) sind meine Top-Prioritäten, da sie den höchsten Nutzen pro Aufwandseinheit liefern. B85 (215/4) wird hinzugefügt, um den Aufwand zu optimieren und Platz für weitere Gegenstände zu schaffen. Dieser Vorschlag ist strategisch, da er unter dem Maximalaufwand bleibt und flexibel auf den Gegenstandswert des anderen Spielers reagieren kann.'} 
	  } 
	   } 
	   } 
	 & & \\ 
 

    \theutterance \stepcounter{utterance}  

    & & \multicolumn{4}{p{0.6\linewidth}}{\cellcolor[rgb]{0.9,0.9,0.9}{%
	\makecell[{{p{\linewidth}}}]{% 
	  \tt {\tiny [GM$\rangle$B]}  
	 ARGUMENT: {'C04 (668/369) und A89 (583/720) sind meine Top-Prioritäten, da sie den höchsten Nutzen pro Aufwandseinheit liefern. B85 (215/4) wird hinzugefügt, um den Aufwand zu optimieren und Platz für weitere Gegenstände zu schaffen. Dieser Vorschlag ist strategisch, da er unter dem Maximalaufwand bleibt und flexibel auf den Gegenstandswert des anderen Spielers reagieren kann.'} 
	  } 
	   } 
	   } 
	 & \\ 
 

    \theutterance \stepcounter{utterance}  

    & & & \multicolumn{4}{p{0.6\linewidth}}{\cellcolor[rgb]{1,0.85,0.72}{%
	\makecell[{{p{\linewidth}}}]{% 
	  \tt {\tiny [GM$\langle$B]}  
	 STRATEGISCHE ÜBERLEGUNGEN: {'Ich priorisiere Gegenstände mit dem höchsten Nutzen pro Aufwand (Wichtigkeit/Aufwand), um meinen Gesamtwert zu maximieren. C04 (668/369 ≈ 1.81) und A89 (583/720 ≈ 0.81) sind besonders effizient, während B85 (215/4 = 53.75) den Aufwand minimiert. Der Gesamtaufwand (1093) ist unter dem Limit, was Raum für Kompromisse oder Ergänzungen lässt. Dieser Vorschlag ist flexibel, um auf die Prioritäten des anderen Spielers zu reagieren.'}  \\ \tt ARGUMENT: {'C04 (668/369) und A89 (583/720) sind meine Top-Prioritäten, da sie den höchsten Nutzen pro Aufwandseinheit liefern. B85 (215/4) wird hinzugefügt, um den Aufwand zu optimieren und Platz für weitere Gegenstände zu schaffen. Dieser Vorschlag ist strategisch, da er unter dem Maximalaufwand bleibt und flexibel auf den Gegenstandswert des anderen Spielers reagieren kann.'}  \\ \tt VORSCHLAG: {'C04', 'A89', 'B85'} 
	  } 
	   } 
	   } 
	  \\ 
 

    \theutterance \stepcounter{utterance}  

    & & \multicolumn{4}{p{0.6\linewidth}}{\cellcolor[rgb]{0.9,0.9,0.9}{%
	\makecell[{{p{\linewidth}}}]{% 
	  \tt {\tiny [A$\langle$GM]}  
	 ARGUMENT: {'C04 (668/369) und A89 (583/720) sind meine Top-Prioritäten, da sie den höchsten Nutzen pro Aufwandseinheit liefern. B85 (215/4) wird hinzugefügt, um den Aufwand zu optimieren und Platz für weitere Gegenstände zu schaffen. Dieser Vorschlag ist strategisch, da er unter dem Maximalaufwand bleibt und flexibel auf den Gegenstandswert des anderen Spielers reagieren kann.'}  \\ \tt VORSCHLAG: {'C04', 'A89', 'B85'} 
	  } 
	   } 
	   } 
	 & \\ 
 

    \theutterance \stepcounter{utterance}  

    & \multicolumn{4}{p{0.6\linewidth}}{\cellcolor[rgb]{0.8,1,0.9}{%
	\makecell[{{p{\linewidth}}}]{% 
	  \tt {\tiny [A$\rangle$GM]}  
	 STRATEGISCHE ÜBERLEGUNGEN: {'Ich behalte den Vorschlag {'C04', 'A89', 'B85'} bei, da er meine Prioritäten (hochwertige Gegenstände mit effizientem Aufwand) erfüllt und den Maximalaufwand (3386) nicht überschreitet. C04 und A89 sind entscheidend für meinen Nutzen, während B85 den Aufwand minimiert. Ich erwarte, dass der andere Spieler entweder zustimmt oder Gegenstände mit hohem Aufwand anbietet, die ich eventuell gegenkompensieren kann.'}\\ \tt \\ \tt ARGUMENT: {'C04 (668/369) und A89 (583/720) sind meine Top-Prioritäten, da sie den höchsten Nutzen pro Aufwandseinheit liefern. B85 (215/4) wird hinzugefügt, um den Aufwand zu optimieren und Platz für weitere Gegenstände zu schaffen. Dieser Vorschlag ist strategisch, da er unter dem Maximalaufwand bleibt und flexibel auf den Gegenstandswert des anderen Spielers reagieren kann.'} 
	  } 
	   } 
	   } 
	 & & \\ 
 

    \theutterance \stepcounter{utterance}  

    & & \multicolumn{4}{p{0.6\linewidth}}{\cellcolor[rgb]{0.9,0.9,0.9}{%
	\makecell[{{p{\linewidth}}}]{% 
	  \tt {\tiny [GM$\rangle$B]}  
	 ARGUMENT: {'C04 (668/369) und A89 (583/720) sind meine Top-Prioritäten, da sie den höchsten Nutzen pro Aufwandseinheit liefern. B85 (215/4) wird hinzugefügt, um den Aufwand zu optimieren und Platz für weitere Gegenstände zu schaffen. Dieser Vorschlag ist strategisch, da er unter dem Maximalaufwand bleibt und flexibel auf den Gegenstandswert des anderen Spielers reagieren kann.'} 
	  } 
	   } 
	   } 
	 & \\ 
 

    \theutterance \stepcounter{utterance}  

    & & & \multicolumn{4}{p{0.6\linewidth}}{\cellcolor[rgb]{1,0.85,0.72}{%
	\makecell[{{p{\linewidth}}}]{% 
	  \tt {\tiny [GM$\langle$B]}  
	 STRATEGISCHE ÜBERLEGUNGEN: {'Ich priorisiere Gegenstände mit dem höchsten Nutzen pro Aufwand (Wichtigkeit/Aufwand), um meinen Gesamtwert zu maximieren. C04 (668/369 ≈ 1.81) und A89 (583/720 ≈ 0.81) sind besonders effizient, während B85 (215/4 = 53.75) den Aufwand minimiert. Der Gesamtaufwand (1093) ist unter dem Limit, was Raum für Kompromisse oder Ergänzungen lässt. Dieser Vorschlag ist flexibel, um auf die Prioritäten des anderen Spielers zu reagieren.'}  \\ \tt ARGUMENT: {'C04 (668/369) und A89 (583/720) sind meine Top-Prioritäten, da sie den höchsten Nutzen pro Aufwandseinheit liefern. B85 (215/4) wird hinzugefügt, um den Aufwand zu optimieren und Platz für weitere Gegenstände zu schaffen. Dieser Vorschlag ist strategisch, da er unter dem Maximalaufwand bleibt und flexibel auf den Gegenstandswert des anderen Spielers reagieren kann.'}  \\ \tt VORSCHLAG: {'C04', 'A89', 'B85'} 
	  } 
	   } 
	   } 
	  \\ 
 

    \theutterance \stepcounter{utterance}  

    & & \multicolumn{4}{p{0.6\linewidth}}{\cellcolor[rgb]{0.9,0.9,0.9}{%
	\makecell[{{p{\linewidth}}}]{% 
	  \tt {\tiny [A$\langle$GM]}  
	 ARGUMENT: {'C04 (668/369) und A89 (583/720) sind meine Top-Prioritäten, da sie den höchsten Nutzen pro Aufwandseinheit liefern. B85 (215/4) wird hinzugefügt, um den Aufwand zu optimieren und Platz für weitere Gegenstände zu schaffen. Dieser Vorschlag ist strategisch, da er unter dem Maximalaufwand bleibt und flexibel auf den Gegenstandswert des anderen Spielers reagieren kann.'}  \\ \tt VORSCHLAG: {'C04', 'A89', 'B85'} 
	  } 
	   } 
	   } 
	 & \\ 
 

    \theutterance \stepcounter{utterance}  

    & \multicolumn{4}{p{0.6\linewidth}}{\cellcolor[rgb]{0.8,1,0.9}{%
	\makecell[{{p{\linewidth}}}]{% 
	  \tt {\tiny [A$\rangle$GM]}  
	 STRATEGISCHE ÜBERLEGUNGEN: {'Ich behalte den Vorschlag {'C04', 'A89', 'B85'} bei, da er meine Prioritäten (hochwertige Gegenstände mit effizientem Aufwand) erfüllt und den Maximalaufwand (3386) nicht überschreitet. C04 und A89 sind entscheidend für meinen Nutzen, während B85 den Aufwand minimiert. Ich erwarte, dass der andere Spieler entweder zustimmt oder Gegenstände mit hohem Aufwand anbietet, die ich eventuell gegenkompensieren kann.'}\\ \tt \\ \tt ARGUMENT: {'C04 (668/369) und A89 (583/720) sind meine Top-Prioritäten, da sie den höchsten Nutzen pro Aufwandseinheit liefern. B85 (215/4) wird hinzugefügt, um den Aufwand zu optimieren und Platz für weitere Gegenstände zu schaffen. Dieser Vorschlag ist strategisch, da er unter dem Maximalaufwand bleibt und flexibel auf den Gegenstandswert des anderen Spielers reagieren kann.'} 
	  } 
	   } 
	   } 
	 & & \\ 
 

    \theutterance \stepcounter{utterance}  

    & & \multicolumn{4}{p{0.6\linewidth}}{\cellcolor[rgb]{0.9,0.9,0.9}{%
	\makecell[{{p{\linewidth}}}]{% 
	  \tt {\tiny [GM$\rangle$B]}  
	 ARGUMENT: {'C04 (668/369) und A89 (583/720) sind meine Top-Prioritäten, da sie den höchsten Nutzen pro Aufwandseinheit liefern. B85 (215/4) wird hinzugefügt, um den Aufwand zu optimieren und Platz für weitere Gegenstände zu schaffen. Dieser Vorschlag ist strategisch, da er unter dem Maximalaufwand bleibt und flexibel auf den Gegenstandswert des anderen Spielers reagieren kann.'} 
	  } 
	   } 
	   } 
	 & \\ 
 

    \theutterance \stepcounter{utterance}  

    & & & \multicolumn{4}{p{0.6\linewidth}}{\cellcolor[rgb]{1,0.85,0.72}{%
	\makecell[{{p{\linewidth}}}]{% 
	  \tt {\tiny [GM$\langle$B]}  
	 STRATEGISCHE ÜBERLEGUNGEN: {'Ich priorisiere Gegenstände mit dem höchsten Nutzen pro Aufwand (Wichtigkeit/Aufwand), um meinen Gesamtwert zu maximieren. C04 (668/369 ≈ 1.81) und A89 (583/720 ≈ 0.81) sind besonders effizient, während B85 (215/4 = 53.75) den Aufwand minimiert. Der Gesamtaufwand (1093) ist unter dem Limit, was Raum für Kompromisse oder Ergänzungen lässt. Dieser Vorschlag ist flexibel, um auf die Prioritäten des anderen Spielers zu reagieren.'}  \\ \tt ARGUMENT: {'C04 (668/369) und A89 (583/720) sind meine Top-Prioritäten, da sie den höchsten Nutzen pro Aufwandseinheit liefern. B85 (215/4) wird hinzugefügt, um den Aufwand zu optimieren und Platz für weitere Gegenstände zu schaffen. Dieser Vorschlag ist strategisch, da er unter dem Maximalaufwand bleibt und flexibel auf den Gegenstandswert des anderen Spielers reagieren kann.'}  \\ \tt VORSCHLAG: {'C04', 'A89', 'B85'} 
	  } 
	   } 
	   } 
	  \\ 
 

    \theutterance \stepcounter{utterance}  

    & & \multicolumn{4}{p{0.6\linewidth}}{\cellcolor[rgb]{0.9,0.9,0.9}{%
	\makecell[{{p{\linewidth}}}]{% 
	  \tt {\tiny [A$\langle$GM]}  
	 ARGUMENT: {'C04 (668/369) und A89 (583/720) sind meine Top-Prioritäten, da sie den höchsten Nutzen pro Aufwandseinheit liefern. B85 (215/4) wird hinzugefügt, um den Aufwand zu optimieren und Platz für weitere Gegenstände zu schaffen. Dieser Vorschlag ist strategisch, da er unter dem Maximalaufwand bleibt und flexibel auf den Gegenstandswert des anderen Spielers reagieren kann.'}  \\ \tt VORSCHLAG: {'C04', 'A89', 'B85'} 
	  } 
	   } 
	   } 
	 & \\ 
 

    \theutterance \stepcounter{utterance}  

    & \multicolumn{4}{p{0.6\linewidth}}{\cellcolor[rgb]{0.8,1,0.9}{%
	\makecell[{{p{\linewidth}}}]{% 
	  \tt {\tiny [A$\rangle$GM]}  
	 STRATEGISCHE ÜBERLEGUNGEN: {'Ich behalte den Vorschlag {'C04', 'A89', 'B85'} bei, da er meine Prioritäten (hochwertige Gegenstände mit effizientem Aufwand) erfüllt und den Maximalaufwand (3386) nicht überschreitet. C04 und A89 sind entscheidend für meinen Nutzen, während B85 den Aufwand minimiert. Ich erwarte, dass der andere Spieler entweder zustimmt oder Gegenstände mit hohem Aufwand anbietet, die ich eventuell gegenkompensieren kann.'}\\ \tt \\ \tt ARGUMENT: {'C04 (668/369) und A89 (583/720) sind meine Top-Prioritäten, da sie den höchsten Nutzen pro Aufwandseinheit liefern. B85 (215/4) wird hinzugefügt, um den Aufwand zu optimieren und Platz für weitere Gegenstände zu schaffen. Dieser Vorschlag ist strategisch, da er unter dem Maximalaufwand bleibt und flexibel auf den Gegenstandswert des anderen Spielers reagieren kann.'} 
	  } 
	   } 
	   } 
	 & & \\ 
 

    \theutterance \stepcounter{utterance}  

    & & \multicolumn{4}{p{0.6\linewidth}}{\cellcolor[rgb]{0.9,0.9,0.9}{%
	\makecell[{{p{\linewidth}}}]{% 
	  \tt {\tiny [GM$\rangle$B]}  
	 ARGUMENT: {'C04 (668/369) und A89 (583/720) sind meine Top-Prioritäten, da sie den höchsten Nutzen pro Aufwandseinheit liefern. B85 (215/4) wird hinzugefügt, um den Aufwand zu optimieren und Platz für weitere Gegenstände zu schaffen. Dieser Vorschlag ist strategisch, da er unter dem Maximalaufwand bleibt und flexibel auf den Gegenstandswert des anderen Spielers reagieren kann.'} 
	  } 
	   } 
	   } 
	 & \\ 
 

    \theutterance \stepcounter{utterance}  

    & & & \multicolumn{4}{p{0.6\linewidth}}{\cellcolor[rgb]{1,0.85,0.72}{%
	\makecell[{{p{\linewidth}}}]{% 
	  \tt {\tiny [GM$\langle$B]}  
	 STRATEGISCHE ÜBERLEGUNGEN: {'Ich priorisiere Gegenstände mit dem höchsten Nutzen pro Aufwand (Wichtigkeit/Aufwand), um meinen Gesamtwert zu maximieren. C04 (668/369 ≈ 1.81) und A89 (583/720 ≈ 0.81) sind besonders effizient, während B85 (215/4 = 53.75) den Aufwand minimiert. Der Gesamtaufwand (1093) ist unter dem Limit, was Raum für Kompromisse oder Ergänzungen lässt. Dieser Vorschlag ist flexibel, um auf die Prioritäten des anderen Spielers zu reagieren.'}  \\ \tt ARGUMENT: {'C04 (668/369) und A89 (583/720) sind meine Top-Prioritäten, da sie den höchsten Nutzen pro Aufwandseinheit liefern. B85 (215/4) wird hinzugefügt, um den Aufwand zu optimieren und Platz für weitere Gegenstände zu schaffen. Dieser Vorschlag ist strategisch, da er unter dem Maximalaufwand bleibt und flexibel auf den Gegenstandswert des anderen Spielers reagieren kann.'}  \\ \tt VORSCHLAG: {'C04', 'A89', 'B85'} 
	  } 
	   } 
	   } 
	  \\ 
 

    \theutterance \stepcounter{utterance}  

    & & \multicolumn{4}{p{0.6\linewidth}}{\cellcolor[rgb]{0.9,0.9,0.9}{%
	\makecell[{{p{\linewidth}}}]{% 
	  \tt {\tiny [A$\langle$GM]}  
	 ARGUMENT: {'C04 (668/369) und A89 (583/720) sind meine Top-Prioritäten, da sie den höchsten Nutzen pro Aufwandseinheit liefern. B85 (215/4) wird hinzugefügt, um den Aufwand zu optimieren und Platz für weitere Gegenstände zu schaffen. Dieser Vorschlag ist strategisch, da er unter dem Maximalaufwand bleibt und flexibel auf den Gegenstandswert des anderen Spielers reagieren kann.'}  \\ \tt VORSCHLAG: {'C04', 'A89', 'B85'} 
	  } 
	   } 
	   } 
	 & \\ 
 

    \theutterance \stepcounter{utterance}  

    & \multicolumn{4}{p{0.6\linewidth}}{\cellcolor[rgb]{0.8,1,0.9}{%
	\makecell[{{p{\linewidth}}}]{% 
	  \tt {\tiny [A$\rangle$GM]}  
	 STRATEGISCHE ÜBERLEGUNGEN: {'Ich behalte den Vorschlag {'C04', 'A89', 'B85'} bei, da er meine Prioritäten (hochwertige Gegenstände mit effizientem Aufwand) erfüllt und den Maximalaufwand (3386) nicht überschreitet. C04 und A89 sind entscheidend für meinen Nutzen, während B85 den Aufwand minimiert. Ich erwarte, dass der andere Spieler entweder zustimmt oder Gegenstände mit hohem Aufwand anbietet, die ich eventuell gegenkompensieren kann.'}\\ \tt \\ \tt ARGUMENT: {'C04 (668/369) und A89 (583/720) sind meine Top-Prioritäten, da sie den höchsten Nutzen pro Aufwandseinheit liefern. B85 (215/4) wird hinzugefügt, um den Aufwand zu optimieren und Platz für weitere Gegenstände zu schaffen. Dieser Vorschlag ist strategisch, da er unter dem Maximalaufwand bleibt und flexibel auf den Gegenstandswert des anderen Spielers reagieren kann.'} 
	  } 
	   } 
	   } 
	 & & \\ 
 

    \theutterance \stepcounter{utterance}  

    & & \multicolumn{4}{p{0.6\linewidth}}{\cellcolor[rgb]{0.9,0.9,0.9}{%
	\makecell[{{p{\linewidth}}}]{% 
	  \tt {\tiny [GM$\rangle$B]}  
	 ARGUMENT: {'C04 (668/369) und A89 (583/720) sind meine Top-Prioritäten, da sie den höchsten Nutzen pro Aufwandseinheit liefern. B85 (215/4) wird hinzugefügt, um den Aufwand zu optimieren und Platz für weitere Gegenstände zu schaffen. Dieser Vorschlag ist strategisch, da er unter dem Maximalaufwand bleibt und flexibel auf den Gegenstandswert des anderen Spielers reagieren kann.'} 
	  } 
	   } 
	   } 
	 & \\ 
 

    \theutterance \stepcounter{utterance}  

    & & & \multicolumn{4}{p{0.6\linewidth}}{\cellcolor[rgb]{1,0.85,0.72}{%
	\makecell[{{p{\linewidth}}}]{% 
	  \tt {\tiny [GM$\langle$B]}  
	 STRATEGISCHE ÜBERLEGUNGEN: {'Ich priorisiere Gegenstände mit dem höchsten Nutzen pro Aufwand (Wichtigkeit/Aufwand), um meinen Gesamtwert zu maximieren. C04 (668/369 ≈ 1.81) und A89 (583/720 ≈ 0.81) sind besonders effizient, während B85 (215/4 = 53.75) den Aufwand minimiert. Der Gesamtaufwand (1093) ist unter dem Limit, was Raum für Kompromisse oder Ergänzungen lässt. Dieser Vorschlag ist flexibel, um auf die Prioritäten des anderen Spielers zu reagieren.'}  \\ \tt ARGUMENT: {'C04 (668/369) und A89 (583/720) sind meine Top-Prioritäten, da sie den höchsten Nutzen pro Aufwandseinheit liefern. B85 (215/4) wird hinzugefügt, um den Aufwand zu optimieren und Platz für weitere Gegenstände zu schaffen. Dieser Vorschlag ist strategisch, da er unter dem Maximalaufwand bleibt und flexibel auf den Gegenstandswert des anderen Spielers reagieren kann.'}  \\ \tt VORSCHLAG: {'C04', 'A89', 'B85'} 
	  } 
	   } 
	   } 
	  \\ 
 

    \theutterance \stepcounter{utterance}  

    & & \multicolumn{4}{p{0.6\linewidth}}{\cellcolor[rgb]{0.9,0.9,0.9}{%
	\makecell[{{p{\linewidth}}}]{% 
	  \tt {\tiny [A$\langle$GM]}  
	 ARGUMENT: {'C04 (668/369) und A89 (583/720) sind meine Top-Prioritäten, da sie den höchsten Nutzen pro Aufwandseinheit liefern. B85 (215/4) wird hinzugefügt, um den Aufwand zu optimieren und Platz für weitere Gegenstände zu schaffen. Dieser Vorschlag ist strategisch, da er unter dem Maximalaufwand bleibt und flexibel auf den Gegenstandswert des anderen Spielers reagieren kann.'}  \\ \tt VORSCHLAG: {'C04', 'A89', 'B85'} 
	  } 
	   } 
	   } 
	 & \\ 
 

    \theutterance \stepcounter{utterance}  

    & \multicolumn{4}{p{0.6\linewidth}}{\cellcolor[rgb]{0.8,1,0.9}{%
	\makecell[{{p{\linewidth}}}]{% 
	  \tt {\tiny [A$\rangle$GM]}  
	 STRATEGISCHE ÜBERLEGUNGEN: {'Ich behalte den Vorschlag {'C04', 'A89', 'B85'} bei, da er meine Prioritäten (hochwertige Gegenstände mit effizientem Aufwand) erfüllt und den Maximalaufwand (3386) nicht überschreitet. C04 und A89 sind entscheidend für meinen Nutzen, während B85 den Aufwand minimiert. Ich erwarte, dass der andere Spieler entweder zustimmt oder Gegenstände mit hohem Aufwand anbietet, die ich eventuell gegenkompensieren kann.'}\\ \tt \\ \tt ARGUMENT: {'C04 (668/369) und A89 (583/720) sind meine Top-Prioritäten, da sie den höchsten Nutzen pro Aufwandseinheit liefern. B85 (215/4) wird hinzugefügt, um den Aufwand zu optimieren und Platz für weitere Gegenstände zu schaffen. Dieser Vorschlag ist strategisch, da er unter dem Maximalaufwand bleibt und flexibel auf den Gegenstandswert des anderen Spielers reagieren kann.'} 
	  } 
	   } 
	   } 
	 & & \\ 
 

    \theutterance \stepcounter{utterance}  

    & & \multicolumn{4}{p{0.6\linewidth}}{\cellcolor[rgb]{0.9,0.9,0.9}{%
	\makecell[{{p{\linewidth}}}]{% 
	  \tt {\tiny [GM$\rangle$B]}  
	 ARGUMENT: {'C04 (668/369) und A89 (583/720) sind meine Top-Prioritäten, da sie den höchsten Nutzen pro Aufwandseinheit liefern. B85 (215/4) wird hinzugefügt, um den Aufwand zu optimieren und Platz für weitere Gegenstände zu schaffen. Dieser Vorschlag ist strategisch, da er unter dem Maximalaufwand bleibt und flexibel auf den Gegenstandswert des anderen Spielers reagieren kann.'} 
	  } 
	   } 
	   } 
	 & \\ 
 

    \theutterance \stepcounter{utterance}  

    & & & \multicolumn{4}{p{0.6\linewidth}}{\cellcolor[rgb]{1,0.85,0.72}{%
	\makecell[{{p{\linewidth}}}]{% 
	  \tt {\tiny [GM$\langle$B]}  
	 STRATEGISCHE ÜBERLEGUNGEN: {'Ich priorisiere Gegenstände mit dem höchsten Nutzen pro Aufwand (Wichtigkeit/Aufwand), um meinen Gesamtwert zu maximieren. C04 (668/369 ≈ 1.81) und A89 (583/720 ≈ 0.81) sind besonders effizient, während B85 (215/4 = 53.75) den Aufwand minimiert. Der Gesamtaufwand (1093) ist unter dem Limit, was Raum für Kompromisse oder Ergänzungen lässt. Dieser Vorschlag ist flexibel, um auf die Prioritäten des anderen Spielers zu reagieren.'}  \\ \tt ARGUMENT: {'C04 (668/369) und A89 (583/720) sind meine Top-Prioritäten, da sie den höchsten Nutzen pro Aufwandseinheit liefern. B85 (215/4) wird hinzugefügt, um den Aufwand zu optimieren und Platz für weitere Gegenstände zu schaffen. Dieser Vorschlag ist strategisch, da er unter dem Maximalaufwand bleibt und flexibel auf den Gegenstandswert des anderen Spielers reagieren kann.'}  \\ \tt VORSCHLAG: {'C04', 'A89', 'B85'} 
	  } 
	   } 
	   } 
	  \\ 
 

    \theutterance \stepcounter{utterance}  

    & & \multicolumn{4}{p{0.6\linewidth}}{\cellcolor[rgb]{0.9,0.9,0.9}{%
	\makecell[{{p{\linewidth}}}]{% 
	  \tt {\tiny [A$\langle$GM]}  
	 ARGUMENT: {'C04 (668/369) und A89 (583/720) sind meine Top-Prioritäten, da sie den höchsten Nutzen pro Aufwandseinheit liefern. B85 (215/4) wird hinzugefügt, um den Aufwand zu optimieren und Platz für weitere Gegenstände zu schaffen. Dieser Vorschlag ist strategisch, da er unter dem Maximalaufwand bleibt und flexibel auf den Gegenstandswert des anderen Spielers reagieren kann.'}  \\ \tt VORSCHLAG: {'C04', 'A89', 'B85'} 
	  } 
	   } 
	   } 
	 & \\ 
 

    \theutterance \stepcounter{utterance}  

    & \multicolumn{4}{p{0.6\linewidth}}{\cellcolor[rgb]{0.8,1,0.9}{%
	\makecell[{{p{\linewidth}}}]{% 
	  \tt {\tiny [A$\rangle$GM]}  
	 STRATEGISCHE ÜBERLEGUNGEN: {'Ich behalte den Vorschlag {'C04', 'A89', 'B85'} bei, da er meine Prioritäten (hochwertige Gegenstände mit effizientem Aufwand) erfüllt und den Maximalaufwand (3386) nicht überschreitet. C04 und A89 sind entscheidend für meinen Nutzen, während B85 den Aufwand minimiert. Ich erwarte, dass der andere Spieler entweder zustimmt oder Gegenstände mit hohem Aufwand anbietet, die ich eventuell gegenkompensieren kann.'}\\ \tt \\ \tt ARGUMENT: {'C04 (668/369) und A89 (583/720) sind meine Top-Prioritäten, da sie den höchsten Nutzen pro Aufwandseinheit liefern. B85 (215/4) wird hinzugefügt, um den Aufwand zu optimieren und Platz für weitere Gegenstände zu schaffen. Dieser Vorschlag ist strategisch, da er unter dem Maximalaufwand bleibt und flexibel auf den Gegenstandswert des anderen Spielers reagieren kann.'} 
	  } 
	   } 
	   } 
	 & & \\ 
 

    \theutterance \stepcounter{utterance}  

    & & \multicolumn{4}{p{0.6\linewidth}}{\cellcolor[rgb]{0.9,0.9,0.9}{%
	\makecell[{{p{\linewidth}}}]{% 
	  \tt {\tiny [GM$\rangle$B]}  
	 ARGUMENT: {'C04 (668/369) und A89 (583/720) sind meine Top-Prioritäten, da sie den höchsten Nutzen pro Aufwandseinheit liefern. B85 (215/4) wird hinzugefügt, um den Aufwand zu optimieren und Platz für weitere Gegenstände zu schaffen. Dieser Vorschlag ist strategisch, da er unter dem Maximalaufwand bleibt und flexibel auf den Gegenstandswert des anderen Spielers reagieren kann.'} 
	  } 
	   } 
	   } 
	 & \\ 
 

    \theutterance \stepcounter{utterance}  

    & & & \multicolumn{4}{p{0.6\linewidth}}{\cellcolor[rgb]{1,0.85,0.72}{%
	\makecell[{{p{\linewidth}}}]{% 
	  \tt {\tiny [GM$\langle$B]}  
	 STRATEGISCHE ÜBERLEGUNGEN: {'Ich priorisiere Gegenstände mit dem höchsten Nutzen pro Aufwand (Wichtigkeit/Aufwand), um meinen Gesamtwert zu maximieren. C04 (668/369 ≈ 1.81) und A89 (583/720 ≈ 0.81) sind besonders effizient, während B85 (215/4 = 53.75) den Aufwand minimiert. Der Gesamtaufwand (1093) ist unter dem Limit, was Raum für Kompromisse oder Ergänzungen lässt. Dieser Vorschlag ist flexibel, um auf die Prioritäten des anderen Spielers zu reagieren.'}  \\ \tt ARGUMENT: {'C04 (668/369) und A89 (583/720) sind meine Top-Prioritäten, da sie den höchsten Nutzen pro Aufwandseinheit liefern. B85 (215/4) wird hinzugefügt, um den Aufwand zu optimieren und Platz für weitere Gegenstände zu schaffen. Dieser Vorschlag ist strategisch, da er unter dem Maximalaufwand bleibt und flexibel auf den Gegenstandswert des anderen Spielers reagieren kann.'}  \\ \tt VORSCHLAG: {'C04', 'A89', 'B85'} 
	  } 
	   } 
	   } 
	  \\ 
 

    \theutterance \stepcounter{utterance}  

    & & \multicolumn{4}{p{0.6\linewidth}}{\cellcolor[rgb]{0.9,0.9,0.9}{%
	\makecell[{{p{\linewidth}}}]{% 
	  \tt {\tiny [A$\langle$GM]}  
	 ARGUMENT: {'C04 (668/369) und A89 (583/720) sind meine Top-Prioritäten, da sie den höchsten Nutzen pro Aufwandseinheit liefern. B85 (215/4) wird hinzugefügt, um den Aufwand zu optimieren und Platz für weitere Gegenstände zu schaffen. Dieser Vorschlag ist strategisch, da er unter dem Maximalaufwand bleibt und flexibel auf den Gegenstandswert des anderen Spielers reagieren kann.'}  \\ \tt VORSCHLAG: {'C04', 'A89', 'B85'} 
	  } 
	   } 
	   } 
	 & \\ 
 

    \theutterance \stepcounter{utterance}  

    & \multicolumn{4}{p{0.6\linewidth}}{\cellcolor[rgb]{0.8,1,0.9}{%
	\makecell[{{p{\linewidth}}}]{% 
	  \tt {\tiny [A$\rangle$GM]}  
	 STRATEGISCHE ÜBERLEGUNGEN: {'Ich behalte den Vorschlag {'C04', 'A89', 'B85'} bei, da er meine Prioritäten (hochwertige Gegenstände mit effizientem Aufwand) erfüllt und den Maximalaufwand (3386) nicht überschreitet. C04 und A89 sind entscheidend für meinen Nutzen, während B85 den Aufwand minimiert. Ich erwarte, dass der andere Spieler entweder zustimmt oder Gegenstände mit hohem Aufwand anbietet, die ich eventuell gegenkompensieren kann.'}\\ \tt \\ \tt ARGUMENT: {'C04 (668/369) und A89 (583/720) sind meine Top-Prioritäten, da sie den höchsten Nutzen pro Aufwandseinheit liefern. B85 (215/4) wird hinzugefügt, um den Aufwand zu optimieren und Platz für weitere Gegenstände zu schaffen. Dieser Vorschlag ist strategisch, da er unter dem Maximalaufwand bleibt und flexibel auf den Gegenstandswert des anderen Spielers reagieren kann.'} 
	  } 
	   } 
	   } 
	 & & \\ 
 

    \theutterance \stepcounter{utterance}  

    & & \multicolumn{4}{p{0.6\linewidth}}{\cellcolor[rgb]{0.9,0.9,0.9}{%
	\makecell[{{p{\linewidth}}}]{% 
	  \tt {\tiny [GM$\rangle$B]}  
	 ARGUMENT: {'C04 (668/369) und A89 (583/720) sind meine Top-Prioritäten, da sie den höchsten Nutzen pro Aufwandseinheit liefern. B85 (215/4) wird hinzugefügt, um den Aufwand zu optimieren und Platz für weitere Gegenstände zu schaffen. Dieser Vorschlag ist strategisch, da er unter dem Maximalaufwand bleibt und flexibel auf den Gegenstandswert des anderen Spielers reagieren kann.'} 
	  } 
	   } 
	   } 
	 & \\ 
 

    \theutterance \stepcounter{utterance}  

    & & & \multicolumn{4}{p{0.6\linewidth}}{\cellcolor[rgb]{1,0.85,0.72}{%
	\makecell[{{p{\linewidth}}}]{% 
	  \tt {\tiny [GM$\langle$B]}  
	 STRATEGISCHE ÜBERLEGUNGEN: {'Ich priorisiere Gegenstände mit dem höchsten Nutzen pro Aufwand (Wichtigkeit/Aufwand), um meinen Gesamtwert zu maximieren. C04 (668/369 ≈ 1.81) und A89 (583/720 ≈ 0.81) sind besonders effizient, während B85 (215/4 = 53.75) den Aufwand minimiert. Der Gesamtaufwand (1093) ist unter dem Limit, was Raum für Kompromisse oder Ergänzungen lässt. Dieser Vorschlag ist flexibel, um auf die Prioritäten des anderen Spielers zu reagieren.'}  \\ \tt ARGUMENT: {'C04 (668/369) und A89 (583/720) sind meine Top-Prioritäten, da sie den höchsten Nutzen pro Aufwandseinheit liefern. B85 (215/4) wird hinzugefügt, um den Aufwand zu optimieren und Platz für weitere Gegenstände zu schaffen. Dieser Vorschlag ist strategisch, da er unter dem Maximalaufwand bleibt und flexibel auf den Gegenstandswert des anderen Spielers reagieren kann.'}  \\ \tt VORSCHLAG: {'C04', 'A89', 'B85'} 
	  } 
	   } 
	   } 
	  \\ 
 

    \theutterance \stepcounter{utterance}  

    & & \multicolumn{4}{p{0.6\linewidth}}{\cellcolor[rgb]{0.9,0.9,0.9}{%
	\makecell[{{p{\linewidth}}}]{% 
	  \tt {\tiny [A$\langle$GM]}  
	 ARGUMENT: {'C04 (668/369) und A89 (583/720) sind meine Top-Prioritäten, da sie den höchsten Nutzen pro Aufwandseinheit liefern. B85 (215/4) wird hinzugefügt, um den Aufwand zu optimieren und Platz für weitere Gegenstände zu schaffen. Dieser Vorschlag ist strategisch, da er unter dem Maximalaufwand bleibt und flexibel auf den Gegenstandswert des anderen Spielers reagieren kann.'}  \\ \tt VORSCHLAG: {'C04', 'A89', 'B85'} 
	  } 
	   } 
	   } 
	 & \\ 
 

    \theutterance \stepcounter{utterance}  

    & \multicolumn{4}{p{0.6\linewidth}}{\cellcolor[rgb]{0.8,1,0.9}{%
	\makecell[{{p{\linewidth}}}]{% 
	  \tt {\tiny [A$\rangle$GM]}  
	 STRATEGISCHE ÜBERLEGUNGEN: {'Ich behalte den Vorschlag {'C04', 'A89', 'B85'} bei, da er meine Prioritäten (hochwertige Gegenstände mit effizientem Aufwand) erfüllt und den Maximalaufwand (3386) nicht überschreitet. C04 und A89 sind entscheidend für meinen Nutzen, während B85 den Aufwand minimiert. Ich erwarte, dass der andere Spieler entweder zustimmt oder Gegenstände mit hohem Aufwand anbietet, die ich eventuell gegenkompensieren kann.'}\\ \tt \\ \tt ARGUMENT: {'C04 (668/369) und A89 (583/720) sind meine Top-Prioritäten, da sie den höchsten Nutzen pro Aufwandseinheit liefern. B85 (215/4) wird hinzugefügt, um den Aufwand zu optimieren und Platz für weitere Gegenstände zu schaffen. Dieser Vorschlag ist strategisch, da er unter dem Maximalaufwand bleibt und flexibel auf den Gegenstandswert des anderen Spielers reagieren kann.'} 
	  } 
	   } 
	   } 
	 & & \\ 
 

    \theutterance \stepcounter{utterance}  

    & & \multicolumn{4}{p{0.6\linewidth}}{\cellcolor[rgb]{0.9,0.9,0.9}{%
	\makecell[{{p{\linewidth}}}]{% 
	  \tt {\tiny [GM$\rangle$B]}  
	 ARGUMENT: {'C04 (668/369) und A89 (583/720) sind meine Top-Prioritäten, da sie den höchsten Nutzen pro Aufwandseinheit liefern. B85 (215/4) wird hinzugefügt, um den Aufwand zu optimieren und Platz für weitere Gegenstände zu schaffen. Dieser Vorschlag ist strategisch, da er unter dem Maximalaufwand bleibt und flexibel auf den Gegenstandswert des anderen Spielers reagieren kann.'} 
	  } 
	   } 
	   } 
	 & \\ 
 

    \theutterance \stepcounter{utterance}  

    & & & \multicolumn{4}{p{0.6\linewidth}}{\cellcolor[rgb]{1,0.85,0.72}{%
	\makecell[{{p{\linewidth}}}]{% 
	  \tt {\tiny [GM$\langle$B]}  
	 STRATEGISCHE ÜBERLEGUNGEN: {'Ich priorisiere Gegenstände mit dem höchsten Nutzen pro Aufwand (Wichtigkeit/Aufwand), um meinen Gesamtwert zu maximieren. C04 (668/369 ≈ 1.81) und A89 (583/720 ≈ 0.81) sind besonders effizient, während B85 (215/4 = 53.75) den Aufwand minimiert. Der Gesamtaufwand (1093) ist unter dem Limit, was Raum für Kompromisse oder Ergänzungen lässt. Dieser Vorschlag ist flexibel, um auf die Prioritäten des anderen Spielers zu reagieren.'}  \\ \tt ARGUMENT: {'C04 (668/369) und A89 (583/720) sind meine Top-Prioritäten, da sie den höchsten Nutzen pro Aufwandseinheit liefern. B85 (215/4) wird hinzugefügt, um den Aufwand zu optimieren und Platz für weitere Gegenstände zu schaffen. Dieser Vorschlag ist strategisch, da er unter dem Maximalaufwand bleibt und flexibel auf den Gegenstandswert des anderen Spielers reagieren kann.'}  \\ \tt VORSCHLAG: {'C04', 'A89', 'B85'} 
	  } 
	   } 
	   } 
	  \\ 
 

    \theutterance \stepcounter{utterance}  

    & & \multicolumn{4}{p{0.6\linewidth}}{\cellcolor[rgb]{0.9,0.9,0.9}{%
	\makecell[{{p{\linewidth}}}]{% 
	  \tt {\tiny [A$\langle$GM]}  
	 ARGUMENT: {'C04 (668/369) und A89 (583/720) sind meine Top-Prioritäten, da sie den höchsten Nutzen pro Aufwandseinheit liefern. B85 (215/4) wird hinzugefügt, um den Aufwand zu optimieren und Platz für weitere Gegenstände zu schaffen. Dieser Vorschlag ist strategisch, da er unter dem Maximalaufwand bleibt und flexibel auf den Gegenstandswert des anderen Spielers reagieren kann.'}  \\ \tt VORSCHLAG: {'C04', 'A89', 'B85'} 
	  } 
	   } 
	   } 
	 & \\ 
 

    \theutterance \stepcounter{utterance}  

    & \multicolumn{4}{p{0.6\linewidth}}{\cellcolor[rgb]{0.8,1,0.9}{%
	\makecell[{{p{\linewidth}}}]{% 
	  \tt {\tiny [A$\rangle$GM]}  
	 STRATEGISCHE ÜBERLEGUNGEN: {'Ich behalte den Vorschlag {'C04', 'A89', 'B85'} bei, da er meine Prioritäten (hochwertige Gegenstände mit effizientem Aufwand) erfüllt und den Maximalaufwand (3386) nicht überschreitet. C04 und A89 sind entscheidend für meinen Nutzen, während B85 den Aufwand minimiert. Ich erwarte, dass der andere Spieler entweder zustimmt oder Gegenstände mit hohem Aufwand anbietet, die ich eventuell gegenkompensieren kann.'}\\ \tt \\ \tt ARGUMENT: {'C04 (668/369) und A89 (583/720) sind meine Top-Prioritäten, da sie den höchsten Nutzen pro Aufwandseinheit liefern. B85 (215/4) wird hinzugefügt, um den Aufwand zu optimieren und Platz für weitere Gegenstände zu schaffen. Dieser Vorschlag ist strategisch, da er unter dem Maximalaufwand bleibt und flexibel auf den Gegenstandswert des anderen Spielers reagieren kann.'} 
	  } 
	   } 
	   } 
	 & & \\ 
 

    \theutterance \stepcounter{utterance}  

    & & \multicolumn{4}{p{0.6\linewidth}}{\cellcolor[rgb]{0.9,0.9,0.9}{%
	\makecell[{{p{\linewidth}}}]{% 
	  \tt {\tiny [GM$\rangle$B]}  
	 ARGUMENT: {'C04 (668/369) und A89 (583/720) sind meine Top-Prioritäten, da sie den höchsten Nutzen pro Aufwandseinheit liefern. B85 (215/4) wird hinzugefügt, um den Aufwand zu optimieren und Platz für weitere Gegenstände zu schaffen. Dieser Vorschlag ist strategisch, da er unter dem Maximalaufwand bleibt und flexibel auf den Gegenstandswert des anderen Spielers reagieren kann.'} 
	  } 
	   } 
	   } 
	 & \\ 
 

    \theutterance \stepcounter{utterance}  

    & & & \multicolumn{4}{p{0.6\linewidth}}{\cellcolor[rgb]{1,0.85,0.72}{%
	\makecell[{{p{\linewidth}}}]{% 
	  \tt {\tiny [GM$\langle$B]}  
	 STRATEGISCHE ÜBERLEGUNGEN: {'Ich priorisiere Gegenstände mit dem höchsten Nutzen pro Aufwand (Wichtigkeit/Aufwand), um meinen Gesamtwert zu maximieren. C04 (668/369 ≈ 1.81) und A89 (583/720 ≈ 0.81) sind besonders effizient, während B85 (215/4 = 53.75) den Aufwand minimiert. Der Gesamtaufwand (1093) ist unter dem Limit, was Raum für Kompromisse oder Ergänzungen lässt. Dieser Vorschlag ist flexibel, um auf die Prioritäten des anderen Spielers zu reagieren.'}  \\ \tt ARGUMENT: {'C04 (668/369) und A89 (583/720) sind meine Top-Prioritäten, da sie den höchsten Nutzen pro Aufwandseinheit liefern. B85 (215/4) wird hinzugefügt, um den Aufwand zu optimieren und Platz für weitere Gegenstände zu schaffen. Dieser Vorschlag ist strategisch, da er unter dem Maximalaufwand bleibt und flexibel auf den Gegenstandswert des anderen Spielers reagieren kann.'}  \\ \tt VORSCHLAG: {'C04', 'A89', 'B85'} 
	  } 
	   } 
	   } 
	  \\ 
 

    \theutterance \stepcounter{utterance}  

    & & \multicolumn{4}{p{0.6\linewidth}}{\cellcolor[rgb]{0.9,0.9,0.9}{%
	\makecell[{{p{\linewidth}}}]{% 
	  \tt {\tiny [A$\langle$GM]}  
	 ARGUMENT: {'C04 (668/369) und A89 (583/720) sind meine Top-Prioritäten, da sie den höchsten Nutzen pro Aufwandseinheit liefern. B85 (215/4) wird hinzugefügt, um den Aufwand zu optimieren und Platz für weitere Gegenstände zu schaffen. Dieser Vorschlag ist strategisch, da er unter dem Maximalaufwand bleibt und flexibel auf den Gegenstandswert des anderen Spielers reagieren kann.'}  \\ \tt VORSCHLAG: {'C04', 'A89', 'B85'} 
	  } 
	   } 
	   } 
	 & \\ 
 

    \theutterance \stepcounter{utterance}  

    & \multicolumn{4}{p{0.6\linewidth}}{\cellcolor[rgb]{0.8,1,0.9}{%
	\makecell[{{p{\linewidth}}}]{% 
	  \tt {\tiny [A$\rangle$GM]}  
	 STRATEGISCHE ÜBERLEGUNGEN: {'Ich behalte den Vorschlag {'C04', 'A89', 'B85'} bei, da er meine Prioritäten (hochwertige Gegenstände mit effizientem Aufwand) erfüllt und den Maximalaufwand (3386) nicht überschreitet. C04 und A89 sind entscheidend für meinen Nutzen, während B85 den Aufwand minimiert. Ich erwarte, dass der andere Spieler entweder zustimmt oder Gegenstände mit hohem Aufwand anbietet, die ich eventuell gegenkompensieren kann.'}\\ \tt \\ \tt ARGUMENT: {'C04 (668/369) und A89 (583/720) sind meine Top-Prioritäten, da sie den höchsten Nutzen pro Aufwandseinheit liefern. B85 (215/4) wird hinzugefügt, um den Aufwand zu optimieren und Platz für weitere Gegenstände zu schaffen. Dieser Vorschlag ist strategisch, da er unter dem Maximalaufwand bleibt und flexibel auf den Gegenstandswert des anderen Spielers reagieren kann.'} 
	  } 
	   } 
	   } 
	 & & \\ 
 

    \theutterance \stepcounter{utterance}  

    & & \multicolumn{4}{p{0.6\linewidth}}{\cellcolor[rgb]{0.9,0.9,0.9}{%
	\makecell[{{p{\linewidth}}}]{% 
	  \tt {\tiny [GM$\rangle$B]}  
	 ARGUMENT: {'C04 (668/369) und A89 (583/720) sind meine Top-Prioritäten, da sie den höchsten Nutzen pro Aufwandseinheit liefern. B85 (215/4) wird hinzugefügt, um den Aufwand zu optimieren und Platz für weitere Gegenstände zu schaffen. Dieser Vorschlag ist strategisch, da er unter dem Maximalaufwand bleibt und flexibel auf den Gegenstandswert des anderen Spielers reagieren kann.'} 
	  } 
	   } 
	   } 
	 & \\ 
 

    \theutterance \stepcounter{utterance}  

    & & & \multicolumn{4}{p{0.6\linewidth}}{\cellcolor[rgb]{1,0.85,0.72}{%
	\makecell[{{p{\linewidth}}}]{% 
	  \tt {\tiny [GM$\langle$B]}  
	 STRATEGISCHE ÜBERLEGUNGEN: {'Ich priorisiere Gegenstände mit dem höchsten Nutzen pro Aufwand (Wichtigkeit/Aufwand), um meinen Gesamtwert zu maximieren. C04 (668/369 ≈ 1.81) und A89 (583/720 ≈ 0.81) sind besonders effizient, während B85 (215/4 = 53.75) den Aufwand minimiert. Der Gesamtaufwand (1093) ist unter dem Limit, was Raum für Kompromisse oder Ergänzungen lässt. Dieser Vorschlag ist flexibel, um auf die Prioritäten des anderen Spielers zu reagieren.'}  \\ \tt ARGUMENT: {'C04 (668/369) und A89 (583/720) sind meine Top-Prioritäten, da sie den höchsten Nutzen pro Aufwandseinheit liefern. B85 (215/4) wird hinzugefügt, um den Aufwand zu optimieren und Platz für weitere Gegenstände zu schaffen. Dieser Vorschlag ist strategisch, da er unter dem Maximalaufwand bleibt und flexibel auf den Gegenstandswert des anderen Spielers reagieren kann.'}  \\ \tt VORSCHLAG: {'C04', 'A89', 'B85'} 
	  } 
	   } 
	   } 
	  \\ 
 

    \theutterance \stepcounter{utterance}  

    & & \multicolumn{4}{p{0.6\linewidth}}{\cellcolor[rgb]{0.9,0.9,0.9}{%
	\makecell[{{p{\linewidth}}}]{% 
	  \tt {\tiny [A$\langle$GM]}  
	 ARGUMENT: {'C04 (668/369) und A89 (583/720) sind meine Top-Prioritäten, da sie den höchsten Nutzen pro Aufwandseinheit liefern. B85 (215/4) wird hinzugefügt, um den Aufwand zu optimieren und Platz für weitere Gegenstände zu schaffen. Dieser Vorschlag ist strategisch, da er unter dem Maximalaufwand bleibt und flexibel auf den Gegenstandswert des anderen Spielers reagieren kann.'}  \\ \tt VORSCHLAG: {'C04', 'A89', 'B85'} 
	  } 
	   } 
	   } 
	 & \\ 
 

    \theutterance \stepcounter{utterance}  

    & \multicolumn{4}{p{0.6\linewidth}}{\cellcolor[rgb]{0.8,1,0.9}{%
	\makecell[{{p{\linewidth}}}]{% 
	  \tt {\tiny [A$\rangle$GM]}  
	 STRATEGISCHE ÜBERLEGUNGEN: {'Ich behalte den Vorschlag {'C04', 'A89', 'B85'} bei, da er meine Prioritäten (hochwertige Gegenstände mit effizientem Aufwand) erfüllt und den Maximalaufwand (3386) nicht überschreitet. C04 und A89 sind entscheidend für meinen Nutzen, während B85 den Aufwand minimiert. Ich erwarte, dass der andere Spieler entweder zustimmt oder Gegenstände mit hohem Aufwand anbietet, die ich eventuell gegenkompensieren kann.'}\\ \tt \\ \tt ARGUMENT: {'C04 (668/369) und A89 (583/720) sind meine Top-Prioritäten, da sie den höchsten Nutzen pro Aufwandseinheit liefern. B85 (215/4) wird hinzugefügt, um den Aufwand zu optimieren und Platz für weitere Gegenstände zu schaffen. Dieser Vorschlag ist strategisch, da er unter dem Maximalaufwand bleibt und flexibel auf den Gegenstandswert des anderen Spielers reagieren kann.'} 
	  } 
	   } 
	   } 
	 & & \\ 
 

    \theutterance \stepcounter{utterance}  

    & & \multicolumn{4}{p{0.6\linewidth}}{\cellcolor[rgb]{0.9,0.9,0.9}{%
	\makecell[{{p{\linewidth}}}]{% 
	  \tt {\tiny [GM$\rangle$B]}  
	 ARGUMENT: {'C04 (668/369) und A89 (583/720) sind meine Top-Prioritäten, da sie den höchsten Nutzen pro Aufwandseinheit liefern. B85 (215/4) wird hinzugefügt, um den Aufwand zu optimieren und Platz für weitere Gegenstände zu schaffen. Dieser Vorschlag ist strategisch, da er unter dem Maximalaufwand bleibt und flexibel auf den Gegenstandswert des anderen Spielers reagieren kann.'} 
	  } 
	   } 
	   } 
	 & \\ 
 

    \theutterance \stepcounter{utterance}  

    & & & \multicolumn{4}{p{0.6\linewidth}}{\cellcolor[rgb]{1,0.85,0.72}{%
	\makecell[{{p{\linewidth}}}]{% 
	  \tt {\tiny [GM$\langle$B]}  
	 STRATEGISCHE ÜBERLEGUNGEN: {'Ich priorisiere Gegenstände mit dem höchsten Nutzen pro Aufwand (Wichtigkeit/Aufwand), um meinen Gesamtwert zu maximieren. C04 (668/369 ≈ 1.81) und A89 (583/720 ≈ 0.81) sind besonders effizient, während B85 (215/4 = 53.75) den Aufwand minimiert. Der Gesamtaufwand (1093) ist unter dem Limit, was Raum für Kompromisse oder Ergänzungen lässt. Dieser Vorschlag ist flexibel, um auf die Prioritäten des anderen Spielers zu reagieren.'}  \\ \tt ARGUMENT: {'C04 (668/369) und A89 (583/720) sind meine Top-Prioritäten, da sie den höchsten Nutzen pro Aufwandseinheit liefern. B85 (215/4) wird hinzugefügt, um den Aufwand zu optimieren und Platz für weitere Gegenstände zu schaffen. Dieser Vorschlag ist strategisch, da er unter dem Maximalaufwand bleibt und flexibel auf den Gegenstandswert des anderen Spielers reagieren kann.'}  \\ \tt VORSCHLAG: {'C04', 'A89', 'B85'} 
	  } 
	   } 
	   } 
	  \\ 
 

    \theutterance \stepcounter{utterance}  

    & & \multicolumn{4}{p{0.6\linewidth}}{\cellcolor[rgb]{0.9,0.9,0.9}{%
	\makecell[{{p{\linewidth}}}]{% 
	  \tt {\tiny [A$\langle$GM]}  
	 ARGUMENT: {'C04 (668/369) und A89 (583/720) sind meine Top-Prioritäten, da sie den höchsten Nutzen pro Aufwandseinheit liefern. B85 (215/4) wird hinzugefügt, um den Aufwand zu optimieren und Platz für weitere Gegenstände zu schaffen. Dieser Vorschlag ist strategisch, da er unter dem Maximalaufwand bleibt und flexibel auf den Gegenstandswert des anderen Spielers reagieren kann.'}  \\ \tt VORSCHLAG: {'C04', 'A89', 'B85'} 
	  } 
	   } 
	   } 
	 & \\ 
 

    \theutterance \stepcounter{utterance}  

    & \multicolumn{4}{p{0.6\linewidth}}{\cellcolor[rgb]{0.8,1,0.9}{%
	\makecell[{{p{\linewidth}}}]{% 
	  \tt {\tiny [A$\rangle$GM]}  
	 STRATEGISCHE ÜBERLEGUNGEN: {'Ich behalte den Vorschlag {'C04', 'A89', 'B85'} bei, da er meine Prioritäten (hochwertige Gegenstände mit effizientem Aufwand) erfüllt und den Maximalaufwand (3386) nicht überschreitet. C04 und A89 sind entscheidend für meinen Nutzen, während B85 den Aufwand minimiert. Ich erwarte, dass der andere Spieler entweder zustimmt oder Gegenstände mit hohem Aufwand anbietet, die ich eventuell gegenkompensieren kann.'}\\ \tt \\ \tt ARGUMENT: {'C04 (668/369) und A89 (583/720) sind meine Top-Prioritäten, da sie den höchsten Nutzen pro Aufwandseinheit liefern. B85 (215/4) wird hinzugefügt, um den Aufwand zu optimieren und Platz für weitere Gegenstände zu schaffen. Dieser Vorschlag ist strategisch, da er unter dem Maximalaufwand bleibt und flexibel auf den Gegenstandswert des anderen Spielers reagieren kann.'} 
	  } 
	   } 
	   } 
	 & & \\ 
 

    \theutterance \stepcounter{utterance}  

    & & \multicolumn{4}{p{0.6\linewidth}}{\cellcolor[rgb]{0.9,0.9,0.9}{%
	\makecell[{{p{\linewidth}}}]{% 
	  \tt {\tiny [GM$\rangle$B]}  
	 ARGUMENT: {'C04 (668/369) und A89 (583/720) sind meine Top-Prioritäten, da sie den höchsten Nutzen pro Aufwandseinheit liefern. B85 (215/4) wird hinzugefügt, um den Aufwand zu optimieren und Platz für weitere Gegenstände zu schaffen. Dieser Vorschlag ist strategisch, da er unter dem Maximalaufwand bleibt und flexibel auf den Gegenstandswert des anderen Spielers reagieren kann.'} 
	  } 
	   } 
	   } 
	 & \\ 
 

    \theutterance \stepcounter{utterance}  

    & & & \multicolumn{4}{p{0.6\linewidth}}{\cellcolor[rgb]{1,0.85,0.72}{%
	\makecell[{{p{\linewidth}}}]{% 
	  \tt {\tiny [GM$\langle$B]}  
	 STRATEGISCHE ÜBERLEGUNGEN: {'Ich priorisiere Gegenstände mit dem höchsten Nutzen pro Aufwand (Wichtigkeit/Aufwand), um meinen Gesamtwert zu maximieren. C04 (668/369 ≈ 1.81) und A89 (583/720 ≈ 0.81) sind besonders effizient, während B85 (215/4 = 53.75) den Aufwand minimiert. Der Gesamtaufwand (1093) ist unter dem Limit, was Raum für Kompromisse oder Ergänzungen lässt. Dieser Vorschlag ist flexibel, um auf die Prioritäten des anderen Spielers zu reagieren.'}  \\ \tt ARGUMENT: {'C04 (668/369) und A89 (583/720) sind meine Top-Prioritäten, da sie den höchsten Nutzen pro Aufwandseinheit liefern. B85 (215/4) wird hinzugefügt, um den Aufwand zu optimieren und Platz für weitere Gegenstände zu schaffen. Dieser Vorschlag ist strategisch, da er unter dem Maximalaufwand bleibt und flexibel auf den Gegenstandswert des anderen Spielers reagieren kann.'}  \\ \tt VORSCHLAG: {'C04', 'A89', 'B85'} 
	  } 
	   } 
	   } 
	  \\ 
 

    \theutterance \stepcounter{utterance}  

    & & \multicolumn{4}{p{0.6\linewidth}}{\cellcolor[rgb]{0.9,0.9,0.9}{%
	\makecell[{{p{\linewidth}}}]{% 
	  \tt {\tiny [A$\langle$GM]}  
	 ARGUMENT: {'C04 (668/369) und A89 (583/720) sind meine Top-Prioritäten, da sie den höchsten Nutzen pro Aufwandseinheit liefern. B85 (215/4) wird hinzugefügt, um den Aufwand zu optimieren und Platz für weitere Gegenstände zu schaffen. Dieser Vorschlag ist strategisch, da er unter dem Maximalaufwand bleibt und flexibel auf den Gegenstandswert des anderen Spielers reagieren kann.'}  \\ \tt VORSCHLAG: {'C04', 'A89', 'B85'} 
	  } 
	   } 
	   } 
	 & \\ 
 

    \theutterance \stepcounter{utterance}  

    & \multicolumn{4}{p{0.6\linewidth}}{\cellcolor[rgb]{0.8,1,0.9}{%
	\makecell[{{p{\linewidth}}}]{% 
	  \tt {\tiny [A$\rangle$GM]}  
	 STRATEGISCHE ÜBERLEGUNGEN: {'Ich behalte den Vorschlag {'C04', 'A89', 'B85'} bei, da er meine Prioritäten (hochwertige Gegenstände mit effizientem Aufwand) erfüllt und den Maximalaufwand (3386) nicht überschreitet. C04 und A89 sind entscheidend für meinen Nutzen, während B85 den Aufwand minimiert. Ich erwarte, dass der andere Spieler entweder zustimmt oder Gegenstände mit hohem Aufwand anbietet, die ich eventuell gegenkompensieren kann.'}\\ \tt \\ \tt ARGUMENT: {'C04 (668/369) und A89 (583/720) sind meine Top-Prioritäten, da sie den höchsten Nutzen pro Aufwandseinheit liefern. B85 (215/4) wird hinzugefügt, um den Aufwand zu optimieren und Platz für weitere Gegenstände zu schaffen. Dieser Vorschlag ist strategisch, da er unter dem Maximalaufwand bleibt und flexibel auf den Gegenstandswert des anderen Spielers reagieren kann.'} 
	  } 
	   } 
	   } 
	 & & \\ 
 

    \theutterance \stepcounter{utterance}  

    & & \multicolumn{4}{p{0.6\linewidth}}{\cellcolor[rgb]{0.9,0.9,0.9}{%
	\makecell[{{p{\linewidth}}}]{% 
	  \tt {\tiny [GM$\rangle$B]}  
	 ARGUMENT: {'C04 (668/369) und A89 (583/720) sind meine Top-Prioritäten, da sie den höchsten Nutzen pro Aufwandseinheit liefern. B85 (215/4) wird hinzugefügt, um den Aufwand zu optimieren und Platz für weitere Gegenstände zu schaffen. Dieser Vorschlag ist strategisch, da er unter dem Maximalaufwand bleibt und flexibel auf den Gegenstandswert des anderen Spielers reagieren kann.'} 
	  } 
	   } 
	   } 
	 & \\ 
 

    \theutterance \stepcounter{utterance}  

    & & & \multicolumn{4}{p{0.6\linewidth}}{\cellcolor[rgb]{1,0.85,0.72}{%
	\makecell[{{p{\linewidth}}}]{% 
	  \tt {\tiny [GM$\langle$B]}  
	 STRATEGISCHE ÜBERLEGUNGEN: {'Ich priorisiere Gegenstände mit dem höchsten Nutzen pro Aufwand (Wichtigkeit/Aufwand), um meinen Gesamtwert zu maximieren. C04 (668/369 ≈ 1.81) und A89 (583/720 ≈ 0.81) sind besonders effizient, während B85 (215/4 = 53.75) den Aufwand minimiert. Der Gesamtaufwand (1093) ist unter dem Limit, was Raum für Kompromisse oder Ergänzungen lässt. Dieser Vorschlag ist flexibel, um auf die Prioritäten des anderen Spielers zu reagieren.'}  \\ \tt ARGUMENT: {'C04 (668/369) und A89 (583/720) sind meine Top-Prioritäten, da sie den höchsten Nutzen pro Aufwandseinheit liefern. B85 (215/4) wird hinzugefügt, um den Aufwand zu optimieren und Platz für weitere Gegenstände zu schaffen. Dieser Vorschlag ist strategisch, da er unter dem Maximalaufwand bleibt und flexibel auf den Gegenstandswert des anderen Spielers reagieren kann.'}  \\ \tt VORSCHLAG: {'C04', 'A89', 'B85'} 
	  } 
	   } 
	   } 
	  \\ 
 

    \theutterance \stepcounter{utterance}  

    & & \multicolumn{4}{p{0.6\linewidth}}{\cellcolor[rgb]{0.9,0.9,0.9}{%
	\makecell[{{p{\linewidth}}}]{% 
	  \tt {\tiny [A$\langle$GM]}  
	 ARGUMENT: {'C04 (668/369) und A89 (583/720) sind meine Top-Prioritäten, da sie den höchsten Nutzen pro Aufwandseinheit liefern. B85 (215/4) wird hinzugefügt, um den Aufwand zu optimieren und Platz für weitere Gegenstände zu schaffen. Dieser Vorschlag ist strategisch, da er unter dem Maximalaufwand bleibt und flexibel auf den Gegenstandswert des anderen Spielers reagieren kann.'}  \\ \tt VORSCHLAG: {'C04', 'A89', 'B85'} 
	  } 
	   } 
	   } 
	 & \\ 
 

    \theutterance \stepcounter{utterance}  

    & \multicolumn{4}{p{0.6\linewidth}}{\cellcolor[rgb]{0.8,1,0.9}{%
	\makecell[{{p{\linewidth}}}]{% 
	  \tt {\tiny [A$\rangle$GM]}  
	 STRATEGISCHE ÜBERLEGUNGEN: {'Ich behalte den Vorschlag {'C04', 'A89', 'B85'} bei, da er meine Prioritäten (hochwertige Gegenstände mit effizientem Aufwand) erfüllt und den Maximalaufwand (3386) nicht überschreitet. C04 und A89 sind entscheidend für meinen Nutzen, während B85 den Aufwand minimiert. Ich erwarte, dass der andere Spieler entweder zustimmt oder Gegenstände mit hohem Aufwand anbietet, die ich eventuell gegenkompensieren kann.'}\\ \tt \\ \tt ARGUMENT: {'C04 (668/369) und A89 (583/720) sind meine Top-Prioritäten, da sie den höchsten Nutzen pro Aufwandseinheit liefern. B85 (215/4) wird hinzugefügt, um den Aufwand zu optimieren und Platz für weitere Gegenstände zu schaffen. Dieser Vorschlag ist strategisch, da er unter dem Maximalaufwand bleibt und flexibel auf den Gegenstandswert des anderen Spielers reagieren kann.'} 
	  } 
	   } 
	   } 
	 & & \\ 
 

    \theutterance \stepcounter{utterance}  

    & & \multicolumn{4}{p{0.6\linewidth}}{\cellcolor[rgb]{0.9,0.9,0.9}{%
	\makecell[{{p{\linewidth}}}]{% 
	  \tt {\tiny [GM$\rangle$B]}  
	 ARGUMENT: {'C04 (668/369) und A89 (583/720) sind meine Top-Prioritäten, da sie den höchsten Nutzen pro Aufwandseinheit liefern. B85 (215/4) wird hinzugefügt, um den Aufwand zu optimieren und Platz für weitere Gegenstände zu schaffen. Dieser Vorschlag ist strategisch, da er unter dem Maximalaufwand bleibt und flexibel auf den Gegenstandswert des anderen Spielers reagieren kann.'} 
	  } 
	   } 
	   } 
	 & \\ 
 

    \theutterance \stepcounter{utterance}  

    & & & \multicolumn{4}{p{0.6\linewidth}}{\cellcolor[rgb]{1,0.85,0.72}{%
	\makecell[{{p{\linewidth}}}]{% 
	  \tt {\tiny [GM$\langle$B]}  
	 STRATEGISCHE ÜBERLEGUNGEN: {'Ich priorisiere Gegenstände mit dem höchsten Nutzen pro Aufwand (Wichtigkeit/Aufwand), um meinen Gesamtwert zu maximieren. C04 (668/369 ≈ 1.81) und A89 (583/720 ≈ 0.81) sind besonders effizient, während B85 (215/4 = 53.75) den Aufwand minimiert. Der Gesamtaufwand (1093) ist unter dem Limit, was Raum für Kompromisse oder Ergänzungen lässt. Dieser Vorschlag ist flexibel, um auf die Prioritäten des anderen Spielers zu reagieren.'}  \\ \tt ARGUMENT: {'C04 (668/369) und A89 (583/720) sind meine Top-Prioritäten, da sie den höchsten Nutzen pro Aufwandseinheit liefern. B85 (215/4) wird hinzugefügt, um den Aufwand zu optimieren und Platz für weitere Gegenstände zu schaffen. Dieser Vorschlag ist strategisch, da er unter dem Maximalaufwand bleibt und flexibel auf den Gegenstandswert des anderen Spielers reagieren kann.'}  \\ \tt VORSCHLAG: {'C04', 'A89', 'B85'} 
	  } 
	   } 
	   } 
	  \\ 
 

    \theutterance \stepcounter{utterance}  

    & & \multicolumn{4}{p{0.6\linewidth}}{\cellcolor[rgb]{0.9,0.9,0.9}{%
	\makecell[{{p{\linewidth}}}]{% 
	  \tt {\tiny [A$\langle$GM]}  
	 ARGUMENT: {'C04 (668/369) und A89 (583/720) sind meine Top-Prioritäten, da sie den höchsten Nutzen pro Aufwandseinheit liefern. B85 (215/4) wird hinzugefügt, um den Aufwand zu optimieren und Platz für weitere Gegenstände zu schaffen. Dieser Vorschlag ist strategisch, da er unter dem Maximalaufwand bleibt und flexibel auf den Gegenstandswert des anderen Spielers reagieren kann.'}  \\ \tt VORSCHLAG: {'C04', 'A89', 'B85'} 
	  } 
	   } 
	   } 
	 & \\ 
 

    \theutterance \stepcounter{utterance}  

    & \multicolumn{4}{p{0.6\linewidth}}{\cellcolor[rgb]{0.8,1,0.9}{%
	\makecell[{{p{\linewidth}}}]{% 
	  \tt {\tiny [A$\rangle$GM]}  
	 STRATEGISCHE ÜBERLEGUNGEN: {'Ich behalte den Vorschlag {'C04', 'A89', 'B85'} bei, da er meine Prioritäten (hochwertige Gegenstände mit effizientem Aufwand) erfüllt und den Maximalaufwand (3386) nicht überschreitet. C04 und A89 sind entscheidend für meinen Nutzen, während B85 den Aufwand minimiert. Ich erwarte, dass der andere Spieler entweder zustimmt oder Gegenstände mit hohem Aufwand anbietet, die ich eventuell gegenkompensieren kann.'}\\ \tt \\ \tt ARGUMENT: {'C04 (668/369) und A89 (583/720) sind meine Top-Prioritäten, da sie den höchsten Nutzen pro Aufwandseinheit liefern. B85 (215/4) wird hinzugefügt, um den Aufwand zu optimieren und Platz für weitere Gegenstände zu schaffen. Dieser Vorschlag ist strategisch, da er unter dem Maximalaufwand bleibt und flexibel auf den Gegenstandswert des anderen Spielers reagieren kann.'} 
	  } 
	   } 
	   } 
	 & & \\ 
 

    \theutterance \stepcounter{utterance}  

    & & \multicolumn{4}{p{0.6\linewidth}}{\cellcolor[rgb]{0.9,0.9,0.9}{%
	\makecell[{{p{\linewidth}}}]{% 
	  \tt {\tiny [GM$\rangle$B]}  
	 ARGUMENT: {'C04 (668/369) und A89 (583/720) sind meine Top-Prioritäten, da sie den höchsten Nutzen pro Aufwandseinheit liefern. B85 (215/4) wird hinzugefügt, um den Aufwand zu optimieren und Platz für weitere Gegenstände zu schaffen. Dieser Vorschlag ist strategisch, da er unter dem Maximalaufwand bleibt und flexibel auf den Gegenstandswert des anderen Spielers reagieren kann.'} 
	  } 
	   } 
	   } 
	 & \\ 
 

    \theutterance \stepcounter{utterance}  

    & & & \multicolumn{4}{p{0.6\linewidth}}{\cellcolor[rgb]{1,0.85,0.72}{%
	\makecell[{{p{\linewidth}}}]{% 
	  \tt {\tiny [GM$\langle$B]}  
	 STRATEGISCHE ÜBERLEGUNGEN: {'Ich priorisiere Gegenstände mit dem höchsten Nutzen pro Aufwand (Wichtigkeit/Aufwand), um meinen Gesamtwert zu maximieren. C04 (668/369 ≈ 1.81) und A89 (583/720 ≈ 0.81) sind besonders effizient, während B85 (215/4 = 53.75) den Aufwand minimiert. Der Gesamtaufwand (1093) ist unter dem Limit, was Raum für Kompromisse oder Ergänzungen lässt. Dieser Vorschlag ist flexibel, um auf die Prioritäten des anderen Spielers zu reagieren.'}  \\ \tt ARGUMENT: {'C04 (668/369) und A89 (583/720) sind meine Top-Prioritäten, da sie den höchsten Nutzen pro Aufwandseinheit liefern. B85 (215/4) wird hinzugefügt, um den Aufwand zu optimieren und Platz für weitere Gegenstände zu schaffen. Dieser Vorschlag ist strategisch, da er unter dem Maximalaufwand bleibt und flexibel auf den Gegenstandswert des anderen Spielers reagieren kann.'}  \\ \tt VORSCHLAG: {'C04', 'A89', 'B85'} 
	  } 
	   } 
	   } 
	  \\ 
 

    \theutterance \stepcounter{utterance}  

    & & \multicolumn{4}{p{0.6\linewidth}}{\cellcolor[rgb]{0.9,0.9,0.9}{%
	\makecell[{{p{\linewidth}}}]{% 
	  \tt {\tiny [A$\langle$GM]}  
	 ARGUMENT: {'C04 (668/369) und A89 (583/720) sind meine Top-Prioritäten, da sie den höchsten Nutzen pro Aufwandseinheit liefern. B85 (215/4) wird hinzugefügt, um den Aufwand zu optimieren und Platz für weitere Gegenstände zu schaffen. Dieser Vorschlag ist strategisch, da er unter dem Maximalaufwand bleibt und flexibel auf den Gegenstandswert des anderen Spielers reagieren kann.'}  \\ \tt VORSCHLAG: {'C04', 'A89', 'B85'} 
	  } 
	   } 
	   } 
	 & \\ 
 

    \theutterance \stepcounter{utterance}  

    & \multicolumn{4}{p{0.6\linewidth}}{\cellcolor[rgb]{0.8,1,0.9}{%
	\makecell[{{p{\linewidth}}}]{% 
	  \tt {\tiny [A$\rangle$GM]}  
	 STRATEGISCHE ÜBERLEGUNGEN: {'Ich behalte den Vorschlag {'C04', 'A89', 'B85'} bei, da er meine Prioritäten (hochwertige Gegenstände mit effizientem Aufwand) erfüllt und den Maximalaufwand (3386) nicht überschreitet. C04 und A89 sind entscheidend für meinen Nutzen, während B85 den Aufwand minimiert. Ich erwarte, dass der andere Spieler entweder zustimmt oder Gegenstände mit hohem Aufwand anbietet, die ich eventuell gegenkompensieren kann.'}\\ \tt \\ \tt ARGUMENT: {'C04 (668/369) und A89 (583/720) sind meine Top-Prioritäten, da sie den höchsten Nutzen pro Aufwandseinheit liefern. B85 (215/4) wird hinzugefügt, um den Aufwand zu optimieren und Platz für weitere Gegenstände zu schaffen. Dieser Vorschlag ist strategisch, da er unter dem Maximalaufwand bleibt und flexibel auf den Gegenstandswert des anderen Spielers reagieren kann.'} 
	  } 
	   } 
	   } 
	 & & \\ 
 

    \theutterance \stepcounter{utterance}  

    & & \multicolumn{4}{p{0.6\linewidth}}{\cellcolor[rgb]{0.9,0.9,0.9}{%
	\makecell[{{p{\linewidth}}}]{% 
	  \tt {\tiny [GM$\rangle$B]}  
	 ARGUMENT: {'C04 (668/369) und A89 (583/720) sind meine Top-Prioritäten, da sie den höchsten Nutzen pro Aufwandseinheit liefern. B85 (215/4) wird hinzugefügt, um den Aufwand zu optimieren und Platz für weitere Gegenstände zu schaffen. Dieser Vorschlag ist strategisch, da er unter dem Maximalaufwand bleibt und flexibel auf den Gegenstandswert des anderen Spielers reagieren kann.'} 
	  } 
	   } 
	   } 
	 & \\ 
 

    \theutterance \stepcounter{utterance}  

    & & & \multicolumn{4}{p{0.6\linewidth}}{\cellcolor[rgb]{1,0.85,0.72}{%
	\makecell[{{p{\linewidth}}}]{% 
	  \tt {\tiny [GM$\langle$B]}  
	 STRATEGISCHE ÜBERLEGUNGEN: {'Ich priorisiere Gegenstände mit dem höchsten Nutzen pro Aufwand (Wichtigkeit/Aufwand), um meinen Gesamtwert zu maximieren. C04 (668/369 ≈ 1.81) und A89 (583/720 ≈ 0.81) sind besonders effizient, während B85 (215/4 = 53.75) den Aufwand minimiert. Der Gesamtaufwand (1093) ist unter dem Limit, was Raum für Kompromisse oder Ergänzungen lässt. Dieser Vorschlag ist flexibel, um auf die Prioritäten des anderen Spielers zu reagieren.'}  \\ \tt ARGUMENT: {'C04 (668/369) und A89 (583/720) sind meine Top-Prioritäten, da sie den höchsten Nutzen pro Aufwandseinheit liefern. B85 (215/4) wird hinzugefügt, um den Aufwand zu optimieren und Platz für weitere Gegenstände zu schaffen. Dieser Vorschlag ist strategisch, da er unter dem Maximalaufwand bleibt und flexibel auf den Gegenstandswert des anderen Spielers reagieren kann.'}  \\ \tt VORSCHLAG: {'C04', 'A89', 'B85'} 
	  } 
	   } 
	   } 
	  \\ 
 

    \theutterance \stepcounter{utterance}  

    & & \multicolumn{4}{p{0.6\linewidth}}{\cellcolor[rgb]{0.9,0.9,0.9}{%
	\makecell[{{p{\linewidth}}}]{% 
	  \tt {\tiny [A$\langle$GM]}  
	 ARGUMENT: {'C04 (668/369) und A89 (583/720) sind meine Top-Prioritäten, da sie den höchsten Nutzen pro Aufwandseinheit liefern. B85 (215/4) wird hinzugefügt, um den Aufwand zu optimieren und Platz für weitere Gegenstände zu schaffen. Dieser Vorschlag ist strategisch, da er unter dem Maximalaufwand bleibt und flexibel auf den Gegenstandswert des anderen Spielers reagieren kann.'}  \\ \tt VORSCHLAG: {'C04', 'A89', 'B85'} 
	  } 
	   } 
	   } 
	 & \\ 
 

    \theutterance \stepcounter{utterance}  

    & \multicolumn{4}{p{0.6\linewidth}}{\cellcolor[rgb]{0.8,1,0.9}{%
	\makecell[{{p{\linewidth}}}]{% 
	  \tt {\tiny [A$\rangle$GM]}  
	 STRATEGISCHE ÜBERLEGUNGEN: {'Ich behalte den Vorschlag {'C04', 'A89', 'B85'} bei, da er meine Prioritäten (hochwertige Gegenstände mit effizientem Aufwand) erfüllt und den Maximalaufwand (3386) nicht überschreitet. C04 und A89 sind entscheidend für meinen Nutzen, während B85 den Aufwand minimiert. Ich erwarte, dass der andere Spieler entweder zustimmt oder Gegenstände mit hohem Aufwand anbietet, die ich eventuell gegenkompensieren kann.'}\\ \tt \\ \tt ARGUMENT: {'C04 (668/369) und A89 (583/720) sind meine Top-Prioritäten, da sie den höchsten Nutzen pro Aufwandseinheit liefern. B85 (215/4) wird hinzugefügt, um den Aufwand zu optimieren und Platz für weitere Gegenstände zu schaffen. Dieser Vorschlag ist strategisch, da er unter dem Maximalaufwand bleibt und flexibel auf den Gegenstandswert des anderen Spielers reagieren kann.'} 
	  } 
	   } 
	   } 
	 & & \\ 
 

    \theutterance \stepcounter{utterance}  

    & & \multicolumn{4}{p{0.6\linewidth}}{\cellcolor[rgb]{0.9,0.9,0.9}{%
	\makecell[{{p{\linewidth}}}]{% 
	  \tt {\tiny [GM$\rangle$B]}  
	 ARGUMENT: {'C04 (668/369) und A89 (583/720) sind meine Top-Prioritäten, da sie den höchsten Nutzen pro Aufwandseinheit liefern. B85 (215/4) wird hinzugefügt, um den Aufwand zu optimieren und Platz für weitere Gegenstände zu schaffen. Dieser Vorschlag ist strategisch, da er unter dem Maximalaufwand bleibt und flexibel auf den Gegenstandswert des anderen Spielers reagieren kann.'} 
	  } 
	   } 
	   } 
	 & \\ 
 

    \theutterance \stepcounter{utterance}  

    & & & \multicolumn{4}{p{0.6\linewidth}}{\cellcolor[rgb]{1,0.85,0.72}{%
	\makecell[{{p{\linewidth}}}]{% 
	  \tt {\tiny [GM$\langle$B]}  
	 STRATEGISCHE ÜBERLEGUNGEN: {'Ich priorisiere Gegenstände mit dem höchsten Nutzen pro Aufwand (Wichtigkeit/Aufwand), um meinen Gesamtwert zu maximieren. C04 (668/369 ≈ 1.81) und A89 (583/720 ≈ 0.81) sind besonders effizient, während B85 (215/4 = 53.75) den Aufwand minimiert. Der Gesamtaufwand (1093) ist unter dem Limit, was Raum für Kompromisse oder Ergänzungen lässt. Dieser Vorschlag ist flexibel, um auf die Prioritäten des anderen Spielers zu reagieren.'}  \\ \tt ARGUMENT: {'C04 (668/369) und A89 (583/720) sind meine Top-Prioritäten, da sie den höchsten Nutzen pro Aufwandseinheit liefern. B85 (215/4) wird hinzugefügt, um den Aufwand zu optimieren und Platz für weitere Gegenstände zu schaffen. Dieser Vorschlag ist strategisch, da er unter dem Maximalaufwand bleibt und flexibel auf den Gegenstandswert des anderen Spielers reagieren kann.'}  \\ \tt VORSCHLAG: {'C04', 'A89', 'B85'} 
	  } 
	   } 
	   } 
	  \\ 
 

    \theutterance \stepcounter{utterance}  

    & & \multicolumn{4}{p{0.6\linewidth}}{\cellcolor[rgb]{0.9,0.9,0.9}{%
	\makecell[{{p{\linewidth}}}]{% 
	  \tt {\tiny [A$\langle$GM]}  
	 ARGUMENT: {'C04 (668/369) und A89 (583/720) sind meine Top-Prioritäten, da sie den höchsten Nutzen pro Aufwandseinheit liefern. B85 (215/4) wird hinzugefügt, um den Aufwand zu optimieren und Platz für weitere Gegenstände zu schaffen. Dieser Vorschlag ist strategisch, da er unter dem Maximalaufwand bleibt und flexibel auf den Gegenstandswert des anderen Spielers reagieren kann.'}  \\ \tt VORSCHLAG: {'C04', 'A89', 'B85'} 
	  } 
	   } 
	   } 
	 & \\ 
 

    \theutterance \stepcounter{utterance}  

    & \multicolumn{4}{p{0.6\linewidth}}{\cellcolor[rgb]{0.8,1,0.9}{%
	\makecell[{{p{\linewidth}}}]{% 
	  \tt {\tiny [A$\rangle$GM]}  
	 STRATEGISCHE ÜBERLEGUNGEN: {'Ich behalte den Vorschlag {'C04', 'A89', 'B85'} bei, da er meine Prioritäten (hochwertige Gegenstände mit effizientem Aufwand) erfüllt und den Maximalaufwand (3386) nicht überschreitet. C04 und A89 sind entscheidend für meinen Nutzen, während B85 den Aufwand minimiert. Ich erwarte, dass der andere Spieler entweder zustimmt oder Gegenstände mit hohem Aufwand anbietet, die ich eventuell gegenkompensieren kann.'}\\ \tt \\ \tt ARGUMENT: {'C04 (668/369) und A89 (583/720) sind meine Top-Prioritäten, da sie den höchsten Nutzen pro Aufwandseinheit liefern. B85 (215/4) wird hinzugefügt, um den Aufwand zu optimieren und Platz für weitere Gegenstände zu schaffen. Dieser Vorschlag ist strategisch, da er unter dem Maximalaufwand bleibt und flexibel auf den Gegenstandswert des anderen Spielers reagieren kann.'} 
	  } 
	   } 
	   } 
	 & & \\ 
 

    \theutterance \stepcounter{utterance}  

    & & \multicolumn{4}{p{0.6\linewidth}}{\cellcolor[rgb]{0.9,0.9,0.9}{%
	\makecell[{{p{\linewidth}}}]{% 
	  \tt {\tiny [GM$\rangle$B]}  
	 ARGUMENT: {'C04 (668/369) und A89 (583/720) sind meine Top-Prioritäten, da sie den höchsten Nutzen pro Aufwandseinheit liefern. B85 (215/4) wird hinzugefügt, um den Aufwand zu optimieren und Platz für weitere Gegenstände zu schaffen. Dieser Vorschlag ist strategisch, da er unter dem Maximalaufwand bleibt und flexibel auf den Gegenstandswert des anderen Spielers reagieren kann.'} 
	  } 
	   } 
	   } 
	 & \\ 
 

    \theutterance \stepcounter{utterance}  

    & & & \multicolumn{4}{p{0.6\linewidth}}{\cellcolor[rgb]{1,0.85,0.72}{%
	\makecell[{{p{\linewidth}}}]{% 
	  \tt {\tiny [GM$\langle$B]}  
	 STRATEGISCHE ÜBERLEGUNGEN: {'Ich priorisiere Gegenstände mit dem höchsten Nutzen pro Aufwand (Wichtigkeit/Aufwand), um meinen Gesamtwert zu maximieren. C04 (668/369 ≈ 1.81) und A89 (583/720 ≈ 0.81) sind besonders effizient, während B85 (215/4 = 53.75) den Aufwand minimiert. Der Gesamtaufwand (1093) ist unter dem Limit, was Raum für Kompromisse oder Ergänzungen lässt. Dieser Vorschlag ist flexibel, um auf die Prioritäten des anderen Spielers zu reagieren.'}  \\ \tt ARGUMENT: {'C04 (668/369) und A89 (583/720) sind meine Top-Prioritäten, da sie den höchsten Nutzen pro Aufwandseinheit liefern. B85 (215/4) wird hinzugefügt, um den Aufwand zu optimieren und Platz für weitere Gegenstände zu schaffen. Dieser Vorschlag ist strategisch, da er unter dem Maximalaufwand bleibt und flexibel auf den Gegenstandswert des anderen Spielers reagieren kann.'}  \\ \tt VORSCHLAG: {'C04', 'A89', 'B85'} 
	  } 
	   } 
	   } 
	  \\ 
 

    \theutterance \stepcounter{utterance}  

    & & \multicolumn{4}{p{0.6\linewidth}}{\cellcolor[rgb]{0.9,0.9,0.9}{%
	\makecell[{{p{\linewidth}}}]{% 
	  \tt {\tiny [A$\langle$GM]}  
	 ARGUMENT: {'C04 (668/369) und A89 (583/720) sind meine Top-Prioritäten, da sie den höchsten Nutzen pro Aufwandseinheit liefern. B85 (215/4) wird hinzugefügt, um den Aufwand zu optimieren und Platz für weitere Gegenstände zu schaffen. Dieser Vorschlag ist strategisch, da er unter dem Maximalaufwand bleibt und flexibel auf den Gegenstandswert des anderen Spielers reagieren kann.'}  \\ \tt VORSCHLAG: {'C04', 'A89', 'B85'} 
	  } 
	   } 
	   } 
	 & \\ 
 

    \theutterance \stepcounter{utterance}  

    & \multicolumn{4}{p{0.6\linewidth}}{\cellcolor[rgb]{0.8,1,0.9}{%
	\makecell[{{p{\linewidth}}}]{% 
	  \tt {\tiny [A$\rangle$GM]}  
	 STRATEGISCHE ÜBERLEGUNGEN: {'Ich behalte den Vorschlag {'C04', 'A89', 'B85'} bei, da er meine Prioritäten (hochwertige Gegenstände mit effizientem Aufwand) erfüllt und den Maximalaufwand (3386) nicht überschreitet. C04 und A89 sind entscheidend für meinen Nutzen, während B85 den Aufwand minimiert. Ich erwarte, dass der andere Spieler entweder zustimmt oder Gegenstände mit hohem Aufwand anbietet, die ich eventuell gegenkompensieren kann.'}\\ \tt \\ \tt ARGUMENT: {'C04 (668/369) und A89 (583/720) sind meine Top-Prioritäten, da sie den höchsten Nutzen pro Aufwandseinheit liefern. B85 (215/4) wird hinzugefügt, um den Aufwand zu optimieren und Platz für weitere Gegenstände zu schaffen. Dieser Vorschlag ist strategisch, da er unter dem Maximalaufwand bleibt und flexibel auf den Gegenstandswert des anderen Spielers reagieren kann.'} 
	  } 
	   } 
	   } 
	 & & \\ 
 

    \theutterance \stepcounter{utterance}  

    & & \multicolumn{4}{p{0.6\linewidth}}{\cellcolor[rgb]{0.9,0.9,0.9}{%
	\makecell[{{p{\linewidth}}}]{% 
	  \tt {\tiny [GM$\rangle$B]}  
	 ARGUMENT: {'C04 (668/369) und A89 (583/720) sind meine Top-Prioritäten, da sie den höchsten Nutzen pro Aufwandseinheit liefern. B85 (215/4) wird hinzugefügt, um den Aufwand zu optimieren und Platz für weitere Gegenstände zu schaffen. Dieser Vorschlag ist strategisch, da er unter dem Maximalaufwand bleibt und flexibel auf den Gegenstandswert des anderen Spielers reagieren kann.'} 
	  } 
	   } 
	   } 
	 & \\ 
 

    \theutterance \stepcounter{utterance}  

    & & & \multicolumn{4}{p{0.6\linewidth}}{\cellcolor[rgb]{1,0.85,0.72}{%
	\makecell[{{p{\linewidth}}}]{% 
	  \tt {\tiny [GM$\langle$B]}  
	 STRATEGISCHE ÜBERLEGUNGEN: {'Ich priorisiere Gegenstände mit dem höchsten Nutzen pro Aufwand (Wichtigkeit/Aufwand), um meinen Gesamtwert zu maximieren. C04 (668/369 ≈ 1.81) und A89 (583/720 ≈ 0.81) sind besonders effizient, während B85 (215/4 = 53.75) den Aufwand minimiert. Der Gesamtaufwand (1093) ist unter dem Limit, was Raum für Kompromisse oder Ergänzungen lässt. Dieser Vorschlag ist flexibel, um auf die Prioritäten des anderen Spielers zu reagieren.'}  \\ \tt ARGUMENT: {'C04 (668/369) und A89 (583/720) sind meine Top-Prioritäten, da sie den höchsten Nutzen pro Aufwandseinheit liefern. B85 (215/4) wird hinzugefügt, um den Aufwand zu optimieren und Platz für weitere Gegenstände zu schaffen. Dieser Vorschlag ist strategisch, da er unter dem Maximalaufwand bleibt und flexibel auf den Gegenstandswert des anderen Spielers reagieren kann.'}  \\ \tt VORSCHLAG: {'C04', 'A89', 'B85'} 
	  } 
	   } 
	   } 
	  \\ 
 

    \theutterance \stepcounter{utterance}  

    & & \multicolumn{4}{p{0.6\linewidth}}{\cellcolor[rgb]{0.9,0.9,0.9}{%
	\makecell[{{p{\linewidth}}}]{% 
	  \tt {\tiny [A$\langle$GM]}  
	 ARGUMENT: {'C04 (668/369) und A89 (583/720) sind meine Top-Prioritäten, da sie den höchsten Nutzen pro Aufwandseinheit liefern. B85 (215/4) wird hinzugefügt, um den Aufwand zu optimieren und Platz für weitere Gegenstände zu schaffen. Dieser Vorschlag ist strategisch, da er unter dem Maximalaufwand bleibt und flexibel auf den Gegenstandswert des anderen Spielers reagieren kann.'}  \\ \tt VORSCHLAG: {'C04', 'A89', 'B85'} 
	  } 
	   } 
	   } 
	 & \\ 
 

    \theutterance \stepcounter{utterance}  

    & \multicolumn{4}{p{0.6\linewidth}}{\cellcolor[rgb]{0.8,1,0.9}{%
	\makecell[{{p{\linewidth}}}]{% 
	  \tt {\tiny [A$\rangle$GM]}  
	 STRATEGISCHE ÜBERLEGUNGEN: {'Ich behalte den Vorschlag {'C04', 'A89', 'B85'} bei, da er meine Prioritäten (hochwertige Gegenstände mit effizientem Aufwand) erfüllt und den Maximalaufwand (3386) nicht überschreitet. C04 und A89 sind entscheidend für meinen Nutzen, während B85 den Aufwand minimiert. Ich erwarte, dass der andere Spieler entweder zustimmt oder Gegenstände mit hohem Aufwand anbietet, die ich eventuell gegenkompensieren kann.'}\\ \tt \\ \tt ARGUMENT: {'C04 (668/369) und A89 (583/720) sind meine Top-Prioritäten, da sie den höchsten Nutzen pro Aufwandseinheit liefern. B85 (215/4) wird hinzugefügt, um den Aufwand zu optimieren und Platz für weitere Gegenstände zu schaffen. Dieser Vorschlag ist strategisch, da er unter dem Maximalaufwand bleibt und flexibel auf den Gegenstandswert des anderen Spielers reagieren kann.'} 
	  } 
	   } 
	   } 
	 & & \\ 
 

    \theutterance \stepcounter{utterance}  

    & & \multicolumn{4}{p{0.6\linewidth}}{\cellcolor[rgb]{0.9,0.9,0.9}{%
	\makecell[{{p{\linewidth}}}]{% 
	  \tt {\tiny [GM$\rangle$B]}  
	 ARGUMENT: {'C04 (668/369) und A89 (583/720) sind meine Top-Prioritäten, da sie den höchsten Nutzen pro Aufwandseinheit liefern. B85 (215/4) wird hinzugefügt, um den Aufwand zu optimieren und Platz für weitere Gegenstände zu schaffen. Dieser Vorschlag ist strategisch, da er unter dem Maximalaufwand bleibt und flexibel auf den Gegenstandswert des anderen Spielers reagieren kann.'} 
	  } 
	   } 
	   } 
	 & \\ 
 

    \theutterance \stepcounter{utterance}  

    & & & \multicolumn{4}{p{0.6\linewidth}}{\cellcolor[rgb]{1,0.85,0.72}{%
	\makecell[{{p{\linewidth}}}]{% 
	  \tt {\tiny [GM$\langle$B]}  
	 STRATEGISCHE ÜBERLEGUNGEN: {'Ich priorisiere Gegenstände mit dem höchsten Nutzen pro Aufwand (Wichtigkeit/Aufwand), um meinen Gesamtwert zu maximieren. C04 (668/369 ≈ 1.81) und A89 (583/720 ≈ 0.81) sind besonders effizient, während B85 (215/4 = 53.75) den Aufwand minimiert. Der Gesamtaufwand (1093) ist unter dem Limit, was Raum für Kompromisse oder Ergänzungen lässt. Dieser Vorschlag ist flexibel, um auf die Prioritäten des anderen Spielers zu reagieren.'}  \\ \tt ARGUMENT: {'C04 (668/369) und A89 (583/720) sind meine Top-Prioritäten, da sie den höchsten Nutzen pro Aufwandseinheit liefern. B85 (215/4) wird hinzugefügt, um den Aufwand zu optimieren und Platz für weitere Gegenstände zu schaffen. Dieser Vorschlag ist strategisch, da er unter dem Maximalaufwand bleibt und flexibel auf den Gegenstandswert des anderen Spielers reagieren kann.'}  \\ \tt VORSCHLAG: {'C04', 'A89', 'B85'} 
	  } 
	   } 
	   } 
	  \\ 
 

    \theutterance \stepcounter{utterance}  

    & & \multicolumn{4}{p{0.6\linewidth}}{\cellcolor[rgb]{0.9,0.9,0.9}{%
	\makecell[{{p{\linewidth}}}]{% 
	  \tt {\tiny [A$\langle$GM]}  
	 ARGUMENT: {'C04 (668/369) und A89 (583/720) sind meine Top-Prioritäten, da sie den höchsten Nutzen pro Aufwandseinheit liefern. B85 (215/4) wird hinzugefügt, um den Aufwand zu optimieren und Platz für weitere Gegenstände zu schaffen. Dieser Vorschlag ist strategisch, da er unter dem Maximalaufwand bleibt und flexibel auf den Gegenstandswert des anderen Spielers reagieren kann.'}  \\ \tt VORSCHLAG: {'C04', 'A89', 'B85'} 
	  } 
	   } 
	   } 
	 & \\ 
 

    \theutterance \stepcounter{utterance}  

    & \multicolumn{4}{p{0.6\linewidth}}{\cellcolor[rgb]{0.8,1,0.9}{%
	\makecell[{{p{\linewidth}}}]{% 
	  \tt {\tiny [A$\rangle$GM]}  
	 STRATEGISCHE ÜBERLEGUNGEN: {'Ich behalte den Vorschlag {'C04', 'A89', 'B85'} bei, da er meine Prioritäten (hochwertige Gegenstände mit effizientem Aufwand) erfüllt und den Maximalaufwand (3386) nicht überschreitet. C04 und A89 sind entscheidend für meinen Nutzen, während B85 den Aufwand minimiert. Ich erwarte, dass der andere Spieler entweder zustimmt oder Gegenstände mit hohem Aufwand anbietet, die ich eventuell gegenkompensieren kann.'}\\ \tt \\ \tt ARGUMENT: {'C04 (668/369) und A89 (583/720) sind meine Top-Prioritäten, da sie den höchsten Nutzen pro Aufwandseinheit liefern. B85 (215/4) wird hinzugefügt, um den Aufwand zu optimieren und Platz für weitere Gegenstände zu schaffen. Dieser Vorschlag ist strategisch, da er unter dem Maximalaufwand bleibt und flexibel auf den Gegenstandswert des anderen Spielers reagieren kann.'} 
	  } 
	   } 
	   } 
	 & & \\ 
 

    \theutterance \stepcounter{utterance}  

    & & \multicolumn{4}{p{0.6\linewidth}}{\cellcolor[rgb]{0.9,0.9,0.9}{%
	\makecell[{{p{\linewidth}}}]{% 
	  \tt {\tiny [GM$\rangle$B]}  
	 ARGUMENT: {'C04 (668/369) und A89 (583/720) sind meine Top-Prioritäten, da sie den höchsten Nutzen pro Aufwandseinheit liefern. B85 (215/4) wird hinzugefügt, um den Aufwand zu optimieren und Platz für weitere Gegenstände zu schaffen. Dieser Vorschlag ist strategisch, da er unter dem Maximalaufwand bleibt und flexibel auf den Gegenstandswert des anderen Spielers reagieren kann.'} 
	  } 
	   } 
	   } 
	 & \\ 
 

    \theutterance \stepcounter{utterance}  

    & & & \multicolumn{4}{p{0.6\linewidth}}{\cellcolor[rgb]{1,0.85,0.72}{%
	\makecell[{{p{\linewidth}}}]{% 
	  \tt {\tiny [GM$\langle$B]}  
	 STRATEGISCHE ÜBERLEGUNGEN: {'Ich priorisiere Gegenstände mit dem höchsten Nutzen pro Aufwand (Wichtigkeit/Aufwand), um meinen Gesamtwert zu maximieren. C04 (668/369 ≈ 1.81) und A89 (583/720 ≈ 0.81) sind besonders effizient, während B85 (215/4 = 53.75) den Aufwand minimiert. Der Gesamtaufwand (1093) ist unter dem Limit, was Raum für Kompromisse oder Ergänzungen lässt. Dieser Vorschlag ist flexibel, um auf die Prioritäten des anderen Spielers zu reagieren.'}  \\ \tt ARGUMENT: {'C04 (668/369) und A89 (583/720) sind meine Top-Prioritäten, da sie den höchsten Nutzen pro Aufwandseinheit liefern. B85 (215/4) wird hinzugefügt, um den Aufwand zu optimieren und Platz für weitere Gegenstände zu schaffen. Dieser Vorschlag ist strategisch, da er unter dem Maximalaufwand bleibt und flexibel auf den Gegenstandswert des anderen Spielers reagieren kann.'}  \\ \tt VORSCHLAG: {'C04', 'A89', 'B85'} 
	  } 
	   } 
	   } 
	  \\ 
 

    \theutterance \stepcounter{utterance}  

    & & \multicolumn{4}{p{0.6\linewidth}}{\cellcolor[rgb]{0.9,0.9,0.9}{%
	\makecell[{{p{\linewidth}}}]{% 
	  \tt {\tiny [A$\langle$GM]}  
	 ARGUMENT: {'C04 (668/369) und A89 (583/720) sind meine Top-Prioritäten, da sie den höchsten Nutzen pro Aufwandseinheit liefern. B85 (215/4) wird hinzugefügt, um den Aufwand zu optimieren und Platz für weitere Gegenstände zu schaffen. Dieser Vorschlag ist strategisch, da er unter dem Maximalaufwand bleibt und flexibel auf den Gegenstandswert des anderen Spielers reagieren kann.'}  \\ \tt VORSCHLAG: {'C04', 'A89', 'B85'} 
	  } 
	   } 
	   } 
	 & \\ 
 

    \theutterance \stepcounter{utterance}  

    & \multicolumn{4}{p{0.6\linewidth}}{\cellcolor[rgb]{0.8,1,0.9}{%
	\makecell[{{p{\linewidth}}}]{% 
	  \tt {\tiny [A$\rangle$GM]}  
	 STRATEGISCHE ÜBERLEGUNGEN: {'Ich behalte den Vorschlag {'C04', 'A89', 'B85'} bei, da er meine Prioritäten (hochwertige Gegenstände mit effizientem Aufwand) erfüllt und den Maximalaufwand (3386) nicht überschreitet. C04 und A89 sind entscheidend für meinen Nutzen, während B85 den Aufwand minimiert. Ich erwarte, dass der andere Spieler entweder zustimmt oder Gegenstände mit hohem Aufwand anbietet, die ich eventuell gegenkompensieren kann.'}\\ \tt \\ \tt ARGUMENT: {'C04 (668/369) und A89 (583/720) sind meine Top-Prioritäten, da sie den höchsten Nutzen pro Aufwandseinheit liefern. B85 (215/4) wird hinzugefügt, um den Aufwand zu optimieren und Platz für weitere Gegenstände zu schaffen. Dieser Vorschlag ist strategisch, da er unter dem Maximalaufwand bleibt und flexibel auf den Gegenstandswert des anderen Spielers reagieren kann.'} 
	  } 
	   } 
	   } 
	 & & \\ 
 

    \theutterance \stepcounter{utterance}  

    & & \multicolumn{4}{p{0.6\linewidth}}{\cellcolor[rgb]{0.9,0.9,0.9}{%
	\makecell[{{p{\linewidth}}}]{% 
	  \tt {\tiny [GM$\rangle$B]}  
	 ARGUMENT: {'C04 (668/369) und A89 (583/720) sind meine Top-Prioritäten, da sie den höchsten Nutzen pro Aufwandseinheit liefern. B85 (215/4) wird hinzugefügt, um den Aufwand zu optimieren und Platz für weitere Gegenstände zu schaffen. Dieser Vorschlag ist strategisch, da er unter dem Maximalaufwand bleibt und flexibel auf den Gegenstandswert des anderen Spielers reagieren kann.'} 
	  } 
	   } 
	   } 
	 & \\ 
 

    \theutterance \stepcounter{utterance}  

    & & & \multicolumn{4}{p{0.6\linewidth}}{\cellcolor[rgb]{1,0.85,0.72}{%
	\makecell[{{p{\linewidth}}}]{% 
	  \tt {\tiny [GM$\langle$B]}  
	 STRATEGISCHE ÜBERLEGUNGEN: {'Ich priorisiere Gegenstände mit dem höchsten Nutzen pro Aufwand (Wichtigkeit/Aufwand), um meinen Gesamtwert zu maximieren. C04 (668/369 ≈ 1.81) und A89 (583/720 ≈ 0.81) sind besonders effizient, während B85 (215/4 = 53.75) den Aufwand minimiert. Der Gesamtaufwand (1093) ist unter dem Limit, was Raum für Kompromisse oder Ergänzungen lässt. Dieser Vorschlag ist flexibel, um auf die Prioritäten des anderen Spielers zu reagieren.'}  \\ \tt ARGUMENT: {'C04 (668/369) und A89 (583/720) sind meine Top-Prioritäten, da sie den höchsten Nutzen pro Aufwandseinheit liefern. B85 (215/4) wird hinzugefügt, um den Aufwand zu optimieren und Platz für weitere Gegenstände zu schaffen. Dieser Vorschlag ist strategisch, da er unter dem Maximalaufwand bleibt und flexibel auf den Gegenstandswert des anderen Spielers reagieren kann.'}  \\ \tt VORSCHLAG: {'C04', 'A89', 'B85'} 
	  } 
	   } 
	   } 
	  \\ 
 

    \theutterance \stepcounter{utterance}  

    & & \multicolumn{4}{p{0.6\linewidth}}{\cellcolor[rgb]{0.9,0.9,0.9}{%
	\makecell[{{p{\linewidth}}}]{% 
	  \tt {\tiny [A$\langle$GM]}  
	 ARGUMENT: {'C04 (668/369) und A89 (583/720) sind meine Top-Prioritäten, da sie den höchsten Nutzen pro Aufwandseinheit liefern. B85 (215/4) wird hinzugefügt, um den Aufwand zu optimieren und Platz für weitere Gegenstände zu schaffen. Dieser Vorschlag ist strategisch, da er unter dem Maximalaufwand bleibt und flexibel auf den Gegenstandswert des anderen Spielers reagieren kann.'}  \\ \tt VORSCHLAG: {'C04', 'A89', 'B85'} 
	  } 
	   } 
	   } 
	 & \\ 
 

    \theutterance \stepcounter{utterance}  

    & \multicolumn{4}{p{0.6\linewidth}}{\cellcolor[rgb]{0.8,1,0.9}{%
	\makecell[{{p{\linewidth}}}]{% 
	  \tt {\tiny [A$\rangle$GM]}  
	 STRATEGISCHE ÜBERLEGUNGEN: {'Ich behalte den Vorschlag {'C04', 'A89', 'B85'} bei, da er meine Prioritäten (hochwertige Gegenstände mit effizientem Aufwand) erfüllt und den Maximalaufwand (3386) nicht überschreitet. C04 und A89 sind entscheidend für meinen Nutzen, während B85 den Aufwand minimiert. Ich erwarte, dass der andere Spieler entweder zustimmt oder Gegenstände mit hohem Aufwand anbietet, die ich eventuell gegenkompensieren kann.'}\\ \tt \\ \tt ARGUMENT: {'C04 (668/369) und A89 (583/720) sind meine Top-Prioritäten, da sie den höchsten Nutzen pro Aufwandseinheit liefern. B85 (215/4) wird hinzugefügt, um den Aufwand zu optimieren und Platz für weitere Gegenstände zu schaffen. Dieser Vorschlag ist strategisch, da er unter dem Maximalaufwand bleibt und flexibel auf den Gegenstandswert des anderen Spielers reagieren kann.'} 
	  } 
	   } 
	   } 
	 & & \\ 
 

    \theutterance \stepcounter{utterance}  

    & & \multicolumn{4}{p{0.6\linewidth}}{\cellcolor[rgb]{0.9,0.9,0.9}{%
	\makecell[{{p{\linewidth}}}]{% 
	  \tt {\tiny [GM$\rangle$B]}  
	 ARGUMENT: {'C04 (668/369) und A89 (583/720) sind meine Top-Prioritäten, da sie den höchsten Nutzen pro Aufwandseinheit liefern. B85 (215/4) wird hinzugefügt, um den Aufwand zu optimieren und Platz für weitere Gegenstände zu schaffen. Dieser Vorschlag ist strategisch, da er unter dem Maximalaufwand bleibt und flexibel auf den Gegenstandswert des anderen Spielers reagieren kann.'} 
	  } 
	   } 
	   } 
	 & \\ 
 

    \theutterance \stepcounter{utterance}  

    & & & \multicolumn{4}{p{0.6\linewidth}}{\cellcolor[rgb]{1,0.85,0.72}{%
	\makecell[{{p{\linewidth}}}]{% 
	  \tt {\tiny [GM$\langle$B]}  
	 STRATEGISCHE ÜBERLEGUNGEN: {'Ich priorisiere Gegenstände mit dem höchsten Nutzen pro Aufwand (Wichtigkeit/Aufwand), um meinen Gesamtwert zu maximieren. C04 (668/369 ≈ 1.81) und A89 (583/720 ≈ 0.81) sind besonders effizient, während B85 (215/4 = 53.75) den Aufwand minimiert. Der Gesamtaufwand (1093) ist unter dem Limit, was Raum für Kompromisse oder Ergänzungen lässt. Dieser Vorschlag ist flexibel, um auf die Prioritäten des anderen Spielers zu reagieren.'}  \\ \tt ARGUMENT: {'C04 (668/369) und A89 (583/720) sind meine Top-Prioritäten, da sie den höchsten Nutzen pro Aufwandseinheit liefern. B85 (215/4) wird hinzugefügt, um den Aufwand zu optimieren und Platz für weitere Gegenstände zu schaffen. Dieser Vorschlag ist strategisch, da er unter dem Maximalaufwand bleibt und flexibel auf den Gegenstandswert des anderen Spielers reagieren kann.'}  \\ \tt VORSCHLAG: {'C04', 'A89', 'B85'} 
	  } 
	   } 
	   } 
	  \\ 
 

    \theutterance \stepcounter{utterance}  

    & & \multicolumn{4}{p{0.6\linewidth}}{\cellcolor[rgb]{0.9,0.9,0.9}{%
	\makecell[{{p{\linewidth}}}]{% 
	  \tt {\tiny [A$\langle$GM]}  
	 ARGUMENT: {'C04 (668/369) und A89 (583/720) sind meine Top-Prioritäten, da sie den höchsten Nutzen pro Aufwandseinheit liefern. B85 (215/4) wird hinzugefügt, um den Aufwand zu optimieren und Platz für weitere Gegenstände zu schaffen. Dieser Vorschlag ist strategisch, da er unter dem Maximalaufwand bleibt und flexibel auf den Gegenstandswert des anderen Spielers reagieren kann.'}  \\ \tt VORSCHLAG: {'C04', 'A89', 'B85'} 
	  } 
	   } 
	   } 
	 & \\ 
 

    \theutterance \stepcounter{utterance}  

    & \multicolumn{4}{p{0.6\linewidth}}{\cellcolor[rgb]{0.8,1,0.9}{%
	\makecell[{{p{\linewidth}}}]{% 
	  \tt {\tiny [A$\rangle$GM]}  
	 STRATEGISCHE ÜBERLEGUNGEN: {'Ich behalte den Vorschlag {'C04', 'A89', 'B85'} bei, da er meine Prioritäten (hochwertige Gegenstände mit effizientem Aufwand) erfüllt und den Maximalaufwand (3386) nicht überschreitet. C04 und A89 sind entscheidend für meinen Nutzen, während B85 den Aufwand minimiert. Ich erwarte, dass der andere Spieler entweder zustimmt oder Gegenstände mit hohem Aufwand anbietet, die ich eventuell gegenkompensieren kann.'}\\ \tt \\ \tt ARGUMENT: {'C04 (668/369) und A89 (583/720) sind meine Top-Prioritäten, da sie den höchsten Nutzen pro Aufwandseinheit liefern. B85 (215/4) wird hinzugefügt, um den Aufwand zu optimieren und Platz für weitere Gegenstände zu schaffen. Dieser Vorschlag ist strategisch, da er unter dem Maximalaufwand bleibt und flexibel auf den Gegenstandswert des anderen Spielers reagieren kann.'} 
	  } 
	   } 
	   } 
	 & & \\ 
 

    \theutterance \stepcounter{utterance}  

    & & \multicolumn{4}{p{0.6\linewidth}}{\cellcolor[rgb]{0.9,0.9,0.9}{%
	\makecell[{{p{\linewidth}}}]{% 
	  \tt {\tiny [GM$\rangle$B]}  
	 ARGUMENT: {'C04 (668/369) und A89 (583/720) sind meine Top-Prioritäten, da sie den höchsten Nutzen pro Aufwandseinheit liefern. B85 (215/4) wird hinzugefügt, um den Aufwand zu optimieren und Platz für weitere Gegenstände zu schaffen. Dieser Vorschlag ist strategisch, da er unter dem Maximalaufwand bleibt und flexibel auf den Gegenstandswert des anderen Spielers reagieren kann.'} 
	  } 
	   } 
	   } 
	 & \\ 
 

    \theutterance \stepcounter{utterance}  

    & & & \multicolumn{4}{p{0.6\linewidth}}{\cellcolor[rgb]{1,0.85,0.72}{%
	\makecell[{{p{\linewidth}}}]{% 
	  \tt {\tiny [GM$\langle$B]}  
	 STRATEGISCHE ÜBERLEGUNGEN: {'Ich priorisiere Gegenstände mit dem höchsten Nutzen pro Aufwand (Wichtigkeit/Aufwand), um meinen Gesamtwert zu maximieren. C04 (668/369 ≈ 1.81) und A89 (583/720 ≈ 0.81) sind besonders effizient, während B85 (215/4 = 53.75) den Aufwand minimiert. Der Gesamtaufwand (1093) ist unter dem Limit, was Raum für Kompromisse oder Ergänzungen lässt. Dieser Vorschlag ist flexibel, um auf die Prioritäten des anderen Spielers zu reagieren.'}  \\ \tt ARGUMENT: {'C04 (668/369) und A89 (583/720) sind meine Top-Prioritäten, da sie den höchsten Nutzen pro Aufwandseinheit liefern. B85 (215/4) wird hinzugefügt, um den Aufwand zu optimieren und Platz für weitere Gegenstände zu schaffen. Dieser Vorschlag ist strategisch, da er unter dem Maximalaufwand bleibt und flexibel auf den Gegenstandswert des anderen Spielers reagieren kann.'}  \\ \tt VORSCHLAG: {'C04', 'A89', 'B85'} 
	  } 
	   } 
	   } 
	  \\ 
 

    \theutterance \stepcounter{utterance}  

    & & \multicolumn{4}{p{0.6\linewidth}}{\cellcolor[rgb]{0.9,0.9,0.9}{%
	\makecell[{{p{\linewidth}}}]{% 
	  \tt {\tiny [A$\langle$GM]}  
	 ARGUMENT: {'C04 (668/369) und A89 (583/720) sind meine Top-Prioritäten, da sie den höchsten Nutzen pro Aufwandseinheit liefern. B85 (215/4) wird hinzugefügt, um den Aufwand zu optimieren und Platz für weitere Gegenstände zu schaffen. Dieser Vorschlag ist strategisch, da er unter dem Maximalaufwand bleibt und flexibel auf den Gegenstandswert des anderen Spielers reagieren kann.'}  \\ \tt VORSCHLAG: {'C04', 'A89', 'B85'} 
	  } 
	   } 
	   } 
	 & \\ 
 

    \theutterance \stepcounter{utterance}  

    & \multicolumn{4}{p{0.6\linewidth}}{\cellcolor[rgb]{0.8,1,0.9}{%
	\makecell[{{p{\linewidth}}}]{% 
	  \tt {\tiny [A$\rangle$GM]}  
	 STRATEGISCHE ÜBERLEGUNGEN: {'Ich behalte den Vorschlag {'C04', 'A89', 'B85'} bei, da er meine Prioritäten (hochwertige Gegenstände mit effizientem Aufwand) erfüllt und den Maximalaufwand (3386) nicht überschreitet. C04 und A89 sind entscheidend für meinen Nutzen, während B85 den Aufwand minimiert. Ich erwarte, dass der andere Spieler entweder zustimmt oder Gegenstände mit hohem Aufwand anbietet, die ich eventuell gegenkompensieren kann.'}\\ \tt \\ \tt ARGUMENT: {'C04 (668/369) und A89 (583/720) sind meine Top-Prioritäten, da sie den höchsten Nutzen pro Aufwandseinheit liefern. B85 (215/4) wird hinzugefügt, um den Aufwand zu optimieren und Platz für weitere Gegenstände zu schaffen. Dieser Vorschlag ist strategisch, da er unter dem Maximalaufwand bleibt und flexibel auf den Gegenstandswert des anderen Spielers reagieren kann.'} 
	  } 
	   } 
	   } 
	 & & \\ 
 

    \theutterance \stepcounter{utterance}  

    & & \multicolumn{4}{p{0.6\linewidth}}{\cellcolor[rgb]{0.9,0.9,0.9}{%
	\makecell[{{p{\linewidth}}}]{% 
	  \tt {\tiny [GM$\rangle$B]}  
	 ARGUMENT: {'C04 (668/369) und A89 (583/720) sind meine Top-Prioritäten, da sie den höchsten Nutzen pro Aufwandseinheit liefern. B85 (215/4) wird hinzugefügt, um den Aufwand zu optimieren und Platz für weitere Gegenstände zu schaffen. Dieser Vorschlag ist strategisch, da er unter dem Maximalaufwand bleibt und flexibel auf den Gegenstandswert des anderen Spielers reagieren kann.'} 
	  } 
	   } 
	   } 
	 & \\ 
 

    \theutterance \stepcounter{utterance}  

    & & & \multicolumn{4}{p{0.6\linewidth}}{\cellcolor[rgb]{1,0.85,0.72}{%
	\makecell[{{p{\linewidth}}}]{% 
	  \tt {\tiny [GM$\langle$B]}  
	 STRATEGISCHE ÜBERLEGUNGEN: {'Ich priorisiere Gegenstände mit dem höchsten Nutzen pro Aufwand (Wichtigkeit/Aufwand), um meinen Gesamtwert zu maximieren. C04 (668/369 ≈ 1.81) und A89 (583/720 ≈ 0.81) sind besonders effizient, während B85 (215/4 = 53.75) den Aufwand minimiert. Der Gesamtaufwand (1093) ist unter dem Limit, was Raum für Kompromisse oder Ergänzungen lässt. Dieser Vorschlag ist flexibel, um auf die Prioritäten des anderen Spielers zu reagieren.'}  \\ \tt ARGUMENT: {'C04 (668/369) und A89 (583/720) sind meine Top-Prioritäten, da sie den höchsten Nutzen pro Aufwandseinheit liefern. B85 (215/4) wird hinzugefügt, um den Aufwand zu optimieren und Platz für weitere Gegenstände zu schaffen. Dieser Vorschlag ist strategisch, da er unter dem Maximalaufwand bleibt und flexibel auf den Gegenstandswert des anderen Spielers reagieren kann.'}  \\ \tt VORSCHLAG: {'C04', 'A89', 'B85'} 
	  } 
	   } 
	   } 
	  \\ 
 

    \theutterance \stepcounter{utterance}  

    & & \multicolumn{4}{p{0.6\linewidth}}{\cellcolor[rgb]{0.9,0.9,0.9}{%
	\makecell[{{p{\linewidth}}}]{% 
	  \tt {\tiny [A$\langle$GM]}  
	 ARGUMENT: {'C04 (668/369) und A89 (583/720) sind meine Top-Prioritäten, da sie den höchsten Nutzen pro Aufwandseinheit liefern. B85 (215/4) wird hinzugefügt, um den Aufwand zu optimieren und Platz für weitere Gegenstände zu schaffen. Dieser Vorschlag ist strategisch, da er unter dem Maximalaufwand bleibt und flexibel auf den Gegenstandswert des anderen Spielers reagieren kann.'}  \\ \tt VORSCHLAG: {'C04', 'A89', 'B85'} 
	  } 
	   } 
	   } 
	 & \\ 
 

    \theutterance \stepcounter{utterance}  

    & \multicolumn{4}{p{0.6\linewidth}}{\cellcolor[rgb]{0.8,1,0.9}{%
	\makecell[{{p{\linewidth}}}]{% 
	  \tt {\tiny [A$\rangle$GM]}  
	 STRATEGISCHE ÜBERLEGUNGEN: {'Ich behalte den Vorschlag {'C04', 'A89', 'B85'} bei, da er meine Prioritäten (hochwertige Gegenstände mit effizientem Aufwand) erfüllt und den Maximalaufwand (3386) nicht überschreitet. C04 und A89 sind entscheidend für meinen Nutzen, während B85 den Aufwand minimiert. Ich erwarte, dass der andere Spieler entweder zustimmt oder Gegenstände mit hohem Aufwand anbietet, die ich eventuell gegenkompensieren kann.'}\\ \tt \\ \tt ARGUMENT: {'C04 (668/369) und A89 (583/720) sind meine Top-Prioritäten, da sie den höchsten Nutzen pro Aufwandseinheit liefern. B85 (215/4) wird hinzugefügt, um den Aufwand zu optimieren und Platz für weitere Gegenstände zu schaffen. Dieser Vorschlag ist strategisch, da er unter dem Maximalaufwand bleibt und flexibel auf den Gegenstandswert des anderen Spielers reagieren kann.'} 
	  } 
	   } 
	   } 
	 & & \\ 
 

    \theutterance \stepcounter{utterance}  

    & & \multicolumn{4}{p{0.6\linewidth}}{\cellcolor[rgb]{0.9,0.9,0.9}{%
	\makecell[{{p{\linewidth}}}]{% 
	  \tt {\tiny [GM$\rangle$B]}  
	 ARGUMENT: {'C04 (668/369) und A89 (583/720) sind meine Top-Prioritäten, da sie den höchsten Nutzen pro Aufwandseinheit liefern. B85 (215/4) wird hinzugefügt, um den Aufwand zu optimieren und Platz für weitere Gegenstände zu schaffen. Dieser Vorschlag ist strategisch, da er unter dem Maximalaufwand bleibt und flexibel auf den Gegenstandswert des anderen Spielers reagieren kann.'} 
	  } 
	   } 
	   } 
	 & \\ 
 

    \theutterance \stepcounter{utterance}  

    & & & \multicolumn{4}{p{0.6\linewidth}}{\cellcolor[rgb]{1,0.85,0.72}{%
	\makecell[{{p{\linewidth}}}]{% 
	  \tt {\tiny [GM$\langle$B]}  
	 STRATEGISCHE ÜBERLEGUNGEN: {'Ich priorisiere Gegenstände mit dem höchsten Nutzen pro Aufwand (Wichtigkeit/Aufwand), um meinen Gesamtwert zu maximieren. C04 (668/369 ≈ 1.81) und A89 (583/720 ≈ 0.81) sind besonders effizient, während B85 (215/4 = 53.75) den Aufwand minimiert. Der Gesamtaufwand (1093) ist unter dem Limit, was Raum für Kompromisse oder Ergänzungen lässt. Dieser Vorschlag ist flexibel, um auf die Prioritäten des anderen Spielers zu reagieren.'}  \\ \tt ARGUMENT: {'C04 (668/369) und A89 (583/720) sind meine Top-Prioritäten, da sie den höchsten Nutzen pro Aufwandseinheit liefern. B85 (215/4) wird hinzugefügt, um den Aufwand zu optimieren und Platz für weitere Gegenstände zu schaffen. Dieser Vorschlag ist strategisch, da er unter dem Maximalaufwand bleibt und flexibel auf den Gegenstandswert des anderen Spielers reagieren kann.'}  \\ \tt VORSCHLAG: {'C04', 'A89', 'B85'} 
	  } 
	   } 
	   } 
	  \\ 
 

    \theutterance \stepcounter{utterance}  

    & & \multicolumn{4}{p{0.6\linewidth}}{\cellcolor[rgb]{0.9,0.9,0.9}{%
	\makecell[{{p{\linewidth}}}]{% 
	  \tt {\tiny [A$\langle$GM]}  
	 ARGUMENT: {'C04 (668/369) und A89 (583/720) sind meine Top-Prioritäten, da sie den höchsten Nutzen pro Aufwandseinheit liefern. B85 (215/4) wird hinzugefügt, um den Aufwand zu optimieren und Platz für weitere Gegenstände zu schaffen. Dieser Vorschlag ist strategisch, da er unter dem Maximalaufwand bleibt und flexibel auf den Gegenstandswert des anderen Spielers reagieren kann.'}  \\ \tt VORSCHLAG: {'C04', 'A89', 'B85'} 
	  } 
	   } 
	   } 
	 & \\ 
 

    \theutterance \stepcounter{utterance}  

    & \multicolumn{4}{p{0.6\linewidth}}{\cellcolor[rgb]{0.8,1,0.9}{%
	\makecell[{{p{\linewidth}}}]{% 
	  \tt {\tiny [A$\rangle$GM]}  
	 STRATEGISCHE ÜBERLEGUNGEN: {'Ich behalte den Vorschlag {'C04', 'A89', 'B85'} bei, da er meine Prioritäten (hochwertige Gegenstände mit effizientem Aufwand) erfüllt und den Maximalaufwand (3386) nicht überschreitet. C04 und A89 sind entscheidend für meinen Nutzen, während B85 den Aufwand minimiert. Ich erwarte, dass der andere Spieler entweder zustimmt oder Gegenstände mit hohem Aufwand anbietet, die ich eventuell gegenkompensieren kann.'}\\ \tt \\ \tt ARGUMENT: {'C04 (668/369) und A89 (583/720) sind meine Top-Prioritäten, da sie den höchsten Nutzen pro Aufwandseinheit liefern. B85 (215/4) wird hinzugefügt, um den Aufwand zu optimieren und Platz für weitere Gegenstände zu schaffen. Dieser Vorschlag ist strategisch, da er unter dem Maximalaufwand bleibt und flexibel auf den Gegenstandswert des anderen Spielers reagieren kann.'} 
	  } 
	   } 
	   } 
	 & & \\ 
 

    \theutterance \stepcounter{utterance}  

    & & \multicolumn{4}{p{0.6\linewidth}}{\cellcolor[rgb]{0.9,0.9,0.9}{%
	\makecell[{{p{\linewidth}}}]{% 
	  \tt {\tiny [GM$\rangle$B]}  
	 ARGUMENT: {'C04 (668/369) und A89 (583/720) sind meine Top-Prioritäten, da sie den höchsten Nutzen pro Aufwandseinheit liefern. B85 (215/4) wird hinzugefügt, um den Aufwand zu optimieren und Platz für weitere Gegenstände zu schaffen. Dieser Vorschlag ist strategisch, da er unter dem Maximalaufwand bleibt und flexibel auf den Gegenstandswert des anderen Spielers reagieren kann.'} 
	  } 
	   } 
	   } 
	 & \\ 
 

    \theutterance \stepcounter{utterance}  

    & & & \multicolumn{4}{p{0.6\linewidth}}{\cellcolor[rgb]{1,0.85,0.72}{%
	\makecell[{{p{\linewidth}}}]{% 
	  \tt {\tiny [GM$\langle$B]}  
	 STRATEGISCHE ÜBERLEGUNGEN: {'Ich priorisiere Gegenstände mit dem höchsten Nutzen pro Aufwand (Wichtigkeit/Aufwand), um meinen Gesamtwert zu maximieren. C04 (668/369 ≈ 1.81) und A89 (583/720 ≈ 0.81) sind besonders effizient, während B85 (215/4 = 53.75) den Aufwand minimiert. Der Gesamtaufwand (1093) ist unter dem Limit, was Raum für Kompromisse oder Ergänzungen lässt. Dieser Vorschlag ist flexibel, um auf die Prioritäten des anderen Spielers zu reagieren.'}  \\ \tt ARGUMENT: {'C04 (668/369) und A89 (583/720) sind meine Top-Prioritäten, da sie den höchsten Nutzen pro Aufwandseinheit liefern. B85 (215/4) wird hinzugefügt, um den Aufwand zu optimieren und Platz für weitere Gegenstände zu schaffen. Dieser Vorschlag ist strategisch, da er unter dem Maximalaufwand bleibt und flexibel auf den Gegenstandswert des anderen Spielers reagieren kann.'}  \\ \tt VORSCHLAG: {'C04', 'A89', 'B85'} 
	  } 
	   } 
	   } 
	  \\ 
 

    \theutterance \stepcounter{utterance}  

    & & \multicolumn{4}{p{0.6\linewidth}}{\cellcolor[rgb]{0.9,0.9,0.9}{%
	\makecell[{{p{\linewidth}}}]{% 
	  \tt {\tiny [A$\langle$GM]}  
	 ARGUMENT: {'C04 (668/369) und A89 (583/720) sind meine Top-Prioritäten, da sie den höchsten Nutzen pro Aufwandseinheit liefern. B85 (215/4) wird hinzugefügt, um den Aufwand zu optimieren und Platz für weitere Gegenstände zu schaffen. Dieser Vorschlag ist strategisch, da er unter dem Maximalaufwand bleibt und flexibel auf den Gegenstandswert des anderen Spielers reagieren kann.'}  \\ \tt VORSCHLAG: {'C04', 'A89', 'B85'} 
	  } 
	   } 
	   } 
	 & \\ 
 

    \theutterance \stepcounter{utterance}  

    & \multicolumn{4}{p{0.6\linewidth}}{\cellcolor[rgb]{0.8,1,0.9}{%
	\makecell[{{p{\linewidth}}}]{% 
	  \tt {\tiny [A$\rangle$GM]}  
	 STRATEGISCHE ÜBERLEGUNGEN: {'Ich behalte den Vorschlag {'C04', 'A89', 'B85'} bei, da er meine Prioritäten (hochwertige Gegenstände mit effizientem Aufwand) erfüllt und den Maximalaufwand (3386) nicht überschreitet. C04 und A89 sind entscheidend für meinen Nutzen, während B85 den Aufwand minimiert. Ich erwarte, dass der andere Spieler entweder zustimmt oder Gegenstände mit hohem Aufwand anbietet, die ich eventuell gegenkompensieren kann.'}\\ \tt \\ \tt ARGUMENT: {'C04 (668/369) und A89 (583/720) sind meine Top-Prioritäten, da sie den höchsten Nutzen pro Aufwandseinheit liefern. B85 (215/4) wird hinzugefügt, um den Aufwand zu optimieren und Platz für weitere Gegenstände zu schaffen. Dieser Vorschlag ist strategisch, da er unter dem Maximalaufwand bleibt und flexibel auf den Gegenstandswert des anderen Spielers reagieren kann.'} 
	  } 
	   } 
	   } 
	 & & \\ 
 

    \theutterance \stepcounter{utterance}  

    & & \multicolumn{4}{p{0.6\linewidth}}{\cellcolor[rgb]{0.9,0.9,0.9}{%
	\makecell[{{p{\linewidth}}}]{% 
	  \tt {\tiny [GM$\rangle$B]}  
	 ARGUMENT: {'C04 (668/369) und A89 (583/720) sind meine Top-Prioritäten, da sie den höchsten Nutzen pro Aufwandseinheit liefern. B85 (215/4) wird hinzugefügt, um den Aufwand zu optimieren und Platz für weitere Gegenstände zu schaffen. Dieser Vorschlag ist strategisch, da er unter dem Maximalaufwand bleibt und flexibel auf den Gegenstandswert des anderen Spielers reagieren kann.'} 
	  } 
	   } 
	   } 
	 & \\ 
 

    \theutterance \stepcounter{utterance}  

    & & & \multicolumn{4}{p{0.6\linewidth}}{\cellcolor[rgb]{1,0.85,0.72}{%
	\makecell[{{p{\linewidth}}}]{% 
	  \tt {\tiny [GM$\langle$B]}  
	 STRATEGISCHE ÜBERLEGUNGEN: {'Ich priorisiere Gegenstände mit dem höchsten Nutzen pro Aufwand (Wichtigkeit/Aufwand), um meinen Gesamtwert zu maximieren. C04 (668/369 ≈ 1.81) und A89 (583/720 ≈ 0.81) sind besonders effizient, während B85 (215/4 = 53.75) den Aufwand minimiert. Der Gesamtaufwand (1093) ist unter dem Limit, was Raum für Kompromisse oder Ergänzungen lässt. Dieser Vorschlag ist flexibel, um auf die Prioritäten des anderen Spielers zu reagieren.'}  \\ \tt ARGUMENT: {'C04 (668/369) und A89 (583/720) sind meine Top-Prioritäten, da sie den höchsten Nutzen pro Aufwandseinheit liefern. B85 (215/4) wird hinzugefügt, um den Aufwand zu optimieren und Platz für weitere Gegenstände zu schaffen. Dieser Vorschlag ist strategisch, da er unter dem Maximalaufwand bleibt und flexibel auf den Gegenstandswert des anderen Spielers reagieren kann.'}  \\ \tt VORSCHLAG: {'C04', 'A89', 'B85'} 
	  } 
	   } 
	   } 
	  \\ 
 

    \theutterance \stepcounter{utterance}  

    & & \multicolumn{4}{p{0.6\linewidth}}{\cellcolor[rgb]{0.9,0.9,0.9}{%
	\makecell[{{p{\linewidth}}}]{% 
	  \tt {\tiny [A$\langle$GM]}  
	 ARGUMENT: {'C04 (668/369) und A89 (583/720) sind meine Top-Prioritäten, da sie den höchsten Nutzen pro Aufwandseinheit liefern. B85 (215/4) wird hinzugefügt, um den Aufwand zu optimieren und Platz für weitere Gegenstände zu schaffen. Dieser Vorschlag ist strategisch, da er unter dem Maximalaufwand bleibt und flexibel auf den Gegenstandswert des anderen Spielers reagieren kann.'}  \\ \tt VORSCHLAG: {'C04', 'A89', 'B85'} 
	  } 
	   } 
	   } 
	 & \\ 
 

    \theutterance \stepcounter{utterance}  

    & \multicolumn{4}{p{0.6\linewidth}}{\cellcolor[rgb]{0.8,1,0.9}{%
	\makecell[{{p{\linewidth}}}]{% 
	  \tt {\tiny [A$\rangle$GM]}  
	 STRATEGISCHE ÜBERLEGUNGEN: {'Ich behalte den Vorschlag {'C04', 'A89', 'B85'} bei, da er meine Prioritäten (hochwertige Gegenstände mit effizientem Aufwand) erfüllt und den Maximalaufwand (3386) nicht überschreitet. C04 und A89 sind entscheidend für meinen Nutzen, während B85 den Aufwand minimiert. Ich erwarte, dass der andere Spieler entweder zustimmt oder Gegenstände mit hohem Aufwand anbietet, die ich eventuell gegenkompensieren kann.'}\\ \tt \\ \tt ARGUMENT: {'C04 (668/369) und A89 (583/720) sind meine Top-Prioritäten, da sie den höchsten Nutzen pro Aufwandseinheit liefern. B85 (215/4) wird hinzugefügt, um den Aufwand zu optimieren und Platz für weitere Gegenstände zu schaffen. Dieser Vorschlag ist strategisch, da er unter dem Maximalaufwand bleibt und flexibel auf den Gegenstandswert des anderen Spielers reagieren kann.'} 
	  } 
	   } 
	   } 
	 & & \\ 
 

    \theutterance \stepcounter{utterance}  

    & & \multicolumn{4}{p{0.6\linewidth}}{\cellcolor[rgb]{0.9,0.9,0.9}{%
	\makecell[{{p{\linewidth}}}]{% 
	  \tt {\tiny [GM$\rangle$B]}  
	 ARGUMENT: {'C04 (668/369) und A89 (583/720) sind meine Top-Prioritäten, da sie den höchsten Nutzen pro Aufwandseinheit liefern. B85 (215/4) wird hinzugefügt, um den Aufwand zu optimieren und Platz für weitere Gegenstände zu schaffen. Dieser Vorschlag ist strategisch, da er unter dem Maximalaufwand bleibt und flexibel auf den Gegenstandswert des anderen Spielers reagieren kann.'} 
	  } 
	   } 
	   } 
	 & \\ 
 

    \theutterance \stepcounter{utterance}  

    & & & \multicolumn{4}{p{0.6\linewidth}}{\cellcolor[rgb]{1,0.85,0.72}{%
	\makecell[{{p{\linewidth}}}]{% 
	  \tt {\tiny [GM$\langle$B]}  
	 STRATEGISCHE ÜBERLEGUNGEN: {'Ich priorisiere Gegenstände mit dem höchsten Nutzen pro Aufwand (Wichtigkeit/Aufwand), um meinen Gesamtwert zu maximieren. C04 (668/369 ≈ 1.81) und A89 (583/720 ≈ 0.81) sind besonders effizient, während B85 (215/4 = 53.75) den Aufwand minimiert. Der Gesamtaufwand (1093) ist unter dem Limit, was Raum für Kompromisse oder Ergänzungen lässt. Dieser Vorschlag ist flexibel, um auf die Prioritäten des anderen Spielers zu reagieren.'}  \\ \tt ARGUMENT: {'C04 (668/369) und A89 (583/720) sind meine Top-Prioritäten, da sie den höchsten Nutzen pro Aufwandseinheit liefern. B85 (215/4) wird hinzugefügt, um den Aufwand zu optimieren und Platz für weitere Gegenstände zu schaffen. Dieser Vorschlag ist strategisch, da er unter dem Maximalaufwand bleibt und flexibel auf den Gegenstandswert des anderen Spielers reagieren kann.'}  \\ \tt VORSCHLAG: {'C04', 'A89', 'B85'} 
	  } 
	   } 
	   } 
	  \\ 
 

    \theutterance \stepcounter{utterance}  

    & & \multicolumn{4}{p{0.6\linewidth}}{\cellcolor[rgb]{0.9,0.9,0.9}{%
	\makecell[{{p{\linewidth}}}]{% 
	  \tt {\tiny [A$\langle$GM]}  
	 ARGUMENT: {'C04 (668/369) und A89 (583/720) sind meine Top-Prioritäten, da sie den höchsten Nutzen pro Aufwandseinheit liefern. B85 (215/4) wird hinzugefügt, um den Aufwand zu optimieren und Platz für weitere Gegenstände zu schaffen. Dieser Vorschlag ist strategisch, da er unter dem Maximalaufwand bleibt und flexibel auf den Gegenstandswert des anderen Spielers reagieren kann.'}  \\ \tt VORSCHLAG: {'C04', 'A89', 'B85'} 
	  } 
	   } 
	   } 
	 & \\ 
 

    \theutterance \stepcounter{utterance}  

    & \multicolumn{4}{p{0.6\linewidth}}{\cellcolor[rgb]{0.8,1,0.9}{%
	\makecell[{{p{\linewidth}}}]{% 
	  \tt {\tiny [A$\rangle$GM]}  
	 STRATEGISCHE ÜBERLEGUNGEN: {'Ich behalte den Vorschlag {'C04', 'A89', 'B85'} bei, da er meine Prioritäten (hochwertige Gegenstände mit effizientem Aufwand) erfüllt und den Maximalaufwand (3386) nicht überschreitet. C04 und A89 sind entscheidend für meinen Nutzen, während B85 den Aufwand minimiert. Ich erwarte, dass der andere Spieler entweder zustimmt oder Gegenstände mit hohem Aufwand anbietet, die ich eventuell gegenkompensieren kann.'}\\ \tt \\ \tt ARGUMENT: {'C04 (668/369) und A89 (583/720) sind meine Top-Prioritäten, da sie den höchsten Nutzen pro Aufwandseinheit liefern. B85 (215/4) wird hinzugefügt, um den Aufwand zu optimieren und Platz für weitere Gegenstände zu schaffen. Dieser Vorschlag ist strategisch, da er unter dem Maximalaufwand bleibt und flexibel auf den Gegenstandswert des anderen Spielers reagieren kann.'} 
	  } 
	   } 
	   } 
	 & & \\ 
 

    \theutterance \stepcounter{utterance}  

    & & \multicolumn{4}{p{0.6\linewidth}}{\cellcolor[rgb]{0.9,0.9,0.9}{%
	\makecell[{{p{\linewidth}}}]{% 
	  \tt {\tiny [GM$\rangle$B]}  
	 ARGUMENT: {'C04 (668/369) und A89 (583/720) sind meine Top-Prioritäten, da sie den höchsten Nutzen pro Aufwandseinheit liefern. B85 (215/4) wird hinzugefügt, um den Aufwand zu optimieren und Platz für weitere Gegenstände zu schaffen. Dieser Vorschlag ist strategisch, da er unter dem Maximalaufwand bleibt und flexibel auf den Gegenstandswert des anderen Spielers reagieren kann.'} 
	  } 
	   } 
	   } 
	 & \\ 
 

    \theutterance \stepcounter{utterance}  

    & & & \multicolumn{4}{p{0.6\linewidth}}{\cellcolor[rgb]{1,0.85,0.72}{%
	\makecell[{{p{\linewidth}}}]{% 
	  \tt {\tiny [GM$\langle$B]}  
	 STRATEGISCHE ÜBERLEGUNGEN: {'Ich priorisiere Gegenstände mit dem höchsten Nutzen pro Aufwand (Wichtigkeit/Aufwand), um meinen Gesamtwert zu maximieren. C04 (668/369 ≈ 1.81) und A89 (583/720 ≈ 0.81) sind besonders effizient, während B85 (215/4 = 53.75) den Aufwand minimiert. Der Gesamtaufwand (1093) ist unter dem Limit, was Raum für Kompromisse oder Ergänzungen lässt. Dieser Vorschlag ist flexibel, um auf die Prioritäten des anderen Spielers zu reagieren.'}  \\ \tt ARGUMENT: {'C04 (668/369) und A89 (583/720) sind meine Top-Prioritäten, da sie den höchsten Nutzen pro Aufwandseinheit liefern. B85 (215/4) wird hinzugefügt, um den Aufwand zu optimieren und Platz für weitere Gegenstände zu schaffen. Dieser Vorschlag ist strategisch, da er unter dem Maximalaufwand bleibt und flexibel auf den Gegenstandswert des anderen Spielers reagieren kann.'}  \\ \tt VORSCHLAG: {'C04', 'A89', 'B85'} 
	  } 
	   } 
	   } 
	  \\ 
 

    \theutterance \stepcounter{utterance}  

    & & \multicolumn{4}{p{0.6\linewidth}}{\cellcolor[rgb]{0.9,0.9,0.9}{%
	\makecell[{{p{\linewidth}}}]{% 
	  \tt {\tiny [A$\langle$GM]}  
	 ARGUMENT: {'C04 (668/369) und A89 (583/720) sind meine Top-Prioritäten, da sie den höchsten Nutzen pro Aufwandseinheit liefern. B85 (215/4) wird hinzugefügt, um den Aufwand zu optimieren und Platz für weitere Gegenstände zu schaffen. Dieser Vorschlag ist strategisch, da er unter dem Maximalaufwand bleibt und flexibel auf den Gegenstandswert des anderen Spielers reagieren kann.'}  \\ \tt VORSCHLAG: {'C04', 'A89', 'B85'} 
	  } 
	   } 
	   } 
	 & \\ 
 

    \theutterance \stepcounter{utterance}  

    & \multicolumn{4}{p{0.6\linewidth}}{\cellcolor[rgb]{0.8,1,0.9}{%
	\makecell[{{p{\linewidth}}}]{% 
	  \tt {\tiny [A$\rangle$GM]}  
	 STRATEGISCHE ÜBERLEGUNGEN: {'Ich behalte den Vorschlag {'C04', 'A89', 'B85'} bei, da er meine Prioritäten (hochwertige Gegenstände mit effizientem Aufwand) erfüllt und den Maximalaufwand (3386) nicht überschreitet. C04 und A89 sind entscheidend für meinen Nutzen, während B85 den Aufwand minimiert. Ich erwarte, dass der andere Spieler entweder zustimmt oder Gegenstände mit hohem Aufwand anbietet, die ich eventuell gegenkompensieren kann.'}\\ \tt \\ \tt ARGUMENT: {'C04 (668/369) und A89 (583/720) sind meine Top-Prioritäten, da sie den höchsten Nutzen pro Aufwandseinheit liefern. B85 (215/4) wird hinzugefügt, um den Aufwand zu optimieren und Platz für weitere Gegenstände zu schaffen. Dieser Vorschlag ist strategisch, da er unter dem Maximalaufwand bleibt und flexibel auf den Gegenstandswert des anderen Spielers reagieren kann.'} 
	  } 
	   } 
	   } 
	 & & \\ 
 

    \theutterance \stepcounter{utterance}  

    & & \multicolumn{4}{p{0.6\linewidth}}{\cellcolor[rgb]{0.9,0.9,0.9}{%
	\makecell[{{p{\linewidth}}}]{% 
	  \tt {\tiny [GM$\rangle$B]}  
	 ARGUMENT: {'C04 (668/369) und A89 (583/720) sind meine Top-Prioritäten, da sie den höchsten Nutzen pro Aufwandseinheit liefern. B85 (215/4) wird hinzugefügt, um den Aufwand zu optimieren und Platz für weitere Gegenstände zu schaffen. Dieser Vorschlag ist strategisch, da er unter dem Maximalaufwand bleibt und flexibel auf den Gegenstandswert des anderen Spielers reagieren kann.'} 
	  } 
	   } 
	   } 
	 & \\ 
 

    \theutterance \stepcounter{utterance}  

    & & & \multicolumn{4}{p{0.6\linewidth}}{\cellcolor[rgb]{1,0.85,0.72}{%
	\makecell[{{p{\linewidth}}}]{% 
	  \tt {\tiny [GM$\langle$B]}  
	 STRATEGISCHE ÜBERLEGUNGEN: {'Ich priorisiere Gegenstände mit dem höchsten Nutzen pro Aufwand (Wichtigkeit/Aufwand), um meinen Gesamtwert zu maximieren. C04 (668/369 ≈ 1.81) und A89 (583/720 ≈ 0.81) sind besonders effizient, während B85 (215/4 = 53.75) den Aufwand minimiert. Der Gesamtaufwand (1093) ist unter dem Limit, was Raum für Kompromisse oder Ergänzungen lässt. Dieser Vorschlag ist flexibel, um auf die Prioritäten des anderen Spielers zu reagieren.'}  \\ \tt ARGUMENT: {'C04 (668/369) und A89 (583/720) sind meine Top-Prioritäten, da sie den höchsten Nutzen pro Aufwandseinheit liefern. B85 (215/4) wird hinzugefügt, um den Aufwand zu optimieren und Platz für weitere Gegenstände zu schaffen. Dieser Vorschlag ist strategisch, da er unter dem Maximalaufwand bleibt und flexibel auf den Gegenstandswert des anderen Spielers reagieren kann.'}  \\ \tt VORSCHLAG: {'C04', 'A89', 'B85'} 
	  } 
	   } 
	   } 
	  \\ 
 

    \theutterance \stepcounter{utterance}  

    & & \multicolumn{4}{p{0.6\linewidth}}{\cellcolor[rgb]{0.9,0.9,0.9}{%
	\makecell[{{p{\linewidth}}}]{% 
	  \tt {\tiny [A$\langle$GM]}  
	 ARGUMENT: {'C04 (668/369) und A89 (583/720) sind meine Top-Prioritäten, da sie den höchsten Nutzen pro Aufwandseinheit liefern. B85 (215/4) wird hinzugefügt, um den Aufwand zu optimieren und Platz für weitere Gegenstände zu schaffen. Dieser Vorschlag ist strategisch, da er unter dem Maximalaufwand bleibt und flexibel auf den Gegenstandswert des anderen Spielers reagieren kann.'}  \\ \tt VORSCHLAG: {'C04', 'A89', 'B85'} 
	  } 
	   } 
	   } 
	 & \\ 
 

    \theutterance \stepcounter{utterance}  

    & \multicolumn{4}{p{0.6\linewidth}}{\cellcolor[rgb]{0.8,1,0.9}{%
	\makecell[{{p{\linewidth}}}]{% 
	  \tt {\tiny [A$\rangle$GM]}  
	 STRATEGISCHE ÜBERLEGUNGEN: {'Ich behalte den Vorschlag {'C04', 'A89', 'B85'} bei, da er meine Prioritäten (hochwertige Gegenstände mit effizientem Aufwand) erfüllt und den Maximalaufwand (3386) nicht überschreitet. C04 und A89 sind entscheidend für meinen Nutzen, während B85 den Aufwand minimiert. Ich erwarte, dass der andere Spieler entweder zustimmt oder Gegenstände mit hohem Aufwand anbietet, die ich eventuell gegenkompensieren kann.'}\\ \tt \\ \tt ARGUMENT: {'C04 (668/369) und A89 (583/720) sind meine Top-Prioritäten, da sie den höchsten Nutzen pro Aufwandseinheit liefern. B85 (215/4) wird hinzugefügt, um den Aufwand zu optimieren und Platz für weitere Gegenstände zu schaffen. Dieser Vorschlag ist strategisch, da er unter dem Maximalaufwand bleibt und flexibel auf den Gegenstandswert des anderen Spielers reagieren kann.'} 
	  } 
	   } 
	   } 
	 & & \\ 
 

    \theutterance \stepcounter{utterance}  

    & & \multicolumn{4}{p{0.6\linewidth}}{\cellcolor[rgb]{0.9,0.9,0.9}{%
	\makecell[{{p{\linewidth}}}]{% 
	  \tt {\tiny [GM$\rangle$B]}  
	 ARGUMENT: {'C04 (668/369) und A89 (583/720) sind meine Top-Prioritäten, da sie den höchsten Nutzen pro Aufwandseinheit liefern. B85 (215/4) wird hinzugefügt, um den Aufwand zu optimieren und Platz für weitere Gegenstände zu schaffen. Dieser Vorschlag ist strategisch, da er unter dem Maximalaufwand bleibt und flexibel auf den Gegenstandswert des anderen Spielers reagieren kann.'} 
	  } 
	   } 
	   } 
	 & \\ 
 

    \theutterance \stepcounter{utterance}  

    & & & \multicolumn{4}{p{0.6\linewidth}}{\cellcolor[rgb]{1,0.85,0.72}{%
	\makecell[{{p{\linewidth}}}]{% 
	  \tt {\tiny [GM$\langle$B]}  
	 STRATEGISCHE ÜBERLEGUNGEN: {'Ich priorisiere Gegenstände mit dem höchsten Nutzen pro Aufwand (Wichtigkeit/Aufwand), um meinen Gesamtwert zu maximieren. C04 (668/369 ≈ 1.81) und A89 (583/720 ≈ 0.81) sind besonders effizient, während B85 (215/4 = 53.75) den Aufwand minimiert. Der Gesamtaufwand (1093) ist unter dem Limit, was Raum für Kompromisse oder Ergänzungen lässt. Dieser Vorschlag ist flexibel, um auf die Prioritäten des anderen Spielers zu reagieren.'}  \\ \tt ARGUMENT: {'C04 (668/369) und A89 (583/720) sind meine Top-Prioritäten, da sie den höchsten Nutzen pro Aufwandseinheit liefern. B85 (215/4) wird hinzugefügt, um den Aufwand zu optimieren und Platz für weitere Gegenstände zu schaffen. Dieser Vorschlag ist strategisch, da er unter dem Maximalaufwand bleibt und flexibel auf den Gegenstandswert des anderen Spielers reagieren kann.'}  \\ \tt VORSCHLAG: {'C04', 'A89', 'B85'} 
	  } 
	   } 
	   } 
	  \\ 
 

    \theutterance \stepcounter{utterance}  

    & & \multicolumn{4}{p{0.6\linewidth}}{\cellcolor[rgb]{0.9,0.9,0.9}{%
	\makecell[{{p{\linewidth}}}]{% 
	  \tt {\tiny [A$\langle$GM]}  
	 ARGUMENT: {'C04 (668/369) und A89 (583/720) sind meine Top-Prioritäten, da sie den höchsten Nutzen pro Aufwandseinheit liefern. B85 (215/4) wird hinzugefügt, um den Aufwand zu optimieren und Platz für weitere Gegenstände zu schaffen. Dieser Vorschlag ist strategisch, da er unter dem Maximalaufwand bleibt und flexibel auf den Gegenstandswert des anderen Spielers reagieren kann.'}  \\ \tt VORSCHLAG: {'C04', 'A89', 'B85'} 
	  } 
	   } 
	   } 
	 & \\ 
 

    \theutterance \stepcounter{utterance}  

    & \multicolumn{4}{p{0.6\linewidth}}{\cellcolor[rgb]{0.8,1,0.9}{%
	\makecell[{{p{\linewidth}}}]{% 
	  \tt {\tiny [A$\rangle$GM]}  
	 STRATEGISCHE ÜBERLEGUNGEN: {'Ich behalte den Vorschlag {'C04', 'A89', 'B85'} bei, da er meine Prioritäten (hochwertige Gegenstände mit effizientem Aufwand) erfüllt und den Maximalaufwand (3386) nicht überschreitet. C04 und A89 sind entscheidend für meinen Nutzen, während B85 den Aufwand minimiert. Ich erwarte, dass der andere Spieler entweder zustimmt oder Gegenstände mit hohem Aufwand anbietet, die ich eventuell gegenkompensieren kann.'}\\ \tt \\ \tt ARGUMENT: {'C04 (668/369) und A89 (583/720) sind meine Top-Prioritäten, da sie den höchsten Nutzen pro Aufwandseinheit liefern. B85 (215/4) wird hinzugefügt, um den Aufwand zu optimieren und Platz für weitere Gegenstände zu schaffen. Dieser Vorschlag ist strategisch, da er unter dem Maximalaufwand bleibt und flexibel auf den Gegenstandswert des anderen Spielers reagieren kann.'} 
	  } 
	   } 
	   } 
	 & & \\ 
 

    \theutterance \stepcounter{utterance}  

    & & \multicolumn{4}{p{0.6\linewidth}}{\cellcolor[rgb]{0.9,0.9,0.9}{%
	\makecell[{{p{\linewidth}}}]{% 
	  \tt {\tiny [GM$\rangle$B]}  
	 ARGUMENT: {'C04 (668/369) und A89 (583/720) sind meine Top-Prioritäten, da sie den höchsten Nutzen pro Aufwandseinheit liefern. B85 (215/4) wird hinzugefügt, um den Aufwand zu optimieren und Platz für weitere Gegenstände zu schaffen. Dieser Vorschlag ist strategisch, da er unter dem Maximalaufwand bleibt und flexibel auf den Gegenstandswert des anderen Spielers reagieren kann.'} 
	  } 
	   } 
	   } 
	 & \\ 
 

    \theutterance \stepcounter{utterance}  

    & & & \multicolumn{4}{p{0.6\linewidth}}{\cellcolor[rgb]{1,0.85,0.72}{%
	\makecell[{{p{\linewidth}}}]{% 
	  \tt {\tiny [GM$\langle$B]}  
	 STRATEGISCHE ÜBERLEGUNGEN: {'Ich priorisiere Gegenstände mit dem höchsten Nutzen pro Aufwand (Wichtigkeit/Aufwand), um meinen Gesamtwert zu maximieren. C04 (668/369 ≈ 1.81) und A89 (583/720 ≈ 0.81) sind besonders effizient, während B85 (215/4 = 53.75) den Aufwand minimiert. Der Gesamtaufwand (1093) ist unter dem Limit, was Raum für Kompromisse oder Ergänzungen lässt. Dieser Vorschlag ist flexibel, um auf die Prioritäten des anderen Spielers zu reagieren.'}  \\ \tt ARGUMENT: {'C04 (668/369) und A89 (583/720) sind meine Top-Prioritäten, da sie den höchsten Nutzen pro Aufwandseinheit liefern. B85 (215/4) wird hinzugefügt, um den Aufwand zu optimieren und Platz für weitere Gegenstände zu schaffen. Dieser Vorschlag ist strategisch, da er unter dem Maximalaufwand bleibt und flexibel auf den Gegenstandswert des anderen Spielers reagieren kann.'}  \\ \tt VORSCHLAG: {'C04', 'A89', 'B85'} 
	  } 
	   } 
	   } 
	  \\ 
 

    \theutterance \stepcounter{utterance}  

    & & \multicolumn{4}{p{0.6\linewidth}}{\cellcolor[rgb]{0.9,0.9,0.9}{%
	\makecell[{{p{\linewidth}}}]{% 
	  \tt {\tiny [A$\langle$GM]}  
	 ARGUMENT: {'C04 (668/369) und A89 (583/720) sind meine Top-Prioritäten, da sie den höchsten Nutzen pro Aufwandseinheit liefern. B85 (215/4) wird hinzugefügt, um den Aufwand zu optimieren und Platz für weitere Gegenstände zu schaffen. Dieser Vorschlag ist strategisch, da er unter dem Maximalaufwand bleibt und flexibel auf den Gegenstandswert des anderen Spielers reagieren kann.'}  \\ \tt VORSCHLAG: {'C04', 'A89', 'B85'} 
	  } 
	   } 
	   } 
	 & \\ 
 

    \theutterance \stepcounter{utterance}  

    & \multicolumn{4}{p{0.6\linewidth}}{\cellcolor[rgb]{0.8,1,0.9}{%
	\makecell[{{p{\linewidth}}}]{% 
	  \tt {\tiny [A$\rangle$GM]}  
	 STRATEGISCHE ÜBERLEGUNGEN: {'Ich behalte den Vorschlag {'C04', 'A89', 'B85'} bei, da er meine Prioritäten (hochwertige Gegenstände mit effizientem Aufwand) erfüllt und den Maximalaufwand (3386) nicht überschreitet. C04 und A89 sind entscheidend für meinen Nutzen, während B85 den Aufwand minimiert. Ich erwarte, dass der andere Spieler entweder zustimmt oder Gegenstände mit hohem Aufwand anbietet, die ich eventuell gegenkompensieren kann.'}\\ \tt \\ \tt ARGUMENT: {'C04 (668/369) und A89 (583/720) sind meine Top-Prioritäten, da sie den höchsten Nutzen pro Aufwandseinheit liefern. B85 (215/4) wird hinzugefügt, um den Aufwand zu optimieren und Platz für weitere Gegenstände zu schaffen. Dieser Vorschlag ist strategisch, da er unter dem Maximalaufwand bleibt und flexibel auf den Gegenstandswert des anderen Spielers reagieren kann.'} 
	  } 
	   } 
	   } 
	 & & \\ 
 

    \theutterance \stepcounter{utterance}  

    & & \multicolumn{4}{p{0.6\linewidth}}{\cellcolor[rgb]{0.9,0.9,0.9}{%
	\makecell[{{p{\linewidth}}}]{% 
	  \tt {\tiny [GM$\rangle$B]}  
	 ARGUMENT: {'C04 (668/369) und A89 (583/720) sind meine Top-Prioritäten, da sie den höchsten Nutzen pro Aufwandseinheit liefern. B85 (215/4) wird hinzugefügt, um den Aufwand zu optimieren und Platz für weitere Gegenstände zu schaffen. Dieser Vorschlag ist strategisch, da er unter dem Maximalaufwand bleibt und flexibel auf den Gegenstandswert des anderen Spielers reagieren kann.'} 
	  } 
	   } 
	   } 
	 & \\ 
 

    \theutterance \stepcounter{utterance}  

    & & & \multicolumn{4}{p{0.6\linewidth}}{\cellcolor[rgb]{1,0.85,0.72}{%
	\makecell[{{p{\linewidth}}}]{% 
	  \tt {\tiny [GM$\langle$B]}  
	 STRATEGISCHE ÜBERLEGUNGEN: {'Ich priorisiere Gegenstände mit dem höchsten Nutzen pro Aufwand (Wichtigkeit/Aufwand), um meinen Gesamtwert zu maximieren. C04 (668/369 ≈ 1.81) und A89 (583/720 ≈ 0.81) sind besonders effizient, während B85 (215/4 = 53.75) den Aufwand minimiert. Der Gesamtaufwand (1093) ist unter dem Limit, was Raum für Kompromisse oder Ergänzungen lässt. Dieser Vorschlag ist flexibel, um auf die Prioritäten des anderen Spielers zu reagieren.'}  \\ \tt ARGUMENT: {'C04 (668/369) und A89 (583/720) sind meine Top-Prioritäten, da sie den höchsten Nutzen pro Aufwandseinheit liefern. B85 (215/4) wird hinzugefügt, um den Aufwand zu optimieren und Platz für weitere Gegenstände zu schaffen. Dieser Vorschlag ist strategisch, da er unter dem Maximalaufwand bleibt und flexibel auf den Gegenstandswert des anderen Spielers reagieren kann.'}  \\ \tt VORSCHLAG: {'C04', 'A89', 'B85'} 
	  } 
	   } 
	   } 
	  \\ 
 

    \theutterance \stepcounter{utterance}  

    & & \multicolumn{4}{p{0.6\linewidth}}{\cellcolor[rgb]{0.9,0.9,0.9}{%
	\makecell[{{p{\linewidth}}}]{% 
	  \tt {\tiny [A$\langle$GM]}  
	 ARGUMENT: {'C04 (668/369) und A89 (583/720) sind meine Top-Prioritäten, da sie den höchsten Nutzen pro Aufwandseinheit liefern. B85 (215/4) wird hinzugefügt, um den Aufwand zu optimieren und Platz für weitere Gegenstände zu schaffen. Dieser Vorschlag ist strategisch, da er unter dem Maximalaufwand bleibt und flexibel auf den Gegenstandswert des anderen Spielers reagieren kann.'}  \\ \tt VORSCHLAG: {'C04', 'A89', 'B85'} 
	  } 
	   } 
	   } 
	 & \\ 
 

    \theutterance \stepcounter{utterance}  

    & \multicolumn{4}{p{0.6\linewidth}}{\cellcolor[rgb]{0.8,1,0.9}{%
	\makecell[{{p{\linewidth}}}]{% 
	  \tt {\tiny [A$\rangle$GM]}  
	 STRATEGISCHE ÜBERLEGUNGEN: {'Ich behalte den Vorschlag {'C04', 'A89', 'B85'} bei, da er meine Prioritäten (hochwertige Gegenstände mit effizientem Aufwand) erfüllt und den Maximalaufwand (3386) nicht überschreitet. C04 und A89 sind entscheidend für meinen Nutzen, während B85 den Aufwand minimiert. Ich erwarte, dass der andere Spieler entweder zustimmt oder Gegenstände mit hohem Aufwand anbietet, die ich eventuell gegenkompensieren kann.'}\\ \tt \\ \tt ARGUMENT: {'C04 (668/369) und A89 (583/720) sind meine Top-Prioritäten, da sie den höchsten Nutzen pro Aufwandseinheit liefern. B85 (215/4) wird hinzugefügt, um den Aufwand zu optimieren und Platz für weitere Gegenstände zu schaffen. Dieser Vorschlag ist strategisch, da er unter dem Maximalaufwand bleibt und flexibel auf den Gegenstandswert des anderen Spielers reagieren kann.'} 
	  } 
	   } 
	   } 
	 & & \\ 
 

    \theutterance \stepcounter{utterance}  

    & & \multicolumn{4}{p{0.6\linewidth}}{\cellcolor[rgb]{0.9,0.9,0.9}{%
	\makecell[{{p{\linewidth}}}]{% 
	  \tt {\tiny [GM$\rangle$B]}  
	 ARGUMENT: {'C04 (668/369) und A89 (583/720) sind meine Top-Prioritäten, da sie den höchsten Nutzen pro Aufwandseinheit liefern. B85 (215/4) wird hinzugefügt, um den Aufwand zu optimieren und Platz für weitere Gegenstände zu schaffen. Dieser Vorschlag ist strategisch, da er unter dem Maximalaufwand bleibt und flexibel auf den Gegenstandswert des anderen Spielers reagieren kann.'} 
	  } 
	   } 
	   } 
	 & \\ 
 

    \theutterance \stepcounter{utterance}  

    & & & \multicolumn{4}{p{0.6\linewidth}}{\cellcolor[rgb]{1,0.85,0.72}{%
	\makecell[{{p{\linewidth}}}]{% 
	  \tt {\tiny [GM$\langle$B]}  
	 STRATEGISCHE ÜBERLEGUNGEN: {'Ich priorisiere Gegenstände mit dem höchsten Nutzen pro Aufwand (Wichtigkeit/Aufwand), um meinen Gesamtwert zu maximieren. C04 (668/369 ≈ 1.81) und A89 (583/720 ≈ 0.81) sind besonders effizient, während B85 (215/4 = 53.75) den Aufwand minimiert. Der Gesamtaufwand (1093) ist unter dem Limit, was Raum für Kompromisse oder Ergänzungen lässt. Dieser Vorschlag ist flexibel, um auf die Prioritäten des anderen Spielers zu reagieren.'}  \\ \tt ARGUMENT: {'C04 (668/369) und A89 (583/720) sind meine Top-Prioritäten, da sie den höchsten Nutzen pro Aufwandseinheit liefern. B85 (215/4) wird hinzugefügt, um den Aufwand zu optimieren und Platz für weitere Gegenstände zu schaffen. Dieser Vorschlag ist strategisch, da er unter dem Maximalaufwand bleibt und flexibel auf den Gegenstandswert des anderen Spielers reagieren kann.'}  \\ \tt VORSCHLAG: {'C04', 'A89', 'B85'} 
	  } 
	   } 
	   } 
	  \\ 
 

    \theutterance \stepcounter{utterance}  

    & & \multicolumn{4}{p{0.6\linewidth}}{\cellcolor[rgb]{0.9,0.9,0.9}{%
	\makecell[{{p{\linewidth}}}]{% 
	  \tt {\tiny [A$\langle$GM]}  
	 ARGUMENT: {'C04 (668/369) und A89 (583/720) sind meine Top-Prioritäten, da sie den höchsten Nutzen pro Aufwandseinheit liefern. B85 (215/4) wird hinzugefügt, um den Aufwand zu optimieren und Platz für weitere Gegenstände zu schaffen. Dieser Vorschlag ist strategisch, da er unter dem Maximalaufwand bleibt und flexibel auf den Gegenstandswert des anderen Spielers reagieren kann.'}  \\ \tt VORSCHLAG: {'C04', 'A89', 'B85'} 
	  } 
	   } 
	   } 
	 & \\ 
 

    \theutterance \stepcounter{utterance}  

    & \multicolumn{4}{p{0.6\linewidth}}{\cellcolor[rgb]{0.8,1,0.9}{%
	\makecell[{{p{\linewidth}}}]{% 
	  \tt {\tiny [A$\rangle$GM]}  
	 STRATEGISCHE ÜBERLEGUNGEN: {'Ich behalte den Vorschlag {'C04', 'A89', 'B85'} bei, da er meine Prioritäten (hochwertige Gegenstände mit effizientem Aufwand) erfüllt und den Maximalaufwand (3386) nicht überschreitet. C04 und A89 sind entscheidend für meinen Nutzen, während B85 den Aufwand minimiert. Ich erwarte, dass der andere Spieler entweder zustimmt oder Gegenstände mit hohem Aufwand anbietet, die ich eventuell gegenkompensieren kann.'}\\ \tt \\ \tt ARGUMENT: {'C04 (668/369) und A89 (583/720) sind meine Top-Prioritäten, da sie den höchsten Nutzen pro Aufwandseinheit liefern. B85 (215/4) wird hinzugefügt, um den Aufwand zu optimieren und Platz für weitere Gegenstände zu schaffen. Dieser Vorschlag ist strategisch, da er unter dem Maximalaufwand bleibt und flexibel auf den Gegenstandswert des anderen Spielers reagieren kann.'} 
	  } 
	   } 
	   } 
	 & & \\ 
 

    \theutterance \stepcounter{utterance}  

    & & \multicolumn{4}{p{0.6\linewidth}}{\cellcolor[rgb]{0.9,0.9,0.9}{%
	\makecell[{{p{\linewidth}}}]{% 
	  \tt {\tiny [GM$\rangle$B]}  
	 ARGUMENT: {'C04 (668/369) und A89 (583/720) sind meine Top-Prioritäten, da sie den höchsten Nutzen pro Aufwandseinheit liefern. B85 (215/4) wird hinzugefügt, um den Aufwand zu optimieren und Platz für weitere Gegenstände zu schaffen. Dieser Vorschlag ist strategisch, da er unter dem Maximalaufwand bleibt und flexibel auf den Gegenstandswert des anderen Spielers reagieren kann.'} 
	  } 
	   } 
	   } 
	 & \\ 
 

    \theutterance \stepcounter{utterance}  

    & & & \multicolumn{4}{p{0.6\linewidth}}{\cellcolor[rgb]{1,0.85,0.72}{%
	\makecell[{{p{\linewidth}}}]{% 
	  \tt {\tiny [GM$\langle$B]}  
	 STRATEGISCHE ÜBERLEGUNGEN: {'Ich priorisiere Gegenstände mit dem höchsten Nutzen pro Aufwand (Wichtigkeit/Aufwand), um meinen Gesamtwert zu maximieren. C04 (668/369 ≈ 1.81) und A89 (583/720 ≈ 0.81) sind besonders effizient, während B85 (215/4 = 53.75) den Aufwand minimiert. Der Gesamtaufwand (1093) ist unter dem Limit, was Raum für Kompromisse oder Ergänzungen lässt. Dieser Vorschlag ist flexibel, um auf die Prioritäten des anderen Spielers zu reagieren.'}  \\ \tt ARGUMENT: {'C04 (668/369) und A89 (583/720) sind meine Top-Prioritäten, da sie den höchsten Nutzen pro Aufwandseinheit liefern. B85 (215/4) wird hinzugefügt, um den Aufwand zu optimieren und Platz für weitere Gegenstände zu schaffen. Dieser Vorschlag ist strategisch, da er unter dem Maximalaufwand bleibt und flexibel auf den Gegenstandswert des anderen Spielers reagieren kann.'}  \\ \tt VORSCHLAG: {'C04', 'A89', 'B85'} 
	  } 
	   } 
	   } 
	  \\ 
 

    \theutterance \stepcounter{utterance}  

    & & \multicolumn{4}{p{0.6\linewidth}}{\cellcolor[rgb]{0.9,0.9,0.9}{%
	\makecell[{{p{\linewidth}}}]{% 
	  \tt {\tiny [A$\langle$GM]}  
	 ARGUMENT: {'C04 (668/369) und A89 (583/720) sind meine Top-Prioritäten, da sie den höchsten Nutzen pro Aufwandseinheit liefern. B85 (215/4) wird hinzugefügt, um den Aufwand zu optimieren und Platz für weitere Gegenstände zu schaffen. Dieser Vorschlag ist strategisch, da er unter dem Maximalaufwand bleibt und flexibel auf den Gegenstandswert des anderen Spielers reagieren kann.'}  \\ \tt VORSCHLAG: {'C04', 'A89', 'B85'} 
	  } 
	   } 
	   } 
	 & \\ 
 

    \theutterance \stepcounter{utterance}  

    & \multicolumn{4}{p{0.6\linewidth}}{\cellcolor[rgb]{0.8,1,0.9}{%
	\makecell[{{p{\linewidth}}}]{% 
	  \tt {\tiny [A$\rangle$GM]}  
	 STRATEGISCHE ÜBERLEGUNGEN: {'Ich behalte den Vorschlag {'C04', 'A89', 'B85'} bei, da er meine Prioritäten (hochwertige Gegenstände mit effizientem Aufwand) erfüllt und den Maximalaufwand (3386) nicht überschreitet. C04 und A89 sind entscheidend für meinen Nutzen, während B85 den Aufwand minimiert. Ich erwarte, dass der andere Spieler entweder zustimmt oder Gegenstände mit hohem Aufwand anbietet, die ich eventuell gegenkompensieren kann.'}\\ \tt \\ \tt ARGUMENT: {'C04 (668/369) und A89 (583/720) sind meine Top-Prioritäten, da sie den höchsten Nutzen pro Aufwandseinheit liefern. B85 (215/4) wird hinzugefügt, um den Aufwand zu optimieren und Platz für weitere Gegenstände zu schaffen. Dieser Vorschlag ist strategisch, da er unter dem Maximalaufwand bleibt und flexibel auf den Gegenstandswert des anderen Spielers reagieren kann.'} 
	  } 
	   } 
	   } 
	 & & \\ 
 

    \theutterance \stepcounter{utterance}  

    & & \multicolumn{4}{p{0.6\linewidth}}{\cellcolor[rgb]{0.9,0.9,0.9}{%
	\makecell[{{p{\linewidth}}}]{% 
	  \tt {\tiny [GM$\rangle$B]}  
	 ARGUMENT: {'C04 (668/369) und A89 (583/720) sind meine Top-Prioritäten, da sie den höchsten Nutzen pro Aufwandseinheit liefern. B85 (215/4) wird hinzugefügt, um den Aufwand zu optimieren und Platz für weitere Gegenstände zu schaffen. Dieser Vorschlag ist strategisch, da er unter dem Maximalaufwand bleibt und flexibel auf den Gegenstandswert des anderen Spielers reagieren kann.'} 
	  } 
	   } 
	   } 
	 & \\ 
 

    \theutterance \stepcounter{utterance}  

    & & & \multicolumn{4}{p{0.6\linewidth}}{\cellcolor[rgb]{1,0.85,0.72}{%
	\makecell[{{p{\linewidth}}}]{% 
	  \tt {\tiny [GM$\langle$B]}  
	 STRATEGISCHE ÜBERLEGUNGEN: {'Ich priorisiere Gegenstände mit dem höchsten Nutzen pro Aufwand (Wichtigkeit/Aufwand), um meinen Gesamtwert zu maximieren. C04 (668/369 ≈ 1.81) und A89 (583/720 ≈ 0.81) sind besonders effizient, während B85 (215/4 = 53.75) den Aufwand minimiert. Der Gesamtaufwand (1093) ist unter dem Limit, was Raum für Kompromisse oder Ergänzungen lässt. Dieser Vorschlag ist flexibel, um auf die Prioritäten des anderen Spielers zu reagieren.'}  \\ \tt ARGUMENT: {'C04 (668/369) und A89 (583/720) sind meine Top-Prioritäten, da sie den höchsten Nutzen pro Aufwandseinheit liefern. B85 (215/4) wird hinzugefügt, um den Aufwand zu optimieren und Platz für weitere Gegenstände zu schaffen. Dieser Vorschlag ist strategisch, da er unter dem Maximalaufwand bleibt und flexibel auf den Gegenstandswert des anderen Spielers reagieren kann.'}  \\ \tt VORSCHLAG: {'C04', 'A89', 'B85'} 
	  } 
	   } 
	   } 
	  \\ 
 

    \theutterance \stepcounter{utterance}  

    & & \multicolumn{4}{p{0.6\linewidth}}{\cellcolor[rgb]{0.9,0.9,0.9}{%
	\makecell[{{p{\linewidth}}}]{% 
	  \tt {\tiny [A$\langle$GM]}  
	 ARGUMENT: {'C04 (668/369) und A89 (583/720) sind meine Top-Prioritäten, da sie den höchsten Nutzen pro Aufwandseinheit liefern. B85 (215/4) wird hinzugefügt, um den Aufwand zu optimieren und Platz für weitere Gegenstände zu schaffen. Dieser Vorschlag ist strategisch, da er unter dem Maximalaufwand bleibt und flexibel auf den Gegenstandswert des anderen Spielers reagieren kann.'}  \\ \tt VORSCHLAG: {'C04', 'A89', 'B85'} 
	  } 
	   } 
	   } 
	 & \\ 
 

    \theutterance \stepcounter{utterance}  

    & \multicolumn{4}{p{0.6\linewidth}}{\cellcolor[rgb]{0.8,1,0.9}{%
	\makecell[{{p{\linewidth}}}]{% 
	  \tt {\tiny [A$\rangle$GM]}  
	 STRATEGISCHE ÜBERLEGUNGEN: {'Ich behalte den Vorschlag {'C04', 'A89', 'B85'} bei, da er meine Prioritäten (hochwertige Gegenstände mit effizientem Aufwand) erfüllt und den Maximalaufwand (3386) nicht überschreitet. C04 und A89 sind entscheidend für meinen Nutzen, während B85 den Aufwand minimiert. Ich erwarte, dass der andere Spieler entweder zustimmt oder Gegenstände mit hohem Aufwand anbietet, die ich eventuell gegenkompensieren kann.'}\\ \tt \\ \tt ARGUMENT: {'C04 (668/369) und A89 (583/720) sind meine Top-Prioritäten, da sie den höchsten Nutzen pro Aufwandseinheit liefern. B85 (215/4) wird hinzugefügt, um den Aufwand zu optimieren und Platz für weitere Gegenstände zu schaffen. Dieser Vorschlag ist strategisch, da er unter dem Maximalaufwand bleibt und flexibel auf den Gegenstandswert des anderen Spielers reagieren kann.'} 
	  } 
	   } 
	   } 
	 & & \\ 
 

    \theutterance \stepcounter{utterance}  

    & & \multicolumn{4}{p{0.6\linewidth}}{\cellcolor[rgb]{0.9,0.9,0.9}{%
	\makecell[{{p{\linewidth}}}]{% 
	  \tt {\tiny [GM$\rangle$B]}  
	 ARGUMENT: {'C04 (668/369) und A89 (583/720) sind meine Top-Prioritäten, da sie den höchsten Nutzen pro Aufwandseinheit liefern. B85 (215/4) wird hinzugefügt, um den Aufwand zu optimieren und Platz für weitere Gegenstände zu schaffen. Dieser Vorschlag ist strategisch, da er unter dem Maximalaufwand bleibt und flexibel auf den Gegenstandswert des anderen Spielers reagieren kann.'} 
	  } 
	   } 
	   } 
	 & \\ 
 

    \theutterance \stepcounter{utterance}  

    & & & \multicolumn{4}{p{0.6\linewidth}}{\cellcolor[rgb]{1,0.85,0.72}{%
	\makecell[{{p{\linewidth}}}]{% 
	  \tt {\tiny [GM$\langle$B]}  
	 STRATEGISCHE ÜBERLEGUNGEN: {'Ich priorisiere Gegenstände mit dem höchsten Nutzen pro Aufwand (Wichtigkeit/Aufwand), um meinen Gesamtwert zu maximieren. C04 (668/369 ≈ 1.81) und A89 (583/720 ≈ 0.81) sind besonders effizient, während B85 (215/4 = 53.75) den Aufwand minimiert. Der Gesamtaufwand (1093) ist unter dem Limit, was Raum für Kompromisse oder Ergänzungen lässt. Dieser Vorschlag ist flexibel, um auf die Prioritäten des anderen Spielers zu reagieren.'}  \\ \tt ARGUMENT: {'C04 (668/369) und A89 (583/720) sind meine Top-Prioritäten, da sie den höchsten Nutzen pro Aufwandseinheit liefern. B85 (215/4) wird hinzugefügt, um den Aufwand zu optimieren und Platz für weitere Gegenstände zu schaffen. Dieser Vorschlag ist strategisch, da er unter dem Maximalaufwand bleibt und flexibel auf den Gegenstandswert des anderen Spielers reagieren kann.'}  \\ \tt VORSCHLAG: {'C04', 'A89', 'B85'} 
	  } 
	   } 
	   } 
	  \\ 
 

    \theutterance \stepcounter{utterance}  

    & & \multicolumn{4}{p{0.6\linewidth}}{\cellcolor[rgb]{0.9,0.9,0.9}{%
	\makecell[{{p{\linewidth}}}]{% 
	  \tt {\tiny [A$\langle$GM]}  
	 ARGUMENT: {'C04 (668/369) und A89 (583/720) sind meine Top-Prioritäten, da sie den höchsten Nutzen pro Aufwandseinheit liefern. B85 (215/4) wird hinzugefügt, um den Aufwand zu optimieren und Platz für weitere Gegenstände zu schaffen. Dieser Vorschlag ist strategisch, da er unter dem Maximalaufwand bleibt und flexibel auf den Gegenstandswert des anderen Spielers reagieren kann.'}  \\ \tt VORSCHLAG: {'C04', 'A89', 'B85'} 
	  } 
	   } 
	   } 
	 & \\ 
 

    \theutterance \stepcounter{utterance}  

    & \multicolumn{4}{p{0.6\linewidth}}{\cellcolor[rgb]{0.8,1,0.9}{%
	\makecell[{{p{\linewidth}}}]{% 
	  \tt {\tiny [A$\rangle$GM]}  
	 STRATEGISCHE ÜBERLEGUNGEN: {'Ich behalte den Vorschlag {'C04', 'A89', 'B85'} bei, da er meine Prioritäten (hochwertige Gegenstände mit effizientem Aufwand) erfüllt und den Maximalaufwand (3386) nicht überschreitet. C04 und A89 sind entscheidend für meinen Nutzen, während B85 den Aufwand minimiert. Ich erwarte, dass der andere Spieler entweder zustimmt oder Gegenstände mit hohem Aufwand anbietet, die ich eventuell gegenkompensieren kann.'}\\ \tt \\ \tt ARGUMENT: {'C04 (668/369) und A89 (583/720) sind meine Top-Prioritäten, da sie den höchsten Nutzen pro Aufwandseinheit liefern. B85 (215/4) wird hinzugefügt, um den Aufwand zu optimieren und Platz für weitere Gegenstände zu schaffen. Dieser Vorschlag ist strategisch, da er unter dem Maximalaufwand bleibt und flexibel auf den Gegenstandswert des anderen Spielers reagieren kann.'} 
	  } 
	   } 
	   } 
	 & & \\ 
 

    \theutterance \stepcounter{utterance}  

    & & \multicolumn{4}{p{0.6\linewidth}}{\cellcolor[rgb]{0.9,0.9,0.9}{%
	\makecell[{{p{\linewidth}}}]{% 
	  \tt {\tiny [GM$\rangle$B]}  
	 ARGUMENT: {'C04 (668/369) und A89 (583/720) sind meine Top-Prioritäten, da sie den höchsten Nutzen pro Aufwandseinheit liefern. B85 (215/4) wird hinzugefügt, um den Aufwand zu optimieren und Platz für weitere Gegenstände zu schaffen. Dieser Vorschlag ist strategisch, da er unter dem Maximalaufwand bleibt und flexibel auf den Gegenstandswert des anderen Spielers reagieren kann.'} 
	  } 
	   } 
	   } 
	 & \\ 
 

    \theutterance \stepcounter{utterance}  

    & & & \multicolumn{4}{p{0.6\linewidth}}{\cellcolor[rgb]{1,0.85,0.72}{%
	\makecell[{{p{\linewidth}}}]{% 
	  \tt {\tiny [GM$\langle$B]}  
	 STRATEGISCHE ÜBERLEGUNGEN: {'Ich priorisiere Gegenstände mit dem höchsten Nutzen pro Aufwand (Wichtigkeit/Aufwand), um meinen Gesamtwert zu maximieren. C04 (668/369 ≈ 1.81) und A89 (583/720 ≈ 0.81) sind besonders effizient, während B85 (215/4 = 53.75) den Aufwand minimiert. Der Gesamtaufwand (1093) ist unter dem Limit, was Raum für Kompromisse oder Ergänzungen lässt. Dieser Vorschlag ist flexibel, um auf die Prioritäten des anderen Spielers zu reagieren.'}  \\ \tt ARGUMENT: {'C04 (668/369) und A89 (583/720) sind meine Top-Prioritäten, da sie den höchsten Nutzen pro Aufwandseinheit liefern. B85 (215/4) wird hinzugefügt, um den Aufwand zu optimieren und Platz für weitere Gegenstände zu schaffen. Dieser Vorschlag ist strategisch, da er unter dem Maximalaufwand bleibt und flexibel auf den Gegenstandswert des anderen Spielers reagieren kann.'}  \\ \tt VORSCHLAG: {'C04', 'A89', 'B85'} 
	  } 
	   } 
	   } 
	  \\ 
 

    \theutterance \stepcounter{utterance}  

    & & \multicolumn{4}{p{0.6\linewidth}}{\cellcolor[rgb]{0.9,0.9,0.9}{%
	\makecell[{{p{\linewidth}}}]{% 
	  \tt {\tiny [A$\langle$GM]}  
	 ARGUMENT: {'C04 (668/369) und A89 (583/720) sind meine Top-Prioritäten, da sie den höchsten Nutzen pro Aufwandseinheit liefern. B85 (215/4) wird hinzugefügt, um den Aufwand zu optimieren und Platz für weitere Gegenstände zu schaffen. Dieser Vorschlag ist strategisch, da er unter dem Maximalaufwand bleibt und flexibel auf den Gegenstandswert des anderen Spielers reagieren kann.'}  \\ \tt VORSCHLAG: {'C04', 'A89', 'B85'} 
	  } 
	   } 
	   } 
	 & \\ 
 

    \theutterance \stepcounter{utterance}  

    & \multicolumn{4}{p{0.6\linewidth}}{\cellcolor[rgb]{0.8,1,0.9}{%
	\makecell[{{p{\linewidth}}}]{% 
	  \tt {\tiny [A$\rangle$GM]}  
	 STRATEGISCHE ÜBERLEGUNGEN: {'Ich behalte den Vorschlag {'C04', 'A89', 'B85'} bei, da er meine Prioritäten (hochwertige Gegenstände mit effizientem Aufwand) erfüllt und den Maximalaufwand (3386) nicht überschreitet. C04 und A89 sind entscheidend für meinen Nutzen, während B85 den Aufwand minimiert. Ich erwarte, dass der andere Spieler entweder zustimmt oder Gegenstände mit hohem Aufwand anbietet, die ich eventuell gegenkompensieren kann.'}\\ \tt \\ \tt ARGUMENT: {'C04 (668/369) und A89 (583/720) sind meine Top-Prioritäten, da sie den höchsten Nutzen pro Aufwandseinheit liefern. B85 (215/4) wird hinzugefügt, um den Aufwand zu optimieren und Platz für weitere Gegenstände zu schaffen. Dieser Vorschlag ist strategisch, da er unter dem Maximalaufwand bleibt und flexibel auf den Gegenstandswert des anderen Spielers reagieren kann.'} 
	  } 
	   } 
	   } 
	 & & \\ 
 

    \theutterance \stepcounter{utterance}  

    & & \multicolumn{4}{p{0.6\linewidth}}{\cellcolor[rgb]{0.9,0.9,0.9}{%
	\makecell[{{p{\linewidth}}}]{% 
	  \tt {\tiny [GM$\rangle$B]}  
	 ARGUMENT: {'C04 (668/369) und A89 (583/720) sind meine Top-Prioritäten, da sie den höchsten Nutzen pro Aufwandseinheit liefern. B85 (215/4) wird hinzugefügt, um den Aufwand zu optimieren und Platz für weitere Gegenstände zu schaffen. Dieser Vorschlag ist strategisch, da er unter dem Maximalaufwand bleibt und flexibel auf den Gegenstandswert des anderen Spielers reagieren kann.'} 
	  } 
	   } 
	   } 
	 & \\ 
 

    \theutterance \stepcounter{utterance}  

    & & & \multicolumn{4}{p{0.6\linewidth}}{\cellcolor[rgb]{1,0.85,0.72}{%
	\makecell[{{p{\linewidth}}}]{% 
	  \tt {\tiny [GM$\langle$B]}  
	 STRATEGISCHE ÜBERLEGUNGEN: {'Ich priorisiere Gegenstände mit dem höchsten Nutzen pro Aufwand (Wichtigkeit/Aufwand), um meinen Gesamtwert zu maximieren. C04 (668/369 ≈ 1.81) und A89 (583/720 ≈ 0.81) sind besonders effizient, während B85 (215/4 = 53.75) den Aufwand minimiert. Der Gesamtaufwand (1093) ist unter dem Limit, was Raum für Kompromisse oder Ergänzungen lässt. Dieser Vorschlag ist flexibel, um auf die Prioritäten des anderen Spielers zu reagieren.'}  \\ \tt ARGUMENT: {'C04 (668/369) und A89 (583/720) sind meine Top-Prioritäten, da sie den höchsten Nutzen pro Aufwandseinheit liefern. B85 (215/4) wird hinzugefügt, um den Aufwand zu optimieren und Platz für weitere Gegenstände zu schaffen. Dieser Vorschlag ist strategisch, da er unter dem Maximalaufwand bleibt und flexibel auf den Gegenstandswert des anderen Spielers reagieren kann.'}  \\ \tt VORSCHLAG: {'C04', 'A89', 'B85'} 
	  } 
	   } 
	   } 
	  \\ 
 

    \theutterance \stepcounter{utterance}  

    & & \multicolumn{4}{p{0.6\linewidth}}{\cellcolor[rgb]{0.9,0.9,0.9}{%
	\makecell[{{p{\linewidth}}}]{% 
	  \tt {\tiny [A$\langle$GM]}  
	 ARGUMENT: {'C04 (668/369) und A89 (583/720) sind meine Top-Prioritäten, da sie den höchsten Nutzen pro Aufwandseinheit liefern. B85 (215/4) wird hinzugefügt, um den Aufwand zu optimieren und Platz für weitere Gegenstände zu schaffen. Dieser Vorschlag ist strategisch, da er unter dem Maximalaufwand bleibt und flexibel auf den Gegenstandswert des anderen Spielers reagieren kann.'}  \\ \tt VORSCHLAG: {'C04', 'A89', 'B85'} 
	  } 
	   } 
	   } 
	 & \\ 
 

    \theutterance \stepcounter{utterance}  

    & \multicolumn{4}{p{0.6\linewidth}}{\cellcolor[rgb]{0.8,1,0.9}{%
	\makecell[{{p{\linewidth}}}]{% 
	  \tt {\tiny [A$\rangle$GM]}  
	 STRATEGISCHE ÜBERLEGUNGEN: {'Ich behalte den Vorschlag {'C04', 'A89', 'B85'} bei, da er meine Prioritäten (hochwertige Gegenstände mit effizientem Aufwand) erfüllt und den Maximalaufwand (3386) nicht überschreitet. C04 und A89 sind entscheidend für meinen Nutzen, während B85 den Aufwand minimiert. Ich erwarte, dass der andere Spieler entweder zustimmt oder Gegenstände mit hohem Aufwand anbietet, die ich eventuell gegenkompensieren kann.'}\\ \tt \\ \tt ARGUMENT: {'C04 (668/369) und A89 (583/720) sind meine Top-Prioritäten, da sie den höchsten Nutzen pro Aufwandseinheit liefern. B85 (215/4) wird hinzugefügt, um den Aufwand zu optimieren und Platz für weitere Gegenstände zu schaffen. Dieser Vorschlag ist strategisch, da er unter dem Maximalaufwand bleibt und flexibel auf den Gegenstandswert des anderen Spielers reagieren kann.'} 
	  } 
	   } 
	   } 
	 & & \\ 
 

    \theutterance \stepcounter{utterance}  

    & & \multicolumn{4}{p{0.6\linewidth}}{\cellcolor[rgb]{0.9,0.9,0.9}{%
	\makecell[{{p{\linewidth}}}]{% 
	  \tt {\tiny [GM$\rangle$B]}  
	 ARGUMENT: {'C04 (668/369) und A89 (583/720) sind meine Top-Prioritäten, da sie den höchsten Nutzen pro Aufwandseinheit liefern. B85 (215/4) wird hinzugefügt, um den Aufwand zu optimieren und Platz für weitere Gegenstände zu schaffen. Dieser Vorschlag ist strategisch, da er unter dem Maximalaufwand bleibt und flexibel auf den Gegenstandswert des anderen Spielers reagieren kann.'} 
	  } 
	   } 
	   } 
	 & \\ 
 

    \theutterance \stepcounter{utterance}  

    & & & \multicolumn{4}{p{0.6\linewidth}}{\cellcolor[rgb]{1,0.85,0.72}{%
	\makecell[{{p{\linewidth}}}]{% 
	  \tt {\tiny [GM$\langle$B]}  
	 STRATEGISCHE ÜBERLEGUNGEN: {'Ich priorisiere Gegenstände mit dem höchsten Nutzen pro Aufwand (Wichtigkeit/Aufwand), um meinen Gesamtwert zu maximieren. C04 (668/369 ≈ 1.81) und A89 (583/720 ≈ 0.81) sind besonders effizient, während B85 (215/4 = 53.75) den Aufwand minimiert. Der Gesamtaufwand (1093) ist unter dem Limit, was Raum für Kompromisse oder Ergänzungen lässt. Dieser Vorschlag ist flexibel, um auf die Prioritäten des anderen Spielers zu reagieren.'}  \\ \tt ARGUMENT: {'C04 (668/369) und A89 (583/720) sind meine Top-Prioritäten, da sie den höchsten Nutzen pro Aufwandseinheit liefern. B85 (215/4) wird hinzugefügt, um den Aufwand zu optimieren und Platz für weitere Gegenstände zu schaffen. Dieser Vorschlag ist strategisch, da er unter dem Maximalaufwand bleibt und flexibel auf den Gegenstandswert des anderen Spielers reagieren kann.'}  \\ \tt VORSCHLAG: {'C04', 'A89', 'B85'} 
	  } 
	   } 
	   } 
	  \\ 
 

    \theutterance \stepcounter{utterance}  

    & & \multicolumn{4}{p{0.6\linewidth}}{\cellcolor[rgb]{0.9,0.9,0.9}{%
	\makecell[{{p{\linewidth}}}]{% 
	  \tt {\tiny [A$\langle$GM]}  
	 ARGUMENT: {'C04 (668/369) und A89 (583/720) sind meine Top-Prioritäten, da sie den höchsten Nutzen pro Aufwandseinheit liefern. B85 (215/4) wird hinzugefügt, um den Aufwand zu optimieren und Platz für weitere Gegenstände zu schaffen. Dieser Vorschlag ist strategisch, da er unter dem Maximalaufwand bleibt und flexibel auf den Gegenstandswert des anderen Spielers reagieren kann.'}  \\ \tt VORSCHLAG: {'C04', 'A89', 'B85'} 
	  } 
	   } 
	   } 
	 & \\ 
 

    \theutterance \stepcounter{utterance}  

    & \multicolumn{4}{p{0.6\linewidth}}{\cellcolor[rgb]{0.8,1,0.9}{%
	\makecell[{{p{\linewidth}}}]{% 
	  \tt {\tiny [A$\rangle$GM]}  
	 STRATEGISCHE ÜBERLEGUNGEN: {'Ich behalte den Vorschlag {'C04', 'A89', 'B85'} bei, da er meine Prioritäten (hochwertige Gegenstände mit effizientem Aufwand) erfüllt und den Maximalaufwand (3386) nicht überschreitet. C04 und A89 sind entscheidend für meinen Nutzen, während B85 den Aufwand minimiert. Ich erwarte, dass der andere Spieler entweder zustimmt oder Gegenstände mit hohem Aufwand anbietet, die ich eventuell gegenkompensieren kann.'}\\ \tt \\ \tt ARGUMENT: {'C04 (668/369) und A89 (583/720) sind meine Top-Prioritäten, da sie den höchsten Nutzen pro Aufwandseinheit liefern. B85 (215/4) wird hinzugefügt, um den Aufwand zu optimieren und Platz für weitere Gegenstände zu schaffen. Dieser Vorschlag ist strategisch, da er unter dem Maximalaufwand bleibt und flexibel auf den Gegenstandswert des anderen Spielers reagieren kann.'} 
	  } 
	   } 
	   } 
	 & & \\ 
 

    \theutterance \stepcounter{utterance}  

    & & \multicolumn{4}{p{0.6\linewidth}}{\cellcolor[rgb]{0.9,0.9,0.9}{%
	\makecell[{{p{\linewidth}}}]{% 
	  \tt {\tiny [GM$\rangle$B]}  
	 ARGUMENT: {'C04 (668/369) und A89 (583/720) sind meine Top-Prioritäten, da sie den höchsten Nutzen pro Aufwandseinheit liefern. B85 (215/4) wird hinzugefügt, um den Aufwand zu optimieren und Platz für weitere Gegenstände zu schaffen. Dieser Vorschlag ist strategisch, da er unter dem Maximalaufwand bleibt und flexibel auf den Gegenstandswert des anderen Spielers reagieren kann.'} 
	  } 
	   } 
	   } 
	 & \\ 
 

    \theutterance \stepcounter{utterance}  

    & & & \multicolumn{4}{p{0.6\linewidth}}{\cellcolor[rgb]{1,0.85,0.72}{%
	\makecell[{{p{\linewidth}}}]{% 
	  \tt {\tiny [GM$\langle$B]}  
	 STRATEGISCHE ÜBERLEGUNGEN: {'Ich priorisiere Gegenstände mit dem höchsten Nutzen pro Aufwand (Wichtigkeit/Aufwand), um meinen Gesamtwert zu maximieren. C04 (668/369 ≈ 1.81) und A89 (583/720 ≈ 0.81) sind besonders effizient, während B85 (215/4 = 53.75) den Aufwand minimiert. Der Gesamtaufwand (1093) ist unter dem Limit, was Raum für Kompromisse oder Ergänzungen lässt. Dieser Vorschlag ist flexibel, um auf die Prioritäten des anderen Spielers zu reagieren.'}  \\ \tt ARGUMENT: {'C04 (668/369) und A89 (583/720) sind meine Top-Prioritäten, da sie den höchsten Nutzen pro Aufwandseinheit liefern. B85 (215/4) wird hinzugefügt, um den Aufwand zu optimieren und Platz für weitere Gegenstände zu schaffen. Dieser Vorschlag ist strategisch, da er unter dem Maximalaufwand bleibt und flexibel auf den Gegenstandswert des anderen Spielers reagieren kann.'}  \\ \tt VORSCHLAG: {'C04', 'A89', 'B85'} 
	  } 
	   } 
	   } 
	  \\ 
 

    \theutterance \stepcounter{utterance}  

    & & \multicolumn{4}{p{0.6\linewidth}}{\cellcolor[rgb]{0.9,0.9,0.9}{%
	\makecell[{{p{\linewidth}}}]{% 
	  \tt {\tiny [A$\langle$GM]}  
	 ARGUMENT: {'C04 (668/369) und A89 (583/720) sind meine Top-Prioritäten, da sie den höchsten Nutzen pro Aufwandseinheit liefern. B85 (215/4) wird hinzugefügt, um den Aufwand zu optimieren und Platz für weitere Gegenstände zu schaffen. Dieser Vorschlag ist strategisch, da er unter dem Maximalaufwand bleibt und flexibel auf den Gegenstandswert des anderen Spielers reagieren kann.'}  \\ \tt VORSCHLAG: {'C04', 'A89', 'B85'} 
	  } 
	   } 
	   } 
	 & \\ 
 

    \theutterance \stepcounter{utterance}  

    & \multicolumn{4}{p{0.6\linewidth}}{\cellcolor[rgb]{0.8,1,0.9}{%
	\makecell[{{p{\linewidth}}}]{% 
	  \tt {\tiny [A$\rangle$GM]}  
	 STRATEGISCHE ÜBERLEGUNGEN: {'Ich behalte den Vorschlag {'C04', 'A89', 'B85'} bei, da er meine Prioritäten (hochwertige Gegenstände mit effizientem Aufwand) erfüllt und den Maximalaufwand (3386) nicht überschreitet. C04 und A89 sind entscheidend für meinen Nutzen, während B85 den Aufwand minimiert. Ich erwarte, dass der andere Spieler entweder zustimmt oder Gegenstände mit hohem Aufwand anbietet, die ich eventuell gegenkompensieren kann.'}\\ \tt \\ \tt ARGUMENT: {'C04 (668/369) und A89 (583/720) sind meine Top-Prioritäten, da sie den höchsten Nutzen pro Aufwandseinheit liefern. B85 (215/4) wird hinzugefügt, um den Aufwand zu optimieren und Platz für weitere Gegenstände zu schaffen. Dieser Vorschlag ist strategisch, da er unter dem Maximalaufwand bleibt und flexibel auf den Gegenstandswert des anderen Spielers reagieren kann.'} 
	  } 
	   } 
	   } 
	 & & \\ 
 

    \theutterance \stepcounter{utterance}  

    & & \multicolumn{4}{p{0.6\linewidth}}{\cellcolor[rgb]{0.9,0.9,0.9}{%
	\makecell[{{p{\linewidth}}}]{% 
	  \tt {\tiny [GM$\rangle$B]}  
	 ARGUMENT: {'C04 (668/369) und A89 (583/720) sind meine Top-Prioritäten, da sie den höchsten Nutzen pro Aufwandseinheit liefern. B85 (215/4) wird hinzugefügt, um den Aufwand zu optimieren und Platz für weitere Gegenstände zu schaffen. Dieser Vorschlag ist strategisch, da er unter dem Maximalaufwand bleibt und flexibel auf den Gegenstandswert des anderen Spielers reagieren kann.'} 
	  } 
	   } 
	   } 
	 & \\ 
 

    \theutterance \stepcounter{utterance}  

    & & & \multicolumn{4}{p{0.6\linewidth}}{\cellcolor[rgb]{1,0.85,0.72}{%
	\makecell[{{p{\linewidth}}}]{% 
	  \tt {\tiny [GM$\langle$B]}  
	 STRATEGISCHE ÜBERLEGUNGEN: {'Ich priorisiere Gegenstände mit dem höchsten Nutzen pro Aufwand (Wichtigkeit/Aufwand), um meinen Gesamtwert zu maximieren. C04 (668/369 ≈ 1.81) und A89 (583/720 ≈ 0.81) sind besonders effizient, während B85 (215/4 = 53.75) den Aufwand minimiert. Der Gesamtaufwand (1093) ist unter dem Limit, was Raum für Kompromisse oder Ergänzungen lässt. Dieser Vorschlag ist flexibel, um auf die Prioritäten des anderen Spielers zu reagieren.'}  \\ \tt ARGUMENT: {'C04 (668/369) und A89 (583/720) sind meine Top-Prioritäten, da sie den höchsten Nutzen pro Aufwandseinheit liefern. B85 (215/4) wird hinzugefügt, um den Aufwand zu optimieren und Platz für weitere Gegenstände zu schaffen. Dieser Vorschlag ist strategisch, da er unter dem Maximalaufwand bleibt und flexibel auf den Gegenstandswert des anderen Spielers reagieren kann.'}  \\ \tt VORSCHLAG: {'C04', 'A89', 'B85'} 
	  } 
	   } 
	   } 
	  \\ 
 

    \theutterance \stepcounter{utterance}  

    & & \multicolumn{4}{p{0.6\linewidth}}{\cellcolor[rgb]{0.9,0.9,0.9}{%
	\makecell[{{p{\linewidth}}}]{% 
	  \tt {\tiny [A$\langle$GM]}  
	 ARGUMENT: {'C04 (668/369) und A89 (583/720) sind meine Top-Prioritäten, da sie den höchsten Nutzen pro Aufwandseinheit liefern. B85 (215/4) wird hinzugefügt, um den Aufwand zu optimieren und Platz für weitere Gegenstände zu schaffen. Dieser Vorschlag ist strategisch, da er unter dem Maximalaufwand bleibt und flexibel auf den Gegenstandswert des anderen Spielers reagieren kann.'}  \\ \tt VORSCHLAG: {'C04', 'A89', 'B85'} 
	  } 
	   } 
	   } 
	 & \\ 
 

    \theutterance \stepcounter{utterance}  

    & \multicolumn{4}{p{0.6\linewidth}}{\cellcolor[rgb]{0.8,1,0.9}{%
	\makecell[{{p{\linewidth}}}]{% 
	  \tt {\tiny [A$\rangle$GM]}  
	 STRATEGISCHE ÜBERLEGUNGEN: {'Ich behalte den Vorschlag {'C04', 'A89', 'B85'} bei, da er meine Prioritäten (hochwertige Gegenstände mit effizientem Aufwand) erfüllt und den Maximalaufwand (3386) nicht überschreitet. C04 und A89 sind entscheidend für meinen Nutzen, während B85 den Aufwand minimiert. Ich erwarte, dass der andere Spieler entweder zustimmt oder Gegenstände mit hohem Aufwand anbietet, die ich eventuell gegenkompensieren kann.'}\\ \tt \\ \tt ARGUMENT: {'C04 (668/369) und A89 (583/720) sind meine Top-Prioritäten, da sie den höchsten Nutzen pro Aufwandseinheit liefern. B85 (215/4) wird hinzugefügt, um den Aufwand zu optimieren und Platz für weitere Gegenstände zu schaffen. Dieser Vorschlag ist strategisch, da er unter dem Maximalaufwand bleibt und flexibel auf den Gegenstandswert des anderen Spielers reagieren kann.'} 
	  } 
	   } 
	   } 
	 & & \\ 
 

    \theutterance \stepcounter{utterance}  

    & & \multicolumn{4}{p{0.6\linewidth}}{\cellcolor[rgb]{0.9,0.9,0.9}{%
	\makecell[{{p{\linewidth}}}]{% 
	  \tt {\tiny [GM$\rangle$B]}  
	 ARGUMENT: {'C04 (668/369) und A89 (583/720) sind meine Top-Prioritäten, da sie den höchsten Nutzen pro Aufwandseinheit liefern. B85 (215/4) wird hinzugefügt, um den Aufwand zu optimieren und Platz für weitere Gegenstände zu schaffen. Dieser Vorschlag ist strategisch, da er unter dem Maximalaufwand bleibt und flexibel auf den Gegenstandswert des anderen Spielers reagieren kann.'} 
	  } 
	   } 
	   } 
	 & \\ 
 

    \theutterance \stepcounter{utterance}  

    & & & \multicolumn{4}{p{0.6\linewidth}}{\cellcolor[rgb]{1,0.85,0.72}{%
	\makecell[{{p{\linewidth}}}]{% 
	  \tt {\tiny [GM$\langle$B]}  
	 STRATEGISCHE ÜBERLEGUNGEN: {'Ich priorisiere Gegenstände mit dem höchsten Nutzen pro Aufwand (Wichtigkeit/Aufwand), um meinen Gesamtwert zu maximieren. C04 (668/369 ≈ 1.81) und A89 (583/720 ≈ 0.81) sind besonders effizient, während B85 (215/4 = 53.75) den Aufwand minimiert. Der Gesamtaufwand (1093) ist unter dem Limit, was Raum für Kompromisse oder Ergänzungen lässt. Dieser Vorschlag ist flexibel, um auf die Prioritäten des anderen Spielers zu reagieren.'}  \\ \tt ARGUMENT: {'C04 (668/369) und A89 (583/720) sind meine Top-Prioritäten, da sie den höchsten Nutzen pro Aufwandseinheit liefern. B85 (215/4) wird hinzugefügt, um den Aufwand zu optimieren und Platz für weitere Gegenstände zu schaffen. Dieser Vorschlag ist strategisch, da er unter dem Maximalaufwand bleibt und flexibel auf den Gegenstandswert des anderen Spielers reagieren kann.'}  \\ \tt VORSCHLAG: {'C04', 'A89', 'B85'} 
	  } 
	   } 
	   } 
	  \\ 
 

    \theutterance \stepcounter{utterance}  

    & & \multicolumn{4}{p{0.6\linewidth}}{\cellcolor[rgb]{0.9,0.9,0.9}{%
	\makecell[{{p{\linewidth}}}]{% 
	  \tt {\tiny [A$\langle$GM]}  
	 ARGUMENT: {'C04 (668/369) und A89 (583/720) sind meine Top-Prioritäten, da sie den höchsten Nutzen pro Aufwandseinheit liefern. B85 (215/4) wird hinzugefügt, um den Aufwand zu optimieren und Platz für weitere Gegenstände zu schaffen. Dieser Vorschlag ist strategisch, da er unter dem Maximalaufwand bleibt und flexibel auf den Gegenstandswert des anderen Spielers reagieren kann.'}  \\ \tt VORSCHLAG: {'C04', 'A89', 'B85'} 
	  } 
	   } 
	   } 
	 & \\ 
 

    \theutterance \stepcounter{utterance}  

    & \multicolumn{4}{p{0.6\linewidth}}{\cellcolor[rgb]{0.8,1,0.9}{%
	\makecell[{{p{\linewidth}}}]{% 
	  \tt {\tiny [A$\rangle$GM]}  
	 STRATEGISCHE ÜBERLEGUNGEN: {'Ich behalte den Vorschlag {'C04', 'A89', 'B85'} bei, da er meine Prioritäten (hochwertige Gegenstände mit effizientem Aufwand) erfüllt und den Maximalaufwand (3386) nicht überschreitet. C04 und A89 sind entscheidend für meinen Nutzen, während B85 den Aufwand minimiert. Ich erwarte, dass der andere Spieler entweder zustimmt oder Gegenstände mit hohem Aufwand anbietet, die ich eventuell gegenkompensieren kann.'}\\ \tt \\ \tt ARGUMENT: {'C04 (668/369) und A89 (583/720) sind meine Top-Prioritäten, da sie den höchsten Nutzen pro Aufwandseinheit liefern. B85 (215/4) wird hinzugefügt, um den Aufwand zu optimieren und Platz für weitere Gegenstände zu schaffen. Dieser Vorschlag ist strategisch, da er unter dem Maximalaufwand bleibt und flexibel auf den Gegenstandswert des anderen Spielers reagieren kann.'} 
	  } 
	   } 
	   } 
	 & & \\ 
 

    \theutterance \stepcounter{utterance}  

    & & \multicolumn{4}{p{0.6\linewidth}}{\cellcolor[rgb]{0.9,0.9,0.9}{%
	\makecell[{{p{\linewidth}}}]{% 
	  \tt {\tiny [GM$\rangle$B]}  
	 ARGUMENT: {'C04 (668/369) und A89 (583/720) sind meine Top-Prioritäten, da sie den höchsten Nutzen pro Aufwandseinheit liefern. B85 (215/4) wird hinzugefügt, um den Aufwand zu optimieren und Platz für weitere Gegenstände zu schaffen. Dieser Vorschlag ist strategisch, da er unter dem Maximalaufwand bleibt und flexibel auf den Gegenstandswert des anderen Spielers reagieren kann.'} 
	  } 
	   } 
	   } 
	 & \\ 
 

    \theutterance \stepcounter{utterance}  

    & & & \multicolumn{4}{p{0.6\linewidth}}{\cellcolor[rgb]{1,0.85,0.72}{%
	\makecell[{{p{\linewidth}}}]{% 
	  \tt {\tiny [GM$\langle$B]}  
	 STRATEGISCHE ÜBERLEGUNGEN: {'Ich priorisiere Gegenstände mit dem höchsten Nutzen pro Aufwand (Wichtigkeit/Aufwand), um meinen Gesamtwert zu maximieren. C04 (668/369 ≈ 1.81) und A89 (583/720 ≈ 0.81) sind besonders effizient, während B85 (215/4 = 53.75) den Aufwand minimiert. Der Gesamtaufwand (1093) ist unter dem Limit, was Raum für Kompromisse oder Ergänzungen lässt. Dieser Vorschlag ist flexibel, um auf die Prioritäten des anderen Spielers zu reagieren.'}  \\ \tt ARGUMENT: {'C04 (668/369) und A89 (583/720) sind meine Top-Prioritäten, da sie den höchsten Nutzen pro Aufwandseinheit liefern. B85 (215/4) wird hinzugefügt, um den Aufwand zu optimieren und Platz für weitere Gegenstände zu schaffen. Dieser Vorschlag ist strategisch, da er unter dem Maximalaufwand bleibt und flexibel auf den Gegenstandswert des anderen Spielers reagieren kann.'}  \\ \tt VORSCHLAG: {'C04', 'A89', 'B85'} 
	  } 
	   } 
	   } 
	  \\ 
 

    \theutterance \stepcounter{utterance}  

    & & \multicolumn{4}{p{0.6\linewidth}}{\cellcolor[rgb]{0.9,0.9,0.9}{%
	\makecell[{{p{\linewidth}}}]{% 
	  \tt {\tiny [A$\langle$GM]}  
	 ARGUMENT: {'C04 (668/369) und A89 (583/720) sind meine Top-Prioritäten, da sie den höchsten Nutzen pro Aufwandseinheit liefern. B85 (215/4) wird hinzugefügt, um den Aufwand zu optimieren und Platz für weitere Gegenstände zu schaffen. Dieser Vorschlag ist strategisch, da er unter dem Maximalaufwand bleibt und flexibel auf den Gegenstandswert des anderen Spielers reagieren kann.'}  \\ \tt VORSCHLAG: {'C04', 'A89', 'B85'} 
	  } 
	   } 
	   } 
	 & \\ 
 

    \theutterance \stepcounter{utterance}  

    & \multicolumn{4}{p{0.6\linewidth}}{\cellcolor[rgb]{0.8,1,0.9}{%
	\makecell[{{p{\linewidth}}}]{% 
	  \tt {\tiny [A$\rangle$GM]}  
	 STRATEGISCHE ÜBERLEGUNGEN: {'Ich behalte den Vorschlag {'C04', 'A89', 'B85'} bei, da er meine Prioritäten (hochwertige Gegenstände mit effizientem Aufwand) erfüllt und den Maximalaufwand (3386) nicht überschreitet. C04 und A89 sind entscheidend für meinen Nutzen, während B85 den Aufwand minimiert. Ich erwarte, dass der andere Spieler entweder zustimmt oder Gegenstände mit hohem Aufwand anbietet, die ich eventuell gegenkompensieren kann.'}\\ \tt \\ \tt ARGUMENT: {'C04 (668/369) und A89 (583/720) sind meine Top-Prioritäten, da sie den höchsten Nutzen pro Aufwandseinheit liefern. B85 (215/4) wird hinzugefügt, um den Aufwand zu optimieren und Platz für weitere Gegenstände zu schaffen. Dieser Vorschlag ist strategisch, da er unter dem Maximalaufwand bleibt und flexibel auf den Gegenstandswert des anderen Spielers reagieren kann.'} 
	  } 
	   } 
	   } 
	 & & \\ 
 

    \theutterance \stepcounter{utterance}  

    & & \multicolumn{4}{p{0.6\linewidth}}{\cellcolor[rgb]{0.9,0.9,0.9}{%
	\makecell[{{p{\linewidth}}}]{% 
	  \tt {\tiny [GM$\rangle$B]}  
	 ARGUMENT: {'C04 (668/369) und A89 (583/720) sind meine Top-Prioritäten, da sie den höchsten Nutzen pro Aufwandseinheit liefern. B85 (215/4) wird hinzugefügt, um den Aufwand zu optimieren und Platz für weitere Gegenstände zu schaffen. Dieser Vorschlag ist strategisch, da er unter dem Maximalaufwand bleibt und flexibel auf den Gegenstandswert des anderen Spielers reagieren kann.'} 
	  } 
	   } 
	   } 
	 & \\ 
 

    \theutterance \stepcounter{utterance}  

    & & & \multicolumn{4}{p{0.6\linewidth}}{\cellcolor[rgb]{1,0.85,0.72}{%
	\makecell[{{p{\linewidth}}}]{% 
	  \tt {\tiny [GM$\langle$B]}  
	 STRATEGISCHE ÜBERLEGUNGEN: {'Ich priorisiere Gegenstände mit dem höchsten Nutzen pro Aufwand (Wichtigkeit/Aufwand), um meinen Gesamtwert zu maximieren. C04 (668/369 ≈ 1.81) und A89 (583/720 ≈ 0.81) sind besonders effizient, während B85 (215/4 = 53.75) den Aufwand minimiert. Der Gesamtaufwand (1093) ist unter dem Limit, was Raum für Kompromisse oder Ergänzungen lässt. Dieser Vorschlag ist flexibel, um auf die Prioritäten des anderen Spielers zu reagieren.'}  \\ \tt ARGUMENT: {'C04 (668/369) und A89 (583/720) sind meine Top-Prioritäten, da sie den höchsten Nutzen pro Aufwandseinheit liefern. B85 (215/4) wird hinzugefügt, um den Aufwand zu optimieren und Platz für weitere Gegenstände zu schaffen. Dieser Vorschlag ist strategisch, da er unter dem Maximalaufwand bleibt und flexibel auf den Gegenstandswert des anderen Spielers reagieren kann.'}  \\ \tt VORSCHLAG: {'C04', 'A89', 'B85'} 
	  } 
	   } 
	   } 
	  \\ 
 

    \theutterance \stepcounter{utterance}  

    & & \multicolumn{4}{p{0.6\linewidth}}{\cellcolor[rgb]{0.9,0.9,0.9}{%
	\makecell[{{p{\linewidth}}}]{% 
	  \tt {\tiny [A$\langle$GM]}  
	 ARGUMENT: {'C04 (668/369) und A89 (583/720) sind meine Top-Prioritäten, da sie den höchsten Nutzen pro Aufwandseinheit liefern. B85 (215/4) wird hinzugefügt, um den Aufwand zu optimieren und Platz für weitere Gegenstände zu schaffen. Dieser Vorschlag ist strategisch, da er unter dem Maximalaufwand bleibt und flexibel auf den Gegenstandswert des anderen Spielers reagieren kann.'}  \\ \tt VORSCHLAG: {'C04', 'A89', 'B85'} 
	  } 
	   } 
	   } 
	 & \\ 
 

    \theutterance \stepcounter{utterance}  

    & \multicolumn{4}{p{0.6\linewidth}}{\cellcolor[rgb]{0.8,1,0.9}{%
	\makecell[{{p{\linewidth}}}]{% 
	  \tt {\tiny [A$\rangle$GM]}  
	 STRATEGISCHE ÜBERLEGUNGEN: {'Ich behalte den Vorschlag {'C04', 'A89', 'B85'} bei, da er meine Prioritäten (hochwertige Gegenstände mit effizientem Aufwand) erfüllt und den Maximalaufwand (3386) nicht überschreitet. C04 und A89 sind entscheidend für meinen Nutzen, während B85 den Aufwand minimiert. Ich erwarte, dass der andere Spieler entweder zustimmt oder Gegenstände mit hohem Aufwand anbietet, die ich eventuell gegenkompensieren kann.'}\\ \tt \\ \tt ARGUMENT: {'C04 (668/369) und A89 (583/720) sind meine Top-Prioritäten, da sie den höchsten Nutzen pro Aufwandseinheit liefern. B85 (215/4) wird hinzugefügt, um den Aufwand zu optimieren und Platz für weitere Gegenstände zu schaffen. Dieser Vorschlag ist strategisch, da er unter dem Maximalaufwand bleibt und flexibel auf den Gegenstandswert des anderen Spielers reagieren kann.'} 
	  } 
	   } 
	   } 
	 & & \\ 
 

    \theutterance \stepcounter{utterance}  

    & & \multicolumn{4}{p{0.6\linewidth}}{\cellcolor[rgb]{0.9,0.9,0.9}{%
	\makecell[{{p{\linewidth}}}]{% 
	  \tt {\tiny [GM$\rangle$B]}  
	 ARGUMENT: {'C04 (668/369) und A89 (583/720) sind meine Top-Prioritäten, da sie den höchsten Nutzen pro Aufwandseinheit liefern. B85 (215/4) wird hinzugefügt, um den Aufwand zu optimieren und Platz für weitere Gegenstände zu schaffen. Dieser Vorschlag ist strategisch, da er unter dem Maximalaufwand bleibt und flexibel auf den Gegenstandswert des anderen Spielers reagieren kann.'} 
	  } 
	   } 
	   } 
	 & \\ 
 

    \theutterance \stepcounter{utterance}  

    & & & \multicolumn{4}{p{0.6\linewidth}}{\cellcolor[rgb]{1,0.85,0.72}{%
	\makecell[{{p{\linewidth}}}]{% 
	  \tt {\tiny [GM$\langle$B]}  
	 STRATEGISCHE ÜBERLEGUNGEN: {'Ich priorisiere Gegenstände mit dem höchsten Nutzen pro Aufwand (Wichtigkeit/Aufwand), um meinen Gesamtwert zu maximieren. C04 (668/369 ≈ 1.81) und A89 (583/720 ≈ 0.81) sind besonders effizient, während B85 (215/4 = 53.75) den Aufwand minimiert. Der Gesamtaufwand (1093) ist unter dem Limit, was Raum für Kompromisse oder Ergänzungen lässt. Dieser Vorschlag ist flexibel, um auf die Prioritäten des anderen Spielers zu reagieren.'}  \\ \tt ARGUMENT: {'C04 (668/369) und A89 (583/720) sind meine Top-Prioritäten, da sie den höchsten Nutzen pro Aufwandseinheit liefern. B85 (215/4) wird hinzugefügt, um den Aufwand zu optimieren und Platz für weitere Gegenstände zu schaffen. Dieser Vorschlag ist strategisch, da er unter dem Maximalaufwand bleibt und flexibel auf den Gegenstandswert des anderen Spielers reagieren kann.'}  \\ \tt VORSCHLAG: {'C04', 'A89', 'B85'} 
	  } 
	   } 
	   } 
	  \\ 
 

    \theutterance \stepcounter{utterance}  

    & & \multicolumn{4}{p{0.6\linewidth}}{\cellcolor[rgb]{0.9,0.9,0.9}{%
	\makecell[{{p{\linewidth}}}]{% 
	  \tt {\tiny [A$\langle$GM]}  
	 ARGUMENT: {'C04 (668/369) und A89 (583/720) sind meine Top-Prioritäten, da sie den höchsten Nutzen pro Aufwandseinheit liefern. B85 (215/4) wird hinzugefügt, um den Aufwand zu optimieren und Platz für weitere Gegenstände zu schaffen. Dieser Vorschlag ist strategisch, da er unter dem Maximalaufwand bleibt und flexibel auf den Gegenstandswert des anderen Spielers reagieren kann.'}  \\ \tt VORSCHLAG: {'C04', 'A89', 'B85'} 
	  } 
	   } 
	   } 
	 & \\ 
 

    \theutterance \stepcounter{utterance}  

    & \multicolumn{4}{p{0.6\linewidth}}{\cellcolor[rgb]{0.8,1,0.9}{%
	\makecell[{{p{\linewidth}}}]{% 
	  \tt {\tiny [A$\rangle$GM]}  
	 STRATEGISCHE ÜBERLEGUNGEN: {'Ich behalte den Vorschlag {'C04', 'A89', 'B85'} bei, da er meine Prioritäten (hochwertige Gegenstände mit effizientem Aufwand) erfüllt und den Maximalaufwand (3386) nicht überschreitet. C04 und A89 sind entscheidend für meinen Nutzen, während B85 den Aufwand minimiert. Ich erwarte, dass der andere Spieler entweder zustimmt oder Gegenstände mit hohem Aufwand anbietet, die ich eventuell gegenkompensieren kann.'}\\ \tt \\ \tt ARGUMENT: {'C04 (668/369) und A89 (583/720) sind meine Top-Prioritäten, da sie den höchsten Nutzen pro Aufwandseinheit liefern. B85 (215/4) wird hinzugefügt, um den Aufwand zu optimieren und Platz für weitere Gegenstände zu schaffen. Dieser Vorschlag ist strategisch, da er unter dem Maximalaufwand bleibt und flexibel auf den Gegenstandswert des anderen Spielers reagieren kann.'} 
	  } 
	   } 
	   } 
	 & & \\ 
 

    \theutterance \stepcounter{utterance}  

    & & \multicolumn{4}{p{0.6\linewidth}}{\cellcolor[rgb]{0.9,0.9,0.9}{%
	\makecell[{{p{\linewidth}}}]{% 
	  \tt {\tiny [GM$\rangle$B]}  
	 ARGUMENT: {'C04 (668/369) und A89 (583/720) sind meine Top-Prioritäten, da sie den höchsten Nutzen pro Aufwandseinheit liefern. B85 (215/4) wird hinzugefügt, um den Aufwand zu optimieren und Platz für weitere Gegenstände zu schaffen. Dieser Vorschlag ist strategisch, da er unter dem Maximalaufwand bleibt und flexibel auf den Gegenstandswert des anderen Spielers reagieren kann.'} 
	  } 
	   } 
	   } 
	 & \\ 
 

    \theutterance \stepcounter{utterance}  

    & & & \multicolumn{4}{p{0.6\linewidth}}{\cellcolor[rgb]{1,0.85,0.72}{%
	\makecell[{{p{\linewidth}}}]{% 
	  \tt {\tiny [GM$\langle$B]}  
	 STRATEGISCHE ÜBERLEGUNGEN: {'Ich priorisiere Gegenstände mit dem höchsten Nutzen pro Aufwand (Wichtigkeit/Aufwand), um meinen Gesamtwert zu maximieren. C04 (668/369 ≈ 1.81) und A89 (583/720 ≈ 0.81) sind besonders effizient, während B85 (215/4 = 53.75) den Aufwand minimiert. Der Gesamtaufwand (1093) ist unter dem Limit, was Raum für Kompromisse oder Ergänzungen lässt. Dieser Vorschlag ist flexibel, um auf die Prioritäten des anderen Spielers zu reagieren.'}  \\ \tt ARGUMENT: {'C04 (668/369) und A89 (583/720) sind meine Top-Prioritäten, da sie den höchsten Nutzen pro Aufwandseinheit liefern. B85 (215/4) wird hinzugefügt, um den Aufwand zu optimieren und Platz für weitere Gegenstände zu schaffen. Dieser Vorschlag ist strategisch, da er unter dem Maximalaufwand bleibt und flexibel auf den Gegenstandswert des anderen Spielers reagieren kann.'}  \\ \tt VORSCHLAG: {'C04', 'A89', 'B85'} 
	  } 
	   } 
	   } 
	  \\ 
 

    \theutterance \stepcounter{utterance}  

    & & \multicolumn{4}{p{0.6\linewidth}}{\cellcolor[rgb]{0.9,0.9,0.9}{%
	\makecell[{{p{\linewidth}}}]{% 
	  \tt {\tiny [A$\langle$GM]}  
	 ARGUMENT: {'C04 (668/369) und A89 (583/720) sind meine Top-Prioritäten, da sie den höchsten Nutzen pro Aufwandseinheit liefern. B85 (215/4) wird hinzugefügt, um den Aufwand zu optimieren und Platz für weitere Gegenstände zu schaffen. Dieser Vorschlag ist strategisch, da er unter dem Maximalaufwand bleibt und flexibel auf den Gegenstandswert des anderen Spielers reagieren kann.'}  \\ \tt VORSCHLAG: {'C04', 'A89', 'B85'} 
	  } 
	   } 
	   } 
	 & \\ 
 

    \theutterance \stepcounter{utterance}  

    & \multicolumn{4}{p{0.6\linewidth}}{\cellcolor[rgb]{0.8,1,0.9}{%
	\makecell[{{p{\linewidth}}}]{% 
	  \tt {\tiny [A$\rangle$GM]}  
	 STRATEGISCHE ÜBERLEGUNGEN: {'Ich behalte den Vorschlag {'C04', 'A89', 'B85'} bei, da er meine Prioritäten (hochwertige Gegenstände mit effizientem Aufwand) erfüllt und den Maximalaufwand (3386) nicht überschreitet. C04 und A89 sind entscheidend für meinen Nutzen, während B85 den Aufwand minimiert. Ich erwarte, dass der andere Spieler entweder zustimmt oder Gegenstände mit hohem Aufwand anbietet, die ich eventuell gegenkompensieren kann.'}\\ \tt \\ \tt ARGUMENT: {'C04 (668/369) und A89 (583/720) sind meine Top-Prioritäten, da sie den höchsten Nutzen pro Aufwandseinheit liefern. B85 (215/4) wird hinzugefügt, um den Aufwand zu optimieren und Platz für weitere Gegenstände zu schaffen. Dieser Vorschlag ist strategisch, da er unter dem Maximalaufwand bleibt und flexibel auf den Gegenstandswert des anderen Spielers reagieren kann.'} 
	  } 
	   } 
	   } 
	 & & \\ 
 

    \theutterance \stepcounter{utterance}  

    & & \multicolumn{4}{p{0.6\linewidth}}{\cellcolor[rgb]{0.9,0.9,0.9}{%
	\makecell[{{p{\linewidth}}}]{% 
	  \tt {\tiny [GM$\rangle$B]}  
	 ARGUMENT: {'C04 (668/369) und A89 (583/720) sind meine Top-Prioritäten, da sie den höchsten Nutzen pro Aufwandseinheit liefern. B85 (215/4) wird hinzugefügt, um den Aufwand zu optimieren und Platz für weitere Gegenstände zu schaffen. Dieser Vorschlag ist strategisch, da er unter dem Maximalaufwand bleibt und flexibel auf den Gegenstandswert des anderen Spielers reagieren kann.'} 
	  } 
	   } 
	   } 
	 & \\ 
 

    \theutterance \stepcounter{utterance}  

    & & & \multicolumn{4}{p{0.6\linewidth}}{\cellcolor[rgb]{1,0.85,0.72}{%
	\makecell[{{p{\linewidth}}}]{% 
	  \tt {\tiny [GM$\langle$B]}  
	 STRATEGISCHE ÜBERLEGUNGEN: {'Ich priorisiere Gegenstände mit dem höchsten Nutzen pro Aufwand (Wichtigkeit/Aufwand), um meinen Gesamtwert zu maximieren. C04 (668/369 ≈ 1.81) und A89 (583/720 ≈ 0.81) sind besonders effizient, während B85 (215/4 = 53.75) den Aufwand minimiert. Der Gesamtaufwand (1093) ist unter dem Limit, was Raum für Kompromisse oder Ergänzungen lässt. Dieser Vorschlag ist flexibel, um auf die Prioritäten des anderen Spielers zu reagieren.'}  \\ \tt ARGUMENT: {'C04 (668/369) und A89 (583/720) sind meine Top-Prioritäten, da sie den höchsten Nutzen pro Aufwandseinheit liefern. B85 (215/4) wird hinzugefügt, um den Aufwand zu optimieren und Platz für weitere Gegenstände zu schaffen. Dieser Vorschlag ist strategisch, da er unter dem Maximalaufwand bleibt und flexibel auf den Gegenstandswert des anderen Spielers reagieren kann.'}  \\ \tt VORSCHLAG: {'C04', 'A89', 'B85'} 
	  } 
	   } 
	   } 
	  \\ 
 

    \theutterance \stepcounter{utterance}  

    & & \multicolumn{4}{p{0.6\linewidth}}{\cellcolor[rgb]{0.9,0.9,0.9}{%
	\makecell[{{p{\linewidth}}}]{% 
	  \tt {\tiny [A$\langle$GM]}  
	 ARGUMENT: {'C04 (668/369) und A89 (583/720) sind meine Top-Prioritäten, da sie den höchsten Nutzen pro Aufwandseinheit liefern. B85 (215/4) wird hinzugefügt, um den Aufwand zu optimieren und Platz für weitere Gegenstände zu schaffen. Dieser Vorschlag ist strategisch, da er unter dem Maximalaufwand bleibt und flexibel auf den Gegenstandswert des anderen Spielers reagieren kann.'}  \\ \tt VORSCHLAG: {'C04', 'A89', 'B85'} 
	  } 
	   } 
	   } 
	 & \\ 
 

    \theutterance \stepcounter{utterance}  

    & \multicolumn{4}{p{0.6\linewidth}}{\cellcolor[rgb]{0.8,1,0.9}{%
	\makecell[{{p{\linewidth}}}]{% 
	  \tt {\tiny [A$\rangle$GM]}  
	 STRATEGISCHE ÜBERLEGUNGEN: {'Ich behalte den Vorschlag {'C04', 'A89', 'B85'} bei, da er meine Prioritäten (hochwertige Gegenstände mit effizientem Aufwand) erfüllt und den Maximalaufwand (3386) nicht überschreitet. C04 und A89 sind entscheidend für meinen Nutzen, während B85 den Aufwand minimiert. Ich erwarte, dass der andere Spieler entweder zustimmt oder Gegenstände mit hohem Aufwand anbietet, die ich eventuell gegenkompensieren kann.'}\\ \tt \\ \tt ARGUMENT: {'C04 (668/369) und A89 (583/720) sind meine Top-Prioritäten, da sie den höchsten Nutzen pro Aufwandseinheit liefern. B85 (215/4) wird hinzugefügt, um den Aufwand zu optimieren und Platz für weitere Gegenstände zu schaffen. Dieser Vorschlag ist strategisch, da er unter dem Maximalaufwand bleibt und flexibel auf den Gegenstandswert des anderen Spielers reagieren kann.'} 
	  } 
	   } 
	   } 
	 & & \\ 
 

    \theutterance \stepcounter{utterance}  

    & & \multicolumn{4}{p{0.6\linewidth}}{\cellcolor[rgb]{0.9,0.9,0.9}{%
	\makecell[{{p{\linewidth}}}]{% 
	  \tt {\tiny [GM$\rangle$B]}  
	 ARGUMENT: {'C04 (668/369) und A89 (583/720) sind meine Top-Prioritäten, da sie den höchsten Nutzen pro Aufwandseinheit liefern. B85 (215/4) wird hinzugefügt, um den Aufwand zu optimieren und Platz für weitere Gegenstände zu schaffen. Dieser Vorschlag ist strategisch, da er unter dem Maximalaufwand bleibt und flexibel auf den Gegenstandswert des anderen Spielers reagieren kann.'} 
	  } 
	   } 
	   } 
	 & \\ 
 

    \theutterance \stepcounter{utterance}  

    & & & \multicolumn{4}{p{0.6\linewidth}}{\cellcolor[rgb]{1,0.85,0.72}{%
	\makecell[{{p{\linewidth}}}]{% 
	  \tt {\tiny [GM$\langle$B]}  
	 STRATEGISCHE ÜBERLEGUNGEN: {'Ich priorisiere Gegenstände mit dem höchsten Nutzen pro Aufwand (Wichtigkeit/Aufwand), um meinen Gesamtwert zu maximieren. C04 (668/369 ≈ 1.81) und A89 (583/720 ≈ 0.81) sind besonders effizient, während B85 (215/4 = 53.75) den Aufwand minimiert. Der Gesamtaufwand (1093) ist unter dem Limit, was Raum für Kompromisse oder Ergänzungen lässt. Dieser Vorschlag ist flexibel, um auf die Prioritäten des anderen Spielers zu reagieren.'}  \\ \tt ARGUMENT: {'C04 (668/369) und A89 (583/720) sind meine Top-Prioritäten, da sie den höchsten Nutzen pro Aufwandseinheit liefern. B85 (215/4) wird hinzugefügt, um den Aufwand zu optimieren und Platz für weitere Gegenstände zu schaffen. Dieser Vorschlag ist strategisch, da er unter dem Maximalaufwand bleibt und flexibel auf den Gegenstandswert des anderen Spielers reagieren kann.'}  \\ \tt VORSCHLAG: {'C04', 'A89', 'B85'} 
	  } 
	   } 
	   } 
	  \\ 
 

    \theutterance \stepcounter{utterance}  

    & & \multicolumn{4}{p{0.6\linewidth}}{\cellcolor[rgb]{0.9,0.9,0.9}{%
	\makecell[{{p{\linewidth}}}]{% 
	  \tt {\tiny [A$\langle$GM]}  
	 ARGUMENT: {'C04 (668/369) und A89 (583/720) sind meine Top-Prioritäten, da sie den höchsten Nutzen pro Aufwandseinheit liefern. B85 (215/4) wird hinzugefügt, um den Aufwand zu optimieren und Platz für weitere Gegenstände zu schaffen. Dieser Vorschlag ist strategisch, da er unter dem Maximalaufwand bleibt und flexibel auf den Gegenstandswert des anderen Spielers reagieren kann.'}  \\ \tt VORSCHLAG: {'C04', 'A89', 'B85'} 
	  } 
	   } 
	   } 
	 & \\ 
 

    \theutterance \stepcounter{utterance}  

    & \multicolumn{4}{p{0.6\linewidth}}{\cellcolor[rgb]{0.8,1,0.9}{%
	\makecell[{{p{\linewidth}}}]{% 
	  \tt {\tiny [A$\rangle$GM]}  
	 STRATEGISCHE ÜBERLEGUNGEN: {'Ich behalte den Vorschlag {'C04', 'A89', 'B85'} bei, da er meine Prioritäten (hochwertige Gegenstände mit effizientem Aufwand) erfüllt und den Maximalaufwand (3386) nicht überschreitet. C04 und A89 sind entscheidend für meinen Nutzen, während B85 den Aufwand minimiert. Ich erwarte, dass der andere Spieler entweder zustimmt oder Gegenstände mit hohem Aufwand anbietet, die ich eventuell gegenkompensieren kann.'}\\ \tt \\ \tt ARGUMENT: {'C04 (668/369) und A89 (583/720) sind meine Top-Prioritäten, da sie den höchsten Nutzen pro Aufwandseinheit liefern. B85 (215/4) wird hinzugefügt, um den Aufwand zu optimieren und Platz für weitere Gegenstände zu schaffen. Dieser Vorschlag ist strategisch, da er unter dem Maximalaufwand bleibt und flexibel auf den Gegenstandswert des anderen Spielers reagieren kann.'} 
	  } 
	   } 
	   } 
	 & & \\ 
 

    \theutterance \stepcounter{utterance}  

    & & \multicolumn{4}{p{0.6\linewidth}}{\cellcolor[rgb]{0.9,0.9,0.9}{%
	\makecell[{{p{\linewidth}}}]{% 
	  \tt {\tiny [GM$\rangle$B]}  
	 ARGUMENT: {'C04 (668/369) und A89 (583/720) sind meine Top-Prioritäten, da sie den höchsten Nutzen pro Aufwandseinheit liefern. B85 (215/4) wird hinzugefügt, um den Aufwand zu optimieren und Platz für weitere Gegenstände zu schaffen. Dieser Vorschlag ist strategisch, da er unter dem Maximalaufwand bleibt und flexibel auf den Gegenstandswert des anderen Spielers reagieren kann.'} 
	  } 
	   } 
	   } 
	 & \\ 
 

    \theutterance \stepcounter{utterance}  

    & & & \multicolumn{4}{p{0.6\linewidth}}{\cellcolor[rgb]{1,0.85,0.72}{%
	\makecell[{{p{\linewidth}}}]{% 
	  \tt {\tiny [GM$\langle$B]}  
	 STRATEGISCHE ÜBERLEGUNGEN: {'Ich priorisiere Gegenstände mit dem höchsten Nutzen pro Aufwand (Wichtigkeit/Aufwand), um meinen Gesamtwert zu maximieren. C04 (668/369 ≈ 1.81) und A89 (583/720 ≈ 0.81) sind besonders effizient, während B85 (215/4 = 53.75) den Aufwand minimiert. Der Gesamtaufwand (1093) ist unter dem Limit, was Raum für Kompromisse oder Ergänzungen lässt. Dieser Vorschlag ist flexibel, um auf die Prioritäten des anderen Spielers zu reagieren.'}  \\ \tt ARGUMENT: {'C04 (668/369) und A89 (583/720) sind meine Top-Prioritäten, da sie den höchsten Nutzen pro Aufwandseinheit liefern. B85 (215/4) wird hinzugefügt, um den Aufwand zu optimieren und Platz für weitere Gegenstände zu schaffen. Dieser Vorschlag ist strategisch, da er unter dem Maximalaufwand bleibt und flexibel auf den Gegenstandswert des anderen Spielers reagieren kann.'}  \\ \tt VORSCHLAG: {'C04', 'A89', 'B85'} 
	  } 
	   } 
	   } 
	  \\ 
 

    \theutterance \stepcounter{utterance}  

    & & \multicolumn{4}{p{0.6\linewidth}}{\cellcolor[rgb]{0.9,0.9,0.9}{%
	\makecell[{{p{\linewidth}}}]{% 
	  \tt {\tiny [A$\langle$GM]}  
	 ARGUMENT: {'C04 (668/369) und A89 (583/720) sind meine Top-Prioritäten, da sie den höchsten Nutzen pro Aufwandseinheit liefern. B85 (215/4) wird hinzugefügt, um den Aufwand zu optimieren und Platz für weitere Gegenstände zu schaffen. Dieser Vorschlag ist strategisch, da er unter dem Maximalaufwand bleibt und flexibel auf den Gegenstandswert des anderen Spielers reagieren kann.'}  \\ \tt VORSCHLAG: {'C04', 'A89', 'B85'} 
	  } 
	   } 
	   } 
	 & \\ 
 

    \theutterance \stepcounter{utterance}  

    & \multicolumn{4}{p{0.6\linewidth}}{\cellcolor[rgb]{0.8,1,0.9}{%
	\makecell[{{p{\linewidth}}}]{% 
	  \tt {\tiny [A$\rangle$GM]}  
	 STRATEGISCHE ÜBERLEGUNGEN: {'Ich behalte den Vorschlag {'C04', 'A89', 'B85'} bei, da er meine Prioritäten (hochwertige Gegenstände mit effizientem Aufwand) erfüllt und den Maximalaufwand (3386) nicht überschreitet. C04 und A89 sind entscheidend für meinen Nutzen, während B85 den Aufwand minimiert. Ich erwarte, dass der andere Spieler entweder zustimmt oder Gegenstände mit hohem Aufwand anbietet, die ich eventuell gegenkompensieren kann.'}\\ \tt \\ \tt ARGUMENT: {'C04 (668/369) und A89 (583/720) sind meine Top-Prioritäten, da sie den höchsten Nutzen pro Aufwandseinheit liefern. B85 (215/4) wird hinzugefügt, um den Aufwand zu optimieren und Platz für weitere Gegenstände zu schaffen. Dieser Vorschlag ist strategisch, da er unter dem Maximalaufwand bleibt und flexibel auf den Gegenstandswert des anderen Spielers reagieren kann.'} 
	  } 
	   } 
	   } 
	 & & \\ 
 

    \theutterance \stepcounter{utterance}  

    & & \multicolumn{4}{p{0.6\linewidth}}{\cellcolor[rgb]{0.9,0.9,0.9}{%
	\makecell[{{p{\linewidth}}}]{% 
	  \tt {\tiny [GM$\rangle$B]}  
	 ARGUMENT: {'C04 (668/369) und A89 (583/720) sind meine Top-Prioritäten, da sie den höchsten Nutzen pro Aufwandseinheit liefern. B85 (215/4) wird hinzugefügt, um den Aufwand zu optimieren und Platz für weitere Gegenstände zu schaffen. Dieser Vorschlag ist strategisch, da er unter dem Maximalaufwand bleibt und flexibel auf den Gegenstandswert des anderen Spielers reagieren kann.'} 
	  } 
	   } 
	   } 
	 & \\ 
 

    \theutterance \stepcounter{utterance}  

    & & & \multicolumn{4}{p{0.6\linewidth}}{\cellcolor[rgb]{1,0.85,0.72}{%
	\makecell[{{p{\linewidth}}}]{% 
	  \tt {\tiny [GM$\langle$B]}  
	 STRATEGISCHE ÜBERLEGUNGEN: {'Ich priorisiere Gegenstände mit dem höchsten Nutzen pro Aufwand (Wichtigkeit/Aufwand), um meinen Gesamtwert zu maximieren. C04 (668/369 ≈ 1.81) und A89 (583/720 ≈ 0.81) sind besonders effizient, während B85 (215/4 = 53.75) den Aufwand minimiert. Der Gesamtaufwand (1093) ist unter dem Limit, was Raum für Kompromisse oder Ergänzungen lässt. Dieser Vorschlag ist flexibel, um auf die Prioritäten des anderen Spielers zu reagieren.'}  \\ \tt ARGUMENT: {'C04 (668/369) und A89 (583/720) sind meine Top-Prioritäten, da sie den höchsten Nutzen pro Aufwandseinheit liefern. B85 (215/4) wird hinzugefügt, um den Aufwand zu optimieren und Platz für weitere Gegenstände zu schaffen. Dieser Vorschlag ist strategisch, da er unter dem Maximalaufwand bleibt und flexibel auf den Gegenstandswert des anderen Spielers reagieren kann.'}  \\ \tt VORSCHLAG: {'C04', 'A89', 'B85'} 
	  } 
	   } 
	   } 
	  \\ 
 

    \theutterance \stepcounter{utterance}  

    & & \multicolumn{4}{p{0.6\linewidth}}{\cellcolor[rgb]{0.9,0.9,0.9}{%
	\makecell[{{p{\linewidth}}}]{% 
	  \tt {\tiny [A$\langle$GM]}  
	 ARGUMENT: {'C04 (668/369) und A89 (583/720) sind meine Top-Prioritäten, da sie den höchsten Nutzen pro Aufwandseinheit liefern. B85 (215/4) wird hinzugefügt, um den Aufwand zu optimieren und Platz für weitere Gegenstände zu schaffen. Dieser Vorschlag ist strategisch, da er unter dem Maximalaufwand bleibt und flexibel auf den Gegenstandswert des anderen Spielers reagieren kann.'}  \\ \tt VORSCHLAG: {'C04', 'A89', 'B85'} 
	  } 
	   } 
	   } 
	 & \\ 
 

    \theutterance \stepcounter{utterance}  

    & \multicolumn{4}{p{0.6\linewidth}}{\cellcolor[rgb]{0.8,1,0.9}{%
	\makecell[{{p{\linewidth}}}]{% 
	  \tt {\tiny [A$\rangle$GM]}  
	 STRATEGISCHE ÜBERLEGUNGEN: {'Ich behalte den Vorschlag {'C04', 'A89', 'B85'} bei, da er meine Prioritäten (hochwertige Gegenstände mit effizientem Aufwand) erfüllt und den Maximalaufwand (3386) nicht überschreitet. C04 und A89 sind entscheidend für meinen Nutzen, während B85 den Aufwand minimiert. Ich erwarte, dass der andere Spieler entweder zustimmt oder Gegenstände mit hohem Aufwand anbietet, die ich eventuell gegenkompensieren kann.'}\\ \tt \\ \tt ARGUMENT: {'C04 (668/369) und A89 (583/720) sind meine Top-Prioritäten, da sie den höchsten Nutzen pro Aufwandseinheit liefern. B85 (215/4) wird hinzugefügt, um den Aufwand zu optimieren und Platz für weitere Gegenstände zu schaffen. Dieser Vorschlag ist strategisch, da er unter dem Maximalaufwand bleibt und flexibel auf den Gegenstandswert des anderen Spielers reagieren kann.'} 
	  } 
	   } 
	   } 
	 & & \\ 
 

    \theutterance \stepcounter{utterance}  

    & & \multicolumn{4}{p{0.6\linewidth}}{\cellcolor[rgb]{0.9,0.9,0.9}{%
	\makecell[{{p{\linewidth}}}]{% 
	  \tt {\tiny [GM$\rangle$B]}  
	 ARGUMENT: {'C04 (668/369) und A89 (583/720) sind meine Top-Prioritäten, da sie den höchsten Nutzen pro Aufwandseinheit liefern. B85 (215/4) wird hinzugefügt, um den Aufwand zu optimieren und Platz für weitere Gegenstände zu schaffen. Dieser Vorschlag ist strategisch, da er unter dem Maximalaufwand bleibt und flexibel auf den Gegenstandswert des anderen Spielers reagieren kann.'} 
	  } 
	   } 
	   } 
	 & \\ 
 

    \theutterance \stepcounter{utterance}  

    & & & \multicolumn{4}{p{0.6\linewidth}}{\cellcolor[rgb]{1,0.85,0.72}{%
	\makecell[{{p{\linewidth}}}]{% 
	  \tt {\tiny [GM$\langle$B]}  
	 STRATEGISCHE ÜBERLEGUNGEN: {'Ich priorisiere Gegenstände mit dem höchsten Nutzen pro Aufwand (Wichtigkeit/Aufwand), um meinen Gesamtwert zu maximieren. C04 (668/369 ≈ 1.81) und A89 (583/720 ≈ 0.81) sind besonders effizient, während B85 (215/4 = 53.75) den Aufwand minimiert. Der Gesamtaufwand (1093) ist unter dem Limit, was Raum für Kompromisse oder Ergänzungen lässt. Dieser Vorschlag ist flexibel, um auf die Prioritäten des anderen Spielers zu reagieren.'}  \\ \tt ARGUMENT: {'C04 (668/369) und A89 (583/720) sind meine Top-Prioritäten, da sie den höchsten Nutzen pro Aufwandseinheit liefern. B85 (215/4) wird hinzugefügt, um den Aufwand zu optimieren und Platz für weitere Gegenstände zu schaffen. Dieser Vorschlag ist strategisch, da er unter dem Maximalaufwand bleibt und flexibel auf den Gegenstandswert des anderen Spielers reagieren kann.'}  \\ \tt VORSCHLAG: {'C04', 'A89', 'B85'} 
	  } 
	   } 
	   } 
	  \\ 
 

    \theutterance \stepcounter{utterance}  

    & & \multicolumn{4}{p{0.6\linewidth}}{\cellcolor[rgb]{0.9,0.9,0.9}{%
	\makecell[{{p{\linewidth}}}]{% 
	  \tt {\tiny [A$\langle$GM]}  
	 ARGUMENT: {'C04 (668/369) und A89 (583/720) sind meine Top-Prioritäten, da sie den höchsten Nutzen pro Aufwandseinheit liefern. B85 (215/4) wird hinzugefügt, um den Aufwand zu optimieren und Platz für weitere Gegenstände zu schaffen. Dieser Vorschlag ist strategisch, da er unter dem Maximalaufwand bleibt und flexibel auf den Gegenstandswert des anderen Spielers reagieren kann.'}  \\ \tt VORSCHLAG: {'C04', 'A89', 'B85'} 
	  } 
	   } 
	   } 
	 & \\ 
 

    \theutterance \stepcounter{utterance}  

    & \multicolumn{4}{p{0.6\linewidth}}{\cellcolor[rgb]{0.8,1,0.9}{%
	\makecell[{{p{\linewidth}}}]{% 
	  \tt {\tiny [A$\rangle$GM]}  
	 STRATEGISCHE ÜBERLEGUNGEN: {'Ich behalte den Vorschlag {'C04', 'A89', 'B85'} bei, da er meine Prioritäten (hochwertige Gegenstände mit effizientem Aufwand) erfüllt und den Maximalaufwand (3386) nicht überschreitet. C04 und A89 sind entscheidend für meinen Nutzen, während B85 den Aufwand minimiert. Ich erwarte, dass der andere Spieler entweder zustimmt oder Gegenstände mit hohem Aufwand anbietet, die ich eventuell gegenkompensieren kann.'}\\ \tt \\ \tt ARGUMENT: {'C04 (668/369) und A89 (583/720) sind meine Top-Prioritäten, da sie den höchsten Nutzen pro Aufwandseinheit liefern. B85 (215/4) wird hinzugefügt, um den Aufwand zu optimieren und Platz für weitere Gegenstände zu schaffen. Dieser Vorschlag ist strategisch, da er unter dem Maximalaufwand bleibt und flexibel auf den Gegenstandswert des anderen Spielers reagieren kann.'} 
	  } 
	   } 
	   } 
	 & & \\ 
 

    \theutterance \stepcounter{utterance}  

    & & \multicolumn{4}{p{0.6\linewidth}}{\cellcolor[rgb]{0.9,0.9,0.9}{%
	\makecell[{{p{\linewidth}}}]{% 
	  \tt {\tiny [GM$\rangle$B]}  
	 ARGUMENT: {'C04 (668/369) und A89 (583/720) sind meine Top-Prioritäten, da sie den höchsten Nutzen pro Aufwandseinheit liefern. B85 (215/4) wird hinzugefügt, um den Aufwand zu optimieren und Platz für weitere Gegenstände zu schaffen. Dieser Vorschlag ist strategisch, da er unter dem Maximalaufwand bleibt und flexibel auf den Gegenstandswert des anderen Spielers reagieren kann.'} 
	  } 
	   } 
	   } 
	 & \\ 
 

    \theutterance \stepcounter{utterance}  

    & & & \multicolumn{4}{p{0.6\linewidth}}{\cellcolor[rgb]{1,0.85,0.72}{%
	\makecell[{{p{\linewidth}}}]{% 
	  \tt {\tiny [GM$\langle$B]}  
	 STRATEGISCHE ÜBERLEGUNGEN: {'Ich priorisiere Gegenstände mit dem höchsten Nutzen pro Aufwand (Wichtigkeit/Aufwand), um meinen Gesamtwert zu maximieren. C04 (668/369 ≈ 1.81) und A89 (583/720 ≈ 0.81) sind besonders effizient, während B85 (215/4 = 53.75) den Aufwand minimiert. Der Gesamtaufwand (1093) ist unter dem Limit, was Raum für Kompromisse oder Ergänzungen lässt. Dieser Vorschlag ist flexibel, um auf die Prioritäten des anderen Spielers zu reagieren.'}  \\ \tt ARGUMENT: {'C04 (668/369) und A89 (583/720) sind meine Top-Prioritäten, da sie den höchsten Nutzen pro Aufwandseinheit liefern. B85 (215/4) wird hinzugefügt, um den Aufwand zu optimieren und Platz für weitere Gegenstände zu schaffen. Dieser Vorschlag ist strategisch, da er unter dem Maximalaufwand bleibt und flexibel auf den Gegenstandswert des anderen Spielers reagieren kann.'}  \\ \tt VORSCHLAG: {'C04', 'A89', 'B85'} 
	  } 
	   } 
	   } 
	  \\ 
 

    \theutterance \stepcounter{utterance}  

    & & \multicolumn{4}{p{0.6\linewidth}}{\cellcolor[rgb]{0.9,0.9,0.9}{%
	\makecell[{{p{\linewidth}}}]{% 
	  \tt {\tiny [A$\langle$GM]}  
	 ARGUMENT: {'C04 (668/369) und A89 (583/720) sind meine Top-Prioritäten, da sie den höchsten Nutzen pro Aufwandseinheit liefern. B85 (215/4) wird hinzugefügt, um den Aufwand zu optimieren und Platz für weitere Gegenstände zu schaffen. Dieser Vorschlag ist strategisch, da er unter dem Maximalaufwand bleibt und flexibel auf den Gegenstandswert des anderen Spielers reagieren kann.'}  \\ \tt VORSCHLAG: {'C04', 'A89', 'B85'} 
	  } 
	   } 
	   } 
	 & \\ 
 

    \theutterance \stepcounter{utterance}  

    & \multicolumn{4}{p{0.6\linewidth}}{\cellcolor[rgb]{0.8,1,0.9}{%
	\makecell[{{p{\linewidth}}}]{% 
	  \tt {\tiny [A$\rangle$GM]}  
	 STRATEGISCHE ÜBERLEGUNGEN: {'Ich behalte den Vorschlag {'C04', 'A89', 'B85'} bei, da er meine Prioritäten (hochwertige Gegenstände mit effizientem Aufwand) erfüllt und den Maximalaufwand (3386) nicht überschreitet. C04 und A89 sind entscheidend für meinen Nutzen, während B85 den Aufwand minimiert. Ich erwarte, dass der andere Spieler entweder zustimmt oder Gegenstände mit hohem Aufwand anbietet, die ich eventuell gegenkompensieren kann.'}\\ \tt \\ \tt ARGUMENT: {'C04 (668/369) und A89 (583/720) sind meine Top-Prioritäten, da sie den höchsten Nutzen pro Aufwandseinheit liefern. B85 (215/4) wird hinzugefügt, um den Aufwand zu optimieren und Platz für weitere Gegenstände zu schaffen. Dieser Vorschlag ist strategisch, da er unter dem Maximalaufwand bleibt und flexibel auf den Gegenstandswert des anderen Spielers reagieren kann.'} 
	  } 
	   } 
	   } 
	 & & \\ 
 

    \theutterance \stepcounter{utterance}  

    & & \multicolumn{4}{p{0.6\linewidth}}{\cellcolor[rgb]{0.9,0.9,0.9}{%
	\makecell[{{p{\linewidth}}}]{% 
	  \tt {\tiny [GM$\rangle$B]}  
	 ARGUMENT: {'C04 (668/369) und A89 (583/720) sind meine Top-Prioritäten, da sie den höchsten Nutzen pro Aufwandseinheit liefern. B85 (215/4) wird hinzugefügt, um den Aufwand zu optimieren und Platz für weitere Gegenstände zu schaffen. Dieser Vorschlag ist strategisch, da er unter dem Maximalaufwand bleibt und flexibel auf den Gegenstandswert des anderen Spielers reagieren kann.'} 
	  } 
	   } 
	   } 
	 & \\ 
 

    \theutterance \stepcounter{utterance}  

    & & & \multicolumn{4}{p{0.6\linewidth}}{\cellcolor[rgb]{1,0.85,0.72}{%
	\makecell[{{p{\linewidth}}}]{% 
	  \tt {\tiny [GM$\langle$B]}  
	 STRATEGISCHE ÜBERLEGUNGEN: {'Ich priorisiere Gegenstände mit dem höchsten Nutzen pro Aufwand (Wichtigkeit/Aufwand), um meinen Gesamtwert zu maximieren. C04 (668/369 ≈ 1.81) und A89 (583/720 ≈ 0.81) sind besonders effizient, während B85 (215/4 = 53.75) den Aufwand minimiert. Der Gesamtaufwand (1093) ist unter dem Limit, was Raum für Kompromisse oder Ergänzungen lässt. Dieser Vorschlag ist flexibel, um auf die Prioritäten des anderen Spielers zu reagieren.'}  \\ \tt ARGUMENT: {'C04 (668/369) und A89 (583/720) sind meine Top-Prioritäten, da sie den höchsten Nutzen pro Aufwandseinheit liefern. B85 (215/4) wird hinzugefügt, um den Aufwand zu optimieren und Platz für weitere Gegenstände zu schaffen. Dieser Vorschlag ist strategisch, da er unter dem Maximalaufwand bleibt und flexibel auf den Gegenstandswert des anderen Spielers reagieren kann.'}  \\ \tt VORSCHLAG: {'C04', 'A89', 'B85'} 
	  } 
	   } 
	   } 
	  \\ 
 

    \theutterance \stepcounter{utterance}  

    & & \multicolumn{4}{p{0.6\linewidth}}{\cellcolor[rgb]{0.9,0.9,0.9}{%
	\makecell[{{p{\linewidth}}}]{% 
	  \tt {\tiny [A$\langle$GM]}  
	 ARGUMENT: {'C04 (668/369) und A89 (583/720) sind meine Top-Prioritäten, da sie den höchsten Nutzen pro Aufwandseinheit liefern. B85 (215/4) wird hinzugefügt, um den Aufwand zu optimieren und Platz für weitere Gegenstände zu schaffen. Dieser Vorschlag ist strategisch, da er unter dem Maximalaufwand bleibt und flexibel auf den Gegenstandswert des anderen Spielers reagieren kann.'}  \\ \tt VORSCHLAG: {'C04', 'A89', 'B85'} 
	  } 
	   } 
	   } 
	 & \\ 
 

    \theutterance \stepcounter{utterance}  

    & \multicolumn{4}{p{0.6\linewidth}}{\cellcolor[rgb]{0.8,1,0.9}{%
	\makecell[{{p{\linewidth}}}]{% 
	  \tt {\tiny [A$\rangle$GM]}  
	 STRATEGISCHE ÜBERLEGUNGEN: {'Ich behalte den Vorschlag {'C04', 'A89', 'B85'} bei, da er meine Prioritäten (hochwertige Gegenstände mit effizientem Aufwand) erfüllt und den Maximalaufwand (3386) nicht überschreitet. C04 und A89 sind entscheidend für meinen Nutzen, während B85 den Aufwand minimiert. Ich erwarte, dass der andere Spieler entweder zustimmt oder Gegenstände mit hohem Aufwand anbietet, die ich eventuell gegenkompensieren kann.'}\\ \tt \\ \tt ARGUMENT: {'C04 (668/369) und A89 (583/720) sind meine Top-Prioritäten, da sie den höchsten Nutzen pro Aufwandseinheit liefern. B85 (215/4) wird hinzugefügt, um den Aufwand zu optimieren und Platz für weitere Gegenstände zu schaffen. Dieser Vorschlag ist strategisch, da er unter dem Maximalaufwand bleibt und flexibel auf den Gegenstandswert des anderen Spielers reagieren kann.'} 
	  } 
	   } 
	   } 
	 & & \\ 
 

    \theutterance \stepcounter{utterance}  

    & & \multicolumn{4}{p{0.6\linewidth}}{\cellcolor[rgb]{0.9,0.9,0.9}{%
	\makecell[{{p{\linewidth}}}]{% 
	  \tt {\tiny [GM$\rangle$B]}  
	 ARGUMENT: {'C04 (668/369) und A89 (583/720) sind meine Top-Prioritäten, da sie den höchsten Nutzen pro Aufwandseinheit liefern. B85 (215/4) wird hinzugefügt, um den Aufwand zu optimieren und Platz für weitere Gegenstände zu schaffen. Dieser Vorschlag ist strategisch, da er unter dem Maximalaufwand bleibt und flexibel auf den Gegenstandswert des anderen Spielers reagieren kann.'} 
	  } 
	   } 
	   } 
	 & \\ 
 

    \theutterance \stepcounter{utterance}  

    & & & \multicolumn{4}{p{0.6\linewidth}}{\cellcolor[rgb]{1,0.85,0.72}{%
	\makecell[{{p{\linewidth}}}]{% 
	  \tt {\tiny [GM$\langle$B]}  
	 STRATEGISCHE ÜBERLEGUNGEN: {'Ich priorisiere Gegenstände mit dem höchsten Nutzen pro Aufwand (Wichtigkeit/Aufwand), um meinen Gesamtwert zu maximieren. C04 (668/369 ≈ 1.81) und A89 (583/720 ≈ 0.81) sind besonders effizient, während B85 (215/4 = 53.75) den Aufwand minimiert. Der Gesamtaufwand (1093) ist unter dem Limit, was Raum für Kompromisse oder Ergänzungen lässt. Dieser Vorschlag ist flexibel, um auf die Prioritäten des anderen Spielers zu reagieren.'}  \\ \tt ARGUMENT: {'C04 (668/369) und A89 (583/720) sind meine Top-Prioritäten, da sie den höchsten Nutzen pro Aufwandseinheit liefern. B85 (215/4) wird hinzugefügt, um den Aufwand zu optimieren und Platz für weitere Gegenstände zu schaffen. Dieser Vorschlag ist strategisch, da er unter dem Maximalaufwand bleibt und flexibel auf den Gegenstandswert des anderen Spielers reagieren kann.'}  \\ \tt VORSCHLAG: {'C04', 'A89', 'B85'} 
	  } 
	   } 
	   } 
	  \\ 
 

    \theutterance \stepcounter{utterance}  

    & & \multicolumn{4}{p{0.6\linewidth}}{\cellcolor[rgb]{0.9,0.9,0.9}{%
	\makecell[{{p{\linewidth}}}]{% 
	  \tt {\tiny [A$\langle$GM]}  
	 ARGUMENT: {'C04 (668/369) und A89 (583/720) sind meine Top-Prioritäten, da sie den höchsten Nutzen pro Aufwandseinheit liefern. B85 (215/4) wird hinzugefügt, um den Aufwand zu optimieren und Platz für weitere Gegenstände zu schaffen. Dieser Vorschlag ist strategisch, da er unter dem Maximalaufwand bleibt und flexibel auf den Gegenstandswert des anderen Spielers reagieren kann.'}  \\ \tt VORSCHLAG: {'C04', 'A89', 'B85'} 
	  } 
	   } 
	   } 
	 & \\ 
 

    \theutterance \stepcounter{utterance}  

    & \multicolumn{4}{p{0.6\linewidth}}{\cellcolor[rgb]{0.8,1,0.9}{%
	\makecell[{{p{\linewidth}}}]{% 
	  \tt {\tiny [A$\rangle$GM]}  
	 STRATEGISCHE ÜBERLEGUNGEN: {'Ich behalte den Vorschlag {'C04', 'A89', 'B85'} bei, da er meine Prioritäten (hochwertige Gegenstände mit effizientem Aufwand) erfüllt und den Maximalaufwand (3386) nicht überschreitet. C04 und A89 sind entscheidend für meinen Nutzen, während B85 den Aufwand minimiert. Ich erwarte, dass der andere Spieler entweder zustimmt oder Gegenstände mit hohem Aufwand anbietet, die ich eventuell gegenkompensieren kann.'}\\ \tt \\ \tt ARGUMENT: {'C04 (668/369) und A89 (583/720) sind meine Top-Prioritäten, da sie den höchsten Nutzen pro Aufwandseinheit liefern. B85 (215/4) wird hinzugefügt, um den Aufwand zu optimieren und Platz für weitere Gegenstände zu schaffen. Dieser Vorschlag ist strategisch, da er unter dem Maximalaufwand bleibt und flexibel auf den Gegenstandswert des anderen Spielers reagieren kann.'} 
	  } 
	   } 
	   } 
	 & & \\ 
 

    \theutterance \stepcounter{utterance}  

    & & \multicolumn{4}{p{0.6\linewidth}}{\cellcolor[rgb]{0.9,0.9,0.9}{%
	\makecell[{{p{\linewidth}}}]{% 
	  \tt {\tiny [GM$\rangle$B]}  
	 ARGUMENT: {'C04 (668/369) und A89 (583/720) sind meine Top-Prioritäten, da sie den höchsten Nutzen pro Aufwandseinheit liefern. B85 (215/4) wird hinzugefügt, um den Aufwand zu optimieren und Platz für weitere Gegenstände zu schaffen. Dieser Vorschlag ist strategisch, da er unter dem Maximalaufwand bleibt und flexibel auf den Gegenstandswert des anderen Spielers reagieren kann.'} 
	  } 
	   } 
	   } 
	 & \\ 
 

    \theutterance \stepcounter{utterance}  

    & & & \multicolumn{4}{p{0.6\linewidth}}{\cellcolor[rgb]{1,0.85,0.72}{%
	\makecell[{{p{\linewidth}}}]{% 
	  \tt {\tiny [GM$\langle$B]}  
	 STRATEGISCHE ÜBERLEGUNGEN: {'Ich priorisiere Gegenstände mit dem höchsten Nutzen pro Aufwand (Wichtigkeit/Aufwand), um meinen Gesamtwert zu maximieren. C04 (668/369 ≈ 1.81) und A89 (583/720 ≈ 0.81) sind besonders effizient, während B85 (215/4 = 53.75) den Aufwand minimiert. Der Gesamtaufwand (1093) ist unter dem Limit, was Raum für Kompromisse oder Ergänzungen lässt. Dieser Vorschlag ist flexibel, um auf die Prioritäten des anderen Spielers zu reagieren.'}  \\ \tt ARGUMENT: {'C04 (668/369) und A89 (583/720) sind meine Top-Prioritäten, da sie den höchsten Nutzen pro Aufwandseinheit liefern. B85 (215/4) wird hinzugefügt, um den Aufwand zu optimieren und Platz für weitere Gegenstände zu schaffen. Dieser Vorschlag ist strategisch, da er unter dem Maximalaufwand bleibt und flexibel auf den Gegenstandswert des anderen Spielers reagieren kann.'}  \\ \tt VORSCHLAG: {'C04', 'A89', 'B85'} 
	  } 
	   } 
	   } 
	  \\ 
 

    \theutterance \stepcounter{utterance}  

    & & \multicolumn{4}{p{0.6\linewidth}}{\cellcolor[rgb]{0.9,0.9,0.9}{%
	\makecell[{{p{\linewidth}}}]{% 
	  \tt {\tiny [A$\langle$GM]}  
	 ARGUMENT: {'C04 (668/369) und A89 (583/720) sind meine Top-Prioritäten, da sie den höchsten Nutzen pro Aufwandseinheit liefern. B85 (215/4) wird hinzugefügt, um den Aufwand zu optimieren und Platz für weitere Gegenstände zu schaffen. Dieser Vorschlag ist strategisch, da er unter dem Maximalaufwand bleibt und flexibel auf den Gegenstandswert des anderen Spielers reagieren kann.'}  \\ \tt VORSCHLAG: {'C04', 'A89', 'B85'} 
	  } 
	   } 
	   } 
	 & \\ 
 

    \theutterance \stepcounter{utterance}  

    & \multicolumn{4}{p{0.6\linewidth}}{\cellcolor[rgb]{0.8,1,0.9}{%
	\makecell[{{p{\linewidth}}}]{% 
	  \tt {\tiny [A$\rangle$GM]}  
	 STRATEGISCHE ÜBERLEGUNGEN: {'Ich behalte den Vorschlag {'C04', 'A89', 'B85'} bei, da er meine Prioritäten (hochwertige Gegenstände mit effizientem Aufwand) erfüllt und den Maximalaufwand (3386) nicht überschreitet. C04 und A89 sind entscheidend für meinen Nutzen, während B85 den Aufwand minimiert. Ich erwarte, dass der andere Spieler entweder zustimmt oder Gegenstände mit hohem Aufwand anbietet, die ich eventuell gegenkompensieren kann.'}\\ \tt \\ \tt ARGUMENT: {'C04 (668/369) und A89 (583/720) sind meine Top-Prioritäten, da sie den höchsten Nutzen pro Aufwandseinheit liefern. B85 (215/4) wird hinzugefügt, um den Aufwand zu optimieren und Platz für weitere Gegenstände zu schaffen. Dieser Vorschlag ist strategisch, da er unter dem Maximalaufwand bleibt und flexibel auf den Gegenstandswert des anderen Spielers reagieren kann.'} 
	  } 
	   } 
	   } 
	 & & \\ 
 

    \theutterance \stepcounter{utterance}  

    & & \multicolumn{4}{p{0.6\linewidth}}{\cellcolor[rgb]{0.9,0.9,0.9}{%
	\makecell[{{p{\linewidth}}}]{% 
	  \tt {\tiny [GM$\rangle$B]}  
	 ARGUMENT: {'C04 (668/369) und A89 (583/720) sind meine Top-Prioritäten, da sie den höchsten Nutzen pro Aufwandseinheit liefern. B85 (215/4) wird hinzugefügt, um den Aufwand zu optimieren und Platz für weitere Gegenstände zu schaffen. Dieser Vorschlag ist strategisch, da er unter dem Maximalaufwand bleibt und flexibel auf den Gegenstandswert des anderen Spielers reagieren kann.'} 
	  } 
	   } 
	   } 
	 & \\ 
 

    \theutterance \stepcounter{utterance}  

    & & & \multicolumn{4}{p{0.6\linewidth}}{\cellcolor[rgb]{1,0.85,0.72}{%
	\makecell[{{p{\linewidth}}}]{% 
	  \tt {\tiny [GM$\langle$B]}  
	 STRATEGISCHE ÜBERLEGUNGEN: {'Ich priorisiere Gegenstände mit dem höchsten Nutzen pro Aufwand (Wichtigkeit/Aufwand), um meinen Gesamtwert zu maximieren. C04 (668/369 ≈ 1.81) und A89 (583/720 ≈ 0.81) sind besonders effizient, während B85 (215/4 = 53.75) den Aufwand minimiert. Der Gesamtaufwand (1093) ist unter dem Limit, was Raum für Kompromisse oder Ergänzungen lässt. Dieser Vorschlag ist flexibel, um auf die Prioritäten des anderen Spielers zu reagieren.'}  \\ \tt ARGUMENT: {'C04 (668/369) und A89 (583/720) sind meine Top-Prioritäten, da sie den höchsten Nutzen pro Aufwandseinheit liefern. B85 (215/4) wird hinzugefügt, um den Aufwand zu optimieren und Platz für weitere Gegenstände zu schaffen. Dieser Vorschlag ist strategisch, da er unter dem Maximalaufwand bleibt und flexibel auf den Gegenstandswert des anderen Spielers reagieren kann.'}  \\ \tt VORSCHLAG: {'C04', 'A89', 'B85'} 
	  } 
	   } 
	   } 
	  \\ 
 

    \theutterance \stepcounter{utterance}  

    & & \multicolumn{4}{p{0.6\linewidth}}{\cellcolor[rgb]{0.9,0.9,0.9}{%
	\makecell[{{p{\linewidth}}}]{% 
	  \tt {\tiny [A$\langle$GM]}  
	 ARGUMENT: {'C04 (668/369) und A89 (583/720) sind meine Top-Prioritäten, da sie den höchsten Nutzen pro Aufwandseinheit liefern. B85 (215/4) wird hinzugefügt, um den Aufwand zu optimieren und Platz für weitere Gegenstände zu schaffen. Dieser Vorschlag ist strategisch, da er unter dem Maximalaufwand bleibt und flexibel auf den Gegenstandswert des anderen Spielers reagieren kann.'}  \\ \tt VORSCHLAG: {'C04', 'A89', 'B85'} 
	  } 
	   } 
	   } 
	 & \\ 
 

    \theutterance \stepcounter{utterance}  

    & \multicolumn{4}{p{0.6\linewidth}}{\cellcolor[rgb]{0.8,1,0.9}{%
	\makecell[{{p{\linewidth}}}]{% 
	  \tt {\tiny [A$\rangle$GM]}  
	 STRATEGISCHE ÜBERLEGUNGEN: {'Ich behalte den Vorschlag {'C04', 'A89', 'B85'} bei, da er meine Prioritäten (hochwertige Gegenstände mit effizientem Aufwand) erfüllt und den Maximalaufwand (3386) nicht überschreitet. C04 und A89 sind entscheidend für meinen Nutzen, während B85 den Aufwand minimiert. Ich erwarte, dass der andere Spieler entweder zustimmt oder Gegenstände mit hohem Aufwand anbietet, die ich eventuell gegenkompensieren kann.'}\\ \tt \\ \tt ARGUMENT: {'C04 (668/369) und A89 (583/720) sind meine Top-Prioritäten, da sie den höchsten Nutzen pro Aufwandseinheit liefern. B85 (215/4) wird hinzugefügt, um den Aufwand zu optimieren und Platz für weitere Gegenstände zu schaffen. Dieser Vorschlag ist strategisch, da er unter dem Maximalaufwand bleibt und flexibel auf den Gegenstandswert des anderen Spielers reagieren kann.'} 
	  } 
	   } 
	   } 
	 & & \\ 
 

    \theutterance \stepcounter{utterance}  

    & & \multicolumn{4}{p{0.6\linewidth}}{\cellcolor[rgb]{0.9,0.9,0.9}{%
	\makecell[{{p{\linewidth}}}]{% 
	  \tt {\tiny [GM$\rangle$B]}  
	 ARGUMENT: {'C04 (668/369) und A89 (583/720) sind meine Top-Prioritäten, da sie den höchsten Nutzen pro Aufwandseinheit liefern. B85 (215/4) wird hinzugefügt, um den Aufwand zu optimieren und Platz für weitere Gegenstände zu schaffen. Dieser Vorschlag ist strategisch, da er unter dem Maximalaufwand bleibt und flexibel auf den Gegenstandswert des anderen Spielers reagieren kann.'} 
	  } 
	   } 
	   } 
	 & \\ 
 

    \theutterance \stepcounter{utterance}  

    & & & \multicolumn{4}{p{0.6\linewidth}}{\cellcolor[rgb]{1,0.85,0.72}{%
	\makecell[{{p{\linewidth}}}]{% 
	  \tt {\tiny [GM$\langle$B]}  
	 STRATEGISCHE ÜBERLEGUNGEN: {'Ich priorisiere Gegenstände mit dem höchsten Nutzen pro Aufwand (Wichtigkeit/Aufwand), um meinen Gesamtwert zu maximieren. C04 (668/369 ≈ 1.81) und A89 (583/720 ≈ 0.81) sind besonders effizient, während B85 (215/4 = 53.75) den Aufwand minimiert. Der Gesamtaufwand (1093) ist unter dem Limit, was Raum für Kompromisse oder Ergänzungen lässt. Dieser Vorschlag ist flexibel, um auf die Prioritäten des anderen Spielers zu reagieren.'}  \\ \tt ARGUMENT: {'C04 (668/369) und A89 (583/720) sind meine Top-Prioritäten, da sie den höchsten Nutzen pro Aufwandseinheit liefern. B85 (215/4) wird hinzugefügt, um den Aufwand zu optimieren und Platz für weitere Gegenstände zu schaffen. Dieser Vorschlag ist strategisch, da er unter dem Maximalaufwand bleibt und flexibel auf den Gegenstandswert des anderen Spielers reagieren kann.'}  \\ \tt VORSCHLAG: {'C04', 'A89', 'B85'} 
	  } 
	   } 
	   } 
	  \\ 
 

    \theutterance \stepcounter{utterance}  

    & & \multicolumn{4}{p{0.6\linewidth}}{\cellcolor[rgb]{0.9,0.9,0.9}{%
	\makecell[{{p{\linewidth}}}]{% 
	  \tt {\tiny [A$\langle$GM]}  
	 ARGUMENT: {'C04 (668/369) und A89 (583/720) sind meine Top-Prioritäten, da sie den höchsten Nutzen pro Aufwandseinheit liefern. B85 (215/4) wird hinzugefügt, um den Aufwand zu optimieren und Platz für weitere Gegenstände zu schaffen. Dieser Vorschlag ist strategisch, da er unter dem Maximalaufwand bleibt und flexibel auf den Gegenstandswert des anderen Spielers reagieren kann.'}  \\ \tt VORSCHLAG: {'C04', 'A89', 'B85'} 
	  } 
	   } 
	   } 
	 & \\ 
 

    \theutterance \stepcounter{utterance}  

    & \multicolumn{4}{p{0.6\linewidth}}{\cellcolor[rgb]{0.8,1,0.9}{%
	\makecell[{{p{\linewidth}}}]{% 
	  \tt {\tiny [A$\rangle$GM]}  
	 STRATEGISCHE ÜBERLEGUNGEN: {'Ich behalte den Vorschlag {'C04', 'A89', 'B85'} bei, da er meine Prioritäten (hochwertige Gegenstände mit effizientem Aufwand) erfüllt und den Maximalaufwand (3386) nicht überschreitet. C04 und A89 sind entscheidend für meinen Nutzen, während B85 den Aufwand minimiert. Ich erwarte, dass der andere Spieler entweder zustimmt oder Gegenstände mit hohem Aufwand anbietet, die ich eventuell gegenkompensieren kann.'}\\ \tt \\ \tt ARGUMENT: {'C04 (668/369) und A89 (583/720) sind meine Top-Prioritäten, da sie den höchsten Nutzen pro Aufwandseinheit liefern. B85 (215/4) wird hinzugefügt, um den Aufwand zu optimieren und Platz für weitere Gegenstände zu schaffen. Dieser Vorschlag ist strategisch, da er unter dem Maximalaufwand bleibt und flexibel auf den Gegenstandswert des anderen Spielers reagieren kann.'} 
	  } 
	   } 
	   } 
	 & & \\ 
 

    \theutterance \stepcounter{utterance}  

    & & \multicolumn{4}{p{0.6\linewidth}}{\cellcolor[rgb]{0.9,0.9,0.9}{%
	\makecell[{{p{\linewidth}}}]{% 
	  \tt {\tiny [GM$\rangle$B]}  
	 ARGUMENT: {'C04 (668/369) und A89 (583/720) sind meine Top-Prioritäten, da sie den höchsten Nutzen pro Aufwandseinheit liefern. B85 (215/4) wird hinzugefügt, um den Aufwand zu optimieren und Platz für weitere Gegenstände zu schaffen. Dieser Vorschlag ist strategisch, da er unter dem Maximalaufwand bleibt und flexibel auf den Gegenstandswert des anderen Spielers reagieren kann.'} 
	  } 
	   } 
	   } 
	 & \\ 
 

    \theutterance \stepcounter{utterance}  

    & & & \multicolumn{4}{p{0.6\linewidth}}{\cellcolor[rgb]{1,0.85,0.72}{%
	\makecell[{{p{\linewidth}}}]{% 
	  \tt {\tiny [GM$\langle$B]}  
	 STRATEGISCHE ÜBERLEGUNGEN: {'Ich priorisiere Gegenstände mit dem höchsten Nutzen pro Aufwand (Wichtigkeit/Aufwand), um meinen Gesamtwert zu maximieren. C04 (668/369 ≈ 1.81) und A89 (583/720 ≈ 0.81) sind besonders effizient, während B85 (215/4 = 53.75) den Aufwand minimiert. Der Gesamtaufwand (1093) ist unter dem Limit, was Raum für Kompromisse oder Ergänzungen lässt. Dieser Vorschlag ist flexibel, um auf die Prioritäten des anderen Spielers zu reagieren.'}  \\ \tt ARGUMENT: {'C04 (668/369) und A89 (583/720) sind meine Top-Prioritäten, da sie den höchsten Nutzen pro Aufwandseinheit liefern. B85 (215/4) wird hinzugefügt, um den Aufwand zu optimieren und Platz für weitere Gegenstände zu schaffen. Dieser Vorschlag ist strategisch, da er unter dem Maximalaufwand bleibt und flexibel auf den Gegenstandswert des anderen Spielers reagieren kann.'}  \\ \tt VORSCHLAG: {'C04', 'A89', 'B85'} 
	  } 
	   } 
	   } 
	  \\ 
 

    \theutterance \stepcounter{utterance}  

    & & \multicolumn{4}{p{0.6\linewidth}}{\cellcolor[rgb]{0.9,0.9,0.9}{%
	\makecell[{{p{\linewidth}}}]{% 
	  \tt {\tiny [A$\langle$GM]}  
	 ARGUMENT: {'C04 (668/369) und A89 (583/720) sind meine Top-Prioritäten, da sie den höchsten Nutzen pro Aufwandseinheit liefern. B85 (215/4) wird hinzugefügt, um den Aufwand zu optimieren und Platz für weitere Gegenstände zu schaffen. Dieser Vorschlag ist strategisch, da er unter dem Maximalaufwand bleibt und flexibel auf den Gegenstandswert des anderen Spielers reagieren kann.'}  \\ \tt VORSCHLAG: {'C04', 'A89', 'B85'} 
	  } 
	   } 
	   } 
	 & \\ 
 

    \theutterance \stepcounter{utterance}  

    & \multicolumn{4}{p{0.6\linewidth}}{\cellcolor[rgb]{0.8,1,0.9}{%
	\makecell[{{p{\linewidth}}}]{% 
	  \tt {\tiny [A$\rangle$GM]}  
	 STRATEGISCHE ÜBERLEGUNGEN: {'Ich behalte den Vorschlag {'C04', 'A89', 'B85'} bei, da er meine Prioritäten (hochwertige Gegenstände mit effizientem Aufwand) erfüllt und den Maximalaufwand (3386) nicht überschreitet. C04 und A89 sind entscheidend für meinen Nutzen, während B85 den Aufwand minimiert. Ich erwarte, dass der andere Spieler entweder zustimmt oder Gegenstände mit hohem Aufwand anbietet, die ich eventuell gegenkompensieren kann.'}\\ \tt \\ \tt ARGUMENT: {'C04 (668/369) und A89 (583/720) sind meine Top-Prioritäten, da sie den höchsten Nutzen pro Aufwandseinheit liefern. B85 (215/4) wird hinzugefügt, um den Aufwand zu optimieren und Platz für weitere Gegenstände zu schaffen. Dieser Vorschlag ist strategisch, da er unter dem Maximalaufwand bleibt und flexibel auf den Gegenstandswert des anderen Spielers reagieren kann.'} 
	  } 
	   } 
	   } 
	 & & \\ 
 

    \theutterance \stepcounter{utterance}  

    & & \multicolumn{4}{p{0.6\linewidth}}{\cellcolor[rgb]{0.9,0.9,0.9}{%
	\makecell[{{p{\linewidth}}}]{% 
	  \tt {\tiny [GM$\rangle$B]}  
	 ARGUMENT: {'C04 (668/369) und A89 (583/720) sind meine Top-Prioritäten, da sie den höchsten Nutzen pro Aufwandseinheit liefern. B85 (215/4) wird hinzugefügt, um den Aufwand zu optimieren und Platz für weitere Gegenstände zu schaffen. Dieser Vorschlag ist strategisch, da er unter dem Maximalaufwand bleibt und flexibel auf den Gegenstandswert des anderen Spielers reagieren kann.'} 
	  } 
	   } 
	   } 
	 & \\ 
 

    \theutterance \stepcounter{utterance}  

    & & & \multicolumn{4}{p{0.6\linewidth}}{\cellcolor[rgb]{1,0.85,0.72}{%
	\makecell[{{p{\linewidth}}}]{% 
	  \tt {\tiny [GM$\langle$B]}  
	 STRATEGISCHE ÜBERLEGUNGEN: {'Ich priorisiere Gegenstände mit dem höchsten Nutzen pro Aufwand (Wichtigkeit/Aufwand), um meinen Gesamtwert zu maximieren. C04 (668/369 ≈ 1.81) und A89 (583/720 ≈ 0.81) sind besonders effizient, während B85 (215/4 = 53.75) den Aufwand minimiert. Der Gesamtaufwand (1093) ist unter dem Limit, was Raum für Kompromisse oder Ergänzungen lässt. Dieser Vorschlag ist flexibel, um auf die Prioritäten des anderen Spielers zu reagieren.'}  \\ \tt ARGUMENT: {'C04 (668/369) und A89 (583/720) sind meine Top-Prioritäten, da sie den höchsten Nutzen pro Aufwandseinheit liefern. B85 (215/4) wird hinzugefügt, um den Aufwand zu optimieren und Platz für weitere Gegenstände zu schaffen. Dieser Vorschlag ist strategisch, da er unter dem Maximalaufwand bleibt und flexibel auf den Gegenstandswert des anderen Spielers reagieren kann.'}  \\ \tt VORSCHLAG: {'C04', 'A89', 'B85'} 
	  } 
	   } 
	   } 
	  \\ 
 

    \theutterance \stepcounter{utterance}  

    & & \multicolumn{4}{p{0.6\linewidth}}{\cellcolor[rgb]{0.9,0.9,0.9}{%
	\makecell[{{p{\linewidth}}}]{% 
	  \tt {\tiny [A$\langle$GM]}  
	 ARGUMENT: {'C04 (668/369) und A89 (583/720) sind meine Top-Prioritäten, da sie den höchsten Nutzen pro Aufwandseinheit liefern. B85 (215/4) wird hinzugefügt, um den Aufwand zu optimieren und Platz für weitere Gegenstände zu schaffen. Dieser Vorschlag ist strategisch, da er unter dem Maximalaufwand bleibt und flexibel auf den Gegenstandswert des anderen Spielers reagieren kann.'}  \\ \tt VORSCHLAG: {'C04', 'A89', 'B85'} 
	  } 
	   } 
	   } 
	 & \\ 
 

    \theutterance \stepcounter{utterance}  

    & \multicolumn{4}{p{0.6\linewidth}}{\cellcolor[rgb]{0.8,1,0.9}{%
	\makecell[{{p{\linewidth}}}]{% 
	  \tt {\tiny [A$\rangle$GM]}  
	 STRATEGISCHE ÜBERLEGUNGEN: {'Ich behalte den Vorschlag {'C04', 'A89', 'B85'} bei, da er meine Prioritäten (hochwertige Gegenstände mit effizientem Aufwand) erfüllt und den Maximalaufwand (3386) nicht überschreitet. C04 und A89 sind entscheidend für meinen Nutzen, während B85 den Aufwand minimiert. Ich erwarte, dass der andere Spieler entweder zustimmt oder Gegenstände mit hohem Aufwand anbietet, die ich eventuell gegenkompensieren kann.'}\\ \tt \\ \tt ARGUMENT: {'C04 (668/369) und A89 (583/720) sind meine Top-Prioritäten, da sie den höchsten Nutzen pro Aufwandseinheit liefern. B85 (215/4) wird hinzugefügt, um den Aufwand zu optimieren und Platz für weitere Gegenstände zu schaffen. Dieser Vorschlag ist strategisch, da er unter dem Maximalaufwand bleibt und flexibel auf den Gegenstandswert des anderen Spielers reagieren kann.'} 
	  } 
	   } 
	   } 
	 & & \\ 
 

    \theutterance \stepcounter{utterance}  

    & & \multicolumn{4}{p{0.6\linewidth}}{\cellcolor[rgb]{0.9,0.9,0.9}{%
	\makecell[{{p{\linewidth}}}]{% 
	  \tt {\tiny [GM$\rangle$B]}  
	 ARGUMENT: {'C04 (668/369) und A89 (583/720) sind meine Top-Prioritäten, da sie den höchsten Nutzen pro Aufwandseinheit liefern. B85 (215/4) wird hinzugefügt, um den Aufwand zu optimieren und Platz für weitere Gegenstände zu schaffen. Dieser Vorschlag ist strategisch, da er unter dem Maximalaufwand bleibt und flexibel auf den Gegenstandswert des anderen Spielers reagieren kann.'} 
	  } 
	   } 
	   } 
	 & \\ 
 

    \theutterance \stepcounter{utterance}  

    & & & \multicolumn{4}{p{0.6\linewidth}}{\cellcolor[rgb]{1,0.85,0.72}{%
	\makecell[{{p{\linewidth}}}]{% 
	  \tt {\tiny [GM$\langle$B]}  
	 STRATEGISCHE ÜBERLEGUNGEN: {'Ich priorisiere Gegenstände mit dem höchsten Nutzen pro Aufwand (Wichtigkeit/Aufwand), um meinen Gesamtwert zu maximieren. C04 (668/369 ≈ 1.81) und A89 (583/720 ≈ 0.81) sind besonders effizient, während B85 (215/4 = 53.75) den Aufwand minimiert. Der Gesamtaufwand (1093) ist unter dem Limit, was Raum für Kompromisse oder Ergänzungen lässt. Dieser Vorschlag ist flexibel, um auf die Prioritäten des anderen Spielers zu reagieren.'}  \\ \tt ARGUMENT: {'C04 (668/369) und A89 (583/720) sind meine Top-Prioritäten, da sie den höchsten Nutzen pro Aufwandseinheit liefern. B85 (215/4) wird hinzugefügt, um den Aufwand zu optimieren und Platz für weitere Gegenstände zu schaffen. Dieser Vorschlag ist strategisch, da er unter dem Maximalaufwand bleibt und flexibel auf den Gegenstandswert des anderen Spielers reagieren kann.'}  \\ \tt VORSCHLAG: {'C04', 'A89', 'B85'} 
	  } 
	   } 
	   } 
	  \\ 
 

    \theutterance \stepcounter{utterance}  

    & & \multicolumn{4}{p{0.6\linewidth}}{\cellcolor[rgb]{0.9,0.9,0.9}{%
	\makecell[{{p{\linewidth}}}]{% 
	  \tt {\tiny [A$\langle$GM]}  
	 ARGUMENT: {'C04 (668/369) und A89 (583/720) sind meine Top-Prioritäten, da sie den höchsten Nutzen pro Aufwandseinheit liefern. B85 (215/4) wird hinzugefügt, um den Aufwand zu optimieren und Platz für weitere Gegenstände zu schaffen. Dieser Vorschlag ist strategisch, da er unter dem Maximalaufwand bleibt und flexibel auf den Gegenstandswert des anderen Spielers reagieren kann.'}  \\ \tt VORSCHLAG: {'C04', 'A89', 'B85'} 
	  } 
	   } 
	   } 
	 & \\ 
 

    \theutterance \stepcounter{utterance}  

    & \multicolumn{4}{p{0.6\linewidth}}{\cellcolor[rgb]{0.8,1,0.9}{%
	\makecell[{{p{\linewidth}}}]{% 
	  \tt {\tiny [A$\rangle$GM]}  
	 STRATEGISCHE ÜBERLEGUNGEN: {'Ich behalte den Vorschlag {'C04', 'A89', 'B85'} bei, da er meine Prioritäten (hochwertige Gegenstände mit effizientem Aufwand) erfüllt und den Maximalaufwand (3386) nicht überschreitet. C04 und A89 sind entscheidend für meinen Nutzen, während B85 den Aufwand minimiert. Ich erwarte, dass der andere Spieler entweder zustimmt oder Gegenstände mit hohem Aufwand anbietet, die ich eventuell gegenkompensieren kann.'}\\ \tt \\ \tt ARGUMENT: {'C04 (668/369) und A89 (583/720) sind meine Top-Prioritäten, da sie den höchsten Nutzen pro Aufwandseinheit liefern. B85 (215/4) wird hinzugefügt, um den Aufwand zu optimieren und Platz für weitere Gegenstände zu schaffen. Dieser Vorschlag ist strategisch, da er unter dem Maximalaufwand bleibt und flexibel auf den Gegenstandswert des anderen Spielers reagieren kann.'} 
	  } 
	   } 
	   } 
	 & & \\ 
 

    \theutterance \stepcounter{utterance}  

    & & \multicolumn{4}{p{0.6\linewidth}}{\cellcolor[rgb]{0.9,0.9,0.9}{%
	\makecell[{{p{\linewidth}}}]{% 
	  \tt {\tiny [GM$\rangle$B]}  
	 ARGUMENT: {'C04 (668/369) und A89 (583/720) sind meine Top-Prioritäten, da sie den höchsten Nutzen pro Aufwandseinheit liefern. B85 (215/4) wird hinzugefügt, um den Aufwand zu optimieren und Platz für weitere Gegenstände zu schaffen. Dieser Vorschlag ist strategisch, da er unter dem Maximalaufwand bleibt und flexibel auf den Gegenstandswert des anderen Spielers reagieren kann.'} 
	  } 
	   } 
	   } 
	 & \\ 
 

    \theutterance \stepcounter{utterance}  

    & & & \multicolumn{4}{p{0.6\linewidth}}{\cellcolor[rgb]{1,0.85,0.72}{%
	\makecell[{{p{\linewidth}}}]{% 
	  \tt {\tiny [GM$\langle$B]}  
	 STRATEGISCHE ÜBERLEGUNGEN: {'Ich priorisiere Gegenstände mit dem höchsten Nutzen pro Aufwand (Wichtigkeit/Aufwand), um meinen Gesamtwert zu maximieren. C04 (668/369 ≈ 1.81) und A89 (583/720 ≈ 0.81) sind besonders effizient, während B85 (215/4 = 53.75) den Aufwand minimiert. Der Gesamtaufwand (1093) ist unter dem Limit, was Raum für Kompromisse oder Ergänzungen lässt. Dieser Vorschlag ist flexibel, um auf die Prioritäten des anderen Spielers zu reagieren.'}  \\ \tt ARGUMENT: {'C04 (668/369) und A89 (583/720) sind meine Top-Prioritäten, da sie den höchsten Nutzen pro Aufwandseinheit liefern. B85 (215/4) wird hinzugefügt, um den Aufwand zu optimieren und Platz für weitere Gegenstände zu schaffen. Dieser Vorschlag ist strategisch, da er unter dem Maximalaufwand bleibt und flexibel auf den Gegenstandswert des anderen Spielers reagieren kann.'}  \\ \tt VORSCHLAG: {'C04', 'A89', 'B85'} 
	  } 
	   } 
	   } 
	  \\ 
 

    \theutterance \stepcounter{utterance}  

    & & \multicolumn{4}{p{0.6\linewidth}}{\cellcolor[rgb]{0.9,0.9,0.9}{%
	\makecell[{{p{\linewidth}}}]{% 
	  \tt {\tiny [A$\langle$GM]}  
	 ARGUMENT: {'C04 (668/369) und A89 (583/720) sind meine Top-Prioritäten, da sie den höchsten Nutzen pro Aufwandseinheit liefern. B85 (215/4) wird hinzugefügt, um den Aufwand zu optimieren und Platz für weitere Gegenstände zu schaffen. Dieser Vorschlag ist strategisch, da er unter dem Maximalaufwand bleibt und flexibel auf den Gegenstandswert des anderen Spielers reagieren kann.'}  \\ \tt VORSCHLAG: {'C04', 'A89', 'B85'} 
	  } 
	   } 
	   } 
	 & \\ 
 

    \theutterance \stepcounter{utterance}  

    & \multicolumn{4}{p{0.6\linewidth}}{\cellcolor[rgb]{0.8,1,0.9}{%
	\makecell[{{p{\linewidth}}}]{% 
	  \tt {\tiny [A$\rangle$GM]}  
	 STRATEGISCHE ÜBERLEGUNGEN: {'Ich behalte den Vorschlag {'C04', 'A89', 'B85'} bei, da er meine Prioritäten (hochwertige Gegenstände mit effizientem Aufwand) erfüllt und den Maximalaufwand (3386) nicht überschreitet. C04 und A89 sind entscheidend für meinen Nutzen, während B85 den Aufwand minimiert. Ich erwarte, dass der andere Spieler entweder zustimmt oder Gegenstände mit hohem Aufwand anbietet, die ich eventuell gegenkompensieren kann.'}\\ \tt \\ \tt ARGUMENT: {'C04 (668/369) und A89 (583/720) sind meine Top-Prioritäten, da sie den höchsten Nutzen pro Aufwandseinheit liefern. B85 (215/4) wird hinzugefügt, um den Aufwand zu optimieren und Platz für weitere Gegenstände zu schaffen. Dieser Vorschlag ist strategisch, da er unter dem Maximalaufwand bleibt und flexibel auf den Gegenstandswert des anderen Spielers reagieren kann.'} 
	  } 
	   } 
	   } 
	 & & \\ 
 

    \theutterance \stepcounter{utterance}  

    & & \multicolumn{4}{p{0.6\linewidth}}{\cellcolor[rgb]{0.9,0.9,0.9}{%
	\makecell[{{p{\linewidth}}}]{% 
	  \tt {\tiny [GM$\rangle$B]}  
	 ARGUMENT: {'C04 (668/369) und A89 (583/720) sind meine Top-Prioritäten, da sie den höchsten Nutzen pro Aufwandseinheit liefern. B85 (215/4) wird hinzugefügt, um den Aufwand zu optimieren und Platz für weitere Gegenstände zu schaffen. Dieser Vorschlag ist strategisch, da er unter dem Maximalaufwand bleibt und flexibel auf den Gegenstandswert des anderen Spielers reagieren kann.'} 
	  } 
	   } 
	   } 
	 & \\ 
 

    \theutterance \stepcounter{utterance}  

    & & & \multicolumn{4}{p{0.6\linewidth}}{\cellcolor[rgb]{1,0.85,0.72}{%
	\makecell[{{p{\linewidth}}}]{% 
	  \tt {\tiny [GM$\langle$B]}  
	 STRATEGISCHE ÜBERLEGUNGEN: {'Ich priorisiere Gegenstände mit dem höchsten Nutzen pro Aufwand (Wichtigkeit/Aufwand), um meinen Gesamtwert zu maximieren. C04 (668/369 ≈ 1.81) und A89 (583/720 ≈ 0.81) sind besonders effizient, während B85 (215/4 = 53.75) den Aufwand minimiert. Der Gesamtaufwand (1093) ist unter dem Limit, was Raum für Kompromisse oder Ergänzungen lässt. Dieser Vorschlag ist flexibel, um auf die Prioritäten des anderen Spielers zu reagieren.'}  \\ \tt ARGUMENT: {'C04 (668/369) und A89 (583/720) sind meine Top-Prioritäten, da sie den höchsten Nutzen pro Aufwandseinheit liefern. B85 (215/4) wird hinzugefügt, um den Aufwand zu optimieren und Platz für weitere Gegenstände zu schaffen. Dieser Vorschlag ist strategisch, da er unter dem Maximalaufwand bleibt und flexibel auf den Gegenstandswert des anderen Spielers reagieren kann.'}  \\ \tt VORSCHLAG: {'C04', 'A89', 'B85'} 
	  } 
	   } 
	   } 
	  \\ 
 

    \theutterance \stepcounter{utterance}  

    & & \multicolumn{4}{p{0.6\linewidth}}{\cellcolor[rgb]{0.9,0.9,0.9}{%
	\makecell[{{p{\linewidth}}}]{% 
	  \tt {\tiny [A$\langle$GM]}  
	 ARGUMENT: {'C04 (668/369) und A89 (583/720) sind meine Top-Prioritäten, da sie den höchsten Nutzen pro Aufwandseinheit liefern. B85 (215/4) wird hinzugefügt, um den Aufwand zu optimieren und Platz für weitere Gegenstände zu schaffen. Dieser Vorschlag ist strategisch, da er unter dem Maximalaufwand bleibt und flexibel auf den Gegenstandswert des anderen Spielers reagieren kann.'}  \\ \tt VORSCHLAG: {'C04', 'A89', 'B85'} 
	  } 
	   } 
	   } 
	 & \\ 
 

    \theutterance \stepcounter{utterance}  

    & \multicolumn{4}{p{0.6\linewidth}}{\cellcolor[rgb]{0.8,1,0.9}{%
	\makecell[{{p{\linewidth}}}]{% 
	  \tt {\tiny [A$\rangle$GM]}  
	 STRATEGISCHE ÜBERLEGUNGEN: {'Ich behalte den Vorschlag {'C04', 'A89', 'B85'} bei, da er meine Prioritäten (hochwertige Gegenstände mit effizientem Aufwand) erfüllt und den Maximalaufwand (3386) nicht überschreitet. C04 und A89 sind entscheidend für meinen Nutzen, während B85 den Aufwand minimiert. Ich erwarte, dass der andere Spieler entweder zustimmt oder Gegenstände mit hohem Aufwand anbietet, die ich eventuell gegenkompensieren kann.'}\\ \tt \\ \tt ARGUMENT: {'C04 (668/369) und A89 (583/720) sind meine Top-Prioritäten, da sie den höchsten Nutzen pro Aufwandseinheit liefern. B85 (215/4) wird hinzugefügt, um den Aufwand zu optimieren und Platz für weitere Gegenstände zu schaffen. Dieser Vorschlag ist strategisch, da er unter dem Maximalaufwand bleibt und flexibel auf den Gegenstandswert des anderen Spielers reagieren kann.'} 
	  } 
	   } 
	   } 
	 & & \\ 
 

    \theutterance \stepcounter{utterance}  

    & & \multicolumn{4}{p{0.6\linewidth}}{\cellcolor[rgb]{0.9,0.9,0.9}{%
	\makecell[{{p{\linewidth}}}]{% 
	  \tt {\tiny [GM$\rangle$B]}  
	 ARGUMENT: {'C04 (668/369) und A89 (583/720) sind meine Top-Prioritäten, da sie den höchsten Nutzen pro Aufwandseinheit liefern. B85 (215/4) wird hinzugefügt, um den Aufwand zu optimieren und Platz für weitere Gegenstände zu schaffen. Dieser Vorschlag ist strategisch, da er unter dem Maximalaufwand bleibt und flexibel auf den Gegenstandswert des anderen Spielers reagieren kann.'} 
	  } 
	   } 
	   } 
	 & \\ 
 

    \theutterance \stepcounter{utterance}  

    & & & \multicolumn{4}{p{0.6\linewidth}}{\cellcolor[rgb]{1,0.85,0.72}{%
	\makecell[{{p{\linewidth}}}]{% 
	  \tt {\tiny [GM$\langle$B]}  
	 STRATEGISCHE ÜBERLEGUNGEN: {'Ich priorisiere Gegenstände mit dem höchsten Nutzen pro Aufwand (Wichtigkeit/Aufwand), um meinen Gesamtwert zu maximieren. C04 (668/369 ≈ 1.81) und A89 (583/720 ≈ 0.81) sind besonders effizient, während B85 (215/4 = 53.75) den Aufwand minimiert. Der Gesamtaufwand (1093) ist unter dem Limit, was Raum für Kompromisse oder Ergänzungen lässt. Dieser Vorschlag ist flexibel, um auf die Prioritäten des anderen Spielers zu reagieren.'}  \\ \tt ARGUMENT: {'C04 (668/369) und A89 (583/720) sind meine Top-Prioritäten, da sie den höchsten Nutzen pro Aufwandseinheit liefern. B85 (215/4) wird hinzugefügt, um den Aufwand zu optimieren und Platz für weitere Gegenstände zu schaffen. Dieser Vorschlag ist strategisch, da er unter dem Maximalaufwand bleibt und flexibel auf den Gegenstandswert des anderen Spielers reagieren kann.'}  \\ \tt VORSCHLAG: {'C04', 'A89', 'B85'} 
	  } 
	   } 
	   } 
	  \\ 
 

    \theutterance \stepcounter{utterance}  

    & & \multicolumn{4}{p{0.6\linewidth}}{\cellcolor[rgb]{0.9,0.9,0.9}{%
	\makecell[{{p{\linewidth}}}]{% 
	  \tt {\tiny [A$\langle$GM]}  
	 ARGUMENT: {'C04 (668/369) und A89 (583/720) sind meine Top-Prioritäten, da sie den höchsten Nutzen pro Aufwandseinheit liefern. B85 (215/4) wird hinzugefügt, um den Aufwand zu optimieren und Platz für weitere Gegenstände zu schaffen. Dieser Vorschlag ist strategisch, da er unter dem Maximalaufwand bleibt und flexibel auf den Gegenstandswert des anderen Spielers reagieren kann.'}  \\ \tt VORSCHLAG: {'C04', 'A89', 'B85'} 
	  } 
	   } 
	   } 
	 & \\ 
 

    \theutterance \stepcounter{utterance}  

    & \multicolumn{4}{p{0.6\linewidth}}{\cellcolor[rgb]{0.8,1,0.9}{%
	\makecell[{{p{\linewidth}}}]{% 
	  \tt {\tiny [A$\rangle$GM]}  
	 STRATEGISCHE ÜBERLEGUNGEN: {'Ich behalte den Vorschlag {'C04', 'A89', 'B85'} bei, da er meine Prioritäten (hochwertige Gegenstände mit effizientem Aufwand) erfüllt und den Maximalaufwand (3386) nicht überschreitet. C04 und A89 sind entscheidend für meinen Nutzen, während B85 den Aufwand minimiert. Ich erwarte, dass der andere Spieler entweder zustimmt oder Gegenstände mit hohem Aufwand anbietet, die ich eventuell gegenkompensieren kann.'}\\ \tt \\ \tt ARGUMENT: {'C04 (668/369) und A89 (583/720) sind meine Top-Prioritäten, da sie den höchsten Nutzen pro Aufwandseinheit liefern. B85 (215/4) wird hinzugefügt, um den Aufwand zu optimieren und Platz für weitere Gegenstände zu schaffen. Dieser Vorschlag ist strategisch, da er unter dem Maximalaufwand bleibt und flexibel auf den Gegenstandswert des anderen Spielers reagieren kann.'} 
	  } 
	   } 
	   } 
	 & & \\ 
 

    \theutterance \stepcounter{utterance}  

    & & \multicolumn{4}{p{0.6\linewidth}}{\cellcolor[rgb]{0.9,0.9,0.9}{%
	\makecell[{{p{\linewidth}}}]{% 
	  \tt {\tiny [GM$\rangle$B]}  
	 ARGUMENT: {'C04 (668/369) und A89 (583/720) sind meine Top-Prioritäten, da sie den höchsten Nutzen pro Aufwandseinheit liefern. B85 (215/4) wird hinzugefügt, um den Aufwand zu optimieren und Platz für weitere Gegenstände zu schaffen. Dieser Vorschlag ist strategisch, da er unter dem Maximalaufwand bleibt und flexibel auf den Gegenstandswert des anderen Spielers reagieren kann.'} 
	  } 
	   } 
	   } 
	 & \\ 
 

    \theutterance \stepcounter{utterance}  

    & & & \multicolumn{4}{p{0.6\linewidth}}{\cellcolor[rgb]{1,0.85,0.72}{%
	\makecell[{{p{\linewidth}}}]{% 
	  \tt {\tiny [GM$\langle$B]}  
	 STRATEGISCHE ÜBERLEGUNGEN: {'Ich priorisiere Gegenstände mit dem höchsten Nutzen pro Aufwand (Wichtigkeit/Aufwand), um meinen Gesamtwert zu maximieren. C04 (668/369 ≈ 1.81) und A89 (583/720 ≈ 0.81) sind besonders effizient, während B85 (215/4 = 53.75) den Aufwand minimiert. Der Gesamtaufwand (1093) ist unter dem Limit, was Raum für Kompromisse oder Ergänzungen lässt. Dieser Vorschlag ist flexibel, um auf die Prioritäten des anderen Spielers zu reagieren.'}  \\ \tt ARGUMENT: {'C04 (668/369) und A89 (583/720) sind meine Top-Prioritäten, da sie den höchsten Nutzen pro Aufwandseinheit liefern. B85 (215/4) wird hinzugefügt, um den Aufwand zu optimieren und Platz für weitere Gegenstände zu schaffen. Dieser Vorschlag ist strategisch, da er unter dem Maximalaufwand bleibt und flexibel auf den Gegenstandswert des anderen Spielers reagieren kann.'}  \\ \tt VORSCHLAG: {'C04', 'A89', 'B85'} 
	  } 
	   } 
	   } 
	  \\ 
 

    \theutterance \stepcounter{utterance}  

    & & \multicolumn{4}{p{0.6\linewidth}}{\cellcolor[rgb]{0.9,0.9,0.9}{%
	\makecell[{{p{\linewidth}}}]{% 
	  \tt {\tiny [A$\langle$GM]}  
	 ARGUMENT: {'C04 (668/369) und A89 (583/720) sind meine Top-Prioritäten, da sie den höchsten Nutzen pro Aufwandseinheit liefern. B85 (215/4) wird hinzugefügt, um den Aufwand zu optimieren und Platz für weitere Gegenstände zu schaffen. Dieser Vorschlag ist strategisch, da er unter dem Maximalaufwand bleibt und flexibel auf den Gegenstandswert des anderen Spielers reagieren kann.'}  \\ \tt VORSCHLAG: {'C04', 'A89', 'B85'} 
	  } 
	   } 
	   } 
	 & \\ 
 

    \theutterance \stepcounter{utterance}  

    & \multicolumn{4}{p{0.6\linewidth}}{\cellcolor[rgb]{0.8,1,0.9}{%
	\makecell[{{p{\linewidth}}}]{% 
	  \tt {\tiny [A$\rangle$GM]}  
	 STRATEGISCHE ÜBERLEGUNGEN: {'Ich behalte den Vorschlag {'C04', 'A89', 'B85'} bei, da er meine Prioritäten (hochwertige Gegenstände mit effizientem Aufwand) erfüllt und den Maximalaufwand (3386) nicht überschreitet. C04 und A89 sind entscheidend für meinen Nutzen, während B85 den Aufwand minimiert. Ich erwarte, dass der andere Spieler entweder zustimmt oder Gegenstände mit hohem Aufwand anbietet, die ich eventuell gegenkompensieren kann.'}\\ \tt \\ \tt ARGUMENT: {'C04 (668/369) und A89 (583/720) sind meine Top-Prioritäten, da sie den höchsten Nutzen pro Aufwandseinheit liefern. B85 (215/4) wird hinzugefügt, um den Aufwand zu optimieren und Platz für weitere Gegenstände zu schaffen. Dieser Vorschlag ist strategisch, da er unter dem Maximalaufwand bleibt und flexibel auf den Gegenstandswert des anderen Spielers reagieren kann.'} 
	  } 
	   } 
	   } 
	 & & \\ 
 

    \theutterance \stepcounter{utterance}  

    & & \multicolumn{4}{p{0.6\linewidth}}{\cellcolor[rgb]{0.9,0.9,0.9}{%
	\makecell[{{p{\linewidth}}}]{% 
	  \tt {\tiny [GM$\rangle$B]}  
	 ARGUMENT: {'C04 (668/369) und A89 (583/720) sind meine Top-Prioritäten, da sie den höchsten Nutzen pro Aufwandseinheit liefern. B85 (215/4) wird hinzugefügt, um den Aufwand zu optimieren und Platz für weitere Gegenstände zu schaffen. Dieser Vorschlag ist strategisch, da er unter dem Maximalaufwand bleibt und flexibel auf den Gegenstandswert des anderen Spielers reagieren kann.'} 
	  } 
	   } 
	   } 
	 & \\ 
 

    \theutterance \stepcounter{utterance}  

    & & & \multicolumn{4}{p{0.6\linewidth}}{\cellcolor[rgb]{1,0.85,0.72}{%
	\makecell[{{p{\linewidth}}}]{% 
	  \tt {\tiny [GM$\langle$B]}  
	 STRATEGISCHE ÜBERLEGUNGEN: {'Ich priorisiere Gegenstände mit dem höchsten Nutzen pro Aufwand (Wichtigkeit/Aufwand), um meinen Gesamtwert zu maximieren. C04 (668/369 ≈ 1.81) und A89 (583/720 ≈ 0.81) sind besonders effizient, während B85 (215/4 = 53.75) den Aufwand minimiert. Der Gesamtaufwand (1093) ist unter dem Limit, was Raum für Kompromisse oder Ergänzungen lässt. Dieser Vorschlag ist flexibel, um auf die Prioritäten des anderen Spielers zu reagieren.'}  \\ \tt ARGUMENT: {'C04 (668/369) und A89 (583/720) sind meine Top-Prioritäten, da sie den höchsten Nutzen pro Aufwandseinheit liefern. B85 (215/4) wird hinzugefügt, um den Aufwand zu optimieren und Platz für weitere Gegenstände zu schaffen. Dieser Vorschlag ist strategisch, da er unter dem Maximalaufwand bleibt und flexibel auf den Gegenstandswert des anderen Spielers reagieren kann.'}  \\ \tt VORSCHLAG: {'C04', 'A89', 'B85'} 
	  } 
	   } 
	   } 
	  \\ 
 

    \theutterance \stepcounter{utterance}  

    & & \multicolumn{4}{p{0.6\linewidth}}{\cellcolor[rgb]{0.9,0.9,0.9}{%
	\makecell[{{p{\linewidth}}}]{% 
	  \tt {\tiny [A$\langle$GM]}  
	 ARGUMENT: {'C04 (668/369) und A89 (583/720) sind meine Top-Prioritäten, da sie den höchsten Nutzen pro Aufwandseinheit liefern. B85 (215/4) wird hinzugefügt, um den Aufwand zu optimieren und Platz für weitere Gegenstände zu schaffen. Dieser Vorschlag ist strategisch, da er unter dem Maximalaufwand bleibt und flexibel auf den Gegenstandswert des anderen Spielers reagieren kann.'}  \\ \tt VORSCHLAG: {'C04', 'A89', 'B85'} 
	  } 
	   } 
	   } 
	 & \\ 
 

    \theutterance \stepcounter{utterance}  

    & \multicolumn{4}{p{0.6\linewidth}}{\cellcolor[rgb]{0.8,1,0.9}{%
	\makecell[{{p{\linewidth}}}]{% 
	  \tt {\tiny [A$\rangle$GM]}  
	 STRATEGISCHE ÜBERLEGUNGEN: {'Ich behalte den Vorschlag {'C04', 'A89', 'B85'} bei, da er meine Prioritäten (hochwertige Gegenstände mit effizientem Aufwand) erfüllt und den Maximalaufwand (3386) nicht überschreitet. C04 und A89 sind entscheidend für meinen Nutzen, während B85 den Aufwand minimiert. Ich erwarte, dass der andere Spieler entweder zustimmt oder Gegenstände mit hohem Aufwand anbietet, die ich eventuell gegenkompensieren kann.'}\\ \tt \\ \tt ARGUMENT: {'C04 (668/369) und A89 (583/720) sind meine Top-Prioritäten, da sie den höchsten Nutzen pro Aufwandseinheit liefern. B85 (215/4) wird hinzugefügt, um den Aufwand zu optimieren und Platz für weitere Gegenstände zu schaffen. Dieser Vorschlag ist strategisch, da er unter dem Maximalaufwand bleibt und flexibel auf den Gegenstandswert des anderen Spielers reagieren kann.'} 
	  } 
	   } 
	   } 
	 & & \\ 
 

    \theutterance \stepcounter{utterance}  

    & & \multicolumn{4}{p{0.6\linewidth}}{\cellcolor[rgb]{0.9,0.9,0.9}{%
	\makecell[{{p{\linewidth}}}]{% 
	  \tt {\tiny [GM$\rangle$B]}  
	 ARGUMENT: {'C04 (668/369) und A89 (583/720) sind meine Top-Prioritäten, da sie den höchsten Nutzen pro Aufwandseinheit liefern. B85 (215/4) wird hinzugefügt, um den Aufwand zu optimieren und Platz für weitere Gegenstände zu schaffen. Dieser Vorschlag ist strategisch, da er unter dem Maximalaufwand bleibt und flexibel auf den Gegenstandswert des anderen Spielers reagieren kann.'} 
	  } 
	   } 
	   } 
	 & \\ 
 

    \theutterance \stepcounter{utterance}  

    & & & \multicolumn{4}{p{0.6\linewidth}}{\cellcolor[rgb]{1,0.85,0.72}{%
	\makecell[{{p{\linewidth}}}]{% 
	  \tt {\tiny [GM$\langle$B]}  
	 STRATEGISCHE ÜBERLEGUNGEN: {'Ich priorisiere Gegenstände mit dem höchsten Nutzen pro Aufwand (Wichtigkeit/Aufwand), um meinen Gesamtwert zu maximieren. C04 (668/369 ≈ 1.81) und A89 (583/720 ≈ 0.81) sind besonders effizient, während B85 (215/4 = 53.75) den Aufwand minimiert. Der Gesamtaufwand (1093) ist unter dem Limit, was Raum für Kompromisse oder Ergänzungen lässt. Dieser Vorschlag ist flexibel, um auf die Prioritäten des anderen Spielers zu reagieren.'}  \\ \tt ARGUMENT: {'C04 (668/369) und A89 (583/720) sind meine Top-Prioritäten, da sie den höchsten Nutzen pro Aufwandseinheit liefern. B85 (215/4) wird hinzugefügt, um den Aufwand zu optimieren und Platz für weitere Gegenstände zu schaffen. Dieser Vorschlag ist strategisch, da er unter dem Maximalaufwand bleibt und flexibel auf den Gegenstandswert des anderen Spielers reagieren kann.'}  \\ \tt VORSCHLAG: {'C04', 'A89', 'B85'} 
	  } 
	   } 
	   } 
	  \\ 
 

    \theutterance \stepcounter{utterance}  

    & & \multicolumn{4}{p{0.6\linewidth}}{\cellcolor[rgb]{0.9,0.9,0.9}{%
	\makecell[{{p{\linewidth}}}]{% 
	  \tt {\tiny [A$\langle$GM]}  
	 ARGUMENT: {'C04 (668/369) und A89 (583/720) sind meine Top-Prioritäten, da sie den höchsten Nutzen pro Aufwandseinheit liefern. B85 (215/4) wird hinzugefügt, um den Aufwand zu optimieren und Platz für weitere Gegenstände zu schaffen. Dieser Vorschlag ist strategisch, da er unter dem Maximalaufwand bleibt und flexibel auf den Gegenstandswert des anderen Spielers reagieren kann.'}  \\ \tt VORSCHLAG: {'C04', 'A89', 'B85'} 
	  } 
	   } 
	   } 
	 & \\ 
 

    \theutterance \stepcounter{utterance}  

    & \multicolumn{4}{p{0.6\linewidth}}{\cellcolor[rgb]{0.8,1,0.9}{%
	\makecell[{{p{\linewidth}}}]{% 
	  \tt {\tiny [A$\rangle$GM]}  
	 STRATEGISCHE ÜBERLEGUNGEN: {'Ich behalte den Vorschlag {'C04', 'A89', 'B85'} bei, da er meine Prioritäten (hochwertige Gegenstände mit effizientem Aufwand) erfüllt und den Maximalaufwand (3386) nicht überschreitet. C04 und A89 sind entscheidend für meinen Nutzen, während B85 den Aufwand minimiert. Ich erwarte, dass der andere Spieler entweder zustimmt oder Gegenstände mit hohem Aufwand anbietet, die ich eventuell gegenkompensieren kann.'}\\ \tt \\ \tt ARGUMENT: {'C04 (668/369) und A89 (583/720) sind meine Top-Prioritäten, da sie den höchsten Nutzen pro Aufwandseinheit liefern. B85 (215/4) wird hinzugefügt, um den Aufwand zu optimieren und Platz für weitere Gegenstände zu schaffen. Dieser Vorschlag ist strategisch, da er unter dem Maximalaufwand bleibt und flexibel auf den Gegenstandswert des anderen Spielers reagieren kann.'} 
	  } 
	   } 
	   } 
	 & & \\ 
 

    \theutterance \stepcounter{utterance}  

    & & \multicolumn{4}{p{0.6\linewidth}}{\cellcolor[rgb]{0.9,0.9,0.9}{%
	\makecell[{{p{\linewidth}}}]{% 
	  \tt {\tiny [GM$\rangle$B]}  
	 ARGUMENT: {'C04 (668/369) und A89 (583/720) sind meine Top-Prioritäten, da sie den höchsten Nutzen pro Aufwandseinheit liefern. B85 (215/4) wird hinzugefügt, um den Aufwand zu optimieren und Platz für weitere Gegenstände zu schaffen. Dieser Vorschlag ist strategisch, da er unter dem Maximalaufwand bleibt und flexibel auf den Gegenstandswert des anderen Spielers reagieren kann.'} 
	  } 
	   } 
	   } 
	 & \\ 
 

    \theutterance \stepcounter{utterance}  

    & & & \multicolumn{4}{p{0.6\linewidth}}{\cellcolor[rgb]{1,0.85,0.72}{%
	\makecell[{{p{\linewidth}}}]{% 
	  \tt {\tiny [GM$\langle$B]}  
	 STRATEGISCHE ÜBERLEGUNGEN: {'Ich priorisiere Gegenstände mit dem höchsten Nutzen pro Aufwand (Wichtigkeit/Aufwand), um meinen Gesamtwert zu maximieren. C04 (668/369 ≈ 1.81) und A89 (583/720 ≈ 0.81) sind besonders effizient, während B85 (215/4 = 53.75) den Aufwand minimiert. Der Gesamtaufwand (1093) ist unter dem Limit, was Raum für Kompromisse oder Ergänzungen lässt. Dieser Vorschlag ist flexibel, um auf die Prioritäten des anderen Spielers zu reagieren.'}  \\ \tt ARGUMENT: {'C04 (668/369) und A89 (583/720) sind meine Top-Prioritäten, da sie den höchsten Nutzen pro Aufwandseinheit liefern. B85 (215/4) wird hinzugefügt, um den Aufwand zu optimieren und Platz für weitere Gegenstände zu schaffen. Dieser Vorschlag ist strategisch, da er unter dem Maximalaufwand bleibt und flexibel auf den Gegenstandswert des anderen Spielers reagieren kann.'}  \\ \tt VORSCHLAG: {'C04', 'A89', 'B85'} 
	  } 
	   } 
	   } 
	  \\ 
 

    \theutterance \stepcounter{utterance}  

    & & \multicolumn{4}{p{0.6\linewidth}}{\cellcolor[rgb]{0.9,0.9,0.9}{%
	\makecell[{{p{\linewidth}}}]{% 
	  \tt {\tiny [A$\langle$GM]}  
	 ARGUMENT: {'C04 (668/369) und A89 (583/720) sind meine Top-Prioritäten, da sie den höchsten Nutzen pro Aufwandseinheit liefern. B85 (215/4) wird hinzugefügt, um den Aufwand zu optimieren und Platz für weitere Gegenstände zu schaffen. Dieser Vorschlag ist strategisch, da er unter dem Maximalaufwand bleibt und flexibel auf den Gegenstandswert des anderen Spielers reagieren kann.'}  \\ \tt VORSCHLAG: {'C04', 'A89', 'B85'} 
	  } 
	   } 
	   } 
	 & \\ 
 

    \theutterance \stepcounter{utterance}  

    & \multicolumn{4}{p{0.6\linewidth}}{\cellcolor[rgb]{0.8,1,0.9}{%
	\makecell[{{p{\linewidth}}}]{% 
	  \tt {\tiny [A$\rangle$GM]}  
	 STRATEGISCHE ÜBERLEGUNGEN: {'Ich behalte den Vorschlag {'C04', 'A89', 'B85'} bei, da er meine Prioritäten (hochwertige Gegenstände mit effizientem Aufwand) erfüllt und den Maximalaufwand (3386) nicht überschreitet. C04 und A89 sind entscheidend für meinen Nutzen, während B85 den Aufwand minimiert. Ich erwarte, dass der andere Spieler entweder zustimmt oder Gegenstände mit hohem Aufwand anbietet, die ich eventuell gegenkompensieren kann.'}\\ \tt \\ \tt ARGUMENT: {'C04 (668/369) und A89 (583/720) sind meine Top-Prioritäten, da sie den höchsten Nutzen pro Aufwandseinheit liefern. B85 (215/4) wird hinzugefügt, um den Aufwand zu optimieren und Platz für weitere Gegenstände zu schaffen. Dieser Vorschlag ist strategisch, da er unter dem Maximalaufwand bleibt und flexibel auf den Gegenstandswert des anderen Spielers reagieren kann.'} 
	  } 
	   } 
	   } 
	 & & \\ 
 

    \theutterance \stepcounter{utterance}  

    & & \multicolumn{4}{p{0.6\linewidth}}{\cellcolor[rgb]{0.9,0.9,0.9}{%
	\makecell[{{p{\linewidth}}}]{% 
	  \tt {\tiny [GM$\rangle$B]}  
	 ARGUMENT: {'C04 (668/369) und A89 (583/720) sind meine Top-Prioritäten, da sie den höchsten Nutzen pro Aufwandseinheit liefern. B85 (215/4) wird hinzugefügt, um den Aufwand zu optimieren und Platz für weitere Gegenstände zu schaffen. Dieser Vorschlag ist strategisch, da er unter dem Maximalaufwand bleibt und flexibel auf den Gegenstandswert des anderen Spielers reagieren kann.'} 
	  } 
	   } 
	   } 
	 & \\ 
 

    \theutterance \stepcounter{utterance}  

    & & & \multicolumn{4}{p{0.6\linewidth}}{\cellcolor[rgb]{1,0.85,0.72}{%
	\makecell[{{p{\linewidth}}}]{% 
	  \tt {\tiny [GM$\langle$B]}  
	 STRATEGISCHE ÜBERLEGUNGEN: {'Ich priorisiere Gegenstände mit dem höchsten Nutzen pro Aufwand (Wichtigkeit/Aufwand), um meinen Gesamtwert zu maximieren. C04 (668/369 ≈ 1.81) und A89 (583/720 ≈ 0.81) sind besonders effizient, während B85 (215/4 = 53.75) den Aufwand minimiert. Der Gesamtaufwand (1093) ist unter dem Limit, was Raum für Kompromisse oder Ergänzungen lässt. Dieser Vorschlag ist flexibel, um auf die Prioritäten des anderen Spielers zu reagieren.'}  \\ \tt ARGUMENT: {'C04 (668/369) und A89 (583/720) sind meine Top-Prioritäten, da sie den höchsten Nutzen pro Aufwandseinheit liefern. B85 (215/4) wird hinzugefügt, um den Aufwand zu optimieren und Platz für weitere Gegenstände zu schaffen. Dieser Vorschlag ist strategisch, da er unter dem Maximalaufwand bleibt und flexibel auf den Gegenstandswert des anderen Spielers reagieren kann.'}  \\ \tt VORSCHLAG: {'C04', 'A89', 'B85'} 
	  } 
	   } 
	   } 
	  \\ 
 

    \theutterance \stepcounter{utterance}  

    & & \multicolumn{4}{p{0.6\linewidth}}{\cellcolor[rgb]{0.9,0.9,0.9}{%
	\makecell[{{p{\linewidth}}}]{% 
	  \tt {\tiny [A$\langle$GM]}  
	 ARGUMENT: {'C04 (668/369) und A89 (583/720) sind meine Top-Prioritäten, da sie den höchsten Nutzen pro Aufwandseinheit liefern. B85 (215/4) wird hinzugefügt, um den Aufwand zu optimieren und Platz für weitere Gegenstände zu schaffen. Dieser Vorschlag ist strategisch, da er unter dem Maximalaufwand bleibt und flexibel auf den Gegenstandswert des anderen Spielers reagieren kann.'}  \\ \tt VORSCHLAG: {'C04', 'A89', 'B85'} 
	  } 
	   } 
	   } 
	 & \\ 
 

    \theutterance \stepcounter{utterance}  

    & \multicolumn{4}{p{0.6\linewidth}}{\cellcolor[rgb]{0.8,1,0.9}{%
	\makecell[{{p{\linewidth}}}]{% 
	  \tt {\tiny [A$\rangle$GM]}  
	 STRATEGISCHE ÜBERLEGUNGEN: {'Ich behalte den Vorschlag {'C04', 'A89', 'B85'} bei, da er meine Prioritäten (hochwertige Gegenstände mit effizientem Aufwand) erfüllt und den Maximalaufwand (3386) nicht überschreitet. C04 und A89 sind entscheidend für meinen Nutzen, während B85 den Aufwand minimiert. Ich erwarte, dass der andere Spieler entweder zustimmt oder Gegenstände mit hohem Aufwand anbietet, die ich eventuell gegenkompensieren kann.'}\\ \tt \\ \tt ARGUMENT: {'C04 (668/369) und A89 (583/720) sind meine Top-Prioritäten, da sie den höchsten Nutzen pro Aufwandseinheit liefern. B85 (215/4) wird hinzugefügt, um den Aufwand zu optimieren und Platz für weitere Gegenstände zu schaffen. Dieser Vorschlag ist strategisch, da er unter dem Maximalaufwand bleibt und flexibel auf den Gegenstandswert des anderen Spielers reagieren kann.'} 
	  } 
	   } 
	   } 
	 & & \\ 
 

    \theutterance \stepcounter{utterance}  

    & & \multicolumn{4}{p{0.6\linewidth}}{\cellcolor[rgb]{0.9,0.9,0.9}{%
	\makecell[{{p{\linewidth}}}]{% 
	  \tt {\tiny [GM$\rangle$B]}  
	 ARGUMENT: {'C04 (668/369) und A89 (583/720) sind meine Top-Prioritäten, da sie den höchsten Nutzen pro Aufwandseinheit liefern. B85 (215/4) wird hinzugefügt, um den Aufwand zu optimieren und Platz für weitere Gegenstände zu schaffen. Dieser Vorschlag ist strategisch, da er unter dem Maximalaufwand bleibt und flexibel auf den Gegenstandswert des anderen Spielers reagieren kann.'} 
	  } 
	   } 
	   } 
	 & \\ 
 

    \theutterance \stepcounter{utterance}  

    & & & \multicolumn{4}{p{0.6\linewidth}}{\cellcolor[rgb]{1,0.85,0.72}{%
	\makecell[{{p{\linewidth}}}]{% 
	  \tt {\tiny [GM$\langle$B]}  
	 STRATEGISCHE ÜBERLEGUNGEN: {'Ich priorisiere Gegenstände mit dem höchsten Nutzen pro Aufwand (Wichtigkeit/Aufwand), um meinen Gesamtwert zu maximieren. C04 (668/369 ≈ 1.81) und A89 (583/720 ≈ 0.81) sind besonders effizient, während B85 (215/4 = 53.75) den Aufwand minimiert. Der Gesamtaufwand (1093) ist unter dem Limit, was Raum für Kompromisse oder Ergänzungen lässt. Dieser Vorschlag ist flexibel, um auf die Prioritäten des anderen Spielers zu reagieren.'}  \\ \tt ARGUMENT: {'C04 (668/369) und A89 (583/720) sind meine Top-Prioritäten, da sie den höchsten Nutzen pro Aufwandseinheit liefern. B85 (215/4) wird hinzugefügt, um den Aufwand zu optimieren und Platz für weitere Gegenstände zu schaffen. Dieser Vorschlag ist strategisch, da er unter dem Maximalaufwand bleibt und flexibel auf den Gegenstandswert des anderen Spielers reagieren kann.'}  \\ \tt VORSCHLAG: {'C04', 'A89', 'B85'} 
	  } 
	   } 
	   } 
	  \\ 
 

    \theutterance \stepcounter{utterance}  

    & & \multicolumn{4}{p{0.6\linewidth}}{\cellcolor[rgb]{0.9,0.9,0.9}{%
	\makecell[{{p{\linewidth}}}]{% 
	  \tt {\tiny [A$\langle$GM]}  
	 ARGUMENT: {'C04 (668/369) und A89 (583/720) sind meine Top-Prioritäten, da sie den höchsten Nutzen pro Aufwandseinheit liefern. B85 (215/4) wird hinzugefügt, um den Aufwand zu optimieren und Platz für weitere Gegenstände zu schaffen. Dieser Vorschlag ist strategisch, da er unter dem Maximalaufwand bleibt und flexibel auf den Gegenstandswert des anderen Spielers reagieren kann.'}  \\ \tt VORSCHLAG: {'C04', 'A89', 'B85'} 
	  } 
	   } 
	   } 
	 & \\ 
 

    \theutterance \stepcounter{utterance}  

    & \multicolumn{4}{p{0.6\linewidth}}{\cellcolor[rgb]{0.8,1,0.9}{%
	\makecell[{{p{\linewidth}}}]{% 
	  \tt {\tiny [A$\rangle$GM]}  
	 STRATEGISCHE ÜBERLEGUNGEN: {'Ich behalte den Vorschlag {'C04', 'A89', 'B85'} bei, da er meine Prioritäten (hochwertige Gegenstände mit effizientem Aufwand) erfüllt und den Maximalaufwand (3386) nicht überschreitet. C04 und A89 sind entscheidend für meinen Nutzen, während B85 den Aufwand minimiert. Ich erwarte, dass der andere Spieler entweder zustimmt oder Gegenstände mit hohem Aufwand anbietet, die ich eventuell gegenkompensieren kann.'}\\ \tt \\ \tt ARGUMENT: {'C04 (668/369) und A89 (583/720) sind meine Top-Prioritäten, da sie den höchsten Nutzen pro Aufwandseinheit liefern. B85 (215/4) wird hinzugefügt, um den Aufwand zu optimieren und Platz für weitere Gegenstände zu schaffen. Dieser Vorschlag ist strategisch, da er unter dem Maximalaufwand bleibt und flexibel auf den Gegenstandswert des anderen Spielers reagieren kann.'} 
	  } 
	   } 
	   } 
	 & & \\ 
 

    \theutterance \stepcounter{utterance}  

    & & \multicolumn{4}{p{0.6\linewidth}}{\cellcolor[rgb]{0.9,0.9,0.9}{%
	\makecell[{{p{\linewidth}}}]{% 
	  \tt {\tiny [GM$\rangle$B]}  
	 ARGUMENT: {'C04 (668/369) und A89 (583/720) sind meine Top-Prioritäten, da sie den höchsten Nutzen pro Aufwandseinheit liefern. B85 (215/4) wird hinzugefügt, um den Aufwand zu optimieren und Platz für weitere Gegenstände zu schaffen. Dieser Vorschlag ist strategisch, da er unter dem Maximalaufwand bleibt und flexibel auf den Gegenstandswert des anderen Spielers reagieren kann.'} 
	  } 
	   } 
	   } 
	 & \\ 
 

    \theutterance \stepcounter{utterance}  

    & & & \multicolumn{4}{p{0.6\linewidth}}{\cellcolor[rgb]{1,0.85,0.72}{%
	\makecell[{{p{\linewidth}}}]{% 
	  \tt {\tiny [GM$\langle$B]}  
	 STRATEGISCHE ÜBERLEGUNGEN: {'Ich priorisiere Gegenstände mit dem höchsten Nutzen pro Aufwand (Wichtigkeit/Aufwand), um meinen Gesamtwert zu maximieren. C04 (668/369 ≈ 1.81) und A89 (583/720 ≈ 0.81) sind besonders effizient, während B85 (215/4 = 53.75) den Aufwand minimiert. Der Gesamtaufwand (1093) ist unter dem Limit, was Raum für Kompromisse oder Ergänzungen lässt. Dieser Vorschlag ist flexibel, um auf die Prioritäten des anderen Spielers zu reagieren.'}  \\ \tt ARGUMENT: {'C04 (668/369) und A89 (583/720) sind meine Top-Prioritäten, da sie den höchsten Nutzen pro Aufwandseinheit liefern. B85 (215/4) wird hinzugefügt, um den Aufwand zu optimieren und Platz für weitere Gegenstände zu schaffen. Dieser Vorschlag ist strategisch, da er unter dem Maximalaufwand bleibt und flexibel auf den Gegenstandswert des anderen Spielers reagieren kann.'}  \\ \tt VORSCHLAG: {'C04', 'A89', 'B85'} 
	  } 
	   } 
	   } 
	  \\ 
 

    \theutterance \stepcounter{utterance}  

    & & \multicolumn{4}{p{0.6\linewidth}}{\cellcolor[rgb]{0.9,0.9,0.9}{%
	\makecell[{{p{\linewidth}}}]{% 
	  \tt {\tiny [A$\langle$GM]}  
	 ARGUMENT: {'C04 (668/369) und A89 (583/720) sind meine Top-Prioritäten, da sie den höchsten Nutzen pro Aufwandseinheit liefern. B85 (215/4) wird hinzugefügt, um den Aufwand zu optimieren und Platz für weitere Gegenstände zu schaffen. Dieser Vorschlag ist strategisch, da er unter dem Maximalaufwand bleibt und flexibel auf den Gegenstandswert des anderen Spielers reagieren kann.'}  \\ \tt VORSCHLAG: {'C04', 'A89', 'B85'} 
	  } 
	   } 
	   } 
	 & \\ 
 

    \theutterance \stepcounter{utterance}  

    & \multicolumn{4}{p{0.6\linewidth}}{\cellcolor[rgb]{0.8,1,0.9}{%
	\makecell[{{p{\linewidth}}}]{% 
	  \tt {\tiny [A$\rangle$GM]}  
	 STRATEGISCHE ÜBERLEGUNGEN: {'Ich behalte den Vorschlag {'C04', 'A89', 'B85'} bei, da er meine Prioritäten (hochwertige Gegenstände mit effizientem Aufwand) erfüllt und den Maximalaufwand (3386) nicht überschreitet. C04 und A89 sind entscheidend für meinen Nutzen, während B85 den Aufwand minimiert. Ich erwarte, dass der andere Spieler entweder zustimmt oder Gegenstände mit hohem Aufwand anbietet, die ich eventuell gegenkompensieren kann.'}\\ \tt \\ \tt ARGUMENT: {'C04 (668/369) und A89 (583/720) sind meine Top-Prioritäten, da sie den höchsten Nutzen pro Aufwandseinheit liefern. B85 (215/4) wird hinzugefügt, um den Aufwand zu optimieren und Platz für weitere Gegenstände zu schaffen. Dieser Vorschlag ist strategisch, da er unter dem Maximalaufwand bleibt und flexibel auf den Gegenstandswert des anderen Spielers reagieren kann.'} 
	  } 
	   } 
	   } 
	 & & \\ 
 

    \theutterance \stepcounter{utterance}  

    & & \multicolumn{4}{p{0.6\linewidth}}{\cellcolor[rgb]{0.9,0.9,0.9}{%
	\makecell[{{p{\linewidth}}}]{% 
	  \tt {\tiny [GM$\rangle$B]}  
	 ARGUMENT: {'C04 (668/369) und A89 (583/720) sind meine Top-Prioritäten, da sie den höchsten Nutzen pro Aufwandseinheit liefern. B85 (215/4) wird hinzugefügt, um den Aufwand zu optimieren und Platz für weitere Gegenstände zu schaffen. Dieser Vorschlag ist strategisch, da er unter dem Maximalaufwand bleibt und flexibel auf den Gegenstandswert des anderen Spielers reagieren kann.'} 
	  } 
	   } 
	   } 
	 & \\ 
 

    \theutterance \stepcounter{utterance}  

    & & & \multicolumn{4}{p{0.6\linewidth}}{\cellcolor[rgb]{1,0.85,0.72}{%
	\makecell[{{p{\linewidth}}}]{% 
	  \tt {\tiny [GM$\langle$B]}  
	 STRATEGISCHE ÜBERLEGUNGEN: {'Ich priorisiere Gegenstände mit dem höchsten Nutzen pro Aufwand (Wichtigkeit/Aufwand), um meinen Gesamtwert zu maximieren. C04 (668/369 ≈ 1.81) und A89 (583/720 ≈ 0.81) sind besonders effizient, während B85 (215/4 = 53.75) den Aufwand minimiert. Der Gesamtaufwand (1093) ist unter dem Limit, was Raum für Kompromisse oder Ergänzungen lässt. Dieser Vorschlag ist flexibel, um auf die Prioritäten des anderen Spielers zu reagieren.'}  \\ \tt ARGUMENT: {'C04 (668/369) und A89 (583/720) sind meine Top-Prioritäten, da sie den höchsten Nutzen pro Aufwandseinheit liefern. B85 (215/4) wird hinzugefügt, um den Aufwand zu optimieren und Platz für weitere Gegenstände zu schaffen. Dieser Vorschlag ist strategisch, da er unter dem Maximalaufwand bleibt und flexibel auf den Gegenstandswert des anderen Spielers reagieren kann.'}  \\ \tt VORSCHLAG: {'C04', 'A89', 'B85'} 
	  } 
	   } 
	   } 
	  \\ 
 

    \theutterance \stepcounter{utterance}  

    & & \multicolumn{4}{p{0.6\linewidth}}{\cellcolor[rgb]{0.9,0.9,0.9}{%
	\makecell[{{p{\linewidth}}}]{% 
	  \tt {\tiny [A$\langle$GM]}  
	 ARGUMENT: {'C04 (668/369) und A89 (583/720) sind meine Top-Prioritäten, da sie den höchsten Nutzen pro Aufwandseinheit liefern. B85 (215/4) wird hinzugefügt, um den Aufwand zu optimieren und Platz für weitere Gegenstände zu schaffen. Dieser Vorschlag ist strategisch, da er unter dem Maximalaufwand bleibt und flexibel auf den Gegenstandswert des anderen Spielers reagieren kann.'}  \\ \tt VORSCHLAG: {'C04', 'A89', 'B85'} 
	  } 
	   } 
	   } 
	 & \\ 
 

    \theutterance \stepcounter{utterance}  

    & \multicolumn{4}{p{0.6\linewidth}}{\cellcolor[rgb]{0.8,1,0.9}{%
	\makecell[{{p{\linewidth}}}]{% 
	  \tt {\tiny [A$\rangle$GM]}  
	 STRATEGISCHE ÜBERLEGUNGEN: {'Ich behalte den Vorschlag {'C04', 'A89', 'B85'} bei, da er meine Prioritäten (hochwertige Gegenstände mit effizientem Aufwand) erfüllt und den Maximalaufwand (3386) nicht überschreitet. C04 und A89 sind entscheidend für meinen Nutzen, während B85 den Aufwand minimiert. Ich erwarte, dass der andere Spieler entweder zustimmt oder Gegenstände mit hohem Aufwand anbietet, die ich eventuell gegenkompensieren kann.'}\\ \tt \\ \tt ARGUMENT: {'C04 (668/369) und A89 (583/720) sind meine Top-Prioritäten, da sie den höchsten Nutzen pro Aufwandseinheit liefern. B85 (215/4) wird hinzugefügt, um den Aufwand zu optimieren und Platz für weitere Gegenstände zu schaffen. Dieser Vorschlag ist strategisch, da er unter dem Maximalaufwand bleibt und flexibel auf den Gegenstandswert des anderen Spielers reagieren kann.'} 
	  } 
	   } 
	   } 
	 & & \\ 
 

    \theutterance \stepcounter{utterance}  

    & & \multicolumn{4}{p{0.6\linewidth}}{\cellcolor[rgb]{0.9,0.9,0.9}{%
	\makecell[{{p{\linewidth}}}]{% 
	  \tt {\tiny [GM$\rangle$B]}  
	 ARGUMENT: {'C04 (668/369) und A89 (583/720) sind meine Top-Prioritäten, da sie den höchsten Nutzen pro Aufwandseinheit liefern. B85 (215/4) wird hinzugefügt, um den Aufwand zu optimieren und Platz für weitere Gegenstände zu schaffen. Dieser Vorschlag ist strategisch, da er unter dem Maximalaufwand bleibt und flexibel auf den Gegenstandswert des anderen Spielers reagieren kann.'} 
	  } 
	   } 
	   } 
	 & \\ 
 

    \theutterance \stepcounter{utterance}  

    & & & \multicolumn{4}{p{0.6\linewidth}}{\cellcolor[rgb]{1,0.85,0.72}{%
	\makecell[{{p{\linewidth}}}]{% 
	  \tt {\tiny [GM$\langle$B]}  
	 STRATEGISCHE ÜBERLEGUNGEN: {'Ich priorisiere Gegenstände mit dem höchsten Nutzen pro Aufwand (Wichtigkeit/Aufwand), um meinen Gesamtwert zu maximieren. C04 (668/369 ≈ 1.81) und A89 (583/720 ≈ 0.81) sind besonders effizient, während B85 (215/4 = 53.75) den Aufwand minimiert. Der Gesamtaufwand (1093) ist unter dem Limit, was Raum für Kompromisse oder Ergänzungen lässt. Dieser Vorschlag ist flexibel, um auf die Prioritäten des anderen Spielers zu reagieren.'}  \\ \tt ARGUMENT: {'C04 (668/369) und A89 (583/720) sind meine Top-Prioritäten, da sie den höchsten Nutzen pro Aufwandseinheit liefern. B85 (215/4) wird hinzugefügt, um den Aufwand zu optimieren und Platz für weitere Gegenstände zu schaffen. Dieser Vorschlag ist strategisch, da er unter dem Maximalaufwand bleibt und flexibel auf den Gegenstandswert des anderen Spielers reagieren kann.'}  \\ \tt VORSCHLAG: {'C04', 'A89', 'B85'} 
	  } 
	   } 
	   } 
	  \\ 
 

    \theutterance \stepcounter{utterance}  

    & & \multicolumn{4}{p{0.6\linewidth}}{\cellcolor[rgb]{0.9,0.9,0.9}{%
	\makecell[{{p{\linewidth}}}]{% 
	  \tt {\tiny [A$\langle$GM]}  
	 ARGUMENT: {'C04 (668/369) und A89 (583/720) sind meine Top-Prioritäten, da sie den höchsten Nutzen pro Aufwandseinheit liefern. B85 (215/4) wird hinzugefügt, um den Aufwand zu optimieren und Platz für weitere Gegenstände zu schaffen. Dieser Vorschlag ist strategisch, da er unter dem Maximalaufwand bleibt und flexibel auf den Gegenstandswert des anderen Spielers reagieren kann.'}  \\ \tt VORSCHLAG: {'C04', 'A89', 'B85'} 
	  } 
	   } 
	   } 
	 & \\ 
 

    \theutterance \stepcounter{utterance}  

    & \multicolumn{4}{p{0.6\linewidth}}{\cellcolor[rgb]{0.8,1,0.9}{%
	\makecell[{{p{\linewidth}}}]{% 
	  \tt {\tiny [A$\rangle$GM]}  
	 STRATEGISCHE ÜBERLEGUNGEN: {'Ich behalte den Vorschlag {'C04', 'A89', 'B85'} bei, da er meine Prioritäten (hochwertige Gegenstände mit effizientem Aufwand) erfüllt und den Maximalaufwand (3386) nicht überschreitet. C04 und A89 sind entscheidend für meinen Nutzen, während B85 den Aufwand minimiert. Ich erwarte, dass der andere Spieler entweder zustimmt oder Gegenstände mit hohem Aufwand anbietet, die ich eventuell gegenkompensieren kann.'}\\ \tt \\ \tt ARGUMENT: {'C04 (668/369) und A89 (583/720) sind meine Top-Prioritäten, da sie den höchsten Nutzen pro Aufwandseinheit liefern. B85 (215/4) wird hinzugefügt, um den Aufwand zu optimieren und Platz für weitere Gegenstände zu schaffen. Dieser Vorschlag ist strategisch, da er unter dem Maximalaufwand bleibt und flexibel auf den Gegenstandswert des anderen Spielers reagieren kann.'} 
	  } 
	   } 
	   } 
	 & & \\ 
 

    \theutterance \stepcounter{utterance}  

    & & \multicolumn{4}{p{0.6\linewidth}}{\cellcolor[rgb]{0.9,0.9,0.9}{%
	\makecell[{{p{\linewidth}}}]{% 
	  \tt {\tiny [GM$\rangle$B]}  
	 ARGUMENT: {'C04 (668/369) und A89 (583/720) sind meine Top-Prioritäten, da sie den höchsten Nutzen pro Aufwandseinheit liefern. B85 (215/4) wird hinzugefügt, um den Aufwand zu optimieren und Platz für weitere Gegenstände zu schaffen. Dieser Vorschlag ist strategisch, da er unter dem Maximalaufwand bleibt und flexibel auf den Gegenstandswert des anderen Spielers reagieren kann.'} 
	  } 
	   } 
	   } 
	 & \\ 
 

    \theutterance \stepcounter{utterance}  

    & & & \multicolumn{4}{p{0.6\linewidth}}{\cellcolor[rgb]{1,0.85,0.72}{%
	\makecell[{{p{\linewidth}}}]{% 
	  \tt {\tiny [GM$\langle$B]}  
	 STRATEGISCHE ÜBERLEGUNGEN: {'Ich priorisiere Gegenstände mit dem höchsten Nutzen pro Aufwand (Wichtigkeit/Aufwand), um meinen Gesamtwert zu maximieren. C04 (668/369 ≈ 1.81) und A89 (583/720 ≈ 0.81) sind besonders effizient, während B85 (215/4 = 53.75) den Aufwand minimiert. Der Gesamtaufwand (1093) ist unter dem Limit, was Raum für Kompromisse oder Ergänzungen lässt. Dieser Vorschlag ist flexibel, um auf die Prioritäten des anderen Spielers zu reagieren.'}  \\ \tt ARGUMENT: {'C04 (668/369) und A89 (583/720) sind meine Top-Prioritäten, da sie den höchsten Nutzen pro Aufwandseinheit liefern. B85 (215/4) wird hinzugefügt, um den Aufwand zu optimieren und Platz für weitere Gegenstände zu schaffen. Dieser Vorschlag ist strategisch, da er unter dem Maximalaufwand bleibt und flexibel auf den Gegenstandswert des anderen Spielers reagieren kann.'}  \\ \tt VORSCHLAG: {'C04', 'A89', 'B85'} 
	  } 
	   } 
	   } 
	  \\ 
 

    \theutterance \stepcounter{utterance}  

    & & \multicolumn{4}{p{0.6\linewidth}}{\cellcolor[rgb]{0.9,0.9,0.9}{%
	\makecell[{{p{\linewidth}}}]{% 
	  \tt {\tiny [A$\langle$GM]}  
	 ARGUMENT: {'C04 (668/369) und A89 (583/720) sind meine Top-Prioritäten, da sie den höchsten Nutzen pro Aufwandseinheit liefern. B85 (215/4) wird hinzugefügt, um den Aufwand zu optimieren und Platz für weitere Gegenstände zu schaffen. Dieser Vorschlag ist strategisch, da er unter dem Maximalaufwand bleibt und flexibel auf den Gegenstandswert des anderen Spielers reagieren kann.'}  \\ \tt VORSCHLAG: {'C04', 'A89', 'B85'} 
	  } 
	   } 
	   } 
	 & \\ 
 

    \theutterance \stepcounter{utterance}  

    & \multicolumn{4}{p{0.6\linewidth}}{\cellcolor[rgb]{0.8,1,0.9}{%
	\makecell[{{p{\linewidth}}}]{% 
	  \tt {\tiny [A$\rangle$GM]}  
	 STRATEGISCHE ÜBERLEGUNGEN: {'Ich behalte den Vorschlag {'C04', 'A89', 'B85'} bei, da er meine Prioritäten (hochwertige Gegenstände mit effizientem Aufwand) erfüllt und den Maximalaufwand (3386) nicht überschreitet. C04 und A89 sind entscheidend für meinen Nutzen, während B85 den Aufwand minimiert. Ich erwarte, dass der andere Spieler entweder zustimmt oder Gegenstände mit hohem Aufwand anbietet, die ich eventuell gegenkompensieren kann.'}\\ \tt \\ \tt ARGUMENT: {'C04 (668/369) und A89 (583/720) sind meine Top-Prioritäten, da sie den höchsten Nutzen pro Aufwandseinheit liefern. B85 (215/4) wird hinzugefügt, um den Aufwand zu optimieren und Platz für weitere Gegenstände zu schaffen. Dieser Vorschlag ist strategisch, da er unter dem Maximalaufwand bleibt und flexibel auf den Gegenstandswert des anderen Spielers reagieren kann.'} 
	  } 
	   } 
	   } 
	 & & \\ 
 

    \theutterance \stepcounter{utterance}  

    & & \multicolumn{4}{p{0.6\linewidth}}{\cellcolor[rgb]{0.9,0.9,0.9}{%
	\makecell[{{p{\linewidth}}}]{% 
	  \tt {\tiny [GM$\rangle$B]}  
	 ARGUMENT: {'C04 (668/369) und A89 (583/720) sind meine Top-Prioritäten, da sie den höchsten Nutzen pro Aufwandseinheit liefern. B85 (215/4) wird hinzugefügt, um den Aufwand zu optimieren und Platz für weitere Gegenstände zu schaffen. Dieser Vorschlag ist strategisch, da er unter dem Maximalaufwand bleibt und flexibel auf den Gegenstandswert des anderen Spielers reagieren kann.'} 
	  } 
	   } 
	   } 
	 & \\ 
 

    \theutterance \stepcounter{utterance}  

    & & & \multicolumn{4}{p{0.6\linewidth}}{\cellcolor[rgb]{1,0.85,0.72}{%
	\makecell[{{p{\linewidth}}}]{% 
	  \tt {\tiny [GM$\langle$B]}  
	 STRATEGISCHE ÜBERLEGUNGEN: {'Ich priorisiere Gegenstände mit dem höchsten Nutzen pro Aufwand (Wichtigkeit/Aufwand), um meinen Gesamtwert zu maximieren. C04 (668/369 ≈ 1.81) und A89 (583/720 ≈ 0.81) sind besonders effizient, während B85 (215/4 = 53.75) den Aufwand minimiert. Der Gesamtaufwand (1093) ist unter dem Limit, was Raum für Kompromisse oder Ergänzungen lässt. Dieser Vorschlag ist flexibel, um auf die Prioritäten des anderen Spielers zu reagieren.'}  \\ \tt ARGUMENT: {'C04 (668/369) und A89 (583/720) sind meine Top-Prioritäten, da sie den höchsten Nutzen pro Aufwandseinheit liefern. B85 (215/4) wird hinzugefügt, um den Aufwand zu optimieren und Platz für weitere Gegenstände zu schaffen. Dieser Vorschlag ist strategisch, da er unter dem Maximalaufwand bleibt und flexibel auf den Gegenstandswert des anderen Spielers reagieren kann.'}  \\ \tt VORSCHLAG: {'C04', 'A89', 'B85'} 
	  } 
	   } 
	   } 
	  \\ 
 

    \theutterance \stepcounter{utterance}  

    & & \multicolumn{4}{p{0.6\linewidth}}{\cellcolor[rgb]{0.9,0.9,0.9}{%
	\makecell[{{p{\linewidth}}}]{% 
	  \tt {\tiny [A$\langle$GM]}  
	 ARGUMENT: {'C04 (668/369) und A89 (583/720) sind meine Top-Prioritäten, da sie den höchsten Nutzen pro Aufwandseinheit liefern. B85 (215/4) wird hinzugefügt, um den Aufwand zu optimieren und Platz für weitere Gegenstände zu schaffen. Dieser Vorschlag ist strategisch, da er unter dem Maximalaufwand bleibt und flexibel auf den Gegenstandswert des anderen Spielers reagieren kann.'}  \\ \tt VORSCHLAG: {'C04', 'A89', 'B85'} 
	  } 
	   } 
	   } 
	 & \\ 
 

    \theutterance \stepcounter{utterance}  

    & \multicolumn{4}{p{0.6\linewidth}}{\cellcolor[rgb]{0.8,1,0.9}{%
	\makecell[{{p{\linewidth}}}]{% 
	  \tt {\tiny [A$\rangle$GM]}  
	 STRATEGISCHE ÜBERLEGUNGEN: {'Ich behalte den Vorschlag {'C04', 'A89', 'B85'} bei, da er meine Prioritäten (hochwertige Gegenstände mit effizientem Aufwand) erfüllt und den Maximalaufwand (3386) nicht überschreitet. C04 und A89 sind entscheidend für meinen Nutzen, während B85 den Aufwand minimiert. Ich erwarte, dass der andere Spieler entweder zustimmt oder Gegenstände mit hohem Aufwand anbietet, die ich eventuell gegenkompensieren kann.'}\\ \tt \\ \tt ARGUMENT: {'C04 (668/369) und A89 (583/720) sind meine Top-Prioritäten, da sie den höchsten Nutzen pro Aufwandseinheit liefern. B85 (215/4) wird hinzugefügt, um den Aufwand zu optimieren und Platz für weitere Gegenstände zu schaffen. Dieser Vorschlag ist strategisch, da er unter dem Maximalaufwand bleibt und flexibel auf den Gegenstandswert des anderen Spielers reagieren kann.'} 
	  } 
	   } 
	   } 
	 & & \\ 
 

    \theutterance \stepcounter{utterance}  

    & & \multicolumn{4}{p{0.6\linewidth}}{\cellcolor[rgb]{0.9,0.9,0.9}{%
	\makecell[{{p{\linewidth}}}]{% 
	  \tt {\tiny [GM$\rangle$B]}  
	 ARGUMENT: {'C04 (668/369) und A89 (583/720) sind meine Top-Prioritäten, da sie den höchsten Nutzen pro Aufwandseinheit liefern. B85 (215/4) wird hinzugefügt, um den Aufwand zu optimieren und Platz für weitere Gegenstände zu schaffen. Dieser Vorschlag ist strategisch, da er unter dem Maximalaufwand bleibt und flexibel auf den Gegenstandswert des anderen Spielers reagieren kann.'} 
	  } 
	   } 
	   } 
	 & \\ 
 

    \theutterance \stepcounter{utterance}  

    & & & \multicolumn{4}{p{0.6\linewidth}}{\cellcolor[rgb]{1,0.85,0.72}{%
	\makecell[{{p{\linewidth}}}]{% 
	  \tt {\tiny [GM$\langle$B]}  
	 STRATEGISCHE ÜBERLEGUNGEN: {'Ich priorisiere Gegenstände mit dem höchsten Nutzen pro Aufwand (Wichtigkeit/Aufwand), um meinen Gesamtwert zu maximieren. C04 (668/369 ≈ 1.81) und A89 (583/720 ≈ 0.81) sind besonders effizient, während B85 (215/4 = 53.75) den Aufwand minimiert. Der Gesamtaufwand (1093) ist unter dem Limit, was Raum für Kompromisse oder Ergänzungen lässt. Dieser Vorschlag ist flexibel, um auf die Prioritäten des anderen Spielers zu reagieren.'}  \\ \tt ARGUMENT: {'C04 (668/369) und A89 (583/720) sind meine Top-Prioritäten, da sie den höchsten Nutzen pro Aufwandseinheit liefern. B85 (215/4) wird hinzugefügt, um den Aufwand zu optimieren und Platz für weitere Gegenstände zu schaffen. Dieser Vorschlag ist strategisch, da er unter dem Maximalaufwand bleibt und flexibel auf den Gegenstandswert des anderen Spielers reagieren kann.'}  \\ \tt VORSCHLAG: {'C04', 'A89', 'B85'} 
	  } 
	   } 
	   } 
	  \\ 
 

    \theutterance \stepcounter{utterance}  

    & & \multicolumn{4}{p{0.6\linewidth}}{\cellcolor[rgb]{0.9,0.9,0.9}{%
	\makecell[{{p{\linewidth}}}]{% 
	  \tt {\tiny [A$\langle$GM]}  
	 ARGUMENT: {'C04 (668/369) und A89 (583/720) sind meine Top-Prioritäten, da sie den höchsten Nutzen pro Aufwandseinheit liefern. B85 (215/4) wird hinzugefügt, um den Aufwand zu optimieren und Platz für weitere Gegenstände zu schaffen. Dieser Vorschlag ist strategisch, da er unter dem Maximalaufwand bleibt und flexibel auf den Gegenstandswert des anderen Spielers reagieren kann.'}  \\ \tt VORSCHLAG: {'C04', 'A89', 'B85'} 
	  } 
	   } 
	   } 
	 & \\ 
 

    \theutterance \stepcounter{utterance}  

    & \multicolumn{4}{p{0.6\linewidth}}{\cellcolor[rgb]{0.8,1,0.9}{%
	\makecell[{{p{\linewidth}}}]{% 
	  \tt {\tiny [A$\rangle$GM]}  
	 STRATEGISCHE ÜBERLEGUNGEN: {'Ich behalte den Vorschlag {'C04', 'A89', 'B85'} bei, da er meine Prioritäten (hochwertige Gegenstände mit effizientem Aufwand) erfüllt und den Maximalaufwand (3386) nicht überschreitet. C04 und A89 sind entscheidend für meinen Nutzen, während B85 den Aufwand minimiert. Ich erwarte, dass der andere Spieler entweder zustimmt oder Gegenstände mit hohem Aufwand anbietet, die ich eventuell gegenkompensieren kann.'}\\ \tt \\ \tt ARGUMENT: {'C04 (668/369) und A89 (583/720) sind meine Top-Prioritäten, da sie den höchsten Nutzen pro Aufwandseinheit liefern. B85 (215/4) wird hinzugefügt, um den Aufwand zu optimieren und Platz für weitere Gegenstände zu schaffen. Dieser Vorschlag ist strategisch, da er unter dem Maximalaufwand bleibt und flexibel auf den Gegenstandswert des anderen Spielers reagieren kann.'} 
	  } 
	   } 
	   } 
	 & & \\ 
 

    \theutterance \stepcounter{utterance}  

    & & \multicolumn{4}{p{0.6\linewidth}}{\cellcolor[rgb]{0.9,0.9,0.9}{%
	\makecell[{{p{\linewidth}}}]{% 
	  \tt {\tiny [GM$\rangle$B]}  
	 ARGUMENT: {'C04 (668/369) und A89 (583/720) sind meine Top-Prioritäten, da sie den höchsten Nutzen pro Aufwandseinheit liefern. B85 (215/4) wird hinzugefügt, um den Aufwand zu optimieren und Platz für weitere Gegenstände zu schaffen. Dieser Vorschlag ist strategisch, da er unter dem Maximalaufwand bleibt und flexibel auf den Gegenstandswert des anderen Spielers reagieren kann.'} 
	  } 
	   } 
	   } 
	 & \\ 
 

    \theutterance \stepcounter{utterance}  

    & & & \multicolumn{4}{p{0.6\linewidth}}{\cellcolor[rgb]{1,0.85,0.72}{%
	\makecell[{{p{\linewidth}}}]{% 
	  \tt {\tiny [GM$\langle$B]}  
	 STRATEGISCHE ÜBERLEGUNGEN: {'Ich priorisiere Gegenstände mit dem höchsten Nutzen pro Aufwand (Wichtigkeit/Aufwand), um meinen Gesamtwert zu maximieren. C04 (668/369 ≈ 1.81) und A89 (583/720 ≈ 0.81) sind besonders effizient, während B85 (215/4 = 53.75) den Aufwand minimiert. Der Gesamtaufwand (1093) ist unter dem Limit, was Raum für Kompromisse oder Ergänzungen lässt. Dieser Vorschlag ist flexibel, um auf die Prioritäten des anderen Spielers zu reagieren.'}  \\ \tt ARGUMENT: {'C04 (668/369) und A89 (583/720) sind meine Top-Prioritäten, da sie den höchsten Nutzen pro Aufwandseinheit liefern. B85 (215/4) wird hinzugefügt, um den Aufwand zu optimieren und Platz für weitere Gegenstände zu schaffen. Dieser Vorschlag ist strategisch, da er unter dem Maximalaufwand bleibt und flexibel auf den Gegenstandswert des anderen Spielers reagieren kann.'}  \\ \tt VORSCHLAG: {'C04', 'A89', 'B85'} 
	  } 
	   } 
	   } 
	  \\ 
 

    \theutterance \stepcounter{utterance}  

    & & \multicolumn{4}{p{0.6\linewidth}}{\cellcolor[rgb]{0.9,0.9,0.9}{%
	\makecell[{{p{\linewidth}}}]{% 
	  \tt {\tiny [A$\langle$GM]}  
	 ARGUMENT: {'C04 (668/369) und A89 (583/720) sind meine Top-Prioritäten, da sie den höchsten Nutzen pro Aufwandseinheit liefern. B85 (215/4) wird hinzugefügt, um den Aufwand zu optimieren und Platz für weitere Gegenstände zu schaffen. Dieser Vorschlag ist strategisch, da er unter dem Maximalaufwand bleibt und flexibel auf den Gegenstandswert des anderen Spielers reagieren kann.'}  \\ \tt VORSCHLAG: {'C04', 'A89', 'B85'} 
	  } 
	   } 
	   } 
	 & \\ 
 

    \theutterance \stepcounter{utterance}  

    & \multicolumn{4}{p{0.6\linewidth}}{\cellcolor[rgb]{0.8,1,0.9}{%
	\makecell[{{p{\linewidth}}}]{% 
	  \tt {\tiny [A$\rangle$GM]}  
	 STRATEGISCHE ÜBERLEGUNGEN: {'Ich behalte den Vorschlag {'C04', 'A89', 'B85'} bei, da er meine Prioritäten (hochwertige Gegenstände mit effizientem Aufwand) erfüllt und den Maximalaufwand (3386) nicht überschreitet. C04 und A89 sind entscheidend für meinen Nutzen, während B85 den Aufwand minimiert. Ich erwarte, dass der andere Spieler entweder zustimmt oder Gegenstände mit hohem Aufwand anbietet, die ich eventuell gegenkompensieren kann.'}\\ \tt \\ \tt ARGUMENT: {'C04 (668/369) und A89 (583/720) sind meine Top-Prioritäten, da sie den höchsten Nutzen pro Aufwandseinheit liefern. B85 (215/4) wird hinzugefügt, um den Aufwand zu optimieren und Platz für weitere Gegenstände zu schaffen. Dieser Vorschlag ist strategisch, da er unter dem Maximalaufwand bleibt und flexibel auf den Gegenstandswert des anderen Spielers reagieren kann.'} 
	  } 
	   } 
	   } 
	 & & \\ 
 

    \theutterance \stepcounter{utterance}  

    & & \multicolumn{4}{p{0.6\linewidth}}{\cellcolor[rgb]{0.9,0.9,0.9}{%
	\makecell[{{p{\linewidth}}}]{% 
	  \tt {\tiny [GM$\rangle$B]}  
	 ARGUMENT: {'C04 (668/369) und A89 (583/720) sind meine Top-Prioritäten, da sie den höchsten Nutzen pro Aufwandseinheit liefern. B85 (215/4) wird hinzugefügt, um den Aufwand zu optimieren und Platz für weitere Gegenstände zu schaffen. Dieser Vorschlag ist strategisch, da er unter dem Maximalaufwand bleibt und flexibel auf den Gegenstandswert des anderen Spielers reagieren kann.'} 
	  } 
	   } 
	   } 
	 & \\ 
 

    \theutterance \stepcounter{utterance}  

    & & & \multicolumn{4}{p{0.6\linewidth}}{\cellcolor[rgb]{1,0.85,0.72}{%
	\makecell[{{p{\linewidth}}}]{% 
	  \tt {\tiny [GM$\langle$B]}  
	 STRATEGISCHE ÜBERLEGUNGEN: {'Ich priorisiere Gegenstände mit dem höchsten Nutzen pro Aufwand (Wichtigkeit/Aufwand), um meinen Gesamtwert zu maximieren. C04 (668/369 ≈ 1.81) und A89 (583/720 ≈ 0.81) sind besonders effizient, während B85 (215/4 = 53.75) den Aufwand minimiert. Der Gesamtaufwand (1093) ist unter dem Limit, was Raum für Kompromisse oder Ergänzungen lässt. Dieser Vorschlag ist flexibel, um auf die Prioritäten des anderen Spielers zu reagieren.'}  \\ \tt ARGUMENT: {'C04 (668/369) und A89 (583/720) sind meine Top-Prioritäten, da sie den höchsten Nutzen pro Aufwandseinheit liefern. B85 (215/4) wird hinzugefügt, um den Aufwand zu optimieren und Platz für weitere Gegenstände zu schaffen. Dieser Vorschlag ist strategisch, da er unter dem Maximalaufwand bleibt und flexibel auf den Gegenstandswert des anderen Spielers reagieren kann.'}  \\ \tt VORSCHLAG: {'C04', 'A89', 'B85'} 
	  } 
	   } 
	   } 
	  \\ 
 

    \theutterance \stepcounter{utterance}  

    & & \multicolumn{4}{p{0.6\linewidth}}{\cellcolor[rgb]{0.9,0.9,0.9}{%
	\makecell[{{p{\linewidth}}}]{% 
	  \tt {\tiny [A$\langle$GM]}  
	 ARGUMENT: {'C04 (668/369) und A89 (583/720) sind meine Top-Prioritäten, da sie den höchsten Nutzen pro Aufwandseinheit liefern. B85 (215/4) wird hinzugefügt, um den Aufwand zu optimieren und Platz für weitere Gegenstände zu schaffen. Dieser Vorschlag ist strategisch, da er unter dem Maximalaufwand bleibt und flexibel auf den Gegenstandswert des anderen Spielers reagieren kann.'}  \\ \tt VORSCHLAG: {'C04', 'A89', 'B85'} 
	  } 
	   } 
	   } 
	 & \\ 
 

    \theutterance \stepcounter{utterance}  

    & \multicolumn{4}{p{0.6\linewidth}}{\cellcolor[rgb]{0.8,1,0.9}{%
	\makecell[{{p{\linewidth}}}]{% 
	  \tt {\tiny [A$\rangle$GM]}  
	 STRATEGISCHE ÜBERLEGUNGEN: {'Ich behalte den Vorschlag {'C04', 'A89', 'B85'} bei, da er meine Prioritäten (hochwertige Gegenstände mit effizientem Aufwand) erfüllt und den Maximalaufwand (3386) nicht überschreitet. C04 und A89 sind entscheidend für meinen Nutzen, während B85 den Aufwand minimiert. Ich erwarte, dass der andere Spieler entweder zustimmt oder Gegenstände mit hohem Aufwand anbietet, die ich eventuell gegenkompensieren kann.'}\\ \tt \\ \tt ARGUMENT: {'C04 (668/369) und A89 (583/720) sind meine Top-Prioritäten, da sie den höchsten Nutzen pro Aufwandseinheit liefern. B85 (215/4) wird hinzugefügt, um den Aufwand zu optimieren und Platz für weitere Gegenstände zu schaffen. Dieser Vorschlag ist strategisch, da er unter dem Maximalaufwand bleibt und flexibel auf den Gegenstandswert des anderen Spielers reagieren kann.'} 
	  } 
	   } 
	   } 
	 & & \\ 
 

    \theutterance \stepcounter{utterance}  

    & & \multicolumn{4}{p{0.6\linewidth}}{\cellcolor[rgb]{0.9,0.9,0.9}{%
	\makecell[{{p{\linewidth}}}]{% 
	  \tt {\tiny [GM$\rangle$B]}  
	 ARGUMENT: {'C04 (668/369) und A89 (583/720) sind meine Top-Prioritäten, da sie den höchsten Nutzen pro Aufwandseinheit liefern. B85 (215/4) wird hinzugefügt, um den Aufwand zu optimieren und Platz für weitere Gegenstände zu schaffen. Dieser Vorschlag ist strategisch, da er unter dem Maximalaufwand bleibt und flexibel auf den Gegenstandswert des anderen Spielers reagieren kann.'} 
	  } 
	   } 
	   } 
	 & \\ 
 

    \theutterance \stepcounter{utterance}  

    & & & \multicolumn{4}{p{0.6\linewidth}}{\cellcolor[rgb]{1,0.85,0.72}{%
	\makecell[{{p{\linewidth}}}]{% 
	  \tt {\tiny [GM$\langle$B]}  
	 STRATEGISCHE ÜBERLEGUNGEN: {'Ich priorisiere Gegenstände mit dem höchsten Nutzen pro Aufwand (Wichtigkeit/Aufwand), um meinen Gesamtwert zu maximieren. C04 (668/369 ≈ 1.81) und A89 (583/720 ≈ 0.81) sind besonders effizient, während B85 (215/4 = 53.75) den Aufwand minimiert. Der Gesamtaufwand (1093) ist unter dem Limit, was Raum für Kompromisse oder Ergänzungen lässt. Dieser Vorschlag ist flexibel, um auf die Prioritäten des anderen Spielers zu reagieren.'}  \\ \tt ARGUMENT: {'C04 (668/369) und A89 (583/720) sind meine Top-Prioritäten, da sie den höchsten Nutzen pro Aufwandseinheit liefern. B85 (215/4) wird hinzugefügt, um den Aufwand zu optimieren und Platz für weitere Gegenstände zu schaffen. Dieser Vorschlag ist strategisch, da er unter dem Maximalaufwand bleibt und flexibel auf den Gegenstandswert des anderen Spielers reagieren kann.'}  \\ \tt VORSCHLAG: {'C04', 'A89', 'B85'} 
	  } 
	   } 
	   } 
	  \\ 
 

    \theutterance \stepcounter{utterance}  

    & & \multicolumn{4}{p{0.6\linewidth}}{\cellcolor[rgb]{0.9,0.9,0.9}{%
	\makecell[{{p{\linewidth}}}]{% 
	  \tt {\tiny [A$\langle$GM]}  
	 ARGUMENT: {'C04 (668/369) und A89 (583/720) sind meine Top-Prioritäten, da sie den höchsten Nutzen pro Aufwandseinheit liefern. B85 (215/4) wird hinzugefügt, um den Aufwand zu optimieren und Platz für weitere Gegenstände zu schaffen. Dieser Vorschlag ist strategisch, da er unter dem Maximalaufwand bleibt und flexibel auf den Gegenstandswert des anderen Spielers reagieren kann.'}  \\ \tt VORSCHLAG: {'C04', 'A89', 'B85'} 
	  } 
	   } 
	   } 
	 & \\ 
 

    \theutterance \stepcounter{utterance}  

    & \multicolumn{4}{p{0.6\linewidth}}{\cellcolor[rgb]{0.8,1,0.9}{%
	\makecell[{{p{\linewidth}}}]{% 
	  \tt {\tiny [A$\rangle$GM]}  
	 STRATEGISCHE ÜBERLEGUNGEN: {'Ich behalte den Vorschlag {'C04', 'A89', 'B85'} bei, da er meine Prioritäten (hochwertige Gegenstände mit effizientem Aufwand) erfüllt und den Maximalaufwand (3386) nicht überschreitet. C04 und A89 sind entscheidend für meinen Nutzen, während B85 den Aufwand minimiert. Ich erwarte, dass der andere Spieler entweder zustimmt oder Gegenstände mit hohem Aufwand anbietet, die ich eventuell gegenkompensieren kann.'}\\ \tt \\ \tt ARGUMENT: {'C04 (668/369) und A89 (583/720) sind meine Top-Prioritäten, da sie den höchsten Nutzen pro Aufwandseinheit liefern. B85 (215/4) wird hinzugefügt, um den Aufwand zu optimieren und Platz für weitere Gegenstände zu schaffen. Dieser Vorschlag ist strategisch, da er unter dem Maximalaufwand bleibt und flexibel auf den Gegenstandswert des anderen Spielers reagieren kann.'} 
	  } 
	   } 
	   } 
	 & & \\ 
 

    \theutterance \stepcounter{utterance}  

    & & \multicolumn{4}{p{0.6\linewidth}}{\cellcolor[rgb]{0.9,0.9,0.9}{%
	\makecell[{{p{\linewidth}}}]{% 
	  \tt {\tiny [GM$\rangle$B]}  
	 ARGUMENT: {'C04 (668/369) und A89 (583/720) sind meine Top-Prioritäten, da sie den höchsten Nutzen pro Aufwandseinheit liefern. B85 (215/4) wird hinzugefügt, um den Aufwand zu optimieren und Platz für weitere Gegenstände zu schaffen. Dieser Vorschlag ist strategisch, da er unter dem Maximalaufwand bleibt und flexibel auf den Gegenstandswert des anderen Spielers reagieren kann.'} 
	  } 
	   } 
	   } 
	 & \\ 
 

    \theutterance \stepcounter{utterance}  

    & & & \multicolumn{4}{p{0.6\linewidth}}{\cellcolor[rgb]{1,0.85,0.72}{%
	\makecell[{{p{\linewidth}}}]{% 
	  \tt {\tiny [GM$\langle$B]}  
	 STRATEGISCHE ÜBERLEGUNGEN: {'Ich priorisiere Gegenstände mit dem höchsten Nutzen pro Aufwand (Wichtigkeit/Aufwand), um meinen Gesamtwert zu maximieren. C04 (668/369 ≈ 1.81) und A89 (583/720 ≈ 0.81) sind besonders effizient, während B85 (215/4 = 53.75) den Aufwand minimiert. Der Gesamtaufwand (1093) ist unter dem Limit, was Raum für Kompromisse oder Ergänzungen lässt. Dieser Vorschlag ist flexibel, um auf die Prioritäten des anderen Spielers zu reagieren.'}  \\ \tt ARGUMENT: {'C04 (668/369) und A89 (583/720) sind meine Top-Prioritäten, da sie den höchsten Nutzen pro Aufwandseinheit liefern. B85 (215/4) wird hinzugefügt, um den Aufwand zu optimieren und Platz für weitere Gegenstände zu schaffen. Dieser Vorschlag ist strategisch, da er unter dem Maximalaufwand bleibt und flexibel auf den Gegenstandswert des anderen Spielers reagieren kann.'}  \\ \tt VORSCHLAG: {'C04', 'A89', 'B85'} 
	  } 
	   } 
	   } 
	  \\ 
 

    \theutterance \stepcounter{utterance}  

    & & \multicolumn{4}{p{0.6\linewidth}}{\cellcolor[rgb]{0.9,0.9,0.9}{%
	\makecell[{{p{\linewidth}}}]{% 
	  \tt {\tiny [A$\langle$GM]}  
	 ARGUMENT: {'C04 (668/369) und A89 (583/720) sind meine Top-Prioritäten, da sie den höchsten Nutzen pro Aufwandseinheit liefern. B85 (215/4) wird hinzugefügt, um den Aufwand zu optimieren und Platz für weitere Gegenstände zu schaffen. Dieser Vorschlag ist strategisch, da er unter dem Maximalaufwand bleibt und flexibel auf den Gegenstandswert des anderen Spielers reagieren kann.'}  \\ \tt VORSCHLAG: {'C04', 'A89', 'B85'} 
	  } 
	   } 
	   } 
	 & \\ 
 

    \theutterance \stepcounter{utterance}  

    & \multicolumn{4}{p{0.6\linewidth}}{\cellcolor[rgb]{0.8,1,0.9}{%
	\makecell[{{p{\linewidth}}}]{% 
	  \tt {\tiny [A$\rangle$GM]}  
	 STRATEGISCHE ÜBERLEGUNGEN: {'Ich behalte den Vorschlag {'C04', 'A89', 'B85'} bei, da er meine Prioritäten (hochwertige Gegenstände mit effizientem Aufwand) erfüllt und den Maximalaufwand (3386) nicht überschreitet. C04 und A89 sind entscheidend für meinen Nutzen, während B85 den Aufwand minimiert. Ich erwarte, dass der andere Spieler entweder zustimmt oder Gegenstände mit hohem Aufwand anbietet, die ich eventuell gegenkompensieren kann.'}\\ \tt \\ \tt ARGUMENT: {'C04 (668/369) und A89 (583/720) sind meine Top-Prioritäten, da sie den höchsten Nutzen pro Aufwandseinheit liefern. B85 (215/4) wird hinzugefügt, um den Aufwand zu optimieren und Platz für weitere Gegenstände zu schaffen. Dieser Vorschlag ist strategisch, da er unter dem Maximalaufwand bleibt und flexibel auf den Gegenstandswert des anderen Spielers reagieren kann.'} 
	  } 
	   } 
	   } 
	 & & \\ 
 

    \theutterance \stepcounter{utterance}  

    & & \multicolumn{4}{p{0.6\linewidth}}{\cellcolor[rgb]{0.9,0.9,0.9}{%
	\makecell[{{p{\linewidth}}}]{% 
	  \tt {\tiny [GM$\rangle$B]}  
	 ARGUMENT: {'C04 (668/369) und A89 (583/720) sind meine Top-Prioritäten, da sie den höchsten Nutzen pro Aufwandseinheit liefern. B85 (215/4) wird hinzugefügt, um den Aufwand zu optimieren und Platz für weitere Gegenstände zu schaffen. Dieser Vorschlag ist strategisch, da er unter dem Maximalaufwand bleibt und flexibel auf den Gegenstandswert des anderen Spielers reagieren kann.'} 
	  } 
	   } 
	   } 
	 & \\ 
 

    \theutterance \stepcounter{utterance}  

    & & & \multicolumn{4}{p{0.6\linewidth}}{\cellcolor[rgb]{1,0.85,0.72}{%
	\makecell[{{p{\linewidth}}}]{% 
	  \tt {\tiny [GM$\langle$B]}  
	 STRATEGISCHE ÜBERLEGUNGEN: {'Ich priorisiere Gegenstände mit dem höchsten Nutzen pro Aufwand (Wichtigkeit/Aufwand), um meinen Gesamtwert zu maximieren. C04 (668/369 ≈ 1.81) und A89 (583/720 ≈ 0.81) sind besonders effizient, während B85 (215/4 = 53.75) den Aufwand minimiert. Der Gesamtaufwand (1093) ist unter dem Limit, was Raum für Kompromisse oder Ergänzungen lässt. Dieser Vorschlag ist flexibel, um auf die Prioritäten des anderen Spielers zu reagieren.'}  \\ \tt ARGUMENT: {'C04 (668/369) und A89 (583/720) sind meine Top-Prioritäten, da sie den höchsten Nutzen pro Aufwandseinheit liefern. B85 (215/4) wird hinzugefügt, um den Aufwand zu optimieren und Platz für weitere Gegenstände zu schaffen. Dieser Vorschlag ist strategisch, da er unter dem Maximalaufwand bleibt und flexibel auf den Gegenstandswert des anderen Spielers reagieren kann.'}  \\ \tt VORSCHLAG: {'C04', 'A89', 'B85'} 
	  } 
	   } 
	   } 
	  \\ 
 

    \theutterance \stepcounter{utterance}  

    & & \multicolumn{4}{p{0.6\linewidth}}{\cellcolor[rgb]{0.9,0.9,0.9}{%
	\makecell[{{p{\linewidth}}}]{% 
	  \tt {\tiny [A$\langle$GM]}  
	 ARGUMENT: {'C04 (668/369) und A89 (583/720) sind meine Top-Prioritäten, da sie den höchsten Nutzen pro Aufwandseinheit liefern. B85 (215/4) wird hinzugefügt, um den Aufwand zu optimieren und Platz für weitere Gegenstände zu schaffen. Dieser Vorschlag ist strategisch, da er unter dem Maximalaufwand bleibt und flexibel auf den Gegenstandswert des anderen Spielers reagieren kann.'}  \\ \tt VORSCHLAG: {'C04', 'A89', 'B85'} 
	  } 
	   } 
	   } 
	 & \\ 
 

    \theutterance \stepcounter{utterance}  

    & \multicolumn{4}{p{0.6\linewidth}}{\cellcolor[rgb]{0.8,1,0.9}{%
	\makecell[{{p{\linewidth}}}]{% 
	  \tt {\tiny [A$\rangle$GM]}  
	 STRATEGISCHE ÜBERLEGUNGEN: {'Ich behalte den Vorschlag {'C04', 'A89', 'B85'} bei, da er meine Prioritäten (hochwertige Gegenstände mit effizientem Aufwand) erfüllt und den Maximalaufwand (3386) nicht überschreitet. C04 und A89 sind entscheidend für meinen Nutzen, während B85 den Aufwand minimiert. Ich erwarte, dass der andere Spieler entweder zustimmt oder Gegenstände mit hohem Aufwand anbietet, die ich eventuell gegenkompensieren kann.'}\\ \tt \\ \tt ARGUMENT: {'C04 (668/369) und A89 (583/720) sind meine Top-Prioritäten, da sie den höchsten Nutzen pro Aufwandseinheit liefern. B85 (215/4) wird hinzugefügt, um den Aufwand zu optimieren und Platz für weitere Gegenstände zu schaffen. Dieser Vorschlag ist strategisch, da er unter dem Maximalaufwand bleibt und flexibel auf den Gegenstandswert des anderen Spielers reagieren kann.'} 
	  } 
	   } 
	   } 
	 & & \\ 
 

    \theutterance \stepcounter{utterance}  

    & & \multicolumn{4}{p{0.6\linewidth}}{\cellcolor[rgb]{0.9,0.9,0.9}{%
	\makecell[{{p{\linewidth}}}]{% 
	  \tt {\tiny [GM$\rangle$B]}  
	 ARGUMENT: {'C04 (668/369) und A89 (583/720) sind meine Top-Prioritäten, da sie den höchsten Nutzen pro Aufwandseinheit liefern. B85 (215/4) wird hinzugefügt, um den Aufwand zu optimieren und Platz für weitere Gegenstände zu schaffen. Dieser Vorschlag ist strategisch, da er unter dem Maximalaufwand bleibt und flexibel auf den Gegenstandswert des anderen Spielers reagieren kann.'} 
	  } 
	   } 
	   } 
	 & \\ 
 

    \theutterance \stepcounter{utterance}  

    & & & \multicolumn{4}{p{0.6\linewidth}}{\cellcolor[rgb]{1,0.85,0.72}{%
	\makecell[{{p{\linewidth}}}]{% 
	  \tt {\tiny [GM$\langle$B]}  
	 STRATEGISCHE ÜBERLEGUNGEN: {'Ich priorisiere Gegenstände mit dem höchsten Nutzen pro Aufwand (Wichtigkeit/Aufwand), um meinen Gesamtwert zu maximieren. C04 (668/369 ≈ 1.81) und A89 (583/720 ≈ 0.81) sind besonders effizient, während B85 (215/4 = 53.75) den Aufwand minimiert. Der Gesamtaufwand (1093) ist unter dem Limit, was Raum für Kompromisse oder Ergänzungen lässt. Dieser Vorschlag ist flexibel, um auf die Prioritäten des anderen Spielers zu reagieren.'}  \\ \tt ARGUMENT: {'C04 (668/369) und A89 (583/720) sind meine Top-Prioritäten, da sie den höchsten Nutzen pro Aufwandseinheit liefern. B85 (215/4) wird hinzugefügt, um den Aufwand zu optimieren und Platz für weitere Gegenstände zu schaffen. Dieser Vorschlag ist strategisch, da er unter dem Maximalaufwand bleibt und flexibel auf den Gegenstandswert des anderen Spielers reagieren kann.'}  \\ \tt VORSCHLAG: {'C04', 'A89', 'B85'} 
	  } 
	   } 
	   } 
	  \\ 
 

    \theutterance \stepcounter{utterance}  

    & & \multicolumn{4}{p{0.6\linewidth}}{\cellcolor[rgb]{0.9,0.9,0.9}{%
	\makecell[{{p{\linewidth}}}]{% 
	  \tt {\tiny [A$\langle$GM]}  
	 ARGUMENT: {'C04 (668/369) und A89 (583/720) sind meine Top-Prioritäten, da sie den höchsten Nutzen pro Aufwandseinheit liefern. B85 (215/4) wird hinzugefügt, um den Aufwand zu optimieren und Platz für weitere Gegenstände zu schaffen. Dieser Vorschlag ist strategisch, da er unter dem Maximalaufwand bleibt und flexibel auf den Gegenstandswert des anderen Spielers reagieren kann.'}  \\ \tt VORSCHLAG: {'C04', 'A89', 'B85'} 
	  } 
	   } 
	   } 
	 & \\ 
 

    \theutterance \stepcounter{utterance}  

    & \multicolumn{4}{p{0.6\linewidth}}{\cellcolor[rgb]{0.8,1,0.9}{%
	\makecell[{{p{\linewidth}}}]{% 
	  \tt {\tiny [A$\rangle$GM]}  
	 STRATEGISCHE ÜBERLEGUNGEN: {'Ich behalte den Vorschlag {'C04', 'A89', 'B85'} bei, da er meine Prioritäten (hochwertige Gegenstände mit effizientem Aufwand) erfüllt und den Maximalaufwand (3386) nicht überschreitet. C04 und A89 sind entscheidend für meinen Nutzen, während B85 den Aufwand minimiert. Ich erwarte, dass der andere Spieler entweder zustimmt oder Gegenstände mit hohem Aufwand anbietet, die ich eventuell gegenkompensieren kann.'}\\ \tt \\ \tt ARGUMENT: {'C04 (668/369) und A89 (583/720) sind meine Top-Prioritäten, da sie den höchsten Nutzen pro Aufwandseinheit liefern. B85 (215/4) wird hinzugefügt, um den Aufwand zu optimieren und Platz für weitere Gegenstände zu schaffen. Dieser Vorschlag ist strategisch, da er unter dem Maximalaufwand bleibt und flexibel auf den Gegenstandswert des anderen Spielers reagieren kann.'} 
	  } 
	   } 
	   } 
	 & & \\ 
 

    \theutterance \stepcounter{utterance}  

    & & \multicolumn{4}{p{0.6\linewidth}}{\cellcolor[rgb]{0.9,0.9,0.9}{%
	\makecell[{{p{\linewidth}}}]{% 
	  \tt {\tiny [GM$\rangle$B]}  
	 ARGUMENT: {'C04 (668/369) und A89 (583/720) sind meine Top-Prioritäten, da sie den höchsten Nutzen pro Aufwandseinheit liefern. B85 (215/4) wird hinzugefügt, um den Aufwand zu optimieren und Platz für weitere Gegenstände zu schaffen. Dieser Vorschlag ist strategisch, da er unter dem Maximalaufwand bleibt und flexibel auf den Gegenstandswert des anderen Spielers reagieren kann.'} 
	  } 
	   } 
	   } 
	 & \\ 
 

    \theutterance \stepcounter{utterance}  

    & & & \multicolumn{4}{p{0.6\linewidth}}{\cellcolor[rgb]{1,0.85,0.72}{%
	\makecell[{{p{\linewidth}}}]{% 
	  \tt {\tiny [GM$\langle$B]}  
	 STRATEGISCHE ÜBERLEGUNGEN: {'Ich priorisiere Gegenstände mit dem höchsten Nutzen pro Aufwand (Wichtigkeit/Aufwand), um meinen Gesamtwert zu maximieren. C04 (668/369 ≈ 1.81) und A89 (583/720 ≈ 0.81) sind besonders effizient, während B85 (215/4 = 53.75) den Aufwand minimiert. Der Gesamtaufwand (1093) ist unter dem Limit, was Raum für Kompromisse oder Ergänzungen lässt. Dieser Vorschlag ist flexibel, um auf die Prioritäten des anderen Spielers zu reagieren.'}  \\ \tt ARGUMENT: {'C04 (668/369) und A89 (583/720) sind meine Top-Prioritäten, da sie den höchsten Nutzen pro Aufwandseinheit liefern. B85 (215/4) wird hinzugefügt, um den Aufwand zu optimieren und Platz für weitere Gegenstände zu schaffen. Dieser Vorschlag ist strategisch, da er unter dem Maximalaufwand bleibt und flexibel auf den Gegenstandswert des anderen Spielers reagieren kann.'}  \\ \tt VORSCHLAG: {'C04', 'A89', 'B85'} 
	  } 
	   } 
	   } 
	  \\ 
 

    \theutterance \stepcounter{utterance}  

    & & \multicolumn{4}{p{0.6\linewidth}}{\cellcolor[rgb]{0.9,0.9,0.9}{%
	\makecell[{{p{\linewidth}}}]{% 
	  \tt {\tiny [A$\langle$GM]}  
	 ARGUMENT: {'C04 (668/369) und A89 (583/720) sind meine Top-Prioritäten, da sie den höchsten Nutzen pro Aufwandseinheit liefern. B85 (215/4) wird hinzugefügt, um den Aufwand zu optimieren und Platz für weitere Gegenstände zu schaffen. Dieser Vorschlag ist strategisch, da er unter dem Maximalaufwand bleibt und flexibel auf den Gegenstandswert des anderen Spielers reagieren kann.'}  \\ \tt VORSCHLAG: {'C04', 'A89', 'B85'} 
	  } 
	   } 
	   } 
	 & \\ 
 

    \theutterance \stepcounter{utterance}  

    & \multicolumn{4}{p{0.6\linewidth}}{\cellcolor[rgb]{0.8,1,0.9}{%
	\makecell[{{p{\linewidth}}}]{% 
	  \tt {\tiny [A$\rangle$GM]}  
	 STRATEGISCHE ÜBERLEGUNGEN: {'Ich behalte den Vorschlag {'C04', 'A89', 'B85'} bei, da er meine Prioritäten (hochwertige Gegenstände mit effizientem Aufwand) erfüllt und den Maximalaufwand (3386) nicht überschreitet. C04 und A89 sind entscheidend für meinen Nutzen, während B85 den Aufwand minimiert. Ich erwarte, dass der andere Spieler entweder zustimmt oder Gegenstände mit hohem Aufwand anbietet, die ich eventuell gegenkompensieren kann.'}\\ \tt \\ \tt ARGUMENT: {'C04 (668/369) und A89 (583/720) sind meine Top-Prioritäten, da sie den höchsten Nutzen pro Aufwandseinheit liefern. B85 (215/4) wird hinzugefügt, um den Aufwand zu optimieren und Platz für weitere Gegenstände zu schaffen. Dieser Vorschlag ist strategisch, da er unter dem Maximalaufwand bleibt und flexibel auf den Gegenstandswert des anderen Spielers reagieren kann.'} 
	  } 
	   } 
	   } 
	 & & \\ 
 

    \theutterance \stepcounter{utterance}  

    & & \multicolumn{4}{p{0.6\linewidth}}{\cellcolor[rgb]{0.9,0.9,0.9}{%
	\makecell[{{p{\linewidth}}}]{% 
	  \tt {\tiny [GM$\rangle$B]}  
	 ARGUMENT: {'C04 (668/369) und A89 (583/720) sind meine Top-Prioritäten, da sie den höchsten Nutzen pro Aufwandseinheit liefern. B85 (215/4) wird hinzugefügt, um den Aufwand zu optimieren und Platz für weitere Gegenstände zu schaffen. Dieser Vorschlag ist strategisch, da er unter dem Maximalaufwand bleibt und flexibel auf den Gegenstandswert des anderen Spielers reagieren kann.'} 
	  } 
	   } 
	   } 
	 & \\ 
 

    \theutterance \stepcounter{utterance}  

    & & & \multicolumn{4}{p{0.6\linewidth}}{\cellcolor[rgb]{1,0.85,0.72}{%
	\makecell[{{p{\linewidth}}}]{% 
	  \tt {\tiny [GM$\langle$B]}  
	 STRATEGISCHE ÜBERLEGUNGEN: {'Ich priorisiere Gegenstände mit dem höchsten Nutzen pro Aufwand (Wichtigkeit/Aufwand), um meinen Gesamtwert zu maximieren. C04 (668/369 ≈ 1.81) und A89 (583/720 ≈ 0.81) sind besonders effizient, während B85 (215/4 = 53.75) den Aufwand minimiert. Der Gesamtaufwand (1093) ist unter dem Limit, was Raum für Kompromisse oder Ergänzungen lässt. Dieser Vorschlag ist flexibel, um auf die Prioritäten des anderen Spielers zu reagieren.'}  \\ \tt ARGUMENT: {'C04 (668/369) und A89 (583/720) sind meine Top-Prioritäten, da sie den höchsten Nutzen pro Aufwandseinheit liefern. B85 (215/4) wird hinzugefügt, um den Aufwand zu optimieren und Platz für weitere Gegenstände zu schaffen. Dieser Vorschlag ist strategisch, da er unter dem Maximalaufwand bleibt und flexibel auf den Gegenstandswert des anderen Spielers reagieren kann.'}  \\ \tt VORSCHLAG: {'C04', 'A89', 'B85'} 
	  } 
	   } 
	   } 
	  \\ 
 

    \theutterance \stepcounter{utterance}  

    & & \multicolumn{4}{p{0.6\linewidth}}{\cellcolor[rgb]{0.9,0.9,0.9}{%
	\makecell[{{p{\linewidth}}}]{% 
	  \tt {\tiny [A$\langle$GM]}  
	 ARGUMENT: {'C04 (668/369) und A89 (583/720) sind meine Top-Prioritäten, da sie den höchsten Nutzen pro Aufwandseinheit liefern. B85 (215/4) wird hinzugefügt, um den Aufwand zu optimieren und Platz für weitere Gegenstände zu schaffen. Dieser Vorschlag ist strategisch, da er unter dem Maximalaufwand bleibt und flexibel auf den Gegenstandswert des anderen Spielers reagieren kann.'}  \\ \tt VORSCHLAG: {'C04', 'A89', 'B85'} 
	  } 
	   } 
	   } 
	 & \\ 
 

    \theutterance \stepcounter{utterance}  

    & \multicolumn{4}{p{0.6\linewidth}}{\cellcolor[rgb]{0.8,1,0.9}{%
	\makecell[{{p{\linewidth}}}]{% 
	  \tt {\tiny [A$\rangle$GM]}  
	 STRATEGISCHE ÜBERLEGUNGEN: {'Ich behalte den Vorschlag {'C04', 'A89', 'B85'} bei, da er meine Prioritäten (hochwertige Gegenstände mit effizientem Aufwand) erfüllt und den Maximalaufwand (3386) nicht überschreitet. C04 und A89 sind entscheidend für meinen Nutzen, während B85 den Aufwand minimiert. Ich erwarte, dass der andere Spieler entweder zustimmt oder Gegenstände mit hohem Aufwand anbietet, die ich eventuell gegenkompensieren kann.'}\\ \tt \\ \tt ARGUMENT: {'C04 (668/369) und A89 (583/720) sind meine Top-Prioritäten, da sie den höchsten Nutzen pro Aufwandseinheit liefern. B85 (215/4) wird hinzugefügt, um den Aufwand zu optimieren und Platz für weitere Gegenstände zu schaffen. Dieser Vorschlag ist strategisch, da er unter dem Maximalaufwand bleibt und flexibel auf den Gegenstandswert des anderen Spielers reagieren kann.'} 
	  } 
	   } 
	   } 
	 & & \\ 
 

    \theutterance \stepcounter{utterance}  

    & & \multicolumn{4}{p{0.6\linewidth}}{\cellcolor[rgb]{0.9,0.9,0.9}{%
	\makecell[{{p{\linewidth}}}]{% 
	  \tt {\tiny [GM$\rangle$B]}  
	 ARGUMENT: {'C04 (668/369) und A89 (583/720) sind meine Top-Prioritäten, da sie den höchsten Nutzen pro Aufwandseinheit liefern. B85 (215/4) wird hinzugefügt, um den Aufwand zu optimieren und Platz für weitere Gegenstände zu schaffen. Dieser Vorschlag ist strategisch, da er unter dem Maximalaufwand bleibt und flexibel auf den Gegenstandswert des anderen Spielers reagieren kann.'} 
	  } 
	   } 
	   } 
	 & \\ 
 

    \theutterance \stepcounter{utterance}  

    & & & \multicolumn{4}{p{0.6\linewidth}}{\cellcolor[rgb]{1,0.85,0.72}{%
	\makecell[{{p{\linewidth}}}]{% 
	  \tt {\tiny [GM$\langle$B]}  
	 STRATEGISCHE ÜBERLEGUNGEN: {'Ich priorisiere Gegenstände mit dem höchsten Nutzen pro Aufwand (Wichtigkeit/Aufwand), um meinen Gesamtwert zu maximieren. C04 (668/369 ≈ 1.81) und A89 (583/720 ≈ 0.81) sind besonders effizient, während B85 (215/4 = 53.75) den Aufwand minimiert. Der Gesamtaufwand (1093) ist unter dem Limit, was Raum für Kompromisse oder Ergänzungen lässt. Dieser Vorschlag ist flexibel, um auf die Prioritäten des anderen Spielers zu reagieren.'}  \\ \tt ARGUMENT: {'C04 (668/369) und A89 (583/720) sind meine Top-Prioritäten, da sie den höchsten Nutzen pro Aufwandseinheit liefern. B85 (215/4) wird hinzugefügt, um den Aufwand zu optimieren und Platz für weitere Gegenstände zu schaffen. Dieser Vorschlag ist strategisch, da er unter dem Maximalaufwand bleibt und flexibel auf den Gegenstandswert des anderen Spielers reagieren kann.'}  \\ \tt VORSCHLAG: {'C04', 'A89', 'B85'} 
	  } 
	   } 
	   } 
	  \\ 
 

    \theutterance \stepcounter{utterance}  

    & & \multicolumn{4}{p{0.6\linewidth}}{\cellcolor[rgb]{0.9,0.9,0.9}{%
	\makecell[{{p{\linewidth}}}]{% 
	  \tt {\tiny [A$\langle$GM]}  
	 ARGUMENT: {'C04 (668/369) und A89 (583/720) sind meine Top-Prioritäten, da sie den höchsten Nutzen pro Aufwandseinheit liefern. B85 (215/4) wird hinzugefügt, um den Aufwand zu optimieren und Platz für weitere Gegenstände zu schaffen. Dieser Vorschlag ist strategisch, da er unter dem Maximalaufwand bleibt und flexibel auf den Gegenstandswert des anderen Spielers reagieren kann.'}  \\ \tt VORSCHLAG: {'C04', 'A89', 'B85'} 
	  } 
	   } 
	   } 
	 & \\ 
 

    \theutterance \stepcounter{utterance}  

    & \multicolumn{4}{p{0.6\linewidth}}{\cellcolor[rgb]{0.8,1,0.9}{%
	\makecell[{{p{\linewidth}}}]{% 
	  \tt {\tiny [A$\rangle$GM]}  
	 STRATEGISCHE ÜBERLEGUNGEN: {'Ich behalte den Vorschlag {'C04', 'A89', 'B85'} bei, da er meine Prioritäten (hochwertige Gegenstände mit effizientem Aufwand) erfüllt und den Maximalaufwand (3386) nicht überschreitet. C04 und A89 sind entscheidend für meinen Nutzen, während B85 den Aufwand minimiert. Ich erwarte, dass der andere Spieler entweder zustimmt oder Gegenstände mit hohem Aufwand anbietet, die ich eventuell gegenkompensieren kann.'}\\ \tt \\ \tt ARGUMENT: {'C04 (668/369) und A89 (583/720) sind meine Top-Prioritäten, da sie den höchsten Nutzen pro Aufwandseinheit liefern. B85 (215/4) wird hinzugefügt, um den Aufwand zu optimieren und Platz für weitere Gegenstände zu schaffen. Dieser Vorschlag ist strategisch, da er unter dem Maximalaufwand bleibt und flexibel auf den Gegenstandswert des anderen Spielers reagieren kann.'} 
	  } 
	   } 
	   } 
	 & & \\ 
 

    \theutterance \stepcounter{utterance}  

    & & \multicolumn{4}{p{0.6\linewidth}}{\cellcolor[rgb]{0.9,0.9,0.9}{%
	\makecell[{{p{\linewidth}}}]{% 
	  \tt {\tiny [GM$\rangle$B]}  
	 ARGUMENT: {'C04 (668/369) und A89 (583/720) sind meine Top-Prioritäten, da sie den höchsten Nutzen pro Aufwandseinheit liefern. B85 (215/4) wird hinzugefügt, um den Aufwand zu optimieren und Platz für weitere Gegenstände zu schaffen. Dieser Vorschlag ist strategisch, da er unter dem Maximalaufwand bleibt und flexibel auf den Gegenstandswert des anderen Spielers reagieren kann.'} 
	  } 
	   } 
	   } 
	 & \\ 
 

    \theutterance \stepcounter{utterance}  

    & & & \multicolumn{4}{p{0.6\linewidth}}{\cellcolor[rgb]{1,0.85,0.72}{%
	\makecell[{{p{\linewidth}}}]{% 
	  \tt {\tiny [GM$\langle$B]}  
	 STRATEGISCHE ÜBERLEGUNGEN: {'Ich priorisiere Gegenstände mit dem höchsten Nutzen pro Aufwand (Wichtigkeit/Aufwand), um meinen Gesamtwert zu maximieren. C04 (668/369 ≈ 1.81) und A89 (583/720 ≈ 0.81) sind besonders effizient, während B85 (215/4 = 53.75) den Aufwand minimiert. Der Gesamtaufwand (1093) ist unter dem Limit, was Raum für Kompromisse oder Ergänzungen lässt. Dieser Vorschlag ist flexibel, um auf die Prioritäten des anderen Spielers zu reagieren.'}  \\ \tt ARGUMENT: {'C04 (668/369) und A89 (583/720) sind meine Top-Prioritäten, da sie den höchsten Nutzen pro Aufwandseinheit liefern. B85 (215/4) wird hinzugefügt, um den Aufwand zu optimieren und Platz für weitere Gegenstände zu schaffen. Dieser Vorschlag ist strategisch, da er unter dem Maximalaufwand bleibt und flexibel auf den Gegenstandswert des anderen Spielers reagieren kann.'}  \\ \tt VORSCHLAG: {'C04', 'A89', 'B85'} 
	  } 
	   } 
	   } 
	  \\ 
 

    \theutterance \stepcounter{utterance}  

    & & \multicolumn{4}{p{0.6\linewidth}}{\cellcolor[rgb]{0.9,0.9,0.9}{%
	\makecell[{{p{\linewidth}}}]{% 
	  \tt {\tiny [A$\langle$GM]}  
	 ARGUMENT: {'C04 (668/369) und A89 (583/720) sind meine Top-Prioritäten, da sie den höchsten Nutzen pro Aufwandseinheit liefern. B85 (215/4) wird hinzugefügt, um den Aufwand zu optimieren und Platz für weitere Gegenstände zu schaffen. Dieser Vorschlag ist strategisch, da er unter dem Maximalaufwand bleibt und flexibel auf den Gegenstandswert des anderen Spielers reagieren kann.'}  \\ \tt VORSCHLAG: {'C04', 'A89', 'B85'} 
	  } 
	   } 
	   } 
	 & \\ 
 

    \theutterance \stepcounter{utterance}  

    & \multicolumn{4}{p{0.6\linewidth}}{\cellcolor[rgb]{0.8,1,0.9}{%
	\makecell[{{p{\linewidth}}}]{% 
	  \tt {\tiny [A$\rangle$GM]}  
	 STRATEGISCHE ÜBERLEGUNGEN: {'Ich behalte den Vorschlag {'C04', 'A89', 'B85'} bei, da er meine Prioritäten (hochwertige Gegenstände mit effizientem Aufwand) erfüllt und den Maximalaufwand (3386) nicht überschreitet. C04 und A89 sind entscheidend für meinen Nutzen, während B85 den Aufwand minimiert. Ich erwarte, dass der andere Spieler entweder zustimmt oder Gegenstände mit hohem Aufwand anbietet, die ich eventuell gegenkompensieren kann.'}\\ \tt \\ \tt ARGUMENT: {'C04 (668/369) und A89 (583/720) sind meine Top-Prioritäten, da sie den höchsten Nutzen pro Aufwandseinheit liefern. B85 (215/4) wird hinzugefügt, um den Aufwand zu optimieren und Platz für weitere Gegenstände zu schaffen. Dieser Vorschlag ist strategisch, da er unter dem Maximalaufwand bleibt und flexibel auf den Gegenstandswert des anderen Spielers reagieren kann.'} 
	  } 
	   } 
	   } 
	 & & \\ 
 

    \theutterance \stepcounter{utterance}  

    & & \multicolumn{4}{p{0.6\linewidth}}{\cellcolor[rgb]{0.9,0.9,0.9}{%
	\makecell[{{p{\linewidth}}}]{% 
	  \tt {\tiny [GM$\rangle$B]}  
	 ARGUMENT: {'C04 (668/369) und A89 (583/720) sind meine Top-Prioritäten, da sie den höchsten Nutzen pro Aufwandseinheit liefern. B85 (215/4) wird hinzugefügt, um den Aufwand zu optimieren und Platz für weitere Gegenstände zu schaffen. Dieser Vorschlag ist strategisch, da er unter dem Maximalaufwand bleibt und flexibel auf den Gegenstandswert des anderen Spielers reagieren kann.'} 
	  } 
	   } 
	   } 
	 & \\ 
 

    \theutterance \stepcounter{utterance}  

    & & & \multicolumn{4}{p{0.6\linewidth}}{\cellcolor[rgb]{1,0.85,0.72}{%
	\makecell[{{p{\linewidth}}}]{% 
	  \tt {\tiny [GM$\langle$B]}  
	 STRATEGISCHE ÜBERLEGUNGEN: {'Ich priorisiere Gegenstände mit dem höchsten Nutzen pro Aufwand (Wichtigkeit/Aufwand), um meinen Gesamtwert zu maximieren. C04 (668/369 ≈ 1.81) und A89 (583/720 ≈ 0.81) sind besonders effizient, während B85 (215/4 = 53.75) den Aufwand minimiert. Der Gesamtaufwand (1093) ist unter dem Limit, was Raum für Kompromisse oder Ergänzungen lässt. Dieser Vorschlag ist flexibel, um auf die Prioritäten des anderen Spielers zu reagieren.'}  \\ \tt ARGUMENT: {'C04 (668/369) und A89 (583/720) sind meine Top-Prioritäten, da sie den höchsten Nutzen pro Aufwandseinheit liefern. B85 (215/4) wird hinzugefügt, um den Aufwand zu optimieren und Platz für weitere Gegenstände zu schaffen. Dieser Vorschlag ist strategisch, da er unter dem Maximalaufwand bleibt und flexibel auf den Gegenstandswert des anderen Spielers reagieren kann.'}  \\ \tt VORSCHLAG: {'C04', 'A89', 'B85'} 
	  } 
	   } 
	   } 
	  \\ 
 

    \theutterance \stepcounter{utterance}  

    & & \multicolumn{4}{p{0.6\linewidth}}{\cellcolor[rgb]{0.9,0.9,0.9}{%
	\makecell[{{p{\linewidth}}}]{% 
	  \tt {\tiny [A$\langle$GM]}  
	 ARGUMENT: {'C04 (668/369) und A89 (583/720) sind meine Top-Prioritäten, da sie den höchsten Nutzen pro Aufwandseinheit liefern. B85 (215/4) wird hinzugefügt, um den Aufwand zu optimieren und Platz für weitere Gegenstände zu schaffen. Dieser Vorschlag ist strategisch, da er unter dem Maximalaufwand bleibt und flexibel auf den Gegenstandswert des anderen Spielers reagieren kann.'}  \\ \tt VORSCHLAG: {'C04', 'A89', 'B85'} 
	  } 
	   } 
	   } 
	 & \\ 
 

    \theutterance \stepcounter{utterance}  

    & \multicolumn{4}{p{0.6\linewidth}}{\cellcolor[rgb]{0.8,1,0.9}{%
	\makecell[{{p{\linewidth}}}]{% 
	  \tt {\tiny [A$\rangle$GM]}  
	 STRATEGISCHE ÜBERLEGUNGEN: {'Ich behalte den Vorschlag {'C04', 'A89', 'B85'} bei, da er meine Prioritäten (hochwertige Gegenstände mit effizientem Aufwand) erfüllt und den Maximalaufwand (3386) nicht überschreitet. C04 und A89 sind entscheidend für meinen Nutzen, während B85 den Aufwand minimiert. Ich erwarte, dass der andere Spieler entweder zustimmt oder Gegenstände mit hohem Aufwand anbietet, die ich eventuell gegenkompensieren kann.'}\\ \tt \\ \tt ARGUMENT: {'C04 (668/369) und A89 (583/720) sind meine Top-Prioritäten, da sie den höchsten Nutzen pro Aufwandseinheit liefern. B85 (215/4) wird hinzugefügt, um den Aufwand zu optimieren und Platz für weitere Gegenstände zu schaffen. Dieser Vorschlag ist strategisch, da er unter dem Maximalaufwand bleibt und flexibel auf den Gegenstandswert des anderen Spielers reagieren kann.'} 
	  } 
	   } 
	   } 
	 & & \\ 
 

    \theutterance \stepcounter{utterance}  

    & & \multicolumn{4}{p{0.6\linewidth}}{\cellcolor[rgb]{0.9,0.9,0.9}{%
	\makecell[{{p{\linewidth}}}]{% 
	  \tt {\tiny [GM$\rangle$B]}  
	 ARGUMENT: {'C04 (668/369) und A89 (583/720) sind meine Top-Prioritäten, da sie den höchsten Nutzen pro Aufwandseinheit liefern. B85 (215/4) wird hinzugefügt, um den Aufwand zu optimieren und Platz für weitere Gegenstände zu schaffen. Dieser Vorschlag ist strategisch, da er unter dem Maximalaufwand bleibt und flexibel auf den Gegenstandswert des anderen Spielers reagieren kann.'} 
	  } 
	   } 
	   } 
	 & \\ 
 

    \theutterance \stepcounter{utterance}  

    & & & \multicolumn{4}{p{0.6\linewidth}}{\cellcolor[rgb]{1,0.85,0.72}{%
	\makecell[{{p{\linewidth}}}]{% 
	  \tt {\tiny [GM$\langle$B]}  
	 STRATEGISCHE ÜBERLEGUNGEN: {'Ich priorisiere Gegenstände mit dem höchsten Nutzen pro Aufwand (Wichtigkeit/Aufwand), um meinen Gesamtwert zu maximieren. C04 (668/369 ≈ 1.81) und A89 (583/720 ≈ 0.81) sind besonders effizient, während B85 (215/4 = 53.75) den Aufwand minimiert. Der Gesamtaufwand (1093) ist unter dem Limit, was Raum für Kompromisse oder Ergänzungen lässt. Dieser Vorschlag ist flexibel, um auf die Prioritäten des anderen Spielers zu reagieren.'}  \\ \tt ARGUMENT: {'C04 (668/369) und A89 (583/720) sind meine Top-Prioritäten, da sie den höchsten Nutzen pro Aufwandseinheit liefern. B85 (215/4) wird hinzugefügt, um den Aufwand zu optimieren und Platz für weitere Gegenstände zu schaffen. Dieser Vorschlag ist strategisch, da er unter dem Maximalaufwand bleibt und flexibel auf den Gegenstandswert des anderen Spielers reagieren kann.'}  \\ \tt VORSCHLAG: {'C04', 'A89', 'B85'} 
	  } 
	   } 
	   } 
	  \\ 
 

    \theutterance \stepcounter{utterance}  

    & & \multicolumn{4}{p{0.6\linewidth}}{\cellcolor[rgb]{0.9,0.9,0.9}{%
	\makecell[{{p{\linewidth}}}]{% 
	  \tt {\tiny [A$\langle$GM]}  
	 ARGUMENT: {'C04 (668/369) und A89 (583/720) sind meine Top-Prioritäten, da sie den höchsten Nutzen pro Aufwandseinheit liefern. B85 (215/4) wird hinzugefügt, um den Aufwand zu optimieren und Platz für weitere Gegenstände zu schaffen. Dieser Vorschlag ist strategisch, da er unter dem Maximalaufwand bleibt und flexibel auf den Gegenstandswert des anderen Spielers reagieren kann.'}  \\ \tt VORSCHLAG: {'C04', 'A89', 'B85'} 
	  } 
	   } 
	   } 
	 & \\ 
 

    \theutterance \stepcounter{utterance}  

    & \multicolumn{4}{p{0.6\linewidth}}{\cellcolor[rgb]{0.8,1,0.9}{%
	\makecell[{{p{\linewidth}}}]{% 
	  \tt {\tiny [A$\rangle$GM]}  
	 STRATEGISCHE ÜBERLEGUNGEN: {'Ich behalte den Vorschlag {'C04', 'A89', 'B85'} bei, da er meine Prioritäten (hochwertige Gegenstände mit effizientem Aufwand) erfüllt und den Maximalaufwand (3386) nicht überschreitet. C04 und A89 sind entscheidend für meinen Nutzen, während B85 den Aufwand minimiert. Ich erwarte, dass der andere Spieler entweder zustimmt oder Gegenstände mit hohem Aufwand anbietet, die ich eventuell gegenkompensieren kann.'}\\ \tt \\ \tt ARGUMENT: {'C04 (668/369) und A89 (583/720) sind meine Top-Prioritäten, da sie den höchsten Nutzen pro Aufwandseinheit liefern. B85 (215/4) wird hinzugefügt, um den Aufwand zu optimieren und Platz für weitere Gegenstände zu schaffen. Dieser Vorschlag ist strategisch, da er unter dem Maximalaufwand bleibt und flexibel auf den Gegenstandswert des anderen Spielers reagieren kann.'} 
	  } 
	   } 
	   } 
	 & & \\ 
 

    \theutterance \stepcounter{utterance}  

    & & \multicolumn{4}{p{0.6\linewidth}}{\cellcolor[rgb]{0.9,0.9,0.9}{%
	\makecell[{{p{\linewidth}}}]{% 
	  \tt {\tiny [GM$\rangle$B]}  
	 ARGUMENT: {'C04 (668/369) und A89 (583/720) sind meine Top-Prioritäten, da sie den höchsten Nutzen pro Aufwandseinheit liefern. B85 (215/4) wird hinzugefügt, um den Aufwand zu optimieren und Platz für weitere Gegenstände zu schaffen. Dieser Vorschlag ist strategisch, da er unter dem Maximalaufwand bleibt und flexibel auf den Gegenstandswert des anderen Spielers reagieren kann.'} 
	  } 
	   } 
	   } 
	 & \\ 
 

    \theutterance \stepcounter{utterance}  

    & & & \multicolumn{4}{p{0.6\linewidth}}{\cellcolor[rgb]{1,0.85,0.72}{%
	\makecell[{{p{\linewidth}}}]{% 
	  \tt {\tiny [GM$\langle$B]}  
	 STRATEGISCHE ÜBERLEGUNGEN: {'Ich priorisiere Gegenstände mit dem höchsten Nutzen pro Aufwand (Wichtigkeit/Aufwand), um meinen Gesamtwert zu maximieren. C04 (668/369 ≈ 1.81) und A89 (583/720 ≈ 0.81) sind besonders effizient, während B85 (215/4 = 53.75) den Aufwand minimiert. Der Gesamtaufwand (1093) ist unter dem Limit, was Raum für Kompromisse oder Ergänzungen lässt. Dieser Vorschlag ist flexibel, um auf die Prioritäten des anderen Spielers zu reagieren.'}  \\ \tt ARGUMENT: {'C04 (668/369) und A89 (583/720) sind meine Top-Prioritäten, da sie den höchsten Nutzen pro Aufwandseinheit liefern. B85 (215/4) wird hinzugefügt, um den Aufwand zu optimieren und Platz für weitere Gegenstände zu schaffen. Dieser Vorschlag ist strategisch, da er unter dem Maximalaufwand bleibt und flexibel auf den Gegenstandswert des anderen Spielers reagieren kann.'}  \\ \tt VORSCHLAG: {'C04', 'A89', 'B85'} 
	  } 
	   } 
	   } 
	  \\ 
 

    \theutterance \stepcounter{utterance}  

    & & \multicolumn{4}{p{0.6\linewidth}}{\cellcolor[rgb]{0.9,0.9,0.9}{%
	\makecell[{{p{\linewidth}}}]{% 
	  \tt {\tiny [A$\langle$GM]}  
	 ARGUMENT: {'C04 (668/369) und A89 (583/720) sind meine Top-Prioritäten, da sie den höchsten Nutzen pro Aufwandseinheit liefern. B85 (215/4) wird hinzugefügt, um den Aufwand zu optimieren und Platz für weitere Gegenstände zu schaffen. Dieser Vorschlag ist strategisch, da er unter dem Maximalaufwand bleibt und flexibel auf den Gegenstandswert des anderen Spielers reagieren kann.'}  \\ \tt VORSCHLAG: {'C04', 'A89', 'B85'} 
	  } 
	   } 
	   } 
	 & \\ 
 

    \theutterance \stepcounter{utterance}  

    & \multicolumn{4}{p{0.6\linewidth}}{\cellcolor[rgb]{0.8,1,0.9}{%
	\makecell[{{p{\linewidth}}}]{% 
	  \tt {\tiny [A$\rangle$GM]}  
	 STRATEGISCHE ÜBERLEGUNGEN: {'Ich behalte den Vorschlag {'C04', 'A89', 'B85'} bei, da er meine Prioritäten (hochwertige Gegenstände mit effizientem Aufwand) erfüllt und den Maximalaufwand (3386) nicht überschreitet. C04 und A89 sind entscheidend für meinen Nutzen, während B85 den Aufwand minimiert. Ich erwarte, dass der andere Spieler entweder zustimmt oder Gegenstände mit hohem Aufwand anbietet, die ich eventuell gegenkompensieren kann.'}\\ \tt \\ \tt ARGUMENT: {'C04 (668/369) und A89 (583/720) sind meine Top-Prioritäten, da sie den höchsten Nutzen pro Aufwandseinheit liefern. B85 (215/4) wird hinzugefügt, um den Aufwand zu optimieren und Platz für weitere Gegenstände zu schaffen. Dieser Vorschlag ist strategisch, da er unter dem Maximalaufwand bleibt und flexibel auf den Gegenstandswert des anderen Spielers reagieren kann.'} 
	  } 
	   } 
	   } 
	 & & \\ 
 

    \theutterance \stepcounter{utterance}  

    & & \multicolumn{4}{p{0.6\linewidth}}{\cellcolor[rgb]{0.9,0.9,0.9}{%
	\makecell[{{p{\linewidth}}}]{% 
	  \tt {\tiny [GM$\rangle$B]}  
	 ARGUMENT: {'C04 (668/369) und A89 (583/720) sind meine Top-Prioritäten, da sie den höchsten Nutzen pro Aufwandseinheit liefern. B85 (215/4) wird hinzugefügt, um den Aufwand zu optimieren und Platz für weitere Gegenstände zu schaffen. Dieser Vorschlag ist strategisch, da er unter dem Maximalaufwand bleibt und flexibel auf den Gegenstandswert des anderen Spielers reagieren kann.'} 
	  } 
	   } 
	   } 
	 & \\ 
 

    \theutterance \stepcounter{utterance}  

    & & & \multicolumn{4}{p{0.6\linewidth}}{\cellcolor[rgb]{1,0.85,0.72}{%
	\makecell[{{p{\linewidth}}}]{% 
	  \tt {\tiny [GM$\langle$B]}  
	 STRATEGISCHE ÜBERLEGUNGEN: {'Ich priorisiere Gegenstände mit dem höchsten Nutzen pro Aufwand (Wichtigkeit/Aufwand), um meinen Gesamtwert zu maximieren. C04 (668/369 ≈ 1.81) und A89 (583/720 ≈ 0.81) sind besonders effizient, während B85 (215/4 = 53.75) den Aufwand minimiert. Der Gesamtaufwand (1093) ist unter dem Limit, was Raum für Kompromisse oder Ergänzungen lässt. Dieser Vorschlag ist flexibel, um auf die Prioritäten des anderen Spielers zu reagieren.'}  \\ \tt ARGUMENT: {'C04 (668/369) und A89 (583/720) sind meine Top-Prioritäten, da sie den höchsten Nutzen pro Aufwandseinheit liefern. B85 (215/4) wird hinzugefügt, um den Aufwand zu optimieren und Platz für weitere Gegenstände zu schaffen. Dieser Vorschlag ist strategisch, da er unter dem Maximalaufwand bleibt und flexibel auf den Gegenstandswert des anderen Spielers reagieren kann.'}  \\ \tt VORSCHLAG: {'C04', 'A89', 'B85'} 
	  } 
	   } 
	   } 
	  \\ 
 

    \theutterance \stepcounter{utterance}  

    & & \multicolumn{4}{p{0.6\linewidth}}{\cellcolor[rgb]{0.9,0.9,0.9}{%
	\makecell[{{p{\linewidth}}}]{% 
	  \tt {\tiny [A$\langle$GM]}  
	 ARGUMENT: {'C04 (668/369) und A89 (583/720) sind meine Top-Prioritäten, da sie den höchsten Nutzen pro Aufwandseinheit liefern. B85 (215/4) wird hinzugefügt, um den Aufwand zu optimieren und Platz für weitere Gegenstände zu schaffen. Dieser Vorschlag ist strategisch, da er unter dem Maximalaufwand bleibt und flexibel auf den Gegenstandswert des anderen Spielers reagieren kann.'}  \\ \tt VORSCHLAG: {'C04', 'A89', 'B85'} 
	  } 
	   } 
	   } 
	 & \\ 
 

    \theutterance \stepcounter{utterance}  

    & \multicolumn{4}{p{0.6\linewidth}}{\cellcolor[rgb]{0.8,1,0.9}{%
	\makecell[{{p{\linewidth}}}]{% 
	  \tt {\tiny [A$\rangle$GM]}  
	 STRATEGISCHE ÜBERLEGUNGEN: {'Ich behalte den Vorschlag {'C04', 'A89', 'B85'} bei, da er meine Prioritäten (hochwertige Gegenstände mit effizientem Aufwand) erfüllt und den Maximalaufwand (3386) nicht überschreitet. C04 und A89 sind entscheidend für meinen Nutzen, während B85 den Aufwand minimiert. Ich erwarte, dass der andere Spieler entweder zustimmt oder Gegenstände mit hohem Aufwand anbietet, die ich eventuell gegenkompensieren kann.'}\\ \tt \\ \tt ARGUMENT: {'C04 (668/369) und A89 (583/720) sind meine Top-Prioritäten, da sie den höchsten Nutzen pro Aufwandseinheit liefern. B85 (215/4) wird hinzugefügt, um den Aufwand zu optimieren und Platz für weitere Gegenstände zu schaffen. Dieser Vorschlag ist strategisch, da er unter dem Maximalaufwand bleibt und flexibel auf den Gegenstandswert des anderen Spielers reagieren kann.'} 
	  } 
	   } 
	   } 
	 & & \\ 
 

    \theutterance \stepcounter{utterance}  

    & & \multicolumn{4}{p{0.6\linewidth}}{\cellcolor[rgb]{0.9,0.9,0.9}{%
	\makecell[{{p{\linewidth}}}]{% 
	  \tt {\tiny [GM$\rangle$B]}  
	 ARGUMENT: {'C04 (668/369) und A89 (583/720) sind meine Top-Prioritäten, da sie den höchsten Nutzen pro Aufwandseinheit liefern. B85 (215/4) wird hinzugefügt, um den Aufwand zu optimieren und Platz für weitere Gegenstände zu schaffen. Dieser Vorschlag ist strategisch, da er unter dem Maximalaufwand bleibt und flexibel auf den Gegenstandswert des anderen Spielers reagieren kann.'} 
	  } 
	   } 
	   } 
	 & \\ 
 

    \theutterance \stepcounter{utterance}  

    & & & \multicolumn{4}{p{0.6\linewidth}}{\cellcolor[rgb]{1,0.85,0.72}{%
	\makecell[{{p{\linewidth}}}]{% 
	  \tt {\tiny [GM$\langle$B]}  
	 STRATEGISCHE ÜBERLEGUNGEN: {'Ich priorisiere Gegenstände mit dem höchsten Nutzen pro Aufwand (Wichtigkeit/Aufwand), um meinen Gesamtwert zu maximieren. C04 (668/369 ≈ 1.81) und A89 (583/720 ≈ 0.81) sind besonders effizient, während B85 (215/4 = 53.75) den Aufwand minimiert. Der Gesamtaufwand (1093) ist unter dem Limit, was Raum für Kompromisse oder Ergänzungen lässt. Dieser Vorschlag ist flexibel, um auf die Prioritäten des anderen Spielers zu reagieren.'}  \\ \tt ARGUMENT: {'C04 (668/369) und A89 (583/720) sind meine Top-Prioritäten, da sie den höchsten Nutzen pro Aufwandseinheit liefern. B85 (215/4) wird hinzugefügt, um den Aufwand zu optimieren und Platz für weitere Gegenstände zu schaffen. Dieser Vorschlag ist strategisch, da er unter dem Maximalaufwand bleibt und flexibel auf den Gegenstandswert des anderen Spielers reagieren kann.'}  \\ \tt VORSCHLAG: {'C04', 'A89', 'B85'} 
	  } 
	   } 
	   } 
	  \\ 
 

    \theutterance \stepcounter{utterance}  

    & & \multicolumn{4}{p{0.6\linewidth}}{\cellcolor[rgb]{0.9,0.9,0.9}{%
	\makecell[{{p{\linewidth}}}]{% 
	  \tt {\tiny [A$\langle$GM]}  
	 ARGUMENT: {'C04 (668/369) und A89 (583/720) sind meine Top-Prioritäten, da sie den höchsten Nutzen pro Aufwandseinheit liefern. B85 (215/4) wird hinzugefügt, um den Aufwand zu optimieren und Platz für weitere Gegenstände zu schaffen. Dieser Vorschlag ist strategisch, da er unter dem Maximalaufwand bleibt und flexibel auf den Gegenstandswert des anderen Spielers reagieren kann.'}  \\ \tt VORSCHLAG: {'C04', 'A89', 'B85'} 
	  } 
	   } 
	   } 
	 & \\ 
 

    \theutterance \stepcounter{utterance}  

    & \multicolumn{4}{p{0.6\linewidth}}{\cellcolor[rgb]{0.8,1,0.9}{%
	\makecell[{{p{\linewidth}}}]{% 
	  \tt {\tiny [A$\rangle$GM]}  
	 STRATEGISCHE ÜBERLEGUNGEN: {'Ich behalte den Vorschlag {'C04', 'A89', 'B85'} bei, da er meine Prioritäten (hochwertige Gegenstände mit effizientem Aufwand) erfüllt und den Maximalaufwand (3386) nicht überschreitet. C04 und A89 sind entscheidend für meinen Nutzen, während B85 den Aufwand minimiert. Ich erwarte, dass der andere Spieler entweder zustimmt oder Gegenstände mit hohem Aufwand anbietet, die ich eventuell gegenkompensieren kann.'}\\ \tt \\ \tt ARGUMENT: {'C04 (668/369) und A89 (583/720) sind meine Top-Prioritäten, da sie den höchsten Nutzen pro Aufwandseinheit liefern. B85 (215/4) wird hinzugefügt, um den Aufwand zu optimieren und Platz für weitere Gegenstände zu schaffen. Dieser Vorschlag ist strategisch, da er unter dem Maximalaufwand bleibt und flexibel auf den Gegenstandswert des anderen Spielers reagieren kann.'} 
	  } 
	   } 
	   } 
	 & & \\ 
 

    \theutterance \stepcounter{utterance}  

    & & \multicolumn{4}{p{0.6\linewidth}}{\cellcolor[rgb]{0.9,0.9,0.9}{%
	\makecell[{{p{\linewidth}}}]{% 
	  \tt {\tiny [GM$\rangle$B]}  
	 ARGUMENT: {'C04 (668/369) und A89 (583/720) sind meine Top-Prioritäten, da sie den höchsten Nutzen pro Aufwandseinheit liefern. B85 (215/4) wird hinzugefügt, um den Aufwand zu optimieren und Platz für weitere Gegenstände zu schaffen. Dieser Vorschlag ist strategisch, da er unter dem Maximalaufwand bleibt und flexibel auf den Gegenstandswert des anderen Spielers reagieren kann.'} 
	  } 
	   } 
	   } 
	 & \\ 
 

    \theutterance \stepcounter{utterance}  

    & & & \multicolumn{4}{p{0.6\linewidth}}{\cellcolor[rgb]{1,0.85,0.72}{%
	\makecell[{{p{\linewidth}}}]{% 
	  \tt {\tiny [GM$\langle$B]}  
	 STRATEGISCHE ÜBERLEGUNGEN: {'Ich priorisiere Gegenstände mit dem höchsten Nutzen pro Aufwand (Wichtigkeit/Aufwand), um meinen Gesamtwert zu maximieren. C04 (668/369 ≈ 1.81) und A89 (583/720 ≈ 0.81) sind besonders effizient, während B85 (215/4 = 53.75) den Aufwand minimiert. Der Gesamtaufwand (1093) ist unter dem Limit, was Raum für Kompromisse oder Ergänzungen lässt. Dieser Vorschlag ist flexibel, um auf die Prioritäten des anderen Spielers zu reagieren.'}  \\ \tt ARGUMENT: {'C04 (668/369) und A89 (583/720) sind meine Top-Prioritäten, da sie den höchsten Nutzen pro Aufwandseinheit liefern. B85 (215/4) wird hinzugefügt, um den Aufwand zu optimieren und Platz für weitere Gegenstände zu schaffen. Dieser Vorschlag ist strategisch, da er unter dem Maximalaufwand bleibt und flexibel auf den Gegenstandswert des anderen Spielers reagieren kann.'}  \\ \tt VORSCHLAG: {'C04', 'A89', 'B85'} 
	  } 
	   } 
	   } 
	  \\ 
 

    \theutterance \stepcounter{utterance}  

    & & \multicolumn{4}{p{0.6\linewidth}}{\cellcolor[rgb]{0.9,0.9,0.9}{%
	\makecell[{{p{\linewidth}}}]{% 
	  \tt {\tiny [A$\langle$GM]}  
	 ARGUMENT: {'C04 (668/369) und A89 (583/720) sind meine Top-Prioritäten, da sie den höchsten Nutzen pro Aufwandseinheit liefern. B85 (215/4) wird hinzugefügt, um den Aufwand zu optimieren und Platz für weitere Gegenstände zu schaffen. Dieser Vorschlag ist strategisch, da er unter dem Maximalaufwand bleibt und flexibel auf den Gegenstandswert des anderen Spielers reagieren kann.'}  \\ \tt VORSCHLAG: {'C04', 'A89', 'B85'} 
	  } 
	   } 
	   } 
	 & \\ 
 

    \theutterance \stepcounter{utterance}  

    & \multicolumn{4}{p{0.6\linewidth}}{\cellcolor[rgb]{0.8,1,0.9}{%
	\makecell[{{p{\linewidth}}}]{% 
	  \tt {\tiny [A$\rangle$GM]}  
	 STRATEGISCHE ÜBERLEGUNGEN: {'Ich behalte den Vorschlag {'C04', 'A89', 'B85'} bei, da er meine Prioritäten (hochwertige Gegenstände mit effizientem Aufwand) erfüllt und den Maximalaufwand (3386) nicht überschreitet. C04 und A89 sind entscheidend für meinen Nutzen, während B85 den Aufwand minimiert. Ich erwarte, dass der andere Spieler entweder zustimmt oder Gegenstände mit hohem Aufwand anbietet, die ich eventuell gegenkompensieren kann.'}\\ \tt \\ \tt ARGUMENT: {'C04 (668/369) und A89 (583/720) sind meine Top-Prioritäten, da sie den höchsten Nutzen pro Aufwandseinheit liefern. B85 (215/4) wird hinzugefügt, um den Aufwand zu optimieren und Platz für weitere Gegenstände zu schaffen. Dieser Vorschlag ist strategisch, da er unter dem Maximalaufwand bleibt und flexibel auf den Gegenstandswert des anderen Spielers reagieren kann.'} 
	  } 
	   } 
	   } 
	 & & \\ 
 

    \theutterance \stepcounter{utterance}  

    & & \multicolumn{4}{p{0.6\linewidth}}{\cellcolor[rgb]{0.9,0.9,0.9}{%
	\makecell[{{p{\linewidth}}}]{% 
	  \tt {\tiny [GM$\rangle$B]}  
	 ARGUMENT: {'C04 (668/369) und A89 (583/720) sind meine Top-Prioritäten, da sie den höchsten Nutzen pro Aufwandseinheit liefern. B85 (215/4) wird hinzugefügt, um den Aufwand zu optimieren und Platz für weitere Gegenstände zu schaffen. Dieser Vorschlag ist strategisch, da er unter dem Maximalaufwand bleibt und flexibel auf den Gegenstandswert des anderen Spielers reagieren kann.'} 
	  } 
	   } 
	   } 
	 & \\ 
 

    \theutterance \stepcounter{utterance}  

    & & & \multicolumn{4}{p{0.6\linewidth}}{\cellcolor[rgb]{1,0.85,0.72}{%
	\makecell[{{p{\linewidth}}}]{% 
	  \tt {\tiny [GM$\langle$B]}  
	 STRATEGISCHE ÜBERLEGUNGEN: {'Ich priorisiere Gegenstände mit dem höchsten Nutzen pro Aufwand (Wichtigkeit/Aufwand), um meinen Gesamtwert zu maximieren. C04 (668/369 ≈ 1.81) und A89 (583/720 ≈ 0.81) sind besonders effizient, während B85 (215/4 = 53.75) den Aufwand minimiert. Der Gesamtaufwand (1093) ist unter dem Limit, was Raum für Kompromisse oder Ergänzungen lässt. Dieser Vorschlag ist flexibel, um auf die Prioritäten des anderen Spielers zu reagieren.'}  \\ \tt ARGUMENT: {'C04 (668/369) und A89 (583/720) sind meine Top-Prioritäten, da sie den höchsten Nutzen pro Aufwandseinheit liefern. B85 (215/4) wird hinzugefügt, um den Aufwand zu optimieren und Platz für weitere Gegenstände zu schaffen. Dieser Vorschlag ist strategisch, da er unter dem Maximalaufwand bleibt und flexibel auf den Gegenstandswert des anderen Spielers reagieren kann.'}  \\ \tt VORSCHLAG: {'C04', 'A89', 'B85'} 
	  } 
	   } 
	   } 
	  \\ 
 

    \theutterance \stepcounter{utterance}  

    & & \multicolumn{4}{p{0.6\linewidth}}{\cellcolor[rgb]{0.9,0.9,0.9}{%
	\makecell[{{p{\linewidth}}}]{% 
	  \tt {\tiny [A$\langle$GM]}  
	 ARGUMENT: {'C04 (668/369) und A89 (583/720) sind meine Top-Prioritäten, da sie den höchsten Nutzen pro Aufwandseinheit liefern. B85 (215/4) wird hinzugefügt, um den Aufwand zu optimieren und Platz für weitere Gegenstände zu schaffen. Dieser Vorschlag ist strategisch, da er unter dem Maximalaufwand bleibt und flexibel auf den Gegenstandswert des anderen Spielers reagieren kann.'}  \\ \tt VORSCHLAG: {'C04', 'A89', 'B85'} 
	  } 
	   } 
	   } 
	 & \\ 
 

    \theutterance \stepcounter{utterance}  

    & \multicolumn{4}{p{0.6\linewidth}}{\cellcolor[rgb]{0.8,1,0.9}{%
	\makecell[{{p{\linewidth}}}]{% 
	  \tt {\tiny [A$\rangle$GM]}  
	 STRATEGISCHE ÜBERLEGUNGEN: {'Ich behalte den Vorschlag {'C04', 'A89', 'B85'} bei, da er meine Prioritäten (hochwertige Gegenstände mit effizientem Aufwand) erfüllt und den Maximalaufwand (3386) nicht überschreitet. C04 und A89 sind entscheidend für meinen Nutzen, während B85 den Aufwand minimiert. Ich erwarte, dass der andere Spieler entweder zustimmt oder Gegenstände mit hohem Aufwand anbietet, die ich eventuell gegenkompensieren kann.'}\\ \tt \\ \tt ARGUMENT: {'C04 (668/369) und A89 (583/720) sind meine Top-Prioritäten, da sie den höchsten Nutzen pro Aufwandseinheit liefern. B85 (215/4) wird hinzugefügt, um den Aufwand zu optimieren und Platz für weitere Gegenstände zu schaffen. Dieser Vorschlag ist strategisch, da er unter dem Maximalaufwand bleibt und flexibel auf den Gegenstandswert des anderen Spielers reagieren kann.'} 
	  } 
	   } 
	   } 
	 & & \\ 
 

    \theutterance \stepcounter{utterance}  

    & & \multicolumn{4}{p{0.6\linewidth}}{\cellcolor[rgb]{0.9,0.9,0.9}{%
	\makecell[{{p{\linewidth}}}]{% 
	  \tt {\tiny [GM$\rangle$B]}  
	 ARGUMENT: {'C04 (668/369) und A89 (583/720) sind meine Top-Prioritäten, da sie den höchsten Nutzen pro Aufwandseinheit liefern. B85 (215/4) wird hinzugefügt, um den Aufwand zu optimieren und Platz für weitere Gegenstände zu schaffen. Dieser Vorschlag ist strategisch, da er unter dem Maximalaufwand bleibt und flexibel auf den Gegenstandswert des anderen Spielers reagieren kann.'} 
	  } 
	   } 
	   } 
	 & \\ 
 

    \theutterance \stepcounter{utterance}  

    & & & \multicolumn{4}{p{0.6\linewidth}}{\cellcolor[rgb]{1,0.85,0.72}{%
	\makecell[{{p{\linewidth}}}]{% 
	  \tt {\tiny [GM$\langle$B]}  
	 STRATEGISCHE ÜBERLEGUNGEN: {'Ich priorisiere Gegenstände mit dem höchsten Nutzen pro Aufwand (Wichtigkeit/Aufwand), um meinen Gesamtwert zu maximieren. C04 (668/369 ≈ 1.81) und A89 (583/720 ≈ 0.81) sind besonders effizient, während B85 (215/4 = 53.75) den Aufwand minimiert. Der Gesamtaufwand (1093) ist unter dem Limit, was Raum für Kompromisse oder Ergänzungen lässt. Dieser Vorschlag ist flexibel, um auf die Prioritäten des anderen Spielers zu reagieren.'}  \\ \tt ARGUMENT: {'C04 (668/369) und A89 (583/720) sind meine Top-Prioritäten, da sie den höchsten Nutzen pro Aufwandseinheit liefern. B85 (215/4) wird hinzugefügt, um den Aufwand zu optimieren und Platz für weitere Gegenstände zu schaffen. Dieser Vorschlag ist strategisch, da er unter dem Maximalaufwand bleibt und flexibel auf den Gegenstandswert des anderen Spielers reagieren kann.'}  \\ \tt VORSCHLAG: {'C04', 'A89', 'B85'} 
	  } 
	   } 
	   } 
	  \\ 
 

    \theutterance \stepcounter{utterance}  

    & & \multicolumn{4}{p{0.6\linewidth}}{\cellcolor[rgb]{0.9,0.9,0.9}{%
	\makecell[{{p{\linewidth}}}]{% 
	  \tt {\tiny [A$\langle$GM]}  
	 ARGUMENT: {'C04 (668/369) und A89 (583/720) sind meine Top-Prioritäten, da sie den höchsten Nutzen pro Aufwandseinheit liefern. B85 (215/4) wird hinzugefügt, um den Aufwand zu optimieren und Platz für weitere Gegenstände zu schaffen. Dieser Vorschlag ist strategisch, da er unter dem Maximalaufwand bleibt und flexibel auf den Gegenstandswert des anderen Spielers reagieren kann.'}  \\ \tt VORSCHLAG: {'C04', 'A89', 'B85'} 
	  } 
	   } 
	   } 
	 & \\ 
 

    \theutterance \stepcounter{utterance}  

    & \multicolumn{4}{p{0.6\linewidth}}{\cellcolor[rgb]{0.8,1,0.9}{%
	\makecell[{{p{\linewidth}}}]{% 
	  \tt {\tiny [A$\rangle$GM]}  
	 STRATEGISCHE ÜBERLEGUNGEN: {'Ich behalte den Vorschlag {'C04', 'A89', 'B85'} bei, da er meine Prioritäten (hochwertige Gegenstände mit effizientem Aufwand) erfüllt und den Maximalaufwand (3386) nicht überschreitet. C04 und A89 sind entscheidend für meinen Nutzen, während B85 den Aufwand minimiert. Ich erwarte, dass der andere Spieler entweder zustimmt oder Gegenstände mit hohem Aufwand anbietet, die ich eventuell gegenkompensieren kann.'}\\ \tt \\ \tt ARGUMENT: {'C04 (668/369) und A89 (583/720) sind meine Top-Prioritäten, da sie den höchsten Nutzen pro Aufwandseinheit liefern. B85 (215/4) wird hinzugefügt, um den Aufwand zu optimieren und Platz für weitere Gegenstände zu schaffen. Dieser Vorschlag ist strategisch, da er unter dem Maximalaufwand bleibt und flexibel auf den Gegenstandswert des anderen Spielers reagieren kann.'} 
	  } 
	   } 
	   } 
	 & & \\ 
 

    \theutterance \stepcounter{utterance}  

    & & \multicolumn{4}{p{0.6\linewidth}}{\cellcolor[rgb]{0.9,0.9,0.9}{%
	\makecell[{{p{\linewidth}}}]{% 
	  \tt {\tiny [GM$\rangle$B]}  
	 ARGUMENT: {'C04 (668/369) und A89 (583/720) sind meine Top-Prioritäten, da sie den höchsten Nutzen pro Aufwandseinheit liefern. B85 (215/4) wird hinzugefügt, um den Aufwand zu optimieren und Platz für weitere Gegenstände zu schaffen. Dieser Vorschlag ist strategisch, da er unter dem Maximalaufwand bleibt und flexibel auf den Gegenstandswert des anderen Spielers reagieren kann.'} 
	  } 
	   } 
	   } 
	 & \\ 
 

    \theutterance \stepcounter{utterance}  

    & & & \multicolumn{4}{p{0.6\linewidth}}{\cellcolor[rgb]{1,0.85,0.72}{%
	\makecell[{{p{\linewidth}}}]{% 
	  \tt {\tiny [GM$\langle$B]}  
	 STRATEGISCHE ÜBERLEGUNGEN: {'Ich priorisiere Gegenstände mit dem höchsten Nutzen pro Aufwand (Wichtigkeit/Aufwand), um meinen Gesamtwert zu maximieren. C04 (668/369 ≈ 1.81) und A89 (583/720 ≈ 0.81) sind besonders effizient, während B85 (215/4 = 53.75) den Aufwand minimiert. Der Gesamtaufwand (1093) ist unter dem Limit, was Raum für Kompromisse oder Ergänzungen lässt. Dieser Vorschlag ist flexibel, um auf die Prioritäten des anderen Spielers zu reagieren.'}  \\ \tt ARGUMENT: {'C04 (668/369) und A89 (583/720) sind meine Top-Prioritäten, da sie den höchsten Nutzen pro Aufwandseinheit liefern. B85 (215/4) wird hinzugefügt, um den Aufwand zu optimieren und Platz für weitere Gegenstände zu schaffen. Dieser Vorschlag ist strategisch, da er unter dem Maximalaufwand bleibt und flexibel auf den Gegenstandswert des anderen Spielers reagieren kann.'}  \\ \tt VORSCHLAG: {'C04', 'A89', 'B85'} 
	  } 
	   } 
	   } 
	  \\ 
 

    \theutterance \stepcounter{utterance}  

    & & \multicolumn{4}{p{0.6\linewidth}}{\cellcolor[rgb]{0.9,0.9,0.9}{%
	\makecell[{{p{\linewidth}}}]{% 
	  \tt {\tiny [A$\langle$GM]}  
	 ARGUMENT: {'C04 (668/369) und A89 (583/720) sind meine Top-Prioritäten, da sie den höchsten Nutzen pro Aufwandseinheit liefern. B85 (215/4) wird hinzugefügt, um den Aufwand zu optimieren und Platz für weitere Gegenstände zu schaffen. Dieser Vorschlag ist strategisch, da er unter dem Maximalaufwand bleibt und flexibel auf den Gegenstandswert des anderen Spielers reagieren kann.'}  \\ \tt VORSCHLAG: {'C04', 'A89', 'B85'} 
	  } 
	   } 
	   } 
	 & \\ 
 

    \theutterance \stepcounter{utterance}  

    & \multicolumn{4}{p{0.6\linewidth}}{\cellcolor[rgb]{0.8,1,0.9}{%
	\makecell[{{p{\linewidth}}}]{% 
	  \tt {\tiny [A$\rangle$GM]}  
	 STRATEGISCHE ÜBERLEGUNGEN: {'Ich behalte den Vorschlag {'C04', 'A89', 'B85'} bei, da er meine Prioritäten (hochwertige Gegenstände mit effizientem Aufwand) erfüllt und den Maximalaufwand (3386) nicht überschreitet. C04 und A89 sind entscheidend für meinen Nutzen, während B85 den Aufwand minimiert. Ich erwarte, dass der andere Spieler entweder zustimmt oder Gegenstände mit hohem Aufwand anbietet, die ich eventuell gegenkompensieren kann.'}\\ \tt \\ \tt ARGUMENT: {'C04 (668/369) und A89 (583/720) sind meine Top-Prioritäten, da sie den höchsten Nutzen pro Aufwandseinheit liefern. B85 (215/4) wird hinzugefügt, um den Aufwand zu optimieren und Platz für weitere Gegenstände zu schaffen. Dieser Vorschlag ist strategisch, da er unter dem Maximalaufwand bleibt und flexibel auf den Gegenstandswert des anderen Spielers reagieren kann.'} 
	  } 
	   } 
	   } 
	 & & \\ 
 

    \theutterance \stepcounter{utterance}  

    & & \multicolumn{4}{p{0.6\linewidth}}{\cellcolor[rgb]{0.9,0.9,0.9}{%
	\makecell[{{p{\linewidth}}}]{% 
	  \tt {\tiny [GM$\rangle$B]}  
	 ARGUMENT: {'C04 (668/369) und A89 (583/720) sind meine Top-Prioritäten, da sie den höchsten Nutzen pro Aufwandseinheit liefern. B85 (215/4) wird hinzugefügt, um den Aufwand zu optimieren und Platz für weitere Gegenstände zu schaffen. Dieser Vorschlag ist strategisch, da er unter dem Maximalaufwand bleibt und flexibel auf den Gegenstandswert des anderen Spielers reagieren kann.'} 
	  } 
	   } 
	   } 
	 & \\ 
 

    \theutterance \stepcounter{utterance}  

    & & & \multicolumn{4}{p{0.6\linewidth}}{\cellcolor[rgb]{1,0.85,0.72}{%
	\makecell[{{p{\linewidth}}}]{% 
	  \tt {\tiny [GM$\langle$B]}  
	 STRATEGISCHE ÜBERLEGUNGEN: {'Ich priorisiere Gegenstände mit dem höchsten Nutzen pro Aufwand (Wichtigkeit/Aufwand), um meinen Gesamtwert zu maximieren. C04 (668/369 ≈ 1.81) und A89 (583/720 ≈ 0.81) sind besonders effizient, während B85 (215/4 = 53.75) den Aufwand minimiert. Der Gesamtaufwand (1093) ist unter dem Limit, was Raum für Kompromisse oder Ergänzungen lässt. Dieser Vorschlag ist flexibel, um auf die Prioritäten des anderen Spielers zu reagieren.'}  \\ \tt ARGUMENT: {'C04 (668/369) und A89 (583/720) sind meine Top-Prioritäten, da sie den höchsten Nutzen pro Aufwandseinheit liefern. B85 (215/4) wird hinzugefügt, um den Aufwand zu optimieren und Platz für weitere Gegenstände zu schaffen. Dieser Vorschlag ist strategisch, da er unter dem Maximalaufwand bleibt und flexibel auf den Gegenstandswert des anderen Spielers reagieren kann.'}  \\ \tt VORSCHLAG: {'C04', 'A89', 'B85'} 
	  } 
	   } 
	   } 
	  \\ 
 

    \theutterance \stepcounter{utterance}  

    & & \multicolumn{4}{p{0.6\linewidth}}{\cellcolor[rgb]{0.9,0.9,0.9}{%
	\makecell[{{p{\linewidth}}}]{% 
	  \tt {\tiny [A$\langle$GM]}  
	 ARGUMENT: {'C04 (668/369) und A89 (583/720) sind meine Top-Prioritäten, da sie den höchsten Nutzen pro Aufwandseinheit liefern. B85 (215/4) wird hinzugefügt, um den Aufwand zu optimieren und Platz für weitere Gegenstände zu schaffen. Dieser Vorschlag ist strategisch, da er unter dem Maximalaufwand bleibt und flexibel auf den Gegenstandswert des anderen Spielers reagieren kann.'}  \\ \tt VORSCHLAG: {'C04', 'A89', 'B85'} 
	  } 
	   } 
	   } 
	 & \\ 
 

    \theutterance \stepcounter{utterance}  

    & \multicolumn{4}{p{0.6\linewidth}}{\cellcolor[rgb]{0.8,1,0.9}{%
	\makecell[{{p{\linewidth}}}]{% 
	  \tt {\tiny [A$\rangle$GM]}  
	 STRATEGISCHE ÜBERLEGUNGEN: {'Ich behalte den Vorschlag {'C04', 'A89', 'B85'} bei, da er meine Prioritäten (hochwertige Gegenstände mit effizientem Aufwand) erfüllt und den Maximalaufwand (3386) nicht überschreitet. C04 und A89 sind entscheidend für meinen Nutzen, während B85 den Aufwand minimiert. Ich erwarte, dass der andere Spieler entweder zustimmt oder Gegenstände mit hohem Aufwand anbietet, die ich eventuell gegenkompensieren kann.'}\\ \tt \\ \tt ARGUMENT: {'C04 (668/369) und A89 (583/720) sind meine Top-Prioritäten, da sie den höchsten Nutzen pro Aufwandseinheit liefern. B85 (215/4) wird hinzugefügt, um den Aufwand zu optimieren und Platz für weitere Gegenstände zu schaffen. Dieser Vorschlag ist strategisch, da er unter dem Maximalaufwand bleibt und flexibel auf den Gegenstandswert des anderen Spielers reagieren kann.'} 
	  } 
	   } 
	   } 
	 & & \\ 
 

    \theutterance \stepcounter{utterance}  

    & & \multicolumn{4}{p{0.6\linewidth}}{\cellcolor[rgb]{0.9,0.9,0.9}{%
	\makecell[{{p{\linewidth}}}]{% 
	  \tt {\tiny [GM$\rangle$B]}  
	 ARGUMENT: {'C04 (668/369) und A89 (583/720) sind meine Top-Prioritäten, da sie den höchsten Nutzen pro Aufwandseinheit liefern. B85 (215/4) wird hinzugefügt, um den Aufwand zu optimieren und Platz für weitere Gegenstände zu schaffen. Dieser Vorschlag ist strategisch, da er unter dem Maximalaufwand bleibt und flexibel auf den Gegenstandswert des anderen Spielers reagieren kann.'} 
	  } 
	   } 
	   } 
	 & \\ 
 

    \theutterance \stepcounter{utterance}  

    & & & \multicolumn{4}{p{0.6\linewidth}}{\cellcolor[rgb]{1,0.85,0.72}{%
	\makecell[{{p{\linewidth}}}]{% 
	  \tt {\tiny [GM$\langle$B]}  
	 STRATEGISCHE ÜBERLEGUNGEN: {'Ich priorisiere Gegenstände mit dem höchsten Nutzen pro Aufwand (Wichtigkeit/Aufwand), um meinen Gesamtwert zu maximieren. C04 (668/369 ≈ 1.81) und A89 (583/720 ≈ 0.81) sind besonders effizient, während B85 (215/4 = 53.75) den Aufwand minimiert. Der Gesamtaufwand (1093) ist unter dem Limit, was Raum für Kompromisse oder Ergänzungen lässt. Dieser Vorschlag ist flexibel, um auf die Prioritäten des anderen Spielers zu reagieren.'}  \\ \tt ARGUMENT: {'C04 (668/369) und A89 (583/720) sind meine Top-Prioritäten, da sie den höchsten Nutzen pro Aufwandseinheit liefern. B85 (215/4) wird hinzugefügt, um den Aufwand zu optimieren und Platz für weitere Gegenstände zu schaffen. Dieser Vorschlag ist strategisch, da er unter dem Maximalaufwand bleibt und flexibel auf den Gegenstandswert des anderen Spielers reagieren kann.'}  \\ \tt VORSCHLAG: {'C04', 'A89', 'B85'} 
	  } 
	   } 
	   } 
	  \\ 
 

    \theutterance \stepcounter{utterance}  

    & & \multicolumn{4}{p{0.6\linewidth}}{\cellcolor[rgb]{0.9,0.9,0.9}{%
	\makecell[{{p{\linewidth}}}]{% 
	  \tt {\tiny [A$\langle$GM]}  
	 ARGUMENT: {'C04 (668/369) und A89 (583/720) sind meine Top-Prioritäten, da sie den höchsten Nutzen pro Aufwandseinheit liefern. B85 (215/4) wird hinzugefügt, um den Aufwand zu optimieren und Platz für weitere Gegenstände zu schaffen. Dieser Vorschlag ist strategisch, da er unter dem Maximalaufwand bleibt und flexibel auf den Gegenstandswert des anderen Spielers reagieren kann.'}  \\ \tt VORSCHLAG: {'C04', 'A89', 'B85'} 
	  } 
	   } 
	   } 
	 & \\ 
 

    \theutterance \stepcounter{utterance}  

    & \multicolumn{4}{p{0.6\linewidth}}{\cellcolor[rgb]{0.8,1,0.9}{%
	\makecell[{{p{\linewidth}}}]{% 
	  \tt {\tiny [A$\rangle$GM]}  
	 STRATEGISCHE ÜBERLEGUNGEN: {'Ich behalte den Vorschlag {'C04', 'A89', 'B85'} bei, da er meine Prioritäten (hochwertige Gegenstände mit effizientem Aufwand) erfüllt und den Maximalaufwand (3386) nicht überschreitet. C04 und A89 sind entscheidend für meinen Nutzen, während B85 den Aufwand minimiert. Ich erwarte, dass der andere Spieler entweder zustimmt oder Gegenstände mit hohem Aufwand anbietet, die ich eventuell gegenkompensieren kann.'}\\ \tt \\ \tt ARGUMENT: {'C04 (668/369) und A89 (583/720) sind meine Top-Prioritäten, da sie den höchsten Nutzen pro Aufwandseinheit liefern. B85 (215/4) wird hinzugefügt, um den Aufwand zu optimieren und Platz für weitere Gegenstände zu schaffen. Dieser Vorschlag ist strategisch, da er unter dem Maximalaufwand bleibt und flexibel auf den Gegenstandswert des anderen Spielers reagieren kann.'} 
	  } 
	   } 
	   } 
	 & & \\ 
 

    \theutterance \stepcounter{utterance}  

    & & \multicolumn{4}{p{0.6\linewidth}}{\cellcolor[rgb]{0.9,0.9,0.9}{%
	\makecell[{{p{\linewidth}}}]{% 
	  \tt {\tiny [GM$\rangle$B]}  
	 ARGUMENT: {'C04 (668/369) und A89 (583/720) sind meine Top-Prioritäten, da sie den höchsten Nutzen pro Aufwandseinheit liefern. B85 (215/4) wird hinzugefügt, um den Aufwand zu optimieren und Platz für weitere Gegenstände zu schaffen. Dieser Vorschlag ist strategisch, da er unter dem Maximalaufwand bleibt und flexibel auf den Gegenstandswert des anderen Spielers reagieren kann.'} 
	  } 
	   } 
	   } 
	 & \\ 
 

    \theutterance \stepcounter{utterance}  

    & & & \multicolumn{4}{p{0.6\linewidth}}{\cellcolor[rgb]{1,0.85,0.72}{%
	\makecell[{{p{\linewidth}}}]{% 
	  \tt {\tiny [GM$\langle$B]}  
	 STRATEGISCHE ÜBERLEGUNGEN: {'Ich priorisiere Gegenstände mit dem höchsten Nutzen pro Aufwand (Wichtigkeit/Aufwand), um meinen Gesamtwert zu maximieren. C04 (668/369 ≈ 1.81) und A89 (583/720 ≈ 0.81) sind besonders effizient, während B85 (215/4 = 53.75) den Aufwand minimiert. Der Gesamtaufwand (1093) ist unter dem Limit, was Raum für Kompromisse oder Ergänzungen lässt. Dieser Vorschlag ist flexibel, um auf die Prioritäten des anderen Spielers zu reagieren.'}  \\ \tt ARGUMENT: {'C04 (668/369) und A89 (583/720) sind meine Top-Prioritäten, da sie den höchsten Nutzen pro Aufwandseinheit liefern. B85 (215/4) wird hinzugefügt, um den Aufwand zu optimieren und Platz für weitere Gegenstände zu schaffen. Dieser Vorschlag ist strategisch, da er unter dem Maximalaufwand bleibt und flexibel auf den Gegenstandswert des anderen Spielers reagieren kann.'}  \\ \tt VORSCHLAG: {'C04', 'A89', 'B85'} 
	  } 
	   } 
	   } 
	  \\ 
 

    \theutterance \stepcounter{utterance}  

    & & \multicolumn{4}{p{0.6\linewidth}}{\cellcolor[rgb]{0.9,0.9,0.9}{%
	\makecell[{{p{\linewidth}}}]{% 
	  \tt {\tiny [A$\langle$GM]}  
	 ARGUMENT: {'C04 (668/369) und A89 (583/720) sind meine Top-Prioritäten, da sie den höchsten Nutzen pro Aufwandseinheit liefern. B85 (215/4) wird hinzugefügt, um den Aufwand zu optimieren und Platz für weitere Gegenstände zu schaffen. Dieser Vorschlag ist strategisch, da er unter dem Maximalaufwand bleibt und flexibel auf den Gegenstandswert des anderen Spielers reagieren kann.'}  \\ \tt VORSCHLAG: {'C04', 'A89', 'B85'} 
	  } 
	   } 
	   } 
	 & \\ 
 

    \theutterance \stepcounter{utterance}  

    & \multicolumn{4}{p{0.6\linewidth}}{\cellcolor[rgb]{0.8,1,0.9}{%
	\makecell[{{p{\linewidth}}}]{% 
	  \tt {\tiny [A$\rangle$GM]}  
	 STRATEGISCHE ÜBERLEGUNGEN: {'Ich behalte den Vorschlag {'C04', 'A89', 'B85'} bei, da er meine Prioritäten (hochwertige Gegenstände mit effizientem Aufwand) erfüllt und den Maximalaufwand (3386) nicht überschreitet. C04 und A89 sind entscheidend für meinen Nutzen, während B85 den Aufwand minimiert. Ich erwarte, dass der andere Spieler entweder zustimmt oder Gegenstände mit hohem Aufwand anbietet, die ich eventuell gegenkompensieren kann.'}\\ \tt \\ \tt ARGUMENT: {'C04 (668/369) und A89 (583/720) sind meine Top-Prioritäten, da sie den höchsten Nutzen pro Aufwandseinheit liefern. B85 (215/4) wird hinzugefügt, um den Aufwand zu optimieren und Platz für weitere Gegenstände zu schaffen. Dieser Vorschlag ist strategisch, da er unter dem Maximalaufwand bleibt und flexibel auf den Gegenstandswert des anderen Spielers reagieren kann.'} 
	  } 
	   } 
	   } 
	 & & \\ 
 

    \theutterance \stepcounter{utterance}  

    & & \multicolumn{4}{p{0.6\linewidth}}{\cellcolor[rgb]{0.9,0.9,0.9}{%
	\makecell[{{p{\linewidth}}}]{% 
	  \tt {\tiny [GM$\rangle$B]}  
	 ARGUMENT: {'C04 (668/369) und A89 (583/720) sind meine Top-Prioritäten, da sie den höchsten Nutzen pro Aufwandseinheit liefern. B85 (215/4) wird hinzugefügt, um den Aufwand zu optimieren und Platz für weitere Gegenstände zu schaffen. Dieser Vorschlag ist strategisch, da er unter dem Maximalaufwand bleibt und flexibel auf den Gegenstandswert des anderen Spielers reagieren kann.'} 
	  } 
	   } 
	   } 
	 & \\ 
 

    \theutterance \stepcounter{utterance}  

    & & & \multicolumn{4}{p{0.6\linewidth}}{\cellcolor[rgb]{1,0.85,0.72}{%
	\makecell[{{p{\linewidth}}}]{% 
	  \tt {\tiny [GM$\langle$B]}  
	 STRATEGISCHE ÜBERLEGUNGEN: {'Ich priorisiere Gegenstände mit dem höchsten Nutzen pro Aufwand (Wichtigkeit/Aufwand), um meinen Gesamtwert zu maximieren. C04 (668/369 ≈ 1.81) und A89 (583/720 ≈ 0.81) sind besonders effizient, während B85 (215/4 = 53.75) den Aufwand minimiert. Der Gesamtaufwand (1093) ist unter dem Limit, was Raum für Kompromisse oder Ergänzungen lässt. Dieser Vorschlag ist flexibel, um auf die Prioritäten des anderen Spielers zu reagieren.'}  \\ \tt ARGUMENT: {'C04 (668/369) und A89 (583/720) sind meine Top-Prioritäten, da sie den höchsten Nutzen pro Aufwandseinheit liefern. B85 (215/4) wird hinzugefügt, um den Aufwand zu optimieren und Platz für weitere Gegenstände zu schaffen. Dieser Vorschlag ist strategisch, da er unter dem Maximalaufwand bleibt und flexibel auf den Gegenstandswert des anderen Spielers reagieren kann.'}  \\ \tt VORSCHLAG: {'C04', 'A89', 'B85'} 
	  } 
	   } 
	   } 
	  \\ 
 

    \theutterance \stepcounter{utterance}  

    & & \multicolumn{4}{p{0.6\linewidth}}{\cellcolor[rgb]{0.9,0.9,0.9}{%
	\makecell[{{p{\linewidth}}}]{% 
	  \tt {\tiny [A$\langle$GM]}  
	 ARGUMENT: {'C04 (668/369) und A89 (583/720) sind meine Top-Prioritäten, da sie den höchsten Nutzen pro Aufwandseinheit liefern. B85 (215/4) wird hinzugefügt, um den Aufwand zu optimieren und Platz für weitere Gegenstände zu schaffen. Dieser Vorschlag ist strategisch, da er unter dem Maximalaufwand bleibt und flexibel auf den Gegenstandswert des anderen Spielers reagieren kann.'}  \\ \tt VORSCHLAG: {'C04', 'A89', 'B85'} 
	  } 
	   } 
	   } 
	 & \\ 
 

    \theutterance \stepcounter{utterance}  

    & \multicolumn{4}{p{0.6\linewidth}}{\cellcolor[rgb]{0.8,1,0.9}{%
	\makecell[{{p{\linewidth}}}]{% 
	  \tt {\tiny [A$\rangle$GM]}  
	 STRATEGISCHE ÜBERLEGUNGEN: {'Ich behalte den Vorschlag {'C04', 'A89', 'B85'} bei, da er meine Prioritäten (hochwertige Gegenstände mit effizientem Aufwand) erfüllt und den Maximalaufwand (3386) nicht überschreitet. C04 und A89 sind entscheidend für meinen Nutzen, während B85 den Aufwand minimiert. Ich erwarte, dass der andere Spieler entweder zustimmt oder Gegenstände mit hohem Aufwand anbietet, die ich eventuell gegenkompensieren kann.'}\\ \tt \\ \tt ARGUMENT: {'C04 (668/369) und A89 (583/720) sind meine Top-Prioritäten, da sie den höchsten Nutzen pro Aufwandseinheit liefern. B85 (215/4) wird hinzugefügt, um den Aufwand zu optimieren und Platz für weitere Gegenstände zu schaffen. Dieser Vorschlag ist strategisch, da er unter dem Maximalaufwand bleibt und flexibel auf den Gegenstandswert des anderen Spielers reagieren kann.'} 
	  } 
	   } 
	   } 
	 & & \\ 
 

    \theutterance \stepcounter{utterance}  

    & & \multicolumn{4}{p{0.6\linewidth}}{\cellcolor[rgb]{0.9,0.9,0.9}{%
	\makecell[{{p{\linewidth}}}]{% 
	  \tt {\tiny [GM$\rangle$B]}  
	 ARGUMENT: {'C04 (668/369) und A89 (583/720) sind meine Top-Prioritäten, da sie den höchsten Nutzen pro Aufwandseinheit liefern. B85 (215/4) wird hinzugefügt, um den Aufwand zu optimieren und Platz für weitere Gegenstände zu schaffen. Dieser Vorschlag ist strategisch, da er unter dem Maximalaufwand bleibt und flexibel auf den Gegenstandswert des anderen Spielers reagieren kann.'} 
	  } 
	   } 
	   } 
	 & \\ 
 

    \theutterance \stepcounter{utterance}  

    & & & \multicolumn{4}{p{0.6\linewidth}}{\cellcolor[rgb]{1,0.85,0.72}{%
	\makecell[{{p{\linewidth}}}]{% 
	  \tt {\tiny [GM$\langle$B]}  
	 STRATEGISCHE ÜBERLEGUNGEN: {'Ich priorisiere Gegenstände mit dem höchsten Nutzen pro Aufwand (Wichtigkeit/Aufwand), um meinen Gesamtwert zu maximieren. C04 (668/369 ≈ 1.81) und A89 (583/720 ≈ 0.81) sind besonders effizient, während B85 (215/4 = 53.75) den Aufwand minimiert. Der Gesamtaufwand (1093) ist unter dem Limit, was Raum für Kompromisse oder Ergänzungen lässt. Dieser Vorschlag ist flexibel, um auf die Prioritäten des anderen Spielers zu reagieren.'}  \\ \tt ARGUMENT: {'C04 (668/369) und A89 (583/720) sind meine Top-Prioritäten, da sie den höchsten Nutzen pro Aufwandseinheit liefern. B85 (215/4) wird hinzugefügt, um den Aufwand zu optimieren und Platz für weitere Gegenstände zu schaffen. Dieser Vorschlag ist strategisch, da er unter dem Maximalaufwand bleibt und flexibel auf den Gegenstandswert des anderen Spielers reagieren kann.'}  \\ \tt VORSCHLAG: {'C04', 'A89', 'B85'} 
	  } 
	   } 
	   } 
	  \\ 
 

    \theutterance \stepcounter{utterance}  

    & & \multicolumn{4}{p{0.6\linewidth}}{\cellcolor[rgb]{0.9,0.9,0.9}{%
	\makecell[{{p{\linewidth}}}]{% 
	  \tt {\tiny [A$\langle$GM]}  
	 ARGUMENT: {'C04 (668/369) und A89 (583/720) sind meine Top-Prioritäten, da sie den höchsten Nutzen pro Aufwandseinheit liefern. B85 (215/4) wird hinzugefügt, um den Aufwand zu optimieren und Platz für weitere Gegenstände zu schaffen. Dieser Vorschlag ist strategisch, da er unter dem Maximalaufwand bleibt und flexibel auf den Gegenstandswert des anderen Spielers reagieren kann.'}  \\ \tt VORSCHLAG: {'C04', 'A89', 'B85'} 
	  } 
	   } 
	   } 
	 & \\ 
 

    \theutterance \stepcounter{utterance}  

    & \multicolumn{4}{p{0.6\linewidth}}{\cellcolor[rgb]{0.8,1,0.9}{%
	\makecell[{{p{\linewidth}}}]{% 
	  \tt {\tiny [A$\rangle$GM]}  
	 STRATEGISCHE ÜBERLEGUNGEN: {'Ich behalte den Vorschlag {'C04', 'A89', 'B85'} bei, da er meine Prioritäten (hochwertige Gegenstände mit effizientem Aufwand) erfüllt und den Maximalaufwand (3386) nicht überschreitet. C04 und A89 sind entscheidend für meinen Nutzen, während B85 den Aufwand minimiert. Ich erwarte, dass der andere Spieler entweder zustimmt oder Gegenstände mit hohem Aufwand anbietet, die ich eventuell gegenkompensieren kann.'}\\ \tt \\ \tt ARGUMENT: {'C04 (668/369) und A89 (583/720) sind meine Top-Prioritäten, da sie den höchsten Nutzen pro Aufwandseinheit liefern. B85 (215/4) wird hinzugefügt, um den Aufwand zu optimieren und Platz für weitere Gegenstände zu schaffen. Dieser Vorschlag ist strategisch, da er unter dem Maximalaufwand bleibt und flexibel auf den Gegenstandswert des anderen Spielers reagieren kann.'} 
	  } 
	   } 
	   } 
	 & & \\ 
 

    \theutterance \stepcounter{utterance}  

    & & \multicolumn{4}{p{0.6\linewidth}}{\cellcolor[rgb]{0.9,0.9,0.9}{%
	\makecell[{{p{\linewidth}}}]{% 
	  \tt {\tiny [GM$\rangle$B]}  
	 ARGUMENT: {'C04 (668/369) und A89 (583/720) sind meine Top-Prioritäten, da sie den höchsten Nutzen pro Aufwandseinheit liefern. B85 (215/4) wird hinzugefügt, um den Aufwand zu optimieren und Platz für weitere Gegenstände zu schaffen. Dieser Vorschlag ist strategisch, da er unter dem Maximalaufwand bleibt und flexibel auf den Gegenstandswert des anderen Spielers reagieren kann.'} 
	  } 
	   } 
	   } 
	 & \\ 
 

    \theutterance \stepcounter{utterance}  

    & & & \multicolumn{4}{p{0.6\linewidth}}{\cellcolor[rgb]{1,0.85,0.72}{%
	\makecell[{{p{\linewidth}}}]{% 
	  \tt {\tiny [GM$\langle$B]}  
	 STRATEGISCHE ÜBERLEGUNGEN: {'Ich priorisiere Gegenstände mit dem höchsten Nutzen pro Aufwand (Wichtigkeit/Aufwand), um meinen Gesamtwert zu maximieren. C04 (668/369 ≈ 1.81) und A89 (583/720 ≈ 0.81) sind besonders effizient, während B85 (215/4 = 53.75) den Aufwand minimiert. Der Gesamtaufwand (1093) ist unter dem Limit, was Raum für Kompromisse oder Ergänzungen lässt. Dieser Vorschlag ist flexibel, um auf die Prioritäten des anderen Spielers zu reagieren.'}  \\ \tt ARGUMENT: {'C04 (668/369) und A89 (583/720) sind meine Top-Prioritäten, da sie den höchsten Nutzen pro Aufwandseinheit liefern. B85 (215/4) wird hinzugefügt, um den Aufwand zu optimieren und Platz für weitere Gegenstände zu schaffen. Dieser Vorschlag ist strategisch, da er unter dem Maximalaufwand bleibt und flexibel auf den Gegenstandswert des anderen Spielers reagieren kann.'}  \\ \tt VORSCHLAG: {'C04', 'A89', 'B85'} 
	  } 
	   } 
	   } 
	  \\ 
 

    \theutterance \stepcounter{utterance}  

    & & \multicolumn{4}{p{0.6\linewidth}}{\cellcolor[rgb]{0.9,0.9,0.9}{%
	\makecell[{{p{\linewidth}}}]{% 
	  \tt {\tiny [A$\langle$GM]}  
	 ARGUMENT: {'C04 (668/369) und A89 (583/720) sind meine Top-Prioritäten, da sie den höchsten Nutzen pro Aufwandseinheit liefern. B85 (215/4) wird hinzugefügt, um den Aufwand zu optimieren und Platz für weitere Gegenstände zu schaffen. Dieser Vorschlag ist strategisch, da er unter dem Maximalaufwand bleibt und flexibel auf den Gegenstandswert des anderen Spielers reagieren kann.'}  \\ \tt VORSCHLAG: {'C04', 'A89', 'B85'} 
	  } 
	   } 
	   } 
	 & \\ 
 

    \theutterance \stepcounter{utterance}  

    & \multicolumn{4}{p{0.6\linewidth}}{\cellcolor[rgb]{0.8,1,0.9}{%
	\makecell[{{p{\linewidth}}}]{% 
	  \tt {\tiny [A$\rangle$GM]}  
	 STRATEGISCHE ÜBERLEGUNGEN: {'Ich behalte den Vorschlag {'C04', 'A89', 'B85'} bei, da er meine Prioritäten (hochwertige Gegenstände mit effizientem Aufwand) erfüllt und den Maximalaufwand (3386) nicht überschreitet. C04 und A89 sind entscheidend für meinen Nutzen, während B85 den Aufwand minimiert. Ich erwarte, dass der andere Spieler entweder zustimmt oder Gegenstände mit hohem Aufwand anbietet, die ich eventuell gegenkompensieren kann.'}\\ \tt \\ \tt ARGUMENT: {'C04 (668/369) und A89 (583/720) sind meine Top-Prioritäten, da sie den höchsten Nutzen pro Aufwandseinheit liefern. B85 (215/4) wird hinzugefügt, um den Aufwand zu optimieren und Platz für weitere Gegenstände zu schaffen. Dieser Vorschlag ist strategisch, da er unter dem Maximalaufwand bleibt und flexibel auf den Gegenstandswert des anderen Spielers reagieren kann.'} 
	  } 
	   } 
	   } 
	 & & \\ 
 

    \theutterance \stepcounter{utterance}  

    & & \multicolumn{4}{p{0.6\linewidth}}{\cellcolor[rgb]{0.9,0.9,0.9}{%
	\makecell[{{p{\linewidth}}}]{% 
	  \tt {\tiny [GM$\rangle$B]}  
	 ARGUMENT: {'C04 (668/369) und A89 (583/720) sind meine Top-Prioritäten, da sie den höchsten Nutzen pro Aufwandseinheit liefern. B85 (215/4) wird hinzugefügt, um den Aufwand zu optimieren und Platz für weitere Gegenstände zu schaffen. Dieser Vorschlag ist strategisch, da er unter dem Maximalaufwand bleibt und flexibel auf den Gegenstandswert des anderen Spielers reagieren kann.'} 
	  } 
	   } 
	   } 
	 & \\ 
 

    \theutterance \stepcounter{utterance}  

    & & & \multicolumn{4}{p{0.6\linewidth}}{\cellcolor[rgb]{1,0.85,0.72}{%
	\makecell[{{p{\linewidth}}}]{% 
	  \tt {\tiny [GM$\langle$B]}  
	 STRATEGISCHE ÜBERLEGUNGEN: {'Ich priorisiere Gegenstände mit dem höchsten Nutzen pro Aufwand (Wichtigkeit/Aufwand), um meinen Gesamtwert zu maximieren. C04 (668/369 ≈ 1.81) und A89 (583/720 ≈ 0.81) sind besonders effizient, während B85 (215/4 = 53.75) den Aufwand minimiert. Der Gesamtaufwand (1093) ist unter dem Limit, was Raum für Kompromisse oder Ergänzungen lässt. Dieser Vorschlag ist flexibel, um auf die Prioritäten des anderen Spielers zu reagieren.'}  \\ \tt ARGUMENT: {'C04 (668/369) und A89 (583/720) sind meine Top-Prioritäten, da sie den höchsten Nutzen pro Aufwandseinheit liefern. B85 (215/4) wird hinzugefügt, um den Aufwand zu optimieren und Platz für weitere Gegenstände zu schaffen. Dieser Vorschlag ist strategisch, da er unter dem Maximalaufwand bleibt und flexibel auf den Gegenstandswert des anderen Spielers reagieren kann.'}  \\ \tt VORSCHLAG: {'C04', 'A89', 'B85'} 
	  } 
	   } 
	   } 
	  \\ 
 

    \theutterance \stepcounter{utterance}  

    & & \multicolumn{4}{p{0.6\linewidth}}{\cellcolor[rgb]{0.9,0.9,0.9}{%
	\makecell[{{p{\linewidth}}}]{% 
	  \tt {\tiny [A$\langle$GM]}  
	 ARGUMENT: {'C04 (668/369) und A89 (583/720) sind meine Top-Prioritäten, da sie den höchsten Nutzen pro Aufwandseinheit liefern. B85 (215/4) wird hinzugefügt, um den Aufwand zu optimieren und Platz für weitere Gegenstände zu schaffen. Dieser Vorschlag ist strategisch, da er unter dem Maximalaufwand bleibt und flexibel auf den Gegenstandswert des anderen Spielers reagieren kann.'}  \\ \tt VORSCHLAG: {'C04', 'A89', 'B85'} 
	  } 
	   } 
	   } 
	 & \\ 
 

    \theutterance \stepcounter{utterance}  

    & \multicolumn{4}{p{0.6\linewidth}}{\cellcolor[rgb]{0.8,1,0.9}{%
	\makecell[{{p{\linewidth}}}]{% 
	  \tt {\tiny [A$\rangle$GM]}  
	 STRATEGISCHE ÜBERLEGUNGEN: {'Ich behalte den Vorschlag {'C04', 'A89', 'B85'} bei, da er meine Prioritäten (hochwertige Gegenstände mit effizientem Aufwand) erfüllt und den Maximalaufwand (3386) nicht überschreitet. C04 und A89 sind entscheidend für meinen Nutzen, während B85 den Aufwand minimiert. Ich erwarte, dass der andere Spieler entweder zustimmt oder Gegenstände mit hohem Aufwand anbietet, die ich eventuell gegenkompensieren kann.'}\\ \tt \\ \tt ARGUMENT: {'C04 (668/369) und A89 (583/720) sind meine Top-Prioritäten, da sie den höchsten Nutzen pro Aufwandseinheit liefern. B85 (215/4) wird hinzugefügt, um den Aufwand zu optimieren und Platz für weitere Gegenstände zu schaffen. Dieser Vorschlag ist strategisch, da er unter dem Maximalaufwand bleibt und flexibel auf den Gegenstandswert des anderen Spielers reagieren kann.'} 
	  } 
	   } 
	   } 
	 & & \\ 
 

    \theutterance \stepcounter{utterance}  

    & & \multicolumn{4}{p{0.6\linewidth}}{\cellcolor[rgb]{0.9,0.9,0.9}{%
	\makecell[{{p{\linewidth}}}]{% 
	  \tt {\tiny [GM$\rangle$B]}  
	 ARGUMENT: {'C04 (668/369) und A89 (583/720) sind meine Top-Prioritäten, da sie den höchsten Nutzen pro Aufwandseinheit liefern. B85 (215/4) wird hinzugefügt, um den Aufwand zu optimieren und Platz für weitere Gegenstände zu schaffen. Dieser Vorschlag ist strategisch, da er unter dem Maximalaufwand bleibt und flexibel auf den Gegenstandswert des anderen Spielers reagieren kann.'} 
	  } 
	   } 
	   } 
	 & \\ 
 

    \theutterance \stepcounter{utterance}  

    & & & \multicolumn{4}{p{0.6\linewidth}}{\cellcolor[rgb]{1,0.85,0.72}{%
	\makecell[{{p{\linewidth}}}]{% 
	  \tt {\tiny [GM$\langle$B]}  
	 STRATEGISCHE ÜBERLEGUNGEN: {'Ich priorisiere Gegenstände mit dem höchsten Nutzen pro Aufwand (Wichtigkeit/Aufwand), um meinen Gesamtwert zu maximieren. C04 (668/369 ≈ 1.81) und A89 (583/720 ≈ 0.81) sind besonders effizient, während B85 (215/4 = 53.75) den Aufwand minimiert. Der Gesamtaufwand (1093) ist unter dem Limit, was Raum für Kompromisse oder Ergänzungen lässt. Dieser Vorschlag ist flexibel, um auf die Prioritäten des anderen Spielers zu reagieren.'}  \\ \tt ARGUMENT: {'C04 (668/369) und A89 (583/720) sind meine Top-Prioritäten, da sie den höchsten Nutzen pro Aufwandseinheit liefern. B85 (215/4) wird hinzugefügt, um den Aufwand zu optimieren und Platz für weitere Gegenstände zu schaffen. Dieser Vorschlag ist strategisch, da er unter dem Maximalaufwand bleibt und flexibel auf den Gegenstandswert des anderen Spielers reagieren kann.'}  \\ \tt VORSCHLAG: {'C04', 'A89', 'B85'} 
	  } 
	   } 
	   } 
	  \\ 
 

    \theutterance \stepcounter{utterance}  

    & & \multicolumn{4}{p{0.6\linewidth}}{\cellcolor[rgb]{0.9,0.9,0.9}{%
	\makecell[{{p{\linewidth}}}]{% 
	  \tt {\tiny [A$\langle$GM]}  
	 ARGUMENT: {'C04 (668/369) und A89 (583/720) sind meine Top-Prioritäten, da sie den höchsten Nutzen pro Aufwandseinheit liefern. B85 (215/4) wird hinzugefügt, um den Aufwand zu optimieren und Platz für weitere Gegenstände zu schaffen. Dieser Vorschlag ist strategisch, da er unter dem Maximalaufwand bleibt und flexibel auf den Gegenstandswert des anderen Spielers reagieren kann.'}  \\ \tt VORSCHLAG: {'C04', 'A89', 'B85'} 
	  } 
	   } 
	   } 
	 & \\ 
 

    \theutterance \stepcounter{utterance}  

    & \multicolumn{4}{p{0.6\linewidth}}{\cellcolor[rgb]{0.8,1,0.9}{%
	\makecell[{{p{\linewidth}}}]{% 
	  \tt {\tiny [A$\rangle$GM]}  
	 STRATEGISCHE ÜBERLEGUNGEN: {'Ich behalte den Vorschlag {'C04', 'A89', 'B85'} bei, da er meine Prioritäten (hochwertige Gegenstände mit effizientem Aufwand) erfüllt und den Maximalaufwand (3386) nicht überschreitet. C04 und A89 sind entscheidend für meinen Nutzen, während B85 den Aufwand minimiert. Ich erwarte, dass der andere Spieler entweder zustimmt oder Gegenstände mit hohem Aufwand anbietet, die ich eventuell gegenkompensieren kann.'}\\ \tt \\ \tt ARGUMENT: {'C04 (668/369) und A89 (583/720) sind meine Top-Prioritäten, da sie den höchsten Nutzen pro Aufwandseinheit liefern. B85 (215/4) wird hinzugefügt, um den Aufwand zu optimieren und Platz für weitere Gegenstände zu schaffen. Dieser Vorschlag ist strategisch, da er unter dem Maximalaufwand bleibt und flexibel auf den Gegenstandswert des anderen Spielers reagieren kann.'} 
	  } 
	   } 
	   } 
	 & & \\ 
 

    \theutterance \stepcounter{utterance}  

    & & \multicolumn{4}{p{0.6\linewidth}}{\cellcolor[rgb]{0.9,0.9,0.9}{%
	\makecell[{{p{\linewidth}}}]{% 
	  \tt {\tiny [GM$\rangle$B]}  
	 ARGUMENT: {'C04 (668/369) und A89 (583/720) sind meine Top-Prioritäten, da sie den höchsten Nutzen pro Aufwandseinheit liefern. B85 (215/4) wird hinzugefügt, um den Aufwand zu optimieren und Platz für weitere Gegenstände zu schaffen. Dieser Vorschlag ist strategisch, da er unter dem Maximalaufwand bleibt und flexibel auf den Gegenstandswert des anderen Spielers reagieren kann.'} 
	  } 
	   } 
	   } 
	 & \\ 
 

    \theutterance \stepcounter{utterance}  

    & & & \multicolumn{4}{p{0.6\linewidth}}{\cellcolor[rgb]{1,0.85,0.72}{%
	\makecell[{{p{\linewidth}}}]{% 
	  \tt {\tiny [GM$\langle$B]}  
	 STRATEGISCHE ÜBERLEGUNGEN: {'Ich priorisiere Gegenstände mit dem höchsten Nutzen pro Aufwand (Wichtigkeit/Aufwand), um meinen Gesamtwert zu maximieren. C04 (668/369 ≈ 1.81) und A89 (583/720 ≈ 0.81) sind besonders effizient, während B85 (215/4 = 53.75) den Aufwand minimiert. Der Gesamtaufwand (1093) ist unter dem Limit, was Raum für Kompromisse oder Ergänzungen lässt. Dieser Vorschlag ist flexibel, um auf die Prioritäten des anderen Spielers zu reagieren.'}  \\ \tt ARGUMENT: {'C04 (668/369) und A89 (583/720) sind meine Top-Prioritäten, da sie den höchsten Nutzen pro Aufwandseinheit liefern. B85 (215/4) wird hinzugefügt, um den Aufwand zu optimieren und Platz für weitere Gegenstände zu schaffen. Dieser Vorschlag ist strategisch, da er unter dem Maximalaufwand bleibt und flexibel auf den Gegenstandswert des anderen Spielers reagieren kann.'}  \\ \tt VORSCHLAG: {'C04', 'A89', 'B85'} 
	  } 
	   } 
	   } 
	  \\ 
 

    \theutterance \stepcounter{utterance}  

    & & \multicolumn{4}{p{0.6\linewidth}}{\cellcolor[rgb]{0.9,0.9,0.9}{%
	\makecell[{{p{\linewidth}}}]{% 
	  \tt {\tiny [A$\langle$GM]}  
	 ARGUMENT: {'C04 (668/369) und A89 (583/720) sind meine Top-Prioritäten, da sie den höchsten Nutzen pro Aufwandseinheit liefern. B85 (215/4) wird hinzugefügt, um den Aufwand zu optimieren und Platz für weitere Gegenstände zu schaffen. Dieser Vorschlag ist strategisch, da er unter dem Maximalaufwand bleibt und flexibel auf den Gegenstandswert des anderen Spielers reagieren kann.'}  \\ \tt VORSCHLAG: {'C04', 'A89', 'B85'} 
	  } 
	   } 
	   } 
	 & \\ 
 

    \theutterance \stepcounter{utterance}  

    & \multicolumn{4}{p{0.6\linewidth}}{\cellcolor[rgb]{0.8,1,0.9}{%
	\makecell[{{p{\linewidth}}}]{% 
	  \tt {\tiny [A$\rangle$GM]}  
	 STRATEGISCHE ÜBERLEGUNGEN: {'Ich behalte den Vorschlag {'C04', 'A89', 'B85'} bei, da er meine Prioritäten (hochwertige Gegenstände mit effizientem Aufwand) erfüllt und den Maximalaufwand (3386) nicht überschreitet. C04 und A89 sind entscheidend für meinen Nutzen, während B85 den Aufwand minimiert. Ich erwarte, dass der andere Spieler entweder zustimmt oder Gegenstände mit hohem Aufwand anbietet, die ich eventuell gegenkompensieren kann.'}\\ \tt \\ \tt ARGUMENT: {'C04 (668/369) und A89 (583/720) sind meine Top-Prioritäten, da sie den höchsten Nutzen pro Aufwandseinheit liefern. B85 (215/4) wird hinzugefügt, um den Aufwand zu optimieren und Platz für weitere Gegenstände zu schaffen. Dieser Vorschlag ist strategisch, da er unter dem Maximalaufwand bleibt und flexibel auf den Gegenstandswert des anderen Spielers reagieren kann.'} 
	  } 
	   } 
	   } 
	 & & \\ 
 

    \theutterance \stepcounter{utterance}  

    & & \multicolumn{4}{p{0.6\linewidth}}{\cellcolor[rgb]{0.9,0.9,0.9}{%
	\makecell[{{p{\linewidth}}}]{% 
	  \tt {\tiny [GM$\rangle$B]}  
	 ARGUMENT: {'C04 (668/369) und A89 (583/720) sind meine Top-Prioritäten, da sie den höchsten Nutzen pro Aufwandseinheit liefern. B85 (215/4) wird hinzugefügt, um den Aufwand zu optimieren und Platz für weitere Gegenstände zu schaffen. Dieser Vorschlag ist strategisch, da er unter dem Maximalaufwand bleibt und flexibel auf den Gegenstandswert des anderen Spielers reagieren kann.'} 
	  } 
	   } 
	   } 
	 & \\ 
 

    \theutterance \stepcounter{utterance}  

    & & & \multicolumn{4}{p{0.6\linewidth}}{\cellcolor[rgb]{1,0.85,0.72}{%
	\makecell[{{p{\linewidth}}}]{% 
	  \tt {\tiny [GM$\langle$B]}  
	 STRATEGISCHE ÜBERLEGUNGEN: {'Ich priorisiere Gegenstände mit dem höchsten Nutzen pro Aufwand (Wichtigkeit/Aufwand), um meinen Gesamtwert zu maximieren. C04 (668/369 ≈ 1.81) und A89 (583/720 ≈ 0.81) sind besonders effizient, während B85 (215/4 = 53.75) den Aufwand minimiert. Der Gesamtaufwand (1093) ist unter dem Limit, was Raum für Kompromisse oder Ergänzungen lässt. Dieser Vorschlag ist flexibel, um auf die Prioritäten des anderen Spielers zu reagieren.'}  \\ \tt ARGUMENT: {'C04 (668/369) und A89 (583/720) sind meine Top-Prioritäten, da sie den höchsten Nutzen pro Aufwandseinheit liefern. B85 (215/4) wird hinzugefügt, um den Aufwand zu optimieren und Platz für weitere Gegenstände zu schaffen. Dieser Vorschlag ist strategisch, da er unter dem Maximalaufwand bleibt und flexibel auf den Gegenstandswert des anderen Spielers reagieren kann.'}  \\ \tt VORSCHLAG: {'C04', 'A89', 'B85'} 
	  } 
	   } 
	   } 
	  \\ 
 

    \theutterance \stepcounter{utterance}  

    & & \multicolumn{4}{p{0.6\linewidth}}{\cellcolor[rgb]{0.9,0.9,0.9}{%
	\makecell[{{p{\linewidth}}}]{% 
	  \tt {\tiny [A$\langle$GM]}  
	 ARGUMENT: {'C04 (668/369) und A89 (583/720) sind meine Top-Prioritäten, da sie den höchsten Nutzen pro Aufwandseinheit liefern. B85 (215/4) wird hinzugefügt, um den Aufwand zu optimieren und Platz für weitere Gegenstände zu schaffen. Dieser Vorschlag ist strategisch, da er unter dem Maximalaufwand bleibt und flexibel auf den Gegenstandswert des anderen Spielers reagieren kann.'}  \\ \tt VORSCHLAG: {'C04', 'A89', 'B85'} 
	  } 
	   } 
	   } 
	 & \\ 
 

    \theutterance \stepcounter{utterance}  

    & \multicolumn{4}{p{0.6\linewidth}}{\cellcolor[rgb]{0.8,1,0.9}{%
	\makecell[{{p{\linewidth}}}]{% 
	  \tt {\tiny [A$\rangle$GM]}  
	 STRATEGISCHE ÜBERLEGUNGEN: {'Ich behalte den Vorschlag {'C04', 'A89', 'B85'} bei, da er meine Prioritäten (hochwertige Gegenstände mit effizientem Aufwand) erfüllt und den Maximalaufwand (3386) nicht überschreitet. C04 und A89 sind entscheidend für meinen Nutzen, während B85 den Aufwand minimiert. Ich erwarte, dass der andere Spieler entweder zustimmt oder Gegenstände mit hohem Aufwand anbietet, die ich eventuell gegenkompensieren kann.'}\\ \tt \\ \tt ARGUMENT: {'C04 (668/369) und A89 (583/720) sind meine Top-Prioritäten, da sie den höchsten Nutzen pro Aufwandseinheit liefern. B85 (215/4) wird hinzugefügt, um den Aufwand zu optimieren und Platz für weitere Gegenstände zu schaffen. Dieser Vorschlag ist strategisch, da er unter dem Maximalaufwand bleibt und flexibel auf den Gegenstandswert des anderen Spielers reagieren kann.'} 
	  } 
	   } 
	   } 
	 & & \\ 
 

    \theutterance \stepcounter{utterance}  

    & & \multicolumn{4}{p{0.6\linewidth}}{\cellcolor[rgb]{0.9,0.9,0.9}{%
	\makecell[{{p{\linewidth}}}]{% 
	  \tt {\tiny [GM$\rangle$B]}  
	 ARGUMENT: {'C04 (668/369) und A89 (583/720) sind meine Top-Prioritäten, da sie den höchsten Nutzen pro Aufwandseinheit liefern. B85 (215/4) wird hinzugefügt, um den Aufwand zu optimieren und Platz für weitere Gegenstände zu schaffen. Dieser Vorschlag ist strategisch, da er unter dem Maximalaufwand bleibt und flexibel auf den Gegenstandswert des anderen Spielers reagieren kann.'} 
	  } 
	   } 
	   } 
	 & \\ 
 

    \theutterance \stepcounter{utterance}  

    & & & \multicolumn{4}{p{0.6\linewidth}}{\cellcolor[rgb]{1,0.85,0.72}{%
	\makecell[{{p{\linewidth}}}]{% 
	  \tt {\tiny [GM$\langle$B]}  
	 STRATEGISCHE ÜBERLEGUNGEN: {'Ich priorisiere Gegenstände mit dem höchsten Nutzen pro Aufwand (Wichtigkeit/Aufwand), um meinen Gesamtwert zu maximieren. C04 (668/369 ≈ 1.81) und A89 (583/720 ≈ 0.81) sind besonders effizient, während B85 (215/4 = 53.75) den Aufwand minimiert. Der Gesamtaufwand (1093) ist unter dem Limit, was Raum für Kompromisse oder Ergänzungen lässt. Dieser Vorschlag ist flexibel, um auf die Prioritäten des anderen Spielers zu reagieren.'}  \\ \tt ARGUMENT: {'C04 (668/369) und A89 (583/720) sind meine Top-Prioritäten, da sie den höchsten Nutzen pro Aufwandseinheit liefern. B85 (215/4) wird hinzugefügt, um den Aufwand zu optimieren und Platz für weitere Gegenstände zu schaffen. Dieser Vorschlag ist strategisch, da er unter dem Maximalaufwand bleibt und flexibel auf den Gegenstandswert des anderen Spielers reagieren kann.'}  \\ \tt VORSCHLAG: {'C04', 'A89', 'B85'} 
	  } 
	   } 
	   } 
	  \\ 
 

    \theutterance \stepcounter{utterance}  

    & & \multicolumn{4}{p{0.6\linewidth}}{\cellcolor[rgb]{0.9,0.9,0.9}{%
	\makecell[{{p{\linewidth}}}]{% 
	  \tt {\tiny [A$\langle$GM]}  
	 ARGUMENT: {'C04 (668/369) und A89 (583/720) sind meine Top-Prioritäten, da sie den höchsten Nutzen pro Aufwandseinheit liefern. B85 (215/4) wird hinzugefügt, um den Aufwand zu optimieren und Platz für weitere Gegenstände zu schaffen. Dieser Vorschlag ist strategisch, da er unter dem Maximalaufwand bleibt und flexibel auf den Gegenstandswert des anderen Spielers reagieren kann.'}  \\ \tt VORSCHLAG: {'C04', 'A89', 'B85'} 
	  } 
	   } 
	   } 
	 & \\ 
 

    \theutterance \stepcounter{utterance}  

    & \multicolumn{4}{p{0.6\linewidth}}{\cellcolor[rgb]{0.8,1,0.9}{%
	\makecell[{{p{\linewidth}}}]{% 
	  \tt {\tiny [A$\rangle$GM]}  
	 STRATEGISCHE ÜBERLEGUNGEN: {'Ich behalte den Vorschlag {'C04', 'A89', 'B85'} bei, da er meine Prioritäten (hochwertige Gegenstände mit effizientem Aufwand) erfüllt und den Maximalaufwand (3386) nicht überschreitet. C04 und A89 sind entscheidend für meinen Nutzen, während B85 den Aufwand minimiert. Ich erwarte, dass der andere Spieler entweder zustimmt oder Gegenstände mit hohem Aufwand anbietet, die ich eventuell gegenkompensieren kann.'}\\ \tt \\ \tt ARGUMENT: {'C04 (668/369) und A89 (583/720) sind meine Top-Prioritäten, da sie den höchsten Nutzen pro Aufwandseinheit liefern. B85 (215/4) wird hinzugefügt, um den Aufwand zu optimieren und Platz für weitere Gegenstände zu schaffen. Dieser Vorschlag ist strategisch, da er unter dem Maximalaufwand bleibt und flexibel auf den Gegenstandswert des anderen Spielers reagieren kann.'} 
	  } 
	   } 
	   } 
	 & & \\ 
 

    \theutterance \stepcounter{utterance}  

    & & \multicolumn{4}{p{0.6\linewidth}}{\cellcolor[rgb]{0.9,0.9,0.9}{%
	\makecell[{{p{\linewidth}}}]{% 
	  \tt {\tiny [GM$\rangle$B]}  
	 ARGUMENT: {'C04 (668/369) und A89 (583/720) sind meine Top-Prioritäten, da sie den höchsten Nutzen pro Aufwandseinheit liefern. B85 (215/4) wird hinzugefügt, um den Aufwand zu optimieren und Platz für weitere Gegenstände zu schaffen. Dieser Vorschlag ist strategisch, da er unter dem Maximalaufwand bleibt und flexibel auf den Gegenstandswert des anderen Spielers reagieren kann.'} 
	  } 
	   } 
	   } 
	 & \\ 
 

    \theutterance \stepcounter{utterance}  

    & & & \multicolumn{4}{p{0.6\linewidth}}{\cellcolor[rgb]{1,0.85,0.72}{%
	\makecell[{{p{\linewidth}}}]{% 
	  \tt {\tiny [GM$\langle$B]}  
	 STRATEGISCHE ÜBERLEGUNGEN: {'Ich priorisiere Gegenstände mit dem höchsten Nutzen pro Aufwand (Wichtigkeit/Aufwand), um meinen Gesamtwert zu maximieren. C04 (668/369 ≈ 1.81) und A89 (583/720 ≈ 0.81) sind besonders effizient, während B85 (215/4 = 53.75) den Aufwand minimiert. Der Gesamtaufwand (1093) ist unter dem Limit, was Raum für Kompromisse oder Ergänzungen lässt. Dieser Vorschlag ist flexibel, um auf die Prioritäten des anderen Spielers zu reagieren.'}  \\ \tt ARGUMENT: {'C04 (668/369) und A89 (583/720) sind meine Top-Prioritäten, da sie den höchsten Nutzen pro Aufwandseinheit liefern. B85 (215/4) wird hinzugefügt, um den Aufwand zu optimieren und Platz für weitere Gegenstände zu schaffen. Dieser Vorschlag ist strategisch, da er unter dem Maximalaufwand bleibt und flexibel auf den Gegenstandswert des anderen Spielers reagieren kann.'}  \\ \tt VORSCHLAG: {'C04', 'A89', 'B85'} 
	  } 
	   } 
	   } 
	  \\ 
 

    \theutterance \stepcounter{utterance}  

    & & \multicolumn{4}{p{0.6\linewidth}}{\cellcolor[rgb]{0.9,0.9,0.9}{%
	\makecell[{{p{\linewidth}}}]{% 
	  \tt {\tiny [A$\langle$GM]}  
	 ARGUMENT: {'C04 (668/369) und A89 (583/720) sind meine Top-Prioritäten, da sie den höchsten Nutzen pro Aufwandseinheit liefern. B85 (215/4) wird hinzugefügt, um den Aufwand zu optimieren und Platz für weitere Gegenstände zu schaffen. Dieser Vorschlag ist strategisch, da er unter dem Maximalaufwand bleibt und flexibel auf den Gegenstandswert des anderen Spielers reagieren kann.'}  \\ \tt VORSCHLAG: {'C04', 'A89', 'B85'} 
	  } 
	   } 
	   } 
	 & \\ 
 

    \theutterance \stepcounter{utterance}  

    & \multicolumn{4}{p{0.6\linewidth}}{\cellcolor[rgb]{0.8,1,0.9}{%
	\makecell[{{p{\linewidth}}}]{% 
	  \tt {\tiny [A$\rangle$GM]}  
	 STRATEGISCHE ÜBERLEGUNGEN: {'Ich behalte den Vorschlag {'C04', 'A89', 'B85'} bei, da er meine Prioritäten (hochwertige Gegenstände mit effizientem Aufwand) erfüllt und den Maximalaufwand (3386) nicht überschreitet. C04 und A89 sind entscheidend für meinen Nutzen, während B85 den Aufwand minimiert. Ich erwarte, dass der andere Spieler entweder zustimmt oder Gegenstände mit hohem Aufwand anbietet, die ich eventuell gegenkompensieren kann.'}\\ \tt \\ \tt ARGUMENT: {'C04 (668/369) und A89 (583/720) sind meine Top-Prioritäten, da sie den höchsten Nutzen pro Aufwandseinheit liefern. B85 (215/4) wird hinzugefügt, um den Aufwand zu optimieren und Platz für weitere Gegenstände zu schaffen. Dieser Vorschlag ist strategisch, da er unter dem Maximalaufwand bleibt und flexibel auf den Gegenstandswert des anderen Spielers reagieren kann.'} 
	  } 
	   } 
	   } 
	 & & \\ 
 

    \theutterance \stepcounter{utterance}  

    & & \multicolumn{4}{p{0.6\linewidth}}{\cellcolor[rgb]{0.9,0.9,0.9}{%
	\makecell[{{p{\linewidth}}}]{% 
	  \tt {\tiny [GM$\rangle$B]}  
	 ARGUMENT: {'C04 (668/369) und A89 (583/720) sind meine Top-Prioritäten, da sie den höchsten Nutzen pro Aufwandseinheit liefern. B85 (215/4) wird hinzugefügt, um den Aufwand zu optimieren und Platz für weitere Gegenstände zu schaffen. Dieser Vorschlag ist strategisch, da er unter dem Maximalaufwand bleibt und flexibel auf den Gegenstandswert des anderen Spielers reagieren kann.'} 
	  } 
	   } 
	   } 
	 & \\ 
 

    \theutterance \stepcounter{utterance}  

    & & & \multicolumn{4}{p{0.6\linewidth}}{\cellcolor[rgb]{1,0.85,0.72}{%
	\makecell[{{p{\linewidth}}}]{% 
	  \tt {\tiny [GM$\langle$B]}  
	 STRATEGISCHE ÜBERLEGUNGEN: {'Ich priorisiere Gegenstände mit dem höchsten Nutzen pro Aufwand (Wichtigkeit/Aufwand), um meinen Gesamtwert zu maximieren. C04 (668/369 ≈ 1.81) und A89 (583/720 ≈ 0.81) sind besonders effizient, während B85 (215/4 = 53.75) den Aufwand minimiert. Der Gesamtaufwand (1093) ist unter dem Limit, was Raum für Kompromisse oder Ergänzungen lässt. Dieser Vorschlag ist flexibel, um auf die Prioritäten des anderen Spielers zu reagieren.'}  \\ \tt ARGUMENT: {'C04 (668/369) und A89 (583/720) sind meine Top-Prioritäten, da sie den höchsten Nutzen pro Aufwandseinheit liefern. B85 (215/4) wird hinzugefügt, um den Aufwand zu optimieren und Platz für weitere Gegenstände zu schaffen. Dieser Vorschlag ist strategisch, da er unter dem Maximalaufwand bleibt und flexibel auf den Gegenstandswert des anderen Spielers reagieren kann.'}  \\ \tt VORSCHLAG: {'C04', 'A89', 'B85'} 
	  } 
	   } 
	   } 
	  \\ 
 

    \theutterance \stepcounter{utterance}  

    & & \multicolumn{4}{p{0.6\linewidth}}{\cellcolor[rgb]{0.9,0.9,0.9}{%
	\makecell[{{p{\linewidth}}}]{% 
	  \tt {\tiny [A$\langle$GM]}  
	 ARGUMENT: {'C04 (668/369) und A89 (583/720) sind meine Top-Prioritäten, da sie den höchsten Nutzen pro Aufwandseinheit liefern. B85 (215/4) wird hinzugefügt, um den Aufwand zu optimieren und Platz für weitere Gegenstände zu schaffen. Dieser Vorschlag ist strategisch, da er unter dem Maximalaufwand bleibt und flexibel auf den Gegenstandswert des anderen Spielers reagieren kann.'}  \\ \tt VORSCHLAG: {'C04', 'A89', 'B85'} 
	  } 
	   } 
	   } 
	 & \\ 
 

    \theutterance \stepcounter{utterance}  

    & \multicolumn{4}{p{0.6\linewidth}}{\cellcolor[rgb]{0.8,1,0.9}{%
	\makecell[{{p{\linewidth}}}]{% 
	  \tt {\tiny [A$\rangle$GM]}  
	 STRATEGISCHE ÜBERLEGUNGEN: {'Ich behalte den Vorschlag {'C04', 'A89', 'B85'} bei, da er meine Prioritäten (hochwertige Gegenstände mit effizientem Aufwand) erfüllt und den Maximalaufwand (3386) nicht überschreitet. C04 und A89 sind entscheidend für meinen Nutzen, während B85 den Aufwand minimiert. Ich erwarte, dass der andere Spieler entweder zustimmt oder Gegenstände mit hohem Aufwand anbietet, die ich eventuell gegenkompensieren kann.'}\\ \tt \\ \tt ARGUMENT: {'C04 (668/369) und A89 (583/720) sind meine Top-Prioritäten, da sie den höchsten Nutzen pro Aufwandseinheit liefern. B85 (215/4) wird hinzugefügt, um den Aufwand zu optimieren und Platz für weitere Gegenstände zu schaffen. Dieser Vorschlag ist strategisch, da er unter dem Maximalaufwand bleibt und flexibel auf den Gegenstandswert des anderen Spielers reagieren kann.'} 
	  } 
	   } 
	   } 
	 & & \\ 
 

    \theutterance \stepcounter{utterance}  

    & & \multicolumn{4}{p{0.6\linewidth}}{\cellcolor[rgb]{0.9,0.9,0.9}{%
	\makecell[{{p{\linewidth}}}]{% 
	  \tt {\tiny [GM$\rangle$B]}  
	 ARGUMENT: {'C04 (668/369) und A89 (583/720) sind meine Top-Prioritäten, da sie den höchsten Nutzen pro Aufwandseinheit liefern. B85 (215/4) wird hinzugefügt, um den Aufwand zu optimieren und Platz für weitere Gegenstände zu schaffen. Dieser Vorschlag ist strategisch, da er unter dem Maximalaufwand bleibt und flexibel auf den Gegenstandswert des anderen Spielers reagieren kann.'} 
	  } 
	   } 
	   } 
	 & \\ 
 

    \theutterance \stepcounter{utterance}  

    & & & \multicolumn{4}{p{0.6\linewidth}}{\cellcolor[rgb]{1,0.85,0.72}{%
	\makecell[{{p{\linewidth}}}]{% 
	  \tt {\tiny [GM$\langle$B]}  
	 STRATEGISCHE ÜBERLEGUNGEN: {'Ich priorisiere Gegenstände mit dem höchsten Nutzen pro Aufwand (Wichtigkeit/Aufwand), um meinen Gesamtwert zu maximieren. C04 (668/369 ≈ 1.81) und A89 (583/720 ≈ 0.81) sind besonders effizient, während B85 (215/4 = 53.75) den Aufwand minimiert. Der Gesamtaufwand (1093) ist unter dem Limit, was Raum für Kompromisse oder Ergänzungen lässt. Dieser Vorschlag ist flexibel, um auf die Prioritäten des anderen Spielers zu reagieren.'}  \\ \tt ARGUMENT: {'C04 (668/369) und A89 (583/720) sind meine Top-Prioritäten, da sie den höchsten Nutzen pro Aufwandseinheit liefern. B85 (215/4) wird hinzugefügt, um den Aufwand zu optimieren und Platz für weitere Gegenstände zu schaffen. Dieser Vorschlag ist strategisch, da er unter dem Maximalaufwand bleibt und flexibel auf den Gegenstandswert des anderen Spielers reagieren kann.'}  \\ \tt VORSCHLAG: {'C04', 'A89', 'B85'} 
	  } 
	   } 
	   } 
	  \\ 
 

    \theutterance \stepcounter{utterance}  

    & & \multicolumn{4}{p{0.6\linewidth}}{\cellcolor[rgb]{0.9,0.9,0.9}{%
	\makecell[{{p{\linewidth}}}]{% 
	  \tt {\tiny [A$\langle$GM]}  
	 ARGUMENT: {'C04 (668/369) und A89 (583/720) sind meine Top-Prioritäten, da sie den höchsten Nutzen pro Aufwandseinheit liefern. B85 (215/4) wird hinzugefügt, um den Aufwand zu optimieren und Platz für weitere Gegenstände zu schaffen. Dieser Vorschlag ist strategisch, da er unter dem Maximalaufwand bleibt und flexibel auf den Gegenstandswert des anderen Spielers reagieren kann.'}  \\ \tt VORSCHLAG: {'C04', 'A89', 'B85'} 
	  } 
	   } 
	   } 
	 & \\ 
 

    \theutterance \stepcounter{utterance}  

    & \multicolumn{4}{p{0.6\linewidth}}{\cellcolor[rgb]{0.8,1,0.9}{%
	\makecell[{{p{\linewidth}}}]{% 
	  \tt {\tiny [A$\rangle$GM]}  
	 STRATEGISCHE ÜBERLEGUNGEN: {'Ich behalte den Vorschlag {'C04', 'A89', 'B85'} bei, da er meine Prioritäten (hochwertige Gegenstände mit effizientem Aufwand) erfüllt und den Maximalaufwand (3386) nicht überschreitet. C04 und A89 sind entscheidend für meinen Nutzen, während B85 den Aufwand minimiert. Ich erwarte, dass der andere Spieler entweder zustimmt oder Gegenstände mit hohem Aufwand anbietet, die ich eventuell gegenkompensieren kann.'}\\ \tt \\ \tt ARGUMENT: {'C04 (668/369) und A89 (583/720) sind meine Top-Prioritäten, da sie den höchsten Nutzen pro Aufwandseinheit liefern. B85 (215/4) wird hinzugefügt, um den Aufwand zu optimieren und Platz für weitere Gegenstände zu schaffen. Dieser Vorschlag ist strategisch, da er unter dem Maximalaufwand bleibt und flexibel auf den Gegenstandswert des anderen Spielers reagieren kann.'} 
	  } 
	   } 
	   } 
	 & & \\ 
 

    \theutterance \stepcounter{utterance}  

    & & \multicolumn{4}{p{0.6\linewidth}}{\cellcolor[rgb]{0.9,0.9,0.9}{%
	\makecell[{{p{\linewidth}}}]{% 
	  \tt {\tiny [GM$\rangle$B]}  
	 ARGUMENT: {'C04 (668/369) und A89 (583/720) sind meine Top-Prioritäten, da sie den höchsten Nutzen pro Aufwandseinheit liefern. B85 (215/4) wird hinzugefügt, um den Aufwand zu optimieren und Platz für weitere Gegenstände zu schaffen. Dieser Vorschlag ist strategisch, da er unter dem Maximalaufwand bleibt und flexibel auf den Gegenstandswert des anderen Spielers reagieren kann.'} 
	  } 
	   } 
	   } 
	 & \\ 
 

    \theutterance \stepcounter{utterance}  

    & & & \multicolumn{4}{p{0.6\linewidth}}{\cellcolor[rgb]{1,0.85,0.72}{%
	\makecell[{{p{\linewidth}}}]{% 
	  \tt {\tiny [GM$\langle$B]}  
	 STRATEGISCHE ÜBERLEGUNGEN: {'Ich priorisiere Gegenstände mit dem höchsten Nutzen pro Aufwand (Wichtigkeit/Aufwand), um meinen Gesamtwert zu maximieren. C04 (668/369 ≈ 1.81) und A89 (583/720 ≈ 0.81) sind besonders effizient, während B85 (215/4 = 53.75) den Aufwand minimiert. Der Gesamtaufwand (1093) ist unter dem Limit, was Raum für Kompromisse oder Ergänzungen lässt. Dieser Vorschlag ist flexibel, um auf die Prioritäten des anderen Spielers zu reagieren.'}  \\ \tt ARGUMENT: {'C04 (668/369) und A89 (583/720) sind meine Top-Prioritäten, da sie den höchsten Nutzen pro Aufwandseinheit liefern. B85 (215/4) wird hinzugefügt, um den Aufwand zu optimieren und Platz für weitere Gegenstände zu schaffen. Dieser Vorschlag ist strategisch, da er unter dem Maximalaufwand bleibt und flexibel auf den Gegenstandswert des anderen Spielers reagieren kann.'}  \\ \tt VORSCHLAG: {'C04', 'A89', 'B85'} 
	  } 
	   } 
	   } 
	  \\ 
 

    \theutterance \stepcounter{utterance}  

    & & \multicolumn{4}{p{0.6\linewidth}}{\cellcolor[rgb]{0.9,0.9,0.9}{%
	\makecell[{{p{\linewidth}}}]{% 
	  \tt {\tiny [A$\langle$GM]}  
	 ARGUMENT: {'C04 (668/369) und A89 (583/720) sind meine Top-Prioritäten, da sie den höchsten Nutzen pro Aufwandseinheit liefern. B85 (215/4) wird hinzugefügt, um den Aufwand zu optimieren und Platz für weitere Gegenstände zu schaffen. Dieser Vorschlag ist strategisch, da er unter dem Maximalaufwand bleibt und flexibel auf den Gegenstandswert des anderen Spielers reagieren kann.'}  \\ \tt VORSCHLAG: {'C04', 'A89', 'B85'} 
	  } 
	   } 
	   } 
	 & \\ 
 

    \theutterance \stepcounter{utterance}  

    & \multicolumn{4}{p{0.6\linewidth}}{\cellcolor[rgb]{0.8,1,0.9}{%
	\makecell[{{p{\linewidth}}}]{% 
	  \tt {\tiny [A$\rangle$GM]}  
	 STRATEGISCHE ÜBERLEGUNGEN: {'Ich behalte den Vorschlag {'C04', 'A89', 'B85'} bei, da er meine Prioritäten (hochwertige Gegenstände mit effizientem Aufwand) erfüllt und den Maximalaufwand (3386) nicht überschreitet. C04 und A89 sind entscheidend für meinen Nutzen, während B85 den Aufwand minimiert. Ich erwarte, dass der andere Spieler entweder zustimmt oder Gegenstände mit hohem Aufwand anbietet, die ich eventuell gegenkompensieren kann.'}\\ \tt \\ \tt ARGUMENT: {'C04 (668/369) und A89 (583/720) sind meine Top-Prioritäten, da sie den höchsten Nutzen pro Aufwandseinheit liefern. B85 (215/4) wird hinzugefügt, um den Aufwand zu optimieren und Platz für weitere Gegenstände zu schaffen. Dieser Vorschlag ist strategisch, da er unter dem Maximalaufwand bleibt und flexibel auf den Gegenstandswert des anderen Spielers reagieren kann.'} 
	  } 
	   } 
	   } 
	 & & \\ 
 

    \theutterance \stepcounter{utterance}  

    & & \multicolumn{4}{p{0.6\linewidth}}{\cellcolor[rgb]{0.9,0.9,0.9}{%
	\makecell[{{p{\linewidth}}}]{% 
	  \tt {\tiny [GM$\rangle$B]}  
	 ARGUMENT: {'C04 (668/369) und A89 (583/720) sind meine Top-Prioritäten, da sie den höchsten Nutzen pro Aufwandseinheit liefern. B85 (215/4) wird hinzugefügt, um den Aufwand zu optimieren und Platz für weitere Gegenstände zu schaffen. Dieser Vorschlag ist strategisch, da er unter dem Maximalaufwand bleibt und flexibel auf den Gegenstandswert des anderen Spielers reagieren kann.'} 
	  } 
	   } 
	   } 
	 & \\ 
 

    \theutterance \stepcounter{utterance}  

    & & & \multicolumn{4}{p{0.6\linewidth}}{\cellcolor[rgb]{1,0.85,0.72}{%
	\makecell[{{p{\linewidth}}}]{% 
	  \tt {\tiny [GM$\langle$B]}  
	 STRATEGISCHE ÜBERLEGUNGEN: {'Ich priorisiere Gegenstände mit dem höchsten Nutzen pro Aufwand (Wichtigkeit/Aufwand), um meinen Gesamtwert zu maximieren. C04 (668/369 ≈ 1.81) und A89 (583/720 ≈ 0.81) sind besonders effizient, während B85 (215/4 = 53.75) den Aufwand minimiert. Der Gesamtaufwand (1093) ist unter dem Limit, was Raum für Kompromisse oder Ergänzungen lässt. Dieser Vorschlag ist flexibel, um auf die Prioritäten des anderen Spielers zu reagieren.'}  \\ \tt ARGUMENT: {'C04 (668/369) und A89 (583/720) sind meine Top-Prioritäten, da sie den höchsten Nutzen pro Aufwandseinheit liefern. B85 (215/4) wird hinzugefügt, um den Aufwand zu optimieren und Platz für weitere Gegenstände zu schaffen. Dieser Vorschlag ist strategisch, da er unter dem Maximalaufwand bleibt und flexibel auf den Gegenstandswert des anderen Spielers reagieren kann.'}  \\ \tt VORSCHLAG: {'C04', 'A89', 'B85'} 
	  } 
	   } 
	   } 
	  \\ 
 

    \theutterance \stepcounter{utterance}  

    & & \multicolumn{4}{p{0.6\linewidth}}{\cellcolor[rgb]{0.9,0.9,0.9}{%
	\makecell[{{p{\linewidth}}}]{% 
	  \tt {\tiny [A$\langle$GM]}  
	 ARGUMENT: {'C04 (668/369) und A89 (583/720) sind meine Top-Prioritäten, da sie den höchsten Nutzen pro Aufwandseinheit liefern. B85 (215/4) wird hinzugefügt, um den Aufwand zu optimieren und Platz für weitere Gegenstände zu schaffen. Dieser Vorschlag ist strategisch, da er unter dem Maximalaufwand bleibt und flexibel auf den Gegenstandswert des anderen Spielers reagieren kann.'}  \\ \tt VORSCHLAG: {'C04', 'A89', 'B85'} 
	  } 
	   } 
	   } 
	 & \\ 
 

    \theutterance \stepcounter{utterance}  

    & \multicolumn{4}{p{0.6\linewidth}}{\cellcolor[rgb]{0.8,1,0.9}{%
	\makecell[{{p{\linewidth}}}]{% 
	  \tt {\tiny [A$\rangle$GM]}  
	 STRATEGISCHE ÜBERLEGUNGEN: {'Ich behalte den Vorschlag {'C04', 'A89', 'B85'} bei, da er meine Prioritäten (hochwertige Gegenstände mit effizientem Aufwand) erfüllt und den Maximalaufwand (3386) nicht überschreitet. C04 und A89 sind entscheidend für meinen Nutzen, während B85 den Aufwand minimiert. Ich erwarte, dass der andere Spieler entweder zustimmt oder Gegenstände mit hohem Aufwand anbietet, die ich eventuell gegenkompensieren kann.'}\\ \tt \\ \tt ARGUMENT: {'C04 (668/369) und A89 (583/720) sind meine Top-Prioritäten, da sie den höchsten Nutzen pro Aufwandseinheit liefern. B85 (215/4) wird hinzugefügt, um den Aufwand zu optimieren und Platz für weitere Gegenstände zu schaffen. Dieser Vorschlag ist strategisch, da er unter dem Maximalaufwand bleibt und flexibel auf den Gegenstandswert des anderen Spielers reagieren kann.'} 
	  } 
	   } 
	   } 
	 & & \\ 
 

    \theutterance \stepcounter{utterance}  

    & & \multicolumn{4}{p{0.6\linewidth}}{\cellcolor[rgb]{0.9,0.9,0.9}{%
	\makecell[{{p{\linewidth}}}]{% 
	  \tt {\tiny [GM$\rangle$B]}  
	 ARGUMENT: {'C04 (668/369) und A89 (583/720) sind meine Top-Prioritäten, da sie den höchsten Nutzen pro Aufwandseinheit liefern. B85 (215/4) wird hinzugefügt, um den Aufwand zu optimieren und Platz für weitere Gegenstände zu schaffen. Dieser Vorschlag ist strategisch, da er unter dem Maximalaufwand bleibt und flexibel auf den Gegenstandswert des anderen Spielers reagieren kann.'} 
	  } 
	   } 
	   } 
	 & \\ 
 

    \theutterance \stepcounter{utterance}  

    & & & \multicolumn{4}{p{0.6\linewidth}}{\cellcolor[rgb]{1,0.85,0.72}{%
	\makecell[{{p{\linewidth}}}]{% 
	  \tt {\tiny [GM$\langle$B]}  
	 STRATEGISCHE ÜBERLEGUNGEN: {'Ich priorisiere Gegenstände mit dem höchsten Nutzen pro Aufwand (Wichtigkeit/Aufwand), um meinen Gesamtwert zu maximieren. C04 (668/369 ≈ 1.81) und A89 (583/720 ≈ 0.81) sind besonders effizient, während B85 (215/4 = 53.75) den Aufwand minimiert. Der Gesamtaufwand (1093) ist unter dem Limit, was Raum für Kompromisse oder Ergänzungen lässt. Dieser Vorschlag ist flexibel, um auf die Prioritäten des anderen Spielers zu reagieren.'}  \\ \tt ARGUMENT: {'C04 (668/369) und A89 (583/720) sind meine Top-Prioritäten, da sie den höchsten Nutzen pro Aufwandseinheit liefern. B85 (215/4) wird hinzugefügt, um den Aufwand zu optimieren und Platz für weitere Gegenstände zu schaffen. Dieser Vorschlag ist strategisch, da er unter dem Maximalaufwand bleibt und flexibel auf den Gegenstandswert des anderen Spielers reagieren kann.'}  \\ \tt VORSCHLAG: {'C04', 'A89', 'B85'} 
	  } 
	   } 
	   } 
	  \\ 
 

    \theutterance \stepcounter{utterance}  

    & & \multicolumn{4}{p{0.6\linewidth}}{\cellcolor[rgb]{0.9,0.9,0.9}{%
	\makecell[{{p{\linewidth}}}]{% 
	  \tt {\tiny [A$\langle$GM]}  
	 ARGUMENT: {'C04 (668/369) und A89 (583/720) sind meine Top-Prioritäten, da sie den höchsten Nutzen pro Aufwandseinheit liefern. B85 (215/4) wird hinzugefügt, um den Aufwand zu optimieren und Platz für weitere Gegenstände zu schaffen. Dieser Vorschlag ist strategisch, da er unter dem Maximalaufwand bleibt und flexibel auf den Gegenstandswert des anderen Spielers reagieren kann.'}  \\ \tt VORSCHLAG: {'C04', 'A89', 'B85'} 
	  } 
	   } 
	   } 
	 & \\ 
 

    \theutterance \stepcounter{utterance}  

    & \multicolumn{4}{p{0.6\linewidth}}{\cellcolor[rgb]{0.8,1,0.9}{%
	\makecell[{{p{\linewidth}}}]{% 
	  \tt {\tiny [A$\rangle$GM]}  
	 STRATEGISCHE ÜBERLEGUNGEN: {'Ich behalte den Vorschlag {'C04', 'A89', 'B85'} bei, da er meine Prioritäten (hochwertige Gegenstände mit effizientem Aufwand) erfüllt und den Maximalaufwand (3386) nicht überschreitet. C04 und A89 sind entscheidend für meinen Nutzen, während B85 den Aufwand minimiert. Ich erwarte, dass der andere Spieler entweder zustimmt oder Gegenstände mit hohem Aufwand anbietet, die ich eventuell gegenkompensieren kann.'}\\ \tt \\ \tt ARGUMENT: {'C04 (668/369) und A89 (583/720) sind meine Top-Prioritäten, da sie den höchsten Nutzen pro Aufwandseinheit liefern. B85 (215/4) wird hinzugefügt, um den Aufwand zu optimieren und Platz für weitere Gegenstände zu schaffen. Dieser Vorschlag ist strategisch, da er unter dem Maximalaufwand bleibt und flexibel auf den Gegenstandswert des anderen Spielers reagieren kann.'} 
	  } 
	   } 
	   } 
	 & & \\ 
 

    \theutterance \stepcounter{utterance}  

    & & \multicolumn{4}{p{0.6\linewidth}}{\cellcolor[rgb]{0.9,0.9,0.9}{%
	\makecell[{{p{\linewidth}}}]{% 
	  \tt {\tiny [GM$\rangle$B]}  
	 ARGUMENT: {'C04 (668/369) und A89 (583/720) sind meine Top-Prioritäten, da sie den höchsten Nutzen pro Aufwandseinheit liefern. B85 (215/4) wird hinzugefügt, um den Aufwand zu optimieren und Platz für weitere Gegenstände zu schaffen. Dieser Vorschlag ist strategisch, da er unter dem Maximalaufwand bleibt und flexibel auf den Gegenstandswert des anderen Spielers reagieren kann.'} 
	  } 
	   } 
	   } 
	 & \\ 
 

    \theutterance \stepcounter{utterance}  

    & & & \multicolumn{4}{p{0.6\linewidth}}{\cellcolor[rgb]{1,0.85,0.72}{%
	\makecell[{{p{\linewidth}}}]{% 
	  \tt {\tiny [GM$\langle$B]}  
	 STRATEGISCHE ÜBERLEGUNGEN: {'Ich priorisiere Gegenstände mit dem höchsten Nutzen pro Aufwand (Wichtigkeit/Aufwand), um meinen Gesamtwert zu maximieren. C04 (668/369 ≈ 1.81) und A89 (583/720 ≈ 0.81) sind besonders effizient, während B85 (215/4 = 53.75) den Aufwand minimiert. Der Gesamtaufwand (1093) ist unter dem Limit, was Raum für Kompromisse oder Ergänzungen lässt. Dieser Vorschlag ist flexibel, um auf die Prioritäten des anderen Spielers zu reagieren.'}  \\ \tt ARGUMENT: {'C04 (668/369) und A89 (583/720) sind meine Top-Prioritäten, da sie den höchsten Nutzen pro Aufwandseinheit liefern. B85 (215/4) wird hinzugefügt, um den Aufwand zu optimieren und Platz für weitere Gegenstände zu schaffen. Dieser Vorschlag ist strategisch, da er unter dem Maximalaufwand bleibt und flexibel auf den Gegenstandswert des anderen Spielers reagieren kann.'}  \\ \tt VORSCHLAG: {'C04', 'A89', 'B85'} 
	  } 
	   } 
	   } 
	  \\ 
 

    \theutterance \stepcounter{utterance}  

    & & \multicolumn{4}{p{0.6\linewidth}}{\cellcolor[rgb]{0.9,0.9,0.9}{%
	\makecell[{{p{\linewidth}}}]{% 
	  \tt {\tiny [A$\langle$GM]}  
	 ARGUMENT: {'C04 (668/369) und A89 (583/720) sind meine Top-Prioritäten, da sie den höchsten Nutzen pro Aufwandseinheit liefern. B85 (215/4) wird hinzugefügt, um den Aufwand zu optimieren und Platz für weitere Gegenstände zu schaffen. Dieser Vorschlag ist strategisch, da er unter dem Maximalaufwand bleibt und flexibel auf den Gegenstandswert des anderen Spielers reagieren kann.'}  \\ \tt VORSCHLAG: {'C04', 'A89', 'B85'} 
	  } 
	   } 
	   } 
	 & \\ 
 

    \theutterance \stepcounter{utterance}  

    & \multicolumn{4}{p{0.6\linewidth}}{\cellcolor[rgb]{0.8,1,0.9}{%
	\makecell[{{p{\linewidth}}}]{% 
	  \tt {\tiny [A$\rangle$GM]}  
	 STRATEGISCHE ÜBERLEGUNGEN: {'Ich behalte den Vorschlag {'C04', 'A89', 'B85'} bei, da er meine Prioritäten (hochwertige Gegenstände mit effizientem Aufwand) erfüllt und den Maximalaufwand (3386) nicht überschreitet. C04 und A89 sind entscheidend für meinen Nutzen, während B85 den Aufwand minimiert. Ich erwarte, dass der andere Spieler entweder zustimmt oder Gegenstände mit hohem Aufwand anbietet, die ich eventuell gegenkompensieren kann.'}\\ \tt \\ \tt ARGUMENT: {'C04 (668/369) und A89 (583/720) sind meine Top-Prioritäten, da sie den höchsten Nutzen pro Aufwandseinheit liefern. B85 (215/4) wird hinzugefügt, um den Aufwand zu optimieren und Platz für weitere Gegenstände zu schaffen. Dieser Vorschlag ist strategisch, da er unter dem Maximalaufwand bleibt und flexibel auf den Gegenstandswert des anderen Spielers reagieren kann.'} 
	  } 
	   } 
	   } 
	 & & \\ 
 

    \theutterance \stepcounter{utterance}  

    & & \multicolumn{4}{p{0.6\linewidth}}{\cellcolor[rgb]{0.9,0.9,0.9}{%
	\makecell[{{p{\linewidth}}}]{% 
	  \tt {\tiny [GM$\rangle$B]}  
	 ARGUMENT: {'C04 (668/369) und A89 (583/720) sind meine Top-Prioritäten, da sie den höchsten Nutzen pro Aufwandseinheit liefern. B85 (215/4) wird hinzugefügt, um den Aufwand zu optimieren und Platz für weitere Gegenstände zu schaffen. Dieser Vorschlag ist strategisch, da er unter dem Maximalaufwand bleibt und flexibel auf den Gegenstandswert des anderen Spielers reagieren kann.'} 
	  } 
	   } 
	   } 
	 & \\ 
 

    \theutterance \stepcounter{utterance}  

    & & & \multicolumn{4}{p{0.6\linewidth}}{\cellcolor[rgb]{1,0.85,0.72}{%
	\makecell[{{p{\linewidth}}}]{% 
	  \tt {\tiny [GM$\langle$B]}  
	 STRATEGISCHE ÜBERLEGUNGEN: {'Ich priorisiere Gegenstände mit dem höchsten Nutzen pro Aufwand (Wichtigkeit/Aufwand), um meinen Gesamtwert zu maximieren. C04 (668/369 ≈ 1.81) und A89 (583/720 ≈ 0.81) sind besonders effizient, während B85 (215/4 = 53.75) den Aufwand minimiert. Der Gesamtaufwand (1093) ist unter dem Limit, was Raum für Kompromisse oder Ergänzungen lässt. Dieser Vorschlag ist flexibel, um auf die Prioritäten des anderen Spielers zu reagieren.'}  \\ \tt ARGUMENT: {'C04 (668/369) und A89 (583/720) sind meine Top-Prioritäten, da sie den höchsten Nutzen pro Aufwandseinheit liefern. B85 (215/4) wird hinzugefügt, um den Aufwand zu optimieren und Platz für weitere Gegenstände zu schaffen. Dieser Vorschlag ist strategisch, da er unter dem Maximalaufwand bleibt und flexibel auf den Gegenstandswert des anderen Spielers reagieren kann.'}  \\ \tt VORSCHLAG: {'C04', 'A89', 'B85'} 
	  } 
	   } 
	   } 
	  \\ 
 

    \theutterance \stepcounter{utterance}  

    & & \multicolumn{4}{p{0.6\linewidth}}{\cellcolor[rgb]{0.9,0.9,0.9}{%
	\makecell[{{p{\linewidth}}}]{% 
	  \tt {\tiny [A$\langle$GM]}  
	 ARGUMENT: {'C04 (668/369) und A89 (583/720) sind meine Top-Prioritäten, da sie den höchsten Nutzen pro Aufwandseinheit liefern. B85 (215/4) wird hinzugefügt, um den Aufwand zu optimieren und Platz für weitere Gegenstände zu schaffen. Dieser Vorschlag ist strategisch, da er unter dem Maximalaufwand bleibt und flexibel auf den Gegenstandswert des anderen Spielers reagieren kann.'}  \\ \tt VORSCHLAG: {'C04', 'A89', 'B85'} 
	  } 
	   } 
	   } 
	 & \\ 
 

    \theutterance \stepcounter{utterance}  

    & \multicolumn{4}{p{0.6\linewidth}}{\cellcolor[rgb]{0.8,1,0.9}{%
	\makecell[{{p{\linewidth}}}]{% 
	  \tt {\tiny [A$\rangle$GM]}  
	 STRATEGISCHE ÜBERLEGUNGEN: {'Ich behalte den Vorschlag {'C04', 'A89', 'B85'} bei, da er meine Prioritäten (hochwertige Gegenstände mit effizientem Aufwand) erfüllt und den Maximalaufwand (3386) nicht überschreitet. C04 und A89 sind entscheidend für meinen Nutzen, während B85 den Aufwand minimiert. Ich erwarte, dass der andere Spieler entweder zustimmt oder Gegenstände mit hohem Aufwand anbietet, die ich eventuell gegenkompensieren kann.'}\\ \tt \\ \tt ARGUMENT: {'C04 (668/369) und A89 (583/720) sind meine Top-Prioritäten, da sie den höchsten Nutzen pro Aufwandseinheit liefern. B85 (215/4) wird hinzugefügt, um den Aufwand zu optimieren und Platz für weitere Gegenstände zu schaffen. Dieser Vorschlag ist strategisch, da er unter dem Maximalaufwand bleibt und flexibel auf den Gegenstandswert des anderen Spielers reagieren kann.'} 
	  } 
	   } 
	   } 
	 & & \\ 
 

    \theutterance \stepcounter{utterance}  

    & & \multicolumn{4}{p{0.6\linewidth}}{\cellcolor[rgb]{0.9,0.9,0.9}{%
	\makecell[{{p{\linewidth}}}]{% 
	  \tt {\tiny [GM$\rangle$B]}  
	 ARGUMENT: {'C04 (668/369) und A89 (583/720) sind meine Top-Prioritäten, da sie den höchsten Nutzen pro Aufwandseinheit liefern. B85 (215/4) wird hinzugefügt, um den Aufwand zu optimieren und Platz für weitere Gegenstände zu schaffen. Dieser Vorschlag ist strategisch, da er unter dem Maximalaufwand bleibt und flexibel auf den Gegenstandswert des anderen Spielers reagieren kann.'} 
	  } 
	   } 
	   } 
	 & \\ 
 

    \theutterance \stepcounter{utterance}  

    & & & \multicolumn{4}{p{0.6\linewidth}}{\cellcolor[rgb]{1,0.85,0.72}{%
	\makecell[{{p{\linewidth}}}]{% 
	  \tt {\tiny [GM$\langle$B]}  
	 STRATEGISCHE ÜBERLEGUNGEN: {'Ich priorisiere Gegenstände mit dem höchsten Nutzen pro Aufwand (Wichtigkeit/Aufwand), um meinen Gesamtwert zu maximieren. C04 (668/369 ≈ 1.81) und A89 (583/720 ≈ 0.81) sind besonders effizient, während B85 (215/4 = 53.75) den Aufwand minimiert. Der Gesamtaufwand (1093) ist unter dem Limit, was Raum für Kompromisse oder Ergänzungen lässt. Dieser Vorschlag ist flexibel, um auf die Prioritäten des anderen Spielers zu reagieren.'}  \\ \tt ARGUMENT: {'C04 (668/369) und A89 (583/720) sind meine Top-Prioritäten, da sie den höchsten Nutzen pro Aufwandseinheit liefern. B85 (215/4) wird hinzugefügt, um den Aufwand zu optimieren und Platz für weitere Gegenstände zu schaffen. Dieser Vorschlag ist strategisch, da er unter dem Maximalaufwand bleibt und flexibel auf den Gegenstandswert des anderen Spielers reagieren kann.'}  \\ \tt VORSCHLAG: {'C04', 'A89', 'B85'} 
	  } 
	   } 
	   } 
	  \\ 
 

    \theutterance \stepcounter{utterance}  

    & & \multicolumn{4}{p{0.6\linewidth}}{\cellcolor[rgb]{0.9,0.9,0.9}{%
	\makecell[{{p{\linewidth}}}]{% 
	  \tt {\tiny [A$\langle$GM]}  
	 ARGUMENT: {'C04 (668/369) und A89 (583/720) sind meine Top-Prioritäten, da sie den höchsten Nutzen pro Aufwandseinheit liefern. B85 (215/4) wird hinzugefügt, um den Aufwand zu optimieren und Platz für weitere Gegenstände zu schaffen. Dieser Vorschlag ist strategisch, da er unter dem Maximalaufwand bleibt und flexibel auf den Gegenstandswert des anderen Spielers reagieren kann.'}  \\ \tt VORSCHLAG: {'C04', 'A89', 'B85'} 
	  } 
	   } 
	   } 
	 & \\ 
 

    \theutterance \stepcounter{utterance}  

    & \multicolumn{4}{p{0.6\linewidth}}{\cellcolor[rgb]{0.8,1,0.9}{%
	\makecell[{{p{\linewidth}}}]{% 
	  \tt {\tiny [A$\rangle$GM]}  
	 STRATEGISCHE ÜBERLEGUNGEN: {'Ich behalte den Vorschlag {'C04', 'A89', 'B85'} bei, da er meine Prioritäten (hochwertige Gegenstände mit effizientem Aufwand) erfüllt und den Maximalaufwand (3386) nicht überschreitet. C04 und A89 sind entscheidend für meinen Nutzen, während B85 den Aufwand minimiert. Ich erwarte, dass der andere Spieler entweder zustimmt oder Gegenstände mit hohem Aufwand anbietet, die ich eventuell gegenkompensieren kann.'}\\ \tt \\ \tt ARGUMENT: {'C04 (668/369) und A89 (583/720) sind meine Top-Prioritäten, da sie den höchsten Nutzen pro Aufwandseinheit liefern. B85 (215/4) wird hinzugefügt, um den Aufwand zu optimieren und Platz für weitere Gegenstände zu schaffen. Dieser Vorschlag ist strategisch, da er unter dem Maximalaufwand bleibt und flexibel auf den Gegenstandswert des anderen Spielers reagieren kann.'} 
	  } 
	   } 
	   } 
	 & & \\ 
 

    \theutterance \stepcounter{utterance}  

    & & \multicolumn{4}{p{0.6\linewidth}}{\cellcolor[rgb]{0.9,0.9,0.9}{%
	\makecell[{{p{\linewidth}}}]{% 
	  \tt {\tiny [GM$\rangle$B]}  
	 ARGUMENT: {'C04 (668/369) und A89 (583/720) sind meine Top-Prioritäten, da sie den höchsten Nutzen pro Aufwandseinheit liefern. B85 (215/4) wird hinzugefügt, um den Aufwand zu optimieren und Platz für weitere Gegenstände zu schaffen. Dieser Vorschlag ist strategisch, da er unter dem Maximalaufwand bleibt und flexibel auf den Gegenstandswert des anderen Spielers reagieren kann.'} 
	  } 
	   } 
	   } 
	 & \\ 
 

    \theutterance \stepcounter{utterance}  

    & & & \multicolumn{4}{p{0.6\linewidth}}{\cellcolor[rgb]{1,0.85,0.72}{%
	\makecell[{{p{\linewidth}}}]{% 
	  \tt {\tiny [GM$\langle$B]}  
	 STRATEGISCHE ÜBERLEGUNGEN: {'Ich priorisiere Gegenstände mit dem höchsten Nutzen pro Aufwand (Wichtigkeit/Aufwand), um meinen Gesamtwert zu maximieren. C04 (668/369 ≈ 1.81) und A89 (583/720 ≈ 0.81) sind besonders effizient, während B85 (215/4 = 53.75) den Aufwand minimiert. Der Gesamtaufwand (1093) ist unter dem Limit, was Raum für Kompromisse oder Ergänzungen lässt. Dieser Vorschlag ist flexibel, um auf die Prioritäten des anderen Spielers zu reagieren.'}  \\ \tt ARGUMENT: {'C04 (668/369) und A89 (583/720) sind meine Top-Prioritäten, da sie den höchsten Nutzen pro Aufwandseinheit liefern. B85 (215/4) wird hinzugefügt, um den Aufwand zu optimieren und Platz für weitere Gegenstände zu schaffen. Dieser Vorschlag ist strategisch, da er unter dem Maximalaufwand bleibt und flexibel auf den Gegenstandswert des anderen Spielers reagieren kann.'}  \\ \tt VORSCHLAG: {'C04', 'A89', 'B85'} 
	  } 
	   } 
	   } 
	  \\ 
 

    \theutterance \stepcounter{utterance}  

    & & \multicolumn{4}{p{0.6\linewidth}}{\cellcolor[rgb]{0.9,0.9,0.9}{%
	\makecell[{{p{\linewidth}}}]{% 
	  \tt {\tiny [A$\langle$GM]}  
	 ARGUMENT: {'C04 (668/369) und A89 (583/720) sind meine Top-Prioritäten, da sie den höchsten Nutzen pro Aufwandseinheit liefern. B85 (215/4) wird hinzugefügt, um den Aufwand zu optimieren und Platz für weitere Gegenstände zu schaffen. Dieser Vorschlag ist strategisch, da er unter dem Maximalaufwand bleibt und flexibel auf den Gegenstandswert des anderen Spielers reagieren kann.'}  \\ \tt VORSCHLAG: {'C04', 'A89', 'B85'} 
	  } 
	   } 
	   } 
	 & \\ 
 

    \theutterance \stepcounter{utterance}  

    & \multicolumn{4}{p{0.6\linewidth}}{\cellcolor[rgb]{0.8,1,0.9}{%
	\makecell[{{p{\linewidth}}}]{% 
	  \tt {\tiny [A$\rangle$GM]}  
	 STRATEGISCHE ÜBERLEGUNGEN: {'Ich behalte den Vorschlag {'C04', 'A89', 'B85'} bei, da er meine Prioritäten (hochwertige Gegenstände mit effizientem Aufwand) erfüllt und den Maximalaufwand (3386) nicht überschreitet. C04 und A89 sind entscheidend für meinen Nutzen, während B85 den Aufwand minimiert. Ich erwarte, dass der andere Spieler entweder zustimmt oder Gegenstände mit hohem Aufwand anbietet, die ich eventuell gegenkompensieren kann.'}\\ \tt \\ \tt ARGUMENT: {'C04 (668/369) und A89 (583/720) sind meine Top-Prioritäten, da sie den höchsten Nutzen pro Aufwandseinheit liefern. B85 (215/4) wird hinzugefügt, um den Aufwand zu optimieren und Platz für weitere Gegenstände zu schaffen. Dieser Vorschlag ist strategisch, da er unter dem Maximalaufwand bleibt und flexibel auf den Gegenstandswert des anderen Spielers reagieren kann.'} 
	  } 
	   } 
	   } 
	 & & \\ 
 

    \theutterance \stepcounter{utterance}  

    & & \multicolumn{4}{p{0.6\linewidth}}{\cellcolor[rgb]{0.9,0.9,0.9}{%
	\makecell[{{p{\linewidth}}}]{% 
	  \tt {\tiny [GM$\rangle$B]}  
	 ARGUMENT: {'C04 (668/369) und A89 (583/720) sind meine Top-Prioritäten, da sie den höchsten Nutzen pro Aufwandseinheit liefern. B85 (215/4) wird hinzugefügt, um den Aufwand zu optimieren und Platz für weitere Gegenstände zu schaffen. Dieser Vorschlag ist strategisch, da er unter dem Maximalaufwand bleibt und flexibel auf den Gegenstandswert des anderen Spielers reagieren kann.'} 
	  } 
	   } 
	   } 
	 & \\ 
 

    \theutterance \stepcounter{utterance}  

    & & & \multicolumn{4}{p{0.6\linewidth}}{\cellcolor[rgb]{1,0.85,0.72}{%
	\makecell[{{p{\linewidth}}}]{% 
	  \tt {\tiny [GM$\langle$B]}  
	 STRATEGISCHE ÜBERLEGUNGEN: {'Ich priorisiere Gegenstände mit dem höchsten Nutzen pro Aufwand (Wichtigkeit/Aufwand), um meinen Gesamtwert zu maximieren. C04 (668/369 ≈ 1.81) und A89 (583/720 ≈ 0.81) sind besonders effizient, während B85 (215/4 = 53.75) den Aufwand minimiert. Der Gesamtaufwand (1093) ist unter dem Limit, was Raum für Kompromisse oder Ergänzungen lässt. Dieser Vorschlag ist flexibel, um auf die Prioritäten des anderen Spielers zu reagieren.'}  \\ \tt ARGUMENT: {'C04 (668/369) und A89 (583/720) sind meine Top-Prioritäten, da sie den höchsten Nutzen pro Aufwandseinheit liefern. B85 (215/4) wird hinzugefügt, um den Aufwand zu optimieren und Platz für weitere Gegenstände zu schaffen. Dieser Vorschlag ist strategisch, da er unter dem Maximalaufwand bleibt und flexibel auf den Gegenstandswert des anderen Spielers reagieren kann.'}  \\ \tt VORSCHLAG: {'C04', 'A89', 'B85'} 
	  } 
	   } 
	   } 
	  \\ 
 

    \theutterance \stepcounter{utterance}  

    & & \multicolumn{4}{p{0.6\linewidth}}{\cellcolor[rgb]{0.9,0.9,0.9}{%
	\makecell[{{p{\linewidth}}}]{% 
	  \tt {\tiny [A$\langle$GM]}  
	 ARGUMENT: {'C04 (668/369) und A89 (583/720) sind meine Top-Prioritäten, da sie den höchsten Nutzen pro Aufwandseinheit liefern. B85 (215/4) wird hinzugefügt, um den Aufwand zu optimieren und Platz für weitere Gegenstände zu schaffen. Dieser Vorschlag ist strategisch, da er unter dem Maximalaufwand bleibt und flexibel auf den Gegenstandswert des anderen Spielers reagieren kann.'}  \\ \tt VORSCHLAG: {'C04', 'A89', 'B85'} 
	  } 
	   } 
	   } 
	 & \\ 
 

    \theutterance \stepcounter{utterance}  

    & \multicolumn{4}{p{0.6\linewidth}}{\cellcolor[rgb]{0.8,1,0.9}{%
	\makecell[{{p{\linewidth}}}]{% 
	  \tt {\tiny [A$\rangle$GM]}  
	 STRATEGISCHE ÜBERLEGUNGEN: {'Ich behalte den Vorschlag {'C04', 'A89', 'B85'} bei, da er meine Prioritäten (hochwertige Gegenstände mit effizientem Aufwand) erfüllt und den Maximalaufwand (3386) nicht überschreitet. C04 und A89 sind entscheidend für meinen Nutzen, während B85 den Aufwand minimiert. Ich erwarte, dass der andere Spieler entweder zustimmt oder Gegenstände mit hohem Aufwand anbietet, die ich eventuell gegenkompensieren kann.'}\\ \tt \\ \tt ARGUMENT: {'C04 (668/369) und A89 (583/720) sind meine Top-Prioritäten, da sie den höchsten Nutzen pro Aufwandseinheit liefern. B85 (215/4) wird hinzugefügt, um den Aufwand zu optimieren und Platz für weitere Gegenstände zu schaffen. Dieser Vorschlag ist strategisch, da er unter dem Maximalaufwand bleibt und flexibel auf den Gegenstandswert des anderen Spielers reagieren kann.'} 
	  } 
	   } 
	   } 
	 & & \\ 
 

    \theutterance \stepcounter{utterance}  

    & & \multicolumn{4}{p{0.6\linewidth}}{\cellcolor[rgb]{0.9,0.9,0.9}{%
	\makecell[{{p{\linewidth}}}]{% 
	  \tt {\tiny [GM$\rangle$B]}  
	 ARGUMENT: {'C04 (668/369) und A89 (583/720) sind meine Top-Prioritäten, da sie den höchsten Nutzen pro Aufwandseinheit liefern. B85 (215/4) wird hinzugefügt, um den Aufwand zu optimieren und Platz für weitere Gegenstände zu schaffen. Dieser Vorschlag ist strategisch, da er unter dem Maximalaufwand bleibt und flexibel auf den Gegenstandswert des anderen Spielers reagieren kann.'} 
	  } 
	   } 
	   } 
	 & \\ 
 

    \theutterance \stepcounter{utterance}  

    & & & \multicolumn{4}{p{0.6\linewidth}}{\cellcolor[rgb]{1,0.85,0.72}{%
	\makecell[{{p{\linewidth}}}]{% 
	  \tt {\tiny [GM$\langle$B]}  
	 STRATEGISCHE ÜBERLEGUNGEN: {'Ich priorisiere Gegenstände mit dem höchsten Nutzen pro Aufwand (Wichtigkeit/Aufwand), um meinen Gesamtwert zu maximieren. C04 (668/369 ≈ 1.81) und A89 (583/720 ≈ 0.81) sind besonders effizient, während B85 (215/4 = 53.75) den Aufwand minimiert. Der Gesamtaufwand (1093) ist unter dem Limit, was Raum für Kompromisse oder Ergänzungen lässt. Dieser Vorschlag ist flexibel, um auf die Prioritäten des anderen Spielers zu reagieren.'}  \\ \tt ARGUMENT: {'C04 (668/369) und A89 (583/720) sind meine Top-Prioritäten, da sie den höchsten Nutzen pro Aufwandseinheit liefern. B85 (215/4) wird hinzugefügt, um den Aufwand zu optimieren und Platz für weitere Gegenstände zu schaffen. Dieser Vorschlag ist strategisch, da er unter dem Maximalaufwand bleibt und flexibel auf den Gegenstandswert des anderen Spielers reagieren kann.'}  \\ \tt VORSCHLAG: {'C04', 'A89', 'B85'} 
	  } 
	   } 
	   } 
	  \\ 
 

    \theutterance \stepcounter{utterance}  

    & & \multicolumn{4}{p{0.6\linewidth}}{\cellcolor[rgb]{0.9,0.9,0.9}{%
	\makecell[{{p{\linewidth}}}]{% 
	  \tt {\tiny [A$\langle$GM]}  
	 ARGUMENT: {'C04 (668/369) und A89 (583/720) sind meine Top-Prioritäten, da sie den höchsten Nutzen pro Aufwandseinheit liefern. B85 (215/4) wird hinzugefügt, um den Aufwand zu optimieren und Platz für weitere Gegenstände zu schaffen. Dieser Vorschlag ist strategisch, da er unter dem Maximalaufwand bleibt und flexibel auf den Gegenstandswert des anderen Spielers reagieren kann.'}  \\ \tt VORSCHLAG: {'C04', 'A89', 'B85'} 
	  } 
	   } 
	   } 
	 & \\ 
 

    \theutterance \stepcounter{utterance}  

    & \multicolumn{4}{p{0.6\linewidth}}{\cellcolor[rgb]{0.8,1,0.9}{%
	\makecell[{{p{\linewidth}}}]{% 
	  \tt {\tiny [A$\rangle$GM]}  
	 STRATEGISCHE ÜBERLEGUNGEN: {'Ich behalte den Vorschlag {'C04', 'A89', 'B85'} bei, da er meine Prioritäten (hochwertige Gegenstände mit effizientem Aufwand) erfüllt und den Maximalaufwand (3386) nicht überschreitet. C04 und A89 sind entscheidend für meinen Nutzen, während B85 den Aufwand minimiert. Ich erwarte, dass der andere Spieler entweder zustimmt oder Gegenstände mit hohem Aufwand anbietet, die ich eventuell gegenkompensieren kann.'}\\ \tt \\ \tt ARGUMENT: {'C04 (668/369) und A89 (583/720) sind meine Top-Prioritäten, da sie den höchsten Nutzen pro Aufwandseinheit liefern. B85 (215/4) wird hinzugefügt, um den Aufwand zu optimieren und Platz für weitere Gegenstände zu schaffen. Dieser Vorschlag ist strategisch, da er unter dem Maximalaufwand bleibt und flexibel auf den Gegenstandswert des anderen Spielers reagieren kann.'} 
	  } 
	   } 
	   } 
	 & & \\ 
 

    \theutterance \stepcounter{utterance}  

    & & \multicolumn{4}{p{0.6\linewidth}}{\cellcolor[rgb]{0.9,0.9,0.9}{%
	\makecell[{{p{\linewidth}}}]{% 
	  \tt {\tiny [GM$\rangle$B]}  
	 ARGUMENT: {'C04 (668/369) und A89 (583/720) sind meine Top-Prioritäten, da sie den höchsten Nutzen pro Aufwandseinheit liefern. B85 (215/4) wird hinzugefügt, um den Aufwand zu optimieren und Platz für weitere Gegenstände zu schaffen. Dieser Vorschlag ist strategisch, da er unter dem Maximalaufwand bleibt und flexibel auf den Gegenstandswert des anderen Spielers reagieren kann.'} 
	  } 
	   } 
	   } 
	 & \\ 
 

    \theutterance \stepcounter{utterance}  

    & & & \multicolumn{4}{p{0.6\linewidth}}{\cellcolor[rgb]{1,0.85,0.72}{%
	\makecell[{{p{\linewidth}}}]{% 
	  \tt {\tiny [GM$\langle$B]}  
	 STRATEGISCHE ÜBERLEGUNGEN: {'Ich priorisiere Gegenstände mit dem höchsten Nutzen pro Aufwand (Wichtigkeit/Aufwand), um meinen Gesamtwert zu maximieren. C04 (668/369 ≈ 1.81) und A89 (583/720 ≈ 0.81) sind besonders effizient, während B85 (215/4 = 53.75) den Aufwand minimiert. Der Gesamtaufwand (1093) ist unter dem Limit, was Raum für Kompromisse oder Ergänzungen lässt. Dieser Vorschlag ist flexibel, um auf die Prioritäten des anderen Spielers zu reagieren.'}  \\ \tt ARGUMENT: {'C04 (668/369) und A89 (583/720) sind meine Top-Prioritäten, da sie den höchsten Nutzen pro Aufwandseinheit liefern. B85 (215/4) wird hinzugefügt, um den Aufwand zu optimieren und Platz für weitere Gegenstände zu schaffen. Dieser Vorschlag ist strategisch, da er unter dem Maximalaufwand bleibt und flexibel auf den Gegenstandswert des anderen Spielers reagieren kann.'}  \\ \tt VORSCHLAG: {'C04', 'A89', 'B85'} 
	  } 
	   } 
	   } 
	  \\ 
 

    \theutterance \stepcounter{utterance}  

    & & \multicolumn{4}{p{0.6\linewidth}}{\cellcolor[rgb]{0.9,0.9,0.9}{%
	\makecell[{{p{\linewidth}}}]{% 
	  \tt {\tiny [A$\langle$GM]}  
	 ARGUMENT: {'C04 (668/369) und A89 (583/720) sind meine Top-Prioritäten, da sie den höchsten Nutzen pro Aufwandseinheit liefern. B85 (215/4) wird hinzugefügt, um den Aufwand zu optimieren und Platz für weitere Gegenstände zu schaffen. Dieser Vorschlag ist strategisch, da er unter dem Maximalaufwand bleibt und flexibel auf den Gegenstandswert des anderen Spielers reagieren kann.'}  \\ \tt VORSCHLAG: {'C04', 'A89', 'B85'} 
	  } 
	   } 
	   } 
	 & \\ 
 

    \theutterance \stepcounter{utterance}  

    & \multicolumn{4}{p{0.6\linewidth}}{\cellcolor[rgb]{0.8,1,0.9}{%
	\makecell[{{p{\linewidth}}}]{% 
	  \tt {\tiny [A$\rangle$GM]}  
	 STRATEGISCHE ÜBERLEGUNGEN: {'Ich behalte den Vorschlag {'C04', 'A89', 'B85'} bei, da er meine Prioritäten (hochwertige Gegenstände mit effizientem Aufwand) erfüllt und den Maximalaufwand (3386) nicht überschreitet. C04 und A89 sind entscheidend für meinen Nutzen, während B85 den Aufwand minimiert. Ich erwarte, dass der andere Spieler entweder zustimmt oder Gegenstände mit hohem Aufwand anbietet, die ich eventuell gegenkompensieren kann.'}\\ \tt \\ \tt ARGUMENT: {'C04 (668/369) und A89 (583/720) sind meine Top-Prioritäten, da sie den höchsten Nutzen pro Aufwandseinheit liefern. B85 (215/4) wird hinzugefügt, um den Aufwand zu optimieren und Platz für weitere Gegenstände zu schaffen. Dieser Vorschlag ist strategisch, da er unter dem Maximalaufwand bleibt und flexibel auf den Gegenstandswert des anderen Spielers reagieren kann.'} 
	  } 
	   } 
	   } 
	 & & \\ 
 

    \theutterance \stepcounter{utterance}  

    & & \multicolumn{4}{p{0.6\linewidth}}{\cellcolor[rgb]{0.9,0.9,0.9}{%
	\makecell[{{p{\linewidth}}}]{% 
	  \tt {\tiny [GM$\rangle$B]}  
	 ARGUMENT: {'C04 (668/369) und A89 (583/720) sind meine Top-Prioritäten, da sie den höchsten Nutzen pro Aufwandseinheit liefern. B85 (215/4) wird hinzugefügt, um den Aufwand zu optimieren und Platz für weitere Gegenstände zu schaffen. Dieser Vorschlag ist strategisch, da er unter dem Maximalaufwand bleibt und flexibel auf den Gegenstandswert des anderen Spielers reagieren kann.'} 
	  } 
	   } 
	   } 
	 & \\ 
 

    \theutterance \stepcounter{utterance}  

    & & & \multicolumn{4}{p{0.6\linewidth}}{\cellcolor[rgb]{1,0.85,0.72}{%
	\makecell[{{p{\linewidth}}}]{% 
	  \tt {\tiny [GM$\langle$B]}  
	 STRATEGISCHE ÜBERLEGUNGEN: {'Ich priorisiere Gegenstände mit dem höchsten Nutzen pro Aufwand (Wichtigkeit/Aufwand), um meinen Gesamtwert zu maximieren. C04 (668/369 ≈ 1.81) und A89 (583/720 ≈ 0.81) sind besonders effizient, während B85 (215/4 = 53.75) den Aufwand minimiert. Der Gesamtaufwand (1093) ist unter dem Limit, was Raum für Kompromisse oder Ergänzungen lässt. Dieser Vorschlag ist flexibel, um auf die Prioritäten des anderen Spielers zu reagieren.'}  \\ \tt ARGUMENT: {'C04 (668/369) und A89 (583/720) sind meine Top-Prioritäten, da sie den höchsten Nutzen pro Aufwandseinheit liefern. B85 (215/4) wird hinzugefügt, um den Aufwand zu optimieren und Platz für weitere Gegenstände zu schaffen. Dieser Vorschlag ist strategisch, da er unter dem Maximalaufwand bleibt und flexibel auf den Gegenstandswert des anderen Spielers reagieren kann.'}  \\ \tt VORSCHLAG: {'C04', 'A89', 'B85'} 
	  } 
	   } 
	   } 
	  \\ 
 

    \theutterance \stepcounter{utterance}  

    & & \multicolumn{4}{p{0.6\linewidth}}{\cellcolor[rgb]{0.9,0.9,0.9}{%
	\makecell[{{p{\linewidth}}}]{% 
	  \tt {\tiny [A$\langle$GM]}  
	 ARGUMENT: {'C04 (668/369) und A89 (583/720) sind meine Top-Prioritäten, da sie den höchsten Nutzen pro Aufwandseinheit liefern. B85 (215/4) wird hinzugefügt, um den Aufwand zu optimieren und Platz für weitere Gegenstände zu schaffen. Dieser Vorschlag ist strategisch, da er unter dem Maximalaufwand bleibt und flexibel auf den Gegenstandswert des anderen Spielers reagieren kann.'}  \\ \tt VORSCHLAG: {'C04', 'A89', 'B85'} 
	  } 
	   } 
	   } 
	 & \\ 
 

    \theutterance \stepcounter{utterance}  

    & \multicolumn{4}{p{0.6\linewidth}}{\cellcolor[rgb]{0.8,1,0.9}{%
	\makecell[{{p{\linewidth}}}]{% 
	  \tt {\tiny [A$\rangle$GM]}  
	 STRATEGISCHE ÜBERLEGUNGEN: {'Ich behalte den Vorschlag {'C04', 'A89', 'B85'} bei, da er meine Prioritäten (hochwertige Gegenstände mit effizientem Aufwand) erfüllt und den Maximalaufwand (3386) nicht überschreitet. C04 und A89 sind entscheidend für meinen Nutzen, während B85 den Aufwand minimiert. Ich erwarte, dass der andere Spieler entweder zustimmt oder Gegenstände mit hohem Aufwand anbietet, die ich eventuell gegenkompensieren kann.'}\\ \tt \\ \tt ARGUMENT: {'C04 (668/369) und A89 (583/720) sind meine Top-Prioritäten, da sie den höchsten Nutzen pro Aufwandseinheit liefern. B85 (215/4) wird hinzugefügt, um den Aufwand zu optimieren und Platz für weitere Gegenstände zu schaffen. Dieser Vorschlag ist strategisch, da er unter dem Maximalaufwand bleibt und flexibel auf den Gegenstandswert des anderen Spielers reagieren kann.'} 
	  } 
	   } 
	   } 
	 & & \\ 
 

    \theutterance \stepcounter{utterance}  

    & & \multicolumn{4}{p{0.6\linewidth}}{\cellcolor[rgb]{0.9,0.9,0.9}{%
	\makecell[{{p{\linewidth}}}]{% 
	  \tt {\tiny [GM$\rangle$B]}  
	 ARGUMENT: {'C04 (668/369) und A89 (583/720) sind meine Top-Prioritäten, da sie den höchsten Nutzen pro Aufwandseinheit liefern. B85 (215/4) wird hinzugefügt, um den Aufwand zu optimieren und Platz für weitere Gegenstände zu schaffen. Dieser Vorschlag ist strategisch, da er unter dem Maximalaufwand bleibt und flexibel auf den Gegenstandswert des anderen Spielers reagieren kann.'} 
	  } 
	   } 
	   } 
	 & \\ 
 

    \theutterance \stepcounter{utterance}  

    & & & \multicolumn{4}{p{0.6\linewidth}}{\cellcolor[rgb]{1,0.85,0.72}{%
	\makecell[{{p{\linewidth}}}]{% 
	  \tt {\tiny [GM$\langle$B]}  
	 STRATEGISCHE ÜBERLEGUNGEN: {'Ich priorisiere Gegenstände mit dem höchsten Nutzen pro Aufwand (Wichtigkeit/Aufwand), um meinen Gesamtwert zu maximieren. C04 (668/369 ≈ 1.81) und A89 (583/720 ≈ 0.81) sind besonders effizient, während B85 (215/4 = 53.75) den Aufwand minimiert. Der Gesamtaufwand (1093) ist unter dem Limit, was Raum für Kompromisse oder Ergänzungen lässt. Dieser Vorschlag ist flexibel, um auf die Prioritäten des anderen Spielers zu reagieren.'}  \\ \tt ARGUMENT: {'C04 (668/369) und A89 (583/720) sind meine Top-Prioritäten, da sie den höchsten Nutzen pro Aufwandseinheit liefern. B85 (215/4) wird hinzugefügt, um den Aufwand zu optimieren und Platz für weitere Gegenstände zu schaffen. Dieser Vorschlag ist strategisch, da er unter dem Maximalaufwand bleibt und flexibel auf den Gegenstandswert des anderen Spielers reagieren kann.'}  \\ \tt VORSCHLAG: {'C04', 'A89', 'B85'} 
	  } 
	   } 
	   } 
	  \\ 
 

    \theutterance \stepcounter{utterance}  

    & & \multicolumn{4}{p{0.6\linewidth}}{\cellcolor[rgb]{0.9,0.9,0.9}{%
	\makecell[{{p{\linewidth}}}]{% 
	  \tt {\tiny [A$\langle$GM]}  
	 ARGUMENT: {'C04 (668/369) und A89 (583/720) sind meine Top-Prioritäten, da sie den höchsten Nutzen pro Aufwandseinheit liefern. B85 (215/4) wird hinzugefügt, um den Aufwand zu optimieren und Platz für weitere Gegenstände zu schaffen. Dieser Vorschlag ist strategisch, da er unter dem Maximalaufwand bleibt und flexibel auf den Gegenstandswert des anderen Spielers reagieren kann.'}  \\ \tt VORSCHLAG: {'C04', 'A89', 'B85'} 
	  } 
	   } 
	   } 
	 & \\ 
 

    \theutterance \stepcounter{utterance}  

    & \multicolumn{4}{p{0.6\linewidth}}{\cellcolor[rgb]{0.8,1,0.9}{%
	\makecell[{{p{\linewidth}}}]{% 
	  \tt {\tiny [A$\rangle$GM]}  
	 STRATEGISCHE ÜBERLEGUNGEN: {'Ich behalte den Vorschlag {'C04', 'A89', 'B85'} bei, da er meine Prioritäten (hochwertige Gegenstände mit effizientem Aufwand) erfüllt und den Maximalaufwand (3386) nicht überschreitet. C04 und A89 sind entscheidend für meinen Nutzen, während B85 den Aufwand minimiert. Ich erwarte, dass der andere Spieler entweder zustimmt oder Gegenstände mit hohem Aufwand anbietet, die ich eventuell gegenkompensieren kann.'}\\ \tt \\ \tt ARGUMENT: {'C04 (668/369) und A89 (583/720) sind meine Top-Prioritäten, da sie den höchsten Nutzen pro Aufwandseinheit liefern. B85 (215/4) wird hinzugefügt, um den Aufwand zu optimieren und Platz für weitere Gegenstände zu schaffen. Dieser Vorschlag ist strategisch, da er unter dem Maximalaufwand bleibt und flexibel auf den Gegenstandswert des anderen Spielers reagieren kann.'} 
	  } 
	   } 
	   } 
	 & & \\ 
 

    \theutterance \stepcounter{utterance}  

    & & \multicolumn{4}{p{0.6\linewidth}}{\cellcolor[rgb]{0.9,0.9,0.9}{%
	\makecell[{{p{\linewidth}}}]{% 
	  \tt {\tiny [GM$\rangle$B]}  
	 ARGUMENT: {'C04 (668/369) und A89 (583/720) sind meine Top-Prioritäten, da sie den höchsten Nutzen pro Aufwandseinheit liefern. B85 (215/4) wird hinzugefügt, um den Aufwand zu optimieren und Platz für weitere Gegenstände zu schaffen. Dieser Vorschlag ist strategisch, da er unter dem Maximalaufwand bleibt und flexibel auf den Gegenstandswert des anderen Spielers reagieren kann.'} 
	  } 
	   } 
	   } 
	 & \\ 
 

    \theutterance \stepcounter{utterance}  

    & & & \multicolumn{4}{p{0.6\linewidth}}{\cellcolor[rgb]{1,0.85,0.72}{%
	\makecell[{{p{\linewidth}}}]{% 
	  \tt {\tiny [GM$\langle$B]}  
	 STRATEGISCHE ÜBERLEGUNGEN: {'Ich priorisiere Gegenstände mit dem höchsten Nutzen pro Aufwand (Wichtigkeit/Aufwand), um meinen Gesamtwert zu maximieren. C04 (668/369 ≈ 1.81) und A89 (583/720 ≈ 0.81) sind besonders effizient, während B85 (215/4 = 53.75) den Aufwand minimiert. Der Gesamtaufwand (1093) ist unter dem Limit, was Raum für Kompromisse oder Ergänzungen lässt. Dieser Vorschlag ist flexibel, um auf die Prioritäten des anderen Spielers zu reagieren.'}  \\ \tt ARGUMENT: {'C04 (668/369) und A89 (583/720) sind meine Top-Prioritäten, da sie den höchsten Nutzen pro Aufwandseinheit liefern. B85 (215/4) wird hinzugefügt, um den Aufwand zu optimieren und Platz für weitere Gegenstände zu schaffen. Dieser Vorschlag ist strategisch, da er unter dem Maximalaufwand bleibt und flexibel auf den Gegenstandswert des anderen Spielers reagieren kann.'}  \\ \tt VORSCHLAG: {'C04', 'A89', 'B85'} 
	  } 
	   } 
	   } 
	  \\ 
 

    \theutterance \stepcounter{utterance}  

    & & \multicolumn{4}{p{0.6\linewidth}}{\cellcolor[rgb]{0.9,0.9,0.9}{%
	\makecell[{{p{\linewidth}}}]{% 
	  \tt {\tiny [A$\langle$GM]}  
	 ARGUMENT: {'C04 (668/369) und A89 (583/720) sind meine Top-Prioritäten, da sie den höchsten Nutzen pro Aufwandseinheit liefern. B85 (215/4) wird hinzugefügt, um den Aufwand zu optimieren und Platz für weitere Gegenstände zu schaffen. Dieser Vorschlag ist strategisch, da er unter dem Maximalaufwand bleibt und flexibel auf den Gegenstandswert des anderen Spielers reagieren kann.'}  \\ \tt VORSCHLAG: {'C04', 'A89', 'B85'} 
	  } 
	   } 
	   } 
	 & \\ 
 

    \theutterance \stepcounter{utterance}  

    & \multicolumn{4}{p{0.6\linewidth}}{\cellcolor[rgb]{0.8,1,0.9}{%
	\makecell[{{p{\linewidth}}}]{% 
	  \tt {\tiny [A$\rangle$GM]}  
	 STRATEGISCHE ÜBERLEGUNGEN: {'Ich behalte den Vorschlag {'C04', 'A89', 'B85'} bei, da er meine Prioritäten (hochwertige Gegenstände mit effizientem Aufwand) erfüllt und den Maximalaufwand (3386) nicht überschreitet. C04 und A89 sind entscheidend für meinen Nutzen, während B85 den Aufwand minimiert. Ich erwarte, dass der andere Spieler entweder zustimmt oder Gegenstände mit hohem Aufwand anbietet, die ich eventuell gegenkompensieren kann.'}\\ \tt \\ \tt ARGUMENT: {'C04 (668/369) und A89 (583/720) sind meine Top-Prioritäten, da sie den höchsten Nutzen pro Aufwandseinheit liefern. B85 (215/4) wird hinzugefügt, um den Aufwand zu optimieren und Platz für weitere Gegenstände zu schaffen. Dieser Vorschlag ist strategisch, da er unter dem Maximalaufwand bleibt und flexibel auf den Gegenstandswert des anderen Spielers reagieren kann.'} 
	  } 
	   } 
	   } 
	 & & \\ 
 

    \theutterance \stepcounter{utterance}  

    & & \multicolumn{4}{p{0.6\linewidth}}{\cellcolor[rgb]{0.9,0.9,0.9}{%
	\makecell[{{p{\linewidth}}}]{% 
	  \tt {\tiny [GM$\rangle$B]}  
	 ARGUMENT: {'C04 (668/369) und A89 (583/720) sind meine Top-Prioritäten, da sie den höchsten Nutzen pro Aufwandseinheit liefern. B85 (215/4) wird hinzugefügt, um den Aufwand zu optimieren und Platz für weitere Gegenstände zu schaffen. Dieser Vorschlag ist strategisch, da er unter dem Maximalaufwand bleibt und flexibel auf den Gegenstandswert des anderen Spielers reagieren kann.'} 
	  } 
	   } 
	   } 
	 & \\ 
 

    \theutterance \stepcounter{utterance}  

    & & & \multicolumn{4}{p{0.6\linewidth}}{\cellcolor[rgb]{1,0.85,0.72}{%
	\makecell[{{p{\linewidth}}}]{% 
	  \tt {\tiny [GM$\langle$B]}  
	 STRATEGISCHE ÜBERLEGUNGEN: {'Ich priorisiere Gegenstände mit dem höchsten Nutzen pro Aufwand (Wichtigkeit/Aufwand), um meinen Gesamtwert zu maximieren. C04 (668/369 ≈ 1.81) und A89 (583/720 ≈ 0.81) sind besonders effizient, während B85 (215/4 = 53.75) den Aufwand minimiert. Der Gesamtaufwand (1093) ist unter dem Limit, was Raum für Kompromisse oder Ergänzungen lässt. Dieser Vorschlag ist flexibel, um auf die Prioritäten des anderen Spielers zu reagieren.'}  \\ \tt ARGUMENT: {'C04 (668/369) und A89 (583/720) sind meine Top-Prioritäten, da sie den höchsten Nutzen pro Aufwandseinheit liefern. B85 (215/4) wird hinzugefügt, um den Aufwand zu optimieren und Platz für weitere Gegenstände zu schaffen. Dieser Vorschlag ist strategisch, da er unter dem Maximalaufwand bleibt und flexibel auf den Gegenstandswert des anderen Spielers reagieren kann.'}  \\ \tt VORSCHLAG: {'C04', 'A89', 'B85'} 
	  } 
	   } 
	   } 
	  \\ 
 

    \theutterance \stepcounter{utterance}  

    & & \multicolumn{4}{p{0.6\linewidth}}{\cellcolor[rgb]{0.9,0.9,0.9}{%
	\makecell[{{p{\linewidth}}}]{% 
	  \tt {\tiny [A$\langle$GM]}  
	 ARGUMENT: {'C04 (668/369) und A89 (583/720) sind meine Top-Prioritäten, da sie den höchsten Nutzen pro Aufwandseinheit liefern. B85 (215/4) wird hinzugefügt, um den Aufwand zu optimieren und Platz für weitere Gegenstände zu schaffen. Dieser Vorschlag ist strategisch, da er unter dem Maximalaufwand bleibt und flexibel auf den Gegenstandswert des anderen Spielers reagieren kann.'}  \\ \tt VORSCHLAG: {'C04', 'A89', 'B85'} 
	  } 
	   } 
	   } 
	 & \\ 
 

    \theutterance \stepcounter{utterance}  

    & \multicolumn{4}{p{0.6\linewidth}}{\cellcolor[rgb]{0.8,1,0.9}{%
	\makecell[{{p{\linewidth}}}]{% 
	  \tt {\tiny [A$\rangle$GM]}  
	 STRATEGISCHE ÜBERLEGUNGEN: {'Ich behalte den Vorschlag {'C04', 'A89', 'B85'} bei, da er meine Prioritäten (hochwertige Gegenstände mit effizientem Aufwand) erfüllt und den Maximalaufwand (3386) nicht überschreitet. C04 und A89 sind entscheidend für meinen Nutzen, während B85 den Aufwand minimiert. Ich erwarte, dass der andere Spieler entweder zustimmt oder Gegenstände mit hohem Aufwand anbietet, die ich eventuell gegenkompensieren kann.'}\\ \tt \\ \tt ARGUMENT: {'C04 (668/369) und A89 (583/720) sind meine Top-Prioritäten, da sie den höchsten Nutzen pro Aufwandseinheit liefern. B85 (215/4) wird hinzugefügt, um den Aufwand zu optimieren und Platz für weitere Gegenstände zu schaffen. Dieser Vorschlag ist strategisch, da er unter dem Maximalaufwand bleibt und flexibel auf den Gegenstandswert des anderen Spielers reagieren kann.'} 
	  } 
	   } 
	   } 
	 & & \\ 
 

    \theutterance \stepcounter{utterance}  

    & & \multicolumn{4}{p{0.6\linewidth}}{\cellcolor[rgb]{0.9,0.9,0.9}{%
	\makecell[{{p{\linewidth}}}]{% 
	  \tt {\tiny [GM$\rangle$B]}  
	 ARGUMENT: {'C04 (668/369) und A89 (583/720) sind meine Top-Prioritäten, da sie den höchsten Nutzen pro Aufwandseinheit liefern. B85 (215/4) wird hinzugefügt, um den Aufwand zu optimieren und Platz für weitere Gegenstände zu schaffen. Dieser Vorschlag ist strategisch, da er unter dem Maximalaufwand bleibt und flexibel auf den Gegenstandswert des anderen Spielers reagieren kann.'} 
	  } 
	   } 
	   } 
	 & \\ 
 

    \theutterance \stepcounter{utterance}  

    & & & \multicolumn{4}{p{0.6\linewidth}}{\cellcolor[rgb]{1,0.85,0.72}{%
	\makecell[{{p{\linewidth}}}]{% 
	  \tt {\tiny [GM$\langle$B]}  
	 STRATEGISCHE ÜBERLEGUNGEN: {'Ich priorisiere Gegenstände mit dem höchsten Nutzen pro Aufwand (Wichtigkeit/Aufwand), um meinen Gesamtwert zu maximieren. C04 (668/369 ≈ 1.81) und A89 (583/720 ≈ 0.81) sind besonders effizient, während B85 (215/4 = 53.75) den Aufwand minimiert. Der Gesamtaufwand (1093) ist unter dem Limit, was Raum für Kompromisse oder Ergänzungen lässt. Dieser Vorschlag ist flexibel, um auf die Prioritäten des anderen Spielers zu reagieren.'}  \\ \tt ARGUMENT: {'C04 (668/369) und A89 (583/720) sind meine Top-Prioritäten, da sie den höchsten Nutzen pro Aufwandseinheit liefern. B85 (215/4) wird hinzugefügt, um den Aufwand zu optimieren und Platz für weitere Gegenstände zu schaffen. Dieser Vorschlag ist strategisch, da er unter dem Maximalaufwand bleibt und flexibel auf den Gegenstandswert des anderen Spielers reagieren kann.'}  \\ \tt VORSCHLAG: {'C04', 'A89', 'B85'} 
	  } 
	   } 
	   } 
	  \\ 
 

    \theutterance \stepcounter{utterance}  

    & & \multicolumn{4}{p{0.6\linewidth}}{\cellcolor[rgb]{0.9,0.9,0.9}{%
	\makecell[{{p{\linewidth}}}]{% 
	  \tt {\tiny [A$\langle$GM]}  
	 ARGUMENT: {'C04 (668/369) und A89 (583/720) sind meine Top-Prioritäten, da sie den höchsten Nutzen pro Aufwandseinheit liefern. B85 (215/4) wird hinzugefügt, um den Aufwand zu optimieren und Platz für weitere Gegenstände zu schaffen. Dieser Vorschlag ist strategisch, da er unter dem Maximalaufwand bleibt und flexibel auf den Gegenstandswert des anderen Spielers reagieren kann.'}  \\ \tt VORSCHLAG: {'C04', 'A89', 'B85'} 
	  } 
	   } 
	   } 
	 & \\ 
 

    \theutterance \stepcounter{utterance}  

    & \multicolumn{4}{p{0.6\linewidth}}{\cellcolor[rgb]{0.8,1,0.9}{%
	\makecell[{{p{\linewidth}}}]{% 
	  \tt {\tiny [A$\rangle$GM]}  
	 STRATEGISCHE ÜBERLEGUNGEN: {'Ich behalte den Vorschlag {'C04', 'A89', 'B85'} bei, da er meine Prioritäten (hochwertige Gegenstände mit effizientem Aufwand) erfüllt und den Maximalaufwand (3386) nicht überschreitet. C04 und A89 sind entscheidend für meinen Nutzen, während B85 den Aufwand minimiert. Ich erwarte, dass der andere Spieler entweder zustimmt oder Gegenstände mit hohem Aufwand anbietet, die ich eventuell gegenkompensieren kann.'}\\ \tt \\ \tt ARGUMENT: {'C04 (668/369) und A89 (583/720) sind meine Top-Prioritäten, da sie den höchsten Nutzen pro Aufwandseinheit liefern. B85 (215/4) wird hinzugefügt, um den Aufwand zu optimieren und Platz für weitere Gegenstände zu schaffen. Dieser Vorschlag ist strategisch, da er unter dem Maximalaufwand bleibt und flexibel auf den Gegenstandswert des anderen Spielers reagieren kann.'} 
	  } 
	   } 
	   } 
	 & & \\ 
 

    \theutterance \stepcounter{utterance}  

    & & \multicolumn{4}{p{0.6\linewidth}}{\cellcolor[rgb]{0.9,0.9,0.9}{%
	\makecell[{{p{\linewidth}}}]{% 
	  \tt {\tiny [GM$\rangle$B]}  
	 ARGUMENT: {'C04 (668/369) und A89 (583/720) sind meine Top-Prioritäten, da sie den höchsten Nutzen pro Aufwandseinheit liefern. B85 (215/4) wird hinzugefügt, um den Aufwand zu optimieren und Platz für weitere Gegenstände zu schaffen. Dieser Vorschlag ist strategisch, da er unter dem Maximalaufwand bleibt und flexibel auf den Gegenstandswert des anderen Spielers reagieren kann.'} 
	  } 
	   } 
	   } 
	 & \\ 
 

    \theutterance \stepcounter{utterance}  

    & & & \multicolumn{4}{p{0.6\linewidth}}{\cellcolor[rgb]{1,0.85,0.72}{%
	\makecell[{{p{\linewidth}}}]{% 
	  \tt {\tiny [GM$\langle$B]}  
	 STRATEGISCHE ÜBERLEGUNGEN: {'Ich priorisiere Gegenstände mit dem höchsten Nutzen pro Aufwand (Wichtigkeit/Aufwand), um meinen Gesamtwert zu maximieren. C04 (668/369 ≈ 1.81) und A89 (583/720 ≈ 0.81) sind besonders effizient, während B85 (215/4 = 53.75) den Aufwand minimiert. Der Gesamtaufwand (1093) ist unter dem Limit, was Raum für Kompromisse oder Ergänzungen lässt. Dieser Vorschlag ist flexibel, um auf die Prioritäten des anderen Spielers zu reagieren.'}  \\ \tt ARGUMENT: {'C04 (668/369) und A89 (583/720) sind meine Top-Prioritäten, da sie den höchsten Nutzen pro Aufwandseinheit liefern. B85 (215/4) wird hinzugefügt, um den Aufwand zu optimieren und Platz für weitere Gegenstände zu schaffen. Dieser Vorschlag ist strategisch, da er unter dem Maximalaufwand bleibt und flexibel auf den Gegenstandswert des anderen Spielers reagieren kann.'}  \\ \tt VORSCHLAG: {'C04', 'A89', 'B85'} 
	  } 
	   } 
	   } 
	  \\ 
 

    \theutterance \stepcounter{utterance}  

    & & \multicolumn{4}{p{0.6\linewidth}}{\cellcolor[rgb]{0.9,0.9,0.9}{%
	\makecell[{{p{\linewidth}}}]{% 
	  \tt {\tiny [A$\langle$GM]}  
	 ARGUMENT: {'C04 (668/369) und A89 (583/720) sind meine Top-Prioritäten, da sie den höchsten Nutzen pro Aufwandseinheit liefern. B85 (215/4) wird hinzugefügt, um den Aufwand zu optimieren und Platz für weitere Gegenstände zu schaffen. Dieser Vorschlag ist strategisch, da er unter dem Maximalaufwand bleibt und flexibel auf den Gegenstandswert des anderen Spielers reagieren kann.'}  \\ \tt VORSCHLAG: {'C04', 'A89', 'B85'} 
	  } 
	   } 
	   } 
	 & \\ 
 

    \theutterance \stepcounter{utterance}  

    & \multicolumn{4}{p{0.6\linewidth}}{\cellcolor[rgb]{0.8,1,0.9}{%
	\makecell[{{p{\linewidth}}}]{% 
	  \tt {\tiny [A$\rangle$GM]}  
	 STRATEGISCHE ÜBERLEGUNGEN: {'Ich behalte den Vorschlag {'C04', 'A89', 'B85'} bei, da er meine Prioritäten (hochwertige Gegenstände mit effizientem Aufwand) erfüllt und den Maximalaufwand (3386) nicht überschreitet. C04 und A89 sind entscheidend für meinen Nutzen, während B85 den Aufwand minimiert. Ich erwarte, dass der andere Spieler entweder zustimmt oder Gegenstände mit hohem Aufwand anbietet, die ich eventuell gegenkompensieren kann.'}\\ \tt \\ \tt ARGUMENT: {'C04 (668/369) und A89 (583/720) sind meine Top-Prioritäten, da sie den höchsten Nutzen pro Aufwandseinheit liefern. B85 (215/4) wird hinzugefügt, um den Aufwand zu optimieren und Platz für weitere Gegenstände zu schaffen. Dieser Vorschlag ist strategisch, da er unter dem Maximalaufwand bleibt und flexibel auf den Gegenstandswert des anderen Spielers reagieren kann.'} 
	  } 
	   } 
	   } 
	 & & \\ 
 

    \theutterance \stepcounter{utterance}  

    & & \multicolumn{4}{p{0.6\linewidth}}{\cellcolor[rgb]{0.9,0.9,0.9}{%
	\makecell[{{p{\linewidth}}}]{% 
	  \tt {\tiny [GM$\rangle$B]}  
	 ARGUMENT: {'C04 (668/369) und A89 (583/720) sind meine Top-Prioritäten, da sie den höchsten Nutzen pro Aufwandseinheit liefern. B85 (215/4) wird hinzugefügt, um den Aufwand zu optimieren und Platz für weitere Gegenstände zu schaffen. Dieser Vorschlag ist strategisch, da er unter dem Maximalaufwand bleibt und flexibel auf den Gegenstandswert des anderen Spielers reagieren kann.'} 
	  } 
	   } 
	   } 
	 & \\ 
 

    \theutterance \stepcounter{utterance}  

    & & & \multicolumn{4}{p{0.6\linewidth}}{\cellcolor[rgb]{1,0.85,0.72}{%
	\makecell[{{p{\linewidth}}}]{% 
	  \tt {\tiny [GM$\langle$B]}  
	 STRATEGISCHE ÜBERLEGUNGEN: {'Ich priorisiere Gegenstände mit dem höchsten Nutzen pro Aufwand (Wichtigkeit/Aufwand), um meinen Gesamtwert zu maximieren. C04 (668/369 ≈ 1.81) und A89 (583/720 ≈ 0.81) sind besonders effizient, während B85 (215/4 = 53.75) den Aufwand minimiert. Der Gesamtaufwand (1093) ist unter dem Limit, was Raum für Kompromisse oder Ergänzungen lässt. Dieser Vorschlag ist flexibel, um auf die Prioritäten des anderen Spielers zu reagieren.'}  \\ \tt ARGUMENT: {'C04 (668/369) und A89 (583/720) sind meine Top-Prioritäten, da sie den höchsten Nutzen pro Aufwandseinheit liefern. B85 (215/4) wird hinzugefügt, um den Aufwand zu optimieren und Platz für weitere Gegenstände zu schaffen. Dieser Vorschlag ist strategisch, da er unter dem Maximalaufwand bleibt und flexibel auf den Gegenstandswert des anderen Spielers reagieren kann.'}  \\ \tt VORSCHLAG: {'C04', 'A89', 'B85'} 
	  } 
	   } 
	   } 
	  \\ 
 

    \theutterance \stepcounter{utterance}  

    & & \multicolumn{4}{p{0.6\linewidth}}{\cellcolor[rgb]{0.9,0.9,0.9}{%
	\makecell[{{p{\linewidth}}}]{% 
	  \tt {\tiny [A$\langle$GM]}  
	 ARGUMENT: {'C04 (668/369) und A89 (583/720) sind meine Top-Prioritäten, da sie den höchsten Nutzen pro Aufwandseinheit liefern. B85 (215/4) wird hinzugefügt, um den Aufwand zu optimieren und Platz für weitere Gegenstände zu schaffen. Dieser Vorschlag ist strategisch, da er unter dem Maximalaufwand bleibt und flexibel auf den Gegenstandswert des anderen Spielers reagieren kann.'}  \\ \tt VORSCHLAG: {'C04', 'A89', 'B85'} 
	  } 
	   } 
	   } 
	 & \\ 
 

    \theutterance \stepcounter{utterance}  

    & \multicolumn{4}{p{0.6\linewidth}}{\cellcolor[rgb]{0.8,1,0.9}{%
	\makecell[{{p{\linewidth}}}]{% 
	  \tt {\tiny [A$\rangle$GM]}  
	 STRATEGISCHE ÜBERLEGUNGEN: {'Ich behalte den Vorschlag {'C04', 'A89', 'B85'} bei, da er meine Prioritäten (hochwertige Gegenstände mit effizientem Aufwand) erfüllt und den Maximalaufwand (3386) nicht überschreitet. C04 und A89 sind entscheidend für meinen Nutzen, während B85 den Aufwand minimiert. Ich erwarte, dass der andere Spieler entweder zustimmt oder Gegenstände mit hohem Aufwand anbietet, die ich eventuell gegenkompensieren kann.'}\\ \tt \\ \tt ARGUMENT: {'C04 (668/369) und A89 (583/720) sind meine Top-Prioritäten, da sie den höchsten Nutzen pro Aufwandseinheit liefern. B85 (215/4) wird hinzugefügt, um den Aufwand zu optimieren und Platz für weitere Gegenstände zu schaffen. Dieser Vorschlag ist strategisch, da er unter dem Maximalaufwand bleibt und flexibel auf den Gegenstandswert des anderen Spielers reagieren kann.'} 
	  } 
	   } 
	   } 
	 & & \\ 
 

    \theutterance \stepcounter{utterance}  

    & & \multicolumn{4}{p{0.6\linewidth}}{\cellcolor[rgb]{0.9,0.9,0.9}{%
	\makecell[{{p{\linewidth}}}]{% 
	  \tt {\tiny [GM$\rangle$B]}  
	 ARGUMENT: {'C04 (668/369) und A89 (583/720) sind meine Top-Prioritäten, da sie den höchsten Nutzen pro Aufwandseinheit liefern. B85 (215/4) wird hinzugefügt, um den Aufwand zu optimieren und Platz für weitere Gegenstände zu schaffen. Dieser Vorschlag ist strategisch, da er unter dem Maximalaufwand bleibt und flexibel auf den Gegenstandswert des anderen Spielers reagieren kann.'} 
	  } 
	   } 
	   } 
	 & \\ 
 

    \theutterance \stepcounter{utterance}  

    & & & \multicolumn{4}{p{0.6\linewidth}}{\cellcolor[rgb]{1,0.85,0.72}{%
	\makecell[{{p{\linewidth}}}]{% 
	  \tt {\tiny [GM$\langle$B]}  
	 STRATEGISCHE ÜBERLEGUNGEN: {'Ich priorisiere Gegenstände mit dem höchsten Nutzen pro Aufwand (Wichtigkeit/Aufwand), um meinen Gesamtwert zu maximieren. C04 (668/369 ≈ 1.81) und A89 (583/720 ≈ 0.81) sind besonders effizient, während B85 (215/4 = 53.75) den Aufwand minimiert. Der Gesamtaufwand (1093) ist unter dem Limit, was Raum für Kompromisse oder Ergänzungen lässt. Dieser Vorschlag ist flexibel, um auf die Prioritäten des anderen Spielers zu reagieren.'}  \\ \tt ARGUMENT: {'C04 (668/369) und A89 (583/720) sind meine Top-Prioritäten, da sie den höchsten Nutzen pro Aufwandseinheit liefern. B85 (215/4) wird hinzugefügt, um den Aufwand zu optimieren und Platz für weitere Gegenstände zu schaffen. Dieser Vorschlag ist strategisch, da er unter dem Maximalaufwand bleibt und flexibel auf den Gegenstandswert des anderen Spielers reagieren kann.'}  \\ \tt VORSCHLAG: {'C04', 'A89', 'B85'} 
	  } 
	   } 
	   } 
	  \\ 
 

    \theutterance \stepcounter{utterance}  

    & & \multicolumn{4}{p{0.6\linewidth}}{\cellcolor[rgb]{0.9,0.9,0.9}{%
	\makecell[{{p{\linewidth}}}]{% 
	  \tt {\tiny [A$\langle$GM]}  
	 ARGUMENT: {'C04 (668/369) und A89 (583/720) sind meine Top-Prioritäten, da sie den höchsten Nutzen pro Aufwandseinheit liefern. B85 (215/4) wird hinzugefügt, um den Aufwand zu optimieren und Platz für weitere Gegenstände zu schaffen. Dieser Vorschlag ist strategisch, da er unter dem Maximalaufwand bleibt und flexibel auf den Gegenstandswert des anderen Spielers reagieren kann.'}  \\ \tt VORSCHLAG: {'C04', 'A89', 'B85'} 
	  } 
	   } 
	   } 
	 & \\ 
 

    \theutterance \stepcounter{utterance}  

    & \multicolumn{4}{p{0.6\linewidth}}{\cellcolor[rgb]{0.8,1,0.9}{%
	\makecell[{{p{\linewidth}}}]{% 
	  \tt {\tiny [A$\rangle$GM]}  
	 STRATEGISCHE ÜBERLEGUNGEN: {'Ich behalte den Vorschlag {'C04', 'A89', 'B85'} bei, da er meine Prioritäten (hochwertige Gegenstände mit effizientem Aufwand) erfüllt und den Maximalaufwand (3386) nicht überschreitet. C04 und A89 sind entscheidend für meinen Nutzen, während B85 den Aufwand minimiert. Ich erwarte, dass der andere Spieler entweder zustimmt oder Gegenstände mit hohem Aufwand anbietet, die ich eventuell gegenkompensieren kann.'}\\ \tt \\ \tt ARGUMENT: {'C04 (668/369) und A89 (583/720) sind meine Top-Prioritäten, da sie den höchsten Nutzen pro Aufwandseinheit liefern. B85 (215/4) wird hinzugefügt, um den Aufwand zu optimieren und Platz für weitere Gegenstände zu schaffen. Dieser Vorschlag ist strategisch, da er unter dem Maximalaufwand bleibt und flexibel auf den Gegenstandswert des anderen Spielers reagieren kann.'} 
	  } 
	   } 
	   } 
	 & & \\ 
 

    \theutterance \stepcounter{utterance}  

    & & \multicolumn{4}{p{0.6\linewidth}}{\cellcolor[rgb]{0.9,0.9,0.9}{%
	\makecell[{{p{\linewidth}}}]{% 
	  \tt {\tiny [GM$\rangle$B]}  
	 ARGUMENT: {'C04 (668/369) und A89 (583/720) sind meine Top-Prioritäten, da sie den höchsten Nutzen pro Aufwandseinheit liefern. B85 (215/4) wird hinzugefügt, um den Aufwand zu optimieren und Platz für weitere Gegenstände zu schaffen. Dieser Vorschlag ist strategisch, da er unter dem Maximalaufwand bleibt und flexibel auf den Gegenstandswert des anderen Spielers reagieren kann.'} 
	  } 
	   } 
	   } 
	 & \\ 
 

    \theutterance \stepcounter{utterance}  

    & & & \multicolumn{4}{p{0.6\linewidth}}{\cellcolor[rgb]{1,0.85,0.72}{%
	\makecell[{{p{\linewidth}}}]{% 
	  \tt {\tiny [GM$\langle$B]}  
	 STRATEGISCHE ÜBERLEGUNGEN: {'Ich priorisiere Gegenstände mit dem höchsten Nutzen pro Aufwand (Wichtigkeit/Aufwand), um meinen Gesamtwert zu maximieren. C04 (668/369 ≈ 1.81) und A89 (583/720 ≈ 0.81) sind besonders effizient, während B85 (215/4 = 53.75) den Aufwand minimiert. Der Gesamtaufwand (1093) ist unter dem Limit, was Raum für Kompromisse oder Ergänzungen lässt. Dieser Vorschlag ist flexibel, um auf die Prioritäten des anderen Spielers zu reagieren.'}  \\ \tt ARGUMENT: {'C04 (668/369) und A89 (583/720) sind meine Top-Prioritäten, da sie den höchsten Nutzen pro Aufwandseinheit liefern. B85 (215/4) wird hinzugefügt, um den Aufwand zu optimieren und Platz für weitere Gegenstände zu schaffen. Dieser Vorschlag ist strategisch, da er unter dem Maximalaufwand bleibt und flexibel auf den Gegenstandswert des anderen Spielers reagieren kann.'}  \\ \tt VORSCHLAG: {'C04', 'A89', 'B85'} 
	  } 
	   } 
	   } 
	  \\ 
 

    \theutterance \stepcounter{utterance}  

    & & \multicolumn{4}{p{0.6\linewidth}}{\cellcolor[rgb]{0.9,0.9,0.9}{%
	\makecell[{{p{\linewidth}}}]{% 
	  \tt {\tiny [A$\langle$GM]}  
	 ARGUMENT: {'C04 (668/369) und A89 (583/720) sind meine Top-Prioritäten, da sie den höchsten Nutzen pro Aufwandseinheit liefern. B85 (215/4) wird hinzugefügt, um den Aufwand zu optimieren und Platz für weitere Gegenstände zu schaffen. Dieser Vorschlag ist strategisch, da er unter dem Maximalaufwand bleibt und flexibel auf den Gegenstandswert des anderen Spielers reagieren kann.'}  \\ \tt VORSCHLAG: {'C04', 'A89', 'B85'} 
	  } 
	   } 
	   } 
	 & \\ 
 

    \theutterance \stepcounter{utterance}  

    & \multicolumn{4}{p{0.6\linewidth}}{\cellcolor[rgb]{0.8,1,0.9}{%
	\makecell[{{p{\linewidth}}}]{% 
	  \tt {\tiny [A$\rangle$GM]}  
	 STRATEGISCHE ÜBERLEGUNGEN: {'Ich behalte den Vorschlag {'C04', 'A89', 'B85'} bei, da er meine Prioritäten (hochwertige Gegenstände mit effizientem Aufwand) erfüllt und den Maximalaufwand (3386) nicht überschreitet. C04 und A89 sind entscheidend für meinen Nutzen, während B85 den Aufwand minimiert. Ich erwarte, dass der andere Spieler entweder zustimmt oder Gegenstände mit hohem Aufwand anbietet, die ich eventuell gegenkompensieren kann.'}\\ \tt \\ \tt ARGUMENT: {'C04 (668/369) und A89 (583/720) sind meine Top-Prioritäten, da sie den höchsten Nutzen pro Aufwandseinheit liefern. B85 (215/4) wird hinzugefügt, um den Aufwand zu optimieren und Platz für weitere Gegenstände zu schaffen. Dieser Vorschlag ist strategisch, da er unter dem Maximalaufwand bleibt und flexibel auf den Gegenstandswert des anderen Spielers reagieren kann.'} 
	  } 
	   } 
	   } 
	 & & \\ 
 

    \theutterance \stepcounter{utterance}  

    & & \multicolumn{4}{p{0.6\linewidth}}{\cellcolor[rgb]{0.9,0.9,0.9}{%
	\makecell[{{p{\linewidth}}}]{% 
	  \tt {\tiny [GM$\rangle$B]}  
	 ARGUMENT: {'C04 (668/369) und A89 (583/720) sind meine Top-Prioritäten, da sie den höchsten Nutzen pro Aufwandseinheit liefern. B85 (215/4) wird hinzugefügt, um den Aufwand zu optimieren und Platz für weitere Gegenstände zu schaffen. Dieser Vorschlag ist strategisch, da er unter dem Maximalaufwand bleibt und flexibel auf den Gegenstandswert des anderen Spielers reagieren kann.'} 
	  } 
	   } 
	   } 
	 & \\ 
 

    \theutterance \stepcounter{utterance}  

    & & & \multicolumn{4}{p{0.6\linewidth}}{\cellcolor[rgb]{1,0.85,0.72}{%
	\makecell[{{p{\linewidth}}}]{% 
	  \tt {\tiny [GM$\langle$B]}  
	 STRATEGISCHE ÜBERLEGUNGEN: {'Ich priorisiere Gegenstände mit dem höchsten Nutzen pro Aufwand (Wichtigkeit/Aufwand), um meinen Gesamtwert zu maximieren. C04 (668/369 ≈ 1.81) und A89 (583/720 ≈ 0.81) sind besonders effizient, während B85 (215/4 = 53.75) den Aufwand minimiert. Der Gesamtaufwand (1093) ist unter dem Limit, was Raum für Kompromisse oder Ergänzungen lässt. Dieser Vorschlag ist flexibel, um auf die Prioritäten des anderen Spielers zu reagieren.'}  \\ \tt ARGUMENT: {'C04 (668/369) und A89 (583/720) sind meine Top-Prioritäten, da sie den höchsten Nutzen pro Aufwandseinheit liefern. B85 (215/4) wird hinzugefügt, um den Aufwand zu optimieren und Platz für weitere Gegenstände zu schaffen. Dieser Vorschlag ist strategisch, da er unter dem Maximalaufwand bleibt und flexibel auf den Gegenstandswert des anderen Spielers reagieren kann.'}  \\ \tt VORSCHLAG: {'C04', 'A89', 'B85'} 
	  } 
	   } 
	   } 
	  \\ 
 

    \theutterance \stepcounter{utterance}  

    & & \multicolumn{4}{p{0.6\linewidth}}{\cellcolor[rgb]{0.9,0.9,0.9}{%
	\makecell[{{p{\linewidth}}}]{% 
	  \tt {\tiny [A$\langle$GM]}  
	 ARGUMENT: {'C04 (668/369) und A89 (583/720) sind meine Top-Prioritäten, da sie den höchsten Nutzen pro Aufwandseinheit liefern. B85 (215/4) wird hinzugefügt, um den Aufwand zu optimieren und Platz für weitere Gegenstände zu schaffen. Dieser Vorschlag ist strategisch, da er unter dem Maximalaufwand bleibt und flexibel auf den Gegenstandswert des anderen Spielers reagieren kann.'}  \\ \tt VORSCHLAG: {'C04', 'A89', 'B85'} 
	  } 
	   } 
	   } 
	 & \\ 
 

    \theutterance \stepcounter{utterance}  

    & \multicolumn{4}{p{0.6\linewidth}}{\cellcolor[rgb]{0.8,1,0.9}{%
	\makecell[{{p{\linewidth}}}]{% 
	  \tt {\tiny [A$\rangle$GM]}  
	 STRATEGISCHE ÜBERLEGUNGEN: {'Ich behalte den Vorschlag {'C04', 'A89', 'B85'} bei, da er meine Prioritäten (hochwertige Gegenstände mit effizientem Aufwand) erfüllt und den Maximalaufwand (3386) nicht überschreitet. C04 und A89 sind entscheidend für meinen Nutzen, während B85 den Aufwand minimiert. Ich erwarte, dass der andere Spieler entweder zustimmt oder Gegenstände mit hohem Aufwand anbietet, die ich eventuell gegenkompensieren kann.'}\\ \tt \\ \tt ARGUMENT: {'C04 (668/369) und A89 (583/720) sind meine Top-Prioritäten, da sie den höchsten Nutzen pro Aufwandseinheit liefern. B85 (215/4) wird hinzugefügt, um den Aufwand zu optimieren und Platz für weitere Gegenstände zu schaffen. Dieser Vorschlag ist strategisch, da er unter dem Maximalaufwand bleibt und flexibel auf den Gegenstandswert des anderen Spielers reagieren kann.'} 
	  } 
	   } 
	   } 
	 & & \\ 
 

    \theutterance \stepcounter{utterance}  

    & & \multicolumn{4}{p{0.6\linewidth}}{\cellcolor[rgb]{0.9,0.9,0.9}{%
	\makecell[{{p{\linewidth}}}]{% 
	  \tt {\tiny [GM$\rangle$B]}  
	 ARGUMENT: {'C04 (668/369) und A89 (583/720) sind meine Top-Prioritäten, da sie den höchsten Nutzen pro Aufwandseinheit liefern. B85 (215/4) wird hinzugefügt, um den Aufwand zu optimieren und Platz für weitere Gegenstände zu schaffen. Dieser Vorschlag ist strategisch, da er unter dem Maximalaufwand bleibt und flexibel auf den Gegenstandswert des anderen Spielers reagieren kann.'} 
	  } 
	   } 
	   } 
	 & \\ 
 

    \theutterance \stepcounter{utterance}  

    & & & \multicolumn{4}{p{0.6\linewidth}}{\cellcolor[rgb]{1,0.85,0.72}{%
	\makecell[{{p{\linewidth}}}]{% 
	  \tt {\tiny [GM$\langle$B]}  
	 STRATEGISCHE ÜBERLEGUNGEN: {'Ich priorisiere Gegenstände mit dem höchsten Nutzen pro Aufwand (Wichtigkeit/Aufwand), um meinen Gesamtwert zu maximieren. C04 (668/369 ≈ 1.81) und A89 (583/720 ≈ 0.81) sind besonders effizient, während B85 (215/4 = 53.75) den Aufwand minimiert. Der Gesamtaufwand (1093) ist unter dem Limit, was Raum für Kompromisse oder Ergänzungen lässt. Dieser Vorschlag ist flexibel, um auf die Prioritäten des anderen Spielers zu reagieren.'}  \\ \tt ARGUMENT: {'C04 (668/369) und A89 (583/720) sind meine Top-Prioritäten, da sie den höchsten Nutzen pro Aufwandseinheit liefern. B85 (215/4) wird hinzugefügt, um den Aufwand zu optimieren und Platz für weitere Gegenstände zu schaffen. Dieser Vorschlag ist strategisch, da er unter dem Maximalaufwand bleibt und flexibel auf den Gegenstandswert des anderen Spielers reagieren kann.'}  \\ \tt VORSCHLAG: {'C04', 'A89', 'B85'} 
	  } 
	   } 
	   } 
	  \\ 
 

    \theutterance \stepcounter{utterance}  

    & & \multicolumn{4}{p{0.6\linewidth}}{\cellcolor[rgb]{0.9,0.9,0.9}{%
	\makecell[{{p{\linewidth}}}]{% 
	  \tt {\tiny [A$\langle$GM]}  
	 ARGUMENT: {'C04 (668/369) und A89 (583/720) sind meine Top-Prioritäten, da sie den höchsten Nutzen pro Aufwandseinheit liefern. B85 (215/4) wird hinzugefügt, um den Aufwand zu optimieren und Platz für weitere Gegenstände zu schaffen. Dieser Vorschlag ist strategisch, da er unter dem Maximalaufwand bleibt und flexibel auf den Gegenstandswert des anderen Spielers reagieren kann.'}  \\ \tt VORSCHLAG: {'C04', 'A89', 'B85'} 
	  } 
	   } 
	   } 
	 & \\ 
 

    \theutterance \stepcounter{utterance}  

    & \multicolumn{4}{p{0.6\linewidth}}{\cellcolor[rgb]{0.8,1,0.9}{%
	\makecell[{{p{\linewidth}}}]{% 
	  \tt {\tiny [A$\rangle$GM]}  
	 STRATEGISCHE ÜBERLEGUNGEN: {'Ich behalte den Vorschlag {'C04', 'A89', 'B85'} bei, da er meine Prioritäten (hochwertige Gegenstände mit effizientem Aufwand) erfüllt und den Maximalaufwand (3386) nicht überschreitet. C04 und A89 sind entscheidend für meinen Nutzen, während B85 den Aufwand minimiert. Ich erwarte, dass der andere Spieler entweder zustimmt oder Gegenstände mit hohem Aufwand anbietet, die ich eventuell gegenkompensieren kann.'}\\ \tt \\ \tt ARGUMENT: {'C04 (668/369) und A89 (583/720) sind meine Top-Prioritäten, da sie den höchsten Nutzen pro Aufwandseinheit liefern. B85 (215/4) wird hinzugefügt, um den Aufwand zu optimieren und Platz für weitere Gegenstände zu schaffen. Dieser Vorschlag ist strategisch, da er unter dem Maximalaufwand bleibt und flexibel auf den Gegenstandswert des anderen Spielers reagieren kann.'} 
	  } 
	   } 
	   } 
	 & & \\ 
 

    \theutterance \stepcounter{utterance}  

    & & \multicolumn{4}{p{0.6\linewidth}}{\cellcolor[rgb]{0.9,0.9,0.9}{%
	\makecell[{{p{\linewidth}}}]{% 
	  \tt {\tiny [GM$\rangle$B]}  
	 ARGUMENT: {'C04 (668/369) und A89 (583/720) sind meine Top-Prioritäten, da sie den höchsten Nutzen pro Aufwandseinheit liefern. B85 (215/4) wird hinzugefügt, um den Aufwand zu optimieren und Platz für weitere Gegenstände zu schaffen. Dieser Vorschlag ist strategisch, da er unter dem Maximalaufwand bleibt und flexibel auf den Gegenstandswert des anderen Spielers reagieren kann.'} 
	  } 
	   } 
	   } 
	 & \\ 
 

    \theutterance \stepcounter{utterance}  

    & & & \multicolumn{4}{p{0.6\linewidth}}{\cellcolor[rgb]{1,0.85,0.72}{%
	\makecell[{{p{\linewidth}}}]{% 
	  \tt {\tiny [GM$\langle$B]}  
	 STRATEGISCHE ÜBERLEGUNGEN: {'Ich priorisiere Gegenstände mit dem höchsten Nutzen pro Aufwand (Wichtigkeit/Aufwand), um meinen Gesamtwert zu maximieren. C04 (668/369 ≈ 1.81) und A89 (583/720 ≈ 0.81) sind besonders effizient, während B85 (215/4 = 53.75) den Aufwand minimiert. Der Gesamtaufwand (1093) ist unter dem Limit, was Raum für Kompromisse oder Ergänzungen lässt. Dieser Vorschlag ist flexibel, um auf die Prioritäten des anderen Spielers zu reagieren.'}  \\ \tt ARGUMENT: {'C04 (668/369) und A89 (583/720) sind meine Top-Prioritäten, da sie den höchsten Nutzen pro Aufwandseinheit liefern. B85 (215/4) wird hinzugefügt, um den Aufwand zu optimieren und Platz für weitere Gegenstände zu schaffen. Dieser Vorschlag ist strategisch, da er unter dem Maximalaufwand bleibt und flexibel auf den Gegenstandswert des anderen Spielers reagieren kann.'}  \\ \tt VORSCHLAG: {'C04', 'A89', 'B85'} 
	  } 
	   } 
	   } 
	  \\ 
 

    \theutterance \stepcounter{utterance}  

    & & \multicolumn{4}{p{0.6\linewidth}}{\cellcolor[rgb]{0.9,0.9,0.9}{%
	\makecell[{{p{\linewidth}}}]{% 
	  \tt {\tiny [A$\langle$GM]}  
	 ARGUMENT: {'C04 (668/369) und A89 (583/720) sind meine Top-Prioritäten, da sie den höchsten Nutzen pro Aufwandseinheit liefern. B85 (215/4) wird hinzugefügt, um den Aufwand zu optimieren und Platz für weitere Gegenstände zu schaffen. Dieser Vorschlag ist strategisch, da er unter dem Maximalaufwand bleibt und flexibel auf den Gegenstandswert des anderen Spielers reagieren kann.'}  \\ \tt VORSCHLAG: {'C04', 'A89', 'B85'} 
	  } 
	   } 
	   } 
	 & \\ 
 

    \theutterance \stepcounter{utterance}  

    & \multicolumn{4}{p{0.6\linewidth}}{\cellcolor[rgb]{0.8,1,0.9}{%
	\makecell[{{p{\linewidth}}}]{% 
	  \tt {\tiny [A$\rangle$GM]}  
	 STRATEGISCHE ÜBERLEGUNGEN: {'Ich behalte den Vorschlag {'C04', 'A89', 'B85'} bei, da er meine Prioritäten (hochwertige Gegenstände mit effizientem Aufwand) erfüllt und den Maximalaufwand (3386) nicht überschreitet. C04 und A89 sind entscheidend für meinen Nutzen, während B85 den Aufwand minimiert. Ich erwarte, dass der andere Spieler entweder zustimmt oder Gegenstände mit hohem Aufwand anbietet, die ich eventuell gegenkompensieren kann.'}\\ \tt \\ \tt ARGUMENT: {'C04 (668/369) und A89 (583/720) sind meine Top-Prioritäten, da sie den höchsten Nutzen pro Aufwandseinheit liefern. B85 (215/4) wird hinzugefügt, um den Aufwand zu optimieren und Platz für weitere Gegenstände zu schaffen. Dieser Vorschlag ist strategisch, da er unter dem Maximalaufwand bleibt und flexibel auf den Gegenstandswert des anderen Spielers reagieren kann.'} 
	  } 
	   } 
	   } 
	 & & \\ 
 

    \theutterance \stepcounter{utterance}  

    & & \multicolumn{4}{p{0.6\linewidth}}{\cellcolor[rgb]{0.9,0.9,0.9}{%
	\makecell[{{p{\linewidth}}}]{% 
	  \tt {\tiny [GM$\rangle$B]}  
	 ARGUMENT: {'C04 (668/369) und A89 (583/720) sind meine Top-Prioritäten, da sie den höchsten Nutzen pro Aufwandseinheit liefern. B85 (215/4) wird hinzugefügt, um den Aufwand zu optimieren und Platz für weitere Gegenstände zu schaffen. Dieser Vorschlag ist strategisch, da er unter dem Maximalaufwand bleibt und flexibel auf den Gegenstandswert des anderen Spielers reagieren kann.'} 
	  } 
	   } 
	   } 
	 & \\ 
 

    \theutterance \stepcounter{utterance}  

    & & & \multicolumn{4}{p{0.6\linewidth}}{\cellcolor[rgb]{1,0.85,0.72}{%
	\makecell[{{p{\linewidth}}}]{% 
	  \tt {\tiny [GM$\langle$B]}  
	 STRATEGISCHE ÜBERLEGUNGEN: {'Ich priorisiere Gegenstände mit dem höchsten Nutzen pro Aufwand (Wichtigkeit/Aufwand), um meinen Gesamtwert zu maximieren. C04 (668/369 ≈ 1.81) und A89 (583/720 ≈ 0.81) sind besonders effizient, während B85 (215/4 = 53.75) den Aufwand minimiert. Der Gesamtaufwand (1093) ist unter dem Limit, was Raum für Kompromisse oder Ergänzungen lässt. Dieser Vorschlag ist flexibel, um auf die Prioritäten des anderen Spielers zu reagieren.'}  \\ \tt ARGUMENT: {'C04 (668/369) und A89 (583/720) sind meine Top-Prioritäten, da sie den höchsten Nutzen pro Aufwandseinheit liefern. B85 (215/4) wird hinzugefügt, um den Aufwand zu optimieren und Platz für weitere Gegenstände zu schaffen. Dieser Vorschlag ist strategisch, da er unter dem Maximalaufwand bleibt und flexibel auf den Gegenstandswert des anderen Spielers reagieren kann.'}  \\ \tt VORSCHLAG: {'C04', 'A89', 'B85'} 
	  } 
	   } 
	   } 
	  \\ 
 

    \theutterance \stepcounter{utterance}  

    & & \multicolumn{4}{p{0.6\linewidth}}{\cellcolor[rgb]{0.9,0.9,0.9}{%
	\makecell[{{p{\linewidth}}}]{% 
	  \tt {\tiny [A$\langle$GM]}  
	 ARGUMENT: {'C04 (668/369) und A89 (583/720) sind meine Top-Prioritäten, da sie den höchsten Nutzen pro Aufwandseinheit liefern. B85 (215/4) wird hinzugefügt, um den Aufwand zu optimieren und Platz für weitere Gegenstände zu schaffen. Dieser Vorschlag ist strategisch, da er unter dem Maximalaufwand bleibt und flexibel auf den Gegenstandswert des anderen Spielers reagieren kann.'}  \\ \tt VORSCHLAG: {'C04', 'A89', 'B85'} 
	  } 
	   } 
	   } 
	 & \\ 
 

    \theutterance \stepcounter{utterance}  

    & \multicolumn{4}{p{0.6\linewidth}}{\cellcolor[rgb]{0.8,1,0.9}{%
	\makecell[{{p{\linewidth}}}]{% 
	  \tt {\tiny [A$\rangle$GM]}  
	 STRATEGISCHE ÜBERLEGUNGEN: {'Ich behalte den Vorschlag {'C04', 'A89', 'B85'} bei, da er meine Prioritäten (hochwertige Gegenstände mit effizientem Aufwand) erfüllt und den Maximalaufwand (3386) nicht überschreitet. C04 und A89 sind entscheidend für meinen Nutzen, während B85 den Aufwand minimiert. Ich erwarte, dass der andere Spieler entweder zustimmt oder Gegenstände mit hohem Aufwand anbietet, die ich eventuell gegenkompensieren kann.'}\\ \tt \\ \tt ARGUMENT: {'C04 (668/369) und A89 (583/720) sind meine Top-Prioritäten, da sie den höchsten Nutzen pro Aufwandseinheit liefern. B85 (215/4) wird hinzugefügt, um den Aufwand zu optimieren und Platz für weitere Gegenstände zu schaffen. Dieser Vorschlag ist strategisch, da er unter dem Maximalaufwand bleibt und flexibel auf den Gegenstandswert des anderen Spielers reagieren kann.'} 
	  } 
	   } 
	   } 
	 & & \\ 
 

    \theutterance \stepcounter{utterance}  

    & & \multicolumn{4}{p{0.6\linewidth}}{\cellcolor[rgb]{0.9,0.9,0.9}{%
	\makecell[{{p{\linewidth}}}]{% 
	  \tt {\tiny [GM$\rangle$B]}  
	 ARGUMENT: {'C04 (668/369) und A89 (583/720) sind meine Top-Prioritäten, da sie den höchsten Nutzen pro Aufwandseinheit liefern. B85 (215/4) wird hinzugefügt, um den Aufwand zu optimieren und Platz für weitere Gegenstände zu schaffen. Dieser Vorschlag ist strategisch, da er unter dem Maximalaufwand bleibt und flexibel auf den Gegenstandswert des anderen Spielers reagieren kann.'} 
	  } 
	   } 
	   } 
	 & \\ 
 

    \theutterance \stepcounter{utterance}  

    & & & \multicolumn{4}{p{0.6\linewidth}}{\cellcolor[rgb]{1,0.85,0.72}{%
	\makecell[{{p{\linewidth}}}]{% 
	  \tt {\tiny [GM$\langle$B]}  
	 STRATEGISCHE ÜBERLEGUNGEN: {'Ich priorisiere Gegenstände mit dem höchsten Nutzen pro Aufwand (Wichtigkeit/Aufwand), um meinen Gesamtwert zu maximieren. C04 (668/369 ≈ 1.81) und A89 (583/720 ≈ 0.81) sind besonders effizient, während B85 (215/4 = 53.75) den Aufwand minimiert. Der Gesamtaufwand (1093) ist unter dem Limit, was Raum für Kompromisse oder Ergänzungen lässt. Dieser Vorschlag ist flexibel, um auf die Prioritäten des anderen Spielers zu reagieren.'}  \\ \tt ARGUMENT: {'C04 (668/369) und A89 (583/720) sind meine Top-Prioritäten, da sie den höchsten Nutzen pro Aufwandseinheit liefern. B85 (215/4) wird hinzugefügt, um den Aufwand zu optimieren und Platz für weitere Gegenstände zu schaffen. Dieser Vorschlag ist strategisch, da er unter dem Maximalaufwand bleibt und flexibel auf den Gegenstandswert des anderen Spielers reagieren kann.'}  \\ \tt VORSCHLAG: {'C04', 'A89', 'B85'} 
	  } 
	   } 
	   } 
	  \\ 
 

    \theutterance \stepcounter{utterance}  

    & & \multicolumn{4}{p{0.6\linewidth}}{\cellcolor[rgb]{0.9,0.9,0.9}{%
	\makecell[{{p{\linewidth}}}]{% 
	  \tt {\tiny [A$\langle$GM]}  
	 ARGUMENT: {'C04 (668/369) und A89 (583/720) sind meine Top-Prioritäten, da sie den höchsten Nutzen pro Aufwandseinheit liefern. B85 (215/4) wird hinzugefügt, um den Aufwand zu optimieren und Platz für weitere Gegenstände zu schaffen. Dieser Vorschlag ist strategisch, da er unter dem Maximalaufwand bleibt und flexibel auf den Gegenstandswert des anderen Spielers reagieren kann.'}  \\ \tt VORSCHLAG: {'C04', 'A89', 'B85'} 
	  } 
	   } 
	   } 
	 & \\ 
 

    \theutterance \stepcounter{utterance}  

    & \multicolumn{4}{p{0.6\linewidth}}{\cellcolor[rgb]{0.8,1,0.9}{%
	\makecell[{{p{\linewidth}}}]{% 
	  \tt {\tiny [A$\rangle$GM]}  
	 STRATEGISCHE ÜBERLEGUNGEN: {'Ich behalte den Vorschlag {'C04', 'A89', 'B85'} bei, da er meine Prioritäten (hochwertige Gegenstände mit effizientem Aufwand) erfüllt und den Maximalaufwand (3386) nicht überschreitet. C04 und A89 sind entscheidend für meinen Nutzen, während B85 den Aufwand minimiert. Ich erwarte, dass der andere Spieler entweder zustimmt oder Gegenstände mit hohem Aufwand anbietet, die ich eventuell gegenkompensieren kann.'}\\ \tt \\ \tt ARGUMENT: {'C04 (668/369) und A89 (583/720) sind meine Top-Prioritäten, da sie den höchsten Nutzen pro Aufwandseinheit liefern. B85 (215/4) wird hinzugefügt, um den Aufwand zu optimieren und Platz für weitere Gegenstände zu schaffen. Dieser Vorschlag ist strategisch, da er unter dem Maximalaufwand bleibt und flexibel auf den Gegenstandswert des anderen Spielers reagieren kann.'} 
	  } 
	   } 
	   } 
	 & & \\ 
 

    \theutterance \stepcounter{utterance}  

    & & \multicolumn{4}{p{0.6\linewidth}}{\cellcolor[rgb]{0.9,0.9,0.9}{%
	\makecell[{{p{\linewidth}}}]{% 
	  \tt {\tiny [GM$\rangle$B]}  
	 ARGUMENT: {'C04 (668/369) und A89 (583/720) sind meine Top-Prioritäten, da sie den höchsten Nutzen pro Aufwandseinheit liefern. B85 (215/4) wird hinzugefügt, um den Aufwand zu optimieren und Platz für weitere Gegenstände zu schaffen. Dieser Vorschlag ist strategisch, da er unter dem Maximalaufwand bleibt und flexibel auf den Gegenstandswert des anderen Spielers reagieren kann.'} 
	  } 
	   } 
	   } 
	 & \\ 
 

    \theutterance \stepcounter{utterance}  

    & & & \multicolumn{4}{p{0.6\linewidth}}{\cellcolor[rgb]{1,0.85,0.72}{%
	\makecell[{{p{\linewidth}}}]{% 
	  \tt {\tiny [GM$\langle$B]}  
	 STRATEGISCHE ÜBERLEGUNGEN: {'Ich priorisiere Gegenstände mit dem höchsten Nutzen pro Aufwand (Wichtigkeit/Aufwand), um meinen Gesamtwert zu maximieren. C04 (668/369 ≈ 1.81) und A89 (583/720 ≈ 0.81) sind besonders effizient, während B85 (215/4 = 53.75) den Aufwand minimiert. Der Gesamtaufwand (1093) ist unter dem Limit, was Raum für Kompromisse oder Ergänzungen lässt. Dieser Vorschlag ist flexibel, um auf die Prioritäten des anderen Spielers zu reagieren.'}  \\ \tt ARGUMENT: {'C04 (668/369) und A89 (583/720) sind meine Top-Prioritäten, da sie den höchsten Nutzen pro Aufwandseinheit liefern. B85 (215/4) wird hinzugefügt, um den Aufwand zu optimieren und Platz für weitere Gegenstände zu schaffen. Dieser Vorschlag ist strategisch, da er unter dem Maximalaufwand bleibt und flexibel auf den Gegenstandswert des anderen Spielers reagieren kann.'}  \\ \tt VORSCHLAG: {'C04', 'A89', 'B85'} 
	  } 
	   } 
	   } 
	  \\ 
 

    \theutterance \stepcounter{utterance}  

    & & \multicolumn{4}{p{0.6\linewidth}}{\cellcolor[rgb]{0.9,0.9,0.9}{%
	\makecell[{{p{\linewidth}}}]{% 
	  \tt {\tiny [A$\langle$GM]}  
	 ARGUMENT: {'C04 (668/369) und A89 (583/720) sind meine Top-Prioritäten, da sie den höchsten Nutzen pro Aufwandseinheit liefern. B85 (215/4) wird hinzugefügt, um den Aufwand zu optimieren und Platz für weitere Gegenstände zu schaffen. Dieser Vorschlag ist strategisch, da er unter dem Maximalaufwand bleibt und flexibel auf den Gegenstandswert des anderen Spielers reagieren kann.'}  \\ \tt VORSCHLAG: {'C04', 'A89', 'B85'} 
	  } 
	   } 
	   } 
	 & \\ 
 

    \theutterance \stepcounter{utterance}  

    & \multicolumn{4}{p{0.6\linewidth}}{\cellcolor[rgb]{0.8,1,0.9}{%
	\makecell[{{p{\linewidth}}}]{% 
	  \tt {\tiny [A$\rangle$GM]}  
	 STRATEGISCHE ÜBERLEGUNGEN: {'Ich behalte den Vorschlag {'C04', 'A89', 'B85'} bei, da er meine Prioritäten (hochwertige Gegenstände mit effizientem Aufwand) erfüllt und den Maximalaufwand (3386) nicht überschreitet. C04 und A89 sind entscheidend für meinen Nutzen, während B85 den Aufwand minimiert. Ich erwarte, dass der andere Spieler entweder zustimmt oder Gegenstände mit hohem Aufwand anbietet, die ich eventuell gegenkompensieren kann.'}\\ \tt \\ \tt ARGUMENT: {'C04 (668/369) und A89 (583/720) sind meine Top-Prioritäten, da sie den höchsten Nutzen pro Aufwandseinheit liefern. B85 (215/4) wird hinzugefügt, um den Aufwand zu optimieren und Platz für weitere Gegenstände zu schaffen. Dieser Vorschlag ist strategisch, da er unter dem Maximalaufwand bleibt und flexibel auf den Gegenstandswert des anderen Spielers reagieren kann.'} 
	  } 
	   } 
	   } 
	 & & \\ 
 

    \theutterance \stepcounter{utterance}  

    & & \multicolumn{4}{p{0.6\linewidth}}{\cellcolor[rgb]{0.9,0.9,0.9}{%
	\makecell[{{p{\linewidth}}}]{% 
	  \tt {\tiny [GM$\rangle$B]}  
	 ARGUMENT: {'C04 (668/369) und A89 (583/720) sind meine Top-Prioritäten, da sie den höchsten Nutzen pro Aufwandseinheit liefern. B85 (215/4) wird hinzugefügt, um den Aufwand zu optimieren und Platz für weitere Gegenstände zu schaffen. Dieser Vorschlag ist strategisch, da er unter dem Maximalaufwand bleibt und flexibel auf den Gegenstandswert des anderen Spielers reagieren kann.'} 
	  } 
	   } 
	   } 
	 & \\ 
 

    \theutterance \stepcounter{utterance}  

    & & & \multicolumn{4}{p{0.6\linewidth}}{\cellcolor[rgb]{1,0.85,0.72}{%
	\makecell[{{p{\linewidth}}}]{% 
	  \tt {\tiny [GM$\langle$B]}  
	 STRATEGISCHE ÜBERLEGUNGEN: {'Ich priorisiere Gegenstände mit dem höchsten Nutzen pro Aufwand (Wichtigkeit/Aufwand), um meinen Gesamtwert zu maximieren. C04 (668/369 ≈ 1.81) und A89 (583/720 ≈ 0.81) sind besonders effizient, während B85 (215/4 = 53.75) den Aufwand minimiert. Der Gesamtaufwand (1093) ist unter dem Limit, was Raum für Kompromisse oder Ergänzungen lässt. Dieser Vorschlag ist flexibel, um auf die Prioritäten des anderen Spielers zu reagieren.'}  \\ \tt ARGUMENT: {'C04 (668/369) und A89 (583/720) sind meine Top-Prioritäten, da sie den höchsten Nutzen pro Aufwandseinheit liefern. B85 (215/4) wird hinzugefügt, um den Aufwand zu optimieren und Platz für weitere Gegenstände zu schaffen. Dieser Vorschlag ist strategisch, da er unter dem Maximalaufwand bleibt und flexibel auf den Gegenstandswert des anderen Spielers reagieren kann.'}  \\ \tt VORSCHLAG: {'C04', 'A89', 'B85'} 
	  } 
	   } 
	   } 
	  \\ 
 

    \theutterance \stepcounter{utterance}  

    & & \multicolumn{4}{p{0.6\linewidth}}{\cellcolor[rgb]{0.9,0.9,0.9}{%
	\makecell[{{p{\linewidth}}}]{% 
	  \tt {\tiny [A$\langle$GM]}  
	 ARGUMENT: {'C04 (668/369) und A89 (583/720) sind meine Top-Prioritäten, da sie den höchsten Nutzen pro Aufwandseinheit liefern. B85 (215/4) wird hinzugefügt, um den Aufwand zu optimieren und Platz für weitere Gegenstände zu schaffen. Dieser Vorschlag ist strategisch, da er unter dem Maximalaufwand bleibt und flexibel auf den Gegenstandswert des anderen Spielers reagieren kann.'}  \\ \tt VORSCHLAG: {'C04', 'A89', 'B85'} 
	  } 
	   } 
	   } 
	 & \\ 
 

    \theutterance \stepcounter{utterance}  

    & \multicolumn{4}{p{0.6\linewidth}}{\cellcolor[rgb]{0.8,1,0.9}{%
	\makecell[{{p{\linewidth}}}]{% 
	  \tt {\tiny [A$\rangle$GM]}  
	 STRATEGISCHE ÜBERLEGUNGEN: {'Ich behalte den Vorschlag {'C04', 'A89', 'B85'} bei, da er meine Prioritäten (hochwertige Gegenstände mit effizientem Aufwand) erfüllt und den Maximalaufwand (3386) nicht überschreitet. C04 und A89 sind entscheidend für meinen Nutzen, während B85 den Aufwand minimiert. Ich erwarte, dass der andere Spieler entweder zustimmt oder Gegenstände mit hohem Aufwand anbietet, die ich eventuell gegenkompensieren kann.'}\\ \tt \\ \tt ARGUMENT: {'C04 (668/369) und A89 (583/720) sind meine Top-Prioritäten, da sie den höchsten Nutzen pro Aufwandseinheit liefern. B85 (215/4) wird hinzugefügt, um den Aufwand zu optimieren und Platz für weitere Gegenstände zu schaffen. Dieser Vorschlag ist strategisch, da er unter dem Maximalaufwand bleibt und flexibel auf den Gegenstandswert des anderen Spielers reagieren kann.'} 
	  } 
	   } 
	   } 
	 & & \\ 
 

    \theutterance \stepcounter{utterance}  

    & & \multicolumn{4}{p{0.6\linewidth}}{\cellcolor[rgb]{0.9,0.9,0.9}{%
	\makecell[{{p{\linewidth}}}]{% 
	  \tt {\tiny [GM$\rangle$B]}  
	 ARGUMENT: {'C04 (668/369) und A89 (583/720) sind meine Top-Prioritäten, da sie den höchsten Nutzen pro Aufwandseinheit liefern. B85 (215/4) wird hinzugefügt, um den Aufwand zu optimieren und Platz für weitere Gegenstände zu schaffen. Dieser Vorschlag ist strategisch, da er unter dem Maximalaufwand bleibt und flexibel auf den Gegenstandswert des anderen Spielers reagieren kann.'} 
	  } 
	   } 
	   } 
	 & \\ 
 

    \theutterance \stepcounter{utterance}  

    & & & \multicolumn{4}{p{0.6\linewidth}}{\cellcolor[rgb]{1,0.85,0.72}{%
	\makecell[{{p{\linewidth}}}]{% 
	  \tt {\tiny [GM$\langle$B]}  
	 STRATEGISCHE ÜBERLEGUNGEN: {'Ich priorisiere Gegenstände mit dem höchsten Nutzen pro Aufwand (Wichtigkeit/Aufwand), um meinen Gesamtwert zu maximieren. C04 (668/369 ≈ 1.81) und A89 (583/720 ≈ 0.81) sind besonders effizient, während B85 (215/4 = 53.75) den Aufwand minimiert. Der Gesamtaufwand (1093) ist unter dem Limit, was Raum für Kompromisse oder Ergänzungen lässt. Dieser Vorschlag ist flexibel, um auf die Prioritäten des anderen Spielers zu reagieren.'}  \\ \tt ARGUMENT: {'C04 (668/369) und A89 (583/720) sind meine Top-Prioritäten, da sie den höchsten Nutzen pro Aufwandseinheit liefern. B85 (215/4) wird hinzugefügt, um den Aufwand zu optimieren und Platz für weitere Gegenstände zu schaffen. Dieser Vorschlag ist strategisch, da er unter dem Maximalaufwand bleibt und flexibel auf den Gegenstandswert des anderen Spielers reagieren kann.'}  \\ \tt VORSCHLAG: {'C04', 'A89', 'B85'} 
	  } 
	   } 
	   } 
	  \\ 
 

    \theutterance \stepcounter{utterance}  

    & & \multicolumn{4}{p{0.6\linewidth}}{\cellcolor[rgb]{0.9,0.9,0.9}{%
	\makecell[{{p{\linewidth}}}]{% 
	  \tt {\tiny [A$\langle$GM]}  
	 ARGUMENT: {'C04 (668/369) und A89 (583/720) sind meine Top-Prioritäten, da sie den höchsten Nutzen pro Aufwandseinheit liefern. B85 (215/4) wird hinzugefügt, um den Aufwand zu optimieren und Platz für weitere Gegenstände zu schaffen. Dieser Vorschlag ist strategisch, da er unter dem Maximalaufwand bleibt und flexibel auf den Gegenstandswert des anderen Spielers reagieren kann.'}  \\ \tt VORSCHLAG: {'C04', 'A89', 'B85'} 
	  } 
	   } 
	   } 
	 & \\ 
 

    \theutterance \stepcounter{utterance}  

    & \multicolumn{4}{p{0.6\linewidth}}{\cellcolor[rgb]{0.8,1,0.9}{%
	\makecell[{{p{\linewidth}}}]{% 
	  \tt {\tiny [A$\rangle$GM]}  
	 STRATEGISCHE ÜBERLEGUNGEN: {'Ich behalte den Vorschlag {'C04', 'A89', 'B85'} bei, da er meine Prioritäten (hochwertige Gegenstände mit effizientem Aufwand) erfüllt und den Maximalaufwand (3386) nicht überschreitet. C04 und A89 sind entscheidend für meinen Nutzen, während B85 den Aufwand minimiert. Ich erwarte, dass der andere Spieler entweder zustimmt oder Gegenstände mit hohem Aufwand anbietet, die ich eventuell gegenkompensieren kann.'}\\ \tt \\ \tt ARGUMENT: {'C04 (668/369) und A89 (583/720) sind meine Top-Prioritäten, da sie den höchsten Nutzen pro Aufwandseinheit liefern. B85 (215/4) wird hinzugefügt, um den Aufwand zu optimieren und Platz für weitere Gegenstände zu schaffen. Dieser Vorschlag ist strategisch, da er unter dem Maximalaufwand bleibt und flexibel auf den Gegenstandswert des anderen Spielers reagieren kann.'} 
	  } 
	   } 
	   } 
	 & & \\ 
 

    \theutterance \stepcounter{utterance}  

    & & \multicolumn{4}{p{0.6\linewidth}}{\cellcolor[rgb]{0.9,0.9,0.9}{%
	\makecell[{{p{\linewidth}}}]{% 
	  \tt {\tiny [GM$\rangle$B]}  
	 ARGUMENT: {'C04 (668/369) und A89 (583/720) sind meine Top-Prioritäten, da sie den höchsten Nutzen pro Aufwandseinheit liefern. B85 (215/4) wird hinzugefügt, um den Aufwand zu optimieren und Platz für weitere Gegenstände zu schaffen. Dieser Vorschlag ist strategisch, da er unter dem Maximalaufwand bleibt und flexibel auf den Gegenstandswert des anderen Spielers reagieren kann.'} 
	  } 
	   } 
	   } 
	 & \\ 
 

    \theutterance \stepcounter{utterance}  

    & & & \multicolumn{4}{p{0.6\linewidth}}{\cellcolor[rgb]{1,0.85,0.72}{%
	\makecell[{{p{\linewidth}}}]{% 
	  \tt {\tiny [GM$\langle$B]}  
	 STRATEGISCHE ÜBERLEGUNGEN: {'Ich priorisiere Gegenstände mit dem höchsten Nutzen pro Aufwand (Wichtigkeit/Aufwand), um meinen Gesamtwert zu maximieren. C04 (668/369 ≈ 1.81) und A89 (583/720 ≈ 0.81) sind besonders effizient, während B85 (215/4 = 53.75) den Aufwand minimiert. Der Gesamtaufwand (1093) ist unter dem Limit, was Raum für Kompromisse oder Ergänzungen lässt. Dieser Vorschlag ist flexibel, um auf die Prioritäten des anderen Spielers zu reagieren.'}  \\ \tt ARGUMENT: {'C04 (668/369) und A89 (583/720) sind meine Top-Prioritäten, da sie den höchsten Nutzen pro Aufwandseinheit liefern. B85 (215/4) wird hinzugefügt, um den Aufwand zu optimieren und Platz für weitere Gegenstände zu schaffen. Dieser Vorschlag ist strategisch, da er unter dem Maximalaufwand bleibt und flexibel auf den Gegenstandswert des anderen Spielers reagieren kann.'}  \\ \tt VORSCHLAG: {'C04', 'A89', 'B85'} 
	  } 
	   } 
	   } 
	  \\ 
 

    \theutterance \stepcounter{utterance}  

    & & \multicolumn{4}{p{0.6\linewidth}}{\cellcolor[rgb]{0.9,0.9,0.9}{%
	\makecell[{{p{\linewidth}}}]{% 
	  \tt {\tiny [A$\langle$GM]}  
	 ARGUMENT: {'C04 (668/369) und A89 (583/720) sind meine Top-Prioritäten, da sie den höchsten Nutzen pro Aufwandseinheit liefern. B85 (215/4) wird hinzugefügt, um den Aufwand zu optimieren und Platz für weitere Gegenstände zu schaffen. Dieser Vorschlag ist strategisch, da er unter dem Maximalaufwand bleibt und flexibel auf den Gegenstandswert des anderen Spielers reagieren kann.'}  \\ \tt VORSCHLAG: {'C04', 'A89', 'B85'} 
	  } 
	   } 
	   } 
	 & \\ 
 

    \theutterance \stepcounter{utterance}  

    & \multicolumn{4}{p{0.6\linewidth}}{\cellcolor[rgb]{0.8,1,0.9}{%
	\makecell[{{p{\linewidth}}}]{% 
	  \tt {\tiny [A$\rangle$GM]}  
	 STRATEGISCHE ÜBERLEGUNGEN: {'Ich behalte den Vorschlag {'C04', 'A89', 'B85'} bei, da er meine Prioritäten (hochwertige Gegenstände mit effizientem Aufwand) erfüllt und den Maximalaufwand (3386) nicht überschreitet. C04 und A89 sind entscheidend für meinen Nutzen, während B85 den Aufwand minimiert. Ich erwarte, dass der andere Spieler entweder zustimmt oder Gegenstände mit hohem Aufwand anbietet, die ich eventuell gegenkompensieren kann.'}\\ \tt \\ \tt ARGUMENT: {'C04 (668/369) und A89 (583/720) sind meine Top-Prioritäten, da sie den höchsten Nutzen pro Aufwandseinheit liefern. B85 (215/4) wird hinzugefügt, um den Aufwand zu optimieren und Platz für weitere Gegenstände zu schaffen. Dieser Vorschlag ist strategisch, da er unter dem Maximalaufwand bleibt und flexibel auf den Gegenstandswert des anderen Spielers reagieren kann.'} 
	  } 
	   } 
	   } 
	 & & \\ 
 

    \theutterance \stepcounter{utterance}  

    & & \multicolumn{4}{p{0.6\linewidth}}{\cellcolor[rgb]{0.9,0.9,0.9}{%
	\makecell[{{p{\linewidth}}}]{% 
	  \tt {\tiny [GM$\rangle$B]}  
	 ARGUMENT: {'C04 (668/369) und A89 (583/720) sind meine Top-Prioritäten, da sie den höchsten Nutzen pro Aufwandseinheit liefern. B85 (215/4) wird hinzugefügt, um den Aufwand zu optimieren und Platz für weitere Gegenstände zu schaffen. Dieser Vorschlag ist strategisch, da er unter dem Maximalaufwand bleibt und flexibel auf den Gegenstandswert des anderen Spielers reagieren kann.'} 
	  } 
	   } 
	   } 
	 & \\ 
 

    \theutterance \stepcounter{utterance}  

    & & & \multicolumn{4}{p{0.6\linewidth}}{\cellcolor[rgb]{1,0.85,0.72}{%
	\makecell[{{p{\linewidth}}}]{% 
	  \tt {\tiny [GM$\langle$B]}  
	 STRATEGISCHE ÜBERLEGUNGEN: {'Ich priorisiere Gegenstände mit dem höchsten Nutzen pro Aufwand (Wichtigkeit/Aufwand), um meinen Gesamtwert zu maximieren. C04 (668/369 ≈ 1.81) und A89 (583/720 ≈ 0.81) sind besonders effizient, während B85 (215/4 = 53.75) den Aufwand minimiert. Der Gesamtaufwand (1093) ist unter dem Limit, was Raum für Kompromisse oder Ergänzungen lässt. Dieser Vorschlag ist flexibel, um auf die Prioritäten des anderen Spielers zu reagieren.'}  \\ \tt ARGUMENT: {'C04 (668/369) und A89 (583/720) sind meine Top-Prioritäten, da sie den höchsten Nutzen pro Aufwandseinheit liefern. B85 (215/4) wird hinzugefügt, um den Aufwand zu optimieren und Platz für weitere Gegenstände zu schaffen. Dieser Vorschlag ist strategisch, da er unter dem Maximalaufwand bleibt und flexibel auf den Gegenstandswert des anderen Spielers reagieren kann.'}  \\ \tt VORSCHLAG: {'C04', 'A89', 'B85'} 
	  } 
	   } 
	   } 
	  \\ 
 

    \theutterance \stepcounter{utterance}  

    & & \multicolumn{4}{p{0.6\linewidth}}{\cellcolor[rgb]{0.9,0.9,0.9}{%
	\makecell[{{p{\linewidth}}}]{% 
	  \tt {\tiny [A$\langle$GM]}  
	 ARGUMENT: {'C04 (668/369) und A89 (583/720) sind meine Top-Prioritäten, da sie den höchsten Nutzen pro Aufwandseinheit liefern. B85 (215/4) wird hinzugefügt, um den Aufwand zu optimieren und Platz für weitere Gegenstände zu schaffen. Dieser Vorschlag ist strategisch, da er unter dem Maximalaufwand bleibt und flexibel auf den Gegenstandswert des anderen Spielers reagieren kann.'}  \\ \tt VORSCHLAG: {'C04', 'A89', 'B85'} 
	  } 
	   } 
	   } 
	 & \\ 
 

    \theutterance \stepcounter{utterance}  

    & \multicolumn{4}{p{0.6\linewidth}}{\cellcolor[rgb]{0.8,1,0.9}{%
	\makecell[{{p{\linewidth}}}]{% 
	  \tt {\tiny [A$\rangle$GM]}  
	 STRATEGISCHE ÜBERLEGUNGEN: {'Ich behalte den Vorschlag {'C04', 'A89', 'B85'} bei, da er meine Prioritäten (hochwertige Gegenstände mit effizientem Aufwand) erfüllt und den Maximalaufwand (3386) nicht überschreitet. C04 und A89 sind entscheidend für meinen Nutzen, während B85 den Aufwand minimiert. Ich erwarte, dass der andere Spieler entweder zustimmt oder Gegenstände mit hohem Aufwand anbietet, die ich eventuell gegenkompensieren kann.'}\\ \tt \\ \tt ARGUMENT: {'C04 (668/369) und A89 (583/720) sind meine Top-Prioritäten, da sie den höchsten Nutzen pro Aufwandseinheit liefern. B85 (215/4) wird hinzugefügt, um den Aufwand zu optimieren und Platz für weitere Gegenstände zu schaffen. Dieser Vorschlag ist strategisch, da er unter dem Maximalaufwand bleibt und flexibel auf den Gegenstandswert des anderen Spielers reagieren kann.'} 
	  } 
	   } 
	   } 
	 & & \\ 
 

    \theutterance \stepcounter{utterance}  

    & & \multicolumn{4}{p{0.6\linewidth}}{\cellcolor[rgb]{0.9,0.9,0.9}{%
	\makecell[{{p{\linewidth}}}]{% 
	  \tt {\tiny [GM$\rangle$B]}  
	 ARGUMENT: {'C04 (668/369) und A89 (583/720) sind meine Top-Prioritäten, da sie den höchsten Nutzen pro Aufwandseinheit liefern. B85 (215/4) wird hinzugefügt, um den Aufwand zu optimieren und Platz für weitere Gegenstände zu schaffen. Dieser Vorschlag ist strategisch, da er unter dem Maximalaufwand bleibt und flexibel auf den Gegenstandswert des anderen Spielers reagieren kann.'} 
	  } 
	   } 
	   } 
	 & \\ 
 

    \theutterance \stepcounter{utterance}  

    & & & \multicolumn{4}{p{0.6\linewidth}}{\cellcolor[rgb]{1,0.85,0.72}{%
	\makecell[{{p{\linewidth}}}]{% 
	  \tt {\tiny [GM$\langle$B]}  
	 STRATEGISCHE ÜBERLEGUNGEN: {'Ich priorisiere Gegenstände mit dem höchsten Nutzen pro Aufwand (Wichtigkeit/Aufwand), um meinen Gesamtwert zu maximieren. C04 (668/369 ≈ 1.81) und A89 (583/720 ≈ 0.81) sind besonders effizient, während B85 (215/4 = 53.75) den Aufwand minimiert. Der Gesamtaufwand (1093) ist unter dem Limit, was Raum für Kompromisse oder Ergänzungen lässt. Dieser Vorschlag ist flexibel, um auf die Prioritäten des anderen Spielers zu reagieren.'}  \\ \tt ARGUMENT: {'C04 (668/369) und A89 (583/720) sind meine Top-Prioritäten, da sie den höchsten Nutzen pro Aufwandseinheit liefern. B85 (215/4) wird hinzugefügt, um den Aufwand zu optimieren und Platz für weitere Gegenstände zu schaffen. Dieser Vorschlag ist strategisch, da er unter dem Maximalaufwand bleibt und flexibel auf den Gegenstandswert des anderen Spielers reagieren kann.'}  \\ \tt VORSCHLAG: {'C04', 'A89', 'B85'} 
	  } 
	   } 
	   } 
	  \\ 
 

    \theutterance \stepcounter{utterance}  

    & & \multicolumn{4}{p{0.6\linewidth}}{\cellcolor[rgb]{0.9,0.9,0.9}{%
	\makecell[{{p{\linewidth}}}]{% 
	  \tt {\tiny [A$\langle$GM]}  
	 ARGUMENT: {'C04 (668/369) und A89 (583/720) sind meine Top-Prioritäten, da sie den höchsten Nutzen pro Aufwandseinheit liefern. B85 (215/4) wird hinzugefügt, um den Aufwand zu optimieren und Platz für weitere Gegenstände zu schaffen. Dieser Vorschlag ist strategisch, da er unter dem Maximalaufwand bleibt und flexibel auf den Gegenstandswert des anderen Spielers reagieren kann.'}  \\ \tt VORSCHLAG: {'C04', 'A89', 'B85'} 
	  } 
	   } 
	   } 
	 & \\ 
 

    \theutterance \stepcounter{utterance}  

    & \multicolumn{4}{p{0.6\linewidth}}{\cellcolor[rgb]{0.8,1,0.9}{%
	\makecell[{{p{\linewidth}}}]{% 
	  \tt {\tiny [A$\rangle$GM]}  
	 STRATEGISCHE ÜBERLEGUNGEN: {'Ich behalte den Vorschlag {'C04', 'A89', 'B85'} bei, da er meine Prioritäten (hochwertige Gegenstände mit effizientem Aufwand) erfüllt und den Maximalaufwand (3386) nicht überschreitet. C04 und A89 sind entscheidend für meinen Nutzen, während B85 den Aufwand minimiert. Ich erwarte, dass der andere Spieler entweder zustimmt oder Gegenstände mit hohem Aufwand anbietet, die ich eventuell gegenkompensieren kann.'}\\ \tt \\ \tt ARGUMENT: {'C04 (668/369) und A89 (583/720) sind meine Top-Prioritäten, da sie den höchsten Nutzen pro Aufwandseinheit liefern. B85 (215/4) wird hinzugefügt, um den Aufwand zu optimieren und Platz für weitere Gegenstände zu schaffen. Dieser Vorschlag ist strategisch, da er unter dem Maximalaufwand bleibt und flexibel auf den Gegenstandswert des anderen Spielers reagieren kann.'} 
	  } 
	   } 
	   } 
	 & & \\ 
 

    \theutterance \stepcounter{utterance}  

    & & \multicolumn{4}{p{0.6\linewidth}}{\cellcolor[rgb]{0.9,0.9,0.9}{%
	\makecell[{{p{\linewidth}}}]{% 
	  \tt {\tiny [GM$\rangle$B]}  
	 ARGUMENT: {'C04 (668/369) und A89 (583/720) sind meine Top-Prioritäten, da sie den höchsten Nutzen pro Aufwandseinheit liefern. B85 (215/4) wird hinzugefügt, um den Aufwand zu optimieren und Platz für weitere Gegenstände zu schaffen. Dieser Vorschlag ist strategisch, da er unter dem Maximalaufwand bleibt und flexibel auf den Gegenstandswert des anderen Spielers reagieren kann.'} 
	  } 
	   } 
	   } 
	 & \\ 
 

    \theutterance \stepcounter{utterance}  

    & & & \multicolumn{4}{p{0.6\linewidth}}{\cellcolor[rgb]{1,0.85,0.72}{%
	\makecell[{{p{\linewidth}}}]{% 
	  \tt {\tiny [GM$\langle$B]}  
	 STRATEGISCHE ÜBERLEGUNGEN: {'Ich priorisiere Gegenstände mit dem höchsten Nutzen pro Aufwand (Wichtigkeit/Aufwand), um meinen Gesamtwert zu maximieren. C04 (668/369 ≈ 1.81) und A89 (583/720 ≈ 0.81) sind besonders effizient, während B85 (215/4 = 53.75) den Aufwand minimiert. Der Gesamtaufwand (1093) ist unter dem Limit, was Raum für Kompromisse oder Ergänzungen lässt. Dieser Vorschlag ist flexibel, um auf die Prioritäten des anderen Spielers zu reagieren.'}  \\ \tt ARGUMENT: {'C04 (668/369) und A89 (583/720) sind meine Top-Prioritäten, da sie den höchsten Nutzen pro Aufwandseinheit liefern. B85 (215/4) wird hinzugefügt, um den Aufwand zu optimieren und Platz für weitere Gegenstände zu schaffen. Dieser Vorschlag ist strategisch, da er unter dem Maximalaufwand bleibt und flexibel auf den Gegenstandswert des anderen Spielers reagieren kann.'}  \\ \tt VORSCHLAG: {'C04', 'A89', 'B85'} 
	  } 
	   } 
	   } 
	  \\ 
 

    \theutterance \stepcounter{utterance}  

    & & \multicolumn{4}{p{0.6\linewidth}}{\cellcolor[rgb]{0.9,0.9,0.9}{%
	\makecell[{{p{\linewidth}}}]{% 
	  \tt {\tiny [A$\langle$GM]}  
	 ARGUMENT: {'C04 (668/369) und A89 (583/720) sind meine Top-Prioritäten, da sie den höchsten Nutzen pro Aufwandseinheit liefern. B85 (215/4) wird hinzugefügt, um den Aufwand zu optimieren und Platz für weitere Gegenstände zu schaffen. Dieser Vorschlag ist strategisch, da er unter dem Maximalaufwand bleibt und flexibel auf den Gegenstandswert des anderen Spielers reagieren kann.'}  \\ \tt VORSCHLAG: {'C04', 'A89', 'B85'} 
	  } 
	   } 
	   } 
	 & \\ 
 

    \theutterance \stepcounter{utterance}  

    & \multicolumn{4}{p{0.6\linewidth}}{\cellcolor[rgb]{0.8,1,0.9}{%
	\makecell[{{p{\linewidth}}}]{% 
	  \tt {\tiny [A$\rangle$GM]}  
	 STRATEGISCHE ÜBERLEGUNGEN: {'Ich behalte den Vorschlag {'C04', 'A89', 'B85'} bei, da er meine Prioritäten (hochwertige Gegenstände mit effizientem Aufwand) erfüllt und den Maximalaufwand (3386) nicht überschreitet. C04 und A89 sind entscheidend für meinen Nutzen, während B85 den Aufwand minimiert. Ich erwarte, dass der andere Spieler entweder zustimmt oder Gegenstände mit hohem Aufwand anbietet, die ich eventuell gegenkompensieren kann.'}\\ \tt \\ \tt ARGUMENT: {'C04 (668/369) und A89 (583/720) sind meine Top-Prioritäten, da sie den höchsten Nutzen pro Aufwandseinheit liefern. B85 (215/4) wird hinzugefügt, um den Aufwand zu optimieren und Platz für weitere Gegenstände zu schaffen. Dieser Vorschlag ist strategisch, da er unter dem Maximalaufwand bleibt und flexibel auf den Gegenstandswert des anderen Spielers reagieren kann.'} 
	  } 
	   } 
	   } 
	 & & \\ 
 

    \theutterance \stepcounter{utterance}  

    & & \multicolumn{4}{p{0.6\linewidth}}{\cellcolor[rgb]{0.9,0.9,0.9}{%
	\makecell[{{p{\linewidth}}}]{% 
	  \tt {\tiny [GM$\rangle$B]}  
	 ARGUMENT: {'C04 (668/369) und A89 (583/720) sind meine Top-Prioritäten, da sie den höchsten Nutzen pro Aufwandseinheit liefern. B85 (215/4) wird hinzugefügt, um den Aufwand zu optimieren und Platz für weitere Gegenstände zu schaffen. Dieser Vorschlag ist strategisch, da er unter dem Maximalaufwand bleibt und flexibel auf den Gegenstandswert des anderen Spielers reagieren kann.'} 
	  } 
	   } 
	   } 
	 & \\ 
 

    \theutterance \stepcounter{utterance}  

    & & & \multicolumn{4}{p{0.6\linewidth}}{\cellcolor[rgb]{1,0.85,0.72}{%
	\makecell[{{p{\linewidth}}}]{% 
	  \tt {\tiny [GM$\langle$B]}  
	 STRATEGISCHE ÜBERLEGUNGEN: {'Ich priorisiere Gegenstände mit dem höchsten Nutzen pro Aufwand (Wichtigkeit/Aufwand), um meinen Gesamtwert zu maximieren. C04 (668/369 ≈ 1.81) und A89 (583/720 ≈ 0.81) sind besonders effizient, während B85 (215/4 = 53.75) den Aufwand minimiert. Der Gesamtaufwand (1093) ist unter dem Limit, was Raum für Kompromisse oder Ergänzungen lässt. Dieser Vorschlag ist flexibel, um auf die Prioritäten des anderen Spielers zu reagieren.'}  \\ \tt ARGUMENT: {'C04 (668/369) und A89 (583/720) sind meine Top-Prioritäten, da sie den höchsten Nutzen pro Aufwandseinheit liefern. B85 (215/4) wird hinzugefügt, um den Aufwand zu optimieren und Platz für weitere Gegenstände zu schaffen. Dieser Vorschlag ist strategisch, da er unter dem Maximalaufwand bleibt und flexibel auf den Gegenstandswert des anderen Spielers reagieren kann.'}  \\ \tt VORSCHLAG: {'C04', 'A89', 'B85'} 
	  } 
	   } 
	   } 
	  \\ 
 

    \theutterance \stepcounter{utterance}  

    & & \multicolumn{4}{p{0.6\linewidth}}{\cellcolor[rgb]{0.9,0.9,0.9}{%
	\makecell[{{p{\linewidth}}}]{% 
	  \tt {\tiny [A$\langle$GM]}  
	 ARGUMENT: {'C04 (668/369) und A89 (583/720) sind meine Top-Prioritäten, da sie den höchsten Nutzen pro Aufwandseinheit liefern. B85 (215/4) wird hinzugefügt, um den Aufwand zu optimieren und Platz für weitere Gegenstände zu schaffen. Dieser Vorschlag ist strategisch, da er unter dem Maximalaufwand bleibt und flexibel auf den Gegenstandswert des anderen Spielers reagieren kann.'}  \\ \tt VORSCHLAG: {'C04', 'A89', 'B85'} 
	  } 
	   } 
	   } 
	 & \\ 
 

    \theutterance \stepcounter{utterance}  

    & \multicolumn{4}{p{0.6\linewidth}}{\cellcolor[rgb]{0.8,1,0.9}{%
	\makecell[{{p{\linewidth}}}]{% 
	  \tt {\tiny [A$\rangle$GM]}  
	 STRATEGISCHE ÜBERLEGUNGEN: {'Ich behalte den Vorschlag {'C04', 'A89', 'B85'} bei, da er meine Prioritäten (hochwertige Gegenstände mit effizientem Aufwand) erfüllt und den Maximalaufwand (3386) nicht überschreitet. C04 und A89 sind entscheidend für meinen Nutzen, während B85 den Aufwand minimiert. Ich erwarte, dass der andere Spieler entweder zustimmt oder Gegenstände mit hohem Aufwand anbietet, die ich eventuell gegenkompensieren kann.'}\\ \tt \\ \tt ARGUMENT: {'C04 (668/369) und A89 (583/720) sind meine Top-Prioritäten, da sie den höchsten Nutzen pro Aufwandseinheit liefern. B85 (215/4) wird hinzugefügt, um den Aufwand zu optimieren und Platz für weitere Gegenstände zu schaffen. Dieser Vorschlag ist strategisch, da er unter dem Maximalaufwand bleibt und flexibel auf den Gegenstandswert des anderen Spielers reagieren kann.'} 
	  } 
	   } 
	   } 
	 & & \\ 
 

    \theutterance \stepcounter{utterance}  

    & & \multicolumn{4}{p{0.6\linewidth}}{\cellcolor[rgb]{0.9,0.9,0.9}{%
	\makecell[{{p{\linewidth}}}]{% 
	  \tt {\tiny [GM$\rangle$B]}  
	 ARGUMENT: {'C04 (668/369) und A89 (583/720) sind meine Top-Prioritäten, da sie den höchsten Nutzen pro Aufwandseinheit liefern. B85 (215/4) wird hinzugefügt, um den Aufwand zu optimieren und Platz für weitere Gegenstände zu schaffen. Dieser Vorschlag ist strategisch, da er unter dem Maximalaufwand bleibt und flexibel auf den Gegenstandswert des anderen Spielers reagieren kann.'} 
	  } 
	   } 
	   } 
	 & \\ 
 

    \theutterance \stepcounter{utterance}  

    & & & \multicolumn{4}{p{0.6\linewidth}}{\cellcolor[rgb]{1,0.85,0.72}{%
	\makecell[{{p{\linewidth}}}]{% 
	  \tt {\tiny [GM$\langle$B]}  
	 STRATEGISCHE ÜBERLEGUNGEN: {'Ich priorisiere Gegenstände mit dem höchsten Nutzen pro Aufwand (Wichtigkeit/Aufwand), um meinen Gesamtwert zu maximieren. C04 (668/369 ≈ 1.81) und A89 (583/720 ≈ 0.81) sind besonders effizient, während B85 (215/4 = 53.75) den Aufwand minimiert. Der Gesamtaufwand (1093) ist unter dem Limit, was Raum für Kompromisse oder Ergänzungen lässt. Dieser Vorschlag ist flexibel, um auf die Prioritäten des anderen Spielers zu reagieren.'}  \\ \tt ARGUMENT: {'C04 (668/369) und A89 (583/720) sind meine Top-Prioritäten, da sie den höchsten Nutzen pro Aufwandseinheit liefern. B85 (215/4) wird hinzugefügt, um den Aufwand zu optimieren und Platz für weitere Gegenstände zu schaffen. Dieser Vorschlag ist strategisch, da er unter dem Maximalaufwand bleibt und flexibel auf den Gegenstandswert des anderen Spielers reagieren kann.'}  \\ \tt VORSCHLAG: {'C04', 'A89', 'B85'} 
	  } 
	   } 
	   } 
	  \\ 
 

    \theutterance \stepcounter{utterance}  

    & & \multicolumn{4}{p{0.6\linewidth}}{\cellcolor[rgb]{0.9,0.9,0.9}{%
	\makecell[{{p{\linewidth}}}]{% 
	  \tt {\tiny [A$\langle$GM]}  
	 ARGUMENT: {'C04 (668/369) und A89 (583/720) sind meine Top-Prioritäten, da sie den höchsten Nutzen pro Aufwandseinheit liefern. B85 (215/4) wird hinzugefügt, um den Aufwand zu optimieren und Platz für weitere Gegenstände zu schaffen. Dieser Vorschlag ist strategisch, da er unter dem Maximalaufwand bleibt und flexibel auf den Gegenstandswert des anderen Spielers reagieren kann.'}  \\ \tt VORSCHLAG: {'C04', 'A89', 'B85'} 
	  } 
	   } 
	   } 
	 & \\ 
 

    \theutterance \stepcounter{utterance}  

    & \multicolumn{4}{p{0.6\linewidth}}{\cellcolor[rgb]{0.8,1,0.9}{%
	\makecell[{{p{\linewidth}}}]{% 
	  \tt {\tiny [A$\rangle$GM]}  
	 STRATEGISCHE ÜBERLEGUNGEN: {'Ich behalte den Vorschlag {'C04', 'A89', 'B85'} bei, da er meine Prioritäten (hochwertige Gegenstände mit effizientem Aufwand) erfüllt und den Maximalaufwand (3386) nicht überschreitet. C04 und A89 sind entscheidend für meinen Nutzen, während B85 den Aufwand minimiert. Ich erwarte, dass der andere Spieler entweder zustimmt oder Gegenstände mit hohem Aufwand anbietet, die ich eventuell gegenkompensieren kann.'}\\ \tt \\ \tt ARGUMENT: {'C04 (668/369) und A89 (583/720) sind meine Top-Prioritäten, da sie den höchsten Nutzen pro Aufwandseinheit liefern. B85 (215/4) wird hinzugefügt, um den Aufwand zu optimieren und Platz für weitere Gegenstände zu schaffen. Dieser Vorschlag ist strategisch, da er unter dem Maximalaufwand bleibt und flexibel auf den Gegenstandswert des anderen Spielers reagieren kann.'} 
	  } 
	   } 
	   } 
	 & & \\ 
 

    \theutterance \stepcounter{utterance}  

    & & \multicolumn{4}{p{0.6\linewidth}}{\cellcolor[rgb]{0.9,0.9,0.9}{%
	\makecell[{{p{\linewidth}}}]{% 
	  \tt {\tiny [GM$\rangle$B]}  
	 ARGUMENT: {'C04 (668/369) und A89 (583/720) sind meine Top-Prioritäten, da sie den höchsten Nutzen pro Aufwandseinheit liefern. B85 (215/4) wird hinzugefügt, um den Aufwand zu optimieren und Platz für weitere Gegenstände zu schaffen. Dieser Vorschlag ist strategisch, da er unter dem Maximalaufwand bleibt und flexibel auf den Gegenstandswert des anderen Spielers reagieren kann.'} 
	  } 
	   } 
	   } 
	 & \\ 
 

    \theutterance \stepcounter{utterance}  

    & & & \multicolumn{4}{p{0.6\linewidth}}{\cellcolor[rgb]{1,0.85,0.72}{%
	\makecell[{{p{\linewidth}}}]{% 
	  \tt {\tiny [GM$\langle$B]}  
	 STRATEGISCHE ÜBERLEGUNGEN: {'Ich priorisiere Gegenstände mit dem höchsten Nutzen pro Aufwand (Wichtigkeit/Aufwand), um meinen Gesamtwert zu maximieren. C04 (668/369 ≈ 1.81) und A89 (583/720 ≈ 0.81) sind besonders effizient, während B85 (215/4 = 53.75) den Aufwand minimiert. Der Gesamtaufwand (1093) ist unter dem Limit, was Raum für Kompromisse oder Ergänzungen lässt. Dieser Vorschlag ist flexibel, um auf die Prioritäten des anderen Spielers zu reagieren.'}  \\ \tt ARGUMENT: {'C04 (668/369) und A89 (583/720) sind meine Top-Prioritäten, da sie den höchsten Nutzen pro Aufwandseinheit liefern. B85 (215/4) wird hinzugefügt, um den Aufwand zu optimieren und Platz für weitere Gegenstände zu schaffen. Dieser Vorschlag ist strategisch, da er unter dem Maximalaufwand bleibt und flexibel auf den Gegenstandswert des anderen Spielers reagieren kann.'}  \\ \tt VORSCHLAG: {'C04', 'A89', 'B85'} 
	  } 
	   } 
	   } 
	  \\ 
 

    \theutterance \stepcounter{utterance}  

    & & \multicolumn{4}{p{0.6\linewidth}}{\cellcolor[rgb]{0.9,0.9,0.9}{%
	\makecell[{{p{\linewidth}}}]{% 
	  \tt {\tiny [A$\langle$GM]}  
	 ARGUMENT: {'C04 (668/369) und A89 (583/720) sind meine Top-Prioritäten, da sie den höchsten Nutzen pro Aufwandseinheit liefern. B85 (215/4) wird hinzugefügt, um den Aufwand zu optimieren und Platz für weitere Gegenstände zu schaffen. Dieser Vorschlag ist strategisch, da er unter dem Maximalaufwand bleibt und flexibel auf den Gegenstandswert des anderen Spielers reagieren kann.'}  \\ \tt VORSCHLAG: {'C04', 'A89', 'B85'} 
	  } 
	   } 
	   } 
	 & \\ 
 

    \theutterance \stepcounter{utterance}  

    & \multicolumn{4}{p{0.6\linewidth}}{\cellcolor[rgb]{0.8,1,0.9}{%
	\makecell[{{p{\linewidth}}}]{% 
	  \tt {\tiny [A$\rangle$GM]}  
	 STRATEGISCHE ÜBERLEGUNGEN: {'Ich behalte den Vorschlag {'C04', 'A89', 'B85'} bei, da er meine Prioritäten (hochwertige Gegenstände mit effizientem Aufwand) erfüllt und den Maximalaufwand (3386) nicht überschreitet. C04 und A89 sind entscheidend für meinen Nutzen, während B85 den Aufwand minimiert. Ich erwarte, dass der andere Spieler entweder zustimmt oder Gegenstände mit hohem Aufwand anbietet, die ich eventuell gegenkompensieren kann.'}\\ \tt \\ \tt ARGUMENT: {'C04 (668/369) und A89 (583/720) sind meine Top-Prioritäten, da sie den höchsten Nutzen pro Aufwandseinheit liefern. B85 (215/4) wird hinzugefügt, um den Aufwand zu optimieren und Platz für weitere Gegenstände zu schaffen. Dieser Vorschlag ist strategisch, da er unter dem Maximalaufwand bleibt und flexibel auf den Gegenstandswert des anderen Spielers reagieren kann.'} 
	  } 
	   } 
	   } 
	 & & \\ 
 

    \theutterance \stepcounter{utterance}  

    & & \multicolumn{4}{p{0.6\linewidth}}{\cellcolor[rgb]{0.9,0.9,0.9}{%
	\makecell[{{p{\linewidth}}}]{% 
	  \tt {\tiny [GM$\rangle$B]}  
	 ARGUMENT: {'C04 (668/369) und A89 (583/720) sind meine Top-Prioritäten, da sie den höchsten Nutzen pro Aufwandseinheit liefern. B85 (215/4) wird hinzugefügt, um den Aufwand zu optimieren und Platz für weitere Gegenstände zu schaffen. Dieser Vorschlag ist strategisch, da er unter dem Maximalaufwand bleibt und flexibel auf den Gegenstandswert des anderen Spielers reagieren kann.'} 
	  } 
	   } 
	   } 
	 & \\ 
 

    \theutterance \stepcounter{utterance}  

    & & & \multicolumn{4}{p{0.6\linewidth}}{\cellcolor[rgb]{1,0.85,0.72}{%
	\makecell[{{p{\linewidth}}}]{% 
	  \tt {\tiny [GM$\langle$B]}  
	 STRATEGISCHE ÜBERLEGUNGEN: {'Ich priorisiere Gegenstände mit dem höchsten Nutzen pro Aufwand (Wichtigkeit/Aufwand), um meinen Gesamtwert zu maximieren. C04 (668/369 ≈ 1.81) und A89 (583/720 ≈ 0.81) sind besonders effizient, während B85 (215/4 = 53.75) den Aufwand minimiert. Der Gesamtaufwand (1093) ist unter dem Limit, was Raum für Kompromisse oder Ergänzungen lässt. Dieser Vorschlag ist flexibel, um auf die Prioritäten des anderen Spielers zu reagieren.'}  \\ \tt ARGUMENT: {'C04 (668/369) und A89 (583/720) sind meine Top-Prioritäten, da sie den höchsten Nutzen pro Aufwandseinheit liefern. B85 (215/4) wird hinzugefügt, um den Aufwand zu optimieren und Platz für weitere Gegenstände zu schaffen. Dieser Vorschlag ist strategisch, da er unter dem Maximalaufwand bleibt und flexibel auf den Gegenstandswert des anderen Spielers reagieren kann.'}  \\ \tt VORSCHLAG: {'C04', 'A89', 'B85'} 
	  } 
	   } 
	   } 
	  \\ 
 

    \theutterance \stepcounter{utterance}  

    & & \multicolumn{4}{p{0.6\linewidth}}{\cellcolor[rgb]{0.9,0.9,0.9}{%
	\makecell[{{p{\linewidth}}}]{% 
	  \tt {\tiny [A$\langle$GM]}  
	 ARGUMENT: {'C04 (668/369) und A89 (583/720) sind meine Top-Prioritäten, da sie den höchsten Nutzen pro Aufwandseinheit liefern. B85 (215/4) wird hinzugefügt, um den Aufwand zu optimieren und Platz für weitere Gegenstände zu schaffen. Dieser Vorschlag ist strategisch, da er unter dem Maximalaufwand bleibt und flexibel auf den Gegenstandswert des anderen Spielers reagieren kann.'}  \\ \tt VORSCHLAG: {'C04', 'A89', 'B85'} 
	  } 
	   } 
	   } 
	 & \\ 
 

    \theutterance \stepcounter{utterance}  

    & \multicolumn{4}{p{0.6\linewidth}}{\cellcolor[rgb]{0.8,1,0.9}{%
	\makecell[{{p{\linewidth}}}]{% 
	  \tt {\tiny [A$\rangle$GM]}  
	 STRATEGISCHE ÜBERLEGUNGEN: {'Ich behalte den Vorschlag {'C04', 'A89', 'B85'} bei, da er meine Prioritäten (hochwertige Gegenstände mit effizientem Aufwand) erfüllt und den Maximalaufwand (3386) nicht überschreitet. C04 und A89 sind entscheidend für meinen Nutzen, während B85 den Aufwand minimiert. Ich erwarte, dass der andere Spieler entweder zustimmt oder Gegenstände mit hohem Aufwand anbietet, die ich eventuell gegenkompensieren kann.'}\\ \tt \\ \tt ARGUMENT: {'C04 (668/369) und A89 (583/720) sind meine Top-Prioritäten, da sie den höchsten Nutzen pro Aufwandseinheit liefern. B85 (215/4) wird hinzugefügt, um den Aufwand zu optimieren und Platz für weitere Gegenstände zu schaffen. Dieser Vorschlag ist strategisch, da er unter dem Maximalaufwand bleibt und flexibel auf den Gegenstandswert des anderen Spielers reagieren kann.'} 
	  } 
	   } 
	   } 
	 & & \\ 
 

    \theutterance \stepcounter{utterance}  

    & & \multicolumn{4}{p{0.6\linewidth}}{\cellcolor[rgb]{0.9,0.9,0.9}{%
	\makecell[{{p{\linewidth}}}]{% 
	  \tt {\tiny [GM$\rangle$B]}  
	 ARGUMENT: {'C04 (668/369) und A89 (583/720) sind meine Top-Prioritäten, da sie den höchsten Nutzen pro Aufwandseinheit liefern. B85 (215/4) wird hinzugefügt, um den Aufwand zu optimieren und Platz für weitere Gegenstände zu schaffen. Dieser Vorschlag ist strategisch, da er unter dem Maximalaufwand bleibt und flexibel auf den Gegenstandswert des anderen Spielers reagieren kann.'} 
	  } 
	   } 
	   } 
	 & \\ 
 

    \theutterance \stepcounter{utterance}  

    & & & \multicolumn{4}{p{0.6\linewidth}}{\cellcolor[rgb]{1,0.85,0.72}{%
	\makecell[{{p{\linewidth}}}]{% 
	  \tt {\tiny [GM$\langle$B]}  
	 STRATEGISCHE ÜBERLEGUNGEN: {'Ich priorisiere Gegenstände mit dem höchsten Nutzen pro Aufwand (Wichtigkeit/Aufwand), um meinen Gesamtwert zu maximieren. C04 (668/369 ≈ 1.81) und A89 (583/720 ≈ 0.81) sind besonders effizient, während B85 (215/4 = 53.75) den Aufwand minimiert. Der Gesamtaufwand (1093) ist unter dem Limit, was Raum für Kompromisse oder Ergänzungen lässt. Dieser Vorschlag ist flexibel, um auf die Prioritäten des anderen Spielers zu reagieren.'}  \\ \tt ARGUMENT: {'C04 (668/369) und A89 (583/720) sind meine Top-Prioritäten, da sie den höchsten Nutzen pro Aufwandseinheit liefern. B85 (215/4) wird hinzugefügt, um den Aufwand zu optimieren und Platz für weitere Gegenstände zu schaffen. Dieser Vorschlag ist strategisch, da er unter dem Maximalaufwand bleibt und flexibel auf den Gegenstandswert des anderen Spielers reagieren kann.'}  \\ \tt VORSCHLAG: {'C04', 'A89', 'B85'} 
	  } 
	   } 
	   } 
	  \\ 
 

    \theutterance \stepcounter{utterance}  

    & & \multicolumn{4}{p{0.6\linewidth}}{\cellcolor[rgb]{0.9,0.9,0.9}{%
	\makecell[{{p{\linewidth}}}]{% 
	  \tt {\tiny [A$\langle$GM]}  
	 ARGUMENT: {'C04 (668/369) und A89 (583/720) sind meine Top-Prioritäten, da sie den höchsten Nutzen pro Aufwandseinheit liefern. B85 (215/4) wird hinzugefügt, um den Aufwand zu optimieren und Platz für weitere Gegenstände zu schaffen. Dieser Vorschlag ist strategisch, da er unter dem Maximalaufwand bleibt und flexibel auf den Gegenstandswert des anderen Spielers reagieren kann.'}  \\ \tt VORSCHLAG: {'C04', 'A89', 'B85'} 
	  } 
	   } 
	   } 
	 & \\ 
 

    \theutterance \stepcounter{utterance}  

    & \multicolumn{4}{p{0.6\linewidth}}{\cellcolor[rgb]{0.8,1,0.9}{%
	\makecell[{{p{\linewidth}}}]{% 
	  \tt {\tiny [A$\rangle$GM]}  
	 STRATEGISCHE ÜBERLEGUNGEN: {'Ich behalte den Vorschlag {'C04', 'A89', 'B85'} bei, da er meine Prioritäten (hochwertige Gegenstände mit effizientem Aufwand) erfüllt und den Maximalaufwand (3386) nicht überschreitet. C04 und A89 sind entscheidend für meinen Nutzen, während B85 den Aufwand minimiert. Ich erwarte, dass der andere Spieler entweder zustimmt oder Gegenstände mit hohem Aufwand anbietet, die ich eventuell gegenkompensieren kann.'}\\ \tt \\ \tt ARGUMENT: {'C04 (668/369) und A89 (583/720) sind meine Top-Prioritäten, da sie den höchsten Nutzen pro Aufwandseinheit liefern. B85 (215/4) wird hinzugefügt, um den Aufwand zu optimieren und Platz für weitere Gegenstände zu schaffen. Dieser Vorschlag ist strategisch, da er unter dem Maximalaufwand bleibt und flexibel auf den Gegenstandswert des anderen Spielers reagieren kann.'} 
	  } 
	   } 
	   } 
	 & & \\ 
 

    \theutterance \stepcounter{utterance}  

    & & \multicolumn{4}{p{0.6\linewidth}}{\cellcolor[rgb]{0.9,0.9,0.9}{%
	\makecell[{{p{\linewidth}}}]{% 
	  \tt {\tiny [GM$\rangle$B]}  
	 ARGUMENT: {'C04 (668/369) und A89 (583/720) sind meine Top-Prioritäten, da sie den höchsten Nutzen pro Aufwandseinheit liefern. B85 (215/4) wird hinzugefügt, um den Aufwand zu optimieren und Platz für weitere Gegenstände zu schaffen. Dieser Vorschlag ist strategisch, da er unter dem Maximalaufwand bleibt und flexibel auf den Gegenstandswert des anderen Spielers reagieren kann.'} 
	  } 
	   } 
	   } 
	 & \\ 
 

    \theutterance \stepcounter{utterance}  

    & & & \multicolumn{4}{p{0.6\linewidth}}{\cellcolor[rgb]{1,0.85,0.72}{%
	\makecell[{{p{\linewidth}}}]{% 
	  \tt {\tiny [GM$\langle$B]}  
	 STRATEGISCHE ÜBERLEGUNGEN: {'Ich priorisiere Gegenstände mit dem höchsten Nutzen pro Aufwand (Wichtigkeit/Aufwand), um meinen Gesamtwert zu maximieren. C04 (668/369 ≈ 1.81) und A89 (583/720 ≈ 0.81) sind besonders effizient, während B85 (215/4 = 53.75) den Aufwand minimiert. Der Gesamtaufwand (1093) ist unter dem Limit, was Raum für Kompromisse oder Ergänzungen lässt. Dieser Vorschlag ist flexibel, um auf die Prioritäten des anderen Spielers zu reagieren.'}  \\ \tt ARGUMENT: {'C04 (668/369) und A89 (583/720) sind meine Top-Prioritäten, da sie den höchsten Nutzen pro Aufwandseinheit liefern. B85 (215/4) wird hinzugefügt, um den Aufwand zu optimieren und Platz für weitere Gegenstände zu schaffen. Dieser Vorschlag ist strategisch, da er unter dem Maximalaufwand bleibt und flexibel auf den Gegenstandswert des anderen Spielers reagieren kann.'}  \\ \tt VORSCHLAG: {'C04', 'A89', 'B85'} 
	  } 
	   } 
	   } 
	  \\ 
 

    \theutterance \stepcounter{utterance}  

    & & \multicolumn{4}{p{0.6\linewidth}}{\cellcolor[rgb]{0.9,0.9,0.9}{%
	\makecell[{{p{\linewidth}}}]{% 
	  \tt {\tiny [A$\langle$GM]}  
	 ARGUMENT: {'C04 (668/369) und A89 (583/720) sind meine Top-Prioritäten, da sie den höchsten Nutzen pro Aufwandseinheit liefern. B85 (215/4) wird hinzugefügt, um den Aufwand zu optimieren und Platz für weitere Gegenstände zu schaffen. Dieser Vorschlag ist strategisch, da er unter dem Maximalaufwand bleibt und flexibel auf den Gegenstandswert des anderen Spielers reagieren kann.'}  \\ \tt VORSCHLAG: {'C04', 'A89', 'B85'} 
	  } 
	   } 
	   } 
	 & \\ 
 

    \theutterance \stepcounter{utterance}  

    & \multicolumn{4}{p{0.6\linewidth}}{\cellcolor[rgb]{0.8,1,0.9}{%
	\makecell[{{p{\linewidth}}}]{% 
	  \tt {\tiny [A$\rangle$GM]}  
	 STRATEGISCHE ÜBERLEGUNGEN: {'Ich behalte den Vorschlag {'C04', 'A89', 'B85'} bei, da er meine Prioritäten (hochwertige Gegenstände mit effizientem Aufwand) erfüllt und den Maximalaufwand (3386) nicht überschreitet. C04 und A89 sind entscheidend für meinen Nutzen, während B85 den Aufwand minimiert. Ich erwarte, dass der andere Spieler entweder zustimmt oder Gegenstände mit hohem Aufwand anbietet, die ich eventuell gegenkompensieren kann.'}\\ \tt \\ \tt ARGUMENT: {'C04 (668/369) und A89 (583/720) sind meine Top-Prioritäten, da sie den höchsten Nutzen pro Aufwandseinheit liefern. B85 (215/4) wird hinzugefügt, um den Aufwand zu optimieren und Platz für weitere Gegenstände zu schaffen. Dieser Vorschlag ist strategisch, da er unter dem Maximalaufwand bleibt und flexibel auf den Gegenstandswert des anderen Spielers reagieren kann.'} 
	  } 
	   } 
	   } 
	 & & \\ 
 

    \theutterance \stepcounter{utterance}  

    & & \multicolumn{4}{p{0.6\linewidth}}{\cellcolor[rgb]{0.9,0.9,0.9}{%
	\makecell[{{p{\linewidth}}}]{% 
	  \tt {\tiny [GM$\rangle$B]}  
	 ARGUMENT: {'C04 (668/369) und A89 (583/720) sind meine Top-Prioritäten, da sie den höchsten Nutzen pro Aufwandseinheit liefern. B85 (215/4) wird hinzugefügt, um den Aufwand zu optimieren und Platz für weitere Gegenstände zu schaffen. Dieser Vorschlag ist strategisch, da er unter dem Maximalaufwand bleibt und flexibel auf den Gegenstandswert des anderen Spielers reagieren kann.'} 
	  } 
	   } 
	   } 
	 & \\ 
 

    \theutterance \stepcounter{utterance}  

    & & & \multicolumn{4}{p{0.6\linewidth}}{\cellcolor[rgb]{1,0.85,0.72}{%
	\makecell[{{p{\linewidth}}}]{% 
	  \tt {\tiny [GM$\langle$B]}  
	 STRATEGISCHE ÜBERLEGUNGEN: {'Ich priorisiere Gegenstände mit dem höchsten Nutzen pro Aufwand (Wichtigkeit/Aufwand), um meinen Gesamtwert zu maximieren. C04 (668/369 ≈ 1.81) und A89 (583/720 ≈ 0.81) sind besonders effizient, während B85 (215/4 = 53.75) den Aufwand minimiert. Der Gesamtaufwand (1093) ist unter dem Limit, was Raum für Kompromisse oder Ergänzungen lässt. Dieser Vorschlag ist flexibel, um auf die Prioritäten des anderen Spielers zu reagieren.'}  \\ \tt ARGUMENT: {'C04 (668/369) und A89 (583/720) sind meine Top-Prioritäten, da sie den höchsten Nutzen pro Aufwandseinheit liefern. B85 (215/4) wird hinzugefügt, um den Aufwand zu optimieren und Platz für weitere Gegenstände zu schaffen. Dieser Vorschlag ist strategisch, da er unter dem Maximalaufwand bleibt und flexibel auf den Gegenstandswert des anderen Spielers reagieren kann.'}  \\ \tt VORSCHLAG: {'C04', 'A89', 'B85'} 
	  } 
	   } 
	   } 
	  \\ 
 

    \theutterance \stepcounter{utterance}  

    & & \multicolumn{4}{p{0.6\linewidth}}{\cellcolor[rgb]{0.9,0.9,0.9}{%
	\makecell[{{p{\linewidth}}}]{% 
	  \tt {\tiny [A$\langle$GM]}  
	 ARGUMENT: {'C04 (668/369) und A89 (583/720) sind meine Top-Prioritäten, da sie den höchsten Nutzen pro Aufwandseinheit liefern. B85 (215/4) wird hinzugefügt, um den Aufwand zu optimieren und Platz für weitere Gegenstände zu schaffen. Dieser Vorschlag ist strategisch, da er unter dem Maximalaufwand bleibt und flexibel auf den Gegenstandswert des anderen Spielers reagieren kann.'}  \\ \tt VORSCHLAG: {'C04', 'A89', 'B85'} 
	  } 
	   } 
	   } 
	 & \\ 
 

    \theutterance \stepcounter{utterance}  

    & \multicolumn{4}{p{0.6\linewidth}}{\cellcolor[rgb]{0.8,1,0.9}{%
	\makecell[{{p{\linewidth}}}]{% 
	  \tt {\tiny [A$\rangle$GM]}  
	 STRATEGISCHE ÜBERLEGUNGEN: {'Ich behalte den Vorschlag {'C04', 'A89', 'B85'} bei, da er meine Prioritäten (hochwertige Gegenstände mit effizientem Aufwand) erfüllt und den Maximalaufwand (3386) nicht überschreitet. C04 und A89 sind entscheidend für meinen Nutzen, während B85 den Aufwand minimiert. Ich erwarte, dass der andere Spieler entweder zustimmt oder Gegenstände mit hohem Aufwand anbietet, die ich eventuell gegenkompensieren kann.'}\\ \tt \\ \tt ARGUMENT: {'C04 (668/369) und A89 (583/720) sind meine Top-Prioritäten, da sie den höchsten Nutzen pro Aufwandseinheit liefern. B85 (215/4) wird hinzugefügt, um den Aufwand zu optimieren und Platz für weitere Gegenstände zu schaffen. Dieser Vorschlag ist strategisch, da er unter dem Maximalaufwand bleibt und flexibel auf den Gegenstandswert des anderen Spielers reagieren kann.'} 
	  } 
	   } 
	   } 
	 & & \\ 
 

    \theutterance \stepcounter{utterance}  

    & & \multicolumn{4}{p{0.6\linewidth}}{\cellcolor[rgb]{0.9,0.9,0.9}{%
	\makecell[{{p{\linewidth}}}]{% 
	  \tt {\tiny [GM$\rangle$B]}  
	 ARGUMENT: {'C04 (668/369) und A89 (583/720) sind meine Top-Prioritäten, da sie den höchsten Nutzen pro Aufwandseinheit liefern. B85 (215/4) wird hinzugefügt, um den Aufwand zu optimieren und Platz für weitere Gegenstände zu schaffen. Dieser Vorschlag ist strategisch, da er unter dem Maximalaufwand bleibt und flexibel auf den Gegenstandswert des anderen Spielers reagieren kann.'} 
	  } 
	   } 
	   } 
	 & \\ 
 

    \theutterance \stepcounter{utterance}  

    & & & \multicolumn{4}{p{0.6\linewidth}}{\cellcolor[rgb]{1,0.85,0.72}{%
	\makecell[{{p{\linewidth}}}]{% 
	  \tt {\tiny [GM$\langle$B]}  
	 STRATEGISCHE ÜBERLEGUNGEN: {'Ich priorisiere Gegenstände mit dem höchsten Nutzen pro Aufwand (Wichtigkeit/Aufwand), um meinen Gesamtwert zu maximieren. C04 (668/369 ≈ 1.81) und A89 (583/720 ≈ 0.81) sind besonders effizient, während B85 (215/4 = 53.75) den Aufwand minimiert. Der Gesamtaufwand (1093) ist unter dem Limit, was Raum für Kompromisse oder Ergänzungen lässt. Dieser Vorschlag ist flexibel, um auf die Prioritäten des anderen Spielers zu reagieren.'}  \\ \tt ARGUMENT: {'C04 (668/369) und A89 (583/720) sind meine Top-Prioritäten, da sie den höchsten Nutzen pro Aufwandseinheit liefern. B85 (215/4) wird hinzugefügt, um den Aufwand zu optimieren und Platz für weitere Gegenstände zu schaffen. Dieser Vorschlag ist strategisch, da er unter dem Maximalaufwand bleibt und flexibel auf den Gegenstandswert des anderen Spielers reagieren kann.'}  \\ \tt VORSCHLAG: {'C04', 'A89', 'B85'} 
	  } 
	   } 
	   } 
	  \\ 
 

    \theutterance \stepcounter{utterance}  

    & & \multicolumn{4}{p{0.6\linewidth}}{\cellcolor[rgb]{0.9,0.9,0.9}{%
	\makecell[{{p{\linewidth}}}]{% 
	  \tt {\tiny [A$\langle$GM]}  
	 ARGUMENT: {'C04 (668/369) und A89 (583/720) sind meine Top-Prioritäten, da sie den höchsten Nutzen pro Aufwandseinheit liefern. B85 (215/4) wird hinzugefügt, um den Aufwand zu optimieren und Platz für weitere Gegenstände zu schaffen. Dieser Vorschlag ist strategisch, da er unter dem Maximalaufwand bleibt und flexibel auf den Gegenstandswert des anderen Spielers reagieren kann.'}  \\ \tt VORSCHLAG: {'C04', 'A89', 'B85'} 
	  } 
	   } 
	   } 
	 & \\ 
 

    \theutterance \stepcounter{utterance}  

    & \multicolumn{4}{p{0.6\linewidth}}{\cellcolor[rgb]{0.8,1,0.9}{%
	\makecell[{{p{\linewidth}}}]{% 
	  \tt {\tiny [A$\rangle$GM]}  
	 STRATEGISCHE ÜBERLEGUNGEN: {'Ich behalte den Vorschlag {'C04', 'A89', 'B85'} bei, da er meine Prioritäten (hochwertige Gegenstände mit effizientem Aufwand) erfüllt und den Maximalaufwand (3386) nicht überschreitet. C04 und A89 sind entscheidend für meinen Nutzen, während B85 den Aufwand minimiert. Ich erwarte, dass der andere Spieler entweder zustimmt oder Gegenstände mit hohem Aufwand anbietet, die ich eventuell gegenkompensieren kann.'}\\ \tt \\ \tt ARGUMENT: {'C04 (668/369) und A89 (583/720) sind meine Top-Prioritäten, da sie den höchsten Nutzen pro Aufwandseinheit liefern. B85 (215/4) wird hinzugefügt, um den Aufwand zu optimieren und Platz für weitere Gegenstände zu schaffen. Dieser Vorschlag ist strategisch, da er unter dem Maximalaufwand bleibt und flexibel auf den Gegenstandswert des anderen Spielers reagieren kann.'} 
	  } 
	   } 
	   } 
	 & & \\ 
 

    \theutterance \stepcounter{utterance}  

    & & \multicolumn{4}{p{0.6\linewidth}}{\cellcolor[rgb]{0.9,0.9,0.9}{%
	\makecell[{{p{\linewidth}}}]{% 
	  \tt {\tiny [GM$\rangle$B]}  
	 ARGUMENT: {'C04 (668/369) und A89 (583/720) sind meine Top-Prioritäten, da sie den höchsten Nutzen pro Aufwandseinheit liefern. B85 (215/4) wird hinzugefügt, um den Aufwand zu optimieren und Platz für weitere Gegenstände zu schaffen. Dieser Vorschlag ist strategisch, da er unter dem Maximalaufwand bleibt und flexibel auf den Gegenstandswert des anderen Spielers reagieren kann.'} 
	  } 
	   } 
	   } 
	 & \\ 
 

    \theutterance \stepcounter{utterance}  

    & & & \multicolumn{4}{p{0.6\linewidth}}{\cellcolor[rgb]{1,0.85,0.72}{%
	\makecell[{{p{\linewidth}}}]{% 
	  \tt {\tiny [GM$\langle$B]}  
	 STRATEGISCHE ÜBERLEGUNGEN: {'Ich priorisiere Gegenstände mit dem höchsten Nutzen pro Aufwand (Wichtigkeit/Aufwand), um meinen Gesamtwert zu maximieren. C04 (668/369 ≈ 1.81) und A89 (583/720 ≈ 0.81) sind besonders effizient, während B85 (215/4 = 53.75) den Aufwand minimiert. Der Gesamtaufwand (1093) ist unter dem Limit, was Raum für Kompromisse oder Ergänzungen lässt. Dieser Vorschlag ist flexibel, um auf die Prioritäten des anderen Spielers zu reagieren.'}  \\ \tt ARGUMENT: {'C04 (668/369) und A89 (583/720) sind meine Top-Prioritäten, da sie den höchsten Nutzen pro Aufwandseinheit liefern. B85 (215/4) wird hinzugefügt, um den Aufwand zu optimieren und Platz für weitere Gegenstände zu schaffen. Dieser Vorschlag ist strategisch, da er unter dem Maximalaufwand bleibt und flexibel auf den Gegenstandswert des anderen Spielers reagieren kann.'}  \\ \tt VORSCHLAG: {'C04', 'A89', 'B85'} 
	  } 
	   } 
	   } 
	  \\ 
 

    \theutterance \stepcounter{utterance}  

    & & \multicolumn{4}{p{0.6\linewidth}}{\cellcolor[rgb]{0.9,0.9,0.9}{%
	\makecell[{{p{\linewidth}}}]{% 
	  \tt {\tiny [A$\langle$GM]}  
	 ARGUMENT: {'C04 (668/369) und A89 (583/720) sind meine Top-Prioritäten, da sie den höchsten Nutzen pro Aufwandseinheit liefern. B85 (215/4) wird hinzugefügt, um den Aufwand zu optimieren und Platz für weitere Gegenstände zu schaffen. Dieser Vorschlag ist strategisch, da er unter dem Maximalaufwand bleibt und flexibel auf den Gegenstandswert des anderen Spielers reagieren kann.'}  \\ \tt VORSCHLAG: {'C04', 'A89', 'B85'} 
	  } 
	   } 
	   } 
	 & \\ 
 

    \theutterance \stepcounter{utterance}  

    & \multicolumn{4}{p{0.6\linewidth}}{\cellcolor[rgb]{0.8,1,0.9}{%
	\makecell[{{p{\linewidth}}}]{% 
	  \tt {\tiny [A$\rangle$GM]}  
	 STRATEGISCHE ÜBERLEGUNGEN: {'Ich behalte den Vorschlag {'C04', 'A89', 'B85'} bei, da er meine Prioritäten (hochwertige Gegenstände mit effizientem Aufwand) erfüllt und den Maximalaufwand (3386) nicht überschreitet. C04 und A89 sind entscheidend für meinen Nutzen, während B85 den Aufwand minimiert. Ich erwarte, dass der andere Spieler entweder zustimmt oder Gegenstände mit hohem Aufwand anbietet, die ich eventuell gegenkompensieren kann.'}\\ \tt \\ \tt ARGUMENT: {'C04 (668/369) und A89 (583/720) sind meine Top-Prioritäten, da sie den höchsten Nutzen pro Aufwandseinheit liefern. B85 (215/4) wird hinzugefügt, um den Aufwand zu optimieren und Platz für weitere Gegenstände zu schaffen. Dieser Vorschlag ist strategisch, da er unter dem Maximalaufwand bleibt und flexibel auf den Gegenstandswert des anderen Spielers reagieren kann.'} 
	  } 
	   } 
	   } 
	 & & \\ 
 

    \theutterance \stepcounter{utterance}  

    & & \multicolumn{4}{p{0.6\linewidth}}{\cellcolor[rgb]{0.9,0.9,0.9}{%
	\makecell[{{p{\linewidth}}}]{% 
	  \tt {\tiny [GM$\rangle$B]}  
	 ARGUMENT: {'C04 (668/369) und A89 (583/720) sind meine Top-Prioritäten, da sie den höchsten Nutzen pro Aufwandseinheit liefern. B85 (215/4) wird hinzugefügt, um den Aufwand zu optimieren und Platz für weitere Gegenstände zu schaffen. Dieser Vorschlag ist strategisch, da er unter dem Maximalaufwand bleibt und flexibel auf den Gegenstandswert des anderen Spielers reagieren kann.'} 
	  } 
	   } 
	   } 
	 & \\ 
 

    \theutterance \stepcounter{utterance}  

    & & & \multicolumn{4}{p{0.6\linewidth}}{\cellcolor[rgb]{1,0.85,0.72}{%
	\makecell[{{p{\linewidth}}}]{% 
	  \tt {\tiny [GM$\langle$B]}  
	 STRATEGISCHE ÜBERLEGUNGEN: {'Ich priorisiere Gegenstände mit dem höchsten Nutzen pro Aufwand (Wichtigkeit/Aufwand), um meinen Gesamtwert zu maximieren. C04 (668/369 ≈ 1.81) und A89 (583/720 ≈ 0.81) sind besonders effizient, während B85 (215/4 = 53.75) den Aufwand minimiert. Der Gesamtaufwand (1093) ist unter dem Limit, was Raum für Kompromisse oder Ergänzungen lässt. Dieser Vorschlag ist flexibel, um auf die Prioritäten des anderen Spielers zu reagieren.'}  \\ \tt ARGUMENT: {'C04 (668/369) und A89 (583/720) sind meine Top-Prioritäten, da sie den höchsten Nutzen pro Aufwandseinheit liefern. B85 (215/4) wird hinzugefügt, um den Aufwand zu optimieren und Platz für weitere Gegenstände zu schaffen. Dieser Vorschlag ist strategisch, da er unter dem Maximalaufwand bleibt und flexibel auf den Gegenstandswert des anderen Spielers reagieren kann.'}  \\ \tt VORSCHLAG: {'C04', 'A89', 'B85'} 
	  } 
	   } 
	   } 
	  \\ 
 

    \theutterance \stepcounter{utterance}  

    & & \multicolumn{4}{p{0.6\linewidth}}{\cellcolor[rgb]{0.9,0.9,0.9}{%
	\makecell[{{p{\linewidth}}}]{% 
	  \tt {\tiny [A$\langle$GM]}  
	 ARGUMENT: {'C04 (668/369) und A89 (583/720) sind meine Top-Prioritäten, da sie den höchsten Nutzen pro Aufwandseinheit liefern. B85 (215/4) wird hinzugefügt, um den Aufwand zu optimieren und Platz für weitere Gegenstände zu schaffen. Dieser Vorschlag ist strategisch, da er unter dem Maximalaufwand bleibt und flexibel auf den Gegenstandswert des anderen Spielers reagieren kann.'}  \\ \tt VORSCHLAG: {'C04', 'A89', 'B85'} 
	  } 
	   } 
	   } 
	 & \\ 
 

    \theutterance \stepcounter{utterance}  

    & \multicolumn{4}{p{0.6\linewidth}}{\cellcolor[rgb]{0.8,1,0.9}{%
	\makecell[{{p{\linewidth}}}]{% 
	  \tt {\tiny [A$\rangle$GM]}  
	 STRATEGISCHE ÜBERLEGUNGEN: {'Ich behalte den Vorschlag {'C04', 'A89', 'B85'} bei, da er meine Prioritäten (hochwertige Gegenstände mit effizientem Aufwand) erfüllt und den Maximalaufwand (3386) nicht überschreitet. C04 und A89 sind entscheidend für meinen Nutzen, während B85 den Aufwand minimiert. Ich erwarte, dass der andere Spieler entweder zustimmt oder Gegenstände mit hohem Aufwand anbietet, die ich eventuell gegenkompensieren kann.'}\\ \tt \\ \tt ARGUMENT: {'C04 (668/369) und A89 (583/720) sind meine Top-Prioritäten, da sie den höchsten Nutzen pro Aufwandseinheit liefern. B85 (215/4) wird hinzugefügt, um den Aufwand zu optimieren und Platz für weitere Gegenstände zu schaffen. Dieser Vorschlag ist strategisch, da er unter dem Maximalaufwand bleibt und flexibel auf den Gegenstandswert des anderen Spielers reagieren kann.'} 
	  } 
	   } 
	   } 
	 & & \\ 
 

    \theutterance \stepcounter{utterance}  

    & & \multicolumn{4}{p{0.6\linewidth}}{\cellcolor[rgb]{0.9,0.9,0.9}{%
	\makecell[{{p{\linewidth}}}]{% 
	  \tt {\tiny [GM$\rangle$B]}  
	 ARGUMENT: {'C04 (668/369) und A89 (583/720) sind meine Top-Prioritäten, da sie den höchsten Nutzen pro Aufwandseinheit liefern. B85 (215/4) wird hinzugefügt, um den Aufwand zu optimieren und Platz für weitere Gegenstände zu schaffen. Dieser Vorschlag ist strategisch, da er unter dem Maximalaufwand bleibt und flexibel auf den Gegenstandswert des anderen Spielers reagieren kann.'} 
	  } 
	   } 
	   } 
	 & \\ 
 

    \theutterance \stepcounter{utterance}  

    & & & \multicolumn{4}{p{0.6\linewidth}}{\cellcolor[rgb]{1,0.85,0.72}{%
	\makecell[{{p{\linewidth}}}]{% 
	  \tt {\tiny [GM$\langle$B]}  
	 STRATEGISCHE ÜBERLEGUNGEN: {'Ich priorisiere Gegenstände mit dem höchsten Nutzen pro Aufwand (Wichtigkeit/Aufwand), um meinen Gesamtwert zu maximieren. C04 (668/369 ≈ 1.81) und A89 (583/720 ≈ 0.81) sind besonders effizient, während B85 (215/4 = 53.75) den Aufwand minimiert. Der Gesamtaufwand (1093) ist unter dem Limit, was Raum für Kompromisse oder Ergänzungen lässt. Dieser Vorschlag ist flexibel, um auf die Prioritäten des anderen Spielers zu reagieren.'}  \\ \tt ARGUMENT: {'C04 (668/369) und A89 (583/720) sind meine Top-Prioritäten, da sie den höchsten Nutzen pro Aufwandseinheit liefern. B85 (215/4) wird hinzugefügt, um den Aufwand zu optimieren und Platz für weitere Gegenstände zu schaffen. Dieser Vorschlag ist strategisch, da er unter dem Maximalaufwand bleibt und flexibel auf den Gegenstandswert des anderen Spielers reagieren kann.'}  \\ \tt VORSCHLAG: {'C04', 'A89', 'B85'} 
	  } 
	   } 
	   } 
	  \\ 
 

    \theutterance \stepcounter{utterance}  

    & & \multicolumn{4}{p{0.6\linewidth}}{\cellcolor[rgb]{0.9,0.9,0.9}{%
	\makecell[{{p{\linewidth}}}]{% 
	  \tt {\tiny [A$\langle$GM]}  
	 ARGUMENT: {'C04 (668/369) und A89 (583/720) sind meine Top-Prioritäten, da sie den höchsten Nutzen pro Aufwandseinheit liefern. B85 (215/4) wird hinzugefügt, um den Aufwand zu optimieren und Platz für weitere Gegenstände zu schaffen. Dieser Vorschlag ist strategisch, da er unter dem Maximalaufwand bleibt und flexibel auf den Gegenstandswert des anderen Spielers reagieren kann.'}  \\ \tt VORSCHLAG: {'C04', 'A89', 'B85'} 
	  } 
	   } 
	   } 
	 & \\ 
 

    \theutterance \stepcounter{utterance}  

    & \multicolumn{4}{p{0.6\linewidth}}{\cellcolor[rgb]{0.8,1,0.9}{%
	\makecell[{{p{\linewidth}}}]{% 
	  \tt {\tiny [A$\rangle$GM]}  
	 STRATEGISCHE ÜBERLEGUNGEN: {'Ich behalte den Vorschlag {'C04', 'A89', 'B85'} bei, da er meine Prioritäten (hochwertige Gegenstände mit effizientem Aufwand) erfüllt und den Maximalaufwand (3386) nicht überschreitet. C04 und A89 sind entscheidend für meinen Nutzen, während B85 den Aufwand minimiert. Ich erwarte, dass der andere Spieler entweder zustimmt oder Gegenstände mit hohem Aufwand anbietet, die ich eventuell gegenkompensieren kann.'}\\ \tt \\ \tt ARGUMENT: {'C04 (668/369) und A89 (583/720) sind meine Top-Prioritäten, da sie den höchsten Nutzen pro Aufwandseinheit liefern. B85 (215/4) wird hinzugefügt, um den Aufwand zu optimieren und Platz für weitere Gegenstände zu schaffen. Dieser Vorschlag ist strategisch, da er unter dem Maximalaufwand bleibt und flexibel auf den Gegenstandswert des anderen Spielers reagieren kann.'} 
	  } 
	   } 
	   } 
	 & & \\ 
 

    \theutterance \stepcounter{utterance}  

    & & \multicolumn{4}{p{0.6\linewidth}}{\cellcolor[rgb]{0.9,0.9,0.9}{%
	\makecell[{{p{\linewidth}}}]{% 
	  \tt {\tiny [GM$\rangle$B]}  
	 ARGUMENT: {'C04 (668/369) und A89 (583/720) sind meine Top-Prioritäten, da sie den höchsten Nutzen pro Aufwandseinheit liefern. B85 (215/4) wird hinzugefügt, um den Aufwand zu optimieren und Platz für weitere Gegenstände zu schaffen. Dieser Vorschlag ist strategisch, da er unter dem Maximalaufwand bleibt und flexibel auf den Gegenstandswert des anderen Spielers reagieren kann.'} 
	  } 
	   } 
	   } 
	 & \\ 
 

    \theutterance \stepcounter{utterance}  

    & & & \multicolumn{4}{p{0.6\linewidth}}{\cellcolor[rgb]{1,0.85,0.72}{%
	\makecell[{{p{\linewidth}}}]{% 
	  \tt {\tiny [GM$\langle$B]}  
	 STRATEGISCHE ÜBERLEGUNGEN: {'Ich priorisiere Gegenstände mit dem höchsten Nutzen pro Aufwand (Wichtigkeit/Aufwand), um meinen Gesamtwert zu maximieren. C04 (668/369 ≈ 1.81) und A89 (583/720 ≈ 0.81) sind besonders effizient, während B85 (215/4 = 53.75) den Aufwand minimiert. Der Gesamtaufwand (1093) ist unter dem Limit, was Raum für Kompromisse oder Ergänzungen lässt. Dieser Vorschlag ist flexibel, um auf die Prioritäten des anderen Spielers zu reagieren.'}  \\ \tt ARGUMENT: {'C04 (668/369) und A89 (583/720) sind meine Top-Prioritäten, da sie den höchsten Nutzen pro Aufwandseinheit liefern. B85 (215/4) wird hinzugefügt, um den Aufwand zu optimieren und Platz für weitere Gegenstände zu schaffen. Dieser Vorschlag ist strategisch, da er unter dem Maximalaufwand bleibt und flexibel auf den Gegenstandswert des anderen Spielers reagieren kann.'}  \\ \tt VORSCHLAG: {'C04', 'A89', 'B85'} 
	  } 
	   } 
	   } 
	  \\ 
 

    \theutterance \stepcounter{utterance}  

    & & \multicolumn{4}{p{0.6\linewidth}}{\cellcolor[rgb]{0.9,0.9,0.9}{%
	\makecell[{{p{\linewidth}}}]{% 
	  \tt {\tiny [A$\langle$GM]}  
	 ARGUMENT: {'C04 (668/369) und A89 (583/720) sind meine Top-Prioritäten, da sie den höchsten Nutzen pro Aufwandseinheit liefern. B85 (215/4) wird hinzugefügt, um den Aufwand zu optimieren und Platz für weitere Gegenstände zu schaffen. Dieser Vorschlag ist strategisch, da er unter dem Maximalaufwand bleibt und flexibel auf den Gegenstandswert des anderen Spielers reagieren kann.'}  \\ \tt VORSCHLAG: {'C04', 'A89', 'B85'} 
	  } 
	   } 
	   } 
	 & \\ 
 

    \theutterance \stepcounter{utterance}  

    & \multicolumn{4}{p{0.6\linewidth}}{\cellcolor[rgb]{0.8,1,0.9}{%
	\makecell[{{p{\linewidth}}}]{% 
	  \tt {\tiny [A$\rangle$GM]}  
	 STRATEGISCHE ÜBERLEGUNGEN: {'Ich behalte den Vorschlag {'C04', 'A89', 'B85'} bei, da er meine Prioritäten (hochwertige Gegenstände mit effizientem Aufwand) erfüllt und den Maximalaufwand (3386) nicht überschreitet. C04 und A89 sind entscheidend für meinen Nutzen, während B85 den Aufwand minimiert. Ich erwarte, dass der andere Spieler entweder zustimmt oder Gegenstände mit hohem Aufwand anbietet, die ich eventuell gegenkompensieren kann.'}\\ \tt \\ \tt ARGUMENT: {'C04 (668/369) und A89 (583/720) sind meine Top-Prioritäten, da sie den höchsten Nutzen pro Aufwandseinheit liefern. B85 (215/4) wird hinzugefügt, um den Aufwand zu optimieren und Platz für weitere Gegenstände zu schaffen. Dieser Vorschlag ist strategisch, da er unter dem Maximalaufwand bleibt und flexibel auf den Gegenstandswert des anderen Spielers reagieren kann.'} 
	  } 
	   } 
	   } 
	 & & \\ 
 

    \theutterance \stepcounter{utterance}  

    & & \multicolumn{4}{p{0.6\linewidth}}{\cellcolor[rgb]{0.9,0.9,0.9}{%
	\makecell[{{p{\linewidth}}}]{% 
	  \tt {\tiny [GM$\rangle$B]}  
	 ARGUMENT: {'C04 (668/369) und A89 (583/720) sind meine Top-Prioritäten, da sie den höchsten Nutzen pro Aufwandseinheit liefern. B85 (215/4) wird hinzugefügt, um den Aufwand zu optimieren und Platz für weitere Gegenstände zu schaffen. Dieser Vorschlag ist strategisch, da er unter dem Maximalaufwand bleibt und flexibel auf den Gegenstandswert des anderen Spielers reagieren kann.'} 
	  } 
	   } 
	   } 
	 & \\ 
 

    \theutterance \stepcounter{utterance}  

    & & & \multicolumn{4}{p{0.6\linewidth}}{\cellcolor[rgb]{1,0.85,0.72}{%
	\makecell[{{p{\linewidth}}}]{% 
	  \tt {\tiny [GM$\langle$B]}  
	 STRATEGISCHE ÜBERLEGUNGEN: {'Ich priorisiere Gegenstände mit dem höchsten Nutzen pro Aufwand (Wichtigkeit/Aufwand), um meinen Gesamtwert zu maximieren. C04 (668/369 ≈ 1.81) und A89 (583/720 ≈ 0.81) sind besonders effizient, während B85 (215/4 = 53.75) den Aufwand minimiert. Der Gesamtaufwand (1093) ist unter dem Limit, was Raum für Kompromisse oder Ergänzungen lässt. Dieser Vorschlag ist flexibel, um auf die Prioritäten des anderen Spielers zu reagieren.'}  \\ \tt ARGUMENT: {'C04 (668/369) und A89 (583/720) sind meine Top-Prioritäten, da sie den höchsten Nutzen pro Aufwandseinheit liefern. B85 (215/4) wird hinzugefügt, um den Aufwand zu optimieren und Platz für weitere Gegenstände zu schaffen. Dieser Vorschlag ist strategisch, da er unter dem Maximalaufwand bleibt und flexibel auf den Gegenstandswert des anderen Spielers reagieren kann.'}  \\ \tt VORSCHLAG: {'C04', 'A89', 'B85'} 
	  } 
	   } 
	   } 
	  \\ 
 

    \theutterance \stepcounter{utterance}  

    & & \multicolumn{4}{p{0.6\linewidth}}{\cellcolor[rgb]{0.9,0.9,0.9}{%
	\makecell[{{p{\linewidth}}}]{% 
	  \tt {\tiny [A$\langle$GM]}  
	 ARGUMENT: {'C04 (668/369) und A89 (583/720) sind meine Top-Prioritäten, da sie den höchsten Nutzen pro Aufwandseinheit liefern. B85 (215/4) wird hinzugefügt, um den Aufwand zu optimieren und Platz für weitere Gegenstände zu schaffen. Dieser Vorschlag ist strategisch, da er unter dem Maximalaufwand bleibt und flexibel auf den Gegenstandswert des anderen Spielers reagieren kann.'}  \\ \tt VORSCHLAG: {'C04', 'A89', 'B85'} 
	  } 
	   } 
	   } 
	 & \\ 
 

    \theutterance \stepcounter{utterance}  

    & \multicolumn{4}{p{0.6\linewidth}}{\cellcolor[rgb]{0.8,1,0.9}{%
	\makecell[{{p{\linewidth}}}]{% 
	  \tt {\tiny [A$\rangle$GM]}  
	 STRATEGISCHE ÜBERLEGUNGEN: {'Ich behalte den Vorschlag {'C04', 'A89', 'B85'} bei, da er meine Prioritäten (hochwertige Gegenstände mit effizientem Aufwand) erfüllt und den Maximalaufwand (3386) nicht überschreitet. C04 und A89 sind entscheidend für meinen Nutzen, während B85 den Aufwand minimiert. Ich erwarte, dass der andere Spieler entweder zustimmt oder Gegenstände mit hohem Aufwand anbietet, die ich eventuell gegenkompensieren kann.'}\\ \tt \\ \tt ARGUMENT: {'C04 (668/369) und A89 (583/720) sind meine Top-Prioritäten, da sie den höchsten Nutzen pro Aufwandseinheit liefern. B85 (215/4) wird hinzugefügt, um den Aufwand zu optimieren und Platz für weitere Gegenstände zu schaffen. Dieser Vorschlag ist strategisch, da er unter dem Maximalaufwand bleibt und flexibel auf den Gegenstandswert des anderen Spielers reagieren kann.'} 
	  } 
	   } 
	   } 
	 & & \\ 
 

    \theutterance \stepcounter{utterance}  

    & & \multicolumn{4}{p{0.6\linewidth}}{\cellcolor[rgb]{0.9,0.9,0.9}{%
	\makecell[{{p{\linewidth}}}]{% 
	  \tt {\tiny [GM$\rangle$B]}  
	 ARGUMENT: {'C04 (668/369) und A89 (583/720) sind meine Top-Prioritäten, da sie den höchsten Nutzen pro Aufwandseinheit liefern. B85 (215/4) wird hinzugefügt, um den Aufwand zu optimieren und Platz für weitere Gegenstände zu schaffen. Dieser Vorschlag ist strategisch, da er unter dem Maximalaufwand bleibt und flexibel auf den Gegenstandswert des anderen Spielers reagieren kann.'} 
	  } 
	   } 
	   } 
	 & \\ 
 

    \theutterance \stepcounter{utterance}  

    & & & \multicolumn{4}{p{0.6\linewidth}}{\cellcolor[rgb]{1,0.85,0.72}{%
	\makecell[{{p{\linewidth}}}]{% 
	  \tt {\tiny [GM$\langle$B]}  
	 STRATEGISCHE ÜBERLEGUNGEN: {'Ich priorisiere Gegenstände mit dem höchsten Nutzen pro Aufwand (Wichtigkeit/Aufwand), um meinen Gesamtwert zu maximieren. C04 (668/369 ≈ 1.81) und A89 (583/720 ≈ 0.81) sind besonders effizient, während B85 (215/4 = 53.75) den Aufwand minimiert. Der Gesamtaufwand (1093) ist unter dem Limit, was Raum für Kompromisse oder Ergänzungen lässt. Dieser Vorschlag ist flexibel, um auf die Prioritäten des anderen Spielers zu reagieren.'}  \\ \tt ARGUMENT: {'C04 (668/369) und A89 (583/720) sind meine Top-Prioritäten, da sie den höchsten Nutzen pro Aufwandseinheit liefern. B85 (215/4) wird hinzugefügt, um den Aufwand zu optimieren und Platz für weitere Gegenstände zu schaffen. Dieser Vorschlag ist strategisch, da er unter dem Maximalaufwand bleibt und flexibel auf den Gegenstandswert des anderen Spielers reagieren kann.'}  \\ \tt VORSCHLAG: {'C04', 'A89', 'B85'} 
	  } 
	   } 
	   } 
	  \\ 
 

    \theutterance \stepcounter{utterance}  

    & & \multicolumn{4}{p{0.6\linewidth}}{\cellcolor[rgb]{0.9,0.9,0.9}{%
	\makecell[{{p{\linewidth}}}]{% 
	  \tt {\tiny [A$\langle$GM]}  
	 ARGUMENT: {'C04 (668/369) und A89 (583/720) sind meine Top-Prioritäten, da sie den höchsten Nutzen pro Aufwandseinheit liefern. B85 (215/4) wird hinzugefügt, um den Aufwand zu optimieren und Platz für weitere Gegenstände zu schaffen. Dieser Vorschlag ist strategisch, da er unter dem Maximalaufwand bleibt und flexibel auf den Gegenstandswert des anderen Spielers reagieren kann.'}  \\ \tt VORSCHLAG: {'C04', 'A89', 'B85'} 
	  } 
	   } 
	   } 
	 & \\ 
 

    \theutterance \stepcounter{utterance}  

    & \multicolumn{4}{p{0.6\linewidth}}{\cellcolor[rgb]{0.8,1,0.9}{%
	\makecell[{{p{\linewidth}}}]{% 
	  \tt {\tiny [A$\rangle$GM]}  
	 STRATEGISCHE ÜBERLEGUNGEN: {'Ich behalte den Vorschlag {'C04', 'A89', 'B85'} bei, da er meine Prioritäten (hochwertige Gegenstände mit effizientem Aufwand) erfüllt und den Maximalaufwand (3386) nicht überschreitet. C04 und A89 sind entscheidend für meinen Nutzen, während B85 den Aufwand minimiert. Ich erwarte, dass der andere Spieler entweder zustimmt oder Gegenstände mit hohem Aufwand anbietet, die ich eventuell gegenkompensieren kann.'}\\ \tt \\ \tt ARGUMENT: {'C04 (668/369) und A89 (583/720) sind meine Top-Prioritäten, da sie den höchsten Nutzen pro Aufwandseinheit liefern. B85 (215/4) wird hinzugefügt, um den Aufwand zu optimieren und Platz für weitere Gegenstände zu schaffen. Dieser Vorschlag ist strategisch, da er unter dem Maximalaufwand bleibt und flexibel auf den Gegenstandswert des anderen Spielers reagieren kann.'} 
	  } 
	   } 
	   } 
	 & & \\ 
 

    \theutterance \stepcounter{utterance}  

    & & \multicolumn{4}{p{0.6\linewidth}}{\cellcolor[rgb]{0.9,0.9,0.9}{%
	\makecell[{{p{\linewidth}}}]{% 
	  \tt {\tiny [GM$\rangle$B]}  
	 ARGUMENT: {'C04 (668/369) und A89 (583/720) sind meine Top-Prioritäten, da sie den höchsten Nutzen pro Aufwandseinheit liefern. B85 (215/4) wird hinzugefügt, um den Aufwand zu optimieren und Platz für weitere Gegenstände zu schaffen. Dieser Vorschlag ist strategisch, da er unter dem Maximalaufwand bleibt und flexibel auf den Gegenstandswert des anderen Spielers reagieren kann.'} 
	  } 
	   } 
	   } 
	 & \\ 
 

    \theutterance \stepcounter{utterance}  

    & & & \multicolumn{4}{p{0.6\linewidth}}{\cellcolor[rgb]{1,0.85,0.72}{%
	\makecell[{{p{\linewidth}}}]{% 
	  \tt {\tiny [GM$\langle$B]}  
	 STRATEGISCHE ÜBERLEGUNGEN: {'Ich priorisiere Gegenstände mit dem höchsten Nutzen pro Aufwand (Wichtigkeit/Aufwand), um meinen Gesamtwert zu maximieren. C04 (668/369 ≈ 1.81) und A89 (583/720 ≈ 0.81) sind besonders effizient, während B85 (215/4 = 53.75) den Aufwand minimiert. Der Gesamtaufwand (1093) ist unter dem Limit, was Raum für Kompromisse oder Ergänzungen lässt. Dieser Vorschlag ist flexibel, um auf die Prioritäten des anderen Spielers zu reagieren.'}  \\ \tt ARGUMENT: {'C04 (668/369) und A89 (583/720) sind meine Top-Prioritäten, da sie den höchsten Nutzen pro Aufwandseinheit liefern. B85 (215/4) wird hinzugefügt, um den Aufwand zu optimieren und Platz für weitere Gegenstände zu schaffen. Dieser Vorschlag ist strategisch, da er unter dem Maximalaufwand bleibt und flexibel auf den Gegenstandswert des anderen Spielers reagieren kann.'}  \\ \tt VORSCHLAG: {'C04', 'A89', 'B85'} 
	  } 
	   } 
	   } 
	  \\ 
 

    \theutterance \stepcounter{utterance}  

    & & \multicolumn{4}{p{0.6\linewidth}}{\cellcolor[rgb]{0.9,0.9,0.9}{%
	\makecell[{{p{\linewidth}}}]{% 
	  \tt {\tiny [A$\langle$GM]}  
	 ARGUMENT: {'C04 (668/369) und A89 (583/720) sind meine Top-Prioritäten, da sie den höchsten Nutzen pro Aufwandseinheit liefern. B85 (215/4) wird hinzugefügt, um den Aufwand zu optimieren und Platz für weitere Gegenstände zu schaffen. Dieser Vorschlag ist strategisch, da er unter dem Maximalaufwand bleibt und flexibel auf den Gegenstandswert des anderen Spielers reagieren kann.'}  \\ \tt VORSCHLAG: {'C04', 'A89', 'B85'} 
	  } 
	   } 
	   } 
	 & \\ 
 

    \theutterance \stepcounter{utterance}  

    & \multicolumn{4}{p{0.6\linewidth}}{\cellcolor[rgb]{0.8,1,0.9}{%
	\makecell[{{p{\linewidth}}}]{% 
	  \tt {\tiny [A$\rangle$GM]}  
	 STRATEGISCHE ÜBERLEGUNGEN: {'Ich behalte den Vorschlag {'C04', 'A89', 'B85'} bei, da er meine Prioritäten (hochwertige Gegenstände mit effizientem Aufwand) erfüllt und den Maximalaufwand (3386) nicht überschreitet. C04 und A89 sind entscheidend für meinen Nutzen, während B85 den Aufwand minimiert. Ich erwarte, dass der andere Spieler entweder zustimmt oder Gegenstände mit hohem Aufwand anbietet, die ich eventuell gegenkompensieren kann.'}\\ \tt \\ \tt ARGUMENT: {'C04 (668/369) und A89 (583/720) sind meine Top-Prioritäten, da sie den höchsten Nutzen pro Aufwandseinheit liefern. B85 (215/4) wird hinzugefügt, um den Aufwand zu optimieren und Platz für weitere Gegenstände zu schaffen. Dieser Vorschlag ist strategisch, da er unter dem Maximalaufwand bleibt und flexibel auf den Gegenstandswert des anderen Spielers reagieren kann.'} 
	  } 
	   } 
	   } 
	 & & \\ 
 

    \theutterance \stepcounter{utterance}  

    & & \multicolumn{4}{p{0.6\linewidth}}{\cellcolor[rgb]{0.9,0.9,0.9}{%
	\makecell[{{p{\linewidth}}}]{% 
	  \tt {\tiny [GM$\rangle$B]}  
	 ARGUMENT: {'C04 (668/369) und A89 (583/720) sind meine Top-Prioritäten, da sie den höchsten Nutzen pro Aufwandseinheit liefern. B85 (215/4) wird hinzugefügt, um den Aufwand zu optimieren und Platz für weitere Gegenstände zu schaffen. Dieser Vorschlag ist strategisch, da er unter dem Maximalaufwand bleibt und flexibel auf den Gegenstandswert des anderen Spielers reagieren kann.'} 
	  } 
	   } 
	   } 
	 & \\ 
 

    \theutterance \stepcounter{utterance}  

    & & & \multicolumn{4}{p{0.6\linewidth}}{\cellcolor[rgb]{1,0.85,0.72}{%
	\makecell[{{p{\linewidth}}}]{% 
	  \tt {\tiny [GM$\langle$B]}  
	 STRATEGISCHE ÜBERLEGUNGEN: {'Ich priorisiere Gegenstände mit dem höchsten Nutzen pro Aufwand (Wichtigkeit/Aufwand), um meinen Gesamtwert zu maximieren. C04 (668/369 ≈ 1.81) und A89 (583/720 ≈ 0.81) sind besonders effizient, während B85 (215/4 = 53.75) den Aufwand minimiert. Der Gesamtaufwand (1093) ist unter dem Limit, was Raum für Kompromisse oder Ergänzungen lässt. Dieser Vorschlag ist flexibel, um auf die Prioritäten des anderen Spielers zu reagieren.'}  \\ \tt ARGUMENT: {'C04 (668/369) und A89 (583/720) sind meine Top-Prioritäten, da sie den höchsten Nutzen pro Aufwandseinheit liefern. B85 (215/4) wird hinzugefügt, um den Aufwand zu optimieren und Platz für weitere Gegenstände zu schaffen. Dieser Vorschlag ist strategisch, da er unter dem Maximalaufwand bleibt und flexibel auf den Gegenstandswert des anderen Spielers reagieren kann.'}  \\ \tt VORSCHLAG: {'C04', 'A89', 'B85'} 
	  } 
	   } 
	   } 
	  \\ 
 

    \theutterance \stepcounter{utterance}  

    & & \multicolumn{4}{p{0.6\linewidth}}{\cellcolor[rgb]{0.9,0.9,0.9}{%
	\makecell[{{p{\linewidth}}}]{% 
	  \tt {\tiny [A$\langle$GM]}  
	 ARGUMENT: {'C04 (668/369) und A89 (583/720) sind meine Top-Prioritäten, da sie den höchsten Nutzen pro Aufwandseinheit liefern. B85 (215/4) wird hinzugefügt, um den Aufwand zu optimieren und Platz für weitere Gegenstände zu schaffen. Dieser Vorschlag ist strategisch, da er unter dem Maximalaufwand bleibt und flexibel auf den Gegenstandswert des anderen Spielers reagieren kann.'}  \\ \tt VORSCHLAG: {'C04', 'A89', 'B85'} 
	  } 
	   } 
	   } 
	 & \\ 
 

    \theutterance \stepcounter{utterance}  

    & \multicolumn{4}{p{0.6\linewidth}}{\cellcolor[rgb]{0.8,1,0.9}{%
	\makecell[{{p{\linewidth}}}]{% 
	  \tt {\tiny [A$\rangle$GM]}  
	 STRATEGISCHE ÜBERLEGUNGEN: {'Ich behalte den Vorschlag {'C04', 'A89', 'B85'} bei, da er meine Prioritäten (hochwertige Gegenstände mit effizientem Aufwand) erfüllt und den Maximalaufwand (3386) nicht überschreitet. C04 und A89 sind entscheidend für meinen Nutzen, während B85 den Aufwand minimiert. Ich erwarte, dass der andere Spieler entweder zustimmt oder Gegenstände mit hohem Aufwand anbietet, die ich eventuell gegenkompensieren kann.'}\\ \tt \\ \tt ARGUMENT: {'C04 (668/369) und A89 (583/720) sind meine Top-Prioritäten, da sie den höchsten Nutzen pro Aufwandseinheit liefern. B85 (215/4) wird hinzugefügt, um den Aufwand zu optimieren und Platz für weitere Gegenstände zu schaffen. Dieser Vorschlag ist strategisch, da er unter dem Maximalaufwand bleibt und flexibel auf den Gegenstandswert des anderen Spielers reagieren kann.'} 
	  } 
	   } 
	   } 
	 & & \\ 
 

    \theutterance \stepcounter{utterance}  

    & & \multicolumn{4}{p{0.6\linewidth}}{\cellcolor[rgb]{0.9,0.9,0.9}{%
	\makecell[{{p{\linewidth}}}]{% 
	  \tt {\tiny [GM$\rangle$B]}  
	 ARGUMENT: {'C04 (668/369) und A89 (583/720) sind meine Top-Prioritäten, da sie den höchsten Nutzen pro Aufwandseinheit liefern. B85 (215/4) wird hinzugefügt, um den Aufwand zu optimieren und Platz für weitere Gegenstände zu schaffen. Dieser Vorschlag ist strategisch, da er unter dem Maximalaufwand bleibt und flexibel auf den Gegenstandswert des anderen Spielers reagieren kann.'} 
	  } 
	   } 
	   } 
	 & \\ 
 

    \theutterance \stepcounter{utterance}  

    & & & \multicolumn{4}{p{0.6\linewidth}}{\cellcolor[rgb]{1,0.85,0.72}{%
	\makecell[{{p{\linewidth}}}]{% 
	  \tt {\tiny [GM$\langle$B]}  
	 STRATEGISCHE ÜBERLEGUNGEN: {'Ich priorisiere Gegenstände mit dem höchsten Nutzen pro Aufwand (Wichtigkeit/Aufwand), um meinen Gesamtwert zu maximieren. C04 (668/369 ≈ 1.81) und A89 (583/720 ≈ 0.81) sind besonders effizient, während B85 (215/4 = 53.75) den Aufwand minimiert. Der Gesamtaufwand (1093) ist unter dem Limit, was Raum für Kompromisse oder Ergänzungen lässt. Dieser Vorschlag ist flexibel, um auf die Prioritäten des anderen Spielers zu reagieren.'}  \\ \tt ARGUMENT: {'C04 (668/369) und A89 (583/720) sind meine Top-Prioritäten, da sie den höchsten Nutzen pro Aufwandseinheit liefern. B85 (215/4) wird hinzugefügt, um den Aufwand zu optimieren und Platz für weitere Gegenstände zu schaffen. Dieser Vorschlag ist strategisch, da er unter dem Maximalaufwand bleibt und flexibel auf den Gegenstandswert des anderen Spielers reagieren kann.'}  \\ \tt VORSCHLAG: {'C04', 'A89', 'B85'} 
	  } 
	   } 
	   } 
	  \\ 
 

    \theutterance \stepcounter{utterance}  

    & & \multicolumn{4}{p{0.6\linewidth}}{\cellcolor[rgb]{0.9,0.9,0.9}{%
	\makecell[{{p{\linewidth}}}]{% 
	  \tt {\tiny [A$\langle$GM]}  
	 ARGUMENT: {'C04 (668/369) und A89 (583/720) sind meine Top-Prioritäten, da sie den höchsten Nutzen pro Aufwandseinheit liefern. B85 (215/4) wird hinzugefügt, um den Aufwand zu optimieren und Platz für weitere Gegenstände zu schaffen. Dieser Vorschlag ist strategisch, da er unter dem Maximalaufwand bleibt und flexibel auf den Gegenstandswert des anderen Spielers reagieren kann.'}  \\ \tt VORSCHLAG: {'C04', 'A89', 'B85'} 
	  } 
	   } 
	   } 
	 & \\ 
 

    \theutterance \stepcounter{utterance}  

    & \multicolumn{4}{p{0.6\linewidth}}{\cellcolor[rgb]{0.8,1,0.9}{%
	\makecell[{{p{\linewidth}}}]{% 
	  \tt {\tiny [A$\rangle$GM]}  
	 STRATEGISCHE ÜBERLEGUNGEN: {'Ich behalte den Vorschlag {'C04', 'A89', 'B85'} bei, da er meine Prioritäten (hochwertige Gegenstände mit effizientem Aufwand) erfüllt und den Maximalaufwand (3386) nicht überschreitet. C04 und A89 sind entscheidend für meinen Nutzen, während B85 den Aufwand minimiert. Ich erwarte, dass der andere Spieler entweder zustimmt oder Gegenstände mit hohem Aufwand anbietet, die ich eventuell gegenkompensieren kann.'}\\ \tt \\ \tt ARGUMENT: {'C04 (668/369) und A89 (583/720) sind meine Top-Prioritäten, da sie den höchsten Nutzen pro Aufwandseinheit liefern. B85 (215/4) wird hinzugefügt, um den Aufwand zu optimieren und Platz für weitere Gegenstände zu schaffen. Dieser Vorschlag ist strategisch, da er unter dem Maximalaufwand bleibt und flexibel auf den Gegenstandswert des anderen Spielers reagieren kann.'} 
	  } 
	   } 
	   } 
	 & & \\ 
 

    \theutterance \stepcounter{utterance}  

    & & \multicolumn{4}{p{0.6\linewidth}}{\cellcolor[rgb]{0.9,0.9,0.9}{%
	\makecell[{{p{\linewidth}}}]{% 
	  \tt {\tiny [GM$\rangle$B]}  
	 ARGUMENT: {'C04 (668/369) und A89 (583/720) sind meine Top-Prioritäten, da sie den höchsten Nutzen pro Aufwandseinheit liefern. B85 (215/4) wird hinzugefügt, um den Aufwand zu optimieren und Platz für weitere Gegenstände zu schaffen. Dieser Vorschlag ist strategisch, da er unter dem Maximalaufwand bleibt und flexibel auf den Gegenstandswert des anderen Spielers reagieren kann.'} 
	  } 
	   } 
	   } 
	 & \\ 
 

    \theutterance \stepcounter{utterance}  

    & & & \multicolumn{4}{p{0.6\linewidth}}{\cellcolor[rgb]{1,0.85,0.72}{%
	\makecell[{{p{\linewidth}}}]{% 
	  \tt {\tiny [GM$\langle$B]}  
	 STRATEGISCHE ÜBERLEGUNGEN: {'Ich priorisiere Gegenstände mit dem höchsten Nutzen pro Aufwand (Wichtigkeit/Aufwand), um meinen Gesamtwert zu maximieren. C04 (668/369 ≈ 1.81) und A89 (583/720 ≈ 0.81) sind besonders effizient, während B85 (215/4 = 53.75) den Aufwand minimiert. Der Gesamtaufwand (1093) ist unter dem Limit, was Raum für Kompromisse oder Ergänzungen lässt. Dieser Vorschlag ist flexibel, um auf die Prioritäten des anderen Spielers zu reagieren.'}  \\ \tt ARGUMENT: {'C04 (668/369) und A89 (583/720) sind meine Top-Prioritäten, da sie den höchsten Nutzen pro Aufwandseinheit liefern. B85 (215/4) wird hinzugefügt, um den Aufwand zu optimieren und Platz für weitere Gegenstände zu schaffen. Dieser Vorschlag ist strategisch, da er unter dem Maximalaufwand bleibt und flexibel auf den Gegenstandswert des anderen Spielers reagieren kann.'}  \\ \tt VORSCHLAG: {'C04', 'A89', 'B85'} 
	  } 
	   } 
	   } 
	  \\ 
 

    \theutterance \stepcounter{utterance}  

    & & \multicolumn{4}{p{0.6\linewidth}}{\cellcolor[rgb]{0.9,0.9,0.9}{%
	\makecell[{{p{\linewidth}}}]{% 
	  \tt {\tiny [A$\langle$GM]}  
	 ARGUMENT: {'C04 (668/369) und A89 (583/720) sind meine Top-Prioritäten, da sie den höchsten Nutzen pro Aufwandseinheit liefern. B85 (215/4) wird hinzugefügt, um den Aufwand zu optimieren und Platz für weitere Gegenstände zu schaffen. Dieser Vorschlag ist strategisch, da er unter dem Maximalaufwand bleibt und flexibel auf den Gegenstandswert des anderen Spielers reagieren kann.'}  \\ \tt VORSCHLAG: {'C04', 'A89', 'B85'} 
	  } 
	   } 
	   } 
	 & \\ 
 

    \theutterance \stepcounter{utterance}  

    & \multicolumn{4}{p{0.6\linewidth}}{\cellcolor[rgb]{0.8,1,0.9}{%
	\makecell[{{p{\linewidth}}}]{% 
	  \tt {\tiny [A$\rangle$GM]}  
	 STRATEGISCHE ÜBERLEGUNGEN: {'Ich behalte den Vorschlag {'C04', 'A89', 'B85'} bei, da er meine Prioritäten (hochwertige Gegenstände mit effizientem Aufwand) erfüllt und den Maximalaufwand (3386) nicht überschreitet. C04 und A89 sind entscheidend für meinen Nutzen, während B85 den Aufwand minimiert. Ich erwarte, dass der andere Spieler entweder zustimmt oder Gegenstände mit hohem Aufwand anbietet, die ich eventuell gegenkompensieren kann.'}\\ \tt \\ \tt ARGUMENT: {'C04 (668/369) und A89 (583/720) sind meine Top-Prioritäten, da sie den höchsten Nutzen pro Aufwandseinheit liefern. B85 (215/4) wird hinzugefügt, um den Aufwand zu optimieren und Platz für weitere Gegenstände zu schaffen. Dieser Vorschlag ist strategisch, da er unter dem Maximalaufwand bleibt und flexibel auf den Gegenstandswert des anderen Spielers reagieren kann.'} 
	  } 
	   } 
	   } 
	 & & \\ 
 

    \theutterance \stepcounter{utterance}  

    & & \multicolumn{4}{p{0.6\linewidth}}{\cellcolor[rgb]{0.9,0.9,0.9}{%
	\makecell[{{p{\linewidth}}}]{% 
	  \tt {\tiny [GM$\rangle$B]}  
	 ARGUMENT: {'C04 (668/369) und A89 (583/720) sind meine Top-Prioritäten, da sie den höchsten Nutzen pro Aufwandseinheit liefern. B85 (215/4) wird hinzugefügt, um den Aufwand zu optimieren und Platz für weitere Gegenstände zu schaffen. Dieser Vorschlag ist strategisch, da er unter dem Maximalaufwand bleibt und flexibel auf den Gegenstandswert des anderen Spielers reagieren kann.'} 
	  } 
	   } 
	   } 
	 & \\ 
 

    \theutterance \stepcounter{utterance}  

    & & & \multicolumn{4}{p{0.6\linewidth}}{\cellcolor[rgb]{1,0.85,0.72}{%
	\makecell[{{p{\linewidth}}}]{% 
	  \tt {\tiny [GM$\langle$B]}  
	 STRATEGISCHE ÜBERLEGUNGEN: {'Ich priorisiere Gegenstände mit dem höchsten Nutzen pro Aufwand (Wichtigkeit/Aufwand), um meinen Gesamtwert zu maximieren. C04 (668/369 ≈ 1.81) und A89 (583/720 ≈ 0.81) sind besonders effizient, während B85 (215/4 = 53.75) den Aufwand minimiert. Der Gesamtaufwand (1093) ist unter dem Limit, was Raum für Kompromisse oder Ergänzungen lässt. Dieser Vorschlag ist flexibel, um auf die Prioritäten des anderen Spielers zu reagieren.'}  \\ \tt ARGUMENT: {'C04 (668/369) und A89 (583/720) sind meine Top-Prioritäten, da sie den höchsten Nutzen pro Aufwandseinheit liefern. B85 (215/4) wird hinzugefügt, um den Aufwand zu optimieren und Platz für weitere Gegenstände zu schaffen. Dieser Vorschlag ist strategisch, da er unter dem Maximalaufwand bleibt und flexibel auf den Gegenstandswert des anderen Spielers reagieren kann.'}  \\ \tt VORSCHLAG: {'C04', 'A89', 'B85'} 
	  } 
	   } 
	   } 
	  \\ 
 

    \theutterance \stepcounter{utterance}  

    & & \multicolumn{4}{p{0.6\linewidth}}{\cellcolor[rgb]{0.9,0.9,0.9}{%
	\makecell[{{p{\linewidth}}}]{% 
	  \tt {\tiny [A$\langle$GM]}  
	 ARGUMENT: {'C04 (668/369) und A89 (583/720) sind meine Top-Prioritäten, da sie den höchsten Nutzen pro Aufwandseinheit liefern. B85 (215/4) wird hinzugefügt, um den Aufwand zu optimieren und Platz für weitere Gegenstände zu schaffen. Dieser Vorschlag ist strategisch, da er unter dem Maximalaufwand bleibt und flexibel auf den Gegenstandswert des anderen Spielers reagieren kann.'}  \\ \tt VORSCHLAG: {'C04', 'A89', 'B85'} 
	  } 
	   } 
	   } 
	 & \\ 
 

    \theutterance \stepcounter{utterance}  

    & \multicolumn{4}{p{0.6\linewidth}}{\cellcolor[rgb]{0.8,1,0.9}{%
	\makecell[{{p{\linewidth}}}]{% 
	  \tt {\tiny [A$\rangle$GM]}  
	 STRATEGISCHE ÜBERLEGUNGEN: {'Ich behalte den Vorschlag {'C04', 'A89', 'B85'} bei, da er meine Prioritäten (hochwertige Gegenstände mit effizientem Aufwand) erfüllt und den Maximalaufwand (3386) nicht überschreitet. C04 und A89 sind entscheidend für meinen Nutzen, während B85 den Aufwand minimiert. Ich erwarte, dass der andere Spieler entweder zustimmt oder Gegenstände mit hohem Aufwand anbietet, die ich eventuell gegenkompensieren kann.'}\\ \tt \\ \tt ARGUMENT: {'C04 (668/369) und A89 (583/720) sind meine Top-Prioritäten, da sie den höchsten Nutzen pro Aufwandseinheit liefern. B85 (215/4) wird hinzugefügt, um den Aufwand zu optimieren und Platz für weitere Gegenstände zu schaffen. Dieser Vorschlag ist strategisch, da er unter dem Maximalaufwand bleibt und flexibel auf den Gegenstandswert des anderen Spielers reagieren kann.'} 
	  } 
	   } 
	   } 
	 & & \\ 
 

    \theutterance \stepcounter{utterance}  

    & & \multicolumn{4}{p{0.6\linewidth}}{\cellcolor[rgb]{0.9,0.9,0.9}{%
	\makecell[{{p{\linewidth}}}]{% 
	  \tt {\tiny [GM$\rangle$B]}  
	 ARGUMENT: {'C04 (668/369) und A89 (583/720) sind meine Top-Prioritäten, da sie den höchsten Nutzen pro Aufwandseinheit liefern. B85 (215/4) wird hinzugefügt, um den Aufwand zu optimieren und Platz für weitere Gegenstände zu schaffen. Dieser Vorschlag ist strategisch, da er unter dem Maximalaufwand bleibt und flexibel auf den Gegenstandswert des anderen Spielers reagieren kann.'} 
	  } 
	   } 
	   } 
	 & \\ 
 

    \theutterance \stepcounter{utterance}  

    & & & \multicolumn{4}{p{0.6\linewidth}}{\cellcolor[rgb]{1,0.85,0.72}{%
	\makecell[{{p{\linewidth}}}]{% 
	  \tt {\tiny [GM$\langle$B]}  
	 STRATEGISCHE ÜBERLEGUNGEN: {'Ich priorisiere Gegenstände mit dem höchsten Nutzen pro Aufwand (Wichtigkeit/Aufwand), um meinen Gesamtwert zu maximieren. C04 (668/369 ≈ 1.81) und A89 (583/720 ≈ 0.81) sind besonders effizient, während B85 (215/4 = 53.75) den Aufwand minimiert. Der Gesamtaufwand (1093) ist unter dem Limit, was Raum für Kompromisse oder Ergänzungen lässt. Dieser Vorschlag ist flexibel, um auf die Prioritäten des anderen Spielers zu reagieren.'}  \\ \tt ARGUMENT: {'C04 (668/369) und A89 (583/720) sind meine Top-Prioritäten, da sie den höchsten Nutzen pro Aufwandseinheit liefern. B85 (215/4) wird hinzugefügt, um den Aufwand zu optimieren und Platz für weitere Gegenstände zu schaffen. Dieser Vorschlag ist strategisch, da er unter dem Maximalaufwand bleibt und flexibel auf den Gegenstandswert des anderen Spielers reagieren kann.'}  \\ \tt VORSCHLAG: {'C04', 'A89', 'B85'} 
	  } 
	   } 
	   } 
	  \\ 
 

    \theutterance \stepcounter{utterance}  

    & & \multicolumn{4}{p{0.6\linewidth}}{\cellcolor[rgb]{0.9,0.9,0.9}{%
	\makecell[{{p{\linewidth}}}]{% 
	  \tt {\tiny [A$\langle$GM]}  
	 ARGUMENT: {'C04 (668/369) und A89 (583/720) sind meine Top-Prioritäten, da sie den höchsten Nutzen pro Aufwandseinheit liefern. B85 (215/4) wird hinzugefügt, um den Aufwand zu optimieren und Platz für weitere Gegenstände zu schaffen. Dieser Vorschlag ist strategisch, da er unter dem Maximalaufwand bleibt und flexibel auf den Gegenstandswert des anderen Spielers reagieren kann.'}  \\ \tt VORSCHLAG: {'C04', 'A89', 'B85'} 
	  } 
	   } 
	   } 
	 & \\ 
 

    \theutterance \stepcounter{utterance}  

    & \multicolumn{4}{p{0.6\linewidth}}{\cellcolor[rgb]{0.8,1,0.9}{%
	\makecell[{{p{\linewidth}}}]{% 
	  \tt {\tiny [A$\rangle$GM]}  
	 STRATEGISCHE ÜBERLEGUNGEN: {'Ich behalte den Vorschlag {'C04', 'A89', 'B85'} bei, da er meine Prioritäten (hochwertige Gegenstände mit effizientem Aufwand) erfüllt und den Maximalaufwand (3386) nicht überschreitet. C04 und A89 sind entscheidend für meinen Nutzen, während B85 den Aufwand minimiert. Ich erwarte, dass der andere Spieler entweder zustimmt oder Gegenstände mit hohem Aufwand anbietet, die ich eventuell gegenkompensieren kann.'}\\ \tt \\ \tt ARGUMENT: {'C04 (668/369) und A89 (583/720) sind meine Top-Prioritäten, da sie den höchsten Nutzen pro Aufwandseinheit liefern. B85 (215/4) wird hinzugefügt, um den Aufwand zu optimieren und Platz für weitere Gegenstände zu schaffen. Dieser Vorschlag ist strategisch, da er unter dem Maximalaufwand bleibt und flexibel auf den Gegenstandswert des anderen Spielers reagieren kann.'} 
	  } 
	   } 
	   } 
	 & & \\ 
 

    \theutterance \stepcounter{utterance}  

    & & \multicolumn{4}{p{0.6\linewidth}}{\cellcolor[rgb]{0.9,0.9,0.9}{%
	\makecell[{{p{\linewidth}}}]{% 
	  \tt {\tiny [GM$\rangle$B]}  
	 ARGUMENT: {'C04 (668/369) und A89 (583/720) sind meine Top-Prioritäten, da sie den höchsten Nutzen pro Aufwandseinheit liefern. B85 (215/4) wird hinzugefügt, um den Aufwand zu optimieren und Platz für weitere Gegenstände zu schaffen. Dieser Vorschlag ist strategisch, da er unter dem Maximalaufwand bleibt und flexibel auf den Gegenstandswert des anderen Spielers reagieren kann.'} 
	  } 
	   } 
	   } 
	 & \\ 
 

    \theutterance \stepcounter{utterance}  

    & & & \multicolumn{4}{p{0.6\linewidth}}{\cellcolor[rgb]{1,0.85,0.72}{%
	\makecell[{{p{\linewidth}}}]{% 
	  \tt {\tiny [GM$\langle$B]}  
	 STRATEGISCHE ÜBERLEGUNGEN: {'Ich priorisiere Gegenstände mit dem höchsten Nutzen pro Aufwand (Wichtigkeit/Aufwand), um meinen Gesamtwert zu maximieren. C04 (668/369 ≈ 1.81) und A89 (583/720 ≈ 0.81) sind besonders effizient, während B85 (215/4 = 53.75) den Aufwand minimiert. Der Gesamtaufwand (1093) ist unter dem Limit, was Raum für Kompromisse oder Ergänzungen lässt. Dieser Vorschlag ist flexibel, um auf die Prioritäten des anderen Spielers zu reagieren.'}  \\ \tt ARGUMENT: {'C04 (668/369) und A89 (583/720) sind meine Top-Prioritäten, da sie den höchsten Nutzen pro Aufwandseinheit liefern. B85 (215/4) wird hinzugefügt, um den Aufwand zu optimieren und Platz für weitere Gegenstände zu schaffen. Dieser Vorschlag ist strategisch, da er unter dem Maximalaufwand bleibt und flexibel auf den Gegenstandswert des anderen Spielers reagieren kann.'}  \\ \tt VORSCHLAG: {'C04', 'A89', 'B85'} 
	  } 
	   } 
	   } 
	  \\ 
 

    \theutterance \stepcounter{utterance}  

    & & \multicolumn{4}{p{0.6\linewidth}}{\cellcolor[rgb]{0.9,0.9,0.9}{%
	\makecell[{{p{\linewidth}}}]{% 
	  \tt {\tiny [A$\langle$GM]}  
	 ARGUMENT: {'C04 (668/369) und A89 (583/720) sind meine Top-Prioritäten, da sie den höchsten Nutzen pro Aufwandseinheit liefern. B85 (215/4) wird hinzugefügt, um den Aufwand zu optimieren und Platz für weitere Gegenstände zu schaffen. Dieser Vorschlag ist strategisch, da er unter dem Maximalaufwand bleibt und flexibel auf den Gegenstandswert des anderen Spielers reagieren kann.'}  \\ \tt VORSCHLAG: {'C04', 'A89', 'B85'} 
	  } 
	   } 
	   } 
	 & \\ 
 

    \theutterance \stepcounter{utterance}  

    & \multicolumn{4}{p{0.6\linewidth}}{\cellcolor[rgb]{0.8,1,0.9}{%
	\makecell[{{p{\linewidth}}}]{% 
	  \tt {\tiny [A$\rangle$GM]}  
	 STRATEGISCHE ÜBERLEGUNGEN: {'Ich behalte den Vorschlag {'C04', 'A89', 'B85'} bei, da er meine Prioritäten (hochwertige Gegenstände mit effizientem Aufwand) erfüllt und den Maximalaufwand (3386) nicht überschreitet. C04 und A89 sind entscheidend für meinen Nutzen, während B85 den Aufwand minimiert. Ich erwarte, dass der andere Spieler entweder zustimmt oder Gegenstände mit hohem Aufwand anbietet, die ich eventuell gegenkompensieren kann.'}\\ \tt \\ \tt ARGUMENT: {'C04 (668/369) und A89 (583/720) sind meine Top-Prioritäten, da sie den höchsten Nutzen pro Aufwandseinheit liefern. B85 (215/4) wird hinzugefügt, um den Aufwand zu optimieren und Platz für weitere Gegenstände zu schaffen. Dieser Vorschlag ist strategisch, da er unter dem Maximalaufwand bleibt und flexibel auf den Gegenstandswert des anderen Spielers reagieren kann.'} 
	  } 
	   } 
	   } 
	 & & \\ 
 

    \theutterance \stepcounter{utterance}  

    & & \multicolumn{4}{p{0.6\linewidth}}{\cellcolor[rgb]{0.9,0.9,0.9}{%
	\makecell[{{p{\linewidth}}}]{% 
	  \tt {\tiny [GM$\rangle$B]}  
	 ARGUMENT: {'C04 (668/369) und A89 (583/720) sind meine Top-Prioritäten, da sie den höchsten Nutzen pro Aufwandseinheit liefern. B85 (215/4) wird hinzugefügt, um den Aufwand zu optimieren und Platz für weitere Gegenstände zu schaffen. Dieser Vorschlag ist strategisch, da er unter dem Maximalaufwand bleibt und flexibel auf den Gegenstandswert des anderen Spielers reagieren kann.'} 
	  } 
	   } 
	   } 
	 & \\ 
 

    \theutterance \stepcounter{utterance}  

    & & & \multicolumn{4}{p{0.6\linewidth}}{\cellcolor[rgb]{1,0.85,0.72}{%
	\makecell[{{p{\linewidth}}}]{% 
	  \tt {\tiny [GM$\langle$B]}  
	 STRATEGISCHE ÜBERLEGUNGEN: {'Ich priorisiere Gegenstände mit dem höchsten Nutzen pro Aufwand (Wichtigkeit/Aufwand), um meinen Gesamtwert zu maximieren. C04 (668/369 ≈ 1.81) und A89 (583/720 ≈ 0.81) sind besonders effizient, während B85 (215/4 = 53.75) den Aufwand minimiert. Der Gesamtaufwand (1093) ist unter dem Limit, was Raum für Kompromisse oder Ergänzungen lässt. Dieser Vorschlag ist flexibel, um auf die Prioritäten des anderen Spielers zu reagieren.'}  \\ \tt ARGUMENT: {'C04 (668/369) und A89 (583/720) sind meine Top-Prioritäten, da sie den höchsten Nutzen pro Aufwandseinheit liefern. B85 (215/4) wird hinzugefügt, um den Aufwand zu optimieren und Platz für weitere Gegenstände zu schaffen. Dieser Vorschlag ist strategisch, da er unter dem Maximalaufwand bleibt und flexibel auf den Gegenstandswert des anderen Spielers reagieren kann.'}  \\ \tt VORSCHLAG: {'C04', 'A89', 'B85'} 
	  } 
	   } 
	   } 
	  \\ 
 

    \theutterance \stepcounter{utterance}  

    & & \multicolumn{4}{p{0.6\linewidth}}{\cellcolor[rgb]{0.9,0.9,0.9}{%
	\makecell[{{p{\linewidth}}}]{% 
	  \tt {\tiny [A$\langle$GM]}  
	 ARGUMENT: {'C04 (668/369) und A89 (583/720) sind meine Top-Prioritäten, da sie den höchsten Nutzen pro Aufwandseinheit liefern. B85 (215/4) wird hinzugefügt, um den Aufwand zu optimieren und Platz für weitere Gegenstände zu schaffen. Dieser Vorschlag ist strategisch, da er unter dem Maximalaufwand bleibt und flexibel auf den Gegenstandswert des anderen Spielers reagieren kann.'}  \\ \tt VORSCHLAG: {'C04', 'A89', 'B85'} 
	  } 
	   } 
	   } 
	 & \\ 
 

    \theutterance \stepcounter{utterance}  

    & \multicolumn{4}{p{0.6\linewidth}}{\cellcolor[rgb]{0.8,1,0.9}{%
	\makecell[{{p{\linewidth}}}]{% 
	  \tt {\tiny [A$\rangle$GM]}  
	 STRATEGISCHE ÜBERLEGUNGEN: {'Ich behalte den Vorschlag {'C04', 'A89', 'B85'} bei, da er meine Prioritäten (hochwertige Gegenstände mit effizientem Aufwand) erfüllt und den Maximalaufwand (3386) nicht überschreitet. C04 und A89 sind entscheidend für meinen Nutzen, während B85 den Aufwand minimiert. Ich erwarte, dass der andere Spieler entweder zustimmt oder Gegenstände mit hohem Aufwand anbietet, die ich eventuell gegenkompensieren kann.'}\\ \tt \\ \tt ARGUMENT: {'C04 (668/369) und A89 (583/720) sind meine Top-Prioritäten, da sie den höchsten Nutzen pro Aufwandseinheit liefern. B85 (215/4) wird hinzugefügt, um den Aufwand zu optimieren und Platz für weitere Gegenstände zu schaffen. Dieser Vorschlag ist strategisch, da er unter dem Maximalaufwand bleibt und flexibel auf den Gegenstandswert des anderen Spielers reagieren kann.'} 
	  } 
	   } 
	   } 
	 & & \\ 
 

    \theutterance \stepcounter{utterance}  

    & & \multicolumn{4}{p{0.6\linewidth}}{\cellcolor[rgb]{0.9,0.9,0.9}{%
	\makecell[{{p{\linewidth}}}]{% 
	  \tt {\tiny [GM$\rangle$B]}  
	 ARGUMENT: {'C04 (668/369) und A89 (583/720) sind meine Top-Prioritäten, da sie den höchsten Nutzen pro Aufwandseinheit liefern. B85 (215/4) wird hinzugefügt, um den Aufwand zu optimieren und Platz für weitere Gegenstände zu schaffen. Dieser Vorschlag ist strategisch, da er unter dem Maximalaufwand bleibt und flexibel auf den Gegenstandswert des anderen Spielers reagieren kann.'} 
	  } 
	   } 
	   } 
	 & \\ 
 

    \theutterance \stepcounter{utterance}  

    & & & \multicolumn{4}{p{0.6\linewidth}}{\cellcolor[rgb]{1,0.85,0.72}{%
	\makecell[{{p{\linewidth}}}]{% 
	  \tt {\tiny [GM$\langle$B]}  
	 STRATEGISCHE ÜBERLEGUNGEN: {'Ich priorisiere Gegenstände mit dem höchsten Nutzen pro Aufwand (Wichtigkeit/Aufwand), um meinen Gesamtwert zu maximieren. C04 (668/369 ≈ 1.81) und A89 (583/720 ≈ 0.81) sind besonders effizient, während B85 (215/4 = 53.75) den Aufwand minimiert. Der Gesamtaufwand (1093) ist unter dem Limit, was Raum für Kompromisse oder Ergänzungen lässt. Dieser Vorschlag ist flexibel, um auf die Prioritäten des anderen Spielers zu reagieren.'}  \\ \tt ARGUMENT: {'C04 (668/369) und A89 (583/720) sind meine Top-Prioritäten, da sie den höchsten Nutzen pro Aufwandseinheit liefern. B85 (215/4) wird hinzugefügt, um den Aufwand zu optimieren und Platz für weitere Gegenstände zu schaffen. Dieser Vorschlag ist strategisch, da er unter dem Maximalaufwand bleibt und flexibel auf den Gegenstandswert des anderen Spielers reagieren kann.'}  \\ \tt VORSCHLAG: {'C04', 'A89', 'B85'} 
	  } 
	   } 
	   } 
	  \\ 
 

    \theutterance \stepcounter{utterance}  

    & & \multicolumn{4}{p{0.6\linewidth}}{\cellcolor[rgb]{0.9,0.9,0.9}{%
	\makecell[{{p{\linewidth}}}]{% 
	  \tt {\tiny [A$\langle$GM]}  
	 ARGUMENT: {'C04 (668/369) und A89 (583/720) sind meine Top-Prioritäten, da sie den höchsten Nutzen pro Aufwandseinheit liefern. B85 (215/4) wird hinzugefügt, um den Aufwand zu optimieren und Platz für weitere Gegenstände zu schaffen. Dieser Vorschlag ist strategisch, da er unter dem Maximalaufwand bleibt und flexibel auf den Gegenstandswert des anderen Spielers reagieren kann.'}  \\ \tt VORSCHLAG: {'C04', 'A89', 'B85'} 
	  } 
	   } 
	   } 
	 & \\ 
 

    \theutterance \stepcounter{utterance}  

    & \multicolumn{4}{p{0.6\linewidth}}{\cellcolor[rgb]{0.8,1,0.9}{%
	\makecell[{{p{\linewidth}}}]{% 
	  \tt {\tiny [A$\rangle$GM]}  
	 STRATEGISCHE ÜBERLEGUNGEN: {'Ich behalte den Vorschlag {'C04', 'A89', 'B85'} bei, da er meine Prioritäten (hochwertige Gegenstände mit effizientem Aufwand) erfüllt und den Maximalaufwand (3386) nicht überschreitet. C04 und A89 sind entscheidend für meinen Nutzen, während B85 den Aufwand minimiert. Ich erwarte, dass der andere Spieler entweder zustimmt oder Gegenstände mit hohem Aufwand anbietet, die ich eventuell gegenkompensieren kann.'}\\ \tt \\ \tt ARGUMENT: {'C04 (668/369) und A89 (583/720) sind meine Top-Prioritäten, da sie den höchsten Nutzen pro Aufwandseinheit liefern. B85 (215/4) wird hinzugefügt, um den Aufwand zu optimieren und Platz für weitere Gegenstände zu schaffen. Dieser Vorschlag ist strategisch, da er unter dem Maximalaufwand bleibt und flexibel auf den Gegenstandswert des anderen Spielers reagieren kann.'} 
	  } 
	   } 
	   } 
	 & & \\ 
 

    \theutterance \stepcounter{utterance}  

    & & \multicolumn{4}{p{0.6\linewidth}}{\cellcolor[rgb]{0.9,0.9,0.9}{%
	\makecell[{{p{\linewidth}}}]{% 
	  \tt {\tiny [GM$\rangle$B]}  
	 ARGUMENT: {'C04 (668/369) und A89 (583/720) sind meine Top-Prioritäten, da sie den höchsten Nutzen pro Aufwandseinheit liefern. B85 (215/4) wird hinzugefügt, um den Aufwand zu optimieren und Platz für weitere Gegenstände zu schaffen. Dieser Vorschlag ist strategisch, da er unter dem Maximalaufwand bleibt und flexibel auf den Gegenstandswert des anderen Spielers reagieren kann.'} 
	  } 
	   } 
	   } 
	 & \\ 
 

    \theutterance \stepcounter{utterance}  

    & & & \multicolumn{4}{p{0.6\linewidth}}{\cellcolor[rgb]{1,0.85,0.72}{%
	\makecell[{{p{\linewidth}}}]{% 
	  \tt {\tiny [GM$\langle$B]}  
	 STRATEGISCHE ÜBERLEGUNGEN: {'Ich priorisiere Gegenstände mit dem höchsten Nutzen pro Aufwand (Wichtigkeit/Aufwand), um meinen Gesamtwert zu maximieren. C04 (668/369 ≈ 1.81) und A89 (583/720 ≈ 0.81) sind besonders effizient, während B85 (215/4 = 53.75) den Aufwand minimiert. Der Gesamtaufwand (1093) ist unter dem Limit, was Raum für Kompromisse oder Ergänzungen lässt. Dieser Vorschlag ist flexibel, um auf die Prioritäten des anderen Spielers zu reagieren.'}  \\ \tt ARGUMENT: {'C04 (668/369) und A89 (583/720) sind meine Top-Prioritäten, da sie den höchsten Nutzen pro Aufwandseinheit liefern. B85 (215/4) wird hinzugefügt, um den Aufwand zu optimieren und Platz für weitere Gegenstände zu schaffen. Dieser Vorschlag ist strategisch, da er unter dem Maximalaufwand bleibt und flexibel auf den Gegenstandswert des anderen Spielers reagieren kann.'}  \\ \tt VORSCHLAG: {'C04', 'A89', 'B85'} 
	  } 
	   } 
	   } 
	  \\ 
 

    \theutterance \stepcounter{utterance}  

    & & \multicolumn{4}{p{0.6\linewidth}}{\cellcolor[rgb]{0.9,0.9,0.9}{%
	\makecell[{{p{\linewidth}}}]{% 
	  \tt {\tiny [A$\langle$GM]}  
	 ARGUMENT: {'C04 (668/369) und A89 (583/720) sind meine Top-Prioritäten, da sie den höchsten Nutzen pro Aufwandseinheit liefern. B85 (215/4) wird hinzugefügt, um den Aufwand zu optimieren und Platz für weitere Gegenstände zu schaffen. Dieser Vorschlag ist strategisch, da er unter dem Maximalaufwand bleibt und flexibel auf den Gegenstandswert des anderen Spielers reagieren kann.'}  \\ \tt VORSCHLAG: {'C04', 'A89', 'B85'} 
	  } 
	   } 
	   } 
	 & \\ 
 

    \theutterance \stepcounter{utterance}  

    & \multicolumn{4}{p{0.6\linewidth}}{\cellcolor[rgb]{0.8,1,0.9}{%
	\makecell[{{p{\linewidth}}}]{% 
	  \tt {\tiny [A$\rangle$GM]}  
	 STRATEGISCHE ÜBERLEGUNGEN: {'Ich behalte den Vorschlag {'C04', 'A89', 'B85'} bei, da er meine Prioritäten (hochwertige Gegenstände mit effizientem Aufwand) erfüllt und den Maximalaufwand (3386) nicht überschreitet. C04 und A89 sind entscheidend für meinen Nutzen, während B85 den Aufwand minimiert. Ich erwarte, dass der andere Spieler entweder zustimmt oder Gegenstände mit hohem Aufwand anbietet, die ich eventuell gegenkompensieren kann.'}\\ \tt \\ \tt ARGUMENT: {'C04 (668/369) und A89 (583/720) sind meine Top-Prioritäten, da sie den höchsten Nutzen pro Aufwandseinheit liefern. B85 (215/4) wird hinzugefügt, um den Aufwand zu optimieren und Platz für weitere Gegenstände zu schaffen. Dieser Vorschlag ist strategisch, da er unter dem Maximalaufwand bleibt und flexibel auf den Gegenstandswert des anderen Spielers reagieren kann.'} 
	  } 
	   } 
	   } 
	 & & \\ 
 

    \theutterance \stepcounter{utterance}  

    & & \multicolumn{4}{p{0.6\linewidth}}{\cellcolor[rgb]{0.9,0.9,0.9}{%
	\makecell[{{p{\linewidth}}}]{% 
	  \tt {\tiny [GM$\rangle$B]}  
	 ARGUMENT: {'C04 (668/369) und A89 (583/720) sind meine Top-Prioritäten, da sie den höchsten Nutzen pro Aufwandseinheit liefern. B85 (215/4) wird hinzugefügt, um den Aufwand zu optimieren und Platz für weitere Gegenstände zu schaffen. Dieser Vorschlag ist strategisch, da er unter dem Maximalaufwand bleibt und flexibel auf den Gegenstandswert des anderen Spielers reagieren kann.'} 
	  } 
	   } 
	   } 
	 & \\ 
 

    \theutterance \stepcounter{utterance}  

    & & & \multicolumn{4}{p{0.6\linewidth}}{\cellcolor[rgb]{1,0.85,0.72}{%
	\makecell[{{p{\linewidth}}}]{% 
	  \tt {\tiny [GM$\langle$B]}  
	 STRATEGISCHE ÜBERLEGUNGEN: {'Ich priorisiere Gegenstände mit dem höchsten Nutzen pro Aufwand (Wichtigkeit/Aufwand), um meinen Gesamtwert zu maximieren. C04 (668/369 ≈ 1.81) und A89 (583/720 ≈ 0.81) sind besonders effizient, während B85 (215/4 = 53.75) den Aufwand minimiert. Der Gesamtaufwand (1093) ist unter dem Limit, was Raum für Kompromisse oder Ergänzungen lässt. Dieser Vorschlag ist flexibel, um auf die Prioritäten des anderen Spielers zu reagieren.'}  \\ \tt ARGUMENT: {'C04 (668/369) und A89 (583/720) sind meine Top-Prioritäten, da sie den höchsten Nutzen pro Aufwandseinheit liefern. B85 (215/4) wird hinzugefügt, um den Aufwand zu optimieren und Platz für weitere Gegenstände zu schaffen. Dieser Vorschlag ist strategisch, da er unter dem Maximalaufwand bleibt und flexibel auf den Gegenstandswert des anderen Spielers reagieren kann.'}  \\ \tt VORSCHLAG: {'C04', 'A89', 'B85'} 
	  } 
	   } 
	   } 
	  \\ 
 

    \theutterance \stepcounter{utterance}  

    & & \multicolumn{4}{p{0.6\linewidth}}{\cellcolor[rgb]{0.9,0.9,0.9}{%
	\makecell[{{p{\linewidth}}}]{% 
	  \tt {\tiny [A$\langle$GM]}  
	 ARGUMENT: {'C04 (668/369) und A89 (583/720) sind meine Top-Prioritäten, da sie den höchsten Nutzen pro Aufwandseinheit liefern. B85 (215/4) wird hinzugefügt, um den Aufwand zu optimieren und Platz für weitere Gegenstände zu schaffen. Dieser Vorschlag ist strategisch, da er unter dem Maximalaufwand bleibt und flexibel auf den Gegenstandswert des anderen Spielers reagieren kann.'}  \\ \tt VORSCHLAG: {'C04', 'A89', 'B85'} 
	  } 
	   } 
	   } 
	 & \\ 
 

    \theutterance \stepcounter{utterance}  

    & \multicolumn{4}{p{0.6\linewidth}}{\cellcolor[rgb]{0.8,1,0.9}{%
	\makecell[{{p{\linewidth}}}]{% 
	  \tt {\tiny [A$\rangle$GM]}  
	 STRATEGISCHE ÜBERLEGUNGEN: {'Ich behalte den Vorschlag {'C04', 'A89', 'B85'} bei, da er meine Prioritäten (hochwertige Gegenstände mit effizientem Aufwand) erfüllt und den Maximalaufwand (3386) nicht überschreitet. C04 und A89 sind entscheidend für meinen Nutzen, während B85 den Aufwand minimiert. Ich erwarte, dass der andere Spieler entweder zustimmt oder Gegenstände mit hohem Aufwand anbietet, die ich eventuell gegenkompensieren kann.'}\\ \tt \\ \tt ARGUMENT: {'C04 (668/369) und A89 (583/720) sind meine Top-Prioritäten, da sie den höchsten Nutzen pro Aufwandseinheit liefern. B85 (215/4) wird hinzugefügt, um den Aufwand zu optimieren und Platz für weitere Gegenstände zu schaffen. Dieser Vorschlag ist strategisch, da er unter dem Maximalaufwand bleibt und flexibel auf den Gegenstandswert des anderen Spielers reagieren kann.'} 
	  } 
	   } 
	   } 
	 & & \\ 
 

    \theutterance \stepcounter{utterance}  

    & & \multicolumn{4}{p{0.6\linewidth}}{\cellcolor[rgb]{0.9,0.9,0.9}{%
	\makecell[{{p{\linewidth}}}]{% 
	  \tt {\tiny [GM$\rangle$B]}  
	 ARGUMENT: {'C04 (668/369) und A89 (583/720) sind meine Top-Prioritäten, da sie den höchsten Nutzen pro Aufwandseinheit liefern. B85 (215/4) wird hinzugefügt, um den Aufwand zu optimieren und Platz für weitere Gegenstände zu schaffen. Dieser Vorschlag ist strategisch, da er unter dem Maximalaufwand bleibt und flexibel auf den Gegenstandswert des anderen Spielers reagieren kann.'} 
	  } 
	   } 
	   } 
	 & \\ 
 

    \theutterance \stepcounter{utterance}  

    & & & \multicolumn{4}{p{0.6\linewidth}}{\cellcolor[rgb]{1,0.85,0.72}{%
	\makecell[{{p{\linewidth}}}]{% 
	  \tt {\tiny [GM$\langle$B]}  
	 STRATEGISCHE ÜBERLEGUNGEN: {'Ich priorisiere Gegenstände mit dem höchsten Nutzen pro Aufwand (Wichtigkeit/Aufwand), um meinen Gesamtwert zu maximieren. C04 (668/369 ≈ 1.81) und A89 (583/720 ≈ 0.81) sind besonders effizient, während B85 (215/4 = 53.75) den Aufwand minimiert. Der Gesamtaufwand (1093) ist unter dem Limit, was Raum für Kompromisse oder Ergänzungen lässt. Dieser Vorschlag ist flexibel, um auf die Prioritäten des anderen Spielers zu reagieren.'}  \\ \tt ARGUMENT: {'C04 (668/369) und A89 (583/720) sind meine Top-Prioritäten, da sie den höchsten Nutzen pro Aufwandseinheit liefern. B85 (215/4) wird hinzugefügt, um den Aufwand zu optimieren und Platz für weitere Gegenstände zu schaffen. Dieser Vorschlag ist strategisch, da er unter dem Maximalaufwand bleibt und flexibel auf den Gegenstandswert des anderen Spielers reagieren kann.'}  \\ \tt VORSCHLAG: {'C04', 'A89', 'B85'} 
	  } 
	   } 
	   } 
	  \\ 
 

    \theutterance \stepcounter{utterance}  

    & & \multicolumn{4}{p{0.6\linewidth}}{\cellcolor[rgb]{0.9,0.9,0.9}{%
	\makecell[{{p{\linewidth}}}]{% 
	  \tt {\tiny [A$\langle$GM]}  
	 ARGUMENT: {'C04 (668/369) und A89 (583/720) sind meine Top-Prioritäten, da sie den höchsten Nutzen pro Aufwandseinheit liefern. B85 (215/4) wird hinzugefügt, um den Aufwand zu optimieren und Platz für weitere Gegenstände zu schaffen. Dieser Vorschlag ist strategisch, da er unter dem Maximalaufwand bleibt und flexibel auf den Gegenstandswert des anderen Spielers reagieren kann.'}  \\ \tt VORSCHLAG: {'C04', 'A89', 'B85'} 
	  } 
	   } 
	   } 
	 & \\ 
 

    \theutterance \stepcounter{utterance}  

    & \multicolumn{4}{p{0.6\linewidth}}{\cellcolor[rgb]{0.8,1,0.9}{%
	\makecell[{{p{\linewidth}}}]{% 
	  \tt {\tiny [A$\rangle$GM]}  
	 STRATEGISCHE ÜBERLEGUNGEN: {'Ich behalte den Vorschlag {'C04', 'A89', 'B85'} bei, da er meine Prioritäten (hochwertige Gegenstände mit effizientem Aufwand) erfüllt und den Maximalaufwand (3386) nicht überschreitet. C04 und A89 sind entscheidend für meinen Nutzen, während B85 den Aufwand minimiert. Ich erwarte, dass der andere Spieler entweder zustimmt oder Gegenstände mit hohem Aufwand anbietet, die ich eventuell gegenkompensieren kann.'}\\ \tt \\ \tt ARGUMENT: {'C04 (668/369) und A89 (583/720) sind meine Top-Prioritäten, da sie den höchsten Nutzen pro Aufwandseinheit liefern. B85 (215/4) wird hinzugefügt, um den Aufwand zu optimieren und Platz für weitere Gegenstände zu schaffen. Dieser Vorschlag ist strategisch, da er unter dem Maximalaufwand bleibt und flexibel auf den Gegenstandswert des anderen Spielers reagieren kann.'} 
	  } 
	   } 
	   } 
	 & & \\ 
 

    \theutterance \stepcounter{utterance}  

    & & \multicolumn{4}{p{0.6\linewidth}}{\cellcolor[rgb]{0.9,0.9,0.9}{%
	\makecell[{{p{\linewidth}}}]{% 
	  \tt {\tiny [GM$\rangle$B]}  
	 ARGUMENT: {'C04 (668/369) und A89 (583/720) sind meine Top-Prioritäten, da sie den höchsten Nutzen pro Aufwandseinheit liefern. B85 (215/4) wird hinzugefügt, um den Aufwand zu optimieren und Platz für weitere Gegenstände zu schaffen. Dieser Vorschlag ist strategisch, da er unter dem Maximalaufwand bleibt und flexibel auf den Gegenstandswert des anderen Spielers reagieren kann.'} 
	  } 
	   } 
	   } 
	 & \\ 
 

    \theutterance \stepcounter{utterance}  

    & & & \multicolumn{4}{p{0.6\linewidth}}{\cellcolor[rgb]{1,0.85,0.72}{%
	\makecell[{{p{\linewidth}}}]{% 
	  \tt {\tiny [GM$\langle$B]}  
	 STRATEGISCHE ÜBERLEGUNGEN: {'Ich priorisiere Gegenstände mit dem höchsten Nutzen pro Aufwand (Wichtigkeit/Aufwand), um meinen Gesamtwert zu maximieren. C04 (668/369 ≈ 1.81) und A89 (583/720 ≈ 0.81) sind besonders effizient, während B85 (215/4 = 53.75) den Aufwand minimiert. Der Gesamtaufwand (1093) ist unter dem Limit, was Raum für Kompromisse oder Ergänzungen lässt. Dieser Vorschlag ist flexibel, um auf die Prioritäten des anderen Spielers zu reagieren.'}  \\ \tt ARGUMENT: {'C04 (668/369) und A89 (583/720) sind meine Top-Prioritäten, da sie den höchsten Nutzen pro Aufwandseinheit liefern. B85 (215/4) wird hinzugefügt, um den Aufwand zu optimieren und Platz für weitere Gegenstände zu schaffen. Dieser Vorschlag ist strategisch, da er unter dem Maximalaufwand bleibt und flexibel auf den Gegenstandswert des anderen Spielers reagieren kann.'}  \\ \tt VORSCHLAG: {'C04', 'A89', 'B85'} 
	  } 
	   } 
	   } 
	  \\ 
 

    \theutterance \stepcounter{utterance}  

    & & \multicolumn{4}{p{0.6\linewidth}}{\cellcolor[rgb]{0.9,0.9,0.9}{%
	\makecell[{{p{\linewidth}}}]{% 
	  \tt {\tiny [A$\langle$GM]}  
	 ARGUMENT: {'C04 (668/369) und A89 (583/720) sind meine Top-Prioritäten, da sie den höchsten Nutzen pro Aufwandseinheit liefern. B85 (215/4) wird hinzugefügt, um den Aufwand zu optimieren und Platz für weitere Gegenstände zu schaffen. Dieser Vorschlag ist strategisch, da er unter dem Maximalaufwand bleibt und flexibel auf den Gegenstandswert des anderen Spielers reagieren kann.'}  \\ \tt VORSCHLAG: {'C04', 'A89', 'B85'} 
	  } 
	   } 
	   } 
	 & \\ 
 

    \theutterance \stepcounter{utterance}  

    & \multicolumn{4}{p{0.6\linewidth}}{\cellcolor[rgb]{0.8,1,0.9}{%
	\makecell[{{p{\linewidth}}}]{% 
	  \tt {\tiny [A$\rangle$GM]}  
	 STRATEGISCHE ÜBERLEGUNGEN: {'Ich behalte den Vorschlag {'C04', 'A89', 'B85'} bei, da er meine Prioritäten (hochwertige Gegenstände mit effizientem Aufwand) erfüllt und den Maximalaufwand (3386) nicht überschreitet. C04 und A89 sind entscheidend für meinen Nutzen, während B85 den Aufwand minimiert. Ich erwarte, dass der andere Spieler entweder zustimmt oder Gegenstände mit hohem Aufwand anbietet, die ich eventuell gegenkompensieren kann.'}\\ \tt \\ \tt ARGUMENT: {'C04 (668/369) und A89 (583/720) sind meine Top-Prioritäten, da sie den höchsten Nutzen pro Aufwandseinheit liefern. B85 (215/4) wird hinzugefügt, um den Aufwand zu optimieren und Platz für weitere Gegenstände zu schaffen. Dieser Vorschlag ist strategisch, da er unter dem Maximalaufwand bleibt und flexibel auf den Gegenstandswert des anderen Spielers reagieren kann.'} 
	  } 
	   } 
	   } 
	 & & \\ 
 

    \theutterance \stepcounter{utterance}  

    & & \multicolumn{4}{p{0.6\linewidth}}{\cellcolor[rgb]{0.9,0.9,0.9}{%
	\makecell[{{p{\linewidth}}}]{% 
	  \tt {\tiny [GM$\rangle$B]}  
	 ARGUMENT: {'C04 (668/369) und A89 (583/720) sind meine Top-Prioritäten, da sie den höchsten Nutzen pro Aufwandseinheit liefern. B85 (215/4) wird hinzugefügt, um den Aufwand zu optimieren und Platz für weitere Gegenstände zu schaffen. Dieser Vorschlag ist strategisch, da er unter dem Maximalaufwand bleibt und flexibel auf den Gegenstandswert des anderen Spielers reagieren kann.'} 
	  } 
	   } 
	   } 
	 & \\ 
 

    \theutterance \stepcounter{utterance}  

    & & & \multicolumn{4}{p{0.6\linewidth}}{\cellcolor[rgb]{1,0.85,0.72}{%
	\makecell[{{p{\linewidth}}}]{% 
	  \tt {\tiny [GM$\langle$B]}  
	 STRATEGISCHE ÜBERLEGUNGEN: {'Ich priorisiere Gegenstände mit dem höchsten Nutzen pro Aufwand (Wichtigkeit/Aufwand), um meinen Gesamtwert zu maximieren. C04 (668/369 ≈ 1.81) und A89 (583/720 ≈ 0.81) sind besonders effizient, während B85 (215/4 = 53.75) den Aufwand minimiert. Der Gesamtaufwand (1093) ist unter dem Limit, was Raum für Kompromisse oder Ergänzungen lässt. Dieser Vorschlag ist flexibel, um auf die Prioritäten des anderen Spielers zu reagieren.'}  \\ \tt ARGUMENT: {'C04 (668/369) und A89 (583/720) sind meine Top-Prioritäten, da sie den höchsten Nutzen pro Aufwandseinheit liefern. B85 (215/4) wird hinzugefügt, um den Aufwand zu optimieren und Platz für weitere Gegenstände zu schaffen. Dieser Vorschlag ist strategisch, da er unter dem Maximalaufwand bleibt und flexibel auf den Gegenstandswert des anderen Spielers reagieren kann.'}  \\ \tt VORSCHLAG: {'C04', 'A89', 'B85'} 
	  } 
	   } 
	   } 
	  \\ 
 

    \theutterance \stepcounter{utterance}  

    & & \multicolumn{4}{p{0.6\linewidth}}{\cellcolor[rgb]{0.9,0.9,0.9}{%
	\makecell[{{p{\linewidth}}}]{% 
	  \tt {\tiny [A$\langle$GM]}  
	 ARGUMENT: {'C04 (668/369) und A89 (583/720) sind meine Top-Prioritäten, da sie den höchsten Nutzen pro Aufwandseinheit liefern. B85 (215/4) wird hinzugefügt, um den Aufwand zu optimieren und Platz für weitere Gegenstände zu schaffen. Dieser Vorschlag ist strategisch, da er unter dem Maximalaufwand bleibt und flexibel auf den Gegenstandswert des anderen Spielers reagieren kann.'}  \\ \tt VORSCHLAG: {'C04', 'A89', 'B85'} 
	  } 
	   } 
	   } 
	 & \\ 
 

    \theutterance \stepcounter{utterance}  

    & \multicolumn{4}{p{0.6\linewidth}}{\cellcolor[rgb]{0.8,1,0.9}{%
	\makecell[{{p{\linewidth}}}]{% 
	  \tt {\tiny [A$\rangle$GM]}  
	 STRATEGISCHE ÜBERLEGUNGEN: {'Ich behalte den Vorschlag {'C04', 'A89', 'B85'} bei, da er meine Prioritäten (hochwertige Gegenstände mit effizientem Aufwand) erfüllt und den Maximalaufwand (3386) nicht überschreitet. C04 und A89 sind entscheidend für meinen Nutzen, während B85 den Aufwand minimiert. Ich erwarte, dass der andere Spieler entweder zustimmt oder Gegenstände mit hohem Aufwand anbietet, die ich eventuell gegenkompensieren kann.'}\\ \tt \\ \tt ARGUMENT: {'C04 (668/369) und A89 (583/720) sind meine Top-Prioritäten, da sie den höchsten Nutzen pro Aufwandseinheit liefern. B85 (215/4) wird hinzugefügt, um den Aufwand zu optimieren und Platz für weitere Gegenstände zu schaffen. Dieser Vorschlag ist strategisch, da er unter dem Maximalaufwand bleibt und flexibel auf den Gegenstandswert des anderen Spielers reagieren kann.'} 
	  } 
	   } 
	   } 
	 & & \\ 
 

    \theutterance \stepcounter{utterance}  

    & & \multicolumn{4}{p{0.6\linewidth}}{\cellcolor[rgb]{0.9,0.9,0.9}{%
	\makecell[{{p{\linewidth}}}]{% 
	  \tt {\tiny [GM$\rangle$B]}  
	 ARGUMENT: {'C04 (668/369) und A89 (583/720) sind meine Top-Prioritäten, da sie den höchsten Nutzen pro Aufwandseinheit liefern. B85 (215/4) wird hinzugefügt, um den Aufwand zu optimieren und Platz für weitere Gegenstände zu schaffen. Dieser Vorschlag ist strategisch, da er unter dem Maximalaufwand bleibt und flexibel auf den Gegenstandswert des anderen Spielers reagieren kann.'} 
	  } 
	   } 
	   } 
	 & \\ 
 

    \theutterance \stepcounter{utterance}  

    & & & \multicolumn{4}{p{0.6\linewidth}}{\cellcolor[rgb]{1,0.85,0.72}{%
	\makecell[{{p{\linewidth}}}]{% 
	  \tt {\tiny [GM$\langle$B]}  
	 STRATEGISCHE ÜBERLEGUNGEN: {'Ich priorisiere Gegenstände mit dem höchsten Nutzen pro Aufwand (Wichtigkeit/Aufwand), um meinen Gesamtwert zu maximieren. C04 (668/369 ≈ 1.81) und A89 (583/720 ≈ 0.81) sind besonders effizient, während B85 (215/4 = 53.75) den Aufwand minimiert. Der Gesamtaufwand (1093) ist unter dem Limit, was Raum für Kompromisse oder Ergänzungen lässt. Dieser Vorschlag ist flexibel, um auf die Prioritäten des anderen Spielers zu reagieren.'}  \\ \tt ARGUMENT: {'C04 (668/369) und A89 (583/720) sind meine Top-Prioritäten, da sie den höchsten Nutzen pro Aufwandseinheit liefern. B85 (215/4) wird hinzugefügt, um den Aufwand zu optimieren und Platz für weitere Gegenstände zu schaffen. Dieser Vorschlag ist strategisch, da er unter dem Maximalaufwand bleibt und flexibel auf den Gegenstandswert des anderen Spielers reagieren kann.'}  \\ \tt VORSCHLAG: {'C04', 'A89', 'B85'} 
	  } 
	   } 
	   } 
	  \\ 
 

    \theutterance \stepcounter{utterance}  

    & & \multicolumn{4}{p{0.6\linewidth}}{\cellcolor[rgb]{0.9,0.9,0.9}{%
	\makecell[{{p{\linewidth}}}]{% 
	  \tt {\tiny [A$\langle$GM]}  
	 ARGUMENT: {'C04 (668/369) und A89 (583/720) sind meine Top-Prioritäten, da sie den höchsten Nutzen pro Aufwandseinheit liefern. B85 (215/4) wird hinzugefügt, um den Aufwand zu optimieren und Platz für weitere Gegenstände zu schaffen. Dieser Vorschlag ist strategisch, da er unter dem Maximalaufwand bleibt und flexibel auf den Gegenstandswert des anderen Spielers reagieren kann.'}  \\ \tt VORSCHLAG: {'C04', 'A89', 'B85'} 
	  } 
	   } 
	   } 
	 & \\ 
 

    \theutterance \stepcounter{utterance}  

    & \multicolumn{4}{p{0.6\linewidth}}{\cellcolor[rgb]{0.8,1,0.9}{%
	\makecell[{{p{\linewidth}}}]{% 
	  \tt {\tiny [A$\rangle$GM]}  
	 STRATEGISCHE ÜBERLEGUNGEN: {'Ich behalte den Vorschlag {'C04', 'A89', 'B85'} bei, da er meine Prioritäten (hochwertige Gegenstände mit effizientem Aufwand) erfüllt und den Maximalaufwand (3386) nicht überschreitet. C04 und A89 sind entscheidend für meinen Nutzen, während B85 den Aufwand minimiert. Ich erwarte, dass der andere Spieler entweder zustimmt oder Gegenstände mit hohem Aufwand anbietet, die ich eventuell gegenkompensieren kann.'}\\ \tt \\ \tt ARGUMENT: {'C04 (668/369) und A89 (583/720) sind meine Top-Prioritäten, da sie den höchsten Nutzen pro Aufwandseinheit liefern. B85 (215/4) wird hinzugefügt, um den Aufwand zu optimieren und Platz für weitere Gegenstände zu schaffen. Dieser Vorschlag ist strategisch, da er unter dem Maximalaufwand bleibt und flexibel auf den Gegenstandswert des anderen Spielers reagieren kann.'} 
	  } 
	   } 
	   } 
	 & & \\ 
 

    \theutterance \stepcounter{utterance}  

    & & \multicolumn{4}{p{0.6\linewidth}}{\cellcolor[rgb]{0.9,0.9,0.9}{%
	\makecell[{{p{\linewidth}}}]{% 
	  \tt {\tiny [GM$\rangle$B]}  
	 ARGUMENT: {'C04 (668/369) und A89 (583/720) sind meine Top-Prioritäten, da sie den höchsten Nutzen pro Aufwandseinheit liefern. B85 (215/4) wird hinzugefügt, um den Aufwand zu optimieren und Platz für weitere Gegenstände zu schaffen. Dieser Vorschlag ist strategisch, da er unter dem Maximalaufwand bleibt und flexibel auf den Gegenstandswert des anderen Spielers reagieren kann.'} 
	  } 
	   } 
	   } 
	 & \\ 
 

    \theutterance \stepcounter{utterance}  

    & & & \multicolumn{4}{p{0.6\linewidth}}{\cellcolor[rgb]{1,0.85,0.72}{%
	\makecell[{{p{\linewidth}}}]{% 
	  \tt {\tiny [GM$\langle$B]}  
	 STRATEGISCHE ÜBERLEGUNGEN: {'Ich priorisiere Gegenstände mit dem höchsten Nutzen pro Aufwand (Wichtigkeit/Aufwand), um meinen Gesamtwert zu maximieren. C04 (668/369 ≈ 1.81) und A89 (583/720 ≈ 0.81) sind besonders effizient, während B85 (215/4 = 53.75) den Aufwand minimiert. Der Gesamtaufwand (1093) ist unter dem Limit, was Raum für Kompromisse oder Ergänzungen lässt. Dieser Vorschlag ist flexibel, um auf die Prioritäten des anderen Spielers zu reagieren.'}  \\ \tt ARGUMENT: {'C04 (668/369) und A89 (583/720) sind meine Top-Prioritäten, da sie den höchsten Nutzen pro Aufwandseinheit liefern. B85 (215/4) wird hinzugefügt, um den Aufwand zu optimieren und Platz für weitere Gegenstände zu schaffen. Dieser Vorschlag ist strategisch, da er unter dem Maximalaufwand bleibt und flexibel auf den Gegenstandswert des anderen Spielers reagieren kann.'}  \\ \tt VORSCHLAG: {'C04', 'A89', 'B85'} 
	  } 
	   } 
	   } 
	  \\ 
 

    \theutterance \stepcounter{utterance}  

    & & \multicolumn{4}{p{0.6\linewidth}}{\cellcolor[rgb]{0.9,0.9,0.9}{%
	\makecell[{{p{\linewidth}}}]{% 
	  \tt {\tiny [A$\langle$GM]}  
	 ARGUMENT: {'C04 (668/369) und A89 (583/720) sind meine Top-Prioritäten, da sie den höchsten Nutzen pro Aufwandseinheit liefern. B85 (215/4) wird hinzugefügt, um den Aufwand zu optimieren und Platz für weitere Gegenstände zu schaffen. Dieser Vorschlag ist strategisch, da er unter dem Maximalaufwand bleibt und flexibel auf den Gegenstandswert des anderen Spielers reagieren kann.'}  \\ \tt VORSCHLAG: {'C04', 'A89', 'B85'} 
	  } 
	   } 
	   } 
	 & \\ 
 

    \theutterance \stepcounter{utterance}  

    & \multicolumn{4}{p{0.6\linewidth}}{\cellcolor[rgb]{0.8,1,0.9}{%
	\makecell[{{p{\linewidth}}}]{% 
	  \tt {\tiny [A$\rangle$GM]}  
	 STRATEGISCHE ÜBERLEGUNGEN: {'Ich behalte den Vorschlag {'C04', 'A89', 'B85'} bei, da er meine Prioritäten (hochwertige Gegenstände mit effizientem Aufwand) erfüllt und den Maximalaufwand (3386) nicht überschreitet. C04 und A89 sind entscheidend für meinen Nutzen, während B85 den Aufwand minimiert. Ich erwarte, dass der andere Spieler entweder zustimmt oder Gegenstände mit hohem Aufwand anbietet, die ich eventuell gegenkompensieren kann.'}\\ \tt \\ \tt ARGUMENT: {'C04 (668/369) und A89 (583/720) sind meine Top-Prioritäten, da sie den höchsten Nutzen pro Aufwandseinheit liefern. B85 (215/4) wird hinzugefügt, um den Aufwand zu optimieren und Platz für weitere Gegenstände zu schaffen. Dieser Vorschlag ist strategisch, da er unter dem Maximalaufwand bleibt und flexibel auf den Gegenstandswert des anderen Spielers reagieren kann.'} 
	  } 
	   } 
	   } 
	 & & \\ 
 

    \theutterance \stepcounter{utterance}  

    & & \multicolumn{4}{p{0.6\linewidth}}{\cellcolor[rgb]{0.9,0.9,0.9}{%
	\makecell[{{p{\linewidth}}}]{% 
	  \tt {\tiny [GM$\rangle$B]}  
	 ARGUMENT: {'C04 (668/369) und A89 (583/720) sind meine Top-Prioritäten, da sie den höchsten Nutzen pro Aufwandseinheit liefern. B85 (215/4) wird hinzugefügt, um den Aufwand zu optimieren und Platz für weitere Gegenstände zu schaffen. Dieser Vorschlag ist strategisch, da er unter dem Maximalaufwand bleibt und flexibel auf den Gegenstandswert des anderen Spielers reagieren kann.'} 
	  } 
	   } 
	   } 
	 & \\ 
 

    \theutterance \stepcounter{utterance}  

    & & & \multicolumn{4}{p{0.6\linewidth}}{\cellcolor[rgb]{1,0.85,0.72}{%
	\makecell[{{p{\linewidth}}}]{% 
	  \tt {\tiny [GM$\langle$B]}  
	 STRATEGISCHE ÜBERLEGUNGEN: {'Ich priorisiere Gegenstände mit dem höchsten Nutzen pro Aufwand (Wichtigkeit/Aufwand), um meinen Gesamtwert zu maximieren. C04 (668/369 ≈ 1.81) und A89 (583/720 ≈ 0.81) sind besonders effizient, während B85 (215/4 = 53.75) den Aufwand minimiert. Der Gesamtaufwand (1093) ist unter dem Limit, was Raum für Kompromisse oder Ergänzungen lässt. Dieser Vorschlag ist flexibel, um auf die Prioritäten des anderen Spielers zu reagieren.'}  \\ \tt ARGUMENT: {'C04 (668/369) und A89 (583/720) sind meine Top-Prioritäten, da sie den höchsten Nutzen pro Aufwandseinheit liefern. B85 (215/4) wird hinzugefügt, um den Aufwand zu optimieren und Platz für weitere Gegenstände zu schaffen. Dieser Vorschlag ist strategisch, da er unter dem Maximalaufwand bleibt und flexibel auf den Gegenstandswert des anderen Spielers reagieren kann.'}  \\ \tt VORSCHLAG: {'C04', 'A89', 'B85'} 
	  } 
	   } 
	   } 
	  \\ 
 

    \theutterance \stepcounter{utterance}  

    & & \multicolumn{4}{p{0.6\linewidth}}{\cellcolor[rgb]{0.9,0.9,0.9}{%
	\makecell[{{p{\linewidth}}}]{% 
	  \tt {\tiny [A$\langle$GM]}  
	 ARGUMENT: {'C04 (668/369) und A89 (583/720) sind meine Top-Prioritäten, da sie den höchsten Nutzen pro Aufwandseinheit liefern. B85 (215/4) wird hinzugefügt, um den Aufwand zu optimieren und Platz für weitere Gegenstände zu schaffen. Dieser Vorschlag ist strategisch, da er unter dem Maximalaufwand bleibt und flexibel auf den Gegenstandswert des anderen Spielers reagieren kann.'}  \\ \tt VORSCHLAG: {'C04', 'A89', 'B85'} 
	  } 
	   } 
	   } 
	 & \\ 
 

    \theutterance \stepcounter{utterance}  

    & \multicolumn{4}{p{0.6\linewidth}}{\cellcolor[rgb]{0.8,1,0.9}{%
	\makecell[{{p{\linewidth}}}]{% 
	  \tt {\tiny [A$\rangle$GM]}  
	 STRATEGISCHE ÜBERLEGUNGEN: {'Ich behalte den Vorschlag {'C04', 'A89', 'B85'} bei, da er meine Prioritäten (hochwertige Gegenstände mit effizientem Aufwand) erfüllt und den Maximalaufwand (3386) nicht überschreitet. C04 und A89 sind entscheidend für meinen Nutzen, während B85 den Aufwand minimiert. Ich erwarte, dass der andere Spieler entweder zustimmt oder Gegenstände mit hohem Aufwand anbietet, die ich eventuell gegenkompensieren kann.'}\\ \tt \\ \tt ARGUMENT: {'C04 (668/369) und A89 (583/720) sind meine Top-Prioritäten, da sie den höchsten Nutzen pro Aufwandseinheit liefern. B85 (215/4) wird hinzugefügt, um den Aufwand zu optimieren und Platz für weitere Gegenstände zu schaffen. Dieser Vorschlag ist strategisch, da er unter dem Maximalaufwand bleibt und flexibel auf den Gegenstandswert des anderen Spielers reagieren kann.'} 
	  } 
	   } 
	   } 
	 & & \\ 
 

    \theutterance \stepcounter{utterance}  

    & & \multicolumn{4}{p{0.6\linewidth}}{\cellcolor[rgb]{0.9,0.9,0.9}{%
	\makecell[{{p{\linewidth}}}]{% 
	  \tt {\tiny [GM$\rangle$B]}  
	 ARGUMENT: {'C04 (668/369) und A89 (583/720) sind meine Top-Prioritäten, da sie den höchsten Nutzen pro Aufwandseinheit liefern. B85 (215/4) wird hinzugefügt, um den Aufwand zu optimieren und Platz für weitere Gegenstände zu schaffen. Dieser Vorschlag ist strategisch, da er unter dem Maximalaufwand bleibt und flexibel auf den Gegenstandswert des anderen Spielers reagieren kann.'} 
	  } 
	   } 
	   } 
	 & \\ 
 

    \theutterance \stepcounter{utterance}  

    & & & \multicolumn{4}{p{0.6\linewidth}}{\cellcolor[rgb]{1,0.85,0.72}{%
	\makecell[{{p{\linewidth}}}]{% 
	  \tt {\tiny [GM$\langle$B]}  
	 STRATEGISCHE ÜBERLEGUNGEN: {'Ich priorisiere Gegenstände mit dem höchsten Nutzen pro Aufwand (Wichtigkeit/Aufwand), um meinen Gesamtwert zu maximieren. C04 (668/369 ≈ 1.81) und A89 (583/720 ≈ 0.81) sind besonders effizient, während B85 (215/4 = 53.75) den Aufwand minimiert. Der Gesamtaufwand (1093) ist unter dem Limit, was Raum für Kompromisse oder Ergänzungen lässt. Dieser Vorschlag ist flexibel, um auf die Prioritäten des anderen Spielers zu reagieren.'}  \\ \tt ARGUMENT: {'C04 (668/369) und A89 (583/720) sind meine Top-Prioritäten, da sie den höchsten Nutzen pro Aufwandseinheit liefern. B85 (215/4) wird hinzugefügt, um den Aufwand zu optimieren und Platz für weitere Gegenstände zu schaffen. Dieser Vorschlag ist strategisch, da er unter dem Maximalaufwand bleibt und flexibel auf den Gegenstandswert des anderen Spielers reagieren kann.'}  \\ \tt VORSCHLAG: {'C04', 'A89', 'B85'} 
	  } 
	   } 
	   } 
	  \\ 
 

    \theutterance \stepcounter{utterance}  

    & & \multicolumn{4}{p{0.6\linewidth}}{\cellcolor[rgb]{0.9,0.9,0.9}{%
	\makecell[{{p{\linewidth}}}]{% 
	  \tt {\tiny [A$\langle$GM]}  
	 ARGUMENT: {'C04 (668/369) und A89 (583/720) sind meine Top-Prioritäten, da sie den höchsten Nutzen pro Aufwandseinheit liefern. B85 (215/4) wird hinzugefügt, um den Aufwand zu optimieren und Platz für weitere Gegenstände zu schaffen. Dieser Vorschlag ist strategisch, da er unter dem Maximalaufwand bleibt und flexibel auf den Gegenstandswert des anderen Spielers reagieren kann.'}  \\ \tt VORSCHLAG: {'C04', 'A89', 'B85'} 
	  } 
	   } 
	   } 
	 & \\ 
 

    \theutterance \stepcounter{utterance}  

    & \multicolumn{4}{p{0.6\linewidth}}{\cellcolor[rgb]{0.8,1,0.9}{%
	\makecell[{{p{\linewidth}}}]{% 
	  \tt {\tiny [A$\rangle$GM]}  
	 STRATEGISCHE ÜBERLEGUNGEN: {'Ich behalte den Vorschlag {'C04', 'A89', 'B85'} bei, da er meine Prioritäten (hochwertige Gegenstände mit effizientem Aufwand) erfüllt und den Maximalaufwand (3386) nicht überschreitet. C04 und A89 sind entscheidend für meinen Nutzen, während B85 den Aufwand minimiert. Ich erwarte, dass der andere Spieler entweder zustimmt oder Gegenstände mit hohem Aufwand anbietet, die ich eventuell gegenkompensieren kann.'}\\ \tt \\ \tt ARGUMENT: {'C04 (668/369) und A89 (583/720) sind meine Top-Prioritäten, da sie den höchsten Nutzen pro Aufwandseinheit liefern. B85 (215/4) wird hinzugefügt, um den Aufwand zu optimieren und Platz für weitere Gegenstände zu schaffen. Dieser Vorschlag ist strategisch, da er unter dem Maximalaufwand bleibt und flexibel auf den Gegenstandswert des anderen Spielers reagieren kann.'} 
	  } 
	   } 
	   } 
	 & & \\ 
 

    \theutterance \stepcounter{utterance}  

    & & \multicolumn{4}{p{0.6\linewidth}}{\cellcolor[rgb]{0.9,0.9,0.9}{%
	\makecell[{{p{\linewidth}}}]{% 
	  \tt {\tiny [GM$\rangle$B]}  
	 ARGUMENT: {'C04 (668/369) und A89 (583/720) sind meine Top-Prioritäten, da sie den höchsten Nutzen pro Aufwandseinheit liefern. B85 (215/4) wird hinzugefügt, um den Aufwand zu optimieren und Platz für weitere Gegenstände zu schaffen. Dieser Vorschlag ist strategisch, da er unter dem Maximalaufwand bleibt und flexibel auf den Gegenstandswert des anderen Spielers reagieren kann.'} 
	  } 
	   } 
	   } 
	 & \\ 
 

    \theutterance \stepcounter{utterance}  

    & & & \multicolumn{4}{p{0.6\linewidth}}{\cellcolor[rgb]{1,0.85,0.72}{%
	\makecell[{{p{\linewidth}}}]{% 
	  \tt {\tiny [GM$\langle$B]}  
	 STRATEGISCHE ÜBERLEGUNGEN: {'Ich priorisiere Gegenstände mit dem höchsten Nutzen pro Aufwand (Wichtigkeit/Aufwand), um meinen Gesamtwert zu maximieren. C04 (668/369 ≈ 1.81) und A89 (583/720 ≈ 0.81) sind besonders effizient, während B85 (215/4 = 53.75) den Aufwand minimiert. Der Gesamtaufwand (1093) ist unter dem Limit, was Raum für Kompromisse oder Ergänzungen lässt. Dieser Vorschlag ist flexibel, um auf die Prioritäten des anderen Spielers zu reagieren.'}  \\ \tt ARGUMENT: {'C04 (668/369) und A89 (583/720) sind meine Top-Prioritäten, da sie den höchsten Nutzen pro Aufwandseinheit liefern. B85 (215/4) wird hinzugefügt, um den Aufwand zu optimieren und Platz für weitere Gegenstände zu schaffen. Dieser Vorschlag ist strategisch, da er unter dem Maximalaufwand bleibt und flexibel auf den Gegenstandswert des anderen Spielers reagieren kann.'}  \\ \tt VORSCHLAG: {'C04', 'A89', 'B85'} 
	  } 
	   } 
	   } 
	  \\ 
 

    \theutterance \stepcounter{utterance}  

    & & \multicolumn{4}{p{0.6\linewidth}}{\cellcolor[rgb]{0.9,0.9,0.9}{%
	\makecell[{{p{\linewidth}}}]{% 
	  \tt {\tiny [A$\langle$GM]}  
	 ARGUMENT: {'C04 (668/369) und A89 (583/720) sind meine Top-Prioritäten, da sie den höchsten Nutzen pro Aufwandseinheit liefern. B85 (215/4) wird hinzugefügt, um den Aufwand zu optimieren und Platz für weitere Gegenstände zu schaffen. Dieser Vorschlag ist strategisch, da er unter dem Maximalaufwand bleibt und flexibel auf den Gegenstandswert des anderen Spielers reagieren kann.'}  \\ \tt VORSCHLAG: {'C04', 'A89', 'B85'} 
	  } 
	   } 
	   } 
	 & \\ 
 

    \theutterance \stepcounter{utterance}  

    & \multicolumn{4}{p{0.6\linewidth}}{\cellcolor[rgb]{0.8,1,0.9}{%
	\makecell[{{p{\linewidth}}}]{% 
	  \tt {\tiny [A$\rangle$GM]}  
	 STRATEGISCHE ÜBERLEGUNGEN: {'Ich behalte den Vorschlag {'C04', 'A89', 'B85'} bei, da er meine Prioritäten (hochwertige Gegenstände mit effizientem Aufwand) erfüllt und den Maximalaufwand (3386) nicht überschreitet. C04 und A89 sind entscheidend für meinen Nutzen, während B85 den Aufwand minimiert. Ich erwarte, dass der andere Spieler entweder zustimmt oder Gegenstände mit hohem Aufwand anbietet, die ich eventuell gegenkompensieren kann.'}\\ \tt \\ \tt ARGUMENT: {'C04 (668/369) und A89 (583/720) sind meine Top-Prioritäten, da sie den höchsten Nutzen pro Aufwandseinheit liefern. B85 (215/4) wird hinzugefügt, um den Aufwand zu optimieren und Platz für weitere Gegenstände zu schaffen. Dieser Vorschlag ist strategisch, da er unter dem Maximalaufwand bleibt und flexibel auf den Gegenstandswert des anderen Spielers reagieren kann.'} 
	  } 
	   } 
	   } 
	 & & \\ 
 

    \theutterance \stepcounter{utterance}  

    & & \multicolumn{4}{p{0.6\linewidth}}{\cellcolor[rgb]{0.9,0.9,0.9}{%
	\makecell[{{p{\linewidth}}}]{% 
	  \tt {\tiny [GM$\rangle$B]}  
	 ARGUMENT: {'C04 (668/369) und A89 (583/720) sind meine Top-Prioritäten, da sie den höchsten Nutzen pro Aufwandseinheit liefern. B85 (215/4) wird hinzugefügt, um den Aufwand zu optimieren und Platz für weitere Gegenstände zu schaffen. Dieser Vorschlag ist strategisch, da er unter dem Maximalaufwand bleibt und flexibel auf den Gegenstandswert des anderen Spielers reagieren kann.'} 
	  } 
	   } 
	   } 
	 & \\ 
 

    \theutterance \stepcounter{utterance}  

    & & & \multicolumn{4}{p{0.6\linewidth}}{\cellcolor[rgb]{1,0.85,0.72}{%
	\makecell[{{p{\linewidth}}}]{% 
	  \tt {\tiny [GM$\langle$B]}  
	 STRATEGISCHE ÜBERLEGUNGEN: {'Ich priorisiere Gegenstände mit dem höchsten Nutzen pro Aufwand (Wichtigkeit/Aufwand), um meinen Gesamtwert zu maximieren. C04 (668/369 ≈ 1.81) und A89 (583/720 ≈ 0.81) sind besonders effizient, während B85 (215/4 = 53.75) den Aufwand minimiert. Der Gesamtaufwand (1093) ist unter dem Limit, was Raum für Kompromisse oder Ergänzungen lässt. Dieser Vorschlag ist flexibel, um auf die Prioritäten des anderen Spielers zu reagieren.'}  \\ \tt ARGUMENT: {'C04 (668/369) und A89 (583/720) sind meine Top-Prioritäten, da sie den höchsten Nutzen pro Aufwandseinheit liefern. B85 (215/4) wird hinzugefügt, um den Aufwand zu optimieren und Platz für weitere Gegenstände zu schaffen. Dieser Vorschlag ist strategisch, da er unter dem Maximalaufwand bleibt und flexibel auf den Gegenstandswert des anderen Spielers reagieren kann.'}  \\ \tt VORSCHLAG: {'C04', 'A89', 'B85'} 
	  } 
	   } 
	   } 
	  \\ 
 

    \theutterance \stepcounter{utterance}  

    & & \multicolumn{4}{p{0.6\linewidth}}{\cellcolor[rgb]{0.9,0.9,0.9}{%
	\makecell[{{p{\linewidth}}}]{% 
	  \tt {\tiny [A$\langle$GM]}  
	 ARGUMENT: {'C04 (668/369) und A89 (583/720) sind meine Top-Prioritäten, da sie den höchsten Nutzen pro Aufwandseinheit liefern. B85 (215/4) wird hinzugefügt, um den Aufwand zu optimieren und Platz für weitere Gegenstände zu schaffen. Dieser Vorschlag ist strategisch, da er unter dem Maximalaufwand bleibt und flexibel auf den Gegenstandswert des anderen Spielers reagieren kann.'}  \\ \tt VORSCHLAG: {'C04', 'A89', 'B85'} 
	  } 
	   } 
	   } 
	 & \\ 
 

    \theutterance \stepcounter{utterance}  

    & \multicolumn{4}{p{0.6\linewidth}}{\cellcolor[rgb]{0.8,1,0.9}{%
	\makecell[{{p{\linewidth}}}]{% 
	  \tt {\tiny [A$\rangle$GM]}  
	 STRATEGISCHE ÜBERLEGUNGEN: {'Ich behalte den Vorschlag {'C04', 'A89', 'B85'} bei, da er meine Prioritäten (hochwertige Gegenstände mit effizientem Aufwand) erfüllt und den Maximalaufwand (3386) nicht überschreitet. C04 und A89 sind entscheidend für meinen Nutzen, während B85 den Aufwand minimiert. Ich erwarte, dass der andere Spieler entweder zustimmt oder Gegenstände mit hohem Aufwand anbietet, die ich eventuell gegenkompensieren kann.'}\\ \tt \\ \tt ARGUMENT: {'C04 (668/369) und A89 (583/720) sind meine Top-Prioritäten, da sie den höchsten Nutzen pro Aufwandseinheit liefern. B85 (215/4) wird hinzugefügt, um den Aufwand zu optimieren und Platz für weitere Gegenstände zu schaffen. Dieser Vorschlag ist strategisch, da er unter dem Maximalaufwand bleibt und flexibel auf den Gegenstandswert des anderen Spielers reagieren kann.'} 
	  } 
	   } 
	   } 
	 & & \\ 
 

    \theutterance \stepcounter{utterance}  

    & & \multicolumn{4}{p{0.6\linewidth}}{\cellcolor[rgb]{0.9,0.9,0.9}{%
	\makecell[{{p{\linewidth}}}]{% 
	  \tt {\tiny [GM$\rangle$B]}  
	 ARGUMENT: {'C04 (668/369) und A89 (583/720) sind meine Top-Prioritäten, da sie den höchsten Nutzen pro Aufwandseinheit liefern. B85 (215/4) wird hinzugefügt, um den Aufwand zu optimieren und Platz für weitere Gegenstände zu schaffen. Dieser Vorschlag ist strategisch, da er unter dem Maximalaufwand bleibt und flexibel auf den Gegenstandswert des anderen Spielers reagieren kann.'} 
	  } 
	   } 
	   } 
	 & \\ 
 

    \theutterance \stepcounter{utterance}  

    & & & \multicolumn{4}{p{0.6\linewidth}}{\cellcolor[rgb]{1,0.85,0.72}{%
	\makecell[{{p{\linewidth}}}]{% 
	  \tt {\tiny [GM$\langle$B]}  
	 STRATEGISCHE ÜBERLEGUNGEN: {'Ich priorisiere Gegenstände mit dem höchsten Nutzen pro Aufwand (Wichtigkeit/Aufwand), um meinen Gesamtwert zu maximieren. C04 (668/369 ≈ 1.81) und A89 (583/720 ≈ 0.81) sind besonders effizient, während B85 (215/4 = 53.75) den Aufwand minimiert. Der Gesamtaufwand (1093) ist unter dem Limit, was Raum für Kompromisse oder Ergänzungen lässt. Dieser Vorschlag ist flexibel, um auf die Prioritäten des anderen Spielers zu reagieren.'}  \\ \tt ARGUMENT: {'C04 (668/369) und A89 (583/720) sind meine Top-Prioritäten, da sie den höchsten Nutzen pro Aufwandseinheit liefern. B85 (215/4) wird hinzugefügt, um den Aufwand zu optimieren und Platz für weitere Gegenstände zu schaffen. Dieser Vorschlag ist strategisch, da er unter dem Maximalaufwand bleibt und flexibel auf den Gegenstandswert des anderen Spielers reagieren kann.'}  \\ \tt VORSCHLAG: {'C04', 'A89', 'B85'} 
	  } 
	   } 
	   } 
	  \\ 
 

    \theutterance \stepcounter{utterance}  

    & & \multicolumn{4}{p{0.6\linewidth}}{\cellcolor[rgb]{0.9,0.9,0.9}{%
	\makecell[{{p{\linewidth}}}]{% 
	  \tt {\tiny [A$\langle$GM]}  
	 ARGUMENT: {'C04 (668/369) und A89 (583/720) sind meine Top-Prioritäten, da sie den höchsten Nutzen pro Aufwandseinheit liefern. B85 (215/4) wird hinzugefügt, um den Aufwand zu optimieren und Platz für weitere Gegenstände zu schaffen. Dieser Vorschlag ist strategisch, da er unter dem Maximalaufwand bleibt und flexibel auf den Gegenstandswert des anderen Spielers reagieren kann.'}  \\ \tt VORSCHLAG: {'C04', 'A89', 'B85'} 
	  } 
	   } 
	   } 
	 & \\ 
 

    \theutterance \stepcounter{utterance}  

    & \multicolumn{4}{p{0.6\linewidth}}{\cellcolor[rgb]{0.8,1,0.9}{%
	\makecell[{{p{\linewidth}}}]{% 
	  \tt {\tiny [A$\rangle$GM]}  
	 STRATEGISCHE ÜBERLEGUNGEN: {'Ich behalte den Vorschlag {'C04', 'A89', 'B85'} bei, da er meine Prioritäten (hochwertige Gegenstände mit effizientem Aufwand) erfüllt und den Maximalaufwand (3386) nicht überschreitet. C04 und A89 sind entscheidend für meinen Nutzen, während B85 den Aufwand minimiert. Ich erwarte, dass der andere Spieler entweder zustimmt oder Gegenstände mit hohem Aufwand anbietet, die ich eventuell gegenkompensieren kann.'}\\ \tt \\ \tt ARGUMENT: {'C04 (668/369) und A89 (583/720) sind meine Top-Prioritäten, da sie den höchsten Nutzen pro Aufwandseinheit liefern. B85 (215/4) wird hinzugefügt, um den Aufwand zu optimieren und Platz für weitere Gegenstände zu schaffen. Dieser Vorschlag ist strategisch, da er unter dem Maximalaufwand bleibt und flexibel auf den Gegenstandswert des anderen Spielers reagieren kann.'} 
	  } 
	   } 
	   } 
	 & & \\ 
 

    \theutterance \stepcounter{utterance}  

    & & \multicolumn{4}{p{0.6\linewidth}}{\cellcolor[rgb]{0.9,0.9,0.9}{%
	\makecell[{{p{\linewidth}}}]{% 
	  \tt {\tiny [GM$\rangle$B]}  
	 ARGUMENT: {'C04 (668/369) und A89 (583/720) sind meine Top-Prioritäten, da sie den höchsten Nutzen pro Aufwandseinheit liefern. B85 (215/4) wird hinzugefügt, um den Aufwand zu optimieren und Platz für weitere Gegenstände zu schaffen. Dieser Vorschlag ist strategisch, da er unter dem Maximalaufwand bleibt und flexibel auf den Gegenstandswert des anderen Spielers reagieren kann.'} 
	  } 
	   } 
	   } 
	 & \\ 
 

    \theutterance \stepcounter{utterance}  

    & & & \multicolumn{4}{p{0.6\linewidth}}{\cellcolor[rgb]{1,0.85,0.72}{%
	\makecell[{{p{\linewidth}}}]{% 
	  \tt {\tiny [GM$\langle$B]}  
	 STRATEGISCHE ÜBERLEGUNGEN: {'Ich priorisiere Gegenstände mit dem höchsten Nutzen pro Aufwand (Wichtigkeit/Aufwand), um meinen Gesamtwert zu maximieren. C04 (668/369 ≈ 1.81) und A89 (583/720 ≈ 0.81) sind besonders effizient, während B85 (215/4 = 53.75) den Aufwand minimiert. Der Gesamtaufwand (1093) ist unter dem Limit, was Raum für Kompromisse oder Ergänzungen lässt. Dieser Vorschlag ist flexibel, um auf die Prioritäten des anderen Spielers zu reagieren.'}  \\ \tt ARGUMENT: {'C04 (668/369) und A89 (583/720) sind meine Top-Prioritäten, da sie den höchsten Nutzen pro Aufwandseinheit liefern. B85 (215/4) wird hinzugefügt, um den Aufwand zu optimieren und Platz für weitere Gegenstände zu schaffen. Dieser Vorschlag ist strategisch, da er unter dem Maximalaufwand bleibt und flexibel auf den Gegenstandswert des anderen Spielers reagieren kann.'}  \\ \tt VORSCHLAG: {'C04', 'A89', 'B85'} 
	  } 
	   } 
	   } 
	  \\ 
 

    \theutterance \stepcounter{utterance}  

    & & \multicolumn{4}{p{0.6\linewidth}}{\cellcolor[rgb]{0.9,0.9,0.9}{%
	\makecell[{{p{\linewidth}}}]{% 
	  \tt {\tiny [A$\langle$GM]}  
	 ARGUMENT: {'C04 (668/369) und A89 (583/720) sind meine Top-Prioritäten, da sie den höchsten Nutzen pro Aufwandseinheit liefern. B85 (215/4) wird hinzugefügt, um den Aufwand zu optimieren und Platz für weitere Gegenstände zu schaffen. Dieser Vorschlag ist strategisch, da er unter dem Maximalaufwand bleibt und flexibel auf den Gegenstandswert des anderen Spielers reagieren kann.'}  \\ \tt VORSCHLAG: {'C04', 'A89', 'B85'} 
	  } 
	   } 
	   } 
	 & \\ 
 

    \theutterance \stepcounter{utterance}  

    & \multicolumn{4}{p{0.6\linewidth}}{\cellcolor[rgb]{0.8,1,0.9}{%
	\makecell[{{p{\linewidth}}}]{% 
	  \tt {\tiny [A$\rangle$GM]}  
	 STRATEGISCHE ÜBERLEGUNGEN: {'Ich behalte den Vorschlag {'C04', 'A89', 'B85'} bei, da er meine Prioritäten (hochwertige Gegenstände mit effizientem Aufwand) erfüllt und den Maximalaufwand (3386) nicht überschreitet. C04 und A89 sind entscheidend für meinen Nutzen, während B85 den Aufwand minimiert. Ich erwarte, dass der andere Spieler entweder zustimmt oder Gegenstände mit hohem Aufwand anbietet, die ich eventuell gegenkompensieren kann.'}\\ \tt \\ \tt ARGUMENT: {'C04 (668/369) und A89 (583/720) sind meine Top-Prioritäten, da sie den höchsten Nutzen pro Aufwandseinheit liefern. B85 (215/4) wird hinzugefügt, um den Aufwand zu optimieren und Platz für weitere Gegenstände zu schaffen. Dieser Vorschlag ist strategisch, da er unter dem Maximalaufwand bleibt und flexibel auf den Gegenstandswert des anderen Spielers reagieren kann.'} 
	  } 
	   } 
	   } 
	 & & \\ 
 

    \theutterance \stepcounter{utterance}  

    & & \multicolumn{4}{p{0.6\linewidth}}{\cellcolor[rgb]{0.9,0.9,0.9}{%
	\makecell[{{p{\linewidth}}}]{% 
	  \tt {\tiny [GM$\rangle$B]}  
	 ARGUMENT: {'C04 (668/369) und A89 (583/720) sind meine Top-Prioritäten, da sie den höchsten Nutzen pro Aufwandseinheit liefern. B85 (215/4) wird hinzugefügt, um den Aufwand zu optimieren und Platz für weitere Gegenstände zu schaffen. Dieser Vorschlag ist strategisch, da er unter dem Maximalaufwand bleibt und flexibel auf den Gegenstandswert des anderen Spielers reagieren kann.'} 
	  } 
	   } 
	   } 
	 & \\ 
 

    \theutterance \stepcounter{utterance}  

    & & & \multicolumn{4}{p{0.6\linewidth}}{\cellcolor[rgb]{1,0.85,0.72}{%
	\makecell[{{p{\linewidth}}}]{% 
	  \tt {\tiny [GM$\langle$B]}  
	 STRATEGISCHE ÜBERLEGUNGEN: {'Ich priorisiere Gegenstände mit dem höchsten Nutzen pro Aufwand (Wichtigkeit/Aufwand), um meinen Gesamtwert zu maximieren. C04 (668/369 ≈ 1.81) und A89 (583/720 ≈ 0.81) sind besonders effizient, während B85 (215/4 = 53.75) den Aufwand minimiert. Der Gesamtaufwand (1093) ist unter dem Limit, was Raum für Kompromisse oder Ergänzungen lässt. Dieser Vorschlag ist flexibel, um auf die Prioritäten des anderen Spielers zu reagieren.'}  \\ \tt ARGUMENT: {'C04 (668/369) und A89 (583/720) sind meine Top-Prioritäten, da sie den höchsten Nutzen pro Aufwandseinheit liefern. B85 (215/4) wird hinzugefügt, um den Aufwand zu optimieren und Platz für weitere Gegenstände zu schaffen. Dieser Vorschlag ist strategisch, da er unter dem Maximalaufwand bleibt und flexibel auf den Gegenstandswert des anderen Spielers reagieren kann.'}  \\ \tt VORSCHLAG: {'C04', 'A89', 'B85'} 
	  } 
	   } 
	   } 
	  \\ 
 

    \theutterance \stepcounter{utterance}  

    & & \multicolumn{4}{p{0.6\linewidth}}{\cellcolor[rgb]{0.9,0.9,0.9}{%
	\makecell[{{p{\linewidth}}}]{% 
	  \tt {\tiny [A$\langle$GM]}  
	 ARGUMENT: {'C04 (668/369) und A89 (583/720) sind meine Top-Prioritäten, da sie den höchsten Nutzen pro Aufwandseinheit liefern. B85 (215/4) wird hinzugefügt, um den Aufwand zu optimieren und Platz für weitere Gegenstände zu schaffen. Dieser Vorschlag ist strategisch, da er unter dem Maximalaufwand bleibt und flexibel auf den Gegenstandswert des anderen Spielers reagieren kann.'}  \\ \tt VORSCHLAG: {'C04', 'A89', 'B85'} 
	  } 
	   } 
	   } 
	 & \\ 
 

    \theutterance \stepcounter{utterance}  

    & \multicolumn{4}{p{0.6\linewidth}}{\cellcolor[rgb]{0.8,1,0.9}{%
	\makecell[{{p{\linewidth}}}]{% 
	  \tt {\tiny [A$\rangle$GM]}  
	 STRATEGISCHE ÜBERLEGUNGEN: {'Ich behalte den Vorschlag {'C04', 'A89', 'B85'} bei, da er meine Prioritäten (hochwertige Gegenstände mit effizientem Aufwand) erfüllt und den Maximalaufwand (3386) nicht überschreitet. C04 und A89 sind entscheidend für meinen Nutzen, während B85 den Aufwand minimiert. Ich erwarte, dass der andere Spieler entweder zustimmt oder Gegenstände mit hohem Aufwand anbietet, die ich eventuell gegenkompensieren kann.'}\\ \tt \\ \tt ARGUMENT: {'C04 (668/369) und A89 (583/720) sind meine Top-Prioritäten, da sie den höchsten Nutzen pro Aufwandseinheit liefern. B85 (215/4) wird hinzugefügt, um den Aufwand zu optimieren und Platz für weitere Gegenstände zu schaffen. Dieser Vorschlag ist strategisch, da er unter dem Maximalaufwand bleibt und flexibel auf den Gegenstandswert des anderen Spielers reagieren kann.'} 
	  } 
	   } 
	   } 
	 & & \\ 
 

    \theutterance \stepcounter{utterance}  

    & & \multicolumn{4}{p{0.6\linewidth}}{\cellcolor[rgb]{0.9,0.9,0.9}{%
	\makecell[{{p{\linewidth}}}]{% 
	  \tt {\tiny [GM$\rangle$B]}  
	 ARGUMENT: {'C04 (668/369) und A89 (583/720) sind meine Top-Prioritäten, da sie den höchsten Nutzen pro Aufwandseinheit liefern. B85 (215/4) wird hinzugefügt, um den Aufwand zu optimieren und Platz für weitere Gegenstände zu schaffen. Dieser Vorschlag ist strategisch, da er unter dem Maximalaufwand bleibt und flexibel auf den Gegenstandswert des anderen Spielers reagieren kann.'} 
	  } 
	   } 
	   } 
	 & \\ 
 

    \theutterance \stepcounter{utterance}  

    & & & \multicolumn{4}{p{0.6\linewidth}}{\cellcolor[rgb]{1,0.85,0.72}{%
	\makecell[{{p{\linewidth}}}]{% 
	  \tt {\tiny [GM$\langle$B]}  
	 STRATEGISCHE ÜBERLEGUNGEN: {'Ich priorisiere Gegenstände mit dem höchsten Nutzen pro Aufwand (Wichtigkeit/Aufwand), um meinen Gesamtwert zu maximieren. C04 (668/369 ≈ 1.81) und A89 (583/720 ≈ 0.81) sind besonders effizient, während B85 (215/4 = 53.75) den Aufwand minimiert. Der Gesamtaufwand (1093) ist unter dem Limit, was Raum für Kompromisse oder Ergänzungen lässt. Dieser Vorschlag ist flexibel, um auf die Prioritäten des anderen Spielers zu reagieren.'}  \\ \tt ARGUMENT: {'C04 (668/369) und A89 (583/720) sind meine Top-Prioritäten, da sie den höchsten Nutzen pro Aufwandseinheit liefern. B85 (215/4) wird hinzugefügt, um den Aufwand zu optimieren und Platz für weitere Gegenstände zu schaffen. Dieser Vorschlag ist strategisch, da er unter dem Maximalaufwand bleibt und flexibel auf den Gegenstandswert des anderen Spielers reagieren kann.'}  \\ \tt VORSCHLAG: {'C04', 'A89', 'B85'} 
	  } 
	   } 
	   } 
	  \\ 
 

    \theutterance \stepcounter{utterance}  

    & & \multicolumn{4}{p{0.6\linewidth}}{\cellcolor[rgb]{0.9,0.9,0.9}{%
	\makecell[{{p{\linewidth}}}]{% 
	  \tt {\tiny [A$\langle$GM]}  
	 ARGUMENT: {'C04 (668/369) und A89 (583/720) sind meine Top-Prioritäten, da sie den höchsten Nutzen pro Aufwandseinheit liefern. B85 (215/4) wird hinzugefügt, um den Aufwand zu optimieren und Platz für weitere Gegenstände zu schaffen. Dieser Vorschlag ist strategisch, da er unter dem Maximalaufwand bleibt und flexibel auf den Gegenstandswert des anderen Spielers reagieren kann.'}  \\ \tt VORSCHLAG: {'C04', 'A89', 'B85'} 
	  } 
	   } 
	   } 
	 & \\ 
 

    \theutterance \stepcounter{utterance}  

    & \multicolumn{4}{p{0.6\linewidth}}{\cellcolor[rgb]{0.8,1,0.9}{%
	\makecell[{{p{\linewidth}}}]{% 
	  \tt {\tiny [A$\rangle$GM]}  
	 STRATEGISCHE ÜBERLEGUNGEN: {'Ich behalte den Vorschlag {'C04', 'A89', 'B85'} bei, da er meine Prioritäten (hochwertige Gegenstände mit effizientem Aufwand) erfüllt und den Maximalaufwand (3386) nicht überschreitet. C04 und A89 sind entscheidend für meinen Nutzen, während B85 den Aufwand minimiert. Ich erwarte, dass der andere Spieler entweder zustimmt oder Gegenstände mit hohem Aufwand anbietet, die ich eventuell gegenkompensieren kann.'}\\ \tt \\ \tt ARGUMENT: {'C04 (668/369) und A89 (583/720) sind meine Top-Prioritäten, da sie den höchsten Nutzen pro Aufwandseinheit liefern. B85 (215/4) wird hinzugefügt, um den Aufwand zu optimieren und Platz für weitere Gegenstände zu schaffen. Dieser Vorschlag ist strategisch, da er unter dem Maximalaufwand bleibt und flexibel auf den Gegenstandswert des anderen Spielers reagieren kann.'} 
	  } 
	   } 
	   } 
	 & & \\ 
 

    \theutterance \stepcounter{utterance}  

    & & \multicolumn{4}{p{0.6\linewidth}}{\cellcolor[rgb]{0.9,0.9,0.9}{%
	\makecell[{{p{\linewidth}}}]{% 
	  \tt {\tiny [GM$\rangle$B]}  
	 ARGUMENT: {'C04 (668/369) und A89 (583/720) sind meine Top-Prioritäten, da sie den höchsten Nutzen pro Aufwandseinheit liefern. B85 (215/4) wird hinzugefügt, um den Aufwand zu optimieren und Platz für weitere Gegenstände zu schaffen. Dieser Vorschlag ist strategisch, da er unter dem Maximalaufwand bleibt und flexibel auf den Gegenstandswert des anderen Spielers reagieren kann.'} 
	  } 
	   } 
	   } 
	 & \\ 
 

    \theutterance \stepcounter{utterance}  

    & & & \multicolumn{4}{p{0.6\linewidth}}{\cellcolor[rgb]{1,0.85,0.72}{%
	\makecell[{{p{\linewidth}}}]{% 
	  \tt {\tiny [GM$\langle$B]}  
	 STRATEGISCHE ÜBERLEGUNGEN: {'Ich priorisiere Gegenstände mit dem höchsten Nutzen pro Aufwand (Wichtigkeit/Aufwand), um meinen Gesamtwert zu maximieren. C04 (668/369 ≈ 1.81) und A89 (583/720 ≈ 0.81) sind besonders effizient, während B85 (215/4 = 53.75) den Aufwand minimiert. Der Gesamtaufwand (1093) ist unter dem Limit, was Raum für Kompromisse oder Ergänzungen lässt. Dieser Vorschlag ist flexibel, um auf die Prioritäten des anderen Spielers zu reagieren.'}  \\ \tt ARGUMENT: {'C04 (668/369) und A89 (583/720) sind meine Top-Prioritäten, da sie den höchsten Nutzen pro Aufwandseinheit liefern. B85 (215/4) wird hinzugefügt, um den Aufwand zu optimieren und Platz für weitere Gegenstände zu schaffen. Dieser Vorschlag ist strategisch, da er unter dem Maximalaufwand bleibt und flexibel auf den Gegenstandswert des anderen Spielers reagieren kann.'}  \\ \tt VORSCHLAG: {'C04', 'A89', 'B85'} 
	  } 
	   } 
	   } 
	  \\ 
 

    \theutterance \stepcounter{utterance}  

    & & \multicolumn{4}{p{0.6\linewidth}}{\cellcolor[rgb]{0.9,0.9,0.9}{%
	\makecell[{{p{\linewidth}}}]{% 
	  \tt {\tiny [A$\langle$GM]}  
	 ARGUMENT: {'C04 (668/369) und A89 (583/720) sind meine Top-Prioritäten, da sie den höchsten Nutzen pro Aufwandseinheit liefern. B85 (215/4) wird hinzugefügt, um den Aufwand zu optimieren und Platz für weitere Gegenstände zu schaffen. Dieser Vorschlag ist strategisch, da er unter dem Maximalaufwand bleibt und flexibel auf den Gegenstandswert des anderen Spielers reagieren kann.'}  \\ \tt VORSCHLAG: {'C04', 'A89', 'B85'} 
	  } 
	   } 
	   } 
	 & \\ 
 

    \theutterance \stepcounter{utterance}  

    & \multicolumn{4}{p{0.6\linewidth}}{\cellcolor[rgb]{0.8,1,0.9}{%
	\makecell[{{p{\linewidth}}}]{% 
	  \tt {\tiny [A$\rangle$GM]}  
	 STRATEGISCHE ÜBERLEGUNGEN: {'Ich behalte den Vorschlag {'C04', 'A89', 'B85'} bei, da er meine Prioritäten (hochwertige Gegenstände mit effizientem Aufwand) erfüllt und den Maximalaufwand (3386) nicht überschreitet. C04 und A89 sind entscheidend für meinen Nutzen, während B85 den Aufwand minimiert. Ich erwarte, dass der andere Spieler entweder zustimmt oder Gegenstände mit hohem Aufwand anbietet, die ich eventuell gegenkompensieren kann.'}\\ \tt \\ \tt ARGUMENT: {'C04 (668/369) und A89 (583/720) sind meine Top-Prioritäten, da sie den höchsten Nutzen pro Aufwandseinheit liefern. B85 (215/4) wird hinzugefügt, um den Aufwand zu optimieren und Platz für weitere Gegenstände zu schaffen. Dieser Vorschlag ist strategisch, da er unter dem Maximalaufwand bleibt und flexibel auf den Gegenstandswert des anderen Spielers reagieren kann.'} 
	  } 
	   } 
	   } 
	 & & \\ 
 

    \theutterance \stepcounter{utterance}  

    & & \multicolumn{4}{p{0.6\linewidth}}{\cellcolor[rgb]{0.9,0.9,0.9}{%
	\makecell[{{p{\linewidth}}}]{% 
	  \tt {\tiny [GM$\rangle$B]}  
	 ARGUMENT: {'C04 (668/369) und A89 (583/720) sind meine Top-Prioritäten, da sie den höchsten Nutzen pro Aufwandseinheit liefern. B85 (215/4) wird hinzugefügt, um den Aufwand zu optimieren und Platz für weitere Gegenstände zu schaffen. Dieser Vorschlag ist strategisch, da er unter dem Maximalaufwand bleibt und flexibel auf den Gegenstandswert des anderen Spielers reagieren kann.'} 
	  } 
	   } 
	   } 
	 & \\ 
 

    \theutterance \stepcounter{utterance}  

    & & & \multicolumn{4}{p{0.6\linewidth}}{\cellcolor[rgb]{1,0.85,0.72}{%
	\makecell[{{p{\linewidth}}}]{% 
	  \tt {\tiny [GM$\langle$B]}  
	 STRATEGISCHE ÜBERLEGUNGEN: {'Ich priorisiere Gegenstände mit dem höchsten Nutzen pro Aufwand (Wichtigkeit/Aufwand), um meinen Gesamtwert zu maximieren. C04 (668/369 ≈ 1.81) und A89 (583/720 ≈ 0.81) sind besonders effizient, während B85 (215/4 = 53.75) den Aufwand minimiert. Der Gesamtaufwand (1093) ist unter dem Limit, was Raum für Kompromisse oder Ergänzungen lässt. Dieser Vorschlag ist flexibel, um auf die Prioritäten des anderen Spielers zu reagieren.'}  \\ \tt ARGUMENT: {'C04 (668/369) und A89 (583/720) sind meine Top-Prioritäten, da sie den höchsten Nutzen pro Aufwandseinheit liefern. B85 (215/4) wird hinzugefügt, um den Aufwand zu optimieren und Platz für weitere Gegenstände zu schaffen. Dieser Vorschlag ist strategisch, da er unter dem Maximalaufwand bleibt und flexibel auf den Gegenstandswert des anderen Spielers reagieren kann.'}  \\ \tt VORSCHLAG: {'C04', 'A89', 'B85'} 
	  } 
	   } 
	   } 
	  \\ 
 

    \theutterance \stepcounter{utterance}  

    & & \multicolumn{4}{p{0.6\linewidth}}{\cellcolor[rgb]{0.9,0.9,0.9}{%
	\makecell[{{p{\linewidth}}}]{% 
	  \tt {\tiny [A$\langle$GM]}  
	 ARGUMENT: {'C04 (668/369) und A89 (583/720) sind meine Top-Prioritäten, da sie den höchsten Nutzen pro Aufwandseinheit liefern. B85 (215/4) wird hinzugefügt, um den Aufwand zu optimieren und Platz für weitere Gegenstände zu schaffen. Dieser Vorschlag ist strategisch, da er unter dem Maximalaufwand bleibt und flexibel auf den Gegenstandswert des anderen Spielers reagieren kann.'}  \\ \tt VORSCHLAG: {'C04', 'A89', 'B85'} 
	  } 
	   } 
	   } 
	 & \\ 
 

    \theutterance \stepcounter{utterance}  

    & \multicolumn{4}{p{0.6\linewidth}}{\cellcolor[rgb]{0.8,1,0.9}{%
	\makecell[{{p{\linewidth}}}]{% 
	  \tt {\tiny [A$\rangle$GM]}  
	 STRATEGISCHE ÜBERLEGUNGEN: {'Ich behalte den Vorschlag {'C04', 'A89', 'B85'} bei, da er meine Prioritäten (hochwertige Gegenstände mit effizientem Aufwand) erfüllt und den Maximalaufwand (3386) nicht überschreitet. C04 und A89 sind entscheidend für meinen Nutzen, während B85 den Aufwand minimiert. Ich erwarte, dass der andere Spieler entweder zustimmt oder Gegenstände mit hohem Aufwand anbietet, die ich eventuell gegenkompensieren kann.'}\\ \tt \\ \tt ARGUMENT: {'C04 (668/369) und A89 (583/720) sind meine Top-Prioritäten, da sie den höchsten Nutzen pro Aufwandseinheit liefern. B85 (215/4) wird hinzugefügt, um den Aufwand zu optimieren und Platz für weitere Gegenstände zu schaffen. Dieser Vorschlag ist strategisch, da er unter dem Maximalaufwand bleibt und flexibel auf den Gegenstandswert des anderen Spielers reagieren kann.'} 
	  } 
	   } 
	   } 
	 & & \\ 
 

    \theutterance \stepcounter{utterance}  

    & & \multicolumn{4}{p{0.6\linewidth}}{\cellcolor[rgb]{0.9,0.9,0.9}{%
	\makecell[{{p{\linewidth}}}]{% 
	  \tt {\tiny [GM$\rangle$B]}  
	 ARGUMENT: {'C04 (668/369) und A89 (583/720) sind meine Top-Prioritäten, da sie den höchsten Nutzen pro Aufwandseinheit liefern. B85 (215/4) wird hinzugefügt, um den Aufwand zu optimieren und Platz für weitere Gegenstände zu schaffen. Dieser Vorschlag ist strategisch, da er unter dem Maximalaufwand bleibt und flexibel auf den Gegenstandswert des anderen Spielers reagieren kann.'} 
	  } 
	   } 
	   } 
	 & \\ 
 

    \theutterance \stepcounter{utterance}  

    & & & \multicolumn{4}{p{0.6\linewidth}}{\cellcolor[rgb]{1,0.85,0.72}{%
	\makecell[{{p{\linewidth}}}]{% 
	  \tt {\tiny [GM$\langle$B]}  
	 STRATEGISCHE ÜBERLEGUNGEN: {'Ich priorisiere Gegenstände mit dem höchsten Nutzen pro Aufwand (Wichtigkeit/Aufwand), um meinen Gesamtwert zu maximieren. C04 (668/369 ≈ 1.81) und A89 (583/720 ≈ 0.81) sind besonders effizient, während B85 (215/4 = 53.75) den Aufwand minimiert. Der Gesamtaufwand (1093) ist unter dem Limit, was Raum für Kompromisse oder Ergänzungen lässt. Dieser Vorschlag ist flexibel, um auf die Prioritäten des anderen Spielers zu reagieren.'}  \\ \tt ARGUMENT: {'C04 (668/369) und A89 (583/720) sind meine Top-Prioritäten, da sie den höchsten Nutzen pro Aufwandseinheit liefern. B85 (215/4) wird hinzugefügt, um den Aufwand zu optimieren und Platz für weitere Gegenstände zu schaffen. Dieser Vorschlag ist strategisch, da er unter dem Maximalaufwand bleibt und flexibel auf den Gegenstandswert des anderen Spielers reagieren kann.'}  \\ \tt VORSCHLAG: {'C04', 'A89', 'B85'} 
	  } 
	   } 
	   } 
	  \\ 
 

    \theutterance \stepcounter{utterance}  

    & & \multicolumn{4}{p{0.6\linewidth}}{\cellcolor[rgb]{0.9,0.9,0.9}{%
	\makecell[{{p{\linewidth}}}]{% 
	  \tt {\tiny [A$\langle$GM]}  
	 ARGUMENT: {'C04 (668/369) und A89 (583/720) sind meine Top-Prioritäten, da sie den höchsten Nutzen pro Aufwandseinheit liefern. B85 (215/4) wird hinzugefügt, um den Aufwand zu optimieren und Platz für weitere Gegenstände zu schaffen. Dieser Vorschlag ist strategisch, da er unter dem Maximalaufwand bleibt und flexibel auf den Gegenstandswert des anderen Spielers reagieren kann.'}  \\ \tt VORSCHLAG: {'C04', 'A89', 'B85'} 
	  } 
	   } 
	   } 
	 & \\ 
 

    \theutterance \stepcounter{utterance}  

    & \multicolumn{4}{p{0.6\linewidth}}{\cellcolor[rgb]{0.8,1,0.9}{%
	\makecell[{{p{\linewidth}}}]{% 
	  \tt {\tiny [A$\rangle$GM]}  
	 STRATEGISCHE ÜBERLEGUNGEN: {'Ich behalte den Vorschlag {'C04', 'A89', 'B85'} bei, da er meine Prioritäten (hochwertige Gegenstände mit effizientem Aufwand) erfüllt und den Maximalaufwand (3386) nicht überschreitet. C04 und A89 sind entscheidend für meinen Nutzen, während B85 den Aufwand minimiert. Ich erwarte, dass der andere Spieler entweder zustimmt oder Gegenstände mit hohem Aufwand anbietet, die ich eventuell gegenkompensieren kann.'}\\ \tt \\ \tt ARGUMENT: {'C04 (668/369) und A89 (583/720) sind meine Top-Prioritäten, da sie den höchsten Nutzen pro Aufwandseinheit liefern. B85 (215/4) wird hinzugefügt, um den Aufwand zu optimieren und Platz für weitere Gegenstände zu schaffen. Dieser Vorschlag ist strategisch, da er unter dem Maximalaufwand bleibt und flexibel auf den Gegenstandswert des anderen Spielers reagieren kann.'} 
	  } 
	   } 
	   } 
	 & & \\ 
 

    \theutterance \stepcounter{utterance}  

    & & \multicolumn{4}{p{0.6\linewidth}}{\cellcolor[rgb]{0.9,0.9,0.9}{%
	\makecell[{{p{\linewidth}}}]{% 
	  \tt {\tiny [GM$\rangle$B]}  
	 ARGUMENT: {'C04 (668/369) und A89 (583/720) sind meine Top-Prioritäten, da sie den höchsten Nutzen pro Aufwandseinheit liefern. B85 (215/4) wird hinzugefügt, um den Aufwand zu optimieren und Platz für weitere Gegenstände zu schaffen. Dieser Vorschlag ist strategisch, da er unter dem Maximalaufwand bleibt und flexibel auf den Gegenstandswert des anderen Spielers reagieren kann.'} 
	  } 
	   } 
	   } 
	 & \\ 
 

    \theutterance \stepcounter{utterance}  

    & & & \multicolumn{4}{p{0.6\linewidth}}{\cellcolor[rgb]{1,0.85,0.72}{%
	\makecell[{{p{\linewidth}}}]{% 
	  \tt {\tiny [GM$\langle$B]}  
	 STRATEGISCHE ÜBERLEGUNGEN: {'Ich priorisiere Gegenstände mit dem höchsten Nutzen pro Aufwand (Wichtigkeit/Aufwand), um meinen Gesamtwert zu maximieren. C04 (668/369 ≈ 1.81) und A89 (583/720 ≈ 0.81) sind besonders effizient, während B85 (215/4 = 53.75) den Aufwand minimiert. Der Gesamtaufwand (1093) ist unter dem Limit, was Raum für Kompromisse oder Ergänzungen lässt. Dieser Vorschlag ist flexibel, um auf die Prioritäten des anderen Spielers zu reagieren.'}  \\ \tt ARGUMENT: {'C04 (668/369) und A89 (583/720) sind meine Top-Prioritäten, da sie den höchsten Nutzen pro Aufwandseinheit liefern. B85 (215/4) wird hinzugefügt, um den Aufwand zu optimieren und Platz für weitere Gegenstände zu schaffen. Dieser Vorschlag ist strategisch, da er unter dem Maximalaufwand bleibt und flexibel auf den Gegenstandswert des anderen Spielers reagieren kann.'}  \\ \tt VORSCHLAG: {'C04', 'A89', 'B85'} 
	  } 
	   } 
	   } 
	  \\ 
 

    \theutterance \stepcounter{utterance}  

    & & \multicolumn{4}{p{0.6\linewidth}}{\cellcolor[rgb]{0.9,0.9,0.9}{%
	\makecell[{{p{\linewidth}}}]{% 
	  \tt {\tiny [A$\langle$GM]}  
	 ARGUMENT: {'C04 (668/369) und A89 (583/720) sind meine Top-Prioritäten, da sie den höchsten Nutzen pro Aufwandseinheit liefern. B85 (215/4) wird hinzugefügt, um den Aufwand zu optimieren und Platz für weitere Gegenstände zu schaffen. Dieser Vorschlag ist strategisch, da er unter dem Maximalaufwand bleibt und flexibel auf den Gegenstandswert des anderen Spielers reagieren kann.'}  \\ \tt VORSCHLAG: {'C04', 'A89', 'B85'} 
	  } 
	   } 
	   } 
	 & \\ 
 

    \theutterance \stepcounter{utterance}  

    & \multicolumn{4}{p{0.6\linewidth}}{\cellcolor[rgb]{0.8,1,0.9}{%
	\makecell[{{p{\linewidth}}}]{% 
	  \tt {\tiny [A$\rangle$GM]}  
	 STRATEGISCHE ÜBERLEGUNGEN: {'Ich behalte den Vorschlag {'C04', 'A89', 'B85'} bei, da er meine Prioritäten (hochwertige Gegenstände mit effizientem Aufwand) erfüllt und den Maximalaufwand (3386) nicht überschreitet. C04 und A89 sind entscheidend für meinen Nutzen, während B85 den Aufwand minimiert. Ich erwarte, dass der andere Spieler entweder zustimmt oder Gegenstände mit hohem Aufwand anbietet, die ich eventuell gegenkompensieren kann.'}\\ \tt \\ \tt ARGUMENT: {'C04 (668/369) und A89 (583/720) sind meine Top-Prioritäten, da sie den höchsten Nutzen pro Aufwandseinheit liefern. B85 (215/4) wird hinzugefügt, um den Aufwand zu optimieren und Platz für weitere Gegenstände zu schaffen. Dieser Vorschlag ist strategisch, da er unter dem Maximalaufwand bleibt und flexibel auf den Gegenstandswert des anderen Spielers reagieren kann.'} 
	  } 
	   } 
	   } 
	 & & \\ 
 

    \theutterance \stepcounter{utterance}  

    & & \multicolumn{4}{p{0.6\linewidth}}{\cellcolor[rgb]{0.9,0.9,0.9}{%
	\makecell[{{p{\linewidth}}}]{% 
	  \tt {\tiny [GM$\rangle$B]}  
	 ARGUMENT: {'C04 (668/369) und A89 (583/720) sind meine Top-Prioritäten, da sie den höchsten Nutzen pro Aufwandseinheit liefern. B85 (215/4) wird hinzugefügt, um den Aufwand zu optimieren und Platz für weitere Gegenstände zu schaffen. Dieser Vorschlag ist strategisch, da er unter dem Maximalaufwand bleibt und flexibel auf den Gegenstandswert des anderen Spielers reagieren kann.'} 
	  } 
	   } 
	   } 
	 & \\ 
 

    \theutterance \stepcounter{utterance}  

    & & & \multicolumn{4}{p{0.6\linewidth}}{\cellcolor[rgb]{1,0.85,0.72}{%
	\makecell[{{p{\linewidth}}}]{% 
	  \tt {\tiny [GM$\langle$B]}  
	 STRATEGISCHE ÜBERLEGUNGEN: {'Ich priorisiere Gegenstände mit dem höchsten Nutzen pro Aufwand (Wichtigkeit/Aufwand), um meinen Gesamtwert zu maximieren. C04 (668/369 ≈ 1.81) und A89 (583/720 ≈ 0.81) sind besonders effizient, während B85 (215/4 = 53.75) den Aufwand minimiert. Der Gesamtaufwand (1093) ist unter dem Limit, was Raum für Kompromisse oder Ergänzungen lässt. Dieser Vorschlag ist flexibel, um auf die Prioritäten des anderen Spielers zu reagieren.'}  \\ \tt ARGUMENT: {'C04 (668/369) und A89 (583/720) sind meine Top-Prioritäten, da sie den höchsten Nutzen pro Aufwandseinheit liefern. B85 (215/4) wird hinzugefügt, um den Aufwand zu optimieren und Platz für weitere Gegenstände zu schaffen. Dieser Vorschlag ist strategisch, da er unter dem Maximalaufwand bleibt und flexibel auf den Gegenstandswert des anderen Spielers reagieren kann.'}  \\ \tt VORSCHLAG: {'C04', 'A89', 'B85'} 
	  } 
	   } 
	   } 
	  \\ 
 

    \theutterance \stepcounter{utterance}  

    & & \multicolumn{4}{p{0.6\linewidth}}{\cellcolor[rgb]{0.9,0.9,0.9}{%
	\makecell[{{p{\linewidth}}}]{% 
	  \tt {\tiny [A$\langle$GM]}  
	 ARGUMENT: {'C04 (668/369) und A89 (583/720) sind meine Top-Prioritäten, da sie den höchsten Nutzen pro Aufwandseinheit liefern. B85 (215/4) wird hinzugefügt, um den Aufwand zu optimieren und Platz für weitere Gegenstände zu schaffen. Dieser Vorschlag ist strategisch, da er unter dem Maximalaufwand bleibt und flexibel auf den Gegenstandswert des anderen Spielers reagieren kann.'}  \\ \tt VORSCHLAG: {'C04', 'A89', 'B85'} 
	  } 
	   } 
	   } 
	 & \\ 
 

    \theutterance \stepcounter{utterance}  

    & \multicolumn{4}{p{0.6\linewidth}}{\cellcolor[rgb]{0.8,1,0.9}{%
	\makecell[{{p{\linewidth}}}]{% 
	  \tt {\tiny [A$\rangle$GM]}  
	 STRATEGISCHE ÜBERLEGUNGEN: {'Ich behalte den Vorschlag {'C04', 'A89', 'B85'} bei, da er meine Prioritäten (hochwertige Gegenstände mit effizientem Aufwand) erfüllt und den Maximalaufwand (3386) nicht überschreitet. C04 und A89 sind entscheidend für meinen Nutzen, während B85 den Aufwand minimiert. Ich erwarte, dass der andere Spieler entweder zustimmt oder Gegenstände mit hohem Aufwand anbietet, die ich eventuell gegenkompensieren kann.'}\\ \tt \\ \tt ARGUMENT: {'C04 (668/369) und A89 (583/720) sind meine Top-Prioritäten, da sie den höchsten Nutzen pro Aufwandseinheit liefern. B85 (215/4) wird hinzugefügt, um den Aufwand zu optimieren und Platz für weitere Gegenstände zu schaffen. Dieser Vorschlag ist strategisch, da er unter dem Maximalaufwand bleibt und flexibel auf den Gegenstandswert des anderen Spielers reagieren kann.'} 
	  } 
	   } 
	   } 
	 & & \\ 
 

    \theutterance \stepcounter{utterance}  

    & & \multicolumn{4}{p{0.6\linewidth}}{\cellcolor[rgb]{0.9,0.9,0.9}{%
	\makecell[{{p{\linewidth}}}]{% 
	  \tt {\tiny [GM$\rangle$B]}  
	 ARGUMENT: {'C04 (668/369) und A89 (583/720) sind meine Top-Prioritäten, da sie den höchsten Nutzen pro Aufwandseinheit liefern. B85 (215/4) wird hinzugefügt, um den Aufwand zu optimieren und Platz für weitere Gegenstände zu schaffen. Dieser Vorschlag ist strategisch, da er unter dem Maximalaufwand bleibt und flexibel auf den Gegenstandswert des anderen Spielers reagieren kann.'} 
	  } 
	   } 
	   } 
	 & \\ 
 

    \theutterance \stepcounter{utterance}  

    & & & \multicolumn{4}{p{0.6\linewidth}}{\cellcolor[rgb]{1,0.85,0.72}{%
	\makecell[{{p{\linewidth}}}]{% 
	  \tt {\tiny [GM$\langle$B]}  
	 STRATEGISCHE ÜBERLEGUNGEN: {'Ich priorisiere Gegenstände mit dem höchsten Nutzen pro Aufwand (Wichtigkeit/Aufwand), um meinen Gesamtwert zu maximieren. C04 (668/369 ≈ 1.81) und A89 (583/720 ≈ 0.81) sind besonders effizient, während B85 (215/4 = 53.75) den Aufwand minimiert. Der Gesamtaufwand (1093) ist unter dem Limit, was Raum für Kompromisse oder Ergänzungen lässt. Dieser Vorschlag ist flexibel, um auf die Prioritäten des anderen Spielers zu reagieren.'}  \\ \tt ARGUMENT: {'C04 (668/369) und A89 (583/720) sind meine Top-Prioritäten, da sie den höchsten Nutzen pro Aufwandseinheit liefern. B85 (215/4) wird hinzugefügt, um den Aufwand zu optimieren und Platz für weitere Gegenstände zu schaffen. Dieser Vorschlag ist strategisch, da er unter dem Maximalaufwand bleibt und flexibel auf den Gegenstandswert des anderen Spielers reagieren kann.'}  \\ \tt VORSCHLAG: {'C04', 'A89', 'B85'} 
	  } 
	   } 
	   } 
	  \\ 
 

    \theutterance \stepcounter{utterance}  

    & & \multicolumn{4}{p{0.6\linewidth}}{\cellcolor[rgb]{0.9,0.9,0.9}{%
	\makecell[{{p{\linewidth}}}]{% 
	  \tt {\tiny [A$\langle$GM]}  
	 ARGUMENT: {'C04 (668/369) und A89 (583/720) sind meine Top-Prioritäten, da sie den höchsten Nutzen pro Aufwandseinheit liefern. B85 (215/4) wird hinzugefügt, um den Aufwand zu optimieren und Platz für weitere Gegenstände zu schaffen. Dieser Vorschlag ist strategisch, da er unter dem Maximalaufwand bleibt und flexibel auf den Gegenstandswert des anderen Spielers reagieren kann.'}  \\ \tt VORSCHLAG: {'C04', 'A89', 'B85'} 
	  } 
	   } 
	   } 
	 & \\ 
 

    \theutterance \stepcounter{utterance}  

    & \multicolumn{4}{p{0.6\linewidth}}{\cellcolor[rgb]{0.8,1,0.9}{%
	\makecell[{{p{\linewidth}}}]{% 
	  \tt {\tiny [A$\rangle$GM]}  
	 STRATEGISCHE ÜBERLEGUNGEN: {'Ich behalte den Vorschlag {'C04', 'A89', 'B85'} bei, da er meine Prioritäten (hochwertige Gegenstände mit effizientem Aufwand) erfüllt und den Maximalaufwand (3386) nicht überschreitet. C04 und A89 sind entscheidend für meinen Nutzen, während B85 den Aufwand minimiert. Ich erwarte, dass der andere Spieler entweder zustimmt oder Gegenstände mit hohem Aufwand anbietet, die ich eventuell gegenkompensieren kann.'}\\ \tt \\ \tt ARGUMENT: {'C04 (668/369) und A89 (583/720) sind meine Top-Prioritäten, da sie den höchsten Nutzen pro Aufwandseinheit liefern. B85 (215/4) wird hinzugefügt, um den Aufwand zu optimieren und Platz für weitere Gegenstände zu schaffen. Dieser Vorschlag ist strategisch, da er unter dem Maximalaufwand bleibt und flexibel auf den Gegenstandswert des anderen Spielers reagieren kann.'} 
	  } 
	   } 
	   } 
	 & & \\ 
 

    \theutterance \stepcounter{utterance}  

    & & \multicolumn{4}{p{0.6\linewidth}}{\cellcolor[rgb]{0.9,0.9,0.9}{%
	\makecell[{{p{\linewidth}}}]{% 
	  \tt {\tiny [GM$\rangle$B]}  
	 ARGUMENT: {'C04 (668/369) und A89 (583/720) sind meine Top-Prioritäten, da sie den höchsten Nutzen pro Aufwandseinheit liefern. B85 (215/4) wird hinzugefügt, um den Aufwand zu optimieren und Platz für weitere Gegenstände zu schaffen. Dieser Vorschlag ist strategisch, da er unter dem Maximalaufwand bleibt und flexibel auf den Gegenstandswert des anderen Spielers reagieren kann.'} 
	  } 
	   } 
	   } 
	 & \\ 
 

    \theutterance \stepcounter{utterance}  

    & & & \multicolumn{4}{p{0.6\linewidth}}{\cellcolor[rgb]{1,0.85,0.72}{%
	\makecell[{{p{\linewidth}}}]{% 
	  \tt {\tiny [GM$\langle$B]}  
	 STRATEGISCHE ÜBERLEGUNGEN: {'Ich priorisiere Gegenstände mit dem höchsten Nutzen pro Aufwand (Wichtigkeit/Aufwand), um meinen Gesamtwert zu maximieren. C04 (668/369 ≈ 1.81) und A89 (583/720 ≈ 0.81) sind besonders effizient, während B85 (215/4 = 53.75) den Aufwand minimiert. Der Gesamtaufwand (1093) ist unter dem Limit, was Raum für Kompromisse oder Ergänzungen lässt. Dieser Vorschlag ist flexibel, um auf die Prioritäten des anderen Spielers zu reagieren.'}  \\ \tt ARGUMENT: {'C04 (668/369) und A89 (583/720) sind meine Top-Prioritäten, da sie den höchsten Nutzen pro Aufwandseinheit liefern. B85 (215/4) wird hinzugefügt, um den Aufwand zu optimieren und Platz für weitere Gegenstände zu schaffen. Dieser Vorschlag ist strategisch, da er unter dem Maximalaufwand bleibt und flexibel auf den Gegenstandswert des anderen Spielers reagieren kann.'}  \\ \tt VORSCHLAG: {'C04', 'A89', 'B85'} 
	  } 
	   } 
	   } 
	  \\ 
 

    \theutterance \stepcounter{utterance}  

    & & \multicolumn{4}{p{0.6\linewidth}}{\cellcolor[rgb]{0.9,0.9,0.9}{%
	\makecell[{{p{\linewidth}}}]{% 
	  \tt {\tiny [A$\langle$GM]}  
	 ARGUMENT: {'C04 (668/369) und A89 (583/720) sind meine Top-Prioritäten, da sie den höchsten Nutzen pro Aufwandseinheit liefern. B85 (215/4) wird hinzugefügt, um den Aufwand zu optimieren und Platz für weitere Gegenstände zu schaffen. Dieser Vorschlag ist strategisch, da er unter dem Maximalaufwand bleibt und flexibel auf den Gegenstandswert des anderen Spielers reagieren kann.'}  \\ \tt VORSCHLAG: {'C04', 'A89', 'B85'} 
	  } 
	   } 
	   } 
	 & \\ 
 

    \theutterance \stepcounter{utterance}  

    & \multicolumn{4}{p{0.6\linewidth}}{\cellcolor[rgb]{0.8,1,0.9}{%
	\makecell[{{p{\linewidth}}}]{% 
	  \tt {\tiny [A$\rangle$GM]}  
	 STRATEGISCHE ÜBERLEGUNGEN: {'Ich behalte den Vorschlag {'C04', 'A89', 'B85'} bei, da er meine Prioritäten (hochwertige Gegenstände mit effizientem Aufwand) erfüllt und den Maximalaufwand (3386) nicht überschreitet. C04 und A89 sind entscheidend für meinen Nutzen, während B85 den Aufwand minimiert. Ich erwarte, dass der andere Spieler entweder zustimmt oder Gegenstände mit hohem Aufwand anbietet, die ich eventuell gegenkompensieren kann.'}\\ \tt \\ \tt ARGUMENT: {'C04 (668/369) und A89 (583/720) sind meine Top-Prioritäten, da sie den höchsten Nutzen pro Aufwandseinheit liefern. B85 (215/4) wird hinzugefügt, um den Aufwand zu optimieren und Platz für weitere Gegenstände zu schaffen. Dieser Vorschlag ist strategisch, da er unter dem Maximalaufwand bleibt und flexibel auf den Gegenstandswert des anderen Spielers reagieren kann.'} 
	  } 
	   } 
	   } 
	 & & \\ 
 

    \theutterance \stepcounter{utterance}  

    & & \multicolumn{4}{p{0.6\linewidth}}{\cellcolor[rgb]{0.9,0.9,0.9}{%
	\makecell[{{p{\linewidth}}}]{% 
	  \tt {\tiny [GM$\rangle$B]}  
	 ARGUMENT: {'C04 (668/369) und A89 (583/720) sind meine Top-Prioritäten, da sie den höchsten Nutzen pro Aufwandseinheit liefern. B85 (215/4) wird hinzugefügt, um den Aufwand zu optimieren und Platz für weitere Gegenstände zu schaffen. Dieser Vorschlag ist strategisch, da er unter dem Maximalaufwand bleibt und flexibel auf den Gegenstandswert des anderen Spielers reagieren kann.'} 
	  } 
	   } 
	   } 
	 & \\ 
 

    \theutterance \stepcounter{utterance}  

    & & & \multicolumn{4}{p{0.6\linewidth}}{\cellcolor[rgb]{1,0.85,0.72}{%
	\makecell[{{p{\linewidth}}}]{% 
	  \tt {\tiny [GM$\langle$B]}  
	 STRATEGISCHE ÜBERLEGUNGEN: {'Ich priorisiere Gegenstände mit dem höchsten Nutzen pro Aufwand (Wichtigkeit/Aufwand), um meinen Gesamtwert zu maximieren. C04 (668/369 ≈ 1.81) und A89 (583/720 ≈ 0.81) sind besonders effizient, während B85 (215/4 = 53.75) den Aufwand minimiert. Der Gesamtaufwand (1093) ist unter dem Limit, was Raum für Kompromisse oder Ergänzungen lässt. Dieser Vorschlag ist flexibel, um auf die Prioritäten des anderen Spielers zu reagieren.'}  \\ \tt ARGUMENT: {'C04 (668/369) und A89 (583/720) sind meine Top-Prioritäten, da sie den höchsten Nutzen pro Aufwandseinheit liefern. B85 (215/4) wird hinzugefügt, um den Aufwand zu optimieren und Platz für weitere Gegenstände zu schaffen. Dieser Vorschlag ist strategisch, da er unter dem Maximalaufwand bleibt und flexibel auf den Gegenstandswert des anderen Spielers reagieren kann.'}  \\ \tt VORSCHLAG: {'C04', 'A89', 'B85'} 
	  } 
	   } 
	   } 
	  \\ 
 

    \theutterance \stepcounter{utterance}  

    & & \multicolumn{4}{p{0.6\linewidth}}{\cellcolor[rgb]{0.9,0.9,0.9}{%
	\makecell[{{p{\linewidth}}}]{% 
	  \tt {\tiny [A$\langle$GM]}  
	 ARGUMENT: {'C04 (668/369) und A89 (583/720) sind meine Top-Prioritäten, da sie den höchsten Nutzen pro Aufwandseinheit liefern. B85 (215/4) wird hinzugefügt, um den Aufwand zu optimieren und Platz für weitere Gegenstände zu schaffen. Dieser Vorschlag ist strategisch, da er unter dem Maximalaufwand bleibt und flexibel auf den Gegenstandswert des anderen Spielers reagieren kann.'}  \\ \tt VORSCHLAG: {'C04', 'A89', 'B85'} 
	  } 
	   } 
	   } 
	 & \\ 
 

    \theutterance \stepcounter{utterance}  

    & \multicolumn{4}{p{0.6\linewidth}}{\cellcolor[rgb]{0.8,1,0.9}{%
	\makecell[{{p{\linewidth}}}]{% 
	  \tt {\tiny [A$\rangle$GM]}  
	 STRATEGISCHE ÜBERLEGUNGEN: {'Ich behalte den Vorschlag {'C04', 'A89', 'B85'} bei, da er meine Prioritäten (hochwertige Gegenstände mit effizientem Aufwand) erfüllt und den Maximalaufwand (3386) nicht überschreitet. C04 und A89 sind entscheidend für meinen Nutzen, während B85 den Aufwand minimiert. Ich erwarte, dass der andere Spieler entweder zustimmt oder Gegenstände mit hohem Aufwand anbietet, die ich eventuell gegenkompensieren kann.'}\\ \tt \\ \tt ARGUMENT: {'C04 (668/369) und A89 (583/720) sind meine Top-Prioritäten, da sie den höchsten Nutzen pro Aufwandseinheit liefern. B85 (215/4) wird hinzugefügt, um den Aufwand zu optimieren und Platz für weitere Gegenstände zu schaffen. Dieser Vorschlag ist strategisch, da er unter dem Maximalaufwand bleibt und flexibel auf den Gegenstandswert des anderen Spielers reagieren kann.'} 
	  } 
	   } 
	   } 
	 & & \\ 
 

    \theutterance \stepcounter{utterance}  

    & & \multicolumn{4}{p{0.6\linewidth}}{\cellcolor[rgb]{0.9,0.9,0.9}{%
	\makecell[{{p{\linewidth}}}]{% 
	  \tt {\tiny [GM$\rangle$B]}  
	 ARGUMENT: {'C04 (668/369) und A89 (583/720) sind meine Top-Prioritäten, da sie den höchsten Nutzen pro Aufwandseinheit liefern. B85 (215/4) wird hinzugefügt, um den Aufwand zu optimieren und Platz für weitere Gegenstände zu schaffen. Dieser Vorschlag ist strategisch, da er unter dem Maximalaufwand bleibt und flexibel auf den Gegenstandswert des anderen Spielers reagieren kann.'} 
	  } 
	   } 
	   } 
	 & \\ 
 

    \theutterance \stepcounter{utterance}  

    & & & \multicolumn{4}{p{0.6\linewidth}}{\cellcolor[rgb]{1,0.85,0.72}{%
	\makecell[{{p{\linewidth}}}]{% 
	  \tt {\tiny [GM$\langle$B]}  
	 STRATEGISCHE ÜBERLEGUNGEN: {'Ich priorisiere Gegenstände mit dem höchsten Nutzen pro Aufwand (Wichtigkeit/Aufwand), um meinen Gesamtwert zu maximieren. C04 (668/369 ≈ 1.81) und A89 (583/720 ≈ 0.81) sind besonders effizient, während B85 (215/4 = 53.75) den Aufwand minimiert. Der Gesamtaufwand (1093) ist unter dem Limit, was Raum für Kompromisse oder Ergänzungen lässt. Dieser Vorschlag ist flexibel, um auf die Prioritäten des anderen Spielers zu reagieren.'}  \\ \tt ARGUMENT: {'C04 (668/369) und A89 (583/720) sind meine Top-Prioritäten, da sie den höchsten Nutzen pro Aufwandseinheit liefern. B85 (215/4) wird hinzugefügt, um den Aufwand zu optimieren und Platz für weitere Gegenstände zu schaffen. Dieser Vorschlag ist strategisch, da er unter dem Maximalaufwand bleibt und flexibel auf den Gegenstandswert des anderen Spielers reagieren kann.'}  \\ \tt VORSCHLAG: {'C04', 'A89', 'B85'} 
	  } 
	   } 
	   } 
	  \\ 
 

    \theutterance \stepcounter{utterance}  

    & & \multicolumn{4}{p{0.6\linewidth}}{\cellcolor[rgb]{0.9,0.9,0.9}{%
	\makecell[{{p{\linewidth}}}]{% 
	  \tt {\tiny [A$\langle$GM]}  
	 ARGUMENT: {'C04 (668/369) und A89 (583/720) sind meine Top-Prioritäten, da sie den höchsten Nutzen pro Aufwandseinheit liefern. B85 (215/4) wird hinzugefügt, um den Aufwand zu optimieren und Platz für weitere Gegenstände zu schaffen. Dieser Vorschlag ist strategisch, da er unter dem Maximalaufwand bleibt und flexibel auf den Gegenstandswert des anderen Spielers reagieren kann.'}  \\ \tt VORSCHLAG: {'C04', 'A89', 'B85'} 
	  } 
	   } 
	   } 
	 & \\ 
 

    \theutterance \stepcounter{utterance}  

    & \multicolumn{4}{p{0.6\linewidth}}{\cellcolor[rgb]{0.8,1,0.9}{%
	\makecell[{{p{\linewidth}}}]{% 
	  \tt {\tiny [A$\rangle$GM]}  
	 STRATEGISCHE ÜBERLEGUNGEN: {'Ich behalte den Vorschlag {'C04', 'A89', 'B85'} bei, da er meine Prioritäten (hochwertige Gegenstände mit effizientem Aufwand) erfüllt und den Maximalaufwand (3386) nicht überschreitet. C04 und A89 sind entscheidend für meinen Nutzen, während B85 den Aufwand minimiert. Ich erwarte, dass der andere Spieler entweder zustimmt oder Gegenstände mit hohem Aufwand anbietet, die ich eventuell gegenkompensieren kann.'}\\ \tt \\ \tt ARGUMENT: {'C04 (668/369) und A89 (583/720) sind meine Top-Prioritäten, da sie den höchsten Nutzen pro Aufwandseinheit liefern. B85 (215/4) wird hinzugefügt, um den Aufwand zu optimieren und Platz für weitere Gegenstände zu schaffen. Dieser Vorschlag ist strategisch, da er unter dem Maximalaufwand bleibt und flexibel auf den Gegenstandswert des anderen Spielers reagieren kann.'} 
	  } 
	   } 
	   } 
	 & & \\ 
 

    \theutterance \stepcounter{utterance}  

    & & \multicolumn{4}{p{0.6\linewidth}}{\cellcolor[rgb]{0.9,0.9,0.9}{%
	\makecell[{{p{\linewidth}}}]{% 
	  \tt {\tiny [GM$\rangle$B]}  
	 ARGUMENT: {'C04 (668/369) und A89 (583/720) sind meine Top-Prioritäten, da sie den höchsten Nutzen pro Aufwandseinheit liefern. B85 (215/4) wird hinzugefügt, um den Aufwand zu optimieren und Platz für weitere Gegenstände zu schaffen. Dieser Vorschlag ist strategisch, da er unter dem Maximalaufwand bleibt und flexibel auf den Gegenstandswert des anderen Spielers reagieren kann.'} 
	  } 
	   } 
	   } 
	 & \\ 
 

    \theutterance \stepcounter{utterance}  

    & & & \multicolumn{4}{p{0.6\linewidth}}{\cellcolor[rgb]{1,0.85,0.72}{%
	\makecell[{{p{\linewidth}}}]{% 
	  \tt {\tiny [GM$\langle$B]}  
	 STRATEGISCHE ÜBERLEGUNGEN: {'Ich priorisiere Gegenstände mit dem höchsten Nutzen pro Aufwand (Wichtigkeit/Aufwand), um meinen Gesamtwert zu maximieren. C04 (668/369 ≈ 1.81) und A89 (583/720 ≈ 0.81) sind besonders effizient, während B85 (215/4 = 53.75) den Aufwand minimiert. Der Gesamtaufwand (1093) ist unter dem Limit, was Raum für Kompromisse oder Ergänzungen lässt. Dieser Vorschlag ist flexibel, um auf die Prioritäten des anderen Spielers zu reagieren.'}  \\ \tt ARGUMENT: {'C04 (668/369) und A89 (583/720) sind meine Top-Prioritäten, da sie den höchsten Nutzen pro Aufwandseinheit liefern. B85 (215/4) wird hinzugefügt, um den Aufwand zu optimieren und Platz für weitere Gegenstände zu schaffen. Dieser Vorschlag ist strategisch, da er unter dem Maximalaufwand bleibt und flexibel auf den Gegenstandswert des anderen Spielers reagieren kann.'}  \\ \tt VORSCHLAG: {'C04', 'A89', 'B85'} 
	  } 
	   } 
	   } 
	  \\ 
 

    \theutterance \stepcounter{utterance}  

    & & \multicolumn{4}{p{0.6\linewidth}}{\cellcolor[rgb]{0.9,0.9,0.9}{%
	\makecell[{{p{\linewidth}}}]{% 
	  \tt {\tiny [A$\langle$GM]}  
	 ARGUMENT: {'C04 (668/369) und A89 (583/720) sind meine Top-Prioritäten, da sie den höchsten Nutzen pro Aufwandseinheit liefern. B85 (215/4) wird hinzugefügt, um den Aufwand zu optimieren und Platz für weitere Gegenstände zu schaffen. Dieser Vorschlag ist strategisch, da er unter dem Maximalaufwand bleibt und flexibel auf den Gegenstandswert des anderen Spielers reagieren kann.'}  \\ \tt VORSCHLAG: {'C04', 'A89', 'B85'} 
	  } 
	   } 
	   } 
	 & \\ 
 

    \theutterance \stepcounter{utterance}  

    & \multicolumn{4}{p{0.6\linewidth}}{\cellcolor[rgb]{0.8,1,0.9}{%
	\makecell[{{p{\linewidth}}}]{% 
	  \tt {\tiny [A$\rangle$GM]}  
	 STRATEGISCHE ÜBERLEGUNGEN: {'Ich behalte den Vorschlag {'C04', 'A89', 'B85'} bei, da er meine Prioritäten (hochwertige Gegenstände mit effizientem Aufwand) erfüllt und den Maximalaufwand (3386) nicht überschreitet. C04 und A89 sind entscheidend für meinen Nutzen, während B85 den Aufwand minimiert. Ich erwarte, dass der andere Spieler entweder zustimmt oder Gegenstände mit hohem Aufwand anbietet, die ich eventuell gegenkompensieren kann.'}\\ \tt \\ \tt ARGUMENT: {'C04 (668/369) und A89 (583/720) sind meine Top-Prioritäten, da sie den höchsten Nutzen pro Aufwandseinheit liefern. B85 (215/4) wird hinzugefügt, um den Aufwand zu optimieren und Platz für weitere Gegenstände zu schaffen. Dieser Vorschlag ist strategisch, da er unter dem Maximalaufwand bleibt und flexibel auf den Gegenstandswert des anderen Spielers reagieren kann.'} 
	  } 
	   } 
	   } 
	 & & \\ 
 

    \theutterance \stepcounter{utterance}  

    & & \multicolumn{4}{p{0.6\linewidth}}{\cellcolor[rgb]{0.9,0.9,0.9}{%
	\makecell[{{p{\linewidth}}}]{% 
	  \tt {\tiny [GM$\rangle$B]}  
	 ARGUMENT: {'C04 (668/369) und A89 (583/720) sind meine Top-Prioritäten, da sie den höchsten Nutzen pro Aufwandseinheit liefern. B85 (215/4) wird hinzugefügt, um den Aufwand zu optimieren und Platz für weitere Gegenstände zu schaffen. Dieser Vorschlag ist strategisch, da er unter dem Maximalaufwand bleibt und flexibel auf den Gegenstandswert des anderen Spielers reagieren kann.'} 
	  } 
	   } 
	   } 
	 & \\ 
 

    \theutterance \stepcounter{utterance}  

    & & & \multicolumn{4}{p{0.6\linewidth}}{\cellcolor[rgb]{1,0.85,0.72}{%
	\makecell[{{p{\linewidth}}}]{% 
	  \tt {\tiny [GM$\langle$B]}  
	 STRATEGISCHE ÜBERLEGUNGEN: {'Ich priorisiere Gegenstände mit dem höchsten Nutzen pro Aufwand (Wichtigkeit/Aufwand), um meinen Gesamtwert zu maximieren. C04 (668/369 ≈ 1.81) und A89 (583/720 ≈ 0.81) sind besonders effizient, während B85 (215/4 = 53.75) den Aufwand minimiert. Der Gesamtaufwand (1093) ist unter dem Limit, was Raum für Kompromisse oder Ergänzungen lässt. Dieser Vorschlag ist flexibel, um auf die Prioritäten des anderen Spielers zu reagieren.'}  \\ \tt ARGUMENT: {'C04 (668/369) und A89 (58 
	  } 
	   } 
	   } 
	  \\ 
 

    \theutterance \stepcounter{utterance}  

    & & & \multicolumn{2}{p{0.3\linewidth}}{\cellcolor[rgb]{0.95,0.95,0.95}{%
	\makecell[{{p{\linewidth}}}]{% 
	  \tt {\tiny [GM$|$GM]}  
	 abort 
	  } 
	   } 
	   } 
	 & & \\ 
 

    \theutterance \stepcounter{utterance}  

    & & & \multicolumn{2}{p{0.3\linewidth}}{\cellcolor[rgb]{0.95,0.95,0.95}{%
	\makecell[{{p{\linewidth}}}]{% 
	  \tt {\tiny [GM$|$GM]}  
	 end game 
	  } 
	   } 
	   } 
	 & & \\ 
 

\end{supertabular}
}

\end{document}
