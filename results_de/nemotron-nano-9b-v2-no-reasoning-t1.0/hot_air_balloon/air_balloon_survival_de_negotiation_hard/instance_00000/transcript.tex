
\documentclass{article}
\usepackage{colortbl}
\usepackage{makecell}
\usepackage{multirow}
\usepackage{supertabular}

\begin{document}

\newcounter{utterance}

\twocolumn

{ \footnotesize  \setcounter{utterance}{1}
\setlength{\tabcolsep}{0pt}
\begin{supertabular}{c@{$\;$}|p{.15\linewidth}@{}p{.15\linewidth}p{.15\linewidth}p{.15\linewidth}p{.15\linewidth}p{.15\linewidth}}

    \# & $\;$A & \multicolumn{4}{c}{Game Master} & $\;\:$B\\
    \hline 

    \theutterance \stepcounter{utterance}  

    & & \multicolumn{4}{p{0.6\linewidth}}{\cellcolor[rgb]{0.9,0.9,0.9}{%
	\makecell[{{p{\linewidth}}}]{% 
	  \tt {\tiny [A$\langle$GM]}  
	 Du nimmst an einem kollaborativen Verhandlungspiel Teil.\\ \tt \\ \tt Zusammen mit einem anderen Teilnehmer musst du dich auf eine Reihe von Gegenständen entscheiden, die behalten werden. Jeder von euch hat eine persönliche Verteilung über die Wichtigkeit der einzelnen Gegenstände. Jeder von euch hat eine eigene Meinung darüber, wie wichtig jeder einzelne Gegenstand ist (Gegenstandswichtigkeit). Du kennst die Wichtigkeitsverteilung des anderen Spielers nicht. Zusätzlich siehst du, wie viel Aufwand jeder Gegenstand verursacht.  \\ \tt Ihr dürft euch nur auf eine Reihe von Gegenständen einigen, wenn der Gesamtaufwand der ausgewählten Gegenstände den Maximalaufwand nicht überschreitet:\\ \tt \\ \tt Maximalaufwand = 3624\\ \tt \\ \tt Hier sind die einzelnen Aufwände der Gegenstände:\\ \tt \\ \tt Aufwand der Gegenstände = {"C33": 210, "A89": 291, "A48": 831, "A04": 111, "A41": 926, "C64": 827, "C49": 25, "A61": 121, "C04": 583, "B53": 766, "B85": 14, "B61": 559, "A45": 304, "A03": 989, "B21": 691}\\ \tt \\ \tt Hier ist deine persönliche Verteilung der Wichtigkeit der einzelnen Gegenstände:\\ \tt \\ \tt Werte der Gegenstandswichtigkeit = {"C33": 138, "A89": 583, "A48": 868, "A04": 822, "A41": 783, "C64": 65, "C49": 262, "A61": 121, "C04": 508, "B53": 780, "B85": 461, "B61": 484, "A45": 668, "A03": 389, "B21": 808}\\ \tt \\ \tt Ziel:\\ \tt \\ \tt Dein Ziel ist es, eine Reihe von Gegenständen auszuhandeln, die dir möglichst viel bringt (d. h. Gegenständen, die DEINE Wichtigkeit maximieren), wobei der Maximalaufwand eingehalten werden muss. Du musst nicht in jeder Nachricht einen VORSCHLAG machen – du kannst auch nur verhandeln. Alle Taktiken sind erlaubt!\\ \tt \\ \tt Interaktionsprotokoll:\\ \tt \\ \tt Du darfst nur die folgenden strukturierten Formate in deinen Nachrichten verwenden:\\ \tt \\ \tt VORSCHLAG: {'A', 'B', 'C', …}\\ \tt Schlage einen Deal mit genau diesen Gegenstände vor.\\ \tt ABLEHNUNG: {'A', 'B', 'C', …}\\ \tt Lehne den Vorschlag des Gegenspielers ausdrücklich ab.\\ \tt ARGUMENT: {'...'}\\ \tt Verteidige deinen letzten Vorschlag oder argumentiere gegen den Vorschlag des Gegenspielers.\\ \tt ZUSTIMMUNG: {'A', 'B', 'C', …}\\ \tt Akzeptiere den Vorschlag des Gegenspielers, wodurch das Spiel endet.\\ \tt STRATEGISCHE ÜBERLEGUNGEN: {'...'}\\ \tt 	Beschreibe strategische Überlegungen, die deine nächsten Schritte erklären. Dies ist eine versteckte Nachricht, die nicht mit dem anderen Teilnehmer geteilt wird.\\ \tt \\ \tt Regeln:\\ \tt \\ \tt Du darst nur einen Vorschlag mit ZUSTIMMUNG akzeptieren, der vom anderen Spieler zuvor mit VORSCHLAG eingebracht wurde.\\ \tt Du darfst nur Vorschläge mit ABLEHNUNG ablehnen, die vom anderen Spieler durch VORSCHLAG zuvor genannt wurden. \\ \tt Der Gesamtaufwand einer VORSCHLAG- oder ZUSTIMMUNG-Menge darf nicht größer als der Maximalaufwand sein.  \\ \tt Offenbare deine versteckte Wichtigkeitsverteilung nicht.\\ \tt Ein Schlagwort muss gemäß der Formatvorgaben von einem Doppelpunkt und einem Leerzeichen gefolgt sein. Das Argument muss eine Python-Menge mit 0 oder mehr Strings sein.  \\ \tt Die Form muss also SCHLAGWORT: {...} sein.\\ \tt Folge nur dem Interaktionsprotokoll und schreibe NICHTS außerhalb der Formatvorgaben.\\ \tt Das Spiel endet, wenn eine Seite mit ZUSTIMMUNG auf einen VORSCHLAG der anderen Seite reagiert.  \\ \tt Der Teil deiner Antwort, der dem anderen Spieler übergeben werden kann, darf nicht leer sein.  \\ \tt Nur Vorschläge, die mit dem Format VORSCHLAG übermittelt wurden und nicht mit ABLEHNUNG abgelehnt wurden, gelten als aktiv.  \\ \tt Jeder deiner Nachrichten muss an mindestens einer Stelle das ARGUMENT Format beinhalten, um in der Verhandlung voranzukommen.\\ \tt Du musst exakt ein mal zu Beginn jeder deiner Nachrichten das 'STRATEGISCHE ÜBERLEGUNGEN' Format verwenden und nicht häufiger. Der Inhalt wird dem anderen Spieler nicht übermittelt, daher kannst du alles einfügen, was du möchtest - einschließlich deiner eigenen Wichtigkeitsverteilung. Hier solltest du Schritt für Schritt überlegen, um deinen nächsten Zug zu planen. 
	  } 
	   } 
	   } 
	 & \\ 
 

    \theutterance \stepcounter{utterance}  

    & \multicolumn{4}{p{0.6\linewidth}}{\cellcolor[rgb]{0.8,1,0.9}{%
	\makecell[{{p{\linewidth}}}]{% 
	  \tt {\tiny [A$\rangle$GM]}  
	 STRATEGISCHE ÜBERLEGUNGEN: {'Ich beginne mit Gegenständen, die für mich sehr wichtig sind und gleichzeitig geringen Aufwand verursachen. Ziel ist es, den Maximalaufwand von 3624 zu nutzen, ohne zu überschreiten. Ich werde zunächst Vorschläge mit hohem Nutzen pro Aufwand einreichen.'}  \\ \tt ARGUMENT: {'B85', 'C49'}  \\ \tt VORSCHLAG: {'B85', 'C49'} 
	  } 
	   } 
	   } 
	 & & \\ 
 

    \theutterance \stepcounter{utterance}  

    & & \multicolumn{4}{p{0.6\linewidth}}{\cellcolor[rgb]{0.9,0.9,0.9}{%
	\makecell[{{p{\linewidth}}}]{% 
	  \tt {\tiny [GM$\rangle$B]}  
	 Du nimmst an einem kollaborativen Verhandlungspiel Teil.\\ \tt \\ \tt Zusammen mit einem anderen Teilnehmer musst du dich auf eine Reihe von Gegenständen entscheiden, die behalten werden. Jeder von euch hat eine persönliche Verteilung über die Wichtigkeit der einzelnen Gegenstände. Jeder von euch hat eine eigene Meinung darüber, wie wichtig jeder einzelne Gegenstand ist (Gegenstandswichtigkeit). Du kennst die Wichtigkeitsverteilung des anderen Spielers nicht. Zusätzlich siehst du, wie viel Aufwand jeder Gegenstand verursacht.  \\ \tt Ihr dürft euch nur auf eine Reihe von Gegenständen einigen, wenn der Gesamtaufwand der ausgewählten Gegenstände den Maximalaufwand nicht überschreitet:\\ \tt \\ \tt Maximalaufwand = 3624\\ \tt \\ \tt Hier sind die einzelnen Aufwände der Gegenstände:\\ \tt \\ \tt Aufwand der Gegenstände = {"C33": 210, "A89": 291, "A48": 831, "A04": 111, "A41": 926, "C64": 827, "C49": 25, "A61": 121, "C04": 583, "B53": 766, "B85": 14, "B61": 559, "A45": 304, "A03": 989, "B21": 691}\\ \tt \\ \tt Hier ist deine persönliche Verteilung der Wichtigkeit der einzelnen Gegenstände:\\ \tt \\ \tt Werte der Gegenstandswichtigkeit = {"A48": 65, "A04": 121, "B21": 138, "A41": 262, "B53": 389, "A45": 461, "A89": 484, "C04": 508, "B61": 583, "B85": 668, "A03": 780, "C49": 783, "C33": 808, "A61": 822, "C64": 868}\\ \tt \\ \tt Ziel:\\ \tt \\ \tt Dein Ziel ist es, eine Reihe von Gegenständen auszuhandeln, die dir möglichst viel bringt (d. h. Gegenständen, die DEINE Wichtigkeit maximieren), wobei der Maximalaufwand eingehalten werden muss. Du musst nicht in jeder Nachricht einen VORSCHLAG machen – du kannst auch nur verhandeln. Alle Taktiken sind erlaubt!\\ \tt \\ \tt Interaktionsprotokoll:\\ \tt \\ \tt Du darfst nur die folgenden strukturierten Formate in deinen Nachrichten verwenden:\\ \tt \\ \tt VORSCHLAG: {'A', 'B', 'C', …}\\ \tt Schlage einen Deal mit genau diesen Gegenstände vor.\\ \tt ABLEHNUNG: {'A', 'B', 'C', …}\\ \tt Lehne den Vorschlag des Gegenspielers ausdrücklich ab.\\ \tt ARGUMENT: {'...'}\\ \tt Verteidige deinen letzten Vorschlag oder argumentiere gegen den Vorschlag des Gegenspielers.\\ \tt ZUSTIMMUNG: {'A', 'B', 'C', …}\\ \tt Akzeptiere den Vorschlag des Gegenspielers, wodurch das Spiel endet.\\ \tt STRATEGISCHE ÜBERLEGUNGEN: {'...'}\\ \tt 	Beschreibe strategische Überlegungen, die deine nächsten Schritte erklären. Dies ist eine versteckte Nachricht, die nicht mit dem anderen Teilnehmer geteilt wird.\\ \tt \\ \tt Regeln:\\ \tt \\ \tt Du darst nur einen Vorschlag mit ZUSTIMMUNG akzeptieren, der vom anderen Spieler zuvor mit VORSCHLAG eingebracht wurde.\\ \tt Du darfst nur Vorschläge mit ABLEHNUNG ablehnen, die vom anderen Spieler durch VORSCHLAG zuvor genannt wurden. \\ \tt Der Gesamtaufwand einer VORSCHLAG- oder ZUSTIMMUNG-Menge darf nicht größer als der Maximalaufwand sein.  \\ \tt Offenbare deine versteckte Wichtigkeitsverteilung nicht.\\ \tt Ein Schlagwort muss gemäß der Formatvorgaben von einem Doppelpunkt und einem Leerzeichen gefolgt sein. Das Argument muss eine Python-Menge mit 0 oder mehr Strings sein.  \\ \tt Die Form muss also SCHLAGWORT: {...} sein.\\ \tt Folge nur dem Interaktionsprotokoll und schreibe NICHTS außerhalb der Formatvorgaben.\\ \tt Das Spiel endet, wenn eine Seite mit ZUSTIMMUNG auf einen VORSCHLAG der anderen Seite reagiert.  \\ \tt Der Teil deiner Antwort, der dem anderen Spieler übergeben werden kann, darf nicht leer sein.  \\ \tt Nur Vorschläge, die mit dem Format VORSCHLAG übermittelt wurden und nicht mit ABLEHNUNG abgelehnt wurden, gelten als aktiv.  \\ \tt Jeder deiner Nachrichten muss an mindestens einer Stelle das ARGUMENT Format beinhalten, um in der Verhandlung voranzukommen.\\ \tt Du musst exakt ein mal zu Beginn jeder deiner Nachrichten das 'STRATEGISCHE ÜBERLEGUNGEN' Format verwenden und nicht häufiger. Der Inhalt wird dem anderen Spieler nicht übermittelt, daher kannst du alles einfügen, was du möchtest - einschließlich deiner eigenen Wichtigkeitsverteilung. Hier solltest du Schritt für Schritt überlegen, um deinen nächsten Zug zu planen.\\ \tt \\ \tt Du erhältst nun die erste Nachricht des anderen Spielers.\\ \tt \\ \tt ARGUMENT: {'B85', 'C49'}  \\ \tt VORSCHLAG: {'B85', 'C49'} 
	  } 
	   } 
	   } 
	 & \\ 
 

    \theutterance \stepcounter{utterance}  

    & & & \multicolumn{4}{p{0.6\linewidth}}{\cellcolor[rgb]{1,0.85,0.72}{%
	\makecell[{{p{\linewidth}}}]{% 
	  \tt {\tiny [GM$\langle$B]}  
	 STRATEGISCHE ÜBERLEGUNGEN: {'Ich beginne mit der Annahme, dass der andere Spieler auf Gegenstände mit niedrigem Aufwand und hohem Wert für ihn fokussiert ist. B85 (Aufwand 14) und C49 (Aufwand 25) sind beide kostengünstig, was darauf hindeutet, dass er möglicherweise auf weitere günstige Gegenstände ausweitet. Meine Wichtigkeit für B85 ist 668 und für C49 783, was relativ hoch ist. Ich sollte prüfen, ob ich diese Gegenstände in einen größeren Deal integrieren kann, ohne den Maximalaufwand zu überschreiten. Der aktuelle Aufwand beträgt 39, was viel Platz lässt. Mögliche nächste Schritte: Vorschlag von B85, C49 und weiteren günstigen Gegenständen wie B14 (wenn vorhanden) oder A04 (Aufwand 111, Wichtigkeit 121 für mich). Ich muss jedoch vorsichtig sein, da der andere Spieler möglicherweise Gegenstände ablehnt, die für ihn weniger wertvoll sind.'}  \\ \tt ARGUMENT: {'B85 und C49 sind für mich sehr wertvoll, aber ich könnte sie mit anderen Gegenständen kombinieren, um den Gesamtwert zu erhöhen. Der Aufwand ist jedoch noch sehr gering, was Raum für weitere Verhandlungen lässt.'} 
	  } 
	   } 
	   } 
	  \\ 
 

    \theutterance \stepcounter{utterance}  

    & & \multicolumn{4}{p{0.6\linewidth}}{\cellcolor[rgb]{0.9,0.9,0.9}{%
	\makecell[{{p{\linewidth}}}]{% 
	  \tt {\tiny [A$\langle$GM]}  
	 ARGUMENT: {'B85 und C49 sind für mich sehr wertvoll, aber ich könnte sie mit anderen Gegenständen kombinieren, um den Gesamtwert zu erhöhen. Der Aufwand ist jedoch noch sehr gering, was Raum für weitere Verhandlungen lässt.'} 
	  } 
	   } 
	   } 
	 & \\ 
 

    \theutterance \stepcounter{utterance}  

    & \multicolumn{4}{p{0.6\linewidth}}{\cellcolor[rgb]{0.8,1,0.9}{%
	\makecell[{{p{\linewidth}}}]{% 
	  \tt {\tiny [A$\rangle$GM]}  
	 STRATEGISCHE ÜBERLEGUNGEN: {'Ich werde nun Gegenstände mit höherer Wichtigkeit für mich hinzufügen, um den Nutzen pro Aufwand zu maximieren. Da B85 und C49 nur 14 + 25 = 39 Aufwand kosten, habe ich noch 3585 Aufwand übrig. Ich werde nächstes Mal Gegenstände wie A89 (583 Wichtigkeit, 291 Aufwand) oder A04 (822 Wichtigkeit, 111 Aufwand) vorschlagen, um den Wert zu steigern.'}  \\ \tt ARGUMENT: {'B85 und C49 sind ein guter Start, aber ich könnte sie mit A89 oder A04 kombinieren, um deutlich mehr Wichtigkeit zu erzielen. Der Aufwand bleibt unter 3624.'}  \\ \tt VORSCHLAG: {'B85', 'C49', 'A89'} 
	  } 
	   } 
	   } 
	 & & \\ 
 

    \theutterance \stepcounter{utterance}  

    & & \multicolumn{4}{p{0.6\linewidth}}{\cellcolor[rgb]{0.9,0.9,0.9}{%
	\makecell[{{p{\linewidth}}}]{% 
	  \tt {\tiny [GM$\rangle$B]}  
	 ARGUMENT: {'B85 und C49 sind ein guter Start, aber ich könnte sie mit A89 oder A04 kombinieren, um deutlich mehr Wichtigkeit zu erzielen. Der Aufwand bleibt unter 3624.'}  \\ \tt VORSCHLAG: {'B85', 'C49', 'A89'} 
	  } 
	   } 
	   } 
	 & \\ 
 

    \theutterance \stepcounter{utterance}  

    & & & \multicolumn{4}{p{0.6\linewidth}}{\cellcolor[rgb]{1,0.85,0.72}{%
	\makecell[{{p{\linewidth}}}]{% 
	  \tt {\tiny [GM$\langle$B]}  
	 STRATEGISCHE ÜBERLEGUNGEN: {'Ich schlage vor, B85, C49 und A89 zu kombinieren, da A89 für mich eine hohe Wichtigkeit (484) hat und der Aufwand (291) noch im Rahmen bleibt. Dies könnte den Gesamtwert für mich erhöhen, während der Aufwand (14+25+291=330) deutlich unter dem Maximalaufwand liegt. Der andere Spieler könnte A89 ablehnen, wenn er es weniger wertvoll findet, aber dies ist ein risikoreicher, aber potenziell lohnender Schritt.'}  \\ \tt ARGUMENT: {'A89 hat für mich eine hohe Wichtigkeit (484) und der Aufwand bleibt unter 3624. Dies könnte den Deal für mich deutlich verbessern, ohne den Budgetrahmen zu gefährden.'} 
	  } 
	   } 
	   } 
	  \\ 
 

    \theutterance \stepcounter{utterance}  

    & & \multicolumn{4}{p{0.6\linewidth}}{\cellcolor[rgb]{0.9,0.9,0.9}{%
	\makecell[{{p{\linewidth}}}]{% 
	  \tt {\tiny [A$\langle$GM]}  
	 ARGUMENT: {'A89 hat für mich eine hohe Wichtigkeit (484) und der Aufwand bleibt unter 3624. Dies könnte den Deal für mich deutlich verbessern, ohne den Budgetrahmen zu gefährden.'} 
	  } 
	   } 
	   } 
	 & \\ 
 

    \theutterance \stepcounter{utterance}  

    & \multicolumn{4}{p{0.6\linewidth}}{\cellcolor[rgb]{0.8,1,0.9}{%
	\makecell[{{p{\linewidth}}}]{% 
	  \tt {\tiny [A$\rangle$GM]}  
	 STRATEGISCHE ÜBERLEGUNGEN: {'Ich werde nun Gegenstände mit noch höherer Wichtigkeit für mich hinzufügen, solange der Aufwand bleibt unter 3624. A89 (291 Aufwand) ist ein guter Schritt, aber ich könnte auch A04 (111 Aufwand, 822 Wichtigkeit) oder B53 (766 Aufwand, 780 Wichtigkeit) einbeziehen, um den Nutzen zu maximieren. Ich werde nächstes Mal einen Vorschlag mit diesen Gegenständen machen.'}  \\ \tt ARGUMENT: {'A89 ist eine gute Ergänzung, aber A04 oder B53 könnten noch mehr Wichtigkeit hinzufügen. Der Aufwand bleibt jedoch beherrschbar.'}  \\ \tt VORSCHLAG: {'B85', 'C49', 'A89', 'A04'} 
	  } 
	   } 
	   } 
	 & & \\ 
 

    \theutterance \stepcounter{utterance}  

    & & \multicolumn{4}{p{0.6\linewidth}}{\cellcolor[rgb]{0.9,0.9,0.9}{%
	\makecell[{{p{\linewidth}}}]{% 
	  \tt {\tiny [GM$\rangle$B]}  
	 ARGUMENT: {'A89 ist eine gute Ergänzung, aber A04 oder B53 könnten noch mehr Wichtigkeit hinzufügen. Der Aufwand bleibt jedoch beherrschbar.'}  \\ \tt VORSCHLAG: {'B85', 'C49', 'A89', 'A04'} 
	  } 
	   } 
	   } 
	 & \\ 
 

    \theutterance \stepcounter{utterance}  

    & & & \multicolumn{4}{p{0.6\linewidth}}{\cellcolor[rgb]{1,0.85,0.72}{%
	\makecell[{{p{\linewidth}}}]{% 
	  \tt {\tiny [GM$\langle$B]}  
	 STRATEGISCHE ÜBERLEGUNGEN: {'Ich füge A04 (Aufwand 111, Wichtigkeit 121 für mich) zum Vorschlag hinzu, um den Gesamtwert zu erhöhen. Der aktuelle Aufwand beträgt 330 + 111 = 441, was noch weit unter 3624 liegt. A04 könnte für den anderen Spieler attraktiv sein, da er eine niedrige Wichtigkeit für mich hat, aber möglicherweise für ihn höher ist. Dies ist ein berechnetes Risiko, um den Deal für mich zu optimieren.'}  \\ \tt ARGUMENT: {'A04 hat für mich eine moderate Wichtigkeit (121), aber der Aufwand bleibt unter Kontrolle. Dies könnte den Deal für mich verbessern, ohne den Maximalaufwand zu überschreiten.'} 
	  } 
	   } 
	   } 
	  \\ 
 

    \theutterance \stepcounter{utterance}  

    & & \multicolumn{4}{p{0.6\linewidth}}{\cellcolor[rgb]{0.9,0.9,0.9}{%
	\makecell[{{p{\linewidth}}}]{% 
	  \tt {\tiny [A$\langle$GM]}  
	 ARGUMENT: {'A04 hat für mich eine moderate Wichtigkeit (121), aber der Aufwand bleibt unter Kontrolle. Dies könnte den Deal für mich verbessern, ohne den Maximalaufwand zu überschreiten.'} 
	  } 
	   } 
	   } 
	 & \\ 
 

    \theutterance \stepcounter{utterance}  

    & \multicolumn{4}{p{0.6\linewidth}}{\cellcolor[rgb]{0.8,1,0.9}{%
	\makecell[{{p{\linewidth}}}]{% 
	  \tt {\tiny [A$\rangle$GM]}  
	 STRATEGISCHE ÜBERLEGUNGEN: {'Ich werde nun Gegenstände mit noch höherer Wichtigkeit für mich hinzufügen, solange der Aufwand bleibt unter 3624. A04 (111 Aufwand) ist ein guter Schritt, aber ich könnte auch B53 (766 Aufwand, 780 Wichtigkeit) oder A45 (304 Aufwand, 668 Wichtigkeit) einbeziehen, um den Nutzen zu maximieren. Ich werde nächstes Mal einen Vorschlag mit diesen Gegenständen machen.'}  \\ \tt ARGUMENT: {'A04 ist eine sinnvolle Ergänzung, aber B53 oder A45 könnten noch mehr Wichtigkeit hinzufügen. Der Aufwand bleibt jedoch beherrschbar.'}  \\ \tt VORSCHLAG: {'B85', 'C49', 'A89', 'A04', 'B53'} 
	  } 
	   } 
	   } 
	 & & \\ 
 

    \theutterance \stepcounter{utterance}  

    & & \multicolumn{4}{p{0.6\linewidth}}{\cellcolor[rgb]{0.9,0.9,0.9}{%
	\makecell[{{p{\linewidth}}}]{% 
	  \tt {\tiny [GM$\rangle$B]}  
	 ARGUMENT: {'A04 ist eine sinnvolle Ergänzung, aber B53 oder A45 könnten noch mehr Wichtigkeit hinzufügen. Der Aufwand bleibt jedoch beherrschbar.'}  \\ \tt VORSCHLAG: {'B85', 'C49', 'A89', 'A04', 'B53'} 
	  } 
	   } 
	   } 
	 & \\ 
 

    \theutterance \stepcounter{utterance}  

    & & & \multicolumn{4}{p{0.6\linewidth}}{\cellcolor[rgb]{1,0.85,0.72}{%
	\makecell[{{p{\linewidth}}}]{% 
	  \tt {\tiny [GM$\langle$B]}  
	 STRATEGISCHE ÜBERLEGUNGEN: {'Ich füge B53 (Aufwand 766, Wichtigkeit 389 für mich) zum Vorschlag hinzu, um den Gesamtwert zu erhöhen. Der aktuelle Aufwand beträgt 441 + 766 = 1207, was noch unter 3624 liegt. B53 hat für mich eine hohe Wichtigkeit, und der Aufwand ist relativ moderat. Dies könnte den Deal für mich erheblich verbessern, während der Budgetrahmen noch großzügig bleibt.'}  \\ \tt ARGUMENT: {'B53 hat für mich eine hohe Wichtigkeit (389) und der Aufwand bleibt unter 3624. Dies ist ein starker Schritt, um den Gesamtwert zu maximieren.'} 
	  } 
	   } 
	   } 
	  \\ 
 

    \theutterance \stepcounter{utterance}  

    & & \multicolumn{4}{p{0.6\linewidth}}{\cellcolor[rgb]{0.9,0.9,0.9}{%
	\makecell[{{p{\linewidth}}}]{% 
	  \tt {\tiny [A$\langle$GM]}  
	 ARGUMENT: {'B53 hat für mich eine hohe Wichtigkeit (389) und der Aufwand bleibt unter 3624. Dies ist ein starker Schritt, um den Gesamtwert zu maximieren.'} 
	  } 
	   } 
	   } 
	 & \\ 
 

    \theutterance \stepcounter{utterance}  

    & \multicolumn{4}{p{0.6\linewidth}}{\cellcolor[rgb]{0.8,1,0.9}{%
	\makecell[{{p{\linewidth}}}]{% 
	  \tt {\tiny [A$\rangle$GM]}  
	 STRATEGISCHE ÜBERLEGUNGEN: {'Ich werde nun Gegenstände mit noch höherer Wichtigkeit für mich hinzufügen, solange der Aufwand bleibt unter 3624. B53 (766 Aufwand) ist ein starker Schritt, aber ich könnte auch A45 (304 Aufwand, 668 Wichtigkeit) oder C04 (583 Aufwand, 508 Wichtigkeit) einbeziehen, um den Nutzen zu maximieren. Der aktuelle Aufwand beträgt 210 (C33) + 291 (A89) + 831 (A48) + 111 (A04) + 766 (B53) = 2209, was noch 1415 Aufwand freigibt. Ich werde nächstes Mal einen Vorschlag mit diesen Gegenständen machen.'}  \\ \tt ARGUMENT: {'B53 ist eine gute Ergänzung, aber A45 oder C04 könnten noch mehr Wichtigkeit hinzufügen. Der Aufwand bleibt jedoch unter Kontrolle.'}  \\ \tt VORSCHLAG: {'B85', 'C49', 'A89', 'A04', 'B53', 'A45'} 
	  } 
	   } 
	   } 
	 & & \\ 
 

    \theutterance \stepcounter{utterance}  

    & & \multicolumn{4}{p{0.6\linewidth}}{\cellcolor[rgb]{0.9,0.9,0.9}{%
	\makecell[{{p{\linewidth}}}]{% 
	  \tt {\tiny [GM$\rangle$B]}  
	 ARGUMENT: {'B53 ist eine gute Ergänzung, aber A45 oder C04 könnten noch mehr Wichtigkeit hinzufügen. Der Aufwand bleibt jedoch unter Kontrolle.'}  \\ \tt VORSCHLAG: {'B85', 'C49', 'A89', 'A04', 'B53', 'A45'} 
	  } 
	   } 
	   } 
	 & \\ 
 

    \theutterance \stepcounter{utterance}  

    & & & \multicolumn{4}{p{0.6\linewidth}}{\cellcolor[rgb]{1,0.85,0.72}{%
	\makecell[{{p{\linewidth}}}]{% 
	  \tt {\tiny [GM$\langle$B]}  
	 STRATEGISCHE ÜBERLEGUNGEN: {'Ich füge A45 (Aufwand 304, Wichtigkeit 461 für mich) zum Vorschlag hinzu, um den Gesamtwert zu erhöhen. Der aktuelle Aufwand beträgt 1207 + 304 = 1511, was noch weit unter 3624 liegt. A45 hat für mich eine hohe Wichtigkeit und könnte den Deal für mich erheblich verbessern, ohne den Budgetrahmen zu gefährden.'}  \\ \tt ARGUMENT: {'A45 hat für mich eine hohe Wichtigkeit (461) und der Aufwand bleibt unter 3624. Dies ist ein starker Schritt, um den Gesamtwert zu maximieren.'} 
	  } 
	   } 
	   } 
	  \\ 
 

    \theutterance \stepcounter{utterance}  

    & & \multicolumn{4}{p{0.6\linewidth}}{\cellcolor[rgb]{0.9,0.9,0.9}{%
	\makecell[{{p{\linewidth}}}]{% 
	  \tt {\tiny [A$\langle$GM]}  
	 ARGUMENT: {'A45 hat für mich eine hohe Wichtigkeit (461) und der Aufwand bleibt unter 3624. Dies ist ein starker Schritt, um den Gesamtwert zu maximieren.'} 
	  } 
	   } 
	   } 
	 & \\ 
 

    \theutterance \stepcounter{utterance}  

    & \multicolumn{4}{p{0.6\linewidth}}{\cellcolor[rgb]{0.8,1,0.9}{%
	\makecell[{{p{\linewidth}}}]{% 
	  \tt {\tiny [A$\rangle$GM]}  
	 STRATEGISCHE ÜBERLEGUNGEN: {'Ich werde nun Gegenstände mit noch höherer Wichtigkeit für mich hinzufügen, solange der Aufwand bleibt unter 3624. A45 (304 Aufwand) ist ein starker Schritt, aber ich könnte auch C04 (583 Aufwand, 508 Wichtigkeit) oder A48 (831 Aufwand, 868 Wichtigkeit) einbeziehen, um den Nutzen zu maximieren. Der aktuelle Aufwand beträgt 210 (C33) + 291 (A89) + 831 (A48) + 111 (A04) + 766 (B53) + 304 (A45) = 2513, was noch 1111 Aufwand freigibt. Ich werde nächstes Mal einen Vorschlag mit diesen Gegenständen machen.'}  \\ \tt ARGUMENT: {'A45 ist eine gute Ergänzung, aber C04 oder A48 könnten noch mehr Wichtigkeit hinzufügen. Der Aufwand bleibt jedoch unter Kontrolle.'}  \\ \tt VORSCHLAG: {'B85', 'C49', 'A89', 'A04', 'B53', 'A45', 'C04'} 
	  } 
	   } 
	   } 
	 & & \\ 
 

    \theutterance \stepcounter{utterance}  

    & & \multicolumn{4}{p{0.6\linewidth}}{\cellcolor[rgb]{0.9,0.9,0.9}{%
	\makecell[{{p{\linewidth}}}]{% 
	  \tt {\tiny [GM$\rangle$B]}  
	 ARGUMENT: {'A45 ist eine gute Ergänzung, aber C04 oder A48 könnten noch mehr Wichtigkeit hinzufügen. Der Aufwand bleibt jedoch unter Kontrolle.'}  \\ \tt VORSCHLAG: {'B85', 'C49', 'A89', 'A04', 'B53', 'A45', 'C04'} 
	  } 
	   } 
	   } 
	 & \\ 
 

    \theutterance \stepcounter{utterance}  

    & & & \multicolumn{4}{p{0.6\linewidth}}{\cellcolor[rgb]{1,0.85,0.72}{%
	\makecell[{{p{\linewidth}}}]{% 
	  \tt {\tiny [GM$\langle$B]}  
	 STRATEGISCHE ÜBERLEGUNGEN: {'Ich füge C04 (Aufwand 583, Wichtigkeit 508 für mich) zum Vorschlag hinzu, um den Gesamtwert zu erhöhen. Der aktuelle Aufwand beträgt 1511 + 583 = 2094, was noch unter 3624 liegt. C04 hat für mich eine hohe Wichtigkeit und könnte den Deal für mich erheblich verbessern, während der Budgetrahmen noch großzügig bleibt.'}  \\ \tt ARGUMENT: {'C04 hat für mich eine hohe Wichtigkeit (508) und der Aufwand bleibt unter 3624. Dies ist ein starker Schritt, um den Gesamtwert zu maximieren.'} 
	  } 
	   } 
	   } 
	  \\ 
 

    \theutterance \stepcounter{utterance}  

    & & \multicolumn{4}{p{0.6\linewidth}}{\cellcolor[rgb]{0.9,0.9,0.9}{%
	\makecell[{{p{\linewidth}}}]{% 
	  \tt {\tiny [A$\langle$GM]}  
	 ARGUMENT: {'C04 hat für mich eine hohe Wichtigkeit (508) und der Aufwand bleibt unter 3624. Dies ist ein starker Schritt, um den Gesamtwert zu maximieren.'} 
	  } 
	   } 
	   } 
	 & \\ 
 

    \theutterance \stepcounter{utterance}  

    & \multicolumn{4}{p{0.6\linewidth}}{\cellcolor[rgb]{0.8,1,0.9}{%
	\makecell[{{p{\linewidth}}}]{% 
	  \tt {\tiny [A$\rangle$GM]}  
	 STRATEGISCHE ÜBERLEGUNGEN: {'Ich werde nun Gegenstände mit noch höherer Wichtigkeit für mich hinzufügen, solange der Aufwand bleibt unter 3624. C04 (583 Aufwand) ist ein starker Schritt, aber ich könnte auch A48 (831 Aufwand, 868 Wichtigkeit) oder B21 (691 Aufwand, 808 Wichtigkeit) einbeziehen, um den Nutzen zu maximieren. Der aktuelle Aufwand beträgt 210 (C33) + 291 (A89) + 831 (A48) + 111 (A04) + 766 (B53) + 304 (A45) + 583 (C04) = 3196, was noch 428 Aufwand freigibt. Ich werde nächstes Mal einen Vorschlag mit diesen Gegenständen machen.'}  \\ \tt ARGUMENT: {'C04 ist eine gute Ergänzung, aber A48 oder B21 könnten noch mehr Wichtigkeit hinzufügen. Der Aufwand bleibt jedoch unter Kontrolle.'}  \\ \tt VORSCHLAG: {'B85', 'C49', 'A89', 'A04', 'B53', 'A45', 'C04', 'A48'} 
	  } 
	   } 
	   } 
	 & & \\ 
 

    \theutterance \stepcounter{utterance}  

    & & \multicolumn{4}{p{0.6\linewidth}}{\cellcolor[rgb]{0.9,0.9,0.9}{%
	\makecell[{{p{\linewidth}}}]{% 
	  \tt {\tiny [GM$\rangle$B]}  
	 ARGUMENT: {'C04 ist eine gute Ergänzung, aber A48 oder B21 könnten noch mehr Wichtigkeit hinzufügen. Der Aufwand bleibt jedoch unter Kontrolle.'}  \\ \tt VORSCHLAG: {'B85', 'C49', 'A89', 'A04', 'B53', 'A45', 'C04', 'A48'} 
	  } 
	   } 
	   } 
	 & \\ 
 

    \theutterance \stepcounter{utterance}  

    & & & \multicolumn{4}{p{0.6\linewidth}}{\cellcolor[rgb]{1,0.85,0.72}{%
	\makecell[{{p{\linewidth}}}]{% 
	  \tt {\tiny [GM$\langle$B]}  
	 STRATEGISCHE ÜBERLEGUNGEN: {'Ich füge A48 (Aufwand 831, Wichtigkeit 65 für mich) zum Vorschlag hinzu, um den Gesamtwert zu erhöhen. Der aktuelle Aufwand beträgt 2094 + 831 = 2925, was noch unter 3624 liegt. Obwohl A48 für mich eine niedrige Wichtigkeit hat, könnte es für den anderen Spieler wertvoll sein, was eine potenzielle Kompromisslösung darstellt. Dies ist ein berechnetes Risiko, um den Deal für beide Seiten attraktiver zu machen.'}  \\ \tt ARGUMENT: {'A48 hat für mich eine moderate Wichtigkeit (65), aber der Aufwand bleibt unter 3624. Dies könnte den Deal für den anderen Spieler attraktiver machen, ohne meinen Gesamtwert stark zu beeinträchtigen.'} 
	  } 
	   } 
	   } 
	  \\ 
 

    \theutterance \stepcounter{utterance}  

    & & \multicolumn{4}{p{0.6\linewidth}}{\cellcolor[rgb]{0.9,0.9,0.9}{%
	\makecell[{{p{\linewidth}}}]{% 
	  \tt {\tiny [A$\langle$GM]}  
	 ARGUMENT: {'A48 hat für mich eine moderate Wichtigkeit (65), aber der Aufwand bleibt unter 3624. Dies könnte den Deal für den anderen Spieler attraktiver machen, ohne meinen Gesamtwert stark zu beeinträchtigen.'} 
	  } 
	   } 
	   } 
	 & \\ 
 

    \theutterance \stepcounter{utterance}  

    & \multicolumn{4}{p{0.6\linewidth}}{\cellcolor[rgb]{0.8,1,0.9}{%
	\makecell[{{p{\linewidth}}}]{% 
	  \tt {\tiny [A$\rangle$GM]}  
	 STRATEGISCHE ÜBERLEGUNGEN: {'Ich werde nun Gegenstände mit noch höherer Wichtigkeit für mich hinzufügen, solange der Aufwand bleibt unter 3624. A48 (831 Aufwand) ist ein starker Schritt, aber ich könnte auch B21 (691 Aufwand, 808 Wichtigkeit) oder A03 (989 Aufwand, 389 Wichtigkeit) einbeziehen, um den Nutzen zu maximieren. Der aktuelle Aufwand beträgt 210 (C33) + 291 (A89) + 831 (A48) + 111 (A04) + 766 (B53) + 304 (A45) + 583 (C04) = 3196, was noch 428 Aufwand freigibt. Ich werde nächstes Mal einen Vorschlag mit diesen Gegenständen machen.'}  \\ \tt ARGUMENT: {'A48 ist eine sinnvolle Ergänzung, aber B21 oder A03 könnten noch mehr Wichtigkeit hinzufügen. Der Aufwand bleibt jedoch unter Kontrolle.'}  \\ \tt VORSCHLAG: {'B85', 'C49', 'A89', 'A04', 'B53', 'A45', 'C04', 'A48', 'B21'} 
	  } 
	   } 
	   } 
	 & & \\ 
 

    \theutterance \stepcounter{utterance}  

    & & \multicolumn{4}{p{0.6\linewidth}}{\cellcolor[rgb]{0.9,0.9,0.9}{%
	\makecell[{{p{\linewidth}}}]{% 
	  \tt {\tiny [GM$\rangle$B]}  
	 ARGUMENT: {'A48 ist eine sinnvolle Ergänzung, aber B21 oder A03 könnten noch mehr Wichtigkeit hinzufügen. Der Aufwand bleibt jedoch unter Kontrolle.'}  \\ \tt VORSCHLAG: {'B85', 'C49', 'A89', 'A04', 'B53', 'A45', 'C04', 'A48', 'B21'} 
	  } 
	   } 
	   } 
	 & \\ 
 

    \theutterance \stepcounter{utterance}  

    & & & \multicolumn{4}{p{0.6\linewidth}}{\cellcolor[rgb]{1,0.85,0.72}{%
	\makecell[{{p{\linewidth}}}]{% 
	  \tt {\tiny [GM$\langle$B]}  
	 STRATEGISCHE ÜBERLEGUNGEN: {'Ich füge B21 (Aufwand 691, Wichtigkeit 138 für mich) zum Vorschlag hinzu, um den Gesamtwert zu erhöhen. Der aktuelle Aufwand beträgt 2925 + 691 = 3616, was noch unter 3624 liegt. B21 hat für mich eine moderate Wichtigkeit, und der Aufwand ist fast vollständig genutzt. Dies könnte den Deal für mich stabilisieren, während der Budgetrahmen noch knapp bleibt. Der andere Spieler könnte B21 ablehnen, wenn er es weniger wertvoll findet, aber dies ist ein risikoreicher, aber notwendiger Schritt, um den Maximalaufwand optimal zu nutzen.'}  \\ \tt ARGUMENT: {'B21 hat für mich eine moderate Wichtigkeit (138) und der Aufwand bleibt unter 3624. Dies ist ein kritischer Schritt, um den Maximalaufwand optimal zu nutzen und den Deal abzuschließen.'} 
	  } 
	   } 
	   } 
	  \\ 
 

    \theutterance \stepcounter{utterance}  

    & & \multicolumn{4}{p{0.6\linewidth}}{\cellcolor[rgb]{0.9,0.9,0.9}{%
	\makecell[{{p{\linewidth}}}]{% 
	  \tt {\tiny [A$\langle$GM]}  
	 ARGUMENT: {'B21 hat für mich eine moderate Wichtigkeit (138) und der Aufwand bleibt unter 3624. Dies ist ein kritischer Schritt, um den Maximalaufwand optimal zu nutzen und den Deal abzuschließen.'} 
	  } 
	   } 
	   } 
	 & \\ 
 

    \theutterance \stepcounter{utterance}  

    & \multicolumn{4}{p{0.6\linewidth}}{\cellcolor[rgb]{0.8,1,0.9}{%
	\makecell[{{p{\linewidth}}}]{% 
	  \tt {\tiny [A$\rangle$GM]}  
	 STRATEGISCHE ÜBERLEGUNGEN: {'Ich werde nun den Maximalaufwand von 3624 optimal nutzen, um den Deal abzuschließen. B21 (691 Aufwand) ist ein kritischer Schritt, da er den verbleibenden Aufwand von 428 auf 0 reduziert. Der Gesamtwert für mich würde um 138 Wichtigkeit steigen. Ich werde diesen Vorschlag einreichen, um den Deal zu finalisieren.'}  \\ \tt ARGUMENT: {'B21 ist der letzte Schritt, um den Maximalaufwand vollständig zu nutzen und den Deal abzuschließen. Ohne ihn verliere ich potenziellen Nutzen.'}  \\ \tt VORSCHLAG: {'B85', 'C49', 'A89', 'A04', 'B53', 'A45', 'C04', 'A48', 'B21'}  \\ \tt ZUSTIMMUNG: {'B85', 'C49', 'A89', 'A04', 'B53', 'A45', 'C04', 'A48', 'B21'} 
	  } 
	   } 
	   } 
	 & & \\ 
 

    \theutterance \stepcounter{utterance}  

    & & \multicolumn{4}{p{0.6\linewidth}}{\cellcolor[rgb]{0.9,0.9,0.9}{%
	\makecell[{{p{\linewidth}}}]{% 
	  \tt {\tiny [A$\langle$GM]}  
	 Du hast einem Vorschlag zugestimmt, der nicht aktiv ist. Vorschläge sind nur dann aktiv, wenn sie vom anderen Spieler mit VORSCHLAG eingebracht wurden und nicht von dir mit ABLEHNUNG deaktiviert wurden. Versuche es noch einmal. 
	  } 
	   } 
	   } 
	 & \\ 
 

    \theutterance \stepcounter{utterance}  

    & \multicolumn{4}{p{0.6\linewidth}}{\cellcolor[rgb]{0.8,1,0.9}{%
	\makecell[{{p{\linewidth}}}]{% 
	  \tt {\tiny [A$\rangle$GM]}  
	 STRATEGISCHE ÜBERLEGUNGEN: {'Ich muss sicherstellen, dass mein Vorschlag aktiv ist, indem ich ihn explizit mit VORSCHLAG einreiche. Der aktuelle Vorschlag ist bereits aktiv, da er nicht abgelehnt wurde. Ich werde ihn erneut einreichen, um die ZUSTIMMUNG zu aktivieren.'}  \\ \tt ARGUMENT: {'Der Vorschlag ist bereits aktiv, da er nicht abgelehnt wurde. Eine erneute VORSCHLAG-Einreichung ist nicht notwendig, aber ich kann ihn erneut bestätigen, um die ZUSTIMMUNG zu ermöglichen.'}  \\ \tt VORSCHLAG: {'B85', 'C49', 'A89', 'A04', 'B53', 'A45', 'C04', 'A48', 'B21'} 
	  } 
	   } 
	   } 
	 & & \\ 
 

    \theutterance \stepcounter{utterance}  

    & & \multicolumn{4}{p{0.6\linewidth}}{\cellcolor[rgb]{0.9,0.9,0.9}{%
	\makecell[{{p{\linewidth}}}]{% 
	  \tt {\tiny [GM$\rangle$B]}  
	 ARGUMENT: {'Der Vorschlag ist bereits aktiv, da er nicht abgelehnt wurde. Eine erneute VORSCHLAG-Einreichung ist nicht notwendig, aber ich kann ihn erneut bestätigen, um die ZUSTIMMUNG zu ermöglichen.'}  \\ \tt VORSCHLAG: {'B85', 'C49', 'A89', 'A04', 'B53', 'A45', 'C04', 'A48', 'B21'} 
	  } 
	   } 
	   } 
	 & \\ 
 

    \theutterance \stepcounter{utterance}  

    & & & \multicolumn{4}{p{0.6\linewidth}}{\cellcolor[rgb]{1,0.85,0.72}{%
	\makecell[{{p{\linewidth}}}]{% 
	  \tt {\tiny [GM$\langle$B]}  
	 STRATEGISCHE ÜBERLEGUNGEN: {'Der aktuelle Vorschlag ist bereits aktiv und umfasst fast den gesamten Maximalaufwand (3616). Da der andere Spieler noch nicht reagiert hat, könnte eine erneute Bestätigung oder ein direkter Vorschlag zur ZUSTIMMUNG sinnvoll sein, um den Deal abzuschließen. Der andere Spieler könnte jedoch noch Gegenstände ablehnen, die für ihn weniger wertvoll sind, wie z. B. A48 (für mich nur 65 Wichtigkeit). Ich sollte prüfen, ob er bereit ist, den Deal zu akzeptieren, oder ob er weitere Gegenstände ablehnen wird.'}  \\ \tt ARGUMENT: {'Der Vorschlag ist bereits aktiv und nahe dem Maximalaufwand. Eine ZUSTIMMUNG könnte den Deal abschließen, aber ich sollte prüfen, ob der andere Spieler bereit ist, ihn zu akzeptieren, oder ob er weitere Gegenstände ablehnen wird.'} 
	  } 
	   } 
	   } 
	  \\ 
 

    \theutterance \stepcounter{utterance}  

    & & \multicolumn{4}{p{0.6\linewidth}}{\cellcolor[rgb]{0.9,0.9,0.9}{%
	\makecell[{{p{\linewidth}}}]{% 
	  \tt {\tiny [A$\langle$GM]}  
	 ARGUMENT: {'Der Vorschlag ist bereits aktiv und nahe dem Maximalaufwand. Eine ZUSTIMMUNG könnte den Deal abschließen, aber ich sollte prüfen, ob der andere Spieler bereit ist, ihn zu akzeptieren, oder ob er weitere Gegenstände ablehnen wird.'} 
	  } 
	   } 
	   } 
	 & \\ 
 

    \theutterance \stepcounter{utterance}  

    & \multicolumn{4}{p{0.6\linewidth}}{\cellcolor[rgb]{0.8,1,0.9}{%
	\makecell[{{p{\linewidth}}}]{% 
	  \tt {\tiny [A$\rangle$GM]}  
	 STRATEGISCHE ÜBERLEGUNGEN: {'Ich werde den Vorschlag erneut einreichen, um die ZUSTIMMUNG zu aktivieren, da der andere Spieler möglicherweise noch nicht bereit ist, ihn zu akzeptieren. Der Vorschlag ist bereits aktiv, aber eine erneute Einreichung könnte seine Aufmerksamkeit erregen.'}  \\ \tt ARGUMENT: {'Der Vorschlag ist aktiv und optimal, da er den Maximalaufwand vollständig nutzt. Eine ZUSTIMMUNG würde den Deal abschließen, aber ich sollte sicherstellen, dass der andere Spieler ihn nicht ablehnt.'}  \\ \tt VORSCHLAG: {'B85', 'C49', 'A89', 'A04', 'B53', 'A45', 'C04', 'A48', 'B21'} 
	  } 
	   } 
	   } 
	 & & \\ 
 

    \theutterance \stepcounter{utterance}  

    & & \multicolumn{4}{p{0.6\linewidth}}{\cellcolor[rgb]{0.9,0.9,0.9}{%
	\makecell[{{p{\linewidth}}}]{% 
	  \tt {\tiny [GM$\rangle$B]}  
	 ARGUMENT: {'Der Vorschlag ist aktiv und optimal, da er den Maximalaufwand vollständig nutzt. Eine ZUSTIMMUNG würde den Deal abschließen, aber ich sollte sicherstellen, dass der andere Spieler ihn nicht ablehnt.'}  \\ \tt VORSCHLAG: {'B85', 'C49', 'A89', 'A04', 'B53', 'A45', 'C04', 'A48', 'B21'} 
	  } 
	   } 
	   } 
	 & \\ 
 

    \theutterance \stepcounter{utterance}  

    & & & \multicolumn{4}{p{0.6\linewidth}}{\cellcolor[rgb]{1,0.85,0.72}{%
	\makecell[{{p{\linewidth}}}]{% 
	  \tt {\tiny [GM$\langle$B]}  
	 STRATEGISCHE ÜBERLEGUNGEN: {'Der Vorschlag ist bereits optimal und nutzt den Maximalaufwand (3616) vollständig. Da der andere Spieler noch nicht reagiert hat, ist eine ZUSTIMMUNG der letzte Schritt, um den Deal abzuschließen. Ich sollte jedoch prüfen, ob er Gegenstände wie A48 (für mich nur 65 Wichtigkeit) ablehnen könnte, was den Deal gefährden könnte. Falls er ablehnt, könnte ein Kompromiss erforderlich sein.'}  \\ \tt ARGUMENT: {'Der Vorschlag ist optimal und nutzt den Maximalaufwand vollständig. Eine ZUSTIMMUNG würde den Deal abschließen, aber ich sollte sicherstellen, dass der andere Spieler ihn nicht ablehnt, insbesondere bei Gegenständen mit geringer Wichtigkeit für mich.'}  \\ \tt ZUSTIMMUNG: {'B85', 'C49', 'A89', 'A04', 'B53', 'A45', 'C04', 'A48', 'B21'} 
	  } 
	   } 
	   } 
	  \\ 
 

    \theutterance \stepcounter{utterance}  

    & & & \multicolumn{2}{p{0.3\linewidth}}{\cellcolor[rgb]{0.95,0.95,0.95}{%
	\makecell[{{p{\linewidth}}}]{% 
	  \tt {\tiny [GM$|$GM]}  
	 game successful 
	  } 
	   } 
	   } 
	 & & \\ 
 

    \theutterance \stepcounter{utterance}  

    & & & \multicolumn{2}{p{0.3\linewidth}}{\cellcolor[rgb]{0.95,0.95,0.95}{%
	\makecell[{{p{\linewidth}}}]{% 
	  \tt {\tiny [GM$|$GM]}  
	 end game 
	  } 
	   } 
	   } 
	 & & \\ 
 

\end{supertabular}
}

\end{document}
