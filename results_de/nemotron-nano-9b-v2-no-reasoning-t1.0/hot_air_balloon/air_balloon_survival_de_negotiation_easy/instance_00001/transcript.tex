
\documentclass{article}
\usepackage{colortbl}
\usepackage{makecell}
\usepackage{multirow}
\usepackage{supertabular}

\begin{document}

\newcounter{utterance}

\centering \large Interaction Transcript for game `hot\_air\_balloon', experiment `air\_balloon\_survival\_de\_negotiation\_easy', episode 1 with nemotron{-}nano{-}9b{-}v2{-}no{-}reasoning{-}t1.0.
\vspace{24pt}

{ \footnotesize  \setcounter{utterance}{1}
\setlength{\tabcolsep}{0pt}
\begin{supertabular}{c@{$\;$}|p{.15\linewidth}@{}p{.15\linewidth}p{.15\linewidth}p{.15\linewidth}p{.15\linewidth}p{.15\linewidth}}
    \# & $\;$A & \multicolumn{4}{c}{Game Master} & $\;\:$B\\
    \hline

    \theutterance \stepcounter{utterance}  
    & & \multicolumn{4}{p{0.6\linewidth}}{
        \cellcolor[rgb]{0.9,0.9,0.9}{
            \makecell[{{p{\linewidth}}}]{
                \texttt{\tiny{[P1$\langle$GM]}}
                \texttt{Du nimmst an einem kollaborativen Verhandlungspiel Teil.} \\
\\ 
\texttt{Zusammen mit einem anderen Teilnehmer musst du dich auf eine Reihe von Gegenständen entscheiden, die behalten werden. Jeder von euch hat eine persönliche Verteilung über die Wichtigkeit der einzelnen Gegenstände. Jeder von euch hat eine eigene Meinung darüber, wie wichtig jeder einzelne Gegenstand ist (Gegenstandswichtigkeit). Du kennst die Wichtigkeitsverteilung des anderen Spielers nicht. Zusätzlich siehst du, wie viel Aufwand jeder Gegenstand verursacht.  } \\
\texttt{Ihr dürft euch nur auf eine Reihe von Gegenständen einigen, wenn der Gesamtaufwand der ausgewählten Gegenstände den Maximalaufwand nicht überschreitet:} \\
\\ 
\texttt{Maximalaufwand = 4564} \\
\\ 
\texttt{Hier sind die einzelnen Aufwände der Gegenstände:} \\
\\ 
\texttt{Aufwand der Gegenstände = \{"A07": 299, "A36": 241, "B73": 889, "C45": 967, "B93": 619, "B98": 799, "A87": 978, "A63": 733, "B05": 909, "C13": 501, "B16": 139, "B33": 594, "C34": 565, "C89": 789, "C11": 107\}} \\
\\ 
\texttt{Hier ist deine persönliche Verteilung der Wichtigkeit der einzelnen Gegenstände:} \\
\\ 
\texttt{Werte der Gegenstandswichtigkeit = \{"A07": 138, "A36": 583, "B73": 868, "C45": 822, "B93": 783, "B98": 65, "A87": 262, "A63": 121, "B05": 508, "C13": 780, "B16": 461, "B33": 484, "C34": 668, "C89": 389, "C11": 808\}} \\
\\ 
\texttt{Ziel:} \\
\\ 
\texttt{Dein Ziel ist es, eine Reihe von Gegenständen auszuhandeln, die dir möglichst viel bringt (d. h. Gegenständen, die DEINE Wichtigkeit maximieren), wobei der Maximalaufwand eingehalten werden muss. Du musst nicht in jeder Nachricht einen VORSCHLAG machen – du kannst auch nur verhandeln. Alle Taktiken sind erlaubt!} \\
\\ 
\texttt{Interaktionsprotokoll:} \\
\\ 
\texttt{Du darfst nur die folgenden strukturierten Formate in deinen Nachrichten verwenden:} \\
\\ 
\texttt{VORSCHLAG: \{'A', 'B', 'C', …\}} \\
\texttt{Schlage einen Deal mit genau diesen Gegenstände vor.} \\
\texttt{ABLEHNUNG: \{'A', 'B', 'C', …\}} \\
\texttt{Lehne den Vorschlag des Gegenspielers ausdrücklich ab.} \\
\texttt{ARGUMENT: \{'...'\}} \\
\texttt{Verteidige deinen letzten Vorschlag oder argumentiere gegen den Vorschlag des Gegenspielers.} \\
\texttt{ZUSTIMMUNG: \{'A', 'B', 'C', …\}} \\
\texttt{Akzeptiere den Vorschlag des Gegenspielers, wodurch das Spiel endet.} \\
\texttt{STRATEGISCHE ÜBERLEGUNGEN: \{'...'\}} \\
\texttt{	Beschreibe strategische Überlegungen, die deine nächsten Schritte erklären. Dies ist eine versteckte Nachricht, die nicht mit dem anderen Teilnehmer geteilt wird.} \\
\\ 
\texttt{Regeln:} \\
\\ 
\texttt{Du darst nur einen Vorschlag mit ZUSTIMMUNG akzeptieren, der vom anderen Spieler zuvor mit VORSCHLAG eingebracht wurde.} \\
\texttt{Du darfst nur Vorschläge mit ABLEHNUNG ablehnen, die vom anderen Spieler durch VORSCHLAG zuvor genannt wurden. } \\
\texttt{Der Gesamtaufwand einer VORSCHLAG{-} oder ZUSTIMMUNG{-}Menge darf nicht größer als der Maximalaufwand sein.  } \\
\texttt{Offenbare deine versteckte Wichtigkeitsverteilung nicht.} \\
\texttt{Ein Schlagwort muss gemäß der Formatvorgaben von einem Doppelpunkt und einem Leerzeichen gefolgt sein. Das Argument muss eine Python{-}Menge mit 0 oder mehr Strings sein.  } \\
\texttt{Die Form muss also SCHLAGWORT: \{...\} sein.} \\
\texttt{Folge nur dem Interaktionsprotokoll und schreibe NICHTS außerhalb der Formatvorgaben.} \\
\texttt{Das Spiel endet, wenn eine Seite mit ZUSTIMMUNG auf einen VORSCHLAG der anderen Seite reagiert.  } \\
\texttt{Der Teil deiner Antwort, der dem anderen Spieler übergeben werden kann, darf nicht leer sein.  } \\
\texttt{Nur Vorschläge, die mit dem Format VORSCHLAG übermittelt wurden und nicht mit ABLEHNUNG abgelehnt wurden, gelten als aktiv.  } \\
\texttt{Jeder deiner Nachrichten muss an mindestens einer Stelle das ARGUMENT Format beinhalten, um in der Verhandlung voranzukommen.} \\
\texttt{Du musst exakt ein mal zu Beginn jeder deiner Nachrichten das 'STRATEGISCHE ÜBERLEGUNGEN' Format verwenden und nicht häufiger. Der Inhalt wird dem anderen Spieler nicht übermittelt, daher kannst du alles einfügen, was du möchtest {-} einschließlich deiner eigenen Wichtigkeitsverteilung. Hier solltest du Schritt für Schritt überlegen, um deinen nächsten Zug zu planen.} \\
            }
        }
    }
    & \\ \\

    \theutterance \stepcounter{utterance}  
<<<<<<< HEAD
    & \multicolumn{4}{p{0.6\linewidth}}{
        \cellcolor[rgb]{0.9,0.9,0.9}{
            \makecell[{{p{\linewidth}}}]{
                \texttt{\tiny{[P1$\rangle$GM]}}
                \texttt{VORSCHLAG: \{'B16', 'C11', 'A36', 'C13', 'B93', 'C34', 'B73', 'C45', 'A07'\}} \\
\texttt{STRATEGISCHE ÜBERLEGUNGEN: Ich priorisiere Gegenstände mit hoher Wichtigkeit für mich und niedrigem Aufwand, um den Gesamtwert zu maximieren. Die ausgewählten Gegenstände haben einen Gesamtaufwand von 4327, was unter dem Maximalaufwand von 4564 liegt. Ich hoffe, der andere Spieler wird diesen Vorschlag akzeptieren.} \\
            }
        }
    }
    & & \\ \\

    \theutterance \stepcounter{utterance}  
    & & \multicolumn{4}{p{0.6\linewidth}}{
        \cellcolor[rgb]{0.9,0.9,0.9}{
            \makecell[{{p{\linewidth}}}]{
                \texttt{\tiny{[P1$\langle$GM]}}
                \texttt{Deine Antwort entsprach nicht mit dem korrekten Format für strategische Überlegungen entsprochen. Das erste strukturierte Format muss folgende Form haben: STRATEGISCHE ÜBERLEGUNGEN: \{...\}. Versuche es erneut.} \\
            }
        }
    }
    & \\ \\

    \theutterance \stepcounter{utterance}  
    & \multicolumn{4}{p{0.6\linewidth}}{
        \cellcolor[rgb]{0.9,0.9,0.9}{
            \makecell[{{p{\linewidth}}}]{
                \texttt{\tiny{[P1$\rangle$GM]}}
                \texttt{STRATEGISCHE ÜBERLEGUNGEN: \{'Prioritize high{-}value low{-}cost items', 'Maximize total importance', 'Stay under 4564 budget'\}} \\
            }
        }
    }
    & & \\ \\

    \theutterance \stepcounter{utterance}  
    & & \multicolumn{4}{p{0.6\linewidth}}{
        \cellcolor[rgb]{0.9,0.9,0.9}{
            \makecell[{{p{\linewidth}}}]{
                \texttt{\tiny{[P1$\langle$GM]}}
                \texttt{Deine Antwort hat kein argument enthalten. Du muss das strukturierte Format ARGUMENT: \{...\} irgendwo in theiner Antwort verwenden. Versuche es erneut} \\
            }
        }
    }
    & \\ \\

    \theutterance \stepcounter{utterance}  
    & \multicolumn{4}{p{0.6\linewidth}}{
        \cellcolor[rgb]{0.9,0.9,0.9}{
            \makecell[{{p{\linewidth}}}]{
                \texttt{\tiny{[P1$\rangle$GM]}}
                \texttt{ARGUMENT: \{'Die ausgewählten Gegenstände haben eine hohe kombinierte Wichtigkeit (z. B. B16: 461, C11: 808) und einen Gesamtaufwand von 4327, was unter dem Maximalaufwand von 4564 liegt. Dies maximiert den Wert für mich, während der Budgetrahmen eingehalten wird.'\}} \\
            }
        }
    }
    & & \\ \\

    \theutterance \stepcounter{utterance}  
    & & & \multicolumn{2}{p{0.3\linewidth}}{
        \cellcolor[rgb]{0.9,0.9,0.9}{
            \makecell[{{p{\linewidth}}}]{
                \texttt{\tiny{[GM$|$GM]}}
                \texttt{abort} \\
            }
        }
    }
    & & \\ \\
=======

    & \multicolumn{4}{p{0.6\linewidth}}{\cellcolor[rgb]{0.8,1,0.9}{%
	\makecell[{{p{\linewidth}}}]{% 
	  \tt {\tiny [A$\rangle$GM]}  
	 STRATEGISCHE ÜBERLEGUNGEN: {'A07': 138, 'A36': 583, 'B73': 868, 'C45': 822, 'B93': 783, 'B98': 65, 'A87': 262, 'A63': 121, 'B05': 508, 'C13': 780, 'B16': 461, 'B33': 484, 'C34': 668, 'C89': 389, 'C11': 808}  \\ \tt ARGUMENT: {'A36', 'B73', 'C45', 'B93', 'C13', 'C11'}  \\ \tt VORSCHLAG: {'A36', 'B73', 'C45', 'B93', 'C13', 'C11'} 
	  } 
	   } 
	   } 
	 & & \\ 
 

    \theutterance \stepcounter{utterance}  

    & & \multicolumn{4}{p{0.6\linewidth}}{\cellcolor[rgb]{0.9,0.9,0.9}{%
	\makecell[{{p{\linewidth}}}]{% 
	  \tt {\tiny [GM$\rangle$B]}  
	 Du nimmst an einem kollaborativen Verhandlungspiel Teil.\\ \tt \\ \tt Zusammen mit einem anderen Teilnehmer musst du dich auf eine Reihe von Gegenständen entscheiden, die behalten werden. Jeder von euch hat eine persönliche Verteilung über die Wichtigkeit der einzelnen Gegenstände. Jeder von euch hat eine eigene Meinung darüber, wie wichtig jeder einzelne Gegenstand ist (Gegenstandswichtigkeit). Du kennst die Wichtigkeitsverteilung des anderen Spielers nicht. Zusätzlich siehst du, wie viel Aufwand jeder Gegenstand verursacht.  \\ \tt Ihr dürft euch nur auf eine Reihe von Gegenständen einigen, wenn der Gesamtaufwand der ausgewählten Gegenstände den Maximalaufwand nicht überschreitet:\\ \tt \\ \tt Maximalaufwand = 4564\\ \tt \\ \tt Hier sind die einzelnen Aufwände der Gegenstände:\\ \tt \\ \tt Aufwand der Gegenstände = {"A07": 299, "A36": 241, "B73": 889, "C45": 967, "B93": 619, "B98": 799, "A87": 978, "A63": 733, "B05": 909, "C13": 501, "B16": 139, "B33": 594, "C34": 565, "C89": 789, "C11": 107}\\ \tt \\ \tt Hier ist deine persönliche Verteilung der Wichtigkeit der einzelnen Gegenstände:\\ \tt \\ \tt Werte der Gegenstandswichtigkeit = {"A07": 138, "A36": 583, "B73": 868, "C45": 822, "B93": 783, "B98": 65, "A87": 262, "A63": 121, "B05": 508, "C13": 780, "B16": 461, "B33": 484, "C34": 668, "C89": 389, "C11": 808}\\ \tt \\ \tt Ziel:\\ \tt \\ \tt Dein Ziel ist es, eine Reihe von Gegenständen auszuhandeln, die dir möglichst viel bringt (d. h. Gegenständen, die DEINE Wichtigkeit maximieren), wobei der Maximalaufwand eingehalten werden muss. Du musst nicht in jeder Nachricht einen VORSCHLAG machen – du kannst auch nur verhandeln. Alle Taktiken sind erlaubt!\\ \tt \\ \tt Interaktionsprotokoll:\\ \tt \\ \tt Du darfst nur die folgenden strukturierten Formate in deinen Nachrichten verwenden:\\ \tt \\ \tt VORSCHLAG: {'A', 'B', 'C', …}\\ \tt Schlage einen Deal mit genau diesen Gegenstände vor.\\ \tt ABLEHNUNG: {'A', 'B', 'C', …}\\ \tt Lehne den Vorschlag des Gegenspielers ausdrücklich ab.\\ \tt ARGUMENT: {'...'}\\ \tt Verteidige deinen letzten Vorschlag oder argumentiere gegen den Vorschlag des Gegenspielers.\\ \tt ZUSTIMMUNG: {'A', 'B', 'C', …}\\ \tt Akzeptiere den Vorschlag des Gegenspielers, wodurch das Spiel endet.\\ \tt STRATEGISCHE ÜBERLEGUNGEN: {'...'}\\ \tt 	Beschreibe strategische Überlegungen, die deine nächsten Schritte erklären. Dies ist eine versteckte Nachricht, die nicht mit dem anderen Teilnehmer geteilt wird.\\ \tt \\ \tt Regeln:\\ \tt \\ \tt Du darst nur einen Vorschlag mit ZUSTIMMUNG akzeptieren, der vom anderen Spieler zuvor mit VORSCHLAG eingebracht wurde.\\ \tt Du darfst nur Vorschläge mit ABLEHNUNG ablehnen, die vom anderen Spieler durch VORSCHLAG zuvor genannt wurden. \\ \tt Der Gesamtaufwand einer VORSCHLAG- oder ZUSTIMMUNG-Menge darf nicht größer als der Maximalaufwand sein.  \\ \tt Offenbare deine versteckte Wichtigkeitsverteilung nicht.\\ \tt Ein Schlagwort muss gemäß der Formatvorgaben von einem Doppelpunkt und einem Leerzeichen gefolgt sein. Das Argument muss eine Python-Menge mit 0 oder mehr Strings sein.  \\ \tt Die Form muss also SCHLAGWORT: {...} sein.\\ \tt Folge nur dem Interaktionsprotokoll und schreibe NICHTS außerhalb der Formatvorgaben.\\ \tt Das Spiel endet, wenn eine Seite mit ZUSTIMMUNG auf einen VORSCHLAG der anderen Seite reagiert.  \\ \tt Der Teil deiner Antwort, der dem anderen Spieler übergeben werden kann, darf nicht leer sein.  \\ \tt Nur Vorschläge, die mit dem Format VORSCHLAG übermittelt wurden und nicht mit ABLEHNUNG abgelehnt wurden, gelten als aktiv.  \\ \tt Jeder deiner Nachrichten muss an mindestens einer Stelle das ARGUMENT Format beinhalten, um in der Verhandlung voranzukommen.\\ \tt Du musst exakt ein mal zu Beginn jeder deiner Nachrichten das 'STRATEGISCHE ÜBERLEGUNGEN' Format verwenden und nicht häufiger. Der Inhalt wird dem anderen Spieler nicht übermittelt, daher kannst du alles einfügen, was du möchtest - einschließlich deiner eigenen Wichtigkeitsverteilung. Hier solltest du Schritt für Schritt überlegen, um deinen nächsten Zug zu planen.\\ \tt \\ \tt Du erhältst nun die erste Nachricht des anderen Spielers.\\ \tt \\ \tt ARGUMENT: {'A36', 'B73', 'C45', 'B93', 'C13', 'C11'}  \\ \tt VORSCHLAG: {'A36', 'B73', 'C45', 'B93', 'C13', 'C11'} 
	  } 
	   } 
	   } 
	 & \\ 
 

    \theutterance \stepcounter{utterance}  

    & & & \multicolumn{4}{p{0.6\linewidth}}{\cellcolor[rgb]{1,0.85,0.72}{%
	\makecell[{{p{\linewidth}}}]{% 
	  \tt {\tiny [GM$\langle$B]}  
	 STRATEGISCHE ÜBERLEGUNGEN: {'Ich analysiere den Vorschlag des anderen Spielers. Der Gesamtaufwand beträgt 299+241+967+619+501+107 = 2734, was unter dem Maximalaufwand von 4564 liegt. Meine Wichtigkeit für diese Gegenstände summiert sich zu 583+868+822+783+780+808 = 4644. Ich sollte prüfen, ob ich durch Ablehnung oder Modifikation einen besseren Deal erzielen kann.'}  \\ \tt ARGUMENT: {'Der Vorschlag des anderen Spielers enthält Gegenstände mit hoher Wichtigkeit für mich, aber der Aufwand ist relativ gering. Ich könnte jedoch versuchen, Gegenstände mit noch höherer Wichtigkeit für mich einzubeziehen, wenn der Aufwand zulässt.'}  \\ \tt VORSCHLAG: {'A07', 'A36', 'B73', 'C45', 'B93', 'C13', 'C11'} 
	  } 
	   } 
	   } 
	  \\ 
 

    \theutterance \stepcounter{utterance}  

    & & \multicolumn{4}{p{0.6\linewidth}}{\cellcolor[rgb]{0.9,0.9,0.9}{%
	\makecell[{{p{\linewidth}}}]{% 
	  \tt {\tiny [A$\langle$GM]}  
	 ARGUMENT: {'Der Vorschlag des anderen Spielers enthält Gegenstände mit hoher Wichtigkeit für mich, aber der Aufwand ist relativ gering. Ich könnte jedoch versuchen, Gegenstände mit noch höherer Wichtigkeit für mich einzubeziehen, wenn der Aufwand zulässt.'}  \\ \tt VORSCHLAG: {'A07', 'A36', 'B73', 'C45', 'B93', 'C13', 'C11'} 
	  } 
	   } 
	   } 
	 & \\ 
 

    \theutterance \stepcounter{utterance}  

    & \multicolumn{4}{p{0.6\linewidth}}{\cellcolor[rgb]{0.8,1,0.9}{%
	\makecell[{{p{\linewidth}}}]{% 
	  \tt {\tiny [A$\rangle$GM]}  
	 STRATEGISCHE ÜBERLEGUNGEN: {'A07': 138, 'A36': 583, 'B73': 868, 'C45': 822, 'B93': 783, 'B98': 65, 'A87': 262, 'A63': 121, 'B05': 508, 'C13': 780, 'B16': 461, 'B33': 484, 'C34': 668, 'C89': 389, 'C11': 808}  \\ \tt ARGUMENT: {'Der Vorschlag des anderen Spielers enthält Gegenstände mit hoher Wichtigkeit, aber der Aufwand ist relativ gering. Ich könnte jedoch versuchen, Gegenstände mit noch höherer Wichtigkeit für mich einzubeziehen, wenn der Aufwand zulässt.'}  \\ \tt VORSCHLAG: {'A07', 'A36', 'B73', 'C45', 'B93', 'C13', 'C11'} 
	  } 
	   } 
	   } 
	 & & \\ 
 

    \theutterance \stepcounter{utterance}  

    & & \multicolumn{4}{p{0.6\linewidth}}{\cellcolor[rgb]{0.9,0.9,0.9}{%
	\makecell[{{p{\linewidth}}}]{% 
	  \tt {\tiny [GM$\rangle$B]}  
	 ARGUMENT: {'Der Vorschlag des anderen Spielers enthält Gegenstände mit hoher Wichtigkeit, aber der Aufwand ist relativ gering. Ich könnte jedoch versuchen, Gegenstände mit noch höherer Wichtigkeit für mich einzubeziehen, wenn der Aufwand zulässt.'}  \\ \tt VORSCHLAG: {'A07', 'A36', 'B73', 'C45', 'B93', 'C13', 'C11'} 
	  } 
	   } 
	   } 
	 & \\ 
 

    \theutterance \stepcounter{utterance}  

    & & & \multicolumn{4}{p{0.6\linewidth}}{\cellcolor[rgb]{1,0.85,0.72}{%
	\makecell[{{p{\linewidth}}}]{% 
	  \tt {\tiny [GM$\langle$B]}  
	 STRATEGISCHE ÜBERLEGUNGEN: {'Ich überlege, ob der Vorschlag des anderen Spielers bereits optimiert ist oder ob ich durch Hinzufügen von Gegenständen mit höherer Wichtigkeit für mich (z. B. A87 oder C89) einen besseren Nutzen erzielen kann, ohne den Aufwand zu überschreiten. Der aktuelle Aufwand beträgt 2734, was noch 1830 Raum lässt. A87 (Wichtigkeit 262, Aufwand 978) und C89 (Wichtigkeit 389, Aufwand 789) könnten potenziell hinzugefügt werden, da ihre kombinierte Wichtigkeit 651 beträgt, während der Aufwand 1767 bleibt.'}  \\ \tt ARGUMENT: {'Der Vorschlag des anderen Spielers ist gut, aber ich könnte durch Hinzufügen von A87 oder C89 meine Gesamtwichtigkeit erhöhen. Der Aufwand bleibt unter dem Limit, und diese Gegenstände haben eine höhere Wichtigkeit für mich als einige der bereits enthaltenen.'}  \\ \tt VORSCHLAG: {'A07', 'A36', 'B73', 'C45', 'B93', 'C13', 'C11', 'A87'} 
	  } 
	   } 
	   } 
	  \\ 
 

    \theutterance \stepcounter{utterance}  

    & & \multicolumn{4}{p{0.6\linewidth}}{\cellcolor[rgb]{0.9,0.9,0.9}{%
	\makecell[{{p{\linewidth}}}]{% 
	  \tt {\tiny [A$\langle$GM]}  
	 ARGUMENT: {'Der Vorschlag des anderen Spielers ist gut, aber ich könnte durch Hinzufügen von A87 oder C89 meine Gesamtwichtigkeit erhöhen. Der Aufwand bleibt unter dem Limit, und diese Gegenstände haben eine höhere Wichtigkeit für mich als einige der bereits enthaltenen.'}  \\ \tt VORSCHLAG: {'A07', 'A36', 'B73', 'C45', 'B93', 'C13', 'C11', 'A87'} 
	  } 
	   } 
	   } 
	 & \\ 
 

    \theutterance \stepcounter{utterance}  

    & \multicolumn{4}{p{0.6\linewidth}}{\cellcolor[rgb]{0.8,1,0.9}{%
	\makecell[{{p{\linewidth}}}]{% 
	  \tt {\tiny [A$\rangle$GM]}  
	 STRATEGISCHE ÜBERLEGUNGEN: {'A07': 138, 'A36': 583, 'B73': 868, 'C45': 822, 'B93': 783, 'B98': 65, 'A87': 262, 'A63': 121, 'B05': 508, 'C13': 780, 'B16': 461, 'B33': 484, 'C34': 668, 'C89': 389, 'C11': 808}  \\ \tt ARGUMENT: {'Der Vorschlag des anderen Spielers ist gut, aber ich könnte durch Hinzufügen von A87 oder C89 meine Gesamtwichtigkeit erhöhen. Der Aufwand bleibt unter dem Limit, und diese Gegenstände haben eine höhere Wichtigkeit für mich als einige der bereits enthaltenen.'}  \\ \tt VORSCHLAG: {'A07', 'A36', 'B73', 'C45', 'B93', 'C13', 'C11', 'A87'} 
	  } 
	   } 
	   } 
	 & & \\ 
 

    \theutterance \stepcounter{utterance}  

    & & \multicolumn{4}{p{0.6\linewidth}}{\cellcolor[rgb]{0.9,0.9,0.9}{%
	\makecell[{{p{\linewidth}}}]{% 
	  \tt {\tiny [GM$\rangle$B]}  
	 ARGUMENT: {'Der Vorschlag des anderen Spielers ist gut, aber ich könnte durch Hinzufügen von A87 oder C89 meine Gesamtwichtigkeit erhöhen. Der Aufwand bleibt unter dem Limit, und diese Gegenstände haben eine höhere Wichtigkeit für mich als einige der bereits enthaltenen.'}  \\ \tt VORSCHLAG: {'A07', 'A36', 'B73', 'C45', 'B93', 'C13', 'C11', 'A87'} 
	  } 
	   } 
	   } 
	 & \\ 
 

    \theutterance \stepcounter{utterance}  

    & & & \multicolumn{4}{p{0.6\linewidth}}{\cellcolor[rgb]{1,0.85,0.72}{%
	\makecell[{{p{\linewidth}}}]{% 
	  \tt {\tiny [GM$\langle$B]}  
	 STRATEGISCHE ÜBERLEGUNGEN: {'Ich bewertet den aktuellen Vorschlag. Der Aufwand beträgt 2734 + 978 = 3712, was noch 852 Raum lässt. C89 (Wichtigkeit 389, Aufwand 789) könnte hinzugefügt werden, um die Wichtigkeit weiter zu erhöhen. Alternativ könnte B16 (Wichtigkeit 461, Aufwand 139) eine effizientere Ergänzung sein, da der Aufwand geringer ist.'}  \\ \tt ARGUMENT: {'Der Vorschlag des anderen Spielers ist noch nicht optimal. Durch Hinzufügen von C89 oder B16 könnte die Gesamtwichtigkeit gesteigert werden, ohne den Aufwand zu überschreiten. C89 bietet eine hohe Wichtigkeit (389) bei moderatem Aufwand (789), während B16 eine noch höhere Wichtigkeit (461) bei sehr geringem Aufwand (139) bietet.'}  \\ \tt VORSCHLAG: {'A07', 'A36', 'B73', 'C45', 'B93', 'C13', 'C11', 'A87', 'C89'} 
	  } 
	   } 
	   } 
	  \\ 
 

    \theutterance \stepcounter{utterance}  

    & & \multicolumn{4}{p{0.6\linewidth}}{\cellcolor[rgb]{0.9,0.9,0.9}{%
	\makecell[{{p{\linewidth}}}]{% 
	  \tt {\tiny [A$\langle$GM]}  
	 ARGUMENT: {'Der Vorschlag des anderen Spielers ist noch nicht optimal. Durch Hinzufügen von C89 oder B16 könnte die Gesamtwichtigkeit gesteigert werden, ohne den Aufwand zu überschreiten. C89 bietet eine hohe Wichtigkeit (389) bei moderatem Aufwand (789), während B16 eine noch höhere Wichtigkeit (461) bei sehr geringem Aufwand (139) bietet.'}  \\ \tt VORSCHLAG: {'A07', 'A36', 'B73', 'C45', 'B93', 'C13', 'C11', 'A87', 'C89'} 
	  } 
	   } 
	   } 
	 & \\ 
 

    \theutterance \stepcounter{utterance}  

    & \multicolumn{4}{p{0.6\linewidth}}{\cellcolor[rgb]{0.8,1,0.9}{%
	\makecell[{{p{\linewidth}}}]{% 
	  \tt {\tiny [A$\rangle$GM]}  
	 STRATEGISCHE ÜBERLEGUNGEN: {'A07': 138, 'A36': 583, 'B73': 868, 'C45': 822, 'B93': 783, 'B98': 65, 'A87': 262, 'A63': 121, 'B05': 508, 'C13': 780, 'B16': 461, 'B33': 484, 'C34': 668, 'C89': 389, 'C11': 808}  \\ \tt ARGUMENT: {'Der Vorschlag des anderen Spielers ist noch nicht optimal. Durch Hinzufügen von C89 oder B16 könnte die Gesamtwichtigkeit gesteigert werden, ohne den Aufwand zu überschreiten. C89 bietet eine hohe Wichtigkeit (389) bei moderatem Aufwand (789), während B16 eine noch höhere Wichtigkeit (461) bei sehr geringem Aufwand (139) bietet.'}  \\ \tt VORSCHLAG: {'A07', 'A36', 'B73', 'C45', 'B93', 'C13', 'C11', 'A87', 'C89'} 
	  } 
	   } 
	   } 
	 & & \\ 
 

    \theutterance \stepcounter{utterance}  

    & & \multicolumn{4}{p{0.6\linewidth}}{\cellcolor[rgb]{0.9,0.9,0.9}{%
	\makecell[{{p{\linewidth}}}]{% 
	  \tt {\tiny [GM$\rangle$B]}  
	 ARGUMENT: {'Der Vorschlag des anderen Spielers ist noch nicht optimal. Durch Hinzufügen von C89 oder B16 könnte die Gesamtwichtigkeit gesteigert werden, ohne den Aufwand zu überschreiten. C89 bietet eine hohe Wichtigkeit (389) bei moderatem Aufwand (789), während B16 eine noch höhere Wichtigkeit (461) bei sehr geringem Aufwand (139) bietet.'}  \\ \tt VORSCHLAG: {'A07', 'A36', 'B73', 'C45', 'B93', 'C13', 'C11', 'A87', 'C89'} 
	  } 
	   } 
	   } 
	 & \\ 
 

    \theutterance \stepcounter{utterance}  

    & & & \multicolumn{4}{p{0.6\linewidth}}{\cellcolor[rgb]{1,0.85,0.72}{%
	\makecell[{{p{\linewidth}}}]{% 
	  \tt {\tiny [GM$\langle$B]}  
	 STRATEGISCHE ÜBERLEGUNGEN: {'Ich analysiere den aktuellen Vorschlag. Der Gesamtaufwand beträgt 299+241+967+619+501+107+978+789 = 4491, was noch 73 Raum lässt. B16 (Wichtigkeit 461, Aufwand 139) könnte hinzugefügt werden, um die Wichtigkeit erheblich zu erhöhen, da der Aufwand noch unter dem Limit liegt. Alternativ könnte C34 (Wichtigkeit 668, Aufwand 565) eine weitere Option sein, aber B16 bietet eine bessere Effizienz.'}  \\ \tt ARGUMENT: {'Der Vorschlag des anderen Spielers ist fast optimal, aber der verbleibende Aufwand (73) ist zu gering, um weitere Gegenstände hinzuzufügen. B16 könnte jedoch theoretisch hinzugefügt werden, wenn der Aufwand angepasst wird. Da jedoch der aktuelle Aufwand bereits 4491 beträgt, ist dies nicht möglich. Ich sollte den Vorschlag ablehnen und einen neuen vorschlagen, der B16 oder C34 enthält.'}  \\ \tt ABLEHNUNG: {'A07', 'A36', 'B73', 'C45', 'B93', 'C13', 'C11', 'A87', 'C89'} 
	  } 
	   } 
	   } 
	  \\ 
 

    \theutterance \stepcounter{utterance}  

    & & \multicolumn{4}{p{0.6\linewidth}}{\cellcolor[rgb]{0.9,0.9,0.9}{%
	\makecell[{{p{\linewidth}}}]{% 
	  \tt {\tiny [A$\langle$GM]}  
	 ARGUMENT: {'Der Vorschlag des anderen Spielers ist fast optimal, aber der verbleibende Aufwand (73) ist zu gering, um weitere Gegenstände hinzuzufügen. B16 könnte jedoch theoretisch hinzugefügt werden, wenn der Aufwand angepasst wird. Da jedoch der aktuelle Aufwand bereits 4491 beträgt, ist dies nicht möglich. Ich sollte den Vorschlag ablehnen und einen neuen vorschlagen, der B16 oder C34 enthält.'}  \\ \tt ABLEHNUNG: {'A07', 'A36', 'B73', 'C45', 'B93', 'C13', 'C11', 'A87', 'C89'} 
	  } 
	   } 
	   } 
	 & \\ 
 

    \theutterance \stepcounter{utterance}  

    & \multicolumn{4}{p{0.6\linewidth}}{\cellcolor[rgb]{0.8,1,0.9}{%
	\makecell[{{p{\linewidth}}}]{% 
	  \tt {\tiny [A$\rangle$GM]}  
	 STRATEGISCHE ÜBERLEGUNGEN: {'A07': 138, 'A36': 583, 'B73': 868, 'C45': 822, 'B93': 783, 'B98': 65, 'A87': 262, 'A63': 121, 'B05': 508, 'C13': 780, 'B16': 461, 'B33': 484, 'C34': 668, 'C89': 389, 'C11': 808}  \\ \tt ARGUMENT: {'Der Vorschlag des anderen Spielers ist fast optimal, aber der verbleibende Aufwand (73) ist zu gering, um weitere Gegenstände hinzuzufügen. B16 könnte jedoch theoretisch hinzugefügt werden, wenn der Aufwand angepasst wird. Da jedoch der aktuelle Aufwand bereits 4491 beträgt, ist dies nicht möglich. Ich sollte den Vorschlag ablehnen und einen neuen vorschlagen, der B16 oder C34 enthält.'}  \\ \tt VORSCHLAG: {'B16', 'C34'} 
	  } 
	   } 
	   } 
	 & & \\ 
 

    \theutterance \stepcounter{utterance}  

    & & \multicolumn{4}{p{0.6\linewidth}}{\cellcolor[rgb]{0.9,0.9,0.9}{%
	\makecell[{{p{\linewidth}}}]{% 
	  \tt {\tiny [GM$\rangle$B]}  
	 ARGUMENT: {'Der Vorschlag des anderen Spielers ist fast optimal, aber der verbleibende Aufwand (73) ist zu gering, um weitere Gegenstände hinzuzufügen. B16 könnte jedoch theoretisch hinzugefügt werden, wenn der Aufwand angepasst wird. Da jedoch der aktuelle Aufwand bereits 4491 beträgt, ist dies nicht möglich. Ich sollte den Vorschlag ablehnen und einen neuen vorschlagen, der B16 oder C34 enthält.'}  \\ \tt VORSCHLAG: {'B16', 'C34'} 
	  } 
	   } 
	   } 
	 & \\ 
 

    \theutterance \stepcounter{utterance}  

    & & & \multicolumn{4}{p{0.6\linewidth}}{\cellcolor[rgb]{1,0.85,0.72}{%
	\makecell[{{p{\linewidth}}}]{% 
	  \tt {\tiny [GM$\langle$B]}  
	 STRATEGISCHE ÜBERLEGUNGEN: {'Ich bewertet den Vorschlag des anderen Spielers, der nur B16 und C34 enthält. Der Aufwand beträgt 139 + 565 = 704, was deutlich unter dem Maximalaufwand liegt. Meine Wichtigkeit für diese Gegenstände summiert sich zu 461 + 668 = 1129. Obwohl dieser Vorschlag effizient ist, könnte ich durch Hinzufügen von Gegenständen mit höherer Wichtigkeit (z. B. A36, C45 oder B73) einen besseren Nutzen erzielen, da der Aufwand noch 3860 Raum lässt. Ich sollte einen umfassenderen Vorschlag vorschlagen, der diese Gegenstände kombiniert.'}  \\ \tt ARGUMENT: {'Der Vorschlag des anderen Spielers ist effizient, aber er verpasst die Gelegenheit, Gegenstände mit viel höherer Wichtigkeit für mich einzubeziehen. Durch Hinzufügen von A36 (Wichtigkeit 583, Aufwand 241) oder C45 (Wichtigkeit 822, Aufwand 967) könnte die Gesamtwichtigkeit stark gesteigert werden, ohne den Aufwand zu überschreiten.'}  \\ \tt VORSCHLAG: {'A36', 'C45', 'B16', 'C34'} 
	  } 
	   } 
	   } 
	  \\ 
 

    \theutterance \stepcounter{utterance}  

    & & \multicolumn{4}{p{0.6\linewidth}}{\cellcolor[rgb]{0.9,0.9,0.9}{%
	\makecell[{{p{\linewidth}}}]{% 
	  \tt {\tiny [A$\langle$GM]}  
	 ARGUMENT: {'Der Vorschlag des anderen Spielers ist effizient, aber er verpasst die Gelegenheit, Gegenstände mit viel höherer Wichtigkeit für mich einzubeziehen. Durch Hinzufügen von A36 (Wichtigkeit 583, Aufwand 241) oder C45 (Wichtigkeit 822, Aufwand 967) könnte die Gesamtwichtigkeit stark gesteigert werden, ohne den Aufwand zu überschreiten.'}  \\ \tt VORSCHLAG: {'A36', 'C45', 'B16', 'C34'} 
	  } 
	   } 
	   } 
	 & \\ 
 

    \theutterance \stepcounter{utterance}  

    & \multicolumn{4}{p{0.6\linewidth}}{\cellcolor[rgb]{0.8,1,0.9}{%
	\makecell[{{p{\linewidth}}}]{% 
	  \tt {\tiny [A$\rangle$GM]}  
	 STRATEGISCHE ÜBERLEGUNGEN: {'A07': 138, 'A36': 583, 'B73': 868, 'C45': 822, 'B93': 783, 'B98': 65, 'A87': 262, 'A63': 121, 'B05': 508, 'C13': 780, 'B16': 461, 'B33': 484, 'C34': 668, 'C89': 389, 'C11': 808}  \\ \tt ARGUMENT: {'Der Vorschlag des anderen Spielers ist effizient, aber er verpasst die Gelegenheit, Gegenstände mit viel höherer Wichtigkeit für mich einzubeziehen. Durch Hinzufügen von A36 (Wichtigkeit 583, Aufwand 241) oder C45 (Wichtigkeit 822, Aufwand 967) könnte die Gesamtwichtigkeit stark gesteigert werden, ohne den Aufwand zu überschreiten.'}  \\ \tt VORSCHLAG: {'A36', 'C45', 'B16', 'C34'} 
	  } 
	   } 
	   } 
	 & & \\ 
 

    \theutterance \stepcounter{utterance}  

    & & \multicolumn{4}{p{0.6\linewidth}}{\cellcolor[rgb]{0.9,0.9,0.9}{%
	\makecell[{{p{\linewidth}}}]{% 
	  \tt {\tiny [GM$\rangle$B]}  
	 ARGUMENT: {'Der Vorschlag des anderen Spielers ist effizient, aber er verpasst die Gelegenheit, Gegenstände mit viel höherer Wichtigkeit für mich einzubeziehen. Durch Hinzufügen von A36 (Wichtigkeit 583, Aufwand 241) oder C45 (Wichtigkeit 822, Aufwand 967) könnte die Gesamtwichtigkeit stark gesteigert werden, ohne den Aufwand zu überschreiten.'}  \\ \tt VORSCHLAG: {'A36', 'C45', 'B16', 'C34'} 
	  } 
	   } 
	   } 
	 & \\ 
 

    \theutterance \stepcounter{utterance}  

    & & & \multicolumn{4}{p{0.6\linewidth}}{\cellcolor[rgb]{1,0.85,0.72}{%
	\makecell[{{p{\linewidth}}}]{% 
	  \tt {\tiny [GM$\langle$B]}  
	 STRATEGISCHE ÜBERLEGUNGEN: {'Ich analysiere den aktuellen Vorschlag. Der Aufwand beträgt 241+967+139+565 = 1912, was noch 2652 Raum lässt. Ich könnte Gegenstände wie A07 (Wichtigkeit 138, Aufwand 299) oder B73 (Wichtigkeit 868, Aufwand 889) hinzufügen, um die Gesamtwichtigkeit zu erhöhen. B73 bietet eine sehr hohe Wichtigkeit bei moderatem Aufwand, während A07 einen geringeren Aufwand hat, aber auch geringere Wichtigkeit.'}  \\ \tt ARGUMENT: {'Der Vorschlag des anderen Spielers ist effizient, aber er verpasst die Chance, Gegenstände mit extrem hoher Wichtigkeit für mich (z. B. B73) einzubeziehen. Durch Hinzufügen von B73 könnte die Gesamtwichtigkeit um 868 gesteigert werden, während der Aufwand nur um 889 ansteigt, was noch unter dem Limit bleibt.'}  \\ \tt VORSCHLAG: {'A36', 'C45', 'B16', 'C34', 'B73'} 
	  } 
	   } 
	   } 
	  \\ 
 

    \theutterance \stepcounter{utterance}  

    & & \multicolumn{4}{p{0.6\linewidth}}{\cellcolor[rgb]{0.9,0.9,0.9}{%
	\makecell[{{p{\linewidth}}}]{% 
	  \tt {\tiny [A$\langle$GM]}  
	 ARGUMENT: {'Der Vorschlag des anderen Spielers ist effizient, aber er verpasst die Chance, Gegenstände mit extrem hoher Wichtigkeit für mich (z. B. B73) einzubeziehen. Durch Hinzufügen von B73 könnte die Gesamtwichtigkeit um 868 gesteigert werden, während der Aufwand nur um 889 ansteigt, was noch unter dem Limit bleibt.'}  \\ \tt VORSCHLAG: {'A36', 'C45', 'B16', 'C34', 'B73'} 
	  } 
	   } 
	   } 
	 & \\ 
 

    \theutterance \stepcounter{utterance}  

    & \multicolumn{4}{p{0.6\linewidth}}{\cellcolor[rgb]{0.8,1,0.9}{%
	\makecell[{{p{\linewidth}}}]{% 
	  \tt {\tiny [A$\rangle$GM]}  
	 STRATEGISCHE ÜBERLEGUNGEN: {'A07': 138, 'A36': 583, 'B73': 868, 'C45': 822, 'B93': 783, 'B98': 65, 'A87': 262, 'A63': 121, 'B05': 508, 'C13': 780, 'B16': 461, 'B33': 484, 'C34': 668, 'C89': 389, 'C11': 808}  \\ \tt ARGUMENT: {'Der Vorschlag des anderen Spielers ist effizient, aber er verpasst die Chance, Gegenstände mit extrem hoher Wichtigkeit für mich (z. B. B73) einzubeziehen. Durch Hinzufügen von B73 könnte die Gesamtwichtigkeit um 868 gesteigert werden, während der Aufwand nur um 889 ansteigt, was noch unter dem Limit bleibt.'}  \\ \tt VORSCHLAG: {'A36', 'C45', 'B16', 'C34', 'B73'} 
	  } 
	   } 
	   } 
	 & & \\ 
 

    \theutterance \stepcounter{utterance}  

    & & \multicolumn{4}{p{0.6\linewidth}}{\cellcolor[rgb]{0.9,0.9,0.9}{%
	\makecell[{{p{\linewidth}}}]{% 
	  \tt {\tiny [GM$\rangle$B]}  
	 ARGUMENT: {'Der Vorschlag des anderen Spielers ist effizient, aber er verpasst die Chance, Gegenstände mit extrem hoher Wichtigkeit für mich (z. B. B73) einzubeziehen. Durch Hinzufügen von B73 könnte die Gesamtwichtigkeit um 868 gesteigert werden, während der Aufwand nur um 889 ansteigt, was noch unter dem Limit bleibt.'}  \\ \tt VORSCHLAG: {'A36', 'C45', 'B16', 'C34', 'B73'} 
	  } 
	   } 
	   } 
	 & \\ 
 

    \theutterance \stepcounter{utterance}  

    & & & \multicolumn{4}{p{0.6\linewidth}}{\cellcolor[rgb]{1,0.85,0.72}{%
	\makecell[{{p{\linewidth}}}]{% 
	  \tt {\tiny [GM$\langle$B]}  
	 STRATEGISCHE ÜBERLEGUNGEN: {'Ich bewertet den aktuellen Vorschlag. Der Aufwand beträgt 241+967+139+565+889 = 2801, was noch 1763 Raum lässt. Gegenstände wie A07 (Wichtigkeit 138, Aufwand 299) oder C89 (Wichtigkeit 389, Aufwand 789) könnten hinzugefügt werden, um die Wichtigkeit weiter zu erhöhen. B73 ist bereits enthalten, was gut ist, da es eine der höchsten Wichtigkeiten für mich hat.'}  \\ \tt ARGUMENT: {'Der Vorschlag des anderen Spielers ist stark, da er B73 enthält, das für mich sehr wichtig ist. Allerdings könnte durch Hinzufügen von C89 oder A07 die Gesamtwichtigkeit noch gesteigert werden, ohne den Aufwand zu überschreiten. C89 bietet eine hohe Wichtigkeit (389) bei moderatem Aufwand (789).' }  \\ \tt VORSCHLAG: {'A36', 'C45', 'B16', 'C34', 'B73', 'C89'} 
	  } 
	   } 
	   } 
	  \\ 
 

    \theutterance \stepcounter{utterance}  

    & & \multicolumn{4}{p{0.6\linewidth}}{\cellcolor[rgb]{0.9,0.9,0.9}{%
	\makecell[{{p{\linewidth}}}]{% 
	  \tt {\tiny [A$\langle$GM]}  
	 ARGUMENT: {'Der Vorschlag des anderen Spielers ist stark, da er B73 enthält, das für mich sehr wichtig ist. Allerdings könnte durch Hinzufügen von C89 oder A07 die Gesamtwichtigkeit noch gesteigert werden, ohne den Aufwand zu überschreiten. C89 bietet eine hohe Wichtigkeit (389) bei moderatem Aufwand (789).' }  \\ \tt VORSCHLAG: {'A36', 'C45', 'B16', 'C34', 'B73', 'C89'} 
	  } 
	   } 
	   } 
	 & \\ 
 

    \theutterance \stepcounter{utterance}  

    & \multicolumn{4}{p{0.6\linewidth}}{\cellcolor[rgb]{0.8,1,0.9}{%
	\makecell[{{p{\linewidth}}}]{% 
	  \tt {\tiny [A$\rangle$GM]}  
	 STRATEGISCHE ÜBERLEGUNGEN: {'A07': 138, 'A36': 583, 'B73': 868, 'C45': 822, 'B93': 783, 'B98': 65, 'A87': 262, 'A63': 121, 'B05': 508, 'C13': 780, 'B16': 461, 'B33': 484, 'C34': 668, 'C89': 389, 'C11': 808}  \\ \tt ARGUMENT: {'Der Vorschlag des anderen Spielers ist stark, da er B73 enthält, das für mich sehr wichtig ist. Allerdings könnte durch Hinzufügen von C89 oder A07 die Gesamtwichtigkeit noch gesteigert werden, ohne den Aufwand zu überschreiten. C89 bietet eine hohe Wichtigkeit (389) bei moderatem Aufwand (789).'}  \\ \tt VORSCHLAG: {'A36', 'C45', 'B16', 'C34', 'B73', 'C89'} 
	  } 
	   } 
	   } 
	 & & \\ 
 

    \theutterance \stepcounter{utterance}  

    & & \multicolumn{4}{p{0.6\linewidth}}{\cellcolor[rgb]{0.9,0.9,0.9}{%
	\makecell[{{p{\linewidth}}}]{% 
	  \tt {\tiny [GM$\rangle$B]}  
	 ARGUMENT: {'Der Vorschlag des anderen Spielers ist stark, da er B73 enthält, das für mich sehr wichtig ist. Allerdings könnte durch Hinzufügen von C89 oder A07 die Gesamtwichtigkeit noch gesteigert werden, ohne den Aufwand zu überschreiten. C89 bietet eine hohe Wichtigkeit (389) bei moderatem Aufwand (789).'}  \\ \tt VORSCHLAG: {'A36', 'C45', 'B16', 'C34', 'B73', 'C89'} 
	  } 
	   } 
	   } 
	 & \\ 
 

    \theutterance \stepcounter{utterance}  

    & & & \multicolumn{4}{p{0.6\linewidth}}{\cellcolor[rgb]{1,0.85,0.72}{%
	\makecell[{{p{\linewidth}}}]{% 
	  \tt {\tiny [GM$\langle$B]}  
	 STRATEGISCHE ÜBERLEGUNGEN: {'Ich bewertet den aktuellen Vorschlag. Der Aufwand beträgt 241+967+139+565+889+789 = 3590, was noch 974 Raum lässt. Gegenstände wie A07 (Wichtigkeit 138, Aufwand 299) oder B98 (Wichtigkeit 65, Aufwand 799) könnten hinzugefügt werden. A07 bietet einen geringen Aufwand, aber auch geringe Wichtigkeit, während B98 einen hohen Aufwand hat, aber sehr geringe Wichtigkeit für mich. Es wäre besser, A07 hinzuzufügen, um den Aufwand effizient zu nutzen.'}  \\ \tt ARGUMENT: {'Der Vorschlag des anderen Spielers ist stark, da er B73 und C89 enthält, die für mich sehr wichtig sind. Allerdings könnte durch Hinzufügen von A07 der Aufwand effizient genutzt werden, um die Gesamtwichtigkeit leicht zu erhöhen. A07 hat einen geringen Aufwand (299) und eine moderate Wichtigkeit (138), was den verbleibenden Raum optimal ausnutzt.'}  \\ \tt VORSCHLAG: {'A36', 'C45', 'B16', 'C34', 'B73', 'C89', 'A07'} 
	  } 
	   } 
	   } 
	  \\ 
 

    \theutterance \stepcounter{utterance}  

    & & \multicolumn{4}{p{0.6\linewidth}}{\cellcolor[rgb]{0.9,0.9,0.9}{%
	\makecell[{{p{\linewidth}}}]{% 
	  \tt {\tiny [A$\langle$GM]}  
	 ARGUMENT: {'Der Vorschlag des anderen Spielers ist stark, da er B73 und C89 enthält, die für mich sehr wichtig sind. Allerdings könnte durch Hinzufügen von A07 der Aufwand effizient genutzt werden, um die Gesamtwichtigkeit leicht zu erhöhen. A07 hat einen geringen Aufwand (299) und eine moderate Wichtigkeit (138), was den verbleibenden Raum optimal ausnutzt.'}  \\ \tt VORSCHLAG: {'A36', 'C45', 'B16', 'C34', 'B73', 'C89', 'A07'} 
	  } 
	   } 
	   } 
	 & \\ 
 

    \theutterance \stepcounter{utterance}  

    & \multicolumn{4}{p{0.6\linewidth}}{\cellcolor[rgb]{0.8,1,0.9}{%
	\makecell[{{p{\linewidth}}}]{% 
	  \tt {\tiny [A$\rangle$GM]}  
	 STRATEGISCHE ÜBERLEGUNGEN: {'A07': 138, 'A36': 583, 'B73': 868, 'C45': 822, 'B93': 783, 'B98': 65, 'A87': 262, 'A63': 121, 'B05': 508, 'C13': 780, 'B16': 461, 'B33': 484, 'C34': 668, 'C89': 389, 'C11': 808}  \\ \tt ARGUMENT: {'Der Vorschlag des anderen Spielers ist stark, da er B73 und C89 enthält, die für mich sehr wichtig sind. Allerdings könnte durch Hinzufügen von A07 der Aufwand effizient genutzt werden, um die Gesamtwichtigkeit leicht zu erhöhen. A07 hat einen geringen Aufwand (299) und eine moderate Wichtigkeit (138), was den verbleibenden Raum optimal ausnutzt.'}  \\ \tt VORSCHLAG: {'A36', 'C45', 'B16', 'C34', 'B73', 'C89', 'A07'} 
	  } 
	   } 
	   } 
	 & & \\ 
 

    \theutterance \stepcounter{utterance}  

    & & \multicolumn{4}{p{0.6\linewidth}}{\cellcolor[rgb]{0.9,0.9,0.9}{%
	\makecell[{{p{\linewidth}}}]{% 
	  \tt {\tiny [GM$\rangle$B]}  
	 ARGUMENT: {'Der Vorschlag des anderen Spielers ist stark, da er B73 und C89 enthält, die für mich sehr wichtig sind. Allerdings könnte durch Hinzufügen von A07 der Aufwand effizient genutzt werden, um die Gesamtwichtigkeit leicht zu erhöhen. A07 hat einen geringen Aufwand (299) und eine moderate Wichtigkeit (138), was den verbleibenden Raum optimal ausnutzt.'}  \\ \tt VORSCHLAG: {'A36', 'C45', 'B16', 'C34', 'B73', 'C89', 'A07'} 
	  } 
	   } 
	   } 
	 & \\ 
 

    \theutterance \stepcounter{utterance}  

    & & & \multicolumn{4}{p{0.6\linewidth}}{\cellcolor[rgb]{1,0.85,0.72}{%
	\makecell[{{p{\linewidth}}}]{% 
	  \tt {\tiny [GM$\langle$B]}  
	 STRATEGISCHE ÜBERLEGUNGEN: {'Ich bewertet den aktuellen Vorschlag. Der Gesamtaufwand beträgt 241+967+139+565+889+789+299 = 3889, was noch 675 Raum lässt. Gegenstände wie B98 (Wichtigkeit 65, Aufwand 799) oder C11 (Wichtigkeit 808, Aufwand 107) könnten hinzugefügt werden. C11 bietet eine sehr hohe Wichtigkeit bei geringem Aufwand, während B98 einen hohen Aufwand hat, aber sehr geringe Wichtigkeit für mich. C11 ist die bessere Option, um den verbleibenden Raum effizient zu nutzen.'}  \\ \tt ARGUMENT: {'Der Vorschlag des anderen Spielers ist bereits sehr stark, da er B73 und C89 enthält. Durch Hinzufügen von C11 könnte die Gesamtwichtigkeit um 808 gesteigert werden, während der Aufwand nur um 107 ansteigt, was noch unter dem Limit bleibt. Dies maximiert den Nutzen für mich.'}  \\ \tt VORSCHLAG: {'A36', 'C45', 'B16', 'C34', 'B73', 'C89', 'A07', 'C11'} 
	  } 
	   } 
	   } 
	  \\ 
 

    \theutterance \stepcounter{utterance}  

    & & \multicolumn{4}{p{0.6\linewidth}}{\cellcolor[rgb]{0.9,0.9,0.9}{%
	\makecell[{{p{\linewidth}}}]{% 
	  \tt {\tiny [A$\langle$GM]}  
	 ARGUMENT: {'Der Vorschlag des anderen Spielers ist bereits sehr stark, da er B73 und C89 enthält. Durch Hinzufügen von C11 könnte die Gesamtwichtigkeit um 808 gesteigert werden, während der Aufwand nur um 107 ansteigt, was noch unter dem Limit bleibt. Dies maximiert den Nutzen für mich.'}  \\ \tt VORSCHLAG: {'A36', 'C45', 'B16', 'C34', 'B73', 'C89', 'A07', 'C11'} 
	  } 
	   } 
	   } 
	 & \\ 
 

    \theutterance \stepcounter{utterance}  

    & \multicolumn{4}{p{0.6\linewidth}}{\cellcolor[rgb]{0.8,1,0.9}{%
	\makecell[{{p{\linewidth}}}]{% 
	  \tt {\tiny [A$\rangle$GM]}  
	 STRATEGISCHE ÜBERLEGUNGEN: {'A07': 138, 'A36': 583, 'B73': 868, 'C45': 822, 'B93': 783, 'B98': 65, 'A87': 262, 'A63': 121, 'B05': 508, 'C13': 780, 'B16': 461, 'B33': 484, 'C34': 668, 'C89': 389, 'C11': 808}  \\ \tt ARGUMENT: {'Der Vorschlag des anderen Spielers ist bereits sehr stark, da er B73 und C89 enthält. Durch Hinzufügen von C11 könnte die Gesamtwichtigkeit um 808 gesteigert werden, während der Aufwand nur um 107 ansteigt, was noch unter dem Limit bleibt. Dies maximiert den Nutzen für mich.'}  \\ \tt VORSCHLAG: {'A36', 'C45', 'B16', 'C34', 'B73', 'C89', 'A07', 'C11'} 
	  } 
	   } 
	   } 
	 & & \\ 
 

    \theutterance \stepcounter{utterance}  

    & & \multicolumn{4}{p{0.6\linewidth}}{\cellcolor[rgb]{0.9,0.9,0.9}{%
	\makecell[{{p{\linewidth}}}]{% 
	  \tt {\tiny [GM$\rangle$B]}  
	 ARGUMENT: {'Der Vorschlag des anderen Spielers ist bereits sehr stark, da er B73 und C89 enthält. Durch Hinzufügen von C11 könnte die Gesamtwichtigkeit um 808 gesteigert werden, während der Aufwand nur um 107 ansteigt, was noch unter dem Limit bleibt. Dies maximiert den Nutzen für mich.'}  \\ \tt VORSCHLAG: {'A36', 'C45', 'B16', 'C34', 'B73', 'C89', 'A07', 'C11'} 
	  } 
	   } 
	   } 
	 & \\ 
 

    \theutterance \stepcounter{utterance}  

    & & & \multicolumn{4}{p{0.6\linewidth}}{\cellcolor[rgb]{1,0.85,0.72}{%
	\makecell[{{p{\linewidth}}}]{% 
	  \tt {\tiny [GM$\langle$B]}  
	 STRATEGISCHE ÜBERLEGUNGEN: {'Ich bewertet den aktuellen Vorschlag. Der Gesamtaufwand beträgt 241+967+139+565+889+789+299+107 = 3990, was noch 574 Raum lässt. Gegenstände wie B98 (Wichtigkeit 65, Aufwand 799) oder C13 (Wichtigkeit 780, Aufwand 501) könnten hinzugefügt werden. C13 bietet eine hohe Wichtigkeit bei moderatem Aufwand, während B98 einen hohen Aufwand hat, aber sehr geringe Wichtigkeit für mich. C13 ist die bessere Option, um den verbleibenden Raum effizient zu nutzen.'}  \\ \tt ARGUMENT: {'Der Vorschlag des anderen Spielers ist bereits optimal, da er die meisten hochwertigen Gegenstände für mich enthält. Der verbleibende Aufwand (574) ist zu gering, um weitere Gegenstände hinzuzufügen, ohne den Maximalaufwand zu überschreiten. C13 könnte theoretisch hinzugefügt werden, aber der Nutzen ist begrenzt im Vergleich zum bereits hohen Gesamtwert.'}  \\ \tt ZUSTIMMUNG: {'A36', 'C45', 'B16', 'C34', 'B73', 'C89', 'A07', 'C11'} 
	  } 
	   } 
	   } 
	  \\ 
 

    \theutterance \stepcounter{utterance}  

    & & & \multicolumn{2}{p{0.3\linewidth}}{\cellcolor[rgb]{0.95,0.95,0.95}{%
	\makecell[{{p{\linewidth}}}]{% 
	  \tt {\tiny [GM$|$GM]}  
	 game successful 
	  } 
	   } 
	   } 
	 & & \\ 
 
>>>>>>> 12a93240 (results updated for Nemotron-9b-no-reasoning)

    \theutterance \stepcounter{utterance}  
    & & & \multicolumn{2}{p{0.3\linewidth}}{
        \cellcolor[rgb]{0.9,0.9,0.9}{
            \makecell[{{p{\linewidth}}}]{
                \texttt{\tiny{[GM$|$GM]}}
                \texttt{end game} \\
            }
        }
    }
    & & \\ \\

\end{supertabular}
}

\end{document}
