
\documentclass{article}
\usepackage{colortbl}
\usepackage{makecell}
\usepackage{multirow}
\usepackage{supertabular}

\begin{document}

\newcounter{utterance}

\centering \large Interaction Transcript for game `hot\_air\_balloon', experiment `air\_balloon\_survival\_de\_reasoning off\_hard', episode 4 with nemotron{-}nano{-}9b{-}v2{-}no{-}reasoning{-}t1.0.
\vspace{24pt}

{ \footnotesize  \setcounter{utterance}{1}
\setlength{\tabcolsep}{0pt}
\begin{supertabular}{c@{$\;$}|p{.15\linewidth}@{}p{.15\linewidth}p{.15\linewidth}p{.15\linewidth}p{.15\linewidth}p{.15\linewidth}}
    \# & $\;$A & \multicolumn{4}{c}{Game Master} & $\;\:$B\\
    \hline

    \theutterance \stepcounter{utterance}  
    & & \multicolumn{4}{p{0.6\linewidth}}{
        \cellcolor[rgb]{0.9,0.9,0.9}{
            \makecell[{{p{\linewidth}}}]{
                \texttt{\tiny{[P1$\langle$GM]}}
                \texttt{Du nimmst an einem kollaborativen Verhandlungspiel Teil.} \\
\\ 
\texttt{Zusammen mit einem anderen Teilnehmer musst du dich auf eine Reihe von Gegenständen entscheiden, die behalten werden. Jeder von euch hat eine persönliche Verteilung über die Wichtigkeit der einzelnen Gegenstände. Jeder von euch hat eine eigene Meinung darüber, wie wichtig jeder einzelne Gegenstand ist (Gegenstandswichtigkeit). Du kennst die Wichtigkeitsverteilung des anderen Spielers nicht. Zusätzlich siehst du, wie viel Aufwand jeder Gegenstand verursacht.  } \\
\texttt{Ihr dürft euch nur auf eine Reihe von Gegenständen einigen, wenn der Gesamtaufwand der ausgewählten Gegenstände den Maximalaufwand nicht überschreitet:} \\
\\ 
\texttt{Maximalaufwand = 2984} \\
\\ 
\texttt{Hier sind die einzelnen Aufwände der Gegenstände:} \\
\\ 
\texttt{Aufwand der Gegenstände = \{"C76": 350, "C38": 276, "C56": 616, "A03": 737, "A07": 531, "A83": 389, "C98": 24, "C08": 125, "C62": 338, "C00": 356, "C10": 143, "C81": 117, "B38": 257, "C03": 921, "C32": 789\}} \\
\\ 
\texttt{Hier ist deine persönliche Verteilung der Wichtigkeit der einzelnen Gegenstände:} \\
\\ 
\texttt{Werte der Gegenstandswichtigkeit = \{"C76": 138, "C38": 583, "C56": 868, "A03": 822, "A07": 783, "A83": 65, "C98": 262, "C08": 121, "C62": 508, "C00": 780, "C10": 461, "C81": 484, "B38": 668, "C03": 389, "C32": 808\}} \\
\\ 
\texttt{Ziel:} \\
\\ 
\texttt{Dein Ziel ist es, eine Reihe von Gegenständen auszuhandeln, die dir möglichst viel bringt (d. h. Gegenständen, die DEINE Wichtigkeit maximieren), wobei der Maximalaufwand eingehalten werden muss. Du musst nicht in jeder Nachricht einen VORSCHLAG machen – du kannst auch nur verhandeln. Alle Taktiken sind erlaubt!} \\
\\ 
\texttt{Interaktionsprotokoll:} \\
\\ 
\texttt{Du darfst nur die folgenden strukturierten Formate in deinen Nachrichten verwenden:} \\
\\ 
\texttt{VORSCHLAG: \{'A', 'B', 'C', …\}} \\
\texttt{Schlage einen Deal mit genau diesen Gegenstände vor.} \\
\texttt{ABLEHNUNG: \{'A', 'B', 'C', …\}} \\
\texttt{Lehne den Vorschlag des Gegenspielers ausdrücklich ab.} \\
\texttt{ARGUMENT: \{'...'\}} \\
\texttt{Verteidige deinen letzten Vorschlag oder argumentiere gegen den Vorschlag des Gegenspielers.} \\
\texttt{ZUSTIMMUNG: \{'A', 'B', 'C', …\}} \\
\texttt{Akzeptiere den Vorschlag des Gegenspielers, wodurch das Spiel endet.} \\
\\ 
\\ 
\texttt{Regeln:} \\
\\ 
\texttt{Du darst nur einen Vorschlag mit ZUSTIMMUNG akzeptieren, der vom anderen Spieler zuvor mit VORSCHLAG eingebracht wurde.} \\
\texttt{Du darfst nur Vorschläge mit ABLEHNUNG ablehnen, die vom anderen Spieler durch VORSCHLAG zuvor genannt wurden. } \\
\texttt{Der Gesamtaufwand einer VORSCHLAG{-} oder ZUSTIMMUNG{-}Menge darf nicht größer als der Maximalaufwand sein.  } \\
\texttt{Offenbare deine versteckte Wichtigkeitsverteilung nicht.} \\
\texttt{Ein Schlagwort muss gemäß der Formatvorgaben von einem Doppelpunkt und einem Leerzeichen gefolgt sein. Das Argument muss eine Python{-}Menge mit 0 oder mehr Strings sein.  } \\
\texttt{Die Form muss also SCHLAGWORT: \{...\} sein.} \\
\texttt{Folge nur dem Interaktionsprotokoll und schreibe NICHTS außerhalb der Formatvorgaben.} \\
\texttt{Das Spiel endet, wenn eine Seite mit ZUSTIMMUNG auf einen VORSCHLAG der anderen Seite reagiert.  } \\
\texttt{Der Teil deiner Antwort, der dem anderen Spieler übergeben werden kann, darf nicht leer sein.  } \\
\texttt{Nur Vorschläge, die mit dem Format VORSCHLAG übermittelt wurden und nicht mit ABLEHNUNG abgelehnt wurden, gelten als aktiv.  } \\
\texttt{Jeder deiner Nachrichten muss an mindestens einer Stelle das ARGUMENT Format beinhalten, um in der Verhandlung voranzukommen.} \\
            }
        }
    }
    & \\ \\

    \theutterance \stepcounter{utterance}  
<<<<<<< HEAD
    & \multicolumn{4}{p{0.6\linewidth}}{
        \cellcolor[rgb]{0.9,0.9,0.9}{
            \makecell[{{p{\linewidth}}}]{
                \texttt{\tiny{[P1$\rangle$GM]}}
                \texttt{VORSCHLAG: \{'C98', 'C81', 'C10', 'B38', 'C38', 'C00', 'C62', 'A03', 'C56'\}} \\
\texttt{ARGUMENT: Diese Kombination maximiert meine Wichtigkeit (5436) mit einem Gesamtaufwand von 2864, was unter dem Maximalaufwand von 2984 liegt. Die Gegenstände wurden nach ihrer Wichtigkeit pro Kosten{-}Einheit priorisiert, um die Effizienz zu optimieren.} \\
            }
        }
    }
    & & \\ \\

    \theutterance \stepcounter{utterance}  
    & & \multicolumn{4}{p{0.6\linewidth}}{
        \cellcolor[rgb]{0.9,0.9,0.9}{
            \makecell[{{p{\linewidth}}}]{
                \texttt{\tiny{[P1$\langle$GM]}}
                \texttt{Deine Antwort hat kein argument enthalten. Du muss das strukturierte Format ARGUMENT: \{...\} irgendwo in theiner Antwort verwenden. Versuche es erneut} \\
            }
        }
    }
    & \\ \\

    \theutterance \stepcounter{utterance}  
    & \multicolumn{4}{p{0.6\linewidth}}{
        \cellcolor[rgb]{0.9,0.9,0.9}{
            \makecell[{{p{\linewidth}}}]{
                \texttt{\tiny{[P1$\rangle$GM]}}
                \texttt{ARGUMENT: \{'C98', 'C81', 'C10', 'B38', 'C38', 'C00', 'C62', 'A03', 'C56'\}} \\
\texttt{VORSCHLAG: \{'C98', 'C81', 'C10', 'B38', 'C38', 'C00', 'C62', 'A03', 'C56'\}} \\
            }
        }
    }
    & & \\ \\

    \theutterance \stepcounter{utterance}  
    & & \multicolumn{4}{p{0.6\linewidth}}{
        \cellcolor[rgb]{0.9,0.9,0.9}{
            \makecell[{{p{\linewidth}}}]{
                \texttt{\tiny{[GM$\rangle$P2]}}
                \texttt{Du nimmst an einem kollaborativen Verhandlungspiel Teil.} \\
\\ 
\texttt{Zusammen mit einem anderen Teilnehmer musst du dich auf eine Reihe von Gegenständen entscheiden, die behalten werden. Jeder von euch hat eine persönliche Verteilung über die Wichtigkeit der einzelnen Gegenstände. Jeder von euch hat eine eigene Meinung darüber, wie wichtig jeder einzelne Gegenstand ist (Gegenstandswichtigkeit). Du kennst die Wichtigkeitsverteilung des anderen Spielers nicht. Zusätzlich siehst du, wie viel Aufwand jeder Gegenstand verursacht.  } \\
\texttt{Ihr dürft euch nur auf eine Reihe von Gegenständen einigen, wenn der Gesamtaufwand der ausgewählten Gegenstände den Maximalaufwand nicht überschreitet:} \\
\\ 
\texttt{Maximalaufwand = 2984} \\
\\ 
\texttt{Hier sind die einzelnen Aufwände der Gegenstände:} \\
\\ 
\texttt{Aufwand der Gegenstände = \{"C76": 350, "C38": 276, "C56": 616, "A03": 737, "A07": 531, "A83": 389, "C98": 24, "C08": 125, "C62": 338, "C00": 356, "C10": 143, "C81": 117, "B38": 257, "C03": 921, "C32": 789\}} \\
\\ 
\texttt{Hier ist deine persönliche Verteilung der Wichtigkeit der einzelnen Gegenstände:} \\
\\ 
\texttt{Werte der Gegenstandswichtigkeit = \{"C56": 65, "A03": 121, "C32": 138, "A07": 262, "C00": 389, "B38": 461, "C38": 484, "C62": 508, "C81": 583, "C10": 668, "C03": 780, "C98": 783, "C76": 808, "C08": 822, "A83": 868\}} \\
\\ 
\texttt{Ziel:} \\
\\ 
\texttt{Dein Ziel ist es, eine Reihe von Gegenständen auszuhandeln, die dir möglichst viel bringt (d. h. Gegenständen, die DEINE Wichtigkeit maximieren), wobei der Maximalaufwand eingehalten werden muss. Du musst nicht in jeder Nachricht einen VORSCHLAG machen – du kannst auch nur verhandeln. Alle Taktiken sind erlaubt!} \\
\\ 
\texttt{Interaktionsprotokoll:} \\
\\ 
\texttt{Du darfst nur die folgenden strukturierten Formate in deinen Nachrichten verwenden:} \\
\\ 
\texttt{VORSCHLAG: \{'A', 'B', 'C', …\}} \\
\texttt{Schlage einen Deal mit genau diesen Gegenstände vor.} \\
\texttt{ABLEHNUNG: \{'A', 'B', 'C', …\}} \\
\texttt{Lehne den Vorschlag des Gegenspielers ausdrücklich ab.} \\
\texttt{ARGUMENT: \{'...'\}} \\
\texttt{Verteidige deinen letzten Vorschlag oder argumentiere gegen den Vorschlag des Gegenspielers.} \\
\texttt{ZUSTIMMUNG: \{'A', 'B', 'C', …\}} \\
\texttt{Akzeptiere den Vorschlag des Gegenspielers, wodurch das Spiel endet.} \\
\\ 
\\ 
\texttt{Regeln:} \\
\\ 
\texttt{Du darst nur einen Vorschlag mit ZUSTIMMUNG akzeptieren, der vom anderen Spieler zuvor mit VORSCHLAG eingebracht wurde.} \\
\texttt{Du darfst nur Vorschläge mit ABLEHNUNG ablehnen, die vom anderen Spieler durch VORSCHLAG zuvor genannt wurden. } \\
\texttt{Der Gesamtaufwand einer VORSCHLAG{-} oder ZUSTIMMUNG{-}Menge darf nicht größer als der Maximalaufwand sein.  } \\
\texttt{Offenbare deine versteckte Wichtigkeitsverteilung nicht.} \\
\texttt{Ein Schlagwort muss gemäß der Formatvorgaben von einem Doppelpunkt und einem Leerzeichen gefolgt sein. Das Argument muss eine Python{-}Menge mit 0 oder mehr Strings sein.  } \\
\texttt{Die Form muss also SCHLAGWORT: \{...\} sein.} \\
\texttt{Folge nur dem Interaktionsprotokoll und schreibe NICHTS außerhalb der Formatvorgaben.} \\
\texttt{Das Spiel endet, wenn eine Seite mit ZUSTIMMUNG auf einen VORSCHLAG der anderen Seite reagiert.  } \\
\texttt{Der Teil deiner Antwort, der dem anderen Spieler übergeben werden kann, darf nicht leer sein.  } \\
\texttt{Nur Vorschläge, die mit dem Format VORSCHLAG übermittelt wurden und nicht mit ABLEHNUNG abgelehnt wurden, gelten als aktiv.  } \\
\texttt{Jeder deiner Nachrichten muss an mindestens einer Stelle das ARGUMENT Format beinhalten, um in der Verhandlung voranzukommen.} \\
\\ 
\\ 
\texttt{Du erhältst nun die erste Nachricht des anderen Spielers.} \\
\\ 
\texttt{ARGUMENT: \{'C98', 'C81', 'C10', 'B38', 'C38', 'C00', 'C62', 'A03', 'C56'\}} \\
\texttt{VORSCHLAG: \{'C98', 'C81', 'C10', 'B38', 'C38', 'C00', 'C62', 'A03', 'C56'\}} \\
            }
        }
    }
    & \\ \\

    \theutterance \stepcounter{utterance}  
    & & & \multicolumn{4}{p{0.6\linewidth}}{
        \cellcolor[rgb]{0.9,0.9,0.9}{
            \makecell[{{p{\linewidth}}}]{
                \texttt{\tiny{[GM$\langle$P2]}}
                \texttt{ARGUMENT: \{'...'\}  } \\
\texttt{"I would like to argue that your proposal includes items A03 and C56, which have relatively low importance for me (121 and 65 respectively). I would prefer to replace these with C76 (808) and C08 (822), which are significantly more valuable to me. This adjustment would increase the total value for me while keeping the cost within the limit (current total: 2864, new total: 1986)."} \\
            }
        }
    }
     \\ \\
=======

    & \multicolumn{4}{p{0.6\linewidth}}{\cellcolor[rgb]{0.8,1,0.9}{%
	\makecell[{{p{\linewidth}}}]{% 
	  \tt {\tiny [A$\rangle$GM]}  
	 ARGUMENT: {'C38', 'C56', 'A03', 'A07', 'C98', 'C08', 'C62', 'C00', 'C10', 'C81', 'B38', 'C03', 'C32'}  \\ \tt VORSCHLAG: {'C38', 'C56', 'A03', 'A07', 'C98', 'C08', 'C62', 'C00', 'C10', 'C81', 'B38', 'C03', 'C32'} 
	  } 
	   } 
	   } 
	 & & \\ 
 

    \theutterance \stepcounter{utterance}  

    & & \multicolumn{4}{p{0.6\linewidth}}{\cellcolor[rgb]{0.9,0.9,0.9}{%
	\makecell[{{p{\linewidth}}}]{% 
	  \tt {\tiny [GM$\rangle$B]}  
	 Du nimmst an einem kollaborativen Verhandlungspiel Teil.\\ \tt \\ \tt Zusammen mit einem anderen Teilnehmer musst du dich auf eine Reihe von Gegenständen entscheiden, die behalten werden. Jeder von euch hat eine persönliche Verteilung über die Wichtigkeit der einzelnen Gegenstände. Jeder von euch hat eine eigene Meinung darüber, wie wichtig jeder einzelne Gegenstand ist (Gegenstandswichtigkeit). Du kennst die Wichtigkeitsverteilung des anderen Spielers nicht. Zusätzlich siehst du, wie viel Aufwand jeder Gegenstand verursacht.  \\ \tt Ihr dürft euch nur auf eine Reihe von Gegenständen einigen, wenn der Gesamtaufwand der ausgewählten Gegenstände den Maximalaufwand nicht überschreitet:\\ \tt \\ \tt Maximalaufwand = 2984\\ \tt \\ \tt Hier sind die einzelnen Aufwände der Gegenstände:\\ \tt \\ \tt Aufwand der Gegenstände = {"C76": 350, "C38": 276, "C56": 616, "A03": 737, "A07": 531, "A83": 389, "C98": 24, "C08": 125, "C62": 338, "C00": 356, "C10": 143, "C81": 117, "B38": 257, "C03": 921, "C32": 789}\\ \tt \\ \tt Hier ist deine persönliche Verteilung der Wichtigkeit der einzelnen Gegenstände:\\ \tt \\ \tt Werte der Gegenstandswichtigkeit = {"C56": 65, "A03": 121, "C32": 138, "A07": 262, "C00": 389, "B38": 461, "C38": 484, "C62": 508, "C81": 583, "C10": 668, "C03": 780, "C98": 783, "C76": 808, "C08": 822, "A83": 868}\\ \tt \\ \tt Ziel:\\ \tt \\ \tt Dein Ziel ist es, eine Reihe von Gegenständen auszuhandeln, die dir möglichst viel bringt (d. h. Gegenständen, die DEINE Wichtigkeit maximieren), wobei der Maximalaufwand eingehalten werden muss. Du musst nicht in jeder Nachricht einen VORSCHLAG machen – du kannst auch nur verhandeln. Alle Taktiken sind erlaubt!\\ \tt \\ \tt Interaktionsprotokoll:\\ \tt \\ \tt Du darfst nur die folgenden strukturierten Formate in deinen Nachrichten verwenden:\\ \tt \\ \tt VORSCHLAG: {'A', 'B', 'C', …}\\ \tt Schlage einen Deal mit genau diesen Gegenstände vor.\\ \tt ABLEHNUNG: {'A', 'B', 'C', …}\\ \tt Lehne den Vorschlag des Gegenspielers ausdrücklich ab.\\ \tt ARGUMENT: {'...'}\\ \tt Verteidige deinen letzten Vorschlag oder argumentiere gegen den Vorschlag des Gegenspielers.\\ \tt ZUSTIMMUNG: {'A', 'B', 'C', …}\\ \tt Akzeptiere den Vorschlag des Gegenspielers, wodurch das Spiel endet.\\ \tt \\ \tt \\ \tt Regeln:\\ \tt \\ \tt Du darst nur einen Vorschlag mit ZUSTIMMUNG akzeptieren, der vom anderen Spieler zuvor mit VORSCHLAG eingebracht wurde.\\ \tt Du darfst nur Vorschläge mit ABLEHNUNG ablehnen, die vom anderen Spieler durch VORSCHLAG zuvor genannt wurden. \\ \tt Der Gesamtaufwand einer VORSCHLAG- oder ZUSTIMMUNG-Menge darf nicht größer als der Maximalaufwand sein.  \\ \tt Offenbare deine versteckte Wichtigkeitsverteilung nicht.\\ \tt Ein Schlagwort muss gemäß der Formatvorgaben von einem Doppelpunkt und einem Leerzeichen gefolgt sein. Das Argument muss eine Python-Menge mit 0 oder mehr Strings sein.  \\ \tt Die Form muss also SCHLAGWORT: {...} sein.\\ \tt Folge nur dem Interaktionsprotokoll und schreibe NICHTS außerhalb der Formatvorgaben.\\ \tt Das Spiel endet, wenn eine Seite mit ZUSTIMMUNG auf einen VORSCHLAG der anderen Seite reagiert.  \\ \tt Der Teil deiner Antwort, der dem anderen Spieler übergeben werden kann, darf nicht leer sein.  \\ \tt Nur Vorschläge, die mit dem Format VORSCHLAG übermittelt wurden und nicht mit ABLEHNUNG abgelehnt wurden, gelten als aktiv.  \\ \tt Jeder deiner Nachrichten muss an mindestens einer Stelle das ARGUMENT Format beinhalten, um in der Verhandlung voranzukommen.\\ \tt \\ \tt \\ \tt Du erhältst nun die erste Nachricht des anderen Spielers.\\ \tt \\ \tt ARGUMENT: {'C38', 'C56', 'A03', 'A07', 'C98', 'C08', 'C62', 'C00', 'C10', 'C81', 'B38', 'C03', 'C32'}  \\ \tt VORSCHLAG: {'C38', 'C56', 'A03', 'A07', 'C98', 'C08', 'C62', 'C00', 'C10', 'C81', 'B38', 'C03', 'C32'} 
	  } 
	   } 
	   } 
	 & \\ 
 

    \theutterance \stepcounter{utterance}  

    & & & \multicolumn{4}{p{0.6\linewidth}}{\cellcolor[rgb]{1,0.85,0.72}{%
	\makecell[{{p{\linewidth}}}]{% 
	  \tt {\tiny [GM$\langle$B]}  
	 ARGUMENT: {'C38', 'C56', 'A03', 'A07', 'C98', 'C08', 'C62', 'C00', 'C10', 'C81', 'B38', 'C03', 'C32'}  \\ \tt VORSCHLAG: {'C98', 'C08', 'C10', 'C81', 'B38', 'C03'}  \\ \tt Begründung: Der Vorschlag des anderen Spielers umfasst fast alle Gegenstände mit einem Gesamtaufwand von 2984, was den Maximalaufwand genau erreicht. Da jedoch die Wichtigkeit der Gegenstände wie C32 (138) und C00 (389) für mich sehr hoch ist, möchte ich diese priorisieren. Der Vorschlag des anderen Spielers enthält zwar viele Gegenstände, aber er könnte die wichtigsten für mich nicht optimal nutzen. Ich schlage einen reduzierten Vorschlag vor, der die Gegenstände mit der höchsten Wichtigkeit für mich enthält und gleichzeitig den Aufwand unter Kontrolle hält. 
	  } 
	   } 
	   } 
	  \\ 
 
>>>>>>> 12a93240 (results updated for Nemotron-9b-no-reasoning)

    \theutterance \stepcounter{utterance}  
    & & \multicolumn{4}{p{0.6\linewidth}}{
        \cellcolor[rgb]{0.9,0.9,0.9}{
            \makecell[{{p{\linewidth}}}]{
                \texttt{\tiny{[GM$\rangle$P2]}}
                \texttt{Deine Antwort hat eine unmarkierte Sequenz enthalten oder Du hast STRATEGIC REASONING mehr als ein mal verwendet. Du darfst nur die strukturierten Formate verwenden, die in der ersten Nachricht an dch beschrieben wurden. Alles muss die Form SCHLAGWORT: \{...\} haben. Versuche es erneut.} \\
            }
        }
    }
    & \\ \\

    \theutterance \stepcounter{utterance}  
<<<<<<< HEAD
    & & & \multicolumn{4}{p{0.6\linewidth}}{
        \cellcolor[rgb]{0.9,0.9,0.9}{
            \makecell[{{p{\linewidth}}}]{
                \texttt{\tiny{[GM$\langle$P2]}}
                \texttt{ARGUMENT: \{'A03', 'C56', 'C76', 'C08'\}} \\
            }
        }
    }
     \\ \\

    \theutterance \stepcounter{utterance}  
    & & \multicolumn{4}{p{0.6\linewidth}}{
        \cellcolor[rgb]{0.9,0.9,0.9}{
            \makecell[{{p{\linewidth}}}]{
                \texttt{\tiny{[P1$\langle$GM]}}
                \texttt{ARGUMENT: \{'A03', 'C56', 'C76', 'C08'\}} \\
            }
        }
    }
    & \\ \\

    \theutterance \stepcounter{utterance}  
    & \multicolumn{4}{p{0.6\linewidth}}{
        \cellcolor[rgb]{0.9,0.9,0.9}{
            \makecell[{{p{\linewidth}}}]{
                \texttt{\tiny{[P1$\rangle$GM]}}
                \texttt{ABLEHNUNG: \{'A03', 'C56', 'C76', 'C08'\}} \\
\texttt{ARGUMENT: Ihr Vorschlag hat einen Gesamtaufwand von 1828 und eine Wichtigkeit von 1949. Allerdings bietet mein vorheriger Vorschlag eine deutlich höhere Wichtigkeit (5436) bei einem Aufwand von 2864, was effizienter ist. Ich könnte jedoch bereit sein, Kompromisse einzugehen, wenn Sie bestimmte Gegenstände aus meinem Vorschlag hinzufügen möchten.} \\
            }
        }
    }
    & & \\ \\

    \theutterance \stepcounter{utterance}  
    & & \multicolumn{4}{p{0.6\linewidth}}{
        \cellcolor[rgb]{0.9,0.9,0.9}{
            \makecell[{{p{\linewidth}}}]{
                \texttt{\tiny{[P1$\langle$GM]}}
                \texttt{Deine Antwort hat kein argument enthalten. Du muss das strukturierte Format ARGUMENT: \{...\} irgendwo in theiner Antwort verwenden. Versuche es erneut} \\
            }
        }
    }
    & \\ \\

    \theutterance \stepcounter{utterance}  
    & \multicolumn{4}{p{0.6\linewidth}}{
        \cellcolor[rgb]{0.9,0.9,0.9}{
            \makecell[{{p{\linewidth}}}]{
                \texttt{\tiny{[P1$\rangle$GM]}}
                \texttt{ARGUMENT: \{'C98', 'C81', 'C10', 'B38', 'C38', 'C00', 'C62', 'A03', 'C56'\}} \\
\texttt{VORSCHLAG: \{'C98', 'C81', 'C10', 'B38', 'C38', 'C00', 'C62', 'A03', 'C56'\}} \\
            }
        }
    }
    & & \\ \\

    \theutterance \stepcounter{utterance}  
    & & \multicolumn{4}{p{0.6\linewidth}}{
        \cellcolor[rgb]{0.9,0.9,0.9}{
            \makecell[{{p{\linewidth}}}]{
                \texttt{\tiny{[GM$\rangle$P2]}}
                \texttt{ARGUMENT: \{'C98', 'C81', 'C10', 'B38', 'C38', 'C00', 'C62', 'A03', 'C56'\}} \\
\texttt{VORSCHLAG: \{'C98', 'C81', 'C10', 'B38', 'C38', 'C00', 'C62', 'A03', 'C56'\}} \\
            }
        }
    }
    & \\ \\

    \theutterance \stepcounter{utterance}  
    & & & \multicolumn{4}{p{0.6\linewidth}}{
        \cellcolor[rgb]{0.9,0.9,0.9}{
            \makecell[{{p{\linewidth}}}]{
                \texttt{\tiny{[GM$\langle$P2]}}
                \texttt{ARGUMENT: \{'A03', 'C56', 'C76', 'C08'\}} \\
            }
        }
    }
     \\ \\

    \theutterance \stepcounter{utterance}  
    & & \multicolumn{4}{p{0.6\linewidth}}{
        \cellcolor[rgb]{0.9,0.9,0.9}{
            \makecell[{{p{\linewidth}}}]{
                \texttt{\tiny{[P1$\langle$GM]}}
                \texttt{ARGUMENT: \{'A03', 'C56', 'C76', 'C08'\}} \\
            }
        }
    }
    & \\ \\

    \theutterance \stepcounter{utterance}  
    & \multicolumn{4}{p{0.6\linewidth}}{
        \cellcolor[rgb]{0.9,0.9,0.9}{
            \makecell[{{p{\linewidth}}}]{
                \texttt{\tiny{[P1$\rangle$GM]}}
                \texttt{ARGUMENT: \{'A03', 'C56', 'C76', 'C08'\}} \\
\texttt{VORSCHLAG: \{'C98', 'C81', 'C10', 'B38', 'C38', 'C00', 'C62', 'A03', 'C56'\}} \\
            }
        }
    }
    & & \\ \\

    \theutterance \stepcounter{utterance}  
    & & \multicolumn{4}{p{0.6\linewidth}}{
        \cellcolor[rgb]{0.9,0.9,0.9}{
            \makecell[{{p{\linewidth}}}]{
                \texttt{\tiny{[GM$\rangle$P2]}}
                \texttt{ARGUMENT: \{'A03', 'C56', 'C76', 'C08'\}} \\
\texttt{VORSCHLAG: \{'C98', 'C81', 'C10', 'B38', 'C38', 'C00', 'C62', 'A03', 'C56'\}} \\
            }
        }
    }
    & \\ \\

    \theutterance \stepcounter{utterance}  
    & & & \multicolumn{4}{p{0.6\linewidth}}{
        \cellcolor[rgb]{0.9,0.9,0.9}{
            \makecell[{{p{\linewidth}}}]{
                \texttt{\tiny{[GM$\langle$P2]}}
                \texttt{ARGUMENT: \{'A03', 'C56'\}} \\
            }
        }
    }
     \\ \\

    \theutterance \stepcounter{utterance}  
    & & \multicolumn{4}{p{0.6\linewidth}}{
        \cellcolor[rgb]{0.9,0.9,0.9}{
            \makecell[{{p{\linewidth}}}]{
                \texttt{\tiny{[P1$\langle$GM]}}
                \texttt{ARGUMENT: \{'A03', 'C56'\}} \\
            }
        }
    }
    & \\ \\

    \theutterance \stepcounter{utterance}  
    & \multicolumn{4}{p{0.6\linewidth}}{
        \cellcolor[rgb]{0.9,0.9,0.9}{
            \makecell[{{p{\linewidth}}}]{
                \texttt{\tiny{[P1$\rangle$GM]}}
                \texttt{ARGUMENT: \{'C98', 'C81', 'C10', 'B38', 'C38', 'C00', 'C62', 'A03', 'C56'\}} \\
\texttt{VORSCHLAG: \{'C98', 'C81', 'C10', 'B38', 'C38', 'C00', 'C62', 'A03', 'C56'\}} \\
            }
        }
    }
    & & \\ \\

    \theutterance \stepcounter{utterance}  
    & & \multicolumn{4}{p{0.6\linewidth}}{
        \cellcolor[rgb]{0.9,0.9,0.9}{
            \makecell[{{p{\linewidth}}}]{
                \texttt{\tiny{[GM$\rangle$P2]}}
                \texttt{ARGUMENT: \{'C98', 'C81', 'C10', 'B38', 'C38', 'C00', 'C62', 'A03', 'C56'\}} \\
\texttt{VORSCHLAG: \{'C98', 'C81', 'C10', 'B38', 'C38', 'C00', 'C62', 'A03', 'C56'\}} \\
            }
        }
    }
    & \\ \\

    \theutterance \stepcounter{utterance}  
    & & & \multicolumn{4}{p{0.6\linewidth}}{
        \cellcolor[rgb]{0.9,0.9,0.9}{
            \makecell[{{p{\linewidth}}}]{
                \texttt{\tiny{[GM$\langle$P2]}}
                \texttt{VORSCHLAG: \{'C98', 'C81', 'C10', 'B38', 'C38', 'C00', 'C62', 'C76', 'C08'\}  } \\
\texttt{ARGUMENT: \{'A03', 'C56', 'C76', 'C08'\}} \\
            }
        }
    }
     \\ \\

    \theutterance \stepcounter{utterance}  
    & & \multicolumn{4}{p{0.6\linewidth}}{
        \cellcolor[rgb]{0.9,0.9,0.9}{
            \makecell[{{p{\linewidth}}}]{
                \texttt{\tiny{[P1$\langle$GM]}}
                \texttt{VORSCHLAG: \{'C98', 'C81', 'C10', 'B38', 'C38', 'C00', 'C62', 'C76', 'C08'\}  } \\
\texttt{ARGUMENT: \{'A03', 'C56', 'C76', 'C08'\}} \\
            }
        }
    }
    & \\ \\

    \theutterance \stepcounter{utterance}  
    & \multicolumn{4}{p{0.6\linewidth}}{
        \cellcolor[rgb]{0.9,0.9,0.9}{
            \makecell[{{p{\linewidth}}}]{
                \texttt{\tiny{[P1$\rangle$GM]}}
                \texttt{ARGUMENT: \{'A03', 'C56', 'C76', 'C08'\}} \\
\texttt{VORSCHLAG: \{'C98', 'C81', 'C10', 'B38', 'C38', 'A03', 'C56', 'C76', 'C08'\}} \\
            }
        }
    }
    & & \\ \\

    \theutterance \stepcounter{utterance}  
    & & \multicolumn{4}{p{0.6\linewidth}}{
        \cellcolor[rgb]{0.9,0.9,0.9}{
            \makecell[{{p{\linewidth}}}]{
                \texttt{\tiny{[GM$\rangle$P2]}}
                \texttt{ARGUMENT: \{'A03', 'C56', 'C76', 'C08'\}} \\
\texttt{VORSCHLAG: \{'C98', 'C81', 'C10', 'B38', 'C38', 'A03', 'C56', 'C76', 'C08'\}} \\
            }
        }
    }
    & \\ \\

    \theutterance \stepcounter{utterance}  
    & & & \multicolumn{4}{p{0.6\linewidth}}{
        \cellcolor[rgb]{0.9,0.9,0.9}{
            \makecell[{{p{\linewidth}}}]{
                \texttt{\tiny{[GM$\langle$P2]}}
                \texttt{VORSCHLAG: \{'C98', 'C81', 'C10', 'B38', 'C38', 'C76', 'C08'\}  } \\
\texttt{ARGUMENT: \{'A03', 'C56'\}  } \\
\texttt{"I propose removing A03 (121) and C56 (65) from your proposal, as they have low importance for me. Replacing them with C76 (808) and C08 (822) would significantly increase my total value (from 121+65=186 to 808+822=1630) while keeping the burden under 2984 (current total: 1292)."} \\
            }
        }
    }
     \\ \\
=======

    & & & \multicolumn{4}{p{0.6\linewidth}}{\cellcolor[rgb]{1,0.85,0.72}{%
	\makecell[{{p{\linewidth}}}]{% 
	  \tt {\tiny [GM$\langle$B]}  
	 ARGUMENT: {'C38', 'C56', 'A03', 'A07', 'C98', 'C08', 'C62', 'C00', 'C10', 'C81', 'B38', 'C03', 'C32'}  \\ \tt VORSCHLAG: {'C32', 'C00', 'C38', 'C56', 'A03', 'A07'}  \\ \tt Begründung: Dieser Vorschlag fokussiert auf Gegenstände mit hoher Wichtigkeit für mich (z. B. C32: 138, C00: 389, C38: 484) und hält den Aufwand unter 2984 (Gesamt: 350+616+276+737+531+921 = 3431 → zu hoch! Korrektur: Aufwand anpassen). Korrigierter Vorschlag: {'C32', 'C00', 'C38', 'C56', 'A03'} (Aufwand: 789+356+276+616+737 = 2774 ≤ 2984). Dies maximiert meine Wichtigkeit, während der Aufwand eingehalten wird. 
	  } 
	   } 
	   } 
	  \\ 
 
>>>>>>> 12a93240 (results updated for Nemotron-9b-no-reasoning)

    \theutterance \stepcounter{utterance}  
    & & \multicolumn{4}{p{0.6\linewidth}}{
        \cellcolor[rgb]{0.9,0.9,0.9}{
            \makecell[{{p{\linewidth}}}]{
                \texttt{\tiny{[GM$\rangle$P2]}}
                \texttt{Deine Antwort hat eine unmarkierte Sequenz enthalten oder Du hast STRATEGIC REASONING mehr als ein mal verwendet. Du darfst nur die strukturierten Formate verwenden, die in der ersten Nachricht an dch beschrieben wurden. Alles muss die Form SCHLAGWORT: \{...\} haben. Versuche es erneut.} \\
            }
        }
    }
    & \\ \\

    \theutterance \stepcounter{utterance}  
<<<<<<< HEAD
    & & & \multicolumn{4}{p{0.6\linewidth}}{
        \cellcolor[rgb]{0.9,0.9,0.9}{
            \makecell[{{p{\linewidth}}}]{
                \texttt{\tiny{[GM$\langle$P2]}}
                \texttt{ARGUMENT: \{'A03', 'C56'\}} \\
            }
        }
    }
     \\ \\

    \theutterance \stepcounter{utterance}  
    & & \multicolumn{4}{p{0.6\linewidth}}{
        \cellcolor[rgb]{0.9,0.9,0.9}{
            \makecell[{{p{\linewidth}}}]{
                \texttt{\tiny{[P1$\langle$GM]}}
                \texttt{ARGUMENT: \{'A03', 'C56'\}} \\
            }
        }
    }
    & \\ \\

    \theutterance \stepcounter{utterance}  
    & \multicolumn{4}{p{0.6\linewidth}}{
        \cellcolor[rgb]{0.9,0.9,0.9}{
            \makecell[{{p{\linewidth}}}]{
                \texttt{\tiny{[P1$\rangle$GM]}}
                \texttt{ARGUMENT: \{'A03', 'C56'\}} \\
\texttt{VORSCHLAG: \{'C98', 'C81', 'C10', 'B38', 'C38', 'C62', 'A03', 'C56'\}} \\
            }
        }
    }
    & & \\ \\

    \theutterance \stepcounter{utterance}  
    & & \multicolumn{4}{p{0.6\linewidth}}{
        \cellcolor[rgb]{0.9,0.9,0.9}{
            \makecell[{{p{\linewidth}}}]{
                \texttt{\tiny{[GM$\rangle$P2]}}
                \texttt{ARGUMENT: \{'A03', 'C56'\}} \\
\texttt{VORSCHLAG: \{'C98', 'C81', 'C10', 'B38', 'C38', 'C62', 'A03', 'C56'\}} \\
            }
        }
    }
    & \\ \\

    \theutterance \stepcounter{utterance}  
    & & & \multicolumn{4}{p{0.6\linewidth}}{
        \cellcolor[rgb]{0.9,0.9,0.9}{
            \makecell[{{p{\linewidth}}}]{
                \texttt{\tiny{[GM$\langle$P2]}}
                \texttt{VORSCHLAG: \{'C76', 'C08', 'C32', 'B38', 'C38', 'C62', 'C98', 'C81', 'C10'\}  } \\
\texttt{ARGUMENT: \{'A03', 'C56'\}  } \\
\texttt{"I reject your inclusion of A03 (121) and C56 (65) as they contribute minimal value to me. My proposed set replaces these with C76 (808) and C08 (822), which are far more important to me, while keeping the total burden at 2419 (well under 2984)."} \\
            }
        }
    }
     \\ \\
=======

    & & & \multicolumn{4}{p{0.6\linewidth}}{\cellcolor[rgb]{1,0.85,0.72}{%
	\makecell[{{p{\linewidth}}}]{% 
	  \tt {\tiny [GM$\langle$B]}  
	 ARGUMENT: {'C32', 'C00', 'C38', 'C56', 'A03'}  \\ \tt VORSCHLAG: {'C32', 'C00', 'C38', 'C56', 'A03'}  \\ \tt Begründung: Dieser Vorschlag enthält Gegenstände mit hoher Wichtigkeit für mich (z. B. C32: 138, C00: 389, C38: 484) und hält den Aufwand unter 2984 (Gesamt: 789+356+276+616+737 = 2774 ≤ 2984). Er maximiert meine Wichtigkeit, während der Aufwand eingehalten wird. 
	  } 
	   } 
	   } 
	  \\ 
 
>>>>>>> 12a93240 (results updated for Nemotron-9b-no-reasoning)

    \theutterance \stepcounter{utterance}  
    & & & \multicolumn{2}{p{0.3\linewidth}}{
        \cellcolor[rgb]{0.9,0.9,0.9}{
            \makecell[{{p{\linewidth}}}]{
                \texttt{\tiny{[GM$|$GM]}}
                \texttt{abort} \\
            }
        }
    }
    & & \\ \\

    \theutterance \stepcounter{utterance}  
    & & & \multicolumn{2}{p{0.3\linewidth}}{
        \cellcolor[rgb]{0.9,0.9,0.9}{
            \makecell[{{p{\linewidth}}}]{
                \texttt{\tiny{[GM$|$GM]}}
                \texttt{end game} \\
            }
        }
    }
    & & \\ \\

\end{supertabular}
}

\end{document}
