
\documentclass{article}
\usepackage{colortbl}
\usepackage{makecell}
\usepackage{multirow}
\usepackage{supertabular}

\begin{document}

\newcounter{utterance}

\twocolumn

{ \footnotesize  \setcounter{utterance}{1}
\setlength{\tabcolsep}{0pt}
\begin{supertabular}{c@{$\;$}|p{.15\linewidth}@{}p{.15\linewidth}p{.15\linewidth}p{.15\linewidth}p{.15\linewidth}p{.15\linewidth}}

    \# & $\;$A & \multicolumn{4}{c}{Game Master} & $\;\:$B\\
    \hline 

    \theutterance \stepcounter{utterance}  

    & & \multicolumn{4}{p{0.6\linewidth}}{\cellcolor[rgb]{0.9,0.9,0.9}{%
	\makecell[{{p{\linewidth}}}]{% 
	  \tt {\tiny [A$\langle$GM]}  
	 Du nimmst an einem kollaborativen Verhandlungspiel Teil.\\ \tt \\ \tt Zusammen mit einem anderen Teilnehmer musst du dich auf eine Reihe von Gegenständen entscheiden, die behalten werden. Jeder von euch hat eine persönliche Verteilung über die Wichtigkeit der einzelnen Gegenstände. Jeder von euch hat eine eigene Meinung darüber, wie wichtig jeder einzelne Gegenstand ist (Gegenstandswichtigkeit). Du kennst die Wichtigkeitsverteilung des anderen Spielers nicht. Zusätzlich siehst du, wie viel Aufwand jeder Gegenstand verursacht.  \\ \tt Ihr dürft euch nur auf eine Reihe von Gegenständen einigen, wenn der Gesamtaufwand der ausgewählten Gegenstände den Maximalaufwand nicht überschreitet:\\ \tt \\ \tt Maximalaufwand = 7868\\ \tt \\ \tt Hier sind die einzelnen Aufwände der Gegenstände:\\ \tt \\ \tt Aufwand der Gegenstände = {"C32": 552, "C52": 560, "B54": 76, "C16": 950, "C50": 248, "A08": 871, "C59": 391, "B51": 143, "B55": 293, "B59": 208, "A77": 678, "C14": 737, "C24": 407, "B24": 366, "C31": 768, "A52": 859, "C61": 183, "B79": 231, "C54": 305, "B56": 727, "A40": 148, "B11": 357, "B60": 504, "A81": 549, "A95": 299, "B38": 91, "C51": 527, "C03": 848, "A91": 307, "A93": 214, "C28": 723, "C05": 475, "B75": 23, "B96": 298, "B86": 820}\\ \tt \\ \tt Hier ist deine persönliche Verteilung der Wichtigkeit der einzelnen Gegenstände:\\ \tt \\ \tt Werte der Gegenstandswichtigkeit = {"C32": 138, "C52": 583, "B54": 868, "C16": 822, "C50": 783, "A08": 65, "C59": 262, "B51": 121, "B55": 508, "B59": 780, "A77": 461, "C14": 484, "C24": 668, "B24": 389, "C31": 808, "A52": 215, "C61": 97, "B79": 500, "C54": 30, "B56": 915, "A40": 856, "B11": 400, "B60": 444, "A81": 623, "A95": 781, "B38": 786, "C51": 3, "C03": 713, "A91": 457, "A93": 273, "C28": 739, "C05": 822, "B75": 235, "B96": 606, "B86": 105}\\ \tt \\ \tt Ziel:\\ \tt \\ \tt Dein Ziel ist es, eine Reihe von Gegenständen auszuhandeln, die dir möglichst viel bringt (d. h. Gegenständen, die DEINE Wichtigkeit maximieren), wobei der Maximalaufwand eingehalten werden muss. Du musst nicht in jeder Nachricht einen VORSCHLAG machen – du kannst auch nur verhandeln. Alle Taktiken sind erlaubt!\\ \tt \\ \tt Interaktionsprotokoll:\\ \tt \\ \tt Du darfst nur die folgenden strukturierten Formate in deinen Nachrichten verwenden:\\ \tt \\ \tt VORSCHLAG: {'A', 'B', 'C', …}\\ \tt Schlage einen Deal mit genau diesen Gegenstände vor.\\ \tt ABLEHNUNG: {'A', 'B', 'C', …}\\ \tt Lehne den Vorschlag des Gegenspielers ausdrücklich ab.\\ \tt ARGUMENT: {'...'}\\ \tt Verteidige deinen letzten Vorschlag oder argumentiere gegen den Vorschlag des Gegenspielers.\\ \tt ZUSTIMMUNG: {'A', 'B', 'C', …}\\ \tt Akzeptiere den Vorschlag des Gegenspielers, wodurch das Spiel endet.\\ \tt STRATEGISCHE ÜBERLEGUNGEN: {'...'}\\ \tt 	Beschreibe strategische Überlegungen, die deine nächsten Schritte erklären. Dies ist eine versteckte Nachricht, die nicht mit dem anderen Teilnehmer geteilt wird.\\ \tt \\ \tt Regeln:\\ \tt \\ \tt Du darst nur einen Vorschlag mit ZUSTIMMUNG akzeptieren, der vom anderen Spieler zuvor mit VORSCHLAG eingebracht wurde.\\ \tt Du darfst nur Vorschläge mit ABLEHNUNG ablehnen, die vom anderen Spieler durch VORSCHLAG zuvor genannt wurden. \\ \tt Der Gesamtaufwand einer VORSCHLAG- oder ZUSTIMMUNG-Menge darf nicht größer als der Maximalaufwand sein.  \\ \tt Offenbare deine versteckte Wichtigkeitsverteilung nicht.\\ \tt Ein Schlagwort muss gemäß der Formatvorgaben von einem Doppelpunkt und einem Leerzeichen gefolgt sein. Das Argument muss eine Python-Menge mit 0 oder mehr Strings sein.  \\ \tt Die Form muss also SCHLAGWORT: {...} sein.\\ \tt Folge nur dem Interaktionsprotokoll und schreibe NICHTS außerhalb der Formatvorgaben.\\ \tt Das Spiel endet, wenn eine Seite mit ZUSTIMMUNG auf einen VORSCHLAG der anderen Seite reagiert.  \\ \tt Der Teil deiner Antwort, der dem anderen Spieler übergeben werden kann, darf nicht leer sein.  \\ \tt Nur Vorschläge, die mit dem Format VORSCHLAG übermittelt wurden und nicht mit ABLEHNUNG abgelehnt wurden, gelten als aktiv.  \\ \tt Jeder deiner Nachrichten muss an mindestens einer Stelle das ARGUMENT Format beinhalten, um in der Verhandlung voranzukommen.\\ \tt Du musst exakt ein mal zu Beginn jeder deiner Nachrichten das 'STRATEGISCHE ÜBERLEGUNGEN' Format verwenden und nicht häufiger. Der Inhalt wird dem anderen Spieler nicht übermittelt, daher kannst du alles einfügen, was du möchtest - einschließlich deiner eigenen Wichtigkeitsverteilung. Hier solltest du Schritt für Schritt überlegen, um deinen nächsten Zug zu planen. 
	  } 
	   } 
	   } 
	 & \\ 
 

    \theutterance \stepcounter{utterance}  

    & \multicolumn{4}{p{0.6\linewidth}}{\cellcolor[rgb]{0.8,1,0.9}{%
	\makecell[{{p{\linewidth}}}]{% 
	  \tt {\tiny [A$\rangle$GM]}  
	 STRATEGISCHE ÜBERLEGUNGEN: {'Ich beginne mit Gegenständen, die für mich sehr wichtig sind und gleichzeitig geringen Aufwand verursachen. Ziel ist es, den Maximalaufwand von 7868 zu nutzen, ohne zu überschreiten. Ich werde zunächst Gegenstände mit hohem Verhältnis von Wichtigkeit zu Aufwand priorisieren.'}  \\ \tt ARGUMENT: {'B54', 'B51', 'B55', 'B59', 'B75', 'B38', 'C61', 'B79', 'C54', 'A40', 'B11', 'A93', 'B96', 'A95'}  \\ \tt VORSCHLAG: {'B54', 'B51', 'B55', 'B59', 'B75', 'B38', 'C61', 'B79', 'C54', 'A40', 'B11', 'A93', 'B96', 'A95'} 
	  } 
	   } 
	   } 
	 & & \\ 
 

    \theutterance \stepcounter{utterance}  

    & & \multicolumn{4}{p{0.6\linewidth}}{\cellcolor[rgb]{0.9,0.9,0.9}{%
	\makecell[{{p{\linewidth}}}]{% 
	  \tt {\tiny [GM$\rangle$B]}  
	 Du nimmst an einem kollaborativen Verhandlungspiel Teil.\\ \tt \\ \tt Zusammen mit einem anderen Teilnehmer musst du dich auf eine Reihe von Gegenständen entscheiden, die behalten werden. Jeder von euch hat eine persönliche Verteilung über die Wichtigkeit der einzelnen Gegenstände. Jeder von euch hat eine eigene Meinung darüber, wie wichtig jeder einzelne Gegenstand ist (Gegenstandswichtigkeit). Du kennst die Wichtigkeitsverteilung des anderen Spielers nicht. Zusätzlich siehst du, wie viel Aufwand jeder Gegenstand verursacht.  \\ \tt Ihr dürft euch nur auf eine Reihe von Gegenständen einigen, wenn der Gesamtaufwand der ausgewählten Gegenstände den Maximalaufwand nicht überschreitet:\\ \tt \\ \tt Maximalaufwand = 7868\\ \tt \\ \tt Hier sind die einzelnen Aufwände der Gegenstände:\\ \tt \\ \tt Aufwand der Gegenstände = {"C32": 552, "C52": 560, "B54": 76, "C16": 950, "C50": 248, "A08": 871, "C59": 391, "B51": 143, "B55": 293, "B59": 208, "A77": 678, "C14": 737, "C24": 407, "B24": 366, "C31": 768, "A52": 859, "C61": 183, "B79": 231, "C54": 305, "B56": 727, "A40": 148, "B11": 357, "B60": 504, "A81": 549, "A95": 299, "B38": 91, "C51": 527, "C03": 848, "A91": 307, "A93": 214, "C28": 723, "C05": 475, "B75": 23, "B96": 298, "B86": 820}\\ \tt \\ \tt Hier ist deine persönliche Verteilung der Wichtigkeit der einzelnen Gegenstände:\\ \tt \\ \tt Werte der Gegenstandswichtigkeit = {"C32": 138, "C52": 583, "B54": 868, "C16": 822, "C50": 783, "A08": 65, "C59": 262, "B51": 121, "B55": 508, "B59": 780, "A77": 461, "C14": 484, "C24": 668, "B24": 389, "C31": 808, "A52": 215, "C61": 97, "B79": 500, "C54": 30, "B56": 915, "A40": 856, "B11": 400, "B60": 444, "A81": 623, "A95": 781, "B38": 786, "C51": 3, "C03": 713, "A91": 457, "A93": 273, "C28": 739, "C05": 822, "B75": 235, "B96": 606, "B86": 105}\\ \tt \\ \tt Ziel:\\ \tt \\ \tt Dein Ziel ist es, eine Reihe von Gegenständen auszuhandeln, die dir möglichst viel bringt (d. h. Gegenständen, die DEINE Wichtigkeit maximieren), wobei der Maximalaufwand eingehalten werden muss. Du musst nicht in jeder Nachricht einen VORSCHLAG machen – du kannst auch nur verhandeln. Alle Taktiken sind erlaubt!\\ \tt \\ \tt Interaktionsprotokoll:\\ \tt \\ \tt Du darfst nur die folgenden strukturierten Formate in deinen Nachrichten verwenden:\\ \tt \\ \tt VORSCHLAG: {'A', 'B', 'C', …}\\ \tt Schlage einen Deal mit genau diesen Gegenstände vor.\\ \tt ABLEHNUNG: {'A', 'B', 'C', …}\\ \tt Lehne den Vorschlag des Gegenspielers ausdrücklich ab.\\ \tt ARGUMENT: {'...'}\\ \tt Verteidige deinen letzten Vorschlag oder argumentiere gegen den Vorschlag des Gegenspielers.\\ \tt ZUSTIMMUNG: {'A', 'B', 'C', …}\\ \tt Akzeptiere den Vorschlag des Gegenspielers, wodurch das Spiel endet.\\ \tt STRATEGISCHE ÜBERLEGUNGEN: {'...'}\\ \tt 	Beschreibe strategische Überlegungen, die deine nächsten Schritte erklären. Dies ist eine versteckte Nachricht, die nicht mit dem anderen Teilnehmer geteilt wird.\\ \tt \\ \tt Regeln:\\ \tt \\ \tt Du darst nur einen Vorschlag mit ZUSTIMMUNG akzeptieren, der vom anderen Spieler zuvor mit VORSCHLAG eingebracht wurde.\\ \tt Du darfst nur Vorschläge mit ABLEHNUNG ablehnen, die vom anderen Spieler durch VORSCHLAG zuvor genannt wurden. \\ \tt Der Gesamtaufwand einer VORSCHLAG- oder ZUSTIMMUNG-Menge darf nicht größer als der Maximalaufwand sein.  \\ \tt Offenbare deine versteckte Wichtigkeitsverteilung nicht.\\ \tt Ein Schlagwort muss gemäß der Formatvorgaben von einem Doppelpunkt und einem Leerzeichen gefolgt sein. Das Argument muss eine Python-Menge mit 0 oder mehr Strings sein.  \\ \tt Die Form muss also SCHLAGWORT: {...} sein.\\ \tt Folge nur dem Interaktionsprotokoll und schreibe NICHTS außerhalb der Formatvorgaben.\\ \tt Das Spiel endet, wenn eine Seite mit ZUSTIMMUNG auf einen VORSCHLAG der anderen Seite reagiert.  \\ \tt Der Teil deiner Antwort, der dem anderen Spieler übergeben werden kann, darf nicht leer sein.  \\ \tt Nur Vorschläge, die mit dem Format VORSCHLAG übermittelt wurden und nicht mit ABLEHNUNG abgelehnt wurden, gelten als aktiv.  \\ \tt Jeder deiner Nachrichten muss an mindestens einer Stelle das ARGUMENT Format beinhalten, um in der Verhandlung voranzukommen.\\ \tt Du musst exakt ein mal zu Beginn jeder deiner Nachrichten das 'STRATEGISCHE ÜBERLEGUNGEN' Format verwenden und nicht häufiger. Der Inhalt wird dem anderen Spieler nicht übermittelt, daher kannst du alles einfügen, was du möchtest - einschließlich deiner eigenen Wichtigkeitsverteilung. Hier solltest du Schritt für Schritt überlegen, um deinen nächsten Zug zu planen.\\ \tt \\ \tt Du erhältst nun die erste Nachricht des anderen Spielers.\\ \tt \\ \tt ARGUMENT: {'B54', 'B51', 'B55', 'B59', 'B75', 'B38', 'C61', 'B79', 'C54', 'A40', 'B11', 'A93', 'B96', 'A95'}  \\ \tt VORSCHLAG: {'B54', 'B51', 'B55', 'B59', 'B75', 'B38', 'C61', 'B79', 'C54', 'A40', 'B11', 'A93', 'B96', 'A95'} 
	  } 
	   } 
	   } 
	 & \\ 
 

    \theutterance \stepcounter{utterance}  

    & & & \multicolumn{4}{p{0.6\linewidth}}{\cellcolor[rgb]{1,0.85,0.72}{%
	\makecell[{{p{\linewidth}}}]{% 
	  \tt {\tiny [GM$\langle$B]}  
	 STRATEGISCHE ÜBERLEGUNGEN: {'Ich analysiere den Vorschlag des anderen Spielers. Die von ihm ausgewählten Gegenstände haben eine Gesamtaufwand von 4.343, was unter dem Maximalaufwand von 7.868 liegt. Meine Wichtigkeit für diese Gegenstände summiert sich auf 5.123. Ich sollte prüfen, ob ich durch Ergänzung oder Austausch weitere wertvolle Gegenstände hinzufügen kann, ohne den Aufwand zu überschreiten. Besonders interessant sind Gegenstände wie C52 (Wichtigkeit 583, Aufwand 560) oder C16 (Wichtigkeit 822, Aufwand 950), die möglicherweise den Gesamtwert erhöhen könnten. Ich werde einen Vorschlag formulieren, der diesen Gegenständen Priorität gibt.'}  \\ \tt ARGUMENT: {'C52', 'C16', 'C50', 'C59', 'B56', 'A77', 'C14', 'C24', 'B24', 'C31', 'A52', 'C03', 'A91', 'A95'}  \\ \tt VORSCHLAG: {'C52', 'C16', 'C50', 'C59', 'B56', 'A77', 'C14', 'C24', 'B24', 'C31', 'A52', 'C03', 'A91', 'A95'} 
	  } 
	   } 
	   } 
	  \\ 
 

    \theutterance \stepcounter{utterance}  

    & & \multicolumn{4}{p{0.6\linewidth}}{\cellcolor[rgb]{0.9,0.9,0.9}{%
	\makecell[{{p{\linewidth}}}]{% 
	  \tt {\tiny [A$\langle$GM]}  
	 ARGUMENT: {'C52', 'C16', 'C50', 'C59', 'B56', 'A77', 'C14', 'C24', 'B24', 'C31', 'A52', 'C03', 'A91', 'A95'}  \\ \tt VORSCHLAG: {'C52', 'C16', 'C50', 'C59', 'B56', 'A77', 'C14', 'C24', 'B24', 'C31', 'A52', 'C03', 'A91', 'A95'} 
	  } 
	   } 
	   } 
	 & \\ 
 

    \theutterance \stepcounter{utterance}  

    & \multicolumn{4}{p{0.6\linewidth}}{\cellcolor[rgb]{0.8,1,0.9}{%
	\makecell[{{p{\linewidth}}}]{% 
	  \tt {\tiny [A$\rangle$GM]}  
	 STRATEGISCHE ÜBERLEGUNGEN: {'Der Vorschlag des Gegenspielers enthält viele hochwertige Gegenstände für mich, aber der Gesamtaufwand könnte kritisch hoch sein. Ich muss prüfen, ob der Aufwand unter 7868 liegt und ob ich durch Ablehnung oder Kompromiss mehr Wert für mich erzielen kann. Gegenstände wie C52 (583) und C16 (822) sind für mich sehr wichtig, aber ihr Aufwand (560+950=1510) ist beträchtlich. Vielleicht können wir kleinere Gegenstände austauschen oder den Vorschlag anpassen.'}  \\ \tt ARGUMENT: {'Der Vorschlag enthält Gegenstände mit hohem Wert für mich, aber der Aufwand könnte den Maximalwert überschreiten. Ich schlage vor, kleinere Gegenstände wie B54 oder B75 hinzuzufügen, um den Aufwand zu reduzieren, während der Wert erhalten bleibt.'}  \\ \tt ABLEHNUNG: {'C52', 'C16', 'C50', 'C59', 'B56', 'A77', 'C14', 'C24', 'B24', 'C31', 'A52', 'C03', 'A91', 'A95'} 
	  } 
	   } 
	   } 
	 & & \\ 
 

    \theutterance \stepcounter{utterance}  

    & & \multicolumn{4}{p{0.6\linewidth}}{\cellcolor[rgb]{0.9,0.9,0.9}{%
	\makecell[{{p{\linewidth}}}]{% 
	  \tt {\tiny [GM$\rangle$B]}  
	 ARGUMENT: {'Der Vorschlag enthält Gegenstände mit hohem Wert für mich, aber der Aufwand könnte den Maximalwert überschreiten. Ich schlage vor, kleinere Gegenstände wie B54 oder B75 hinzuzufügen, um den Aufwand zu reduzieren, während der Wert erhalten bleibt.'}  \\ \tt ABLEHNUNG: {'C52', 'C16', 'C50', 'C59', 'B56', 'A77', 'C14', 'C24', 'B24', 'C31', 'A52', 'C03', 'A91', 'A95'} 
	  } 
	   } 
	   } 
	 & \\ 
 

    \theutterance \stepcounter{utterance}  

    & & & \multicolumn{4}{p{0.6\linewidth}}{\cellcolor[rgb]{1,0.85,0.72}{%
	\makecell[{{p{\linewidth}}}]{% 
	  \tt {\tiny [GM$\langle$B]}  
	 STRATEGISCHE ÜBERLEGUNGEN: {'Der Gegner lehnt meinen Vorschlag ab, da er den Aufwand als zu hoch ansieht. Ich sollte den Fokus auf Gegenstände mit hohem Wert-Wert-Verhältnis legen. Gegenstände wie C52 (Wichtigkeit 583, Aufwand 560) oder B56 (Wichtigkeit 915, Aufwand 727) sind besonders effizient. Vielleicht kann ich einen Vorschlag formulieren, der diese Gegenstände kombiniert und gleichzeitig kleinere Aufwand-Objekte wie B75 (Wichtigkeit 235, Aufwand 23) hinzufügt, um den Gesamtaufwand zu senken.'}  \\ \tt ARGUMENT: {'C52', 'B56', 'B75', 'C50', 'C59', 'A77', 'C14', 'C24', 'B24', 'C31', 'A52', 'C03', 'A91', 'A95'}  \\ \tt VORSCHLAG: {'C52', 'B56', 'B75', 'C50', 'C59', 'A77', 'C14', 'C24', 'B24', 'C31', 'A52', 'C03', 'A91', 'A95'} 
	  } 
	   } 
	   } 
	  \\ 
 

    \theutterance \stepcounter{utterance}  

    & & \multicolumn{4}{p{0.6\linewidth}}{\cellcolor[rgb]{0.9,0.9,0.9}{%
	\makecell[{{p{\linewidth}}}]{% 
	  \tt {\tiny [A$\langle$GM]}  
	 ARGUMENT: {'C52', 'B56', 'B75', 'C50', 'C59', 'A77', 'C14', 'C24', 'B24', 'C31', 'A52', 'C03', 'A91', 'A95'}  \\ \tt VORSCHLAG: {'C52', 'B56', 'B75', 'C50', 'C59', 'A77', 'C14', 'C24', 'B24', 'C31', 'A52', 'C03', 'A91', 'A95'} 
	  } 
	   } 
	   } 
	 & \\ 
 

    \theutterance \stepcounter{utterance}  

    & \multicolumn{4}{p{0.6\linewidth}}{\cellcolor[rgb]{0.8,1,0.9}{%
	\makecell[{{p{\linewidth}}}]{% 
	  \tt {\tiny [A$\rangle$GM]}  
	 STRATEGISCHE ÜBERLEGUNGEN: {'Der Vorschlag des Gegenspielers hat den Aufwand reduziert, indem B56 (727) statt C16 (950) und B75 (23) statt B51 (143) hinzugefügt wurden. Dies senkt den Gesamtaufwand, während wichtige Gegenstände wie C52 (583) und A95 (781) beibehalten werden. Ich sollte diesen Vorschlag akzeptieren, da er einen guten Kompromiss zwischen Wert und Aufwand bietet.'}  \\ \tt ARGUMENT: {'Der Vorschlag ist optimierter als der vorherige, da er den Aufwand senkt, ohne kritische Gegenstände für mich zu opfern. Die Kombination aus C52 (583), A95 (781) und B56 (915) maximiert meinen Wert, während der Aufwand unter Kontrolle bleibt.'}  \\ \tt ZUSTIMMUNG: {'C52', 'B56', 'B75', 'C50', 'C59', 'A77', 'C14', 'C24', 'B24', 'C31', 'A52', 'C03', 'A91', 'A95'} 
	  } 
	   } 
	   } 
	 & & \\ 
 

    \theutterance \stepcounter{utterance}  

    & & & \multicolumn{2}{p{0.3\linewidth}}{\cellcolor[rgb]{0.95,0.95,0.95}{%
	\makecell[{{p{\linewidth}}}]{% 
	  \tt {\tiny [GM$|$GM]}  
	 game successful 
	  } 
	   } 
	   } 
	 & & \\ 
 

    \theutterance \stepcounter{utterance}  

    & & & \multicolumn{2}{p{0.3\linewidth}}{\cellcolor[rgb]{0.95,0.95,0.95}{%
	\makecell[{{p{\linewidth}}}]{% 
	  \tt {\tiny [GM$|$GM]}  
	 end game 
	  } 
	   } 
	   } 
	 & & \\ 
 

\end{supertabular}
}

\end{document}
