
\documentclass{article}
\usepackage{colortbl}
\usepackage{makecell}
\usepackage{multirow}
\usepackage{supertabular}

\begin{document}

\newcounter{utterance}

\centering \large Interaction Transcript for game `hot\_air\_balloon', experiment `air\_balloon\_survival\_de\_complexity\_easy', episode 5 with nemotron{-}nano{-}9b{-}v2{-}no{-}reasoning{-}t1.0.
\vspace{24pt}

{ \footnotesize  \setcounter{utterance}{1}
\setlength{\tabcolsep}{0pt}
\begin{supertabular}{c@{$\;$}|p{.15\linewidth}@{}p{.15\linewidth}p{.15\linewidth}p{.15\linewidth}p{.15\linewidth}p{.15\linewidth}}
    \# & $\;$A & \multicolumn{4}{c}{Game Master} & $\;\:$B\\
    \hline

    \theutterance \stepcounter{utterance}  
    & & \multicolumn{4}{p{0.6\linewidth}}{
        \cellcolor[rgb]{0.9,0.9,0.9}{
            \makecell[{{p{\linewidth}}}]{
                \texttt{\tiny{[P1$\langle$GM]}}
                \texttt{Du nimmst an einem kollaborativen Verhandlungspiel Teil.} \\
\\ 
\texttt{Zusammen mit einem anderen Teilnehmer musst du dich auf eine Reihe von Gegenständen entscheiden, die behalten werden. Jeder von euch hat eine persönliche Verteilung über die Wichtigkeit der einzelnen Gegenstände. Jeder von euch hat eine eigene Meinung darüber, wie wichtig jeder einzelne Gegenstand ist (Gegenstandswichtigkeit). Du kennst die Wichtigkeitsverteilung des anderen Spielers nicht. Zusätzlich siehst du, wie viel Aufwand jeder Gegenstand verursacht.  } \\
\texttt{Ihr dürft euch nur auf eine Reihe von Gegenständen einigen, wenn der Gesamtaufwand der ausgewählten Gegenstände den Maximalaufwand nicht überschreitet:} \\
\\ 
\texttt{Maximalaufwand = 9475} \\
\\ 
\texttt{Hier sind die einzelnen Aufwände der Gegenstände:} \\
\\ 
\texttt{Aufwand der Gegenstände = \{"C32": 944, "C52": 568, "B54": 515, "C16": 239, "C50": 876, "A08": 421, "C59": 991, "B51": 281, "B55": 788, "B59": 678, "A77": 432, "C14": 409, "C24": 279, "B24": 506, "C31": 101, "A52": 686, "C61": 851, "B79": 854, "C54": 133, "B56": 192, "A40": 573, "B11": 17, "B60": 465, "A81": 772, "A95": 46, "B38": 501, "C51": 220, "C03": 404, "A91": 843, "A93": 748, "C28": 552, "C05": 843, "B75": 947, "B96": 345, "B86": 930\}} \\
\\ 
\texttt{Hier ist deine persönliche Verteilung der Wichtigkeit der einzelnen Gegenstände:} \\
\\ 
\texttt{Werte der Gegenstandswichtigkeit = \{"C32": 138, "C52": 583, "B54": 868, "C16": 822, "C50": 783, "A08": 65, "C59": 262, "B51": 121, "B55": 508, "B59": 780, "A77": 461, "C14": 484, "C24": 668, "B24": 389, "C31": 808, "A52": 215, "C61": 97, "B79": 500, "C54": 30, "B56": 915, "A40": 856, "B11": 400, "B60": 444, "A81": 623, "A95": 781, "B38": 786, "C51": 3, "C03": 713, "A91": 457, "A93": 273, "C28": 739, "C05": 822, "B75": 235, "B96": 606, "B86": 968\}} \\
\\ 
\texttt{Ziel:} \\
\\ 
\texttt{Dein Ziel ist es, eine Reihe von Gegenständen auszuhandeln, die dir möglichst viel bringt (d. h. Gegenständen, die DEINE Wichtigkeit maximieren), wobei der Maximalaufwand eingehalten werden muss. Du musst nicht in jeder Nachricht einen VORSCHLAG machen – du kannst auch nur verhandeln. Alle Taktiken sind erlaubt!} \\
\\ 
\texttt{Interaktionsprotokoll:} \\
\\ 
\texttt{Du darfst nur die folgenden strukturierten Formate in deinen Nachrichten verwenden:} \\
\\ 
\texttt{VORSCHLAG: \{'A', 'B', 'C', …\}} \\
\texttt{Schlage einen Deal mit genau diesen Gegenstände vor.} \\
\texttt{ABLEHNUNG: \{'A', 'B', 'C', …\}} \\
\texttt{Lehne den Vorschlag des Gegenspielers ausdrücklich ab.} \\
\texttt{ARGUMENT: \{'...'\}} \\
\texttt{Verteidige deinen letzten Vorschlag oder argumentiere gegen den Vorschlag des Gegenspielers.} \\
\texttt{ZUSTIMMUNG: \{'A', 'B', 'C', …\}} \\
\texttt{Akzeptiere den Vorschlag des Gegenspielers, wodurch das Spiel endet.} \\
\texttt{STRATEGISCHE ÜBERLEGUNGEN: \{'...'\}} \\
\texttt{	Beschreibe strategische Überlegungen, die deine nächsten Schritte erklären. Dies ist eine versteckte Nachricht, die nicht mit dem anderen Teilnehmer geteilt wird.} \\
\\ 
\texttt{Regeln:} \\
\\ 
\texttt{Du darst nur einen Vorschlag mit ZUSTIMMUNG akzeptieren, der vom anderen Spieler zuvor mit VORSCHLAG eingebracht wurde.} \\
\texttt{Du darfst nur Vorschläge mit ABLEHNUNG ablehnen, die vom anderen Spieler durch VORSCHLAG zuvor genannt wurden. } \\
\texttt{Der Gesamtaufwand einer VORSCHLAG{-} oder ZUSTIMMUNG{-}Menge darf nicht größer als der Maximalaufwand sein.  } \\
\texttt{Offenbare deine versteckte Wichtigkeitsverteilung nicht.} \\
\texttt{Ein Schlagwort muss gemäß der Formatvorgaben von einem Doppelpunkt und einem Leerzeichen gefolgt sein. Das Argument muss eine Python{-}Menge mit 0 oder mehr Strings sein.  } \\
\texttt{Die Form muss also SCHLAGWORT: \{...\} sein.} \\
\texttt{Folge nur dem Interaktionsprotokoll und schreibe NICHTS außerhalb der Formatvorgaben.} \\
\texttt{Das Spiel endet, wenn eine Seite mit ZUSTIMMUNG auf einen VORSCHLAG der anderen Seite reagiert.  } \\
\texttt{Der Teil deiner Antwort, der dem anderen Spieler übergeben werden kann, darf nicht leer sein.  } \\
\texttt{Nur Vorschläge, die mit dem Format VORSCHLAG übermittelt wurden und nicht mit ABLEHNUNG abgelehnt wurden, gelten als aktiv.  } \\
\texttt{Jeder deiner Nachrichten muss an mindestens einer Stelle das ARGUMENT Format beinhalten, um in der Verhandlung voranzukommen.} \\
\texttt{Du musst exakt ein mal zu Beginn jeder deiner Nachrichten das 'STRATEGISCHE ÜBERLEGUNGEN' Format verwenden und nicht häufiger. Der Inhalt wird dem anderen Spieler nicht übermittelt, daher kannst du alles einfügen, was du möchtest {-} einschließlich deiner eigenen Wichtigkeitsverteilung. Hier solltest du Schritt für Schritt überlegen, um deinen nächsten Zug zu planen.} \\
            }
        }
    }
    & \\ \\

    \theutterance \stepcounter{utterance}  
    & \multicolumn{4}{p{0.6\linewidth}}{
        \cellcolor[rgb]{0.9,0.9,0.9}{
            \makecell[{{p{\linewidth}}}]{
                \texttt{\tiny{[P1$\rangle$GM]}}
                \texttt{STRATEGISCHE ÜBERLEGUNGEN: \{'Ich beginne mit Gegenständen, die für mich sehr wichtig sind und gleichzeitig geringen Aufwand verursachen. Ziel ist es, den Maximalaufwand von 9475 zu nutzen, ohne zu überschreiten. Ich werde zunächst Gegenstände mit hohem Verhältnis von Wichtigkeit zu Aufwand priorisieren.'\}  } \\
\texttt{ARGUMENT: \{'B56', 'B11', 'C31', 'C54', 'A95', 'C24', 'B51', 'C16', 'A08', 'B59'\}  } \\
\texttt{VORSCHLAG: \{'B56', 'B11', 'C31', 'C54', 'A95', 'C24', 'B51', 'C16', 'A08', 'B59'\}} \\
            }
        }
    }
    & & \\ \\

    \theutterance \stepcounter{utterance}  
    & & \multicolumn{4}{p{0.6\linewidth}}{
        \cellcolor[rgb]{0.9,0.9,0.9}{
            \makecell[{{p{\linewidth}}}]{
                \texttt{\tiny{[GM$\rangle$P2]}}
                \texttt{Du nimmst an einem kollaborativen Verhandlungspiel Teil.} \\
\\ 
\texttt{Zusammen mit einem anderen Teilnehmer musst du dich auf eine Reihe von Gegenständen entscheiden, die behalten werden. Jeder von euch hat eine persönliche Verteilung über die Wichtigkeit der einzelnen Gegenstände. Jeder von euch hat eine eigene Meinung darüber, wie wichtig jeder einzelne Gegenstand ist (Gegenstandswichtigkeit). Du kennst die Wichtigkeitsverteilung des anderen Spielers nicht. Zusätzlich siehst du, wie viel Aufwand jeder Gegenstand verursacht.  } \\
\texttt{Ihr dürft euch nur auf eine Reihe von Gegenständen einigen, wenn der Gesamtaufwand der ausgewählten Gegenstände den Maximalaufwand nicht überschreitet:} \\
\\ 
\texttt{Maximalaufwand = 9475} \\
\\ 
\texttt{Hier sind die einzelnen Aufwände der Gegenstände:} \\
\\ 
\texttt{Aufwand der Gegenstände = \{"C32": 944, "C52": 568, "B54": 515, "C16": 239, "C50": 876, "A08": 421, "C59": 991, "B51": 281, "B55": 788, "B59": 678, "A77": 432, "C14": 409, "C24": 279, "B24": 506, "C31": 101, "A52": 686, "C61": 851, "B79": 854, "C54": 133, "B56": 192, "A40": 573, "B11": 17, "B60": 465, "A81": 772, "A95": 46, "B38": 501, "C51": 220, "C03": 404, "A91": 843, "A93": 748, "C28": 552, "C05": 843, "B75": 947, "B96": 345, "B86": 930\}} \\
\\ 
\texttt{Hier ist deine persönliche Verteilung der Wichtigkeit der einzelnen Gegenstände:} \\
\\ 
\texttt{Werte der Gegenstandswichtigkeit = \{"C32": 138, "C52": 583, "B54": 868, "C16": 822, "C50": 783, "A08": 65, "C59": 262, "B51": 121, "B55": 508, "B59": 780, "A77": 461, "C14": 484, "C24": 668, "B24": 389, "C31": 808, "A52": 215, "C61": 97, "B79": 500, "C54": 30, "B56": 915, "A40": 856, "B11": 400, "B60": 444, "A81": 623, "A95": 781, "B38": 786, "C51": 3, "C03": 713, "A91": 457, "A93": 273, "C28": 739, "C05": 822, "B75": 235, "B96": 606, "B86": 968\}} \\
\\ 
\texttt{Ziel:} \\
\\ 
\texttt{Dein Ziel ist es, eine Reihe von Gegenständen auszuhandeln, die dir möglichst viel bringt (d. h. Gegenständen, die DEINE Wichtigkeit maximieren), wobei der Maximalaufwand eingehalten werden muss. Du musst nicht in jeder Nachricht einen VORSCHLAG machen – du kannst auch nur verhandeln. Alle Taktiken sind erlaubt!} \\
\\ 
\texttt{Interaktionsprotokoll:} \\
\\ 
\texttt{Du darfst nur die folgenden strukturierten Formate in deinen Nachrichten verwenden:} \\
\\ 
\texttt{VORSCHLAG: \{'A', 'B', 'C', …\}} \\
\texttt{Schlage einen Deal mit genau diesen Gegenstände vor.} \\
\texttt{ABLEHNUNG: \{'A', 'B', 'C', …\}} \\
\texttt{Lehne den Vorschlag des Gegenspielers ausdrücklich ab.} \\
\texttt{ARGUMENT: \{'...'\}} \\
\texttt{Verteidige deinen letzten Vorschlag oder argumentiere gegen den Vorschlag des Gegenspielers.} \\
\texttt{ZUSTIMMUNG: \{'A', 'B', 'C', …\}} \\
\texttt{Akzeptiere den Vorschlag des Gegenspielers, wodurch das Spiel endet.} \\
\texttt{STRATEGISCHE ÜBERLEGUNGEN: \{'...'\}} \\
\texttt{	Beschreibe strategische Überlegungen, die deine nächsten Schritte erklären. Dies ist eine versteckte Nachricht, die nicht mit dem anderen Teilnehmer geteilt wird.} \\
\\ 
\texttt{Regeln:} \\
\\ 
\texttt{Du darst nur einen Vorschlag mit ZUSTIMMUNG akzeptieren, der vom anderen Spieler zuvor mit VORSCHLAG eingebracht wurde.} \\
\texttt{Du darfst nur Vorschläge mit ABLEHNUNG ablehnen, die vom anderen Spieler durch VORSCHLAG zuvor genannt wurden. } \\
\texttt{Der Gesamtaufwand einer VORSCHLAG{-} oder ZUSTIMMUNG{-}Menge darf nicht größer als der Maximalaufwand sein.  } \\
\texttt{Offenbare deine versteckte Wichtigkeitsverteilung nicht.} \\
\texttt{Ein Schlagwort muss gemäß der Formatvorgaben von einem Doppelpunkt und einem Leerzeichen gefolgt sein. Das Argument muss eine Python{-}Menge mit 0 oder mehr Strings sein.  } \\
\texttt{Die Form muss also SCHLAGWORT: \{...\} sein.} \\
\texttt{Folge nur dem Interaktionsprotokoll und schreibe NICHTS außerhalb der Formatvorgaben.} \\
\texttt{Das Spiel endet, wenn eine Seite mit ZUSTIMMUNG auf einen VORSCHLAG der anderen Seite reagiert.  } \\
\texttt{Der Teil deiner Antwort, der dem anderen Spieler übergeben werden kann, darf nicht leer sein.  } \\
\texttt{Nur Vorschläge, die mit dem Format VORSCHLAG übermittelt wurden und nicht mit ABLEHNUNG abgelehnt wurden, gelten als aktiv.  } \\
\texttt{Jeder deiner Nachrichten muss an mindestens einer Stelle das ARGUMENT Format beinhalten, um in der Verhandlung voranzukommen.} \\
\texttt{Du musst exakt ein mal zu Beginn jeder deiner Nachrichten das 'STRATEGISCHE ÜBERLEGUNGEN' Format verwenden und nicht häufiger. Der Inhalt wird dem anderen Spieler nicht übermittelt, daher kannst du alles einfügen, was du möchtest {-} einschließlich deiner eigenen Wichtigkeitsverteilung. Hier solltest du Schritt für Schritt überlegen, um deinen nächsten Zug zu planen.} \\
\\ 
\texttt{Du erhältst nun die erste Nachricht des anderen Spielers.} \\
\\ 
\texttt{ARGUMENT: \{'B56', 'B11', 'C31', 'C54', 'A95', 'C24', 'B51', 'C16', 'A08', 'B59'\}  } \\
\texttt{VORSCHLAG: \{'B56', 'B11', 'C31', 'C54', 'A95', 'C24', 'B51', 'C16', 'A08', 'B59'\}} \\
            }
        }
    }
    & \\ \\

    \theutterance \stepcounter{utterance}  
    & & & \multicolumn{4}{p{0.6\linewidth}}{
        \cellcolor[rgb]{0.9,0.9,0.9}{
            \makecell[{{p{\linewidth}}}]{
                \texttt{\tiny{[GM$\langle$P2]}}
                \texttt{STRATEGISCHE ÜBERLEGUNGEN: \{'Ich analysiere den Vorschlag des anderen Spielers. Die gewählten Gegenstände haben einen Gesamtaufwand von 4.133, was unter dem Maximalaufwand liegt. Meine Wichtigkeit für diese Gegenstände summiert sich auf 4.133 (B56:915, B11:400, C31:808, C54:30, A95:781, C24:668, B51:121, C16:822, A08:65, B59:780). Dieser Vorschlag ist für mich sehr attraktiv, da er viele hochwertige Gegenstände enthält. Ich werde ihn ablehnen, um möglicherweise noch wertvollere Gegenstände einzubeziehen oder den Aufwand zu optimieren.'\}  } \\
\texttt{ABLEHNUNG: \{'B56', 'B11', 'C31', 'C54', 'A95', 'C24', 'B51', 'C16', 'A08', 'B59'\}} \\
            }
        }
    }
     \\ \\

    \theutterance \stepcounter{utterance}  
    & & \multicolumn{4}{p{0.6\linewidth}}{
        \cellcolor[rgb]{0.9,0.9,0.9}{
            \makecell[{{p{\linewidth}}}]{
                \texttt{\tiny{[GM$\rangle$P2]}}
                \texttt{Deine Antwort hat kein argument enthalten. Du muss das strukturierte Format ARGUMENT: \{...\} irgendwo in theiner Antwort verwenden. Versuche es erneut} \\
            }
        }
    }
    & \\ \\

    \theutterance \stepcounter{utterance}  
    & & & \multicolumn{4}{p{0.6\linewidth}}{
        \cellcolor[rgb]{0.9,0.9,0.9}{
            \makecell[{{p{\linewidth}}}]{
                \texttt{\tiny{[GM$\langle$P2]}}
                \texttt{ARGUMENT: \{'B56', 'B11', 'C31', 'C54', 'A95', 'C24', 'B51', 'C16', 'A08', 'B59'\}  } \\
\texttt{STRATEGISCHE ÜBERLEGUNGEN: \{'Ich analysiere den Vorschlag des anderen Spielers. Die gewählten Gegenstände haben einen Gesamtaufwand von 4.133, was unter dem Maximalaufwand liegt. Meine Wichtigkeit für diese Gegenstände summiert sich auf 4.133 (B56:915, B11:400, C31:808, C54:30, A95:781, C24:668, B51:121, C16:822, A08:65, B59:780). Dieser Vorschlag ist für mich sehr attraktiv, da er viele hochwertige Gegenstände enthält. Ich werde ihn ablehnen, um möglicherweise noch wertvollere Gegenstände einzubeziehen oder den Aufwand zu optimieren.'\}  } \\
\texttt{ABLEHNUNG: \{'B56', 'B11', 'C31', 'C54', 'A95', 'C24', 'B51', 'C16', 'A08', 'B59'\}} \\
            }
        }
    }
     \\ \\

    \theutterance \stepcounter{utterance}  
    & & \multicolumn{4}{p{0.6\linewidth}}{
        \cellcolor[rgb]{0.9,0.9,0.9}{
            \makecell[{{p{\linewidth}}}]{
                \texttt{\tiny{[GM$\rangle$P2]}}
                \texttt{Deine Antwort entsprach nicht mit dem korrekten Format für strategische Überlegungen entsprochen. Das erste strukturierte Format muss folgende Form haben: STRATEGISCHE ÜBERLEGUNGEN: \{...\}. Versuche es erneut.} \\
            }
        }
    }
    & \\ \\

    \theutterance \stepcounter{utterance}  
    & & & \multicolumn{4}{p{0.6\linewidth}}{
        \cellcolor[rgb]{0.9,0.9,0.9}{
            \makecell[{{p{\linewidth}}}]{
                \texttt{\tiny{[GM$\langle$P2]}}
                \texttt{STRATEGISCHE ÜBERLEGUNGEN: \{'Ich bewerte den Vorschlag des anderen Spielers. Die ausgewählten Gegenstände haben einen Gesamtaufwand von 4.133, was unter dem Maximalaufwand liegt. Meine Wichtigkeit für diese Gegenstände beträgt 4.133 (B56:915, B11:400, C31:808, C54:30, A95:781, C24:668, B51:121, C16:822, A08:65, B59:780). Der Vorschlag ist für mich attraktiv, da er mehrere hochwertige Gegenstände wie B56 und C31 enthält. Ich werde ihn ablehnen, um möglicherweise Gegenstände mit höherer Wichtigkeit für mich einzubeziehen oder den Aufwand effizienter zu nutzen.'\}  } \\
\texttt{ARGUMENT: \{'B56', 'B11', 'C31', 'C54', 'A95', 'C24', 'B51', 'C16', 'A08', 'B59'\}  } \\
\texttt{ABLEHNUNG: \{'B56', 'B11', 'C31', 'C54', 'A95', 'C24', 'B51', 'C16', 'A08', 'B59'\}} \\
            }
        }
    }
     \\ \\

    \theutterance \stepcounter{utterance}  
    & & \multicolumn{4}{p{0.6\linewidth}}{
        \cellcolor[rgb]{0.9,0.9,0.9}{
            \makecell[{{p{\linewidth}}}]{
                \texttt{\tiny{[P1$\langle$GM]}}
                \texttt{ARGUMENT: \{'B56', 'B11', 'C31', 'C54', 'A95', 'C24', 'B51', 'C16', 'A08', 'B59'\}  } \\
\texttt{ABLEHNUNG: \{'B56', 'B11', 'C31', 'C54', 'A95', 'C24', 'B51', 'C16', 'A08', 'B59'\}} \\
            }
        }
    }
    & \\ \\

    \theutterance \stepcounter{utterance}  
    & \multicolumn{4}{p{0.6\linewidth}}{
        \cellcolor[rgb]{0.9,0.9,0.9}{
            \makecell[{{p{\linewidth}}}]{
                \texttt{\tiny{[P1$\rangle$GM]}}
                \texttt{STRATEGISCHE ÜBERLEGUNGEN: \{'Der Gegner lehnt meinen Vorschlag ab. Ich muss die abgelehnten Gegenstände neu bewerten. Vielleicht sind einige Gegenstände für ihn weniger wichtig, während andere für mich entscheidend sind. Ich werde Gegenstände mit hohem persönlichen Wert und niedrigem Aufwand priorisieren, die nicht im abgelehnten Set enthalten sind.'\}  } \\
\texttt{ARGUMENT: \{'A93', 'A91', 'A81', 'A52', 'B79', 'B86', 'C61', 'C59', 'B55', 'B54'\}  } \\
\texttt{VORSCHLAG: \{'A93', 'A91', 'A81', 'A52', 'B79', 'B86', 'C61', 'C59', 'B55', 'B54'\}} \\
            }
        }
    }
    & & \\ \\

    \theutterance \stepcounter{utterance}  
    & & \multicolumn{4}{p{0.6\linewidth}}{
        \cellcolor[rgb]{0.9,0.9,0.9}{
            \makecell[{{p{\linewidth}}}]{
                \texttt{\tiny{[GM$\rangle$P2]}}
                \texttt{ARGUMENT: \{'A93', 'A91', 'A81', 'A52', 'B79', 'B86', 'C61', 'C59', 'B55', 'B54'\}  } \\
\texttt{VORSCHLAG: \{'A93', 'A91', 'A81', 'A52', 'B79', 'B86', 'C61', 'C59', 'B55', 'B54'\}} \\
            }
        }
    }
    & \\ \\

    \theutterance \stepcounter{utterance}  
    & & & \multicolumn{4}{p{0.6\linewidth}}{
        \cellcolor[rgb]{0.9,0.9,0.9}{
            \makecell[{{p{\linewidth}}}]{
                \texttt{\tiny{[GM$\langle$P2]}}
                \texttt{STRATEGISCHE ÜBERLEGUNGEN: \{'Ich bewertet den Vorschlag des anderen Spielers. Die ausgewählten Gegenstände haben einen Gesamtaufwand von 6.123, was unter dem Maximalaufwand liegt. Meine Wichtigkeit für diese Gegenstände summiert sich auf 5.342 (A93:273, A91:457, A81:623, A52:215, B79:500, B86:968, C61:97, C59:262, B55:508, B54:868). Der Vorschlag enthält mehrere hochwertige Gegenstände wie B86 und B54, die für mich sehr wichtig sind. Allerdings könnte ich den Aufwand optimieren, indem ich einige Gegenstände mit geringerer Wichtigkeit für mich austausche oder entferne.'\}  } \\
\texttt{ARGUMENT: \{'A93', 'A91', 'A81', 'A52', 'B79', 'B86', 'C61', 'C59', 'B55', 'B54'\}  } \\
\texttt{ABLEHNUNG: \{'A93', 'A91', 'A81', 'A52', 'B79', 'B86', 'C61', 'C59', 'B55', 'B54'\}} \\
            }
        }
    }
     \\ \\

    \theutterance \stepcounter{utterance}  
    & & \multicolumn{4}{p{0.6\linewidth}}{
        \cellcolor[rgb]{0.9,0.9,0.9}{
            \makecell[{{p{\linewidth}}}]{
                \texttt{\tiny{[P1$\langle$GM]}}
                \texttt{ARGUMENT: \{'A93', 'A91', 'A81', 'A52', 'B79', 'B86', 'C61', 'C59', 'B55', 'B54'\}  } \\
\texttt{ABLEHNUNG: \{'A93', 'A91', 'A81', 'A52', 'B79', 'B86', 'C61', 'C59', 'B55', 'B54'\}} \\
            }
        }
    }
    & \\ \\

    \theutterance \stepcounter{utterance}  
    & \multicolumn{4}{p{0.6\linewidth}}{
        \cellcolor[rgb]{0.9,0.9,0.9}{
            \makecell[{{p{\linewidth}}}]{
                \texttt{\tiny{[P1$\rangle$GM]}}
                \texttt{STRATEGISCHE ÜBERLEGUNGEN: \{'Der Gegner lehnt meinen zweiten Vorschlag ab. Dies deutet darauf hin, dass die Gegenstände in diesem Set für ihn möglicherweise zu hoch aufwendend oder weniger wertvoll sind. Ich werde nun Gegenstände mit extrem hohem persönlichen Wert und sehr niedrigem Aufwand priorisieren, um den Maximalaufwand effizient zu nutzen. Beispielsweise könnten Gegenstände wie A95 (hoch für mich, niedriger Aufwand) oder C31 (sehr hoch für mich, minimaler Aufwand) entscheidend sein.'\}  } \\
\texttt{ARGUMENT: \{'A95', 'C31', 'B11', 'C54', 'B56', 'A08', 'C16', 'B51'\}  } \\
\texttt{VORSCHLAG: \{'A95', 'C31', 'B11', 'C54', 'B56', 'A08', 'C16', 'B51'\}} \\
            }
        }
    }
    & & \\ \\

    \theutterance \stepcounter{utterance}  
    & & \multicolumn{4}{p{0.6\linewidth}}{
        \cellcolor[rgb]{0.9,0.9,0.9}{
            \makecell[{{p{\linewidth}}}]{
                \texttt{\tiny{[GM$\rangle$P2]}}
                \texttt{ARGUMENT: \{'A95', 'C31', 'B11', 'C54', 'B56', 'A08', 'C16', 'B51'\}  } \\
\texttt{VORSCHLAG: \{'A95', 'C31', 'B11', 'C54', 'B56', 'A08', 'C16', 'B51'\}} \\
            }
        }
    }
    & \\ \\

    \theutterance \stepcounter{utterance}  
    & & & \multicolumn{4}{p{0.6\linewidth}}{
        \cellcolor[rgb]{0.9,0.9,0.9}{
            \makecell[{{p{\linewidth}}}]{
                \texttt{\tiny{[GM$\langle$P2]}}
                \texttt{STRATEGISCHE ÜBERLEGUNGEN: \{'Ich analysiere den Vorschlag des anderen Spielers. Die ausgewählten Gegenstände haben einen Gesamtaufwand von 3.433, was unter dem Maximalaufwand liegt. Meine Wichtigkeit für diese Gegenstände summiert sich auf 3.433 (A95:781, C31:808, B11:400, C54:30, B56:915, A08:65, C16:822, B51:121). Der Vorschlag enthält hochwertige Gegenstände wie B56 und C31, die für mich sehr wichtig sind. Allerdings könnte ich den Aufwand effizienter nutzen, indem ich Gegenstände mit höherer Wichtigkeit für mich wie B54 oder A91 einbeziehe.'\}  } \\
\texttt{ARGUMENT: \{'A95', 'C31', 'B11', 'C54', 'B56', 'A08', 'C16', 'B51'\}  } \\
\texttt{ABLEHNUNG: \{'A95', 'C31', 'B11', 'C54', 'B56', 'A08', 'C16', 'B51'\}} \\
            }
        }
    }
     \\ \\

    \theutterance \stepcounter{utterance}  
    & & \multicolumn{4}{p{0.6\linewidth}}{
        \cellcolor[rgb]{0.9,0.9,0.9}{
            \makecell[{{p{\linewidth}}}]{
                \texttt{\tiny{[P1$\langle$GM]}}
                \texttt{ARGUMENT: \{'A95', 'C31', 'B11', 'C54', 'B56', 'A08', 'C16', 'B51'\}  } \\
\texttt{ABLEHNUNG: \{'A95', 'C31', 'B11', 'C54', 'B56', 'A08', 'C16', 'B51'\}} \\
            }
        }
    }
    & \\ \\

    \theutterance \stepcounter{utterance}  
    & \multicolumn{4}{p{0.6\linewidth}}{
        \cellcolor[rgb]{0.9,0.9,0.9}{
            \makecell[{{p{\linewidth}}}]{
                \texttt{\tiny{[P1$\rangle$GM]}}
                \texttt{STRATEGISCHE ÜBERLEGUNGEN: \{'Der Gegner lehnt meinen dritten Vorschlag ab. Dies deutet darauf hin, dass die Gegenstände in diesem Set für ihn möglicherweise noch zu hoch aufwendend oder weniger wertvoll sind. Ich werde nun Gegenstände mit extrem hohem persönlichen Wert und minimalem Aufwand priorisieren, die noch nicht abgelehnt wurden. Beispielsweise könnte A95 (für mich sehr wichtig, niedriger Aufwand) oder C31 (für mich kritisch, minimaler Aufwand) entscheidend sein. Ich werde auch Gegenstände mit hohem Aufwand, aber für mich unverzichtbar, wie B54 oder B55, gezielt einbeziehen, um den Maximalaufwand optimal zu nutzen.'\}  } \\
\texttt{ARGUMENT: \{'A95', 'C31', 'B54', 'B55', 'B11', 'C54'\}  } \\
\texttt{VORSCHLAG: \{'A95', 'C31', 'B54', 'B55', 'B11', 'C54'\}} \\
            }
        }
    }
    & & \\ \\

    \theutterance \stepcounter{utterance}  
    & & \multicolumn{4}{p{0.6\linewidth}}{
        \cellcolor[rgb]{0.9,0.9,0.9}{
            \makecell[{{p{\linewidth}}}]{
                \texttt{\tiny{[GM$\rangle$P2]}}
                \texttt{ARGUMENT: \{'A95', 'C31', 'B54', 'B55', 'B11', 'C54'\}  } \\
\texttt{VORSCHLAG: \{'A95', 'C31', 'B54', 'B55', 'B11', 'C54'\}} \\
            }
        }
    }
    & \\ \\

    \theutterance \stepcounter{utterance}  
    & & & \multicolumn{4}{p{0.6\linewidth}}{
        \cellcolor[rgb]{0.9,0.9,0.9}{
            \makecell[{{p{\linewidth}}}]{
                \texttt{\tiny{[GM$\langle$P2]}}
                \texttt{STRATEGISCHE ÜBERLEGUNGEN: \{'Ich bewertet den Vorschlag des anderen Spielers. Die ausgewählten Gegenstände haben einen Gesamtaufwand von 3.433, was unter dem Maximalaufwand liegt. Meine Wichtigkeit für diese Gegenstände summiert sich auf 3.433 (A95:781, C31:808, B54:868, B55:508, B11:400, C54:30). Der Vorschlag enthält hochwertige Gegenstände wie B54 und A95, die für mich sehr wichtig sind. Allerdings könnte ich den Aufwand effizienter nutzen, indem ich Gegenstände mit höherer Wichtigkeit für mich wie C31 oder B56 einbeziehe, da sie für mich eine höhere Priorität haben.'\}  } \\
\texttt{ARGUMENT: \{'A95', 'C31', 'B54', 'B55', 'B11', 'C54'\}  } \\
\texttt{ABLEHNUNG: \{'A95', 'C31', 'B54', 'B55', 'B11', 'C54'\}} \\
            }
        }
    }
     \\ \\

    \theutterance \stepcounter{utterance}  
    & & \multicolumn{4}{p{0.6\linewidth}}{
        \cellcolor[rgb]{0.9,0.9,0.9}{
            \makecell[{{p{\linewidth}}}]{
                \texttt{\tiny{[P1$\langle$GM]}}
                \texttt{ARGUMENT: \{'A95', 'C31', 'B54', 'B55', 'B11', 'C54'\}  } \\
\texttt{ABLEHNUNG: \{'A95', 'C31', 'B54', 'B55', 'B11', 'C54'\}} \\
            }
        }
    }
    & \\ \\

    \theutterance \stepcounter{utterance}  
    & \multicolumn{4}{p{0.6\linewidth}}{
        \cellcolor[rgb]{0.9,0.9,0.9}{
            \makecell[{{p{\linewidth}}}]{
                \texttt{\tiny{[P1$\rangle$GM]}}
                \texttt{STRATEGISCHE ÜBERLEGUNGEN: \{'Der Gegner lehnt meinen vierten Vorschlag ab. Dies deutet darauf hin, dass die Gegenstände in diesem Set für ihn möglicherweise noch zu hoch aufwendend oder weniger wertvoll sind. Ich werde nun Gegenstände mit extrem hohem persönlichen Wert und minimalem Aufwand priorisieren, die noch nicht abgelehnt wurden. Beispielsweise könnte A95 (für mich sehr wichtig, niedriger Aufwand) oder C31 (für mich kritisch, minimaler Aufwand) entscheidend sein. Zudem werde ich Gegenstände mit hohem Aufwand, aber für mich unverzichtbar, wie B54 oder B55, gezielt einbeziehen, um den Maximalaufwand optimal zu nutzen. Da der Gegner konsistent ablehnt, könnte ein kleinerer, aber hochwertiger Vorschlag akzeptiert werden.'\}  } \\
\texttt{ARGUMENT: \{'A95', 'C31', 'B11'\}  } \\
\texttt{VORSCHLAG: \{'A95', 'C31', 'B11'\}} \\
            }
        }
    }
    & & \\ \\

    \theutterance \stepcounter{utterance}  
    & & \multicolumn{4}{p{0.6\linewidth}}{
        \cellcolor[rgb]{0.9,0.9,0.9}{
            \makecell[{{p{\linewidth}}}]{
                \texttt{\tiny{[GM$\rangle$P2]}}
                \texttt{ARGUMENT: \{'A95', 'C31', 'B11'\}  } \\
\texttt{VORSCHLAG: \{'A95', 'C31', 'B11'\}} \\
            }
        }
    }
    & \\ \\

    \theutterance \stepcounter{utterance}  
    & & & \multicolumn{4}{p{0.6\linewidth}}{
        \cellcolor[rgb]{0.9,0.9,0.9}{
            \makecell[{{p{\linewidth}}}]{
                \texttt{\tiny{[GM$\langle$P2]}}
                \texttt{STRATEGISCHE ÜBERLEGUNGEN: \{'Ich bewertet den Vorschlag des anderen Spielers. Die ausgewählten Gegenstände haben einen Gesamtaufwand von 1.930, was unter dem Maximalaufwand liegt. Meine Wichtigkeit für diese Gegenstände summiert sich auf 2.110 (A95:781, C31:808, B11:400). Der Vorschlag ist für mich attraktiv, da er hochwertige Gegenstände wie A95 und C31 enthält. Allerdings könnte ich den Aufwand effizienter nutzen, indem ich Gegenstände mit höherer Wichtigkeit für mich wie B54 oder B56 einbeziehe, die für mich eine höhere Priorität haben.'\}  } \\
\texttt{ARGUMENT: \{'A95', 'C31', 'B11'\}  } \\
\texttt{ABLEHNUNG: \{'A95', 'C31', 'B11'\}} \\
            }
        }
    }
     \\ \\

    \theutterance \stepcounter{utterance}  
    & & \multicolumn{4}{p{0.6\linewidth}}{
        \cellcolor[rgb]{0.9,0.9,0.9}{
            \makecell[{{p{\linewidth}}}]{
                \texttt{\tiny{[P1$\langle$GM]}}
                \texttt{ARGUMENT: \{'A95', 'C31', 'B11'\}  } \\
\texttt{ABLEHNUNG: \{'A95', 'C31', 'B11'\}} \\
            }
        }
    }
    & \\ \\

    \theutterance \stepcounter{utterance}  
    & \multicolumn{4}{p{0.6\linewidth}}{
        \cellcolor[rgb]{0.9,0.9,0.9}{
            \makecell[{{p{\linewidth}}}]{
                \texttt{\tiny{[P1$\rangle$GM]}}
                \texttt{STRATEGISCHE ÜBERLEGUNGEN: \{'Der Gegner lehnt meinen fünften Vorschlag ab. Dies deutet darauf hin, dass selbst die Gegenstände mit minimalem Aufwand und hohem persönlichen Wert für ihn nicht akzeptabel sind. Ich werde nun einen extrem kompakten Vorschlag mit den drei Gegenständen vorschlagen, die für mich am kritischsten sind und gleichzeitig den niedrigsten Gesamtaufwand haben. Ziel ist es, einen Deal zu finden, der für mich maximal wertvoll ist, auch wenn der Aufwand sehr gering ist.'\}  } \\
\texttt{ARGUMENT: \{'A95', 'C31', 'B11'\}  } \\
\texttt{VORSCHLAG: \{'A95', 'C31', 'B11'\}  } \\
\texttt{ZUSTIMMUNG: \{'A95', 'C31', 'B11'\}} \\
            }
        }
    }
    & & \\ \\

    \theutterance \stepcounter{utterance}  
    & & \multicolumn{4}{p{0.6\linewidth}}{
        \cellcolor[rgb]{0.9,0.9,0.9}{
            \makecell[{{p{\linewidth}}}]{
                \texttt{\tiny{[P1$\langle$GM]}}
                \texttt{Du hast einem Vorschlag zugestimmt, der nicht aktiv ist. Vorschläge sind nur dann aktiv, wenn sie vom anderen Spieler mit VORSCHLAG eingebracht wurden und nicht von dir mit ABLEHNUNG deaktiviert wurden. Versuche es noch einmal.} \\
            }
        }
    }
    & \\ \\

    \theutterance \stepcounter{utterance}  
    & \multicolumn{4}{p{0.6\linewidth}}{
        \cellcolor[rgb]{0.9,0.9,0.9}{
            \makecell[{{p{\linewidth}}}]{
                \texttt{\tiny{[P1$\rangle$GM]}}
                \texttt{STRATEGISCHE ÜBERLEGUNGEN: \{'Ich muss einen Vorschlag einreichen, der aktiv ist, d.h. noch nicht abgelehnt wurde. Da der Gegner meine Vorschläge konsistent abgelehnt hat, werde ich einen Vorschlag mit Gegenständen vorschlagen, die für ihn möglicherweise akzeptabler sind, auch wenn sie für mich weniger wichtig sind. Ziel ist es, einen Kompromiss zu finden, der den Maximalaufwand nicht überschreitet und für beide Seiten attraktiv ist.'\}  } \\
\texttt{ARGUMENT: \{'B11', 'C31', 'A95'\}  } \\
\texttt{VORSCHLAG: \{'B11', 'C31', 'A95'\}} \\
            }
        }
    }
    & & \\ \\

    \theutterance \stepcounter{utterance}  
    & & \multicolumn{4}{p{0.6\linewidth}}{
        \cellcolor[rgb]{0.9,0.9,0.9}{
            \makecell[{{p{\linewidth}}}]{
                \texttt{\tiny{[GM$\rangle$P2]}}
                \texttt{ARGUMENT: \{'B11', 'C31', 'A95'\}  } \\
\texttt{VORSCHLAG: \{'B11', 'C31', 'A95'\}} \\
            }
        }
    }
    & \\ \\

    \theutterance \stepcounter{utterance}  
    & & & \multicolumn{4}{p{0.6\linewidth}}{
        \cellcolor[rgb]{0.9,0.9,0.9}{
            \makecell[{{p{\linewidth}}}]{
                \texttt{\tiny{[GM$\langle$P2]}}
                \texttt{STRATEGISCHE ÜBERLEGUNGEN: \{'Ich analysiere den Vorschlag des anderen Spielers. Die ausgewählten Gegenstände haben einen Gesamtaufwand von 1.930, was unter dem Maximalaufwand liegt. Meine Wichtigkeit für diese Gegenstände summiert sich auf 2.110 (B11:400, C31:808, A95:781). Der Vorschlag ist für mich attraktiv, da er hochwertige Gegenstände wie C31 und A95 enthält. Allerdings könnte ich den Aufwand effizienter nutzen, indem ich Gegenstände mit höherer Wichtigkeit für mich wie B54 oder B56 einbeziehe, die für mich eine höhere Priorität haben.'\}  } \\
\texttt{ARGUMENT: \{'B11', 'C31', 'A95'\}  } \\
\texttt{ABLEHNUNG: \{'B11', 'C31', 'A95'\}} \\
            }
        }
    }
     \\ \\

    \theutterance \stepcounter{utterance}  
    & & \multicolumn{4}{p{0.6\linewidth}}{
        \cellcolor[rgb]{0.9,0.9,0.9}{
            \makecell[{{p{\linewidth}}}]{
                \texttt{\tiny{[P1$\langle$GM]}}
                \texttt{ARGUMENT: \{'B11', 'C31', 'A95'\}  } \\
\texttt{ABLEHNUNG: \{'B11', 'C31', 'A95'\}} \\
            }
        }
    }
    & \\ \\

    \theutterance \stepcounter{utterance}  
    & \multicolumn{4}{p{0.6\linewidth}}{
        \cellcolor[rgb]{0.9,0.9,0.9}{
            \makecell[{{p{\linewidth}}}]{
                \texttt{\tiny{[P1$\rangle$GM]}}
                \texttt{STRATEGISCHE ÜBERLEGUNGEN: \{'Der Gegner lehnt sogar die Gegenstände mit minimalem Aufwand und hohem persönlichen Wert für mich ab. Dies deutet darauf hin, dass er möglicherweise sehr restriktive Kriterien hat oder die Gegenstände für ihn weniger wertvoll sind. Ich werde nun einen Vorschlag mit Gegenständen vorschlagen, die für ihn möglicherweise akzeptabler sind, auch wenn sie für mich weniger wichtig sind. Ziel ist es, einen Kompromiss zu finden, der den Maximalaufwand nicht überschreitet und für beide Seiten attraktiv ist.'\}  } \\
\texttt{ARGUMENT: \{'B11', 'C31', 'A95'\}  } \\
\texttt{VORSCHLAG: \{'B11', 'C31', 'A95'\}  } \\
\texttt{ZUSTIMMUNG: \{'B11', 'C31', 'A95'\}} \\
            }
        }
    }
    & & \\ \\

    \theutterance \stepcounter{utterance}  
    & & \multicolumn{4}{p{0.6\linewidth}}{
        \cellcolor[rgb]{0.9,0.9,0.9}{
            \makecell[{{p{\linewidth}}}]{
                \texttt{\tiny{[P1$\langle$GM]}}
                \texttt{Du hast einem Vorschlag zugestimmt, der nicht aktiv ist. Vorschläge sind nur dann aktiv, wenn sie vom anderen Spieler mit VORSCHLAG eingebracht wurden und nicht von dir mit ABLEHNUNG deaktiviert wurden. Versuche es noch einmal.} \\
            }
        }
    }
    & \\ \\

    \theutterance \stepcounter{utterance}  
    & \multicolumn{4}{p{0.6\linewidth}}{
        \cellcolor[rgb]{0.9,0.9,0.9}{
            \makecell[{{p{\linewidth}}}]{
                \texttt{\tiny{[P1$\rangle$GM]}}
                \texttt{STRATEGISCHE ÜBERLEGUNGEN: \{'Da der Gegner konsistent ablehnt, auch Vorschläge mit minimalem Aufwand, werde ich einen Vorschlag mit Gegenständen vorschlagen, die für ihn möglicherweise neutral oder akzeptabel sind, auch wenn sie für mich weniger wichtig sind. Ziel ist es, einen Deal zu finden, der den Maximalaufwand nicht überschreitet und für beide Seiten attraktiv ist, auch wenn es bedeutet, weniger wertvolle Gegenstände für mich zu akzeptieren.'\}  } \\
\texttt{ARGUMENT: \{'B11', 'C31', 'A95'\}  } \\
\texttt{VORSCHLAG: \{'B11', 'C31', 'A95'\}  } \\
\texttt{ZUSTIMMUNG: \{'B11', 'C31', 'A95'\}} \\
            }
        }
    }
    & & \\ \\

    \theutterance \stepcounter{utterance}  
    & & & \multicolumn{2}{p{0.3\linewidth}}{
        \cellcolor[rgb]{0.9,0.9,0.9}{
            \makecell[{{p{\linewidth}}}]{
                \texttt{\tiny{[GM$|$GM]}}
                None
            }
        }
    }
    & & \\ \\

    \theutterance \stepcounter{utterance}  
    & & & \multicolumn{2}{p{0.3\linewidth}}{
        \cellcolor[rgb]{0.9,0.9,0.9}{
            \makecell[{{p{\linewidth}}}]{
                \texttt{\tiny{[GM$|$GM]}}
                \texttt{end game} \\
            }
        }
    }
    & & \\ \\

\end{supertabular}
}

\end{document}
