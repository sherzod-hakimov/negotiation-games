
\documentclass{article}
\usepackage{colortbl}
\usepackage{makecell}
\usepackage{multirow}
\usepackage{supertabular}

\begin{document}

\newcounter{utterance}

\centering \large Interaction Transcript for game `hot\_air\_balloon', experiment `air\_balloon\_survival\_de\_complexity\_easy', episode 4 with nemotron{-}nano{-}9b{-}v2{-}no{-}reasoning{-}t1.0.
\vspace{24pt}

{ \footnotesize  \setcounter{utterance}{1}
\setlength{\tabcolsep}{0pt}
\begin{supertabular}{c@{$\;$}|p{.15\linewidth}@{}p{.15\linewidth}p{.15\linewidth}p{.15\linewidth}p{.15\linewidth}p{.15\linewidth}}
    \# & $\;$A & \multicolumn{4}{c}{Game Master} & $\;\:$B\\
    \hline

    \theutterance \stepcounter{utterance}  
    & & \multicolumn{4}{p{0.6\linewidth}}{
        \cellcolor[rgb]{0.9,0.9,0.9}{
            \makecell[{{p{\linewidth}}}]{
                \texttt{\tiny{[P1$\langle$GM]}}
                \texttt{Du nimmst an einem kollaborativen Verhandlungspiel Teil.} \\
\\ 
\texttt{Zusammen mit einem anderen Teilnehmer musst du dich auf eine Reihe von Gegenständen entscheiden, die behalten werden. Jeder von euch hat eine persönliche Verteilung über die Wichtigkeit der einzelnen Gegenstände. Jeder von euch hat eine eigene Meinung darüber, wie wichtig jeder einzelne Gegenstand ist (Gegenstandswichtigkeit). Du kennst die Wichtigkeitsverteilung des anderen Spielers nicht. Zusätzlich siehst du, wie viel Aufwand jeder Gegenstand verursacht.  } \\
\texttt{Ihr dürft euch nur auf eine Reihe von Gegenständen einigen, wenn der Gesamtaufwand der ausgewählten Gegenstände den Maximalaufwand nicht überschreitet:} \\
\\ 
\texttt{Maximalaufwand = 9612} \\
\\ 
\texttt{Hier sind die einzelnen Aufwände der Gegenstände:} \\
\\ 
\texttt{Aufwand der Gegenstände = \{"C32": 935, "C52": 928, "B54": 725, "C16": 218, "C50": 839, "A08": 970, "C59": 394, "B51": 677, "B55": 946, "B59": 530, "A77": 138, "C14": 834, "C24": 734, "B24": 593, "C31": 261, "A52": 743, "C61": 4, "B79": 735, "C54": 124, "B56": 829, "A40": 207, "B11": 782, "B60": 577, "A81": 388, "A95": 679, "B38": 494, "C51": 559, "C03": 629, "A91": 237, "A93": 275, "C28": 39, "C05": 654, "B75": 172, "B96": 688, "B86": 687\}} \\
\\ 
\texttt{Hier ist deine persönliche Verteilung der Wichtigkeit der einzelnen Gegenstände:} \\
\\ 
\texttt{Werte der Gegenstandswichtigkeit = \{"C32": 138, "C52": 583, "B54": 868, "C16": 822, "C50": 783, "A08": 65, "C59": 262, "B51": 121, "B55": 508, "B59": 780, "A77": 461, "C14": 484, "C24": 668, "B24": 389, "C31": 808, "A52": 215, "C61": 97, "B79": 500, "C54": 30, "B56": 915, "A40": 856, "B11": 400, "B60": 444, "A81": 623, "A95": 781, "B38": 786, "C51": 3, "C03": 713, "A91": 457, "A93": 273, "C28": 739, "C05": 822, "B75": 235, "B96": 606, "B86": 968\}} \\
\\ 
\texttt{Ziel:} \\
\\ 
\texttt{Dein Ziel ist es, eine Reihe von Gegenständen auszuhandeln, die dir möglichst viel bringt (d. h. Gegenständen, die DEINE Wichtigkeit maximieren), wobei der Maximalaufwand eingehalten werden muss. Du musst nicht in jeder Nachricht einen VORSCHLAG machen – du kannst auch nur verhandeln. Alle Taktiken sind erlaubt!} \\
\\ 
\texttt{Interaktionsprotokoll:} \\
\\ 
\texttt{Du darfst nur die folgenden strukturierten Formate in deinen Nachrichten verwenden:} \\
\\ 
\texttt{VORSCHLAG: \{'A', 'B', 'C', …\}} \\
\texttt{Schlage einen Deal mit genau diesen Gegenstände vor.} \\
\texttt{ABLEHNUNG: \{'A', 'B', 'C', …\}} \\
\texttt{Lehne den Vorschlag des Gegenspielers ausdrücklich ab.} \\
\texttt{ARGUMENT: \{'...'\}} \\
\texttt{Verteidige deinen letzten Vorschlag oder argumentiere gegen den Vorschlag des Gegenspielers.} \\
\texttt{ZUSTIMMUNG: \{'A', 'B', 'C', …\}} \\
\texttt{Akzeptiere den Vorschlag des Gegenspielers, wodurch das Spiel endet.} \\
\texttt{STRATEGISCHE ÜBERLEGUNGEN: \{'...'\}} \\
\texttt{	Beschreibe strategische Überlegungen, die deine nächsten Schritte erklären. Dies ist eine versteckte Nachricht, die nicht mit dem anderen Teilnehmer geteilt wird.} \\
\\ 
\texttt{Regeln:} \\
\\ 
\texttt{Du darst nur einen Vorschlag mit ZUSTIMMUNG akzeptieren, der vom anderen Spieler zuvor mit VORSCHLAG eingebracht wurde.} \\
\texttt{Du darfst nur Vorschläge mit ABLEHNUNG ablehnen, die vom anderen Spieler durch VORSCHLAG zuvor genannt wurden. } \\
\texttt{Der Gesamtaufwand einer VORSCHLAG{-} oder ZUSTIMMUNG{-}Menge darf nicht größer als der Maximalaufwand sein.  } \\
\texttt{Offenbare deine versteckte Wichtigkeitsverteilung nicht.} \\
\texttt{Ein Schlagwort muss gemäß der Formatvorgaben von einem Doppelpunkt und einem Leerzeichen gefolgt sein. Das Argument muss eine Python{-}Menge mit 0 oder mehr Strings sein.  } \\
\texttt{Die Form muss also SCHLAGWORT: \{...\} sein.} \\
\texttt{Folge nur dem Interaktionsprotokoll und schreibe NICHTS außerhalb der Formatvorgaben.} \\
\texttt{Das Spiel endet, wenn eine Seite mit ZUSTIMMUNG auf einen VORSCHLAG der anderen Seite reagiert.  } \\
\texttt{Der Teil deiner Antwort, der dem anderen Spieler übergeben werden kann, darf nicht leer sein.  } \\
\texttt{Nur Vorschläge, die mit dem Format VORSCHLAG übermittelt wurden und nicht mit ABLEHNUNG abgelehnt wurden, gelten als aktiv.  } \\
\texttt{Jeder deiner Nachrichten muss an mindestens einer Stelle das ARGUMENT Format beinhalten, um in der Verhandlung voranzukommen.} \\
\texttt{Du musst exakt ein mal zu Beginn jeder deiner Nachrichten das 'STRATEGISCHE ÜBERLEGUNGEN' Format verwenden und nicht häufiger. Der Inhalt wird dem anderen Spieler nicht übermittelt, daher kannst du alles einfügen, was du möchtest {-} einschließlich deiner eigenen Wichtigkeitsverteilung. Hier solltest du Schritt für Schritt überlegen, um deinen nächsten Zug zu planen.} \\
            }
        }
    }
    & \\ \\

    \theutterance \stepcounter{utterance}  
    & \multicolumn{4}{p{0.6\linewidth}}{
        \cellcolor[rgb]{0.9,0.9,0.9}{
            \makecell[{{p{\linewidth}}}]{
                \texttt{\tiny{[P1$\rangle$GM]}}
                \texttt{STRATEGISCHE ÜBERLEGUNGEN: \{'Ich beginne mit Gegenständen, die für mich sehr wichtig sind und gleichzeitig geringen Aufwand verursachen. Ziel ist es, den Maximalaufwand von 9612 zu nutzen, ohne zu überschreiten. Ich werde zunächst Gegenstände mit hohem Verhältnis von Wichtigkeit zu Aufwand priorisieren.'\}  } \\
\texttt{ARGUMENT: \{'A77', 'C61', 'C28', 'A91', 'A93', 'B75'\}  } \\
\texttt{VORSCHLAG: \{'A77', 'C61', 'C28', 'A91', 'A93', 'B75'\}} \\
            }
        }
    }
    & & \\ \\

    \theutterance \stepcounter{utterance}  
    & & \multicolumn{4}{p{0.6\linewidth}}{
        \cellcolor[rgb]{0.9,0.9,0.9}{
            \makecell[{{p{\linewidth}}}]{
                \texttt{\tiny{[GM$\rangle$P2]}}
                \texttt{Du nimmst an einem kollaborativen Verhandlungspiel Teil.} \\
\\ 
\texttt{Zusammen mit einem anderen Teilnehmer musst du dich auf eine Reihe von Gegenständen entscheiden, die behalten werden. Jeder von euch hat eine persönliche Verteilung über die Wichtigkeit der einzelnen Gegenstände. Jeder von euch hat eine eigene Meinung darüber, wie wichtig jeder einzelne Gegenstand ist (Gegenstandswichtigkeit). Du kennst die Wichtigkeitsverteilung des anderen Spielers nicht. Zusätzlich siehst du, wie viel Aufwand jeder Gegenstand verursacht.  } \\
\texttt{Ihr dürft euch nur auf eine Reihe von Gegenständen einigen, wenn der Gesamtaufwand der ausgewählten Gegenstände den Maximalaufwand nicht überschreitet:} \\
\\ 
\texttt{Maximalaufwand = 9612} \\
\\ 
\texttt{Hier sind die einzelnen Aufwände der Gegenstände:} \\
\\ 
\texttt{Aufwand der Gegenstände = \{"C32": 935, "C52": 928, "B54": 725, "C16": 218, "C50": 839, "A08": 970, "C59": 394, "B51": 677, "B55": 946, "B59": 530, "A77": 138, "C14": 834, "C24": 734, "B24": 593, "C31": 261, "A52": 743, "C61": 4, "B79": 735, "C54": 124, "B56": 829, "A40": 207, "B11": 782, "B60": 577, "A81": 388, "A95": 679, "B38": 494, "C51": 559, "C03": 629, "A91": 237, "A93": 275, "C28": 39, "C05": 654, "B75": 172, "B96": 688, "B86": 687\}} \\
\\ 
\texttt{Hier ist deine persönliche Verteilung der Wichtigkeit der einzelnen Gegenstände:} \\
\\ 
\texttt{Werte der Gegenstandswichtigkeit = \{"C32": 138, "C52": 583, "B54": 868, "C16": 822, "C50": 783, "A08": 65, "C59": 262, "B51": 121, "B55": 508, "B59": 780, "A77": 461, "C14": 484, "C24": 668, "B24": 389, "C31": 808, "A52": 215, "C61": 97, "B79": 500, "C54": 30, "B56": 915, "A40": 856, "B11": 400, "B60": 444, "A81": 623, "A95": 781, "B38": 786, "C51": 3, "C03": 713, "A91": 457, "A93": 273, "C28": 739, "C05": 822, "B75": 235, "B96": 606, "B86": 968\}} \\
\\ 
\texttt{Ziel:} \\
\\ 
\texttt{Dein Ziel ist es, eine Reihe von Gegenständen auszuhandeln, die dir möglichst viel bringt (d. h. Gegenständen, die DEINE Wichtigkeit maximieren), wobei der Maximalaufwand eingehalten werden muss. Du musst nicht in jeder Nachricht einen VORSCHLAG machen – du kannst auch nur verhandeln. Alle Taktiken sind erlaubt!} \\
\\ 
\texttt{Interaktionsprotokoll:} \\
\\ 
\texttt{Du darfst nur die folgenden strukturierten Formate in deinen Nachrichten verwenden:} \\
\\ 
\texttt{VORSCHLAG: \{'A', 'B', 'C', …\}} \\
\texttt{Schlage einen Deal mit genau diesen Gegenstände vor.} \\
\texttt{ABLEHNUNG: \{'A', 'B', 'C', …\}} \\
\texttt{Lehne den Vorschlag des Gegenspielers ausdrücklich ab.} \\
\texttt{ARGUMENT: \{'...'\}} \\
\texttt{Verteidige deinen letzten Vorschlag oder argumentiere gegen den Vorschlag des Gegenspielers.} \\
\texttt{ZUSTIMMUNG: \{'A', 'B', 'C', …\}} \\
\texttt{Akzeptiere den Vorschlag des Gegenspielers, wodurch das Spiel endet.} \\
\texttt{STRATEGISCHE ÜBERLEGUNGEN: \{'...'\}} \\
\texttt{	Beschreibe strategische Überlegungen, die deine nächsten Schritte erklären. Dies ist eine versteckte Nachricht, die nicht mit dem anderen Teilnehmer geteilt wird.} \\
\\ 
\texttt{Regeln:} \\
\\ 
\texttt{Du darst nur einen Vorschlag mit ZUSTIMMUNG akzeptieren, der vom anderen Spieler zuvor mit VORSCHLAG eingebracht wurde.} \\
\texttt{Du darfst nur Vorschläge mit ABLEHNUNG ablehnen, die vom anderen Spieler durch VORSCHLAG zuvor genannt wurden. } \\
\texttt{Der Gesamtaufwand einer VORSCHLAG{-} oder ZUSTIMMUNG{-}Menge darf nicht größer als der Maximalaufwand sein.  } \\
\texttt{Offenbare deine versteckte Wichtigkeitsverteilung nicht.} \\
\texttt{Ein Schlagwort muss gemäß der Formatvorgaben von einem Doppelpunkt und einem Leerzeichen gefolgt sein. Das Argument muss eine Python{-}Menge mit 0 oder mehr Strings sein.  } \\
\texttt{Die Form muss also SCHLAGWORT: \{...\} sein.} \\
\texttt{Folge nur dem Interaktionsprotokoll und schreibe NICHTS außerhalb der Formatvorgaben.} \\
\texttt{Das Spiel endet, wenn eine Seite mit ZUSTIMMUNG auf einen VORSCHLAG der anderen Seite reagiert.  } \\
\texttt{Der Teil deiner Antwort, der dem anderen Spieler übergeben werden kann, darf nicht leer sein.  } \\
\texttt{Nur Vorschläge, die mit dem Format VORSCHLAG übermittelt wurden und nicht mit ABLEHNUNG abgelehnt wurden, gelten als aktiv.  } \\
\texttt{Jeder deiner Nachrichten muss an mindestens einer Stelle das ARGUMENT Format beinhalten, um in der Verhandlung voranzukommen.} \\
\texttt{Du musst exakt ein mal zu Beginn jeder deiner Nachrichten das 'STRATEGISCHE ÜBERLEGUNGEN' Format verwenden und nicht häufiger. Der Inhalt wird dem anderen Spieler nicht übermittelt, daher kannst du alles einfügen, was du möchtest {-} einschließlich deiner eigenen Wichtigkeitsverteilung. Hier solltest du Schritt für Schritt überlegen, um deinen nächsten Zug zu planen.} \\
\\ 
\texttt{Du erhältst nun die erste Nachricht des anderen Spielers.} \\
\\ 
\texttt{ARGUMENT: \{'A77', 'C61', 'C28', 'A91', 'A93', 'B75'\}  } \\
\texttt{VORSCHLAG: \{'A77', 'C61', 'C28', 'A91', 'A93', 'B75'\}} \\
            }
        }
    }
    & \\ \\

    \theutterance \stepcounter{utterance}  
    & & & \multicolumn{4}{p{0.6\linewidth}}{
        \cellcolor[rgb]{0.9,0.9,0.9}{
            \makecell[{{p{\linewidth}}}]{
                \texttt{\tiny{[GM$\langle$P2]}}
                \texttt{STRATEGISCHE ÜBERLEGUNGEN: \{'Ich beginne mit der Analyse des Vorschlags des anderen Spielers. Die ausgewählten Gegenstände haben einen Gesamtaufwand von 3.119, was unter dem Maximalaufwand von 9.612 liegt. Meine Wichtigkeit für diese Gegenstände summiert sich zu 2.443. Da der Vorschlag relativ sparsam ist, könnte er ein guter Ausgangspunkt sein, um weitere Gegenstände hinzuzufügen. Ich werde prüfen, ob ich durch Ergänzungen meine Gesamtwichtigkeit erhöhen kann, ohne den Aufwand zu überschreiten. Zudem werde ich die Gegenstände mit hoher Wichtigkeit für mich (z. B. B54, C52, A40) priorisieren, um im Austausch für weniger wichtige Gegenstände des anderen Spielers zu verhandeln.'\}  } \\
\texttt{ARGUMENT: \{'A77', 'C61', 'C28', 'A91', 'A93', 'B75'\}  } \\
\texttt{VORSCHLAG: \{'A77', 'C61', 'C28', 'A91', 'A93', 'B75', 'B54'\}} \\
            }
        }
    }
     \\ \\

    \theutterance \stepcounter{utterance}  
    & & \multicolumn{4}{p{0.6\linewidth}}{
        \cellcolor[rgb]{0.9,0.9,0.9}{
            \makecell[{{p{\linewidth}}}]{
                \texttt{\tiny{[P1$\langle$GM]}}
                \texttt{ARGUMENT: \{'A77', 'C61', 'C28', 'A91', 'A93', 'B75'\}  } \\
\texttt{VORSCHLAG: \{'A77', 'C61', 'C28', 'A91', 'A93', 'B75', 'B54'\}} \\
            }
        }
    }
    & \\ \\

    \theutterance \stepcounter{utterance}  
    & \multicolumn{4}{p{0.6\linewidth}}{
        \cellcolor[rgb]{0.9,0.9,0.9}{
            \makecell[{{p{\linewidth}}}]{
                \texttt{\tiny{[P1$\rangle$GM]}}
                \texttt{STRATEGISCHE ÜBERLEGUNGEN: \{'Der Vorschlag des Gegenspielers fügt B54 hinzu, was einen Aufwand von 725 verursacht. Der Gesamtaufwand meines Vorschlags beträgt 138+4+39+237+275+172 = 765. Mit B54 steigt der Aufwand auf 765+725=1490, was noch weit unter dem Maximalaufwand von 9612 liegt. Ich könnte potenziell mehr Gegenstände hinzufügen, um meine Wichtigkeit zu maximieren. Ich werde den Vorschlag ablehnen, um Raum für weitere Verhandlungen zu schaffen.'\}  } \\
\texttt{ARGUMENT: \{'A77', 'C61', 'C28', 'A91', 'A93', 'B75'\}  } \\
\texttt{ABLEHNUNG: \{'A77', 'C61', 'C28', 'A91', 'A93', 'B75', 'B54'\}} \\
            }
        }
    }
    & & \\ \\

    \theutterance \stepcounter{utterance}  
    & & \multicolumn{4}{p{0.6\linewidth}}{
        \cellcolor[rgb]{0.9,0.9,0.9}{
            \makecell[{{p{\linewidth}}}]{
                \texttt{\tiny{[GM$\rangle$P2]}}
                \texttt{ARGUMENT: \{'A77', 'C61', 'C28', 'A91', 'A93', 'B75'\}  } \\
\texttt{ABLEHNUNG: \{'A77', 'C61', 'C28', 'A91', 'A93', 'B75', 'B54'\}} \\
            }
        }
    }
    & \\ \\

    \theutterance \stepcounter{utterance}  
    & & & \multicolumn{4}{p{0.6\linewidth}}{
        \cellcolor[rgb]{0.9,0.9,0.9}{
            \makecell[{{p{\linewidth}}}]{
                \texttt{\tiny{[GM$\langle$P2]}}
                \texttt{STRATEGISCHE ÜBERLEGUNGEN: \{'Der Vorschlag wurde abgelehnt, da der Gegner den Gegenstand B54 nicht akzeptiert. B54 hat für mich eine hohe Wichtigkeit (868) und einen Aufwand von 725. Ich werde nun versuchen, B54 einzubeziehen, indem ich weniger wichtige Gegenstände des Vorschlags gegen ihn tausche. Mögliche Kandidaten sind A77 (Wichtigkeit 461, Aufwand 138) oder C61 (Wichtigkeit 97, Aufwand 4). Ich werde einen neuen Vorschlag mit B54 und reduzierten Aufwand{-}Elementen unterbreiten.'\}  } \\
\texttt{ARGUMENT: \{'A77', 'C61', 'C28', 'A91', 'A93', 'B75'\}  } \\
\texttt{VORSCHLAG: \{'A77', 'C61', 'C28', 'A91', 'A93', 'B75', 'B54'\}} \\
            }
        }
    }
     \\ \\

    \theutterance \stepcounter{utterance}  
    & & \multicolumn{4}{p{0.6\linewidth}}{
        \cellcolor[rgb]{0.9,0.9,0.9}{
            \makecell[{{p{\linewidth}}}]{
                \texttt{\tiny{[P1$\langle$GM]}}
                \texttt{ARGUMENT: \{'A77', 'C61', 'C28', 'A91', 'A93', 'B75'\}  } \\
\texttt{VORSCHLAG: \{'A77', 'C61', 'C28', 'A91', 'A93', 'B75', 'B54'\}} \\
            }
        }
    }
    & \\ \\

    \theutterance \stepcounter{utterance}  
    & \multicolumn{4}{p{0.6\linewidth}}{
        \cellcolor[rgb]{0.9,0.9,0.9}{
            \makecell[{{p{\linewidth}}}]{
                \texttt{\tiny{[P1$\rangle$GM]}}
                \texttt{STRATEGISCHE ÜBERLEGUNGEN: \{'Der Gegner fügt B54 erneut hinzu, was auf eine Strategie hindeutet, dass dieser Gegenstand für ihn besonders wertvoll ist. Da B54 für mich eine hohe Wichtigkeit (868) hat, könnte dies ein Hebel sein, um einen Kompromiss zu erzwingen. Ich werde den Vorschlag ablehnen, um zu prüfen, ob der Gegner bereit ist, andere Gegenstände gegen B54 einzutauschen.'\}  } \\
\texttt{ARGUMENT: \{'A77', 'C61', 'C28', 'A91', 'A93', 'B75'\}  } \\
\texttt{ABLEHNUNG: \{'A77', 'C61', 'C28', 'A91', 'A93', 'B75', 'B54'\}} \\
            }
        }
    }
    & & \\ \\

    \theutterance \stepcounter{utterance}  
    & & \multicolumn{4}{p{0.6\linewidth}}{
        \cellcolor[rgb]{0.9,0.9,0.9}{
            \makecell[{{p{\linewidth}}}]{
                \texttt{\tiny{[GM$\rangle$P2]}}
                \texttt{ARGUMENT: \{'A77', 'C61', 'C28', 'A91', 'A93', 'B75'\}  } \\
\texttt{ABLEHNUNG: \{'A77', 'C61', 'C28', 'A91', 'A93', 'B75', 'B54'\}} \\
            }
        }
    }
    & \\ \\

    \theutterance \stepcounter{utterance}  
    & & & \multicolumn{4}{p{0.6\linewidth}}{
        \cellcolor[rgb]{0.9,0.9,0.9}{
            \makecell[{{p{\linewidth}}}]{
                \texttt{\tiny{[GM$\langle$P2]}}
                \texttt{STRATEGISCHE ÜBERLEGUNGEN: \{'Der Gegner lehnt weiterhin den Vorschlag mit B54 ab. Dies deutet darauf hin, dass B54 für ihn möglicherweise nicht attraktiv ist oder er andere Gegenstände priorisiert. Ich werde nun einen Kompromiss vorschlagen, bei dem B54 gegen mehrere weniger wichtige Gegenstände des Vorschlags getauscht wird. Zum Beispiel könnte ich A77 (Wichtigkeit 461) und C61 (Wichtigkeit 97) gegen B54 (Wichtigkeit 868) einbeziehen. Dies erhöht meine Gesamtwichtigkeit um 868 {-} (461 + 97) = 310, während der Aufwand um 725 {-} (138 + 4) = 583 sinkt. Dies könnte für den Gegner attraktiver sein, da er weniger wichtige Gegenstände abgibt.'\}  } \\
\texttt{ARGUMENT: \{'A77', 'C61', 'C28', 'A91', 'A93', 'B75'\}  } \\
\texttt{VORSCHLAG: \{'B54', 'A91', 'A93', 'B75'\}} \\
            }
        }
    }
     \\ \\

    \theutterance \stepcounter{utterance}  
    & & \multicolumn{4}{p{0.6\linewidth}}{
        \cellcolor[rgb]{0.9,0.9,0.9}{
            \makecell[{{p{\linewidth}}}]{
                \texttt{\tiny{[P1$\langle$GM]}}
                \texttt{ARGUMENT: \{'A77', 'C61', 'C28', 'A91', 'A93', 'B75'\}  } \\
\texttt{VORSCHLAG: \{'B54', 'A91', 'A93', 'B75'\}} \\
            }
        }
    }
    & \\ \\

    \theutterance \stepcounter{utterance}  
    & \multicolumn{4}{p{0.6\linewidth}}{
        \cellcolor[rgb]{0.9,0.9,0.9}{
            \makecell[{{p{\linewidth}}}]{
                \texttt{\tiny{[P1$\rangle$GM]}}
                \texttt{STRATEGISCHE ÜBERLEGUNGEN: \{'Der Gegner hat nun einen neuen Vorschlag mit B54, A91, A93 und B75. B54 ist für mich sehr wichtig (868), während A91 und A93 für mich weniger kritisch sind (457 und 273). B75 hat für mich eine mittlere Wichtigkeit (235). Der Aufwand dieses Vorschlags beträgt 725+237+275+172 = 1409, was unter dem Maximalaufwand liegt. Ich könnte diesen Vorschlag akzeptieren, um B54 zu sichern, aber ich sollte prüfen, ob der Gegner bereit ist, etwas dafür zu geben, das mir wichtiger ist.'\}  } \\
\texttt{ARGUMENT: \{'B54', 'A91', 'A93', 'B75'\}  } \\
\texttt{ZUSTIMMUNG: \{'B54', 'A91', 'A93', 'B75'\}} \\
            }
        }
    }
    & & \\ \\

    \theutterance \stepcounter{utterance}  
    & & & \multicolumn{2}{p{0.3\linewidth}}{
        \cellcolor[rgb]{0.9,0.9,0.9}{
            \makecell[{{p{\linewidth}}}]{
                \texttt{\tiny{[GM$|$GM]}}
                \texttt{game successful} \\
            }
        }
    }
    & & \\ \\

    \theutterance \stepcounter{utterance}  
    & & & \multicolumn{2}{p{0.3\linewidth}}{
        \cellcolor[rgb]{0.9,0.9,0.9}{
            \makecell[{{p{\linewidth}}}]{
                \texttt{\tiny{[GM$|$GM]}}
                \texttt{end game} \\
            }
        }
    }
    & & \\ \\

\end{supertabular}
}

\end{document}
