
\documentclass{article}
\usepackage{colortbl}
\usepackage{makecell}
\usepackage{multirow}
\usepackage{supertabular}

\begin{document}

\newcounter{utterance}

\twocolumn

{ \footnotesize  \setcounter{utterance}{1}
\setlength{\tabcolsep}{0pt}
\begin{supertabular}{c@{$\;$}|p{.15\linewidth}@{}p{.15\linewidth}p{.15\linewidth}p{.15\linewidth}p{.15\linewidth}p{.15\linewidth}}

    \# & $\;$A & \multicolumn{4}{c}{Game Master} & $\;\:$B\\
    \hline 

    \theutterance \stepcounter{utterance}  

    & & \multicolumn{4}{p{0.6\linewidth}}{\cellcolor[rgb]{0.9,0.9,0.9}{%
	\makecell[{{p{\linewidth}}}]{% 
	  \tt {\tiny [A$\langle$GM]}  
	 Ich bin der Spielleiter, und du spielst ein kooperatives Spiel mit dem folgenden Raster als Spielbrett:\\ \tt \\ \tt ```\\ \tt  1234567\\ \tt ╔═══════╗\\ \tt ║I◌◌◌◌◌◌║ 1\\ \tt ║◌◌◌◌◌◌◌║ 2\\ \tt ║◌◌◌T◌┌─╢ 3\\ \tt ║◌◌◌┌─┤◌║ 4\\ \tt ║◌◌W│C├─╢ 5\\ \tt ╟───┼─┘H║ 6\\ \tt ║◌◌◌│◌◌◌║ 7\\ \tt ╚═══╧═══╝\\ \tt \\ \tt ```\\ \tt \\ \tt * Am oberen Rand sind die x-Koordinaten angegeben, die nach rechts hin zunehmen, und am rechten Rand die y-Koordinaten, die nach unten zunehmen.\\ \tt * Die folgenden Objekte sind zufällig auf deinem Raster platziert: 'W', 'I', 'T', 'C', 'H'.\\ \tt \\ \tt \\ \tt Dein Mitspieler sieht eine Variation des Spielbretts, auf dem dieselben Objekte an anderen zufälligen Positionen platziert sind. Du kannst das Spielbrett des anderen Spielers nicht sehen, und er kann deins nicht sehen.\\ \tt \\ \tt \\ \tt **Ziel des Spiels:**\\ \tt \\ \tt Beide Spieler müssen die Objekte auf ihrem jeweiligen Spielbrett so bewegen, dass identische Objekte am Ende an denselben Koordinaten liegen. Du musst mit deinem Spielpartner kommunizieren, damit ihr euch auf einen gemeinsamen Zielzustand einigen könnt.\\ \tt \\ \tt \\ \tt **Spielregeln:**\\ \tt \\ \tt * In jedem Spielzug kannst du genau eine der folgenden zwei Aktionen ausführen:\\ \tt 1. `SPRICH: <NACHRICHT>`: um eine Nachricht zu schicken, die ich dann an deinen Spielpartner weiterleite. Als Nachricht gilt alles bis zum nächsten Zeilenumbruch.\\ \tt 2. `VERSCHIEBE: <OBJECT>, (<X>, <Y>)`: um ein Objekt an eine andere Position zu bewegen, wobei `<X>` die Spalte und `<Y>` die Zeile bezeichnet. Ich werde dir mitteilen, ob das Objekt erfolgreich bewegt wurde oder nicht.\\ \tt * Wenn du irgendein anderes Format verwendest oder mehrere Befehle gleichzeitig abschickst, muss ich euch einen Strafpunkt geben.\\ \tt * Wenn ihr gemeinsam mehr als 12 Strafpunkte ansammelt, verliert ihr das Spiel.\\ \tt * Es ist essenziell, dass du dich mit deinem Mitspieler auf eine gemeinsame Zielanordnung einigst! Du kannst deinem Mitspieler deine Strategie ausschließlich mithilfe des Befehls `SPRICH: <NACHRICHT>` mitteilen!\\ \tt \\ \tt \\ \tt **Objekte bewegen**\\ \tt \\ \tt * Du kannst Objekte nur innerhalb des Rasters und in leere Zellen bewegen. Leere Zellen erkennst am Zeichen '◌'.\\ \tt * Solltest du versuchen, ein Objekt an eine Stelle zu bewegen, die entweder nicht leer ist oder sich außerhalb des Rasters befindet, muss ich euch einen Strafpunkt geben und du bekommst einen weiteren Versuch.\\ \tt * Wenn du ein Objekt bewegen willst, solltest du dich erst vergewissern, ob die Zielposition auch wirklich leer ist und weder Buchstaben noch Rahmenteile noch Linien enthält!\\ \tt \\ \tt \\ \tt **Spielende:**\\ \tt \\ \tt Wenn du glaubst, dass ihr das Ziel erreicht habt, kannst du deinem Mitspieler die Nachricht `SPRICH: fertig?` schicken. Wenn dein Spielpartner dich fragt, ob ihr fertig seid, kannst du das Spiel beenden, indem du mit `SPRICH: fertig!` antwortest.\\ \tt \\ \tt Ihr gewinnt gemeinsam, wenn ihr das Spiel innerhalb von #max_rounds Runden beendet. Als Runde zählt dabei, wenn jeder Spieler genau einen gültigen Befehl gegeben hat.\\ \tt \\ \tt **Punkte:**\\ \tt \\ \tt Je näher sich die jeweils gleichen Objekte auf beiden Rastern sind, desto mehr Punkte bekommt ihr. Für Strafpunkte gibt es Punktabzug. Könnt ihr den Rekord brechen?\\ \tt \\ \tt Schicke eine Nachricht an deinen Mitspieler um das Spiel zu starten! 
	  } 
	   } 
	   } 
	 & \\ 
 

    \theutterance \stepcounter{utterance}  

    & \multicolumn{4}{p{0.6\linewidth}}{\cellcolor[rgb]{0.8,1,0.9}{%
	\makecell[{{p{\linewidth}}}]{% 
	  \tt {\tiny [A$\rangle$GM]}  
	 SPRICH: Lasst uns alle Objekte in die oberste Zeile bewegen. Ich beginne mit W, dann I, T, C, H. Du solltest dasselbe tun, aber passe die Reihenfolge an deine Position an. Wir müssen uns auf eine gemeinsame Zielanordnung einigen! 
	  } 
	   } 
	   } 
	 & & \\ 
 

    \theutterance \stepcounter{utterance}  

    & & & \multicolumn{2}{p{0.3\linewidth}}{\cellcolor[rgb]{0.95,0.95,0.95}{%
	\makecell[{{p{\linewidth}}}]{% 
	  \tt {\tiny [GM$|$GM]}  
	 SPRICH: Lasst uns alle Objekte in die oberste Zeile bewegen. Ich beginne mit W, dann I, T, C, H. Du solltest dasselbe tun, aber passe die Reihenfolge an deine Position an. Wir müssen uns auf eine gemeinsame Zielanordnung einigen! 
	  } 
	   } 
	   } 
	 & & \\ 
 

    \theutterance \stepcounter{utterance}  

    & & \multicolumn{4}{p{0.6\linewidth}}{\cellcolor[rgb]{0.9,0.9,0.9}{%
	\makecell[{{p{\linewidth}}}]{% 
	  \tt {\tiny [GM$\rangle$B]}  
	 Ich bin der Spielleiter, und du spielst ein kooperatives Spiel mit dem folgenden Raster als Spielbrett:\\ \tt \\ \tt ```\\ \tt  1234567\\ \tt ╔═══════╗\\ \tt ║◌◌H◌◌◌◌║ 1\\ \tt ║◌◌◌C◌◌◌║ 2\\ \tt ║◌◌◌◌◌┌─╢ 3\\ \tt ║◌◌◌┌─┤◌║ 4\\ \tt ║◌IW│◌├─╢ 5\\ \tt ╟───┼─┘◌║ 6\\ \tt ║T◌◌│◌◌◌║ 7\\ \tt ╚═══╧═══╝\\ \tt \\ \tt ```\\ \tt \\ \tt * Am oberen Rand sind die x-Koordinaten angegeben, die nach rechts hin zunehmen, und am rechten Rand die y-Koordinaten, die nach unten zunehmen.\\ \tt * Die folgenden Objekte sind zufällig auf deinem Raster platziert: 'W', 'I', 'T', 'C', 'H'.\\ \tt \\ \tt \\ \tt Dein Mitspieler sieht eine Variation des Spielbretts, auf dem dieselben Objekte an anderen zufälligen Positionen platziert sind. Du kannst das Spielbrett des anderen Spielers nicht sehen, und er kann deins nicht sehen.\\ \tt \\ \tt \\ \tt **Ziel des Spiels:**\\ \tt \\ \tt Beide Spieler müssen die Objekte auf ihrem jeweiligen Spielbrett so bewegen, dass identische Objekte am Ende an denselben Koordinaten liegen. Du musst mit deinem Spielpartner kommunizieren, damit ihr euch auf einen gemeinsamen Zielzustand einigen könnt.\\ \tt \\ \tt \\ \tt **Spielregeln:**\\ \tt \\ \tt * In jedem Spielzug kannst du genau eine der folgenden zwei Aktionen ausführen:\\ \tt 1. `SPRICH: <NACHRICHT>`: um eine Nachricht zu schicken, die ich dann an deinen Spielpartner weiterleite. Als Nachricht gilt alles bis zum nächsten Zeilenumbruch.\\ \tt 2. `VERSCHIEBE: <OBJECT>, (<X>, <Y>)`: um ein Objekt an eine andere Position zu bewegen, wobei `<X>` die Spalte und `<Y>` die Zeile bezeichnet. Ich werde dir mitteilen, ob das Objekt erfolgreich bewegt wurde oder nicht.\\ \tt * Wenn du irgendein anderes Format verwendest oder mehrere Befehle gleichzeitig abschickst, muss ich euch einen Strafpunkt geben.\\ \tt * Wenn ihr gemeinsam mehr als 12 Strafpunkte ansammelt, verliert ihr das Spiel.\\ \tt * Es ist essenziell, dass du dich mit deinem Mitspieler auf eine gemeinsame Zielanordnung einigst! Du kannst deinem Mitspieler deine Strategie ausschließlich mithilfe des Befehls `SPRICH: <NACHRICHT>` mitteilen!\\ \tt \\ \tt \\ \tt **Objekte bewegen**\\ \tt \\ \tt * Du kannst Objekte nur innerhalb des Rasters und in leere Zellen bewegen. Leere Zellen erkennst am Zeichen '◌'.\\ \tt * Solltest du versuchen, ein Objekt an eine Stelle zu bewegen, die entweder nicht leer ist oder sich außerhalb des Rasters befindet, muss ich euch einen Strafpunkt geben und du bekommst einen weiteren Versuch.\\ \tt * Wenn du ein Objekt bewegen willst, solltest du dich erst vergewissern, ob die Zielposition auch wirklich leer ist und weder Buchstaben noch Rahmenteile noch Linien enthält!\\ \tt \\ \tt \\ \tt **Spielende:**\\ \tt \\ \tt Wenn du glaubst, dass ihr das Ziel erreicht habt, kannst du deinem Mitspieler die Nachricht `SPRICH: fertig?` schicken. Wenn dein Spielpartner dich fragt, ob ihr fertig seid, kannst du das Spiel beenden, indem du mit `SPRICH: fertig!` antwortest.\\ \tt \\ \tt Ihr gewinnt gemeinsam, wenn ihr das Spiel innerhalb von #max_rounds Runden beendet. Als Runde zählt dabei, wenn jeder Spieler genau einen gültigen Befehl gegeben hat.\\ \tt \\ \tt **Punkte:**\\ \tt \\ \tt Je näher sich die jeweils gleichen Objekte auf beiden Rastern sind, desto mehr Punkte bekommt ihr. Für Strafpunkte gibt es Punktabzug. Könnt ihr den Rekord brechen?\\ \tt \\ \tt Dein Mitspieler hat das Spiel mit dieser Nachricht begonnen:\\ \tt \\ \tt "Lasst uns alle Objekte in die oberste Zeile bewegen. Ich beginne mit W, dann I, T, C, H. Du solltest dasselbe tun, aber passe die Reihenfolge an deine Position an. Wir müssen uns auf eine gemeinsame Zielanordnung einigen!"\\ \tt \\ \tt Wie lautet dein erster Befehl? 
	  } 
	   } 
	   } 
	 & \\ 
 

    \theutterance \stepcounter{utterance}  

    & & & \multicolumn{4}{p{0.6\linewidth}}{\cellcolor[rgb]{1,0.85,0.72}{%
	\makecell[{{p{\linewidth}}}]{% 
	  \tt {\tiny [GM$\langle$B]}  
	 SPRICH: Lasst uns alle Objekte in die oberste Zeile (y=1) bewegen. Ich beginne mit W, dann I, T, C, H. Du solltest dasselbe tun, aber passe die Reihenfolge an deine Position an. Wir müssen uns auf eine gemeinsame Zielanordnung einigen! 
	  } 
	   } 
	   } 
	  \\ 
 

    \theutterance \stepcounter{utterance}  

    & & & \multicolumn{2}{p{0.3\linewidth}}{\cellcolor[rgb]{0.95,0.95,0.95}{%
	\makecell[{{p{\linewidth}}}]{% 
	  \tt {\tiny [GM$|$GM]}  
	 SPRICH: Lasst uns alle Objekte in die oberste Zeile (y=1) bewegen. Ich beginne mit W, dann I, T, C, H. Du solltest dasselbe tun, aber passe die Reihenfolge an deine Position an. Wir müssen uns auf eine gemeinsame Zielanordnung einigen! 
	  } 
	   } 
	   } 
	 & & \\ 
 

    \theutterance \stepcounter{utterance}  

    & & \multicolumn{4}{p{0.6\linewidth}}{\cellcolor[rgb]{0.9,0.9,0.9}{%
	\makecell[{{p{\linewidth}}}]{% 
	  \tt {\tiny [A$\langle$GM]}  
	 Deine Nachricht wurde an deinen Mitspieler übermittelt.\\ \tt \\ \tt \\ \tt Ihr seid gerade bei Runde 1 von maximal 20.\\ \tt Ihr habt gemeinsam 0 von maximal 12 Strafpunkten angesammelt.\\ \tt Dein Mitspieler hat dir diese Nachricht geschickt:\\ \tt \\ \tt \\ \tt "Lasst uns alle Objekte in die oberste Zeile (y=1) bewegen. Ich beginne mit W, dann I, T, C, H. Du solltest dasselbe tun, aber passe die Reihenfolge an deine Position an. Wir müssen uns auf eine gemeinsame Zielanordnung einigen!"\\ \tt \\ \tt \\ \tt Wie lautet dein nächster Befehl? 
	  } 
	   } 
	   } 
	 & \\ 
 

    \theutterance \stepcounter{utterance}  

    & \multicolumn{4}{p{0.6\linewidth}}{\cellcolor[rgb]{0.8,1,0.9}{%
	\makecell[{{p{\linewidth}}}]{% 
	  \tt {\tiny [A$\rangle$GM]}  
	 SPRICH: Lasst uns die genauen x-Koordinaten für jedes Objekt in der obersten Zeile (y=1) klären. Ich schlage vor: W bei (2,1), I bei (3,1), T bei (4,1), C bei (5,1) und H bei (6,1). Passt das für dich? Wenn ja, starte ich mit dem Verschieben von W nach (2,1). 
	  } 
	   } 
	   } 
	 & & \\ 
 

    \theutterance \stepcounter{utterance}  

    & & & \multicolumn{2}{p{0.3\linewidth}}{\cellcolor[rgb]{0.95,0.95,0.95}{%
	\makecell[{{p{\linewidth}}}]{% 
	  \tt {\tiny [GM$|$GM]}  
	 SPRICH: Lasst uns die genauen x-Koordinaten für jedes Objekt in der obersten Zeile (y=1) klären. Ich schlage vor: W bei (2,1), I bei (3,1), T bei (4,1), C bei (5,1) und H bei (6,1). Passt das für dich? Wenn ja, starte ich mit dem Verschieben von W nach (2,1). 
	  } 
	   } 
	   } 
	 & & \\ 
 

    \theutterance \stepcounter{utterance}  

    & & \multicolumn{4}{p{0.6\linewidth}}{\cellcolor[rgb]{0.9,0.9,0.9}{%
	\makecell[{{p{\linewidth}}}]{% 
	  \tt {\tiny [GM$\rangle$B]}  
	 Deine Nachricht wurde an deinen Mitspieler übermittelt.\\ \tt \\ \tt \\ \tt Ihr seid gerade bei Runde 2 von maximal 20.\\ \tt Ihr habt gemeinsam 0 von maximal 12 Strafpunkten angesammelt.\\ \tt Dein Mitspieler hat dir diese Nachricht geschickt:\\ \tt \\ \tt \\ \tt "Lasst uns die genauen x-Koordinaten für jedes Objekt in der obersten Zeile (y=1) klären. Ich schlage vor: W bei (2,1), I bei (3,1), T bei (4,1), C bei (5,1) und H bei (6,1). Passt das für dich? Wenn ja, starte ich mit dem Verschieben von W nach (2,1)."\\ \tt \\ \tt \\ \tt Wie lautet dein nächster Befehl? 
	  } 
	   } 
	   } 
	 & \\ 
 

    \theutterance \stepcounter{utterance}  

    & & & \multicolumn{4}{p{0.6\linewidth}}{\cellcolor[rgb]{1,0.85,0.72}{%
	\makecell[{{p{\linewidth}}}]{% 
	  \tt {\tiny [GM$\langle$B]}  
	 VERSCHIEBE: W, (2,1) 
	  } 
	   } 
	   } 
	  \\ 
 

    \theutterance \stepcounter{utterance}  

    & & & \multicolumn{2}{p{0.3\linewidth}}{\cellcolor[rgb]{0.95,0.95,0.95}{%
	\makecell[{{p{\linewidth}}}]{% 
	  \tt {\tiny [GM$|$GM]}  
	 VERSCHIEBE: W, (2,1) 
	  } 
	   } 
	   } 
	 & & \\ 
 

    \theutterance \stepcounter{utterance}  

    & & & \multicolumn{2}{p{0.3\linewidth}}{\cellcolor[rgb]{0.95,0.95,0.95}{%
	\makecell[{{p{\linewidth}}}]{% 
	  \tt {\tiny [GM$|$GM]}  
	 'W' erfolgreich nach (2,1) verschoben. Dein Raster sieht jetzt so aus:\\ \tt ```\\ \tt  1234567\\ \tt ╔═══════╗\\ \tt ║◌WH◌◌◌◌║ 1\\ \tt ║◌◌◌C◌◌◌║ 2\\ \tt ║◌◌◌◌◌┌─╢ 3\\ \tt ║◌◌◌┌─┤◌║ 4\\ \tt ║◌I◌│◌├─╢ 5\\ \tt ╟───┼─┘◌║ 6\\ \tt ║T◌◌│◌◌◌║ 7\\ \tt ╚═══╧═══╝\\ \tt ```\\ \tt  
	  } 
	   } 
	   } 
	 & & \\ 
 

    \theutterance \stepcounter{utterance}  

    & & \multicolumn{4}{p{0.6\linewidth}}{\cellcolor[rgb]{0.9,0.9,0.9}{%
	\makecell[{{p{\linewidth}}}]{% 
	  \tt {\tiny [A$\langle$GM]}  
	 Deine Nachricht wurde an deinen Mitspieler übermittelt.\\ \tt \\ \tt \\ \tt Ihr seid gerade bei Runde 2 von maximal 20.\\ \tt Ihr habt gemeinsam 0 von maximal 12 Strafpunkten angesammelt.\\ \tt Dein Mitspieler hat ein Objekt auf seinem Raster bewegt.\\ \tt \\ \tt \\ \tt Wie lautet dein nächster Befehl? 
	  } 
	   } 
	   } 
	 & \\ 
 

    \theutterance \stepcounter{utterance}  

    & \multicolumn{4}{p{0.6\linewidth}}{\cellcolor[rgb]{0.8,1,0.9}{%
	\makecell[{{p{\linewidth}}}]{% 
	  \tt {\tiny [A$\rangle$GM]}  
	 VERSCHIEBE: I, (3,1) 
	  } 
	   } 
	   } 
	 & & \\ 
 

    \theutterance \stepcounter{utterance}  

    & & & \multicolumn{2}{p{0.3\linewidth}}{\cellcolor[rgb]{0.95,0.95,0.95}{%
	\makecell[{{p{\linewidth}}}]{% 
	  \tt {\tiny [GM$|$GM]}  
	 VERSCHIEBE: I, (3,1) 
	  } 
	   } 
	   } 
	 & & \\ 
 

    \theutterance \stepcounter{utterance}  

    & & & \multicolumn{2}{p{0.3\linewidth}}{\cellcolor[rgb]{0.95,0.95,0.95}{%
	\makecell[{{p{\linewidth}}}]{% 
	  \tt {\tiny [GM$|$GM]}  
	 'I' erfolgreich nach (3,1) verschoben. Dein Raster sieht jetzt so aus:\\ \tt ```\\ \tt  1234567\\ \tt ╔═══════╗\\ \tt ║◌◌I◌◌◌◌║ 1\\ \tt ║◌◌◌◌◌◌◌║ 2\\ \tt ║◌◌◌T◌┌─╢ 3\\ \tt ║◌◌◌┌─┤◌║ 4\\ \tt ║◌◌W│C├─╢ 5\\ \tt ╟───┼─┘H║ 6\\ \tt ║◌◌◌│◌◌◌║ 7\\ \tt ╚═══╧═══╝\\ \tt ```\\ \tt  
	  } 
	   } 
	   } 
	 & & \\ 
 

    \theutterance \stepcounter{utterance}  

    & & \multicolumn{4}{p{0.6\linewidth}}{\cellcolor[rgb]{0.9,0.9,0.9}{%
	\makecell[{{p{\linewidth}}}]{% 
	  \tt {\tiny [GM$\rangle$B]}  
	 'W' erfolgreich nach (2,1) verschoben. Dein Raster sieht jetzt so aus:\\ \tt ```\\ \tt  1234567\\ \tt ╔═══════╗\\ \tt ║◌WH◌◌◌◌║ 1\\ \tt ║◌◌◌C◌◌◌║ 2\\ \tt ║◌◌◌◌◌┌─╢ 3\\ \tt ║◌◌◌┌─┤◌║ 4\\ \tt ║◌I◌│◌├─╢ 5\\ \tt ╟───┼─┘◌║ 6\\ \tt ║T◌◌│◌◌◌║ 7\\ \tt ╚═══╧═══╝\\ \tt ```\\ \tt Ihr seid gerade bei Runde 3 von maximal 20.\\ \tt Ihr habt gemeinsam 0 von maximal 12 Strafpunkten angesammelt.\\ \tt Dein Mitspieler hat ein Objekt auf seinem Raster bewegt.\\ \tt \\ \tt \\ \tt Wie lautet dein nächster Befehl? 
	  } 
	   } 
	   } 
	 & \\ 
 

    \theutterance \stepcounter{utterance}  

    & & & \multicolumn{4}{p{0.6\linewidth}}{\cellcolor[rgb]{1,0.85,0.72}{%
	\makecell[{{p{\linewidth}}}]{% 
	  \tt {\tiny [GM$\langle$B]}  
	 VERSCHIEBE: I, (3,1) 
	  } 
	   } 
	   } 
	  \\ 
 

    \theutterance \stepcounter{utterance}  

    & & & \multicolumn{2}{p{0.3\linewidth}}{\cellcolor[rgb]{0.95,0.95,0.95}{%
	\makecell[{{p{\linewidth}}}]{% 
	  \tt {\tiny [GM$|$GM]}  
	 VERSCHIEBE: I, (3,1) 
	  } 
	   } 
	   } 
	 & & \\ 
 

    \theutterance \stepcounter{utterance}  

    & & & \multicolumn{2}{p{0.3\linewidth}}{\cellcolor[rgb]{0.95,0.95,0.95}{%
	\makecell[{{p{\linewidth}}}]{% 
	  \tt {\tiny [GM$|$GM]}  
	 Strafpunkt: (3,1) ist nicht leer, sondern enthält 'H'.\\ \tt \\ \tt Ihr habt gemeinsam 1 von maximal 12 Strafpunkten angesammelt.\\ \tt Bitte versuche es noch mal! 
	  } 
	   } 
	   } 
	 & & \\ 
 

    \theutterance \stepcounter{utterance}  

    & & \multicolumn{4}{p{0.6\linewidth}}{\cellcolor[rgb]{0.9,0.9,0.9}{%
	\makecell[{{p{\linewidth}}}]{% 
	  \tt {\tiny [GM$\rangle$B]}  
	 Strafpunkt: (3,1) ist nicht leer, sondern enthält 'H'.\\ \tt \\ \tt Ihr habt gemeinsam 1 von maximal 12 Strafpunkten angesammelt.\\ \tt Bitte versuche es noch mal! 
	  } 
	   } 
	   } 
	 & \\ 
 

    \theutterance \stepcounter{utterance}  

    & & & \multicolumn{4}{p{0.6\linewidth}}{\cellcolor[rgb]{1,0.85,0.72}{%
	\makecell[{{p{\linewidth}}}]{% 
	  \tt {\tiny [GM$\langle$B]}  
	 VERSCHIEBE: H, (4,1) 
	  } 
	   } 
	   } 
	  \\ 
 

    \theutterance \stepcounter{utterance}  

    & & & \multicolumn{2}{p{0.3\linewidth}}{\cellcolor[rgb]{0.95,0.95,0.95}{%
	\makecell[{{p{\linewidth}}}]{% 
	  \tt {\tiny [GM$|$GM]}  
	 VERSCHIEBE: H, (4,1) 
	  } 
	   } 
	   } 
	 & & \\ 
 

    \theutterance \stepcounter{utterance}  

    & & & \multicolumn{2}{p{0.3\linewidth}}{\cellcolor[rgb]{0.95,0.95,0.95}{%
	\makecell[{{p{\linewidth}}}]{% 
	  \tt {\tiny [GM$|$GM]}  
	 'H' erfolgreich nach (4,1) verschoben. Dein Raster sieht jetzt so aus:\\ \tt ```\\ \tt  1234567\\ \tt ╔═══════╗\\ \tt ║◌W◌H◌◌◌║ 1\\ \tt ║◌◌◌C◌◌◌║ 2\\ \tt ║◌◌◌◌◌┌─╢ 3\\ \tt ║◌◌◌┌─┤◌║ 4\\ \tt ║◌I◌│◌├─╢ 5\\ \tt ╟───┼─┘◌║ 6\\ \tt ║T◌◌│◌◌◌║ 7\\ \tt ╚═══╧═══╝\\ \tt ```\\ \tt  
	  } 
	   } 
	   } 
	 & & \\ 
 

    \theutterance \stepcounter{utterance}  

    & & \multicolumn{4}{p{0.6\linewidth}}{\cellcolor[rgb]{0.9,0.9,0.9}{%
	\makecell[{{p{\linewidth}}}]{% 
	  \tt {\tiny [A$\langle$GM]}  
	 'I' erfolgreich nach (3,1) verschoben. Dein Raster sieht jetzt so aus:\\ \tt ```\\ \tt  1234567\\ \tt ╔═══════╗\\ \tt ║◌◌I◌◌◌◌║ 1\\ \tt ║◌◌◌◌◌◌◌║ 2\\ \tt ║◌◌◌T◌┌─╢ 3\\ \tt ║◌◌◌┌─┤◌║ 4\\ \tt ║◌◌W│C├─╢ 5\\ \tt ╟───┼─┘H║ 6\\ \tt ║◌◌◌│◌◌◌║ 7\\ \tt ╚═══╧═══╝\\ \tt ```\\ \tt Ihr seid gerade bei Runde 3 von maximal 20.\\ \tt Ihr habt gemeinsam 1 von maximal 12 Strafpunkten angesammelt.\\ \tt Dein Mitspieler hat ein Objekt auf seinem Raster bewegt.\\ \tt \\ \tt \\ \tt Wie lautet dein nächster Befehl? 
	  } 
	   } 
	   } 
	 & \\ 
 

    \theutterance \stepcounter{utterance}  

    & \multicolumn{4}{p{0.6\linewidth}}{\cellcolor[rgb]{0.8,1,0.9}{%
	\makecell[{{p{\linewidth}}}]{% 
	  \tt {\tiny [A$\rangle$GM]}  
	 VERSCHIEBE: T, (4,1) 
	  } 
	   } 
	   } 
	 & & \\ 
 

    \theutterance \stepcounter{utterance}  

    & & & \multicolumn{2}{p{0.3\linewidth}}{\cellcolor[rgb]{0.95,0.95,0.95}{%
	\makecell[{{p{\linewidth}}}]{% 
	  \tt {\tiny [GM$|$GM]}  
	 VERSCHIEBE: T, (4,1) 
	  } 
	   } 
	   } 
	 & & \\ 
 

    \theutterance \stepcounter{utterance}  

    & & & \multicolumn{2}{p{0.3\linewidth}}{\cellcolor[rgb]{0.95,0.95,0.95}{%
	\makecell[{{p{\linewidth}}}]{% 
	  \tt {\tiny [GM$|$GM]}  
	 'T' erfolgreich nach (4,1) verschoben. Dein Raster sieht jetzt so aus:\\ \tt ```\\ \tt  1234567\\ \tt ╔═══════╗\\ \tt ║◌◌IT◌◌◌║ 1\\ \tt ║◌◌◌◌◌◌◌║ 2\\ \tt ║◌◌◌◌◌┌─╢ 3\\ \tt ║◌◌◌┌─┤◌║ 4\\ \tt ║◌◌W│C├─╢ 5\\ \tt ╟───┼─┘H║ 6\\ \tt ║◌◌◌│◌◌◌║ 7\\ \tt ╚═══╧═══╝\\ \tt ```\\ \tt  
	  } 
	   } 
	   } 
	 & & \\ 
 

    \theutterance \stepcounter{utterance}  

    & & \multicolumn{4}{p{0.6\linewidth}}{\cellcolor[rgb]{0.9,0.9,0.9}{%
	\makecell[{{p{\linewidth}}}]{% 
	  \tt {\tiny [GM$\rangle$B]}  
	 'H' erfolgreich nach (4,1) verschoben. Dein Raster sieht jetzt so aus:\\ \tt ```\\ \tt  1234567\\ \tt ╔═══════╗\\ \tt ║◌W◌H◌◌◌║ 1\\ \tt ║◌◌◌C◌◌◌║ 2\\ \tt ║◌◌◌◌◌┌─╢ 3\\ \tt ║◌◌◌┌─┤◌║ 4\\ \tt ║◌I◌│◌├─╢ 5\\ \tt ╟───┼─┘◌║ 6\\ \tt ║T◌◌│◌◌◌║ 7\\ \tt ╚═══╧═══╝\\ \tt ```\\ \tt Ihr seid gerade bei Runde 4 von maximal 20.\\ \tt Ihr habt gemeinsam 1 von maximal 12 Strafpunkten angesammelt.\\ \tt Dein Mitspieler hat ein Objekt auf seinem Raster bewegt.\\ \tt \\ \tt \\ \tt Wie lautet dein nächster Befehl? 
	  } 
	   } 
	   } 
	 & \\ 
 

    \theutterance \stepcounter{utterance}  

    & & & \multicolumn{4}{p{0.6\linewidth}}{\cellcolor[rgb]{1,0.85,0.72}{%
	\makecell[{{p{\linewidth}}}]{% 
	  \tt {\tiny [GM$\langle$B]}  
	 VERSCHIEBE: I, (3,1) 
	  } 
	   } 
	   } 
	  \\ 
 

    \theutterance \stepcounter{utterance}  

    & & & \multicolumn{2}{p{0.3\linewidth}}{\cellcolor[rgb]{0.95,0.95,0.95}{%
	\makecell[{{p{\linewidth}}}]{% 
	  \tt {\tiny [GM$|$GM]}  
	 VERSCHIEBE: I, (3,1) 
	  } 
	   } 
	   } 
	 & & \\ 
 

    \theutterance \stepcounter{utterance}  

    & & & \multicolumn{2}{p{0.3\linewidth}}{\cellcolor[rgb]{0.95,0.95,0.95}{%
	\makecell[{{p{\linewidth}}}]{% 
	  \tt {\tiny [GM$|$GM]}  
	 'I' erfolgreich nach (3,1) verschoben. Dein Raster sieht jetzt so aus:\\ \tt ```\\ \tt  1234567\\ \tt ╔═══════╗\\ \tt ║◌WIH◌◌◌║ 1\\ \tt ║◌◌◌C◌◌◌║ 2\\ \tt ║◌◌◌◌◌┌─╢ 3\\ \tt ║◌◌◌┌─┤◌║ 4\\ \tt ║◌◌◌│◌├─╢ 5\\ \tt ╟───┼─┘◌║ 6\\ \tt ║T◌◌│◌◌◌║ 7\\ \tt ╚═══╧═══╝\\ \tt ```\\ \tt  
	  } 
	   } 
	   } 
	 & & \\ 
 

    \theutterance \stepcounter{utterance}  

    & & \multicolumn{4}{p{0.6\linewidth}}{\cellcolor[rgb]{0.9,0.9,0.9}{%
	\makecell[{{p{\linewidth}}}]{% 
	  \tt {\tiny [A$\langle$GM]}  
	 'T' erfolgreich nach (4,1) verschoben. Dein Raster sieht jetzt so aus:\\ \tt ```\\ \tt  1234567\\ \tt ╔═══════╗\\ \tt ║◌◌IT◌◌◌║ 1\\ \tt ║◌◌◌◌◌◌◌║ 2\\ \tt ║◌◌◌◌◌┌─╢ 3\\ \tt ║◌◌◌┌─┤◌║ 4\\ \tt ║◌◌W│C├─╢ 5\\ \tt ╟───┼─┘H║ 6\\ \tt ║◌◌◌│◌◌◌║ 7\\ \tt ╚═══╧═══╝\\ \tt ```\\ \tt Ihr seid gerade bei Runde 4 von maximal 20.\\ \tt Ihr habt gemeinsam 1 von maximal 12 Strafpunkten angesammelt.\\ \tt Dein Mitspieler hat ein Objekt auf seinem Raster bewegt.\\ \tt \\ \tt \\ \tt Wie lautet dein nächster Befehl? 
	  } 
	   } 
	   } 
	 & \\ 
 

    \theutterance \stepcounter{utterance}  

    & \multicolumn{4}{p{0.6\linewidth}}{\cellcolor[rgb]{0.8,1,0.9}{%
	\makecell[{{p{\linewidth}}}]{% 
	  \tt {\tiny [A$\rangle$GM]}  
	 VERSCHIEBE: C, (5,1) 
	  } 
	   } 
	   } 
	 & & \\ 
 

    \theutterance \stepcounter{utterance}  

    & & & \multicolumn{2}{p{0.3\linewidth}}{\cellcolor[rgb]{0.95,0.95,0.95}{%
	\makecell[{{p{\linewidth}}}]{% 
	  \tt {\tiny [GM$|$GM]}  
	 VERSCHIEBE: C, (5,1) 
	  } 
	   } 
	   } 
	 & & \\ 
 

    \theutterance \stepcounter{utterance}  

    & & & \multicolumn{2}{p{0.3\linewidth}}{\cellcolor[rgb]{0.95,0.95,0.95}{%
	\makecell[{{p{\linewidth}}}]{% 
	  \tt {\tiny [GM$|$GM]}  
	 'C' erfolgreich nach (5,1) verschoben. Dein Raster sieht jetzt so aus:\\ \tt ```\\ \tt  1234567\\ \tt ╔═══════╗\\ \tt ║◌◌ITC◌◌║ 1\\ \tt ║◌◌◌◌◌◌◌║ 2\\ \tt ║◌◌◌◌◌┌─╢ 3\\ \tt ║◌◌◌┌─┤◌║ 4\\ \tt ║◌◌W│◌├─╢ 5\\ \tt ╟───┼─┘H║ 6\\ \tt ║◌◌◌│◌◌◌║ 7\\ \tt ╚═══╧═══╝\\ \tt ```\\ \tt  
	  } 
	   } 
	   } 
	 & & \\ 
 

    \theutterance \stepcounter{utterance}  

    & & \multicolumn{4}{p{0.6\linewidth}}{\cellcolor[rgb]{0.9,0.9,0.9}{%
	\makecell[{{p{\linewidth}}}]{% 
	  \tt {\tiny [GM$\rangle$B]}  
	 'I' erfolgreich nach (3,1) verschoben. Dein Raster sieht jetzt so aus:\\ \tt ```\\ \tt  1234567\\ \tt ╔═══════╗\\ \tt ║◌WIH◌◌◌║ 1\\ \tt ║◌◌◌C◌◌◌║ 2\\ \tt ║◌◌◌◌◌┌─╢ 3\\ \tt ║◌◌◌┌─┤◌║ 4\\ \tt ║◌◌◌│◌├─╢ 5\\ \tt ╟───┼─┘◌║ 6\\ \tt ║T◌◌│◌◌◌║ 7\\ \tt ╚═══╧═══╝\\ \tt ```\\ \tt Ihr seid gerade bei Runde 5 von maximal 20.\\ \tt Ihr habt gemeinsam 1 von maximal 12 Strafpunkten angesammelt.\\ \tt Dein Mitspieler hat ein Objekt auf seinem Raster bewegt.\\ \tt \\ \tt \\ \tt Wie lautet dein nächster Befehl? 
	  } 
	   } 
	   } 
	 & \\ 
 

    \theutterance \stepcounter{utterance}  

    & & & \multicolumn{4}{p{0.6\linewidth}}{\cellcolor[rgb]{1,0.85,0.72}{%
	\makecell[{{p{\linewidth}}}]{% 
	  \tt {\tiny [GM$\langle$B]}  
	 VERSCHIEBE: H, (6,1) 
	  } 
	   } 
	   } 
	  \\ 
 

    \theutterance \stepcounter{utterance}  

    & & & \multicolumn{2}{p{0.3\linewidth}}{\cellcolor[rgb]{0.95,0.95,0.95}{%
	\makecell[{{p{\linewidth}}}]{% 
	  \tt {\tiny [GM$|$GM]}  
	 VERSCHIEBE: H, (6,1) 
	  } 
	   } 
	   } 
	 & & \\ 
 

    \theutterance \stepcounter{utterance}  

    & & & \multicolumn{2}{p{0.3\linewidth}}{\cellcolor[rgb]{0.95,0.95,0.95}{%
	\makecell[{{p{\linewidth}}}]{% 
	  \tt {\tiny [GM$|$GM]}  
	 'H' erfolgreich nach (6,1) verschoben. Dein Raster sieht jetzt so aus:\\ \tt ```\\ \tt  1234567\\ \tt ╔═══════╗\\ \tt ║◌WI◌◌H◌║ 1\\ \tt ║◌◌◌C◌◌◌║ 2\\ \tt ║◌◌◌◌◌┌─╢ 3\\ \tt ║◌◌◌┌─┤◌║ 4\\ \tt ║◌◌◌│◌├─╢ 5\\ \tt ╟───┼─┘◌║ 6\\ \tt ║T◌◌│◌◌◌║ 7\\ \tt ╚═══╧═══╝\\ \tt ```\\ \tt  
	  } 
	   } 
	   } 
	 & & \\ 
 

    \theutterance \stepcounter{utterance}  

    & & \multicolumn{4}{p{0.6\linewidth}}{\cellcolor[rgb]{0.9,0.9,0.9}{%
	\makecell[{{p{\linewidth}}}]{% 
	  \tt {\tiny [A$\langle$GM]}  
	 'C' erfolgreich nach (5,1) verschoben. Dein Raster sieht jetzt so aus:\\ \tt ```\\ \tt  1234567\\ \tt ╔═══════╗\\ \tt ║◌◌ITC◌◌║ 1\\ \tt ║◌◌◌◌◌◌◌║ 2\\ \tt ║◌◌◌◌◌┌─╢ 3\\ \tt ║◌◌◌┌─┤◌║ 4\\ \tt ║◌◌W│◌├─╢ 5\\ \tt ╟───┼─┘H║ 6\\ \tt ║◌◌◌│◌◌◌║ 7\\ \tt ╚═══╧═══╝\\ \tt ```\\ \tt Ihr seid gerade bei Runde 5 von maximal 20.\\ \tt Ihr habt gemeinsam 1 von maximal 12 Strafpunkten angesammelt.\\ \tt Dein Mitspieler hat ein Objekt auf seinem Raster bewegt.\\ \tt \\ \tt \\ \tt Wie lautet dein nächster Befehl? 
	  } 
	   } 
	   } 
	 & \\ 
 

    \theutterance \stepcounter{utterance}  

    & \multicolumn{4}{p{0.6\linewidth}}{\cellcolor[rgb]{0.8,1,0.9}{%
	\makecell[{{p{\linewidth}}}]{% 
	  \tt {\tiny [A$\rangle$GM]}  
	 VERSCHIEBE: H, (6,1) 
	  } 
	   } 
	   } 
	 & & \\ 
 

    \theutterance \stepcounter{utterance}  

    & & & \multicolumn{2}{p{0.3\linewidth}}{\cellcolor[rgb]{0.95,0.95,0.95}{%
	\makecell[{{p{\linewidth}}}]{% 
	  \tt {\tiny [GM$|$GM]}  
	 VERSCHIEBE: H, (6,1) 
	  } 
	   } 
	   } 
	 & & \\ 
 

    \theutterance \stepcounter{utterance}  

    & & & \multicolumn{2}{p{0.3\linewidth}}{\cellcolor[rgb]{0.95,0.95,0.95}{%
	\makecell[{{p{\linewidth}}}]{% 
	  \tt {\tiny [GM$|$GM]}  
	 'H' erfolgreich nach (6,1) verschoben. Dein Raster sieht jetzt so aus:\\ \tt ```\\ \tt  1234567\\ \tt ╔═══════╗\\ \tt ║◌◌ITCH◌║ 1\\ \tt ║◌◌◌◌◌◌◌║ 2\\ \tt ║◌◌◌◌◌┌─╢ 3\\ \tt ║◌◌◌┌─┤◌║ 4\\ \tt ║◌◌W│◌├─╢ 5\\ \tt ╟───┼─┘◌║ 6\\ \tt ║◌◌◌│◌◌◌║ 7\\ \tt ╚═══╧═══╝\\ \tt ```\\ \tt  
	  } 
	   } 
	   } 
	 & & \\ 
 

    \theutterance \stepcounter{utterance}  

    & & \multicolumn{4}{p{0.6\linewidth}}{\cellcolor[rgb]{0.9,0.9,0.9}{%
	\makecell[{{p{\linewidth}}}]{% 
	  \tt {\tiny [GM$\rangle$B]}  
	 'H' erfolgreich nach (6,1) verschoben. Dein Raster sieht jetzt so aus:\\ \tt ```\\ \tt  1234567\\ \tt ╔═══════╗\\ \tt ║◌WI◌◌H◌║ 1\\ \tt ║◌◌◌C◌◌◌║ 2\\ \tt ║◌◌◌◌◌┌─╢ 3\\ \tt ║◌◌◌┌─┤◌║ 4\\ \tt ║◌◌◌│◌├─╢ 5\\ \tt ╟───┼─┘◌║ 6\\ \tt ║T◌◌│◌◌◌║ 7\\ \tt ╚═══╧═══╝\\ \tt ```\\ \tt Ihr seid gerade bei Runde 6 von maximal 20.\\ \tt Ihr habt gemeinsam 1 von maximal 12 Strafpunkten angesammelt.\\ \tt Dein Mitspieler hat ein Objekt auf seinem Raster bewegt.\\ \tt \\ \tt \\ \tt Wie lautet dein nächster Befehl? 
	  } 
	   } 
	   } 
	 & \\ 
 

    \theutterance \stepcounter{utterance}  

    & & & \multicolumn{4}{p{0.6\linewidth}}{\cellcolor[rgb]{1,0.85,0.72}{%
	\makecell[{{p{\linewidth}}}]{% 
	  \tt {\tiny [GM$\langle$B]}  
	 VERSCHIEBE: C, (5,1) 
	  } 
	   } 
	   } 
	  \\ 
 

    \theutterance \stepcounter{utterance}  

    & & & \multicolumn{2}{p{0.3\linewidth}}{\cellcolor[rgb]{0.95,0.95,0.95}{%
	\makecell[{{p{\linewidth}}}]{% 
	  \tt {\tiny [GM$|$GM]}  
	 VERSCHIEBE: C, (5,1) 
	  } 
	   } 
	   } 
	 & & \\ 
 

    \theutterance \stepcounter{utterance}  

    & & & \multicolumn{2}{p{0.3\linewidth}}{\cellcolor[rgb]{0.95,0.95,0.95}{%
	\makecell[{{p{\linewidth}}}]{% 
	  \tt {\tiny [GM$|$GM]}  
	 'C' erfolgreich nach (5,1) verschoben. Dein Raster sieht jetzt so aus:\\ \tt ```\\ \tt  1234567\\ \tt ╔═══════╗\\ \tt ║◌WI◌CH◌║ 1\\ \tt ║◌◌◌◌◌◌◌║ 2\\ \tt ║◌◌◌◌◌┌─╢ 3\\ \tt ║◌◌◌┌─┤◌║ 4\\ \tt ║◌◌◌│◌├─╢ 5\\ \tt ╟───┼─┘◌║ 6\\ \tt ║T◌◌│◌◌◌║ 7\\ \tt ╚═══╧═══╝\\ \tt ```\\ \tt  
	  } 
	   } 
	   } 
	 & & \\ 
 

    \theutterance \stepcounter{utterance}  

    & & \multicolumn{4}{p{0.6\linewidth}}{\cellcolor[rgb]{0.9,0.9,0.9}{%
	\makecell[{{p{\linewidth}}}]{% 
	  \tt {\tiny [A$\langle$GM]}  
	 'H' erfolgreich nach (6,1) verschoben. Dein Raster sieht jetzt so aus:\\ \tt ```\\ \tt  1234567\\ \tt ╔═══════╗\\ \tt ║◌◌ITCH◌║ 1\\ \tt ║◌◌◌◌◌◌◌║ 2\\ \tt ║◌◌◌◌◌┌─╢ 3\\ \tt ║◌◌◌┌─┤◌║ 4\\ \tt ║◌◌W│◌├─╢ 5\\ \tt ╟───┼─┘◌║ 6\\ \tt ║◌◌◌│◌◌◌║ 7\\ \tt ╚═══╧═══╝\\ \tt ```\\ \tt Ihr seid gerade bei Runde 6 von maximal 20.\\ \tt Ihr habt gemeinsam 1 von maximal 12 Strafpunkten angesammelt.\\ \tt Dein Mitspieler hat ein Objekt auf seinem Raster bewegt.\\ \tt \\ \tt \\ \tt Wie lautet dein nächster Befehl? 
	  } 
	   } 
	   } 
	 & \\ 
 

    \theutterance \stepcounter{utterance}  

    & \multicolumn{4}{p{0.6\linewidth}}{\cellcolor[rgb]{0.8,1,0.9}{%
	\makecell[{{p{\linewidth}}}]{% 
	  \tt {\tiny [A$\rangle$GM]}  
	 VERSCHIEBE: W, (2,1) 
	  } 
	   } 
	   } 
	 & & \\ 
 

    \theutterance \stepcounter{utterance}  

    & & & \multicolumn{2}{p{0.3\linewidth}}{\cellcolor[rgb]{0.95,0.95,0.95}{%
	\makecell[{{p{\linewidth}}}]{% 
	  \tt {\tiny [GM$|$GM]}  
	 VERSCHIEBE: W, (2,1) 
	  } 
	   } 
	   } 
	 & & \\ 
 

    \theutterance \stepcounter{utterance}  

    & & & \multicolumn{2}{p{0.3\linewidth}}{\cellcolor[rgb]{0.95,0.95,0.95}{%
	\makecell[{{p{\linewidth}}}]{% 
	  \tt {\tiny [GM$|$GM]}  
	 'W' erfolgreich nach (2,1) verschoben. Dein Raster sieht jetzt so aus:\\ \tt ```\\ \tt  1234567\\ \tt ╔═══════╗\\ \tt ║◌WITCH◌║ 1\\ \tt ║◌◌◌◌◌◌◌║ 2\\ \tt ║◌◌◌◌◌┌─╢ 3\\ \tt ║◌◌◌┌─┤◌║ 4\\ \tt ║◌◌◌│◌├─╢ 5\\ \tt ╟───┼─┘◌║ 6\\ \tt ║◌◌◌│◌◌◌║ 7\\ \tt ╚═══╧═══╝\\ \tt ```\\ \tt  
	  } 
	   } 
	   } 
	 & & \\ 
 

    \theutterance \stepcounter{utterance}  

    & & \multicolumn{4}{p{0.6\linewidth}}{\cellcolor[rgb]{0.9,0.9,0.9}{%
	\makecell[{{p{\linewidth}}}]{% 
	  \tt {\tiny [GM$\rangle$B]}  
	 'C' erfolgreich nach (5,1) verschoben. Dein Raster sieht jetzt so aus:\\ \tt ```\\ \tt  1234567\\ \tt ╔═══════╗\\ \tt ║◌WI◌CH◌║ 1\\ \tt ║◌◌◌◌◌◌◌║ 2\\ \tt ║◌◌◌◌◌┌─╢ 3\\ \tt ║◌◌◌┌─┤◌║ 4\\ \tt ║◌◌◌│◌├─╢ 5\\ \tt ╟───┼─┘◌║ 6\\ \tt ║T◌◌│◌◌◌║ 7\\ \tt ╚═══╧═══╝\\ \tt ```\\ \tt Ihr seid gerade bei Runde 7 von maximal 20.\\ \tt Ihr habt gemeinsam 1 von maximal 12 Strafpunkten angesammelt.\\ \tt Dein Mitspieler hat ein Objekt auf seinem Raster bewegt.\\ \tt \\ \tt \\ \tt Wie lautet dein nächster Befehl? 
	  } 
	   } 
	   } 
	 & \\ 
 

    \theutterance \stepcounter{utterance}  

    & & & \multicolumn{4}{p{0.6\linewidth}}{\cellcolor[rgb]{1,0.85,0.72}{%
	\makecell[{{p{\linewidth}}}]{% 
	  \tt {\tiny [GM$\langle$B]}  
	 VERSCHIEBE: T, (4,1) 
	  } 
	   } 
	   } 
	  \\ 
 

    \theutterance \stepcounter{utterance}  

    & & & \multicolumn{2}{p{0.3\linewidth}}{\cellcolor[rgb]{0.95,0.95,0.95}{%
	\makecell[{{p{\linewidth}}}]{% 
	  \tt {\tiny [GM$|$GM]}  
	 VERSCHIEBE: T, (4,1) 
	  } 
	   } 
	   } 
	 & & \\ 
 

    \theutterance \stepcounter{utterance}  

    & & & \multicolumn{2}{p{0.3\linewidth}}{\cellcolor[rgb]{0.95,0.95,0.95}{%
	\makecell[{{p{\linewidth}}}]{% 
	  \tt {\tiny [GM$|$GM]}  
	 'T' erfolgreich nach (4,1) verschoben. Dein Raster sieht jetzt so aus:\\ \tt ```\\ \tt  1234567\\ \tt ╔═══════╗\\ \tt ║◌WITCH◌║ 1\\ \tt ║◌◌◌◌◌◌◌║ 2\\ \tt ║◌◌◌◌◌┌─╢ 3\\ \tt ║◌◌◌┌─┤◌║ 4\\ \tt ║◌◌◌│◌├─╢ 5\\ \tt ╟───┼─┘◌║ 6\\ \tt ║◌◌◌│◌◌◌║ 7\\ \tt ╚═══╧═══╝\\ \tt ```\\ \tt  
	  } 
	   } 
	   } 
	 & & \\ 
 

    \theutterance \stepcounter{utterance}  

    & & \multicolumn{4}{p{0.6\linewidth}}{\cellcolor[rgb]{0.9,0.9,0.9}{%
	\makecell[{{p{\linewidth}}}]{% 
	  \tt {\tiny [A$\langle$GM]}  
	 'W' erfolgreich nach (2,1) verschoben. Dein Raster sieht jetzt so aus:\\ \tt ```\\ \tt  1234567\\ \tt ╔═══════╗\\ \tt ║◌WITCH◌║ 1\\ \tt ║◌◌◌◌◌◌◌║ 2\\ \tt ║◌◌◌◌◌┌─╢ 3\\ \tt ║◌◌◌┌─┤◌║ 4\\ \tt ║◌◌◌│◌├─╢ 5\\ \tt ╟───┼─┘◌║ 6\\ \tt ║◌◌◌│◌◌◌║ 7\\ \tt ╚═══╧═══╝\\ \tt ```\\ \tt Ihr seid gerade bei Runde 7 von maximal 20.\\ \tt Ihr habt gemeinsam 1 von maximal 12 Strafpunkten angesammelt.\\ \tt Dein Mitspieler hat ein Objekt auf seinem Raster bewegt.\\ \tt \\ \tt \\ \tt Wie lautet dein nächster Befehl? 
	  } 
	   } 
	   } 
	 & \\ 
 

    \theutterance \stepcounter{utterance}  

    & \multicolumn{4}{p{0.6\linewidth}}{\cellcolor[rgb]{0.8,1,0.9}{%
	\makecell[{{p{\linewidth}}}]{% 
	  \tt {\tiny [A$\rangle$GM]}  
	  
	  } 
	   } 
	   } 
	 & & \\ 
 

    \theutterance \stepcounter{utterance}  

    & & & \multicolumn{2}{p{0.3\linewidth}}{\cellcolor[rgb]{0.95,0.95,0.95}{%
	\makecell[{{p{\linewidth}}}]{% 
	  \tt {\tiny [GM$|$GM]}  
	  
	  } 
	   } 
	   } 
	 & & \\ 
 

    \theutterance \stepcounter{utterance}  

    & & & \multicolumn{2}{p{0.3\linewidth}}{\cellcolor[rgb]{0.95,0.95,0.95}{%
	\makecell[{{p{\linewidth}}}]{% 
	  \tt {\tiny [GM$|$GM]}  
	 Invalid response format 
	  } 
	   } 
	   } 
	 & & \\ 
 

    \theutterance \stepcounter{utterance}  

    & & \multicolumn{4}{p{0.6\linewidth}}{\cellcolor[rgb]{0.9,0.9,0.9}{%
	\makecell[{{p{\linewidth}}}]{% 
	  \tt {\tiny [A$\langle$GM]}  
	 Strafpunkt: Deine Nachricht entspricht keinem gültigen Format! Stelle sicher, dass deine Antwort ausschließlich einen der beiden Befehle `SPRICH: <NACHRICHT>` und `VERSCHIEBE: <OBJECT>, (<X>, <Y>)` und sonst nichts enthält!\\ \tt \\ \tt Ihr habt gemeinsam 2 von maximal 12 Strafpunkten angesammelt.\\ \tt Bitte versuche es noch mal! 
	  } 
	   } 
	   } 
	 & \\ 
 

    \theutterance \stepcounter{utterance}  

    & \multicolumn{4}{p{0.6\linewidth}}{\cellcolor[rgb]{0.8,1,0.9}{%
	\makecell[{{p{\linewidth}}}]{% 
	  \tt {\tiny [A$\rangle$GM]}  
	 VERSCHIEBE: W,(2,1) 
	  } 
	   } 
	   } 
	 & & \\ 
 

    \theutterance \stepcounter{utterance}  

    & & & \multicolumn{2}{p{0.3\linewidth}}{\cellcolor[rgb]{0.95,0.95,0.95}{%
	\makecell[{{p{\linewidth}}}]{% 
	  \tt {\tiny [GM$|$GM]}  
	 VERSCHIEBE: W,(2,1) 
	  } 
	   } 
	   } 
	 & & \\ 
 

    \theutterance \stepcounter{utterance}  

    & & & \multicolumn{2}{p{0.3\linewidth}}{\cellcolor[rgb]{0.95,0.95,0.95}{%
	\makecell[{{p{\linewidth}}}]{% 
	  \tt {\tiny [GM$|$GM]}  
	 Invalid response format 
	  } 
	   } 
	   } 
	 & & \\ 
 

    \theutterance \stepcounter{utterance}  

    & & \multicolumn{4}{p{0.6\linewidth}}{\cellcolor[rgb]{0.9,0.9,0.9}{%
	\makecell[{{p{\linewidth}}}]{% 
	  \tt {\tiny [A$\langle$GM]}  
	 Strafpunkt: Deine Nachricht entspricht keinem gültigen Format! Stelle sicher, dass deine Antwort ausschließlich einen der beiden Befehle `SPRICH: <NACHRICHT>` und `VERSCHIEBE: <OBJECT>, (<X>, <Y>)` und sonst nichts enthält!\\ \tt \\ \tt Ihr habt gemeinsam 3 von maximal 12 Strafpunkten angesammelt.\\ \tt Bitte versuche es noch mal! 
	  } 
	   } 
	   } 
	 & \\ 
 

    \theutterance \stepcounter{utterance}  

    & \multicolumn{4}{p{0.6\linewidth}}{\cellcolor[rgb]{0.8,1,0.9}{%
	\makecell[{{p{\linewidth}}}]{% 
	  \tt {\tiny [A$\rangle$GM]}  
	 VERSCHIEBE: W, (2,1) 
	  } 
	   } 
	   } 
	 & & \\ 
 

    \theutterance \stepcounter{utterance}  

    & & & \multicolumn{2}{p{0.3\linewidth}}{\cellcolor[rgb]{0.95,0.95,0.95}{%
	\makecell[{{p{\linewidth}}}]{% 
	  \tt {\tiny [GM$|$GM]}  
	 VERSCHIEBE: W, (2,1) 
	  } 
	   } 
	   } 
	 & & \\ 
 

    \theutterance \stepcounter{utterance}  

    & & & \multicolumn{2}{p{0.3\linewidth}}{\cellcolor[rgb]{0.95,0.95,0.95}{%
	\makecell[{{p{\linewidth}}}]{% 
	  \tt {\tiny [GM$|$GM]}  
	 Strafpunkt: (2,1) ist nicht leer, sondern enthält 'W'.\\ \tt \\ \tt Ihr habt gemeinsam 4 von maximal 12 Strafpunkten angesammelt.\\ \tt Bitte versuche es noch mal! 
	  } 
	   } 
	   } 
	 & & \\ 
 

    \theutterance \stepcounter{utterance}  

    & & \multicolumn{4}{p{0.6\linewidth}}{\cellcolor[rgb]{0.9,0.9,0.9}{%
	\makecell[{{p{\linewidth}}}]{% 
	  \tt {\tiny [A$\langle$GM]}  
	 Strafpunkt: (2,1) ist nicht leer, sondern enthält 'W'.\\ \tt \\ \tt Ihr habt gemeinsam 4 von maximal 12 Strafpunkten angesammelt.\\ \tt Bitte versuche es noch mal! 
	  } 
	   } 
	   } 
	 & \\ 
 

    \theutterance \stepcounter{utterance}  

    & \multicolumn{4}{p{0.6\linewidth}}{\cellcolor[rgb]{0.8,1,0.9}{%
	\makecell[{{p{\linewidth}}}]{% 
	  \tt {\tiny [A$\rangle$GM]}  
	 SPRICH: Alle Objekte sind jetzt in der obersten Zeile (y=1) auf meinem Raster. Kannst du bestätigen, ob du dieselbe Zielanordnung hast? Falls nicht, passen wir die Koordinaten an! 
	  } 
	   } 
	   } 
	 & & \\ 
 

    \theutterance \stepcounter{utterance}  

    & & & \multicolumn{2}{p{0.3\linewidth}}{\cellcolor[rgb]{0.95,0.95,0.95}{%
	\makecell[{{p{\linewidth}}}]{% 
	  \tt {\tiny [GM$|$GM]}  
	 SPRICH: Alle Objekte sind jetzt in der obersten Zeile (y=1) auf meinem Raster. Kannst du bestätigen, ob du dieselbe Zielanordnung hast? Falls nicht, passen wir die Koordinaten an! 
	  } 
	   } 
	   } 
	 & & \\ 
 

    \theutterance \stepcounter{utterance}  

    & & \multicolumn{4}{p{0.6\linewidth}}{\cellcolor[rgb]{0.9,0.9,0.9}{%
	\makecell[{{p{\linewidth}}}]{% 
	  \tt {\tiny [GM$\rangle$B]}  
	 'T' erfolgreich nach (4,1) verschoben. Dein Raster sieht jetzt so aus:\\ \tt ```\\ \tt  1234567\\ \tt ╔═══════╗\\ \tt ║◌WITCH◌║ 1\\ \tt ║◌◌◌◌◌◌◌║ 2\\ \tt ║◌◌◌◌◌┌─╢ 3\\ \tt ║◌◌◌┌─┤◌║ 4\\ \tt ║◌◌◌│◌├─╢ 5\\ \tt ╟───┼─┘◌║ 6\\ \tt ║◌◌◌│◌◌◌║ 7\\ \tt ╚═══╧═══╝\\ \tt ```\\ \tt Ihr seid gerade bei Runde 8 von maximal 20.\\ \tt Ihr habt gemeinsam 4 von maximal 12 Strafpunkten angesammelt.\\ \tt Dein Mitspieler hat dir diese Nachricht geschickt:\\ \tt \\ \tt \\ \tt "Alle Objekte sind jetzt in der obersten Zeile (y=1) auf meinem Raster. Kannst du bestätigen, ob du dieselbe Zielanordnung hast? Falls nicht, passen wir die Koordinaten an!"\\ \tt \\ \tt \\ \tt Wie lautet dein nächster Befehl? 
	  } 
	   } 
	   } 
	 & \\ 
 

    \theutterance \stepcounter{utterance}  

    & & & \multicolumn{4}{p{0.6\linewidth}}{\cellcolor[rgb]{1,0.85,0.72}{%
	\makecell[{{p{\linewidth}}}]{% 
	  \tt {\tiny [GM$\langle$B]}  
	 SPRICH: Ja, wir haben dieselbe Zielanordnung! Alle Objekte sind in der obersten Zeile an den Koordinaten (2,1) für W, (3,1) für I, (4,1) für T, (5,1) für C und (6,1) für H positioniert. \\ \tt \\ \tt Jetzt kann dein Mitspieler mit `SPRICH: fertig!` das Spiel beenden, wenn er bestätigt, dass auch er die gleiche Anordnung hat. 
	  } 
	   } 
	   } 
	  \\ 
 

    \theutterance \stepcounter{utterance}  

    & & & \multicolumn{2}{p{0.3\linewidth}}{\cellcolor[rgb]{0.95,0.95,0.95}{%
	\makecell[{{p{\linewidth}}}]{% 
	  \tt {\tiny [GM$|$GM]}  
	 SPRICH: Ja, wir haben dieselbe Zielanordnung! Alle Objekte sind in der obersten Zeile an den Koordinaten (2,1) für W, (3,1) für I, (4,1) für T, (5,1) für C und (6,1) für H positioniert. \\ \tt \\ \tt Jetzt kann dein Mitspieler mit `SPRICH: fertig!` das Spiel beenden, wenn er bestätigt, dass auch er die gleiche Anordnung hat. 
	  } 
	   } 
	   } 
	 & & \\ 
 

    \theutterance \stepcounter{utterance}  

    & & & \multicolumn{2}{p{0.3\linewidth}}{\cellcolor[rgb]{0.95,0.95,0.95}{%
	\makecell[{{p{\linewidth}}}]{% 
	  \tt {\tiny [GM$|$GM]}  
	 Invalid format: head is not empty: 'SPRICH: Ja, wir haben dieselbe Zielanordnung! Alle Objekte sind in der obersten Zeile an den Koordinaten (2,1) für W, (3,1) für I, (4,1) für T, (5,1) für C und (6,1) für H positioniert. \\ \tt \\ \tt Jetzt kann dein Mitspieler mit ' 
	  } 
	   } 
	   } 
	 & & \\ 
 

    \theutterance \stepcounter{utterance}  

    & & \multicolumn{4}{p{0.6\linewidth}}{\cellcolor[rgb]{0.9,0.9,0.9}{%
	\makecell[{{p{\linewidth}}}]{% 
	  \tt {\tiny [GM$\rangle$B]}  
	 Strafpunkt: Deine Nachricht darf nichts vor dem Befehl enthalten! Stelle sicher, dass deine Antwort ausschließlich einen der beiden Befehle `SPRICH: <NACHRICHT>` und `VERSCHIEBE: <OBJECT>, (<X>, <Y>)` und sonst nichts enthält!\\ \tt \\ \tt Ihr habt gemeinsam 5 von maximal 12 Strafpunkten angesammelt.\\ \tt Bitte versuche es noch mal! 
	  } 
	   } 
	   } 
	 & \\ 
 

    \theutterance \stepcounter{utterance}  

    & & & \multicolumn{4}{p{0.6\linewidth}}{\cellcolor[rgb]{1,0.85,0.72}{%
	\makecell[{{p{\linewidth}}}]{% 
	  \tt {\tiny [GM$\langle$B]}  
	 SPRICH: Ja, wir haben dieselbe Zielanordnung! 
	  } 
	   } 
	   } 
	  \\ 
 

    \theutterance \stepcounter{utterance}  

    & & & \multicolumn{2}{p{0.3\linewidth}}{\cellcolor[rgb]{0.95,0.95,0.95}{%
	\makecell[{{p{\linewidth}}}]{% 
	  \tt {\tiny [GM$|$GM]}  
	 SPRICH: Ja, wir haben dieselbe Zielanordnung! 
	  } 
	   } 
	   } 
	 & & \\ 
 

    \theutterance \stepcounter{utterance}  

    & & \multicolumn{4}{p{0.6\linewidth}}{\cellcolor[rgb]{0.9,0.9,0.9}{%
	\makecell[{{p{\linewidth}}}]{% 
	  \tt {\tiny [A$\langle$GM]}  
	 Deine Nachricht wurde an deinen Mitspieler übermittelt.\\ \tt \\ \tt \\ \tt Ihr seid gerade bei Runde 8 von maximal 20.\\ \tt Ihr habt gemeinsam 5 von maximal 12 Strafpunkten angesammelt.\\ \tt Dein Mitspieler hat dir diese Nachricht geschickt:\\ \tt \\ \tt \\ \tt "Ja, wir haben dieselbe Zielanordnung!"\\ \tt \\ \tt \\ \tt Wie lautet dein nächster Befehl? 
	  } 
	   } 
	   } 
	 & \\ 
 

    \theutterance \stepcounter{utterance}  

    & \multicolumn{4}{p{0.6\linewidth}}{\cellcolor[rgb]{0.8,1,0.9}{%
	\makecell[{{p{\linewidth}}}]{% 
	  \tt {\tiny [A$\rangle$GM]}  
	 SPRICH: Alle Objekte sind auf meiner Seite in der obersten Zeile (y=1) positioniert. Kannst du bestätigen, dass deine Objekte ebenfalls an denselben Koordinaten liegen? Wenn ja, können wir das Spiel beenden! 
	  } 
	   } 
	   } 
	 & & \\ 
 

    \theutterance \stepcounter{utterance}  

    & & & \multicolumn{2}{p{0.3\linewidth}}{\cellcolor[rgb]{0.95,0.95,0.95}{%
	\makecell[{{p{\linewidth}}}]{% 
	  \tt {\tiny [GM$|$GM]}  
	 SPRICH: Alle Objekte sind auf meiner Seite in der obersten Zeile (y=1) positioniert. Kannst du bestätigen, dass deine Objekte ebenfalls an denselben Koordinaten liegen? Wenn ja, können wir das Spiel beenden! 
	  } 
	   } 
	   } 
	 & & \\ 
 

    \theutterance \stepcounter{utterance}  

    & & \multicolumn{4}{p{0.6\linewidth}}{\cellcolor[rgb]{0.9,0.9,0.9}{%
	\makecell[{{p{\linewidth}}}]{% 
	  \tt {\tiny [GM$\rangle$B]}  
	 Deine Nachricht wurde an deinen Mitspieler übermittelt.\\ \tt \\ \tt \\ \tt Ihr seid gerade bei Runde 9 von maximal 20.\\ \tt Ihr habt gemeinsam 5 von maximal 12 Strafpunkten angesammelt.\\ \tt Dein Mitspieler hat dir diese Nachricht geschickt:\\ \tt \\ \tt \\ \tt "Alle Objekte sind auf meiner Seite in der obersten Zeile (y=1) positioniert. Kannst du bestätigen, dass deine Objekte ebenfalls an denselben Koordinaten liegen? Wenn ja, können wir das Spiel beenden!"\\ \tt \\ \tt \\ \tt Wie lautet dein nächster Befehl? 
	  } 
	   } 
	   } 
	 & \\ 
 

    \theutterance \stepcounter{utterance}  

    & & & \multicolumn{4}{p{0.6\linewidth}}{\cellcolor[rgb]{1,0.85,0.72}{%
	\makecell[{{p{\linewidth}}}]{% 
	  \tt {\tiny [GM$\langle$B]}  
	 SPRICH: Ja, alle Objekte sind an denselben Koordinaten positioniert! 
	  } 
	   } 
	   } 
	  \\ 
 

    \theutterance \stepcounter{utterance}  

    & & & \multicolumn{2}{p{0.3\linewidth}}{\cellcolor[rgb]{0.95,0.95,0.95}{%
	\makecell[{{p{\linewidth}}}]{% 
	  \tt {\tiny [GM$|$GM]}  
	 SPRICH: Ja, alle Objekte sind an denselben Koordinaten positioniert! 
	  } 
	   } 
	   } 
	 & & \\ 
 

    \theutterance \stepcounter{utterance}  

    & & \multicolumn{4}{p{0.6\linewidth}}{\cellcolor[rgb]{0.9,0.9,0.9}{%
	\makecell[{{p{\linewidth}}}]{% 
	  \tt {\tiny [A$\langle$GM]}  
	 Deine Nachricht wurde an deinen Mitspieler übermittelt.\\ \tt \\ \tt \\ \tt Ihr seid gerade bei Runde 9 von maximal 20.\\ \tt Ihr habt gemeinsam 5 von maximal 12 Strafpunkten angesammelt.\\ \tt Dein Mitspieler hat dir diese Nachricht geschickt:\\ \tt \\ \tt \\ \tt "Ja, alle Objekte sind an denselben Koordinaten positioniert!"\\ \tt \\ \tt \\ \tt Wie lautet dein nächster Befehl? 
	  } 
	   } 
	   } 
	 & \\ 
 

    \theutterance \stepcounter{utterance}  

    & \multicolumn{4}{p{0.6\linewidth}}{\cellcolor[rgb]{0.8,1,0.9}{%
	\makecell[{{p{\linewidth}}}]{% 
	  \tt {\tiny [A$\rangle$GM]}  
	 SPRICH: fertig? 
	  } 
	   } 
	   } 
	 & & \\ 
 

    \theutterance \stepcounter{utterance}  

    & & & \multicolumn{2}{p{0.3\linewidth}}{\cellcolor[rgb]{0.95,0.95,0.95}{%
	\makecell[{{p{\linewidth}}}]{% 
	  \tt {\tiny [GM$|$GM]}  
	 SPRICH: fertig? 
	  } 
	   } 
	   } 
	 & & \\ 
 

    \theutterance \stepcounter{utterance}  

    & & \multicolumn{4}{p{0.6\linewidth}}{\cellcolor[rgb]{0.9,0.9,0.9}{%
	\makecell[{{p{\linewidth}}}]{% 
	  \tt {\tiny [GM$\rangle$B]}  
	 Deine Nachricht wurde an deinen Mitspieler übermittelt.\\ \tt \\ \tt \\ \tt Ihr seid gerade bei Runde 10 von maximal 20.\\ \tt Ihr habt gemeinsam 5 von maximal 12 Strafpunkten angesammelt.\\ \tt Dein Mitspieler hat dir diese Nachricht geschickt:\\ \tt \\ \tt \\ \tt "fertig?"\\ \tt \\ \tt \\ \tt Wie lautet dein nächster Befehl? 
	  } 
	   } 
	   } 
	 & \\ 
 

    \theutterance \stepcounter{utterance}  

    & & & \multicolumn{4}{p{0.6\linewidth}}{\cellcolor[rgb]{1,0.85,0.72}{%
	\makecell[{{p{\linewidth}}}]{% 
	  \tt {\tiny [GM$\langle$B]}  
	 SPRICH: fertig! 
	  } 
	   } 
	   } 
	  \\ 
 

    \theutterance \stepcounter{utterance}  

    & & & \multicolumn{2}{p{0.3\linewidth}}{\cellcolor[rgb]{0.95,0.95,0.95}{%
	\makecell[{{p{\linewidth}}}]{% 
	  \tt {\tiny [GM$|$GM]}  
	 SPRICH: fertig! 
	  } 
	   } 
	   } 
	 & & \\ 
 

    \theutterance \stepcounter{utterance}  

    & & & \multicolumn{2}{p{0.3\linewidth}}{\cellcolor[rgb]{0.95,0.95,0.95}{%
	\makecell[{{p{\linewidth}}}]{% 
	  \tt {\tiny [GM$|$GM]}  
	 true 
	  } 
	   } 
	   } 
	 & & \\ 
 

    \theutterance \stepcounter{utterance}  

    & & & \multicolumn{2}{p{0.3\linewidth}}{\cellcolor[rgb]{0.95,0.95,0.95}{%
	\makecell[{{p{\linewidth}}}]{% 
	  \tt {\tiny [GM$|$GM]}  
	 * success: True\\ \tt * lose: False\\ \tt * aborted: False\\ \tt -------\\ \tt * Shifts: 9.00\\ \tt * Max Shifts: 8.00\\ \tt * Min Shifts: 4.00\\ \tt * End Distance Sum: 0.00\\ \tt * Init Distance Sum: 18.69\\ \tt * Expected Distance Sum: 20.95\\ \tt * Penalties: 5.00\\ \tt * Max Penalties: 12.00\\ \tt * Rounds: 10.00\\ \tt * Max Rounds: 20.00\\ \tt * Object Count: 5.00\\ \tt  
	  } 
	   } 
	   } 
	 & & \\ 
 

    \theutterance \stepcounter{utterance}  

    & & & \multicolumn{2}{p{0.3\linewidth}}{\cellcolor[rgb]{0.95,0.95,0.95}{%
	\makecell[{{p{\linewidth}}}]{% 
	  \tt {\tiny [GM$|$GM]}  
	 * Main Score: 0.00\\ \tt \\ \tt -------\\ \tt * Distance Score: 1.00\\ \tt * Consistency Score: 0.00\\ \tt * Coverage Score: 1.00\\ \tt * Penalty Score: 0.73\\ \tt * Alternative Penalty Score: 0.84\\ \tt * Alternative Main Score: 84.21\\ \tt \\ \tt -------\\ \tt * Shifts: 9.00\\ \tt * Max Shifts: 8.00\\ \tt * Min Shifts: 4.00\\ \tt * End Distance Sum: 0.00\\ \tt * Init Distance Sum: 18.69\\ \tt * Expected Distance Sum: 20.95\\ \tt * Penalties: 5.00\\ \tt * Max Penalties: 12.00\\ \tt * Rounds: 10.00\\ \tt * Max Rounds: 20.00\\ \tt * Object Count: 5.00\\ \tt  
	  } 
	   } 
	   } 
	 & & \\ 
 

\end{supertabular}
}

\end{document}
