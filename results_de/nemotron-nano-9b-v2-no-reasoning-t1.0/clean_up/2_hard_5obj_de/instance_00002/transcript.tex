
\documentclass{article}
\usepackage{colortbl}
\usepackage{makecell}
\usepackage{multirow}
\usepackage{supertabular}

\begin{document}

\newcounter{utterance}

\centering \large Interaction Transcript for game `clean\_up', experiment `2\_hard\_5obj\_de', episode 2 with nemotron{-}nano{-}9b{-}v2{-}no{-}reasoning{-}t1.0.
\vspace{24pt}

{ \footnotesize  \setcounter{utterance}{1}
\setlength{\tabcolsep}{0pt}
\begin{supertabular}{c@{$\;$}|p{.15\linewidth}@{}p{.15\linewidth}p{.15\linewidth}p{.15\linewidth}p{.15\linewidth}p{.15\linewidth}}
    \# & $\;$A & \multicolumn{4}{c}{Game Master} & $\;\:$B\\
<<<<<<< HEAD
    \hline

    \theutterance \stepcounter{utterance}  
    & & \multicolumn{4}{p{0.6\linewidth}}{
        \cellcolor[rgb]{0.9,0.9,0.9}{
            \makecell[{{p{\linewidth}}}]{
                \texttt{\tiny{[P1$\langle$GM]}}
                \texttt{Ich bin der Spielleiter, und du spielst ein kooperatives Spiel mit dem folgenden Raster als Spielbrett:} \\
\\ 
\texttt{```} \\
\texttt{ 1234567} \\
\texttt{╔═╤═╤═╤═╗} \\
\texttt{║T│◌│◌│◌║ 1} \\
\texttt{╟─┼─┤I│◌║ 2} \\
\texttt{║W│◌│◌│H║ 3} \\
\texttt{║◌│◌├─┴─╢ 4} \\
\texttt{╟─┴─┤◌◌◌║ 5} \\
\texttt{║◌C◌└─┬─╢ 6} \\
\texttt{║◌◌◌◌◌│◌║ 7} \\
\texttt{╚═════╧═╝} \\
\\ 
\texttt{```} \\
\\ 
\texttt{* Am oberen Rand sind die x{-}Koordinaten angegeben, die nach rechts hin zunehmen, und am rechten Rand die y{-}Koordinaten, die nach unten zunehmen.} \\
\texttt{* Die folgenden Objekte sind zufällig auf deinem Raster platziert: 'W', 'I', 'T', 'C', 'H'.} \\
\\ 
\\ 
\texttt{Dein Mitspieler sieht eine Variation des Spielbretts, auf dem dieselben Objekte an anderen zufälligen Positionen platziert sind. Du kannst das Spielbrett des anderen Spielers nicht sehen, und er kann deins nicht sehen.} \\
\\ 
\\ 
\texttt{**Ziel des Spiels:**} \\
\\ 
\texttt{Beide Spieler müssen die Objekte auf ihrem jeweiligen Spielbrett so bewegen, dass identische Objekte am Ende an denselben Koordinaten liegen. Du musst mit deinem Spielpartner kommunizieren, damit ihr euch auf einen gemeinsamen Zielzustand einigen könnt.} \\
\\ 
\\ 
\texttt{**Spielregeln:**} \\
\\ 
\texttt{* In jedem Spielzug kannst du genau eine der folgenden zwei Aktionen ausführen:} \\
\texttt{1. `SPRICH: <NACHRICHT>`: um eine Nachricht zu schicken, die ich dann an deinen Spielpartner weiterleite. Als Nachricht gilt alles bis zum nächsten Zeilenumbruch.} \\
\texttt{2. `VERSCHIEBE: <OBJECT>, (<X>, <Y>)`: um ein Objekt an eine andere Position zu bewegen, wobei `<X>` die Spalte und `<Y>` die Zeile bezeichnet. Ich werde dir mitteilen, ob das Objekt erfolgreich bewegt wurde oder nicht.} \\
\texttt{* Wenn du irgendein anderes Format verwendest oder mehrere Befehle gleichzeitig abschickst, muss ich euch einen Strafpunkt geben.} \\
\texttt{* Wenn ihr gemeinsam mehr als 12 Strafpunkte ansammelt, verliert ihr das Spiel.} \\
\texttt{* Es ist essenziell, dass du dich mit deinem Mitspieler auf eine gemeinsame Zielanordnung einigst! Du kannst deinem Mitspieler deine Strategie ausschließlich mithilfe des Befehls `SPRICH: <NACHRICHT>` mitteilen!} \\
\\ 
\\ 
\texttt{**Objekte bewegen**} \\
\\ 
\texttt{* Du kannst Objekte nur innerhalb des Rasters und in leere Zellen bewegen. Leere Zellen erkennst am Zeichen '◌'.} \\
\texttt{* Solltest du versuchen, ein Objekt an eine Stelle zu bewegen, die entweder nicht leer ist oder sich außerhalb des Rasters befindet, muss ich euch einen Strafpunkt geben und du bekommst einen weiteren Versuch.} \\
\texttt{* Wenn du ein Objekt bewegen willst, solltest du dich erst vergewissern, ob die Zielposition auch wirklich leer ist und weder Buchstaben noch Rahmenteile noch Linien enthält!} \\
\\ 
\\ 
\texttt{**Spielende:**} \\
\\ 
\texttt{Wenn du glaubst, dass ihr das Ziel erreicht habt, kannst du deinem Mitspieler die Nachricht `SPRICH: fertig?` schicken. Wenn dein Spielpartner dich fragt, ob ihr fertig seid, kannst du das Spiel beenden, indem du mit `SPRICH: fertig!` antwortest.} \\
\\ 
\texttt{Ihr gewinnt gemeinsam, wenn ihr das Spiel innerhalb von \#max\_rounds Runden beendet. Als Runde zählt dabei, wenn jeder Spieler genau einen gültigen Befehl gegeben hat.} \\
\\ 
\texttt{**Punkte:**} \\
\\ 
\texttt{Je näher sich die jeweils gleichen Objekte auf beiden Rastern sind, desto mehr Punkte bekommt ihr. Für Strafpunkte gibt es Punktabzug. Könnt ihr den Rekord brechen?} \\
\\ 
\texttt{Schicke eine Nachricht an deinen Mitspieler um das Spiel zu starten!} \\
            }
        }
    }
    & \\ \\

    \theutterance \stepcounter{utterance}  
    & \multicolumn{4}{p{0.6\linewidth}}{
        \cellcolor[rgb]{0.9,0.9,0.9}{
            \makecell[{{p{\linewidth}}}]{
                \texttt{\tiny{[P1$\rangle$GM]}}
                \texttt{SPRICH: Spiel gestartet! Ich sende jetzt meine aktuelle Position der Objekte. Bitte teile mir deine Positionen mit, damit wir uns auf eine gemeinsame Zielanordnung einigen können.} \\
            }
        }
    }
    & & \\ \\

    \theutterance \stepcounter{utterance}  
    & & & \multicolumn{2}{p{0.3\linewidth}}{
        \cellcolor[rgb]{0.9,0.9,0.9}{
            \makecell[{{p{\linewidth}}}]{
                \texttt{\tiny{[GM$|$GM]}}
                \texttt{SPRICH: Spiel gestartet! Ich sende jetzt meine aktuelle Position der Objekte. Bitte teile mir deine Positionen mit, damit wir uns auf eine gemeinsame Zielanordnung einigen können.} \\
            }
        }
    }
    & & \\ \\

    \theutterance \stepcounter{utterance}  
    & & \multicolumn{4}{p{0.6\linewidth}}{
        \cellcolor[rgb]{0.9,0.9,0.9}{
            \makecell[{{p{\linewidth}}}]{
                \texttt{\tiny{[GM$\rangle$P2]}}
                \texttt{Ich bin der Spielleiter, und du spielst ein kooperatives Spiel mit dem folgenden Raster als Spielbrett:} \\
\\ 
\texttt{```} \\
\texttt{ 1234567} \\
\texttt{╔═╤═╤═╤═╗} \\
\texttt{║◌│◌│◌│◌║ 1} \\
\texttt{╟─┼─┤◌│◌║ 2} \\
\texttt{║◌│W│◌│◌║ 3} \\
\texttt{║◌│C├─┴─╢ 4} \\
\texttt{╟─┴─┤I◌◌║ 5} \\
\texttt{║◌H◌└─┬─╢ 6} \\
\texttt{║◌◌T◌◌│◌║ 7} \\
\texttt{╚═════╧═╝} \\
\\ 
\texttt{```} \\
\\ 
\texttt{* Am oberen Rand sind die x{-}Koordinaten angegeben, die nach rechts hin zunehmen, und am rechten Rand die y{-}Koordinaten, die nach unten zunehmen.} \\
\texttt{* Die folgenden Objekte sind zufällig auf deinem Raster platziert: 'W', 'I', 'T', 'C', 'H'.} \\
\\ 
\\ 
\texttt{Dein Mitspieler sieht eine Variation des Spielbretts, auf dem dieselben Objekte an anderen zufälligen Positionen platziert sind. Du kannst das Spielbrett des anderen Spielers nicht sehen, und er kann deins nicht sehen.} \\
\\ 
\\ 
\texttt{**Ziel des Spiels:**} \\
\\ 
\texttt{Beide Spieler müssen die Objekte auf ihrem jeweiligen Spielbrett so bewegen, dass identische Objekte am Ende an denselben Koordinaten liegen. Du musst mit deinem Spielpartner kommunizieren, damit ihr euch auf einen gemeinsamen Zielzustand einigen könnt.} \\
\\ 
\\ 
\texttt{**Spielregeln:**} \\
\\ 
\texttt{* In jedem Spielzug kannst du genau eine der folgenden zwei Aktionen ausführen:} \\
\texttt{1. `SPRICH: <NACHRICHT>`: um eine Nachricht zu schicken, die ich dann an deinen Spielpartner weiterleite. Als Nachricht gilt alles bis zum nächsten Zeilenumbruch.} \\
\texttt{2. `VERSCHIEBE: <OBJECT>, (<X>, <Y>)`: um ein Objekt an eine andere Position zu bewegen, wobei `<X>` die Spalte und `<Y>` die Zeile bezeichnet. Ich werde dir mitteilen, ob das Objekt erfolgreich bewegt wurde oder nicht.} \\
\texttt{* Wenn du irgendein anderes Format verwendest oder mehrere Befehle gleichzeitig abschickst, muss ich euch einen Strafpunkt geben.} \\
\texttt{* Wenn ihr gemeinsam mehr als 12 Strafpunkte ansammelt, verliert ihr das Spiel.} \\
\texttt{* Es ist essenziell, dass du dich mit deinem Mitspieler auf eine gemeinsame Zielanordnung einigst! Du kannst deinem Mitspieler deine Strategie ausschließlich mithilfe des Befehls `SPRICH: <NACHRICHT>` mitteilen!} \\
\\ 
\\ 
\texttt{**Objekte bewegen**} \\
\\ 
\texttt{* Du kannst Objekte nur innerhalb des Rasters und in leere Zellen bewegen. Leere Zellen erkennst am Zeichen '◌'.} \\
\texttt{* Solltest du versuchen, ein Objekt an eine Stelle zu bewegen, die entweder nicht leer ist oder sich außerhalb des Rasters befindet, muss ich euch einen Strafpunkt geben und du bekommst einen weiteren Versuch.} \\
\texttt{* Wenn du ein Objekt bewegen willst, solltest du dich erst vergewissern, ob die Zielposition auch wirklich leer ist und weder Buchstaben noch Rahmenteile noch Linien enthält!} \\
\\ 
\\ 
\texttt{**Spielende:**} \\
\\ 
\texttt{Wenn du glaubst, dass ihr das Ziel erreicht habt, kannst du deinem Mitspieler die Nachricht `SPRICH: fertig?` schicken. Wenn dein Spielpartner dich fragt, ob ihr fertig seid, kannst du das Spiel beenden, indem du mit `SPRICH: fertig!` antwortest.} \\
\\ 
\texttt{Ihr gewinnt gemeinsam, wenn ihr das Spiel innerhalb von \#max\_rounds Runden beendet. Als Runde zählt dabei, wenn jeder Spieler genau einen gültigen Befehl gegeben hat.} \\
\\ 
\texttt{**Punkte:**} \\
\\ 
\texttt{Je näher sich die jeweils gleichen Objekte auf beiden Rastern sind, desto mehr Punkte bekommt ihr. Für Strafpunkte gibt es Punktabzug. Könnt ihr den Rekord brechen?} \\
\\ 
\texttt{Dein Mitspieler hat das Spiel mit dieser Nachricht begonnen:} \\
\\ 
\texttt{"Spiel gestartet! Ich sende jetzt meine aktuelle Position der Objekte. Bitte teile mir deine Positionen mit, damit wir uns auf eine gemeinsame Zielanordnung einigen können."} \\
\\ 
\texttt{Wie lautet dein erster Befehl?} \\
            }
        }
    }
    & \\ \\

    \theutterance \stepcounter{utterance}  
    & & & \multicolumn{4}{p{0.6\linewidth}}{
        \cellcolor[rgb]{0.9,0.9,0.9}{
            \makecell[{{p{\linewidth}}}]{
                \texttt{\tiny{[GM$\langle$P2]}}
                \texttt{SPRICH: W: (3,3), C: (4,4), I: (3,5), H: (2,6), T: (4,7)} \\
            }
        }
    }
     \\ \\

    \theutterance \stepcounter{utterance}  
    & & & \multicolumn{2}{p{0.3\linewidth}}{
        \cellcolor[rgb]{0.9,0.9,0.9}{
            \makecell[{{p{\linewidth}}}]{
                \texttt{\tiny{[GM$|$GM]}}
                \texttt{SPRICH: W: (3,3), C: (4,4), I: (3,5), H: (2,6), T: (4,7)} \\
            }
        }
    }
    & & \\ \\

    \theutterance \stepcounter{utterance}  
    & & \multicolumn{4}{p{0.6\linewidth}}{
        \cellcolor[rgb]{0.9,0.9,0.9}{
            \makecell[{{p{\linewidth}}}]{
                \texttt{\tiny{[P1$\langle$GM]}}
                \texttt{Deine Nachricht wurde an deinen Mitspieler übermittelt.} \\
\\ 
\\ 
\texttt{Ihr seid gerade bei Runde 1 von maximal 20.} \\
\texttt{Ihr habt gemeinsam 0 von maximal 12 Strafpunkten angesammelt.} \\
\texttt{Dein Mitspieler hat dir diese Nachricht geschickt:} \\
\\ 
\\ 
\texttt{"W: (3,3), C: (4,4), I: (3,5), H: (2,6), T: (4,7)"} \\
\\ 
\\ 
\texttt{Wie lautet dein nächster Befehl?} \\
            }
        }
    }
    & \\ \\

    \theutterance \stepcounter{utterance}  
    & \multicolumn{4}{p{0.6\linewidth}}{
        \cellcolor[rgb]{0.9,0.9,0.9}{
            \makecell[{{p{\linewidth}}}]{
                \texttt{\tiny{[P1$\rangle$GM]}}
                \texttt{SPRICH: Kannst du mir die Positionen meiner Objekte auf meinem Raster mitteilen? Ich benötige diese Informationen, um eine gemeinsame Zielanordnung zu planen.} \\
            }
        }
    }
    & & \\ \\

    \theutterance \stepcounter{utterance}  
    & & & \multicolumn{2}{p{0.3\linewidth}}{
        \cellcolor[rgb]{0.9,0.9,0.9}{
            \makecell[{{p{\linewidth}}}]{
                \texttt{\tiny{[GM$|$GM]}}
                \texttt{SPRICH: Kannst du mir die Positionen meiner Objekte auf meinem Raster mitteilen? Ich benötige diese Informationen, um eine gemeinsame Zielanordnung zu planen.} \\
            }
        }
    }
    & & \\ \\

    \theutterance \stepcounter{utterance}  
    & & \multicolumn{4}{p{0.6\linewidth}}{
        \cellcolor[rgb]{0.9,0.9,0.9}{
            \makecell[{{p{\linewidth}}}]{
                \texttt{\tiny{[GM$\rangle$P2]}}
                \texttt{Deine Nachricht wurde an deinen Mitspieler übermittelt.} \\
\\ 
\\ 
\texttt{Ihr seid gerade bei Runde 2 von maximal 20.} \\
\texttt{Ihr habt gemeinsam 0 von maximal 12 Strafpunkten angesammelt.} \\
\texttt{Dein Mitspieler hat dir diese Nachricht geschickt:} \\
\\ 
\\ 
\texttt{"Kannst du mir die Positionen meiner Objekte auf meinem Raster mitteilen? Ich benötige diese Informationen, um eine gemeinsame Zielanordnung zu planen."} \\
\\ 
\\ 
\texttt{Wie lautet dein nächster Befehl?} \\
            }
        }
    }
    & \\ \\

    \theutterance \stepcounter{utterance}  
    & & & \multicolumn{4}{p{0.6\linewidth}}{
        \cellcolor[rgb]{0.9,0.9,0.9}{
            \makecell[{{p{\linewidth}}}]{
                \texttt{\tiny{[GM$\langle$P2]}}
                \texttt{SPRICH: Bitte sende mir die Positionen deiner Objekte, damit wir eine gemeinsame Zielanordnung planen können.} \\
            }
        }
    }
     \\ \\

    \theutterance \stepcounter{utterance}  
    & & & \multicolumn{2}{p{0.3\linewidth}}{
        \cellcolor[rgb]{0.9,0.9,0.9}{
            \makecell[{{p{\linewidth}}}]{
                \texttt{\tiny{[GM$|$GM]}}
                \texttt{SPRICH: Bitte sende mir die Positionen deiner Objekte, damit wir eine gemeinsame Zielanordnung planen können.} \\
            }
        }
    }
    & & \\ \\

    \theutterance \stepcounter{utterance}  
    & & \multicolumn{4}{p{0.6\linewidth}}{
        \cellcolor[rgb]{0.9,0.9,0.9}{
            \makecell[{{p{\linewidth}}}]{
                \texttt{\tiny{[P1$\langle$GM]}}
                \texttt{Deine Nachricht wurde an deinen Mitspieler übermittelt.} \\
\\ 
\\ 
\texttt{Ihr seid gerade bei Runde 2 von maximal 20.} \\
\texttt{Ihr habt gemeinsam 0 von maximal 12 Strafpunkten angesammelt.} \\
\texttt{Dein Mitspieler hat dir diese Nachricht geschickt:} \\
\\ 
\\ 
\texttt{"Bitte sende mir die Positionen deiner Objekte, damit wir eine gemeinsame Zielanordnung planen können."} \\
\\ 
\\ 
\texttt{Wie lautet dein nächster Befehl?} \\
            }
        }
    }
    & \\ \\

    \theutterance \stepcounter{utterance}  
    & \multicolumn{4}{p{0.6\linewidth}}{
        \cellcolor[rgb]{0.9,0.9,0.9}{
            \makecell[{{p{\linewidth}}}]{
                \texttt{\tiny{[P1$\rangle$GM]}}
                \texttt{SPRICH: W: (3,3), C: (4,4), I: (3,5), H: (2,6), T: (4,7)  } \\
\\ 
\texttt{*(Ich sende jetzt meine aktuellen Objektpositionen, damit wir eine gemeinsame Zielanordnung planen können.)*} \\
            }
        }
    }
    & & \\ \\

    \theutterance \stepcounter{utterance}  
    & & & \multicolumn{2}{p{0.3\linewidth}}{
        \cellcolor[rgb]{0.9,0.9,0.9}{
            \makecell[{{p{\linewidth}}}]{
                \texttt{\tiny{[GM$|$GM]}}
                \texttt{SPRICH: W: (3,3), C: (4,4), I: (3,5), H: (2,6), T: (4,7)  } \\
\\ 
\texttt{*(Ich sende jetzt meine aktuellen Objektpositionen, damit wir eine gemeinsame Zielanordnung planen können.)*} \\
            }
        }
    }
    & & \\ \\

    \theutterance \stepcounter{utterance}  
    & & & \multicolumn{2}{p{0.3\linewidth}}{
        \cellcolor[rgb]{0.9,0.9,0.9}{
            \makecell[{{p{\linewidth}}}]{
                \texttt{\tiny{[GM$|$GM]}}
                \texttt{Invalid format: tail is not empty: '} \\
\\ 
\texttt{*(Ich sende jetzt meine aktuellen Objektpositionen, damit wir eine gemeinsame Zielanordnung planen können.)*} \\
            }
        }
    }
    & & \\ \\

    \theutterance \stepcounter{utterance}  
    & & \multicolumn{4}{p{0.6\linewidth}}{
        \cellcolor[rgb]{0.9,0.9,0.9}{
            \makecell[{{p{\linewidth}}}]{
                \texttt{\tiny{[P1$\langle$GM]}}
                \texttt{Strafpunkt: Deine Nachricht darf nichts nach dem Befehl enthalten! Stelle sicher, dass deine Antwort ausschließlich einen der beiden Befehle `SPRICH: <NACHRICHT>` und `VERSCHIEBE: <OBJECT>, (<X>, <Y>)` und sonst nichts enthält!} \\
\\ 
\texttt{Ihr habt gemeinsam 1 von maximal 12 Strafpunkten angesammelt.} \\
\texttt{Bitte versuche es noch mal!} \\
            }
        }
    }
    & \\ \\

    \theutterance \stepcounter{utterance}  
    & \multicolumn{4}{p{0.6\linewidth}}{
        \cellcolor[rgb]{0.9,0.9,0.9}{
            \makecell[{{p{\linewidth}}}]{
                \texttt{\tiny{[P1$\rangle$GM]}}
                \texttt{SPRICH: W: (3,3), C: (4,4), I: (3,5), H: (2,6), T: (4,7)} \\
            }
        }
    }
    & & \\ \\

    \theutterance \stepcounter{utterance}  
    & & & \multicolumn{2}{p{0.3\linewidth}}{
        \cellcolor[rgb]{0.9,0.9,0.9}{
            \makecell[{{p{\linewidth}}}]{
                \texttt{\tiny{[GM$|$GM]}}
                \texttt{SPRICH: W: (3,3), C: (4,4), I: (3,5), H: (2,6), T: (4,7)} \\
            }
        }
    }
    & & \\ \\

    \theutterance \stepcounter{utterance}  
    & & \multicolumn{4}{p{0.6\linewidth}}{
        \cellcolor[rgb]{0.9,0.9,0.9}{
            \makecell[{{p{\linewidth}}}]{
                \texttt{\tiny{[GM$\rangle$P2]}}
                \texttt{Deine Nachricht wurde an deinen Mitspieler übermittelt.} \\
\\ 
\\ 
\texttt{Ihr seid gerade bei Runde 3 von maximal 20.} \\
\texttt{Ihr habt gemeinsam 1 von maximal 12 Strafpunkten angesammelt.} \\
\texttt{Dein Mitspieler hat dir diese Nachricht geschickt:} \\
\\ 
\\ 
\texttt{"W: (3,3), C: (4,4), I: (3,5), H: (2,6), T: (4,7)"} \\
\\ 
\\ 
\texttt{Wie lautet dein nächster Befehl?} \\
            }
        }
    }
    & \\ \\

    \theutterance \stepcounter{utterance}  
    & & & \multicolumn{4}{p{0.6\linewidth}}{
        \cellcolor[rgb]{0.9,0.9,0.9}{
            \makecell[{{p{\linewidth}}}]{
                \texttt{\tiny{[GM$\langle$P2]}}
                \texttt{SPRICH: Wir haben nun beide unsere Objektpositionen. Lassen Sie uns eine gemeinsame Zielanordnung planen. Zum Beispiel: W nach (4,4), C nach (3,3), I nach (4,5), H nach (3,6), T nach (4,7). Was denken Sie?} \\
            }
        }
    }
     \\ \\

    \theutterance \stepcounter{utterance}  
    & & & \multicolumn{2}{p{0.3\linewidth}}{
        \cellcolor[rgb]{0.9,0.9,0.9}{
            \makecell[{{p{\linewidth}}}]{
                \texttt{\tiny{[GM$|$GM]}}
                \texttt{SPRICH: Wir haben nun beide unsere Objektpositionen. Lassen Sie uns eine gemeinsame Zielanordnung planen. Zum Beispiel: W nach (4,4), C nach (3,3), I nach (4,5), H nach (3,6), T nach (4,7). Was denken Sie?} \\
            }
        }
    }
    & & \\ \\

    \theutterance \stepcounter{utterance}  
    & & \multicolumn{4}{p{0.6\linewidth}}{
        \cellcolor[rgb]{0.9,0.9,0.9}{
            \makecell[{{p{\linewidth}}}]{
                \texttt{\tiny{[P1$\langle$GM]}}
                \texttt{Deine Nachricht wurde an deinen Mitspieler übermittelt.} \\
\\ 
\\ 
\texttt{Ihr seid gerade bei Runde 3 von maximal 20.} \\
\texttt{Ihr habt gemeinsam 1 von maximal 12 Strafpunkten angesammelt.} \\
\texttt{Dein Mitspieler hat dir diese Nachricht geschickt:} \\
\\ 
\\ 
\texttt{"Wir haben nun beide unsere Objektpositionen. Lassen Sie uns eine gemeinsame Zielanordnung planen. Zum Beispiel: W nach (4,4), C nach (3,3), I nach (4,5), H nach (3,6), T nach (4,7). Was denken Sie?"} \\
\\ 
\\ 
\texttt{Wie lautet dein nächster Befehl?} \\
            }
        }
    }
    & \\ \\

    \theutterance \stepcounter{utterance}  
    & \multicolumn{4}{p{0.6\linewidth}}{
        \cellcolor[rgb]{0.9,0.9,0.9}{
            \makecell[{{p{\linewidth}}}]{
                \texttt{\tiny{[P1$\rangle$GM]}}
                \texttt{SPRICH: Wir stimmen der Zielanordnung zu: W (4,4), C (3,3), I (4,5), H (3,6), T (4,7). Bitte bestätige dies und sende deine aktuellen Objektpositionen, damit wir mit dem Verschieben beginnen können.} \\
            }
        }
    }
    & & \\ \\

    \theutterance \stepcounter{utterance}  
    & & & \multicolumn{2}{p{0.3\linewidth}}{
        \cellcolor[rgb]{0.9,0.9,0.9}{
            \makecell[{{p{\linewidth}}}]{
                \texttt{\tiny{[GM$|$GM]}}
                \texttt{SPRICH: Wir stimmen der Zielanordnung zu: W (4,4), C (3,3), I (4,5), H (3,6), T (4,7). Bitte bestätige dies und sende deine aktuellen Objektpositionen, damit wir mit dem Verschieben beginnen können.} \\
            }
        }
    }
    & & \\ \\

    \theutterance \stepcounter{utterance}  
    & & \multicolumn{4}{p{0.6\linewidth}}{
        \cellcolor[rgb]{0.9,0.9,0.9}{
            \makecell[{{p{\linewidth}}}]{
                \texttt{\tiny{[GM$\rangle$P2]}}
                \texttt{Deine Nachricht wurde an deinen Mitspieler übermittelt.} \\
\\ 
\\ 
\texttt{Ihr seid gerade bei Runde 4 von maximal 20.} \\
\texttt{Ihr habt gemeinsam 1 von maximal 12 Strafpunkten angesammelt.} \\
\texttt{Dein Mitspieler hat dir diese Nachricht geschickt:} \\
\\ 
\\ 
\texttt{"Wir stimmen der Zielanordnung zu: W (4,4), C (3,3), I (4,5), H (3,6), T (4,7). Bitte bestätige dies und sende deine aktuellen Objektpositionen, damit wir mit dem Verschieben beginnen können."} \\
\\ 
\\ 
\texttt{Wie lautet dein nächster Befehl?} \\
            }
        }
    }
    & \\ \\

    \theutterance \stepcounter{utterance}  
    & & & \multicolumn{4}{p{0.6\linewidth}}{
        \cellcolor[rgb]{0.9,0.9,0.9}{
            \makecell[{{p{\linewidth}}}]{
                \texttt{\tiny{[GM$\langle$P2]}}
                \texttt{SPRICH: Bestätige die Zielanordnung: W (4,4), C (3,3), I (4,5), H (3,6), T (4,7). Meine aktuellen Positionen sind W: (3,3), C: (4,4), I: (3,5), H: (2,6), T: (4,7).} \\
            }
        }
    }
     \\ \\

    \theutterance \stepcounter{utterance}  
    & & & \multicolumn{2}{p{0.3\linewidth}}{
        \cellcolor[rgb]{0.9,0.9,0.9}{
            \makecell[{{p{\linewidth}}}]{
                \texttt{\tiny{[GM$|$GM]}}
                \texttt{SPRICH: Bestätige die Zielanordnung: W (4,4), C (3,3), I (4,5), H (3,6), T (4,7). Meine aktuellen Positionen sind W: (3,3), C: (4,4), I: (3,5), H: (2,6), T: (4,7).} \\
            }
        }
    }
    & & \\ \\

    \theutterance \stepcounter{utterance}  
    & & \multicolumn{4}{p{0.6\linewidth}}{
        \cellcolor[rgb]{0.9,0.9,0.9}{
            \makecell[{{p{\linewidth}}}]{
                \texttt{\tiny{[P1$\langle$GM]}}
                \texttt{Deine Nachricht wurde an deinen Mitspieler übermittelt.} \\
\\ 
\\ 
\texttt{Ihr seid gerade bei Runde 4 von maximal 20.} \\
\texttt{Ihr habt gemeinsam 1 von maximal 12 Strafpunkten angesammelt.} \\
\texttt{Dein Mitspieler hat dir diese Nachricht geschickt:} \\
\\ 
\\ 
\texttt{"Bestätige die Zielanordnung: W (4,4), C (3,3), I (4,5), H (3,6), T (4,7). Meine aktuellen Positionen sind W: (3,3), C: (4,4), I: (3,5), H: (2,6), T: (4,7)."} \\
\\ 
\\ 
\texttt{Wie lautet dein nächster Befehl?} \\
            }
        }
    }
    & \\ \\

    \theutterance \stepcounter{utterance}  
    & \multicolumn{4}{p{0.6\linewidth}}{
        \cellcolor[rgb]{0.9,0.9,0.9}{
            \makecell[{{p{\linewidth}}}]{
                \texttt{\tiny{[P1$\rangle$GM]}}
                \texttt{VERSCHIEBE: I, (4,5)} \\
            }
        }
    }
    & & \\ \\

    \theutterance \stepcounter{utterance}  
    & & & \multicolumn{2}{p{0.3\linewidth}}{
        \cellcolor[rgb]{0.9,0.9,0.9}{
            \makecell[{{p{\linewidth}}}]{
                \texttt{\tiny{[GM$|$GM]}}
                \texttt{VERSCHIEBE: I, (4,5)} \\
            }
        }
    }
    & & \\ \\

    \theutterance \stepcounter{utterance}  
    & & & \multicolumn{2}{p{0.3\linewidth}}{
        \cellcolor[rgb]{0.9,0.9,0.9}{
            \makecell[{{p{\linewidth}}}]{
                \texttt{\tiny{[GM$|$GM]}}
                \texttt{Strafpunkt: (4,5) ist nicht leer, sondern enthält '┤'.} \\
\\ 
\texttt{Ihr habt gemeinsam 2 von maximal 12 Strafpunkten angesammelt.} \\
\texttt{Bitte versuche es noch mal!} \\
            }
        }
    }
    & & \\ \\

    \theutterance \stepcounter{utterance}  
    & & \multicolumn{4}{p{0.6\linewidth}}{
        \cellcolor[rgb]{0.9,0.9,0.9}{
            \makecell[{{p{\linewidth}}}]{
                \texttt{\tiny{[P1$\langle$GM]}}
                \texttt{Strafpunkt: (4,5) ist nicht leer, sondern enthält '┤'.} \\
\\ 
\texttt{Ihr habt gemeinsam 2 von maximal 12 Strafpunkten angesammelt.} \\
\texttt{Bitte versuche es noch mal!} \\
            }
        }
    }
    & \\ \\

    \theutterance \stepcounter{utterance}  
    & \multicolumn{4}{p{0.6\linewidth}}{
        \cellcolor[rgb]{0.9,0.9,0.9}{
            \makecell[{{p{\linewidth}}}]{
                \texttt{\tiny{[P1$\rangle$GM]}}
                \texttt{VERSCHIEBE: I, (3,6)} \\
            }
        }
    }
    & & \\ \\

    \theutterance \stepcounter{utterance}  
    & & & \multicolumn{2}{p{0.3\linewidth}}{
        \cellcolor[rgb]{0.9,0.9,0.9}{
            \makecell[{{p{\linewidth}}}]{
                \texttt{\tiny{[GM$|$GM]}}
                \texttt{VERSCHIEBE: I, (3,6)} \\
            }
        }
    }
    & & \\ \\

    \theutterance \stepcounter{utterance}  
    & & & \multicolumn{2}{p{0.3\linewidth}}{
        \cellcolor[rgb]{0.9,0.9,0.9}{
            \makecell[{{p{\linewidth}}}]{
                \texttt{\tiny{[GM$|$GM]}}
                \texttt{'I' erfolgreich nach (3,6) verschoben. Dein Raster sieht jetzt so aus:} \\
\texttt{```} \\
\texttt{ 1234567} \\
\texttt{╔═╤═╤═╤═╗} \\
\texttt{║T│◌│◌│◌║ 1} \\
\texttt{╟─┼─┤◌│◌║ 2} \\
\texttt{║W│◌│◌│H║ 3} \\
\texttt{║◌│◌├─┴─╢ 4} \\
\texttt{╟─┴─┤◌◌◌║ 5} \\
\texttt{║◌CI└─┬─╢ 6} \\
\texttt{║◌◌◌◌◌│◌║ 7} \\
\texttt{╚═════╧═╝} \\
\texttt{```} \\
            }
        }
    }
    & & \\ \\

    \theutterance \stepcounter{utterance}  
    & & \multicolumn{4}{p{0.6\linewidth}}{
        \cellcolor[rgb]{0.9,0.9,0.9}{
            \makecell[{{p{\linewidth}}}]{
                \texttt{\tiny{[GM$\rangle$P2]}}
                \texttt{Deine Nachricht wurde an deinen Mitspieler übermittelt.} \\
\\ 
\\ 
\texttt{Ihr seid gerade bei Runde 5 von maximal 20.} \\
\texttt{Ihr habt gemeinsam 2 von maximal 12 Strafpunkten angesammelt.} \\
\texttt{Dein Mitspieler hat ein Objekt auf seinem Raster bewegt.} \\
\\ 
\\ 
\texttt{Wie lautet dein nächster Befehl?} \\
            }
        }
    }
    & \\ \\

    \theutterance \stepcounter{utterance}  
    & & & \multicolumn{4}{p{0.6\linewidth}}{
        \cellcolor[rgb]{0.9,0.9,0.9}{
            \makecell[{{p{\linewidth}}}]{
                \texttt{\tiny{[GM$\langle$P2]}}
                \texttt{VERSCHIEBE: W, (4,3)} \\
            }
        }
    }
     \\ \\

    \theutterance \stepcounter{utterance}  
    & & & \multicolumn{2}{p{0.3\linewidth}}{
        \cellcolor[rgb]{0.9,0.9,0.9}{
            \makecell[{{p{\linewidth}}}]{
                \texttt{\tiny{[GM$|$GM]}}
                \texttt{VERSCHIEBE: W, (4,3)} \\
            }
        }
    }
    & & \\ \\

    \theutterance \stepcounter{utterance}  
    & & & \multicolumn{2}{p{0.3\linewidth}}{
        \cellcolor[rgb]{0.9,0.9,0.9}{
            \makecell[{{p{\linewidth}}}]{
                \texttt{\tiny{[GM$|$GM]}}
                \texttt{Strafpunkt: (4,3) ist nicht leer, sondern enthält '│'.} \\
\\ 
\texttt{Ihr habt gemeinsam 3 von maximal 12 Strafpunkten angesammelt.} \\
\texttt{Bitte versuche es noch mal!} \\
            }
        }
    }
    & & \\ \\

    \theutterance \stepcounter{utterance}  
    & & \multicolumn{4}{p{0.6\linewidth}}{
        \cellcolor[rgb]{0.9,0.9,0.9}{
            \makecell[{{p{\linewidth}}}]{
                \texttt{\tiny{[GM$\rangle$P2]}}
                \texttt{Strafpunkt: (4,3) ist nicht leer, sondern enthält '│'.} \\
\\ 
\texttt{Ihr habt gemeinsam 3 von maximal 12 Strafpunkten angesammelt.} \\
\texttt{Bitte versuche es noch mal!} \\
            }
        }
    }
    & \\ \\

    \theutterance \stepcounter{utterance}  
    & & & \multicolumn{4}{p{0.6\linewidth}}{
        \cellcolor[rgb]{0.9,0.9,0.9}{
            \makecell[{{p{\linewidth}}}]{
                \texttt{\tiny{[GM$\langle$P2]}}
                \texttt{VERSCHIEBE: W, (3,2)} \\
            }
        }
    }
     \\ \\

    \theutterance \stepcounter{utterance}  
    & & & \multicolumn{2}{p{0.3\linewidth}}{
        \cellcolor[rgb]{0.9,0.9,0.9}{
            \makecell[{{p{\linewidth}}}]{
                \texttt{\tiny{[GM$|$GM]}}
                \texttt{VERSCHIEBE: W, (3,2)} \\
            }
        }
    }
    & & \\ \\

    \theutterance \stepcounter{utterance}  
    & & & \multicolumn{2}{p{0.3\linewidth}}{
        \cellcolor[rgb]{0.9,0.9,0.9}{
            \makecell[{{p{\linewidth}}}]{
                \texttt{\tiny{[GM$|$GM]}}
                \texttt{Strafpunkt: (3,2) ist nicht leer, sondern enthält '─'.} \\
\\ 
\texttt{Ihr habt gemeinsam 4 von maximal 12 Strafpunkten angesammelt.} \\
\texttt{Bitte versuche es noch mal!} \\
            }
        }
    }
    & & \\ \\

    \theutterance \stepcounter{utterance}  
    & & \multicolumn{4}{p{0.6\linewidth}}{
        \cellcolor[rgb]{0.9,0.9,0.9}{
            \makecell[{{p{\linewidth}}}]{
                \texttt{\tiny{[GM$\rangle$P2]}}
                \texttt{Strafpunkt: (3,2) ist nicht leer, sondern enthält '─'.} \\
\\ 
\texttt{Ihr habt gemeinsam 4 von maximal 12 Strafpunkten angesammelt.} \\
\texttt{Bitte versuche es noch mal!} \\
            }
        }
    }
    & \\ \\

    \theutterance \stepcounter{utterance}  
    & & & \multicolumn{4}{p{0.6\linewidth}}{
        \cellcolor[rgb]{0.9,0.9,0.9}{
            \makecell[{{p{\linewidth}}}]{
                \texttt{\tiny{[GM$\langle$P2]}}
                \texttt{VERSCHIEBE: W, (3,2)  } \\
\texttt{*(Hinweis: Laut dem System ist (3,2) ein '─', also ein Wall. Dies könnte auf eine Diskrepanz zwischen der tatsächlichen Spielbrettdarstellung und der Kommunikation zurückzuführen sein. Alternativ könnte es sich um einen Fehler in der Koordinatenangabe handeln. Versuche, die Zielanordnung anzupassen oder andere Objekte zu verschieben, um den Weg für 'W' freizumachen.)*} \\
            }
        }
    }
     \\ \\

    \theutterance \stepcounter{utterance}  
    & & & \multicolumn{2}{p{0.3\linewidth}}{
        \cellcolor[rgb]{0.9,0.9,0.9}{
            \makecell[{{p{\linewidth}}}]{
                \texttt{\tiny{[GM$|$GM]}}
                \texttt{VERSCHIEBE: W, (3,2)  } \\
\texttt{*(Hinweis: Laut dem System ist (3,2) ein '─', also ein Wall. Dies könnte auf eine Diskrepanz zwischen der tatsächlichen Spielbrettdarstellung und der Kommunikation zurückzuführen sein. Alternativ könnte es sich um einen Fehler in der Koordinatenangabe handeln. Versuche, die Zielanordnung anzupassen oder andere Objekte zu verschieben, um den Weg für 'W' freizumachen.)*} \\
            }
        }
    }
    & & \\ \\

    \theutterance \stepcounter{utterance}  
    & & & \multicolumn{2}{p{0.3\linewidth}}{
        \cellcolor[rgb]{0.9,0.9,0.9}{
            \makecell[{{p{\linewidth}}}]{
                \texttt{\tiny{[GM$|$GM]}}
                \texttt{Invalid format: tail is not empty: '  } \\
\texttt{*(Hinweis: Laut dem System ist (3,2) ein '─', also ein Wall. Dies könnte auf eine Diskrepanz zwischen der tatsächlichen Spielbrettdarstellung und der Kommunikation zurückzuführen sein. Alternativ könnte es sich um einen Fehler in der Koordinatenangabe handeln. Versuche, die Zielanordnung anzupassen oder andere Objekte zu verschieben, um den Weg für 'W' freizumachen.)*} \\
            }
        }
    }
    & & \\ \\

    \theutterance \stepcounter{utterance}  
    & & \multicolumn{4}{p{0.6\linewidth}}{
        \cellcolor[rgb]{0.9,0.9,0.9}{
            \makecell[{{p{\linewidth}}}]{
                \texttt{\tiny{[GM$\rangle$P2]}}
                \texttt{Strafpunkt: Deine Nachricht darf nichts nach dem Befehl enthalten! Stelle sicher, dass deine Antwort ausschließlich einen der beiden Befehle `SPRICH: <NACHRICHT>` und `VERSCHIEBE: <OBJECT>, (<X>, <Y>)` und sonst nichts enthält!} \\
\\ 
\texttt{Ihr habt gemeinsam 5 von maximal 12 Strafpunkten angesammelt.} \\
\texttt{Bitte versuche es noch mal!} \\
            }
        }
    }
    & \\ \\

    \theutterance \stepcounter{utterance}  
    & & & \multicolumn{4}{p{0.6\linewidth}}{
        \cellcolor[rgb]{0.9,0.9,0.9}{
            \makecell[{{p{\linewidth}}}]{
                \texttt{\tiny{[GM$\langle$P2]}}
                \texttt{VERSCHIEBE: W, (4,3)} \\
            }
        }
    }
     \\ \\

    \theutterance \stepcounter{utterance}  
    & & & \multicolumn{2}{p{0.3\linewidth}}{
        \cellcolor[rgb]{0.9,0.9,0.9}{
            \makecell[{{p{\linewidth}}}]{
                \texttt{\tiny{[GM$|$GM]}}
                \texttt{VERSCHIEBE: W, (4,3)} \\
            }
        }
    }
    & & \\ \\

    \theutterance \stepcounter{utterance}  
    & & & \multicolumn{2}{p{0.3\linewidth}}{
        \cellcolor[rgb]{0.9,0.9,0.9}{
            \makecell[{{p{\linewidth}}}]{
                \texttt{\tiny{[GM$|$GM]}}
                \texttt{Strafpunkt: (4,3) ist nicht leer, sondern enthält '│'.} \\
\\ 
\texttt{Ihr habt gemeinsam 6 von maximal 12 Strafpunkten angesammelt.} \\
\texttt{Bitte versuche es noch mal!} \\
            }
        }
    }
    & & \\ \\

    \theutterance \stepcounter{utterance}  
    & & \multicolumn{4}{p{0.6\linewidth}}{
        \cellcolor[rgb]{0.9,0.9,0.9}{
            \makecell[{{p{\linewidth}}}]{
                \texttt{\tiny{[GM$\rangle$P2]}}
                \texttt{Strafpunkt: (4,3) ist nicht leer, sondern enthält '│'.} \\
\\ 
\texttt{Ihr habt gemeinsam 6 von maximal 12 Strafpunkten angesammelt.} \\
\texttt{Bitte versuche es noch mal!} \\
            }
        }
    }
    & \\ \\

    \theutterance \stepcounter{utterance}  
    & & & \multicolumn{4}{p{0.6\linewidth}}{
        \cellcolor[rgb]{0.9,0.9,0.9}{
            \makecell[{{p{\linewidth}}}]{
                \texttt{\tiny{[GM$\langle$P2]}}
                \texttt{VERSCHIEBE: W, (3,4)} \\
            }
        }
    }
     \\ \\

    \theutterance \stepcounter{utterance}  
    & & & \multicolumn{2}{p{0.3\linewidth}}{
        \cellcolor[rgb]{0.9,0.9,0.9}{
            \makecell[{{p{\linewidth}}}]{
                \texttt{\tiny{[GM$|$GM]}}
                \texttt{VERSCHIEBE: W, (3,4)} \\
            }
        }
    }
    & & \\ \\

    \theutterance \stepcounter{utterance}  
    & & & \multicolumn{2}{p{0.3\linewidth}}{
        \cellcolor[rgb]{0.9,0.9,0.9}{
            \makecell[{{p{\linewidth}}}]{
                \texttt{\tiny{[GM$|$GM]}}
                \texttt{Strafpunkt: (3,4) ist nicht leer, sondern enthält 'C'.} \\
\\ 
\texttt{Ihr habt gemeinsam 7 von maximal 12 Strafpunkten angesammelt.} \\
\texttt{Bitte versuche es noch mal!} \\
            }
        }
    }
    & & \\ \\

    \theutterance \stepcounter{utterance}  
    & & \multicolumn{4}{p{0.6\linewidth}}{
        \cellcolor[rgb]{0.9,0.9,0.9}{
            \makecell[{{p{\linewidth}}}]{
                \texttt{\tiny{[GM$\rangle$P2]}}
                \texttt{Strafpunkt: (3,4) ist nicht leer, sondern enthält 'C'.} \\
\\ 
\texttt{Ihr habt gemeinsam 7 von maximal 12 Strafpunkten angesammelt.} \\
\texttt{Bitte versuche es noch mal!} \\
            }
        }
    }
    & \\ \\

    \theutterance \stepcounter{utterance}  
    & & & \multicolumn{4}{p{0.6\linewidth}}{
        \cellcolor[rgb]{0.9,0.9,0.9}{
            \makecell[{{p{\linewidth}}}]{
                \texttt{\tiny{[GM$\langle$P2]}}
                \texttt{VERSCHIEBE: C, (4,4)} \\
            }
        }
    }
     \\ \\

    \theutterance \stepcounter{utterance}  
    & & & \multicolumn{2}{p{0.3\linewidth}}{
        \cellcolor[rgb]{0.9,0.9,0.9}{
            \makecell[{{p{\linewidth}}}]{
                \texttt{\tiny{[GM$|$GM]}}
                \texttt{VERSCHIEBE: C, (4,4)} \\
            }
        }
    }
    & & \\ \\

    \theutterance \stepcounter{utterance}  
    & & & \multicolumn{2}{p{0.3\linewidth}}{
        \cellcolor[rgb]{0.9,0.9,0.9}{
            \makecell[{{p{\linewidth}}}]{
                \texttt{\tiny{[GM$|$GM]}}
                \texttt{Strafpunkt: (4,4) ist nicht leer, sondern enthält '├'.} \\
\\ 
\texttt{Ihr habt gemeinsam 8 von maximal 12 Strafpunkten angesammelt.} \\
\texttt{Bitte versuche es noch mal!} \\
            }
        }
    }
    & & \\ \\

    \theutterance \stepcounter{utterance}  
    & & \multicolumn{4}{p{0.6\linewidth}}{
        \cellcolor[rgb]{0.9,0.9,0.9}{
            \makecell[{{p{\linewidth}}}]{
                \texttt{\tiny{[GM$\rangle$P2]}}
                \texttt{Strafpunkt: (4,4) ist nicht leer, sondern enthält '├'.} \\
\\ 
\texttt{Ihr habt gemeinsam 8 von maximal 12 Strafpunkten angesammelt.} \\
\texttt{Bitte versuche es noch mal!} \\
            }
        }
    }
    & \\ \\

    \theutterance \stepcounter{utterance}  
    & & & \multicolumn{4}{p{0.6\linewidth}}{
        \cellcolor[rgb]{0.9,0.9,0.9}{
            \makecell[{{p{\linewidth}}}]{
                \texttt{\tiny{[GM$\langle$P2]}}
                \texttt{VERSCHIEBE: W, (4,3)} \\
            }
        }
    }
     \\ \\

    \theutterance \stepcounter{utterance}  
    & & & \multicolumn{2}{p{0.3\linewidth}}{
        \cellcolor[rgb]{0.9,0.9,0.9}{
            \makecell[{{p{\linewidth}}}]{
                \texttt{\tiny{[GM$|$GM]}}
                \texttt{VERSCHIEBE: W, (4,3)} \\
            }
        }
    }
    & & \\ \\

    \theutterance \stepcounter{utterance}  
    & & & \multicolumn{2}{p{0.3\linewidth}}{
        \cellcolor[rgb]{0.9,0.9,0.9}{
            \makecell[{{p{\linewidth}}}]{
                \texttt{\tiny{[GM$|$GM]}}
                \texttt{Strafpunkt: (4,3) ist nicht leer, sondern enthält '│'.} \\
\\ 
\texttt{Ihr habt gemeinsam 9 von maximal 12 Strafpunkten angesammelt.} \\
\texttt{Bitte versuche es noch mal!} \\
            }
        }
    }
    & & \\ \\

    \theutterance \stepcounter{utterance}  
    & & \multicolumn{4}{p{0.6\linewidth}}{
        \cellcolor[rgb]{0.9,0.9,0.9}{
            \makecell[{{p{\linewidth}}}]{
                \texttt{\tiny{[GM$\rangle$P2]}}
                \texttt{Strafpunkt: (4,3) ist nicht leer, sondern enthält '│'.} \\
\\ 
\texttt{Ihr habt gemeinsam 9 von maximal 12 Strafpunkten angesammelt.} \\
\texttt{Bitte versuche es noch mal!} \\
            }
        }
    }
    & \\ \\

    \theutterance \stepcounter{utterance}  
    & & & \multicolumn{4}{p{0.6\linewidth}}{
        \cellcolor[rgb]{0.9,0.9,0.9}{
            \makecell[{{p{\linewidth}}}]{
                \texttt{\tiny{[GM$\langle$P2]}}
                \texttt{VERSCHIEBE: W, (3,4)} \\
            }
        }
    }
     \\ \\

    \theutterance \stepcounter{utterance}  
    & & & \multicolumn{2}{p{0.3\linewidth}}{
        \cellcolor[rgb]{0.9,0.9,0.9}{
            \makecell[{{p{\linewidth}}}]{
                \texttt{\tiny{[GM$|$GM]}}
                \texttt{VERSCHIEBE: W, (3,4)} \\
            }
        }
    }
    & & \\ \\

    \theutterance \stepcounter{utterance}  
    & & & \multicolumn{2}{p{0.3\linewidth}}{
        \cellcolor[rgb]{0.9,0.9,0.9}{
            \makecell[{{p{\linewidth}}}]{
                \texttt{\tiny{[GM$|$GM]}}
                \texttt{Strafpunkt: (3,4) ist nicht leer, sondern enthält 'C'.} \\
\\ 
\texttt{Ihr habt gemeinsam 10 von maximal 12 Strafpunkten angesammelt.} \\
\texttt{Bitte versuche es noch mal!} \\
            }
        }
    }
    & & \\ \\

    \theutterance \stepcounter{utterance}  
    & & \multicolumn{4}{p{0.6\linewidth}}{
        \cellcolor[rgb]{0.9,0.9,0.9}{
            \makecell[{{p{\linewidth}}}]{
                \texttt{\tiny{[GM$\rangle$P2]}}
                \texttt{Strafpunkt: (3,4) ist nicht leer, sondern enthält 'C'.} \\
\\ 
\texttt{Ihr habt gemeinsam 10 von maximal 12 Strafpunkten angesammelt.} \\
\texttt{Bitte versuche es noch mal!} \\
            }
        }
    }
    & \\ \\

    \theutterance \stepcounter{utterance}  
    & & & \multicolumn{4}{p{0.6\linewidth}}{
        \cellcolor[rgb]{0.9,0.9,0.9}{
            \makecell[{{p{\linewidth}}}]{
                \texttt{\tiny{[GM$\langle$P2]}}
                \texttt{VERSCHIEBE: C, (4,5)} \\
            }
        }
    }
     \\ \\

    \theutterance \stepcounter{utterance}  
    & & & \multicolumn{2}{p{0.3\linewidth}}{
        \cellcolor[rgb]{0.9,0.9,0.9}{
            \makecell[{{p{\linewidth}}}]{
                \texttt{\tiny{[GM$|$GM]}}
                \texttt{VERSCHIEBE: C, (4,5)} \\
            }
        }
    }
    & & \\ \\

    \theutterance \stepcounter{utterance}  
    & & & \multicolumn{2}{p{0.3\linewidth}}{
        \cellcolor[rgb]{0.9,0.9,0.9}{
            \makecell[{{p{\linewidth}}}]{
                \texttt{\tiny{[GM$|$GM]}}
                \texttt{Strafpunkt: (4,5) ist nicht leer, sondern enthält '┤'.} \\
\\ 
\texttt{Ihr habt gemeinsam 11 von maximal 12 Strafpunkten angesammelt.} \\
\texttt{Bitte versuche es noch mal!} \\
            }
        }
    }
    & & \\ \\

    \theutterance \stepcounter{utterance}  
    & & \multicolumn{4}{p{0.6\linewidth}}{
        \cellcolor[rgb]{0.9,0.9,0.9}{
            \makecell[{{p{\linewidth}}}]{
                \texttt{\tiny{[GM$\rangle$P2]}}
                \texttt{Strafpunkt: (4,5) ist nicht leer, sondern enthält '┤'.} \\
\\ 
\texttt{Ihr habt gemeinsam 11 von maximal 12 Strafpunkten angesammelt.} \\
\texttt{Bitte versuche es noch mal!} \\
            }
        }
    }
    & \\ \\

    \theutterance \stepcounter{utterance}  
    & & & \multicolumn{4}{p{0.6\linewidth}}{
        \cellcolor[rgb]{0.9,0.9,0.9}{
            \makecell[{{p{\linewidth}}}]{
                \texttt{\tiny{[GM$\langle$P2]}}
                \texttt{VERSCHIEBE: W, (3,4)} \\
            }
        }
    }
     \\ \\

    \theutterance \stepcounter{utterance}  
    & & & \multicolumn{2}{p{0.3\linewidth}}{
        \cellcolor[rgb]{0.9,0.9,0.9}{
            \makecell[{{p{\linewidth}}}]{
                \texttt{\tiny{[GM$|$GM]}}
                \texttt{VERSCHIEBE: W, (3,4)} \\
            }
        }
    }
    & & \\ \\

    \theutterance \stepcounter{utterance}  
    & & & \multicolumn{2}{p{0.3\linewidth}}{
        \cellcolor[rgb]{0.9,0.9,0.9}{
            \makecell[{{p{\linewidth}}}]{
                \texttt{\tiny{[GM$|$GM]}}
                \texttt{Strafpunkt: (3,4) ist nicht leer, sondern enthält 'C'.} \\
\\ 
\texttt{Ihr habt gemeinsam 12 von maximal 12 Strafpunkten angesammelt.} \\
\texttt{Bitte versuche es noch mal!} \\
            }
        }
    }
    & & \\ \\

    \theutterance \stepcounter{utterance}  
    & & \multicolumn{4}{p{0.6\linewidth}}{
        \cellcolor[rgb]{0.9,0.9,0.9}{
            \makecell[{{p{\linewidth}}}]{
                \texttt{\tiny{[GM$\rangle$P2]}}
                \texttt{Strafpunkt: (3,4) ist nicht leer, sondern enthält 'C'.} \\
\\ 
\texttt{Ihr habt gemeinsam 12 von maximal 12 Strafpunkten angesammelt.} \\
\texttt{Bitte versuche es noch mal!} \\
            }
        }
    }
    & \\ \\

    \theutterance \stepcounter{utterance}  
    & & & \multicolumn{4}{p{0.6\linewidth}}{
        \cellcolor[rgb]{0.9,0.9,0.9}{
            \makecell[{{p{\linewidth}}}]{
                \texttt{\tiny{[GM$\langle$P2]}}
                \texttt{VERSCHIEBE: W, (2,3)} \\
            }
        }
    }
     \\ \\

    \theutterance \stepcounter{utterance}  
    & & & \multicolumn{2}{p{0.3\linewidth}}{
        \cellcolor[rgb]{0.9,0.9,0.9}{
            \makecell[{{p{\linewidth}}}]{
                \texttt{\tiny{[GM$|$GM]}}
                \texttt{VERSCHIEBE: W, (2,3)} \\
            }
        }
    }
    & & \\ \\

    \theutterance \stepcounter{utterance}  
    & & & \multicolumn{2}{p{0.3\linewidth}}{
        \cellcolor[rgb]{0.9,0.9,0.9}{
            \makecell[{{p{\linewidth}}}]{
                \texttt{\tiny{[GM$|$GM]}}
                \texttt{Strafpunkt: (2,3) ist nicht leer, sondern enthält '│'.} \\
\\ 
\texttt{Ihr habt gemeinsam 13 von maximal 12 Strafpunkten angesammelt.} \\
\texttt{Bitte versuche es noch mal!} \\
            }
        }
    }
    & & \\ \\

    \theutterance \stepcounter{utterance}  
    & & & \multicolumn{2}{p{0.3\linewidth}}{
        \cellcolor[rgb]{0.9,0.9,0.9}{
            \makecell[{{p{\linewidth}}}]{
                \texttt{\tiny{[GM$|$GM]}}
                \texttt{Maximum number of penalties exceeded} \\
            }
        }
    }
    & & \\ \\

    \theutterance \stepcounter{utterance}  
    & & & \multicolumn{2}{p{0.3\linewidth}}{
        \cellcolor[rgb]{0.9,0.9,0.9}{
            \makecell[{{p{\linewidth}}}]{
                \texttt{\tiny{[GM$|$GM]}}
                \texttt{* success: False} \\
\texttt{* lose: True} \\
\texttt{* aborted: True} \\
\texttt{{-}{-}{-}{-}{-}{-}{-}} \\
\texttt{* Shifts: 0.00} \\
\texttt{* Max Shifts: 8.00} \\
\texttt{* Min Shifts: 4.00} \\
\texttt{* End Distance Sum: 18.63} \\
\texttt{* Init Distance Sum: 19.39} \\
\texttt{* Expected Distance Sum: 20.95} \\
\texttt{* Penalties: 13.00} \\
\texttt{* Max Penalties: 12.00} \\
\texttt{* Rounds: 4.00} \\
\texttt{* Max Rounds: 20.00} \\
\texttt{* Object Count: 5.00} \\
            }
        }
    }
    & & \\ \\

    \theutterance \stepcounter{utterance}  
    & & & \multicolumn{2}{p{0.3\linewidth}}{
        \cellcolor[rgb]{0.9,0.9,0.9}{
            \makecell[{{p{\linewidth}}}]{
                \texttt{\tiny{[GM$|$GM]}}
                \texttt{Initial Grids:} \\
\texttt{```} \\
\texttt{Player 1:     Player 2:} \\
\texttt{ 1234567       1234567} \\
\texttt{╔═╤═╤═╤═╗     ╔═╤═╤═╤═╗} \\
\texttt{║T│◌│◌│◌║ 1   ║◌│◌│◌│◌║ 1} \\
\texttt{╟─┼─┤I│◌║ 2   ╟─┼─┤◌│◌║ 2} \\
\texttt{║W│◌│◌│H║ 3   ║◌│W│◌│◌║ 3} \\
\texttt{║◌│◌├─┴─╢ 4   ║◌│C├─┴─╢ 4} \\
\texttt{╟─┴─┤◌◌◌║ 5   ╟─┴─┤I◌◌║ 5} \\
\texttt{║◌C◌└─┬─╢ 6   ║◌H◌└─┬─╢ 6} \\
\texttt{║◌◌◌◌◌│◌║ 7   ║◌◌T◌◌│◌║ 7} \\
\texttt{╚═════╧═╝     ╚═════╧═╝} \\
\texttt{```} \\
\\ 
\texttt{Final Grids:} \\
\texttt{```} \\
\texttt{Player 1:     Player 2:} \\
\texttt{ 1234567       1234567} \\
\texttt{╔═╤═╤═╤═╗     ╔═╤═╤═╤═╗} \\
\texttt{║T│◌│◌│◌║ 1   ║◌│◌│◌│◌║ 1} \\
\texttt{╟─┼─┤◌│◌║ 2   ╟─┼─┤◌│◌║ 2} \\
\texttt{║W│◌│◌│H║ 3   ║◌│W│◌│◌║ 3} \\
\texttt{║◌│◌├─┴─╢ 4   ║◌│C├─┴─╢ 4} \\
\texttt{╟─┴─┤◌◌◌║ 5   ╟─┴─┤I◌◌║ 5} \\
\texttt{║◌CI└─┬─╢ 6   ║◌H◌└─┬─╢ 6} \\
\texttt{║◌◌◌◌◌│◌║ 7   ║◌◌T◌◌│◌║ 7} \\
\texttt{╚═════╧═╝     ╚═════╧═╝} \\
\texttt{```} \\
            }
        }
    }
    & & \\ \\
=======
    \hline 

    \theutterance \stepcounter{utterance}  

    & & \multicolumn{4}{p{0.6\linewidth}}{\cellcolor[rgb]{0.9,0.9,0.9}{%
	\makecell[{{p{\linewidth}}}]{% 
	  \tt {\tiny [A$\langle$GM]}  
	 Ich bin der Spielleiter, und du spielst ein kooperatives Spiel mit dem folgenden Raster als Spielbrett:\\ \tt \\ \tt ```\\ \tt  1234567\\ \tt ╔═╤═╤═╤═╗\\ \tt ║T│◌│◌│◌║ 1\\ \tt ╟─┼─┤I│◌║ 2\\ \tt ║W│◌│◌│H║ 3\\ \tt ║◌│◌├─┴─╢ 4\\ \tt ╟─┴─┤◌◌◌║ 5\\ \tt ║◌C◌└─┬─╢ 6\\ \tt ║◌◌◌◌◌│◌║ 7\\ \tt ╚═════╧═╝\\ \tt \\ \tt ```\\ \tt \\ \tt * Am oberen Rand sind die x-Koordinaten angegeben, die nach rechts hin zunehmen, und am rechten Rand die y-Koordinaten, die nach unten zunehmen.\\ \tt * Die folgenden Objekte sind zufällig auf deinem Raster platziert: 'W', 'I', 'T', 'C', 'H'.\\ \tt \\ \tt \\ \tt Dein Mitspieler sieht eine Variation des Spielbretts, auf dem dieselben Objekte an anderen zufälligen Positionen platziert sind. Du kannst das Spielbrett des anderen Spielers nicht sehen, und er kann deins nicht sehen.\\ \tt \\ \tt \\ \tt **Ziel des Spiels:**\\ \tt \\ \tt Beide Spieler müssen die Objekte auf ihrem jeweiligen Spielbrett so bewegen, dass identische Objekte am Ende an denselben Koordinaten liegen. Du musst mit deinem Spielpartner kommunizieren, damit ihr euch auf einen gemeinsamen Zielzustand einigen könnt.\\ \tt \\ \tt \\ \tt **Spielregeln:**\\ \tt \\ \tt * In jedem Spielzug kannst du genau eine der folgenden zwei Aktionen ausführen:\\ \tt 1. `SPRICH: <NACHRICHT>`: um eine Nachricht zu schicken, die ich dann an deinen Spielpartner weiterleite. Als Nachricht gilt alles bis zum nächsten Zeilenumbruch.\\ \tt 2. `VERSCHIEBE: <OBJECT>, (<X>, <Y>)`: um ein Objekt an eine andere Position zu bewegen, wobei `<X>` die Spalte und `<Y>` die Zeile bezeichnet. Ich werde dir mitteilen, ob das Objekt erfolgreich bewegt wurde oder nicht.\\ \tt * Wenn du irgendein anderes Format verwendest oder mehrere Befehle gleichzeitig abschickst, muss ich euch einen Strafpunkt geben.\\ \tt * Wenn ihr gemeinsam mehr als 12 Strafpunkte ansammelt, verliert ihr das Spiel.\\ \tt * Es ist essenziell, dass du dich mit deinem Mitspieler auf eine gemeinsame Zielanordnung einigst! Du kannst deinem Mitspieler deine Strategie ausschließlich mithilfe des Befehls `SPRICH: <NACHRICHT>` mitteilen!\\ \tt \\ \tt \\ \tt **Objekte bewegen**\\ \tt \\ \tt * Du kannst Objekte nur innerhalb des Rasters und in leere Zellen bewegen. Leere Zellen erkennst am Zeichen '◌'.\\ \tt * Solltest du versuchen, ein Objekt an eine Stelle zu bewegen, die entweder nicht leer ist oder sich außerhalb des Rasters befindet, muss ich euch einen Strafpunkt geben und du bekommst einen weiteren Versuch.\\ \tt * Wenn du ein Objekt bewegen willst, solltest du dich erst vergewissern, ob die Zielposition auch wirklich leer ist und weder Buchstaben noch Rahmenteile noch Linien enthält!\\ \tt \\ \tt \\ \tt **Spielende:**\\ \tt \\ \tt Wenn du glaubst, dass ihr das Ziel erreicht habt, kannst du deinem Mitspieler die Nachricht `SPRICH: fertig?` schicken. Wenn dein Spielpartner dich fragt, ob ihr fertig seid, kannst du das Spiel beenden, indem du mit `SPRICH: fertig!` antwortest.\\ \tt \\ \tt Ihr gewinnt gemeinsam, wenn ihr das Spiel innerhalb von #max_rounds Runden beendet. Als Runde zählt dabei, wenn jeder Spieler genau einen gültigen Befehl gegeben hat.\\ \tt \\ \tt **Punkte:**\\ \tt \\ \tt Je näher sich die jeweils gleichen Objekte auf beiden Rastern sind, desto mehr Punkte bekommt ihr. Für Strafpunkte gibt es Punktabzug. Könnt ihr den Rekord brechen?\\ \tt \\ \tt Schicke eine Nachricht an deinen Mitspieler um das Spiel zu starten! 
	  } 
	   } 
	   } 
	 & \\ 
 

    \theutterance \stepcounter{utterance}  

    & \multicolumn{4}{p{0.6\linewidth}}{\cellcolor[rgb]{0.8,1,0.9}{%
	\makecell[{{p{\linewidth}}}]{% 
	  \tt {\tiny [A$\rangle$GM]}  
	 SPRICH: Hallo! Ich bin bereit, das Spiel zu starten. Bitte sende mir deine aktuelle Spielbrett-Konfiguration, damit wir uns auf eine gemeinsame Strategie einigen können. 😊 
	  } 
	   } 
	   } 
	 & & \\ 
 

    \theutterance \stepcounter{utterance}  

    & & & \multicolumn{2}{p{0.3\linewidth}}{\cellcolor[rgb]{0.95,0.95,0.95}{%
	\makecell[{{p{\linewidth}}}]{% 
	  \tt {\tiny [GM$|$GM]}  
	 SPRICH: Hallo! Ich bin bereit, das Spiel zu starten. Bitte sende mir deine aktuelle Spielbrett-Konfiguration, damit wir uns auf eine gemeinsame Strategie einigen können. 😊 
	  } 
	   } 
	   } 
	 & & \\ 
 

    \theutterance \stepcounter{utterance}  

    & & \multicolumn{4}{p{0.6\linewidth}}{\cellcolor[rgb]{0.9,0.9,0.9}{%
	\makecell[{{p{\linewidth}}}]{% 
	  \tt {\tiny [GM$\rangle$B]}  
	 Ich bin der Spielleiter, und du spielst ein kooperatives Spiel mit dem folgenden Raster als Spielbrett:\\ \tt \\ \tt ```\\ \tt  1234567\\ \tt ╔═╤═╤═╤═╗\\ \tt ║◌│◌│◌│◌║ 1\\ \tt ╟─┼─┤◌│◌║ 2\\ \tt ║◌│W│◌│◌║ 3\\ \tt ║◌│C├─┴─╢ 4\\ \tt ╟─┴─┤I◌◌║ 5\\ \tt ║◌H◌└─┬─╢ 6\\ \tt ║◌◌T◌◌│◌║ 7\\ \tt ╚═════╧═╝\\ \tt \\ \tt ```\\ \tt \\ \tt * Am oberen Rand sind die x-Koordinaten angegeben, die nach rechts hin zunehmen, und am rechten Rand die y-Koordinaten, die nach unten zunehmen.\\ \tt * Die folgenden Objekte sind zufällig auf deinem Raster platziert: 'W', 'I', 'T', 'C', 'H'.\\ \tt \\ \tt \\ \tt Dein Mitspieler sieht eine Variation des Spielbretts, auf dem dieselben Objekte an anderen zufälligen Positionen platziert sind. Du kannst das Spielbrett des anderen Spielers nicht sehen, und er kann deins nicht sehen.\\ \tt \\ \tt \\ \tt **Ziel des Spiels:**\\ \tt \\ \tt Beide Spieler müssen die Objekte auf ihrem jeweiligen Spielbrett so bewegen, dass identische Objekte am Ende an denselben Koordinaten liegen. Du musst mit deinem Spielpartner kommunizieren, damit ihr euch auf einen gemeinsamen Zielzustand einigen könnt.\\ \tt \\ \tt \\ \tt **Spielregeln:**\\ \tt \\ \tt * In jedem Spielzug kannst du genau eine der folgenden zwei Aktionen ausführen:\\ \tt 1. `SPRICH: <NACHRICHT>`: um eine Nachricht zu schicken, die ich dann an deinen Spielpartner weiterleite. Als Nachricht gilt alles bis zum nächsten Zeilenumbruch.\\ \tt 2. `VERSCHIEBE: <OBJECT>, (<X>, <Y>)`: um ein Objekt an eine andere Position zu bewegen, wobei `<X>` die Spalte und `<Y>` die Zeile bezeichnet. Ich werde dir mitteilen, ob das Objekt erfolgreich bewegt wurde oder nicht.\\ \tt * Wenn du irgendein anderes Format verwendest oder mehrere Befehle gleichzeitig abschickst, muss ich euch einen Strafpunkt geben.\\ \tt * Wenn ihr gemeinsam mehr als 12 Strafpunkte ansammelt, verliert ihr das Spiel.\\ \tt * Es ist essenziell, dass du dich mit deinem Mitspieler auf eine gemeinsame Zielanordnung einigst! Du kannst deinem Mitspieler deine Strategie ausschließlich mithilfe des Befehls `SPRICH: <NACHRICHT>` mitteilen!\\ \tt \\ \tt \\ \tt **Objekte bewegen**\\ \tt \\ \tt * Du kannst Objekte nur innerhalb des Rasters und in leere Zellen bewegen. Leere Zellen erkennst am Zeichen '◌'.\\ \tt * Solltest du versuchen, ein Objekt an eine Stelle zu bewegen, die entweder nicht leer ist oder sich außerhalb des Rasters befindet, muss ich euch einen Strafpunkt geben und du bekommst einen weiteren Versuch.\\ \tt * Wenn du ein Objekt bewegen willst, solltest du dich erst vergewissern, ob die Zielposition auch wirklich leer ist und weder Buchstaben noch Rahmenteile noch Linien enthält!\\ \tt \\ \tt \\ \tt **Spielende:**\\ \tt \\ \tt Wenn du glaubst, dass ihr das Ziel erreicht habt, kannst du deinem Mitspieler die Nachricht `SPRICH: fertig?` schicken. Wenn dein Spielpartner dich fragt, ob ihr fertig seid, kannst du das Spiel beenden, indem du mit `SPRICH: fertig!` antwortest.\\ \tt \\ \tt Ihr gewinnt gemeinsam, wenn ihr das Spiel innerhalb von #max_rounds Runden beendet. Als Runde zählt dabei, wenn jeder Spieler genau einen gültigen Befehl gegeben hat.\\ \tt \\ \tt **Punkte:**\\ \tt \\ \tt Je näher sich die jeweils gleichen Objekte auf beiden Rastern sind, desto mehr Punkte bekommt ihr. Für Strafpunkte gibt es Punktabzug. Könnt ihr den Rekord brechen?\\ \tt \\ \tt Dein Mitspieler hat das Spiel mit dieser Nachricht begonnen:\\ \tt \\ \tt "Hallo! Ich bin bereit, das Spiel zu starten. Bitte sende mir deine aktuelle Spielbrett-Konfiguration, damit wir uns auf eine gemeinsame Strategie einigen können. 😊"\\ \tt \\ \tt Wie lautet dein erster Befehl? 
	  } 
	   } 
	   } 
	 & \\ 
 

    \theutterance \stepcounter{utterance}  

    & & & \multicolumn{4}{p{0.6\linewidth}}{\cellcolor[rgb]{1,0.85,0.72}{%
	\makecell[{{p{\linewidth}}}]{% 
	  \tt {\tiny [GM$\langle$B]}  
	 SPRICH: Hallo! Ich bin bereit, das Spiel zu starten. Bitte sende mir deine aktuelle Spielbrett-Konfiguration, damit wir uns auf eine gemeinsame Strategie einigen können. 😊 
	  } 
	   } 
	   } 
	  \\ 
 

    \theutterance \stepcounter{utterance}  

    & & & \multicolumn{2}{p{0.3\linewidth}}{\cellcolor[rgb]{0.95,0.95,0.95}{%
	\makecell[{{p{\linewidth}}}]{% 
	  \tt {\tiny [GM$|$GM]}  
	 SPRICH: Hallo! Ich bin bereit, das Spiel zu starten. Bitte sende mir deine aktuelle Spielbrett-Konfiguration, damit wir uns auf eine gemeinsame Strategie einigen können. 😊 
	  } 
	   } 
	   } 
	 & & \\ 
 

    \theutterance \stepcounter{utterance}  

    & & \multicolumn{4}{p{0.6\linewidth}}{\cellcolor[rgb]{0.9,0.9,0.9}{%
	\makecell[{{p{\linewidth}}}]{% 
	  \tt {\tiny [A$\langle$GM]}  
	 Deine Nachricht wurde an deinen Mitspieler übermittelt.\\ \tt \\ \tt \\ \tt Ihr seid gerade bei Runde 1 von maximal 20.\\ \tt Ihr habt gemeinsam 0 von maximal 12 Strafpunkten angesammelt.\\ \tt Dein Mitspieler hat dir diese Nachricht geschickt:\\ \tt \\ \tt \\ \tt "Hallo! Ich bin bereit, das Spiel zu starten. Bitte sende mir deine aktuelle Spielbrett-Konfiguration, damit wir uns auf eine gemeinsame Strategie einigen können. 😊"\\ \tt \\ \tt \\ \tt Wie lautet dein nächster Befehl? 
	  } 
	   } 
	   } 
	 & \\ 
 

    \theutterance \stepcounter{utterance}  

    & \multicolumn{4}{p{0.6\linewidth}}{\cellcolor[rgb]{0.8,1,0.9}{%
	\makecell[{{p{\linewidth}}}]{% 
	  \tt {\tiny [A$\rangle$GM]}  
	 SPRICH: Hallo! Ich bin bereit, das Spiel zu starten. Bitte sende mir deine aktuelle Spielbrett-Konfiguration, damit wir uns auf eine gemeinsame Strategie einigen können. 😊 
	  } 
	   } 
	   } 
	 & & \\ 
 

    \theutterance \stepcounter{utterance}  

    & & & \multicolumn{2}{p{0.3\linewidth}}{\cellcolor[rgb]{0.95,0.95,0.95}{%
	\makecell[{{p{\linewidth}}}]{% 
	  \tt {\tiny [GM$|$GM]}  
	 SPRICH: Hallo! Ich bin bereit, das Spiel zu starten. Bitte sende mir deine aktuelle Spielbrett-Konfiguration, damit wir uns auf eine gemeinsame Strategie einigen können. 😊 
	  } 
	   } 
	   } 
	 & & \\ 
 

    \theutterance \stepcounter{utterance}  

    & & \multicolumn{4}{p{0.6\linewidth}}{\cellcolor[rgb]{0.9,0.9,0.9}{%
	\makecell[{{p{\linewidth}}}]{% 
	  \tt {\tiny [GM$\rangle$B]}  
	 Deine Nachricht wurde an deinen Mitspieler übermittelt.\\ \tt \\ \tt \\ \tt Ihr seid gerade bei Runde 2 von maximal 20.\\ \tt Ihr habt gemeinsam 0 von maximal 12 Strafpunkten angesammelt.\\ \tt Dein Mitspieler hat dir diese Nachricht geschickt:\\ \tt \\ \tt \\ \tt "Hallo! Ich bin bereit, das Spiel zu starten. Bitte sende mir deine aktuelle Spielbrett-Konfiguration, damit wir uns auf eine gemeinsame Strategie einigen können. 😊"\\ \tt \\ \tt \\ \tt Wie lautet dein nächster Befehl? 
	  } 
	   } 
	   } 
	 & \\ 
 

    \theutterance \stepcounter{utterance}  

    & & & \multicolumn{4}{p{0.6\linewidth}}{\cellcolor[rgb]{1,0.85,0.72}{%
	\makecell[{{p{\linewidth}}}]{% 
	  \tt {\tiny [GM$\langle$B]}  
	 SPRICH: Hier ist mein aktuelles Spielbrett:  \\ \tt ```\\ \tt  1234567\\ \tt ╔═╤═╤═╤═╗\\ \tt ║◌│◌│◌│◌║ 1\\ \tt ╟─┼─┤◌│◌║ 2\\ \tt ║◌│W│◌│◌║ 3\\ \tt ║◌│C├─┴─╢ 4\\ \tt ╟─┴─┤I◌◌║ 5\\ \tt ║◌H◌└─┬─╢ 6\\ \tt ║◌◌T◌◌│◌║ 7\\ \tt ╚═════╧═╝\\ \tt ```  \\ \tt Bitte sende mir deine aktuelle Konfiguration, damit wir eine Strategie planen können. 😊 
	  } 
	   } 
	   } 
	  \\ 
 

    \theutterance \stepcounter{utterance}  

    & & & \multicolumn{2}{p{0.3\linewidth}}{\cellcolor[rgb]{0.95,0.95,0.95}{%
	\makecell[{{p{\linewidth}}}]{% 
	  \tt {\tiny [GM$|$GM]}  
	 SPRICH: Hier ist mein aktuelles Spielbrett:  \\ \tt ```\\ \tt  1234567\\ \tt ╔═╤═╤═╤═╗\\ \tt ║◌│◌│◌│◌║ 1\\ \tt ╟─┼─┤◌│◌║ 2\\ \tt ║◌│W│◌│◌║ 3\\ \tt ║◌│C├─┴─╢ 4\\ \tt ╟─┴─┤I◌◌║ 5\\ \tt ║◌H◌└─┬─╢ 6\\ \tt ║◌◌T◌◌│◌║ 7\\ \tt ╚═════╧═╝\\ \tt ```  \\ \tt Bitte sende mir deine aktuelle Konfiguration, damit wir eine Strategie planen können. 😊 
	  } 
	   } 
	   } 
	 & & \\ 
 

    \theutterance \stepcounter{utterance}  

    & & & \multicolumn{2}{p{0.3\linewidth}}{\cellcolor[rgb]{0.95,0.95,0.95}{%
	\makecell[{{p{\linewidth}}}]{% 
	  \tt {\tiny [GM$|$GM]}  
	 Invalid format: tail is not empty: '\\ \tt \\ \tt  1234567\\ \tt ╔═╤═╤═╤═╗\\ \tt ║◌│◌│◌│◌║ 1\\ \tt ╟─┼─┤◌│◌║ 2\\ \tt ║◌│W│◌│◌║ 3\\ \tt ║◌│C├─┴─╢ 4\\ \tt ╟─┴─┤I◌◌║ 5\\ \tt ║◌H◌└─┬─╢ 6\\ \tt ║◌◌T◌◌│◌║ 7\\ \tt ╚═════╧═╝\\ \tt   \\ \tt Bitte sende mir deine aktuelle Konfiguration, damit wir eine Strategie planen können. 😊 
	  } 
	   } 
	   } 
	 & & \\ 
 

    \theutterance \stepcounter{utterance}  

    & & \multicolumn{4}{p{0.6\linewidth}}{\cellcolor[rgb]{0.9,0.9,0.9}{%
	\makecell[{{p{\linewidth}}}]{% 
	  \tt {\tiny [GM$\rangle$B]}  
	 Strafpunkt: Deine Nachricht darf nichts nach dem Befehl enthalten! Stelle sicher, dass deine Antwort ausschließlich einen der beiden Befehle `SPRICH: <NACHRICHT>` und `VERSCHIEBE: <OBJECT>, (<X>, <Y>)` und sonst nichts enthält!\\ \tt \\ \tt Ihr habt gemeinsam 1 von maximal 12 Strafpunkten angesammelt.\\ \tt Bitte versuche es noch mal! 
	  } 
	   } 
	   } 
	 & \\ 
 

    \theutterance \stepcounter{utterance}  

    & & & \multicolumn{4}{p{0.6\linewidth}}{\cellcolor[rgb]{1,0.85,0.72}{%
	\makecell[{{p{\linewidth}}}]{% 
	  \tt {\tiny [GM$\langle$B]}  
	 SPRICH: Hier ist mein aktuelles Spielbrett:  \\ \tt ```\\ \tt  1234567\\ \tt ╔═╤═╤═╤═╗\\ \tt ║◌│◌│◌│◌║ 1\\ \tt ╟─┼─┤◌│◌║ 2\\ \tt ║◌│W│◌│◌║ 3\\ \tt ║◌│C├─┴─╢ 4\\ \tt ╟─┴─┤I◌◌║ 5\\ \tt ║◌H◌└─┬─╢ 6\\ \tt ║◌◌T◌◌│◌║ 7\\ \tt ╚═════╧═╝\\ \tt ``` 
	  } 
	   } 
	   } 
	  \\ 
 

    \theutterance \stepcounter{utterance}  

    & & & \multicolumn{2}{p{0.3\linewidth}}{\cellcolor[rgb]{0.95,0.95,0.95}{%
	\makecell[{{p{\linewidth}}}]{% 
	  \tt {\tiny [GM$|$GM]}  
	 SPRICH: Hier ist mein aktuelles Spielbrett:  \\ \tt ```\\ \tt  1234567\\ \tt ╔═╤═╤═╤═╗\\ \tt ║◌│◌│◌│◌║ 1\\ \tt ╟─┼─┤◌│◌║ 2\\ \tt ║◌│W│◌│◌║ 3\\ \tt ║◌│C├─┴─╢ 4\\ \tt ╟─┴─┤I◌◌║ 5\\ \tt ║◌H◌└─┬─╢ 6\\ \tt ║◌◌T◌◌│◌║ 7\\ \tt ╚═════╧═╝\\ \tt ``` 
	  } 
	   } 
	   } 
	 & & \\ 
 

    \theutterance \stepcounter{utterance}  

    & & & \multicolumn{2}{p{0.3\linewidth}}{\cellcolor[rgb]{0.95,0.95,0.95}{%
	\makecell[{{p{\linewidth}}}]{% 
	  \tt {\tiny [GM$|$GM]}  
	 Invalid format: tail is not empty: '\\ \tt \\ \tt  1234567\\ \tt ╔═╤═╤═╤═╗\\ \tt ║◌│◌│◌│◌║ 1\\ \tt ╟─┼─┤◌│◌║ 2\\ \tt ║◌│W│◌│◌║ 3\\ \tt ║◌│C├─┴─╢ 4\\ \tt ╟─┴─┤I◌◌║ 5\\ \tt ║◌H◌└─┬─╢ 6\\ \tt ║◌◌T◌◌│◌║ 7\\ \tt ╚═════╧═╝ 
	  } 
	   } 
	   } 
	 & & \\ 
 

    \theutterance \stepcounter{utterance}  

    & & \multicolumn{4}{p{0.6\linewidth}}{\cellcolor[rgb]{0.9,0.9,0.9}{%
	\makecell[{{p{\linewidth}}}]{% 
	  \tt {\tiny [GM$\rangle$B]}  
	 Strafpunkt: Deine Nachricht darf nichts nach dem Befehl enthalten! Stelle sicher, dass deine Antwort ausschließlich einen der beiden Befehle `SPRICH: <NACHRICHT>` und `VERSCHIEBE: <OBJECT>, (<X>, <Y>)` und sonst nichts enthält!\\ \tt \\ \tt Ihr habt gemeinsam 2 von maximal 12 Strafpunkten angesammelt.\\ \tt Bitte versuche es noch mal! 
	  } 
	   } 
	   } 
	 & \\ 
 

    \theutterance \stepcounter{utterance}  

    & & & \multicolumn{4}{p{0.6\linewidth}}{\cellcolor[rgb]{1,0.85,0.72}{%
	\makecell[{{p{\linewidth}}}]{% 
	  \tt {\tiny [GM$\langle$B]}  
	 SPRICH: Hier ist mein aktuelles Spielbrett:  \\ \tt ```\\ \tt  1234567\\ \tt ╔═╤═╤═╤═╗\\ \tt ║◌│◌│◌│◌║ 1\\ \tt ╟─┼─┤◌│◌║ 2\\ \tt ║◌│W│◌│◌║ 3\\ \tt ║◌│C├─┴─╢ 4\\ \tt ╟─┴─┤I◌◌║ 5\\ \tt ║◌H◌└─┬─╢ 6\\ \tt ║◌◌T◌◌│◌║ 7\\ \tt ╚═════╧═╝\\ \tt ``` 
	  } 
	   } 
	   } 
	  \\ 
 

    \theutterance \stepcounter{utterance}  

    & & & \multicolumn{2}{p{0.3\linewidth}}{\cellcolor[rgb]{0.95,0.95,0.95}{%
	\makecell[{{p{\linewidth}}}]{% 
	  \tt {\tiny [GM$|$GM]}  
	 SPRICH: Hier ist mein aktuelles Spielbrett:  \\ \tt ```\\ \tt  1234567\\ \tt ╔═╤═╤═╤═╗\\ \tt ║◌│◌│◌│◌║ 1\\ \tt ╟─┼─┤◌│◌║ 2\\ \tt ║◌│W│◌│◌║ 3\\ \tt ║◌│C├─┴─╢ 4\\ \tt ╟─┴─┤I◌◌║ 5\\ \tt ║◌H◌└─┬─╢ 6\\ \tt ║◌◌T◌◌│◌║ 7\\ \tt ╚═════╧═╝\\ \tt ``` 
	  } 
	   } 
	   } 
	 & & \\ 
 

    \theutterance \stepcounter{utterance}  

    & & & \multicolumn{2}{p{0.3\linewidth}}{\cellcolor[rgb]{0.95,0.95,0.95}{%
	\makecell[{{p{\linewidth}}}]{% 
	  \tt {\tiny [GM$|$GM]}  
	 Invalid format: tail is not empty: '\\ \tt \\ \tt  1234567\\ \tt ╔═╤═╤═╤═╗\\ \tt ║◌│◌│◌│◌║ 1\\ \tt ╟─┼─┤◌│◌║ 2\\ \tt ║◌│W│◌│◌║ 3\\ \tt ║◌│C├─┴─╢ 4\\ \tt ╟─┴─┤I◌◌║ 5\\ \tt ║◌H◌└─┬─╢ 6\\ \tt ║◌◌T◌◌│◌║ 7\\ \tt ╚═════╧═╝ 
	  } 
	   } 
	   } 
	 & & \\ 
 

    \theutterance \stepcounter{utterance}  

    & & \multicolumn{4}{p{0.6\linewidth}}{\cellcolor[rgb]{0.9,0.9,0.9}{%
	\makecell[{{p{\linewidth}}}]{% 
	  \tt {\tiny [GM$\rangle$B]}  
	 Strafpunkt: Deine Nachricht darf nichts nach dem Befehl enthalten! Stelle sicher, dass deine Antwort ausschließlich einen der beiden Befehle `SPRICH: <NACHRICHT>` und `VERSCHIEBE: <OBJECT>, (<X>, <Y>)` und sonst nichts enthält!\\ \tt \\ \tt Ihr habt gemeinsam 3 von maximal 12 Strafpunkten angesammelt.\\ \tt Bitte versuche es noch mal! 
	  } 
	   } 
	   } 
	 & \\ 
 

    \theutterance \stepcounter{utterance}  

    & & & \multicolumn{4}{p{0.6\linewidth}}{\cellcolor[rgb]{1,0.85,0.72}{%
	\makecell[{{p{\linewidth}}}]{% 
	  \tt {\tiny [GM$\langle$B]}  
	 SPRICH: Hier ist mein aktuelles Spielbrett:  \\ \tt ```\\ \tt  1234567\\ \tt ╔═╤═╤═╤═╗\\ \tt ║◌│◌│◌│◌║ 1\\ \tt ╟─┼─┤◌│◌║ 2\\ \tt ║◌│W│◌│◌║ 3\\ \tt ║◌│C├─┴─╢ 4\\ \tt ╟─┴─┤I◌◌║ 5\\ \tt ║◌H◌└─┬─╢ 6\\ \tt ║◌◌T◌◌│◌║ 7\\ \tt ╚═════╧═╝\\ \tt ``` 
	  } 
	   } 
	   } 
	  \\ 
 

    \theutterance \stepcounter{utterance}  

    & & & \multicolumn{2}{p{0.3\linewidth}}{\cellcolor[rgb]{0.95,0.95,0.95}{%
	\makecell[{{p{\linewidth}}}]{% 
	  \tt {\tiny [GM$|$GM]}  
	 SPRICH: Hier ist mein aktuelles Spielbrett:  \\ \tt ```\\ \tt  1234567\\ \tt ╔═╤═╤═╤═╗\\ \tt ║◌│◌│◌│◌║ 1\\ \tt ╟─┼─┤◌│◌║ 2\\ \tt ║◌│W│◌│◌║ 3\\ \tt ║◌│C├─┴─╢ 4\\ \tt ╟─┴─┤I◌◌║ 5\\ \tt ║◌H◌└─┬─╢ 6\\ \tt ║◌◌T◌◌│◌║ 7\\ \tt ╚═════╧═╝\\ \tt ``` 
	  } 
	   } 
	   } 
	 & & \\ 
 

    \theutterance \stepcounter{utterance}  

    & & & \multicolumn{2}{p{0.3\linewidth}}{\cellcolor[rgb]{0.95,0.95,0.95}{%
	\makecell[{{p{\linewidth}}}]{% 
	  \tt {\tiny [GM$|$GM]}  
	 Invalid format: tail is not empty: '\\ \tt \\ \tt  1234567\\ \tt ╔═╤═╤═╤═╗\\ \tt ║◌│◌│◌│◌║ 1\\ \tt ╟─┼─┤◌│◌║ 2\\ \tt ║◌│W│◌│◌║ 3\\ \tt ║◌│C├─┴─╢ 4\\ \tt ╟─┴─┤I◌◌║ 5\\ \tt ║◌H◌└─┬─╢ 6\\ \tt ║◌◌T◌◌│◌║ 7\\ \tt ╚═════╧═╝ 
	  } 
	   } 
	   } 
	 & & \\ 
 

    \theutterance \stepcounter{utterance}  

    & & \multicolumn{4}{p{0.6\linewidth}}{\cellcolor[rgb]{0.9,0.9,0.9}{%
	\makecell[{{p{\linewidth}}}]{% 
	  \tt {\tiny [GM$\rangle$B]}  
	 Strafpunkt: Deine Nachricht darf nichts nach dem Befehl enthalten! Stelle sicher, dass deine Antwort ausschließlich einen der beiden Befehle `SPRICH: <NACHRICHT>` und `VERSCHIEBE: <OBJECT>, (<X>, <Y>)` und sonst nichts enthält!\\ \tt \\ \tt Ihr habt gemeinsam 4 von maximal 12 Strafpunkten angesammelt.\\ \tt Bitte versuche es noch mal! 
	  } 
	   } 
	   } 
	 & \\ 
 

    \theutterance \stepcounter{utterance}  

    & & & \multicolumn{4}{p{0.6\linewidth}}{\cellcolor[rgb]{1,0.85,0.72}{%
	\makecell[{{p{\linewidth}}}]{% 
	  \tt {\tiny [GM$\langle$B]}  
	 SPRICH: Hier ist mein aktuelles Spielbrett:  \\ \tt ```\\ \tt  1234567\\ \tt ╔═╤═╤═╤═╗\\ \tt ║◌│◌│◌│◌║ 1\\ \tt ╟─┼─┤◌│◌║ 2\\ \tt ║◌│W│◌│◌║ 3\\ \tt ║◌│C├─┴─╢ 4\\ \tt ╟─┴─┤I◌◌║ 5\\ \tt ║◌H◌└─┬─╢ 6\\ \tt ║◌◌T◌◌│◌║ 7\\ \tt ╚═════╧═╝\\ \tt ``` 
	  } 
	   } 
	   } 
	  \\ 
 

    \theutterance \stepcounter{utterance}  

    & & & \multicolumn{2}{p{0.3\linewidth}}{\cellcolor[rgb]{0.95,0.95,0.95}{%
	\makecell[{{p{\linewidth}}}]{% 
	  \tt {\tiny [GM$|$GM]}  
	 SPRICH: Hier ist mein aktuelles Spielbrett:  \\ \tt ```\\ \tt  1234567\\ \tt ╔═╤═╤═╤═╗\\ \tt ║◌│◌│◌│◌║ 1\\ \tt ╟─┼─┤◌│◌║ 2\\ \tt ║◌│W│◌│◌║ 3\\ \tt ║◌│C├─┴─╢ 4\\ \tt ╟─┴─┤I◌◌║ 5\\ \tt ║◌H◌└─┬─╢ 6\\ \tt ║◌◌T◌◌│◌║ 7\\ \tt ╚═════╧═╝\\ \tt ``` 
	  } 
	   } 
	   } 
	 & & \\ 
 

    \theutterance \stepcounter{utterance}  

    & & & \multicolumn{2}{p{0.3\linewidth}}{\cellcolor[rgb]{0.95,0.95,0.95}{%
	\makecell[{{p{\linewidth}}}]{% 
	  \tt {\tiny [GM$|$GM]}  
	 Invalid format: tail is not empty: '\\ \tt \\ \tt  1234567\\ \tt ╔═╤═╤═╤═╗\\ \tt ║◌│◌│◌│◌║ 1\\ \tt ╟─┼─┤◌│◌║ 2\\ \tt ║◌│W│◌│◌║ 3\\ \tt ║◌│C├─┴─╢ 4\\ \tt ╟─┴─┤I◌◌║ 5\\ \tt ║◌H◌└─┬─╢ 6\\ \tt ║◌◌T◌◌│◌║ 7\\ \tt ╚═════╧═╝ 
	  } 
	   } 
	   } 
	 & & \\ 
 

    \theutterance \stepcounter{utterance}  

    & & \multicolumn{4}{p{0.6\linewidth}}{\cellcolor[rgb]{0.9,0.9,0.9}{%
	\makecell[{{p{\linewidth}}}]{% 
	  \tt {\tiny [GM$\rangle$B]}  
	 Strafpunkt: Deine Nachricht darf nichts nach dem Befehl enthalten! Stelle sicher, dass deine Antwort ausschließlich einen der beiden Befehle `SPRICH: <NACHRICHT>` und `VERSCHIEBE: <OBJECT>, (<X>, <Y>)` und sonst nichts enthält!\\ \tt \\ \tt Ihr habt gemeinsam 5 von maximal 12 Strafpunkten angesammelt.\\ \tt Bitte versuche es noch mal! 
	  } 
	   } 
	   } 
	 & \\ 
 

    \theutterance \stepcounter{utterance}  

    & & & \multicolumn{4}{p{0.6\linewidth}}{\cellcolor[rgb]{1,0.85,0.72}{%
	\makecell[{{p{\linewidth}}}]{% 
	  \tt {\tiny [GM$\langle$B]}  
	 SPRICH: Hier ist mein aktuelles Spielbrett:  \\ \tt ```\\ \tt  1234567\\ \tt ╔═╤═╤═╤═╗\\ \tt ║◌│◌│◌│◌║ 1\\ \tt ╟─┼─┤◌│◌║ 2\\ \tt ║◌│W│◌│◌║ 3\\ \tt ║◌│C├─┴─╢ 4\\ \tt ╟─┴─┤I◌◌║ 5\\ \tt ║◌H◌└─┬─╢ 6\\ \tt ║◌◌T◌◌│◌║ 7\\ \tt ╚═════╧═╝\\ \tt ``` 
	  } 
	   } 
	   } 
	  \\ 
 

    \theutterance \stepcounter{utterance}  

    & & & \multicolumn{2}{p{0.3\linewidth}}{\cellcolor[rgb]{0.95,0.95,0.95}{%
	\makecell[{{p{\linewidth}}}]{% 
	  \tt {\tiny [GM$|$GM]}  
	 SPRICH: Hier ist mein aktuelles Spielbrett:  \\ \tt ```\\ \tt  1234567\\ \tt ╔═╤═╤═╤═╗\\ \tt ║◌│◌│◌│◌║ 1\\ \tt ╟─┼─┤◌│◌║ 2\\ \tt ║◌│W│◌│◌║ 3\\ \tt ║◌│C├─┴─╢ 4\\ \tt ╟─┴─┤I◌◌║ 5\\ \tt ║◌H◌└─┬─╢ 6\\ \tt ║◌◌T◌◌│◌║ 7\\ \tt ╚═════╧═╝\\ \tt ``` 
	  } 
	   } 
	   } 
	 & & \\ 
 

    \theutterance \stepcounter{utterance}  

    & & & \multicolumn{2}{p{0.3\linewidth}}{\cellcolor[rgb]{0.95,0.95,0.95}{%
	\makecell[{{p{\linewidth}}}]{% 
	  \tt {\tiny [GM$|$GM]}  
	 Invalid format: tail is not empty: '\\ \tt \\ \tt  1234567\\ \tt ╔═╤═╤═╤═╗\\ \tt ║◌│◌│◌│◌║ 1\\ \tt ╟─┼─┤◌│◌║ 2\\ \tt ║◌│W│◌│◌║ 3\\ \tt ║◌│C├─┴─╢ 4\\ \tt ╟─┴─┤I◌◌║ 5\\ \tt ║◌H◌└─┬─╢ 6\\ \tt ║◌◌T◌◌│◌║ 7\\ \tt ╚═════╧═╝ 
	  } 
	   } 
	   } 
	 & & \\ 
 

    \theutterance \stepcounter{utterance}  

    & & \multicolumn{4}{p{0.6\linewidth}}{\cellcolor[rgb]{0.9,0.9,0.9}{%
	\makecell[{{p{\linewidth}}}]{% 
	  \tt {\tiny [GM$\rangle$B]}  
	 Strafpunkt: Deine Nachricht darf nichts nach dem Befehl enthalten! Stelle sicher, dass deine Antwort ausschließlich einen der beiden Befehle `SPRICH: <NACHRICHT>` und `VERSCHIEBE: <OBJECT>, (<X>, <Y>)` und sonst nichts enthält!\\ \tt \\ \tt Ihr habt gemeinsam 6 von maximal 12 Strafpunkten angesammelt.\\ \tt Bitte versuche es noch mal! 
	  } 
	   } 
	   } 
	 & \\ 
 

    \theutterance \stepcounter{utterance}  

    & & & \multicolumn{4}{p{0.6\linewidth}}{\cellcolor[rgb]{1,0.85,0.72}{%
	\makecell[{{p{\linewidth}}}]{% 
	  \tt {\tiny [GM$\langle$B]}  
	 SPRICH: Hier ist mein aktuelles Spielbrett:  \\ \tt ```\\ \tt  1234567\\ \tt ╔═╤═╤═╤═╗\\ \tt ║◌│◌│◌│◌║ 1\\ \tt ╟─┼─┤◌│◌║ 2\\ \tt ║◌│W│◌│◌║ 3\\ \tt ║◌│C├─┴─╢ 4\\ \tt ╟─┴─┤I◌◌║ 5\\ \tt ║◌H◌└─┬─╢ 6\\ \tt ║◌◌T◌◌│◌║ 7\\ \tt ╚═════╧═╝\\ \tt ``` 
	  } 
	   } 
	   } 
	  \\ 
 

    \theutterance \stepcounter{utterance}  

    & & & \multicolumn{2}{p{0.3\linewidth}}{\cellcolor[rgb]{0.95,0.95,0.95}{%
	\makecell[{{p{\linewidth}}}]{% 
	  \tt {\tiny [GM$|$GM]}  
	 SPRICH: Hier ist mein aktuelles Spielbrett:  \\ \tt ```\\ \tt  1234567\\ \tt ╔═╤═╤═╤═╗\\ \tt ║◌│◌│◌│◌║ 1\\ \tt ╟─┼─┤◌│◌║ 2\\ \tt ║◌│W│◌│◌║ 3\\ \tt ║◌│C├─┴─╢ 4\\ \tt ╟─┴─┤I◌◌║ 5\\ \tt ║◌H◌└─┬─╢ 6\\ \tt ║◌◌T◌◌│◌║ 7\\ \tt ╚═════╧═╝\\ \tt ``` 
	  } 
	   } 
	   } 
	 & & \\ 
 

    \theutterance \stepcounter{utterance}  

    & & & \multicolumn{2}{p{0.3\linewidth}}{\cellcolor[rgb]{0.95,0.95,0.95}{%
	\makecell[{{p{\linewidth}}}]{% 
	  \tt {\tiny [GM$|$GM]}  
	 Invalid format: tail is not empty: '\\ \tt \\ \tt  1234567\\ \tt ╔═╤═╤═╤═╗\\ \tt ║◌│◌│◌│◌║ 1\\ \tt ╟─┼─┤◌│◌║ 2\\ \tt ║◌│W│◌│◌║ 3\\ \tt ║◌│C├─┴─╢ 4\\ \tt ╟─┴─┤I◌◌║ 5\\ \tt ║◌H◌└─┬─╢ 6\\ \tt ║◌◌T◌◌│◌║ 7\\ \tt ╚═════╧═╝ 
	  } 
	   } 
	   } 
	 & & \\ 
 

    \theutterance \stepcounter{utterance}  

    & & \multicolumn{4}{p{0.6\linewidth}}{\cellcolor[rgb]{0.9,0.9,0.9}{%
	\makecell[{{p{\linewidth}}}]{% 
	  \tt {\tiny [GM$\rangle$B]}  
	 Strafpunkt: Deine Nachricht darf nichts nach dem Befehl enthalten! Stelle sicher, dass deine Antwort ausschließlich einen der beiden Befehle `SPRICH: <NACHRICHT>` und `VERSCHIEBE: <OBJECT>, (<X>, <Y>)` und sonst nichts enthält!\\ \tt \\ \tt Ihr habt gemeinsam 7 von maximal 12 Strafpunkten angesammelt.\\ \tt Bitte versuche es noch mal! 
	  } 
	   } 
	   } 
	 & \\ 
 

    \theutterance \stepcounter{utterance}  

    & & & \multicolumn{4}{p{0.6\linewidth}}{\cellcolor[rgb]{1,0.85,0.72}{%
	\makecell[{{p{\linewidth}}}]{% 
	  \tt {\tiny [GM$\langle$B]}  
	 SPRICH: Hier ist mein aktuelles Spielbrett:  \\ \tt ```\\ \tt  1234567\\ \tt ╔═╤═╤═╤═╗\\ \tt ║◌│◌│◌│◌║ 1\\ \tt ╟─┼─┤◌│◌║ 2\\ \tt ║◌│W│◌│◌║ 3\\ \tt ║◌│C├─┴─╢ 4\\ \tt ╟─┴─┤I◌◌║ 5\\ \tt ║◌H◌└─┬─╢ 6\\ \tt ║◌◌T◌◌│◌║ 7\\ \tt ╚═════╧═╝\\ \tt ``` 
	  } 
	   } 
	   } 
	  \\ 
 

    \theutterance \stepcounter{utterance}  

    & & & \multicolumn{2}{p{0.3\linewidth}}{\cellcolor[rgb]{0.95,0.95,0.95}{%
	\makecell[{{p{\linewidth}}}]{% 
	  \tt {\tiny [GM$|$GM]}  
	 SPRICH: Hier ist mein aktuelles Spielbrett:  \\ \tt ```\\ \tt  1234567\\ \tt ╔═╤═╤═╤═╗\\ \tt ║◌│◌│◌│◌║ 1\\ \tt ╟─┼─┤◌│◌║ 2\\ \tt ║◌│W│◌│◌║ 3\\ \tt ║◌│C├─┴─╢ 4\\ \tt ╟─┴─┤I◌◌║ 5\\ \tt ║◌H◌└─┬─╢ 6\\ \tt ║◌◌T◌◌│◌║ 7\\ \tt ╚═════╧═╝\\ \tt ``` 
	  } 
	   } 
	   } 
	 & & \\ 
 

    \theutterance \stepcounter{utterance}  

    & & & \multicolumn{2}{p{0.3\linewidth}}{\cellcolor[rgb]{0.95,0.95,0.95}{%
	\makecell[{{p{\linewidth}}}]{% 
	  \tt {\tiny [GM$|$GM]}  
	 Invalid format: tail is not empty: '\\ \tt \\ \tt  1234567\\ \tt ╔═╤═╤═╤═╗\\ \tt ║◌│◌│◌│◌║ 1\\ \tt ╟─┼─┤◌│◌║ 2\\ \tt ║◌│W│◌│◌║ 3\\ \tt ║◌│C├─┴─╢ 4\\ \tt ╟─┴─┤I◌◌║ 5\\ \tt ║◌H◌└─┬─╢ 6\\ \tt ║◌◌T◌◌│◌║ 7\\ \tt ╚═════╧═╝ 
	  } 
	   } 
	   } 
	 & & \\ 
 

    \theutterance \stepcounter{utterance}  

    & & \multicolumn{4}{p{0.6\linewidth}}{\cellcolor[rgb]{0.9,0.9,0.9}{%
	\makecell[{{p{\linewidth}}}]{% 
	  \tt {\tiny [GM$\rangle$B]}  
	 Strafpunkt: Deine Nachricht darf nichts nach dem Befehl enthalten! Stelle sicher, dass deine Antwort ausschließlich einen der beiden Befehle `SPRICH: <NACHRICHT>` und `VERSCHIEBE: <OBJECT>, (<X>, <Y>)` und sonst nichts enthält!\\ \tt \\ \tt Ihr habt gemeinsam 8 von maximal 12 Strafpunkten angesammelt.\\ \tt Bitte versuche es noch mal! 
	  } 
	   } 
	   } 
	 & \\ 
 

    \theutterance \stepcounter{utterance}  

    & & & \multicolumn{4}{p{0.6\linewidth}}{\cellcolor[rgb]{1,0.85,0.72}{%
	\makecell[{{p{\linewidth}}}]{% 
	  \tt {\tiny [GM$\langle$B]}  
	 SPRICH: Hier ist mein aktuelles Spielbrett:  \\ \tt ```\\ \tt  1234567\\ \tt ╔═╤═╤═╤═╗\\ \tt ║◌│◌│◌│◌║ 1\\ \tt ╟─┼─┤◌│◌║ 2\\ \tt ║◌│W│◌│◌║ 3\\ \tt ║◌│C├─┴─╢ 4\\ \tt ╟─┴─┤I◌◌║ 5\\ \tt ║◌H◌└─┬─╢ 6\\ \tt ║◌◌T◌◌│◌║ 7\\ \tt ╚═════╧═╝\\ \tt ``` 
	  } 
	   } 
	   } 
	  \\ 
 

    \theutterance \stepcounter{utterance}  

    & & & \multicolumn{2}{p{0.3\linewidth}}{\cellcolor[rgb]{0.95,0.95,0.95}{%
	\makecell[{{p{\linewidth}}}]{% 
	  \tt {\tiny [GM$|$GM]}  
	 SPRICH: Hier ist mein aktuelles Spielbrett:  \\ \tt ```\\ \tt  1234567\\ \tt ╔═╤═╤═╤═╗\\ \tt ║◌│◌│◌│◌║ 1\\ \tt ╟─┼─┤◌│◌║ 2\\ \tt ║◌│W│◌│◌║ 3\\ \tt ║◌│C├─┴─╢ 4\\ \tt ╟─┴─┤I◌◌║ 5\\ \tt ║◌H◌└─┬─╢ 6\\ \tt ║◌◌T◌◌│◌║ 7\\ \tt ╚═════╧═╝\\ \tt ``` 
	  } 
	   } 
	   } 
	 & & \\ 
 

    \theutterance \stepcounter{utterance}  

    & & & \multicolumn{2}{p{0.3\linewidth}}{\cellcolor[rgb]{0.95,0.95,0.95}{%
	\makecell[{{p{\linewidth}}}]{% 
	  \tt {\tiny [GM$|$GM]}  
	 Invalid format: tail is not empty: '\\ \tt \\ \tt  1234567\\ \tt ╔═╤═╤═╤═╗\\ \tt ║◌│◌│◌│◌║ 1\\ \tt ╟─┼─┤◌│◌║ 2\\ \tt ║◌│W│◌│◌║ 3\\ \tt ║◌│C├─┴─╢ 4\\ \tt ╟─┴─┤I◌◌║ 5\\ \tt ║◌H◌└─┬─╢ 6\\ \tt ║◌◌T◌◌│◌║ 7\\ \tt ╚═════╧═╝ 
	  } 
	   } 
	   } 
	 & & \\ 
 

    \theutterance \stepcounter{utterance}  

    & & \multicolumn{4}{p{0.6\linewidth}}{\cellcolor[rgb]{0.9,0.9,0.9}{%
	\makecell[{{p{\linewidth}}}]{% 
	  \tt {\tiny [GM$\rangle$B]}  
	 Strafpunkt: Deine Nachricht darf nichts nach dem Befehl enthalten! Stelle sicher, dass deine Antwort ausschließlich einen der beiden Befehle `SPRICH: <NACHRICHT>` und `VERSCHIEBE: <OBJECT>, (<X>, <Y>)` und sonst nichts enthält!\\ \tt \\ \tt Ihr habt gemeinsam 9 von maximal 12 Strafpunkten angesammelt.\\ \tt Bitte versuche es noch mal! 
	  } 
	   } 
	   } 
	 & \\ 
 

    \theutterance \stepcounter{utterance}  

    & & & \multicolumn{4}{p{0.6\linewidth}}{\cellcolor[rgb]{1,0.85,0.72}{%
	\makecell[{{p{\linewidth}}}]{% 
	  \tt {\tiny [GM$\langle$B]}  
	 SPRICH: Hier ist mein aktuelles Spielbrett:  \\ \tt ```\\ \tt  1234567\\ \tt ╔═╤═╤═╤═╗\\ \tt ║◌│◌│◌│◌║ 1\\ \tt ╟─┼─┤◌│◌║ 2\\ \tt ║◌│W│◌│◌║ 3\\ \tt ║◌│C├─┴─╢ 4\\ \tt ╟─┴─┤I◌◌║ 5\\ \tt ║◌H◌└─┬─╢ 6\\ \tt ║◌◌T◌◌│◌║ 7\\ \tt ╚═════╧═╝\\ \tt ``` 
	  } 
	   } 
	   } 
	  \\ 
 

    \theutterance \stepcounter{utterance}  

    & & & \multicolumn{2}{p{0.3\linewidth}}{\cellcolor[rgb]{0.95,0.95,0.95}{%
	\makecell[{{p{\linewidth}}}]{% 
	  \tt {\tiny [GM$|$GM]}  
	 SPRICH: Hier ist mein aktuelles Spielbrett:  \\ \tt ```\\ \tt  1234567\\ \tt ╔═╤═╤═╤═╗\\ \tt ║◌│◌│◌│◌║ 1\\ \tt ╟─┼─┤◌│◌║ 2\\ \tt ║◌│W│◌│◌║ 3\\ \tt ║◌│C├─┴─╢ 4\\ \tt ╟─┴─┤I◌◌║ 5\\ \tt ║◌H◌└─┬─╢ 6\\ \tt ║◌◌T◌◌│◌║ 7\\ \tt ╚═════╧═╝\\ \tt ``` 
	  } 
	   } 
	   } 
	 & & \\ 
 

    \theutterance \stepcounter{utterance}  

    & & & \multicolumn{2}{p{0.3\linewidth}}{\cellcolor[rgb]{0.95,0.95,0.95}{%
	\makecell[{{p{\linewidth}}}]{% 
	  \tt {\tiny [GM$|$GM]}  
	 Invalid format: tail is not empty: '\\ \tt \\ \tt  1234567\\ \tt ╔═╤═╤═╤═╗\\ \tt ║◌│◌│◌│◌║ 1\\ \tt ╟─┼─┤◌│◌║ 2\\ \tt ║◌│W│◌│◌║ 3\\ \tt ║◌│C├─┴─╢ 4\\ \tt ╟─┴─┤I◌◌║ 5\\ \tt ║◌H◌└─┬─╢ 6\\ \tt ║◌◌T◌◌│◌║ 7\\ \tt ╚═════╧═╝ 
	  } 
	   } 
	   } 
	 & & \\ 
 

    \theutterance \stepcounter{utterance}  

    & & \multicolumn{4}{p{0.6\linewidth}}{\cellcolor[rgb]{0.9,0.9,0.9}{%
	\makecell[{{p{\linewidth}}}]{% 
	  \tt {\tiny [GM$\rangle$B]}  
	 Strafpunkt: Deine Nachricht darf nichts nach dem Befehl enthalten! Stelle sicher, dass deine Antwort ausschließlich einen der beiden Befehle `SPRICH: <NACHRICHT>` und `VERSCHIEBE: <OBJECT>, (<X>, <Y>)` und sonst nichts enthält!\\ \tt \\ \tt Ihr habt gemeinsam 10 von maximal 12 Strafpunkten angesammelt.\\ \tt Bitte versuche es noch mal! 
	  } 
	   } 
	   } 
	 & \\ 
 

    \theutterance \stepcounter{utterance}  

    & & & \multicolumn{4}{p{0.6\linewidth}}{\cellcolor[rgb]{1,0.85,0.72}{%
	\makecell[{{p{\linewidth}}}]{% 
	  \tt {\tiny [GM$\langle$B]}  
	 SPRICH: Hier ist mein aktuelles Spielbrett:  \\ \tt ```\\ \tt  1234567\\ \tt ╔═╤═╤═╤═╗\\ \tt ║◌│◌│◌│◌║ 1\\ \tt ╟─┼─┤◌│◌║ 2\\ \tt ║◌│W│◌│◌║ 3\\ \tt ║◌│C├─┴─╢ 4\\ \tt ╟─┴─┤I◌◌║ 5\\ \tt ║◌H◌└─┬─╢ 6\\ \tt ║◌◌T◌◌│◌║ 7\\ \tt ╚═════╧═╝\\ \tt ``` 
	  } 
	   } 
	   } 
	  \\ 
 

    \theutterance \stepcounter{utterance}  

    & & & \multicolumn{2}{p{0.3\linewidth}}{\cellcolor[rgb]{0.95,0.95,0.95}{%
	\makecell[{{p{\linewidth}}}]{% 
	  \tt {\tiny [GM$|$GM]}  
	 SPRICH: Hier ist mein aktuelles Spielbrett:  \\ \tt ```\\ \tt  1234567\\ \tt ╔═╤═╤═╤═╗\\ \tt ║◌│◌│◌│◌║ 1\\ \tt ╟─┼─┤◌│◌║ 2\\ \tt ║◌│W│◌│◌║ 3\\ \tt ║◌│C├─┴─╢ 4\\ \tt ╟─┴─┤I◌◌║ 5\\ \tt ║◌H◌└─┬─╢ 6\\ \tt ║◌◌T◌◌│◌║ 7\\ \tt ╚═════╧═╝\\ \tt ``` 
	  } 
	   } 
	   } 
	 & & \\ 
 

    \theutterance \stepcounter{utterance}  

    & & & \multicolumn{2}{p{0.3\linewidth}}{\cellcolor[rgb]{0.95,0.95,0.95}{%
	\makecell[{{p{\linewidth}}}]{% 
	  \tt {\tiny [GM$|$GM]}  
	 Invalid format: tail is not empty: '\\ \tt \\ \tt  1234567\\ \tt ╔═╤═╤═╤═╗\\ \tt ║◌│◌│◌│◌║ 1\\ \tt ╟─┼─┤◌│◌║ 2\\ \tt ║◌│W│◌│◌║ 3\\ \tt ║◌│C├─┴─╢ 4\\ \tt ╟─┴─┤I◌◌║ 5\\ \tt ║◌H◌└─┬─╢ 6\\ \tt ║◌◌T◌◌│◌║ 7\\ \tt ╚═════╧═╝ 
	  } 
	   } 
	   } 
	 & & \\ 
 

    \theutterance \stepcounter{utterance}  

    & & \multicolumn{4}{p{0.6\linewidth}}{\cellcolor[rgb]{0.9,0.9,0.9}{%
	\makecell[{{p{\linewidth}}}]{% 
	  \tt {\tiny [GM$\rangle$B]}  
	 Strafpunkt: Deine Nachricht darf nichts nach dem Befehl enthalten! Stelle sicher, dass deine Antwort ausschließlich einen der beiden Befehle `SPRICH: <NACHRICHT>` und `VERSCHIEBE: <OBJECT>, (<X>, <Y>)` und sonst nichts enthält!\\ \tt \\ \tt Ihr habt gemeinsam 11 von maximal 12 Strafpunkten angesammelt.\\ \tt Bitte versuche es noch mal! 
	  } 
	   } 
	   } 
	 & \\ 
 

    \theutterance \stepcounter{utterance}  

    & & & \multicolumn{4}{p{0.6\linewidth}}{\cellcolor[rgb]{1,0.85,0.72}{%
	\makecell[{{p{\linewidth}}}]{% 
	  \tt {\tiny [GM$\langle$B]}  
	 SPRICH: Hier ist mein aktuelles Spielbrett:  \\ \tt ```\\ \tt  1234567\\ \tt ╔═╤═╤═╤═╗\\ \tt ║◌│◌│◌│◌║ 1\\ \tt ╟─┼─┤◌│◌║ 2\\ \tt ║◌│W│◌│◌║ 3\\ \tt ║◌│C├─┴─╢ 4\\ \tt ╟─┴─┤I◌◌║ 5\\ \tt ║◌H◌└─┬─╢ 6\\ \tt ║◌◌T◌◌│◌║ 7\\ \tt ╚═════╧═╝\\ \tt ``` 
	  } 
	   } 
	   } 
	  \\ 
 

    \theutterance \stepcounter{utterance}  

    & & & \multicolumn{2}{p{0.3\linewidth}}{\cellcolor[rgb]{0.95,0.95,0.95}{%
	\makecell[{{p{\linewidth}}}]{% 
	  \tt {\tiny [GM$|$GM]}  
	 SPRICH: Hier ist mein aktuelles Spielbrett:  \\ \tt ```\\ \tt  1234567\\ \tt ╔═╤═╤═╤═╗\\ \tt ║◌│◌│◌│◌║ 1\\ \tt ╟─┼─┤◌│◌║ 2\\ \tt ║◌│W│◌│◌║ 3\\ \tt ║◌│C├─┴─╢ 4\\ \tt ╟─┴─┤I◌◌║ 5\\ \tt ║◌H◌└─┬─╢ 6\\ \tt ║◌◌T◌◌│◌║ 7\\ \tt ╚═════╧═╝\\ \tt ``` 
	  } 
	   } 
	   } 
	 & & \\ 
 

    \theutterance \stepcounter{utterance}  

    & & & \multicolumn{2}{p{0.3\linewidth}}{\cellcolor[rgb]{0.95,0.95,0.95}{%
	\makecell[{{p{\linewidth}}}]{% 
	  \tt {\tiny [GM$|$GM]}  
	 Invalid format: tail is not empty: '\\ \tt \\ \tt  1234567\\ \tt ╔═╤═╤═╤═╗\\ \tt ║◌│◌│◌│◌║ 1\\ \tt ╟─┼─┤◌│◌║ 2\\ \tt ║◌│W│◌│◌║ 3\\ \tt ║◌│C├─┴─╢ 4\\ \tt ╟─┴─┤I◌◌║ 5\\ \tt ║◌H◌└─┬─╢ 6\\ \tt ║◌◌T◌◌│◌║ 7\\ \tt ╚═════╧═╝ 
	  } 
	   } 
	   } 
	 & & \\ 
 

    \theutterance \stepcounter{utterance}  

    & & \multicolumn{4}{p{0.6\linewidth}}{\cellcolor[rgb]{0.9,0.9,0.9}{%
	\makecell[{{p{\linewidth}}}]{% 
	  \tt {\tiny [GM$\rangle$B]}  
	 Strafpunkt: Deine Nachricht darf nichts nach dem Befehl enthalten! Stelle sicher, dass deine Antwort ausschließlich einen der beiden Befehle `SPRICH: <NACHRICHT>` und `VERSCHIEBE: <OBJECT>, (<X>, <Y>)` und sonst nichts enthält!\\ \tt \\ \tt Ihr habt gemeinsam 12 von maximal 12 Strafpunkten angesammelt.\\ \tt Bitte versuche es noch mal! 
	  } 
	   } 
	   } 
	 & \\ 
 

    \theutterance \stepcounter{utterance}  

    & & & \multicolumn{4}{p{0.6\linewidth}}{\cellcolor[rgb]{1,0.85,0.72}{%
	\makecell[{{p{\linewidth}}}]{% 
	  \tt {\tiny [GM$\langle$B]}  
	 SPRICH: Ich entschuldige mich für die wiederholten Fehler. Hier ist mein Spielbrett:  \\ \tt ```\\ \tt  1234567\\ \tt ╔═╤═╤═╤═╗\\ \tt ║◌│◌│◌│◌║ 1\\ \tt ╟─┼─┤◌│◌║ 2\\ \tt ║◌│W│◌│◌║ 3\\ \tt ║◌│C├─┴─╢ 4\\ \tt ╟─┴─┤I◌◌║ 5\\ \tt ║◌H◌└─┬─╢ 6\\ \tt ║◌◌T◌◌│◌║ 7\\ \tt ╚═════╧═╝\\ \tt ``` 
	  } 
	   } 
	   } 
	  \\ 
 

    \theutterance \stepcounter{utterance}  

    & & & \multicolumn{2}{p{0.3\linewidth}}{\cellcolor[rgb]{0.95,0.95,0.95}{%
	\makecell[{{p{\linewidth}}}]{% 
	  \tt {\tiny [GM$|$GM]}  
	 SPRICH: Ich entschuldige mich für die wiederholten Fehler. Hier ist mein Spielbrett:  \\ \tt ```\\ \tt  1234567\\ \tt ╔═╤═╤═╤═╗\\ \tt ║◌│◌│◌│◌║ 1\\ \tt ╟─┼─┤◌│◌║ 2\\ \tt ║◌│W│◌│◌║ 3\\ \tt ║◌│C├─┴─╢ 4\\ \tt ╟─┴─┤I◌◌║ 5\\ \tt ║◌H◌└─┬─╢ 6\\ \tt ║◌◌T◌◌│◌║ 7\\ \tt ╚═════╧═╝\\ \tt ``` 
	  } 
	   } 
	   } 
	 & & \\ 
 

    \theutterance \stepcounter{utterance}  

    & & & \multicolumn{2}{p{0.3\linewidth}}{\cellcolor[rgb]{0.95,0.95,0.95}{%
	\makecell[{{p{\linewidth}}}]{% 
	  \tt {\tiny [GM$|$GM]}  
	 Invalid format: tail is not empty: '\\ \tt \\ \tt  1234567\\ \tt ╔═╤═╤═╤═╗\\ \tt ║◌│◌│◌│◌║ 1\\ \tt ╟─┼─┤◌│◌║ 2\\ \tt ║◌│W│◌│◌║ 3\\ \tt ║◌│C├─┴─╢ 4\\ \tt ╟─┴─┤I◌◌║ 5\\ \tt ║◌H◌└─┬─╢ 6\\ \tt ║◌◌T◌◌│◌║ 7\\ \tt ╚═════╧═╝ 
	  } 
	   } 
	   } 
	 & & \\ 
 

    \theutterance \stepcounter{utterance}  

    & & & \multicolumn{2}{p{0.3\linewidth}}{\cellcolor[rgb]{0.95,0.95,0.95}{%
	\makecell[{{p{\linewidth}}}]{% 
	  \tt {\tiny [GM$|$GM]}  
	 Maximum number of penalties exceeded 
	  } 
	   } 
	   } 
	 & & \\ 
 

    \theutterance \stepcounter{utterance}  

    & & & \multicolumn{2}{p{0.3\linewidth}}{\cellcolor[rgb]{0.95,0.95,0.95}{%
	\makecell[{{p{\linewidth}}}]{% 
	  \tt {\tiny [GM$|$GM]}  
	 * success: False\\ \tt * lose: True\\ \tt * aborted: True\\ \tt -------\\ \tt * Shifts: 0.00\\ \tt * Max Shifts: 8.00\\ \tt * Min Shifts: 4.00\\ \tt * End Distance Sum: 19.39\\ \tt * Init Distance Sum: 19.39\\ \tt * Expected Distance Sum: 20.95\\ \tt * Penalties: 13.00\\ \tt * Max Penalties: 12.00\\ \tt * Rounds: 1.00\\ \tt * Max Rounds: 20.00\\ \tt * Object Count: 5.00\\ \tt  
	  } 
	   } 
	   } 
	 & & \\ 
 

    \theutterance \stepcounter{utterance}  

    & & & \multicolumn{2}{p{0.3\linewidth}}{\cellcolor[rgb]{0.95,0.95,0.95}{%
	\makecell[{{p{\linewidth}}}]{% 
	  \tt {\tiny [GM$|$GM]}  
	 * Main Score: 0.00\\ \tt \\ \tt -------\\ \tt * Distance Score: 0.04\\ \tt * Coverage Score: 0.03\\ \tt * Penalty Score: 0.00\\ \tt * Alternative Penalty Score: 0.00\\ \tt * Alternative Main Score: 0.00\\ \tt \\ \tt -------\\ \tt * Shifts: 0.00\\ \tt * Max Shifts: 8.00\\ \tt * Min Shifts: 4.00\\ \tt * End Distance Sum: 19.39\\ \tt * Init Distance Sum: 19.39\\ \tt * Expected Distance Sum: 20.95\\ \tt * Penalties: 13.00\\ \tt * Max Penalties: 12.00\\ \tt * Rounds: 1.00\\ \tt * Max Rounds: 20.00\\ \tt * Object Count: 5.00\\ \tt  
	  } 
	   } 
	   } 
	 & & \\ 
 
>>>>>>> 12a93240 (results updated for Nemotron-9b-no-reasoning)

\end{supertabular}
}

\end{document}
