
\documentclass{article}
\usepackage{colortbl}
\usepackage{makecell}
\usepackage{multirow}
\usepackage{supertabular}

\begin{document}

\newcounter{utterance}

\centering \large Interaction Transcript for game `clean\_up', experiment `0\_easy\_7obj\_de', episode 1 with nemotron{-}nano{-}9b{-}v2{-}no{-}reasoning{-}t1.0.
\vspace{24pt}

{ \footnotesize  \setcounter{utterance}{1}
\setlength{\tabcolsep}{0pt}
\begin{supertabular}{c@{$\;$}|p{.15\linewidth}@{}p{.15\linewidth}p{.15\linewidth}p{.15\linewidth}p{.15\linewidth}p{.15\linewidth}}
    \# & $\;$A & \multicolumn{4}{c}{Game Master} & $\;\:$B\\
    \hline

    \theutterance \stepcounter{utterance}  
    & & \multicolumn{4}{p{0.6\linewidth}}{
        \cellcolor[rgb]{0.9,0.9,0.9}{
            \makecell[{{p{\linewidth}}}]{
                \texttt{\tiny{[P1$\langle$GM]}}
                \texttt{Ich bin der Spielleiter, und du spielst ein kooperatives Spiel mit dem folgenden Raster als Spielbrett:} \\
\\ 
\texttt{```} \\
\texttt{ 1234567} \\
\texttt{╔═══════╗} \\
\texttt{║◌◌P◌◌◌◌║ 1} \\
\texttt{║◌◌◌◌◌◌◌║ 2} \\
\texttt{║◌◌◌◌D┌─╢ 3} \\
\texttt{║◌TM┌─┤◌║ 4} \\
\texttt{║◌◌O│◌├─╢ 5} \\
\texttt{╟───┼─┘◌║ 6} \\
\texttt{║◌◌◌│◌AS║ 7} \\
\texttt{╚═══╧═══╝} \\
\\ 
\texttt{```} \\
\\ 
\texttt{* Am oberen Rand sind die x{-}Koordinaten angegeben, die nach rechts hin zunehmen, und am rechten Rand die y{-}Koordinaten, die nach unten zunehmen.} \\
\texttt{* Die folgenden Objekte sind zufällig auf deinem Raster platziert: 'P', 'O', 'T', 'S', 'D', 'A', 'M'.} \\
\\ 
\\ 
\texttt{Dein Mitspieler sieht eine Variation des Spielbretts, auf dem dieselben Objekte an anderen zufälligen Positionen platziert sind. Du kannst das Spielbrett des anderen Spielers nicht sehen, und er kann deins nicht sehen.} \\
\\ 
\\ 
\texttt{**Ziel des Spiels:**} \\
\\ 
\texttt{Beide Spieler müssen die Objekte auf ihrem jeweiligen Spielbrett so bewegen, dass identische Objekte am Ende an denselben Koordinaten liegen. Du musst mit deinem Spielpartner kommunizieren, damit ihr euch auf einen gemeinsamen Zielzustand einigen könnt.} \\
\\ 
\\ 
\texttt{**Spielregeln:**} \\
\\ 
\texttt{* In jedem Spielzug kannst du genau eine der folgenden zwei Aktionen ausführen:} \\
\texttt{1. `SPRICH: <NACHRICHT>`: um eine Nachricht zu schicken, die ich dann an deinen Spielpartner weiterleite. Als Nachricht gilt alles bis zum nächsten Zeilenumbruch.} \\
\texttt{2. `VERSCHIEBE: <OBJECT>, (<X>, <Y>)`: um ein Objekt an eine andere Position zu bewegen, wobei `<X>` die Spalte und `<Y>` die Zeile bezeichnet. Ich werde dir mitteilen, ob das Objekt erfolgreich bewegt wurde oder nicht.} \\
\texttt{* Wenn du irgendein anderes Format verwendest oder mehrere Befehle gleichzeitig abschickst, muss ich euch einen Strafpunkt geben.} \\
\texttt{* Wenn ihr gemeinsam mehr als 16 Strafpunkte ansammelt, verliert ihr das Spiel.} \\
\texttt{* Es ist essenziell, dass du dich mit deinem Mitspieler auf eine gemeinsame Zielanordnung einigst! Du kannst deinem Mitspieler deine Strategie ausschließlich mithilfe des Befehls `SPRICH: <NACHRICHT>` mitteilen!} \\
\\ 
\\ 
\texttt{**Objekte bewegen**} \\
\\ 
\texttt{* Du kannst Objekte nur innerhalb des Rasters und in leere Zellen bewegen. Leere Zellen erkennst am Zeichen '◌'.} \\
\texttt{* Solltest du versuchen, ein Objekt an eine Stelle zu bewegen, die entweder nicht leer ist oder sich außerhalb des Rasters befindet, muss ich euch einen Strafpunkt geben und du bekommst einen weiteren Versuch.} \\
\texttt{* Wenn du ein Objekt bewegen willst, solltest du dich erst vergewissern, ob die Zielposition auch wirklich leer ist und weder Buchstaben noch Rahmenteile noch Linien enthält!} \\
\\ 
\\ 
\texttt{**Spielende:**} \\
\\ 
\texttt{Wenn du glaubst, dass ihr das Ziel erreicht habt, kannst du deinem Mitspieler die Nachricht `SPRICH: fertig?` schicken. Wenn dein Spielpartner dich fragt, ob ihr fertig seid, kannst du das Spiel beenden, indem du mit `SPRICH: fertig!` antwortest.} \\
\\ 
\texttt{Ihr gewinnt gemeinsam, wenn ihr das Spiel innerhalb von \#max\_rounds Runden beendet. Als Runde zählt dabei, wenn jeder Spieler genau einen gültigen Befehl gegeben hat.} \\
\\ 
\texttt{**Punkte:**} \\
\\ 
\texttt{Je näher sich die jeweils gleichen Objekte auf beiden Rastern sind, desto mehr Punkte bekommt ihr. Für Strafpunkte gibt es Punktabzug. Könnt ihr den Rekord brechen?} \\
\\ 
\texttt{Schicke eine Nachricht an deinen Mitspieler um das Spiel zu starten!} \\
            }
        }
    }
    & \\ \\

    \theutterance \stepcounter{utterance}  
    & \multicolumn{4}{p{0.6\linewidth}}{
        \cellcolor[rgb]{0.9,0.9,0.9}{
            \makecell[{{p{\linewidth}}}]{
                \texttt{\tiny{[P1$\rangle$GM]}}
                \texttt{SPRICH: Hallo! Ich bin der erste Spieler. Ich schlage vor, wir beginnen damit, uns auf eine gemeinsame Zielanordnung zu einigen. Könntest du mir deine aktuelle Position der Objekte beschreiben? So können wir gemeinsam einen Plan erstellen.  } \\
\texttt{(Beispiel: "Mein 'P' ist in Zeile 1, Spalte 3. Dein 'P' ist in Zeile 2, Spalte 4.")  } \\
\texttt{Sobald du deine Positionen teilst, können wir uns auf die nächsten Schritte einigen!} \\
            }
        }
    }
    & & \\ \\

    \theutterance \stepcounter{utterance}  
    & & & \multicolumn{2}{p{0.3\linewidth}}{
        \cellcolor[rgb]{0.9,0.9,0.9}{
            \makecell[{{p{\linewidth}}}]{
                \texttt{\tiny{[GM$|$GM]}}
                \texttt{SPRICH: Hallo! Ich bin der erste Spieler. Ich schlage vor, wir beginnen damit, uns auf eine gemeinsame Zielanordnung zu einigen. Könntest du mir deine aktuelle Position der Objekte beschreiben? So können wir gemeinsam einen Plan erstellen.  } \\
\texttt{(Beispiel: "Mein 'P' ist in Zeile 1, Spalte 3. Dein 'P' ist in Zeile 2, Spalte 4.")  } \\
\texttt{Sobald du deine Positionen teilst, können wir uns auf die nächsten Schritte einigen!} \\
            }
        }
    }
    & & \\ \\

    \theutterance \stepcounter{utterance}  
    & & & \multicolumn{2}{p{0.3\linewidth}}{
        \cellcolor[rgb]{0.9,0.9,0.9}{
            \makecell[{{p{\linewidth}}}]{
                \texttt{\tiny{[GM$|$GM]}}
                \texttt{Invalid format: tail is not empty: '} \\
\texttt{(Beispiel: "Mein 'P' ist in Zeile 1, Spalte 3. Dein 'P' ist in Zeile 2, Spalte 4.")  } \\
\texttt{Sobald du deine Positionen teilst, können wir uns auf die nächsten Schritte einigen!} \\
            }
        }
    }
    & & \\ \\

    \theutterance \stepcounter{utterance}  
    & & \multicolumn{4}{p{0.6\linewidth}}{
        \cellcolor[rgb]{0.9,0.9,0.9}{
            \makecell[{{p{\linewidth}}}]{
                \texttt{\tiny{[P1$\langle$GM]}}
                \texttt{Strafpunkt: Deine Nachricht darf nichts nach dem Befehl enthalten! Stelle sicher, dass deine Antwort ausschließlich einen der beiden Befehle `SPRICH: <NACHRICHT>` und `VERSCHIEBE: <OBJECT>, (<X>, <Y>)` und sonst nichts enthält!} \\
\\ 
\texttt{Ihr habt gemeinsam 1 von maximal 16 Strafpunkten angesammelt.} \\
\texttt{Bitte versuche es noch mal!} \\
            }
        }
    }
    & \\ \\

    \theutterance \stepcounter{utterance}  
    & \multicolumn{4}{p{0.6\linewidth}}{
        \cellcolor[rgb]{0.9,0.9,0.9}{
            \makecell[{{p{\linewidth}}}]{
                \texttt{\tiny{[P1$\rangle$GM]}}
                \texttt{SPRICH: Hallo! Ich bin der erste Spieler. Ich schlage vor, wir beginnen damit, uns auf eine gemeinsame Zielanordnung zu einigen. Könntest du mir deine aktuelle Position der Objekte beschreiben? So können wir gemeinsam einen Plan erstellen.} \\
            }
        }
    }
    & & \\ \\

    \theutterance \stepcounter{utterance}  
    & & & \multicolumn{2}{p{0.3\linewidth}}{
        \cellcolor[rgb]{0.9,0.9,0.9}{
            \makecell[{{p{\linewidth}}}]{
                \texttt{\tiny{[GM$|$GM]}}
                \texttt{SPRICH: Hallo! Ich bin der erste Spieler. Ich schlage vor, wir beginnen damit, uns auf eine gemeinsame Zielanordnung zu einigen. Könntest du mir deine aktuelle Position der Objekte beschreiben? So können wir gemeinsam einen Plan erstellen.} \\
            }
        }
    }
    & & \\ \\

    \theutterance \stepcounter{utterance}  
    & & \multicolumn{4}{p{0.6\linewidth}}{
        \cellcolor[rgb]{0.9,0.9,0.9}{
            \makecell[{{p{\linewidth}}}]{
                \texttt{\tiny{[GM$\rangle$P2]}}
                \texttt{Ich bin der Spielleiter, und du spielst ein kooperatives Spiel mit dem folgenden Raster als Spielbrett:} \\
\\ 
\texttt{```} \\
\texttt{ 1234567} \\
\texttt{╔═══════╗} \\
\texttt{║T◌◌◌S◌◌║ 1} \\
\texttt{║◌◌◌D◌◌◌║ 2} \\
\texttt{║◌◌O◌◌┌─╢ 3} \\
\texttt{║◌M◌┌─┤P║ 4} \\
\texttt{║◌◌◌│◌├─╢ 5} \\
\texttt{╟───┼─┘◌║ 6} \\
\texttt{║◌◌◌│◌A◌║ 7} \\
\texttt{╚═══╧═══╝} \\
\\ 
\texttt{```} \\
\\ 
\texttt{* Am oberen Rand sind die x{-}Koordinaten angegeben, die nach rechts hin zunehmen, und am rechten Rand die y{-}Koordinaten, die nach unten zunehmen.} \\
\texttt{* Die folgenden Objekte sind zufällig auf deinem Raster platziert: 'P', 'O', 'T', 'S', 'D', 'A', 'M'.} \\
\\ 
\\ 
\texttt{Dein Mitspieler sieht eine Variation des Spielbretts, auf dem dieselben Objekte an anderen zufälligen Positionen platziert sind. Du kannst das Spielbrett des anderen Spielers nicht sehen, und er kann deins nicht sehen.} \\
\\ 
\\ 
\texttt{**Ziel des Spiels:**} \\
\\ 
\texttt{Beide Spieler müssen die Objekte auf ihrem jeweiligen Spielbrett so bewegen, dass identische Objekte am Ende an denselben Koordinaten liegen. Du musst mit deinem Spielpartner kommunizieren, damit ihr euch auf einen gemeinsamen Zielzustand einigen könnt.} \\
\\ 
\\ 
\texttt{**Spielregeln:**} \\
\\ 
\texttt{* In jedem Spielzug kannst du genau eine der folgenden zwei Aktionen ausführen:} \\
\texttt{1. `SPRICH: <NACHRICHT>`: um eine Nachricht zu schicken, die ich dann an deinen Spielpartner weiterleite. Als Nachricht gilt alles bis zum nächsten Zeilenumbruch.} \\
\texttt{2. `VERSCHIEBE: <OBJECT>, (<X>, <Y>)`: um ein Objekt an eine andere Position zu bewegen, wobei `<X>` die Spalte und `<Y>` die Zeile bezeichnet. Ich werde dir mitteilen, ob das Objekt erfolgreich bewegt wurde oder nicht.} \\
\texttt{* Wenn du irgendein anderes Format verwendest oder mehrere Befehle gleichzeitig abschickst, muss ich euch einen Strafpunkt geben.} \\
\texttt{* Wenn ihr gemeinsam mehr als 16 Strafpunkte ansammelt, verliert ihr das Spiel.} \\
\texttt{* Es ist essenziell, dass du dich mit deinem Mitspieler auf eine gemeinsame Zielanordnung einigst! Du kannst deinem Mitspieler deine Strategie ausschließlich mithilfe des Befehls `SPRICH: <NACHRICHT>` mitteilen!} \\
\\ 
\\ 
\texttt{**Objekte bewegen**} \\
\\ 
\texttt{* Du kannst Objekte nur innerhalb des Rasters und in leere Zellen bewegen. Leere Zellen erkennst am Zeichen '◌'.} \\
\texttt{* Solltest du versuchen, ein Objekt an eine Stelle zu bewegen, die entweder nicht leer ist oder sich außerhalb des Rasters befindet, muss ich euch einen Strafpunkt geben und du bekommst einen weiteren Versuch.} \\
\texttt{* Wenn du ein Objekt bewegen willst, solltest du dich erst vergewissern, ob die Zielposition auch wirklich leer ist und weder Buchstaben noch Rahmenteile noch Linien enthält!} \\
\\ 
\\ 
\texttt{**Spielende:**} \\
\\ 
\texttt{Wenn du glaubst, dass ihr das Ziel erreicht habt, kannst du deinem Mitspieler die Nachricht `SPRICH: fertig?` schicken. Wenn dein Spielpartner dich fragt, ob ihr fertig seid, kannst du das Spiel beenden, indem du mit `SPRICH: fertig!` antwortest.} \\
\\ 
\texttt{Ihr gewinnt gemeinsam, wenn ihr das Spiel innerhalb von \#max\_rounds Runden beendet. Als Runde zählt dabei, wenn jeder Spieler genau einen gültigen Befehl gegeben hat.} \\
\\ 
\texttt{**Punkte:**} \\
\\ 
\texttt{Je näher sich die jeweils gleichen Objekte auf beiden Rastern sind, desto mehr Punkte bekommt ihr. Für Strafpunkte gibt es Punktabzug. Könnt ihr den Rekord brechen?} \\
\\ 
\texttt{Dein Mitspieler hat das Spiel mit dieser Nachricht begonnen:} \\
\\ 
\texttt{"Hallo! Ich bin der erste Spieler. Ich schlage vor, wir beginnen damit, uns auf eine gemeinsame Zielanordnung zu einigen. Könntest du mir deine aktuelle Position der Objekte beschreiben? So können wir gemeinsam einen Plan erstellen."} \\
\\ 
\texttt{Wie lautet dein erster Befehl?} \\
            }
        }
    }
    & \\ \\

    \theutterance \stepcounter{utterance}  
    & & & \multicolumn{4}{p{0.6\linewidth}}{
        \cellcolor[rgb]{0.9,0.9,0.9}{
            \makecell[{{p{\linewidth}}}]{
                \texttt{\tiny{[GM$\langle$P2]}}
                \texttt{SPRICH: Hallo! Ich bin der zweite Spieler. Meine Objekte sind wie folgt positioniert:  } \\
\texttt{{-} P: (4,4)  } \\
\texttt{{-} O: (3,3)  } \\
\texttt{{-} T: (1,1)  } \\
\texttt{{-} S: (4,1)  } \\
\texttt{{-} D: (2,2)  } \\
\texttt{{-} A: (7,7)  } \\
\texttt{{-} M: (4,2)  } \\
\\ 
\texttt{Ich schlage vor, wir beginnen damit, uns auf eine gemeinsame Zielanordnung zu einigen. Könntest du mir deine aktuelle Position der Objekte beschreiben?} \\
            }
        }
    }
     \\ \\

    \theutterance \stepcounter{utterance}  
    & & & \multicolumn{2}{p{0.3\linewidth}}{
        \cellcolor[rgb]{0.9,0.9,0.9}{
            \makecell[{{p{\linewidth}}}]{
                \texttt{\tiny{[GM$|$GM]}}
                \texttt{SPRICH: Hallo! Ich bin der zweite Spieler. Meine Objekte sind wie folgt positioniert:  } \\
\texttt{{-} P: (4,4)  } \\
\texttt{{-} O: (3,3)  } \\
\texttt{{-} T: (1,1)  } \\
\texttt{{-} S: (4,1)  } \\
\texttt{{-} D: (2,2)  } \\
\texttt{{-} A: (7,7)  } \\
\texttt{{-} M: (4,2)  } \\
\\ 
\texttt{Ich schlage vor, wir beginnen damit, uns auf eine gemeinsame Zielanordnung zu einigen. Könntest du mir deine aktuelle Position der Objekte beschreiben?} \\
            }
        }
    }
    & & \\ \\

    \theutterance \stepcounter{utterance}  
    & & & \multicolumn{2}{p{0.3\linewidth}}{
        \cellcolor[rgb]{0.9,0.9,0.9}{
            \makecell[{{p{\linewidth}}}]{
                \texttt{\tiny{[GM$|$GM]}}
                \texttt{Invalid format: tail is not empty: '} \\
\texttt{{-} P: (4,4)  } \\
\texttt{{-} O: (3,3)  } \\
\texttt{{-} T: (1,1)  } \\
\texttt{{-} S: (4,1)  } \\
\texttt{{-} D: (2,2)  } \\
\texttt{{-} A: (7,7)  } \\
\texttt{{-} M: (4,2)  } \\
\\ 
\texttt{Ich schlage vor, wir beginnen damit, uns auf eine gemeinsame Zielanordnung zu einigen. Könntest du mir deine aktuelle Position der Objekte beschreiben?} \\
            }
        }
    }
    & & \\ \\

    \theutterance \stepcounter{utterance}  
    & & \multicolumn{4}{p{0.6\linewidth}}{
        \cellcolor[rgb]{0.9,0.9,0.9}{
            \makecell[{{p{\linewidth}}}]{
                \texttt{\tiny{[GM$\rangle$P2]}}
                \texttt{Strafpunkt: Deine Nachricht darf nichts nach dem Befehl enthalten! Stelle sicher, dass deine Antwort ausschließlich einen der beiden Befehle `SPRICH: <NACHRICHT>` und `VERSCHIEBE: <OBJECT>, (<X>, <Y>)` und sonst nichts enthält!} \\
\\ 
\texttt{Ihr habt gemeinsam 2 von maximal 16 Strafpunkten angesammelt.} \\
\texttt{Bitte versuche es noch mal!} \\
            }
        }
    }
    & \\ \\

    \theutterance \stepcounter{utterance}  
    & & & \multicolumn{4}{p{0.6\linewidth}}{
        \cellcolor[rgb]{0.9,0.9,0.9}{
            \makecell[{{p{\linewidth}}}]{
                \texttt{\tiny{[GM$\langle$P2]}}
                \texttt{SPRICH: Hallo! Ich bin der zweite Spieler. Meine Objekte sind wie folgt positioniert:  } \\
\texttt{{-} P: (4,4)  } \\
\texttt{{-} O: (3,3)  } \\
\texttt{{-} T: (1,1)  } \\
\texttt{{-} S: (4,1)  } \\
\texttt{{-} D: (2,2)  } \\
\texttt{{-} A: (7,7)  } \\
\texttt{{-} M: (4,2)} \\
            }
        }
    }
     \\ \\

    \theutterance \stepcounter{utterance}  
    & & & \multicolumn{2}{p{0.3\linewidth}}{
        \cellcolor[rgb]{0.9,0.9,0.9}{
            \makecell[{{p{\linewidth}}}]{
                \texttt{\tiny{[GM$|$GM]}}
                \texttt{SPRICH: Hallo! Ich bin der zweite Spieler. Meine Objekte sind wie folgt positioniert:  } \\
\texttt{{-} P: (4,4)  } \\
\texttt{{-} O: (3,3)  } \\
\texttt{{-} T: (1,1)  } \\
\texttt{{-} S: (4,1)  } \\
\texttt{{-} D: (2,2)  } \\
\texttt{{-} A: (7,7)  } \\
\texttt{{-} M: (4,2)} \\
            }
        }
    }
    & & \\ \\

    \theutterance \stepcounter{utterance}  
    & & & \multicolumn{2}{p{0.3\linewidth}}{
        \cellcolor[rgb]{0.9,0.9,0.9}{
            \makecell[{{p{\linewidth}}}]{
                \texttt{\tiny{[GM$|$GM]}}
                \texttt{Invalid format: tail is not empty: '} \\
\texttt{{-} P: (4,4)  } \\
\texttt{{-} O: (3,3)  } \\
\texttt{{-} T: (1,1)  } \\
\texttt{{-} S: (4,1)  } \\
\texttt{{-} D: (2,2)  } \\
\texttt{{-} A: (7,7)  } \\
\texttt{{-} M: (4,2)} \\
            }
        }
    }
    & & \\ \\

    \theutterance \stepcounter{utterance}  
    & & \multicolumn{4}{p{0.6\linewidth}}{
        \cellcolor[rgb]{0.9,0.9,0.9}{
            \makecell[{{p{\linewidth}}}]{
                \texttt{\tiny{[GM$\rangle$P2]}}
                \texttt{Strafpunkt: Deine Nachricht darf nichts nach dem Befehl enthalten! Stelle sicher, dass deine Antwort ausschließlich einen der beiden Befehle `SPRICH: <NACHRICHT>` und `VERSCHIEBE: <OBJECT>, (<X>, <Y>)` und sonst nichts enthält!} \\
\\ 
\texttt{Ihr habt gemeinsam 3 von maximal 16 Strafpunkten angesammelt.} \\
\texttt{Bitte versuche es noch mal!} \\
            }
        }
    }
    & \\ \\

    \theutterance \stepcounter{utterance}  
    & & & \multicolumn{4}{p{0.6\linewidth}}{
        \cellcolor[rgb]{0.9,0.9,0.9}{
            \makecell[{{p{\linewidth}}}]{
                \texttt{\tiny{[GM$\langle$P2]}}
                \texttt{SPRICH: Hallo! Ich bin der zweite Spieler. Meine Objekte sind wie folgt positioniert:  } \\
\texttt{{-} P: (4,4)  } \\
\texttt{{-} O: (3,3)  } \\
\texttt{{-} T: (1,1)  } \\
\texttt{{-} S: (4,1)  } \\
\texttt{{-} D: (2,2)  } \\
\texttt{{-} A: (7,7)  } \\
\texttt{{-} M: (4,2)} \\
            }
        }
    }
     \\ \\

    \theutterance \stepcounter{utterance}  
    & & & \multicolumn{2}{p{0.3\linewidth}}{
        \cellcolor[rgb]{0.9,0.9,0.9}{
            \makecell[{{p{\linewidth}}}]{
                \texttt{\tiny{[GM$|$GM]}}
                \texttt{SPRICH: Hallo! Ich bin der zweite Spieler. Meine Objekte sind wie folgt positioniert:  } \\
\texttt{{-} P: (4,4)  } \\
\texttt{{-} O: (3,3)  } \\
\texttt{{-} T: (1,1)  } \\
\texttt{{-} S: (4,1)  } \\
\texttt{{-} D: (2,2)  } \\
\texttt{{-} A: (7,7)  } \\
\texttt{{-} M: (4,2)} \\
            }
        }
    }
    & & \\ \\

    \theutterance \stepcounter{utterance}  
    & & & \multicolumn{2}{p{0.3\linewidth}}{
        \cellcolor[rgb]{0.9,0.9,0.9}{
            \makecell[{{p{\linewidth}}}]{
                \texttt{\tiny{[GM$|$GM]}}
                \texttt{Invalid format: tail is not empty: '} \\
\texttt{{-} P: (4,4)  } \\
\texttt{{-} O: (3,3)  } \\
\texttt{{-} T: (1,1)  } \\
\texttt{{-} S: (4,1)  } \\
\texttt{{-} D: (2,2)  } \\
\texttt{{-} A: (7,7)  } \\
\texttt{{-} M: (4,2)} \\
            }
        }
    }
    & & \\ \\

    \theutterance \stepcounter{utterance}  
    & & \multicolumn{4}{p{0.6\linewidth}}{
        \cellcolor[rgb]{0.9,0.9,0.9}{
            \makecell[{{p{\linewidth}}}]{
                \texttt{\tiny{[GM$\rangle$P2]}}
                \texttt{Strafpunkt: Deine Nachricht darf nichts nach dem Befehl enthalten! Stelle sicher, dass deine Antwort ausschließlich einen der beiden Befehle `SPRICH: <NACHRICHT>` und `VERSCHIEBE: <OBJECT>, (<X>, <Y>)` und sonst nichts enthält!} \\
\\ 
\texttt{Ihr habt gemeinsam 4 von maximal 16 Strafpunkten angesammelt.} \\
\texttt{Bitte versuche es noch mal!} \\
            }
        }
    }
    & \\ \\

    \theutterance \stepcounter{utterance}  
    & & & \multicolumn{4}{p{0.6\linewidth}}{
        \cellcolor[rgb]{0.9,0.9,0.9}{
            \makecell[{{p{\linewidth}}}]{
                \texttt{\tiny{[GM$\langle$P2]}}
                \texttt{SPRICH: Hallo! Ich bin der zweite Spieler. Meine Objekte sind wie folgt positioniert:  } \\
\texttt{{-} P: (4,4)  } \\
\texttt{{-} O: (3,3)  } \\
\texttt{{-} T: (1,1)  } \\
\texttt{{-} S: (4,1)  } \\
\texttt{{-} D: (2,2)  } \\
\texttt{{-} A: (7,7)  } \\
\texttt{{-} M: (4,2)} \\
            }
        }
    }
     \\ \\

    \theutterance \stepcounter{utterance}  
    & & & \multicolumn{2}{p{0.3\linewidth}}{
        \cellcolor[rgb]{0.9,0.9,0.9}{
            \makecell[{{p{\linewidth}}}]{
                \texttt{\tiny{[GM$|$GM]}}
                \texttt{SPRICH: Hallo! Ich bin der zweite Spieler. Meine Objekte sind wie folgt positioniert:  } \\
\texttt{{-} P: (4,4)  } \\
\texttt{{-} O: (3,3)  } \\
\texttt{{-} T: (1,1)  } \\
\texttt{{-} S: (4,1)  } \\
\texttt{{-} D: (2,2)  } \\
\texttt{{-} A: (7,7)  } \\
\texttt{{-} M: (4,2)} \\
            }
        }
    }
    & & \\ \\

    \theutterance \stepcounter{utterance}  
    & & & \multicolumn{2}{p{0.3\linewidth}}{
        \cellcolor[rgb]{0.9,0.9,0.9}{
            \makecell[{{p{\linewidth}}}]{
                \texttt{\tiny{[GM$|$GM]}}
                \texttt{Invalid format: tail is not empty: '} \\
\texttt{{-} P: (4,4)  } \\
\texttt{{-} O: (3,3)  } \\
\texttt{{-} T: (1,1)  } \\
\texttt{{-} S: (4,1)  } \\
\texttt{{-} D: (2,2)  } \\
\texttt{{-} A: (7,7)  } \\
\texttt{{-} M: (4,2)} \\
            }
        }
    }
    & & \\ \\

    \theutterance \stepcounter{utterance}  
    & & \multicolumn{4}{p{0.6\linewidth}}{
        \cellcolor[rgb]{0.9,0.9,0.9}{
            \makecell[{{p{\linewidth}}}]{
                \texttt{\tiny{[GM$\rangle$P2]}}
                \texttt{Strafpunkt: Deine Nachricht darf nichts nach dem Befehl enthalten! Stelle sicher, dass deine Antwort ausschließlich einen der beiden Befehle `SPRICH: <NACHRICHT>` und `VERSCHIEBE: <OBJECT>, (<X>, <Y>)` und sonst nichts enthält!} \\
\\ 
\texttt{Ihr habt gemeinsam 5 von maximal 16 Strafpunkten angesammelt.} \\
\texttt{Bitte versuche es noch mal!} \\
            }
        }
    }
    & \\ \\

    \theutterance \stepcounter{utterance}  
    & & & \multicolumn{4}{p{0.6\linewidth}}{
        \cellcolor[rgb]{0.9,0.9,0.9}{
            \makecell[{{p{\linewidth}}}]{
                \texttt{\tiny{[GM$\langle$P2]}}
                \texttt{SPRICH: Hallo! Ich bin der zweite Spieler. Meine Objekte sind wie folgt positioniert:  } \\
\texttt{{-} P: (4,4)  } \\
\texttt{{-} O: (3,3)  } \\
\texttt{{-} T: (1,1)  } \\
\texttt{{-} S: (4,1)  } \\
\texttt{{-} D: (2,2)  } \\
\texttt{{-} A: (7,7)  } \\
\texttt{{-} M: (4,2)} \\
            }
        }
    }
     \\ \\

    \theutterance \stepcounter{utterance}  
    & & & \multicolumn{2}{p{0.3\linewidth}}{
        \cellcolor[rgb]{0.9,0.9,0.9}{
            \makecell[{{p{\linewidth}}}]{
                \texttt{\tiny{[GM$|$GM]}}
                \texttt{SPRICH: Hallo! Ich bin der zweite Spieler. Meine Objekte sind wie folgt positioniert:  } \\
\texttt{{-} P: (4,4)  } \\
\texttt{{-} O: (3,3)  } \\
\texttt{{-} T: (1,1)  } \\
\texttt{{-} S: (4,1)  } \\
\texttt{{-} D: (2,2)  } \\
\texttt{{-} A: (7,7)  } \\
\texttt{{-} M: (4,2)} \\
            }
        }
    }
    & & \\ \\

    \theutterance \stepcounter{utterance}  
    & & & \multicolumn{2}{p{0.3\linewidth}}{
        \cellcolor[rgb]{0.9,0.9,0.9}{
            \makecell[{{p{\linewidth}}}]{
                \texttt{\tiny{[GM$|$GM]}}
                \texttt{Invalid format: tail is not empty: '} \\
\texttt{{-} P: (4,4)  } \\
\texttt{{-} O: (3,3)  } \\
\texttt{{-} T: (1,1)  } \\
\texttt{{-} S: (4,1)  } \\
\texttt{{-} D: (2,2)  } \\
\texttt{{-} A: (7,7)  } \\
\texttt{{-} M: (4,2)} \\
            }
        }
    }
    & & \\ \\

    \theutterance \stepcounter{utterance}  
    & & \multicolumn{4}{p{0.6\linewidth}}{
        \cellcolor[rgb]{0.9,0.9,0.9}{
            \makecell[{{p{\linewidth}}}]{
                \texttt{\tiny{[GM$\rangle$P2]}}
                \texttt{Strafpunkt: Deine Nachricht darf nichts nach dem Befehl enthalten! Stelle sicher, dass deine Antwort ausschließlich einen der beiden Befehle `SPRICH: <NACHRICHT>` und `VERSCHIEBE: <OBJECT>, (<X>, <Y>)` und sonst nichts enthält!} \\
\\ 
\texttt{Ihr habt gemeinsam 6 von maximal 16 Strafpunkten angesammelt.} \\
\texttt{Bitte versuche es noch mal!} \\
            }
        }
    }
    & \\ \\

    \theutterance \stepcounter{utterance}  
    & & & \multicolumn{4}{p{0.6\linewidth}}{
        \cellcolor[rgb]{0.9,0.9,0.9}{
            \makecell[{{p{\linewidth}}}]{
                \texttt{\tiny{[GM$\langle$P2]}}
                \texttt{SPRICH: Hallo! Ich bin der zweite Spieler. Meine Objekte sind wie folgt positioniert:  } \\
\texttt{{-} P: (4,4)  } \\
\texttt{{-} O: (3,3)  } \\
\texttt{{-} T: (1,1)  } \\
\texttt{{-} S: (4,1)  } \\
\texttt{{-} D: (2,2)  } \\
\texttt{{-} A: (7,7)  } \\
\texttt{{-} M: (4,2)} \\
            }
        }
    }
     \\ \\

    \theutterance \stepcounter{utterance}  
    & & & \multicolumn{2}{p{0.3\linewidth}}{
        \cellcolor[rgb]{0.9,0.9,0.9}{
            \makecell[{{p{\linewidth}}}]{
                \texttt{\tiny{[GM$|$GM]}}
                \texttt{SPRICH: Hallo! Ich bin der zweite Spieler. Meine Objekte sind wie folgt positioniert:  } \\
\texttt{{-} P: (4,4)  } \\
\texttt{{-} O: (3,3)  } \\
\texttt{{-} T: (1,1)  } \\
\texttt{{-} S: (4,1)  } \\
\texttt{{-} D: (2,2)  } \\
\texttt{{-} A: (7,7)  } \\
\texttt{{-} M: (4,2)} \\
            }
        }
    }
    & & \\ \\

    \theutterance \stepcounter{utterance}  
    & & & \multicolumn{2}{p{0.3\linewidth}}{
        \cellcolor[rgb]{0.9,0.9,0.9}{
            \makecell[{{p{\linewidth}}}]{
                \texttt{\tiny{[GM$|$GM]}}
                \texttt{Invalid format: tail is not empty: '} \\
\texttt{{-} P: (4,4)  } \\
\texttt{{-} O: (3,3)  } \\
\texttt{{-} T: (1,1)  } \\
\texttt{{-} S: (4,1)  } \\
\texttt{{-} D: (2,2)  } \\
\texttt{{-} A: (7,7)  } \\
\texttt{{-} M: (4,2)} \\
            }
        }
    }
    & & \\ \\

    \theutterance \stepcounter{utterance}  
    & & \multicolumn{4}{p{0.6\linewidth}}{
        \cellcolor[rgb]{0.9,0.9,0.9}{
            \makecell[{{p{\linewidth}}}]{
                \texttt{\tiny{[GM$\rangle$P2]}}
                \texttt{Strafpunkt: Deine Nachricht darf nichts nach dem Befehl enthalten! Stelle sicher, dass deine Antwort ausschließlich einen der beiden Befehle `SPRICH: <NACHRICHT>` und `VERSCHIEBE: <OBJECT>, (<X>, <Y>)` und sonst nichts enthält!} \\
\\ 
\texttt{Ihr habt gemeinsam 7 von maximal 16 Strafpunkten angesammelt.} \\
\texttt{Bitte versuche es noch mal!} \\
            }
        }
    }
    & \\ \\

    \theutterance \stepcounter{utterance}  
    & & & \multicolumn{4}{p{0.6\linewidth}}{
        \cellcolor[rgb]{0.9,0.9,0.9}{
            \makecell[{{p{\linewidth}}}]{
                \texttt{\tiny{[GM$\langle$P2]}}
                \texttt{SPRICH: Hallo! Ich bin der zweite Spieler. Meine Objekte sind wie folgt positioniert:  } \\
\texttt{{-} P: (4,4)  } \\
\texttt{{-} O: (3,3)  } \\
\texttt{{-} T: (1,1)  } \\
\texttt{{-} S: (4,1)  } \\
\texttt{{-} D: (2,2)  } \\
\texttt{{-} A: (7,7)  } \\
\texttt{{-} M: (4,2)} \\
            }
        }
    }
     \\ \\

    \theutterance \stepcounter{utterance}  
    & & & \multicolumn{2}{p{0.3\linewidth}}{
        \cellcolor[rgb]{0.9,0.9,0.9}{
            \makecell[{{p{\linewidth}}}]{
                \texttt{\tiny{[GM$|$GM]}}
                \texttt{SPRICH: Hallo! Ich bin der zweite Spieler. Meine Objekte sind wie folgt positioniert:  } \\
\texttt{{-} P: (4,4)  } \\
\texttt{{-} O: (3,3)  } \\
\texttt{{-} T: (1,1)  } \\
\texttt{{-} S: (4,1)  } \\
\texttt{{-} D: (2,2)  } \\
\texttt{{-} A: (7,7)  } \\
\texttt{{-} M: (4,2)} \\
            }
        }
    }
    & & \\ \\

    \theutterance \stepcounter{utterance}  
    & & & \multicolumn{2}{p{0.3\linewidth}}{
        \cellcolor[rgb]{0.9,0.9,0.9}{
            \makecell[{{p{\linewidth}}}]{
                \texttt{\tiny{[GM$|$GM]}}
                \texttt{Invalid format: tail is not empty: '} \\
\texttt{{-} P: (4,4)  } \\
\texttt{{-} O: (3,3)  } \\
\texttt{{-} T: (1,1)  } \\
\texttt{{-} S: (4,1)  } \\
\texttt{{-} D: (2,2)  } \\
\texttt{{-} A: (7,7)  } \\
\texttt{{-} M: (4,2)} \\
            }
        }
    }
    & & \\ \\

    \theutterance \stepcounter{utterance}  
    & & \multicolumn{4}{p{0.6\linewidth}}{
        \cellcolor[rgb]{0.9,0.9,0.9}{
            \makecell[{{p{\linewidth}}}]{
                \texttt{\tiny{[GM$\rangle$P2]}}
                \texttt{Strafpunkt: Deine Nachricht darf nichts nach dem Befehl enthalten! Stelle sicher, dass deine Antwort ausschließlich einen der beiden Befehle `SPRICH: <NACHRICHT>` und `VERSCHIEBE: <OBJECT>, (<X>, <Y>)` und sonst nichts enthält!} \\
\\ 
\texttt{Ihr habt gemeinsam 8 von maximal 16 Strafpunkten angesammelt.} \\
\texttt{Bitte versuche es noch mal!} \\
            }
        }
    }
    & \\ \\

    \theutterance \stepcounter{utterance}  
    & & & \multicolumn{4}{p{0.6\linewidth}}{
        \cellcolor[rgb]{0.9,0.9,0.9}{
            \makecell[{{p{\linewidth}}}]{
                \texttt{\tiny{[GM$\langle$P2]}}
                \texttt{SPRICH: Hallo! Ich bin der zweite Spieler. Meine Objekte sind wie folgt positioniert:  } \\
\texttt{{-} P: (4,4)  } \\
\texttt{{-} O: (3,3)  } \\
\texttt{{-} T: (1,1)  } \\
\texttt{{-} S: (4,1)  } \\
\texttt{{-} D: (2,2)  } \\
\texttt{{-} A: (7,7)  } \\
\texttt{{-} M: (4,2)} \\
            }
        }
    }
     \\ \\

    \theutterance \stepcounter{utterance}  
    & & & \multicolumn{2}{p{0.3\linewidth}}{
        \cellcolor[rgb]{0.9,0.9,0.9}{
            \makecell[{{p{\linewidth}}}]{
                \texttt{\tiny{[GM$|$GM]}}
                \texttt{SPRICH: Hallo! Ich bin der zweite Spieler. Meine Objekte sind wie folgt positioniert:  } \\
\texttt{{-} P: (4,4)  } \\
\texttt{{-} O: (3,3)  } \\
\texttt{{-} T: (1,1)  } \\
\texttt{{-} S: (4,1)  } \\
\texttt{{-} D: (2,2)  } \\
\texttt{{-} A: (7,7)  } \\
\texttt{{-} M: (4,2)} \\
            }
        }
    }
    & & \\ \\

    \theutterance \stepcounter{utterance}  
    & & & \multicolumn{2}{p{0.3\linewidth}}{
        \cellcolor[rgb]{0.9,0.9,0.9}{
            \makecell[{{p{\linewidth}}}]{
                \texttt{\tiny{[GM$|$GM]}}
                \texttt{Invalid format: tail is not empty: '} \\
\texttt{{-} P: (4,4)  } \\
\texttt{{-} O: (3,3)  } \\
\texttt{{-} T: (1,1)  } \\
\texttt{{-} S: (4,1)  } \\
\texttt{{-} D: (2,2)  } \\
\texttt{{-} A: (7,7)  } \\
\texttt{{-} M: (4,2)} \\
            }
        }
    }
    & & \\ \\

    \theutterance \stepcounter{utterance}  
    & & \multicolumn{4}{p{0.6\linewidth}}{
        \cellcolor[rgb]{0.9,0.9,0.9}{
            \makecell[{{p{\linewidth}}}]{
                \texttt{\tiny{[GM$\rangle$P2]}}
                \texttt{Strafpunkt: Deine Nachricht darf nichts nach dem Befehl enthalten! Stelle sicher, dass deine Antwort ausschließlich einen der beiden Befehle `SPRICH: <NACHRICHT>` und `VERSCHIEBE: <OBJECT>, (<X>, <Y>)` und sonst nichts enthält!} \\
\\ 
\texttt{Ihr habt gemeinsam 9 von maximal 16 Strafpunkten angesammelt.} \\
\texttt{Bitte versuche es noch mal!} \\
            }
        }
    }
    & \\ \\

    \theutterance \stepcounter{utterance}  
    & & & \multicolumn{4}{p{0.6\linewidth}}{
        \cellcolor[rgb]{0.9,0.9,0.9}{
            \makecell[{{p{\linewidth}}}]{
                \texttt{\tiny{[GM$\langle$P2]}}
                \texttt{SPRICH: Hallo! Ich bin der zweite Spieler. Meine Objekte sind wie folgt positioniert:  } \\
\texttt{{-} P: (4,4)  } \\
\texttt{{-} O: (3,3)  } \\
\texttt{{-} T: (1,1)  } \\
\texttt{{-} S: (4,1)  } \\
\texttt{{-} D: (2,2)  } \\
\texttt{{-} A: (7,7)  } \\
\texttt{{-} M: (4,2)} \\
            }
        }
    }
     \\ \\

    \theutterance \stepcounter{utterance}  
    & & & \multicolumn{2}{p{0.3\linewidth}}{
        \cellcolor[rgb]{0.9,0.9,0.9}{
            \makecell[{{p{\linewidth}}}]{
                \texttt{\tiny{[GM$|$GM]}}
                \texttt{SPRICH: Hallo! Ich bin der zweite Spieler. Meine Objekte sind wie folgt positioniert:  } \\
\texttt{{-} P: (4,4)  } \\
\texttt{{-} O: (3,3)  } \\
\texttt{{-} T: (1,1)  } \\
\texttt{{-} S: (4,1)  } \\
\texttt{{-} D: (2,2)  } \\
\texttt{{-} A: (7,7)  } \\
\texttt{{-} M: (4,2)} \\
            }
        }
    }
    & & \\ \\

    \theutterance \stepcounter{utterance}  
    & & & \multicolumn{2}{p{0.3\linewidth}}{
        \cellcolor[rgb]{0.9,0.9,0.9}{
            \makecell[{{p{\linewidth}}}]{
                \texttt{\tiny{[GM$|$GM]}}
                \texttt{Invalid format: tail is not empty: '} \\
\texttt{{-} P: (4,4)  } \\
\texttt{{-} O: (3,3)  } \\
\texttt{{-} T: (1,1)  } \\
\texttt{{-} S: (4,1)  } \\
\texttt{{-} D: (2,2)  } \\
\texttt{{-} A: (7,7)  } \\
\texttt{{-} M: (4,2)} \\
            }
        }
    }
    & & \\ \\

    \theutterance \stepcounter{utterance}  
    & & \multicolumn{4}{p{0.6\linewidth}}{
        \cellcolor[rgb]{0.9,0.9,0.9}{
            \makecell[{{p{\linewidth}}}]{
                \texttt{\tiny{[GM$\rangle$P2]}}
                \texttt{Strafpunkt: Deine Nachricht darf nichts nach dem Befehl enthalten! Stelle sicher, dass deine Antwort ausschließlich einen der beiden Befehle `SPRICH: <NACHRICHT>` und `VERSCHIEBE: <OBJECT>, (<X>, <Y>)` und sonst nichts enthält!} \\
\\ 
\texttt{Ihr habt gemeinsam 10 von maximal 16 Strafpunkten angesammelt.} \\
\texttt{Bitte versuche es noch mal!} \\
            }
        }
    }
    & \\ \\

    \theutterance \stepcounter{utterance}  
    & & & \multicolumn{4}{p{0.6\linewidth}}{
        \cellcolor[rgb]{0.9,0.9,0.9}{
            \makecell[{{p{\linewidth}}}]{
                \texttt{\tiny{[GM$\langle$P2]}}
                \texttt{SPRICH: Hallo! Ich bin der zweite Spieler. Meine Objekte sind wie folgt positioniert:  } \\
\texttt{{-} P: (4,4)  } \\
\texttt{{-} O: (3,3)  } \\
\texttt{{-} T: (1,1)  } \\
\texttt{{-} S: (4,1)  } \\
\texttt{{-} D: (2,2)  } \\
\texttt{{-} A: (7,7)  } \\
\texttt{{-} M: (4,2)} \\
            }
        }
    }
     \\ \\

    \theutterance \stepcounter{utterance}  
    & & & \multicolumn{2}{p{0.3\linewidth}}{
        \cellcolor[rgb]{0.9,0.9,0.9}{
            \makecell[{{p{\linewidth}}}]{
                \texttt{\tiny{[GM$|$GM]}}
                \texttt{SPRICH: Hallo! Ich bin der zweite Spieler. Meine Objekte sind wie folgt positioniert:  } \\
\texttt{{-} P: (4,4)  } \\
\texttt{{-} O: (3,3)  } \\
\texttt{{-} T: (1,1)  } \\
\texttt{{-} S: (4,1)  } \\
\texttt{{-} D: (2,2)  } \\
\texttt{{-} A: (7,7)  } \\
\texttt{{-} M: (4,2)} \\
            }
        }
    }
    & & \\ \\

    \theutterance \stepcounter{utterance}  
    & & & \multicolumn{2}{p{0.3\linewidth}}{
        \cellcolor[rgb]{0.9,0.9,0.9}{
            \makecell[{{p{\linewidth}}}]{
                \texttt{\tiny{[GM$|$GM]}}
                \texttt{Invalid format: tail is not empty: '} \\
\texttt{{-} P: (4,4)  } \\
\texttt{{-} O: (3,3)  } \\
\texttt{{-} T: (1,1)  } \\
\texttt{{-} S: (4,1)  } \\
\texttt{{-} D: (2,2)  } \\
\texttt{{-} A: (7,7)  } \\
\texttt{{-} M: (4,2)} \\
            }
        }
    }
    & & \\ \\

    \theutterance \stepcounter{utterance}  
    & & \multicolumn{4}{p{0.6\linewidth}}{
        \cellcolor[rgb]{0.9,0.9,0.9}{
            \makecell[{{p{\linewidth}}}]{
                \texttt{\tiny{[GM$\rangle$P2]}}
                \texttt{Strafpunkt: Deine Nachricht darf nichts nach dem Befehl enthalten! Stelle sicher, dass deine Antwort ausschließlich einen der beiden Befehle `SPRICH: <NACHRICHT>` und `VERSCHIEBE: <OBJECT>, (<X>, <Y>)` und sonst nichts enthält!} \\
\\ 
\texttt{Ihr habt gemeinsam 11 von maximal 16 Strafpunkten angesammelt.} \\
\texttt{Bitte versuche es noch mal!} \\
            }
        }
    }
    & \\ \\

    \theutterance \stepcounter{utterance}  
    & & & \multicolumn{4}{p{0.6\linewidth}}{
        \cellcolor[rgb]{0.9,0.9,0.9}{
            \makecell[{{p{\linewidth}}}]{
                \texttt{\tiny{[GM$\langle$P2]}}
                \texttt{SPRICH: Hallo! Ich bin der zweite Spieler. Meine Objekte sind wie folgt positioniert:  } \\
\texttt{{-} P: (4,4)  } \\
\texttt{{-} O: (3,3)  } \\
\texttt{{-} T: (1,1)  } \\
\texttt{{-} S: (4,1)  } \\
\texttt{{-} D: (2,2)  } \\
\texttt{{-} A: (7,7)  } \\
\texttt{{-} M: (4,2)} \\
            }
        }
    }
     \\ \\

    \theutterance \stepcounter{utterance}  
    & & & \multicolumn{2}{p{0.3\linewidth}}{
        \cellcolor[rgb]{0.9,0.9,0.9}{
            \makecell[{{p{\linewidth}}}]{
                \texttt{\tiny{[GM$|$GM]}}
                \texttt{SPRICH: Hallo! Ich bin der zweite Spieler. Meine Objekte sind wie folgt positioniert:  } \\
\texttt{{-} P: (4,4)  } \\
\texttt{{-} O: (3,3)  } \\
\texttt{{-} T: (1,1)  } \\
\texttt{{-} S: (4,1)  } \\
\texttt{{-} D: (2,2)  } \\
\texttt{{-} A: (7,7)  } \\
\texttt{{-} M: (4,2)} \\
            }
        }
    }
    & & \\ \\

    \theutterance \stepcounter{utterance}  
    & & & \multicolumn{2}{p{0.3\linewidth}}{
        \cellcolor[rgb]{0.9,0.9,0.9}{
            \makecell[{{p{\linewidth}}}]{
                \texttt{\tiny{[GM$|$GM]}}
                \texttt{Invalid format: tail is not empty: '} \\
\texttt{{-} P: (4,4)  } \\
\texttt{{-} O: (3,3)  } \\
\texttt{{-} T: (1,1)  } \\
\texttt{{-} S: (4,1)  } \\
\texttt{{-} D: (2,2)  } \\
\texttt{{-} A: (7,7)  } \\
\texttt{{-} M: (4,2)} \\
            }
        }
    }
    & & \\ \\

    \theutterance \stepcounter{utterance}  
    & & \multicolumn{4}{p{0.6\linewidth}}{
        \cellcolor[rgb]{0.9,0.9,0.9}{
            \makecell[{{p{\linewidth}}}]{
                \texttt{\tiny{[GM$\rangle$P2]}}
                \texttt{Strafpunkt: Deine Nachricht darf nichts nach dem Befehl enthalten! Stelle sicher, dass deine Antwort ausschließlich einen der beiden Befehle `SPRICH: <NACHRICHT>` und `VERSCHIEBE: <OBJECT>, (<X>, <Y>)` und sonst nichts enthält!} \\
\\ 
\texttt{Ihr habt gemeinsam 12 von maximal 16 Strafpunkten angesammelt.} \\
\texttt{Bitte versuche es noch mal!} \\
            }
        }
    }
    & \\ \\

    \theutterance \stepcounter{utterance}  
    & & & \multicolumn{4}{p{0.6\linewidth}}{
        \cellcolor[rgb]{0.9,0.9,0.9}{
            \makecell[{{p{\linewidth}}}]{
                \texttt{\tiny{[GM$\langle$P2]}}
                \texttt{SPRICH: Hallo! Ich bin der zweite Spieler. Meine Objekte sind wie folgt positioniert:  } \\
\texttt{{-} P: (4,4)  } \\
\texttt{{-} O: (3,3)  } \\
\texttt{{-} T: (1,1)  } \\
\texttt{{-} S: (4,1)  } \\
\texttt{{-} D: (2,2)  } \\
\texttt{{-} A: (7,7)  } \\
\texttt{{-} M: (4,2)} \\
            }
        }
    }
     \\ \\

    \theutterance \stepcounter{utterance}  
    & & & \multicolumn{2}{p{0.3\linewidth}}{
        \cellcolor[rgb]{0.9,0.9,0.9}{
            \makecell[{{p{\linewidth}}}]{
                \texttt{\tiny{[GM$|$GM]}}
                \texttt{SPRICH: Hallo! Ich bin der zweite Spieler. Meine Objekte sind wie folgt positioniert:  } \\
\texttt{{-} P: (4,4)  } \\
\texttt{{-} O: (3,3)  } \\
\texttt{{-} T: (1,1)  } \\
\texttt{{-} S: (4,1)  } \\
\texttt{{-} D: (2,2)  } \\
\texttt{{-} A: (7,7)  } \\
\texttt{{-} M: (4,2)} \\
            }
        }
    }
    & & \\ \\

    \theutterance \stepcounter{utterance}  
    & & & \multicolumn{2}{p{0.3\linewidth}}{
        \cellcolor[rgb]{0.9,0.9,0.9}{
            \makecell[{{p{\linewidth}}}]{
                \texttt{\tiny{[GM$|$GM]}}
                \texttt{Invalid format: tail is not empty: '} \\
\texttt{{-} P: (4,4)  } \\
\texttt{{-} O: (3,3)  } \\
\texttt{{-} T: (1,1)  } \\
\texttt{{-} S: (4,1)  } \\
\texttt{{-} D: (2,2)  } \\
\texttt{{-} A: (7,7)  } \\
\texttt{{-} M: (4,2)} \\
            }
        }
    }
    & & \\ \\

    \theutterance \stepcounter{utterance}  
    & & \multicolumn{4}{p{0.6\linewidth}}{
        \cellcolor[rgb]{0.9,0.9,0.9}{
            \makecell[{{p{\linewidth}}}]{
                \texttt{\tiny{[GM$\rangle$P2]}}
                \texttt{Strafpunkt: Deine Nachricht darf nichts nach dem Befehl enthalten! Stelle sicher, dass deine Antwort ausschließlich einen der beiden Befehle `SPRICH: <NACHRICHT>` und `VERSCHIEBE: <OBJECT>, (<X>, <Y>)` und sonst nichts enthält!} \\
\\ 
\texttt{Ihr habt gemeinsam 13 von maximal 16 Strafpunkten angesammelt.} \\
\texttt{Bitte versuche es noch mal!} \\
            }
        }
    }
    & \\ \\

    \theutterance \stepcounter{utterance}  
    & & & \multicolumn{4}{p{0.6\linewidth}}{
        \cellcolor[rgb]{0.9,0.9,0.9}{
            \makecell[{{p{\linewidth}}}]{
                \texttt{\tiny{[GM$\langle$P2]}}
                \texttt{SPRICH: Hallo! Ich bin der zweite Spieler. Meine Objekte sind wie folgt positioniert:  } \\
\texttt{{-} P: (4,4)  } \\
\texttt{{-} O: (3,3)  } \\
\texttt{{-} T: (1,1)  } \\
\texttt{{-} S: (4,1)  } \\
\texttt{{-} D: (2,2)  } \\
\texttt{{-} A: (7,7)  } \\
\texttt{{-} M: (4,2)} \\
            }
        }
    }
     \\ \\

    \theutterance \stepcounter{utterance}  
    & & & \multicolumn{2}{p{0.3\linewidth}}{
        \cellcolor[rgb]{0.9,0.9,0.9}{
            \makecell[{{p{\linewidth}}}]{
                \texttt{\tiny{[GM$|$GM]}}
                \texttt{SPRICH: Hallo! Ich bin der zweite Spieler. Meine Objekte sind wie folgt positioniert:  } \\
\texttt{{-} P: (4,4)  } \\
\texttt{{-} O: (3,3)  } \\
\texttt{{-} T: (1,1)  } \\
\texttt{{-} S: (4,1)  } \\
\texttt{{-} D: (2,2)  } \\
\texttt{{-} A: (7,7)  } \\
\texttt{{-} M: (4,2)} \\
            }
        }
    }
    & & \\ \\

    \theutterance \stepcounter{utterance}  
    & & & \multicolumn{2}{p{0.3\linewidth}}{
        \cellcolor[rgb]{0.9,0.9,0.9}{
            \makecell[{{p{\linewidth}}}]{
                \texttt{\tiny{[GM$|$GM]}}
                \texttt{Invalid format: tail is not empty: '} \\
\texttt{{-} P: (4,4)  } \\
\texttt{{-} O: (3,3)  } \\
\texttt{{-} T: (1,1)  } \\
\texttt{{-} S: (4,1)  } \\
\texttt{{-} D: (2,2)  } \\
\texttt{{-} A: (7,7)  } \\
\texttt{{-} M: (4,2)} \\
            }
        }
    }
    & & \\ \\

    \theutterance \stepcounter{utterance}  
    & & \multicolumn{4}{p{0.6\linewidth}}{
        \cellcolor[rgb]{0.9,0.9,0.9}{
            \makecell[{{p{\linewidth}}}]{
                \texttt{\tiny{[GM$\rangle$P2]}}
                \texttt{Strafpunkt: Deine Nachricht darf nichts nach dem Befehl enthalten! Stelle sicher, dass deine Antwort ausschließlich einen der beiden Befehle `SPRICH: <NACHRICHT>` und `VERSCHIEBE: <OBJECT>, (<X>, <Y>)` und sonst nichts enthält!} \\
\\ 
\texttt{Ihr habt gemeinsam 14 von maximal 16 Strafpunkten angesammelt.} \\
\texttt{Bitte versuche es noch mal!} \\
            }
        }
    }
    & \\ \\

    \theutterance \stepcounter{utterance}  
    & & & \multicolumn{4}{p{0.6\linewidth}}{
        \cellcolor[rgb]{0.9,0.9,0.9}{
            \makecell[{{p{\linewidth}}}]{
                \texttt{\tiny{[GM$\langle$P2]}}
                \texttt{SPRICH: Hallo! Ich bin der zweite Spieler. Meine Objekte sind wie folgt positioniert:  } \\
\texttt{{-} P: (4,4)  } \\
\texttt{{-} O: (3,3)  } \\
\texttt{{-} T: (1,1)  } \\
\texttt{{-} S: (4,1)  } \\
\texttt{{-} D: (2,2)  } \\
\texttt{{-} A: (7,7)  } \\
\texttt{{-} M: (4,2)} \\
            }
        }
    }
     \\ \\

    \theutterance \stepcounter{utterance}  
    & & & \multicolumn{2}{p{0.3\linewidth}}{
        \cellcolor[rgb]{0.9,0.9,0.9}{
            \makecell[{{p{\linewidth}}}]{
                \texttt{\tiny{[GM$|$GM]}}
                \texttt{SPRICH: Hallo! Ich bin der zweite Spieler. Meine Objekte sind wie folgt positioniert:  } \\
\texttt{{-} P: (4,4)  } \\
\texttt{{-} O: (3,3)  } \\
\texttt{{-} T: (1,1)  } \\
\texttt{{-} S: (4,1)  } \\
\texttt{{-} D: (2,2)  } \\
\texttt{{-} A: (7,7)  } \\
\texttt{{-} M: (4,2)} \\
            }
        }
    }
    & & \\ \\

    \theutterance \stepcounter{utterance}  
    & & & \multicolumn{2}{p{0.3\linewidth}}{
        \cellcolor[rgb]{0.9,0.9,0.9}{
            \makecell[{{p{\linewidth}}}]{
                \texttt{\tiny{[GM$|$GM]}}
                \texttt{Invalid format: tail is not empty: '} \\
\texttt{{-} P: (4,4)  } \\
\texttt{{-} O: (3,3)  } \\
\texttt{{-} T: (1,1)  } \\
\texttt{{-} S: (4,1)  } \\
\texttt{{-} D: (2,2)  } \\
\texttt{{-} A: (7,7)  } \\
\texttt{{-} M: (4,2)} \\
            }
        }
    }
    & & \\ \\

    \theutterance \stepcounter{utterance}  
    & & \multicolumn{4}{p{0.6\linewidth}}{
        \cellcolor[rgb]{0.9,0.9,0.9}{
            \makecell[{{p{\linewidth}}}]{
                \texttt{\tiny{[GM$\rangle$P2]}}
                \texttt{Strafpunkt: Deine Nachricht darf nichts nach dem Befehl enthalten! Stelle sicher, dass deine Antwort ausschließlich einen der beiden Befehle `SPRICH: <NACHRICHT>` und `VERSCHIEBE: <OBJECT>, (<X>, <Y>)` und sonst nichts enthält!} \\
\\ 
\texttt{Ihr habt gemeinsam 15 von maximal 16 Strafpunkten angesammelt.} \\
\texttt{Bitte versuche es noch mal!} \\
            }
        }
    }
    & \\ \\

    \theutterance \stepcounter{utterance}  
    & & & \multicolumn{4}{p{0.6\linewidth}}{
        \cellcolor[rgb]{0.9,0.9,0.9}{
            \makecell[{{p{\linewidth}}}]{
                \texttt{\tiny{[GM$\langle$P2]}}
                \texttt{SPRICH: Hallo! Ich bin der zweite Spieler. Meine Objekte sind wie folgt positioniert:  } \\
\texttt{{-} P: (4,4)  } \\
\texttt{{-} O: (3,3)  } \\
\texttt{{-} T: (1,1)  } \\
\texttt{{-} S: (4,1)  } \\
\texttt{{-} D: (2,2)  } \\
\texttt{{-} A: (7,7)  } \\
\texttt{{-} M: (4,2)} \\
            }
        }
    }
     \\ \\

    \theutterance \stepcounter{utterance}  
    & & & \multicolumn{2}{p{0.3\linewidth}}{
        \cellcolor[rgb]{0.9,0.9,0.9}{
            \makecell[{{p{\linewidth}}}]{
                \texttt{\tiny{[GM$|$GM]}}
                \texttt{SPRICH: Hallo! Ich bin der zweite Spieler. Meine Objekte sind wie folgt positioniert:  } \\
\texttt{{-} P: (4,4)  } \\
\texttt{{-} O: (3,3)  } \\
\texttt{{-} T: (1,1)  } \\
\texttt{{-} S: (4,1)  } \\
\texttt{{-} D: (2,2)  } \\
\texttt{{-} A: (7,7)  } \\
\texttt{{-} M: (4,2)} \\
            }
        }
    }
    & & \\ \\

    \theutterance \stepcounter{utterance}  
    & & & \multicolumn{2}{p{0.3\linewidth}}{
        \cellcolor[rgb]{0.9,0.9,0.9}{
            \makecell[{{p{\linewidth}}}]{
                \texttt{\tiny{[GM$|$GM]}}
                \texttt{Invalid format: tail is not empty: '} \\
\texttt{{-} P: (4,4)  } \\
\texttt{{-} O: (3,3)  } \\
\texttt{{-} T: (1,1)  } \\
\texttt{{-} S: (4,1)  } \\
\texttt{{-} D: (2,2)  } \\
\texttt{{-} A: (7,7)  } \\
\texttt{{-} M: (4,2)} \\
            }
        }
    }
    & & \\ \\

    \theutterance \stepcounter{utterance}  
    & & \multicolumn{4}{p{0.6\linewidth}}{
        \cellcolor[rgb]{0.9,0.9,0.9}{
            \makecell[{{p{\linewidth}}}]{
                \texttt{\tiny{[GM$\rangle$P2]}}
                \texttt{Strafpunkt: Deine Nachricht darf nichts nach dem Befehl enthalten! Stelle sicher, dass deine Antwort ausschließlich einen der beiden Befehle `SPRICH: <NACHRICHT>` und `VERSCHIEBE: <OBJECT>, (<X>, <Y>)` und sonst nichts enthält!} \\
\\ 
\texttt{Ihr habt gemeinsam 16 von maximal 16 Strafpunkten angesammelt.} \\
\texttt{Bitte versuche es noch mal!} \\
            }
        }
    }
    & \\ \\

    \theutterance \stepcounter{utterance}  
    & & & \multicolumn{4}{p{0.6\linewidth}}{
        \cellcolor[rgb]{0.9,0.9,0.9}{
            \makecell[{{p{\linewidth}}}]{
                \texttt{\tiny{[GM$\langle$P2]}}
                \texttt{SPRICH: Hallo! Ich bin der zweite Spieler. Meine Objekte sind wie folgt positioniert:  } \\
\texttt{{-} P: (4,4)  } \\
\texttt{{-} O: (3,3)  } \\
\texttt{{-} T: (1,1)  } \\
\texttt{{-} S: (4,1)  } \\
\texttt{{-} D: (2,2)  } \\
\texttt{{-} A: (7,7)  } \\
\texttt{{-} M: (4,2)} \\
            }
        }
    }
     \\ \\

    \theutterance \stepcounter{utterance}  
    & & & \multicolumn{2}{p{0.3\linewidth}}{
        \cellcolor[rgb]{0.9,0.9,0.9}{
            \makecell[{{p{\linewidth}}}]{
                \texttt{\tiny{[GM$|$GM]}}
                \texttt{SPRICH: Hallo! Ich bin der zweite Spieler. Meine Objekte sind wie folgt positioniert:  } \\
\texttt{{-} P: (4,4)  } \\
\texttt{{-} O: (3,3)  } \\
\texttt{{-} T: (1,1)  } \\
\texttt{{-} S: (4,1)  } \\
\texttt{{-} D: (2,2)  } \\
\texttt{{-} A: (7,7)  } \\
\texttt{{-} M: (4,2)} \\
            }
        }
    }
    & & \\ \\

    \theutterance \stepcounter{utterance}  
    & & & \multicolumn{2}{p{0.3\linewidth}}{
        \cellcolor[rgb]{0.9,0.9,0.9}{
            \makecell[{{p{\linewidth}}}]{
                \texttt{\tiny{[GM$|$GM]}}
                \texttt{Invalid format: tail is not empty: '} \\
\texttt{{-} P: (4,4)  } \\
\texttt{{-} O: (3,3)  } \\
\texttt{{-} T: (1,1)  } \\
\texttt{{-} S: (4,1)  } \\
\texttt{{-} D: (2,2)  } \\
\texttt{{-} A: (7,7)  } \\
\texttt{{-} M: (4,2)} \\
            }
        }
    }
    & & \\ \\

    \theutterance \stepcounter{utterance}  
    & & & \multicolumn{2}{p{0.3\linewidth}}{
        \cellcolor[rgb]{0.9,0.9,0.9}{
            \makecell[{{p{\linewidth}}}]{
                \texttt{\tiny{[GM$|$GM]}}
                \texttt{Maximum number of penalties exceeded} \\
            }
        }
    }
    & & \\ \\

    \theutterance \stepcounter{utterance}  
    & & & \multicolumn{2}{p{0.3\linewidth}}{
        \cellcolor[rgb]{0.9,0.9,0.9}{
            \makecell[{{p{\linewidth}}}]{
                \texttt{\tiny{[GM$|$GM]}}
                \texttt{* success: False} \\
\texttt{* lose: True} \\
\texttt{* aborted: True} \\
\texttt{{-}{-}{-}{-}{-}{-}{-}} \\
\texttt{* Shifts: 0.00} \\
\texttt{* Max Shifts: 12.00} \\
\texttt{* Min Shifts: 6.00} \\
\texttt{* End Distance Sum: 18.90} \\
\texttt{* Init Distance Sum: 18.90} \\
\texttt{* Expected Distance Sum: 29.33} \\
\texttt{* Penalties: 17.00} \\
\texttt{* Max Penalties: 16.00} \\
\texttt{* Rounds: 0.00} \\
\texttt{* Max Rounds: 28.00} \\
\texttt{* Object Count: 7.00} \\
            }
        }
    }
    & & \\ \\

    \theutterance \stepcounter{utterance}  
    & & & \multicolumn{2}{p{0.3\linewidth}}{
        \cellcolor[rgb]{0.9,0.9,0.9}{
            \makecell[{{p{\linewidth}}}]{
                \texttt{\tiny{[GM$|$GM]}}
                \texttt{Initial Grids:} \\
\texttt{```} \\
\texttt{Player 1:     Player 2:} \\
\texttt{ 1234567       1234567} \\
\texttt{╔═══════╗     ╔═══════╗} \\
\texttt{║◌◌P◌◌◌◌║ 1   ║T◌◌◌S◌◌║ 1} \\
\texttt{║◌◌◌◌◌◌◌║ 2   ║◌◌◌D◌◌◌║ 2} \\
\texttt{║◌◌◌◌D┌─╢ 3   ║◌◌O◌◌┌─╢ 3} \\
\texttt{║◌TM┌─┤◌║ 4   ║◌M◌┌─┤P║ 4} \\
\texttt{║◌◌O│◌├─╢ 5   ║◌◌◌│◌├─╢ 5} \\
\texttt{╟───┼─┘◌║ 6   ╟───┼─┘◌║ 6} \\
\texttt{║◌◌◌│◌AS║ 7   ║◌◌◌│◌A◌║ 7} \\
\texttt{╚═══╧═══╝     ╚═══╧═══╝} \\
\texttt{```} \\
\\ 
\texttt{Final Grids:} \\
\texttt{```} \\
\texttt{Player 1:     Player 2:} \\
\texttt{ 1234567       1234567} \\
\texttt{╔═══════╗     ╔═══════╗} \\
\texttt{║◌◌P◌◌◌◌║ 1   ║T◌◌◌S◌◌║ 1} \\
\texttt{║◌◌◌◌◌◌◌║ 2   ║◌◌◌D◌◌◌║ 2} \\
\texttt{║◌◌◌◌D┌─╢ 3   ║◌◌O◌◌┌─╢ 3} \\
\texttt{║◌TM┌─┤◌║ 4   ║◌M◌┌─┤P║ 4} \\
\texttt{║◌◌O│◌├─╢ 5   ║◌◌◌│◌├─╢ 5} \\
\texttt{╟───┼─┘◌║ 6   ╟───┼─┘◌║ 6} \\
\texttt{║◌◌◌│◌AS║ 7   ║◌◌◌│◌A◌║ 7} \\
\texttt{╚═══╧═══╝     ╚═══╧═══╝} \\
\texttt{```} \\
            }
        }
    }
    & & \\ \\

\end{supertabular}
}

\end{document}
