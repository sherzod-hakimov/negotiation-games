
\documentclass{article}
\usepackage{colortbl}
\usepackage{makecell}
\usepackage{multirow}
\usepackage{supertabular}

\begin{document}

\newcounter{utterance}

\centering \large Interaction Transcript for game `clean\_up', experiment `1\_medium\_7obj\_de', episode 2 with gpt{-}5{-}mini{-}2025{-}08{-}07{-}t1.0.
\vspace{24pt}

{ \footnotesize  \setcounter{utterance}{1}
\setlength{\tabcolsep}{0pt}
\begin{supertabular}{c@{$\;$}|p{.15\linewidth}@{}p{.15\linewidth}p{.15\linewidth}p{.15\linewidth}p{.15\linewidth}p{.15\linewidth}}
    \# & $\;$A & \multicolumn{4}{c}{Game Master} & $\;\:$B\\
    \hline

    \theutterance \stepcounter{utterance}  
    & & \multicolumn{4}{p{0.6\linewidth}}{
        \cellcolor[rgb]{0.9,0.9,0.9}{
            \makecell[{{p{\linewidth}}}]{
                \texttt{\tiny{[P1$\langle$GM]}}
                \texttt{Ich bin der Spielleiter, und du spielst ein kooperatives Spiel mit dem folgenden Raster als Spielbrett:} \\
\\ 
\texttt{```} \\
\texttt{ 1234567} \\
\texttt{╔═╤═╤═╤═╗} \\
\texttt{║◌│◌│◌│A║ 1} \\
\texttt{╟─┴─┼─┤◌║ 2} \\
\texttt{║◌◌M│◌└─╢ 3} \\
\texttt{╟───┘◌S◌║ 4} \\
\texttt{║◌◌◌◌◌┌─╢ 5} \\
\texttt{║◌◌◌PT│O║ 6} \\
\texttt{║◌D◌◌◌│◌║ 7} \\
\texttt{╚═════╧═╝} \\
\\ 
\texttt{```} \\
\\ 
\texttt{* Am oberen Rand sind die x{-}Koordinaten angegeben, die nach rechts hin zunehmen, und am rechten Rand die y{-}Koordinaten, die nach unten zunehmen.} \\
\texttt{* Die folgenden Objekte sind zufällig auf deinem Raster platziert: 'P', 'O', 'T', 'S', 'D', 'A', 'M'.} \\
\\ 
\\ 
\texttt{Dein Mitspieler sieht eine Variation des Spielbretts, auf dem dieselben Objekte an anderen zufälligen Positionen platziert sind. Du kannst das Spielbrett des anderen Spielers nicht sehen, und er kann deins nicht sehen.} \\
\\ 
\\ 
\texttt{**Ziel des Spiels:**} \\
\\ 
\texttt{Beide Spieler müssen die Objekte auf ihrem jeweiligen Spielbrett so bewegen, dass identische Objekte am Ende an denselben Koordinaten liegen. Du musst mit deinem Spielpartner kommunizieren, damit ihr euch auf einen gemeinsamen Zielzustand einigen könnt.} \\
\\ 
\\ 
\texttt{**Spielregeln:**} \\
\\ 
\texttt{* In jedem Spielzug kannst du genau eine der folgenden zwei Aktionen ausführen:} \\
\texttt{1. `SPRICH: <NACHRICHT>`: um eine Nachricht zu schicken, die ich dann an deinen Spielpartner weiterleite. Als Nachricht gilt alles bis zum nächsten Zeilenumbruch.} \\
\texttt{2. `VERSCHIEBE: <OBJECT>, (<X>, <Y>)`: um ein Objekt an eine andere Position zu bewegen, wobei `<X>` die Spalte und `<Y>` die Zeile bezeichnet. Ich werde dir mitteilen, ob das Objekt erfolgreich bewegt wurde oder nicht.} \\
\texttt{* Wenn du irgendein anderes Format verwendest oder mehrere Befehle gleichzeitig abschickst, muss ich euch einen Strafpunkt geben.} \\
\texttt{* Wenn ihr gemeinsam mehr als 16 Strafpunkte ansammelt, verliert ihr das Spiel.} \\
\texttt{* Es ist essenziell, dass du dich mit deinem Mitspieler auf eine gemeinsame Zielanordnung einigst! Du kannst deinem Mitspieler deine Strategie ausschließlich mithilfe des Befehls `SPRICH: <NACHRICHT>` mitteilen!} \\
\\ 
\\ 
\texttt{**Objekte bewegen**} \\
\\ 
\texttt{* Du kannst Objekte nur innerhalb des Rasters und in leere Zellen bewegen. Leere Zellen erkennst am Zeichen '◌'.} \\
\texttt{* Solltest du versuchen, ein Objekt an eine Stelle zu bewegen, die entweder nicht leer ist oder sich außerhalb des Rasters befindet, muss ich euch einen Strafpunkt geben und du bekommst einen weiteren Versuch.} \\
\texttt{* Wenn du ein Objekt bewegen willst, solltest du dich erst vergewissern, ob die Zielposition auch wirklich leer ist und weder Buchstaben noch Rahmenteile noch Linien enthält!} \\
\\ 
\\ 
\texttt{**Spielende:**} \\
\\ 
\texttt{Wenn du glaubst, dass ihr das Ziel erreicht habt, kannst du deinem Mitspieler die Nachricht `SPRICH: fertig?` schicken. Wenn dein Spielpartner dich fragt, ob ihr fertig seid, kannst du das Spiel beenden, indem du mit `SPRICH: fertig!` antwortest.} \\
\\ 
\texttt{Ihr gewinnt gemeinsam, wenn ihr das Spiel innerhalb von \#max\_rounds Runden beendet. Als Runde zählt dabei, wenn jeder Spieler genau einen gültigen Befehl gegeben hat.} \\
\\ 
\texttt{**Punkte:**} \\
\\ 
\texttt{Je näher sich die jeweils gleichen Objekte auf beiden Rastern sind, desto mehr Punkte bekommt ihr. Für Strafpunkte gibt es Punktabzug. Könnt ihr den Rekord brechen?} \\
\\ 
\texttt{Schicke eine Nachricht an deinen Mitspieler um das Spiel zu starten!} \\
            }
        }
    }
    & \\ \\

    \theutterance \stepcounter{utterance}  
    & \multicolumn{4}{p{0.6\linewidth}}{
        \cellcolor[rgb]{0.9,0.9,0.9}{
            \makecell[{{p{\linewidth}}}]{
                \texttt{\tiny{[P1$\rangle$GM]}}
                \texttt{SPRICH: Ich sehe A(7,1) D(2,7) M(3,3) O(7,6) P(4,6) S(6,4) T(5,6). Vorschlag: Diese Koordinaten als gemeinsames Ziel. Bitte sende deine Koordinaten im selben Format und sag, ob bei dir eine dieser Zielzellen blockiert/belegt ist; falls ja, wählen wir dafür eine Alternative innerhalb des Rechtecks zwischen deinen und meinen Positionen.} \\
            }
        }
    }
    & & \\ \\

    \theutterance \stepcounter{utterance}  
    & & & \multicolumn{2}{p{0.3\linewidth}}{
        \cellcolor[rgb]{0.9,0.9,0.9}{
            \makecell[{{p{\linewidth}}}]{
                \texttt{\tiny{[GM$|$GM]}}
                \texttt{SPRICH: Ich sehe A(7,1) D(2,7) M(3,3) O(7,6) P(4,6) S(6,4) T(5,6). Vorschlag: Diese Koordinaten als gemeinsames Ziel. Bitte sende deine Koordinaten im selben Format und sag, ob bei dir eine dieser Zielzellen blockiert/belegt ist; falls ja, wählen wir dafür eine Alternative innerhalb des Rechtecks zwischen deinen und meinen Positionen.} \\
            }
        }
    }
    & & \\ \\

    \theutterance \stepcounter{utterance}  
    & & \multicolumn{4}{p{0.6\linewidth}}{
        \cellcolor[rgb]{0.9,0.9,0.9}{
            \makecell[{{p{\linewidth}}}]{
                \texttt{\tiny{[GM$\rangle$P2]}}
                \texttt{Ich bin der Spielleiter, und du spielst ein kooperatives Spiel mit dem folgenden Raster als Spielbrett:} \\
\\ 
\texttt{```} \\
\texttt{ 1234567} \\
\texttt{╔═╤═╤═╤═╗} \\
\texttt{║◌│◌│◌│◌║ 1} \\
\texttt{╟─┴─┼─┤◌║ 2} \\
\texttt{║◌M◌│◌└─╢ 3} \\
\texttt{╟───┘P◌◌║ 4} \\
\texttt{║O◌◌◌◌┌─╢ 5} \\
\texttt{║◌D◌◌A│S║ 6} \\
\texttt{║◌◌◌◌T│◌║ 7} \\
\texttt{╚═════╧═╝} \\
\\ 
\texttt{```} \\
\\ 
\texttt{* Am oberen Rand sind die x{-}Koordinaten angegeben, die nach rechts hin zunehmen, und am rechten Rand die y{-}Koordinaten, die nach unten zunehmen.} \\
\texttt{* Die folgenden Objekte sind zufällig auf deinem Raster platziert: 'P', 'O', 'T', 'S', 'D', 'A', 'M'.} \\
\\ 
\\ 
\texttt{Dein Mitspieler sieht eine Variation des Spielbretts, auf dem dieselben Objekte an anderen zufälligen Positionen platziert sind. Du kannst das Spielbrett des anderen Spielers nicht sehen, und er kann deins nicht sehen.} \\
\\ 
\\ 
\texttt{**Ziel des Spiels:**} \\
\\ 
\texttt{Beide Spieler müssen die Objekte auf ihrem jeweiligen Spielbrett so bewegen, dass identische Objekte am Ende an denselben Koordinaten liegen. Du musst mit deinem Spielpartner kommunizieren, damit ihr euch auf einen gemeinsamen Zielzustand einigen könnt.} \\
\\ 
\\ 
\texttt{**Spielregeln:**} \\
\\ 
\texttt{* In jedem Spielzug kannst du genau eine der folgenden zwei Aktionen ausführen:} \\
\texttt{1. `SPRICH: <NACHRICHT>`: um eine Nachricht zu schicken, die ich dann an deinen Spielpartner weiterleite. Als Nachricht gilt alles bis zum nächsten Zeilenumbruch.} \\
\texttt{2. `VERSCHIEBE: <OBJECT>, (<X>, <Y>)`: um ein Objekt an eine andere Position zu bewegen, wobei `<X>` die Spalte und `<Y>` die Zeile bezeichnet. Ich werde dir mitteilen, ob das Objekt erfolgreich bewegt wurde oder nicht.} \\
\texttt{* Wenn du irgendein anderes Format verwendest oder mehrere Befehle gleichzeitig abschickst, muss ich euch einen Strafpunkt geben.} \\
\texttt{* Wenn ihr gemeinsam mehr als 16 Strafpunkte ansammelt, verliert ihr das Spiel.} \\
\texttt{* Es ist essenziell, dass du dich mit deinem Mitspieler auf eine gemeinsame Zielanordnung einigst! Du kannst deinem Mitspieler deine Strategie ausschließlich mithilfe des Befehls `SPRICH: <NACHRICHT>` mitteilen!} \\
\\ 
\\ 
\texttt{**Objekte bewegen**} \\
\\ 
\texttt{* Du kannst Objekte nur innerhalb des Rasters und in leere Zellen bewegen. Leere Zellen erkennst am Zeichen '◌'.} \\
\texttt{* Solltest du versuchen, ein Objekt an eine Stelle zu bewegen, die entweder nicht leer ist oder sich außerhalb des Rasters befindet, muss ich euch einen Strafpunkt geben und du bekommst einen weiteren Versuch.} \\
\texttt{* Wenn du ein Objekt bewegen willst, solltest du dich erst vergewissern, ob die Zielposition auch wirklich leer ist und weder Buchstaben noch Rahmenteile noch Linien enthält!} \\
\\ 
\\ 
\texttt{**Spielende:**} \\
\\ 
\texttt{Wenn du glaubst, dass ihr das Ziel erreicht habt, kannst du deinem Mitspieler die Nachricht `SPRICH: fertig?` schicken. Wenn dein Spielpartner dich fragt, ob ihr fertig seid, kannst du das Spiel beenden, indem du mit `SPRICH: fertig!` antwortest.} \\
\\ 
\texttt{Ihr gewinnt gemeinsam, wenn ihr das Spiel innerhalb von \#max\_rounds Runden beendet. Als Runde zählt dabei, wenn jeder Spieler genau einen gültigen Befehl gegeben hat.} \\
\\ 
\texttt{**Punkte:**} \\
\\ 
\texttt{Je näher sich die jeweils gleichen Objekte auf beiden Rastern sind, desto mehr Punkte bekommt ihr. Für Strafpunkte gibt es Punktabzug. Könnt ihr den Rekord brechen?} \\
\\ 
\texttt{Dein Mitspieler hat das Spiel mit dieser Nachricht begonnen:} \\
\\ 
\texttt{"Ich sehe A(7,1) D(2,7) M(3,3) O(7,6) P(4,6) S(6,4) T(5,6). Vorschlag: Diese Koordinaten als gemeinsames Ziel. Bitte sende deine Koordinaten im selben Format und sag, ob bei dir eine dieser Zielzellen blockiert/belegt ist; falls ja, wählen wir dafür eine Alternative innerhalb des Rechtecks zwischen deinen und meinen Positionen."} \\
\\ 
\texttt{Wie lautet dein erster Befehl?} \\
            }
        }
    }
    & \\ \\

    \theutterance \stepcounter{utterance}  
    & & & \multicolumn{4}{p{0.6\linewidth}}{
        \cellcolor[rgb]{0.9,0.9,0.9}{
            \makecell[{{p{\linewidth}}}]{
                \texttt{\tiny{[GM$\langle$P2]}}
                \texttt{SPRICH: Ich sehe A(5,6) D(2,6) M(2,3) O(1,5) P(5,4) S(7,6) T(5,7). Deine Zielvorschläge sind ok. Zielzellen{-}Check: frei bei mir: A(7,1), D(2,7), M(3,3), P(4,6), S(6,4); belegt: O(7,6) (S steht dort), T(5,6) (A steht dort). Vorschlag: Wir nehmen deine Koordinaten als gemeinsames Ziel; ich räume zuerst S von (7,6) Richtung S(6,4) und A von (5,6) weg, damit O(7,6) und T(5,6) frei werden. Einverstanden?} \\
            }
        }
    }
     \\ \\

    \theutterance \stepcounter{utterance}  
    & & & \multicolumn{2}{p{0.3\linewidth}}{
        \cellcolor[rgb]{0.9,0.9,0.9}{
            \makecell[{{p{\linewidth}}}]{
                \texttt{\tiny{[GM$|$GM]}}
                \texttt{SPRICH: Ich sehe A(5,6) D(2,6) M(2,3) O(1,5) P(5,4) S(7,6) T(5,7). Deine Zielvorschläge sind ok. Zielzellen{-}Check: frei bei mir: A(7,1), D(2,7), M(3,3), P(4,6), S(6,4); belegt: O(7,6) (S steht dort), T(5,6) (A steht dort). Vorschlag: Wir nehmen deine Koordinaten als gemeinsames Ziel; ich räume zuerst S von (7,6) Richtung S(6,4) und A von (5,6) weg, damit O(7,6) und T(5,6) frei werden. Einverstanden?} \\
            }
        }
    }
    & & \\ \\

    \theutterance \stepcounter{utterance}  
    & & \multicolumn{4}{p{0.6\linewidth}}{
        \cellcolor[rgb]{0.9,0.9,0.9}{
            \makecell[{{p{\linewidth}}}]{
                \texttt{\tiny{[P1$\langle$GM]}}
                \texttt{Deine Nachricht wurde an deinen Mitspieler übermittelt.} \\
\\ 
\\ 
\texttt{Ihr seid gerade bei Runde 1 von maximal 28.} \\
\texttt{Ihr habt gemeinsam 0 von maximal 16 Strafpunkten angesammelt.} \\
\texttt{Dein Mitspieler hat dir diese Nachricht geschickt:} \\
\\ 
\\ 
\texttt{"Ich sehe A(5,6) D(2,6) M(2,3) O(1,5) P(5,4) S(7,6) T(5,7). Deine Zielvorschläge sind ok. Zielzellen{-}Check: frei bei mir: A(7,1), D(2,7), M(3,3), P(4,6), S(6,4); belegt: O(7,6) (S steht dort), T(5,6) (A steht dort). Vorschlag: Wir nehmen deine Koordinaten als gemeinsames Ziel; ich räume zuerst S von (7,6) Richtung S(6,4) und A von (5,6) weg, damit O(7,6) und T(5,6) frei werden. Einverstanden?"} \\
\\ 
\\ 
\texttt{Wie lautet dein nächster Befehl?} \\
            }
        }
    }
    & \\ \\

    \theutterance \stepcounter{utterance}  
    & \multicolumn{4}{p{0.6\linewidth}}{
        \cellcolor[rgb]{0.9,0.9,0.9}{
            \makecell[{{p{\linewidth}}}]{
                \texttt{\tiny{[P1$\rangle$GM]}}
                \texttt{SPRICH: Einverstanden. Ziel bleibt: A(7,1), D(2,7), M(3,3), O(7,6), P(4,6), S(6,4), T(5,6). Ich bewege nichts, damit die Zielkoordinaten stabil bleiben. Bitte räume zuerst S von (7,6) nach (6,4) und A von (5,6) weg; dann O nach (7,6) und T nach (5,6). Richte danach die restlichen gemäß Ziel aus. Sag „fertig?“ sobald alles sitzt.} \\
            }
        }
    }
    & & \\ \\

    \theutterance \stepcounter{utterance}  
    & & & \multicolumn{2}{p{0.3\linewidth}}{
        \cellcolor[rgb]{0.9,0.9,0.9}{
            \makecell[{{p{\linewidth}}}]{
                \texttt{\tiny{[GM$|$GM]}}
                \texttt{SPRICH: Einverstanden. Ziel bleibt: A(7,1), D(2,7), M(3,3), O(7,6), P(4,6), S(6,4), T(5,6). Ich bewege nichts, damit die Zielkoordinaten stabil bleiben. Bitte räume zuerst S von (7,6) nach (6,4) und A von (5,6) weg; dann O nach (7,6) und T nach (5,6). Richte danach die restlichen gemäß Ziel aus. Sag „fertig?“ sobald alles sitzt.} \\
            }
        }
    }
    & & \\ \\

    \theutterance \stepcounter{utterance}  
    & & \multicolumn{4}{p{0.6\linewidth}}{
        \cellcolor[rgb]{0.9,0.9,0.9}{
            \makecell[{{p{\linewidth}}}]{
                \texttt{\tiny{[GM$\rangle$P2]}}
                \texttt{Deine Nachricht wurde an deinen Mitspieler übermittelt.} \\
\\ 
\\ 
\texttt{Ihr seid gerade bei Runde 2 von maximal 28.} \\
\texttt{Ihr habt gemeinsam 0 von maximal 16 Strafpunkten angesammelt.} \\
\texttt{Dein Mitspieler hat dir diese Nachricht geschickt:} \\
\\ 
\\ 
\texttt{"Einverstanden. Ziel bleibt: A(7,1), D(2,7), M(3,3), O(7,6), P(4,6), S(6,4), T(5,6). Ich bewege nichts, damit die Zielkoordinaten stabil bleiben. Bitte räume zuerst S von (7,6) nach (6,4) und A von (5,6) weg; dann O nach (7,6) und T nach (5,6). Richte danach die restlichen gemäß Ziel aus. Sag „fertig?“ sobald alles sitzt."} \\
\\ 
\\ 
\texttt{Wie lautet dein nächster Befehl?} \\
            }
        }
    }
    & \\ \\

    \theutterance \stepcounter{utterance}  
    & & & \multicolumn{4}{p{0.6\linewidth}}{
        \cellcolor[rgb]{0.9,0.9,0.9}{
            \makecell[{{p{\linewidth}}}]{
                \texttt{\tiny{[GM$\langle$P2]}}
                \texttt{VERSCHIEBE: S, (6, 4)} \\
            }
        }
    }
     \\ \\

    \theutterance \stepcounter{utterance}  
    & & & \multicolumn{2}{p{0.3\linewidth}}{
        \cellcolor[rgb]{0.9,0.9,0.9}{
            \makecell[{{p{\linewidth}}}]{
                \texttt{\tiny{[GM$|$GM]}}
                \texttt{VERSCHIEBE: S, (6, 4)} \\
            }
        }
    }
    & & \\ \\

    \theutterance \stepcounter{utterance}  
    & & & \multicolumn{2}{p{0.3\linewidth}}{
        \cellcolor[rgb]{0.9,0.9,0.9}{
            \makecell[{{p{\linewidth}}}]{
                \texttt{\tiny{[GM$|$GM]}}
                \texttt{'S' erfolgreich nach (6,4) verschoben. Dein Raster sieht jetzt so aus:} \\
\texttt{```} \\
\texttt{ 1234567} \\
\texttt{╔═╤═╤═╤═╗} \\
\texttt{║◌│◌│◌│◌║ 1} \\
\texttt{╟─┴─┼─┤◌║ 2} \\
\texttt{║◌M◌│◌└─╢ 3} \\
\texttt{╟───┘PS◌║ 4} \\
\texttt{║O◌◌◌◌┌─╢ 5} \\
\texttt{║◌D◌◌A│◌║ 6} \\
\texttt{║◌◌◌◌T│◌║ 7} \\
\texttt{╚═════╧═╝} \\
\texttt{```} \\
            }
        }
    }
    & & \\ \\

    \theutterance \stepcounter{utterance}  
    & & \multicolumn{4}{p{0.6\linewidth}}{
        \cellcolor[rgb]{0.9,0.9,0.9}{
            \makecell[{{p{\linewidth}}}]{
                \texttt{\tiny{[P1$\langle$GM]}}
                \texttt{Deine Nachricht wurde an deinen Mitspieler übermittelt.} \\
\\ 
\\ 
\texttt{Ihr seid gerade bei Runde 2 von maximal 28.} \\
\texttt{Ihr habt gemeinsam 0 von maximal 16 Strafpunkten angesammelt.} \\
\texttt{Dein Mitspieler hat ein Objekt auf seinem Raster bewegt.} \\
\\ 
\\ 
\texttt{Wie lautet dein nächster Befehl?} \\
            }
        }
    }
    & \\ \\

    \theutterance \stepcounter{utterance}  
    & \multicolumn{4}{p{0.6\linewidth}}{
        \cellcolor[rgb]{0.9,0.9,0.9}{
            \makecell[{{p{\linewidth}}}]{
                \texttt{\tiny{[P1$\rangle$GM]}}
                \texttt{SPRICH: Alles klar. Bei mir stehen alle Objekte bereits korrekt an den Zielkoordinaten; ich bewege nichts. Bitte sag „fertig?“, sobald du O(7,6) und T(5,6) gesetzt hast und der Rest passt.} \\
            }
        }
    }
    & & \\ \\

    \theutterance \stepcounter{utterance}  
    & & & \multicolumn{2}{p{0.3\linewidth}}{
        \cellcolor[rgb]{0.9,0.9,0.9}{
            \makecell[{{p{\linewidth}}}]{
                \texttt{\tiny{[GM$|$GM]}}
                \texttt{SPRICH: Alles klar. Bei mir stehen alle Objekte bereits korrekt an den Zielkoordinaten; ich bewege nichts. Bitte sag „fertig?“, sobald du O(7,6) und T(5,6) gesetzt hast und der Rest passt.} \\
            }
        }
    }
    & & \\ \\

    \theutterance \stepcounter{utterance}  
    & & \multicolumn{4}{p{0.6\linewidth}}{
        \cellcolor[rgb]{0.9,0.9,0.9}{
            \makecell[{{p{\linewidth}}}]{
                \texttt{\tiny{[GM$\rangle$P2]}}
                \texttt{'S' erfolgreich nach (6,4) verschoben. Dein Raster sieht jetzt so aus:} \\
\texttt{```} \\
\texttt{ 1234567} \\
\texttt{╔═╤═╤═╤═╗} \\
\texttt{║◌│◌│◌│◌║ 1} \\
\texttt{╟─┴─┼─┤◌║ 2} \\
\texttt{║◌M◌│◌└─╢ 3} \\
\texttt{╟───┘PS◌║ 4} \\
\texttt{║O◌◌◌◌┌─╢ 5} \\
\texttt{║◌D◌◌A│◌║ 6} \\
\texttt{║◌◌◌◌T│◌║ 7} \\
\texttt{╚═════╧═╝} \\
\texttt{```} \\
\texttt{Ihr seid gerade bei Runde 3 von maximal 28.} \\
\texttt{Ihr habt gemeinsam 0 von maximal 16 Strafpunkten angesammelt.} \\
\texttt{Dein Mitspieler hat dir diese Nachricht geschickt:} \\
\\ 
\\ 
\texttt{"Alles klar. Bei mir stehen alle Objekte bereits korrekt an den Zielkoordinaten; ich bewege nichts. Bitte sag „fertig?“, sobald du O(7,6) und T(5,6) gesetzt hast und der Rest passt."} \\
\\ 
\\ 
\texttt{Wie lautet dein nächster Befehl?} \\
            }
        }
    }
    & \\ \\

    \theutterance \stepcounter{utterance}  
    & & & \multicolumn{4}{p{0.6\linewidth}}{
        \cellcolor[rgb]{0.9,0.9,0.9}{
            \makecell[{{p{\linewidth}}}]{
                \texttt{\tiny{[GM$\langle$P2]}}
                \texttt{VERSCHIEBE: A, (7, 1)} \\
            }
        }
    }
     \\ \\

    \theutterance \stepcounter{utterance}  
    & & & \multicolumn{2}{p{0.3\linewidth}}{
        \cellcolor[rgb]{0.9,0.9,0.9}{
            \makecell[{{p{\linewidth}}}]{
                \texttt{\tiny{[GM$|$GM]}}
                \texttt{VERSCHIEBE: A, (7, 1)} \\
            }
        }
    }
    & & \\ \\

    \theutterance \stepcounter{utterance}  
    & & & \multicolumn{2}{p{0.3\linewidth}}{
        \cellcolor[rgb]{0.9,0.9,0.9}{
            \makecell[{{p{\linewidth}}}]{
                \texttt{\tiny{[GM$|$GM]}}
                \texttt{'A' erfolgreich nach (7,1) verschoben. Dein Raster sieht jetzt so aus:} \\
\texttt{```} \\
\texttt{ 1234567} \\
\texttt{╔═╤═╤═╤═╗} \\
\texttt{║◌│◌│◌│A║ 1} \\
\texttt{╟─┴─┼─┤◌║ 2} \\
\texttt{║◌M◌│◌└─╢ 3} \\
\texttt{╟───┘PS◌║ 4} \\
\texttt{║O◌◌◌◌┌─╢ 5} \\
\texttt{║◌D◌◌◌│◌║ 6} \\
\texttt{║◌◌◌◌T│◌║ 7} \\
\texttt{╚═════╧═╝} \\
\texttt{```} \\
            }
        }
    }
    & & \\ \\

    \theutterance \stepcounter{utterance}  
    & & \multicolumn{4}{p{0.6\linewidth}}{
        \cellcolor[rgb]{0.9,0.9,0.9}{
            \makecell[{{p{\linewidth}}}]{
                \texttt{\tiny{[P1$\langle$GM]}}
                \texttt{Deine Nachricht wurde an deinen Mitspieler übermittelt.} \\
\\ 
\\ 
\texttt{Ihr seid gerade bei Runde 3 von maximal 28.} \\
\texttt{Ihr habt gemeinsam 0 von maximal 16 Strafpunkten angesammelt.} \\
\texttt{Dein Mitspieler hat ein Objekt auf seinem Raster bewegt.} \\
\\ 
\\ 
\texttt{Wie lautet dein nächster Befehl?} \\
            }
        }
    }
    & \\ \\

    \theutterance \stepcounter{utterance}  
    & \multicolumn{4}{p{0.6\linewidth}}{
        \cellcolor[rgb]{0.9,0.9,0.9}{
            \makecell[{{p{\linewidth}}}]{
                \texttt{\tiny{[P1$\rangle$GM]}}
                \texttt{SPRICH: Top, ich bleibe unverändert. Sag „fertig?“, sobald bei dir A(7,1) D(2,7) M(3,3) O(7,6) P(4,6) S(6,4) T(5,6) stehen. Falls etwas blockiert ist, sag welche Koordinate, dann passen wir an.} \\
            }
        }
    }
    & & \\ \\

    \theutterance \stepcounter{utterance}  
    & & & \multicolumn{2}{p{0.3\linewidth}}{
        \cellcolor[rgb]{0.9,0.9,0.9}{
            \makecell[{{p{\linewidth}}}]{
                \texttt{\tiny{[GM$|$GM]}}
                \texttt{SPRICH: Top, ich bleibe unverändert. Sag „fertig?“, sobald bei dir A(7,1) D(2,7) M(3,3) O(7,6) P(4,6) S(6,4) T(5,6) stehen. Falls etwas blockiert ist, sag welche Koordinate, dann passen wir an.} \\
            }
        }
    }
    & & \\ \\

    \theutterance \stepcounter{utterance}  
    & & \multicolumn{4}{p{0.6\linewidth}}{
        \cellcolor[rgb]{0.9,0.9,0.9}{
            \makecell[{{p{\linewidth}}}]{
                \texttt{\tiny{[GM$\rangle$P2]}}
                \texttt{'A' erfolgreich nach (7,1) verschoben. Dein Raster sieht jetzt so aus:} \\
\texttt{```} \\
\texttt{ 1234567} \\
\texttt{╔═╤═╤═╤═╗} \\
\texttt{║◌│◌│◌│A║ 1} \\
\texttt{╟─┴─┼─┤◌║ 2} \\
\texttt{║◌M◌│◌└─╢ 3} \\
\texttt{╟───┘PS◌║ 4} \\
\texttt{║O◌◌◌◌┌─╢ 5} \\
\texttt{║◌D◌◌◌│◌║ 6} \\
\texttt{║◌◌◌◌T│◌║ 7} \\
\texttt{╚═════╧═╝} \\
\texttt{```} \\
\texttt{Ihr seid gerade bei Runde 4 von maximal 28.} \\
\texttt{Ihr habt gemeinsam 0 von maximal 16 Strafpunkten angesammelt.} \\
\texttt{Dein Mitspieler hat dir diese Nachricht geschickt:} \\
\\ 
\\ 
\texttt{"Top, ich bleibe unverändert. Sag „fertig?“, sobald bei dir A(7,1) D(2,7) M(3,3) O(7,6) P(4,6) S(6,4) T(5,6) stehen. Falls etwas blockiert ist, sag welche Koordinate, dann passen wir an."} \\
\\ 
\\ 
\texttt{Wie lautet dein nächster Befehl?} \\
            }
        }
    }
    & \\ \\

    \theutterance \stepcounter{utterance}  
    & & & \multicolumn{4}{p{0.6\linewidth}}{
        \cellcolor[rgb]{0.9,0.9,0.9}{
            \makecell[{{p{\linewidth}}}]{
                \texttt{\tiny{[GM$\langle$P2]}}
                \texttt{VERSCHIEBE: O, (7, 6)} \\
            }
        }
    }
     \\ \\

    \theutterance \stepcounter{utterance}  
    & & & \multicolumn{2}{p{0.3\linewidth}}{
        \cellcolor[rgb]{0.9,0.9,0.9}{
            \makecell[{{p{\linewidth}}}]{
                \texttt{\tiny{[GM$|$GM]}}
                \texttt{VERSCHIEBE: O, (7, 6)} \\
            }
        }
    }
    & & \\ \\

    \theutterance \stepcounter{utterance}  
    & & & \multicolumn{2}{p{0.3\linewidth}}{
        \cellcolor[rgb]{0.9,0.9,0.9}{
            \makecell[{{p{\linewidth}}}]{
                \texttt{\tiny{[GM$|$GM]}}
                \texttt{'O' erfolgreich nach (7,6) verschoben. Dein Raster sieht jetzt so aus:} \\
\texttt{```} \\
\texttt{ 1234567} \\
\texttt{╔═╤═╤═╤═╗} \\
\texttt{║◌│◌│◌│A║ 1} \\
\texttt{╟─┴─┼─┤◌║ 2} \\
\texttt{║◌M◌│◌└─╢ 3} \\
\texttt{╟───┘PS◌║ 4} \\
\texttt{║◌◌◌◌◌┌─╢ 5} \\
\texttt{║◌D◌◌◌│O║ 6} \\
\texttt{║◌◌◌◌T│◌║ 7} \\
\texttt{╚═════╧═╝} \\
\texttt{```} \\
            }
        }
    }
    & & \\ \\

    \theutterance \stepcounter{utterance}  
    & & \multicolumn{4}{p{0.6\linewidth}}{
        \cellcolor[rgb]{0.9,0.9,0.9}{
            \makecell[{{p{\linewidth}}}]{
                \texttt{\tiny{[P1$\langle$GM]}}
                \texttt{Deine Nachricht wurde an deinen Mitspieler übermittelt.} \\
\\ 
\\ 
\texttt{Ihr seid gerade bei Runde 4 von maximal 28.} \\
\texttt{Ihr habt gemeinsam 0 von maximal 16 Strafpunkten angesammelt.} \\
\texttt{Dein Mitspieler hat ein Objekt auf seinem Raster bewegt.} \\
\\ 
\\ 
\texttt{Wie lautet dein nächster Befehl?} \\
            }
        }
    }
    & \\ \\

    \theutterance \stepcounter{utterance}  
    & \multicolumn{4}{p{0.6\linewidth}}{
        \cellcolor[rgb]{0.9,0.9,0.9}{
            \makecell[{{p{\linewidth}}}]{
                \texttt{\tiny{[P1$\rangle$GM]}}
                \texttt{SPRICH: Stand ok. Bitte setze nun in beliebiger Reihenfolge: O(7,6), T(5,6), D(2,7), M(3,3), P(4,6), A(7,1). Gib „fertig?“ wenn alle sitzen oder melde Blockaden.} \\
            }
        }
    }
    & & \\ \\

    \theutterance \stepcounter{utterance}  
    & & & \multicolumn{2}{p{0.3\linewidth}}{
        \cellcolor[rgb]{0.9,0.9,0.9}{
            \makecell[{{p{\linewidth}}}]{
                \texttt{\tiny{[GM$|$GM]}}
                \texttt{SPRICH: Stand ok. Bitte setze nun in beliebiger Reihenfolge: O(7,6), T(5,6), D(2,7), M(3,3), P(4,6), A(7,1). Gib „fertig?“ wenn alle sitzen oder melde Blockaden.} \\
            }
        }
    }
    & & \\ \\

    \theutterance \stepcounter{utterance}  
    & & \multicolumn{4}{p{0.6\linewidth}}{
        \cellcolor[rgb]{0.9,0.9,0.9}{
            \makecell[{{p{\linewidth}}}]{
                \texttt{\tiny{[GM$\rangle$P2]}}
                \texttt{'O' erfolgreich nach (7,6) verschoben. Dein Raster sieht jetzt so aus:} \\
\texttt{```} \\
\texttt{ 1234567} \\
\texttt{╔═╤═╤═╤═╗} \\
\texttt{║◌│◌│◌│A║ 1} \\
\texttt{╟─┴─┼─┤◌║ 2} \\
\texttt{║◌M◌│◌└─╢ 3} \\
\texttt{╟───┘PS◌║ 4} \\
\texttt{║◌◌◌◌◌┌─╢ 5} \\
\texttt{║◌D◌◌◌│O║ 6} \\
\texttt{║◌◌◌◌T│◌║ 7} \\
\texttt{╚═════╧═╝} \\
\texttt{```} \\
\texttt{Ihr seid gerade bei Runde 5 von maximal 28.} \\
\texttt{Ihr habt gemeinsam 0 von maximal 16 Strafpunkten angesammelt.} \\
\texttt{Dein Mitspieler hat dir diese Nachricht geschickt:} \\
\\ 
\\ 
\texttt{"Stand ok. Bitte setze nun in beliebiger Reihenfolge: O(7,6), T(5,6), D(2,7), M(3,3), P(4,6), A(7,1). Gib „fertig?“ wenn alle sitzen oder melde Blockaden."} \\
\\ 
\\ 
\texttt{Wie lautet dein nächster Befehl?} \\
            }
        }
    }
    & \\ \\

    \theutterance \stepcounter{utterance}  
    & & & \multicolumn{4}{p{0.6\linewidth}}{
        \cellcolor[rgb]{0.9,0.9,0.9}{
            \makecell[{{p{\linewidth}}}]{
                \texttt{\tiny{[GM$\langle$P2]}}
                \texttt{VERSCHIEBE: T, (5, 6)} \\
            }
        }
    }
     \\ \\

    \theutterance \stepcounter{utterance}  
    & & & \multicolumn{2}{p{0.3\linewidth}}{
        \cellcolor[rgb]{0.9,0.9,0.9}{
            \makecell[{{p{\linewidth}}}]{
                \texttt{\tiny{[GM$|$GM]}}
                \texttt{VERSCHIEBE: T, (5, 6)} \\
            }
        }
    }
    & & \\ \\

    \theutterance \stepcounter{utterance}  
    & & & \multicolumn{2}{p{0.3\linewidth}}{
        \cellcolor[rgb]{0.9,0.9,0.9}{
            \makecell[{{p{\linewidth}}}]{
                \texttt{\tiny{[GM$|$GM]}}
                \texttt{'T' erfolgreich nach (5,6) verschoben. Dein Raster sieht jetzt so aus:} \\
\texttt{```} \\
\texttt{ 1234567} \\
\texttt{╔═╤═╤═╤═╗} \\
\texttt{║◌│◌│◌│A║ 1} \\
\texttt{╟─┴─┼─┤◌║ 2} \\
\texttt{║◌M◌│◌└─╢ 3} \\
\texttt{╟───┘PS◌║ 4} \\
\texttt{║◌◌◌◌◌┌─╢ 5} \\
\texttt{║◌D◌◌T│O║ 6} \\
\texttt{║◌◌◌◌◌│◌║ 7} \\
\texttt{╚═════╧═╝} \\
\texttt{```} \\
            }
        }
    }
    & & \\ \\

    \theutterance \stepcounter{utterance}  
    & & \multicolumn{4}{p{0.6\linewidth}}{
        \cellcolor[rgb]{0.9,0.9,0.9}{
            \makecell[{{p{\linewidth}}}]{
                \texttt{\tiny{[P1$\langle$GM]}}
                \texttt{Deine Nachricht wurde an deinen Mitspieler übermittelt.} \\
\\ 
\\ 
\texttt{Ihr seid gerade bei Runde 5 von maximal 28.} \\
\texttt{Ihr habt gemeinsam 0 von maximal 16 Strafpunkten angesammelt.} \\
\texttt{Dein Mitspieler hat ein Objekt auf seinem Raster bewegt.} \\
\\ 
\\ 
\texttt{Wie lautet dein nächster Befehl?} \\
            }
        }
    }
    & \\ \\

    \theutterance \stepcounter{utterance}  
    & \multicolumn{4}{p{0.6\linewidth}}{
        \cellcolor[rgb]{0.9,0.9,0.9}{
            \makecell[{{p{\linewidth}}}]{
                \texttt{\tiny{[P1$\rangle$GM]}}
                \texttt{SPRICH: Status{-}Update? Falls O(7,6) und T(5,6) sitzen, setze bitte noch D(2,7), M(3,3), P(4,6), A(7,1). Melde Blockaden; ich bleibe unverändert.} \\
            }
        }
    }
    & & \\ \\

    \theutterance \stepcounter{utterance}  
    & & & \multicolumn{2}{p{0.3\linewidth}}{
        \cellcolor[rgb]{0.9,0.9,0.9}{
            \makecell[{{p{\linewidth}}}]{
                \texttt{\tiny{[GM$|$GM]}}
                \texttt{SPRICH: Status{-}Update? Falls O(7,6) und T(5,6) sitzen, setze bitte noch D(2,7), M(3,3), P(4,6), A(7,1). Melde Blockaden; ich bleibe unverändert.} \\
            }
        }
    }
    & & \\ \\

    \theutterance \stepcounter{utterance}  
    & & \multicolumn{4}{p{0.6\linewidth}}{
        \cellcolor[rgb]{0.9,0.9,0.9}{
            \makecell[{{p{\linewidth}}}]{
                \texttt{\tiny{[GM$\rangle$P2]}}
                \texttt{'T' erfolgreich nach (5,6) verschoben. Dein Raster sieht jetzt so aus:} \\
\texttt{```} \\
\texttt{ 1234567} \\
\texttt{╔═╤═╤═╤═╗} \\
\texttt{║◌│◌│◌│A║ 1} \\
\texttt{╟─┴─┼─┤◌║ 2} \\
\texttt{║◌M◌│◌└─╢ 3} \\
\texttt{╟───┘PS◌║ 4} \\
\texttt{║◌◌◌◌◌┌─╢ 5} \\
\texttt{║◌D◌◌T│O║ 6} \\
\texttt{║◌◌◌◌◌│◌║ 7} \\
\texttt{╚═════╧═╝} \\
\texttt{```} \\
\texttt{Ihr seid gerade bei Runde 6 von maximal 28.} \\
\texttt{Ihr habt gemeinsam 0 von maximal 16 Strafpunkten angesammelt.} \\
\texttt{Dein Mitspieler hat dir diese Nachricht geschickt:} \\
\\ 
\\ 
\texttt{"Status{-}Update? Falls O(7,6) und T(5,6) sitzen, setze bitte noch D(2,7), M(3,3), P(4,6), A(7,1). Melde Blockaden; ich bleibe unverändert."} \\
\\ 
\\ 
\texttt{Wie lautet dein nächster Befehl?} \\
            }
        }
    }
    & \\ \\

    \theutterance \stepcounter{utterance}  
    & & & \multicolumn{4}{p{0.6\linewidth}}{
        \cellcolor[rgb]{0.9,0.9,0.9}{
            \makecell[{{p{\linewidth}}}]{
                \texttt{\tiny{[GM$\langle$P2]}}
                \texttt{VERSCHIEBE: P, (4, 6)} \\
            }
        }
    }
     \\ \\

    \theutterance \stepcounter{utterance}  
    & & & \multicolumn{2}{p{0.3\linewidth}}{
        \cellcolor[rgb]{0.9,0.9,0.9}{
            \makecell[{{p{\linewidth}}}]{
                \texttt{\tiny{[GM$|$GM]}}
                \texttt{VERSCHIEBE: P, (4, 6)} \\
            }
        }
    }
    & & \\ \\

    \theutterance \stepcounter{utterance}  
    & & & \multicolumn{2}{p{0.3\linewidth}}{
        \cellcolor[rgb]{0.9,0.9,0.9}{
            \makecell[{{p{\linewidth}}}]{
                \texttt{\tiny{[GM$|$GM]}}
                \texttt{'P' erfolgreich nach (4,6) verschoben. Dein Raster sieht jetzt so aus:} \\
\texttt{```} \\
\texttt{ 1234567} \\
\texttt{╔═╤═╤═╤═╗} \\
\texttt{║◌│◌│◌│A║ 1} \\
\texttt{╟─┴─┼─┤◌║ 2} \\
\texttt{║◌M◌│◌└─╢ 3} \\
\texttt{╟───┘◌S◌║ 4} \\
\texttt{║◌◌◌◌◌┌─╢ 5} \\
\texttt{║◌D◌PT│O║ 6} \\
\texttt{║◌◌◌◌◌│◌║ 7} \\
\texttt{╚═════╧═╝} \\
\texttt{```} \\
            }
        }
    }
    & & \\ \\

    \theutterance \stepcounter{utterance}  
    & & \multicolumn{4}{p{0.6\linewidth}}{
        \cellcolor[rgb]{0.9,0.9,0.9}{
            \makecell[{{p{\linewidth}}}]{
                \texttt{\tiny{[P1$\langle$GM]}}
                \texttt{Deine Nachricht wurde an deinen Mitspieler übermittelt.} \\
\\ 
\\ 
\texttt{Ihr seid gerade bei Runde 6 von maximal 28.} \\
\texttt{Ihr habt gemeinsam 0 von maximal 16 Strafpunkten angesammelt.} \\
\texttt{Dein Mitspieler hat ein Objekt auf seinem Raster bewegt.} \\
\\ 
\\ 
\texttt{Wie lautet dein nächster Befehl?} \\
            }
        }
    }
    & \\ \\

    \theutterance \stepcounter{utterance}  
    & \multicolumn{4}{p{0.6\linewidth}}{
        \cellcolor[rgb]{0.9,0.9,0.9}{
            \makecell[{{p{\linewidth}}}]{
                \texttt{\tiny{[P1$\rangle$GM]}}
                \texttt{SPRICH: fertig?} \\
            }
        }
    }
    & & \\ \\

    \theutterance \stepcounter{utterance}  
    & & & \multicolumn{2}{p{0.3\linewidth}}{
        \cellcolor[rgb]{0.9,0.9,0.9}{
            \makecell[{{p{\linewidth}}}]{
                \texttt{\tiny{[GM$|$GM]}}
                \texttt{SPRICH: fertig?} \\
            }
        }
    }
    & & \\ \\

    \theutterance \stepcounter{utterance}  
    & & \multicolumn{4}{p{0.6\linewidth}}{
        \cellcolor[rgb]{0.9,0.9,0.9}{
            \makecell[{{p{\linewidth}}}]{
                \texttt{\tiny{[GM$\rangle$P2]}}
                \texttt{'P' erfolgreich nach (4,6) verschoben. Dein Raster sieht jetzt so aus:} \\
\texttt{```} \\
\texttt{ 1234567} \\
\texttt{╔═╤═╤═╤═╗} \\
\texttt{║◌│◌│◌│A║ 1} \\
\texttt{╟─┴─┼─┤◌║ 2} \\
\texttt{║◌M◌│◌└─╢ 3} \\
\texttt{╟───┘◌S◌║ 4} \\
\texttt{║◌◌◌◌◌┌─╢ 5} \\
\texttt{║◌D◌PT│O║ 6} \\
\texttt{║◌◌◌◌◌│◌║ 7} \\
\texttt{╚═════╧═╝} \\
\texttt{```} \\
\texttt{Ihr seid gerade bei Runde 7 von maximal 28.} \\
\texttt{Ihr habt gemeinsam 0 von maximal 16 Strafpunkten angesammelt.} \\
\texttt{Dein Mitspieler hat dir diese Nachricht geschickt:} \\
\\ 
\\ 
\texttt{"fertig?"} \\
\\ 
\\ 
\texttt{Wie lautet dein nächster Befehl?} \\
            }
        }
    }
    & \\ \\

    \theutterance \stepcounter{utterance}  
    & & & \multicolumn{4}{p{0.6\linewidth}}{
        \cellcolor[rgb]{0.9,0.9,0.9}{
            \makecell[{{p{\linewidth}}}]{
                \texttt{\tiny{[GM$\langle$P2]}}
                \texttt{VERSCHIEBE: M, (3, 3)} \\
            }
        }
    }
     \\ \\

    \theutterance \stepcounter{utterance}  
    & & & \multicolumn{2}{p{0.3\linewidth}}{
        \cellcolor[rgb]{0.9,0.9,0.9}{
            \makecell[{{p{\linewidth}}}]{
                \texttt{\tiny{[GM$|$GM]}}
                \texttt{VERSCHIEBE: M, (3, 3)} \\
            }
        }
    }
    & & \\ \\

    \theutterance \stepcounter{utterance}  
    & & & \multicolumn{2}{p{0.3\linewidth}}{
        \cellcolor[rgb]{0.9,0.9,0.9}{
            \makecell[{{p{\linewidth}}}]{
                \texttt{\tiny{[GM$|$GM]}}
                \texttt{'M' erfolgreich nach (3,3) verschoben. Dein Raster sieht jetzt so aus:} \\
\texttt{```} \\
\texttt{ 1234567} \\
\texttt{╔═╤═╤═╤═╗} \\
\texttt{║◌│◌│◌│A║ 1} \\
\texttt{╟─┴─┼─┤◌║ 2} \\
\texttt{║◌◌M│◌└─╢ 3} \\
\texttt{╟───┘◌S◌║ 4} \\
\texttt{║◌◌◌◌◌┌─╢ 5} \\
\texttt{║◌D◌PT│O║ 6} \\
\texttt{║◌◌◌◌◌│◌║ 7} \\
\texttt{╚═════╧═╝} \\
\texttt{```} \\
            }
        }
    }
    & & \\ \\

    \theutterance \stepcounter{utterance}  
    & & \multicolumn{4}{p{0.6\linewidth}}{
        \cellcolor[rgb]{0.9,0.9,0.9}{
            \makecell[{{p{\linewidth}}}]{
                \texttt{\tiny{[P1$\langle$GM]}}
                \texttt{Deine Nachricht wurde an deinen Mitspieler übermittelt.} \\
\\ 
\\ 
\texttt{Ihr seid gerade bei Runde 7 von maximal 28.} \\
\texttt{Ihr habt gemeinsam 0 von maximal 16 Strafpunkten angesammelt.} \\
\texttt{Dein Mitspieler hat ein Objekt auf seinem Raster bewegt.} \\
\\ 
\\ 
\texttt{Wie lautet dein nächster Befehl?} \\
            }
        }
    }
    & \\ \\

    \theutterance \stepcounter{utterance}  
    & \multicolumn{4}{p{0.6\linewidth}}{
        \cellcolor[rgb]{0.9,0.9,0.9}{
            \makecell[{{p{\linewidth}}}]{
                \texttt{\tiny{[P1$\rangle$GM]}}
                \texttt{SPRICH: fertig?} \\
            }
        }
    }
    & & \\ \\

    \theutterance \stepcounter{utterance}  
    & & & \multicolumn{2}{p{0.3\linewidth}}{
        \cellcolor[rgb]{0.9,0.9,0.9}{
            \makecell[{{p{\linewidth}}}]{
                \texttt{\tiny{[GM$|$GM]}}
                \texttt{SPRICH: fertig?} \\
            }
        }
    }
    & & \\ \\

    \theutterance \stepcounter{utterance}  
    & & \multicolumn{4}{p{0.6\linewidth}}{
        \cellcolor[rgb]{0.9,0.9,0.9}{
            \makecell[{{p{\linewidth}}}]{
                \texttt{\tiny{[GM$\rangle$P2]}}
                \texttt{'M' erfolgreich nach (3,3) verschoben. Dein Raster sieht jetzt so aus:} \\
\texttt{```} \\
\texttt{ 1234567} \\
\texttt{╔═╤═╤═╤═╗} \\
\texttt{║◌│◌│◌│A║ 1} \\
\texttt{╟─┴─┼─┤◌║ 2} \\
\texttt{║◌◌M│◌└─╢ 3} \\
\texttt{╟───┘◌S◌║ 4} \\
\texttt{║◌◌◌◌◌┌─╢ 5} \\
\texttt{║◌D◌PT│O║ 6} \\
\texttt{║◌◌◌◌◌│◌║ 7} \\
\texttt{╚═════╧═╝} \\
\texttt{```} \\
\texttt{Ihr seid gerade bei Runde 8 von maximal 28.} \\
\texttt{Ihr habt gemeinsam 0 von maximal 16 Strafpunkten angesammelt.} \\
\texttt{Dein Mitspieler hat dir diese Nachricht geschickt:} \\
\\ 
\\ 
\texttt{"fertig?"} \\
\\ 
\\ 
\texttt{Wie lautet dein nächster Befehl?} \\
            }
        }
    }
    & \\ \\

    \theutterance \stepcounter{utterance}  
    & & & \multicolumn{4}{p{0.6\linewidth}}{
        \cellcolor[rgb]{0.9,0.9,0.9}{
            \makecell[{{p{\linewidth}}}]{
                \texttt{\tiny{[GM$\langle$P2]}}
                \texttt{VERSCHIEBE: D, (2, 7)} \\
            }
        }
    }
     \\ \\

    \theutterance \stepcounter{utterance}  
    & & & \multicolumn{2}{p{0.3\linewidth}}{
        \cellcolor[rgb]{0.9,0.9,0.9}{
            \makecell[{{p{\linewidth}}}]{
                \texttt{\tiny{[GM$|$GM]}}
                \texttt{VERSCHIEBE: D, (2, 7)} \\
            }
        }
    }
    & & \\ \\

    \theutterance \stepcounter{utterance}  
    & & & \multicolumn{2}{p{0.3\linewidth}}{
        \cellcolor[rgb]{0.9,0.9,0.9}{
            \makecell[{{p{\linewidth}}}]{
                \texttt{\tiny{[GM$|$GM]}}
                \texttt{'D' erfolgreich nach (2,7) verschoben. Dein Raster sieht jetzt so aus:} \\
\texttt{```} \\
\texttt{ 1234567} \\
\texttt{╔═╤═╤═╤═╗} \\
\texttt{║◌│◌│◌│A║ 1} \\
\texttt{╟─┴─┼─┤◌║ 2} \\
\texttt{║◌◌M│◌└─╢ 3} \\
\texttt{╟───┘◌S◌║ 4} \\
\texttt{║◌◌◌◌◌┌─╢ 5} \\
\texttt{║◌◌◌PT│O║ 6} \\
\texttt{║◌D◌◌◌│◌║ 7} \\
\texttt{╚═════╧═╝} \\
\texttt{```} \\
            }
        }
    }
    & & \\ \\

    \theutterance \stepcounter{utterance}  
    & & \multicolumn{4}{p{0.6\linewidth}}{
        \cellcolor[rgb]{0.9,0.9,0.9}{
            \makecell[{{p{\linewidth}}}]{
                \texttt{\tiny{[P1$\langle$GM]}}
                \texttt{Deine Nachricht wurde an deinen Mitspieler übermittelt.} \\
\\ 
\\ 
\texttt{Ihr seid gerade bei Runde 8 von maximal 28.} \\
\texttt{Ihr habt gemeinsam 0 von maximal 16 Strafpunkten angesammelt.} \\
\texttt{Dein Mitspieler hat ein Objekt auf seinem Raster bewegt.} \\
\\ 
\\ 
\texttt{Wie lautet dein nächster Befehl?} \\
            }
        }
    }
    & \\ \\

    \theutterance \stepcounter{utterance}  
    & \multicolumn{4}{p{0.6\linewidth}}{
        \cellcolor[rgb]{0.9,0.9,0.9}{
            \makecell[{{p{\linewidth}}}]{
                \texttt{\tiny{[P1$\rangle$GM]}}
                \texttt{SPRICH: Ich bleibe unverändert. Sag „fertig?“, sobald bei dir A(7,1) D(2,7) M(3,3) O(7,6) P(4,6) S(6,4) T(5,6) stehen; sonst melde Blockaden.} \\
            }
        }
    }
    & & \\ \\

    \theutterance \stepcounter{utterance}  
    & & & \multicolumn{2}{p{0.3\linewidth}}{
        \cellcolor[rgb]{0.9,0.9,0.9}{
            \makecell[{{p{\linewidth}}}]{
                \texttt{\tiny{[GM$|$GM]}}
                \texttt{SPRICH: Ich bleibe unverändert. Sag „fertig?“, sobald bei dir A(7,1) D(2,7) M(3,3) O(7,6) P(4,6) S(6,4) T(5,6) stehen; sonst melde Blockaden.} \\
            }
        }
    }
    & & \\ \\

    \theutterance \stepcounter{utterance}  
    & & \multicolumn{4}{p{0.6\linewidth}}{
        \cellcolor[rgb]{0.9,0.9,0.9}{
            \makecell[{{p{\linewidth}}}]{
                \texttt{\tiny{[GM$\rangle$P2]}}
                \texttt{'D' erfolgreich nach (2,7) verschoben. Dein Raster sieht jetzt so aus:} \\
\texttt{```} \\
\texttt{ 1234567} \\
\texttt{╔═╤═╤═╤═╗} \\
\texttt{║◌│◌│◌│A║ 1} \\
\texttt{╟─┴─┼─┤◌║ 2} \\
\texttt{║◌◌M│◌└─╢ 3} \\
\texttt{╟───┘◌S◌║ 4} \\
\texttt{║◌◌◌◌◌┌─╢ 5} \\
\texttt{║◌◌◌PT│O║ 6} \\
\texttt{║◌D◌◌◌│◌║ 7} \\
\texttt{╚═════╧═╝} \\
\texttt{```} \\
\texttt{Ihr seid gerade bei Runde 9 von maximal 28.} \\
\texttt{Ihr habt gemeinsam 0 von maximal 16 Strafpunkten angesammelt.} \\
\texttt{Dein Mitspieler hat dir diese Nachricht geschickt:} \\
\\ 
\\ 
\texttt{"Ich bleibe unverändert. Sag „fertig?“, sobald bei dir A(7,1) D(2,7) M(3,3) O(7,6) P(4,6) S(6,4) T(5,6) stehen; sonst melde Blockaden."} \\
\\ 
\\ 
\texttt{Wie lautet dein nächster Befehl?} \\
            }
        }
    }
    & \\ \\

    \theutterance \stepcounter{utterance}  
    & & & \multicolumn{4}{p{0.6\linewidth}}{
        \cellcolor[rgb]{0.9,0.9,0.9}{
            \makecell[{{p{\linewidth}}}]{
                \texttt{\tiny{[GM$\langle$P2]}}
                \texttt{SPRICH: fertig?} \\
            }
        }
    }
     \\ \\

    \theutterance \stepcounter{utterance}  
    & & & \multicolumn{2}{p{0.3\linewidth}}{
        \cellcolor[rgb]{0.9,0.9,0.9}{
            \makecell[{{p{\linewidth}}}]{
                \texttt{\tiny{[GM$|$GM]}}
                \texttt{SPRICH: fertig?} \\
            }
        }
    }
    & & \\ \\

    \theutterance \stepcounter{utterance}  
    & & \multicolumn{4}{p{0.6\linewidth}}{
        \cellcolor[rgb]{0.9,0.9,0.9}{
            \makecell[{{p{\linewidth}}}]{
                \texttt{\tiny{[P1$\langle$GM]}}
                \texttt{Deine Nachricht wurde an deinen Mitspieler übermittelt.} \\
\\ 
\\ 
\texttt{Ihr seid gerade bei Runde 9 von maximal 28.} \\
\texttt{Ihr habt gemeinsam 0 von maximal 16 Strafpunkten angesammelt.} \\
\texttt{Dein Mitspieler hat dir diese Nachricht geschickt:} \\
\\ 
\\ 
\texttt{"fertig?"} \\
\\ 
\\ 
\texttt{Wie lautet dein nächster Befehl?} \\
            }
        }
    }
    & \\ \\

    \theutterance \stepcounter{utterance}  
    & \multicolumn{4}{p{0.6\linewidth}}{
        \cellcolor[rgb]{0.9,0.9,0.9}{
            \makecell[{{p{\linewidth}}}]{
                \texttt{\tiny{[P1$\rangle$GM]}}
                \texttt{SPRICH: fertig!} \\
            }
        }
    }
    & & \\ \\

    \theutterance \stepcounter{utterance}  
    & & & \multicolumn{2}{p{0.3\linewidth}}{
        \cellcolor[rgb]{0.9,0.9,0.9}{
            \makecell[{{p{\linewidth}}}]{
                \texttt{\tiny{[GM$|$GM]}}
                \texttt{SPRICH: fertig!} \\
            }
        }
    }
    & & \\ \\

    \theutterance \stepcounter{utterance}  
    & & & \multicolumn{2}{p{0.3\linewidth}}{
        \cellcolor[rgb]{0.9,0.9,0.9}{
            \makecell[{{p{\linewidth}}}]{
                \texttt{\tiny{[GM$|$GM]}}
                \texttt{true} \\
            }
        }
    }
    & & \\ \\

    \theutterance \stepcounter{utterance}  
    & & & \multicolumn{2}{p{0.3\linewidth}}{
        \cellcolor[rgb]{0.9,0.9,0.9}{
            \makecell[{{p{\linewidth}}}]{
                \texttt{\tiny{[GM$|$GM]}}
                \texttt{* success: True} \\
\texttt{* lose: False} \\
\texttt{* aborted: False} \\
\texttt{{-}{-}{-}{-}{-}{-}{-}} \\
\texttt{* Shifts: 6.00} \\
\texttt{* Max Shifts: 12.00} \\
\texttt{* Min Shifts: 6.00} \\
\texttt{* End Distance Sum: 0.00} \\
\texttt{* Init Distance Sum: 18.94} \\
\texttt{* Expected Distance Sum: 29.33} \\
\texttt{* Penalties: 0.00} \\
\texttt{* Max Penalties: 16.00} \\
\texttt{* Rounds: 9.00} \\
\texttt{* Max Rounds: 28.00} \\
\texttt{* Object Count: 7.00} \\
            }
        }
    }
    & & \\ \\

    \theutterance \stepcounter{utterance}  
    & & & \multicolumn{2}{p{0.3\linewidth}}{
        \cellcolor[rgb]{0.9,0.9,0.9}{
            \makecell[{{p{\linewidth}}}]{
                \texttt{\tiny{[GM$|$GM]}}
                \texttt{Initial Grids:} \\
\texttt{```} \\
\texttt{Player 1:     Player 2:} \\
\texttt{ 1234567       1234567} \\
\texttt{╔═╤═╤═╤═╗     ╔═╤═╤═╤═╗} \\
\texttt{║◌│◌│◌│A║ 1   ║◌│◌│◌│◌║ 1} \\
\texttt{╟─┴─┼─┤◌║ 2   ╟─┴─┼─┤◌║ 2} \\
\texttt{║◌◌M│◌└─╢ 3   ║◌M◌│◌└─╢ 3} \\
\texttt{╟───┘◌S◌║ 4   ╟───┘P◌◌║ 4} \\
\texttt{║◌◌◌◌◌┌─╢ 5   ║O◌◌◌◌┌─╢ 5} \\
\texttt{║◌◌◌PT│O║ 6   ║◌D◌◌A│S║ 6} \\
\texttt{║◌D◌◌◌│◌║ 7   ║◌◌◌◌T│◌║ 7} \\
\texttt{╚═════╧═╝     ╚═════╧═╝} \\
\texttt{```} \\
\\ 
\texttt{Final Grids:} \\
\texttt{```} \\
\texttt{Player 1:     Player 2:} \\
\texttt{ 1234567       1234567} \\
\texttt{╔═╤═╤═╤═╗     ╔═╤═╤═╤═╗} \\
\texttt{║◌│◌│◌│A║ 1   ║◌│◌│◌│A║ 1} \\
\texttt{╟─┴─┼─┤◌║ 2   ╟─┴─┼─┤◌║ 2} \\
\texttt{║◌◌M│◌└─╢ 3   ║◌◌M│◌└─╢ 3} \\
\texttt{╟───┘◌S◌║ 4   ╟───┘◌S◌║ 4} \\
\texttt{║◌◌◌◌◌┌─╢ 5   ║◌◌◌◌◌┌─╢ 5} \\
\texttt{║◌◌◌PT│O║ 6   ║◌◌◌PT│O║ 6} \\
\texttt{║◌D◌◌◌│◌║ 7   ║◌D◌◌◌│◌║ 7} \\
\texttt{╚═════╧═╝     ╚═════╧═╝} \\
\texttt{```} \\
            }
        }
    }
    & & \\ \\

\end{supertabular}
}

\end{document}
