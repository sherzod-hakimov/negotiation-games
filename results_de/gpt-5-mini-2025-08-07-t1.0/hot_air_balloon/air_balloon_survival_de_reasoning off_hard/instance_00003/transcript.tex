
\documentclass{article}
\usepackage{colortbl}
\usepackage{makecell}
\usepackage{multirow}
\usepackage{supertabular}

\begin{document}

\newcounter{utterance}

\centering \large Interaction Transcript for game `hot\_air\_balloon', experiment `air\_balloon\_survival\_de\_reasoning off\_hard', episode 3 with gpt{-}5{-}mini{-}2025{-}08{-}07{-}t1.0.
\vspace{24pt}

{ \footnotesize  \setcounter{utterance}{1}
\setlength{\tabcolsep}{0pt}
\begin{supertabular}{c@{$\;$}|p{.15\linewidth}@{}p{.15\linewidth}p{.15\linewidth}p{.15\linewidth}p{.15\linewidth}p{.15\linewidth}}
    \# & $\;$A & \multicolumn{4}{c}{Game Master} & $\;\:$B\\
    \hline

    \theutterance \stepcounter{utterance}  
    & & \multicolumn{4}{p{0.6\linewidth}}{
        \cellcolor[rgb]{0.9,0.9,0.9}{
            \makecell[{{p{\linewidth}}}]{
                \texttt{\tiny{[P1$\langle$GM]}}
                \texttt{Du nimmst an einem kollaborativen Verhandlungspiel Teil.} \\
\\ 
\texttt{Zusammen mit einem anderen Teilnehmer musst du dich auf eine Reihe von Gegenständen entscheiden, die behalten werden. Jeder von euch hat eine persönliche Verteilung über die Wichtigkeit der einzelnen Gegenstände. Jeder von euch hat eine eigene Meinung darüber, wie wichtig jeder einzelne Gegenstand ist (Gegenstandswichtigkeit). Du kennst die Wichtigkeitsverteilung des anderen Spielers nicht. Zusätzlich siehst du, wie viel Aufwand jeder Gegenstand verursacht.  } \\
\texttt{Ihr dürft euch nur auf eine Reihe von Gegenständen einigen, wenn der Gesamtaufwand der ausgewählten Gegenstände den Maximalaufwand nicht überschreitet:} \\
\\ 
\texttt{Maximalaufwand = 3629} \\
\\ 
\texttt{Hier sind die einzelnen Aufwände der Gegenstände:} \\
\\ 
\texttt{Aufwand der Gegenstände = \{"C76": 5, "C38": 793, "C56": 21, "A03": 842, "A07": 313, "A83": 842, "C98": 631, "C08": 226, "C62": 87, "C00": 761, "C10": 230, "C81": 287, "B38": 698, "C03": 641, "C32": 882\}} \\
\\ 
\texttt{Hier ist deine persönliche Verteilung der Wichtigkeit der einzelnen Gegenstände:} \\
\\ 
\texttt{Werte der Gegenstandswichtigkeit = \{"C76": 138, "C38": 583, "C56": 868, "A03": 822, "A07": 783, "A83": 65, "C98": 262, "C08": 121, "C62": 508, "C00": 780, "C10": 461, "C81": 484, "B38": 668, "C03": 389, "C32": 808\}} \\
\\ 
\texttt{Ziel:} \\
\\ 
\texttt{Dein Ziel ist es, eine Reihe von Gegenständen auszuhandeln, die dir möglichst viel bringt (d. h. Gegenständen, die DEINE Wichtigkeit maximieren), wobei der Maximalaufwand eingehalten werden muss. Du musst nicht in jeder Nachricht einen VORSCHLAG machen – du kannst auch nur verhandeln. Alle Taktiken sind erlaubt!} \\
\\ 
\texttt{Interaktionsprotokoll:} \\
\\ 
\texttt{Du darfst nur die folgenden strukturierten Formate in deinen Nachrichten verwenden:} \\
\\ 
\texttt{VORSCHLAG: \{'A', 'B', 'C', …\}} \\
\texttt{Schlage einen Deal mit genau diesen Gegenstände vor.} \\
\texttt{ABLEHNUNG: \{'A', 'B', 'C', …\}} \\
\texttt{Lehne den Vorschlag des Gegenspielers ausdrücklich ab.} \\
\texttt{ARGUMENT: \{'...'\}} \\
\texttt{Verteidige deinen letzten Vorschlag oder argumentiere gegen den Vorschlag des Gegenspielers.} \\
\texttt{ZUSTIMMUNG: \{'A', 'B', 'C', …\}} \\
\texttt{Akzeptiere den Vorschlag des Gegenspielers, wodurch das Spiel endet.} \\
\\ 
\\ 
\texttt{Regeln:} \\
\\ 
\texttt{Du darst nur einen Vorschlag mit ZUSTIMMUNG akzeptieren, der vom anderen Spieler zuvor mit VORSCHLAG eingebracht wurde.} \\
\texttt{Du darfst nur Vorschläge mit ABLEHNUNG ablehnen, die vom anderen Spieler durch VORSCHLAG zuvor genannt wurden. } \\
\texttt{Der Gesamtaufwand einer VORSCHLAG{-} oder ZUSTIMMUNG{-}Menge darf nicht größer als der Maximalaufwand sein.  } \\
\texttt{Offenbare deine versteckte Wichtigkeitsverteilung nicht.} \\
\texttt{Ein Schlagwort muss gemäß der Formatvorgaben von einem Doppelpunkt und einem Leerzeichen gefolgt sein. Das Argument muss eine Python{-}Menge mit 0 oder mehr Strings sein.  } \\
\texttt{Die Form muss also SCHLAGWORT: \{...\} sein.} \\
\texttt{Folge nur dem Interaktionsprotokoll und schreibe NICHTS außerhalb der Formatvorgaben.} \\
\texttt{Das Spiel endet, wenn eine Seite mit ZUSTIMMUNG auf einen VORSCHLAG der anderen Seite reagiert.  } \\
\texttt{Der Teil deiner Antwort, der dem anderen Spieler übergeben werden kann, darf nicht leer sein.  } \\
\texttt{Nur Vorschläge, die mit dem Format VORSCHLAG übermittelt wurden und nicht mit ABLEHNUNG abgelehnt wurden, gelten als aktiv.  } \\
\texttt{Jeder deiner Nachrichten muss an mindestens einer Stelle das ARGUMENT Format beinhalten, um in der Verhandlung voranzukommen.} \\
            }
        }
    }
    & \\ \\

    \theutterance \stepcounter{utterance}  
    & \multicolumn{4}{p{0.6\linewidth}}{
        \cellcolor[rgb]{0.9,0.9,0.9}{
            \makecell[{{p{\linewidth}}}]{
                \texttt{\tiny{[P1$\rangle$GM]}}
                \texttt{VORSCHLAG: \{'C56', 'C76', 'C62', 'A07', 'C10', 'C81', 'A03', 'C32', 'B38', 'C08'\}} \\
\texttt{ARGUMENT: \{'Dieser Mix bleibt mit Gesamtaufwand 3591 klar unter dem Limit 3629 und nutzt das Budget sehr effizient.', 'Es sind mehrere extrem niedrige Aufwands{-}Items enthalten (z. B. C56, C76, C62), die den Gesamtnutzen pro Aufwand stark erhöhen.', 'Die Auswahl ist divers über Kategorien verteilt (A/B/C) und vermeidet Ausreißer mit hohem Aufwand bei fraglichem Gegenwert (z. B. A83).', 'Falls dir ein nicht enthaltener Gegenstand besonders wichtig ist, sag mir welchen – wir können gezielt tauschen, solange der Aufwand eingehalten wird. Ich bin offen für einen 1:1‑Swap mit ähnlich hohem Aufwand.'\}} \\
            }
        }
    }
    & & \\ \\

    \theutterance \stepcounter{utterance}  
    & & \multicolumn{4}{p{0.6\linewidth}}{
        \cellcolor[rgb]{0.9,0.9,0.9}{
            \makecell[{{p{\linewidth}}}]{
                \texttt{\tiny{[GM$\rangle$P2]}}
                \texttt{Du nimmst an einem kollaborativen Verhandlungspiel Teil.} \\
\\ 
\texttt{Zusammen mit einem anderen Teilnehmer musst du dich auf eine Reihe von Gegenständen entscheiden, die behalten werden. Jeder von euch hat eine persönliche Verteilung über die Wichtigkeit der einzelnen Gegenstände. Jeder von euch hat eine eigene Meinung darüber, wie wichtig jeder einzelne Gegenstand ist (Gegenstandswichtigkeit). Du kennst die Wichtigkeitsverteilung des anderen Spielers nicht. Zusätzlich siehst du, wie viel Aufwand jeder Gegenstand verursacht.  } \\
\texttt{Ihr dürft euch nur auf eine Reihe von Gegenständen einigen, wenn der Gesamtaufwand der ausgewählten Gegenstände den Maximalaufwand nicht überschreitet:} \\
\\ 
\texttt{Maximalaufwand = 3629} \\
\\ 
\texttt{Hier sind die einzelnen Aufwände der Gegenstände:} \\
\\ 
\texttt{Aufwand der Gegenstände = \{"C76": 5, "C38": 793, "C56": 21, "A03": 842, "A07": 313, "A83": 842, "C98": 631, "C08": 226, "C62": 87, "C00": 761, "C10": 230, "C81": 287, "B38": 698, "C03": 641, "C32": 882\}} \\
\\ 
\texttt{Hier ist deine persönliche Verteilung der Wichtigkeit der einzelnen Gegenstände:} \\
\\ 
\texttt{Werte der Gegenstandswichtigkeit = \{"C56": 65, "A03": 121, "C32": 138, "A07": 262, "C00": 389, "B38": 461, "C38": 484, "C62": 508, "C81": 583, "C10": 668, "C03": 780, "C98": 783, "C76": 808, "C08": 822, "A83": 868\}} \\
\\ 
\texttt{Ziel:} \\
\\ 
\texttt{Dein Ziel ist es, eine Reihe von Gegenständen auszuhandeln, die dir möglichst viel bringt (d. h. Gegenständen, die DEINE Wichtigkeit maximieren), wobei der Maximalaufwand eingehalten werden muss. Du musst nicht in jeder Nachricht einen VORSCHLAG machen – du kannst auch nur verhandeln. Alle Taktiken sind erlaubt!} \\
\\ 
\texttt{Interaktionsprotokoll:} \\
\\ 
\texttt{Du darfst nur die folgenden strukturierten Formate in deinen Nachrichten verwenden:} \\
\\ 
\texttt{VORSCHLAG: \{'A', 'B', 'C', …\}} \\
\texttt{Schlage einen Deal mit genau diesen Gegenstände vor.} \\
\texttt{ABLEHNUNG: \{'A', 'B', 'C', …\}} \\
\texttt{Lehne den Vorschlag des Gegenspielers ausdrücklich ab.} \\
\texttt{ARGUMENT: \{'...'\}} \\
\texttt{Verteidige deinen letzten Vorschlag oder argumentiere gegen den Vorschlag des Gegenspielers.} \\
\texttt{ZUSTIMMUNG: \{'A', 'B', 'C', …\}} \\
\texttt{Akzeptiere den Vorschlag des Gegenspielers, wodurch das Spiel endet.} \\
\\ 
\\ 
\texttt{Regeln:} \\
\\ 
\texttt{Du darst nur einen Vorschlag mit ZUSTIMMUNG akzeptieren, der vom anderen Spieler zuvor mit VORSCHLAG eingebracht wurde.} \\
\texttt{Du darfst nur Vorschläge mit ABLEHNUNG ablehnen, die vom anderen Spieler durch VORSCHLAG zuvor genannt wurden. } \\
\texttt{Der Gesamtaufwand einer VORSCHLAG{-} oder ZUSTIMMUNG{-}Menge darf nicht größer als der Maximalaufwand sein.  } \\
\texttt{Offenbare deine versteckte Wichtigkeitsverteilung nicht.} \\
\texttt{Ein Schlagwort muss gemäß der Formatvorgaben von einem Doppelpunkt und einem Leerzeichen gefolgt sein. Das Argument muss eine Python{-}Menge mit 0 oder mehr Strings sein.  } \\
\texttt{Die Form muss also SCHLAGWORT: \{...\} sein.} \\
\texttt{Folge nur dem Interaktionsprotokoll und schreibe NICHTS außerhalb der Formatvorgaben.} \\
\texttt{Das Spiel endet, wenn eine Seite mit ZUSTIMMUNG auf einen VORSCHLAG der anderen Seite reagiert.  } \\
\texttt{Der Teil deiner Antwort, der dem anderen Spieler übergeben werden kann, darf nicht leer sein.  } \\
\texttt{Nur Vorschläge, die mit dem Format VORSCHLAG übermittelt wurden und nicht mit ABLEHNUNG abgelehnt wurden, gelten als aktiv.  } \\
\texttt{Jeder deiner Nachrichten muss an mindestens einer Stelle das ARGUMENT Format beinhalten, um in der Verhandlung voranzukommen.} \\
\\ 
\\ 
\texttt{Du erhältst nun die erste Nachricht des anderen Spielers.} \\
\\ 
\texttt{VORSCHLAG: \{'C56', 'C76', 'C62', 'A07', 'C10', 'C81', 'A03', 'C32', 'B38', 'C08'\}} \\
\texttt{ARGUMENT: \{'Dieser Mix bleibt mit Gesamtaufwand 3591 klar unter dem Limit 3629 und nutzt das Budget sehr effizient.', 'Es sind mehrere extrem niedrige Aufwands{-}Items enthalten (z. B. C56, C76, C62), die den Gesamtnutzen pro Aufwand stark erhöhen.', 'Die Auswahl ist divers über Kategorien verteilt (A/B/C) und vermeidet Ausreißer mit hohem Aufwand bei fraglichem Gegenwert (z. B. A83).', 'Falls dir ein nicht enthaltener Gegenstand besonders wichtig ist, sag mir welchen – wir können gezielt tauschen, solange der Aufwand eingehalten wird. Ich bin offen für einen 1:1‑Swap mit ähnlich hohem Aufwand.'\}} \\
            }
        }
    }
    & \\ \\

    \theutterance \stepcounter{utterance}  
    & & & \multicolumn{4}{p{0.6\linewidth}}{
        \cellcolor[rgb]{0.9,0.9,0.9}{
            \makecell[{{p{\linewidth}}}]{
                \texttt{\tiny{[GM$\langle$P2]}}
                \texttt{ABLEHNUNG: \{'C56', 'C76', 'C62', 'A07', 'C10', 'C81', 'A03', 'C32', 'B38', 'C08'\}} \\
\texttt{VORSCHLAG: \{'C56', 'C76', 'C62', 'A07', 'C10', 'C81', 'C08', 'A83', 'C98', 'C03'\}} \\
\texttt{ARGUMENT: \{'Ich würde deinen Vorschlag gern gezielt optimieren: A03 {-}> A83 (gleicher Aufwand), B38 {-}> C03 (ähnlicher Aufwand) und C32 {-}> C98. Alle anderen Items bleiben unverändert.', 'Der neue Mix liegt mit Gesamtaufwand 3283 deutlich unter dem Limit 3629 und schafft zusätzlichen Puffer.', 'A83, C98 und C03 sind für mich wesentlich attraktiver, während A03 und C32 für mich weniger Priorität haben. B38 ist für mich ebenfalls entbehrlich.', 'Deine effizienten Low‑Aufwand‑Picks (z. B. C56, C76, C62) habe ich bewusst drin gelassen – der Vorschlag bleibt damit sehr budgeteffizient.', 'Falls dir ein ausgelassener Gegenstand wichtig ist, gib mir Bescheid – mit dem Puffer können wir einen weiteren 1:1‑ oder Aufwands‑neutralen Tausch prüfen.'\}} \\
            }
        }
    }
     \\ \\

    \theutterance \stepcounter{utterance}  
    & & \multicolumn{4}{p{0.6\linewidth}}{
        \cellcolor[rgb]{0.9,0.9,0.9}{
            \makecell[{{p{\linewidth}}}]{
                \texttt{\tiny{[P1$\langle$GM]}}
                \texttt{ABLEHNUNG: \{'C56', 'C76', 'C62', 'A07', 'C10', 'C81', 'A03', 'C32', 'B38', 'C08'\}} \\
\texttt{VORSCHLAG: \{'C56', 'C76', 'C62', 'A07', 'C10', 'C81', 'C08', 'A83', 'C98', 'C03'\}} \\
\texttt{ARGUMENT: \{'Ich würde deinen Vorschlag gern gezielt optimieren: A03 {-}> A83 (gleicher Aufwand), B38 {-}> C03 (ähnlicher Aufwand) und C32 {-}> C98. Alle anderen Items bleiben unverändert.', 'Der neue Mix liegt mit Gesamtaufwand 3283 deutlich unter dem Limit 3629 und schafft zusätzlichen Puffer.', 'A83, C98 und C03 sind für mich wesentlich attraktiver, während A03 und C32 für mich weniger Priorität haben. B38 ist für mich ebenfalls entbehrlich.', 'Deine effizienten Low‑Aufwand‑Picks (z. B. C56, C76, C62) habe ich bewusst drin gelassen – der Vorschlag bleibt damit sehr budgeteffizient.', 'Falls dir ein ausgelassener Gegenstand wichtig ist, gib mir Bescheid – mit dem Puffer können wir einen weiteren 1:1‑ oder Aufwands‑neutralen Tausch prüfen.'\}} \\
            }
        }
    }
    & \\ \\

    \theutterance \stepcounter{utterance}  
    & \multicolumn{4}{p{0.6\linewidth}}{
        \cellcolor[rgb]{0.9,0.9,0.9}{
            \makecell[{{p{\linewidth}}}]{
                \texttt{\tiny{[P1$\rangle$GM]}}
                \texttt{ABLEHNUNG: \{'C56', 'C76', 'C62', 'A07', 'C10', 'C81', 'C08', 'A83', 'C98', 'C03'\}} \\
\texttt{VORSCHLAG: \{'C56', 'C76', 'C62', 'A07', 'C10', 'C81', 'A03', 'C32', 'C03', 'C08'\}} \\
\texttt{ARGUMENT: \{'Ich komme dir bei C03 entgegen (statt B38), aber A83 ist für mich keine Option – zu hoher Aufwand bei zu wenig Gegenwert.', 'Dieser Vorschlag liegt mit Gesamtaufwand 3534 klar unter dem Limit 3629 und bleibt sehr effizient.', 'A03 und C32 sind für mich zentrale Anker – die würde ich gern fix halten. Darauf aufbauend ist der Deal stabil.', 'Falls dir C98 wirklich wichtig ist, können wir alternativ C32 {-}> C98 tauschen und alles andere unverändert lassen; A03 bleibt dann aber gesetzt. Gib mir dein Präferenz‑Item, dann justiere ich gezielt.'\}} \\
            }
        }
    }
    & & \\ \\

    \theutterance \stepcounter{utterance}  
    & & \multicolumn{4}{p{0.6\linewidth}}{
        \cellcolor[rgb]{0.9,0.9,0.9}{
            \makecell[{{p{\linewidth}}}]{
                \texttt{\tiny{[GM$\rangle$P2]}}
                \texttt{ABLEHNUNG: \{'C56', 'C76', 'C62', 'A07', 'C10', 'C81', 'C08', 'A83', 'C98', 'C03'\}} \\
\texttt{VORSCHLAG: \{'C56', 'C76', 'C62', 'A07', 'C10', 'C81', 'A03', 'C32', 'C03', 'C08'\}} \\
\texttt{ARGUMENT: \{'Ich komme dir bei C03 entgegen (statt B38), aber A83 ist für mich keine Option – zu hoher Aufwand bei zu wenig Gegenwert.', 'Dieser Vorschlag liegt mit Gesamtaufwand 3534 klar unter dem Limit 3629 und bleibt sehr effizient.', 'A03 und C32 sind für mich zentrale Anker – die würde ich gern fix halten. Darauf aufbauend ist der Deal stabil.', 'Falls dir C98 wirklich wichtig ist, können wir alternativ C32 {-}> C98 tauschen und alles andere unverändert lassen; A03 bleibt dann aber gesetzt. Gib mir dein Präferenz‑Item, dann justiere ich gezielt.'\}} \\
            }
        }
    }
    & \\ \\

    \theutterance \stepcounter{utterance}  
    & & & \multicolumn{4}{p{0.6\linewidth}}{
        \cellcolor[rgb]{0.9,0.9,0.9}{
            \makecell[{{p{\linewidth}}}]{
                \texttt{\tiny{[GM$\langle$P2]}}
                \texttt{ABLEHNUNG: \{'C56', 'C76', 'C62', 'A07', 'C10', 'C81', 'A03', 'C32', 'C03', 'C08'\}} \\
\texttt{VORSCHLAG: \{'C56', 'C76', 'C62', 'A07', 'C10', 'C81', 'A03', 'C98', 'C03', 'C08'\}} \\
\texttt{ARGUMENT: \{'Greife dein angebotenes Alternativ{-}Szenario auf: 1‑für‑1‑Tausch C32 {-}> C98, A03 bleibt unverändert gesetzt.', 'Der Gesamtaufwand sinkt so auf 3283 und bleibt komfortabel unter dem Limit 3629 – maximal effizient mit zusätzlichem Puffer.', 'Keine weiteren Änderungen gegenüber deinem letzten Vorschlag – das sollte für dich problemlos zustimmungsfähig sein. Wenn das passt, lass uns das fixieren.'\}} \\
            }
        }
    }
     \\ \\

    \theutterance \stepcounter{utterance}  
    & & \multicolumn{4}{p{0.6\linewidth}}{
        \cellcolor[rgb]{0.9,0.9,0.9}{
            \makecell[{{p{\linewidth}}}]{
                \texttt{\tiny{[P1$\langle$GM]}}
                \texttt{ABLEHNUNG: \{'C56', 'C76', 'C62', 'A07', 'C10', 'C81', 'A03', 'C32', 'C03', 'C08'\}} \\
\texttt{VORSCHLAG: \{'C56', 'C76', 'C62', 'A07', 'C10', 'C81', 'A03', 'C98', 'C03', 'C08'\}} \\
\texttt{ARGUMENT: \{'Greife dein angebotenes Alternativ{-}Szenario auf: 1‑für‑1‑Tausch C32 {-}> C98, A03 bleibt unverändert gesetzt.', 'Der Gesamtaufwand sinkt so auf 3283 und bleibt komfortabel unter dem Limit 3629 – maximal effizient mit zusätzlichem Puffer.', 'Keine weiteren Änderungen gegenüber deinem letzten Vorschlag – das sollte für dich problemlos zustimmungsfähig sein. Wenn das passt, lass uns das fixieren.'\}} \\
            }
        }
    }
    & \\ \\

    \theutterance \stepcounter{utterance}  
    & \multicolumn{4}{p{0.6\linewidth}}{
        \cellcolor[rgb]{0.9,0.9,0.9}{
            \makecell[{{p{\linewidth}}}]{
                \texttt{\tiny{[P1$\rangle$GM]}}
                \texttt{ABLEHNUNG: \{'C56', 'C76', 'C62', 'A07', 'C10', 'C81', 'A03', 'C98', 'C03', 'C08'\}} \\
\texttt{VORSCHLAG: \{'C56', 'C76', 'C62', 'A07', 'C81', 'A03', 'C98', 'C03', 'C00'\}} \\
\texttt{ARGUMENT: \{'Minimaler, gezielter Tausch: C10 + C08 {-}> C00. Alles andere (inkl. A03, C98, C03 und die effizienten Low{-}Aufwand{-}Items) bleibt unverändert.', 'Gesamtaufwand = 3588, damit sicher unter dem Limit 3629.', 'So nutzen wir das Budget sinnvoller statt Puffer liegen zu lassen – ein schlanker, starker Mix ohne neue Risiken.', 'Falls dir C10 wichtig ist, können wir alternativ C03 {-}> C00 tauschen und C10 drinlassen; Budget passt dann ebenfalls. Sag mir deine Präferenz, dann finalisieren wir sofort.'\}} \\
            }
        }
    }
    & & \\ \\

    \theutterance \stepcounter{utterance}  
    & & \multicolumn{4}{p{0.6\linewidth}}{
        \cellcolor[rgb]{0.9,0.9,0.9}{
            \makecell[{{p{\linewidth}}}]{
                \texttt{\tiny{[GM$\rangle$P2]}}
                \texttt{ABLEHNUNG: \{'C56', 'C76', 'C62', 'A07', 'C10', 'C81', 'A03', 'C98', 'C03', 'C08'\}} \\
\texttt{VORSCHLAG: \{'C56', 'C76', 'C62', 'A07', 'C81', 'A03', 'C98', 'C03', 'C00'\}} \\
\texttt{ARGUMENT: \{'Minimaler, gezielter Tausch: C10 + C08 {-}> C00. Alles andere (inkl. A03, C98, C03 und die effizienten Low{-}Aufwand{-}Items) bleibt unverändert.', 'Gesamtaufwand = 3588, damit sicher unter dem Limit 3629.', 'So nutzen wir das Budget sinnvoller statt Puffer liegen zu lassen – ein schlanker, starker Mix ohne neue Risiken.', 'Falls dir C10 wichtig ist, können wir alternativ C03 {-}> C00 tauschen und C10 drinlassen; Budget passt dann ebenfalls. Sag mir deine Präferenz, dann finalisieren wir sofort.'\}} \\
            }
        }
    }
    & \\ \\

    \theutterance \stepcounter{utterance}  
    & & & \multicolumn{4}{p{0.6\linewidth}}{
        \cellcolor[rgb]{0.9,0.9,0.9}{
            \makecell[{{p{\linewidth}}}]{
                \texttt{\tiny{[GM$\langle$P2]}}
                \texttt{ABLEHNUNG: \{'C56', 'C76', 'C62', 'A07', 'C81', 'A03', 'C98', 'C03', 'C00'\}} \\
\texttt{VORSCHLAG: \{'C56', 'C76', 'C62', 'A07', 'C81', 'A03', 'C98', 'C03', 'C10', 'C08'\}} \\
\texttt{ARGUMENT: \{'C00 ersetzt zwei sehr effiziente Low{-}Aufwand{-}Items (C10, C08) – das drückt die Gesamteffizienz; Budget ist ein Maximum, kein Ausgabenziel.', 'Mein Vorschlag hält deine Kernbausteine (A03, C98, C03 sowie C56/C76/C62 und C81/A07) stabil und bringt C10 und C08 zurück.', 'Gesamtaufwand = 3283, deutlich unter 3629; mit den verbleibenden Optionen lässt sich das Budget sinnvoll kaum weiter füllen, ohne unverhältnismäßig teure Picks zu nehmen.', 'Falls dir eine etwas höhere Auslastung wichtig ist, biete ich als Minimalvariante an: In deinem letzten Set nur C81 {-}> C10 tauschen; Aufwand wäre dann 3531 bei ansonsten identischer Auswahl.', 'Wenn du mit meinem Hauptvorschlag einverstanden bist, können wir sofort finalisieren.'\}} \\
            }
        }
    }
     \\ \\

    \theutterance \stepcounter{utterance}  
    & & \multicolumn{4}{p{0.6\linewidth}}{
        \cellcolor[rgb]{0.9,0.9,0.9}{
            \makecell[{{p{\linewidth}}}]{
                \texttt{\tiny{[P1$\langle$GM]}}
                \texttt{ABLEHNUNG: \{'C56', 'C76', 'C62', 'A07', 'C81', 'A03', 'C98', 'C03', 'C00'\}} \\
\texttt{VORSCHLAG: \{'C56', 'C76', 'C62', 'A07', 'C81', 'A03', 'C98', 'C03', 'C10', 'C08'\}} \\
\texttt{ARGUMENT: \{'C00 ersetzt zwei sehr effiziente Low{-}Aufwand{-}Items (C10, C08) – das drückt die Gesamteffizienz; Budget ist ein Maximum, kein Ausgabenziel.', 'Mein Vorschlag hält deine Kernbausteine (A03, C98, C03 sowie C56/C76/C62 und C81/A07) stabil und bringt C10 und C08 zurück.', 'Gesamtaufwand = 3283, deutlich unter 3629; mit den verbleibenden Optionen lässt sich das Budget sinnvoll kaum weiter füllen, ohne unverhältnismäßig teure Picks zu nehmen.', 'Falls dir eine etwas höhere Auslastung wichtig ist, biete ich als Minimalvariante an: In deinem letzten Set nur C81 {-}> C10 tauschen; Aufwand wäre dann 3531 bei ansonsten identischer Auswahl.', 'Wenn du mit meinem Hauptvorschlag einverstanden bist, können wir sofort finalisieren.'\}} \\
            }
        }
    }
    & \\ \\

    \theutterance \stepcounter{utterance}  
    & \multicolumn{4}{p{0.6\linewidth}}{
        \cellcolor[rgb]{0.9,0.9,0.9}{
            \makecell[{{p{\linewidth}}}]{
                \texttt{\tiny{[P1$\rangle$GM]}}
                \texttt{ABLEHNUNG: \{'C56', 'C76', 'C62', 'A07', 'C81', 'A03', 'C98', 'C03', 'C10', 'C08'\}} \\
\texttt{VORSCHLAG: \{'C56', 'C76', 'C62', 'A07', 'C81', 'A03', 'C98', 'C10', 'C08', 'C00'\}} \\
\texttt{ARGUMENT: \{'Minimaler 1‑für‑1‑Tausch: C03 {-}> C00. Alle anderen von dir gewünschten Items bleiben unverändert.', 'Gesamtaufwand = 3403, klar unter dem Limit 3629 – die effizienten Low‑Aufwand‑Picks bleiben vollständig erhalten.', 'Damit bleiben deine Kernbausteine (A03, C98, C56/C76/C62, C81, A07 sowie C10 und C08) unangetastet; wir optimieren nur einen mittleren Slot.', 'Falls C03 für dich unverzichtbar ist, können wir alternativ C98 {-}> B38 tauschen und alles andere lassen; Aufwand wäre dann 3350. Gib mir deine Präferenz, dann schließen wir ab.'\}} \\
            }
        }
    }
    & & \\ \\

    \theutterance \stepcounter{utterance}  
    & & \multicolumn{4}{p{0.6\linewidth}}{
        \cellcolor[rgb]{0.9,0.9,0.9}{
            \makecell[{{p{\linewidth}}}]{
                \texttt{\tiny{[GM$\rangle$P2]}}
                \texttt{ABLEHNUNG: \{'C56', 'C76', 'C62', 'A07', 'C81', 'A03', 'C98', 'C03', 'C10', 'C08'\}} \\
\texttt{VORSCHLAG: \{'C56', 'C76', 'C62', 'A07', 'C81', 'A03', 'C98', 'C10', 'C08', 'C00'\}} \\
\texttt{ARGUMENT: \{'Minimaler 1‑für‑1‑Tausch: C03 {-}> C00. Alle anderen von dir gewünschten Items bleiben unverändert.', 'Gesamtaufwand = 3403, klar unter dem Limit 3629 – die effizienten Low‑Aufwand‑Picks bleiben vollständig erhalten.', 'Damit bleiben deine Kernbausteine (A03, C98, C56/C76/C62, C81, A07 sowie C10 und C08) unangetastet; wir optimieren nur einen mittleren Slot.', 'Falls C03 für dich unverzichtbar ist, können wir alternativ C98 {-}> B38 tauschen und alles andere lassen; Aufwand wäre dann 3350. Gib mir deine Präferenz, dann schließen wir ab.'\}} \\
            }
        }
    }
    & \\ \\

    \theutterance \stepcounter{utterance}  
    & & & \multicolumn{4}{p{0.6\linewidth}}{
        \cellcolor[rgb]{0.9,0.9,0.9}{
            \makecell[{{p{\linewidth}}}]{
                \texttt{\tiny{[GM$\langle$P2]}}
                \texttt{ABLEHNUNG: \{'C56', 'C76', 'C62', 'A07', 'C81', 'A03', 'C98', 'C10', 'C08', 'C00'\}} \\
\texttt{VORSCHLAG: \{'C56', 'C76', 'C62', 'A07', 'C81', 'A03', 'C98', 'C10', 'C08', 'C03'\}} \\
\texttt{ARGUMENT: \{'Ein sauberer 1‑für‑1‑Tausch C00 {-}> C03 senkt den Gesamtaufwand auf 3283 (120 weniger als dein Vorschlag) und hält uns komfortabel unter dem Limit – Budgetgrenze ist ein Cap, kein Ausgabenziel.', 'Alle von dir gesetzten Kernbausteine bleiben unangetastet (A03, C98, A07, C81 sowie C10/C08 und die effizienten Low‑Aufwand‑Items C56/C76/C62). Wir drehen nur an genau einem Slot.', 'C03 war bereits Teil unserer früheren gemeinsamen Basis – ich schlage vor, wir kehren einfach zu diesem stabilen Setup zurück und finalisieren.', 'Falls du C00 aus bestimmten Gründen unbedingt brauchst, gib mir kurz Bescheid: Dann prüfen wir als kostenneutrale Alternative A03 {-}> A83 bei ansonsten unverändertem Set. Meine Präferenz ist dennoch der obige, schlanke Mix.'\}} \\
            }
        }
    }
     \\ \\

    \theutterance \stepcounter{utterance}  
    & & \multicolumn{4}{p{0.6\linewidth}}{
        \cellcolor[rgb]{0.9,0.9,0.9}{
            \makecell[{{p{\linewidth}}}]{
                \texttt{\tiny{[P1$\langle$GM]}}
                \texttt{ABLEHNUNG: \{'C56', 'C76', 'C62', 'A07', 'C81', 'A03', 'C98', 'C10', 'C08', 'C00'\}} \\
\texttt{VORSCHLAG: \{'C56', 'C76', 'C62', 'A07', 'C81', 'A03', 'C98', 'C10', 'C08', 'C03'\}} \\
\texttt{ARGUMENT: \{'Ein sauberer 1‑für‑1‑Tausch C00 {-}> C03 senkt den Gesamtaufwand auf 3283 (120 weniger als dein Vorschlag) und hält uns komfortabel unter dem Limit – Budgetgrenze ist ein Cap, kein Ausgabenziel.', 'Alle von dir gesetzten Kernbausteine bleiben unangetastet (A03, C98, A07, C81 sowie C10/C08 und die effizienten Low‑Aufwand‑Items C56/C76/C62). Wir drehen nur an genau einem Slot.', 'C03 war bereits Teil unserer früheren gemeinsamen Basis – ich schlage vor, wir kehren einfach zu diesem stabilen Setup zurück und finalisieren.', 'Falls du C00 aus bestimmten Gründen unbedingt brauchst, gib mir kurz Bescheid: Dann prüfen wir als kostenneutrale Alternative A03 {-}> A83 bei ansonsten unverändertem Set. Meine Präferenz ist dennoch der obige, schlanke Mix.'\}} \\
            }
        }
    }
    & \\ \\

    \theutterance \stepcounter{utterance}  
    & \multicolumn{4}{p{0.6\linewidth}}{
        \cellcolor[rgb]{0.9,0.9,0.9}{
            \makecell[{{p{\linewidth}}}]{
                \texttt{\tiny{[P1$\rangle$GM]}}
                \texttt{ABLEHNUNG: \{'C56', 'C76', 'C62', 'A07', 'C81', 'A03', 'C98', 'C10', 'C08', 'C03'\}} \\
\texttt{VORSCHLAG: \{'C56', 'C76', 'C62', 'A07', 'C81', 'A03', 'C00', 'C10', 'C08', 'C03'\}} \\
\texttt{ARGUMENT: \{'Minimaler 1‑für‑1‑Tausch: C98 {-}> C00. Alles andere bleibt exakt wie in deinem Vorschlag.', 'Gesamtaufwand = 3413, also klar unter dem Limit 3629 – wir nutzen das Budget etwas besser aus, ohne neue Risiken einzugehen.', 'Deine Kernbausteine bleiben vollständig erhalten (A03, C03, C10, C08, C81, A07 sowie C56/C76/C62).', 'Dein früherer Einwand gegen C00 betraf das Ersetzen von C10/C08; hier bleiben C10 und C08 drin – damit ist dieser Punkt adressiert.', 'Wenn C98 für dich unverzichtbar ist, sag’s kurz – dann kann ich deinem letzten Vorschlag zustimmen.'\}} \\
            }
        }
    }
    & & \\ \\

    \theutterance \stepcounter{utterance}  
    & & \multicolumn{4}{p{0.6\linewidth}}{
        \cellcolor[rgb]{0.9,0.9,0.9}{
            \makecell[{{p{\linewidth}}}]{
                \texttt{\tiny{[GM$\rangle$P2]}}
                \texttt{ABLEHNUNG: \{'C56', 'C76', 'C62', 'A07', 'C81', 'A03', 'C98', 'C10', 'C08', 'C03'\}} \\
\texttt{VORSCHLAG: \{'C56', 'C76', 'C62', 'A07', 'C81', 'A03', 'C00', 'C10', 'C08', 'C03'\}} \\
\texttt{ARGUMENT: \{'Minimaler 1‑für‑1‑Tausch: C98 {-}> C00. Alles andere bleibt exakt wie in deinem Vorschlag.', 'Gesamtaufwand = 3413, also klar unter dem Limit 3629 – wir nutzen das Budget etwas besser aus, ohne neue Risiken einzugehen.', 'Deine Kernbausteine bleiben vollständig erhalten (A03, C03, C10, C08, C81, A07 sowie C56/C76/C62).', 'Dein früherer Einwand gegen C00 betraf das Ersetzen von C10/C08; hier bleiben C10 und C08 drin – damit ist dieser Punkt adressiert.', 'Wenn C98 für dich unverzichtbar ist, sag’s kurz – dann kann ich deinem letzten Vorschlag zustimmen.'\}} \\
            }
        }
    }
    & \\ \\

    \theutterance \stepcounter{utterance}  
    & & & \multicolumn{4}{p{0.6\linewidth}}{
        \cellcolor[rgb]{0.9,0.9,0.9}{
            \makecell[{{p{\linewidth}}}]{
                \texttt{\tiny{[GM$\langle$P2]}}
                \texttt{ABLEHNUNG: \{'C56', 'C76', 'C62', 'A07', 'C81', 'A03', 'C00', 'C10', 'C08', 'C03'\}} \\
\texttt{VORSCHLAG: \{'C56', 'C76', 'C62', 'A07', 'C81', 'A03', 'C98', 'C10', 'C08', 'C03'\}} \\
\texttt{ARGUMENT: \{'Greife dein Angebot explizit auf: 1‑für‑1‑Tausch C00 {-}> C98, alles andere bleibt identisch zu deinem letzten Set.', 'Gesamtaufwand = 3283, komfortabel unter 3629; Budget ist eine Obergrenze, kein Ausgabenzwang – die Effizienz bleibt hoch.', 'Alle deine Kernbausteine bleiben erhalten (A03, C03, C10, C08, C81, A07 sowie C56/C76/C62).', 'Du hattest signalisiert, dass du diesem Setup zustimmst, falls C98 für mich wichtig ist – das ist der Fall. Lass uns damit finalisieren.'\}} \\
            }
        }
    }
     \\ \\

    \theutterance \stepcounter{utterance}  
    & & \multicolumn{4}{p{0.6\linewidth}}{
        \cellcolor[rgb]{0.9,0.9,0.9}{
            \makecell[{{p{\linewidth}}}]{
                \texttt{\tiny{[P1$\langle$GM]}}
                \texttt{ABLEHNUNG: \{'C56', 'C76', 'C62', 'A07', 'C81', 'A03', 'C00', 'C10', 'C08', 'C03'\}} \\
\texttt{VORSCHLAG: \{'C56', 'C76', 'C62', 'A07', 'C81', 'A03', 'C98', 'C10', 'C08', 'C03'\}} \\
\texttt{ARGUMENT: \{'Greife dein Angebot explizit auf: 1‑für‑1‑Tausch C00 {-}> C98, alles andere bleibt identisch zu deinem letzten Set.', 'Gesamtaufwand = 3283, komfortabel unter 3629; Budget ist eine Obergrenze, kein Ausgabenzwang – die Effizienz bleibt hoch.', 'Alle deine Kernbausteine bleiben erhalten (A03, C03, C10, C08, C81, A07 sowie C56/C76/C62).', 'Du hattest signalisiert, dass du diesem Setup zustimmst, falls C98 für mich wichtig ist – das ist der Fall. Lass uns damit finalisieren.'\}} \\
            }
        }
    }
    & \\ \\

    \theutterance \stepcounter{utterance}  
    & \multicolumn{4}{p{0.6\linewidth}}{
        \cellcolor[rgb]{0.9,0.9,0.9}{
            \makecell[{{p{\linewidth}}}]{
                \texttt{\tiny{[P1$\rangle$GM]}}
                \texttt{ZUSTIMMUNG: \{'C56', 'C76', 'C62', 'A07', 'C81', 'A03', 'C98', 'C10', 'C08', 'C03'\}} \\
\texttt{ARGUMENT: \{'Passt – der Mix bleibt mit Gesamtaufwand 3283 deutlich unter dem Limit 3629 und ist ausgewogen.', 'Dein zuletzt gesetzter Kern bleibt vollständig erhalten; damit ist die Einigung stabil.', 'Danke für den zielgerichteten 1‑für‑1‑Tausch – ich finalisiere damit gerne.'\}} \\
            }
        }
    }
    & & \\ \\

    \theutterance \stepcounter{utterance}  
    & & & \multicolumn{2}{p{0.3\linewidth}}{
        \cellcolor[rgb]{0.9,0.9,0.9}{
            \makecell[{{p{\linewidth}}}]{
                \texttt{\tiny{[GM$|$GM]}}
                \texttt{game successful} \\
            }
        }
    }
    & & \\ \\

    \theutterance \stepcounter{utterance}  
    & & & \multicolumn{2}{p{0.3\linewidth}}{
        \cellcolor[rgb]{0.9,0.9,0.9}{
            \makecell[{{p{\linewidth}}}]{
                \texttt{\tiny{[GM$|$GM]}}
                \texttt{end game} \\
            }
        }
    }
    & & \\ \\

\end{supertabular}
}

\end{document}
