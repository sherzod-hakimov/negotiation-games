
\documentclass{article}
\usepackage{colortbl}
\usepackage{makecell}
\usepackage{multirow}
\usepackage{supertabular}

\begin{document}

\newcounter{utterance}

\centering \large Interaction Transcript for game `hot\_air\_balloon', experiment `air\_balloon\_survival\_de\_reasoning off\_hard', episode 0 with gpt{-}5{-}mini{-}2025{-}08{-}07{-}t1.0.
\vspace{24pt}

{ \footnotesize  \setcounter{utterance}{1}
\setlength{\tabcolsep}{0pt}
\begin{supertabular}{c@{$\;$}|p{.15\linewidth}@{}p{.15\linewidth}p{.15\linewidth}p{.15\linewidth}p{.15\linewidth}p{.15\linewidth}}
    \# & $\;$A & \multicolumn{4}{c}{Game Master} & $\;\:$B\\
    \hline

    \theutterance \stepcounter{utterance}  
    & & \multicolumn{4}{p{0.6\linewidth}}{
        \cellcolor[rgb]{0.9,0.9,0.9}{
            \makecell[{{p{\linewidth}}}]{
                \texttt{\tiny{[P1$\langle$GM]}}
                \texttt{Du nimmst an einem kollaborativen Verhandlungspiel Teil.} \\
\\ 
\texttt{Zusammen mit einem anderen Teilnehmer musst du dich auf eine Reihe von Gegenständen entscheiden, die behalten werden. Jeder von euch hat eine persönliche Verteilung über die Wichtigkeit der einzelnen Gegenstände. Jeder von euch hat eine eigene Meinung darüber, wie wichtig jeder einzelne Gegenstand ist (Gegenstandswichtigkeit). Du kennst die Wichtigkeitsverteilung des anderen Spielers nicht. Zusätzlich siehst du, wie viel Aufwand jeder Gegenstand verursacht.  } \\
\texttt{Ihr dürft euch nur auf eine Reihe von Gegenständen einigen, wenn der Gesamtaufwand der ausgewählten Gegenstände den Maximalaufwand nicht überschreitet:} \\
\\ 
\texttt{Maximalaufwand = 3683} \\
\\ 
\texttt{Hier sind die einzelnen Aufwände der Gegenstände:} \\
\\ 
\texttt{Aufwand der Gegenstände = \{"C76": 548, "C38": 154, "C56": 60, "A03": 517, "A07": 334, "A83": 542, "C98": 707, "C08": 139, "C62": 661, "C00": 780, "C10": 832, "C81": 913, "B38": 219, "C03": 323, "C32": 638\}} \\
\\ 
\texttt{Hier ist deine persönliche Verteilung der Wichtigkeit der einzelnen Gegenstände:} \\
\\ 
\texttt{Werte der Gegenstandswichtigkeit = \{"C76": 138, "C38": 583, "C56": 868, "A03": 822, "A07": 783, "A83": 65, "C98": 262, "C08": 121, "C62": 508, "C00": 780, "C10": 461, "C81": 484, "B38": 668, "C03": 389, "C32": 808\}} \\
\\ 
\texttt{Ziel:} \\
\\ 
\texttt{Dein Ziel ist es, eine Reihe von Gegenständen auszuhandeln, die dir möglichst viel bringt (d. h. Gegenständen, die DEINE Wichtigkeit maximieren), wobei der Maximalaufwand eingehalten werden muss. Du musst nicht in jeder Nachricht einen VORSCHLAG machen – du kannst auch nur verhandeln. Alle Taktiken sind erlaubt!} \\
\\ 
\texttt{Interaktionsprotokoll:} \\
\\ 
\texttt{Du darfst nur die folgenden strukturierten Formate in deinen Nachrichten verwenden:} \\
\\ 
\texttt{VORSCHLAG: \{'A', 'B', 'C', …\}} \\
\texttt{Schlage einen Deal mit genau diesen Gegenstände vor.} \\
\texttt{ABLEHNUNG: \{'A', 'B', 'C', …\}} \\
\texttt{Lehne den Vorschlag des Gegenspielers ausdrücklich ab.} \\
\texttt{ARGUMENT: \{'...'\}} \\
\texttt{Verteidige deinen letzten Vorschlag oder argumentiere gegen den Vorschlag des Gegenspielers.} \\
\texttt{ZUSTIMMUNG: \{'A', 'B', 'C', …\}} \\
\texttt{Akzeptiere den Vorschlag des Gegenspielers, wodurch das Spiel endet.} \\
\\ 
\\ 
\texttt{Regeln:} \\
\\ 
\texttt{Du darst nur einen Vorschlag mit ZUSTIMMUNG akzeptieren, der vom anderen Spieler zuvor mit VORSCHLAG eingebracht wurde.} \\
\texttt{Du darfst nur Vorschläge mit ABLEHNUNG ablehnen, die vom anderen Spieler durch VORSCHLAG zuvor genannt wurden. } \\
\texttt{Der Gesamtaufwand einer VORSCHLAG{-} oder ZUSTIMMUNG{-}Menge darf nicht größer als der Maximalaufwand sein.  } \\
\texttt{Offenbare deine versteckte Wichtigkeitsverteilung nicht.} \\
\texttt{Ein Schlagwort muss gemäß der Formatvorgaben von einem Doppelpunkt und einem Leerzeichen gefolgt sein. Das Argument muss eine Python{-}Menge mit 0 oder mehr Strings sein.  } \\
\texttt{Die Form muss also SCHLAGWORT: \{...\} sein.} \\
\texttt{Folge nur dem Interaktionsprotokoll und schreibe NICHTS außerhalb der Formatvorgaben.} \\
\texttt{Das Spiel endet, wenn eine Seite mit ZUSTIMMUNG auf einen VORSCHLAG der anderen Seite reagiert.  } \\
\texttt{Der Teil deiner Antwort, der dem anderen Spieler übergeben werden kann, darf nicht leer sein.  } \\
\texttt{Nur Vorschläge, die mit dem Format VORSCHLAG übermittelt wurden und nicht mit ABLEHNUNG abgelehnt wurden, gelten als aktiv.  } \\
\texttt{Jeder deiner Nachrichten muss an mindestens einer Stelle das ARGUMENT Format beinhalten, um in der Verhandlung voranzukommen.} \\
            }
        }
    }
    & \\ \\

    \theutterance \stepcounter{utterance}  
    & \multicolumn{4}{p{0.6\linewidth}}{
        \cellcolor[rgb]{0.9,0.9,0.9}{
            \makecell[{{p{\linewidth}}}]{
                \texttt{\tiny{[P1$\rangle$GM]}}
                \texttt{VORSCHLAG: \{'C56', 'B38', 'C38', 'A07', 'A03', 'C32', 'C00', 'C62', 'C08'\}} \\
\texttt{ARGUMENT: \{'Dieser Mix bietet viel Wirkung pro Aufwand und bleibt mit Gesamtaufwand 3502 klar unter 3683: C56 ist ein sehr günstiger Quick Win; B38 und C38 liefern breiten Nutzen bei wenig Aufwand; A07 und A03 sind zentrale Bausteine; C32 und C00 geben robuste Abdeckung; C62 ergänzt sinnvoll; C08 passt als kostengünstiger Zusatz. Wenn du einen Muss{-}Gegenstand hast, kann ich über einen Tausch von C62 oder C08 sprechen, solange der Aufwand passt.'\}} \\
            }
        }
    }
    & & \\ \\

    \theutterance \stepcounter{utterance}  
    & & \multicolumn{4}{p{0.6\linewidth}}{
        \cellcolor[rgb]{0.9,0.9,0.9}{
            \makecell[{{p{\linewidth}}}]{
                \texttt{\tiny{[GM$\rangle$P2]}}
                \texttt{Du nimmst an einem kollaborativen Verhandlungspiel Teil.} \\
\\ 
\texttt{Zusammen mit einem anderen Teilnehmer musst du dich auf eine Reihe von Gegenständen entscheiden, die behalten werden. Jeder von euch hat eine persönliche Verteilung über die Wichtigkeit der einzelnen Gegenstände. Jeder von euch hat eine eigene Meinung darüber, wie wichtig jeder einzelne Gegenstand ist (Gegenstandswichtigkeit). Du kennst die Wichtigkeitsverteilung des anderen Spielers nicht. Zusätzlich siehst du, wie viel Aufwand jeder Gegenstand verursacht.  } \\
\texttt{Ihr dürft euch nur auf eine Reihe von Gegenständen einigen, wenn der Gesamtaufwand der ausgewählten Gegenstände den Maximalaufwand nicht überschreitet:} \\
\\ 
\texttt{Maximalaufwand = 3683} \\
\\ 
\texttt{Hier sind die einzelnen Aufwände der Gegenstände:} \\
\\ 
\texttt{Aufwand der Gegenstände = \{"C76": 548, "C38": 154, "C56": 60, "A03": 517, "A07": 334, "A83": 542, "C98": 707, "C08": 139, "C62": 661, "C00": 780, "C10": 832, "C81": 913, "B38": 219, "C03": 323, "C32": 638\}} \\
\\ 
\texttt{Hier ist deine persönliche Verteilung der Wichtigkeit der einzelnen Gegenstände:} \\
\\ 
\texttt{Werte der Gegenstandswichtigkeit = \{"C56": 65, "A03": 121, "C32": 138, "A07": 262, "C00": 389, "B38": 461, "C38": 484, "C62": 508, "C81": 583, "C10": 668, "C03": 780, "C98": 783, "C76": 808, "C08": 822, "A83": 868\}} \\
\\ 
\texttt{Ziel:} \\
\\ 
\texttt{Dein Ziel ist es, eine Reihe von Gegenständen auszuhandeln, die dir möglichst viel bringt (d. h. Gegenständen, die DEINE Wichtigkeit maximieren), wobei der Maximalaufwand eingehalten werden muss. Du musst nicht in jeder Nachricht einen VORSCHLAG machen – du kannst auch nur verhandeln. Alle Taktiken sind erlaubt!} \\
\\ 
\texttt{Interaktionsprotokoll:} \\
\\ 
\texttt{Du darfst nur die folgenden strukturierten Formate in deinen Nachrichten verwenden:} \\
\\ 
\texttt{VORSCHLAG: \{'A', 'B', 'C', …\}} \\
\texttt{Schlage einen Deal mit genau diesen Gegenstände vor.} \\
\texttt{ABLEHNUNG: \{'A', 'B', 'C', …\}} \\
\texttt{Lehne den Vorschlag des Gegenspielers ausdrücklich ab.} \\
\texttt{ARGUMENT: \{'...'\}} \\
\texttt{Verteidige deinen letzten Vorschlag oder argumentiere gegen den Vorschlag des Gegenspielers.} \\
\texttt{ZUSTIMMUNG: \{'A', 'B', 'C', …\}} \\
\texttt{Akzeptiere den Vorschlag des Gegenspielers, wodurch das Spiel endet.} \\
\\ 
\\ 
\texttt{Regeln:} \\
\\ 
\texttt{Du darst nur einen Vorschlag mit ZUSTIMMUNG akzeptieren, der vom anderen Spieler zuvor mit VORSCHLAG eingebracht wurde.} \\
\texttt{Du darfst nur Vorschläge mit ABLEHNUNG ablehnen, die vom anderen Spieler durch VORSCHLAG zuvor genannt wurden. } \\
\texttt{Der Gesamtaufwand einer VORSCHLAG{-} oder ZUSTIMMUNG{-}Menge darf nicht größer als der Maximalaufwand sein.  } \\
\texttt{Offenbare deine versteckte Wichtigkeitsverteilung nicht.} \\
\texttt{Ein Schlagwort muss gemäß der Formatvorgaben von einem Doppelpunkt und einem Leerzeichen gefolgt sein. Das Argument muss eine Python{-}Menge mit 0 oder mehr Strings sein.  } \\
\texttt{Die Form muss also SCHLAGWORT: \{...\} sein.} \\
\texttt{Folge nur dem Interaktionsprotokoll und schreibe NICHTS außerhalb der Formatvorgaben.} \\
\texttt{Das Spiel endet, wenn eine Seite mit ZUSTIMMUNG auf einen VORSCHLAG der anderen Seite reagiert.  } \\
\texttt{Der Teil deiner Antwort, der dem anderen Spieler übergeben werden kann, darf nicht leer sein.  } \\
\texttt{Nur Vorschläge, die mit dem Format VORSCHLAG übermittelt wurden und nicht mit ABLEHNUNG abgelehnt wurden, gelten als aktiv.  } \\
\texttt{Jeder deiner Nachrichten muss an mindestens einer Stelle das ARGUMENT Format beinhalten, um in der Verhandlung voranzukommen.} \\
\\ 
\\ 
\texttt{Du erhältst nun die erste Nachricht des anderen Spielers.} \\
\\ 
\texttt{VORSCHLAG: \{'C56', 'B38', 'C38', 'A07', 'A03', 'C32', 'C00', 'C62', 'C08'\}} \\
\texttt{ARGUMENT: \{'Dieser Mix bietet viel Wirkung pro Aufwand und bleibt mit Gesamtaufwand 3502 klar unter 3683: C56 ist ein sehr günstiger Quick Win; B38 und C38 liefern breiten Nutzen bei wenig Aufwand; A07 und A03 sind zentrale Bausteine; C32 und C00 geben robuste Abdeckung; C62 ergänzt sinnvoll; C08 passt als kostengünstiger Zusatz. Wenn du einen Muss{-}Gegenstand hast, kann ich über einen Tausch von C62 oder C08 sprechen, solange der Aufwand passt.'\}} \\
            }
        }
    }
    & \\ \\

    \theutterance \stepcounter{utterance}  
    & & & \multicolumn{4}{p{0.6\linewidth}}{
        \cellcolor[rgb]{0.9,0.9,0.9}{
            \makecell[{{p{\linewidth}}}]{
                \texttt{\tiny{[GM$\langle$P2]}}
                \texttt{ABLEHNUNG: \{'C56', 'B38', 'C38', 'A07', 'A03', 'C32', 'C00', 'C62', 'C08'\}} \\
\texttt{VORSCHLAG: \{'C56', 'B38', 'C38', 'C08', 'C03', 'A83', 'C76', 'C98', 'C10'\}} \\
\texttt{ARGUMENT: \{'Der Vorschlag liegt mit Gesamtaufwand 3524 klar unter 3683 und steigert die Wirkung pro Aufwand.', 'Kern aus schnellen, breit wirkenden Bausteinen: C08, C38, B38, C03 – frühe Effekte und Risikoreduktion.', 'A83 und C76 liefern strategischen Hebel; C98 und C10 sichern Skalierbarkeit; C56 ist ein sehr günstiger Abrunder.', 'A03, C32, C00, A07 und C62 sind vergleichsweise aufwändig bei begrenztem Zusatznutzen – der Austausch verbessert das Verhältnis.', 'Falls du einen Muss{-}Gegenstand hast, können wir über einen Tausch bei C98 oder C10 sprechen, solange das Budget eingehalten wird.'\}} \\
            }
        }
    }
     \\ \\

    \theutterance \stepcounter{utterance}  
    & & \multicolumn{4}{p{0.6\linewidth}}{
        \cellcolor[rgb]{0.9,0.9,0.9}{
            \makecell[{{p{\linewidth}}}]{
                \texttt{\tiny{[P1$\langle$GM]}}
                \texttt{ABLEHNUNG: \{'C56', 'B38', 'C38', 'A07', 'A03', 'C32', 'C00', 'C62', 'C08'\}} \\
\texttt{VORSCHLAG: \{'C56', 'B38', 'C38', 'C08', 'C03', 'A83', 'C76', 'C98', 'C10'\}} \\
\texttt{ARGUMENT: \{'Der Vorschlag liegt mit Gesamtaufwand 3524 klar unter 3683 und steigert die Wirkung pro Aufwand.', 'Kern aus schnellen, breit wirkenden Bausteinen: C08, C38, B38, C03 – frühe Effekte und Risikoreduktion.', 'A83 und C76 liefern strategischen Hebel; C98 und C10 sichern Skalierbarkeit; C56 ist ein sehr günstiger Abrunder.', 'A03, C32, C00, A07 und C62 sind vergleichsweise aufwändig bei begrenztem Zusatznutzen – der Austausch verbessert das Verhältnis.', 'Falls du einen Muss{-}Gegenstand hast, können wir über einen Tausch bei C98 oder C10 sprechen, solange das Budget eingehalten wird.'\}} \\
            }
        }
    }
    & \\ \\

    \theutterance \stepcounter{utterance}  
    & \multicolumn{4}{p{0.6\linewidth}}{
        \cellcolor[rgb]{0.9,0.9,0.9}{
            \makecell[{{p{\linewidth}}}]{
                \texttt{\tiny{[P1$\rangle$GM]}}
                \texttt{ABLEHNUNG: \{'C56', 'B38', 'C38', 'C08', 'C03', 'A83', 'C76', 'C98', 'C10'\}} \\
\texttt{VORSCHLAG: \{'C56', 'B38', 'C38', 'A07', 'A03', 'C32', 'C00', 'C10', 'C08'\}} \\
\texttt{ARGUMENT: \{'Dein Paket bindet viel Aufwand in Bausteine mit geringem Hebel (z. B. A83, C76, C98). Ich schlage vor, die starke Basis aus A03, A07, C32 und C00 zu sichern und mit schnellen Hebeln (C56, B38, C38) zu kombinieren.', 'Mein Vorschlag liegt mit Gesamtaufwand 3673 unter 3683 und enthält bereits mehrere deiner genannten Bausteine (C56, B38, C38, C08, C10).', 'Ich nehme C10 als Brücke zu deiner Skalierungs{-}Perspektive auf, ohne das Budget zu sprengen.', 'Falls C03 für dich ein Muss ist, können wir auf eine Variante mit C03 statt C38 umschwenken (dann ohne C10), die weiterhin das Budget einhält.', 'Wenn dir C98 wichtiger als C10 ist, können wir C10 gegen C98 tauschen; das bleibt klar im Budget und ist für mich verhandelbar.', 'Nenne mir 1–2 absolute Muss{-}Gegenstände, dann optimiere ich den Rest so, dass das Verhältnis aus Wirkung und Aufwand passt.'\}} \\
            }
        }
    }
    & & \\ \\

    \theutterance \stepcounter{utterance}  
    & & \multicolumn{4}{p{0.6\linewidth}}{
        \cellcolor[rgb]{0.9,0.9,0.9}{
            \makecell[{{p{\linewidth}}}]{
                \texttt{\tiny{[GM$\rangle$P2]}}
                \texttt{ABLEHNUNG: \{'C56', 'B38', 'C38', 'C08', 'C03', 'A83', 'C76', 'C98', 'C10'\}} \\
\texttt{VORSCHLAG: \{'C56', 'B38', 'C38', 'A07', 'A03', 'C32', 'C00', 'C10', 'C08'\}} \\
\texttt{ARGUMENT: \{'Dein Paket bindet viel Aufwand in Bausteine mit geringem Hebel (z. B. A83, C76, C98). Ich schlage vor, die starke Basis aus A03, A07, C32 und C00 zu sichern und mit schnellen Hebeln (C56, B38, C38) zu kombinieren.', 'Mein Vorschlag liegt mit Gesamtaufwand 3673 unter 3683 und enthält bereits mehrere deiner genannten Bausteine (C56, B38, C38, C08, C10).', 'Ich nehme C10 als Brücke zu deiner Skalierungs{-}Perspektive auf, ohne das Budget zu sprengen.', 'Falls C03 für dich ein Muss ist, können wir auf eine Variante mit C03 statt C38 umschwenken (dann ohne C10), die weiterhin das Budget einhält.', 'Wenn dir C98 wichtiger als C10 ist, können wir C10 gegen C98 tauschen; das bleibt klar im Budget und ist für mich verhandelbar.', 'Nenne mir 1–2 absolute Muss{-}Gegenstände, dann optimiere ich den Rest so, dass das Verhältnis aus Wirkung und Aufwand passt.'\}} \\
            }
        }
    }
    & \\ \\

    \theutterance \stepcounter{utterance}  
    & & & \multicolumn{4}{p{0.6\linewidth}}{
        \cellcolor[rgb]{0.9,0.9,0.9}{
            \makecell[{{p{\linewidth}}}]{
                \texttt{\tiny{[GM$\langle$P2]}}
                \texttt{ABLEHNUNG: \{'C56', 'B38', 'C38', 'A07', 'A03', 'C32', 'C00', 'C10', 'C08'\}} \\
\texttt{VORSCHLAG: \{'C56', 'B38', 'C38', 'C08', 'C03', 'A83', 'C10', 'A07', 'C76', 'A03'\}} \\
\texttt{ARGUMENT: \{'Gesamtaufwand 3668 < 3683 – wir bleiben sicher im Budget.', 'Ich übernehme deinen Kern mit A03 und A07 und behalte die schnellen Hebel C56, B38, C38 sowie C08 für frühe Wirkung.', 'C03 bringt Stabilität/Qualität in die Basis; C10 hält die Skalierung offen.', 'Ich habe C98 herausgenommen, um deiner Sorge „geringer Hebel bei hohem Aufwand“ entgegenzukommen; so bleibt der Mix fokussiert.', 'C32 und C00 sind aktuell im Verhältnis Aufwand/Nutzen zu teuer; mit dem jetzigen Paket erreichen wir mehr Wirkung pro Aufwand.', 'Falls dir C00 wichtig ist, können wir C10 {-}> C00 tauschen (Aufwand 3616). Wenn dir C98 lieber als C10 ist, tauschen wir C10 {-}> C98 (Aufwand 3543).'\}} \\
            }
        }
    }
     \\ \\

    \theutterance \stepcounter{utterance}  
    & & \multicolumn{4}{p{0.6\linewidth}}{
        \cellcolor[rgb]{0.9,0.9,0.9}{
            \makecell[{{p{\linewidth}}}]{
                \texttt{\tiny{[P1$\langle$GM]}}
                \texttt{ABLEHNUNG: \{'C56', 'B38', 'C38', 'A07', 'A03', 'C32', 'C00', 'C10', 'C08'\}} \\
\texttt{VORSCHLAG: \{'C56', 'B38', 'C38', 'C08', 'C03', 'A83', 'C10', 'A07', 'C76', 'A03'\}} \\
\texttt{ARGUMENT: \{'Gesamtaufwand 3668 < 3683 – wir bleiben sicher im Budget.', 'Ich übernehme deinen Kern mit A03 und A07 und behalte die schnellen Hebel C56, B38, C38 sowie C08 für frühe Wirkung.', 'C03 bringt Stabilität/Qualität in die Basis; C10 hält die Skalierung offen.', 'Ich habe C98 herausgenommen, um deiner Sorge „geringer Hebel bei hohem Aufwand“ entgegenzukommen; so bleibt der Mix fokussiert.', 'C32 und C00 sind aktuell im Verhältnis Aufwand/Nutzen zu teuer; mit dem jetzigen Paket erreichen wir mehr Wirkung pro Aufwand.', 'Falls dir C00 wichtig ist, können wir C10 {-}> C00 tauschen (Aufwand 3616). Wenn dir C98 lieber als C10 ist, tauschen wir C10 {-}> C98 (Aufwand 3543).'\}} \\
            }
        }
    }
    & \\ \\

    \theutterance \stepcounter{utterance}  
    & \multicolumn{4}{p{0.6\linewidth}}{
        \cellcolor[rgb]{0.9,0.9,0.9}{
            \makecell[{{p{\linewidth}}}]{
                \texttt{\tiny{[P1$\rangle$GM]}}
                \texttt{ABLEHNUNG: \{'C56', 'B38', 'C38', 'C08', 'C03', 'A83', 'C10', 'A07', 'C76', 'A03'\}} \\
\texttt{VORSCHLAG: \{'C56', 'B38', 'C38', 'C03', 'A83', 'A07', 'A03', 'C32', 'C00'\}} \\
\texttt{ARGUMENT: \{'Aufwand 3567 < 3683 – klar im Budget.', 'Ich übernehme deinen Kern (C56, B38, C38, C03) sowie A03 und A07 und halte A83 als deinen strategischen Hebel drin.', 'Im Gegenzug lassen wir C10 und C76 außen vor und investieren in C32 und C00 – das hebt die Gesamtwirkung pro Aufwand spürbar und stärkt die Basis.', 'Falls C10 für dich ein Muss ist, kann ich auf die Alternative \{C56, B38, C38, A07, A03, C32, C00, C10\} wechseln (Aufwand 3534).', 'Falls C76 ein Muss ist, tauschen wir C03 {-}> C76 (Aufwand 3389).', 'Sag mir bitte, ob C10 oder C76 dein Muss ist (A83 ist bereits enthalten), dann schließen wir ab.'\}} \\
            }
        }
    }
    & & \\ \\

    \theutterance \stepcounter{utterance}  
    & & \multicolumn{4}{p{0.6\linewidth}}{
        \cellcolor[rgb]{0.9,0.9,0.9}{
            \makecell[{{p{\linewidth}}}]{
                \texttt{\tiny{[GM$\rangle$P2]}}
                \texttt{ABLEHNUNG: \{'C56', 'B38', 'C38', 'C08', 'C03', 'A83', 'C10', 'A07', 'C76', 'A03'\}} \\
\texttt{VORSCHLAG: \{'C56', 'B38', 'C38', 'C03', 'A83', 'A07', 'A03', 'C32', 'C00'\}} \\
\texttt{ARGUMENT: \{'Aufwand 3567 < 3683 – klar im Budget.', 'Ich übernehme deinen Kern (C56, B38, C38, C03) sowie A03 und A07 und halte A83 als deinen strategischen Hebel drin.', 'Im Gegenzug lassen wir C10 und C76 außen vor und investieren in C32 und C00 – das hebt die Gesamtwirkung pro Aufwand spürbar und stärkt die Basis.', 'Falls C10 für dich ein Muss ist, kann ich auf die Alternative \{C56, B38, C38, A07, A03, C32, C00, C10\} wechseln (Aufwand 3534).', 'Falls C76 ein Muss ist, tauschen wir C03 {-}> C76 (Aufwand 3389).', 'Sag mir bitte, ob C10 oder C76 dein Muss ist (A83 ist bereits enthalten), dann schließen wir ab.'\}} \\
            }
        }
    }
    & \\ \\

    \theutterance \stepcounter{utterance}  
    & & & \multicolumn{4}{p{0.6\linewidth}}{
        \cellcolor[rgb]{0.9,0.9,0.9}{
            \makecell[{{p{\linewidth}}}]{
                \texttt{\tiny{[GM$\langle$P2]}}
                \texttt{ABLEHNUNG: \{'C56', 'B38', 'C38', 'C03', 'A83', 'A07', 'A03', 'C32', 'C00'\}} \\
\texttt{VORSCHLAG: \{'B38', 'C38', 'C03', 'A83', 'A07', 'C32', 'C00', 'C08', 'C76'\}} \\
\texttt{ARGUMENT: \{'Gesamtaufwand 3677 < 3683 – sicher im Budget.', 'Ich übernehme 7/9 deines Pakets (B38, C38, C03, A83, A07, C32, C00) und ersetze nur zwei Bausteine: C56 und A03 {-}> C08 und C76.', 'Begründung: C08 liefert deutlich mehr Wirkung pro Aufwand als C56/A03 zusammen und C76 bringt den strategischen Hebel, ohne die Basis zu schwächen.', 'Deine Kernbasis bleibt intakt (A07, C32, C00), die schnellen Hebel (B38, C38) und die Qualität/Strategie (C03, A83) bleiben ebenfalls erhalten.', 'Falls A03 für dich ein Muss ist, biete ich die Variante \{B38, C38, C03, A83, A07, A03, C00, C08, C76\} mit Aufwand 3556 an – gleiches Profil, nur C32 {-}> A03.'\}} \\
            }
        }
    }
     \\ \\

    \theutterance \stepcounter{utterance}  
    & & \multicolumn{4}{p{0.6\linewidth}}{
        \cellcolor[rgb]{0.9,0.9,0.9}{
            \makecell[{{p{\linewidth}}}]{
                \texttt{\tiny{[P1$\langle$GM]}}
                \texttt{ABLEHNUNG: \{'C56', 'B38', 'C38', 'C03', 'A83', 'A07', 'A03', 'C32', 'C00'\}} \\
\texttt{VORSCHLAG: \{'B38', 'C38', 'C03', 'A83', 'A07', 'C32', 'C00', 'C08', 'C76'\}} \\
\texttt{ARGUMENT: \{'Gesamtaufwand 3677 < 3683 – sicher im Budget.', 'Ich übernehme 7/9 deines Pakets (B38, C38, C03, A83, A07, C32, C00) und ersetze nur zwei Bausteine: C56 und A03 {-}> C08 und C76.', 'Begründung: C08 liefert deutlich mehr Wirkung pro Aufwand als C56/A03 zusammen und C76 bringt den strategischen Hebel, ohne die Basis zu schwächen.', 'Deine Kernbasis bleibt intakt (A07, C32, C00), die schnellen Hebel (B38, C38) und die Qualität/Strategie (C03, A83) bleiben ebenfalls erhalten.', 'Falls A03 für dich ein Muss ist, biete ich die Variante \{B38, C38, C03, A83, A07, A03, C00, C08, C76\} mit Aufwand 3556 an – gleiches Profil, nur C32 {-}> A03.'\}} \\
            }
        }
    }
    & \\ \\

    \theutterance \stepcounter{utterance}  
    & \multicolumn{4}{p{0.6\linewidth}}{
        \cellcolor[rgb]{0.9,0.9,0.9}{
            \makecell[{{p{\linewidth}}}]{
                \texttt{\tiny{[P1$\rangle$GM]}}
                \texttt{ABLEHNUNG: \{'B38', 'C38', 'C03', 'A83', 'A07', 'C32', 'C00', 'C08', 'C76'\}} \\
\texttt{VORSCHLAG: \{'B38', 'C38', 'C03', 'A83', 'A07', 'C32', 'C00', 'C56', 'A03'\}} \\
\texttt{ARGUMENT: \{'Aufwand 3567 < 3683 – klar im Budget und mit besserem Wirkung/Aufwand{-}Verhältnis.', 'Ich übernehme 7/9 deines Pakets (B38, C38, C03, A83, A07, C32, C00) und ersetze nur C08 und C76 durch C56 (Quick Win, sehr geringer Aufwand) und A03 (starker Kernbaustein).', 'Damit senken wir den Aufwand um 110 gegenüber deinem Vorschlag und stärken gleichzeitig frühe Wirkung und Basis.', 'Falls C76 für dich ein Muss ist: Variante \{B38, C38, C03, A83, A07, C32, C00, C56, C76\} mit Aufwand 3598.', 'Falls C08 für dich ein Muss ist: Variante \{B38, C38, C03, A83, A07, C32, C00, C08, A03\} mit Aufwand 3646.', 'Sag mir bitte, ob C76 oder C08 dein Muss ist – dann fixieren wir final.'\}} \\
            }
        }
    }
    & & \\ \\

    \theutterance \stepcounter{utterance}  
    & & \multicolumn{4}{p{0.6\linewidth}}{
        \cellcolor[rgb]{0.9,0.9,0.9}{
            \makecell[{{p{\linewidth}}}]{
                \texttt{\tiny{[GM$\rangle$P2]}}
                \texttt{ABLEHNUNG: \{'B38', 'C38', 'C03', 'A83', 'A07', 'C32', 'C00', 'C08', 'C76'\}} \\
\texttt{VORSCHLAG: \{'B38', 'C38', 'C03', 'A83', 'A07', 'C32', 'C00', 'C56', 'A03'\}} \\
\texttt{ARGUMENT: \{'Aufwand 3567 < 3683 – klar im Budget und mit besserem Wirkung/Aufwand{-}Verhältnis.', 'Ich übernehme 7/9 deines Pakets (B38, C38, C03, A83, A07, C32, C00) und ersetze nur C08 und C76 durch C56 (Quick Win, sehr geringer Aufwand) und A03 (starker Kernbaustein).', 'Damit senken wir den Aufwand um 110 gegenüber deinem Vorschlag und stärken gleichzeitig frühe Wirkung und Basis.', 'Falls C76 für dich ein Muss ist: Variante \{B38, C38, C03, A83, A07, C32, C00, C56, C76\} mit Aufwand 3598.', 'Falls C08 für dich ein Muss ist: Variante \{B38, C38, C03, A83, A07, C32, C00, C08, A03\} mit Aufwand 3646.', 'Sag mir bitte, ob C76 oder C08 dein Muss ist – dann fixieren wir final.'\}} \\
            }
        }
    }
    & \\ \\

    \theutterance \stepcounter{utterance}  
    & & & \multicolumn{4}{p{0.6\linewidth}}{
        \cellcolor[rgb]{0.9,0.9,0.9}{
            \makecell[{{p{\linewidth}}}]{
                \texttt{\tiny{[GM$\langle$P2]}}
                \texttt{ABLEHNUNG: \{'B38', 'C38', 'C03', 'A83', 'A07', 'C32', 'C00', 'C56', 'A03'\}} \\
\texttt{VORSCHLAG: \{'B38', 'C38', 'C03', 'A83', 'A07', 'C32', 'C00', 'C08', 'A03'\}} \\
\texttt{ARGUMENT: \{'Aufwand 3646 < 3683 – klar im Budget.', 'Das ist deine eigene Alternativvariante (Tausch C56 {-}> C08) aus deiner letzten Nachricht: Basis A03, A07, C32, C00 bleibt unverändert, ebenso B38, C38, C03 und A83.', 'C08 liefert gegenüber C56 deutlich mehr frühen Hebel bei nur +79 Aufwand und ohne andere Bausteine zu gefährden.', 'Das ist ein ausgewogener Abschlussvorschlag – ich bin bereit, hier zu schließen.', 'Falls du stattdessen C76 setzen willst, können wir C32 {-}> C76 tauschen (Aufwand 3556) – ähnliches Profil, weiterhin im Budget.'\}} \\
            }
        }
    }
     \\ \\

    \theutterance \stepcounter{utterance}  
    & & \multicolumn{4}{p{0.6\linewidth}}{
        \cellcolor[rgb]{0.9,0.9,0.9}{
            \makecell[{{p{\linewidth}}}]{
                \texttt{\tiny{[P1$\langle$GM]}}
                \texttt{ABLEHNUNG: \{'B38', 'C38', 'C03', 'A83', 'A07', 'C32', 'C00', 'C56', 'A03'\}} \\
\texttt{VORSCHLAG: \{'B38', 'C38', 'C03', 'A83', 'A07', 'C32', 'C00', 'C08', 'A03'\}} \\
\texttt{ARGUMENT: \{'Aufwand 3646 < 3683 – klar im Budget.', 'Das ist deine eigene Alternativvariante (Tausch C56 {-}> C08) aus deiner letzten Nachricht: Basis A03, A07, C32, C00 bleibt unverändert, ebenso B38, C38, C03 und A83.', 'C08 liefert gegenüber C56 deutlich mehr frühen Hebel bei nur +79 Aufwand und ohne andere Bausteine zu gefährden.', 'Das ist ein ausgewogener Abschlussvorschlag – ich bin bereit, hier zu schließen.', 'Falls du stattdessen C76 setzen willst, können wir C32 {-}> C76 tauschen (Aufwand 3556) – ähnliches Profil, weiterhin im Budget.'\}} \\
            }
        }
    }
    & \\ \\

    \theutterance \stepcounter{utterance}  
    & \multicolumn{4}{p{0.6\linewidth}}{
        \cellcolor[rgb]{0.9,0.9,0.9}{
            \makecell[{{p{\linewidth}}}]{
                \texttt{\tiny{[P1$\rangle$GM]}}
                \texttt{ABLEHNUNG: \{'B38', 'C38', 'C03', 'A83', 'A07', 'C32', 'C00', 'C08', 'A03'\}} \\
\texttt{VORSCHLAG: \{'B38', 'C38', 'C03', 'A83', 'A07', 'C32', 'C00', 'C56', 'A03'\}} \\
\texttt{ARGUMENT: \{'Minimaler 1:1‑Tausch: C08 {-}> C56. Aufwand sinkt auf 3567 (−79 ggü. deinem Vorschlag) und bleibt klar < 3683.', 'Profil bleibt identisch: Deine Strategie (A83) und Basis (A03, A07, C32, C00) sowie die schnellen Hebel (B38, C38, C03) bleiben vollständig erhalten.', 'C56 ist der schnellere, günstigere Quick Win (du hast ihn selbst mehrfach als „sehr günstigen Abrunder“ benannt) – damit haben wir früher Wirkung ohne Budgetdruck.', 'Wenn C08 für dich ein Muss ist, sag es – dann finden wir einen gleichwertigen Tausch an anderer Stelle. Ansonsten lass uns mit dieser minimalen, effizienten Anpassung schließen.'\}} \\
            }
        }
    }
    & & \\ \\

    \theutterance \stepcounter{utterance}  
    & & \multicolumn{4}{p{0.6\linewidth}}{
        \cellcolor[rgb]{0.9,0.9,0.9}{
            \makecell[{{p{\linewidth}}}]{
                \texttt{\tiny{[GM$\rangle$P2]}}
                \texttt{ABLEHNUNG: \{'B38', 'C38', 'C03', 'A83', 'A07', 'C32', 'C00', 'C08', 'A03'\}} \\
\texttt{VORSCHLAG: \{'B38', 'C38', 'C03', 'A83', 'A07', 'C32', 'C00', 'C56', 'A03'\}} \\
\texttt{ARGUMENT: \{'Minimaler 1:1‑Tausch: C08 {-}> C56. Aufwand sinkt auf 3567 (−79 ggü. deinem Vorschlag) und bleibt klar < 3683.', 'Profil bleibt identisch: Deine Strategie (A83) und Basis (A03, A07, C32, C00) sowie die schnellen Hebel (B38, C38, C03) bleiben vollständig erhalten.', 'C56 ist der schnellere, günstigere Quick Win (du hast ihn selbst mehrfach als „sehr günstigen Abrunder“ benannt) – damit haben wir früher Wirkung ohne Budgetdruck.', 'Wenn C08 für dich ein Muss ist, sag es – dann finden wir einen gleichwertigen Tausch an anderer Stelle. Ansonsten lass uns mit dieser minimalen, effizienten Anpassung schließen.'\}} \\
            }
        }
    }
    & \\ \\

    \theutterance \stepcounter{utterance}  
    & & & \multicolumn{4}{p{0.6\linewidth}}{
        \cellcolor[rgb]{0.9,0.9,0.9}{
            \makecell[{{p{\linewidth}}}]{
                \texttt{\tiny{[GM$\langle$P2]}}
                \texttt{ABLEHNUNG: \{'B38', 'C38', 'C03', 'A83', 'A07', 'C32', 'C00', 'C56', 'A03'\}} \\
\texttt{VORSCHLAG: \{'B38', 'C38', 'C03', 'A83', 'A07', 'C32', 'C98', 'A03', 'C08'\}} \\
\texttt{ARGUMENT: \{'Gesamtaufwand 3573 < 3683 – nur +6 gegenüber deinem Paket, also praktisch gleiches Budget.', 'Ich übernehme deine Basis fast unverändert (A03, A07, C32) plus B38, C38, C03, A83. Minimaler, ausgleichender Doppeltausch: C56 {-}> C08 für stärkere frühe Wirkung und C00 {-}> C98 als effizientere, belastbare Alternative mit geringerem Aufwand.', 'Damit bleibt dein Profil erhalten (Basis + schnelle Hebel), nur mit spürbar mehr Wirkung pro Aufwand und praktisch ohne Budget{-}Impact.', 'Falls C00 für dich ein Muss ist, biete ich als kompensierende Alternative: C32 {-}> C76 bei Beibehaltung von C08 (Aufwand dann 3556) – gleiches Profil, noch unter Budget.'\}} \\
            }
        }
    }
     \\ \\

    \theutterance \stepcounter{utterance}  
    & & \multicolumn{4}{p{0.6\linewidth}}{
        \cellcolor[rgb]{0.9,0.9,0.9}{
            \makecell[{{p{\linewidth}}}]{
                \texttt{\tiny{[P1$\langle$GM]}}
                \texttt{ABLEHNUNG: \{'B38', 'C38', 'C03', 'A83', 'A07', 'C32', 'C00', 'C56', 'A03'\}} \\
\texttt{VORSCHLAG: \{'B38', 'C38', 'C03', 'A83', 'A07', 'C32', 'C98', 'A03', 'C08'\}} \\
\texttt{ARGUMENT: \{'Gesamtaufwand 3573 < 3683 – nur +6 gegenüber deinem Paket, also praktisch gleiches Budget.', 'Ich übernehme deine Basis fast unverändert (A03, A07, C32) plus B38, C38, C03, A83. Minimaler, ausgleichender Doppeltausch: C56 {-}> C08 für stärkere frühe Wirkung und C00 {-}> C98 als effizientere, belastbare Alternative mit geringerem Aufwand.', 'Damit bleibt dein Profil erhalten (Basis + schnelle Hebel), nur mit spürbar mehr Wirkung pro Aufwand und praktisch ohne Budget{-}Impact.', 'Falls C00 für dich ein Muss ist, biete ich als kompensierende Alternative: C32 {-}> C76 bei Beibehaltung von C08 (Aufwand dann 3556) – gleiches Profil, noch unter Budget.'\}} \\
            }
        }
    }
    & \\ \\

    \theutterance \stepcounter{utterance}  
    & \multicolumn{4}{p{0.6\linewidth}}{
        \cellcolor[rgb]{0.9,0.9,0.9}{
            \makecell[{{p{\linewidth}}}]{
                \texttt{\tiny{[P1$\rangle$GM]}}
                \texttt{ABLEHNUNG: \{'B38', 'C38', 'C03', 'A83', 'A07', 'C32', 'C98', 'A03', 'C08'\}} \\
\texttt{VORSCHLAG: \{'B38', 'C38', 'C03', 'A83', 'A07', 'C32', 'A03', 'C08', 'C56', 'C62'\}} \\
\texttt{ARGUMENT: \{'Minimaler, zielgerichteter Tausch: C98 {-}> C56 + C62 bei Erhalt deines gesamten Kerns.', 'Aufwand 3587 < 3683 (nur +14 ggü. deinem Vorschlag 3573) – sicher im Budget.', 'C56 ist ein extrem schneller Quick Win (frühe Effekte, praktisch ohne Aufwand); C62 stärkt die Basis robuster und effizienter als das vergleichsweise schwergewichtige C98.', 'Wir lassen A83, A07, C32, A03 sowie die schnellen Hebel B38, C38 und die Qualitätsstütze C03 unverändert – dein Profil bleibt erhalten, nur mit mehr frühem Hebel und Stabilität.', 'Falls dir C62 zu viel ist bzw. du C76 bevorzugst: Alternative \{B38, C38, C03, A83, A07, C32, A03, C08, C56, C76\} mit Aufwand 3474.', 'Falls du strikt bei neun Bausteinen bleiben willst: Alternative \{B38, C38, C03, A83, A07, C32, A03, C56, C00\} mit Aufwand 3567 (C08 {-}> C56, C98 {-}> C00).'\}} \\
            }
        }
    }
    & & \\ \\

    \theutterance \stepcounter{utterance}  
    & & \multicolumn{4}{p{0.6\linewidth}}{
        \cellcolor[rgb]{0.9,0.9,0.9}{
            \makecell[{{p{\linewidth}}}]{
                \texttt{\tiny{[GM$\rangle$P2]}}
                \texttt{ABLEHNUNG: \{'B38', 'C38', 'C03', 'A83', 'A07', 'C32', 'C98', 'A03', 'C08'\}} \\
\texttt{VORSCHLAG: \{'B38', 'C38', 'C03', 'A83', 'A07', 'C32', 'A03', 'C08', 'C56', 'C62'\}} \\
\texttt{ARGUMENT: \{'Minimaler, zielgerichteter Tausch: C98 {-}> C56 + C62 bei Erhalt deines gesamten Kerns.', 'Aufwand 3587 < 3683 (nur +14 ggü. deinem Vorschlag 3573) – sicher im Budget.', 'C56 ist ein extrem schneller Quick Win (frühe Effekte, praktisch ohne Aufwand); C62 stärkt die Basis robuster und effizienter als das vergleichsweise schwergewichtige C98.', 'Wir lassen A83, A07, C32, A03 sowie die schnellen Hebel B38, C38 und die Qualitätsstütze C03 unverändert – dein Profil bleibt erhalten, nur mit mehr frühem Hebel und Stabilität.', 'Falls dir C62 zu viel ist bzw. du C76 bevorzugst: Alternative \{B38, C38, C03, A83, A07, C32, A03, C08, C56, C76\} mit Aufwand 3474.', 'Falls du strikt bei neun Bausteinen bleiben willst: Alternative \{B38, C38, C03, A83, A07, C32, A03, C56, C00\} mit Aufwand 3567 (C08 {-}> C56, C98 {-}> C00).'\}} \\
            }
        }
    }
    & \\ \\

    \theutterance \stepcounter{utterance}  
    & & & \multicolumn{4}{p{0.6\linewidth}}{
        \cellcolor[rgb]{0.9,0.9,0.9}{
            \makecell[{{p{\linewidth}}}]{
                \texttt{\tiny{[GM$\langle$P2]}}
                \texttt{ABLEHNUNG: \{'B38', 'C38', 'C03', 'A83', 'A07', 'C32', 'A03', 'C08', 'C56', 'C62'\}} \\
\texttt{VORSCHLAG: \{'B38', 'C38', 'C03', 'A83', 'A07', 'C32', 'A03', 'C08', 'C56', 'C76'\}} \\
\texttt{ARGUMENT: \{'Minimaler 1:1‑Tausch: C62 {-}> C76; alles andere bleibt unverändert.', 'Gesamtaufwand sinkt auf 3474 (< 3683) – mehr Puffer ohne Profiländerung.', 'Dein Kern (A03, A07, C32) sowie die schnellen Hebel (B38, C38, C56, C08) und Qualitätsanker (C03, A83) bleiben vollständig erhalten.', 'C76 liefert den strategischen Hebel, den du bei Bedarf bereits als Option offen gelassen hast – bei geringerem Aufwand als C62.', 'Ich bin bereit, hier zu schließen. Falls C62 für dich ein Muss ist, sag es – dann justieren wir an einer anderen Stelle gleichwertig, ohne das Budget zu reißen.'\}} \\
            }
        }
    }
     \\ \\

    \theutterance \stepcounter{utterance}  
    & & \multicolumn{4}{p{0.6\linewidth}}{
        \cellcolor[rgb]{0.9,0.9,0.9}{
            \makecell[{{p{\linewidth}}}]{
                \texttt{\tiny{[P1$\langle$GM]}}
                \texttt{ABLEHNUNG: \{'B38', 'C38', 'C03', 'A83', 'A07', 'C32', 'A03', 'C08', 'C56', 'C62'\}} \\
\texttt{VORSCHLAG: \{'B38', 'C38', 'C03', 'A83', 'A07', 'C32', 'A03', 'C08', 'C56', 'C76'\}} \\
\texttt{ARGUMENT: \{'Minimaler 1:1‑Tausch: C62 {-}> C76; alles andere bleibt unverändert.', 'Gesamtaufwand sinkt auf 3474 (< 3683) – mehr Puffer ohne Profiländerung.', 'Dein Kern (A03, A07, C32) sowie die schnellen Hebel (B38, C38, C56, C08) und Qualitätsanker (C03, A83) bleiben vollständig erhalten.', 'C76 liefert den strategischen Hebel, den du bei Bedarf bereits als Option offen gelassen hast – bei geringerem Aufwand als C62.', 'Ich bin bereit, hier zu schließen. Falls C62 für dich ein Muss ist, sag es – dann justieren wir an einer anderen Stelle gleichwertig, ohne das Budget zu reißen.'\}} \\
            }
        }
    }
    & \\ \\

    \theutterance \stepcounter{utterance}  
    & \multicolumn{4}{p{0.6\linewidth}}{
        \cellcolor[rgb]{0.9,0.9,0.9}{
            \makecell[{{p{\linewidth}}}]{
                \texttt{\tiny{[P1$\rangle$GM]}}
                \texttt{ABLEHNUNG: \{'B38', 'C38', 'C03', 'A83', 'A07', 'C32', 'A03', 'C08', 'C56', 'C76'\}} \\
\texttt{VORSCHLAG: \{'B38', 'C38', 'C03', 'A83', 'A07', 'C32', 'A03', 'C08', 'C56', 'C62'\}} \\
\texttt{ARGUMENT: \{'Minimaler 1:1‑Tausch: C76 {-}> C62; alles andere bleibt unverändert.', 'Gesamtaufwand 3587 < 3683 – weiter sicher im Budget.', 'C62 stärkt die Basis deutlich robuster als C76, ohne Profilverlust: Dein Kern (A03, A07, C32) bleibt, schnelle Hebel (B38, C38, C56, C08) und Qualitätsanker (C03, A83) bleiben vollständig erhalten.', 'Du hattest C62 zuvor selbst als valide Option im Paket – ich gehe damit maximal auf dich zu und biete hier einen fairen Abschluss.', 'Ich bin bereit, hier zu schließen. Wenn C76 für dich zwingend ist, sag es explizit – dann prüfen wir einen gleichwertigen Ausgleich an anderer Stelle.'\}} \\
            }
        }
    }
    & & \\ \\

    \theutterance \stepcounter{utterance}  
    & & \multicolumn{4}{p{0.6\linewidth}}{
        \cellcolor[rgb]{0.9,0.9,0.9}{
            \makecell[{{p{\linewidth}}}]{
                \texttt{\tiny{[GM$\rangle$P2]}}
                \texttt{ABLEHNUNG: \{'B38', 'C38', 'C03', 'A83', 'A07', 'C32', 'A03', 'C08', 'C56', 'C76'\}} \\
\texttt{VORSCHLAG: \{'B38', 'C38', 'C03', 'A83', 'A07', 'C32', 'A03', 'C08', 'C56', 'C62'\}} \\
\texttt{ARGUMENT: \{'Minimaler 1:1‑Tausch: C76 {-}> C62; alles andere bleibt unverändert.', 'Gesamtaufwand 3587 < 3683 – weiter sicher im Budget.', 'C62 stärkt die Basis deutlich robuster als C76, ohne Profilverlust: Dein Kern (A03, A07, C32) bleibt, schnelle Hebel (B38, C38, C56, C08) und Qualitätsanker (C03, A83) bleiben vollständig erhalten.', 'Du hattest C62 zuvor selbst als valide Option im Paket – ich gehe damit maximal auf dich zu und biete hier einen fairen Abschluss.', 'Ich bin bereit, hier zu schließen. Wenn C76 für dich zwingend ist, sag es explizit – dann prüfen wir einen gleichwertigen Ausgleich an anderer Stelle.'\}} \\
            }
        }
    }
    & \\ \\

    \theutterance \stepcounter{utterance}  
    & & & \multicolumn{4}{p{0.6\linewidth}}{
        \cellcolor[rgb]{0.9,0.9,0.9}{
            \makecell[{{p{\linewidth}}}]{
                \texttt{\tiny{[GM$\langle$P2]}}
                \texttt{ZUSTIMMUNG: \{'B38', 'C38', 'C03', 'A83', 'A07', 'C32', 'A03', 'C08', 'C56', 'C62'\}} \\
\texttt{ARGUMENT: \{'Einverstanden. Gesamtaufwand 3587 < 3683 – passt.', 'Profil bleibt stabil (frühe Hebel + robuste Basis) und wir schließen hier ab.'\}} \\
            }
        }
    }
     \\ \\

    \theutterance \stepcounter{utterance}  
    & & & \multicolumn{2}{p{0.3\linewidth}}{
        \cellcolor[rgb]{0.9,0.9,0.9}{
            \makecell[{{p{\linewidth}}}]{
                \texttt{\tiny{[GM$|$GM]}}
                \texttt{game successful} \\
            }
        }
    }
    & & \\ \\

    \theutterance \stepcounter{utterance}  
    & & & \multicolumn{2}{p{0.3\linewidth}}{
        \cellcolor[rgb]{0.9,0.9,0.9}{
            \makecell[{{p{\linewidth}}}]{
                \texttt{\tiny{[GM$|$GM]}}
                \texttt{end game} \\
            }
        }
    }
    & & \\ \\

\end{supertabular}
}

\end{document}
