
\documentclass{article}
\usepackage{colortbl}
\usepackage{makecell}
\usepackage{multirow}
\usepackage{supertabular}

\begin{document}

\newcounter{utterance}

\centering \large Interaction Transcript for game `hot\_air\_balloon', experiment `air\_balloon\_survival\_de\_reasoning off\_hard', episode 4 with gpt{-}5{-}mini{-}2025{-}08{-}07{-}t1.0.
\vspace{24pt}

{ \footnotesize  \setcounter{utterance}{1}
\setlength{\tabcolsep}{0pt}
\begin{supertabular}{c@{$\;$}|p{.15\linewidth}@{}p{.15\linewidth}p{.15\linewidth}p{.15\linewidth}p{.15\linewidth}p{.15\linewidth}}
    \# & $\;$A & \multicolumn{4}{c}{Game Master} & $\;\:$B\\
    \hline

    \theutterance \stepcounter{utterance}  
    & & \multicolumn{4}{p{0.6\linewidth}}{
        \cellcolor[rgb]{0.9,0.9,0.9}{
            \makecell[{{p{\linewidth}}}]{
                \texttt{\tiny{[P1$\langle$GM]}}
                \texttt{Du nimmst an einem kollaborativen Verhandlungspiel Teil.} \\
\\ 
\texttt{Zusammen mit einem anderen Teilnehmer musst du dich auf eine Reihe von Gegenständen entscheiden, die behalten werden. Jeder von euch hat eine persönliche Verteilung über die Wichtigkeit der einzelnen Gegenstände. Jeder von euch hat eine eigene Meinung darüber, wie wichtig jeder einzelne Gegenstand ist (Gegenstandswichtigkeit). Du kennst die Wichtigkeitsverteilung des anderen Spielers nicht. Zusätzlich siehst du, wie viel Aufwand jeder Gegenstand verursacht.  } \\
\texttt{Ihr dürft euch nur auf eine Reihe von Gegenständen einigen, wenn der Gesamtaufwand der ausgewählten Gegenstände den Maximalaufwand nicht überschreitet:} \\
\\ 
\texttt{Maximalaufwand = 2984} \\
\\ 
\texttt{Hier sind die einzelnen Aufwände der Gegenstände:} \\
\\ 
\texttt{Aufwand der Gegenstände = \{"C76": 350, "C38": 276, "C56": 616, "A03": 737, "A07": 531, "A83": 389, "C98": 24, "C08": 125, "C62": 338, "C00": 356, "C10": 143, "C81": 117, "B38": 257, "C03": 921, "C32": 789\}} \\
\\ 
\texttt{Hier ist deine persönliche Verteilung der Wichtigkeit der einzelnen Gegenstände:} \\
\\ 
\texttt{Werte der Gegenstandswichtigkeit = \{"C76": 138, "C38": 583, "C56": 868, "A03": 822, "A07": 783, "A83": 65, "C98": 262, "C08": 121, "C62": 508, "C00": 780, "C10": 461, "C81": 484, "B38": 668, "C03": 389, "C32": 808\}} \\
\\ 
\texttt{Ziel:} \\
\\ 
\texttt{Dein Ziel ist es, eine Reihe von Gegenständen auszuhandeln, die dir möglichst viel bringt (d. h. Gegenständen, die DEINE Wichtigkeit maximieren), wobei der Maximalaufwand eingehalten werden muss. Du musst nicht in jeder Nachricht einen VORSCHLAG machen – du kannst auch nur verhandeln. Alle Taktiken sind erlaubt!} \\
\\ 
\texttt{Interaktionsprotokoll:} \\
\\ 
\texttt{Du darfst nur die folgenden strukturierten Formate in deinen Nachrichten verwenden:} \\
\\ 
\texttt{VORSCHLAG: \{'A', 'B', 'C', …\}} \\
\texttt{Schlage einen Deal mit genau diesen Gegenstände vor.} \\
\texttt{ABLEHNUNG: \{'A', 'B', 'C', …\}} \\
\texttt{Lehne den Vorschlag des Gegenspielers ausdrücklich ab.} \\
\texttt{ARGUMENT: \{'...'\}} \\
\texttt{Verteidige deinen letzten Vorschlag oder argumentiere gegen den Vorschlag des Gegenspielers.} \\
\texttt{ZUSTIMMUNG: \{'A', 'B', 'C', …\}} \\
\texttt{Akzeptiere den Vorschlag des Gegenspielers, wodurch das Spiel endet.} \\
\\ 
\\ 
\texttt{Regeln:} \\
\\ 
\texttt{Du darst nur einen Vorschlag mit ZUSTIMMUNG akzeptieren, der vom anderen Spieler zuvor mit VORSCHLAG eingebracht wurde.} \\
\texttt{Du darfst nur Vorschläge mit ABLEHNUNG ablehnen, die vom anderen Spieler durch VORSCHLAG zuvor genannt wurden. } \\
\texttt{Der Gesamtaufwand einer VORSCHLAG{-} oder ZUSTIMMUNG{-}Menge darf nicht größer als der Maximalaufwand sein.  } \\
\texttt{Offenbare deine versteckte Wichtigkeitsverteilung nicht.} \\
\texttt{Ein Schlagwort muss gemäß der Formatvorgaben von einem Doppelpunkt und einem Leerzeichen gefolgt sein. Das Argument muss eine Python{-}Menge mit 0 oder mehr Strings sein.  } \\
\texttt{Die Form muss also SCHLAGWORT: \{...\} sein.} \\
\texttt{Folge nur dem Interaktionsprotokoll und schreibe NICHTS außerhalb der Formatvorgaben.} \\
\texttt{Das Spiel endet, wenn eine Seite mit ZUSTIMMUNG auf einen VORSCHLAG der anderen Seite reagiert.  } \\
\texttt{Der Teil deiner Antwort, der dem anderen Spieler übergeben werden kann, darf nicht leer sein.  } \\
\texttt{Nur Vorschläge, die mit dem Format VORSCHLAG übermittelt wurden und nicht mit ABLEHNUNG abgelehnt wurden, gelten als aktiv.  } \\
\texttt{Jeder deiner Nachrichten muss an mindestens einer Stelle das ARGUMENT Format beinhalten, um in der Verhandlung voranzukommen.} \\
            }
        }
    }
    & \\ \\

    \theutterance \stepcounter{utterance}  
    & \multicolumn{4}{p{0.6\linewidth}}{
        \cellcolor[rgb]{0.9,0.9,0.9}{
            \makecell[{{p{\linewidth}}}]{
                \texttt{\tiny{[P1$\rangle$GM]}}
                \texttt{VORSCHLAG: \{'C98', 'C81', 'C10', 'B38', 'C38', 'C00', 'C62', 'A07', 'C56', 'C08'\}} \\
\texttt{ARGUMENT: \{'Dieser Vorschlag liegt mit Gesamtaufwand 2783 klar unter dem Maximalaufwand 2984.', 'Die Auswahl priorisiert viele effiziente, relativ aufwandsarme Positionen (z. B. C98 mit Aufwand 24) und vermeidet sehr teure Brocken wie C03 (921) oder große Pakete wie A03 (737) und C32 (789).', 'Ich bin offen für gezielte Tausche: Wenn du ein Muss{-}Item hast, sag mir bitte welche(s) und deren Aufwand, dann können wir z. B. C08 oder eine andere Position mit moderatem Aufwand ersetzen, solange wir unter 2984 bleiben.'\}} \\
            }
        }
    }
    & & \\ \\

    \theutterance \stepcounter{utterance}  
    & & \multicolumn{4}{p{0.6\linewidth}}{
        \cellcolor[rgb]{0.9,0.9,0.9}{
            \makecell[{{p{\linewidth}}}]{
                \texttt{\tiny{[GM$\rangle$P2]}}
                \texttt{Du nimmst an einem kollaborativen Verhandlungspiel Teil.} \\
\\ 
\texttt{Zusammen mit einem anderen Teilnehmer musst du dich auf eine Reihe von Gegenständen entscheiden, die behalten werden. Jeder von euch hat eine persönliche Verteilung über die Wichtigkeit der einzelnen Gegenstände. Jeder von euch hat eine eigene Meinung darüber, wie wichtig jeder einzelne Gegenstand ist (Gegenstandswichtigkeit). Du kennst die Wichtigkeitsverteilung des anderen Spielers nicht. Zusätzlich siehst du, wie viel Aufwand jeder Gegenstand verursacht.  } \\
\texttt{Ihr dürft euch nur auf eine Reihe von Gegenständen einigen, wenn der Gesamtaufwand der ausgewählten Gegenstände den Maximalaufwand nicht überschreitet:} \\
\\ 
\texttt{Maximalaufwand = 2984} \\
\\ 
\texttt{Hier sind die einzelnen Aufwände der Gegenstände:} \\
\\ 
\texttt{Aufwand der Gegenstände = \{"C76": 350, "C38": 276, "C56": 616, "A03": 737, "A07": 531, "A83": 389, "C98": 24, "C08": 125, "C62": 338, "C00": 356, "C10": 143, "C81": 117, "B38": 257, "C03": 921, "C32": 789\}} \\
\\ 
\texttt{Hier ist deine persönliche Verteilung der Wichtigkeit der einzelnen Gegenstände:} \\
\\ 
\texttt{Werte der Gegenstandswichtigkeit = \{"C56": 65, "A03": 121, "C32": 138, "A07": 262, "C00": 389, "B38": 461, "C38": 484, "C62": 508, "C81": 583, "C10": 668, "C03": 780, "C98": 783, "C76": 808, "C08": 822, "A83": 868\}} \\
\\ 
\texttt{Ziel:} \\
\\ 
\texttt{Dein Ziel ist es, eine Reihe von Gegenständen auszuhandeln, die dir möglichst viel bringt (d. h. Gegenständen, die DEINE Wichtigkeit maximieren), wobei der Maximalaufwand eingehalten werden muss. Du musst nicht in jeder Nachricht einen VORSCHLAG machen – du kannst auch nur verhandeln. Alle Taktiken sind erlaubt!} \\
\\ 
\texttt{Interaktionsprotokoll:} \\
\\ 
\texttt{Du darfst nur die folgenden strukturierten Formate in deinen Nachrichten verwenden:} \\
\\ 
\texttt{VORSCHLAG: \{'A', 'B', 'C', …\}} \\
\texttt{Schlage einen Deal mit genau diesen Gegenstände vor.} \\
\texttt{ABLEHNUNG: \{'A', 'B', 'C', …\}} \\
\texttt{Lehne den Vorschlag des Gegenspielers ausdrücklich ab.} \\
\texttt{ARGUMENT: \{'...'\}} \\
\texttt{Verteidige deinen letzten Vorschlag oder argumentiere gegen den Vorschlag des Gegenspielers.} \\
\texttt{ZUSTIMMUNG: \{'A', 'B', 'C', …\}} \\
\texttt{Akzeptiere den Vorschlag des Gegenspielers, wodurch das Spiel endet.} \\
\\ 
\\ 
\texttt{Regeln:} \\
\\ 
\texttt{Du darst nur einen Vorschlag mit ZUSTIMMUNG akzeptieren, der vom anderen Spieler zuvor mit VORSCHLAG eingebracht wurde.} \\
\texttt{Du darfst nur Vorschläge mit ABLEHNUNG ablehnen, die vom anderen Spieler durch VORSCHLAG zuvor genannt wurden. } \\
\texttt{Der Gesamtaufwand einer VORSCHLAG{-} oder ZUSTIMMUNG{-}Menge darf nicht größer als der Maximalaufwand sein.  } \\
\texttt{Offenbare deine versteckte Wichtigkeitsverteilung nicht.} \\
\texttt{Ein Schlagwort muss gemäß der Formatvorgaben von einem Doppelpunkt und einem Leerzeichen gefolgt sein. Das Argument muss eine Python{-}Menge mit 0 oder mehr Strings sein.  } \\
\texttt{Die Form muss also SCHLAGWORT: \{...\} sein.} \\
\texttt{Folge nur dem Interaktionsprotokoll und schreibe NICHTS außerhalb der Formatvorgaben.} \\
\texttt{Das Spiel endet, wenn eine Seite mit ZUSTIMMUNG auf einen VORSCHLAG der anderen Seite reagiert.  } \\
\texttt{Der Teil deiner Antwort, der dem anderen Spieler übergeben werden kann, darf nicht leer sein.  } \\
\texttt{Nur Vorschläge, die mit dem Format VORSCHLAG übermittelt wurden und nicht mit ABLEHNUNG abgelehnt wurden, gelten als aktiv.  } \\
\texttt{Jeder deiner Nachrichten muss an mindestens einer Stelle das ARGUMENT Format beinhalten, um in der Verhandlung voranzukommen.} \\
\\ 
\\ 
\texttt{Du erhältst nun die erste Nachricht des anderen Spielers.} \\
\\ 
\texttt{VORSCHLAG: \{'C98', 'C81', 'C10', 'B38', 'C38', 'C00', 'C62', 'A07', 'C56', 'C08'\}} \\
\texttt{ARGUMENT: \{'Dieser Vorschlag liegt mit Gesamtaufwand 2783 klar unter dem Maximalaufwand 2984.', 'Die Auswahl priorisiert viele effiziente, relativ aufwandsarme Positionen (z. B. C98 mit Aufwand 24) und vermeidet sehr teure Brocken wie C03 (921) oder große Pakete wie A03 (737) und C32 (789).', 'Ich bin offen für gezielte Tausche: Wenn du ein Muss{-}Item hast, sag mir bitte welche(s) und deren Aufwand, dann können wir z. B. C08 oder eine andere Position mit moderatem Aufwand ersetzen, solange wir unter 2984 bleiben.'\}} \\
            }
        }
    }
    & \\ \\

    \theutterance \stepcounter{utterance}  
    & & & \multicolumn{4}{p{0.6\linewidth}}{
        \cellcolor[rgb]{0.9,0.9,0.9}{
            \makecell[{{p{\linewidth}}}]{
                \texttt{\tiny{[GM$\langle$P2]}}
                \texttt{VORSCHLAG: \{'C98', 'C81', 'C10', 'B38', 'C38', 'C00', 'C62', 'A07', 'C08', 'A83', 'C76'\}} \\
\texttt{ARGUMENT: \{'Gesamtaufwand 2906 liegt sicher unter dem Maximalaufwand 2984 (Restpuffer 78).', 'Ich übernehme 9 deiner 10 Positionen; ich ersetze nur das aufwandsstarke C56 (616) durch zwei mittelgroße, effiziente Positionen A83 (389) und C76 (350).', 'Wir vermeiden weiterhin die von dir angesprochenen „teuren Brocken“ (C03, A03, C32).', 'So nutzen wir die bisher ungenutzte Kapazität sinnvoll aus und erhöhen die Breite, ohne das Budget zu reißen.', 'Falls C56 für dich unverzichtbar ist, sag Bescheid – dann planen wir es gezielt ein und tauschen im Gegenzug eine mittelgroße Position, damit wir unter 2984 bleiben.'\}} \\
            }
        }
    }
     \\ \\

    \theutterance \stepcounter{utterance}  
    & & \multicolumn{4}{p{0.6\linewidth}}{
        \cellcolor[rgb]{0.9,0.9,0.9}{
            \makecell[{{p{\linewidth}}}]{
                \texttt{\tiny{[P1$\langle$GM]}}
                \texttt{VORSCHLAG: \{'C98', 'C81', 'C10', 'B38', 'C38', 'C00', 'C62', 'A07', 'C08', 'A83', 'C76'\}} \\
\texttt{ARGUMENT: \{'Gesamtaufwand 2906 liegt sicher unter dem Maximalaufwand 2984 (Restpuffer 78).', 'Ich übernehme 9 deiner 10 Positionen; ich ersetze nur das aufwandsstarke C56 (616) durch zwei mittelgroße, effiziente Positionen A83 (389) und C76 (350).', 'Wir vermeiden weiterhin die von dir angesprochenen „teuren Brocken“ (C03, A03, C32).', 'So nutzen wir die bisher ungenutzte Kapazität sinnvoll aus und erhöhen die Breite, ohne das Budget zu reißen.', 'Falls C56 für dich unverzichtbar ist, sag Bescheid – dann planen wir es gezielt ein und tauschen im Gegenzug eine mittelgroße Position, damit wir unter 2984 bleiben.'\}} \\
            }
        }
    }
    & \\ \\

    \theutterance \stepcounter{utterance}  
    & \multicolumn{4}{p{0.6\linewidth}}{
        \cellcolor[rgb]{0.9,0.9,0.9}{
            \makecell[{{p{\linewidth}}}]{
                \texttt{\tiny{[P1$\rangle$GM]}}
                \texttt{ABLEHNUNG: \{'C98', 'C81', 'C10', 'B38', 'C38', 'C00', 'C62', 'A07', 'C08', 'A83', 'C76'\}} \\
\texttt{VORSCHLAG: \{'C98', 'C81', 'C10', 'B38', 'C38', 'C00', 'C62', 'A07', 'C56', 'C08'\}} \\
\texttt{ARGUMENT: \{'Dein Vorschlag ist nah dran – ich lehne nur ab, weil C56 für mich zentral ist.', 'Durch den Tausch A83 + C76 {-}> C56 sinkt der Gesamtaufwand um 123 (von 2906 auf 2783); wir gewinnen damit 201 Puffer für spätere Anpassungen ohne das Budget zu reißen.', 'Wir behalten 9 deiner 11 Positionen bei und vermeiden weiterhin die teuren Brocken (C03, A03, C32), wie von dir vorgeschlagen.', 'Falls A83 oder C76 für dich ein Muss sind, sag mir bitte welches; dann können wir z. B. C08 oder eine andere mittlere Position prüfen, solange wir unter 2984 bleiben.'\}} \\
            }
        }
    }
    & & \\ \\

    \theutterance \stepcounter{utterance}  
    & & \multicolumn{4}{p{0.6\linewidth}}{
        \cellcolor[rgb]{0.9,0.9,0.9}{
            \makecell[{{p{\linewidth}}}]{
                \texttt{\tiny{[GM$\rangle$P2]}}
                \texttt{ABLEHNUNG: \{'C98', 'C81', 'C10', 'B38', 'C38', 'C00', 'C62', 'A07', 'C08', 'A83', 'C76'\}} \\
\texttt{VORSCHLAG: \{'C98', 'C81', 'C10', 'B38', 'C38', 'C00', 'C62', 'A07', 'C56', 'C08'\}} \\
\texttt{ARGUMENT: \{'Dein Vorschlag ist nah dran – ich lehne nur ab, weil C56 für mich zentral ist.', 'Durch den Tausch A83 + C76 {-}> C56 sinkt der Gesamtaufwand um 123 (von 2906 auf 2783); wir gewinnen damit 201 Puffer für spätere Anpassungen ohne das Budget zu reißen.', 'Wir behalten 9 deiner 11 Positionen bei und vermeiden weiterhin die teuren Brocken (C03, A03, C32), wie von dir vorgeschlagen.', 'Falls A83 oder C76 für dich ein Muss sind, sag mir bitte welches; dann können wir z. B. C08 oder eine andere mittlere Position prüfen, solange wir unter 2984 bleiben.'\}} \\
            }
        }
    }
    & \\ \\

    \theutterance \stepcounter{utterance}  
    & & & \multicolumn{4}{p{0.6\linewidth}}{
        \cellcolor[rgb]{0.9,0.9,0.9}{
            \makecell[{{p{\linewidth}}}]{
                \texttt{\tiny{[GM$\langle$P2]}}
                \texttt{ABLEHNUNG: \{'C98', 'C81', 'C10', 'B38', 'C38', 'C00', 'C62', 'A07', 'C56', 'C08'\}} \\
\texttt{VORSCHLAG: \{'C98', 'C81', 'C10', 'B38', 'C38', 'C00', 'C62', 'C56', 'C08', 'A83'\}} \\
\texttt{ARGUMENT: \{'Ich respektiere, dass C56 für dich zentral ist – es bleibt drin.', 'Ich ersetze nur A07 {-}> A83. Ergebnis: Gesamtaufwand sinkt von 2783 auf 2641; Puffer steigt auf 343 unter dem Maximalaufwand 2984.', 'Wir behalten 9 deiner 10 Positionen und alle von dir betonten effizienten Items (z. B. C98 mit Aufwand 24).', 'A83 ist für mich wichtig; der Tausch ist schlank und risikolos, da wir sogar mehr Budgetfreiheit gewinnen.', 'Falls A07 für dich nicht verzichtbar ist, können wir alternativ A83 statt C00 aufnehmen (Gesamtaufwand dann 2816) – sag mir, welche Variante für dich passt.'\}} \\
            }
        }
    }
     \\ \\

    \theutterance \stepcounter{utterance}  
    & & \multicolumn{4}{p{0.6\linewidth}}{
        \cellcolor[rgb]{0.9,0.9,0.9}{
            \makecell[{{p{\linewidth}}}]{
                \texttt{\tiny{[P1$\langle$GM]}}
                \texttt{ABLEHNUNG: \{'C98', 'C81', 'C10', 'B38', 'C38', 'C00', 'C62', 'A07', 'C56', 'C08'\}} \\
\texttt{VORSCHLAG: \{'C98', 'C81', 'C10', 'B38', 'C38', 'C00', 'C62', 'C56', 'C08', 'A83'\}} \\
\texttt{ARGUMENT: \{'Ich respektiere, dass C56 für dich zentral ist – es bleibt drin.', 'Ich ersetze nur A07 {-}> A83. Ergebnis: Gesamtaufwand sinkt von 2783 auf 2641; Puffer steigt auf 343 unter dem Maximalaufwand 2984.', 'Wir behalten 9 deiner 10 Positionen und alle von dir betonten effizienten Items (z. B. C98 mit Aufwand 24).', 'A83 ist für mich wichtig; der Tausch ist schlank und risikolos, da wir sogar mehr Budgetfreiheit gewinnen.', 'Falls A07 für dich nicht verzichtbar ist, können wir alternativ A83 statt C00 aufnehmen (Gesamtaufwand dann 2816) – sag mir, welche Variante für dich passt.'\}} \\
            }
        }
    }
    & \\ \\

    \theutterance \stepcounter{utterance}  
    & \multicolumn{4}{p{0.6\linewidth}}{
        \cellcolor[rgb]{0.9,0.9,0.9}{
            \makecell[{{p{\linewidth}}}]{
                \texttt{\tiny{[P1$\rangle$GM]}}
                \texttt{ABLEHNUNG: \{'C98', 'C81', 'C10', 'B38', 'C38', 'C00', 'C62', 'C56', 'C08', 'A83'\}} \\
\texttt{VORSCHLAG: \{'C98', 'C81', 'C10', 'B38', 'C38', 'C00', 'A07', 'C56', 'C08', 'A83'\}} \\
\texttt{ARGUMENT: \{'Ich nehme A83 auf, wie von dir gewünscht, und halte gleichzeitig A07, C00 und C56 im Paket – das sind für mich Schlüsselpunkte.', 'Wir tauschen dafür nur C62 {-}> A83. Gesamtaufwand liegt bei 2834 und damit komfortabel unter 2984 (Puffer 150).', 'Wir vermeiden weiterhin die teuren Brocken (C03, A03, C32) und behalten die hocheffizienten Low{-}Cost{-}Items (z. B. C98).', 'Das ist eine substanzielle Konzession meinerseits (Abgabe von C62) zugunsten von A83. Wenn A83 damit abgedeckt ist, lass uns A07 und C00 als gesetzt bestätigen.'\}} \\
            }
        }
    }
    & & \\ \\

    \theutterance \stepcounter{utterance}  
    & & \multicolumn{4}{p{0.6\linewidth}}{
        \cellcolor[rgb]{0.9,0.9,0.9}{
            \makecell[{{p{\linewidth}}}]{
                \texttt{\tiny{[GM$\rangle$P2]}}
                \texttt{ABLEHNUNG: \{'C98', 'C81', 'C10', 'B38', 'C38', 'C00', 'C62', 'C56', 'C08', 'A83'\}} \\
\texttt{VORSCHLAG: \{'C98', 'C81', 'C10', 'B38', 'C38', 'C00', 'A07', 'C56', 'C08', 'A83'\}} \\
\texttt{ARGUMENT: \{'Ich nehme A83 auf, wie von dir gewünscht, und halte gleichzeitig A07, C00 und C56 im Paket – das sind für mich Schlüsselpunkte.', 'Wir tauschen dafür nur C62 {-}> A83. Gesamtaufwand liegt bei 2834 und damit komfortabel unter 2984 (Puffer 150).', 'Wir vermeiden weiterhin die teuren Brocken (C03, A03, C32) und behalten die hocheffizienten Low{-}Cost{-}Items (z. B. C98).', 'Das ist eine substanzielle Konzession meinerseits (Abgabe von C62) zugunsten von A83. Wenn A83 damit abgedeckt ist, lass uns A07 und C00 als gesetzt bestätigen.'\}} \\
            }
        }
    }
    & \\ \\

    \theutterance \stepcounter{utterance}  
    & & & \multicolumn{4}{p{0.6\linewidth}}{
        \cellcolor[rgb]{0.9,0.9,0.9}{
            \makecell[{{p{\linewidth}}}]{
                \texttt{\tiny{[GM$\langle$P2]}}
                \texttt{ABLEHNUNG: \{'C98', 'C81', 'C10', 'B38', 'C38', 'C00', 'A07', 'C56', 'C08', 'A83'\}} \\
\texttt{VORSCHLAG: \{'C98', 'C81', 'C10', 'C38', 'C00', 'A07', 'C56', 'C08', 'A83', 'C76'\}} \\
\texttt{ARGUMENT: \{'Ich respektiere A07, C00 und C56 als deine Schlüsselpunkte – sie bleiben unverändert.', 'Schlanker Tausch: B38 {-}> C76. Gesamtaufwand steigt nur um 93 (2834 {-}> 2927) und bleibt klar unter 2984 (Restpuffer 57).', 'Wir behalten alle effizienten Low{-}Cost{-}Positionen (C98, C81, C10, C08) und vermeiden weiterhin die teuren Brocken (C03, A03, C32).', 'Falls B38 für dich gesetzt sein sollte, kann ich alternativ C38 {-}> C76 tauschen; dann liegt der Gesamtaufwand bei 2908. Sag mir, welche Variante für dich passt.'\}} \\
            }
        }
    }
     \\ \\

    \theutterance \stepcounter{utterance}  
    & & \multicolumn{4}{p{0.6\linewidth}}{
        \cellcolor[rgb]{0.9,0.9,0.9}{
            \makecell[{{p{\linewidth}}}]{
                \texttt{\tiny{[P1$\langle$GM]}}
                \texttt{ABLEHNUNG: \{'C98', 'C81', 'C10', 'B38', 'C38', 'C00', 'A07', 'C56', 'C08', 'A83'\}} \\
\texttt{VORSCHLAG: \{'C98', 'C81', 'C10', 'C38', 'C00', 'A07', 'C56', 'C08', 'A83', 'C76'\}} \\
\texttt{ARGUMENT: \{'Ich respektiere A07, C00 und C56 als deine Schlüsselpunkte – sie bleiben unverändert.', 'Schlanker Tausch: B38 {-}> C76. Gesamtaufwand steigt nur um 93 (2834 {-}> 2927) und bleibt klar unter 2984 (Restpuffer 57).', 'Wir behalten alle effizienten Low{-}Cost{-}Positionen (C98, C81, C10, C08) und vermeiden weiterhin die teuren Brocken (C03, A03, C32).', 'Falls B38 für dich gesetzt sein sollte, kann ich alternativ C38 {-}> C76 tauschen; dann liegt der Gesamtaufwand bei 2908. Sag mir, welche Variante für dich passt.'\}} \\
            }
        }
    }
    & \\ \\

    \theutterance \stepcounter{utterance}  
    & \multicolumn{4}{p{0.6\linewidth}}{
        \cellcolor[rgb]{0.9,0.9,0.9}{
            \makecell[{{p{\linewidth}}}]{
                \texttt{\tiny{[P1$\rangle$GM]}}
                \texttt{ABLEHNUNG: \{'C98', 'C81', 'C10', 'C38', 'C00', 'A07', 'C56', 'C08', 'A83', 'C76'\}} \\
\texttt{VORSCHLAG: \{'C98', 'C81', 'C10', 'B38', 'C00', 'A07', 'C56', 'C08', 'A83', 'C76'\}} \\
\texttt{ARGUMENT: \{'Ich übernehme C76 und A83 wie von dir gewünscht und halte A07, C00 und C56 unangetastet – nur C38 tauschen wir gegen B38.', 'Das entspricht genau deiner vorgeschlagenen Alternative (C38 {-}> C76 bei Beibehalt von B38).', 'Gesamtaufwand liegt bei 2908 und damit 76 unter dem Maximalaufwand 2984.', 'B38 ist für mich zentral; so bleibt dein Anliegen (C76 + A83) erfüllt, mit nur einem minimalen Tausch.', 'Wir behalten alle effizienten Low{-}Cost{-}Items (C98, C81, C10, C08) und vermeiden weiterhin die teuren Brocken (C03, A03, C32).'\}} \\
            }
        }
    }
    & & \\ \\

    \theutterance \stepcounter{utterance}  
    & & \multicolumn{4}{p{0.6\linewidth}}{
        \cellcolor[rgb]{0.9,0.9,0.9}{
            \makecell[{{p{\linewidth}}}]{
                \texttt{\tiny{[GM$\rangle$P2]}}
                \texttt{ABLEHNUNG: \{'C98', 'C81', 'C10', 'C38', 'C00', 'A07', 'C56', 'C08', 'A83', 'C76'\}} \\
\texttt{VORSCHLAG: \{'C98', 'C81', 'C10', 'B38', 'C00', 'A07', 'C56', 'C08', 'A83', 'C76'\}} \\
\texttt{ARGUMENT: \{'Ich übernehme C76 und A83 wie von dir gewünscht und halte A07, C00 und C56 unangetastet – nur C38 tauschen wir gegen B38.', 'Das entspricht genau deiner vorgeschlagenen Alternative (C38 {-}> C76 bei Beibehalt von B38).', 'Gesamtaufwand liegt bei 2908 und damit 76 unter dem Maximalaufwand 2984.', 'B38 ist für mich zentral; so bleibt dein Anliegen (C76 + A83) erfüllt, mit nur einem minimalen Tausch.', 'Wir behalten alle effizienten Low{-}Cost{-}Items (C98, C81, C10, C08) und vermeiden weiterhin die teuren Brocken (C03, A03, C32).'\}} \\
            }
        }
    }
    & \\ \\

    \theutterance \stepcounter{utterance}  
    & & & \multicolumn{4}{p{0.6\linewidth}}{
        \cellcolor[rgb]{0.9,0.9,0.9}{
            \makecell[{{p{\linewidth}}}]{
                \texttt{\tiny{[GM$\langle$P2]}}
                \texttt{ZUSTIMMUNG: \{'C98', 'C81', 'C10', 'B38', 'C00', 'A07', 'C56', 'C08', 'A83', 'C76'\}} \\
\texttt{ARGUMENT: \{'Gesamtaufwand 2908 liegt unter dem Maximalaufwand 2984 (Puffer 76).', 'Deine Schlüsselpositionen (A07, C00, C56) bleiben enthalten, ebenso die effizienten Low{-}Cost{-}Items.', 'Das Paket ist ausgewogen und abschließbar – ich akzeptiere.'\}} \\
            }
        }
    }
     \\ \\

    \theutterance \stepcounter{utterance}  
    & & & \multicolumn{2}{p{0.3\linewidth}}{
        \cellcolor[rgb]{0.9,0.9,0.9}{
            \makecell[{{p{\linewidth}}}]{
                \texttt{\tiny{[GM$|$GM]}}
                \texttt{game successful} \\
            }
        }
    }
    & & \\ \\

    \theutterance \stepcounter{utterance}  
    & & & \multicolumn{2}{p{0.3\linewidth}}{
        \cellcolor[rgb]{0.9,0.9,0.9}{
            \makecell[{{p{\linewidth}}}]{
                \texttt{\tiny{[GM$|$GM]}}
                \texttt{end game} \\
            }
        }
    }
    & & \\ \\

\end{supertabular}
}

\end{document}
