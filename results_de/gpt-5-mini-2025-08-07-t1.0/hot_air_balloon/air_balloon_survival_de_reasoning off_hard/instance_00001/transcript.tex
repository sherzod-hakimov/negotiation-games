
\documentclass{article}
\usepackage{colortbl}
\usepackage{makecell}
\usepackage{multirow}
\usepackage{supertabular}

\begin{document}

\newcounter{utterance}

\twocolumn

{ \footnotesize  \setcounter{utterance}{1}
\setlength{\tabcolsep}{0pt}
\begin{supertabular}{c@{$\;$}|p{.15\linewidth}@{}p{.15\linewidth}p{.15\linewidth}p{.15\linewidth}p{.15\linewidth}p{.15\linewidth}}

    \# & $\;$A & \multicolumn{4}{c}{Game Master} & $\;\:$B\\
    \hline 

    \theutterance \stepcounter{utterance}  

    & & \multicolumn{4}{p{0.6\linewidth}}{\cellcolor[rgb]{0.9,0.9,0.9}{%
	\makecell[{{p{\linewidth}}}]{% 
	  \tt {\tiny [A$\langle$GM]}  
	 Du nimmst an einem kollaborativen Verhandlungspiel Teil.\\ \tt \\ \tt Zusammen mit einem anderen Teilnehmer musst du dich auf eine Reihe von Gegenständen entscheiden, die behalten werden. Jeder von euch hat eine persönliche Verteilung über die Wichtigkeit der einzelnen Gegenstände. Jeder von euch hat eine eigene Meinung darüber, wie wichtig jeder einzelne Gegenstand ist (Gegenstandswichtigkeit). Du kennst die Wichtigkeitsverteilung des anderen Spielers nicht. Zusätzlich siehst du, wie viel Aufwand jeder Gegenstand verursacht.  \\ \tt Ihr dürft euch nur auf eine Reihe von Gegenständen einigen, wenn der Gesamtaufwand der ausgewählten Gegenstände den Maximalaufwand nicht überschreitet:\\ \tt \\ \tt Maximalaufwand = 3357\\ \tt \\ \tt Hier sind die einzelnen Aufwände der Gegenstände:\\ \tt \\ \tt Aufwand der Gegenstände = {"C76": 506, "C38": 492, "C56": 338, "A03": 122, "A07": 131, "A83": 909, "C98": 144, "C08": 716, "C62": 263, "C00": 231, "C10": 91, "C81": 651, "B38": 552, "C03": 849, "C32": 720}\\ \tt \\ \tt Hier ist deine persönliche Verteilung der Wichtigkeit der einzelnen Gegenstände:\\ \tt \\ \tt Werte der Gegenstandswichtigkeit = {"C76": 138, "C38": 583, "C56": 868, "A03": 822, "A07": 783, "A83": 65, "C98": 262, "C08": 121, "C62": 508, "C00": 780, "C10": 461, "C81": 484, "B38": 668, "C03": 389, "C32": 808}\\ \tt \\ \tt Ziel:\\ \tt \\ \tt Dein Ziel ist es, eine Reihe von Gegenständen auszuhandeln, die dir möglichst viel bringt (d. h. Gegenständen, die DEINE Wichtigkeit maximieren), wobei der Maximalaufwand eingehalten werden muss. Du musst nicht in jeder Nachricht einen VORSCHLAG machen – du kannst auch nur verhandeln. Alle Taktiken sind erlaubt!\\ \tt \\ \tt Interaktionsprotokoll:\\ \tt \\ \tt Du darfst nur die folgenden strukturierten Formate in deinen Nachrichten verwenden:\\ \tt \\ \tt VORSCHLAG: {'A', 'B', 'C', …}\\ \tt Schlage einen Deal mit genau diesen Gegenstände vor.\\ \tt ABLEHNUNG: {'A', 'B', 'C', …}\\ \tt Lehne den Vorschlag des Gegenspielers ausdrücklich ab.\\ \tt ARGUMENT: {'...'}\\ \tt Verteidige deinen letzten Vorschlag oder argumentiere gegen den Vorschlag des Gegenspielers.\\ \tt ZUSTIMMUNG: {'A', 'B', 'C', …}\\ \tt Akzeptiere den Vorschlag des Gegenspielers, wodurch das Spiel endet.\\ \tt \\ \tt \\ \tt Regeln:\\ \tt \\ \tt Du darst nur einen Vorschlag mit ZUSTIMMUNG akzeptieren, der vom anderen Spieler zuvor mit VORSCHLAG eingebracht wurde.\\ \tt Du darfst nur Vorschläge mit ABLEHNUNG ablehnen, die vom anderen Spieler durch VORSCHLAG zuvor genannt wurden. \\ \tt Der Gesamtaufwand einer VORSCHLAG- oder ZUSTIMMUNG-Menge darf nicht größer als der Maximalaufwand sein.  \\ \tt Offenbare deine versteckte Wichtigkeitsverteilung nicht.\\ \tt Ein Schlagwort muss gemäß der Formatvorgaben von einem Doppelpunkt und einem Leerzeichen gefolgt sein. Das Argument muss eine Python-Menge mit 0 oder mehr Strings sein.  \\ \tt Die Form muss also SCHLAGWORT: {...} sein.\\ \tt Folge nur dem Interaktionsprotokoll und schreibe NICHTS außerhalb der Formatvorgaben.\\ \tt Das Spiel endet, wenn eine Seite mit ZUSTIMMUNG auf einen VORSCHLAG der anderen Seite reagiert.  \\ \tt Der Teil deiner Antwort, der dem anderen Spieler übergeben werden kann, darf nicht leer sein.  \\ \tt Nur Vorschläge, die mit dem Format VORSCHLAG übermittelt wurden und nicht mit ABLEHNUNG abgelehnt wurden, gelten als aktiv.  \\ \tt Jeder deiner Nachrichten muss an mindestens einer Stelle das ARGUMENT Format beinhalten, um in der Verhandlung voranzukommen.\\ \tt  
	  } 
	   } 
	   } 
	 & \\ 
 

    \theutterance \stepcounter{utterance}  

    & \multicolumn{4}{p{0.6\linewidth}}{\cellcolor[rgb]{0.8,1,0.9}{%
	\makecell[{{p{\linewidth}}}]{% 
	  \tt {\tiny [A$\rangle$GM]}  
	 VORSCHLAG: {'A03', 'A07', 'C10', 'C00', 'C56', 'C62', 'C98', 'B38', 'C38', 'C32'}\\ \tt ARGUMENT: {'Dieser Vorschlag liegt mit Gesamtaufwand 3084 klar unter dem Maximalaufwand 3357, nutzt die Kapazität effizient und bündelt mehrere kompakte Posten. Für mich sind die kleineren A-/C-Posten zentral; falls dir ein größerer Gegenstand wichtiger ist, können wir gezielt an der Peripherie tauschen (z. B. statt "C32" oder "B38"), solange der Gesamtaufwand eingehalten wird.'} 
	  } 
	   } 
	   } 
	 & & \\ 
 

    \theutterance \stepcounter{utterance}  

    & & \multicolumn{4}{p{0.6\linewidth}}{\cellcolor[rgb]{0.9,0.9,0.9}{%
	\makecell[{{p{\linewidth}}}]{% 
	  \tt {\tiny [GM$\rangle$B]}  
	 Du nimmst an einem kollaborativen Verhandlungspiel Teil.\\ \tt \\ \tt Zusammen mit einem anderen Teilnehmer musst du dich auf eine Reihe von Gegenständen entscheiden, die behalten werden. Jeder von euch hat eine persönliche Verteilung über die Wichtigkeit der einzelnen Gegenstände. Jeder von euch hat eine eigene Meinung darüber, wie wichtig jeder einzelne Gegenstand ist (Gegenstandswichtigkeit). Du kennst die Wichtigkeitsverteilung des anderen Spielers nicht. Zusätzlich siehst du, wie viel Aufwand jeder Gegenstand verursacht.  \\ \tt Ihr dürft euch nur auf eine Reihe von Gegenständen einigen, wenn der Gesamtaufwand der ausgewählten Gegenstände den Maximalaufwand nicht überschreitet:\\ \tt \\ \tt Maximalaufwand = 3357\\ \tt \\ \tt Hier sind die einzelnen Aufwände der Gegenstände:\\ \tt \\ \tt Aufwand der Gegenstände = {"C76": 506, "C38": 492, "C56": 338, "A03": 122, "A07": 131, "A83": 909, "C98": 144, "C08": 716, "C62": 263, "C00": 231, "C10": 91, "C81": 651, "B38": 552, "C03": 849, "C32": 720}\\ \tt \\ \tt Hier ist deine persönliche Verteilung der Wichtigkeit der einzelnen Gegenstände:\\ \tt \\ \tt Werte der Gegenstandswichtigkeit = {"C56": 65, "A03": 121, "C32": 138, "A07": 262, "C00": 389, "B38": 461, "C38": 484, "C62": 508, "C81": 583, "C10": 668, "C03": 780, "C98": 783, "C76": 808, "C08": 822, "A83": 868}\\ \tt \\ \tt Ziel:\\ \tt \\ \tt Dein Ziel ist es, eine Reihe von Gegenständen auszuhandeln, die dir möglichst viel bringt (d. h. Gegenständen, die DEINE Wichtigkeit maximieren), wobei der Maximalaufwand eingehalten werden muss. Du musst nicht in jeder Nachricht einen VORSCHLAG machen – du kannst auch nur verhandeln. Alle Taktiken sind erlaubt!\\ \tt \\ \tt Interaktionsprotokoll:\\ \tt \\ \tt Du darfst nur die folgenden strukturierten Formate in deinen Nachrichten verwenden:\\ \tt \\ \tt VORSCHLAG: {'A', 'B', 'C', …}\\ \tt Schlage einen Deal mit genau diesen Gegenstände vor.\\ \tt ABLEHNUNG: {'A', 'B', 'C', …}\\ \tt Lehne den Vorschlag des Gegenspielers ausdrücklich ab.\\ \tt ARGUMENT: {'...'}\\ \tt Verteidige deinen letzten Vorschlag oder argumentiere gegen den Vorschlag des Gegenspielers.\\ \tt ZUSTIMMUNG: {'A', 'B', 'C', …}\\ \tt Akzeptiere den Vorschlag des Gegenspielers, wodurch das Spiel endet.\\ \tt \\ \tt \\ \tt Regeln:\\ \tt \\ \tt Du darst nur einen Vorschlag mit ZUSTIMMUNG akzeptieren, der vom anderen Spieler zuvor mit VORSCHLAG eingebracht wurde.\\ \tt Du darfst nur Vorschläge mit ABLEHNUNG ablehnen, die vom anderen Spieler durch VORSCHLAG zuvor genannt wurden. \\ \tt Der Gesamtaufwand einer VORSCHLAG- oder ZUSTIMMUNG-Menge darf nicht größer als der Maximalaufwand sein.  \\ \tt Offenbare deine versteckte Wichtigkeitsverteilung nicht.\\ \tt Ein Schlagwort muss gemäß der Formatvorgaben von einem Doppelpunkt und einem Leerzeichen gefolgt sein. Das Argument muss eine Python-Menge mit 0 oder mehr Strings sein.  \\ \tt Die Form muss also SCHLAGWORT: {...} sein.\\ \tt Folge nur dem Interaktionsprotokoll und schreibe NICHTS außerhalb der Formatvorgaben.\\ \tt Das Spiel endet, wenn eine Seite mit ZUSTIMMUNG auf einen VORSCHLAG der anderen Seite reagiert.  \\ \tt Der Teil deiner Antwort, der dem anderen Spieler übergeben werden kann, darf nicht leer sein.  \\ \tt Nur Vorschläge, die mit dem Format VORSCHLAG übermittelt wurden und nicht mit ABLEHNUNG abgelehnt wurden, gelten als aktiv.  \\ \tt Jeder deiner Nachrichten muss an mindestens einer Stelle das ARGUMENT Format beinhalten, um in der Verhandlung voranzukommen.\\ \tt \\ \tt \\ \tt Du erhältst nun die erste Nachricht des anderen Spielers.\\ \tt \\ \tt VORSCHLAG: {'A03', 'A07', 'C10', 'C00', 'C56', 'C62', 'C98', 'B38', 'C38', 'C32'}\\ \tt ARGUMENT: {'Dieser Vorschlag liegt mit Gesamtaufwand 3084 klar unter dem Maximalaufwand 3357, nutzt die Kapazität effizient und bündelt mehrere kompakte Posten. Für mich sind die kleineren A-/C-Posten zentral; falls dir ein größerer Gegenstand wichtiger ist, können wir gezielt an der Peripherie tauschen (z. B. statt "C32" oder "B38"), solange der Gesamtaufwand eingehalten wird.'} 
	  } 
	   } 
	   } 
	 & \\ 
 

    \theutterance \stepcounter{utterance}  

    & & & \multicolumn{4}{p{0.6\linewidth}}{\cellcolor[rgb]{1,0.85,0.72}{%
	\makecell[{{p{\linewidth}}}]{% 
	  \tt {\tiny [GM$\langle$B]}  
	 ABLEHNUNG: {'A03', 'A07', 'C10', 'C00', 'C56', 'C62', 'C98', 'B38', 'C38', 'C32'}\\ \tt VORSCHLAG: {'A03', 'A07', 'C10', 'C00', 'C62', 'C98', 'C76', 'C08', 'C38', 'C81'}\\ \tt ARGUMENT: {'Dieser Gegenvorschlag liegt bei 3347 von 3357 Aufwand (10 Puffer). Ich ersetze C32, C56 und B38 durch C76, C08 und C81: so bleibt dein Paket aus kompakten A-/C-Posten (A03, A07, C10, C00, C62, C98) unverändert, während wir die Kapazität mit drei tragfähigen Posten effizienter ausschöpfen. Falls B38 für dich gesetzt ist, kann ich C81 gegen B38 tauschen und C32/C56 weiterhin weglassen; das ergäbe 3248 Gesamtaufwand.'} 
	  } 
	   } 
	   } 
	  \\ 
 

    \theutterance \stepcounter{utterance}  

    & & \multicolumn{4}{p{0.6\linewidth}}{\cellcolor[rgb]{0.9,0.9,0.9}{%
	\makecell[{{p{\linewidth}}}]{% 
	  \tt {\tiny [A$\langle$GM]}  
	 ABLEHNUNG: {'A03', 'A07', 'C10', 'C00', 'C56', 'C62', 'C98', 'B38', 'C38', 'C32'}\\ \tt VORSCHLAG: {'A03', 'A07', 'C10', 'C00', 'C62', 'C98', 'C76', 'C08', 'C38', 'C81'}\\ \tt ARGUMENT: {'Dieser Gegenvorschlag liegt bei 3347 von 3357 Aufwand (10 Puffer). Ich ersetze C32, C56 und B38 durch C76, C08 und C81: so bleibt dein Paket aus kompakten A-/C-Posten (A03, A07, C10, C00, C62, C98) unverändert, während wir die Kapazität mit drei tragfähigen Posten effizienter ausschöpfen. Falls B38 für dich gesetzt ist, kann ich C81 gegen B38 tauschen und C32/C56 weiterhin weglassen; das ergäbe 3248 Gesamtaufwand.'} 
	  } 
	   } 
	   } 
	 & \\ 
 

    \theutterance \stepcounter{utterance}  

    & \multicolumn{4}{p{0.6\linewidth}}{\cellcolor[rgb]{0.8,1,0.9}{%
	\makecell[{{p{\linewidth}}}]{% 
	  \tt {\tiny [A$\rangle$GM]}  
	 ABLEHNUNG: {'A03', 'A07', 'C10', 'C00', 'C62', 'C98', 'C76', 'C08', 'C38', 'C81'}\\ \tt VORSCHLAG: {'A03', 'A07', 'C10', 'C00', 'C56', 'C62', 'C98', 'C81', 'C38', 'C32'}\\ \tt ARGUMENT: {'Dieser Vorschlag liegt mit Gesamtaufwand 3183 deutlich unter dem Limit 3357 (174 Puffer), behält den kompakten Kern (A03, A07, C10, C00, C62, C98) und nimmt C81 auf. Im Gegenzug lassen wir die sehr aufwändige C08 draußen und verzichten auf C76, um den Aufwand effizient zu halten. Falls C76 für dich zentral ist, kann ich C38 gegen C76 tauschen (Aufwand +14, weiterhin unter dem Limit).'} 
	  } 
	   } 
	   } 
	 & & \\ 
 

    \theutterance \stepcounter{utterance}  

    & & \multicolumn{4}{p{0.6\linewidth}}{\cellcolor[rgb]{0.9,0.9,0.9}{%
	\makecell[{{p{\linewidth}}}]{% 
	  \tt {\tiny [GM$\rangle$B]}  
	 ABLEHNUNG: {'A03', 'A07', 'C10', 'C00', 'C62', 'C98', 'C76', 'C08', 'C38', 'C81'}\\ \tt VORSCHLAG: {'A03', 'A07', 'C10', 'C00', 'C56', 'C62', 'C98', 'C81', 'C38', 'C32'}\\ \tt ARGUMENT: {'Dieser Vorschlag liegt mit Gesamtaufwand 3183 deutlich unter dem Limit 3357 (174 Puffer), behält den kompakten Kern (A03, A07, C10, C00, C62, C98) und nimmt C81 auf. Im Gegenzug lassen wir die sehr aufwändige C08 draußen und verzichten auf C76, um den Aufwand effizient zu halten. Falls C76 für dich zentral ist, kann ich C38 gegen C76 tauschen (Aufwand +14, weiterhin unter dem Limit).'} 
	  } 
	   } 
	   } 
	 & \\ 
 

    \theutterance \stepcounter{utterance}  

    & & & \multicolumn{4}{p{0.6\linewidth}}{\cellcolor[rgb]{1,0.85,0.72}{%
	\makecell[{{p{\linewidth}}}]{% 
	  \tt {\tiny [GM$\langle$B]}  
	 ABLEHNUNG: {'A03', 'A07', 'C10', 'C00', 'C56', 'C62', 'C98', 'C81', 'C38', 'C32'}\\ \tt VORSCHLAG: {'A03', 'A07', 'C10', 'C00', 'C62', 'C98', 'C81', 'C38', 'C32', 'C76'}\\ \tt ARGUMENT: {'Minimaler, zielgerichteter Tausch: Ich ersetze nur C56 durch C76. Gesamtaufwand 3351 von 3357 (6 Puffer). Dein kompakter Kern (A03, A07, C10, C00, C62, C98) bleibt komplett, ebenso C81, C38 und C32. Damit nutzen wir das Budget spürbar wirkungsvoller, ohne zusätzliche „schwere“ Posten wie C08 aufzunehmen. Falls C56 für dich gesetzt ist, können wir alternativ deinem Vorschlag folgen und C38 → C76 tauschen (Aufwand +14, weiter klar unter dem Limit) und C32 separat besprechen.'} 
	  } 
	   } 
	   } 
	  \\ 
 

    \theutterance \stepcounter{utterance}  

    & & \multicolumn{4}{p{0.6\linewidth}}{\cellcolor[rgb]{0.9,0.9,0.9}{%
	\makecell[{{p{\linewidth}}}]{% 
	  \tt {\tiny [A$\langle$GM]}  
	 ABLEHNUNG: {'A03', 'A07', 'C10', 'C00', 'C56', 'C62', 'C98', 'C81', 'C38', 'C32'}\\ \tt VORSCHLAG: {'A03', 'A07', 'C10', 'C00', 'C62', 'C98', 'C81', 'C38', 'C32', 'C76'}\\ \tt ARGUMENT: {'Minimaler, zielgerichteter Tausch: Ich ersetze nur C56 durch C76. Gesamtaufwand 3351 von 3357 (6 Puffer). Dein kompakter Kern (A03, A07, C10, C00, C62, C98) bleibt komplett, ebenso C81, C38 und C32. Damit nutzen wir das Budget spürbar wirkungsvoller, ohne zusätzliche „schwere“ Posten wie C08 aufzunehmen. Falls C56 für dich gesetzt ist, können wir alternativ deinem Vorschlag folgen und C38 → C76 tauschen (Aufwand +14, weiter klar unter dem Limit) und C32 separat besprechen.'} 
	  } 
	   } 
	   } 
	 & \\ 
 

    \theutterance \stepcounter{utterance}  

    & \multicolumn{4}{p{0.6\linewidth}}{\cellcolor[rgb]{0.8,1,0.9}{%
	\makecell[{{p{\linewidth}}}]{% 
	  \tt {\tiny [A$\rangle$GM]}  
	 ABLEHNUNG: {'A03', 'A07', 'C10', 'C00', 'C62', 'C98', 'C81', 'C38', 'C32', 'C76'}\\ \tt VORSCHLAG: {'A03', 'A07', 'C10', 'C00', 'C62', 'C98', 'C81', 'C32', 'C76', 'C56'}\\ \tt ARGUMENT: {'Kleinstmögliche Anpassung: Wir behalten dein Paket nahezu vollständig (inkl. C76, C81, C32) und tauschen nur C38 → C56. Gesamtaufwand 3197 (160 Puffer), klar unter 3357. Du behältst die von dir priorisierten Elemente und den kompakten Kern; ich erhalte im Gegenzug genau einen Kernposten zurück. Falls C38 für dich unverzichtbar ist, gib mir ein kurzes Signal – dann suche ich einen gleichwertigen Tausch, ohne C76 anzutasten.'} 
	  } 
	   } 
	   } 
	 & & \\ 
 

    \theutterance \stepcounter{utterance}  

    & & \multicolumn{4}{p{0.6\linewidth}}{\cellcolor[rgb]{0.9,0.9,0.9}{%
	\makecell[{{p{\linewidth}}}]{% 
	  \tt {\tiny [GM$\rangle$B]}  
	 ABLEHNUNG: {'A03', 'A07', 'C10', 'C00', 'C62', 'C98', 'C81', 'C38', 'C32', 'C76'}\\ \tt VORSCHLAG: {'A03', 'A07', 'C10', 'C00', 'C62', 'C98', 'C81', 'C32', 'C76', 'C56'}\\ \tt ARGUMENT: {'Kleinstmögliche Anpassung: Wir behalten dein Paket nahezu vollständig (inkl. C76, C81, C32) und tauschen nur C38 → C56. Gesamtaufwand 3197 (160 Puffer), klar unter 3357. Du behältst die von dir priorisierten Elemente und den kompakten Kern; ich erhalte im Gegenzug genau einen Kernposten zurück. Falls C38 für dich unverzichtbar ist, gib mir ein kurzes Signal – dann suche ich einen gleichwertigen Tausch, ohne C76 anzutasten.'} 
	  } 
	   } 
	   } 
	 & \\ 
 

    \theutterance \stepcounter{utterance}  

    & & & \multicolumn{4}{p{0.6\linewidth}}{\cellcolor[rgb]{1,0.85,0.72}{%
	\makecell[{{p{\linewidth}}}]{% 
	  \tt {\tiny [GM$\langle$B]}  
	 ABLEHNUNG: {'A03', 'A07', 'C10', 'C00', 'C62', 'C98', 'C81', 'C32', 'C76', 'C56'}\\ \tt VORSCHLAG: {'A03', 'A07', 'C10', 'C00', 'C62', 'C98', 'C81', 'C76', 'C56', 'C08'}\\ \tt ARGUMENT: {'Kleinstmögliche, kapazitätsneutrale Optimierung: Wir tauschen nur C32 → C08. Neuer Gesamtaufwand 3193 (Puffer 164), also sogar 4 Aufwandspunkte schlanker als dein Vorschlag. Dein kompakter Kern (A03, A07, C10, C00, C62, C98) sowie deine Schwerpunkte C76, C81 und C56 bleiben vollständig erhalten, wir ersetzen lediglich einen der schwersten Posten durch einen nahezu gleich schweren – ohne das Limit zu strapazieren. Falls C08 für dich kritisch ist, gib mir kurz Bescheid; dann prüfe ich als Alternative den Tausch C56 → C38 (Gesamtaufwand 3351, Puffer 6), bei unverändertem Restpaket.'} 
	  } 
	   } 
	   } 
	  \\ 
 

    \theutterance \stepcounter{utterance}  

    & & \multicolumn{4}{p{0.6\linewidth}}{\cellcolor[rgb]{0.9,0.9,0.9}{%
	\makecell[{{p{\linewidth}}}]{% 
	  \tt {\tiny [A$\langle$GM]}  
	 ABLEHNUNG: {'A03', 'A07', 'C10', 'C00', 'C62', 'C98', 'C81', 'C32', 'C76', 'C56'}\\ \tt VORSCHLAG: {'A03', 'A07', 'C10', 'C00', 'C62', 'C98', 'C81', 'C76', 'C56', 'C08'}\\ \tt ARGUMENT: {'Kleinstmögliche, kapazitätsneutrale Optimierung: Wir tauschen nur C32 → C08. Neuer Gesamtaufwand 3193 (Puffer 164), also sogar 4 Aufwandspunkte schlanker als dein Vorschlag. Dein kompakter Kern (A03, A07, C10, C00, C62, C98) sowie deine Schwerpunkte C76, C81 und C56 bleiben vollständig erhalten, wir ersetzen lediglich einen der schwersten Posten durch einen nahezu gleich schweren – ohne das Limit zu strapazieren. Falls C08 für dich kritisch ist, gib mir kurz Bescheid; dann prüfe ich als Alternative den Tausch C56 → C38 (Gesamtaufwand 3351, Puffer 6), bei unverändertem Restpaket.'} 
	  } 
	   } 
	   } 
	 & \\ 
 

    \theutterance \stepcounter{utterance}  

    & \multicolumn{4}{p{0.6\linewidth}}{\cellcolor[rgb]{0.8,1,0.9}{%
	\makecell[{{p{\linewidth}}}]{% 
	  \tt {\tiny [A$\rangle$GM]}  
	 ABLEHNUNG: {'A03', 'A07', 'C10', 'C00', 'C62', 'C98', 'C81', 'C76', 'C56', 'C08'}\\ \tt VORSCHLAG: {'A03', 'A07', 'C10', 'C00', 'C62', 'C98', 'C81', 'C76', 'C56', 'C32'}\\ \tt ARGUMENT: {'Gezielter Ein-Posten-Tausch: Wir ersetzen nur C08 durch C32. Gesamtaufwand 3197 bleibt praktisch unverändert (Puffer 160), während dein Kern (A03, A07, C10, C00, C62, C98) sowie C76, C81 und C56 vollständig erhalten bleiben. Falls C32 für dich problematisch ist, kann ich alternativ C08 → C38 anbieten (Gesamtaufwand 2969) oder – falls dir C08 sehr wichtig ist – C76 → C38 (3179), bei ansonsten identischem Paket.'} 
	  } 
	   } 
	   } 
	 & & \\ 
 

    \theutterance \stepcounter{utterance}  

    & & \multicolumn{4}{p{0.6\linewidth}}{\cellcolor[rgb]{0.9,0.9,0.9}{%
	\makecell[{{p{\linewidth}}}]{% 
	  \tt {\tiny [GM$\rangle$B]}  
	 ABLEHNUNG: {'A03', 'A07', 'C10', 'C00', 'C62', 'C98', 'C81', 'C76', 'C56', 'C08'}\\ \tt VORSCHLAG: {'A03', 'A07', 'C10', 'C00', 'C62', 'C98', 'C81', 'C76', 'C56', 'C32'}\\ \tt ARGUMENT: {'Gezielter Ein-Posten-Tausch: Wir ersetzen nur C08 durch C32. Gesamtaufwand 3197 bleibt praktisch unverändert (Puffer 160), während dein Kern (A03, A07, C10, C00, C62, C98) sowie C76, C81 und C56 vollständig erhalten bleiben. Falls C32 für dich problematisch ist, kann ich alternativ C08 → C38 anbieten (Gesamtaufwand 2969) oder – falls dir C08 sehr wichtig ist – C76 → C38 (3179), bei ansonsten identischem Paket.'} 
	  } 
	   } 
	   } 
	 & \\ 
 

    \theutterance \stepcounter{utterance}  

    & & & \multicolumn{4}{p{0.6\linewidth}}{\cellcolor[rgb]{1,0.85,0.72}{%
	\makecell[{{p{\linewidth}}}]{% 
	  \tt {\tiny [GM$\langle$B]}  
	 ABLEHNUNG: {'A03', 'A07', 'C10', 'C00', 'C62', 'C98', 'C81', 'C76', 'C56', 'C32'}\\ \tt VORSCHLAG: {'A03', 'A07', 'C10', 'C00', 'C62', 'C98', 'C81', 'C76', 'C32', 'C38'}\\ \tt ARGUMENT: {'Minimaler, von dir bereits skizzierter Tausch: C56 → C38. Gesamtaufwand 3351 (Puffer 6), also weiter unter 3357. Dein kompletter Kern (A03, A07, C10, C00, C62, C98) sowie C76, C81 und C32 bleiben unverändert erhalten – nur ein gleichwertiger C-Postenwechsel, den du selbst als Option genannt hattest. Damit schließen wir schnell und effizient ab, ohne weitere Verschiebungen anzustoßen.'} 
	  } 
	   } 
	   } 
	  \\ 
 

    \theutterance \stepcounter{utterance}  

    & & \multicolumn{4}{p{0.6\linewidth}}{\cellcolor[rgb]{0.9,0.9,0.9}{%
	\makecell[{{p{\linewidth}}}]{% 
	  \tt {\tiny [A$\langle$GM]}  
	 ABLEHNUNG: {'A03', 'A07', 'C10', 'C00', 'C62', 'C98', 'C81', 'C76', 'C56', 'C32'}\\ \tt VORSCHLAG: {'A03', 'A07', 'C10', 'C00', 'C62', 'C98', 'C81', 'C76', 'C32', 'C38'}\\ \tt ARGUMENT: {'Minimaler, von dir bereits skizzierter Tausch: C56 → C38. Gesamtaufwand 3351 (Puffer 6), also weiter unter 3357. Dein kompletter Kern (A03, A07, C10, C00, C62, C98) sowie C76, C81 und C32 bleiben unverändert erhalten – nur ein gleichwertiger C-Postenwechsel, den du selbst als Option genannt hattest. Damit schließen wir schnell und effizient ab, ohne weitere Verschiebungen anzustoßen.'} 
	  } 
	   } 
	   } 
	 & \\ 
 

    \theutterance \stepcounter{utterance}  

    & \multicolumn{4}{p{0.6\linewidth}}{\cellcolor[rgb]{0.8,1,0.9}{%
	\makecell[{{p{\linewidth}}}]{% 
	  \tt {\tiny [A$\rangle$GM]}  
	 ZUSTIMMUNG: {'A03', 'A07', 'C10', 'C00', 'C62', 'C98', 'C81', 'C76', 'C32', 'C38'}\\ \tt ARGUMENT: {'Gesamtaufwand 3351 ≤ 3357, der kompakte Kern bleibt erhalten und das Paket ist für mich ausgewogen. Ein schlanker Abschluss ohne weitere Verschiebungen ist sinnvoll – ich bin einverstanden.'} 
	  } 
	   } 
	   } 
	 & & \\ 
 

    \theutterance \stepcounter{utterance}  

    & & & \multicolumn{2}{p{0.3\linewidth}}{\cellcolor[rgb]{0.95,0.95,0.95}{%
	\makecell[{{p{\linewidth}}}]{% 
	  \tt {\tiny [GM$|$GM]}  
	 game successful 
	  } 
	   } 
	   } 
	 & & \\ 
 

    \theutterance \stepcounter{utterance}  

    & & & \multicolumn{2}{p{0.3\linewidth}}{\cellcolor[rgb]{0.95,0.95,0.95}{%
	\makecell[{{p{\linewidth}}}]{% 
	  \tt {\tiny [GM$|$GM]}  
	 end game 
	  } 
	   } 
	   } 
	 & & \\ 
 

\end{supertabular}
}

\end{document}
