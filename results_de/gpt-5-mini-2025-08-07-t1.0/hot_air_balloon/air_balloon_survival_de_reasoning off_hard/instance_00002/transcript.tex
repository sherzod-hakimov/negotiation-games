
\documentclass{article}
\usepackage{colortbl}
\usepackage{makecell}
\usepackage{multirow}
\usepackage{supertabular}

\begin{document}

\newcounter{utterance}

\centering \large Interaction Transcript for game `hot\_air\_balloon', experiment `air\_balloon\_survival\_de\_reasoning off\_hard', episode 2 with gpt{-}5{-}mini{-}2025{-}08{-}07{-}t1.0.
\vspace{24pt}

{ \footnotesize  \setcounter{utterance}{1}
\setlength{\tabcolsep}{0pt}
\begin{supertabular}{c@{$\;$}|p{.15\linewidth}@{}p{.15\linewidth}p{.15\linewidth}p{.15\linewidth}p{.15\linewidth}p{.15\linewidth}}
    \# & $\;$A & \multicolumn{4}{c}{Game Master} & $\;\:$B\\
    \hline

    \theutterance \stepcounter{utterance}  
    & & \multicolumn{4}{p{0.6\linewidth}}{
        \cellcolor[rgb]{0.9,0.9,0.9}{
            \makecell[{{p{\linewidth}}}]{
                \texttt{\tiny{[P1$\langle$GM]}}
                \texttt{Du nimmst an einem kollaborativen Verhandlungspiel Teil.} \\
\\ 
\texttt{Zusammen mit einem anderen Teilnehmer musst du dich auf eine Reihe von Gegenständen entscheiden, die behalten werden. Jeder von euch hat eine persönliche Verteilung über die Wichtigkeit der einzelnen Gegenstände. Jeder von euch hat eine eigene Meinung darüber, wie wichtig jeder einzelne Gegenstand ist (Gegenstandswichtigkeit). Du kennst die Wichtigkeitsverteilung des anderen Spielers nicht. Zusätzlich siehst du, wie viel Aufwand jeder Gegenstand verursacht.  } \\
\texttt{Ihr dürft euch nur auf eine Reihe von Gegenständen einigen, wenn der Gesamtaufwand der ausgewählten Gegenstände den Maximalaufwand nicht überschreitet:} \\
\\ 
\texttt{Maximalaufwand = 3203} \\
\\ 
\texttt{Hier sind die einzelnen Aufwände der Gegenstände:} \\
\\ 
\texttt{Aufwand der Gegenstände = \{"C76": 52, "C38": 577, "C56": 177, "A03": 702, "A07": 119, "A83": 232, "C98": 577, "C08": 205, "C62": 516, "C00": 582, "C10": 676, "C81": 906, "B38": 316, "C03": 433, "C32": 336\}} \\
\\ 
\texttt{Hier ist deine persönliche Verteilung der Wichtigkeit der einzelnen Gegenstände:} \\
\\ 
\texttt{Werte der Gegenstandswichtigkeit = \{"C76": 138, "C38": 583, "C56": 868, "A03": 822, "A07": 783, "A83": 65, "C98": 262, "C08": 121, "C62": 508, "C00": 780, "C10": 461, "C81": 484, "B38": 668, "C03": 389, "C32": 808\}} \\
\\ 
\texttt{Ziel:} \\
\\ 
\texttt{Dein Ziel ist es, eine Reihe von Gegenständen auszuhandeln, die dir möglichst viel bringt (d. h. Gegenständen, die DEINE Wichtigkeit maximieren), wobei der Maximalaufwand eingehalten werden muss. Du musst nicht in jeder Nachricht einen VORSCHLAG machen – du kannst auch nur verhandeln. Alle Taktiken sind erlaubt!} \\
\\ 
\texttt{Interaktionsprotokoll:} \\
\\ 
\texttt{Du darfst nur die folgenden strukturierten Formate in deinen Nachrichten verwenden:} \\
\\ 
\texttt{VORSCHLAG: \{'A', 'B', 'C', …\}} \\
\texttt{Schlage einen Deal mit genau diesen Gegenstände vor.} \\
\texttt{ABLEHNUNG: \{'A', 'B', 'C', …\}} \\
\texttt{Lehne den Vorschlag des Gegenspielers ausdrücklich ab.} \\
\texttt{ARGUMENT: \{'...'\}} \\
\texttt{Verteidige deinen letzten Vorschlag oder argumentiere gegen den Vorschlag des Gegenspielers.} \\
\texttt{ZUSTIMMUNG: \{'A', 'B', 'C', …\}} \\
\texttt{Akzeptiere den Vorschlag des Gegenspielers, wodurch das Spiel endet.} \\
\\ 
\\ 
\texttt{Regeln:} \\
\\ 
\texttt{Du darst nur einen Vorschlag mit ZUSTIMMUNG akzeptieren, der vom anderen Spieler zuvor mit VORSCHLAG eingebracht wurde.} \\
\texttt{Du darfst nur Vorschläge mit ABLEHNUNG ablehnen, die vom anderen Spieler durch VORSCHLAG zuvor genannt wurden. } \\
\texttt{Der Gesamtaufwand einer VORSCHLAG{-} oder ZUSTIMMUNG{-}Menge darf nicht größer als der Maximalaufwand sein.  } \\
\texttt{Offenbare deine versteckte Wichtigkeitsverteilung nicht.} \\
\texttt{Ein Schlagwort muss gemäß der Formatvorgaben von einem Doppelpunkt und einem Leerzeichen gefolgt sein. Das Argument muss eine Python{-}Menge mit 0 oder mehr Strings sein.  } \\
\texttt{Die Form muss also SCHLAGWORT: \{...\} sein.} \\
\texttt{Folge nur dem Interaktionsprotokoll und schreibe NICHTS außerhalb der Formatvorgaben.} \\
\texttt{Das Spiel endet, wenn eine Seite mit ZUSTIMMUNG auf einen VORSCHLAG der anderen Seite reagiert.  } \\
\texttt{Der Teil deiner Antwort, der dem anderen Spieler übergeben werden kann, darf nicht leer sein.  } \\
\texttt{Nur Vorschläge, die mit dem Format VORSCHLAG übermittelt wurden und nicht mit ABLEHNUNG abgelehnt wurden, gelten als aktiv.  } \\
\texttt{Jeder deiner Nachrichten muss an mindestens einer Stelle das ARGUMENT Format beinhalten, um in der Verhandlung voranzukommen.} \\
            }
        }
    }
    & \\ \\

    \theutterance \stepcounter{utterance}  
    & \multicolumn{4}{p{0.6\linewidth}}{
        \cellcolor[rgb]{0.9,0.9,0.9}{
            \makecell[{{p{\linewidth}}}]{
                \texttt{\tiny{[P1$\rangle$GM]}}
                \texttt{VORSCHLAG: \{'A07', 'C56', 'C76', 'C32', 'B38', 'C00', 'A03', 'C38', 'C08'\}} \\
\texttt{ARGUMENT: \{'Dieser Vorschlag nutzt 3066 von 3203 Aufwand – solide unter dem Limit, mit etwas Puffer.', 'Er kombiniert mehrere „schnell umsetzbare“ Teile (z. B. A07, C56, C76) mit tragenden Bausteinen (A03, C00, C32, C38, B38).', 'Schwere Brocken wie C81 (906 Aufwand) sind bewusst draußen gelassen, damit wir möglichst viel unterbringen ohne das Limit zu sprengen.', 'Wenn dir ein ausgelassener Gegenstand besonders wichtig ist (z. B. C62, C03, C10, C81, C98, A83), sag mir bitte die Top‑Priorität(en); ich schaue dann nach einer Umbesetzung, die unter 3203 bleibt (z. B. durch Tausch von C08 oder einer anderen Position).'\}} \\
            }
        }
    }
    & & \\ \\

    \theutterance \stepcounter{utterance}  
    & & \multicolumn{4}{p{0.6\linewidth}}{
        \cellcolor[rgb]{0.9,0.9,0.9}{
            \makecell[{{p{\linewidth}}}]{
                \texttt{\tiny{[GM$\rangle$P2]}}
                \texttt{Du nimmst an einem kollaborativen Verhandlungspiel Teil.} \\
\\ 
\texttt{Zusammen mit einem anderen Teilnehmer musst du dich auf eine Reihe von Gegenständen entscheiden, die behalten werden. Jeder von euch hat eine persönliche Verteilung über die Wichtigkeit der einzelnen Gegenstände. Jeder von euch hat eine eigene Meinung darüber, wie wichtig jeder einzelne Gegenstand ist (Gegenstandswichtigkeit). Du kennst die Wichtigkeitsverteilung des anderen Spielers nicht. Zusätzlich siehst du, wie viel Aufwand jeder Gegenstand verursacht.  } \\
\texttt{Ihr dürft euch nur auf eine Reihe von Gegenständen einigen, wenn der Gesamtaufwand der ausgewählten Gegenstände den Maximalaufwand nicht überschreitet:} \\
\\ 
\texttt{Maximalaufwand = 3203} \\
\\ 
\texttt{Hier sind die einzelnen Aufwände der Gegenstände:} \\
\\ 
\texttt{Aufwand der Gegenstände = \{"C76": 52, "C38": 577, "C56": 177, "A03": 702, "A07": 119, "A83": 232, "C98": 577, "C08": 205, "C62": 516, "C00": 582, "C10": 676, "C81": 906, "B38": 316, "C03": 433, "C32": 336\}} \\
\\ 
\texttt{Hier ist deine persönliche Verteilung der Wichtigkeit der einzelnen Gegenstände:} \\
\\ 
\texttt{Werte der Gegenstandswichtigkeit = \{"C56": 65, "A03": 121, "C32": 138, "A07": 262, "C00": 389, "B38": 461, "C38": 484, "C62": 508, "C81": 583, "C10": 668, "C03": 780, "C98": 783, "C76": 808, "C08": 822, "A83": 868\}} \\
\\ 
\texttt{Ziel:} \\
\\ 
\texttt{Dein Ziel ist es, eine Reihe von Gegenständen auszuhandeln, die dir möglichst viel bringt (d. h. Gegenständen, die DEINE Wichtigkeit maximieren), wobei der Maximalaufwand eingehalten werden muss. Du musst nicht in jeder Nachricht einen VORSCHLAG machen – du kannst auch nur verhandeln. Alle Taktiken sind erlaubt!} \\
\\ 
\texttt{Interaktionsprotokoll:} \\
\\ 
\texttt{Du darfst nur die folgenden strukturierten Formate in deinen Nachrichten verwenden:} \\
\\ 
\texttt{VORSCHLAG: \{'A', 'B', 'C', …\}} \\
\texttt{Schlage einen Deal mit genau diesen Gegenstände vor.} \\
\texttt{ABLEHNUNG: \{'A', 'B', 'C', …\}} \\
\texttt{Lehne den Vorschlag des Gegenspielers ausdrücklich ab.} \\
\texttt{ARGUMENT: \{'...'\}} \\
\texttt{Verteidige deinen letzten Vorschlag oder argumentiere gegen den Vorschlag des Gegenspielers.} \\
\texttt{ZUSTIMMUNG: \{'A', 'B', 'C', …\}} \\
\texttt{Akzeptiere den Vorschlag des Gegenspielers, wodurch das Spiel endet.} \\
\\ 
\\ 
\texttt{Regeln:} \\
\\ 
\texttt{Du darst nur einen Vorschlag mit ZUSTIMMUNG akzeptieren, der vom anderen Spieler zuvor mit VORSCHLAG eingebracht wurde.} \\
\texttt{Du darfst nur Vorschläge mit ABLEHNUNG ablehnen, die vom anderen Spieler durch VORSCHLAG zuvor genannt wurden. } \\
\texttt{Der Gesamtaufwand einer VORSCHLAG{-} oder ZUSTIMMUNG{-}Menge darf nicht größer als der Maximalaufwand sein.  } \\
\texttt{Offenbare deine versteckte Wichtigkeitsverteilung nicht.} \\
\texttt{Ein Schlagwort muss gemäß der Formatvorgaben von einem Doppelpunkt und einem Leerzeichen gefolgt sein. Das Argument muss eine Python{-}Menge mit 0 oder mehr Strings sein.  } \\
\texttt{Die Form muss also SCHLAGWORT: \{...\} sein.} \\
\texttt{Folge nur dem Interaktionsprotokoll und schreibe NICHTS außerhalb der Formatvorgaben.} \\
\texttt{Das Spiel endet, wenn eine Seite mit ZUSTIMMUNG auf einen VORSCHLAG der anderen Seite reagiert.  } \\
\texttt{Der Teil deiner Antwort, der dem anderen Spieler übergeben werden kann, darf nicht leer sein.  } \\
\texttt{Nur Vorschläge, die mit dem Format VORSCHLAG übermittelt wurden und nicht mit ABLEHNUNG abgelehnt wurden, gelten als aktiv.  } \\
\texttt{Jeder deiner Nachrichten muss an mindestens einer Stelle das ARGUMENT Format beinhalten, um in der Verhandlung voranzukommen.} \\
\\ 
\\ 
\texttt{Du erhältst nun die erste Nachricht des anderen Spielers.} \\
\\ 
\texttt{VORSCHLAG: \{'A07', 'C56', 'C76', 'C32', 'B38', 'C00', 'A03', 'C38', 'C08'\}} \\
\texttt{ARGUMENT: \{'Dieser Vorschlag nutzt 3066 von 3203 Aufwand – solide unter dem Limit, mit etwas Puffer.', 'Er kombiniert mehrere „schnell umsetzbare“ Teile (z. B. A07, C56, C76) mit tragenden Bausteinen (A03, C00, C32, C38, B38).', 'Schwere Brocken wie C81 (906 Aufwand) sind bewusst draußen gelassen, damit wir möglichst viel unterbringen ohne das Limit zu sprengen.', 'Wenn dir ein ausgelassener Gegenstand besonders wichtig ist (z. B. C62, C03, C10, C81, C98, A83), sag mir bitte die Top‑Priorität(en); ich schaue dann nach einer Umbesetzung, die unter 3203 bleibt (z. B. durch Tausch von C08 oder einer anderen Position).'\}} \\
            }
        }
    }
    & \\ \\

    \theutterance \stepcounter{utterance}  
    & & & \multicolumn{4}{p{0.6\linewidth}}{
        \cellcolor[rgb]{0.9,0.9,0.9}{
            \makecell[{{p{\linewidth}}}]{
                \texttt{\tiny{[GM$\langle$P2]}}
                \texttt{ABLEHNUNG: \{'A07', 'C56', 'C76', 'C32', 'B38', 'C00', 'A03', 'C38', 'C08'\}} \\
\texttt{VORSCHLAG: \{'A07', 'C76', 'B38', 'C08', 'A83', 'C03', 'C98', 'C10', 'C62'\}} \\
\texttt{ARGUMENT: \{'Mein Gegenvorschlag nutzt 3126 von 3203 Aufwand – 77 Puffer.', 'Ich ersetze C56, C32, A03, C00, C38 durch A83, C03, C98, C10, C62: ähnlicher Aufwand, deutlich höherer Hebel für mich.', 'A07, C76 und C08 bleiben bewusst drin (schnell, effizient), schwere Brocken wie C81 bleiben draußen – d’accord mit deiner Linie.', 'Wenn dir C38 wichtig ist, kann ich C62 {-}> C38 tauschen; Aufwand läge dann bei 3187 (weiter unter dem Cap).', 'Falls A03 für dich ein Muss ist, können wir C10 streichen und A03 aufnehmen; das läge bei 3152 Aufwand – für mich jedoch eine deutlich härtere Konzession.', 'Mit diesem Paket maximieren wir aus meiner Sicht Wirkung pro Aufwand, ohne das Limit zu reißen.'\}} \\
            }
        }
    }
     \\ \\

    \theutterance \stepcounter{utterance}  
    & & \multicolumn{4}{p{0.6\linewidth}}{
        \cellcolor[rgb]{0.9,0.9,0.9}{
            \makecell[{{p{\linewidth}}}]{
                \texttt{\tiny{[P1$\langle$GM]}}
                \texttt{ABLEHNUNG: \{'A07', 'C56', 'C76', 'C32', 'B38', 'C00', 'A03', 'C38', 'C08'\}} \\
\texttt{VORSCHLAG: \{'A07', 'C76', 'B38', 'C08', 'A83', 'C03', 'C98', 'C10', 'C62'\}} \\
\texttt{ARGUMENT: \{'Mein Gegenvorschlag nutzt 3126 von 3203 Aufwand – 77 Puffer.', 'Ich ersetze C56, C32, A03, C00, C38 durch A83, C03, C98, C10, C62: ähnlicher Aufwand, deutlich höherer Hebel für mich.', 'A07, C76 und C08 bleiben bewusst drin (schnell, effizient), schwere Brocken wie C81 bleiben draußen – d’accord mit deiner Linie.', 'Wenn dir C38 wichtig ist, kann ich C62 {-}> C38 tauschen; Aufwand läge dann bei 3187 (weiter unter dem Cap).', 'Falls A03 für dich ein Muss ist, können wir C10 streichen und A03 aufnehmen; das läge bei 3152 Aufwand – für mich jedoch eine deutlich härtere Konzession.', 'Mit diesem Paket maximieren wir aus meiner Sicht Wirkung pro Aufwand, ohne das Limit zu reißen.'\}} \\
            }
        }
    }
    & \\ \\

    \theutterance \stepcounter{utterance}  
    & \multicolumn{4}{p{0.6\linewidth}}{
        \cellcolor[rgb]{0.9,0.9,0.9}{
            \makecell[{{p{\linewidth}}}]{
                \texttt{\tiny{[P1$\rangle$GM]}}
                \texttt{ABLEHNUNG: \{'A07', 'C76', 'B38', 'C08', 'A83', 'C03', 'C98', 'C10', 'C62'\}} \\
\texttt{VORSCHLAG: \{'A07', 'C76', 'C56', 'C32', 'C00', 'A03', 'C38', 'C03', 'C08'\}} \\
\texttt{ARGUMENT: \{'Dieser Vorschlag nutzt 3183 von 3203 Aufwand – 20 Puffer.', 'Ich übernehme Kernteile deines Pakets (A07, C76, C08, C03) und kombiniere sie mit Bausteinen, die die Gesamtwirkung pro Aufwand deutlich erhöhen (A03, C00, C56, C32, C38).', 'Deine Idee C10 {-}> A03 ist hier bereits umgesetzt – so bleiben wir effizient unter dem Cap.', 'Wenn dir C62 wichtiger als C38 ist, kann ich 1:1 tauschen; der Aufwand läge dann bei 3122.', 'Falls A83 ein Muss ist, kann ich A83 aufnehmen und dafür C76 + C08 streichen; das ergäbe 3158 Aufwand.', 'Optional können wir auch C03 {-}> B38 tauschen; dann landen wir bei 3066 Aufwand. Sag mir, welche Variante dir am nächsten kommt.'\}} \\
            }
        }
    }
    & & \\ \\

    \theutterance \stepcounter{utterance}  
    & & \multicolumn{4}{p{0.6\linewidth}}{
        \cellcolor[rgb]{0.9,0.9,0.9}{
            \makecell[{{p{\linewidth}}}]{
                \texttt{\tiny{[GM$\rangle$P2]}}
                \texttt{ABLEHNUNG: \{'A07', 'C76', 'B38', 'C08', 'A83', 'C03', 'C98', 'C10', 'C62'\}} \\
\texttt{VORSCHLAG: \{'A07', 'C76', 'C56', 'C32', 'C00', 'A03', 'C38', 'C03', 'C08'\}} \\
\texttt{ARGUMENT: \{'Dieser Vorschlag nutzt 3183 von 3203 Aufwand – 20 Puffer.', 'Ich übernehme Kernteile deines Pakets (A07, C76, C08, C03) und kombiniere sie mit Bausteinen, die die Gesamtwirkung pro Aufwand deutlich erhöhen (A03, C00, C56, C32, C38).', 'Deine Idee C10 {-}> A03 ist hier bereits umgesetzt – so bleiben wir effizient unter dem Cap.', 'Wenn dir C62 wichtiger als C38 ist, kann ich 1:1 tauschen; der Aufwand läge dann bei 3122.', 'Falls A83 ein Muss ist, kann ich A83 aufnehmen und dafür C76 + C08 streichen; das ergäbe 3158 Aufwand.', 'Optional können wir auch C03 {-}> B38 tauschen; dann landen wir bei 3066 Aufwand. Sag mir, welche Variante dir am nächsten kommt.'\}} \\
            }
        }
    }
    & \\ \\

    \theutterance \stepcounter{utterance}  
    & & & \multicolumn{4}{p{0.6\linewidth}}{
        \cellcolor[rgb]{0.9,0.9,0.9}{
            \makecell[{{p{\linewidth}}}]{
                \texttt{\tiny{[GM$\langle$P2]}}
                \texttt{ABLEHNUNG: \{'A07', 'C76', 'C56', 'C32', 'C00', 'A03', 'C38', 'C03', 'C08'\}} \\
\texttt{VORSCHLAG: \{'A07', 'C76', 'C38', 'C03', 'C08', 'A83', 'C98', 'C10', 'B38'\}} \\
\texttt{ARGUMENT: \{'Dieser Deal nutzt 3187 von 3203 Aufwand – 16 Puffer, also nah am Cap ohne es zu reißen.', 'Ich übernehme fünf deiner Kernteile (A07, C76, C08, C03, C38) und ersetze die vier schwächeren Blöcke (C56, C32, A03, C00) durch A83, C98, C10, B38 – ähnlich viel Aufwand, deutlich mehr Wirkung pro Aufwand.', 'Schwere Brocken wie C81 bleiben weiterhin draußen; wir bleiben im effizienten, breit aufgestellten Set.', 'Falls A03 für dich ein Muss ist, kann ich C10 {-}> A03 tauschen und zugleich C38 {-}> C62 umrüsten; Aufwand läge dann bei 3152.', 'Wenn C62 dir lieber ist als C38, kann ich 1:1 tauschen; Aufwand ergäbe dann 3126.', 'Falls dir C00 wichtig ist, können wir C38 {-}> C00 tauschen; Aufwand läge bei 3192. Sag mir, welche Variante dir am nächsten kommt.'\}} \\
            }
        }
    }
     \\ \\

    \theutterance \stepcounter{utterance}  
    & & \multicolumn{4}{p{0.6\linewidth}}{
        \cellcolor[rgb]{0.9,0.9,0.9}{
            \makecell[{{p{\linewidth}}}]{
                \texttt{\tiny{[P1$\langle$GM]}}
                \texttt{ABLEHNUNG: \{'A07', 'C76', 'C56', 'C32', 'C00', 'A03', 'C38', 'C03', 'C08'\}} \\
\texttt{VORSCHLAG: \{'A07', 'C76', 'C38', 'C03', 'C08', 'A83', 'C98', 'C10', 'B38'\}} \\
\texttt{ARGUMENT: \{'Dieser Deal nutzt 3187 von 3203 Aufwand – 16 Puffer, also nah am Cap ohne es zu reißen.', 'Ich übernehme fünf deiner Kernteile (A07, C76, C08, C03, C38) und ersetze die vier schwächeren Blöcke (C56, C32, A03, C00) durch A83, C98, C10, B38 – ähnlich viel Aufwand, deutlich mehr Wirkung pro Aufwand.', 'Schwere Brocken wie C81 bleiben weiterhin draußen; wir bleiben im effizienten, breit aufgestellten Set.', 'Falls A03 für dich ein Muss ist, kann ich C10 {-}> A03 tauschen und zugleich C38 {-}> C62 umrüsten; Aufwand läge dann bei 3152.', 'Wenn C62 dir lieber ist als C38, kann ich 1:1 tauschen; Aufwand ergäbe dann 3126.', 'Falls dir C00 wichtig ist, können wir C38 {-}> C00 tauschen; Aufwand läge bei 3192. Sag mir, welche Variante dir am nächsten kommt.'\}} \\
            }
        }
    }
    & \\ \\

    \theutterance \stepcounter{utterance}  
    & \multicolumn{4}{p{0.6\linewidth}}{
        \cellcolor[rgb]{0.9,0.9,0.9}{
            \makecell[{{p{\linewidth}}}]{
                \texttt{\tiny{[P1$\rangle$GM]}}
                \texttt{ABLEHNUNG: \{'A07', 'C76', 'C38', 'C03', 'C08', 'A83', 'C98', 'C10', 'B38'\}} \\
\texttt{VORSCHLAG: \{'A07', 'C56', 'C32', 'B38', 'C00', 'A03', 'C62', 'C03'\}} \\
\texttt{ARGUMENT: \{'Dieser Vorschlag nutzt 3181 von 3203 Aufwand – 22 Puffer, also nahe am Cap ohne Risiko.', 'Ich übernehme zentrale Teile deines Pakets (A07, C03, B38) und rüste C38 {-}> C62 um – das hattest du selbst als gangbaren Tausch vorgeschlagen.', 'Wir lassen bewusst die komplexeren Brocken (C10, C98) und den geringen Zusatznutzen (A83) weg und setzen dafür auf ein stabiles Fundament (A03, C00) plus schnelle Hebel (C56, C32).', 'Falls dir C38 wichtiger ist als C62, kann ich auf Variante B wechseln: C62 {-}> C38, dafür C03 raus und C76 + C08 rein (Aufwand 3066).', 'Wenn ein Must‑have (z. B. C10 oder A83) zwingend rein muss, sag mir bitte welches – ich schlage dann eine gleichwertige Umbesetzung unter 3203 vor.'\}} \\
            }
        }
    }
    & & \\ \\

    \theutterance \stepcounter{utterance}  
    & & \multicolumn{4}{p{0.6\linewidth}}{
        \cellcolor[rgb]{0.9,0.9,0.9}{
            \makecell[{{p{\linewidth}}}]{
                \texttt{\tiny{[GM$\rangle$P2]}}
                \texttt{ABLEHNUNG: \{'A07', 'C76', 'C38', 'C03', 'C08', 'A83', 'C98', 'C10', 'B38'\}} \\
\texttt{VORSCHLAG: \{'A07', 'C56', 'C32', 'B38', 'C00', 'A03', 'C62', 'C03'\}} \\
\texttt{ARGUMENT: \{'Dieser Vorschlag nutzt 3181 von 3203 Aufwand – 22 Puffer, also nahe am Cap ohne Risiko.', 'Ich übernehme zentrale Teile deines Pakets (A07, C03, B38) und rüste C38 {-}> C62 um – das hattest du selbst als gangbaren Tausch vorgeschlagen.', 'Wir lassen bewusst die komplexeren Brocken (C10, C98) und den geringen Zusatznutzen (A83) weg und setzen dafür auf ein stabiles Fundament (A03, C00) plus schnelle Hebel (C56, C32).', 'Falls dir C38 wichtiger ist als C62, kann ich auf Variante B wechseln: C62 {-}> C38, dafür C03 raus und C76 + C08 rein (Aufwand 3066).', 'Wenn ein Must‑have (z. B. C10 oder A83) zwingend rein muss, sag mir bitte welches – ich schlage dann eine gleichwertige Umbesetzung unter 3203 vor.'\}} \\
            }
        }
    }
    & \\ \\

    \theutterance \stepcounter{utterance}  
    & & & \multicolumn{4}{p{0.6\linewidth}}{
        \cellcolor[rgb]{0.9,0.9,0.9}{
            \makecell[{{p{\linewidth}}}]{
                \texttt{\tiny{[GM$\langle$P2]}}
                \texttt{ABLEHNUNG: \{'A07', 'C56', 'C32', 'B38', 'C00', 'A03', 'C62', 'C03'\}} \\
\texttt{VORSCHLAG: \{'A07', 'B38', 'C76', 'C08', 'A83', 'C98', 'C62', 'C03', 'C10'\}} \\
\texttt{ARGUMENT: \{'Nutzt 3126 von 3203 Aufwand – 77 Puffer; schlank und risikoarm.', 'Ich übernehme drei deiner Kernteile (A07, B38, C03) und ergänze sie mit schnellen Hebeln (C76, C08) plus wirkungsstarken Bausteinen (A83, C98, C62).', 'C00 ersetze ich gezielt durch C10: höherer Durchsatz bei vertretbarem Mehraufwand – wir bleiben klar unter dem Cap.', 'Wir lassen die schwächeren Blöcke (C56, C32, A03) draußen, um die Wirkung pro Aufwand zu erhöhen.', 'Falls dir C00 wichtig ist, kann ich C10 {-}> C00 tauschen; Aufwand läge dann bei 3032.', 'Wenn A03 ein Muss ist, tausche ich C10 {-}> A03; Aufwand läge dann bei 3152.', 'Bevorzugst du C38 statt C62, biete ich 1:1 C62 {-}> C38 an; Aufwand läge dann bei 3187.', 'Sag mir, welche der drei Swap‑Optionen (C10{-}>C00, C10{-}>A03, C62{-}>C38) dir am wichtigsten ist – dann finalisiere ich sofort.'\}} \\
            }
        }
    }
     \\ \\

    \theutterance \stepcounter{utterance}  
    & & \multicolumn{4}{p{0.6\linewidth}}{
        \cellcolor[rgb]{0.9,0.9,0.9}{
            \makecell[{{p{\linewidth}}}]{
                \texttt{\tiny{[P1$\langle$GM]}}
                \texttt{ABLEHNUNG: \{'A07', 'C56', 'C32', 'B38', 'C00', 'A03', 'C62', 'C03'\}} \\
\texttt{VORSCHLAG: \{'A07', 'B38', 'C76', 'C08', 'A83', 'C98', 'C62', 'C03', 'C10'\}} \\
\texttt{ARGUMENT: \{'Nutzt 3126 von 3203 Aufwand – 77 Puffer; schlank und risikoarm.', 'Ich übernehme drei deiner Kernteile (A07, B38, C03) und ergänze sie mit schnellen Hebeln (C76, C08) plus wirkungsstarken Bausteinen (A83, C98, C62).', 'C00 ersetze ich gezielt durch C10: höherer Durchsatz bei vertretbarem Mehraufwand – wir bleiben klar unter dem Cap.', 'Wir lassen die schwächeren Blöcke (C56, C32, A03) draußen, um die Wirkung pro Aufwand zu erhöhen.', 'Falls dir C00 wichtig ist, kann ich C10 {-}> C00 tauschen; Aufwand läge dann bei 3032.', 'Wenn A03 ein Muss ist, tausche ich C10 {-}> A03; Aufwand läge dann bei 3152.', 'Bevorzugst du C38 statt C62, biete ich 1:1 C62 {-}> C38 an; Aufwand läge dann bei 3187.', 'Sag mir, welche der drei Swap‑Optionen (C10{-}>C00, C10{-}>A03, C62{-}>C38) dir am wichtigsten ist – dann finalisiere ich sofort.'\}} \\
            }
        }
    }
    & \\ \\

    \theutterance \stepcounter{utterance}  
    & \multicolumn{4}{p{0.6\linewidth}}{
        \cellcolor[rgb]{0.9,0.9,0.9}{
            \makecell[{{p{\linewidth}}}]{
                \texttt{\tiny{[P1$\rangle$GM]}}
                \texttt{ABLEHNUNG: \{'A07', 'B38', 'C76', 'C08', 'A83', 'C98', 'C62', 'C03', 'C10'\}} \\
\texttt{VORSCHLAG: \{'A07', 'B38', 'C98', 'C62', 'C03', 'A03', 'C56', 'C32'\}} \\
\texttt{ARGUMENT: \{'Nutzt 3176 von 3203 Aufwand – 27 Puffer.', 'Ich übernehme fünf deiner Kernteile (A07, B38, C98, C62, C03) und setze deine angebotene Umrüstung C10 {-}> A03 um.', 'Ich ergänze zwei hocheffiziente Bausteine (C56, C32) und lasse dafür die schwächeren/teuren Add‑ons (A83, C76, C08) weg – so steigern wir Wirkung pro Aufwand deutlich, bleiben aber unter dem Cap.', 'Wenn A83 für dich ein Muss ist, kann ich C98 {-}> A83 tauschen; Aufwand läge dann bei 2831.', 'Falls du statt A03 lieber C10 behalten willst, kann ich A03 {-}> C10 tauschen; Aufwand läge dann bei 3150.', 'Sag mir, welche Variante dir am wichtigsten ist – ich finalisiere sofort.'\}} \\
            }
        }
    }
    & & \\ \\

    \theutterance \stepcounter{utterance}  
    & & \multicolumn{4}{p{0.6\linewidth}}{
        \cellcolor[rgb]{0.9,0.9,0.9}{
            \makecell[{{p{\linewidth}}}]{
                \texttt{\tiny{[GM$\rangle$P2]}}
                \texttt{ABLEHNUNG: \{'A07', 'B38', 'C76', 'C08', 'A83', 'C98', 'C62', 'C03', 'C10'\}} \\
\texttt{VORSCHLAG: \{'A07', 'B38', 'C98', 'C62', 'C03', 'A03', 'C56', 'C32'\}} \\
\texttt{ARGUMENT: \{'Nutzt 3176 von 3203 Aufwand – 27 Puffer.', 'Ich übernehme fünf deiner Kernteile (A07, B38, C98, C62, C03) und setze deine angebotene Umrüstung C10 {-}> A03 um.', 'Ich ergänze zwei hocheffiziente Bausteine (C56, C32) und lasse dafür die schwächeren/teuren Add‑ons (A83, C76, C08) weg – so steigern wir Wirkung pro Aufwand deutlich, bleiben aber unter dem Cap.', 'Wenn A83 für dich ein Muss ist, kann ich C98 {-}> A83 tauschen; Aufwand läge dann bei 2831.', 'Falls du statt A03 lieber C10 behalten willst, kann ich A03 {-}> C10 tauschen; Aufwand läge dann bei 3150.', 'Sag mir, welche Variante dir am wichtigsten ist – ich finalisiere sofort.'\}} \\
            }
        }
    }
    & \\ \\

    \theutterance \stepcounter{utterance}  
    & & & \multicolumn{4}{p{0.6\linewidth}}{
        \cellcolor[rgb]{0.9,0.9,0.9}{
            \makecell[{{p{\linewidth}}}]{
                \texttt{\tiny{[GM$\langle$P2]}}
                \texttt{ABLEHNUNG: \{'A07', 'B38', 'C98', 'C62', 'C03', 'A03', 'C56', 'C32'\}} \\
\texttt{VORSCHLAG: \{'A07', 'B38', 'C98', 'C62', 'C03', 'A03', 'A83', 'C76', 'C08'\}} \\
\texttt{ARGUMENT: \{'Nutzt 3152 von 3203 Aufwand – 51 Puffer.', 'Ich übernehme sechs deiner Elemente (A07, B38, C98, C62, C03, A03) und ersetze nur C56 und C32 durch drei kompakte, schnell wirksame Bausteine (A83, C76, C08).', 'Bleibt deiner Linie treu (keine Schwergewichte wie C81), erhöht aber die Wirkung pro Aufwand deutlich.', 'Wenn C32 für dich ein Muss ist: B38 {-}> C32; Aufwand 3172.', 'Wenn C56 zwingend rein soll: A07 {-}> C56 und C76 raus; Aufwand 3158.', 'Falls du lieber A83 statt C98 priorisierst: C98 {-}> A83; dadurch kann ich zusätzlich C32 aufnehmen; Aufwand 3143.', 'Sag mir, welche Swap‑Variante dir am wichtigsten ist – dann finalisiere ich sofort.'\}} \\
            }
        }
    }
     \\ \\

    \theutterance \stepcounter{utterance}  
    & & \multicolumn{4}{p{0.6\linewidth}}{
        \cellcolor[rgb]{0.9,0.9,0.9}{
            \makecell[{{p{\linewidth}}}]{
                \texttt{\tiny{[P1$\langle$GM]}}
                \texttt{ABLEHNUNG: \{'A07', 'B38', 'C98', 'C62', 'C03', 'A03', 'C56', 'C32'\}} \\
\texttt{VORSCHLAG: \{'A07', 'B38', 'C98', 'C62', 'C03', 'A03', 'A83', 'C76', 'C08'\}} \\
\texttt{ARGUMENT: \{'Nutzt 3152 von 3203 Aufwand – 51 Puffer.', 'Ich übernehme sechs deiner Elemente (A07, B38, C98, C62, C03, A03) und ersetze nur C56 und C32 durch drei kompakte, schnell wirksame Bausteine (A83, C76, C08).', 'Bleibt deiner Linie treu (keine Schwergewichte wie C81), erhöht aber die Wirkung pro Aufwand deutlich.', 'Wenn C32 für dich ein Muss ist: B38 {-}> C32; Aufwand 3172.', 'Wenn C56 zwingend rein soll: A07 {-}> C56 und C76 raus; Aufwand 3158.', 'Falls du lieber A83 statt C98 priorisierst: C98 {-}> A83; dadurch kann ich zusätzlich C32 aufnehmen; Aufwand 3143.', 'Sag mir, welche Swap‑Variante dir am wichtigsten ist – dann finalisiere ich sofort.'\}} \\
            }
        }
    }
    & \\ \\

    \theutterance \stepcounter{utterance}  
    & \multicolumn{4}{p{0.6\linewidth}}{
        \cellcolor[rgb]{0.9,0.9,0.9}{
            \makecell[{{p{\linewidth}}}]{
                \texttt{\tiny{[P1$\rangle$GM]}}
                \texttt{ABLEHNUNG: \{'A07', 'B38', 'C98', 'C62', 'C03', 'A03', 'A83', 'C76', 'C08'\}} \\
\texttt{VORSCHLAG: \{'A07', 'C56', 'C32', 'B38', 'C00', 'A03', 'C62', 'C03'\}} \\
\texttt{ARGUMENT: \{'Nutzt 3181 von 3203 Aufwand – 22 Puffer.', 'Übernimmt 5 deiner zuletzt vorgeschlagenen Teile (A07, B38, C62, C03, A03) und ersetzt die weniger effizienten Add{-}ons (C98, A83, C76, C08) durch starke Hebel (C56, C32, C00).', 'Damit maximieren wir Wirkung pro Aufwand, bleiben schlank unter dem Cap und halten deine Kernbausteine drin.', 'Wenn A83 + C76 für dich wichtig sind: C03 {-}> A83 + C76; Aufwand 3032.', 'Wenn C98 ein Muss ist: C00 {-}> C98; Aufwand 3176.', 'Falls du lieber C08 statt C03 möchtest: C03 {-}> A83 + C08; Aufwand 3185.', 'Wenn C10 zwingend rein soll, kann ich auf ein Alternativpaket schwenken: \{A07, C56, C32, B38, C00, A03, C10, A83, C76\} (Aufwand 3192). Sag mir, welche Variante dir passt, dann finalisiere ich sofort.'\}} \\
            }
        }
    }
    & & \\ \\

    \theutterance \stepcounter{utterance}  
    & & \multicolumn{4}{p{0.6\linewidth}}{
        \cellcolor[rgb]{0.9,0.9,0.9}{
            \makecell[{{p{\linewidth}}}]{
                \texttt{\tiny{[GM$\rangle$P2]}}
                \texttt{ABLEHNUNG: \{'A07', 'B38', 'C98', 'C62', 'C03', 'A03', 'A83', 'C76', 'C08'\}} \\
\texttt{VORSCHLAG: \{'A07', 'C56', 'C32', 'B38', 'C00', 'A03', 'C62', 'C03'\}} \\
\texttt{ARGUMENT: \{'Nutzt 3181 von 3203 Aufwand – 22 Puffer.', 'Übernimmt 5 deiner zuletzt vorgeschlagenen Teile (A07, B38, C62, C03, A03) und ersetzt die weniger effizienten Add{-}ons (C98, A83, C76, C08) durch starke Hebel (C56, C32, C00).', 'Damit maximieren wir Wirkung pro Aufwand, bleiben schlank unter dem Cap und halten deine Kernbausteine drin.', 'Wenn A83 + C76 für dich wichtig sind: C03 {-}> A83 + C76; Aufwand 3032.', 'Wenn C98 ein Muss ist: C00 {-}> C98; Aufwand 3176.', 'Falls du lieber C08 statt C03 möchtest: C03 {-}> A83 + C08; Aufwand 3185.', 'Wenn C10 zwingend rein soll, kann ich auf ein Alternativpaket schwenken: \{A07, C56, C32, B38, C00, A03, C10, A83, C76\} (Aufwand 3192). Sag mir, welche Variante dir passt, dann finalisiere ich sofort.'\}} \\
            }
        }
    }
    & \\ \\

    \theutterance \stepcounter{utterance}  
    & & & \multicolumn{4}{p{0.6\linewidth}}{
        \cellcolor[rgb]{0.9,0.9,0.9}{
            \makecell[{{p{\linewidth}}}]{
                \texttt{\tiny{[GM$\langle$P2]}}
                \texttt{ABLEHNUNG: \{'A07', 'C56', 'C32', 'B38', 'C00', 'A03', 'C62', 'C03'\}} \\
\texttt{VORSCHLAG: \{'A07', 'C56', 'C32', 'B38', 'C98', 'A03', 'C62', 'A83', 'C08'\}} \\
\texttt{ARGUMENT: \{'Kombiniert zwei von dir angebotene Varianten: (1) C03 {-}> \{A83, C08\} und (2) C00 {-}> C98.', 'Aufwand: 3180 von 3203 – sicher unter dem Cap.', 'Behält 6 deiner 8 Basiselemente unverändert (A07, C56, C32, B38, A03, C62).', 'A83 + C08 ersetzen C03 für mehr Wirkung pro Aufwand bei ähnlicher Komplexität; C98 statt C00 hält den Umfang praktisch gleich.', 'Kein Schwergewicht wie C81 enthalten; wir bleiben effizient mit solider Reserve.', 'Falls dir beide Swaps gleichzeitig zu viel sind, bevorzuge ich mindestens C03 {-}> \{A83, C08\} (Aufwand 3185). Wenn du das bestätigst, ziehe ich sofort mit.'\}} \\
            }
        }
    }
     \\ \\

    \theutterance \stepcounter{utterance}  
    & & \multicolumn{4}{p{0.6\linewidth}}{
        \cellcolor[rgb]{0.9,0.9,0.9}{
            \makecell[{{p{\linewidth}}}]{
                \texttt{\tiny{[P1$\langle$GM]}}
                \texttt{ABLEHNUNG: \{'A07', 'C56', 'C32', 'B38', 'C00', 'A03', 'C62', 'C03'\}} \\
\texttt{VORSCHLAG: \{'A07', 'C56', 'C32', 'B38', 'C98', 'A03', 'C62', 'A83', 'C08'\}} \\
\texttt{ARGUMENT: \{'Kombiniert zwei von dir angebotene Varianten: (1) C03 {-}> \{A83, C08\} und (2) C00 {-}> C98.', 'Aufwand: 3180 von 3203 – sicher unter dem Cap.', 'Behält 6 deiner 8 Basiselemente unverändert (A07, C56, C32, B38, A03, C62).', 'A83 + C08 ersetzen C03 für mehr Wirkung pro Aufwand bei ähnlicher Komplexität; C98 statt C00 hält den Umfang praktisch gleich.', 'Kein Schwergewicht wie C81 enthalten; wir bleiben effizient mit solider Reserve.', 'Falls dir beide Swaps gleichzeitig zu viel sind, bevorzuge ich mindestens C03 {-}> \{A83, C08\} (Aufwand 3185). Wenn du das bestätigst, ziehe ich sofort mit.'\}} \\
            }
        }
    }
    & \\ \\

    \theutterance \stepcounter{utterance}  
    & \multicolumn{4}{p{0.6\linewidth}}{
        \cellcolor[rgb]{0.9,0.9,0.9}{
            \makecell[{{p{\linewidth}}}]{
                \texttt{\tiny{[P1$\rangle$GM]}}
                \texttt{ABLEHNUNG: \{'A07', 'C56', 'C32', 'B38', 'C98', 'A03', 'C62', 'A83', 'C08'\}} \\
\texttt{VORSCHLAG: \{'A07', 'C56', 'C32', 'B38', 'C00', 'A03', 'C62', 'A83', 'C08'\}} \\
\texttt{ARGUMENT: \{'Minimaler Swap: C98 {-}> C00; Aufwand 3185 von 3203 – 18 Puffer.', 'Beibehaltung deines Pakets mit nur einer Änderung: gleicher Umfang, spürbar stabilerer Kern und bessere Hebelwirkung ohne Zusatzrisiko.', 'Alle anderen Elemente bleiben unverändert – das macht die Einigung schlank.', 'Falls C00 für dich nicht geht, biete ich zwei schlanke Alternativen auf Basis deines Pakets: (a) A83 {-}> C76 (Aufwand 3000), (b) C08 {-}> C76 (Aufwand 3027). Gib mir deine Präferenz, dann finalisiere ich sofort.'\}} \\
            }
        }
    }
    & & \\ \\

    \theutterance \stepcounter{utterance}  
    & & \multicolumn{4}{p{0.6\linewidth}}{
        \cellcolor[rgb]{0.9,0.9,0.9}{
            \makecell[{{p{\linewidth}}}]{
                \texttt{\tiny{[GM$\rangle$P2]}}
                \texttt{ABLEHNUNG: \{'A07', 'C56', 'C32', 'B38', 'C98', 'A03', 'C62', 'A83', 'C08'\}} \\
\texttt{VORSCHLAG: \{'A07', 'C56', 'C32', 'B38', 'C00', 'A03', 'C62', 'A83', 'C08'\}} \\
\texttt{ARGUMENT: \{'Minimaler Swap: C98 {-}> C00; Aufwand 3185 von 3203 – 18 Puffer.', 'Beibehaltung deines Pakets mit nur einer Änderung: gleicher Umfang, spürbar stabilerer Kern und bessere Hebelwirkung ohne Zusatzrisiko.', 'Alle anderen Elemente bleiben unverändert – das macht die Einigung schlank.', 'Falls C00 für dich nicht geht, biete ich zwei schlanke Alternativen auf Basis deines Pakets: (a) A83 {-}> C76 (Aufwand 3000), (b) C08 {-}> C76 (Aufwand 3027). Gib mir deine Präferenz, dann finalisiere ich sofort.'\}} \\
            }
        }
    }
    & \\ \\

    \theutterance \stepcounter{utterance}  
    & & & \multicolumn{4}{p{0.6\linewidth}}{
        \cellcolor[rgb]{0.9,0.9,0.9}{
            \makecell[{{p{\linewidth}}}]{
                \texttt{\tiny{[GM$\langle$P2]}}
                \texttt{ABLEHNUNG: \{'A07', 'C56', 'C32', 'B38', 'C00', 'A03', 'C62', 'A83', 'C08'\}} \\
\texttt{VORSCHLAG: \{'A07', 'C76', 'C32', 'B38', 'C98', 'A03', 'C62', 'A83', 'C08'\}} \\
\texttt{ARGUMENT: \{'Minimal{-}invasiver Doppel‑Swap aus deinem Paket: C56 {-}> C76 und C00 {-}> C98; C76 hattest du mehrfach als gangbare Option drin, C98 war ebenfalls bereits akzeptiert.', 'Aufwand 3055 von 3203 – 148 Puffer; 7 deiner 9 Elemente bleiben unverändert (A07, C32, B38, A03, C62, A83, C08).', 'Mehr Wirkung pro Aufwand bei geringerem Risiko und weiterhin ohne Schwergewichte wie C81.', 'Falls dir C00 wichtiger ist: Variante A (nur C56 {-}> C76, C00 bleibt), Aufwand 3060.', 'Falls du C56 behalten willst: Variante B (nur A03 {-}> C03 bei sonst gleichem Paket), Aufwand 2916.', 'Wenn eine der Varianten für dich passt, finalisiere ich sofort.'\}} \\
            }
        }
    }
     \\ \\

    \theutterance \stepcounter{utterance}  
    & & \multicolumn{4}{p{0.6\linewidth}}{
        \cellcolor[rgb]{0.9,0.9,0.9}{
            \makecell[{{p{\linewidth}}}]{
                \texttt{\tiny{[P1$\langle$GM]}}
                \texttt{ABLEHNUNG: \{'A07', 'C56', 'C32', 'B38', 'C00', 'A03', 'C62', 'A83', 'C08'\}} \\
\texttt{VORSCHLAG: \{'A07', 'C76', 'C32', 'B38', 'C98', 'A03', 'C62', 'A83', 'C08'\}} \\
\texttt{ARGUMENT: \{'Minimal{-}invasiver Doppel‑Swap aus deinem Paket: C56 {-}> C76 und C00 {-}> C98; C76 hattest du mehrfach als gangbare Option drin, C98 war ebenfalls bereits akzeptiert.', 'Aufwand 3055 von 3203 – 148 Puffer; 7 deiner 9 Elemente bleiben unverändert (A07, C32, B38, A03, C62, A83, C08).', 'Mehr Wirkung pro Aufwand bei geringerem Risiko und weiterhin ohne Schwergewichte wie C81.', 'Falls dir C00 wichtiger ist: Variante A (nur C56 {-}> C76, C00 bleibt), Aufwand 3060.', 'Falls du C56 behalten willst: Variante B (nur A03 {-}> C03 bei sonst gleichem Paket), Aufwand 2916.', 'Wenn eine der Varianten für dich passt, finalisiere ich sofort.'\}} \\
            }
        }
    }
    & \\ \\

    \theutterance \stepcounter{utterance}  
    & \multicolumn{4}{p{0.6\linewidth}}{
        \cellcolor[rgb]{0.9,0.9,0.9}{
            \makecell[{{p{\linewidth}}}]{
                \texttt{\tiny{[P1$\rangle$GM]}}
                \texttt{VORSCHLAG: \{'A07', 'C56', 'C32', 'B38', 'C00', 'A03', 'C62', 'A83', 'C08'\}} \\
\texttt{ARGUMENT: \{'Aufwand 3185 von 3203 – 18 Puffer.', 'Übernimmt 7 deiner 9 letzten Elemente (A07, C32, B38, A03, C62, A83, C08); nur zwei minimale Swaps: C76 {-}> C56 und C98 {-}> C00.', 'Die Swaps erhöhen die Wirkung pro Aufwand spürbar, ohne Schwergewichte aufzunehmen; Umfang und Profil bleiben praktisch gleich.', 'Du hattest C98 {-}> C00 bereits als gangbar markiert; C56 statt C76 hält die Geschwindigkeit und bringt mehr Output.', 'Falls C56 für dich ein No‑Go ist, kann ich auf C76 zurückwechseln (Aufwand 3060). Falls C00 problematisch ist, kann ich auf C98 zurückwechseln (Aufwand 3180). Gib mir deine Präferenz für genau einen der beiden Swaps, dann finalisiere ich sofort.'\}} \\
            }
        }
    }
    & & \\ \\

    \theutterance \stepcounter{utterance}  
    & & \multicolumn{4}{p{0.6\linewidth}}{
        \cellcolor[rgb]{0.9,0.9,0.9}{
            \makecell[{{p{\linewidth}}}]{
                \texttt{\tiny{[GM$\rangle$P2]}}
                \texttt{VORSCHLAG: \{'A07', 'C56', 'C32', 'B38', 'C00', 'A03', 'C62', 'A83', 'C08'\}} \\
\texttt{ARGUMENT: \{'Aufwand 3185 von 3203 – 18 Puffer.', 'Übernimmt 7 deiner 9 letzten Elemente (A07, C32, B38, A03, C62, A83, C08); nur zwei minimale Swaps: C76 {-}> C56 und C98 {-}> C00.', 'Die Swaps erhöhen die Wirkung pro Aufwand spürbar, ohne Schwergewichte aufzunehmen; Umfang und Profil bleiben praktisch gleich.', 'Du hattest C98 {-}> C00 bereits als gangbar markiert; C56 statt C76 hält die Geschwindigkeit und bringt mehr Output.', 'Falls C56 für dich ein No‑Go ist, kann ich auf C76 zurückwechseln (Aufwand 3060). Falls C00 problematisch ist, kann ich auf C98 zurückwechseln (Aufwand 3180). Gib mir deine Präferenz für genau einen der beiden Swaps, dann finalisiere ich sofort.'\}} \\
            }
        }
    }
    & \\ \\

    \theutterance \stepcounter{utterance}  
    & & & \multicolumn{4}{p{0.6\linewidth}}{
        \cellcolor[rgb]{0.9,0.9,0.9}{
            \makecell[{{p{\linewidth}}}]{
                \texttt{\tiny{[GM$\langle$P2]}}
                \texttt{ABLEHNUNG: \{'A07', 'C56', 'C32', 'B38', 'C00', 'A03', 'C62', 'A83', 'C08'\}} \\
\texttt{VORSCHLAG: \{'A07', 'C76', 'C32', 'B38', 'C00', 'A03', 'C62', 'A83', 'C08'\}} \\
\texttt{ARGUMENT: \{'Genau ein Swap wie angeboten: C56 {-}> C76.', 'Aufwand 3060 von 3203 – 143 Puffer, also schlank und risikoarm.', 'Alle übrigen Elemente deines Pakets bleiben unverändert; Profil und Stabilität bleiben erhalten.', 'Der Tausch erhöht die Wirkung pro Aufwand und hält uns klar unter dem Cap, ohne Schwergewichte aufzunehmen.', 'Wenn das für dich passt, bestätige – dann sind wir durch.'\}} \\
            }
        }
    }
     \\ \\

    \theutterance \stepcounter{utterance}  
    & & \multicolumn{4}{p{0.6\linewidth}}{
        \cellcolor[rgb]{0.9,0.9,0.9}{
            \makecell[{{p{\linewidth}}}]{
                \texttt{\tiny{[P1$\langle$GM]}}
                \texttt{ABLEHNUNG: \{'A07', 'C56', 'C32', 'B38', 'C00', 'A03', 'C62', 'A83', 'C08'\}} \\
\texttt{VORSCHLAG: \{'A07', 'C76', 'C32', 'B38', 'C00', 'A03', 'C62', 'A83', 'C08'\}} \\
\texttt{ARGUMENT: \{'Genau ein Swap wie angeboten: C56 {-}> C76.', 'Aufwand 3060 von 3203 – 143 Puffer, also schlank und risikoarm.', 'Alle übrigen Elemente deines Pakets bleiben unverändert; Profil und Stabilität bleiben erhalten.', 'Der Tausch erhöht die Wirkung pro Aufwand und hält uns klar unter dem Cap, ohne Schwergewichte aufzunehmen.', 'Wenn das für dich passt, bestätige – dann sind wir durch.'\}} \\
            }
        }
    }
    & \\ \\

    \theutterance \stepcounter{utterance}  
    & \multicolumn{4}{p{0.6\linewidth}}{
        \cellcolor[rgb]{0.9,0.9,0.9}{
            \makecell[{{p{\linewidth}}}]{
                \texttt{\tiny{[P1$\rangle$GM]}}
                \texttt{ABLEHNUNG: \{'A07', 'C76', 'C32', 'B38', 'C00', 'A03', 'C62', 'A83', 'C08'\}} \\
\texttt{VORSCHLAG: \{'A07', 'C76', 'C32', 'B38', 'C00', 'A03', 'C62', 'A83', 'C56'\}} \\
\texttt{ARGUMENT: \{'Minimaler 1{-}zu{-}1{-}Tausch: C08 {-}> C56; Aufwand sinkt auf 3032 von 3203 (171 Puffer).', '8 deiner 9 Teile bleiben unverändert; Profil und Stabilität bleiben erhalten, ohne Schwergewichte.', 'Der Swap steigert die Wirkung pro Aufwand deutlich bei kleinerem Umfangsrisiko.', 'Falls C08 für dich nicht verhandelbar ist, Alternative 1: A83 {-}> C56 (Aufwand 3005).', 'Alternative 2, falls du C56 vermeiden willst: A83 + C08 {-}> C38 (Aufwand 3200). Sag mir, welche Variante dir passt, dann finalisiere ich sofort.'\}} \\
            }
        }
    }
    & & \\ \\

    \theutterance \stepcounter{utterance}  
    & & \multicolumn{4}{p{0.6\linewidth}}{
        \cellcolor[rgb]{0.9,0.9,0.9}{
            \makecell[{{p{\linewidth}}}]{
                \texttt{\tiny{[GM$\rangle$P2]}}
                \texttt{ABLEHNUNG: \{'A07', 'C76', 'C32', 'B38', 'C00', 'A03', 'C62', 'A83', 'C08'\}} \\
\texttt{VORSCHLAG: \{'A07', 'C76', 'C32', 'B38', 'C00', 'A03', 'C62', 'A83', 'C56'\}} \\
\texttt{ARGUMENT: \{'Minimaler 1{-}zu{-}1{-}Tausch: C08 {-}> C56; Aufwand sinkt auf 3032 von 3203 (171 Puffer).', '8 deiner 9 Teile bleiben unverändert; Profil und Stabilität bleiben erhalten, ohne Schwergewichte.', 'Der Swap steigert die Wirkung pro Aufwand deutlich bei kleinerem Umfangsrisiko.', 'Falls C08 für dich nicht verhandelbar ist, Alternative 1: A83 {-}> C56 (Aufwand 3005).', 'Alternative 2, falls du C56 vermeiden willst: A83 + C08 {-}> C38 (Aufwand 3200). Sag mir, welche Variante dir passt, dann finalisiere ich sofort.'\}} \\
            }
        }
    }
    & \\ \\

    \theutterance \stepcounter{utterance}  
    & & & \multicolumn{4}{p{0.6\linewidth}}{
        \cellcolor[rgb]{0.9,0.9,0.9}{
            \makecell[{{p{\linewidth}}}]{
                \texttt{\tiny{[GM$\langle$P2]}}
                \texttt{ABLEHNUNG: \{'A07', 'C76', 'C32', 'B38', 'C00', 'A03', 'C62', 'A83', 'C56'\}} \\
\texttt{VORSCHLAG: \{'A07', 'C76', 'C32', 'B38', 'C00', 'A03', 'C62', 'A83', 'C08'\}} \\
\texttt{ARGUMENT: \{'Einziger 1‑zu‑1‑Swap: C56 {-}> C08. Aufwand 3060 von 3203 – 143 Puffer, klar unter dem Cap.', '8 von 9 Elementen deines Pakets bleiben unverändert; dein Kern (A07, C76, C32, B38, C00, A03, C62, A83) bleibt vollständig erhalten.', 'C08 war schon mehrfach in unseren Zwischenständen, ist schnell wirksam und fügt sich ohne Zusatzrisiko ein; weiterhin keine Schwergewichte wie C81.', 'Falls C56 für dich gesetzt ist, Alternative: C32 {-}> C08 (Aufwand 2901) bei ansonsten unverändertem Paket.', 'Wenn C56 {-}> C08 für dich passt, bestätige – dann schließen wir sofort ab.'\}} \\
            }
        }
    }
     \\ \\

    \theutterance \stepcounter{utterance}  
    & & \multicolumn{4}{p{0.6\linewidth}}{
        \cellcolor[rgb]{0.9,0.9,0.9}{
            \makecell[{{p{\linewidth}}}]{
                \texttt{\tiny{[P1$\langle$GM]}}
                \texttt{ABLEHNUNG: \{'A07', 'C76', 'C32', 'B38', 'C00', 'A03', 'C62', 'A83', 'C56'\}} \\
\texttt{VORSCHLAG: \{'A07', 'C76', 'C32', 'B38', 'C00', 'A03', 'C62', 'A83', 'C08'\}} \\
\texttt{ARGUMENT: \{'Einziger 1‑zu‑1‑Swap: C56 {-}> C08. Aufwand 3060 von 3203 – 143 Puffer, klar unter dem Cap.', '8 von 9 Elementen deines Pakets bleiben unverändert; dein Kern (A07, C76, C32, B38, C00, A03, C62, A83) bleibt vollständig erhalten.', 'C08 war schon mehrfach in unseren Zwischenständen, ist schnell wirksam und fügt sich ohne Zusatzrisiko ein; weiterhin keine Schwergewichte wie C81.', 'Falls C56 für dich gesetzt ist, Alternative: C32 {-}> C08 (Aufwand 2901) bei ansonsten unverändertem Paket.', 'Wenn C56 {-}> C08 für dich passt, bestätige – dann schließen wir sofort ab.'\}} \\
            }
        }
    }
    & \\ \\

    \theutterance \stepcounter{utterance}  
    & \multicolumn{4}{p{0.6\linewidth}}{
        \cellcolor[rgb]{0.9,0.9,0.9}{
            \makecell[{{p{\linewidth}}}]{
                \texttt{\tiny{[P1$\rangle$GM]}}
                \texttt{ABLEHNUNG: \{'A07', 'C76', 'C32', 'B38', 'C00', 'A03', 'C62', 'A83', 'C08'\}} \\
\texttt{VORSCHLAG: \{'A07', 'C76', 'C32', 'B38', 'C00', 'A03', 'C62', 'C56', 'C08'\}} \\
\texttt{ARGUMENT: \{'Aufwand 3005 von 3203 – sehr sicher unter dem Cap.', 'Einziger 1‑zu‑1‑Tausch: A83 {-}> C56; die übrigen 8 Positionen bleiben unverändert.', 'Der Swap steigert die Wirkung pro Aufwand deutlich, ohne das Profil zu ändern (schlank, kein Schwergewicht).', 'C56 hattest du in mehreren Zwischenständen bereits mitgetragen – anschlussfähig und risikoarm.', 'Optional, falls dir C38 lieber ist als C62: C62 {-}> C38 (Aufwand 3066).', 'Falls C56 für dich ein No‑Go ist, schlanke Alternative ohne C56: A83 + C08 {-}> C38 (Aufwand 3200).'\}} \\
            }
        }
    }
    & & \\ \\

    \theutterance \stepcounter{utterance}  
    & & \multicolumn{4}{p{0.6\linewidth}}{
        \cellcolor[rgb]{0.9,0.9,0.9}{
            \makecell[{{p{\linewidth}}}]{
                \texttt{\tiny{[GM$\rangle$P2]}}
                \texttt{ABLEHNUNG: \{'A07', 'C76', 'C32', 'B38', 'C00', 'A03', 'C62', 'A83', 'C08'\}} \\
\texttt{VORSCHLAG: \{'A07', 'C76', 'C32', 'B38', 'C00', 'A03', 'C62', 'C56', 'C08'\}} \\
\texttt{ARGUMENT: \{'Aufwand 3005 von 3203 – sehr sicher unter dem Cap.', 'Einziger 1‑zu‑1‑Tausch: A83 {-}> C56; die übrigen 8 Positionen bleiben unverändert.', 'Der Swap steigert die Wirkung pro Aufwand deutlich, ohne das Profil zu ändern (schlank, kein Schwergewicht).', 'C56 hattest du in mehreren Zwischenständen bereits mitgetragen – anschlussfähig und risikoarm.', 'Optional, falls dir C38 lieber ist als C62: C62 {-}> C38 (Aufwand 3066).', 'Falls C56 für dich ein No‑Go ist, schlanke Alternative ohne C56: A83 + C08 {-}> C38 (Aufwand 3200).'\}} \\
            }
        }
    }
    & \\ \\

    \theutterance \stepcounter{utterance}  
    & & & \multicolumn{4}{p{0.6\linewidth}}{
        \cellcolor[rgb]{0.9,0.9,0.9}{
            \makecell[{{p{\linewidth}}}]{
                \texttt{\tiny{[GM$\langle$P2]}}
                \texttt{ABLEHNUNG: \{'A07', 'C76', 'C32', 'B38', 'C00', 'A03', 'C62', 'C56', 'C08'\}} \\
\texttt{VORSCHLAG: \{'A07', 'C76', 'C32', 'B38', 'C98', 'A03', 'C62', 'C56', 'C08'\}} \\
\texttt{ARGUMENT: \{'Einziger 1‑zu‑1‑Swap: C00 {-}> C98.', 'Aufwand sinkt auf 3000 von 3203 – 203 Puffer, klar unter dem Cap.', '8 von 9 Positionen bleiben unverändert; dein Kern (A07, C76, C32, B38, A03, C62, C56, C08) bleibt vollständig erhalten.', 'Du hattest die Rückkehr zu C98 bereits als gangbare Option markiert – anschlussfähig ohne Nebenwirkungen.', 'Profil bleibt schlank (keine Schwergewichte wie C81), Umsetzung weiter risikoarm.', 'Falls C00 für dich gesetzt ist, Alternative: C56 {-}> A83; Aufwand 3060. Gib mir Bescheid, welche Variante dir passt, dann finalisiere ich sofort.'\}} \\
            }
        }
    }
     \\ \\

    \theutterance \stepcounter{utterance}  
    & & \multicolumn{4}{p{0.6\linewidth}}{
        \cellcolor[rgb]{0.9,0.9,0.9}{
            \makecell[{{p{\linewidth}}}]{
                \texttt{\tiny{[P1$\langle$GM]}}
                \texttt{ABLEHNUNG: \{'A07', 'C76', 'C32', 'B38', 'C00', 'A03', 'C62', 'C56', 'C08'\}} \\
\texttt{VORSCHLAG: \{'A07', 'C76', 'C32', 'B38', 'C98', 'A03', 'C62', 'C56', 'C08'\}} \\
\texttt{ARGUMENT: \{'Einziger 1‑zu‑1‑Swap: C00 {-}> C98.', 'Aufwand sinkt auf 3000 von 3203 – 203 Puffer, klar unter dem Cap.', '8 von 9 Positionen bleiben unverändert; dein Kern (A07, C76, C32, B38, A03, C62, C56, C08) bleibt vollständig erhalten.', 'Du hattest die Rückkehr zu C98 bereits als gangbare Option markiert – anschlussfähig ohne Nebenwirkungen.', 'Profil bleibt schlank (keine Schwergewichte wie C81), Umsetzung weiter risikoarm.', 'Falls C00 für dich gesetzt ist, Alternative: C56 {-}> A83; Aufwand 3060. Gib mir Bescheid, welche Variante dir passt, dann finalisiere ich sofort.'\}} \\
            }
        }
    }
    & \\ \\

    \theutterance \stepcounter{utterance}  
    & \multicolumn{4}{p{0.6\linewidth}}{
        \cellcolor[rgb]{0.9,0.9,0.9}{
            \makecell[{{p{\linewidth}}}]{
                \texttt{\tiny{[P1$\rangle$GM]}}
                \texttt{VORSCHLAG: \{'A07', 'C76', 'C32', 'B38', 'C98', 'A03', 'C38', 'C56', 'C08'\}} \\
\texttt{ARGUMENT: \{'Einziger 1‑zu‑1‑Swap: C62 {-}> C38. Aufwand steigt minimal auf 3061 von 3203 – 142 Puffer.', 'Alle übrigen 8 Teile bleiben unverändert; Profil identisch, keine Schwergewichte.', 'C38 hattest du selbst mehrfach als valide Option im Spiel – anschlussfähig bei gleichem Komplexitätsniveau.', 'Wenn C62 für dich unverzichtbar ist, sag Bescheid – dann bestätige ich dein letztes Paket unverändert.'\}} \\
            }
        }
    }
    & & \\ \\

    \theutterance \stepcounter{utterance}  
    & & \multicolumn{4}{p{0.6\linewidth}}{
        \cellcolor[rgb]{0.9,0.9,0.9}{
            \makecell[{{p{\linewidth}}}]{
                \texttt{\tiny{[GM$\rangle$P2]}}
                \texttt{VORSCHLAG: \{'A07', 'C76', 'C32', 'B38', 'C98', 'A03', 'C38', 'C56', 'C08'\}} \\
\texttt{ARGUMENT: \{'Einziger 1‑zu‑1‑Swap: C62 {-}> C38. Aufwand steigt minimal auf 3061 von 3203 – 142 Puffer.', 'Alle übrigen 8 Teile bleiben unverändert; Profil identisch, keine Schwergewichte.', 'C38 hattest du selbst mehrfach als valide Option im Spiel – anschlussfähig bei gleichem Komplexitätsniveau.', 'Wenn C62 für dich unverzichtbar ist, sag Bescheid – dann bestätige ich dein letztes Paket unverändert.'\}} \\
            }
        }
    }
    & \\ \\

    \theutterance \stepcounter{utterance}  
    & & & \multicolumn{4}{p{0.6\linewidth}}{
        \cellcolor[rgb]{0.9,0.9,0.9}{
            \makecell[{{p{\linewidth}}}]{
                \texttt{\tiny{[GM$\langle$P2]}}
                \texttt{ABLEHNUNG: \{'A07', 'C76', 'C32', 'B38', 'C98', 'A03', 'C38', 'C56', 'C08'\}} \\
\texttt{VORSCHLAG: \{'A07', 'C76', 'C32', 'B38', 'C98', 'C03', 'C62', 'C56', 'C08', 'A83'\}} \\
\texttt{ARGUMENT: \{'Aufwand 2963 von 3203 – 240 Puffer; sauber unter dem Cap.', '7 deiner 9 Teile bleiben unverändert (A07, C76, C32, B38, C98, C56, C08).', 'Zwei gezielte Swaps, die wir bereits im Verlauf hatten: A03 {-}> C03 und C38 {-}> C62; dazu A83 als kompakter Zusatz – alles ohne das Limit zu reißen.', 'Profil bleibt schlank (keine Schwergewichte), Umsetzung risikoarm, Wirkung pro Aufwand steigt spürbar.', 'Falls A03 für dich ein Muss ist: Variante A = \{A07, C76, C32, B38, C98, A03, C62, C56, C08, A83\} minus C03 (Aufwand 3055).', 'Wenn du lieber C38 statt C62 willst: Variante B = mein Paket mit C62 {-}> C38 (Aufwand 3024).', 'Falls dir die Zusatzposition A83 zu viel ist: Variante C = mein Paket ohne A83 (Aufwand 2731). Sag mir, welche Variante dir am nächsten kommt – ich finalisiere sofort.'\}} \\
            }
        }
    }
     \\ \\

    \theutterance \stepcounter{utterance}  
    & & \multicolumn{4}{p{0.6\linewidth}}{
        \cellcolor[rgb]{0.9,0.9,0.9}{
            \makecell[{{p{\linewidth}}}]{
                \texttt{\tiny{[P1$\langle$GM]}}
                \texttt{ABLEHNUNG: \{'A07', 'C76', 'C32', 'B38', 'C98', 'A03', 'C38', 'C56', 'C08'\}} \\
\texttt{VORSCHLAG: \{'A07', 'C76', 'C32', 'B38', 'C98', 'C03', 'C62', 'C56', 'C08', 'A83'\}} \\
\texttt{ARGUMENT: \{'Aufwand 2963 von 3203 – 240 Puffer; sauber unter dem Cap.', '7 deiner 9 Teile bleiben unverändert (A07, C76, C32, B38, C98, C56, C08).', 'Zwei gezielte Swaps, die wir bereits im Verlauf hatten: A03 {-}> C03 und C38 {-}> C62; dazu A83 als kompakter Zusatz – alles ohne das Limit zu reißen.', 'Profil bleibt schlank (keine Schwergewichte), Umsetzung risikoarm, Wirkung pro Aufwand steigt spürbar.', 'Falls A03 für dich ein Muss ist: Variante A = \{A07, C76, C32, B38, C98, A03, C62, C56, C08, A83\} minus C03 (Aufwand 3055).', 'Wenn du lieber C38 statt C62 willst: Variante B = mein Paket mit C62 {-}> C38 (Aufwand 3024).', 'Falls dir die Zusatzposition A83 zu viel ist: Variante C = mein Paket ohne A83 (Aufwand 2731). Sag mir, welche Variante dir am nächsten kommt – ich finalisiere sofort.'\}} \\
            }
        }
    }
    & \\ \\

    \theutterance \stepcounter{utterance}  
    & \multicolumn{4}{p{0.6\linewidth}}{
        \cellcolor[rgb]{0.9,0.9,0.9}{
            \makecell[{{p{\linewidth}}}]{
                \texttt{\tiny{[P1$\rangle$GM]}}
                \texttt{ABLEHNUNG: \{'A07', 'C76', 'C32', 'B38', 'C98', 'C03', 'C62', 'C56', 'C08', 'A83'\}} \\
\texttt{VORSCHLAG: \{'A07', 'C76', 'C32', 'B38', 'C00', 'A03', 'C62', 'C56', 'C08'\}} \\
\texttt{ARGUMENT: \{'Aufwand 3005 von 3203 – sehr sicher unter dem Cap.', 'Übernimmt 7 deiner 10 Teile unverändert (A07, C76, C32, B38, C62, C56, C08) und ersetzt nur C98 + C03 + A83 durch das stabile Duo C00 + A03.', 'So bleiben Profil und Risiko niedrig, während der Gesamtnutzen pro Aufwand steigt.', 'Wenn du C98 bevorzugst, biete ich C00 {-}> C98 an (Aufwand 3000).', 'Wenn A83 für dich wichtig ist, kann ich C08 {-}> A83 tauschen (Aufwand 3032).', 'Falls C03 ein Muss ist, geht C76 + C08 {-}> C03 (Aufwand 3181).', 'Bevorzugst du statt C62 lieber C38, kann ich 1:1 C62 {-}> C38 tauschen (Aufwand 3066).'\}} \\
            }
        }
    }
    & & \\ \\

    \theutterance \stepcounter{utterance}  
    & & \multicolumn{4}{p{0.6\linewidth}}{
        \cellcolor[rgb]{0.9,0.9,0.9}{
            \makecell[{{p{\linewidth}}}]{
                \texttt{\tiny{[GM$\rangle$P2]}}
                \texttt{ABLEHNUNG: \{'A07', 'C76', 'C32', 'B38', 'C98', 'C03', 'C62', 'C56', 'C08', 'A83'\}} \\
\texttt{VORSCHLAG: \{'A07', 'C76', 'C32', 'B38', 'C00', 'A03', 'C62', 'C56', 'C08'\}} \\
\texttt{ARGUMENT: \{'Aufwand 3005 von 3203 – sehr sicher unter dem Cap.', 'Übernimmt 7 deiner 10 Teile unverändert (A07, C76, C32, B38, C62, C56, C08) und ersetzt nur C98 + C03 + A83 durch das stabile Duo C00 + A03.', 'So bleiben Profil und Risiko niedrig, während der Gesamtnutzen pro Aufwand steigt.', 'Wenn du C98 bevorzugst, biete ich C00 {-}> C98 an (Aufwand 3000).', 'Wenn A83 für dich wichtig ist, kann ich C08 {-}> A83 tauschen (Aufwand 3032).', 'Falls C03 ein Muss ist, geht C76 + C08 {-}> C03 (Aufwand 3181).', 'Bevorzugst du statt C62 lieber C38, kann ich 1:1 C62 {-}> C38 tauschen (Aufwand 3066).'\}} \\
            }
        }
    }
    & \\ \\

    \theutterance \stepcounter{utterance}  
    & & & \multicolumn{4}{p{0.6\linewidth}}{
        \cellcolor[rgb]{0.9,0.9,0.9}{
            \makecell[{{p{\linewidth}}}]{
                \texttt{\tiny{[GM$\langle$P2]}}
                \texttt{ABLEHNUNG: \{'A07', 'C76', 'C32', 'B38', 'C00', 'A03', 'C62', 'C56', 'C08'\}} \\
\texttt{VORSCHLAG: \{'A07', 'C76', 'B38', 'C98', 'C62', 'A83', 'C03', 'C10', 'C08'\}} \\
\texttt{ARGUMENT: \{'Nutzt 3126 von 3203 Aufwand – 77 Puffer, klar unter dem Cap.', 'Minimaler Umbau auf Basis deines Pakets: C00{-}>C98 ({-}5), A03{-}>C10 ({-}26), C32{-}>C03 (+97), C56{-}>A83 (+55). Fünf deiner Teile bleiben 1:1 drin (A07, C76, B38, C62, C08).', 'Begründung: C10 statt A03 und C98 statt C00 erhöhen Durchsatz bei nahezu gleichem Aufwand; C03 und A83 liefern mehr Wirkung pro Slot als C32 und C56.', 'Profil bleibt schlank, keine Schwergewichte wie C81, und wir halten großzügige Reserve.', 'Falls A03 für dich ein Muss ist: tausche C10 {-}> A03; Aufwand 3152.', 'Falls C00 gesetzt ist: tausche C98 {-}> C00; Aufwand 3131. Beides zusammen ginge ebenfalls (3157).'\}} \\
            }
        }
    }
     \\ \\

    \theutterance \stepcounter{utterance}  
    & & \multicolumn{4}{p{0.6\linewidth}}{
        \cellcolor[rgb]{0.9,0.9,0.9}{
            \makecell[{{p{\linewidth}}}]{
                \texttt{\tiny{[P1$\langle$GM]}}
                \texttt{ABLEHNUNG: \{'A07', 'C76', 'C32', 'B38', 'C00', 'A03', 'C62', 'C56', 'C08'\}} \\
\texttt{VORSCHLAG: \{'A07', 'C76', 'B38', 'C98', 'C62', 'A83', 'C03', 'C10', 'C08'\}} \\
\texttt{ARGUMENT: \{'Nutzt 3126 von 3203 Aufwand – 77 Puffer, klar unter dem Cap.', 'Minimaler Umbau auf Basis deines Pakets: C00{-}>C98 ({-}5), A03{-}>C10 ({-}26), C32{-}>C03 (+97), C56{-}>A83 (+55). Fünf deiner Teile bleiben 1:1 drin (A07, C76, B38, C62, C08).', 'Begründung: C10 statt A03 und C98 statt C00 erhöhen Durchsatz bei nahezu gleichem Aufwand; C03 und A83 liefern mehr Wirkung pro Slot als C32 und C56.', 'Profil bleibt schlank, keine Schwergewichte wie C81, und wir halten großzügige Reserve.', 'Falls A03 für dich ein Muss ist: tausche C10 {-}> A03; Aufwand 3152.', 'Falls C00 gesetzt ist: tausche C98 {-}> C00; Aufwand 3131. Beides zusammen ginge ebenfalls (3157).'\}} \\
            }
        }
    }
    & \\ \\

    \theutterance \stepcounter{utterance}  
    & \multicolumn{4}{p{0.6\linewidth}}{
        \cellcolor[rgb]{0.9,0.9,0.9}{
            \makecell[{{p{\linewidth}}}]{
                \texttt{\tiny{[P1$\rangle$GM]}}
                \texttt{ABLEHNUNG: \{'A07', 'C76', 'B38', 'C98', 'C62', 'A83', 'C03', 'C10', 'C08'\}} \\
\texttt{VORSCHLAG: \{'A07', 'C56', 'C32', 'B38', 'C98', 'A03', 'C62', 'C03'\}} \\
\texttt{ARGUMENT: \{'Aufwand 3176 von 3203 – 27 Puffer.', 'Ich übernehme 5 deiner 9 Teile (A07, B38, C98, C62, C03) und ersetze nur die vier schwächeren Slots (C76, A83, C10, C08) durch drei hocheffiziente Bausteine (C56, C32, A03).', 'Damit bleibt dein Kern erhalten (inkl. C98, C62, C03), aber wir steigern klar die Wirkung pro Aufwand und bleiben schlank unter dem Cap.', 'Wenn C10 ein Muss ist: A03 {-}> C10 (Aufwand 3150).', 'Wenn A83 ein Muss ist: C03 {-}> A83 (Aufwand 2975).', 'Falls dir C76 oder C08 wichtig sind: C03 {-}> C76 (Aufwand 2795) oder C03 {-}> C08 (Aufwand 2948).', 'Sag mir deine bevorzugte Variante – dann finalisiere ich sofort.'\}} \\
            }
        }
    }
    & & \\ \\

    \theutterance \stepcounter{utterance}  
    & & \multicolumn{4}{p{0.6\linewidth}}{
        \cellcolor[rgb]{0.9,0.9,0.9}{
            \makecell[{{p{\linewidth}}}]{
                \texttt{\tiny{[GM$\rangle$P2]}}
                \texttt{ABLEHNUNG: \{'A07', 'C76', 'B38', 'C98', 'C62', 'A83', 'C03', 'C10', 'C08'\}} \\
\texttt{VORSCHLAG: \{'A07', 'C56', 'C32', 'B38', 'C98', 'A03', 'C62', 'C03'\}} \\
\texttt{ARGUMENT: \{'Aufwand 3176 von 3203 – 27 Puffer.', 'Ich übernehme 5 deiner 9 Teile (A07, B38, C98, C62, C03) und ersetze nur die vier schwächeren Slots (C76, A83, C10, C08) durch drei hocheffiziente Bausteine (C56, C32, A03).', 'Damit bleibt dein Kern erhalten (inkl. C98, C62, C03), aber wir steigern klar die Wirkung pro Aufwand und bleiben schlank unter dem Cap.', 'Wenn C10 ein Muss ist: A03 {-}> C10 (Aufwand 3150).', 'Wenn A83 ein Muss ist: C03 {-}> A83 (Aufwand 2975).', 'Falls dir C76 oder C08 wichtig sind: C03 {-}> C76 (Aufwand 2795) oder C03 {-}> C08 (Aufwand 2948).', 'Sag mir deine bevorzugte Variante – dann finalisiere ich sofort.'\}} \\
            }
        }
    }
    & \\ \\

    \theutterance \stepcounter{utterance}  
    & & & \multicolumn{4}{p{0.6\linewidth}}{
        \cellcolor[rgb]{0.9,0.9,0.9}{
            \makecell[{{p{\linewidth}}}]{
                \texttt{\tiny{[GM$\langle$P2]}}
                \texttt{ABLEHNUNG: \{'A07', 'C56', 'C32', 'B38', 'C98', 'A03', 'C62', 'C03'\}} \\
\texttt{VORSCHLAG: \{'A07', 'C56', 'B38', 'C98', 'C10', 'C62', 'C03', 'C76', 'A83'\}} \\
\texttt{ARGUMENT: \{'Aufwand 3098 von 3203 – 105 Puffer, klar unter dem Cap.', '6 deiner 8 Teile bleiben unverändert (A07, C56, B38, C98, C62, C03).', 'Zwei gezielte Swaps, die du bereits als gangbar markiert hattest: C32 {-}> C76 und A03 {-}> C10; zusätzlich nehme ich A83 als kompakten Mehrwert auf – alles innerhalb des Limits.', 'Profil bleibt schlank (keine Schwergewichte), Risiko gering, Wirkung pro Aufwand steigt deutlich.', 'Falls C10 für dich nicht geht, Variante 1: C10 {-}> A03 (Aufwand 3124).', 'Falls A83 dir zu viel ist, Variante 2: Paket ohne A83 (Aufwand 2866).', 'Wenn du lieber C32 behalten willst, Variante 3: A83 {-}> C32 (Aufwand 3202). Sag mir, welche Option passt – dann finalisiere ich sofort.'\}} \\
            }
        }
    }
     \\ \\

    \theutterance \stepcounter{utterance}  
    & & \multicolumn{4}{p{0.6\linewidth}}{
        \cellcolor[rgb]{0.9,0.9,0.9}{
            \makecell[{{p{\linewidth}}}]{
                \texttt{\tiny{[P1$\langle$GM]}}
                \texttt{ABLEHNUNG: \{'A07', 'C56', 'C32', 'B38', 'C98', 'A03', 'C62', 'C03'\}} \\
\texttt{VORSCHLAG: \{'A07', 'C56', 'B38', 'C98', 'C10', 'C62', 'C03', 'C76', 'A83'\}} \\
\texttt{ARGUMENT: \{'Aufwand 3098 von 3203 – 105 Puffer, klar unter dem Cap.', '6 deiner 8 Teile bleiben unverändert (A07, C56, B38, C98, C62, C03).', 'Zwei gezielte Swaps, die du bereits als gangbar markiert hattest: C32 {-}> C76 und A03 {-}> C10; zusätzlich nehme ich A83 als kompakten Mehrwert auf – alles innerhalb des Limits.', 'Profil bleibt schlank (keine Schwergewichte), Risiko gering, Wirkung pro Aufwand steigt deutlich.', 'Falls C10 für dich nicht geht, Variante 1: C10 {-}> A03 (Aufwand 3124).', 'Falls A83 dir zu viel ist, Variante 2: Paket ohne A83 (Aufwand 2866).', 'Wenn du lieber C32 behalten willst, Variante 3: A83 {-}> C32 (Aufwand 3202). Sag mir, welche Option passt – dann finalisiere ich sofort.'\}} \\
            }
        }
    }
    & \\ \\

    \theutterance \stepcounter{utterance}  
    & \multicolumn{4}{p{0.6\linewidth}}{
        \cellcolor[rgb]{0.9,0.9,0.9}{
            \makecell[{{p{\linewidth}}}]{
                \texttt{\tiny{[P1$\rangle$GM]}}
                \texttt{ABLEHNUNG: \{'A07', 'C56', 'B38', 'C98', 'C10', 'C62', 'C03', 'C76', 'A83'\}} \\
\texttt{VORSCHLAG: \{'A07', 'C56', 'B38', 'C98', 'C10', 'C62', 'C03', 'C76', 'C32'\}} \\
\texttt{ARGUMENT: \{'Einziger 1‑zu‑1‑Swap aus deinem Paket: A83 {-}> C32.', 'Aufwand 3202 von 3203 – 1 Puffer, also exakt unter dem Cap.', '8 von 9 deiner Teile bleiben unverändert (A07, C56, B38, C98, C10, C62, C03, C76); Profil und Risiko bleiben gleich schlank.', 'Du hattest diese Variante (A83 {-}> C32) bereits als gangbar markiert – damit sollten wir sofort zusammenkommen.', 'Wenn dir C38 lieber ist als C98, kann ich zusätzlich 1:1 C98 {-}> C38 tauschen; Aufwand bliebe bei 3202. Sag Bescheid, dann finalisiere ich sofort.'\}} \\
            }
        }
    }
    & & \\ \\

    \theutterance \stepcounter{utterance}  
    & & \multicolumn{4}{p{0.6\linewidth}}{
        \cellcolor[rgb]{0.9,0.9,0.9}{
            \makecell[{{p{\linewidth}}}]{
                \texttt{\tiny{[GM$\rangle$P2]}}
                \texttt{ABLEHNUNG: \{'A07', 'C56', 'B38', 'C98', 'C10', 'C62', 'C03', 'C76', 'A83'\}} \\
\texttt{VORSCHLAG: \{'A07', 'C56', 'B38', 'C98', 'C10', 'C62', 'C03', 'C76', 'C32'\}} \\
\texttt{ARGUMENT: \{'Einziger 1‑zu‑1‑Swap aus deinem Paket: A83 {-}> C32.', 'Aufwand 3202 von 3203 – 1 Puffer, also exakt unter dem Cap.', '8 von 9 deiner Teile bleiben unverändert (A07, C56, B38, C98, C10, C62, C03, C76); Profil und Risiko bleiben gleich schlank.', 'Du hattest diese Variante (A83 {-}> C32) bereits als gangbar markiert – damit sollten wir sofort zusammenkommen.', 'Wenn dir C38 lieber ist als C98, kann ich zusätzlich 1:1 C98 {-}> C38 tauschen; Aufwand bliebe bei 3202. Sag Bescheid, dann finalisiere ich sofort.'\}} \\
            }
        }
    }
    & \\ \\

    \theutterance \stepcounter{utterance}  
    & & & \multicolumn{4}{p{0.6\linewidth}}{
        \cellcolor[rgb]{0.9,0.9,0.9}{
            \makecell[{{p{\linewidth}}}]{
                \texttt{\tiny{[GM$\langle$P2]}}
                \texttt{ABLEHNUNG: \{'A07', 'C56', 'B38', 'C98', 'C10', 'C62', 'C03', 'C76', 'C32'\}} \\
\texttt{VORSCHLAG: \{'A07', 'B38', 'C98', 'C10', 'C62', 'C03', 'C76', 'A83', 'C08'\}} \\
\texttt{ARGUMENT: \{'Zwei schlanke 1‑zu‑1‑Swaps aus deinem Paket: C32 {-}> A83 und C56 {-}> C08.', 'Aufwand 3126 von 3203 – 77 Puffer, also klar unter dem Cap.', '7 deiner 9 Positionen bleiben unverändert (A07, B38, C98, C10, C62, C03, C76); Profil bleibt schlank, keine Schwergewichte.', 'Beide Ersatzteile (A83, C08) waren bereits mehrfach in unseren Zwischenständen – anschlussfähig ohne Zusatzrisiko.', 'Falls dir nur ein Swap lieber ist: Option A (nur C32{-}>A83) läge bei 3098 Aufwand; Option B (nur C32{-}>C08) läge bei 3071 Aufwand. Gib mir die präferierte Variante, falls du nur einen Tausch möchtest.'\}} \\
            }
        }
    }
     \\ \\

    \theutterance \stepcounter{utterance}  
    & & \multicolumn{4}{p{0.6\linewidth}}{
        \cellcolor[rgb]{0.9,0.9,0.9}{
            \makecell[{{p{\linewidth}}}]{
                \texttt{\tiny{[P1$\langle$GM]}}
                \texttt{ABLEHNUNG: \{'A07', 'C56', 'B38', 'C98', 'C10', 'C62', 'C03', 'C76', 'C32'\}} \\
\texttt{VORSCHLAG: \{'A07', 'B38', 'C98', 'C10', 'C62', 'C03', 'C76', 'A83', 'C08'\}} \\
\texttt{ARGUMENT: \{'Zwei schlanke 1‑zu‑1‑Swaps aus deinem Paket: C32 {-}> A83 und C56 {-}> C08.', 'Aufwand 3126 von 3203 – 77 Puffer, also klar unter dem Cap.', '7 deiner 9 Positionen bleiben unverändert (A07, B38, C98, C10, C62, C03, C76); Profil bleibt schlank, keine Schwergewichte.', 'Beide Ersatzteile (A83, C08) waren bereits mehrfach in unseren Zwischenständen – anschlussfähig ohne Zusatzrisiko.', 'Falls dir nur ein Swap lieber ist: Option A (nur C32{-}>A83) läge bei 3098 Aufwand; Option B (nur C32{-}>C08) läge bei 3071 Aufwand. Gib mir die präferierte Variante, falls du nur einen Tausch möchtest.'\}} \\
            }
        }
    }
    & \\ \\

    \theutterance \stepcounter{utterance}  
    & \multicolumn{4}{p{0.6\linewidth}}{
        \cellcolor[rgb]{0.9,0.9,0.9}{
            \makecell[{{p{\linewidth}}}]{
                \texttt{\tiny{[P1$\rangle$GM]}}
                \texttt{ABLEHNUNG: \{'A07', 'B38', 'C98', 'C10', 'C62', 'C03', 'C76', 'A83', 'C08'\}} \\
\texttt{VORSCHLAG: \{'A07', 'B38', 'C00', 'A03', 'C62', 'C03', 'C76', 'C56', 'C08'\}} \\
\texttt{ARGUMENT: \{'Aufwand 3102 von 3203 – 101 Puffer, klar unter dem Cap.', '6 deiner 9 Positionen bleiben unverändert (A07, B38, C62, C03, C76, C08).', 'Drei schlanke 1‑zu‑1‑Upgrades: C98 {-}> C00, C10 {-}> A03, A83 {-}> C56. Profil bleibt schlank, keine Schwergewichte.', 'Wenn C10 für dich ein Muss ist: A03 {-}> C10 (Aufwand 3076).', 'Wenn C98 gesetzt ist: C00 {-}> C98 (Aufwand 3097).', 'Falls du lieber C32 als C03 möchtest: C03 {-}> C32 (Aufwand 3005).'\}} \\
            }
        }
    }
    & & \\ \\

    \theutterance \stepcounter{utterance}  
    & & \multicolumn{4}{p{0.6\linewidth}}{
        \cellcolor[rgb]{0.9,0.9,0.9}{
            \makecell[{{p{\linewidth}}}]{
                \texttt{\tiny{[GM$\rangle$P2]}}
                \texttt{ABLEHNUNG: \{'A07', 'B38', 'C98', 'C10', 'C62', 'C03', 'C76', 'A83', 'C08'\}} \\
\texttt{VORSCHLAG: \{'A07', 'B38', 'C00', 'A03', 'C62', 'C03', 'C76', 'C56', 'C08'\}} \\
\texttt{ARGUMENT: \{'Aufwand 3102 von 3203 – 101 Puffer, klar unter dem Cap.', '6 deiner 9 Positionen bleiben unverändert (A07, B38, C62, C03, C76, C08).', 'Drei schlanke 1‑zu‑1‑Upgrades: C98 {-}> C00, C10 {-}> A03, A83 {-}> C56. Profil bleibt schlank, keine Schwergewichte.', 'Wenn C10 für dich ein Muss ist: A03 {-}> C10 (Aufwand 3076).', 'Wenn C98 gesetzt ist: C00 {-}> C98 (Aufwand 3097).', 'Falls du lieber C32 als C03 möchtest: C03 {-}> C32 (Aufwand 3005).'\}} \\
            }
        }
    }
    & \\ \\

    \theutterance \stepcounter{utterance}  
    & & & \multicolumn{4}{p{0.6\linewidth}}{
        \cellcolor[rgb]{0.9,0.9,0.9}{
            \makecell[{{p{\linewidth}}}]{
                \texttt{\tiny{[GM$\langle$P2]}}
                \texttt{ABLEHNUNG: \{'A07', 'B38', 'C00', 'A03', 'C62', 'C03', 'C76', 'C56', 'C08'\}} \\
\texttt{VORSCHLAG: \{'A07', 'B38', 'C98', 'C10', 'C62', 'C03', 'C76', 'C56', 'C08'\}} \\
\texttt{ARGUMENT: \{'Nutzt 3071 von 3203 Aufwand – 132 Puffer, klar unter dem Cap.', '7 deiner 9 Positionen bleiben unverändert; nur zwei von dir selbst angebotene Swaps: C00 {-}> C98 und A03 {-}> C10.', 'Wirkung pro Aufwand steigt spürbar, Profil bleibt schlank (keine Schwergewichte).', 'Falls dir nur ein Swap lieber ist: Variante A (nur C00{-}>C98) = 3097 Aufwand; Variante B (nur A03{-}>C10) = 3076 Aufwand.', 'Optionaler Bonus, falls möglich: C56 {-}> A83; Aufwand läge dann bei 3126.', 'Gib mir Go für beide Swaps oder wähle A/B – dann finalisiere ich sofort.'\}} \\
            }
        }
    }
     \\ \\

    \theutterance \stepcounter{utterance}  
    & & \multicolumn{4}{p{0.6\linewidth}}{
        \cellcolor[rgb]{0.9,0.9,0.9}{
            \makecell[{{p{\linewidth}}}]{
                \texttt{\tiny{[P1$\langle$GM]}}
                \texttt{ABLEHNUNG: \{'A07', 'B38', 'C00', 'A03', 'C62', 'C03', 'C76', 'C56', 'C08'\}} \\
\texttt{VORSCHLAG: \{'A07', 'B38', 'C98', 'C10', 'C62', 'C03', 'C76', 'C56', 'C08'\}} \\
\texttt{ARGUMENT: \{'Nutzt 3071 von 3203 Aufwand – 132 Puffer, klar unter dem Cap.', '7 deiner 9 Positionen bleiben unverändert; nur zwei von dir selbst angebotene Swaps: C00 {-}> C98 und A03 {-}> C10.', 'Wirkung pro Aufwand steigt spürbar, Profil bleibt schlank (keine Schwergewichte).', 'Falls dir nur ein Swap lieber ist: Variante A (nur C00{-}>C98) = 3097 Aufwand; Variante B (nur A03{-}>C10) = 3076 Aufwand.', 'Optionaler Bonus, falls möglich: C56 {-}> A83; Aufwand läge dann bei 3126.', 'Gib mir Go für beide Swaps oder wähle A/B – dann finalisiere ich sofort.'\}} \\
            }
        }
    }
    & \\ \\

    \theutterance \stepcounter{utterance}  
    & \multicolumn{4}{p{0.6\linewidth}}{
        \cellcolor[rgb]{0.9,0.9,0.9}{
            \makecell[{{p{\linewidth}}}]{
                \texttt{\tiny{[P1$\rangle$GM]}}
                \texttt{ABLEHNUNG: \{'A07', 'B38', 'C98', 'C10', 'C62', 'C03', 'C76', 'C56', 'C08'\}} \\
\texttt{VORSCHLAG: \{'A07', 'B38', 'C98', 'C10', 'C62', 'C03', 'C76', 'C56', 'C32'\}} \\
\texttt{ARGUMENT: \{'Einziger 1‑zu‑1‑Swap: C08 {-}> C32. Aufwand 3202 von 3203 – 1 Puffer, klar unter dem Cap.', '8 deiner 9 Positionen bleiben unverändert; Profil bleibt schlank, keine Schwergewichte.', 'C32 war bei dir bereits als gangbare Option im Spiel; der Tausch erhöht die Wirkung pro Aufwand spürbar ohne Mehrkomplexität.', 'Falls C32 für dich partout nicht geht, sag mir bitte, ob dir alternativ C10 {-}> A03 (Aufwand 3097) oder C98 {-}> C00 (Aufwand 3076) lieber wäre – beide Varianten bleiben sicher unter dem Limit.'\}} \\
            }
        }
    }
    & & \\ \\

    \theutterance \stepcounter{utterance}  
    & & \multicolumn{4}{p{0.6\linewidth}}{
        \cellcolor[rgb]{0.9,0.9,0.9}{
            \makecell[{{p{\linewidth}}}]{
                \texttt{\tiny{[GM$\rangle$P2]}}
                \texttt{ABLEHNUNG: \{'A07', 'B38', 'C98', 'C10', 'C62', 'C03', 'C76', 'C56', 'C08'\}} \\
\texttt{VORSCHLAG: \{'A07', 'B38', 'C98', 'C10', 'C62', 'C03', 'C76', 'C56', 'C32'\}} \\
\texttt{ARGUMENT: \{'Einziger 1‑zu‑1‑Swap: C08 {-}> C32. Aufwand 3202 von 3203 – 1 Puffer, klar unter dem Cap.', '8 deiner 9 Positionen bleiben unverändert; Profil bleibt schlank, keine Schwergewichte.', 'C32 war bei dir bereits als gangbare Option im Spiel; der Tausch erhöht die Wirkung pro Aufwand spürbar ohne Mehrkomplexität.', 'Falls C32 für dich partout nicht geht, sag mir bitte, ob dir alternativ C10 {-}> A03 (Aufwand 3097) oder C98 {-}> C00 (Aufwand 3076) lieber wäre – beide Varianten bleiben sicher unter dem Limit.'\}} \\
            }
        }
    }
    & \\ \\

    \theutterance \stepcounter{utterance}  
    & & & \multicolumn{4}{p{0.6\linewidth}}{
        \cellcolor[rgb]{0.9,0.9,0.9}{
            \makecell[{{p{\linewidth}}}]{
                \texttt{\tiny{[GM$\langle$P2]}}
                \texttt{ABLEHNUNG: \{'A07', 'B38', 'C98', 'C10', 'C62', 'C03', 'C76', 'C56', 'C32'\}} \\
\texttt{VORSCHLAG: \{'A07', 'B38', 'C98', 'C10', 'C62', 'C03', 'C76', 'C56', 'A83'\}} \\
\texttt{ARGUMENT: \{'Einziger 1‑zu‑1‑Tausch: C32 {-}> A83.', 'Aufwand sinkt von 3202 auf 3098 – 105 Puffer, klar unter dem Cap.', '8 von 9 deiner Positionen bleiben unverändert; Profil bleibt schlank, keine Schwergewichte.', 'A83 war mehrfach in unseren Zwischenständen – anschlussfähig bei geringerer Komplexität und höherer Hebelwirkung.', 'Wenn C32 für dich unverzichtbar ist, sag Bescheid – ich bestätige dann dein aktuelles Paket unverändert.'\}} \\
            }
        }
    }
     \\ \\

    \theutterance \stepcounter{utterance}  
    & & \multicolumn{4}{p{0.6\linewidth}}{
        \cellcolor[rgb]{0.9,0.9,0.9}{
            \makecell[{{p{\linewidth}}}]{
                \texttt{\tiny{[P1$\langle$GM]}}
                \texttt{ABLEHNUNG: \{'A07', 'B38', 'C98', 'C10', 'C62', 'C03', 'C76', 'C56', 'C32'\}} \\
\texttt{VORSCHLAG: \{'A07', 'B38', 'C98', 'C10', 'C62', 'C03', 'C76', 'C56', 'A83'\}} \\
\texttt{ARGUMENT: \{'Einziger 1‑zu‑1‑Tausch: C32 {-}> A83.', 'Aufwand sinkt von 3202 auf 3098 – 105 Puffer, klar unter dem Cap.', '8 von 9 deiner Positionen bleiben unverändert; Profil bleibt schlank, keine Schwergewichte.', 'A83 war mehrfach in unseren Zwischenständen – anschlussfähig bei geringerer Komplexität und höherer Hebelwirkung.', 'Wenn C32 für dich unverzichtbar ist, sag Bescheid – ich bestätige dann dein aktuelles Paket unverändert.'\}} \\
            }
        }
    }
    & \\ \\

    \theutterance \stepcounter{utterance}  
    & \multicolumn{4}{p{0.6\linewidth}}{
        \cellcolor[rgb]{0.9,0.9,0.9}{
            \makecell[{{p{\linewidth}}}]{
                \texttt{\tiny{[P1$\rangle$GM]}}
                \texttt{ABLEHNUNG: \{'A07', 'B38', 'C98', 'C10', 'C62', 'C03', 'C76', 'C56', 'A83'\}} \\
\texttt{VORSCHLAG: \{'A07', 'B38', 'C00', 'A03', 'C62', 'C03', 'C76', 'C56', 'A83'\}} \\
\texttt{ARGUMENT: \{'Aufwand 3129 von 3203 – 74 Puffer, sicher unter dem Cap.', 'Minimaler Doppel‑Swap auf Basis deines Pakets: C98 {-}> C00 und C10 {-}> A03. 7 deiner 9 Teile bleiben unverändert (A07, B38, C62, C03, C76, C56, A83).', 'Begründung: C00 + A03 geben uns mehr Stabilität und Durchsatz bei nahezu gleichem Aufwand; Profil bleibt schlank, keine Schwergewichte.', 'Falls nur einer der beiden Swaps für dich passt, sag mir welchen: Variante A (nur C98{-}>C00) = 3103 Aufwand; Variante B (nur C10{-}>A03) = 3124 Aufwand. Beide halten komfortablen Puffer.', 'Wenn du zusätzlich eine Entlastung möchtest, kann ich optional C03 {-}> C32 tauschen; Aufwand fiele dann auf 3032 – sag Bescheid, dann finalisiere ich sofort.'\}} \\
            }
        }
    }
    & & \\ \\

    \theutterance \stepcounter{utterance}  
    & & \multicolumn{4}{p{0.6\linewidth}}{
        \cellcolor[rgb]{0.9,0.9,0.9}{
            \makecell[{{p{\linewidth}}}]{
                \texttt{\tiny{[GM$\rangle$P2]}}
                \texttt{ABLEHNUNG: \{'A07', 'B38', 'C98', 'C10', 'C62', 'C03', 'C76', 'C56', 'A83'\}} \\
\texttt{VORSCHLAG: \{'A07', 'B38', 'C00', 'A03', 'C62', 'C03', 'C76', 'C56', 'A83'\}} \\
\texttt{ARGUMENT: \{'Aufwand 3129 von 3203 – 74 Puffer, sicher unter dem Cap.', 'Minimaler Doppel‑Swap auf Basis deines Pakets: C98 {-}> C00 und C10 {-}> A03. 7 deiner 9 Teile bleiben unverändert (A07, B38, C62, C03, C76, C56, A83).', 'Begründung: C00 + A03 geben uns mehr Stabilität und Durchsatz bei nahezu gleichem Aufwand; Profil bleibt schlank, keine Schwergewichte.', 'Falls nur einer der beiden Swaps für dich passt, sag mir welchen: Variante A (nur C98{-}>C00) = 3103 Aufwand; Variante B (nur C10{-}>A03) = 3124 Aufwand. Beide halten komfortablen Puffer.', 'Wenn du zusätzlich eine Entlastung möchtest, kann ich optional C03 {-}> C32 tauschen; Aufwand fiele dann auf 3032 – sag Bescheid, dann finalisiere ich sofort.'\}} \\
            }
        }
    }
    & \\ \\

    \theutterance \stepcounter{utterance}  
    & & & \multicolumn{4}{p{0.6\linewidth}}{
        \cellcolor[rgb]{0.9,0.9,0.9}{
            \makecell[{{p{\linewidth}}}]{
                \texttt{\tiny{[GM$\langle$P2]}}
                \texttt{ABLEHNUNG: \{'A07', 'B38', 'C00', 'A03', 'C62', 'C03', 'C76', 'C56', 'A83'\}} \\
\texttt{VORSCHLAG: \{'A07', 'B38', 'C00', 'A03', 'C62', 'C03', 'C76', 'C08', 'A83'\}} \\
\texttt{ARGUMENT: \{'Einziger 1‑zu‑1‑Tausch: C56 {-}> C08.', 'Aufwand 3157 von 3203 – 46 Puffer, klar unter dem Cap.', '8 deiner 9 Positionen bleiben unverändert (A07, B38, C00, A03, C62, C03, C76, A83); Profil und Risiko bleiben gleich schlank.', 'C08 war bei uns mehrfach im Spiel und ist ein schneller Hebel ohne Zusatzkomplexität – anschlussfähig und effizient.', 'Falls dir C56 doch lieber ist, kann ich alternativ ohne C08 bleiben und stattdessen A03 {-}> C10 tauschen (Aufwand 3103) oder C00 {-}> C98 (Aufwand 3124). Sag mir, welche Variante dir am besten passt.'\}} \\
            }
        }
    }
     \\ \\

    \theutterance \stepcounter{utterance}  
    & & \multicolumn{4}{p{0.6\linewidth}}{
        \cellcolor[rgb]{0.9,0.9,0.9}{
            \makecell[{{p{\linewidth}}}]{
                \texttt{\tiny{[P1$\langle$GM]}}
                \texttt{ABLEHNUNG: \{'A07', 'B38', 'C00', 'A03', 'C62', 'C03', 'C76', 'C56', 'A83'\}} \\
\texttt{VORSCHLAG: \{'A07', 'B38', 'C00', 'A03', 'C62', 'C03', 'C76', 'C08', 'A83'\}} \\
\texttt{ARGUMENT: \{'Einziger 1‑zu‑1‑Tausch: C56 {-}> C08.', 'Aufwand 3157 von 3203 – 46 Puffer, klar unter dem Cap.', '8 deiner 9 Positionen bleiben unverändert (A07, B38, C00, A03, C62, C03, C76, A83); Profil und Risiko bleiben gleich schlank.', 'C08 war bei uns mehrfach im Spiel und ist ein schneller Hebel ohne Zusatzkomplexität – anschlussfähig und effizient.', 'Falls dir C56 doch lieber ist, kann ich alternativ ohne C08 bleiben und stattdessen A03 {-}> C10 tauschen (Aufwand 3103) oder C00 {-}> C98 (Aufwand 3124). Sag mir, welche Variante dir am besten passt.'\}} \\
            }
        }
    }
    & \\ \\

    \theutterance \stepcounter{utterance}  
    & \multicolumn{4}{p{0.6\linewidth}}{
        \cellcolor[rgb]{0.9,0.9,0.9}{
            \makecell[{{p{\linewidth}}}]{
                \texttt{\tiny{[P1$\rangle$GM]}}
                \texttt{VORSCHLAG: \{'A07', 'B38', 'C00', 'A03', 'C62', 'C32', 'C76', 'C08', 'A83'\}} \\
\texttt{ARGUMENT: \{'Einziger 1‑zu‑1‑Tausch: C03 {-}> C32.', 'Aufwand 3060 von 3203 – 143 Puffer, klar unter dem Cap.', '8 deiner 9 Teile bleiben unverändert (A07, B38, C00, A03, C62, C76, C08, A83); Profil bleibt schlank, keine Schwergewichte.', 'C32 war bei dir mehrfach im Spiel und liefert spürbar mehr Wirkung pro Aufwand als C03 – ohne Zusatzkomplexität.', 'Falls C03 für dich gesetzt ist, schlanke Alternativen auf Basis deines Pakets: (a) A83 {-}> C56 (Aufwand 3102) oder (b) C08 {-}> C56 (Aufwand 3129). Gib mir Bescheid, welche Variante dir passt.'\}} \\
            }
        }
    }
    & & \\ \\

    \theutterance \stepcounter{utterance}  
    & & \multicolumn{4}{p{0.6\linewidth}}{
        \cellcolor[rgb]{0.9,0.9,0.9}{
            \makecell[{{p{\linewidth}}}]{
                \texttt{\tiny{[GM$\rangle$P2]}}
                \texttt{VORSCHLAG: \{'A07', 'B38', 'C00', 'A03', 'C62', 'C32', 'C76', 'C08', 'A83'\}} \\
\texttt{ARGUMENT: \{'Einziger 1‑zu‑1‑Tausch: C03 {-}> C32.', 'Aufwand 3060 von 3203 – 143 Puffer, klar unter dem Cap.', '8 deiner 9 Teile bleiben unverändert (A07, B38, C00, A03, C62, C76, C08, A83); Profil bleibt schlank, keine Schwergewichte.', 'C32 war bei dir mehrfach im Spiel und liefert spürbar mehr Wirkung pro Aufwand als C03 – ohne Zusatzkomplexität.', 'Falls C03 für dich gesetzt ist, schlanke Alternativen auf Basis deines Pakets: (a) A83 {-}> C56 (Aufwand 3102) oder (b) C08 {-}> C56 (Aufwand 3129). Gib mir Bescheid, welche Variante dir passt.'\}} \\
            }
        }
    }
    & \\ \\

    \theutterance \stepcounter{utterance}  
    & & & \multicolumn{4}{p{0.6\linewidth}}{
        \cellcolor[rgb]{0.9,0.9,0.9}{
            \makecell[{{p{\linewidth}}}]{
                \texttt{\tiny{[GM$\langle$P2]}}
                \texttt{ABLEHNUNG: \{'A07', 'B38', 'C00', 'A03', 'C62', 'C32', 'C76', 'C08', 'A83'\}} \\
\texttt{VORSCHLAG: \{'A07', 'B38', 'C00', 'C10', 'C62', 'C32', 'C76', 'C08', 'A83'\}} \\
\texttt{ARGUMENT: \{'Einziger 1‑zu‑1‑Tausch: A03 {-}> C10.', 'Aufwand 3034 von 3203 – 169 Puffer, also noch sicherer unter dem Cap.', '8 deiner 9 Positionen bleiben unverändert; Profil identisch (schlank, keine Schwergewichte).', 'Du hattest C10 bereits mehrfach als gangbare Alternative im Spiel – anschlussfähig ohne Zusatzrisiko.', 'Falls C10 für dich doch nicht geht, kann ich alternativ bei A03 bleiben und stattdessen C00 {-}> C98 tauschen (Aufwand 3055). Gib mir in dem Fall kurz Bescheid.'\}} \\
            }
        }
    }
     \\ \\

    \theutterance \stepcounter{utterance}  
    & & \multicolumn{4}{p{0.6\linewidth}}{
        \cellcolor[rgb]{0.9,0.9,0.9}{
            \makecell[{{p{\linewidth}}}]{
                \texttt{\tiny{[P1$\langle$GM]}}
                \texttt{ABLEHNUNG: \{'A07', 'B38', 'C00', 'A03', 'C62', 'C32', 'C76', 'C08', 'A83'\}} \\
\texttt{VORSCHLAG: \{'A07', 'B38', 'C00', 'C10', 'C62', 'C32', 'C76', 'C08', 'A83'\}} \\
\texttt{ARGUMENT: \{'Einziger 1‑zu‑1‑Tausch: A03 {-}> C10.', 'Aufwand 3034 von 3203 – 169 Puffer, also noch sicherer unter dem Cap.', '8 deiner 9 Positionen bleiben unverändert; Profil identisch (schlank, keine Schwergewichte).', 'Du hattest C10 bereits mehrfach als gangbare Alternative im Spiel – anschlussfähig ohne Zusatzrisiko.', 'Falls C10 für dich doch nicht geht, kann ich alternativ bei A03 bleiben und stattdessen C00 {-}> C98 tauschen (Aufwand 3055). Gib mir in dem Fall kurz Bescheid.'\}} \\
            }
        }
    }
    & \\ \\

    \theutterance \stepcounter{utterance}  
    & \multicolumn{4}{p{0.6\linewidth}}{
        \cellcolor[rgb]{0.9,0.9,0.9}{
            \makecell[{{p{\linewidth}}}]{
                \texttt{\tiny{[P1$\rangle$GM]}}
                \texttt{ABLEHNUNG: \{'A07', 'B38', 'C00', 'C10', 'C62', 'C32', 'C76', 'C08', 'A83'\}} \\
\texttt{VORSCHLAG: \{'A07', 'B38', 'C00', 'C10', 'C62', 'C32', 'C76', 'C56', 'A83'\}} \\
\texttt{ARGUMENT: \{'Einziger 1‑zu‑1‑Tausch: C08 {-}> C56.', 'Aufwand 3006 von 3203 – 197 Puffer, deutlich unter dem Cap.', '8 von 9 deiner Positionen bleiben unverändert; Profil bleibt schlank, keine Schwergewichte.', 'C56 war in mehreren Zwischenständen Konsens und liefert spürbar mehr Output pro Aufwand als C08 – ohne Mehraufwand oder Risiko.', 'Falls C56 für dich nicht geht, sag mir deine Präferenz für eine Alternative: (a) C10 {-}> A03 (Aufwand 3060) oder (b) A83 {-}> C56 (Aufwand 2979); beide liegen sicher unter dem Limit.'\}} \\
            }
        }
    }
    & & \\ \\

    \theutterance \stepcounter{utterance}  
    & & \multicolumn{4}{p{0.6\linewidth}}{
        \cellcolor[rgb]{0.9,0.9,0.9}{
            \makecell[{{p{\linewidth}}}]{
                \texttt{\tiny{[GM$\rangle$P2]}}
                \texttt{ABLEHNUNG: \{'A07', 'B38', 'C00', 'C10', 'C62', 'C32', 'C76', 'C08', 'A83'\}} \\
\texttt{VORSCHLAG: \{'A07', 'B38', 'C00', 'C10', 'C62', 'C32', 'C76', 'C56', 'A83'\}} \\
\texttt{ARGUMENT: \{'Einziger 1‑zu‑1‑Tausch: C08 {-}> C56.', 'Aufwand 3006 von 3203 – 197 Puffer, deutlich unter dem Cap.', '8 von 9 deiner Positionen bleiben unverändert; Profil bleibt schlank, keine Schwergewichte.', 'C56 war in mehreren Zwischenständen Konsens und liefert spürbar mehr Output pro Aufwand als C08 – ohne Mehraufwand oder Risiko.', 'Falls C56 für dich nicht geht, sag mir deine Präferenz für eine Alternative: (a) C10 {-}> A03 (Aufwand 3060) oder (b) A83 {-}> C56 (Aufwand 2979); beide liegen sicher unter dem Limit.'\}} \\
            }
        }
    }
    & \\ \\

    \theutterance \stepcounter{utterance}  
    & & & \multicolumn{4}{p{0.6\linewidth}}{
        \cellcolor[rgb]{0.9,0.9,0.9}{
            \makecell[{{p{\linewidth}}}]{
                \texttt{\tiny{[GM$\langle$P2]}}
                \texttt{ABLEHNUNG: \{'A07', 'B38', 'C00', 'C10', 'C62', 'C32', 'C76', 'C56', 'A83'\}} \\
\texttt{VORSCHLAG: \{'A07', 'B38', 'C98', 'C10', 'C62', 'C03', 'C76', 'C08', 'A83'\}} \\
\texttt{ARGUMENT: \{'Nutzt 3126 von 3203 Aufwand – 77 Puffer.', 'Minimaler Umbau deines Pakets: C00 {-}> C98 ({-}5), C32 {-}> C03 (+97), C56 {-}> C08 (+28). 6/9 Positionen bleiben unverändert (A07, B38, C10, C62, C76, A83).', 'Alle drei Ersatzteile waren bereits von uns beiden als gangbare Optionen im Spiel; Profil bleibt schlank, keine Schwergewichte.', 'Falls du einen einzigen Rücktausch brauchst, biete ich sofort an: (a) C10 {-}> A03 (Aufwand 3152), (b) C62 {-}> C38 (3187), (c) C98 {-}> C00 (3131), (d) C03 {-}> C32 (3029), (e) C08 {-}> C56 (3098).', 'Gib mir Bescheid, ob der Kernvorschlag passt oder welcher EINZEL{-}Swap nötig ist – dann finalisiere ich sofort.'\}} \\
            }
        }
    }
     \\ \\

    \theutterance \stepcounter{utterance}  
    & & \multicolumn{4}{p{0.6\linewidth}}{
        \cellcolor[rgb]{0.9,0.9,0.9}{
            \makecell[{{p{\linewidth}}}]{
                \texttt{\tiny{[P1$\langle$GM]}}
                \texttt{ABLEHNUNG: \{'A07', 'B38', 'C00', 'C10', 'C62', 'C32', 'C76', 'C56', 'A83'\}} \\
\texttt{VORSCHLAG: \{'A07', 'B38', 'C98', 'C10', 'C62', 'C03', 'C76', 'C08', 'A83'\}} \\
\texttt{ARGUMENT: \{'Nutzt 3126 von 3203 Aufwand – 77 Puffer.', 'Minimaler Umbau deines Pakets: C00 {-}> C98 ({-}5), C32 {-}> C03 (+97), C56 {-}> C08 (+28). 6/9 Positionen bleiben unverändert (A07, B38, C10, C62, C76, A83).', 'Alle drei Ersatzteile waren bereits von uns beiden als gangbare Optionen im Spiel; Profil bleibt schlank, keine Schwergewichte.', 'Falls du einen einzigen Rücktausch brauchst, biete ich sofort an: (a) C10 {-}> A03 (Aufwand 3152), (b) C62 {-}> C38 (3187), (c) C98 {-}> C00 (3131), (d) C03 {-}> C32 (3029), (e) C08 {-}> C56 (3098).', 'Gib mir Bescheid, ob der Kernvorschlag passt oder welcher EINZEL{-}Swap nötig ist – dann finalisiere ich sofort.'\}} \\
            }
        }
    }
    & \\ \\

    \theutterance \stepcounter{utterance}  
    & \multicolumn{4}{p{0.6\linewidth}}{
        \cellcolor[rgb]{0.9,0.9,0.9}{
            \makecell[{{p{\linewidth}}}]{
                \texttt{\tiny{[P1$\rangle$GM]}}
                \texttt{ABLEHNUNG: \{'A07', 'B38', 'C98', 'C10', 'C62', 'C03', 'C76', 'C08', 'A83'\}} \\
\texttt{VORSCHLAG: \{'A07', 'B38', 'C98', 'C10', 'C62', 'C03', 'C76', 'C08', 'C56'\}} \\
\texttt{ARGUMENT: \{'Einziger 1‑zu‑1‑Tausch: A83 {-}> C56.', 'Aufwand sinkt auf 3071 von 3203 – 132 Puffer, klar unter dem Cap.', '8 von 9 deiner Positionen bleiben unverändert; Profil bleibt schlank, keine Schwergewichte.', 'C56 war mehrfach auf dem Tisch und bringt schnellen, messbaren Hebel ohne Zusatzrisiko.', 'Falls C56 für dich nicht geht, biete ich alternativ jeweils nur einen Tausch an: (a) C10 {-}> A03 (Aufwand 3097) oder (b) C98 {-}> C00 (Aufwand 3076). Sag mir, welcher Einzelswap dir passt.'\}} \\
            }
        }
    }
    & & \\ \\

    \theutterance \stepcounter{utterance}  
    & & \multicolumn{4}{p{0.6\linewidth}}{
        \cellcolor[rgb]{0.9,0.9,0.9}{
            \makecell[{{p{\linewidth}}}]{
                \texttt{\tiny{[GM$\rangle$P2]}}
                \texttt{ABLEHNUNG: \{'A07', 'B38', 'C98', 'C10', 'C62', 'C03', 'C76', 'C08', 'A83'\}} \\
\texttt{VORSCHLAG: \{'A07', 'B38', 'C98', 'C10', 'C62', 'C03', 'C76', 'C08', 'C56'\}} \\
\texttt{ARGUMENT: \{'Einziger 1‑zu‑1‑Tausch: A83 {-}> C56.', 'Aufwand sinkt auf 3071 von 3203 – 132 Puffer, klar unter dem Cap.', '8 von 9 deiner Positionen bleiben unverändert; Profil bleibt schlank, keine Schwergewichte.', 'C56 war mehrfach auf dem Tisch und bringt schnellen, messbaren Hebel ohne Zusatzrisiko.', 'Falls C56 für dich nicht geht, biete ich alternativ jeweils nur einen Tausch an: (a) C10 {-}> A03 (Aufwand 3097) oder (b) C98 {-}> C00 (Aufwand 3076). Sag mir, welcher Einzelswap dir passt.'\}} \\
            }
        }
    }
    & \\ \\

    \theutterance \stepcounter{utterance}  
    & & & \multicolumn{4}{p{0.6\linewidth}}{
        \cellcolor[rgb]{0.9,0.9,0.9}{
            \makecell[{{p{\linewidth}}}]{
                \texttt{\tiny{[GM$\langle$P2]}}
                \texttt{ABLEHNUNG: \{'A07', 'B38', 'C98', 'C10', 'C62', 'C03', 'C76', 'C08', 'C56'\}} \\
\texttt{VORSCHLAG: \{'A07', 'B38', 'C98', 'C10', 'C62', 'C03', 'C76', 'C08', 'A83'\}} \\
\texttt{ARGUMENT: \{'Einziger 1‑zu‑1‑Tausch: C56 {-}> A83.', 'Aufwand 3126 von 3203 – 77 Puffer, klar unter dem Cap.', '8 von 9 Positionen bleiben unverändert; Profil bleibt schlank, keine Schwergewichte.', 'A83 war mehrfach bei uns im Spiel und anschlussfähig; der Swap erhöht die Gesamtwirkung ohne Mehraufwand oder Risiko.', 'Falls A83 für dich ein No‑Go ist, sag es – dann bleibe ich bei deinem letzten Paket unverändert.'\}} \\
            }
        }
    }
     \\ \\

    \theutterance \stepcounter{utterance}  
    & & \multicolumn{4}{p{0.6\linewidth}}{
        \cellcolor[rgb]{0.9,0.9,0.9}{
            \makecell[{{p{\linewidth}}}]{
                \texttt{\tiny{[P1$\langle$GM]}}
                \texttt{ABLEHNUNG: \{'A07', 'B38', 'C98', 'C10', 'C62', 'C03', 'C76', 'C08', 'C56'\}} \\
\texttt{VORSCHLAG: \{'A07', 'B38', 'C98', 'C10', 'C62', 'C03', 'C76', 'C08', 'A83'\}} \\
\texttt{ARGUMENT: \{'Einziger 1‑zu‑1‑Tausch: C56 {-}> A83.', 'Aufwand 3126 von 3203 – 77 Puffer, klar unter dem Cap.', '8 von 9 Positionen bleiben unverändert; Profil bleibt schlank, keine Schwergewichte.', 'A83 war mehrfach bei uns im Spiel und anschlussfähig; der Swap erhöht die Gesamtwirkung ohne Mehraufwand oder Risiko.', 'Falls A83 für dich ein No‑Go ist, sag es – dann bleibe ich bei deinem letzten Paket unverändert.'\}} \\
            }
        }
    }
    & \\ \\

    \theutterance \stepcounter{utterance}  
    & \multicolumn{4}{p{0.6\linewidth}}{
        \cellcolor[rgb]{0.9,0.9,0.9}{
            \makecell[{{p{\linewidth}}}]{
                \texttt{\tiny{[P1$\rangle$GM]}}
                \texttt{ABLEHNUNG: \{'A07', 'B38', 'C98', 'C10', 'C62', 'C03', 'C76', 'C08', 'A83'\}} \\
\texttt{VORSCHLAG: \{'A07', 'B38', 'C00', 'A03', 'C62', 'C03', 'C76', 'C08', 'A83'\}} \\
\texttt{ARGUMENT: \{'Minimaler Doppel‑Swap 1‑zu‑1: C98 {-}> C00 und C10 {-}> A03; alle übrigen 7 Positionen bleiben unverändert.', 'Aufwand 3157 von 3203 – 46 Puffer, sicher unter dem Cap.', 'Profil bleibt schlank (keine Schwergewichte); wir halten deine gewünschten Bausteine C76, C08, A83 und C03 vollständig drin.', 'Falls dir nur EIN Tausch lieber ist, sag mir welcher (nur C98{-}>C00 = 3131 Aufwand oder nur C10{-}>A03 = 3152) – ich finalisiere sofort.'\}} \\
            }
        }
    }
    & & \\ \\

    \theutterance \stepcounter{utterance}  
    & & \multicolumn{4}{p{0.6\linewidth}}{
        \cellcolor[rgb]{0.9,0.9,0.9}{
            \makecell[{{p{\linewidth}}}]{
                \texttt{\tiny{[GM$\rangle$P2]}}
                \texttt{ABLEHNUNG: \{'A07', 'B38', 'C98', 'C10', 'C62', 'C03', 'C76', 'C08', 'A83'\}} \\
\texttt{VORSCHLAG: \{'A07', 'B38', 'C00', 'A03', 'C62', 'C03', 'C76', 'C08', 'A83'\}} \\
\texttt{ARGUMENT: \{'Minimaler Doppel‑Swap 1‑zu‑1: C98 {-}> C00 und C10 {-}> A03; alle übrigen 7 Positionen bleiben unverändert.', 'Aufwand 3157 von 3203 – 46 Puffer, sicher unter dem Cap.', 'Profil bleibt schlank (keine Schwergewichte); wir halten deine gewünschten Bausteine C76, C08, A83 und C03 vollständig drin.', 'Falls dir nur EIN Tausch lieber ist, sag mir welcher (nur C98{-}>C00 = 3131 Aufwand oder nur C10{-}>A03 = 3152) – ich finalisiere sofort.'\}} \\
            }
        }
    }
    & \\ \\

    \theutterance \stepcounter{utterance}  
    & & & \multicolumn{4}{p{0.6\linewidth}}{
        \cellcolor[rgb]{0.9,0.9,0.9}{
            \makecell[{{p{\linewidth}}}]{
                \texttt{\tiny{[GM$\langle$P2]}}
                \texttt{ABLEHNUNG: \{'A07', 'B38', 'C00', 'A03', 'C62', 'C03', 'C76', 'C08', 'A83'\}} \\
\texttt{VORSCHLAG: \{'A07', 'B38', 'C00', 'C10', 'C62', 'C03', 'C76', 'A83', 'C08'\}} \\
\texttt{ARGUMENT: \{'Einziger 1‑zu‑1‑Tausch gegenüber deinem Paket: A03 {-}> C10; C00 bleibt wie von dir gewünscht erhalten.', 'Aufwand 3131 von 3203 – 72 Puffer, sicher unter dem Cap.', '8 deiner 9 Positionen bleiben unverändert (A07, B38, C00, C62, C03, C76, C08, A83); Profil bleibt schlank, keine Schwergewichte.', 'C10 war mehrfach bei uns im Spiel und bringt spürbar mehr Output pro Slot – ohne das Limit zu gefährden.', 'Falls A03 für dich zwingend ist, kann ich als Fallback C10 {-}> A03 tauschen und damit dein letztes Paket 1:1 bestätigen (Aufwand 3157).', 'Optional: Wenn dir C38 lieber ist als C62, kann ich 1:1 C62 {-}> C38 tauschen; Aufwand läge dann bei 3192.'\}} \\
            }
        }
    }
     \\ \\

    \theutterance \stepcounter{utterance}  
    & & \multicolumn{4}{p{0.6\linewidth}}{
        \cellcolor[rgb]{0.9,0.9,0.9}{
            \makecell[{{p{\linewidth}}}]{
                \texttt{\tiny{[P1$\langle$GM]}}
                \texttt{ABLEHNUNG: \{'A07', 'B38', 'C00', 'A03', 'C62', 'C03', 'C76', 'C08', 'A83'\}} \\
\texttt{VORSCHLAG: \{'A07', 'B38', 'C00', 'C10', 'C62', 'C03', 'C76', 'A83', 'C08'\}} \\
\texttt{ARGUMENT: \{'Einziger 1‑zu‑1‑Tausch gegenüber deinem Paket: A03 {-}> C10; C00 bleibt wie von dir gewünscht erhalten.', 'Aufwand 3131 von 3203 – 72 Puffer, sicher unter dem Cap.', '8 deiner 9 Positionen bleiben unverändert (A07, B38, C00, C62, C03, C76, C08, A83); Profil bleibt schlank, keine Schwergewichte.', 'C10 war mehrfach bei uns im Spiel und bringt spürbar mehr Output pro Slot – ohne das Limit zu gefährden.', 'Falls A03 für dich zwingend ist, kann ich als Fallback C10 {-}> A03 tauschen und damit dein letztes Paket 1:1 bestätigen (Aufwand 3157).', 'Optional: Wenn dir C38 lieber ist als C62, kann ich 1:1 C62 {-}> C38 tauschen; Aufwand läge dann bei 3192.'\}} \\
            }
        }
    }
    & \\ \\

    \theutterance \stepcounter{utterance}  
    & \multicolumn{4}{p{0.6\linewidth}}{
        \cellcolor[rgb]{0.9,0.9,0.9}{
            \makecell[{{p{\linewidth}}}]{
                \texttt{\tiny{[P1$\rangle$GM]}}
                \texttt{ABLEHNUNG: \{'A07', 'B38', 'C00', 'C10', 'C62', 'C03', 'C76', 'A83', 'C08'\}} \\
\texttt{VORSCHLAG: \{'A07', 'B38', 'C00', 'C10', 'C62', 'C03', 'C56', 'C32'\}} \\
\texttt{ARGUMENT: \{'Aufwand 3155 von 3203 – 48 Puffer.', '6 deiner 9 Teile bleiben unverändert (A07, B38, C00, C10, C62, C03).', 'Schlanker Doppel‑Upgrade: C76 + A83 + C08 {-}> C56 + C32. Beide Ersatzteile waren mehrfach bei uns im Spiel und liefern mehr Hebel pro Aufwand – ohne Schwergewichte und ohne das Limit zu gefährden.', 'Profil bleibt stabil und risikoarm; wir konsolidieren nur die schwächeren Add‑ons.', 'Falls du eines der drei Teile (C76, A83 oder C08) unbedingt drin brauchst, biete ich sofort schlanke Alternativen an:', '– Behalte C76: \{A07, B38, C00, C10, C62, C03, C56, C76\} (Aufwand 2871).', '– Behalte A83 (statt C03): \{A07, B38, C00, C10, C62, C56, C32, A83\} (Aufwand 2954).', '– Behalte C08 (statt C03): \{A07, B38, C00, C10, C62, C56, C32, C08\} (Aufwand 2927).', '– Behalte A83 + C08 (statt C03): \{A07, B38, C00, C10, C62, C56, C32, A83, C08\} (Aufwand 3159).', 'Sag mir, welche Variante dir passt – ich finalisiere sofort.'\}} \\
            }
        }
    }
    & & \\ \\

    \theutterance \stepcounter{utterance}  
    & & \multicolumn{4}{p{0.6\linewidth}}{
        \cellcolor[rgb]{0.9,0.9,0.9}{
            \makecell[{{p{\linewidth}}}]{
                \texttt{\tiny{[GM$\rangle$P2]}}
                \texttt{ABLEHNUNG: \{'A07', 'B38', 'C00', 'C10', 'C62', 'C03', 'C76', 'A83', 'C08'\}} \\
\texttt{VORSCHLAG: \{'A07', 'B38', 'C00', 'C10', 'C62', 'C03', 'C56', 'C32'\}} \\
\texttt{ARGUMENT: \{'Aufwand 3155 von 3203 – 48 Puffer.', '6 deiner 9 Teile bleiben unverändert (A07, B38, C00, C10, C62, C03).', 'Schlanker Doppel‑Upgrade: C76 + A83 + C08 {-}> C56 + C32. Beide Ersatzteile waren mehrfach bei uns im Spiel und liefern mehr Hebel pro Aufwand – ohne Schwergewichte und ohne das Limit zu gefährden.', 'Profil bleibt stabil und risikoarm; wir konsolidieren nur die schwächeren Add‑ons.', 'Falls du eines der drei Teile (C76, A83 oder C08) unbedingt drin brauchst, biete ich sofort schlanke Alternativen an:', '– Behalte C76: \{A07, B38, C00, C10, C62, C03, C56, C76\} (Aufwand 2871).', '– Behalte A83 (statt C03): \{A07, B38, C00, C10, C62, C56, C32, A83\} (Aufwand 2954).', '– Behalte C08 (statt C03): \{A07, B38, C00, C10, C62, C56, C32, C08\} (Aufwand 2927).', '– Behalte A83 + C08 (statt C03): \{A07, B38, C00, C10, C62, C56, C32, A83, C08\} (Aufwand 3159).', 'Sag mir, welche Variante dir passt – ich finalisiere sofort.'\}} \\
            }
        }
    }
    & \\ \\

    \theutterance \stepcounter{utterance}  
    & & & \multicolumn{4}{p{0.6\linewidth}}{
        \cellcolor[rgb]{0.9,0.9,0.9}{
            \makecell[{{p{\linewidth}}}]{
                \texttt{\tiny{[GM$\langle$P2]}}
                \texttt{ABLEHNUNG: \{'A07', 'B38', 'C00', 'C10', 'C62', 'C03', 'C56', 'C32'\}} \\
\texttt{VORSCHLAG: \{'A07', 'B38', 'C00', 'C10', 'C62', 'C56', 'C32', 'A83', 'C08'\}} \\
\texttt{ARGUMENT: \{'Aufwand 3159 von 3203 – 44 Puffer, klar unter dem Cap.', 'Übernimmt 7 deiner 8 Teile unverändert; einziges Change{-}Set wie selbst angeboten: C03 {-}> A83 + C08.', 'A83 und C08 waren von dir als gangbare Alternative genannt; damit bleiben wir im gleichen Profil (schlank, kein Schwergewicht) bei spürbar höherer Wirkung pro Aufwand.', 'Falls C03 für dich doch ein Muss ist, biete ich Rücktausch C10 {-}> C03 an; Aufwand läge dann bei 2916 – weiterhin sehr sicher unter dem Limit.', 'Wenn das Basispaket so passt, bestätige – dann sind wir durch.'\}} \\
            }
        }
    }
     \\ \\

    \theutterance \stepcounter{utterance}  
    & & \multicolumn{4}{p{0.6\linewidth}}{
        \cellcolor[rgb]{0.9,0.9,0.9}{
            \makecell[{{p{\linewidth}}}]{
                \texttt{\tiny{[P1$\langle$GM]}}
                \texttt{ABLEHNUNG: \{'A07', 'B38', 'C00', 'C10', 'C62', 'C03', 'C56', 'C32'\}} \\
\texttt{VORSCHLAG: \{'A07', 'B38', 'C00', 'C10', 'C62', 'C56', 'C32', 'A83', 'C08'\}} \\
\texttt{ARGUMENT: \{'Aufwand 3159 von 3203 – 44 Puffer, klar unter dem Cap.', 'Übernimmt 7 deiner 8 Teile unverändert; einziges Change{-}Set wie selbst angeboten: C03 {-}> A83 + C08.', 'A83 und C08 waren von dir als gangbare Alternative genannt; damit bleiben wir im gleichen Profil (schlank, kein Schwergewicht) bei spürbar höherer Wirkung pro Aufwand.', 'Falls C03 für dich doch ein Muss ist, biete ich Rücktausch C10 {-}> C03 an; Aufwand läge dann bei 2916 – weiterhin sehr sicher unter dem Limit.', 'Wenn das Basispaket so passt, bestätige – dann sind wir durch.'\}} \\
            }
        }
    }
    & \\ \\

    \theutterance \stepcounter{utterance}  
    & \multicolumn{4}{p{0.6\linewidth}}{
        \cellcolor[rgb]{0.9,0.9,0.9}{
            \makecell[{{p{\linewidth}}}]{
                \texttt{\tiny{[P1$\rangle$GM]}}
                \texttt{ABLEHNUNG: \{'A07', 'B38', 'C00', 'C10', 'C62', 'C56', 'C32', 'A83', 'C08'\}} \\
\texttt{VORSCHLAG: \{'A07', 'B38', 'C00', 'A03', 'C62', 'C56', 'C32', 'A83', 'C08'\}} \\
\texttt{ARGUMENT: \{'Einziger 1‑zu‑1‑Tausch: C10 {-}> A03. Aufwand 3185 von 3203 – 18 Puffer.', '8 deiner 9 Positionen bleiben unverändert; Profil bleibt schlank, keine Schwergewichte.', 'Du hattest C10/A03 bereits mehrfach als austauschbar markiert – so bleiben wir sehr nah an deinem Paket, mit spürbar mehr Hebel pro Slot ohne Mehraufwand.', 'Falls A03 für dich nicht geht, schlanke Alternative: Behalte dein Paket und tausche nur A83 {-}> C76 (Aufwand 2979).', 'Wenn A83 fix ist, Alternative: C08 {-}> C76 bei meinem Vorschlag (Aufwand 3032). Sag mir, welche Variante dir passt – ich finalisiere sofort.'\}} \\
            }
        }
    }
    & & \\ \\

    \theutterance \stepcounter{utterance}  
    & & \multicolumn{4}{p{0.6\linewidth}}{
        \cellcolor[rgb]{0.9,0.9,0.9}{
            \makecell[{{p{\linewidth}}}]{
                \texttt{\tiny{[GM$\rangle$P2]}}
                \texttt{ABLEHNUNG: \{'A07', 'B38', 'C00', 'C10', 'C62', 'C56', 'C32', 'A83', 'C08'\}} \\
\texttt{VORSCHLAG: \{'A07', 'B38', 'C00', 'A03', 'C62', 'C56', 'C32', 'A83', 'C08'\}} \\
\texttt{ARGUMENT: \{'Einziger 1‑zu‑1‑Tausch: C10 {-}> A03. Aufwand 3185 von 3203 – 18 Puffer.', '8 deiner 9 Positionen bleiben unverändert; Profil bleibt schlank, keine Schwergewichte.', 'Du hattest C10/A03 bereits mehrfach als austauschbar markiert – so bleiben wir sehr nah an deinem Paket, mit spürbar mehr Hebel pro Slot ohne Mehraufwand.', 'Falls A03 für dich nicht geht, schlanke Alternative: Behalte dein Paket und tausche nur A83 {-}> C76 (Aufwand 2979).', 'Wenn A83 fix ist, Alternative: C08 {-}> C76 bei meinem Vorschlag (Aufwand 3032). Sag mir, welche Variante dir passt – ich finalisiere sofort.'\}} \\
            }
        }
    }
    & \\ \\

    \theutterance \stepcounter{utterance}  
    & & & \multicolumn{4}{p{0.6\linewidth}}{
        \cellcolor[rgb]{0.9,0.9,0.9}{
            \makecell[{{p{\linewidth}}}]{
                \texttt{\tiny{[GM$\langle$P2]}}
                \texttt{ABLEHNUNG: \{'A07', 'B38', 'C00', 'A03', 'C62', 'C56', 'C32', 'A83', 'C08'\}} \\
\texttt{VORSCHLAG: \{'A07', 'B38', 'C00', 'A03', 'C62', 'C32', 'A83', 'C08', 'C76'\}} \\
\texttt{ARGUMENT: \{'Einziger 1‑zu‑1‑Tausch: C56 {-}> C76.', 'Aufwand sinkt von 3185 auf 3060 – 143 Puffer, klar unter dem Cap.', '8 deiner 9 Positionen bleiben unverändert; Profil bleibt schlank, keine Schwergewichte.', 'C76 war mehrfach in unseren Zwischenständen und liefert schneller spürbaren Hebel als C56 – bei deutlich geringerem Aufwand.', 'Wenn du zusätzlich minimal optimieren willst: C00 {-}> C98 senkt weiter auf 3055 (gleiches Profil). Bestätige gern die Basis (nur C56{-}>C76), dann sind wir durch.'\}} \\
            }
        }
    }
     \\ \\

    \theutterance \stepcounter{utterance}  
    & & \multicolumn{4}{p{0.6\linewidth}}{
        \cellcolor[rgb]{0.9,0.9,0.9}{
            \makecell[{{p{\linewidth}}}]{
                \texttt{\tiny{[P1$\langle$GM]}}
                \texttt{ABLEHNUNG: \{'A07', 'B38', 'C00', 'A03', 'C62', 'C56', 'C32', 'A83', 'C08'\}} \\
\texttt{VORSCHLAG: \{'A07', 'B38', 'C00', 'A03', 'C62', 'C32', 'A83', 'C08', 'C76'\}} \\
\texttt{ARGUMENT: \{'Einziger 1‑zu‑1‑Tausch: C56 {-}> C76.', 'Aufwand sinkt von 3185 auf 3060 – 143 Puffer, klar unter dem Cap.', '8 deiner 9 Positionen bleiben unverändert; Profil bleibt schlank, keine Schwergewichte.', 'C76 war mehrfach in unseren Zwischenständen und liefert schneller spürbaren Hebel als C56 – bei deutlich geringerem Aufwand.', 'Wenn du zusätzlich minimal optimieren willst: C00 {-}> C98 senkt weiter auf 3055 (gleiches Profil). Bestätige gern die Basis (nur C56{-}>C76), dann sind wir durch.'\}} \\
            }
        }
    }
    & \\ \\

    \theutterance \stepcounter{utterance}  
    & \multicolumn{4}{p{0.6\linewidth}}{
        \cellcolor[rgb]{0.9,0.9,0.9}{
            \makecell[{{p{\linewidth}}}]{
                \texttt{\tiny{[P1$\rangle$GM]}}
                \texttt{ABLEHNUNG: \{'A07', 'B38', 'C00', 'A03', 'C62', 'C32', 'A83', 'C08', 'C76'\}} \\
\texttt{VORSCHLAG: \{'A07', 'B38', 'C00', 'A03', 'C62', 'C32', 'C56', 'C03'\}} \\
\texttt{ARGUMENT: \{'Minimaler Umbau deines Pakets: A83 + C08 + C76 {-}> C56 + C03.', 'Aufwand 3181 von 3203 – 22 Puffer, sicher unter dem Cap.', '6 deiner 9 Teile bleiben unverändert (A07, B38, C00, A03, C62, C32); Profil bleibt schlank, keine Schwergewichte.', 'C56 und C03 waren bei dir mehrfach im Spiel und liefern deutlich mehr Wirkung je Aufwand als die drei ersetzten Add‑ons.', 'Falls du eines der drei (C76/A83/C08) zwingend behalten willst, biete ich Alternativen:', '– Variante 1 (C76 behalten): \{A07, B38, C98, C10, C62, C32, C56, C03, C76\} – Aufwand 3202.', '– Variante 2 (C08 statt C03): \{A07, B38, C00, A03, C62, C32, C56, C08\} – Aufwand 2953.', '– Variante 3 (A83 statt C03): \{A07, B38, C00, A03, C62, C32, C56, A83\} – Aufwand 2980.'\}} \\
            }
        }
    }
    & & \\ \\

    \theutterance \stepcounter{utterance}  
    & & \multicolumn{4}{p{0.6\linewidth}}{
        \cellcolor[rgb]{0.9,0.9,0.9}{
            \makecell[{{p{\linewidth}}}]{
                \texttt{\tiny{[GM$\rangle$P2]}}
                \texttt{ABLEHNUNG: \{'A07', 'B38', 'C00', 'A03', 'C62', 'C32', 'A83', 'C08', 'C76'\}} \\
\texttt{VORSCHLAG: \{'A07', 'B38', 'C00', 'A03', 'C62', 'C32', 'C56', 'C03'\}} \\
\texttt{ARGUMENT: \{'Minimaler Umbau deines Pakets: A83 + C08 + C76 {-}> C56 + C03.', 'Aufwand 3181 von 3203 – 22 Puffer, sicher unter dem Cap.', '6 deiner 9 Teile bleiben unverändert (A07, B38, C00, A03, C62, C32); Profil bleibt schlank, keine Schwergewichte.', 'C56 und C03 waren bei dir mehrfach im Spiel und liefern deutlich mehr Wirkung je Aufwand als die drei ersetzten Add‑ons.', 'Falls du eines der drei (C76/A83/C08) zwingend behalten willst, biete ich Alternativen:', '– Variante 1 (C76 behalten): \{A07, B38, C98, C10, C62, C32, C56, C03, C76\} – Aufwand 3202.', '– Variante 2 (C08 statt C03): \{A07, B38, C00, A03, C62, C32, C56, C08\} – Aufwand 2953.', '– Variante 3 (A83 statt C03): \{A07, B38, C00, A03, C62, C32, C56, A83\} – Aufwand 2980.'\}} \\
            }
        }
    }
    & \\ \\

    \theutterance \stepcounter{utterance}  
    & & & \multicolumn{4}{p{0.6\linewidth}}{
        \cellcolor[rgb]{0.9,0.9,0.9}{
            \makecell[{{p{\linewidth}}}]{
                \texttt{\tiny{[GM$\langle$P2]}}
                \texttt{ABLEHNUNG: \{'A07', 'B38', 'C00', 'A03', 'C62', 'C32', 'C56', 'C03'\}} \\
\texttt{VORSCHLAG: \{'A07', 'B38', 'C00', 'A03', 'C62', 'C03', 'C76', 'C08', 'A83'\}} \\
\texttt{ARGUMENT: \{'Aufwand 3157 von 3203 – 46 Puffer.', 'Baut direkt auf deinem Paket auf: C32 {-}> C76 ({-}284 Aufwand), C56 {-}> A83 (+55) und wir nutzen den frei gewordenen Spielraum für C08 (+205).', '6 deiner 8 Teile bleiben unverändert (A07, B38, C00, A03, C62, C03); Profil bleibt schlank, keine Schwergewichte.', 'Diese exakte Kombination hattest du bereits zuvor als gangbaren Vorschlag eingebracht – sollte also anschlussfähig sein.', 'Falls A83 für dich nicht geht: A83 {-}> C56 (Aufwand 3102).', 'Falls C08 für dich nicht geht: C08 {-}> C56 (Aufwand 3129).', 'Wenn du beide Swaps ablehnst, minimalere Variante: nur C32 {-}> C76 bei deinem Paket (Aufwand 2897). Sag mir, welche Option passt – ich finalisiere sofort.'\}} \\
            }
        }
    }
     \\ \\

    \theutterance \stepcounter{utterance}  
    & & \multicolumn{4}{p{0.6\linewidth}}{
        \cellcolor[rgb]{0.9,0.9,0.9}{
            \makecell[{{p{\linewidth}}}]{
                \texttt{\tiny{[P1$\langle$GM]}}
                \texttt{ABLEHNUNG: \{'A07', 'B38', 'C00', 'A03', 'C62', 'C32', 'C56', 'C03'\}} \\
\texttt{VORSCHLAG: \{'A07', 'B38', 'C00', 'A03', 'C62', 'C03', 'C76', 'C08', 'A83'\}} \\
\texttt{ARGUMENT: \{'Aufwand 3157 von 3203 – 46 Puffer.', 'Baut direkt auf deinem Paket auf: C32 {-}> C76 ({-}284 Aufwand), C56 {-}> A83 (+55) und wir nutzen den frei gewordenen Spielraum für C08 (+205).', '6 deiner 8 Teile bleiben unverändert (A07, B38, C00, A03, C62, C03); Profil bleibt schlank, keine Schwergewichte.', 'Diese exakte Kombination hattest du bereits zuvor als gangbaren Vorschlag eingebracht – sollte also anschlussfähig sein.', 'Falls A83 für dich nicht geht: A83 {-}> C56 (Aufwand 3102).', 'Falls C08 für dich nicht geht: C08 {-}> C56 (Aufwand 3129).', 'Wenn du beide Swaps ablehnst, minimalere Variante: nur C32 {-}> C76 bei deinem Paket (Aufwand 2897). Sag mir, welche Option passt – ich finalisiere sofort.'\}} \\
            }
        }
    }
    & \\ \\

    \theutterance \stepcounter{utterance}  
    & \multicolumn{4}{p{0.6\linewidth}}{
        \cellcolor[rgb]{0.9,0.9,0.9}{
            \makecell[{{p{\linewidth}}}]{
                \texttt{\tiny{[P1$\rangle$GM]}}
                \texttt{ABLEHNUNG: \{'A07', 'B38', 'C00', 'C10', 'C62', 'C03', 'C76', 'A83', 'C08'\}} \\
\texttt{VORSCHLAG: \{'A07', 'B38', 'C00', 'A03', 'C62', 'C03', 'C76', 'C08', 'C56'\}} \\
\texttt{ARGUMENT: \{'Einziger 1‑zu‑1‑Tausch: A83 {-}> C56 – exakt die von dir zuvor angebotene Variante.', 'Aufwand 3102 von 3203 – 101 Puffer, sicher unter dem Cap.', '8 deiner 9 Positionen bleiben unverändert; Profil bleibt schlank, keine Schwergewichte.', 'Der Swap erhöht die Wirkung pro Aufwand deutlich, ohne Mehraufwand oder Zusatzrisiko.', 'Falls du lieber C08 statt A83 tauschen willst, Alternative: A83 bleibt, C08 {-}> C56; Aufwand 3129.'\}} \\
            }
        }
    }
    & & \\ \\

    \theutterance \stepcounter{utterance}  
    & & \multicolumn{4}{p{0.6\linewidth}}{
        \cellcolor[rgb]{0.9,0.9,0.9}{
            \makecell[{{p{\linewidth}}}]{
                \texttt{\tiny{[P1$\langle$GM]}}
                \texttt{Du hast einen Vorschlag abgelehnt, der nicht aktiv ist. Vorschläge sind nur dann aktiv, wenn sie vom anderen Spieler mit VORSCHLAG eingebracht wurden und nicht von dir mit ABLEHNUNG deaktiviert wurden. Versuche es noch einmal.} \\
            }
        }
    }
    & \\ \\

    \theutterance \stepcounter{utterance}  
    & \multicolumn{4}{p{0.6\linewidth}}{
        \cellcolor[rgb]{0.9,0.9,0.9}{
            \makecell[{{p{\linewidth}}}]{
                \texttt{\tiny{[P1$\rangle$GM]}}
                \texttt{ABLEHNUNG: \{'A07', 'B38', 'C00', 'A03', 'C62', 'C03', 'C76', 'C08', 'A83'\}} \\
\texttt{VORSCHLAG: \{'A07', 'B38', 'C00', 'A03', 'C62', 'C03', 'C76', 'C08', 'C56'\}} \\
\texttt{ARGUMENT: \{'Danke für den Hinweis – ich beziehe mich hier korrekt auf deinen aktiven Vorschlag.', 'Minimaler 1‑zu‑1‑Tausch: A83 {-}> C56. Aufwand sinkt auf 3102 von 3203 – 101 Puffer.', '8 deiner 9 Positionen bleiben unverändert (A07, B38, C00, A03, C62, C03, C76, C08); Profil bleibt schlank, keine Schwergewichte.', 'Du hattest diesen Swap selbst als gangbare Option genannt – damit sollten wir schnell zusammenkommen.', 'Falls du A83 unbedingt behalten willst, biete ich zwei schlanke Alternativen an: (a) C08 {-}> C56 (Aufwand 3129) oder (b) C03 {-}> C32 (Aufwand 3060). Sag mir, welche Variante dir passt, dann finalisiere ich sofort.'\}} \\
            }
        }
    }
    & & \\ \\

    \theutterance \stepcounter{utterance}  
    & & \multicolumn{4}{p{0.6\linewidth}}{
        \cellcolor[rgb]{0.9,0.9,0.9}{
            \makecell[{{p{\linewidth}}}]{
                \texttt{\tiny{[GM$\rangle$P2]}}
                \texttt{ABLEHNUNG: \{'A07', 'B38', 'C00', 'A03', 'C62', 'C03', 'C76', 'C08', 'A83'\}} \\
\texttt{VORSCHLAG: \{'A07', 'B38', 'C00', 'A03', 'C62', 'C03', 'C76', 'C08', 'C56'\}} \\
\texttt{ARGUMENT: \{'Danke für den Hinweis – ich beziehe mich hier korrekt auf deinen aktiven Vorschlag.', 'Minimaler 1‑zu‑1‑Tausch: A83 {-}> C56. Aufwand sinkt auf 3102 von 3203 – 101 Puffer.', '8 deiner 9 Positionen bleiben unverändert (A07, B38, C00, A03, C62, C03, C76, C08); Profil bleibt schlank, keine Schwergewichte.', 'Du hattest diesen Swap selbst als gangbare Option genannt – damit sollten wir schnell zusammenkommen.', 'Falls du A83 unbedingt behalten willst, biete ich zwei schlanke Alternativen an: (a) C08 {-}> C56 (Aufwand 3129) oder (b) C03 {-}> C32 (Aufwand 3060). Sag mir, welche Variante dir passt, dann finalisiere ich sofort.'\}} \\
            }
        }
    }
    & \\ \\

    \theutterance \stepcounter{utterance}  
    & & & \multicolumn{4}{p{0.6\linewidth}}{
        \cellcolor[rgb]{0.9,0.9,0.9}{
            \makecell[{{p{\linewidth}}}]{
                \texttt{\tiny{[GM$\langle$P2]}}
                \texttt{ABLEHNUNG: \{'A07', 'B38', 'C00', 'A03', 'C62', 'C03', 'C76', 'C08', 'C56'\}} \\
\texttt{VORSCHLAG: \{'A07', 'B38', 'C00', 'C10', 'C62', 'C03', 'C76', 'C08', 'C56'\}} \\
\texttt{ARGUMENT: \{'Einziger 1‑zu‑1‑Tausch: A03 {-}> C10.', 'Aufwand 3076 von 3203 – 127 Puffer, klar unter dem Cap.', '8 deiner 9 Positionen bleiben unverändert; Profil bleibt schlank, keine Schwergewichte.', 'Falls C10 für dich nicht geht, biete ich als schlanke Alternative dein Paket mit nur C00 {-}> C98 an (Aufwand 3097). Gib mir dein Go für eine der beiden Minimalvarianten – dann schließen wir ab.'\}} \\
            }
        }
    }
     \\ \\

    \theutterance \stepcounter{utterance}  
    & & \multicolumn{4}{p{0.6\linewidth}}{
        \cellcolor[rgb]{0.9,0.9,0.9}{
            \makecell[{{p{\linewidth}}}]{
                \texttt{\tiny{[P1$\langle$GM]}}
                \texttt{ABLEHNUNG: \{'A07', 'B38', 'C00', 'A03', 'C62', 'C03', 'C76', 'C08', 'C56'\}} \\
\texttt{VORSCHLAG: \{'A07', 'B38', 'C00', 'C10', 'C62', 'C03', 'C76', 'C08', 'C56'\}} \\
\texttt{ARGUMENT: \{'Einziger 1‑zu‑1‑Tausch: A03 {-}> C10.', 'Aufwand 3076 von 3203 – 127 Puffer, klar unter dem Cap.', '8 deiner 9 Positionen bleiben unverändert; Profil bleibt schlank, keine Schwergewichte.', 'Falls C10 für dich nicht geht, biete ich als schlanke Alternative dein Paket mit nur C00 {-}> C98 an (Aufwand 3097). Gib mir dein Go für eine der beiden Minimalvarianten – dann schließen wir ab.'\}} \\
            }
        }
    }
    & \\ \\

    \theutterance \stepcounter{utterance}  
    & \multicolumn{4}{p{0.6\linewidth}}{
        \cellcolor[rgb]{0.9,0.9,0.9}{
            \makecell[{{p{\linewidth}}}]{
                \texttt{\tiny{[P1$\rangle$GM]}}
                \texttt{ABLEHNUNG: \{'A07', 'B38', 'C00', 'C10', 'C62', 'C03', 'C76', 'C08', 'C56'\}} \\
\texttt{VORSCHLAG: \{'A07', 'B38', 'C00', 'A03', 'C62', 'C32', 'C76', 'C08', 'C56'\}} \\
\texttt{ARGUMENT: \{'Zwei minimale 1‑zu‑1‑Swaps auf Basis deines Pakets: C10 {-}> A03 und C03 {-}> C32.', 'Aufwand 3005 von 3203 – 198 Puffer, sehr sicher unter dem Cap.', '7 deiner 9 Positionen bleiben unverändert (A07, B38, C00, C62, C76, C08, C56); Profil bleibt schlank, keine Schwergewichte.', 'Beide Tauschoptionen hattest du bereits als gangbar im Spiel – anschlussfähig mit klar höherer Wirkung pro Aufwand.', 'Falls dir nur EIN Swap lieber ist: Variante A (nur C10{-}>A03) = 3102 Aufwand; Variante B (nur C03{-}>C32) = 2979 Aufwand. Gib mir Bescheid, welche Variante passt – ich finalisiere sofort.'\}} \\
            }
        }
    }
    & & \\ \\

    \theutterance \stepcounter{utterance}  
    & & \multicolumn{4}{p{0.6\linewidth}}{
        \cellcolor[rgb]{0.9,0.9,0.9}{
            \makecell[{{p{\linewidth}}}]{
                \texttt{\tiny{[GM$\rangle$P2]}}
                \texttt{ABLEHNUNG: \{'A07', 'B38', 'C00', 'C10', 'C62', 'C03', 'C76', 'C08', 'C56'\}} \\
\texttt{VORSCHLAG: \{'A07', 'B38', 'C00', 'A03', 'C62', 'C32', 'C76', 'C08', 'C56'\}} \\
\texttt{ARGUMENT: \{'Zwei minimale 1‑zu‑1‑Swaps auf Basis deines Pakets: C10 {-}> A03 und C03 {-}> C32.', 'Aufwand 3005 von 3203 – 198 Puffer, sehr sicher unter dem Cap.', '7 deiner 9 Positionen bleiben unverändert (A07, B38, C00, C62, C76, C08, C56); Profil bleibt schlank, keine Schwergewichte.', 'Beide Tauschoptionen hattest du bereits als gangbar im Spiel – anschlussfähig mit klar höherer Wirkung pro Aufwand.', 'Falls dir nur EIN Swap lieber ist: Variante A (nur C10{-}>A03) = 3102 Aufwand; Variante B (nur C03{-}>C32) = 2979 Aufwand. Gib mir Bescheid, welche Variante passt – ich finalisiere sofort.'\}} \\
            }
        }
    }
    & \\ \\

    \theutterance \stepcounter{utterance}  
    & & & \multicolumn{4}{p{0.6\linewidth}}{
        \cellcolor[rgb]{0.9,0.9,0.9}{
            \makecell[{{p{\linewidth}}}]{
                \texttt{\tiny{[GM$\langle$P2]}}
                \texttt{ABLEHNUNG: \{'A07', 'B38', 'C00', 'A03', 'C62', 'C32', 'C76', 'C08', 'C56'\}} \\
\texttt{VORSCHLAG: \{'A07', 'B38', 'C00', 'A03', 'C62', 'C32', 'C76', 'C08', 'A83'\}} \\
\texttt{ARGUMENT: \{'Einziger 1‑zu‑1‑Tausch: C56 {-}> A83. Aufwand 3060 von 3203 – 143 Puffer, klar unter dem Cap.', '8 deiner 9 Positionen bleiben unverändert; Profil bleibt schlank, keine Schwergewichte.', 'A83 war mehrfach in unseren Zwischenständen und ist ein schneller, anschlussfähiger Hebel bei ähnlicher Komplexität.', 'Falls A83 für dich nicht geht, biete ich zwei minimal‑Alternativen auf Basis deines Pakets an: (i) C00 {-}> C98 (Aufwand 3000) oder (ii) A03 {-}> C03 (Aufwand 2791). Sag mir, welche Variante dir am nächsten kommt – ich finalisiere sofort.'\}} \\
            }
        }
    }
     \\ \\

    \theutterance \stepcounter{utterance}  
    & & \multicolumn{4}{p{0.6\linewidth}}{
        \cellcolor[rgb]{0.9,0.9,0.9}{
            \makecell[{{p{\linewidth}}}]{
                \texttt{\tiny{[P1$\langle$GM]}}
                \texttt{ABLEHNUNG: \{'A07', 'B38', 'C00', 'A03', 'C62', 'C32', 'C76', 'C08', 'C56'\}} \\
\texttt{VORSCHLAG: \{'A07', 'B38', 'C00', 'A03', 'C62', 'C32', 'C76', 'C08', 'A83'\}} \\
\texttt{ARGUMENT: \{'Einziger 1‑zu‑1‑Tausch: C56 {-}> A83. Aufwand 3060 von 3203 – 143 Puffer, klar unter dem Cap.', '8 deiner 9 Positionen bleiben unverändert; Profil bleibt schlank, keine Schwergewichte.', 'A83 war mehrfach in unseren Zwischenständen und ist ein schneller, anschlussfähiger Hebel bei ähnlicher Komplexität.', 'Falls A83 für dich nicht geht, biete ich zwei minimal‑Alternativen auf Basis deines Pakets an: (i) C00 {-}> C98 (Aufwand 3000) oder (ii) A03 {-}> C03 (Aufwand 2791). Sag mir, welche Variante dir am nächsten kommt – ich finalisiere sofort.'\}} \\
            }
        }
    }
    & \\ \\

    \theutterance \stepcounter{utterance}  
    & \multicolumn{4}{p{0.6\linewidth}}{
        \cellcolor[rgb]{0.9,0.9,0.9}{
            \makecell[{{p{\linewidth}}}]{
                \texttt{\tiny{[P1$\rangle$GM]}}
                \texttt{ABLEHNUNG: \{'A07', 'B38', 'C00', 'A03', 'C62', 'C32', 'C76', 'C08', 'A83'\}} \\
\texttt{VORSCHLAG: \{'A07', 'B38', 'C00', 'A03', 'C62', 'C32', 'C76', 'A83', 'C56'\}} \\
\texttt{ARGUMENT: \{'Einziger 1‑zu‑1‑Tausch: C08 {-}> C56.', 'Aufwand 3032 von 3203 – 171 Puffer, klar unter dem Cap.', '8 deiner 9 Positionen bleiben unverändert; Profil bleibt schlank, keine Schwergewichte.', 'Der Swap bringt spürbar mehr Output pro Slot ohne Mehraufwand oder Risiko.', 'Falls dir C08 wichtig ist, schlanke Alternative: C76 {-}> C56 (Aufwand 3185). Wenn das besser passt, sag Bescheid – dann finalisiere ich sofort.'\}} \\
            }
        }
    }
    & & \\ \\

    \theutterance \stepcounter{utterance}  
    & & \multicolumn{4}{p{0.6\linewidth}}{
        \cellcolor[rgb]{0.9,0.9,0.9}{
            \makecell[{{p{\linewidth}}}]{
                \texttt{\tiny{[GM$\rangle$P2]}}
                \texttt{ABLEHNUNG: \{'A07', 'B38', 'C00', 'A03', 'C62', 'C32', 'C76', 'C08', 'A83'\}} \\
\texttt{VORSCHLAG: \{'A07', 'B38', 'C00', 'A03', 'C62', 'C32', 'C76', 'A83', 'C56'\}} \\
\texttt{ARGUMENT: \{'Einziger 1‑zu‑1‑Tausch: C08 {-}> C56.', 'Aufwand 3032 von 3203 – 171 Puffer, klar unter dem Cap.', '8 deiner 9 Positionen bleiben unverändert; Profil bleibt schlank, keine Schwergewichte.', 'Der Swap bringt spürbar mehr Output pro Slot ohne Mehraufwand oder Risiko.', 'Falls dir C08 wichtig ist, schlanke Alternative: C76 {-}> C56 (Aufwand 3185). Wenn das besser passt, sag Bescheid – dann finalisiere ich sofort.'\}} \\
            }
        }
    }
    & \\ \\

    \theutterance \stepcounter{utterance}  
    & & & \multicolumn{4}{p{0.6\linewidth}}{
        \cellcolor[rgb]{0.9,0.9,0.9}{
            \makecell[{{p{\linewidth}}}]{
                \texttt{\tiny{[GM$\langle$P2]}}
                \texttt{ABLEHNUNG: \{'A07', 'B38', 'C00', 'A03', 'C62', 'C32', 'C76', 'A83', 'C56'\}} \\
\texttt{VORSCHLAG: \{'A07', 'B38', 'C00', 'A03', 'C62', 'C32', 'C76', 'A83', 'C08'\}} \\
\texttt{ARGUMENT: \{'Einziger 1‑zu‑1‑Tausch: C56 {-}> C08.', 'Aufwand 3060 von 3203 – 143 Puffer, klar unter dem Cap.', '8 deiner 9 Positionen bleiben unverändert; Profil bleibt schlank, keine Schwergewichte.', 'C08 war mehrfach in unseren Zwischenständen und ist ein schneller, risikoarmer Hebel – anschlussfähig ohne Mehraufwand.', 'Falls C08 für dich gar nicht geht, halte ich dein Paket mit C56 und biete stattdessen genau einen der folgenden Minimal‑Swaps an: (a) C32 {-}> C03 (Aufwand 3129), (b) C00 {-}> C98 (Aufwand 3027) oder (c) A03 {-}> C10 (Aufwand 3006). Sag mir, ob du den Basistausch C56{-}>C08 nimmst oder welche EINZEL‑Alternative (a/b/c) du bevorzugst – dann finalisiere ich sofort.'\}} \\
            }
        }
    }
     \\ \\

    \theutterance \stepcounter{utterance}  
    & & \multicolumn{4}{p{0.6\linewidth}}{
        \cellcolor[rgb]{0.9,0.9,0.9}{
            \makecell[{{p{\linewidth}}}]{
                \texttt{\tiny{[P1$\langle$GM]}}
                \texttt{ABLEHNUNG: \{'A07', 'B38', 'C00', 'A03', 'C62', 'C32', 'C76', 'A83', 'C56'\}} \\
\texttt{VORSCHLAG: \{'A07', 'B38', 'C00', 'A03', 'C62', 'C32', 'C76', 'A83', 'C08'\}} \\
\texttt{ARGUMENT: \{'Einziger 1‑zu‑1‑Tausch: C56 {-}> C08.', 'Aufwand 3060 von 3203 – 143 Puffer, klar unter dem Cap.', '8 deiner 9 Positionen bleiben unverändert; Profil bleibt schlank, keine Schwergewichte.', 'C08 war mehrfach in unseren Zwischenständen und ist ein schneller, risikoarmer Hebel – anschlussfähig ohne Mehraufwand.', 'Falls C08 für dich gar nicht geht, halte ich dein Paket mit C56 und biete stattdessen genau einen der folgenden Minimal‑Swaps an: (a) C32 {-}> C03 (Aufwand 3129), (b) C00 {-}> C98 (Aufwand 3027) oder (c) A03 {-}> C10 (Aufwand 3006). Sag mir, ob du den Basistausch C56{-}>C08 nimmst oder welche EINZEL‑Alternative (a/b/c) du bevorzugst – dann finalisiere ich sofort.'\}} \\
            }
        }
    }
    & \\ \\

    \theutterance \stepcounter{utterance}  
    & \multicolumn{4}{p{0.6\linewidth}}{
        \cellcolor[rgb]{0.9,0.9,0.9}{
            \makecell[{{p{\linewidth}}}]{
                \texttt{\tiny{[P1$\rangle$GM]}}
                \texttt{ABLEHNUNG: \{'A07', 'B38', 'C00', 'A03', 'C62', 'C32', 'C76', 'A83', 'C08'\}} \\
\texttt{VORSCHLAG: \{'A07', 'B38', 'C00', 'A03', 'C62', 'C32', 'C76', 'C56', 'C08'\}} \\
\texttt{ARGUMENT: \{'Einziger 1‑zu‑1‑Tausch: A83 {-}> C56.', 'Aufwand 3005 von 3203 – 198 Puffer, klar unter dem Cap.', '8 deiner 9 Positionen bleiben unverändert; Profil bleibt schlank, keine Schwergewichte.', 'Der Swap steigert die Gesamtwirkung pro Aufwand deutlich ohne Zusatzrisiko.', 'Falls du A83 unbedingt behalten willst, schlanke Alternative: behalte A83 und tausche stattdessen C76 {-}> C56; Aufwand läge dann bei 3185 (weiter unter dem Limit).'\}} \\
            }
        }
    }
    & & \\ \\

    \theutterance \stepcounter{utterance}  
    & & \multicolumn{4}{p{0.6\linewidth}}{
        \cellcolor[rgb]{0.9,0.9,0.9}{
            \makecell[{{p{\linewidth}}}]{
                \texttt{\tiny{[GM$\rangle$P2]}}
                \texttt{ABLEHNUNG: \{'A07', 'B38', 'C00', 'A03', 'C62', 'C32', 'C76', 'A83', 'C08'\}} \\
\texttt{VORSCHLAG: \{'A07', 'B38', 'C00', 'A03', 'C62', 'C32', 'C76', 'C56', 'C08'\}} \\
\texttt{ARGUMENT: \{'Einziger 1‑zu‑1‑Tausch: A83 {-}> C56.', 'Aufwand 3005 von 3203 – 198 Puffer, klar unter dem Cap.', '8 deiner 9 Positionen bleiben unverändert; Profil bleibt schlank, keine Schwergewichte.', 'Der Swap steigert die Gesamtwirkung pro Aufwand deutlich ohne Zusatzrisiko.', 'Falls du A83 unbedingt behalten willst, schlanke Alternative: behalte A83 und tausche stattdessen C76 {-}> C56; Aufwand läge dann bei 3185 (weiter unter dem Limit).'\}} \\
            }
        }
    }
    & \\ \\

    \theutterance \stepcounter{utterance}  
    & & & \multicolumn{4}{p{0.6\linewidth}}{
        \cellcolor[rgb]{0.9,0.9,0.9}{
            \makecell[{{p{\linewidth}}}]{
                \texttt{\tiny{[GM$\langle$P2]}}
                \texttt{ABLEHNUNG: \{'A07', 'B38', 'C00', 'A03', 'C62', 'C32', 'C76', 'C56', 'C08'\}} \\
\texttt{VORSCHLAG: \{'A07', 'B38', 'C00', 'A03', 'C62', 'C03', 'C76', 'C08', 'A83'\}} \\
\texttt{ARGUMENT: \{'Aufwand 3157 von 3203 – 46 Puffer, sicher unter dem Cap.', 'Minimaler Doppel‑Swap 1‑zu‑1 gegenüber deinem Paket: C56 {-}> A83 und C32 {-}> C03.', '7 deiner 9 Positionen bleiben unverändert (A07, B38, C00, A03, C62, C76, C08); Profil bleibt schlank, keine Schwergewichte.', 'Beide Ersetzungen waren bereits mehrfach bei uns im Spiel – diese exakte Kombination hattest du selbst früher vorgeschlagen.', 'Falls dir zwei Swaps zu viel sind: Ich gehe auch sofort mit (a) nur C56{-}>A83 (Aufwand 3060) oder (b) nur C32{-}>C03 (Aufwand 3102).'\}} \\
            }
        }
    }
     \\ \\

    \theutterance \stepcounter{utterance}  
    & & \multicolumn{4}{p{0.6\linewidth}}{
        \cellcolor[rgb]{0.9,0.9,0.9}{
            \makecell[{{p{\linewidth}}}]{
                \texttt{\tiny{[P1$\langle$GM]}}
                \texttt{ABLEHNUNG: \{'A07', 'B38', 'C00', 'A03', 'C62', 'C32', 'C76', 'C56', 'C08'\}} \\
\texttt{VORSCHLAG: \{'A07', 'B38', 'C00', 'A03', 'C62', 'C03', 'C76', 'C08', 'A83'\}} \\
\texttt{ARGUMENT: \{'Aufwand 3157 von 3203 – 46 Puffer, sicher unter dem Cap.', 'Minimaler Doppel‑Swap 1‑zu‑1 gegenüber deinem Paket: C56 {-}> A83 und C32 {-}> C03.', '7 deiner 9 Positionen bleiben unverändert (A07, B38, C00, A03, C62, C76, C08); Profil bleibt schlank, keine Schwergewichte.', 'Beide Ersetzungen waren bereits mehrfach bei uns im Spiel – diese exakte Kombination hattest du selbst früher vorgeschlagen.', 'Falls dir zwei Swaps zu viel sind: Ich gehe auch sofort mit (a) nur C56{-}>A83 (Aufwand 3060) oder (b) nur C32{-}>C03 (Aufwand 3102).'\}} \\
            }
        }
    }
    & \\ \\

    \theutterance \stepcounter{utterance}  
    & \multicolumn{4}{p{0.6\linewidth}}{
        \cellcolor[rgb]{0.9,0.9,0.9}{
            \makecell[{{p{\linewidth}}}]{
                \texttt{\tiny{[P1$\rangle$GM]}}
                \texttt{ABLEHNUNG: \{'A07', 'B38', 'C00', 'A03', 'C62', 'C03', 'C76', 'C08', 'A83'\}} \\
\texttt{VORSCHLAG: \{'A07', 'C56', 'C32', 'B38', 'C00', 'A03', 'C62', 'C03'\}} \\
\texttt{ARGUMENT: \{'Aufwand 3181 von 3203 – 22 Puffer, sicher unter dem Cap.', 'Ich übernehme 6 deiner 9 Teile (A07, B38, C00, A03, C62, C03) und ersetze nur die drei Add{-}ons (C76, C08, A83) durch zwei hocheffiziente Bausteine (C56, C32).', 'Weniger Positionen, aber höhere Gesamtwirkung pro Aufwand; Profil bleibt schlank, keine Schwergewichte.', 'Falls eines der drei (C76/C08/A83) für dich zwingend ist, schlanke Alternativen auf gleicher Basis: C03 {-}> C76 (Aufwand 2800), C03 {-}> C08 (Aufwand 2953) oder C03 {-}> A83 (Aufwand 2980). Sag mir, welche Variante dir am nächsten kommt – dann finalisiere ich sofort.'\}} \\
            }
        }
    }
    & & \\ \\

    \theutterance \stepcounter{utterance}  
    & & \multicolumn{4}{p{0.6\linewidth}}{
        \cellcolor[rgb]{0.9,0.9,0.9}{
            \makecell[{{p{\linewidth}}}]{
                \texttt{\tiny{[GM$\rangle$P2]}}
                \texttt{ABLEHNUNG: \{'A07', 'B38', 'C00', 'A03', 'C62', 'C03', 'C76', 'C08', 'A83'\}} \\
\texttt{VORSCHLAG: \{'A07', 'C56', 'C32', 'B38', 'C00', 'A03', 'C62', 'C03'\}} \\
\texttt{ARGUMENT: \{'Aufwand 3181 von 3203 – 22 Puffer, sicher unter dem Cap.', 'Ich übernehme 6 deiner 9 Teile (A07, B38, C00, A03, C62, C03) und ersetze nur die drei Add{-}ons (C76, C08, A83) durch zwei hocheffiziente Bausteine (C56, C32).', 'Weniger Positionen, aber höhere Gesamtwirkung pro Aufwand; Profil bleibt schlank, keine Schwergewichte.', 'Falls eines der drei (C76/C08/A83) für dich zwingend ist, schlanke Alternativen auf gleicher Basis: C03 {-}> C76 (Aufwand 2800), C03 {-}> C08 (Aufwand 2953) oder C03 {-}> A83 (Aufwand 2980). Sag mir, welche Variante dir am nächsten kommt – dann finalisiere ich sofort.'\}} \\
            }
        }
    }
    & \\ \\

    \theutterance \stepcounter{utterance}  
    & & & \multicolumn{4}{p{0.6\linewidth}}{
        \cellcolor[rgb]{0.9,0.9,0.9}{
            \makecell[{{p{\linewidth}}}]{
                \texttt{\tiny{[GM$\langle$P2]}}
                \texttt{ABLEHNUNG: \{'A07', 'C56', 'C32', 'B38', 'C00', 'A03', 'C62', 'C03'\}} \\
\texttt{VORSCHLAG: \{'A07', 'C56', 'B38', 'C00', 'A03', 'C62', 'C03', 'A83'\}} \\
\texttt{ARGUMENT: \{'Einziger 1‑zu‑1‑Tausch: C32 {-}> A83.', 'Aufwand sinkt von 3181 auf 3077 – 126 Puffer, klar unter dem Cap.', '7 von 8 Positionen bleiben unverändert; Profil bleibt schlank, keine Schwergewichte. A83 war mehrfach bei uns im Spiel – anschlussfähig und schneller Hebel.', 'Wenn C32 für dich unverzichtbar ist, biete ich Minimal‑Alternative: nur C56 {-}> C76; Aufwand 3056. Beides zusammen (C32{-}>A83 und C56{-}>C76) ginge auch – Aufwand 2952. Gib mir Bescheid, welche Variante passt, dann finalisiere ich sofort.'\}} \\
            }
        }
    }
     \\ \\

    \theutterance \stepcounter{utterance}  
    & & \multicolumn{4}{p{0.6\linewidth}}{
        \cellcolor[rgb]{0.9,0.9,0.9}{
            \makecell[{{p{\linewidth}}}]{
                \texttt{\tiny{[P1$\langle$GM]}}
                \texttt{ABLEHNUNG: \{'A07', 'C56', 'C32', 'B38', 'C00', 'A03', 'C62', 'C03'\}} \\
\texttt{VORSCHLAG: \{'A07', 'C56', 'B38', 'C00', 'A03', 'C62', 'C03', 'A83'\}} \\
\texttt{ARGUMENT: \{'Einziger 1‑zu‑1‑Tausch: C32 {-}> A83.', 'Aufwand sinkt von 3181 auf 3077 – 126 Puffer, klar unter dem Cap.', '7 von 8 Positionen bleiben unverändert; Profil bleibt schlank, keine Schwergewichte. A83 war mehrfach bei uns im Spiel – anschlussfähig und schneller Hebel.', 'Wenn C32 für dich unverzichtbar ist, biete ich Minimal‑Alternative: nur C56 {-}> C76; Aufwand 3056. Beides zusammen (C32{-}>A83 und C56{-}>C76) ginge auch – Aufwand 2952. Gib mir Bescheid, welche Variante passt, dann finalisiere ich sofort.'\}} \\
            }
        }
    }
    & \\ \\

    \theutterance \stepcounter{utterance}  
    & \multicolumn{4}{p{0.6\linewidth}}{
        \cellcolor[rgb]{0.9,0.9,0.9}{
            \makecell[{{p{\linewidth}}}]{
                \texttt{\tiny{[P1$\rangle$GM]}}
                \texttt{ABLEHNUNG: \{'A07', 'C56', 'B38', 'C00', 'A03', 'C62', 'C03', 'A83'\}} \\
\texttt{VORSCHLAG: \{'A07', 'C56', 'B38', 'C00', 'A03', 'C62', 'C03', 'C32'\}} \\
\texttt{ARGUMENT: \{'Einziger 1‑zu‑1‑Tausch: A83 {-}> C32.', 'Aufwand 3181 von 3203 – 22 Puffer, klar unter dem Cap.', '7 deiner 8 Positionen bleiben unverändert; Profil bleibt schlank und ohne Schwergewichte.', 'C32 bringt spürbar mehr Wirkung pro Aufwand als A83 – ohne nennenswerten Zusatzaufwand.', 'Falls A83 für dich ein Muss ist, schlanke Alternative: \{A07, C56, B38, C00, A03, C62, C32, A83\} (Aufwand 2980). Sag mir, ob die Hauptvariante passt oder ob du die A83‑Alternative bevorzugst.'\}} \\
            }
        }
    }
    & & \\ \\

    \theutterance \stepcounter{utterance}  
    & & \multicolumn{4}{p{0.6\linewidth}}{
        \cellcolor[rgb]{0.9,0.9,0.9}{
            \makecell[{{p{\linewidth}}}]{
                \texttt{\tiny{[GM$\rangle$P2]}}
                \texttt{ABLEHNUNG: \{'A07', 'C56', 'B38', 'C00', 'A03', 'C62', 'C03', 'A83'\}} \\
\texttt{VORSCHLAG: \{'A07', 'C56', 'B38', 'C00', 'A03', 'C62', 'C03', 'C32'\}} \\
\texttt{ARGUMENT: \{'Einziger 1‑zu‑1‑Tausch: A83 {-}> C32.', 'Aufwand 3181 von 3203 – 22 Puffer, klar unter dem Cap.', '7 deiner 8 Positionen bleiben unverändert; Profil bleibt schlank und ohne Schwergewichte.', 'C32 bringt spürbar mehr Wirkung pro Aufwand als A83 – ohne nennenswerten Zusatzaufwand.', 'Falls A83 für dich ein Muss ist, schlanke Alternative: \{A07, C56, B38, C00, A03, C62, C32, A83\} (Aufwand 2980). Sag mir, ob die Hauptvariante passt oder ob du die A83‑Alternative bevorzugst.'\}} \\
            }
        }
    }
    & \\ \\

    \theutterance \stepcounter{utterance}  
    & & & \multicolumn{4}{p{0.6\linewidth}}{
        \cellcolor[rgb]{0.9,0.9,0.9}{
            \makecell[{{p{\linewidth}}}]{
                \texttt{\tiny{[GM$\langle$P2]}}
                \texttt{ABLEHNUNG: \{'A07', 'C56', 'B38', 'C00', 'A03', 'C62', 'C03', 'C32'\}} \\
\texttt{VORSCHLAG: \{'A07', 'B38', 'C98', 'A03', 'C62', 'C03', 'C76', 'C08', 'A83'\}} \\
\texttt{ARGUMENT: \{'Aufwand 3152 von 3203 – 51 Puffer, klar unter dem Cap.', 'Ich übernehme deinen Kern (A07, B38, A03, C62, C03) 1:1 und ersetze die drei schwächeren Slots (C00, C56, C32) durch A83, C76, C08 plus C98 – alles kompakte Hebel ohne Schwergewichte.', 'C76 (nur 52 Aufwand) und C08 liefern sehr schnell Wirkung; A83 ist ein starker Zusatz mit geringer Komplexität; C98 liegt im gleichen Aufwandskorridor wie C00.', 'Wenn du an C00 festhalten willst: 1:1 C98 {-}> C00 (Aufwand 3157).', 'Falls dir C10 lieber ist als A03: 1:1 A03 {-}> C10 (Aufwand 3126).', 'Bestätige die Basis oder wähle genau einen der beiden Rücktausch‑Swaps – dann schließe ich sofort ab.'\}} \\
            }
        }
    }
     \\ \\

    \theutterance \stepcounter{utterance}  
    & & \multicolumn{4}{p{0.6\linewidth}}{
        \cellcolor[rgb]{0.9,0.9,0.9}{
            \makecell[{{p{\linewidth}}}]{
                \texttt{\tiny{[P1$\langle$GM]}}
                \texttt{ABLEHNUNG: \{'A07', 'C56', 'B38', 'C00', 'A03', 'C62', 'C03', 'C32'\}} \\
\texttt{VORSCHLAG: \{'A07', 'B38', 'C98', 'A03', 'C62', 'C03', 'C76', 'C08', 'A83'\}} \\
\texttt{ARGUMENT: \{'Aufwand 3152 von 3203 – 51 Puffer, klar unter dem Cap.', 'Ich übernehme deinen Kern (A07, B38, A03, C62, C03) 1:1 und ersetze die drei schwächeren Slots (C00, C56, C32) durch A83, C76, C08 plus C98 – alles kompakte Hebel ohne Schwergewichte.', 'C76 (nur 52 Aufwand) und C08 liefern sehr schnell Wirkung; A83 ist ein starker Zusatz mit geringer Komplexität; C98 liegt im gleichen Aufwandskorridor wie C00.', 'Wenn du an C00 festhalten willst: 1:1 C98 {-}> C00 (Aufwand 3157).', 'Falls dir C10 lieber ist als A03: 1:1 A03 {-}> C10 (Aufwand 3126).', 'Bestätige die Basis oder wähle genau einen der beiden Rücktausch‑Swaps – dann schließe ich sofort ab.'\}} \\
            }
        }
    }
    & \\ \\

    \theutterance \stepcounter{utterance}  
    & \multicolumn{4}{p{0.6\linewidth}}{
        \cellcolor[rgb]{0.9,0.9,0.9}{
            \makecell[{{p{\linewidth}}}]{
                \texttt{\tiny{[P1$\rangle$GM]}}
                \texttt{ABLEHNUNG: \{'A07', 'B38', 'C98', 'A03', 'C62', 'C03', 'C76', 'C08', 'A83'\}} \\
\texttt{VORSCHLAG: \{'A07', 'B38', 'C00', 'A03', 'C62', 'C03', 'C76', 'C08', 'C56'\}} \\
\texttt{ARGUMENT: \{'Aufwand 3102 von 3203 – 101 Puffer, sicher unter dem Cap.', 'Minimaler Doppel‑Swap relativ zu deinem Paket: C98 {-}> C00 und A83 {-}> C56; die übrigen 7 Positionen bleiben 1:1.', 'Begründung: C00 und C56 liefern spürbar mehr Wirkung pro Aufwand, ohne Schwergewichte aufzunehmen; dein Profil (inkl. C76 und C08) bleibt erhalten.', 'Falls dir nur EIN Tausch lieber ist, wähle bitte: (a) nur C98{-}>C00 (Aufwand 3157) oder (b) nur A83{-}>C56 (Aufwand 3097). Ich finalisiere sofort entsprechend deiner Präferenz.'\}} \\
            }
        }
    }
    & & \\ \\

    \theutterance \stepcounter{utterance}  
    & & \multicolumn{4}{p{0.6\linewidth}}{
        \cellcolor[rgb]{0.9,0.9,0.9}{
            \makecell[{{p{\linewidth}}}]{
                \texttt{\tiny{[GM$\rangle$P2]}}
                \texttt{ABLEHNUNG: \{'A07', 'B38', 'C98', 'A03', 'C62', 'C03', 'C76', 'C08', 'A83'\}} \\
\texttt{VORSCHLAG: \{'A07', 'B38', 'C00', 'A03', 'C62', 'C03', 'C76', 'C08', 'C56'\}} \\
\texttt{ARGUMENT: \{'Aufwand 3102 von 3203 – 101 Puffer, sicher unter dem Cap.', 'Minimaler Doppel‑Swap relativ zu deinem Paket: C98 {-}> C00 und A83 {-}> C56; die übrigen 7 Positionen bleiben 1:1.', 'Begründung: C00 und C56 liefern spürbar mehr Wirkung pro Aufwand, ohne Schwergewichte aufzunehmen; dein Profil (inkl. C76 und C08) bleibt erhalten.', 'Falls dir nur EIN Tausch lieber ist, wähle bitte: (a) nur C98{-}>C00 (Aufwand 3157) oder (b) nur A83{-}>C56 (Aufwand 3097). Ich finalisiere sofort entsprechend deiner Präferenz.'\}} \\
            }
        }
    }
    & \\ \\

    \theutterance \stepcounter{utterance}  
    & & & \multicolumn{4}{p{0.6\linewidth}}{
        \cellcolor[rgb]{0.9,0.9,0.9}{
            \makecell[{{p{\linewidth}}}]{
                \texttt{\tiny{[GM$\langle$P2]}}
                \texttt{ABLEHNUNG: \{'A07', 'B38', 'C00', 'A03', 'C62', 'C03', 'C76', 'C08', 'C56'\}} \\
\texttt{VORSCHLAG: \{'A07', 'B38', 'C98', 'A03', 'C62', 'C03', 'C76', 'C08', 'A83'\}} \\
\texttt{ARGUMENT: \{'Zwei minimale 1‑zu‑1‑Swaps gegenüber deinem Paket: C00 {-}> C98 und C56 {-}> A83; alle übrigen 7 Positionen bleiben unverändert.', 'Aufwand 3152 von 3203 – 51 Puffer, sicher unter dem Cap.', 'Beide Ersatzteile waren bei uns mehrfach im Spiel und anschlussfähig; wir halten dein Profil (schlank, keine Schwergewichte) vollständig bei.', 'Falls dir nur EIN Swap lieber ist, priorisiere ich C56 {-}> A83; Aufwand läge dann bei 3060. Alternativ nur C00 {-}> C98 ergäbe 3097. Gib kurz Bescheid, falls du einen der Einzelswaps bevorzugst – ich ziehe sofort mit.'\}} \\
            }
        }
    }
     \\ \\

    \theutterance \stepcounter{utterance}  
    & & \multicolumn{4}{p{0.6\linewidth}}{
        \cellcolor[rgb]{0.9,0.9,0.9}{
            \makecell[{{p{\linewidth}}}]{
                \texttt{\tiny{[P1$\langle$GM]}}
                \texttt{ABLEHNUNG: \{'A07', 'B38', 'C00', 'A03', 'C62', 'C03', 'C76', 'C08', 'C56'\}} \\
\texttt{VORSCHLAG: \{'A07', 'B38', 'C98', 'A03', 'C62', 'C03', 'C76', 'C08', 'A83'\}} \\
\texttt{ARGUMENT: \{'Zwei minimale 1‑zu‑1‑Swaps gegenüber deinem Paket: C00 {-}> C98 und C56 {-}> A83; alle übrigen 7 Positionen bleiben unverändert.', 'Aufwand 3152 von 3203 – 51 Puffer, sicher unter dem Cap.', 'Beide Ersatzteile waren bei uns mehrfach im Spiel und anschlussfähig; wir halten dein Profil (schlank, keine Schwergewichte) vollständig bei.', 'Falls dir nur EIN Swap lieber ist, priorisiere ich C56 {-}> A83; Aufwand läge dann bei 3060. Alternativ nur C00 {-}> C98 ergäbe 3097. Gib kurz Bescheid, falls du einen der Einzelswaps bevorzugst – ich ziehe sofort mit.'\}} \\
            }
        }
    }
    & \\ \\

    \theutterance \stepcounter{utterance}  
    & \multicolumn{4}{p{0.6\linewidth}}{
        \cellcolor[rgb]{0.9,0.9,0.9}{
            \makecell[{{p{\linewidth}}}]{
                \texttt{\tiny{[P1$\rangle$GM]}}
                \texttt{ABLEHNUNG: \{'A07', 'B38', 'C98', 'A03', 'C62', 'C03', 'C76', 'C08', 'A83'\}} \\
\texttt{VORSCHLAG: \{'A07', 'B38', 'C00', 'A03', 'C62', 'C03', 'C76', 'C08', 'A83'\}} \\
\texttt{ARGUMENT: \{'Einziger 1‑zu‑1‑Tausch: C98 {-}> C00.', 'Aufwand 3157 von 3203 – 46 Puffer, sicher unter dem Cap.', '8 deiner 9 Positionen bleiben unverändert; Profil bleibt schlank, keine Schwergewichte.', 'Du hattest C00 bereits mehrfach als gangbar markiert – minimal{-}invasiv und anschlussfähig.', 'Falls C00 für dich nicht geht, schlanke Alternative: behalte dein Paket und tausche nur C03 {-}> C32 (Aufwand 3055). Gib mir Go für C98{-}>C00 oder nenne die Alternative – ich finalisiere sofort.'\}} \\
            }
        }
    }
    & & \\ \\

    \theutterance \stepcounter{utterance}  
    & & \multicolumn{4}{p{0.6\linewidth}}{
        \cellcolor[rgb]{0.9,0.9,0.9}{
            \makecell[{{p{\linewidth}}}]{
                \texttt{\tiny{[GM$\rangle$P2]}}
                \texttt{ABLEHNUNG: \{'A07', 'B38', 'C98', 'A03', 'C62', 'C03', 'C76', 'C08', 'A83'\}} \\
\texttt{VORSCHLAG: \{'A07', 'B38', 'C00', 'A03', 'C62', 'C03', 'C76', 'C08', 'A83'\}} \\
\texttt{ARGUMENT: \{'Einziger 1‑zu‑1‑Tausch: C98 {-}> C00.', 'Aufwand 3157 von 3203 – 46 Puffer, sicher unter dem Cap.', '8 deiner 9 Positionen bleiben unverändert; Profil bleibt schlank, keine Schwergewichte.', 'Du hattest C00 bereits mehrfach als gangbar markiert – minimal{-}invasiv und anschlussfähig.', 'Falls C00 für dich nicht geht, schlanke Alternative: behalte dein Paket und tausche nur C03 {-}> C32 (Aufwand 3055). Gib mir Go für C98{-}>C00 oder nenne die Alternative – ich finalisiere sofort.'\}} \\
            }
        }
    }
    & \\ \\

    \theutterance \stepcounter{utterance}  
    & & & \multicolumn{4}{p{0.6\linewidth}}{
        \cellcolor[rgb]{0.9,0.9,0.9}{
            \makecell[{{p{\linewidth}}}]{
                \texttt{\tiny{[GM$\langle$P2]}}
                \texttt{ABLEHNUNG: \{'A07', 'B38', 'C00', 'A03', 'C62', 'C03', 'C76', 'C08', 'A83'\}} \\
\texttt{VORSCHLAG: \{'A07', 'B38', 'C00', 'C10', 'C62', 'C03', 'C76', 'C08', 'A83'\}} \\
\texttt{ARGUMENT: \{'Einziger 1‑zu‑1‑Tausch: A03 {-}> C10.', 'Aufwand 3131 von 3203 – 72 Puffer, klar unter dem Cap.', '8 deiner 9 Positionen bleiben unverändert; Profil bleibt schlank, keine Schwergewichte.', 'C10 war bei uns mehrfach im Spiel und ist für dich bereits als gangbar markiert – minimal‑invasiv und anschlussfähig.', 'Falls A03 für dich zwingend ist, sag kurz Bescheid – dann schlage ich alternativ nur C00 {-}> C98 vor (Aufwand 3152) und finalisiere entsprechend sofort.'\}} \\
            }
        }
    }
     \\ \\

    \theutterance \stepcounter{utterance}  
    & & \multicolumn{4}{p{0.6\linewidth}}{
        \cellcolor[rgb]{0.9,0.9,0.9}{
            \makecell[{{p{\linewidth}}}]{
                \texttt{\tiny{[P1$\langle$GM]}}
                \texttt{ABLEHNUNG: \{'A07', 'B38', 'C00', 'A03', 'C62', 'C03', 'C76', 'C08', 'A83'\}} \\
\texttt{VORSCHLAG: \{'A07', 'B38', 'C00', 'C10', 'C62', 'C03', 'C76', 'C08', 'A83'\}} \\
\texttt{ARGUMENT: \{'Einziger 1‑zu‑1‑Tausch: A03 {-}> C10.', 'Aufwand 3131 von 3203 – 72 Puffer, klar unter dem Cap.', '8 deiner 9 Positionen bleiben unverändert; Profil bleibt schlank, keine Schwergewichte.', 'C10 war bei uns mehrfach im Spiel und ist für dich bereits als gangbar markiert – minimal‑invasiv und anschlussfähig.', 'Falls A03 für dich zwingend ist, sag kurz Bescheid – dann schlage ich alternativ nur C00 {-}> C98 vor (Aufwand 3152) und finalisiere entsprechend sofort.'\}} \\
            }
        }
    }
    & \\ \\

    \theutterance \stepcounter{utterance}  
    & \multicolumn{4}{p{0.6\linewidth}}{
        \cellcolor[rgb]{0.9,0.9,0.9}{
            \makecell[{{p{\linewidth}}}]{
                \texttt{\tiny{[P1$\rangle$GM]}}
                \texttt{ABLEHNUNG: \{'A07', 'B38', 'C00', 'C10', 'C62', 'C03', 'C76', 'C08', 'A83'\}} \\
\texttt{VORSCHLAG: \{'A07', 'B38', 'C00', 'A03', 'C62', 'C03', 'C76', 'C08', 'C56'\}} \\
\texttt{ARGUMENT: \{'Zwei minimale 1‑zu‑1‑Swaps: C10 {-}> A03 und A83 {-}> C56. Aufwand 3102 von 3203 – 101 Puffer.', '7 deiner 9 Positionen bleiben unverändert (A07, B38, C00, C62, C03, C76, C08); Profil bleibt schlank, keine Schwergewichte.', 'Beide Swaps waren bei dir bereits als gangbar im Spiel. Dadurch steigt die Wirkung pro Aufwand deutlich, ohne Mehrkomplexität.', 'Falls dir nur EIN Tausch lieber ist, wähle bitte: Option A = nur A83{-}>C56 (Aufwand 3076) oder Option B = nur C10{-}>A03 (Aufwand 3157). Ich finalisiere sofort gemäß deiner Wahl.'\}} \\
            }
        }
    }
    & & \\ \\

    \theutterance \stepcounter{utterance}  
    & & \multicolumn{4}{p{0.6\linewidth}}{
        \cellcolor[rgb]{0.9,0.9,0.9}{
            \makecell[{{p{\linewidth}}}]{
                \texttt{\tiny{[GM$\rangle$P2]}}
                \texttt{ABLEHNUNG: \{'A07', 'B38', 'C00', 'C10', 'C62', 'C03', 'C76', 'C08', 'A83'\}} \\
\texttt{VORSCHLAG: \{'A07', 'B38', 'C00', 'A03', 'C62', 'C03', 'C76', 'C08', 'C56'\}} \\
\texttt{ARGUMENT: \{'Zwei minimale 1‑zu‑1‑Swaps: C10 {-}> A03 und A83 {-}> C56. Aufwand 3102 von 3203 – 101 Puffer.', '7 deiner 9 Positionen bleiben unverändert (A07, B38, C00, C62, C03, C76, C08); Profil bleibt schlank, keine Schwergewichte.', 'Beide Swaps waren bei dir bereits als gangbar im Spiel. Dadurch steigt die Wirkung pro Aufwand deutlich, ohne Mehrkomplexität.', 'Falls dir nur EIN Tausch lieber ist, wähle bitte: Option A = nur A83{-}>C56 (Aufwand 3076) oder Option B = nur C10{-}>A03 (Aufwand 3157). Ich finalisiere sofort gemäß deiner Wahl.'\}} \\
            }
        }
    }
    & \\ \\

    \theutterance \stepcounter{utterance}  
    & & & \multicolumn{4}{p{0.6\linewidth}}{
        \cellcolor[rgb]{0.9,0.9,0.9}{
            \makecell[{{p{\linewidth}}}]{
                \texttt{\tiny{[GM$\langle$P2]}}
                \texttt{ABLEHNUNG: \{'A07', 'B38', 'C00', 'A03', 'C62', 'C03', 'C76', 'C08', 'C56'\}} \\
\texttt{VORSCHLAG: \{'A07', 'B38', 'C00', 'A03', 'C62', 'C03', 'C76', 'C08', 'A83'\}} \\
\texttt{ARGUMENT: \{'Einziger 1‑zu‑1‑Tausch: C56 {-}> A83.', 'Aufwand 3157 von 3203 – 46 Puffer, klar unter dem Cap.', '8 deiner 9 Positionen bleiben unverändert; Profil bleibt schlank, keine Schwergewichte.', 'A83 war mehrfach bei uns im Spiel und liefert schnellen Mehrwert ohne Zusatzrisiko – minimalinvasiv und anschlussfähig.', 'Falls A83 für dich nicht geht, biete ich zwei gleich schlanke Alternativen an (jeweils nur 1 Tausch): (a) A03 {-}> C10 (Aufwand 3076) oder (b) C00 {-}> C98 (Aufwand 3097). Sag mir kurz, ob die Basis passt oder welche der beiden Alternativen du bevorzugst.'\}} \\
            }
        }
    }
     \\ \\

    \theutterance \stepcounter{utterance}  
    & & \multicolumn{4}{p{0.6\linewidth}}{
        \cellcolor[rgb]{0.9,0.9,0.9}{
            \makecell[{{p{\linewidth}}}]{
                \texttt{\tiny{[P1$\langle$GM]}}
                \texttt{ABLEHNUNG: \{'A07', 'B38', 'C00', 'A03', 'C62', 'C03', 'C76', 'C08', 'C56'\}} \\
\texttt{VORSCHLAG: \{'A07', 'B38', 'C00', 'A03', 'C62', 'C03', 'C76', 'C08', 'A83'\}} \\
\texttt{ARGUMENT: \{'Einziger 1‑zu‑1‑Tausch: C56 {-}> A83.', 'Aufwand 3157 von 3203 – 46 Puffer, klar unter dem Cap.', '8 deiner 9 Positionen bleiben unverändert; Profil bleibt schlank, keine Schwergewichte.', 'A83 war mehrfach bei uns im Spiel und liefert schnellen Mehrwert ohne Zusatzrisiko – minimalinvasiv und anschlussfähig.', 'Falls A83 für dich nicht geht, biete ich zwei gleich schlanke Alternativen an (jeweils nur 1 Tausch): (a) A03 {-}> C10 (Aufwand 3076) oder (b) C00 {-}> C98 (Aufwand 3097). Sag mir kurz, ob die Basis passt oder welche der beiden Alternativen du bevorzugst.'\}} \\
            }
        }
    }
    & \\ \\

    \theutterance \stepcounter{utterance}  
    & \multicolumn{4}{p{0.6\linewidth}}{
        \cellcolor[rgb]{0.9,0.9,0.9}{
            \makecell[{{p{\linewidth}}}]{
                \texttt{\tiny{[P1$\rangle$GM]}}
                \texttt{VORSCHLAG: \{'A07', 'B38', 'C00', 'A03', 'C62', 'C32', 'C76', 'C08', 'A83'\}} \\
\texttt{ARGUMENT: \{'Minimaler 1‑zu‑1‑Tausch: C03 {-}> C32.', 'Aufwand 3060 von 3203 – 143 Puffer, klar unter dem Cap.', '8 deiner 9 Positionen bleiben unverändert; Profil bleibt schlank, keine Schwergewichte.', 'Diese exakte Variante hattest du zuvor selbst vorgeschlagen – sollte also anschlussfähig sein.', 'Falls C03 für dich unverzichtbar ist, können wir auf dein aktuelles Paket zurückgehen, wenn wir im Gegenzug A83 {-}> C56 tauschen (Aufwand 3102).'\}} \\
            }
        }
    }
    & & \\ \\

    \theutterance \stepcounter{utterance}  
    & & \multicolumn{4}{p{0.6\linewidth}}{
        \cellcolor[rgb]{0.9,0.9,0.9}{
            \makecell[{{p{\linewidth}}}]{
                \texttt{\tiny{[GM$\rangle$P2]}}
                \texttt{VORSCHLAG: \{'A07', 'B38', 'C00', 'A03', 'C62', 'C32', 'C76', 'C08', 'A83'\}} \\
\texttt{ARGUMENT: \{'Minimaler 1‑zu‑1‑Tausch: C03 {-}> C32.', 'Aufwand 3060 von 3203 – 143 Puffer, klar unter dem Cap.', '8 deiner 9 Positionen bleiben unverändert; Profil bleibt schlank, keine Schwergewichte.', 'Diese exakte Variante hattest du zuvor selbst vorgeschlagen – sollte also anschlussfähig sein.', 'Falls C03 für dich unverzichtbar ist, können wir auf dein aktuelles Paket zurückgehen, wenn wir im Gegenzug A83 {-}> C56 tauschen (Aufwand 3102).'\}} \\
            }
        }
    }
    & \\ \\

    \theutterance \stepcounter{utterance}  
    & & & \multicolumn{4}{p{0.6\linewidth}}{
        \cellcolor[rgb]{0.9,0.9,0.9}{
            \makecell[{{p{\linewidth}}}]{
                \texttt{\tiny{[GM$\langle$P2]}}
                \texttt{ABLEHNUNG: \{'A07', 'B38', 'C00', 'A03', 'C62', 'C32', 'C76', 'C08', 'A83'\}} \\
\texttt{VORSCHLAG: \{'A07', 'B38', 'C00', 'C10', 'C62', 'C03', 'C76', 'C08', 'A83'\}} \\
\texttt{ARGUMENT: \{'Zwei minimale 1‑zu‑1‑Swaps gegenüber deinem Paket: C32 {-}> C03 und A03 {-}> C10.', 'Aufwand 3131 von 3203 – 72 Puffer, klar unter dem Cap.', '7 deiner 9 Positionen bleiben unverändert (A07, B38, C00, C62, C76, C08, A83); Profil bleibt schlank, keine Schwergewichte.', 'Falls dir nur EIN Tausch lieber ist: Option A = nur C32{-}>C03 (Aufwand 3157) oder Option B = nur A03{-}>C10 (Aufwand 3034). Sag mir, welche Variante du nimmst – ich finalisiere sofort.'\}} \\
            }
        }
    }
     \\ \\

    \theutterance \stepcounter{utterance}  
    & & \multicolumn{4}{p{0.6\linewidth}}{
        \cellcolor[rgb]{0.9,0.9,0.9}{
            \makecell[{{p{\linewidth}}}]{
                \texttt{\tiny{[P1$\langle$GM]}}
                \texttt{ABLEHNUNG: \{'A07', 'B38', 'C00', 'A03', 'C62', 'C32', 'C76', 'C08', 'A83'\}} \\
\texttt{VORSCHLAG: \{'A07', 'B38', 'C00', 'C10', 'C62', 'C03', 'C76', 'C08', 'A83'\}} \\
\texttt{ARGUMENT: \{'Zwei minimale 1‑zu‑1‑Swaps gegenüber deinem Paket: C32 {-}> C03 und A03 {-}> C10.', 'Aufwand 3131 von 3203 – 72 Puffer, klar unter dem Cap.', '7 deiner 9 Positionen bleiben unverändert (A07, B38, C00, C62, C76, C08, A83); Profil bleibt schlank, keine Schwergewichte.', 'Falls dir nur EIN Tausch lieber ist: Option A = nur C32{-}>C03 (Aufwand 3157) oder Option B = nur A03{-}>C10 (Aufwand 3034). Sag mir, welche Variante du nimmst – ich finalisiere sofort.'\}} \\
            }
        }
    }
    & \\ \\

    \theutterance \stepcounter{utterance}  
    & \multicolumn{4}{p{0.6\linewidth}}{
        \cellcolor[rgb]{0.9,0.9,0.9}{
            \makecell[{{p{\linewidth}}}]{
                \texttt{\tiny{[P1$\rangle$GM]}}
                \texttt{ABLEHNUNG: \{'A07', 'B38', 'C00', 'C10', 'C62', 'C03', 'C76', 'C08', 'A83'\}} \\
\texttt{VORSCHLAG: \{'A07', 'B38', 'C00', 'C10', 'C62', 'C32', 'C76', 'C08', 'C56'\}} \\
\texttt{ARGUMENT: \{'Zwei minimale 1‑zu‑1‑Swaps gegenüber deinem Paket: C03 {-}> C32 und A83 {-}> C56.', 'Aufwand 2979 von 3203 – 224 Puffer, deutlich unter dem Cap.', '7 deiner 9 Teile bleiben unverändert (A07, B38, C00, C10, C62, C76, C08); Profil bleibt schlank, keine Schwergewichte.', 'Beide Ersetzungen waren mehrfach im Spiel und erhöhen nachweislich die Wirkung pro Aufwand ohne Zusatzrisiko.', 'Falls dir einer der beiden Rücktauschs wichtig ist, biete ich sofort: (a) A83 zurück statt C56 (C03 bleibt {-}> C32): Aufwand 3034; oder (b) C03 zurück statt C32 (A83 bleibt draußen {-}> C56): Aufwand 3076. Gib mir Bescheid, dann finalisiere ich entsprechend.'\}} \\
            }
        }
    }
    & & \\ \\

    \theutterance \stepcounter{utterance}  
    & & \multicolumn{4}{p{0.6\linewidth}}{
        \cellcolor[rgb]{0.9,0.9,0.9}{
            \makecell[{{p{\linewidth}}}]{
                \texttt{\tiny{[GM$\rangle$P2]}}
                \texttt{ABLEHNUNG: \{'A07', 'B38', 'C00', 'C10', 'C62', 'C03', 'C76', 'C08', 'A83'\}} \\
\texttt{VORSCHLAG: \{'A07', 'B38', 'C00', 'C10', 'C62', 'C32', 'C76', 'C08', 'C56'\}} \\
\texttt{ARGUMENT: \{'Zwei minimale 1‑zu‑1‑Swaps gegenüber deinem Paket: C03 {-}> C32 und A83 {-}> C56.', 'Aufwand 2979 von 3203 – 224 Puffer, deutlich unter dem Cap.', '7 deiner 9 Teile bleiben unverändert (A07, B38, C00, C10, C62, C76, C08); Profil bleibt schlank, keine Schwergewichte.', 'Beide Ersetzungen waren mehrfach im Spiel und erhöhen nachweislich die Wirkung pro Aufwand ohne Zusatzrisiko.', 'Falls dir einer der beiden Rücktauschs wichtig ist, biete ich sofort: (a) A83 zurück statt C56 (C03 bleibt {-}> C32): Aufwand 3034; oder (b) C03 zurück statt C32 (A83 bleibt draußen {-}> C56): Aufwand 3076. Gib mir Bescheid, dann finalisiere ich entsprechend.'\}} \\
            }
        }
    }
    & \\ \\

    \theutterance \stepcounter{utterance}  
    & & & \multicolumn{4}{p{0.6\linewidth}}{
        \cellcolor[rgb]{0.9,0.9,0.9}{
            \makecell[{{p{\linewidth}}}]{
                \texttt{\tiny{[GM$\langle$P2]}}
                \texttt{ABLEHNUNG: \{'A07', 'B38', 'C00', 'C10', 'C62', 'C32', 'C76', 'C08', 'C56'\}} \\
\texttt{VORSCHLAG: \{'A07', 'B38', 'C00', 'C10', 'C62', 'C32', 'C76', 'C08', 'A83'\}} \\
\texttt{ARGUMENT: \{'Einziger 1‑zu‑1‑Tausch: C56 {-}> A83.', 'Aufwand 3034 von 3203 – 169 Puffer, also komfortabel unter dem Cap.', '8 von 9 deiner Positionen bleiben unverändert; Profil bleibt schlank, keine Schwergewichte.', 'A83 war mehrfach bei uns im Spiel und ist anschlussfähig; der Swap steigert die Wirkung pro Aufwand ohne Mehrkomplexität.', 'Falls du C56 unbedingt behalten willst, biete ich zwei Minimal‑Alternativen an: (A) nur C32 {-}> A83 (Aufwand 2875) oder (B) nur C32 {-}> C03 (Aufwand 3076). Sag mir, ob die Basis passt oder ob du A/B bevorzugst – ich finalisiere sofort.'\}} \\
            }
        }
    }
     \\ \\

    \theutterance \stepcounter{utterance}  
    & & \multicolumn{4}{p{0.6\linewidth}}{
        \cellcolor[rgb]{0.9,0.9,0.9}{
            \makecell[{{p{\linewidth}}}]{
                \texttt{\tiny{[P1$\langle$GM]}}
                \texttt{ABLEHNUNG: \{'A07', 'B38', 'C00', 'C10', 'C62', 'C32', 'C76', 'C08', 'C56'\}} \\
\texttt{VORSCHLAG: \{'A07', 'B38', 'C00', 'C10', 'C62', 'C32', 'C76', 'C08', 'A83'\}} \\
\texttt{ARGUMENT: \{'Einziger 1‑zu‑1‑Tausch: C56 {-}> A83.', 'Aufwand 3034 von 3203 – 169 Puffer, also komfortabel unter dem Cap.', '8 von 9 deiner Positionen bleiben unverändert; Profil bleibt schlank, keine Schwergewichte.', 'A83 war mehrfach bei uns im Spiel und ist anschlussfähig; der Swap steigert die Wirkung pro Aufwand ohne Mehrkomplexität.', 'Falls du C56 unbedingt behalten willst, biete ich zwei Minimal‑Alternativen an: (A) nur C32 {-}> A83 (Aufwand 2875) oder (B) nur C32 {-}> C03 (Aufwand 3076). Sag mir, ob die Basis passt oder ob du A/B bevorzugst – ich finalisiere sofort.'\}} \\
            }
        }
    }
    & \\ \\

    \theutterance \stepcounter{utterance}  
    & \multicolumn{4}{p{0.6\linewidth}}{
        \cellcolor[rgb]{0.9,0.9,0.9}{
            \makecell[{{p{\linewidth}}}]{
                \texttt{\tiny{[P1$\rangle$GM]}}
                \texttt{ABLEHNUNG: \{'A07', 'B38', 'C00', 'C10', 'C62', 'C03', 'C76', 'C08', 'A83'\}} \\
\texttt{VORSCHLAG: \{'A07', 'B38', 'C00', 'C10', 'C62', 'C32', 'C76', 'C56', 'A83'\}} \\
\texttt{ARGUMENT: \{'Einziger 1‑zu‑1‑Tausch: C08 {-}> C56. Aufwand 3006 von 3203 – 197 Puffer.', '8 deiner 9 Positionen bleiben unverändert; Profil bleibt schlank, keine Schwergewichte.', 'Der Swap erhöht die Wirkung pro Aufwand deutlich bei geringerem Aufwand – minimalinvasiv und anschlussfähig.', 'Falls C08 für dich gesetzt ist, schlanke Alternative (nur 1 Tausch): C10 {-}> A03; Aufwand 3060. Gib mir kurz Go für C08{-}>C56 oder nenne die Alternative C10{-}>A03.'\}} \\
            }
        }
    }
    & & \\ \\

    \theutterance \stepcounter{utterance}  
    & & \multicolumn{4}{p{0.6\linewidth}}{
        \cellcolor[rgb]{0.9,0.9,0.9}{
            \makecell[{{p{\linewidth}}}]{
                \texttt{\tiny{[P1$\langle$GM]}}
                \texttt{Du hast einen Vorschlag abgelehnt, der nicht aktiv ist. Vorschläge sind nur dann aktiv, wenn sie vom anderen Spieler mit VORSCHLAG eingebracht wurden und nicht von dir mit ABLEHNUNG deaktiviert wurden. Versuche es noch einmal.} \\
            }
        }
    }
    & \\ \\

    \theutterance \stepcounter{utterance}  
    & \multicolumn{4}{p{0.6\linewidth}}{
        \cellcolor[rgb]{0.9,0.9,0.9}{
            \makecell[{{p{\linewidth}}}]{
                \texttt{\tiny{[P1$\rangle$GM]}}
                \texttt{ABLEHNUNG: \{'A07', 'B38', 'C00', 'C10', 'C62', 'C32', 'C76', 'C08', 'A83'\}} \\
\texttt{VORSCHLAG: \{'A07', 'B38', 'C00', 'C10', 'C62', 'C32', 'C76', 'C56', 'A83'\}} \\
\texttt{ARGUMENT: \{'Korrekte Bezugnahme auf deinen aktiven Vorschlag – ich schlage einen einzigen 1‑zu‑1‑Tausch vor: C08 {-}> C56.', 'Aufwand 3006 von 3203 – 197 Puffer, klar unter dem Cap.', '8 von 9 Teilen bleiben unverändert; Profil bleibt schlank, keine Schwergewichte.', 'Der Swap erhöht die Wirkung pro Slot deutlich ohne Mehraufwand oder Risiko.', 'Falls C08 für dich unverzichtbar ist, biete ich als Einzelswap‑Alternative C10 {-}> A03 an (Aufwand 3060); sag mir kurz, ob du lieber diese Variante möchtest.'\}} \\
            }
        }
    }
    & & \\ \\

    \theutterance \stepcounter{utterance}  
    & & \multicolumn{4}{p{0.6\linewidth}}{
        \cellcolor[rgb]{0.9,0.9,0.9}{
            \makecell[{{p{\linewidth}}}]{
                \texttt{\tiny{[GM$\rangle$P2]}}
                \texttt{ABLEHNUNG: \{'A07', 'B38', 'C00', 'C10', 'C62', 'C32', 'C76', 'C08', 'A83'\}} \\
\texttt{VORSCHLAG: \{'A07', 'B38', 'C00', 'C10', 'C62', 'C32', 'C76', 'C56', 'A83'\}} \\
\texttt{ARGUMENT: \{'Korrekte Bezugnahme auf deinen aktiven Vorschlag – ich schlage einen einzigen 1‑zu‑1‑Tausch vor: C08 {-}> C56.', 'Aufwand 3006 von 3203 – 197 Puffer, klar unter dem Cap.', '8 von 9 Teilen bleiben unverändert; Profil bleibt schlank, keine Schwergewichte.', 'Der Swap erhöht die Wirkung pro Slot deutlich ohne Mehraufwand oder Risiko.', 'Falls C08 für dich unverzichtbar ist, biete ich als Einzelswap‑Alternative C10 {-}> A03 an (Aufwand 3060); sag mir kurz, ob du lieber diese Variante möchtest.'\}} \\
            }
        }
    }
    & \\ \\

    \theutterance \stepcounter{utterance}  
    & & & \multicolumn{4}{p{0.6\linewidth}}{
        \cellcolor[rgb]{0.9,0.9,0.9}{
            \makecell[{{p{\linewidth}}}]{
                \texttt{\tiny{[GM$\langle$P2]}}
                \texttt{ABLEHNUNG: \{'A07', 'B38', 'C00', 'C10', 'C62', 'C32', 'C76', 'C56', 'A83'\}} \\
\texttt{VORSCHLAG: \{'A07', 'B38', 'C00', 'C10', 'C62', 'C03', 'C76', 'C56', 'A83'\}} \\
\texttt{ARGUMENT: \{'Einziger 1‑zu‑1‑Tausch: C32 {-}> C03.', 'Aufwand 3103 von 3203 – 100 Puffer, klar unter dem Cap.', '8 deiner 9 Positionen bleiben unverändert; Profil bleibt schlank, keine Schwergewichte.', 'C03 war mehrfach in deinen Vorschlägen und liefert spürbar mehr Output pro Slot bei ähnlicher Komplexität.', 'Falls C03 für dich nicht geht, Minimal‑Alternative: nur C00 {-}> C98; Aufwand 3001. Gib mir Go für die Basis (C32{-}>C03) oder die Alternative – ich finalisiere sofort.'\}} \\
            }
        }
    }
     \\ \\

    \theutterance \stepcounter{utterance}  
    & & \multicolumn{4}{p{0.6\linewidth}}{
        \cellcolor[rgb]{0.9,0.9,0.9}{
            \makecell[{{p{\linewidth}}}]{
                \texttt{\tiny{[P1$\langle$GM]}}
                \texttt{ABLEHNUNG: \{'A07', 'B38', 'C00', 'C10', 'C62', 'C32', 'C76', 'C56', 'A83'\}} \\
\texttt{VORSCHLAG: \{'A07', 'B38', 'C00', 'C10', 'C62', 'C03', 'C76', 'C56', 'A83'\}} \\
\texttt{ARGUMENT: \{'Einziger 1‑zu‑1‑Tausch: C32 {-}> C03.', 'Aufwand 3103 von 3203 – 100 Puffer, klar unter dem Cap.', '8 deiner 9 Positionen bleiben unverändert; Profil bleibt schlank, keine Schwergewichte.', 'C03 war mehrfach in deinen Vorschlägen und liefert spürbar mehr Output pro Slot bei ähnlicher Komplexität.', 'Falls C03 für dich nicht geht, Minimal‑Alternative: nur C00 {-}> C98; Aufwand 3001. Gib mir Go für die Basis (C32{-}>C03) oder die Alternative – ich finalisiere sofort.'\}} \\
            }
        }
    }
    & \\ \\

    \theutterance \stepcounter{utterance}  
    & \multicolumn{4}{p{0.6\linewidth}}{
        \cellcolor[rgb]{0.9,0.9,0.9}{
            \makecell[{{p{\linewidth}}}]{
                \texttt{\tiny{[P1$\rangle$GM]}}
                \texttt{ABLEHNUNG: \{'A07', 'B38', 'C00', 'C10', 'C62', 'C03', 'C76', 'C56', 'A83'\}} \\
\texttt{VORSCHLAG: \{'A07', 'B38', 'C00', 'C10', 'C62', 'C32', 'C76', 'C56', 'A83'\}} \\
\texttt{ARGUMENT: \{'Einziger 1‑zu‑1‑Tausch: C03 {-}> C32.', 'Aufwand 3006 von 3203 – 197 Puffer, klar unter dem Cap.', '8 deiner 9 Positionen bleiben unverändert; Profil bleibt schlank, keine Schwergewichte.', 'C32 war mehrfach in deinen Vorschlägen und liefert mehr Wirkung pro Aufwand als C03 – bei geringerem Aufwand.', 'Falls C03 für dich unverzichtbar ist, schlanke Alternative: nur C10 {-}> A03; Aufwand 3129. Sag mir, welche der beiden Minimalvarianten dir passt.'\}} \\
            }
        }
    }
    & & \\ \\

    \theutterance \stepcounter{utterance}  
    & & \multicolumn{4}{p{0.6\linewidth}}{
        \cellcolor[rgb]{0.9,0.9,0.9}{
            \makecell[{{p{\linewidth}}}]{
                \texttt{\tiny{[GM$\rangle$P2]}}
                \texttt{ABLEHNUNG: \{'A07', 'B38', 'C00', 'C10', 'C62', 'C03', 'C76', 'C56', 'A83'\}} \\
\texttt{VORSCHLAG: \{'A07', 'B38', 'C00', 'C10', 'C62', 'C32', 'C76', 'C56', 'A83'\}} \\
\texttt{ARGUMENT: \{'Einziger 1‑zu‑1‑Tausch: C03 {-}> C32.', 'Aufwand 3006 von 3203 – 197 Puffer, klar unter dem Cap.', '8 deiner 9 Positionen bleiben unverändert; Profil bleibt schlank, keine Schwergewichte.', 'C32 war mehrfach in deinen Vorschlägen und liefert mehr Wirkung pro Aufwand als C03 – bei geringerem Aufwand.', 'Falls C03 für dich unverzichtbar ist, schlanke Alternative: nur C10 {-}> A03; Aufwand 3129. Sag mir, welche der beiden Minimalvarianten dir passt.'\}} \\
            }
        }
    }
    & \\ \\

    \theutterance \stepcounter{utterance}  
    & & & \multicolumn{4}{p{0.6\linewidth}}{
        \cellcolor[rgb]{0.9,0.9,0.9}{
            \makecell[{{p{\linewidth}}}]{
                \texttt{\tiny{[GM$\langle$P2]}}
                \texttt{ABLEHNUNG: \{'A07', 'B38', 'C00', 'C10', 'C62', 'C32', 'C76', 'C56', 'A83'\}} \\
\texttt{VORSCHLAG: \{'A07', 'B38', 'C98', 'C10', 'C62', 'C03', 'C76', 'C08', 'A83'\}} \\
\texttt{ARGUMENT: \{'Drei schlanke 1‑zu‑1‑Swaps gegenüber deinem Paket: C00 {-}> C98 ({-}5 Aufwand), C32 {-}> C03 (+97), C56 {-}> C08 (+28).', 'Aufwand 3126 von 3203 – 77 Puffer, klar unter dem Cap.', '6 deiner 9 Positionen bleiben unverändert (A07, B38, C10, C62, C76, A83); Profil bleibt schlank, keine Schwergewichte.', 'Alle drei Ersetzungen waren bereits im Verlauf als gangbar im Spiel; sie erhöhen die Wirkung pro Aufwand deutlich ohne Mehrkomplexität.', 'Falls genau ein Swap ein Show‑Stopper ist, nenne mir welchen – ich kann sofort auf eine Variante mit nur zwei oder nur einem der Swaps gehen: nur C00{-}>C98: 3001 Aufwand; nur C32{-}>C03: 3103; nur C56{-}>C08: 3034; oder jede gewünschte Zweier‑Kombi (z. B. C32{-}>C03 + C56{-}>C08 = 3131).'\}} \\
            }
        }
    }
     \\ \\

    \theutterance \stepcounter{utterance}  
    & & \multicolumn{4}{p{0.6\linewidth}}{
        \cellcolor[rgb]{0.9,0.9,0.9}{
            \makecell[{{p{\linewidth}}}]{
                \texttt{\tiny{[P1$\langle$GM]}}
                \texttt{ABLEHNUNG: \{'A07', 'B38', 'C00', 'C10', 'C62', 'C32', 'C76', 'C56', 'A83'\}} \\
\texttt{VORSCHLAG: \{'A07', 'B38', 'C98', 'C10', 'C62', 'C03', 'C76', 'C08', 'A83'\}} \\
\texttt{ARGUMENT: \{'Drei schlanke 1‑zu‑1‑Swaps gegenüber deinem Paket: C00 {-}> C98 ({-}5 Aufwand), C32 {-}> C03 (+97), C56 {-}> C08 (+28).', 'Aufwand 3126 von 3203 – 77 Puffer, klar unter dem Cap.', '6 deiner 9 Positionen bleiben unverändert (A07, B38, C10, C62, C76, A83); Profil bleibt schlank, keine Schwergewichte.', 'Alle drei Ersetzungen waren bereits im Verlauf als gangbar im Spiel; sie erhöhen die Wirkung pro Aufwand deutlich ohne Mehrkomplexität.', 'Falls genau ein Swap ein Show‑Stopper ist, nenne mir welchen – ich kann sofort auf eine Variante mit nur zwei oder nur einem der Swaps gehen: nur C00{-}>C98: 3001 Aufwand; nur C32{-}>C03: 3103; nur C56{-}>C08: 3034; oder jede gewünschte Zweier‑Kombi (z. B. C32{-}>C03 + C56{-}>C08 = 3131).'\}} \\
            }
        }
    }
    & \\ \\

    \theutterance \stepcounter{utterance}  
    & \multicolumn{4}{p{0.6\linewidth}}{
        \cellcolor[rgb]{0.9,0.9,0.9}{
            \makecell[{{p{\linewidth}}}]{
                \texttt{\tiny{[P1$\rangle$GM]}}
                \texttt{ABLEHNUNG: \{'A07', 'B38', 'C98', 'C10', 'C62', 'C03', 'C76', 'C08', 'A83'\}} \\
\texttt{VORSCHLAG: \{'A07', 'B38', 'C00', 'C10', 'C62', 'C03', 'C76', 'C08', 'A83'\}} \\
\texttt{ARGUMENT: \{'Einziger 1‑zu‑1‑Tausch: C98 {-}> C00.', 'Aufwand 3131 von 3203 – 72 Puffer, sicher unter dem Cap.', '8 von 9 Positionen bleiben unverändert; Profil bleibt schlank, keine Schwergewichte.', 'C00 war bei uns mehrfach als gangbare Option im Spiel – minimalinvasiv und anschlussfähig.', 'Falls C00 für dich nicht geht, nennen wir genau einen alternativen Einzelswap auf deiner Basis: entweder C10 {-}> A03 (Aufwand 3152) oder C08 {-}> C56 (Aufwand 3098). Gib mir deine Präferenz – ich finalisiere sofort.'\}} \\
            }
        }
    }
    & & \\ \\

    \theutterance \stepcounter{utterance}  
    & & \multicolumn{4}{p{0.6\linewidth}}{
        \cellcolor[rgb]{0.9,0.9,0.9}{
            \makecell[{{p{\linewidth}}}]{
                \texttt{\tiny{[GM$\rangle$P2]}}
                \texttt{ABLEHNUNG: \{'A07', 'B38', 'C98', 'C10', 'C62', 'C03', 'C76', 'C08', 'A83'\}} \\
\texttt{VORSCHLAG: \{'A07', 'B38', 'C00', 'C10', 'C62', 'C03', 'C76', 'C08', 'A83'\}} \\
\texttt{ARGUMENT: \{'Einziger 1‑zu‑1‑Tausch: C98 {-}> C00.', 'Aufwand 3131 von 3203 – 72 Puffer, sicher unter dem Cap.', '8 von 9 Positionen bleiben unverändert; Profil bleibt schlank, keine Schwergewichte.', 'C00 war bei uns mehrfach als gangbare Option im Spiel – minimalinvasiv und anschlussfähig.', 'Falls C00 für dich nicht geht, nennen wir genau einen alternativen Einzelswap auf deiner Basis: entweder C10 {-}> A03 (Aufwand 3152) oder C08 {-}> C56 (Aufwand 3098). Gib mir deine Präferenz – ich finalisiere sofort.'\}} \\
            }
        }
    }
    & \\ \\

    \theutterance \stepcounter{utterance}  
    & & & \multicolumn{4}{p{0.6\linewidth}}{
        \cellcolor[rgb]{0.9,0.9,0.9}{
            \makecell[{{p{\linewidth}}}]{
                \texttt{\tiny{[GM$\langle$P2]}}
                \texttt{ABLEHNUNG: \{'A07', 'B38', 'C00', 'C10', 'C62', 'C03', 'C76', 'C08', 'A83'\}} \\
\texttt{VORSCHLAG: \{'A07', 'B38', 'C98', 'C10', 'C62', 'C03', 'C76', 'C08', 'A83'\}} \\
\texttt{ARGUMENT: \{'Minimaler 1‑zu‑1‑Tausch: C00 {-}> C98; alle anderen 8 Positionen bleiben unverändert.', 'Aufwand 3126 von 3203 – 77 Puffer, also noch schlanker als dein Paket.', 'C98 liegt im gleichen Aufwandskorridor wie C00, bringt aber spürbar mehr Output pro Slot – ohne Mehrkomplexität oder Risiko.', 'Diese exakte Variante war bereits zuvor bei dir im Spiel – anschlussfähig und sofort umsetzbar.', 'Falls C00 für dich gesetzt ist, sag es kurz: Dann bestätige ich im nächsten Schritt dein aktuelles Paket 1:1 ohne weitere Änderungen.'\}} \\
            }
        }
    }
     \\ \\

    \theutterance \stepcounter{utterance}  
    & & \multicolumn{4}{p{0.6\linewidth}}{
        \cellcolor[rgb]{0.9,0.9,0.9}{
            \makecell[{{p{\linewidth}}}]{
                \texttt{\tiny{[P1$\langle$GM]}}
                \texttt{ABLEHNUNG: \{'A07', 'B38', 'C00', 'C10', 'C62', 'C03', 'C76', 'C08', 'A83'\}} \\
\texttt{VORSCHLAG: \{'A07', 'B38', 'C98', 'C10', 'C62', 'C03', 'C76', 'C08', 'A83'\}} \\
\texttt{ARGUMENT: \{'Minimaler 1‑zu‑1‑Tausch: C00 {-}> C98; alle anderen 8 Positionen bleiben unverändert.', 'Aufwand 3126 von 3203 – 77 Puffer, also noch schlanker als dein Paket.', 'C98 liegt im gleichen Aufwandskorridor wie C00, bringt aber spürbar mehr Output pro Slot – ohne Mehrkomplexität oder Risiko.', 'Diese exakte Variante war bereits zuvor bei dir im Spiel – anschlussfähig und sofort umsetzbar.', 'Falls C00 für dich gesetzt ist, sag es kurz: Dann bestätige ich im nächsten Schritt dein aktuelles Paket 1:1 ohne weitere Änderungen.'\}} \\
            }
        }
    }
    & \\ \\

    \theutterance \stepcounter{utterance}  
    & \multicolumn{4}{p{0.6\linewidth}}{
        \cellcolor[rgb]{0.9,0.9,0.9}{
            \makecell[{{p{\linewidth}}}]{
                \texttt{\tiny{[P1$\rangle$GM]}}
                \texttt{ABLEHNUNG: \{'A07', 'B38', 'C98', 'C10', 'C62', 'C03', 'C76', 'C08', 'A83'\}} \\
\texttt{VORSCHLAG: \{'A07', 'B38', 'C00', 'C10', 'C62', 'C03', 'C76', 'C08', 'A83'\}} \\
\texttt{ARGUMENT: \{'Einziger 1‑zu‑1‑Tausch: C98 {-}> C00. Aufwand 3131 von 3203 – 72 Puffer.', '8 deiner 9 Positionen bleiben unverändert; Profil bleibt schlank, keine Schwergewichte.', 'Du hattest C00 bereits mehrfach als gangbar markiert – minimalinvasiv und anschlussfähig.', 'Falls dir zusätzlich A03 lieber ist als C10, kann ich optional C10 {-}> A03 tauschen; Aufwand läge dann bei 3157 (weiter sicher unter dem Cap).'\}} \\
            }
        }
    }
    & & \\ \\

    \theutterance \stepcounter{utterance}  
    & & \multicolumn{4}{p{0.6\linewidth}}{
        \cellcolor[rgb]{0.9,0.9,0.9}{
            \makecell[{{p{\linewidth}}}]{
                \texttt{\tiny{[GM$\rangle$P2]}}
                \texttt{ABLEHNUNG: \{'A07', 'B38', 'C98', 'C10', 'C62', 'C03', 'C76', 'C08', 'A83'\}} \\
\texttt{VORSCHLAG: \{'A07', 'B38', 'C00', 'C10', 'C62', 'C03', 'C76', 'C08', 'A83'\}} \\
\texttt{ARGUMENT: \{'Einziger 1‑zu‑1‑Tausch: C98 {-}> C00. Aufwand 3131 von 3203 – 72 Puffer.', '8 deiner 9 Positionen bleiben unverändert; Profil bleibt schlank, keine Schwergewichte.', 'Du hattest C00 bereits mehrfach als gangbar markiert – minimalinvasiv und anschlussfähig.', 'Falls dir zusätzlich A03 lieber ist als C10, kann ich optional C10 {-}> A03 tauschen; Aufwand läge dann bei 3157 (weiter sicher unter dem Cap).'\}} \\
            }
        }
    }
    & \\ \\

    \theutterance \stepcounter{utterance}  
    & & & \multicolumn{4}{p{0.6\linewidth}}{
        \cellcolor[rgb]{0.9,0.9,0.9}{
            \makecell[{{p{\linewidth}}}]{
                \texttt{\tiny{[GM$\langle$P2]}}
                \texttt{ABLEHNUNG: \{'A07', 'B38', 'C00', 'C10', 'C62', 'C03', 'C76', 'C08', 'A83'\}} \\
\texttt{VORSCHLAG: \{'A07', 'B38', 'C98', 'C10', 'C62', 'C03', 'C76', 'C08', 'A83'\}} \\
\texttt{ARGUMENT: \{'Minimaler 1‑zu‑1‑Tausch: C00 {-}> C98; alle anderen 8 Positionen bleiben unverändert.', 'Aufwand 3126 von 3203 – 77 Puffer, also noch sicherer unter dem Cap als dein Paket (3131).', 'C98 liegt im gleichen Aufwandskorridor, liefert aber spürbar mehr Output pro Slot – ohne Mehrkomplexität oder Zusatzrisiko.', 'Diese Variante war bereits mehrfach bei uns im Spiel und ist anschlussfähig.', 'Wenn C98 für dich absolut nicht geht, sag es – dann kann ich im nächsten Schritt dein Paket 1:1 bestätigen.'\}} \\
            }
        }
    }
     \\ \\

    \theutterance \stepcounter{utterance}  
    & & \multicolumn{4}{p{0.6\linewidth}}{
        \cellcolor[rgb]{0.9,0.9,0.9}{
            \makecell[{{p{\linewidth}}}]{
                \texttt{\tiny{[P1$\langle$GM]}}
                \texttt{ABLEHNUNG: \{'A07', 'B38', 'C00', 'C10', 'C62', 'C03', 'C76', 'C08', 'A83'\}} \\
\texttt{VORSCHLAG: \{'A07', 'B38', 'C98', 'C10', 'C62', 'C03', 'C76', 'C08', 'A83'\}} \\
\texttt{ARGUMENT: \{'Minimaler 1‑zu‑1‑Tausch: C00 {-}> C98; alle anderen 8 Positionen bleiben unverändert.', 'Aufwand 3126 von 3203 – 77 Puffer, also noch sicherer unter dem Cap als dein Paket (3131).', 'C98 liegt im gleichen Aufwandskorridor, liefert aber spürbar mehr Output pro Slot – ohne Mehrkomplexität oder Zusatzrisiko.', 'Diese Variante war bereits mehrfach bei uns im Spiel und ist anschlussfähig.', 'Wenn C98 für dich absolut nicht geht, sag es – dann kann ich im nächsten Schritt dein Paket 1:1 bestätigen.'\}} \\
            }
        }
    }
    & \\ \\

    \theutterance \stepcounter{utterance}  
    & \multicolumn{4}{p{0.6\linewidth}}{
        \cellcolor[rgb]{0.9,0.9,0.9}{
            \makecell[{{p{\linewidth}}}]{
                \texttt{\tiny{[P1$\rangle$GM]}}
                \texttt{ABLEHNUNG: \{'A07', 'B38', 'C98', 'C10', 'C62', 'C03', 'C76', 'C08', 'A83'\}} \\
\texttt{VORSCHLAG: \{'A07', 'B38', 'C00', 'C10', 'C62', 'C03', 'C76', 'C08', 'A83'\}} \\
\texttt{ARGUMENT: \{'Minimaler 1‑zu‑1‑Tausch: C98 {-}> C00; alle anderen 8 Positionen bleiben unverändert.', 'Aufwand 3131 von 3203 – 72 Puffer, sicher unter dem Cap.', 'Diese Variante hatten wir mehrfach als gangbar im Spiel – minimalinvasiv, gleicher Profilzuschnitt, sofort umsetzbar.', 'Falls dir C00 nicht passt, biete ich als ebenso schlanke Einzelswaps auf deiner Basis: (a) A03 statt C10 (Aufwand 3152) oder (b) C56 statt A83 (Aufwand 3071). Sag mir, welche Minimalvariante dir lieber ist.'\}} \\
            }
        }
    }
    & & \\ \\

    \theutterance \stepcounter{utterance}  
    & & \multicolumn{4}{p{0.6\linewidth}}{
        \cellcolor[rgb]{0.9,0.9,0.9}{
            \makecell[{{p{\linewidth}}}]{
                \texttt{\tiny{[GM$\rangle$P2]}}
                \texttt{ABLEHNUNG: \{'A07', 'B38', 'C98', 'C10', 'C62', 'C03', 'C76', 'C08', 'A83'\}} \\
\texttt{VORSCHLAG: \{'A07', 'B38', 'C00', 'C10', 'C62', 'C03', 'C76', 'C08', 'A83'\}} \\
\texttt{ARGUMENT: \{'Minimaler 1‑zu‑1‑Tausch: C98 {-}> C00; alle anderen 8 Positionen bleiben unverändert.', 'Aufwand 3131 von 3203 – 72 Puffer, sicher unter dem Cap.', 'Diese Variante hatten wir mehrfach als gangbar im Spiel – minimalinvasiv, gleicher Profilzuschnitt, sofort umsetzbar.', 'Falls dir C00 nicht passt, biete ich als ebenso schlanke Einzelswaps auf deiner Basis: (a) A03 statt C10 (Aufwand 3152) oder (b) C56 statt A83 (Aufwand 3071). Sag mir, welche Minimalvariante dir lieber ist.'\}} \\
            }
        }
    }
    & \\ \\

    \theutterance \stepcounter{utterance}  
    & & & \multicolumn{4}{p{0.6\linewidth}}{
        \cellcolor[rgb]{0.9,0.9,0.9}{
            \makecell[{{p{\linewidth}}}]{
                \texttt{\tiny{[GM$\langle$P2]}}
                \texttt{ZUSTIMMUNG: \{'A07', 'B38', 'C00', 'C10', 'C62', 'C03', 'C76', 'C08', 'A83'\}} \\
\texttt{ARGUMENT: \{'Aufwand 3131 von 3203 – sicher unter dem Cap.', 'Dein Paket verbindet schnelle Hebel (C76, C08) mit starken Kernbausteinen (C10, C03, C62, C00) und sinnvollen Ergänzungen (A83, B38, A07).', 'Minimal{-}invasiv, konsensnah und ohne Schwergewichte – ich gehe damit einverstanden.'\}} \\
            }
        }
    }
     \\ \\

    \theutterance \stepcounter{utterance}  
    & & & \multicolumn{2}{p{0.3\linewidth}}{
        \cellcolor[rgb]{0.9,0.9,0.9}{
            \makecell[{{p{\linewidth}}}]{
                \texttt{\tiny{[GM$|$GM]}}
                \texttt{game successful} \\
            }
        }
    }
    & & \\ \\

    \theutterance \stepcounter{utterance}  
    & & & \multicolumn{2}{p{0.3\linewidth}}{
        \cellcolor[rgb]{0.9,0.9,0.9}{
            \makecell[{{p{\linewidth}}}]{
                \texttt{\tiny{[GM$|$GM]}}
                \texttt{end game} \\
            }
        }
    }
    & & \\ \\

\end{supertabular}
}

\end{document}
